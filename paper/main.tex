% !TEX program = xelatex
% !BIB program = bibtex
\documentclass[conference,compsoc]{IEEEtran} 
% \usepackage[utf8]{inputenc}
\usepackage{fontspec}
\usepackage{array,multirow,graphicx}
\usepackage{csquotes}
\usepackage{amsthm}
\usepackage{amsmath}
\usepackage{amssymb}
\usepackage{bm}
\usepackage{braket}
\usepackage{graphicx}
\usepackage{verbatimbox}
\usepackage{wrapfig, fancyvrb}
\usepackage{subcaption}
\usepackage{float}
\usepackage{hyperref}
\usepackage{cleveref}
\usepackage{pgfplots}
\usepackage{placeins}

% Make \paragraph bold
\makeatletter
\def\theparagraph{\thesubsubsection.\@arabic\c@paragraph}
\def\paragraph{\@startsection{paragraph}{4}{\z@}%
                                    {3.25ex \@plus1ex \@minus.2ex}%
                                    {-1em}%
                                    {\normalfont\normalsize\bfseries}}
\makeatother


\DeclareMathOperator*{\argmax}{arg\,max}
\DeclareMathOperator*{\argmin}{arg\,min}

\usepackage[backend=bibtex,style=authoryear]{biblatex}
\usepackage{listings}
\usepackage{tikz-qtree}

\addbibresource{ref.bib}

\newcommand{\action}[1]{\scriptsize{\textsc{#1}}}

\theoremstyle{definition}
\newtheorem{definition}{Definition}[section]

\title{Bayesian Optimisation using an Ensemble of Neural Networks}
\author{Thomas M. Pethick}
\date{September 2018}

\begin{document}

\maketitle
\thispagestyle{plain} % give it pagenumbers
\pagestyle{plain}

\begin{abstract}
    State-of-the-art hyperparameter optimization of machine learning algorithms has been achieved with Bayesian Optimization using Gaussian Processes (GPs). 
    However, computational cost for this model scales cubically in the number of observations.
    Recent methods replaces the GP with a Bayesian linear regressor on basis functions learned by a neural network -- an approach that scale linearly.
    The objective is to explore where this method fails and investigate whether an ensemble of neural networks improves convergence in those scenarios.
    The preliminary investigation suggests that an ensemble is beneficial in problems with low effective dimensionality and in spaces of sufficient dimensionality (i.e. 6 or above).
\end{abstract}

\section{Introduction}

    The performance of machine learning algorithms is heavily dependent on the configuration of so-called \emph{hyperparameters}.
    In particular, for a deep neural network architecture this could be the number of units for each layer, the number of layers and the activation functions.
    We can treat the possible collection of hyperparameters as a space which we will call the \emph{hyperparameter space}.
    In this framework the objective is to find the point in this space that yields the best performance.
    What we have done is to frame it as an optimization problem, more precisely formulated mathematically as finding the optimal configuration,
        \begin{equation}
            \bm{x}^\star = \argmax_{\bm{x}=\mathcal{X}}f(\bm{x}).
        \end{equation}

    \noindent where $\mathcal{X}$ in this particular instance is the hyperparameter space and $f$ is a function providing a measure of performance of the machine learning algorithm for a a certain hyperparameter configuration.

    This problem belongs to an interesting subclass of optimization problems.
    First of all, $f$ is very expensive to evaluate.
    In our particular instance it is required to run the machine learning algorithm to completion to evaluate $f$.
    This could take days for a single evaluation considering the training required for recent deep learning methods.
    Secondly, $f$ is considered a "black-box" in the sense that it has no special structure such as linearity or concavity that would make the problem easy and no first- or second-order derivative is observed that would similarly simplify the problem.
    %Optimization problems in which minimal assumptions are made about the problem .

    Several solutions have been proposed to this problem.
    Naive proposals include non-adaptive algorithms of which \textsc{Random Sampling} is one.
    It samples $N$ points uniformly at random from $\mathcal{X}$ and picks the one that optimizes $f(\cdot)$.
    Another similarly simple algorithms is the \textsc{Grid Search} which instead selects the $N$ points from a grid -- more precisely the cartesian product between a finite subset of each dimension.

    If we instead query $f$ sequentially we can make an informed choice of the next point to evaluate given the already observed evaluations.
    These methods are commonly referred to as \emph{Bayesian Optimization} since the way they update their belief about what point to pick based on previous observation is fundamentally Bayesian.
    The assumption of expensive evaluation justifies spending computational effort on making informed choices.

    Numerous suggestions on how to model the belief of $f$ exists with most methods capturing it (at least partially) using \emph{Gaussian Processes} (GPs).
    Examples of this includes \textsc{Spearmint} \parencite{snoek_practical_2012} with a purely GP based approach, \textsc{SMAC} \parencite{hutter_sequential_2011} using Random Forest and \textsc{HyperOpt} \parencite{bergstra_algorithms_2011} using Tree-structured Parzen Estimator.
    These methods are computationally restricted by an expensive cost of $\mathcal{O}(n^3)$ for $n$ observations.
    
    More recently, methods using a Bayesian linear regressor on features from a neural network were proposed to obtain \emph{scalable} Bayesian Optimization \parencite{snoek_scalable_2015}.
    This was coined Deep Network for Global Optimization (DNGO).
    
    It is a well-known fact that the random initialization of neural network weights can have a significant impact on the networks performance.
    To mitigate this effect it is natural to consider a common statistical method called bagging in which an ensemble of the model is aggregated.
    Thus, we propose an extension of DNGO that uses an ensemble of neural networks.

    \subsection{Contributions}
        
        Our contributions are two-fold:

        \begin{itemize}
            \item The behaviour of DNGO is badly understood.
             We first present a qualitative analysis based on several categorized benchmarks in order to find failure cases in which an ensemble might help.
            \item Secondly, we propose a generalization of DNGO which uses an ensemble of neural network and investigate in which scenarios this is advantageous.
        \end{itemize}

    % - assume familiarity with common linear algebra notation, gaussian distribution, neural network

    \subsection{Related work}

        Another extension of DNGO has been proposed in which a distribution is places on all weights thus obtaining a Bayesian neural network \parencite{springenberg_bayesian_2016}.
        Even though it similarly attempts to make the method more robust it does so through other means.
        It is natural to compare these methods however.

        More recently it was proposed in \parencite{li_hyperband:_2016} that $2\times$\textsc{Random Search} (i.e. allowing to sample twice as many points) where competitive to Bayesian Optimization methods suggesting much simpler algorithms.
        However, experimental results in \parencite{golovin_google_2017} suggest that this is only the case in sufficiently high dimensions (specifically 16 or more).

    \subsection{Overview}

        The paper will be structured as follows.
        First the necessary background will be covered in \cref{sec:background} consisting of Bayesian Optimization specifically with a gaussian processes and the neural network approach referred to as DNGO.
        We are then equipped in \cref{sec:method} to specify the three categories of models being tested: GP, DNGO and ensembled DNGO.
        The results are then tested on several benchmarks and hyperparameter optimization tasks covered in \Cref{sec:exp}. We follow up by a discussion in \cref{sec:discussion} and close with a conclusion in \cref{sec:conclusion}.

\section{Background}\label{sec:background}

  \subsection{Bayesian Optimization}

        Bayesian Optimization is a sequential optimization strategy well-suited for finding an optimum in few iterations.
        It does so by incorporating a prior belief of the objective function $f$ and subsequently refining this through a Bayesian posterior as observations are made.
        The posterior now expresses our updated belief of the objective function given data.
        To know where to query the next observation an \emph{acquisition function} can be induced on the statistical model that guides exploration by using the uncertainty in the posterior.
        Thus the next point $\bm{x}_{t+1}$ at time step $t$ is selected by maximizing the acquisition function.
        \Cref{fig:bo} illustrates three steps of such a process where the input space is $\mathbb{R}^1$.

        To summaries the process involves two components, namely the prior over the objective function and the acquisition function.
        Since the acquisition function is defined based on the posterior function which arises from the prior, we will leave this topic until \cref{sec:acq}.
        
        Let $\mathcal{D}_{1:t} = \set{\bm{x}_{1:t}, y_{1:t}}$ be the sequence of observation up to time step $t$ in the Bayesian Optimization procedure.
        We combine the \emph{prior distribution} $P(f)$ with the \emph{likelihood function} $P(D_{1:t}\mid f)$ to obtain the \emph{posterior distribution},
        \begin{equation}
            P(f\mid\mathcal{D}_{1:t}) = \frac{P(\mathcal{D}_{1:t}\mid f)P(f)}{P(\mathcal{D}_{1:t})} \propto P(\mathcal{D}_{1:t}\mid f)P(f).
        \end{equation}

        This is essentially what makes it Bayesian since it is an application of Bayes rule.
        Notice that we can drop the normalization constant given it is irrelevant for an optimization task.
        This posterior is sometimes also referred to as the \emph{response surface} or \emph{surrogate function} since it estimates the objective function.

        One commonly used prior over $f$ is the Gaussian Process covered in \cref{sec:gp}.
        It is interesting since it yields an analytical expression for the posterior as well as for other reasons which will become apparent in the subsequent section.

        For a more complete treatment and overview of BO we refer to \parencite{shahriari_taking_2016}.

        \begin{figure*}[t]
            \centering
            %% Creator: Matplotlib, PGF backend
%%
%% To include the figure in your LaTeX document, write
%%   \input{<filename>.pgf}
%%
%% Make sure the required packages are loaded in your preamble
%%   \usepackage{pgf}
%%
%% Figures using additional raster images can only be included by \input if
%% they are in the same directory as the main LaTeX file. For loading figures
%% from other directories you can use the `import` package
%%   \usepackage{import}
%% and then include the figures with
%%   \import{<path to file>}{<filename>.pgf}
%%
%% Matplotlib used the following preamble
%%   \usepackage{gensymb}
%%   \usepackage{fontspec}
%%   \setmainfont{DejaVu Serif}
%%   \setsansfont{Arial}
%%   \setmonofont{DejaVu Sans Mono}
%%
\begingroup%
\makeatletter%
\begin{pgfpicture}%
\pgfpathrectangle{\pgfpointorigin}{\pgfqpoint{6.000000in}{2.500000in}}%
\pgfusepath{use as bounding box, clip}%
\begin{pgfscope}%
\pgfsetbuttcap%
\pgfsetmiterjoin%
\definecolor{currentfill}{rgb}{1.000000,1.000000,1.000000}%
\pgfsetfillcolor{currentfill}%
\pgfsetlinewidth{0.000000pt}%
\definecolor{currentstroke}{rgb}{1.000000,1.000000,1.000000}%
\pgfsetstrokecolor{currentstroke}%
\pgfsetdash{}{0pt}%
\pgfpathmoveto{\pgfqpoint{0.000000in}{0.000000in}}%
\pgfpathlineto{\pgfqpoint{6.000000in}{0.000000in}}%
\pgfpathlineto{\pgfqpoint{6.000000in}{2.500000in}}%
\pgfpathlineto{\pgfqpoint{0.000000in}{2.500000in}}%
\pgfpathclose%
\pgfusepath{fill}%
\end{pgfscope}%
\begin{pgfscope}%
\pgfsetbuttcap%
\pgfsetmiterjoin%
\definecolor{currentfill}{rgb}{0.917647,0.917647,0.949020}%
\pgfsetfillcolor{currentfill}%
\pgfsetlinewidth{0.000000pt}%
\definecolor{currentstroke}{rgb}{0.000000,0.000000,0.000000}%
\pgfsetstrokecolor{currentstroke}%
\pgfsetstrokeopacity{0.000000}%
\pgfsetdash{}{0pt}%
\pgfpathmoveto{\pgfqpoint{0.150000in}{0.150000in}}%
\pgfpathlineto{\pgfqpoint{5.850000in}{0.150000in}}%
\pgfpathlineto{\pgfqpoint{5.850000in}{2.350000in}}%
\pgfpathlineto{\pgfqpoint{0.150000in}{2.350000in}}%
\pgfpathclose%
\pgfusepath{fill}%
\end{pgfscope}%
\begin{pgfscope}%
\pgfpathrectangle{\pgfqpoint{0.150000in}{0.150000in}}{\pgfqpoint{5.700000in}{2.200000in}}%
\pgfusepath{clip}%
\pgfsetbuttcap%
\pgfsetroundjoin%
\definecolor{currentfill}{rgb}{0.000000,0.000000,0.000000}%
\pgfsetfillcolor{currentfill}%
\pgfsetlinewidth{1.505625pt}%
\definecolor{currentstroke}{rgb}{0.000000,0.000000,0.000000}%
\pgfsetstrokecolor{currentstroke}%
\pgfsetdash{}{0pt}%
\pgfpathmoveto{\pgfqpoint{1.105833in}{1.234693in}}%
\pgfpathlineto{\pgfqpoint{1.189167in}{1.318027in}}%
\pgfpathmoveto{\pgfqpoint{1.105833in}{1.318027in}}%
\pgfpathlineto{\pgfqpoint{1.189167in}{1.234693in}}%
\pgfusepath{stroke,fill}%
\end{pgfscope}%
\begin{pgfscope}%
\pgfpathrectangle{\pgfqpoint{0.150000in}{0.150000in}}{\pgfqpoint{5.700000in}{2.200000in}}%
\pgfusepath{clip}%
\pgfsetbuttcap%
\pgfsetroundjoin%
\definecolor{currentfill}{rgb}{0.000000,0.000000,0.000000}%
\pgfsetfillcolor{currentfill}%
\pgfsetlinewidth{1.505625pt}%
\definecolor{currentstroke}{rgb}{0.000000,0.000000,0.000000}%
\pgfsetstrokecolor{currentstroke}%
\pgfsetdash{}{0pt}%
\pgfpathmoveto{\pgfqpoint{5.333333in}{0.888654in}}%
\pgfpathlineto{\pgfqpoint{5.416667in}{0.971987in}}%
\pgfpathmoveto{\pgfqpoint{5.333333in}{0.971987in}}%
\pgfpathlineto{\pgfqpoint{5.416667in}{0.888654in}}%
\pgfusepath{stroke,fill}%
\end{pgfscope}%
\begin{pgfscope}%
\pgfpathrectangle{\pgfqpoint{0.150000in}{0.150000in}}{\pgfqpoint{5.700000in}{2.200000in}}%
\pgfusepath{clip}%
\pgfsetbuttcap%
\pgfsetroundjoin%
\definecolor{currentfill}{rgb}{0.447059,0.623529,0.811765}%
\pgfsetfillcolor{currentfill}%
\pgfsetfillopacity{0.200000}%
\pgfsetlinewidth{0.501875pt}%
\definecolor{currentstroke}{rgb}{0.125490,0.290196,0.529412}%
\pgfsetstrokecolor{currentstroke}%
\pgfsetstrokeopacity{0.200000}%
\pgfsetdash{}{0pt}%
\pgfpathmoveto{\pgfqpoint{0.090625in}{1.253372in}}%
\pgfpathlineto{\pgfqpoint{0.090625in}{0.636116in}}%
\pgfpathlineto{\pgfqpoint{0.122491in}{0.652545in}}%
\pgfpathlineto{\pgfqpoint{0.154356in}{0.669530in}}%
\pgfpathlineto{\pgfqpoint{0.186222in}{0.687056in}}%
\pgfpathlineto{\pgfqpoint{0.218087in}{0.705104in}}%
\pgfpathlineto{\pgfqpoint{0.249953in}{0.723653in}}%
\pgfpathlineto{\pgfqpoint{0.281818in}{0.742678in}}%
\pgfpathlineto{\pgfqpoint{0.313684in}{0.762151in}}%
\pgfpathlineto{\pgfqpoint{0.345550in}{0.782040in}}%
\pgfpathlineto{\pgfqpoint{0.377415in}{0.802311in}}%
\pgfpathlineto{\pgfqpoint{0.409281in}{0.822927in}}%
\pgfpathlineto{\pgfqpoint{0.441146in}{0.843846in}}%
\pgfpathlineto{\pgfqpoint{0.473012in}{0.865026in}}%
\pgfpathlineto{\pgfqpoint{0.504878in}{0.886419in}}%
\pgfpathlineto{\pgfqpoint{0.536743in}{0.907976in}}%
\pgfpathlineto{\pgfqpoint{0.568609in}{0.929646in}}%
\pgfpathlineto{\pgfqpoint{0.600474in}{0.951376in}}%
\pgfpathlineto{\pgfqpoint{0.632340in}{0.973110in}}%
\pgfpathlineto{\pgfqpoint{0.664205in}{0.994791in}}%
\pgfpathlineto{\pgfqpoint{0.696071in}{1.016361in}}%
\pgfpathlineto{\pgfqpoint{0.727937in}{1.037761in}}%
\pgfpathlineto{\pgfqpoint{0.759802in}{1.058930in}}%
\pgfpathlineto{\pgfqpoint{0.791668in}{1.079809in}}%
\pgfpathlineto{\pgfqpoint{0.823533in}{1.100339in}}%
\pgfpathlineto{\pgfqpoint{0.855399in}{1.120458in}}%
\pgfpathlineto{\pgfqpoint{0.887264in}{1.140110in}}%
\pgfpathlineto{\pgfqpoint{0.919130in}{1.159237in}}%
\pgfpathlineto{\pgfqpoint{0.950996in}{1.177783in}}%
\pgfpathlineto{\pgfqpoint{0.982861in}{1.195695in}}%
\pgfpathlineto{\pgfqpoint{1.014727in}{1.212921in}}%
\pgfpathlineto{\pgfqpoint{1.046592in}{1.229412in}}%
\pgfpathlineto{\pgfqpoint{1.078458in}{1.245118in}}%
\pgfpathlineto{\pgfqpoint{1.110323in}{1.259984in}}%
\pgfpathlineto{\pgfqpoint{1.142189in}{1.273628in}}%
\pgfpathlineto{\pgfqpoint{1.174055in}{1.264732in}}%
\pgfpathlineto{\pgfqpoint{1.205920in}{1.250171in}}%
\pgfpathlineto{\pgfqpoint{1.237786in}{1.234737in}}%
\pgfpathlineto{\pgfqpoint{1.269651in}{1.218503in}}%
\pgfpathlineto{\pgfqpoint{1.301517in}{1.201516in}}%
\pgfpathlineto{\pgfqpoint{1.333383in}{1.183828in}}%
\pgfpathlineto{\pgfqpoint{1.365248in}{1.165487in}}%
\pgfpathlineto{\pgfqpoint{1.397114in}{1.146548in}}%
\pgfpathlineto{\pgfqpoint{1.428979in}{1.127065in}}%
\pgfpathlineto{\pgfqpoint{1.460845in}{1.107095in}}%
\pgfpathlineto{\pgfqpoint{1.492710in}{1.086697in}}%
\pgfpathlineto{\pgfqpoint{1.524576in}{1.065928in}}%
\pgfpathlineto{\pgfqpoint{1.556442in}{1.044849in}}%
\pgfpathlineto{\pgfqpoint{1.588307in}{1.023520in}}%
\pgfpathlineto{\pgfqpoint{1.620173in}{1.002001in}}%
\pgfpathlineto{\pgfqpoint{1.652038in}{0.980351in}}%
\pgfpathlineto{\pgfqpoint{1.683904in}{0.958630in}}%
\pgfpathlineto{\pgfqpoint{1.715769in}{0.936894in}}%
\pgfpathlineto{\pgfqpoint{1.747635in}{0.915199in}}%
\pgfpathlineto{\pgfqpoint{1.779501in}{0.893600in}}%
\pgfpathlineto{\pgfqpoint{1.811366in}{0.872149in}}%
\pgfpathlineto{\pgfqpoint{1.843232in}{0.850896in}}%
\pgfpathlineto{\pgfqpoint{1.875097in}{0.829888in}}%
\pgfpathlineto{\pgfqpoint{1.906963in}{0.809170in}}%
\pgfpathlineto{\pgfqpoint{1.938829in}{0.788783in}}%
\pgfpathlineto{\pgfqpoint{1.970694in}{0.768768in}}%
\pgfpathlineto{\pgfqpoint{2.002560in}{0.749158in}}%
\pgfpathlineto{\pgfqpoint{2.034425in}{0.729987in}}%
\pgfpathlineto{\pgfqpoint{2.066291in}{0.711285in}}%
\pgfpathlineto{\pgfqpoint{2.098156in}{0.693076in}}%
\pgfpathlineto{\pgfqpoint{2.130022in}{0.675385in}}%
\pgfpathlineto{\pgfqpoint{2.161888in}{0.658230in}}%
\pgfpathlineto{\pgfqpoint{2.193753in}{0.641628in}}%
\pgfpathlineto{\pgfqpoint{2.225619in}{0.625592in}}%
\pgfpathlineto{\pgfqpoint{2.257484in}{0.610133in}}%
\pgfpathlineto{\pgfqpoint{2.289350in}{0.595259in}}%
\pgfpathlineto{\pgfqpoint{2.321215in}{0.580973in}}%
\pgfpathlineto{\pgfqpoint{2.353081in}{0.567280in}}%
\pgfpathlineto{\pgfqpoint{2.384947in}{0.554177in}}%
\pgfpathlineto{\pgfqpoint{2.416812in}{0.541664in}}%
\pgfpathlineto{\pgfqpoint{2.448678in}{0.529735in}}%
\pgfpathlineto{\pgfqpoint{2.480543in}{0.518384in}}%
\pgfpathlineto{\pgfqpoint{2.512409in}{0.507604in}}%
\pgfpathlineto{\pgfqpoint{2.544274in}{0.497384in}}%
\pgfpathlineto{\pgfqpoint{2.576140in}{0.487714in}}%
\pgfpathlineto{\pgfqpoint{2.608006in}{0.478582in}}%
\pgfpathlineto{\pgfqpoint{2.639871in}{0.469974in}}%
\pgfpathlineto{\pgfqpoint{2.671737in}{0.461877in}}%
\pgfpathlineto{\pgfqpoint{2.703602in}{0.454276in}}%
\pgfpathlineto{\pgfqpoint{2.735468in}{0.447156in}}%
\pgfpathlineto{\pgfqpoint{2.767334in}{0.440502in}}%
\pgfpathlineto{\pgfqpoint{2.799199in}{0.434297in}}%
\pgfpathlineto{\pgfqpoint{2.831065in}{0.428526in}}%
\pgfpathlineto{\pgfqpoint{2.862930in}{0.423174in}}%
\pgfpathlineto{\pgfqpoint{2.894796in}{0.418225in}}%
\pgfpathlineto{\pgfqpoint{2.926661in}{0.413662in}}%
\pgfpathlineto{\pgfqpoint{2.958527in}{0.409472in}}%
\pgfpathlineto{\pgfqpoint{2.990393in}{0.405640in}}%
\pgfpathlineto{\pgfqpoint{3.022258in}{0.402152in}}%
\pgfpathlineto{\pgfqpoint{3.054124in}{0.398994in}}%
\pgfpathlineto{\pgfqpoint{3.085989in}{0.396153in}}%
\pgfpathlineto{\pgfqpoint{3.117855in}{0.393618in}}%
\pgfpathlineto{\pgfqpoint{3.149720in}{0.391376in}}%
\pgfpathlineto{\pgfqpoint{3.181586in}{0.389418in}}%
\pgfpathlineto{\pgfqpoint{3.213452in}{0.387733in}}%
\pgfpathlineto{\pgfqpoint{3.245317in}{0.386314in}}%
\pgfpathlineto{\pgfqpoint{3.277183in}{0.385151in}}%
\pgfpathlineto{\pgfqpoint{3.309048in}{0.384239in}}%
\pgfpathlineto{\pgfqpoint{3.340914in}{0.383570in}}%
\pgfpathlineto{\pgfqpoint{3.372780in}{0.383141in}}%
\pgfpathlineto{\pgfqpoint{3.404645in}{0.382946in}}%
\pgfpathlineto{\pgfqpoint{3.436511in}{0.382983in}}%
\pgfpathlineto{\pgfqpoint{3.468376in}{0.383250in}}%
\pgfpathlineto{\pgfqpoint{3.500242in}{0.383745in}}%
\pgfpathlineto{\pgfqpoint{3.532107in}{0.384469in}}%
\pgfpathlineto{\pgfqpoint{3.563973in}{0.385421in}}%
\pgfpathlineto{\pgfqpoint{3.595839in}{0.386605in}}%
\pgfpathlineto{\pgfqpoint{3.627704in}{0.388022in}}%
\pgfpathlineto{\pgfqpoint{3.659570in}{0.389677in}}%
\pgfpathlineto{\pgfqpoint{3.691435in}{0.391573in}}%
\pgfpathlineto{\pgfqpoint{3.723301in}{0.393715in}}%
\pgfpathlineto{\pgfqpoint{3.755166in}{0.396111in}}%
\pgfpathlineto{\pgfqpoint{3.787032in}{0.398766in}}%
\pgfpathlineto{\pgfqpoint{3.818898in}{0.401688in}}%
\pgfpathlineto{\pgfqpoint{3.850763in}{0.404886in}}%
\pgfpathlineto{\pgfqpoint{3.882629in}{0.408367in}}%
\pgfpathlineto{\pgfqpoint{3.914494in}{0.412142in}}%
\pgfpathlineto{\pgfqpoint{3.946360in}{0.416218in}}%
\pgfpathlineto{\pgfqpoint{3.978226in}{0.420608in}}%
\pgfpathlineto{\pgfqpoint{4.010091in}{0.425319in}}%
\pgfpathlineto{\pgfqpoint{4.041957in}{0.430364in}}%
\pgfpathlineto{\pgfqpoint{4.073822in}{0.435751in}}%
\pgfpathlineto{\pgfqpoint{4.105688in}{0.441490in}}%
\pgfpathlineto{\pgfqpoint{4.137553in}{0.447592in}}%
\pgfpathlineto{\pgfqpoint{4.169419in}{0.454066in}}%
\pgfpathlineto{\pgfqpoint{4.201285in}{0.460921in}}%
\pgfpathlineto{\pgfqpoint{4.233150in}{0.468164in}}%
\pgfpathlineto{\pgfqpoint{4.265016in}{0.475803in}}%
\pgfpathlineto{\pgfqpoint{4.296881in}{0.483845in}}%
\pgfpathlineto{\pgfqpoint{4.328747in}{0.492295in}}%
\pgfpathlineto{\pgfqpoint{4.360612in}{0.501156in}}%
\pgfpathlineto{\pgfqpoint{4.392478in}{0.510432in}}%
\pgfpathlineto{\pgfqpoint{4.424344in}{0.520124in}}%
\pgfpathlineto{\pgfqpoint{4.456209in}{0.530231in}}%
\pgfpathlineto{\pgfqpoint{4.488075in}{0.540751in}}%
\pgfpathlineto{\pgfqpoint{4.519940in}{0.551680in}}%
\pgfpathlineto{\pgfqpoint{4.551806in}{0.563012in}}%
\pgfpathlineto{\pgfqpoint{4.583671in}{0.574739in}}%
\pgfpathlineto{\pgfqpoint{4.615537in}{0.586852in}}%
\pgfpathlineto{\pgfqpoint{4.647403in}{0.599336in}}%
\pgfpathlineto{\pgfqpoint{4.679268in}{0.612179in}}%
\pgfpathlineto{\pgfqpoint{4.711134in}{0.625364in}}%
\pgfpathlineto{\pgfqpoint{4.742999in}{0.638872in}}%
\pgfpathlineto{\pgfqpoint{4.774865in}{0.652682in}}%
\pgfpathlineto{\pgfqpoint{4.806731in}{0.666771in}}%
\pgfpathlineto{\pgfqpoint{4.838596in}{0.681115in}}%
\pgfpathlineto{\pgfqpoint{4.870462in}{0.695685in}}%
\pgfpathlineto{\pgfqpoint{4.902327in}{0.710454in}}%
\pgfpathlineto{\pgfqpoint{4.934193in}{0.725391in}}%
\pgfpathlineto{\pgfqpoint{4.966058in}{0.740464in}}%
\pgfpathlineto{\pgfqpoint{4.997924in}{0.755640in}}%
\pgfpathlineto{\pgfqpoint{5.029790in}{0.770883in}}%
\pgfpathlineto{\pgfqpoint{5.061655in}{0.786158in}}%
\pgfpathlineto{\pgfqpoint{5.093521in}{0.801429in}}%
\pgfpathlineto{\pgfqpoint{5.125386in}{0.816658in}}%
\pgfpathlineto{\pgfqpoint{5.157252in}{0.831808in}}%
\pgfpathlineto{\pgfqpoint{5.189117in}{0.846841in}}%
\pgfpathlineto{\pgfqpoint{5.220983in}{0.861718in}}%
\pgfpathlineto{\pgfqpoint{5.252849in}{0.876404in}}%
\pgfpathlineto{\pgfqpoint{5.284714in}{0.890858in}}%
\pgfpathlineto{\pgfqpoint{5.316580in}{0.905039in}}%
\pgfpathlineto{\pgfqpoint{5.348445in}{0.918883in}}%
\pgfpathlineto{\pgfqpoint{5.380311in}{0.927600in}}%
\pgfpathlineto{\pgfqpoint{5.412177in}{0.914314in}}%
\pgfpathlineto{\pgfqpoint{5.444042in}{0.900345in}}%
\pgfpathlineto{\pgfqpoint{5.475908in}{0.886067in}}%
\pgfpathlineto{\pgfqpoint{5.507773in}{0.871531in}}%
\pgfpathlineto{\pgfqpoint{5.539639in}{0.856777in}}%
\pgfpathlineto{\pgfqpoint{5.571504in}{0.841843in}}%
\pgfpathlineto{\pgfqpoint{5.603370in}{0.826766in}}%
\pgfpathlineto{\pgfqpoint{5.635236in}{0.811585in}}%
\pgfpathlineto{\pgfqpoint{5.667101in}{0.796337in}}%
\pgfpathlineto{\pgfqpoint{5.698967in}{0.781060in}}%
\pgfpathlineto{\pgfqpoint{5.730832in}{0.765790in}}%
\pgfpathlineto{\pgfqpoint{5.762698in}{0.750565in}}%
\pgfpathlineto{\pgfqpoint{5.794563in}{0.735418in}}%
\pgfpathlineto{\pgfqpoint{5.826429in}{0.720385in}}%
\pgfpathlineto{\pgfqpoint{5.858295in}{0.705499in}}%
\pgfpathlineto{\pgfqpoint{5.890160in}{0.690790in}}%
\pgfpathlineto{\pgfqpoint{5.922026in}{0.676289in}}%
\pgfpathlineto{\pgfqpoint{5.953891in}{0.662024in}}%
\pgfpathlineto{\pgfqpoint{5.985757in}{0.648021in}}%
\pgfpathlineto{\pgfqpoint{6.017622in}{0.634305in}}%
\pgfpathlineto{\pgfqpoint{6.049488in}{0.620896in}}%
\pgfpathlineto{\pgfqpoint{6.081354in}{0.607817in}}%
\pgfpathlineto{\pgfqpoint{6.113219in}{0.595083in}}%
\pgfpathlineto{\pgfqpoint{6.145085in}{0.582712in}}%
\pgfpathlineto{\pgfqpoint{6.176950in}{0.570716in}}%
\pgfpathlineto{\pgfqpoint{6.208816in}{0.559106in}}%
\pgfpathlineto{\pgfqpoint{6.240682in}{0.547893in}}%
\pgfpathlineto{\pgfqpoint{6.272547in}{0.537082in}}%
\pgfpathlineto{\pgfqpoint{6.304413in}{0.526679in}}%
\pgfpathlineto{\pgfqpoint{6.336278in}{0.516686in}}%
\pgfpathlineto{\pgfqpoint{6.368144in}{0.507105in}}%
\pgfpathlineto{\pgfqpoint{6.400009in}{0.497935in}}%
\pgfpathlineto{\pgfqpoint{6.431875in}{0.489174in}}%
\pgfpathlineto{\pgfqpoint{6.431875in}{1.106431in}}%
\pgfpathlineto{\pgfqpoint{6.431875in}{1.106431in}}%
\pgfpathlineto{\pgfqpoint{6.400009in}{1.107878in}}%
\pgfpathlineto{\pgfqpoint{6.368144in}{1.109111in}}%
\pgfpathlineto{\pgfqpoint{6.336278in}{1.110107in}}%
\pgfpathlineto{\pgfqpoint{6.304413in}{1.110837in}}%
\pgfpathlineto{\pgfqpoint{6.272547in}{1.111277in}}%
\pgfpathlineto{\pgfqpoint{6.240682in}{1.111400in}}%
\pgfpathlineto{\pgfqpoint{6.208816in}{1.111181in}}%
\pgfpathlineto{\pgfqpoint{6.176950in}{1.110594in}}%
\pgfpathlineto{\pgfqpoint{6.145085in}{1.109614in}}%
\pgfpathlineto{\pgfqpoint{6.113219in}{1.108217in}}%
\pgfpathlineto{\pgfqpoint{6.081354in}{1.106380in}}%
\pgfpathlineto{\pgfqpoint{6.049488in}{1.104083in}}%
\pgfpathlineto{\pgfqpoint{6.017622in}{1.101304in}}%
\pgfpathlineto{\pgfqpoint{5.985757in}{1.098024in}}%
\pgfpathlineto{\pgfqpoint{5.953891in}{1.094228in}}%
\pgfpathlineto{\pgfqpoint{5.922026in}{1.089900in}}%
\pgfpathlineto{\pgfqpoint{5.890160in}{1.085028in}}%
\pgfpathlineto{\pgfqpoint{5.858295in}{1.079601in}}%
\pgfpathlineto{\pgfqpoint{5.826429in}{1.073611in}}%
\pgfpathlineto{\pgfqpoint{5.794563in}{1.067053in}}%
\pgfpathlineto{\pgfqpoint{5.762698in}{1.059924in}}%
\pgfpathlineto{\pgfqpoint{5.730832in}{1.052224in}}%
\pgfpathlineto{\pgfqpoint{5.698967in}{1.043956in}}%
\pgfpathlineto{\pgfqpoint{5.667101in}{1.035125in}}%
\pgfpathlineto{\pgfqpoint{5.635236in}{1.025740in}}%
\pgfpathlineto{\pgfqpoint{5.603370in}{1.015812in}}%
\pgfpathlineto{\pgfqpoint{5.571504in}{1.005354in}}%
\pgfpathlineto{\pgfqpoint{5.539639in}{0.994383in}}%
\pgfpathlineto{\pgfqpoint{5.507773in}{0.982920in}}%
\pgfpathlineto{\pgfqpoint{5.475908in}{0.970987in}}%
\pgfpathlineto{\pgfqpoint{5.444042in}{0.958611in}}%
\pgfpathlineto{\pgfqpoint{5.412177in}{0.945834in}}%
\pgfpathlineto{\pgfqpoint{5.380311in}{0.933027in}}%
\pgfpathlineto{\pgfqpoint{5.348445in}{0.941505in}}%
\pgfpathlineto{\pgfqpoint{5.316580in}{0.954394in}}%
\pgfpathlineto{\pgfqpoint{5.284714in}{0.966910in}}%
\pgfpathlineto{\pgfqpoint{5.252849in}{0.978994in}}%
\pgfpathlineto{\pgfqpoint{5.220983in}{0.990617in}}%
\pgfpathlineto{\pgfqpoint{5.189117in}{1.001755in}}%
\pgfpathlineto{\pgfqpoint{5.157252in}{1.012386in}}%
\pgfpathlineto{\pgfqpoint{5.125386in}{1.022493in}}%
\pgfpathlineto{\pgfqpoint{5.093521in}{1.032062in}}%
\pgfpathlineto{\pgfqpoint{5.061655in}{1.041079in}}%
\pgfpathlineto{\pgfqpoint{5.029790in}{1.049537in}}%
\pgfpathlineto{\pgfqpoint{4.997924in}{1.057427in}}%
\pgfpathlineto{\pgfqpoint{4.966058in}{1.064748in}}%
\pgfpathlineto{\pgfqpoint{4.934193in}{1.071498in}}%
\pgfpathlineto{\pgfqpoint{4.902327in}{1.077678in}}%
\pgfpathlineto{\pgfqpoint{4.870462in}{1.083295in}}%
\pgfpathlineto{\pgfqpoint{4.838596in}{1.088354in}}%
\pgfpathlineto{\pgfqpoint{4.806731in}{1.092865in}}%
\pgfpathlineto{\pgfqpoint{4.774865in}{1.096841in}}%
\pgfpathlineto{\pgfqpoint{4.742999in}{1.100295in}}%
\pgfpathlineto{\pgfqpoint{4.711134in}{1.103245in}}%
\pgfpathlineto{\pgfqpoint{4.679268in}{1.105707in}}%
\pgfpathlineto{\pgfqpoint{4.647403in}{1.107703in}}%
\pgfpathlineto{\pgfqpoint{4.615537in}{1.109252in}}%
\pgfpathlineto{\pgfqpoint{4.583671in}{1.110379in}}%
\pgfpathlineto{\pgfqpoint{4.551806in}{1.111107in}}%
\pgfpathlineto{\pgfqpoint{4.519940in}{1.111461in}}%
\pgfpathlineto{\pgfqpoint{4.488075in}{1.111465in}}%
\pgfpathlineto{\pgfqpoint{4.456209in}{1.111148in}}%
\pgfpathlineto{\pgfqpoint{4.424344in}{1.110534in}}%
\pgfpathlineto{\pgfqpoint{4.392478in}{1.109650in}}%
\pgfpathlineto{\pgfqpoint{4.360612in}{1.108524in}}%
\pgfpathlineto{\pgfqpoint{4.328747in}{1.107181in}}%
\pgfpathlineto{\pgfqpoint{4.296881in}{1.105647in}}%
\pgfpathlineto{\pgfqpoint{4.265016in}{1.103949in}}%
\pgfpathlineto{\pgfqpoint{4.233150in}{1.102110in}}%
\pgfpathlineto{\pgfqpoint{4.201285in}{1.100155in}}%
\pgfpathlineto{\pgfqpoint{4.169419in}{1.098108in}}%
\pgfpathlineto{\pgfqpoint{4.137553in}{1.095990in}}%
\pgfpathlineto{\pgfqpoint{4.105688in}{1.093824in}}%
\pgfpathlineto{\pgfqpoint{4.073822in}{1.091629in}}%
\pgfpathlineto{\pgfqpoint{4.041957in}{1.089424in}}%
\pgfpathlineto{\pgfqpoint{4.010091in}{1.087228in}}%
\pgfpathlineto{\pgfqpoint{3.978226in}{1.085057in}}%
\pgfpathlineto{\pgfqpoint{3.946360in}{1.082927in}}%
\pgfpathlineto{\pgfqpoint{3.914494in}{1.080853in}}%
\pgfpathlineto{\pgfqpoint{3.882629in}{1.078848in}}%
\pgfpathlineto{\pgfqpoint{3.850763in}{1.076925in}}%
\pgfpathlineto{\pgfqpoint{3.818898in}{1.075095in}}%
\pgfpathlineto{\pgfqpoint{3.787032in}{1.073368in}}%
\pgfpathlineto{\pgfqpoint{3.755166in}{1.071756in}}%
\pgfpathlineto{\pgfqpoint{3.723301in}{1.070265in}}%
\pgfpathlineto{\pgfqpoint{3.691435in}{1.068906in}}%
\pgfpathlineto{\pgfqpoint{3.659570in}{1.067685in}}%
\pgfpathlineto{\pgfqpoint{3.627704in}{1.066609in}}%
\pgfpathlineto{\pgfqpoint{3.595839in}{1.065687in}}%
\pgfpathlineto{\pgfqpoint{3.563973in}{1.064923in}}%
\pgfpathlineto{\pgfqpoint{3.532107in}{1.064324in}}%
\pgfpathlineto{\pgfqpoint{3.500242in}{1.063896in}}%
\pgfpathlineto{\pgfqpoint{3.468376in}{1.063646in}}%
\pgfpathlineto{\pgfqpoint{3.436511in}{1.063579in}}%
\pgfpathlineto{\pgfqpoint{3.404645in}{1.063701in}}%
\pgfpathlineto{\pgfqpoint{3.372780in}{1.064018in}}%
\pgfpathlineto{\pgfqpoint{3.340914in}{1.064537in}}%
\pgfpathlineto{\pgfqpoint{3.309048in}{1.065263in}}%
\pgfpathlineto{\pgfqpoint{3.277183in}{1.066205in}}%
\pgfpathlineto{\pgfqpoint{3.245317in}{1.067367in}}%
\pgfpathlineto{\pgfqpoint{3.213452in}{1.068758in}}%
\pgfpathlineto{\pgfqpoint{3.181586in}{1.070384in}}%
\pgfpathlineto{\pgfqpoint{3.149720in}{1.072253in}}%
\pgfpathlineto{\pgfqpoint{3.117855in}{1.074372in}}%
\pgfpathlineto{\pgfqpoint{3.085989in}{1.076749in}}%
\pgfpathlineto{\pgfqpoint{3.054124in}{1.079390in}}%
\pgfpathlineto{\pgfqpoint{3.022258in}{1.082303in}}%
\pgfpathlineto{\pgfqpoint{2.990393in}{1.085495in}}%
\pgfpathlineto{\pgfqpoint{2.958527in}{1.088974in}}%
\pgfpathlineto{\pgfqpoint{2.926661in}{1.092744in}}%
\pgfpathlineto{\pgfqpoint{2.894796in}{1.096812in}}%
\pgfpathlineto{\pgfqpoint{2.862930in}{1.101182in}}%
\pgfpathlineto{\pgfqpoint{2.831065in}{1.105860in}}%
\pgfpathlineto{\pgfqpoint{2.799199in}{1.110847in}}%
\pgfpathlineto{\pgfqpoint{2.767334in}{1.116146in}}%
\pgfpathlineto{\pgfqpoint{2.735468in}{1.121759in}}%
\pgfpathlineto{\pgfqpoint{2.703602in}{1.127683in}}%
\pgfpathlineto{\pgfqpoint{2.671737in}{1.133916in}}%
\pgfpathlineto{\pgfqpoint{2.639871in}{1.140455in}}%
\pgfpathlineto{\pgfqpoint{2.608006in}{1.147293in}}%
\pgfpathlineto{\pgfqpoint{2.576140in}{1.154423in}}%
\pgfpathlineto{\pgfqpoint{2.544274in}{1.161833in}}%
\pgfpathlineto{\pgfqpoint{2.512409in}{1.169512in}}%
\pgfpathlineto{\pgfqpoint{2.480543in}{1.177445in}}%
\pgfpathlineto{\pgfqpoint{2.448678in}{1.185613in}}%
\pgfpathlineto{\pgfqpoint{2.416812in}{1.193997in}}%
\pgfpathlineto{\pgfqpoint{2.384947in}{1.202575in}}%
\pgfpathlineto{\pgfqpoint{2.353081in}{1.211321in}}%
\pgfpathlineto{\pgfqpoint{2.321215in}{1.220208in}}%
\pgfpathlineto{\pgfqpoint{2.289350in}{1.229205in}}%
\pgfpathlineto{\pgfqpoint{2.257484in}{1.238278in}}%
\pgfpathlineto{\pgfqpoint{2.225619in}{1.247394in}}%
\pgfpathlineto{\pgfqpoint{2.193753in}{1.256514in}}%
\pgfpathlineto{\pgfqpoint{2.161888in}{1.265597in}}%
\pgfpathlineto{\pgfqpoint{2.130022in}{1.274603in}}%
\pgfpathlineto{\pgfqpoint{2.098156in}{1.283486in}}%
\pgfpathlineto{\pgfqpoint{2.066291in}{1.292201in}}%
\pgfpathlineto{\pgfqpoint{2.034425in}{1.300702in}}%
\pgfpathlineto{\pgfqpoint{2.002560in}{1.308938in}}%
\pgfpathlineto{\pgfqpoint{1.970694in}{1.316862in}}%
\pgfpathlineto{\pgfqpoint{1.938829in}{1.324423in}}%
\pgfpathlineto{\pgfqpoint{1.906963in}{1.331570in}}%
\pgfpathlineto{\pgfqpoint{1.875097in}{1.338254in}}%
\pgfpathlineto{\pgfqpoint{1.843232in}{1.344424in}}%
\pgfpathlineto{\pgfqpoint{1.811366in}{1.350029in}}%
\pgfpathlineto{\pgfqpoint{1.779501in}{1.355023in}}%
\pgfpathlineto{\pgfqpoint{1.747635in}{1.359358in}}%
\pgfpathlineto{\pgfqpoint{1.715769in}{1.362987in}}%
\pgfpathlineto{\pgfqpoint{1.683904in}{1.365869in}}%
\pgfpathlineto{\pgfqpoint{1.652038in}{1.367961in}}%
\pgfpathlineto{\pgfqpoint{1.620173in}{1.369225in}}%
\pgfpathlineto{\pgfqpoint{1.588307in}{1.369627in}}%
\pgfpathlineto{\pgfqpoint{1.556442in}{1.369133in}}%
\pgfpathlineto{\pgfqpoint{1.524576in}{1.367716in}}%
\pgfpathlineto{\pgfqpoint{1.492710in}{1.365351in}}%
\pgfpathlineto{\pgfqpoint{1.460845in}{1.362017in}}%
\pgfpathlineto{\pgfqpoint{1.428979in}{1.357698in}}%
\pgfpathlineto{\pgfqpoint{1.397114in}{1.352384in}}%
\pgfpathlineto{\pgfqpoint{1.365248in}{1.346066in}}%
\pgfpathlineto{\pgfqpoint{1.333383in}{1.338742in}}%
\pgfpathlineto{\pgfqpoint{1.301517in}{1.330415in}}%
\pgfpathlineto{\pgfqpoint{1.269651in}{1.321093in}}%
\pgfpathlineto{\pgfqpoint{1.237786in}{1.310789in}}%
\pgfpathlineto{\pgfqpoint{1.205920in}{1.299525in}}%
\pgfpathlineto{\pgfqpoint{1.174055in}{1.287354in}}%
\pgfpathlineto{\pgfqpoint{1.142189in}{1.279055in}}%
\pgfpathlineto{\pgfqpoint{1.110323in}{1.291504in}}%
\pgfpathlineto{\pgfqpoint{1.078458in}{1.303384in}}%
\pgfpathlineto{\pgfqpoint{1.046592in}{1.314331in}}%
\pgfpathlineto{\pgfqpoint{1.014727in}{1.324310in}}%
\pgfpathlineto{\pgfqpoint{0.982861in}{1.333301in}}%
\pgfpathlineto{\pgfqpoint{0.950996in}{1.341294in}}%
\pgfpathlineto{\pgfqpoint{0.919130in}{1.348282in}}%
\pgfpathlineto{\pgfqpoint{0.887264in}{1.354265in}}%
\pgfpathlineto{\pgfqpoint{0.855399in}{1.359247in}}%
\pgfpathlineto{\pgfqpoint{0.823533in}{1.363235in}}%
\pgfpathlineto{\pgfqpoint{0.791668in}{1.366243in}}%
\pgfpathlineto{\pgfqpoint{0.759802in}{1.368290in}}%
\pgfpathlineto{\pgfqpoint{0.727937in}{1.369395in}}%
\pgfpathlineto{\pgfqpoint{0.696071in}{1.369587in}}%
\pgfpathlineto{\pgfqpoint{0.664205in}{1.368893in}}%
\pgfpathlineto{\pgfqpoint{0.632340in}{1.367348in}}%
\pgfpathlineto{\pgfqpoint{0.600474in}{1.364987in}}%
\pgfpathlineto{\pgfqpoint{0.568609in}{1.361850in}}%
\pgfpathlineto{\pgfqpoint{0.536743in}{1.357979in}}%
\pgfpathlineto{\pgfqpoint{0.504878in}{1.353417in}}%
\pgfpathlineto{\pgfqpoint{0.473012in}{1.348212in}}%
\pgfpathlineto{\pgfqpoint{0.441146in}{1.342410in}}%
\pgfpathlineto{\pgfqpoint{0.409281in}{1.336061in}}%
\pgfpathlineto{\pgfqpoint{0.377415in}{1.329213in}}%
\pgfpathlineto{\pgfqpoint{0.345550in}{1.321918in}}%
\pgfpathlineto{\pgfqpoint{0.313684in}{1.314225in}}%
\pgfpathlineto{\pgfqpoint{0.281818in}{1.306185in}}%
\pgfpathlineto{\pgfqpoint{0.249953in}{1.297848in}}%
\pgfpathlineto{\pgfqpoint{0.218087in}{1.289263in}}%
\pgfpathlineto{\pgfqpoint{0.186222in}{1.280477in}}%
\pgfpathlineto{\pgfqpoint{0.154356in}{1.271537in}}%
\pgfpathlineto{\pgfqpoint{0.122491in}{1.262487in}}%
\pgfpathlineto{\pgfqpoint{0.090625in}{1.253372in}}%
\pgfpathclose%
\pgfusepath{stroke,fill}%
\end{pgfscope}%
\begin{pgfscope}%
\pgfpathrectangle{\pgfqpoint{0.150000in}{0.150000in}}{\pgfqpoint{5.700000in}{2.200000in}}%
\pgfusepath{clip}%
\pgfsetbuttcap%
\pgfsetroundjoin%
\definecolor{currentfill}{rgb}{0.000000,0.501961,0.000000}%
\pgfsetfillcolor{currentfill}%
\pgfsetfillopacity{0.100000}%
\pgfsetlinewidth{1.003750pt}%
\definecolor{currentstroke}{rgb}{0.000000,0.501961,0.000000}%
\pgfsetstrokecolor{currentstroke}%
\pgfsetstrokeopacity{0.100000}%
\pgfsetdash{}{0pt}%
\pgfpathmoveto{\pgfqpoint{0.150000in}{0.150000in}}%
\pgfpathlineto{\pgfqpoint{0.150000in}{0.474362in}}%
\pgfpathlineto{\pgfqpoint{0.161423in}{0.475201in}}%
\pgfpathlineto{\pgfqpoint{0.172846in}{0.476012in}}%
\pgfpathlineto{\pgfqpoint{0.184269in}{0.476791in}}%
\pgfpathlineto{\pgfqpoint{0.195691in}{0.477538in}}%
\pgfpathlineto{\pgfqpoint{0.207114in}{0.478252in}}%
\pgfpathlineto{\pgfqpoint{0.218537in}{0.478930in}}%
\pgfpathlineto{\pgfqpoint{0.229960in}{0.479572in}}%
\pgfpathlineto{\pgfqpoint{0.241383in}{0.480177in}}%
\pgfpathlineto{\pgfqpoint{0.252806in}{0.480742in}}%
\pgfpathlineto{\pgfqpoint{0.264228in}{0.481266in}}%
\pgfpathlineto{\pgfqpoint{0.275651in}{0.481749in}}%
\pgfpathlineto{\pgfqpoint{0.287074in}{0.482189in}}%
\pgfpathlineto{\pgfqpoint{0.298497in}{0.482583in}}%
\pgfpathlineto{\pgfqpoint{0.309920in}{0.482932in}}%
\pgfpathlineto{\pgfqpoint{0.321343in}{0.483233in}}%
\pgfpathlineto{\pgfqpoint{0.332766in}{0.483486in}}%
\pgfpathlineto{\pgfqpoint{0.344188in}{0.483688in}}%
\pgfpathlineto{\pgfqpoint{0.355611in}{0.483840in}}%
\pgfpathlineto{\pgfqpoint{0.367034in}{0.483938in}}%
\pgfpathlineto{\pgfqpoint{0.378457in}{0.483982in}}%
\pgfpathlineto{\pgfqpoint{0.389880in}{0.483971in}}%
\pgfpathlineto{\pgfqpoint{0.401303in}{0.483903in}}%
\pgfpathlineto{\pgfqpoint{0.412725in}{0.483777in}}%
\pgfpathlineto{\pgfqpoint{0.424148in}{0.483592in}}%
\pgfpathlineto{\pgfqpoint{0.435571in}{0.483347in}}%
\pgfpathlineto{\pgfqpoint{0.446994in}{0.483040in}}%
\pgfpathlineto{\pgfqpoint{0.458417in}{0.482670in}}%
\pgfpathlineto{\pgfqpoint{0.469840in}{0.482236in}}%
\pgfpathlineto{\pgfqpoint{0.481263in}{0.481738in}}%
\pgfpathlineto{\pgfqpoint{0.492685in}{0.481172in}}%
\pgfpathlineto{\pgfqpoint{0.504108in}{0.480540in}}%
\pgfpathlineto{\pgfqpoint{0.515531in}{0.479839in}}%
\pgfpathlineto{\pgfqpoint{0.526954in}{0.479069in}}%
\pgfpathlineto{\pgfqpoint{0.538377in}{0.478229in}}%
\pgfpathlineto{\pgfqpoint{0.549800in}{0.477317in}}%
\pgfpathlineto{\pgfqpoint{0.561222in}{0.476332in}}%
\pgfpathlineto{\pgfqpoint{0.572645in}{0.475275in}}%
\pgfpathlineto{\pgfqpoint{0.584068in}{0.474144in}}%
\pgfpathlineto{\pgfqpoint{0.595491in}{0.472937in}}%
\pgfpathlineto{\pgfqpoint{0.606914in}{0.471655in}}%
\pgfpathlineto{\pgfqpoint{0.618337in}{0.470297in}}%
\pgfpathlineto{\pgfqpoint{0.629760in}{0.468862in}}%
\pgfpathlineto{\pgfqpoint{0.641182in}{0.467349in}}%
\pgfpathlineto{\pgfqpoint{0.652605in}{0.465757in}}%
\pgfpathlineto{\pgfqpoint{0.664028in}{0.464086in}}%
\pgfpathlineto{\pgfqpoint{0.675451in}{0.462336in}}%
\pgfpathlineto{\pgfqpoint{0.686874in}{0.460506in}}%
\pgfpathlineto{\pgfqpoint{0.698297in}{0.458596in}}%
\pgfpathlineto{\pgfqpoint{0.709719in}{0.456604in}}%
\pgfpathlineto{\pgfqpoint{0.721142in}{0.454532in}}%
\pgfpathlineto{\pgfqpoint{0.732565in}{0.452378in}}%
\pgfpathlineto{\pgfqpoint{0.743988in}{0.450142in}}%
\pgfpathlineto{\pgfqpoint{0.755411in}{0.447824in}}%
\pgfpathlineto{\pgfqpoint{0.766834in}{0.445424in}}%
\pgfpathlineto{\pgfqpoint{0.778257in}{0.442942in}}%
\pgfpathlineto{\pgfqpoint{0.789679in}{0.440377in}}%
\pgfpathlineto{\pgfqpoint{0.801102in}{0.437730in}}%
\pgfpathlineto{\pgfqpoint{0.812525in}{0.435001in}}%
\pgfpathlineto{\pgfqpoint{0.823948in}{0.432190in}}%
\pgfpathlineto{\pgfqpoint{0.835371in}{0.429297in}}%
\pgfpathlineto{\pgfqpoint{0.846794in}{0.426321in}}%
\pgfpathlineto{\pgfqpoint{0.858216in}{0.423264in}}%
\pgfpathlineto{\pgfqpoint{0.869639in}{0.420126in}}%
\pgfpathlineto{\pgfqpoint{0.881062in}{0.416907in}}%
\pgfpathlineto{\pgfqpoint{0.892485in}{0.413608in}}%
\pgfpathlineto{\pgfqpoint{0.903908in}{0.410228in}}%
\pgfpathlineto{\pgfqpoint{0.915331in}{0.406768in}}%
\pgfpathlineto{\pgfqpoint{0.926754in}{0.403230in}}%
\pgfpathlineto{\pgfqpoint{0.938176in}{0.399613in}}%
\pgfpathlineto{\pgfqpoint{0.949599in}{0.395919in}}%
\pgfpathlineto{\pgfqpoint{0.961022in}{0.392147in}}%
\pgfpathlineto{\pgfqpoint{0.972445in}{0.388299in}}%
\pgfpathlineto{\pgfqpoint{0.983868in}{0.384376in}}%
\pgfpathlineto{\pgfqpoint{0.995291in}{0.380378in}}%
\pgfpathlineto{\pgfqpoint{1.006713in}{0.376306in}}%
\pgfpathlineto{\pgfqpoint{1.018136in}{0.372162in}}%
\pgfpathlineto{\pgfqpoint{1.029559in}{0.367946in}}%
\pgfpathlineto{\pgfqpoint{1.040982in}{0.363660in}}%
\pgfpathlineto{\pgfqpoint{1.052405in}{0.359305in}}%
\pgfpathlineto{\pgfqpoint{1.063828in}{0.354882in}}%
\pgfpathlineto{\pgfqpoint{1.075251in}{0.350393in}}%
\pgfpathlineto{\pgfqpoint{1.086673in}{0.345839in}}%
\pgfpathlineto{\pgfqpoint{1.098096in}{0.341224in}}%
\pgfpathlineto{\pgfqpoint{1.109519in}{0.336552in}}%
\pgfpathlineto{\pgfqpoint{1.120942in}{0.331831in}}%
\pgfpathlineto{\pgfqpoint{1.132365in}{0.327097in}}%
\pgfpathlineto{\pgfqpoint{1.143788in}{0.322675in}}%
\pgfpathlineto{\pgfqpoint{1.155210in}{0.324100in}}%
\pgfpathlineto{\pgfqpoint{1.166633in}{0.328750in}}%
\pgfpathlineto{\pgfqpoint{1.178056in}{0.333488in}}%
\pgfpathlineto{\pgfqpoint{1.189479in}{0.338193in}}%
\pgfpathlineto{\pgfqpoint{1.200902in}{0.342846in}}%
\pgfpathlineto{\pgfqpoint{1.212325in}{0.347440in}}%
\pgfpathlineto{\pgfqpoint{1.223747in}{0.351972in}}%
\pgfpathlineto{\pgfqpoint{1.235170in}{0.356438in}}%
\pgfpathlineto{\pgfqpoint{1.246593in}{0.360837in}}%
\pgfpathlineto{\pgfqpoint{1.258016in}{0.365169in}}%
\pgfpathlineto{\pgfqpoint{1.269439in}{0.369430in}}%
\pgfpathlineto{\pgfqpoint{1.280862in}{0.373621in}}%
\pgfpathlineto{\pgfqpoint{1.292285in}{0.377740in}}%
\pgfpathlineto{\pgfqpoint{1.303707in}{0.381786in}}%
\pgfpathlineto{\pgfqpoint{1.315130in}{0.385758in}}%
\pgfpathlineto{\pgfqpoint{1.326553in}{0.389655in}}%
\pgfpathlineto{\pgfqpoint{1.337976in}{0.393476in}}%
\pgfpathlineto{\pgfqpoint{1.349399in}{0.397221in}}%
\pgfpathlineto{\pgfqpoint{1.360822in}{0.400888in}}%
\pgfpathlineto{\pgfqpoint{1.372244in}{0.404478in}}%
\pgfpathlineto{\pgfqpoint{1.383667in}{0.407989in}}%
\pgfpathlineto{\pgfqpoint{1.395090in}{0.411420in}}%
\pgfpathlineto{\pgfqpoint{1.406513in}{0.414772in}}%
\pgfpathlineto{\pgfqpoint{1.417936in}{0.418044in}}%
\pgfpathlineto{\pgfqpoint{1.429359in}{0.421235in}}%
\pgfpathlineto{\pgfqpoint{1.440782in}{0.424344in}}%
\pgfpathlineto{\pgfqpoint{1.452204in}{0.427373in}}%
\pgfpathlineto{\pgfqpoint{1.463627in}{0.430319in}}%
\pgfpathlineto{\pgfqpoint{1.475050in}{0.433184in}}%
\pgfpathlineto{\pgfqpoint{1.486473in}{0.435967in}}%
\pgfpathlineto{\pgfqpoint{1.497896in}{0.438667in}}%
\pgfpathlineto{\pgfqpoint{1.509319in}{0.441285in}}%
\pgfpathlineto{\pgfqpoint{1.520741in}{0.443821in}}%
\pgfpathlineto{\pgfqpoint{1.532164in}{0.446275in}}%
\pgfpathlineto{\pgfqpoint{1.543587in}{0.448646in}}%
\pgfpathlineto{\pgfqpoint{1.555010in}{0.450935in}}%
\pgfpathlineto{\pgfqpoint{1.566433in}{0.453143in}}%
\pgfpathlineto{\pgfqpoint{1.577856in}{0.455268in}}%
\pgfpathlineto{\pgfqpoint{1.589279in}{0.457312in}}%
\pgfpathlineto{\pgfqpoint{1.600701in}{0.459276in}}%
\pgfpathlineto{\pgfqpoint{1.612124in}{0.461158in}}%
\pgfpathlineto{\pgfqpoint{1.623547in}{0.462960in}}%
\pgfpathlineto{\pgfqpoint{1.634970in}{0.464683in}}%
\pgfpathlineto{\pgfqpoint{1.646393in}{0.466326in}}%
\pgfpathlineto{\pgfqpoint{1.657816in}{0.467890in}}%
\pgfpathlineto{\pgfqpoint{1.669238in}{0.469376in}}%
\pgfpathlineto{\pgfqpoint{1.680661in}{0.470785in}}%
\pgfpathlineto{\pgfqpoint{1.692084in}{0.472116in}}%
\pgfpathlineto{\pgfqpoint{1.703507in}{0.473372in}}%
\pgfpathlineto{\pgfqpoint{1.714930in}{0.474552in}}%
\pgfpathlineto{\pgfqpoint{1.726353in}{0.475658in}}%
\pgfpathlineto{\pgfqpoint{1.737776in}{0.476690in}}%
\pgfpathlineto{\pgfqpoint{1.749198in}{0.477649in}}%
\pgfpathlineto{\pgfqpoint{1.760621in}{0.478536in}}%
\pgfpathlineto{\pgfqpoint{1.772044in}{0.479352in}}%
\pgfpathlineto{\pgfqpoint{1.783467in}{0.480098in}}%
\pgfpathlineto{\pgfqpoint{1.794890in}{0.480775in}}%
\pgfpathlineto{\pgfqpoint{1.806313in}{0.481384in}}%
\pgfpathlineto{\pgfqpoint{1.817735in}{0.481927in}}%
\pgfpathlineto{\pgfqpoint{1.829158in}{0.482403in}}%
\pgfpathlineto{\pgfqpoint{1.840581in}{0.482815in}}%
\pgfpathlineto{\pgfqpoint{1.852004in}{0.483163in}}%
\pgfpathlineto{\pgfqpoint{1.863427in}{0.483449in}}%
\pgfpathlineto{\pgfqpoint{1.874850in}{0.483673in}}%
\pgfpathlineto{\pgfqpoint{1.886273in}{0.483838in}}%
\pgfpathlineto{\pgfqpoint{1.897695in}{0.483944in}}%
\pgfpathlineto{\pgfqpoint{1.909118in}{0.483993in}}%
\pgfpathlineto{\pgfqpoint{1.920541in}{0.483985in}}%
\pgfpathlineto{\pgfqpoint{1.931964in}{0.483923in}}%
\pgfpathlineto{\pgfqpoint{1.943387in}{0.483806in}}%
\pgfpathlineto{\pgfqpoint{1.954810in}{0.483638in}}%
\pgfpathlineto{\pgfqpoint{1.966232in}{0.483419in}}%
\pgfpathlineto{\pgfqpoint{1.977655in}{0.483150in}}%
\pgfpathlineto{\pgfqpoint{1.989078in}{0.482833in}}%
\pgfpathlineto{\pgfqpoint{2.000501in}{0.482469in}}%
\pgfpathlineto{\pgfqpoint{2.011924in}{0.482059in}}%
\pgfpathlineto{\pgfqpoint{2.023347in}{0.481606in}}%
\pgfpathlineto{\pgfqpoint{2.034770in}{0.481110in}}%
\pgfpathlineto{\pgfqpoint{2.046192in}{0.480572in}}%
\pgfpathlineto{\pgfqpoint{2.057615in}{0.479994in}}%
\pgfpathlineto{\pgfqpoint{2.069038in}{0.479378in}}%
\pgfpathlineto{\pgfqpoint{2.080461in}{0.478725in}}%
\pgfpathlineto{\pgfqpoint{2.091884in}{0.478036in}}%
\pgfpathlineto{\pgfqpoint{2.103307in}{0.477312in}}%
\pgfpathlineto{\pgfqpoint{2.114729in}{0.476555in}}%
\pgfpathlineto{\pgfqpoint{2.126152in}{0.475767in}}%
\pgfpathlineto{\pgfqpoint{2.137575in}{0.474948in}}%
\pgfpathlineto{\pgfqpoint{2.148998in}{0.474100in}}%
\pgfpathlineto{\pgfqpoint{2.160421in}{0.473225in}}%
\pgfpathlineto{\pgfqpoint{2.171844in}{0.472323in}}%
\pgfpathlineto{\pgfqpoint{2.183267in}{0.471397in}}%
\pgfpathlineto{\pgfqpoint{2.194689in}{0.470447in}}%
\pgfpathlineto{\pgfqpoint{2.206112in}{0.469474in}}%
\pgfpathlineto{\pgfqpoint{2.217535in}{0.468480in}}%
\pgfpathlineto{\pgfqpoint{2.228958in}{0.467467in}}%
\pgfpathlineto{\pgfqpoint{2.240381in}{0.466435in}}%
\pgfpathlineto{\pgfqpoint{2.251804in}{0.465386in}}%
\pgfpathlineto{\pgfqpoint{2.263226in}{0.464320in}}%
\pgfpathlineto{\pgfqpoint{2.274649in}{0.463240in}}%
\pgfpathlineto{\pgfqpoint{2.286072in}{0.462147in}}%
\pgfpathlineto{\pgfqpoint{2.297495in}{0.461040in}}%
\pgfpathlineto{\pgfqpoint{2.308918in}{0.459923in}}%
\pgfpathlineto{\pgfqpoint{2.320341in}{0.458795in}}%
\pgfpathlineto{\pgfqpoint{2.331764in}{0.457658in}}%
\pgfpathlineto{\pgfqpoint{2.343186in}{0.456513in}}%
\pgfpathlineto{\pgfqpoint{2.354609in}{0.455361in}}%
\pgfpathlineto{\pgfqpoint{2.366032in}{0.454204in}}%
\pgfpathlineto{\pgfqpoint{2.377455in}{0.453041in}}%
\pgfpathlineto{\pgfqpoint{2.388878in}{0.451875in}}%
\pgfpathlineto{\pgfqpoint{2.400301in}{0.450705in}}%
\pgfpathlineto{\pgfqpoint{2.411723in}{0.449534in}}%
\pgfpathlineto{\pgfqpoint{2.423146in}{0.448361in}}%
\pgfpathlineto{\pgfqpoint{2.434569in}{0.447189in}}%
\pgfpathlineto{\pgfqpoint{2.445992in}{0.446017in}}%
\pgfpathlineto{\pgfqpoint{2.457415in}{0.444847in}}%
\pgfpathlineto{\pgfqpoint{2.468838in}{0.443679in}}%
\pgfpathlineto{\pgfqpoint{2.480261in}{0.442514in}}%
\pgfpathlineto{\pgfqpoint{2.491683in}{0.441354in}}%
\pgfpathlineto{\pgfqpoint{2.503106in}{0.440198in}}%
\pgfpathlineto{\pgfqpoint{2.514529in}{0.439047in}}%
\pgfpathlineto{\pgfqpoint{2.525952in}{0.437903in}}%
\pgfpathlineto{\pgfqpoint{2.537375in}{0.436766in}}%
\pgfpathlineto{\pgfqpoint{2.548798in}{0.435636in}}%
\pgfpathlineto{\pgfqpoint{2.560220in}{0.434514in}}%
\pgfpathlineto{\pgfqpoint{2.571643in}{0.433401in}}%
\pgfpathlineto{\pgfqpoint{2.583066in}{0.432297in}}%
\pgfpathlineto{\pgfqpoint{2.594489in}{0.431202in}}%
\pgfpathlineto{\pgfqpoint{2.605912in}{0.430119in}}%
\pgfpathlineto{\pgfqpoint{2.617335in}{0.429046in}}%
\pgfpathlineto{\pgfqpoint{2.628758in}{0.427984in}}%
\pgfpathlineto{\pgfqpoint{2.640180in}{0.426934in}}%
\pgfpathlineto{\pgfqpoint{2.651603in}{0.425897in}}%
\pgfpathlineto{\pgfqpoint{2.663026in}{0.424871in}}%
\pgfpathlineto{\pgfqpoint{2.674449in}{0.423859in}}%
\pgfpathlineto{\pgfqpoint{2.685872in}{0.422861in}}%
\pgfpathlineto{\pgfqpoint{2.697295in}{0.421876in}}%
\pgfpathlineto{\pgfqpoint{2.708717in}{0.420904in}}%
\pgfpathlineto{\pgfqpoint{2.720140in}{0.419948in}}%
\pgfpathlineto{\pgfqpoint{2.731563in}{0.419005in}}%
\pgfpathlineto{\pgfqpoint{2.742986in}{0.418078in}}%
\pgfpathlineto{\pgfqpoint{2.754409in}{0.417166in}}%
\pgfpathlineto{\pgfqpoint{2.765832in}{0.416269in}}%
\pgfpathlineto{\pgfqpoint{2.777255in}{0.415387in}}%
\pgfpathlineto{\pgfqpoint{2.788677in}{0.414521in}}%
\pgfpathlineto{\pgfqpoint{2.800100in}{0.413671in}}%
\pgfpathlineto{\pgfqpoint{2.811523in}{0.412837in}}%
\pgfpathlineto{\pgfqpoint{2.822946in}{0.412019in}}%
\pgfpathlineto{\pgfqpoint{2.834369in}{0.411218in}}%
\pgfpathlineto{\pgfqpoint{2.845792in}{0.410432in}}%
\pgfpathlineto{\pgfqpoint{2.857214in}{0.409663in}}%
\pgfpathlineto{\pgfqpoint{2.868637in}{0.408911in}}%
\pgfpathlineto{\pgfqpoint{2.880060in}{0.408175in}}%
\pgfpathlineto{\pgfqpoint{2.891483in}{0.407456in}}%
\pgfpathlineto{\pgfqpoint{2.902906in}{0.406753in}}%
\pgfpathlineto{\pgfqpoint{2.914329in}{0.406067in}}%
\pgfpathlineto{\pgfqpoint{2.925752in}{0.405397in}}%
\pgfpathlineto{\pgfqpoint{2.937174in}{0.404745in}}%
\pgfpathlineto{\pgfqpoint{2.948597in}{0.404108in}}%
\pgfpathlineto{\pgfqpoint{2.960020in}{0.403489in}}%
\pgfpathlineto{\pgfqpoint{2.971443in}{0.402886in}}%
\pgfpathlineto{\pgfqpoint{2.982866in}{0.402299in}}%
\pgfpathlineto{\pgfqpoint{2.994289in}{0.401729in}}%
\pgfpathlineto{\pgfqpoint{3.005711in}{0.401176in}}%
\pgfpathlineto{\pgfqpoint{3.017134in}{0.400639in}}%
\pgfpathlineto{\pgfqpoint{3.028557in}{0.400118in}}%
\pgfpathlineto{\pgfqpoint{3.039980in}{0.399613in}}%
\pgfpathlineto{\pgfqpoint{3.051403in}{0.399124in}}%
\pgfpathlineto{\pgfqpoint{3.062826in}{0.398651in}}%
\pgfpathlineto{\pgfqpoint{3.074248in}{0.398194in}}%
\pgfpathlineto{\pgfqpoint{3.085671in}{0.397753in}}%
\pgfpathlineto{\pgfqpoint{3.097094in}{0.397328in}}%
\pgfpathlineto{\pgfqpoint{3.108517in}{0.396918in}}%
\pgfpathlineto{\pgfqpoint{3.119940in}{0.396524in}}%
\pgfpathlineto{\pgfqpoint{3.131363in}{0.396144in}}%
\pgfpathlineto{\pgfqpoint{3.142786in}{0.395780in}}%
\pgfpathlineto{\pgfqpoint{3.154208in}{0.395431in}}%
\pgfpathlineto{\pgfqpoint{3.165631in}{0.395097in}}%
\pgfpathlineto{\pgfqpoint{3.177054in}{0.394778in}}%
\pgfpathlineto{\pgfqpoint{3.188477in}{0.394473in}}%
\pgfpathlineto{\pgfqpoint{3.199900in}{0.394183in}}%
\pgfpathlineto{\pgfqpoint{3.211323in}{0.393907in}}%
\pgfpathlineto{\pgfqpoint{3.222745in}{0.393646in}}%
\pgfpathlineto{\pgfqpoint{3.234168in}{0.393398in}}%
\pgfpathlineto{\pgfqpoint{3.245591in}{0.393164in}}%
\pgfpathlineto{\pgfqpoint{3.257014in}{0.392944in}}%
\pgfpathlineto{\pgfqpoint{3.268437in}{0.392737in}}%
\pgfpathlineto{\pgfqpoint{3.279860in}{0.392544in}}%
\pgfpathlineto{\pgfqpoint{3.291283in}{0.392364in}}%
\pgfpathlineto{\pgfqpoint{3.302705in}{0.392197in}}%
\pgfpathlineto{\pgfqpoint{3.314128in}{0.392043in}}%
\pgfpathlineto{\pgfqpoint{3.325551in}{0.391902in}}%
\pgfpathlineto{\pgfqpoint{3.336974in}{0.391773in}}%
\pgfpathlineto{\pgfqpoint{3.348397in}{0.391657in}}%
\pgfpathlineto{\pgfqpoint{3.359820in}{0.391553in}}%
\pgfpathlineto{\pgfqpoint{3.371242in}{0.391461in}}%
\pgfpathlineto{\pgfqpoint{3.382665in}{0.391380in}}%
\pgfpathlineto{\pgfqpoint{3.394088in}{0.391312in}}%
\pgfpathlineto{\pgfqpoint{3.405511in}{0.391255in}}%
\pgfpathlineto{\pgfqpoint{3.416934in}{0.391209in}}%
\pgfpathlineto{\pgfqpoint{3.428357in}{0.391174in}}%
\pgfpathlineto{\pgfqpoint{3.439780in}{0.391151in}}%
\pgfpathlineto{\pgfqpoint{3.451202in}{0.391138in}}%
\pgfpathlineto{\pgfqpoint{3.462625in}{0.391135in}}%
\pgfpathlineto{\pgfqpoint{3.474048in}{0.391143in}}%
\pgfpathlineto{\pgfqpoint{3.485471in}{0.391161in}}%
\pgfpathlineto{\pgfqpoint{3.496894in}{0.391189in}}%
\pgfpathlineto{\pgfqpoint{3.508317in}{0.391227in}}%
\pgfpathlineto{\pgfqpoint{3.519739in}{0.391274in}}%
\pgfpathlineto{\pgfqpoint{3.531162in}{0.391331in}}%
\pgfpathlineto{\pgfqpoint{3.542585in}{0.391397in}}%
\pgfpathlineto{\pgfqpoint{3.554008in}{0.391471in}}%
\pgfpathlineto{\pgfqpoint{3.565431in}{0.391555in}}%
\pgfpathlineto{\pgfqpoint{3.576854in}{0.391646in}}%
\pgfpathlineto{\pgfqpoint{3.588277in}{0.391746in}}%
\pgfpathlineto{\pgfqpoint{3.599699in}{0.391853in}}%
\pgfpathlineto{\pgfqpoint{3.611122in}{0.391969in}}%
\pgfpathlineto{\pgfqpoint{3.622545in}{0.392091in}}%
\pgfpathlineto{\pgfqpoint{3.633968in}{0.392221in}}%
\pgfpathlineto{\pgfqpoint{3.645391in}{0.392358in}}%
\pgfpathlineto{\pgfqpoint{3.656814in}{0.392501in}}%
\pgfpathlineto{\pgfqpoint{3.668236in}{0.392650in}}%
\pgfpathlineto{\pgfqpoint{3.679659in}{0.392805in}}%
\pgfpathlineto{\pgfqpoint{3.691082in}{0.392966in}}%
\pgfpathlineto{\pgfqpoint{3.702505in}{0.393132in}}%
\pgfpathlineto{\pgfqpoint{3.713928in}{0.393303in}}%
\pgfpathlineto{\pgfqpoint{3.725351in}{0.393479in}}%
\pgfpathlineto{\pgfqpoint{3.736774in}{0.393659in}}%
\pgfpathlineto{\pgfqpoint{3.748196in}{0.393843in}}%
\pgfpathlineto{\pgfqpoint{3.759619in}{0.394030in}}%
\pgfpathlineto{\pgfqpoint{3.771042in}{0.394220in}}%
\pgfpathlineto{\pgfqpoint{3.782465in}{0.394414in}}%
\pgfpathlineto{\pgfqpoint{3.793888in}{0.394609in}}%
\pgfpathlineto{\pgfqpoint{3.805311in}{0.394807in}}%
\pgfpathlineto{\pgfqpoint{3.816733in}{0.395006in}}%
\pgfpathlineto{\pgfqpoint{3.828156in}{0.395206in}}%
\pgfpathlineto{\pgfqpoint{3.839579in}{0.395406in}}%
\pgfpathlineto{\pgfqpoint{3.851002in}{0.395607in}}%
\pgfpathlineto{\pgfqpoint{3.862425in}{0.395807in}}%
\pgfpathlineto{\pgfqpoint{3.873848in}{0.396007in}}%
\pgfpathlineto{\pgfqpoint{3.885271in}{0.396205in}}%
\pgfpathlineto{\pgfqpoint{3.896693in}{0.396401in}}%
\pgfpathlineto{\pgfqpoint{3.908116in}{0.396594in}}%
\pgfpathlineto{\pgfqpoint{3.919539in}{0.396785in}}%
\pgfpathlineto{\pgfqpoint{3.930962in}{0.396972in}}%
\pgfpathlineto{\pgfqpoint{3.942385in}{0.397154in}}%
\pgfpathlineto{\pgfqpoint{3.953808in}{0.397332in}}%
\pgfpathlineto{\pgfqpoint{3.965230in}{0.397505in}}%
\pgfpathlineto{\pgfqpoint{3.976653in}{0.397671in}}%
\pgfpathlineto{\pgfqpoint{3.988076in}{0.397831in}}%
\pgfpathlineto{\pgfqpoint{3.999499in}{0.397984in}}%
\pgfpathlineto{\pgfqpoint{4.010922in}{0.398128in}}%
\pgfpathlineto{\pgfqpoint{4.022345in}{0.398264in}}%
\pgfpathlineto{\pgfqpoint{4.033768in}{0.398391in}}%
\pgfpathlineto{\pgfqpoint{4.045190in}{0.398507in}}%
\pgfpathlineto{\pgfqpoint{4.056613in}{0.398613in}}%
\pgfpathlineto{\pgfqpoint{4.068036in}{0.398707in}}%
\pgfpathlineto{\pgfqpoint{4.079459in}{0.398788in}}%
\pgfpathlineto{\pgfqpoint{4.090882in}{0.398857in}}%
\pgfpathlineto{\pgfqpoint{4.102305in}{0.398911in}}%
\pgfpathlineto{\pgfqpoint{4.113727in}{0.398951in}}%
\pgfpathlineto{\pgfqpoint{4.125150in}{0.398976in}}%
\pgfpathlineto{\pgfqpoint{4.136573in}{0.398984in}}%
\pgfpathlineto{\pgfqpoint{4.147996in}{0.398975in}}%
\pgfpathlineto{\pgfqpoint{4.159419in}{0.398948in}}%
\pgfpathlineto{\pgfqpoint{4.170842in}{0.398902in}}%
\pgfpathlineto{\pgfqpoint{4.182265in}{0.398837in}}%
\pgfpathlineto{\pgfqpoint{4.193687in}{0.398751in}}%
\pgfpathlineto{\pgfqpoint{4.205110in}{0.398644in}}%
\pgfpathlineto{\pgfqpoint{4.216533in}{0.398514in}}%
\pgfpathlineto{\pgfqpoint{4.227956in}{0.398361in}}%
\pgfpathlineto{\pgfqpoint{4.239379in}{0.398183in}}%
\pgfpathlineto{\pgfqpoint{4.250802in}{0.397981in}}%
\pgfpathlineto{\pgfqpoint{4.262224in}{0.397753in}}%
\pgfpathlineto{\pgfqpoint{4.273647in}{0.397498in}}%
\pgfpathlineto{\pgfqpoint{4.285070in}{0.397215in}}%
\pgfpathlineto{\pgfqpoint{4.296493in}{0.396903in}}%
\pgfpathlineto{\pgfqpoint{4.307916in}{0.396561in}}%
\pgfpathlineto{\pgfqpoint{4.319339in}{0.396189in}}%
\pgfpathlineto{\pgfqpoint{4.330762in}{0.395785in}}%
\pgfpathlineto{\pgfqpoint{4.342184in}{0.395349in}}%
\pgfpathlineto{\pgfqpoint{4.353607in}{0.394879in}}%
\pgfpathlineto{\pgfqpoint{4.365030in}{0.394375in}}%
\pgfpathlineto{\pgfqpoint{4.376453in}{0.393835in}}%
\pgfpathlineto{\pgfqpoint{4.387876in}{0.393259in}}%
\pgfpathlineto{\pgfqpoint{4.399299in}{0.392646in}}%
\pgfpathlineto{\pgfqpoint{4.410721in}{0.391994in}}%
\pgfpathlineto{\pgfqpoint{4.422144in}{0.391304in}}%
\pgfpathlineto{\pgfqpoint{4.433567in}{0.390573in}}%
\pgfpathlineto{\pgfqpoint{4.444990in}{0.389802in}}%
\pgfpathlineto{\pgfqpoint{4.456413in}{0.388988in}}%
\pgfpathlineto{\pgfqpoint{4.467836in}{0.388132in}}%
\pgfpathlineto{\pgfqpoint{4.479259in}{0.387232in}}%
\pgfpathlineto{\pgfqpoint{4.490681in}{0.386288in}}%
\pgfpathlineto{\pgfqpoint{4.502104in}{0.385298in}}%
\pgfpathlineto{\pgfqpoint{4.513527in}{0.384262in}}%
\pgfpathlineto{\pgfqpoint{4.524950in}{0.383179in}}%
\pgfpathlineto{\pgfqpoint{4.536373in}{0.382048in}}%
\pgfpathlineto{\pgfqpoint{4.547796in}{0.380868in}}%
\pgfpathlineto{\pgfqpoint{4.559218in}{0.379639in}}%
\pgfpathlineto{\pgfqpoint{4.570641in}{0.378359in}}%
\pgfpathlineto{\pgfqpoint{4.582064in}{0.377029in}}%
\pgfpathlineto{\pgfqpoint{4.593487in}{0.375646in}}%
\pgfpathlineto{\pgfqpoint{4.604910in}{0.374211in}}%
\pgfpathlineto{\pgfqpoint{4.616333in}{0.372723in}}%
\pgfpathlineto{\pgfqpoint{4.627756in}{0.371182in}}%
\pgfpathlineto{\pgfqpoint{4.639178in}{0.369585in}}%
\pgfpathlineto{\pgfqpoint{4.650601in}{0.367934in}}%
\pgfpathlineto{\pgfqpoint{4.662024in}{0.366227in}}%
\pgfpathlineto{\pgfqpoint{4.673447in}{0.364464in}}%
\pgfpathlineto{\pgfqpoint{4.684870in}{0.362643in}}%
\pgfpathlineto{\pgfqpoint{4.696293in}{0.360766in}}%
\pgfpathlineto{\pgfqpoint{4.707715in}{0.358831in}}%
\pgfpathlineto{\pgfqpoint{4.719138in}{0.356837in}}%
\pgfpathlineto{\pgfqpoint{4.730561in}{0.354785in}}%
\pgfpathlineto{\pgfqpoint{4.741984in}{0.352674in}}%
\pgfpathlineto{\pgfqpoint{4.753407in}{0.350503in}}%
\pgfpathlineto{\pgfqpoint{4.764830in}{0.348272in}}%
\pgfpathlineto{\pgfqpoint{4.776253in}{0.345982in}}%
\pgfpathlineto{\pgfqpoint{4.787675in}{0.343631in}}%
\pgfpathlineto{\pgfqpoint{4.799098in}{0.341220in}}%
\pgfpathlineto{\pgfqpoint{4.810521in}{0.338748in}}%
\pgfpathlineto{\pgfqpoint{4.821944in}{0.336215in}}%
\pgfpathlineto{\pgfqpoint{4.833367in}{0.333621in}}%
\pgfpathlineto{\pgfqpoint{4.844790in}{0.330966in}}%
\pgfpathlineto{\pgfqpoint{4.856212in}{0.328249in}}%
\pgfpathlineto{\pgfqpoint{4.867635in}{0.325472in}}%
\pgfpathlineto{\pgfqpoint{4.879058in}{0.322634in}}%
\pgfpathlineto{\pgfqpoint{4.890481in}{0.319734in}}%
\pgfpathlineto{\pgfqpoint{4.901904in}{0.316774in}}%
\pgfpathlineto{\pgfqpoint{4.913327in}{0.313753in}}%
\pgfpathlineto{\pgfqpoint{4.924749in}{0.310671in}}%
\pgfpathlineto{\pgfqpoint{4.936172in}{0.307529in}}%
\pgfpathlineto{\pgfqpoint{4.947595in}{0.304326in}}%
\pgfpathlineto{\pgfqpoint{4.959018in}{0.301064in}}%
\pgfpathlineto{\pgfqpoint{4.970441in}{0.297743in}}%
\pgfpathlineto{\pgfqpoint{4.981864in}{0.294362in}}%
\pgfpathlineto{\pgfqpoint{4.993287in}{0.290923in}}%
\pgfpathlineto{\pgfqpoint{5.004709in}{0.287425in}}%
\pgfpathlineto{\pgfqpoint{5.016132in}{0.283870in}}%
\pgfpathlineto{\pgfqpoint{5.027555in}{0.280258in}}%
\pgfpathlineto{\pgfqpoint{5.038978in}{0.276589in}}%
\pgfpathlineto{\pgfqpoint{5.050401in}{0.272864in}}%
\pgfpathlineto{\pgfqpoint{5.061824in}{0.269084in}}%
\pgfpathlineto{\pgfqpoint{5.073246in}{0.265250in}}%
\pgfpathlineto{\pgfqpoint{5.084669in}{0.261361in}}%
\pgfpathlineto{\pgfqpoint{5.096092in}{0.257420in}}%
\pgfpathlineto{\pgfqpoint{5.107515in}{0.253426in}}%
\pgfpathlineto{\pgfqpoint{5.118938in}{0.249381in}}%
\pgfpathlineto{\pgfqpoint{5.130361in}{0.245285in}}%
\pgfpathlineto{\pgfqpoint{5.141784in}{0.241140in}}%
\pgfpathlineto{\pgfqpoint{5.153206in}{0.236946in}}%
\pgfpathlineto{\pgfqpoint{5.164629in}{0.232704in}}%
\pgfpathlineto{\pgfqpoint{5.176052in}{0.228416in}}%
\pgfpathlineto{\pgfqpoint{5.187475in}{0.224082in}}%
\pgfpathlineto{\pgfqpoint{5.198898in}{0.219704in}}%
\pgfpathlineto{\pgfqpoint{5.210321in}{0.215282in}}%
\pgfpathlineto{\pgfqpoint{5.221743in}{0.210818in}}%
\pgfpathlineto{\pgfqpoint{5.233166in}{0.206314in}}%
\pgfpathlineto{\pgfqpoint{5.244589in}{0.201769in}}%
\pgfpathlineto{\pgfqpoint{5.256012in}{0.197187in}}%
\pgfpathlineto{\pgfqpoint{5.267435in}{0.192567in}}%
\pgfpathlineto{\pgfqpoint{5.278858in}{0.187912in}}%
\pgfpathlineto{\pgfqpoint{5.290281in}{0.183222in}}%
\pgfpathlineto{\pgfqpoint{5.301703in}{0.178501in}}%
\pgfpathlineto{\pgfqpoint{5.313126in}{0.173749in}}%
\pgfpathlineto{\pgfqpoint{5.324549in}{0.168969in}}%
\pgfpathlineto{\pgfqpoint{5.335972in}{0.164166in}}%
\pgfpathlineto{\pgfqpoint{5.347395in}{0.159349in}}%
\pgfpathlineto{\pgfqpoint{5.358818in}{0.154545in}}%
\pgfpathlineto{\pgfqpoint{5.370240in}{0.150000in}}%
\pgfpathlineto{\pgfqpoint{5.381663in}{0.150689in}}%
\pgfpathlineto{\pgfqpoint{5.393086in}{0.155342in}}%
\pgfpathlineto{\pgfqpoint{5.404509in}{0.160152in}}%
\pgfpathlineto{\pgfqpoint{5.415932in}{0.164968in}}%
\pgfpathlineto{\pgfqpoint{5.427355in}{0.169767in}}%
\pgfpathlineto{\pgfqpoint{5.438778in}{0.174542in}}%
\pgfpathlineto{\pgfqpoint{5.450200in}{0.179289in}}%
\pgfpathlineto{\pgfqpoint{5.461623in}{0.184006in}}%
\pgfpathlineto{\pgfqpoint{5.473046in}{0.188689in}}%
\pgfpathlineto{\pgfqpoint{5.484469in}{0.193339in}}%
\pgfpathlineto{\pgfqpoint{5.495892in}{0.197952in}}%
\pgfpathlineto{\pgfqpoint{5.507315in}{0.202529in}}%
\pgfpathlineto{\pgfqpoint{5.518737in}{0.207066in}}%
\pgfpathlineto{\pgfqpoint{5.530160in}{0.211564in}}%
\pgfpathlineto{\pgfqpoint{5.541583in}{0.216021in}}%
\pgfpathlineto{\pgfqpoint{5.553006in}{0.220435in}}%
\pgfpathlineto{\pgfqpoint{5.564429in}{0.224806in}}%
\pgfpathlineto{\pgfqpoint{5.575852in}{0.229133in}}%
\pgfpathlineto{\pgfqpoint{5.587275in}{0.233413in}}%
\pgfpathlineto{\pgfqpoint{5.598697in}{0.237647in}}%
\pgfpathlineto{\pgfqpoint{5.610120in}{0.241833in}}%
\pgfpathlineto{\pgfqpoint{5.621543in}{0.245970in}}%
\pgfpathlineto{\pgfqpoint{5.632966in}{0.250057in}}%
\pgfpathlineto{\pgfqpoint{5.644389in}{0.254093in}}%
\pgfpathlineto{\pgfqpoint{5.655812in}{0.258078in}}%
\pgfpathlineto{\pgfqpoint{5.667234in}{0.262011in}}%
\pgfpathlineto{\pgfqpoint{5.678657in}{0.265890in}}%
\pgfpathlineto{\pgfqpoint{5.690080in}{0.269716in}}%
\pgfpathlineto{\pgfqpoint{5.701503in}{0.273487in}}%
\pgfpathlineto{\pgfqpoint{5.712926in}{0.277202in}}%
\pgfpathlineto{\pgfqpoint{5.724349in}{0.280861in}}%
\pgfpathlineto{\pgfqpoint{5.735772in}{0.284464in}}%
\pgfpathlineto{\pgfqpoint{5.747194in}{0.288009in}}%
\pgfpathlineto{\pgfqpoint{5.758617in}{0.291497in}}%
\pgfpathlineto{\pgfqpoint{5.770040in}{0.294926in}}%
\pgfpathlineto{\pgfqpoint{5.781463in}{0.298296in}}%
\pgfpathlineto{\pgfqpoint{5.792886in}{0.301608in}}%
\pgfpathlineto{\pgfqpoint{5.804309in}{0.304860in}}%
\pgfpathlineto{\pgfqpoint{5.815731in}{0.308052in}}%
\pgfpathlineto{\pgfqpoint{5.827154in}{0.311183in}}%
\pgfpathlineto{\pgfqpoint{5.838577in}{0.314255in}}%
\pgfpathlineto{\pgfqpoint{5.850000in}{0.317265in}}%
\pgfpathlineto{\pgfqpoint{5.850000in}{0.150000in}}%
\pgfpathlineto{\pgfqpoint{5.850000in}{0.150000in}}%
\pgfpathlineto{\pgfqpoint{5.838577in}{0.150000in}}%
\pgfpathlineto{\pgfqpoint{5.827154in}{0.150000in}}%
\pgfpathlineto{\pgfqpoint{5.815731in}{0.150000in}}%
\pgfpathlineto{\pgfqpoint{5.804309in}{0.150000in}}%
\pgfpathlineto{\pgfqpoint{5.792886in}{0.150000in}}%
\pgfpathlineto{\pgfqpoint{5.781463in}{0.150000in}}%
\pgfpathlineto{\pgfqpoint{5.770040in}{0.150000in}}%
\pgfpathlineto{\pgfqpoint{5.758617in}{0.150000in}}%
\pgfpathlineto{\pgfqpoint{5.747194in}{0.150000in}}%
\pgfpathlineto{\pgfqpoint{5.735772in}{0.150000in}}%
\pgfpathlineto{\pgfqpoint{5.724349in}{0.150000in}}%
\pgfpathlineto{\pgfqpoint{5.712926in}{0.150000in}}%
\pgfpathlineto{\pgfqpoint{5.701503in}{0.150000in}}%
\pgfpathlineto{\pgfqpoint{5.690080in}{0.150000in}}%
\pgfpathlineto{\pgfqpoint{5.678657in}{0.150000in}}%
\pgfpathlineto{\pgfqpoint{5.667234in}{0.150000in}}%
\pgfpathlineto{\pgfqpoint{5.655812in}{0.150000in}}%
\pgfpathlineto{\pgfqpoint{5.644389in}{0.150000in}}%
\pgfpathlineto{\pgfqpoint{5.632966in}{0.150000in}}%
\pgfpathlineto{\pgfqpoint{5.621543in}{0.150000in}}%
\pgfpathlineto{\pgfqpoint{5.610120in}{0.150000in}}%
\pgfpathlineto{\pgfqpoint{5.598697in}{0.150000in}}%
\pgfpathlineto{\pgfqpoint{5.587275in}{0.150000in}}%
\pgfpathlineto{\pgfqpoint{5.575852in}{0.150000in}}%
\pgfpathlineto{\pgfqpoint{5.564429in}{0.150000in}}%
\pgfpathlineto{\pgfqpoint{5.553006in}{0.150000in}}%
\pgfpathlineto{\pgfqpoint{5.541583in}{0.150000in}}%
\pgfpathlineto{\pgfqpoint{5.530160in}{0.150000in}}%
\pgfpathlineto{\pgfqpoint{5.518737in}{0.150000in}}%
\pgfpathlineto{\pgfqpoint{5.507315in}{0.150000in}}%
\pgfpathlineto{\pgfqpoint{5.495892in}{0.150000in}}%
\pgfpathlineto{\pgfqpoint{5.484469in}{0.150000in}}%
\pgfpathlineto{\pgfqpoint{5.473046in}{0.150000in}}%
\pgfpathlineto{\pgfqpoint{5.461623in}{0.150000in}}%
\pgfpathlineto{\pgfqpoint{5.450200in}{0.150000in}}%
\pgfpathlineto{\pgfqpoint{5.438778in}{0.150000in}}%
\pgfpathlineto{\pgfqpoint{5.427355in}{0.150000in}}%
\pgfpathlineto{\pgfqpoint{5.415932in}{0.150000in}}%
\pgfpathlineto{\pgfqpoint{5.404509in}{0.150000in}}%
\pgfpathlineto{\pgfqpoint{5.393086in}{0.150000in}}%
\pgfpathlineto{\pgfqpoint{5.381663in}{0.150000in}}%
\pgfpathlineto{\pgfqpoint{5.370240in}{0.150000in}}%
\pgfpathlineto{\pgfqpoint{5.358818in}{0.150000in}}%
\pgfpathlineto{\pgfqpoint{5.347395in}{0.150000in}}%
\pgfpathlineto{\pgfqpoint{5.335972in}{0.150000in}}%
\pgfpathlineto{\pgfqpoint{5.324549in}{0.150000in}}%
\pgfpathlineto{\pgfqpoint{5.313126in}{0.150000in}}%
\pgfpathlineto{\pgfqpoint{5.301703in}{0.150000in}}%
\pgfpathlineto{\pgfqpoint{5.290281in}{0.150000in}}%
\pgfpathlineto{\pgfqpoint{5.278858in}{0.150000in}}%
\pgfpathlineto{\pgfqpoint{5.267435in}{0.150000in}}%
\pgfpathlineto{\pgfqpoint{5.256012in}{0.150000in}}%
\pgfpathlineto{\pgfqpoint{5.244589in}{0.150000in}}%
\pgfpathlineto{\pgfqpoint{5.233166in}{0.150000in}}%
\pgfpathlineto{\pgfqpoint{5.221743in}{0.150000in}}%
\pgfpathlineto{\pgfqpoint{5.210321in}{0.150000in}}%
\pgfpathlineto{\pgfqpoint{5.198898in}{0.150000in}}%
\pgfpathlineto{\pgfqpoint{5.187475in}{0.150000in}}%
\pgfpathlineto{\pgfqpoint{5.176052in}{0.150000in}}%
\pgfpathlineto{\pgfqpoint{5.164629in}{0.150000in}}%
\pgfpathlineto{\pgfqpoint{5.153206in}{0.150000in}}%
\pgfpathlineto{\pgfqpoint{5.141784in}{0.150000in}}%
\pgfpathlineto{\pgfqpoint{5.130361in}{0.150000in}}%
\pgfpathlineto{\pgfqpoint{5.118938in}{0.150000in}}%
\pgfpathlineto{\pgfqpoint{5.107515in}{0.150000in}}%
\pgfpathlineto{\pgfqpoint{5.096092in}{0.150000in}}%
\pgfpathlineto{\pgfqpoint{5.084669in}{0.150000in}}%
\pgfpathlineto{\pgfqpoint{5.073246in}{0.150000in}}%
\pgfpathlineto{\pgfqpoint{5.061824in}{0.150000in}}%
\pgfpathlineto{\pgfqpoint{5.050401in}{0.150000in}}%
\pgfpathlineto{\pgfqpoint{5.038978in}{0.150000in}}%
\pgfpathlineto{\pgfqpoint{5.027555in}{0.150000in}}%
\pgfpathlineto{\pgfqpoint{5.016132in}{0.150000in}}%
\pgfpathlineto{\pgfqpoint{5.004709in}{0.150000in}}%
\pgfpathlineto{\pgfqpoint{4.993287in}{0.150000in}}%
\pgfpathlineto{\pgfqpoint{4.981864in}{0.150000in}}%
\pgfpathlineto{\pgfqpoint{4.970441in}{0.150000in}}%
\pgfpathlineto{\pgfqpoint{4.959018in}{0.150000in}}%
\pgfpathlineto{\pgfqpoint{4.947595in}{0.150000in}}%
\pgfpathlineto{\pgfqpoint{4.936172in}{0.150000in}}%
\pgfpathlineto{\pgfqpoint{4.924749in}{0.150000in}}%
\pgfpathlineto{\pgfqpoint{4.913327in}{0.150000in}}%
\pgfpathlineto{\pgfqpoint{4.901904in}{0.150000in}}%
\pgfpathlineto{\pgfqpoint{4.890481in}{0.150000in}}%
\pgfpathlineto{\pgfqpoint{4.879058in}{0.150000in}}%
\pgfpathlineto{\pgfqpoint{4.867635in}{0.150000in}}%
\pgfpathlineto{\pgfqpoint{4.856212in}{0.150000in}}%
\pgfpathlineto{\pgfqpoint{4.844790in}{0.150000in}}%
\pgfpathlineto{\pgfqpoint{4.833367in}{0.150000in}}%
\pgfpathlineto{\pgfqpoint{4.821944in}{0.150000in}}%
\pgfpathlineto{\pgfqpoint{4.810521in}{0.150000in}}%
\pgfpathlineto{\pgfqpoint{4.799098in}{0.150000in}}%
\pgfpathlineto{\pgfqpoint{4.787675in}{0.150000in}}%
\pgfpathlineto{\pgfqpoint{4.776253in}{0.150000in}}%
\pgfpathlineto{\pgfqpoint{4.764830in}{0.150000in}}%
\pgfpathlineto{\pgfqpoint{4.753407in}{0.150000in}}%
\pgfpathlineto{\pgfqpoint{4.741984in}{0.150000in}}%
\pgfpathlineto{\pgfqpoint{4.730561in}{0.150000in}}%
\pgfpathlineto{\pgfqpoint{4.719138in}{0.150000in}}%
\pgfpathlineto{\pgfqpoint{4.707715in}{0.150000in}}%
\pgfpathlineto{\pgfqpoint{4.696293in}{0.150000in}}%
\pgfpathlineto{\pgfqpoint{4.684870in}{0.150000in}}%
\pgfpathlineto{\pgfqpoint{4.673447in}{0.150000in}}%
\pgfpathlineto{\pgfqpoint{4.662024in}{0.150000in}}%
\pgfpathlineto{\pgfqpoint{4.650601in}{0.150000in}}%
\pgfpathlineto{\pgfqpoint{4.639178in}{0.150000in}}%
\pgfpathlineto{\pgfqpoint{4.627756in}{0.150000in}}%
\pgfpathlineto{\pgfqpoint{4.616333in}{0.150000in}}%
\pgfpathlineto{\pgfqpoint{4.604910in}{0.150000in}}%
\pgfpathlineto{\pgfqpoint{4.593487in}{0.150000in}}%
\pgfpathlineto{\pgfqpoint{4.582064in}{0.150000in}}%
\pgfpathlineto{\pgfqpoint{4.570641in}{0.150000in}}%
\pgfpathlineto{\pgfqpoint{4.559218in}{0.150000in}}%
\pgfpathlineto{\pgfqpoint{4.547796in}{0.150000in}}%
\pgfpathlineto{\pgfqpoint{4.536373in}{0.150000in}}%
\pgfpathlineto{\pgfqpoint{4.524950in}{0.150000in}}%
\pgfpathlineto{\pgfqpoint{4.513527in}{0.150000in}}%
\pgfpathlineto{\pgfqpoint{4.502104in}{0.150000in}}%
\pgfpathlineto{\pgfqpoint{4.490681in}{0.150000in}}%
\pgfpathlineto{\pgfqpoint{4.479259in}{0.150000in}}%
\pgfpathlineto{\pgfqpoint{4.467836in}{0.150000in}}%
\pgfpathlineto{\pgfqpoint{4.456413in}{0.150000in}}%
\pgfpathlineto{\pgfqpoint{4.444990in}{0.150000in}}%
\pgfpathlineto{\pgfqpoint{4.433567in}{0.150000in}}%
\pgfpathlineto{\pgfqpoint{4.422144in}{0.150000in}}%
\pgfpathlineto{\pgfqpoint{4.410721in}{0.150000in}}%
\pgfpathlineto{\pgfqpoint{4.399299in}{0.150000in}}%
\pgfpathlineto{\pgfqpoint{4.387876in}{0.150000in}}%
\pgfpathlineto{\pgfqpoint{4.376453in}{0.150000in}}%
\pgfpathlineto{\pgfqpoint{4.365030in}{0.150000in}}%
\pgfpathlineto{\pgfqpoint{4.353607in}{0.150000in}}%
\pgfpathlineto{\pgfqpoint{4.342184in}{0.150000in}}%
\pgfpathlineto{\pgfqpoint{4.330762in}{0.150000in}}%
\pgfpathlineto{\pgfqpoint{4.319339in}{0.150000in}}%
\pgfpathlineto{\pgfqpoint{4.307916in}{0.150000in}}%
\pgfpathlineto{\pgfqpoint{4.296493in}{0.150000in}}%
\pgfpathlineto{\pgfqpoint{4.285070in}{0.150000in}}%
\pgfpathlineto{\pgfqpoint{4.273647in}{0.150000in}}%
\pgfpathlineto{\pgfqpoint{4.262224in}{0.150000in}}%
\pgfpathlineto{\pgfqpoint{4.250802in}{0.150000in}}%
\pgfpathlineto{\pgfqpoint{4.239379in}{0.150000in}}%
\pgfpathlineto{\pgfqpoint{4.227956in}{0.150000in}}%
\pgfpathlineto{\pgfqpoint{4.216533in}{0.150000in}}%
\pgfpathlineto{\pgfqpoint{4.205110in}{0.150000in}}%
\pgfpathlineto{\pgfqpoint{4.193687in}{0.150000in}}%
\pgfpathlineto{\pgfqpoint{4.182265in}{0.150000in}}%
\pgfpathlineto{\pgfqpoint{4.170842in}{0.150000in}}%
\pgfpathlineto{\pgfqpoint{4.159419in}{0.150000in}}%
\pgfpathlineto{\pgfqpoint{4.147996in}{0.150000in}}%
\pgfpathlineto{\pgfqpoint{4.136573in}{0.150000in}}%
\pgfpathlineto{\pgfqpoint{4.125150in}{0.150000in}}%
\pgfpathlineto{\pgfqpoint{4.113727in}{0.150000in}}%
\pgfpathlineto{\pgfqpoint{4.102305in}{0.150000in}}%
\pgfpathlineto{\pgfqpoint{4.090882in}{0.150000in}}%
\pgfpathlineto{\pgfqpoint{4.079459in}{0.150000in}}%
\pgfpathlineto{\pgfqpoint{4.068036in}{0.150000in}}%
\pgfpathlineto{\pgfqpoint{4.056613in}{0.150000in}}%
\pgfpathlineto{\pgfqpoint{4.045190in}{0.150000in}}%
\pgfpathlineto{\pgfqpoint{4.033768in}{0.150000in}}%
\pgfpathlineto{\pgfqpoint{4.022345in}{0.150000in}}%
\pgfpathlineto{\pgfqpoint{4.010922in}{0.150000in}}%
\pgfpathlineto{\pgfqpoint{3.999499in}{0.150000in}}%
\pgfpathlineto{\pgfqpoint{3.988076in}{0.150000in}}%
\pgfpathlineto{\pgfqpoint{3.976653in}{0.150000in}}%
\pgfpathlineto{\pgfqpoint{3.965230in}{0.150000in}}%
\pgfpathlineto{\pgfqpoint{3.953808in}{0.150000in}}%
\pgfpathlineto{\pgfqpoint{3.942385in}{0.150000in}}%
\pgfpathlineto{\pgfqpoint{3.930962in}{0.150000in}}%
\pgfpathlineto{\pgfqpoint{3.919539in}{0.150000in}}%
\pgfpathlineto{\pgfqpoint{3.908116in}{0.150000in}}%
\pgfpathlineto{\pgfqpoint{3.896693in}{0.150000in}}%
\pgfpathlineto{\pgfqpoint{3.885271in}{0.150000in}}%
\pgfpathlineto{\pgfqpoint{3.873848in}{0.150000in}}%
\pgfpathlineto{\pgfqpoint{3.862425in}{0.150000in}}%
\pgfpathlineto{\pgfqpoint{3.851002in}{0.150000in}}%
\pgfpathlineto{\pgfqpoint{3.839579in}{0.150000in}}%
\pgfpathlineto{\pgfqpoint{3.828156in}{0.150000in}}%
\pgfpathlineto{\pgfqpoint{3.816733in}{0.150000in}}%
\pgfpathlineto{\pgfqpoint{3.805311in}{0.150000in}}%
\pgfpathlineto{\pgfqpoint{3.793888in}{0.150000in}}%
\pgfpathlineto{\pgfqpoint{3.782465in}{0.150000in}}%
\pgfpathlineto{\pgfqpoint{3.771042in}{0.150000in}}%
\pgfpathlineto{\pgfqpoint{3.759619in}{0.150000in}}%
\pgfpathlineto{\pgfqpoint{3.748196in}{0.150000in}}%
\pgfpathlineto{\pgfqpoint{3.736774in}{0.150000in}}%
\pgfpathlineto{\pgfqpoint{3.725351in}{0.150000in}}%
\pgfpathlineto{\pgfqpoint{3.713928in}{0.150000in}}%
\pgfpathlineto{\pgfqpoint{3.702505in}{0.150000in}}%
\pgfpathlineto{\pgfqpoint{3.691082in}{0.150000in}}%
\pgfpathlineto{\pgfqpoint{3.679659in}{0.150000in}}%
\pgfpathlineto{\pgfqpoint{3.668236in}{0.150000in}}%
\pgfpathlineto{\pgfqpoint{3.656814in}{0.150000in}}%
\pgfpathlineto{\pgfqpoint{3.645391in}{0.150000in}}%
\pgfpathlineto{\pgfqpoint{3.633968in}{0.150000in}}%
\pgfpathlineto{\pgfqpoint{3.622545in}{0.150000in}}%
\pgfpathlineto{\pgfqpoint{3.611122in}{0.150000in}}%
\pgfpathlineto{\pgfqpoint{3.599699in}{0.150000in}}%
\pgfpathlineto{\pgfqpoint{3.588277in}{0.150000in}}%
\pgfpathlineto{\pgfqpoint{3.576854in}{0.150000in}}%
\pgfpathlineto{\pgfqpoint{3.565431in}{0.150000in}}%
\pgfpathlineto{\pgfqpoint{3.554008in}{0.150000in}}%
\pgfpathlineto{\pgfqpoint{3.542585in}{0.150000in}}%
\pgfpathlineto{\pgfqpoint{3.531162in}{0.150000in}}%
\pgfpathlineto{\pgfqpoint{3.519739in}{0.150000in}}%
\pgfpathlineto{\pgfqpoint{3.508317in}{0.150000in}}%
\pgfpathlineto{\pgfqpoint{3.496894in}{0.150000in}}%
\pgfpathlineto{\pgfqpoint{3.485471in}{0.150000in}}%
\pgfpathlineto{\pgfqpoint{3.474048in}{0.150000in}}%
\pgfpathlineto{\pgfqpoint{3.462625in}{0.150000in}}%
\pgfpathlineto{\pgfqpoint{3.451202in}{0.150000in}}%
\pgfpathlineto{\pgfqpoint{3.439780in}{0.150000in}}%
\pgfpathlineto{\pgfqpoint{3.428357in}{0.150000in}}%
\pgfpathlineto{\pgfqpoint{3.416934in}{0.150000in}}%
\pgfpathlineto{\pgfqpoint{3.405511in}{0.150000in}}%
\pgfpathlineto{\pgfqpoint{3.394088in}{0.150000in}}%
\pgfpathlineto{\pgfqpoint{3.382665in}{0.150000in}}%
\pgfpathlineto{\pgfqpoint{3.371242in}{0.150000in}}%
\pgfpathlineto{\pgfqpoint{3.359820in}{0.150000in}}%
\pgfpathlineto{\pgfqpoint{3.348397in}{0.150000in}}%
\pgfpathlineto{\pgfqpoint{3.336974in}{0.150000in}}%
\pgfpathlineto{\pgfqpoint{3.325551in}{0.150000in}}%
\pgfpathlineto{\pgfqpoint{3.314128in}{0.150000in}}%
\pgfpathlineto{\pgfqpoint{3.302705in}{0.150000in}}%
\pgfpathlineto{\pgfqpoint{3.291283in}{0.150000in}}%
\pgfpathlineto{\pgfqpoint{3.279860in}{0.150000in}}%
\pgfpathlineto{\pgfqpoint{3.268437in}{0.150000in}}%
\pgfpathlineto{\pgfqpoint{3.257014in}{0.150000in}}%
\pgfpathlineto{\pgfqpoint{3.245591in}{0.150000in}}%
\pgfpathlineto{\pgfqpoint{3.234168in}{0.150000in}}%
\pgfpathlineto{\pgfqpoint{3.222745in}{0.150000in}}%
\pgfpathlineto{\pgfqpoint{3.211323in}{0.150000in}}%
\pgfpathlineto{\pgfqpoint{3.199900in}{0.150000in}}%
\pgfpathlineto{\pgfqpoint{3.188477in}{0.150000in}}%
\pgfpathlineto{\pgfqpoint{3.177054in}{0.150000in}}%
\pgfpathlineto{\pgfqpoint{3.165631in}{0.150000in}}%
\pgfpathlineto{\pgfqpoint{3.154208in}{0.150000in}}%
\pgfpathlineto{\pgfqpoint{3.142786in}{0.150000in}}%
\pgfpathlineto{\pgfqpoint{3.131363in}{0.150000in}}%
\pgfpathlineto{\pgfqpoint{3.119940in}{0.150000in}}%
\pgfpathlineto{\pgfqpoint{3.108517in}{0.150000in}}%
\pgfpathlineto{\pgfqpoint{3.097094in}{0.150000in}}%
\pgfpathlineto{\pgfqpoint{3.085671in}{0.150000in}}%
\pgfpathlineto{\pgfqpoint{3.074248in}{0.150000in}}%
\pgfpathlineto{\pgfqpoint{3.062826in}{0.150000in}}%
\pgfpathlineto{\pgfqpoint{3.051403in}{0.150000in}}%
\pgfpathlineto{\pgfqpoint{3.039980in}{0.150000in}}%
\pgfpathlineto{\pgfqpoint{3.028557in}{0.150000in}}%
\pgfpathlineto{\pgfqpoint{3.017134in}{0.150000in}}%
\pgfpathlineto{\pgfqpoint{3.005711in}{0.150000in}}%
\pgfpathlineto{\pgfqpoint{2.994289in}{0.150000in}}%
\pgfpathlineto{\pgfqpoint{2.982866in}{0.150000in}}%
\pgfpathlineto{\pgfqpoint{2.971443in}{0.150000in}}%
\pgfpathlineto{\pgfqpoint{2.960020in}{0.150000in}}%
\pgfpathlineto{\pgfqpoint{2.948597in}{0.150000in}}%
\pgfpathlineto{\pgfqpoint{2.937174in}{0.150000in}}%
\pgfpathlineto{\pgfqpoint{2.925752in}{0.150000in}}%
\pgfpathlineto{\pgfqpoint{2.914329in}{0.150000in}}%
\pgfpathlineto{\pgfqpoint{2.902906in}{0.150000in}}%
\pgfpathlineto{\pgfqpoint{2.891483in}{0.150000in}}%
\pgfpathlineto{\pgfqpoint{2.880060in}{0.150000in}}%
\pgfpathlineto{\pgfqpoint{2.868637in}{0.150000in}}%
\pgfpathlineto{\pgfqpoint{2.857214in}{0.150000in}}%
\pgfpathlineto{\pgfqpoint{2.845792in}{0.150000in}}%
\pgfpathlineto{\pgfqpoint{2.834369in}{0.150000in}}%
\pgfpathlineto{\pgfqpoint{2.822946in}{0.150000in}}%
\pgfpathlineto{\pgfqpoint{2.811523in}{0.150000in}}%
\pgfpathlineto{\pgfqpoint{2.800100in}{0.150000in}}%
\pgfpathlineto{\pgfqpoint{2.788677in}{0.150000in}}%
\pgfpathlineto{\pgfqpoint{2.777255in}{0.150000in}}%
\pgfpathlineto{\pgfqpoint{2.765832in}{0.150000in}}%
\pgfpathlineto{\pgfqpoint{2.754409in}{0.150000in}}%
\pgfpathlineto{\pgfqpoint{2.742986in}{0.150000in}}%
\pgfpathlineto{\pgfqpoint{2.731563in}{0.150000in}}%
\pgfpathlineto{\pgfqpoint{2.720140in}{0.150000in}}%
\pgfpathlineto{\pgfqpoint{2.708717in}{0.150000in}}%
\pgfpathlineto{\pgfqpoint{2.697295in}{0.150000in}}%
\pgfpathlineto{\pgfqpoint{2.685872in}{0.150000in}}%
\pgfpathlineto{\pgfqpoint{2.674449in}{0.150000in}}%
\pgfpathlineto{\pgfqpoint{2.663026in}{0.150000in}}%
\pgfpathlineto{\pgfqpoint{2.651603in}{0.150000in}}%
\pgfpathlineto{\pgfqpoint{2.640180in}{0.150000in}}%
\pgfpathlineto{\pgfqpoint{2.628758in}{0.150000in}}%
\pgfpathlineto{\pgfqpoint{2.617335in}{0.150000in}}%
\pgfpathlineto{\pgfqpoint{2.605912in}{0.150000in}}%
\pgfpathlineto{\pgfqpoint{2.594489in}{0.150000in}}%
\pgfpathlineto{\pgfqpoint{2.583066in}{0.150000in}}%
\pgfpathlineto{\pgfqpoint{2.571643in}{0.150000in}}%
\pgfpathlineto{\pgfqpoint{2.560220in}{0.150000in}}%
\pgfpathlineto{\pgfqpoint{2.548798in}{0.150000in}}%
\pgfpathlineto{\pgfqpoint{2.537375in}{0.150000in}}%
\pgfpathlineto{\pgfqpoint{2.525952in}{0.150000in}}%
\pgfpathlineto{\pgfqpoint{2.514529in}{0.150000in}}%
\pgfpathlineto{\pgfqpoint{2.503106in}{0.150000in}}%
\pgfpathlineto{\pgfqpoint{2.491683in}{0.150000in}}%
\pgfpathlineto{\pgfqpoint{2.480261in}{0.150000in}}%
\pgfpathlineto{\pgfqpoint{2.468838in}{0.150000in}}%
\pgfpathlineto{\pgfqpoint{2.457415in}{0.150000in}}%
\pgfpathlineto{\pgfqpoint{2.445992in}{0.150000in}}%
\pgfpathlineto{\pgfqpoint{2.434569in}{0.150000in}}%
\pgfpathlineto{\pgfqpoint{2.423146in}{0.150000in}}%
\pgfpathlineto{\pgfqpoint{2.411723in}{0.150000in}}%
\pgfpathlineto{\pgfqpoint{2.400301in}{0.150000in}}%
\pgfpathlineto{\pgfqpoint{2.388878in}{0.150000in}}%
\pgfpathlineto{\pgfqpoint{2.377455in}{0.150000in}}%
\pgfpathlineto{\pgfqpoint{2.366032in}{0.150000in}}%
\pgfpathlineto{\pgfqpoint{2.354609in}{0.150000in}}%
\pgfpathlineto{\pgfqpoint{2.343186in}{0.150000in}}%
\pgfpathlineto{\pgfqpoint{2.331764in}{0.150000in}}%
\pgfpathlineto{\pgfqpoint{2.320341in}{0.150000in}}%
\pgfpathlineto{\pgfqpoint{2.308918in}{0.150000in}}%
\pgfpathlineto{\pgfqpoint{2.297495in}{0.150000in}}%
\pgfpathlineto{\pgfqpoint{2.286072in}{0.150000in}}%
\pgfpathlineto{\pgfqpoint{2.274649in}{0.150000in}}%
\pgfpathlineto{\pgfqpoint{2.263226in}{0.150000in}}%
\pgfpathlineto{\pgfqpoint{2.251804in}{0.150000in}}%
\pgfpathlineto{\pgfqpoint{2.240381in}{0.150000in}}%
\pgfpathlineto{\pgfqpoint{2.228958in}{0.150000in}}%
\pgfpathlineto{\pgfqpoint{2.217535in}{0.150000in}}%
\pgfpathlineto{\pgfqpoint{2.206112in}{0.150000in}}%
\pgfpathlineto{\pgfqpoint{2.194689in}{0.150000in}}%
\pgfpathlineto{\pgfqpoint{2.183267in}{0.150000in}}%
\pgfpathlineto{\pgfqpoint{2.171844in}{0.150000in}}%
\pgfpathlineto{\pgfqpoint{2.160421in}{0.150000in}}%
\pgfpathlineto{\pgfqpoint{2.148998in}{0.150000in}}%
\pgfpathlineto{\pgfqpoint{2.137575in}{0.150000in}}%
\pgfpathlineto{\pgfqpoint{2.126152in}{0.150000in}}%
\pgfpathlineto{\pgfqpoint{2.114729in}{0.150000in}}%
\pgfpathlineto{\pgfqpoint{2.103307in}{0.150000in}}%
\pgfpathlineto{\pgfqpoint{2.091884in}{0.150000in}}%
\pgfpathlineto{\pgfqpoint{2.080461in}{0.150000in}}%
\pgfpathlineto{\pgfqpoint{2.069038in}{0.150000in}}%
\pgfpathlineto{\pgfqpoint{2.057615in}{0.150000in}}%
\pgfpathlineto{\pgfqpoint{2.046192in}{0.150000in}}%
\pgfpathlineto{\pgfqpoint{2.034770in}{0.150000in}}%
\pgfpathlineto{\pgfqpoint{2.023347in}{0.150000in}}%
\pgfpathlineto{\pgfqpoint{2.011924in}{0.150000in}}%
\pgfpathlineto{\pgfqpoint{2.000501in}{0.150000in}}%
\pgfpathlineto{\pgfqpoint{1.989078in}{0.150000in}}%
\pgfpathlineto{\pgfqpoint{1.977655in}{0.150000in}}%
\pgfpathlineto{\pgfqpoint{1.966232in}{0.150000in}}%
\pgfpathlineto{\pgfqpoint{1.954810in}{0.150000in}}%
\pgfpathlineto{\pgfqpoint{1.943387in}{0.150000in}}%
\pgfpathlineto{\pgfqpoint{1.931964in}{0.150000in}}%
\pgfpathlineto{\pgfqpoint{1.920541in}{0.150000in}}%
\pgfpathlineto{\pgfqpoint{1.909118in}{0.150000in}}%
\pgfpathlineto{\pgfqpoint{1.897695in}{0.150000in}}%
\pgfpathlineto{\pgfqpoint{1.886273in}{0.150000in}}%
\pgfpathlineto{\pgfqpoint{1.874850in}{0.150000in}}%
\pgfpathlineto{\pgfqpoint{1.863427in}{0.150000in}}%
\pgfpathlineto{\pgfqpoint{1.852004in}{0.150000in}}%
\pgfpathlineto{\pgfqpoint{1.840581in}{0.150000in}}%
\pgfpathlineto{\pgfqpoint{1.829158in}{0.150000in}}%
\pgfpathlineto{\pgfqpoint{1.817735in}{0.150000in}}%
\pgfpathlineto{\pgfqpoint{1.806313in}{0.150000in}}%
\pgfpathlineto{\pgfqpoint{1.794890in}{0.150000in}}%
\pgfpathlineto{\pgfqpoint{1.783467in}{0.150000in}}%
\pgfpathlineto{\pgfqpoint{1.772044in}{0.150000in}}%
\pgfpathlineto{\pgfqpoint{1.760621in}{0.150000in}}%
\pgfpathlineto{\pgfqpoint{1.749198in}{0.150000in}}%
\pgfpathlineto{\pgfqpoint{1.737776in}{0.150000in}}%
\pgfpathlineto{\pgfqpoint{1.726353in}{0.150000in}}%
\pgfpathlineto{\pgfqpoint{1.714930in}{0.150000in}}%
\pgfpathlineto{\pgfqpoint{1.703507in}{0.150000in}}%
\pgfpathlineto{\pgfqpoint{1.692084in}{0.150000in}}%
\pgfpathlineto{\pgfqpoint{1.680661in}{0.150000in}}%
\pgfpathlineto{\pgfqpoint{1.669238in}{0.150000in}}%
\pgfpathlineto{\pgfqpoint{1.657816in}{0.150000in}}%
\pgfpathlineto{\pgfqpoint{1.646393in}{0.150000in}}%
\pgfpathlineto{\pgfqpoint{1.634970in}{0.150000in}}%
\pgfpathlineto{\pgfqpoint{1.623547in}{0.150000in}}%
\pgfpathlineto{\pgfqpoint{1.612124in}{0.150000in}}%
\pgfpathlineto{\pgfqpoint{1.600701in}{0.150000in}}%
\pgfpathlineto{\pgfqpoint{1.589279in}{0.150000in}}%
\pgfpathlineto{\pgfqpoint{1.577856in}{0.150000in}}%
\pgfpathlineto{\pgfqpoint{1.566433in}{0.150000in}}%
\pgfpathlineto{\pgfqpoint{1.555010in}{0.150000in}}%
\pgfpathlineto{\pgfqpoint{1.543587in}{0.150000in}}%
\pgfpathlineto{\pgfqpoint{1.532164in}{0.150000in}}%
\pgfpathlineto{\pgfqpoint{1.520741in}{0.150000in}}%
\pgfpathlineto{\pgfqpoint{1.509319in}{0.150000in}}%
\pgfpathlineto{\pgfqpoint{1.497896in}{0.150000in}}%
\pgfpathlineto{\pgfqpoint{1.486473in}{0.150000in}}%
\pgfpathlineto{\pgfqpoint{1.475050in}{0.150000in}}%
\pgfpathlineto{\pgfqpoint{1.463627in}{0.150000in}}%
\pgfpathlineto{\pgfqpoint{1.452204in}{0.150000in}}%
\pgfpathlineto{\pgfqpoint{1.440782in}{0.150000in}}%
\pgfpathlineto{\pgfqpoint{1.429359in}{0.150000in}}%
\pgfpathlineto{\pgfqpoint{1.417936in}{0.150000in}}%
\pgfpathlineto{\pgfqpoint{1.406513in}{0.150000in}}%
\pgfpathlineto{\pgfqpoint{1.395090in}{0.150000in}}%
\pgfpathlineto{\pgfqpoint{1.383667in}{0.150000in}}%
\pgfpathlineto{\pgfqpoint{1.372244in}{0.150000in}}%
\pgfpathlineto{\pgfqpoint{1.360822in}{0.150000in}}%
\pgfpathlineto{\pgfqpoint{1.349399in}{0.150000in}}%
\pgfpathlineto{\pgfqpoint{1.337976in}{0.150000in}}%
\pgfpathlineto{\pgfqpoint{1.326553in}{0.150000in}}%
\pgfpathlineto{\pgfqpoint{1.315130in}{0.150000in}}%
\pgfpathlineto{\pgfqpoint{1.303707in}{0.150000in}}%
\pgfpathlineto{\pgfqpoint{1.292285in}{0.150000in}}%
\pgfpathlineto{\pgfqpoint{1.280862in}{0.150000in}}%
\pgfpathlineto{\pgfqpoint{1.269439in}{0.150000in}}%
\pgfpathlineto{\pgfqpoint{1.258016in}{0.150000in}}%
\pgfpathlineto{\pgfqpoint{1.246593in}{0.150000in}}%
\pgfpathlineto{\pgfqpoint{1.235170in}{0.150000in}}%
\pgfpathlineto{\pgfqpoint{1.223747in}{0.150000in}}%
\pgfpathlineto{\pgfqpoint{1.212325in}{0.150000in}}%
\pgfpathlineto{\pgfqpoint{1.200902in}{0.150000in}}%
\pgfpathlineto{\pgfqpoint{1.189479in}{0.150000in}}%
\pgfpathlineto{\pgfqpoint{1.178056in}{0.150000in}}%
\pgfpathlineto{\pgfqpoint{1.166633in}{0.150000in}}%
\pgfpathlineto{\pgfqpoint{1.155210in}{0.150000in}}%
\pgfpathlineto{\pgfqpoint{1.143788in}{0.150000in}}%
\pgfpathlineto{\pgfqpoint{1.132365in}{0.150000in}}%
\pgfpathlineto{\pgfqpoint{1.120942in}{0.150000in}}%
\pgfpathlineto{\pgfqpoint{1.109519in}{0.150000in}}%
\pgfpathlineto{\pgfqpoint{1.098096in}{0.150000in}}%
\pgfpathlineto{\pgfqpoint{1.086673in}{0.150000in}}%
\pgfpathlineto{\pgfqpoint{1.075251in}{0.150000in}}%
\pgfpathlineto{\pgfqpoint{1.063828in}{0.150000in}}%
\pgfpathlineto{\pgfqpoint{1.052405in}{0.150000in}}%
\pgfpathlineto{\pgfqpoint{1.040982in}{0.150000in}}%
\pgfpathlineto{\pgfqpoint{1.029559in}{0.150000in}}%
\pgfpathlineto{\pgfqpoint{1.018136in}{0.150000in}}%
\pgfpathlineto{\pgfqpoint{1.006713in}{0.150000in}}%
\pgfpathlineto{\pgfqpoint{0.995291in}{0.150000in}}%
\pgfpathlineto{\pgfqpoint{0.983868in}{0.150000in}}%
\pgfpathlineto{\pgfqpoint{0.972445in}{0.150000in}}%
\pgfpathlineto{\pgfqpoint{0.961022in}{0.150000in}}%
\pgfpathlineto{\pgfqpoint{0.949599in}{0.150000in}}%
\pgfpathlineto{\pgfqpoint{0.938176in}{0.150000in}}%
\pgfpathlineto{\pgfqpoint{0.926754in}{0.150000in}}%
\pgfpathlineto{\pgfqpoint{0.915331in}{0.150000in}}%
\pgfpathlineto{\pgfqpoint{0.903908in}{0.150000in}}%
\pgfpathlineto{\pgfqpoint{0.892485in}{0.150000in}}%
\pgfpathlineto{\pgfqpoint{0.881062in}{0.150000in}}%
\pgfpathlineto{\pgfqpoint{0.869639in}{0.150000in}}%
\pgfpathlineto{\pgfqpoint{0.858216in}{0.150000in}}%
\pgfpathlineto{\pgfqpoint{0.846794in}{0.150000in}}%
\pgfpathlineto{\pgfqpoint{0.835371in}{0.150000in}}%
\pgfpathlineto{\pgfqpoint{0.823948in}{0.150000in}}%
\pgfpathlineto{\pgfqpoint{0.812525in}{0.150000in}}%
\pgfpathlineto{\pgfqpoint{0.801102in}{0.150000in}}%
\pgfpathlineto{\pgfqpoint{0.789679in}{0.150000in}}%
\pgfpathlineto{\pgfqpoint{0.778257in}{0.150000in}}%
\pgfpathlineto{\pgfqpoint{0.766834in}{0.150000in}}%
\pgfpathlineto{\pgfqpoint{0.755411in}{0.150000in}}%
\pgfpathlineto{\pgfqpoint{0.743988in}{0.150000in}}%
\pgfpathlineto{\pgfqpoint{0.732565in}{0.150000in}}%
\pgfpathlineto{\pgfqpoint{0.721142in}{0.150000in}}%
\pgfpathlineto{\pgfqpoint{0.709719in}{0.150000in}}%
\pgfpathlineto{\pgfqpoint{0.698297in}{0.150000in}}%
\pgfpathlineto{\pgfqpoint{0.686874in}{0.150000in}}%
\pgfpathlineto{\pgfqpoint{0.675451in}{0.150000in}}%
\pgfpathlineto{\pgfqpoint{0.664028in}{0.150000in}}%
\pgfpathlineto{\pgfqpoint{0.652605in}{0.150000in}}%
\pgfpathlineto{\pgfqpoint{0.641182in}{0.150000in}}%
\pgfpathlineto{\pgfqpoint{0.629760in}{0.150000in}}%
\pgfpathlineto{\pgfqpoint{0.618337in}{0.150000in}}%
\pgfpathlineto{\pgfqpoint{0.606914in}{0.150000in}}%
\pgfpathlineto{\pgfqpoint{0.595491in}{0.150000in}}%
\pgfpathlineto{\pgfqpoint{0.584068in}{0.150000in}}%
\pgfpathlineto{\pgfqpoint{0.572645in}{0.150000in}}%
\pgfpathlineto{\pgfqpoint{0.561222in}{0.150000in}}%
\pgfpathlineto{\pgfqpoint{0.549800in}{0.150000in}}%
\pgfpathlineto{\pgfqpoint{0.538377in}{0.150000in}}%
\pgfpathlineto{\pgfqpoint{0.526954in}{0.150000in}}%
\pgfpathlineto{\pgfqpoint{0.515531in}{0.150000in}}%
\pgfpathlineto{\pgfqpoint{0.504108in}{0.150000in}}%
\pgfpathlineto{\pgfqpoint{0.492685in}{0.150000in}}%
\pgfpathlineto{\pgfqpoint{0.481263in}{0.150000in}}%
\pgfpathlineto{\pgfqpoint{0.469840in}{0.150000in}}%
\pgfpathlineto{\pgfqpoint{0.458417in}{0.150000in}}%
\pgfpathlineto{\pgfqpoint{0.446994in}{0.150000in}}%
\pgfpathlineto{\pgfqpoint{0.435571in}{0.150000in}}%
\pgfpathlineto{\pgfqpoint{0.424148in}{0.150000in}}%
\pgfpathlineto{\pgfqpoint{0.412725in}{0.150000in}}%
\pgfpathlineto{\pgfqpoint{0.401303in}{0.150000in}}%
\pgfpathlineto{\pgfqpoint{0.389880in}{0.150000in}}%
\pgfpathlineto{\pgfqpoint{0.378457in}{0.150000in}}%
\pgfpathlineto{\pgfqpoint{0.367034in}{0.150000in}}%
\pgfpathlineto{\pgfqpoint{0.355611in}{0.150000in}}%
\pgfpathlineto{\pgfqpoint{0.344188in}{0.150000in}}%
\pgfpathlineto{\pgfqpoint{0.332766in}{0.150000in}}%
\pgfpathlineto{\pgfqpoint{0.321343in}{0.150000in}}%
\pgfpathlineto{\pgfqpoint{0.309920in}{0.150000in}}%
\pgfpathlineto{\pgfqpoint{0.298497in}{0.150000in}}%
\pgfpathlineto{\pgfqpoint{0.287074in}{0.150000in}}%
\pgfpathlineto{\pgfqpoint{0.275651in}{0.150000in}}%
\pgfpathlineto{\pgfqpoint{0.264228in}{0.150000in}}%
\pgfpathlineto{\pgfqpoint{0.252806in}{0.150000in}}%
\pgfpathlineto{\pgfqpoint{0.241383in}{0.150000in}}%
\pgfpathlineto{\pgfqpoint{0.229960in}{0.150000in}}%
\pgfpathlineto{\pgfqpoint{0.218537in}{0.150000in}}%
\pgfpathlineto{\pgfqpoint{0.207114in}{0.150000in}}%
\pgfpathlineto{\pgfqpoint{0.195691in}{0.150000in}}%
\pgfpathlineto{\pgfqpoint{0.184269in}{0.150000in}}%
\pgfpathlineto{\pgfqpoint{0.172846in}{0.150000in}}%
\pgfpathlineto{\pgfqpoint{0.161423in}{0.150000in}}%
\pgfpathlineto{\pgfqpoint{0.150000in}{0.150000in}}%
\pgfpathclose%
\pgfusepath{stroke,fill}%
\end{pgfscope}%
\begin{pgfscope}%
\pgfpathrectangle{\pgfqpoint{0.150000in}{0.150000in}}{\pgfqpoint{5.700000in}{2.200000in}}%
\pgfusepath{clip}%
\pgfsetroundcap%
\pgfsetroundjoin%
\pgfsetlinewidth{2.007500pt}%
\definecolor{currentstroke}{rgb}{0.203922,0.396078,0.643137}%
\pgfsetstrokecolor{currentstroke}%
\pgfsetdash{}{0pt}%
\pgfpathmoveto{\pgfqpoint{0.136111in}{0.963080in}}%
\pgfpathlineto{\pgfqpoint{0.249953in}{1.010751in}}%
\pgfpathlineto{\pgfqpoint{0.536743in}{1.132977in}}%
\pgfpathlineto{\pgfqpoint{0.632340in}{1.170229in}}%
\pgfpathlineto{\pgfqpoint{0.727937in}{1.203578in}}%
\pgfpathlineto{\pgfqpoint{0.791668in}{1.223026in}}%
\pgfpathlineto{\pgfqpoint{0.855399in}{1.239853in}}%
\pgfpathlineto{\pgfqpoint{0.919130in}{1.253760in}}%
\pgfpathlineto{\pgfqpoint{0.982861in}{1.264498in}}%
\pgfpathlineto{\pgfqpoint{1.046592in}{1.271872in}}%
\pgfpathlineto{\pgfqpoint{1.110323in}{1.275744in}}%
\pgfpathlineto{\pgfqpoint{1.174055in}{1.276043in}}%
\pgfpathlineto{\pgfqpoint{1.237786in}{1.272763in}}%
\pgfpathlineto{\pgfqpoint{1.301517in}{1.265966in}}%
\pgfpathlineto{\pgfqpoint{1.365248in}{1.255777in}}%
\pgfpathlineto{\pgfqpoint{1.428979in}{1.242382in}}%
\pgfpathlineto{\pgfqpoint{1.492710in}{1.226024in}}%
\pgfpathlineto{\pgfqpoint{1.556442in}{1.206991in}}%
\pgfpathlineto{\pgfqpoint{1.652038in}{1.174156in}}%
\pgfpathlineto{\pgfqpoint{1.747635in}{1.137278in}}%
\pgfpathlineto{\pgfqpoint{1.875097in}{1.084071in}}%
\pgfpathlineto{\pgfqpoint{2.161888in}{0.961913in}}%
\pgfpathlineto{\pgfqpoint{2.257484in}{0.924206in}}%
\pgfpathlineto{\pgfqpoint{2.353081in}{0.889300in}}%
\pgfpathlineto{\pgfqpoint{2.448678in}{0.857674in}}%
\pgfpathlineto{\pgfqpoint{2.544274in}{0.829609in}}%
\pgfpathlineto{\pgfqpoint{2.639871in}{0.805215in}}%
\pgfpathlineto{\pgfqpoint{2.735468in}{0.784457in}}%
\pgfpathlineto{\pgfqpoint{2.831065in}{0.767193in}}%
\pgfpathlineto{\pgfqpoint{2.926661in}{0.753203in}}%
\pgfpathlineto{\pgfqpoint{3.022258in}{0.742227in}}%
\pgfpathlineto{\pgfqpoint{3.117855in}{0.733995in}}%
\pgfpathlineto{\pgfqpoint{3.213452in}{0.728246in}}%
\pgfpathlineto{\pgfqpoint{3.340914in}{0.724053in}}%
\pgfpathlineto{\pgfqpoint{3.468376in}{0.723448in}}%
\pgfpathlineto{\pgfqpoint{3.595839in}{0.726146in}}%
\pgfpathlineto{\pgfqpoint{3.723301in}{0.731990in}}%
\pgfpathlineto{\pgfqpoint{3.850763in}{0.740905in}}%
\pgfpathlineto{\pgfqpoint{3.978226in}{0.752832in}}%
\pgfpathlineto{\pgfqpoint{4.105688in}{0.767657in}}%
\pgfpathlineto{\pgfqpoint{4.265016in}{0.789876in}}%
\pgfpathlineto{\pgfqpoint{4.424344in}{0.815329in}}%
\pgfpathlineto{\pgfqpoint{4.870462in}{0.889490in}}%
\pgfpathlineto{\pgfqpoint{4.997924in}{0.906533in}}%
\pgfpathlineto{\pgfqpoint{5.093521in}{0.916745in}}%
\pgfpathlineto{\pgfqpoint{5.189117in}{0.924298in}}%
\pgfpathlineto{\pgfqpoint{5.284714in}{0.928884in}}%
\pgfpathlineto{\pgfqpoint{5.380311in}{0.930313in}}%
\pgfpathlineto{\pgfqpoint{5.475908in}{0.928527in}}%
\pgfpathlineto{\pgfqpoint{5.571504in}{0.923598in}}%
\pgfpathlineto{\pgfqpoint{5.667101in}{0.915731in}}%
\pgfpathlineto{\pgfqpoint{5.762698in}{0.905244in}}%
\pgfpathlineto{\pgfqpoint{5.863889in}{0.891735in}}%
\pgfpathlineto{\pgfqpoint{5.863889in}{0.891735in}}%
\pgfusepath{stroke}%
\end{pgfscope}%
\begin{pgfscope}%
\pgfpathrectangle{\pgfqpoint{0.150000in}{0.150000in}}{\pgfqpoint{5.700000in}{2.200000in}}%
\pgfusepath{clip}%
\pgfsetroundcap%
\pgfsetroundjoin%
\pgfsetlinewidth{1.505625pt}%
\definecolor{currentstroke}{rgb}{0.000000,0.501961,0.000000}%
\pgfsetstrokecolor{currentstroke}%
\pgfsetdash{}{0pt}%
\pgfpathmoveto{\pgfqpoint{0.150000in}{0.474362in}}%
\pgfpathlineto{\pgfqpoint{0.264228in}{0.481266in}}%
\pgfpathlineto{\pgfqpoint{0.355611in}{0.483840in}}%
\pgfpathlineto{\pgfqpoint{0.435571in}{0.483347in}}%
\pgfpathlineto{\pgfqpoint{0.515531in}{0.479839in}}%
\pgfpathlineto{\pgfqpoint{0.595491in}{0.472937in}}%
\pgfpathlineto{\pgfqpoint{0.664028in}{0.464086in}}%
\pgfpathlineto{\pgfqpoint{0.732565in}{0.452378in}}%
\pgfpathlineto{\pgfqpoint{0.801102in}{0.437730in}}%
\pgfpathlineto{\pgfqpoint{0.869639in}{0.420126in}}%
\pgfpathlineto{\pgfqpoint{0.938176in}{0.399613in}}%
\pgfpathlineto{\pgfqpoint{1.018136in}{0.372162in}}%
\pgfpathlineto{\pgfqpoint{1.098096in}{0.341224in}}%
\pgfpathlineto{\pgfqpoint{1.143788in}{0.322675in}}%
\pgfpathlineto{\pgfqpoint{1.155210in}{0.324100in}}%
\pgfpathlineto{\pgfqpoint{1.280862in}{0.373621in}}%
\pgfpathlineto{\pgfqpoint{1.360822in}{0.400888in}}%
\pgfpathlineto{\pgfqpoint{1.429359in}{0.421235in}}%
\pgfpathlineto{\pgfqpoint{1.497896in}{0.438667in}}%
\pgfpathlineto{\pgfqpoint{1.566433in}{0.453143in}}%
\pgfpathlineto{\pgfqpoint{1.634970in}{0.464683in}}%
\pgfpathlineto{\pgfqpoint{1.703507in}{0.473372in}}%
\pgfpathlineto{\pgfqpoint{1.783467in}{0.480098in}}%
\pgfpathlineto{\pgfqpoint{1.863427in}{0.483449in}}%
\pgfpathlineto{\pgfqpoint{1.943387in}{0.483806in}}%
\pgfpathlineto{\pgfqpoint{2.034770in}{0.481110in}}%
\pgfpathlineto{\pgfqpoint{2.137575in}{0.474948in}}%
\pgfpathlineto{\pgfqpoint{2.274649in}{0.463240in}}%
\pgfpathlineto{\pgfqpoint{2.822946in}{0.412019in}}%
\pgfpathlineto{\pgfqpoint{2.982866in}{0.402299in}}%
\pgfpathlineto{\pgfqpoint{3.142786in}{0.395780in}}%
\pgfpathlineto{\pgfqpoint{3.314128in}{0.392043in}}%
\pgfpathlineto{\pgfqpoint{3.496894in}{0.391189in}}%
\pgfpathlineto{\pgfqpoint{3.725351in}{0.393479in}}%
\pgfpathlineto{\pgfqpoint{4.159419in}{0.398948in}}%
\pgfpathlineto{\pgfqpoint{4.296493in}{0.396903in}}%
\pgfpathlineto{\pgfqpoint{4.410721in}{0.391994in}}%
\pgfpathlineto{\pgfqpoint{4.513527in}{0.384262in}}%
\pgfpathlineto{\pgfqpoint{4.604910in}{0.374211in}}%
\pgfpathlineto{\pgfqpoint{4.696293in}{0.360766in}}%
\pgfpathlineto{\pgfqpoint{4.776253in}{0.345982in}}%
\pgfpathlineto{\pgfqpoint{4.856212in}{0.328249in}}%
\pgfpathlineto{\pgfqpoint{4.936172in}{0.307529in}}%
\pgfpathlineto{\pgfqpoint{5.016132in}{0.283870in}}%
\pgfpathlineto{\pgfqpoint{5.107515in}{0.253426in}}%
\pgfpathlineto{\pgfqpoint{5.198898in}{0.219704in}}%
\pgfpathlineto{\pgfqpoint{5.301703in}{0.178501in}}%
\pgfpathlineto{\pgfqpoint{5.370240in}{0.150000in}}%
\pgfpathlineto{\pgfqpoint{5.381663in}{0.150689in}}%
\pgfpathlineto{\pgfqpoint{5.598697in}{0.237647in}}%
\pgfpathlineto{\pgfqpoint{5.690080in}{0.269716in}}%
\pgfpathlineto{\pgfqpoint{5.781463in}{0.298296in}}%
\pgfpathlineto{\pgfqpoint{5.850000in}{0.317265in}}%
\pgfpathlineto{\pgfqpoint{5.850000in}{0.317265in}}%
\pgfusepath{stroke}%
\end{pgfscope}%
\begin{pgfscope}%
\pgfpathrectangle{\pgfqpoint{0.150000in}{0.150000in}}{\pgfqpoint{5.700000in}{2.200000in}}%
\pgfusepath{clip}%
\pgfsetbuttcap%
\pgfsetroundjoin%
\pgfsetlinewidth{1.505625pt}%
\definecolor{currentstroke}{rgb}{0.000000,0.000000,0.000000}%
\pgfsetstrokecolor{currentstroke}%
\pgfsetdash{{3.000000pt}{3.000000pt}}{0.000000pt}%
\pgfpathmoveto{\pgfqpoint{0.150000in}{0.810914in}}%
\pgfpathlineto{\pgfqpoint{0.229960in}{0.827485in}}%
\pgfpathlineto{\pgfqpoint{0.298497in}{0.844453in}}%
\pgfpathlineto{\pgfqpoint{0.367034in}{0.864529in}}%
\pgfpathlineto{\pgfqpoint{0.424148in}{0.884079in}}%
\pgfpathlineto{\pgfqpoint{0.481263in}{0.906614in}}%
\pgfpathlineto{\pgfqpoint{0.538377in}{0.932543in}}%
\pgfpathlineto{\pgfqpoint{0.595491in}{0.962245in}}%
\pgfpathlineto{\pgfqpoint{0.652605in}{0.995983in}}%
\pgfpathlineto{\pgfqpoint{0.709719in}{1.033771in}}%
\pgfpathlineto{\pgfqpoint{0.766834in}{1.075180in}}%
\pgfpathlineto{\pgfqpoint{0.961022in}{1.219987in}}%
\pgfpathlineto{\pgfqpoint{0.995291in}{1.239964in}}%
\pgfpathlineto{\pgfqpoint{1.029559in}{1.256161in}}%
\pgfpathlineto{\pgfqpoint{1.063828in}{1.267922in}}%
\pgfpathlineto{\pgfqpoint{1.086673in}{1.273094in}}%
\pgfpathlineto{\pgfqpoint{1.109519in}{1.276082in}}%
\pgfpathlineto{\pgfqpoint{1.132365in}{1.276928in}}%
\pgfpathlineto{\pgfqpoint{1.166633in}{1.274445in}}%
\pgfpathlineto{\pgfqpoint{1.200902in}{1.268079in}}%
\pgfpathlineto{\pgfqpoint{1.246593in}{1.255202in}}%
\pgfpathlineto{\pgfqpoint{1.383667in}{1.211971in}}%
\pgfpathlineto{\pgfqpoint{1.417936in}{1.205535in}}%
\pgfpathlineto{\pgfqpoint{1.452204in}{1.202364in}}%
\pgfpathlineto{\pgfqpoint{1.486473in}{1.202887in}}%
\pgfpathlineto{\pgfqpoint{1.520741in}{1.207337in}}%
\pgfpathlineto{\pgfqpoint{1.555010in}{1.215766in}}%
\pgfpathlineto{\pgfqpoint{1.589279in}{1.228055in}}%
\pgfpathlineto{\pgfqpoint{1.623547in}{1.243920in}}%
\pgfpathlineto{\pgfqpoint{1.657816in}{1.262929in}}%
\pgfpathlineto{\pgfqpoint{1.703507in}{1.292175in}}%
\pgfpathlineto{\pgfqpoint{1.772044in}{1.340944in}}%
\pgfpathlineto{\pgfqpoint{1.852004in}{1.397319in}}%
\pgfpathlineto{\pgfqpoint{1.897695in}{1.425274in}}%
\pgfpathlineto{\pgfqpoint{1.931964in}{1.442737in}}%
\pgfpathlineto{\pgfqpoint{1.966232in}{1.456438in}}%
\pgfpathlineto{\pgfqpoint{2.000501in}{1.465831in}}%
\pgfpathlineto{\pgfqpoint{2.023347in}{1.469500in}}%
\pgfpathlineto{\pgfqpoint{2.046192in}{1.471006in}}%
\pgfpathlineto{\pgfqpoint{2.069038in}{1.470309in}}%
\pgfpathlineto{\pgfqpoint{2.091884in}{1.467409in}}%
\pgfpathlineto{\pgfqpoint{2.114729in}{1.462340in}}%
\pgfpathlineto{\pgfqpoint{2.148998in}{1.450826in}}%
\pgfpathlineto{\pgfqpoint{2.183267in}{1.434958in}}%
\pgfpathlineto{\pgfqpoint{2.217535in}{1.415246in}}%
\pgfpathlineto{\pgfqpoint{2.251804in}{1.392314in}}%
\pgfpathlineto{\pgfqpoint{2.297495in}{1.357965in}}%
\pgfpathlineto{\pgfqpoint{2.377455in}{1.292569in}}%
\pgfpathlineto{\pgfqpoint{2.457415in}{1.228451in}}%
\pgfpathlineto{\pgfqpoint{2.503106in}{1.195421in}}%
\pgfpathlineto{\pgfqpoint{2.548798in}{1.166345in}}%
\pgfpathlineto{\pgfqpoint{2.583066in}{1.147540in}}%
\pgfpathlineto{\pgfqpoint{2.617335in}{1.131490in}}%
\pgfpathlineto{\pgfqpoint{2.651603in}{1.118251in}}%
\pgfpathlineto{\pgfqpoint{2.685872in}{1.107793in}}%
\pgfpathlineto{\pgfqpoint{2.720140in}{1.100017in}}%
\pgfpathlineto{\pgfqpoint{2.754409in}{1.094765in}}%
\pgfpathlineto{\pgfqpoint{2.800100in}{1.091343in}}%
\pgfpathlineto{\pgfqpoint{2.845792in}{1.091512in}}%
\pgfpathlineto{\pgfqpoint{2.891483in}{1.094698in}}%
\pgfpathlineto{\pgfqpoint{2.948597in}{1.102058in}}%
\pgfpathlineto{\pgfqpoint{3.017134in}{1.114505in}}%
\pgfpathlineto{\pgfqpoint{3.108517in}{1.134817in}}%
\pgfpathlineto{\pgfqpoint{3.439780in}{1.211941in}}%
\pgfpathlineto{\pgfqpoint{3.542585in}{1.231060in}}%
\pgfpathlineto{\pgfqpoint{3.633968in}{1.244863in}}%
\pgfpathlineto{\pgfqpoint{3.725351in}{1.255270in}}%
\pgfpathlineto{\pgfqpoint{3.805311in}{1.261370in}}%
\pgfpathlineto{\pgfqpoint{3.885271in}{1.264524in}}%
\pgfpathlineto{\pgfqpoint{3.965230in}{1.264655in}}%
\pgfpathlineto{\pgfqpoint{4.045190in}{1.261739in}}%
\pgfpathlineto{\pgfqpoint{4.125150in}{1.255806in}}%
\pgfpathlineto{\pgfqpoint{4.205110in}{1.246938in}}%
\pgfpathlineto{\pgfqpoint{4.285070in}{1.235268in}}%
\pgfpathlineto{\pgfqpoint{4.376453in}{1.218729in}}%
\pgfpathlineto{\pgfqpoint{4.467836in}{1.199095in}}%
\pgfpathlineto{\pgfqpoint{4.570641in}{1.173793in}}%
\pgfpathlineto{\pgfqpoint{4.684870in}{1.142416in}}%
\pgfpathlineto{\pgfqpoint{4.821944in}{1.101476in}}%
\pgfpathlineto{\pgfqpoint{5.050401in}{1.029416in}}%
\pgfpathlineto{\pgfqpoint{5.278858in}{0.958471in}}%
\pgfpathlineto{\pgfqpoint{5.427355in}{0.915597in}}%
\pgfpathlineto{\pgfqpoint{5.553006in}{0.882283in}}%
\pgfpathlineto{\pgfqpoint{5.678657in}{0.852131in}}%
\pgfpathlineto{\pgfqpoint{5.804309in}{0.825366in}}%
\pgfpathlineto{\pgfqpoint{5.850000in}{0.816489in}}%
\pgfpathlineto{\pgfqpoint{5.850000in}{0.816489in}}%
\pgfusepath{stroke}%
\end{pgfscope}%
\begin{pgfscope}%
\pgfpathrectangle{\pgfqpoint{0.150000in}{0.150000in}}{\pgfqpoint{5.700000in}{2.200000in}}%
\pgfusepath{clip}%
\pgfsetbuttcap%
\pgfsetroundjoin%
\definecolor{currentfill}{rgb}{1.000000,0.000000,0.000000}%
\pgfsetfillcolor{currentfill}%
\pgfsetlinewidth{1.003750pt}%
\definecolor{currentstroke}{rgb}{1.000000,0.000000,0.000000}%
\pgfsetstrokecolor{currentstroke}%
\pgfsetdash{}{0pt}%
\pgfsys@defobject{currentmarker}{\pgfqpoint{-0.041667in}{-0.041667in}}{\pgfqpoint{0.041667in}{0.041667in}}{%
\pgfpathmoveto{\pgfqpoint{0.000000in}{-0.041667in}}%
\pgfpathcurveto{\pgfqpoint{0.011050in}{-0.041667in}}{\pgfqpoint{0.021649in}{-0.037276in}}{\pgfqpoint{0.029463in}{-0.029463in}}%
\pgfpathcurveto{\pgfqpoint{0.037276in}{-0.021649in}}{\pgfqpoint{0.041667in}{-0.011050in}}{\pgfqpoint{0.041667in}{0.000000in}}%
\pgfpathcurveto{\pgfqpoint{0.041667in}{0.011050in}}{\pgfqpoint{0.037276in}{0.021649in}}{\pgfqpoint{0.029463in}{0.029463in}}%
\pgfpathcurveto{\pgfqpoint{0.021649in}{0.037276in}}{\pgfqpoint{0.011050in}{0.041667in}}{\pgfqpoint{0.000000in}{0.041667in}}%
\pgfpathcurveto{\pgfqpoint{-0.011050in}{0.041667in}}{\pgfqpoint{-0.021649in}{0.037276in}}{\pgfqpoint{-0.029463in}{0.029463in}}%
\pgfpathcurveto{\pgfqpoint{-0.037276in}{0.021649in}}{\pgfqpoint{-0.041667in}{0.011050in}}{\pgfqpoint{-0.041667in}{0.000000in}}%
\pgfpathcurveto{\pgfqpoint{-0.041667in}{-0.011050in}}{\pgfqpoint{-0.037276in}{-0.021649in}}{\pgfqpoint{-0.029463in}{-0.029463in}}%
\pgfpathcurveto{\pgfqpoint{-0.021649in}{-0.037276in}}{\pgfqpoint{-0.011050in}{-0.041667in}}{\pgfqpoint{0.000000in}{-0.041667in}}%
\pgfpathclose%
\pgfusepath{stroke,fill}%
}%
\begin{pgfscope}%
\pgfsys@transformshift{5.375000in}{0.930321in}%
\pgfsys@useobject{currentmarker}{}%
\end{pgfscope}%
\end{pgfscope}%
\begin{pgfscope}%
\pgfpathrectangle{\pgfqpoint{0.150000in}{0.150000in}}{\pgfqpoint{5.700000in}{2.200000in}}%
\pgfusepath{clip}%
\pgfsetbuttcap%
\pgfsetmiterjoin%
\definecolor{currentfill}{rgb}{1.000000,0.000000,0.000000}%
\pgfsetfillcolor{currentfill}%
\pgfsetlinewidth{1.003750pt}%
\definecolor{currentstroke}{rgb}{1.000000,0.000000,0.000000}%
\pgfsetstrokecolor{currentstroke}%
\pgfsetdash{}{0pt}%
\pgfsys@defobject{currentmarker}{\pgfqpoint{-0.041667in}{-0.041667in}}{\pgfqpoint{0.041667in}{0.041667in}}{%
\pgfpathmoveto{\pgfqpoint{-0.000000in}{-0.041667in}}%
\pgfpathlineto{\pgfqpoint{0.041667in}{0.041667in}}%
\pgfpathlineto{\pgfqpoint{-0.041667in}{0.041667in}}%
\pgfpathclose%
\pgfusepath{stroke,fill}%
}%
\begin{pgfscope}%
\pgfsys@transformshift{1.909118in}{0.538993in}%
\pgfsys@useobject{currentmarker}{}%
\end{pgfscope}%
\end{pgfscope}%
\begin{pgfscope}%
\pgfsetrectcap%
\pgfsetmiterjoin%
\pgfsetlinewidth{0.000000pt}%
\definecolor{currentstroke}{rgb}{1.000000,1.000000,1.000000}%
\pgfsetstrokecolor{currentstroke}%
\pgfsetdash{}{0pt}%
\pgfpathmoveto{\pgfqpoint{0.150000in}{0.150000in}}%
\pgfpathlineto{\pgfqpoint{0.150000in}{2.350000in}}%
\pgfusepath{}%
\end{pgfscope}%
\begin{pgfscope}%
\pgfsetrectcap%
\pgfsetmiterjoin%
\pgfsetlinewidth{0.000000pt}%
\definecolor{currentstroke}{rgb}{1.000000,1.000000,1.000000}%
\pgfsetstrokecolor{currentstroke}%
\pgfsetdash{}{0pt}%
\pgfpathmoveto{\pgfqpoint{5.850000in}{0.150000in}}%
\pgfpathlineto{\pgfqpoint{5.850000in}{2.350000in}}%
\pgfusepath{}%
\end{pgfscope}%
\begin{pgfscope}%
\pgfsetrectcap%
\pgfsetmiterjoin%
\pgfsetlinewidth{0.000000pt}%
\definecolor{currentstroke}{rgb}{1.000000,1.000000,1.000000}%
\pgfsetstrokecolor{currentstroke}%
\pgfsetdash{}{0pt}%
\pgfpathmoveto{\pgfqpoint{0.150000in}{0.150000in}}%
\pgfpathlineto{\pgfqpoint{5.850000in}{0.150000in}}%
\pgfusepath{}%
\end{pgfscope}%
\begin{pgfscope}%
\pgfsetrectcap%
\pgfsetmiterjoin%
\pgfsetlinewidth{0.000000pt}%
\definecolor{currentstroke}{rgb}{1.000000,1.000000,1.000000}%
\pgfsetstrokecolor{currentstroke}%
\pgfsetdash{}{0pt}%
\pgfpathmoveto{\pgfqpoint{0.150000in}{2.350000in}}%
\pgfpathlineto{\pgfqpoint{5.850000in}{2.350000in}}%
\pgfusepath{}%
\end{pgfscope}%
\begin{pgfscope}%
\pgfsetroundcap%
\pgfsetroundjoin%
\pgfsetlinewidth{2.007500pt}%
\definecolor{currentstroke}{rgb}{0.203922,0.396078,0.643137}%
\pgfsetstrokecolor{currentstroke}%
\pgfsetdash{}{0pt}%
\pgfpathmoveto{\pgfqpoint{4.254590in}{2.204470in}}%
\pgfpathlineto{\pgfqpoint{4.476812in}{2.204470in}}%
\pgfusepath{stroke}%
\end{pgfscope}%
\begin{pgfscope}%
\definecolor{textcolor}{rgb}{0.150000,0.150000,0.150000}%
\pgfsetstrokecolor{textcolor}%
\pgfsetfillcolor{textcolor}%
\pgftext[x=4.565701in,y=2.165582in,left,base]{\color{textcolor}\rmfamily\fontsize{8.000000}{9.600000}\selectfont Mean}%
\end{pgfscope}%
\begin{pgfscope}%
\pgfsetroundcap%
\pgfsetroundjoin%
\pgfsetlinewidth{1.505625pt}%
\definecolor{currentstroke}{rgb}{0.000000,0.501961,0.000000}%
\pgfsetstrokecolor{currentstroke}%
\pgfsetdash{}{0pt}%
\pgfpathmoveto{\pgfqpoint{4.254590in}{2.041385in}}%
\pgfpathlineto{\pgfqpoint{4.476812in}{2.041385in}}%
\pgfusepath{stroke}%
\end{pgfscope}%
\begin{pgfscope}%
\definecolor{textcolor}{rgb}{0.150000,0.150000,0.150000}%
\pgfsetstrokecolor{textcolor}%
\pgfsetfillcolor{textcolor}%
\pgftext[x=4.565701in,y=2.002496in,left,base]{\color{textcolor}\rmfamily\fontsize{8.000000}{9.600000}\selectfont Acquisition function}%
\end{pgfscope}%
\begin{pgfscope}%
\pgfsetbuttcap%
\pgfsetroundjoin%
\pgfsetlinewidth{1.505625pt}%
\definecolor{currentstroke}{rgb}{0.000000,0.000000,0.000000}%
\pgfsetstrokecolor{currentstroke}%
\pgfsetdash{{3.000000pt}{3.000000pt}}{0.000000pt}%
\pgfpathmoveto{\pgfqpoint{4.254590in}{1.878299in}}%
\pgfpathlineto{\pgfqpoint{4.476812in}{1.878299in}}%
\pgfusepath{stroke}%
\end{pgfscope}%
\begin{pgfscope}%
\definecolor{textcolor}{rgb}{0.150000,0.150000,0.150000}%
\pgfsetstrokecolor{textcolor}%
\pgfsetfillcolor{textcolor}%
\pgftext[x=4.565701in,y=1.839410in,left,base]{\color{textcolor}\rmfamily\fontsize{8.000000}{9.600000}\selectfont \(\displaystyle \mathrm{f}\)}%
\end{pgfscope}%
\begin{pgfscope}%
\pgfsetbuttcap%
\pgfsetroundjoin%
\definecolor{currentfill}{rgb}{1.000000,0.000000,0.000000}%
\pgfsetfillcolor{currentfill}%
\pgfsetlinewidth{1.003750pt}%
\definecolor{currentstroke}{rgb}{1.000000,0.000000,0.000000}%
\pgfsetstrokecolor{currentstroke}%
\pgfsetdash{}{0pt}%
\pgfsys@defobject{currentmarker}{\pgfqpoint{-0.041667in}{-0.041667in}}{\pgfqpoint{0.041667in}{0.041667in}}{%
\pgfpathmoveto{\pgfqpoint{0.000000in}{-0.041667in}}%
\pgfpathcurveto{\pgfqpoint{0.011050in}{-0.041667in}}{\pgfqpoint{0.021649in}{-0.037276in}}{\pgfqpoint{0.029463in}{-0.029463in}}%
\pgfpathcurveto{\pgfqpoint{0.037276in}{-0.021649in}}{\pgfqpoint{0.041667in}{-0.011050in}}{\pgfqpoint{0.041667in}{0.000000in}}%
\pgfpathcurveto{\pgfqpoint{0.041667in}{0.011050in}}{\pgfqpoint{0.037276in}{0.021649in}}{\pgfqpoint{0.029463in}{0.029463in}}%
\pgfpathcurveto{\pgfqpoint{0.021649in}{0.037276in}}{\pgfqpoint{0.011050in}{0.041667in}}{\pgfqpoint{0.000000in}{0.041667in}}%
\pgfpathcurveto{\pgfqpoint{-0.011050in}{0.041667in}}{\pgfqpoint{-0.021649in}{0.037276in}}{\pgfqpoint{-0.029463in}{0.029463in}}%
\pgfpathcurveto{\pgfqpoint{-0.037276in}{0.021649in}}{\pgfqpoint{-0.041667in}{0.011050in}}{\pgfqpoint{-0.041667in}{0.000000in}}%
\pgfpathcurveto{\pgfqpoint{-0.041667in}{-0.011050in}}{\pgfqpoint{-0.037276in}{-0.021649in}}{\pgfqpoint{-0.029463in}{-0.029463in}}%
\pgfpathcurveto{\pgfqpoint{-0.021649in}{-0.037276in}}{\pgfqpoint{-0.011050in}{-0.041667in}}{\pgfqpoint{0.000000in}{-0.041667in}}%
\pgfpathclose%
\pgfusepath{stroke,fill}%
}%
\begin{pgfscope}%
\pgfsys@transformshift{4.365701in}{1.715213in}%
\pgfsys@useobject{currentmarker}{}%
\end{pgfscope}%
\end{pgfscope}%
\begin{pgfscope}%
\definecolor{textcolor}{rgb}{0.150000,0.150000,0.150000}%
\pgfsetstrokecolor{textcolor}%
\pgfsetfillcolor{textcolor}%
\pgftext[x=4.565701in,y=1.676324in,left,base]{\color{textcolor}\rmfamily\fontsize{8.000000}{9.600000}\selectfont Previous observation}%
\end{pgfscope}%
\begin{pgfscope}%
\pgfsetbuttcap%
\pgfsetmiterjoin%
\definecolor{currentfill}{rgb}{1.000000,0.000000,0.000000}%
\pgfsetfillcolor{currentfill}%
\pgfsetlinewidth{1.003750pt}%
\definecolor{currentstroke}{rgb}{1.000000,0.000000,0.000000}%
\pgfsetstrokecolor{currentstroke}%
\pgfsetdash{}{0pt}%
\pgfsys@defobject{currentmarker}{\pgfqpoint{-0.041667in}{-0.041667in}}{\pgfqpoint{0.041667in}{0.041667in}}{%
\pgfpathmoveto{\pgfqpoint{-0.000000in}{-0.041667in}}%
\pgfpathlineto{\pgfqpoint{0.041667in}{0.041667in}}%
\pgfpathlineto{\pgfqpoint{-0.041667in}{0.041667in}}%
\pgfpathclose%
\pgfusepath{stroke,fill}%
}%
\begin{pgfscope}%
\pgfsys@transformshift{4.365701in}{1.552127in}%
\pgfsys@useobject{currentmarker}{}%
\end{pgfscope}%
\end{pgfscope}%
\begin{pgfscope}%
\definecolor{textcolor}{rgb}{0.150000,0.150000,0.150000}%
\pgfsetstrokecolor{textcolor}%
\pgfsetfillcolor{textcolor}%
\pgftext[x=4.565701in,y=1.513238in,left,base]{\color{textcolor}\rmfamily\fontsize{8.000000}{9.600000}\selectfont Next point to query}%
\end{pgfscope}%
\begin{pgfscope}%
\pgfsetbuttcap%
\pgfsetroundjoin%
\definecolor{currentfill}{rgb}{0.000000,0.000000,0.000000}%
\pgfsetfillcolor{currentfill}%
\pgfsetlinewidth{1.505625pt}%
\definecolor{currentstroke}{rgb}{0.000000,0.000000,0.000000}%
\pgfsetstrokecolor{currentstroke}%
\pgfsetdash{}{0pt}%
\pgfpathmoveto{\pgfqpoint{4.324034in}{1.336079in}}%
\pgfpathlineto{\pgfqpoint{4.407368in}{1.419412in}}%
\pgfpathmoveto{\pgfqpoint{4.324034in}{1.419412in}}%
\pgfpathlineto{\pgfqpoint{4.407368in}{1.336079in}}%
\pgfusepath{stroke,fill}%
\end{pgfscope}%
\begin{pgfscope}%
\definecolor{textcolor}{rgb}{0.150000,0.150000,0.150000}%
\pgfsetstrokecolor{textcolor}%
\pgfsetfillcolor{textcolor}%
\pgftext[x=4.565701in,y=1.348579in,left,base]{\color{textcolor}\rmfamily\fontsize{8.000000}{9.600000}\selectfont Data}%
\end{pgfscope}%
\begin{pgfscope}%
\pgfsetbuttcap%
\pgfsetmiterjoin%
\definecolor{currentfill}{rgb}{0.447059,0.623529,0.811765}%
\pgfsetfillcolor{currentfill}%
\pgfsetfillopacity{0.200000}%
\pgfsetlinewidth{0.501875pt}%
\definecolor{currentstroke}{rgb}{0.125490,0.290196,0.529412}%
\pgfsetstrokecolor{currentstroke}%
\pgfsetstrokeopacity{0.200000}%
\pgfsetdash{}{0pt}%
\pgfpathmoveto{\pgfqpoint{4.254590in}{1.185493in}}%
\pgfpathlineto{\pgfqpoint{4.476812in}{1.185493in}}%
\pgfpathlineto{\pgfqpoint{4.476812in}{1.263271in}}%
\pgfpathlineto{\pgfqpoint{4.254590in}{1.263271in}}%
\pgfpathclose%
\pgfusepath{stroke,fill}%
\end{pgfscope}%
\begin{pgfscope}%
\definecolor{textcolor}{rgb}{0.150000,0.150000,0.150000}%
\pgfsetstrokecolor{textcolor}%
\pgfsetfillcolor{textcolor}%
\pgftext[x=4.565701in,y=1.185493in,left,base]{\color{textcolor}\rmfamily\fontsize{8.000000}{9.600000}\selectfont Confidence}%
\end{pgfscope}%
\end{pgfpicture}%
\makeatother%
\endgroup%

            %% Creator: Matplotlib, PGF backend
%%
%% To include the figure in your LaTeX document, write
%%   \input{<filename>.pgf}
%%
%% Make sure the required packages are loaded in your preamble
%%   \usepackage{pgf}
%%
%% Figures using additional raster images can only be included by \input if
%% they are in the same directory as the main LaTeX file. For loading figures
%% from other directories you can use the `import` package
%%   \usepackage{import}
%% and then include the figures with
%%   \import{<path to file>}{<filename>.pgf}
%%
%% Matplotlib used the following preamble
%%   \usepackage{gensymb}
%%   \usepackage{fontspec}
%%   \setmainfont{DejaVu Serif}
%%   \setsansfont{Arial}
%%   \setmonofont{DejaVu Sans Mono}
%%
\begingroup%
\makeatletter%
\begin{pgfpicture}%
\pgfpathrectangle{\pgfpointorigin}{\pgfqpoint{6.000000in}{2.500000in}}%
\pgfusepath{use as bounding box, clip}%
\begin{pgfscope}%
\pgfsetbuttcap%
\pgfsetmiterjoin%
\definecolor{currentfill}{rgb}{1.000000,1.000000,1.000000}%
\pgfsetfillcolor{currentfill}%
\pgfsetlinewidth{0.000000pt}%
\definecolor{currentstroke}{rgb}{1.000000,1.000000,1.000000}%
\pgfsetstrokecolor{currentstroke}%
\pgfsetdash{}{0pt}%
\pgfpathmoveto{\pgfqpoint{0.000000in}{0.000000in}}%
\pgfpathlineto{\pgfqpoint{6.000000in}{0.000000in}}%
\pgfpathlineto{\pgfqpoint{6.000000in}{2.500000in}}%
\pgfpathlineto{\pgfqpoint{0.000000in}{2.500000in}}%
\pgfpathclose%
\pgfusepath{fill}%
\end{pgfscope}%
\begin{pgfscope}%
\pgfsetbuttcap%
\pgfsetmiterjoin%
\definecolor{currentfill}{rgb}{0.917647,0.917647,0.949020}%
\pgfsetfillcolor{currentfill}%
\pgfsetlinewidth{0.000000pt}%
\definecolor{currentstroke}{rgb}{0.000000,0.000000,0.000000}%
\pgfsetstrokecolor{currentstroke}%
\pgfsetstrokeopacity{0.000000}%
\pgfsetdash{}{0pt}%
\pgfpathmoveto{\pgfqpoint{0.150000in}{0.150000in}}%
\pgfpathlineto{\pgfqpoint{5.850000in}{0.150000in}}%
\pgfpathlineto{\pgfqpoint{5.850000in}{2.350000in}}%
\pgfpathlineto{\pgfqpoint{0.150000in}{2.350000in}}%
\pgfpathclose%
\pgfusepath{fill}%
\end{pgfscope}%
\begin{pgfscope}%
\pgfpathrectangle{\pgfqpoint{0.150000in}{0.150000in}}{\pgfqpoint{5.700000in}{2.200000in}}%
\pgfusepath{clip}%
\pgfsetbuttcap%
\pgfsetroundjoin%
\definecolor{currentfill}{rgb}{0.000000,0.000000,0.000000}%
\pgfsetfillcolor{currentfill}%
\pgfsetlinewidth{1.505625pt}%
\definecolor{currentstroke}{rgb}{0.000000,0.000000,0.000000}%
\pgfsetstrokecolor{currentstroke}%
\pgfsetdash{}{0pt}%
\pgfpathmoveto{\pgfqpoint{1.105833in}{1.234693in}}%
\pgfpathlineto{\pgfqpoint{1.189167in}{1.318027in}}%
\pgfpathmoveto{\pgfqpoint{1.105833in}{1.318027in}}%
\pgfpathlineto{\pgfqpoint{1.189167in}{1.234693in}}%
\pgfusepath{stroke,fill}%
\end{pgfscope}%
\begin{pgfscope}%
\pgfpathrectangle{\pgfqpoint{0.150000in}{0.150000in}}{\pgfqpoint{5.700000in}{2.200000in}}%
\pgfusepath{clip}%
\pgfsetbuttcap%
\pgfsetroundjoin%
\definecolor{currentfill}{rgb}{0.000000,0.000000,0.000000}%
\pgfsetfillcolor{currentfill}%
\pgfsetlinewidth{1.505625pt}%
\definecolor{currentstroke}{rgb}{0.000000,0.000000,0.000000}%
\pgfsetstrokecolor{currentstroke}%
\pgfsetdash{}{0pt}%
\pgfpathmoveto{\pgfqpoint{5.333333in}{0.888654in}}%
\pgfpathlineto{\pgfqpoint{5.416667in}{0.971987in}}%
\pgfpathmoveto{\pgfqpoint{5.333333in}{0.971987in}}%
\pgfpathlineto{\pgfqpoint{5.416667in}{0.888654in}}%
\pgfusepath{stroke,fill}%
\end{pgfscope}%
\begin{pgfscope}%
\pgfpathrectangle{\pgfqpoint{0.150000in}{0.150000in}}{\pgfqpoint{5.700000in}{2.200000in}}%
\pgfusepath{clip}%
\pgfsetbuttcap%
\pgfsetroundjoin%
\definecolor{currentfill}{rgb}{0.000000,0.000000,0.000000}%
\pgfsetfillcolor{currentfill}%
\pgfsetlinewidth{1.505625pt}%
\definecolor{currentstroke}{rgb}{0.000000,0.000000,0.000000}%
\pgfsetstrokecolor{currentstroke}%
\pgfsetdash{}{0pt}%
\pgfpathmoveto{\pgfqpoint{1.867452in}{1.389807in}}%
\pgfpathlineto{\pgfqpoint{1.950785in}{1.473140in}}%
\pgfpathmoveto{\pgfqpoint{1.867452in}{1.473140in}}%
\pgfpathlineto{\pgfqpoint{1.950785in}{1.389807in}}%
\pgfusepath{stroke,fill}%
\end{pgfscope}%
\begin{pgfscope}%
\pgfpathrectangle{\pgfqpoint{0.150000in}{0.150000in}}{\pgfqpoint{5.700000in}{2.200000in}}%
\pgfusepath{clip}%
\pgfsetbuttcap%
\pgfsetroundjoin%
\definecolor{currentfill}{rgb}{0.447059,0.623529,0.811765}%
\pgfsetfillcolor{currentfill}%
\pgfsetfillopacity{0.200000}%
\pgfsetlinewidth{0.501875pt}%
\definecolor{currentstroke}{rgb}{0.125490,0.290196,0.529412}%
\pgfsetstrokecolor{currentstroke}%
\pgfsetstrokeopacity{0.200000}%
\pgfsetdash{}{0pt}%
\pgfpathmoveto{\pgfqpoint{0.090625in}{1.122662in}}%
\pgfpathlineto{\pgfqpoint{0.090625in}{0.529679in}}%
\pgfpathlineto{\pgfqpoint{0.122491in}{0.542930in}}%
\pgfpathlineto{\pgfqpoint{0.154356in}{0.556936in}}%
\pgfpathlineto{\pgfqpoint{0.186222in}{0.571709in}}%
\pgfpathlineto{\pgfqpoint{0.218087in}{0.587261in}}%
\pgfpathlineto{\pgfqpoint{0.249953in}{0.603598in}}%
\pgfpathlineto{\pgfqpoint{0.281818in}{0.620725in}}%
\pgfpathlineto{\pgfqpoint{0.313684in}{0.638642in}}%
\pgfpathlineto{\pgfqpoint{0.345550in}{0.657345in}}%
\pgfpathlineto{\pgfqpoint{0.377415in}{0.676825in}}%
\pgfpathlineto{\pgfqpoint{0.409281in}{0.697070in}}%
\pgfpathlineto{\pgfqpoint{0.441146in}{0.718062in}}%
\pgfpathlineto{\pgfqpoint{0.473012in}{0.739780in}}%
\pgfpathlineto{\pgfqpoint{0.504878in}{0.762197in}}%
\pgfpathlineto{\pgfqpoint{0.536743in}{0.785280in}}%
\pgfpathlineto{\pgfqpoint{0.568609in}{0.808991in}}%
\pgfpathlineto{\pgfqpoint{0.600474in}{0.833289in}}%
\pgfpathlineto{\pgfqpoint{0.632340in}{0.858126in}}%
\pgfpathlineto{\pgfqpoint{0.664205in}{0.883449in}}%
\pgfpathlineto{\pgfqpoint{0.696071in}{0.909200in}}%
\pgfpathlineto{\pgfqpoint{0.727937in}{0.935317in}}%
\pgfpathlineto{\pgfqpoint{0.759802in}{0.961732in}}%
\pgfpathlineto{\pgfqpoint{0.791668in}{0.988374in}}%
\pgfpathlineto{\pgfqpoint{0.823533in}{1.015167in}}%
\pgfpathlineto{\pgfqpoint{0.855399in}{1.042031in}}%
\pgfpathlineto{\pgfqpoint{0.887264in}{1.068883in}}%
\pgfpathlineto{\pgfqpoint{0.919130in}{1.095636in}}%
\pgfpathlineto{\pgfqpoint{0.950996in}{1.122202in}}%
\pgfpathlineto{\pgfqpoint{0.982861in}{1.148488in}}%
\pgfpathlineto{\pgfqpoint{1.014727in}{1.174401in}}%
\pgfpathlineto{\pgfqpoint{1.046592in}{1.199846in}}%
\pgfpathlineto{\pgfqpoint{1.078458in}{1.224721in}}%
\pgfpathlineto{\pgfqpoint{1.110323in}{1.248908in}}%
\pgfpathlineto{\pgfqpoint{1.142189in}{1.271843in}}%
\pgfpathlineto{\pgfqpoint{1.174055in}{1.281551in}}%
\pgfpathlineto{\pgfqpoint{1.205920in}{1.288208in}}%
\pgfpathlineto{\pgfqpoint{1.237786in}{1.294942in}}%
\pgfpathlineto{\pgfqpoint{1.269651in}{1.301791in}}%
\pgfpathlineto{\pgfqpoint{1.301517in}{1.308752in}}%
\pgfpathlineto{\pgfqpoint{1.333383in}{1.315812in}}%
\pgfpathlineto{\pgfqpoint{1.365248in}{1.322957in}}%
\pgfpathlineto{\pgfqpoint{1.397114in}{1.330166in}}%
\pgfpathlineto{\pgfqpoint{1.428979in}{1.337418in}}%
\pgfpathlineto{\pgfqpoint{1.460845in}{1.344690in}}%
\pgfpathlineto{\pgfqpoint{1.492710in}{1.351953in}}%
\pgfpathlineto{\pgfqpoint{1.524576in}{1.359182in}}%
\pgfpathlineto{\pgfqpoint{1.556442in}{1.366345in}}%
\pgfpathlineto{\pgfqpoint{1.588307in}{1.373411in}}%
\pgfpathlineto{\pgfqpoint{1.620173in}{1.380346in}}%
\pgfpathlineto{\pgfqpoint{1.652038in}{1.387117in}}%
\pgfpathlineto{\pgfqpoint{1.683904in}{1.393689in}}%
\pgfpathlineto{\pgfqpoint{1.715769in}{1.400026in}}%
\pgfpathlineto{\pgfqpoint{1.747635in}{1.406091in}}%
\pgfpathlineto{\pgfqpoint{1.779501in}{1.411849in}}%
\pgfpathlineto{\pgfqpoint{1.811366in}{1.417262in}}%
\pgfpathlineto{\pgfqpoint{1.843232in}{1.422289in}}%
\pgfpathlineto{\pgfqpoint{1.875097in}{1.426867in}}%
\pgfpathlineto{\pgfqpoint{1.906963in}{1.430133in}}%
\pgfpathlineto{\pgfqpoint{1.938829in}{1.418898in}}%
\pgfpathlineto{\pgfqpoint{1.970694in}{1.404280in}}%
\pgfpathlineto{\pgfqpoint{2.002560in}{1.388253in}}%
\pgfpathlineto{\pgfqpoint{2.034425in}{1.370928in}}%
\pgfpathlineto{\pgfqpoint{2.066291in}{1.352387in}}%
\pgfpathlineto{\pgfqpoint{2.098156in}{1.332714in}}%
\pgfpathlineto{\pgfqpoint{2.130022in}{1.311994in}}%
\pgfpathlineto{\pgfqpoint{2.161888in}{1.290313in}}%
\pgfpathlineto{\pgfqpoint{2.193753in}{1.267760in}}%
\pgfpathlineto{\pgfqpoint{2.225619in}{1.244428in}}%
\pgfpathlineto{\pgfqpoint{2.257484in}{1.220406in}}%
\pgfpathlineto{\pgfqpoint{2.289350in}{1.195790in}}%
\pgfpathlineto{\pgfqpoint{2.321215in}{1.170670in}}%
\pgfpathlineto{\pgfqpoint{2.353081in}{1.145139in}}%
\pgfpathlineto{\pgfqpoint{2.384947in}{1.119289in}}%
\pgfpathlineto{\pgfqpoint{2.416812in}{1.093208in}}%
\pgfpathlineto{\pgfqpoint{2.448678in}{1.066984in}}%
\pgfpathlineto{\pgfqpoint{2.480543in}{1.040703in}}%
\pgfpathlineto{\pgfqpoint{2.512409in}{1.014446in}}%
\pgfpathlineto{\pgfqpoint{2.544274in}{0.988292in}}%
\pgfpathlineto{\pgfqpoint{2.576140in}{0.962317in}}%
\pgfpathlineto{\pgfqpoint{2.608006in}{0.936591in}}%
\pgfpathlineto{\pgfqpoint{2.639871in}{0.911182in}}%
\pgfpathlineto{\pgfqpoint{2.671737in}{0.886153in}}%
\pgfpathlineto{\pgfqpoint{2.703602in}{0.861562in}}%
\pgfpathlineto{\pgfqpoint{2.735468in}{0.837463in}}%
\pgfpathlineto{\pgfqpoint{2.767334in}{0.813904in}}%
\pgfpathlineto{\pgfqpoint{2.799199in}{0.790932in}}%
\pgfpathlineto{\pgfqpoint{2.831065in}{0.768584in}}%
\pgfpathlineto{\pgfqpoint{2.862930in}{0.746896in}}%
\pgfpathlineto{\pgfqpoint{2.894796in}{0.725899in}}%
\pgfpathlineto{\pgfqpoint{2.926661in}{0.705618in}}%
\pgfpathlineto{\pgfqpoint{2.958527in}{0.686075in}}%
\pgfpathlineto{\pgfqpoint{2.990393in}{0.667288in}}%
\pgfpathlineto{\pgfqpoint{3.022258in}{0.649271in}}%
\pgfpathlineto{\pgfqpoint{3.054124in}{0.632033in}}%
\pgfpathlineto{\pgfqpoint{3.085989in}{0.615581in}}%
\pgfpathlineto{\pgfqpoint{3.117855in}{0.599918in}}%
\pgfpathlineto{\pgfqpoint{3.149720in}{0.585045in}}%
\pgfpathlineto{\pgfqpoint{3.181586in}{0.570960in}}%
\pgfpathlineto{\pgfqpoint{3.213452in}{0.557657in}}%
\pgfpathlineto{\pgfqpoint{3.245317in}{0.545130in}}%
\pgfpathlineto{\pgfqpoint{3.277183in}{0.533371in}}%
\pgfpathlineto{\pgfqpoint{3.309048in}{0.522368in}}%
\pgfpathlineto{\pgfqpoint{3.340914in}{0.512112in}}%
\pgfpathlineto{\pgfqpoint{3.372780in}{0.502588in}}%
\pgfpathlineto{\pgfqpoint{3.404645in}{0.493785in}}%
\pgfpathlineto{\pgfqpoint{3.436511in}{0.485687in}}%
\pgfpathlineto{\pgfqpoint{3.468376in}{0.478280in}}%
\pgfpathlineto{\pgfqpoint{3.500242in}{0.471550in}}%
\pgfpathlineto{\pgfqpoint{3.532107in}{0.465483in}}%
\pgfpathlineto{\pgfqpoint{3.563973in}{0.460065in}}%
\pgfpathlineto{\pgfqpoint{3.595839in}{0.455280in}}%
\pgfpathlineto{\pgfqpoint{3.627704in}{0.451117in}}%
\pgfpathlineto{\pgfqpoint{3.659570in}{0.447561in}}%
\pgfpathlineto{\pgfqpoint{3.691435in}{0.444602in}}%
\pgfpathlineto{\pgfqpoint{3.723301in}{0.442227in}}%
\pgfpathlineto{\pgfqpoint{3.755166in}{0.440426in}}%
\pgfpathlineto{\pgfqpoint{3.787032in}{0.439189in}}%
\pgfpathlineto{\pgfqpoint{3.818898in}{0.438508in}}%
\pgfpathlineto{\pgfqpoint{3.850763in}{0.438375in}}%
\pgfpathlineto{\pgfqpoint{3.882629in}{0.438782in}}%
\pgfpathlineto{\pgfqpoint{3.914494in}{0.439724in}}%
\pgfpathlineto{\pgfqpoint{3.946360in}{0.441196in}}%
\pgfpathlineto{\pgfqpoint{3.978226in}{0.443192in}}%
\pgfpathlineto{\pgfqpoint{4.010091in}{0.445709in}}%
\pgfpathlineto{\pgfqpoint{4.041957in}{0.448745in}}%
\pgfpathlineto{\pgfqpoint{4.073822in}{0.452296in}}%
\pgfpathlineto{\pgfqpoint{4.105688in}{0.456361in}}%
\pgfpathlineto{\pgfqpoint{4.137553in}{0.460937in}}%
\pgfpathlineto{\pgfqpoint{4.169419in}{0.466023in}}%
\pgfpathlineto{\pgfqpoint{4.201285in}{0.471617in}}%
\pgfpathlineto{\pgfqpoint{4.233150in}{0.477718in}}%
\pgfpathlineto{\pgfqpoint{4.265016in}{0.484324in}}%
\pgfpathlineto{\pgfqpoint{4.296881in}{0.491431in}}%
\pgfpathlineto{\pgfqpoint{4.328747in}{0.499038in}}%
\pgfpathlineto{\pgfqpoint{4.360612in}{0.507141in}}%
\pgfpathlineto{\pgfqpoint{4.392478in}{0.515734in}}%
\pgfpathlineto{\pgfqpoint{4.424344in}{0.524813in}}%
\pgfpathlineto{\pgfqpoint{4.456209in}{0.534371in}}%
\pgfpathlineto{\pgfqpoint{4.488075in}{0.544400in}}%
\pgfpathlineto{\pgfqpoint{4.519940in}{0.554891in}}%
\pgfpathlineto{\pgfqpoint{4.551806in}{0.565832in}}%
\pgfpathlineto{\pgfqpoint{4.583671in}{0.577210in}}%
\pgfpathlineto{\pgfqpoint{4.615537in}{0.589013in}}%
\pgfpathlineto{\pgfqpoint{4.647403in}{0.601223in}}%
\pgfpathlineto{\pgfqpoint{4.679268in}{0.613823in}}%
\pgfpathlineto{\pgfqpoint{4.711134in}{0.626792in}}%
\pgfpathlineto{\pgfqpoint{4.742999in}{0.640110in}}%
\pgfpathlineto{\pgfqpoint{4.774865in}{0.653752in}}%
\pgfpathlineto{\pgfqpoint{4.806731in}{0.667694in}}%
\pgfpathlineto{\pgfqpoint{4.838596in}{0.681907in}}%
\pgfpathlineto{\pgfqpoint{4.870462in}{0.696364in}}%
\pgfpathlineto{\pgfqpoint{4.902327in}{0.711033in}}%
\pgfpathlineto{\pgfqpoint{4.934193in}{0.725883in}}%
\pgfpathlineto{\pgfqpoint{4.966058in}{0.740880in}}%
\pgfpathlineto{\pgfqpoint{4.997924in}{0.755989in}}%
\pgfpathlineto{\pgfqpoint{5.029790in}{0.771174in}}%
\pgfpathlineto{\pgfqpoint{5.061655in}{0.786400in}}%
\pgfpathlineto{\pgfqpoint{5.093521in}{0.801627in}}%
\pgfpathlineto{\pgfqpoint{5.125386in}{0.816818in}}%
\pgfpathlineto{\pgfqpoint{5.157252in}{0.831936in}}%
\pgfpathlineto{\pgfqpoint{5.189117in}{0.846940in}}%
\pgfpathlineto{\pgfqpoint{5.220983in}{0.861794in}}%
\pgfpathlineto{\pgfqpoint{5.252849in}{0.876459in}}%
\pgfpathlineto{\pgfqpoint{5.284714in}{0.890895in}}%
\pgfpathlineto{\pgfqpoint{5.316580in}{0.905062in}}%
\pgfpathlineto{\pgfqpoint{5.348445in}{0.918892in}}%
\pgfpathlineto{\pgfqpoint{5.380311in}{0.927598in}}%
\pgfpathlineto{\pgfqpoint{5.412177in}{0.914303in}}%
\pgfpathlineto{\pgfqpoint{5.444042in}{0.900326in}}%
\pgfpathlineto{\pgfqpoint{5.475908in}{0.886042in}}%
\pgfpathlineto{\pgfqpoint{5.507773in}{0.871501in}}%
\pgfpathlineto{\pgfqpoint{5.539639in}{0.856742in}}%
\pgfpathlineto{\pgfqpoint{5.571504in}{0.841805in}}%
\pgfpathlineto{\pgfqpoint{5.603370in}{0.826725in}}%
\pgfpathlineto{\pgfqpoint{5.635236in}{0.811542in}}%
\pgfpathlineto{\pgfqpoint{5.667101in}{0.796292in}}%
\pgfpathlineto{\pgfqpoint{5.698967in}{0.781014in}}%
\pgfpathlineto{\pgfqpoint{5.730832in}{0.765744in}}%
\pgfpathlineto{\pgfqpoint{5.762698in}{0.750517in}}%
\pgfpathlineto{\pgfqpoint{5.794563in}{0.735371in}}%
\pgfpathlineto{\pgfqpoint{5.826429in}{0.720338in}}%
\pgfpathlineto{\pgfqpoint{5.858295in}{0.705452in}}%
\pgfpathlineto{\pgfqpoint{5.890160in}{0.690744in}}%
\pgfpathlineto{\pgfqpoint{5.922026in}{0.676243in}}%
\pgfpathlineto{\pgfqpoint{5.953891in}{0.661979in}}%
\pgfpathlineto{\pgfqpoint{5.985757in}{0.647977in}}%
\pgfpathlineto{\pgfqpoint{6.017622in}{0.634262in}}%
\pgfpathlineto{\pgfqpoint{6.049488in}{0.620855in}}%
\pgfpathlineto{\pgfqpoint{6.081354in}{0.607776in}}%
\pgfpathlineto{\pgfqpoint{6.113219in}{0.595044in}}%
\pgfpathlineto{\pgfqpoint{6.145085in}{0.582674in}}%
\pgfpathlineto{\pgfqpoint{6.176950in}{0.570679in}}%
\pgfpathlineto{\pgfqpoint{6.208816in}{0.559071in}}%
\pgfpathlineto{\pgfqpoint{6.240682in}{0.547859in}}%
\pgfpathlineto{\pgfqpoint{6.272547in}{0.537049in}}%
\pgfpathlineto{\pgfqpoint{6.304413in}{0.526647in}}%
\pgfpathlineto{\pgfqpoint{6.336278in}{0.516656in}}%
\pgfpathlineto{\pgfqpoint{6.368144in}{0.507077in}}%
\pgfpathlineto{\pgfqpoint{6.400009in}{0.497908in}}%
\pgfpathlineto{\pgfqpoint{6.431875in}{0.489148in}}%
\pgfpathlineto{\pgfqpoint{6.431875in}{1.106405in}}%
\pgfpathlineto{\pgfqpoint{6.431875in}{1.106405in}}%
\pgfpathlineto{\pgfqpoint{6.400009in}{1.107851in}}%
\pgfpathlineto{\pgfqpoint{6.368144in}{1.109083in}}%
\pgfpathlineto{\pgfqpoint{6.336278in}{1.110077in}}%
\pgfpathlineto{\pgfqpoint{6.304413in}{1.110806in}}%
\pgfpathlineto{\pgfqpoint{6.272547in}{1.111245in}}%
\pgfpathlineto{\pgfqpoint{6.240682in}{1.111366in}}%
\pgfpathlineto{\pgfqpoint{6.208816in}{1.111146in}}%
\pgfpathlineto{\pgfqpoint{6.176950in}{1.110557in}}%
\pgfpathlineto{\pgfqpoint{6.145085in}{1.109576in}}%
\pgfpathlineto{\pgfqpoint{6.113219in}{1.108177in}}%
\pgfpathlineto{\pgfqpoint{6.081354in}{1.106340in}}%
\pgfpathlineto{\pgfqpoint{6.049488in}{1.104041in}}%
\pgfpathlineto{\pgfqpoint{6.017622in}{1.101261in}}%
\pgfpathlineto{\pgfqpoint{5.985757in}{1.097980in}}%
\pgfpathlineto{\pgfqpoint{5.953891in}{1.094183in}}%
\pgfpathlineto{\pgfqpoint{5.922026in}{1.089854in}}%
\pgfpathlineto{\pgfqpoint{5.890160in}{1.084981in}}%
\pgfpathlineto{\pgfqpoint{5.858295in}{1.079554in}}%
\pgfpathlineto{\pgfqpoint{5.826429in}{1.073563in}}%
\pgfpathlineto{\pgfqpoint{5.794563in}{1.067005in}}%
\pgfpathlineto{\pgfqpoint{5.762698in}{1.059877in}}%
\pgfpathlineto{\pgfqpoint{5.730832in}{1.052177in}}%
\pgfpathlineto{\pgfqpoint{5.698967in}{1.043910in}}%
\pgfpathlineto{\pgfqpoint{5.667101in}{1.035080in}}%
\pgfpathlineto{\pgfqpoint{5.635236in}{1.025697in}}%
\pgfpathlineto{\pgfqpoint{5.603370in}{1.015770in}}%
\pgfpathlineto{\pgfqpoint{5.571504in}{1.005315in}}%
\pgfpathlineto{\pgfqpoint{5.539639in}{0.994348in}}%
\pgfpathlineto{\pgfqpoint{5.507773in}{0.982889in}}%
\pgfpathlineto{\pgfqpoint{5.475908in}{0.970961in}}%
\pgfpathlineto{\pgfqpoint{5.444042in}{0.958592in}}%
\pgfpathlineto{\pgfqpoint{5.412177in}{0.945823in}}%
\pgfpathlineto{\pgfqpoint{5.380311in}{0.933025in}}%
\pgfpathlineto{\pgfqpoint{5.348445in}{0.941514in}}%
\pgfpathlineto{\pgfqpoint{5.316580in}{0.954416in}}%
\pgfpathlineto{\pgfqpoint{5.284714in}{0.966947in}}%
\pgfpathlineto{\pgfqpoint{5.252849in}{0.979049in}}%
\pgfpathlineto{\pgfqpoint{5.220983in}{0.990693in}}%
\pgfpathlineto{\pgfqpoint{5.189117in}{1.001855in}}%
\pgfpathlineto{\pgfqpoint{5.157252in}{1.012514in}}%
\pgfpathlineto{\pgfqpoint{5.125386in}{1.022654in}}%
\pgfpathlineto{\pgfqpoint{5.093521in}{1.032260in}}%
\pgfpathlineto{\pgfqpoint{5.061655in}{1.041321in}}%
\pgfpathlineto{\pgfqpoint{5.029790in}{1.049828in}}%
\pgfpathlineto{\pgfqpoint{4.997924in}{1.057777in}}%
\pgfpathlineto{\pgfqpoint{4.966058in}{1.065163in}}%
\pgfpathlineto{\pgfqpoint{4.934193in}{1.071989in}}%
\pgfpathlineto{\pgfqpoint{4.902327in}{1.078257in}}%
\pgfpathlineto{\pgfqpoint{4.870462in}{1.083972in}}%
\pgfpathlineto{\pgfqpoint{4.838596in}{1.089145in}}%
\pgfpathlineto{\pgfqpoint{4.806731in}{1.093785in}}%
\pgfpathlineto{\pgfqpoint{4.774865in}{1.097908in}}%
\pgfpathlineto{\pgfqpoint{4.742999in}{1.101530in}}%
\pgfpathlineto{\pgfqpoint{4.711134in}{1.104668in}}%
\pgfpathlineto{\pgfqpoint{4.679268in}{1.107345in}}%
\pgfpathlineto{\pgfqpoint{4.647403in}{1.109582in}}%
\pgfpathlineto{\pgfqpoint{4.615537in}{1.111404in}}%
\pgfpathlineto{\pgfqpoint{4.583671in}{1.112838in}}%
\pgfpathlineto{\pgfqpoint{4.551806in}{1.113911in}}%
\pgfpathlineto{\pgfqpoint{4.519940in}{1.114652in}}%
\pgfpathlineto{\pgfqpoint{4.488075in}{1.115090in}}%
\pgfpathlineto{\pgfqpoint{4.456209in}{1.115257in}}%
\pgfpathlineto{\pgfqpoint{4.424344in}{1.115185in}}%
\pgfpathlineto{\pgfqpoint{4.392478in}{1.114904in}}%
\pgfpathlineto{\pgfqpoint{4.360612in}{1.114447in}}%
\pgfpathlineto{\pgfqpoint{4.328747in}{1.113848in}}%
\pgfpathlineto{\pgfqpoint{4.296881in}{1.113138in}}%
\pgfpathlineto{\pgfqpoint{4.265016in}{1.112350in}}%
\pgfpathlineto{\pgfqpoint{4.233150in}{1.111516in}}%
\pgfpathlineto{\pgfqpoint{4.201285in}{1.110668in}}%
\pgfpathlineto{\pgfqpoint{4.169419in}{1.109837in}}%
\pgfpathlineto{\pgfqpoint{4.137553in}{1.109054in}}%
\pgfpathlineto{\pgfqpoint{4.105688in}{1.108348in}}%
\pgfpathlineto{\pgfqpoint{4.073822in}{1.107749in}}%
\pgfpathlineto{\pgfqpoint{4.041957in}{1.107285in}}%
\pgfpathlineto{\pgfqpoint{4.010091in}{1.106982in}}%
\pgfpathlineto{\pgfqpoint{3.978226in}{1.106866in}}%
\pgfpathlineto{\pgfqpoint{3.946360in}{1.106962in}}%
\pgfpathlineto{\pgfqpoint{3.914494in}{1.107295in}}%
\pgfpathlineto{\pgfqpoint{3.882629in}{1.107885in}}%
\pgfpathlineto{\pgfqpoint{3.850763in}{1.108754in}}%
\pgfpathlineto{\pgfqpoint{3.818898in}{1.109921in}}%
\pgfpathlineto{\pgfqpoint{3.787032in}{1.111404in}}%
\pgfpathlineto{\pgfqpoint{3.755166in}{1.113220in}}%
\pgfpathlineto{\pgfqpoint{3.723301in}{1.115384in}}%
\pgfpathlineto{\pgfqpoint{3.691435in}{1.117909in}}%
\pgfpathlineto{\pgfqpoint{3.659570in}{1.120806in}}%
\pgfpathlineto{\pgfqpoint{3.627704in}{1.124085in}}%
\pgfpathlineto{\pgfqpoint{3.595839in}{1.127755in}}%
\pgfpathlineto{\pgfqpoint{3.563973in}{1.131821in}}%
\pgfpathlineto{\pgfqpoint{3.532107in}{1.136288in}}%
\pgfpathlineto{\pgfqpoint{3.500242in}{1.141157in}}%
\pgfpathlineto{\pgfqpoint{3.468376in}{1.146429in}}%
\pgfpathlineto{\pgfqpoint{3.436511in}{1.152101in}}%
\pgfpathlineto{\pgfqpoint{3.404645in}{1.158170in}}%
\pgfpathlineto{\pgfqpoint{3.372780in}{1.164629in}}%
\pgfpathlineto{\pgfqpoint{3.340914in}{1.171468in}}%
\pgfpathlineto{\pgfqpoint{3.309048in}{1.178678in}}%
\pgfpathlineto{\pgfqpoint{3.277183in}{1.186243in}}%
\pgfpathlineto{\pgfqpoint{3.245317in}{1.194149in}}%
\pgfpathlineto{\pgfqpoint{3.213452in}{1.202377in}}%
\pgfpathlineto{\pgfqpoint{3.181586in}{1.210906in}}%
\pgfpathlineto{\pgfqpoint{3.149720in}{1.219714in}}%
\pgfpathlineto{\pgfqpoint{3.117855in}{1.228776in}}%
\pgfpathlineto{\pgfqpoint{3.085989in}{1.238063in}}%
\pgfpathlineto{\pgfqpoint{3.054124in}{1.247547in}}%
\pgfpathlineto{\pgfqpoint{3.022258in}{1.257197in}}%
\pgfpathlineto{\pgfqpoint{2.990393in}{1.266978in}}%
\pgfpathlineto{\pgfqpoint{2.958527in}{1.276856in}}%
\pgfpathlineto{\pgfqpoint{2.926661in}{1.286794in}}%
\pgfpathlineto{\pgfqpoint{2.894796in}{1.296756in}}%
\pgfpathlineto{\pgfqpoint{2.862930in}{1.306700in}}%
\pgfpathlineto{\pgfqpoint{2.831065in}{1.316589in}}%
\pgfpathlineto{\pgfqpoint{2.799199in}{1.326382in}}%
\pgfpathlineto{\pgfqpoint{2.767334in}{1.336037in}}%
\pgfpathlineto{\pgfqpoint{2.735468in}{1.345514in}}%
\pgfpathlineto{\pgfqpoint{2.703602in}{1.354772in}}%
\pgfpathlineto{\pgfqpoint{2.671737in}{1.363772in}}%
\pgfpathlineto{\pgfqpoint{2.639871in}{1.372472in}}%
\pgfpathlineto{\pgfqpoint{2.608006in}{1.380836in}}%
\pgfpathlineto{\pgfqpoint{2.576140in}{1.388826in}}%
\pgfpathlineto{\pgfqpoint{2.544274in}{1.396406in}}%
\pgfpathlineto{\pgfqpoint{2.512409in}{1.403543in}}%
\pgfpathlineto{\pgfqpoint{2.480543in}{1.410206in}}%
\pgfpathlineto{\pgfqpoint{2.448678in}{1.416365in}}%
\pgfpathlineto{\pgfqpoint{2.416812in}{1.421995in}}%
\pgfpathlineto{\pgfqpoint{2.384947in}{1.427072in}}%
\pgfpathlineto{\pgfqpoint{2.353081in}{1.431576in}}%
\pgfpathlineto{\pgfqpoint{2.321215in}{1.435489in}}%
\pgfpathlineto{\pgfqpoint{2.289350in}{1.438799in}}%
\pgfpathlineto{\pgfqpoint{2.257484in}{1.441494in}}%
\pgfpathlineto{\pgfqpoint{2.225619in}{1.443568in}}%
\pgfpathlineto{\pgfqpoint{2.193753in}{1.445019in}}%
\pgfpathlineto{\pgfqpoint{2.161888in}{1.445846in}}%
\pgfpathlineto{\pgfqpoint{2.130022in}{1.446055in}}%
\pgfpathlineto{\pgfqpoint{2.098156in}{1.445653in}}%
\pgfpathlineto{\pgfqpoint{2.066291in}{1.444652in}}%
\pgfpathlineto{\pgfqpoint{2.034425in}{1.443067in}}%
\pgfpathlineto{\pgfqpoint{2.002560in}{1.440919in}}%
\pgfpathlineto{\pgfqpoint{1.970694in}{1.438236in}}%
\pgfpathlineto{\pgfqpoint{1.938829in}{1.435080in}}%
\pgfpathlineto{\pgfqpoint{1.906963in}{1.433386in}}%
\pgfpathlineto{\pgfqpoint{1.875097in}{1.444238in}}%
\pgfpathlineto{\pgfqpoint{1.843232in}{1.454425in}}%
\pgfpathlineto{\pgfqpoint{1.811366in}{1.463067in}}%
\pgfpathlineto{\pgfqpoint{1.779501in}{1.470092in}}%
\pgfpathlineto{\pgfqpoint{1.747635in}{1.475461in}}%
\pgfpathlineto{\pgfqpoint{1.715769in}{1.479145in}}%
\pgfpathlineto{\pgfqpoint{1.683904in}{1.481125in}}%
\pgfpathlineto{\pgfqpoint{1.652038in}{1.481388in}}%
\pgfpathlineto{\pgfqpoint{1.620173in}{1.479932in}}%
\pgfpathlineto{\pgfqpoint{1.588307in}{1.476761in}}%
\pgfpathlineto{\pgfqpoint{1.556442in}{1.471887in}}%
\pgfpathlineto{\pgfqpoint{1.524576in}{1.465332in}}%
\pgfpathlineto{\pgfqpoint{1.492710in}{1.457124in}}%
\pgfpathlineto{\pgfqpoint{1.460845in}{1.447298in}}%
\pgfpathlineto{\pgfqpoint{1.428979in}{1.435897in}}%
\pgfpathlineto{\pgfqpoint{1.397114in}{1.422970in}}%
\pgfpathlineto{\pgfqpoint{1.365248in}{1.408575in}}%
\pgfpathlineto{\pgfqpoint{1.333383in}{1.392774in}}%
\pgfpathlineto{\pgfqpoint{1.301517in}{1.375636in}}%
\pgfpathlineto{\pgfqpoint{1.269651in}{1.357235in}}%
\pgfpathlineto{\pgfqpoint{1.237786in}{1.337651in}}%
\pgfpathlineto{\pgfqpoint{1.205920in}{1.316975in}}%
\pgfpathlineto{\pgfqpoint{1.174055in}{1.295343in}}%
\pgfpathlineto{\pgfqpoint{1.142189in}{1.275978in}}%
\pgfpathlineto{\pgfqpoint{1.110323in}{1.269151in}}%
\pgfpathlineto{\pgfqpoint{1.078458in}{1.262985in}}%
\pgfpathlineto{\pgfqpoint{1.046592in}{1.257010in}}%
\pgfpathlineto{\pgfqpoint{1.014727in}{1.251202in}}%
\pgfpathlineto{\pgfqpoint{0.982861in}{1.245551in}}%
\pgfpathlineto{\pgfqpoint{0.950996in}{1.240053in}}%
\pgfpathlineto{\pgfqpoint{0.919130in}{1.234701in}}%
\pgfpathlineto{\pgfqpoint{0.887264in}{1.229488in}}%
\pgfpathlineto{\pgfqpoint{0.855399in}{1.224406in}}%
\pgfpathlineto{\pgfqpoint{0.823533in}{1.219448in}}%
\pgfpathlineto{\pgfqpoint{0.791668in}{1.214603in}}%
\pgfpathlineto{\pgfqpoint{0.759802in}{1.209865in}}%
\pgfpathlineto{\pgfqpoint{0.727937in}{1.205222in}}%
\pgfpathlineto{\pgfqpoint{0.696071in}{1.200666in}}%
\pgfpathlineto{\pgfqpoint{0.664205in}{1.196187in}}%
\pgfpathlineto{\pgfqpoint{0.632340in}{1.191778in}}%
\pgfpathlineto{\pgfqpoint{0.600474in}{1.187430in}}%
\pgfpathlineto{\pgfqpoint{0.568609in}{1.183136in}}%
\pgfpathlineto{\pgfqpoint{0.536743in}{1.178888in}}%
\pgfpathlineto{\pgfqpoint{0.504878in}{1.174680in}}%
\pgfpathlineto{\pgfqpoint{0.473012in}{1.170509in}}%
\pgfpathlineto{\pgfqpoint{0.441146in}{1.166368in}}%
\pgfpathlineto{\pgfqpoint{0.409281in}{1.162256in}}%
\pgfpathlineto{\pgfqpoint{0.377415in}{1.158170in}}%
\pgfpathlineto{\pgfqpoint{0.345550in}{1.154109in}}%
\pgfpathlineto{\pgfqpoint{0.313684in}{1.150072in}}%
\pgfpathlineto{\pgfqpoint{0.281818in}{1.146060in}}%
\pgfpathlineto{\pgfqpoint{0.249953in}{1.142075in}}%
\pgfpathlineto{\pgfqpoint{0.218087in}{1.138120in}}%
\pgfpathlineto{\pgfqpoint{0.186222in}{1.134197in}}%
\pgfpathlineto{\pgfqpoint{0.154356in}{1.130309in}}%
\pgfpathlineto{\pgfqpoint{0.122491in}{1.126463in}}%
\pgfpathlineto{\pgfqpoint{0.090625in}{1.122662in}}%
\pgfpathclose%
\pgfusepath{stroke,fill}%
\end{pgfscope}%
\begin{pgfscope}%
\pgfpathrectangle{\pgfqpoint{0.150000in}{0.150000in}}{\pgfqpoint{5.700000in}{2.200000in}}%
\pgfusepath{clip}%
\pgfsetbuttcap%
\pgfsetroundjoin%
\definecolor{currentfill}{rgb}{0.000000,0.501961,0.000000}%
\pgfsetfillcolor{currentfill}%
\pgfsetfillopacity{0.100000}%
\pgfsetlinewidth{1.003750pt}%
\definecolor{currentstroke}{rgb}{0.000000,0.501961,0.000000}%
\pgfsetstrokecolor{currentstroke}%
\pgfsetstrokeopacity{0.100000}%
\pgfsetdash{}{0pt}%
\pgfpathmoveto{\pgfqpoint{0.150000in}{0.150000in}}%
\pgfpathlineto{\pgfqpoint{0.150000in}{0.396729in}}%
\pgfpathlineto{\pgfqpoint{0.161423in}{0.396438in}}%
\pgfpathlineto{\pgfqpoint{0.172846in}{0.396125in}}%
\pgfpathlineto{\pgfqpoint{0.184269in}{0.395791in}}%
\pgfpathlineto{\pgfqpoint{0.195691in}{0.395433in}}%
\pgfpathlineto{\pgfqpoint{0.207114in}{0.395053in}}%
\pgfpathlineto{\pgfqpoint{0.218537in}{0.394651in}}%
\pgfpathlineto{\pgfqpoint{0.229960in}{0.394225in}}%
\pgfpathlineto{\pgfqpoint{0.241383in}{0.393775in}}%
\pgfpathlineto{\pgfqpoint{0.252806in}{0.393303in}}%
\pgfpathlineto{\pgfqpoint{0.264228in}{0.392806in}}%
\pgfpathlineto{\pgfqpoint{0.275651in}{0.392286in}}%
\pgfpathlineto{\pgfqpoint{0.287074in}{0.391742in}}%
\pgfpathlineto{\pgfqpoint{0.298497in}{0.391175in}}%
\pgfpathlineto{\pgfqpoint{0.309920in}{0.390583in}}%
\pgfpathlineto{\pgfqpoint{0.321343in}{0.389968in}}%
\pgfpathlineto{\pgfqpoint{0.332766in}{0.389329in}}%
\pgfpathlineto{\pgfqpoint{0.344188in}{0.388666in}}%
\pgfpathlineto{\pgfqpoint{0.355611in}{0.387979in}}%
\pgfpathlineto{\pgfqpoint{0.367034in}{0.387269in}}%
\pgfpathlineto{\pgfqpoint{0.378457in}{0.386536in}}%
\pgfpathlineto{\pgfqpoint{0.389880in}{0.385779in}}%
\pgfpathlineto{\pgfqpoint{0.401303in}{0.385000in}}%
\pgfpathlineto{\pgfqpoint{0.412725in}{0.384198in}}%
\pgfpathlineto{\pgfqpoint{0.424148in}{0.383373in}}%
\pgfpathlineto{\pgfqpoint{0.435571in}{0.382527in}}%
\pgfpathlineto{\pgfqpoint{0.446994in}{0.381658in}}%
\pgfpathlineto{\pgfqpoint{0.458417in}{0.380769in}}%
\pgfpathlineto{\pgfqpoint{0.469840in}{0.379858in}}%
\pgfpathlineto{\pgfqpoint{0.481263in}{0.378927in}}%
\pgfpathlineto{\pgfqpoint{0.492685in}{0.377976in}}%
\pgfpathlineto{\pgfqpoint{0.504108in}{0.377006in}}%
\pgfpathlineto{\pgfqpoint{0.515531in}{0.376017in}}%
\pgfpathlineto{\pgfqpoint{0.526954in}{0.375009in}}%
\pgfpathlineto{\pgfqpoint{0.538377in}{0.373984in}}%
\pgfpathlineto{\pgfqpoint{0.549800in}{0.372942in}}%
\pgfpathlineto{\pgfqpoint{0.561222in}{0.371884in}}%
\pgfpathlineto{\pgfqpoint{0.572645in}{0.370810in}}%
\pgfpathlineto{\pgfqpoint{0.584068in}{0.369721in}}%
\pgfpathlineto{\pgfqpoint{0.595491in}{0.368619in}}%
\pgfpathlineto{\pgfqpoint{0.606914in}{0.367503in}}%
\pgfpathlineto{\pgfqpoint{0.618337in}{0.366375in}}%
\pgfpathlineto{\pgfqpoint{0.629760in}{0.365235in}}%
\pgfpathlineto{\pgfqpoint{0.641182in}{0.364085in}}%
\pgfpathlineto{\pgfqpoint{0.652605in}{0.362925in}}%
\pgfpathlineto{\pgfqpoint{0.664028in}{0.361757in}}%
\pgfpathlineto{\pgfqpoint{0.675451in}{0.360581in}}%
\pgfpathlineto{\pgfqpoint{0.686874in}{0.359398in}}%
\pgfpathlineto{\pgfqpoint{0.698297in}{0.358210in}}%
\pgfpathlineto{\pgfqpoint{0.709719in}{0.357018in}}%
\pgfpathlineto{\pgfqpoint{0.721142in}{0.355822in}}%
\pgfpathlineto{\pgfqpoint{0.732565in}{0.354624in}}%
\pgfpathlineto{\pgfqpoint{0.743988in}{0.353425in}}%
\pgfpathlineto{\pgfqpoint{0.755411in}{0.352226in}}%
\pgfpathlineto{\pgfqpoint{0.766834in}{0.351029in}}%
\pgfpathlineto{\pgfqpoint{0.778257in}{0.349834in}}%
\pgfpathlineto{\pgfqpoint{0.789679in}{0.348643in}}%
\pgfpathlineto{\pgfqpoint{0.801102in}{0.347456in}}%
\pgfpathlineto{\pgfqpoint{0.812525in}{0.346276in}}%
\pgfpathlineto{\pgfqpoint{0.823948in}{0.345104in}}%
\pgfpathlineto{\pgfqpoint{0.835371in}{0.343940in}}%
\pgfpathlineto{\pgfqpoint{0.846794in}{0.342787in}}%
\pgfpathlineto{\pgfqpoint{0.858216in}{0.341645in}}%
\pgfpathlineto{\pgfqpoint{0.869639in}{0.340516in}}%
\pgfpathlineto{\pgfqpoint{0.881062in}{0.339401in}}%
\pgfpathlineto{\pgfqpoint{0.892485in}{0.338301in}}%
\pgfpathlineto{\pgfqpoint{0.903908in}{0.337218in}}%
\pgfpathlineto{\pgfqpoint{0.915331in}{0.336153in}}%
\pgfpathlineto{\pgfqpoint{0.926754in}{0.335108in}}%
\pgfpathlineto{\pgfqpoint{0.938176in}{0.334083in}}%
\pgfpathlineto{\pgfqpoint{0.949599in}{0.333080in}}%
\pgfpathlineto{\pgfqpoint{0.961022in}{0.332101in}}%
\pgfpathlineto{\pgfqpoint{0.972445in}{0.331147in}}%
\pgfpathlineto{\pgfqpoint{0.983868in}{0.330218in}}%
\pgfpathlineto{\pgfqpoint{0.995291in}{0.329317in}}%
\pgfpathlineto{\pgfqpoint{1.006713in}{0.328445in}}%
\pgfpathlineto{\pgfqpoint{1.018136in}{0.327603in}}%
\pgfpathlineto{\pgfqpoint{1.029559in}{0.326793in}}%
\pgfpathlineto{\pgfqpoint{1.040982in}{0.326016in}}%
\pgfpathlineto{\pgfqpoint{1.052405in}{0.325272in}}%
\pgfpathlineto{\pgfqpoint{1.063828in}{0.324565in}}%
\pgfpathlineto{\pgfqpoint{1.075251in}{0.323896in}}%
\pgfpathlineto{\pgfqpoint{1.086673in}{0.323267in}}%
\pgfpathlineto{\pgfqpoint{1.098096in}{0.322681in}}%
\pgfpathlineto{\pgfqpoint{1.109519in}{0.322145in}}%
\pgfpathlineto{\pgfqpoint{1.120942in}{0.321672in}}%
\pgfpathlineto{\pgfqpoint{1.132365in}{0.321311in}}%
\pgfpathlineto{\pgfqpoint{1.143788in}{0.321439in}}%
\pgfpathlineto{\pgfqpoint{1.155210in}{0.324772in}}%
\pgfpathlineto{\pgfqpoint{1.166633in}{0.330033in}}%
\pgfpathlineto{\pgfqpoint{1.178056in}{0.335406in}}%
\pgfpathlineto{\pgfqpoint{1.189479in}{0.340728in}}%
\pgfpathlineto{\pgfqpoint{1.200902in}{0.345967in}}%
\pgfpathlineto{\pgfqpoint{1.212325in}{0.351111in}}%
\pgfpathlineto{\pgfqpoint{1.223747in}{0.356154in}}%
\pgfpathlineto{\pgfqpoint{1.235170in}{0.361090in}}%
\pgfpathlineto{\pgfqpoint{1.246593in}{0.365916in}}%
\pgfpathlineto{\pgfqpoint{1.258016in}{0.370629in}}%
\pgfpathlineto{\pgfqpoint{1.269439in}{0.375226in}}%
\pgfpathlineto{\pgfqpoint{1.280862in}{0.379704in}}%
\pgfpathlineto{\pgfqpoint{1.292285in}{0.384059in}}%
\pgfpathlineto{\pgfqpoint{1.303707in}{0.388290in}}%
\pgfpathlineto{\pgfqpoint{1.315130in}{0.392393in}}%
\pgfpathlineto{\pgfqpoint{1.326553in}{0.396367in}}%
\pgfpathlineto{\pgfqpoint{1.337976in}{0.400209in}}%
\pgfpathlineto{\pgfqpoint{1.349399in}{0.403916in}}%
\pgfpathlineto{\pgfqpoint{1.360822in}{0.407486in}}%
\pgfpathlineto{\pgfqpoint{1.372244in}{0.410918in}}%
\pgfpathlineto{\pgfqpoint{1.383667in}{0.414209in}}%
\pgfpathlineto{\pgfqpoint{1.395090in}{0.417358in}}%
\pgfpathlineto{\pgfqpoint{1.406513in}{0.420361in}}%
\pgfpathlineto{\pgfqpoint{1.417936in}{0.423218in}}%
\pgfpathlineto{\pgfqpoint{1.429359in}{0.425927in}}%
\pgfpathlineto{\pgfqpoint{1.440782in}{0.428487in}}%
\pgfpathlineto{\pgfqpoint{1.452204in}{0.430895in}}%
\pgfpathlineto{\pgfqpoint{1.463627in}{0.433150in}}%
\pgfpathlineto{\pgfqpoint{1.475050in}{0.435252in}}%
\pgfpathlineto{\pgfqpoint{1.486473in}{0.437198in}}%
\pgfpathlineto{\pgfqpoint{1.497896in}{0.438988in}}%
\pgfpathlineto{\pgfqpoint{1.509319in}{0.440621in}}%
\pgfpathlineto{\pgfqpoint{1.520741in}{0.442095in}}%
\pgfpathlineto{\pgfqpoint{1.532164in}{0.443410in}}%
\pgfpathlineto{\pgfqpoint{1.543587in}{0.444565in}}%
\pgfpathlineto{\pgfqpoint{1.555010in}{0.445560in}}%
\pgfpathlineto{\pgfqpoint{1.566433in}{0.446394in}}%
\pgfpathlineto{\pgfqpoint{1.577856in}{0.447066in}}%
\pgfpathlineto{\pgfqpoint{1.589279in}{0.447577in}}%
\pgfpathlineto{\pgfqpoint{1.600701in}{0.447925in}}%
\pgfpathlineto{\pgfqpoint{1.612124in}{0.448111in}}%
\pgfpathlineto{\pgfqpoint{1.623547in}{0.448136in}}%
\pgfpathlineto{\pgfqpoint{1.634970in}{0.447998in}}%
\pgfpathlineto{\pgfqpoint{1.646393in}{0.447698in}}%
\pgfpathlineto{\pgfqpoint{1.657816in}{0.447237in}}%
\pgfpathlineto{\pgfqpoint{1.669238in}{0.446615in}}%
\pgfpathlineto{\pgfqpoint{1.680661in}{0.445832in}}%
\pgfpathlineto{\pgfqpoint{1.692084in}{0.444889in}}%
\pgfpathlineto{\pgfqpoint{1.703507in}{0.443787in}}%
\pgfpathlineto{\pgfqpoint{1.714930in}{0.442526in}}%
\pgfpathlineto{\pgfqpoint{1.726353in}{0.441107in}}%
\pgfpathlineto{\pgfqpoint{1.737776in}{0.439532in}}%
\pgfpathlineto{\pgfqpoint{1.749198in}{0.437802in}}%
\pgfpathlineto{\pgfqpoint{1.760621in}{0.435917in}}%
\pgfpathlineto{\pgfqpoint{1.772044in}{0.433880in}}%
\pgfpathlineto{\pgfqpoint{1.783467in}{0.431691in}}%
\pgfpathlineto{\pgfqpoint{1.794890in}{0.429352in}}%
\pgfpathlineto{\pgfqpoint{1.806313in}{0.426865in}}%
\pgfpathlineto{\pgfqpoint{1.817735in}{0.424233in}}%
\pgfpathlineto{\pgfqpoint{1.829158in}{0.421456in}}%
\pgfpathlineto{\pgfqpoint{1.840581in}{0.418538in}}%
\pgfpathlineto{\pgfqpoint{1.852004in}{0.415482in}}%
\pgfpathlineto{\pgfqpoint{1.863427in}{0.412293in}}%
\pgfpathlineto{\pgfqpoint{1.874850in}{0.408979in}}%
\pgfpathlineto{\pgfqpoint{1.886273in}{0.405561in}}%
\pgfpathlineto{\pgfqpoint{1.897695in}{0.402125in}}%
\pgfpathlineto{\pgfqpoint{1.909118in}{0.399559in}}%
\pgfpathlineto{\pgfqpoint{1.920541in}{0.400631in}}%
\pgfpathlineto{\pgfqpoint{1.931964in}{0.402706in}}%
\pgfpathlineto{\pgfqpoint{1.943387in}{0.404897in}}%
\pgfpathlineto{\pgfqpoint{1.954810in}{0.407117in}}%
\pgfpathlineto{\pgfqpoint{1.966232in}{0.409348in}}%
\pgfpathlineto{\pgfqpoint{1.977655in}{0.411579in}}%
\pgfpathlineto{\pgfqpoint{1.989078in}{0.413808in}}%
\pgfpathlineto{\pgfqpoint{2.000501in}{0.416030in}}%
\pgfpathlineto{\pgfqpoint{2.011924in}{0.418244in}}%
\pgfpathlineto{\pgfqpoint{2.023347in}{0.420447in}}%
\pgfpathlineto{\pgfqpoint{2.034770in}{0.422637in}}%
\pgfpathlineto{\pgfqpoint{2.046192in}{0.424813in}}%
\pgfpathlineto{\pgfqpoint{2.057615in}{0.426972in}}%
\pgfpathlineto{\pgfqpoint{2.069038in}{0.429114in}}%
\pgfpathlineto{\pgfqpoint{2.080461in}{0.431236in}}%
\pgfpathlineto{\pgfqpoint{2.091884in}{0.433337in}}%
\pgfpathlineto{\pgfqpoint{2.103307in}{0.435415in}}%
\pgfpathlineto{\pgfqpoint{2.114729in}{0.437469in}}%
\pgfpathlineto{\pgfqpoint{2.126152in}{0.439497in}}%
\pgfpathlineto{\pgfqpoint{2.137575in}{0.441498in}}%
\pgfpathlineto{\pgfqpoint{2.148998in}{0.443471in}}%
\pgfpathlineto{\pgfqpoint{2.160421in}{0.445414in}}%
\pgfpathlineto{\pgfqpoint{2.171844in}{0.447326in}}%
\pgfpathlineto{\pgfqpoint{2.183267in}{0.449205in}}%
\pgfpathlineto{\pgfqpoint{2.194689in}{0.451050in}}%
\pgfpathlineto{\pgfqpoint{2.206112in}{0.452861in}}%
\pgfpathlineto{\pgfqpoint{2.217535in}{0.454636in}}%
\pgfpathlineto{\pgfqpoint{2.228958in}{0.456373in}}%
\pgfpathlineto{\pgfqpoint{2.240381in}{0.458072in}}%
\pgfpathlineto{\pgfqpoint{2.251804in}{0.459732in}}%
\pgfpathlineto{\pgfqpoint{2.263226in}{0.461352in}}%
\pgfpathlineto{\pgfqpoint{2.274649in}{0.462930in}}%
\pgfpathlineto{\pgfqpoint{2.286072in}{0.464467in}}%
\pgfpathlineto{\pgfqpoint{2.297495in}{0.465960in}}%
\pgfpathlineto{\pgfqpoint{2.308918in}{0.467409in}}%
\pgfpathlineto{\pgfqpoint{2.320341in}{0.468814in}}%
\pgfpathlineto{\pgfqpoint{2.331764in}{0.470174in}}%
\pgfpathlineto{\pgfqpoint{2.343186in}{0.471488in}}%
\pgfpathlineto{\pgfqpoint{2.354609in}{0.472755in}}%
\pgfpathlineto{\pgfqpoint{2.366032in}{0.473976in}}%
\pgfpathlineto{\pgfqpoint{2.377455in}{0.475148in}}%
\pgfpathlineto{\pgfqpoint{2.388878in}{0.476273in}}%
\pgfpathlineto{\pgfqpoint{2.400301in}{0.477349in}}%
\pgfpathlineto{\pgfqpoint{2.411723in}{0.478377in}}%
\pgfpathlineto{\pgfqpoint{2.423146in}{0.479355in}}%
\pgfpathlineto{\pgfqpoint{2.434569in}{0.480284in}}%
\pgfpathlineto{\pgfqpoint{2.445992in}{0.481163in}}%
\pgfpathlineto{\pgfqpoint{2.457415in}{0.481992in}}%
\pgfpathlineto{\pgfqpoint{2.468838in}{0.482772in}}%
\pgfpathlineto{\pgfqpoint{2.480261in}{0.483501in}}%
\pgfpathlineto{\pgfqpoint{2.491683in}{0.484181in}}%
\pgfpathlineto{\pgfqpoint{2.503106in}{0.484810in}}%
\pgfpathlineto{\pgfqpoint{2.514529in}{0.485390in}}%
\pgfpathlineto{\pgfqpoint{2.525952in}{0.485919in}}%
\pgfpathlineto{\pgfqpoint{2.537375in}{0.486400in}}%
\pgfpathlineto{\pgfqpoint{2.548798in}{0.486830in}}%
\pgfpathlineto{\pgfqpoint{2.560220in}{0.487212in}}%
\pgfpathlineto{\pgfqpoint{2.571643in}{0.487545in}}%
\pgfpathlineto{\pgfqpoint{2.583066in}{0.487829in}}%
\pgfpathlineto{\pgfqpoint{2.594489in}{0.488065in}}%
\pgfpathlineto{\pgfqpoint{2.605912in}{0.488253in}}%
\pgfpathlineto{\pgfqpoint{2.617335in}{0.488394in}}%
\pgfpathlineto{\pgfqpoint{2.628758in}{0.488489in}}%
\pgfpathlineto{\pgfqpoint{2.640180in}{0.488537in}}%
\pgfpathlineto{\pgfqpoint{2.651603in}{0.488539in}}%
\pgfpathlineto{\pgfqpoint{2.663026in}{0.488497in}}%
\pgfpathlineto{\pgfqpoint{2.674449in}{0.488410in}}%
\pgfpathlineto{\pgfqpoint{2.685872in}{0.488280in}}%
\pgfpathlineto{\pgfqpoint{2.697295in}{0.488106in}}%
\pgfpathlineto{\pgfqpoint{2.708717in}{0.487891in}}%
\pgfpathlineto{\pgfqpoint{2.720140in}{0.487634in}}%
\pgfpathlineto{\pgfqpoint{2.731563in}{0.487336in}}%
\pgfpathlineto{\pgfqpoint{2.742986in}{0.486998in}}%
\pgfpathlineto{\pgfqpoint{2.754409in}{0.486622in}}%
\pgfpathlineto{\pgfqpoint{2.765832in}{0.486208in}}%
\pgfpathlineto{\pgfqpoint{2.777255in}{0.485756in}}%
\pgfpathlineto{\pgfqpoint{2.788677in}{0.485269in}}%
\pgfpathlineto{\pgfqpoint{2.800100in}{0.484746in}}%
\pgfpathlineto{\pgfqpoint{2.811523in}{0.484189in}}%
\pgfpathlineto{\pgfqpoint{2.822946in}{0.483598in}}%
\pgfpathlineto{\pgfqpoint{2.834369in}{0.482976in}}%
\pgfpathlineto{\pgfqpoint{2.845792in}{0.482322in}}%
\pgfpathlineto{\pgfqpoint{2.857214in}{0.481638in}}%
\pgfpathlineto{\pgfqpoint{2.868637in}{0.480925in}}%
\pgfpathlineto{\pgfqpoint{2.880060in}{0.480184in}}%
\pgfpathlineto{\pgfqpoint{2.891483in}{0.479416in}}%
\pgfpathlineto{\pgfqpoint{2.902906in}{0.478622in}}%
\pgfpathlineto{\pgfqpoint{2.914329in}{0.477804in}}%
\pgfpathlineto{\pgfqpoint{2.925752in}{0.476961in}}%
\pgfpathlineto{\pgfqpoint{2.937174in}{0.476096in}}%
\pgfpathlineto{\pgfqpoint{2.948597in}{0.475210in}}%
\pgfpathlineto{\pgfqpoint{2.960020in}{0.474303in}}%
\pgfpathlineto{\pgfqpoint{2.971443in}{0.473378in}}%
\pgfpathlineto{\pgfqpoint{2.982866in}{0.472434in}}%
\pgfpathlineto{\pgfqpoint{2.994289in}{0.471473in}}%
\pgfpathlineto{\pgfqpoint{3.005711in}{0.470496in}}%
\pgfpathlineto{\pgfqpoint{3.017134in}{0.469504in}}%
\pgfpathlineto{\pgfqpoint{3.028557in}{0.468499in}}%
\pgfpathlineto{\pgfqpoint{3.039980in}{0.467481in}}%
\pgfpathlineto{\pgfqpoint{3.051403in}{0.466452in}}%
\pgfpathlineto{\pgfqpoint{3.062826in}{0.465412in}}%
\pgfpathlineto{\pgfqpoint{3.074248in}{0.464363in}}%
\pgfpathlineto{\pgfqpoint{3.085671in}{0.463306in}}%
\pgfpathlineto{\pgfqpoint{3.097094in}{0.462241in}}%
\pgfpathlineto{\pgfqpoint{3.108517in}{0.461170in}}%
\pgfpathlineto{\pgfqpoint{3.119940in}{0.460095in}}%
\pgfpathlineto{\pgfqpoint{3.131363in}{0.459015in}}%
\pgfpathlineto{\pgfqpoint{3.142786in}{0.457932in}}%
\pgfpathlineto{\pgfqpoint{3.154208in}{0.456846in}}%
\pgfpathlineto{\pgfqpoint{3.165631in}{0.455760in}}%
\pgfpathlineto{\pgfqpoint{3.177054in}{0.454673in}}%
\pgfpathlineto{\pgfqpoint{3.188477in}{0.453587in}}%
\pgfpathlineto{\pgfqpoint{3.199900in}{0.452503in}}%
\pgfpathlineto{\pgfqpoint{3.211323in}{0.451421in}}%
\pgfpathlineto{\pgfqpoint{3.222745in}{0.450342in}}%
\pgfpathlineto{\pgfqpoint{3.234168in}{0.449268in}}%
\pgfpathlineto{\pgfqpoint{3.245591in}{0.448198in}}%
\pgfpathlineto{\pgfqpoint{3.257014in}{0.447135in}}%
\pgfpathlineto{\pgfqpoint{3.268437in}{0.446078in}}%
\pgfpathlineto{\pgfqpoint{3.279860in}{0.445029in}}%
\pgfpathlineto{\pgfqpoint{3.291283in}{0.443987in}}%
\pgfpathlineto{\pgfqpoint{3.302705in}{0.442955in}}%
\pgfpathlineto{\pgfqpoint{3.314128in}{0.441932in}}%
\pgfpathlineto{\pgfqpoint{3.325551in}{0.440919in}}%
\pgfpathlineto{\pgfqpoint{3.336974in}{0.439917in}}%
\pgfpathlineto{\pgfqpoint{3.348397in}{0.438926in}}%
\pgfpathlineto{\pgfqpoint{3.359820in}{0.437947in}}%
\pgfpathlineto{\pgfqpoint{3.371242in}{0.436981in}}%
\pgfpathlineto{\pgfqpoint{3.382665in}{0.436028in}}%
\pgfpathlineto{\pgfqpoint{3.394088in}{0.435089in}}%
\pgfpathlineto{\pgfqpoint{3.405511in}{0.434164in}}%
\pgfpathlineto{\pgfqpoint{3.416934in}{0.433253in}}%
\pgfpathlineto{\pgfqpoint{3.428357in}{0.432357in}}%
\pgfpathlineto{\pgfqpoint{3.439780in}{0.431476in}}%
\pgfpathlineto{\pgfqpoint{3.451202in}{0.430612in}}%
\pgfpathlineto{\pgfqpoint{3.462625in}{0.429763in}}%
\pgfpathlineto{\pgfqpoint{3.474048in}{0.428930in}}%
\pgfpathlineto{\pgfqpoint{3.485471in}{0.428115in}}%
\pgfpathlineto{\pgfqpoint{3.496894in}{0.427316in}}%
\pgfpathlineto{\pgfqpoint{3.508317in}{0.426534in}}%
\pgfpathlineto{\pgfqpoint{3.519739in}{0.425770in}}%
\pgfpathlineto{\pgfqpoint{3.531162in}{0.425023in}}%
\pgfpathlineto{\pgfqpoint{3.542585in}{0.424294in}}%
\pgfpathlineto{\pgfqpoint{3.554008in}{0.423583in}}%
\pgfpathlineto{\pgfqpoint{3.565431in}{0.422889in}}%
\pgfpathlineto{\pgfqpoint{3.576854in}{0.422214in}}%
\pgfpathlineto{\pgfqpoint{3.588277in}{0.421556in}}%
\pgfpathlineto{\pgfqpoint{3.599699in}{0.420916in}}%
\pgfpathlineto{\pgfqpoint{3.611122in}{0.420295in}}%
\pgfpathlineto{\pgfqpoint{3.622545in}{0.419691in}}%
\pgfpathlineto{\pgfqpoint{3.633968in}{0.419105in}}%
\pgfpathlineto{\pgfqpoint{3.645391in}{0.418537in}}%
\pgfpathlineto{\pgfqpoint{3.656814in}{0.417986in}}%
\pgfpathlineto{\pgfqpoint{3.668236in}{0.417453in}}%
\pgfpathlineto{\pgfqpoint{3.679659in}{0.416938in}}%
\pgfpathlineto{\pgfqpoint{3.691082in}{0.416439in}}%
\pgfpathlineto{\pgfqpoint{3.702505in}{0.415957in}}%
\pgfpathlineto{\pgfqpoint{3.713928in}{0.415492in}}%
\pgfpathlineto{\pgfqpoint{3.725351in}{0.415043in}}%
\pgfpathlineto{\pgfqpoint{3.736774in}{0.414610in}}%
\pgfpathlineto{\pgfqpoint{3.748196in}{0.414193in}}%
\pgfpathlineto{\pgfqpoint{3.759619in}{0.413792in}}%
\pgfpathlineto{\pgfqpoint{3.771042in}{0.413405in}}%
\pgfpathlineto{\pgfqpoint{3.782465in}{0.413032in}}%
\pgfpathlineto{\pgfqpoint{3.793888in}{0.412674in}}%
\pgfpathlineto{\pgfqpoint{3.805311in}{0.412330in}}%
\pgfpathlineto{\pgfqpoint{3.816733in}{0.411998in}}%
\pgfpathlineto{\pgfqpoint{3.828156in}{0.411680in}}%
\pgfpathlineto{\pgfqpoint{3.839579in}{0.411373in}}%
\pgfpathlineto{\pgfqpoint{3.851002in}{0.411078in}}%
\pgfpathlineto{\pgfqpoint{3.862425in}{0.410795in}}%
\pgfpathlineto{\pgfqpoint{3.873848in}{0.410521in}}%
\pgfpathlineto{\pgfqpoint{3.885271in}{0.410258in}}%
\pgfpathlineto{\pgfqpoint{3.896693in}{0.410004in}}%
\pgfpathlineto{\pgfqpoint{3.908116in}{0.409758in}}%
\pgfpathlineto{\pgfqpoint{3.919539in}{0.409520in}}%
\pgfpathlineto{\pgfqpoint{3.930962in}{0.409289in}}%
\pgfpathlineto{\pgfqpoint{3.942385in}{0.409065in}}%
\pgfpathlineto{\pgfqpoint{3.953808in}{0.408847in}}%
\pgfpathlineto{\pgfqpoint{3.965230in}{0.408634in}}%
\pgfpathlineto{\pgfqpoint{3.976653in}{0.408424in}}%
\pgfpathlineto{\pgfqpoint{3.988076in}{0.408219in}}%
\pgfpathlineto{\pgfqpoint{3.999499in}{0.408015in}}%
\pgfpathlineto{\pgfqpoint{4.010922in}{0.407814in}}%
\pgfpathlineto{\pgfqpoint{4.022345in}{0.407614in}}%
\pgfpathlineto{\pgfqpoint{4.033768in}{0.407413in}}%
\pgfpathlineto{\pgfqpoint{4.045190in}{0.407212in}}%
\pgfpathlineto{\pgfqpoint{4.056613in}{0.407009in}}%
\pgfpathlineto{\pgfqpoint{4.068036in}{0.406804in}}%
\pgfpathlineto{\pgfqpoint{4.079459in}{0.406595in}}%
\pgfpathlineto{\pgfqpoint{4.090882in}{0.406381in}}%
\pgfpathlineto{\pgfqpoint{4.102305in}{0.406162in}}%
\pgfpathlineto{\pgfqpoint{4.113727in}{0.405937in}}%
\pgfpathlineto{\pgfqpoint{4.125150in}{0.405704in}}%
\pgfpathlineto{\pgfqpoint{4.136573in}{0.405463in}}%
\pgfpathlineto{\pgfqpoint{4.147996in}{0.405212in}}%
\pgfpathlineto{\pgfqpoint{4.159419in}{0.404952in}}%
\pgfpathlineto{\pgfqpoint{4.170842in}{0.404679in}}%
\pgfpathlineto{\pgfqpoint{4.182265in}{0.404394in}}%
\pgfpathlineto{\pgfqpoint{4.193687in}{0.404096in}}%
\pgfpathlineto{\pgfqpoint{4.205110in}{0.403783in}}%
\pgfpathlineto{\pgfqpoint{4.216533in}{0.403455in}}%
\pgfpathlineto{\pgfqpoint{4.227956in}{0.403110in}}%
\pgfpathlineto{\pgfqpoint{4.239379in}{0.402747in}}%
\pgfpathlineto{\pgfqpoint{4.250802in}{0.402365in}}%
\pgfpathlineto{\pgfqpoint{4.262224in}{0.401963in}}%
\pgfpathlineto{\pgfqpoint{4.273647in}{0.401540in}}%
\pgfpathlineto{\pgfqpoint{4.285070in}{0.401096in}}%
\pgfpathlineto{\pgfqpoint{4.296493in}{0.400628in}}%
\pgfpathlineto{\pgfqpoint{4.307916in}{0.400135in}}%
\pgfpathlineto{\pgfqpoint{4.319339in}{0.399618in}}%
\pgfpathlineto{\pgfqpoint{4.330762in}{0.399074in}}%
\pgfpathlineto{\pgfqpoint{4.342184in}{0.398502in}}%
\pgfpathlineto{\pgfqpoint{4.353607in}{0.397902in}}%
\pgfpathlineto{\pgfqpoint{4.365030in}{0.397272in}}%
\pgfpathlineto{\pgfqpoint{4.376453in}{0.396611in}}%
\pgfpathlineto{\pgfqpoint{4.387876in}{0.395918in}}%
\pgfpathlineto{\pgfqpoint{4.399299in}{0.395192in}}%
\pgfpathlineto{\pgfqpoint{4.410721in}{0.394433in}}%
\pgfpathlineto{\pgfqpoint{4.422144in}{0.393638in}}%
\pgfpathlineto{\pgfqpoint{4.433567in}{0.392807in}}%
\pgfpathlineto{\pgfqpoint{4.444990in}{0.391939in}}%
\pgfpathlineto{\pgfqpoint{4.456413in}{0.391033in}}%
\pgfpathlineto{\pgfqpoint{4.467836in}{0.390087in}}%
\pgfpathlineto{\pgfqpoint{4.479259in}{0.389101in}}%
\pgfpathlineto{\pgfqpoint{4.490681in}{0.388074in}}%
\pgfpathlineto{\pgfqpoint{4.502104in}{0.387005in}}%
\pgfpathlineto{\pgfqpoint{4.513527in}{0.385893in}}%
\pgfpathlineto{\pgfqpoint{4.524950in}{0.384737in}}%
\pgfpathlineto{\pgfqpoint{4.536373in}{0.383536in}}%
\pgfpathlineto{\pgfqpoint{4.547796in}{0.382288in}}%
\pgfpathlineto{\pgfqpoint{4.559218in}{0.380994in}}%
\pgfpathlineto{\pgfqpoint{4.570641in}{0.379653in}}%
\pgfpathlineto{\pgfqpoint{4.582064in}{0.378262in}}%
\pgfpathlineto{\pgfqpoint{4.593487in}{0.376823in}}%
\pgfpathlineto{\pgfqpoint{4.604910in}{0.375333in}}%
\pgfpathlineto{\pgfqpoint{4.616333in}{0.373793in}}%
\pgfpathlineto{\pgfqpoint{4.627756in}{0.372200in}}%
\pgfpathlineto{\pgfqpoint{4.639178in}{0.370556in}}%
\pgfpathlineto{\pgfqpoint{4.650601in}{0.368858in}}%
\pgfpathlineto{\pgfqpoint{4.662024in}{0.367107in}}%
\pgfpathlineto{\pgfqpoint{4.673447in}{0.365301in}}%
\pgfpathlineto{\pgfqpoint{4.684870in}{0.363440in}}%
\pgfpathlineto{\pgfqpoint{4.696293in}{0.361524in}}%
\pgfpathlineto{\pgfqpoint{4.707715in}{0.359551in}}%
\pgfpathlineto{\pgfqpoint{4.719138in}{0.357522in}}%
\pgfpathlineto{\pgfqpoint{4.730561in}{0.355436in}}%
\pgfpathlineto{\pgfqpoint{4.741984in}{0.353292in}}%
\pgfpathlineto{\pgfqpoint{4.753407in}{0.351090in}}%
\pgfpathlineto{\pgfqpoint{4.764830in}{0.348830in}}%
\pgfpathlineto{\pgfqpoint{4.776253in}{0.346511in}}%
\pgfpathlineto{\pgfqpoint{4.787675in}{0.344133in}}%
\pgfpathlineto{\pgfqpoint{4.799098in}{0.341695in}}%
\pgfpathlineto{\pgfqpoint{4.810521in}{0.339198in}}%
\pgfpathlineto{\pgfqpoint{4.821944in}{0.336642in}}%
\pgfpathlineto{\pgfqpoint{4.833367in}{0.334025in}}%
\pgfpathlineto{\pgfqpoint{4.844790in}{0.331348in}}%
\pgfpathlineto{\pgfqpoint{4.856212in}{0.328612in}}%
\pgfpathlineto{\pgfqpoint{4.867635in}{0.325815in}}%
\pgfpathlineto{\pgfqpoint{4.879058in}{0.322957in}}%
\pgfpathlineto{\pgfqpoint{4.890481in}{0.320040in}}%
\pgfpathlineto{\pgfqpoint{4.901904in}{0.317062in}}%
\pgfpathlineto{\pgfqpoint{4.913327in}{0.314025in}}%
\pgfpathlineto{\pgfqpoint{4.924749in}{0.310928in}}%
\pgfpathlineto{\pgfqpoint{4.936172in}{0.307771in}}%
\pgfpathlineto{\pgfqpoint{4.947595in}{0.304554in}}%
\pgfpathlineto{\pgfqpoint{4.959018in}{0.301279in}}%
\pgfpathlineto{\pgfqpoint{4.970441in}{0.297945in}}%
\pgfpathlineto{\pgfqpoint{4.981864in}{0.294552in}}%
\pgfpathlineto{\pgfqpoint{4.993287in}{0.291101in}}%
\pgfpathlineto{\pgfqpoint{5.004709in}{0.287593in}}%
\pgfpathlineto{\pgfqpoint{5.016132in}{0.284027in}}%
\pgfpathlineto{\pgfqpoint{5.027555in}{0.280405in}}%
\pgfpathlineto{\pgfqpoint{5.038978in}{0.276726in}}%
\pgfpathlineto{\pgfqpoint{5.050401in}{0.272993in}}%
\pgfpathlineto{\pgfqpoint{5.061824in}{0.269204in}}%
\pgfpathlineto{\pgfqpoint{5.073246in}{0.265361in}}%
\pgfpathlineto{\pgfqpoint{5.084669in}{0.261465in}}%
\pgfpathlineto{\pgfqpoint{5.096092in}{0.257516in}}%
\pgfpathlineto{\pgfqpoint{5.107515in}{0.253515in}}%
\pgfpathlineto{\pgfqpoint{5.118938in}{0.249464in}}%
\pgfpathlineto{\pgfqpoint{5.130361in}{0.245362in}}%
\pgfpathlineto{\pgfqpoint{5.141784in}{0.241210in}}%
\pgfpathlineto{\pgfqpoint{5.153206in}{0.237011in}}%
\pgfpathlineto{\pgfqpoint{5.164629in}{0.232764in}}%
\pgfpathlineto{\pgfqpoint{5.176052in}{0.228470in}}%
\pgfpathlineto{\pgfqpoint{5.187475in}{0.224132in}}%
\pgfpathlineto{\pgfqpoint{5.198898in}{0.219749in}}%
\pgfpathlineto{\pgfqpoint{5.210321in}{0.215323in}}%
\pgfpathlineto{\pgfqpoint{5.221743in}{0.210855in}}%
\pgfpathlineto{\pgfqpoint{5.233166in}{0.206346in}}%
\pgfpathlineto{\pgfqpoint{5.244589in}{0.201798in}}%
\pgfpathlineto{\pgfqpoint{5.256012in}{0.197212in}}%
\pgfpathlineto{\pgfqpoint{5.267435in}{0.192589in}}%
\pgfpathlineto{\pgfqpoint{5.278858in}{0.187931in}}%
\pgfpathlineto{\pgfqpoint{5.290281in}{0.183239in}}%
\pgfpathlineto{\pgfqpoint{5.301703in}{0.178514in}}%
\pgfpathlineto{\pgfqpoint{5.313126in}{0.173760in}}%
\pgfpathlineto{\pgfqpoint{5.324549in}{0.168978in}}%
\pgfpathlineto{\pgfqpoint{5.335972in}{0.164173in}}%
\pgfpathlineto{\pgfqpoint{5.347395in}{0.159353in}}%
\pgfpathlineto{\pgfqpoint{5.358818in}{0.154547in}}%
\pgfpathlineto{\pgfqpoint{5.370240in}{0.150000in}}%
\pgfpathlineto{\pgfqpoint{5.381663in}{0.150687in}}%
\pgfpathlineto{\pgfqpoint{5.393086in}{0.155338in}}%
\pgfpathlineto{\pgfqpoint{5.404509in}{0.160147in}}%
\pgfpathlineto{\pgfqpoint{5.415932in}{0.164961in}}%
\pgfpathlineto{\pgfqpoint{5.427355in}{0.169759in}}%
\pgfpathlineto{\pgfqpoint{5.438778in}{0.174533in}}%
\pgfpathlineto{\pgfqpoint{5.450200in}{0.179279in}}%
\pgfpathlineto{\pgfqpoint{5.461623in}{0.183994in}}%
\pgfpathlineto{\pgfqpoint{5.473046in}{0.188676in}}%
\pgfpathlineto{\pgfqpoint{5.484469in}{0.193325in}}%
\pgfpathlineto{\pgfqpoint{5.495892in}{0.197937in}}%
\pgfpathlineto{\pgfqpoint{5.507315in}{0.202513in}}%
\pgfpathlineto{\pgfqpoint{5.518737in}{0.207050in}}%
\pgfpathlineto{\pgfqpoint{5.530160in}{0.211547in}}%
\pgfpathlineto{\pgfqpoint{5.541583in}{0.216003in}}%
\pgfpathlineto{\pgfqpoint{5.553006in}{0.220416in}}%
\pgfpathlineto{\pgfqpoint{5.564429in}{0.224787in}}%
\pgfpathlineto{\pgfqpoint{5.575852in}{0.229112in}}%
\pgfpathlineto{\pgfqpoint{5.587275in}{0.233392in}}%
\pgfpathlineto{\pgfqpoint{5.598697in}{0.237625in}}%
\pgfpathlineto{\pgfqpoint{5.610120in}{0.241811in}}%
\pgfpathlineto{\pgfqpoint{5.621543in}{0.245948in}}%
\pgfpathlineto{\pgfqpoint{5.632966in}{0.250034in}}%
\pgfpathlineto{\pgfqpoint{5.644389in}{0.254071in}}%
\pgfpathlineto{\pgfqpoint{5.655812in}{0.258055in}}%
\pgfpathlineto{\pgfqpoint{5.667234in}{0.261988in}}%
\pgfpathlineto{\pgfqpoint{5.678657in}{0.265867in}}%
\pgfpathlineto{\pgfqpoint{5.690080in}{0.269692in}}%
\pgfpathlineto{\pgfqpoint{5.701503in}{0.273463in}}%
\pgfpathlineto{\pgfqpoint{5.712926in}{0.277178in}}%
\pgfpathlineto{\pgfqpoint{5.724349in}{0.280837in}}%
\pgfpathlineto{\pgfqpoint{5.735772in}{0.284439in}}%
\pgfpathlineto{\pgfqpoint{5.747194in}{0.287985in}}%
\pgfpathlineto{\pgfqpoint{5.758617in}{0.291472in}}%
\pgfpathlineto{\pgfqpoint{5.770040in}{0.294901in}}%
\pgfpathlineto{\pgfqpoint{5.781463in}{0.298272in}}%
\pgfpathlineto{\pgfqpoint{5.792886in}{0.301583in}}%
\pgfpathlineto{\pgfqpoint{5.804309in}{0.304835in}}%
\pgfpathlineto{\pgfqpoint{5.815731in}{0.308027in}}%
\pgfpathlineto{\pgfqpoint{5.827154in}{0.311159in}}%
\pgfpathlineto{\pgfqpoint{5.838577in}{0.314230in}}%
\pgfpathlineto{\pgfqpoint{5.850000in}{0.317241in}}%
\pgfpathlineto{\pgfqpoint{5.850000in}{0.150000in}}%
\pgfpathlineto{\pgfqpoint{5.850000in}{0.150000in}}%
\pgfpathlineto{\pgfqpoint{5.838577in}{0.150000in}}%
\pgfpathlineto{\pgfqpoint{5.827154in}{0.150000in}}%
\pgfpathlineto{\pgfqpoint{5.815731in}{0.150000in}}%
\pgfpathlineto{\pgfqpoint{5.804309in}{0.150000in}}%
\pgfpathlineto{\pgfqpoint{5.792886in}{0.150000in}}%
\pgfpathlineto{\pgfqpoint{5.781463in}{0.150000in}}%
\pgfpathlineto{\pgfqpoint{5.770040in}{0.150000in}}%
\pgfpathlineto{\pgfqpoint{5.758617in}{0.150000in}}%
\pgfpathlineto{\pgfqpoint{5.747194in}{0.150000in}}%
\pgfpathlineto{\pgfqpoint{5.735772in}{0.150000in}}%
\pgfpathlineto{\pgfqpoint{5.724349in}{0.150000in}}%
\pgfpathlineto{\pgfqpoint{5.712926in}{0.150000in}}%
\pgfpathlineto{\pgfqpoint{5.701503in}{0.150000in}}%
\pgfpathlineto{\pgfqpoint{5.690080in}{0.150000in}}%
\pgfpathlineto{\pgfqpoint{5.678657in}{0.150000in}}%
\pgfpathlineto{\pgfqpoint{5.667234in}{0.150000in}}%
\pgfpathlineto{\pgfqpoint{5.655812in}{0.150000in}}%
\pgfpathlineto{\pgfqpoint{5.644389in}{0.150000in}}%
\pgfpathlineto{\pgfqpoint{5.632966in}{0.150000in}}%
\pgfpathlineto{\pgfqpoint{5.621543in}{0.150000in}}%
\pgfpathlineto{\pgfqpoint{5.610120in}{0.150000in}}%
\pgfpathlineto{\pgfqpoint{5.598697in}{0.150000in}}%
\pgfpathlineto{\pgfqpoint{5.587275in}{0.150000in}}%
\pgfpathlineto{\pgfqpoint{5.575852in}{0.150000in}}%
\pgfpathlineto{\pgfqpoint{5.564429in}{0.150000in}}%
\pgfpathlineto{\pgfqpoint{5.553006in}{0.150000in}}%
\pgfpathlineto{\pgfqpoint{5.541583in}{0.150000in}}%
\pgfpathlineto{\pgfqpoint{5.530160in}{0.150000in}}%
\pgfpathlineto{\pgfqpoint{5.518737in}{0.150000in}}%
\pgfpathlineto{\pgfqpoint{5.507315in}{0.150000in}}%
\pgfpathlineto{\pgfqpoint{5.495892in}{0.150000in}}%
\pgfpathlineto{\pgfqpoint{5.484469in}{0.150000in}}%
\pgfpathlineto{\pgfqpoint{5.473046in}{0.150000in}}%
\pgfpathlineto{\pgfqpoint{5.461623in}{0.150000in}}%
\pgfpathlineto{\pgfqpoint{5.450200in}{0.150000in}}%
\pgfpathlineto{\pgfqpoint{5.438778in}{0.150000in}}%
\pgfpathlineto{\pgfqpoint{5.427355in}{0.150000in}}%
\pgfpathlineto{\pgfqpoint{5.415932in}{0.150000in}}%
\pgfpathlineto{\pgfqpoint{5.404509in}{0.150000in}}%
\pgfpathlineto{\pgfqpoint{5.393086in}{0.150000in}}%
\pgfpathlineto{\pgfqpoint{5.381663in}{0.150000in}}%
\pgfpathlineto{\pgfqpoint{5.370240in}{0.150000in}}%
\pgfpathlineto{\pgfqpoint{5.358818in}{0.150000in}}%
\pgfpathlineto{\pgfqpoint{5.347395in}{0.150000in}}%
\pgfpathlineto{\pgfqpoint{5.335972in}{0.150000in}}%
\pgfpathlineto{\pgfqpoint{5.324549in}{0.150000in}}%
\pgfpathlineto{\pgfqpoint{5.313126in}{0.150000in}}%
\pgfpathlineto{\pgfqpoint{5.301703in}{0.150000in}}%
\pgfpathlineto{\pgfqpoint{5.290281in}{0.150000in}}%
\pgfpathlineto{\pgfqpoint{5.278858in}{0.150000in}}%
\pgfpathlineto{\pgfqpoint{5.267435in}{0.150000in}}%
\pgfpathlineto{\pgfqpoint{5.256012in}{0.150000in}}%
\pgfpathlineto{\pgfqpoint{5.244589in}{0.150000in}}%
\pgfpathlineto{\pgfqpoint{5.233166in}{0.150000in}}%
\pgfpathlineto{\pgfqpoint{5.221743in}{0.150000in}}%
\pgfpathlineto{\pgfqpoint{5.210321in}{0.150000in}}%
\pgfpathlineto{\pgfqpoint{5.198898in}{0.150000in}}%
\pgfpathlineto{\pgfqpoint{5.187475in}{0.150000in}}%
\pgfpathlineto{\pgfqpoint{5.176052in}{0.150000in}}%
\pgfpathlineto{\pgfqpoint{5.164629in}{0.150000in}}%
\pgfpathlineto{\pgfqpoint{5.153206in}{0.150000in}}%
\pgfpathlineto{\pgfqpoint{5.141784in}{0.150000in}}%
\pgfpathlineto{\pgfqpoint{5.130361in}{0.150000in}}%
\pgfpathlineto{\pgfqpoint{5.118938in}{0.150000in}}%
\pgfpathlineto{\pgfqpoint{5.107515in}{0.150000in}}%
\pgfpathlineto{\pgfqpoint{5.096092in}{0.150000in}}%
\pgfpathlineto{\pgfqpoint{5.084669in}{0.150000in}}%
\pgfpathlineto{\pgfqpoint{5.073246in}{0.150000in}}%
\pgfpathlineto{\pgfqpoint{5.061824in}{0.150000in}}%
\pgfpathlineto{\pgfqpoint{5.050401in}{0.150000in}}%
\pgfpathlineto{\pgfqpoint{5.038978in}{0.150000in}}%
\pgfpathlineto{\pgfqpoint{5.027555in}{0.150000in}}%
\pgfpathlineto{\pgfqpoint{5.016132in}{0.150000in}}%
\pgfpathlineto{\pgfqpoint{5.004709in}{0.150000in}}%
\pgfpathlineto{\pgfqpoint{4.993287in}{0.150000in}}%
\pgfpathlineto{\pgfqpoint{4.981864in}{0.150000in}}%
\pgfpathlineto{\pgfqpoint{4.970441in}{0.150000in}}%
\pgfpathlineto{\pgfqpoint{4.959018in}{0.150000in}}%
\pgfpathlineto{\pgfqpoint{4.947595in}{0.150000in}}%
\pgfpathlineto{\pgfqpoint{4.936172in}{0.150000in}}%
\pgfpathlineto{\pgfqpoint{4.924749in}{0.150000in}}%
\pgfpathlineto{\pgfqpoint{4.913327in}{0.150000in}}%
\pgfpathlineto{\pgfqpoint{4.901904in}{0.150000in}}%
\pgfpathlineto{\pgfqpoint{4.890481in}{0.150000in}}%
\pgfpathlineto{\pgfqpoint{4.879058in}{0.150000in}}%
\pgfpathlineto{\pgfqpoint{4.867635in}{0.150000in}}%
\pgfpathlineto{\pgfqpoint{4.856212in}{0.150000in}}%
\pgfpathlineto{\pgfqpoint{4.844790in}{0.150000in}}%
\pgfpathlineto{\pgfqpoint{4.833367in}{0.150000in}}%
\pgfpathlineto{\pgfqpoint{4.821944in}{0.150000in}}%
\pgfpathlineto{\pgfqpoint{4.810521in}{0.150000in}}%
\pgfpathlineto{\pgfqpoint{4.799098in}{0.150000in}}%
\pgfpathlineto{\pgfqpoint{4.787675in}{0.150000in}}%
\pgfpathlineto{\pgfqpoint{4.776253in}{0.150000in}}%
\pgfpathlineto{\pgfqpoint{4.764830in}{0.150000in}}%
\pgfpathlineto{\pgfqpoint{4.753407in}{0.150000in}}%
\pgfpathlineto{\pgfqpoint{4.741984in}{0.150000in}}%
\pgfpathlineto{\pgfqpoint{4.730561in}{0.150000in}}%
\pgfpathlineto{\pgfqpoint{4.719138in}{0.150000in}}%
\pgfpathlineto{\pgfqpoint{4.707715in}{0.150000in}}%
\pgfpathlineto{\pgfqpoint{4.696293in}{0.150000in}}%
\pgfpathlineto{\pgfqpoint{4.684870in}{0.150000in}}%
\pgfpathlineto{\pgfqpoint{4.673447in}{0.150000in}}%
\pgfpathlineto{\pgfqpoint{4.662024in}{0.150000in}}%
\pgfpathlineto{\pgfqpoint{4.650601in}{0.150000in}}%
\pgfpathlineto{\pgfqpoint{4.639178in}{0.150000in}}%
\pgfpathlineto{\pgfqpoint{4.627756in}{0.150000in}}%
\pgfpathlineto{\pgfqpoint{4.616333in}{0.150000in}}%
\pgfpathlineto{\pgfqpoint{4.604910in}{0.150000in}}%
\pgfpathlineto{\pgfqpoint{4.593487in}{0.150000in}}%
\pgfpathlineto{\pgfqpoint{4.582064in}{0.150000in}}%
\pgfpathlineto{\pgfqpoint{4.570641in}{0.150000in}}%
\pgfpathlineto{\pgfqpoint{4.559218in}{0.150000in}}%
\pgfpathlineto{\pgfqpoint{4.547796in}{0.150000in}}%
\pgfpathlineto{\pgfqpoint{4.536373in}{0.150000in}}%
\pgfpathlineto{\pgfqpoint{4.524950in}{0.150000in}}%
\pgfpathlineto{\pgfqpoint{4.513527in}{0.150000in}}%
\pgfpathlineto{\pgfqpoint{4.502104in}{0.150000in}}%
\pgfpathlineto{\pgfqpoint{4.490681in}{0.150000in}}%
\pgfpathlineto{\pgfqpoint{4.479259in}{0.150000in}}%
\pgfpathlineto{\pgfqpoint{4.467836in}{0.150000in}}%
\pgfpathlineto{\pgfqpoint{4.456413in}{0.150000in}}%
\pgfpathlineto{\pgfqpoint{4.444990in}{0.150000in}}%
\pgfpathlineto{\pgfqpoint{4.433567in}{0.150000in}}%
\pgfpathlineto{\pgfqpoint{4.422144in}{0.150000in}}%
\pgfpathlineto{\pgfqpoint{4.410721in}{0.150000in}}%
\pgfpathlineto{\pgfqpoint{4.399299in}{0.150000in}}%
\pgfpathlineto{\pgfqpoint{4.387876in}{0.150000in}}%
\pgfpathlineto{\pgfqpoint{4.376453in}{0.150000in}}%
\pgfpathlineto{\pgfqpoint{4.365030in}{0.150000in}}%
\pgfpathlineto{\pgfqpoint{4.353607in}{0.150000in}}%
\pgfpathlineto{\pgfqpoint{4.342184in}{0.150000in}}%
\pgfpathlineto{\pgfqpoint{4.330762in}{0.150000in}}%
\pgfpathlineto{\pgfqpoint{4.319339in}{0.150000in}}%
\pgfpathlineto{\pgfqpoint{4.307916in}{0.150000in}}%
\pgfpathlineto{\pgfqpoint{4.296493in}{0.150000in}}%
\pgfpathlineto{\pgfqpoint{4.285070in}{0.150000in}}%
\pgfpathlineto{\pgfqpoint{4.273647in}{0.150000in}}%
\pgfpathlineto{\pgfqpoint{4.262224in}{0.150000in}}%
\pgfpathlineto{\pgfqpoint{4.250802in}{0.150000in}}%
\pgfpathlineto{\pgfqpoint{4.239379in}{0.150000in}}%
\pgfpathlineto{\pgfqpoint{4.227956in}{0.150000in}}%
\pgfpathlineto{\pgfqpoint{4.216533in}{0.150000in}}%
\pgfpathlineto{\pgfqpoint{4.205110in}{0.150000in}}%
\pgfpathlineto{\pgfqpoint{4.193687in}{0.150000in}}%
\pgfpathlineto{\pgfqpoint{4.182265in}{0.150000in}}%
\pgfpathlineto{\pgfqpoint{4.170842in}{0.150000in}}%
\pgfpathlineto{\pgfqpoint{4.159419in}{0.150000in}}%
\pgfpathlineto{\pgfqpoint{4.147996in}{0.150000in}}%
\pgfpathlineto{\pgfqpoint{4.136573in}{0.150000in}}%
\pgfpathlineto{\pgfqpoint{4.125150in}{0.150000in}}%
\pgfpathlineto{\pgfqpoint{4.113727in}{0.150000in}}%
\pgfpathlineto{\pgfqpoint{4.102305in}{0.150000in}}%
\pgfpathlineto{\pgfqpoint{4.090882in}{0.150000in}}%
\pgfpathlineto{\pgfqpoint{4.079459in}{0.150000in}}%
\pgfpathlineto{\pgfqpoint{4.068036in}{0.150000in}}%
\pgfpathlineto{\pgfqpoint{4.056613in}{0.150000in}}%
\pgfpathlineto{\pgfqpoint{4.045190in}{0.150000in}}%
\pgfpathlineto{\pgfqpoint{4.033768in}{0.150000in}}%
\pgfpathlineto{\pgfqpoint{4.022345in}{0.150000in}}%
\pgfpathlineto{\pgfqpoint{4.010922in}{0.150000in}}%
\pgfpathlineto{\pgfqpoint{3.999499in}{0.150000in}}%
\pgfpathlineto{\pgfqpoint{3.988076in}{0.150000in}}%
\pgfpathlineto{\pgfqpoint{3.976653in}{0.150000in}}%
\pgfpathlineto{\pgfqpoint{3.965230in}{0.150000in}}%
\pgfpathlineto{\pgfqpoint{3.953808in}{0.150000in}}%
\pgfpathlineto{\pgfqpoint{3.942385in}{0.150000in}}%
\pgfpathlineto{\pgfqpoint{3.930962in}{0.150000in}}%
\pgfpathlineto{\pgfqpoint{3.919539in}{0.150000in}}%
\pgfpathlineto{\pgfqpoint{3.908116in}{0.150000in}}%
\pgfpathlineto{\pgfqpoint{3.896693in}{0.150000in}}%
\pgfpathlineto{\pgfqpoint{3.885271in}{0.150000in}}%
\pgfpathlineto{\pgfqpoint{3.873848in}{0.150000in}}%
\pgfpathlineto{\pgfqpoint{3.862425in}{0.150000in}}%
\pgfpathlineto{\pgfqpoint{3.851002in}{0.150000in}}%
\pgfpathlineto{\pgfqpoint{3.839579in}{0.150000in}}%
\pgfpathlineto{\pgfqpoint{3.828156in}{0.150000in}}%
\pgfpathlineto{\pgfqpoint{3.816733in}{0.150000in}}%
\pgfpathlineto{\pgfqpoint{3.805311in}{0.150000in}}%
\pgfpathlineto{\pgfqpoint{3.793888in}{0.150000in}}%
\pgfpathlineto{\pgfqpoint{3.782465in}{0.150000in}}%
\pgfpathlineto{\pgfqpoint{3.771042in}{0.150000in}}%
\pgfpathlineto{\pgfqpoint{3.759619in}{0.150000in}}%
\pgfpathlineto{\pgfqpoint{3.748196in}{0.150000in}}%
\pgfpathlineto{\pgfqpoint{3.736774in}{0.150000in}}%
\pgfpathlineto{\pgfqpoint{3.725351in}{0.150000in}}%
\pgfpathlineto{\pgfqpoint{3.713928in}{0.150000in}}%
\pgfpathlineto{\pgfqpoint{3.702505in}{0.150000in}}%
\pgfpathlineto{\pgfqpoint{3.691082in}{0.150000in}}%
\pgfpathlineto{\pgfqpoint{3.679659in}{0.150000in}}%
\pgfpathlineto{\pgfqpoint{3.668236in}{0.150000in}}%
\pgfpathlineto{\pgfqpoint{3.656814in}{0.150000in}}%
\pgfpathlineto{\pgfqpoint{3.645391in}{0.150000in}}%
\pgfpathlineto{\pgfqpoint{3.633968in}{0.150000in}}%
\pgfpathlineto{\pgfqpoint{3.622545in}{0.150000in}}%
\pgfpathlineto{\pgfqpoint{3.611122in}{0.150000in}}%
\pgfpathlineto{\pgfqpoint{3.599699in}{0.150000in}}%
\pgfpathlineto{\pgfqpoint{3.588277in}{0.150000in}}%
\pgfpathlineto{\pgfqpoint{3.576854in}{0.150000in}}%
\pgfpathlineto{\pgfqpoint{3.565431in}{0.150000in}}%
\pgfpathlineto{\pgfqpoint{3.554008in}{0.150000in}}%
\pgfpathlineto{\pgfqpoint{3.542585in}{0.150000in}}%
\pgfpathlineto{\pgfqpoint{3.531162in}{0.150000in}}%
\pgfpathlineto{\pgfqpoint{3.519739in}{0.150000in}}%
\pgfpathlineto{\pgfqpoint{3.508317in}{0.150000in}}%
\pgfpathlineto{\pgfqpoint{3.496894in}{0.150000in}}%
\pgfpathlineto{\pgfqpoint{3.485471in}{0.150000in}}%
\pgfpathlineto{\pgfqpoint{3.474048in}{0.150000in}}%
\pgfpathlineto{\pgfqpoint{3.462625in}{0.150000in}}%
\pgfpathlineto{\pgfqpoint{3.451202in}{0.150000in}}%
\pgfpathlineto{\pgfqpoint{3.439780in}{0.150000in}}%
\pgfpathlineto{\pgfqpoint{3.428357in}{0.150000in}}%
\pgfpathlineto{\pgfqpoint{3.416934in}{0.150000in}}%
\pgfpathlineto{\pgfqpoint{3.405511in}{0.150000in}}%
\pgfpathlineto{\pgfqpoint{3.394088in}{0.150000in}}%
\pgfpathlineto{\pgfqpoint{3.382665in}{0.150000in}}%
\pgfpathlineto{\pgfqpoint{3.371242in}{0.150000in}}%
\pgfpathlineto{\pgfqpoint{3.359820in}{0.150000in}}%
\pgfpathlineto{\pgfqpoint{3.348397in}{0.150000in}}%
\pgfpathlineto{\pgfqpoint{3.336974in}{0.150000in}}%
\pgfpathlineto{\pgfqpoint{3.325551in}{0.150000in}}%
\pgfpathlineto{\pgfqpoint{3.314128in}{0.150000in}}%
\pgfpathlineto{\pgfqpoint{3.302705in}{0.150000in}}%
\pgfpathlineto{\pgfqpoint{3.291283in}{0.150000in}}%
\pgfpathlineto{\pgfqpoint{3.279860in}{0.150000in}}%
\pgfpathlineto{\pgfqpoint{3.268437in}{0.150000in}}%
\pgfpathlineto{\pgfqpoint{3.257014in}{0.150000in}}%
\pgfpathlineto{\pgfqpoint{3.245591in}{0.150000in}}%
\pgfpathlineto{\pgfqpoint{3.234168in}{0.150000in}}%
\pgfpathlineto{\pgfqpoint{3.222745in}{0.150000in}}%
\pgfpathlineto{\pgfqpoint{3.211323in}{0.150000in}}%
\pgfpathlineto{\pgfqpoint{3.199900in}{0.150000in}}%
\pgfpathlineto{\pgfqpoint{3.188477in}{0.150000in}}%
\pgfpathlineto{\pgfqpoint{3.177054in}{0.150000in}}%
\pgfpathlineto{\pgfqpoint{3.165631in}{0.150000in}}%
\pgfpathlineto{\pgfqpoint{3.154208in}{0.150000in}}%
\pgfpathlineto{\pgfqpoint{3.142786in}{0.150000in}}%
\pgfpathlineto{\pgfqpoint{3.131363in}{0.150000in}}%
\pgfpathlineto{\pgfqpoint{3.119940in}{0.150000in}}%
\pgfpathlineto{\pgfqpoint{3.108517in}{0.150000in}}%
\pgfpathlineto{\pgfqpoint{3.097094in}{0.150000in}}%
\pgfpathlineto{\pgfqpoint{3.085671in}{0.150000in}}%
\pgfpathlineto{\pgfqpoint{3.074248in}{0.150000in}}%
\pgfpathlineto{\pgfqpoint{3.062826in}{0.150000in}}%
\pgfpathlineto{\pgfqpoint{3.051403in}{0.150000in}}%
\pgfpathlineto{\pgfqpoint{3.039980in}{0.150000in}}%
\pgfpathlineto{\pgfqpoint{3.028557in}{0.150000in}}%
\pgfpathlineto{\pgfqpoint{3.017134in}{0.150000in}}%
\pgfpathlineto{\pgfqpoint{3.005711in}{0.150000in}}%
\pgfpathlineto{\pgfqpoint{2.994289in}{0.150000in}}%
\pgfpathlineto{\pgfqpoint{2.982866in}{0.150000in}}%
\pgfpathlineto{\pgfqpoint{2.971443in}{0.150000in}}%
\pgfpathlineto{\pgfqpoint{2.960020in}{0.150000in}}%
\pgfpathlineto{\pgfqpoint{2.948597in}{0.150000in}}%
\pgfpathlineto{\pgfqpoint{2.937174in}{0.150000in}}%
\pgfpathlineto{\pgfqpoint{2.925752in}{0.150000in}}%
\pgfpathlineto{\pgfqpoint{2.914329in}{0.150000in}}%
\pgfpathlineto{\pgfqpoint{2.902906in}{0.150000in}}%
\pgfpathlineto{\pgfqpoint{2.891483in}{0.150000in}}%
\pgfpathlineto{\pgfqpoint{2.880060in}{0.150000in}}%
\pgfpathlineto{\pgfqpoint{2.868637in}{0.150000in}}%
\pgfpathlineto{\pgfqpoint{2.857214in}{0.150000in}}%
\pgfpathlineto{\pgfqpoint{2.845792in}{0.150000in}}%
\pgfpathlineto{\pgfqpoint{2.834369in}{0.150000in}}%
\pgfpathlineto{\pgfqpoint{2.822946in}{0.150000in}}%
\pgfpathlineto{\pgfqpoint{2.811523in}{0.150000in}}%
\pgfpathlineto{\pgfqpoint{2.800100in}{0.150000in}}%
\pgfpathlineto{\pgfqpoint{2.788677in}{0.150000in}}%
\pgfpathlineto{\pgfqpoint{2.777255in}{0.150000in}}%
\pgfpathlineto{\pgfqpoint{2.765832in}{0.150000in}}%
\pgfpathlineto{\pgfqpoint{2.754409in}{0.150000in}}%
\pgfpathlineto{\pgfqpoint{2.742986in}{0.150000in}}%
\pgfpathlineto{\pgfqpoint{2.731563in}{0.150000in}}%
\pgfpathlineto{\pgfqpoint{2.720140in}{0.150000in}}%
\pgfpathlineto{\pgfqpoint{2.708717in}{0.150000in}}%
\pgfpathlineto{\pgfqpoint{2.697295in}{0.150000in}}%
\pgfpathlineto{\pgfqpoint{2.685872in}{0.150000in}}%
\pgfpathlineto{\pgfqpoint{2.674449in}{0.150000in}}%
\pgfpathlineto{\pgfqpoint{2.663026in}{0.150000in}}%
\pgfpathlineto{\pgfqpoint{2.651603in}{0.150000in}}%
\pgfpathlineto{\pgfqpoint{2.640180in}{0.150000in}}%
\pgfpathlineto{\pgfqpoint{2.628758in}{0.150000in}}%
\pgfpathlineto{\pgfqpoint{2.617335in}{0.150000in}}%
\pgfpathlineto{\pgfqpoint{2.605912in}{0.150000in}}%
\pgfpathlineto{\pgfqpoint{2.594489in}{0.150000in}}%
\pgfpathlineto{\pgfqpoint{2.583066in}{0.150000in}}%
\pgfpathlineto{\pgfqpoint{2.571643in}{0.150000in}}%
\pgfpathlineto{\pgfqpoint{2.560220in}{0.150000in}}%
\pgfpathlineto{\pgfqpoint{2.548798in}{0.150000in}}%
\pgfpathlineto{\pgfqpoint{2.537375in}{0.150000in}}%
\pgfpathlineto{\pgfqpoint{2.525952in}{0.150000in}}%
\pgfpathlineto{\pgfqpoint{2.514529in}{0.150000in}}%
\pgfpathlineto{\pgfqpoint{2.503106in}{0.150000in}}%
\pgfpathlineto{\pgfqpoint{2.491683in}{0.150000in}}%
\pgfpathlineto{\pgfqpoint{2.480261in}{0.150000in}}%
\pgfpathlineto{\pgfqpoint{2.468838in}{0.150000in}}%
\pgfpathlineto{\pgfqpoint{2.457415in}{0.150000in}}%
\pgfpathlineto{\pgfqpoint{2.445992in}{0.150000in}}%
\pgfpathlineto{\pgfqpoint{2.434569in}{0.150000in}}%
\pgfpathlineto{\pgfqpoint{2.423146in}{0.150000in}}%
\pgfpathlineto{\pgfqpoint{2.411723in}{0.150000in}}%
\pgfpathlineto{\pgfqpoint{2.400301in}{0.150000in}}%
\pgfpathlineto{\pgfqpoint{2.388878in}{0.150000in}}%
\pgfpathlineto{\pgfqpoint{2.377455in}{0.150000in}}%
\pgfpathlineto{\pgfqpoint{2.366032in}{0.150000in}}%
\pgfpathlineto{\pgfqpoint{2.354609in}{0.150000in}}%
\pgfpathlineto{\pgfqpoint{2.343186in}{0.150000in}}%
\pgfpathlineto{\pgfqpoint{2.331764in}{0.150000in}}%
\pgfpathlineto{\pgfqpoint{2.320341in}{0.150000in}}%
\pgfpathlineto{\pgfqpoint{2.308918in}{0.150000in}}%
\pgfpathlineto{\pgfqpoint{2.297495in}{0.150000in}}%
\pgfpathlineto{\pgfqpoint{2.286072in}{0.150000in}}%
\pgfpathlineto{\pgfqpoint{2.274649in}{0.150000in}}%
\pgfpathlineto{\pgfqpoint{2.263226in}{0.150000in}}%
\pgfpathlineto{\pgfqpoint{2.251804in}{0.150000in}}%
\pgfpathlineto{\pgfqpoint{2.240381in}{0.150000in}}%
\pgfpathlineto{\pgfqpoint{2.228958in}{0.150000in}}%
\pgfpathlineto{\pgfqpoint{2.217535in}{0.150000in}}%
\pgfpathlineto{\pgfqpoint{2.206112in}{0.150000in}}%
\pgfpathlineto{\pgfqpoint{2.194689in}{0.150000in}}%
\pgfpathlineto{\pgfqpoint{2.183267in}{0.150000in}}%
\pgfpathlineto{\pgfqpoint{2.171844in}{0.150000in}}%
\pgfpathlineto{\pgfqpoint{2.160421in}{0.150000in}}%
\pgfpathlineto{\pgfqpoint{2.148998in}{0.150000in}}%
\pgfpathlineto{\pgfqpoint{2.137575in}{0.150000in}}%
\pgfpathlineto{\pgfqpoint{2.126152in}{0.150000in}}%
\pgfpathlineto{\pgfqpoint{2.114729in}{0.150000in}}%
\pgfpathlineto{\pgfqpoint{2.103307in}{0.150000in}}%
\pgfpathlineto{\pgfqpoint{2.091884in}{0.150000in}}%
\pgfpathlineto{\pgfqpoint{2.080461in}{0.150000in}}%
\pgfpathlineto{\pgfqpoint{2.069038in}{0.150000in}}%
\pgfpathlineto{\pgfqpoint{2.057615in}{0.150000in}}%
\pgfpathlineto{\pgfqpoint{2.046192in}{0.150000in}}%
\pgfpathlineto{\pgfqpoint{2.034770in}{0.150000in}}%
\pgfpathlineto{\pgfqpoint{2.023347in}{0.150000in}}%
\pgfpathlineto{\pgfqpoint{2.011924in}{0.150000in}}%
\pgfpathlineto{\pgfqpoint{2.000501in}{0.150000in}}%
\pgfpathlineto{\pgfqpoint{1.989078in}{0.150000in}}%
\pgfpathlineto{\pgfqpoint{1.977655in}{0.150000in}}%
\pgfpathlineto{\pgfqpoint{1.966232in}{0.150000in}}%
\pgfpathlineto{\pgfqpoint{1.954810in}{0.150000in}}%
\pgfpathlineto{\pgfqpoint{1.943387in}{0.150000in}}%
\pgfpathlineto{\pgfqpoint{1.931964in}{0.150000in}}%
\pgfpathlineto{\pgfqpoint{1.920541in}{0.150000in}}%
\pgfpathlineto{\pgfqpoint{1.909118in}{0.150000in}}%
\pgfpathlineto{\pgfqpoint{1.897695in}{0.150000in}}%
\pgfpathlineto{\pgfqpoint{1.886273in}{0.150000in}}%
\pgfpathlineto{\pgfqpoint{1.874850in}{0.150000in}}%
\pgfpathlineto{\pgfqpoint{1.863427in}{0.150000in}}%
\pgfpathlineto{\pgfqpoint{1.852004in}{0.150000in}}%
\pgfpathlineto{\pgfqpoint{1.840581in}{0.150000in}}%
\pgfpathlineto{\pgfqpoint{1.829158in}{0.150000in}}%
\pgfpathlineto{\pgfqpoint{1.817735in}{0.150000in}}%
\pgfpathlineto{\pgfqpoint{1.806313in}{0.150000in}}%
\pgfpathlineto{\pgfqpoint{1.794890in}{0.150000in}}%
\pgfpathlineto{\pgfqpoint{1.783467in}{0.150000in}}%
\pgfpathlineto{\pgfqpoint{1.772044in}{0.150000in}}%
\pgfpathlineto{\pgfqpoint{1.760621in}{0.150000in}}%
\pgfpathlineto{\pgfqpoint{1.749198in}{0.150000in}}%
\pgfpathlineto{\pgfqpoint{1.737776in}{0.150000in}}%
\pgfpathlineto{\pgfqpoint{1.726353in}{0.150000in}}%
\pgfpathlineto{\pgfqpoint{1.714930in}{0.150000in}}%
\pgfpathlineto{\pgfqpoint{1.703507in}{0.150000in}}%
\pgfpathlineto{\pgfqpoint{1.692084in}{0.150000in}}%
\pgfpathlineto{\pgfqpoint{1.680661in}{0.150000in}}%
\pgfpathlineto{\pgfqpoint{1.669238in}{0.150000in}}%
\pgfpathlineto{\pgfqpoint{1.657816in}{0.150000in}}%
\pgfpathlineto{\pgfqpoint{1.646393in}{0.150000in}}%
\pgfpathlineto{\pgfqpoint{1.634970in}{0.150000in}}%
\pgfpathlineto{\pgfqpoint{1.623547in}{0.150000in}}%
\pgfpathlineto{\pgfqpoint{1.612124in}{0.150000in}}%
\pgfpathlineto{\pgfqpoint{1.600701in}{0.150000in}}%
\pgfpathlineto{\pgfqpoint{1.589279in}{0.150000in}}%
\pgfpathlineto{\pgfqpoint{1.577856in}{0.150000in}}%
\pgfpathlineto{\pgfqpoint{1.566433in}{0.150000in}}%
\pgfpathlineto{\pgfqpoint{1.555010in}{0.150000in}}%
\pgfpathlineto{\pgfqpoint{1.543587in}{0.150000in}}%
\pgfpathlineto{\pgfqpoint{1.532164in}{0.150000in}}%
\pgfpathlineto{\pgfqpoint{1.520741in}{0.150000in}}%
\pgfpathlineto{\pgfqpoint{1.509319in}{0.150000in}}%
\pgfpathlineto{\pgfqpoint{1.497896in}{0.150000in}}%
\pgfpathlineto{\pgfqpoint{1.486473in}{0.150000in}}%
\pgfpathlineto{\pgfqpoint{1.475050in}{0.150000in}}%
\pgfpathlineto{\pgfqpoint{1.463627in}{0.150000in}}%
\pgfpathlineto{\pgfqpoint{1.452204in}{0.150000in}}%
\pgfpathlineto{\pgfqpoint{1.440782in}{0.150000in}}%
\pgfpathlineto{\pgfqpoint{1.429359in}{0.150000in}}%
\pgfpathlineto{\pgfqpoint{1.417936in}{0.150000in}}%
\pgfpathlineto{\pgfqpoint{1.406513in}{0.150000in}}%
\pgfpathlineto{\pgfqpoint{1.395090in}{0.150000in}}%
\pgfpathlineto{\pgfqpoint{1.383667in}{0.150000in}}%
\pgfpathlineto{\pgfqpoint{1.372244in}{0.150000in}}%
\pgfpathlineto{\pgfqpoint{1.360822in}{0.150000in}}%
\pgfpathlineto{\pgfqpoint{1.349399in}{0.150000in}}%
\pgfpathlineto{\pgfqpoint{1.337976in}{0.150000in}}%
\pgfpathlineto{\pgfqpoint{1.326553in}{0.150000in}}%
\pgfpathlineto{\pgfqpoint{1.315130in}{0.150000in}}%
\pgfpathlineto{\pgfqpoint{1.303707in}{0.150000in}}%
\pgfpathlineto{\pgfqpoint{1.292285in}{0.150000in}}%
\pgfpathlineto{\pgfqpoint{1.280862in}{0.150000in}}%
\pgfpathlineto{\pgfqpoint{1.269439in}{0.150000in}}%
\pgfpathlineto{\pgfqpoint{1.258016in}{0.150000in}}%
\pgfpathlineto{\pgfqpoint{1.246593in}{0.150000in}}%
\pgfpathlineto{\pgfqpoint{1.235170in}{0.150000in}}%
\pgfpathlineto{\pgfqpoint{1.223747in}{0.150000in}}%
\pgfpathlineto{\pgfqpoint{1.212325in}{0.150000in}}%
\pgfpathlineto{\pgfqpoint{1.200902in}{0.150000in}}%
\pgfpathlineto{\pgfqpoint{1.189479in}{0.150000in}}%
\pgfpathlineto{\pgfqpoint{1.178056in}{0.150000in}}%
\pgfpathlineto{\pgfqpoint{1.166633in}{0.150000in}}%
\pgfpathlineto{\pgfqpoint{1.155210in}{0.150000in}}%
\pgfpathlineto{\pgfqpoint{1.143788in}{0.150000in}}%
\pgfpathlineto{\pgfqpoint{1.132365in}{0.150000in}}%
\pgfpathlineto{\pgfqpoint{1.120942in}{0.150000in}}%
\pgfpathlineto{\pgfqpoint{1.109519in}{0.150000in}}%
\pgfpathlineto{\pgfqpoint{1.098096in}{0.150000in}}%
\pgfpathlineto{\pgfqpoint{1.086673in}{0.150000in}}%
\pgfpathlineto{\pgfqpoint{1.075251in}{0.150000in}}%
\pgfpathlineto{\pgfqpoint{1.063828in}{0.150000in}}%
\pgfpathlineto{\pgfqpoint{1.052405in}{0.150000in}}%
\pgfpathlineto{\pgfqpoint{1.040982in}{0.150000in}}%
\pgfpathlineto{\pgfqpoint{1.029559in}{0.150000in}}%
\pgfpathlineto{\pgfqpoint{1.018136in}{0.150000in}}%
\pgfpathlineto{\pgfqpoint{1.006713in}{0.150000in}}%
\pgfpathlineto{\pgfqpoint{0.995291in}{0.150000in}}%
\pgfpathlineto{\pgfqpoint{0.983868in}{0.150000in}}%
\pgfpathlineto{\pgfqpoint{0.972445in}{0.150000in}}%
\pgfpathlineto{\pgfqpoint{0.961022in}{0.150000in}}%
\pgfpathlineto{\pgfqpoint{0.949599in}{0.150000in}}%
\pgfpathlineto{\pgfqpoint{0.938176in}{0.150000in}}%
\pgfpathlineto{\pgfqpoint{0.926754in}{0.150000in}}%
\pgfpathlineto{\pgfqpoint{0.915331in}{0.150000in}}%
\pgfpathlineto{\pgfqpoint{0.903908in}{0.150000in}}%
\pgfpathlineto{\pgfqpoint{0.892485in}{0.150000in}}%
\pgfpathlineto{\pgfqpoint{0.881062in}{0.150000in}}%
\pgfpathlineto{\pgfqpoint{0.869639in}{0.150000in}}%
\pgfpathlineto{\pgfqpoint{0.858216in}{0.150000in}}%
\pgfpathlineto{\pgfqpoint{0.846794in}{0.150000in}}%
\pgfpathlineto{\pgfqpoint{0.835371in}{0.150000in}}%
\pgfpathlineto{\pgfqpoint{0.823948in}{0.150000in}}%
\pgfpathlineto{\pgfqpoint{0.812525in}{0.150000in}}%
\pgfpathlineto{\pgfqpoint{0.801102in}{0.150000in}}%
\pgfpathlineto{\pgfqpoint{0.789679in}{0.150000in}}%
\pgfpathlineto{\pgfqpoint{0.778257in}{0.150000in}}%
\pgfpathlineto{\pgfqpoint{0.766834in}{0.150000in}}%
\pgfpathlineto{\pgfqpoint{0.755411in}{0.150000in}}%
\pgfpathlineto{\pgfqpoint{0.743988in}{0.150000in}}%
\pgfpathlineto{\pgfqpoint{0.732565in}{0.150000in}}%
\pgfpathlineto{\pgfqpoint{0.721142in}{0.150000in}}%
\pgfpathlineto{\pgfqpoint{0.709719in}{0.150000in}}%
\pgfpathlineto{\pgfqpoint{0.698297in}{0.150000in}}%
\pgfpathlineto{\pgfqpoint{0.686874in}{0.150000in}}%
\pgfpathlineto{\pgfqpoint{0.675451in}{0.150000in}}%
\pgfpathlineto{\pgfqpoint{0.664028in}{0.150000in}}%
\pgfpathlineto{\pgfqpoint{0.652605in}{0.150000in}}%
\pgfpathlineto{\pgfqpoint{0.641182in}{0.150000in}}%
\pgfpathlineto{\pgfqpoint{0.629760in}{0.150000in}}%
\pgfpathlineto{\pgfqpoint{0.618337in}{0.150000in}}%
\pgfpathlineto{\pgfqpoint{0.606914in}{0.150000in}}%
\pgfpathlineto{\pgfqpoint{0.595491in}{0.150000in}}%
\pgfpathlineto{\pgfqpoint{0.584068in}{0.150000in}}%
\pgfpathlineto{\pgfqpoint{0.572645in}{0.150000in}}%
\pgfpathlineto{\pgfqpoint{0.561222in}{0.150000in}}%
\pgfpathlineto{\pgfqpoint{0.549800in}{0.150000in}}%
\pgfpathlineto{\pgfqpoint{0.538377in}{0.150000in}}%
\pgfpathlineto{\pgfqpoint{0.526954in}{0.150000in}}%
\pgfpathlineto{\pgfqpoint{0.515531in}{0.150000in}}%
\pgfpathlineto{\pgfqpoint{0.504108in}{0.150000in}}%
\pgfpathlineto{\pgfqpoint{0.492685in}{0.150000in}}%
\pgfpathlineto{\pgfqpoint{0.481263in}{0.150000in}}%
\pgfpathlineto{\pgfqpoint{0.469840in}{0.150000in}}%
\pgfpathlineto{\pgfqpoint{0.458417in}{0.150000in}}%
\pgfpathlineto{\pgfqpoint{0.446994in}{0.150000in}}%
\pgfpathlineto{\pgfqpoint{0.435571in}{0.150000in}}%
\pgfpathlineto{\pgfqpoint{0.424148in}{0.150000in}}%
\pgfpathlineto{\pgfqpoint{0.412725in}{0.150000in}}%
\pgfpathlineto{\pgfqpoint{0.401303in}{0.150000in}}%
\pgfpathlineto{\pgfqpoint{0.389880in}{0.150000in}}%
\pgfpathlineto{\pgfqpoint{0.378457in}{0.150000in}}%
\pgfpathlineto{\pgfqpoint{0.367034in}{0.150000in}}%
\pgfpathlineto{\pgfqpoint{0.355611in}{0.150000in}}%
\pgfpathlineto{\pgfqpoint{0.344188in}{0.150000in}}%
\pgfpathlineto{\pgfqpoint{0.332766in}{0.150000in}}%
\pgfpathlineto{\pgfqpoint{0.321343in}{0.150000in}}%
\pgfpathlineto{\pgfqpoint{0.309920in}{0.150000in}}%
\pgfpathlineto{\pgfqpoint{0.298497in}{0.150000in}}%
\pgfpathlineto{\pgfqpoint{0.287074in}{0.150000in}}%
\pgfpathlineto{\pgfqpoint{0.275651in}{0.150000in}}%
\pgfpathlineto{\pgfqpoint{0.264228in}{0.150000in}}%
\pgfpathlineto{\pgfqpoint{0.252806in}{0.150000in}}%
\pgfpathlineto{\pgfqpoint{0.241383in}{0.150000in}}%
\pgfpathlineto{\pgfqpoint{0.229960in}{0.150000in}}%
\pgfpathlineto{\pgfqpoint{0.218537in}{0.150000in}}%
\pgfpathlineto{\pgfqpoint{0.207114in}{0.150000in}}%
\pgfpathlineto{\pgfqpoint{0.195691in}{0.150000in}}%
\pgfpathlineto{\pgfqpoint{0.184269in}{0.150000in}}%
\pgfpathlineto{\pgfqpoint{0.172846in}{0.150000in}}%
\pgfpathlineto{\pgfqpoint{0.161423in}{0.150000in}}%
\pgfpathlineto{\pgfqpoint{0.150000in}{0.150000in}}%
\pgfpathclose%
\pgfusepath{stroke,fill}%
\end{pgfscope}%
\begin{pgfscope}%
\pgfpathrectangle{\pgfqpoint{0.150000in}{0.150000in}}{\pgfqpoint{5.700000in}{2.200000in}}%
\pgfusepath{clip}%
\pgfsetroundcap%
\pgfsetroundjoin%
\pgfsetlinewidth{2.007500pt}%
\definecolor{currentstroke}{rgb}{0.203922,0.396078,0.643137}%
\pgfsetstrokecolor{currentstroke}%
\pgfsetdash{}{0pt}%
\pgfpathmoveto{\pgfqpoint{0.136111in}{0.838512in}}%
\pgfpathlineto{\pgfqpoint{0.218087in}{0.862690in}}%
\pgfpathlineto{\pgfqpoint{0.313684in}{0.894357in}}%
\pgfpathlineto{\pgfqpoint{0.409281in}{0.929663in}}%
\pgfpathlineto{\pgfqpoint{0.504878in}{0.968439in}}%
\pgfpathlineto{\pgfqpoint{0.600474in}{1.010360in}}%
\pgfpathlineto{\pgfqpoint{0.727937in}{1.070269in}}%
\pgfpathlineto{\pgfqpoint{0.887264in}{1.149185in}}%
\pgfpathlineto{\pgfqpoint{1.078458in}{1.243853in}}%
\pgfpathlineto{\pgfqpoint{1.174055in}{1.288447in}}%
\pgfpathlineto{\pgfqpoint{1.269651in}{1.329513in}}%
\pgfpathlineto{\pgfqpoint{1.333383in}{1.354293in}}%
\pgfpathlineto{\pgfqpoint{1.397114in}{1.376568in}}%
\pgfpathlineto{\pgfqpoint{1.460845in}{1.395994in}}%
\pgfpathlineto{\pgfqpoint{1.524576in}{1.412257in}}%
\pgfpathlineto{\pgfqpoint{1.588307in}{1.425086in}}%
\pgfpathlineto{\pgfqpoint{1.652038in}{1.434253in}}%
\pgfpathlineto{\pgfqpoint{1.715769in}{1.439586in}}%
\pgfpathlineto{\pgfqpoint{1.779501in}{1.440971in}}%
\pgfpathlineto{\pgfqpoint{1.843232in}{1.438357in}}%
\pgfpathlineto{\pgfqpoint{1.906963in}{1.431760in}}%
\pgfpathlineto{\pgfqpoint{1.970694in}{1.421258in}}%
\pgfpathlineto{\pgfqpoint{2.034425in}{1.406997in}}%
\pgfpathlineto{\pgfqpoint{2.098156in}{1.389184in}}%
\pgfpathlineto{\pgfqpoint{2.161888in}{1.368079in}}%
\pgfpathlineto{\pgfqpoint{2.225619in}{1.343998in}}%
\pgfpathlineto{\pgfqpoint{2.289350in}{1.317294in}}%
\pgfpathlineto{\pgfqpoint{2.384947in}{1.273180in}}%
\pgfpathlineto{\pgfqpoint{2.480543in}{1.225454in}}%
\pgfpathlineto{\pgfqpoint{2.671737in}{1.124962in}}%
\pgfpathlineto{\pgfqpoint{2.831065in}{1.042587in}}%
\pgfpathlineto{\pgfqpoint{2.926661in}{0.996206in}}%
\pgfpathlineto{\pgfqpoint{3.022258in}{0.953234in}}%
\pgfpathlineto{\pgfqpoint{3.117855in}{0.914347in}}%
\pgfpathlineto{\pgfqpoint{3.213452in}{0.880017in}}%
\pgfpathlineto{\pgfqpoint{3.277183in}{0.859807in}}%
\pgfpathlineto{\pgfqpoint{3.340914in}{0.841790in}}%
\pgfpathlineto{\pgfqpoint{3.404645in}{0.825977in}}%
\pgfpathlineto{\pgfqpoint{3.468376in}{0.812354in}}%
\pgfpathlineto{\pgfqpoint{3.532107in}{0.800885in}}%
\pgfpathlineto{\pgfqpoint{3.627704in}{0.787601in}}%
\pgfpathlineto{\pgfqpoint{3.723301in}{0.778806in}}%
\pgfpathlineto{\pgfqpoint{3.818898in}{0.774215in}}%
\pgfpathlineto{\pgfqpoint{3.914494in}{0.773509in}}%
\pgfpathlineto{\pgfqpoint{4.010091in}{0.776346in}}%
\pgfpathlineto{\pgfqpoint{4.105688in}{0.782355in}}%
\pgfpathlineto{\pgfqpoint{4.201285in}{0.791143in}}%
\pgfpathlineto{\pgfqpoint{4.328747in}{0.806443in}}%
\pgfpathlineto{\pgfqpoint{4.488075in}{0.829745in}}%
\pgfpathlineto{\pgfqpoint{4.934193in}{0.898936in}}%
\pgfpathlineto{\pgfqpoint{5.061655in}{0.913860in}}%
\pgfpathlineto{\pgfqpoint{5.157252in}{0.922225in}}%
\pgfpathlineto{\pgfqpoint{5.252849in}{0.927754in}}%
\pgfpathlineto{\pgfqpoint{5.348445in}{0.930203in}}%
\pgfpathlineto{\pgfqpoint{5.444042in}{0.929459in}}%
\pgfpathlineto{\pgfqpoint{5.539639in}{0.925545in}}%
\pgfpathlineto{\pgfqpoint{5.635236in}{0.918619in}}%
\pgfpathlineto{\pgfqpoint{5.730832in}{0.908960in}}%
\pgfpathlineto{\pgfqpoint{5.858295in}{0.892503in}}%
\pgfpathlineto{\pgfqpoint{5.863889in}{0.891688in}}%
\pgfpathlineto{\pgfqpoint{5.863889in}{0.891688in}}%
\pgfusepath{stroke}%
\end{pgfscope}%
\begin{pgfscope}%
\pgfpathrectangle{\pgfqpoint{0.150000in}{0.150000in}}{\pgfqpoint{5.700000in}{2.200000in}}%
\pgfusepath{clip}%
\pgfsetroundcap%
\pgfsetroundjoin%
\pgfsetlinewidth{1.505625pt}%
\definecolor{currentstroke}{rgb}{0.000000,0.501961,0.000000}%
\pgfsetstrokecolor{currentstroke}%
\pgfsetdash{}{0pt}%
\pgfpathmoveto{\pgfqpoint{0.150000in}{0.396729in}}%
\pgfpathlineto{\pgfqpoint{0.287074in}{0.391742in}}%
\pgfpathlineto{\pgfqpoint{0.412725in}{0.384198in}}%
\pgfpathlineto{\pgfqpoint{0.549800in}{0.372942in}}%
\pgfpathlineto{\pgfqpoint{0.743988in}{0.353425in}}%
\pgfpathlineto{\pgfqpoint{0.961022in}{0.332101in}}%
\pgfpathlineto{\pgfqpoint{1.075251in}{0.323896in}}%
\pgfpathlineto{\pgfqpoint{1.143788in}{0.321439in}}%
\pgfpathlineto{\pgfqpoint{1.155210in}{0.324772in}}%
\pgfpathlineto{\pgfqpoint{1.258016in}{0.370629in}}%
\pgfpathlineto{\pgfqpoint{1.315130in}{0.392393in}}%
\pgfpathlineto{\pgfqpoint{1.372244in}{0.410918in}}%
\pgfpathlineto{\pgfqpoint{1.429359in}{0.425927in}}%
\pgfpathlineto{\pgfqpoint{1.486473in}{0.437198in}}%
\pgfpathlineto{\pgfqpoint{1.532164in}{0.443410in}}%
\pgfpathlineto{\pgfqpoint{1.577856in}{0.447066in}}%
\pgfpathlineto{\pgfqpoint{1.623547in}{0.448136in}}%
\pgfpathlineto{\pgfqpoint{1.669238in}{0.446615in}}%
\pgfpathlineto{\pgfqpoint{1.714930in}{0.442526in}}%
\pgfpathlineto{\pgfqpoint{1.772044in}{0.433880in}}%
\pgfpathlineto{\pgfqpoint{1.829158in}{0.421456in}}%
\pgfpathlineto{\pgfqpoint{1.886273in}{0.405561in}}%
\pgfpathlineto{\pgfqpoint{1.909118in}{0.399559in}}%
\pgfpathlineto{\pgfqpoint{1.931964in}{0.402706in}}%
\pgfpathlineto{\pgfqpoint{2.171844in}{0.447326in}}%
\pgfpathlineto{\pgfqpoint{2.274649in}{0.462930in}}%
\pgfpathlineto{\pgfqpoint{2.366032in}{0.473976in}}%
\pgfpathlineto{\pgfqpoint{2.457415in}{0.481992in}}%
\pgfpathlineto{\pgfqpoint{2.548798in}{0.486830in}}%
\pgfpathlineto{\pgfqpoint{2.640180in}{0.488537in}}%
\pgfpathlineto{\pgfqpoint{2.731563in}{0.487336in}}%
\pgfpathlineto{\pgfqpoint{2.834369in}{0.482976in}}%
\pgfpathlineto{\pgfqpoint{2.960020in}{0.474303in}}%
\pgfpathlineto{\pgfqpoint{3.131363in}{0.459015in}}%
\pgfpathlineto{\pgfqpoint{3.428357in}{0.432357in}}%
\pgfpathlineto{\pgfqpoint{3.588277in}{0.421556in}}%
\pgfpathlineto{\pgfqpoint{3.736774in}{0.414610in}}%
\pgfpathlineto{\pgfqpoint{3.908116in}{0.409758in}}%
\pgfpathlineto{\pgfqpoint{4.307916in}{0.400135in}}%
\pgfpathlineto{\pgfqpoint{4.422144in}{0.393638in}}%
\pgfpathlineto{\pgfqpoint{4.524950in}{0.384737in}}%
\pgfpathlineto{\pgfqpoint{4.616333in}{0.373793in}}%
\pgfpathlineto{\pgfqpoint{4.707715in}{0.359551in}}%
\pgfpathlineto{\pgfqpoint{4.787675in}{0.344133in}}%
\pgfpathlineto{\pgfqpoint{4.867635in}{0.325815in}}%
\pgfpathlineto{\pgfqpoint{4.947595in}{0.304554in}}%
\pgfpathlineto{\pgfqpoint{5.027555in}{0.280405in}}%
\pgfpathlineto{\pgfqpoint{5.118938in}{0.249464in}}%
\pgfpathlineto{\pgfqpoint{5.210321in}{0.215323in}}%
\pgfpathlineto{\pgfqpoint{5.313126in}{0.173760in}}%
\pgfpathlineto{\pgfqpoint{5.370240in}{0.150000in}}%
\pgfpathlineto{\pgfqpoint{5.381663in}{0.150687in}}%
\pgfpathlineto{\pgfqpoint{5.598697in}{0.237625in}}%
\pgfpathlineto{\pgfqpoint{5.690080in}{0.269692in}}%
\pgfpathlineto{\pgfqpoint{5.781463in}{0.298272in}}%
\pgfpathlineto{\pgfqpoint{5.850000in}{0.317241in}}%
\pgfpathlineto{\pgfqpoint{5.850000in}{0.317241in}}%
\pgfusepath{stroke}%
\end{pgfscope}%
\begin{pgfscope}%
\pgfpathrectangle{\pgfqpoint{0.150000in}{0.150000in}}{\pgfqpoint{5.700000in}{2.200000in}}%
\pgfusepath{clip}%
\pgfsetbuttcap%
\pgfsetroundjoin%
\pgfsetlinewidth{1.505625pt}%
\definecolor{currentstroke}{rgb}{0.000000,0.000000,0.000000}%
\pgfsetstrokecolor{currentstroke}%
\pgfsetdash{{3.000000pt}{3.000000pt}}{0.000000pt}%
\pgfpathmoveto{\pgfqpoint{0.150000in}{0.810914in}}%
\pgfpathlineto{\pgfqpoint{0.229960in}{0.827485in}}%
\pgfpathlineto{\pgfqpoint{0.298497in}{0.844453in}}%
\pgfpathlineto{\pgfqpoint{0.367034in}{0.864529in}}%
\pgfpathlineto{\pgfqpoint{0.424148in}{0.884079in}}%
\pgfpathlineto{\pgfqpoint{0.481263in}{0.906614in}}%
\pgfpathlineto{\pgfqpoint{0.538377in}{0.932543in}}%
\pgfpathlineto{\pgfqpoint{0.595491in}{0.962245in}}%
\pgfpathlineto{\pgfqpoint{0.652605in}{0.995983in}}%
\pgfpathlineto{\pgfqpoint{0.709719in}{1.033771in}}%
\pgfpathlineto{\pgfqpoint{0.766834in}{1.075180in}}%
\pgfpathlineto{\pgfqpoint{0.961022in}{1.219987in}}%
\pgfpathlineto{\pgfqpoint{0.995291in}{1.239964in}}%
\pgfpathlineto{\pgfqpoint{1.029559in}{1.256161in}}%
\pgfpathlineto{\pgfqpoint{1.063828in}{1.267922in}}%
\pgfpathlineto{\pgfqpoint{1.086673in}{1.273094in}}%
\pgfpathlineto{\pgfqpoint{1.109519in}{1.276082in}}%
\pgfpathlineto{\pgfqpoint{1.132365in}{1.276928in}}%
\pgfpathlineto{\pgfqpoint{1.166633in}{1.274445in}}%
\pgfpathlineto{\pgfqpoint{1.200902in}{1.268079in}}%
\pgfpathlineto{\pgfqpoint{1.246593in}{1.255202in}}%
\pgfpathlineto{\pgfqpoint{1.383667in}{1.211971in}}%
\pgfpathlineto{\pgfqpoint{1.417936in}{1.205535in}}%
\pgfpathlineto{\pgfqpoint{1.452204in}{1.202364in}}%
\pgfpathlineto{\pgfqpoint{1.486473in}{1.202887in}}%
\pgfpathlineto{\pgfqpoint{1.520741in}{1.207337in}}%
\pgfpathlineto{\pgfqpoint{1.555010in}{1.215766in}}%
\pgfpathlineto{\pgfqpoint{1.589279in}{1.228055in}}%
\pgfpathlineto{\pgfqpoint{1.623547in}{1.243920in}}%
\pgfpathlineto{\pgfqpoint{1.657816in}{1.262929in}}%
\pgfpathlineto{\pgfqpoint{1.703507in}{1.292175in}}%
\pgfpathlineto{\pgfqpoint{1.772044in}{1.340944in}}%
\pgfpathlineto{\pgfqpoint{1.852004in}{1.397319in}}%
\pgfpathlineto{\pgfqpoint{1.897695in}{1.425274in}}%
\pgfpathlineto{\pgfqpoint{1.931964in}{1.442737in}}%
\pgfpathlineto{\pgfqpoint{1.966232in}{1.456438in}}%
\pgfpathlineto{\pgfqpoint{2.000501in}{1.465831in}}%
\pgfpathlineto{\pgfqpoint{2.023347in}{1.469500in}}%
\pgfpathlineto{\pgfqpoint{2.046192in}{1.471006in}}%
\pgfpathlineto{\pgfqpoint{2.069038in}{1.470309in}}%
\pgfpathlineto{\pgfqpoint{2.091884in}{1.467409in}}%
\pgfpathlineto{\pgfqpoint{2.114729in}{1.462340in}}%
\pgfpathlineto{\pgfqpoint{2.148998in}{1.450826in}}%
\pgfpathlineto{\pgfqpoint{2.183267in}{1.434958in}}%
\pgfpathlineto{\pgfqpoint{2.217535in}{1.415246in}}%
\pgfpathlineto{\pgfqpoint{2.251804in}{1.392314in}}%
\pgfpathlineto{\pgfqpoint{2.297495in}{1.357965in}}%
\pgfpathlineto{\pgfqpoint{2.377455in}{1.292569in}}%
\pgfpathlineto{\pgfqpoint{2.457415in}{1.228451in}}%
\pgfpathlineto{\pgfqpoint{2.503106in}{1.195421in}}%
\pgfpathlineto{\pgfqpoint{2.548798in}{1.166345in}}%
\pgfpathlineto{\pgfqpoint{2.583066in}{1.147540in}}%
\pgfpathlineto{\pgfqpoint{2.617335in}{1.131490in}}%
\pgfpathlineto{\pgfqpoint{2.651603in}{1.118251in}}%
\pgfpathlineto{\pgfqpoint{2.685872in}{1.107793in}}%
\pgfpathlineto{\pgfqpoint{2.720140in}{1.100017in}}%
\pgfpathlineto{\pgfqpoint{2.754409in}{1.094765in}}%
\pgfpathlineto{\pgfqpoint{2.800100in}{1.091343in}}%
\pgfpathlineto{\pgfqpoint{2.845792in}{1.091512in}}%
\pgfpathlineto{\pgfqpoint{2.891483in}{1.094698in}}%
\pgfpathlineto{\pgfqpoint{2.948597in}{1.102058in}}%
\pgfpathlineto{\pgfqpoint{3.017134in}{1.114505in}}%
\pgfpathlineto{\pgfqpoint{3.108517in}{1.134817in}}%
\pgfpathlineto{\pgfqpoint{3.439780in}{1.211941in}}%
\pgfpathlineto{\pgfqpoint{3.542585in}{1.231060in}}%
\pgfpathlineto{\pgfqpoint{3.633968in}{1.244863in}}%
\pgfpathlineto{\pgfqpoint{3.725351in}{1.255270in}}%
\pgfpathlineto{\pgfqpoint{3.805311in}{1.261370in}}%
\pgfpathlineto{\pgfqpoint{3.885271in}{1.264524in}}%
\pgfpathlineto{\pgfqpoint{3.965230in}{1.264655in}}%
\pgfpathlineto{\pgfqpoint{4.045190in}{1.261739in}}%
\pgfpathlineto{\pgfqpoint{4.125150in}{1.255806in}}%
\pgfpathlineto{\pgfqpoint{4.205110in}{1.246938in}}%
\pgfpathlineto{\pgfqpoint{4.285070in}{1.235268in}}%
\pgfpathlineto{\pgfqpoint{4.376453in}{1.218729in}}%
\pgfpathlineto{\pgfqpoint{4.467836in}{1.199095in}}%
\pgfpathlineto{\pgfqpoint{4.570641in}{1.173793in}}%
\pgfpathlineto{\pgfqpoint{4.684870in}{1.142416in}}%
\pgfpathlineto{\pgfqpoint{4.821944in}{1.101476in}}%
\pgfpathlineto{\pgfqpoint{5.050401in}{1.029416in}}%
\pgfpathlineto{\pgfqpoint{5.278858in}{0.958471in}}%
\pgfpathlineto{\pgfqpoint{5.427355in}{0.915597in}}%
\pgfpathlineto{\pgfqpoint{5.553006in}{0.882283in}}%
\pgfpathlineto{\pgfqpoint{5.678657in}{0.852131in}}%
\pgfpathlineto{\pgfqpoint{5.804309in}{0.825366in}}%
\pgfpathlineto{\pgfqpoint{5.850000in}{0.816489in}}%
\pgfpathlineto{\pgfqpoint{5.850000in}{0.816489in}}%
\pgfusepath{stroke}%
\end{pgfscope}%
\begin{pgfscope}%
\pgfpathrectangle{\pgfqpoint{0.150000in}{0.150000in}}{\pgfqpoint{5.700000in}{2.200000in}}%
\pgfusepath{clip}%
\pgfsetbuttcap%
\pgfsetroundjoin%
\definecolor{currentfill}{rgb}{1.000000,0.000000,0.000000}%
\pgfsetfillcolor{currentfill}%
\pgfsetlinewidth{1.003750pt}%
\definecolor{currentstroke}{rgb}{1.000000,0.000000,0.000000}%
\pgfsetstrokecolor{currentstroke}%
\pgfsetdash{}{0pt}%
\pgfsys@defobject{currentmarker}{\pgfqpoint{-0.041667in}{-0.041667in}}{\pgfqpoint{0.041667in}{0.041667in}}{%
\pgfpathmoveto{\pgfqpoint{0.000000in}{-0.041667in}}%
\pgfpathcurveto{\pgfqpoint{0.011050in}{-0.041667in}}{\pgfqpoint{0.021649in}{-0.037276in}}{\pgfqpoint{0.029463in}{-0.029463in}}%
\pgfpathcurveto{\pgfqpoint{0.037276in}{-0.021649in}}{\pgfqpoint{0.041667in}{-0.011050in}}{\pgfqpoint{0.041667in}{0.000000in}}%
\pgfpathcurveto{\pgfqpoint{0.041667in}{0.011050in}}{\pgfqpoint{0.037276in}{0.021649in}}{\pgfqpoint{0.029463in}{0.029463in}}%
\pgfpathcurveto{\pgfqpoint{0.021649in}{0.037276in}}{\pgfqpoint{0.011050in}{0.041667in}}{\pgfqpoint{0.000000in}{0.041667in}}%
\pgfpathcurveto{\pgfqpoint{-0.011050in}{0.041667in}}{\pgfqpoint{-0.021649in}{0.037276in}}{\pgfqpoint{-0.029463in}{0.029463in}}%
\pgfpathcurveto{\pgfqpoint{-0.037276in}{0.021649in}}{\pgfqpoint{-0.041667in}{0.011050in}}{\pgfqpoint{-0.041667in}{0.000000in}}%
\pgfpathcurveto{\pgfqpoint{-0.041667in}{-0.011050in}}{\pgfqpoint{-0.037276in}{-0.021649in}}{\pgfqpoint{-0.029463in}{-0.029463in}}%
\pgfpathcurveto{\pgfqpoint{-0.021649in}{-0.037276in}}{\pgfqpoint{-0.011050in}{-0.041667in}}{\pgfqpoint{0.000000in}{-0.041667in}}%
\pgfpathclose%
\pgfusepath{stroke,fill}%
}%
\begin{pgfscope}%
\pgfsys@transformshift{1.909118in}{1.431474in}%
\pgfsys@useobject{currentmarker}{}%
\end{pgfscope}%
\end{pgfscope}%
\begin{pgfscope}%
\pgfpathrectangle{\pgfqpoint{0.150000in}{0.150000in}}{\pgfqpoint{5.700000in}{2.200000in}}%
\pgfusepath{clip}%
\pgfsetbuttcap%
\pgfsetmiterjoin%
\definecolor{currentfill}{rgb}{1.000000,0.000000,0.000000}%
\pgfsetfillcolor{currentfill}%
\pgfsetlinewidth{1.003750pt}%
\definecolor{currentstroke}{rgb}{1.000000,0.000000,0.000000}%
\pgfsetstrokecolor{currentstroke}%
\pgfsetdash{}{0pt}%
\pgfsys@defobject{currentmarker}{\pgfqpoint{-0.041667in}{-0.041667in}}{\pgfqpoint{0.041667in}{0.041667in}}{%
\pgfpathmoveto{\pgfqpoint{-0.000000in}{-0.041667in}}%
\pgfpathlineto{\pgfqpoint{0.041667in}{0.041667in}}%
\pgfpathlineto{\pgfqpoint{-0.041667in}{0.041667in}}%
\pgfpathclose%
\pgfusepath{stroke,fill}%
}%
\begin{pgfscope}%
\pgfsys@transformshift{2.651603in}{0.543539in}%
\pgfsys@useobject{currentmarker}{}%
\end{pgfscope}%
\end{pgfscope}%
\begin{pgfscope}%
\pgfsetrectcap%
\pgfsetmiterjoin%
\pgfsetlinewidth{0.000000pt}%
\definecolor{currentstroke}{rgb}{1.000000,1.000000,1.000000}%
\pgfsetstrokecolor{currentstroke}%
\pgfsetdash{}{0pt}%
\pgfpathmoveto{\pgfqpoint{0.150000in}{0.150000in}}%
\pgfpathlineto{\pgfqpoint{0.150000in}{2.350000in}}%
\pgfusepath{}%
\end{pgfscope}%
\begin{pgfscope}%
\pgfsetrectcap%
\pgfsetmiterjoin%
\pgfsetlinewidth{0.000000pt}%
\definecolor{currentstroke}{rgb}{1.000000,1.000000,1.000000}%
\pgfsetstrokecolor{currentstroke}%
\pgfsetdash{}{0pt}%
\pgfpathmoveto{\pgfqpoint{5.850000in}{0.150000in}}%
\pgfpathlineto{\pgfqpoint{5.850000in}{2.350000in}}%
\pgfusepath{}%
\end{pgfscope}%
\begin{pgfscope}%
\pgfsetrectcap%
\pgfsetmiterjoin%
\pgfsetlinewidth{0.000000pt}%
\definecolor{currentstroke}{rgb}{1.000000,1.000000,1.000000}%
\pgfsetstrokecolor{currentstroke}%
\pgfsetdash{}{0pt}%
\pgfpathmoveto{\pgfqpoint{0.150000in}{0.150000in}}%
\pgfpathlineto{\pgfqpoint{5.850000in}{0.150000in}}%
\pgfusepath{}%
\end{pgfscope}%
\begin{pgfscope}%
\pgfsetrectcap%
\pgfsetmiterjoin%
\pgfsetlinewidth{0.000000pt}%
\definecolor{currentstroke}{rgb}{1.000000,1.000000,1.000000}%
\pgfsetstrokecolor{currentstroke}%
\pgfsetdash{}{0pt}%
\pgfpathmoveto{\pgfqpoint{0.150000in}{2.350000in}}%
\pgfpathlineto{\pgfqpoint{5.850000in}{2.350000in}}%
\pgfusepath{}%
\end{pgfscope}%
\begin{pgfscope}%
\pgfsetroundcap%
\pgfsetroundjoin%
\pgfsetlinewidth{2.007500pt}%
\definecolor{currentstroke}{rgb}{0.203922,0.396078,0.643137}%
\pgfsetstrokecolor{currentstroke}%
\pgfsetdash{}{0pt}%
\pgfpathmoveto{\pgfqpoint{4.254590in}{2.204470in}}%
\pgfpathlineto{\pgfqpoint{4.476812in}{2.204470in}}%
\pgfusepath{stroke}%
\end{pgfscope}%
\begin{pgfscope}%
\definecolor{textcolor}{rgb}{0.150000,0.150000,0.150000}%
\pgfsetstrokecolor{textcolor}%
\pgfsetfillcolor{textcolor}%
\pgftext[x=4.565701in,y=2.165582in,left,base]{\color{textcolor}\rmfamily\fontsize{8.000000}{9.600000}\selectfont Mean}%
\end{pgfscope}%
\begin{pgfscope}%
\pgfsetroundcap%
\pgfsetroundjoin%
\pgfsetlinewidth{1.505625pt}%
\definecolor{currentstroke}{rgb}{0.000000,0.501961,0.000000}%
\pgfsetstrokecolor{currentstroke}%
\pgfsetdash{}{0pt}%
\pgfpathmoveto{\pgfqpoint{4.254590in}{2.041385in}}%
\pgfpathlineto{\pgfqpoint{4.476812in}{2.041385in}}%
\pgfusepath{stroke}%
\end{pgfscope}%
\begin{pgfscope}%
\definecolor{textcolor}{rgb}{0.150000,0.150000,0.150000}%
\pgfsetstrokecolor{textcolor}%
\pgfsetfillcolor{textcolor}%
\pgftext[x=4.565701in,y=2.002496in,left,base]{\color{textcolor}\rmfamily\fontsize{8.000000}{9.600000}\selectfont Acquisition function}%
\end{pgfscope}%
\begin{pgfscope}%
\pgfsetbuttcap%
\pgfsetroundjoin%
\pgfsetlinewidth{1.505625pt}%
\definecolor{currentstroke}{rgb}{0.000000,0.000000,0.000000}%
\pgfsetstrokecolor{currentstroke}%
\pgfsetdash{{3.000000pt}{3.000000pt}}{0.000000pt}%
\pgfpathmoveto{\pgfqpoint{4.254590in}{1.878299in}}%
\pgfpathlineto{\pgfqpoint{4.476812in}{1.878299in}}%
\pgfusepath{stroke}%
\end{pgfscope}%
\begin{pgfscope}%
\definecolor{textcolor}{rgb}{0.150000,0.150000,0.150000}%
\pgfsetstrokecolor{textcolor}%
\pgfsetfillcolor{textcolor}%
\pgftext[x=4.565701in,y=1.839410in,left,base]{\color{textcolor}\rmfamily\fontsize{8.000000}{9.600000}\selectfont \(\displaystyle \mathrm{f}\)}%
\end{pgfscope}%
\begin{pgfscope}%
\pgfsetbuttcap%
\pgfsetroundjoin%
\definecolor{currentfill}{rgb}{1.000000,0.000000,0.000000}%
\pgfsetfillcolor{currentfill}%
\pgfsetlinewidth{1.003750pt}%
\definecolor{currentstroke}{rgb}{1.000000,0.000000,0.000000}%
\pgfsetstrokecolor{currentstroke}%
\pgfsetdash{}{0pt}%
\pgfsys@defobject{currentmarker}{\pgfqpoint{-0.041667in}{-0.041667in}}{\pgfqpoint{0.041667in}{0.041667in}}{%
\pgfpathmoveto{\pgfqpoint{0.000000in}{-0.041667in}}%
\pgfpathcurveto{\pgfqpoint{0.011050in}{-0.041667in}}{\pgfqpoint{0.021649in}{-0.037276in}}{\pgfqpoint{0.029463in}{-0.029463in}}%
\pgfpathcurveto{\pgfqpoint{0.037276in}{-0.021649in}}{\pgfqpoint{0.041667in}{-0.011050in}}{\pgfqpoint{0.041667in}{0.000000in}}%
\pgfpathcurveto{\pgfqpoint{0.041667in}{0.011050in}}{\pgfqpoint{0.037276in}{0.021649in}}{\pgfqpoint{0.029463in}{0.029463in}}%
\pgfpathcurveto{\pgfqpoint{0.021649in}{0.037276in}}{\pgfqpoint{0.011050in}{0.041667in}}{\pgfqpoint{0.000000in}{0.041667in}}%
\pgfpathcurveto{\pgfqpoint{-0.011050in}{0.041667in}}{\pgfqpoint{-0.021649in}{0.037276in}}{\pgfqpoint{-0.029463in}{0.029463in}}%
\pgfpathcurveto{\pgfqpoint{-0.037276in}{0.021649in}}{\pgfqpoint{-0.041667in}{0.011050in}}{\pgfqpoint{-0.041667in}{0.000000in}}%
\pgfpathcurveto{\pgfqpoint{-0.041667in}{-0.011050in}}{\pgfqpoint{-0.037276in}{-0.021649in}}{\pgfqpoint{-0.029463in}{-0.029463in}}%
\pgfpathcurveto{\pgfqpoint{-0.021649in}{-0.037276in}}{\pgfqpoint{-0.011050in}{-0.041667in}}{\pgfqpoint{0.000000in}{-0.041667in}}%
\pgfpathclose%
\pgfusepath{stroke,fill}%
}%
\begin{pgfscope}%
\pgfsys@transformshift{4.365701in}{1.715213in}%
\pgfsys@useobject{currentmarker}{}%
\end{pgfscope}%
\end{pgfscope}%
\begin{pgfscope}%
\definecolor{textcolor}{rgb}{0.150000,0.150000,0.150000}%
\pgfsetstrokecolor{textcolor}%
\pgfsetfillcolor{textcolor}%
\pgftext[x=4.565701in,y=1.676324in,left,base]{\color{textcolor}\rmfamily\fontsize{8.000000}{9.600000}\selectfont Previous observation}%
\end{pgfscope}%
\begin{pgfscope}%
\pgfsetbuttcap%
\pgfsetmiterjoin%
\definecolor{currentfill}{rgb}{1.000000,0.000000,0.000000}%
\pgfsetfillcolor{currentfill}%
\pgfsetlinewidth{1.003750pt}%
\definecolor{currentstroke}{rgb}{1.000000,0.000000,0.000000}%
\pgfsetstrokecolor{currentstroke}%
\pgfsetdash{}{0pt}%
\pgfsys@defobject{currentmarker}{\pgfqpoint{-0.041667in}{-0.041667in}}{\pgfqpoint{0.041667in}{0.041667in}}{%
\pgfpathmoveto{\pgfqpoint{-0.000000in}{-0.041667in}}%
\pgfpathlineto{\pgfqpoint{0.041667in}{0.041667in}}%
\pgfpathlineto{\pgfqpoint{-0.041667in}{0.041667in}}%
\pgfpathclose%
\pgfusepath{stroke,fill}%
}%
\begin{pgfscope}%
\pgfsys@transformshift{4.365701in}{1.552127in}%
\pgfsys@useobject{currentmarker}{}%
\end{pgfscope}%
\end{pgfscope}%
\begin{pgfscope}%
\definecolor{textcolor}{rgb}{0.150000,0.150000,0.150000}%
\pgfsetstrokecolor{textcolor}%
\pgfsetfillcolor{textcolor}%
\pgftext[x=4.565701in,y=1.513238in,left,base]{\color{textcolor}\rmfamily\fontsize{8.000000}{9.600000}\selectfont Next point to query}%
\end{pgfscope}%
\begin{pgfscope}%
\pgfsetbuttcap%
\pgfsetroundjoin%
\definecolor{currentfill}{rgb}{0.000000,0.000000,0.000000}%
\pgfsetfillcolor{currentfill}%
\pgfsetlinewidth{1.505625pt}%
\definecolor{currentstroke}{rgb}{0.000000,0.000000,0.000000}%
\pgfsetstrokecolor{currentstroke}%
\pgfsetdash{}{0pt}%
\pgfpathmoveto{\pgfqpoint{4.324034in}{1.336079in}}%
\pgfpathlineto{\pgfqpoint{4.407368in}{1.419412in}}%
\pgfpathmoveto{\pgfqpoint{4.324034in}{1.419412in}}%
\pgfpathlineto{\pgfqpoint{4.407368in}{1.336079in}}%
\pgfusepath{stroke,fill}%
\end{pgfscope}%
\begin{pgfscope}%
\definecolor{textcolor}{rgb}{0.150000,0.150000,0.150000}%
\pgfsetstrokecolor{textcolor}%
\pgfsetfillcolor{textcolor}%
\pgftext[x=4.565701in,y=1.348579in,left,base]{\color{textcolor}\rmfamily\fontsize{8.000000}{9.600000}\selectfont Data}%
\end{pgfscope}%
\begin{pgfscope}%
\pgfsetbuttcap%
\pgfsetmiterjoin%
\definecolor{currentfill}{rgb}{0.447059,0.623529,0.811765}%
\pgfsetfillcolor{currentfill}%
\pgfsetfillopacity{0.200000}%
\pgfsetlinewidth{0.501875pt}%
\definecolor{currentstroke}{rgb}{0.125490,0.290196,0.529412}%
\pgfsetstrokecolor{currentstroke}%
\pgfsetstrokeopacity{0.200000}%
\pgfsetdash{}{0pt}%
\pgfpathmoveto{\pgfqpoint{4.254590in}{1.185493in}}%
\pgfpathlineto{\pgfqpoint{4.476812in}{1.185493in}}%
\pgfpathlineto{\pgfqpoint{4.476812in}{1.263271in}}%
\pgfpathlineto{\pgfqpoint{4.254590in}{1.263271in}}%
\pgfpathclose%
\pgfusepath{stroke,fill}%
\end{pgfscope}%
\begin{pgfscope}%
\definecolor{textcolor}{rgb}{0.150000,0.150000,0.150000}%
\pgfsetstrokecolor{textcolor}%
\pgfsetfillcolor{textcolor}%
\pgftext[x=4.565701in,y=1.185493in,left,base]{\color{textcolor}\rmfamily\fontsize{8.000000}{9.600000}\selectfont Confidence}%
\end{pgfscope}%
\end{pgfpicture}%
\makeatother%
\endgroup%

            %\includegraphics[height=0.7\textwidth]{fig/bo.pdf}
            \caption{
                Two steps of Bayesian Optimization in which there are initially two observation.
                At every step the posterior distribution is derived and the acquisition function is maximized to determine the next point to query.
                Exactly how the acquisition function is connected to the mean and variance is covered \cref{sec:acq}.}
            \label{fig:bo}
        \end{figure*}

    \subsection{Gaussian Processes}\label{sec:gp}
        % \item Use a Gaussian Process (GP) to model $p_\mathcal{M}(f|\bm{\lambda})$. Under GP prior acquisition functions depends only on predictive mean function and predictive variance function.

        % \item Assuming at every time step $t$, gaussian noise $\epsilon_t \sim \mathcal{N}(0,\sigma^2)$ on function evaluations $y_t = f(\bm{\lambda}_t) + \epsilon_t$. Denote observations $\mathcal{D}_t = \set{\bm{\lambda}_\tau, y_\tau}_\tau^t$,

        % % say: Assuming $f$ is sampled from GP
        % % noisy observation
        % % fix observations

        %     % \begin{equation}\mu(\bm{\lambda}; \set{\bm{\lambda}_n, f(\bm{\lambda}_n)}, \bm{\Theta})\end{equation}
        %     % \begin{equation}\sigma^2(\bm{\lambda}; \set{\bm{\lambda}_n, f(\bm{\lambda}_n)}, \bm{\Theta}),\end{equation}
        %     % TODO: specify
        %     \begin{align}
        %         \mu_t(\bm{\lambda}) &= \bm{k}_t(\bm{\lambda})^\top (\bm{K}_t + \sigma^2 \bm{I})^{-1} \bm{y}_t \\
        %         \sigma^2_t(\bm{\lambda}) &= \bm{k}(\bm{\lambda}, \bm{\lambda}) - \bm{k}_t(\bm{\lambda})^\top (\bm{K}_t + \sigma^2 \bm{I})^{-1} \bm{k}_t(\bm{\lambda}),
        %     \end{align}
                
        % where $\bm{K}_t = [\bm{k}(\bm{\lambda}_i,\bm{\lambda}_j)]_{\set{\bm{\lambda}_i,\bm{\lambda}_j\in \mathcal{D}_t}}$ and $\bm{k}_t(\bm{\lambda}) = [\bm{k}(\bm{\lambda}_i, \bm{\lambda})]_{\set{\bm{\lambda}_i \in \mathcal{D}_t}}$

        % ... we can frame the problem in terms of nonlinear regression
        What we have done by treating our belief about $f$ as a distribution over function and applying Bayes rule to obtain the posterior is essentially what is called \emph{nonlinear Bayesian regression}.

        A GP is one possible prior over functions $P(f)$.
        % TODO: ... with useful properties such as..

        \begin{definition}{A Gaussian Process}
            is a collection of random variables, any finite number of which have a joint Gaussian distribution \parencite{rasmussen_gaussian_2006}.
        \end{definition}
            
        This can be viewed as an extension of the multivariate gaussian distribution to infinite dimensions.
        The GP is a distribution over functions completely specified by its mean function $m$ and covariance function $k$ similarly to how a gaussian distribution is a distribution over a random variable fully specified by its mean and covariance.
        We write it,
            \begin{equation}
                f(\bm{x}) \sim \mathcal{GP}(\mu(\bm{x}), k(\bm{x}, \bm{x}')).
            \end{equation}
            % TODO: what is x'?

        To understand how it is a distribution over functions consider the GP to be defined on the continuous space $\mathbb{R}^D$ (consequently the GP assigns a random variable for every index $\bm{x}$ in the infinite index set $\set{\bm{x} \in \mathbb{R}^D}$).
        Then $f(\bm{x})$ is the random variable for every location $\bm{x} \in \mathbb{R}^D$.
        Thus sampling from this generative model gives you one possible function over the domain $\mathbb{R}^D$.
        % TODO: mention that non-infinite set

        %We will in the following chapter assume zero mean for simplicity of notation.
        % TODO: why can we do this?

        \subsubsection{Covariance Function}\label{sec:covar}

            % The kernel trick allows us to specify an intuitive similarity between pairs of points, rather than a feature map %, which in practice can be hard to define (human loop)

            The covariance function $k(\bm{x},\bm{x}')$ (also known as the \emph{kernel}) captures the correlation between points in the process and thus determines the structure of possible functions.
            Notice, however, that it specifies the similarly between the \emph{function evaluation} of two points even though it is determined by the input.
            A common choice of kernel is the \emph{Squared Exponential} (SE),
            \begin{equation}
                k(\bm{x},\bm{x}') = \sigma_0^2 \exp (-\frac{1}{2 l^2} |\bm{x} - \bm{x}'| ^ 2).
            \end{equation}

            It has two so-called \emph{hyperparameters}, the output variance $\sigma_0^2$ and $l$ referred to as the \emph{length scale}.
            The parameter $\sigma_0^2$ is simply a scale factor that controls how far away from the function mean value varies, and is thus neglectable considering normalization of the output. % TODO: diagonal is sigma^2
            More interesting is the length scale.
            The effect on sampled functions from a GP based on the SE kernel using different length scales is apparent in \cref{fig:rbflengthscale}.
            Intuitively, the length scale roughly corresponds to the distance necessary to move for a change to occur in the function value.
            Thus samples from gaussian process prior with $l=0.02$ fluctuates the most.
            Obviously, these hyperparameters have a big influence on how well we model the unknown function $f$.
            In \cref{sec:choosehyper} we will cover various methods on how to learn these directly from data.
            
            The kernel has a few characteristics.
            
            First, notice how smoothly the sampled function varies in \cref{fig:rbflengthscale}.
            This is a consequence of every prior of SE being infinitely differentiable. 
            This is an important property since it becomes problematic when we want to model variables that are originally discrete (see \cref{sec:discussion}).

            Secondly, if we disregard the output variance, the covariance between two points specified by the above kernel is almost 1 for very close inputs but decreases as the distance grows effectively capturing that inputs close to each other have high influence on each other.
            This assumption is essential in allowing us to estimate which areas have potential in e.g. hyperparameter optimization.

            Lastly, it should be noted that SE belongs to a whole class of kernels called \emph{stationary kernels} since they are translationally invariant.
            This becomes apparent by observing that the kernel can be written as a function of the difference $|\bm{x}-\bm{x}'|$.
            We will return to the importance of this in \cref{sec:discussion} as well.

            Now, returning to the visualisation, the samples in \cref{fig:rbflengthscale} where drawn using the \emph{covariance matrix} which we denote by $\bm{K}$.
            Given a set of points $X$, $\bm{K}$ can be constructed from the covariance function by element-wise application for all pairs of points,
            \begin{equation}
                \bm{K} = \begin{bmatrix}
                k(\bm{x}_1, \bm{x}_1) & \cdots & k(\bm{x}_1, \bm{x}_t) \\
                \vdots & \ddots & \vdots \\
                k(\bm{x}_t, \bm{x}_1) & \cdots & k(\bm{x}_t, \bm{x}_t) \\
            \end{bmatrix}
            \end{equation}
            
            This notation will be useful when we discuss \emph{GP regression} in \cref{sec:gpreg}.

            The literature on covariance function is rich including isotropic, non-stationary (such as periodic) and ways to combine kernels \parencite{duvenaud_automatic_2014}.
            For a thorough overview we refer to \parencite[ch. 5]{rasmussen_gaussian_2006}.
            % - positive semidefinite? ($v^TKv \geq 0$) => necessary for consistancy

            \begin{figure*}[t]
            \centering
            \begin{subfigure}[t]{0.31\textwidth}
                \centering
                %% Creator: Matplotlib, PGF backend
%%
%% To include the figure in your LaTeX document, write
%%   \input{<filename>.pgf}
%%
%% Make sure the required packages are loaded in your preamble
%%   \usepackage{pgf}
%%
%% Figures using additional raster images can only be included by \input if
%% they are in the same directory as the main LaTeX file. For loading figures
%% from other directories you can use the `import` package
%%   \usepackage{import}
%% and then include the figures with
%%   \import{<path to file>}{<filename>.pgf}
%%
%% Matplotlib used the following preamble
%%   \usepackage{gensymb}
%%   \usepackage{fontspec}
%%   \setmainfont{DejaVu Serif}
%%   \setsansfont{Arial}
%%   \setmonofont{DejaVu Sans Mono}
%%
\begingroup%
\makeatletter%
\begin{pgfpicture}%
\pgfpathrectangle{\pgfpointorigin}{\pgfqpoint{2.200000in}{3.000000in}}%
\pgfusepath{use as bounding box, clip}%
\begin{pgfscope}%
\pgfsetbuttcap%
\pgfsetmiterjoin%
\definecolor{currentfill}{rgb}{1.000000,1.000000,1.000000}%
\pgfsetfillcolor{currentfill}%
\pgfsetlinewidth{0.000000pt}%
\definecolor{currentstroke}{rgb}{1.000000,1.000000,1.000000}%
\pgfsetstrokecolor{currentstroke}%
\pgfsetdash{}{0pt}%
\pgfpathmoveto{\pgfqpoint{0.000000in}{0.000000in}}%
\pgfpathlineto{\pgfqpoint{2.200000in}{0.000000in}}%
\pgfpathlineto{\pgfqpoint{2.200000in}{3.000000in}}%
\pgfpathlineto{\pgfqpoint{0.000000in}{3.000000in}}%
\pgfpathclose%
\pgfusepath{fill}%
\end{pgfscope}%
\begin{pgfscope}%
\pgfsetbuttcap%
\pgfsetmiterjoin%
\definecolor{currentfill}{rgb}{0.917647,0.917647,0.949020}%
\pgfsetfillcolor{currentfill}%
\pgfsetlinewidth{0.000000pt}%
\definecolor{currentstroke}{rgb}{0.000000,0.000000,0.000000}%
\pgfsetstrokecolor{currentstroke}%
\pgfsetstrokeopacity{0.000000}%
\pgfsetdash{}{0pt}%
\pgfpathmoveto{\pgfqpoint{0.275000in}{0.375000in}}%
\pgfpathlineto{\pgfqpoint{1.980000in}{0.375000in}}%
\pgfpathlineto{\pgfqpoint{1.980000in}{2.640000in}}%
\pgfpathlineto{\pgfqpoint{0.275000in}{2.640000in}}%
\pgfpathclose%
\pgfusepath{fill}%
\end{pgfscope}%
\begin{pgfscope}%
\pgfpathrectangle{\pgfqpoint{0.275000in}{0.375000in}}{\pgfqpoint{1.705000in}{2.265000in}}%
\pgfusepath{clip}%
\pgfsetroundcap%
\pgfsetroundjoin%
\pgfsetlinewidth{0.803000pt}%
\definecolor{currentstroke}{rgb}{1.000000,1.000000,1.000000}%
\pgfsetstrokecolor{currentstroke}%
\pgfsetdash{}{0pt}%
\pgfpathmoveto{\pgfqpoint{0.352500in}{0.375000in}}%
\pgfpathlineto{\pgfqpoint{0.352500in}{2.640000in}}%
\pgfusepath{stroke}%
\end{pgfscope}%
\begin{pgfscope}%
\definecolor{textcolor}{rgb}{0.150000,0.150000,0.150000}%
\pgfsetstrokecolor{textcolor}%
\pgfsetfillcolor{textcolor}%
\pgftext[x=0.352500in,y=0.326389in,,top]{\color{textcolor}\rmfamily\fontsize{8.000000}{9.600000}\selectfont \(\displaystyle 0.00\)}%
\end{pgfscope}%
\begin{pgfscope}%
\pgfpathrectangle{\pgfqpoint{0.275000in}{0.375000in}}{\pgfqpoint{1.705000in}{2.265000in}}%
\pgfusepath{clip}%
\pgfsetroundcap%
\pgfsetroundjoin%
\pgfsetlinewidth{0.803000pt}%
\definecolor{currentstroke}{rgb}{1.000000,1.000000,1.000000}%
\pgfsetstrokecolor{currentstroke}%
\pgfsetdash{}{0pt}%
\pgfpathmoveto{\pgfqpoint{0.740000in}{0.375000in}}%
\pgfpathlineto{\pgfqpoint{0.740000in}{2.640000in}}%
\pgfusepath{stroke}%
\end{pgfscope}%
\begin{pgfscope}%
\definecolor{textcolor}{rgb}{0.150000,0.150000,0.150000}%
\pgfsetstrokecolor{textcolor}%
\pgfsetfillcolor{textcolor}%
\pgftext[x=0.740000in,y=0.326389in,,top]{\color{textcolor}\rmfamily\fontsize{8.000000}{9.600000}\selectfont \(\displaystyle 0.25\)}%
\end{pgfscope}%
\begin{pgfscope}%
\pgfpathrectangle{\pgfqpoint{0.275000in}{0.375000in}}{\pgfqpoint{1.705000in}{2.265000in}}%
\pgfusepath{clip}%
\pgfsetroundcap%
\pgfsetroundjoin%
\pgfsetlinewidth{0.803000pt}%
\definecolor{currentstroke}{rgb}{1.000000,1.000000,1.000000}%
\pgfsetstrokecolor{currentstroke}%
\pgfsetdash{}{0pt}%
\pgfpathmoveto{\pgfqpoint{1.127500in}{0.375000in}}%
\pgfpathlineto{\pgfqpoint{1.127500in}{2.640000in}}%
\pgfusepath{stroke}%
\end{pgfscope}%
\begin{pgfscope}%
\definecolor{textcolor}{rgb}{0.150000,0.150000,0.150000}%
\pgfsetstrokecolor{textcolor}%
\pgfsetfillcolor{textcolor}%
\pgftext[x=1.127500in,y=0.326389in,,top]{\color{textcolor}\rmfamily\fontsize{8.000000}{9.600000}\selectfont \(\displaystyle 0.50\)}%
\end{pgfscope}%
\begin{pgfscope}%
\pgfpathrectangle{\pgfqpoint{0.275000in}{0.375000in}}{\pgfqpoint{1.705000in}{2.265000in}}%
\pgfusepath{clip}%
\pgfsetroundcap%
\pgfsetroundjoin%
\pgfsetlinewidth{0.803000pt}%
\definecolor{currentstroke}{rgb}{1.000000,1.000000,1.000000}%
\pgfsetstrokecolor{currentstroke}%
\pgfsetdash{}{0pt}%
\pgfpathmoveto{\pgfqpoint{1.515000in}{0.375000in}}%
\pgfpathlineto{\pgfqpoint{1.515000in}{2.640000in}}%
\pgfusepath{stroke}%
\end{pgfscope}%
\begin{pgfscope}%
\definecolor{textcolor}{rgb}{0.150000,0.150000,0.150000}%
\pgfsetstrokecolor{textcolor}%
\pgfsetfillcolor{textcolor}%
\pgftext[x=1.515000in,y=0.326389in,,top]{\color{textcolor}\rmfamily\fontsize{8.000000}{9.600000}\selectfont \(\displaystyle 0.75\)}%
\end{pgfscope}%
\begin{pgfscope}%
\pgfpathrectangle{\pgfqpoint{0.275000in}{0.375000in}}{\pgfqpoint{1.705000in}{2.265000in}}%
\pgfusepath{clip}%
\pgfsetroundcap%
\pgfsetroundjoin%
\pgfsetlinewidth{0.803000pt}%
\definecolor{currentstroke}{rgb}{1.000000,1.000000,1.000000}%
\pgfsetstrokecolor{currentstroke}%
\pgfsetdash{}{0pt}%
\pgfpathmoveto{\pgfqpoint{1.902500in}{0.375000in}}%
\pgfpathlineto{\pgfqpoint{1.902500in}{2.640000in}}%
\pgfusepath{stroke}%
\end{pgfscope}%
\begin{pgfscope}%
\definecolor{textcolor}{rgb}{0.150000,0.150000,0.150000}%
\pgfsetstrokecolor{textcolor}%
\pgfsetfillcolor{textcolor}%
\pgftext[x=1.902500in,y=0.326389in,,top]{\color{textcolor}\rmfamily\fontsize{8.000000}{9.600000}\selectfont \(\displaystyle 1.00\)}%
\end{pgfscope}%
\begin{pgfscope}%
\pgfpathrectangle{\pgfqpoint{0.275000in}{0.375000in}}{\pgfqpoint{1.705000in}{2.265000in}}%
\pgfusepath{clip}%
\pgfsetroundcap%
\pgfsetroundjoin%
\pgfsetlinewidth{0.803000pt}%
\definecolor{currentstroke}{rgb}{1.000000,1.000000,1.000000}%
\pgfsetstrokecolor{currentstroke}%
\pgfsetdash{}{0pt}%
\pgfpathmoveto{\pgfqpoint{0.275000in}{0.441090in}}%
\pgfpathlineto{\pgfqpoint{1.980000in}{0.441090in}}%
\pgfusepath{stroke}%
\end{pgfscope}%
\begin{pgfscope}%
\definecolor{textcolor}{rgb}{0.150000,0.150000,0.150000}%
\pgfsetstrokecolor{textcolor}%
\pgfsetfillcolor{textcolor}%
\pgftext[x=0.075538in,y=0.398881in,left,base]{\color{textcolor}\rmfamily\fontsize{8.000000}{9.600000}\selectfont \(\displaystyle -3\)}%
\end{pgfscope}%
\begin{pgfscope}%
\pgfpathrectangle{\pgfqpoint{0.275000in}{0.375000in}}{\pgfqpoint{1.705000in}{2.265000in}}%
\pgfusepath{clip}%
\pgfsetroundcap%
\pgfsetroundjoin%
\pgfsetlinewidth{0.803000pt}%
\definecolor{currentstroke}{rgb}{1.000000,1.000000,1.000000}%
\pgfsetstrokecolor{currentstroke}%
\pgfsetdash{}{0pt}%
\pgfpathmoveto{\pgfqpoint{0.275000in}{0.833999in}}%
\pgfpathlineto{\pgfqpoint{1.980000in}{0.833999in}}%
\pgfusepath{stroke}%
\end{pgfscope}%
\begin{pgfscope}%
\definecolor{textcolor}{rgb}{0.150000,0.150000,0.150000}%
\pgfsetstrokecolor{textcolor}%
\pgfsetfillcolor{textcolor}%
\pgftext[x=0.075538in,y=0.791789in,left,base]{\color{textcolor}\rmfamily\fontsize{8.000000}{9.600000}\selectfont \(\displaystyle -2\)}%
\end{pgfscope}%
\begin{pgfscope}%
\pgfpathrectangle{\pgfqpoint{0.275000in}{0.375000in}}{\pgfqpoint{1.705000in}{2.265000in}}%
\pgfusepath{clip}%
\pgfsetroundcap%
\pgfsetroundjoin%
\pgfsetlinewidth{0.803000pt}%
\definecolor{currentstroke}{rgb}{1.000000,1.000000,1.000000}%
\pgfsetstrokecolor{currentstroke}%
\pgfsetdash{}{0pt}%
\pgfpathmoveto{\pgfqpoint{0.275000in}{1.226907in}}%
\pgfpathlineto{\pgfqpoint{1.980000in}{1.226907in}}%
\pgfusepath{stroke}%
\end{pgfscope}%
\begin{pgfscope}%
\definecolor{textcolor}{rgb}{0.150000,0.150000,0.150000}%
\pgfsetstrokecolor{textcolor}%
\pgfsetfillcolor{textcolor}%
\pgftext[x=0.075538in,y=1.184698in,left,base]{\color{textcolor}\rmfamily\fontsize{8.000000}{9.600000}\selectfont \(\displaystyle -1\)}%
\end{pgfscope}%
\begin{pgfscope}%
\pgfpathrectangle{\pgfqpoint{0.275000in}{0.375000in}}{\pgfqpoint{1.705000in}{2.265000in}}%
\pgfusepath{clip}%
\pgfsetroundcap%
\pgfsetroundjoin%
\pgfsetlinewidth{0.803000pt}%
\definecolor{currentstroke}{rgb}{1.000000,1.000000,1.000000}%
\pgfsetstrokecolor{currentstroke}%
\pgfsetdash{}{0pt}%
\pgfpathmoveto{\pgfqpoint{0.275000in}{1.619816in}}%
\pgfpathlineto{\pgfqpoint{1.980000in}{1.619816in}}%
\pgfusepath{stroke}%
\end{pgfscope}%
\begin{pgfscope}%
\definecolor{textcolor}{rgb}{0.150000,0.150000,0.150000}%
\pgfsetstrokecolor{textcolor}%
\pgfsetfillcolor{textcolor}%
\pgftext[x=0.167360in,y=1.577606in,left,base]{\color{textcolor}\rmfamily\fontsize{8.000000}{9.600000}\selectfont \(\displaystyle 0\)}%
\end{pgfscope}%
\begin{pgfscope}%
\pgfpathrectangle{\pgfqpoint{0.275000in}{0.375000in}}{\pgfqpoint{1.705000in}{2.265000in}}%
\pgfusepath{clip}%
\pgfsetroundcap%
\pgfsetroundjoin%
\pgfsetlinewidth{0.803000pt}%
\definecolor{currentstroke}{rgb}{1.000000,1.000000,1.000000}%
\pgfsetstrokecolor{currentstroke}%
\pgfsetdash{}{0pt}%
\pgfpathmoveto{\pgfqpoint{0.275000in}{2.012724in}}%
\pgfpathlineto{\pgfqpoint{1.980000in}{2.012724in}}%
\pgfusepath{stroke}%
\end{pgfscope}%
\begin{pgfscope}%
\definecolor{textcolor}{rgb}{0.150000,0.150000,0.150000}%
\pgfsetstrokecolor{textcolor}%
\pgfsetfillcolor{textcolor}%
\pgftext[x=0.167360in,y=1.970515in,left,base]{\color{textcolor}\rmfamily\fontsize{8.000000}{9.600000}\selectfont \(\displaystyle 1\)}%
\end{pgfscope}%
\begin{pgfscope}%
\pgfpathrectangle{\pgfqpoint{0.275000in}{0.375000in}}{\pgfqpoint{1.705000in}{2.265000in}}%
\pgfusepath{clip}%
\pgfsetroundcap%
\pgfsetroundjoin%
\pgfsetlinewidth{0.803000pt}%
\definecolor{currentstroke}{rgb}{1.000000,1.000000,1.000000}%
\pgfsetstrokecolor{currentstroke}%
\pgfsetdash{}{0pt}%
\pgfpathmoveto{\pgfqpoint{0.275000in}{2.405633in}}%
\pgfpathlineto{\pgfqpoint{1.980000in}{2.405633in}}%
\pgfusepath{stroke}%
\end{pgfscope}%
\begin{pgfscope}%
\definecolor{textcolor}{rgb}{0.150000,0.150000,0.150000}%
\pgfsetstrokecolor{textcolor}%
\pgfsetfillcolor{textcolor}%
\pgftext[x=0.167360in,y=2.363423in,left,base]{\color{textcolor}\rmfamily\fontsize{8.000000}{9.600000}\selectfont \(\displaystyle 2\)}%
\end{pgfscope}%
\begin{pgfscope}%
\pgfpathrectangle{\pgfqpoint{0.275000in}{0.375000in}}{\pgfqpoint{1.705000in}{2.265000in}}%
\pgfusepath{clip}%
\pgfsetroundcap%
\pgfsetroundjoin%
\pgfsetlinewidth{1.505625pt}%
\definecolor{currentstroke}{rgb}{0.121569,0.466667,0.705882}%
\pgfsetstrokecolor{currentstroke}%
\pgfsetdash{}{0pt}%
\pgfpathmoveto{\pgfqpoint{0.352500in}{1.732308in}}%
\pgfpathlineto{\pgfqpoint{0.377350in}{1.759924in}}%
\pgfpathlineto{\pgfqpoint{0.399093in}{1.779376in}}%
\pgfpathlineto{\pgfqpoint{0.417730in}{1.792256in}}%
\pgfpathlineto{\pgfqpoint{0.436368in}{1.801498in}}%
\pgfpathlineto{\pgfqpoint{0.455005in}{1.807061in}}%
\pgfpathlineto{\pgfqpoint{0.473642in}{1.808985in}}%
\pgfpathlineto{\pgfqpoint{0.492280in}{1.807386in}}%
\pgfpathlineto{\pgfqpoint{0.510917in}{1.802448in}}%
\pgfpathlineto{\pgfqpoint{0.529554in}{1.794415in}}%
\pgfpathlineto{\pgfqpoint{0.551298in}{1.781531in}}%
\pgfpathlineto{\pgfqpoint{0.576147in}{1.762850in}}%
\pgfpathlineto{\pgfqpoint{0.604103in}{1.737889in}}%
\pgfpathlineto{\pgfqpoint{0.644484in}{1.697243in}}%
\pgfpathlineto{\pgfqpoint{0.719033in}{1.621495in}}%
\pgfpathlineto{\pgfqpoint{0.750095in}{1.594592in}}%
\pgfpathlineto{\pgfqpoint{0.778051in}{1.574614in}}%
\pgfpathlineto{\pgfqpoint{0.802901in}{1.560912in}}%
\pgfpathlineto{\pgfqpoint{0.824644in}{1.552387in}}%
\pgfpathlineto{\pgfqpoint{0.846388in}{1.547264in}}%
\pgfpathlineto{\pgfqpoint{0.868131in}{1.545596in}}%
\pgfpathlineto{\pgfqpoint{0.889875in}{1.547336in}}%
\pgfpathlineto{\pgfqpoint{0.911618in}{1.552345in}}%
\pgfpathlineto{\pgfqpoint{0.933362in}{1.560398in}}%
\pgfpathlineto{\pgfqpoint{0.958211in}{1.572950in}}%
\pgfpathlineto{\pgfqpoint{0.986167in}{1.590704in}}%
\pgfpathlineto{\pgfqpoint{1.020336in}{1.616438in}}%
\pgfpathlineto{\pgfqpoint{1.066929in}{1.655956in}}%
\pgfpathlineto{\pgfqpoint{1.175646in}{1.749779in}}%
\pgfpathlineto{\pgfqpoint{1.222239in}{1.785282in}}%
\pgfpathlineto{\pgfqpoint{1.268833in}{1.816790in}}%
\pgfpathlineto{\pgfqpoint{1.315426in}{1.844577in}}%
\pgfpathlineto{\pgfqpoint{1.368231in}{1.872331in}}%
\pgfpathlineto{\pgfqpoint{1.424143in}{1.898188in}}%
\pgfpathlineto{\pgfqpoint{1.480055in}{1.920662in}}%
\pgfpathlineto{\pgfqpoint{1.529755in}{1.937456in}}%
\pgfpathlineto{\pgfqpoint{1.576348in}{1.949993in}}%
\pgfpathlineto{\pgfqpoint{1.619835in}{1.958591in}}%
\pgfpathlineto{\pgfqpoint{1.663322in}{1.964151in}}%
\pgfpathlineto{\pgfqpoint{1.709915in}{1.967052in}}%
\pgfpathlineto{\pgfqpoint{1.768933in}{1.967381in}}%
\pgfpathlineto{\pgfqpoint{1.874544in}{1.967146in}}%
\pgfpathlineto{\pgfqpoint{1.902500in}{1.969437in}}%
\pgfpathlineto{\pgfqpoint{1.902500in}{1.969437in}}%
\pgfusepath{stroke}%
\end{pgfscope}%
\begin{pgfscope}%
\pgfpathrectangle{\pgfqpoint{0.275000in}{0.375000in}}{\pgfqpoint{1.705000in}{2.265000in}}%
\pgfusepath{clip}%
\pgfsetroundcap%
\pgfsetroundjoin%
\pgfsetlinewidth{1.505625pt}%
\definecolor{currentstroke}{rgb}{1.000000,0.498039,0.054902}%
\pgfsetstrokecolor{currentstroke}%
\pgfsetdash{}{0pt}%
\pgfpathmoveto{\pgfqpoint{0.352500in}{1.435113in}}%
\pgfpathlineto{\pgfqpoint{0.405306in}{1.307238in}}%
\pgfpathlineto{\pgfqpoint{0.442580in}{1.224862in}}%
\pgfpathlineto{\pgfqpoint{0.473642in}{1.162971in}}%
\pgfpathlineto{\pgfqpoint{0.504704in}{1.107999in}}%
\pgfpathlineto{\pgfqpoint{0.532660in}{1.064710in}}%
\pgfpathlineto{\pgfqpoint{0.560616in}{1.027301in}}%
\pgfpathlineto{\pgfqpoint{0.585466in}{0.998883in}}%
\pgfpathlineto{\pgfqpoint{0.610316in}{0.974873in}}%
\pgfpathlineto{\pgfqpoint{0.635165in}{0.955129in}}%
\pgfpathlineto{\pgfqpoint{0.660015in}{0.939540in}}%
\pgfpathlineto{\pgfqpoint{0.681759in}{0.929262in}}%
\pgfpathlineto{\pgfqpoint{0.703502in}{0.922132in}}%
\pgfpathlineto{\pgfqpoint{0.725245in}{0.918207in}}%
\pgfpathlineto{\pgfqpoint{0.746989in}{0.917588in}}%
\pgfpathlineto{\pgfqpoint{0.768732in}{0.920419in}}%
\pgfpathlineto{\pgfqpoint{0.790476in}{0.926866in}}%
\pgfpathlineto{\pgfqpoint{0.812219in}{0.937111in}}%
\pgfpathlineto{\pgfqpoint{0.833963in}{0.951326in}}%
\pgfpathlineto{\pgfqpoint{0.855706in}{0.969657in}}%
\pgfpathlineto{\pgfqpoint{0.877450in}{0.992200in}}%
\pgfpathlineto{\pgfqpoint{0.899193in}{1.018982in}}%
\pgfpathlineto{\pgfqpoint{0.924043in}{1.054695in}}%
\pgfpathlineto{\pgfqpoint{0.951999in}{1.101058in}}%
\pgfpathlineto{\pgfqpoint{0.983061in}{1.159449in}}%
\pgfpathlineto{\pgfqpoint{1.020336in}{1.237139in}}%
\pgfpathlineto{\pgfqpoint{1.076247in}{1.362272in}}%
\pgfpathlineto{\pgfqpoint{1.135266in}{1.492210in}}%
\pgfpathlineto{\pgfqpoint{1.169434in}{1.560263in}}%
\pgfpathlineto{\pgfqpoint{1.197390in}{1.609743in}}%
\pgfpathlineto{\pgfqpoint{1.222239in}{1.648178in}}%
\pgfpathlineto{\pgfqpoint{1.247089in}{1.680972in}}%
\pgfpathlineto{\pgfqpoint{1.268833in}{1.704948in}}%
\pgfpathlineto{\pgfqpoint{1.290576in}{1.724622in}}%
\pgfpathlineto{\pgfqpoint{1.312320in}{1.740252in}}%
\pgfpathlineto{\pgfqpoint{1.334063in}{1.752230in}}%
\pgfpathlineto{\pgfqpoint{1.358913in}{1.762089in}}%
\pgfpathlineto{\pgfqpoint{1.383763in}{1.768691in}}%
\pgfpathlineto{\pgfqpoint{1.414825in}{1.773766in}}%
\pgfpathlineto{\pgfqpoint{1.480055in}{1.780139in}}%
\pgfpathlineto{\pgfqpoint{1.526648in}{1.786569in}}%
\pgfpathlineto{\pgfqpoint{1.567029in}{1.795426in}}%
\pgfpathlineto{\pgfqpoint{1.610516in}{1.808233in}}%
\pgfpathlineto{\pgfqpoint{1.722340in}{1.843172in}}%
\pgfpathlineto{\pgfqpoint{1.753402in}{1.848899in}}%
\pgfpathlineto{\pgfqpoint{1.781358in}{1.851175in}}%
\pgfpathlineto{\pgfqpoint{1.809314in}{1.850286in}}%
\pgfpathlineto{\pgfqpoint{1.837270in}{1.845996in}}%
\pgfpathlineto{\pgfqpoint{1.865225in}{1.838268in}}%
\pgfpathlineto{\pgfqpoint{1.893181in}{1.827255in}}%
\pgfpathlineto{\pgfqpoint{1.902500in}{1.822906in}}%
\pgfpathlineto{\pgfqpoint{1.902500in}{1.822906in}}%
\pgfusepath{stroke}%
\end{pgfscope}%
\begin{pgfscope}%
\pgfpathrectangle{\pgfqpoint{0.275000in}{0.375000in}}{\pgfqpoint{1.705000in}{2.265000in}}%
\pgfusepath{clip}%
\pgfsetroundcap%
\pgfsetroundjoin%
\pgfsetlinewidth{1.505625pt}%
\definecolor{currentstroke}{rgb}{0.172549,0.627451,0.172549}%
\pgfsetstrokecolor{currentstroke}%
\pgfsetdash{}{0pt}%
\pgfpathmoveto{\pgfqpoint{0.352500in}{1.916846in}}%
\pgfpathlineto{\pgfqpoint{0.374243in}{1.945421in}}%
\pgfpathlineto{\pgfqpoint{0.392881in}{1.965388in}}%
\pgfpathlineto{\pgfqpoint{0.411518in}{1.981014in}}%
\pgfpathlineto{\pgfqpoint{0.430155in}{1.992267in}}%
\pgfpathlineto{\pgfqpoint{0.445686in}{1.998362in}}%
\pgfpathlineto{\pgfqpoint{0.461217in}{2.001591in}}%
\pgfpathlineto{\pgfqpoint{0.476748in}{2.002117in}}%
\pgfpathlineto{\pgfqpoint{0.495386in}{1.999484in}}%
\pgfpathlineto{\pgfqpoint{0.514023in}{1.993716in}}%
\pgfpathlineto{\pgfqpoint{0.535767in}{1.983718in}}%
\pgfpathlineto{\pgfqpoint{0.563722in}{1.967097in}}%
\pgfpathlineto{\pgfqpoint{0.622740in}{1.926693in}}%
\pgfpathlineto{\pgfqpoint{0.656909in}{1.905554in}}%
\pgfpathlineto{\pgfqpoint{0.684865in}{1.891960in}}%
\pgfpathlineto{\pgfqpoint{0.709714in}{1.883599in}}%
\pgfpathlineto{\pgfqpoint{0.731458in}{1.879540in}}%
\pgfpathlineto{\pgfqpoint{0.753201in}{1.878681in}}%
\pgfpathlineto{\pgfqpoint{0.774945in}{1.881033in}}%
\pgfpathlineto{\pgfqpoint{0.796688in}{1.886494in}}%
\pgfpathlineto{\pgfqpoint{0.821538in}{1.896287in}}%
\pgfpathlineto{\pgfqpoint{0.846388in}{1.909486in}}%
\pgfpathlineto{\pgfqpoint{0.874344in}{1.927824in}}%
\pgfpathlineto{\pgfqpoint{0.908512in}{1.954270in}}%
\pgfpathlineto{\pgfqpoint{0.948893in}{1.989680in}}%
\pgfpathlineto{\pgfqpoint{1.004805in}{2.043107in}}%
\pgfpathlineto{\pgfqpoint{1.132159in}{2.166333in}}%
\pgfpathlineto{\pgfqpoint{1.172540in}{2.200443in}}%
\pgfpathlineto{\pgfqpoint{1.203602in}{2.222886in}}%
\pgfpathlineto{\pgfqpoint{1.231558in}{2.239201in}}%
\pgfpathlineto{\pgfqpoint{1.256408in}{2.249846in}}%
\pgfpathlineto{\pgfqpoint{1.278151in}{2.255647in}}%
\pgfpathlineto{\pgfqpoint{1.296789in}{2.257675in}}%
\pgfpathlineto{\pgfqpoint{1.315426in}{2.256728in}}%
\pgfpathlineto{\pgfqpoint{1.334063in}{2.252596in}}%
\pgfpathlineto{\pgfqpoint{1.352700in}{2.245104in}}%
\pgfpathlineto{\pgfqpoint{1.371338in}{2.234127in}}%
\pgfpathlineto{\pgfqpoint{1.389975in}{2.219592in}}%
\pgfpathlineto{\pgfqpoint{1.408612in}{2.201487in}}%
\pgfpathlineto{\pgfqpoint{1.430356in}{2.175928in}}%
\pgfpathlineto{\pgfqpoint{1.452099in}{2.145783in}}%
\pgfpathlineto{\pgfqpoint{1.476949in}{2.106135in}}%
\pgfpathlineto{\pgfqpoint{1.504905in}{2.055720in}}%
\pgfpathlineto{\pgfqpoint{1.539073in}{1.987501in}}%
\pgfpathlineto{\pgfqpoint{1.591879in}{1.873893in}}%
\pgfpathlineto{\pgfqpoint{1.657109in}{1.734820in}}%
\pgfpathlineto{\pgfqpoint{1.691278in}{1.668909in}}%
\pgfpathlineto{\pgfqpoint{1.719233in}{1.620811in}}%
\pgfpathlineto{\pgfqpoint{1.744083in}{1.583316in}}%
\pgfpathlineto{\pgfqpoint{1.768933in}{1.551240in}}%
\pgfpathlineto{\pgfqpoint{1.790676in}{1.527793in}}%
\pgfpathlineto{\pgfqpoint{1.812420in}{1.508655in}}%
\pgfpathlineto{\pgfqpoint{1.834163in}{1.493677in}}%
\pgfpathlineto{\pgfqpoint{1.855907in}{1.482580in}}%
\pgfpathlineto{\pgfqpoint{1.877650in}{1.474957in}}%
\pgfpathlineto{\pgfqpoint{1.902500in}{1.469837in}}%
\pgfpathlineto{\pgfqpoint{1.902500in}{1.469837in}}%
\pgfusepath{stroke}%
\end{pgfscope}%
\begin{pgfscope}%
\pgfpathrectangle{\pgfqpoint{0.275000in}{0.375000in}}{\pgfqpoint{1.705000in}{2.265000in}}%
\pgfusepath{clip}%
\pgfsetroundcap%
\pgfsetroundjoin%
\pgfsetlinewidth{1.505625pt}%
\definecolor{currentstroke}{rgb}{0.839216,0.152941,0.156863}%
\pgfsetstrokecolor{currentstroke}%
\pgfsetdash{}{0pt}%
\pgfpathmoveto{\pgfqpoint{0.352500in}{2.537045in}}%
\pgfpathlineto{\pgfqpoint{0.371137in}{2.535158in}}%
\pgfpathlineto{\pgfqpoint{0.389775in}{2.530420in}}%
\pgfpathlineto{\pgfqpoint{0.408412in}{2.522723in}}%
\pgfpathlineto{\pgfqpoint{0.427049in}{2.512030in}}%
\pgfpathlineto{\pgfqpoint{0.448793in}{2.495830in}}%
\pgfpathlineto{\pgfqpoint{0.470536in}{2.475812in}}%
\pgfpathlineto{\pgfqpoint{0.495386in}{2.448677in}}%
\pgfpathlineto{\pgfqpoint{0.523342in}{2.413520in}}%
\pgfpathlineto{\pgfqpoint{0.557510in}{2.365521in}}%
\pgfpathlineto{\pgfqpoint{0.616528in}{2.276221in}}%
\pgfpathlineto{\pgfqpoint{0.672440in}{2.193584in}}%
\pgfpathlineto{\pgfqpoint{0.709714in}{2.144021in}}%
\pgfpathlineto{\pgfqpoint{0.740777in}{2.107749in}}%
\pgfpathlineto{\pgfqpoint{0.768732in}{2.079676in}}%
\pgfpathlineto{\pgfqpoint{0.793582in}{2.058636in}}%
\pgfpathlineto{\pgfqpoint{0.818432in}{2.041383in}}%
\pgfpathlineto{\pgfqpoint{0.843282in}{2.027903in}}%
\pgfpathlineto{\pgfqpoint{0.868131in}{2.018070in}}%
\pgfpathlineto{\pgfqpoint{0.892981in}{2.011661in}}%
\pgfpathlineto{\pgfqpoint{0.917831in}{2.008347in}}%
\pgfpathlineto{\pgfqpoint{0.945787in}{2.007788in}}%
\pgfpathlineto{\pgfqpoint{0.979955in}{2.010590in}}%
\pgfpathlineto{\pgfqpoint{1.035867in}{2.019210in}}%
\pgfpathlineto{\pgfqpoint{1.079354in}{2.024589in}}%
\pgfpathlineto{\pgfqpoint{1.107310in}{2.025175in}}%
\pgfpathlineto{\pgfqpoint{1.132159in}{2.022704in}}%
\pgfpathlineto{\pgfqpoint{1.153903in}{2.017624in}}%
\pgfpathlineto{\pgfqpoint{1.175646in}{2.009366in}}%
\pgfpathlineto{\pgfqpoint{1.197390in}{1.997570in}}%
\pgfpathlineto{\pgfqpoint{1.219133in}{1.981963in}}%
\pgfpathlineto{\pgfqpoint{1.240877in}{1.962371in}}%
\pgfpathlineto{\pgfqpoint{1.262620in}{1.938726in}}%
\pgfpathlineto{\pgfqpoint{1.287470in}{1.906800in}}%
\pgfpathlineto{\pgfqpoint{1.312320in}{1.869882in}}%
\pgfpathlineto{\pgfqpoint{1.340276in}{1.822917in}}%
\pgfpathlineto{\pgfqpoint{1.374444in}{1.758945in}}%
\pgfpathlineto{\pgfqpoint{1.417931in}{1.670211in}}%
\pgfpathlineto{\pgfqpoint{1.517330in}{1.464342in}}%
\pgfpathlineto{\pgfqpoint{1.548392in}{1.407685in}}%
\pgfpathlineto{\pgfqpoint{1.576348in}{1.362940in}}%
\pgfpathlineto{\pgfqpoint{1.601197in}{1.329296in}}%
\pgfpathlineto{\pgfqpoint{1.622941in}{1.305276in}}%
\pgfpathlineto{\pgfqpoint{1.641578in}{1.289075in}}%
\pgfpathlineto{\pgfqpoint{1.660215in}{1.277150in}}%
\pgfpathlineto{\pgfqpoint{1.675746in}{1.270582in}}%
\pgfpathlineto{\pgfqpoint{1.691278in}{1.267120in}}%
\pgfpathlineto{\pgfqpoint{1.706809in}{1.266763in}}%
\pgfpathlineto{\pgfqpoint{1.722340in}{1.269465in}}%
\pgfpathlineto{\pgfqpoint{1.737871in}{1.275138in}}%
\pgfpathlineto{\pgfqpoint{1.756508in}{1.285677in}}%
\pgfpathlineto{\pgfqpoint{1.775145in}{1.299991in}}%
\pgfpathlineto{\pgfqpoint{1.796889in}{1.320928in}}%
\pgfpathlineto{\pgfqpoint{1.821738in}{1.349501in}}%
\pgfpathlineto{\pgfqpoint{1.852801in}{1.390217in}}%
\pgfpathlineto{\pgfqpoint{1.902500in}{1.459971in}}%
\pgfpathlineto{\pgfqpoint{1.902500in}{1.459971in}}%
\pgfusepath{stroke}%
\end{pgfscope}%
\begin{pgfscope}%
\pgfpathrectangle{\pgfqpoint{0.275000in}{0.375000in}}{\pgfqpoint{1.705000in}{2.265000in}}%
\pgfusepath{clip}%
\pgfsetroundcap%
\pgfsetroundjoin%
\pgfsetlinewidth{1.505625pt}%
\definecolor{currentstroke}{rgb}{0.580392,0.403922,0.741176}%
\pgfsetstrokecolor{currentstroke}%
\pgfsetdash{}{0pt}%
\pgfpathmoveto{\pgfqpoint{0.352500in}{1.185870in}}%
\pgfpathlineto{\pgfqpoint{0.374243in}{1.154382in}}%
\pgfpathlineto{\pgfqpoint{0.392881in}{1.132569in}}%
\pgfpathlineto{\pgfqpoint{0.411518in}{1.115500in}}%
\pgfpathlineto{\pgfqpoint{0.430155in}{1.103102in}}%
\pgfpathlineto{\pgfqpoint{0.445686in}{1.096267in}}%
\pgfpathlineto{\pgfqpoint{0.461217in}{1.092533in}}%
\pgfpathlineto{\pgfqpoint{0.476748in}{1.091817in}}%
\pgfpathlineto{\pgfqpoint{0.492280in}{1.094024in}}%
\pgfpathlineto{\pgfqpoint{0.507811in}{1.099054in}}%
\pgfpathlineto{\pgfqpoint{0.526448in}{1.108658in}}%
\pgfpathlineto{\pgfqpoint{0.545085in}{1.121958in}}%
\pgfpathlineto{\pgfqpoint{0.566829in}{1.141843in}}%
\pgfpathlineto{\pgfqpoint{0.588572in}{1.166056in}}%
\pgfpathlineto{\pgfqpoint{0.613422in}{1.198464in}}%
\pgfpathlineto{\pgfqpoint{0.641378in}{1.240090in}}%
\pgfpathlineto{\pgfqpoint{0.675546in}{1.296758in}}%
\pgfpathlineto{\pgfqpoint{0.725245in}{1.385907in}}%
\pgfpathlineto{\pgfqpoint{0.793582in}{1.508053in}}%
\pgfpathlineto{\pgfqpoint{0.827751in}{1.563020in}}%
\pgfpathlineto{\pgfqpoint{0.855706in}{1.602641in}}%
\pgfpathlineto{\pgfqpoint{0.880556in}{1.632816in}}%
\pgfpathlineto{\pgfqpoint{0.902300in}{1.654771in}}%
\pgfpathlineto{\pgfqpoint{0.924043in}{1.672192in}}%
\pgfpathlineto{\pgfqpoint{0.942680in}{1.683303in}}%
\pgfpathlineto{\pgfqpoint{0.961318in}{1.690754in}}%
\pgfpathlineto{\pgfqpoint{0.979955in}{1.694470in}}%
\pgfpathlineto{\pgfqpoint{0.995486in}{1.694692in}}%
\pgfpathlineto{\pgfqpoint{1.011017in}{1.692310in}}%
\pgfpathlineto{\pgfqpoint{1.029654in}{1.686064in}}%
\pgfpathlineto{\pgfqpoint{1.048292in}{1.676220in}}%
\pgfpathlineto{\pgfqpoint{1.066929in}{1.662934in}}%
\pgfpathlineto{\pgfqpoint{1.088672in}{1.643360in}}%
\pgfpathlineto{\pgfqpoint{1.113522in}{1.616137in}}%
\pgfpathlineto{\pgfqpoint{1.141478in}{1.580272in}}%
\pgfpathlineto{\pgfqpoint{1.175646in}{1.530863in}}%
\pgfpathlineto{\pgfqpoint{1.284364in}{1.368958in}}%
\pgfpathlineto{\pgfqpoint{1.309213in}{1.339422in}}%
\pgfpathlineto{\pgfqpoint{1.330957in}{1.318176in}}%
\pgfpathlineto{\pgfqpoint{1.349594in}{1.303972in}}%
\pgfpathlineto{\pgfqpoint{1.368231in}{1.293862in}}%
\pgfpathlineto{\pgfqpoint{1.383763in}{1.288763in}}%
\pgfpathlineto{\pgfqpoint{1.399294in}{1.286798in}}%
\pgfpathlineto{\pgfqpoint{1.414825in}{1.288007in}}%
\pgfpathlineto{\pgfqpoint{1.430356in}{1.292373in}}%
\pgfpathlineto{\pgfqpoint{1.445887in}{1.299816in}}%
\pgfpathlineto{\pgfqpoint{1.464524in}{1.312615in}}%
\pgfpathlineto{\pgfqpoint{1.483161in}{1.329304in}}%
\pgfpathlineto{\pgfqpoint{1.504905in}{1.353087in}}%
\pgfpathlineto{\pgfqpoint{1.529755in}{1.384888in}}%
\pgfpathlineto{\pgfqpoint{1.563923in}{1.434019in}}%
\pgfpathlineto{\pgfqpoint{1.647791in}{1.557097in}}%
\pgfpathlineto{\pgfqpoint{1.675746in}{1.592085in}}%
\pgfpathlineto{\pgfqpoint{1.700596in}{1.618774in}}%
\pgfpathlineto{\pgfqpoint{1.725446in}{1.640854in}}%
\pgfpathlineto{\pgfqpoint{1.747189in}{1.656322in}}%
\pgfpathlineto{\pgfqpoint{1.768933in}{1.668343in}}%
\pgfpathlineto{\pgfqpoint{1.793783in}{1.678276in}}%
\pgfpathlineto{\pgfqpoint{1.818632in}{1.684819in}}%
\pgfpathlineto{\pgfqpoint{1.849694in}{1.689539in}}%
\pgfpathlineto{\pgfqpoint{1.902500in}{1.693664in}}%
\pgfpathlineto{\pgfqpoint{1.902500in}{1.693664in}}%
\pgfusepath{stroke}%
\end{pgfscope}%
\begin{pgfscope}%
\pgfpathrectangle{\pgfqpoint{0.275000in}{0.375000in}}{\pgfqpoint{1.705000in}{2.265000in}}%
\pgfusepath{clip}%
\pgfsetroundcap%
\pgfsetroundjoin%
\pgfsetlinewidth{1.505625pt}%
\definecolor{currentstroke}{rgb}{0.549020,0.337255,0.294118}%
\pgfsetstrokecolor{currentstroke}%
\pgfsetdash{}{0pt}%
\pgfpathmoveto{\pgfqpoint{0.352500in}{1.493148in}}%
\pgfpathlineto{\pgfqpoint{0.380456in}{1.532407in}}%
\pgfpathlineto{\pgfqpoint{0.405306in}{1.561755in}}%
\pgfpathlineto{\pgfqpoint{0.427049in}{1.582392in}}%
\pgfpathlineto{\pgfqpoint{0.445686in}{1.596007in}}%
\pgfpathlineto{\pgfqpoint{0.464324in}{1.605716in}}%
\pgfpathlineto{\pgfqpoint{0.482961in}{1.611502in}}%
\pgfpathlineto{\pgfqpoint{0.498492in}{1.613390in}}%
\pgfpathlineto{\pgfqpoint{0.517129in}{1.612301in}}%
\pgfpathlineto{\pgfqpoint{0.535767in}{1.607815in}}%
\pgfpathlineto{\pgfqpoint{0.554404in}{1.600296in}}%
\pgfpathlineto{\pgfqpoint{0.576147in}{1.588265in}}%
\pgfpathlineto{\pgfqpoint{0.604103in}{1.568838in}}%
\pgfpathlineto{\pgfqpoint{0.644484in}{1.536284in}}%
\pgfpathlineto{\pgfqpoint{0.703502in}{1.489036in}}%
\pgfpathlineto{\pgfqpoint{0.734564in}{1.468480in}}%
\pgfpathlineto{\pgfqpoint{0.759414in}{1.455532in}}%
\pgfpathlineto{\pgfqpoint{0.784264in}{1.446235in}}%
\pgfpathlineto{\pgfqpoint{0.806007in}{1.441327in}}%
\pgfpathlineto{\pgfqpoint{0.827751in}{1.439520in}}%
\pgfpathlineto{\pgfqpoint{0.849494in}{1.440801in}}%
\pgfpathlineto{\pgfqpoint{0.871237in}{1.445079in}}%
\pgfpathlineto{\pgfqpoint{0.896087in}{1.453446in}}%
\pgfpathlineto{\pgfqpoint{0.920937in}{1.465242in}}%
\pgfpathlineto{\pgfqpoint{0.948893in}{1.482185in}}%
\pgfpathlineto{\pgfqpoint{0.979955in}{1.504958in}}%
\pgfpathlineto{\pgfqpoint{1.014123in}{1.533941in}}%
\pgfpathlineto{\pgfqpoint{1.057610in}{1.575209in}}%
\pgfpathlineto{\pgfqpoint{1.144584in}{1.663703in}}%
\pgfpathlineto{\pgfqpoint{1.191177in}{1.708523in}}%
\pgfpathlineto{\pgfqpoint{1.225346in}{1.737543in}}%
\pgfpathlineto{\pgfqpoint{1.256408in}{1.759795in}}%
\pgfpathlineto{\pgfqpoint{1.281258in}{1.774121in}}%
\pgfpathlineto{\pgfqpoint{1.306107in}{1.784938in}}%
\pgfpathlineto{\pgfqpoint{1.327851in}{1.791300in}}%
\pgfpathlineto{\pgfqpoint{1.349594in}{1.794637in}}%
\pgfpathlineto{\pgfqpoint{1.371338in}{1.794899in}}%
\pgfpathlineto{\pgfqpoint{1.393081in}{1.792107in}}%
\pgfpathlineto{\pgfqpoint{1.414825in}{1.786344in}}%
\pgfpathlineto{\pgfqpoint{1.439674in}{1.776303in}}%
\pgfpathlineto{\pgfqpoint{1.464524in}{1.762848in}}%
\pgfpathlineto{\pgfqpoint{1.492480in}{1.744042in}}%
\pgfpathlineto{\pgfqpoint{1.523542in}{1.719172in}}%
\pgfpathlineto{\pgfqpoint{1.557710in}{1.687727in}}%
\pgfpathlineto{\pgfqpoint{1.594985in}{1.649335in}}%
\pgfpathlineto{\pgfqpoint{1.638472in}{1.599904in}}%
\pgfpathlineto{\pgfqpoint{1.681959in}{1.545796in}}%
\pgfpathlineto{\pgfqpoint{1.725446in}{1.486943in}}%
\pgfpathlineto{\pgfqpoint{1.772039in}{1.418603in}}%
\pgfpathlineto{\pgfqpoint{1.827951in}{1.330689in}}%
\pgfpathlineto{\pgfqpoint{1.902500in}{1.212417in}}%
\pgfpathlineto{\pgfqpoint{1.902500in}{1.212417in}}%
\pgfusepath{stroke}%
\end{pgfscope}%
\begin{pgfscope}%
\pgfpathrectangle{\pgfqpoint{0.275000in}{0.375000in}}{\pgfqpoint{1.705000in}{2.265000in}}%
\pgfusepath{clip}%
\pgfsetroundcap%
\pgfsetroundjoin%
\pgfsetlinewidth{1.505625pt}%
\definecolor{currentstroke}{rgb}{0.890196,0.466667,0.760784}%
\pgfsetstrokecolor{currentstroke}%
\pgfsetdash{}{0pt}%
\pgfpathmoveto{\pgfqpoint{0.352500in}{1.469205in}}%
\pgfpathlineto{\pgfqpoint{0.380456in}{1.406768in}}%
\pgfpathlineto{\pgfqpoint{0.405306in}{1.358180in}}%
\pgfpathlineto{\pgfqpoint{0.430155in}{1.316158in}}%
\pgfpathlineto{\pgfqpoint{0.455005in}{1.280607in}}%
\pgfpathlineto{\pgfqpoint{0.476748in}{1.254642in}}%
\pgfpathlineto{\pgfqpoint{0.498492in}{1.233266in}}%
\pgfpathlineto{\pgfqpoint{0.520235in}{1.216233in}}%
\pgfpathlineto{\pgfqpoint{0.541979in}{1.203264in}}%
\pgfpathlineto{\pgfqpoint{0.563722in}{1.194046in}}%
\pgfpathlineto{\pgfqpoint{0.585466in}{1.188237in}}%
\pgfpathlineto{\pgfqpoint{0.607209in}{1.185467in}}%
\pgfpathlineto{\pgfqpoint{0.632059in}{1.185510in}}%
\pgfpathlineto{\pgfqpoint{0.660015in}{1.188876in}}%
\pgfpathlineto{\pgfqpoint{0.694183in}{1.196369in}}%
\pgfpathlineto{\pgfqpoint{0.815326in}{1.226336in}}%
\pgfpathlineto{\pgfqpoint{0.846388in}{1.229295in}}%
\pgfpathlineto{\pgfqpoint{0.877450in}{1.229278in}}%
\pgfpathlineto{\pgfqpoint{0.914724in}{1.225901in}}%
\pgfpathlineto{\pgfqpoint{1.004805in}{1.215488in}}%
\pgfpathlineto{\pgfqpoint{1.029654in}{1.216598in}}%
\pgfpathlineto{\pgfqpoint{1.051398in}{1.220587in}}%
\pgfpathlineto{\pgfqpoint{1.073141in}{1.228067in}}%
\pgfpathlineto{\pgfqpoint{1.091779in}{1.237657in}}%
\pgfpathlineto{\pgfqpoint{1.110416in}{1.250444in}}%
\pgfpathlineto{\pgfqpoint{1.129053in}{1.266573in}}%
\pgfpathlineto{\pgfqpoint{1.150797in}{1.289667in}}%
\pgfpathlineto{\pgfqpoint{1.172540in}{1.317238in}}%
\pgfpathlineto{\pgfqpoint{1.197390in}{1.353792in}}%
\pgfpathlineto{\pgfqpoint{1.225346in}{1.400349in}}%
\pgfpathlineto{\pgfqpoint{1.265726in}{1.474499in}}%
\pgfpathlineto{\pgfqpoint{1.340276in}{1.613251in}}%
\pgfpathlineto{\pgfqpoint{1.371338in}{1.663659in}}%
\pgfpathlineto{\pgfqpoint{1.396187in}{1.698050in}}%
\pgfpathlineto{\pgfqpoint{1.417931in}{1.722847in}}%
\pgfpathlineto{\pgfqpoint{1.436568in}{1.739704in}}%
\pgfpathlineto{\pgfqpoint{1.455205in}{1.752220in}}%
\pgfpathlineto{\pgfqpoint{1.470736in}{1.759215in}}%
\pgfpathlineto{\pgfqpoint{1.486268in}{1.763040in}}%
\pgfpathlineto{\pgfqpoint{1.501799in}{1.763702in}}%
\pgfpathlineto{\pgfqpoint{1.517330in}{1.761252in}}%
\pgfpathlineto{\pgfqpoint{1.532861in}{1.755785in}}%
\pgfpathlineto{\pgfqpoint{1.551498in}{1.745439in}}%
\pgfpathlineto{\pgfqpoint{1.570135in}{1.731254in}}%
\pgfpathlineto{\pgfqpoint{1.591879in}{1.710355in}}%
\pgfpathlineto{\pgfqpoint{1.616728in}{1.681590in}}%
\pgfpathlineto{\pgfqpoint{1.644684in}{1.644397in}}%
\pgfpathlineto{\pgfqpoint{1.685065in}{1.585135in}}%
\pgfpathlineto{\pgfqpoint{1.759614in}{1.475141in}}%
\pgfpathlineto{\pgfqpoint{1.790676in}{1.434927in}}%
\pgfpathlineto{\pgfqpoint{1.818632in}{1.403696in}}%
\pgfpathlineto{\pgfqpoint{1.843482in}{1.380622in}}%
\pgfpathlineto{\pgfqpoint{1.865225in}{1.364392in}}%
\pgfpathlineto{\pgfqpoint{1.886969in}{1.352020in}}%
\pgfpathlineto{\pgfqpoint{1.902500in}{1.345573in}}%
\pgfpathlineto{\pgfqpoint{1.902500in}{1.345573in}}%
\pgfusepath{stroke}%
\end{pgfscope}%
\begin{pgfscope}%
\pgfpathrectangle{\pgfqpoint{0.275000in}{0.375000in}}{\pgfqpoint{1.705000in}{2.265000in}}%
\pgfusepath{clip}%
\pgfsetroundcap%
\pgfsetroundjoin%
\pgfsetlinewidth{1.505625pt}%
\definecolor{currentstroke}{rgb}{0.498039,0.498039,0.498039}%
\pgfsetstrokecolor{currentstroke}%
\pgfsetdash{}{0pt}%
\pgfpathmoveto{\pgfqpoint{0.352500in}{2.450389in}}%
\pgfpathlineto{\pgfqpoint{0.368031in}{2.445479in}}%
\pgfpathlineto{\pgfqpoint{0.383562in}{2.437514in}}%
\pgfpathlineto{\pgfqpoint{0.402199in}{2.424094in}}%
\pgfpathlineto{\pgfqpoint{0.420837in}{2.406720in}}%
\pgfpathlineto{\pgfqpoint{0.442580in}{2.381906in}}%
\pgfpathlineto{\pgfqpoint{0.467430in}{2.348333in}}%
\pgfpathlineto{\pgfqpoint{0.495386in}{2.305202in}}%
\pgfpathlineto{\pgfqpoint{0.532660in}{2.241647in}}%
\pgfpathlineto{\pgfqpoint{0.641378in}{2.052035in}}%
\pgfpathlineto{\pgfqpoint{0.672440in}{2.005221in}}%
\pgfpathlineto{\pgfqpoint{0.700396in}{1.968095in}}%
\pgfpathlineto{\pgfqpoint{0.728352in}{1.936278in}}%
\pgfpathlineto{\pgfqpoint{0.753201in}{1.912640in}}%
\pgfpathlineto{\pgfqpoint{0.778051in}{1.893339in}}%
\pgfpathlineto{\pgfqpoint{0.802901in}{1.878166in}}%
\pgfpathlineto{\pgfqpoint{0.827751in}{1.866767in}}%
\pgfpathlineto{\pgfqpoint{0.852600in}{1.858669in}}%
\pgfpathlineto{\pgfqpoint{0.880556in}{1.852804in}}%
\pgfpathlineto{\pgfqpoint{0.914724in}{1.849062in}}%
\pgfpathlineto{\pgfqpoint{0.967530in}{1.847045in}}%
\pgfpathlineto{\pgfqpoint{1.026548in}{1.843991in}}%
\pgfpathlineto{\pgfqpoint{1.060716in}{1.839282in}}%
\pgfpathlineto{\pgfqpoint{1.091779in}{1.832025in}}%
\pgfpathlineto{\pgfqpoint{1.119734in}{1.822538in}}%
\pgfpathlineto{\pgfqpoint{1.147690in}{1.809862in}}%
\pgfpathlineto{\pgfqpoint{1.175646in}{1.793686in}}%
\pgfpathlineto{\pgfqpoint{1.203602in}{1.773749in}}%
\pgfpathlineto{\pgfqpoint{1.231558in}{1.749830in}}%
\pgfpathlineto{\pgfqpoint{1.259514in}{1.721742in}}%
\pgfpathlineto{\pgfqpoint{1.287470in}{1.689327in}}%
\pgfpathlineto{\pgfqpoint{1.315426in}{1.652462in}}%
\pgfpathlineto{\pgfqpoint{1.346488in}{1.606174in}}%
\pgfpathlineto{\pgfqpoint{1.377550in}{1.554224in}}%
\pgfpathlineto{\pgfqpoint{1.411718in}{1.490579in}}%
\pgfpathlineto{\pgfqpoint{1.445887in}{1.420326in}}%
\pgfpathlineto{\pgfqpoint{1.483161in}{1.336664in}}%
\pgfpathlineto{\pgfqpoint{1.526648in}{1.231062in}}%
\pgfpathlineto{\pgfqpoint{1.582560in}{1.086375in}}%
\pgfpathlineto{\pgfqpoint{1.685065in}{0.819546in}}%
\pgfpathlineto{\pgfqpoint{1.722340in}{0.731486in}}%
\pgfpathlineto{\pgfqpoint{1.753402in}{0.665263in}}%
\pgfpathlineto{\pgfqpoint{1.781358in}{0.612585in}}%
\pgfpathlineto{\pgfqpoint{1.806207in}{0.572043in}}%
\pgfpathlineto{\pgfqpoint{1.827951in}{0.541828in}}%
\pgfpathlineto{\pgfqpoint{1.849694in}{0.516767in}}%
\pgfpathlineto{\pgfqpoint{1.868332in}{0.499489in}}%
\pgfpathlineto{\pgfqpoint{1.886969in}{0.486121in}}%
\pgfpathlineto{\pgfqpoint{1.902500in}{0.477955in}}%
\pgfpathlineto{\pgfqpoint{1.902500in}{0.477955in}}%
\pgfusepath{stroke}%
\end{pgfscope}%
\begin{pgfscope}%
\pgfpathrectangle{\pgfqpoint{0.275000in}{0.375000in}}{\pgfqpoint{1.705000in}{2.265000in}}%
\pgfusepath{clip}%
\pgfsetroundcap%
\pgfsetroundjoin%
\pgfsetlinewidth{1.505625pt}%
\definecolor{currentstroke}{rgb}{0.737255,0.741176,0.133333}%
\pgfsetstrokecolor{currentstroke}%
\pgfsetdash{}{0pt}%
\pgfpathmoveto{\pgfqpoint{0.352500in}{0.724078in}}%
\pgfpathlineto{\pgfqpoint{0.371137in}{0.732752in}}%
\pgfpathlineto{\pgfqpoint{0.389775in}{0.744654in}}%
\pgfpathlineto{\pgfqpoint{0.408412in}{0.759931in}}%
\pgfpathlineto{\pgfqpoint{0.430155in}{0.782094in}}%
\pgfpathlineto{\pgfqpoint{0.451899in}{0.808857in}}%
\pgfpathlineto{\pgfqpoint{0.476748in}{0.844731in}}%
\pgfpathlineto{\pgfqpoint{0.504704in}{0.890967in}}%
\pgfpathlineto{\pgfqpoint{0.538873in}{0.953814in}}%
\pgfpathlineto{\pgfqpoint{0.644484in}{1.153045in}}%
\pgfpathlineto{\pgfqpoint{0.672440in}{1.196417in}}%
\pgfpathlineto{\pgfqpoint{0.694183in}{1.224976in}}%
\pgfpathlineto{\pgfqpoint{0.715927in}{1.248355in}}%
\pgfpathlineto{\pgfqpoint{0.734564in}{1.263987in}}%
\pgfpathlineto{\pgfqpoint{0.753201in}{1.275443in}}%
\pgfpathlineto{\pgfqpoint{0.771839in}{1.282756in}}%
\pgfpathlineto{\pgfqpoint{0.787370in}{1.285795in}}%
\pgfpathlineto{\pgfqpoint{0.806007in}{1.286009in}}%
\pgfpathlineto{\pgfqpoint{0.824644in}{1.282833in}}%
\pgfpathlineto{\pgfqpoint{0.846388in}{1.275476in}}%
\pgfpathlineto{\pgfqpoint{0.871237in}{1.263355in}}%
\pgfpathlineto{\pgfqpoint{0.908512in}{1.240978in}}%
\pgfpathlineto{\pgfqpoint{0.955105in}{1.213508in}}%
\pgfpathlineto{\pgfqpoint{0.979955in}{1.202488in}}%
\pgfpathlineto{\pgfqpoint{1.001698in}{1.196443in}}%
\pgfpathlineto{\pgfqpoint{1.020336in}{1.194620in}}%
\pgfpathlineto{\pgfqpoint{1.035867in}{1.195828in}}%
\pgfpathlineto{\pgfqpoint{1.051398in}{1.199757in}}%
\pgfpathlineto{\pgfqpoint{1.066929in}{1.206595in}}%
\pgfpathlineto{\pgfqpoint{1.082460in}{1.216486in}}%
\pgfpathlineto{\pgfqpoint{1.101097in}{1.232528in}}%
\pgfpathlineto{\pgfqpoint{1.119734in}{1.253215in}}%
\pgfpathlineto{\pgfqpoint{1.138372in}{1.278564in}}%
\pgfpathlineto{\pgfqpoint{1.160115in}{1.313939in}}%
\pgfpathlineto{\pgfqpoint{1.181859in}{1.355337in}}%
\pgfpathlineto{\pgfqpoint{1.206708in}{1.409549in}}%
\pgfpathlineto{\pgfqpoint{1.234664in}{1.478444in}}%
\pgfpathlineto{\pgfqpoint{1.265726in}{1.563257in}}%
\pgfpathlineto{\pgfqpoint{1.306107in}{1.683060in}}%
\pgfpathlineto{\pgfqpoint{1.442781in}{2.097653in}}%
\pgfpathlineto{\pgfqpoint{1.473843in}{2.178335in}}%
\pgfpathlineto{\pgfqpoint{1.501799in}{2.242600in}}%
\pgfpathlineto{\pgfqpoint{1.526648in}{2.292068in}}%
\pgfpathlineto{\pgfqpoint{1.548392in}{2.328912in}}%
\pgfpathlineto{\pgfqpoint{1.570135in}{2.359443in}}%
\pgfpathlineto{\pgfqpoint{1.588773in}{2.380471in}}%
\pgfpathlineto{\pgfqpoint{1.607410in}{2.396738in}}%
\pgfpathlineto{\pgfqpoint{1.622941in}{2.406699in}}%
\pgfpathlineto{\pgfqpoint{1.638472in}{2.413471in}}%
\pgfpathlineto{\pgfqpoint{1.654003in}{2.417166in}}%
\pgfpathlineto{\pgfqpoint{1.669534in}{2.417927in}}%
\pgfpathlineto{\pgfqpoint{1.685065in}{2.415925in}}%
\pgfpathlineto{\pgfqpoint{1.703702in}{2.410160in}}%
\pgfpathlineto{\pgfqpoint{1.722340in}{2.401097in}}%
\pgfpathlineto{\pgfqpoint{1.744083in}{2.386927in}}%
\pgfpathlineto{\pgfqpoint{1.768933in}{2.366881in}}%
\pgfpathlineto{\pgfqpoint{1.799995in}{2.337680in}}%
\pgfpathlineto{\pgfqpoint{1.855907in}{2.279823in}}%
\pgfpathlineto{\pgfqpoint{1.902500in}{2.232545in}}%
\pgfpathlineto{\pgfqpoint{1.902500in}{2.232545in}}%
\pgfusepath{stroke}%
\end{pgfscope}%
\begin{pgfscope}%
\pgfpathrectangle{\pgfqpoint{0.275000in}{0.375000in}}{\pgfqpoint{1.705000in}{2.265000in}}%
\pgfusepath{clip}%
\pgfsetroundcap%
\pgfsetroundjoin%
\pgfsetlinewidth{1.505625pt}%
\definecolor{currentstroke}{rgb}{0.090196,0.745098,0.811765}%
\pgfsetstrokecolor{currentstroke}%
\pgfsetdash{}{0pt}%
\pgfpathmoveto{\pgfqpoint{0.352500in}{1.625172in}}%
\pgfpathlineto{\pgfqpoint{0.374243in}{1.638432in}}%
\pgfpathlineto{\pgfqpoint{0.395987in}{1.647623in}}%
\pgfpathlineto{\pgfqpoint{0.417730in}{1.652944in}}%
\pgfpathlineto{\pgfqpoint{0.439474in}{1.654672in}}%
\pgfpathlineto{\pgfqpoint{0.461217in}{1.653148in}}%
\pgfpathlineto{\pgfqpoint{0.486067in}{1.647908in}}%
\pgfpathlineto{\pgfqpoint{0.510917in}{1.639503in}}%
\pgfpathlineto{\pgfqpoint{0.541979in}{1.625440in}}%
\pgfpathlineto{\pgfqpoint{0.579254in}{1.604767in}}%
\pgfpathlineto{\pgfqpoint{0.632059in}{1.571397in}}%
\pgfpathlineto{\pgfqpoint{0.728352in}{1.509959in}}%
\pgfpathlineto{\pgfqpoint{0.765626in}{1.489978in}}%
\pgfpathlineto{\pgfqpoint{0.796688in}{1.476685in}}%
\pgfpathlineto{\pgfqpoint{0.824644in}{1.468227in}}%
\pgfpathlineto{\pgfqpoint{0.849494in}{1.464126in}}%
\pgfpathlineto{\pgfqpoint{0.871237in}{1.463581in}}%
\pgfpathlineto{\pgfqpoint{0.892981in}{1.466172in}}%
\pgfpathlineto{\pgfqpoint{0.914724in}{1.472117in}}%
\pgfpathlineto{\pgfqpoint{0.936468in}{1.481551in}}%
\pgfpathlineto{\pgfqpoint{0.958211in}{1.494509in}}%
\pgfpathlineto{\pgfqpoint{0.979955in}{1.510911in}}%
\pgfpathlineto{\pgfqpoint{1.004805in}{1.533619in}}%
\pgfpathlineto{\pgfqpoint{1.032761in}{1.563630in}}%
\pgfpathlineto{\pgfqpoint{1.066929in}{1.605350in}}%
\pgfpathlineto{\pgfqpoint{1.119734in}{1.675701in}}%
\pgfpathlineto{\pgfqpoint{1.175646in}{1.748752in}}%
\pgfpathlineto{\pgfqpoint{1.209815in}{1.788287in}}%
\pgfpathlineto{\pgfqpoint{1.237771in}{1.816166in}}%
\pgfpathlineto{\pgfqpoint{1.262620in}{1.837014in}}%
\pgfpathlineto{\pgfqpoint{1.287470in}{1.853937in}}%
\pgfpathlineto{\pgfqpoint{1.312320in}{1.866946in}}%
\pgfpathlineto{\pgfqpoint{1.337169in}{1.876263in}}%
\pgfpathlineto{\pgfqpoint{1.362019in}{1.882294in}}%
\pgfpathlineto{\pgfqpoint{1.389975in}{1.885861in}}%
\pgfpathlineto{\pgfqpoint{1.424143in}{1.887000in}}%
\pgfpathlineto{\pgfqpoint{1.520436in}{1.888219in}}%
\pgfpathlineto{\pgfqpoint{1.548392in}{1.892501in}}%
\pgfpathlineto{\pgfqpoint{1.576348in}{1.900073in}}%
\pgfpathlineto{\pgfqpoint{1.601197in}{1.909956in}}%
\pgfpathlineto{\pgfqpoint{1.626047in}{1.922959in}}%
\pgfpathlineto{\pgfqpoint{1.654003in}{1.941239in}}%
\pgfpathlineto{\pgfqpoint{1.685065in}{1.965591in}}%
\pgfpathlineto{\pgfqpoint{1.725446in}{2.001864in}}%
\pgfpathlineto{\pgfqpoint{1.803101in}{2.072914in}}%
\pgfpathlineto{\pgfqpoint{1.831057in}{2.093491in}}%
\pgfpathlineto{\pgfqpoint{1.852801in}{2.105750in}}%
\pgfpathlineto{\pgfqpoint{1.871438in}{2.112946in}}%
\pgfpathlineto{\pgfqpoint{1.890075in}{2.116561in}}%
\pgfpathlineto{\pgfqpoint{1.902500in}{2.116758in}}%
\pgfpathlineto{\pgfqpoint{1.902500in}{2.116758in}}%
\pgfusepath{stroke}%
\end{pgfscope}%
\begin{pgfscope}%
\pgfsetrectcap%
\pgfsetmiterjoin%
\pgfsetlinewidth{0.000000pt}%
\definecolor{currentstroke}{rgb}{1.000000,1.000000,1.000000}%
\pgfsetstrokecolor{currentstroke}%
\pgfsetdash{}{0pt}%
\pgfpathmoveto{\pgfqpoint{0.275000in}{0.375000in}}%
\pgfpathlineto{\pgfqpoint{0.275000in}{2.640000in}}%
\pgfusepath{}%
\end{pgfscope}%
\begin{pgfscope}%
\pgfsetrectcap%
\pgfsetmiterjoin%
\pgfsetlinewidth{0.000000pt}%
\definecolor{currentstroke}{rgb}{1.000000,1.000000,1.000000}%
\pgfsetstrokecolor{currentstroke}%
\pgfsetdash{}{0pt}%
\pgfpathmoveto{\pgfqpoint{1.980000in}{0.375000in}}%
\pgfpathlineto{\pgfqpoint{1.980000in}{2.640000in}}%
\pgfusepath{}%
\end{pgfscope}%
\begin{pgfscope}%
\pgfsetrectcap%
\pgfsetmiterjoin%
\pgfsetlinewidth{0.000000pt}%
\definecolor{currentstroke}{rgb}{1.000000,1.000000,1.000000}%
\pgfsetstrokecolor{currentstroke}%
\pgfsetdash{}{0pt}%
\pgfpathmoveto{\pgfqpoint{0.275000in}{0.375000in}}%
\pgfpathlineto{\pgfqpoint{1.980000in}{0.375000in}}%
\pgfusepath{}%
\end{pgfscope}%
\begin{pgfscope}%
\pgfsetrectcap%
\pgfsetmiterjoin%
\pgfsetlinewidth{0.000000pt}%
\definecolor{currentstroke}{rgb}{1.000000,1.000000,1.000000}%
\pgfsetstrokecolor{currentstroke}%
\pgfsetdash{}{0pt}%
\pgfpathmoveto{\pgfqpoint{0.275000in}{2.640000in}}%
\pgfpathlineto{\pgfqpoint{1.980000in}{2.640000in}}%
\pgfusepath{}%
\end{pgfscope}%
\end{pgfpicture}%
\makeatother%
\endgroup%

            \end{subfigure}
            \begin{subfigure}[t]{0.31\textwidth}
                \centering
                %% Creator: Matplotlib, PGF backend
%%
%% To include the figure in your LaTeX document, write
%%   \input{<filename>.pgf}
%%
%% Make sure the required packages are loaded in your preamble
%%   \usepackage{pgf}
%%
%% Figures using additional raster images can only be included by \input if
%% they are in the same directory as the main LaTeX file. For loading figures
%% from other directories you can use the `import` package
%%   \usepackage{import}
%% and then include the figures with
%%   \import{<path to file>}{<filename>.pgf}
%%
%% Matplotlib used the following preamble
%%   \usepackage{gensymb}
%%   \usepackage{fontspec}
%%   \setmainfont{DejaVu Serif}
%%   \setsansfont{Arial}
%%   \setmonofont{DejaVu Sans Mono}
%%
\begingroup%
\makeatletter%
\begin{pgfpicture}%
\pgfpathrectangle{\pgfpointorigin}{\pgfqpoint{2.200000in}{3.000000in}}%
\pgfusepath{use as bounding box, clip}%
\begin{pgfscope}%
\pgfsetbuttcap%
\pgfsetmiterjoin%
\definecolor{currentfill}{rgb}{1.000000,1.000000,1.000000}%
\pgfsetfillcolor{currentfill}%
\pgfsetlinewidth{0.000000pt}%
\definecolor{currentstroke}{rgb}{1.000000,1.000000,1.000000}%
\pgfsetstrokecolor{currentstroke}%
\pgfsetdash{}{0pt}%
\pgfpathmoveto{\pgfqpoint{0.000000in}{0.000000in}}%
\pgfpathlineto{\pgfqpoint{2.200000in}{0.000000in}}%
\pgfpathlineto{\pgfqpoint{2.200000in}{3.000000in}}%
\pgfpathlineto{\pgfqpoint{0.000000in}{3.000000in}}%
\pgfpathclose%
\pgfusepath{fill}%
\end{pgfscope}%
\begin{pgfscope}%
\pgfsetbuttcap%
\pgfsetmiterjoin%
\definecolor{currentfill}{rgb}{0.917647,0.917647,0.949020}%
\pgfsetfillcolor{currentfill}%
\pgfsetlinewidth{0.000000pt}%
\definecolor{currentstroke}{rgb}{0.000000,0.000000,0.000000}%
\pgfsetstrokecolor{currentstroke}%
\pgfsetstrokeopacity{0.000000}%
\pgfsetdash{}{0pt}%
\pgfpathmoveto{\pgfqpoint{0.275000in}{0.375000in}}%
\pgfpathlineto{\pgfqpoint{1.980000in}{0.375000in}}%
\pgfpathlineto{\pgfqpoint{1.980000in}{2.640000in}}%
\pgfpathlineto{\pgfqpoint{0.275000in}{2.640000in}}%
\pgfpathclose%
\pgfusepath{fill}%
\end{pgfscope}%
\begin{pgfscope}%
\pgfpathrectangle{\pgfqpoint{0.275000in}{0.375000in}}{\pgfqpoint{1.705000in}{2.265000in}}%
\pgfusepath{clip}%
\pgfsetroundcap%
\pgfsetroundjoin%
\pgfsetlinewidth{0.803000pt}%
\definecolor{currentstroke}{rgb}{1.000000,1.000000,1.000000}%
\pgfsetstrokecolor{currentstroke}%
\pgfsetdash{}{0pt}%
\pgfpathmoveto{\pgfqpoint{0.352500in}{0.375000in}}%
\pgfpathlineto{\pgfqpoint{0.352500in}{2.640000in}}%
\pgfusepath{stroke}%
\end{pgfscope}%
\begin{pgfscope}%
\definecolor{textcolor}{rgb}{0.150000,0.150000,0.150000}%
\pgfsetstrokecolor{textcolor}%
\pgfsetfillcolor{textcolor}%
\pgftext[x=0.352500in,y=0.326389in,,top]{\color{textcolor}\rmfamily\fontsize{8.000000}{9.600000}\selectfont \(\displaystyle 0.00\)}%
\end{pgfscope}%
\begin{pgfscope}%
\pgfpathrectangle{\pgfqpoint{0.275000in}{0.375000in}}{\pgfqpoint{1.705000in}{2.265000in}}%
\pgfusepath{clip}%
\pgfsetroundcap%
\pgfsetroundjoin%
\pgfsetlinewidth{0.803000pt}%
\definecolor{currentstroke}{rgb}{1.000000,1.000000,1.000000}%
\pgfsetstrokecolor{currentstroke}%
\pgfsetdash{}{0pt}%
\pgfpathmoveto{\pgfqpoint{0.740000in}{0.375000in}}%
\pgfpathlineto{\pgfqpoint{0.740000in}{2.640000in}}%
\pgfusepath{stroke}%
\end{pgfscope}%
\begin{pgfscope}%
\definecolor{textcolor}{rgb}{0.150000,0.150000,0.150000}%
\pgfsetstrokecolor{textcolor}%
\pgfsetfillcolor{textcolor}%
\pgftext[x=0.740000in,y=0.326389in,,top]{\color{textcolor}\rmfamily\fontsize{8.000000}{9.600000}\selectfont \(\displaystyle 0.25\)}%
\end{pgfscope}%
\begin{pgfscope}%
\pgfpathrectangle{\pgfqpoint{0.275000in}{0.375000in}}{\pgfqpoint{1.705000in}{2.265000in}}%
\pgfusepath{clip}%
\pgfsetroundcap%
\pgfsetroundjoin%
\pgfsetlinewidth{0.803000pt}%
\definecolor{currentstroke}{rgb}{1.000000,1.000000,1.000000}%
\pgfsetstrokecolor{currentstroke}%
\pgfsetdash{}{0pt}%
\pgfpathmoveto{\pgfqpoint{1.127500in}{0.375000in}}%
\pgfpathlineto{\pgfqpoint{1.127500in}{2.640000in}}%
\pgfusepath{stroke}%
\end{pgfscope}%
\begin{pgfscope}%
\definecolor{textcolor}{rgb}{0.150000,0.150000,0.150000}%
\pgfsetstrokecolor{textcolor}%
\pgfsetfillcolor{textcolor}%
\pgftext[x=1.127500in,y=0.326389in,,top]{\color{textcolor}\rmfamily\fontsize{8.000000}{9.600000}\selectfont \(\displaystyle 0.50\)}%
\end{pgfscope}%
\begin{pgfscope}%
\pgfpathrectangle{\pgfqpoint{0.275000in}{0.375000in}}{\pgfqpoint{1.705000in}{2.265000in}}%
\pgfusepath{clip}%
\pgfsetroundcap%
\pgfsetroundjoin%
\pgfsetlinewidth{0.803000pt}%
\definecolor{currentstroke}{rgb}{1.000000,1.000000,1.000000}%
\pgfsetstrokecolor{currentstroke}%
\pgfsetdash{}{0pt}%
\pgfpathmoveto{\pgfqpoint{1.515000in}{0.375000in}}%
\pgfpathlineto{\pgfqpoint{1.515000in}{2.640000in}}%
\pgfusepath{stroke}%
\end{pgfscope}%
\begin{pgfscope}%
\definecolor{textcolor}{rgb}{0.150000,0.150000,0.150000}%
\pgfsetstrokecolor{textcolor}%
\pgfsetfillcolor{textcolor}%
\pgftext[x=1.515000in,y=0.326389in,,top]{\color{textcolor}\rmfamily\fontsize{8.000000}{9.600000}\selectfont \(\displaystyle 0.75\)}%
\end{pgfscope}%
\begin{pgfscope}%
\pgfpathrectangle{\pgfqpoint{0.275000in}{0.375000in}}{\pgfqpoint{1.705000in}{2.265000in}}%
\pgfusepath{clip}%
\pgfsetroundcap%
\pgfsetroundjoin%
\pgfsetlinewidth{0.803000pt}%
\definecolor{currentstroke}{rgb}{1.000000,1.000000,1.000000}%
\pgfsetstrokecolor{currentstroke}%
\pgfsetdash{}{0pt}%
\pgfpathmoveto{\pgfqpoint{1.902500in}{0.375000in}}%
\pgfpathlineto{\pgfqpoint{1.902500in}{2.640000in}}%
\pgfusepath{stroke}%
\end{pgfscope}%
\begin{pgfscope}%
\definecolor{textcolor}{rgb}{0.150000,0.150000,0.150000}%
\pgfsetstrokecolor{textcolor}%
\pgfsetfillcolor{textcolor}%
\pgftext[x=1.902500in,y=0.326389in,,top]{\color{textcolor}\rmfamily\fontsize{8.000000}{9.600000}\selectfont \(\displaystyle 1.00\)}%
\end{pgfscope}%
\begin{pgfscope}%
\pgfpathrectangle{\pgfqpoint{0.275000in}{0.375000in}}{\pgfqpoint{1.705000in}{2.265000in}}%
\pgfusepath{clip}%
\pgfsetroundcap%
\pgfsetroundjoin%
\pgfsetlinewidth{0.803000pt}%
\definecolor{currentstroke}{rgb}{1.000000,1.000000,1.000000}%
\pgfsetstrokecolor{currentstroke}%
\pgfsetdash{}{0pt}%
\pgfpathmoveto{\pgfqpoint{0.275000in}{0.393824in}}%
\pgfpathlineto{\pgfqpoint{1.980000in}{0.393824in}}%
\pgfusepath{stroke}%
\end{pgfscope}%
\begin{pgfscope}%
\definecolor{textcolor}{rgb}{0.150000,0.150000,0.150000}%
\pgfsetstrokecolor{textcolor}%
\pgfsetfillcolor{textcolor}%
\pgftext[x=0.075538in,y=0.351615in,left,base]{\color{textcolor}\rmfamily\fontsize{8.000000}{9.600000}\selectfont \(\displaystyle -4\)}%
\end{pgfscope}%
\begin{pgfscope}%
\pgfpathrectangle{\pgfqpoint{0.275000in}{0.375000in}}{\pgfqpoint{1.705000in}{2.265000in}}%
\pgfusepath{clip}%
\pgfsetroundcap%
\pgfsetroundjoin%
\pgfsetlinewidth{0.803000pt}%
\definecolor{currentstroke}{rgb}{1.000000,1.000000,1.000000}%
\pgfsetstrokecolor{currentstroke}%
\pgfsetdash{}{0pt}%
\pgfpathmoveto{\pgfqpoint{0.275000in}{0.707393in}}%
\pgfpathlineto{\pgfqpoint{1.980000in}{0.707393in}}%
\pgfusepath{stroke}%
\end{pgfscope}%
\begin{pgfscope}%
\definecolor{textcolor}{rgb}{0.150000,0.150000,0.150000}%
\pgfsetstrokecolor{textcolor}%
\pgfsetfillcolor{textcolor}%
\pgftext[x=0.075538in,y=0.665184in,left,base]{\color{textcolor}\rmfamily\fontsize{8.000000}{9.600000}\selectfont \(\displaystyle -3\)}%
\end{pgfscope}%
\begin{pgfscope}%
\pgfpathrectangle{\pgfqpoint{0.275000in}{0.375000in}}{\pgfqpoint{1.705000in}{2.265000in}}%
\pgfusepath{clip}%
\pgfsetroundcap%
\pgfsetroundjoin%
\pgfsetlinewidth{0.803000pt}%
\definecolor{currentstroke}{rgb}{1.000000,1.000000,1.000000}%
\pgfsetstrokecolor{currentstroke}%
\pgfsetdash{}{0pt}%
\pgfpathmoveto{\pgfqpoint{0.275000in}{1.020962in}}%
\pgfpathlineto{\pgfqpoint{1.980000in}{1.020962in}}%
\pgfusepath{stroke}%
\end{pgfscope}%
\begin{pgfscope}%
\definecolor{textcolor}{rgb}{0.150000,0.150000,0.150000}%
\pgfsetstrokecolor{textcolor}%
\pgfsetfillcolor{textcolor}%
\pgftext[x=0.075538in,y=0.978753in,left,base]{\color{textcolor}\rmfamily\fontsize{8.000000}{9.600000}\selectfont \(\displaystyle -2\)}%
\end{pgfscope}%
\begin{pgfscope}%
\pgfpathrectangle{\pgfqpoint{0.275000in}{0.375000in}}{\pgfqpoint{1.705000in}{2.265000in}}%
\pgfusepath{clip}%
\pgfsetroundcap%
\pgfsetroundjoin%
\pgfsetlinewidth{0.803000pt}%
\definecolor{currentstroke}{rgb}{1.000000,1.000000,1.000000}%
\pgfsetstrokecolor{currentstroke}%
\pgfsetdash{}{0pt}%
\pgfpathmoveto{\pgfqpoint{0.275000in}{1.334531in}}%
\pgfpathlineto{\pgfqpoint{1.980000in}{1.334531in}}%
\pgfusepath{stroke}%
\end{pgfscope}%
\begin{pgfscope}%
\definecolor{textcolor}{rgb}{0.150000,0.150000,0.150000}%
\pgfsetstrokecolor{textcolor}%
\pgfsetfillcolor{textcolor}%
\pgftext[x=0.075538in,y=1.292322in,left,base]{\color{textcolor}\rmfamily\fontsize{8.000000}{9.600000}\selectfont \(\displaystyle -1\)}%
\end{pgfscope}%
\begin{pgfscope}%
\pgfpathrectangle{\pgfqpoint{0.275000in}{0.375000in}}{\pgfqpoint{1.705000in}{2.265000in}}%
\pgfusepath{clip}%
\pgfsetroundcap%
\pgfsetroundjoin%
\pgfsetlinewidth{0.803000pt}%
\definecolor{currentstroke}{rgb}{1.000000,1.000000,1.000000}%
\pgfsetstrokecolor{currentstroke}%
\pgfsetdash{}{0pt}%
\pgfpathmoveto{\pgfqpoint{0.275000in}{1.648100in}}%
\pgfpathlineto{\pgfqpoint{1.980000in}{1.648100in}}%
\pgfusepath{stroke}%
\end{pgfscope}%
\begin{pgfscope}%
\definecolor{textcolor}{rgb}{0.150000,0.150000,0.150000}%
\pgfsetstrokecolor{textcolor}%
\pgfsetfillcolor{textcolor}%
\pgftext[x=0.167360in,y=1.605890in,left,base]{\color{textcolor}\rmfamily\fontsize{8.000000}{9.600000}\selectfont \(\displaystyle 0\)}%
\end{pgfscope}%
\begin{pgfscope}%
\pgfpathrectangle{\pgfqpoint{0.275000in}{0.375000in}}{\pgfqpoint{1.705000in}{2.265000in}}%
\pgfusepath{clip}%
\pgfsetroundcap%
\pgfsetroundjoin%
\pgfsetlinewidth{0.803000pt}%
\definecolor{currentstroke}{rgb}{1.000000,1.000000,1.000000}%
\pgfsetstrokecolor{currentstroke}%
\pgfsetdash{}{0pt}%
\pgfpathmoveto{\pgfqpoint{0.275000in}{1.961668in}}%
\pgfpathlineto{\pgfqpoint{1.980000in}{1.961668in}}%
\pgfusepath{stroke}%
\end{pgfscope}%
\begin{pgfscope}%
\definecolor{textcolor}{rgb}{0.150000,0.150000,0.150000}%
\pgfsetstrokecolor{textcolor}%
\pgfsetfillcolor{textcolor}%
\pgftext[x=0.167360in,y=1.919459in,left,base]{\color{textcolor}\rmfamily\fontsize{8.000000}{9.600000}\selectfont \(\displaystyle 1\)}%
\end{pgfscope}%
\begin{pgfscope}%
\pgfpathrectangle{\pgfqpoint{0.275000in}{0.375000in}}{\pgfqpoint{1.705000in}{2.265000in}}%
\pgfusepath{clip}%
\pgfsetroundcap%
\pgfsetroundjoin%
\pgfsetlinewidth{0.803000pt}%
\definecolor{currentstroke}{rgb}{1.000000,1.000000,1.000000}%
\pgfsetstrokecolor{currentstroke}%
\pgfsetdash{}{0pt}%
\pgfpathmoveto{\pgfqpoint{0.275000in}{2.275237in}}%
\pgfpathlineto{\pgfqpoint{1.980000in}{2.275237in}}%
\pgfusepath{stroke}%
\end{pgfscope}%
\begin{pgfscope}%
\definecolor{textcolor}{rgb}{0.150000,0.150000,0.150000}%
\pgfsetstrokecolor{textcolor}%
\pgfsetfillcolor{textcolor}%
\pgftext[x=0.167360in,y=2.233028in,left,base]{\color{textcolor}\rmfamily\fontsize{8.000000}{9.600000}\selectfont \(\displaystyle 2\)}%
\end{pgfscope}%
\begin{pgfscope}%
\pgfpathrectangle{\pgfqpoint{0.275000in}{0.375000in}}{\pgfqpoint{1.705000in}{2.265000in}}%
\pgfusepath{clip}%
\pgfsetroundcap%
\pgfsetroundjoin%
\pgfsetlinewidth{0.803000pt}%
\definecolor{currentstroke}{rgb}{1.000000,1.000000,1.000000}%
\pgfsetstrokecolor{currentstroke}%
\pgfsetdash{}{0pt}%
\pgfpathmoveto{\pgfqpoint{0.275000in}{2.588806in}}%
\pgfpathlineto{\pgfqpoint{1.980000in}{2.588806in}}%
\pgfusepath{stroke}%
\end{pgfscope}%
\begin{pgfscope}%
\definecolor{textcolor}{rgb}{0.150000,0.150000,0.150000}%
\pgfsetstrokecolor{textcolor}%
\pgfsetfillcolor{textcolor}%
\pgftext[x=0.167360in,y=2.546597in,left,base]{\color{textcolor}\rmfamily\fontsize{8.000000}{9.600000}\selectfont \(\displaystyle 3\)}%
\end{pgfscope}%
\begin{pgfscope}%
\pgfpathrectangle{\pgfqpoint{0.275000in}{0.375000in}}{\pgfqpoint{1.705000in}{2.265000in}}%
\pgfusepath{clip}%
\pgfsetroundcap%
\pgfsetroundjoin%
\pgfsetlinewidth{1.505625pt}%
\definecolor{currentstroke}{rgb}{0.121569,0.466667,0.705882}%
\pgfsetstrokecolor{currentstroke}%
\pgfsetdash{}{0pt}%
\pgfpathmoveto{\pgfqpoint{0.352500in}{1.347402in}}%
\pgfpathlineto{\pgfqpoint{0.371137in}{1.402784in}}%
\pgfpathlineto{\pgfqpoint{0.386668in}{1.438748in}}%
\pgfpathlineto{\pgfqpoint{0.399093in}{1.460826in}}%
\pgfpathlineto{\pgfqpoint{0.414624in}{1.481796in}}%
\pgfpathlineto{\pgfqpoint{0.427049in}{1.494179in}}%
\pgfpathlineto{\pgfqpoint{0.439474in}{1.502117in}}%
\pgfpathlineto{\pgfqpoint{0.448793in}{1.504032in}}%
\pgfpathlineto{\pgfqpoint{0.455005in}{1.502781in}}%
\pgfpathlineto{\pgfqpoint{0.461217in}{1.499121in}}%
\pgfpathlineto{\pgfqpoint{0.467430in}{1.492778in}}%
\pgfpathlineto{\pgfqpoint{0.476748in}{1.477866in}}%
\pgfpathlineto{\pgfqpoint{0.486067in}{1.456478in}}%
\pgfpathlineto{\pgfqpoint{0.498492in}{1.419277in}}%
\pgfpathlineto{\pgfqpoint{0.538873in}{1.286700in}}%
\pgfpathlineto{\pgfqpoint{0.548191in}{1.268579in}}%
\pgfpathlineto{\pgfqpoint{0.554404in}{1.261305in}}%
\pgfpathlineto{\pgfqpoint{0.560616in}{1.258141in}}%
\pgfpathlineto{\pgfqpoint{0.566829in}{1.259102in}}%
\pgfpathlineto{\pgfqpoint{0.573041in}{1.264022in}}%
\pgfpathlineto{\pgfqpoint{0.579254in}{1.272566in}}%
\pgfpathlineto{\pgfqpoint{0.588572in}{1.291110in}}%
\pgfpathlineto{\pgfqpoint{0.604103in}{1.332153in}}%
\pgfpathlineto{\pgfqpoint{0.638272in}{1.426592in}}%
\pgfpathlineto{\pgfqpoint{0.660015in}{1.474470in}}%
\pgfpathlineto{\pgfqpoint{0.684865in}{1.528936in}}%
\pgfpathlineto{\pgfqpoint{0.703502in}{1.579107in}}%
\pgfpathlineto{\pgfqpoint{0.728352in}{1.658216in}}%
\pgfpathlineto{\pgfqpoint{0.753201in}{1.736005in}}%
\pgfpathlineto{\pgfqpoint{0.765626in}{1.766997in}}%
\pgfpathlineto{\pgfqpoint{0.774945in}{1.783891in}}%
\pgfpathlineto{\pgfqpoint{0.781157in}{1.791204in}}%
\pgfpathlineto{\pgfqpoint{0.787370in}{1.794762in}}%
\pgfpathlineto{\pgfqpoint{0.793582in}{1.794076in}}%
\pgfpathlineto{\pgfqpoint{0.799795in}{1.788709in}}%
\pgfpathlineto{\pgfqpoint{0.806007in}{1.778315in}}%
\pgfpathlineto{\pgfqpoint{0.812219in}{1.762683in}}%
\pgfpathlineto{\pgfqpoint{0.821538in}{1.729388in}}%
\pgfpathlineto{\pgfqpoint{0.833963in}{1.668148in}}%
\pgfpathlineto{\pgfqpoint{0.852600in}{1.552316in}}%
\pgfpathlineto{\pgfqpoint{0.871237in}{1.437295in}}%
\pgfpathlineto{\pgfqpoint{0.880556in}{1.392438in}}%
\pgfpathlineto{\pgfqpoint{0.889875in}{1.362038in}}%
\pgfpathlineto{\pgfqpoint{0.896087in}{1.351445in}}%
\pgfpathlineto{\pgfqpoint{0.899193in}{1.349296in}}%
\pgfpathlineto{\pgfqpoint{0.902300in}{1.349305in}}%
\pgfpathlineto{\pgfqpoint{0.905406in}{1.351491in}}%
\pgfpathlineto{\pgfqpoint{0.911618in}{1.362345in}}%
\pgfpathlineto{\pgfqpoint{0.917831in}{1.381543in}}%
\pgfpathlineto{\pgfqpoint{0.927149in}{1.424449in}}%
\pgfpathlineto{\pgfqpoint{0.939574in}{1.502180in}}%
\pgfpathlineto{\pgfqpoint{0.979955in}{1.780106in}}%
\pgfpathlineto{\pgfqpoint{0.989274in}{1.822858in}}%
\pgfpathlineto{\pgfqpoint{0.998592in}{1.852363in}}%
\pgfpathlineto{\pgfqpoint{1.004805in}{1.864758in}}%
\pgfpathlineto{\pgfqpoint{1.011017in}{1.871818in}}%
\pgfpathlineto{\pgfqpoint{1.017229in}{1.874241in}}%
\pgfpathlineto{\pgfqpoint{1.023442in}{1.872919in}}%
\pgfpathlineto{\pgfqpoint{1.032761in}{1.866172in}}%
\pgfpathlineto{\pgfqpoint{1.054504in}{1.846798in}}%
\pgfpathlineto{\pgfqpoint{1.060716in}{1.844437in}}%
\pgfpathlineto{\pgfqpoint{1.066929in}{1.844616in}}%
\pgfpathlineto{\pgfqpoint{1.073141in}{1.847580in}}%
\pgfpathlineto{\pgfqpoint{1.079354in}{1.853335in}}%
\pgfpathlineto{\pgfqpoint{1.088672in}{1.866640in}}%
\pgfpathlineto{\pgfqpoint{1.104203in}{1.896809in}}%
\pgfpathlineto{\pgfqpoint{1.122841in}{1.932035in}}%
\pgfpathlineto{\pgfqpoint{1.132159in}{1.943465in}}%
\pgfpathlineto{\pgfqpoint{1.138372in}{1.947521in}}%
\pgfpathlineto{\pgfqpoint{1.144584in}{1.948345in}}%
\pgfpathlineto{\pgfqpoint{1.150797in}{1.945784in}}%
\pgfpathlineto{\pgfqpoint{1.157009in}{1.939829in}}%
\pgfpathlineto{\pgfqpoint{1.166328in}{1.924798in}}%
\pgfpathlineto{\pgfqpoint{1.175646in}{1.903143in}}%
\pgfpathlineto{\pgfqpoint{1.188071in}{1.865727in}}%
\pgfpathlineto{\pgfqpoint{1.206708in}{1.796639in}}%
\pgfpathlineto{\pgfqpoint{1.250195in}{1.626754in}}%
\pgfpathlineto{\pgfqpoint{1.262620in}{1.593686in}}%
\pgfpathlineto{\pgfqpoint{1.271939in}{1.578591in}}%
\pgfpathlineto{\pgfqpoint{1.278151in}{1.573953in}}%
\pgfpathlineto{\pgfqpoint{1.284364in}{1.573943in}}%
\pgfpathlineto{\pgfqpoint{1.290576in}{1.578618in}}%
\pgfpathlineto{\pgfqpoint{1.296789in}{1.587855in}}%
\pgfpathlineto{\pgfqpoint{1.306107in}{1.609549in}}%
\pgfpathlineto{\pgfqpoint{1.318532in}{1.650265in}}%
\pgfpathlineto{\pgfqpoint{1.343382in}{1.749919in}}%
\pgfpathlineto{\pgfqpoint{1.358913in}{1.807157in}}%
\pgfpathlineto{\pgfqpoint{1.371338in}{1.842950in}}%
\pgfpathlineto{\pgfqpoint{1.380656in}{1.862437in}}%
\pgfpathlineto{\pgfqpoint{1.389975in}{1.875230in}}%
\pgfpathlineto{\pgfqpoint{1.396187in}{1.880045in}}%
\pgfpathlineto{\pgfqpoint{1.402400in}{1.881948in}}%
\pgfpathlineto{\pgfqpoint{1.408612in}{1.881009in}}%
\pgfpathlineto{\pgfqpoint{1.414825in}{1.877295in}}%
\pgfpathlineto{\pgfqpoint{1.424143in}{1.866639in}}%
\pgfpathlineto{\pgfqpoint{1.433462in}{1.849995in}}%
\pgfpathlineto{\pgfqpoint{1.442781in}{1.827492in}}%
\pgfpathlineto{\pgfqpoint{1.455205in}{1.788847in}}%
\pgfpathlineto{\pgfqpoint{1.470736in}{1.729029in}}%
\pgfpathlineto{\pgfqpoint{1.504905in}{1.591382in}}%
\pgfpathlineto{\pgfqpoint{1.514223in}{1.565926in}}%
\pgfpathlineto{\pgfqpoint{1.520436in}{1.554723in}}%
\pgfpathlineto{\pgfqpoint{1.526648in}{1.548899in}}%
\pgfpathlineto{\pgfqpoint{1.532861in}{1.548886in}}%
\pgfpathlineto{\pgfqpoint{1.539073in}{1.554875in}}%
\pgfpathlineto{\pgfqpoint{1.545286in}{1.566794in}}%
\pgfpathlineto{\pgfqpoint{1.554604in}{1.594985in}}%
\pgfpathlineto{\pgfqpoint{1.567029in}{1.648154in}}%
\pgfpathlineto{\pgfqpoint{1.613622in}{1.873632in}}%
\pgfpathlineto{\pgfqpoint{1.626047in}{1.910721in}}%
\pgfpathlineto{\pgfqpoint{1.635366in}{1.929630in}}%
\pgfpathlineto{\pgfqpoint{1.644684in}{1.942000in}}%
\pgfpathlineto{\pgfqpoint{1.654003in}{1.949279in}}%
\pgfpathlineto{\pgfqpoint{1.663322in}{1.952953in}}%
\pgfpathlineto{\pgfqpoint{1.675746in}{1.954264in}}%
\pgfpathlineto{\pgfqpoint{1.688171in}{1.952736in}}%
\pgfpathlineto{\pgfqpoint{1.700596in}{1.948188in}}%
\pgfpathlineto{\pgfqpoint{1.709915in}{1.942026in}}%
\pgfpathlineto{\pgfqpoint{1.719233in}{1.932775in}}%
\pgfpathlineto{\pgfqpoint{1.731658in}{1.914883in}}%
\pgfpathlineto{\pgfqpoint{1.744083in}{1.890731in}}%
\pgfpathlineto{\pgfqpoint{1.762720in}{1.846537in}}%
\pgfpathlineto{\pgfqpoint{1.784464in}{1.795379in}}%
\pgfpathlineto{\pgfqpoint{1.796889in}{1.772433in}}%
\pgfpathlineto{\pgfqpoint{1.809314in}{1.756148in}}%
\pgfpathlineto{\pgfqpoint{1.821738in}{1.745901in}}%
\pgfpathlineto{\pgfqpoint{1.849694in}{1.727860in}}%
\pgfpathlineto{\pgfqpoint{1.859013in}{1.715863in}}%
\pgfpathlineto{\pgfqpoint{1.868332in}{1.697191in}}%
\pgfpathlineto{\pgfqpoint{1.877650in}{1.670233in}}%
\pgfpathlineto{\pgfqpoint{1.886969in}{1.634206in}}%
\pgfpathlineto{\pgfqpoint{1.899394in}{1.572677in}}%
\pgfpathlineto{\pgfqpoint{1.902500in}{1.555229in}}%
\pgfpathlineto{\pgfqpoint{1.902500in}{1.555229in}}%
\pgfusepath{stroke}%
\end{pgfscope}%
\begin{pgfscope}%
\pgfpathrectangle{\pgfqpoint{0.275000in}{0.375000in}}{\pgfqpoint{1.705000in}{2.265000in}}%
\pgfusepath{clip}%
\pgfsetroundcap%
\pgfsetroundjoin%
\pgfsetlinewidth{1.505625pt}%
\definecolor{currentstroke}{rgb}{1.000000,0.498039,0.054902}%
\pgfsetstrokecolor{currentstroke}%
\pgfsetdash{}{0pt}%
\pgfpathmoveto{\pgfqpoint{0.352500in}{1.826429in}}%
\pgfpathlineto{\pgfqpoint{0.364925in}{1.886703in}}%
\pgfpathlineto{\pgfqpoint{0.380456in}{1.980821in}}%
\pgfpathlineto{\pgfqpoint{0.402199in}{2.137196in}}%
\pgfpathlineto{\pgfqpoint{0.420837in}{2.266994in}}%
\pgfpathlineto{\pgfqpoint{0.433262in}{2.333847in}}%
\pgfpathlineto{\pgfqpoint{0.442580in}{2.366959in}}%
\pgfpathlineto{\pgfqpoint{0.448793in}{2.379225in}}%
\pgfpathlineto{\pgfqpoint{0.451899in}{2.382174in}}%
\pgfpathlineto{\pgfqpoint{0.455005in}{2.382931in}}%
\pgfpathlineto{\pgfqpoint{0.458111in}{2.381460in}}%
\pgfpathlineto{\pgfqpoint{0.464324in}{2.371786in}}%
\pgfpathlineto{\pgfqpoint{0.470536in}{2.353210in}}%
\pgfpathlineto{\pgfqpoint{0.479855in}{2.309407in}}%
\pgfpathlineto{\pgfqpoint{0.489173in}{2.248383in}}%
\pgfpathlineto{\pgfqpoint{0.501598in}{2.145683in}}%
\pgfpathlineto{\pgfqpoint{0.526448in}{1.904011in}}%
\pgfpathlineto{\pgfqpoint{0.545085in}{1.732542in}}%
\pgfpathlineto{\pgfqpoint{0.557510in}{1.639749in}}%
\pgfpathlineto{\pgfqpoint{0.569935in}{1.570573in}}%
\pgfpathlineto{\pgfqpoint{0.579254in}{1.535444in}}%
\pgfpathlineto{\pgfqpoint{0.585466in}{1.519917in}}%
\pgfpathlineto{\pgfqpoint{0.591678in}{1.510400in}}%
\pgfpathlineto{\pgfqpoint{0.597891in}{1.506505in}}%
\pgfpathlineto{\pgfqpoint{0.604103in}{1.507761in}}%
\pgfpathlineto{\pgfqpoint{0.610316in}{1.513646in}}%
\pgfpathlineto{\pgfqpoint{0.616528in}{1.523615in}}%
\pgfpathlineto{\pgfqpoint{0.625847in}{1.545004in}}%
\pgfpathlineto{\pgfqpoint{0.638272in}{1.582620in}}%
\pgfpathlineto{\pgfqpoint{0.684865in}{1.738431in}}%
\pgfpathlineto{\pgfqpoint{0.694183in}{1.756021in}}%
\pgfpathlineto{\pgfqpoint{0.700396in}{1.762575in}}%
\pgfpathlineto{\pgfqpoint{0.706608in}{1.764500in}}%
\pgfpathlineto{\pgfqpoint{0.712821in}{1.761552in}}%
\pgfpathlineto{\pgfqpoint{0.719033in}{1.753682in}}%
\pgfpathlineto{\pgfqpoint{0.725245in}{1.741051in}}%
\pgfpathlineto{\pgfqpoint{0.734564in}{1.714050in}}%
\pgfpathlineto{\pgfqpoint{0.746989in}{1.666523in}}%
\pgfpathlineto{\pgfqpoint{0.781157in}{1.527348in}}%
\pgfpathlineto{\pgfqpoint{0.790476in}{1.502581in}}%
\pgfpathlineto{\pgfqpoint{0.796688in}{1.491583in}}%
\pgfpathlineto{\pgfqpoint{0.802901in}{1.485345in}}%
\pgfpathlineto{\pgfqpoint{0.809113in}{1.483927in}}%
\pgfpathlineto{\pgfqpoint{0.815326in}{1.487185in}}%
\pgfpathlineto{\pgfqpoint{0.821538in}{1.494786in}}%
\pgfpathlineto{\pgfqpoint{0.830857in}{1.513210in}}%
\pgfpathlineto{\pgfqpoint{0.843282in}{1.547395in}}%
\pgfpathlineto{\pgfqpoint{0.883662in}{1.669518in}}%
\pgfpathlineto{\pgfqpoint{0.896087in}{1.696314in}}%
\pgfpathlineto{\pgfqpoint{0.908512in}{1.716926in}}%
\pgfpathlineto{\pgfqpoint{0.927149in}{1.740845in}}%
\pgfpathlineto{\pgfqpoint{0.958211in}{1.775373in}}%
\pgfpathlineto{\pgfqpoint{0.967530in}{1.782763in}}%
\pgfpathlineto{\pgfqpoint{0.976849in}{1.785987in}}%
\pgfpathlineto{\pgfqpoint{0.983061in}{1.784734in}}%
\pgfpathlineto{\pgfqpoint{0.989274in}{1.779998in}}%
\pgfpathlineto{\pgfqpoint{0.995486in}{1.771197in}}%
\pgfpathlineto{\pgfqpoint{1.001698in}{1.757859in}}%
\pgfpathlineto{\pgfqpoint{1.011017in}{1.728712in}}%
\pgfpathlineto{\pgfqpoint{1.020336in}{1.688459in}}%
\pgfpathlineto{\pgfqpoint{1.032761in}{1.619179in}}%
\pgfpathlineto{\pgfqpoint{1.051398in}{1.493575in}}%
\pgfpathlineto{\pgfqpoint{1.073141in}{1.351317in}}%
\pgfpathlineto{\pgfqpoint{1.082460in}{1.305421in}}%
\pgfpathlineto{\pgfqpoint{1.091779in}{1.274882in}}%
\pgfpathlineto{\pgfqpoint{1.097991in}{1.264758in}}%
\pgfpathlineto{\pgfqpoint{1.101097in}{1.263054in}}%
\pgfpathlineto{\pgfqpoint{1.104203in}{1.263675in}}%
\pgfpathlineto{\pgfqpoint{1.107310in}{1.266665in}}%
\pgfpathlineto{\pgfqpoint{1.113522in}{1.279823in}}%
\pgfpathlineto{\pgfqpoint{1.119734in}{1.302473in}}%
\pgfpathlineto{\pgfqpoint{1.129053in}{1.353352in}}%
\pgfpathlineto{\pgfqpoint{1.138372in}{1.422208in}}%
\pgfpathlineto{\pgfqpoint{1.153903in}{1.566509in}}%
\pgfpathlineto{\pgfqpoint{1.200496in}{2.030342in}}%
\pgfpathlineto{\pgfqpoint{1.216027in}{2.143676in}}%
\pgfpathlineto{\pgfqpoint{1.228452in}{2.212342in}}%
\pgfpathlineto{\pgfqpoint{1.240877in}{2.262347in}}%
\pgfpathlineto{\pgfqpoint{1.250195in}{2.288930in}}%
\pgfpathlineto{\pgfqpoint{1.259514in}{2.307538in}}%
\pgfpathlineto{\pgfqpoint{1.268833in}{2.319571in}}%
\pgfpathlineto{\pgfqpoint{1.278151in}{2.326439in}}%
\pgfpathlineto{\pgfqpoint{1.287470in}{2.329411in}}%
\pgfpathlineto{\pgfqpoint{1.296789in}{2.329465in}}%
\pgfpathlineto{\pgfqpoint{1.309213in}{2.325839in}}%
\pgfpathlineto{\pgfqpoint{1.318532in}{2.320187in}}%
\pgfpathlineto{\pgfqpoint{1.327851in}{2.311193in}}%
\pgfpathlineto{\pgfqpoint{1.337169in}{2.297594in}}%
\pgfpathlineto{\pgfqpoint{1.346488in}{2.277881in}}%
\pgfpathlineto{\pgfqpoint{1.355807in}{2.250606in}}%
\pgfpathlineto{\pgfqpoint{1.365125in}{2.214745in}}%
\pgfpathlineto{\pgfqpoint{1.377550in}{2.153259in}}%
\pgfpathlineto{\pgfqpoint{1.393081in}{2.058270in}}%
\pgfpathlineto{\pgfqpoint{1.424143in}{1.862438in}}%
\pgfpathlineto{\pgfqpoint{1.433462in}{1.820560in}}%
\pgfpathlineto{\pgfqpoint{1.442781in}{1.793549in}}%
\pgfpathlineto{\pgfqpoint{1.448993in}{1.785057in}}%
\pgfpathlineto{\pgfqpoint{1.452099in}{1.783820in}}%
\pgfpathlineto{\pgfqpoint{1.455205in}{1.784607in}}%
\pgfpathlineto{\pgfqpoint{1.461418in}{1.792177in}}%
\pgfpathlineto{\pgfqpoint{1.467630in}{1.807381in}}%
\pgfpathlineto{\pgfqpoint{1.476949in}{1.842797in}}%
\pgfpathlineto{\pgfqpoint{1.489374in}{1.907262in}}%
\pgfpathlineto{\pgfqpoint{1.517330in}{2.062113in}}%
\pgfpathlineto{\pgfqpoint{1.526648in}{2.097431in}}%
\pgfpathlineto{\pgfqpoint{1.532861in}{2.112862in}}%
\pgfpathlineto{\pgfqpoint{1.539073in}{2.121080in}}%
\pgfpathlineto{\pgfqpoint{1.542179in}{2.122409in}}%
\pgfpathlineto{\pgfqpoint{1.545286in}{2.121894in}}%
\pgfpathlineto{\pgfqpoint{1.551498in}{2.115502in}}%
\pgfpathlineto{\pgfqpoint{1.557710in}{2.102478in}}%
\pgfpathlineto{\pgfqpoint{1.567029in}{2.072581in}}%
\pgfpathlineto{\pgfqpoint{1.582560in}{2.006162in}}%
\pgfpathlineto{\pgfqpoint{1.601197in}{1.928818in}}%
\pgfpathlineto{\pgfqpoint{1.610516in}{1.902425in}}%
\pgfpathlineto{\pgfqpoint{1.616728in}{1.891762in}}%
\pgfpathlineto{\pgfqpoint{1.622941in}{1.887282in}}%
\pgfpathlineto{\pgfqpoint{1.626047in}{1.887423in}}%
\pgfpathlineto{\pgfqpoint{1.632260in}{1.892394in}}%
\pgfpathlineto{\pgfqpoint{1.638472in}{1.903263in}}%
\pgfpathlineto{\pgfqpoint{1.647791in}{1.928995in}}%
\pgfpathlineto{\pgfqpoint{1.660215in}{1.975064in}}%
\pgfpathlineto{\pgfqpoint{1.681959in}{2.056178in}}%
\pgfpathlineto{\pgfqpoint{1.691278in}{2.078455in}}%
\pgfpathlineto{\pgfqpoint{1.697490in}{2.086116in}}%
\pgfpathlineto{\pgfqpoint{1.700596in}{2.087468in}}%
\pgfpathlineto{\pgfqpoint{1.703702in}{2.087064in}}%
\pgfpathlineto{\pgfqpoint{1.709915in}{2.080788in}}%
\pgfpathlineto{\pgfqpoint{1.716127in}{2.067066in}}%
\pgfpathlineto{\pgfqpoint{1.722340in}{2.045963in}}%
\pgfpathlineto{\pgfqpoint{1.731658in}{2.001249in}}%
\pgfpathlineto{\pgfqpoint{1.744083in}{1.920670in}}%
\pgfpathlineto{\pgfqpoint{1.759614in}{1.796353in}}%
\pgfpathlineto{\pgfqpoint{1.803101in}{1.434537in}}%
\pgfpathlineto{\pgfqpoint{1.815526in}{1.356729in}}%
\pgfpathlineto{\pgfqpoint{1.824845in}{1.311840in}}%
\pgfpathlineto{\pgfqpoint{1.834163in}{1.280167in}}%
\pgfpathlineto{\pgfqpoint{1.840376in}{1.266890in}}%
\pgfpathlineto{\pgfqpoint{1.846588in}{1.260044in}}%
\pgfpathlineto{\pgfqpoint{1.852801in}{1.259596in}}%
\pgfpathlineto{\pgfqpoint{1.859013in}{1.265346in}}%
\pgfpathlineto{\pgfqpoint{1.865225in}{1.276916in}}%
\pgfpathlineto{\pgfqpoint{1.874544in}{1.303946in}}%
\pgfpathlineto{\pgfqpoint{1.886969in}{1.354024in}}%
\pgfpathlineto{\pgfqpoint{1.902500in}{1.428816in}}%
\pgfpathlineto{\pgfqpoint{1.902500in}{1.428816in}}%
\pgfusepath{stroke}%
\end{pgfscope}%
\begin{pgfscope}%
\pgfpathrectangle{\pgfqpoint{0.275000in}{0.375000in}}{\pgfqpoint{1.705000in}{2.265000in}}%
\pgfusepath{clip}%
\pgfsetroundcap%
\pgfsetroundjoin%
\pgfsetlinewidth{1.505625pt}%
\definecolor{currentstroke}{rgb}{0.172549,0.627451,0.172549}%
\pgfsetstrokecolor{currentstroke}%
\pgfsetdash{}{0pt}%
\pgfpathmoveto{\pgfqpoint{0.352500in}{1.358839in}}%
\pgfpathlineto{\pgfqpoint{0.364925in}{1.348168in}}%
\pgfpathlineto{\pgfqpoint{0.374243in}{1.343651in}}%
\pgfpathlineto{\pgfqpoint{0.383562in}{1.342091in}}%
\pgfpathlineto{\pgfqpoint{0.392881in}{1.343990in}}%
\pgfpathlineto{\pgfqpoint{0.402199in}{1.350176in}}%
\pgfpathlineto{\pgfqpoint{0.411518in}{1.361578in}}%
\pgfpathlineto{\pgfqpoint{0.420837in}{1.378972in}}%
\pgfpathlineto{\pgfqpoint{0.430155in}{1.402748in}}%
\pgfpathlineto{\pgfqpoint{0.442580in}{1.443969in}}%
\pgfpathlineto{\pgfqpoint{0.461217in}{1.520683in}}%
\pgfpathlineto{\pgfqpoint{0.486067in}{1.623739in}}%
\pgfpathlineto{\pgfqpoint{0.498492in}{1.661453in}}%
\pgfpathlineto{\pgfqpoint{0.507811in}{1.679263in}}%
\pgfpathlineto{\pgfqpoint{0.514023in}{1.685421in}}%
\pgfpathlineto{\pgfqpoint{0.520235in}{1.686868in}}%
\pgfpathlineto{\pgfqpoint{0.526448in}{1.683729in}}%
\pgfpathlineto{\pgfqpoint{0.532660in}{1.676350in}}%
\pgfpathlineto{\pgfqpoint{0.541979in}{1.658580in}}%
\pgfpathlineto{\pgfqpoint{0.554404in}{1.626527in}}%
\pgfpathlineto{\pgfqpoint{0.576147in}{1.568362in}}%
\pgfpathlineto{\pgfqpoint{0.585466in}{1.550068in}}%
\pgfpathlineto{\pgfqpoint{0.594785in}{1.538424in}}%
\pgfpathlineto{\pgfqpoint{0.600997in}{1.534618in}}%
\pgfpathlineto{\pgfqpoint{0.607209in}{1.533838in}}%
\pgfpathlineto{\pgfqpoint{0.613422in}{1.535736in}}%
\pgfpathlineto{\pgfqpoint{0.622740in}{1.542461in}}%
\pgfpathlineto{\pgfqpoint{0.656909in}{1.574258in}}%
\pgfpathlineto{\pgfqpoint{0.666227in}{1.577621in}}%
\pgfpathlineto{\pgfqpoint{0.675546in}{1.577932in}}%
\pgfpathlineto{\pgfqpoint{0.706608in}{1.573997in}}%
\pgfpathlineto{\pgfqpoint{0.715927in}{1.578275in}}%
\pgfpathlineto{\pgfqpoint{0.725245in}{1.587772in}}%
\pgfpathlineto{\pgfqpoint{0.734564in}{1.603156in}}%
\pgfpathlineto{\pgfqpoint{0.743883in}{1.624446in}}%
\pgfpathlineto{\pgfqpoint{0.756308in}{1.661038in}}%
\pgfpathlineto{\pgfqpoint{0.771839in}{1.716781in}}%
\pgfpathlineto{\pgfqpoint{0.796688in}{1.818830in}}%
\pgfpathlineto{\pgfqpoint{0.830857in}{1.972533in}}%
\pgfpathlineto{\pgfqpoint{0.896087in}{2.274878in}}%
\pgfpathlineto{\pgfqpoint{0.905406in}{2.302430in}}%
\pgfpathlineto{\pgfqpoint{0.914724in}{2.320152in}}%
\pgfpathlineto{\pgfqpoint{0.920937in}{2.325535in}}%
\pgfpathlineto{\pgfqpoint{0.927149in}{2.325221in}}%
\pgfpathlineto{\pgfqpoint{0.933362in}{2.318887in}}%
\pgfpathlineto{\pgfqpoint{0.939574in}{2.306371in}}%
\pgfpathlineto{\pgfqpoint{0.948893in}{2.276131in}}%
\pgfpathlineto{\pgfqpoint{0.958211in}{2.233105in}}%
\pgfpathlineto{\pgfqpoint{0.970636in}{2.159686in}}%
\pgfpathlineto{\pgfqpoint{1.011017in}{1.900883in}}%
\pgfpathlineto{\pgfqpoint{1.020336in}{1.863542in}}%
\pgfpathlineto{\pgfqpoint{1.026548in}{1.847531in}}%
\pgfpathlineto{\pgfqpoint{1.032761in}{1.839314in}}%
\pgfpathlineto{\pgfqpoint{1.035867in}{1.838217in}}%
\pgfpathlineto{\pgfqpoint{1.038973in}{1.839133in}}%
\pgfpathlineto{\pgfqpoint{1.045185in}{1.846902in}}%
\pgfpathlineto{\pgfqpoint{1.051398in}{1.862203in}}%
\pgfpathlineto{\pgfqpoint{1.060716in}{1.897595in}}%
\pgfpathlineto{\pgfqpoint{1.073141in}{1.962065in}}%
\pgfpathlineto{\pgfqpoint{1.104203in}{2.137218in}}%
\pgfpathlineto{\pgfqpoint{1.113522in}{2.173955in}}%
\pgfpathlineto{\pgfqpoint{1.122841in}{2.197200in}}%
\pgfpathlineto{\pgfqpoint{1.129053in}{2.204414in}}%
\pgfpathlineto{\pgfqpoint{1.132159in}{2.205476in}}%
\pgfpathlineto{\pgfqpoint{1.135266in}{2.204851in}}%
\pgfpathlineto{\pgfqpoint{1.141478in}{2.198672in}}%
\pgfpathlineto{\pgfqpoint{1.147690in}{2.186270in}}%
\pgfpathlineto{\pgfqpoint{1.157009in}{2.157332in}}%
\pgfpathlineto{\pgfqpoint{1.169434in}{2.103676in}}%
\pgfpathlineto{\pgfqpoint{1.194284in}{1.973231in}}%
\pgfpathlineto{\pgfqpoint{1.212921in}{1.884540in}}%
\pgfpathlineto{\pgfqpoint{1.225346in}{1.840579in}}%
\pgfpathlineto{\pgfqpoint{1.234664in}{1.817599in}}%
\pgfpathlineto{\pgfqpoint{1.243983in}{1.803113in}}%
\pgfpathlineto{\pgfqpoint{1.253302in}{1.795929in}}%
\pgfpathlineto{\pgfqpoint{1.262620in}{1.793856in}}%
\pgfpathlineto{\pgfqpoint{1.281258in}{1.792516in}}%
\pgfpathlineto{\pgfqpoint{1.287470in}{1.789015in}}%
\pgfpathlineto{\pgfqpoint{1.293682in}{1.782256in}}%
\pgfpathlineto{\pgfqpoint{1.299895in}{1.771351in}}%
\pgfpathlineto{\pgfqpoint{1.306107in}{1.755586in}}%
\pgfpathlineto{\pgfqpoint{1.315426in}{1.721805in}}%
\pgfpathlineto{\pgfqpoint{1.324744in}{1.675401in}}%
\pgfpathlineto{\pgfqpoint{1.337169in}{1.595280in}}%
\pgfpathlineto{\pgfqpoint{1.355807in}{1.447435in}}%
\pgfpathlineto{\pgfqpoint{1.383763in}{1.221920in}}%
\pgfpathlineto{\pgfqpoint{1.396187in}{1.144034in}}%
\pgfpathlineto{\pgfqpoint{1.405506in}{1.100930in}}%
\pgfpathlineto{\pgfqpoint{1.414825in}{1.072750in}}%
\pgfpathlineto{\pgfqpoint{1.421037in}{1.062521in}}%
\pgfpathlineto{\pgfqpoint{1.427249in}{1.059011in}}%
\pgfpathlineto{\pgfqpoint{1.433462in}{1.061881in}}%
\pgfpathlineto{\pgfqpoint{1.439674in}{1.070607in}}%
\pgfpathlineto{\pgfqpoint{1.448993in}{1.093153in}}%
\pgfpathlineto{\pgfqpoint{1.461418in}{1.136403in}}%
\pgfpathlineto{\pgfqpoint{1.508011in}{1.315931in}}%
\pgfpathlineto{\pgfqpoint{1.520436in}{1.345004in}}%
\pgfpathlineto{\pgfqpoint{1.529755in}{1.359905in}}%
\pgfpathlineto{\pgfqpoint{1.542179in}{1.372883in}}%
\pgfpathlineto{\pgfqpoint{1.567029in}{1.394284in}}%
\pgfpathlineto{\pgfqpoint{1.576348in}{1.408138in}}%
\pgfpathlineto{\pgfqpoint{1.585666in}{1.428269in}}%
\pgfpathlineto{\pgfqpoint{1.594985in}{1.455647in}}%
\pgfpathlineto{\pgfqpoint{1.607410in}{1.503244in}}%
\pgfpathlineto{\pgfqpoint{1.626047in}{1.591607in}}%
\pgfpathlineto{\pgfqpoint{1.650897in}{1.709359in}}%
\pgfpathlineto{\pgfqpoint{1.663322in}{1.753460in}}%
\pgfpathlineto{\pgfqpoint{1.672640in}{1.776088in}}%
\pgfpathlineto{\pgfqpoint{1.678853in}{1.785644in}}%
\pgfpathlineto{\pgfqpoint{1.685065in}{1.790696in}}%
\pgfpathlineto{\pgfqpoint{1.691278in}{1.791356in}}%
\pgfpathlineto{\pgfqpoint{1.697490in}{1.787879in}}%
\pgfpathlineto{\pgfqpoint{1.703702in}{1.780638in}}%
\pgfpathlineto{\pgfqpoint{1.713021in}{1.763776in}}%
\pgfpathlineto{\pgfqpoint{1.725446in}{1.733039in}}%
\pgfpathlineto{\pgfqpoint{1.772039in}{1.605714in}}%
\pgfpathlineto{\pgfqpoint{1.784464in}{1.583228in}}%
\pgfpathlineto{\pgfqpoint{1.793783in}{1.571633in}}%
\pgfpathlineto{\pgfqpoint{1.803101in}{1.564831in}}%
\pgfpathlineto{\pgfqpoint{1.812420in}{1.562863in}}%
\pgfpathlineto{\pgfqpoint{1.821738in}{1.565551in}}%
\pgfpathlineto{\pgfqpoint{1.831057in}{1.572430in}}%
\pgfpathlineto{\pgfqpoint{1.843482in}{1.586629in}}%
\pgfpathlineto{\pgfqpoint{1.871438in}{1.622303in}}%
\pgfpathlineto{\pgfqpoint{1.880757in}{1.628422in}}%
\pgfpathlineto{\pgfqpoint{1.886969in}{1.629176in}}%
\pgfpathlineto{\pgfqpoint{1.893181in}{1.626715in}}%
\pgfpathlineto{\pgfqpoint{1.899394in}{1.620685in}}%
\pgfpathlineto{\pgfqpoint{1.902500in}{1.616256in}}%
\pgfpathlineto{\pgfqpoint{1.902500in}{1.616256in}}%
\pgfusepath{stroke}%
\end{pgfscope}%
\begin{pgfscope}%
\pgfpathrectangle{\pgfqpoint{0.275000in}{0.375000in}}{\pgfqpoint{1.705000in}{2.265000in}}%
\pgfusepath{clip}%
\pgfsetroundcap%
\pgfsetroundjoin%
\pgfsetlinewidth{1.505625pt}%
\definecolor{currentstroke}{rgb}{0.839216,0.152941,0.156863}%
\pgfsetstrokecolor{currentstroke}%
\pgfsetdash{}{0pt}%
\pgfpathmoveto{\pgfqpoint{0.352500in}{1.718245in}}%
\pgfpathlineto{\pgfqpoint{0.371137in}{1.552224in}}%
\pgfpathlineto{\pgfqpoint{0.380456in}{1.489208in}}%
\pgfpathlineto{\pgfqpoint{0.389775in}{1.446365in}}%
\pgfpathlineto{\pgfqpoint{0.395987in}{1.430919in}}%
\pgfpathlineto{\pgfqpoint{0.399093in}{1.427437in}}%
\pgfpathlineto{\pgfqpoint{0.402199in}{1.426870in}}%
\pgfpathlineto{\pgfqpoint{0.405306in}{1.429259in}}%
\pgfpathlineto{\pgfqpoint{0.411518in}{1.442949in}}%
\pgfpathlineto{\pgfqpoint{0.417730in}{1.468370in}}%
\pgfpathlineto{\pgfqpoint{0.427049in}{1.527333in}}%
\pgfpathlineto{\pgfqpoint{0.436368in}{1.608463in}}%
\pgfpathlineto{\pgfqpoint{0.448793in}{1.743209in}}%
\pgfpathlineto{\pgfqpoint{0.498492in}{2.322322in}}%
\pgfpathlineto{\pgfqpoint{0.510917in}{2.418576in}}%
\pgfpathlineto{\pgfqpoint{0.520235in}{2.470729in}}%
\pgfpathlineto{\pgfqpoint{0.529554in}{2.506345in}}%
\pgfpathlineto{\pgfqpoint{0.535767in}{2.521776in}}%
\pgfpathlineto{\pgfqpoint{0.541979in}{2.531431in}}%
\pgfpathlineto{\pgfqpoint{0.548191in}{2.536198in}}%
\pgfpathlineto{\pgfqpoint{0.554404in}{2.537031in}}%
\pgfpathlineto{\pgfqpoint{0.560616in}{2.534904in}}%
\pgfpathlineto{\pgfqpoint{0.569935in}{2.528204in}}%
\pgfpathlineto{\pgfqpoint{0.594785in}{2.507789in}}%
\pgfpathlineto{\pgfqpoint{0.604103in}{2.503797in}}%
\pgfpathlineto{\pgfqpoint{0.613422in}{2.502542in}}%
\pgfpathlineto{\pgfqpoint{0.628953in}{2.504474in}}%
\pgfpathlineto{\pgfqpoint{0.641378in}{2.505560in}}%
\pgfpathlineto{\pgfqpoint{0.650696in}{2.502953in}}%
\pgfpathlineto{\pgfqpoint{0.656909in}{2.498366in}}%
\pgfpathlineto{\pgfqpoint{0.663121in}{2.490784in}}%
\pgfpathlineto{\pgfqpoint{0.672440in}{2.472648in}}%
\pgfpathlineto{\pgfqpoint{0.681759in}{2.445256in}}%
\pgfpathlineto{\pgfqpoint{0.691077in}{2.407996in}}%
\pgfpathlineto{\pgfqpoint{0.703502in}{2.343451in}}%
\pgfpathlineto{\pgfqpoint{0.719033in}{2.243024in}}%
\pgfpathlineto{\pgfqpoint{0.750095in}{2.011995in}}%
\pgfpathlineto{\pgfqpoint{0.778051in}{1.814177in}}%
\pgfpathlineto{\pgfqpoint{0.802901in}{1.660536in}}%
\pgfpathlineto{\pgfqpoint{0.827751in}{1.527122in}}%
\pgfpathlineto{\pgfqpoint{0.846388in}{1.442608in}}%
\pgfpathlineto{\pgfqpoint{0.861919in}{1.384831in}}%
\pgfpathlineto{\pgfqpoint{0.874344in}{1.347966in}}%
\pgfpathlineto{\pgfqpoint{0.886769in}{1.320288in}}%
\pgfpathlineto{\pgfqpoint{0.896087in}{1.306371in}}%
\pgfpathlineto{\pgfqpoint{0.902300in}{1.300861in}}%
\pgfpathlineto{\pgfqpoint{0.908512in}{1.298804in}}%
\pgfpathlineto{\pgfqpoint{0.914724in}{1.300654in}}%
\pgfpathlineto{\pgfqpoint{0.920937in}{1.306910in}}%
\pgfpathlineto{\pgfqpoint{0.927149in}{1.318084in}}%
\pgfpathlineto{\pgfqpoint{0.933362in}{1.334658in}}%
\pgfpathlineto{\pgfqpoint{0.942680in}{1.370477in}}%
\pgfpathlineto{\pgfqpoint{0.951999in}{1.420016in}}%
\pgfpathlineto{\pgfqpoint{0.964424in}{1.506580in}}%
\pgfpathlineto{\pgfqpoint{0.979955in}{1.640406in}}%
\pgfpathlineto{\pgfqpoint{1.011017in}{1.914486in}}%
\pgfpathlineto{\pgfqpoint{1.020336in}{1.974212in}}%
\pgfpathlineto{\pgfqpoint{1.029654in}{2.014805in}}%
\pgfpathlineto{\pgfqpoint{1.035867in}{2.029739in}}%
\pgfpathlineto{\pgfqpoint{1.038973in}{2.033378in}}%
\pgfpathlineto{\pgfqpoint{1.042079in}{2.034438in}}%
\pgfpathlineto{\pgfqpoint{1.045185in}{2.032925in}}%
\pgfpathlineto{\pgfqpoint{1.051398in}{2.022331in}}%
\pgfpathlineto{\pgfqpoint{1.057610in}{2.002115in}}%
\pgfpathlineto{\pgfqpoint{1.066929in}{1.955678in}}%
\pgfpathlineto{\pgfqpoint{1.079354in}{1.870125in}}%
\pgfpathlineto{\pgfqpoint{1.104203in}{1.662648in}}%
\pgfpathlineto{\pgfqpoint{1.122841in}{1.519737in}}%
\pgfpathlineto{\pgfqpoint{1.135266in}{1.445494in}}%
\pgfpathlineto{\pgfqpoint{1.144584in}{1.403497in}}%
\pgfpathlineto{\pgfqpoint{1.153903in}{1.373568in}}%
\pgfpathlineto{\pgfqpoint{1.163221in}{1.355244in}}%
\pgfpathlineto{\pgfqpoint{1.169434in}{1.349026in}}%
\pgfpathlineto{\pgfqpoint{1.175646in}{1.347206in}}%
\pgfpathlineto{\pgfqpoint{1.181859in}{1.349402in}}%
\pgfpathlineto{\pgfqpoint{1.188071in}{1.355214in}}%
\pgfpathlineto{\pgfqpoint{1.197390in}{1.369815in}}%
\pgfpathlineto{\pgfqpoint{1.209815in}{1.398069in}}%
\pgfpathlineto{\pgfqpoint{1.228452in}{1.452086in}}%
\pgfpathlineto{\pgfqpoint{1.253302in}{1.524884in}}%
\pgfpathlineto{\pgfqpoint{1.265726in}{1.551972in}}%
\pgfpathlineto{\pgfqpoint{1.275045in}{1.565076in}}%
\pgfpathlineto{\pgfqpoint{1.281258in}{1.569785in}}%
\pgfpathlineto{\pgfqpoint{1.287470in}{1.571104in}}%
\pgfpathlineto{\pgfqpoint{1.293682in}{1.569062in}}%
\pgfpathlineto{\pgfqpoint{1.299895in}{1.563849in}}%
\pgfpathlineto{\pgfqpoint{1.309213in}{1.550908in}}%
\pgfpathlineto{\pgfqpoint{1.321638in}{1.527087in}}%
\pgfpathlineto{\pgfqpoint{1.343382in}{1.483641in}}%
\pgfpathlineto{\pgfqpoint{1.352700in}{1.470374in}}%
\pgfpathlineto{\pgfqpoint{1.362019in}{1.462407in}}%
\pgfpathlineto{\pgfqpoint{1.371338in}{1.459789in}}%
\pgfpathlineto{\pgfqpoint{1.380656in}{1.461362in}}%
\pgfpathlineto{\pgfqpoint{1.399294in}{1.467328in}}%
\pgfpathlineto{\pgfqpoint{1.405506in}{1.466712in}}%
\pgfpathlineto{\pgfqpoint{1.411718in}{1.463176in}}%
\pgfpathlineto{\pgfqpoint{1.417931in}{1.455875in}}%
\pgfpathlineto{\pgfqpoint{1.424143in}{1.444143in}}%
\pgfpathlineto{\pgfqpoint{1.433462in}{1.417342in}}%
\pgfpathlineto{\pgfqpoint{1.442781in}{1.379284in}}%
\pgfpathlineto{\pgfqpoint{1.455205in}{1.313042in}}%
\pgfpathlineto{\pgfqpoint{1.476949in}{1.173380in}}%
\pgfpathlineto{\pgfqpoint{1.495586in}{1.060907in}}%
\pgfpathlineto{\pgfqpoint{1.508011in}{1.004623in}}%
\pgfpathlineto{\pgfqpoint{1.517330in}{0.976122in}}%
\pgfpathlineto{\pgfqpoint{1.523542in}{0.964211in}}%
\pgfpathlineto{\pgfqpoint{1.529755in}{0.957995in}}%
\pgfpathlineto{\pgfqpoint{1.535967in}{0.957303in}}%
\pgfpathlineto{\pgfqpoint{1.542179in}{0.961829in}}%
\pgfpathlineto{\pgfqpoint{1.548392in}{0.971177in}}%
\pgfpathlineto{\pgfqpoint{1.557710in}{0.993262in}}%
\pgfpathlineto{\pgfqpoint{1.567029in}{1.023691in}}%
\pgfpathlineto{\pgfqpoint{1.579454in}{1.075055in}}%
\pgfpathlineto{\pgfqpoint{1.594985in}{1.153932in}}%
\pgfpathlineto{\pgfqpoint{1.610516in}{1.247834in}}%
\pgfpathlineto{\pgfqpoint{1.629153in}{1.379981in}}%
\pgfpathlineto{\pgfqpoint{1.654003in}{1.582964in}}%
\pgfpathlineto{\pgfqpoint{1.685065in}{1.836349in}}%
\pgfpathlineto{\pgfqpoint{1.697490in}{1.917429in}}%
\pgfpathlineto{\pgfqpoint{1.706809in}{1.963993in}}%
\pgfpathlineto{\pgfqpoint{1.716127in}{1.995326in}}%
\pgfpathlineto{\pgfqpoint{1.722340in}{2.006647in}}%
\pgfpathlineto{\pgfqpoint{1.725446in}{2.009246in}}%
\pgfpathlineto{\pgfqpoint{1.728552in}{2.009741in}}%
\pgfpathlineto{\pgfqpoint{1.731658in}{2.008099in}}%
\pgfpathlineto{\pgfqpoint{1.737871in}{1.998337in}}%
\pgfpathlineto{\pgfqpoint{1.744083in}{1.979953in}}%
\pgfpathlineto{\pgfqpoint{1.753402in}{1.936753in}}%
\pgfpathlineto{\pgfqpoint{1.762720in}{1.876433in}}%
\pgfpathlineto{\pgfqpoint{1.775145in}{1.774606in}}%
\pgfpathlineto{\pgfqpoint{1.821738in}{1.361384in}}%
\pgfpathlineto{\pgfqpoint{1.831057in}{1.311757in}}%
\pgfpathlineto{\pgfqpoint{1.837270in}{1.290167in}}%
\pgfpathlineto{\pgfqpoint{1.843482in}{1.278580in}}%
\pgfpathlineto{\pgfqpoint{1.846588in}{1.276661in}}%
\pgfpathlineto{\pgfqpoint{1.849694in}{1.277350in}}%
\pgfpathlineto{\pgfqpoint{1.852801in}{1.280644in}}%
\pgfpathlineto{\pgfqpoint{1.859013in}{1.294927in}}%
\pgfpathlineto{\pgfqpoint{1.865225in}{1.319072in}}%
\pgfpathlineto{\pgfqpoint{1.874544in}{1.372078in}}%
\pgfpathlineto{\pgfqpoint{1.886969in}{1.468019in}}%
\pgfpathlineto{\pgfqpoint{1.902500in}{1.611347in}}%
\pgfpathlineto{\pgfqpoint{1.902500in}{1.611347in}}%
\pgfusepath{stroke}%
\end{pgfscope}%
\begin{pgfscope}%
\pgfpathrectangle{\pgfqpoint{0.275000in}{0.375000in}}{\pgfqpoint{1.705000in}{2.265000in}}%
\pgfusepath{clip}%
\pgfsetroundcap%
\pgfsetroundjoin%
\pgfsetlinewidth{1.505625pt}%
\definecolor{currentstroke}{rgb}{0.580392,0.403922,0.741176}%
\pgfsetstrokecolor{currentstroke}%
\pgfsetdash{}{0pt}%
\pgfpathmoveto{\pgfqpoint{0.352500in}{1.631904in}}%
\pgfpathlineto{\pgfqpoint{0.368031in}{1.465812in}}%
\pgfpathlineto{\pgfqpoint{0.389775in}{1.193917in}}%
\pgfpathlineto{\pgfqpoint{0.423943in}{0.760583in}}%
\pgfpathlineto{\pgfqpoint{0.436368in}{0.634873in}}%
\pgfpathlineto{\pgfqpoint{0.445686in}{0.561549in}}%
\pgfpathlineto{\pgfqpoint{0.455005in}{0.509915in}}%
\pgfpathlineto{\pgfqpoint{0.461217in}{0.488752in}}%
\pgfpathlineto{\pgfqpoint{0.467430in}{0.478725in}}%
\pgfpathlineto{\pgfqpoint{0.470536in}{0.477955in}}%
\pgfpathlineto{\pgfqpoint{0.473642in}{0.480018in}}%
\pgfpathlineto{\pgfqpoint{0.479855in}{0.492567in}}%
\pgfpathlineto{\pgfqpoint{0.486067in}{0.516065in}}%
\pgfpathlineto{\pgfqpoint{0.495386in}{0.570603in}}%
\pgfpathlineto{\pgfqpoint{0.504704in}{0.645841in}}%
\pgfpathlineto{\pgfqpoint{0.517129in}{0.772220in}}%
\pgfpathlineto{\pgfqpoint{0.535767in}{0.997614in}}%
\pgfpathlineto{\pgfqpoint{0.573041in}{1.457900in}}%
\pgfpathlineto{\pgfqpoint{0.591678in}{1.644660in}}%
\pgfpathlineto{\pgfqpoint{0.607209in}{1.768287in}}%
\pgfpathlineto{\pgfqpoint{0.622740in}{1.865482in}}%
\pgfpathlineto{\pgfqpoint{0.638272in}{1.943269in}}%
\pgfpathlineto{\pgfqpoint{0.660015in}{2.035253in}}%
\pgfpathlineto{\pgfqpoint{0.691077in}{2.154499in}}%
\pgfpathlineto{\pgfqpoint{0.706608in}{2.203740in}}%
\pgfpathlineto{\pgfqpoint{0.715927in}{2.225483in}}%
\pgfpathlineto{\pgfqpoint{0.725245in}{2.238847in}}%
\pgfpathlineto{\pgfqpoint{0.731458in}{2.242182in}}%
\pgfpathlineto{\pgfqpoint{0.737670in}{2.240547in}}%
\pgfpathlineto{\pgfqpoint{0.743883in}{2.233649in}}%
\pgfpathlineto{\pgfqpoint{0.750095in}{2.221353in}}%
\pgfpathlineto{\pgfqpoint{0.759414in}{2.192918in}}%
\pgfpathlineto{\pgfqpoint{0.768732in}{2.153340in}}%
\pgfpathlineto{\pgfqpoint{0.781157in}{2.086347in}}%
\pgfpathlineto{\pgfqpoint{0.806007in}{1.927585in}}%
\pgfpathlineto{\pgfqpoint{0.824644in}{1.814808in}}%
\pgfpathlineto{\pgfqpoint{0.837069in}{1.753683in}}%
\pgfpathlineto{\pgfqpoint{0.849494in}{1.707681in}}%
\pgfpathlineto{\pgfqpoint{0.858813in}{1.683730in}}%
\pgfpathlineto{\pgfqpoint{0.868131in}{1.668534in}}%
\pgfpathlineto{\pgfqpoint{0.874344in}{1.662906in}}%
\pgfpathlineto{\pgfqpoint{0.880556in}{1.660531in}}%
\pgfpathlineto{\pgfqpoint{0.886769in}{1.661061in}}%
\pgfpathlineto{\pgfqpoint{0.892981in}{1.664130in}}%
\pgfpathlineto{\pgfqpoint{0.902300in}{1.672689in}}%
\pgfpathlineto{\pgfqpoint{0.914724in}{1.689726in}}%
\pgfpathlineto{\pgfqpoint{0.930256in}{1.717176in}}%
\pgfpathlineto{\pgfqpoint{0.948893in}{1.756529in}}%
\pgfpathlineto{\pgfqpoint{0.967530in}{1.802806in}}%
\pgfpathlineto{\pgfqpoint{0.989274in}{1.865959in}}%
\pgfpathlineto{\pgfqpoint{1.029654in}{1.986287in}}%
\pgfpathlineto{\pgfqpoint{1.045185in}{2.021022in}}%
\pgfpathlineto{\pgfqpoint{1.057610in}{2.042207in}}%
\pgfpathlineto{\pgfqpoint{1.073141in}{2.062820in}}%
\pgfpathlineto{\pgfqpoint{1.097991in}{2.094904in}}%
\pgfpathlineto{\pgfqpoint{1.110416in}{2.116026in}}%
\pgfpathlineto{\pgfqpoint{1.125947in}{2.149359in}}%
\pgfpathlineto{\pgfqpoint{1.147690in}{2.205315in}}%
\pgfpathlineto{\pgfqpoint{1.175646in}{2.276192in}}%
\pgfpathlineto{\pgfqpoint{1.194284in}{2.314198in}}%
\pgfpathlineto{\pgfqpoint{1.209815in}{2.339221in}}%
\pgfpathlineto{\pgfqpoint{1.225346in}{2.358677in}}%
\pgfpathlineto{\pgfqpoint{1.237771in}{2.369843in}}%
\pgfpathlineto{\pgfqpoint{1.247089in}{2.374861in}}%
\pgfpathlineto{\pgfqpoint{1.256408in}{2.376060in}}%
\pgfpathlineto{\pgfqpoint{1.262620in}{2.374227in}}%
\pgfpathlineto{\pgfqpoint{1.268833in}{2.369915in}}%
\pgfpathlineto{\pgfqpoint{1.275045in}{2.362816in}}%
\pgfpathlineto{\pgfqpoint{1.284364in}{2.346363in}}%
\pgfpathlineto{\pgfqpoint{1.293682in}{2.322373in}}%
\pgfpathlineto{\pgfqpoint{1.303001in}{2.290573in}}%
\pgfpathlineto{\pgfqpoint{1.315426in}{2.236438in}}%
\pgfpathlineto{\pgfqpoint{1.330957in}{2.152729in}}%
\pgfpathlineto{\pgfqpoint{1.358913in}{1.978037in}}%
\pgfpathlineto{\pgfqpoint{1.383763in}{1.828405in}}%
\pgfpathlineto{\pgfqpoint{1.402400in}{1.733775in}}%
\pgfpathlineto{\pgfqpoint{1.421037in}{1.655463in}}%
\pgfpathlineto{\pgfqpoint{1.445887in}{1.566627in}}%
\pgfpathlineto{\pgfqpoint{1.492480in}{1.402117in}}%
\pgfpathlineto{\pgfqpoint{1.520436in}{1.306715in}}%
\pgfpathlineto{\pgfqpoint{1.535967in}{1.264180in}}%
\pgfpathlineto{\pgfqpoint{1.548392in}{1.238977in}}%
\pgfpathlineto{\pgfqpoint{1.557710in}{1.226330in}}%
\pgfpathlineto{\pgfqpoint{1.567029in}{1.219717in}}%
\pgfpathlineto{\pgfqpoint{1.573241in}{1.218925in}}%
\pgfpathlineto{\pgfqpoint{1.579454in}{1.221189in}}%
\pgfpathlineto{\pgfqpoint{1.585666in}{1.226628in}}%
\pgfpathlineto{\pgfqpoint{1.594985in}{1.240920in}}%
\pgfpathlineto{\pgfqpoint{1.604304in}{1.262642in}}%
\pgfpathlineto{\pgfqpoint{1.616728in}{1.302650in}}%
\pgfpathlineto{\pgfqpoint{1.632260in}{1.367214in}}%
\pgfpathlineto{\pgfqpoint{1.666428in}{1.518648in}}%
\pgfpathlineto{\pgfqpoint{1.675746in}{1.546747in}}%
\pgfpathlineto{\pgfqpoint{1.681959in}{1.559091in}}%
\pgfpathlineto{\pgfqpoint{1.688171in}{1.565425in}}%
\pgfpathlineto{\pgfqpoint{1.691278in}{1.566155in}}%
\pgfpathlineto{\pgfqpoint{1.694384in}{1.565190in}}%
\pgfpathlineto{\pgfqpoint{1.700596in}{1.558059in}}%
\pgfpathlineto{\pgfqpoint{1.706809in}{1.543965in}}%
\pgfpathlineto{\pgfqpoint{1.716127in}{1.510301in}}%
\pgfpathlineto{\pgfqpoint{1.725446in}{1.463376in}}%
\pgfpathlineto{\pgfqpoint{1.740977in}{1.364738in}}%
\pgfpathlineto{\pgfqpoint{1.765827in}{1.200988in}}%
\pgfpathlineto{\pgfqpoint{1.778252in}{1.140822in}}%
\pgfpathlineto{\pgfqpoint{1.787570in}{1.112962in}}%
\pgfpathlineto{\pgfqpoint{1.793783in}{1.104000in}}%
\pgfpathlineto{\pgfqpoint{1.796889in}{1.102556in}}%
\pgfpathlineto{\pgfqpoint{1.799995in}{1.103165in}}%
\pgfpathlineto{\pgfqpoint{1.803101in}{1.105825in}}%
\pgfpathlineto{\pgfqpoint{1.809314in}{1.117215in}}%
\pgfpathlineto{\pgfqpoint{1.815526in}{1.136388in}}%
\pgfpathlineto{\pgfqpoint{1.824845in}{1.178432in}}%
\pgfpathlineto{\pgfqpoint{1.837270in}{1.254815in}}%
\pgfpathlineto{\pgfqpoint{1.855907in}{1.395884in}}%
\pgfpathlineto{\pgfqpoint{1.886969in}{1.632964in}}%
\pgfpathlineto{\pgfqpoint{1.902500in}{1.727819in}}%
\pgfpathlineto{\pgfqpoint{1.902500in}{1.727819in}}%
\pgfusepath{stroke}%
\end{pgfscope}%
\begin{pgfscope}%
\pgfpathrectangle{\pgfqpoint{0.275000in}{0.375000in}}{\pgfqpoint{1.705000in}{2.265000in}}%
\pgfusepath{clip}%
\pgfsetroundcap%
\pgfsetroundjoin%
\pgfsetlinewidth{1.505625pt}%
\definecolor{currentstroke}{rgb}{0.549020,0.337255,0.294118}%
\pgfsetstrokecolor{currentstroke}%
\pgfsetdash{}{0pt}%
\pgfpathmoveto{\pgfqpoint{0.352500in}{1.849331in}}%
\pgfpathlineto{\pgfqpoint{0.361819in}{1.826501in}}%
\pgfpathlineto{\pgfqpoint{0.374243in}{1.785670in}}%
\pgfpathlineto{\pgfqpoint{0.399093in}{1.688506in}}%
\pgfpathlineto{\pgfqpoint{0.427049in}{1.582531in}}%
\pgfpathlineto{\pgfqpoint{0.458111in}{1.478300in}}%
\pgfpathlineto{\pgfqpoint{0.510917in}{1.311765in}}%
\pgfpathlineto{\pgfqpoint{0.526448in}{1.272090in}}%
\pgfpathlineto{\pgfqpoint{0.535767in}{1.254401in}}%
\pgfpathlineto{\pgfqpoint{0.545085in}{1.243118in}}%
\pgfpathlineto{\pgfqpoint{0.551298in}{1.239850in}}%
\pgfpathlineto{\pgfqpoint{0.557510in}{1.240409in}}%
\pgfpathlineto{\pgfqpoint{0.563722in}{1.245094in}}%
\pgfpathlineto{\pgfqpoint{0.569935in}{1.254129in}}%
\pgfpathlineto{\pgfqpoint{0.579254in}{1.276147in}}%
\pgfpathlineto{\pgfqpoint{0.588572in}{1.308442in}}%
\pgfpathlineto{\pgfqpoint{0.600997in}{1.367040in}}%
\pgfpathlineto{\pgfqpoint{0.613422in}{1.441836in}}%
\pgfpathlineto{\pgfqpoint{0.632059in}{1.577666in}}%
\pgfpathlineto{\pgfqpoint{0.675546in}{1.909673in}}%
\pgfpathlineto{\pgfqpoint{0.687971in}{1.977705in}}%
\pgfpathlineto{\pgfqpoint{0.697290in}{2.014220in}}%
\pgfpathlineto{\pgfqpoint{0.706608in}{2.037153in}}%
\pgfpathlineto{\pgfqpoint{0.712821in}{2.044882in}}%
\pgfpathlineto{\pgfqpoint{0.719033in}{2.046838in}}%
\pgfpathlineto{\pgfqpoint{0.725245in}{2.043452in}}%
\pgfpathlineto{\pgfqpoint{0.731458in}{2.035293in}}%
\pgfpathlineto{\pgfqpoint{0.740777in}{2.015606in}}%
\pgfpathlineto{\pgfqpoint{0.753201in}{1.979417in}}%
\pgfpathlineto{\pgfqpoint{0.799795in}{1.831167in}}%
\pgfpathlineto{\pgfqpoint{0.812219in}{1.804535in}}%
\pgfpathlineto{\pgfqpoint{0.824644in}{1.785686in}}%
\pgfpathlineto{\pgfqpoint{0.837069in}{1.773619in}}%
\pgfpathlineto{\pgfqpoint{0.849494in}{1.766382in}}%
\pgfpathlineto{\pgfqpoint{0.871237in}{1.758528in}}%
\pgfpathlineto{\pgfqpoint{0.902300in}{1.748940in}}%
\pgfpathlineto{\pgfqpoint{0.911618in}{1.748961in}}%
\pgfpathlineto{\pgfqpoint{0.920937in}{1.752506in}}%
\pgfpathlineto{\pgfqpoint{0.930256in}{1.760765in}}%
\pgfpathlineto{\pgfqpoint{0.939574in}{1.774448in}}%
\pgfpathlineto{\pgfqpoint{0.948893in}{1.793535in}}%
\pgfpathlineto{\pgfqpoint{0.964424in}{1.834560in}}%
\pgfpathlineto{\pgfqpoint{0.989274in}{1.902035in}}%
\pgfpathlineto{\pgfqpoint{0.998592in}{1.919425in}}%
\pgfpathlineto{\pgfqpoint{1.004805in}{1.926833in}}%
\pgfpathlineto{\pgfqpoint{1.011017in}{1.930447in}}%
\pgfpathlineto{\pgfqpoint{1.017229in}{1.930117in}}%
\pgfpathlineto{\pgfqpoint{1.023442in}{1.925915in}}%
\pgfpathlineto{\pgfqpoint{1.029654in}{1.918136in}}%
\pgfpathlineto{\pgfqpoint{1.038973in}{1.900902in}}%
\pgfpathlineto{\pgfqpoint{1.057610in}{1.855857in}}%
\pgfpathlineto{\pgfqpoint{1.073141in}{1.821792in}}%
\pgfpathlineto{\pgfqpoint{1.082460in}{1.808503in}}%
\pgfpathlineto{\pgfqpoint{1.088672in}{1.803759in}}%
\pgfpathlineto{\pgfqpoint{1.094885in}{1.802615in}}%
\pgfpathlineto{\pgfqpoint{1.101097in}{1.805097in}}%
\pgfpathlineto{\pgfqpoint{1.107310in}{1.811031in}}%
\pgfpathlineto{\pgfqpoint{1.116628in}{1.825564in}}%
\pgfpathlineto{\pgfqpoint{1.129053in}{1.852397in}}%
\pgfpathlineto{\pgfqpoint{1.157009in}{1.915887in}}%
\pgfpathlineto{\pgfqpoint{1.166328in}{1.930294in}}%
\pgfpathlineto{\pgfqpoint{1.175646in}{1.939205in}}%
\pgfpathlineto{\pgfqpoint{1.181859in}{1.942015in}}%
\pgfpathlineto{\pgfqpoint{1.188071in}{1.942497in}}%
\pgfpathlineto{\pgfqpoint{1.197390in}{1.939552in}}%
\pgfpathlineto{\pgfqpoint{1.209815in}{1.931143in}}%
\pgfpathlineto{\pgfqpoint{1.231558in}{1.915639in}}%
\pgfpathlineto{\pgfqpoint{1.240877in}{1.912621in}}%
\pgfpathlineto{\pgfqpoint{1.250195in}{1.913164in}}%
\pgfpathlineto{\pgfqpoint{1.259514in}{1.917577in}}%
\pgfpathlineto{\pgfqpoint{1.268833in}{1.925727in}}%
\pgfpathlineto{\pgfqpoint{1.281258in}{1.941545in}}%
\pgfpathlineto{\pgfqpoint{1.299895in}{1.972201in}}%
\pgfpathlineto{\pgfqpoint{1.324744in}{2.013772in}}%
\pgfpathlineto{\pgfqpoint{1.334063in}{2.025236in}}%
\pgfpathlineto{\pgfqpoint{1.343382in}{2.031946in}}%
\pgfpathlineto{\pgfqpoint{1.349594in}{2.032850in}}%
\pgfpathlineto{\pgfqpoint{1.355807in}{2.030227in}}%
\pgfpathlineto{\pgfqpoint{1.362019in}{2.023520in}}%
\pgfpathlineto{\pgfqpoint{1.368231in}{2.012231in}}%
\pgfpathlineto{\pgfqpoint{1.377550in}{1.985854in}}%
\pgfpathlineto{\pgfqpoint{1.386869in}{1.947512in}}%
\pgfpathlineto{\pgfqpoint{1.399294in}{1.878176in}}%
\pgfpathlineto{\pgfqpoint{1.414825in}{1.767415in}}%
\pgfpathlineto{\pgfqpoint{1.458312in}{1.436677in}}%
\pgfpathlineto{\pgfqpoint{1.470736in}{1.371287in}}%
\pgfpathlineto{\pgfqpoint{1.480055in}{1.336075in}}%
\pgfpathlineto{\pgfqpoint{1.489374in}{1.312591in}}%
\pgfpathlineto{\pgfqpoint{1.495586in}{1.302935in}}%
\pgfpathlineto{\pgfqpoint{1.501799in}{1.297522in}}%
\pgfpathlineto{\pgfqpoint{1.508011in}{1.295785in}}%
\pgfpathlineto{\pgfqpoint{1.514223in}{1.297119in}}%
\pgfpathlineto{\pgfqpoint{1.523542in}{1.303583in}}%
\pgfpathlineto{\pgfqpoint{1.535967in}{1.317748in}}%
\pgfpathlineto{\pgfqpoint{1.560817in}{1.353603in}}%
\pgfpathlineto{\pgfqpoint{1.585666in}{1.387496in}}%
\pgfpathlineto{\pgfqpoint{1.598091in}{1.400231in}}%
\pgfpathlineto{\pgfqpoint{1.607410in}{1.406635in}}%
\pgfpathlineto{\pgfqpoint{1.616728in}{1.409587in}}%
\pgfpathlineto{\pgfqpoint{1.626047in}{1.408565in}}%
\pgfpathlineto{\pgfqpoint{1.635366in}{1.403334in}}%
\pgfpathlineto{\pgfqpoint{1.644684in}{1.394029in}}%
\pgfpathlineto{\pgfqpoint{1.657109in}{1.376277in}}%
\pgfpathlineto{\pgfqpoint{1.678853in}{1.337231in}}%
\pgfpathlineto{\pgfqpoint{1.700596in}{1.300470in}}%
\pgfpathlineto{\pgfqpoint{1.713021in}{1.284588in}}%
\pgfpathlineto{\pgfqpoint{1.725446in}{1.273481in}}%
\pgfpathlineto{\pgfqpoint{1.734765in}{1.268469in}}%
\pgfpathlineto{\pgfqpoint{1.744083in}{1.266510in}}%
\pgfpathlineto{\pgfqpoint{1.753402in}{1.267998in}}%
\pgfpathlineto{\pgfqpoint{1.762720in}{1.273545in}}%
\pgfpathlineto{\pgfqpoint{1.772039in}{1.283890in}}%
\pgfpathlineto{\pgfqpoint{1.781358in}{1.299708in}}%
\pgfpathlineto{\pgfqpoint{1.790676in}{1.321363in}}%
\pgfpathlineto{\pgfqpoint{1.803101in}{1.358898in}}%
\pgfpathlineto{\pgfqpoint{1.821738in}{1.428110in}}%
\pgfpathlineto{\pgfqpoint{1.843482in}{1.508719in}}%
\pgfpathlineto{\pgfqpoint{1.855907in}{1.544726in}}%
\pgfpathlineto{\pgfqpoint{1.865225in}{1.564603in}}%
\pgfpathlineto{\pgfqpoint{1.874544in}{1.578415in}}%
\pgfpathlineto{\pgfqpoint{1.883863in}{1.587281in}}%
\pgfpathlineto{\pgfqpoint{1.899394in}{1.596506in}}%
\pgfpathlineto{\pgfqpoint{1.902500in}{1.598328in}}%
\pgfpathlineto{\pgfqpoint{1.902500in}{1.598328in}}%
\pgfusepath{stroke}%
\end{pgfscope}%
\begin{pgfscope}%
\pgfpathrectangle{\pgfqpoint{0.275000in}{0.375000in}}{\pgfqpoint{1.705000in}{2.265000in}}%
\pgfusepath{clip}%
\pgfsetroundcap%
\pgfsetroundjoin%
\pgfsetlinewidth{1.505625pt}%
\definecolor{currentstroke}{rgb}{0.890196,0.466667,0.760784}%
\pgfsetstrokecolor{currentstroke}%
\pgfsetdash{}{0pt}%
\pgfpathmoveto{\pgfqpoint{0.352500in}{1.992764in}}%
\pgfpathlineto{\pgfqpoint{0.358712in}{1.996324in}}%
\pgfpathlineto{\pgfqpoint{0.364925in}{1.994145in}}%
\pgfpathlineto{\pgfqpoint{0.371137in}{1.986353in}}%
\pgfpathlineto{\pgfqpoint{0.377350in}{1.973277in}}%
\pgfpathlineto{\pgfqpoint{0.386668in}{1.944854in}}%
\pgfpathlineto{\pgfqpoint{0.399093in}{1.893756in}}%
\pgfpathlineto{\pgfqpoint{0.414624in}{1.815579in}}%
\pgfpathlineto{\pgfqpoint{0.439474in}{1.673083in}}%
\pgfpathlineto{\pgfqpoint{0.498492in}{1.326376in}}%
\pgfpathlineto{\pgfqpoint{0.510917in}{1.276697in}}%
\pgfpathlineto{\pgfqpoint{0.520235in}{1.251340in}}%
\pgfpathlineto{\pgfqpoint{0.526448in}{1.240733in}}%
\pgfpathlineto{\pgfqpoint{0.532660in}{1.235281in}}%
\pgfpathlineto{\pgfqpoint{0.538873in}{1.234873in}}%
\pgfpathlineto{\pgfqpoint{0.545085in}{1.239218in}}%
\pgfpathlineto{\pgfqpoint{0.551298in}{1.247855in}}%
\pgfpathlineto{\pgfqpoint{0.560616in}{1.267523in}}%
\pgfpathlineto{\pgfqpoint{0.573041in}{1.302426in}}%
\pgfpathlineto{\pgfqpoint{0.607209in}{1.403577in}}%
\pgfpathlineto{\pgfqpoint{0.619634in}{1.429241in}}%
\pgfpathlineto{\pgfqpoint{0.628953in}{1.442820in}}%
\pgfpathlineto{\pgfqpoint{0.638272in}{1.451949in}}%
\pgfpathlineto{\pgfqpoint{0.650696in}{1.458951in}}%
\pgfpathlineto{\pgfqpoint{0.675546in}{1.469492in}}%
\pgfpathlineto{\pgfqpoint{0.684865in}{1.478079in}}%
\pgfpathlineto{\pgfqpoint{0.694183in}{1.492135in}}%
\pgfpathlineto{\pgfqpoint{0.703502in}{1.513237in}}%
\pgfpathlineto{\pgfqpoint{0.712821in}{1.542569in}}%
\pgfpathlineto{\pgfqpoint{0.722139in}{1.580835in}}%
\pgfpathlineto{\pgfqpoint{0.734564in}{1.646004in}}%
\pgfpathlineto{\pgfqpoint{0.750095in}{1.748530in}}%
\pgfpathlineto{\pgfqpoint{0.768732in}{1.895092in}}%
\pgfpathlineto{\pgfqpoint{0.809113in}{2.221712in}}%
\pgfpathlineto{\pgfqpoint{0.821538in}{2.296698in}}%
\pgfpathlineto{\pgfqpoint{0.830857in}{2.337103in}}%
\pgfpathlineto{\pgfqpoint{0.837069in}{2.355157in}}%
\pgfpathlineto{\pgfqpoint{0.843282in}{2.365541in}}%
\pgfpathlineto{\pgfqpoint{0.846388in}{2.367774in}}%
\pgfpathlineto{\pgfqpoint{0.849494in}{2.368020in}}%
\pgfpathlineto{\pgfqpoint{0.852600in}{2.366288in}}%
\pgfpathlineto{\pgfqpoint{0.858813in}{2.357017in}}%
\pgfpathlineto{\pgfqpoint{0.865025in}{2.340376in}}%
\pgfpathlineto{\pgfqpoint{0.874344in}{2.303138in}}%
\pgfpathlineto{\pgfqpoint{0.886769in}{2.235877in}}%
\pgfpathlineto{\pgfqpoint{0.930256in}{1.980139in}}%
\pgfpathlineto{\pgfqpoint{0.942680in}{1.934283in}}%
\pgfpathlineto{\pgfqpoint{0.951999in}{1.911170in}}%
\pgfpathlineto{\pgfqpoint{0.961318in}{1.896519in}}%
\pgfpathlineto{\pgfqpoint{0.970636in}{1.888396in}}%
\pgfpathlineto{\pgfqpoint{0.979955in}{1.884474in}}%
\pgfpathlineto{\pgfqpoint{1.004805in}{1.877476in}}%
\pgfpathlineto{\pgfqpoint{1.014123in}{1.871251in}}%
\pgfpathlineto{\pgfqpoint{1.023442in}{1.861525in}}%
\pgfpathlineto{\pgfqpoint{1.035867in}{1.842688in}}%
\pgfpathlineto{\pgfqpoint{1.048292in}{1.817538in}}%
\pgfpathlineto{\pgfqpoint{1.063823in}{1.778972in}}%
\pgfpathlineto{\pgfqpoint{1.091779in}{1.698847in}}%
\pgfpathlineto{\pgfqpoint{1.116628in}{1.630744in}}%
\pgfpathlineto{\pgfqpoint{1.129053in}{1.604484in}}%
\pgfpathlineto{\pgfqpoint{1.138372in}{1.591100in}}%
\pgfpathlineto{\pgfqpoint{1.144584in}{1.586153in}}%
\pgfpathlineto{\pgfqpoint{1.150797in}{1.585003in}}%
\pgfpathlineto{\pgfqpoint{1.157009in}{1.588141in}}%
\pgfpathlineto{\pgfqpoint{1.163221in}{1.595995in}}%
\pgfpathlineto{\pgfqpoint{1.169434in}{1.608890in}}%
\pgfpathlineto{\pgfqpoint{1.178753in}{1.638056in}}%
\pgfpathlineto{\pgfqpoint{1.188071in}{1.678887in}}%
\pgfpathlineto{\pgfqpoint{1.200496in}{1.749590in}}%
\pgfpathlineto{\pgfqpoint{1.219133in}{1.879584in}}%
\pgfpathlineto{\pgfqpoint{1.247089in}{2.078181in}}%
\pgfpathlineto{\pgfqpoint{1.259514in}{2.148792in}}%
\pgfpathlineto{\pgfqpoint{1.268833in}{2.189713in}}%
\pgfpathlineto{\pgfqpoint{1.278151in}{2.218561in}}%
\pgfpathlineto{\pgfqpoint{1.284364in}{2.230479in}}%
\pgfpathlineto{\pgfqpoint{1.290576in}{2.236187in}}%
\pgfpathlineto{\pgfqpoint{1.293682in}{2.236623in}}%
\pgfpathlineto{\pgfqpoint{1.296789in}{2.235400in}}%
\pgfpathlineto{\pgfqpoint{1.303001in}{2.227849in}}%
\pgfpathlineto{\pgfqpoint{1.309213in}{2.213282in}}%
\pgfpathlineto{\pgfqpoint{1.315426in}{2.191475in}}%
\pgfpathlineto{\pgfqpoint{1.324744in}{2.144826in}}%
\pgfpathlineto{\pgfqpoint{1.334063in}{2.081271in}}%
\pgfpathlineto{\pgfqpoint{1.346488in}{1.971027in}}%
\pgfpathlineto{\pgfqpoint{1.362019in}{1.797481in}}%
\pgfpathlineto{\pgfqpoint{1.386869in}{1.472867in}}%
\pgfpathlineto{\pgfqpoint{1.408612in}{1.204198in}}%
\pgfpathlineto{\pgfqpoint{1.421037in}{1.084216in}}%
\pgfpathlineto{\pgfqpoint{1.430356in}{1.018081in}}%
\pgfpathlineto{\pgfqpoint{1.439674in}{0.975475in}}%
\pgfpathlineto{\pgfqpoint{1.445887in}{0.960843in}}%
\pgfpathlineto{\pgfqpoint{1.448993in}{0.957698in}}%
\pgfpathlineto{\pgfqpoint{1.452099in}{0.957313in}}%
\pgfpathlineto{\pgfqpoint{1.455205in}{0.959649in}}%
\pgfpathlineto{\pgfqpoint{1.461418in}{0.972262in}}%
\pgfpathlineto{\pgfqpoint{1.467630in}{0.994952in}}%
\pgfpathlineto{\pgfqpoint{1.476949in}{1.046125in}}%
\pgfpathlineto{\pgfqpoint{1.489374in}{1.141109in}}%
\pgfpathlineto{\pgfqpoint{1.504905in}{1.290051in}}%
\pgfpathlineto{\pgfqpoint{1.554604in}{1.794322in}}%
\pgfpathlineto{\pgfqpoint{1.567029in}{1.886163in}}%
\pgfpathlineto{\pgfqpoint{1.576348in}{1.938404in}}%
\pgfpathlineto{\pgfqpoint{1.585666in}{1.974711in}}%
\pgfpathlineto{\pgfqpoint{1.591879in}{1.989683in}}%
\pgfpathlineto{\pgfqpoint{1.598091in}{1.997251in}}%
\pgfpathlineto{\pgfqpoint{1.601197in}{1.998309in}}%
\pgfpathlineto{\pgfqpoint{1.604304in}{1.997600in}}%
\pgfpathlineto{\pgfqpoint{1.610516in}{1.991096in}}%
\pgfpathlineto{\pgfqpoint{1.616728in}{1.978277in}}%
\pgfpathlineto{\pgfqpoint{1.626047in}{1.948754in}}%
\pgfpathlineto{\pgfqpoint{1.638472in}{1.894729in}}%
\pgfpathlineto{\pgfqpoint{1.685065in}{1.668252in}}%
\pgfpathlineto{\pgfqpoint{1.694384in}{1.638655in}}%
\pgfpathlineto{\pgfqpoint{1.703702in}{1.618380in}}%
\pgfpathlineto{\pgfqpoint{1.709915in}{1.610345in}}%
\pgfpathlineto{\pgfqpoint{1.716127in}{1.606723in}}%
\pgfpathlineto{\pgfqpoint{1.722340in}{1.607388in}}%
\pgfpathlineto{\pgfqpoint{1.728552in}{1.612073in}}%
\pgfpathlineto{\pgfqpoint{1.734765in}{1.620371in}}%
\pgfpathlineto{\pgfqpoint{1.744083in}{1.638345in}}%
\pgfpathlineto{\pgfqpoint{1.762720in}{1.684480in}}%
\pgfpathlineto{\pgfqpoint{1.775145in}{1.712354in}}%
\pgfpathlineto{\pgfqpoint{1.784464in}{1.725962in}}%
\pgfpathlineto{\pgfqpoint{1.790676in}{1.729880in}}%
\pgfpathlineto{\pgfqpoint{1.796889in}{1.728890in}}%
\pgfpathlineto{\pgfqpoint{1.803101in}{1.722578in}}%
\pgfpathlineto{\pgfqpoint{1.809314in}{1.710801in}}%
\pgfpathlineto{\pgfqpoint{1.818632in}{1.683296in}}%
\pgfpathlineto{\pgfqpoint{1.831057in}{1.631292in}}%
\pgfpathlineto{\pgfqpoint{1.874544in}{1.428560in}}%
\pgfpathlineto{\pgfqpoint{1.883863in}{1.404389in}}%
\pgfpathlineto{\pgfqpoint{1.890075in}{1.394497in}}%
\pgfpathlineto{\pgfqpoint{1.896288in}{1.389671in}}%
\pgfpathlineto{\pgfqpoint{1.902500in}{1.389803in}}%
\pgfpathlineto{\pgfqpoint{1.902500in}{1.389803in}}%
\pgfusepath{stroke}%
\end{pgfscope}%
\begin{pgfscope}%
\pgfpathrectangle{\pgfqpoint{0.275000in}{0.375000in}}{\pgfqpoint{1.705000in}{2.265000in}}%
\pgfusepath{clip}%
\pgfsetroundcap%
\pgfsetroundjoin%
\pgfsetlinewidth{1.505625pt}%
\definecolor{currentstroke}{rgb}{0.498039,0.498039,0.498039}%
\pgfsetstrokecolor{currentstroke}%
\pgfsetdash{}{0pt}%
\pgfpathmoveto{\pgfqpoint{0.352500in}{1.882487in}}%
\pgfpathlineto{\pgfqpoint{0.364925in}{1.914251in}}%
\pgfpathlineto{\pgfqpoint{0.374243in}{1.930721in}}%
\pgfpathlineto{\pgfqpoint{0.383562in}{1.941626in}}%
\pgfpathlineto{\pgfqpoint{0.392881in}{1.948066in}}%
\pgfpathlineto{\pgfqpoint{0.405306in}{1.952274in}}%
\pgfpathlineto{\pgfqpoint{0.423943in}{1.957811in}}%
\pgfpathlineto{\pgfqpoint{0.433262in}{1.964249in}}%
\pgfpathlineto{\pgfqpoint{0.442580in}{1.975325in}}%
\pgfpathlineto{\pgfqpoint{0.451899in}{1.992215in}}%
\pgfpathlineto{\pgfqpoint{0.461217in}{2.015534in}}%
\pgfpathlineto{\pgfqpoint{0.473642in}{2.056427in}}%
\pgfpathlineto{\pgfqpoint{0.489173in}{2.119806in}}%
\pgfpathlineto{\pgfqpoint{0.517129in}{2.237357in}}%
\pgfpathlineto{\pgfqpoint{0.526448in}{2.266541in}}%
\pgfpathlineto{\pgfqpoint{0.535767in}{2.285917in}}%
\pgfpathlineto{\pgfqpoint{0.541979in}{2.292293in}}%
\pgfpathlineto{\pgfqpoint{0.548191in}{2.292918in}}%
\pgfpathlineto{\pgfqpoint{0.554404in}{2.287580in}}%
\pgfpathlineto{\pgfqpoint{0.560616in}{2.276294in}}%
\pgfpathlineto{\pgfqpoint{0.569935in}{2.248827in}}%
\pgfpathlineto{\pgfqpoint{0.579254in}{2.210399in}}%
\pgfpathlineto{\pgfqpoint{0.594785in}{2.129895in}}%
\pgfpathlineto{\pgfqpoint{0.622740in}{1.981098in}}%
\pgfpathlineto{\pgfqpoint{0.635165in}{1.931907in}}%
\pgfpathlineto{\pgfqpoint{0.644484in}{1.905493in}}%
\pgfpathlineto{\pgfqpoint{0.653803in}{1.887948in}}%
\pgfpathlineto{\pgfqpoint{0.663121in}{1.877942in}}%
\pgfpathlineto{\pgfqpoint{0.672440in}{1.873275in}}%
\pgfpathlineto{\pgfqpoint{0.700396in}{1.864682in}}%
\pgfpathlineto{\pgfqpoint{0.709714in}{1.856183in}}%
\pgfpathlineto{\pgfqpoint{0.719033in}{1.843011in}}%
\pgfpathlineto{\pgfqpoint{0.731458in}{1.818805in}}%
\pgfpathlineto{\pgfqpoint{0.778051in}{1.716968in}}%
\pgfpathlineto{\pgfqpoint{0.790476in}{1.701449in}}%
\pgfpathlineto{\pgfqpoint{0.799795in}{1.694277in}}%
\pgfpathlineto{\pgfqpoint{0.809113in}{1.690470in}}%
\pgfpathlineto{\pgfqpoint{0.818432in}{1.689499in}}%
\pgfpathlineto{\pgfqpoint{0.830857in}{1.691889in}}%
\pgfpathlineto{\pgfqpoint{0.843282in}{1.698125in}}%
\pgfpathlineto{\pgfqpoint{0.855706in}{1.708437in}}%
\pgfpathlineto{\pgfqpoint{0.868131in}{1.723365in}}%
\pgfpathlineto{\pgfqpoint{0.880556in}{1.743190in}}%
\pgfpathlineto{\pgfqpoint{0.896087in}{1.774016in}}%
\pgfpathlineto{\pgfqpoint{0.930256in}{1.846035in}}%
\pgfpathlineto{\pgfqpoint{0.939574in}{1.859545in}}%
\pgfpathlineto{\pgfqpoint{0.945787in}{1.865351in}}%
\pgfpathlineto{\pgfqpoint{0.951999in}{1.867974in}}%
\pgfpathlineto{\pgfqpoint{0.958211in}{1.866911in}}%
\pgfpathlineto{\pgfqpoint{0.964424in}{1.861717in}}%
\pgfpathlineto{\pgfqpoint{0.970636in}{1.852031in}}%
\pgfpathlineto{\pgfqpoint{0.976849in}{1.837602in}}%
\pgfpathlineto{\pgfqpoint{0.986167in}{1.806869in}}%
\pgfpathlineto{\pgfqpoint{0.995486in}{1.765613in}}%
\pgfpathlineto{\pgfqpoint{1.007911in}{1.696705in}}%
\pgfpathlineto{\pgfqpoint{1.051398in}{1.434598in}}%
\pgfpathlineto{\pgfqpoint{1.060716in}{1.400032in}}%
\pgfpathlineto{\pgfqpoint{1.066929in}{1.384919in}}%
\pgfpathlineto{\pgfqpoint{1.073141in}{1.376564in}}%
\pgfpathlineto{\pgfqpoint{1.079354in}{1.374981in}}%
\pgfpathlineto{\pgfqpoint{1.085566in}{1.379883in}}%
\pgfpathlineto{\pgfqpoint{1.091779in}{1.390698in}}%
\pgfpathlineto{\pgfqpoint{1.101097in}{1.416173in}}%
\pgfpathlineto{\pgfqpoint{1.113522in}{1.461812in}}%
\pgfpathlineto{\pgfqpoint{1.141478in}{1.568885in}}%
\pgfpathlineto{\pgfqpoint{1.153903in}{1.601980in}}%
\pgfpathlineto{\pgfqpoint{1.163221in}{1.617866in}}%
\pgfpathlineto{\pgfqpoint{1.169434in}{1.624194in}}%
\pgfpathlineto{\pgfqpoint{1.175646in}{1.627355in}}%
\pgfpathlineto{\pgfqpoint{1.181859in}{1.627679in}}%
\pgfpathlineto{\pgfqpoint{1.188071in}{1.625567in}}%
\pgfpathlineto{\pgfqpoint{1.197390in}{1.618790in}}%
\pgfpathlineto{\pgfqpoint{1.209815in}{1.605358in}}%
\pgfpathlineto{\pgfqpoint{1.237771in}{1.573248in}}%
\pgfpathlineto{\pgfqpoint{1.247089in}{1.567070in}}%
\pgfpathlineto{\pgfqpoint{1.253302in}{1.565583in}}%
\pgfpathlineto{\pgfqpoint{1.259514in}{1.566811in}}%
\pgfpathlineto{\pgfqpoint{1.265726in}{1.571284in}}%
\pgfpathlineto{\pgfqpoint{1.271939in}{1.579502in}}%
\pgfpathlineto{\pgfqpoint{1.278151in}{1.591907in}}%
\pgfpathlineto{\pgfqpoint{1.287470in}{1.619053in}}%
\pgfpathlineto{\pgfqpoint{1.296789in}{1.656792in}}%
\pgfpathlineto{\pgfqpoint{1.309213in}{1.722412in}}%
\pgfpathlineto{\pgfqpoint{1.327851in}{1.842560in}}%
\pgfpathlineto{\pgfqpoint{1.346488in}{1.960869in}}%
\pgfpathlineto{\pgfqpoint{1.358913in}{2.020409in}}%
\pgfpathlineto{\pgfqpoint{1.368231in}{2.048221in}}%
\pgfpathlineto{\pgfqpoint{1.374444in}{2.057361in}}%
\pgfpathlineto{\pgfqpoint{1.377550in}{2.058984in}}%
\pgfpathlineto{\pgfqpoint{1.380656in}{2.058637in}}%
\pgfpathlineto{\pgfqpoint{1.386869in}{2.052151in}}%
\pgfpathlineto{\pgfqpoint{1.393081in}{2.038357in}}%
\pgfpathlineto{\pgfqpoint{1.402400in}{2.005742in}}%
\pgfpathlineto{\pgfqpoint{1.414825in}{1.946134in}}%
\pgfpathlineto{\pgfqpoint{1.442781in}{1.801491in}}%
\pgfpathlineto{\pgfqpoint{1.452099in}{1.767103in}}%
\pgfpathlineto{\pgfqpoint{1.458312in}{1.751252in}}%
\pgfpathlineto{\pgfqpoint{1.464524in}{1.741830in}}%
\pgfpathlineto{\pgfqpoint{1.470736in}{1.739192in}}%
\pgfpathlineto{\pgfqpoint{1.473843in}{1.740453in}}%
\pgfpathlineto{\pgfqpoint{1.480055in}{1.748046in}}%
\pgfpathlineto{\pgfqpoint{1.486268in}{1.762089in}}%
\pgfpathlineto{\pgfqpoint{1.495586in}{1.793847in}}%
\pgfpathlineto{\pgfqpoint{1.508011in}{1.851105in}}%
\pgfpathlineto{\pgfqpoint{1.539073in}{2.004051in}}%
\pgfpathlineto{\pgfqpoint{1.548392in}{2.034061in}}%
\pgfpathlineto{\pgfqpoint{1.554604in}{2.046767in}}%
\pgfpathlineto{\pgfqpoint{1.560817in}{2.053075in}}%
\pgfpathlineto{\pgfqpoint{1.563923in}{2.053786in}}%
\pgfpathlineto{\pgfqpoint{1.570135in}{2.050423in}}%
\pgfpathlineto{\pgfqpoint{1.576348in}{2.041060in}}%
\pgfpathlineto{\pgfqpoint{1.585666in}{2.017423in}}%
\pgfpathlineto{\pgfqpoint{1.598091in}{1.973650in}}%
\pgfpathlineto{\pgfqpoint{1.622941in}{1.881561in}}%
\pgfpathlineto{\pgfqpoint{1.632260in}{1.856389in}}%
\pgfpathlineto{\pgfqpoint{1.641578in}{1.839233in}}%
\pgfpathlineto{\pgfqpoint{1.650897in}{1.829876in}}%
\pgfpathlineto{\pgfqpoint{1.660215in}{1.826724in}}%
\pgfpathlineto{\pgfqpoint{1.672640in}{1.827547in}}%
\pgfpathlineto{\pgfqpoint{1.681959in}{1.827804in}}%
\pgfpathlineto{\pgfqpoint{1.688171in}{1.826070in}}%
\pgfpathlineto{\pgfqpoint{1.694384in}{1.821904in}}%
\pgfpathlineto{\pgfqpoint{1.700596in}{1.814728in}}%
\pgfpathlineto{\pgfqpoint{1.709915in}{1.797610in}}%
\pgfpathlineto{\pgfqpoint{1.719233in}{1.772816in}}%
\pgfpathlineto{\pgfqpoint{1.731658in}{1.729862in}}%
\pgfpathlineto{\pgfqpoint{1.765827in}{1.604009in}}%
\pgfpathlineto{\pgfqpoint{1.775145in}{1.583306in}}%
\pgfpathlineto{\pgfqpoint{1.781358in}{1.575610in}}%
\pgfpathlineto{\pgfqpoint{1.787570in}{1.573410in}}%
\pgfpathlineto{\pgfqpoint{1.793783in}{1.577000in}}%
\pgfpathlineto{\pgfqpoint{1.799995in}{1.586460in}}%
\pgfpathlineto{\pgfqpoint{1.806207in}{1.601656in}}%
\pgfpathlineto{\pgfqpoint{1.815526in}{1.634412in}}%
\pgfpathlineto{\pgfqpoint{1.827951in}{1.693509in}}%
\pgfpathlineto{\pgfqpoint{1.849694in}{1.820129in}}%
\pgfpathlineto{\pgfqpoint{1.868332in}{1.924939in}}%
\pgfpathlineto{\pgfqpoint{1.880757in}{1.980050in}}%
\pgfpathlineto{\pgfqpoint{1.890075in}{2.009918in}}%
\pgfpathlineto{\pgfqpoint{1.899394in}{2.028638in}}%
\pgfpathlineto{\pgfqpoint{1.902500in}{2.032330in}}%
\pgfpathlineto{\pgfqpoint{1.902500in}{2.032330in}}%
\pgfusepath{stroke}%
\end{pgfscope}%
\begin{pgfscope}%
\pgfpathrectangle{\pgfqpoint{0.275000in}{0.375000in}}{\pgfqpoint{1.705000in}{2.265000in}}%
\pgfusepath{clip}%
\pgfsetroundcap%
\pgfsetroundjoin%
\pgfsetlinewidth{1.505625pt}%
\definecolor{currentstroke}{rgb}{0.737255,0.741176,0.133333}%
\pgfsetstrokecolor{currentstroke}%
\pgfsetdash{}{0pt}%
\pgfpathmoveto{\pgfqpoint{0.352500in}{1.732104in}}%
\pgfpathlineto{\pgfqpoint{0.361819in}{1.768447in}}%
\pgfpathlineto{\pgfqpoint{0.374243in}{1.829445in}}%
\pgfpathlineto{\pgfqpoint{0.408412in}{2.007238in}}%
\pgfpathlineto{\pgfqpoint{0.417730in}{2.037494in}}%
\pgfpathlineto{\pgfqpoint{0.423943in}{2.048916in}}%
\pgfpathlineto{\pgfqpoint{0.427049in}{2.051610in}}%
\pgfpathlineto{\pgfqpoint{0.430155in}{2.052152in}}%
\pgfpathlineto{\pgfqpoint{0.433262in}{2.050460in}}%
\pgfpathlineto{\pgfqpoint{0.439474in}{2.040151in}}%
\pgfpathlineto{\pgfqpoint{0.445686in}{2.020440in}}%
\pgfpathlineto{\pgfqpoint{0.455005in}{1.973603in}}%
\pgfpathlineto{\pgfqpoint{0.464324in}{1.907874in}}%
\pgfpathlineto{\pgfqpoint{0.476748in}{1.797231in}}%
\pgfpathlineto{\pgfqpoint{0.523342in}{1.354031in}}%
\pgfpathlineto{\pgfqpoint{0.535767in}{1.279287in}}%
\pgfpathlineto{\pgfqpoint{0.545085in}{1.242381in}}%
\pgfpathlineto{\pgfqpoint{0.551298in}{1.226957in}}%
\pgfpathlineto{\pgfqpoint{0.557510in}{1.218634in}}%
\pgfpathlineto{\pgfqpoint{0.563722in}{1.217070in}}%
\pgfpathlineto{\pgfqpoint{0.569935in}{1.221847in}}%
\pgfpathlineto{\pgfqpoint{0.576147in}{1.232500in}}%
\pgfpathlineto{\pgfqpoint{0.585466in}{1.258424in}}%
\pgfpathlineto{\pgfqpoint{0.594785in}{1.294766in}}%
\pgfpathlineto{\pgfqpoint{0.607209in}{1.356322in}}%
\pgfpathlineto{\pgfqpoint{0.625847in}{1.466879in}}%
\pgfpathlineto{\pgfqpoint{0.653803in}{1.634293in}}%
\pgfpathlineto{\pgfqpoint{0.666227in}{1.693523in}}%
\pgfpathlineto{\pgfqpoint{0.678652in}{1.737329in}}%
\pgfpathlineto{\pgfqpoint{0.687971in}{1.759627in}}%
\pgfpathlineto{\pgfqpoint{0.697290in}{1.774038in}}%
\pgfpathlineto{\pgfqpoint{0.706608in}{1.782795in}}%
\pgfpathlineto{\pgfqpoint{0.728352in}{1.799166in}}%
\pgfpathlineto{\pgfqpoint{0.737670in}{1.812593in}}%
\pgfpathlineto{\pgfqpoint{0.746989in}{1.833789in}}%
\pgfpathlineto{\pgfqpoint{0.756308in}{1.864018in}}%
\pgfpathlineto{\pgfqpoint{0.768732in}{1.917808in}}%
\pgfpathlineto{\pgfqpoint{0.790476in}{2.033123in}}%
\pgfpathlineto{\pgfqpoint{0.806007in}{2.108326in}}%
\pgfpathlineto{\pgfqpoint{0.815326in}{2.140698in}}%
\pgfpathlineto{\pgfqpoint{0.821538in}{2.154655in}}%
\pgfpathlineto{\pgfqpoint{0.827751in}{2.161603in}}%
\pgfpathlineto{\pgfqpoint{0.830857in}{2.162290in}}%
\pgfpathlineto{\pgfqpoint{0.833963in}{2.161072in}}%
\pgfpathlineto{\pgfqpoint{0.840175in}{2.152899in}}%
\pgfpathlineto{\pgfqpoint{0.846388in}{2.137220in}}%
\pgfpathlineto{\pgfqpoint{0.855706in}{2.100630in}}%
\pgfpathlineto{\pgfqpoint{0.868131in}{2.031602in}}%
\pgfpathlineto{\pgfqpoint{0.886769in}{1.902493in}}%
\pgfpathlineto{\pgfqpoint{0.911618in}{1.732862in}}%
\pgfpathlineto{\pgfqpoint{0.924043in}{1.666597in}}%
\pgfpathlineto{\pgfqpoint{0.936468in}{1.618237in}}%
\pgfpathlineto{\pgfqpoint{0.945787in}{1.594056in}}%
\pgfpathlineto{\pgfqpoint{0.955105in}{1.579422in}}%
\pgfpathlineto{\pgfqpoint{0.961318in}{1.574301in}}%
\pgfpathlineto{\pgfqpoint{0.967530in}{1.572306in}}%
\pgfpathlineto{\pgfqpoint{0.973742in}{1.572900in}}%
\pgfpathlineto{\pgfqpoint{0.983061in}{1.577490in}}%
\pgfpathlineto{\pgfqpoint{0.995486in}{1.588050in}}%
\pgfpathlineto{\pgfqpoint{1.017229in}{1.611762in}}%
\pgfpathlineto{\pgfqpoint{1.038973in}{1.639594in}}%
\pgfpathlineto{\pgfqpoint{1.051398in}{1.659645in}}%
\pgfpathlineto{\pgfqpoint{1.063823in}{1.685216in}}%
\pgfpathlineto{\pgfqpoint{1.076247in}{1.718359in}}%
\pgfpathlineto{\pgfqpoint{1.088672in}{1.760698in}}%
\pgfpathlineto{\pgfqpoint{1.101097in}{1.813045in}}%
\pgfpathlineto{\pgfqpoint{1.116628in}{1.891815in}}%
\pgfpathlineto{\pgfqpoint{1.144584in}{2.055076in}}%
\pgfpathlineto{\pgfqpoint{1.163221in}{2.155664in}}%
\pgfpathlineto{\pgfqpoint{1.175646in}{2.206577in}}%
\pgfpathlineto{\pgfqpoint{1.184965in}{2.232205in}}%
\pgfpathlineto{\pgfqpoint{1.191177in}{2.242337in}}%
\pgfpathlineto{\pgfqpoint{1.197390in}{2.246546in}}%
\pgfpathlineto{\pgfqpoint{1.203602in}{2.244713in}}%
\pgfpathlineto{\pgfqpoint{1.209815in}{2.236900in}}%
\pgfpathlineto{\pgfqpoint{1.216027in}{2.223334in}}%
\pgfpathlineto{\pgfqpoint{1.225346in}{2.193074in}}%
\pgfpathlineto{\pgfqpoint{1.237771in}{2.137156in}}%
\pgfpathlineto{\pgfqpoint{1.253302in}{2.049728in}}%
\pgfpathlineto{\pgfqpoint{1.334063in}{1.563145in}}%
\pgfpathlineto{\pgfqpoint{1.349594in}{1.493217in}}%
\pgfpathlineto{\pgfqpoint{1.362019in}{1.450919in}}%
\pgfpathlineto{\pgfqpoint{1.371338in}{1.429218in}}%
\pgfpathlineto{\pgfqpoint{1.377550in}{1.420049in}}%
\pgfpathlineto{\pgfqpoint{1.383763in}{1.415275in}}%
\pgfpathlineto{\pgfqpoint{1.389975in}{1.414895in}}%
\pgfpathlineto{\pgfqpoint{1.396187in}{1.418781in}}%
\pgfpathlineto{\pgfqpoint{1.402400in}{1.426692in}}%
\pgfpathlineto{\pgfqpoint{1.411718in}{1.445347in}}%
\pgfpathlineto{\pgfqpoint{1.424143in}{1.480691in}}%
\pgfpathlineto{\pgfqpoint{1.439674in}{1.536894in}}%
\pgfpathlineto{\pgfqpoint{1.461418in}{1.629190in}}%
\pgfpathlineto{\pgfqpoint{1.489374in}{1.762755in}}%
\pgfpathlineto{\pgfqpoint{1.526648in}{1.942770in}}%
\pgfpathlineto{\pgfqpoint{1.539073in}{1.988675in}}%
\pgfpathlineto{\pgfqpoint{1.548392in}{2.013448in}}%
\pgfpathlineto{\pgfqpoint{1.554604in}{2.024438in}}%
\pgfpathlineto{\pgfqpoint{1.560817in}{2.030636in}}%
\pgfpathlineto{\pgfqpoint{1.567029in}{2.031917in}}%
\pgfpathlineto{\pgfqpoint{1.573241in}{2.028340in}}%
\pgfpathlineto{\pgfqpoint{1.579454in}{2.020159in}}%
\pgfpathlineto{\pgfqpoint{1.588773in}{2.000251in}}%
\pgfpathlineto{\pgfqpoint{1.601197in}{1.962928in}}%
\pgfpathlineto{\pgfqpoint{1.641578in}{1.829849in}}%
\pgfpathlineto{\pgfqpoint{1.654003in}{1.803370in}}%
\pgfpathlineto{\pgfqpoint{1.663322in}{1.789618in}}%
\pgfpathlineto{\pgfqpoint{1.675746in}{1.777658in}}%
\pgfpathlineto{\pgfqpoint{1.706809in}{1.753826in}}%
\pgfpathlineto{\pgfqpoint{1.716127in}{1.740023in}}%
\pgfpathlineto{\pgfqpoint{1.725446in}{1.719840in}}%
\pgfpathlineto{\pgfqpoint{1.734765in}{1.691996in}}%
\pgfpathlineto{\pgfqpoint{1.747189in}{1.641892in}}%
\pgfpathlineto{\pgfqpoint{1.759614in}{1.577558in}}%
\pgfpathlineto{\pgfqpoint{1.778252in}{1.461915in}}%
\pgfpathlineto{\pgfqpoint{1.806207in}{1.287893in}}%
\pgfpathlineto{\pgfqpoint{1.818632in}{1.228585in}}%
\pgfpathlineto{\pgfqpoint{1.827951in}{1.195940in}}%
\pgfpathlineto{\pgfqpoint{1.837270in}{1.174275in}}%
\pgfpathlineto{\pgfqpoint{1.843482in}{1.165770in}}%
\pgfpathlineto{\pgfqpoint{1.849694in}{1.161611in}}%
\pgfpathlineto{\pgfqpoint{1.855907in}{1.161260in}}%
\pgfpathlineto{\pgfqpoint{1.862119in}{1.164054in}}%
\pgfpathlineto{\pgfqpoint{1.871438in}{1.172506in}}%
\pgfpathlineto{\pgfqpoint{1.902500in}{1.208815in}}%
\pgfpathlineto{\pgfqpoint{1.902500in}{1.208815in}}%
\pgfusepath{stroke}%
\end{pgfscope}%
\begin{pgfscope}%
\pgfpathrectangle{\pgfqpoint{0.275000in}{0.375000in}}{\pgfqpoint{1.705000in}{2.265000in}}%
\pgfusepath{clip}%
\pgfsetroundcap%
\pgfsetroundjoin%
\pgfsetlinewidth{1.505625pt}%
\definecolor{currentstroke}{rgb}{0.090196,0.745098,0.811765}%
\pgfsetstrokecolor{currentstroke}%
\pgfsetdash{}{0pt}%
\pgfpathmoveto{\pgfqpoint{0.352500in}{1.352309in}}%
\pgfpathlineto{\pgfqpoint{0.371137in}{1.374562in}}%
\pgfpathlineto{\pgfqpoint{0.389775in}{1.401900in}}%
\pgfpathlineto{\pgfqpoint{0.408412in}{1.435731in}}%
\pgfpathlineto{\pgfqpoint{0.430155in}{1.483433in}}%
\pgfpathlineto{\pgfqpoint{0.451899in}{1.531118in}}%
\pgfpathlineto{\pgfqpoint{0.464324in}{1.551311in}}%
\pgfpathlineto{\pgfqpoint{0.473642in}{1.560314in}}%
\pgfpathlineto{\pgfqpoint{0.479855in}{1.562802in}}%
\pgfpathlineto{\pgfqpoint{0.486067in}{1.562307in}}%
\pgfpathlineto{\pgfqpoint{0.492280in}{1.558841in}}%
\pgfpathlineto{\pgfqpoint{0.501598in}{1.548425in}}%
\pgfpathlineto{\pgfqpoint{0.510917in}{1.532657in}}%
\pgfpathlineto{\pgfqpoint{0.526448in}{1.498229in}}%
\pgfpathlineto{\pgfqpoint{0.600997in}{1.319035in}}%
\pgfpathlineto{\pgfqpoint{0.622740in}{1.275483in}}%
\pgfpathlineto{\pgfqpoint{0.644484in}{1.238869in}}%
\pgfpathlineto{\pgfqpoint{0.669334in}{1.196724in}}%
\pgfpathlineto{\pgfqpoint{0.681759in}{1.169215in}}%
\pgfpathlineto{\pgfqpoint{0.694183in}{1.134516in}}%
\pgfpathlineto{\pgfqpoint{0.709714in}{1.081364in}}%
\pgfpathlineto{\pgfqpoint{0.740777in}{0.971184in}}%
\pgfpathlineto{\pgfqpoint{0.750095in}{0.949161in}}%
\pgfpathlineto{\pgfqpoint{0.756308in}{0.940231in}}%
\pgfpathlineto{\pgfqpoint{0.762520in}{0.936874in}}%
\pgfpathlineto{\pgfqpoint{0.765626in}{0.937506in}}%
\pgfpathlineto{\pgfqpoint{0.771839in}{0.943740in}}%
\pgfpathlineto{\pgfqpoint{0.778051in}{0.956926in}}%
\pgfpathlineto{\pgfqpoint{0.784264in}{0.977260in}}%
\pgfpathlineto{\pgfqpoint{0.793582in}{1.021003in}}%
\pgfpathlineto{\pgfqpoint{0.802901in}{1.079495in}}%
\pgfpathlineto{\pgfqpoint{0.815326in}{1.176150in}}%
\pgfpathlineto{\pgfqpoint{0.865025in}{1.595193in}}%
\pgfpathlineto{\pgfqpoint{0.877450in}{1.662270in}}%
\pgfpathlineto{\pgfqpoint{0.886769in}{1.696045in}}%
\pgfpathlineto{\pgfqpoint{0.892981in}{1.710678in}}%
\pgfpathlineto{\pgfqpoint{0.899193in}{1.719309in}}%
\pgfpathlineto{\pgfqpoint{0.905406in}{1.722376in}}%
\pgfpathlineto{\pgfqpoint{0.911618in}{1.720438in}}%
\pgfpathlineto{\pgfqpoint{0.917831in}{1.714135in}}%
\pgfpathlineto{\pgfqpoint{0.927149in}{1.697999in}}%
\pgfpathlineto{\pgfqpoint{0.939574in}{1.667615in}}%
\pgfpathlineto{\pgfqpoint{0.961318in}{1.603287in}}%
\pgfpathlineto{\pgfqpoint{0.992380in}{1.512738in}}%
\pgfpathlineto{\pgfqpoint{1.014123in}{1.457624in}}%
\pgfpathlineto{\pgfqpoint{1.029654in}{1.425535in}}%
\pgfpathlineto{\pgfqpoint{1.038973in}{1.411491in}}%
\pgfpathlineto{\pgfqpoint{1.048292in}{1.403337in}}%
\pgfpathlineto{\pgfqpoint{1.054504in}{1.402091in}}%
\pgfpathlineto{\pgfqpoint{1.060716in}{1.404834in}}%
\pgfpathlineto{\pgfqpoint{1.066929in}{1.412050in}}%
\pgfpathlineto{\pgfqpoint{1.073141in}{1.424128in}}%
\pgfpathlineto{\pgfqpoint{1.082460in}{1.451882in}}%
\pgfpathlineto{\pgfqpoint{1.091779in}{1.491286in}}%
\pgfpathlineto{\pgfqpoint{1.104203in}{1.560528in}}%
\pgfpathlineto{\pgfqpoint{1.119734in}{1.667620in}}%
\pgfpathlineto{\pgfqpoint{1.163221in}{1.982388in}}%
\pgfpathlineto{\pgfqpoint{1.175646in}{2.049275in}}%
\pgfpathlineto{\pgfqpoint{1.188071in}{2.098242in}}%
\pgfpathlineto{\pgfqpoint{1.197390in}{2.121937in}}%
\pgfpathlineto{\pgfqpoint{1.203602in}{2.131220in}}%
\pgfpathlineto{\pgfqpoint{1.209815in}{2.135179in}}%
\pgfpathlineto{\pgfqpoint{1.216027in}{2.133777in}}%
\pgfpathlineto{\pgfqpoint{1.222239in}{2.127031in}}%
\pgfpathlineto{\pgfqpoint{1.228452in}{2.115026in}}%
\pgfpathlineto{\pgfqpoint{1.237771in}{2.087547in}}%
\pgfpathlineto{\pgfqpoint{1.247089in}{2.049593in}}%
\pgfpathlineto{\pgfqpoint{1.259514in}{1.985399in}}%
\pgfpathlineto{\pgfqpoint{1.281258in}{1.851332in}}%
\pgfpathlineto{\pgfqpoint{1.303001in}{1.722041in}}%
\pgfpathlineto{\pgfqpoint{1.315426in}{1.664699in}}%
\pgfpathlineto{\pgfqpoint{1.324744in}{1.633523in}}%
\pgfpathlineto{\pgfqpoint{1.334063in}{1.613716in}}%
\pgfpathlineto{\pgfqpoint{1.340276in}{1.606956in}}%
\pgfpathlineto{\pgfqpoint{1.346488in}{1.605206in}}%
\pgfpathlineto{\pgfqpoint{1.352700in}{1.608180in}}%
\pgfpathlineto{\pgfqpoint{1.358913in}{1.615467in}}%
\pgfpathlineto{\pgfqpoint{1.368231in}{1.633338in}}%
\pgfpathlineto{\pgfqpoint{1.380656in}{1.666869in}}%
\pgfpathlineto{\pgfqpoint{1.427249in}{1.805579in}}%
\pgfpathlineto{\pgfqpoint{1.436568in}{1.822720in}}%
\pgfpathlineto{\pgfqpoint{1.445887in}{1.834324in}}%
\pgfpathlineto{\pgfqpoint{1.455205in}{1.840656in}}%
\pgfpathlineto{\pgfqpoint{1.464524in}{1.842461in}}%
\pgfpathlineto{\pgfqpoint{1.473843in}{1.840785in}}%
\pgfpathlineto{\pgfqpoint{1.486268in}{1.835028in}}%
\pgfpathlineto{\pgfqpoint{1.504905in}{1.822464in}}%
\pgfpathlineto{\pgfqpoint{1.520436in}{1.808386in}}%
\pgfpathlineto{\pgfqpoint{1.529755in}{1.796608in}}%
\pgfpathlineto{\pgfqpoint{1.539073in}{1.780569in}}%
\pgfpathlineto{\pgfqpoint{1.548392in}{1.758678in}}%
\pgfpathlineto{\pgfqpoint{1.557710in}{1.729543in}}%
\pgfpathlineto{\pgfqpoint{1.570135in}{1.677942in}}%
\pgfpathlineto{\pgfqpoint{1.582560in}{1.611797in}}%
\pgfpathlineto{\pgfqpoint{1.601197in}{1.491959in}}%
\pgfpathlineto{\pgfqpoint{1.629153in}{1.309784in}}%
\pgfpathlineto{\pgfqpoint{1.641578in}{1.248778in}}%
\pgfpathlineto{\pgfqpoint{1.650897in}{1.217101in}}%
\pgfpathlineto{\pgfqpoint{1.657109in}{1.203566in}}%
\pgfpathlineto{\pgfqpoint{1.663322in}{1.196255in}}%
\pgfpathlineto{\pgfqpoint{1.669534in}{1.195006in}}%
\pgfpathlineto{\pgfqpoint{1.675746in}{1.199399in}}%
\pgfpathlineto{\pgfqpoint{1.681959in}{1.208774in}}%
\pgfpathlineto{\pgfqpoint{1.691278in}{1.230245in}}%
\pgfpathlineto{\pgfqpoint{1.709915in}{1.286445in}}%
\pgfpathlineto{\pgfqpoint{1.725446in}{1.328967in}}%
\pgfpathlineto{\pgfqpoint{1.734765in}{1.346344in}}%
\pgfpathlineto{\pgfqpoint{1.740977in}{1.353494in}}%
\pgfpathlineto{\pgfqpoint{1.747189in}{1.356944in}}%
\pgfpathlineto{\pgfqpoint{1.753402in}{1.356848in}}%
\pgfpathlineto{\pgfqpoint{1.759614in}{1.353594in}}%
\pgfpathlineto{\pgfqpoint{1.768933in}{1.344119in}}%
\pgfpathlineto{\pgfqpoint{1.799995in}{1.305345in}}%
\pgfpathlineto{\pgfqpoint{1.806207in}{1.301895in}}%
\pgfpathlineto{\pgfqpoint{1.812420in}{1.300970in}}%
\pgfpathlineto{\pgfqpoint{1.818632in}{1.302612in}}%
\pgfpathlineto{\pgfqpoint{1.827951in}{1.309390in}}%
\pgfpathlineto{\pgfqpoint{1.840376in}{1.323773in}}%
\pgfpathlineto{\pgfqpoint{1.855907in}{1.341898in}}%
\pgfpathlineto{\pgfqpoint{1.865225in}{1.348139in}}%
\pgfpathlineto{\pgfqpoint{1.871438in}{1.349137in}}%
\pgfpathlineto{\pgfqpoint{1.877650in}{1.347304in}}%
\pgfpathlineto{\pgfqpoint{1.883863in}{1.342709in}}%
\pgfpathlineto{\pgfqpoint{1.893181in}{1.331571in}}%
\pgfpathlineto{\pgfqpoint{1.902500in}{1.317739in}}%
\pgfpathlineto{\pgfqpoint{1.902500in}{1.317739in}}%
\pgfusepath{stroke}%
\end{pgfscope}%
\begin{pgfscope}%
\pgfsetrectcap%
\pgfsetmiterjoin%
\pgfsetlinewidth{0.000000pt}%
\definecolor{currentstroke}{rgb}{1.000000,1.000000,1.000000}%
\pgfsetstrokecolor{currentstroke}%
\pgfsetdash{}{0pt}%
\pgfpathmoveto{\pgfqpoint{0.275000in}{0.375000in}}%
\pgfpathlineto{\pgfqpoint{0.275000in}{2.640000in}}%
\pgfusepath{}%
\end{pgfscope}%
\begin{pgfscope}%
\pgfsetrectcap%
\pgfsetmiterjoin%
\pgfsetlinewidth{0.000000pt}%
\definecolor{currentstroke}{rgb}{1.000000,1.000000,1.000000}%
\pgfsetstrokecolor{currentstroke}%
\pgfsetdash{}{0pt}%
\pgfpathmoveto{\pgfqpoint{1.980000in}{0.375000in}}%
\pgfpathlineto{\pgfqpoint{1.980000in}{2.640000in}}%
\pgfusepath{}%
\end{pgfscope}%
\begin{pgfscope}%
\pgfsetrectcap%
\pgfsetmiterjoin%
\pgfsetlinewidth{0.000000pt}%
\definecolor{currentstroke}{rgb}{1.000000,1.000000,1.000000}%
\pgfsetstrokecolor{currentstroke}%
\pgfsetdash{}{0pt}%
\pgfpathmoveto{\pgfqpoint{0.275000in}{0.375000in}}%
\pgfpathlineto{\pgfqpoint{1.980000in}{0.375000in}}%
\pgfusepath{}%
\end{pgfscope}%
\begin{pgfscope}%
\pgfsetrectcap%
\pgfsetmiterjoin%
\pgfsetlinewidth{0.000000pt}%
\definecolor{currentstroke}{rgb}{1.000000,1.000000,1.000000}%
\pgfsetstrokecolor{currentstroke}%
\pgfsetdash{}{0pt}%
\pgfpathmoveto{\pgfqpoint{0.275000in}{2.640000in}}%
\pgfpathlineto{\pgfqpoint{1.980000in}{2.640000in}}%
\pgfusepath{}%
\end{pgfscope}%
\end{pgfpicture}%
\makeatother%
\endgroup%

            \end{subfigure}
            \begin{subfigure}[t]{0.31\textwidth}
                \centering
                %% Creator: Matplotlib, PGF backend
%%
%% To include the figure in your LaTeX document, write
%%   \input{<filename>.pgf}
%%
%% Make sure the required packages are loaded in your preamble
%%   \usepackage{pgf}
%%
%% Figures using additional raster images can only be included by \input if
%% they are in the same directory as the main LaTeX file. For loading figures
%% from other directories you can use the `import` package
%%   \usepackage{import}
%% and then include the figures with
%%   \import{<path to file>}{<filename>.pgf}
%%
%% Matplotlib used the following preamble
%%   \usepackage{gensymb}
%%   \usepackage{fontspec}
%%   \setmainfont{DejaVu Serif}
%%   \setsansfont{Arial}
%%   \setmonofont{DejaVu Sans Mono}
%%
\begingroup%
\makeatletter%
\begin{pgfpicture}%
\pgfpathrectangle{\pgfpointorigin}{\pgfqpoint{2.200000in}{3.000000in}}%
\pgfusepath{use as bounding box, clip}%
\begin{pgfscope}%
\pgfsetbuttcap%
\pgfsetmiterjoin%
\definecolor{currentfill}{rgb}{1.000000,1.000000,1.000000}%
\pgfsetfillcolor{currentfill}%
\pgfsetlinewidth{0.000000pt}%
\definecolor{currentstroke}{rgb}{1.000000,1.000000,1.000000}%
\pgfsetstrokecolor{currentstroke}%
\pgfsetdash{}{0pt}%
\pgfpathmoveto{\pgfqpoint{0.000000in}{0.000000in}}%
\pgfpathlineto{\pgfqpoint{2.200000in}{0.000000in}}%
\pgfpathlineto{\pgfqpoint{2.200000in}{3.000000in}}%
\pgfpathlineto{\pgfqpoint{0.000000in}{3.000000in}}%
\pgfpathclose%
\pgfusepath{fill}%
\end{pgfscope}%
\begin{pgfscope}%
\pgfsetbuttcap%
\pgfsetmiterjoin%
\definecolor{currentfill}{rgb}{0.917647,0.917647,0.949020}%
\pgfsetfillcolor{currentfill}%
\pgfsetlinewidth{0.000000pt}%
\definecolor{currentstroke}{rgb}{0.000000,0.000000,0.000000}%
\pgfsetstrokecolor{currentstroke}%
\pgfsetstrokeopacity{0.000000}%
\pgfsetdash{}{0pt}%
\pgfpathmoveto{\pgfqpoint{0.275000in}{0.375000in}}%
\pgfpathlineto{\pgfqpoint{1.980000in}{0.375000in}}%
\pgfpathlineto{\pgfqpoint{1.980000in}{2.640000in}}%
\pgfpathlineto{\pgfqpoint{0.275000in}{2.640000in}}%
\pgfpathclose%
\pgfusepath{fill}%
\end{pgfscope}%
\begin{pgfscope}%
\pgfpathrectangle{\pgfqpoint{0.275000in}{0.375000in}}{\pgfqpoint{1.705000in}{2.265000in}}%
\pgfusepath{clip}%
\pgfsetroundcap%
\pgfsetroundjoin%
\pgfsetlinewidth{0.803000pt}%
\definecolor{currentstroke}{rgb}{1.000000,1.000000,1.000000}%
\pgfsetstrokecolor{currentstroke}%
\pgfsetdash{}{0pt}%
\pgfpathmoveto{\pgfqpoint{0.352500in}{0.375000in}}%
\pgfpathlineto{\pgfqpoint{0.352500in}{2.640000in}}%
\pgfusepath{stroke}%
\end{pgfscope}%
\begin{pgfscope}%
\definecolor{textcolor}{rgb}{0.150000,0.150000,0.150000}%
\pgfsetstrokecolor{textcolor}%
\pgfsetfillcolor{textcolor}%
\pgftext[x=0.352500in,y=0.326389in,,top]{\color{textcolor}\rmfamily\fontsize{8.000000}{9.600000}\selectfont \(\displaystyle 0.00\)}%
\end{pgfscope}%
\begin{pgfscope}%
\pgfpathrectangle{\pgfqpoint{0.275000in}{0.375000in}}{\pgfqpoint{1.705000in}{2.265000in}}%
\pgfusepath{clip}%
\pgfsetroundcap%
\pgfsetroundjoin%
\pgfsetlinewidth{0.803000pt}%
\definecolor{currentstroke}{rgb}{1.000000,1.000000,1.000000}%
\pgfsetstrokecolor{currentstroke}%
\pgfsetdash{}{0pt}%
\pgfpathmoveto{\pgfqpoint{0.740000in}{0.375000in}}%
\pgfpathlineto{\pgfqpoint{0.740000in}{2.640000in}}%
\pgfusepath{stroke}%
\end{pgfscope}%
\begin{pgfscope}%
\definecolor{textcolor}{rgb}{0.150000,0.150000,0.150000}%
\pgfsetstrokecolor{textcolor}%
\pgfsetfillcolor{textcolor}%
\pgftext[x=0.740000in,y=0.326389in,,top]{\color{textcolor}\rmfamily\fontsize{8.000000}{9.600000}\selectfont \(\displaystyle 0.25\)}%
\end{pgfscope}%
\begin{pgfscope}%
\pgfpathrectangle{\pgfqpoint{0.275000in}{0.375000in}}{\pgfqpoint{1.705000in}{2.265000in}}%
\pgfusepath{clip}%
\pgfsetroundcap%
\pgfsetroundjoin%
\pgfsetlinewidth{0.803000pt}%
\definecolor{currentstroke}{rgb}{1.000000,1.000000,1.000000}%
\pgfsetstrokecolor{currentstroke}%
\pgfsetdash{}{0pt}%
\pgfpathmoveto{\pgfqpoint{1.127500in}{0.375000in}}%
\pgfpathlineto{\pgfqpoint{1.127500in}{2.640000in}}%
\pgfusepath{stroke}%
\end{pgfscope}%
\begin{pgfscope}%
\definecolor{textcolor}{rgb}{0.150000,0.150000,0.150000}%
\pgfsetstrokecolor{textcolor}%
\pgfsetfillcolor{textcolor}%
\pgftext[x=1.127500in,y=0.326389in,,top]{\color{textcolor}\rmfamily\fontsize{8.000000}{9.600000}\selectfont \(\displaystyle 0.50\)}%
\end{pgfscope}%
\begin{pgfscope}%
\pgfpathrectangle{\pgfqpoint{0.275000in}{0.375000in}}{\pgfqpoint{1.705000in}{2.265000in}}%
\pgfusepath{clip}%
\pgfsetroundcap%
\pgfsetroundjoin%
\pgfsetlinewidth{0.803000pt}%
\definecolor{currentstroke}{rgb}{1.000000,1.000000,1.000000}%
\pgfsetstrokecolor{currentstroke}%
\pgfsetdash{}{0pt}%
\pgfpathmoveto{\pgfqpoint{1.515000in}{0.375000in}}%
\pgfpathlineto{\pgfqpoint{1.515000in}{2.640000in}}%
\pgfusepath{stroke}%
\end{pgfscope}%
\begin{pgfscope}%
\definecolor{textcolor}{rgb}{0.150000,0.150000,0.150000}%
\pgfsetstrokecolor{textcolor}%
\pgfsetfillcolor{textcolor}%
\pgftext[x=1.515000in,y=0.326389in,,top]{\color{textcolor}\rmfamily\fontsize{8.000000}{9.600000}\selectfont \(\displaystyle 0.75\)}%
\end{pgfscope}%
\begin{pgfscope}%
\pgfpathrectangle{\pgfqpoint{0.275000in}{0.375000in}}{\pgfqpoint{1.705000in}{2.265000in}}%
\pgfusepath{clip}%
\pgfsetroundcap%
\pgfsetroundjoin%
\pgfsetlinewidth{0.803000pt}%
\definecolor{currentstroke}{rgb}{1.000000,1.000000,1.000000}%
\pgfsetstrokecolor{currentstroke}%
\pgfsetdash{}{0pt}%
\pgfpathmoveto{\pgfqpoint{1.902500in}{0.375000in}}%
\pgfpathlineto{\pgfqpoint{1.902500in}{2.640000in}}%
\pgfusepath{stroke}%
\end{pgfscope}%
\begin{pgfscope}%
\definecolor{textcolor}{rgb}{0.150000,0.150000,0.150000}%
\pgfsetstrokecolor{textcolor}%
\pgfsetfillcolor{textcolor}%
\pgftext[x=1.902500in,y=0.326389in,,top]{\color{textcolor}\rmfamily\fontsize{8.000000}{9.600000}\selectfont \(\displaystyle 1.00\)}%
\end{pgfscope}%
\begin{pgfscope}%
\pgfpathrectangle{\pgfqpoint{0.275000in}{0.375000in}}{\pgfqpoint{1.705000in}{2.265000in}}%
\pgfusepath{clip}%
\pgfsetroundcap%
\pgfsetroundjoin%
\pgfsetlinewidth{0.803000pt}%
\definecolor{currentstroke}{rgb}{1.000000,1.000000,1.000000}%
\pgfsetstrokecolor{currentstroke}%
\pgfsetdash{}{0pt}%
\pgfpathmoveto{\pgfqpoint{0.275000in}{0.484078in}}%
\pgfpathlineto{\pgfqpoint{1.980000in}{0.484078in}}%
\pgfusepath{stroke}%
\end{pgfscope}%
\begin{pgfscope}%
\definecolor{textcolor}{rgb}{0.150000,0.150000,0.150000}%
\pgfsetstrokecolor{textcolor}%
\pgfsetfillcolor{textcolor}%
\pgftext[x=0.075538in,y=0.441869in,left,base]{\color{textcolor}\rmfamily\fontsize{8.000000}{9.600000}\selectfont \(\displaystyle -3\)}%
\end{pgfscope}%
\begin{pgfscope}%
\pgfpathrectangle{\pgfqpoint{0.275000in}{0.375000in}}{\pgfqpoint{1.705000in}{2.265000in}}%
\pgfusepath{clip}%
\pgfsetroundcap%
\pgfsetroundjoin%
\pgfsetlinewidth{0.803000pt}%
\definecolor{currentstroke}{rgb}{1.000000,1.000000,1.000000}%
\pgfsetstrokecolor{currentstroke}%
\pgfsetdash{}{0pt}%
\pgfpathmoveto{\pgfqpoint{0.275000in}{0.811564in}}%
\pgfpathlineto{\pgfqpoint{1.980000in}{0.811564in}}%
\pgfusepath{stroke}%
\end{pgfscope}%
\begin{pgfscope}%
\definecolor{textcolor}{rgb}{0.150000,0.150000,0.150000}%
\pgfsetstrokecolor{textcolor}%
\pgfsetfillcolor{textcolor}%
\pgftext[x=0.075538in,y=0.769355in,left,base]{\color{textcolor}\rmfamily\fontsize{8.000000}{9.600000}\selectfont \(\displaystyle -2\)}%
\end{pgfscope}%
\begin{pgfscope}%
\pgfpathrectangle{\pgfqpoint{0.275000in}{0.375000in}}{\pgfqpoint{1.705000in}{2.265000in}}%
\pgfusepath{clip}%
\pgfsetroundcap%
\pgfsetroundjoin%
\pgfsetlinewidth{0.803000pt}%
\definecolor{currentstroke}{rgb}{1.000000,1.000000,1.000000}%
\pgfsetstrokecolor{currentstroke}%
\pgfsetdash{}{0pt}%
\pgfpathmoveto{\pgfqpoint{0.275000in}{1.139050in}}%
\pgfpathlineto{\pgfqpoint{1.980000in}{1.139050in}}%
\pgfusepath{stroke}%
\end{pgfscope}%
\begin{pgfscope}%
\definecolor{textcolor}{rgb}{0.150000,0.150000,0.150000}%
\pgfsetstrokecolor{textcolor}%
\pgfsetfillcolor{textcolor}%
\pgftext[x=0.075538in,y=1.096841in,left,base]{\color{textcolor}\rmfamily\fontsize{8.000000}{9.600000}\selectfont \(\displaystyle -1\)}%
\end{pgfscope}%
\begin{pgfscope}%
\pgfpathrectangle{\pgfqpoint{0.275000in}{0.375000in}}{\pgfqpoint{1.705000in}{2.265000in}}%
\pgfusepath{clip}%
\pgfsetroundcap%
\pgfsetroundjoin%
\pgfsetlinewidth{0.803000pt}%
\definecolor{currentstroke}{rgb}{1.000000,1.000000,1.000000}%
\pgfsetstrokecolor{currentstroke}%
\pgfsetdash{}{0pt}%
\pgfpathmoveto{\pgfqpoint{0.275000in}{1.466536in}}%
\pgfpathlineto{\pgfqpoint{1.980000in}{1.466536in}}%
\pgfusepath{stroke}%
\end{pgfscope}%
\begin{pgfscope}%
\definecolor{textcolor}{rgb}{0.150000,0.150000,0.150000}%
\pgfsetstrokecolor{textcolor}%
\pgfsetfillcolor{textcolor}%
\pgftext[x=0.167360in,y=1.424327in,left,base]{\color{textcolor}\rmfamily\fontsize{8.000000}{9.600000}\selectfont \(\displaystyle 0\)}%
\end{pgfscope}%
\begin{pgfscope}%
\pgfpathrectangle{\pgfqpoint{0.275000in}{0.375000in}}{\pgfqpoint{1.705000in}{2.265000in}}%
\pgfusepath{clip}%
\pgfsetroundcap%
\pgfsetroundjoin%
\pgfsetlinewidth{0.803000pt}%
\definecolor{currentstroke}{rgb}{1.000000,1.000000,1.000000}%
\pgfsetstrokecolor{currentstroke}%
\pgfsetdash{}{0pt}%
\pgfpathmoveto{\pgfqpoint{0.275000in}{1.794022in}}%
\pgfpathlineto{\pgfqpoint{1.980000in}{1.794022in}}%
\pgfusepath{stroke}%
\end{pgfscope}%
\begin{pgfscope}%
\definecolor{textcolor}{rgb}{0.150000,0.150000,0.150000}%
\pgfsetstrokecolor{textcolor}%
\pgfsetfillcolor{textcolor}%
\pgftext[x=0.167360in,y=1.751813in,left,base]{\color{textcolor}\rmfamily\fontsize{8.000000}{9.600000}\selectfont \(\displaystyle 1\)}%
\end{pgfscope}%
\begin{pgfscope}%
\pgfpathrectangle{\pgfqpoint{0.275000in}{0.375000in}}{\pgfqpoint{1.705000in}{2.265000in}}%
\pgfusepath{clip}%
\pgfsetroundcap%
\pgfsetroundjoin%
\pgfsetlinewidth{0.803000pt}%
\definecolor{currentstroke}{rgb}{1.000000,1.000000,1.000000}%
\pgfsetstrokecolor{currentstroke}%
\pgfsetdash{}{0pt}%
\pgfpathmoveto{\pgfqpoint{0.275000in}{2.121508in}}%
\pgfpathlineto{\pgfqpoint{1.980000in}{2.121508in}}%
\pgfusepath{stroke}%
\end{pgfscope}%
\begin{pgfscope}%
\definecolor{textcolor}{rgb}{0.150000,0.150000,0.150000}%
\pgfsetstrokecolor{textcolor}%
\pgfsetfillcolor{textcolor}%
\pgftext[x=0.167360in,y=2.079299in,left,base]{\color{textcolor}\rmfamily\fontsize{8.000000}{9.600000}\selectfont \(\displaystyle 2\)}%
\end{pgfscope}%
\begin{pgfscope}%
\pgfpathrectangle{\pgfqpoint{0.275000in}{0.375000in}}{\pgfqpoint{1.705000in}{2.265000in}}%
\pgfusepath{clip}%
\pgfsetroundcap%
\pgfsetroundjoin%
\pgfsetlinewidth{0.803000pt}%
\definecolor{currentstroke}{rgb}{1.000000,1.000000,1.000000}%
\pgfsetstrokecolor{currentstroke}%
\pgfsetdash{}{0pt}%
\pgfpathmoveto{\pgfqpoint{0.275000in}{2.448994in}}%
\pgfpathlineto{\pgfqpoint{1.980000in}{2.448994in}}%
\pgfusepath{stroke}%
\end{pgfscope}%
\begin{pgfscope}%
\definecolor{textcolor}{rgb}{0.150000,0.150000,0.150000}%
\pgfsetstrokecolor{textcolor}%
\pgfsetfillcolor{textcolor}%
\pgftext[x=0.167360in,y=2.406785in,left,base]{\color{textcolor}\rmfamily\fontsize{8.000000}{9.600000}\selectfont \(\displaystyle 3\)}%
\end{pgfscope}%
\begin{pgfscope}%
\pgfpathrectangle{\pgfqpoint{0.275000in}{0.375000in}}{\pgfqpoint{1.705000in}{2.265000in}}%
\pgfusepath{clip}%
\pgfsetroundcap%
\pgfsetroundjoin%
\pgfsetlinewidth{1.505625pt}%
\definecolor{currentstroke}{rgb}{0.121569,0.466667,0.705882}%
\pgfsetstrokecolor{currentstroke}%
\pgfsetdash{}{0pt}%
\pgfpathmoveto{\pgfqpoint{0.352500in}{1.264999in}}%
\pgfpathlineto{\pgfqpoint{0.355606in}{1.280051in}}%
\pgfpathlineto{\pgfqpoint{0.358712in}{1.285829in}}%
\pgfpathlineto{\pgfqpoint{0.361819in}{1.282305in}}%
\pgfpathlineto{\pgfqpoint{0.364925in}{1.270077in}}%
\pgfpathlineto{\pgfqpoint{0.371137in}{1.224608in}}%
\pgfpathlineto{\pgfqpoint{0.389775in}{1.053616in}}%
\pgfpathlineto{\pgfqpoint{0.395987in}{1.024548in}}%
\pgfpathlineto{\pgfqpoint{0.399093in}{1.017253in}}%
\pgfpathlineto{\pgfqpoint{0.402199in}{1.014082in}}%
\pgfpathlineto{\pgfqpoint{0.405306in}{1.014269in}}%
\pgfpathlineto{\pgfqpoint{0.411518in}{1.021655in}}%
\pgfpathlineto{\pgfqpoint{0.417730in}{1.034907in}}%
\pgfpathlineto{\pgfqpoint{0.423943in}{1.053520in}}%
\pgfpathlineto{\pgfqpoint{0.430155in}{1.080807in}}%
\pgfpathlineto{\pgfqpoint{0.436368in}{1.121166in}}%
\pgfpathlineto{\pgfqpoint{0.445686in}{1.208171in}}%
\pgfpathlineto{\pgfqpoint{0.458111in}{1.335268in}}%
\pgfpathlineto{\pgfqpoint{0.464324in}{1.373413in}}%
\pgfpathlineto{\pgfqpoint{0.467430in}{1.380481in}}%
\pgfpathlineto{\pgfqpoint{0.470536in}{1.378372in}}%
\pgfpathlineto{\pgfqpoint{0.473642in}{1.366941in}}%
\pgfpathlineto{\pgfqpoint{0.479855in}{1.318739in}}%
\pgfpathlineto{\pgfqpoint{0.501598in}{1.084668in}}%
\pgfpathlineto{\pgfqpoint{0.504704in}{1.072465in}}%
\pgfpathlineto{\pgfqpoint{0.507811in}{1.070128in}}%
\pgfpathlineto{\pgfqpoint{0.510917in}{1.077905in}}%
\pgfpathlineto{\pgfqpoint{0.514023in}{1.095464in}}%
\pgfpathlineto{\pgfqpoint{0.520235in}{1.155990in}}%
\pgfpathlineto{\pgfqpoint{0.545085in}{1.472258in}}%
\pgfpathlineto{\pgfqpoint{0.551298in}{1.501975in}}%
\pgfpathlineto{\pgfqpoint{0.554404in}{1.504786in}}%
\pgfpathlineto{\pgfqpoint{0.557510in}{1.499771in}}%
\pgfpathlineto{\pgfqpoint{0.560616in}{1.487395in}}%
\pgfpathlineto{\pgfqpoint{0.566829in}{1.443212in}}%
\pgfpathlineto{\pgfqpoint{0.576147in}{1.340642in}}%
\pgfpathlineto{\pgfqpoint{0.597891in}{1.067509in}}%
\pgfpathlineto{\pgfqpoint{0.604103in}{1.018223in}}%
\pgfpathlineto{\pgfqpoint{0.610316in}{0.993578in}}%
\pgfpathlineto{\pgfqpoint{0.613422in}{0.991125in}}%
\pgfpathlineto{\pgfqpoint{0.616528in}{0.995102in}}%
\pgfpathlineto{\pgfqpoint{0.619634in}{1.005112in}}%
\pgfpathlineto{\pgfqpoint{0.625847in}{1.040761in}}%
\pgfpathlineto{\pgfqpoint{0.635165in}{1.121051in}}%
\pgfpathlineto{\pgfqpoint{0.653803in}{1.290041in}}%
\pgfpathlineto{\pgfqpoint{0.660015in}{1.326732in}}%
\pgfpathlineto{\pgfqpoint{0.666227in}{1.348398in}}%
\pgfpathlineto{\pgfqpoint{0.672440in}{1.356794in}}%
\pgfpathlineto{\pgfqpoint{0.678652in}{1.357541in}}%
\pgfpathlineto{\pgfqpoint{0.684865in}{1.358612in}}%
\pgfpathlineto{\pgfqpoint{0.687971in}{1.361623in}}%
\pgfpathlineto{\pgfqpoint{0.694183in}{1.375176in}}%
\pgfpathlineto{\pgfqpoint{0.703502in}{1.410557in}}%
\pgfpathlineto{\pgfqpoint{0.712821in}{1.441436in}}%
\pgfpathlineto{\pgfqpoint{0.715927in}{1.445702in}}%
\pgfpathlineto{\pgfqpoint{0.719033in}{1.445920in}}%
\pgfpathlineto{\pgfqpoint{0.722139in}{1.442080in}}%
\pgfpathlineto{\pgfqpoint{0.728352in}{1.424202in}}%
\pgfpathlineto{\pgfqpoint{0.746989in}{1.355750in}}%
\pgfpathlineto{\pgfqpoint{0.756308in}{1.339811in}}%
\pgfpathlineto{\pgfqpoint{0.762520in}{1.329944in}}%
\pgfpathlineto{\pgfqpoint{0.768732in}{1.313496in}}%
\pgfpathlineto{\pgfqpoint{0.774945in}{1.286627in}}%
\pgfpathlineto{\pgfqpoint{0.784264in}{1.227812in}}%
\pgfpathlineto{\pgfqpoint{0.802901in}{1.098423in}}%
\pgfpathlineto{\pgfqpoint{0.812219in}{1.055729in}}%
\pgfpathlineto{\pgfqpoint{0.818432in}{1.040045in}}%
\pgfpathlineto{\pgfqpoint{0.821538in}{1.036651in}}%
\pgfpathlineto{\pgfqpoint{0.824644in}{1.036663in}}%
\pgfpathlineto{\pgfqpoint{0.827751in}{1.040541in}}%
\pgfpathlineto{\pgfqpoint{0.830857in}{1.048836in}}%
\pgfpathlineto{\pgfqpoint{0.837069in}{1.081134in}}%
\pgfpathlineto{\pgfqpoint{0.843282in}{1.138186in}}%
\pgfpathlineto{\pgfqpoint{0.849494in}{1.222330in}}%
\pgfpathlineto{\pgfqpoint{0.861919in}{1.454178in}}%
\pgfpathlineto{\pgfqpoint{0.874344in}{1.676592in}}%
\pgfpathlineto{\pgfqpoint{0.880556in}{1.738719in}}%
\pgfpathlineto{\pgfqpoint{0.883662in}{1.752058in}}%
\pgfpathlineto{\pgfqpoint{0.886769in}{1.753260in}}%
\pgfpathlineto{\pgfqpoint{0.889875in}{1.743072in}}%
\pgfpathlineto{\pgfqpoint{0.896087in}{1.695088in}}%
\pgfpathlineto{\pgfqpoint{0.914724in}{1.506170in}}%
\pgfpathlineto{\pgfqpoint{0.920937in}{1.481079in}}%
\pgfpathlineto{\pgfqpoint{0.924043in}{1.478290in}}%
\pgfpathlineto{\pgfqpoint{0.927149in}{1.480880in}}%
\pgfpathlineto{\pgfqpoint{0.933362in}{1.496612in}}%
\pgfpathlineto{\pgfqpoint{0.942680in}{1.522376in}}%
\pgfpathlineto{\pgfqpoint{0.945787in}{1.525525in}}%
\pgfpathlineto{\pgfqpoint{0.948893in}{1.524037in}}%
\pgfpathlineto{\pgfqpoint{0.951999in}{1.517199in}}%
\pgfpathlineto{\pgfqpoint{0.958211in}{1.486248in}}%
\pgfpathlineto{\pgfqpoint{0.964424in}{1.433508in}}%
\pgfpathlineto{\pgfqpoint{0.976849in}{1.288826in}}%
\pgfpathlineto{\pgfqpoint{0.986167in}{1.186399in}}%
\pgfpathlineto{\pgfqpoint{0.992380in}{1.140889in}}%
\pgfpathlineto{\pgfqpoint{0.995486in}{1.128095in}}%
\pgfpathlineto{\pgfqpoint{0.998592in}{1.122873in}}%
\pgfpathlineto{\pgfqpoint{1.001698in}{1.125624in}}%
\pgfpathlineto{\pgfqpoint{1.004805in}{1.136480in}}%
\pgfpathlineto{\pgfqpoint{1.011017in}{1.181555in}}%
\pgfpathlineto{\pgfqpoint{1.020336in}{1.295382in}}%
\pgfpathlineto{\pgfqpoint{1.035867in}{1.499777in}}%
\pgfpathlineto{\pgfqpoint{1.042079in}{1.540424in}}%
\pgfpathlineto{\pgfqpoint{1.045185in}{1.546045in}}%
\pgfpathlineto{\pgfqpoint{1.048292in}{1.541324in}}%
\pgfpathlineto{\pgfqpoint{1.051398in}{1.526612in}}%
\pgfpathlineto{\pgfqpoint{1.057610in}{1.471412in}}%
\pgfpathlineto{\pgfqpoint{1.070035in}{1.306757in}}%
\pgfpathlineto{\pgfqpoint{1.082460in}{1.160621in}}%
\pgfpathlineto{\pgfqpoint{1.091779in}{1.093008in}}%
\pgfpathlineto{\pgfqpoint{1.097991in}{1.067689in}}%
\pgfpathlineto{\pgfqpoint{1.101097in}{1.061091in}}%
\pgfpathlineto{\pgfqpoint{1.104203in}{1.059096in}}%
\pgfpathlineto{\pgfqpoint{1.107310in}{1.062419in}}%
\pgfpathlineto{\pgfqpoint{1.110416in}{1.071943in}}%
\pgfpathlineto{\pgfqpoint{1.116628in}{1.113342in}}%
\pgfpathlineto{\pgfqpoint{1.122841in}{1.189470in}}%
\pgfpathlineto{\pgfqpoint{1.132159in}{1.370463in}}%
\pgfpathlineto{\pgfqpoint{1.150797in}{1.858591in}}%
\pgfpathlineto{\pgfqpoint{1.160115in}{2.068365in}}%
\pgfpathlineto{\pgfqpoint{1.166328in}{2.168410in}}%
\pgfpathlineto{\pgfqpoint{1.172540in}{2.230552in}}%
\pgfpathlineto{\pgfqpoint{1.175646in}{2.247175in}}%
\pgfpathlineto{\pgfqpoint{1.178753in}{2.254571in}}%
\pgfpathlineto{\pgfqpoint{1.181859in}{2.253357in}}%
\pgfpathlineto{\pgfqpoint{1.184965in}{2.244397in}}%
\pgfpathlineto{\pgfqpoint{1.191177in}{2.207774in}}%
\pgfpathlineto{\pgfqpoint{1.216027in}{2.015104in}}%
\pgfpathlineto{\pgfqpoint{1.222239in}{1.993870in}}%
\pgfpathlineto{\pgfqpoint{1.237771in}{1.959880in}}%
\pgfpathlineto{\pgfqpoint{1.243983in}{1.932366in}}%
\pgfpathlineto{\pgfqpoint{1.253302in}{1.867672in}}%
\pgfpathlineto{\pgfqpoint{1.268833in}{1.750744in}}%
\pgfpathlineto{\pgfqpoint{1.278151in}{1.708007in}}%
\pgfpathlineto{\pgfqpoint{1.287470in}{1.673957in}}%
\pgfpathlineto{\pgfqpoint{1.293682in}{1.639087in}}%
\pgfpathlineto{\pgfqpoint{1.299895in}{1.583961in}}%
\pgfpathlineto{\pgfqpoint{1.309213in}{1.459200in}}%
\pgfpathlineto{\pgfqpoint{1.330957in}{1.118553in}}%
\pgfpathlineto{\pgfqpoint{1.337169in}{1.058375in}}%
\pgfpathlineto{\pgfqpoint{1.343382in}{1.025250in}}%
\pgfpathlineto{\pgfqpoint{1.346488in}{1.018377in}}%
\pgfpathlineto{\pgfqpoint{1.349594in}{1.017284in}}%
\pgfpathlineto{\pgfqpoint{1.352700in}{1.021311in}}%
\pgfpathlineto{\pgfqpoint{1.358913in}{1.042021in}}%
\pgfpathlineto{\pgfqpoint{1.368231in}{1.096160in}}%
\pgfpathlineto{\pgfqpoint{1.393081in}{1.275628in}}%
\pgfpathlineto{\pgfqpoint{1.396187in}{1.284858in}}%
\pgfpathlineto{\pgfqpoint{1.399294in}{1.286823in}}%
\pgfpathlineto{\pgfqpoint{1.402400in}{1.280655in}}%
\pgfpathlineto{\pgfqpoint{1.405506in}{1.266020in}}%
\pgfpathlineto{\pgfqpoint{1.411718in}{1.213335in}}%
\pgfpathlineto{\pgfqpoint{1.430356in}{1.003327in}}%
\pgfpathlineto{\pgfqpoint{1.436568in}{0.974026in}}%
\pgfpathlineto{\pgfqpoint{1.439674in}{0.971612in}}%
\pgfpathlineto{\pgfqpoint{1.442781in}{0.976323in}}%
\pgfpathlineto{\pgfqpoint{1.448993in}{1.000630in}}%
\pgfpathlineto{\pgfqpoint{1.461418in}{1.056816in}}%
\pgfpathlineto{\pgfqpoint{1.467630in}{1.069376in}}%
\pgfpathlineto{\pgfqpoint{1.470736in}{1.071389in}}%
\pgfpathlineto{\pgfqpoint{1.476949in}{1.070817in}}%
\pgfpathlineto{\pgfqpoint{1.483161in}{1.070239in}}%
\pgfpathlineto{\pgfqpoint{1.489374in}{1.074244in}}%
\pgfpathlineto{\pgfqpoint{1.504905in}{1.092294in}}%
\pgfpathlineto{\pgfqpoint{1.511117in}{1.093030in}}%
\pgfpathlineto{\pgfqpoint{1.517330in}{1.091973in}}%
\pgfpathlineto{\pgfqpoint{1.520436in}{1.093287in}}%
\pgfpathlineto{\pgfqpoint{1.523542in}{1.097527in}}%
\pgfpathlineto{\pgfqpoint{1.526648in}{1.105917in}}%
\pgfpathlineto{\pgfqpoint{1.532861in}{1.138798in}}%
\pgfpathlineto{\pgfqpoint{1.539073in}{1.195473in}}%
\pgfpathlineto{\pgfqpoint{1.548392in}{1.318106in}}%
\pgfpathlineto{\pgfqpoint{1.576348in}{1.732164in}}%
\pgfpathlineto{\pgfqpoint{1.582560in}{1.789447in}}%
\pgfpathlineto{\pgfqpoint{1.588773in}{1.817912in}}%
\pgfpathlineto{\pgfqpoint{1.591879in}{1.819539in}}%
\pgfpathlineto{\pgfqpoint{1.594985in}{1.812380in}}%
\pgfpathlineto{\pgfqpoint{1.598091in}{1.796626in}}%
\pgfpathlineto{\pgfqpoint{1.604304in}{1.742149in}}%
\pgfpathlineto{\pgfqpoint{1.616728in}{1.580277in}}%
\pgfpathlineto{\pgfqpoint{1.626047in}{1.468448in}}%
\pgfpathlineto{\pgfqpoint{1.632260in}{1.420891in}}%
\pgfpathlineto{\pgfqpoint{1.638472in}{1.400481in}}%
\pgfpathlineto{\pgfqpoint{1.641578in}{1.399925in}}%
\pgfpathlineto{\pgfqpoint{1.644684in}{1.404859in}}%
\pgfpathlineto{\pgfqpoint{1.650897in}{1.427263in}}%
\pgfpathlineto{\pgfqpoint{1.663322in}{1.496225in}}%
\pgfpathlineto{\pgfqpoint{1.685065in}{1.621609in}}%
\pgfpathlineto{\pgfqpoint{1.691278in}{1.645183in}}%
\pgfpathlineto{\pgfqpoint{1.694384in}{1.651147in}}%
\pgfpathlineto{\pgfqpoint{1.697490in}{1.652053in}}%
\pgfpathlineto{\pgfqpoint{1.700596in}{1.647286in}}%
\pgfpathlineto{\pgfqpoint{1.703702in}{1.636599in}}%
\pgfpathlineto{\pgfqpoint{1.709915in}{1.598761in}}%
\pgfpathlineto{\pgfqpoint{1.731658in}{1.428419in}}%
\pgfpathlineto{\pgfqpoint{1.737871in}{1.407838in}}%
\pgfpathlineto{\pgfqpoint{1.750296in}{1.380226in}}%
\pgfpathlineto{\pgfqpoint{1.756508in}{1.347186in}}%
\pgfpathlineto{\pgfqpoint{1.762720in}{1.292316in}}%
\pgfpathlineto{\pgfqpoint{1.784464in}{1.066078in}}%
\pgfpathlineto{\pgfqpoint{1.787570in}{1.053909in}}%
\pgfpathlineto{\pgfqpoint{1.790676in}{1.050896in}}%
\pgfpathlineto{\pgfqpoint{1.793783in}{1.057459in}}%
\pgfpathlineto{\pgfqpoint{1.796889in}{1.073572in}}%
\pgfpathlineto{\pgfqpoint{1.803101in}{1.132174in}}%
\pgfpathlineto{\pgfqpoint{1.812420in}{1.266084in}}%
\pgfpathlineto{\pgfqpoint{1.824845in}{1.437024in}}%
\pgfpathlineto{\pgfqpoint{1.831057in}{1.473912in}}%
\pgfpathlineto{\pgfqpoint{1.834163in}{1.474151in}}%
\pgfpathlineto{\pgfqpoint{1.837270in}{1.461878in}}%
\pgfpathlineto{\pgfqpoint{1.843482in}{1.404130in}}%
\pgfpathlineto{\pgfqpoint{1.862119in}{1.157918in}}%
\pgfpathlineto{\pgfqpoint{1.865225in}{1.137703in}}%
\pgfpathlineto{\pgfqpoint{1.868332in}{1.128770in}}%
\pgfpathlineto{\pgfqpoint{1.871438in}{1.131444in}}%
\pgfpathlineto{\pgfqpoint{1.874544in}{1.145269in}}%
\pgfpathlineto{\pgfqpoint{1.880757in}{1.201283in}}%
\pgfpathlineto{\pgfqpoint{1.893181in}{1.371438in}}%
\pgfpathlineto{\pgfqpoint{1.902500in}{1.488725in}}%
\pgfpathlineto{\pgfqpoint{1.902500in}{1.488725in}}%
\pgfusepath{stroke}%
\end{pgfscope}%
\begin{pgfscope}%
\pgfpathrectangle{\pgfqpoint{0.275000in}{0.375000in}}{\pgfqpoint{1.705000in}{2.265000in}}%
\pgfusepath{clip}%
\pgfsetroundcap%
\pgfsetroundjoin%
\pgfsetlinewidth{1.505625pt}%
\definecolor{currentstroke}{rgb}{1.000000,0.498039,0.054902}%
\pgfsetstrokecolor{currentstroke}%
\pgfsetdash{}{0pt}%
\pgfpathmoveto{\pgfqpoint{0.352500in}{1.083737in}}%
\pgfpathlineto{\pgfqpoint{0.361819in}{1.204694in}}%
\pgfpathlineto{\pgfqpoint{0.377350in}{1.417972in}}%
\pgfpathlineto{\pgfqpoint{0.383562in}{1.462644in}}%
\pgfpathlineto{\pgfqpoint{0.386668in}{1.469950in}}%
\pgfpathlineto{\pgfqpoint{0.389775in}{1.466737in}}%
\pgfpathlineto{\pgfqpoint{0.392881in}{1.453520in}}%
\pgfpathlineto{\pgfqpoint{0.399093in}{1.402331in}}%
\pgfpathlineto{\pgfqpoint{0.420837in}{1.171705in}}%
\pgfpathlineto{\pgfqpoint{0.427049in}{1.140834in}}%
\pgfpathlineto{\pgfqpoint{0.430155in}{1.133788in}}%
\pgfpathlineto{\pgfqpoint{0.433262in}{1.132076in}}%
\pgfpathlineto{\pgfqpoint{0.436368in}{1.135391in}}%
\pgfpathlineto{\pgfqpoint{0.439474in}{1.143374in}}%
\pgfpathlineto{\pgfqpoint{0.445686in}{1.171668in}}%
\pgfpathlineto{\pgfqpoint{0.455005in}{1.236413in}}%
\pgfpathlineto{\pgfqpoint{0.470536in}{1.346902in}}%
\pgfpathlineto{\pgfqpoint{0.476748in}{1.370134in}}%
\pgfpathlineto{\pgfqpoint{0.479855in}{1.374611in}}%
\pgfpathlineto{\pgfqpoint{0.482961in}{1.373982in}}%
\pgfpathlineto{\pgfqpoint{0.486067in}{1.368200in}}%
\pgfpathlineto{\pgfqpoint{0.492280in}{1.341656in}}%
\pgfpathlineto{\pgfqpoint{0.498492in}{1.297038in}}%
\pgfpathlineto{\pgfqpoint{0.510917in}{1.172036in}}%
\pgfpathlineto{\pgfqpoint{0.520235in}{1.082082in}}%
\pgfpathlineto{\pgfqpoint{0.526448in}{1.048460in}}%
\pgfpathlineto{\pgfqpoint{0.529554in}{1.044561in}}%
\pgfpathlineto{\pgfqpoint{0.532660in}{1.050822in}}%
\pgfpathlineto{\pgfqpoint{0.535767in}{1.067759in}}%
\pgfpathlineto{\pgfqpoint{0.541979in}{1.132517in}}%
\pgfpathlineto{\pgfqpoint{0.551298in}{1.286399in}}%
\pgfpathlineto{\pgfqpoint{0.563722in}{1.504129in}}%
\pgfpathlineto{\pgfqpoint{0.569935in}{1.581786in}}%
\pgfpathlineto{\pgfqpoint{0.576147in}{1.628752in}}%
\pgfpathlineto{\pgfqpoint{0.579254in}{1.641015in}}%
\pgfpathlineto{\pgfqpoint{0.582360in}{1.646612in}}%
\pgfpathlineto{\pgfqpoint{0.585466in}{1.646384in}}%
\pgfpathlineto{\pgfqpoint{0.588572in}{1.641244in}}%
\pgfpathlineto{\pgfqpoint{0.594785in}{1.619814in}}%
\pgfpathlineto{\pgfqpoint{0.607209in}{1.553822in}}%
\pgfpathlineto{\pgfqpoint{0.619634in}{1.488171in}}%
\pgfpathlineto{\pgfqpoint{0.632059in}{1.439955in}}%
\pgfpathlineto{\pgfqpoint{0.641378in}{1.400232in}}%
\pgfpathlineto{\pgfqpoint{0.647590in}{1.358564in}}%
\pgfpathlineto{\pgfqpoint{0.653803in}{1.298260in}}%
\pgfpathlineto{\pgfqpoint{0.663121in}{1.175847in}}%
\pgfpathlineto{\pgfqpoint{0.678652in}{0.971540in}}%
\pgfpathlineto{\pgfqpoint{0.684865in}{0.927825in}}%
\pgfpathlineto{\pgfqpoint{0.687971in}{0.918642in}}%
\pgfpathlineto{\pgfqpoint{0.691077in}{0.918284in}}%
\pgfpathlineto{\pgfqpoint{0.694183in}{0.926509in}}%
\pgfpathlineto{\pgfqpoint{0.700396in}{0.965920in}}%
\pgfpathlineto{\pgfqpoint{0.709714in}{1.065106in}}%
\pgfpathlineto{\pgfqpoint{0.725245in}{1.245911in}}%
\pgfpathlineto{\pgfqpoint{0.734564in}{1.318858in}}%
\pgfpathlineto{\pgfqpoint{0.740777in}{1.348135in}}%
\pgfpathlineto{\pgfqpoint{0.746989in}{1.366179in}}%
\pgfpathlineto{\pgfqpoint{0.759414in}{1.395872in}}%
\pgfpathlineto{\pgfqpoint{0.765626in}{1.421997in}}%
\pgfpathlineto{\pgfqpoint{0.771839in}{1.462422in}}%
\pgfpathlineto{\pgfqpoint{0.781157in}{1.548860in}}%
\pgfpathlineto{\pgfqpoint{0.793582in}{1.670266in}}%
\pgfpathlineto{\pgfqpoint{0.799795in}{1.704349in}}%
\pgfpathlineto{\pgfqpoint{0.802901in}{1.709558in}}%
\pgfpathlineto{\pgfqpoint{0.806007in}{1.705940in}}%
\pgfpathlineto{\pgfqpoint{0.809113in}{1.693527in}}%
\pgfpathlineto{\pgfqpoint{0.815326in}{1.645493in}}%
\pgfpathlineto{\pgfqpoint{0.840175in}{1.397739in}}%
\pgfpathlineto{\pgfqpoint{0.846388in}{1.376826in}}%
\pgfpathlineto{\pgfqpoint{0.849494in}{1.373698in}}%
\pgfpathlineto{\pgfqpoint{0.852600in}{1.374556in}}%
\pgfpathlineto{\pgfqpoint{0.855706in}{1.378688in}}%
\pgfpathlineto{\pgfqpoint{0.861919in}{1.394465in}}%
\pgfpathlineto{\pgfqpoint{0.871237in}{1.432301in}}%
\pgfpathlineto{\pgfqpoint{0.883662in}{1.502831in}}%
\pgfpathlineto{\pgfqpoint{0.892981in}{1.551929in}}%
\pgfpathlineto{\pgfqpoint{0.896087in}{1.561374in}}%
\pgfpathlineto{\pgfqpoint{0.899193in}{1.564973in}}%
\pgfpathlineto{\pgfqpoint{0.902300in}{1.561478in}}%
\pgfpathlineto{\pgfqpoint{0.905406in}{1.549977in}}%
\pgfpathlineto{\pgfqpoint{0.911618in}{1.501801in}}%
\pgfpathlineto{\pgfqpoint{0.920937in}{1.378149in}}%
\pgfpathlineto{\pgfqpoint{0.933362in}{1.202205in}}%
\pgfpathlineto{\pgfqpoint{0.939574in}{1.151804in}}%
\pgfpathlineto{\pgfqpoint{0.942680in}{1.141102in}}%
\pgfpathlineto{\pgfqpoint{0.945787in}{1.140207in}}%
\pgfpathlineto{\pgfqpoint{0.948893in}{1.148331in}}%
\pgfpathlineto{\pgfqpoint{0.955105in}{1.185642in}}%
\pgfpathlineto{\pgfqpoint{0.973742in}{1.325855in}}%
\pgfpathlineto{\pgfqpoint{0.979955in}{1.343824in}}%
\pgfpathlineto{\pgfqpoint{0.983061in}{1.345418in}}%
\pgfpathlineto{\pgfqpoint{0.986167in}{1.342560in}}%
\pgfpathlineto{\pgfqpoint{0.992380in}{1.325808in}}%
\pgfpathlineto{\pgfqpoint{1.001698in}{1.282641in}}%
\pgfpathlineto{\pgfqpoint{1.023442in}{1.170434in}}%
\pgfpathlineto{\pgfqpoint{1.029654in}{1.150745in}}%
\pgfpathlineto{\pgfqpoint{1.032761in}{1.145166in}}%
\pgfpathlineto{\pgfqpoint{1.035867in}{1.143030in}}%
\pgfpathlineto{\pgfqpoint{1.038973in}{1.144726in}}%
\pgfpathlineto{\pgfqpoint{1.042079in}{1.150550in}}%
\pgfpathlineto{\pgfqpoint{1.048292in}{1.175143in}}%
\pgfpathlineto{\pgfqpoint{1.054504in}{1.216496in}}%
\pgfpathlineto{\pgfqpoint{1.063823in}{1.304494in}}%
\pgfpathlineto{\pgfqpoint{1.079354in}{1.492105in}}%
\pgfpathlineto{\pgfqpoint{1.094885in}{1.673509in}}%
\pgfpathlineto{\pgfqpoint{1.101097in}{1.717688in}}%
\pgfpathlineto{\pgfqpoint{1.104203in}{1.728882in}}%
\pgfpathlineto{\pgfqpoint{1.107310in}{1.731713in}}%
\pgfpathlineto{\pgfqpoint{1.110416in}{1.725875in}}%
\pgfpathlineto{\pgfqpoint{1.113522in}{1.711582in}}%
\pgfpathlineto{\pgfqpoint{1.119734in}{1.661235in}}%
\pgfpathlineto{\pgfqpoint{1.138372in}{1.472840in}}%
\pgfpathlineto{\pgfqpoint{1.141478in}{1.458674in}}%
\pgfpathlineto{\pgfqpoint{1.144584in}{1.453680in}}%
\pgfpathlineto{\pgfqpoint{1.147690in}{1.458344in}}%
\pgfpathlineto{\pgfqpoint{1.150797in}{1.472541in}}%
\pgfpathlineto{\pgfqpoint{1.157009in}{1.526147in}}%
\pgfpathlineto{\pgfqpoint{1.181859in}{1.812162in}}%
\pgfpathlineto{\pgfqpoint{1.188071in}{1.835164in}}%
\pgfpathlineto{\pgfqpoint{1.191177in}{1.835973in}}%
\pgfpathlineto{\pgfqpoint{1.194284in}{1.830443in}}%
\pgfpathlineto{\pgfqpoint{1.200496in}{1.804576in}}%
\pgfpathlineto{\pgfqpoint{1.216027in}{1.721730in}}%
\pgfpathlineto{\pgfqpoint{1.219133in}{1.714704in}}%
\pgfpathlineto{\pgfqpoint{1.222239in}{1.713975in}}%
\pgfpathlineto{\pgfqpoint{1.225346in}{1.720249in}}%
\pgfpathlineto{\pgfqpoint{1.231558in}{1.754074in}}%
\pgfpathlineto{\pgfqpoint{1.240877in}{1.843943in}}%
\pgfpathlineto{\pgfqpoint{1.250195in}{1.932823in}}%
\pgfpathlineto{\pgfqpoint{1.253302in}{1.950777in}}%
\pgfpathlineto{\pgfqpoint{1.256408in}{1.959333in}}%
\pgfpathlineto{\pgfqpoint{1.259514in}{1.957064in}}%
\pgfpathlineto{\pgfqpoint{1.262620in}{1.943189in}}%
\pgfpathlineto{\pgfqpoint{1.268833in}{1.880984in}}%
\pgfpathlineto{\pgfqpoint{1.278151in}{1.719878in}}%
\pgfpathlineto{\pgfqpoint{1.293682in}{1.418953in}}%
\pgfpathlineto{\pgfqpoint{1.299895in}{1.337545in}}%
\pgfpathlineto{\pgfqpoint{1.306107in}{1.291997in}}%
\pgfpathlineto{\pgfqpoint{1.309213in}{1.282899in}}%
\pgfpathlineto{\pgfqpoint{1.312320in}{1.282453in}}%
\pgfpathlineto{\pgfqpoint{1.315426in}{1.290121in}}%
\pgfpathlineto{\pgfqpoint{1.321638in}{1.327465in}}%
\pgfpathlineto{\pgfqpoint{1.327851in}{1.390625in}}%
\pgfpathlineto{\pgfqpoint{1.337169in}{1.524966in}}%
\pgfpathlineto{\pgfqpoint{1.358913in}{1.874690in}}%
\pgfpathlineto{\pgfqpoint{1.365125in}{1.921004in}}%
\pgfpathlineto{\pgfqpoint{1.368231in}{1.927542in}}%
\pgfpathlineto{\pgfqpoint{1.371338in}{1.922676in}}%
\pgfpathlineto{\pgfqpoint{1.374444in}{1.907060in}}%
\pgfpathlineto{\pgfqpoint{1.380656in}{1.849615in}}%
\pgfpathlineto{\pgfqpoint{1.399294in}{1.639787in}}%
\pgfpathlineto{\pgfqpoint{1.405506in}{1.611214in}}%
\pgfpathlineto{\pgfqpoint{1.408612in}{1.607171in}}%
\pgfpathlineto{\pgfqpoint{1.411718in}{1.608465in}}%
\pgfpathlineto{\pgfqpoint{1.424143in}{1.629157in}}%
\pgfpathlineto{\pgfqpoint{1.427249in}{1.628292in}}%
\pgfpathlineto{\pgfqpoint{1.430356in}{1.621834in}}%
\pgfpathlineto{\pgfqpoint{1.433462in}{1.608948in}}%
\pgfpathlineto{\pgfqpoint{1.439674in}{1.563229in}}%
\pgfpathlineto{\pgfqpoint{1.448993in}{1.455557in}}%
\pgfpathlineto{\pgfqpoint{1.461418in}{1.305528in}}%
\pgfpathlineto{\pgfqpoint{1.467630in}{1.261516in}}%
\pgfpathlineto{\pgfqpoint{1.470736in}{1.252539in}}%
\pgfpathlineto{\pgfqpoint{1.473843in}{1.253327in}}%
\pgfpathlineto{\pgfqpoint{1.476949in}{1.264115in}}%
\pgfpathlineto{\pgfqpoint{1.483161in}{1.314197in}}%
\pgfpathlineto{\pgfqpoint{1.492480in}{1.443427in}}%
\pgfpathlineto{\pgfqpoint{1.508011in}{1.678727in}}%
\pgfpathlineto{\pgfqpoint{1.514223in}{1.735313in}}%
\pgfpathlineto{\pgfqpoint{1.517330in}{1.750178in}}%
\pgfpathlineto{\pgfqpoint{1.520436in}{1.755256in}}%
\pgfpathlineto{\pgfqpoint{1.523542in}{1.750333in}}%
\pgfpathlineto{\pgfqpoint{1.526648in}{1.735510in}}%
\pgfpathlineto{\pgfqpoint{1.532861in}{1.677963in}}%
\pgfpathlineto{\pgfqpoint{1.542179in}{1.535208in}}%
\pgfpathlineto{\pgfqpoint{1.560817in}{1.203189in}}%
\pgfpathlineto{\pgfqpoint{1.567029in}{1.142338in}}%
\pgfpathlineto{\pgfqpoint{1.570135in}{1.131050in}}%
\pgfpathlineto{\pgfqpoint{1.573241in}{1.134393in}}%
\pgfpathlineto{\pgfqpoint{1.576348in}{1.152991in}}%
\pgfpathlineto{\pgfqpoint{1.582560in}{1.234698in}}%
\pgfpathlineto{\pgfqpoint{1.591879in}{1.443921in}}%
\pgfpathlineto{\pgfqpoint{1.607410in}{1.824019in}}%
\pgfpathlineto{\pgfqpoint{1.613622in}{1.915814in}}%
\pgfpathlineto{\pgfqpoint{1.616728in}{1.941477in}}%
\pgfpathlineto{\pgfqpoint{1.619835in}{1.953345in}}%
\pgfpathlineto{\pgfqpoint{1.622941in}{1.952129in}}%
\pgfpathlineto{\pgfqpoint{1.626047in}{1.939222in}}%
\pgfpathlineto{\pgfqpoint{1.632260in}{1.886628in}}%
\pgfpathlineto{\pgfqpoint{1.647791in}{1.720146in}}%
\pgfpathlineto{\pgfqpoint{1.654003in}{1.686425in}}%
\pgfpathlineto{\pgfqpoint{1.657109in}{1.681002in}}%
\pgfpathlineto{\pgfqpoint{1.660215in}{1.682889in}}%
\pgfpathlineto{\pgfqpoint{1.663322in}{1.691201in}}%
\pgfpathlineto{\pgfqpoint{1.669534in}{1.721828in}}%
\pgfpathlineto{\pgfqpoint{1.685065in}{1.812816in}}%
\pgfpathlineto{\pgfqpoint{1.691278in}{1.833397in}}%
\pgfpathlineto{\pgfqpoint{1.694384in}{1.838374in}}%
\pgfpathlineto{\pgfqpoint{1.697490in}{1.839518in}}%
\pgfpathlineto{\pgfqpoint{1.700596in}{1.836560in}}%
\pgfpathlineto{\pgfqpoint{1.703702in}{1.829127in}}%
\pgfpathlineto{\pgfqpoint{1.709915in}{1.798758in}}%
\pgfpathlineto{\pgfqpoint{1.716127in}{1.743678in}}%
\pgfpathlineto{\pgfqpoint{1.722340in}{1.660196in}}%
\pgfpathlineto{\pgfqpoint{1.731658in}{1.484904in}}%
\pgfpathlineto{\pgfqpoint{1.750296in}{1.100787in}}%
\pgfpathlineto{\pgfqpoint{1.756508in}{1.026087in}}%
\pgfpathlineto{\pgfqpoint{1.759614in}{1.006662in}}%
\pgfpathlineto{\pgfqpoint{1.762720in}{1.000174in}}%
\pgfpathlineto{\pgfqpoint{1.765827in}{1.006728in}}%
\pgfpathlineto{\pgfqpoint{1.768933in}{1.025954in}}%
\pgfpathlineto{\pgfqpoint{1.775145in}{1.098823in}}%
\pgfpathlineto{\pgfqpoint{1.784464in}{1.272586in}}%
\pgfpathlineto{\pgfqpoint{1.809314in}{1.789359in}}%
\pgfpathlineto{\pgfqpoint{1.818632in}{1.914153in}}%
\pgfpathlineto{\pgfqpoint{1.824845in}{1.968934in}}%
\pgfpathlineto{\pgfqpoint{1.831057in}{2.000251in}}%
\pgfpathlineto{\pgfqpoint{1.834163in}{2.006535in}}%
\pgfpathlineto{\pgfqpoint{1.837270in}{2.006138in}}%
\pgfpathlineto{\pgfqpoint{1.840376in}{1.998749in}}%
\pgfpathlineto{\pgfqpoint{1.846588in}{1.962336in}}%
\pgfpathlineto{\pgfqpoint{1.852801in}{1.898096in}}%
\pgfpathlineto{\pgfqpoint{1.862119in}{1.761490in}}%
\pgfpathlineto{\pgfqpoint{1.886969in}{1.370343in}}%
\pgfpathlineto{\pgfqpoint{1.899394in}{1.229956in}}%
\pgfpathlineto{\pgfqpoint{1.902500in}{1.202247in}}%
\pgfpathlineto{\pgfqpoint{1.902500in}{1.202247in}}%
\pgfusepath{stroke}%
\end{pgfscope}%
\begin{pgfscope}%
\pgfpathrectangle{\pgfqpoint{0.275000in}{0.375000in}}{\pgfqpoint{1.705000in}{2.265000in}}%
\pgfusepath{clip}%
\pgfsetroundcap%
\pgfsetroundjoin%
\pgfsetlinewidth{1.505625pt}%
\definecolor{currentstroke}{rgb}{0.172549,0.627451,0.172549}%
\pgfsetstrokecolor{currentstroke}%
\pgfsetdash{}{0pt}%
\pgfpathmoveto{\pgfqpoint{0.352500in}{1.098538in}}%
\pgfpathlineto{\pgfqpoint{0.358712in}{1.185835in}}%
\pgfpathlineto{\pgfqpoint{0.368031in}{1.376556in}}%
\pgfpathlineto{\pgfqpoint{0.395987in}{2.009225in}}%
\pgfpathlineto{\pgfqpoint{0.405306in}{2.140899in}}%
\pgfpathlineto{\pgfqpoint{0.411518in}{2.198732in}}%
\pgfpathlineto{\pgfqpoint{0.417730in}{2.234757in}}%
\pgfpathlineto{\pgfqpoint{0.420837in}{2.244749in}}%
\pgfpathlineto{\pgfqpoint{0.423943in}{2.249032in}}%
\pgfpathlineto{\pgfqpoint{0.427049in}{2.247015in}}%
\pgfpathlineto{\pgfqpoint{0.430155in}{2.237888in}}%
\pgfpathlineto{\pgfqpoint{0.433262in}{2.220718in}}%
\pgfpathlineto{\pgfqpoint{0.439474in}{2.158664in}}%
\pgfpathlineto{\pgfqpoint{0.445686in}{2.056019in}}%
\pgfpathlineto{\pgfqpoint{0.455005in}{1.832076in}}%
\pgfpathlineto{\pgfqpoint{0.476748in}{1.261810in}}%
\pgfpathlineto{\pgfqpoint{0.482961in}{1.173699in}}%
\pgfpathlineto{\pgfqpoint{0.489173in}{1.135524in}}%
\pgfpathlineto{\pgfqpoint{0.492280in}{1.133090in}}%
\pgfpathlineto{\pgfqpoint{0.495386in}{1.139450in}}%
\pgfpathlineto{\pgfqpoint{0.501598in}{1.169272in}}%
\pgfpathlineto{\pgfqpoint{0.510917in}{1.219592in}}%
\pgfpathlineto{\pgfqpoint{0.514023in}{1.229065in}}%
\pgfpathlineto{\pgfqpoint{0.517129in}{1.232258in}}%
\pgfpathlineto{\pgfqpoint{0.520235in}{1.228284in}}%
\pgfpathlineto{\pgfqpoint{0.523342in}{1.216736in}}%
\pgfpathlineto{\pgfqpoint{0.529554in}{1.171498in}}%
\pgfpathlineto{\pgfqpoint{0.538873in}{1.059635in}}%
\pgfpathlineto{\pgfqpoint{0.557510in}{0.814018in}}%
\pgfpathlineto{\pgfqpoint{0.563722in}{0.774337in}}%
\pgfpathlineto{\pgfqpoint{0.566829in}{0.769276in}}%
\pgfpathlineto{\pgfqpoint{0.569935in}{0.775391in}}%
\pgfpathlineto{\pgfqpoint{0.573041in}{0.793204in}}%
\pgfpathlineto{\pgfqpoint{0.579254in}{0.863807in}}%
\pgfpathlineto{\pgfqpoint{0.585466in}{0.976332in}}%
\pgfpathlineto{\pgfqpoint{0.597891in}{1.278451in}}%
\pgfpathlineto{\pgfqpoint{0.613422in}{1.642636in}}%
\pgfpathlineto{\pgfqpoint{0.622740in}{1.787148in}}%
\pgfpathlineto{\pgfqpoint{0.628953in}{1.846405in}}%
\pgfpathlineto{\pgfqpoint{0.635165in}{1.880003in}}%
\pgfpathlineto{\pgfqpoint{0.641378in}{1.894070in}}%
\pgfpathlineto{\pgfqpoint{0.644484in}{1.895921in}}%
\pgfpathlineto{\pgfqpoint{0.647590in}{1.895426in}}%
\pgfpathlineto{\pgfqpoint{0.653803in}{1.890327in}}%
\pgfpathlineto{\pgfqpoint{0.666227in}{1.877906in}}%
\pgfpathlineto{\pgfqpoint{0.684865in}{1.863506in}}%
\pgfpathlineto{\pgfqpoint{0.691077in}{1.852777in}}%
\pgfpathlineto{\pgfqpoint{0.709714in}{1.810385in}}%
\pgfpathlineto{\pgfqpoint{0.712821in}{1.808028in}}%
\pgfpathlineto{\pgfqpoint{0.715927in}{1.808609in}}%
\pgfpathlineto{\pgfqpoint{0.719033in}{1.812286in}}%
\pgfpathlineto{\pgfqpoint{0.725245in}{1.827827in}}%
\pgfpathlineto{\pgfqpoint{0.737670in}{1.866277in}}%
\pgfpathlineto{\pgfqpoint{0.740777in}{1.870239in}}%
\pgfpathlineto{\pgfqpoint{0.743883in}{1.869719in}}%
\pgfpathlineto{\pgfqpoint{0.746989in}{1.864100in}}%
\pgfpathlineto{\pgfqpoint{0.753201in}{1.837144in}}%
\pgfpathlineto{\pgfqpoint{0.762520in}{1.766151in}}%
\pgfpathlineto{\pgfqpoint{0.774945in}{1.663374in}}%
\pgfpathlineto{\pgfqpoint{0.781157in}{1.630427in}}%
\pgfpathlineto{\pgfqpoint{0.784264in}{1.621569in}}%
\pgfpathlineto{\pgfqpoint{0.787370in}{1.618345in}}%
\pgfpathlineto{\pgfqpoint{0.790476in}{1.620865in}}%
\pgfpathlineto{\pgfqpoint{0.793582in}{1.628983in}}%
\pgfpathlineto{\pgfqpoint{0.799795in}{1.660110in}}%
\pgfpathlineto{\pgfqpoint{0.821538in}{1.806886in}}%
\pgfpathlineto{\pgfqpoint{0.824644in}{1.812545in}}%
\pgfpathlineto{\pgfqpoint{0.827751in}{1.810183in}}%
\pgfpathlineto{\pgfqpoint{0.830857in}{1.799072in}}%
\pgfpathlineto{\pgfqpoint{0.837069in}{1.750011in}}%
\pgfpathlineto{\pgfqpoint{0.843282in}{1.669242in}}%
\pgfpathlineto{\pgfqpoint{0.868131in}{1.300664in}}%
\pgfpathlineto{\pgfqpoint{0.874344in}{1.262808in}}%
\pgfpathlineto{\pgfqpoint{0.877450in}{1.255562in}}%
\pgfpathlineto{\pgfqpoint{0.880556in}{1.255566in}}%
\pgfpathlineto{\pgfqpoint{0.883662in}{1.262200in}}%
\pgfpathlineto{\pgfqpoint{0.889875in}{1.292603in}}%
\pgfpathlineto{\pgfqpoint{0.896087in}{1.341464in}}%
\pgfpathlineto{\pgfqpoint{0.908512in}{1.474957in}}%
\pgfpathlineto{\pgfqpoint{0.924043in}{1.643777in}}%
\pgfpathlineto{\pgfqpoint{0.930256in}{1.687823in}}%
\pgfpathlineto{\pgfqpoint{0.936468in}{1.711353in}}%
\pgfpathlineto{\pgfqpoint{0.939574in}{1.715566in}}%
\pgfpathlineto{\pgfqpoint{0.942680in}{1.715347in}}%
\pgfpathlineto{\pgfqpoint{0.945787in}{1.711371in}}%
\pgfpathlineto{\pgfqpoint{0.951999in}{1.695081in}}%
\pgfpathlineto{\pgfqpoint{0.961318in}{1.658019in}}%
\pgfpathlineto{\pgfqpoint{0.970636in}{1.606975in}}%
\pgfpathlineto{\pgfqpoint{0.976849in}{1.559022in}}%
\pgfpathlineto{\pgfqpoint{0.986167in}{1.456527in}}%
\pgfpathlineto{\pgfqpoint{1.001698in}{1.226619in}}%
\pgfpathlineto{\pgfqpoint{1.011017in}{1.103471in}}%
\pgfpathlineto{\pgfqpoint{1.017229in}{1.047553in}}%
\pgfpathlineto{\pgfqpoint{1.023442in}{1.018268in}}%
\pgfpathlineto{\pgfqpoint{1.026548in}{1.013911in}}%
\pgfpathlineto{\pgfqpoint{1.029654in}{1.016132in}}%
\pgfpathlineto{\pgfqpoint{1.032761in}{1.024463in}}%
\pgfpathlineto{\pgfqpoint{1.038973in}{1.056418in}}%
\pgfpathlineto{\pgfqpoint{1.054504in}{1.162831in}}%
\pgfpathlineto{\pgfqpoint{1.057610in}{1.173979in}}%
\pgfpathlineto{\pgfqpoint{1.060716in}{1.177179in}}%
\pgfpathlineto{\pgfqpoint{1.063823in}{1.170910in}}%
\pgfpathlineto{\pgfqpoint{1.066929in}{1.154158in}}%
\pgfpathlineto{\pgfqpoint{1.073141in}{1.088360in}}%
\pgfpathlineto{\pgfqpoint{1.082460in}{0.924056in}}%
\pgfpathlineto{\pgfqpoint{1.097991in}{0.632160in}}%
\pgfpathlineto{\pgfqpoint{1.104203in}{0.568768in}}%
\pgfpathlineto{\pgfqpoint{1.107310in}{0.555072in}}%
\pgfpathlineto{\pgfqpoint{1.110416in}{0.554007in}}%
\pgfpathlineto{\pgfqpoint{1.113522in}{0.565442in}}%
\pgfpathlineto{\pgfqpoint{1.119734in}{0.623409in}}%
\pgfpathlineto{\pgfqpoint{1.125947in}{0.721576in}}%
\pgfpathlineto{\pgfqpoint{1.138372in}{0.995423in}}%
\pgfpathlineto{\pgfqpoint{1.153903in}{1.337186in}}%
\pgfpathlineto{\pgfqpoint{1.160115in}{1.422807in}}%
\pgfpathlineto{\pgfqpoint{1.166328in}{1.461649in}}%
\pgfpathlineto{\pgfqpoint{1.169434in}{1.462534in}}%
\pgfpathlineto{\pgfqpoint{1.172540in}{1.451569in}}%
\pgfpathlineto{\pgfqpoint{1.178753in}{1.398303in}}%
\pgfpathlineto{\pgfqpoint{1.188071in}{1.263396in}}%
\pgfpathlineto{\pgfqpoint{1.206708in}{0.969037in}}%
\pgfpathlineto{\pgfqpoint{1.212921in}{0.905913in}}%
\pgfpathlineto{\pgfqpoint{1.219133in}{0.870015in}}%
\pgfpathlineto{\pgfqpoint{1.222239in}{0.862061in}}%
\pgfpathlineto{\pgfqpoint{1.225346in}{0.860097in}}%
\pgfpathlineto{\pgfqpoint{1.228452in}{0.863367in}}%
\pgfpathlineto{\pgfqpoint{1.234664in}{0.882162in}}%
\pgfpathlineto{\pgfqpoint{1.243983in}{0.929046in}}%
\pgfpathlineto{\pgfqpoint{1.275045in}{1.108287in}}%
\pgfpathlineto{\pgfqpoint{1.281258in}{1.125553in}}%
\pgfpathlineto{\pgfqpoint{1.284364in}{1.129540in}}%
\pgfpathlineto{\pgfqpoint{1.287470in}{1.130616in}}%
\pgfpathlineto{\pgfqpoint{1.290576in}{1.129231in}}%
\pgfpathlineto{\pgfqpoint{1.299895in}{1.117586in}}%
\pgfpathlineto{\pgfqpoint{1.306107in}{1.111710in}}%
\pgfpathlineto{\pgfqpoint{1.309213in}{1.111170in}}%
\pgfpathlineto{\pgfqpoint{1.315426in}{1.115179in}}%
\pgfpathlineto{\pgfqpoint{1.324744in}{1.124542in}}%
\pgfpathlineto{\pgfqpoint{1.327851in}{1.124248in}}%
\pgfpathlineto{\pgfqpoint{1.330957in}{1.120172in}}%
\pgfpathlineto{\pgfqpoint{1.334063in}{1.111221in}}%
\pgfpathlineto{\pgfqpoint{1.340276in}{1.075755in}}%
\pgfpathlineto{\pgfqpoint{1.346488in}{1.015770in}}%
\pgfpathlineto{\pgfqpoint{1.355807in}{0.890549in}}%
\pgfpathlineto{\pgfqpoint{1.371338in}{0.674426in}}%
\pgfpathlineto{\pgfqpoint{1.377550in}{0.615753in}}%
\pgfpathlineto{\pgfqpoint{1.383763in}{0.583697in}}%
\pgfpathlineto{\pgfqpoint{1.386869in}{0.578975in}}%
\pgfpathlineto{\pgfqpoint{1.389975in}{0.582271in}}%
\pgfpathlineto{\pgfqpoint{1.393081in}{0.593805in}}%
\pgfpathlineto{\pgfqpoint{1.399294in}{0.641718in}}%
\pgfpathlineto{\pgfqpoint{1.405506in}{0.720941in}}%
\pgfpathlineto{\pgfqpoint{1.414825in}{0.884201in}}%
\pgfpathlineto{\pgfqpoint{1.430356in}{1.171069in}}%
\pgfpathlineto{\pgfqpoint{1.436568in}{1.251568in}}%
\pgfpathlineto{\pgfqpoint{1.442781in}{1.300917in}}%
\pgfpathlineto{\pgfqpoint{1.448993in}{1.320769in}}%
\pgfpathlineto{\pgfqpoint{1.452099in}{1.321892in}}%
\pgfpathlineto{\pgfqpoint{1.455205in}{1.318888in}}%
\pgfpathlineto{\pgfqpoint{1.473843in}{1.286577in}}%
\pgfpathlineto{\pgfqpoint{1.480055in}{1.287069in}}%
\pgfpathlineto{\pgfqpoint{1.486268in}{1.290026in}}%
\pgfpathlineto{\pgfqpoint{1.489374in}{1.289947in}}%
\pgfpathlineto{\pgfqpoint{1.492480in}{1.287396in}}%
\pgfpathlineto{\pgfqpoint{1.495586in}{1.281454in}}%
\pgfpathlineto{\pgfqpoint{1.501799in}{1.257162in}}%
\pgfpathlineto{\pgfqpoint{1.508011in}{1.216500in}}%
\pgfpathlineto{\pgfqpoint{1.526648in}{1.072776in}}%
\pgfpathlineto{\pgfqpoint{1.532861in}{1.050082in}}%
\pgfpathlineto{\pgfqpoint{1.535967in}{1.046657in}}%
\pgfpathlineto{\pgfqpoint{1.539073in}{1.048496in}}%
\pgfpathlineto{\pgfqpoint{1.542179in}{1.055332in}}%
\pgfpathlineto{\pgfqpoint{1.548392in}{1.082488in}}%
\pgfpathlineto{\pgfqpoint{1.554604in}{1.125247in}}%
\pgfpathlineto{\pgfqpoint{1.563923in}{1.214336in}}%
\pgfpathlineto{\pgfqpoint{1.579454in}{1.406954in}}%
\pgfpathlineto{\pgfqpoint{1.588773in}{1.513252in}}%
\pgfpathlineto{\pgfqpoint{1.594985in}{1.560777in}}%
\pgfpathlineto{\pgfqpoint{1.598091in}{1.574698in}}%
\pgfpathlineto{\pgfqpoint{1.601197in}{1.581457in}}%
\pgfpathlineto{\pgfqpoint{1.604304in}{1.581057in}}%
\pgfpathlineto{\pgfqpoint{1.607410in}{1.573881in}}%
\pgfpathlineto{\pgfqpoint{1.613622in}{1.542376in}}%
\pgfpathlineto{\pgfqpoint{1.626047in}{1.443924in}}%
\pgfpathlineto{\pgfqpoint{1.635366in}{1.377553in}}%
\pgfpathlineto{\pgfqpoint{1.641578in}{1.351063in}}%
\pgfpathlineto{\pgfqpoint{1.644684in}{1.345195in}}%
\pgfpathlineto{\pgfqpoint{1.647791in}{1.344994in}}%
\pgfpathlineto{\pgfqpoint{1.650897in}{1.350918in}}%
\pgfpathlineto{\pgfqpoint{1.654003in}{1.363285in}}%
\pgfpathlineto{\pgfqpoint{1.660215in}{1.407468in}}%
\pgfpathlineto{\pgfqpoint{1.669534in}{1.514019in}}%
\pgfpathlineto{\pgfqpoint{1.685065in}{1.703500in}}%
\pgfpathlineto{\pgfqpoint{1.691278in}{1.745826in}}%
\pgfpathlineto{\pgfqpoint{1.694384in}{1.756569in}}%
\pgfpathlineto{\pgfqpoint{1.697490in}{1.760718in}}%
\pgfpathlineto{\pgfqpoint{1.700596in}{1.759047in}}%
\pgfpathlineto{\pgfqpoint{1.703702in}{1.752578in}}%
\pgfpathlineto{\pgfqpoint{1.709915in}{1.729613in}}%
\pgfpathlineto{\pgfqpoint{1.719233in}{1.681972in}}%
\pgfpathlineto{\pgfqpoint{1.728552in}{1.620414in}}%
\pgfpathlineto{\pgfqpoint{1.737871in}{1.529071in}}%
\pgfpathlineto{\pgfqpoint{1.747189in}{1.397114in}}%
\pgfpathlineto{\pgfqpoint{1.765827in}{1.114942in}}%
\pgfpathlineto{\pgfqpoint{1.772039in}{1.067633in}}%
\pgfpathlineto{\pgfqpoint{1.775145in}{1.058352in}}%
\pgfpathlineto{\pgfqpoint{1.778252in}{1.058509in}}%
\pgfpathlineto{\pgfqpoint{1.781358in}{1.067174in}}%
\pgfpathlineto{\pgfqpoint{1.787570in}{1.103591in}}%
\pgfpathlineto{\pgfqpoint{1.799995in}{1.194156in}}%
\pgfpathlineto{\pgfqpoint{1.806207in}{1.219985in}}%
\pgfpathlineto{\pgfqpoint{1.809314in}{1.225355in}}%
\pgfpathlineto{\pgfqpoint{1.812420in}{1.226187in}}%
\pgfpathlineto{\pgfqpoint{1.815526in}{1.223451in}}%
\pgfpathlineto{\pgfqpoint{1.827951in}{1.204359in}}%
\pgfpathlineto{\pgfqpoint{1.831057in}{1.204337in}}%
\pgfpathlineto{\pgfqpoint{1.834163in}{1.208204in}}%
\pgfpathlineto{\pgfqpoint{1.837270in}{1.216387in}}%
\pgfpathlineto{\pgfqpoint{1.843482in}{1.245725in}}%
\pgfpathlineto{\pgfqpoint{1.852801in}{1.315183in}}%
\pgfpathlineto{\pgfqpoint{1.868332in}{1.442538in}}%
\pgfpathlineto{\pgfqpoint{1.874544in}{1.477114in}}%
\pgfpathlineto{\pgfqpoint{1.880757in}{1.496039in}}%
\pgfpathlineto{\pgfqpoint{1.883863in}{1.499709in}}%
\pgfpathlineto{\pgfqpoint{1.886969in}{1.500074in}}%
\pgfpathlineto{\pgfqpoint{1.893181in}{1.493794in}}%
\pgfpathlineto{\pgfqpoint{1.902500in}{1.479890in}}%
\pgfpathlineto{\pgfqpoint{1.902500in}{1.479890in}}%
\pgfusepath{stroke}%
\end{pgfscope}%
\begin{pgfscope}%
\pgfpathrectangle{\pgfqpoint{0.275000in}{0.375000in}}{\pgfqpoint{1.705000in}{2.265000in}}%
\pgfusepath{clip}%
\pgfsetroundcap%
\pgfsetroundjoin%
\pgfsetlinewidth{1.505625pt}%
\definecolor{currentstroke}{rgb}{0.839216,0.152941,0.156863}%
\pgfsetstrokecolor{currentstroke}%
\pgfsetdash{}{0pt}%
\pgfpathmoveto{\pgfqpoint{0.352500in}{1.753876in}}%
\pgfpathlineto{\pgfqpoint{0.358712in}{1.735894in}}%
\pgfpathlineto{\pgfqpoint{0.368031in}{1.705275in}}%
\pgfpathlineto{\pgfqpoint{0.371137in}{1.699834in}}%
\pgfpathlineto{\pgfqpoint{0.374243in}{1.698164in}}%
\pgfpathlineto{\pgfqpoint{0.377350in}{1.700303in}}%
\pgfpathlineto{\pgfqpoint{0.383562in}{1.713233in}}%
\pgfpathlineto{\pgfqpoint{0.389775in}{1.728230in}}%
\pgfpathlineto{\pgfqpoint{0.392881in}{1.731927in}}%
\pgfpathlineto{\pgfqpoint{0.395987in}{1.730658in}}%
\pgfpathlineto{\pgfqpoint{0.399093in}{1.722966in}}%
\pgfpathlineto{\pgfqpoint{0.402199in}{1.707890in}}%
\pgfpathlineto{\pgfqpoint{0.408412in}{1.654792in}}%
\pgfpathlineto{\pgfqpoint{0.417730in}{1.530406in}}%
\pgfpathlineto{\pgfqpoint{0.433262in}{1.317453in}}%
\pgfpathlineto{\pgfqpoint{0.439474in}{1.265691in}}%
\pgfpathlineto{\pgfqpoint{0.445686in}{1.240465in}}%
\pgfpathlineto{\pgfqpoint{0.448793in}{1.237636in}}%
\pgfpathlineto{\pgfqpoint{0.451899in}{1.240985in}}%
\pgfpathlineto{\pgfqpoint{0.455005in}{1.250202in}}%
\pgfpathlineto{\pgfqpoint{0.461217in}{1.285007in}}%
\pgfpathlineto{\pgfqpoint{0.467430in}{1.339568in}}%
\pgfpathlineto{\pgfqpoint{0.476748in}{1.451601in}}%
\pgfpathlineto{\pgfqpoint{0.510917in}{1.916675in}}%
\pgfpathlineto{\pgfqpoint{0.517129in}{1.965946in}}%
\pgfpathlineto{\pgfqpoint{0.523342in}{1.994109in}}%
\pgfpathlineto{\pgfqpoint{0.526448in}{1.999795in}}%
\pgfpathlineto{\pgfqpoint{0.529554in}{1.999958in}}%
\pgfpathlineto{\pgfqpoint{0.532660in}{1.994879in}}%
\pgfpathlineto{\pgfqpoint{0.538873in}{1.970929in}}%
\pgfpathlineto{\pgfqpoint{0.548191in}{1.910702in}}%
\pgfpathlineto{\pgfqpoint{0.594785in}{1.552537in}}%
\pgfpathlineto{\pgfqpoint{0.607209in}{1.402794in}}%
\pgfpathlineto{\pgfqpoint{0.644484in}{0.906253in}}%
\pgfpathlineto{\pgfqpoint{0.650696in}{0.862405in}}%
\pgfpathlineto{\pgfqpoint{0.653803in}{0.852260in}}%
\pgfpathlineto{\pgfqpoint{0.656909in}{0.851123in}}%
\pgfpathlineto{\pgfqpoint{0.660015in}{0.859296in}}%
\pgfpathlineto{\pgfqpoint{0.666227in}{0.901917in}}%
\pgfpathlineto{\pgfqpoint{0.675546in}{1.011248in}}%
\pgfpathlineto{\pgfqpoint{0.687971in}{1.158284in}}%
\pgfpathlineto{\pgfqpoint{0.694183in}{1.202943in}}%
\pgfpathlineto{\pgfqpoint{0.700396in}{1.222054in}}%
\pgfpathlineto{\pgfqpoint{0.703502in}{1.222581in}}%
\pgfpathlineto{\pgfqpoint{0.706608in}{1.217803in}}%
\pgfpathlineto{\pgfqpoint{0.712821in}{1.194977in}}%
\pgfpathlineto{\pgfqpoint{0.722139in}{1.137556in}}%
\pgfpathlineto{\pgfqpoint{0.743883in}{0.983270in}}%
\pgfpathlineto{\pgfqpoint{0.750095in}{0.956415in}}%
\pgfpathlineto{\pgfqpoint{0.753201in}{0.948845in}}%
\pgfpathlineto{\pgfqpoint{0.756308in}{0.945834in}}%
\pgfpathlineto{\pgfqpoint{0.759414in}{0.947756in}}%
\pgfpathlineto{\pgfqpoint{0.762520in}{0.954846in}}%
\pgfpathlineto{\pgfqpoint{0.768732in}{0.984554in}}%
\pgfpathlineto{\pgfqpoint{0.774945in}{1.032738in}}%
\pgfpathlineto{\pgfqpoint{0.802901in}{1.285485in}}%
\pgfpathlineto{\pgfqpoint{0.809113in}{1.313240in}}%
\pgfpathlineto{\pgfqpoint{0.815326in}{1.329736in}}%
\pgfpathlineto{\pgfqpoint{0.821538in}{1.339077in}}%
\pgfpathlineto{\pgfqpoint{0.833963in}{1.353509in}}%
\pgfpathlineto{\pgfqpoint{0.840175in}{1.365867in}}%
\pgfpathlineto{\pgfqpoint{0.846388in}{1.384778in}}%
\pgfpathlineto{\pgfqpoint{0.855706in}{1.426155in}}%
\pgfpathlineto{\pgfqpoint{0.871237in}{1.516807in}}%
\pgfpathlineto{\pgfqpoint{0.880556in}{1.567301in}}%
\pgfpathlineto{\pgfqpoint{0.886769in}{1.588880in}}%
\pgfpathlineto{\pgfqpoint{0.889875in}{1.593584in}}%
\pgfpathlineto{\pgfqpoint{0.892981in}{1.593198in}}%
\pgfpathlineto{\pgfqpoint{0.896087in}{1.587139in}}%
\pgfpathlineto{\pgfqpoint{0.902300in}{1.556999in}}%
\pgfpathlineto{\pgfqpoint{0.908512in}{1.504564in}}%
\pgfpathlineto{\pgfqpoint{0.933362in}{1.254206in}}%
\pgfpathlineto{\pgfqpoint{0.936468in}{1.240871in}}%
\pgfpathlineto{\pgfqpoint{0.939574in}{1.235009in}}%
\pgfpathlineto{\pgfqpoint{0.942680in}{1.237240in}}%
\pgfpathlineto{\pgfqpoint{0.945787in}{1.248061in}}%
\pgfpathlineto{\pgfqpoint{0.951999in}{1.296857in}}%
\pgfpathlineto{\pgfqpoint{0.958211in}{1.382577in}}%
\pgfpathlineto{\pgfqpoint{0.967530in}{1.573954in}}%
\pgfpathlineto{\pgfqpoint{0.995486in}{2.237794in}}%
\pgfpathlineto{\pgfqpoint{1.001698in}{2.317734in}}%
\pgfpathlineto{\pgfqpoint{1.007911in}{2.358938in}}%
\pgfpathlineto{\pgfqpoint{1.011017in}{2.365543in}}%
\pgfpathlineto{\pgfqpoint{1.014123in}{2.363465in}}%
\pgfpathlineto{\pgfqpoint{1.017229in}{2.353334in}}%
\pgfpathlineto{\pgfqpoint{1.023442in}{2.311836in}}%
\pgfpathlineto{\pgfqpoint{1.032761in}{2.209310in}}%
\pgfpathlineto{\pgfqpoint{1.057610in}{1.901150in}}%
\pgfpathlineto{\pgfqpoint{1.063823in}{1.856054in}}%
\pgfpathlineto{\pgfqpoint{1.070035in}{1.828024in}}%
\pgfpathlineto{\pgfqpoint{1.076247in}{1.813622in}}%
\pgfpathlineto{\pgfqpoint{1.082460in}{1.808437in}}%
\pgfpathlineto{\pgfqpoint{1.088672in}{1.808869in}}%
\pgfpathlineto{\pgfqpoint{1.094885in}{1.813081in}}%
\pgfpathlineto{\pgfqpoint{1.101097in}{1.820789in}}%
\pgfpathlineto{\pgfqpoint{1.110416in}{1.838591in}}%
\pgfpathlineto{\pgfqpoint{1.119734in}{1.856859in}}%
\pgfpathlineto{\pgfqpoint{1.122841in}{1.860024in}}%
\pgfpathlineto{\pgfqpoint{1.125947in}{1.860321in}}%
\pgfpathlineto{\pgfqpoint{1.129053in}{1.856988in}}%
\pgfpathlineto{\pgfqpoint{1.132159in}{1.849473in}}%
\pgfpathlineto{\pgfqpoint{1.138372in}{1.821381in}}%
\pgfpathlineto{\pgfqpoint{1.150797in}{1.732291in}}%
\pgfpathlineto{\pgfqpoint{1.157009in}{1.693510in}}%
\pgfpathlineto{\pgfqpoint{1.163221in}{1.674085in}}%
\pgfpathlineto{\pgfqpoint{1.166328in}{1.673596in}}%
\pgfpathlineto{\pgfqpoint{1.169434in}{1.679338in}}%
\pgfpathlineto{\pgfqpoint{1.175646in}{1.706995in}}%
\pgfpathlineto{\pgfqpoint{1.188071in}{1.794706in}}%
\pgfpathlineto{\pgfqpoint{1.197390in}{1.853911in}}%
\pgfpathlineto{\pgfqpoint{1.206708in}{1.895011in}}%
\pgfpathlineto{\pgfqpoint{1.225346in}{1.964700in}}%
\pgfpathlineto{\pgfqpoint{1.240877in}{2.033050in}}%
\pgfpathlineto{\pgfqpoint{1.247089in}{2.046720in}}%
\pgfpathlineto{\pgfqpoint{1.250195in}{2.047133in}}%
\pgfpathlineto{\pgfqpoint{1.253302in}{2.042482in}}%
\pgfpathlineto{\pgfqpoint{1.259514in}{2.017430in}}%
\pgfpathlineto{\pgfqpoint{1.265726in}{1.973906in}}%
\pgfpathlineto{\pgfqpoint{1.293682in}{1.742569in}}%
\pgfpathlineto{\pgfqpoint{1.306107in}{1.650607in}}%
\pgfpathlineto{\pgfqpoint{1.315426in}{1.550610in}}%
\pgfpathlineto{\pgfqpoint{1.327851in}{1.368020in}}%
\pgfpathlineto{\pgfqpoint{1.346488in}{1.085924in}}%
\pgfpathlineto{\pgfqpoint{1.355807in}{0.993384in}}%
\pgfpathlineto{\pgfqpoint{1.362019in}{0.957496in}}%
\pgfpathlineto{\pgfqpoint{1.368231in}{0.940219in}}%
\pgfpathlineto{\pgfqpoint{1.371338in}{0.937153in}}%
\pgfpathlineto{\pgfqpoint{1.374444in}{0.936845in}}%
\pgfpathlineto{\pgfqpoint{1.380656in}{0.941552in}}%
\pgfpathlineto{\pgfqpoint{1.405506in}{0.972467in}}%
\pgfpathlineto{\pgfqpoint{1.411718in}{0.986965in}}%
\pgfpathlineto{\pgfqpoint{1.417931in}{1.011138in}}%
\pgfpathlineto{\pgfqpoint{1.424143in}{1.047881in}}%
\pgfpathlineto{\pgfqpoint{1.433462in}{1.126900in}}%
\pgfpathlineto{\pgfqpoint{1.452099in}{1.307121in}}%
\pgfpathlineto{\pgfqpoint{1.458312in}{1.341639in}}%
\pgfpathlineto{\pgfqpoint{1.461418in}{1.349750in}}%
\pgfpathlineto{\pgfqpoint{1.464524in}{1.351339in}}%
\pgfpathlineto{\pgfqpoint{1.467630in}{1.346594in}}%
\pgfpathlineto{\pgfqpoint{1.473843in}{1.320939in}}%
\pgfpathlineto{\pgfqpoint{1.492480in}{1.217287in}}%
\pgfpathlineto{\pgfqpoint{1.495586in}{1.212530in}}%
\pgfpathlineto{\pgfqpoint{1.498692in}{1.214163in}}%
\pgfpathlineto{\pgfqpoint{1.501799in}{1.222347in}}%
\pgfpathlineto{\pgfqpoint{1.508011in}{1.256850in}}%
\pgfpathlineto{\pgfqpoint{1.517330in}{1.339867in}}%
\pgfpathlineto{\pgfqpoint{1.532861in}{1.485399in}}%
\pgfpathlineto{\pgfqpoint{1.542179in}{1.545519in}}%
\pgfpathlineto{\pgfqpoint{1.548392in}{1.571297in}}%
\pgfpathlineto{\pgfqpoint{1.554604in}{1.586054in}}%
\pgfpathlineto{\pgfqpoint{1.557710in}{1.589515in}}%
\pgfpathlineto{\pgfqpoint{1.560817in}{1.590560in}}%
\pgfpathlineto{\pgfqpoint{1.563923in}{1.589433in}}%
\pgfpathlineto{\pgfqpoint{1.570135in}{1.582189in}}%
\pgfpathlineto{\pgfqpoint{1.582560in}{1.564510in}}%
\pgfpathlineto{\pgfqpoint{1.585666in}{1.563676in}}%
\pgfpathlineto{\pgfqpoint{1.588773in}{1.565700in}}%
\pgfpathlineto{\pgfqpoint{1.591879in}{1.571139in}}%
\pgfpathlineto{\pgfqpoint{1.598091in}{1.593730in}}%
\pgfpathlineto{\pgfqpoint{1.604304in}{1.632525in}}%
\pgfpathlineto{\pgfqpoint{1.613622in}{1.715246in}}%
\pgfpathlineto{\pgfqpoint{1.629153in}{1.855840in}}%
\pgfpathlineto{\pgfqpoint{1.635366in}{1.884432in}}%
\pgfpathlineto{\pgfqpoint{1.638472in}{1.888947in}}%
\pgfpathlineto{\pgfqpoint{1.641578in}{1.886488in}}%
\pgfpathlineto{\pgfqpoint{1.644684in}{1.877148in}}%
\pgfpathlineto{\pgfqpoint{1.650897in}{1.839714in}}%
\pgfpathlineto{\pgfqpoint{1.660215in}{1.748700in}}%
\pgfpathlineto{\pgfqpoint{1.691278in}{1.400175in}}%
\pgfpathlineto{\pgfqpoint{1.700596in}{1.340557in}}%
\pgfpathlineto{\pgfqpoint{1.706809in}{1.318586in}}%
\pgfpathlineto{\pgfqpoint{1.713021in}{1.308575in}}%
\pgfpathlineto{\pgfqpoint{1.716127in}{1.307147in}}%
\pgfpathlineto{\pgfqpoint{1.719233in}{1.307661in}}%
\pgfpathlineto{\pgfqpoint{1.725446in}{1.313554in}}%
\pgfpathlineto{\pgfqpoint{1.731658in}{1.324920in}}%
\pgfpathlineto{\pgfqpoint{1.740977in}{1.350175in}}%
\pgfpathlineto{\pgfqpoint{1.756508in}{1.404255in}}%
\pgfpathlineto{\pgfqpoint{1.765827in}{1.444747in}}%
\pgfpathlineto{\pgfqpoint{1.772039in}{1.482153in}}%
\pgfpathlineto{\pgfqpoint{1.778252in}{1.533496in}}%
\pgfpathlineto{\pgfqpoint{1.787570in}{1.639081in}}%
\pgfpathlineto{\pgfqpoint{1.803101in}{1.825159in}}%
\pgfpathlineto{\pgfqpoint{1.809314in}{1.860645in}}%
\pgfpathlineto{\pgfqpoint{1.812420in}{1.863531in}}%
\pgfpathlineto{\pgfqpoint{1.815526in}{1.855784in}}%
\pgfpathlineto{\pgfqpoint{1.818632in}{1.837771in}}%
\pgfpathlineto{\pgfqpoint{1.824845in}{1.776167in}}%
\pgfpathlineto{\pgfqpoint{1.840376in}{1.588294in}}%
\pgfpathlineto{\pgfqpoint{1.846588in}{1.556139in}}%
\pgfpathlineto{\pgfqpoint{1.849694in}{1.556398in}}%
\pgfpathlineto{\pgfqpoint{1.852801in}{1.567850in}}%
\pgfpathlineto{\pgfqpoint{1.859013in}{1.620650in}}%
\pgfpathlineto{\pgfqpoint{1.880757in}{1.886363in}}%
\pgfpathlineto{\pgfqpoint{1.883863in}{1.901550in}}%
\pgfpathlineto{\pgfqpoint{1.886969in}{1.905546in}}%
\pgfpathlineto{\pgfqpoint{1.890075in}{1.897596in}}%
\pgfpathlineto{\pgfqpoint{1.893181in}{1.877410in}}%
\pgfpathlineto{\pgfqpoint{1.899394in}{1.801414in}}%
\pgfpathlineto{\pgfqpoint{1.902500in}{1.747170in}}%
\pgfpathlineto{\pgfqpoint{1.902500in}{1.747170in}}%
\pgfusepath{stroke}%
\end{pgfscope}%
\begin{pgfscope}%
\pgfpathrectangle{\pgfqpoint{0.275000in}{0.375000in}}{\pgfqpoint{1.705000in}{2.265000in}}%
\pgfusepath{clip}%
\pgfsetroundcap%
\pgfsetroundjoin%
\pgfsetlinewidth{1.505625pt}%
\definecolor{currentstroke}{rgb}{0.580392,0.403922,0.741176}%
\pgfsetstrokecolor{currentstroke}%
\pgfsetdash{}{0pt}%
\pgfpathmoveto{\pgfqpoint{0.352500in}{0.800841in}}%
\pgfpathlineto{\pgfqpoint{0.358712in}{0.831930in}}%
\pgfpathlineto{\pgfqpoint{0.361819in}{0.837882in}}%
\pgfpathlineto{\pgfqpoint{0.364925in}{0.837273in}}%
\pgfpathlineto{\pgfqpoint{0.368031in}{0.830573in}}%
\pgfpathlineto{\pgfqpoint{0.374243in}{0.803002in}}%
\pgfpathlineto{\pgfqpoint{0.386668in}{0.738746in}}%
\pgfpathlineto{\pgfqpoint{0.389775in}{0.732657in}}%
\pgfpathlineto{\pgfqpoint{0.392881in}{0.734399in}}%
\pgfpathlineto{\pgfqpoint{0.395987in}{0.745334in}}%
\pgfpathlineto{\pgfqpoint{0.402199in}{0.797489in}}%
\pgfpathlineto{\pgfqpoint{0.408412in}{0.888033in}}%
\pgfpathlineto{\pgfqpoint{0.436368in}{1.381476in}}%
\pgfpathlineto{\pgfqpoint{0.442580in}{1.433584in}}%
\pgfpathlineto{\pgfqpoint{0.448793in}{1.463381in}}%
\pgfpathlineto{\pgfqpoint{0.455005in}{1.479091in}}%
\pgfpathlineto{\pgfqpoint{0.461217in}{1.487431in}}%
\pgfpathlineto{\pgfqpoint{0.467430in}{1.491676in}}%
\pgfpathlineto{\pgfqpoint{0.473642in}{1.492201in}}%
\pgfpathlineto{\pgfqpoint{0.479855in}{1.488592in}}%
\pgfpathlineto{\pgfqpoint{0.495386in}{1.473046in}}%
\pgfpathlineto{\pgfqpoint{0.498492in}{1.473043in}}%
\pgfpathlineto{\pgfqpoint{0.501598in}{1.475559in}}%
\pgfpathlineto{\pgfqpoint{0.504704in}{1.481102in}}%
\pgfpathlineto{\pgfqpoint{0.510917in}{1.502498in}}%
\pgfpathlineto{\pgfqpoint{0.517129in}{1.537582in}}%
\pgfpathlineto{\pgfqpoint{0.541979in}{1.702000in}}%
\pgfpathlineto{\pgfqpoint{0.545085in}{1.706708in}}%
\pgfpathlineto{\pgfqpoint{0.548191in}{1.704571in}}%
\pgfpathlineto{\pgfqpoint{0.551298in}{1.695269in}}%
\pgfpathlineto{\pgfqpoint{0.557510in}{1.655920in}}%
\pgfpathlineto{\pgfqpoint{0.566829in}{1.558349in}}%
\pgfpathlineto{\pgfqpoint{0.579254in}{1.419159in}}%
\pgfpathlineto{\pgfqpoint{0.585466in}{1.370696in}}%
\pgfpathlineto{\pgfqpoint{0.591678in}{1.341898in}}%
\pgfpathlineto{\pgfqpoint{0.597891in}{1.329326in}}%
\pgfpathlineto{\pgfqpoint{0.604103in}{1.326402in}}%
\pgfpathlineto{\pgfqpoint{0.613422in}{1.327841in}}%
\pgfpathlineto{\pgfqpoint{0.622740in}{1.331435in}}%
\pgfpathlineto{\pgfqpoint{0.628953in}{1.337745in}}%
\pgfpathlineto{\pgfqpoint{0.635165in}{1.349854in}}%
\pgfpathlineto{\pgfqpoint{0.641378in}{1.368887in}}%
\pgfpathlineto{\pgfqpoint{0.650696in}{1.409377in}}%
\pgfpathlineto{\pgfqpoint{0.666227in}{1.484748in}}%
\pgfpathlineto{\pgfqpoint{0.672440in}{1.502681in}}%
\pgfpathlineto{\pgfqpoint{0.675546in}{1.506303in}}%
\pgfpathlineto{\pgfqpoint{0.678652in}{1.505853in}}%
\pgfpathlineto{\pgfqpoint{0.681759in}{1.501265in}}%
\pgfpathlineto{\pgfqpoint{0.687971in}{1.480755in}}%
\pgfpathlineto{\pgfqpoint{0.697290in}{1.430854in}}%
\pgfpathlineto{\pgfqpoint{0.722139in}{1.272100in}}%
\pgfpathlineto{\pgfqpoint{0.746989in}{1.089156in}}%
\pgfpathlineto{\pgfqpoint{0.750095in}{1.080302in}}%
\pgfpathlineto{\pgfqpoint{0.753201in}{1.077604in}}%
\pgfpathlineto{\pgfqpoint{0.756308in}{1.081151in}}%
\pgfpathlineto{\pgfqpoint{0.759414in}{1.090600in}}%
\pgfpathlineto{\pgfqpoint{0.765626in}{1.124083in}}%
\pgfpathlineto{\pgfqpoint{0.778051in}{1.220389in}}%
\pgfpathlineto{\pgfqpoint{0.802901in}{1.413017in}}%
\pgfpathlineto{\pgfqpoint{0.812219in}{1.463835in}}%
\pgfpathlineto{\pgfqpoint{0.815326in}{1.472301in}}%
\pgfpathlineto{\pgfqpoint{0.818432in}{1.474934in}}%
\pgfpathlineto{\pgfqpoint{0.821538in}{1.471028in}}%
\pgfpathlineto{\pgfqpoint{0.824644in}{1.460156in}}%
\pgfpathlineto{\pgfqpoint{0.830857in}{1.417461in}}%
\pgfpathlineto{\pgfqpoint{0.840175in}{1.309730in}}%
\pgfpathlineto{\pgfqpoint{0.858813in}{1.065278in}}%
\pgfpathlineto{\pgfqpoint{0.865025in}{1.018454in}}%
\pgfpathlineto{\pgfqpoint{0.868131in}{1.005983in}}%
\pgfpathlineto{\pgfqpoint{0.871237in}{1.000988in}}%
\pgfpathlineto{\pgfqpoint{0.874344in}{1.003096in}}%
\pgfpathlineto{\pgfqpoint{0.877450in}{1.011561in}}%
\pgfpathlineto{\pgfqpoint{0.883662in}{1.043133in}}%
\pgfpathlineto{\pgfqpoint{0.920937in}{1.298519in}}%
\pgfpathlineto{\pgfqpoint{0.930256in}{1.411300in}}%
\pgfpathlineto{\pgfqpoint{0.961318in}{1.861675in}}%
\pgfpathlineto{\pgfqpoint{0.967530in}{1.899499in}}%
\pgfpathlineto{\pgfqpoint{0.970636in}{1.909570in}}%
\pgfpathlineto{\pgfqpoint{0.973742in}{1.914247in}}%
\pgfpathlineto{\pgfqpoint{0.976849in}{1.913851in}}%
\pgfpathlineto{\pgfqpoint{0.979955in}{1.908541in}}%
\pgfpathlineto{\pgfqpoint{0.986167in}{1.882696in}}%
\pgfpathlineto{\pgfqpoint{0.992380in}{1.833775in}}%
\pgfpathlineto{\pgfqpoint{0.998592in}{1.757804in}}%
\pgfpathlineto{\pgfqpoint{1.007911in}{1.594244in}}%
\pgfpathlineto{\pgfqpoint{1.023442in}{1.300091in}}%
\pgfpathlineto{\pgfqpoint{1.029654in}{1.241481in}}%
\pgfpathlineto{\pgfqpoint{1.032761in}{1.235232in}}%
\pgfpathlineto{\pgfqpoint{1.035867in}{1.245881in}}%
\pgfpathlineto{\pgfqpoint{1.038973in}{1.273375in}}%
\pgfpathlineto{\pgfqpoint{1.045185in}{1.373815in}}%
\pgfpathlineto{\pgfqpoint{1.057610in}{1.679555in}}%
\pgfpathlineto{\pgfqpoint{1.066929in}{1.893416in}}%
\pgfpathlineto{\pgfqpoint{1.073141in}{1.988452in}}%
\pgfpathlineto{\pgfqpoint{1.079354in}{2.036298in}}%
\pgfpathlineto{\pgfqpoint{1.082460in}{2.043324in}}%
\pgfpathlineto{\pgfqpoint{1.085566in}{2.040496in}}%
\pgfpathlineto{\pgfqpoint{1.088672in}{2.029255in}}%
\pgfpathlineto{\pgfqpoint{1.094885in}{1.987907in}}%
\pgfpathlineto{\pgfqpoint{1.110416in}{1.840820in}}%
\pgfpathlineto{\pgfqpoint{1.122841in}{1.736267in}}%
\pgfpathlineto{\pgfqpoint{1.132159in}{1.678412in}}%
\pgfpathlineto{\pgfqpoint{1.141478in}{1.636260in}}%
\pgfpathlineto{\pgfqpoint{1.157009in}{1.583266in}}%
\pgfpathlineto{\pgfqpoint{1.166328in}{1.558253in}}%
\pgfpathlineto{\pgfqpoint{1.169434in}{1.553559in}}%
\pgfpathlineto{\pgfqpoint{1.172540in}{1.552246in}}%
\pgfpathlineto{\pgfqpoint{1.175646in}{1.555402in}}%
\pgfpathlineto{\pgfqpoint{1.178753in}{1.564066in}}%
\pgfpathlineto{\pgfqpoint{1.184965in}{1.601068in}}%
\pgfpathlineto{\pgfqpoint{1.191177in}{1.666054in}}%
\pgfpathlineto{\pgfqpoint{1.200496in}{1.805774in}}%
\pgfpathlineto{\pgfqpoint{1.219133in}{2.096535in}}%
\pgfpathlineto{\pgfqpoint{1.225346in}{2.157177in}}%
\pgfpathlineto{\pgfqpoint{1.231558in}{2.188062in}}%
\pgfpathlineto{\pgfqpoint{1.234664in}{2.190926in}}%
\pgfpathlineto{\pgfqpoint{1.237771in}{2.184855in}}%
\pgfpathlineto{\pgfqpoint{1.240877in}{2.169661in}}%
\pgfpathlineto{\pgfqpoint{1.247089in}{2.112696in}}%
\pgfpathlineto{\pgfqpoint{1.256408in}{1.974945in}}%
\pgfpathlineto{\pgfqpoint{1.268833in}{1.775204in}}%
\pgfpathlineto{\pgfqpoint{1.275045in}{1.708327in}}%
\pgfpathlineto{\pgfqpoint{1.281258in}{1.675920in}}%
\pgfpathlineto{\pgfqpoint{1.284364in}{1.671640in}}%
\pgfpathlineto{\pgfqpoint{1.287470in}{1.673399in}}%
\pgfpathlineto{\pgfqpoint{1.299895in}{1.696821in}}%
\pgfpathlineto{\pgfqpoint{1.303001in}{1.695565in}}%
\pgfpathlineto{\pgfqpoint{1.306107in}{1.687892in}}%
\pgfpathlineto{\pgfqpoint{1.309213in}{1.672809in}}%
\pgfpathlineto{\pgfqpoint{1.315426in}{1.619066in}}%
\pgfpathlineto{\pgfqpoint{1.324744in}{1.486775in}}%
\pgfpathlineto{\pgfqpoint{1.346488in}{1.125173in}}%
\pgfpathlineto{\pgfqpoint{1.352700in}{1.061437in}}%
\pgfpathlineto{\pgfqpoint{1.358913in}{1.029057in}}%
\pgfpathlineto{\pgfqpoint{1.362019in}{1.025111in}}%
\pgfpathlineto{\pgfqpoint{1.365125in}{1.029146in}}%
\pgfpathlineto{\pgfqpoint{1.368231in}{1.040837in}}%
\pgfpathlineto{\pgfqpoint{1.374444in}{1.085603in}}%
\pgfpathlineto{\pgfqpoint{1.380656in}{1.156429in}}%
\pgfpathlineto{\pgfqpoint{1.389975in}{1.306993in}}%
\pgfpathlineto{\pgfqpoint{1.402400in}{1.575393in}}%
\pgfpathlineto{\pgfqpoint{1.417931in}{1.917894in}}%
\pgfpathlineto{\pgfqpoint{1.424143in}{2.009615in}}%
\pgfpathlineto{\pgfqpoint{1.430356in}{2.055820in}}%
\pgfpathlineto{\pgfqpoint{1.433462in}{2.059863in}}%
\pgfpathlineto{\pgfqpoint{1.436568in}{2.051009in}}%
\pgfpathlineto{\pgfqpoint{1.439674in}{2.029657in}}%
\pgfpathlineto{\pgfqpoint{1.445887in}{1.953126in}}%
\pgfpathlineto{\pgfqpoint{1.455205in}{1.777673in}}%
\pgfpathlineto{\pgfqpoint{1.467630in}{1.531943in}}%
\pgfpathlineto{\pgfqpoint{1.473843in}{1.455295in}}%
\pgfpathlineto{\pgfqpoint{1.476949in}{1.437121in}}%
\pgfpathlineto{\pgfqpoint{1.480055in}{1.434256in}}%
\pgfpathlineto{\pgfqpoint{1.483161in}{1.447138in}}%
\pgfpathlineto{\pgfqpoint{1.489374in}{1.517603in}}%
\pgfpathlineto{\pgfqpoint{1.498692in}{1.705183in}}%
\pgfpathlineto{\pgfqpoint{1.511117in}{1.976975in}}%
\pgfpathlineto{\pgfqpoint{1.517330in}{2.065574in}}%
\pgfpathlineto{\pgfqpoint{1.520436in}{2.090304in}}%
\pgfpathlineto{\pgfqpoint{1.523542in}{2.100568in}}%
\pgfpathlineto{\pgfqpoint{1.526648in}{2.096074in}}%
\pgfpathlineto{\pgfqpoint{1.529755in}{2.077130in}}%
\pgfpathlineto{\pgfqpoint{1.535967in}{1.999889in}}%
\pgfpathlineto{\pgfqpoint{1.545286in}{1.812903in}}%
\pgfpathlineto{\pgfqpoint{1.560817in}{1.481992in}}%
\pgfpathlineto{\pgfqpoint{1.567029in}{1.400383in}}%
\pgfpathlineto{\pgfqpoint{1.573241in}{1.363143in}}%
\pgfpathlineto{\pgfqpoint{1.576348in}{1.360328in}}%
\pgfpathlineto{\pgfqpoint{1.579454in}{1.366282in}}%
\pgfpathlineto{\pgfqpoint{1.585666in}{1.396237in}}%
\pgfpathlineto{\pgfqpoint{1.594985in}{1.450874in}}%
\pgfpathlineto{\pgfqpoint{1.601197in}{1.470346in}}%
\pgfpathlineto{\pgfqpoint{1.604304in}{1.471625in}}%
\pgfpathlineto{\pgfqpoint{1.607410in}{1.467063in}}%
\pgfpathlineto{\pgfqpoint{1.613622in}{1.442698in}}%
\pgfpathlineto{\pgfqpoint{1.626047in}{1.362855in}}%
\pgfpathlineto{\pgfqpoint{1.635366in}{1.309924in}}%
\pgfpathlineto{\pgfqpoint{1.641578in}{1.289219in}}%
\pgfpathlineto{\pgfqpoint{1.644684in}{1.284561in}}%
\pgfpathlineto{\pgfqpoint{1.647791in}{1.284072in}}%
\pgfpathlineto{\pgfqpoint{1.650897in}{1.287905in}}%
\pgfpathlineto{\pgfqpoint{1.657109in}{1.308445in}}%
\pgfpathlineto{\pgfqpoint{1.663322in}{1.344119in}}%
\pgfpathlineto{\pgfqpoint{1.685065in}{1.491939in}}%
\pgfpathlineto{\pgfqpoint{1.691278in}{1.510615in}}%
\pgfpathlineto{\pgfqpoint{1.694384in}{1.514065in}}%
\pgfpathlineto{\pgfqpoint{1.697490in}{1.514071in}}%
\pgfpathlineto{\pgfqpoint{1.703702in}{1.506479in}}%
\pgfpathlineto{\pgfqpoint{1.716127in}{1.484276in}}%
\pgfpathlineto{\pgfqpoint{1.719233in}{1.481613in}}%
\pgfpathlineto{\pgfqpoint{1.722340in}{1.481049in}}%
\pgfpathlineto{\pgfqpoint{1.725446in}{1.482728in}}%
\pgfpathlineto{\pgfqpoint{1.731658in}{1.492413in}}%
\pgfpathlineto{\pgfqpoint{1.753402in}{1.541473in}}%
\pgfpathlineto{\pgfqpoint{1.756508in}{1.541828in}}%
\pgfpathlineto{\pgfqpoint{1.759614in}{1.538914in}}%
\pgfpathlineto{\pgfqpoint{1.762720in}{1.532505in}}%
\pgfpathlineto{\pgfqpoint{1.768933in}{1.509109in}}%
\pgfpathlineto{\pgfqpoint{1.778252in}{1.451209in}}%
\pgfpathlineto{\pgfqpoint{1.803101in}{1.268241in}}%
\pgfpathlineto{\pgfqpoint{1.809314in}{1.244811in}}%
\pgfpathlineto{\pgfqpoint{1.812420in}{1.239773in}}%
\pgfpathlineto{\pgfqpoint{1.815526in}{1.239659in}}%
\pgfpathlineto{\pgfqpoint{1.818632in}{1.244601in}}%
\pgfpathlineto{\pgfqpoint{1.824845in}{1.269115in}}%
\pgfpathlineto{\pgfqpoint{1.834163in}{1.334363in}}%
\pgfpathlineto{\pgfqpoint{1.846588in}{1.431787in}}%
\pgfpathlineto{\pgfqpoint{1.852801in}{1.461575in}}%
\pgfpathlineto{\pgfqpoint{1.855907in}{1.467168in}}%
\pgfpathlineto{\pgfqpoint{1.859013in}{1.465066in}}%
\pgfpathlineto{\pgfqpoint{1.862119in}{1.454377in}}%
\pgfpathlineto{\pgfqpoint{1.868332in}{1.405087in}}%
\pgfpathlineto{\pgfqpoint{1.874544in}{1.318914in}}%
\pgfpathlineto{\pgfqpoint{1.886969in}{1.073903in}}%
\pgfpathlineto{\pgfqpoint{1.896288in}{0.902536in}}%
\pgfpathlineto{\pgfqpoint{1.902500in}{0.833713in}}%
\pgfpathlineto{\pgfqpoint{1.902500in}{0.833713in}}%
\pgfusepath{stroke}%
\end{pgfscope}%
\begin{pgfscope}%
\pgfpathrectangle{\pgfqpoint{0.275000in}{0.375000in}}{\pgfqpoint{1.705000in}{2.265000in}}%
\pgfusepath{clip}%
\pgfsetroundcap%
\pgfsetroundjoin%
\pgfsetlinewidth{1.505625pt}%
\definecolor{currentstroke}{rgb}{0.549020,0.337255,0.294118}%
\pgfsetstrokecolor{currentstroke}%
\pgfsetdash{}{0pt}%
\pgfpathmoveto{\pgfqpoint{0.352500in}{1.518393in}}%
\pgfpathlineto{\pgfqpoint{0.358712in}{1.512817in}}%
\pgfpathlineto{\pgfqpoint{0.364925in}{1.503218in}}%
\pgfpathlineto{\pgfqpoint{0.374243in}{1.481032in}}%
\pgfpathlineto{\pgfqpoint{0.383562in}{1.457188in}}%
\pgfpathlineto{\pgfqpoint{0.386668in}{1.452232in}}%
\pgfpathlineto{\pgfqpoint{0.389775in}{1.450318in}}%
\pgfpathlineto{\pgfqpoint{0.392881in}{1.452302in}}%
\pgfpathlineto{\pgfqpoint{0.395987in}{1.458846in}}%
\pgfpathlineto{\pgfqpoint{0.402199in}{1.486767in}}%
\pgfpathlineto{\pgfqpoint{0.411518in}{1.560269in}}%
\pgfpathlineto{\pgfqpoint{0.427049in}{1.690928in}}%
\pgfpathlineto{\pgfqpoint{0.433262in}{1.717883in}}%
\pgfpathlineto{\pgfqpoint{0.436368in}{1.723841in}}%
\pgfpathlineto{\pgfqpoint{0.439474in}{1.725148in}}%
\pgfpathlineto{\pgfqpoint{0.442580in}{1.722423in}}%
\pgfpathlineto{\pgfqpoint{0.448793in}{1.708077in}}%
\pgfpathlineto{\pgfqpoint{0.495386in}{1.562829in}}%
\pgfpathlineto{\pgfqpoint{0.504704in}{1.540724in}}%
\pgfpathlineto{\pgfqpoint{0.510917in}{1.532315in}}%
\pgfpathlineto{\pgfqpoint{0.514023in}{1.530705in}}%
\pgfpathlineto{\pgfqpoint{0.517129in}{1.530895in}}%
\pgfpathlineto{\pgfqpoint{0.523342in}{1.535942in}}%
\pgfpathlineto{\pgfqpoint{0.535767in}{1.551238in}}%
\pgfpathlineto{\pgfqpoint{0.538873in}{1.552490in}}%
\pgfpathlineto{\pgfqpoint{0.541979in}{1.551666in}}%
\pgfpathlineto{\pgfqpoint{0.548191in}{1.543331in}}%
\pgfpathlineto{\pgfqpoint{0.563722in}{1.510655in}}%
\pgfpathlineto{\pgfqpoint{0.566829in}{1.511283in}}%
\pgfpathlineto{\pgfqpoint{0.569935in}{1.517392in}}%
\pgfpathlineto{\pgfqpoint{0.573041in}{1.529966in}}%
\pgfpathlineto{\pgfqpoint{0.579254in}{1.576343in}}%
\pgfpathlineto{\pgfqpoint{0.588572in}{1.692632in}}%
\pgfpathlineto{\pgfqpoint{0.604103in}{1.913182in}}%
\pgfpathlineto{\pgfqpoint{0.610316in}{1.973776in}}%
\pgfpathlineto{\pgfqpoint{0.616528in}{2.010972in}}%
\pgfpathlineto{\pgfqpoint{0.622740in}{2.028484in}}%
\pgfpathlineto{\pgfqpoint{0.625847in}{2.031879in}}%
\pgfpathlineto{\pgfqpoint{0.628953in}{2.032742in}}%
\pgfpathlineto{\pgfqpoint{0.632059in}{2.031641in}}%
\pgfpathlineto{\pgfqpoint{0.638272in}{2.024595in}}%
\pgfpathlineto{\pgfqpoint{0.644484in}{2.010711in}}%
\pgfpathlineto{\pgfqpoint{0.650696in}{1.988393in}}%
\pgfpathlineto{\pgfqpoint{0.663121in}{1.923720in}}%
\pgfpathlineto{\pgfqpoint{0.669334in}{1.894942in}}%
\pgfpathlineto{\pgfqpoint{0.672440in}{1.885867in}}%
\pgfpathlineto{\pgfqpoint{0.675546in}{1.881856in}}%
\pgfpathlineto{\pgfqpoint{0.678652in}{1.883617in}}%
\pgfpathlineto{\pgfqpoint{0.681759in}{1.891343in}}%
\pgfpathlineto{\pgfqpoint{0.687971in}{1.922397in}}%
\pgfpathlineto{\pgfqpoint{0.700396in}{1.998509in}}%
\pgfpathlineto{\pgfqpoint{0.703502in}{2.006146in}}%
\pgfpathlineto{\pgfqpoint{0.706608in}{2.004263in}}%
\pgfpathlineto{\pgfqpoint{0.709714in}{1.991038in}}%
\pgfpathlineto{\pgfqpoint{0.712821in}{1.965318in}}%
\pgfpathlineto{\pgfqpoint{0.719033in}{1.875652in}}%
\pgfpathlineto{\pgfqpoint{0.728352in}{1.662434in}}%
\pgfpathlineto{\pgfqpoint{0.746989in}{1.180703in}}%
\pgfpathlineto{\pgfqpoint{0.756308in}{1.024081in}}%
\pgfpathlineto{\pgfqpoint{0.762520in}{0.970137in}}%
\pgfpathlineto{\pgfqpoint{0.765626in}{0.959499in}}%
\pgfpathlineto{\pgfqpoint{0.768732in}{0.960056in}}%
\pgfpathlineto{\pgfqpoint{0.771839in}{0.971906in}}%
\pgfpathlineto{\pgfqpoint{0.778051in}{1.028932in}}%
\pgfpathlineto{\pgfqpoint{0.784264in}{1.126396in}}%
\pgfpathlineto{\pgfqpoint{0.812219in}{1.647364in}}%
\pgfpathlineto{\pgfqpoint{0.815326in}{1.663461in}}%
\pgfpathlineto{\pgfqpoint{0.818432in}{1.665061in}}%
\pgfpathlineto{\pgfqpoint{0.821538in}{1.652200in}}%
\pgfpathlineto{\pgfqpoint{0.827751in}{1.586253in}}%
\pgfpathlineto{\pgfqpoint{0.837069in}{1.411022in}}%
\pgfpathlineto{\pgfqpoint{0.855706in}{1.016445in}}%
\pgfpathlineto{\pgfqpoint{0.861919in}{0.941370in}}%
\pgfpathlineto{\pgfqpoint{0.865025in}{0.922247in}}%
\pgfpathlineto{\pgfqpoint{0.868131in}{0.916292in}}%
\pgfpathlineto{\pgfqpoint{0.871237in}{0.923452in}}%
\pgfpathlineto{\pgfqpoint{0.874344in}{0.943083in}}%
\pgfpathlineto{\pgfqpoint{0.880556in}{1.014487in}}%
\pgfpathlineto{\pgfqpoint{0.892981in}{1.229920in}}%
\pgfpathlineto{\pgfqpoint{0.905406in}{1.437565in}}%
\pgfpathlineto{\pgfqpoint{0.914724in}{1.551676in}}%
\pgfpathlineto{\pgfqpoint{0.924043in}{1.635379in}}%
\pgfpathlineto{\pgfqpoint{0.933362in}{1.698509in}}%
\pgfpathlineto{\pgfqpoint{0.942680in}{1.741853in}}%
\pgfpathlineto{\pgfqpoint{0.948893in}{1.758028in}}%
\pgfpathlineto{\pgfqpoint{0.955105in}{1.764167in}}%
\pgfpathlineto{\pgfqpoint{0.958211in}{1.763947in}}%
\pgfpathlineto{\pgfqpoint{0.964424in}{1.757989in}}%
\pgfpathlineto{\pgfqpoint{0.970636in}{1.744821in}}%
\pgfpathlineto{\pgfqpoint{0.976849in}{1.721821in}}%
\pgfpathlineto{\pgfqpoint{0.983061in}{1.683280in}}%
\pgfpathlineto{\pgfqpoint{0.989274in}{1.622823in}}%
\pgfpathlineto{\pgfqpoint{0.998592in}{1.486932in}}%
\pgfpathlineto{\pgfqpoint{1.017229in}{1.185964in}}%
\pgfpathlineto{\pgfqpoint{1.023442in}{1.143167in}}%
\pgfpathlineto{\pgfqpoint{1.026548in}{1.139597in}}%
\pgfpathlineto{\pgfqpoint{1.029654in}{1.147773in}}%
\pgfpathlineto{\pgfqpoint{1.035867in}{1.194563in}}%
\pgfpathlineto{\pgfqpoint{1.045185in}{1.310363in}}%
\pgfpathlineto{\pgfqpoint{1.057610in}{1.456746in}}%
\pgfpathlineto{\pgfqpoint{1.063823in}{1.498819in}}%
\pgfpathlineto{\pgfqpoint{1.066929in}{1.509981in}}%
\pgfpathlineto{\pgfqpoint{1.070035in}{1.514701in}}%
\pgfpathlineto{\pgfqpoint{1.073141in}{1.513444in}}%
\pgfpathlineto{\pgfqpoint{1.076247in}{1.506914in}}%
\pgfpathlineto{\pgfqpoint{1.082460in}{1.481788in}}%
\pgfpathlineto{\pgfqpoint{1.101097in}{1.386996in}}%
\pgfpathlineto{\pgfqpoint{1.107310in}{1.368945in}}%
\pgfpathlineto{\pgfqpoint{1.116628in}{1.353342in}}%
\pgfpathlineto{\pgfqpoint{1.122841in}{1.342991in}}%
\pgfpathlineto{\pgfqpoint{1.129053in}{1.327921in}}%
\pgfpathlineto{\pgfqpoint{1.144584in}{1.283074in}}%
\pgfpathlineto{\pgfqpoint{1.147690in}{1.279929in}}%
\pgfpathlineto{\pgfqpoint{1.150797in}{1.281349in}}%
\pgfpathlineto{\pgfqpoint{1.153903in}{1.288334in}}%
\pgfpathlineto{\pgfqpoint{1.157009in}{1.301692in}}%
\pgfpathlineto{\pgfqpoint{1.163221in}{1.349611in}}%
\pgfpathlineto{\pgfqpoint{1.169434in}{1.426888in}}%
\pgfpathlineto{\pgfqpoint{1.178753in}{1.593364in}}%
\pgfpathlineto{\pgfqpoint{1.194284in}{1.954072in}}%
\pgfpathlineto{\pgfqpoint{1.206708in}{2.211912in}}%
\pgfpathlineto{\pgfqpoint{1.212921in}{2.282708in}}%
\pgfpathlineto{\pgfqpoint{1.216027in}{2.295800in}}%
\pgfpathlineto{\pgfqpoint{1.219133in}{2.292057in}}%
\pgfpathlineto{\pgfqpoint{1.222239in}{2.270887in}}%
\pgfpathlineto{\pgfqpoint{1.228452in}{2.178332in}}%
\pgfpathlineto{\pgfqpoint{1.237771in}{1.945279in}}%
\pgfpathlineto{\pgfqpoint{1.253302in}{1.538104in}}%
\pgfpathlineto{\pgfqpoint{1.259514in}{1.439109in}}%
\pgfpathlineto{\pgfqpoint{1.265726in}{1.387260in}}%
\pgfpathlineto{\pgfqpoint{1.268833in}{1.376448in}}%
\pgfpathlineto{\pgfqpoint{1.271939in}{1.373482in}}%
\pgfpathlineto{\pgfqpoint{1.275045in}{1.376546in}}%
\pgfpathlineto{\pgfqpoint{1.281258in}{1.394313in}}%
\pgfpathlineto{\pgfqpoint{1.290576in}{1.435213in}}%
\pgfpathlineto{\pgfqpoint{1.299895in}{1.490714in}}%
\pgfpathlineto{\pgfqpoint{1.309213in}{1.574017in}}%
\pgfpathlineto{\pgfqpoint{1.321638in}{1.731827in}}%
\pgfpathlineto{\pgfqpoint{1.330957in}{1.847942in}}%
\pgfpathlineto{\pgfqpoint{1.337169in}{1.896496in}}%
\pgfpathlineto{\pgfqpoint{1.340276in}{1.906743in}}%
\pgfpathlineto{\pgfqpoint{1.343382in}{1.905996in}}%
\pgfpathlineto{\pgfqpoint{1.346488in}{1.893640in}}%
\pgfpathlineto{\pgfqpoint{1.352700in}{1.834487in}}%
\pgfpathlineto{\pgfqpoint{1.358913in}{1.735687in}}%
\pgfpathlineto{\pgfqpoint{1.383763in}{1.271744in}}%
\pgfpathlineto{\pgfqpoint{1.389975in}{1.220030in}}%
\pgfpathlineto{\pgfqpoint{1.393081in}{1.209419in}}%
\pgfpathlineto{\pgfqpoint{1.396187in}{1.208452in}}%
\pgfpathlineto{\pgfqpoint{1.399294in}{1.216182in}}%
\pgfpathlineto{\pgfqpoint{1.405506in}{1.252229in}}%
\pgfpathlineto{\pgfqpoint{1.427249in}{1.425451in}}%
\pgfpathlineto{\pgfqpoint{1.433462in}{1.449433in}}%
\pgfpathlineto{\pgfqpoint{1.436568in}{1.454913in}}%
\pgfpathlineto{\pgfqpoint{1.439674in}{1.456222in}}%
\pgfpathlineto{\pgfqpoint{1.442781in}{1.453656in}}%
\pgfpathlineto{\pgfqpoint{1.448993in}{1.438648in}}%
\pgfpathlineto{\pgfqpoint{1.470736in}{1.365318in}}%
\pgfpathlineto{\pgfqpoint{1.473843in}{1.366223in}}%
\pgfpathlineto{\pgfqpoint{1.476949in}{1.372865in}}%
\pgfpathlineto{\pgfqpoint{1.483161in}{1.404462in}}%
\pgfpathlineto{\pgfqpoint{1.489374in}{1.458094in}}%
\pgfpathlineto{\pgfqpoint{1.508011in}{1.647322in}}%
\pgfpathlineto{\pgfqpoint{1.514223in}{1.677970in}}%
\pgfpathlineto{\pgfqpoint{1.517330in}{1.682635in}}%
\pgfpathlineto{\pgfqpoint{1.520436in}{1.680096in}}%
\pgfpathlineto{\pgfqpoint{1.523542in}{1.670767in}}%
\pgfpathlineto{\pgfqpoint{1.529755in}{1.634847in}}%
\pgfpathlineto{\pgfqpoint{1.539073in}{1.553164in}}%
\pgfpathlineto{\pgfqpoint{1.557710in}{1.378291in}}%
\pgfpathlineto{\pgfqpoint{1.563923in}{1.336110in}}%
\pgfpathlineto{\pgfqpoint{1.570135in}{1.312092in}}%
\pgfpathlineto{\pgfqpoint{1.573241in}{1.309230in}}%
\pgfpathlineto{\pgfqpoint{1.576348in}{1.313572in}}%
\pgfpathlineto{\pgfqpoint{1.579454in}{1.325674in}}%
\pgfpathlineto{\pgfqpoint{1.585666in}{1.373574in}}%
\pgfpathlineto{\pgfqpoint{1.594985in}{1.494289in}}%
\pgfpathlineto{\pgfqpoint{1.610516in}{1.710416in}}%
\pgfpathlineto{\pgfqpoint{1.616728in}{1.755104in}}%
\pgfpathlineto{\pgfqpoint{1.619835in}{1.763382in}}%
\pgfpathlineto{\pgfqpoint{1.622941in}{1.762185in}}%
\pgfpathlineto{\pgfqpoint{1.626047in}{1.752217in}}%
\pgfpathlineto{\pgfqpoint{1.632260in}{1.711515in}}%
\pgfpathlineto{\pgfqpoint{1.647791in}{1.583002in}}%
\pgfpathlineto{\pgfqpoint{1.654003in}{1.558994in}}%
\pgfpathlineto{\pgfqpoint{1.657109in}{1.556193in}}%
\pgfpathlineto{\pgfqpoint{1.660215in}{1.559041in}}%
\pgfpathlineto{\pgfqpoint{1.666428in}{1.577583in}}%
\pgfpathlineto{\pgfqpoint{1.678853in}{1.626083in}}%
\pgfpathlineto{\pgfqpoint{1.681959in}{1.632460in}}%
\pgfpathlineto{\pgfqpoint{1.685065in}{1.634522in}}%
\pgfpathlineto{\pgfqpoint{1.688171in}{1.631803in}}%
\pgfpathlineto{\pgfqpoint{1.691278in}{1.624216in}}%
\pgfpathlineto{\pgfqpoint{1.697490in}{1.596063in}}%
\pgfpathlineto{\pgfqpoint{1.713021in}{1.506968in}}%
\pgfpathlineto{\pgfqpoint{1.716127in}{1.499689in}}%
\pgfpathlineto{\pgfqpoint{1.719233in}{1.499727in}}%
\pgfpathlineto{\pgfqpoint{1.722340in}{1.507955in}}%
\pgfpathlineto{\pgfqpoint{1.728552in}{1.549027in}}%
\pgfpathlineto{\pgfqpoint{1.737871in}{1.652582in}}%
\pgfpathlineto{\pgfqpoint{1.747189in}{1.745494in}}%
\pgfpathlineto{\pgfqpoint{1.750296in}{1.761087in}}%
\pgfpathlineto{\pgfqpoint{1.753402in}{1.765738in}}%
\pgfpathlineto{\pgfqpoint{1.756508in}{1.758703in}}%
\pgfpathlineto{\pgfqpoint{1.759614in}{1.740177in}}%
\pgfpathlineto{\pgfqpoint{1.765827in}{1.673821in}}%
\pgfpathlineto{\pgfqpoint{1.784464in}{1.418344in}}%
\pgfpathlineto{\pgfqpoint{1.790676in}{1.374360in}}%
\pgfpathlineto{\pgfqpoint{1.793783in}{1.364144in}}%
\pgfpathlineto{\pgfqpoint{1.796889in}{1.360976in}}%
\pgfpathlineto{\pgfqpoint{1.799995in}{1.363735in}}%
\pgfpathlineto{\pgfqpoint{1.806207in}{1.381641in}}%
\pgfpathlineto{\pgfqpoint{1.824845in}{1.462438in}}%
\pgfpathlineto{\pgfqpoint{1.834163in}{1.513782in}}%
\pgfpathlineto{\pgfqpoint{1.843482in}{1.592987in}}%
\pgfpathlineto{\pgfqpoint{1.855907in}{1.741171in}}%
\pgfpathlineto{\pgfqpoint{1.865225in}{1.847871in}}%
\pgfpathlineto{\pgfqpoint{1.871438in}{1.895151in}}%
\pgfpathlineto{\pgfqpoint{1.874544in}{1.908308in}}%
\pgfpathlineto{\pgfqpoint{1.877650in}{1.913773in}}%
\pgfpathlineto{\pgfqpoint{1.880757in}{1.911521in}}%
\pgfpathlineto{\pgfqpoint{1.883863in}{1.901916in}}%
\pgfpathlineto{\pgfqpoint{1.890075in}{1.863828in}}%
\pgfpathlineto{\pgfqpoint{1.902500in}{1.746700in}}%
\pgfpathlineto{\pgfqpoint{1.902500in}{1.746700in}}%
\pgfusepath{stroke}%
\end{pgfscope}%
\begin{pgfscope}%
\pgfpathrectangle{\pgfqpoint{0.275000in}{0.375000in}}{\pgfqpoint{1.705000in}{2.265000in}}%
\pgfusepath{clip}%
\pgfsetroundcap%
\pgfsetroundjoin%
\pgfsetlinewidth{1.505625pt}%
\definecolor{currentstroke}{rgb}{0.890196,0.466667,0.760784}%
\pgfsetstrokecolor{currentstroke}%
\pgfsetdash{}{0pt}%
\pgfpathmoveto{\pgfqpoint{0.352500in}{1.624705in}}%
\pgfpathlineto{\pgfqpoint{0.355606in}{1.626649in}}%
\pgfpathlineto{\pgfqpoint{0.361819in}{1.625855in}}%
\pgfpathlineto{\pgfqpoint{0.380456in}{1.619229in}}%
\pgfpathlineto{\pgfqpoint{0.386668in}{1.609910in}}%
\pgfpathlineto{\pgfqpoint{0.392881in}{1.589331in}}%
\pgfpathlineto{\pgfqpoint{0.399093in}{1.555171in}}%
\pgfpathlineto{\pgfqpoint{0.420837in}{1.403519in}}%
\pgfpathlineto{\pgfqpoint{0.423943in}{1.395990in}}%
\pgfpathlineto{\pgfqpoint{0.427049in}{1.396126in}}%
\pgfpathlineto{\pgfqpoint{0.430155in}{1.404463in}}%
\pgfpathlineto{\pgfqpoint{0.436368in}{1.444690in}}%
\pgfpathlineto{\pgfqpoint{0.458111in}{1.648636in}}%
\pgfpathlineto{\pgfqpoint{0.461217in}{1.656322in}}%
\pgfpathlineto{\pgfqpoint{0.464324in}{1.654548in}}%
\pgfpathlineto{\pgfqpoint{0.467430in}{1.643556in}}%
\pgfpathlineto{\pgfqpoint{0.473642in}{1.598077in}}%
\pgfpathlineto{\pgfqpoint{0.501598in}{1.330682in}}%
\pgfpathlineto{\pgfqpoint{0.523342in}{1.201624in}}%
\pgfpathlineto{\pgfqpoint{0.532660in}{1.116275in}}%
\pgfpathlineto{\pgfqpoint{0.563722in}{0.792913in}}%
\pgfpathlineto{\pgfqpoint{0.569935in}{0.760406in}}%
\pgfpathlineto{\pgfqpoint{0.573041in}{0.752613in}}%
\pgfpathlineto{\pgfqpoint{0.576147in}{0.751237in}}%
\pgfpathlineto{\pgfqpoint{0.579254in}{0.756758in}}%
\pgfpathlineto{\pgfqpoint{0.582360in}{0.769534in}}%
\pgfpathlineto{\pgfqpoint{0.588572in}{0.817425in}}%
\pgfpathlineto{\pgfqpoint{0.594785in}{0.893727in}}%
\pgfpathlineto{\pgfqpoint{0.607209in}{1.104377in}}%
\pgfpathlineto{\pgfqpoint{0.619634in}{1.312005in}}%
\pgfpathlineto{\pgfqpoint{0.625847in}{1.385026in}}%
\pgfpathlineto{\pgfqpoint{0.632059in}{1.430349in}}%
\pgfpathlineto{\pgfqpoint{0.638272in}{1.450066in}}%
\pgfpathlineto{\pgfqpoint{0.641378in}{1.452167in}}%
\pgfpathlineto{\pgfqpoint{0.644484in}{1.450333in}}%
\pgfpathlineto{\pgfqpoint{0.650696in}{1.438443in}}%
\pgfpathlineto{\pgfqpoint{0.660015in}{1.410217in}}%
\pgfpathlineto{\pgfqpoint{0.678652in}{1.341749in}}%
\pgfpathlineto{\pgfqpoint{0.691077in}{1.284780in}}%
\pgfpathlineto{\pgfqpoint{0.712821in}{1.179330in}}%
\pgfpathlineto{\pgfqpoint{0.719033in}{1.164836in}}%
\pgfpathlineto{\pgfqpoint{0.722139in}{1.162226in}}%
\pgfpathlineto{\pgfqpoint{0.725245in}{1.162610in}}%
\pgfpathlineto{\pgfqpoint{0.731458in}{1.170466in}}%
\pgfpathlineto{\pgfqpoint{0.740777in}{1.187005in}}%
\pgfpathlineto{\pgfqpoint{0.743883in}{1.189340in}}%
\pgfpathlineto{\pgfqpoint{0.746989in}{1.188126in}}%
\pgfpathlineto{\pgfqpoint{0.750095in}{1.182308in}}%
\pgfpathlineto{\pgfqpoint{0.753201in}{1.171012in}}%
\pgfpathlineto{\pgfqpoint{0.759414in}{1.129895in}}%
\pgfpathlineto{\pgfqpoint{0.765626in}{1.064428in}}%
\pgfpathlineto{\pgfqpoint{0.784264in}{0.829059in}}%
\pgfpathlineto{\pgfqpoint{0.787370in}{0.808542in}}%
\pgfpathlineto{\pgfqpoint{0.790476in}{0.799675in}}%
\pgfpathlineto{\pgfqpoint{0.793582in}{0.803827in}}%
\pgfpathlineto{\pgfqpoint{0.796688in}{0.821531in}}%
\pgfpathlineto{\pgfqpoint{0.802901in}{0.894981in}}%
\pgfpathlineto{\pgfqpoint{0.815326in}{1.127328in}}%
\pgfpathlineto{\pgfqpoint{0.824644in}{1.267537in}}%
\pgfpathlineto{\pgfqpoint{0.827751in}{1.291779in}}%
\pgfpathlineto{\pgfqpoint{0.830857in}{1.302569in}}%
\pgfpathlineto{\pgfqpoint{0.833963in}{1.300124in}}%
\pgfpathlineto{\pgfqpoint{0.837069in}{1.285661in}}%
\pgfpathlineto{\pgfqpoint{0.843282in}{1.229632in}}%
\pgfpathlineto{\pgfqpoint{0.855706in}{1.093320in}}%
\pgfpathlineto{\pgfqpoint{0.861919in}{1.056178in}}%
\pgfpathlineto{\pgfqpoint{0.865025in}{1.050457in}}%
\pgfpathlineto{\pgfqpoint{0.868131in}{1.053224in}}%
\pgfpathlineto{\pgfqpoint{0.871237in}{1.063513in}}%
\pgfpathlineto{\pgfqpoint{0.877450in}{1.100050in}}%
\pgfpathlineto{\pgfqpoint{0.889875in}{1.177715in}}%
\pgfpathlineto{\pgfqpoint{0.892981in}{1.186793in}}%
\pgfpathlineto{\pgfqpoint{0.896087in}{1.189238in}}%
\pgfpathlineto{\pgfqpoint{0.899193in}{1.184643in}}%
\pgfpathlineto{\pgfqpoint{0.902300in}{1.173082in}}%
\pgfpathlineto{\pgfqpoint{0.908512in}{1.131411in}}%
\pgfpathlineto{\pgfqpoint{0.917831in}{1.038341in}}%
\pgfpathlineto{\pgfqpoint{0.930256in}{0.911017in}}%
\pgfpathlineto{\pgfqpoint{0.936468in}{0.867699in}}%
\pgfpathlineto{\pgfqpoint{0.942680in}{0.848059in}}%
\pgfpathlineto{\pgfqpoint{0.945787in}{0.848552in}}%
\pgfpathlineto{\pgfqpoint{0.948893in}{0.856157in}}%
\pgfpathlineto{\pgfqpoint{0.955105in}{0.891314in}}%
\pgfpathlineto{\pgfqpoint{0.964424in}{0.978306in}}%
\pgfpathlineto{\pgfqpoint{0.973742in}{1.063628in}}%
\pgfpathlineto{\pgfqpoint{0.979955in}{1.092187in}}%
\pgfpathlineto{\pgfqpoint{0.983061in}{1.093269in}}%
\pgfpathlineto{\pgfqpoint{0.986167in}{1.084690in}}%
\pgfpathlineto{\pgfqpoint{0.992380in}{1.040200in}}%
\pgfpathlineto{\pgfqpoint{1.004805in}{0.890454in}}%
\pgfpathlineto{\pgfqpoint{1.011017in}{0.826383in}}%
\pgfpathlineto{\pgfqpoint{1.017229in}{0.794351in}}%
\pgfpathlineto{\pgfqpoint{1.020336in}{0.793110in}}%
\pgfpathlineto{\pgfqpoint{1.023442in}{0.801674in}}%
\pgfpathlineto{\pgfqpoint{1.029654in}{0.843899in}}%
\pgfpathlineto{\pgfqpoint{1.057610in}{1.105066in}}%
\pgfpathlineto{\pgfqpoint{1.079354in}{1.226358in}}%
\pgfpathlineto{\pgfqpoint{1.088672in}{1.321447in}}%
\pgfpathlineto{\pgfqpoint{1.104203in}{1.532401in}}%
\pgfpathlineto{\pgfqpoint{1.119734in}{1.728716in}}%
\pgfpathlineto{\pgfqpoint{1.129053in}{1.818683in}}%
\pgfpathlineto{\pgfqpoint{1.138372in}{1.880961in}}%
\pgfpathlineto{\pgfqpoint{1.144584in}{1.902330in}}%
\pgfpathlineto{\pgfqpoint{1.147690in}{1.905913in}}%
\pgfpathlineto{\pgfqpoint{1.150797in}{1.904493in}}%
\pgfpathlineto{\pgfqpoint{1.153903in}{1.898037in}}%
\pgfpathlineto{\pgfqpoint{1.160115in}{1.870557in}}%
\pgfpathlineto{\pgfqpoint{1.166328in}{1.825762in}}%
\pgfpathlineto{\pgfqpoint{1.178753in}{1.701816in}}%
\pgfpathlineto{\pgfqpoint{1.191177in}{1.581599in}}%
\pgfpathlineto{\pgfqpoint{1.197390in}{1.549317in}}%
\pgfpathlineto{\pgfqpoint{1.200496in}{1.544981in}}%
\pgfpathlineto{\pgfqpoint{1.203602in}{1.549858in}}%
\pgfpathlineto{\pgfqpoint{1.206708in}{1.564501in}}%
\pgfpathlineto{\pgfqpoint{1.212921in}{1.623047in}}%
\pgfpathlineto{\pgfqpoint{1.222239in}{1.772379in}}%
\pgfpathlineto{\pgfqpoint{1.250195in}{2.289565in}}%
\pgfpathlineto{\pgfqpoint{1.259514in}{2.384971in}}%
\pgfpathlineto{\pgfqpoint{1.265726in}{2.418299in}}%
\pgfpathlineto{\pgfqpoint{1.268833in}{2.425859in}}%
\pgfpathlineto{\pgfqpoint{1.271939in}{2.427443in}}%
\pgfpathlineto{\pgfqpoint{1.275045in}{2.423185in}}%
\pgfpathlineto{\pgfqpoint{1.278151in}{2.413296in}}%
\pgfpathlineto{\pgfqpoint{1.284364in}{2.377896in}}%
\pgfpathlineto{\pgfqpoint{1.293682in}{2.292929in}}%
\pgfpathlineto{\pgfqpoint{1.309213in}{2.106887in}}%
\pgfpathlineto{\pgfqpoint{1.330957in}{1.816433in}}%
\pgfpathlineto{\pgfqpoint{1.349594in}{1.511700in}}%
\pgfpathlineto{\pgfqpoint{1.368231in}{1.219582in}}%
\pgfpathlineto{\pgfqpoint{1.377550in}{1.110647in}}%
\pgfpathlineto{\pgfqpoint{1.386869in}{1.034113in}}%
\pgfpathlineto{\pgfqpoint{1.393081in}{1.002806in}}%
\pgfpathlineto{\pgfqpoint{1.399294in}{0.988209in}}%
\pgfpathlineto{\pgfqpoint{1.402400in}{0.987242in}}%
\pgfpathlineto{\pgfqpoint{1.405506in}{0.990413in}}%
\pgfpathlineto{\pgfqpoint{1.411718in}{1.008488in}}%
\pgfpathlineto{\pgfqpoint{1.417931in}{1.040370in}}%
\pgfpathlineto{\pgfqpoint{1.427249in}{1.106923in}}%
\pgfpathlineto{\pgfqpoint{1.448993in}{1.271850in}}%
\pgfpathlineto{\pgfqpoint{1.458312in}{1.318352in}}%
\pgfpathlineto{\pgfqpoint{1.476949in}{1.396478in}}%
\pgfpathlineto{\pgfqpoint{1.486268in}{1.455473in}}%
\pgfpathlineto{\pgfqpoint{1.504905in}{1.587063in}}%
\pgfpathlineto{\pgfqpoint{1.511117in}{1.612212in}}%
\pgfpathlineto{\pgfqpoint{1.514223in}{1.617447in}}%
\pgfpathlineto{\pgfqpoint{1.517330in}{1.616662in}}%
\pgfpathlineto{\pgfqpoint{1.520436in}{1.609041in}}%
\pgfpathlineto{\pgfqpoint{1.523542in}{1.593851in}}%
\pgfpathlineto{\pgfqpoint{1.529755in}{1.538750in}}%
\pgfpathlineto{\pgfqpoint{1.535967in}{1.450628in}}%
\pgfpathlineto{\pgfqpoint{1.560817in}{1.014983in}}%
\pgfpathlineto{\pgfqpoint{1.563923in}{0.993223in}}%
\pgfpathlineto{\pgfqpoint{1.567029in}{0.985872in}}%
\pgfpathlineto{\pgfqpoint{1.570135in}{0.993451in}}%
\pgfpathlineto{\pgfqpoint{1.573241in}{1.015670in}}%
\pgfpathlineto{\pgfqpoint{1.579454in}{1.099027in}}%
\pgfpathlineto{\pgfqpoint{1.588773in}{1.287503in}}%
\pgfpathlineto{\pgfqpoint{1.601197in}{1.541819in}}%
\pgfpathlineto{\pgfqpoint{1.607410in}{1.631560in}}%
\pgfpathlineto{\pgfqpoint{1.613622in}{1.688408in}}%
\pgfpathlineto{\pgfqpoint{1.619835in}{1.717145in}}%
\pgfpathlineto{\pgfqpoint{1.626047in}{1.727635in}}%
\pgfpathlineto{\pgfqpoint{1.647791in}{1.741714in}}%
\pgfpathlineto{\pgfqpoint{1.657109in}{1.751088in}}%
\pgfpathlineto{\pgfqpoint{1.663322in}{1.754357in}}%
\pgfpathlineto{\pgfqpoint{1.681959in}{1.759925in}}%
\pgfpathlineto{\pgfqpoint{1.694384in}{1.767531in}}%
\pgfpathlineto{\pgfqpoint{1.697490in}{1.766184in}}%
\pgfpathlineto{\pgfqpoint{1.700596in}{1.761771in}}%
\pgfpathlineto{\pgfqpoint{1.703702in}{1.753391in}}%
\pgfpathlineto{\pgfqpoint{1.709915in}{1.721917in}}%
\pgfpathlineto{\pgfqpoint{1.716127in}{1.668942in}}%
\pgfpathlineto{\pgfqpoint{1.725446in}{1.557204in}}%
\pgfpathlineto{\pgfqpoint{1.737871in}{1.406363in}}%
\pgfpathlineto{\pgfqpoint{1.744083in}{1.360921in}}%
\pgfpathlineto{\pgfqpoint{1.747189in}{1.349887in}}%
\pgfpathlineto{\pgfqpoint{1.750296in}{1.346917in}}%
\pgfpathlineto{\pgfqpoint{1.753402in}{1.351419in}}%
\pgfpathlineto{\pgfqpoint{1.759614in}{1.377429in}}%
\pgfpathlineto{\pgfqpoint{1.768933in}{1.427716in}}%
\pgfpathlineto{\pgfqpoint{1.772039in}{1.438364in}}%
\pgfpathlineto{\pgfqpoint{1.775145in}{1.442866in}}%
\pgfpathlineto{\pgfqpoint{1.778252in}{1.440223in}}%
\pgfpathlineto{\pgfqpoint{1.781358in}{1.430200in}}%
\pgfpathlineto{\pgfqpoint{1.787570in}{1.391015in}}%
\pgfpathlineto{\pgfqpoint{1.799995in}{1.290651in}}%
\pgfpathlineto{\pgfqpoint{1.806207in}{1.267185in}}%
\pgfpathlineto{\pgfqpoint{1.809314in}{1.268061in}}%
\pgfpathlineto{\pgfqpoint{1.812420in}{1.277811in}}%
\pgfpathlineto{\pgfqpoint{1.818632in}{1.320530in}}%
\pgfpathlineto{\pgfqpoint{1.834163in}{1.462418in}}%
\pgfpathlineto{\pgfqpoint{1.837270in}{1.475809in}}%
\pgfpathlineto{\pgfqpoint{1.840376in}{1.479072in}}%
\pgfpathlineto{\pgfqpoint{1.843482in}{1.471082in}}%
\pgfpathlineto{\pgfqpoint{1.846588in}{1.451446in}}%
\pgfpathlineto{\pgfqpoint{1.852801in}{1.379553in}}%
\pgfpathlineto{\pgfqpoint{1.865225in}{1.159560in}}%
\pgfpathlineto{\pgfqpoint{1.874544in}{1.017448in}}%
\pgfpathlineto{\pgfqpoint{1.880757in}{0.971470in}}%
\pgfpathlineto{\pgfqpoint{1.883863in}{0.966483in}}%
\pgfpathlineto{\pgfqpoint{1.886969in}{0.973185in}}%
\pgfpathlineto{\pgfqpoint{1.890075in}{0.990421in}}%
\pgfpathlineto{\pgfqpoint{1.896288in}{1.049061in}}%
\pgfpathlineto{\pgfqpoint{1.902500in}{1.124491in}}%
\pgfpathlineto{\pgfqpoint{1.902500in}{1.124491in}}%
\pgfusepath{stroke}%
\end{pgfscope}%
\begin{pgfscope}%
\pgfpathrectangle{\pgfqpoint{0.275000in}{0.375000in}}{\pgfqpoint{1.705000in}{2.265000in}}%
\pgfusepath{clip}%
\pgfsetroundcap%
\pgfsetroundjoin%
\pgfsetlinewidth{1.505625pt}%
\definecolor{currentstroke}{rgb}{0.498039,0.498039,0.498039}%
\pgfsetstrokecolor{currentstroke}%
\pgfsetdash{}{0pt}%
\pgfpathmoveto{\pgfqpoint{0.352500in}{1.635882in}}%
\pgfpathlineto{\pgfqpoint{0.361819in}{1.727936in}}%
\pgfpathlineto{\pgfqpoint{0.371137in}{1.792705in}}%
\pgfpathlineto{\pgfqpoint{0.380456in}{1.838231in}}%
\pgfpathlineto{\pgfqpoint{0.389775in}{1.873106in}}%
\pgfpathlineto{\pgfqpoint{0.395987in}{1.889089in}}%
\pgfpathlineto{\pgfqpoint{0.399093in}{1.893644in}}%
\pgfpathlineto{\pgfqpoint{0.402199in}{1.895323in}}%
\pgfpathlineto{\pgfqpoint{0.405306in}{1.893840in}}%
\pgfpathlineto{\pgfqpoint{0.408412in}{1.889118in}}%
\pgfpathlineto{\pgfqpoint{0.414624in}{1.870814in}}%
\pgfpathlineto{\pgfqpoint{0.427049in}{1.814149in}}%
\pgfpathlineto{\pgfqpoint{0.448793in}{1.703523in}}%
\pgfpathlineto{\pgfqpoint{0.458111in}{1.627367in}}%
\pgfpathlineto{\pgfqpoint{0.470536in}{1.488302in}}%
\pgfpathlineto{\pgfqpoint{0.486067in}{1.315295in}}%
\pgfpathlineto{\pgfqpoint{0.501598in}{1.182274in}}%
\pgfpathlineto{\pgfqpoint{0.510917in}{1.119823in}}%
\pgfpathlineto{\pgfqpoint{0.517129in}{1.091661in}}%
\pgfpathlineto{\pgfqpoint{0.520235in}{1.083472in}}%
\pgfpathlineto{\pgfqpoint{0.523342in}{1.079785in}}%
\pgfpathlineto{\pgfqpoint{0.526448in}{1.080739in}}%
\pgfpathlineto{\pgfqpoint{0.529554in}{1.086207in}}%
\pgfpathlineto{\pgfqpoint{0.535767in}{1.108965in}}%
\pgfpathlineto{\pgfqpoint{0.545085in}{1.162840in}}%
\pgfpathlineto{\pgfqpoint{0.560616in}{1.276003in}}%
\pgfpathlineto{\pgfqpoint{0.569935in}{1.366973in}}%
\pgfpathlineto{\pgfqpoint{0.579254in}{1.491286in}}%
\pgfpathlineto{\pgfqpoint{0.600997in}{1.819244in}}%
\pgfpathlineto{\pgfqpoint{0.607209in}{1.869618in}}%
\pgfpathlineto{\pgfqpoint{0.610316in}{1.881690in}}%
\pgfpathlineto{\pgfqpoint{0.613422in}{1.884763in}}%
\pgfpathlineto{\pgfqpoint{0.616528in}{1.879124in}}%
\pgfpathlineto{\pgfqpoint{0.619634in}{1.865345in}}%
\pgfpathlineto{\pgfqpoint{0.625847in}{1.816458in}}%
\pgfpathlineto{\pgfqpoint{0.635165in}{1.701263in}}%
\pgfpathlineto{\pgfqpoint{0.647590in}{1.489149in}}%
\pgfpathlineto{\pgfqpoint{0.675546in}{0.935023in}}%
\pgfpathlineto{\pgfqpoint{0.681759in}{0.881679in}}%
\pgfpathlineto{\pgfqpoint{0.684865in}{0.876832in}}%
\pgfpathlineto{\pgfqpoint{0.687971in}{0.887990in}}%
\pgfpathlineto{\pgfqpoint{0.691077in}{0.915349in}}%
\pgfpathlineto{\pgfqpoint{0.697290in}{1.015445in}}%
\pgfpathlineto{\pgfqpoint{0.706608in}{1.247693in}}%
\pgfpathlineto{\pgfqpoint{0.722139in}{1.643209in}}%
\pgfpathlineto{\pgfqpoint{0.728352in}{1.737147in}}%
\pgfpathlineto{\pgfqpoint{0.734564in}{1.778002in}}%
\pgfpathlineto{\pgfqpoint{0.737670in}{1.779269in}}%
\pgfpathlineto{\pgfqpoint{0.740777in}{1.769370in}}%
\pgfpathlineto{\pgfqpoint{0.746989in}{1.723670in}}%
\pgfpathlineto{\pgfqpoint{0.765626in}{1.547069in}}%
\pgfpathlineto{\pgfqpoint{0.768732in}{1.533510in}}%
\pgfpathlineto{\pgfqpoint{0.771839in}{1.527624in}}%
\pgfpathlineto{\pgfqpoint{0.774945in}{1.529433in}}%
\pgfpathlineto{\pgfqpoint{0.778051in}{1.538431in}}%
\pgfpathlineto{\pgfqpoint{0.784264in}{1.573725in}}%
\pgfpathlineto{\pgfqpoint{0.802901in}{1.708787in}}%
\pgfpathlineto{\pgfqpoint{0.809113in}{1.731455in}}%
\pgfpathlineto{\pgfqpoint{0.812219in}{1.736686in}}%
\pgfpathlineto{\pgfqpoint{0.815326in}{1.738156in}}%
\pgfpathlineto{\pgfqpoint{0.818432in}{1.736224in}}%
\pgfpathlineto{\pgfqpoint{0.824644in}{1.723489in}}%
\pgfpathlineto{\pgfqpoint{0.830857in}{1.700205in}}%
\pgfpathlineto{\pgfqpoint{0.837069in}{1.666383in}}%
\pgfpathlineto{\pgfqpoint{0.846388in}{1.594317in}}%
\pgfpathlineto{\pgfqpoint{0.858813in}{1.463928in}}%
\pgfpathlineto{\pgfqpoint{0.874344in}{1.299099in}}%
\pgfpathlineto{\pgfqpoint{0.880556in}{1.255794in}}%
\pgfpathlineto{\pgfqpoint{0.886769in}{1.236277in}}%
\pgfpathlineto{\pgfqpoint{0.889875in}{1.236908in}}%
\pgfpathlineto{\pgfqpoint{0.892981in}{1.244719in}}%
\pgfpathlineto{\pgfqpoint{0.899193in}{1.280566in}}%
\pgfpathlineto{\pgfqpoint{0.908512in}{1.370652in}}%
\pgfpathlineto{\pgfqpoint{0.920937in}{1.497588in}}%
\pgfpathlineto{\pgfqpoint{0.930256in}{1.558767in}}%
\pgfpathlineto{\pgfqpoint{0.948893in}{1.643277in}}%
\pgfpathlineto{\pgfqpoint{0.958211in}{1.721848in}}%
\pgfpathlineto{\pgfqpoint{0.970636in}{1.831454in}}%
\pgfpathlineto{\pgfqpoint{0.973742in}{1.845755in}}%
\pgfpathlineto{\pgfqpoint{0.976849in}{1.850849in}}%
\pgfpathlineto{\pgfqpoint{0.979955in}{1.845661in}}%
\pgfpathlineto{\pgfqpoint{0.983061in}{1.829899in}}%
\pgfpathlineto{\pgfqpoint{0.989274in}{1.769599in}}%
\pgfpathlineto{\pgfqpoint{1.011017in}{1.484193in}}%
\pgfpathlineto{\pgfqpoint{1.017229in}{1.454546in}}%
\pgfpathlineto{\pgfqpoint{1.020336in}{1.453541in}}%
\pgfpathlineto{\pgfqpoint{1.023442in}{1.461121in}}%
\pgfpathlineto{\pgfqpoint{1.029654in}{1.497585in}}%
\pgfpathlineto{\pgfqpoint{1.054504in}{1.699160in}}%
\pgfpathlineto{\pgfqpoint{1.057610in}{1.709034in}}%
\pgfpathlineto{\pgfqpoint{1.060716in}{1.713518in}}%
\pgfpathlineto{\pgfqpoint{1.063823in}{1.712788in}}%
\pgfpathlineto{\pgfqpoint{1.066929in}{1.707216in}}%
\pgfpathlineto{\pgfqpoint{1.073141in}{1.683588in}}%
\pgfpathlineto{\pgfqpoint{1.082460in}{1.624409in}}%
\pgfpathlineto{\pgfqpoint{1.091779in}{1.543446in}}%
\pgfpathlineto{\pgfqpoint{1.104203in}{1.403469in}}%
\pgfpathlineto{\pgfqpoint{1.132159in}{1.058715in}}%
\pgfpathlineto{\pgfqpoint{1.160115in}{0.774783in}}%
\pgfpathlineto{\pgfqpoint{1.181859in}{0.506044in}}%
\pgfpathlineto{\pgfqpoint{1.184965in}{0.486921in}}%
\pgfpathlineto{\pgfqpoint{1.188071in}{0.477955in}}%
\pgfpathlineto{\pgfqpoint{1.191177in}{0.480402in}}%
\pgfpathlineto{\pgfqpoint{1.194284in}{0.495013in}}%
\pgfpathlineto{\pgfqpoint{1.200496in}{0.560754in}}%
\pgfpathlineto{\pgfqpoint{1.206708in}{0.669102in}}%
\pgfpathlineto{\pgfqpoint{1.237771in}{1.309841in}}%
\pgfpathlineto{\pgfqpoint{1.250195in}{1.454349in}}%
\pgfpathlineto{\pgfqpoint{1.268833in}{1.625065in}}%
\pgfpathlineto{\pgfqpoint{1.275045in}{1.658552in}}%
\pgfpathlineto{\pgfqpoint{1.278151in}{1.664022in}}%
\pgfpathlineto{\pgfqpoint{1.281258in}{1.659616in}}%
\pgfpathlineto{\pgfqpoint{1.284364in}{1.643954in}}%
\pgfpathlineto{\pgfqpoint{1.290576in}{1.576218in}}%
\pgfpathlineto{\pgfqpoint{1.296789in}{1.462948in}}%
\pgfpathlineto{\pgfqpoint{1.321638in}{0.938366in}}%
\pgfpathlineto{\pgfqpoint{1.327851in}{0.891972in}}%
\pgfpathlineto{\pgfqpoint{1.330957in}{0.886460in}}%
\pgfpathlineto{\pgfqpoint{1.334063in}{0.891258in}}%
\pgfpathlineto{\pgfqpoint{1.337169in}{0.904786in}}%
\pgfpathlineto{\pgfqpoint{1.343382in}{0.951366in}}%
\pgfpathlineto{\pgfqpoint{1.352700in}{1.050842in}}%
\pgfpathlineto{\pgfqpoint{1.365125in}{1.222541in}}%
\pgfpathlineto{\pgfqpoint{1.393081in}{1.673437in}}%
\pgfpathlineto{\pgfqpoint{1.399294in}{1.712331in}}%
\pgfpathlineto{\pgfqpoint{1.402400in}{1.716902in}}%
\pgfpathlineto{\pgfqpoint{1.405506in}{1.712170in}}%
\pgfpathlineto{\pgfqpoint{1.408612in}{1.699246in}}%
\pgfpathlineto{\pgfqpoint{1.414825in}{1.655073in}}%
\pgfpathlineto{\pgfqpoint{1.433462in}{1.500801in}}%
\pgfpathlineto{\pgfqpoint{1.439674in}{1.475823in}}%
\pgfpathlineto{\pgfqpoint{1.442781in}{1.470431in}}%
\pgfpathlineto{\pgfqpoint{1.445887in}{1.469451in}}%
\pgfpathlineto{\pgfqpoint{1.448993in}{1.472389in}}%
\pgfpathlineto{\pgfqpoint{1.455205in}{1.487419in}}%
\pgfpathlineto{\pgfqpoint{1.467630in}{1.534747in}}%
\pgfpathlineto{\pgfqpoint{1.480055in}{1.579178in}}%
\pgfpathlineto{\pgfqpoint{1.489374in}{1.603144in}}%
\pgfpathlineto{\pgfqpoint{1.495586in}{1.612156in}}%
\pgfpathlineto{\pgfqpoint{1.498692in}{1.613416in}}%
\pgfpathlineto{\pgfqpoint{1.501799in}{1.611780in}}%
\pgfpathlineto{\pgfqpoint{1.504905in}{1.606667in}}%
\pgfpathlineto{\pgfqpoint{1.511117in}{1.584086in}}%
\pgfpathlineto{\pgfqpoint{1.517330in}{1.543716in}}%
\pgfpathlineto{\pgfqpoint{1.526648in}{1.455929in}}%
\pgfpathlineto{\pgfqpoint{1.542179in}{1.300782in}}%
\pgfpathlineto{\pgfqpoint{1.551498in}{1.236520in}}%
\pgfpathlineto{\pgfqpoint{1.579454in}{1.083258in}}%
\pgfpathlineto{\pgfqpoint{1.604304in}{0.916901in}}%
\pgfpathlineto{\pgfqpoint{1.610516in}{0.892597in}}%
\pgfpathlineto{\pgfqpoint{1.613622in}{0.886180in}}%
\pgfpathlineto{\pgfqpoint{1.616728in}{0.884531in}}%
\pgfpathlineto{\pgfqpoint{1.619835in}{0.888305in}}%
\pgfpathlineto{\pgfqpoint{1.622941in}{0.898010in}}%
\pgfpathlineto{\pgfqpoint{1.629153in}{0.935970in}}%
\pgfpathlineto{\pgfqpoint{1.638472in}{1.032411in}}%
\pgfpathlineto{\pgfqpoint{1.650897in}{1.176564in}}%
\pgfpathlineto{\pgfqpoint{1.657109in}{1.222161in}}%
\pgfpathlineto{\pgfqpoint{1.660215in}{1.233341in}}%
\pgfpathlineto{\pgfqpoint{1.663322in}{1.236058in}}%
\pgfpathlineto{\pgfqpoint{1.666428in}{1.230454in}}%
\pgfpathlineto{\pgfqpoint{1.672640in}{1.197290in}}%
\pgfpathlineto{\pgfqpoint{1.681959in}{1.112092in}}%
\pgfpathlineto{\pgfqpoint{1.697490in}{0.965217in}}%
\pgfpathlineto{\pgfqpoint{1.703702in}{0.925908in}}%
\pgfpathlineto{\pgfqpoint{1.709915in}{0.905196in}}%
\pgfpathlineto{\pgfqpoint{1.713021in}{0.903580in}}%
\pgfpathlineto{\pgfqpoint{1.716127in}{0.908815in}}%
\pgfpathlineto{\pgfqpoint{1.719233in}{0.921613in}}%
\pgfpathlineto{\pgfqpoint{1.725446in}{0.971830in}}%
\pgfpathlineto{\pgfqpoint{1.731658in}{1.055049in}}%
\pgfpathlineto{\pgfqpoint{1.740977in}{1.228834in}}%
\pgfpathlineto{\pgfqpoint{1.759614in}{1.590401in}}%
\pgfpathlineto{\pgfqpoint{1.765827in}{1.662787in}}%
\pgfpathlineto{\pgfqpoint{1.768933in}{1.683974in}}%
\pgfpathlineto{\pgfqpoint{1.772039in}{1.694324in}}%
\pgfpathlineto{\pgfqpoint{1.775145in}{1.693683in}}%
\pgfpathlineto{\pgfqpoint{1.778252in}{1.682318in}}%
\pgfpathlineto{\pgfqpoint{1.784464in}{1.630731in}}%
\pgfpathlineto{\pgfqpoint{1.793783in}{1.504688in}}%
\pgfpathlineto{\pgfqpoint{1.806207in}{1.337787in}}%
\pgfpathlineto{\pgfqpoint{1.812420in}{1.291698in}}%
\pgfpathlineto{\pgfqpoint{1.815526in}{1.281919in}}%
\pgfpathlineto{\pgfqpoint{1.818632in}{1.280946in}}%
\pgfpathlineto{\pgfqpoint{1.821738in}{1.287992in}}%
\pgfpathlineto{\pgfqpoint{1.827951in}{1.320379in}}%
\pgfpathlineto{\pgfqpoint{1.840376in}{1.402447in}}%
\pgfpathlineto{\pgfqpoint{1.846588in}{1.422240in}}%
\pgfpathlineto{\pgfqpoint{1.849694in}{1.423120in}}%
\pgfpathlineto{\pgfqpoint{1.852801in}{1.417987in}}%
\pgfpathlineto{\pgfqpoint{1.859013in}{1.392517in}}%
\pgfpathlineto{\pgfqpoint{1.871438in}{1.312121in}}%
\pgfpathlineto{\pgfqpoint{1.902500in}{1.101311in}}%
\pgfpathlineto{\pgfqpoint{1.902500in}{1.101311in}}%
\pgfusepath{stroke}%
\end{pgfscope}%
\begin{pgfscope}%
\pgfpathrectangle{\pgfqpoint{0.275000in}{0.375000in}}{\pgfqpoint{1.705000in}{2.265000in}}%
\pgfusepath{clip}%
\pgfsetroundcap%
\pgfsetroundjoin%
\pgfsetlinewidth{1.505625pt}%
\definecolor{currentstroke}{rgb}{0.737255,0.741176,0.133333}%
\pgfsetstrokecolor{currentstroke}%
\pgfsetdash{}{0pt}%
\pgfpathmoveto{\pgfqpoint{0.352500in}{1.781489in}}%
\pgfpathlineto{\pgfqpoint{0.358712in}{1.765328in}}%
\pgfpathlineto{\pgfqpoint{0.364925in}{1.733216in}}%
\pgfpathlineto{\pgfqpoint{0.371137in}{1.682817in}}%
\pgfpathlineto{\pgfqpoint{0.383562in}{1.545407in}}%
\pgfpathlineto{\pgfqpoint{0.395987in}{1.419676in}}%
\pgfpathlineto{\pgfqpoint{0.402199in}{1.379923in}}%
\pgfpathlineto{\pgfqpoint{0.408412in}{1.358260in}}%
\pgfpathlineto{\pgfqpoint{0.411518in}{1.353541in}}%
\pgfpathlineto{\pgfqpoint{0.414624in}{1.352229in}}%
\pgfpathlineto{\pgfqpoint{0.417730in}{1.353703in}}%
\pgfpathlineto{\pgfqpoint{0.423943in}{1.362336in}}%
\pgfpathlineto{\pgfqpoint{0.436368in}{1.384335in}}%
\pgfpathlineto{\pgfqpoint{0.442580in}{1.389179in}}%
\pgfpathlineto{\pgfqpoint{0.445686in}{1.388643in}}%
\pgfpathlineto{\pgfqpoint{0.448793in}{1.385674in}}%
\pgfpathlineto{\pgfqpoint{0.455005in}{1.370962in}}%
\pgfpathlineto{\pgfqpoint{0.461217in}{1.341784in}}%
\pgfpathlineto{\pgfqpoint{0.467430in}{1.295025in}}%
\pgfpathlineto{\pgfqpoint{0.476748in}{1.191123in}}%
\pgfpathlineto{\pgfqpoint{0.492280in}{0.993069in}}%
\pgfpathlineto{\pgfqpoint{0.498492in}{0.947882in}}%
\pgfpathlineto{\pgfqpoint{0.501598in}{0.939292in}}%
\pgfpathlineto{\pgfqpoint{0.504704in}{0.940810in}}%
\pgfpathlineto{\pgfqpoint{0.507811in}{0.952089in}}%
\pgfpathlineto{\pgfqpoint{0.514023in}{0.998927in}}%
\pgfpathlineto{\pgfqpoint{0.532660in}{1.179404in}}%
\pgfpathlineto{\pgfqpoint{0.538873in}{1.209117in}}%
\pgfpathlineto{\pgfqpoint{0.548191in}{1.231469in}}%
\pgfpathlineto{\pgfqpoint{0.554404in}{1.251149in}}%
\pgfpathlineto{\pgfqpoint{0.560616in}{1.287827in}}%
\pgfpathlineto{\pgfqpoint{0.569935in}{1.378453in}}%
\pgfpathlineto{\pgfqpoint{0.585466in}{1.551195in}}%
\pgfpathlineto{\pgfqpoint{0.591678in}{1.594110in}}%
\pgfpathlineto{\pgfqpoint{0.594785in}{1.605855in}}%
\pgfpathlineto{\pgfqpoint{0.597891in}{1.610533in}}%
\pgfpathlineto{\pgfqpoint{0.600997in}{1.607998in}}%
\pgfpathlineto{\pgfqpoint{0.604103in}{1.598321in}}%
\pgfpathlineto{\pgfqpoint{0.610316in}{1.558770in}}%
\pgfpathlineto{\pgfqpoint{0.619634in}{1.458547in}}%
\pgfpathlineto{\pgfqpoint{0.638272in}{1.234049in}}%
\pgfpathlineto{\pgfqpoint{0.644484in}{1.202577in}}%
\pgfpathlineto{\pgfqpoint{0.647590in}{1.202164in}}%
\pgfpathlineto{\pgfqpoint{0.650696in}{1.213023in}}%
\pgfpathlineto{\pgfqpoint{0.656909in}{1.268317in}}%
\pgfpathlineto{\pgfqpoint{0.663121in}{1.361793in}}%
\pgfpathlineto{\pgfqpoint{0.684865in}{1.736262in}}%
\pgfpathlineto{\pgfqpoint{0.691077in}{1.784811in}}%
\pgfpathlineto{\pgfqpoint{0.694183in}{1.793774in}}%
\pgfpathlineto{\pgfqpoint{0.697290in}{1.792774in}}%
\pgfpathlineto{\pgfqpoint{0.700396in}{1.782473in}}%
\pgfpathlineto{\pgfqpoint{0.706608in}{1.738006in}}%
\pgfpathlineto{\pgfqpoint{0.715927in}{1.631018in}}%
\pgfpathlineto{\pgfqpoint{0.734564in}{1.400276in}}%
\pgfpathlineto{\pgfqpoint{0.746989in}{1.292387in}}%
\pgfpathlineto{\pgfqpoint{0.771839in}{1.120656in}}%
\pgfpathlineto{\pgfqpoint{0.778051in}{1.089448in}}%
\pgfpathlineto{\pgfqpoint{0.784264in}{1.071990in}}%
\pgfpathlineto{\pgfqpoint{0.787370in}{1.069142in}}%
\pgfpathlineto{\pgfqpoint{0.790476in}{1.070016in}}%
\pgfpathlineto{\pgfqpoint{0.796688in}{1.080434in}}%
\pgfpathlineto{\pgfqpoint{0.806007in}{1.102440in}}%
\pgfpathlineto{\pgfqpoint{0.809113in}{1.106892in}}%
\pgfpathlineto{\pgfqpoint{0.812219in}{1.108323in}}%
\pgfpathlineto{\pgfqpoint{0.815326in}{1.106214in}}%
\pgfpathlineto{\pgfqpoint{0.818432in}{1.100432in}}%
\pgfpathlineto{\pgfqpoint{0.824644in}{1.079321in}}%
\pgfpathlineto{\pgfqpoint{0.837069in}{1.026337in}}%
\pgfpathlineto{\pgfqpoint{0.843282in}{1.014303in}}%
\pgfpathlineto{\pgfqpoint{0.846388in}{1.015639in}}%
\pgfpathlineto{\pgfqpoint{0.849494in}{1.022768in}}%
\pgfpathlineto{\pgfqpoint{0.855706in}{1.055733in}}%
\pgfpathlineto{\pgfqpoint{0.861919in}{1.113847in}}%
\pgfpathlineto{\pgfqpoint{0.871237in}{1.241979in}}%
\pgfpathlineto{\pgfqpoint{0.892981in}{1.573986in}}%
\pgfpathlineto{\pgfqpoint{0.899193in}{1.619068in}}%
\pgfpathlineto{\pgfqpoint{0.902300in}{1.627325in}}%
\pgfpathlineto{\pgfqpoint{0.905406in}{1.626132in}}%
\pgfpathlineto{\pgfqpoint{0.908512in}{1.616321in}}%
\pgfpathlineto{\pgfqpoint{0.914724in}{1.577001in}}%
\pgfpathlineto{\pgfqpoint{0.927149in}{1.481990in}}%
\pgfpathlineto{\pgfqpoint{0.930256in}{1.468231in}}%
\pgfpathlineto{\pgfqpoint{0.933362in}{1.461922in}}%
\pgfpathlineto{\pgfqpoint{0.936468in}{1.463850in}}%
\pgfpathlineto{\pgfqpoint{0.939574in}{1.474173in}}%
\pgfpathlineto{\pgfqpoint{0.945787in}{1.517707in}}%
\pgfpathlineto{\pgfqpoint{0.955105in}{1.620946in}}%
\pgfpathlineto{\pgfqpoint{0.970636in}{1.799407in}}%
\pgfpathlineto{\pgfqpoint{0.983061in}{1.904057in}}%
\pgfpathlineto{\pgfqpoint{0.995486in}{1.985213in}}%
\pgfpathlineto{\pgfqpoint{1.004805in}{2.032216in}}%
\pgfpathlineto{\pgfqpoint{1.011017in}{2.053324in}}%
\pgfpathlineto{\pgfqpoint{1.017229in}{2.063422in}}%
\pgfpathlineto{\pgfqpoint{1.020336in}{2.063566in}}%
\pgfpathlineto{\pgfqpoint{1.023442in}{2.059951in}}%
\pgfpathlineto{\pgfqpoint{1.026548in}{2.052118in}}%
\pgfpathlineto{\pgfqpoint{1.032761in}{2.021510in}}%
\pgfpathlineto{\pgfqpoint{1.038973in}{1.966507in}}%
\pgfpathlineto{\pgfqpoint{1.045185in}{1.882129in}}%
\pgfpathlineto{\pgfqpoint{1.054504in}{1.698132in}}%
\pgfpathlineto{\pgfqpoint{1.076247in}{1.203614in}}%
\pgfpathlineto{\pgfqpoint{1.082460in}{1.142199in}}%
\pgfpathlineto{\pgfqpoint{1.085566in}{1.136952in}}%
\pgfpathlineto{\pgfqpoint{1.088672in}{1.149691in}}%
\pgfpathlineto{\pgfqpoint{1.091779in}{1.180004in}}%
\pgfpathlineto{\pgfqpoint{1.097991in}{1.286878in}}%
\pgfpathlineto{\pgfqpoint{1.122841in}{1.823957in}}%
\pgfpathlineto{\pgfqpoint{1.125947in}{1.844184in}}%
\pgfpathlineto{\pgfqpoint{1.129053in}{1.847431in}}%
\pgfpathlineto{\pgfqpoint{1.132159in}{1.834281in}}%
\pgfpathlineto{\pgfqpoint{1.138372in}{1.765729in}}%
\pgfpathlineto{\pgfqpoint{1.150797in}{1.541776in}}%
\pgfpathlineto{\pgfqpoint{1.160115in}{1.397458in}}%
\pgfpathlineto{\pgfqpoint{1.166328in}{1.343593in}}%
\pgfpathlineto{\pgfqpoint{1.169434in}{1.330762in}}%
\pgfpathlineto{\pgfqpoint{1.172540in}{1.326635in}}%
\pgfpathlineto{\pgfqpoint{1.175646in}{1.330282in}}%
\pgfpathlineto{\pgfqpoint{1.178753in}{1.340672in}}%
\pgfpathlineto{\pgfqpoint{1.184965in}{1.378150in}}%
\pgfpathlineto{\pgfqpoint{1.191177in}{1.434963in}}%
\pgfpathlineto{\pgfqpoint{1.197390in}{1.512583in}}%
\pgfpathlineto{\pgfqpoint{1.206708in}{1.679269in}}%
\pgfpathlineto{\pgfqpoint{1.216027in}{1.911286in}}%
\pgfpathlineto{\pgfqpoint{1.234664in}{2.404168in}}%
\pgfpathlineto{\pgfqpoint{1.240877in}{2.499808in}}%
\pgfpathlineto{\pgfqpoint{1.243983in}{2.525906in}}%
\pgfpathlineto{\pgfqpoint{1.247089in}{2.537045in}}%
\pgfpathlineto{\pgfqpoint{1.250195in}{2.533794in}}%
\pgfpathlineto{\pgfqpoint{1.253302in}{2.517360in}}%
\pgfpathlineto{\pgfqpoint{1.259514in}{2.452153in}}%
\pgfpathlineto{\pgfqpoint{1.271939in}{2.253590in}}%
\pgfpathlineto{\pgfqpoint{1.284364in}{2.068272in}}%
\pgfpathlineto{\pgfqpoint{1.290576in}{2.006919in}}%
\pgfpathlineto{\pgfqpoint{1.296789in}{1.973582in}}%
\pgfpathlineto{\pgfqpoint{1.299895in}{1.967965in}}%
\pgfpathlineto{\pgfqpoint{1.303001in}{1.969576in}}%
\pgfpathlineto{\pgfqpoint{1.306107in}{1.978023in}}%
\pgfpathlineto{\pgfqpoint{1.312320in}{2.012733in}}%
\pgfpathlineto{\pgfqpoint{1.321638in}{2.094558in}}%
\pgfpathlineto{\pgfqpoint{1.337169in}{2.234243in}}%
\pgfpathlineto{\pgfqpoint{1.343382in}{2.268911in}}%
\pgfpathlineto{\pgfqpoint{1.346488in}{2.278960in}}%
\pgfpathlineto{\pgfqpoint{1.349594in}{2.283430in}}%
\pgfpathlineto{\pgfqpoint{1.352700in}{2.281820in}}%
\pgfpathlineto{\pgfqpoint{1.355807in}{2.273671in}}%
\pgfpathlineto{\pgfqpoint{1.362019in}{2.236342in}}%
\pgfpathlineto{\pgfqpoint{1.368231in}{2.170140in}}%
\pgfpathlineto{\pgfqpoint{1.377550in}{2.022397in}}%
\pgfpathlineto{\pgfqpoint{1.417931in}{1.285048in}}%
\pgfpathlineto{\pgfqpoint{1.427249in}{1.193142in}}%
\pgfpathlineto{\pgfqpoint{1.433462in}{1.165162in}}%
\pgfpathlineto{\pgfqpoint{1.436568in}{1.163255in}}%
\pgfpathlineto{\pgfqpoint{1.439674in}{1.170003in}}%
\pgfpathlineto{\pgfqpoint{1.442781in}{1.185538in}}%
\pgfpathlineto{\pgfqpoint{1.448993in}{1.241925in}}%
\pgfpathlineto{\pgfqpoint{1.458312in}{1.377731in}}%
\pgfpathlineto{\pgfqpoint{1.486268in}{1.840962in}}%
\pgfpathlineto{\pgfqpoint{1.495586in}{1.937026in}}%
\pgfpathlineto{\pgfqpoint{1.501799in}{1.973671in}}%
\pgfpathlineto{\pgfqpoint{1.504905in}{1.981697in}}%
\pgfpathlineto{\pgfqpoint{1.508011in}{1.981844in}}%
\pgfpathlineto{\pgfqpoint{1.511117in}{1.973512in}}%
\pgfpathlineto{\pgfqpoint{1.514223in}{1.956400in}}%
\pgfpathlineto{\pgfqpoint{1.520436in}{1.896900in}}%
\pgfpathlineto{\pgfqpoint{1.532861in}{1.715087in}}%
\pgfpathlineto{\pgfqpoint{1.542179in}{1.597015in}}%
\pgfpathlineto{\pgfqpoint{1.548392in}{1.558838in}}%
\pgfpathlineto{\pgfqpoint{1.551498in}{1.553959in}}%
\pgfpathlineto{\pgfqpoint{1.554604in}{1.557560in}}%
\pgfpathlineto{\pgfqpoint{1.560817in}{1.583086in}}%
\pgfpathlineto{\pgfqpoint{1.570135in}{1.632006in}}%
\pgfpathlineto{\pgfqpoint{1.573241in}{1.642075in}}%
\pgfpathlineto{\pgfqpoint{1.576348in}{1.646238in}}%
\pgfpathlineto{\pgfqpoint{1.579454in}{1.643572in}}%
\pgfpathlineto{\pgfqpoint{1.582560in}{1.633654in}}%
\pgfpathlineto{\pgfqpoint{1.588773in}{1.592444in}}%
\pgfpathlineto{\pgfqpoint{1.594985in}{1.526140in}}%
\pgfpathlineto{\pgfqpoint{1.604304in}{1.392911in}}%
\pgfpathlineto{\pgfqpoint{1.629153in}{1.012936in}}%
\pgfpathlineto{\pgfqpoint{1.635366in}{0.956515in}}%
\pgfpathlineto{\pgfqpoint{1.641578in}{0.928254in}}%
\pgfpathlineto{\pgfqpoint{1.644684in}{0.924966in}}%
\pgfpathlineto{\pgfqpoint{1.647791in}{0.928465in}}%
\pgfpathlineto{\pgfqpoint{1.650897in}{0.938071in}}%
\pgfpathlineto{\pgfqpoint{1.657109in}{0.971778in}}%
\pgfpathlineto{\pgfqpoint{1.681959in}{1.138836in}}%
\pgfpathlineto{\pgfqpoint{1.691278in}{1.171690in}}%
\pgfpathlineto{\pgfqpoint{1.700596in}{1.202239in}}%
\pgfpathlineto{\pgfqpoint{1.706809in}{1.232520in}}%
\pgfpathlineto{\pgfqpoint{1.716127in}{1.300070in}}%
\pgfpathlineto{\pgfqpoint{1.731658in}{1.425394in}}%
\pgfpathlineto{\pgfqpoint{1.737871in}{1.449999in}}%
\pgfpathlineto{\pgfqpoint{1.740977in}{1.452547in}}%
\pgfpathlineto{\pgfqpoint{1.744083in}{1.448163in}}%
\pgfpathlineto{\pgfqpoint{1.747189in}{1.437086in}}%
\pgfpathlineto{\pgfqpoint{1.753402in}{1.398011in}}%
\pgfpathlineto{\pgfqpoint{1.772039in}{1.255361in}}%
\pgfpathlineto{\pgfqpoint{1.775145in}{1.245960in}}%
\pgfpathlineto{\pgfqpoint{1.778252in}{1.244143in}}%
\pgfpathlineto{\pgfqpoint{1.781358in}{1.250333in}}%
\pgfpathlineto{\pgfqpoint{1.784464in}{1.264421in}}%
\pgfpathlineto{\pgfqpoint{1.790676in}{1.313188in}}%
\pgfpathlineto{\pgfqpoint{1.809314in}{1.500586in}}%
\pgfpathlineto{\pgfqpoint{1.815526in}{1.526206in}}%
\pgfpathlineto{\pgfqpoint{1.818632in}{1.526938in}}%
\pgfpathlineto{\pgfqpoint{1.821738in}{1.519771in}}%
\pgfpathlineto{\pgfqpoint{1.827951in}{1.485338in}}%
\pgfpathlineto{\pgfqpoint{1.849694in}{1.319270in}}%
\pgfpathlineto{\pgfqpoint{1.855907in}{1.301525in}}%
\pgfpathlineto{\pgfqpoint{1.859013in}{1.298986in}}%
\pgfpathlineto{\pgfqpoint{1.862119in}{1.299583in}}%
\pgfpathlineto{\pgfqpoint{1.874544in}{1.309238in}}%
\pgfpathlineto{\pgfqpoint{1.877650in}{1.307439in}}%
\pgfpathlineto{\pgfqpoint{1.880757in}{1.302083in}}%
\pgfpathlineto{\pgfqpoint{1.886969in}{1.279134in}}%
\pgfpathlineto{\pgfqpoint{1.893181in}{1.240378in}}%
\pgfpathlineto{\pgfqpoint{1.902500in}{1.163141in}}%
\pgfpathlineto{\pgfqpoint{1.902500in}{1.163141in}}%
\pgfusepath{stroke}%
\end{pgfscope}%
\begin{pgfscope}%
\pgfpathrectangle{\pgfqpoint{0.275000in}{0.375000in}}{\pgfqpoint{1.705000in}{2.265000in}}%
\pgfusepath{clip}%
\pgfsetroundcap%
\pgfsetroundjoin%
\pgfsetlinewidth{1.505625pt}%
\definecolor{currentstroke}{rgb}{0.090196,0.745098,0.811765}%
\pgfsetstrokecolor{currentstroke}%
\pgfsetdash{}{0pt}%
\pgfpathmoveto{\pgfqpoint{0.352500in}{0.878516in}}%
\pgfpathlineto{\pgfqpoint{0.358712in}{0.813297in}}%
\pgfpathlineto{\pgfqpoint{0.361819in}{0.812097in}}%
\pgfpathlineto{\pgfqpoint{0.364925in}{0.832334in}}%
\pgfpathlineto{\pgfqpoint{0.371137in}{0.932719in}}%
\pgfpathlineto{\pgfqpoint{0.380456in}{1.192739in}}%
\pgfpathlineto{\pgfqpoint{0.392881in}{1.570680in}}%
\pgfpathlineto{\pgfqpoint{0.399093in}{1.699250in}}%
\pgfpathlineto{\pgfqpoint{0.405306in}{1.763791in}}%
\pgfpathlineto{\pgfqpoint{0.408412in}{1.772587in}}%
\pgfpathlineto{\pgfqpoint{0.411518in}{1.768177in}}%
\pgfpathlineto{\pgfqpoint{0.417730in}{1.731809in}}%
\pgfpathlineto{\pgfqpoint{0.427049in}{1.660647in}}%
\pgfpathlineto{\pgfqpoint{0.433262in}{1.636365in}}%
\pgfpathlineto{\pgfqpoint{0.436368in}{1.635815in}}%
\pgfpathlineto{\pgfqpoint{0.439474in}{1.642975in}}%
\pgfpathlineto{\pgfqpoint{0.445686in}{1.675494in}}%
\pgfpathlineto{\pgfqpoint{0.458111in}{1.754221in}}%
\pgfpathlineto{\pgfqpoint{0.461217in}{1.764169in}}%
\pgfpathlineto{\pgfqpoint{0.464324in}{1.767307in}}%
\pgfpathlineto{\pgfqpoint{0.467430in}{1.763212in}}%
\pgfpathlineto{\pgfqpoint{0.470536in}{1.752100in}}%
\pgfpathlineto{\pgfqpoint{0.476748in}{1.712500in}}%
\pgfpathlineto{\pgfqpoint{0.495386in}{1.568793in}}%
\pgfpathlineto{\pgfqpoint{0.501598in}{1.550185in}}%
\pgfpathlineto{\pgfqpoint{0.504704in}{1.549058in}}%
\pgfpathlineto{\pgfqpoint{0.507811in}{1.552871in}}%
\pgfpathlineto{\pgfqpoint{0.514023in}{1.571634in}}%
\pgfpathlineto{\pgfqpoint{0.526448in}{1.615663in}}%
\pgfpathlineto{\pgfqpoint{0.529554in}{1.619702in}}%
\pgfpathlineto{\pgfqpoint{0.532660in}{1.618845in}}%
\pgfpathlineto{\pgfqpoint{0.535767in}{1.612737in}}%
\pgfpathlineto{\pgfqpoint{0.541979in}{1.585941in}}%
\pgfpathlineto{\pgfqpoint{0.560616in}{1.482159in}}%
\pgfpathlineto{\pgfqpoint{0.563722in}{1.477311in}}%
\pgfpathlineto{\pgfqpoint{0.566829in}{1.478060in}}%
\pgfpathlineto{\pgfqpoint{0.569935in}{1.483835in}}%
\pgfpathlineto{\pgfqpoint{0.585466in}{1.529519in}}%
\pgfpathlineto{\pgfqpoint{0.588572in}{1.526713in}}%
\pgfpathlineto{\pgfqpoint{0.591678in}{1.515440in}}%
\pgfpathlineto{\pgfqpoint{0.597891in}{1.464675in}}%
\pgfpathlineto{\pgfqpoint{0.604103in}{1.379552in}}%
\pgfpathlineto{\pgfqpoint{0.622740in}{1.089492in}}%
\pgfpathlineto{\pgfqpoint{0.628953in}{1.044997in}}%
\pgfpathlineto{\pgfqpoint{0.632059in}{1.037451in}}%
\pgfpathlineto{\pgfqpoint{0.635165in}{1.038686in}}%
\pgfpathlineto{\pgfqpoint{0.638272in}{1.047142in}}%
\pgfpathlineto{\pgfqpoint{0.644484in}{1.077703in}}%
\pgfpathlineto{\pgfqpoint{0.656909in}{1.141097in}}%
\pgfpathlineto{\pgfqpoint{0.663121in}{1.159073in}}%
\pgfpathlineto{\pgfqpoint{0.672440in}{1.179709in}}%
\pgfpathlineto{\pgfqpoint{0.678652in}{1.203397in}}%
\pgfpathlineto{\pgfqpoint{0.684865in}{1.244088in}}%
\pgfpathlineto{\pgfqpoint{0.691077in}{1.304407in}}%
\pgfpathlineto{\pgfqpoint{0.700396in}{1.427744in}}%
\pgfpathlineto{\pgfqpoint{0.712821in}{1.636483in}}%
\pgfpathlineto{\pgfqpoint{0.731458in}{2.014089in}}%
\pgfpathlineto{\pgfqpoint{0.746989in}{2.326776in}}%
\pgfpathlineto{\pgfqpoint{0.756308in}{2.459163in}}%
\pgfpathlineto{\pgfqpoint{0.762520in}{2.506859in}}%
\pgfpathlineto{\pgfqpoint{0.765626in}{2.516210in}}%
\pgfpathlineto{\pgfqpoint{0.768732in}{2.515246in}}%
\pgfpathlineto{\pgfqpoint{0.771839in}{2.503747in}}%
\pgfpathlineto{\pgfqpoint{0.778051in}{2.449633in}}%
\pgfpathlineto{\pgfqpoint{0.784264in}{2.358091in}}%
\pgfpathlineto{\pgfqpoint{0.796688in}{2.106416in}}%
\pgfpathlineto{\pgfqpoint{0.809113in}{1.865179in}}%
\pgfpathlineto{\pgfqpoint{0.818432in}{1.748876in}}%
\pgfpathlineto{\pgfqpoint{0.824644in}{1.708282in}}%
\pgfpathlineto{\pgfqpoint{0.827751in}{1.698065in}}%
\pgfpathlineto{\pgfqpoint{0.830857in}{1.693802in}}%
\pgfpathlineto{\pgfqpoint{0.833963in}{1.694883in}}%
\pgfpathlineto{\pgfqpoint{0.837069in}{1.700710in}}%
\pgfpathlineto{\pgfqpoint{0.843282in}{1.724230in}}%
\pgfpathlineto{\pgfqpoint{0.852600in}{1.779175in}}%
\pgfpathlineto{\pgfqpoint{0.865025in}{1.852219in}}%
\pgfpathlineto{\pgfqpoint{0.871237in}{1.871541in}}%
\pgfpathlineto{\pgfqpoint{0.874344in}{1.874683in}}%
\pgfpathlineto{\pgfqpoint{0.877450in}{1.873338in}}%
\pgfpathlineto{\pgfqpoint{0.880556in}{1.867760in}}%
\pgfpathlineto{\pgfqpoint{0.886769in}{1.846037in}}%
\pgfpathlineto{\pgfqpoint{0.896087in}{1.797121in}}%
\pgfpathlineto{\pgfqpoint{0.914724in}{1.678731in}}%
\pgfpathlineto{\pgfqpoint{0.945787in}{1.468009in}}%
\pgfpathlineto{\pgfqpoint{0.961318in}{1.389837in}}%
\pgfpathlineto{\pgfqpoint{0.973742in}{1.341649in}}%
\pgfpathlineto{\pgfqpoint{0.983061in}{1.318178in}}%
\pgfpathlineto{\pgfqpoint{0.992380in}{1.293890in}}%
\pgfpathlineto{\pgfqpoint{0.998592in}{1.266656in}}%
\pgfpathlineto{\pgfqpoint{1.004805in}{1.224728in}}%
\pgfpathlineto{\pgfqpoint{1.014123in}{1.133784in}}%
\pgfpathlineto{\pgfqpoint{1.042079in}{0.828348in}}%
\pgfpathlineto{\pgfqpoint{1.048292in}{0.792255in}}%
\pgfpathlineto{\pgfqpoint{1.051398in}{0.782199in}}%
\pgfpathlineto{\pgfqpoint{1.054504in}{0.778173in}}%
\pgfpathlineto{\pgfqpoint{1.057610in}{0.780660in}}%
\pgfpathlineto{\pgfqpoint{1.060716in}{0.790081in}}%
\pgfpathlineto{\pgfqpoint{1.066929in}{0.830992in}}%
\pgfpathlineto{\pgfqpoint{1.073141in}{0.902383in}}%
\pgfpathlineto{\pgfqpoint{1.082460in}{1.064839in}}%
\pgfpathlineto{\pgfqpoint{1.094885in}{1.359124in}}%
\pgfpathlineto{\pgfqpoint{1.107310in}{1.653063in}}%
\pgfpathlineto{\pgfqpoint{1.113522in}{1.752413in}}%
\pgfpathlineto{\pgfqpoint{1.116628in}{1.781759in}}%
\pgfpathlineto{\pgfqpoint{1.119734in}{1.795371in}}%
\pgfpathlineto{\pgfqpoint{1.122841in}{1.792470in}}%
\pgfpathlineto{\pgfqpoint{1.125947in}{1.773082in}}%
\pgfpathlineto{\pgfqpoint{1.132159in}{1.689078in}}%
\pgfpathlineto{\pgfqpoint{1.141478in}{1.484110in}}%
\pgfpathlineto{\pgfqpoint{1.153903in}{1.195184in}}%
\pgfpathlineto{\pgfqpoint{1.160115in}{1.097423in}}%
\pgfpathlineto{\pgfqpoint{1.166328in}{1.051060in}}%
\pgfpathlineto{\pgfqpoint{1.169434in}{1.048787in}}%
\pgfpathlineto{\pgfqpoint{1.172540in}{1.060110in}}%
\pgfpathlineto{\pgfqpoint{1.178753in}{1.118955in}}%
\pgfpathlineto{\pgfqpoint{1.188071in}{1.267836in}}%
\pgfpathlineto{\pgfqpoint{1.200496in}{1.474496in}}%
\pgfpathlineto{\pgfqpoint{1.206708in}{1.544991in}}%
\pgfpathlineto{\pgfqpoint{1.212921in}{1.586151in}}%
\pgfpathlineto{\pgfqpoint{1.219133in}{1.602291in}}%
\pgfpathlineto{\pgfqpoint{1.222239in}{1.603511in}}%
\pgfpathlineto{\pgfqpoint{1.225346in}{1.601605in}}%
\pgfpathlineto{\pgfqpoint{1.231558in}{1.591955in}}%
\pgfpathlineto{\pgfqpoint{1.256408in}{1.540438in}}%
\pgfpathlineto{\pgfqpoint{1.259514in}{1.538915in}}%
\pgfpathlineto{\pgfqpoint{1.262620in}{1.540499in}}%
\pgfpathlineto{\pgfqpoint{1.265726in}{1.545867in}}%
\pgfpathlineto{\pgfqpoint{1.271939in}{1.570021in}}%
\pgfpathlineto{\pgfqpoint{1.278151in}{1.613658in}}%
\pgfpathlineto{\pgfqpoint{1.287470in}{1.713550in}}%
\pgfpathlineto{\pgfqpoint{1.299895in}{1.896627in}}%
\pgfpathlineto{\pgfqpoint{1.330957in}{2.408769in}}%
\pgfpathlineto{\pgfqpoint{1.334063in}{2.432082in}}%
\pgfpathlineto{\pgfqpoint{1.337169in}{2.443089in}}%
\pgfpathlineto{\pgfqpoint{1.340276in}{2.440362in}}%
\pgfpathlineto{\pgfqpoint{1.343382in}{2.422980in}}%
\pgfpathlineto{\pgfqpoint{1.349594in}{2.343798in}}%
\pgfpathlineto{\pgfqpoint{1.355807in}{2.211925in}}%
\pgfpathlineto{\pgfqpoint{1.380656in}{1.601319in}}%
\pgfpathlineto{\pgfqpoint{1.386869in}{1.547449in}}%
\pgfpathlineto{\pgfqpoint{1.389975in}{1.542892in}}%
\pgfpathlineto{\pgfqpoint{1.393081in}{1.551822in}}%
\pgfpathlineto{\pgfqpoint{1.399294in}{1.601469in}}%
\pgfpathlineto{\pgfqpoint{1.417931in}{1.798194in}}%
\pgfpathlineto{\pgfqpoint{1.421037in}{1.812185in}}%
\pgfpathlineto{\pgfqpoint{1.424143in}{1.817626in}}%
\pgfpathlineto{\pgfqpoint{1.427249in}{1.814786in}}%
\pgfpathlineto{\pgfqpoint{1.430356in}{1.804512in}}%
\pgfpathlineto{\pgfqpoint{1.436568in}{1.767254in}}%
\pgfpathlineto{\pgfqpoint{1.452099in}{1.659719in}}%
\pgfpathlineto{\pgfqpoint{1.455205in}{1.648848in}}%
\pgfpathlineto{\pgfqpoint{1.458312in}{1.644170in}}%
\pgfpathlineto{\pgfqpoint{1.461418in}{1.646201in}}%
\pgfpathlineto{\pgfqpoint{1.464524in}{1.655109in}}%
\pgfpathlineto{\pgfqpoint{1.470736in}{1.692423in}}%
\pgfpathlineto{\pgfqpoint{1.480055in}{1.783185in}}%
\pgfpathlineto{\pgfqpoint{1.489374in}{1.875851in}}%
\pgfpathlineto{\pgfqpoint{1.495586in}{1.910750in}}%
\pgfpathlineto{\pgfqpoint{1.498692in}{1.914320in}}%
\pgfpathlineto{\pgfqpoint{1.501799in}{1.906804in}}%
\pgfpathlineto{\pgfqpoint{1.504905in}{1.887443in}}%
\pgfpathlineto{\pgfqpoint{1.511117in}{1.812999in}}%
\pgfpathlineto{\pgfqpoint{1.520436in}{1.626500in}}%
\pgfpathlineto{\pgfqpoint{1.539073in}{1.206266in}}%
\pgfpathlineto{\pgfqpoint{1.545286in}{1.131389in}}%
\pgfpathlineto{\pgfqpoint{1.548392in}{1.115387in}}%
\pgfpathlineto{\pgfqpoint{1.551498in}{1.114560in}}%
\pgfpathlineto{\pgfqpoint{1.554604in}{1.128622in}}%
\pgfpathlineto{\pgfqpoint{1.560817in}{1.196345in}}%
\pgfpathlineto{\pgfqpoint{1.570135in}{1.360174in}}%
\pgfpathlineto{\pgfqpoint{1.579454in}{1.519524in}}%
\pgfpathlineto{\pgfqpoint{1.585666in}{1.581396in}}%
\pgfpathlineto{\pgfqpoint{1.588773in}{1.592872in}}%
\pgfpathlineto{\pgfqpoint{1.591879in}{1.590345in}}%
\pgfpathlineto{\pgfqpoint{1.594985in}{1.574001in}}%
\pgfpathlineto{\pgfqpoint{1.601197in}{1.503962in}}%
\pgfpathlineto{\pgfqpoint{1.610516in}{1.333005in}}%
\pgfpathlineto{\pgfqpoint{1.629153in}{0.966878in}}%
\pgfpathlineto{\pgfqpoint{1.635366in}{0.887875in}}%
\pgfpathlineto{\pgfqpoint{1.641578in}{0.844442in}}%
\pgfpathlineto{\pgfqpoint{1.644684in}{0.837450in}}%
\pgfpathlineto{\pgfqpoint{1.647791in}{0.840605in}}%
\pgfpathlineto{\pgfqpoint{1.650897in}{0.853827in}}%
\pgfpathlineto{\pgfqpoint{1.657109in}{0.908467in}}%
\pgfpathlineto{\pgfqpoint{1.666428in}{1.043461in}}%
\pgfpathlineto{\pgfqpoint{1.681959in}{1.283622in}}%
\pgfpathlineto{\pgfqpoint{1.688171in}{1.344242in}}%
\pgfpathlineto{\pgfqpoint{1.694384in}{1.378349in}}%
\pgfpathlineto{\pgfqpoint{1.700596in}{1.392850in}}%
\pgfpathlineto{\pgfqpoint{1.716127in}{1.411408in}}%
\pgfpathlineto{\pgfqpoint{1.722340in}{1.430070in}}%
\pgfpathlineto{\pgfqpoint{1.728552in}{1.458238in}}%
\pgfpathlineto{\pgfqpoint{1.737871in}{1.518791in}}%
\pgfpathlineto{\pgfqpoint{1.747189in}{1.602245in}}%
\pgfpathlineto{\pgfqpoint{1.762720in}{1.754257in}}%
\pgfpathlineto{\pgfqpoint{1.768933in}{1.786205in}}%
\pgfpathlineto{\pgfqpoint{1.772039in}{1.789252in}}%
\pgfpathlineto{\pgfqpoint{1.775145in}{1.782275in}}%
\pgfpathlineto{\pgfqpoint{1.778252in}{1.764956in}}%
\pgfpathlineto{\pgfqpoint{1.784464in}{1.701371in}}%
\pgfpathlineto{\pgfqpoint{1.793783in}{1.556712in}}%
\pgfpathlineto{\pgfqpoint{1.806207in}{1.361104in}}%
\pgfpathlineto{\pgfqpoint{1.815526in}{1.263002in}}%
\pgfpathlineto{\pgfqpoint{1.824845in}{1.204354in}}%
\pgfpathlineto{\pgfqpoint{1.843482in}{1.118707in}}%
\pgfpathlineto{\pgfqpoint{1.849694in}{1.097120in}}%
\pgfpathlineto{\pgfqpoint{1.852801in}{1.091110in}}%
\pgfpathlineto{\pgfqpoint{1.855907in}{1.089753in}}%
\pgfpathlineto{\pgfqpoint{1.859013in}{1.094023in}}%
\pgfpathlineto{\pgfqpoint{1.862119in}{1.104664in}}%
\pgfpathlineto{\pgfqpoint{1.868332in}{1.146331in}}%
\pgfpathlineto{\pgfqpoint{1.877650in}{1.253243in}}%
\pgfpathlineto{\pgfqpoint{1.893181in}{1.446206in}}%
\pgfpathlineto{\pgfqpoint{1.899394in}{1.478952in}}%
\pgfpathlineto{\pgfqpoint{1.902500in}{1.478899in}}%
\pgfpathlineto{\pgfqpoint{1.902500in}{1.478899in}}%
\pgfusepath{stroke}%
\end{pgfscope}%
\begin{pgfscope}%
\pgfsetrectcap%
\pgfsetmiterjoin%
\pgfsetlinewidth{0.000000pt}%
\definecolor{currentstroke}{rgb}{1.000000,1.000000,1.000000}%
\pgfsetstrokecolor{currentstroke}%
\pgfsetdash{}{0pt}%
\pgfpathmoveto{\pgfqpoint{0.275000in}{0.375000in}}%
\pgfpathlineto{\pgfqpoint{0.275000in}{2.640000in}}%
\pgfusepath{}%
\end{pgfscope}%
\begin{pgfscope}%
\pgfsetrectcap%
\pgfsetmiterjoin%
\pgfsetlinewidth{0.000000pt}%
\definecolor{currentstroke}{rgb}{1.000000,1.000000,1.000000}%
\pgfsetstrokecolor{currentstroke}%
\pgfsetdash{}{0pt}%
\pgfpathmoveto{\pgfqpoint{1.980000in}{0.375000in}}%
\pgfpathlineto{\pgfqpoint{1.980000in}{2.640000in}}%
\pgfusepath{}%
\end{pgfscope}%
\begin{pgfscope}%
\pgfsetrectcap%
\pgfsetmiterjoin%
\pgfsetlinewidth{0.000000pt}%
\definecolor{currentstroke}{rgb}{1.000000,1.000000,1.000000}%
\pgfsetstrokecolor{currentstroke}%
\pgfsetdash{}{0pt}%
\pgfpathmoveto{\pgfqpoint{0.275000in}{0.375000in}}%
\pgfpathlineto{\pgfqpoint{1.980000in}{0.375000in}}%
\pgfusepath{}%
\end{pgfscope}%
\begin{pgfscope}%
\pgfsetrectcap%
\pgfsetmiterjoin%
\pgfsetlinewidth{0.000000pt}%
\definecolor{currentstroke}{rgb}{1.000000,1.000000,1.000000}%
\pgfsetstrokecolor{currentstroke}%
\pgfsetdash{}{0pt}%
\pgfpathmoveto{\pgfqpoint{0.275000in}{2.640000in}}%
\pgfpathlineto{\pgfqpoint{1.980000in}{2.640000in}}%
\pgfusepath{}%
\end{pgfscope}%
\end{pgfpicture}%
\makeatother%
\endgroup%

            \end{subfigure}
            \caption{Each plot contains 10 sampled functions drawn form a gaussian process prior with SE covariance function using a length scale of $0.2$, $0.05$ and $0.02$ from left to right. Notice how the length scale intuitively governs how quickly the function value changes.}
            \label{fig:rbflengthscale}
            \end{figure*}


        \subsubsection{Prior Mean}\label{sec:priormean}

            So far we have only discussed the prior covariance effect on the GP by so far assuming zero mean throughout the input space.
            It provides a principled way of incorporating expert knowledge as done by DNGO covered in \cref{sec:dngo}.
            In practice the prior mean is mostly constant thus leaving the posterior mean to be inferred through the likelihood of the data.

        \subsubsection{GP Regression}\label{sec:gpreg}
      
            For Bayesian Optimization we are interested in estimating the function value at an unobserved point based on the statistical model.
            This is a Bayesian regression problem and specifically with a gaussian prior it is referred to as GP regression.

            As exemplified by \cref{fig:rbflengthscale} it is possible to sample from the prior GP.
            Recall that by the definition of GP a (finite) set of points induces a joint gaussian distribution.
            Thus sampling from a GP prior $\mathcal{GP}(\mu(\bm{x}), k(\bm{x},\bm{x}'))$ is simply done through a gaussian distribution with covariance matrix $\bm{K}$ constructed from $k(\bm{x},\bm{x}')$,
            \begin{equation}
                f \sim \mathcal{N}(\mu(\bm{x}), \bm{K}).
            \end{equation}
        
            We will now consider how to predict a new sample from a learned posterior function, i.e. define the \emph{predictive posterior distribution}.
            This will be done for only single predictions, since this is sufficient in the context of Bayesian Optimization (see \parencite{rasmussen_gaussian_2006} for generalization).

            Assume we have observed $\mathcal{D}_t = \set{\bm{x}_{1:t},\bm{\mathrm{f}}_{1:t}}$ for which we have constructed the covariance matrix $\bm{K}$ and we wish to predict $\mathrm{f}_{t+1}$ at $\bm{x}_{t+1}$.

            To get a distribution over the prediction let us first imagine that all $\bm{\mathrm{f}}_{1:t+1}$ are drawn at the same time.
            Again, by the property of the GP they are sampled from a joint Gaussian distribution,
            \begin{equation} 
                \begin{bmatrix}\bm{\mathrm{f}}_{1:t}\\\mathrm{f}_{t+1}\end{bmatrix} \sim \mathcal{N}(\mu(\bm{x}_{1:t+1}),
                \begin{bmatrix}
                \bm{K}&\bm{k}\\
                \bm{k}^T&k(\bm{x}_{t+1},\bm{x}_{t+1})
                \end{bmatrix}
                ),
            \end{equation}
            
            \noindent where $\bm{k} = [k(\bm{x}_{t+1}, \bm{x}_1), \cdots, k(\bm{x}_{t+1}, \bm{x}_t)]$. 
            The primary difference is that the covariance matrix have been expanded to capture correlation between $\mathrm{f}_{t+1}$ and all previous evaluations.

            From here it is relatively straightforward to condition on $\bm{\mathrm{f}}_{1:t}$, thus arriving at the predictive posterior distribution (see \parencite[A.2]{rasmussen_gaussian_2006} for the full derivation),
            \begin{equation}\label{eq:gppred}%
                \mathrm{f}_{t+1}\mid \bm{x}_{t+1},\bm{x}_{1:t},\bm{\mathrm{f}}_{1:t} \sim \bm{N}(\mu_t(\bm{x}_{t+1}),\sigma_t(\bm{x}_{t+1})^2)  
            \end{equation}

            where 
            \begin{align}
                \mu_t(\bm{x}_{t+1})          &= \bm{k}^\top\bm{K}^{-1}\bm{\mathrm{f}}_{1:t} \\
                \text{and }\sigma_t(\bm{x}_{t+1})^2 &= k(\bm{x}_{t+1}, \bm{x}_{t+1}) − \bm{k}^\top \bm{K}^{-1}\bm{k}.
            \end{align}

            \noindent So the predictive posterior distribution is likewise gaussian, meaning $\mu_t(\cdot)$ and $\sigma_t^2(\cdot)$ are sufficient statistics.
            
            Returning to the Bayesian Optimization setting, our statistical model now captures a mean and variance over all points in the input space which is sufficient for defining the acquisition function. % TODO: rephrase when BO chapter have been done..

            \paragraph{Noisy function evaluation}

                In many application of Bayesian Optimization it is only possible to access $f$ through some noisy evaluation.

                We add Gaussian noise $\epsilon_i \stackrel{iid}\sim \mathcal{N}(0,\sigma_{\mathrm{noise}}^2)$ to function evaluations $y_i = \mathrm{f}_i+ \epsilon_i$. Denote observations $\mathcal{D}_t = \set{\bm{x}_{1:t}, \bm{\mathrm{y}}_{1:t}}$ and the next point $\bm{x}_{t+1}$ with function evaluation $\bm{\mathrm{y}}_{t+1}$.
                
                The likelihood of $\bm{\mathrm{y}}_{1:t+1}$ thus becomes,
                    \begin{equation}
                        \bm{\mathrm{y}}_{1:t+1} \mid \bm{\mathrm{f}} \sim \mathcal{N}(\bm{\mathrm{f}}, \sigma_{\mathrm{noise}}^2 \bm{I}).
                    \end{equation} 

                Integrate out $\bm{\mathrm{f}}$ to get the marginal likelihood,
                    \begin{equation}
                        P(\bm{\mathrm{y}}) 
                    = \int P(\bm{\mathrm{y}}\mid \bm{\mathrm{f}})P(\bm{\mathrm{f}})\ \mathrm{d} \bm{\mathrm{f}} 
                    = \mathcal{N}(\mu(\bm{x}_{1:t+1}), \bm{K} + \sigma_{\mathrm{noise}}^2 \bm{I}).
                    \end{equation}

                From here the same method used to get \eqref{eq:gppred} by conditioning on $\mathcal{D}_t$ can be applied to obtain the predictive posterior distribution,
                    \begin{equation}
                        \mathrm{y}_{t+1} \mid \bm{\mathrm{y}}_{1:t} \sim \mathcal{N}(m_{t}(\bm{x}_{t+1}), \sigma_{t}^2(\bm{x}_{t+1}))
                    \end{equation}

                where
                    \begin{align}
                        \mu_t(\bm{x}_{t+1})          &= \bm{k}^\top (\bm{K} + \sigma_{\mathrm{noise}}^2 \bm{I})^{-1} \bm{\mathrm{y}}\label{eq:gppredmu} \\
                        \text{and }\sigma_t^2(\bm{x}_{t+1}) &= k(\bm{x}_{t+1}, \bm{x}_{t+1}) - \bm{k}^\top (\bm{K} + \sigma_{\mathrm{noise}}^2 \bm{I})^{-1} \bm{k}.\label{eq:gppredsigma}
                    \end{align}
                    
                What have changed from the noise free setting is simply the diagonal $\sigma_{\mathrm{noise}}^2$ term in the kernel and that we now observe $\bm{y}$ instead of $\bm{f}$ directly.

            \paragraph{Running time}\label{sec:runningtime}

                We have obtained an analytical expression for the predictive posterior distribution thus providing well-calibrated uncertainty estimates.
                However, the cost of this exact inference is $\mathcal{O}(n^3)$ where $n$ is number of observations.
                The cost is due to the need for inverting the covariance matrix $\bm{K}$ with size $n \times n$. % In practice the cholesky decomposition is computed.
                By storing $\bm{K}^{-1}$ the mean and variance can be calculated in $\mathcal{O}(n^2)$ and $\mathcal{O}(n)$.

                New samples in the Bayesian Optimization setting could easily be handled by adding the additional row and column for every update so subsequent predictions becomes $\mathcal{O}(n^2)$.
                However, $\bm{K}^{-1}$ would have to be recomputed everytime the hyperparameters of the kernel are updated which is desireable to do based on data as covered in \cref{sec:choosehyper}.

                This computational constraint is the motivation for DNGO introduced in \cref{sec:dngo}.

        \subsubsection{Choosing hyperparameters}\label{sec:choosehyper}
        
            It became apparent in \cref{sec:covar} that the hyperparameters of the kernel have prevailing impact on the type of function the GP describes.
            This section will cover how to infer these hyperparameters $\theta$ from observations $\mathcal{D}_t=\set{\bm{x}_{1:t},\bm{\mathrm{y}}_{1:t}}$.

            \paragraph{MLE and MAP}
                The simplest solution to consider is where we choose the hyperparameters $\theta$ which maximizes $P(\bm{\mathrm{y}}|\bm{x}_{1:t},\theta)$,
                    \begin{equation}
                        \hat{\theta}_n^{MLE} = \argmax_{\theta} P(\bm{\mathrm{y}}_{1:t}\mid\bm{x}_{1:t},\theta)
                    \end{equation}

                \noindent Since we are maximizing the likelihood this is referred to as the \emph{maximum likelihood estimation} (MLE).

                A natural extension to MLE includes a prior over the hyperparameters (refered to as a \emph{hyperprior}), thus getting the \emph{maximum a posteriori} (MAP), 
                    \begin{equation}\hat{\theta}_n^{MAP} = \argmax_{\theta} P(\bm{\mathrm{y}}_{1:t}\mid\bm{x}_{1:t},\theta) P(\theta)\end{equation}

                \noindent We arrive at this by first considering maximizing the posterior $P(\theta\mid\bm{\mathrm{y}}_{1:t},\bm{x}_{1:t})$ instead.
                Using bayes rule and dropping the normalization constant we obtain the expression,

                \begin{equation}
                    \begin{split}
                        P(\theta\mid\bm{\mathrm{y}}_{1:t},\bm{x}_{1:t})
                        &= \frac{P(\bm{\mathrm{y}}_{1:t}\mid\bm{x}_{1:t}, \theta)P(\theta)}
                        {P(\bm{\mathrm{y}}_{1:t}\mid \bm{x}_{1:t})}\\
                        &\propto P(\bm{\mathrm{y}}_{1:t}\mid\bm{x}_{1:t},\theta)P(\theta)
                        .
                    \end{split}
                \end{equation}

                We can use this unnormalized posterior since the normalization constant does not vary across the parameters we are optimizing.
                Placing some kind of prior can be crucial, especially when few data points are available as is the case in Bayesian Optimization.
                Common priors include uniform priors to specify hard constraints, normal priors to suggest it lies near some nominal value and log-normal for strictly positive parameters.
                Notice how MLE is a special case of MAP in which a uniform prior has been placed over the hyperparameters.
                The specific priors used for this problemset will be discussed in \cref{sec:priors}.

                % In practise we minimize the negative log marginal likelihood l(θ)
                % include expression (see human loop)
                An analytical expression can be derived for both MLE and MAP \parencite{rasmussen_gaussian_2006} including their gradient.
                Thus both can be optimized using gradient decent methods like Conjugate Gradient or quasi-Newton methods.
                Note that in practice the \emph{negative logarithm} of the objective is minimized since optimization problem are commonly stated as minimization problems and secondly to prevent numerical overflow.

            \paragraph{Fully Bayesian}\label{sec:hypmcmc}

                Both MLE and MAP provides a \emph{point estimate} of the hyperparameters which are then used to get the predictive posterior distribution used for predicting $\mathrm{y}_{t+1}$ at $\bm{x}_{t+1}$.
                We will now consider the so-called \emph{fully Bayesian} treatment in which the hyperparameters are marginalized out,
                    \begin{equation} 
                        P(\mathrm{y}_{t+1}|\bm{x}_{t+1},\mathcal{D}_t) = \int P(\mathrm{y}_{t+1}|\bm{x}_{t+1},\mathcal{D}_t,\theta) P(\theta|\mathcal{D}_t)\ \mathrm{d}\theta.
                    \end{equation}

                This integral is generally intractable so instead we approximate it by sampling $\theta$,
                \begin{equation} 
                    P(\mathrm{y}_{t+1}|\bm{x}_{t+1},\mathcal{D}_t) \approx \frac{1}{N} \sum_{\theta=1}^{N} P(\mathrm{y}_{t+1}|\bm{x}_{t+1},\mathcal{D}_t,\theta),
                \end{equation}

                \noindent in which $\set{\theta_n}_{n=1}^N$ is sampled from the posterior $P(\theta|\mathcal{D}_t)$. 
                Since it is even impossible to sample directly from this posterior, Markov Chain Monte Carlo (MCMC) techniques are usually utilized to produce samples from an identical distribution in the limit of an infinitely long chain of samples. % TODO: what to call P(theta|D)
                For further details on these techniques we refer to \parencite{berg_introduction_2004}.

                % - hyperparameters directly from training data
                % - (SVM, splines requires cross validation)
   
        \subsubsection{Automatic Relevance Detection}\label{sec:ard}
          
            Automatic Relevance Detection (ARD) is a method for multi-dimensional input useful when many dimensions are irrelevant.
            In the case of the SE kernel it learns an individual length scale $l_d$ for each input dimension $x_d$,
            \begin{equation}
                k(\bm{x},\bm{x}') = \sigma^2 \exp \bigg[ -\frac{1}{2} \sum_{d=1}^{D} \Big(\frac{|x_d - x_d'|}{l_d}\Big)^2 \bigg].
            \end{equation}
        
            \noindent This way a dimension $x_d$ can be rendered irrelevant by simply increasing $l_d$.

    \subsection{Acquisition function}\label{sec:acq}
    
        Now that we have discussed the statistical model used to capture our belief about the unknown function $f$ we return to discuss how to decide which subsequent point to evaluate.

        A whole range of different acquisition functions exist e.g. probability of improvement (PI),
        expected improvement (EI), upper confidence bounds (UCB), which all try to trade off exploration and exploitation;
        that is have optima either where uncertainty is high (exploration) or where the model prediction is high (exploitation).

        To optimize the acquisition function we often need to query it many time.
        It is therefore important that the acquisition function is much cheaper to evaluate (or approximate) than the black box $f$.
        One way of achieving this is by defining the acquisition function on a statistical model with known analytical form such as the GP.

        % motivate why
        All the above mentioned acquisitions functions do this and we will focus on the UCB defined as,
            \begin{equation}
                \alpha_{UCB}(\bm{x}) = \mu(\bm{x}) + \kappa \sigma(\bm{x}).
            \end{equation}
            % TODO: decide on mu vs m

        Together with PI and EI it is part of a class of acquisition function guided by the principle of \emph{being optimistic in the face of uncertainty}.
        That is, higher variance at a given point can only lead to an increased acquisition value.
        USB is especially attractive since it has a nice intuitive interpretation, follows directly from a visually plot of the underlying surrogate model (see \cref{fig:bo}) and provides fast convergence in Bayesian Optimization tasks. % TODO: add cite add ref to plot
        Recently, good theoretical bounds for the \emph{Cumulative Regret} (see \cref{sec:performance} for definition) have even been shown by careful choice of a time depended $\kappa$ \parencite{srinivas_gaussian_2012}.

        Let us turn to how we define the acquisition function when there is uncertainty about kernel hyperparameter $\theta$.
        Define $\alpha(\bm{x};\theta)$ as the acquisition function with known hyperparameter $\theta$.
        We can marginalize out $\theta$ given the data $\mathcal{D}_t$ at timestep $t$,
            \begin{equation}
            \alpha_t(\bm{x}) = \mathbb{E}_{\theta|\mathcal{D}_t}[\alpha(\bm{x};\theta)]
                        = \int \alpha(\bm{x};\theta)p(\theta|\mathcal{D}_t) \ \mathrm{d}\theta.
            \end{equation}

        One simple way of dealing with this expression is to approximate $P(\theta|\mathcal{D}_t)$ by a point estimate (i.e. a Dirac distribution) at either $\hat{\theta}_t^{MLE}$ or $\hat{\theta}_t^{MAP}$ obtained as described in \cref{sec:choosehyper}.

        However, since the uncertainty in the response surface is crucial to guide exploration in Bayesian Optimization, it is desirable to also incorporate the uncertainty about $\theta$.
        Thus we approximate the marginalization by sampling $\theta$,
            \begin{equation}
              \mathbb{E}_{\theta|\mathcal{D}_t}[\alpha(x;\theta)] \approx \frac{1}{N} \sum_{i=1}^{N} \alpha(\bm{x};\theta_i).
            \end{equation}
            
        \noindent with $N$ samples from the posterior $P(\theta|\mathcal{D}_t)$ using MCMC as in \cref{sec:hypmcmc}.
        This effectively averages over the samples similarly to \cref{sec:hypmcmc} but maps the posterior statistics through the acquisition function before averaging.

        % in all cases the acqusition function can now be 
        % multistarted quasi-Newton hill climbers [e.g., the limitedmemory Broyden–Fletcher–Goldfarb–Shanno (L-BFGS) method]

    \subsection{DNGO}\label{sec:dngo}

        This section will describe the method proposed in \parencite{snoek_scalable_2015} for making Bayesian Optimization scalable.
        The problem it addresses is that the running time at every step in GP based Bayesian Optimization is cubic in the number of observations (see \cref{sec:runningtime}).
        This becomes prohibitively large when using a parallel Bayesian Optimization approach \parencite{snoek_practical_2012} which allows them to make many (expensive) function evaluations than previously possible with the common sequential approach.

        The proposed solution proceeds in two steps in which DNGO first learns a set of basis function using a neural network. 
        Subsequently it fits a Bayesian linear regressor to the input transformed by the neural network to capture the uncertainty.

        To make this model precise denote the output of the neural network $\bm{\phi}(\cdot) = [\phi_1(\cdot), \cdots, \phi_D(\cdot)]^\top$ where $D$ is the number of basis functions.
        Given an input set $\set{\bm{x}_i}_{i=1}^n$ we can define what is commonly referred to as the \emph{design matrix},
        \begin{equation}
            \bm{\Phi} = \begin{bmatrix}
            \phi_1(\bm{x}_1) & \cdots & \phi_d(\bm{x}_1) \\
            \vdots & \ddots & \vdots \\
            \phi_1(\bm{x}_n) & \cdots & \phi_d(\bm{x}_n) \\
        \end{bmatrix}.
        \end{equation}

        The network weights are trained by adding a linear layer and then training on a dataset $\set{(\bm{x}_i, \mathrm{y}_i)}_{i=1}^n$ using backpropagation and stocastic gradient descent with momentum.
        Exactly what dataset will be covered in \cref{sec:methoddngo}.

        Now, given observation $\mathcal{D}_t = \set{\bm{x}_{1:t},\bm{\mathrm{f}}_{1:t}}$ at step $t$ in the Bayesian Optimization procedure it constructs the design matrix by applying the neural network.
        It then fits a Bayesian linear regressor providing predictive statistics as follows,

            \begin{align}
                % ; \set{\bm{\lambda}_n, f(\bm{\lambda}_n)}, \bm{\Theta}
                % ; \set{\bm{\lambda}_n, f(\bm{\lambda}_n)}, \bm{\Theta}
                \mu(\bm{x}) &= \bm{\mathrm{m}}^\top \bm{\phi}(\bm{x})\\ %+ \eta(\bm{x})
                \sigma^2(\bm{x}) &= \bm{\phi}(\bm{x})^\top \bm{K}^{-1} \bm{\phi}(\bm{x}) + 1/\beta,
            \end{align}

        where 

            \begin{align}
                \bm{\mathrm{m}} & = \beta \bm{K}^{-1} \bm{\Phi}^\top \bm{\mathrm{y}}\\
                \bm{K} & = \beta \bm{\Phi}^\top \bm{\Phi} + \bm{I} \alpha
            \end{align}
            
        Notice the similarity with the predictive mean \eqref{eq:gppredmu} and variance \eqref{eq:gppredsigma} for GP.
        The primary difference is that $\bm{K}$ now grows with the size of the output dimensionality of the neural network.
        So the running time becomes $\mathcal{O}(n)$ instead of $\mathcal{O}(n^3)$ for $n$ observations. % TODO: why linear?

        \subsubsection{network architecture}\label{sec:dngoarch}
            
            The original paper specifies a network of three layers with 50 units in each using the ReLU activation function in the first two layers and TanH in the final layer.

\section{Method}\label{sec:method}

    This section will cover the three models we are comparing, GP, DNGO and DNGO with an ensemble of neural networks, including details on how they are compared.

    There are many choices to make both for the GP approach and the DNGO architecture.
    This section summaries the choices and the reasons behind them.
    For further detail we refer to the specific implementation at \parencite{thomas_m._pethick_ensembled_2018}.

    \subsection{GP}
        
        The primary decision in GP is the choice of kernel.
        Here we will compare DNGO against a GP using the SE kernel.
        This is done for its smoothness assumption, simplicity and experimentally verified usefulness in Bayesian Optimization tasks \parencite{snoek_practical_2012, wang_bayesian_2013, bergstra_algorithms_2011}.
        Hyperpriors for this kernel are discussed in \cref{sec:priors}.
        
        A critical reader might question why more complicated kernels like the Matérn \parencite{rasmussen_gaussian_2006} were not considered.
        We decide to use the simplest kernel that performs well, which it is expected to do especially on the benchmark functions used, since most are smooth.

        % MCMC was done with a burn-in of 100, sub sampling interval of 10, step size of $10^{-1}$ and leapfrog steps of 20 similarly to \parencite{gpyopt2016}.

    \subsection{DNGO}\label{sec:methoddngo}

        The original network architecture as described in \cref{sec:dngoarch} has been kept except for the use of the Adam optimizer \parencite{kingma_adam:_2014} instead of SGD.

        It was not clear how they trained the neural network.
        Thus we will experiment with three different approaches:
        
        \begin{itemize}
            \item training on a fixed set $\mathcal{D}_{m}$ of $m$ randomly sampled observations;
            \item retraining on all $\mathcal{D}_t$ observations at every timestep $t$ without resetting weights, and;
            \item retraining on all $\mathcal{D}_t$ observations at every timestep $t$ but resetting the weight.
        \end{itemize}

        Results are covered in \cref{sec:exp} in which testing of hyperparameters such as weight decay and number of epochs are also included. % since exact specification was left out..

    \subsection{Ensemble}

        By introducing an ensemble of neural networks we obtain a set of values for the acquisition function at a given point -- one for each neural network.
        This is because for all $N$ neural networks we fit a Bayesian linear regressor separately from which the acquisition function is evaluated.

        The acquisition function $\alpha(\bm{x})$ is then defined as some \emph{aggregate} over this set of values,
            \begin{equation}
                \alpha(\bm{x}) = \mathsf{A}(\set{\alpha(\bm{x});\theta_i}_{i=1}{N}).
            \end{equation}

        \noindent where $\mathsf{A} :\mathbb{R}^{N}\to\mathbb R$ is the aggregator function
        and $\theta_i$ denotes both the hyperparameters of the Bayesian linear regressor and the weights of $i$'th neural network. 
        
        Apart from varying $N$ we will consider different method of aggregating this set of values.
        Specifically \emph{mean}, \emph{median} and \emph{max}, as these are naturally expected to make very different exploration-exploitation trade-offs.

    \subsection{Normalization}
        
        Both the input and output space are normalized to have mean $0$ and variance $1$ in each dimension.
        The primary benefit is that hyperpriors in the GP setting and the neural network become independent of the scale of the space.
        This allows us to specify useful problem-independent hyperpriors.
        %Additionally the optimization strategies such as BFGS used for hyperparameter optimization of the GP and the acquisition function are tailored for a small values.
        %This is primarily because of ensured numerical stability in this interval.

    \subsection{Hyperpriors}\label{sec:priors}

        By normalizing to a fixed interval we can allow ourselves to give high prior weight to certain volumes of the hyperparameter space.
        That is because e.g. a length scale above 1 would assume almost no change throughout the input space (recall that the length scale is approximately the distance required for a change to happen).
        
        Inspired by \parencite{swersky_freeze-thaw_2014} we choose the three hyperpriors for the GP to be,

        \begin{equation}
            \begin{split}
                l                         & \sim \mathrm{lognormal}(0, 1) \\
                \sigma_0^2                & \sim \mathrm{lognormal}(0, 1) \\
                \sigma_{\mathrm{noise}}^2 & \sim \mathrm{horseshoe}(0.1)
            \end{split}
        \end{equation} % TODO: fix sigma_0. not necessary because of norrmaliation?

        Similarly for the Bayesian linear regressor the following priors are picked,

        \begin{equation}
            \begin{split}
                \alpha & \sim \mathrm{lognormal}(0, 1) \\
                \beta  & \sim \mathrm{horseshoe}(0.1)
            \end{split}
        \end{equation}

        The log-normal prior ensures positive values with a mean 1 thus preventing unnecessarily large values from being considered.
        The horseshoe prior (described in detail in \parencite{carvalho_handling_2009}) puts high weight on 0.
        Its Cauchy-like tail allows for strong observed values to remain large a posteriori but simultaneously allows for severe shrinkage for zero element.
        We can thus completely disregard noise in a noiseless setting.

        % - Fitting without noise and no regularization: no variance in BLR layer. Possible fixes: (theoretically)
        % - ensure that alpha is never 0 (otherwise fits perfectly => no variance) (is this a hack?)

        For guidelines on hyperpriors for gaussian processes we recommend the user guide to Stan \parencite{stan_development_team_stan_2017}.

        %- Gamma prior on noise: \url{https://www.researchgate.net/figure/51717248_fig1_Gamma-prior-on-the-total-noise-variance-A-Gamma-prior-is-assumed-for-the-hype}rparameter
        %- HalfT: \url{https://github.com/stan-dev/stan/releases/download/v2.16.0/stan-reference-2.16.0.pdf}
    
    \subsection{Performance Evaluation}\label{sec:performance}

        We evaluate and compare the performance with two regret based metrices: \emph{Simple Regret} and \emph{Cumulative Regret}.

        Simple Regret measures how far the current best point is from the true optimum.
        To make it precise, define the optimal point as $\bm{x}^\star = \argmax_{\bm{x}\in \mathcal{X}}=f(\bm{x})$ and the current best at round $t$ as $\hat{\bm{x}}_t = \argmax_{\bm{x} \in \bm{x}_{1:t}} f(\bm{x})$.
        Then Simple Regret as used in \parencite{snoek_scalable_2015, springenberg_bayesian_2016, golovin_google_2017} is,
        \begin{equation}
            r_t^{\mathrm{simple}} = |f(\bm{x}^\star) - f(\hat{\bm{x}}_t)|.
        \end{equation}
        
        With this measure it is possible to compare how well different models have performed at a given round.
        However, it does not give a clear indication of the exploration-exploitation trade-off being made.
        % This only shows indirectly through the gradient if the Simple Regret is plotted over multiple steps, since a gradient of 0 indicates that no better s

        To captures this Cumulative Regret is naturally considered.
        Given the true optimum point $\bm{x}^\star$ and observation choice $\bm{x}_t$ at round $t$ \emph{instantanous regret} is defined as $r^{\mathrm{instant}}_t = |f(\bm{x}^\star) - f(\bm{x}_t)|$.
        Cumulative Regret is defined as the sum of the instantaneous regret over all rounds \parencite{srinivas_gaussian_2012}, 
            \begin{equation}
                R^{\mathrm{cumulative}}_T = \sum_{t=1}^T r^{\mathrm{instant}}_t.
            \end{equation}

        % TODO: 
        % This is stop exploring in the limit.
        % This gives as indication as to how this metric can be used.
        % if it hits a Plateau => stopped exploring
        % This is not completely true as exploration could lead to 

        % - acc regret for exploration/explotation tradeoff
        % - only explore / only exploits => linear regret!
        % - 
        
        % - Beale, Branin, Ellipsoidal, Rastrigin, Rosenbrock, Six, Hump Camel, Sphere, and Styblinski
        When comparing models a model A is said to outperform a model B if the average Simple Regret across all runs in the final round is smaller for A than B. 
        % Interchangeably, A performs better than B.
        This is especially used in \cref{sec:exp}.

\section{Experiments}\label{sec:exp}

    For comparing and testing optimization it is beneficial to have benchmark functions that are fast to evaluate and have known global optima.
    It is inherently problematic to decide on these synthetic function as the motivation for Bayesian Optimization is that $f$ is a black-box function.
    The benchmarks might thus suffer from design biases -- common is a placement of the optimum at the domain midpoint.
    However, they are still useful to understand the behavior of different methods.
    In \parencite{dewancker_stratified_2016} it is attempted to avoid the design bias by introducing a collection of benchmark categories based on a broad range of characteristics.
    We will focus on four categories that show interesting differences between GP, DNGO and DNGO ensemble:

    \begin{itemize}
        \item \textbf{Oscillatory} being a simple form of non-stationarity in which $f(x) = S(X) + T(x)$ where $T$ introduces a long-range trend while $S$ has a shorter scale oscillatory behaviour.
        \item \textbf{Mostly boring} which can be seen as a type of non-stationarity having small gradient in most of the domain except a very small region in which the optimum is found.
        \item \textbf{Nonsmooth} by having discontinuous
        derivatives on manifolds of at least one dimension. % TODO: quotes directed.. 
        \item \textbf{Discrete values} being a specifically interesting instance of nonsmooth function in the application of hyperparameter optimization.
    \end{itemize}

    % Apart from these categories we ensure to include bound_min, unscaled
    % TODO: argue why
    The specific benchmarks picked are a subset from both \parencite{eggensperger_towards_2013} and \parencite{dewancker_stratified_2016}.

    We will not try to aggregate performance over several benchmark as is the original idea behind \parencite{dewancker_stratified_2016}.
    This is primarily because a larger test set then the one picked is required.
    We are not interested in overall performance but rather in providing a preliminary investigation of the characteristics of DNGO and the ensembled DNGO.
    However, we do test it in a real hyperparameter optimization setting for a machine learning task as covered in \cref{sec:highdim}.

    % High dim
    The models are tested on some higher dimensional problems but we have chosen to restrict these to 10 dimensions so optimization of the acquisition function would not become a problem.
    A characteristic of acquisition functions is that they are mostly boring \parencite{lyu_batch_2018}.
    This is a problem in high dimension where optimization of the acquisition function from some set of random initialization cannot escape the large flat region.
    This could easily become the limiting factor for the model and render a comparison useless.

    For the 8D problems and higher (including machine learning tasks) we chose to initialize the Bayesian Optimization procedure with 100 random samples similarly to what is done in \parencite{lyu_batch_2018}.

    % Focus will primarily be on these benchmarks but hyperparameter optimization for machine learning tasks are...

    % Naming
    All plots will adhere to a specific naming convention.
    There are four different models \emph{GP}, \emph{DNGO fixed}, \emph{DNGO retrain} and \emph{DNGO retrain-reset} of which the last three refers to the different methods of training the neural network as described in \cref{sec:methoddngo}.
    
    MCMC will be added as suffix if the fully Bayesian approach is used.
    For the DNGO methods a prefix is added indicating the size of the ensemble if one such is used.
    
    As an example \emph{10 $\times$ DNGO fixed MCMC} specifies an ensemble of 10 DNGO networks for which the neural network is trained on a fixed dataset and MCMC is used to sample hyperparameters for the Bayesian linear regressor.
 
    In \cref{sec:exp-dngo} and \cref{sec:exp-training} we experiment with different hyperparameters of the DNGO architecture which are subsequently fixed for the remaining experiments.
    In \cref{sec:exp-embeddings} the promising case of embedded function are presented.
    Then the four categories of benchmarks are considered in \cref{sec:expbenchmark} and hyperparameter optimization in its own section \cref{sec:highdim}.
    % Finally, in .. we consider properties related to the scalability of DNGO.
    % TODO:

    The experiments can be reproduced using the accompanying source code at \parencite{thomas_m._pethick_ensembled_2018}.

    \subsection{Hyperparameter Settings for DNGO}\label{sec:exp-dngo}

        We compared different configurations of number of training epoch and $l_2$ regularization for the network evaluated on Branin, Hartmann3 and Hartmann6 from \parencite{eggensperger_towards_2013}.
        The model was rather robust with regards to these parameters so it was decided to use 1,000 epochs for pragmatic reasons and a small $l_2$ as suggested by \parencite{snoek_scalable_2015} of $10^{-4}$.

    \subsection{Training}\label{sec:exp-training}

        We investigated the effect of the size of the initial random samples for the DNGO with a trained network of fixed size.
        This was done on Hartmann3 as it had previously been shown to yield small incremental improvements per round with low variance which made a good candidate for method comparison.

        Increasing the sample size yielded faster convergence and, in the case of 200 samples also an improved optimum (see \cref{sec:apptraining}).
        That the sample size sets an upper bound for how well $f$ can be modeled also shows across Branin, Hartmann3 and Hartmann6 when compared with the other training strategies from \cref{sec:methoddngo}.
        \emph{DNGO retrain-reset} performs better than \emph{DNGO fixed} with the latter seemingly stuck.
        Thus we have decided to continue further experiments with DNGO retrain-reset which is computationally more expensive (though not asymptotically!) but which can update the weights based on incoming observations.

    \subsection{Embeddings}\label{sec:exp-embeddings}

        Low effective dimensionality is known to be a common characteristic of hyperparameter optimization \parencite{bergstra_random_2012, chen_joint_2012, wang_bayesian_2013}.
        In particular, a function of two variables is said to have low effective dimensionality if it can be approximated by a function of one variable.
        Thus we consider embedding the objective function $f$ into a higher dimensional space: the additional dimensions have no effect on the function value.

        The Simple Regret result is plotted in \cref{fig:embedding} for the embedding of the SinOne benchmark in a 2-, 3-, and 4-dimensional space.
        In all cases the ensembled DNGO does better than DNGO and the DNGO better than GP when averaging over runs.
        We will return to an explanation of this in the discussion.

        It should be noted that the methods were indistinguishable from random sampling at higher dimensions.

        %- Embedding
        %- less confidence with ensemble (find measure)
        % - Cccumulative Regret (ensemble stops exploring randomly at some point <= Acc regret stagnates)

        \begin{figure*}[t]
            \centering
            \begin{subfigure}[t]{0.3\textwidth}
                \centering
                % \resizebox{.95\linewidth}{!}{}
                %% Creator: Matplotlib, PGF backend
%%
%% To include the figure in your LaTeX document, write
%%   \input{<filename>.pgf}
%%
%% Make sure the required packages are loaded in your preamble
%%   \usepackage{pgf}
%%
%% Figures using additional raster images can only be included by \input if
%% they are in the same directory as the main LaTeX file. For loading figures
%% from other directories you can use the `import` package
%%   \usepackage{import}
%% and then include the figures with
%%   \import{<path to file>}{<filename>.pgf}
%%
%% Matplotlib used the following preamble
%%   \usepackage{gensymb}
%%   \usepackage{fontspec}
%%   \setmainfont{DejaVu Serif}
%%   \setsansfont{Arial}
%%   \setmonofont{DejaVu Sans Mono}
%%
\begingroup%
\makeatletter%
\begin{pgfpicture}%
\pgfpathrectangle{\pgfpointorigin}{\pgfqpoint{2.300000in}{3.000000in}}%
\pgfusepath{use as bounding box, clip}%
\begin{pgfscope}%
\pgfsetbuttcap%
\pgfsetmiterjoin%
\definecolor{currentfill}{rgb}{1.000000,1.000000,1.000000}%
\pgfsetfillcolor{currentfill}%
\pgfsetlinewidth{0.000000pt}%
\definecolor{currentstroke}{rgb}{1.000000,1.000000,1.000000}%
\pgfsetstrokecolor{currentstroke}%
\pgfsetdash{}{0pt}%
\pgfpathmoveto{\pgfqpoint{0.000000in}{0.000000in}}%
\pgfpathlineto{\pgfqpoint{2.300000in}{0.000000in}}%
\pgfpathlineto{\pgfqpoint{2.300000in}{3.000000in}}%
\pgfpathlineto{\pgfqpoint{0.000000in}{3.000000in}}%
\pgfpathclose%
\pgfusepath{fill}%
\end{pgfscope}%
\begin{pgfscope}%
\pgfsetbuttcap%
\pgfsetmiterjoin%
\definecolor{currentfill}{rgb}{0.917647,0.917647,0.949020}%
\pgfsetfillcolor{currentfill}%
\pgfsetlinewidth{0.000000pt}%
\definecolor{currentstroke}{rgb}{0.000000,0.000000,0.000000}%
\pgfsetstrokecolor{currentstroke}%
\pgfsetstrokeopacity{0.000000}%
\pgfsetdash{}{0pt}%
\pgfpathmoveto{\pgfqpoint{0.287500in}{0.375000in}}%
\pgfpathlineto{\pgfqpoint{2.070000in}{0.375000in}}%
\pgfpathlineto{\pgfqpoint{2.070000in}{2.640000in}}%
\pgfpathlineto{\pgfqpoint{0.287500in}{2.640000in}}%
\pgfpathclose%
\pgfusepath{fill}%
\end{pgfscope}%
\begin{pgfscope}%
\pgfpathrectangle{\pgfqpoint{0.287500in}{0.375000in}}{\pgfqpoint{1.782500in}{2.265000in}}%
\pgfusepath{clip}%
\pgfsetroundcap%
\pgfsetroundjoin%
\pgfsetlinewidth{0.803000pt}%
\definecolor{currentstroke}{rgb}{1.000000,1.000000,1.000000}%
\pgfsetstrokecolor{currentstroke}%
\pgfsetdash{}{0pt}%
\pgfpathmoveto{\pgfqpoint{0.368523in}{0.375000in}}%
\pgfpathlineto{\pgfqpoint{0.368523in}{2.640000in}}%
\pgfusepath{stroke}%
\end{pgfscope}%
\begin{pgfscope}%
\definecolor{textcolor}{rgb}{0.150000,0.150000,0.150000}%
\pgfsetstrokecolor{textcolor}%
\pgfsetfillcolor{textcolor}%
\pgftext[x=0.368523in,y=0.326389in,,top]{\color{textcolor}\rmfamily\fontsize{8.000000}{9.600000}\selectfont \(\displaystyle 0\)}%
\end{pgfscope}%
\begin{pgfscope}%
\pgfpathrectangle{\pgfqpoint{0.287500in}{0.375000in}}{\pgfqpoint{1.782500in}{2.265000in}}%
\pgfusepath{clip}%
\pgfsetroundcap%
\pgfsetroundjoin%
\pgfsetlinewidth{0.803000pt}%
\definecolor{currentstroke}{rgb}{1.000000,1.000000,1.000000}%
\pgfsetstrokecolor{currentstroke}%
\pgfsetdash{}{0pt}%
\pgfpathmoveto{\pgfqpoint{0.771621in}{0.375000in}}%
\pgfpathlineto{\pgfqpoint{0.771621in}{2.640000in}}%
\pgfusepath{stroke}%
\end{pgfscope}%
\begin{pgfscope}%
\definecolor{textcolor}{rgb}{0.150000,0.150000,0.150000}%
\pgfsetstrokecolor{textcolor}%
\pgfsetfillcolor{textcolor}%
\pgftext[x=0.771621in,y=0.326389in,,top]{\color{textcolor}\rmfamily\fontsize{8.000000}{9.600000}\selectfont \(\displaystyle 50\)}%
\end{pgfscope}%
\begin{pgfscope}%
\pgfpathrectangle{\pgfqpoint{0.287500in}{0.375000in}}{\pgfqpoint{1.782500in}{2.265000in}}%
\pgfusepath{clip}%
\pgfsetroundcap%
\pgfsetroundjoin%
\pgfsetlinewidth{0.803000pt}%
\definecolor{currentstroke}{rgb}{1.000000,1.000000,1.000000}%
\pgfsetstrokecolor{currentstroke}%
\pgfsetdash{}{0pt}%
\pgfpathmoveto{\pgfqpoint{1.174719in}{0.375000in}}%
\pgfpathlineto{\pgfqpoint{1.174719in}{2.640000in}}%
\pgfusepath{stroke}%
\end{pgfscope}%
\begin{pgfscope}%
\definecolor{textcolor}{rgb}{0.150000,0.150000,0.150000}%
\pgfsetstrokecolor{textcolor}%
\pgfsetfillcolor{textcolor}%
\pgftext[x=1.174719in,y=0.326389in,,top]{\color{textcolor}\rmfamily\fontsize{8.000000}{9.600000}\selectfont \(\displaystyle 100\)}%
\end{pgfscope}%
\begin{pgfscope}%
\pgfpathrectangle{\pgfqpoint{0.287500in}{0.375000in}}{\pgfqpoint{1.782500in}{2.265000in}}%
\pgfusepath{clip}%
\pgfsetroundcap%
\pgfsetroundjoin%
\pgfsetlinewidth{0.803000pt}%
\definecolor{currentstroke}{rgb}{1.000000,1.000000,1.000000}%
\pgfsetstrokecolor{currentstroke}%
\pgfsetdash{}{0pt}%
\pgfpathmoveto{\pgfqpoint{1.577817in}{0.375000in}}%
\pgfpathlineto{\pgfqpoint{1.577817in}{2.640000in}}%
\pgfusepath{stroke}%
\end{pgfscope}%
\begin{pgfscope}%
\definecolor{textcolor}{rgb}{0.150000,0.150000,0.150000}%
\pgfsetstrokecolor{textcolor}%
\pgfsetfillcolor{textcolor}%
\pgftext[x=1.577817in,y=0.326389in,,top]{\color{textcolor}\rmfamily\fontsize{8.000000}{9.600000}\selectfont \(\displaystyle 150\)}%
\end{pgfscope}%
\begin{pgfscope}%
\pgfpathrectangle{\pgfqpoint{0.287500in}{0.375000in}}{\pgfqpoint{1.782500in}{2.265000in}}%
\pgfusepath{clip}%
\pgfsetroundcap%
\pgfsetroundjoin%
\pgfsetlinewidth{0.803000pt}%
\definecolor{currentstroke}{rgb}{1.000000,1.000000,1.000000}%
\pgfsetstrokecolor{currentstroke}%
\pgfsetdash{}{0pt}%
\pgfpathmoveto{\pgfqpoint{1.980915in}{0.375000in}}%
\pgfpathlineto{\pgfqpoint{1.980915in}{2.640000in}}%
\pgfusepath{stroke}%
\end{pgfscope}%
\begin{pgfscope}%
\definecolor{textcolor}{rgb}{0.150000,0.150000,0.150000}%
\pgfsetstrokecolor{textcolor}%
\pgfsetfillcolor{textcolor}%
\pgftext[x=1.980915in,y=0.326389in,,top]{\color{textcolor}\rmfamily\fontsize{8.000000}{9.600000}\selectfont \(\displaystyle 200\)}%
\end{pgfscope}%
\begin{pgfscope}%
\definecolor{textcolor}{rgb}{0.150000,0.150000,0.150000}%
\pgfsetstrokecolor{textcolor}%
\pgfsetfillcolor{textcolor}%
\pgftext[x=1.178750in,y=0.163303in,,top]{\color{textcolor}\rmfamily\fontsize{8.000000}{9.600000}\selectfont Step}%
\end{pgfscope}%
\begin{pgfscope}%
\pgfpathrectangle{\pgfqpoint{0.287500in}{0.375000in}}{\pgfqpoint{1.782500in}{2.265000in}}%
\pgfusepath{clip}%
\pgfsetroundcap%
\pgfsetroundjoin%
\pgfsetlinewidth{0.803000pt}%
\definecolor{currentstroke}{rgb}{1.000000,1.000000,1.000000}%
\pgfsetstrokecolor{currentstroke}%
\pgfsetdash{}{0pt}%
\pgfpathmoveto{\pgfqpoint{0.287500in}{0.634590in}}%
\pgfpathlineto{\pgfqpoint{2.070000in}{0.634590in}}%
\pgfusepath{stroke}%
\end{pgfscope}%
\begin{pgfscope}%
\definecolor{textcolor}{rgb}{0.150000,0.150000,0.150000}%
\pgfsetstrokecolor{textcolor}%
\pgfsetfillcolor{textcolor}%
\pgftext[x=-0.017284in,y=0.592381in,left,base]{\color{textcolor}\rmfamily\fontsize{8.000000}{9.600000}\selectfont \(\displaystyle 10^{-7}\)}%
\end{pgfscope}%
\begin{pgfscope}%
\pgfpathrectangle{\pgfqpoint{0.287500in}{0.375000in}}{\pgfqpoint{1.782500in}{2.265000in}}%
\pgfusepath{clip}%
\pgfsetroundcap%
\pgfsetroundjoin%
\pgfsetlinewidth{0.803000pt}%
\definecolor{currentstroke}{rgb}{1.000000,1.000000,1.000000}%
\pgfsetstrokecolor{currentstroke}%
\pgfsetdash{}{0pt}%
\pgfpathmoveto{\pgfqpoint{0.287500in}{0.911591in}}%
\pgfpathlineto{\pgfqpoint{2.070000in}{0.911591in}}%
\pgfusepath{stroke}%
\end{pgfscope}%
\begin{pgfscope}%
\definecolor{textcolor}{rgb}{0.150000,0.150000,0.150000}%
\pgfsetstrokecolor{textcolor}%
\pgfsetfillcolor{textcolor}%
\pgftext[x=-0.017284in,y=0.869382in,left,base]{\color{textcolor}\rmfamily\fontsize{8.000000}{9.600000}\selectfont \(\displaystyle 10^{-6}\)}%
\end{pgfscope}%
\begin{pgfscope}%
\pgfpathrectangle{\pgfqpoint{0.287500in}{0.375000in}}{\pgfqpoint{1.782500in}{2.265000in}}%
\pgfusepath{clip}%
\pgfsetroundcap%
\pgfsetroundjoin%
\pgfsetlinewidth{0.803000pt}%
\definecolor{currentstroke}{rgb}{1.000000,1.000000,1.000000}%
\pgfsetstrokecolor{currentstroke}%
\pgfsetdash{}{0pt}%
\pgfpathmoveto{\pgfqpoint{0.287500in}{1.188592in}}%
\pgfpathlineto{\pgfqpoint{2.070000in}{1.188592in}}%
\pgfusepath{stroke}%
\end{pgfscope}%
\begin{pgfscope}%
\definecolor{textcolor}{rgb}{0.150000,0.150000,0.150000}%
\pgfsetstrokecolor{textcolor}%
\pgfsetfillcolor{textcolor}%
\pgftext[x=-0.017284in,y=1.146382in,left,base]{\color{textcolor}\rmfamily\fontsize{8.000000}{9.600000}\selectfont \(\displaystyle 10^{-5}\)}%
\end{pgfscope}%
\begin{pgfscope}%
\pgfpathrectangle{\pgfqpoint{0.287500in}{0.375000in}}{\pgfqpoint{1.782500in}{2.265000in}}%
\pgfusepath{clip}%
\pgfsetroundcap%
\pgfsetroundjoin%
\pgfsetlinewidth{0.803000pt}%
\definecolor{currentstroke}{rgb}{1.000000,1.000000,1.000000}%
\pgfsetstrokecolor{currentstroke}%
\pgfsetdash{}{0pt}%
\pgfpathmoveto{\pgfqpoint{0.287500in}{1.465593in}}%
\pgfpathlineto{\pgfqpoint{2.070000in}{1.465593in}}%
\pgfusepath{stroke}%
\end{pgfscope}%
\begin{pgfscope}%
\definecolor{textcolor}{rgb}{0.150000,0.150000,0.150000}%
\pgfsetstrokecolor{textcolor}%
\pgfsetfillcolor{textcolor}%
\pgftext[x=-0.017284in,y=1.423383in,left,base]{\color{textcolor}\rmfamily\fontsize{8.000000}{9.600000}\selectfont \(\displaystyle 10^{-4}\)}%
\end{pgfscope}%
\begin{pgfscope}%
\pgfpathrectangle{\pgfqpoint{0.287500in}{0.375000in}}{\pgfqpoint{1.782500in}{2.265000in}}%
\pgfusepath{clip}%
\pgfsetroundcap%
\pgfsetroundjoin%
\pgfsetlinewidth{0.803000pt}%
\definecolor{currentstroke}{rgb}{1.000000,1.000000,1.000000}%
\pgfsetstrokecolor{currentstroke}%
\pgfsetdash{}{0pt}%
\pgfpathmoveto{\pgfqpoint{0.287500in}{1.742593in}}%
\pgfpathlineto{\pgfqpoint{2.070000in}{1.742593in}}%
\pgfusepath{stroke}%
\end{pgfscope}%
\begin{pgfscope}%
\definecolor{textcolor}{rgb}{0.150000,0.150000,0.150000}%
\pgfsetstrokecolor{textcolor}%
\pgfsetfillcolor{textcolor}%
\pgftext[x=-0.017284in,y=1.700384in,left,base]{\color{textcolor}\rmfamily\fontsize{8.000000}{9.600000}\selectfont \(\displaystyle 10^{-3}\)}%
\end{pgfscope}%
\begin{pgfscope}%
\pgfpathrectangle{\pgfqpoint{0.287500in}{0.375000in}}{\pgfqpoint{1.782500in}{2.265000in}}%
\pgfusepath{clip}%
\pgfsetroundcap%
\pgfsetroundjoin%
\pgfsetlinewidth{0.803000pt}%
\definecolor{currentstroke}{rgb}{1.000000,1.000000,1.000000}%
\pgfsetstrokecolor{currentstroke}%
\pgfsetdash{}{0pt}%
\pgfpathmoveto{\pgfqpoint{0.287500in}{2.019594in}}%
\pgfpathlineto{\pgfqpoint{2.070000in}{2.019594in}}%
\pgfusepath{stroke}%
\end{pgfscope}%
\begin{pgfscope}%
\definecolor{textcolor}{rgb}{0.150000,0.150000,0.150000}%
\pgfsetstrokecolor{textcolor}%
\pgfsetfillcolor{textcolor}%
\pgftext[x=-0.017284in,y=1.977385in,left,base]{\color{textcolor}\rmfamily\fontsize{8.000000}{9.600000}\selectfont \(\displaystyle 10^{-2}\)}%
\end{pgfscope}%
\begin{pgfscope}%
\pgfpathrectangle{\pgfqpoint{0.287500in}{0.375000in}}{\pgfqpoint{1.782500in}{2.265000in}}%
\pgfusepath{clip}%
\pgfsetroundcap%
\pgfsetroundjoin%
\pgfsetlinewidth{0.803000pt}%
\definecolor{currentstroke}{rgb}{1.000000,1.000000,1.000000}%
\pgfsetstrokecolor{currentstroke}%
\pgfsetdash{}{0pt}%
\pgfpathmoveto{\pgfqpoint{0.287500in}{2.296595in}}%
\pgfpathlineto{\pgfqpoint{2.070000in}{2.296595in}}%
\pgfusepath{stroke}%
\end{pgfscope}%
\begin{pgfscope}%
\definecolor{textcolor}{rgb}{0.150000,0.150000,0.150000}%
\pgfsetstrokecolor{textcolor}%
\pgfsetfillcolor{textcolor}%
\pgftext[x=-0.017284in,y=2.254386in,left,base]{\color{textcolor}\rmfamily\fontsize{8.000000}{9.600000}\selectfont \(\displaystyle 10^{-1}\)}%
\end{pgfscope}%
\begin{pgfscope}%
\pgfpathrectangle{\pgfqpoint{0.287500in}{0.375000in}}{\pgfqpoint{1.782500in}{2.265000in}}%
\pgfusepath{clip}%
\pgfsetroundcap%
\pgfsetroundjoin%
\pgfsetlinewidth{0.803000pt}%
\definecolor{currentstroke}{rgb}{1.000000,1.000000,1.000000}%
\pgfsetstrokecolor{currentstroke}%
\pgfsetdash{}{0pt}%
\pgfpathmoveto{\pgfqpoint{0.287500in}{2.573596in}}%
\pgfpathlineto{\pgfqpoint{2.070000in}{2.573596in}}%
\pgfusepath{stroke}%
\end{pgfscope}%
\begin{pgfscope}%
\definecolor{textcolor}{rgb}{0.150000,0.150000,0.150000}%
\pgfsetstrokecolor{textcolor}%
\pgfsetfillcolor{textcolor}%
\pgftext[x=0.062962in,y=2.531386in,left,base]{\color{textcolor}\rmfamily\fontsize{8.000000}{9.600000}\selectfont \(\displaystyle 10^{0}\)}%
\end{pgfscope}%
\begin{pgfscope}%
\definecolor{textcolor}{rgb}{0.150000,0.150000,0.150000}%
\pgfsetstrokecolor{textcolor}%
\pgfsetfillcolor{textcolor}%
\pgftext[x=-0.072840in,y=1.507500in,,bottom,rotate=90.000000]{\color{textcolor}\rmfamily\fontsize{8.000000}{9.600000}\selectfont Simple Regret}%
\end{pgfscope}%
\begin{pgfscope}%
\pgfpathrectangle{\pgfqpoint{0.287500in}{0.375000in}}{\pgfqpoint{1.782500in}{2.265000in}}%
\pgfusepath{clip}%
\pgfsetbuttcap%
\pgfsetroundjoin%
\definecolor{currentfill}{rgb}{0.121569,0.466667,0.705882}%
\pgfsetfillcolor{currentfill}%
\pgfsetfillopacity{0.200000}%
\pgfsetlinewidth{0.000000pt}%
\definecolor{currentstroke}{rgb}{0.000000,0.000000,0.000000}%
\pgfsetstrokecolor{currentstroke}%
\pgfsetdash{}{0pt}%
\pgfpathmoveto{\pgfqpoint{0.368523in}{2.417519in}}%
\pgfpathlineto{\pgfqpoint{0.368523in}{2.496347in}}%
\pgfpathlineto{\pgfqpoint{0.376585in}{2.468603in}}%
\pgfpathlineto{\pgfqpoint{0.384647in}{2.446383in}}%
\pgfpathlineto{\pgfqpoint{0.392709in}{2.433408in}}%
\pgfpathlineto{\pgfqpoint{0.400771in}{2.433408in}}%
\pgfpathlineto{\pgfqpoint{0.408833in}{2.433408in}}%
\pgfpathlineto{\pgfqpoint{0.416895in}{2.405992in}}%
\pgfpathlineto{\pgfqpoint{0.424956in}{2.361739in}}%
\pgfpathlineto{\pgfqpoint{0.433018in}{2.361173in}}%
\pgfpathlineto{\pgfqpoint{0.441080in}{2.360513in}}%
\pgfpathlineto{\pgfqpoint{0.449142in}{2.352573in}}%
\pgfpathlineto{\pgfqpoint{0.457204in}{2.351434in}}%
\pgfpathlineto{\pgfqpoint{0.465266in}{2.334018in}}%
\pgfpathlineto{\pgfqpoint{0.473328in}{2.331450in}}%
\pgfpathlineto{\pgfqpoint{0.481390in}{2.318775in}}%
\pgfpathlineto{\pgfqpoint{0.489452in}{2.307902in}}%
\pgfpathlineto{\pgfqpoint{0.497514in}{2.307902in}}%
\pgfpathlineto{\pgfqpoint{0.505576in}{2.249683in}}%
\pgfpathlineto{\pgfqpoint{0.513638in}{2.249683in}}%
\pgfpathlineto{\pgfqpoint{0.521700in}{2.242643in}}%
\pgfpathlineto{\pgfqpoint{0.529762in}{2.241489in}}%
\pgfpathlineto{\pgfqpoint{0.537824in}{2.241489in}}%
\pgfpathlineto{\pgfqpoint{0.545886in}{2.241489in}}%
\pgfpathlineto{\pgfqpoint{0.553948in}{2.241489in}}%
\pgfpathlineto{\pgfqpoint{0.562010in}{2.239758in}}%
\pgfpathlineto{\pgfqpoint{0.570072in}{2.230542in}}%
\pgfpathlineto{\pgfqpoint{0.578134in}{2.223185in}}%
\pgfpathlineto{\pgfqpoint{0.586196in}{2.223185in}}%
\pgfpathlineto{\pgfqpoint{0.594258in}{2.223185in}}%
\pgfpathlineto{\pgfqpoint{0.602320in}{2.216598in}}%
\pgfpathlineto{\pgfqpoint{0.610382in}{2.212616in}}%
\pgfpathlineto{\pgfqpoint{0.618444in}{2.212616in}}%
\pgfpathlineto{\pgfqpoint{0.626506in}{2.212616in}}%
\pgfpathlineto{\pgfqpoint{0.634568in}{2.212616in}}%
\pgfpathlineto{\pgfqpoint{0.642629in}{2.212616in}}%
\pgfpathlineto{\pgfqpoint{0.650691in}{2.189606in}}%
\pgfpathlineto{\pgfqpoint{0.658753in}{2.189606in}}%
\pgfpathlineto{\pgfqpoint{0.666815in}{2.189278in}}%
\pgfpathlineto{\pgfqpoint{0.674877in}{2.189278in}}%
\pgfpathlineto{\pgfqpoint{0.682939in}{2.138598in}}%
\pgfpathlineto{\pgfqpoint{0.691001in}{2.138598in}}%
\pgfpathlineto{\pgfqpoint{0.699063in}{2.138598in}}%
\pgfpathlineto{\pgfqpoint{0.707125in}{2.138598in}}%
\pgfpathlineto{\pgfqpoint{0.715187in}{2.090551in}}%
\pgfpathlineto{\pgfqpoint{0.723249in}{2.090551in}}%
\pgfpathlineto{\pgfqpoint{0.731311in}{2.090551in}}%
\pgfpathlineto{\pgfqpoint{0.739373in}{2.085300in}}%
\pgfpathlineto{\pgfqpoint{0.747435in}{2.085300in}}%
\pgfpathlineto{\pgfqpoint{0.755497in}{2.085300in}}%
\pgfpathlineto{\pgfqpoint{0.763559in}{2.082236in}}%
\pgfpathlineto{\pgfqpoint{0.771621in}{2.071467in}}%
\pgfpathlineto{\pgfqpoint{0.779683in}{2.071243in}}%
\pgfpathlineto{\pgfqpoint{0.787745in}{2.065477in}}%
\pgfpathlineto{\pgfqpoint{0.795807in}{1.982083in}}%
\pgfpathlineto{\pgfqpoint{0.803869in}{1.982083in}}%
\pgfpathlineto{\pgfqpoint{0.811931in}{1.982083in}}%
\pgfpathlineto{\pgfqpoint{0.819993in}{1.982083in}}%
\pgfpathlineto{\pgfqpoint{0.828055in}{1.982083in}}%
\pgfpathlineto{\pgfqpoint{0.836117in}{1.942549in}}%
\pgfpathlineto{\pgfqpoint{0.844179in}{1.942549in}}%
\pgfpathlineto{\pgfqpoint{0.852241in}{1.942549in}}%
\pgfpathlineto{\pgfqpoint{0.860302in}{1.937929in}}%
\pgfpathlineto{\pgfqpoint{0.868364in}{1.937533in}}%
\pgfpathlineto{\pgfqpoint{0.876426in}{1.937533in}}%
\pgfpathlineto{\pgfqpoint{0.884488in}{1.929593in}}%
\pgfpathlineto{\pgfqpoint{0.892550in}{1.929593in}}%
\pgfpathlineto{\pgfqpoint{0.900612in}{1.929593in}}%
\pgfpathlineto{\pgfqpoint{0.908674in}{1.929593in}}%
\pgfpathlineto{\pgfqpoint{0.916736in}{1.929593in}}%
\pgfpathlineto{\pgfqpoint{0.924798in}{1.929593in}}%
\pgfpathlineto{\pgfqpoint{0.932860in}{1.929593in}}%
\pgfpathlineto{\pgfqpoint{0.940922in}{1.929593in}}%
\pgfpathlineto{\pgfqpoint{0.948984in}{1.929593in}}%
\pgfpathlineto{\pgfqpoint{0.957046in}{1.929593in}}%
\pgfpathlineto{\pgfqpoint{0.965108in}{1.929035in}}%
\pgfpathlineto{\pgfqpoint{0.973170in}{1.929035in}}%
\pgfpathlineto{\pgfqpoint{0.981232in}{1.929035in}}%
\pgfpathlineto{\pgfqpoint{0.989294in}{1.929035in}}%
\pgfpathlineto{\pgfqpoint{0.997356in}{1.929035in}}%
\pgfpathlineto{\pgfqpoint{1.005418in}{1.929035in}}%
\pgfpathlineto{\pgfqpoint{1.013480in}{1.929035in}}%
\pgfpathlineto{\pgfqpoint{1.021542in}{1.929035in}}%
\pgfpathlineto{\pgfqpoint{1.029604in}{1.929035in}}%
\pgfpathlineto{\pgfqpoint{1.037666in}{1.929035in}}%
\pgfpathlineto{\pgfqpoint{1.045728in}{1.929035in}}%
\pgfpathlineto{\pgfqpoint{1.053790in}{1.929035in}}%
\pgfpathlineto{\pgfqpoint{1.061852in}{1.929035in}}%
\pgfpathlineto{\pgfqpoint{1.069914in}{1.929035in}}%
\pgfpathlineto{\pgfqpoint{1.077975in}{1.929035in}}%
\pgfpathlineto{\pgfqpoint{1.086037in}{1.929035in}}%
\pgfpathlineto{\pgfqpoint{1.094099in}{1.929035in}}%
\pgfpathlineto{\pgfqpoint{1.102161in}{1.929035in}}%
\pgfpathlineto{\pgfqpoint{1.110223in}{1.929035in}}%
\pgfpathlineto{\pgfqpoint{1.118285in}{1.929035in}}%
\pgfpathlineto{\pgfqpoint{1.126347in}{1.929035in}}%
\pgfpathlineto{\pgfqpoint{1.134409in}{1.929035in}}%
\pgfpathlineto{\pgfqpoint{1.142471in}{1.929035in}}%
\pgfpathlineto{\pgfqpoint{1.150533in}{1.929035in}}%
\pgfpathlineto{\pgfqpoint{1.158595in}{1.929035in}}%
\pgfpathlineto{\pgfqpoint{1.166657in}{1.929035in}}%
\pgfpathlineto{\pgfqpoint{1.174719in}{1.929035in}}%
\pgfpathlineto{\pgfqpoint{1.182781in}{1.929035in}}%
\pgfpathlineto{\pgfqpoint{1.190843in}{1.929035in}}%
\pgfpathlineto{\pgfqpoint{1.198905in}{1.929035in}}%
\pgfpathlineto{\pgfqpoint{1.206967in}{1.929035in}}%
\pgfpathlineto{\pgfqpoint{1.215029in}{1.929035in}}%
\pgfpathlineto{\pgfqpoint{1.223091in}{1.929035in}}%
\pgfpathlineto{\pgfqpoint{1.231153in}{1.929035in}}%
\pgfpathlineto{\pgfqpoint{1.239215in}{1.929035in}}%
\pgfpathlineto{\pgfqpoint{1.247277in}{1.929035in}}%
\pgfpathlineto{\pgfqpoint{1.255339in}{1.929035in}}%
\pgfpathlineto{\pgfqpoint{1.263401in}{1.929035in}}%
\pgfpathlineto{\pgfqpoint{1.271463in}{1.929035in}}%
\pgfpathlineto{\pgfqpoint{1.279525in}{1.929035in}}%
\pgfpathlineto{\pgfqpoint{1.287586in}{1.929035in}}%
\pgfpathlineto{\pgfqpoint{1.295648in}{1.929035in}}%
\pgfpathlineto{\pgfqpoint{1.303710in}{1.929035in}}%
\pgfpathlineto{\pgfqpoint{1.311772in}{1.929035in}}%
\pgfpathlineto{\pgfqpoint{1.319834in}{1.929035in}}%
\pgfpathlineto{\pgfqpoint{1.327896in}{1.929035in}}%
\pgfpathlineto{\pgfqpoint{1.335958in}{1.929035in}}%
\pgfpathlineto{\pgfqpoint{1.344020in}{1.929035in}}%
\pgfpathlineto{\pgfqpoint{1.352082in}{1.929035in}}%
\pgfpathlineto{\pgfqpoint{1.360144in}{1.916985in}}%
\pgfpathlineto{\pgfqpoint{1.368206in}{1.916985in}}%
\pgfpathlineto{\pgfqpoint{1.376268in}{1.916985in}}%
\pgfpathlineto{\pgfqpoint{1.384330in}{1.916985in}}%
\pgfpathlineto{\pgfqpoint{1.392392in}{1.916985in}}%
\pgfpathlineto{\pgfqpoint{1.400454in}{1.916985in}}%
\pgfpathlineto{\pgfqpoint{1.408516in}{1.916985in}}%
\pgfpathlineto{\pgfqpoint{1.416578in}{1.916985in}}%
\pgfpathlineto{\pgfqpoint{1.424640in}{1.916985in}}%
\pgfpathlineto{\pgfqpoint{1.432702in}{1.916985in}}%
\pgfpathlineto{\pgfqpoint{1.440764in}{1.916985in}}%
\pgfpathlineto{\pgfqpoint{1.448826in}{1.916985in}}%
\pgfpathlineto{\pgfqpoint{1.456888in}{1.916985in}}%
\pgfpathlineto{\pgfqpoint{1.464950in}{1.916985in}}%
\pgfpathlineto{\pgfqpoint{1.473012in}{1.916985in}}%
\pgfpathlineto{\pgfqpoint{1.481074in}{1.916985in}}%
\pgfpathlineto{\pgfqpoint{1.489136in}{1.916985in}}%
\pgfpathlineto{\pgfqpoint{1.497198in}{1.916985in}}%
\pgfpathlineto{\pgfqpoint{1.505259in}{1.916985in}}%
\pgfpathlineto{\pgfqpoint{1.513321in}{1.916985in}}%
\pgfpathlineto{\pgfqpoint{1.521383in}{1.916985in}}%
\pgfpathlineto{\pgfqpoint{1.529445in}{1.916985in}}%
\pgfpathlineto{\pgfqpoint{1.537507in}{1.916985in}}%
\pgfpathlineto{\pgfqpoint{1.545569in}{1.916985in}}%
\pgfpathlineto{\pgfqpoint{1.553631in}{1.916985in}}%
\pgfpathlineto{\pgfqpoint{1.561693in}{1.916985in}}%
\pgfpathlineto{\pgfqpoint{1.569755in}{1.916985in}}%
\pgfpathlineto{\pgfqpoint{1.577817in}{1.916985in}}%
\pgfpathlineto{\pgfqpoint{1.585879in}{1.916985in}}%
\pgfpathlineto{\pgfqpoint{1.593941in}{1.916985in}}%
\pgfpathlineto{\pgfqpoint{1.602003in}{1.916985in}}%
\pgfpathlineto{\pgfqpoint{1.610065in}{1.916985in}}%
\pgfpathlineto{\pgfqpoint{1.618127in}{1.916985in}}%
\pgfpathlineto{\pgfqpoint{1.626189in}{1.916985in}}%
\pgfpathlineto{\pgfqpoint{1.634251in}{1.916985in}}%
\pgfpathlineto{\pgfqpoint{1.642313in}{1.916985in}}%
\pgfpathlineto{\pgfqpoint{1.650375in}{1.916985in}}%
\pgfpathlineto{\pgfqpoint{1.658437in}{1.916985in}}%
\pgfpathlineto{\pgfqpoint{1.666499in}{1.916985in}}%
\pgfpathlineto{\pgfqpoint{1.674561in}{1.916985in}}%
\pgfpathlineto{\pgfqpoint{1.682623in}{1.916985in}}%
\pgfpathlineto{\pgfqpoint{1.690685in}{1.916985in}}%
\pgfpathlineto{\pgfqpoint{1.698747in}{1.916985in}}%
\pgfpathlineto{\pgfqpoint{1.706809in}{1.916985in}}%
\pgfpathlineto{\pgfqpoint{1.714871in}{1.916985in}}%
\pgfpathlineto{\pgfqpoint{1.722932in}{1.916985in}}%
\pgfpathlineto{\pgfqpoint{1.730994in}{1.916985in}}%
\pgfpathlineto{\pgfqpoint{1.739056in}{1.916985in}}%
\pgfpathlineto{\pgfqpoint{1.747118in}{1.916985in}}%
\pgfpathlineto{\pgfqpoint{1.755180in}{1.916985in}}%
\pgfpathlineto{\pgfqpoint{1.763242in}{1.916985in}}%
\pgfpathlineto{\pgfqpoint{1.771304in}{1.916985in}}%
\pgfpathlineto{\pgfqpoint{1.779366in}{1.916985in}}%
\pgfpathlineto{\pgfqpoint{1.787428in}{1.916985in}}%
\pgfpathlineto{\pgfqpoint{1.795490in}{1.916985in}}%
\pgfpathlineto{\pgfqpoint{1.803552in}{1.916985in}}%
\pgfpathlineto{\pgfqpoint{1.811614in}{1.916985in}}%
\pgfpathlineto{\pgfqpoint{1.819676in}{1.916985in}}%
\pgfpathlineto{\pgfqpoint{1.827738in}{1.916985in}}%
\pgfpathlineto{\pgfqpoint{1.835800in}{1.916985in}}%
\pgfpathlineto{\pgfqpoint{1.843862in}{1.916985in}}%
\pgfpathlineto{\pgfqpoint{1.851924in}{1.916985in}}%
\pgfpathlineto{\pgfqpoint{1.859986in}{1.916985in}}%
\pgfpathlineto{\pgfqpoint{1.868048in}{1.916985in}}%
\pgfpathlineto{\pgfqpoint{1.876110in}{1.916985in}}%
\pgfpathlineto{\pgfqpoint{1.884172in}{1.916985in}}%
\pgfpathlineto{\pgfqpoint{1.892234in}{1.916985in}}%
\pgfpathlineto{\pgfqpoint{1.900296in}{1.916985in}}%
\pgfpathlineto{\pgfqpoint{1.908358in}{1.916985in}}%
\pgfpathlineto{\pgfqpoint{1.916420in}{1.916985in}}%
\pgfpathlineto{\pgfqpoint{1.924482in}{1.916985in}}%
\pgfpathlineto{\pgfqpoint{1.932544in}{1.916985in}}%
\pgfpathlineto{\pgfqpoint{1.940605in}{1.916985in}}%
\pgfpathlineto{\pgfqpoint{1.948667in}{1.916985in}}%
\pgfpathlineto{\pgfqpoint{1.956729in}{1.916985in}}%
\pgfpathlineto{\pgfqpoint{1.964791in}{1.916985in}}%
\pgfpathlineto{\pgfqpoint{1.972853in}{1.916985in}}%
\pgfpathlineto{\pgfqpoint{1.980915in}{1.916985in}}%
\pgfpathlineto{\pgfqpoint{1.988977in}{1.916985in}}%
\pgfpathlineto{\pgfqpoint{1.988977in}{1.449566in}}%
\pgfpathlineto{\pgfqpoint{1.988977in}{1.449566in}}%
\pgfpathlineto{\pgfqpoint{1.980915in}{1.449566in}}%
\pgfpathlineto{\pgfqpoint{1.972853in}{1.449566in}}%
\pgfpathlineto{\pgfqpoint{1.964791in}{1.449566in}}%
\pgfpathlineto{\pgfqpoint{1.956729in}{1.449566in}}%
\pgfpathlineto{\pgfqpoint{1.948667in}{1.449566in}}%
\pgfpathlineto{\pgfqpoint{1.940605in}{1.449566in}}%
\pgfpathlineto{\pgfqpoint{1.932544in}{1.449566in}}%
\pgfpathlineto{\pgfqpoint{1.924482in}{1.449566in}}%
\pgfpathlineto{\pgfqpoint{1.916420in}{1.449566in}}%
\pgfpathlineto{\pgfqpoint{1.908358in}{1.449566in}}%
\pgfpathlineto{\pgfqpoint{1.900296in}{1.449566in}}%
\pgfpathlineto{\pgfqpoint{1.892234in}{1.449566in}}%
\pgfpathlineto{\pgfqpoint{1.884172in}{1.449566in}}%
\pgfpathlineto{\pgfqpoint{1.876110in}{1.449566in}}%
\pgfpathlineto{\pgfqpoint{1.868048in}{1.449566in}}%
\pgfpathlineto{\pgfqpoint{1.859986in}{1.449566in}}%
\pgfpathlineto{\pgfqpoint{1.851924in}{1.449566in}}%
\pgfpathlineto{\pgfqpoint{1.843862in}{1.449566in}}%
\pgfpathlineto{\pgfqpoint{1.835800in}{1.449566in}}%
\pgfpathlineto{\pgfqpoint{1.827738in}{1.449566in}}%
\pgfpathlineto{\pgfqpoint{1.819676in}{1.449566in}}%
\pgfpathlineto{\pgfqpoint{1.811614in}{1.449566in}}%
\pgfpathlineto{\pgfqpoint{1.803552in}{1.449566in}}%
\pgfpathlineto{\pgfqpoint{1.795490in}{1.449566in}}%
\pgfpathlineto{\pgfqpoint{1.787428in}{1.449566in}}%
\pgfpathlineto{\pgfqpoint{1.779366in}{1.449566in}}%
\pgfpathlineto{\pgfqpoint{1.771304in}{1.449566in}}%
\pgfpathlineto{\pgfqpoint{1.763242in}{1.449566in}}%
\pgfpathlineto{\pgfqpoint{1.755180in}{1.449566in}}%
\pgfpathlineto{\pgfqpoint{1.747118in}{1.449566in}}%
\pgfpathlineto{\pgfqpoint{1.739056in}{1.449566in}}%
\pgfpathlineto{\pgfqpoint{1.730994in}{1.449566in}}%
\pgfpathlineto{\pgfqpoint{1.722932in}{1.449566in}}%
\pgfpathlineto{\pgfqpoint{1.714871in}{1.449566in}}%
\pgfpathlineto{\pgfqpoint{1.706809in}{1.449566in}}%
\pgfpathlineto{\pgfqpoint{1.698747in}{1.449566in}}%
\pgfpathlineto{\pgfqpoint{1.690685in}{1.449566in}}%
\pgfpathlineto{\pgfqpoint{1.682623in}{1.449566in}}%
\pgfpathlineto{\pgfqpoint{1.674561in}{1.449566in}}%
\pgfpathlineto{\pgfqpoint{1.666499in}{1.449566in}}%
\pgfpathlineto{\pgfqpoint{1.658437in}{1.449566in}}%
\pgfpathlineto{\pgfqpoint{1.650375in}{1.449566in}}%
\pgfpathlineto{\pgfqpoint{1.642313in}{1.449566in}}%
\pgfpathlineto{\pgfqpoint{1.634251in}{1.449566in}}%
\pgfpathlineto{\pgfqpoint{1.626189in}{1.449566in}}%
\pgfpathlineto{\pgfqpoint{1.618127in}{1.449566in}}%
\pgfpathlineto{\pgfqpoint{1.610065in}{1.449566in}}%
\pgfpathlineto{\pgfqpoint{1.602003in}{1.449566in}}%
\pgfpathlineto{\pgfqpoint{1.593941in}{1.449566in}}%
\pgfpathlineto{\pgfqpoint{1.585879in}{1.449566in}}%
\pgfpathlineto{\pgfqpoint{1.577817in}{1.449566in}}%
\pgfpathlineto{\pgfqpoint{1.569755in}{1.449566in}}%
\pgfpathlineto{\pgfqpoint{1.561693in}{1.449566in}}%
\pgfpathlineto{\pgfqpoint{1.553631in}{1.449566in}}%
\pgfpathlineto{\pgfqpoint{1.545569in}{1.449566in}}%
\pgfpathlineto{\pgfqpoint{1.537507in}{1.449566in}}%
\pgfpathlineto{\pgfqpoint{1.529445in}{1.449566in}}%
\pgfpathlineto{\pgfqpoint{1.521383in}{1.449566in}}%
\pgfpathlineto{\pgfqpoint{1.513321in}{1.449566in}}%
\pgfpathlineto{\pgfqpoint{1.505259in}{1.449566in}}%
\pgfpathlineto{\pgfqpoint{1.497198in}{1.449566in}}%
\pgfpathlineto{\pgfqpoint{1.489136in}{1.449566in}}%
\pgfpathlineto{\pgfqpoint{1.481074in}{1.449566in}}%
\pgfpathlineto{\pgfqpoint{1.473012in}{1.449566in}}%
\pgfpathlineto{\pgfqpoint{1.464950in}{1.449566in}}%
\pgfpathlineto{\pgfqpoint{1.456888in}{1.449566in}}%
\pgfpathlineto{\pgfqpoint{1.448826in}{1.449566in}}%
\pgfpathlineto{\pgfqpoint{1.440764in}{1.449566in}}%
\pgfpathlineto{\pgfqpoint{1.432702in}{1.449566in}}%
\pgfpathlineto{\pgfqpoint{1.424640in}{1.449566in}}%
\pgfpathlineto{\pgfqpoint{1.416578in}{1.449566in}}%
\pgfpathlineto{\pgfqpoint{1.408516in}{1.449566in}}%
\pgfpathlineto{\pgfqpoint{1.400454in}{1.449566in}}%
\pgfpathlineto{\pgfqpoint{1.392392in}{1.449566in}}%
\pgfpathlineto{\pgfqpoint{1.384330in}{1.449566in}}%
\pgfpathlineto{\pgfqpoint{1.376268in}{1.449566in}}%
\pgfpathlineto{\pgfqpoint{1.368206in}{1.449566in}}%
\pgfpathlineto{\pgfqpoint{1.360144in}{1.449566in}}%
\pgfpathlineto{\pgfqpoint{1.352082in}{1.661162in}}%
\pgfpathlineto{\pgfqpoint{1.344020in}{1.661162in}}%
\pgfpathlineto{\pgfqpoint{1.335958in}{1.661162in}}%
\pgfpathlineto{\pgfqpoint{1.327896in}{1.661162in}}%
\pgfpathlineto{\pgfqpoint{1.319834in}{1.661162in}}%
\pgfpathlineto{\pgfqpoint{1.311772in}{1.661162in}}%
\pgfpathlineto{\pgfqpoint{1.303710in}{1.661162in}}%
\pgfpathlineto{\pgfqpoint{1.295648in}{1.661162in}}%
\pgfpathlineto{\pgfqpoint{1.287586in}{1.661162in}}%
\pgfpathlineto{\pgfqpoint{1.279525in}{1.661162in}}%
\pgfpathlineto{\pgfqpoint{1.271463in}{1.661162in}}%
\pgfpathlineto{\pgfqpoint{1.263401in}{1.661162in}}%
\pgfpathlineto{\pgfqpoint{1.255339in}{1.661162in}}%
\pgfpathlineto{\pgfqpoint{1.247277in}{1.661162in}}%
\pgfpathlineto{\pgfqpoint{1.239215in}{1.661162in}}%
\pgfpathlineto{\pgfqpoint{1.231153in}{1.661162in}}%
\pgfpathlineto{\pgfqpoint{1.223091in}{1.661162in}}%
\pgfpathlineto{\pgfqpoint{1.215029in}{1.661162in}}%
\pgfpathlineto{\pgfqpoint{1.206967in}{1.661162in}}%
\pgfpathlineto{\pgfqpoint{1.198905in}{1.661162in}}%
\pgfpathlineto{\pgfqpoint{1.190843in}{1.661162in}}%
\pgfpathlineto{\pgfqpoint{1.182781in}{1.661162in}}%
\pgfpathlineto{\pgfqpoint{1.174719in}{1.661162in}}%
\pgfpathlineto{\pgfqpoint{1.166657in}{1.661162in}}%
\pgfpathlineto{\pgfqpoint{1.158595in}{1.661162in}}%
\pgfpathlineto{\pgfqpoint{1.150533in}{1.661162in}}%
\pgfpathlineto{\pgfqpoint{1.142471in}{1.661162in}}%
\pgfpathlineto{\pgfqpoint{1.134409in}{1.661162in}}%
\pgfpathlineto{\pgfqpoint{1.126347in}{1.661162in}}%
\pgfpathlineto{\pgfqpoint{1.118285in}{1.661162in}}%
\pgfpathlineto{\pgfqpoint{1.110223in}{1.661162in}}%
\pgfpathlineto{\pgfqpoint{1.102161in}{1.661162in}}%
\pgfpathlineto{\pgfqpoint{1.094099in}{1.661162in}}%
\pgfpathlineto{\pgfqpoint{1.086037in}{1.661162in}}%
\pgfpathlineto{\pgfqpoint{1.077975in}{1.661162in}}%
\pgfpathlineto{\pgfqpoint{1.069914in}{1.661162in}}%
\pgfpathlineto{\pgfqpoint{1.061852in}{1.661162in}}%
\pgfpathlineto{\pgfqpoint{1.053790in}{1.661162in}}%
\pgfpathlineto{\pgfqpoint{1.045728in}{1.661162in}}%
\pgfpathlineto{\pgfqpoint{1.037666in}{1.661162in}}%
\pgfpathlineto{\pgfqpoint{1.029604in}{1.661162in}}%
\pgfpathlineto{\pgfqpoint{1.021542in}{1.661162in}}%
\pgfpathlineto{\pgfqpoint{1.013480in}{1.661162in}}%
\pgfpathlineto{\pgfqpoint{1.005418in}{1.661162in}}%
\pgfpathlineto{\pgfqpoint{0.997356in}{1.661162in}}%
\pgfpathlineto{\pgfqpoint{0.989294in}{1.661162in}}%
\pgfpathlineto{\pgfqpoint{0.981232in}{1.661162in}}%
\pgfpathlineto{\pgfqpoint{0.973170in}{1.661162in}}%
\pgfpathlineto{\pgfqpoint{0.965108in}{1.661162in}}%
\pgfpathlineto{\pgfqpoint{0.957046in}{1.670086in}}%
\pgfpathlineto{\pgfqpoint{0.948984in}{1.670086in}}%
\pgfpathlineto{\pgfqpoint{0.940922in}{1.670086in}}%
\pgfpathlineto{\pgfqpoint{0.932860in}{1.670086in}}%
\pgfpathlineto{\pgfqpoint{0.924798in}{1.670086in}}%
\pgfpathlineto{\pgfqpoint{0.916736in}{1.670086in}}%
\pgfpathlineto{\pgfqpoint{0.908674in}{1.670086in}}%
\pgfpathlineto{\pgfqpoint{0.900612in}{1.670086in}}%
\pgfpathlineto{\pgfqpoint{0.892550in}{1.670086in}}%
\pgfpathlineto{\pgfqpoint{0.884488in}{1.670086in}}%
\pgfpathlineto{\pgfqpoint{0.876426in}{1.692584in}}%
\pgfpathlineto{\pgfqpoint{0.868364in}{1.692584in}}%
\pgfpathlineto{\pgfqpoint{0.860302in}{1.698110in}}%
\pgfpathlineto{\pgfqpoint{0.852241in}{1.738473in}}%
\pgfpathlineto{\pgfqpoint{0.844179in}{1.738473in}}%
\pgfpathlineto{\pgfqpoint{0.836117in}{1.738473in}}%
\pgfpathlineto{\pgfqpoint{0.828055in}{1.696270in}}%
\pgfpathlineto{\pgfqpoint{0.819993in}{1.696270in}}%
\pgfpathlineto{\pgfqpoint{0.811931in}{1.696270in}}%
\pgfpathlineto{\pgfqpoint{0.803869in}{1.696270in}}%
\pgfpathlineto{\pgfqpoint{0.795807in}{1.696270in}}%
\pgfpathlineto{\pgfqpoint{0.787745in}{1.794391in}}%
\pgfpathlineto{\pgfqpoint{0.779683in}{1.807211in}}%
\pgfpathlineto{\pgfqpoint{0.771621in}{1.810952in}}%
\pgfpathlineto{\pgfqpoint{0.763559in}{1.888895in}}%
\pgfpathlineto{\pgfqpoint{0.755497in}{1.913388in}}%
\pgfpathlineto{\pgfqpoint{0.747435in}{1.913388in}}%
\pgfpathlineto{\pgfqpoint{0.739373in}{1.913388in}}%
\pgfpathlineto{\pgfqpoint{0.731311in}{1.944564in}}%
\pgfpathlineto{\pgfqpoint{0.723249in}{1.944564in}}%
\pgfpathlineto{\pgfqpoint{0.715187in}{1.944564in}}%
\pgfpathlineto{\pgfqpoint{0.707125in}{1.988609in}}%
\pgfpathlineto{\pgfqpoint{0.699063in}{1.988609in}}%
\pgfpathlineto{\pgfqpoint{0.691001in}{1.988609in}}%
\pgfpathlineto{\pgfqpoint{0.682939in}{1.988609in}}%
\pgfpathlineto{\pgfqpoint{0.674877in}{2.032370in}}%
\pgfpathlineto{\pgfqpoint{0.666815in}{2.032370in}}%
\pgfpathlineto{\pgfqpoint{0.658753in}{2.035267in}}%
\pgfpathlineto{\pgfqpoint{0.650691in}{2.035267in}}%
\pgfpathlineto{\pgfqpoint{0.642629in}{2.024560in}}%
\pgfpathlineto{\pgfqpoint{0.634568in}{2.024560in}}%
\pgfpathlineto{\pgfqpoint{0.626506in}{2.024560in}}%
\pgfpathlineto{\pgfqpoint{0.618444in}{2.024560in}}%
\pgfpathlineto{\pgfqpoint{0.610382in}{2.024560in}}%
\pgfpathlineto{\pgfqpoint{0.602320in}{2.036116in}}%
\pgfpathlineto{\pgfqpoint{0.594258in}{2.072747in}}%
\pgfpathlineto{\pgfqpoint{0.586196in}{2.072747in}}%
\pgfpathlineto{\pgfqpoint{0.578134in}{2.072747in}}%
\pgfpathlineto{\pgfqpoint{0.570072in}{2.093426in}}%
\pgfpathlineto{\pgfqpoint{0.562010in}{2.126082in}}%
\pgfpathlineto{\pgfqpoint{0.553948in}{2.125680in}}%
\pgfpathlineto{\pgfqpoint{0.545886in}{2.125680in}}%
\pgfpathlineto{\pgfqpoint{0.537824in}{2.125680in}}%
\pgfpathlineto{\pgfqpoint{0.529762in}{2.125680in}}%
\pgfpathlineto{\pgfqpoint{0.521700in}{2.125390in}}%
\pgfpathlineto{\pgfqpoint{0.513638in}{2.139826in}}%
\pgfpathlineto{\pgfqpoint{0.505576in}{2.139826in}}%
\pgfpathlineto{\pgfqpoint{0.497514in}{2.147091in}}%
\pgfpathlineto{\pgfqpoint{0.489452in}{2.147091in}}%
\pgfpathlineto{\pgfqpoint{0.481390in}{2.201292in}}%
\pgfpathlineto{\pgfqpoint{0.473328in}{2.232751in}}%
\pgfpathlineto{\pgfqpoint{0.465266in}{2.238512in}}%
\pgfpathlineto{\pgfqpoint{0.457204in}{2.260812in}}%
\pgfpathlineto{\pgfqpoint{0.449142in}{2.264567in}}%
\pgfpathlineto{\pgfqpoint{0.441080in}{2.288635in}}%
\pgfpathlineto{\pgfqpoint{0.433018in}{2.289428in}}%
\pgfpathlineto{\pgfqpoint{0.424956in}{2.291125in}}%
\pgfpathlineto{\pgfqpoint{0.416895in}{2.319606in}}%
\pgfpathlineto{\pgfqpoint{0.408833in}{2.352062in}}%
\pgfpathlineto{\pgfqpoint{0.400771in}{2.352062in}}%
\pgfpathlineto{\pgfqpoint{0.392709in}{2.352062in}}%
\pgfpathlineto{\pgfqpoint{0.384647in}{2.366994in}}%
\pgfpathlineto{\pgfqpoint{0.376585in}{2.402866in}}%
\pgfpathlineto{\pgfqpoint{0.368523in}{2.417519in}}%
\pgfpathclose%
\pgfusepath{fill}%
\end{pgfscope}%
\begin{pgfscope}%
\pgfpathrectangle{\pgfqpoint{0.287500in}{0.375000in}}{\pgfqpoint{1.782500in}{2.265000in}}%
\pgfusepath{clip}%
\pgfsetbuttcap%
\pgfsetroundjoin%
\definecolor{currentfill}{rgb}{1.000000,0.498039,0.054902}%
\pgfsetfillcolor{currentfill}%
\pgfsetfillopacity{0.200000}%
\pgfsetlinewidth{0.000000pt}%
\definecolor{currentstroke}{rgb}{0.000000,0.000000,0.000000}%
\pgfsetstrokecolor{currentstroke}%
\pgfsetdash{}{0pt}%
\pgfpathmoveto{\pgfqpoint{0.368523in}{2.413106in}}%
\pgfpathlineto{\pgfqpoint{0.368523in}{2.503926in}}%
\pgfpathlineto{\pgfqpoint{0.376585in}{2.447606in}}%
\pgfpathlineto{\pgfqpoint{0.384647in}{2.445873in}}%
\pgfpathlineto{\pgfqpoint{0.392709in}{2.416244in}}%
\pgfpathlineto{\pgfqpoint{0.400771in}{2.416244in}}%
\pgfpathlineto{\pgfqpoint{0.408833in}{2.416244in}}%
\pgfpathlineto{\pgfqpoint{0.416895in}{2.416244in}}%
\pgfpathlineto{\pgfqpoint{0.424956in}{2.416244in}}%
\pgfpathlineto{\pgfqpoint{0.433018in}{2.416244in}}%
\pgfpathlineto{\pgfqpoint{0.441080in}{2.416244in}}%
\pgfpathlineto{\pgfqpoint{0.449142in}{2.416244in}}%
\pgfpathlineto{\pgfqpoint{0.457204in}{2.327431in}}%
\pgfpathlineto{\pgfqpoint{0.465266in}{2.327431in}}%
\pgfpathlineto{\pgfqpoint{0.473328in}{2.327431in}}%
\pgfpathlineto{\pgfqpoint{0.481390in}{2.327431in}}%
\pgfpathlineto{\pgfqpoint{0.489452in}{2.327431in}}%
\pgfpathlineto{\pgfqpoint{0.497514in}{2.310448in}}%
\pgfpathlineto{\pgfqpoint{0.505576in}{2.310448in}}%
\pgfpathlineto{\pgfqpoint{0.513638in}{2.310448in}}%
\pgfpathlineto{\pgfqpoint{0.521700in}{2.310448in}}%
\pgfpathlineto{\pgfqpoint{0.529762in}{2.310448in}}%
\pgfpathlineto{\pgfqpoint{0.537824in}{2.310448in}}%
\pgfpathlineto{\pgfqpoint{0.545886in}{2.310448in}}%
\pgfpathlineto{\pgfqpoint{0.553948in}{2.304960in}}%
\pgfpathlineto{\pgfqpoint{0.562010in}{2.304960in}}%
\pgfpathlineto{\pgfqpoint{0.570072in}{2.239759in}}%
\pgfpathlineto{\pgfqpoint{0.578134in}{2.236600in}}%
\pgfpathlineto{\pgfqpoint{0.586196in}{2.235300in}}%
\pgfpathlineto{\pgfqpoint{0.594258in}{2.206811in}}%
\pgfpathlineto{\pgfqpoint{0.602320in}{2.206811in}}%
\pgfpathlineto{\pgfqpoint{0.610382in}{2.178270in}}%
\pgfpathlineto{\pgfqpoint{0.618444in}{2.178270in}}%
\pgfpathlineto{\pgfqpoint{0.626506in}{2.178270in}}%
\pgfpathlineto{\pgfqpoint{0.634568in}{2.148316in}}%
\pgfpathlineto{\pgfqpoint{0.642629in}{2.144578in}}%
\pgfpathlineto{\pgfqpoint{0.650691in}{2.144578in}}%
\pgfpathlineto{\pgfqpoint{0.658753in}{2.056010in}}%
\pgfpathlineto{\pgfqpoint{0.666815in}{2.056010in}}%
\pgfpathlineto{\pgfqpoint{0.674877in}{2.051366in}}%
\pgfpathlineto{\pgfqpoint{0.682939in}{2.048819in}}%
\pgfpathlineto{\pgfqpoint{0.691001in}{2.034536in}}%
\pgfpathlineto{\pgfqpoint{0.699063in}{1.927243in}}%
\pgfpathlineto{\pgfqpoint{0.707125in}{1.927243in}}%
\pgfpathlineto{\pgfqpoint{0.715187in}{1.924862in}}%
\pgfpathlineto{\pgfqpoint{0.723249in}{1.924862in}}%
\pgfpathlineto{\pgfqpoint{0.731311in}{1.857962in}}%
\pgfpathlineto{\pgfqpoint{0.739373in}{1.722108in}}%
\pgfpathlineto{\pgfqpoint{0.747435in}{1.722108in}}%
\pgfpathlineto{\pgfqpoint{0.755497in}{1.722108in}}%
\pgfpathlineto{\pgfqpoint{0.763559in}{1.705000in}}%
\pgfpathlineto{\pgfqpoint{0.771621in}{1.683725in}}%
\pgfpathlineto{\pgfqpoint{0.779683in}{1.683725in}}%
\pgfpathlineto{\pgfqpoint{0.787745in}{1.683725in}}%
\pgfpathlineto{\pgfqpoint{0.795807in}{1.683725in}}%
\pgfpathlineto{\pgfqpoint{0.803869in}{1.487322in}}%
\pgfpathlineto{\pgfqpoint{0.811931in}{1.487322in}}%
\pgfpathlineto{\pgfqpoint{0.819993in}{1.487322in}}%
\pgfpathlineto{\pgfqpoint{0.828055in}{1.487322in}}%
\pgfpathlineto{\pgfqpoint{0.836117in}{1.431006in}}%
\pgfpathlineto{\pgfqpoint{0.844179in}{1.431006in}}%
\pgfpathlineto{\pgfqpoint{0.852241in}{1.431006in}}%
\pgfpathlineto{\pgfqpoint{0.860302in}{1.431006in}}%
\pgfpathlineto{\pgfqpoint{0.868364in}{1.295696in}}%
\pgfpathlineto{\pgfqpoint{0.876426in}{1.278204in}}%
\pgfpathlineto{\pgfqpoint{0.884488in}{1.180779in}}%
\pgfpathlineto{\pgfqpoint{0.892550in}{1.180779in}}%
\pgfpathlineto{\pgfqpoint{0.900612in}{1.180779in}}%
\pgfpathlineto{\pgfqpoint{0.908674in}{1.180779in}}%
\pgfpathlineto{\pgfqpoint{0.916736in}{1.180779in}}%
\pgfpathlineto{\pgfqpoint{0.924798in}{1.163803in}}%
\pgfpathlineto{\pgfqpoint{0.932860in}{1.163803in}}%
\pgfpathlineto{\pgfqpoint{0.940922in}{1.163803in}}%
\pgfpathlineto{\pgfqpoint{0.948984in}{1.114289in}}%
\pgfpathlineto{\pgfqpoint{0.957046in}{1.114289in}}%
\pgfpathlineto{\pgfqpoint{0.965108in}{1.109102in}}%
\pgfpathlineto{\pgfqpoint{0.973170in}{1.109102in}}%
\pgfpathlineto{\pgfqpoint{0.981232in}{1.109102in}}%
\pgfpathlineto{\pgfqpoint{0.989294in}{1.109102in}}%
\pgfpathlineto{\pgfqpoint{0.997356in}{1.109102in}}%
\pgfpathlineto{\pgfqpoint{1.005418in}{1.109102in}}%
\pgfpathlineto{\pgfqpoint{1.013480in}{1.109102in}}%
\pgfpathlineto{\pgfqpoint{1.021542in}{1.109102in}}%
\pgfpathlineto{\pgfqpoint{1.029604in}{1.109102in}}%
\pgfpathlineto{\pgfqpoint{1.037666in}{1.109102in}}%
\pgfpathlineto{\pgfqpoint{1.045728in}{1.109102in}}%
\pgfpathlineto{\pgfqpoint{1.053790in}{1.109102in}}%
\pgfpathlineto{\pgfqpoint{1.061852in}{1.109102in}}%
\pgfpathlineto{\pgfqpoint{1.069914in}{1.109102in}}%
\pgfpathlineto{\pgfqpoint{1.077975in}{1.103242in}}%
\pgfpathlineto{\pgfqpoint{1.086037in}{1.103242in}}%
\pgfpathlineto{\pgfqpoint{1.094099in}{1.103242in}}%
\pgfpathlineto{\pgfqpoint{1.102161in}{1.103242in}}%
\pgfpathlineto{\pgfqpoint{1.110223in}{1.103242in}}%
\pgfpathlineto{\pgfqpoint{1.118285in}{1.102732in}}%
\pgfpathlineto{\pgfqpoint{1.126347in}{1.102732in}}%
\pgfpathlineto{\pgfqpoint{1.134409in}{1.102732in}}%
\pgfpathlineto{\pgfqpoint{1.142471in}{1.093545in}}%
\pgfpathlineto{\pgfqpoint{1.150533in}{1.093545in}}%
\pgfpathlineto{\pgfqpoint{1.158595in}{1.093545in}}%
\pgfpathlineto{\pgfqpoint{1.166657in}{1.093545in}}%
\pgfpathlineto{\pgfqpoint{1.174719in}{1.093545in}}%
\pgfpathlineto{\pgfqpoint{1.182781in}{1.093545in}}%
\pgfpathlineto{\pgfqpoint{1.190843in}{1.093545in}}%
\pgfpathlineto{\pgfqpoint{1.198905in}{1.093545in}}%
\pgfpathlineto{\pgfqpoint{1.206967in}{1.093545in}}%
\pgfpathlineto{\pgfqpoint{1.215029in}{1.093545in}}%
\pgfpathlineto{\pgfqpoint{1.223091in}{1.093545in}}%
\pgfpathlineto{\pgfqpoint{1.231153in}{1.093545in}}%
\pgfpathlineto{\pgfqpoint{1.239215in}{1.093545in}}%
\pgfpathlineto{\pgfqpoint{1.247277in}{1.093545in}}%
\pgfpathlineto{\pgfqpoint{1.255339in}{0.979053in}}%
\pgfpathlineto{\pgfqpoint{1.263401in}{0.979053in}}%
\pgfpathlineto{\pgfqpoint{1.271463in}{0.979053in}}%
\pgfpathlineto{\pgfqpoint{1.279525in}{0.979053in}}%
\pgfpathlineto{\pgfqpoint{1.287586in}{0.979053in}}%
\pgfpathlineto{\pgfqpoint{1.295648in}{0.979053in}}%
\pgfpathlineto{\pgfqpoint{1.303710in}{0.979053in}}%
\pgfpathlineto{\pgfqpoint{1.311772in}{0.979053in}}%
\pgfpathlineto{\pgfqpoint{1.319834in}{0.979041in}}%
\pgfpathlineto{\pgfqpoint{1.327896in}{0.979041in}}%
\pgfpathlineto{\pgfqpoint{1.335958in}{0.979041in}}%
\pgfpathlineto{\pgfqpoint{1.344020in}{0.971645in}}%
\pgfpathlineto{\pgfqpoint{1.352082in}{0.971645in}}%
\pgfpathlineto{\pgfqpoint{1.360144in}{0.971645in}}%
\pgfpathlineto{\pgfqpoint{1.368206in}{0.971645in}}%
\pgfpathlineto{\pgfqpoint{1.376268in}{0.971645in}}%
\pgfpathlineto{\pgfqpoint{1.384330in}{0.895758in}}%
\pgfpathlineto{\pgfqpoint{1.392392in}{0.895758in}}%
\pgfpathlineto{\pgfqpoint{1.400454in}{0.895758in}}%
\pgfpathlineto{\pgfqpoint{1.408516in}{0.895758in}}%
\pgfpathlineto{\pgfqpoint{1.416578in}{0.895758in}}%
\pgfpathlineto{\pgfqpoint{1.424640in}{0.895758in}}%
\pgfpathlineto{\pgfqpoint{1.432702in}{0.736820in}}%
\pgfpathlineto{\pgfqpoint{1.440764in}{0.736820in}}%
\pgfpathlineto{\pgfqpoint{1.448826in}{0.736820in}}%
\pgfpathlineto{\pgfqpoint{1.456888in}{0.736820in}}%
\pgfpathlineto{\pgfqpoint{1.464950in}{0.736820in}}%
\pgfpathlineto{\pgfqpoint{1.473012in}{0.736820in}}%
\pgfpathlineto{\pgfqpoint{1.481074in}{0.736820in}}%
\pgfpathlineto{\pgfqpoint{1.489136in}{0.736820in}}%
\pgfpathlineto{\pgfqpoint{1.497198in}{0.736820in}}%
\pgfpathlineto{\pgfqpoint{1.505259in}{0.736820in}}%
\pgfpathlineto{\pgfqpoint{1.513321in}{0.736820in}}%
\pgfpathlineto{\pgfqpoint{1.521383in}{0.736820in}}%
\pgfpathlineto{\pgfqpoint{1.529445in}{0.736820in}}%
\pgfpathlineto{\pgfqpoint{1.537507in}{0.736820in}}%
\pgfpathlineto{\pgfqpoint{1.545569in}{0.736820in}}%
\pgfpathlineto{\pgfqpoint{1.553631in}{0.736820in}}%
\pgfpathlineto{\pgfqpoint{1.561693in}{0.736820in}}%
\pgfpathlineto{\pgfqpoint{1.569755in}{0.736820in}}%
\pgfpathlineto{\pgfqpoint{1.577817in}{0.736820in}}%
\pgfpathlineto{\pgfqpoint{1.585879in}{0.736820in}}%
\pgfpathlineto{\pgfqpoint{1.593941in}{0.736820in}}%
\pgfpathlineto{\pgfqpoint{1.602003in}{0.736820in}}%
\pgfpathlineto{\pgfqpoint{1.610065in}{0.736820in}}%
\pgfpathlineto{\pgfqpoint{1.618127in}{0.736820in}}%
\pgfpathlineto{\pgfqpoint{1.626189in}{0.731215in}}%
\pgfpathlineto{\pgfqpoint{1.634251in}{0.731215in}}%
\pgfpathlineto{\pgfqpoint{1.642313in}{0.731215in}}%
\pgfpathlineto{\pgfqpoint{1.650375in}{0.731215in}}%
\pgfpathlineto{\pgfqpoint{1.658437in}{0.731215in}}%
\pgfpathlineto{\pgfqpoint{1.666499in}{0.731215in}}%
\pgfpathlineto{\pgfqpoint{1.674561in}{0.731215in}}%
\pgfpathlineto{\pgfqpoint{1.682623in}{0.731215in}}%
\pgfpathlineto{\pgfqpoint{1.690685in}{0.731215in}}%
\pgfpathlineto{\pgfqpoint{1.698747in}{0.731215in}}%
\pgfpathlineto{\pgfqpoint{1.706809in}{0.731215in}}%
\pgfpathlineto{\pgfqpoint{1.714871in}{0.731215in}}%
\pgfpathlineto{\pgfqpoint{1.722932in}{0.731215in}}%
\pgfpathlineto{\pgfqpoint{1.730994in}{0.731215in}}%
\pgfpathlineto{\pgfqpoint{1.739056in}{0.731215in}}%
\pgfpathlineto{\pgfqpoint{1.747118in}{0.731215in}}%
\pgfpathlineto{\pgfqpoint{1.755180in}{0.731215in}}%
\pgfpathlineto{\pgfqpoint{1.763242in}{0.731215in}}%
\pgfpathlineto{\pgfqpoint{1.771304in}{0.731215in}}%
\pgfpathlineto{\pgfqpoint{1.779366in}{0.731215in}}%
\pgfpathlineto{\pgfqpoint{1.787428in}{0.731215in}}%
\pgfpathlineto{\pgfqpoint{1.795490in}{0.731215in}}%
\pgfpathlineto{\pgfqpoint{1.803552in}{0.731215in}}%
\pgfpathlineto{\pgfqpoint{1.811614in}{0.731215in}}%
\pgfpathlineto{\pgfqpoint{1.819676in}{0.731215in}}%
\pgfpathlineto{\pgfqpoint{1.827738in}{0.731215in}}%
\pgfpathlineto{\pgfqpoint{1.835800in}{0.731215in}}%
\pgfpathlineto{\pgfqpoint{1.843862in}{0.731215in}}%
\pgfpathlineto{\pgfqpoint{1.851924in}{0.731215in}}%
\pgfpathlineto{\pgfqpoint{1.859986in}{0.731215in}}%
\pgfpathlineto{\pgfqpoint{1.868048in}{0.731215in}}%
\pgfpathlineto{\pgfqpoint{1.876110in}{0.731215in}}%
\pgfpathlineto{\pgfqpoint{1.884172in}{0.731215in}}%
\pgfpathlineto{\pgfqpoint{1.892234in}{0.731215in}}%
\pgfpathlineto{\pgfqpoint{1.900296in}{0.731215in}}%
\pgfpathlineto{\pgfqpoint{1.908358in}{0.731215in}}%
\pgfpathlineto{\pgfqpoint{1.916420in}{0.731215in}}%
\pgfpathlineto{\pgfqpoint{1.924482in}{0.731215in}}%
\pgfpathlineto{\pgfqpoint{1.932544in}{0.730712in}}%
\pgfpathlineto{\pgfqpoint{1.940605in}{0.730712in}}%
\pgfpathlineto{\pgfqpoint{1.948667in}{0.730712in}}%
\pgfpathlineto{\pgfqpoint{1.956729in}{0.730712in}}%
\pgfpathlineto{\pgfqpoint{1.964791in}{0.730712in}}%
\pgfpathlineto{\pgfqpoint{1.972853in}{0.730712in}}%
\pgfpathlineto{\pgfqpoint{1.980915in}{0.730712in}}%
\pgfpathlineto{\pgfqpoint{1.988977in}{0.730712in}}%
\pgfpathlineto{\pgfqpoint{1.988977in}{0.477955in}}%
\pgfpathlineto{\pgfqpoint{1.988977in}{0.477955in}}%
\pgfpathlineto{\pgfqpoint{1.980915in}{0.477955in}}%
\pgfpathlineto{\pgfqpoint{1.972853in}{0.477955in}}%
\pgfpathlineto{\pgfqpoint{1.964791in}{0.477955in}}%
\pgfpathlineto{\pgfqpoint{1.956729in}{0.477955in}}%
\pgfpathlineto{\pgfqpoint{1.948667in}{0.477955in}}%
\pgfpathlineto{\pgfqpoint{1.940605in}{0.477955in}}%
\pgfpathlineto{\pgfqpoint{1.932544in}{0.477955in}}%
\pgfpathlineto{\pgfqpoint{1.924482in}{0.484693in}}%
\pgfpathlineto{\pgfqpoint{1.916420in}{0.484693in}}%
\pgfpathlineto{\pgfqpoint{1.908358in}{0.484693in}}%
\pgfpathlineto{\pgfqpoint{1.900296in}{0.484693in}}%
\pgfpathlineto{\pgfqpoint{1.892234in}{0.484693in}}%
\pgfpathlineto{\pgfqpoint{1.884172in}{0.484693in}}%
\pgfpathlineto{\pgfqpoint{1.876110in}{0.484693in}}%
\pgfpathlineto{\pgfqpoint{1.868048in}{0.484693in}}%
\pgfpathlineto{\pgfqpoint{1.859986in}{0.484693in}}%
\pgfpathlineto{\pgfqpoint{1.851924in}{0.484693in}}%
\pgfpathlineto{\pgfqpoint{1.843862in}{0.484693in}}%
\pgfpathlineto{\pgfqpoint{1.835800in}{0.484693in}}%
\pgfpathlineto{\pgfqpoint{1.827738in}{0.484693in}}%
\pgfpathlineto{\pgfqpoint{1.819676in}{0.484693in}}%
\pgfpathlineto{\pgfqpoint{1.811614in}{0.484693in}}%
\pgfpathlineto{\pgfqpoint{1.803552in}{0.484693in}}%
\pgfpathlineto{\pgfqpoint{1.795490in}{0.484693in}}%
\pgfpathlineto{\pgfqpoint{1.787428in}{0.484693in}}%
\pgfpathlineto{\pgfqpoint{1.779366in}{0.484693in}}%
\pgfpathlineto{\pgfqpoint{1.771304in}{0.484693in}}%
\pgfpathlineto{\pgfqpoint{1.763242in}{0.484693in}}%
\pgfpathlineto{\pgfqpoint{1.755180in}{0.484693in}}%
\pgfpathlineto{\pgfqpoint{1.747118in}{0.484693in}}%
\pgfpathlineto{\pgfqpoint{1.739056in}{0.484693in}}%
\pgfpathlineto{\pgfqpoint{1.730994in}{0.484693in}}%
\pgfpathlineto{\pgfqpoint{1.722932in}{0.484693in}}%
\pgfpathlineto{\pgfqpoint{1.714871in}{0.484693in}}%
\pgfpathlineto{\pgfqpoint{1.706809in}{0.484693in}}%
\pgfpathlineto{\pgfqpoint{1.698747in}{0.484693in}}%
\pgfpathlineto{\pgfqpoint{1.690685in}{0.484693in}}%
\pgfpathlineto{\pgfqpoint{1.682623in}{0.484693in}}%
\pgfpathlineto{\pgfqpoint{1.674561in}{0.484693in}}%
\pgfpathlineto{\pgfqpoint{1.666499in}{0.484693in}}%
\pgfpathlineto{\pgfqpoint{1.658437in}{0.484693in}}%
\pgfpathlineto{\pgfqpoint{1.650375in}{0.484693in}}%
\pgfpathlineto{\pgfqpoint{1.642313in}{0.484693in}}%
\pgfpathlineto{\pgfqpoint{1.634251in}{0.484693in}}%
\pgfpathlineto{\pgfqpoint{1.626189in}{0.484693in}}%
\pgfpathlineto{\pgfqpoint{1.618127in}{0.536782in}}%
\pgfpathlineto{\pgfqpoint{1.610065in}{0.536782in}}%
\pgfpathlineto{\pgfqpoint{1.602003in}{0.536782in}}%
\pgfpathlineto{\pgfqpoint{1.593941in}{0.536782in}}%
\pgfpathlineto{\pgfqpoint{1.585879in}{0.536782in}}%
\pgfpathlineto{\pgfqpoint{1.577817in}{0.536782in}}%
\pgfpathlineto{\pgfqpoint{1.569755in}{0.536782in}}%
\pgfpathlineto{\pgfqpoint{1.561693in}{0.536782in}}%
\pgfpathlineto{\pgfqpoint{1.553631in}{0.536782in}}%
\pgfpathlineto{\pgfqpoint{1.545569in}{0.536782in}}%
\pgfpathlineto{\pgfqpoint{1.537507in}{0.536782in}}%
\pgfpathlineto{\pgfqpoint{1.529445in}{0.536782in}}%
\pgfpathlineto{\pgfqpoint{1.521383in}{0.536782in}}%
\pgfpathlineto{\pgfqpoint{1.513321in}{0.536782in}}%
\pgfpathlineto{\pgfqpoint{1.505259in}{0.536782in}}%
\pgfpathlineto{\pgfqpoint{1.497198in}{0.536782in}}%
\pgfpathlineto{\pgfqpoint{1.489136in}{0.536782in}}%
\pgfpathlineto{\pgfqpoint{1.481074in}{0.536782in}}%
\pgfpathlineto{\pgfqpoint{1.473012in}{0.536782in}}%
\pgfpathlineto{\pgfqpoint{1.464950in}{0.536782in}}%
\pgfpathlineto{\pgfqpoint{1.456888in}{0.536782in}}%
\pgfpathlineto{\pgfqpoint{1.448826in}{0.536782in}}%
\pgfpathlineto{\pgfqpoint{1.440764in}{0.536782in}}%
\pgfpathlineto{\pgfqpoint{1.432702in}{0.536782in}}%
\pgfpathlineto{\pgfqpoint{1.424640in}{0.699182in}}%
\pgfpathlineto{\pgfqpoint{1.416578in}{0.699182in}}%
\pgfpathlineto{\pgfqpoint{1.408516in}{0.699182in}}%
\pgfpathlineto{\pgfqpoint{1.400454in}{0.699182in}}%
\pgfpathlineto{\pgfqpoint{1.392392in}{0.699182in}}%
\pgfpathlineto{\pgfqpoint{1.384330in}{0.699182in}}%
\pgfpathlineto{\pgfqpoint{1.376268in}{0.727320in}}%
\pgfpathlineto{\pgfqpoint{1.368206in}{0.727320in}}%
\pgfpathlineto{\pgfqpoint{1.360144in}{0.727320in}}%
\pgfpathlineto{\pgfqpoint{1.352082in}{0.727320in}}%
\pgfpathlineto{\pgfqpoint{1.344020in}{0.727320in}}%
\pgfpathlineto{\pgfqpoint{1.335958in}{0.793213in}}%
\pgfpathlineto{\pgfqpoint{1.327896in}{0.793213in}}%
\pgfpathlineto{\pgfqpoint{1.319834in}{0.793213in}}%
\pgfpathlineto{\pgfqpoint{1.311772in}{0.793330in}}%
\pgfpathlineto{\pgfqpoint{1.303710in}{0.793330in}}%
\pgfpathlineto{\pgfqpoint{1.295648in}{0.793330in}}%
\pgfpathlineto{\pgfqpoint{1.287586in}{0.793330in}}%
\pgfpathlineto{\pgfqpoint{1.279525in}{0.793330in}}%
\pgfpathlineto{\pgfqpoint{1.271463in}{0.793330in}}%
\pgfpathlineto{\pgfqpoint{1.263401in}{0.793330in}}%
\pgfpathlineto{\pgfqpoint{1.255339in}{0.793330in}}%
\pgfpathlineto{\pgfqpoint{1.247277in}{0.833096in}}%
\pgfpathlineto{\pgfqpoint{1.239215in}{0.833096in}}%
\pgfpathlineto{\pgfqpoint{1.231153in}{0.833096in}}%
\pgfpathlineto{\pgfqpoint{1.223091in}{0.833096in}}%
\pgfpathlineto{\pgfqpoint{1.215029in}{0.833096in}}%
\pgfpathlineto{\pgfqpoint{1.206967in}{0.833096in}}%
\pgfpathlineto{\pgfqpoint{1.198905in}{0.833096in}}%
\pgfpathlineto{\pgfqpoint{1.190843in}{0.833096in}}%
\pgfpathlineto{\pgfqpoint{1.182781in}{0.833096in}}%
\pgfpathlineto{\pgfqpoint{1.174719in}{0.833096in}}%
\pgfpathlineto{\pgfqpoint{1.166657in}{0.833096in}}%
\pgfpathlineto{\pgfqpoint{1.158595in}{0.833096in}}%
\pgfpathlineto{\pgfqpoint{1.150533in}{0.833096in}}%
\pgfpathlineto{\pgfqpoint{1.142471in}{0.833096in}}%
\pgfpathlineto{\pgfqpoint{1.134409in}{0.909305in}}%
\pgfpathlineto{\pgfqpoint{1.126347in}{0.909305in}}%
\pgfpathlineto{\pgfqpoint{1.118285in}{0.909305in}}%
\pgfpathlineto{\pgfqpoint{1.110223in}{0.914181in}}%
\pgfpathlineto{\pgfqpoint{1.102161in}{0.914181in}}%
\pgfpathlineto{\pgfqpoint{1.094099in}{0.914181in}}%
\pgfpathlineto{\pgfqpoint{1.086037in}{0.914181in}}%
\pgfpathlineto{\pgfqpoint{1.077975in}{0.914181in}}%
\pgfpathlineto{\pgfqpoint{1.069914in}{0.954089in}}%
\pgfpathlineto{\pgfqpoint{1.061852in}{0.954089in}}%
\pgfpathlineto{\pgfqpoint{1.053790in}{0.954089in}}%
\pgfpathlineto{\pgfqpoint{1.045728in}{0.954089in}}%
\pgfpathlineto{\pgfqpoint{1.037666in}{0.954089in}}%
\pgfpathlineto{\pgfqpoint{1.029604in}{0.954089in}}%
\pgfpathlineto{\pgfqpoint{1.021542in}{0.954089in}}%
\pgfpathlineto{\pgfqpoint{1.013480in}{0.954089in}}%
\pgfpathlineto{\pgfqpoint{1.005418in}{0.954089in}}%
\pgfpathlineto{\pgfqpoint{0.997356in}{0.954089in}}%
\pgfpathlineto{\pgfqpoint{0.989294in}{0.954089in}}%
\pgfpathlineto{\pgfqpoint{0.981232in}{0.954089in}}%
\pgfpathlineto{\pgfqpoint{0.973170in}{0.954089in}}%
\pgfpathlineto{\pgfqpoint{0.965108in}{0.954089in}}%
\pgfpathlineto{\pgfqpoint{0.957046in}{0.975716in}}%
\pgfpathlineto{\pgfqpoint{0.948984in}{0.975716in}}%
\pgfpathlineto{\pgfqpoint{0.940922in}{1.088767in}}%
\pgfpathlineto{\pgfqpoint{0.932860in}{1.088767in}}%
\pgfpathlineto{\pgfqpoint{0.924798in}{1.088767in}}%
\pgfpathlineto{\pgfqpoint{0.916736in}{1.096176in}}%
\pgfpathlineto{\pgfqpoint{0.908674in}{1.096176in}}%
\pgfpathlineto{\pgfqpoint{0.900612in}{1.096176in}}%
\pgfpathlineto{\pgfqpoint{0.892550in}{1.096176in}}%
\pgfpathlineto{\pgfqpoint{0.884488in}{1.096176in}}%
\pgfpathlineto{\pgfqpoint{0.876426in}{1.174212in}}%
\pgfpathlineto{\pgfqpoint{0.868364in}{1.177165in}}%
\pgfpathlineto{\pgfqpoint{0.860302in}{1.270566in}}%
\pgfpathlineto{\pgfqpoint{0.852241in}{1.270566in}}%
\pgfpathlineto{\pgfqpoint{0.844179in}{1.270566in}}%
\pgfpathlineto{\pgfqpoint{0.836117in}{1.270566in}}%
\pgfpathlineto{\pgfqpoint{0.828055in}{1.285928in}}%
\pgfpathlineto{\pgfqpoint{0.819993in}{1.285928in}}%
\pgfpathlineto{\pgfqpoint{0.811931in}{1.285928in}}%
\pgfpathlineto{\pgfqpoint{0.803869in}{1.285928in}}%
\pgfpathlineto{\pgfqpoint{0.795807in}{1.539507in}}%
\pgfpathlineto{\pgfqpoint{0.787745in}{1.539507in}}%
\pgfpathlineto{\pgfqpoint{0.779683in}{1.539507in}}%
\pgfpathlineto{\pgfqpoint{0.771621in}{1.539507in}}%
\pgfpathlineto{\pgfqpoint{0.763559in}{1.610674in}}%
\pgfpathlineto{\pgfqpoint{0.755497in}{1.653123in}}%
\pgfpathlineto{\pgfqpoint{0.747435in}{1.653123in}}%
\pgfpathlineto{\pgfqpoint{0.739373in}{1.653123in}}%
\pgfpathlineto{\pgfqpoint{0.731311in}{1.675616in}}%
\pgfpathlineto{\pgfqpoint{0.723249in}{1.790957in}}%
\pgfpathlineto{\pgfqpoint{0.715187in}{1.790957in}}%
\pgfpathlineto{\pgfqpoint{0.707125in}{1.803901in}}%
\pgfpathlineto{\pgfqpoint{0.699063in}{1.803901in}}%
\pgfpathlineto{\pgfqpoint{0.691001in}{1.905400in}}%
\pgfpathlineto{\pgfqpoint{0.682939in}{1.959625in}}%
\pgfpathlineto{\pgfqpoint{0.674877in}{1.968759in}}%
\pgfpathlineto{\pgfqpoint{0.666815in}{1.969630in}}%
\pgfpathlineto{\pgfqpoint{0.658753in}{1.969630in}}%
\pgfpathlineto{\pgfqpoint{0.650691in}{2.021763in}}%
\pgfpathlineto{\pgfqpoint{0.642629in}{2.021763in}}%
\pgfpathlineto{\pgfqpoint{0.634568in}{2.038295in}}%
\pgfpathlineto{\pgfqpoint{0.626506in}{2.106387in}}%
\pgfpathlineto{\pgfqpoint{0.618444in}{2.106387in}}%
\pgfpathlineto{\pgfqpoint{0.610382in}{2.106387in}}%
\pgfpathlineto{\pgfqpoint{0.602320in}{2.114578in}}%
\pgfpathlineto{\pgfqpoint{0.594258in}{2.114578in}}%
\pgfpathlineto{\pgfqpoint{0.586196in}{2.139999in}}%
\pgfpathlineto{\pgfqpoint{0.578134in}{2.147407in}}%
\pgfpathlineto{\pgfqpoint{0.570072in}{2.161832in}}%
\pgfpathlineto{\pgfqpoint{0.562010in}{2.180441in}}%
\pgfpathlineto{\pgfqpoint{0.553948in}{2.180441in}}%
\pgfpathlineto{\pgfqpoint{0.545886in}{2.203445in}}%
\pgfpathlineto{\pgfqpoint{0.537824in}{2.203445in}}%
\pgfpathlineto{\pgfqpoint{0.529762in}{2.203445in}}%
\pgfpathlineto{\pgfqpoint{0.521700in}{2.203445in}}%
\pgfpathlineto{\pgfqpoint{0.513638in}{2.203445in}}%
\pgfpathlineto{\pgfqpoint{0.505576in}{2.203445in}}%
\pgfpathlineto{\pgfqpoint{0.497514in}{2.203445in}}%
\pgfpathlineto{\pgfqpoint{0.489452in}{2.206628in}}%
\pgfpathlineto{\pgfqpoint{0.481390in}{2.206628in}}%
\pgfpathlineto{\pgfqpoint{0.473328in}{2.206628in}}%
\pgfpathlineto{\pgfqpoint{0.465266in}{2.206628in}}%
\pgfpathlineto{\pgfqpoint{0.457204in}{2.206628in}}%
\pgfpathlineto{\pgfqpoint{0.449142in}{2.313990in}}%
\pgfpathlineto{\pgfqpoint{0.441080in}{2.313990in}}%
\pgfpathlineto{\pgfqpoint{0.433018in}{2.313990in}}%
\pgfpathlineto{\pgfqpoint{0.424956in}{2.313990in}}%
\pgfpathlineto{\pgfqpoint{0.416895in}{2.313990in}}%
\pgfpathlineto{\pgfqpoint{0.408833in}{2.313990in}}%
\pgfpathlineto{\pgfqpoint{0.400771in}{2.313990in}}%
\pgfpathlineto{\pgfqpoint{0.392709in}{2.313990in}}%
\pgfpathlineto{\pgfqpoint{0.384647in}{2.322427in}}%
\pgfpathlineto{\pgfqpoint{0.376585in}{2.322863in}}%
\pgfpathlineto{\pgfqpoint{0.368523in}{2.413106in}}%
\pgfpathclose%
\pgfusepath{fill}%
\end{pgfscope}%
\begin{pgfscope}%
\pgfpathrectangle{\pgfqpoint{0.287500in}{0.375000in}}{\pgfqpoint{1.782500in}{2.265000in}}%
\pgfusepath{clip}%
\pgfsetbuttcap%
\pgfsetroundjoin%
\definecolor{currentfill}{rgb}{0.172549,0.627451,0.172549}%
\pgfsetfillcolor{currentfill}%
\pgfsetfillopacity{0.200000}%
\pgfsetlinewidth{0.000000pt}%
\definecolor{currentstroke}{rgb}{0.000000,0.000000,0.000000}%
\pgfsetstrokecolor{currentstroke}%
\pgfsetdash{}{0pt}%
\pgfpathmoveto{\pgfqpoint{0.368523in}{2.483219in}}%
\pgfpathlineto{\pgfqpoint{0.368523in}{2.537045in}}%
\pgfpathlineto{\pgfqpoint{0.376585in}{2.462778in}}%
\pgfpathlineto{\pgfqpoint{0.384647in}{2.449322in}}%
\pgfpathlineto{\pgfqpoint{0.392709in}{2.449322in}}%
\pgfpathlineto{\pgfqpoint{0.400771in}{2.449322in}}%
\pgfpathlineto{\pgfqpoint{0.408833in}{2.445358in}}%
\pgfpathlineto{\pgfqpoint{0.416895in}{2.435187in}}%
\pgfpathlineto{\pgfqpoint{0.424956in}{2.435187in}}%
\pgfpathlineto{\pgfqpoint{0.433018in}{2.419096in}}%
\pgfpathlineto{\pgfqpoint{0.441080in}{2.381199in}}%
\pgfpathlineto{\pgfqpoint{0.449142in}{2.381199in}}%
\pgfpathlineto{\pgfqpoint{0.457204in}{2.379959in}}%
\pgfpathlineto{\pgfqpoint{0.465266in}{2.379959in}}%
\pgfpathlineto{\pgfqpoint{0.473328in}{2.379959in}}%
\pgfpathlineto{\pgfqpoint{0.481390in}{2.379959in}}%
\pgfpathlineto{\pgfqpoint{0.489452in}{2.379959in}}%
\pgfpathlineto{\pgfqpoint{0.497514in}{2.379959in}}%
\pgfpathlineto{\pgfqpoint{0.505576in}{2.379959in}}%
\pgfpathlineto{\pgfqpoint{0.513638in}{2.318966in}}%
\pgfpathlineto{\pgfqpoint{0.521700in}{2.318966in}}%
\pgfpathlineto{\pgfqpoint{0.529762in}{2.318966in}}%
\pgfpathlineto{\pgfqpoint{0.537824in}{2.318966in}}%
\pgfpathlineto{\pgfqpoint{0.545886in}{2.314736in}}%
\pgfpathlineto{\pgfqpoint{0.553948in}{2.274224in}}%
\pgfpathlineto{\pgfqpoint{0.562010in}{2.273823in}}%
\pgfpathlineto{\pgfqpoint{0.570072in}{2.238177in}}%
\pgfpathlineto{\pgfqpoint{0.578134in}{2.238177in}}%
\pgfpathlineto{\pgfqpoint{0.586196in}{2.238177in}}%
\pgfpathlineto{\pgfqpoint{0.594258in}{2.238177in}}%
\pgfpathlineto{\pgfqpoint{0.602320in}{2.238177in}}%
\pgfpathlineto{\pgfqpoint{0.610382in}{2.238177in}}%
\pgfpathlineto{\pgfqpoint{0.618444in}{2.238177in}}%
\pgfpathlineto{\pgfqpoint{0.626506in}{2.238177in}}%
\pgfpathlineto{\pgfqpoint{0.634568in}{2.238177in}}%
\pgfpathlineto{\pgfqpoint{0.642629in}{2.238177in}}%
\pgfpathlineto{\pgfqpoint{0.650691in}{2.238177in}}%
\pgfpathlineto{\pgfqpoint{0.658753in}{2.238177in}}%
\pgfpathlineto{\pgfqpoint{0.666815in}{2.238177in}}%
\pgfpathlineto{\pgfqpoint{0.674877in}{2.238177in}}%
\pgfpathlineto{\pgfqpoint{0.682939in}{2.238177in}}%
\pgfpathlineto{\pgfqpoint{0.691001in}{2.214776in}}%
\pgfpathlineto{\pgfqpoint{0.699063in}{2.214776in}}%
\pgfpathlineto{\pgfqpoint{0.707125in}{2.214776in}}%
\pgfpathlineto{\pgfqpoint{0.715187in}{2.214776in}}%
\pgfpathlineto{\pgfqpoint{0.723249in}{2.214776in}}%
\pgfpathlineto{\pgfqpoint{0.731311in}{2.214776in}}%
\pgfpathlineto{\pgfqpoint{0.739373in}{2.214776in}}%
\pgfpathlineto{\pgfqpoint{0.747435in}{2.213426in}}%
\pgfpathlineto{\pgfqpoint{0.755497in}{2.213426in}}%
\pgfpathlineto{\pgfqpoint{0.763559in}{2.093008in}}%
\pgfpathlineto{\pgfqpoint{0.771621in}{2.091988in}}%
\pgfpathlineto{\pgfqpoint{0.779683in}{2.052405in}}%
\pgfpathlineto{\pgfqpoint{0.787745in}{2.052405in}}%
\pgfpathlineto{\pgfqpoint{0.795807in}{2.050540in}}%
\pgfpathlineto{\pgfqpoint{0.803869in}{2.048920in}}%
\pgfpathlineto{\pgfqpoint{0.811931in}{2.048920in}}%
\pgfpathlineto{\pgfqpoint{0.819993in}{2.048920in}}%
\pgfpathlineto{\pgfqpoint{0.828055in}{2.048819in}}%
\pgfpathlineto{\pgfqpoint{0.836117in}{2.048819in}}%
\pgfpathlineto{\pgfqpoint{0.844179in}{2.048819in}}%
\pgfpathlineto{\pgfqpoint{0.852241in}{2.048819in}}%
\pgfpathlineto{\pgfqpoint{0.860302in}{2.048819in}}%
\pgfpathlineto{\pgfqpoint{0.868364in}{2.048819in}}%
\pgfpathlineto{\pgfqpoint{0.876426in}{2.048819in}}%
\pgfpathlineto{\pgfqpoint{0.884488in}{1.525334in}}%
\pgfpathlineto{\pgfqpoint{0.892550in}{1.523070in}}%
\pgfpathlineto{\pgfqpoint{0.900612in}{1.523070in}}%
\pgfpathlineto{\pgfqpoint{0.908674in}{1.523070in}}%
\pgfpathlineto{\pgfqpoint{0.916736in}{1.515342in}}%
\pgfpathlineto{\pgfqpoint{0.924798in}{1.515342in}}%
\pgfpathlineto{\pgfqpoint{0.932860in}{1.472807in}}%
\pgfpathlineto{\pgfqpoint{0.940922in}{1.472807in}}%
\pgfpathlineto{\pgfqpoint{0.948984in}{1.472807in}}%
\pgfpathlineto{\pgfqpoint{0.957046in}{1.472807in}}%
\pgfpathlineto{\pgfqpoint{0.965108in}{1.457383in}}%
\pgfpathlineto{\pgfqpoint{0.973170in}{1.457383in}}%
\pgfpathlineto{\pgfqpoint{0.981232in}{1.457383in}}%
\pgfpathlineto{\pgfqpoint{0.989294in}{1.457383in}}%
\pgfpathlineto{\pgfqpoint{0.997356in}{1.457383in}}%
\pgfpathlineto{\pgfqpoint{1.005418in}{1.457383in}}%
\pgfpathlineto{\pgfqpoint{1.013480in}{1.457383in}}%
\pgfpathlineto{\pgfqpoint{1.021542in}{1.457383in}}%
\pgfpathlineto{\pgfqpoint{1.029604in}{1.457383in}}%
\pgfpathlineto{\pgfqpoint{1.037666in}{1.457383in}}%
\pgfpathlineto{\pgfqpoint{1.045728in}{1.457379in}}%
\pgfpathlineto{\pgfqpoint{1.053790in}{1.435317in}}%
\pgfpathlineto{\pgfqpoint{1.061852in}{1.418120in}}%
\pgfpathlineto{\pgfqpoint{1.069914in}{1.418120in}}%
\pgfpathlineto{\pgfqpoint{1.077975in}{1.418120in}}%
\pgfpathlineto{\pgfqpoint{1.086037in}{1.418120in}}%
\pgfpathlineto{\pgfqpoint{1.094099in}{1.418120in}}%
\pgfpathlineto{\pgfqpoint{1.102161in}{1.418120in}}%
\pgfpathlineto{\pgfqpoint{1.110223in}{1.418120in}}%
\pgfpathlineto{\pgfqpoint{1.118285in}{1.418120in}}%
\pgfpathlineto{\pgfqpoint{1.126347in}{1.418120in}}%
\pgfpathlineto{\pgfqpoint{1.134409in}{1.418120in}}%
\pgfpathlineto{\pgfqpoint{1.142471in}{1.418120in}}%
\pgfpathlineto{\pgfqpoint{1.150533in}{1.418120in}}%
\pgfpathlineto{\pgfqpoint{1.158595in}{1.418120in}}%
\pgfpathlineto{\pgfqpoint{1.166657in}{1.418120in}}%
\pgfpathlineto{\pgfqpoint{1.174719in}{1.418120in}}%
\pgfpathlineto{\pgfqpoint{1.182781in}{1.418120in}}%
\pgfpathlineto{\pgfqpoint{1.190843in}{1.418120in}}%
\pgfpathlineto{\pgfqpoint{1.198905in}{1.418120in}}%
\pgfpathlineto{\pgfqpoint{1.206967in}{1.382989in}}%
\pgfpathlineto{\pgfqpoint{1.215029in}{1.382989in}}%
\pgfpathlineto{\pgfqpoint{1.223091in}{1.382989in}}%
\pgfpathlineto{\pgfqpoint{1.231153in}{1.382989in}}%
\pgfpathlineto{\pgfqpoint{1.239215in}{1.382989in}}%
\pgfpathlineto{\pgfqpoint{1.247277in}{1.213820in}}%
\pgfpathlineto{\pgfqpoint{1.255339in}{1.213820in}}%
\pgfpathlineto{\pgfqpoint{1.263401in}{1.213820in}}%
\pgfpathlineto{\pgfqpoint{1.271463in}{1.213820in}}%
\pgfpathlineto{\pgfqpoint{1.279525in}{1.213820in}}%
\pgfpathlineto{\pgfqpoint{1.287586in}{1.213820in}}%
\pgfpathlineto{\pgfqpoint{1.295648in}{1.213820in}}%
\pgfpathlineto{\pgfqpoint{1.303710in}{1.006749in}}%
\pgfpathlineto{\pgfqpoint{1.311772in}{1.006749in}}%
\pgfpathlineto{\pgfqpoint{1.319834in}{1.006749in}}%
\pgfpathlineto{\pgfqpoint{1.327896in}{1.006749in}}%
\pgfpathlineto{\pgfqpoint{1.335958in}{1.006749in}}%
\pgfpathlineto{\pgfqpoint{1.344020in}{1.006749in}}%
\pgfpathlineto{\pgfqpoint{1.352082in}{1.006749in}}%
\pgfpathlineto{\pgfqpoint{1.360144in}{1.006749in}}%
\pgfpathlineto{\pgfqpoint{1.368206in}{1.006749in}}%
\pgfpathlineto{\pgfqpoint{1.376268in}{1.006749in}}%
\pgfpathlineto{\pgfqpoint{1.384330in}{1.006749in}}%
\pgfpathlineto{\pgfqpoint{1.392392in}{0.991829in}}%
\pgfpathlineto{\pgfqpoint{1.400454in}{0.991829in}}%
\pgfpathlineto{\pgfqpoint{1.408516in}{0.991829in}}%
\pgfpathlineto{\pgfqpoint{1.416578in}{0.991829in}}%
\pgfpathlineto{\pgfqpoint{1.424640in}{0.962535in}}%
\pgfpathlineto{\pgfqpoint{1.432702in}{0.962535in}}%
\pgfpathlineto{\pgfqpoint{1.440764in}{0.962535in}}%
\pgfpathlineto{\pgfqpoint{1.448826in}{0.962535in}}%
\pgfpathlineto{\pgfqpoint{1.456888in}{0.962535in}}%
\pgfpathlineto{\pgfqpoint{1.464950in}{0.962535in}}%
\pgfpathlineto{\pgfqpoint{1.473012in}{0.962535in}}%
\pgfpathlineto{\pgfqpoint{1.481074in}{0.962535in}}%
\pgfpathlineto{\pgfqpoint{1.489136in}{0.962535in}}%
\pgfpathlineto{\pgfqpoint{1.497198in}{0.962535in}}%
\pgfpathlineto{\pgfqpoint{1.505259in}{0.962535in}}%
\pgfpathlineto{\pgfqpoint{1.513321in}{0.962535in}}%
\pgfpathlineto{\pgfqpoint{1.521383in}{0.962535in}}%
\pgfpathlineto{\pgfqpoint{1.529445in}{0.962535in}}%
\pgfpathlineto{\pgfqpoint{1.537507in}{0.962535in}}%
\pgfpathlineto{\pgfqpoint{1.545569in}{0.962535in}}%
\pgfpathlineto{\pgfqpoint{1.553631in}{0.962535in}}%
\pgfpathlineto{\pgfqpoint{1.561693in}{0.962535in}}%
\pgfpathlineto{\pgfqpoint{1.569755in}{0.962535in}}%
\pgfpathlineto{\pgfqpoint{1.577817in}{0.962535in}}%
\pgfpathlineto{\pgfqpoint{1.585879in}{0.962535in}}%
\pgfpathlineto{\pgfqpoint{1.593941in}{0.961954in}}%
\pgfpathlineto{\pgfqpoint{1.602003in}{0.961954in}}%
\pgfpathlineto{\pgfqpoint{1.610065in}{0.961954in}}%
\pgfpathlineto{\pgfqpoint{1.618127in}{0.961954in}}%
\pgfpathlineto{\pgfqpoint{1.626189in}{0.961954in}}%
\pgfpathlineto{\pgfqpoint{1.634251in}{0.961954in}}%
\pgfpathlineto{\pgfqpoint{1.642313in}{0.961954in}}%
\pgfpathlineto{\pgfqpoint{1.650375in}{0.961954in}}%
\pgfpathlineto{\pgfqpoint{1.658437in}{0.961954in}}%
\pgfpathlineto{\pgfqpoint{1.666499in}{0.961954in}}%
\pgfpathlineto{\pgfqpoint{1.674561in}{0.961954in}}%
\pgfpathlineto{\pgfqpoint{1.682623in}{0.895830in}}%
\pgfpathlineto{\pgfqpoint{1.690685in}{0.895830in}}%
\pgfpathlineto{\pgfqpoint{1.698747in}{0.895830in}}%
\pgfpathlineto{\pgfqpoint{1.706809in}{0.895830in}}%
\pgfpathlineto{\pgfqpoint{1.714871in}{0.895830in}}%
\pgfpathlineto{\pgfqpoint{1.722932in}{0.895830in}}%
\pgfpathlineto{\pgfqpoint{1.730994in}{0.895830in}}%
\pgfpathlineto{\pgfqpoint{1.739056in}{0.895830in}}%
\pgfpathlineto{\pgfqpoint{1.747118in}{0.895830in}}%
\pgfpathlineto{\pgfqpoint{1.755180in}{0.895830in}}%
\pgfpathlineto{\pgfqpoint{1.763242in}{0.895830in}}%
\pgfpathlineto{\pgfqpoint{1.771304in}{0.895830in}}%
\pgfpathlineto{\pgfqpoint{1.779366in}{0.895830in}}%
\pgfpathlineto{\pgfqpoint{1.787428in}{0.895830in}}%
\pgfpathlineto{\pgfqpoint{1.795490in}{0.895830in}}%
\pgfpathlineto{\pgfqpoint{1.803552in}{0.895830in}}%
\pgfpathlineto{\pgfqpoint{1.811614in}{0.895830in}}%
\pgfpathlineto{\pgfqpoint{1.819676in}{0.895830in}}%
\pgfpathlineto{\pgfqpoint{1.827738in}{0.895830in}}%
\pgfpathlineto{\pgfqpoint{1.835800in}{0.895830in}}%
\pgfpathlineto{\pgfqpoint{1.843862in}{0.895830in}}%
\pgfpathlineto{\pgfqpoint{1.851924in}{0.895830in}}%
\pgfpathlineto{\pgfqpoint{1.859986in}{0.841630in}}%
\pgfpathlineto{\pgfqpoint{1.868048in}{0.841630in}}%
\pgfpathlineto{\pgfqpoint{1.876110in}{0.841630in}}%
\pgfpathlineto{\pgfqpoint{1.884172in}{0.841630in}}%
\pgfpathlineto{\pgfqpoint{1.892234in}{0.841630in}}%
\pgfpathlineto{\pgfqpoint{1.900296in}{0.841630in}}%
\pgfpathlineto{\pgfqpoint{1.908358in}{0.841630in}}%
\pgfpathlineto{\pgfqpoint{1.916420in}{0.841630in}}%
\pgfpathlineto{\pgfqpoint{1.924482in}{0.841630in}}%
\pgfpathlineto{\pgfqpoint{1.932544in}{0.841630in}}%
\pgfpathlineto{\pgfqpoint{1.940605in}{0.841630in}}%
\pgfpathlineto{\pgfqpoint{1.948667in}{0.841630in}}%
\pgfpathlineto{\pgfqpoint{1.956729in}{0.841630in}}%
\pgfpathlineto{\pgfqpoint{1.964791in}{0.841630in}}%
\pgfpathlineto{\pgfqpoint{1.972853in}{0.841630in}}%
\pgfpathlineto{\pgfqpoint{1.980915in}{0.841630in}}%
\pgfpathlineto{\pgfqpoint{1.988977in}{0.841630in}}%
\pgfpathlineto{\pgfqpoint{1.988977in}{0.740287in}}%
\pgfpathlineto{\pgfqpoint{1.988977in}{0.740287in}}%
\pgfpathlineto{\pgfqpoint{1.980915in}{0.740287in}}%
\pgfpathlineto{\pgfqpoint{1.972853in}{0.740287in}}%
\pgfpathlineto{\pgfqpoint{1.964791in}{0.740287in}}%
\pgfpathlineto{\pgfqpoint{1.956729in}{0.740287in}}%
\pgfpathlineto{\pgfqpoint{1.948667in}{0.740287in}}%
\pgfpathlineto{\pgfqpoint{1.940605in}{0.740287in}}%
\pgfpathlineto{\pgfqpoint{1.932544in}{0.740287in}}%
\pgfpathlineto{\pgfqpoint{1.924482in}{0.740287in}}%
\pgfpathlineto{\pgfqpoint{1.916420in}{0.740287in}}%
\pgfpathlineto{\pgfqpoint{1.908358in}{0.740287in}}%
\pgfpathlineto{\pgfqpoint{1.900296in}{0.740287in}}%
\pgfpathlineto{\pgfqpoint{1.892234in}{0.740287in}}%
\pgfpathlineto{\pgfqpoint{1.884172in}{0.740287in}}%
\pgfpathlineto{\pgfqpoint{1.876110in}{0.740287in}}%
\pgfpathlineto{\pgfqpoint{1.868048in}{0.740287in}}%
\pgfpathlineto{\pgfqpoint{1.859986in}{0.740287in}}%
\pgfpathlineto{\pgfqpoint{1.851924in}{0.757451in}}%
\pgfpathlineto{\pgfqpoint{1.843862in}{0.757451in}}%
\pgfpathlineto{\pgfqpoint{1.835800in}{0.757451in}}%
\pgfpathlineto{\pgfqpoint{1.827738in}{0.757451in}}%
\pgfpathlineto{\pgfqpoint{1.819676in}{0.757451in}}%
\pgfpathlineto{\pgfqpoint{1.811614in}{0.757451in}}%
\pgfpathlineto{\pgfqpoint{1.803552in}{0.757451in}}%
\pgfpathlineto{\pgfqpoint{1.795490in}{0.757451in}}%
\pgfpathlineto{\pgfqpoint{1.787428in}{0.757451in}}%
\pgfpathlineto{\pgfqpoint{1.779366in}{0.757451in}}%
\pgfpathlineto{\pgfqpoint{1.771304in}{0.757451in}}%
\pgfpathlineto{\pgfqpoint{1.763242in}{0.757451in}}%
\pgfpathlineto{\pgfqpoint{1.755180in}{0.757451in}}%
\pgfpathlineto{\pgfqpoint{1.747118in}{0.757451in}}%
\pgfpathlineto{\pgfqpoint{1.739056in}{0.757451in}}%
\pgfpathlineto{\pgfqpoint{1.730994in}{0.757451in}}%
\pgfpathlineto{\pgfqpoint{1.722932in}{0.757451in}}%
\pgfpathlineto{\pgfqpoint{1.714871in}{0.757451in}}%
\pgfpathlineto{\pgfqpoint{1.706809in}{0.757451in}}%
\pgfpathlineto{\pgfqpoint{1.698747in}{0.757451in}}%
\pgfpathlineto{\pgfqpoint{1.690685in}{0.757451in}}%
\pgfpathlineto{\pgfqpoint{1.682623in}{0.757451in}}%
\pgfpathlineto{\pgfqpoint{1.674561in}{0.774411in}}%
\pgfpathlineto{\pgfqpoint{1.666499in}{0.774411in}}%
\pgfpathlineto{\pgfqpoint{1.658437in}{0.774411in}}%
\pgfpathlineto{\pgfqpoint{1.650375in}{0.774411in}}%
\pgfpathlineto{\pgfqpoint{1.642313in}{0.774411in}}%
\pgfpathlineto{\pgfqpoint{1.634251in}{0.774411in}}%
\pgfpathlineto{\pgfqpoint{1.626189in}{0.774411in}}%
\pgfpathlineto{\pgfqpoint{1.618127in}{0.774411in}}%
\pgfpathlineto{\pgfqpoint{1.610065in}{0.774411in}}%
\pgfpathlineto{\pgfqpoint{1.602003in}{0.774411in}}%
\pgfpathlineto{\pgfqpoint{1.593941in}{0.774411in}}%
\pgfpathlineto{\pgfqpoint{1.585879in}{0.778362in}}%
\pgfpathlineto{\pgfqpoint{1.577817in}{0.778362in}}%
\pgfpathlineto{\pgfqpoint{1.569755in}{0.778362in}}%
\pgfpathlineto{\pgfqpoint{1.561693in}{0.778362in}}%
\pgfpathlineto{\pgfqpoint{1.553631in}{0.778362in}}%
\pgfpathlineto{\pgfqpoint{1.545569in}{0.778362in}}%
\pgfpathlineto{\pgfqpoint{1.537507in}{0.778362in}}%
\pgfpathlineto{\pgfqpoint{1.529445in}{0.778362in}}%
\pgfpathlineto{\pgfqpoint{1.521383in}{0.778362in}}%
\pgfpathlineto{\pgfqpoint{1.513321in}{0.778362in}}%
\pgfpathlineto{\pgfqpoint{1.505259in}{0.778362in}}%
\pgfpathlineto{\pgfqpoint{1.497198in}{0.778362in}}%
\pgfpathlineto{\pgfqpoint{1.489136in}{0.778362in}}%
\pgfpathlineto{\pgfqpoint{1.481074in}{0.778362in}}%
\pgfpathlineto{\pgfqpoint{1.473012in}{0.778362in}}%
\pgfpathlineto{\pgfqpoint{1.464950in}{0.778362in}}%
\pgfpathlineto{\pgfqpoint{1.456888in}{0.778362in}}%
\pgfpathlineto{\pgfqpoint{1.448826in}{0.778362in}}%
\pgfpathlineto{\pgfqpoint{1.440764in}{0.778362in}}%
\pgfpathlineto{\pgfqpoint{1.432702in}{0.778362in}}%
\pgfpathlineto{\pgfqpoint{1.424640in}{0.778362in}}%
\pgfpathlineto{\pgfqpoint{1.416578in}{0.890649in}}%
\pgfpathlineto{\pgfqpoint{1.408516in}{0.890649in}}%
\pgfpathlineto{\pgfqpoint{1.400454in}{0.890649in}}%
\pgfpathlineto{\pgfqpoint{1.392392in}{0.890649in}}%
\pgfpathlineto{\pgfqpoint{1.384330in}{0.894530in}}%
\pgfpathlineto{\pgfqpoint{1.376268in}{0.894530in}}%
\pgfpathlineto{\pgfqpoint{1.368206in}{0.894530in}}%
\pgfpathlineto{\pgfqpoint{1.360144in}{0.894530in}}%
\pgfpathlineto{\pgfqpoint{1.352082in}{0.894530in}}%
\pgfpathlineto{\pgfqpoint{1.344020in}{0.894530in}}%
\pgfpathlineto{\pgfqpoint{1.335958in}{0.894530in}}%
\pgfpathlineto{\pgfqpoint{1.327896in}{0.894530in}}%
\pgfpathlineto{\pgfqpoint{1.319834in}{0.894530in}}%
\pgfpathlineto{\pgfqpoint{1.311772in}{0.894530in}}%
\pgfpathlineto{\pgfqpoint{1.303710in}{0.894530in}}%
\pgfpathlineto{\pgfqpoint{1.295648in}{0.954524in}}%
\pgfpathlineto{\pgfqpoint{1.287586in}{0.954524in}}%
\pgfpathlineto{\pgfqpoint{1.279525in}{0.954524in}}%
\pgfpathlineto{\pgfqpoint{1.271463in}{0.954524in}}%
\pgfpathlineto{\pgfqpoint{1.263401in}{0.954524in}}%
\pgfpathlineto{\pgfqpoint{1.255339in}{0.954524in}}%
\pgfpathlineto{\pgfqpoint{1.247277in}{0.954524in}}%
\pgfpathlineto{\pgfqpoint{1.239215in}{1.056455in}}%
\pgfpathlineto{\pgfqpoint{1.231153in}{1.056455in}}%
\pgfpathlineto{\pgfqpoint{1.223091in}{1.056455in}}%
\pgfpathlineto{\pgfqpoint{1.215029in}{1.056455in}}%
\pgfpathlineto{\pgfqpoint{1.206967in}{1.056455in}}%
\pgfpathlineto{\pgfqpoint{1.198905in}{1.083884in}}%
\pgfpathlineto{\pgfqpoint{1.190843in}{1.083884in}}%
\pgfpathlineto{\pgfqpoint{1.182781in}{1.083884in}}%
\pgfpathlineto{\pgfqpoint{1.174719in}{1.083884in}}%
\pgfpathlineto{\pgfqpoint{1.166657in}{1.083884in}}%
\pgfpathlineto{\pgfqpoint{1.158595in}{1.083884in}}%
\pgfpathlineto{\pgfqpoint{1.150533in}{1.083884in}}%
\pgfpathlineto{\pgfqpoint{1.142471in}{1.083884in}}%
\pgfpathlineto{\pgfqpoint{1.134409in}{1.083884in}}%
\pgfpathlineto{\pgfqpoint{1.126347in}{1.083884in}}%
\pgfpathlineto{\pgfqpoint{1.118285in}{1.083884in}}%
\pgfpathlineto{\pgfqpoint{1.110223in}{1.083884in}}%
\pgfpathlineto{\pgfqpoint{1.102161in}{1.083884in}}%
\pgfpathlineto{\pgfqpoint{1.094099in}{1.083884in}}%
\pgfpathlineto{\pgfqpoint{1.086037in}{1.083884in}}%
\pgfpathlineto{\pgfqpoint{1.077975in}{1.083884in}}%
\pgfpathlineto{\pgfqpoint{1.069914in}{1.083884in}}%
\pgfpathlineto{\pgfqpoint{1.061852in}{1.083884in}}%
\pgfpathlineto{\pgfqpoint{1.053790in}{1.180357in}}%
\pgfpathlineto{\pgfqpoint{1.045728in}{1.190517in}}%
\pgfpathlineto{\pgfqpoint{1.037666in}{1.190581in}}%
\pgfpathlineto{\pgfqpoint{1.029604in}{1.190581in}}%
\pgfpathlineto{\pgfqpoint{1.021542in}{1.190581in}}%
\pgfpathlineto{\pgfqpoint{1.013480in}{1.190581in}}%
\pgfpathlineto{\pgfqpoint{1.005418in}{1.190581in}}%
\pgfpathlineto{\pgfqpoint{0.997356in}{1.190581in}}%
\pgfpathlineto{\pgfqpoint{0.989294in}{1.190581in}}%
\pgfpathlineto{\pgfqpoint{0.981232in}{1.190581in}}%
\pgfpathlineto{\pgfqpoint{0.973170in}{1.190581in}}%
\pgfpathlineto{\pgfqpoint{0.965108in}{1.190581in}}%
\pgfpathlineto{\pgfqpoint{0.957046in}{1.304672in}}%
\pgfpathlineto{\pgfqpoint{0.948984in}{1.304672in}}%
\pgfpathlineto{\pgfqpoint{0.940922in}{1.304672in}}%
\pgfpathlineto{\pgfqpoint{0.932860in}{1.304672in}}%
\pgfpathlineto{\pgfqpoint{0.924798in}{1.376430in}}%
\pgfpathlineto{\pgfqpoint{0.916736in}{1.376430in}}%
\pgfpathlineto{\pgfqpoint{0.908674in}{1.416475in}}%
\pgfpathlineto{\pgfqpoint{0.900612in}{1.416475in}}%
\pgfpathlineto{\pgfqpoint{0.892550in}{1.416475in}}%
\pgfpathlineto{\pgfqpoint{0.884488in}{1.431922in}}%
\pgfpathlineto{\pgfqpoint{0.876426in}{1.709458in}}%
\pgfpathlineto{\pgfqpoint{0.868364in}{1.709458in}}%
\pgfpathlineto{\pgfqpoint{0.860302in}{1.709458in}}%
\pgfpathlineto{\pgfqpoint{0.852241in}{1.709458in}}%
\pgfpathlineto{\pgfqpoint{0.844179in}{1.709458in}}%
\pgfpathlineto{\pgfqpoint{0.836117in}{1.709458in}}%
\pgfpathlineto{\pgfqpoint{0.828055in}{1.709458in}}%
\pgfpathlineto{\pgfqpoint{0.819993in}{1.712193in}}%
\pgfpathlineto{\pgfqpoint{0.811931in}{1.712193in}}%
\pgfpathlineto{\pgfqpoint{0.803869in}{1.712193in}}%
\pgfpathlineto{\pgfqpoint{0.795807in}{1.748199in}}%
\pgfpathlineto{\pgfqpoint{0.787745in}{1.780438in}}%
\pgfpathlineto{\pgfqpoint{0.779683in}{1.780438in}}%
\pgfpathlineto{\pgfqpoint{0.771621in}{1.945548in}}%
\pgfpathlineto{\pgfqpoint{0.763559in}{1.953282in}}%
\pgfpathlineto{\pgfqpoint{0.755497in}{2.040965in}}%
\pgfpathlineto{\pgfqpoint{0.747435in}{2.040965in}}%
\pgfpathlineto{\pgfqpoint{0.739373in}{2.050163in}}%
\pgfpathlineto{\pgfqpoint{0.731311in}{2.050163in}}%
\pgfpathlineto{\pgfqpoint{0.723249in}{2.050163in}}%
\pgfpathlineto{\pgfqpoint{0.715187in}{2.050163in}}%
\pgfpathlineto{\pgfqpoint{0.707125in}{2.050163in}}%
\pgfpathlineto{\pgfqpoint{0.699063in}{2.050163in}}%
\pgfpathlineto{\pgfqpoint{0.691001in}{2.050163in}}%
\pgfpathlineto{\pgfqpoint{0.682939in}{2.131818in}}%
\pgfpathlineto{\pgfqpoint{0.674877in}{2.131818in}}%
\pgfpathlineto{\pgfqpoint{0.666815in}{2.131818in}}%
\pgfpathlineto{\pgfqpoint{0.658753in}{2.131818in}}%
\pgfpathlineto{\pgfqpoint{0.650691in}{2.131818in}}%
\pgfpathlineto{\pgfqpoint{0.642629in}{2.131818in}}%
\pgfpathlineto{\pgfqpoint{0.634568in}{2.131818in}}%
\pgfpathlineto{\pgfqpoint{0.626506in}{2.131818in}}%
\pgfpathlineto{\pgfqpoint{0.618444in}{2.131818in}}%
\pgfpathlineto{\pgfqpoint{0.610382in}{2.131818in}}%
\pgfpathlineto{\pgfqpoint{0.602320in}{2.131818in}}%
\pgfpathlineto{\pgfqpoint{0.594258in}{2.131818in}}%
\pgfpathlineto{\pgfqpoint{0.586196in}{2.131818in}}%
\pgfpathlineto{\pgfqpoint{0.578134in}{2.131818in}}%
\pgfpathlineto{\pgfqpoint{0.570072in}{2.131818in}}%
\pgfpathlineto{\pgfqpoint{0.562010in}{2.163426in}}%
\pgfpathlineto{\pgfqpoint{0.553948in}{2.166437in}}%
\pgfpathlineto{\pgfqpoint{0.545886in}{2.235436in}}%
\pgfpathlineto{\pgfqpoint{0.537824in}{2.251851in}}%
\pgfpathlineto{\pgfqpoint{0.529762in}{2.251851in}}%
\pgfpathlineto{\pgfqpoint{0.521700in}{2.251851in}}%
\pgfpathlineto{\pgfqpoint{0.513638in}{2.251851in}}%
\pgfpathlineto{\pgfqpoint{0.505576in}{2.315525in}}%
\pgfpathlineto{\pgfqpoint{0.497514in}{2.315525in}}%
\pgfpathlineto{\pgfqpoint{0.489452in}{2.315525in}}%
\pgfpathlineto{\pgfqpoint{0.481390in}{2.315525in}}%
\pgfpathlineto{\pgfqpoint{0.473328in}{2.315525in}}%
\pgfpathlineto{\pgfqpoint{0.465266in}{2.315525in}}%
\pgfpathlineto{\pgfqpoint{0.457204in}{2.315525in}}%
\pgfpathlineto{\pgfqpoint{0.449142in}{2.321389in}}%
\pgfpathlineto{\pgfqpoint{0.441080in}{2.321389in}}%
\pgfpathlineto{\pgfqpoint{0.433018in}{2.352101in}}%
\pgfpathlineto{\pgfqpoint{0.424956in}{2.375735in}}%
\pgfpathlineto{\pgfqpoint{0.416895in}{2.375735in}}%
\pgfpathlineto{\pgfqpoint{0.408833in}{2.383561in}}%
\pgfpathlineto{\pgfqpoint{0.400771in}{2.385763in}}%
\pgfpathlineto{\pgfqpoint{0.392709in}{2.385763in}}%
\pgfpathlineto{\pgfqpoint{0.384647in}{2.385763in}}%
\pgfpathlineto{\pgfqpoint{0.376585in}{2.391380in}}%
\pgfpathlineto{\pgfqpoint{0.368523in}{2.483219in}}%
\pgfpathclose%
\pgfusepath{fill}%
\end{pgfscope}%
\begin{pgfscope}%
\pgfpathrectangle{\pgfqpoint{0.287500in}{0.375000in}}{\pgfqpoint{1.782500in}{2.265000in}}%
\pgfusepath{clip}%
\pgfsetbuttcap%
\pgfsetroundjoin%
\definecolor{currentfill}{rgb}{0.839216,0.152941,0.156863}%
\pgfsetfillcolor{currentfill}%
\pgfsetfillopacity{0.200000}%
\pgfsetlinewidth{0.000000pt}%
\definecolor{currentstroke}{rgb}{0.000000,0.000000,0.000000}%
\pgfsetstrokecolor{currentstroke}%
\pgfsetdash{}{0pt}%
\pgfpathmoveto{\pgfqpoint{0.368523in}{2.491663in}}%
\pgfpathlineto{\pgfqpoint{0.368523in}{2.525463in}}%
\pgfpathlineto{\pgfqpoint{0.376585in}{2.516202in}}%
\pgfpathlineto{\pgfqpoint{0.384647in}{2.469387in}}%
\pgfpathlineto{\pgfqpoint{0.392709in}{2.464068in}}%
\pgfpathlineto{\pgfqpoint{0.400771in}{2.451287in}}%
\pgfpathlineto{\pgfqpoint{0.408833in}{2.413284in}}%
\pgfpathlineto{\pgfqpoint{0.416895in}{2.413273in}}%
\pgfpathlineto{\pgfqpoint{0.424956in}{2.409564in}}%
\pgfpathlineto{\pgfqpoint{0.433018in}{2.405300in}}%
\pgfpathlineto{\pgfqpoint{0.441080in}{2.405300in}}%
\pgfpathlineto{\pgfqpoint{0.449142in}{2.405300in}}%
\pgfpathlineto{\pgfqpoint{0.457204in}{2.402233in}}%
\pgfpathlineto{\pgfqpoint{0.465266in}{2.389492in}}%
\pgfpathlineto{\pgfqpoint{0.473328in}{2.373578in}}%
\pgfpathlineto{\pgfqpoint{0.481390in}{2.373578in}}%
\pgfpathlineto{\pgfqpoint{0.489452in}{2.373507in}}%
\pgfpathlineto{\pgfqpoint{0.497514in}{2.367378in}}%
\pgfpathlineto{\pgfqpoint{0.505576in}{2.367378in}}%
\pgfpathlineto{\pgfqpoint{0.513638in}{2.367378in}}%
\pgfpathlineto{\pgfqpoint{0.521700in}{2.367378in}}%
\pgfpathlineto{\pgfqpoint{0.529762in}{2.367378in}}%
\pgfpathlineto{\pgfqpoint{0.537824in}{2.367378in}}%
\pgfpathlineto{\pgfqpoint{0.545886in}{2.367161in}}%
\pgfpathlineto{\pgfqpoint{0.553948in}{2.367161in}}%
\pgfpathlineto{\pgfqpoint{0.562010in}{2.367161in}}%
\pgfpathlineto{\pgfqpoint{0.570072in}{2.367161in}}%
\pgfpathlineto{\pgfqpoint{0.578134in}{2.366847in}}%
\pgfpathlineto{\pgfqpoint{0.586196in}{2.366847in}}%
\pgfpathlineto{\pgfqpoint{0.594258in}{2.366847in}}%
\pgfpathlineto{\pgfqpoint{0.602320in}{2.280395in}}%
\pgfpathlineto{\pgfqpoint{0.610382in}{2.259199in}}%
\pgfpathlineto{\pgfqpoint{0.618444in}{2.259199in}}%
\pgfpathlineto{\pgfqpoint{0.626506in}{2.259199in}}%
\pgfpathlineto{\pgfqpoint{0.634568in}{2.259199in}}%
\pgfpathlineto{\pgfqpoint{0.642629in}{2.259199in}}%
\pgfpathlineto{\pgfqpoint{0.650691in}{2.259199in}}%
\pgfpathlineto{\pgfqpoint{0.658753in}{2.257369in}}%
\pgfpathlineto{\pgfqpoint{0.666815in}{2.257369in}}%
\pgfpathlineto{\pgfqpoint{0.674877in}{2.257369in}}%
\pgfpathlineto{\pgfqpoint{0.682939in}{2.253233in}}%
\pgfpathlineto{\pgfqpoint{0.691001in}{2.248647in}}%
\pgfpathlineto{\pgfqpoint{0.699063in}{2.248647in}}%
\pgfpathlineto{\pgfqpoint{0.707125in}{2.243431in}}%
\pgfpathlineto{\pgfqpoint{0.715187in}{2.235744in}}%
\pgfpathlineto{\pgfqpoint{0.723249in}{2.235744in}}%
\pgfpathlineto{\pgfqpoint{0.731311in}{2.235373in}}%
\pgfpathlineto{\pgfqpoint{0.739373in}{2.235373in}}%
\pgfpathlineto{\pgfqpoint{0.747435in}{2.235373in}}%
\pgfpathlineto{\pgfqpoint{0.755497in}{2.235373in}}%
\pgfpathlineto{\pgfqpoint{0.763559in}{2.235373in}}%
\pgfpathlineto{\pgfqpoint{0.771621in}{2.235373in}}%
\pgfpathlineto{\pgfqpoint{0.779683in}{2.235373in}}%
\pgfpathlineto{\pgfqpoint{0.787745in}{2.235373in}}%
\pgfpathlineto{\pgfqpoint{0.795807in}{2.235373in}}%
\pgfpathlineto{\pgfqpoint{0.803869in}{2.235373in}}%
\pgfpathlineto{\pgfqpoint{0.811931in}{2.235373in}}%
\pgfpathlineto{\pgfqpoint{0.819993in}{2.235373in}}%
\pgfpathlineto{\pgfqpoint{0.828055in}{2.235373in}}%
\pgfpathlineto{\pgfqpoint{0.836117in}{2.204914in}}%
\pgfpathlineto{\pgfqpoint{0.844179in}{2.204914in}}%
\pgfpathlineto{\pgfqpoint{0.852241in}{2.204382in}}%
\pgfpathlineto{\pgfqpoint{0.860302in}{2.204382in}}%
\pgfpathlineto{\pgfqpoint{0.868364in}{2.204382in}}%
\pgfpathlineto{\pgfqpoint{0.876426in}{2.204192in}}%
\pgfpathlineto{\pgfqpoint{0.884488in}{2.204192in}}%
\pgfpathlineto{\pgfqpoint{0.892550in}{2.204192in}}%
\pgfpathlineto{\pgfqpoint{0.900612in}{2.204192in}}%
\pgfpathlineto{\pgfqpoint{0.908674in}{2.204192in}}%
\pgfpathlineto{\pgfqpoint{0.916736in}{2.204192in}}%
\pgfpathlineto{\pgfqpoint{0.924798in}{2.204192in}}%
\pgfpathlineto{\pgfqpoint{0.932860in}{2.204192in}}%
\pgfpathlineto{\pgfqpoint{0.940922in}{2.204192in}}%
\pgfpathlineto{\pgfqpoint{0.948984in}{2.204192in}}%
\pgfpathlineto{\pgfqpoint{0.957046in}{2.204192in}}%
\pgfpathlineto{\pgfqpoint{0.965108in}{2.204192in}}%
\pgfpathlineto{\pgfqpoint{0.973170in}{2.204192in}}%
\pgfpathlineto{\pgfqpoint{0.981232in}{2.204192in}}%
\pgfpathlineto{\pgfqpoint{0.989294in}{2.204192in}}%
\pgfpathlineto{\pgfqpoint{0.997356in}{2.204192in}}%
\pgfpathlineto{\pgfqpoint{1.005418in}{2.204192in}}%
\pgfpathlineto{\pgfqpoint{1.013480in}{2.204192in}}%
\pgfpathlineto{\pgfqpoint{1.021542in}{2.204192in}}%
\pgfpathlineto{\pgfqpoint{1.029604in}{2.175698in}}%
\pgfpathlineto{\pgfqpoint{1.037666in}{2.175698in}}%
\pgfpathlineto{\pgfqpoint{1.045728in}{2.175698in}}%
\pgfpathlineto{\pgfqpoint{1.053790in}{2.175698in}}%
\pgfpathlineto{\pgfqpoint{1.061852in}{2.167055in}}%
\pgfpathlineto{\pgfqpoint{1.069914in}{2.167055in}}%
\pgfpathlineto{\pgfqpoint{1.077975in}{2.167055in}}%
\pgfpathlineto{\pgfqpoint{1.086037in}{2.167055in}}%
\pgfpathlineto{\pgfqpoint{1.094099in}{2.166402in}}%
\pgfpathlineto{\pgfqpoint{1.102161in}{2.166402in}}%
\pgfpathlineto{\pgfqpoint{1.110223in}{2.166402in}}%
\pgfpathlineto{\pgfqpoint{1.118285in}{2.166402in}}%
\pgfpathlineto{\pgfqpoint{1.126347in}{2.166402in}}%
\pgfpathlineto{\pgfqpoint{1.134409in}{2.166402in}}%
\pgfpathlineto{\pgfqpoint{1.142471in}{2.166348in}}%
\pgfpathlineto{\pgfqpoint{1.150533in}{2.166348in}}%
\pgfpathlineto{\pgfqpoint{1.158595in}{1.972563in}}%
\pgfpathlineto{\pgfqpoint{1.166657in}{1.972563in}}%
\pgfpathlineto{\pgfqpoint{1.174719in}{1.972563in}}%
\pgfpathlineto{\pgfqpoint{1.182781in}{1.972563in}}%
\pgfpathlineto{\pgfqpoint{1.190843in}{1.972563in}}%
\pgfpathlineto{\pgfqpoint{1.198905in}{1.972563in}}%
\pgfpathlineto{\pgfqpoint{1.206967in}{1.972563in}}%
\pgfpathlineto{\pgfqpoint{1.215029in}{1.972563in}}%
\pgfpathlineto{\pgfqpoint{1.223091in}{1.972563in}}%
\pgfpathlineto{\pgfqpoint{1.231153in}{1.972563in}}%
\pgfpathlineto{\pgfqpoint{1.239215in}{1.972563in}}%
\pgfpathlineto{\pgfqpoint{1.247277in}{1.972563in}}%
\pgfpathlineto{\pgfqpoint{1.255339in}{1.972563in}}%
\pgfpathlineto{\pgfqpoint{1.263401in}{1.972563in}}%
\pgfpathlineto{\pgfqpoint{1.271463in}{1.972563in}}%
\pgfpathlineto{\pgfqpoint{1.279525in}{1.972563in}}%
\pgfpathlineto{\pgfqpoint{1.287586in}{1.972563in}}%
\pgfpathlineto{\pgfqpoint{1.295648in}{1.972563in}}%
\pgfpathlineto{\pgfqpoint{1.303710in}{1.972563in}}%
\pgfpathlineto{\pgfqpoint{1.311772in}{1.972563in}}%
\pgfpathlineto{\pgfqpoint{1.319834in}{1.972563in}}%
\pgfpathlineto{\pgfqpoint{1.327896in}{1.972563in}}%
\pgfpathlineto{\pgfqpoint{1.335958in}{1.972563in}}%
\pgfpathlineto{\pgfqpoint{1.344020in}{1.972563in}}%
\pgfpathlineto{\pgfqpoint{1.352082in}{1.972563in}}%
\pgfpathlineto{\pgfqpoint{1.360144in}{1.972563in}}%
\pgfpathlineto{\pgfqpoint{1.368206in}{1.972563in}}%
\pgfpathlineto{\pgfqpoint{1.376268in}{1.972563in}}%
\pgfpathlineto{\pgfqpoint{1.384330in}{1.972563in}}%
\pgfpathlineto{\pgfqpoint{1.392392in}{1.972563in}}%
\pgfpathlineto{\pgfqpoint{1.400454in}{1.972563in}}%
\pgfpathlineto{\pgfqpoint{1.408516in}{1.972563in}}%
\pgfpathlineto{\pgfqpoint{1.416578in}{1.972563in}}%
\pgfpathlineto{\pgfqpoint{1.424640in}{1.972563in}}%
\pgfpathlineto{\pgfqpoint{1.432702in}{1.972563in}}%
\pgfpathlineto{\pgfqpoint{1.440764in}{1.972563in}}%
\pgfpathlineto{\pgfqpoint{1.448826in}{1.972563in}}%
\pgfpathlineto{\pgfqpoint{1.456888in}{1.972563in}}%
\pgfpathlineto{\pgfqpoint{1.464950in}{1.972563in}}%
\pgfpathlineto{\pgfqpoint{1.473012in}{1.972563in}}%
\pgfpathlineto{\pgfqpoint{1.481074in}{1.972563in}}%
\pgfpathlineto{\pgfqpoint{1.489136in}{1.972563in}}%
\pgfpathlineto{\pgfqpoint{1.497198in}{1.972563in}}%
\pgfpathlineto{\pgfqpoint{1.505259in}{1.972563in}}%
\pgfpathlineto{\pgfqpoint{1.513321in}{1.972561in}}%
\pgfpathlineto{\pgfqpoint{1.521383in}{1.972561in}}%
\pgfpathlineto{\pgfqpoint{1.529445in}{1.972514in}}%
\pgfpathlineto{\pgfqpoint{1.537507in}{1.854779in}}%
\pgfpathlineto{\pgfqpoint{1.545569in}{1.854779in}}%
\pgfpathlineto{\pgfqpoint{1.553631in}{1.854779in}}%
\pgfpathlineto{\pgfqpoint{1.561693in}{1.854779in}}%
\pgfpathlineto{\pgfqpoint{1.569755in}{1.854779in}}%
\pgfpathlineto{\pgfqpoint{1.577817in}{1.823339in}}%
\pgfpathlineto{\pgfqpoint{1.585879in}{1.823339in}}%
\pgfpathlineto{\pgfqpoint{1.593941in}{1.823339in}}%
\pgfpathlineto{\pgfqpoint{1.602003in}{1.823339in}}%
\pgfpathlineto{\pgfqpoint{1.610065in}{1.823339in}}%
\pgfpathlineto{\pgfqpoint{1.618127in}{1.823339in}}%
\pgfpathlineto{\pgfqpoint{1.626189in}{1.823339in}}%
\pgfpathlineto{\pgfqpoint{1.634251in}{1.823339in}}%
\pgfpathlineto{\pgfqpoint{1.642313in}{1.823339in}}%
\pgfpathlineto{\pgfqpoint{1.650375in}{1.823339in}}%
\pgfpathlineto{\pgfqpoint{1.658437in}{1.823339in}}%
\pgfpathlineto{\pgfqpoint{1.666499in}{1.823339in}}%
\pgfpathlineto{\pgfqpoint{1.674561in}{1.823331in}}%
\pgfpathlineto{\pgfqpoint{1.682623in}{1.823331in}}%
\pgfpathlineto{\pgfqpoint{1.690685in}{1.823331in}}%
\pgfpathlineto{\pgfqpoint{1.698747in}{1.823331in}}%
\pgfpathlineto{\pgfqpoint{1.706809in}{1.823331in}}%
\pgfpathlineto{\pgfqpoint{1.714871in}{1.823331in}}%
\pgfpathlineto{\pgfqpoint{1.722932in}{1.823331in}}%
\pgfpathlineto{\pgfqpoint{1.730994in}{1.823331in}}%
\pgfpathlineto{\pgfqpoint{1.739056in}{1.823331in}}%
\pgfpathlineto{\pgfqpoint{1.747118in}{1.823331in}}%
\pgfpathlineto{\pgfqpoint{1.755180in}{1.823331in}}%
\pgfpathlineto{\pgfqpoint{1.763242in}{1.823331in}}%
\pgfpathlineto{\pgfqpoint{1.771304in}{1.823331in}}%
\pgfpathlineto{\pgfqpoint{1.779366in}{1.726330in}}%
\pgfpathlineto{\pgfqpoint{1.787428in}{1.726322in}}%
\pgfpathlineto{\pgfqpoint{1.795490in}{1.726322in}}%
\pgfpathlineto{\pgfqpoint{1.803552in}{1.599399in}}%
\pgfpathlineto{\pgfqpoint{1.811614in}{1.599399in}}%
\pgfpathlineto{\pgfqpoint{1.819676in}{1.599399in}}%
\pgfpathlineto{\pgfqpoint{1.827738in}{1.599388in}}%
\pgfpathlineto{\pgfqpoint{1.835800in}{1.599388in}}%
\pgfpathlineto{\pgfqpoint{1.843862in}{1.599388in}}%
\pgfpathlineto{\pgfqpoint{1.851924in}{1.599388in}}%
\pgfpathlineto{\pgfqpoint{1.859986in}{1.599388in}}%
\pgfpathlineto{\pgfqpoint{1.868048in}{1.599388in}}%
\pgfpathlineto{\pgfqpoint{1.876110in}{1.583160in}}%
\pgfpathlineto{\pgfqpoint{1.884172in}{1.583160in}}%
\pgfpathlineto{\pgfqpoint{1.892234in}{1.583160in}}%
\pgfpathlineto{\pgfqpoint{1.900296in}{1.578039in}}%
\pgfpathlineto{\pgfqpoint{1.908358in}{1.578039in}}%
\pgfpathlineto{\pgfqpoint{1.916420in}{1.578039in}}%
\pgfpathlineto{\pgfqpoint{1.924482in}{1.578039in}}%
\pgfpathlineto{\pgfqpoint{1.932544in}{1.578039in}}%
\pgfpathlineto{\pgfqpoint{1.940605in}{1.481635in}}%
\pgfpathlineto{\pgfqpoint{1.948667in}{1.481635in}}%
\pgfpathlineto{\pgfqpoint{1.956729in}{1.481635in}}%
\pgfpathlineto{\pgfqpoint{1.964791in}{1.471412in}}%
\pgfpathlineto{\pgfqpoint{1.972853in}{1.135130in}}%
\pgfpathlineto{\pgfqpoint{1.980915in}{1.135130in}}%
\pgfpathlineto{\pgfqpoint{1.988977in}{1.135130in}}%
\pgfpathlineto{\pgfqpoint{1.988977in}{1.007428in}}%
\pgfpathlineto{\pgfqpoint{1.988977in}{1.007428in}}%
\pgfpathlineto{\pgfqpoint{1.980915in}{1.007428in}}%
\pgfpathlineto{\pgfqpoint{1.972853in}{1.007428in}}%
\pgfpathlineto{\pgfqpoint{1.964791in}{1.178872in}}%
\pgfpathlineto{\pgfqpoint{1.956729in}{1.276037in}}%
\pgfpathlineto{\pgfqpoint{1.948667in}{1.276037in}}%
\pgfpathlineto{\pgfqpoint{1.940605in}{1.276037in}}%
\pgfpathlineto{\pgfqpoint{1.932544in}{1.314849in}}%
\pgfpathlineto{\pgfqpoint{1.924482in}{1.314849in}}%
\pgfpathlineto{\pgfqpoint{1.916420in}{1.314849in}}%
\pgfpathlineto{\pgfqpoint{1.908358in}{1.314849in}}%
\pgfpathlineto{\pgfqpoint{1.900296in}{1.314849in}}%
\pgfpathlineto{\pgfqpoint{1.892234in}{1.370871in}}%
\pgfpathlineto{\pgfqpoint{1.884172in}{1.370875in}}%
\pgfpathlineto{\pgfqpoint{1.876110in}{1.370875in}}%
\pgfpathlineto{\pgfqpoint{1.868048in}{1.459776in}}%
\pgfpathlineto{\pgfqpoint{1.859986in}{1.459776in}}%
\pgfpathlineto{\pgfqpoint{1.851924in}{1.459780in}}%
\pgfpathlineto{\pgfqpoint{1.843862in}{1.459780in}}%
\pgfpathlineto{\pgfqpoint{1.835800in}{1.459780in}}%
\pgfpathlineto{\pgfqpoint{1.827738in}{1.459781in}}%
\pgfpathlineto{\pgfqpoint{1.819676in}{1.459869in}}%
\pgfpathlineto{\pgfqpoint{1.811614in}{1.459869in}}%
\pgfpathlineto{\pgfqpoint{1.803552in}{1.459869in}}%
\pgfpathlineto{\pgfqpoint{1.795490in}{1.572354in}}%
\pgfpathlineto{\pgfqpoint{1.787428in}{1.572354in}}%
\pgfpathlineto{\pgfqpoint{1.779366in}{1.572421in}}%
\pgfpathlineto{\pgfqpoint{1.771304in}{1.600648in}}%
\pgfpathlineto{\pgfqpoint{1.763242in}{1.600649in}}%
\pgfpathlineto{\pgfqpoint{1.755180in}{1.600649in}}%
\pgfpathlineto{\pgfqpoint{1.747118in}{1.600649in}}%
\pgfpathlineto{\pgfqpoint{1.739056in}{1.600649in}}%
\pgfpathlineto{\pgfqpoint{1.730994in}{1.600649in}}%
\pgfpathlineto{\pgfqpoint{1.722932in}{1.600649in}}%
\pgfpathlineto{\pgfqpoint{1.714871in}{1.600649in}}%
\pgfpathlineto{\pgfqpoint{1.706809in}{1.600649in}}%
\pgfpathlineto{\pgfqpoint{1.698747in}{1.600649in}}%
\pgfpathlineto{\pgfqpoint{1.690685in}{1.600649in}}%
\pgfpathlineto{\pgfqpoint{1.682623in}{1.600649in}}%
\pgfpathlineto{\pgfqpoint{1.674561in}{1.600649in}}%
\pgfpathlineto{\pgfqpoint{1.666499in}{1.600738in}}%
\pgfpathlineto{\pgfqpoint{1.658437in}{1.600738in}}%
\pgfpathlineto{\pgfqpoint{1.650375in}{1.600738in}}%
\pgfpathlineto{\pgfqpoint{1.642313in}{1.600738in}}%
\pgfpathlineto{\pgfqpoint{1.634251in}{1.600738in}}%
\pgfpathlineto{\pgfqpoint{1.626189in}{1.600738in}}%
\pgfpathlineto{\pgfqpoint{1.618127in}{1.600738in}}%
\pgfpathlineto{\pgfqpoint{1.610065in}{1.600738in}}%
\pgfpathlineto{\pgfqpoint{1.602003in}{1.600738in}}%
\pgfpathlineto{\pgfqpoint{1.593941in}{1.600738in}}%
\pgfpathlineto{\pgfqpoint{1.585879in}{1.600738in}}%
\pgfpathlineto{\pgfqpoint{1.577817in}{1.600738in}}%
\pgfpathlineto{\pgfqpoint{1.569755in}{1.719317in}}%
\pgfpathlineto{\pgfqpoint{1.561693in}{1.719317in}}%
\pgfpathlineto{\pgfqpoint{1.553631in}{1.719317in}}%
\pgfpathlineto{\pgfqpoint{1.545569in}{1.719317in}}%
\pgfpathlineto{\pgfqpoint{1.537507in}{1.719317in}}%
\pgfpathlineto{\pgfqpoint{1.529445in}{1.760847in}}%
\pgfpathlineto{\pgfqpoint{1.521383in}{1.761362in}}%
\pgfpathlineto{\pgfqpoint{1.513321in}{1.761362in}}%
\pgfpathlineto{\pgfqpoint{1.505259in}{1.761381in}}%
\pgfpathlineto{\pgfqpoint{1.497198in}{1.761381in}}%
\pgfpathlineto{\pgfqpoint{1.489136in}{1.761381in}}%
\pgfpathlineto{\pgfqpoint{1.481074in}{1.761381in}}%
\pgfpathlineto{\pgfqpoint{1.473012in}{1.761381in}}%
\pgfpathlineto{\pgfqpoint{1.464950in}{1.761381in}}%
\pgfpathlineto{\pgfqpoint{1.456888in}{1.761381in}}%
\pgfpathlineto{\pgfqpoint{1.448826in}{1.761381in}}%
\pgfpathlineto{\pgfqpoint{1.440764in}{1.761381in}}%
\pgfpathlineto{\pgfqpoint{1.432702in}{1.761381in}}%
\pgfpathlineto{\pgfqpoint{1.424640in}{1.761381in}}%
\pgfpathlineto{\pgfqpoint{1.416578in}{1.761381in}}%
\pgfpathlineto{\pgfqpoint{1.408516in}{1.761381in}}%
\pgfpathlineto{\pgfqpoint{1.400454in}{1.761381in}}%
\pgfpathlineto{\pgfqpoint{1.392392in}{1.761381in}}%
\pgfpathlineto{\pgfqpoint{1.384330in}{1.761381in}}%
\pgfpathlineto{\pgfqpoint{1.376268in}{1.761381in}}%
\pgfpathlineto{\pgfqpoint{1.368206in}{1.761381in}}%
\pgfpathlineto{\pgfqpoint{1.360144in}{1.761381in}}%
\pgfpathlineto{\pgfqpoint{1.352082in}{1.761381in}}%
\pgfpathlineto{\pgfqpoint{1.344020in}{1.761381in}}%
\pgfpathlineto{\pgfqpoint{1.335958in}{1.761381in}}%
\pgfpathlineto{\pgfqpoint{1.327896in}{1.761381in}}%
\pgfpathlineto{\pgfqpoint{1.319834in}{1.761381in}}%
\pgfpathlineto{\pgfqpoint{1.311772in}{1.761381in}}%
\pgfpathlineto{\pgfqpoint{1.303710in}{1.761381in}}%
\pgfpathlineto{\pgfqpoint{1.295648in}{1.761381in}}%
\pgfpathlineto{\pgfqpoint{1.287586in}{1.761381in}}%
\pgfpathlineto{\pgfqpoint{1.279525in}{1.761381in}}%
\pgfpathlineto{\pgfqpoint{1.271463in}{1.761381in}}%
\pgfpathlineto{\pgfqpoint{1.263401in}{1.761381in}}%
\pgfpathlineto{\pgfqpoint{1.255339in}{1.761381in}}%
\pgfpathlineto{\pgfqpoint{1.247277in}{1.761381in}}%
\pgfpathlineto{\pgfqpoint{1.239215in}{1.761381in}}%
\pgfpathlineto{\pgfqpoint{1.231153in}{1.761381in}}%
\pgfpathlineto{\pgfqpoint{1.223091in}{1.761381in}}%
\pgfpathlineto{\pgfqpoint{1.215029in}{1.761381in}}%
\pgfpathlineto{\pgfqpoint{1.206967in}{1.761381in}}%
\pgfpathlineto{\pgfqpoint{1.198905in}{1.761381in}}%
\pgfpathlineto{\pgfqpoint{1.190843in}{1.761381in}}%
\pgfpathlineto{\pgfqpoint{1.182781in}{1.761381in}}%
\pgfpathlineto{\pgfqpoint{1.174719in}{1.761381in}}%
\pgfpathlineto{\pgfqpoint{1.166657in}{1.761381in}}%
\pgfpathlineto{\pgfqpoint{1.158595in}{1.761381in}}%
\pgfpathlineto{\pgfqpoint{1.150533in}{1.855807in}}%
\pgfpathlineto{\pgfqpoint{1.142471in}{1.855807in}}%
\pgfpathlineto{\pgfqpoint{1.134409in}{1.856982in}}%
\pgfpathlineto{\pgfqpoint{1.126347in}{1.856982in}}%
\pgfpathlineto{\pgfqpoint{1.118285in}{1.856982in}}%
\pgfpathlineto{\pgfqpoint{1.110223in}{1.856982in}}%
\pgfpathlineto{\pgfqpoint{1.102161in}{1.856982in}}%
\pgfpathlineto{\pgfqpoint{1.094099in}{1.856982in}}%
\pgfpathlineto{\pgfqpoint{1.086037in}{1.870398in}}%
\pgfpathlineto{\pgfqpoint{1.077975in}{1.870398in}}%
\pgfpathlineto{\pgfqpoint{1.069914in}{1.870398in}}%
\pgfpathlineto{\pgfqpoint{1.061852in}{1.870398in}}%
\pgfpathlineto{\pgfqpoint{1.053790in}{1.964864in}}%
\pgfpathlineto{\pgfqpoint{1.045728in}{1.964864in}}%
\pgfpathlineto{\pgfqpoint{1.037666in}{1.964864in}}%
\pgfpathlineto{\pgfqpoint{1.029604in}{1.964864in}}%
\pgfpathlineto{\pgfqpoint{1.021542in}{2.074039in}}%
\pgfpathlineto{\pgfqpoint{1.013480in}{2.074039in}}%
\pgfpathlineto{\pgfqpoint{1.005418in}{2.074039in}}%
\pgfpathlineto{\pgfqpoint{0.997356in}{2.074039in}}%
\pgfpathlineto{\pgfqpoint{0.989294in}{2.074039in}}%
\pgfpathlineto{\pgfqpoint{0.981232in}{2.074039in}}%
\pgfpathlineto{\pgfqpoint{0.973170in}{2.074039in}}%
\pgfpathlineto{\pgfqpoint{0.965108in}{2.074039in}}%
\pgfpathlineto{\pgfqpoint{0.957046in}{2.074039in}}%
\pgfpathlineto{\pgfqpoint{0.948984in}{2.074039in}}%
\pgfpathlineto{\pgfqpoint{0.940922in}{2.074039in}}%
\pgfpathlineto{\pgfqpoint{0.932860in}{2.074039in}}%
\pgfpathlineto{\pgfqpoint{0.924798in}{2.074039in}}%
\pgfpathlineto{\pgfqpoint{0.916736in}{2.074039in}}%
\pgfpathlineto{\pgfqpoint{0.908674in}{2.074039in}}%
\pgfpathlineto{\pgfqpoint{0.900612in}{2.074039in}}%
\pgfpathlineto{\pgfqpoint{0.892550in}{2.074039in}}%
\pgfpathlineto{\pgfqpoint{0.884488in}{2.074039in}}%
\pgfpathlineto{\pgfqpoint{0.876426in}{2.074039in}}%
\pgfpathlineto{\pgfqpoint{0.868364in}{2.074342in}}%
\pgfpathlineto{\pgfqpoint{0.860302in}{2.074342in}}%
\pgfpathlineto{\pgfqpoint{0.852241in}{2.074342in}}%
\pgfpathlineto{\pgfqpoint{0.844179in}{2.075175in}}%
\pgfpathlineto{\pgfqpoint{0.836117in}{2.075175in}}%
\pgfpathlineto{\pgfqpoint{0.828055in}{2.086513in}}%
\pgfpathlineto{\pgfqpoint{0.819993in}{2.086513in}}%
\pgfpathlineto{\pgfqpoint{0.811931in}{2.086513in}}%
\pgfpathlineto{\pgfqpoint{0.803869in}{2.086513in}}%
\pgfpathlineto{\pgfqpoint{0.795807in}{2.086513in}}%
\pgfpathlineto{\pgfqpoint{0.787745in}{2.086513in}}%
\pgfpathlineto{\pgfqpoint{0.779683in}{2.086513in}}%
\pgfpathlineto{\pgfqpoint{0.771621in}{2.086513in}}%
\pgfpathlineto{\pgfqpoint{0.763559in}{2.086513in}}%
\pgfpathlineto{\pgfqpoint{0.755497in}{2.086513in}}%
\pgfpathlineto{\pgfqpoint{0.747435in}{2.086513in}}%
\pgfpathlineto{\pgfqpoint{0.739373in}{2.086513in}}%
\pgfpathlineto{\pgfqpoint{0.731311in}{2.086513in}}%
\pgfpathlineto{\pgfqpoint{0.723249in}{2.089463in}}%
\pgfpathlineto{\pgfqpoint{0.715187in}{2.089463in}}%
\pgfpathlineto{\pgfqpoint{0.707125in}{2.120225in}}%
\pgfpathlineto{\pgfqpoint{0.699063in}{2.121771in}}%
\pgfpathlineto{\pgfqpoint{0.691001in}{2.121771in}}%
\pgfpathlineto{\pgfqpoint{0.682939in}{2.134342in}}%
\pgfpathlineto{\pgfqpoint{0.674877in}{2.135614in}}%
\pgfpathlineto{\pgfqpoint{0.666815in}{2.135614in}}%
\pgfpathlineto{\pgfqpoint{0.658753in}{2.135614in}}%
\pgfpathlineto{\pgfqpoint{0.650691in}{2.141505in}}%
\pgfpathlineto{\pgfqpoint{0.642629in}{2.141505in}}%
\pgfpathlineto{\pgfqpoint{0.634568in}{2.141505in}}%
\pgfpathlineto{\pgfqpoint{0.626506in}{2.141505in}}%
\pgfpathlineto{\pgfqpoint{0.618444in}{2.141505in}}%
\pgfpathlineto{\pgfqpoint{0.610382in}{2.141505in}}%
\pgfpathlineto{\pgfqpoint{0.602320in}{2.148165in}}%
\pgfpathlineto{\pgfqpoint{0.594258in}{2.170457in}}%
\pgfpathlineto{\pgfqpoint{0.586196in}{2.170457in}}%
\pgfpathlineto{\pgfqpoint{0.578134in}{2.170457in}}%
\pgfpathlineto{\pgfqpoint{0.570072in}{2.172644in}}%
\pgfpathlineto{\pgfqpoint{0.562010in}{2.172644in}}%
\pgfpathlineto{\pgfqpoint{0.553948in}{2.172644in}}%
\pgfpathlineto{\pgfqpoint{0.545886in}{2.172644in}}%
\pgfpathlineto{\pgfqpoint{0.537824in}{2.174739in}}%
\pgfpathlineto{\pgfqpoint{0.529762in}{2.174739in}}%
\pgfpathlineto{\pgfqpoint{0.521700in}{2.174739in}}%
\pgfpathlineto{\pgfqpoint{0.513638in}{2.174739in}}%
\pgfpathlineto{\pgfqpoint{0.505576in}{2.174739in}}%
\pgfpathlineto{\pgfqpoint{0.497514in}{2.174739in}}%
\pgfpathlineto{\pgfqpoint{0.489452in}{2.217069in}}%
\pgfpathlineto{\pgfqpoint{0.481390in}{2.217463in}}%
\pgfpathlineto{\pgfqpoint{0.473328in}{2.217463in}}%
\pgfpathlineto{\pgfqpoint{0.465266in}{2.269535in}}%
\pgfpathlineto{\pgfqpoint{0.457204in}{2.320661in}}%
\pgfpathlineto{\pgfqpoint{0.449142in}{2.333013in}}%
\pgfpathlineto{\pgfqpoint{0.441080in}{2.333013in}}%
\pgfpathlineto{\pgfqpoint{0.433018in}{2.333013in}}%
\pgfpathlineto{\pgfqpoint{0.424956in}{2.343637in}}%
\pgfpathlineto{\pgfqpoint{0.416895in}{2.361984in}}%
\pgfpathlineto{\pgfqpoint{0.408833in}{2.362017in}}%
\pgfpathlineto{\pgfqpoint{0.400771in}{2.406652in}}%
\pgfpathlineto{\pgfqpoint{0.392709in}{2.441693in}}%
\pgfpathlineto{\pgfqpoint{0.384647in}{2.455797in}}%
\pgfpathlineto{\pgfqpoint{0.376585in}{2.487209in}}%
\pgfpathlineto{\pgfqpoint{0.368523in}{2.491663in}}%
\pgfpathclose%
\pgfusepath{fill}%
\end{pgfscope}%
\begin{pgfscope}%
\pgfpathrectangle{\pgfqpoint{0.287500in}{0.375000in}}{\pgfqpoint{1.782500in}{2.265000in}}%
\pgfusepath{clip}%
\pgfsetroundcap%
\pgfsetroundjoin%
\pgfsetlinewidth{1.505625pt}%
\definecolor{currentstroke}{rgb}{0.121569,0.466667,0.705882}%
\pgfsetstrokecolor{currentstroke}%
\pgfsetdash{}{0pt}%
\pgfpathmoveto{\pgfqpoint{0.368523in}{2.463277in}}%
\pgfpathlineto{\pgfqpoint{0.384647in}{2.413122in}}%
\pgfpathlineto{\pgfqpoint{0.392709in}{2.399484in}}%
\pgfpathlineto{\pgfqpoint{0.408833in}{2.399484in}}%
\pgfpathlineto{\pgfqpoint{0.416895in}{2.370392in}}%
\pgfpathlineto{\pgfqpoint{0.424956in}{2.331540in}}%
\pgfpathlineto{\pgfqpoint{0.441080in}{2.329864in}}%
\pgfpathlineto{\pgfqpoint{0.449142in}{2.316445in}}%
\pgfpathlineto{\pgfqpoint{0.457204in}{2.314461in}}%
\pgfpathlineto{\pgfqpoint{0.465266in}{2.295504in}}%
\pgfpathlineto{\pgfqpoint{0.473328in}{2.291951in}}%
\pgfpathlineto{\pgfqpoint{0.489452in}{2.252577in}}%
\pgfpathlineto{\pgfqpoint{0.497514in}{2.252577in}}%
\pgfpathlineto{\pgfqpoint{0.505576in}{2.206881in}}%
\pgfpathlineto{\pgfqpoint{0.513638in}{2.206881in}}%
\pgfpathlineto{\pgfqpoint{0.521700in}{2.197770in}}%
\pgfpathlineto{\pgfqpoint{0.545886in}{2.197013in}}%
\pgfpathlineto{\pgfqpoint{0.562010in}{2.195875in}}%
\pgfpathlineto{\pgfqpoint{0.570072in}{2.180546in}}%
\pgfpathlineto{\pgfqpoint{0.578134in}{2.170092in}}%
\pgfpathlineto{\pgfqpoint{0.594258in}{2.170092in}}%
\pgfpathlineto{\pgfqpoint{0.602320in}{2.157436in}}%
\pgfpathlineto{\pgfqpoint{0.610382in}{2.152108in}}%
\pgfpathlineto{\pgfqpoint{0.642629in}{2.152108in}}%
\pgfpathlineto{\pgfqpoint{0.650691in}{2.135656in}}%
\pgfpathlineto{\pgfqpoint{0.674877in}{2.134775in}}%
\pgfpathlineto{\pgfqpoint{0.682939in}{2.085606in}}%
\pgfpathlineto{\pgfqpoint{0.707125in}{2.085606in}}%
\pgfpathlineto{\pgfqpoint{0.715187in}{2.038463in}}%
\pgfpathlineto{\pgfqpoint{0.731311in}{2.038463in}}%
\pgfpathlineto{\pgfqpoint{0.739373in}{2.027748in}}%
\pgfpathlineto{\pgfqpoint{0.755497in}{2.027748in}}%
\pgfpathlineto{\pgfqpoint{0.763559in}{2.020829in}}%
\pgfpathlineto{\pgfqpoint{0.771621in}{2.001143in}}%
\pgfpathlineto{\pgfqpoint{0.779683in}{2.000561in}}%
\pgfpathlineto{\pgfqpoint{0.787745in}{1.994107in}}%
\pgfpathlineto{\pgfqpoint{0.795807in}{1.909388in}}%
\pgfpathlineto{\pgfqpoint{0.828055in}{1.909388in}}%
\pgfpathlineto{\pgfqpoint{0.836117in}{1.879415in}}%
\pgfpathlineto{\pgfqpoint{0.852241in}{1.879415in}}%
\pgfpathlineto{\pgfqpoint{0.860302in}{1.869906in}}%
\pgfpathlineto{\pgfqpoint{0.876426in}{1.868907in}}%
\pgfpathlineto{\pgfqpoint{0.884488in}{1.859373in}}%
\pgfpathlineto{\pgfqpoint{1.352082in}{1.857975in}}%
\pgfpathlineto{\pgfqpoint{1.360144in}{1.836045in}}%
\pgfpathlineto{\pgfqpoint{1.988977in}{1.836045in}}%
\pgfpathlineto{\pgfqpoint{1.988977in}{1.836045in}}%
\pgfusepath{stroke}%
\end{pgfscope}%
\begin{pgfscope}%
\pgfpathrectangle{\pgfqpoint{0.287500in}{0.375000in}}{\pgfqpoint{1.782500in}{2.265000in}}%
\pgfusepath{clip}%
\pgfsetroundcap%
\pgfsetroundjoin%
\pgfsetlinewidth{1.505625pt}%
\definecolor{currentstroke}{rgb}{1.000000,0.498039,0.054902}%
\pgfsetstrokecolor{currentstroke}%
\pgfsetdash{}{0pt}%
\pgfpathmoveto{\pgfqpoint{0.368523in}{2.464586in}}%
\pgfpathlineto{\pgfqpoint{0.376585in}{2.397671in}}%
\pgfpathlineto{\pgfqpoint{0.384647in}{2.396306in}}%
\pgfpathlineto{\pgfqpoint{0.392709in}{2.373102in}}%
\pgfpathlineto{\pgfqpoint{0.449142in}{2.373102in}}%
\pgfpathlineto{\pgfqpoint{0.457204in}{2.278623in}}%
\pgfpathlineto{\pgfqpoint{0.489452in}{2.278623in}}%
\pgfpathlineto{\pgfqpoint{0.497514in}{2.265798in}}%
\pgfpathlineto{\pgfqpoint{0.545886in}{2.265798in}}%
\pgfpathlineto{\pgfqpoint{0.553948in}{2.255089in}}%
\pgfpathlineto{\pgfqpoint{0.562010in}{2.255089in}}%
\pgfpathlineto{\pgfqpoint{0.570072in}{2.204999in}}%
\pgfpathlineto{\pgfqpoint{0.578134in}{2.197821in}}%
\pgfpathlineto{\pgfqpoint{0.586196in}{2.194441in}}%
\pgfpathlineto{\pgfqpoint{0.594258in}{2.166988in}}%
\pgfpathlineto{\pgfqpoint{0.602320in}{2.166988in}}%
\pgfpathlineto{\pgfqpoint{0.610382in}{2.145768in}}%
\pgfpathlineto{\pgfqpoint{0.626506in}{2.145768in}}%
\pgfpathlineto{\pgfqpoint{0.634568in}{2.102727in}}%
\pgfpathlineto{\pgfqpoint{0.642629in}{2.095191in}}%
\pgfpathlineto{\pgfqpoint{0.650691in}{2.095191in}}%
\pgfpathlineto{\pgfqpoint{0.658753in}{2.018212in}}%
\pgfpathlineto{\pgfqpoint{0.666815in}{2.018212in}}%
\pgfpathlineto{\pgfqpoint{0.682939in}{2.010039in}}%
\pgfpathlineto{\pgfqpoint{0.691001in}{1.983376in}}%
\pgfpathlineto{\pgfqpoint{0.699063in}{1.877706in}}%
\pgfpathlineto{\pgfqpoint{0.707125in}{1.877706in}}%
\pgfpathlineto{\pgfqpoint{0.715187in}{1.872408in}}%
\pgfpathlineto{\pgfqpoint{0.723249in}{1.872408in}}%
\pgfpathlineto{\pgfqpoint{0.731311in}{1.794342in}}%
\pgfpathlineto{\pgfqpoint{0.739373in}{1.690714in}}%
\pgfpathlineto{\pgfqpoint{0.755497in}{1.690714in}}%
\pgfpathlineto{\pgfqpoint{0.763559in}{1.664468in}}%
\pgfpathlineto{\pgfqpoint{0.771621in}{1.628598in}}%
\pgfpathlineto{\pgfqpoint{0.795807in}{1.628598in}}%
\pgfpathlineto{\pgfqpoint{0.803869in}{1.420195in}}%
\pgfpathlineto{\pgfqpoint{0.828055in}{1.420195in}}%
\pgfpathlineto{\pgfqpoint{0.836117in}{1.372009in}}%
\pgfpathlineto{\pgfqpoint{0.860302in}{1.372009in}}%
\pgfpathlineto{\pgfqpoint{0.868364in}{1.247550in}}%
\pgfpathlineto{\pgfqpoint{0.876426in}{1.234505in}}%
\pgfpathlineto{\pgfqpoint{0.884488in}{1.143608in}}%
\pgfpathlineto{\pgfqpoint{0.916736in}{1.143608in}}%
\pgfpathlineto{\pgfqpoint{0.924798in}{1.130114in}}%
\pgfpathlineto{\pgfqpoint{0.940922in}{1.130114in}}%
\pgfpathlineto{\pgfqpoint{0.948984in}{1.060604in}}%
\pgfpathlineto{\pgfqpoint{0.957046in}{1.060604in}}%
\pgfpathlineto{\pgfqpoint{0.965108in}{1.051356in}}%
\pgfpathlineto{\pgfqpoint{1.069914in}{1.051356in}}%
\pgfpathlineto{\pgfqpoint{1.077975in}{1.038334in}}%
\pgfpathlineto{\pgfqpoint{1.134409in}{1.037017in}}%
\pgfpathlineto{\pgfqpoint{1.142471in}{1.018092in}}%
\pgfpathlineto{\pgfqpoint{1.247277in}{1.018092in}}%
\pgfpathlineto{\pgfqpoint{1.255339in}{0.914778in}}%
\pgfpathlineto{\pgfqpoint{1.335958in}{0.914745in}}%
\pgfpathlineto{\pgfqpoint{1.344020in}{0.898127in}}%
\pgfpathlineto{\pgfqpoint{1.376268in}{0.898127in}}%
\pgfpathlineto{\pgfqpoint{1.384330in}{0.829476in}}%
\pgfpathlineto{\pgfqpoint{1.424640in}{0.829476in}}%
\pgfpathlineto{\pgfqpoint{1.432702in}{0.669929in}}%
\pgfpathlineto{\pgfqpoint{1.618127in}{0.669929in}}%
\pgfpathlineto{\pgfqpoint{1.626189in}{0.657420in}}%
\pgfpathlineto{\pgfqpoint{1.988977in}{0.656153in}}%
\pgfpathlineto{\pgfqpoint{1.988977in}{0.656153in}}%
\pgfusepath{stroke}%
\end{pgfscope}%
\begin{pgfscope}%
\pgfpathrectangle{\pgfqpoint{0.287500in}{0.375000in}}{\pgfqpoint{1.782500in}{2.265000in}}%
\pgfusepath{clip}%
\pgfsetroundcap%
\pgfsetroundjoin%
\pgfsetlinewidth{1.505625pt}%
\definecolor{currentstroke}{rgb}{0.172549,0.627451,0.172549}%
\pgfsetstrokecolor{currentstroke}%
\pgfsetdash{}{0pt}%
\pgfpathmoveto{\pgfqpoint{0.368523in}{2.511717in}}%
\pgfpathlineto{\pgfqpoint{0.376585in}{2.430460in}}%
\pgfpathlineto{\pgfqpoint{0.384647in}{2.420046in}}%
\pgfpathlineto{\pgfqpoint{0.400771in}{2.420046in}}%
\pgfpathlineto{\pgfqpoint{0.408833in}{2.416783in}}%
\pgfpathlineto{\pgfqpoint{0.416895in}{2.407554in}}%
\pgfpathlineto{\pgfqpoint{0.424956in}{2.407554in}}%
\pgfpathlineto{\pgfqpoint{0.433018in}{2.388472in}}%
\pgfpathlineto{\pgfqpoint{0.441080in}{2.353421in}}%
\pgfpathlineto{\pgfqpoint{0.449142in}{2.353421in}}%
\pgfpathlineto{\pgfqpoint{0.457204in}{2.350338in}}%
\pgfpathlineto{\pgfqpoint{0.505576in}{2.350338in}}%
\pgfpathlineto{\pgfqpoint{0.513638in}{2.288295in}}%
\pgfpathlineto{\pgfqpoint{0.537824in}{2.288295in}}%
\pgfpathlineto{\pgfqpoint{0.545886in}{2.279474in}}%
\pgfpathlineto{\pgfqpoint{0.553948in}{2.229328in}}%
\pgfpathlineto{\pgfqpoint{0.562010in}{2.228119in}}%
\pgfpathlineto{\pgfqpoint{0.570072in}{2.193729in}}%
\pgfpathlineto{\pgfqpoint{0.682939in}{2.193729in}}%
\pgfpathlineto{\pgfqpoint{0.691001in}{2.154847in}}%
\pgfpathlineto{\pgfqpoint{0.739373in}{2.154847in}}%
\pgfpathlineto{\pgfqpoint{0.747435in}{2.151810in}}%
\pgfpathlineto{\pgfqpoint{0.755497in}{2.151810in}}%
\pgfpathlineto{\pgfqpoint{0.763559in}{2.039025in}}%
\pgfpathlineto{\pgfqpoint{0.771621in}{2.036308in}}%
\pgfpathlineto{\pgfqpoint{0.779683in}{1.975703in}}%
\pgfpathlineto{\pgfqpoint{0.787745in}{1.975703in}}%
\pgfpathlineto{\pgfqpoint{0.803869in}{1.966902in}}%
\pgfpathlineto{\pgfqpoint{0.876426in}{1.966637in}}%
\pgfpathlineto{\pgfqpoint{0.884488in}{1.485111in}}%
\pgfpathlineto{\pgfqpoint{0.892550in}{1.478548in}}%
\pgfpathlineto{\pgfqpoint{0.908674in}{1.478548in}}%
\pgfpathlineto{\pgfqpoint{0.916736in}{1.461570in}}%
\pgfpathlineto{\pgfqpoint{0.924798in}{1.461570in}}%
\pgfpathlineto{\pgfqpoint{0.932860in}{1.412110in}}%
\pgfpathlineto{\pgfqpoint{0.957046in}{1.412110in}}%
\pgfpathlineto{\pgfqpoint{0.965108in}{1.381228in}}%
\pgfpathlineto{\pgfqpoint{1.045728in}{1.381217in}}%
\pgfpathlineto{\pgfqpoint{1.061852in}{1.336262in}}%
\pgfpathlineto{\pgfqpoint{1.198905in}{1.336262in}}%
\pgfpathlineto{\pgfqpoint{1.206967in}{1.301643in}}%
\pgfpathlineto{\pgfqpoint{1.239215in}{1.301643in}}%
\pgfpathlineto{\pgfqpoint{1.247277in}{1.138497in}}%
\pgfpathlineto{\pgfqpoint{1.295648in}{1.138497in}}%
\pgfpathlineto{\pgfqpoint{1.303710in}{0.960488in}}%
\pgfpathlineto{\pgfqpoint{1.384330in}{0.960488in}}%
\pgfpathlineto{\pgfqpoint{1.392392in}{0.949034in}}%
\pgfpathlineto{\pgfqpoint{1.416578in}{0.949034in}}%
\pgfpathlineto{\pgfqpoint{1.424640in}{0.898559in}}%
\pgfpathlineto{\pgfqpoint{1.674561in}{0.897332in}}%
\pgfpathlineto{\pgfqpoint{1.682623in}{0.842196in}}%
\pgfpathlineto{\pgfqpoint{1.851924in}{0.842196in}}%
\pgfpathlineto{\pgfqpoint{1.859986in}{0.798782in}}%
\pgfpathlineto{\pgfqpoint{1.988977in}{0.798782in}}%
\pgfpathlineto{\pgfqpoint{1.988977in}{0.798782in}}%
\pgfusepath{stroke}%
\end{pgfscope}%
\begin{pgfscope}%
\pgfpathrectangle{\pgfqpoint{0.287500in}{0.375000in}}{\pgfqpoint{1.782500in}{2.265000in}}%
\pgfusepath{clip}%
\pgfsetroundcap%
\pgfsetroundjoin%
\pgfsetlinewidth{1.505625pt}%
\definecolor{currentstroke}{rgb}{0.839216,0.152941,0.156863}%
\pgfsetstrokecolor{currentstroke}%
\pgfsetdash{}{0pt}%
\pgfpathmoveto{\pgfqpoint{0.368523in}{2.508859in}}%
\pgfpathlineto{\pgfqpoint{0.376585in}{2.501816in}}%
\pgfpathlineto{\pgfqpoint{0.384647in}{2.462426in}}%
\pgfpathlineto{\pgfqpoint{0.392709in}{2.452811in}}%
\pgfpathlineto{\pgfqpoint{0.400771in}{2.429861in}}%
\pgfpathlineto{\pgfqpoint{0.408833in}{2.389025in}}%
\pgfpathlineto{\pgfqpoint{0.416895in}{2.389005in}}%
\pgfpathlineto{\pgfqpoint{0.424956in}{2.379357in}}%
\pgfpathlineto{\pgfqpoint{0.433018in}{2.372645in}}%
\pgfpathlineto{\pgfqpoint{0.449142in}{2.372645in}}%
\pgfpathlineto{\pgfqpoint{0.457204in}{2.366146in}}%
\pgfpathlineto{\pgfqpoint{0.473328in}{2.315575in}}%
\pgfpathlineto{\pgfqpoint{0.489452in}{2.315428in}}%
\pgfpathlineto{\pgfqpoint{0.497514in}{2.301807in}}%
\pgfpathlineto{\pgfqpoint{0.594258in}{2.300598in}}%
\pgfpathlineto{\pgfqpoint{0.602320in}{2.228391in}}%
\pgfpathlineto{\pgfqpoint{0.610382in}{2.211298in}}%
\pgfpathlineto{\pgfqpoint{0.650691in}{2.211298in}}%
\pgfpathlineto{\pgfqpoint{0.658753in}{2.208286in}}%
\pgfpathlineto{\pgfqpoint{0.674877in}{2.208286in}}%
\pgfpathlineto{\pgfqpoint{0.682939in}{2.204981in}}%
\pgfpathlineto{\pgfqpoint{0.691001in}{2.198113in}}%
\pgfpathlineto{\pgfqpoint{0.699063in}{2.198113in}}%
\pgfpathlineto{\pgfqpoint{0.707125in}{2.193933in}}%
\pgfpathlineto{\pgfqpoint{0.715187in}{2.180103in}}%
\pgfpathlineto{\pgfqpoint{0.828055in}{2.179098in}}%
\pgfpathlineto{\pgfqpoint{0.836117in}{2.153588in}}%
\pgfpathlineto{\pgfqpoint{1.021542in}{2.152753in}}%
\pgfpathlineto{\pgfqpoint{1.029604in}{2.106996in}}%
\pgfpathlineto{\pgfqpoint{1.053790in}{2.106996in}}%
\pgfpathlineto{\pgfqpoint{1.061852in}{2.088013in}}%
\pgfpathlineto{\pgfqpoint{1.086037in}{2.088013in}}%
\pgfpathlineto{\pgfqpoint{1.094099in}{2.086311in}}%
\pgfpathlineto{\pgfqpoint{1.150533in}{2.086170in}}%
\pgfpathlineto{\pgfqpoint{1.158595in}{1.903805in}}%
\pgfpathlineto{\pgfqpoint{1.529445in}{1.903678in}}%
\pgfpathlineto{\pgfqpoint{1.537507in}{1.801911in}}%
\pgfpathlineto{\pgfqpoint{1.569755in}{1.801911in}}%
\pgfpathlineto{\pgfqpoint{1.577817in}{1.752813in}}%
\pgfpathlineto{\pgfqpoint{1.771304in}{1.752793in}}%
\pgfpathlineto{\pgfqpoint{1.779366in}{1.668844in}}%
\pgfpathlineto{\pgfqpoint{1.795490in}{1.668822in}}%
\pgfpathlineto{\pgfqpoint{1.803552in}{1.545467in}}%
\pgfpathlineto{\pgfqpoint{1.868048in}{1.545434in}}%
\pgfpathlineto{\pgfqpoint{1.876110in}{1.514225in}}%
\pgfpathlineto{\pgfqpoint{1.892234in}{1.514224in}}%
\pgfpathlineto{\pgfqpoint{1.900296in}{1.502279in}}%
\pgfpathlineto{\pgfqpoint{1.932544in}{1.502279in}}%
\pgfpathlineto{\pgfqpoint{1.940605in}{1.413794in}}%
\pgfpathlineto{\pgfqpoint{1.956729in}{1.413794in}}%
\pgfpathlineto{\pgfqpoint{1.964791in}{1.392730in}}%
\pgfpathlineto{\pgfqpoint{1.972853in}{1.084366in}}%
\pgfpathlineto{\pgfqpoint{1.988977in}{1.084366in}}%
\pgfpathlineto{\pgfqpoint{1.988977in}{1.084366in}}%
\pgfusepath{stroke}%
\end{pgfscope}%
\begin{pgfscope}%
\pgfsetrectcap%
\pgfsetmiterjoin%
\pgfsetlinewidth{0.000000pt}%
\definecolor{currentstroke}{rgb}{1.000000,1.000000,1.000000}%
\pgfsetstrokecolor{currentstroke}%
\pgfsetdash{}{0pt}%
\pgfpathmoveto{\pgfqpoint{0.287500in}{0.375000in}}%
\pgfpathlineto{\pgfqpoint{0.287500in}{2.640000in}}%
\pgfusepath{}%
\end{pgfscope}%
\begin{pgfscope}%
\pgfsetrectcap%
\pgfsetmiterjoin%
\pgfsetlinewidth{0.000000pt}%
\definecolor{currentstroke}{rgb}{1.000000,1.000000,1.000000}%
\pgfsetstrokecolor{currentstroke}%
\pgfsetdash{}{0pt}%
\pgfpathmoveto{\pgfqpoint{2.070000in}{0.375000in}}%
\pgfpathlineto{\pgfqpoint{2.070000in}{2.640000in}}%
\pgfusepath{}%
\end{pgfscope}%
\begin{pgfscope}%
\pgfsetrectcap%
\pgfsetmiterjoin%
\pgfsetlinewidth{0.000000pt}%
\definecolor{currentstroke}{rgb}{1.000000,1.000000,1.000000}%
\pgfsetstrokecolor{currentstroke}%
\pgfsetdash{}{0pt}%
\pgfpathmoveto{\pgfqpoint{0.287500in}{0.375000in}}%
\pgfpathlineto{\pgfqpoint{2.070000in}{0.375000in}}%
\pgfusepath{}%
\end{pgfscope}%
\begin{pgfscope}%
\pgfsetrectcap%
\pgfsetmiterjoin%
\pgfsetlinewidth{0.000000pt}%
\definecolor{currentstroke}{rgb}{1.000000,1.000000,1.000000}%
\pgfsetstrokecolor{currentstroke}%
\pgfsetdash{}{0pt}%
\pgfpathmoveto{\pgfqpoint{0.287500in}{2.640000in}}%
\pgfpathlineto{\pgfqpoint{2.070000in}{2.640000in}}%
\pgfusepath{}%
\end{pgfscope}%
\begin{pgfscope}%
\definecolor{textcolor}{rgb}{0.150000,0.150000,0.150000}%
\pgfsetstrokecolor{textcolor}%
\pgfsetfillcolor{textcolor}%
\pgftext[x=1.178750in,y=2.723333in,,base]{\color{textcolor}\rmfamily\fontsize{8.000000}{9.600000}\selectfont Embedded SinOne in 1D}%
\end{pgfscope}%
\begin{pgfscope}%
\pgfsetroundcap%
\pgfsetroundjoin%
\pgfsetlinewidth{1.505625pt}%
\definecolor{currentstroke}{rgb}{0.121569,0.466667,0.705882}%
\pgfsetstrokecolor{currentstroke}%
\pgfsetdash{}{0pt}%
\pgfpathmoveto{\pgfqpoint{0.772205in}{2.530853in}}%
\pgfpathlineto{\pgfqpoint{0.938872in}{2.530853in}}%
\pgfusepath{stroke}%
\end{pgfscope}%
\begin{pgfscope}%
\definecolor{textcolor}{rgb}{0.150000,0.150000,0.150000}%
\pgfsetstrokecolor{textcolor}%
\pgfsetfillcolor{textcolor}%
\pgftext[x=1.005539in,y=2.501686in,left,base]{\color{textcolor}\rmfamily\fontsize{6.000000}{7.200000}\selectfont random}%
\end{pgfscope}%
\begin{pgfscope}%
\pgfsetroundcap%
\pgfsetroundjoin%
\pgfsetlinewidth{1.505625pt}%
\definecolor{currentstroke}{rgb}{1.000000,0.498039,0.054902}%
\pgfsetstrokecolor{currentstroke}%
\pgfsetdash{}{0pt}%
\pgfpathmoveto{\pgfqpoint{0.772205in}{2.408538in}}%
\pgfpathlineto{\pgfqpoint{0.938872in}{2.408538in}}%
\pgfusepath{stroke}%
\end{pgfscope}%
\begin{pgfscope}%
\definecolor{textcolor}{rgb}{0.150000,0.150000,0.150000}%
\pgfsetstrokecolor{textcolor}%
\pgfsetfillcolor{textcolor}%
\pgftext[x=1.005539in,y=2.379372in,left,base]{\color{textcolor}\rmfamily\fontsize{6.000000}{7.200000}\selectfont 5 x DNGO retrain-reset}%
\end{pgfscope}%
\begin{pgfscope}%
\pgfsetroundcap%
\pgfsetroundjoin%
\pgfsetlinewidth{1.505625pt}%
\definecolor{currentstroke}{rgb}{0.172549,0.627451,0.172549}%
\pgfsetstrokecolor{currentstroke}%
\pgfsetdash{}{0pt}%
\pgfpathmoveto{\pgfqpoint{0.772205in}{2.286224in}}%
\pgfpathlineto{\pgfqpoint{0.938872in}{2.286224in}}%
\pgfusepath{stroke}%
\end{pgfscope}%
\begin{pgfscope}%
\definecolor{textcolor}{rgb}{0.150000,0.150000,0.150000}%
\pgfsetstrokecolor{textcolor}%
\pgfsetfillcolor{textcolor}%
\pgftext[x=1.005539in,y=2.257057in,left,base]{\color{textcolor}\rmfamily\fontsize{6.000000}{7.200000}\selectfont DNGO retrain-reset}%
\end{pgfscope}%
\begin{pgfscope}%
\pgfsetroundcap%
\pgfsetroundjoin%
\pgfsetlinewidth{1.505625pt}%
\definecolor{currentstroke}{rgb}{0.839216,0.152941,0.156863}%
\pgfsetstrokecolor{currentstroke}%
\pgfsetdash{}{0pt}%
\pgfpathmoveto{\pgfqpoint{0.772205in}{2.163910in}}%
\pgfpathlineto{\pgfqpoint{0.938872in}{2.163910in}}%
\pgfusepath{stroke}%
\end{pgfscope}%
\begin{pgfscope}%
\definecolor{textcolor}{rgb}{0.150000,0.150000,0.150000}%
\pgfsetstrokecolor{textcolor}%
\pgfsetfillcolor{textcolor}%
\pgftext[x=1.005539in,y=2.134743in,left,base]{\color{textcolor}\rmfamily\fontsize{6.000000}{7.200000}\selectfont GP}%
\end{pgfscope}%
\end{pgfpicture}%
\makeatother%
\endgroup%

            \end{subfigure}
            \begin{subfigure}[t]{0.3\textwidth}
                \centering
                % \resizebox{.95\linewidth}{!}{}
                %% Creator: Matplotlib, PGF backend
%%
%% To include the figure in your LaTeX document, write
%%   \input{<filename>.pgf}
%%
%% Make sure the required packages are loaded in your preamble
%%   \usepackage{pgf}
%%
%% Figures using additional raster images can only be included by \input if
%% they are in the same directory as the main LaTeX file. For loading figures
%% from other directories you can use the `import` package
%%   \usepackage{import}
%% and then include the figures with
%%   \import{<path to file>}{<filename>.pgf}
%%
%% Matplotlib used the following preamble
%%   \usepackage{gensymb}
%%   \usepackage{fontspec}
%%   \setmainfont{DejaVu Serif}
%%   \setsansfont{Arial}
%%   \setmonofont{DejaVu Sans Mono}
%%
\begingroup%
\makeatletter%
\begin{pgfpicture}%
\pgfpathrectangle{\pgfpointorigin}{\pgfqpoint{2.300000in}{3.000000in}}%
\pgfusepath{use as bounding box, clip}%
\begin{pgfscope}%
\pgfsetbuttcap%
\pgfsetmiterjoin%
\definecolor{currentfill}{rgb}{1.000000,1.000000,1.000000}%
\pgfsetfillcolor{currentfill}%
\pgfsetlinewidth{0.000000pt}%
\definecolor{currentstroke}{rgb}{1.000000,1.000000,1.000000}%
\pgfsetstrokecolor{currentstroke}%
\pgfsetdash{}{0pt}%
\pgfpathmoveto{\pgfqpoint{0.000000in}{0.000000in}}%
\pgfpathlineto{\pgfqpoint{2.300000in}{0.000000in}}%
\pgfpathlineto{\pgfqpoint{2.300000in}{3.000000in}}%
\pgfpathlineto{\pgfqpoint{0.000000in}{3.000000in}}%
\pgfpathclose%
\pgfusepath{fill}%
\end{pgfscope}%
\begin{pgfscope}%
\pgfsetbuttcap%
\pgfsetmiterjoin%
\definecolor{currentfill}{rgb}{0.917647,0.917647,0.949020}%
\pgfsetfillcolor{currentfill}%
\pgfsetlinewidth{0.000000pt}%
\definecolor{currentstroke}{rgb}{0.000000,0.000000,0.000000}%
\pgfsetstrokecolor{currentstroke}%
\pgfsetstrokeopacity{0.000000}%
\pgfsetdash{}{0pt}%
\pgfpathmoveto{\pgfqpoint{0.287500in}{0.375000in}}%
\pgfpathlineto{\pgfqpoint{2.070000in}{0.375000in}}%
\pgfpathlineto{\pgfqpoint{2.070000in}{2.640000in}}%
\pgfpathlineto{\pgfqpoint{0.287500in}{2.640000in}}%
\pgfpathclose%
\pgfusepath{fill}%
\end{pgfscope}%
\begin{pgfscope}%
\pgfpathrectangle{\pgfqpoint{0.287500in}{0.375000in}}{\pgfqpoint{1.782500in}{2.265000in}}%
\pgfusepath{clip}%
\pgfsetroundcap%
\pgfsetroundjoin%
\pgfsetlinewidth{0.803000pt}%
\definecolor{currentstroke}{rgb}{1.000000,1.000000,1.000000}%
\pgfsetstrokecolor{currentstroke}%
\pgfsetdash{}{0pt}%
\pgfpathmoveto{\pgfqpoint{0.368523in}{0.375000in}}%
\pgfpathlineto{\pgfqpoint{0.368523in}{2.640000in}}%
\pgfusepath{stroke}%
\end{pgfscope}%
\begin{pgfscope}%
\definecolor{textcolor}{rgb}{0.150000,0.150000,0.150000}%
\pgfsetstrokecolor{textcolor}%
\pgfsetfillcolor{textcolor}%
\pgftext[x=0.368523in,y=0.326389in,,top]{\color{textcolor}\rmfamily\fontsize{8.000000}{9.600000}\selectfont \(\displaystyle 0\)}%
\end{pgfscope}%
\begin{pgfscope}%
\pgfpathrectangle{\pgfqpoint{0.287500in}{0.375000in}}{\pgfqpoint{1.782500in}{2.265000in}}%
\pgfusepath{clip}%
\pgfsetroundcap%
\pgfsetroundjoin%
\pgfsetlinewidth{0.803000pt}%
\definecolor{currentstroke}{rgb}{1.000000,1.000000,1.000000}%
\pgfsetstrokecolor{currentstroke}%
\pgfsetdash{}{0pt}%
\pgfpathmoveto{\pgfqpoint{0.771621in}{0.375000in}}%
\pgfpathlineto{\pgfqpoint{0.771621in}{2.640000in}}%
\pgfusepath{stroke}%
\end{pgfscope}%
\begin{pgfscope}%
\definecolor{textcolor}{rgb}{0.150000,0.150000,0.150000}%
\pgfsetstrokecolor{textcolor}%
\pgfsetfillcolor{textcolor}%
\pgftext[x=0.771621in,y=0.326389in,,top]{\color{textcolor}\rmfamily\fontsize{8.000000}{9.600000}\selectfont \(\displaystyle 50\)}%
\end{pgfscope}%
\begin{pgfscope}%
\pgfpathrectangle{\pgfqpoint{0.287500in}{0.375000in}}{\pgfqpoint{1.782500in}{2.265000in}}%
\pgfusepath{clip}%
\pgfsetroundcap%
\pgfsetroundjoin%
\pgfsetlinewidth{0.803000pt}%
\definecolor{currentstroke}{rgb}{1.000000,1.000000,1.000000}%
\pgfsetstrokecolor{currentstroke}%
\pgfsetdash{}{0pt}%
\pgfpathmoveto{\pgfqpoint{1.174719in}{0.375000in}}%
\pgfpathlineto{\pgfqpoint{1.174719in}{2.640000in}}%
\pgfusepath{stroke}%
\end{pgfscope}%
\begin{pgfscope}%
\definecolor{textcolor}{rgb}{0.150000,0.150000,0.150000}%
\pgfsetstrokecolor{textcolor}%
\pgfsetfillcolor{textcolor}%
\pgftext[x=1.174719in,y=0.326389in,,top]{\color{textcolor}\rmfamily\fontsize{8.000000}{9.600000}\selectfont \(\displaystyle 100\)}%
\end{pgfscope}%
\begin{pgfscope}%
\pgfpathrectangle{\pgfqpoint{0.287500in}{0.375000in}}{\pgfqpoint{1.782500in}{2.265000in}}%
\pgfusepath{clip}%
\pgfsetroundcap%
\pgfsetroundjoin%
\pgfsetlinewidth{0.803000pt}%
\definecolor{currentstroke}{rgb}{1.000000,1.000000,1.000000}%
\pgfsetstrokecolor{currentstroke}%
\pgfsetdash{}{0pt}%
\pgfpathmoveto{\pgfqpoint{1.577817in}{0.375000in}}%
\pgfpathlineto{\pgfqpoint{1.577817in}{2.640000in}}%
\pgfusepath{stroke}%
\end{pgfscope}%
\begin{pgfscope}%
\definecolor{textcolor}{rgb}{0.150000,0.150000,0.150000}%
\pgfsetstrokecolor{textcolor}%
\pgfsetfillcolor{textcolor}%
\pgftext[x=1.577817in,y=0.326389in,,top]{\color{textcolor}\rmfamily\fontsize{8.000000}{9.600000}\selectfont \(\displaystyle 150\)}%
\end{pgfscope}%
\begin{pgfscope}%
\pgfpathrectangle{\pgfqpoint{0.287500in}{0.375000in}}{\pgfqpoint{1.782500in}{2.265000in}}%
\pgfusepath{clip}%
\pgfsetroundcap%
\pgfsetroundjoin%
\pgfsetlinewidth{0.803000pt}%
\definecolor{currentstroke}{rgb}{1.000000,1.000000,1.000000}%
\pgfsetstrokecolor{currentstroke}%
\pgfsetdash{}{0pt}%
\pgfpathmoveto{\pgfqpoint{1.980915in}{0.375000in}}%
\pgfpathlineto{\pgfqpoint{1.980915in}{2.640000in}}%
\pgfusepath{stroke}%
\end{pgfscope}%
\begin{pgfscope}%
\definecolor{textcolor}{rgb}{0.150000,0.150000,0.150000}%
\pgfsetstrokecolor{textcolor}%
\pgfsetfillcolor{textcolor}%
\pgftext[x=1.980915in,y=0.326389in,,top]{\color{textcolor}\rmfamily\fontsize{8.000000}{9.600000}\selectfont \(\displaystyle 200\)}%
\end{pgfscope}%
\begin{pgfscope}%
\definecolor{textcolor}{rgb}{0.150000,0.150000,0.150000}%
\pgfsetstrokecolor{textcolor}%
\pgfsetfillcolor{textcolor}%
\pgftext[x=1.178750in,y=0.163303in,,top]{\color{textcolor}\rmfamily\fontsize{8.000000}{9.600000}\selectfont Step}%
\end{pgfscope}%
\begin{pgfscope}%
\pgfpathrectangle{\pgfqpoint{0.287500in}{0.375000in}}{\pgfqpoint{1.782500in}{2.265000in}}%
\pgfusepath{clip}%
\pgfsetroundcap%
\pgfsetroundjoin%
\pgfsetlinewidth{0.803000pt}%
\definecolor{currentstroke}{rgb}{1.000000,1.000000,1.000000}%
\pgfsetstrokecolor{currentstroke}%
\pgfsetdash{}{0pt}%
\pgfpathmoveto{\pgfqpoint{0.287500in}{0.635407in}}%
\pgfpathlineto{\pgfqpoint{2.070000in}{0.635407in}}%
\pgfusepath{stroke}%
\end{pgfscope}%
\begin{pgfscope}%
\definecolor{textcolor}{rgb}{0.150000,0.150000,0.150000}%
\pgfsetstrokecolor{textcolor}%
\pgfsetfillcolor{textcolor}%
\pgftext[x=-0.017284in,y=0.593198in,left,base]{\color{textcolor}\rmfamily\fontsize{8.000000}{9.600000}\selectfont \(\displaystyle 10^{-6}\)}%
\end{pgfscope}%
\begin{pgfscope}%
\pgfpathrectangle{\pgfqpoint{0.287500in}{0.375000in}}{\pgfqpoint{1.782500in}{2.265000in}}%
\pgfusepath{clip}%
\pgfsetroundcap%
\pgfsetroundjoin%
\pgfsetlinewidth{0.803000pt}%
\definecolor{currentstroke}{rgb}{1.000000,1.000000,1.000000}%
\pgfsetstrokecolor{currentstroke}%
\pgfsetdash{}{0pt}%
\pgfpathmoveto{\pgfqpoint{0.287500in}{0.961125in}}%
\pgfpathlineto{\pgfqpoint{2.070000in}{0.961125in}}%
\pgfusepath{stroke}%
\end{pgfscope}%
\begin{pgfscope}%
\definecolor{textcolor}{rgb}{0.150000,0.150000,0.150000}%
\pgfsetstrokecolor{textcolor}%
\pgfsetfillcolor{textcolor}%
\pgftext[x=-0.017284in,y=0.918916in,left,base]{\color{textcolor}\rmfamily\fontsize{8.000000}{9.600000}\selectfont \(\displaystyle 10^{-5}\)}%
\end{pgfscope}%
\begin{pgfscope}%
\pgfpathrectangle{\pgfqpoint{0.287500in}{0.375000in}}{\pgfqpoint{1.782500in}{2.265000in}}%
\pgfusepath{clip}%
\pgfsetroundcap%
\pgfsetroundjoin%
\pgfsetlinewidth{0.803000pt}%
\definecolor{currentstroke}{rgb}{1.000000,1.000000,1.000000}%
\pgfsetstrokecolor{currentstroke}%
\pgfsetdash{}{0pt}%
\pgfpathmoveto{\pgfqpoint{0.287500in}{1.286843in}}%
\pgfpathlineto{\pgfqpoint{2.070000in}{1.286843in}}%
\pgfusepath{stroke}%
\end{pgfscope}%
\begin{pgfscope}%
\definecolor{textcolor}{rgb}{0.150000,0.150000,0.150000}%
\pgfsetstrokecolor{textcolor}%
\pgfsetfillcolor{textcolor}%
\pgftext[x=-0.017284in,y=1.244633in,left,base]{\color{textcolor}\rmfamily\fontsize{8.000000}{9.600000}\selectfont \(\displaystyle 10^{-4}\)}%
\end{pgfscope}%
\begin{pgfscope}%
\pgfpathrectangle{\pgfqpoint{0.287500in}{0.375000in}}{\pgfqpoint{1.782500in}{2.265000in}}%
\pgfusepath{clip}%
\pgfsetroundcap%
\pgfsetroundjoin%
\pgfsetlinewidth{0.803000pt}%
\definecolor{currentstroke}{rgb}{1.000000,1.000000,1.000000}%
\pgfsetstrokecolor{currentstroke}%
\pgfsetdash{}{0pt}%
\pgfpathmoveto{\pgfqpoint{0.287500in}{1.612560in}}%
\pgfpathlineto{\pgfqpoint{2.070000in}{1.612560in}}%
\pgfusepath{stroke}%
\end{pgfscope}%
\begin{pgfscope}%
\definecolor{textcolor}{rgb}{0.150000,0.150000,0.150000}%
\pgfsetstrokecolor{textcolor}%
\pgfsetfillcolor{textcolor}%
\pgftext[x=-0.017284in,y=1.570351in,left,base]{\color{textcolor}\rmfamily\fontsize{8.000000}{9.600000}\selectfont \(\displaystyle 10^{-3}\)}%
\end{pgfscope}%
\begin{pgfscope}%
\pgfpathrectangle{\pgfqpoint{0.287500in}{0.375000in}}{\pgfqpoint{1.782500in}{2.265000in}}%
\pgfusepath{clip}%
\pgfsetroundcap%
\pgfsetroundjoin%
\pgfsetlinewidth{0.803000pt}%
\definecolor{currentstroke}{rgb}{1.000000,1.000000,1.000000}%
\pgfsetstrokecolor{currentstroke}%
\pgfsetdash{}{0pt}%
\pgfpathmoveto{\pgfqpoint{0.287500in}{1.938278in}}%
\pgfpathlineto{\pgfqpoint{2.070000in}{1.938278in}}%
\pgfusepath{stroke}%
\end{pgfscope}%
\begin{pgfscope}%
\definecolor{textcolor}{rgb}{0.150000,0.150000,0.150000}%
\pgfsetstrokecolor{textcolor}%
\pgfsetfillcolor{textcolor}%
\pgftext[x=-0.017284in,y=1.896069in,left,base]{\color{textcolor}\rmfamily\fontsize{8.000000}{9.600000}\selectfont \(\displaystyle 10^{-2}\)}%
\end{pgfscope}%
\begin{pgfscope}%
\pgfpathrectangle{\pgfqpoint{0.287500in}{0.375000in}}{\pgfqpoint{1.782500in}{2.265000in}}%
\pgfusepath{clip}%
\pgfsetroundcap%
\pgfsetroundjoin%
\pgfsetlinewidth{0.803000pt}%
\definecolor{currentstroke}{rgb}{1.000000,1.000000,1.000000}%
\pgfsetstrokecolor{currentstroke}%
\pgfsetdash{}{0pt}%
\pgfpathmoveto{\pgfqpoint{0.287500in}{2.263996in}}%
\pgfpathlineto{\pgfqpoint{2.070000in}{2.263996in}}%
\pgfusepath{stroke}%
\end{pgfscope}%
\begin{pgfscope}%
\definecolor{textcolor}{rgb}{0.150000,0.150000,0.150000}%
\pgfsetstrokecolor{textcolor}%
\pgfsetfillcolor{textcolor}%
\pgftext[x=-0.017284in,y=2.221787in,left,base]{\color{textcolor}\rmfamily\fontsize{8.000000}{9.600000}\selectfont \(\displaystyle 10^{-1}\)}%
\end{pgfscope}%
\begin{pgfscope}%
\pgfpathrectangle{\pgfqpoint{0.287500in}{0.375000in}}{\pgfqpoint{1.782500in}{2.265000in}}%
\pgfusepath{clip}%
\pgfsetroundcap%
\pgfsetroundjoin%
\pgfsetlinewidth{0.803000pt}%
\definecolor{currentstroke}{rgb}{1.000000,1.000000,1.000000}%
\pgfsetstrokecolor{currentstroke}%
\pgfsetdash{}{0pt}%
\pgfpathmoveto{\pgfqpoint{0.287500in}{2.589714in}}%
\pgfpathlineto{\pgfqpoint{2.070000in}{2.589714in}}%
\pgfusepath{stroke}%
\end{pgfscope}%
\begin{pgfscope}%
\definecolor{textcolor}{rgb}{0.150000,0.150000,0.150000}%
\pgfsetstrokecolor{textcolor}%
\pgfsetfillcolor{textcolor}%
\pgftext[x=0.062962in,y=2.547504in,left,base]{\color{textcolor}\rmfamily\fontsize{8.000000}{9.600000}\selectfont \(\displaystyle 10^{0}\)}%
\end{pgfscope}%
\begin{pgfscope}%
\definecolor{textcolor}{rgb}{0.150000,0.150000,0.150000}%
\pgfsetstrokecolor{textcolor}%
\pgfsetfillcolor{textcolor}%
\pgftext[x=-0.072840in,y=1.507500in,,bottom,rotate=90.000000]{\color{textcolor}\rmfamily\fontsize{8.000000}{9.600000}\selectfont Simple Regret}%
\end{pgfscope}%
\begin{pgfscope}%
\pgfpathrectangle{\pgfqpoint{0.287500in}{0.375000in}}{\pgfqpoint{1.782500in}{2.265000in}}%
\pgfusepath{clip}%
\pgfsetbuttcap%
\pgfsetroundjoin%
\definecolor{currentfill}{rgb}{0.121569,0.466667,0.705882}%
\pgfsetfillcolor{currentfill}%
\pgfsetfillopacity{0.200000}%
\pgfsetlinewidth{0.000000pt}%
\definecolor{currentstroke}{rgb}{0.000000,0.000000,0.000000}%
\pgfsetstrokecolor{currentstroke}%
\pgfsetdash{}{0pt}%
\pgfpathmoveto{\pgfqpoint{0.368523in}{2.468958in}}%
\pgfpathlineto{\pgfqpoint{0.368523in}{2.525361in}}%
\pgfpathlineto{\pgfqpoint{0.376585in}{2.487937in}}%
\pgfpathlineto{\pgfqpoint{0.384647in}{2.470226in}}%
\pgfpathlineto{\pgfqpoint{0.392709in}{2.447242in}}%
\pgfpathlineto{\pgfqpoint{0.400771in}{2.406452in}}%
\pgfpathlineto{\pgfqpoint{0.408833in}{2.384202in}}%
\pgfpathlineto{\pgfqpoint{0.416895in}{2.383165in}}%
\pgfpathlineto{\pgfqpoint{0.424956in}{2.375825in}}%
\pgfpathlineto{\pgfqpoint{0.433018in}{2.362816in}}%
\pgfpathlineto{\pgfqpoint{0.441080in}{2.362816in}}%
\pgfpathlineto{\pgfqpoint{0.449142in}{2.362816in}}%
\pgfpathlineto{\pgfqpoint{0.457204in}{2.321562in}}%
\pgfpathlineto{\pgfqpoint{0.465266in}{2.321562in}}%
\pgfpathlineto{\pgfqpoint{0.473328in}{2.302255in}}%
\pgfpathlineto{\pgfqpoint{0.481390in}{2.301640in}}%
\pgfpathlineto{\pgfqpoint{0.489452in}{2.296924in}}%
\pgfpathlineto{\pgfqpoint{0.497514in}{2.275816in}}%
\pgfpathlineto{\pgfqpoint{0.505576in}{2.246029in}}%
\pgfpathlineto{\pgfqpoint{0.513638in}{2.246029in}}%
\pgfpathlineto{\pgfqpoint{0.521700in}{2.239090in}}%
\pgfpathlineto{\pgfqpoint{0.529762in}{2.239090in}}%
\pgfpathlineto{\pgfqpoint{0.537824in}{2.225356in}}%
\pgfpathlineto{\pgfqpoint{0.545886in}{2.225356in}}%
\pgfpathlineto{\pgfqpoint{0.553948in}{2.210359in}}%
\pgfpathlineto{\pgfqpoint{0.562010in}{2.210359in}}%
\pgfpathlineto{\pgfqpoint{0.570072in}{2.210359in}}%
\pgfpathlineto{\pgfqpoint{0.578134in}{2.200588in}}%
\pgfpathlineto{\pgfqpoint{0.586196in}{2.200588in}}%
\pgfpathlineto{\pgfqpoint{0.594258in}{2.180715in}}%
\pgfpathlineto{\pgfqpoint{0.602320in}{2.180715in}}%
\pgfpathlineto{\pgfqpoint{0.610382in}{2.180715in}}%
\pgfpathlineto{\pgfqpoint{0.618444in}{2.180715in}}%
\pgfpathlineto{\pgfqpoint{0.626506in}{2.141212in}}%
\pgfpathlineto{\pgfqpoint{0.634568in}{2.104889in}}%
\pgfpathlineto{\pgfqpoint{0.642629in}{2.104889in}}%
\pgfpathlineto{\pgfqpoint{0.650691in}{2.104889in}}%
\pgfpathlineto{\pgfqpoint{0.658753in}{2.104889in}}%
\pgfpathlineto{\pgfqpoint{0.666815in}{2.088493in}}%
\pgfpathlineto{\pgfqpoint{0.674877in}{2.088493in}}%
\pgfpathlineto{\pgfqpoint{0.682939in}{2.088493in}}%
\pgfpathlineto{\pgfqpoint{0.691001in}{2.083617in}}%
\pgfpathlineto{\pgfqpoint{0.699063in}{2.076319in}}%
\pgfpathlineto{\pgfqpoint{0.707125in}{2.076319in}}%
\pgfpathlineto{\pgfqpoint{0.715187in}{2.076319in}}%
\pgfpathlineto{\pgfqpoint{0.723249in}{2.076319in}}%
\pgfpathlineto{\pgfqpoint{0.731311in}{2.076319in}}%
\pgfpathlineto{\pgfqpoint{0.739373in}{2.076319in}}%
\pgfpathlineto{\pgfqpoint{0.747435in}{2.076319in}}%
\pgfpathlineto{\pgfqpoint{0.755497in}{2.076319in}}%
\pgfpathlineto{\pgfqpoint{0.763559in}{2.076319in}}%
\pgfpathlineto{\pgfqpoint{0.771621in}{2.076319in}}%
\pgfpathlineto{\pgfqpoint{0.779683in}{2.076319in}}%
\pgfpathlineto{\pgfqpoint{0.787745in}{2.076319in}}%
\pgfpathlineto{\pgfqpoint{0.795807in}{2.076319in}}%
\pgfpathlineto{\pgfqpoint{0.803869in}{2.076319in}}%
\pgfpathlineto{\pgfqpoint{0.811931in}{2.074961in}}%
\pgfpathlineto{\pgfqpoint{0.819993in}{2.074961in}}%
\pgfpathlineto{\pgfqpoint{0.828055in}{2.074961in}}%
\pgfpathlineto{\pgfqpoint{0.836117in}{2.074961in}}%
\pgfpathlineto{\pgfqpoint{0.844179in}{2.074961in}}%
\pgfpathlineto{\pgfqpoint{0.852241in}{2.074961in}}%
\pgfpathlineto{\pgfqpoint{0.860302in}{2.074961in}}%
\pgfpathlineto{\pgfqpoint{0.868364in}{2.046505in}}%
\pgfpathlineto{\pgfqpoint{0.876426in}{2.036839in}}%
\pgfpathlineto{\pgfqpoint{0.884488in}{2.036839in}}%
\pgfpathlineto{\pgfqpoint{0.892550in}{1.890795in}}%
\pgfpathlineto{\pgfqpoint{0.900612in}{1.890795in}}%
\pgfpathlineto{\pgfqpoint{0.908674in}{1.890795in}}%
\pgfpathlineto{\pgfqpoint{0.916736in}{1.890795in}}%
\pgfpathlineto{\pgfqpoint{0.924798in}{1.890795in}}%
\pgfpathlineto{\pgfqpoint{0.932860in}{1.890782in}}%
\pgfpathlineto{\pgfqpoint{0.940922in}{1.890782in}}%
\pgfpathlineto{\pgfqpoint{0.948984in}{1.890782in}}%
\pgfpathlineto{\pgfqpoint{0.957046in}{1.890782in}}%
\pgfpathlineto{\pgfqpoint{0.965108in}{1.890782in}}%
\pgfpathlineto{\pgfqpoint{0.973170in}{1.890782in}}%
\pgfpathlineto{\pgfqpoint{0.981232in}{1.890782in}}%
\pgfpathlineto{\pgfqpoint{0.989294in}{1.890782in}}%
\pgfpathlineto{\pgfqpoint{0.997356in}{1.890782in}}%
\pgfpathlineto{\pgfqpoint{1.005418in}{1.890782in}}%
\pgfpathlineto{\pgfqpoint{1.013480in}{1.890782in}}%
\pgfpathlineto{\pgfqpoint{1.021542in}{1.890782in}}%
\pgfpathlineto{\pgfqpoint{1.029604in}{1.890782in}}%
\pgfpathlineto{\pgfqpoint{1.037666in}{1.890782in}}%
\pgfpathlineto{\pgfqpoint{1.045728in}{1.890782in}}%
\pgfpathlineto{\pgfqpoint{1.053790in}{1.890782in}}%
\pgfpathlineto{\pgfqpoint{1.061852in}{1.890782in}}%
\pgfpathlineto{\pgfqpoint{1.069914in}{1.890782in}}%
\pgfpathlineto{\pgfqpoint{1.077975in}{1.890782in}}%
\pgfpathlineto{\pgfqpoint{1.086037in}{1.886638in}}%
\pgfpathlineto{\pgfqpoint{1.094099in}{1.886638in}}%
\pgfpathlineto{\pgfqpoint{1.102161in}{1.886638in}}%
\pgfpathlineto{\pgfqpoint{1.110223in}{1.875216in}}%
\pgfpathlineto{\pgfqpoint{1.118285in}{1.875216in}}%
\pgfpathlineto{\pgfqpoint{1.126347in}{1.875216in}}%
\pgfpathlineto{\pgfqpoint{1.134409in}{1.875216in}}%
\pgfpathlineto{\pgfqpoint{1.142471in}{1.875216in}}%
\pgfpathlineto{\pgfqpoint{1.150533in}{1.875216in}}%
\pgfpathlineto{\pgfqpoint{1.158595in}{1.875008in}}%
\pgfpathlineto{\pgfqpoint{1.166657in}{1.875008in}}%
\pgfpathlineto{\pgfqpoint{1.174719in}{1.875008in}}%
\pgfpathlineto{\pgfqpoint{1.182781in}{1.875008in}}%
\pgfpathlineto{\pgfqpoint{1.190843in}{1.875008in}}%
\pgfpathlineto{\pgfqpoint{1.198905in}{1.875008in}}%
\pgfpathlineto{\pgfqpoint{1.206967in}{1.866204in}}%
\pgfpathlineto{\pgfqpoint{1.215029in}{1.866204in}}%
\pgfpathlineto{\pgfqpoint{1.223091in}{1.866204in}}%
\pgfpathlineto{\pgfqpoint{1.231153in}{1.866204in}}%
\pgfpathlineto{\pgfqpoint{1.239215in}{1.866204in}}%
\pgfpathlineto{\pgfqpoint{1.247277in}{1.866204in}}%
\pgfpathlineto{\pgfqpoint{1.255339in}{1.866204in}}%
\pgfpathlineto{\pgfqpoint{1.263401in}{1.866204in}}%
\pgfpathlineto{\pgfqpoint{1.271463in}{1.866204in}}%
\pgfpathlineto{\pgfqpoint{1.279525in}{1.866204in}}%
\pgfpathlineto{\pgfqpoint{1.287586in}{1.866204in}}%
\pgfpathlineto{\pgfqpoint{1.295648in}{1.866204in}}%
\pgfpathlineto{\pgfqpoint{1.303710in}{1.866204in}}%
\pgfpathlineto{\pgfqpoint{1.311772in}{1.866204in}}%
\pgfpathlineto{\pgfqpoint{1.319834in}{1.866204in}}%
\pgfpathlineto{\pgfqpoint{1.327896in}{1.790514in}}%
\pgfpathlineto{\pgfqpoint{1.335958in}{1.790514in}}%
\pgfpathlineto{\pgfqpoint{1.344020in}{1.790514in}}%
\pgfpathlineto{\pgfqpoint{1.352082in}{1.790514in}}%
\pgfpathlineto{\pgfqpoint{1.360144in}{1.790514in}}%
\pgfpathlineto{\pgfqpoint{1.368206in}{1.790514in}}%
\pgfpathlineto{\pgfqpoint{1.376268in}{1.790514in}}%
\pgfpathlineto{\pgfqpoint{1.384330in}{1.790514in}}%
\pgfpathlineto{\pgfqpoint{1.392392in}{1.790514in}}%
\pgfpathlineto{\pgfqpoint{1.400454in}{1.790514in}}%
\pgfpathlineto{\pgfqpoint{1.408516in}{1.790514in}}%
\pgfpathlineto{\pgfqpoint{1.416578in}{1.790514in}}%
\pgfpathlineto{\pgfqpoint{1.424640in}{1.790514in}}%
\pgfpathlineto{\pgfqpoint{1.432702in}{1.790514in}}%
\pgfpathlineto{\pgfqpoint{1.440764in}{1.790514in}}%
\pgfpathlineto{\pgfqpoint{1.448826in}{1.790514in}}%
\pgfpathlineto{\pgfqpoint{1.456888in}{1.790514in}}%
\pgfpathlineto{\pgfqpoint{1.464950in}{1.790514in}}%
\pgfpathlineto{\pgfqpoint{1.473012in}{1.790514in}}%
\pgfpathlineto{\pgfqpoint{1.481074in}{1.790514in}}%
\pgfpathlineto{\pgfqpoint{1.489136in}{1.790514in}}%
\pgfpathlineto{\pgfqpoint{1.497198in}{1.790514in}}%
\pgfpathlineto{\pgfqpoint{1.505259in}{1.790514in}}%
\pgfpathlineto{\pgfqpoint{1.513321in}{1.790514in}}%
\pgfpathlineto{\pgfqpoint{1.521383in}{1.790514in}}%
\pgfpathlineto{\pgfqpoint{1.529445in}{1.790514in}}%
\pgfpathlineto{\pgfqpoint{1.537507in}{1.790514in}}%
\pgfpathlineto{\pgfqpoint{1.545569in}{1.790514in}}%
\pgfpathlineto{\pgfqpoint{1.553631in}{1.790514in}}%
\pgfpathlineto{\pgfqpoint{1.561693in}{1.790514in}}%
\pgfpathlineto{\pgfqpoint{1.569755in}{1.790514in}}%
\pgfpathlineto{\pgfqpoint{1.577817in}{1.790514in}}%
\pgfpathlineto{\pgfqpoint{1.585879in}{1.790514in}}%
\pgfpathlineto{\pgfqpoint{1.593941in}{1.790514in}}%
\pgfpathlineto{\pgfqpoint{1.602003in}{1.790514in}}%
\pgfpathlineto{\pgfqpoint{1.610065in}{1.790514in}}%
\pgfpathlineto{\pgfqpoint{1.618127in}{1.790514in}}%
\pgfpathlineto{\pgfqpoint{1.626189in}{1.790514in}}%
\pgfpathlineto{\pgfqpoint{1.634251in}{1.790514in}}%
\pgfpathlineto{\pgfqpoint{1.642313in}{1.790514in}}%
\pgfpathlineto{\pgfqpoint{1.650375in}{1.790514in}}%
\pgfpathlineto{\pgfqpoint{1.658437in}{1.790514in}}%
\pgfpathlineto{\pgfqpoint{1.666499in}{1.746831in}}%
\pgfpathlineto{\pgfqpoint{1.674561in}{1.746831in}}%
\pgfpathlineto{\pgfqpoint{1.682623in}{1.746831in}}%
\pgfpathlineto{\pgfqpoint{1.690685in}{1.746831in}}%
\pgfpathlineto{\pgfqpoint{1.698747in}{1.746831in}}%
\pgfpathlineto{\pgfqpoint{1.706809in}{1.746831in}}%
\pgfpathlineto{\pgfqpoint{1.714871in}{1.746831in}}%
\pgfpathlineto{\pgfqpoint{1.722932in}{1.746831in}}%
\pgfpathlineto{\pgfqpoint{1.730994in}{1.746831in}}%
\pgfpathlineto{\pgfqpoint{1.739056in}{1.746831in}}%
\pgfpathlineto{\pgfqpoint{1.747118in}{1.746831in}}%
\pgfpathlineto{\pgfqpoint{1.755180in}{1.746831in}}%
\pgfpathlineto{\pgfqpoint{1.763242in}{1.746831in}}%
\pgfpathlineto{\pgfqpoint{1.771304in}{1.746831in}}%
\pgfpathlineto{\pgfqpoint{1.779366in}{1.746831in}}%
\pgfpathlineto{\pgfqpoint{1.787428in}{1.746831in}}%
\pgfpathlineto{\pgfqpoint{1.795490in}{1.746831in}}%
\pgfpathlineto{\pgfqpoint{1.803552in}{1.746831in}}%
\pgfpathlineto{\pgfqpoint{1.811614in}{1.746831in}}%
\pgfpathlineto{\pgfqpoint{1.819676in}{1.746831in}}%
\pgfpathlineto{\pgfqpoint{1.827738in}{1.746831in}}%
\pgfpathlineto{\pgfqpoint{1.835800in}{1.746831in}}%
\pgfpathlineto{\pgfqpoint{1.843862in}{1.746831in}}%
\pgfpathlineto{\pgfqpoint{1.851924in}{1.746831in}}%
\pgfpathlineto{\pgfqpoint{1.859986in}{1.746831in}}%
\pgfpathlineto{\pgfqpoint{1.868048in}{1.746831in}}%
\pgfpathlineto{\pgfqpoint{1.876110in}{1.746831in}}%
\pgfpathlineto{\pgfqpoint{1.884172in}{1.746831in}}%
\pgfpathlineto{\pgfqpoint{1.892234in}{1.746831in}}%
\pgfpathlineto{\pgfqpoint{1.900296in}{1.746831in}}%
\pgfpathlineto{\pgfqpoint{1.908358in}{1.746831in}}%
\pgfpathlineto{\pgfqpoint{1.916420in}{1.746831in}}%
\pgfpathlineto{\pgfqpoint{1.924482in}{1.746831in}}%
\pgfpathlineto{\pgfqpoint{1.932544in}{1.746831in}}%
\pgfpathlineto{\pgfqpoint{1.940605in}{1.746831in}}%
\pgfpathlineto{\pgfqpoint{1.948667in}{1.746831in}}%
\pgfpathlineto{\pgfqpoint{1.956729in}{1.679714in}}%
\pgfpathlineto{\pgfqpoint{1.964791in}{1.679714in}}%
\pgfpathlineto{\pgfqpoint{1.972853in}{1.624587in}}%
\pgfpathlineto{\pgfqpoint{1.980915in}{1.624587in}}%
\pgfpathlineto{\pgfqpoint{1.988977in}{1.624587in}}%
\pgfpathlineto{\pgfqpoint{1.988977in}{1.433588in}}%
\pgfpathlineto{\pgfqpoint{1.988977in}{1.433588in}}%
\pgfpathlineto{\pgfqpoint{1.980915in}{1.433588in}}%
\pgfpathlineto{\pgfqpoint{1.972853in}{1.433588in}}%
\pgfpathlineto{\pgfqpoint{1.964791in}{1.460673in}}%
\pgfpathlineto{\pgfqpoint{1.956729in}{1.460673in}}%
\pgfpathlineto{\pgfqpoint{1.948667in}{1.543028in}}%
\pgfpathlineto{\pgfqpoint{1.940605in}{1.543028in}}%
\pgfpathlineto{\pgfqpoint{1.932544in}{1.543028in}}%
\pgfpathlineto{\pgfqpoint{1.924482in}{1.543028in}}%
\pgfpathlineto{\pgfqpoint{1.916420in}{1.543028in}}%
\pgfpathlineto{\pgfqpoint{1.908358in}{1.543028in}}%
\pgfpathlineto{\pgfqpoint{1.900296in}{1.543028in}}%
\pgfpathlineto{\pgfqpoint{1.892234in}{1.543028in}}%
\pgfpathlineto{\pgfqpoint{1.884172in}{1.543028in}}%
\pgfpathlineto{\pgfqpoint{1.876110in}{1.543028in}}%
\pgfpathlineto{\pgfqpoint{1.868048in}{1.543028in}}%
\pgfpathlineto{\pgfqpoint{1.859986in}{1.543028in}}%
\pgfpathlineto{\pgfqpoint{1.851924in}{1.543028in}}%
\pgfpathlineto{\pgfqpoint{1.843862in}{1.543028in}}%
\pgfpathlineto{\pgfqpoint{1.835800in}{1.543028in}}%
\pgfpathlineto{\pgfqpoint{1.827738in}{1.543028in}}%
\pgfpathlineto{\pgfqpoint{1.819676in}{1.543028in}}%
\pgfpathlineto{\pgfqpoint{1.811614in}{1.543028in}}%
\pgfpathlineto{\pgfqpoint{1.803552in}{1.543028in}}%
\pgfpathlineto{\pgfqpoint{1.795490in}{1.543028in}}%
\pgfpathlineto{\pgfqpoint{1.787428in}{1.543028in}}%
\pgfpathlineto{\pgfqpoint{1.779366in}{1.543028in}}%
\pgfpathlineto{\pgfqpoint{1.771304in}{1.543028in}}%
\pgfpathlineto{\pgfqpoint{1.763242in}{1.543028in}}%
\pgfpathlineto{\pgfqpoint{1.755180in}{1.543028in}}%
\pgfpathlineto{\pgfqpoint{1.747118in}{1.543028in}}%
\pgfpathlineto{\pgfqpoint{1.739056in}{1.543028in}}%
\pgfpathlineto{\pgfqpoint{1.730994in}{1.543028in}}%
\pgfpathlineto{\pgfqpoint{1.722932in}{1.543028in}}%
\pgfpathlineto{\pgfqpoint{1.714871in}{1.543028in}}%
\pgfpathlineto{\pgfqpoint{1.706809in}{1.543028in}}%
\pgfpathlineto{\pgfqpoint{1.698747in}{1.543028in}}%
\pgfpathlineto{\pgfqpoint{1.690685in}{1.543028in}}%
\pgfpathlineto{\pgfqpoint{1.682623in}{1.543028in}}%
\pgfpathlineto{\pgfqpoint{1.674561in}{1.543028in}}%
\pgfpathlineto{\pgfqpoint{1.666499in}{1.543028in}}%
\pgfpathlineto{\pgfqpoint{1.658437in}{1.620941in}}%
\pgfpathlineto{\pgfqpoint{1.650375in}{1.620941in}}%
\pgfpathlineto{\pgfqpoint{1.642313in}{1.620941in}}%
\pgfpathlineto{\pgfqpoint{1.634251in}{1.620941in}}%
\pgfpathlineto{\pgfqpoint{1.626189in}{1.620941in}}%
\pgfpathlineto{\pgfqpoint{1.618127in}{1.620941in}}%
\pgfpathlineto{\pgfqpoint{1.610065in}{1.620941in}}%
\pgfpathlineto{\pgfqpoint{1.602003in}{1.620941in}}%
\pgfpathlineto{\pgfqpoint{1.593941in}{1.620941in}}%
\pgfpathlineto{\pgfqpoint{1.585879in}{1.620941in}}%
\pgfpathlineto{\pgfqpoint{1.577817in}{1.620941in}}%
\pgfpathlineto{\pgfqpoint{1.569755in}{1.620941in}}%
\pgfpathlineto{\pgfqpoint{1.561693in}{1.620941in}}%
\pgfpathlineto{\pgfqpoint{1.553631in}{1.620941in}}%
\pgfpathlineto{\pgfqpoint{1.545569in}{1.620941in}}%
\pgfpathlineto{\pgfqpoint{1.537507in}{1.620941in}}%
\pgfpathlineto{\pgfqpoint{1.529445in}{1.620941in}}%
\pgfpathlineto{\pgfqpoint{1.521383in}{1.620941in}}%
\pgfpathlineto{\pgfqpoint{1.513321in}{1.620941in}}%
\pgfpathlineto{\pgfqpoint{1.505259in}{1.620941in}}%
\pgfpathlineto{\pgfqpoint{1.497198in}{1.620941in}}%
\pgfpathlineto{\pgfqpoint{1.489136in}{1.620941in}}%
\pgfpathlineto{\pgfqpoint{1.481074in}{1.620941in}}%
\pgfpathlineto{\pgfqpoint{1.473012in}{1.620941in}}%
\pgfpathlineto{\pgfqpoint{1.464950in}{1.620941in}}%
\pgfpathlineto{\pgfqpoint{1.456888in}{1.620941in}}%
\pgfpathlineto{\pgfqpoint{1.448826in}{1.620941in}}%
\pgfpathlineto{\pgfqpoint{1.440764in}{1.620941in}}%
\pgfpathlineto{\pgfqpoint{1.432702in}{1.620941in}}%
\pgfpathlineto{\pgfqpoint{1.424640in}{1.620941in}}%
\pgfpathlineto{\pgfqpoint{1.416578in}{1.620941in}}%
\pgfpathlineto{\pgfqpoint{1.408516in}{1.620941in}}%
\pgfpathlineto{\pgfqpoint{1.400454in}{1.620941in}}%
\pgfpathlineto{\pgfqpoint{1.392392in}{1.620941in}}%
\pgfpathlineto{\pgfqpoint{1.384330in}{1.620941in}}%
\pgfpathlineto{\pgfqpoint{1.376268in}{1.620941in}}%
\pgfpathlineto{\pgfqpoint{1.368206in}{1.620941in}}%
\pgfpathlineto{\pgfqpoint{1.360144in}{1.620941in}}%
\pgfpathlineto{\pgfqpoint{1.352082in}{1.620941in}}%
\pgfpathlineto{\pgfqpoint{1.344020in}{1.620941in}}%
\pgfpathlineto{\pgfqpoint{1.335958in}{1.620941in}}%
\pgfpathlineto{\pgfqpoint{1.327896in}{1.620941in}}%
\pgfpathlineto{\pgfqpoint{1.319834in}{1.668174in}}%
\pgfpathlineto{\pgfqpoint{1.311772in}{1.668174in}}%
\pgfpathlineto{\pgfqpoint{1.303710in}{1.668174in}}%
\pgfpathlineto{\pgfqpoint{1.295648in}{1.668174in}}%
\pgfpathlineto{\pgfqpoint{1.287586in}{1.668174in}}%
\pgfpathlineto{\pgfqpoint{1.279525in}{1.668174in}}%
\pgfpathlineto{\pgfqpoint{1.271463in}{1.668174in}}%
\pgfpathlineto{\pgfqpoint{1.263401in}{1.668174in}}%
\pgfpathlineto{\pgfqpoint{1.255339in}{1.668174in}}%
\pgfpathlineto{\pgfqpoint{1.247277in}{1.668174in}}%
\pgfpathlineto{\pgfqpoint{1.239215in}{1.668174in}}%
\pgfpathlineto{\pgfqpoint{1.231153in}{1.668174in}}%
\pgfpathlineto{\pgfqpoint{1.223091in}{1.668174in}}%
\pgfpathlineto{\pgfqpoint{1.215029in}{1.668174in}}%
\pgfpathlineto{\pgfqpoint{1.206967in}{1.668174in}}%
\pgfpathlineto{\pgfqpoint{1.198905in}{1.663384in}}%
\pgfpathlineto{\pgfqpoint{1.190843in}{1.663384in}}%
\pgfpathlineto{\pgfqpoint{1.182781in}{1.663384in}}%
\pgfpathlineto{\pgfqpoint{1.174719in}{1.663384in}}%
\pgfpathlineto{\pgfqpoint{1.166657in}{1.663384in}}%
\pgfpathlineto{\pgfqpoint{1.158595in}{1.663384in}}%
\pgfpathlineto{\pgfqpoint{1.150533in}{1.665425in}}%
\pgfpathlineto{\pgfqpoint{1.142471in}{1.665425in}}%
\pgfpathlineto{\pgfqpoint{1.134409in}{1.665425in}}%
\pgfpathlineto{\pgfqpoint{1.126347in}{1.665425in}}%
\pgfpathlineto{\pgfqpoint{1.118285in}{1.665425in}}%
\pgfpathlineto{\pgfqpoint{1.110223in}{1.665425in}}%
\pgfpathlineto{\pgfqpoint{1.102161in}{1.657677in}}%
\pgfpathlineto{\pgfqpoint{1.094099in}{1.657677in}}%
\pgfpathlineto{\pgfqpoint{1.086037in}{1.657677in}}%
\pgfpathlineto{\pgfqpoint{1.077975in}{1.686867in}}%
\pgfpathlineto{\pgfqpoint{1.069914in}{1.686867in}}%
\pgfpathlineto{\pgfqpoint{1.061852in}{1.686867in}}%
\pgfpathlineto{\pgfqpoint{1.053790in}{1.686867in}}%
\pgfpathlineto{\pgfqpoint{1.045728in}{1.686867in}}%
\pgfpathlineto{\pgfqpoint{1.037666in}{1.686867in}}%
\pgfpathlineto{\pgfqpoint{1.029604in}{1.686867in}}%
\pgfpathlineto{\pgfqpoint{1.021542in}{1.686867in}}%
\pgfpathlineto{\pgfqpoint{1.013480in}{1.686867in}}%
\pgfpathlineto{\pgfqpoint{1.005418in}{1.686867in}}%
\pgfpathlineto{\pgfqpoint{0.997356in}{1.686867in}}%
\pgfpathlineto{\pgfqpoint{0.989294in}{1.686867in}}%
\pgfpathlineto{\pgfqpoint{0.981232in}{1.686867in}}%
\pgfpathlineto{\pgfqpoint{0.973170in}{1.686867in}}%
\pgfpathlineto{\pgfqpoint{0.965108in}{1.686867in}}%
\pgfpathlineto{\pgfqpoint{0.957046in}{1.686867in}}%
\pgfpathlineto{\pgfqpoint{0.948984in}{1.686867in}}%
\pgfpathlineto{\pgfqpoint{0.940922in}{1.686867in}}%
\pgfpathlineto{\pgfqpoint{0.932860in}{1.686867in}}%
\pgfpathlineto{\pgfqpoint{0.924798in}{1.686997in}}%
\pgfpathlineto{\pgfqpoint{0.916736in}{1.686997in}}%
\pgfpathlineto{\pgfqpoint{0.908674in}{1.686997in}}%
\pgfpathlineto{\pgfqpoint{0.900612in}{1.686997in}}%
\pgfpathlineto{\pgfqpoint{0.892550in}{1.686997in}}%
\pgfpathlineto{\pgfqpoint{0.884488in}{1.569545in}}%
\pgfpathlineto{\pgfqpoint{0.876426in}{1.569545in}}%
\pgfpathlineto{\pgfqpoint{0.868364in}{1.738289in}}%
\pgfpathlineto{\pgfqpoint{0.860302in}{1.862638in}}%
\pgfpathlineto{\pgfqpoint{0.852241in}{1.862638in}}%
\pgfpathlineto{\pgfqpoint{0.844179in}{1.862638in}}%
\pgfpathlineto{\pgfqpoint{0.836117in}{1.862638in}}%
\pgfpathlineto{\pgfqpoint{0.828055in}{1.862638in}}%
\pgfpathlineto{\pgfqpoint{0.819993in}{1.862638in}}%
\pgfpathlineto{\pgfqpoint{0.811931in}{1.862638in}}%
\pgfpathlineto{\pgfqpoint{0.803869in}{1.873030in}}%
\pgfpathlineto{\pgfqpoint{0.795807in}{1.873030in}}%
\pgfpathlineto{\pgfqpoint{0.787745in}{1.873030in}}%
\pgfpathlineto{\pgfqpoint{0.779683in}{1.873030in}}%
\pgfpathlineto{\pgfqpoint{0.771621in}{1.873030in}}%
\pgfpathlineto{\pgfqpoint{0.763559in}{1.873030in}}%
\pgfpathlineto{\pgfqpoint{0.755497in}{1.873030in}}%
\pgfpathlineto{\pgfqpoint{0.747435in}{1.873030in}}%
\pgfpathlineto{\pgfqpoint{0.739373in}{1.873030in}}%
\pgfpathlineto{\pgfqpoint{0.731311in}{1.873030in}}%
\pgfpathlineto{\pgfqpoint{0.723249in}{1.873030in}}%
\pgfpathlineto{\pgfqpoint{0.715187in}{1.873030in}}%
\pgfpathlineto{\pgfqpoint{0.707125in}{1.873030in}}%
\pgfpathlineto{\pgfqpoint{0.699063in}{1.873030in}}%
\pgfpathlineto{\pgfqpoint{0.691001in}{1.909021in}}%
\pgfpathlineto{\pgfqpoint{0.682939in}{1.930385in}}%
\pgfpathlineto{\pgfqpoint{0.674877in}{1.930385in}}%
\pgfpathlineto{\pgfqpoint{0.666815in}{1.930385in}}%
\pgfpathlineto{\pgfqpoint{0.658753in}{1.967378in}}%
\pgfpathlineto{\pgfqpoint{0.650691in}{1.967378in}}%
\pgfpathlineto{\pgfqpoint{0.642629in}{1.967378in}}%
\pgfpathlineto{\pgfqpoint{0.634568in}{1.967378in}}%
\pgfpathlineto{\pgfqpoint{0.626506in}{2.027363in}}%
\pgfpathlineto{\pgfqpoint{0.618444in}{2.055714in}}%
\pgfpathlineto{\pgfqpoint{0.610382in}{2.055714in}}%
\pgfpathlineto{\pgfqpoint{0.602320in}{2.055714in}}%
\pgfpathlineto{\pgfqpoint{0.594258in}{2.055714in}}%
\pgfpathlineto{\pgfqpoint{0.586196in}{2.069087in}}%
\pgfpathlineto{\pgfqpoint{0.578134in}{2.069087in}}%
\pgfpathlineto{\pgfqpoint{0.570072in}{2.104219in}}%
\pgfpathlineto{\pgfqpoint{0.562010in}{2.104219in}}%
\pgfpathlineto{\pgfqpoint{0.553948in}{2.104219in}}%
\pgfpathlineto{\pgfqpoint{0.545886in}{2.114972in}}%
\pgfpathlineto{\pgfqpoint{0.537824in}{2.114972in}}%
\pgfpathlineto{\pgfqpoint{0.529762in}{2.132716in}}%
\pgfpathlineto{\pgfqpoint{0.521700in}{2.132716in}}%
\pgfpathlineto{\pgfqpoint{0.513638in}{2.149372in}}%
\pgfpathlineto{\pgfqpoint{0.505576in}{2.149372in}}%
\pgfpathlineto{\pgfqpoint{0.497514in}{2.159185in}}%
\pgfpathlineto{\pgfqpoint{0.489452in}{2.200059in}}%
\pgfpathlineto{\pgfqpoint{0.481390in}{2.209564in}}%
\pgfpathlineto{\pgfqpoint{0.473328in}{2.215833in}}%
\pgfpathlineto{\pgfqpoint{0.465266in}{2.228237in}}%
\pgfpathlineto{\pgfqpoint{0.457204in}{2.228237in}}%
\pgfpathlineto{\pgfqpoint{0.449142in}{2.246742in}}%
\pgfpathlineto{\pgfqpoint{0.441080in}{2.246742in}}%
\pgfpathlineto{\pgfqpoint{0.433018in}{2.246742in}}%
\pgfpathlineto{\pgfqpoint{0.424956in}{2.277699in}}%
\pgfpathlineto{\pgfqpoint{0.416895in}{2.300031in}}%
\pgfpathlineto{\pgfqpoint{0.408833in}{2.307496in}}%
\pgfpathlineto{\pgfqpoint{0.400771in}{2.327084in}}%
\pgfpathlineto{\pgfqpoint{0.392709in}{2.370298in}}%
\pgfpathlineto{\pgfqpoint{0.384647in}{2.385506in}}%
\pgfpathlineto{\pgfqpoint{0.376585in}{2.417307in}}%
\pgfpathlineto{\pgfqpoint{0.368523in}{2.468958in}}%
\pgfpathclose%
\pgfusepath{fill}%
\end{pgfscope}%
\begin{pgfscope}%
\pgfpathrectangle{\pgfqpoint{0.287500in}{0.375000in}}{\pgfqpoint{1.782500in}{2.265000in}}%
\pgfusepath{clip}%
\pgfsetbuttcap%
\pgfsetroundjoin%
\definecolor{currentfill}{rgb}{1.000000,0.498039,0.054902}%
\pgfsetfillcolor{currentfill}%
\pgfsetfillopacity{0.200000}%
\pgfsetlinewidth{0.000000pt}%
\definecolor{currentstroke}{rgb}{0.000000,0.000000,0.000000}%
\pgfsetstrokecolor{currentstroke}%
\pgfsetdash{}{0pt}%
\pgfpathmoveto{\pgfqpoint{0.368523in}{2.430219in}}%
\pgfpathlineto{\pgfqpoint{0.368523in}{2.504728in}}%
\pgfpathlineto{\pgfqpoint{0.376585in}{2.399809in}}%
\pgfpathlineto{\pgfqpoint{0.384647in}{2.379552in}}%
\pgfpathlineto{\pgfqpoint{0.392709in}{2.379552in}}%
\pgfpathlineto{\pgfqpoint{0.400771in}{2.379552in}}%
\pgfpathlineto{\pgfqpoint{0.408833in}{2.379552in}}%
\pgfpathlineto{\pgfqpoint{0.416895in}{2.379552in}}%
\pgfpathlineto{\pgfqpoint{0.424956in}{2.379552in}}%
\pgfpathlineto{\pgfqpoint{0.433018in}{2.379552in}}%
\pgfpathlineto{\pgfqpoint{0.441080in}{2.379552in}}%
\pgfpathlineto{\pgfqpoint{0.449142in}{2.365593in}}%
\pgfpathlineto{\pgfqpoint{0.457204in}{2.336552in}}%
\pgfpathlineto{\pgfqpoint{0.465266in}{2.271713in}}%
\pgfpathlineto{\pgfqpoint{0.473328in}{2.156697in}}%
\pgfpathlineto{\pgfqpoint{0.481390in}{2.156697in}}%
\pgfpathlineto{\pgfqpoint{0.489452in}{2.134674in}}%
\pgfpathlineto{\pgfqpoint{0.497514in}{2.134674in}}%
\pgfpathlineto{\pgfqpoint{0.505576in}{2.134674in}}%
\pgfpathlineto{\pgfqpoint{0.513638in}{2.134674in}}%
\pgfpathlineto{\pgfqpoint{0.521700in}{2.094901in}}%
\pgfpathlineto{\pgfqpoint{0.529762in}{2.094901in}}%
\pgfpathlineto{\pgfqpoint{0.537824in}{2.094901in}}%
\pgfpathlineto{\pgfqpoint{0.545886in}{2.094901in}}%
\pgfpathlineto{\pgfqpoint{0.553948in}{2.094901in}}%
\pgfpathlineto{\pgfqpoint{0.562010in}{2.068264in}}%
\pgfpathlineto{\pgfqpoint{0.570072in}{2.068264in}}%
\pgfpathlineto{\pgfqpoint{0.578134in}{2.068264in}}%
\pgfpathlineto{\pgfqpoint{0.586196in}{2.068264in}}%
\pgfpathlineto{\pgfqpoint{0.594258in}{2.068264in}}%
\pgfpathlineto{\pgfqpoint{0.602320in}{2.068264in}}%
\pgfpathlineto{\pgfqpoint{0.610382in}{2.068264in}}%
\pgfpathlineto{\pgfqpoint{0.618444in}{2.068264in}}%
\pgfpathlineto{\pgfqpoint{0.626506in}{2.068264in}}%
\pgfpathlineto{\pgfqpoint{0.634568in}{2.068264in}}%
\pgfpathlineto{\pgfqpoint{0.642629in}{2.068264in}}%
\pgfpathlineto{\pgfqpoint{0.650691in}{2.068264in}}%
\pgfpathlineto{\pgfqpoint{0.658753in}{2.068264in}}%
\pgfpathlineto{\pgfqpoint{0.666815in}{2.068264in}}%
\pgfpathlineto{\pgfqpoint{0.674877in}{2.020462in}}%
\pgfpathlineto{\pgfqpoint{0.682939in}{2.020462in}}%
\pgfpathlineto{\pgfqpoint{0.691001in}{2.020462in}}%
\pgfpathlineto{\pgfqpoint{0.699063in}{1.980783in}}%
\pgfpathlineto{\pgfqpoint{0.707125in}{1.980783in}}%
\pgfpathlineto{\pgfqpoint{0.715187in}{1.931200in}}%
\pgfpathlineto{\pgfqpoint{0.723249in}{1.931200in}}%
\pgfpathlineto{\pgfqpoint{0.731311in}{1.931200in}}%
\pgfpathlineto{\pgfqpoint{0.739373in}{1.931200in}}%
\pgfpathlineto{\pgfqpoint{0.747435in}{1.931200in}}%
\pgfpathlineto{\pgfqpoint{0.755497in}{1.931200in}}%
\pgfpathlineto{\pgfqpoint{0.763559in}{1.931200in}}%
\pgfpathlineto{\pgfqpoint{0.771621in}{1.931200in}}%
\pgfpathlineto{\pgfqpoint{0.779683in}{1.931200in}}%
\pgfpathlineto{\pgfqpoint{0.787745in}{1.930272in}}%
\pgfpathlineto{\pgfqpoint{0.795807in}{1.930272in}}%
\pgfpathlineto{\pgfqpoint{0.803869in}{1.717816in}}%
\pgfpathlineto{\pgfqpoint{0.811931in}{1.717816in}}%
\pgfpathlineto{\pgfqpoint{0.819993in}{1.638675in}}%
\pgfpathlineto{\pgfqpoint{0.828055in}{1.638675in}}%
\pgfpathlineto{\pgfqpoint{0.836117in}{1.624260in}}%
\pgfpathlineto{\pgfqpoint{0.844179in}{1.624260in}}%
\pgfpathlineto{\pgfqpoint{0.852241in}{1.613949in}}%
\pgfpathlineto{\pgfqpoint{0.860302in}{1.613949in}}%
\pgfpathlineto{\pgfqpoint{0.868364in}{1.613245in}}%
\pgfpathlineto{\pgfqpoint{0.876426in}{1.613245in}}%
\pgfpathlineto{\pgfqpoint{0.884488in}{1.312175in}}%
\pgfpathlineto{\pgfqpoint{0.892550in}{1.312175in}}%
\pgfpathlineto{\pgfqpoint{0.900612in}{1.312175in}}%
\pgfpathlineto{\pgfqpoint{0.908674in}{0.973341in}}%
\pgfpathlineto{\pgfqpoint{0.916736in}{0.973341in}}%
\pgfpathlineto{\pgfqpoint{0.924798in}{0.973341in}}%
\pgfpathlineto{\pgfqpoint{0.932860in}{0.973341in}}%
\pgfpathlineto{\pgfqpoint{0.940922in}{0.973341in}}%
\pgfpathlineto{\pgfqpoint{0.948984in}{0.973341in}}%
\pgfpathlineto{\pgfqpoint{0.957046in}{0.973341in}}%
\pgfpathlineto{\pgfqpoint{0.965108in}{0.881257in}}%
\pgfpathlineto{\pgfqpoint{0.973170in}{0.881257in}}%
\pgfpathlineto{\pgfqpoint{0.981232in}{0.881257in}}%
\pgfpathlineto{\pgfqpoint{0.989294in}{0.881257in}}%
\pgfpathlineto{\pgfqpoint{0.997356in}{0.881257in}}%
\pgfpathlineto{\pgfqpoint{1.005418in}{0.881257in}}%
\pgfpathlineto{\pgfqpoint{1.013480in}{0.881257in}}%
\pgfpathlineto{\pgfqpoint{1.021542in}{0.881257in}}%
\pgfpathlineto{\pgfqpoint{1.029604in}{0.881257in}}%
\pgfpathlineto{\pgfqpoint{1.037666in}{0.881257in}}%
\pgfpathlineto{\pgfqpoint{1.045728in}{0.881257in}}%
\pgfpathlineto{\pgfqpoint{1.053790in}{0.881257in}}%
\pgfpathlineto{\pgfqpoint{1.061852in}{0.881257in}}%
\pgfpathlineto{\pgfqpoint{1.069914in}{0.881257in}}%
\pgfpathlineto{\pgfqpoint{1.077975in}{0.861107in}}%
\pgfpathlineto{\pgfqpoint{1.086037in}{0.861107in}}%
\pgfpathlineto{\pgfqpoint{1.094099in}{0.861107in}}%
\pgfpathlineto{\pgfqpoint{1.102161in}{0.857395in}}%
\pgfpathlineto{\pgfqpoint{1.110223in}{0.843708in}}%
\pgfpathlineto{\pgfqpoint{1.118285in}{0.843708in}}%
\pgfpathlineto{\pgfqpoint{1.126347in}{0.843708in}}%
\pgfpathlineto{\pgfqpoint{1.134409in}{0.843708in}}%
\pgfpathlineto{\pgfqpoint{1.142471in}{0.843708in}}%
\pgfpathlineto{\pgfqpoint{1.150533in}{0.843708in}}%
\pgfpathlineto{\pgfqpoint{1.158595in}{0.843708in}}%
\pgfpathlineto{\pgfqpoint{1.166657in}{0.843708in}}%
\pgfpathlineto{\pgfqpoint{1.174719in}{0.843708in}}%
\pgfpathlineto{\pgfqpoint{1.182781in}{0.843708in}}%
\pgfpathlineto{\pgfqpoint{1.190843in}{0.843708in}}%
\pgfpathlineto{\pgfqpoint{1.198905in}{0.843708in}}%
\pgfpathlineto{\pgfqpoint{1.206967in}{0.843708in}}%
\pgfpathlineto{\pgfqpoint{1.215029in}{0.843708in}}%
\pgfpathlineto{\pgfqpoint{1.223091in}{0.843708in}}%
\pgfpathlineto{\pgfqpoint{1.231153in}{0.843708in}}%
\pgfpathlineto{\pgfqpoint{1.239215in}{0.843708in}}%
\pgfpathlineto{\pgfqpoint{1.247277in}{0.843708in}}%
\pgfpathlineto{\pgfqpoint{1.255339in}{0.843708in}}%
\pgfpathlineto{\pgfqpoint{1.263401in}{0.843708in}}%
\pgfpathlineto{\pgfqpoint{1.271463in}{0.843708in}}%
\pgfpathlineto{\pgfqpoint{1.279525in}{0.843708in}}%
\pgfpathlineto{\pgfqpoint{1.287586in}{0.836642in}}%
\pgfpathlineto{\pgfqpoint{1.295648in}{0.836642in}}%
\pgfpathlineto{\pgfqpoint{1.303710in}{0.836642in}}%
\pgfpathlineto{\pgfqpoint{1.311772in}{0.836642in}}%
\pgfpathlineto{\pgfqpoint{1.319834in}{0.836642in}}%
\pgfpathlineto{\pgfqpoint{1.327896in}{0.836642in}}%
\pgfpathlineto{\pgfqpoint{1.335958in}{0.836642in}}%
\pgfpathlineto{\pgfqpoint{1.344020in}{0.836642in}}%
\pgfpathlineto{\pgfqpoint{1.352082in}{0.836642in}}%
\pgfpathlineto{\pgfqpoint{1.360144in}{0.836642in}}%
\pgfpathlineto{\pgfqpoint{1.368206in}{0.836642in}}%
\pgfpathlineto{\pgfqpoint{1.376268in}{0.836642in}}%
\pgfpathlineto{\pgfqpoint{1.384330in}{0.836642in}}%
\pgfpathlineto{\pgfqpoint{1.392392in}{0.836642in}}%
\pgfpathlineto{\pgfqpoint{1.400454in}{0.836642in}}%
\pgfpathlineto{\pgfqpoint{1.408516in}{0.699043in}}%
\pgfpathlineto{\pgfqpoint{1.416578in}{0.682135in}}%
\pgfpathlineto{\pgfqpoint{1.424640in}{0.677171in}}%
\pgfpathlineto{\pgfqpoint{1.432702in}{0.677171in}}%
\pgfpathlineto{\pgfqpoint{1.440764in}{0.677171in}}%
\pgfpathlineto{\pgfqpoint{1.448826in}{0.677171in}}%
\pgfpathlineto{\pgfqpoint{1.456888in}{0.677171in}}%
\pgfpathlineto{\pgfqpoint{1.464950in}{0.677171in}}%
\pgfpathlineto{\pgfqpoint{1.473012in}{0.677171in}}%
\pgfpathlineto{\pgfqpoint{1.481074in}{0.677171in}}%
\pgfpathlineto{\pgfqpoint{1.489136in}{0.677171in}}%
\pgfpathlineto{\pgfqpoint{1.497198in}{0.677171in}}%
\pgfpathlineto{\pgfqpoint{1.505259in}{0.677171in}}%
\pgfpathlineto{\pgfqpoint{1.513321in}{0.677171in}}%
\pgfpathlineto{\pgfqpoint{1.521383in}{0.677171in}}%
\pgfpathlineto{\pgfqpoint{1.529445in}{0.677171in}}%
\pgfpathlineto{\pgfqpoint{1.537507in}{0.676627in}}%
\pgfpathlineto{\pgfqpoint{1.545569in}{0.676627in}}%
\pgfpathlineto{\pgfqpoint{1.553631in}{0.676627in}}%
\pgfpathlineto{\pgfqpoint{1.561693in}{0.676627in}}%
\pgfpathlineto{\pgfqpoint{1.569755in}{0.676627in}}%
\pgfpathlineto{\pgfqpoint{1.577817in}{0.676627in}}%
\pgfpathlineto{\pgfqpoint{1.585879in}{0.676488in}}%
\pgfpathlineto{\pgfqpoint{1.593941in}{0.676488in}}%
\pgfpathlineto{\pgfqpoint{1.602003in}{0.676488in}}%
\pgfpathlineto{\pgfqpoint{1.610065in}{0.676488in}}%
\pgfpathlineto{\pgfqpoint{1.618127in}{0.676488in}}%
\pgfpathlineto{\pgfqpoint{1.626189in}{0.676488in}}%
\pgfpathlineto{\pgfqpoint{1.634251in}{0.676488in}}%
\pgfpathlineto{\pgfqpoint{1.642313in}{0.676488in}}%
\pgfpathlineto{\pgfqpoint{1.650375in}{0.676488in}}%
\pgfpathlineto{\pgfqpoint{1.658437in}{0.676488in}}%
\pgfpathlineto{\pgfqpoint{1.666499in}{0.676488in}}%
\pgfpathlineto{\pgfqpoint{1.674561in}{0.676488in}}%
\pgfpathlineto{\pgfqpoint{1.682623in}{0.676488in}}%
\pgfpathlineto{\pgfqpoint{1.690685in}{0.635142in}}%
\pgfpathlineto{\pgfqpoint{1.698747in}{0.635142in}}%
\pgfpathlineto{\pgfqpoint{1.706809in}{0.635142in}}%
\pgfpathlineto{\pgfqpoint{1.714871in}{0.635142in}}%
\pgfpathlineto{\pgfqpoint{1.722932in}{0.635142in}}%
\pgfpathlineto{\pgfqpoint{1.730994in}{0.635142in}}%
\pgfpathlineto{\pgfqpoint{1.739056in}{0.635142in}}%
\pgfpathlineto{\pgfqpoint{1.747118in}{0.635142in}}%
\pgfpathlineto{\pgfqpoint{1.755180in}{0.635142in}}%
\pgfpathlineto{\pgfqpoint{1.763242in}{0.635142in}}%
\pgfpathlineto{\pgfqpoint{1.771304in}{0.635142in}}%
\pgfpathlineto{\pgfqpoint{1.779366in}{0.635142in}}%
\pgfpathlineto{\pgfqpoint{1.787428in}{0.635142in}}%
\pgfpathlineto{\pgfqpoint{1.795490in}{0.635142in}}%
\pgfpathlineto{\pgfqpoint{1.803552in}{0.635142in}}%
\pgfpathlineto{\pgfqpoint{1.811614in}{0.635142in}}%
\pgfpathlineto{\pgfqpoint{1.819676in}{0.635142in}}%
\pgfpathlineto{\pgfqpoint{1.827738in}{0.635142in}}%
\pgfpathlineto{\pgfqpoint{1.835800in}{0.635142in}}%
\pgfpathlineto{\pgfqpoint{1.843862in}{0.635142in}}%
\pgfpathlineto{\pgfqpoint{1.851924in}{0.635142in}}%
\pgfpathlineto{\pgfqpoint{1.859986in}{0.635142in}}%
\pgfpathlineto{\pgfqpoint{1.868048in}{0.635142in}}%
\pgfpathlineto{\pgfqpoint{1.876110in}{0.635142in}}%
\pgfpathlineto{\pgfqpoint{1.884172in}{0.635142in}}%
\pgfpathlineto{\pgfqpoint{1.892234in}{0.635142in}}%
\pgfpathlineto{\pgfqpoint{1.900296in}{0.635142in}}%
\pgfpathlineto{\pgfqpoint{1.908358in}{0.635142in}}%
\pgfpathlineto{\pgfqpoint{1.916420in}{0.635142in}}%
\pgfpathlineto{\pgfqpoint{1.924482in}{0.635142in}}%
\pgfpathlineto{\pgfqpoint{1.932544in}{0.635142in}}%
\pgfpathlineto{\pgfqpoint{1.940605in}{0.635142in}}%
\pgfpathlineto{\pgfqpoint{1.948667in}{0.635142in}}%
\pgfpathlineto{\pgfqpoint{1.956729in}{0.635142in}}%
\pgfpathlineto{\pgfqpoint{1.964791in}{0.635142in}}%
\pgfpathlineto{\pgfqpoint{1.972853in}{0.635142in}}%
\pgfpathlineto{\pgfqpoint{1.980915in}{0.635142in}}%
\pgfpathlineto{\pgfqpoint{1.988977in}{0.635142in}}%
\pgfpathlineto{\pgfqpoint{1.988977in}{0.477955in}}%
\pgfpathlineto{\pgfqpoint{1.988977in}{0.477955in}}%
\pgfpathlineto{\pgfqpoint{1.980915in}{0.477955in}}%
\pgfpathlineto{\pgfqpoint{1.972853in}{0.477955in}}%
\pgfpathlineto{\pgfqpoint{1.964791in}{0.477955in}}%
\pgfpathlineto{\pgfqpoint{1.956729in}{0.477955in}}%
\pgfpathlineto{\pgfqpoint{1.948667in}{0.477955in}}%
\pgfpathlineto{\pgfqpoint{1.940605in}{0.477955in}}%
\pgfpathlineto{\pgfqpoint{1.932544in}{0.477955in}}%
\pgfpathlineto{\pgfqpoint{1.924482in}{0.477955in}}%
\pgfpathlineto{\pgfqpoint{1.916420in}{0.477955in}}%
\pgfpathlineto{\pgfqpoint{1.908358in}{0.477955in}}%
\pgfpathlineto{\pgfqpoint{1.900296in}{0.477955in}}%
\pgfpathlineto{\pgfqpoint{1.892234in}{0.477955in}}%
\pgfpathlineto{\pgfqpoint{1.884172in}{0.477955in}}%
\pgfpathlineto{\pgfqpoint{1.876110in}{0.477955in}}%
\pgfpathlineto{\pgfqpoint{1.868048in}{0.477955in}}%
\pgfpathlineto{\pgfqpoint{1.859986in}{0.477955in}}%
\pgfpathlineto{\pgfqpoint{1.851924in}{0.477955in}}%
\pgfpathlineto{\pgfqpoint{1.843862in}{0.477955in}}%
\pgfpathlineto{\pgfqpoint{1.835800in}{0.477955in}}%
\pgfpathlineto{\pgfqpoint{1.827738in}{0.477955in}}%
\pgfpathlineto{\pgfqpoint{1.819676in}{0.477955in}}%
\pgfpathlineto{\pgfqpoint{1.811614in}{0.477955in}}%
\pgfpathlineto{\pgfqpoint{1.803552in}{0.477955in}}%
\pgfpathlineto{\pgfqpoint{1.795490in}{0.477955in}}%
\pgfpathlineto{\pgfqpoint{1.787428in}{0.477955in}}%
\pgfpathlineto{\pgfqpoint{1.779366in}{0.477955in}}%
\pgfpathlineto{\pgfqpoint{1.771304in}{0.477955in}}%
\pgfpathlineto{\pgfqpoint{1.763242in}{0.477955in}}%
\pgfpathlineto{\pgfqpoint{1.755180in}{0.477955in}}%
\pgfpathlineto{\pgfqpoint{1.747118in}{0.477955in}}%
\pgfpathlineto{\pgfqpoint{1.739056in}{0.477955in}}%
\pgfpathlineto{\pgfqpoint{1.730994in}{0.477955in}}%
\pgfpathlineto{\pgfqpoint{1.722932in}{0.477955in}}%
\pgfpathlineto{\pgfqpoint{1.714871in}{0.477955in}}%
\pgfpathlineto{\pgfqpoint{1.706809in}{0.477955in}}%
\pgfpathlineto{\pgfqpoint{1.698747in}{0.477955in}}%
\pgfpathlineto{\pgfqpoint{1.690685in}{0.477955in}}%
\pgfpathlineto{\pgfqpoint{1.682623in}{0.525133in}}%
\pgfpathlineto{\pgfqpoint{1.674561in}{0.525133in}}%
\pgfpathlineto{\pgfqpoint{1.666499in}{0.525133in}}%
\pgfpathlineto{\pgfqpoint{1.658437in}{0.525133in}}%
\pgfpathlineto{\pgfqpoint{1.650375in}{0.525133in}}%
\pgfpathlineto{\pgfqpoint{1.642313in}{0.525133in}}%
\pgfpathlineto{\pgfqpoint{1.634251in}{0.525133in}}%
\pgfpathlineto{\pgfqpoint{1.626189in}{0.525133in}}%
\pgfpathlineto{\pgfqpoint{1.618127in}{0.525133in}}%
\pgfpathlineto{\pgfqpoint{1.610065in}{0.525133in}}%
\pgfpathlineto{\pgfqpoint{1.602003in}{0.525133in}}%
\pgfpathlineto{\pgfqpoint{1.593941in}{0.525133in}}%
\pgfpathlineto{\pgfqpoint{1.585879in}{0.525133in}}%
\pgfpathlineto{\pgfqpoint{1.577817in}{0.526219in}}%
\pgfpathlineto{\pgfqpoint{1.569755in}{0.526219in}}%
\pgfpathlineto{\pgfqpoint{1.561693in}{0.526219in}}%
\pgfpathlineto{\pgfqpoint{1.553631in}{0.526219in}}%
\pgfpathlineto{\pgfqpoint{1.545569in}{0.526219in}}%
\pgfpathlineto{\pgfqpoint{1.537507in}{0.526219in}}%
\pgfpathlineto{\pgfqpoint{1.529445in}{0.528800in}}%
\pgfpathlineto{\pgfqpoint{1.521383in}{0.528800in}}%
\pgfpathlineto{\pgfqpoint{1.513321in}{0.528800in}}%
\pgfpathlineto{\pgfqpoint{1.505259in}{0.528800in}}%
\pgfpathlineto{\pgfqpoint{1.497198in}{0.528800in}}%
\pgfpathlineto{\pgfqpoint{1.489136in}{0.528800in}}%
\pgfpathlineto{\pgfqpoint{1.481074in}{0.528800in}}%
\pgfpathlineto{\pgfqpoint{1.473012in}{0.528800in}}%
\pgfpathlineto{\pgfqpoint{1.464950in}{0.528800in}}%
\pgfpathlineto{\pgfqpoint{1.456888in}{0.528800in}}%
\pgfpathlineto{\pgfqpoint{1.448826in}{0.528800in}}%
\pgfpathlineto{\pgfqpoint{1.440764in}{0.528800in}}%
\pgfpathlineto{\pgfqpoint{1.432702in}{0.528800in}}%
\pgfpathlineto{\pgfqpoint{1.424640in}{0.528800in}}%
\pgfpathlineto{\pgfqpoint{1.416578in}{0.558678in}}%
\pgfpathlineto{\pgfqpoint{1.408516in}{0.570290in}}%
\pgfpathlineto{\pgfqpoint{1.400454in}{0.674389in}}%
\pgfpathlineto{\pgfqpoint{1.392392in}{0.674389in}}%
\pgfpathlineto{\pgfqpoint{1.384330in}{0.674389in}}%
\pgfpathlineto{\pgfqpoint{1.376268in}{0.674389in}}%
\pgfpathlineto{\pgfqpoint{1.368206in}{0.674389in}}%
\pgfpathlineto{\pgfqpoint{1.360144in}{0.674389in}}%
\pgfpathlineto{\pgfqpoint{1.352082in}{0.674389in}}%
\pgfpathlineto{\pgfqpoint{1.344020in}{0.674389in}}%
\pgfpathlineto{\pgfqpoint{1.335958in}{0.674389in}}%
\pgfpathlineto{\pgfqpoint{1.327896in}{0.674389in}}%
\pgfpathlineto{\pgfqpoint{1.319834in}{0.674389in}}%
\pgfpathlineto{\pgfqpoint{1.311772in}{0.674389in}}%
\pgfpathlineto{\pgfqpoint{1.303710in}{0.674389in}}%
\pgfpathlineto{\pgfqpoint{1.295648in}{0.674389in}}%
\pgfpathlineto{\pgfqpoint{1.287586in}{0.674389in}}%
\pgfpathlineto{\pgfqpoint{1.279525in}{0.694514in}}%
\pgfpathlineto{\pgfqpoint{1.271463in}{0.694514in}}%
\pgfpathlineto{\pgfqpoint{1.263401in}{0.694514in}}%
\pgfpathlineto{\pgfqpoint{1.255339in}{0.694514in}}%
\pgfpathlineto{\pgfqpoint{1.247277in}{0.694514in}}%
\pgfpathlineto{\pgfqpoint{1.239215in}{0.694514in}}%
\pgfpathlineto{\pgfqpoint{1.231153in}{0.694514in}}%
\pgfpathlineto{\pgfqpoint{1.223091in}{0.694514in}}%
\pgfpathlineto{\pgfqpoint{1.215029in}{0.694514in}}%
\pgfpathlineto{\pgfqpoint{1.206967in}{0.694514in}}%
\pgfpathlineto{\pgfqpoint{1.198905in}{0.694514in}}%
\pgfpathlineto{\pgfqpoint{1.190843in}{0.694514in}}%
\pgfpathlineto{\pgfqpoint{1.182781in}{0.694514in}}%
\pgfpathlineto{\pgfqpoint{1.174719in}{0.694514in}}%
\pgfpathlineto{\pgfqpoint{1.166657in}{0.694514in}}%
\pgfpathlineto{\pgfqpoint{1.158595in}{0.694514in}}%
\pgfpathlineto{\pgfqpoint{1.150533in}{0.694514in}}%
\pgfpathlineto{\pgfqpoint{1.142471in}{0.694514in}}%
\pgfpathlineto{\pgfqpoint{1.134409in}{0.694514in}}%
\pgfpathlineto{\pgfqpoint{1.126347in}{0.694514in}}%
\pgfpathlineto{\pgfqpoint{1.118285in}{0.694514in}}%
\pgfpathlineto{\pgfqpoint{1.110223in}{0.694514in}}%
\pgfpathlineto{\pgfqpoint{1.102161in}{0.719611in}}%
\pgfpathlineto{\pgfqpoint{1.094099in}{0.731883in}}%
\pgfpathlineto{\pgfqpoint{1.086037in}{0.731883in}}%
\pgfpathlineto{\pgfqpoint{1.077975in}{0.731883in}}%
\pgfpathlineto{\pgfqpoint{1.069914in}{0.793660in}}%
\pgfpathlineto{\pgfqpoint{1.061852in}{0.793660in}}%
\pgfpathlineto{\pgfqpoint{1.053790in}{0.793660in}}%
\pgfpathlineto{\pgfqpoint{1.045728in}{0.793660in}}%
\pgfpathlineto{\pgfqpoint{1.037666in}{0.793660in}}%
\pgfpathlineto{\pgfqpoint{1.029604in}{0.793660in}}%
\pgfpathlineto{\pgfqpoint{1.021542in}{0.793660in}}%
\pgfpathlineto{\pgfqpoint{1.013480in}{0.793660in}}%
\pgfpathlineto{\pgfqpoint{1.005418in}{0.793660in}}%
\pgfpathlineto{\pgfqpoint{0.997356in}{0.793660in}}%
\pgfpathlineto{\pgfqpoint{0.989294in}{0.793660in}}%
\pgfpathlineto{\pgfqpoint{0.981232in}{0.793660in}}%
\pgfpathlineto{\pgfqpoint{0.973170in}{0.793660in}}%
\pgfpathlineto{\pgfqpoint{0.965108in}{0.793660in}}%
\pgfpathlineto{\pgfqpoint{0.957046in}{0.815332in}}%
\pgfpathlineto{\pgfqpoint{0.948984in}{0.815332in}}%
\pgfpathlineto{\pgfqpoint{0.940922in}{0.815332in}}%
\pgfpathlineto{\pgfqpoint{0.932860in}{0.815332in}}%
\pgfpathlineto{\pgfqpoint{0.924798in}{0.815332in}}%
\pgfpathlineto{\pgfqpoint{0.916736in}{0.815332in}}%
\pgfpathlineto{\pgfqpoint{0.908674in}{0.815332in}}%
\pgfpathlineto{\pgfqpoint{0.900612in}{0.998405in}}%
\pgfpathlineto{\pgfqpoint{0.892550in}{0.998405in}}%
\pgfpathlineto{\pgfqpoint{0.884488in}{0.998405in}}%
\pgfpathlineto{\pgfqpoint{0.876426in}{1.314272in}}%
\pgfpathlineto{\pgfqpoint{0.868364in}{1.314272in}}%
\pgfpathlineto{\pgfqpoint{0.860302in}{1.323915in}}%
\pgfpathlineto{\pgfqpoint{0.852241in}{1.323915in}}%
\pgfpathlineto{\pgfqpoint{0.844179in}{1.394778in}}%
\pgfpathlineto{\pgfqpoint{0.836117in}{1.394778in}}%
\pgfpathlineto{\pgfqpoint{0.828055in}{1.487977in}}%
\pgfpathlineto{\pgfqpoint{0.819993in}{1.487977in}}%
\pgfpathlineto{\pgfqpoint{0.811931in}{1.601817in}}%
\pgfpathlineto{\pgfqpoint{0.803869in}{1.601817in}}%
\pgfpathlineto{\pgfqpoint{0.795807in}{1.682494in}}%
\pgfpathlineto{\pgfqpoint{0.787745in}{1.682494in}}%
\pgfpathlineto{\pgfqpoint{0.779683in}{1.691728in}}%
\pgfpathlineto{\pgfqpoint{0.771621in}{1.691728in}}%
\pgfpathlineto{\pgfqpoint{0.763559in}{1.691728in}}%
\pgfpathlineto{\pgfqpoint{0.755497in}{1.691728in}}%
\pgfpathlineto{\pgfqpoint{0.747435in}{1.691728in}}%
\pgfpathlineto{\pgfqpoint{0.739373in}{1.691728in}}%
\pgfpathlineto{\pgfqpoint{0.731311in}{1.691728in}}%
\pgfpathlineto{\pgfqpoint{0.723249in}{1.691728in}}%
\pgfpathlineto{\pgfqpoint{0.715187in}{1.691728in}}%
\pgfpathlineto{\pgfqpoint{0.707125in}{1.837514in}}%
\pgfpathlineto{\pgfqpoint{0.699063in}{1.837514in}}%
\pgfpathlineto{\pgfqpoint{0.691001in}{1.920787in}}%
\pgfpathlineto{\pgfqpoint{0.682939in}{1.920787in}}%
\pgfpathlineto{\pgfqpoint{0.674877in}{1.920787in}}%
\pgfpathlineto{\pgfqpoint{0.666815in}{1.951696in}}%
\pgfpathlineto{\pgfqpoint{0.658753in}{1.951696in}}%
\pgfpathlineto{\pgfqpoint{0.650691in}{1.951696in}}%
\pgfpathlineto{\pgfqpoint{0.642629in}{1.951696in}}%
\pgfpathlineto{\pgfqpoint{0.634568in}{1.951696in}}%
\pgfpathlineto{\pgfqpoint{0.626506in}{1.951696in}}%
\pgfpathlineto{\pgfqpoint{0.618444in}{1.951696in}}%
\pgfpathlineto{\pgfqpoint{0.610382in}{1.951696in}}%
\pgfpathlineto{\pgfqpoint{0.602320in}{1.951696in}}%
\pgfpathlineto{\pgfqpoint{0.594258in}{1.951696in}}%
\pgfpathlineto{\pgfqpoint{0.586196in}{1.951696in}}%
\pgfpathlineto{\pgfqpoint{0.578134in}{1.951696in}}%
\pgfpathlineto{\pgfqpoint{0.570072in}{1.951696in}}%
\pgfpathlineto{\pgfqpoint{0.562010in}{1.951696in}}%
\pgfpathlineto{\pgfqpoint{0.553948in}{1.981480in}}%
\pgfpathlineto{\pgfqpoint{0.545886in}{1.981480in}}%
\pgfpathlineto{\pgfqpoint{0.537824in}{1.981480in}}%
\pgfpathlineto{\pgfqpoint{0.529762in}{1.981480in}}%
\pgfpathlineto{\pgfqpoint{0.521700in}{1.981480in}}%
\pgfpathlineto{\pgfqpoint{0.513638in}{2.060582in}}%
\pgfpathlineto{\pgfqpoint{0.505576in}{2.060582in}}%
\pgfpathlineto{\pgfqpoint{0.497514in}{2.060582in}}%
\pgfpathlineto{\pgfqpoint{0.489452in}{2.060582in}}%
\pgfpathlineto{\pgfqpoint{0.481390in}{2.090823in}}%
\pgfpathlineto{\pgfqpoint{0.473328in}{2.090823in}}%
\pgfpathlineto{\pgfqpoint{0.465266in}{2.143517in}}%
\pgfpathlineto{\pgfqpoint{0.457204in}{2.220226in}}%
\pgfpathlineto{\pgfqpoint{0.449142in}{2.230211in}}%
\pgfpathlineto{\pgfqpoint{0.441080in}{2.266386in}}%
\pgfpathlineto{\pgfqpoint{0.433018in}{2.266386in}}%
\pgfpathlineto{\pgfqpoint{0.424956in}{2.266386in}}%
\pgfpathlineto{\pgfqpoint{0.416895in}{2.266386in}}%
\pgfpathlineto{\pgfqpoint{0.408833in}{2.266386in}}%
\pgfpathlineto{\pgfqpoint{0.400771in}{2.266386in}}%
\pgfpathlineto{\pgfqpoint{0.392709in}{2.266386in}}%
\pgfpathlineto{\pgfqpoint{0.384647in}{2.266386in}}%
\pgfpathlineto{\pgfqpoint{0.376585in}{2.272568in}}%
\pgfpathlineto{\pgfqpoint{0.368523in}{2.430219in}}%
\pgfpathclose%
\pgfusepath{fill}%
\end{pgfscope}%
\begin{pgfscope}%
\pgfpathrectangle{\pgfqpoint{0.287500in}{0.375000in}}{\pgfqpoint{1.782500in}{2.265000in}}%
\pgfusepath{clip}%
\pgfsetbuttcap%
\pgfsetroundjoin%
\definecolor{currentfill}{rgb}{0.172549,0.627451,0.172549}%
\pgfsetfillcolor{currentfill}%
\pgfsetfillopacity{0.200000}%
\pgfsetlinewidth{0.000000pt}%
\definecolor{currentstroke}{rgb}{0.000000,0.000000,0.000000}%
\pgfsetstrokecolor{currentstroke}%
\pgfsetdash{}{0pt}%
\pgfpathmoveto{\pgfqpoint{0.368523in}{2.472209in}}%
\pgfpathlineto{\pgfqpoint{0.368523in}{2.530239in}}%
\pgfpathlineto{\pgfqpoint{0.376585in}{2.462937in}}%
\pgfpathlineto{\pgfqpoint{0.384647in}{2.450173in}}%
\pgfpathlineto{\pgfqpoint{0.392709in}{2.426483in}}%
\pgfpathlineto{\pgfqpoint{0.400771in}{2.382965in}}%
\pgfpathlineto{\pgfqpoint{0.408833in}{2.382965in}}%
\pgfpathlineto{\pgfqpoint{0.416895in}{2.382965in}}%
\pgfpathlineto{\pgfqpoint{0.424956in}{2.382965in}}%
\pgfpathlineto{\pgfqpoint{0.433018in}{2.382965in}}%
\pgfpathlineto{\pgfqpoint{0.441080in}{2.382965in}}%
\pgfpathlineto{\pgfqpoint{0.449142in}{2.360993in}}%
\pgfpathlineto{\pgfqpoint{0.457204in}{2.360993in}}%
\pgfpathlineto{\pgfqpoint{0.465266in}{2.360993in}}%
\pgfpathlineto{\pgfqpoint{0.473328in}{2.344557in}}%
\pgfpathlineto{\pgfqpoint{0.481390in}{2.344557in}}%
\pgfpathlineto{\pgfqpoint{0.489452in}{2.344557in}}%
\pgfpathlineto{\pgfqpoint{0.497514in}{2.288542in}}%
\pgfpathlineto{\pgfqpoint{0.505576in}{2.288542in}}%
\pgfpathlineto{\pgfqpoint{0.513638in}{2.288542in}}%
\pgfpathlineto{\pgfqpoint{0.521700in}{2.288291in}}%
\pgfpathlineto{\pgfqpoint{0.529762in}{2.288291in}}%
\pgfpathlineto{\pgfqpoint{0.537824in}{2.288291in}}%
\pgfpathlineto{\pgfqpoint{0.545886in}{2.288291in}}%
\pgfpathlineto{\pgfqpoint{0.553948in}{2.275071in}}%
\pgfpathlineto{\pgfqpoint{0.562010in}{2.270024in}}%
\pgfpathlineto{\pgfqpoint{0.570072in}{2.225910in}}%
\pgfpathlineto{\pgfqpoint{0.578134in}{2.225910in}}%
\pgfpathlineto{\pgfqpoint{0.586196in}{2.225910in}}%
\pgfpathlineto{\pgfqpoint{0.594258in}{2.218577in}}%
\pgfpathlineto{\pgfqpoint{0.602320in}{2.169920in}}%
\pgfpathlineto{\pgfqpoint{0.610382in}{1.943234in}}%
\pgfpathlineto{\pgfqpoint{0.618444in}{1.943234in}}%
\pgfpathlineto{\pgfqpoint{0.626506in}{1.943234in}}%
\pgfpathlineto{\pgfqpoint{0.634568in}{1.943234in}}%
\pgfpathlineto{\pgfqpoint{0.642629in}{1.943234in}}%
\pgfpathlineto{\pgfqpoint{0.650691in}{1.943234in}}%
\pgfpathlineto{\pgfqpoint{0.658753in}{1.943234in}}%
\pgfpathlineto{\pgfqpoint{0.666815in}{1.943234in}}%
\pgfpathlineto{\pgfqpoint{0.674877in}{1.943234in}}%
\pgfpathlineto{\pgfqpoint{0.682939in}{1.943234in}}%
\pgfpathlineto{\pgfqpoint{0.691001in}{1.943234in}}%
\pgfpathlineto{\pgfqpoint{0.699063in}{1.943234in}}%
\pgfpathlineto{\pgfqpoint{0.707125in}{1.846696in}}%
\pgfpathlineto{\pgfqpoint{0.715187in}{1.846696in}}%
\pgfpathlineto{\pgfqpoint{0.723249in}{1.846696in}}%
\pgfpathlineto{\pgfqpoint{0.731311in}{1.846696in}}%
\pgfpathlineto{\pgfqpoint{0.739373in}{1.846696in}}%
\pgfpathlineto{\pgfqpoint{0.747435in}{1.846696in}}%
\pgfpathlineto{\pgfqpoint{0.755497in}{1.846351in}}%
\pgfpathlineto{\pgfqpoint{0.763559in}{1.846351in}}%
\pgfpathlineto{\pgfqpoint{0.771621in}{1.846351in}}%
\pgfpathlineto{\pgfqpoint{0.779683in}{1.846351in}}%
\pgfpathlineto{\pgfqpoint{0.787745in}{1.846351in}}%
\pgfpathlineto{\pgfqpoint{0.795807in}{1.846351in}}%
\pgfpathlineto{\pgfqpoint{0.803869in}{1.846351in}}%
\pgfpathlineto{\pgfqpoint{0.811931in}{1.846351in}}%
\pgfpathlineto{\pgfqpoint{0.819993in}{1.716282in}}%
\pgfpathlineto{\pgfqpoint{0.828055in}{1.716282in}}%
\pgfpathlineto{\pgfqpoint{0.836117in}{1.716282in}}%
\pgfpathlineto{\pgfqpoint{0.844179in}{1.711679in}}%
\pgfpathlineto{\pgfqpoint{0.852241in}{1.711679in}}%
\pgfpathlineto{\pgfqpoint{0.860302in}{1.711679in}}%
\pgfpathlineto{\pgfqpoint{0.868364in}{1.711679in}}%
\pgfpathlineto{\pgfqpoint{0.876426in}{1.711679in}}%
\pgfpathlineto{\pgfqpoint{0.884488in}{1.685262in}}%
\pgfpathlineto{\pgfqpoint{0.892550in}{1.685262in}}%
\pgfpathlineto{\pgfqpoint{0.900612in}{1.602665in}}%
\pgfpathlineto{\pgfqpoint{0.908674in}{1.602665in}}%
\pgfpathlineto{\pgfqpoint{0.916736in}{1.602665in}}%
\pgfpathlineto{\pgfqpoint{0.924798in}{1.542029in}}%
\pgfpathlineto{\pgfqpoint{0.932860in}{1.542029in}}%
\pgfpathlineto{\pgfqpoint{0.940922in}{1.542029in}}%
\pgfpathlineto{\pgfqpoint{0.948984in}{1.542029in}}%
\pgfpathlineto{\pgfqpoint{0.957046in}{1.542029in}}%
\pgfpathlineto{\pgfqpoint{0.965108in}{1.542029in}}%
\pgfpathlineto{\pgfqpoint{0.973170in}{1.537414in}}%
\pgfpathlineto{\pgfqpoint{0.981232in}{1.537414in}}%
\pgfpathlineto{\pgfqpoint{0.989294in}{1.537414in}}%
\pgfpathlineto{\pgfqpoint{0.997356in}{1.399268in}}%
\pgfpathlineto{\pgfqpoint{1.005418in}{1.399268in}}%
\pgfpathlineto{\pgfqpoint{1.013480in}{1.399268in}}%
\pgfpathlineto{\pgfqpoint{1.021542in}{1.399268in}}%
\pgfpathlineto{\pgfqpoint{1.029604in}{1.395444in}}%
\pgfpathlineto{\pgfqpoint{1.037666in}{1.395444in}}%
\pgfpathlineto{\pgfqpoint{1.045728in}{1.395444in}}%
\pgfpathlineto{\pgfqpoint{1.053790in}{1.330480in}}%
\pgfpathlineto{\pgfqpoint{1.061852in}{1.325601in}}%
\pgfpathlineto{\pgfqpoint{1.069914in}{1.289451in}}%
\pgfpathlineto{\pgfqpoint{1.077975in}{1.289451in}}%
\pgfpathlineto{\pgfqpoint{1.086037in}{1.289451in}}%
\pgfpathlineto{\pgfqpoint{1.094099in}{1.289451in}}%
\pgfpathlineto{\pgfqpoint{1.102161in}{1.289451in}}%
\pgfpathlineto{\pgfqpoint{1.110223in}{1.289451in}}%
\pgfpathlineto{\pgfqpoint{1.118285in}{1.289451in}}%
\pgfpathlineto{\pgfqpoint{1.126347in}{1.216084in}}%
\pgfpathlineto{\pgfqpoint{1.134409in}{1.211264in}}%
\pgfpathlineto{\pgfqpoint{1.142471in}{1.211264in}}%
\pgfpathlineto{\pgfqpoint{1.150533in}{1.211264in}}%
\pgfpathlineto{\pgfqpoint{1.158595in}{1.211264in}}%
\pgfpathlineto{\pgfqpoint{1.166657in}{1.211264in}}%
\pgfpathlineto{\pgfqpoint{1.174719in}{1.211264in}}%
\pgfpathlineto{\pgfqpoint{1.182781in}{1.211264in}}%
\pgfpathlineto{\pgfqpoint{1.190843in}{1.211264in}}%
\pgfpathlineto{\pgfqpoint{1.198905in}{1.211264in}}%
\pgfpathlineto{\pgfqpoint{1.206967in}{1.211264in}}%
\pgfpathlineto{\pgfqpoint{1.215029in}{1.211264in}}%
\pgfpathlineto{\pgfqpoint{1.223091in}{1.211264in}}%
\pgfpathlineto{\pgfqpoint{1.231153in}{1.211264in}}%
\pgfpathlineto{\pgfqpoint{1.239215in}{1.211264in}}%
\pgfpathlineto{\pgfqpoint{1.247277in}{1.211264in}}%
\pgfpathlineto{\pgfqpoint{1.255339in}{1.211264in}}%
\pgfpathlineto{\pgfqpoint{1.263401in}{1.211264in}}%
\pgfpathlineto{\pgfqpoint{1.271463in}{1.211264in}}%
\pgfpathlineto{\pgfqpoint{1.279525in}{1.211264in}}%
\pgfpathlineto{\pgfqpoint{1.287586in}{1.130363in}}%
\pgfpathlineto{\pgfqpoint{1.295648in}{1.130363in}}%
\pgfpathlineto{\pgfqpoint{1.303710in}{1.130363in}}%
\pgfpathlineto{\pgfqpoint{1.311772in}{1.130363in}}%
\pgfpathlineto{\pgfqpoint{1.319834in}{1.130363in}}%
\pgfpathlineto{\pgfqpoint{1.327896in}{1.130363in}}%
\pgfpathlineto{\pgfqpoint{1.335958in}{1.130363in}}%
\pgfpathlineto{\pgfqpoint{1.344020in}{1.085108in}}%
\pgfpathlineto{\pgfqpoint{1.352082in}{1.085108in}}%
\pgfpathlineto{\pgfqpoint{1.360144in}{1.085071in}}%
\pgfpathlineto{\pgfqpoint{1.368206in}{1.085071in}}%
\pgfpathlineto{\pgfqpoint{1.376268in}{1.085071in}}%
\pgfpathlineto{\pgfqpoint{1.384330in}{1.085071in}}%
\pgfpathlineto{\pgfqpoint{1.392392in}{1.085071in}}%
\pgfpathlineto{\pgfqpoint{1.400454in}{1.006282in}}%
\pgfpathlineto{\pgfqpoint{1.408516in}{1.006282in}}%
\pgfpathlineto{\pgfqpoint{1.416578in}{1.005922in}}%
\pgfpathlineto{\pgfqpoint{1.424640in}{1.005922in}}%
\pgfpathlineto{\pgfqpoint{1.432702in}{1.005922in}}%
\pgfpathlineto{\pgfqpoint{1.440764in}{1.005922in}}%
\pgfpathlineto{\pgfqpoint{1.448826in}{1.005922in}}%
\pgfpathlineto{\pgfqpoint{1.456888in}{1.005922in}}%
\pgfpathlineto{\pgfqpoint{1.464950in}{1.005922in}}%
\pgfpathlineto{\pgfqpoint{1.473012in}{1.005922in}}%
\pgfpathlineto{\pgfqpoint{1.481074in}{1.005922in}}%
\pgfpathlineto{\pgfqpoint{1.489136in}{0.998583in}}%
\pgfpathlineto{\pgfqpoint{1.497198in}{0.998583in}}%
\pgfpathlineto{\pgfqpoint{1.505259in}{0.998583in}}%
\pgfpathlineto{\pgfqpoint{1.513321in}{0.998583in}}%
\pgfpathlineto{\pgfqpoint{1.521383in}{0.928149in}}%
\pgfpathlineto{\pgfqpoint{1.529445in}{0.928149in}}%
\pgfpathlineto{\pgfqpoint{1.537507in}{0.928149in}}%
\pgfpathlineto{\pgfqpoint{1.545569in}{0.918781in}}%
\pgfpathlineto{\pgfqpoint{1.553631in}{0.870062in}}%
\pgfpathlineto{\pgfqpoint{1.561693in}{0.870062in}}%
\pgfpathlineto{\pgfqpoint{1.569755in}{0.870062in}}%
\pgfpathlineto{\pgfqpoint{1.577817in}{0.870062in}}%
\pgfpathlineto{\pgfqpoint{1.585879in}{0.870062in}}%
\pgfpathlineto{\pgfqpoint{1.593941in}{0.870062in}}%
\pgfpathlineto{\pgfqpoint{1.602003in}{0.870062in}}%
\pgfpathlineto{\pgfqpoint{1.610065in}{0.870062in}}%
\pgfpathlineto{\pgfqpoint{1.618127in}{0.870062in}}%
\pgfpathlineto{\pgfqpoint{1.626189in}{0.870062in}}%
\pgfpathlineto{\pgfqpoint{1.634251in}{0.870062in}}%
\pgfpathlineto{\pgfqpoint{1.642313in}{0.870062in}}%
\pgfpathlineto{\pgfqpoint{1.650375in}{0.870062in}}%
\pgfpathlineto{\pgfqpoint{1.658437in}{0.870062in}}%
\pgfpathlineto{\pgfqpoint{1.666499in}{0.870062in}}%
\pgfpathlineto{\pgfqpoint{1.674561in}{0.870062in}}%
\pgfpathlineto{\pgfqpoint{1.682623in}{0.870062in}}%
\pgfpathlineto{\pgfqpoint{1.690685in}{0.870062in}}%
\pgfpathlineto{\pgfqpoint{1.698747in}{0.870062in}}%
\pgfpathlineto{\pgfqpoint{1.706809in}{0.870062in}}%
\pgfpathlineto{\pgfqpoint{1.714871in}{0.870062in}}%
\pgfpathlineto{\pgfqpoint{1.722932in}{0.870062in}}%
\pgfpathlineto{\pgfqpoint{1.730994in}{0.870062in}}%
\pgfpathlineto{\pgfqpoint{1.739056in}{0.870062in}}%
\pgfpathlineto{\pgfqpoint{1.747118in}{0.870062in}}%
\pgfpathlineto{\pgfqpoint{1.755180in}{0.870062in}}%
\pgfpathlineto{\pgfqpoint{1.763242in}{0.870062in}}%
\pgfpathlineto{\pgfqpoint{1.771304in}{0.870062in}}%
\pgfpathlineto{\pgfqpoint{1.779366in}{0.870062in}}%
\pgfpathlineto{\pgfqpoint{1.787428in}{0.870062in}}%
\pgfpathlineto{\pgfqpoint{1.795490in}{0.870062in}}%
\pgfpathlineto{\pgfqpoint{1.803552in}{0.870062in}}%
\pgfpathlineto{\pgfqpoint{1.811614in}{0.870062in}}%
\pgfpathlineto{\pgfqpoint{1.819676in}{0.870062in}}%
\pgfpathlineto{\pgfqpoint{1.827738in}{0.865977in}}%
\pgfpathlineto{\pgfqpoint{1.835800in}{0.865977in}}%
\pgfpathlineto{\pgfqpoint{1.843862in}{0.865977in}}%
\pgfpathlineto{\pgfqpoint{1.851924in}{0.865977in}}%
\pgfpathlineto{\pgfqpoint{1.859986in}{0.865977in}}%
\pgfpathlineto{\pgfqpoint{1.868048in}{0.865977in}}%
\pgfpathlineto{\pgfqpoint{1.876110in}{0.865977in}}%
\pgfpathlineto{\pgfqpoint{1.884172in}{0.865977in}}%
\pgfpathlineto{\pgfqpoint{1.892234in}{0.865977in}}%
\pgfpathlineto{\pgfqpoint{1.900296in}{0.865880in}}%
\pgfpathlineto{\pgfqpoint{1.908358in}{0.865880in}}%
\pgfpathlineto{\pgfqpoint{1.916420in}{0.865880in}}%
\pgfpathlineto{\pgfqpoint{1.924482in}{0.865880in}}%
\pgfpathlineto{\pgfqpoint{1.932544in}{0.865880in}}%
\pgfpathlineto{\pgfqpoint{1.940605in}{0.865880in}}%
\pgfpathlineto{\pgfqpoint{1.948667in}{0.865880in}}%
\pgfpathlineto{\pgfqpoint{1.956729in}{0.865880in}}%
\pgfpathlineto{\pgfqpoint{1.964791in}{0.865880in}}%
\pgfpathlineto{\pgfqpoint{1.972853in}{0.865880in}}%
\pgfpathlineto{\pgfqpoint{1.980915in}{0.865880in}}%
\pgfpathlineto{\pgfqpoint{1.988977in}{0.865880in}}%
\pgfpathlineto{\pgfqpoint{1.988977in}{0.648316in}}%
\pgfpathlineto{\pgfqpoint{1.988977in}{0.648316in}}%
\pgfpathlineto{\pgfqpoint{1.980915in}{0.648316in}}%
\pgfpathlineto{\pgfqpoint{1.972853in}{0.648316in}}%
\pgfpathlineto{\pgfqpoint{1.964791in}{0.648316in}}%
\pgfpathlineto{\pgfqpoint{1.956729in}{0.648316in}}%
\pgfpathlineto{\pgfqpoint{1.948667in}{0.648316in}}%
\pgfpathlineto{\pgfqpoint{1.940605in}{0.648316in}}%
\pgfpathlineto{\pgfqpoint{1.932544in}{0.648316in}}%
\pgfpathlineto{\pgfqpoint{1.924482in}{0.648316in}}%
\pgfpathlineto{\pgfqpoint{1.916420in}{0.648316in}}%
\pgfpathlineto{\pgfqpoint{1.908358in}{0.648316in}}%
\pgfpathlineto{\pgfqpoint{1.900296in}{0.648316in}}%
\pgfpathlineto{\pgfqpoint{1.892234in}{0.649237in}}%
\pgfpathlineto{\pgfqpoint{1.884172in}{0.649237in}}%
\pgfpathlineto{\pgfqpoint{1.876110in}{0.649237in}}%
\pgfpathlineto{\pgfqpoint{1.868048in}{0.649237in}}%
\pgfpathlineto{\pgfqpoint{1.859986in}{0.649237in}}%
\pgfpathlineto{\pgfqpoint{1.851924in}{0.649237in}}%
\pgfpathlineto{\pgfqpoint{1.843862in}{0.649237in}}%
\pgfpathlineto{\pgfqpoint{1.835800in}{0.649237in}}%
\pgfpathlineto{\pgfqpoint{1.827738in}{0.649237in}}%
\pgfpathlineto{\pgfqpoint{1.819676in}{0.680446in}}%
\pgfpathlineto{\pgfqpoint{1.811614in}{0.680446in}}%
\pgfpathlineto{\pgfqpoint{1.803552in}{0.680446in}}%
\pgfpathlineto{\pgfqpoint{1.795490in}{0.680446in}}%
\pgfpathlineto{\pgfqpoint{1.787428in}{0.680446in}}%
\pgfpathlineto{\pgfqpoint{1.779366in}{0.680446in}}%
\pgfpathlineto{\pgfqpoint{1.771304in}{0.680446in}}%
\pgfpathlineto{\pgfqpoint{1.763242in}{0.680446in}}%
\pgfpathlineto{\pgfqpoint{1.755180in}{0.680446in}}%
\pgfpathlineto{\pgfqpoint{1.747118in}{0.680446in}}%
\pgfpathlineto{\pgfqpoint{1.739056in}{0.680446in}}%
\pgfpathlineto{\pgfqpoint{1.730994in}{0.680446in}}%
\pgfpathlineto{\pgfqpoint{1.722932in}{0.680446in}}%
\pgfpathlineto{\pgfqpoint{1.714871in}{0.680446in}}%
\pgfpathlineto{\pgfqpoint{1.706809in}{0.680446in}}%
\pgfpathlineto{\pgfqpoint{1.698747in}{0.680446in}}%
\pgfpathlineto{\pgfqpoint{1.690685in}{0.680446in}}%
\pgfpathlineto{\pgfqpoint{1.682623in}{0.680446in}}%
\pgfpathlineto{\pgfqpoint{1.674561in}{0.680446in}}%
\pgfpathlineto{\pgfqpoint{1.666499in}{0.680446in}}%
\pgfpathlineto{\pgfqpoint{1.658437in}{0.680446in}}%
\pgfpathlineto{\pgfqpoint{1.650375in}{0.680446in}}%
\pgfpathlineto{\pgfqpoint{1.642313in}{0.680446in}}%
\pgfpathlineto{\pgfqpoint{1.634251in}{0.680446in}}%
\pgfpathlineto{\pgfqpoint{1.626189in}{0.680446in}}%
\pgfpathlineto{\pgfqpoint{1.618127in}{0.680446in}}%
\pgfpathlineto{\pgfqpoint{1.610065in}{0.680446in}}%
\pgfpathlineto{\pgfqpoint{1.602003in}{0.680446in}}%
\pgfpathlineto{\pgfqpoint{1.593941in}{0.680446in}}%
\pgfpathlineto{\pgfqpoint{1.585879in}{0.680446in}}%
\pgfpathlineto{\pgfqpoint{1.577817in}{0.680446in}}%
\pgfpathlineto{\pgfqpoint{1.569755in}{0.680446in}}%
\pgfpathlineto{\pgfqpoint{1.561693in}{0.680446in}}%
\pgfpathlineto{\pgfqpoint{1.553631in}{0.680446in}}%
\pgfpathlineto{\pgfqpoint{1.545569in}{0.786458in}}%
\pgfpathlineto{\pgfqpoint{1.537507in}{0.791885in}}%
\pgfpathlineto{\pgfqpoint{1.529445in}{0.791885in}}%
\pgfpathlineto{\pgfqpoint{1.521383in}{0.791885in}}%
\pgfpathlineto{\pgfqpoint{1.513321in}{0.882655in}}%
\pgfpathlineto{\pgfqpoint{1.505259in}{0.882655in}}%
\pgfpathlineto{\pgfqpoint{1.497198in}{0.882655in}}%
\pgfpathlineto{\pgfqpoint{1.489136in}{0.882655in}}%
\pgfpathlineto{\pgfqpoint{1.481074in}{0.914193in}}%
\pgfpathlineto{\pgfqpoint{1.473012in}{0.914193in}}%
\pgfpathlineto{\pgfqpoint{1.464950in}{0.914193in}}%
\pgfpathlineto{\pgfqpoint{1.456888in}{0.914193in}}%
\pgfpathlineto{\pgfqpoint{1.448826in}{0.914193in}}%
\pgfpathlineto{\pgfqpoint{1.440764in}{0.914193in}}%
\pgfpathlineto{\pgfqpoint{1.432702in}{0.914193in}}%
\pgfpathlineto{\pgfqpoint{1.424640in}{0.914193in}}%
\pgfpathlineto{\pgfqpoint{1.416578in}{0.914193in}}%
\pgfpathlineto{\pgfqpoint{1.408516in}{0.914677in}}%
\pgfpathlineto{\pgfqpoint{1.400454in}{0.914677in}}%
\pgfpathlineto{\pgfqpoint{1.392392in}{0.998040in}}%
\pgfpathlineto{\pgfqpoint{1.384330in}{0.998040in}}%
\pgfpathlineto{\pgfqpoint{1.376268in}{0.998040in}}%
\pgfpathlineto{\pgfqpoint{1.368206in}{0.998040in}}%
\pgfpathlineto{\pgfqpoint{1.360144in}{0.998040in}}%
\pgfpathlineto{\pgfqpoint{1.352082in}{0.998226in}}%
\pgfpathlineto{\pgfqpoint{1.344020in}{0.998226in}}%
\pgfpathlineto{\pgfqpoint{1.335958in}{1.005182in}}%
\pgfpathlineto{\pgfqpoint{1.327896in}{1.005182in}}%
\pgfpathlineto{\pgfqpoint{1.319834in}{1.005182in}}%
\pgfpathlineto{\pgfqpoint{1.311772in}{1.005182in}}%
\pgfpathlineto{\pgfqpoint{1.303710in}{1.005182in}}%
\pgfpathlineto{\pgfqpoint{1.295648in}{1.005182in}}%
\pgfpathlineto{\pgfqpoint{1.287586in}{1.005182in}}%
\pgfpathlineto{\pgfqpoint{1.279525in}{1.074293in}}%
\pgfpathlineto{\pgfqpoint{1.271463in}{1.074293in}}%
\pgfpathlineto{\pgfqpoint{1.263401in}{1.074293in}}%
\pgfpathlineto{\pgfqpoint{1.255339in}{1.074293in}}%
\pgfpathlineto{\pgfqpoint{1.247277in}{1.074293in}}%
\pgfpathlineto{\pgfqpoint{1.239215in}{1.074293in}}%
\pgfpathlineto{\pgfqpoint{1.231153in}{1.074293in}}%
\pgfpathlineto{\pgfqpoint{1.223091in}{1.074293in}}%
\pgfpathlineto{\pgfqpoint{1.215029in}{1.074293in}}%
\pgfpathlineto{\pgfqpoint{1.206967in}{1.074293in}}%
\pgfpathlineto{\pgfqpoint{1.198905in}{1.074293in}}%
\pgfpathlineto{\pgfqpoint{1.190843in}{1.074293in}}%
\pgfpathlineto{\pgfqpoint{1.182781in}{1.074293in}}%
\pgfpathlineto{\pgfqpoint{1.174719in}{1.074293in}}%
\pgfpathlineto{\pgfqpoint{1.166657in}{1.074293in}}%
\pgfpathlineto{\pgfqpoint{1.158595in}{1.074293in}}%
\pgfpathlineto{\pgfqpoint{1.150533in}{1.074293in}}%
\pgfpathlineto{\pgfqpoint{1.142471in}{1.074293in}}%
\pgfpathlineto{\pgfqpoint{1.134409in}{1.074293in}}%
\pgfpathlineto{\pgfqpoint{1.126347in}{1.095094in}}%
\pgfpathlineto{\pgfqpoint{1.118285in}{1.113394in}}%
\pgfpathlineto{\pgfqpoint{1.110223in}{1.113394in}}%
\pgfpathlineto{\pgfqpoint{1.102161in}{1.113394in}}%
\pgfpathlineto{\pgfqpoint{1.094099in}{1.113394in}}%
\pgfpathlineto{\pgfqpoint{1.086037in}{1.113394in}}%
\pgfpathlineto{\pgfqpoint{1.077975in}{1.113394in}}%
\pgfpathlineto{\pgfqpoint{1.069914in}{1.113394in}}%
\pgfpathlineto{\pgfqpoint{1.061852in}{1.198533in}}%
\pgfpathlineto{\pgfqpoint{1.053790in}{1.225274in}}%
\pgfpathlineto{\pgfqpoint{1.045728in}{1.288495in}}%
\pgfpathlineto{\pgfqpoint{1.037666in}{1.288495in}}%
\pgfpathlineto{\pgfqpoint{1.029604in}{1.288495in}}%
\pgfpathlineto{\pgfqpoint{1.021542in}{1.305327in}}%
\pgfpathlineto{\pgfqpoint{1.013480in}{1.305327in}}%
\pgfpathlineto{\pgfqpoint{1.005418in}{1.305327in}}%
\pgfpathlineto{\pgfqpoint{0.997356in}{1.305327in}}%
\pgfpathlineto{\pgfqpoint{0.989294in}{1.334980in}}%
\pgfpathlineto{\pgfqpoint{0.981232in}{1.334980in}}%
\pgfpathlineto{\pgfqpoint{0.973170in}{1.334980in}}%
\pgfpathlineto{\pgfqpoint{0.965108in}{1.359258in}}%
\pgfpathlineto{\pgfqpoint{0.957046in}{1.359258in}}%
\pgfpathlineto{\pgfqpoint{0.948984in}{1.359258in}}%
\pgfpathlineto{\pgfqpoint{0.940922in}{1.359258in}}%
\pgfpathlineto{\pgfqpoint{0.932860in}{1.359258in}}%
\pgfpathlineto{\pgfqpoint{0.924798in}{1.359258in}}%
\pgfpathlineto{\pgfqpoint{0.916736in}{1.462634in}}%
\pgfpathlineto{\pgfqpoint{0.908674in}{1.462634in}}%
\pgfpathlineto{\pgfqpoint{0.900612in}{1.462634in}}%
\pgfpathlineto{\pgfqpoint{0.892550in}{1.574267in}}%
\pgfpathlineto{\pgfqpoint{0.884488in}{1.574267in}}%
\pgfpathlineto{\pgfqpoint{0.876426in}{1.600160in}}%
\pgfpathlineto{\pgfqpoint{0.868364in}{1.600160in}}%
\pgfpathlineto{\pgfqpoint{0.860302in}{1.600160in}}%
\pgfpathlineto{\pgfqpoint{0.852241in}{1.600160in}}%
\pgfpathlineto{\pgfqpoint{0.844179in}{1.600160in}}%
\pgfpathlineto{\pgfqpoint{0.836117in}{1.603470in}}%
\pgfpathlineto{\pgfqpoint{0.828055in}{1.603470in}}%
\pgfpathlineto{\pgfqpoint{0.819993in}{1.603470in}}%
\pgfpathlineto{\pgfqpoint{0.811931in}{1.677328in}}%
\pgfpathlineto{\pgfqpoint{0.803869in}{1.677328in}}%
\pgfpathlineto{\pgfqpoint{0.795807in}{1.677328in}}%
\pgfpathlineto{\pgfqpoint{0.787745in}{1.677328in}}%
\pgfpathlineto{\pgfqpoint{0.779683in}{1.677328in}}%
\pgfpathlineto{\pgfqpoint{0.771621in}{1.677328in}}%
\pgfpathlineto{\pgfqpoint{0.763559in}{1.677328in}}%
\pgfpathlineto{\pgfqpoint{0.755497in}{1.677328in}}%
\pgfpathlineto{\pgfqpoint{0.747435in}{1.678609in}}%
\pgfpathlineto{\pgfqpoint{0.739373in}{1.678609in}}%
\pgfpathlineto{\pgfqpoint{0.731311in}{1.678609in}}%
\pgfpathlineto{\pgfqpoint{0.723249in}{1.678609in}}%
\pgfpathlineto{\pgfqpoint{0.715187in}{1.678609in}}%
\pgfpathlineto{\pgfqpoint{0.707125in}{1.678609in}}%
\pgfpathlineto{\pgfqpoint{0.699063in}{1.780453in}}%
\pgfpathlineto{\pgfqpoint{0.691001in}{1.780453in}}%
\pgfpathlineto{\pgfqpoint{0.682939in}{1.780453in}}%
\pgfpathlineto{\pgfqpoint{0.674877in}{1.780453in}}%
\pgfpathlineto{\pgfqpoint{0.666815in}{1.780453in}}%
\pgfpathlineto{\pgfqpoint{0.658753in}{1.780453in}}%
\pgfpathlineto{\pgfqpoint{0.650691in}{1.780453in}}%
\pgfpathlineto{\pgfqpoint{0.642629in}{1.780453in}}%
\pgfpathlineto{\pgfqpoint{0.634568in}{1.780453in}}%
\pgfpathlineto{\pgfqpoint{0.626506in}{1.780453in}}%
\pgfpathlineto{\pgfqpoint{0.618444in}{1.780453in}}%
\pgfpathlineto{\pgfqpoint{0.610382in}{1.780453in}}%
\pgfpathlineto{\pgfqpoint{0.602320in}{1.900642in}}%
\pgfpathlineto{\pgfqpoint{0.594258in}{2.041993in}}%
\pgfpathlineto{\pgfqpoint{0.586196in}{2.084754in}}%
\pgfpathlineto{\pgfqpoint{0.578134in}{2.084754in}}%
\pgfpathlineto{\pgfqpoint{0.570072in}{2.084754in}}%
\pgfpathlineto{\pgfqpoint{0.562010in}{2.105823in}}%
\pgfpathlineto{\pgfqpoint{0.553948in}{2.135912in}}%
\pgfpathlineto{\pgfqpoint{0.545886in}{2.154540in}}%
\pgfpathlineto{\pgfqpoint{0.537824in}{2.154540in}}%
\pgfpathlineto{\pgfqpoint{0.529762in}{2.154540in}}%
\pgfpathlineto{\pgfqpoint{0.521700in}{2.154540in}}%
\pgfpathlineto{\pgfqpoint{0.513638in}{2.156514in}}%
\pgfpathlineto{\pgfqpoint{0.505576in}{2.156514in}}%
\pgfpathlineto{\pgfqpoint{0.497514in}{2.156514in}}%
\pgfpathlineto{\pgfqpoint{0.489452in}{2.175658in}}%
\pgfpathlineto{\pgfqpoint{0.481390in}{2.175658in}}%
\pgfpathlineto{\pgfqpoint{0.473328in}{2.175658in}}%
\pgfpathlineto{\pgfqpoint{0.465266in}{2.219891in}}%
\pgfpathlineto{\pgfqpoint{0.457204in}{2.219891in}}%
\pgfpathlineto{\pgfqpoint{0.449142in}{2.219891in}}%
\pgfpathlineto{\pgfqpoint{0.441080in}{2.227297in}}%
\pgfpathlineto{\pgfqpoint{0.433018in}{2.227297in}}%
\pgfpathlineto{\pgfqpoint{0.424956in}{2.227297in}}%
\pgfpathlineto{\pgfqpoint{0.416895in}{2.227297in}}%
\pgfpathlineto{\pgfqpoint{0.408833in}{2.227297in}}%
\pgfpathlineto{\pgfqpoint{0.400771in}{2.227297in}}%
\pgfpathlineto{\pgfqpoint{0.392709in}{2.239687in}}%
\pgfpathlineto{\pgfqpoint{0.384647in}{2.291673in}}%
\pgfpathlineto{\pgfqpoint{0.376585in}{2.297592in}}%
\pgfpathlineto{\pgfqpoint{0.368523in}{2.472209in}}%
\pgfpathclose%
\pgfusepath{fill}%
\end{pgfscope}%
\begin{pgfscope}%
\pgfpathrectangle{\pgfqpoint{0.287500in}{0.375000in}}{\pgfqpoint{1.782500in}{2.265000in}}%
\pgfusepath{clip}%
\pgfsetbuttcap%
\pgfsetroundjoin%
\definecolor{currentfill}{rgb}{0.839216,0.152941,0.156863}%
\pgfsetfillcolor{currentfill}%
\pgfsetfillopacity{0.200000}%
\pgfsetlinewidth{0.000000pt}%
\definecolor{currentstroke}{rgb}{0.000000,0.000000,0.000000}%
\pgfsetstrokecolor{currentstroke}%
\pgfsetdash{}{0pt}%
\pgfpathmoveto{\pgfqpoint{0.368523in}{2.484263in}}%
\pgfpathlineto{\pgfqpoint{0.368523in}{2.537045in}}%
\pgfpathlineto{\pgfqpoint{0.376585in}{2.492263in}}%
\pgfpathlineto{\pgfqpoint{0.384647in}{2.462626in}}%
\pgfpathlineto{\pgfqpoint{0.392709in}{2.351805in}}%
\pgfpathlineto{\pgfqpoint{0.400771in}{2.319723in}}%
\pgfpathlineto{\pgfqpoint{0.408833in}{2.178553in}}%
\pgfpathlineto{\pgfqpoint{0.416895in}{2.141989in}}%
\pgfpathlineto{\pgfqpoint{0.424956in}{2.141989in}}%
\pgfpathlineto{\pgfqpoint{0.433018in}{2.139324in}}%
\pgfpathlineto{\pgfqpoint{0.441080in}{2.118215in}}%
\pgfpathlineto{\pgfqpoint{0.449142in}{2.118215in}}%
\pgfpathlineto{\pgfqpoint{0.457204in}{2.118069in}}%
\pgfpathlineto{\pgfqpoint{0.465266in}{2.118069in}}%
\pgfpathlineto{\pgfqpoint{0.473328in}{2.116738in}}%
\pgfpathlineto{\pgfqpoint{0.481390in}{2.116691in}}%
\pgfpathlineto{\pgfqpoint{0.489452in}{2.116691in}}%
\pgfpathlineto{\pgfqpoint{0.497514in}{2.116691in}}%
\pgfpathlineto{\pgfqpoint{0.505576in}{2.116691in}}%
\pgfpathlineto{\pgfqpoint{0.513638in}{2.116691in}}%
\pgfpathlineto{\pgfqpoint{0.521700in}{2.116691in}}%
\pgfpathlineto{\pgfqpoint{0.529762in}{2.116112in}}%
\pgfpathlineto{\pgfqpoint{0.537824in}{2.116112in}}%
\pgfpathlineto{\pgfqpoint{0.545886in}{2.116112in}}%
\pgfpathlineto{\pgfqpoint{0.553948in}{2.116044in}}%
\pgfpathlineto{\pgfqpoint{0.562010in}{2.116044in}}%
\pgfpathlineto{\pgfqpoint{0.570072in}{2.116044in}}%
\pgfpathlineto{\pgfqpoint{0.578134in}{2.116044in}}%
\pgfpathlineto{\pgfqpoint{0.586196in}{2.116044in}}%
\pgfpathlineto{\pgfqpoint{0.594258in}{2.116044in}}%
\pgfpathlineto{\pgfqpoint{0.602320in}{2.115735in}}%
\pgfpathlineto{\pgfqpoint{0.610382in}{2.115735in}}%
\pgfpathlineto{\pgfqpoint{0.618444in}{2.115731in}}%
\pgfpathlineto{\pgfqpoint{0.626506in}{2.115731in}}%
\pgfpathlineto{\pgfqpoint{0.634568in}{2.110551in}}%
\pgfpathlineto{\pgfqpoint{0.642629in}{2.110544in}}%
\pgfpathlineto{\pgfqpoint{0.650691in}{2.110539in}}%
\pgfpathlineto{\pgfqpoint{0.658753in}{2.110539in}}%
\pgfpathlineto{\pgfqpoint{0.666815in}{2.110539in}}%
\pgfpathlineto{\pgfqpoint{0.674877in}{2.110538in}}%
\pgfpathlineto{\pgfqpoint{0.682939in}{2.106372in}}%
\pgfpathlineto{\pgfqpoint{0.691001in}{2.106372in}}%
\pgfpathlineto{\pgfqpoint{0.699063in}{2.106372in}}%
\pgfpathlineto{\pgfqpoint{0.707125in}{2.106372in}}%
\pgfpathlineto{\pgfqpoint{0.715187in}{2.106372in}}%
\pgfpathlineto{\pgfqpoint{0.723249in}{2.106372in}}%
\pgfpathlineto{\pgfqpoint{0.731311in}{2.106372in}}%
\pgfpathlineto{\pgfqpoint{0.739373in}{2.106372in}}%
\pgfpathlineto{\pgfqpoint{0.747435in}{2.106372in}}%
\pgfpathlineto{\pgfqpoint{0.755497in}{2.106372in}}%
\pgfpathlineto{\pgfqpoint{0.763559in}{2.106372in}}%
\pgfpathlineto{\pgfqpoint{0.771621in}{2.106372in}}%
\pgfpathlineto{\pgfqpoint{0.779683in}{2.106372in}}%
\pgfpathlineto{\pgfqpoint{0.787745in}{2.041162in}}%
\pgfpathlineto{\pgfqpoint{0.795807in}{2.041162in}}%
\pgfpathlineto{\pgfqpoint{0.803869in}{2.041162in}}%
\pgfpathlineto{\pgfqpoint{0.811931in}{2.041162in}}%
\pgfpathlineto{\pgfqpoint{0.819993in}{2.041162in}}%
\pgfpathlineto{\pgfqpoint{0.828055in}{2.041162in}}%
\pgfpathlineto{\pgfqpoint{0.836117in}{2.041162in}}%
\pgfpathlineto{\pgfqpoint{0.844179in}{2.041162in}}%
\pgfpathlineto{\pgfqpoint{0.852241in}{2.041162in}}%
\pgfpathlineto{\pgfqpoint{0.860302in}{2.041162in}}%
\pgfpathlineto{\pgfqpoint{0.868364in}{2.041162in}}%
\pgfpathlineto{\pgfqpoint{0.876426in}{2.041162in}}%
\pgfpathlineto{\pgfqpoint{0.884488in}{2.041162in}}%
\pgfpathlineto{\pgfqpoint{0.892550in}{2.041162in}}%
\pgfpathlineto{\pgfqpoint{0.900612in}{2.041162in}}%
\pgfpathlineto{\pgfqpoint{0.908674in}{2.041162in}}%
\pgfpathlineto{\pgfqpoint{0.916736in}{2.041162in}}%
\pgfpathlineto{\pgfqpoint{0.924798in}{2.041162in}}%
\pgfpathlineto{\pgfqpoint{0.932860in}{2.041162in}}%
\pgfpathlineto{\pgfqpoint{0.940922in}{2.041162in}}%
\pgfpathlineto{\pgfqpoint{0.948984in}{2.041162in}}%
\pgfpathlineto{\pgfqpoint{0.957046in}{2.041162in}}%
\pgfpathlineto{\pgfqpoint{0.965108in}{2.041162in}}%
\pgfpathlineto{\pgfqpoint{0.973170in}{2.041162in}}%
\pgfpathlineto{\pgfqpoint{0.981232in}{2.041162in}}%
\pgfpathlineto{\pgfqpoint{0.989294in}{2.041162in}}%
\pgfpathlineto{\pgfqpoint{0.997356in}{2.040406in}}%
\pgfpathlineto{\pgfqpoint{1.005418in}{2.040406in}}%
\pgfpathlineto{\pgfqpoint{1.013480in}{2.040406in}}%
\pgfpathlineto{\pgfqpoint{1.021542in}{2.040406in}}%
\pgfpathlineto{\pgfqpoint{1.029604in}{2.040406in}}%
\pgfpathlineto{\pgfqpoint{1.037666in}{2.040406in}}%
\pgfpathlineto{\pgfqpoint{1.045728in}{2.040406in}}%
\pgfpathlineto{\pgfqpoint{1.053790in}{2.040406in}}%
\pgfpathlineto{\pgfqpoint{1.061852in}{2.040406in}}%
\pgfpathlineto{\pgfqpoint{1.069914in}{2.040406in}}%
\pgfpathlineto{\pgfqpoint{1.077975in}{2.040406in}}%
\pgfpathlineto{\pgfqpoint{1.086037in}{2.040406in}}%
\pgfpathlineto{\pgfqpoint{1.094099in}{2.040406in}}%
\pgfpathlineto{\pgfqpoint{1.102161in}{2.040406in}}%
\pgfpathlineto{\pgfqpoint{1.110223in}{2.040406in}}%
\pgfpathlineto{\pgfqpoint{1.118285in}{2.040406in}}%
\pgfpathlineto{\pgfqpoint{1.126347in}{2.040406in}}%
\pgfpathlineto{\pgfqpoint{1.134409in}{2.040406in}}%
\pgfpathlineto{\pgfqpoint{1.142471in}{2.040406in}}%
\pgfpathlineto{\pgfqpoint{1.150533in}{2.040406in}}%
\pgfpathlineto{\pgfqpoint{1.158595in}{2.040406in}}%
\pgfpathlineto{\pgfqpoint{1.166657in}{2.040406in}}%
\pgfpathlineto{\pgfqpoint{1.174719in}{2.040406in}}%
\pgfpathlineto{\pgfqpoint{1.182781in}{2.040406in}}%
\pgfpathlineto{\pgfqpoint{1.190843in}{2.040406in}}%
\pgfpathlineto{\pgfqpoint{1.198905in}{2.040406in}}%
\pgfpathlineto{\pgfqpoint{1.206967in}{2.039044in}}%
\pgfpathlineto{\pgfqpoint{1.215029in}{2.039044in}}%
\pgfpathlineto{\pgfqpoint{1.223091in}{2.039044in}}%
\pgfpathlineto{\pgfqpoint{1.231153in}{2.039044in}}%
\pgfpathlineto{\pgfqpoint{1.239215in}{2.039044in}}%
\pgfpathlineto{\pgfqpoint{1.247277in}{2.039044in}}%
\pgfpathlineto{\pgfqpoint{1.255339in}{2.039044in}}%
\pgfpathlineto{\pgfqpoint{1.263401in}{2.039044in}}%
\pgfpathlineto{\pgfqpoint{1.271463in}{2.039044in}}%
\pgfpathlineto{\pgfqpoint{1.279525in}{2.039044in}}%
\pgfpathlineto{\pgfqpoint{1.287586in}{2.039044in}}%
\pgfpathlineto{\pgfqpoint{1.295648in}{2.039044in}}%
\pgfpathlineto{\pgfqpoint{1.303710in}{2.039044in}}%
\pgfpathlineto{\pgfqpoint{1.311772in}{2.039044in}}%
\pgfpathlineto{\pgfqpoint{1.319834in}{2.039044in}}%
\pgfpathlineto{\pgfqpoint{1.327896in}{2.039044in}}%
\pgfpathlineto{\pgfqpoint{1.335958in}{2.039044in}}%
\pgfpathlineto{\pgfqpoint{1.344020in}{2.039044in}}%
\pgfpathlineto{\pgfqpoint{1.352082in}{2.039044in}}%
\pgfpathlineto{\pgfqpoint{1.360144in}{2.039044in}}%
\pgfpathlineto{\pgfqpoint{1.368206in}{2.039044in}}%
\pgfpathlineto{\pgfqpoint{1.376268in}{2.039044in}}%
\pgfpathlineto{\pgfqpoint{1.384330in}{2.039044in}}%
\pgfpathlineto{\pgfqpoint{1.392392in}{2.039044in}}%
\pgfpathlineto{\pgfqpoint{1.400454in}{2.039044in}}%
\pgfpathlineto{\pgfqpoint{1.408516in}{2.039044in}}%
\pgfpathlineto{\pgfqpoint{1.416578in}{2.039044in}}%
\pgfpathlineto{\pgfqpoint{1.424640in}{2.039044in}}%
\pgfpathlineto{\pgfqpoint{1.432702in}{2.039044in}}%
\pgfpathlineto{\pgfqpoint{1.440764in}{2.039044in}}%
\pgfpathlineto{\pgfqpoint{1.448826in}{2.039044in}}%
\pgfpathlineto{\pgfqpoint{1.456888in}{2.039044in}}%
\pgfpathlineto{\pgfqpoint{1.464950in}{2.039044in}}%
\pgfpathlineto{\pgfqpoint{1.473012in}{2.039044in}}%
\pgfpathlineto{\pgfqpoint{1.481074in}{2.039044in}}%
\pgfpathlineto{\pgfqpoint{1.489136in}{2.039044in}}%
\pgfpathlineto{\pgfqpoint{1.497198in}{2.039044in}}%
\pgfpathlineto{\pgfqpoint{1.505259in}{2.039044in}}%
\pgfpathlineto{\pgfqpoint{1.513321in}{2.039044in}}%
\pgfpathlineto{\pgfqpoint{1.521383in}{2.039044in}}%
\pgfpathlineto{\pgfqpoint{1.529445in}{2.039044in}}%
\pgfpathlineto{\pgfqpoint{1.537507in}{2.039044in}}%
\pgfpathlineto{\pgfqpoint{1.545569in}{2.039044in}}%
\pgfpathlineto{\pgfqpoint{1.553631in}{2.039044in}}%
\pgfpathlineto{\pgfqpoint{1.561693in}{2.039044in}}%
\pgfpathlineto{\pgfqpoint{1.569755in}{2.039044in}}%
\pgfpathlineto{\pgfqpoint{1.577817in}{2.039044in}}%
\pgfpathlineto{\pgfqpoint{1.585879in}{2.039044in}}%
\pgfpathlineto{\pgfqpoint{1.593941in}{2.039044in}}%
\pgfpathlineto{\pgfqpoint{1.602003in}{2.039044in}}%
\pgfpathlineto{\pgfqpoint{1.610065in}{2.039044in}}%
\pgfpathlineto{\pgfqpoint{1.618127in}{2.039044in}}%
\pgfpathlineto{\pgfqpoint{1.626189in}{2.039044in}}%
\pgfpathlineto{\pgfqpoint{1.634251in}{2.039044in}}%
\pgfpathlineto{\pgfqpoint{1.642313in}{2.039044in}}%
\pgfpathlineto{\pgfqpoint{1.650375in}{2.039044in}}%
\pgfpathlineto{\pgfqpoint{1.658437in}{2.039044in}}%
\pgfpathlineto{\pgfqpoint{1.666499in}{2.039044in}}%
\pgfpathlineto{\pgfqpoint{1.674561in}{2.039044in}}%
\pgfpathlineto{\pgfqpoint{1.682623in}{1.747774in}}%
\pgfpathlineto{\pgfqpoint{1.690685in}{1.747774in}}%
\pgfpathlineto{\pgfqpoint{1.698747in}{1.747774in}}%
\pgfpathlineto{\pgfqpoint{1.706809in}{1.492897in}}%
\pgfpathlineto{\pgfqpoint{1.714871in}{1.492897in}}%
\pgfpathlineto{\pgfqpoint{1.722932in}{1.492897in}}%
\pgfpathlineto{\pgfqpoint{1.730994in}{1.492897in}}%
\pgfpathlineto{\pgfqpoint{1.739056in}{1.492897in}}%
\pgfpathlineto{\pgfqpoint{1.747118in}{1.492897in}}%
\pgfpathlineto{\pgfqpoint{1.755180in}{1.482121in}}%
\pgfpathlineto{\pgfqpoint{1.763242in}{1.482121in}}%
\pgfpathlineto{\pgfqpoint{1.771304in}{1.482121in}}%
\pgfpathlineto{\pgfqpoint{1.779366in}{1.482121in}}%
\pgfpathlineto{\pgfqpoint{1.787428in}{1.482121in}}%
\pgfpathlineto{\pgfqpoint{1.795490in}{1.482121in}}%
\pgfpathlineto{\pgfqpoint{1.803552in}{1.482121in}}%
\pgfpathlineto{\pgfqpoint{1.811614in}{1.482121in}}%
\pgfpathlineto{\pgfqpoint{1.819676in}{1.482121in}}%
\pgfpathlineto{\pgfqpoint{1.827738in}{1.482121in}}%
\pgfpathlineto{\pgfqpoint{1.835800in}{1.482121in}}%
\pgfpathlineto{\pgfqpoint{1.843862in}{1.482121in}}%
\pgfpathlineto{\pgfqpoint{1.851924in}{1.482121in}}%
\pgfpathlineto{\pgfqpoint{1.859986in}{1.482121in}}%
\pgfpathlineto{\pgfqpoint{1.868048in}{1.479202in}}%
\pgfpathlineto{\pgfqpoint{1.876110in}{1.479137in}}%
\pgfpathlineto{\pgfqpoint{1.884172in}{1.479137in}}%
\pgfpathlineto{\pgfqpoint{1.892234in}{1.479137in}}%
\pgfpathlineto{\pgfqpoint{1.900296in}{1.479137in}}%
\pgfpathlineto{\pgfqpoint{1.908358in}{1.360635in}}%
\pgfpathlineto{\pgfqpoint{1.916420in}{1.360635in}}%
\pgfpathlineto{\pgfqpoint{1.924482in}{1.360635in}}%
\pgfpathlineto{\pgfqpoint{1.932544in}{1.360635in}}%
\pgfpathlineto{\pgfqpoint{1.940605in}{1.360635in}}%
\pgfpathlineto{\pgfqpoint{1.948667in}{1.360635in}}%
\pgfpathlineto{\pgfqpoint{1.956729in}{1.360635in}}%
\pgfpathlineto{\pgfqpoint{1.964791in}{1.360635in}}%
\pgfpathlineto{\pgfqpoint{1.972853in}{1.360635in}}%
\pgfpathlineto{\pgfqpoint{1.980915in}{1.360635in}}%
\pgfpathlineto{\pgfqpoint{1.988977in}{1.360635in}}%
\pgfpathlineto{\pgfqpoint{1.988977in}{1.183223in}}%
\pgfpathlineto{\pgfqpoint{1.988977in}{1.183223in}}%
\pgfpathlineto{\pgfqpoint{1.980915in}{1.183223in}}%
\pgfpathlineto{\pgfqpoint{1.972853in}{1.183223in}}%
\pgfpathlineto{\pgfqpoint{1.964791in}{1.183223in}}%
\pgfpathlineto{\pgfqpoint{1.956729in}{1.183223in}}%
\pgfpathlineto{\pgfqpoint{1.948667in}{1.183223in}}%
\pgfpathlineto{\pgfqpoint{1.940605in}{1.183223in}}%
\pgfpathlineto{\pgfqpoint{1.932544in}{1.183223in}}%
\pgfpathlineto{\pgfqpoint{1.924482in}{1.183223in}}%
\pgfpathlineto{\pgfqpoint{1.916420in}{1.183223in}}%
\pgfpathlineto{\pgfqpoint{1.908358in}{1.183223in}}%
\pgfpathlineto{\pgfqpoint{1.900296in}{1.233687in}}%
\pgfpathlineto{\pgfqpoint{1.892234in}{1.233687in}}%
\pgfpathlineto{\pgfqpoint{1.884172in}{1.233687in}}%
\pgfpathlineto{\pgfqpoint{1.876110in}{1.233687in}}%
\pgfpathlineto{\pgfqpoint{1.868048in}{1.234398in}}%
\pgfpathlineto{\pgfqpoint{1.859986in}{1.261458in}}%
\pgfpathlineto{\pgfqpoint{1.851924in}{1.261458in}}%
\pgfpathlineto{\pgfqpoint{1.843862in}{1.261458in}}%
\pgfpathlineto{\pgfqpoint{1.835800in}{1.261458in}}%
\pgfpathlineto{\pgfqpoint{1.827738in}{1.261458in}}%
\pgfpathlineto{\pgfqpoint{1.819676in}{1.261458in}}%
\pgfpathlineto{\pgfqpoint{1.811614in}{1.261458in}}%
\pgfpathlineto{\pgfqpoint{1.803552in}{1.261458in}}%
\pgfpathlineto{\pgfqpoint{1.795490in}{1.261458in}}%
\pgfpathlineto{\pgfqpoint{1.787428in}{1.261458in}}%
\pgfpathlineto{\pgfqpoint{1.779366in}{1.261458in}}%
\pgfpathlineto{\pgfqpoint{1.771304in}{1.261458in}}%
\pgfpathlineto{\pgfqpoint{1.763242in}{1.261458in}}%
\pgfpathlineto{\pgfqpoint{1.755180in}{1.261458in}}%
\pgfpathlineto{\pgfqpoint{1.747118in}{1.325807in}}%
\pgfpathlineto{\pgfqpoint{1.739056in}{1.325807in}}%
\pgfpathlineto{\pgfqpoint{1.730994in}{1.325807in}}%
\pgfpathlineto{\pgfqpoint{1.722932in}{1.325807in}}%
\pgfpathlineto{\pgfqpoint{1.714871in}{1.325807in}}%
\pgfpathlineto{\pgfqpoint{1.706809in}{1.325807in}}%
\pgfpathlineto{\pgfqpoint{1.698747in}{1.424257in}}%
\pgfpathlineto{\pgfqpoint{1.690685in}{1.424257in}}%
\pgfpathlineto{\pgfqpoint{1.682623in}{1.424257in}}%
\pgfpathlineto{\pgfqpoint{1.674561in}{1.632824in}}%
\pgfpathlineto{\pgfqpoint{1.666499in}{1.632824in}}%
\pgfpathlineto{\pgfqpoint{1.658437in}{1.632824in}}%
\pgfpathlineto{\pgfqpoint{1.650375in}{1.632824in}}%
\pgfpathlineto{\pgfqpoint{1.642313in}{1.632824in}}%
\pgfpathlineto{\pgfqpoint{1.634251in}{1.632824in}}%
\pgfpathlineto{\pgfqpoint{1.626189in}{1.632824in}}%
\pgfpathlineto{\pgfqpoint{1.618127in}{1.632824in}}%
\pgfpathlineto{\pgfqpoint{1.610065in}{1.632824in}}%
\pgfpathlineto{\pgfqpoint{1.602003in}{1.632824in}}%
\pgfpathlineto{\pgfqpoint{1.593941in}{1.632824in}}%
\pgfpathlineto{\pgfqpoint{1.585879in}{1.632824in}}%
\pgfpathlineto{\pgfqpoint{1.577817in}{1.632824in}}%
\pgfpathlineto{\pgfqpoint{1.569755in}{1.632824in}}%
\pgfpathlineto{\pgfqpoint{1.561693in}{1.632824in}}%
\pgfpathlineto{\pgfqpoint{1.553631in}{1.632824in}}%
\pgfpathlineto{\pgfqpoint{1.545569in}{1.632824in}}%
\pgfpathlineto{\pgfqpoint{1.537507in}{1.632824in}}%
\pgfpathlineto{\pgfqpoint{1.529445in}{1.632824in}}%
\pgfpathlineto{\pgfqpoint{1.521383in}{1.632824in}}%
\pgfpathlineto{\pgfqpoint{1.513321in}{1.632824in}}%
\pgfpathlineto{\pgfqpoint{1.505259in}{1.632824in}}%
\pgfpathlineto{\pgfqpoint{1.497198in}{1.632824in}}%
\pgfpathlineto{\pgfqpoint{1.489136in}{1.632824in}}%
\pgfpathlineto{\pgfqpoint{1.481074in}{1.632824in}}%
\pgfpathlineto{\pgfqpoint{1.473012in}{1.632824in}}%
\pgfpathlineto{\pgfqpoint{1.464950in}{1.632824in}}%
\pgfpathlineto{\pgfqpoint{1.456888in}{1.632824in}}%
\pgfpathlineto{\pgfqpoint{1.448826in}{1.632824in}}%
\pgfpathlineto{\pgfqpoint{1.440764in}{1.632824in}}%
\pgfpathlineto{\pgfqpoint{1.432702in}{1.632824in}}%
\pgfpathlineto{\pgfqpoint{1.424640in}{1.632824in}}%
\pgfpathlineto{\pgfqpoint{1.416578in}{1.632824in}}%
\pgfpathlineto{\pgfqpoint{1.408516in}{1.632824in}}%
\pgfpathlineto{\pgfqpoint{1.400454in}{1.632824in}}%
\pgfpathlineto{\pgfqpoint{1.392392in}{1.632824in}}%
\pgfpathlineto{\pgfqpoint{1.384330in}{1.632824in}}%
\pgfpathlineto{\pgfqpoint{1.376268in}{1.632824in}}%
\pgfpathlineto{\pgfqpoint{1.368206in}{1.632824in}}%
\pgfpathlineto{\pgfqpoint{1.360144in}{1.632824in}}%
\pgfpathlineto{\pgfqpoint{1.352082in}{1.632824in}}%
\pgfpathlineto{\pgfqpoint{1.344020in}{1.632824in}}%
\pgfpathlineto{\pgfqpoint{1.335958in}{1.632824in}}%
\pgfpathlineto{\pgfqpoint{1.327896in}{1.632824in}}%
\pgfpathlineto{\pgfqpoint{1.319834in}{1.632824in}}%
\pgfpathlineto{\pgfqpoint{1.311772in}{1.632824in}}%
\pgfpathlineto{\pgfqpoint{1.303710in}{1.632824in}}%
\pgfpathlineto{\pgfqpoint{1.295648in}{1.632824in}}%
\pgfpathlineto{\pgfqpoint{1.287586in}{1.632824in}}%
\pgfpathlineto{\pgfqpoint{1.279525in}{1.632824in}}%
\pgfpathlineto{\pgfqpoint{1.271463in}{1.632824in}}%
\pgfpathlineto{\pgfqpoint{1.263401in}{1.632824in}}%
\pgfpathlineto{\pgfqpoint{1.255339in}{1.632824in}}%
\pgfpathlineto{\pgfqpoint{1.247277in}{1.632824in}}%
\pgfpathlineto{\pgfqpoint{1.239215in}{1.632824in}}%
\pgfpathlineto{\pgfqpoint{1.231153in}{1.632824in}}%
\pgfpathlineto{\pgfqpoint{1.223091in}{1.632824in}}%
\pgfpathlineto{\pgfqpoint{1.215029in}{1.632824in}}%
\pgfpathlineto{\pgfqpoint{1.206967in}{1.632824in}}%
\pgfpathlineto{\pgfqpoint{1.198905in}{1.666910in}}%
\pgfpathlineto{\pgfqpoint{1.190843in}{1.666910in}}%
\pgfpathlineto{\pgfqpoint{1.182781in}{1.666910in}}%
\pgfpathlineto{\pgfqpoint{1.174719in}{1.666910in}}%
\pgfpathlineto{\pgfqpoint{1.166657in}{1.666910in}}%
\pgfpathlineto{\pgfqpoint{1.158595in}{1.666910in}}%
\pgfpathlineto{\pgfqpoint{1.150533in}{1.666910in}}%
\pgfpathlineto{\pgfqpoint{1.142471in}{1.666910in}}%
\pgfpathlineto{\pgfqpoint{1.134409in}{1.666910in}}%
\pgfpathlineto{\pgfqpoint{1.126347in}{1.666910in}}%
\pgfpathlineto{\pgfqpoint{1.118285in}{1.666910in}}%
\pgfpathlineto{\pgfqpoint{1.110223in}{1.666910in}}%
\pgfpathlineto{\pgfqpoint{1.102161in}{1.666910in}}%
\pgfpathlineto{\pgfqpoint{1.094099in}{1.666910in}}%
\pgfpathlineto{\pgfqpoint{1.086037in}{1.666910in}}%
\pgfpathlineto{\pgfqpoint{1.077975in}{1.666910in}}%
\pgfpathlineto{\pgfqpoint{1.069914in}{1.666910in}}%
\pgfpathlineto{\pgfqpoint{1.061852in}{1.666910in}}%
\pgfpathlineto{\pgfqpoint{1.053790in}{1.666910in}}%
\pgfpathlineto{\pgfqpoint{1.045728in}{1.666910in}}%
\pgfpathlineto{\pgfqpoint{1.037666in}{1.666910in}}%
\pgfpathlineto{\pgfqpoint{1.029604in}{1.666910in}}%
\pgfpathlineto{\pgfqpoint{1.021542in}{1.666910in}}%
\pgfpathlineto{\pgfqpoint{1.013480in}{1.666910in}}%
\pgfpathlineto{\pgfqpoint{1.005418in}{1.666910in}}%
\pgfpathlineto{\pgfqpoint{0.997356in}{1.666910in}}%
\pgfpathlineto{\pgfqpoint{0.989294in}{1.682327in}}%
\pgfpathlineto{\pgfqpoint{0.981232in}{1.682327in}}%
\pgfpathlineto{\pgfqpoint{0.973170in}{1.682327in}}%
\pgfpathlineto{\pgfqpoint{0.965108in}{1.682327in}}%
\pgfpathlineto{\pgfqpoint{0.957046in}{1.682327in}}%
\pgfpathlineto{\pgfqpoint{0.948984in}{1.682327in}}%
\pgfpathlineto{\pgfqpoint{0.940922in}{1.682327in}}%
\pgfpathlineto{\pgfqpoint{0.932860in}{1.682327in}}%
\pgfpathlineto{\pgfqpoint{0.924798in}{1.682327in}}%
\pgfpathlineto{\pgfqpoint{0.916736in}{1.682327in}}%
\pgfpathlineto{\pgfqpoint{0.908674in}{1.682327in}}%
\pgfpathlineto{\pgfqpoint{0.900612in}{1.682327in}}%
\pgfpathlineto{\pgfqpoint{0.892550in}{1.682327in}}%
\pgfpathlineto{\pgfqpoint{0.884488in}{1.682327in}}%
\pgfpathlineto{\pgfqpoint{0.876426in}{1.682327in}}%
\pgfpathlineto{\pgfqpoint{0.868364in}{1.682327in}}%
\pgfpathlineto{\pgfqpoint{0.860302in}{1.682327in}}%
\pgfpathlineto{\pgfqpoint{0.852241in}{1.682327in}}%
\pgfpathlineto{\pgfqpoint{0.844179in}{1.682327in}}%
\pgfpathlineto{\pgfqpoint{0.836117in}{1.682327in}}%
\pgfpathlineto{\pgfqpoint{0.828055in}{1.682327in}}%
\pgfpathlineto{\pgfqpoint{0.819993in}{1.682327in}}%
\pgfpathlineto{\pgfqpoint{0.811931in}{1.682327in}}%
\pgfpathlineto{\pgfqpoint{0.803869in}{1.682327in}}%
\pgfpathlineto{\pgfqpoint{0.795807in}{1.682327in}}%
\pgfpathlineto{\pgfqpoint{0.787745in}{1.682327in}}%
\pgfpathlineto{\pgfqpoint{0.779683in}{1.927071in}}%
\pgfpathlineto{\pgfqpoint{0.771621in}{1.927071in}}%
\pgfpathlineto{\pgfqpoint{0.763559in}{1.927071in}}%
\pgfpathlineto{\pgfqpoint{0.755497in}{1.927071in}}%
\pgfpathlineto{\pgfqpoint{0.747435in}{1.927071in}}%
\pgfpathlineto{\pgfqpoint{0.739373in}{1.927071in}}%
\pgfpathlineto{\pgfqpoint{0.731311in}{1.927071in}}%
\pgfpathlineto{\pgfqpoint{0.723249in}{1.927071in}}%
\pgfpathlineto{\pgfqpoint{0.715187in}{1.927071in}}%
\pgfpathlineto{\pgfqpoint{0.707125in}{1.927071in}}%
\pgfpathlineto{\pgfqpoint{0.699063in}{1.927071in}}%
\pgfpathlineto{\pgfqpoint{0.691001in}{1.927072in}}%
\pgfpathlineto{\pgfqpoint{0.682939in}{1.927072in}}%
\pgfpathlineto{\pgfqpoint{0.674877in}{1.931028in}}%
\pgfpathlineto{\pgfqpoint{0.666815in}{1.931029in}}%
\pgfpathlineto{\pgfqpoint{0.658753in}{1.931029in}}%
\pgfpathlineto{\pgfqpoint{0.650691in}{1.931029in}}%
\pgfpathlineto{\pgfqpoint{0.642629in}{1.931075in}}%
\pgfpathlineto{\pgfqpoint{0.634568in}{1.931132in}}%
\pgfpathlineto{\pgfqpoint{0.626506in}{1.936194in}}%
\pgfpathlineto{\pgfqpoint{0.618444in}{1.936194in}}%
\pgfpathlineto{\pgfqpoint{0.610382in}{1.936221in}}%
\pgfpathlineto{\pgfqpoint{0.602320in}{1.936221in}}%
\pgfpathlineto{\pgfqpoint{0.594258in}{1.936583in}}%
\pgfpathlineto{\pgfqpoint{0.586196in}{1.936583in}}%
\pgfpathlineto{\pgfqpoint{0.578134in}{1.936583in}}%
\pgfpathlineto{\pgfqpoint{0.570072in}{1.936583in}}%
\pgfpathlineto{\pgfqpoint{0.562010in}{1.936583in}}%
\pgfpathlineto{\pgfqpoint{0.553948in}{1.936583in}}%
\pgfpathlineto{\pgfqpoint{0.545886in}{1.937132in}}%
\pgfpathlineto{\pgfqpoint{0.537824in}{1.937132in}}%
\pgfpathlineto{\pgfqpoint{0.529762in}{1.937132in}}%
\pgfpathlineto{\pgfqpoint{0.521700in}{1.937799in}}%
\pgfpathlineto{\pgfqpoint{0.513638in}{1.937799in}}%
\pgfpathlineto{\pgfqpoint{0.505576in}{1.937799in}}%
\pgfpathlineto{\pgfqpoint{0.497514in}{1.937799in}}%
\pgfpathlineto{\pgfqpoint{0.489452in}{1.937799in}}%
\pgfpathlineto{\pgfqpoint{0.481390in}{1.937799in}}%
\pgfpathlineto{\pgfqpoint{0.473328in}{1.938181in}}%
\pgfpathlineto{\pgfqpoint{0.465266in}{1.948361in}}%
\pgfpathlineto{\pgfqpoint{0.457204in}{1.948361in}}%
\pgfpathlineto{\pgfqpoint{0.449142in}{1.949511in}}%
\pgfpathlineto{\pgfqpoint{0.441080in}{1.949511in}}%
\pgfpathlineto{\pgfqpoint{0.433018in}{2.048800in}}%
\pgfpathlineto{\pgfqpoint{0.424956in}{2.050058in}}%
\pgfpathlineto{\pgfqpoint{0.416895in}{2.050058in}}%
\pgfpathlineto{\pgfqpoint{0.408833in}{2.068141in}}%
\pgfpathlineto{\pgfqpoint{0.400771in}{2.102904in}}%
\pgfpathlineto{\pgfqpoint{0.392709in}{2.223799in}}%
\pgfpathlineto{\pgfqpoint{0.384647in}{2.412793in}}%
\pgfpathlineto{\pgfqpoint{0.376585in}{2.445515in}}%
\pgfpathlineto{\pgfqpoint{0.368523in}{2.484263in}}%
\pgfpathclose%
\pgfusepath{fill}%
\end{pgfscope}%
\begin{pgfscope}%
\pgfpathrectangle{\pgfqpoint{0.287500in}{0.375000in}}{\pgfqpoint{1.782500in}{2.265000in}}%
\pgfusepath{clip}%
\pgfsetroundcap%
\pgfsetroundjoin%
\pgfsetlinewidth{1.505625pt}%
\definecolor{currentstroke}{rgb}{0.121569,0.466667,0.705882}%
\pgfsetstrokecolor{currentstroke}%
\pgfsetdash{}{0pt}%
\pgfpathmoveto{\pgfqpoint{0.368523in}{2.499952in}}%
\pgfpathlineto{\pgfqpoint{0.376585in}{2.456985in}}%
\pgfpathlineto{\pgfqpoint{0.392709in}{2.413938in}}%
\pgfpathlineto{\pgfqpoint{0.400771in}{2.372263in}}%
\pgfpathlineto{\pgfqpoint{0.408833in}{2.350986in}}%
\pgfpathlineto{\pgfqpoint{0.416895in}{2.347619in}}%
\pgfpathlineto{\pgfqpoint{0.424956in}{2.335105in}}%
\pgfpathlineto{\pgfqpoint{0.433018in}{2.316365in}}%
\pgfpathlineto{\pgfqpoint{0.449142in}{2.316365in}}%
\pgfpathlineto{\pgfqpoint{0.457204in}{2.282460in}}%
\pgfpathlineto{\pgfqpoint{0.465266in}{2.282460in}}%
\pgfpathlineto{\pgfqpoint{0.473328in}{2.265544in}}%
\pgfpathlineto{\pgfqpoint{0.481390in}{2.262965in}}%
\pgfpathlineto{\pgfqpoint{0.489452in}{2.256625in}}%
\pgfpathlineto{\pgfqpoint{0.505576in}{2.205800in}}%
\pgfpathlineto{\pgfqpoint{0.513638in}{2.205800in}}%
\pgfpathlineto{\pgfqpoint{0.521700in}{2.195675in}}%
\pgfpathlineto{\pgfqpoint{0.529762in}{2.195675in}}%
\pgfpathlineto{\pgfqpoint{0.537824in}{2.180669in}}%
\pgfpathlineto{\pgfqpoint{0.545886in}{2.180669in}}%
\pgfpathlineto{\pgfqpoint{0.553948in}{2.167019in}}%
\pgfpathlineto{\pgfqpoint{0.570072in}{2.167019in}}%
\pgfpathlineto{\pgfqpoint{0.578134in}{2.149597in}}%
\pgfpathlineto{\pgfqpoint{0.586196in}{2.149597in}}%
\pgfpathlineto{\pgfqpoint{0.594258in}{2.131595in}}%
\pgfpathlineto{\pgfqpoint{0.618444in}{2.131595in}}%
\pgfpathlineto{\pgfqpoint{0.634568in}{2.052224in}}%
\pgfpathlineto{\pgfqpoint{0.658753in}{2.052224in}}%
\pgfpathlineto{\pgfqpoint{0.666815in}{2.030467in}}%
\pgfpathlineto{\pgfqpoint{0.682939in}{2.030467in}}%
\pgfpathlineto{\pgfqpoint{0.691001in}{2.021702in}}%
\pgfpathlineto{\pgfqpoint{0.699063in}{2.008425in}}%
\pgfpathlineto{\pgfqpoint{0.803869in}{2.008425in}}%
\pgfpathlineto{\pgfqpoint{0.811931in}{2.005377in}}%
\pgfpathlineto{\pgfqpoint{0.860302in}{2.005377in}}%
\pgfpathlineto{\pgfqpoint{0.868364in}{1.963620in}}%
\pgfpathlineto{\pgfqpoint{0.876426in}{1.943894in}}%
\pgfpathlineto{\pgfqpoint{0.884488in}{1.943894in}}%
\pgfpathlineto{\pgfqpoint{0.892550in}{1.822804in}}%
\pgfpathlineto{\pgfqpoint{1.077975in}{1.822768in}}%
\pgfpathlineto{\pgfqpoint{1.086037in}{1.814163in}}%
\pgfpathlineto{\pgfqpoint{1.102161in}{1.814163in}}%
\pgfpathlineto{\pgfqpoint{1.110223in}{1.806096in}}%
\pgfpathlineto{\pgfqpoint{1.198905in}{1.805551in}}%
\pgfpathlineto{\pgfqpoint{1.206967in}{1.799335in}}%
\pgfpathlineto{\pgfqpoint{1.319834in}{1.799335in}}%
\pgfpathlineto{\pgfqpoint{1.327896in}{1.729747in}}%
\pgfpathlineto{\pgfqpoint{1.658437in}{1.729747in}}%
\pgfpathlineto{\pgfqpoint{1.666499in}{1.678838in}}%
\pgfpathlineto{\pgfqpoint{1.948667in}{1.678838in}}%
\pgfpathlineto{\pgfqpoint{1.956729in}{1.608929in}}%
\pgfpathlineto{\pgfqpoint{1.964791in}{1.608929in}}%
\pgfpathlineto{\pgfqpoint{1.972853in}{1.559136in}}%
\pgfpathlineto{\pgfqpoint{1.988977in}{1.559136in}}%
\pgfpathlineto{\pgfqpoint{1.988977in}{1.559136in}}%
\pgfusepath{stroke}%
\end{pgfscope}%
\begin{pgfscope}%
\pgfpathrectangle{\pgfqpoint{0.287500in}{0.375000in}}{\pgfqpoint{1.782500in}{2.265000in}}%
\pgfusepath{clip}%
\pgfsetroundcap%
\pgfsetroundjoin%
\pgfsetlinewidth{1.505625pt}%
\definecolor{currentstroke}{rgb}{1.000000,0.498039,0.054902}%
\pgfsetstrokecolor{currentstroke}%
\pgfsetdash{}{0pt}%
\pgfpathmoveto{\pgfqpoint{0.368523in}{2.470394in}}%
\pgfpathlineto{\pgfqpoint{0.376585in}{2.346863in}}%
\pgfpathlineto{\pgfqpoint{0.384647in}{2.331139in}}%
\pgfpathlineto{\pgfqpoint{0.441080in}{2.331139in}}%
\pgfpathlineto{\pgfqpoint{0.457204in}{2.287094in}}%
\pgfpathlineto{\pgfqpoint{0.465266in}{2.218470in}}%
\pgfpathlineto{\pgfqpoint{0.473328in}{2.125847in}}%
\pgfpathlineto{\pgfqpoint{0.481390in}{2.125847in}}%
\pgfpathlineto{\pgfqpoint{0.489452in}{2.100505in}}%
\pgfpathlineto{\pgfqpoint{0.513638in}{2.100505in}}%
\pgfpathlineto{\pgfqpoint{0.521700in}{2.046403in}}%
\pgfpathlineto{\pgfqpoint{0.553948in}{2.046403in}}%
\pgfpathlineto{\pgfqpoint{0.562010in}{2.018727in}}%
\pgfpathlineto{\pgfqpoint{0.666815in}{2.018727in}}%
\pgfpathlineto{\pgfqpoint{0.674877in}{1.976686in}}%
\pgfpathlineto{\pgfqpoint{0.691001in}{1.976686in}}%
\pgfpathlineto{\pgfqpoint{0.699063in}{1.923039in}}%
\pgfpathlineto{\pgfqpoint{0.707125in}{1.923039in}}%
\pgfpathlineto{\pgfqpoint{0.715187in}{1.851813in}}%
\pgfpathlineto{\pgfqpoint{0.779683in}{1.851813in}}%
\pgfpathlineto{\pgfqpoint{0.787745in}{1.849509in}}%
\pgfpathlineto{\pgfqpoint{0.795807in}{1.849509in}}%
\pgfpathlineto{\pgfqpoint{0.803869in}{1.668466in}}%
\pgfpathlineto{\pgfqpoint{0.811931in}{1.668466in}}%
\pgfpathlineto{\pgfqpoint{0.819993in}{1.578835in}}%
\pgfpathlineto{\pgfqpoint{0.828055in}{1.578835in}}%
\pgfpathlineto{\pgfqpoint{0.836117in}{1.546619in}}%
\pgfpathlineto{\pgfqpoint{0.844179in}{1.546619in}}%
\pgfpathlineto{\pgfqpoint{0.852241in}{1.527155in}}%
\pgfpathlineto{\pgfqpoint{0.860302in}{1.527155in}}%
\pgfpathlineto{\pgfqpoint{0.868364in}{1.525364in}}%
\pgfpathlineto{\pgfqpoint{0.876426in}{1.525364in}}%
\pgfpathlineto{\pgfqpoint{0.884488in}{1.222621in}}%
\pgfpathlineto{\pgfqpoint{0.900612in}{1.222621in}}%
\pgfpathlineto{\pgfqpoint{0.908674in}{0.911514in}}%
\pgfpathlineto{\pgfqpoint{0.957046in}{0.911514in}}%
\pgfpathlineto{\pgfqpoint{0.965108in}{0.841881in}}%
\pgfpathlineto{\pgfqpoint{1.069914in}{0.841881in}}%
\pgfpathlineto{\pgfqpoint{1.077975in}{0.807546in}}%
\pgfpathlineto{\pgfqpoint{1.094099in}{0.807546in}}%
\pgfpathlineto{\pgfqpoint{1.102161in}{0.801250in}}%
\pgfpathlineto{\pgfqpoint{1.110223in}{0.784286in}}%
\pgfpathlineto{\pgfqpoint{1.279525in}{0.784286in}}%
\pgfpathlineto{\pgfqpoint{1.287586in}{0.773696in}}%
\pgfpathlineto{\pgfqpoint{1.400454in}{0.773696in}}%
\pgfpathlineto{\pgfqpoint{1.408516in}{0.645627in}}%
\pgfpathlineto{\pgfqpoint{1.424640in}{0.617979in}}%
\pgfpathlineto{\pgfqpoint{1.682623in}{0.616466in}}%
\pgfpathlineto{\pgfqpoint{1.690685in}{0.573534in}}%
\pgfpathlineto{\pgfqpoint{1.988977in}{0.573534in}}%
\pgfpathlineto{\pgfqpoint{1.988977in}{0.573534in}}%
\pgfusepath{stroke}%
\end{pgfscope}%
\begin{pgfscope}%
\pgfpathrectangle{\pgfqpoint{0.287500in}{0.375000in}}{\pgfqpoint{1.782500in}{2.265000in}}%
\pgfusepath{clip}%
\pgfsetroundcap%
\pgfsetroundjoin%
\pgfsetlinewidth{1.505625pt}%
\definecolor{currentstroke}{rgb}{0.172549,0.627451,0.172549}%
\pgfsetstrokecolor{currentstroke}%
\pgfsetdash{}{0pt}%
\pgfpathmoveto{\pgfqpoint{0.368523in}{2.502665in}}%
\pgfpathlineto{\pgfqpoint{0.376585in}{2.399190in}}%
\pgfpathlineto{\pgfqpoint{0.384647in}{2.388215in}}%
\pgfpathlineto{\pgfqpoint{0.400771in}{2.321766in}}%
\pgfpathlineto{\pgfqpoint{0.441080in}{2.321766in}}%
\pgfpathlineto{\pgfqpoint{0.449142in}{2.303875in}}%
\pgfpathlineto{\pgfqpoint{0.465266in}{2.303875in}}%
\pgfpathlineto{\pgfqpoint{0.473328in}{2.279907in}}%
\pgfpathlineto{\pgfqpoint{0.489452in}{2.279907in}}%
\pgfpathlineto{\pgfqpoint{0.497514in}{2.234123in}}%
\pgfpathlineto{\pgfqpoint{0.545886in}{2.233350in}}%
\pgfpathlineto{\pgfqpoint{0.562010in}{2.206572in}}%
\pgfpathlineto{\pgfqpoint{0.570072in}{2.168777in}}%
\pgfpathlineto{\pgfqpoint{0.586196in}{2.168777in}}%
\pgfpathlineto{\pgfqpoint{0.594258in}{2.152031in}}%
\pgfpathlineto{\pgfqpoint{0.602320in}{2.085896in}}%
\pgfpathlineto{\pgfqpoint{0.610382in}{1.880150in}}%
\pgfpathlineto{\pgfqpoint{0.699063in}{1.880150in}}%
\pgfpathlineto{\pgfqpoint{0.707125in}{1.782251in}}%
\pgfpathlineto{\pgfqpoint{0.811931in}{1.781670in}}%
\pgfpathlineto{\pgfqpoint{0.819993in}{1.667987in}}%
\pgfpathlineto{\pgfqpoint{0.836117in}{1.667987in}}%
\pgfpathlineto{\pgfqpoint{0.844179in}{1.663816in}}%
\pgfpathlineto{\pgfqpoint{0.876426in}{1.663816in}}%
\pgfpathlineto{\pgfqpoint{0.884488in}{1.637575in}}%
\pgfpathlineto{\pgfqpoint{0.892550in}{1.637575in}}%
\pgfpathlineto{\pgfqpoint{0.900612in}{1.545860in}}%
\pgfpathlineto{\pgfqpoint{0.916736in}{1.545860in}}%
\pgfpathlineto{\pgfqpoint{0.924798in}{1.474011in}}%
\pgfpathlineto{\pgfqpoint{0.965108in}{1.474011in}}%
\pgfpathlineto{\pgfqpoint{0.973170in}{1.465037in}}%
\pgfpathlineto{\pgfqpoint{0.989294in}{1.465037in}}%
\pgfpathlineto{\pgfqpoint{0.997356in}{1.357551in}}%
\pgfpathlineto{\pgfqpoint{1.021542in}{1.357551in}}%
\pgfpathlineto{\pgfqpoint{1.029604in}{1.349131in}}%
\pgfpathlineto{\pgfqpoint{1.045728in}{1.349131in}}%
\pgfpathlineto{\pgfqpoint{1.053790in}{1.284767in}}%
\pgfpathlineto{\pgfqpoint{1.061852in}{1.272708in}}%
\pgfpathlineto{\pgfqpoint{1.069914in}{1.223032in}}%
\pgfpathlineto{\pgfqpoint{1.118285in}{1.223032in}}%
\pgfpathlineto{\pgfqpoint{1.126347in}{1.165113in}}%
\pgfpathlineto{\pgfqpoint{1.134409in}{1.155360in}}%
\pgfpathlineto{\pgfqpoint{1.279525in}{1.155360in}}%
\pgfpathlineto{\pgfqpoint{1.287586in}{1.078061in}}%
\pgfpathlineto{\pgfqpoint{1.335958in}{1.078061in}}%
\pgfpathlineto{\pgfqpoint{1.344020in}{1.046001in}}%
\pgfpathlineto{\pgfqpoint{1.392392in}{1.045907in}}%
\pgfpathlineto{\pgfqpoint{1.400454in}{0.965419in}}%
\pgfpathlineto{\pgfqpoint{1.481074in}{0.965014in}}%
\pgfpathlineto{\pgfqpoint{1.489136in}{0.949256in}}%
\pgfpathlineto{\pgfqpoint{1.513321in}{0.949256in}}%
\pgfpathlineto{\pgfqpoint{1.521383in}{0.872455in}}%
\pgfpathlineto{\pgfqpoint{1.537507in}{0.872455in}}%
\pgfpathlineto{\pgfqpoint{1.545569in}{0.864272in}}%
\pgfpathlineto{\pgfqpoint{1.553631in}{0.800473in}}%
\pgfpathlineto{\pgfqpoint{1.819676in}{0.800473in}}%
\pgfpathlineto{\pgfqpoint{1.827738in}{0.790715in}}%
\pgfpathlineto{\pgfqpoint{1.988977in}{0.790459in}}%
\pgfpathlineto{\pgfqpoint{1.988977in}{0.790459in}}%
\pgfusepath{stroke}%
\end{pgfscope}%
\begin{pgfscope}%
\pgfpathrectangle{\pgfqpoint{0.287500in}{0.375000in}}{\pgfqpoint{1.782500in}{2.265000in}}%
\pgfusepath{clip}%
\pgfsetroundcap%
\pgfsetroundjoin%
\pgfsetlinewidth{1.505625pt}%
\definecolor{currentstroke}{rgb}{0.839216,0.152941,0.156863}%
\pgfsetstrokecolor{currentstroke}%
\pgfsetdash{}{0pt}%
\pgfpathmoveto{\pgfqpoint{0.368523in}{2.511722in}}%
\pgfpathlineto{\pgfqpoint{0.376585in}{2.469587in}}%
\pgfpathlineto{\pgfqpoint{0.384647in}{2.438589in}}%
\pgfpathlineto{\pgfqpoint{0.392709in}{2.298621in}}%
\pgfpathlineto{\pgfqpoint{0.400771in}{2.244446in}}%
\pgfpathlineto{\pgfqpoint{0.408833in}{2.131063in}}%
\pgfpathlineto{\pgfqpoint{0.416895in}{2.101007in}}%
\pgfpathlineto{\pgfqpoint{0.424956in}{2.101007in}}%
\pgfpathlineto{\pgfqpoint{0.433018in}{2.098860in}}%
\pgfpathlineto{\pgfqpoint{0.441080in}{2.053614in}}%
\pgfpathlineto{\pgfqpoint{0.465266in}{2.053215in}}%
\pgfpathlineto{\pgfqpoint{0.473328in}{2.049717in}}%
\pgfpathlineto{\pgfqpoint{0.626506in}{2.048476in}}%
\pgfpathlineto{\pgfqpoint{0.634568in}{2.043324in}}%
\pgfpathlineto{\pgfqpoint{0.674877in}{2.043290in}}%
\pgfpathlineto{\pgfqpoint{0.682939in}{2.039173in}}%
\pgfpathlineto{\pgfqpoint{0.779683in}{2.039173in}}%
\pgfpathlineto{\pgfqpoint{0.787745in}{1.947383in}}%
\pgfpathlineto{\pgfqpoint{0.989294in}{1.947383in}}%
\pgfpathlineto{\pgfqpoint{0.997356in}{1.945497in}}%
\pgfpathlineto{\pgfqpoint{1.198905in}{1.945497in}}%
\pgfpathlineto{\pgfqpoint{1.206967in}{1.941973in}}%
\pgfpathlineto{\pgfqpoint{1.674561in}{1.941973in}}%
\pgfpathlineto{\pgfqpoint{1.682623in}{1.657200in}}%
\pgfpathlineto{\pgfqpoint{1.698747in}{1.657200in}}%
\pgfpathlineto{\pgfqpoint{1.706809in}{1.428705in}}%
\pgfpathlineto{\pgfqpoint{1.747118in}{1.428705in}}%
\pgfpathlineto{\pgfqpoint{1.755180in}{1.406108in}}%
\pgfpathlineto{\pgfqpoint{1.859986in}{1.406108in}}%
\pgfpathlineto{\pgfqpoint{1.868048in}{1.398924in}}%
\pgfpathlineto{\pgfqpoint{1.900296in}{1.398753in}}%
\pgfpathlineto{\pgfqpoint{1.908358in}{1.293888in}}%
\pgfpathlineto{\pgfqpoint{1.988977in}{1.293888in}}%
\pgfpathlineto{\pgfqpoint{1.988977in}{1.293888in}}%
\pgfusepath{stroke}%
\end{pgfscope}%
\begin{pgfscope}%
\pgfsetrectcap%
\pgfsetmiterjoin%
\pgfsetlinewidth{0.000000pt}%
\definecolor{currentstroke}{rgb}{1.000000,1.000000,1.000000}%
\pgfsetstrokecolor{currentstroke}%
\pgfsetdash{}{0pt}%
\pgfpathmoveto{\pgfqpoint{0.287500in}{0.375000in}}%
\pgfpathlineto{\pgfqpoint{0.287500in}{2.640000in}}%
\pgfusepath{}%
\end{pgfscope}%
\begin{pgfscope}%
\pgfsetrectcap%
\pgfsetmiterjoin%
\pgfsetlinewidth{0.000000pt}%
\definecolor{currentstroke}{rgb}{1.000000,1.000000,1.000000}%
\pgfsetstrokecolor{currentstroke}%
\pgfsetdash{}{0pt}%
\pgfpathmoveto{\pgfqpoint{2.070000in}{0.375000in}}%
\pgfpathlineto{\pgfqpoint{2.070000in}{2.640000in}}%
\pgfusepath{}%
\end{pgfscope}%
\begin{pgfscope}%
\pgfsetrectcap%
\pgfsetmiterjoin%
\pgfsetlinewidth{0.000000pt}%
\definecolor{currentstroke}{rgb}{1.000000,1.000000,1.000000}%
\pgfsetstrokecolor{currentstroke}%
\pgfsetdash{}{0pt}%
\pgfpathmoveto{\pgfqpoint{0.287500in}{0.375000in}}%
\pgfpathlineto{\pgfqpoint{2.070000in}{0.375000in}}%
\pgfusepath{}%
\end{pgfscope}%
\begin{pgfscope}%
\pgfsetrectcap%
\pgfsetmiterjoin%
\pgfsetlinewidth{0.000000pt}%
\definecolor{currentstroke}{rgb}{1.000000,1.000000,1.000000}%
\pgfsetstrokecolor{currentstroke}%
\pgfsetdash{}{0pt}%
\pgfpathmoveto{\pgfqpoint{0.287500in}{2.640000in}}%
\pgfpathlineto{\pgfqpoint{2.070000in}{2.640000in}}%
\pgfusepath{}%
\end{pgfscope}%
\begin{pgfscope}%
\definecolor{textcolor}{rgb}{0.150000,0.150000,0.150000}%
\pgfsetstrokecolor{textcolor}%
\pgfsetfillcolor{textcolor}%
\pgftext[x=1.178750in,y=2.723333in,,base]{\color{textcolor}\rmfamily\fontsize{8.000000}{9.600000}\selectfont Embedded SinOne in 2D}%
\end{pgfscope}%
\begin{pgfscope}%
\pgfsetroundcap%
\pgfsetroundjoin%
\pgfsetlinewidth{1.505625pt}%
\definecolor{currentstroke}{rgb}{0.121569,0.466667,0.705882}%
\pgfsetstrokecolor{currentstroke}%
\pgfsetdash{}{0pt}%
\pgfpathmoveto{\pgfqpoint{0.772205in}{2.530853in}}%
\pgfpathlineto{\pgfqpoint{0.938872in}{2.530853in}}%
\pgfusepath{stroke}%
\end{pgfscope}%
\begin{pgfscope}%
\definecolor{textcolor}{rgb}{0.150000,0.150000,0.150000}%
\pgfsetstrokecolor{textcolor}%
\pgfsetfillcolor{textcolor}%
\pgftext[x=1.005539in,y=2.501686in,left,base]{\color{textcolor}\rmfamily\fontsize{6.000000}{7.200000}\selectfont random}%
\end{pgfscope}%
\begin{pgfscope}%
\pgfsetroundcap%
\pgfsetroundjoin%
\pgfsetlinewidth{1.505625pt}%
\definecolor{currentstroke}{rgb}{1.000000,0.498039,0.054902}%
\pgfsetstrokecolor{currentstroke}%
\pgfsetdash{}{0pt}%
\pgfpathmoveto{\pgfqpoint{0.772205in}{2.408538in}}%
\pgfpathlineto{\pgfqpoint{0.938872in}{2.408538in}}%
\pgfusepath{stroke}%
\end{pgfscope}%
\begin{pgfscope}%
\definecolor{textcolor}{rgb}{0.150000,0.150000,0.150000}%
\pgfsetstrokecolor{textcolor}%
\pgfsetfillcolor{textcolor}%
\pgftext[x=1.005539in,y=2.379372in,left,base]{\color{textcolor}\rmfamily\fontsize{6.000000}{7.200000}\selectfont 5 x DNGO retrain-reset}%
\end{pgfscope}%
\begin{pgfscope}%
\pgfsetroundcap%
\pgfsetroundjoin%
\pgfsetlinewidth{1.505625pt}%
\definecolor{currentstroke}{rgb}{0.172549,0.627451,0.172549}%
\pgfsetstrokecolor{currentstroke}%
\pgfsetdash{}{0pt}%
\pgfpathmoveto{\pgfqpoint{0.772205in}{2.286224in}}%
\pgfpathlineto{\pgfqpoint{0.938872in}{2.286224in}}%
\pgfusepath{stroke}%
\end{pgfscope}%
\begin{pgfscope}%
\definecolor{textcolor}{rgb}{0.150000,0.150000,0.150000}%
\pgfsetstrokecolor{textcolor}%
\pgfsetfillcolor{textcolor}%
\pgftext[x=1.005539in,y=2.257057in,left,base]{\color{textcolor}\rmfamily\fontsize{6.000000}{7.200000}\selectfont DNGO retrain-reset}%
\end{pgfscope}%
\begin{pgfscope}%
\pgfsetroundcap%
\pgfsetroundjoin%
\pgfsetlinewidth{1.505625pt}%
\definecolor{currentstroke}{rgb}{0.839216,0.152941,0.156863}%
\pgfsetstrokecolor{currentstroke}%
\pgfsetdash{}{0pt}%
\pgfpathmoveto{\pgfqpoint{0.772205in}{2.163910in}}%
\pgfpathlineto{\pgfqpoint{0.938872in}{2.163910in}}%
\pgfusepath{stroke}%
\end{pgfscope}%
\begin{pgfscope}%
\definecolor{textcolor}{rgb}{0.150000,0.150000,0.150000}%
\pgfsetstrokecolor{textcolor}%
\pgfsetfillcolor{textcolor}%
\pgftext[x=1.005539in,y=2.134743in,left,base]{\color{textcolor}\rmfamily\fontsize{6.000000}{7.200000}\selectfont GP}%
\end{pgfscope}%
\end{pgfpicture}%
\makeatother%
\endgroup%

            \end{subfigure}
            \begin{subfigure}[t]{0.3\textwidth}
                \centering
                % \resizebox{.95\linewidth}{!}{}
                %% Creator: Matplotlib, PGF backend
%%
%% To include the figure in your LaTeX document, write
%%   \input{<filename>.pgf}
%%
%% Make sure the required packages are loaded in your preamble
%%   \usepackage{pgf}
%%
%% Figures using additional raster images can only be included by \input if
%% they are in the same directory as the main LaTeX file. For loading figures
%% from other directories you can use the `import` package
%%   \usepackage{import}
%% and then include the figures with
%%   \import{<path to file>}{<filename>.pgf}
%%
%% Matplotlib used the following preamble
%%   \usepackage{gensymb}
%%   \usepackage{fontspec}
%%   \setmainfont{DejaVu Serif}
%%   \setsansfont{Arial}
%%   \setmonofont{DejaVu Sans Mono}
%%
\begingroup%
\makeatletter%
\begin{pgfpicture}%
\pgfpathrectangle{\pgfpointorigin}{\pgfqpoint{2.300000in}{3.000000in}}%
\pgfusepath{use as bounding box, clip}%
\begin{pgfscope}%
\pgfsetbuttcap%
\pgfsetmiterjoin%
\definecolor{currentfill}{rgb}{1.000000,1.000000,1.000000}%
\pgfsetfillcolor{currentfill}%
\pgfsetlinewidth{0.000000pt}%
\definecolor{currentstroke}{rgb}{1.000000,1.000000,1.000000}%
\pgfsetstrokecolor{currentstroke}%
\pgfsetdash{}{0pt}%
\pgfpathmoveto{\pgfqpoint{0.000000in}{0.000000in}}%
\pgfpathlineto{\pgfqpoint{2.300000in}{0.000000in}}%
\pgfpathlineto{\pgfqpoint{2.300000in}{3.000000in}}%
\pgfpathlineto{\pgfqpoint{0.000000in}{3.000000in}}%
\pgfpathclose%
\pgfusepath{fill}%
\end{pgfscope}%
\begin{pgfscope}%
\pgfsetbuttcap%
\pgfsetmiterjoin%
\definecolor{currentfill}{rgb}{0.917647,0.917647,0.949020}%
\pgfsetfillcolor{currentfill}%
\pgfsetlinewidth{0.000000pt}%
\definecolor{currentstroke}{rgb}{0.000000,0.000000,0.000000}%
\pgfsetstrokecolor{currentstroke}%
\pgfsetstrokeopacity{0.000000}%
\pgfsetdash{}{0pt}%
\pgfpathmoveto{\pgfqpoint{0.287500in}{0.375000in}}%
\pgfpathlineto{\pgfqpoint{2.070000in}{0.375000in}}%
\pgfpathlineto{\pgfqpoint{2.070000in}{2.640000in}}%
\pgfpathlineto{\pgfqpoint{0.287500in}{2.640000in}}%
\pgfpathclose%
\pgfusepath{fill}%
\end{pgfscope}%
\begin{pgfscope}%
\pgfpathrectangle{\pgfqpoint{0.287500in}{0.375000in}}{\pgfqpoint{1.782500in}{2.265000in}}%
\pgfusepath{clip}%
\pgfsetroundcap%
\pgfsetroundjoin%
\pgfsetlinewidth{0.803000pt}%
\definecolor{currentstroke}{rgb}{1.000000,1.000000,1.000000}%
\pgfsetstrokecolor{currentstroke}%
\pgfsetdash{}{0pt}%
\pgfpathmoveto{\pgfqpoint{0.368523in}{0.375000in}}%
\pgfpathlineto{\pgfqpoint{0.368523in}{2.640000in}}%
\pgfusepath{stroke}%
\end{pgfscope}%
\begin{pgfscope}%
\definecolor{textcolor}{rgb}{0.150000,0.150000,0.150000}%
\pgfsetstrokecolor{textcolor}%
\pgfsetfillcolor{textcolor}%
\pgftext[x=0.368523in,y=0.326389in,,top]{\color{textcolor}\rmfamily\fontsize{8.000000}{9.600000}\selectfont \(\displaystyle 0\)}%
\end{pgfscope}%
\begin{pgfscope}%
\pgfpathrectangle{\pgfqpoint{0.287500in}{0.375000in}}{\pgfqpoint{1.782500in}{2.265000in}}%
\pgfusepath{clip}%
\pgfsetroundcap%
\pgfsetroundjoin%
\pgfsetlinewidth{0.803000pt}%
\definecolor{currentstroke}{rgb}{1.000000,1.000000,1.000000}%
\pgfsetstrokecolor{currentstroke}%
\pgfsetdash{}{0pt}%
\pgfpathmoveto{\pgfqpoint{0.771621in}{0.375000in}}%
\pgfpathlineto{\pgfqpoint{0.771621in}{2.640000in}}%
\pgfusepath{stroke}%
\end{pgfscope}%
\begin{pgfscope}%
\definecolor{textcolor}{rgb}{0.150000,0.150000,0.150000}%
\pgfsetstrokecolor{textcolor}%
\pgfsetfillcolor{textcolor}%
\pgftext[x=0.771621in,y=0.326389in,,top]{\color{textcolor}\rmfamily\fontsize{8.000000}{9.600000}\selectfont \(\displaystyle 50\)}%
\end{pgfscope}%
\begin{pgfscope}%
\pgfpathrectangle{\pgfqpoint{0.287500in}{0.375000in}}{\pgfqpoint{1.782500in}{2.265000in}}%
\pgfusepath{clip}%
\pgfsetroundcap%
\pgfsetroundjoin%
\pgfsetlinewidth{0.803000pt}%
\definecolor{currentstroke}{rgb}{1.000000,1.000000,1.000000}%
\pgfsetstrokecolor{currentstroke}%
\pgfsetdash{}{0pt}%
\pgfpathmoveto{\pgfqpoint{1.174719in}{0.375000in}}%
\pgfpathlineto{\pgfqpoint{1.174719in}{2.640000in}}%
\pgfusepath{stroke}%
\end{pgfscope}%
\begin{pgfscope}%
\definecolor{textcolor}{rgb}{0.150000,0.150000,0.150000}%
\pgfsetstrokecolor{textcolor}%
\pgfsetfillcolor{textcolor}%
\pgftext[x=1.174719in,y=0.326389in,,top]{\color{textcolor}\rmfamily\fontsize{8.000000}{9.600000}\selectfont \(\displaystyle 100\)}%
\end{pgfscope}%
\begin{pgfscope}%
\pgfpathrectangle{\pgfqpoint{0.287500in}{0.375000in}}{\pgfqpoint{1.782500in}{2.265000in}}%
\pgfusepath{clip}%
\pgfsetroundcap%
\pgfsetroundjoin%
\pgfsetlinewidth{0.803000pt}%
\definecolor{currentstroke}{rgb}{1.000000,1.000000,1.000000}%
\pgfsetstrokecolor{currentstroke}%
\pgfsetdash{}{0pt}%
\pgfpathmoveto{\pgfqpoint{1.577817in}{0.375000in}}%
\pgfpathlineto{\pgfqpoint{1.577817in}{2.640000in}}%
\pgfusepath{stroke}%
\end{pgfscope}%
\begin{pgfscope}%
\definecolor{textcolor}{rgb}{0.150000,0.150000,0.150000}%
\pgfsetstrokecolor{textcolor}%
\pgfsetfillcolor{textcolor}%
\pgftext[x=1.577817in,y=0.326389in,,top]{\color{textcolor}\rmfamily\fontsize{8.000000}{9.600000}\selectfont \(\displaystyle 150\)}%
\end{pgfscope}%
\begin{pgfscope}%
\pgfpathrectangle{\pgfqpoint{0.287500in}{0.375000in}}{\pgfqpoint{1.782500in}{2.265000in}}%
\pgfusepath{clip}%
\pgfsetroundcap%
\pgfsetroundjoin%
\pgfsetlinewidth{0.803000pt}%
\definecolor{currentstroke}{rgb}{1.000000,1.000000,1.000000}%
\pgfsetstrokecolor{currentstroke}%
\pgfsetdash{}{0pt}%
\pgfpathmoveto{\pgfqpoint{1.980915in}{0.375000in}}%
\pgfpathlineto{\pgfqpoint{1.980915in}{2.640000in}}%
\pgfusepath{stroke}%
\end{pgfscope}%
\begin{pgfscope}%
\definecolor{textcolor}{rgb}{0.150000,0.150000,0.150000}%
\pgfsetstrokecolor{textcolor}%
\pgfsetfillcolor{textcolor}%
\pgftext[x=1.980915in,y=0.326389in,,top]{\color{textcolor}\rmfamily\fontsize{8.000000}{9.600000}\selectfont \(\displaystyle 200\)}%
\end{pgfscope}%
\begin{pgfscope}%
\definecolor{textcolor}{rgb}{0.150000,0.150000,0.150000}%
\pgfsetstrokecolor{textcolor}%
\pgfsetfillcolor{textcolor}%
\pgftext[x=1.178750in,y=0.163303in,,top]{\color{textcolor}\rmfamily\fontsize{8.000000}{9.600000}\selectfont Step}%
\end{pgfscope}%
\begin{pgfscope}%
\pgfpathrectangle{\pgfqpoint{0.287500in}{0.375000in}}{\pgfqpoint{1.782500in}{2.265000in}}%
\pgfusepath{clip}%
\pgfsetroundcap%
\pgfsetroundjoin%
\pgfsetlinewidth{0.803000pt}%
\definecolor{currentstroke}{rgb}{1.000000,1.000000,1.000000}%
\pgfsetstrokecolor{currentstroke}%
\pgfsetdash{}{0pt}%
\pgfpathmoveto{\pgfqpoint{0.287500in}{0.559856in}}%
\pgfpathlineto{\pgfqpoint{2.070000in}{0.559856in}}%
\pgfusepath{stroke}%
\end{pgfscope}%
\begin{pgfscope}%
\definecolor{textcolor}{rgb}{0.150000,0.150000,0.150000}%
\pgfsetstrokecolor{textcolor}%
\pgfsetfillcolor{textcolor}%
\pgftext[x=-0.017284in,y=0.517647in,left,base]{\color{textcolor}\rmfamily\fontsize{8.000000}{9.600000}\selectfont \(\displaystyle 10^{-6}\)}%
\end{pgfscope}%
\begin{pgfscope}%
\pgfpathrectangle{\pgfqpoint{0.287500in}{0.375000in}}{\pgfqpoint{1.782500in}{2.265000in}}%
\pgfusepath{clip}%
\pgfsetroundcap%
\pgfsetroundjoin%
\pgfsetlinewidth{0.803000pt}%
\definecolor{currentstroke}{rgb}{1.000000,1.000000,1.000000}%
\pgfsetstrokecolor{currentstroke}%
\pgfsetdash{}{0pt}%
\pgfpathmoveto{\pgfqpoint{0.287500in}{0.902189in}}%
\pgfpathlineto{\pgfqpoint{2.070000in}{0.902189in}}%
\pgfusepath{stroke}%
\end{pgfscope}%
\begin{pgfscope}%
\definecolor{textcolor}{rgb}{0.150000,0.150000,0.150000}%
\pgfsetstrokecolor{textcolor}%
\pgfsetfillcolor{textcolor}%
\pgftext[x=-0.017284in,y=0.859979in,left,base]{\color{textcolor}\rmfamily\fontsize{8.000000}{9.600000}\selectfont \(\displaystyle 10^{-5}\)}%
\end{pgfscope}%
\begin{pgfscope}%
\pgfpathrectangle{\pgfqpoint{0.287500in}{0.375000in}}{\pgfqpoint{1.782500in}{2.265000in}}%
\pgfusepath{clip}%
\pgfsetroundcap%
\pgfsetroundjoin%
\pgfsetlinewidth{0.803000pt}%
\definecolor{currentstroke}{rgb}{1.000000,1.000000,1.000000}%
\pgfsetstrokecolor{currentstroke}%
\pgfsetdash{}{0pt}%
\pgfpathmoveto{\pgfqpoint{0.287500in}{1.244521in}}%
\pgfpathlineto{\pgfqpoint{2.070000in}{1.244521in}}%
\pgfusepath{stroke}%
\end{pgfscope}%
\begin{pgfscope}%
\definecolor{textcolor}{rgb}{0.150000,0.150000,0.150000}%
\pgfsetstrokecolor{textcolor}%
\pgfsetfillcolor{textcolor}%
\pgftext[x=-0.017284in,y=1.202312in,left,base]{\color{textcolor}\rmfamily\fontsize{8.000000}{9.600000}\selectfont \(\displaystyle 10^{-4}\)}%
\end{pgfscope}%
\begin{pgfscope}%
\pgfpathrectangle{\pgfqpoint{0.287500in}{0.375000in}}{\pgfqpoint{1.782500in}{2.265000in}}%
\pgfusepath{clip}%
\pgfsetroundcap%
\pgfsetroundjoin%
\pgfsetlinewidth{0.803000pt}%
\definecolor{currentstroke}{rgb}{1.000000,1.000000,1.000000}%
\pgfsetstrokecolor{currentstroke}%
\pgfsetdash{}{0pt}%
\pgfpathmoveto{\pgfqpoint{0.287500in}{1.586853in}}%
\pgfpathlineto{\pgfqpoint{2.070000in}{1.586853in}}%
\pgfusepath{stroke}%
\end{pgfscope}%
\begin{pgfscope}%
\definecolor{textcolor}{rgb}{0.150000,0.150000,0.150000}%
\pgfsetstrokecolor{textcolor}%
\pgfsetfillcolor{textcolor}%
\pgftext[x=-0.017284in,y=1.544644in,left,base]{\color{textcolor}\rmfamily\fontsize{8.000000}{9.600000}\selectfont \(\displaystyle 10^{-3}\)}%
\end{pgfscope}%
\begin{pgfscope}%
\pgfpathrectangle{\pgfqpoint{0.287500in}{0.375000in}}{\pgfqpoint{1.782500in}{2.265000in}}%
\pgfusepath{clip}%
\pgfsetroundcap%
\pgfsetroundjoin%
\pgfsetlinewidth{0.803000pt}%
\definecolor{currentstroke}{rgb}{1.000000,1.000000,1.000000}%
\pgfsetstrokecolor{currentstroke}%
\pgfsetdash{}{0pt}%
\pgfpathmoveto{\pgfqpoint{0.287500in}{1.929186in}}%
\pgfpathlineto{\pgfqpoint{2.070000in}{1.929186in}}%
\pgfusepath{stroke}%
\end{pgfscope}%
\begin{pgfscope}%
\definecolor{textcolor}{rgb}{0.150000,0.150000,0.150000}%
\pgfsetstrokecolor{textcolor}%
\pgfsetfillcolor{textcolor}%
\pgftext[x=-0.017284in,y=1.886976in,left,base]{\color{textcolor}\rmfamily\fontsize{8.000000}{9.600000}\selectfont \(\displaystyle 10^{-2}\)}%
\end{pgfscope}%
\begin{pgfscope}%
\pgfpathrectangle{\pgfqpoint{0.287500in}{0.375000in}}{\pgfqpoint{1.782500in}{2.265000in}}%
\pgfusepath{clip}%
\pgfsetroundcap%
\pgfsetroundjoin%
\pgfsetlinewidth{0.803000pt}%
\definecolor{currentstroke}{rgb}{1.000000,1.000000,1.000000}%
\pgfsetstrokecolor{currentstroke}%
\pgfsetdash{}{0pt}%
\pgfpathmoveto{\pgfqpoint{0.287500in}{2.271518in}}%
\pgfpathlineto{\pgfqpoint{2.070000in}{2.271518in}}%
\pgfusepath{stroke}%
\end{pgfscope}%
\begin{pgfscope}%
\definecolor{textcolor}{rgb}{0.150000,0.150000,0.150000}%
\pgfsetstrokecolor{textcolor}%
\pgfsetfillcolor{textcolor}%
\pgftext[x=-0.017284in,y=2.229309in,left,base]{\color{textcolor}\rmfamily\fontsize{8.000000}{9.600000}\selectfont \(\displaystyle 10^{-1}\)}%
\end{pgfscope}%
\begin{pgfscope}%
\pgfpathrectangle{\pgfqpoint{0.287500in}{0.375000in}}{\pgfqpoint{1.782500in}{2.265000in}}%
\pgfusepath{clip}%
\pgfsetroundcap%
\pgfsetroundjoin%
\pgfsetlinewidth{0.803000pt}%
\definecolor{currentstroke}{rgb}{1.000000,1.000000,1.000000}%
\pgfsetstrokecolor{currentstroke}%
\pgfsetdash{}{0pt}%
\pgfpathmoveto{\pgfqpoint{0.287500in}{2.613850in}}%
\pgfpathlineto{\pgfqpoint{2.070000in}{2.613850in}}%
\pgfusepath{stroke}%
\end{pgfscope}%
\begin{pgfscope}%
\definecolor{textcolor}{rgb}{0.150000,0.150000,0.150000}%
\pgfsetstrokecolor{textcolor}%
\pgfsetfillcolor{textcolor}%
\pgftext[x=0.062962in,y=2.571641in,left,base]{\color{textcolor}\rmfamily\fontsize{8.000000}{9.600000}\selectfont \(\displaystyle 10^{0}\)}%
\end{pgfscope}%
\begin{pgfscope}%
\definecolor{textcolor}{rgb}{0.150000,0.150000,0.150000}%
\pgfsetstrokecolor{textcolor}%
\pgfsetfillcolor{textcolor}%
\pgftext[x=-0.072840in,y=1.507500in,,bottom,rotate=90.000000]{\color{textcolor}\rmfamily\fontsize{8.000000}{9.600000}\selectfont Simple Regret}%
\end{pgfscope}%
\begin{pgfscope}%
\pgfpathrectangle{\pgfqpoint{0.287500in}{0.375000in}}{\pgfqpoint{1.782500in}{2.265000in}}%
\pgfusepath{clip}%
\pgfsetbuttcap%
\pgfsetroundjoin%
\definecolor{currentfill}{rgb}{0.121569,0.466667,0.705882}%
\pgfsetfillcolor{currentfill}%
\pgfsetfillopacity{0.200000}%
\pgfsetlinewidth{0.000000pt}%
\definecolor{currentstroke}{rgb}{0.000000,0.000000,0.000000}%
\pgfsetstrokecolor{currentstroke}%
\pgfsetdash{}{0pt}%
\pgfpathmoveto{\pgfqpoint{0.368523in}{2.493626in}}%
\pgfpathlineto{\pgfqpoint{0.368523in}{2.537045in}}%
\pgfpathlineto{\pgfqpoint{0.376585in}{2.495202in}}%
\pgfpathlineto{\pgfqpoint{0.384647in}{2.448215in}}%
\pgfpathlineto{\pgfqpoint{0.392709in}{2.448215in}}%
\pgfpathlineto{\pgfqpoint{0.400771in}{2.437297in}}%
\pgfpathlineto{\pgfqpoint{0.408833in}{2.396083in}}%
\pgfpathlineto{\pgfqpoint{0.416895in}{2.359975in}}%
\pgfpathlineto{\pgfqpoint{0.424956in}{2.328553in}}%
\pgfpathlineto{\pgfqpoint{0.433018in}{2.321420in}}%
\pgfpathlineto{\pgfqpoint{0.441080in}{2.321420in}}%
\pgfpathlineto{\pgfqpoint{0.449142in}{2.321420in}}%
\pgfpathlineto{\pgfqpoint{0.457204in}{2.316937in}}%
\pgfpathlineto{\pgfqpoint{0.465266in}{2.316937in}}%
\pgfpathlineto{\pgfqpoint{0.473328in}{2.242351in}}%
\pgfpathlineto{\pgfqpoint{0.481390in}{2.218052in}}%
\pgfpathlineto{\pgfqpoint{0.489452in}{2.218052in}}%
\pgfpathlineto{\pgfqpoint{0.497514in}{2.218052in}}%
\pgfpathlineto{\pgfqpoint{0.505576in}{2.218052in}}%
\pgfpathlineto{\pgfqpoint{0.513638in}{2.218052in}}%
\pgfpathlineto{\pgfqpoint{0.521700in}{2.218052in}}%
\pgfpathlineto{\pgfqpoint{0.529762in}{2.201791in}}%
\pgfpathlineto{\pgfqpoint{0.537824in}{2.201791in}}%
\pgfpathlineto{\pgfqpoint{0.545886in}{2.189465in}}%
\pgfpathlineto{\pgfqpoint{0.553948in}{2.189465in}}%
\pgfpathlineto{\pgfqpoint{0.562010in}{2.189465in}}%
\pgfpathlineto{\pgfqpoint{0.570072in}{2.189465in}}%
\pgfpathlineto{\pgfqpoint{0.578134in}{2.189464in}}%
\pgfpathlineto{\pgfqpoint{0.586196in}{2.189464in}}%
\pgfpathlineto{\pgfqpoint{0.594258in}{2.189464in}}%
\pgfpathlineto{\pgfqpoint{0.602320in}{2.183737in}}%
\pgfpathlineto{\pgfqpoint{0.610382in}{2.183737in}}%
\pgfpathlineto{\pgfqpoint{0.618444in}{2.183737in}}%
\pgfpathlineto{\pgfqpoint{0.626506in}{2.183737in}}%
\pgfpathlineto{\pgfqpoint{0.634568in}{2.183737in}}%
\pgfpathlineto{\pgfqpoint{0.642629in}{2.183737in}}%
\pgfpathlineto{\pgfqpoint{0.650691in}{2.183737in}}%
\pgfpathlineto{\pgfqpoint{0.658753in}{2.183737in}}%
\pgfpathlineto{\pgfqpoint{0.666815in}{2.183737in}}%
\pgfpathlineto{\pgfqpoint{0.674877in}{2.183737in}}%
\pgfpathlineto{\pgfqpoint{0.682939in}{2.171263in}}%
\pgfpathlineto{\pgfqpoint{0.691001in}{2.169357in}}%
\pgfpathlineto{\pgfqpoint{0.699063in}{2.169357in}}%
\pgfpathlineto{\pgfqpoint{0.707125in}{2.167162in}}%
\pgfpathlineto{\pgfqpoint{0.715187in}{2.167162in}}%
\pgfpathlineto{\pgfqpoint{0.723249in}{2.167162in}}%
\pgfpathlineto{\pgfqpoint{0.731311in}{2.167162in}}%
\pgfpathlineto{\pgfqpoint{0.739373in}{2.167162in}}%
\pgfpathlineto{\pgfqpoint{0.747435in}{2.150059in}}%
\pgfpathlineto{\pgfqpoint{0.755497in}{2.150059in}}%
\pgfpathlineto{\pgfqpoint{0.763559in}{2.150059in}}%
\pgfpathlineto{\pgfqpoint{0.771621in}{2.150059in}}%
\pgfpathlineto{\pgfqpoint{0.779683in}{2.121961in}}%
\pgfpathlineto{\pgfqpoint{0.787745in}{2.121961in}}%
\pgfpathlineto{\pgfqpoint{0.795807in}{2.110744in}}%
\pgfpathlineto{\pgfqpoint{0.803869in}{2.110744in}}%
\pgfpathlineto{\pgfqpoint{0.811931in}{2.110744in}}%
\pgfpathlineto{\pgfqpoint{0.819993in}{2.110744in}}%
\pgfpathlineto{\pgfqpoint{0.828055in}{2.110744in}}%
\pgfpathlineto{\pgfqpoint{0.836117in}{2.110744in}}%
\pgfpathlineto{\pgfqpoint{0.844179in}{2.110744in}}%
\pgfpathlineto{\pgfqpoint{0.852241in}{2.110744in}}%
\pgfpathlineto{\pgfqpoint{0.860302in}{2.110744in}}%
\pgfpathlineto{\pgfqpoint{0.868364in}{2.110744in}}%
\pgfpathlineto{\pgfqpoint{0.876426in}{2.110744in}}%
\pgfpathlineto{\pgfqpoint{0.884488in}{2.110744in}}%
\pgfpathlineto{\pgfqpoint{0.892550in}{2.079965in}}%
\pgfpathlineto{\pgfqpoint{0.900612in}{2.079965in}}%
\pgfpathlineto{\pgfqpoint{0.908674in}{2.063963in}}%
\pgfpathlineto{\pgfqpoint{0.916736in}{2.063963in}}%
\pgfpathlineto{\pgfqpoint{0.924798in}{2.063963in}}%
\pgfpathlineto{\pgfqpoint{0.932860in}{2.063963in}}%
\pgfpathlineto{\pgfqpoint{0.940922in}{2.063963in}}%
\pgfpathlineto{\pgfqpoint{0.948984in}{2.063963in}}%
\pgfpathlineto{\pgfqpoint{0.957046in}{2.063963in}}%
\pgfpathlineto{\pgfqpoint{0.965108in}{2.063963in}}%
\pgfpathlineto{\pgfqpoint{0.973170in}{2.063963in}}%
\pgfpathlineto{\pgfqpoint{0.981232in}{2.063963in}}%
\pgfpathlineto{\pgfqpoint{0.989294in}{2.063963in}}%
\pgfpathlineto{\pgfqpoint{0.997356in}{2.063963in}}%
\pgfpathlineto{\pgfqpoint{1.005418in}{2.025568in}}%
\pgfpathlineto{\pgfqpoint{1.013480in}{2.025568in}}%
\pgfpathlineto{\pgfqpoint{1.021542in}{2.025568in}}%
\pgfpathlineto{\pgfqpoint{1.029604in}{2.025568in}}%
\pgfpathlineto{\pgfqpoint{1.037666in}{2.025568in}}%
\pgfpathlineto{\pgfqpoint{1.045728in}{1.924734in}}%
\pgfpathlineto{\pgfqpoint{1.053790in}{1.924734in}}%
\pgfpathlineto{\pgfqpoint{1.061852in}{1.924734in}}%
\pgfpathlineto{\pgfqpoint{1.069914in}{1.924734in}}%
\pgfpathlineto{\pgfqpoint{1.077975in}{1.924734in}}%
\pgfpathlineto{\pgfqpoint{1.086037in}{1.905586in}}%
\pgfpathlineto{\pgfqpoint{1.094099in}{1.905586in}}%
\pgfpathlineto{\pgfqpoint{1.102161in}{1.905586in}}%
\pgfpathlineto{\pgfqpoint{1.110223in}{1.905586in}}%
\pgfpathlineto{\pgfqpoint{1.118285in}{1.905586in}}%
\pgfpathlineto{\pgfqpoint{1.126347in}{1.905586in}}%
\pgfpathlineto{\pgfqpoint{1.134409in}{1.905586in}}%
\pgfpathlineto{\pgfqpoint{1.142471in}{1.884122in}}%
\pgfpathlineto{\pgfqpoint{1.150533in}{1.884122in}}%
\pgfpathlineto{\pgfqpoint{1.158595in}{1.884122in}}%
\pgfpathlineto{\pgfqpoint{1.166657in}{1.884122in}}%
\pgfpathlineto{\pgfqpoint{1.174719in}{1.884122in}}%
\pgfpathlineto{\pgfqpoint{1.182781in}{1.884122in}}%
\pgfpathlineto{\pgfqpoint{1.190843in}{1.884122in}}%
\pgfpathlineto{\pgfqpoint{1.198905in}{1.884122in}}%
\pgfpathlineto{\pgfqpoint{1.206967in}{1.884122in}}%
\pgfpathlineto{\pgfqpoint{1.215029in}{1.884122in}}%
\pgfpathlineto{\pgfqpoint{1.223091in}{1.884122in}}%
\pgfpathlineto{\pgfqpoint{1.231153in}{1.884122in}}%
\pgfpathlineto{\pgfqpoint{1.239215in}{1.884122in}}%
\pgfpathlineto{\pgfqpoint{1.247277in}{1.884122in}}%
\pgfpathlineto{\pgfqpoint{1.255339in}{1.884122in}}%
\pgfpathlineto{\pgfqpoint{1.263401in}{1.884122in}}%
\pgfpathlineto{\pgfqpoint{1.271463in}{1.884122in}}%
\pgfpathlineto{\pgfqpoint{1.279525in}{1.884122in}}%
\pgfpathlineto{\pgfqpoint{1.287586in}{1.884122in}}%
\pgfpathlineto{\pgfqpoint{1.295648in}{1.884122in}}%
\pgfpathlineto{\pgfqpoint{1.303710in}{1.884122in}}%
\pgfpathlineto{\pgfqpoint{1.311772in}{1.884122in}}%
\pgfpathlineto{\pgfqpoint{1.319834in}{1.884122in}}%
\pgfpathlineto{\pgfqpoint{1.327896in}{1.884122in}}%
\pgfpathlineto{\pgfqpoint{1.335958in}{1.884122in}}%
\pgfpathlineto{\pgfqpoint{1.344020in}{1.884122in}}%
\pgfpathlineto{\pgfqpoint{1.352082in}{1.884122in}}%
\pgfpathlineto{\pgfqpoint{1.360144in}{1.884122in}}%
\pgfpathlineto{\pgfqpoint{1.368206in}{1.884122in}}%
\pgfpathlineto{\pgfqpoint{1.376268in}{1.853253in}}%
\pgfpathlineto{\pgfqpoint{1.384330in}{1.853253in}}%
\pgfpathlineto{\pgfqpoint{1.392392in}{1.853253in}}%
\pgfpathlineto{\pgfqpoint{1.400454in}{1.765446in}}%
\pgfpathlineto{\pgfqpoint{1.408516in}{1.765446in}}%
\pgfpathlineto{\pgfqpoint{1.416578in}{1.765446in}}%
\pgfpathlineto{\pgfqpoint{1.424640in}{1.765446in}}%
\pgfpathlineto{\pgfqpoint{1.432702in}{1.765446in}}%
\pgfpathlineto{\pgfqpoint{1.440764in}{1.765446in}}%
\pgfpathlineto{\pgfqpoint{1.448826in}{1.765446in}}%
\pgfpathlineto{\pgfqpoint{1.456888in}{1.765446in}}%
\pgfpathlineto{\pgfqpoint{1.464950in}{1.717086in}}%
\pgfpathlineto{\pgfqpoint{1.473012in}{1.717086in}}%
\pgfpathlineto{\pgfqpoint{1.481074in}{1.717086in}}%
\pgfpathlineto{\pgfqpoint{1.489136in}{1.717086in}}%
\pgfpathlineto{\pgfqpoint{1.497198in}{1.717086in}}%
\pgfpathlineto{\pgfqpoint{1.505259in}{1.717086in}}%
\pgfpathlineto{\pgfqpoint{1.513321in}{1.717086in}}%
\pgfpathlineto{\pgfqpoint{1.521383in}{1.717086in}}%
\pgfpathlineto{\pgfqpoint{1.529445in}{1.717086in}}%
\pgfpathlineto{\pgfqpoint{1.537507in}{1.717086in}}%
\pgfpathlineto{\pgfqpoint{1.545569in}{1.717086in}}%
\pgfpathlineto{\pgfqpoint{1.553631in}{1.717086in}}%
\pgfpathlineto{\pgfqpoint{1.561693in}{1.717086in}}%
\pgfpathlineto{\pgfqpoint{1.569755in}{1.717086in}}%
\pgfpathlineto{\pgfqpoint{1.577817in}{1.717086in}}%
\pgfpathlineto{\pgfqpoint{1.585879in}{1.717086in}}%
\pgfpathlineto{\pgfqpoint{1.593941in}{1.717086in}}%
\pgfpathlineto{\pgfqpoint{1.602003in}{1.717086in}}%
\pgfpathlineto{\pgfqpoint{1.610065in}{1.717086in}}%
\pgfpathlineto{\pgfqpoint{1.618127in}{1.717086in}}%
\pgfpathlineto{\pgfqpoint{1.626189in}{1.717086in}}%
\pgfpathlineto{\pgfqpoint{1.634251in}{1.714005in}}%
\pgfpathlineto{\pgfqpoint{1.642313in}{1.714005in}}%
\pgfpathlineto{\pgfqpoint{1.650375in}{1.714005in}}%
\pgfpathlineto{\pgfqpoint{1.658437in}{1.714005in}}%
\pgfpathlineto{\pgfqpoint{1.666499in}{1.714005in}}%
\pgfpathlineto{\pgfqpoint{1.674561in}{1.714005in}}%
\pgfpathlineto{\pgfqpoint{1.682623in}{1.714005in}}%
\pgfpathlineto{\pgfqpoint{1.690685in}{1.714005in}}%
\pgfpathlineto{\pgfqpoint{1.698747in}{1.714005in}}%
\pgfpathlineto{\pgfqpoint{1.706809in}{1.714005in}}%
\pgfpathlineto{\pgfqpoint{1.714871in}{1.714005in}}%
\pgfpathlineto{\pgfqpoint{1.722932in}{1.714005in}}%
\pgfpathlineto{\pgfqpoint{1.730994in}{1.714005in}}%
\pgfpathlineto{\pgfqpoint{1.739056in}{1.714005in}}%
\pgfpathlineto{\pgfqpoint{1.747118in}{1.714005in}}%
\pgfpathlineto{\pgfqpoint{1.755180in}{1.714005in}}%
\pgfpathlineto{\pgfqpoint{1.763242in}{1.714005in}}%
\pgfpathlineto{\pgfqpoint{1.771304in}{1.714005in}}%
\pgfpathlineto{\pgfqpoint{1.779366in}{1.714005in}}%
\pgfpathlineto{\pgfqpoint{1.787428in}{1.714005in}}%
\pgfpathlineto{\pgfqpoint{1.795490in}{1.714005in}}%
\pgfpathlineto{\pgfqpoint{1.803552in}{1.714005in}}%
\pgfpathlineto{\pgfqpoint{1.811614in}{1.714005in}}%
\pgfpathlineto{\pgfqpoint{1.819676in}{1.714005in}}%
\pgfpathlineto{\pgfqpoint{1.827738in}{1.714005in}}%
\pgfpathlineto{\pgfqpoint{1.835800in}{1.714005in}}%
\pgfpathlineto{\pgfqpoint{1.843862in}{1.714005in}}%
\pgfpathlineto{\pgfqpoint{1.851924in}{1.714005in}}%
\pgfpathlineto{\pgfqpoint{1.859986in}{1.714005in}}%
\pgfpathlineto{\pgfqpoint{1.868048in}{1.714005in}}%
\pgfpathlineto{\pgfqpoint{1.876110in}{1.714005in}}%
\pgfpathlineto{\pgfqpoint{1.884172in}{1.713141in}}%
\pgfpathlineto{\pgfqpoint{1.892234in}{1.713141in}}%
\pgfpathlineto{\pgfqpoint{1.900296in}{1.713141in}}%
\pgfpathlineto{\pgfqpoint{1.908358in}{1.713141in}}%
\pgfpathlineto{\pgfqpoint{1.916420in}{1.713141in}}%
\pgfpathlineto{\pgfqpoint{1.924482in}{1.713141in}}%
\pgfpathlineto{\pgfqpoint{1.932544in}{1.713141in}}%
\pgfpathlineto{\pgfqpoint{1.940605in}{1.713141in}}%
\pgfpathlineto{\pgfqpoint{1.948667in}{1.713141in}}%
\pgfpathlineto{\pgfqpoint{1.956729in}{1.713141in}}%
\pgfpathlineto{\pgfqpoint{1.964791in}{1.713141in}}%
\pgfpathlineto{\pgfqpoint{1.972853in}{1.713141in}}%
\pgfpathlineto{\pgfqpoint{1.980915in}{1.713141in}}%
\pgfpathlineto{\pgfqpoint{1.988977in}{1.713141in}}%
\pgfpathlineto{\pgfqpoint{1.988977in}{1.471803in}}%
\pgfpathlineto{\pgfqpoint{1.988977in}{1.471803in}}%
\pgfpathlineto{\pgfqpoint{1.980915in}{1.471803in}}%
\pgfpathlineto{\pgfqpoint{1.972853in}{1.471803in}}%
\pgfpathlineto{\pgfqpoint{1.964791in}{1.471803in}}%
\pgfpathlineto{\pgfqpoint{1.956729in}{1.471803in}}%
\pgfpathlineto{\pgfqpoint{1.948667in}{1.471803in}}%
\pgfpathlineto{\pgfqpoint{1.940605in}{1.471803in}}%
\pgfpathlineto{\pgfqpoint{1.932544in}{1.471803in}}%
\pgfpathlineto{\pgfqpoint{1.924482in}{1.471803in}}%
\pgfpathlineto{\pgfqpoint{1.916420in}{1.471803in}}%
\pgfpathlineto{\pgfqpoint{1.908358in}{1.471803in}}%
\pgfpathlineto{\pgfqpoint{1.900296in}{1.471803in}}%
\pgfpathlineto{\pgfqpoint{1.892234in}{1.471803in}}%
\pgfpathlineto{\pgfqpoint{1.884172in}{1.471803in}}%
\pgfpathlineto{\pgfqpoint{1.876110in}{1.480466in}}%
\pgfpathlineto{\pgfqpoint{1.868048in}{1.480466in}}%
\pgfpathlineto{\pgfqpoint{1.859986in}{1.480466in}}%
\pgfpathlineto{\pgfqpoint{1.851924in}{1.480466in}}%
\pgfpathlineto{\pgfqpoint{1.843862in}{1.480466in}}%
\pgfpathlineto{\pgfqpoint{1.835800in}{1.480466in}}%
\pgfpathlineto{\pgfqpoint{1.827738in}{1.480466in}}%
\pgfpathlineto{\pgfqpoint{1.819676in}{1.480466in}}%
\pgfpathlineto{\pgfqpoint{1.811614in}{1.480466in}}%
\pgfpathlineto{\pgfqpoint{1.803552in}{1.480466in}}%
\pgfpathlineto{\pgfqpoint{1.795490in}{1.480466in}}%
\pgfpathlineto{\pgfqpoint{1.787428in}{1.480466in}}%
\pgfpathlineto{\pgfqpoint{1.779366in}{1.480466in}}%
\pgfpathlineto{\pgfqpoint{1.771304in}{1.480466in}}%
\pgfpathlineto{\pgfqpoint{1.763242in}{1.480466in}}%
\pgfpathlineto{\pgfqpoint{1.755180in}{1.480466in}}%
\pgfpathlineto{\pgfqpoint{1.747118in}{1.480466in}}%
\pgfpathlineto{\pgfqpoint{1.739056in}{1.480466in}}%
\pgfpathlineto{\pgfqpoint{1.730994in}{1.480466in}}%
\pgfpathlineto{\pgfqpoint{1.722932in}{1.480466in}}%
\pgfpathlineto{\pgfqpoint{1.714871in}{1.480466in}}%
\pgfpathlineto{\pgfqpoint{1.706809in}{1.480466in}}%
\pgfpathlineto{\pgfqpoint{1.698747in}{1.480466in}}%
\pgfpathlineto{\pgfqpoint{1.690685in}{1.480466in}}%
\pgfpathlineto{\pgfqpoint{1.682623in}{1.480466in}}%
\pgfpathlineto{\pgfqpoint{1.674561in}{1.480466in}}%
\pgfpathlineto{\pgfqpoint{1.666499in}{1.480466in}}%
\pgfpathlineto{\pgfqpoint{1.658437in}{1.480466in}}%
\pgfpathlineto{\pgfqpoint{1.650375in}{1.480466in}}%
\pgfpathlineto{\pgfqpoint{1.642313in}{1.480466in}}%
\pgfpathlineto{\pgfqpoint{1.634251in}{1.480466in}}%
\pgfpathlineto{\pgfqpoint{1.626189in}{1.501242in}}%
\pgfpathlineto{\pgfqpoint{1.618127in}{1.501242in}}%
\pgfpathlineto{\pgfqpoint{1.610065in}{1.501242in}}%
\pgfpathlineto{\pgfqpoint{1.602003in}{1.501242in}}%
\pgfpathlineto{\pgfqpoint{1.593941in}{1.501242in}}%
\pgfpathlineto{\pgfqpoint{1.585879in}{1.501242in}}%
\pgfpathlineto{\pgfqpoint{1.577817in}{1.501242in}}%
\pgfpathlineto{\pgfqpoint{1.569755in}{1.501242in}}%
\pgfpathlineto{\pgfqpoint{1.561693in}{1.501242in}}%
\pgfpathlineto{\pgfqpoint{1.553631in}{1.501242in}}%
\pgfpathlineto{\pgfqpoint{1.545569in}{1.501242in}}%
\pgfpathlineto{\pgfqpoint{1.537507in}{1.501242in}}%
\pgfpathlineto{\pgfqpoint{1.529445in}{1.501242in}}%
\pgfpathlineto{\pgfqpoint{1.521383in}{1.501242in}}%
\pgfpathlineto{\pgfqpoint{1.513321in}{1.501242in}}%
\pgfpathlineto{\pgfqpoint{1.505259in}{1.501242in}}%
\pgfpathlineto{\pgfqpoint{1.497198in}{1.501242in}}%
\pgfpathlineto{\pgfqpoint{1.489136in}{1.501242in}}%
\pgfpathlineto{\pgfqpoint{1.481074in}{1.501242in}}%
\pgfpathlineto{\pgfqpoint{1.473012in}{1.501242in}}%
\pgfpathlineto{\pgfqpoint{1.464950in}{1.501242in}}%
\pgfpathlineto{\pgfqpoint{1.456888in}{1.569403in}}%
\pgfpathlineto{\pgfqpoint{1.448826in}{1.569403in}}%
\pgfpathlineto{\pgfqpoint{1.440764in}{1.569403in}}%
\pgfpathlineto{\pgfqpoint{1.432702in}{1.569403in}}%
\pgfpathlineto{\pgfqpoint{1.424640in}{1.569403in}}%
\pgfpathlineto{\pgfqpoint{1.416578in}{1.569403in}}%
\pgfpathlineto{\pgfqpoint{1.408516in}{1.569403in}}%
\pgfpathlineto{\pgfqpoint{1.400454in}{1.569403in}}%
\pgfpathlineto{\pgfqpoint{1.392392in}{1.623352in}}%
\pgfpathlineto{\pgfqpoint{1.384330in}{1.623352in}}%
\pgfpathlineto{\pgfqpoint{1.376268in}{1.623352in}}%
\pgfpathlineto{\pgfqpoint{1.368206in}{1.652354in}}%
\pgfpathlineto{\pgfqpoint{1.360144in}{1.652354in}}%
\pgfpathlineto{\pgfqpoint{1.352082in}{1.652354in}}%
\pgfpathlineto{\pgfqpoint{1.344020in}{1.652354in}}%
\pgfpathlineto{\pgfqpoint{1.335958in}{1.652354in}}%
\pgfpathlineto{\pgfqpoint{1.327896in}{1.652354in}}%
\pgfpathlineto{\pgfqpoint{1.319834in}{1.652354in}}%
\pgfpathlineto{\pgfqpoint{1.311772in}{1.652354in}}%
\pgfpathlineto{\pgfqpoint{1.303710in}{1.652354in}}%
\pgfpathlineto{\pgfqpoint{1.295648in}{1.652354in}}%
\pgfpathlineto{\pgfqpoint{1.287586in}{1.652354in}}%
\pgfpathlineto{\pgfqpoint{1.279525in}{1.652354in}}%
\pgfpathlineto{\pgfqpoint{1.271463in}{1.652354in}}%
\pgfpathlineto{\pgfqpoint{1.263401in}{1.652354in}}%
\pgfpathlineto{\pgfqpoint{1.255339in}{1.652354in}}%
\pgfpathlineto{\pgfqpoint{1.247277in}{1.652354in}}%
\pgfpathlineto{\pgfqpoint{1.239215in}{1.652354in}}%
\pgfpathlineto{\pgfqpoint{1.231153in}{1.652354in}}%
\pgfpathlineto{\pgfqpoint{1.223091in}{1.652354in}}%
\pgfpathlineto{\pgfqpoint{1.215029in}{1.652354in}}%
\pgfpathlineto{\pgfqpoint{1.206967in}{1.652354in}}%
\pgfpathlineto{\pgfqpoint{1.198905in}{1.652354in}}%
\pgfpathlineto{\pgfqpoint{1.190843in}{1.652354in}}%
\pgfpathlineto{\pgfqpoint{1.182781in}{1.652354in}}%
\pgfpathlineto{\pgfqpoint{1.174719in}{1.652354in}}%
\pgfpathlineto{\pgfqpoint{1.166657in}{1.652354in}}%
\pgfpathlineto{\pgfqpoint{1.158595in}{1.652354in}}%
\pgfpathlineto{\pgfqpoint{1.150533in}{1.652354in}}%
\pgfpathlineto{\pgfqpoint{1.142471in}{1.652354in}}%
\pgfpathlineto{\pgfqpoint{1.134409in}{1.715564in}}%
\pgfpathlineto{\pgfqpoint{1.126347in}{1.715564in}}%
\pgfpathlineto{\pgfqpoint{1.118285in}{1.715564in}}%
\pgfpathlineto{\pgfqpoint{1.110223in}{1.715564in}}%
\pgfpathlineto{\pgfqpoint{1.102161in}{1.715564in}}%
\pgfpathlineto{\pgfqpoint{1.094099in}{1.715564in}}%
\pgfpathlineto{\pgfqpoint{1.086037in}{1.715564in}}%
\pgfpathlineto{\pgfqpoint{1.077975in}{1.747065in}}%
\pgfpathlineto{\pgfqpoint{1.069914in}{1.747065in}}%
\pgfpathlineto{\pgfqpoint{1.061852in}{1.747065in}}%
\pgfpathlineto{\pgfqpoint{1.053790in}{1.747065in}}%
\pgfpathlineto{\pgfqpoint{1.045728in}{1.747065in}}%
\pgfpathlineto{\pgfqpoint{1.037666in}{1.821385in}}%
\pgfpathlineto{\pgfqpoint{1.029604in}{1.821385in}}%
\pgfpathlineto{\pgfqpoint{1.021542in}{1.821385in}}%
\pgfpathlineto{\pgfqpoint{1.013480in}{1.821385in}}%
\pgfpathlineto{\pgfqpoint{1.005418in}{1.821385in}}%
\pgfpathlineto{\pgfqpoint{0.997356in}{1.859181in}}%
\pgfpathlineto{\pgfqpoint{0.989294in}{1.859181in}}%
\pgfpathlineto{\pgfqpoint{0.981232in}{1.859181in}}%
\pgfpathlineto{\pgfqpoint{0.973170in}{1.859181in}}%
\pgfpathlineto{\pgfqpoint{0.965108in}{1.859181in}}%
\pgfpathlineto{\pgfqpoint{0.957046in}{1.859181in}}%
\pgfpathlineto{\pgfqpoint{0.948984in}{1.859181in}}%
\pgfpathlineto{\pgfqpoint{0.940922in}{1.859181in}}%
\pgfpathlineto{\pgfqpoint{0.932860in}{1.859181in}}%
\pgfpathlineto{\pgfqpoint{0.924798in}{1.859181in}}%
\pgfpathlineto{\pgfqpoint{0.916736in}{1.859181in}}%
\pgfpathlineto{\pgfqpoint{0.908674in}{1.859181in}}%
\pgfpathlineto{\pgfqpoint{0.900612in}{1.899268in}}%
\pgfpathlineto{\pgfqpoint{0.892550in}{1.899268in}}%
\pgfpathlineto{\pgfqpoint{0.884488in}{1.967927in}}%
\pgfpathlineto{\pgfqpoint{0.876426in}{1.967927in}}%
\pgfpathlineto{\pgfqpoint{0.868364in}{1.967927in}}%
\pgfpathlineto{\pgfqpoint{0.860302in}{1.967927in}}%
\pgfpathlineto{\pgfqpoint{0.852241in}{1.967927in}}%
\pgfpathlineto{\pgfqpoint{0.844179in}{1.967927in}}%
\pgfpathlineto{\pgfqpoint{0.836117in}{1.967927in}}%
\pgfpathlineto{\pgfqpoint{0.828055in}{1.967927in}}%
\pgfpathlineto{\pgfqpoint{0.819993in}{1.967927in}}%
\pgfpathlineto{\pgfqpoint{0.811931in}{1.967927in}}%
\pgfpathlineto{\pgfqpoint{0.803869in}{1.967927in}}%
\pgfpathlineto{\pgfqpoint{0.795807in}{1.967927in}}%
\pgfpathlineto{\pgfqpoint{0.787745in}{1.978623in}}%
\pgfpathlineto{\pgfqpoint{0.779683in}{1.978623in}}%
\pgfpathlineto{\pgfqpoint{0.771621in}{1.985125in}}%
\pgfpathlineto{\pgfqpoint{0.763559in}{1.985125in}}%
\pgfpathlineto{\pgfqpoint{0.755497in}{1.985125in}}%
\pgfpathlineto{\pgfqpoint{0.747435in}{1.985125in}}%
\pgfpathlineto{\pgfqpoint{0.739373in}{1.981446in}}%
\pgfpathlineto{\pgfqpoint{0.731311in}{1.981446in}}%
\pgfpathlineto{\pgfqpoint{0.723249in}{1.981446in}}%
\pgfpathlineto{\pgfqpoint{0.715187in}{1.981446in}}%
\pgfpathlineto{\pgfqpoint{0.707125in}{1.981446in}}%
\pgfpathlineto{\pgfqpoint{0.699063in}{1.985180in}}%
\pgfpathlineto{\pgfqpoint{0.691001in}{1.985180in}}%
\pgfpathlineto{\pgfqpoint{0.682939in}{1.984589in}}%
\pgfpathlineto{\pgfqpoint{0.674877in}{2.034832in}}%
\pgfpathlineto{\pgfqpoint{0.666815in}{2.034832in}}%
\pgfpathlineto{\pgfqpoint{0.658753in}{2.034832in}}%
\pgfpathlineto{\pgfqpoint{0.650691in}{2.034832in}}%
\pgfpathlineto{\pgfqpoint{0.642629in}{2.034832in}}%
\pgfpathlineto{\pgfqpoint{0.634568in}{2.034832in}}%
\pgfpathlineto{\pgfqpoint{0.626506in}{2.034832in}}%
\pgfpathlineto{\pgfqpoint{0.618444in}{2.034832in}}%
\pgfpathlineto{\pgfqpoint{0.610382in}{2.034832in}}%
\pgfpathlineto{\pgfqpoint{0.602320in}{2.034832in}}%
\pgfpathlineto{\pgfqpoint{0.594258in}{2.063318in}}%
\pgfpathlineto{\pgfqpoint{0.586196in}{2.063318in}}%
\pgfpathlineto{\pgfqpoint{0.578134in}{2.063318in}}%
\pgfpathlineto{\pgfqpoint{0.570072in}{2.063329in}}%
\pgfpathlineto{\pgfqpoint{0.562010in}{2.063329in}}%
\pgfpathlineto{\pgfqpoint{0.553948in}{2.063329in}}%
\pgfpathlineto{\pgfqpoint{0.545886in}{2.063329in}}%
\pgfpathlineto{\pgfqpoint{0.537824in}{2.079949in}}%
\pgfpathlineto{\pgfqpoint{0.529762in}{2.079949in}}%
\pgfpathlineto{\pgfqpoint{0.521700in}{2.092956in}}%
\pgfpathlineto{\pgfqpoint{0.513638in}{2.092956in}}%
\pgfpathlineto{\pgfqpoint{0.505576in}{2.092956in}}%
\pgfpathlineto{\pgfqpoint{0.497514in}{2.092956in}}%
\pgfpathlineto{\pgfqpoint{0.489452in}{2.092956in}}%
\pgfpathlineto{\pgfqpoint{0.481390in}{2.092956in}}%
\pgfpathlineto{\pgfqpoint{0.473328in}{2.103147in}}%
\pgfpathlineto{\pgfqpoint{0.465266in}{2.149539in}}%
\pgfpathlineto{\pgfqpoint{0.457204in}{2.149539in}}%
\pgfpathlineto{\pgfqpoint{0.449142in}{2.147666in}}%
\pgfpathlineto{\pgfqpoint{0.441080in}{2.147666in}}%
\pgfpathlineto{\pgfqpoint{0.433018in}{2.147666in}}%
\pgfpathlineto{\pgfqpoint{0.424956in}{2.173301in}}%
\pgfpathlineto{\pgfqpoint{0.416895in}{2.216270in}}%
\pgfpathlineto{\pgfqpoint{0.408833in}{2.301672in}}%
\pgfpathlineto{\pgfqpoint{0.400771in}{2.329246in}}%
\pgfpathlineto{\pgfqpoint{0.392709in}{2.352059in}}%
\pgfpathlineto{\pgfqpoint{0.384647in}{2.352059in}}%
\pgfpathlineto{\pgfqpoint{0.376585in}{2.407676in}}%
\pgfpathlineto{\pgfqpoint{0.368523in}{2.493626in}}%
\pgfpathclose%
\pgfusepath{fill}%
\end{pgfscope}%
\begin{pgfscope}%
\pgfpathrectangle{\pgfqpoint{0.287500in}{0.375000in}}{\pgfqpoint{1.782500in}{2.265000in}}%
\pgfusepath{clip}%
\pgfsetbuttcap%
\pgfsetroundjoin%
\definecolor{currentfill}{rgb}{1.000000,0.498039,0.054902}%
\pgfsetfillcolor{currentfill}%
\pgfsetfillopacity{0.200000}%
\pgfsetlinewidth{0.000000pt}%
\definecolor{currentstroke}{rgb}{0.000000,0.000000,0.000000}%
\pgfsetstrokecolor{currentstroke}%
\pgfsetdash{}{0pt}%
\pgfpathmoveto{\pgfqpoint{0.368523in}{2.430236in}}%
\pgfpathlineto{\pgfqpoint{0.368523in}{2.481047in}}%
\pgfpathlineto{\pgfqpoint{0.376585in}{2.441953in}}%
\pgfpathlineto{\pgfqpoint{0.384647in}{2.441953in}}%
\pgfpathlineto{\pgfqpoint{0.392709in}{2.439197in}}%
\pgfpathlineto{\pgfqpoint{0.400771in}{2.439197in}}%
\pgfpathlineto{\pgfqpoint{0.408833in}{2.439197in}}%
\pgfpathlineto{\pgfqpoint{0.416895in}{2.356277in}}%
\pgfpathlineto{\pgfqpoint{0.424956in}{2.356277in}}%
\pgfpathlineto{\pgfqpoint{0.433018in}{2.356277in}}%
\pgfpathlineto{\pgfqpoint{0.441080in}{2.356277in}}%
\pgfpathlineto{\pgfqpoint{0.449142in}{2.356277in}}%
\pgfpathlineto{\pgfqpoint{0.457204in}{2.356277in}}%
\pgfpathlineto{\pgfqpoint{0.465266in}{2.323815in}}%
\pgfpathlineto{\pgfqpoint{0.473328in}{2.323815in}}%
\pgfpathlineto{\pgfqpoint{0.481390in}{2.323815in}}%
\pgfpathlineto{\pgfqpoint{0.489452in}{2.323815in}}%
\pgfpathlineto{\pgfqpoint{0.497514in}{2.323815in}}%
\pgfpathlineto{\pgfqpoint{0.505576in}{2.323815in}}%
\pgfpathlineto{\pgfqpoint{0.513638in}{2.323815in}}%
\pgfpathlineto{\pgfqpoint{0.521700in}{2.323815in}}%
\pgfpathlineto{\pgfqpoint{0.529762in}{2.323815in}}%
\pgfpathlineto{\pgfqpoint{0.537824in}{2.271500in}}%
\pgfpathlineto{\pgfqpoint{0.545886in}{2.261363in}}%
\pgfpathlineto{\pgfqpoint{0.553948in}{2.261363in}}%
\pgfpathlineto{\pgfqpoint{0.562010in}{2.261363in}}%
\pgfpathlineto{\pgfqpoint{0.570072in}{2.261363in}}%
\pgfpathlineto{\pgfqpoint{0.578134in}{2.261363in}}%
\pgfpathlineto{\pgfqpoint{0.586196in}{2.261363in}}%
\pgfpathlineto{\pgfqpoint{0.594258in}{2.261363in}}%
\pgfpathlineto{\pgfqpoint{0.602320in}{2.261363in}}%
\pgfpathlineto{\pgfqpoint{0.610382in}{2.193454in}}%
\pgfpathlineto{\pgfqpoint{0.618444in}{2.193454in}}%
\pgfpathlineto{\pgfqpoint{0.626506in}{2.193454in}}%
\pgfpathlineto{\pgfqpoint{0.634568in}{2.193454in}}%
\pgfpathlineto{\pgfqpoint{0.642629in}{2.193454in}}%
\pgfpathlineto{\pgfqpoint{0.650691in}{2.193454in}}%
\pgfpathlineto{\pgfqpoint{0.658753in}{2.193454in}}%
\pgfpathlineto{\pgfqpoint{0.666815in}{2.193454in}}%
\pgfpathlineto{\pgfqpoint{0.674877in}{2.193454in}}%
\pgfpathlineto{\pgfqpoint{0.682939in}{2.193454in}}%
\pgfpathlineto{\pgfqpoint{0.691001in}{2.193454in}}%
\pgfpathlineto{\pgfqpoint{0.699063in}{2.193454in}}%
\pgfpathlineto{\pgfqpoint{0.707125in}{2.193454in}}%
\pgfpathlineto{\pgfqpoint{0.715187in}{2.193454in}}%
\pgfpathlineto{\pgfqpoint{0.723249in}{2.193454in}}%
\pgfpathlineto{\pgfqpoint{0.731311in}{2.193454in}}%
\pgfpathlineto{\pgfqpoint{0.739373in}{2.127652in}}%
\pgfpathlineto{\pgfqpoint{0.747435in}{2.127652in}}%
\pgfpathlineto{\pgfqpoint{0.755497in}{2.123389in}}%
\pgfpathlineto{\pgfqpoint{0.763559in}{2.115860in}}%
\pgfpathlineto{\pgfqpoint{0.771621in}{2.115860in}}%
\pgfpathlineto{\pgfqpoint{0.779683in}{2.115860in}}%
\pgfpathlineto{\pgfqpoint{0.787745in}{2.115860in}}%
\pgfpathlineto{\pgfqpoint{0.795807in}{2.115860in}}%
\pgfpathlineto{\pgfqpoint{0.803869in}{2.115860in}}%
\pgfpathlineto{\pgfqpoint{0.811931in}{2.115860in}}%
\pgfpathlineto{\pgfqpoint{0.819993in}{2.115860in}}%
\pgfpathlineto{\pgfqpoint{0.828055in}{2.115860in}}%
\pgfpathlineto{\pgfqpoint{0.836117in}{2.107522in}}%
\pgfpathlineto{\pgfqpoint{0.844179in}{2.107522in}}%
\pgfpathlineto{\pgfqpoint{0.852241in}{2.106249in}}%
\pgfpathlineto{\pgfqpoint{0.860302in}{2.106249in}}%
\pgfpathlineto{\pgfqpoint{0.868364in}{2.105959in}}%
\pgfpathlineto{\pgfqpoint{0.876426in}{2.105959in}}%
\pgfpathlineto{\pgfqpoint{0.884488in}{2.105711in}}%
\pgfpathlineto{\pgfqpoint{0.892550in}{2.105711in}}%
\pgfpathlineto{\pgfqpoint{0.900612in}{2.105711in}}%
\pgfpathlineto{\pgfqpoint{0.908674in}{2.105711in}}%
\pgfpathlineto{\pgfqpoint{0.916736in}{2.105104in}}%
\pgfpathlineto{\pgfqpoint{0.924798in}{2.105025in}}%
\pgfpathlineto{\pgfqpoint{0.932860in}{2.105025in}}%
\pgfpathlineto{\pgfqpoint{0.940922in}{2.105025in}}%
\pgfpathlineto{\pgfqpoint{0.948984in}{1.933101in}}%
\pgfpathlineto{\pgfqpoint{0.957046in}{1.933101in}}%
\pgfpathlineto{\pgfqpoint{0.965108in}{1.933101in}}%
\pgfpathlineto{\pgfqpoint{0.973170in}{1.932114in}}%
\pgfpathlineto{\pgfqpoint{0.981232in}{1.932114in}}%
\pgfpathlineto{\pgfqpoint{0.989294in}{1.932114in}}%
\pgfpathlineto{\pgfqpoint{0.997356in}{1.859620in}}%
\pgfpathlineto{\pgfqpoint{1.005418in}{1.858973in}}%
\pgfpathlineto{\pgfqpoint{1.013480in}{1.858973in}}%
\pgfpathlineto{\pgfqpoint{1.021542in}{1.858973in}}%
\pgfpathlineto{\pgfqpoint{1.029604in}{1.858973in}}%
\pgfpathlineto{\pgfqpoint{1.037666in}{1.858973in}}%
\pgfpathlineto{\pgfqpoint{1.045728in}{1.858973in}}%
\pgfpathlineto{\pgfqpoint{1.053790in}{1.858973in}}%
\pgfpathlineto{\pgfqpoint{1.061852in}{1.858973in}}%
\pgfpathlineto{\pgfqpoint{1.069914in}{1.858973in}}%
\pgfpathlineto{\pgfqpoint{1.077975in}{1.858973in}}%
\pgfpathlineto{\pgfqpoint{1.086037in}{1.858973in}}%
\pgfpathlineto{\pgfqpoint{1.094099in}{1.858973in}}%
\pgfpathlineto{\pgfqpoint{1.102161in}{1.858931in}}%
\pgfpathlineto{\pgfqpoint{1.110223in}{1.888380in}}%
\pgfpathlineto{\pgfqpoint{1.118285in}{1.797899in}}%
\pgfpathlineto{\pgfqpoint{1.126347in}{1.797899in}}%
\pgfpathlineto{\pgfqpoint{1.134409in}{1.797899in}}%
\pgfpathlineto{\pgfqpoint{1.142471in}{1.575853in}}%
\pgfpathlineto{\pgfqpoint{1.150533in}{1.575853in}}%
\pgfpathlineto{\pgfqpoint{1.158595in}{1.575853in}}%
\pgfpathlineto{\pgfqpoint{1.166657in}{1.575853in}}%
\pgfpathlineto{\pgfqpoint{1.174719in}{1.575853in}}%
\pgfpathlineto{\pgfqpoint{1.182781in}{1.575853in}}%
\pgfpathlineto{\pgfqpoint{1.190843in}{1.575853in}}%
\pgfpathlineto{\pgfqpoint{1.198905in}{1.575853in}}%
\pgfpathlineto{\pgfqpoint{1.206967in}{1.575853in}}%
\pgfpathlineto{\pgfqpoint{1.215029in}{1.575853in}}%
\pgfpathlineto{\pgfqpoint{1.223091in}{1.575853in}}%
\pgfpathlineto{\pgfqpoint{1.231153in}{1.575853in}}%
\pgfpathlineto{\pgfqpoint{1.239215in}{1.575853in}}%
\pgfpathlineto{\pgfqpoint{1.247277in}{1.575853in}}%
\pgfpathlineto{\pgfqpoint{1.255339in}{1.575853in}}%
\pgfpathlineto{\pgfqpoint{1.263401in}{1.575853in}}%
\pgfpathlineto{\pgfqpoint{1.271463in}{1.575853in}}%
\pgfpathlineto{\pgfqpoint{1.279525in}{1.575853in}}%
\pgfpathlineto{\pgfqpoint{1.287586in}{1.066712in}}%
\pgfpathlineto{\pgfqpoint{1.295648in}{1.066712in}}%
\pgfpathlineto{\pgfqpoint{1.303710in}{1.066712in}}%
\pgfpathlineto{\pgfqpoint{1.311772in}{1.066712in}}%
\pgfpathlineto{\pgfqpoint{1.319834in}{1.066712in}}%
\pgfpathlineto{\pgfqpoint{1.327896in}{1.066712in}}%
\pgfpathlineto{\pgfqpoint{1.335958in}{1.066712in}}%
\pgfpathlineto{\pgfqpoint{1.344020in}{1.066712in}}%
\pgfpathlineto{\pgfqpoint{1.352082in}{1.066712in}}%
\pgfpathlineto{\pgfqpoint{1.360144in}{1.066582in}}%
\pgfpathlineto{\pgfqpoint{1.368206in}{1.066582in}}%
\pgfpathlineto{\pgfqpoint{1.376268in}{1.066582in}}%
\pgfpathlineto{\pgfqpoint{1.384330in}{1.066582in}}%
\pgfpathlineto{\pgfqpoint{1.392392in}{1.066582in}}%
\pgfpathlineto{\pgfqpoint{1.400454in}{1.066582in}}%
\pgfpathlineto{\pgfqpoint{1.408516in}{1.066582in}}%
\pgfpathlineto{\pgfqpoint{1.416578in}{1.066582in}}%
\pgfpathlineto{\pgfqpoint{1.424640in}{1.066582in}}%
\pgfpathlineto{\pgfqpoint{1.432702in}{1.066582in}}%
\pgfpathlineto{\pgfqpoint{1.440764in}{1.066582in}}%
\pgfpathlineto{\pgfqpoint{1.448826in}{1.066582in}}%
\pgfpathlineto{\pgfqpoint{1.456888in}{1.066582in}}%
\pgfpathlineto{\pgfqpoint{1.464950in}{1.066582in}}%
\pgfpathlineto{\pgfqpoint{1.473012in}{1.066582in}}%
\pgfpathlineto{\pgfqpoint{1.481074in}{1.066582in}}%
\pgfpathlineto{\pgfqpoint{1.489136in}{1.066582in}}%
\pgfpathlineto{\pgfqpoint{1.497198in}{1.066582in}}%
\pgfpathlineto{\pgfqpoint{1.505259in}{1.066582in}}%
\pgfpathlineto{\pgfqpoint{1.513321in}{1.066582in}}%
\pgfpathlineto{\pgfqpoint{1.521383in}{1.066582in}}%
\pgfpathlineto{\pgfqpoint{1.529445in}{1.066582in}}%
\pgfpathlineto{\pgfqpoint{1.537507in}{1.066582in}}%
\pgfpathlineto{\pgfqpoint{1.545569in}{1.066582in}}%
\pgfpathlineto{\pgfqpoint{1.553631in}{1.066582in}}%
\pgfpathlineto{\pgfqpoint{1.561693in}{1.066582in}}%
\pgfpathlineto{\pgfqpoint{1.569755in}{1.066582in}}%
\pgfpathlineto{\pgfqpoint{1.577817in}{1.066582in}}%
\pgfpathlineto{\pgfqpoint{1.585879in}{1.066582in}}%
\pgfpathlineto{\pgfqpoint{1.593941in}{0.885932in}}%
\pgfpathlineto{\pgfqpoint{1.602003in}{0.885932in}}%
\pgfpathlineto{\pgfqpoint{1.610065in}{0.885932in}}%
\pgfpathlineto{\pgfqpoint{1.618127in}{0.885932in}}%
\pgfpathlineto{\pgfqpoint{1.626189in}{0.885932in}}%
\pgfpathlineto{\pgfqpoint{1.634251in}{0.885932in}}%
\pgfpathlineto{\pgfqpoint{1.642313in}{0.885932in}}%
\pgfpathlineto{\pgfqpoint{1.650375in}{0.885932in}}%
\pgfpathlineto{\pgfqpoint{1.658437in}{0.885932in}}%
\pgfpathlineto{\pgfqpoint{1.666499in}{0.885932in}}%
\pgfpathlineto{\pgfqpoint{1.674561in}{0.885932in}}%
\pgfpathlineto{\pgfqpoint{1.682623in}{0.885932in}}%
\pgfpathlineto{\pgfqpoint{1.690685in}{0.885932in}}%
\pgfpathlineto{\pgfqpoint{1.698747in}{0.885932in}}%
\pgfpathlineto{\pgfqpoint{1.706809in}{0.885932in}}%
\pgfpathlineto{\pgfqpoint{1.714871in}{0.885932in}}%
\pgfpathlineto{\pgfqpoint{1.722932in}{0.885932in}}%
\pgfpathlineto{\pgfqpoint{1.730994in}{0.885932in}}%
\pgfpathlineto{\pgfqpoint{1.739056in}{0.885932in}}%
\pgfpathlineto{\pgfqpoint{1.747118in}{0.885932in}}%
\pgfpathlineto{\pgfqpoint{1.755180in}{0.885932in}}%
\pgfpathlineto{\pgfqpoint{1.763242in}{0.700699in}}%
\pgfpathlineto{\pgfqpoint{1.771304in}{0.700699in}}%
\pgfpathlineto{\pgfqpoint{1.779366in}{0.700699in}}%
\pgfpathlineto{\pgfqpoint{1.787428in}{0.700699in}}%
\pgfpathlineto{\pgfqpoint{1.795490in}{0.700699in}}%
\pgfpathlineto{\pgfqpoint{1.803552in}{0.700699in}}%
\pgfpathlineto{\pgfqpoint{1.811614in}{0.700699in}}%
\pgfpathlineto{\pgfqpoint{1.819676in}{0.700699in}}%
\pgfpathlineto{\pgfqpoint{1.827738in}{0.700699in}}%
\pgfpathlineto{\pgfqpoint{1.835800in}{0.700699in}}%
\pgfpathlineto{\pgfqpoint{1.843862in}{0.700699in}}%
\pgfpathlineto{\pgfqpoint{1.851924in}{0.700699in}}%
\pgfpathlineto{\pgfqpoint{1.859986in}{0.700699in}}%
\pgfpathlineto{\pgfqpoint{1.868048in}{0.700699in}}%
\pgfpathlineto{\pgfqpoint{1.876110in}{0.700699in}}%
\pgfpathlineto{\pgfqpoint{1.884172in}{0.700699in}}%
\pgfpathlineto{\pgfqpoint{1.892234in}{0.700699in}}%
\pgfpathlineto{\pgfqpoint{1.900296in}{0.700699in}}%
\pgfpathlineto{\pgfqpoint{1.908358in}{0.700699in}}%
\pgfpathlineto{\pgfqpoint{1.916420in}{0.700699in}}%
\pgfpathlineto{\pgfqpoint{1.924482in}{0.700699in}}%
\pgfpathlineto{\pgfqpoint{1.932544in}{0.700699in}}%
\pgfpathlineto{\pgfqpoint{1.940605in}{0.700699in}}%
\pgfpathlineto{\pgfqpoint{1.948667in}{0.700699in}}%
\pgfpathlineto{\pgfqpoint{1.956729in}{0.700699in}}%
\pgfpathlineto{\pgfqpoint{1.964791in}{0.700699in}}%
\pgfpathlineto{\pgfqpoint{1.972853in}{0.700699in}}%
\pgfpathlineto{\pgfqpoint{1.980915in}{0.700699in}}%
\pgfpathlineto{\pgfqpoint{1.988977in}{0.700699in}}%
\pgfpathlineto{\pgfqpoint{1.988977in}{0.477955in}}%
\pgfpathlineto{\pgfqpoint{1.988977in}{0.477955in}}%
\pgfpathlineto{\pgfqpoint{1.980915in}{0.477955in}}%
\pgfpathlineto{\pgfqpoint{1.972853in}{0.477955in}}%
\pgfpathlineto{\pgfqpoint{1.964791in}{0.477955in}}%
\pgfpathlineto{\pgfqpoint{1.956729in}{0.477955in}}%
\pgfpathlineto{\pgfqpoint{1.948667in}{0.477955in}}%
\pgfpathlineto{\pgfqpoint{1.940605in}{0.477955in}}%
\pgfpathlineto{\pgfqpoint{1.932544in}{0.477955in}}%
\pgfpathlineto{\pgfqpoint{1.924482in}{0.477955in}}%
\pgfpathlineto{\pgfqpoint{1.916420in}{0.477955in}}%
\pgfpathlineto{\pgfqpoint{1.908358in}{0.477955in}}%
\pgfpathlineto{\pgfqpoint{1.900296in}{0.477955in}}%
\pgfpathlineto{\pgfqpoint{1.892234in}{0.477955in}}%
\pgfpathlineto{\pgfqpoint{1.884172in}{0.477955in}}%
\pgfpathlineto{\pgfqpoint{1.876110in}{0.477955in}}%
\pgfpathlineto{\pgfqpoint{1.868048in}{0.477955in}}%
\pgfpathlineto{\pgfqpoint{1.859986in}{0.477955in}}%
\pgfpathlineto{\pgfqpoint{1.851924in}{0.477955in}}%
\pgfpathlineto{\pgfqpoint{1.843862in}{0.477955in}}%
\pgfpathlineto{\pgfqpoint{1.835800in}{0.477955in}}%
\pgfpathlineto{\pgfqpoint{1.827738in}{0.477955in}}%
\pgfpathlineto{\pgfqpoint{1.819676in}{0.477955in}}%
\pgfpathlineto{\pgfqpoint{1.811614in}{0.477955in}}%
\pgfpathlineto{\pgfqpoint{1.803552in}{0.477955in}}%
\pgfpathlineto{\pgfqpoint{1.795490in}{0.477955in}}%
\pgfpathlineto{\pgfqpoint{1.787428in}{0.477955in}}%
\pgfpathlineto{\pgfqpoint{1.779366in}{0.477955in}}%
\pgfpathlineto{\pgfqpoint{1.771304in}{0.477955in}}%
\pgfpathlineto{\pgfqpoint{1.763242in}{0.477955in}}%
\pgfpathlineto{\pgfqpoint{1.755180in}{0.658115in}}%
\pgfpathlineto{\pgfqpoint{1.747118in}{0.658115in}}%
\pgfpathlineto{\pgfqpoint{1.739056in}{0.658115in}}%
\pgfpathlineto{\pgfqpoint{1.730994in}{0.658115in}}%
\pgfpathlineto{\pgfqpoint{1.722932in}{0.658115in}}%
\pgfpathlineto{\pgfqpoint{1.714871in}{0.658115in}}%
\pgfpathlineto{\pgfqpoint{1.706809in}{0.658115in}}%
\pgfpathlineto{\pgfqpoint{1.698747in}{0.658115in}}%
\pgfpathlineto{\pgfqpoint{1.690685in}{0.658115in}}%
\pgfpathlineto{\pgfqpoint{1.682623in}{0.658115in}}%
\pgfpathlineto{\pgfqpoint{1.674561in}{0.658115in}}%
\pgfpathlineto{\pgfqpoint{1.666499in}{0.658115in}}%
\pgfpathlineto{\pgfqpoint{1.658437in}{0.658115in}}%
\pgfpathlineto{\pgfqpoint{1.650375in}{0.658115in}}%
\pgfpathlineto{\pgfqpoint{1.642313in}{0.658115in}}%
\pgfpathlineto{\pgfqpoint{1.634251in}{0.658115in}}%
\pgfpathlineto{\pgfqpoint{1.626189in}{0.658115in}}%
\pgfpathlineto{\pgfqpoint{1.618127in}{0.658115in}}%
\pgfpathlineto{\pgfqpoint{1.610065in}{0.658115in}}%
\pgfpathlineto{\pgfqpoint{1.602003in}{0.658115in}}%
\pgfpathlineto{\pgfqpoint{1.593941in}{0.658115in}}%
\pgfpathlineto{\pgfqpoint{1.585879in}{0.837289in}}%
\pgfpathlineto{\pgfqpoint{1.577817in}{0.837289in}}%
\pgfpathlineto{\pgfqpoint{1.569755in}{0.837289in}}%
\pgfpathlineto{\pgfqpoint{1.561693in}{0.837289in}}%
\pgfpathlineto{\pgfqpoint{1.553631in}{0.837289in}}%
\pgfpathlineto{\pgfqpoint{1.545569in}{0.837289in}}%
\pgfpathlineto{\pgfqpoint{1.537507in}{0.837289in}}%
\pgfpathlineto{\pgfqpoint{1.529445in}{0.837289in}}%
\pgfpathlineto{\pgfqpoint{1.521383in}{0.837289in}}%
\pgfpathlineto{\pgfqpoint{1.513321in}{0.837289in}}%
\pgfpathlineto{\pgfqpoint{1.505259in}{0.837289in}}%
\pgfpathlineto{\pgfqpoint{1.497198in}{0.837289in}}%
\pgfpathlineto{\pgfqpoint{1.489136in}{0.837289in}}%
\pgfpathlineto{\pgfqpoint{1.481074in}{0.837289in}}%
\pgfpathlineto{\pgfqpoint{1.473012in}{0.837289in}}%
\pgfpathlineto{\pgfqpoint{1.464950in}{0.837289in}}%
\pgfpathlineto{\pgfqpoint{1.456888in}{0.837289in}}%
\pgfpathlineto{\pgfqpoint{1.448826in}{0.837289in}}%
\pgfpathlineto{\pgfqpoint{1.440764in}{0.837289in}}%
\pgfpathlineto{\pgfqpoint{1.432702in}{0.837289in}}%
\pgfpathlineto{\pgfqpoint{1.424640in}{0.837289in}}%
\pgfpathlineto{\pgfqpoint{1.416578in}{0.837289in}}%
\pgfpathlineto{\pgfqpoint{1.408516in}{0.837289in}}%
\pgfpathlineto{\pgfqpoint{1.400454in}{0.837289in}}%
\pgfpathlineto{\pgfqpoint{1.392392in}{0.837289in}}%
\pgfpathlineto{\pgfqpoint{1.384330in}{0.837289in}}%
\pgfpathlineto{\pgfqpoint{1.376268in}{0.837289in}}%
\pgfpathlineto{\pgfqpoint{1.368206in}{0.837289in}}%
\pgfpathlineto{\pgfqpoint{1.360144in}{0.837289in}}%
\pgfpathlineto{\pgfqpoint{1.352082in}{0.838521in}}%
\pgfpathlineto{\pgfqpoint{1.344020in}{0.838521in}}%
\pgfpathlineto{\pgfqpoint{1.335958in}{0.838521in}}%
\pgfpathlineto{\pgfqpoint{1.327896in}{0.838521in}}%
\pgfpathlineto{\pgfqpoint{1.319834in}{0.838521in}}%
\pgfpathlineto{\pgfqpoint{1.311772in}{0.838521in}}%
\pgfpathlineto{\pgfqpoint{1.303710in}{0.838521in}}%
\pgfpathlineto{\pgfqpoint{1.295648in}{0.838521in}}%
\pgfpathlineto{\pgfqpoint{1.287586in}{0.838521in}}%
\pgfpathlineto{\pgfqpoint{1.279525in}{1.345988in}}%
\pgfpathlineto{\pgfqpoint{1.271463in}{1.345988in}}%
\pgfpathlineto{\pgfqpoint{1.263401in}{1.345988in}}%
\pgfpathlineto{\pgfqpoint{1.255339in}{1.345988in}}%
\pgfpathlineto{\pgfqpoint{1.247277in}{1.345988in}}%
\pgfpathlineto{\pgfqpoint{1.239215in}{1.345988in}}%
\pgfpathlineto{\pgfqpoint{1.231153in}{1.345988in}}%
\pgfpathlineto{\pgfqpoint{1.223091in}{1.345988in}}%
\pgfpathlineto{\pgfqpoint{1.215029in}{1.345988in}}%
\pgfpathlineto{\pgfqpoint{1.206967in}{1.345988in}}%
\pgfpathlineto{\pgfqpoint{1.198905in}{1.345988in}}%
\pgfpathlineto{\pgfqpoint{1.190843in}{1.345988in}}%
\pgfpathlineto{\pgfqpoint{1.182781in}{1.345988in}}%
\pgfpathlineto{\pgfqpoint{1.174719in}{1.345988in}}%
\pgfpathlineto{\pgfqpoint{1.166657in}{1.345988in}}%
\pgfpathlineto{\pgfqpoint{1.158595in}{1.345988in}}%
\pgfpathlineto{\pgfqpoint{1.150533in}{1.345988in}}%
\pgfpathlineto{\pgfqpoint{1.142471in}{1.345988in}}%
\pgfpathlineto{\pgfqpoint{1.134409in}{1.567990in}}%
\pgfpathlineto{\pgfqpoint{1.126347in}{1.567990in}}%
\pgfpathlineto{\pgfqpoint{1.118285in}{1.567990in}}%
\pgfpathlineto{\pgfqpoint{1.110223in}{1.658466in}}%
\pgfpathlineto{\pgfqpoint{1.102161in}{1.542693in}}%
\pgfpathlineto{\pgfqpoint{1.094099in}{1.543328in}}%
\pgfpathlineto{\pgfqpoint{1.086037in}{1.543328in}}%
\pgfpathlineto{\pgfqpoint{1.077975in}{1.543328in}}%
\pgfpathlineto{\pgfqpoint{1.069914in}{1.543328in}}%
\pgfpathlineto{\pgfqpoint{1.061852in}{1.543328in}}%
\pgfpathlineto{\pgfqpoint{1.053790in}{1.543328in}}%
\pgfpathlineto{\pgfqpoint{1.045728in}{1.543328in}}%
\pgfpathlineto{\pgfqpoint{1.037666in}{1.543328in}}%
\pgfpathlineto{\pgfqpoint{1.029604in}{1.543328in}}%
\pgfpathlineto{\pgfqpoint{1.021542in}{1.543328in}}%
\pgfpathlineto{\pgfqpoint{1.013480in}{1.543328in}}%
\pgfpathlineto{\pgfqpoint{1.005418in}{1.543328in}}%
\pgfpathlineto{\pgfqpoint{0.997356in}{1.552555in}}%
\pgfpathlineto{\pgfqpoint{0.989294in}{1.621128in}}%
\pgfpathlineto{\pgfqpoint{0.981232in}{1.621128in}}%
\pgfpathlineto{\pgfqpoint{0.973170in}{1.621128in}}%
\pgfpathlineto{\pgfqpoint{0.965108in}{1.634382in}}%
\pgfpathlineto{\pgfqpoint{0.957046in}{1.634382in}}%
\pgfpathlineto{\pgfqpoint{0.948984in}{1.634382in}}%
\pgfpathlineto{\pgfqpoint{0.940922in}{1.793748in}}%
\pgfpathlineto{\pgfqpoint{0.932860in}{1.793748in}}%
\pgfpathlineto{\pgfqpoint{0.924798in}{1.793748in}}%
\pgfpathlineto{\pgfqpoint{0.916736in}{1.794872in}}%
\pgfpathlineto{\pgfqpoint{0.908674in}{1.803152in}}%
\pgfpathlineto{\pgfqpoint{0.900612in}{1.803152in}}%
\pgfpathlineto{\pgfqpoint{0.892550in}{1.803152in}}%
\pgfpathlineto{\pgfqpoint{0.884488in}{1.803152in}}%
\pgfpathlineto{\pgfqpoint{0.876426in}{1.806515in}}%
\pgfpathlineto{\pgfqpoint{0.868364in}{1.806515in}}%
\pgfpathlineto{\pgfqpoint{0.860302in}{1.810366in}}%
\pgfpathlineto{\pgfqpoint{0.852241in}{1.810366in}}%
\pgfpathlineto{\pgfqpoint{0.844179in}{1.825917in}}%
\pgfpathlineto{\pgfqpoint{0.836117in}{1.825917in}}%
\pgfpathlineto{\pgfqpoint{0.828055in}{1.832512in}}%
\pgfpathlineto{\pgfqpoint{0.819993in}{1.832512in}}%
\pgfpathlineto{\pgfqpoint{0.811931in}{1.832512in}}%
\pgfpathlineto{\pgfqpoint{0.803869in}{1.832512in}}%
\pgfpathlineto{\pgfqpoint{0.795807in}{1.832512in}}%
\pgfpathlineto{\pgfqpoint{0.787745in}{1.832512in}}%
\pgfpathlineto{\pgfqpoint{0.779683in}{1.832512in}}%
\pgfpathlineto{\pgfqpoint{0.771621in}{1.832512in}}%
\pgfpathlineto{\pgfqpoint{0.763559in}{1.832512in}}%
\pgfpathlineto{\pgfqpoint{0.755497in}{1.897993in}}%
\pgfpathlineto{\pgfqpoint{0.747435in}{1.922220in}}%
\pgfpathlineto{\pgfqpoint{0.739373in}{1.922220in}}%
\pgfpathlineto{\pgfqpoint{0.731311in}{2.077498in}}%
\pgfpathlineto{\pgfqpoint{0.723249in}{2.077498in}}%
\pgfpathlineto{\pgfqpoint{0.715187in}{2.077498in}}%
\pgfpathlineto{\pgfqpoint{0.707125in}{2.077498in}}%
\pgfpathlineto{\pgfqpoint{0.699063in}{2.077498in}}%
\pgfpathlineto{\pgfqpoint{0.691001in}{2.077498in}}%
\pgfpathlineto{\pgfqpoint{0.682939in}{2.077498in}}%
\pgfpathlineto{\pgfqpoint{0.674877in}{2.077498in}}%
\pgfpathlineto{\pgfqpoint{0.666815in}{2.077498in}}%
\pgfpathlineto{\pgfqpoint{0.658753in}{2.077498in}}%
\pgfpathlineto{\pgfqpoint{0.650691in}{2.077498in}}%
\pgfpathlineto{\pgfqpoint{0.642629in}{2.077498in}}%
\pgfpathlineto{\pgfqpoint{0.634568in}{2.077498in}}%
\pgfpathlineto{\pgfqpoint{0.626506in}{2.077498in}}%
\pgfpathlineto{\pgfqpoint{0.618444in}{2.077498in}}%
\pgfpathlineto{\pgfqpoint{0.610382in}{2.077498in}}%
\pgfpathlineto{\pgfqpoint{0.602320in}{2.114800in}}%
\pgfpathlineto{\pgfqpoint{0.594258in}{2.114800in}}%
\pgfpathlineto{\pgfqpoint{0.586196in}{2.114800in}}%
\pgfpathlineto{\pgfqpoint{0.578134in}{2.114800in}}%
\pgfpathlineto{\pgfqpoint{0.570072in}{2.114800in}}%
\pgfpathlineto{\pgfqpoint{0.562010in}{2.114800in}}%
\pgfpathlineto{\pgfqpoint{0.553948in}{2.114800in}}%
\pgfpathlineto{\pgfqpoint{0.545886in}{2.114800in}}%
\pgfpathlineto{\pgfqpoint{0.537824in}{2.166458in}}%
\pgfpathlineto{\pgfqpoint{0.529762in}{2.183219in}}%
\pgfpathlineto{\pgfqpoint{0.521700in}{2.183219in}}%
\pgfpathlineto{\pgfqpoint{0.513638in}{2.183219in}}%
\pgfpathlineto{\pgfqpoint{0.505576in}{2.183219in}}%
\pgfpathlineto{\pgfqpoint{0.497514in}{2.183219in}}%
\pgfpathlineto{\pgfqpoint{0.489452in}{2.183219in}}%
\pgfpathlineto{\pgfqpoint{0.481390in}{2.183219in}}%
\pgfpathlineto{\pgfqpoint{0.473328in}{2.183219in}}%
\pgfpathlineto{\pgfqpoint{0.465266in}{2.183219in}}%
\pgfpathlineto{\pgfqpoint{0.457204in}{2.243122in}}%
\pgfpathlineto{\pgfqpoint{0.449142in}{2.243122in}}%
\pgfpathlineto{\pgfqpoint{0.441080in}{2.243122in}}%
\pgfpathlineto{\pgfqpoint{0.433018in}{2.243122in}}%
\pgfpathlineto{\pgfqpoint{0.424956in}{2.243122in}}%
\pgfpathlineto{\pgfqpoint{0.416895in}{2.243122in}}%
\pgfpathlineto{\pgfqpoint{0.408833in}{2.340927in}}%
\pgfpathlineto{\pgfqpoint{0.400771in}{2.340927in}}%
\pgfpathlineto{\pgfqpoint{0.392709in}{2.340927in}}%
\pgfpathlineto{\pgfqpoint{0.384647in}{2.342005in}}%
\pgfpathlineto{\pgfqpoint{0.376585in}{2.342005in}}%
\pgfpathlineto{\pgfqpoint{0.368523in}{2.430236in}}%
\pgfpathclose%
\pgfusepath{fill}%
\end{pgfscope}%
\begin{pgfscope}%
\pgfpathrectangle{\pgfqpoint{0.287500in}{0.375000in}}{\pgfqpoint{1.782500in}{2.265000in}}%
\pgfusepath{clip}%
\pgfsetbuttcap%
\pgfsetroundjoin%
\definecolor{currentfill}{rgb}{0.172549,0.627451,0.172549}%
\pgfsetfillcolor{currentfill}%
\pgfsetfillopacity{0.200000}%
\pgfsetlinewidth{0.000000pt}%
\definecolor{currentstroke}{rgb}{0.000000,0.000000,0.000000}%
\pgfsetstrokecolor{currentstroke}%
\pgfsetdash{}{0pt}%
\pgfpathmoveto{\pgfqpoint{0.368523in}{2.450125in}}%
\pgfpathlineto{\pgfqpoint{0.368523in}{2.517466in}}%
\pgfpathlineto{\pgfqpoint{0.376585in}{2.500839in}}%
\pgfpathlineto{\pgfqpoint{0.384647in}{2.492280in}}%
\pgfpathlineto{\pgfqpoint{0.392709in}{2.457959in}}%
\pgfpathlineto{\pgfqpoint{0.400771in}{2.452002in}}%
\pgfpathlineto{\pgfqpoint{0.408833in}{2.452002in}}%
\pgfpathlineto{\pgfqpoint{0.416895in}{2.452002in}}%
\pgfpathlineto{\pgfqpoint{0.424956in}{2.452002in}}%
\pgfpathlineto{\pgfqpoint{0.433018in}{2.452002in}}%
\pgfpathlineto{\pgfqpoint{0.441080in}{2.440217in}}%
\pgfpathlineto{\pgfqpoint{0.449142in}{2.440217in}}%
\pgfpathlineto{\pgfqpoint{0.457204in}{2.438582in}}%
\pgfpathlineto{\pgfqpoint{0.465266in}{2.431392in}}%
\pgfpathlineto{\pgfqpoint{0.473328in}{2.431392in}}%
\pgfpathlineto{\pgfqpoint{0.481390in}{2.431392in}}%
\pgfpathlineto{\pgfqpoint{0.489452in}{2.413620in}}%
\pgfpathlineto{\pgfqpoint{0.497514in}{2.344300in}}%
\pgfpathlineto{\pgfqpoint{0.505576in}{2.344300in}}%
\pgfpathlineto{\pgfqpoint{0.513638in}{2.344300in}}%
\pgfpathlineto{\pgfqpoint{0.521700in}{2.200328in}}%
\pgfpathlineto{\pgfqpoint{0.529762in}{2.200328in}}%
\pgfpathlineto{\pgfqpoint{0.537824in}{2.200328in}}%
\pgfpathlineto{\pgfqpoint{0.545886in}{2.200328in}}%
\pgfpathlineto{\pgfqpoint{0.553948in}{2.200328in}}%
\pgfpathlineto{\pgfqpoint{0.562010in}{2.200328in}}%
\pgfpathlineto{\pgfqpoint{0.570072in}{2.200328in}}%
\pgfpathlineto{\pgfqpoint{0.578134in}{2.200328in}}%
\pgfpathlineto{\pgfqpoint{0.586196in}{2.200328in}}%
\pgfpathlineto{\pgfqpoint{0.594258in}{2.132712in}}%
\pgfpathlineto{\pgfqpoint{0.602320in}{2.132712in}}%
\pgfpathlineto{\pgfqpoint{0.610382in}{2.132712in}}%
\pgfpathlineto{\pgfqpoint{0.618444in}{2.132712in}}%
\pgfpathlineto{\pgfqpoint{0.626506in}{2.132712in}}%
\pgfpathlineto{\pgfqpoint{0.634568in}{2.132712in}}%
\pgfpathlineto{\pgfqpoint{0.642629in}{2.132712in}}%
\pgfpathlineto{\pgfqpoint{0.650691in}{2.072814in}}%
\pgfpathlineto{\pgfqpoint{0.658753in}{2.072814in}}%
\pgfpathlineto{\pgfqpoint{0.666815in}{2.072814in}}%
\pgfpathlineto{\pgfqpoint{0.674877in}{2.072814in}}%
\pgfpathlineto{\pgfqpoint{0.682939in}{2.072814in}}%
\pgfpathlineto{\pgfqpoint{0.691001in}{2.072814in}}%
\pgfpathlineto{\pgfqpoint{0.699063in}{2.072814in}}%
\pgfpathlineto{\pgfqpoint{0.707125in}{2.061052in}}%
\pgfpathlineto{\pgfqpoint{0.715187in}{2.061052in}}%
\pgfpathlineto{\pgfqpoint{0.723249in}{2.061052in}}%
\pgfpathlineto{\pgfqpoint{0.731311in}{2.061052in}}%
\pgfpathlineto{\pgfqpoint{0.739373in}{2.061052in}}%
\pgfpathlineto{\pgfqpoint{0.747435in}{2.061052in}}%
\pgfpathlineto{\pgfqpoint{0.755497in}{2.061052in}}%
\pgfpathlineto{\pgfqpoint{0.763559in}{2.061052in}}%
\pgfpathlineto{\pgfqpoint{0.771621in}{2.061052in}}%
\pgfpathlineto{\pgfqpoint{0.779683in}{2.061052in}}%
\pgfpathlineto{\pgfqpoint{0.787745in}{2.061052in}}%
\pgfpathlineto{\pgfqpoint{0.795807in}{2.061052in}}%
\pgfpathlineto{\pgfqpoint{0.803869in}{2.061052in}}%
\pgfpathlineto{\pgfqpoint{0.811931in}{2.061052in}}%
\pgfpathlineto{\pgfqpoint{0.819993in}{2.061052in}}%
\pgfpathlineto{\pgfqpoint{0.828055in}{2.061052in}}%
\pgfpathlineto{\pgfqpoint{0.836117in}{2.055640in}}%
\pgfpathlineto{\pgfqpoint{0.844179in}{2.055640in}}%
\pgfpathlineto{\pgfqpoint{0.852241in}{1.964130in}}%
\pgfpathlineto{\pgfqpoint{0.860302in}{1.964130in}}%
\pgfpathlineto{\pgfqpoint{0.868364in}{1.964130in}}%
\pgfpathlineto{\pgfqpoint{0.876426in}{1.964130in}}%
\pgfpathlineto{\pgfqpoint{0.884488in}{1.964130in}}%
\pgfpathlineto{\pgfqpoint{0.892550in}{1.964130in}}%
\pgfpathlineto{\pgfqpoint{0.900612in}{1.964130in}}%
\pgfpathlineto{\pgfqpoint{0.908674in}{1.964130in}}%
\pgfpathlineto{\pgfqpoint{0.916736in}{1.964130in}}%
\pgfpathlineto{\pgfqpoint{0.924798in}{1.964130in}}%
\pgfpathlineto{\pgfqpoint{0.932860in}{1.964130in}}%
\pgfpathlineto{\pgfqpoint{0.940922in}{1.964130in}}%
\pgfpathlineto{\pgfqpoint{0.948984in}{1.964130in}}%
\pgfpathlineto{\pgfqpoint{0.957046in}{1.964130in}}%
\pgfpathlineto{\pgfqpoint{0.965108in}{1.964130in}}%
\pgfpathlineto{\pgfqpoint{0.973170in}{1.964130in}}%
\pgfpathlineto{\pgfqpoint{0.981232in}{1.964130in}}%
\pgfpathlineto{\pgfqpoint{0.989294in}{1.964130in}}%
\pgfpathlineto{\pgfqpoint{0.997356in}{1.964130in}}%
\pgfpathlineto{\pgfqpoint{1.005418in}{1.948016in}}%
\pgfpathlineto{\pgfqpoint{1.013480in}{1.948016in}}%
\pgfpathlineto{\pgfqpoint{1.021542in}{1.948016in}}%
\pgfpathlineto{\pgfqpoint{1.029604in}{1.948016in}}%
\pgfpathlineto{\pgfqpoint{1.037666in}{1.948016in}}%
\pgfpathlineto{\pgfqpoint{1.045728in}{1.948016in}}%
\pgfpathlineto{\pgfqpoint{1.053790in}{1.948016in}}%
\pgfpathlineto{\pgfqpoint{1.061852in}{1.948016in}}%
\pgfpathlineto{\pgfqpoint{1.069914in}{1.948016in}}%
\pgfpathlineto{\pgfqpoint{1.077975in}{1.948016in}}%
\pgfpathlineto{\pgfqpoint{1.086037in}{1.948016in}}%
\pgfpathlineto{\pgfqpoint{1.094099in}{1.948016in}}%
\pgfpathlineto{\pgfqpoint{1.102161in}{1.948016in}}%
\pgfpathlineto{\pgfqpoint{1.110223in}{1.948016in}}%
\pgfpathlineto{\pgfqpoint{1.118285in}{1.948016in}}%
\pgfpathlineto{\pgfqpoint{1.126347in}{1.948016in}}%
\pgfpathlineto{\pgfqpoint{1.134409in}{1.948016in}}%
\pgfpathlineto{\pgfqpoint{1.142471in}{1.948016in}}%
\pgfpathlineto{\pgfqpoint{1.150533in}{1.948016in}}%
\pgfpathlineto{\pgfqpoint{1.158595in}{1.948016in}}%
\pgfpathlineto{\pgfqpoint{1.166657in}{1.948016in}}%
\pgfpathlineto{\pgfqpoint{1.174719in}{1.948016in}}%
\pgfpathlineto{\pgfqpoint{1.182781in}{1.948016in}}%
\pgfpathlineto{\pgfqpoint{1.190843in}{1.948016in}}%
\pgfpathlineto{\pgfqpoint{1.198905in}{1.948016in}}%
\pgfpathlineto{\pgfqpoint{1.206967in}{1.948016in}}%
\pgfpathlineto{\pgfqpoint{1.215029in}{1.946255in}}%
\pgfpathlineto{\pgfqpoint{1.223091in}{1.946255in}}%
\pgfpathlineto{\pgfqpoint{1.231153in}{1.946255in}}%
\pgfpathlineto{\pgfqpoint{1.239215in}{1.946255in}}%
\pgfpathlineto{\pgfqpoint{1.247277in}{1.946255in}}%
\pgfpathlineto{\pgfqpoint{1.255339in}{1.946255in}}%
\pgfpathlineto{\pgfqpoint{1.263401in}{1.946255in}}%
\pgfpathlineto{\pgfqpoint{1.271463in}{1.946255in}}%
\pgfpathlineto{\pgfqpoint{1.279525in}{1.946255in}}%
\pgfpathlineto{\pgfqpoint{1.287586in}{1.946255in}}%
\pgfpathlineto{\pgfqpoint{1.295648in}{1.946255in}}%
\pgfpathlineto{\pgfqpoint{1.303710in}{1.946255in}}%
\pgfpathlineto{\pgfqpoint{1.311772in}{1.946255in}}%
\pgfpathlineto{\pgfqpoint{1.319834in}{1.946255in}}%
\pgfpathlineto{\pgfqpoint{1.327896in}{1.946255in}}%
\pgfpathlineto{\pgfqpoint{1.335958in}{1.941584in}}%
\pgfpathlineto{\pgfqpoint{1.344020in}{1.941584in}}%
\pgfpathlineto{\pgfqpoint{1.352082in}{1.941584in}}%
\pgfpathlineto{\pgfqpoint{1.360144in}{1.941584in}}%
\pgfpathlineto{\pgfqpoint{1.368206in}{1.941584in}}%
\pgfpathlineto{\pgfqpoint{1.376268in}{1.763149in}}%
\pgfpathlineto{\pgfqpoint{1.384330in}{1.763149in}}%
\pgfpathlineto{\pgfqpoint{1.392392in}{1.763149in}}%
\pgfpathlineto{\pgfqpoint{1.400454in}{1.763149in}}%
\pgfpathlineto{\pgfqpoint{1.408516in}{1.763149in}}%
\pgfpathlineto{\pgfqpoint{1.416578in}{1.763149in}}%
\pgfpathlineto{\pgfqpoint{1.424640in}{1.763149in}}%
\pgfpathlineto{\pgfqpoint{1.432702in}{1.763149in}}%
\pgfpathlineto{\pgfqpoint{1.440764in}{1.763149in}}%
\pgfpathlineto{\pgfqpoint{1.448826in}{1.718375in}}%
\pgfpathlineto{\pgfqpoint{1.456888in}{1.712297in}}%
\pgfpathlineto{\pgfqpoint{1.464950in}{1.501598in}}%
\pgfpathlineto{\pgfqpoint{1.473012in}{1.501598in}}%
\pgfpathlineto{\pgfqpoint{1.481074in}{1.501598in}}%
\pgfpathlineto{\pgfqpoint{1.489136in}{1.501598in}}%
\pgfpathlineto{\pgfqpoint{1.497198in}{1.501598in}}%
\pgfpathlineto{\pgfqpoint{1.505259in}{1.501598in}}%
\pgfpathlineto{\pgfqpoint{1.513321in}{1.501598in}}%
\pgfpathlineto{\pgfqpoint{1.521383in}{1.501598in}}%
\pgfpathlineto{\pgfqpoint{1.529445in}{1.501598in}}%
\pgfpathlineto{\pgfqpoint{1.537507in}{1.501598in}}%
\pgfpathlineto{\pgfqpoint{1.545569in}{1.501598in}}%
\pgfpathlineto{\pgfqpoint{1.553631in}{1.501598in}}%
\pgfpathlineto{\pgfqpoint{1.561693in}{1.409626in}}%
\pgfpathlineto{\pgfqpoint{1.569755in}{1.409626in}}%
\pgfpathlineto{\pgfqpoint{1.577817in}{1.409626in}}%
\pgfpathlineto{\pgfqpoint{1.585879in}{1.409626in}}%
\pgfpathlineto{\pgfqpoint{1.593941in}{1.409626in}}%
\pgfpathlineto{\pgfqpoint{1.602003in}{1.409626in}}%
\pgfpathlineto{\pgfqpoint{1.610065in}{1.409626in}}%
\pgfpathlineto{\pgfqpoint{1.618127in}{1.409626in}}%
\pgfpathlineto{\pgfqpoint{1.626189in}{1.409626in}}%
\pgfpathlineto{\pgfqpoint{1.634251in}{1.409626in}}%
\pgfpathlineto{\pgfqpoint{1.642313in}{1.409626in}}%
\pgfpathlineto{\pgfqpoint{1.650375in}{1.409626in}}%
\pgfpathlineto{\pgfqpoint{1.658437in}{1.409626in}}%
\pgfpathlineto{\pgfqpoint{1.666499in}{1.409626in}}%
\pgfpathlineto{\pgfqpoint{1.674561in}{1.409626in}}%
\pgfpathlineto{\pgfqpoint{1.682623in}{1.409626in}}%
\pgfpathlineto{\pgfqpoint{1.690685in}{1.409626in}}%
\pgfpathlineto{\pgfqpoint{1.698747in}{1.409626in}}%
\pgfpathlineto{\pgfqpoint{1.706809in}{1.409626in}}%
\pgfpathlineto{\pgfqpoint{1.714871in}{1.409626in}}%
\pgfpathlineto{\pgfqpoint{1.722932in}{1.409626in}}%
\pgfpathlineto{\pgfqpoint{1.730994in}{1.409626in}}%
\pgfpathlineto{\pgfqpoint{1.739056in}{1.409626in}}%
\pgfpathlineto{\pgfqpoint{1.747118in}{1.409626in}}%
\pgfpathlineto{\pgfqpoint{1.755180in}{1.409626in}}%
\pgfpathlineto{\pgfqpoint{1.763242in}{1.409626in}}%
\pgfpathlineto{\pgfqpoint{1.771304in}{1.409626in}}%
\pgfpathlineto{\pgfqpoint{1.779366in}{1.409626in}}%
\pgfpathlineto{\pgfqpoint{1.787428in}{1.409626in}}%
\pgfpathlineto{\pgfqpoint{1.795490in}{1.409626in}}%
\pgfpathlineto{\pgfqpoint{1.803552in}{1.409626in}}%
\pgfpathlineto{\pgfqpoint{1.811614in}{1.409626in}}%
\pgfpathlineto{\pgfqpoint{1.819676in}{1.298488in}}%
\pgfpathlineto{\pgfqpoint{1.827738in}{1.298488in}}%
\pgfpathlineto{\pgfqpoint{1.835800in}{1.298488in}}%
\pgfpathlineto{\pgfqpoint{1.843862in}{1.179941in}}%
\pgfpathlineto{\pgfqpoint{1.851924in}{1.179941in}}%
\pgfpathlineto{\pgfqpoint{1.859986in}{1.179941in}}%
\pgfpathlineto{\pgfqpoint{1.868048in}{1.179941in}}%
\pgfpathlineto{\pgfqpoint{1.876110in}{1.179941in}}%
\pgfpathlineto{\pgfqpoint{1.884172in}{1.179941in}}%
\pgfpathlineto{\pgfqpoint{1.892234in}{1.179941in}}%
\pgfpathlineto{\pgfqpoint{1.900296in}{1.179941in}}%
\pgfpathlineto{\pgfqpoint{1.908358in}{1.179941in}}%
\pgfpathlineto{\pgfqpoint{1.916420in}{1.179941in}}%
\pgfpathlineto{\pgfqpoint{1.924482in}{1.179941in}}%
\pgfpathlineto{\pgfqpoint{1.932544in}{1.179941in}}%
\pgfpathlineto{\pgfqpoint{1.940605in}{1.168028in}}%
\pgfpathlineto{\pgfqpoint{1.948667in}{1.168028in}}%
\pgfpathlineto{\pgfqpoint{1.956729in}{1.168028in}}%
\pgfpathlineto{\pgfqpoint{1.964791in}{1.168028in}}%
\pgfpathlineto{\pgfqpoint{1.972853in}{1.168028in}}%
\pgfpathlineto{\pgfqpoint{1.980915in}{1.168028in}}%
\pgfpathlineto{\pgfqpoint{1.988977in}{1.168028in}}%
\pgfpathlineto{\pgfqpoint{1.988977in}{0.845691in}}%
\pgfpathlineto{\pgfqpoint{1.988977in}{0.845691in}}%
\pgfpathlineto{\pgfqpoint{1.980915in}{0.845691in}}%
\pgfpathlineto{\pgfqpoint{1.972853in}{0.845691in}}%
\pgfpathlineto{\pgfqpoint{1.964791in}{0.845691in}}%
\pgfpathlineto{\pgfqpoint{1.956729in}{0.845691in}}%
\pgfpathlineto{\pgfqpoint{1.948667in}{0.845691in}}%
\pgfpathlineto{\pgfqpoint{1.940605in}{0.845691in}}%
\pgfpathlineto{\pgfqpoint{1.932544in}{0.949440in}}%
\pgfpathlineto{\pgfqpoint{1.924482in}{0.949440in}}%
\pgfpathlineto{\pgfqpoint{1.916420in}{0.949440in}}%
\pgfpathlineto{\pgfqpoint{1.908358in}{0.949440in}}%
\pgfpathlineto{\pgfqpoint{1.900296in}{0.949440in}}%
\pgfpathlineto{\pgfqpoint{1.892234in}{0.949440in}}%
\pgfpathlineto{\pgfqpoint{1.884172in}{0.949440in}}%
\pgfpathlineto{\pgfqpoint{1.876110in}{0.949440in}}%
\pgfpathlineto{\pgfqpoint{1.868048in}{0.949440in}}%
\pgfpathlineto{\pgfqpoint{1.859986in}{0.949440in}}%
\pgfpathlineto{\pgfqpoint{1.851924in}{0.949440in}}%
\pgfpathlineto{\pgfqpoint{1.843862in}{0.949440in}}%
\pgfpathlineto{\pgfqpoint{1.835800in}{1.101361in}}%
\pgfpathlineto{\pgfqpoint{1.827738in}{1.101361in}}%
\pgfpathlineto{\pgfqpoint{1.819676in}{1.101361in}}%
\pgfpathlineto{\pgfqpoint{1.811614in}{1.199551in}}%
\pgfpathlineto{\pgfqpoint{1.803552in}{1.199551in}}%
\pgfpathlineto{\pgfqpoint{1.795490in}{1.199551in}}%
\pgfpathlineto{\pgfqpoint{1.787428in}{1.199551in}}%
\pgfpathlineto{\pgfqpoint{1.779366in}{1.199551in}}%
\pgfpathlineto{\pgfqpoint{1.771304in}{1.199551in}}%
\pgfpathlineto{\pgfqpoint{1.763242in}{1.199551in}}%
\pgfpathlineto{\pgfqpoint{1.755180in}{1.199551in}}%
\pgfpathlineto{\pgfqpoint{1.747118in}{1.199551in}}%
\pgfpathlineto{\pgfqpoint{1.739056in}{1.199551in}}%
\pgfpathlineto{\pgfqpoint{1.730994in}{1.199551in}}%
\pgfpathlineto{\pgfqpoint{1.722932in}{1.199551in}}%
\pgfpathlineto{\pgfqpoint{1.714871in}{1.199551in}}%
\pgfpathlineto{\pgfqpoint{1.706809in}{1.199551in}}%
\pgfpathlineto{\pgfqpoint{1.698747in}{1.199551in}}%
\pgfpathlineto{\pgfqpoint{1.690685in}{1.199551in}}%
\pgfpathlineto{\pgfqpoint{1.682623in}{1.199551in}}%
\pgfpathlineto{\pgfqpoint{1.674561in}{1.199551in}}%
\pgfpathlineto{\pgfqpoint{1.666499in}{1.199551in}}%
\pgfpathlineto{\pgfqpoint{1.658437in}{1.199551in}}%
\pgfpathlineto{\pgfqpoint{1.650375in}{1.199551in}}%
\pgfpathlineto{\pgfqpoint{1.642313in}{1.199551in}}%
\pgfpathlineto{\pgfqpoint{1.634251in}{1.199551in}}%
\pgfpathlineto{\pgfqpoint{1.626189in}{1.199551in}}%
\pgfpathlineto{\pgfqpoint{1.618127in}{1.199551in}}%
\pgfpathlineto{\pgfqpoint{1.610065in}{1.199551in}}%
\pgfpathlineto{\pgfqpoint{1.602003in}{1.199551in}}%
\pgfpathlineto{\pgfqpoint{1.593941in}{1.199551in}}%
\pgfpathlineto{\pgfqpoint{1.585879in}{1.199551in}}%
\pgfpathlineto{\pgfqpoint{1.577817in}{1.199551in}}%
\pgfpathlineto{\pgfqpoint{1.569755in}{1.199551in}}%
\pgfpathlineto{\pgfqpoint{1.561693in}{1.199551in}}%
\pgfpathlineto{\pgfqpoint{1.553631in}{1.354285in}}%
\pgfpathlineto{\pgfqpoint{1.545569in}{1.354285in}}%
\pgfpathlineto{\pgfqpoint{1.537507in}{1.354285in}}%
\pgfpathlineto{\pgfqpoint{1.529445in}{1.354285in}}%
\pgfpathlineto{\pgfqpoint{1.521383in}{1.354285in}}%
\pgfpathlineto{\pgfqpoint{1.513321in}{1.354285in}}%
\pgfpathlineto{\pgfqpoint{1.505259in}{1.354285in}}%
\pgfpathlineto{\pgfqpoint{1.497198in}{1.354285in}}%
\pgfpathlineto{\pgfqpoint{1.489136in}{1.354285in}}%
\pgfpathlineto{\pgfqpoint{1.481074in}{1.354285in}}%
\pgfpathlineto{\pgfqpoint{1.473012in}{1.354285in}}%
\pgfpathlineto{\pgfqpoint{1.464950in}{1.354285in}}%
\pgfpathlineto{\pgfqpoint{1.456888in}{1.477718in}}%
\pgfpathlineto{\pgfqpoint{1.448826in}{1.507350in}}%
\pgfpathlineto{\pgfqpoint{1.440764in}{1.520440in}}%
\pgfpathlineto{\pgfqpoint{1.432702in}{1.520440in}}%
\pgfpathlineto{\pgfqpoint{1.424640in}{1.520440in}}%
\pgfpathlineto{\pgfqpoint{1.416578in}{1.520440in}}%
\pgfpathlineto{\pgfqpoint{1.408516in}{1.520440in}}%
\pgfpathlineto{\pgfqpoint{1.400454in}{1.520440in}}%
\pgfpathlineto{\pgfqpoint{1.392392in}{1.520440in}}%
\pgfpathlineto{\pgfqpoint{1.384330in}{1.520440in}}%
\pgfpathlineto{\pgfqpoint{1.376268in}{1.520440in}}%
\pgfpathlineto{\pgfqpoint{1.368206in}{1.592414in}}%
\pgfpathlineto{\pgfqpoint{1.360144in}{1.592414in}}%
\pgfpathlineto{\pgfqpoint{1.352082in}{1.592414in}}%
\pgfpathlineto{\pgfqpoint{1.344020in}{1.592414in}}%
\pgfpathlineto{\pgfqpoint{1.335958in}{1.592414in}}%
\pgfpathlineto{\pgfqpoint{1.327896in}{1.648591in}}%
\pgfpathlineto{\pgfqpoint{1.319834in}{1.648591in}}%
\pgfpathlineto{\pgfqpoint{1.311772in}{1.648591in}}%
\pgfpathlineto{\pgfqpoint{1.303710in}{1.648591in}}%
\pgfpathlineto{\pgfqpoint{1.295648in}{1.648591in}}%
\pgfpathlineto{\pgfqpoint{1.287586in}{1.648591in}}%
\pgfpathlineto{\pgfqpoint{1.279525in}{1.648591in}}%
\pgfpathlineto{\pgfqpoint{1.271463in}{1.648591in}}%
\pgfpathlineto{\pgfqpoint{1.263401in}{1.648591in}}%
\pgfpathlineto{\pgfqpoint{1.255339in}{1.648591in}}%
\pgfpathlineto{\pgfqpoint{1.247277in}{1.648591in}}%
\pgfpathlineto{\pgfqpoint{1.239215in}{1.648591in}}%
\pgfpathlineto{\pgfqpoint{1.231153in}{1.648591in}}%
\pgfpathlineto{\pgfqpoint{1.223091in}{1.648591in}}%
\pgfpathlineto{\pgfqpoint{1.215029in}{1.648591in}}%
\pgfpathlineto{\pgfqpoint{1.206967in}{1.669916in}}%
\pgfpathlineto{\pgfqpoint{1.198905in}{1.669916in}}%
\pgfpathlineto{\pgfqpoint{1.190843in}{1.669916in}}%
\pgfpathlineto{\pgfqpoint{1.182781in}{1.669916in}}%
\pgfpathlineto{\pgfqpoint{1.174719in}{1.669916in}}%
\pgfpathlineto{\pgfqpoint{1.166657in}{1.669916in}}%
\pgfpathlineto{\pgfqpoint{1.158595in}{1.669916in}}%
\pgfpathlineto{\pgfqpoint{1.150533in}{1.669916in}}%
\pgfpathlineto{\pgfqpoint{1.142471in}{1.669916in}}%
\pgfpathlineto{\pgfqpoint{1.134409in}{1.669916in}}%
\pgfpathlineto{\pgfqpoint{1.126347in}{1.669916in}}%
\pgfpathlineto{\pgfqpoint{1.118285in}{1.669916in}}%
\pgfpathlineto{\pgfqpoint{1.110223in}{1.669916in}}%
\pgfpathlineto{\pgfqpoint{1.102161in}{1.669916in}}%
\pgfpathlineto{\pgfqpoint{1.094099in}{1.669916in}}%
\pgfpathlineto{\pgfqpoint{1.086037in}{1.669916in}}%
\pgfpathlineto{\pgfqpoint{1.077975in}{1.669916in}}%
\pgfpathlineto{\pgfqpoint{1.069914in}{1.669916in}}%
\pgfpathlineto{\pgfqpoint{1.061852in}{1.669916in}}%
\pgfpathlineto{\pgfqpoint{1.053790in}{1.669916in}}%
\pgfpathlineto{\pgfqpoint{1.045728in}{1.669916in}}%
\pgfpathlineto{\pgfqpoint{1.037666in}{1.669916in}}%
\pgfpathlineto{\pgfqpoint{1.029604in}{1.669916in}}%
\pgfpathlineto{\pgfqpoint{1.021542in}{1.669916in}}%
\pgfpathlineto{\pgfqpoint{1.013480in}{1.669916in}}%
\pgfpathlineto{\pgfqpoint{1.005418in}{1.669916in}}%
\pgfpathlineto{\pgfqpoint{0.997356in}{1.753278in}}%
\pgfpathlineto{\pgfqpoint{0.989294in}{1.753278in}}%
\pgfpathlineto{\pgfqpoint{0.981232in}{1.753278in}}%
\pgfpathlineto{\pgfqpoint{0.973170in}{1.753278in}}%
\pgfpathlineto{\pgfqpoint{0.965108in}{1.753278in}}%
\pgfpathlineto{\pgfqpoint{0.957046in}{1.753278in}}%
\pgfpathlineto{\pgfqpoint{0.948984in}{1.753278in}}%
\pgfpathlineto{\pgfqpoint{0.940922in}{1.753278in}}%
\pgfpathlineto{\pgfqpoint{0.932860in}{1.753278in}}%
\pgfpathlineto{\pgfqpoint{0.924798in}{1.753278in}}%
\pgfpathlineto{\pgfqpoint{0.916736in}{1.753278in}}%
\pgfpathlineto{\pgfqpoint{0.908674in}{1.753278in}}%
\pgfpathlineto{\pgfqpoint{0.900612in}{1.753278in}}%
\pgfpathlineto{\pgfqpoint{0.892550in}{1.753278in}}%
\pgfpathlineto{\pgfqpoint{0.884488in}{1.753278in}}%
\pgfpathlineto{\pgfqpoint{0.876426in}{1.753278in}}%
\pgfpathlineto{\pgfqpoint{0.868364in}{1.753278in}}%
\pgfpathlineto{\pgfqpoint{0.860302in}{1.753278in}}%
\pgfpathlineto{\pgfqpoint{0.852241in}{1.753278in}}%
\pgfpathlineto{\pgfqpoint{0.844179in}{1.784477in}}%
\pgfpathlineto{\pgfqpoint{0.836117in}{1.784477in}}%
\pgfpathlineto{\pgfqpoint{0.828055in}{1.786428in}}%
\pgfpathlineto{\pgfqpoint{0.819993in}{1.786428in}}%
\pgfpathlineto{\pgfqpoint{0.811931in}{1.786428in}}%
\pgfpathlineto{\pgfqpoint{0.803869in}{1.786428in}}%
\pgfpathlineto{\pgfqpoint{0.795807in}{1.786428in}}%
\pgfpathlineto{\pgfqpoint{0.787745in}{1.786428in}}%
\pgfpathlineto{\pgfqpoint{0.779683in}{1.786428in}}%
\pgfpathlineto{\pgfqpoint{0.771621in}{1.786428in}}%
\pgfpathlineto{\pgfqpoint{0.763559in}{1.786428in}}%
\pgfpathlineto{\pgfqpoint{0.755497in}{1.786428in}}%
\pgfpathlineto{\pgfqpoint{0.747435in}{1.786428in}}%
\pgfpathlineto{\pgfqpoint{0.739373in}{1.786428in}}%
\pgfpathlineto{\pgfqpoint{0.731311in}{1.786428in}}%
\pgfpathlineto{\pgfqpoint{0.723249in}{1.786428in}}%
\pgfpathlineto{\pgfqpoint{0.715187in}{1.786428in}}%
\pgfpathlineto{\pgfqpoint{0.707125in}{1.786428in}}%
\pgfpathlineto{\pgfqpoint{0.699063in}{1.869932in}}%
\pgfpathlineto{\pgfqpoint{0.691001in}{1.869932in}}%
\pgfpathlineto{\pgfqpoint{0.682939in}{1.869932in}}%
\pgfpathlineto{\pgfqpoint{0.674877in}{1.869932in}}%
\pgfpathlineto{\pgfqpoint{0.666815in}{1.869932in}}%
\pgfpathlineto{\pgfqpoint{0.658753in}{1.869932in}}%
\pgfpathlineto{\pgfqpoint{0.650691in}{1.869932in}}%
\pgfpathlineto{\pgfqpoint{0.642629in}{1.973036in}}%
\pgfpathlineto{\pgfqpoint{0.634568in}{1.973036in}}%
\pgfpathlineto{\pgfqpoint{0.626506in}{1.973036in}}%
\pgfpathlineto{\pgfqpoint{0.618444in}{1.973036in}}%
\pgfpathlineto{\pgfqpoint{0.610382in}{1.973036in}}%
\pgfpathlineto{\pgfqpoint{0.602320in}{1.973036in}}%
\pgfpathlineto{\pgfqpoint{0.594258in}{1.973036in}}%
\pgfpathlineto{\pgfqpoint{0.586196in}{2.072420in}}%
\pgfpathlineto{\pgfqpoint{0.578134in}{2.072420in}}%
\pgfpathlineto{\pgfqpoint{0.570072in}{2.072420in}}%
\pgfpathlineto{\pgfqpoint{0.562010in}{2.072420in}}%
\pgfpathlineto{\pgfqpoint{0.553948in}{2.072420in}}%
\pgfpathlineto{\pgfqpoint{0.545886in}{2.072420in}}%
\pgfpathlineto{\pgfqpoint{0.537824in}{2.072420in}}%
\pgfpathlineto{\pgfqpoint{0.529762in}{2.072420in}}%
\pgfpathlineto{\pgfqpoint{0.521700in}{2.072420in}}%
\pgfpathlineto{\pgfqpoint{0.513638in}{2.188095in}}%
\pgfpathlineto{\pgfqpoint{0.505576in}{2.188095in}}%
\pgfpathlineto{\pgfqpoint{0.497514in}{2.188095in}}%
\pgfpathlineto{\pgfqpoint{0.489452in}{2.267980in}}%
\pgfpathlineto{\pgfqpoint{0.481390in}{2.335143in}}%
\pgfpathlineto{\pgfqpoint{0.473328in}{2.335143in}}%
\pgfpathlineto{\pgfqpoint{0.465266in}{2.335143in}}%
\pgfpathlineto{\pgfqpoint{0.457204in}{2.365084in}}%
\pgfpathlineto{\pgfqpoint{0.449142in}{2.365575in}}%
\pgfpathlineto{\pgfqpoint{0.441080in}{2.365575in}}%
\pgfpathlineto{\pgfqpoint{0.433018in}{2.374394in}}%
\pgfpathlineto{\pgfqpoint{0.424956in}{2.374394in}}%
\pgfpathlineto{\pgfqpoint{0.416895in}{2.374394in}}%
\pgfpathlineto{\pgfqpoint{0.408833in}{2.374394in}}%
\pgfpathlineto{\pgfqpoint{0.400771in}{2.374394in}}%
\pgfpathlineto{\pgfqpoint{0.392709in}{2.377642in}}%
\pgfpathlineto{\pgfqpoint{0.384647in}{2.438009in}}%
\pgfpathlineto{\pgfqpoint{0.376585in}{2.443548in}}%
\pgfpathlineto{\pgfqpoint{0.368523in}{2.450125in}}%
\pgfpathclose%
\pgfusepath{fill}%
\end{pgfscope}%
\begin{pgfscope}%
\pgfpathrectangle{\pgfqpoint{0.287500in}{0.375000in}}{\pgfqpoint{1.782500in}{2.265000in}}%
\pgfusepath{clip}%
\pgfsetbuttcap%
\pgfsetroundjoin%
\definecolor{currentfill}{rgb}{0.839216,0.152941,0.156863}%
\pgfsetfillcolor{currentfill}%
\pgfsetfillopacity{0.200000}%
\pgfsetlinewidth{0.000000pt}%
\definecolor{currentstroke}{rgb}{0.000000,0.000000,0.000000}%
\pgfsetstrokecolor{currentstroke}%
\pgfsetdash{}{0pt}%
\pgfpathmoveto{\pgfqpoint{0.368523in}{2.519071in}}%
\pgfpathlineto{\pgfqpoint{0.368523in}{2.530857in}}%
\pgfpathlineto{\pgfqpoint{0.376585in}{2.527920in}}%
\pgfpathlineto{\pgfqpoint{0.384647in}{2.500393in}}%
\pgfpathlineto{\pgfqpoint{0.392709in}{2.479854in}}%
\pgfpathlineto{\pgfqpoint{0.400771in}{2.479309in}}%
\pgfpathlineto{\pgfqpoint{0.408833in}{2.478966in}}%
\pgfpathlineto{\pgfqpoint{0.416895in}{2.472879in}}%
\pgfpathlineto{\pgfqpoint{0.424956in}{2.443424in}}%
\pgfpathlineto{\pgfqpoint{0.433018in}{2.443065in}}%
\pgfpathlineto{\pgfqpoint{0.441080in}{2.443065in}}%
\pgfpathlineto{\pgfqpoint{0.449142in}{2.442598in}}%
\pgfpathlineto{\pgfqpoint{0.457204in}{2.442598in}}%
\pgfpathlineto{\pgfqpoint{0.465266in}{2.442598in}}%
\pgfpathlineto{\pgfqpoint{0.473328in}{2.442598in}}%
\pgfpathlineto{\pgfqpoint{0.481390in}{2.440889in}}%
\pgfpathlineto{\pgfqpoint{0.489452in}{2.440803in}}%
\pgfpathlineto{\pgfqpoint{0.497514in}{2.440803in}}%
\pgfpathlineto{\pgfqpoint{0.505576in}{2.440803in}}%
\pgfpathlineto{\pgfqpoint{0.513638in}{2.440648in}}%
\pgfpathlineto{\pgfqpoint{0.521700in}{2.440648in}}%
\pgfpathlineto{\pgfqpoint{0.529762in}{2.440648in}}%
\pgfpathlineto{\pgfqpoint{0.537824in}{2.428582in}}%
\pgfpathlineto{\pgfqpoint{0.545886in}{2.411843in}}%
\pgfpathlineto{\pgfqpoint{0.553948in}{2.411145in}}%
\pgfpathlineto{\pgfqpoint{0.562010in}{2.411145in}}%
\pgfpathlineto{\pgfqpoint{0.570072in}{2.410862in}}%
\pgfpathlineto{\pgfqpoint{0.578134in}{2.410862in}}%
\pgfpathlineto{\pgfqpoint{0.586196in}{2.396935in}}%
\pgfpathlineto{\pgfqpoint{0.594258in}{2.395720in}}%
\pgfpathlineto{\pgfqpoint{0.602320in}{2.395720in}}%
\pgfpathlineto{\pgfqpoint{0.610382in}{2.395720in}}%
\pgfpathlineto{\pgfqpoint{0.618444in}{2.395716in}}%
\pgfpathlineto{\pgfqpoint{0.626506in}{2.395716in}}%
\pgfpathlineto{\pgfqpoint{0.634568in}{2.394752in}}%
\pgfpathlineto{\pgfqpoint{0.642629in}{2.394752in}}%
\pgfpathlineto{\pgfqpoint{0.650691in}{2.394752in}}%
\pgfpathlineto{\pgfqpoint{0.658753in}{2.394752in}}%
\pgfpathlineto{\pgfqpoint{0.666815in}{2.394752in}}%
\pgfpathlineto{\pgfqpoint{0.674877in}{2.394752in}}%
\pgfpathlineto{\pgfqpoint{0.682939in}{2.394752in}}%
\pgfpathlineto{\pgfqpoint{0.691001in}{2.394752in}}%
\pgfpathlineto{\pgfqpoint{0.699063in}{2.394752in}}%
\pgfpathlineto{\pgfqpoint{0.707125in}{2.394752in}}%
\pgfpathlineto{\pgfqpoint{0.715187in}{2.394752in}}%
\pgfpathlineto{\pgfqpoint{0.723249in}{2.394752in}}%
\pgfpathlineto{\pgfqpoint{0.731311in}{2.394752in}}%
\pgfpathlineto{\pgfqpoint{0.739373in}{2.394752in}}%
\pgfpathlineto{\pgfqpoint{0.747435in}{2.394752in}}%
\pgfpathlineto{\pgfqpoint{0.755497in}{2.394752in}}%
\pgfpathlineto{\pgfqpoint{0.763559in}{2.394672in}}%
\pgfpathlineto{\pgfqpoint{0.771621in}{2.394672in}}%
\pgfpathlineto{\pgfqpoint{0.779683in}{2.394672in}}%
\pgfpathlineto{\pgfqpoint{0.787745in}{2.394672in}}%
\pgfpathlineto{\pgfqpoint{0.795807in}{2.394672in}}%
\pgfpathlineto{\pgfqpoint{0.803869in}{2.394672in}}%
\pgfpathlineto{\pgfqpoint{0.811931in}{2.394672in}}%
\pgfpathlineto{\pgfqpoint{0.819993in}{2.394672in}}%
\pgfpathlineto{\pgfqpoint{0.828055in}{2.394672in}}%
\pgfpathlineto{\pgfqpoint{0.836117in}{2.394672in}}%
\pgfpathlineto{\pgfqpoint{0.844179in}{2.394672in}}%
\pgfpathlineto{\pgfqpoint{0.852241in}{2.394672in}}%
\pgfpathlineto{\pgfqpoint{0.860302in}{2.394672in}}%
\pgfpathlineto{\pgfqpoint{0.868364in}{2.394672in}}%
\pgfpathlineto{\pgfqpoint{0.876426in}{2.394672in}}%
\pgfpathlineto{\pgfqpoint{0.884488in}{2.394672in}}%
\pgfpathlineto{\pgfqpoint{0.892550in}{2.394672in}}%
\pgfpathlineto{\pgfqpoint{0.900612in}{2.394672in}}%
\pgfpathlineto{\pgfqpoint{0.908674in}{2.394672in}}%
\pgfpathlineto{\pgfqpoint{0.916736in}{2.394672in}}%
\pgfpathlineto{\pgfqpoint{0.924798in}{2.394672in}}%
\pgfpathlineto{\pgfqpoint{0.932860in}{2.394667in}}%
\pgfpathlineto{\pgfqpoint{0.940922in}{2.394667in}}%
\pgfpathlineto{\pgfqpoint{0.948984in}{2.394667in}}%
\pgfpathlineto{\pgfqpoint{0.957046in}{2.394667in}}%
\pgfpathlineto{\pgfqpoint{0.965108in}{2.394667in}}%
\pgfpathlineto{\pgfqpoint{0.973170in}{2.331538in}}%
\pgfpathlineto{\pgfqpoint{0.981232in}{2.306800in}}%
\pgfpathlineto{\pgfqpoint{0.989294in}{2.279003in}}%
\pgfpathlineto{\pgfqpoint{0.997356in}{2.265426in}}%
\pgfpathlineto{\pgfqpoint{1.005418in}{2.262089in}}%
\pgfpathlineto{\pgfqpoint{1.013480in}{2.262089in}}%
\pgfpathlineto{\pgfqpoint{1.021542in}{2.262089in}}%
\pgfpathlineto{\pgfqpoint{1.029604in}{2.262089in}}%
\pgfpathlineto{\pgfqpoint{1.037666in}{2.262089in}}%
\pgfpathlineto{\pgfqpoint{1.045728in}{2.262089in}}%
\pgfpathlineto{\pgfqpoint{1.053790in}{2.262089in}}%
\pgfpathlineto{\pgfqpoint{1.061852in}{2.262089in}}%
\pgfpathlineto{\pgfqpoint{1.069914in}{2.261936in}}%
\pgfpathlineto{\pgfqpoint{1.077975in}{2.261266in}}%
\pgfpathlineto{\pgfqpoint{1.086037in}{2.261266in}}%
\pgfpathlineto{\pgfqpoint{1.094099in}{2.261266in}}%
\pgfpathlineto{\pgfqpoint{1.102161in}{2.261266in}}%
\pgfpathlineto{\pgfqpoint{1.110223in}{2.261266in}}%
\pgfpathlineto{\pgfqpoint{1.118285in}{2.261022in}}%
\pgfpathlineto{\pgfqpoint{1.126347in}{2.225058in}}%
\pgfpathlineto{\pgfqpoint{1.134409in}{2.225058in}}%
\pgfpathlineto{\pgfqpoint{1.142471in}{2.225058in}}%
\pgfpathlineto{\pgfqpoint{1.150533in}{2.223907in}}%
\pgfpathlineto{\pgfqpoint{1.158595in}{2.223907in}}%
\pgfpathlineto{\pgfqpoint{1.166657in}{2.223907in}}%
\pgfpathlineto{\pgfqpoint{1.174719in}{2.223907in}}%
\pgfpathlineto{\pgfqpoint{1.182781in}{2.223413in}}%
\pgfpathlineto{\pgfqpoint{1.190843in}{2.223413in}}%
\pgfpathlineto{\pgfqpoint{1.198905in}{2.223413in}}%
\pgfpathlineto{\pgfqpoint{1.206967in}{2.223413in}}%
\pgfpathlineto{\pgfqpoint{1.215029in}{2.223413in}}%
\pgfpathlineto{\pgfqpoint{1.223091in}{2.223413in}}%
\pgfpathlineto{\pgfqpoint{1.231153in}{2.223413in}}%
\pgfpathlineto{\pgfqpoint{1.239215in}{2.223413in}}%
\pgfpathlineto{\pgfqpoint{1.247277in}{2.223413in}}%
\pgfpathlineto{\pgfqpoint{1.255339in}{2.223413in}}%
\pgfpathlineto{\pgfqpoint{1.263401in}{2.223413in}}%
\pgfpathlineto{\pgfqpoint{1.271463in}{2.223413in}}%
\pgfpathlineto{\pgfqpoint{1.279525in}{2.223413in}}%
\pgfpathlineto{\pgfqpoint{1.287586in}{2.223413in}}%
\pgfpathlineto{\pgfqpoint{1.295648in}{2.223413in}}%
\pgfpathlineto{\pgfqpoint{1.303710in}{2.223413in}}%
\pgfpathlineto{\pgfqpoint{1.311772in}{2.223399in}}%
\pgfpathlineto{\pgfqpoint{1.319834in}{2.223399in}}%
\pgfpathlineto{\pgfqpoint{1.327896in}{2.223399in}}%
\pgfpathlineto{\pgfqpoint{1.335958in}{2.223399in}}%
\pgfpathlineto{\pgfqpoint{1.344020in}{2.223313in}}%
\pgfpathlineto{\pgfqpoint{1.352082in}{2.223313in}}%
\pgfpathlineto{\pgfqpoint{1.360144in}{2.223313in}}%
\pgfpathlineto{\pgfqpoint{1.368206in}{2.223313in}}%
\pgfpathlineto{\pgfqpoint{1.376268in}{2.223313in}}%
\pgfpathlineto{\pgfqpoint{1.384330in}{2.223313in}}%
\pgfpathlineto{\pgfqpoint{1.392392in}{2.223313in}}%
\pgfpathlineto{\pgfqpoint{1.400454in}{2.223182in}}%
\pgfpathlineto{\pgfqpoint{1.408516in}{2.223182in}}%
\pgfpathlineto{\pgfqpoint{1.416578in}{2.223182in}}%
\pgfpathlineto{\pgfqpoint{1.424640in}{2.223182in}}%
\pgfpathlineto{\pgfqpoint{1.432702in}{2.223182in}}%
\pgfpathlineto{\pgfqpoint{1.440764in}{2.223182in}}%
\pgfpathlineto{\pgfqpoint{1.448826in}{2.223182in}}%
\pgfpathlineto{\pgfqpoint{1.456888in}{2.223182in}}%
\pgfpathlineto{\pgfqpoint{1.464950in}{2.223182in}}%
\pgfpathlineto{\pgfqpoint{1.473012in}{2.223182in}}%
\pgfpathlineto{\pgfqpoint{1.481074in}{2.223182in}}%
\pgfpathlineto{\pgfqpoint{1.489136in}{2.223182in}}%
\pgfpathlineto{\pgfqpoint{1.497198in}{2.223182in}}%
\pgfpathlineto{\pgfqpoint{1.505259in}{2.223182in}}%
\pgfpathlineto{\pgfqpoint{1.513321in}{2.223182in}}%
\pgfpathlineto{\pgfqpoint{1.521383in}{2.223182in}}%
\pgfpathlineto{\pgfqpoint{1.529445in}{2.223182in}}%
\pgfpathlineto{\pgfqpoint{1.537507in}{2.223182in}}%
\pgfpathlineto{\pgfqpoint{1.545569in}{2.223182in}}%
\pgfpathlineto{\pgfqpoint{1.553631in}{2.223182in}}%
\pgfpathlineto{\pgfqpoint{1.561693in}{2.223182in}}%
\pgfpathlineto{\pgfqpoint{1.569755in}{2.223182in}}%
\pgfpathlineto{\pgfqpoint{1.577817in}{2.223182in}}%
\pgfpathlineto{\pgfqpoint{1.585879in}{2.223182in}}%
\pgfpathlineto{\pgfqpoint{1.593941in}{2.223182in}}%
\pgfpathlineto{\pgfqpoint{1.602003in}{2.223182in}}%
\pgfpathlineto{\pgfqpoint{1.610065in}{2.223182in}}%
\pgfpathlineto{\pgfqpoint{1.618127in}{2.223182in}}%
\pgfpathlineto{\pgfqpoint{1.626189in}{2.223182in}}%
\pgfpathlineto{\pgfqpoint{1.634251in}{2.223182in}}%
\pgfpathlineto{\pgfqpoint{1.642313in}{2.223172in}}%
\pgfpathlineto{\pgfqpoint{1.650375in}{2.223172in}}%
\pgfpathlineto{\pgfqpoint{1.658437in}{2.223172in}}%
\pgfpathlineto{\pgfqpoint{1.666499in}{2.223172in}}%
\pgfpathlineto{\pgfqpoint{1.674561in}{2.223172in}}%
\pgfpathlineto{\pgfqpoint{1.682623in}{2.223172in}}%
\pgfpathlineto{\pgfqpoint{1.690685in}{2.223172in}}%
\pgfpathlineto{\pgfqpoint{1.698747in}{2.223172in}}%
\pgfpathlineto{\pgfqpoint{1.706809in}{2.223172in}}%
\pgfpathlineto{\pgfqpoint{1.714871in}{2.223172in}}%
\pgfpathlineto{\pgfqpoint{1.722932in}{2.223172in}}%
\pgfpathlineto{\pgfqpoint{1.730994in}{2.223172in}}%
\pgfpathlineto{\pgfqpoint{1.739056in}{2.223170in}}%
\pgfpathlineto{\pgfqpoint{1.747118in}{2.223170in}}%
\pgfpathlineto{\pgfqpoint{1.755180in}{2.223170in}}%
\pgfpathlineto{\pgfqpoint{1.763242in}{2.223170in}}%
\pgfpathlineto{\pgfqpoint{1.771304in}{2.223170in}}%
\pgfpathlineto{\pgfqpoint{1.779366in}{2.223170in}}%
\pgfpathlineto{\pgfqpoint{1.787428in}{2.223170in}}%
\pgfpathlineto{\pgfqpoint{1.795490in}{2.223170in}}%
\pgfpathlineto{\pgfqpoint{1.803552in}{2.223170in}}%
\pgfpathlineto{\pgfqpoint{1.811614in}{2.223170in}}%
\pgfpathlineto{\pgfqpoint{1.819676in}{2.178310in}}%
\pgfpathlineto{\pgfqpoint{1.827738in}{2.178310in}}%
\pgfpathlineto{\pgfqpoint{1.835800in}{2.178310in}}%
\pgfpathlineto{\pgfqpoint{1.843862in}{2.178310in}}%
\pgfpathlineto{\pgfqpoint{1.851924in}{2.178310in}}%
\pgfpathlineto{\pgfqpoint{1.859986in}{2.178310in}}%
\pgfpathlineto{\pgfqpoint{1.868048in}{2.178310in}}%
\pgfpathlineto{\pgfqpoint{1.876110in}{2.170728in}}%
\pgfpathlineto{\pgfqpoint{1.884172in}{2.170728in}}%
\pgfpathlineto{\pgfqpoint{1.892234in}{2.170728in}}%
\pgfpathlineto{\pgfqpoint{1.900296in}{2.170728in}}%
\pgfpathlineto{\pgfqpoint{1.908358in}{2.170728in}}%
\pgfpathlineto{\pgfqpoint{1.916420in}{2.170728in}}%
\pgfpathlineto{\pgfqpoint{1.924482in}{2.170728in}}%
\pgfpathlineto{\pgfqpoint{1.932544in}{2.170728in}}%
\pgfpathlineto{\pgfqpoint{1.940605in}{2.170728in}}%
\pgfpathlineto{\pgfqpoint{1.948667in}{2.170728in}}%
\pgfpathlineto{\pgfqpoint{1.956729in}{2.170728in}}%
\pgfpathlineto{\pgfqpoint{1.964791in}{2.170728in}}%
\pgfpathlineto{\pgfqpoint{1.972853in}{2.170728in}}%
\pgfpathlineto{\pgfqpoint{1.980915in}{2.170728in}}%
\pgfpathlineto{\pgfqpoint{1.988977in}{2.170728in}}%
\pgfpathlineto{\pgfqpoint{1.988977in}{2.099659in}}%
\pgfpathlineto{\pgfqpoint{1.988977in}{2.099659in}}%
\pgfpathlineto{\pgfqpoint{1.980915in}{2.099659in}}%
\pgfpathlineto{\pgfqpoint{1.972853in}{2.099659in}}%
\pgfpathlineto{\pgfqpoint{1.964791in}{2.099659in}}%
\pgfpathlineto{\pgfqpoint{1.956729in}{2.099659in}}%
\pgfpathlineto{\pgfqpoint{1.948667in}{2.099659in}}%
\pgfpathlineto{\pgfqpoint{1.940605in}{2.099659in}}%
\pgfpathlineto{\pgfqpoint{1.932544in}{2.099659in}}%
\pgfpathlineto{\pgfqpoint{1.924482in}{2.099659in}}%
\pgfpathlineto{\pgfqpoint{1.916420in}{2.099659in}}%
\pgfpathlineto{\pgfqpoint{1.908358in}{2.099659in}}%
\pgfpathlineto{\pgfqpoint{1.900296in}{2.099659in}}%
\pgfpathlineto{\pgfqpoint{1.892234in}{2.099659in}}%
\pgfpathlineto{\pgfqpoint{1.884172in}{2.099659in}}%
\pgfpathlineto{\pgfqpoint{1.876110in}{2.099659in}}%
\pgfpathlineto{\pgfqpoint{1.868048in}{2.106051in}}%
\pgfpathlineto{\pgfqpoint{1.859986in}{2.106051in}}%
\pgfpathlineto{\pgfqpoint{1.851924in}{2.106051in}}%
\pgfpathlineto{\pgfqpoint{1.843862in}{2.106051in}}%
\pgfpathlineto{\pgfqpoint{1.835800in}{2.106051in}}%
\pgfpathlineto{\pgfqpoint{1.827738in}{2.106051in}}%
\pgfpathlineto{\pgfqpoint{1.819676in}{2.106051in}}%
\pgfpathlineto{\pgfqpoint{1.811614in}{2.124061in}}%
\pgfpathlineto{\pgfqpoint{1.803552in}{2.124061in}}%
\pgfpathlineto{\pgfqpoint{1.795490in}{2.124061in}}%
\pgfpathlineto{\pgfqpoint{1.787428in}{2.124061in}}%
\pgfpathlineto{\pgfqpoint{1.779366in}{2.124061in}}%
\pgfpathlineto{\pgfqpoint{1.771304in}{2.124061in}}%
\pgfpathlineto{\pgfqpoint{1.763242in}{2.124061in}}%
\pgfpathlineto{\pgfqpoint{1.755180in}{2.124061in}}%
\pgfpathlineto{\pgfqpoint{1.747118in}{2.124061in}}%
\pgfpathlineto{\pgfqpoint{1.739056in}{2.124061in}}%
\pgfpathlineto{\pgfqpoint{1.730994in}{2.124082in}}%
\pgfpathlineto{\pgfqpoint{1.722932in}{2.124082in}}%
\pgfpathlineto{\pgfqpoint{1.714871in}{2.124082in}}%
\pgfpathlineto{\pgfqpoint{1.706809in}{2.124082in}}%
\pgfpathlineto{\pgfqpoint{1.698747in}{2.124082in}}%
\pgfpathlineto{\pgfqpoint{1.690685in}{2.124082in}}%
\pgfpathlineto{\pgfqpoint{1.682623in}{2.124082in}}%
\pgfpathlineto{\pgfqpoint{1.674561in}{2.124082in}}%
\pgfpathlineto{\pgfqpoint{1.666499in}{2.124082in}}%
\pgfpathlineto{\pgfqpoint{1.658437in}{2.124082in}}%
\pgfpathlineto{\pgfqpoint{1.650375in}{2.124082in}}%
\pgfpathlineto{\pgfqpoint{1.642313in}{2.124082in}}%
\pgfpathlineto{\pgfqpoint{1.634251in}{2.124104in}}%
\pgfpathlineto{\pgfqpoint{1.626189in}{2.124104in}}%
\pgfpathlineto{\pgfqpoint{1.618127in}{2.124104in}}%
\pgfpathlineto{\pgfqpoint{1.610065in}{2.124104in}}%
\pgfpathlineto{\pgfqpoint{1.602003in}{2.124104in}}%
\pgfpathlineto{\pgfqpoint{1.593941in}{2.124104in}}%
\pgfpathlineto{\pgfqpoint{1.585879in}{2.124104in}}%
\pgfpathlineto{\pgfqpoint{1.577817in}{2.124104in}}%
\pgfpathlineto{\pgfqpoint{1.569755in}{2.124104in}}%
\pgfpathlineto{\pgfqpoint{1.561693in}{2.124104in}}%
\pgfpathlineto{\pgfqpoint{1.553631in}{2.124104in}}%
\pgfpathlineto{\pgfqpoint{1.545569in}{2.124104in}}%
\pgfpathlineto{\pgfqpoint{1.537507in}{2.124104in}}%
\pgfpathlineto{\pgfqpoint{1.529445in}{2.124104in}}%
\pgfpathlineto{\pgfqpoint{1.521383in}{2.124104in}}%
\pgfpathlineto{\pgfqpoint{1.513321in}{2.124104in}}%
\pgfpathlineto{\pgfqpoint{1.505259in}{2.124104in}}%
\pgfpathlineto{\pgfqpoint{1.497198in}{2.124104in}}%
\pgfpathlineto{\pgfqpoint{1.489136in}{2.124104in}}%
\pgfpathlineto{\pgfqpoint{1.481074in}{2.124104in}}%
\pgfpathlineto{\pgfqpoint{1.473012in}{2.124104in}}%
\pgfpathlineto{\pgfqpoint{1.464950in}{2.124104in}}%
\pgfpathlineto{\pgfqpoint{1.456888in}{2.124104in}}%
\pgfpathlineto{\pgfqpoint{1.448826in}{2.124104in}}%
\pgfpathlineto{\pgfqpoint{1.440764in}{2.124104in}}%
\pgfpathlineto{\pgfqpoint{1.432702in}{2.124104in}}%
\pgfpathlineto{\pgfqpoint{1.424640in}{2.124104in}}%
\pgfpathlineto{\pgfqpoint{1.416578in}{2.124104in}}%
\pgfpathlineto{\pgfqpoint{1.408516in}{2.124104in}}%
\pgfpathlineto{\pgfqpoint{1.400454in}{2.124104in}}%
\pgfpathlineto{\pgfqpoint{1.392392in}{2.124378in}}%
\pgfpathlineto{\pgfqpoint{1.384330in}{2.124378in}}%
\pgfpathlineto{\pgfqpoint{1.376268in}{2.124378in}}%
\pgfpathlineto{\pgfqpoint{1.368206in}{2.124378in}}%
\pgfpathlineto{\pgfqpoint{1.360144in}{2.124378in}}%
\pgfpathlineto{\pgfqpoint{1.352082in}{2.124378in}}%
\pgfpathlineto{\pgfqpoint{1.344020in}{2.124378in}}%
\pgfpathlineto{\pgfqpoint{1.335958in}{2.124558in}}%
\pgfpathlineto{\pgfqpoint{1.327896in}{2.124558in}}%
\pgfpathlineto{\pgfqpoint{1.319834in}{2.124558in}}%
\pgfpathlineto{\pgfqpoint{1.311772in}{2.124558in}}%
\pgfpathlineto{\pgfqpoint{1.303710in}{2.124587in}}%
\pgfpathlineto{\pgfqpoint{1.295648in}{2.124587in}}%
\pgfpathlineto{\pgfqpoint{1.287586in}{2.124587in}}%
\pgfpathlineto{\pgfqpoint{1.279525in}{2.124587in}}%
\pgfpathlineto{\pgfqpoint{1.271463in}{2.124587in}}%
\pgfpathlineto{\pgfqpoint{1.263401in}{2.124587in}}%
\pgfpathlineto{\pgfqpoint{1.255339in}{2.124587in}}%
\pgfpathlineto{\pgfqpoint{1.247277in}{2.124587in}}%
\pgfpathlineto{\pgfqpoint{1.239215in}{2.124587in}}%
\pgfpathlineto{\pgfqpoint{1.231153in}{2.124587in}}%
\pgfpathlineto{\pgfqpoint{1.223091in}{2.124587in}}%
\pgfpathlineto{\pgfqpoint{1.215029in}{2.124587in}}%
\pgfpathlineto{\pgfqpoint{1.206967in}{2.124587in}}%
\pgfpathlineto{\pgfqpoint{1.198905in}{2.124587in}}%
\pgfpathlineto{\pgfqpoint{1.190843in}{2.124587in}}%
\pgfpathlineto{\pgfqpoint{1.182781in}{2.124587in}}%
\pgfpathlineto{\pgfqpoint{1.174719in}{2.130070in}}%
\pgfpathlineto{\pgfqpoint{1.166657in}{2.130070in}}%
\pgfpathlineto{\pgfqpoint{1.158595in}{2.130070in}}%
\pgfpathlineto{\pgfqpoint{1.150533in}{2.130070in}}%
\pgfpathlineto{\pgfqpoint{1.142471in}{2.140587in}}%
\pgfpathlineto{\pgfqpoint{1.134409in}{2.140587in}}%
\pgfpathlineto{\pgfqpoint{1.126347in}{2.140587in}}%
\pgfpathlineto{\pgfqpoint{1.118285in}{2.202149in}}%
\pgfpathlineto{\pgfqpoint{1.110223in}{2.202968in}}%
\pgfpathlineto{\pgfqpoint{1.102161in}{2.202968in}}%
\pgfpathlineto{\pgfqpoint{1.094099in}{2.202968in}}%
\pgfpathlineto{\pgfqpoint{1.086037in}{2.202968in}}%
\pgfpathlineto{\pgfqpoint{1.077975in}{2.202968in}}%
\pgfpathlineto{\pgfqpoint{1.069914in}{2.205074in}}%
\pgfpathlineto{\pgfqpoint{1.061852in}{2.205529in}}%
\pgfpathlineto{\pgfqpoint{1.053790in}{2.205529in}}%
\pgfpathlineto{\pgfqpoint{1.045728in}{2.205529in}}%
\pgfpathlineto{\pgfqpoint{1.037666in}{2.205529in}}%
\pgfpathlineto{\pgfqpoint{1.029604in}{2.205529in}}%
\pgfpathlineto{\pgfqpoint{1.021542in}{2.205529in}}%
\pgfpathlineto{\pgfqpoint{1.013480in}{2.205529in}}%
\pgfpathlineto{\pgfqpoint{1.005418in}{2.205529in}}%
\pgfpathlineto{\pgfqpoint{0.997356in}{2.213565in}}%
\pgfpathlineto{\pgfqpoint{0.989294in}{2.229222in}}%
\pgfpathlineto{\pgfqpoint{0.981232in}{2.240332in}}%
\pgfpathlineto{\pgfqpoint{0.973170in}{2.245364in}}%
\pgfpathlineto{\pgfqpoint{0.965108in}{2.255047in}}%
\pgfpathlineto{\pgfqpoint{0.957046in}{2.255047in}}%
\pgfpathlineto{\pgfqpoint{0.948984in}{2.255047in}}%
\pgfpathlineto{\pgfqpoint{0.940922in}{2.255047in}}%
\pgfpathlineto{\pgfqpoint{0.932860in}{2.255047in}}%
\pgfpathlineto{\pgfqpoint{0.924798in}{2.255048in}}%
\pgfpathlineto{\pgfqpoint{0.916736in}{2.255048in}}%
\pgfpathlineto{\pgfqpoint{0.908674in}{2.255048in}}%
\pgfpathlineto{\pgfqpoint{0.900612in}{2.255048in}}%
\pgfpathlineto{\pgfqpoint{0.892550in}{2.255048in}}%
\pgfpathlineto{\pgfqpoint{0.884488in}{2.255048in}}%
\pgfpathlineto{\pgfqpoint{0.876426in}{2.255048in}}%
\pgfpathlineto{\pgfqpoint{0.868364in}{2.255048in}}%
\pgfpathlineto{\pgfqpoint{0.860302in}{2.255048in}}%
\pgfpathlineto{\pgfqpoint{0.852241in}{2.255048in}}%
\pgfpathlineto{\pgfqpoint{0.844179in}{2.255048in}}%
\pgfpathlineto{\pgfqpoint{0.836117in}{2.255048in}}%
\pgfpathlineto{\pgfqpoint{0.828055in}{2.255048in}}%
\pgfpathlineto{\pgfqpoint{0.819993in}{2.255048in}}%
\pgfpathlineto{\pgfqpoint{0.811931in}{2.255048in}}%
\pgfpathlineto{\pgfqpoint{0.803869in}{2.255048in}}%
\pgfpathlineto{\pgfqpoint{0.795807in}{2.255048in}}%
\pgfpathlineto{\pgfqpoint{0.787745in}{2.255048in}}%
\pgfpathlineto{\pgfqpoint{0.779683in}{2.255048in}}%
\pgfpathlineto{\pgfqpoint{0.771621in}{2.255048in}}%
\pgfpathlineto{\pgfqpoint{0.763559in}{2.255048in}}%
\pgfpathlineto{\pgfqpoint{0.755497in}{2.255312in}}%
\pgfpathlineto{\pgfqpoint{0.747435in}{2.255312in}}%
\pgfpathlineto{\pgfqpoint{0.739373in}{2.255312in}}%
\pgfpathlineto{\pgfqpoint{0.731311in}{2.255312in}}%
\pgfpathlineto{\pgfqpoint{0.723249in}{2.255312in}}%
\pgfpathlineto{\pgfqpoint{0.715187in}{2.255312in}}%
\pgfpathlineto{\pgfqpoint{0.707125in}{2.255312in}}%
\pgfpathlineto{\pgfqpoint{0.699063in}{2.255312in}}%
\pgfpathlineto{\pgfqpoint{0.691001in}{2.255312in}}%
\pgfpathlineto{\pgfqpoint{0.682939in}{2.255312in}}%
\pgfpathlineto{\pgfqpoint{0.674877in}{2.255312in}}%
\pgfpathlineto{\pgfqpoint{0.666815in}{2.255312in}}%
\pgfpathlineto{\pgfqpoint{0.658753in}{2.255312in}}%
\pgfpathlineto{\pgfqpoint{0.650691in}{2.255315in}}%
\pgfpathlineto{\pgfqpoint{0.642629in}{2.255315in}}%
\pgfpathlineto{\pgfqpoint{0.634568in}{2.255315in}}%
\pgfpathlineto{\pgfqpoint{0.626506in}{2.260165in}}%
\pgfpathlineto{\pgfqpoint{0.618444in}{2.260165in}}%
\pgfpathlineto{\pgfqpoint{0.610382in}{2.260179in}}%
\pgfpathlineto{\pgfqpoint{0.602320in}{2.260179in}}%
\pgfpathlineto{\pgfqpoint{0.594258in}{2.260179in}}%
\pgfpathlineto{\pgfqpoint{0.586196in}{2.266207in}}%
\pgfpathlineto{\pgfqpoint{0.578134in}{2.310034in}}%
\pgfpathlineto{\pgfqpoint{0.570072in}{2.310034in}}%
\pgfpathlineto{\pgfqpoint{0.562010in}{2.311350in}}%
\pgfpathlineto{\pgfqpoint{0.553948in}{2.311350in}}%
\pgfpathlineto{\pgfqpoint{0.545886in}{2.313638in}}%
\pgfpathlineto{\pgfqpoint{0.537824in}{2.335929in}}%
\pgfpathlineto{\pgfqpoint{0.529762in}{2.345063in}}%
\pgfpathlineto{\pgfqpoint{0.521700in}{2.345063in}}%
\pgfpathlineto{\pgfqpoint{0.513638in}{2.345063in}}%
\pgfpathlineto{\pgfqpoint{0.505576in}{2.345132in}}%
\pgfpathlineto{\pgfqpoint{0.497514in}{2.345132in}}%
\pgfpathlineto{\pgfqpoint{0.489452in}{2.345132in}}%
\pgfpathlineto{\pgfqpoint{0.481390in}{2.345594in}}%
\pgfpathlineto{\pgfqpoint{0.473328in}{2.353728in}}%
\pgfpathlineto{\pgfqpoint{0.465266in}{2.353728in}}%
\pgfpathlineto{\pgfqpoint{0.457204in}{2.353728in}}%
\pgfpathlineto{\pgfqpoint{0.449142in}{2.353728in}}%
\pgfpathlineto{\pgfqpoint{0.441080in}{2.355677in}}%
\pgfpathlineto{\pgfqpoint{0.433018in}{2.355677in}}%
\pgfpathlineto{\pgfqpoint{0.424956in}{2.355826in}}%
\pgfpathlineto{\pgfqpoint{0.416895in}{2.401728in}}%
\pgfpathlineto{\pgfqpoint{0.408833in}{2.405416in}}%
\pgfpathlineto{\pgfqpoint{0.400771in}{2.407614in}}%
\pgfpathlineto{\pgfqpoint{0.392709in}{2.408003in}}%
\pgfpathlineto{\pgfqpoint{0.384647in}{2.437872in}}%
\pgfpathlineto{\pgfqpoint{0.376585in}{2.488840in}}%
\pgfpathlineto{\pgfqpoint{0.368523in}{2.519071in}}%
\pgfpathclose%
\pgfusepath{fill}%
\end{pgfscope}%
\begin{pgfscope}%
\pgfpathrectangle{\pgfqpoint{0.287500in}{0.375000in}}{\pgfqpoint{1.782500in}{2.265000in}}%
\pgfusepath{clip}%
\pgfsetroundcap%
\pgfsetroundjoin%
\pgfsetlinewidth{1.505625pt}%
\definecolor{currentstroke}{rgb}{0.121569,0.466667,0.705882}%
\pgfsetstrokecolor{currentstroke}%
\pgfsetdash{}{0pt}%
\pgfpathmoveto{\pgfqpoint{0.368523in}{2.516915in}}%
\pgfpathlineto{\pgfqpoint{0.384647in}{2.407779in}}%
\pgfpathlineto{\pgfqpoint{0.392709in}{2.407779in}}%
\pgfpathlineto{\pgfqpoint{0.400771in}{2.392879in}}%
\pgfpathlineto{\pgfqpoint{0.408833in}{2.356249in}}%
\pgfpathlineto{\pgfqpoint{0.416895in}{2.304848in}}%
\pgfpathlineto{\pgfqpoint{0.424956in}{2.270333in}}%
\pgfpathlineto{\pgfqpoint{0.433018in}{2.258601in}}%
\pgfpathlineto{\pgfqpoint{0.449142in}{2.258601in}}%
\pgfpathlineto{\pgfqpoint{0.457204in}{2.255649in}}%
\pgfpathlineto{\pgfqpoint{0.465266in}{2.255649in}}%
\pgfpathlineto{\pgfqpoint{0.473328in}{2.188479in}}%
\pgfpathlineto{\pgfqpoint{0.481390in}{2.168290in}}%
\pgfpathlineto{\pgfqpoint{0.521700in}{2.168290in}}%
\pgfpathlineto{\pgfqpoint{0.529762in}{2.153017in}}%
\pgfpathlineto{\pgfqpoint{0.537824in}{2.153017in}}%
\pgfpathlineto{\pgfqpoint{0.545886in}{2.139391in}}%
\pgfpathlineto{\pgfqpoint{0.594258in}{2.139387in}}%
\pgfpathlineto{\pgfqpoint{0.602320in}{2.127196in}}%
\pgfpathlineto{\pgfqpoint{0.674877in}{2.127196in}}%
\pgfpathlineto{\pgfqpoint{0.682939in}{2.105480in}}%
\pgfpathlineto{\pgfqpoint{0.715187in}{2.101593in}}%
\pgfpathlineto{\pgfqpoint{0.739373in}{2.101593in}}%
\pgfpathlineto{\pgfqpoint{0.747435in}{2.089379in}}%
\pgfpathlineto{\pgfqpoint{0.771621in}{2.089379in}}%
\pgfpathlineto{\pgfqpoint{0.779683in}{2.066936in}}%
\pgfpathlineto{\pgfqpoint{0.787745in}{2.066936in}}%
\pgfpathlineto{\pgfqpoint{0.795807in}{2.055863in}}%
\pgfpathlineto{\pgfqpoint{0.884488in}{2.055863in}}%
\pgfpathlineto{\pgfqpoint{0.892550in}{2.015529in}}%
\pgfpathlineto{\pgfqpoint{0.900612in}{2.015529in}}%
\pgfpathlineto{\pgfqpoint{0.908674in}{1.994351in}}%
\pgfpathlineto{\pgfqpoint{0.997356in}{1.994351in}}%
\pgfpathlineto{\pgfqpoint{1.005418in}{1.956078in}}%
\pgfpathlineto{\pgfqpoint{1.037666in}{1.956078in}}%
\pgfpathlineto{\pgfqpoint{1.045728in}{1.860996in}}%
\pgfpathlineto{\pgfqpoint{1.077975in}{1.860996in}}%
\pgfpathlineto{\pgfqpoint{1.086037in}{1.839068in}}%
\pgfpathlineto{\pgfqpoint{1.134409in}{1.839068in}}%
\pgfpathlineto{\pgfqpoint{1.142471in}{1.809455in}}%
\pgfpathlineto{\pgfqpoint{1.368206in}{1.809455in}}%
\pgfpathlineto{\pgfqpoint{1.376268in}{1.778912in}}%
\pgfpathlineto{\pgfqpoint{1.392392in}{1.778912in}}%
\pgfpathlineto{\pgfqpoint{1.400454in}{1.697637in}}%
\pgfpathlineto{\pgfqpoint{1.456888in}{1.697637in}}%
\pgfpathlineto{\pgfqpoint{1.464950in}{1.645311in}}%
\pgfpathlineto{\pgfqpoint{1.626189in}{1.645311in}}%
\pgfpathlineto{\pgfqpoint{1.634251in}{1.639032in}}%
\pgfpathlineto{\pgfqpoint{1.876110in}{1.639032in}}%
\pgfpathlineto{\pgfqpoint{1.884172in}{1.636854in}}%
\pgfpathlineto{\pgfqpoint{1.988977in}{1.636854in}}%
\pgfpathlineto{\pgfqpoint{1.988977in}{1.636854in}}%
\pgfusepath{stroke}%
\end{pgfscope}%
\begin{pgfscope}%
\pgfpathrectangle{\pgfqpoint{0.287500in}{0.375000in}}{\pgfqpoint{1.782500in}{2.265000in}}%
\pgfusepath{clip}%
\pgfsetroundcap%
\pgfsetroundjoin%
\pgfsetlinewidth{1.505625pt}%
\definecolor{currentstroke}{rgb}{1.000000,0.498039,0.054902}%
\pgfsetstrokecolor{currentstroke}%
\pgfsetdash{}{0pt}%
\pgfpathmoveto{\pgfqpoint{0.368523in}{2.456471in}}%
\pgfpathlineto{\pgfqpoint{0.376585in}{2.397667in}}%
\pgfpathlineto{\pgfqpoint{0.384647in}{2.397667in}}%
\pgfpathlineto{\pgfqpoint{0.392709in}{2.395521in}}%
\pgfpathlineto{\pgfqpoint{0.408833in}{2.395521in}}%
\pgfpathlineto{\pgfqpoint{0.416895in}{2.307345in}}%
\pgfpathlineto{\pgfqpoint{0.457204in}{2.307345in}}%
\pgfpathlineto{\pgfqpoint{0.465266in}{2.266065in}}%
\pgfpathlineto{\pgfqpoint{0.529762in}{2.266065in}}%
\pgfpathlineto{\pgfqpoint{0.537824in}{2.225391in}}%
\pgfpathlineto{\pgfqpoint{0.545886in}{2.201839in}}%
\pgfpathlineto{\pgfqpoint{0.602320in}{2.201839in}}%
\pgfpathlineto{\pgfqpoint{0.610382in}{2.143571in}}%
\pgfpathlineto{\pgfqpoint{0.731311in}{2.143571in}}%
\pgfpathlineto{\pgfqpoint{0.739373in}{2.053151in}}%
\pgfpathlineto{\pgfqpoint{0.747435in}{2.053151in}}%
\pgfpathlineto{\pgfqpoint{0.755497in}{2.044756in}}%
\pgfpathlineto{\pgfqpoint{0.763559in}{2.027503in}}%
\pgfpathlineto{\pgfqpoint{0.828055in}{2.027503in}}%
\pgfpathlineto{\pgfqpoint{0.836117in}{2.019413in}}%
\pgfpathlineto{\pgfqpoint{0.844179in}{2.019413in}}%
\pgfpathlineto{\pgfqpoint{0.852241in}{2.016177in}}%
\pgfpathlineto{\pgfqpoint{0.908674in}{2.014776in}}%
\pgfpathlineto{\pgfqpoint{0.924798in}{2.013013in}}%
\pgfpathlineto{\pgfqpoint{0.940922in}{2.013013in}}%
\pgfpathlineto{\pgfqpoint{0.948984in}{1.842659in}}%
\pgfpathlineto{\pgfqpoint{0.965108in}{1.842659in}}%
\pgfpathlineto{\pgfqpoint{0.973170in}{1.840137in}}%
\pgfpathlineto{\pgfqpoint{0.989294in}{1.840137in}}%
\pgfpathlineto{\pgfqpoint{0.997356in}{1.768121in}}%
\pgfpathlineto{\pgfqpoint{1.005418in}{1.766442in}}%
\pgfpathlineto{\pgfqpoint{1.102161in}{1.766330in}}%
\pgfpathlineto{\pgfqpoint{1.110223in}{1.808871in}}%
\pgfpathlineto{\pgfqpoint{1.118285in}{1.718391in}}%
\pgfpathlineto{\pgfqpoint{1.134409in}{1.718391in}}%
\pgfpathlineto{\pgfqpoint{1.142471in}{1.496353in}}%
\pgfpathlineto{\pgfqpoint{1.279525in}{1.496353in}}%
\pgfpathlineto{\pgfqpoint{1.287586in}{0.987534in}}%
\pgfpathlineto{\pgfqpoint{1.585879in}{0.987192in}}%
\pgfpathlineto{\pgfqpoint{1.593941in}{0.806827in}}%
\pgfpathlineto{\pgfqpoint{1.755180in}{0.806827in}}%
\pgfpathlineto{\pgfqpoint{1.763242in}{0.622589in}}%
\pgfpathlineto{\pgfqpoint{1.988977in}{0.622589in}}%
\pgfpathlineto{\pgfqpoint{1.988977in}{0.622589in}}%
\pgfusepath{stroke}%
\end{pgfscope}%
\begin{pgfscope}%
\pgfpathrectangle{\pgfqpoint{0.287500in}{0.375000in}}{\pgfqpoint{1.782500in}{2.265000in}}%
\pgfusepath{clip}%
\pgfsetroundcap%
\pgfsetroundjoin%
\pgfsetlinewidth{1.505625pt}%
\definecolor{currentstroke}{rgb}{0.172549,0.627451,0.172549}%
\pgfsetstrokecolor{currentstroke}%
\pgfsetdash{}{0pt}%
\pgfpathmoveto{\pgfqpoint{0.368523in}{2.485824in}}%
\pgfpathlineto{\pgfqpoint{0.376585in}{2.473439in}}%
\pgfpathlineto{\pgfqpoint{0.384647in}{2.466188in}}%
\pgfpathlineto{\pgfqpoint{0.392709in}{2.421081in}}%
\pgfpathlineto{\pgfqpoint{0.400771in}{2.416195in}}%
\pgfpathlineto{\pgfqpoint{0.433018in}{2.416195in}}%
\pgfpathlineto{\pgfqpoint{0.441080in}{2.405595in}}%
\pgfpathlineto{\pgfqpoint{0.457204in}{2.404422in}}%
\pgfpathlineto{\pgfqpoint{0.465266in}{2.388456in}}%
\pgfpathlineto{\pgfqpoint{0.481390in}{2.388456in}}%
\pgfpathlineto{\pgfqpoint{0.489452in}{2.354367in}}%
\pgfpathlineto{\pgfqpoint{0.497514in}{2.282014in}}%
\pgfpathlineto{\pgfqpoint{0.513638in}{2.282014in}}%
\pgfpathlineto{\pgfqpoint{0.521700in}{2.146518in}}%
\pgfpathlineto{\pgfqpoint{0.586196in}{2.146518in}}%
\pgfpathlineto{\pgfqpoint{0.594258in}{2.069463in}}%
\pgfpathlineto{\pgfqpoint{0.642629in}{2.069463in}}%
\pgfpathlineto{\pgfqpoint{0.650691in}{1.998873in}}%
\pgfpathlineto{\pgfqpoint{0.699063in}{1.998873in}}%
\pgfpathlineto{\pgfqpoint{0.707125in}{1.973964in}}%
\pgfpathlineto{\pgfqpoint{0.828055in}{1.973964in}}%
\pgfpathlineto{\pgfqpoint{0.836117in}{1.969073in}}%
\pgfpathlineto{\pgfqpoint{0.844179in}{1.969073in}}%
\pgfpathlineto{\pgfqpoint{0.852241in}{1.888463in}}%
\pgfpathlineto{\pgfqpoint{0.997356in}{1.888463in}}%
\pgfpathlineto{\pgfqpoint{1.005418in}{1.860414in}}%
\pgfpathlineto{\pgfqpoint{1.206967in}{1.860414in}}%
\pgfpathlineto{\pgfqpoint{1.215029in}{1.855949in}}%
\pgfpathlineto{\pgfqpoint{1.327896in}{1.855949in}}%
\pgfpathlineto{\pgfqpoint{1.335958in}{1.845488in}}%
\pgfpathlineto{\pgfqpoint{1.368206in}{1.845488in}}%
\pgfpathlineto{\pgfqpoint{1.376268in}{1.681275in}}%
\pgfpathlineto{\pgfqpoint{1.440764in}{1.681275in}}%
\pgfpathlineto{\pgfqpoint{1.448826in}{1.642671in}}%
\pgfpathlineto{\pgfqpoint{1.456888in}{1.631906in}}%
\pgfpathlineto{\pgfqpoint{1.464950in}{1.441854in}}%
\pgfpathlineto{\pgfqpoint{1.553631in}{1.441854in}}%
\pgfpathlineto{\pgfqpoint{1.561693in}{1.334124in}}%
\pgfpathlineto{\pgfqpoint{1.811614in}{1.334124in}}%
\pgfpathlineto{\pgfqpoint{1.819676in}{1.225839in}}%
\pgfpathlineto{\pgfqpoint{1.835800in}{1.225839in}}%
\pgfpathlineto{\pgfqpoint{1.843862in}{1.100320in}}%
\pgfpathlineto{\pgfqpoint{1.932544in}{1.100320in}}%
\pgfpathlineto{\pgfqpoint{1.940605in}{1.074727in}}%
\pgfpathlineto{\pgfqpoint{1.988977in}{1.074727in}}%
\pgfpathlineto{\pgfqpoint{1.988977in}{1.074727in}}%
\pgfusepath{stroke}%
\end{pgfscope}%
\begin{pgfscope}%
\pgfpathrectangle{\pgfqpoint{0.287500in}{0.375000in}}{\pgfqpoint{1.782500in}{2.265000in}}%
\pgfusepath{clip}%
\pgfsetroundcap%
\pgfsetroundjoin%
\pgfsetlinewidth{1.505625pt}%
\definecolor{currentstroke}{rgb}{0.839216,0.152941,0.156863}%
\pgfsetstrokecolor{currentstroke}%
\pgfsetdash{}{0pt}%
\pgfpathmoveto{\pgfqpoint{0.368523in}{2.524770in}}%
\pgfpathlineto{\pgfqpoint{0.376585in}{2.508635in}}%
\pgfpathlineto{\pgfqpoint{0.384647in}{2.470765in}}%
\pgfpathlineto{\pgfqpoint{0.392709in}{2.446361in}}%
\pgfpathlineto{\pgfqpoint{0.408833in}{2.444784in}}%
\pgfpathlineto{\pgfqpoint{0.416895in}{2.439671in}}%
\pgfpathlineto{\pgfqpoint{0.424956in}{2.403728in}}%
\pgfpathlineto{\pgfqpoint{0.473328in}{2.402418in}}%
\pgfpathlineto{\pgfqpoint{0.481390in}{2.398304in}}%
\pgfpathlineto{\pgfqpoint{0.529762in}{2.397956in}}%
\pgfpathlineto{\pgfqpoint{0.537824in}{2.386978in}}%
\pgfpathlineto{\pgfqpoint{0.545886in}{2.368191in}}%
\pgfpathlineto{\pgfqpoint{0.562010in}{2.366914in}}%
\pgfpathlineto{\pgfqpoint{0.578134in}{2.366258in}}%
\pgfpathlineto{\pgfqpoint{0.586196in}{2.342229in}}%
\pgfpathlineto{\pgfqpoint{0.594258in}{2.339512in}}%
\pgfpathlineto{\pgfqpoint{0.626506in}{2.339505in}}%
\pgfpathlineto{\pgfqpoint{0.634568in}{2.337352in}}%
\pgfpathlineto{\pgfqpoint{0.965108in}{2.337211in}}%
\pgfpathlineto{\pgfqpoint{0.973170in}{2.292386in}}%
\pgfpathlineto{\pgfqpoint{0.981232in}{2.275519in}}%
\pgfpathlineto{\pgfqpoint{0.989294in}{2.254883in}}%
\pgfpathlineto{\pgfqpoint{0.997356in}{2.240388in}}%
\pgfpathlineto{\pgfqpoint{1.005418in}{2.235004in}}%
\pgfpathlineto{\pgfqpoint{1.069914in}{2.234721in}}%
\pgfpathlineto{\pgfqpoint{1.086037in}{2.233434in}}%
\pgfpathlineto{\pgfqpoint{1.118285in}{2.232943in}}%
\pgfpathlineto{\pgfqpoint{1.126347in}{2.186562in}}%
\pgfpathlineto{\pgfqpoint{1.142471in}{2.186562in}}%
\pgfpathlineto{\pgfqpoint{1.150533in}{2.181862in}}%
\pgfpathlineto{\pgfqpoint{1.174719in}{2.181862in}}%
\pgfpathlineto{\pgfqpoint{1.182781in}{2.179534in}}%
\pgfpathlineto{\pgfqpoint{1.811614in}{2.179188in}}%
\pgfpathlineto{\pgfqpoint{1.819676in}{2.144651in}}%
\pgfpathlineto{\pgfqpoint{1.868048in}{2.144651in}}%
\pgfpathlineto{\pgfqpoint{1.876110in}{2.137554in}}%
\pgfpathlineto{\pgfqpoint{1.988977in}{2.137554in}}%
\pgfpathlineto{\pgfqpoint{1.988977in}{2.137554in}}%
\pgfusepath{stroke}%
\end{pgfscope}%
\begin{pgfscope}%
\pgfsetrectcap%
\pgfsetmiterjoin%
\pgfsetlinewidth{0.000000pt}%
\definecolor{currentstroke}{rgb}{1.000000,1.000000,1.000000}%
\pgfsetstrokecolor{currentstroke}%
\pgfsetdash{}{0pt}%
\pgfpathmoveto{\pgfqpoint{0.287500in}{0.375000in}}%
\pgfpathlineto{\pgfqpoint{0.287500in}{2.640000in}}%
\pgfusepath{}%
\end{pgfscope}%
\begin{pgfscope}%
\pgfsetrectcap%
\pgfsetmiterjoin%
\pgfsetlinewidth{0.000000pt}%
\definecolor{currentstroke}{rgb}{1.000000,1.000000,1.000000}%
\pgfsetstrokecolor{currentstroke}%
\pgfsetdash{}{0pt}%
\pgfpathmoveto{\pgfqpoint{2.070000in}{0.375000in}}%
\pgfpathlineto{\pgfqpoint{2.070000in}{2.640000in}}%
\pgfusepath{}%
\end{pgfscope}%
\begin{pgfscope}%
\pgfsetrectcap%
\pgfsetmiterjoin%
\pgfsetlinewidth{0.000000pt}%
\definecolor{currentstroke}{rgb}{1.000000,1.000000,1.000000}%
\pgfsetstrokecolor{currentstroke}%
\pgfsetdash{}{0pt}%
\pgfpathmoveto{\pgfqpoint{0.287500in}{0.375000in}}%
\pgfpathlineto{\pgfqpoint{2.070000in}{0.375000in}}%
\pgfusepath{}%
\end{pgfscope}%
\begin{pgfscope}%
\pgfsetrectcap%
\pgfsetmiterjoin%
\pgfsetlinewidth{0.000000pt}%
\definecolor{currentstroke}{rgb}{1.000000,1.000000,1.000000}%
\pgfsetstrokecolor{currentstroke}%
\pgfsetdash{}{0pt}%
\pgfpathmoveto{\pgfqpoint{0.287500in}{2.640000in}}%
\pgfpathlineto{\pgfqpoint{2.070000in}{2.640000in}}%
\pgfusepath{}%
\end{pgfscope}%
\begin{pgfscope}%
\definecolor{textcolor}{rgb}{0.150000,0.150000,0.150000}%
\pgfsetstrokecolor{textcolor}%
\pgfsetfillcolor{textcolor}%
\pgftext[x=1.178750in,y=2.723333in,,base]{\color{textcolor}\rmfamily\fontsize{8.000000}{9.600000}\selectfont Embedded SinOne in 3D}%
\end{pgfscope}%
\begin{pgfscope}%
\pgfsetroundcap%
\pgfsetroundjoin%
\pgfsetlinewidth{1.505625pt}%
\definecolor{currentstroke}{rgb}{0.121569,0.466667,0.705882}%
\pgfsetstrokecolor{currentstroke}%
\pgfsetdash{}{0pt}%
\pgfpathmoveto{\pgfqpoint{0.362500in}{0.863443in}}%
\pgfpathlineto{\pgfqpoint{0.529167in}{0.863443in}}%
\pgfusepath{stroke}%
\end{pgfscope}%
\begin{pgfscope}%
\definecolor{textcolor}{rgb}{0.150000,0.150000,0.150000}%
\pgfsetstrokecolor{textcolor}%
\pgfsetfillcolor{textcolor}%
\pgftext[x=0.595833in,y=0.834277in,left,base]{\color{textcolor}\rmfamily\fontsize{6.000000}{7.200000}\selectfont random}%
\end{pgfscope}%
\begin{pgfscope}%
\pgfsetroundcap%
\pgfsetroundjoin%
\pgfsetlinewidth{1.505625pt}%
\definecolor{currentstroke}{rgb}{1.000000,0.498039,0.054902}%
\pgfsetstrokecolor{currentstroke}%
\pgfsetdash{}{0pt}%
\pgfpathmoveto{\pgfqpoint{0.362500in}{0.741129in}}%
\pgfpathlineto{\pgfqpoint{0.529167in}{0.741129in}}%
\pgfusepath{stroke}%
\end{pgfscope}%
\begin{pgfscope}%
\definecolor{textcolor}{rgb}{0.150000,0.150000,0.150000}%
\pgfsetstrokecolor{textcolor}%
\pgfsetfillcolor{textcolor}%
\pgftext[x=0.595833in,y=0.711962in,left,base]{\color{textcolor}\rmfamily\fontsize{6.000000}{7.200000}\selectfont 5 x DNGO retrain-reset}%
\end{pgfscope}%
\begin{pgfscope}%
\pgfsetroundcap%
\pgfsetroundjoin%
\pgfsetlinewidth{1.505625pt}%
\definecolor{currentstroke}{rgb}{0.172549,0.627451,0.172549}%
\pgfsetstrokecolor{currentstroke}%
\pgfsetdash{}{0pt}%
\pgfpathmoveto{\pgfqpoint{0.362500in}{0.618815in}}%
\pgfpathlineto{\pgfqpoint{0.529167in}{0.618815in}}%
\pgfusepath{stroke}%
\end{pgfscope}%
\begin{pgfscope}%
\definecolor{textcolor}{rgb}{0.150000,0.150000,0.150000}%
\pgfsetstrokecolor{textcolor}%
\pgfsetfillcolor{textcolor}%
\pgftext[x=0.595833in,y=0.589648in,left,base]{\color{textcolor}\rmfamily\fontsize{6.000000}{7.200000}\selectfont DNGO retrain-reset}%
\end{pgfscope}%
\begin{pgfscope}%
\pgfsetroundcap%
\pgfsetroundjoin%
\pgfsetlinewidth{1.505625pt}%
\definecolor{currentstroke}{rgb}{0.839216,0.152941,0.156863}%
\pgfsetstrokecolor{currentstroke}%
\pgfsetdash{}{0pt}%
\pgfpathmoveto{\pgfqpoint{0.362500in}{0.496500in}}%
\pgfpathlineto{\pgfqpoint{0.529167in}{0.496500in}}%
\pgfusepath{stroke}%
\end{pgfscope}%
\begin{pgfscope}%
\definecolor{textcolor}{rgb}{0.150000,0.150000,0.150000}%
\pgfsetstrokecolor{textcolor}%
\pgfsetfillcolor{textcolor}%
\pgftext[x=0.595833in,y=0.467334in,left,base]{\color{textcolor}\rmfamily\fontsize{6.000000}{7.200000}\selectfont GP}%
\end{pgfscope}%
\end{pgfpicture}%
\makeatother%
\endgroup%

            \end{subfigure}
            \caption{All plots shows the average Simple Regret over 10 runs with a $1/4$ standard derivation confidence interval.
            Note how the ensembled DNGO consistently outperforms the DNGO.
            See \cref{sec:appembedding} for accompanying Cumulative Regret plots.}
            \label{fig:embedding}
        \end{figure*}

    \subsection{Benchmarks}\label{sec:expbenchmark}

        The benchmarks were divided into four categories.
        However the most interesting observations are between different categories and has to do with dimensionality.

        DNGO consistently outperforms GP on smaller dimensional problems if they are nonsmooth or oscillatory (see e.g. Alpine01 and Griewank in \cref{sec:appbenchmark} respectively).
        For Alpone01 this most likely comes down to the choice of the infinitely differentiable SE kernel for the GP however, since it performs significantly better on the smooth Branin and Hartmann3 functions.
        Further investigation of Griewank showed that the GP only captured the long-range trend of $T$ (as defined in \cref{sec:exp}).

        Griewank provides an interesting insight into the ensemble in that you should not just blindly throw computational power after regularizing through an ensemble.
        In this benchmark the ensemble performed \emph{worse} than DNGO.
        The reason becomes apparent when plotting the sequence of observations made in the 2D space over time.
        The ensemble tends to get stuck in a local optimum whereas DNGO explores a bigger part of the space.
        This reinforced the intuition that an averaging ensemble would decrease random exploration which in some instances could be useful to some degree.
        See \cref{sec:appbenchmark} for details and \cref{fig:braningexploit} for a similar effect.

        There was no significant improvement by using an ensemble of DNGO in the low dimensions discussed so far.
        However in higher dimensions an ensemble improved performance in different ways even though it did not significantly outperform GP across all benchmarks.
        Firstly, it led to faster convergence for nonsmooth Corana 4D function and oscillatory Dolan 5D function.
        More interestingly it outperformed DNGO on the mostly boring Hartmann6 and LennardJones6 as well as the higher dimensional Rosenbrock 8D (see \cref{fig:benchmark}).

        We have to note though that these results most likely have been influenced by optimizing the network primarily on Hartmann3 and Branin.
        This is problematic because we might overfit on lower dimensional spaces.
        One could imagine that an improvement to the network could have made the gain achieved in higher dimensions by an ensemble insignificant.

        \begin{figure*}[t]
            \centering
            \begin{subfigure}[t]{0.45\textwidth}
                \centering
                % \resizebox{\linewidth}{!}{}
                %% Creator: Matplotlib, PGF backend
%%
%% To include the figure in your LaTeX document, write
%%   \input{<filename>.pgf}
%%
%% Make sure the required packages are loaded in your preamble
%%   \usepackage{pgf}
%%
%% Figures using additional raster images can only be included by \input if
%% they are in the same directory as the main LaTeX file. For loading figures
%% from other directories you can use the `import` package
%%   \usepackage{import}
%% and then include the figures with
%%   \import{<path to file>}{<filename>.pgf}
%%
%% Matplotlib used the following preamble
%%   \usepackage{gensymb}
%%   \usepackage{fontspec}
%%   \setmainfont{DejaVu Serif}
%%   \setsansfont{Arial}
%%   \setmonofont{DejaVu Sans Mono}
%%
\begingroup%
\makeatletter%
\begin{pgfpicture}%
\pgfpathrectangle{\pgfpointorigin}{\pgfqpoint{3.390000in}{2.095135in}}%
\pgfusepath{use as bounding box, clip}%
\begin{pgfscope}%
\pgfsetbuttcap%
\pgfsetmiterjoin%
\definecolor{currentfill}{rgb}{1.000000,1.000000,1.000000}%
\pgfsetfillcolor{currentfill}%
\pgfsetlinewidth{0.000000pt}%
\definecolor{currentstroke}{rgb}{1.000000,1.000000,1.000000}%
\pgfsetstrokecolor{currentstroke}%
\pgfsetdash{}{0pt}%
\pgfpathmoveto{\pgfqpoint{0.000000in}{0.000000in}}%
\pgfpathlineto{\pgfqpoint{3.390000in}{0.000000in}}%
\pgfpathlineto{\pgfqpoint{3.390000in}{2.095135in}}%
\pgfpathlineto{\pgfqpoint{0.000000in}{2.095135in}}%
\pgfpathclose%
\pgfusepath{fill}%
\end{pgfscope}%
\begin{pgfscope}%
\pgfsetbuttcap%
\pgfsetmiterjoin%
\definecolor{currentfill}{rgb}{0.917647,0.917647,0.949020}%
\pgfsetfillcolor{currentfill}%
\pgfsetlinewidth{0.000000pt}%
\definecolor{currentstroke}{rgb}{0.000000,0.000000,0.000000}%
\pgfsetstrokecolor{currentstroke}%
\pgfsetstrokeopacity{0.000000}%
\pgfsetdash{}{0pt}%
\pgfpathmoveto{\pgfqpoint{0.423750in}{0.261892in}}%
\pgfpathlineto{\pgfqpoint{3.051000in}{0.261892in}}%
\pgfpathlineto{\pgfqpoint{3.051000in}{1.843719in}}%
\pgfpathlineto{\pgfqpoint{0.423750in}{1.843719in}}%
\pgfpathclose%
\pgfusepath{fill}%
\end{pgfscope}%
\begin{pgfscope}%
\pgfpathrectangle{\pgfqpoint{0.423750in}{0.261892in}}{\pgfqpoint{2.627250in}{1.581827in}}%
\pgfusepath{clip}%
\pgfsetroundcap%
\pgfsetroundjoin%
\pgfsetlinewidth{0.803000pt}%
\definecolor{currentstroke}{rgb}{1.000000,1.000000,1.000000}%
\pgfsetstrokecolor{currentstroke}%
\pgfsetdash{}{0pt}%
\pgfpathmoveto{\pgfqpoint{0.543170in}{0.261892in}}%
\pgfpathlineto{\pgfqpoint{0.543170in}{1.843719in}}%
\pgfusepath{stroke}%
\end{pgfscope}%
\begin{pgfscope}%
\definecolor{textcolor}{rgb}{0.150000,0.150000,0.150000}%
\pgfsetstrokecolor{textcolor}%
\pgfsetfillcolor{textcolor}%
\pgftext[x=0.543170in,y=0.213281in,,top]{\color{textcolor}\rmfamily\fontsize{8.000000}{9.600000}\selectfont \(\displaystyle 0\)}%
\end{pgfscope}%
\begin{pgfscope}%
\pgfpathrectangle{\pgfqpoint{0.423750in}{0.261892in}}{\pgfqpoint{2.627250in}{1.581827in}}%
\pgfusepath{clip}%
\pgfsetroundcap%
\pgfsetroundjoin%
\pgfsetlinewidth{0.803000pt}%
\definecolor{currentstroke}{rgb}{1.000000,1.000000,1.000000}%
\pgfsetstrokecolor{currentstroke}%
\pgfsetdash{}{0pt}%
\pgfpathmoveto{\pgfqpoint{1.088469in}{0.261892in}}%
\pgfpathlineto{\pgfqpoint{1.088469in}{1.843719in}}%
\pgfusepath{stroke}%
\end{pgfscope}%
\begin{pgfscope}%
\definecolor{textcolor}{rgb}{0.150000,0.150000,0.150000}%
\pgfsetstrokecolor{textcolor}%
\pgfsetfillcolor{textcolor}%
\pgftext[x=1.088469in,y=0.213281in,,top]{\color{textcolor}\rmfamily\fontsize{8.000000}{9.600000}\selectfont \(\displaystyle 50\)}%
\end{pgfscope}%
\begin{pgfscope}%
\pgfpathrectangle{\pgfqpoint{0.423750in}{0.261892in}}{\pgfqpoint{2.627250in}{1.581827in}}%
\pgfusepath{clip}%
\pgfsetroundcap%
\pgfsetroundjoin%
\pgfsetlinewidth{0.803000pt}%
\definecolor{currentstroke}{rgb}{1.000000,1.000000,1.000000}%
\pgfsetstrokecolor{currentstroke}%
\pgfsetdash{}{0pt}%
\pgfpathmoveto{\pgfqpoint{1.633768in}{0.261892in}}%
\pgfpathlineto{\pgfqpoint{1.633768in}{1.843719in}}%
\pgfusepath{stroke}%
\end{pgfscope}%
\begin{pgfscope}%
\definecolor{textcolor}{rgb}{0.150000,0.150000,0.150000}%
\pgfsetstrokecolor{textcolor}%
\pgfsetfillcolor{textcolor}%
\pgftext[x=1.633768in,y=0.213281in,,top]{\color{textcolor}\rmfamily\fontsize{8.000000}{9.600000}\selectfont \(\displaystyle 100\)}%
\end{pgfscope}%
\begin{pgfscope}%
\pgfpathrectangle{\pgfqpoint{0.423750in}{0.261892in}}{\pgfqpoint{2.627250in}{1.581827in}}%
\pgfusepath{clip}%
\pgfsetroundcap%
\pgfsetroundjoin%
\pgfsetlinewidth{0.803000pt}%
\definecolor{currentstroke}{rgb}{1.000000,1.000000,1.000000}%
\pgfsetstrokecolor{currentstroke}%
\pgfsetdash{}{0pt}%
\pgfpathmoveto{\pgfqpoint{2.179067in}{0.261892in}}%
\pgfpathlineto{\pgfqpoint{2.179067in}{1.843719in}}%
\pgfusepath{stroke}%
\end{pgfscope}%
\begin{pgfscope}%
\definecolor{textcolor}{rgb}{0.150000,0.150000,0.150000}%
\pgfsetstrokecolor{textcolor}%
\pgfsetfillcolor{textcolor}%
\pgftext[x=2.179067in,y=0.213281in,,top]{\color{textcolor}\rmfamily\fontsize{8.000000}{9.600000}\selectfont \(\displaystyle 150\)}%
\end{pgfscope}%
\begin{pgfscope}%
\pgfpathrectangle{\pgfqpoint{0.423750in}{0.261892in}}{\pgfqpoint{2.627250in}{1.581827in}}%
\pgfusepath{clip}%
\pgfsetroundcap%
\pgfsetroundjoin%
\pgfsetlinewidth{0.803000pt}%
\definecolor{currentstroke}{rgb}{1.000000,1.000000,1.000000}%
\pgfsetstrokecolor{currentstroke}%
\pgfsetdash{}{0pt}%
\pgfpathmoveto{\pgfqpoint{2.724366in}{0.261892in}}%
\pgfpathlineto{\pgfqpoint{2.724366in}{1.843719in}}%
\pgfusepath{stroke}%
\end{pgfscope}%
\begin{pgfscope}%
\definecolor{textcolor}{rgb}{0.150000,0.150000,0.150000}%
\pgfsetstrokecolor{textcolor}%
\pgfsetfillcolor{textcolor}%
\pgftext[x=2.724366in,y=0.213281in,,top]{\color{textcolor}\rmfamily\fontsize{8.000000}{9.600000}\selectfont \(\displaystyle 200\)}%
\end{pgfscope}%
\begin{pgfscope}%
\definecolor{textcolor}{rgb}{0.150000,0.150000,0.150000}%
\pgfsetstrokecolor{textcolor}%
\pgfsetfillcolor{textcolor}%
\pgftext[x=1.737375in,y=0.050195in,,top]{\color{textcolor}\rmfamily\fontsize{8.000000}{9.600000}\selectfont Step}%
\end{pgfscope}%
\begin{pgfscope}%
\pgfpathrectangle{\pgfqpoint{0.423750in}{0.261892in}}{\pgfqpoint{2.627250in}{1.581827in}}%
\pgfusepath{clip}%
\pgfsetroundcap%
\pgfsetroundjoin%
\pgfsetlinewidth{0.803000pt}%
\definecolor{currentstroke}{rgb}{1.000000,1.000000,1.000000}%
\pgfsetstrokecolor{currentstroke}%
\pgfsetdash{}{0pt}%
\pgfpathmoveto{\pgfqpoint{0.423750in}{0.589646in}}%
\pgfpathlineto{\pgfqpoint{3.051000in}{0.589646in}}%
\pgfusepath{stroke}%
\end{pgfscope}%
\begin{pgfscope}%
\definecolor{textcolor}{rgb}{0.150000,0.150000,0.150000}%
\pgfsetstrokecolor{textcolor}%
\pgfsetfillcolor{textcolor}%
\pgftext[x=0.199212in,y=0.547437in,left,base]{\color{textcolor}\rmfamily\fontsize{8.000000}{9.600000}\selectfont \(\displaystyle 10^{1}\)}%
\end{pgfscope}%
\begin{pgfscope}%
\pgfpathrectangle{\pgfqpoint{0.423750in}{0.261892in}}{\pgfqpoint{2.627250in}{1.581827in}}%
\pgfusepath{clip}%
\pgfsetroundcap%
\pgfsetroundjoin%
\pgfsetlinewidth{0.803000pt}%
\definecolor{currentstroke}{rgb}{1.000000,1.000000,1.000000}%
\pgfsetstrokecolor{currentstroke}%
\pgfsetdash{}{0pt}%
\pgfpathmoveto{\pgfqpoint{0.423750in}{0.983092in}}%
\pgfpathlineto{\pgfqpoint{3.051000in}{0.983092in}}%
\pgfusepath{stroke}%
\end{pgfscope}%
\begin{pgfscope}%
\definecolor{textcolor}{rgb}{0.150000,0.150000,0.150000}%
\pgfsetstrokecolor{textcolor}%
\pgfsetfillcolor{textcolor}%
\pgftext[x=0.199212in,y=0.940883in,left,base]{\color{textcolor}\rmfamily\fontsize{8.000000}{9.600000}\selectfont \(\displaystyle 10^{2}\)}%
\end{pgfscope}%
\begin{pgfscope}%
\pgfpathrectangle{\pgfqpoint{0.423750in}{0.261892in}}{\pgfqpoint{2.627250in}{1.581827in}}%
\pgfusepath{clip}%
\pgfsetroundcap%
\pgfsetroundjoin%
\pgfsetlinewidth{0.803000pt}%
\definecolor{currentstroke}{rgb}{1.000000,1.000000,1.000000}%
\pgfsetstrokecolor{currentstroke}%
\pgfsetdash{}{0pt}%
\pgfpathmoveto{\pgfqpoint{0.423750in}{1.376538in}}%
\pgfpathlineto{\pgfqpoint{3.051000in}{1.376538in}}%
\pgfusepath{stroke}%
\end{pgfscope}%
\begin{pgfscope}%
\definecolor{textcolor}{rgb}{0.150000,0.150000,0.150000}%
\pgfsetstrokecolor{textcolor}%
\pgfsetfillcolor{textcolor}%
\pgftext[x=0.199212in,y=1.334329in,left,base]{\color{textcolor}\rmfamily\fontsize{8.000000}{9.600000}\selectfont \(\displaystyle 10^{3}\)}%
\end{pgfscope}%
\begin{pgfscope}%
\pgfpathrectangle{\pgfqpoint{0.423750in}{0.261892in}}{\pgfqpoint{2.627250in}{1.581827in}}%
\pgfusepath{clip}%
\pgfsetroundcap%
\pgfsetroundjoin%
\pgfsetlinewidth{0.803000pt}%
\definecolor{currentstroke}{rgb}{1.000000,1.000000,1.000000}%
\pgfsetstrokecolor{currentstroke}%
\pgfsetdash{}{0pt}%
\pgfpathmoveto{\pgfqpoint{0.423750in}{1.769984in}}%
\pgfpathlineto{\pgfqpoint{3.051000in}{1.769984in}}%
\pgfusepath{stroke}%
\end{pgfscope}%
\begin{pgfscope}%
\definecolor{textcolor}{rgb}{0.150000,0.150000,0.150000}%
\pgfsetstrokecolor{textcolor}%
\pgfsetfillcolor{textcolor}%
\pgftext[x=0.199212in,y=1.727774in,left,base]{\color{textcolor}\rmfamily\fontsize{8.000000}{9.600000}\selectfont \(\displaystyle 10^{4}\)}%
\end{pgfscope}%
\begin{pgfscope}%
\definecolor{textcolor}{rgb}{0.150000,0.150000,0.150000}%
\pgfsetstrokecolor{textcolor}%
\pgfsetfillcolor{textcolor}%
\pgftext[x=0.143657in,y=1.052805in,,bottom,rotate=90.000000]{\color{textcolor}\rmfamily\fontsize{8.000000}{9.600000}\selectfont Simple Regret}%
\end{pgfscope}%
\begin{pgfscope}%
\pgfpathrectangle{\pgfqpoint{0.423750in}{0.261892in}}{\pgfqpoint{2.627250in}{1.581827in}}%
\pgfusepath{clip}%
\pgfsetbuttcap%
\pgfsetroundjoin%
\definecolor{currentfill}{rgb}{0.121569,0.466667,0.705882}%
\pgfsetfillcolor{currentfill}%
\pgfsetfillopacity{0.200000}%
\pgfsetlinewidth{0.000000pt}%
\definecolor{currentstroke}{rgb}{0.000000,0.000000,0.000000}%
\pgfsetstrokecolor{currentstroke}%
\pgfsetdash{}{0pt}%
\pgfpathmoveto{\pgfqpoint{0.543170in}{1.446721in}}%
\pgfpathlineto{\pgfqpoint{0.543170in}{1.727448in}}%
\pgfpathlineto{\pgfqpoint{0.554076in}{1.694036in}}%
\pgfpathlineto{\pgfqpoint{0.564982in}{1.528347in}}%
\pgfpathlineto{\pgfqpoint{0.575888in}{1.496461in}}%
\pgfpathlineto{\pgfqpoint{0.586794in}{1.404556in}}%
\pgfpathlineto{\pgfqpoint{0.597700in}{1.373816in}}%
\pgfpathlineto{\pgfqpoint{0.608606in}{1.373816in}}%
\pgfpathlineto{\pgfqpoint{0.619512in}{1.373816in}}%
\pgfpathlineto{\pgfqpoint{0.630418in}{1.339499in}}%
\pgfpathlineto{\pgfqpoint{0.641324in}{1.332346in}}%
\pgfpathlineto{\pgfqpoint{0.652230in}{1.332346in}}%
\pgfpathlineto{\pgfqpoint{0.663136in}{1.313414in}}%
\pgfpathlineto{\pgfqpoint{0.674042in}{1.313414in}}%
\pgfpathlineto{\pgfqpoint{0.684948in}{1.235206in}}%
\pgfpathlineto{\pgfqpoint{0.695854in}{1.235206in}}%
\pgfpathlineto{\pgfqpoint{0.706760in}{1.171007in}}%
\pgfpathlineto{\pgfqpoint{0.717666in}{1.171007in}}%
\pgfpathlineto{\pgfqpoint{0.728572in}{1.171007in}}%
\pgfpathlineto{\pgfqpoint{0.739478in}{1.169909in}}%
\pgfpathlineto{\pgfqpoint{0.750384in}{1.169909in}}%
\pgfpathlineto{\pgfqpoint{0.761290in}{1.169909in}}%
\pgfpathlineto{\pgfqpoint{0.772196in}{1.169909in}}%
\pgfpathlineto{\pgfqpoint{0.783102in}{1.169909in}}%
\pgfpathlineto{\pgfqpoint{0.794008in}{1.162593in}}%
\pgfpathlineto{\pgfqpoint{0.804914in}{1.162593in}}%
\pgfpathlineto{\pgfqpoint{0.815820in}{1.161211in}}%
\pgfpathlineto{\pgfqpoint{0.826726in}{1.161211in}}%
\pgfpathlineto{\pgfqpoint{0.837632in}{1.161211in}}%
\pgfpathlineto{\pgfqpoint{0.848538in}{1.161211in}}%
\pgfpathlineto{\pgfqpoint{0.859444in}{1.157823in}}%
\pgfpathlineto{\pgfqpoint{0.870350in}{1.157823in}}%
\pgfpathlineto{\pgfqpoint{0.881256in}{0.992068in}}%
\pgfpathlineto{\pgfqpoint{0.892162in}{0.985066in}}%
\pgfpathlineto{\pgfqpoint{0.903068in}{0.979502in}}%
\pgfpathlineto{\pgfqpoint{0.913974in}{0.979502in}}%
\pgfpathlineto{\pgfqpoint{0.924880in}{0.979502in}}%
\pgfpathlineto{\pgfqpoint{0.935786in}{0.979502in}}%
\pgfpathlineto{\pgfqpoint{0.946692in}{0.979502in}}%
\pgfpathlineto{\pgfqpoint{0.957598in}{0.979502in}}%
\pgfpathlineto{\pgfqpoint{0.968504in}{0.979502in}}%
\pgfpathlineto{\pgfqpoint{0.979410in}{0.979502in}}%
\pgfpathlineto{\pgfqpoint{0.990316in}{0.977909in}}%
\pgfpathlineto{\pgfqpoint{1.001222in}{0.977909in}}%
\pgfpathlineto{\pgfqpoint{1.012127in}{0.977909in}}%
\pgfpathlineto{\pgfqpoint{1.023033in}{0.977909in}}%
\pgfpathlineto{\pgfqpoint{1.033939in}{0.946505in}}%
\pgfpathlineto{\pgfqpoint{1.044845in}{0.945160in}}%
\pgfpathlineto{\pgfqpoint{1.055751in}{0.945160in}}%
\pgfpathlineto{\pgfqpoint{1.066657in}{0.945160in}}%
\pgfpathlineto{\pgfqpoint{1.077563in}{0.945160in}}%
\pgfpathlineto{\pgfqpoint{1.088469in}{0.945160in}}%
\pgfpathlineto{\pgfqpoint{1.099375in}{0.945160in}}%
\pgfpathlineto{\pgfqpoint{1.110281in}{0.945160in}}%
\pgfpathlineto{\pgfqpoint{1.121187in}{0.938893in}}%
\pgfpathlineto{\pgfqpoint{1.132093in}{0.938893in}}%
\pgfpathlineto{\pgfqpoint{1.142999in}{0.938893in}}%
\pgfpathlineto{\pgfqpoint{1.153905in}{0.938893in}}%
\pgfpathlineto{\pgfqpoint{1.164811in}{0.938893in}}%
\pgfpathlineto{\pgfqpoint{1.175717in}{0.938893in}}%
\pgfpathlineto{\pgfqpoint{1.186623in}{0.938893in}}%
\pgfpathlineto{\pgfqpoint{1.197529in}{0.938893in}}%
\pgfpathlineto{\pgfqpoint{1.208435in}{0.938893in}}%
\pgfpathlineto{\pgfqpoint{1.219341in}{0.938893in}}%
\pgfpathlineto{\pgfqpoint{1.230247in}{0.922772in}}%
\pgfpathlineto{\pgfqpoint{1.241153in}{0.922772in}}%
\pgfpathlineto{\pgfqpoint{1.252059in}{0.922772in}}%
\pgfpathlineto{\pgfqpoint{1.262965in}{0.922772in}}%
\pgfpathlineto{\pgfqpoint{1.273871in}{0.920727in}}%
\pgfpathlineto{\pgfqpoint{1.284777in}{0.920727in}}%
\pgfpathlineto{\pgfqpoint{1.295683in}{0.920727in}}%
\pgfpathlineto{\pgfqpoint{1.306589in}{0.920727in}}%
\pgfpathlineto{\pgfqpoint{1.317495in}{0.920727in}}%
\pgfpathlineto{\pgfqpoint{1.328401in}{0.920727in}}%
\pgfpathlineto{\pgfqpoint{1.339307in}{0.920727in}}%
\pgfpathlineto{\pgfqpoint{1.350213in}{0.920727in}}%
\pgfpathlineto{\pgfqpoint{1.361119in}{0.920727in}}%
\pgfpathlineto{\pgfqpoint{1.372025in}{0.920727in}}%
\pgfpathlineto{\pgfqpoint{1.382931in}{0.920727in}}%
\pgfpathlineto{\pgfqpoint{1.393837in}{0.920727in}}%
\pgfpathlineto{\pgfqpoint{1.404743in}{0.920727in}}%
\pgfpathlineto{\pgfqpoint{1.415649in}{0.920727in}}%
\pgfpathlineto{\pgfqpoint{1.426555in}{0.920727in}}%
\pgfpathlineto{\pgfqpoint{1.437461in}{0.920727in}}%
\pgfpathlineto{\pgfqpoint{1.448367in}{0.920727in}}%
\pgfpathlineto{\pgfqpoint{1.459273in}{0.920727in}}%
\pgfpathlineto{\pgfqpoint{1.470179in}{0.920727in}}%
\pgfpathlineto{\pgfqpoint{1.481085in}{0.917794in}}%
\pgfpathlineto{\pgfqpoint{1.491991in}{0.917794in}}%
\pgfpathlineto{\pgfqpoint{1.502896in}{0.917794in}}%
\pgfpathlineto{\pgfqpoint{1.513802in}{0.890503in}}%
\pgfpathlineto{\pgfqpoint{1.524708in}{0.890503in}}%
\pgfpathlineto{\pgfqpoint{1.535614in}{0.890503in}}%
\pgfpathlineto{\pgfqpoint{1.546520in}{0.890503in}}%
\pgfpathlineto{\pgfqpoint{1.557426in}{0.871390in}}%
\pgfpathlineto{\pgfqpoint{1.568332in}{0.871390in}}%
\pgfpathlineto{\pgfqpoint{1.579238in}{0.871390in}}%
\pgfpathlineto{\pgfqpoint{1.590144in}{0.871390in}}%
\pgfpathlineto{\pgfqpoint{1.601050in}{0.871390in}}%
\pgfpathlineto{\pgfqpoint{1.611956in}{0.871390in}}%
\pgfpathlineto{\pgfqpoint{1.622862in}{0.871390in}}%
\pgfpathlineto{\pgfqpoint{1.633768in}{0.871390in}}%
\pgfpathlineto{\pgfqpoint{1.644674in}{0.871390in}}%
\pgfpathlineto{\pgfqpoint{1.655580in}{0.871390in}}%
\pgfpathlineto{\pgfqpoint{1.666486in}{0.871390in}}%
\pgfpathlineto{\pgfqpoint{1.677392in}{0.871390in}}%
\pgfpathlineto{\pgfqpoint{1.688298in}{0.871390in}}%
\pgfpathlineto{\pgfqpoint{1.699204in}{0.871390in}}%
\pgfpathlineto{\pgfqpoint{1.710110in}{0.871390in}}%
\pgfpathlineto{\pgfqpoint{1.721016in}{0.871390in}}%
\pgfpathlineto{\pgfqpoint{1.731922in}{0.871390in}}%
\pgfpathlineto{\pgfqpoint{1.742828in}{0.871390in}}%
\pgfpathlineto{\pgfqpoint{1.753734in}{0.871390in}}%
\pgfpathlineto{\pgfqpoint{1.764640in}{0.871390in}}%
\pgfpathlineto{\pgfqpoint{1.775546in}{0.871390in}}%
\pgfpathlineto{\pgfqpoint{1.786452in}{0.871390in}}%
\pgfpathlineto{\pgfqpoint{1.797358in}{0.871390in}}%
\pgfpathlineto{\pgfqpoint{1.808264in}{0.871390in}}%
\pgfpathlineto{\pgfqpoint{1.819170in}{0.871390in}}%
\pgfpathlineto{\pgfqpoint{1.830076in}{0.871390in}}%
\pgfpathlineto{\pgfqpoint{1.840982in}{0.871390in}}%
\pgfpathlineto{\pgfqpoint{1.851888in}{0.871390in}}%
\pgfpathlineto{\pgfqpoint{1.862794in}{0.871390in}}%
\pgfpathlineto{\pgfqpoint{1.873700in}{0.871390in}}%
\pgfpathlineto{\pgfqpoint{1.884606in}{0.871390in}}%
\pgfpathlineto{\pgfqpoint{1.895512in}{0.871390in}}%
\pgfpathlineto{\pgfqpoint{1.906418in}{0.871390in}}%
\pgfpathlineto{\pgfqpoint{1.917324in}{0.871390in}}%
\pgfpathlineto{\pgfqpoint{1.928230in}{0.871390in}}%
\pgfpathlineto{\pgfqpoint{1.939136in}{0.871390in}}%
\pgfpathlineto{\pgfqpoint{1.950042in}{0.871390in}}%
\pgfpathlineto{\pgfqpoint{1.960948in}{0.871390in}}%
\pgfpathlineto{\pgfqpoint{1.971854in}{0.871390in}}%
\pgfpathlineto{\pgfqpoint{1.982759in}{0.871390in}}%
\pgfpathlineto{\pgfqpoint{1.993665in}{0.871390in}}%
\pgfpathlineto{\pgfqpoint{2.004571in}{0.871390in}}%
\pgfpathlineto{\pgfqpoint{2.015477in}{0.871390in}}%
\pgfpathlineto{\pgfqpoint{2.026383in}{0.871390in}}%
\pgfpathlineto{\pgfqpoint{2.037289in}{0.871390in}}%
\pgfpathlineto{\pgfqpoint{2.048195in}{0.871390in}}%
\pgfpathlineto{\pgfqpoint{2.059101in}{0.871390in}}%
\pgfpathlineto{\pgfqpoint{2.070007in}{0.871390in}}%
\pgfpathlineto{\pgfqpoint{2.080913in}{0.871390in}}%
\pgfpathlineto{\pgfqpoint{2.091819in}{0.871390in}}%
\pgfpathlineto{\pgfqpoint{2.102725in}{0.871390in}}%
\pgfpathlineto{\pgfqpoint{2.113631in}{0.871390in}}%
\pgfpathlineto{\pgfqpoint{2.124537in}{0.871390in}}%
\pgfpathlineto{\pgfqpoint{2.135443in}{0.871390in}}%
\pgfpathlineto{\pgfqpoint{2.146349in}{0.871390in}}%
\pgfpathlineto{\pgfqpoint{2.157255in}{0.844950in}}%
\pgfpathlineto{\pgfqpoint{2.168161in}{0.844950in}}%
\pgfpathlineto{\pgfqpoint{2.179067in}{0.844950in}}%
\pgfpathlineto{\pgfqpoint{2.189973in}{0.844950in}}%
\pgfpathlineto{\pgfqpoint{2.200879in}{0.844950in}}%
\pgfpathlineto{\pgfqpoint{2.211785in}{0.844950in}}%
\pgfpathlineto{\pgfqpoint{2.222691in}{0.844950in}}%
\pgfpathlineto{\pgfqpoint{2.233597in}{0.841754in}}%
\pgfpathlineto{\pgfqpoint{2.244503in}{0.841754in}}%
\pgfpathlineto{\pgfqpoint{2.255409in}{0.841754in}}%
\pgfpathlineto{\pgfqpoint{2.266315in}{0.841754in}}%
\pgfpathlineto{\pgfqpoint{2.277221in}{0.841754in}}%
\pgfpathlineto{\pgfqpoint{2.288127in}{0.841754in}}%
\pgfpathlineto{\pgfqpoint{2.299033in}{0.841754in}}%
\pgfpathlineto{\pgfqpoint{2.309939in}{0.841754in}}%
\pgfpathlineto{\pgfqpoint{2.320845in}{0.841754in}}%
\pgfpathlineto{\pgfqpoint{2.331751in}{0.841754in}}%
\pgfpathlineto{\pgfqpoint{2.342657in}{0.841754in}}%
\pgfpathlineto{\pgfqpoint{2.353563in}{0.841754in}}%
\pgfpathlineto{\pgfqpoint{2.364469in}{0.841754in}}%
\pgfpathlineto{\pgfqpoint{2.375375in}{0.841754in}}%
\pgfpathlineto{\pgfqpoint{2.386281in}{0.841754in}}%
\pgfpathlineto{\pgfqpoint{2.397187in}{0.840440in}}%
\pgfpathlineto{\pgfqpoint{2.408093in}{0.840440in}}%
\pgfpathlineto{\pgfqpoint{2.418999in}{0.840440in}}%
\pgfpathlineto{\pgfqpoint{2.429905in}{0.840440in}}%
\pgfpathlineto{\pgfqpoint{2.440811in}{0.840440in}}%
\pgfpathlineto{\pgfqpoint{2.451717in}{0.840440in}}%
\pgfpathlineto{\pgfqpoint{2.462623in}{0.840440in}}%
\pgfpathlineto{\pgfqpoint{2.473528in}{0.840440in}}%
\pgfpathlineto{\pgfqpoint{2.484434in}{0.840440in}}%
\pgfpathlineto{\pgfqpoint{2.495340in}{0.840440in}}%
\pgfpathlineto{\pgfqpoint{2.506246in}{0.840440in}}%
\pgfpathlineto{\pgfqpoint{2.517152in}{0.840440in}}%
\pgfpathlineto{\pgfqpoint{2.528058in}{0.840440in}}%
\pgfpathlineto{\pgfqpoint{2.538964in}{0.840440in}}%
\pgfpathlineto{\pgfqpoint{2.549870in}{0.840440in}}%
\pgfpathlineto{\pgfqpoint{2.560776in}{0.840440in}}%
\pgfpathlineto{\pgfqpoint{2.571682in}{0.840440in}}%
\pgfpathlineto{\pgfqpoint{2.582588in}{0.840440in}}%
\pgfpathlineto{\pgfqpoint{2.593494in}{0.840440in}}%
\pgfpathlineto{\pgfqpoint{2.604400in}{0.840440in}}%
\pgfpathlineto{\pgfqpoint{2.615306in}{0.834792in}}%
\pgfpathlineto{\pgfqpoint{2.626212in}{0.834792in}}%
\pgfpathlineto{\pgfqpoint{2.637118in}{0.834792in}}%
\pgfpathlineto{\pgfqpoint{2.648024in}{0.834792in}}%
\pgfpathlineto{\pgfqpoint{2.658930in}{0.834792in}}%
\pgfpathlineto{\pgfqpoint{2.669836in}{0.834792in}}%
\pgfpathlineto{\pgfqpoint{2.680742in}{0.834792in}}%
\pgfpathlineto{\pgfqpoint{2.691648in}{0.834792in}}%
\pgfpathlineto{\pgfqpoint{2.702554in}{0.834792in}}%
\pgfpathlineto{\pgfqpoint{2.713460in}{0.834792in}}%
\pgfpathlineto{\pgfqpoint{2.724366in}{0.834792in}}%
\pgfpathlineto{\pgfqpoint{2.735272in}{0.834792in}}%
\pgfpathlineto{\pgfqpoint{2.746178in}{0.834792in}}%
\pgfpathlineto{\pgfqpoint{2.757084in}{0.834792in}}%
\pgfpathlineto{\pgfqpoint{2.767990in}{0.810027in}}%
\pgfpathlineto{\pgfqpoint{2.778896in}{0.810027in}}%
\pgfpathlineto{\pgfqpoint{2.789802in}{0.810027in}}%
\pgfpathlineto{\pgfqpoint{2.800708in}{0.810027in}}%
\pgfpathlineto{\pgfqpoint{2.811614in}{0.810027in}}%
\pgfpathlineto{\pgfqpoint{2.822520in}{0.810027in}}%
\pgfpathlineto{\pgfqpoint{2.833426in}{0.810027in}}%
\pgfpathlineto{\pgfqpoint{2.844332in}{0.810027in}}%
\pgfpathlineto{\pgfqpoint{2.855238in}{0.810027in}}%
\pgfpathlineto{\pgfqpoint{2.866144in}{0.810027in}}%
\pgfpathlineto{\pgfqpoint{2.877050in}{0.810027in}}%
\pgfpathlineto{\pgfqpoint{2.887956in}{0.810027in}}%
\pgfpathlineto{\pgfqpoint{2.898862in}{0.810027in}}%
\pgfpathlineto{\pgfqpoint{2.909768in}{0.810027in}}%
\pgfpathlineto{\pgfqpoint{2.920674in}{0.810027in}}%
\pgfpathlineto{\pgfqpoint{2.931580in}{0.810027in}}%
\pgfpathlineto{\pgfqpoint{2.931580in}{0.745856in}}%
\pgfpathlineto{\pgfqpoint{2.931580in}{0.745856in}}%
\pgfpathlineto{\pgfqpoint{2.920674in}{0.745856in}}%
\pgfpathlineto{\pgfqpoint{2.909768in}{0.745856in}}%
\pgfpathlineto{\pgfqpoint{2.898862in}{0.745856in}}%
\pgfpathlineto{\pgfqpoint{2.887956in}{0.745856in}}%
\pgfpathlineto{\pgfqpoint{2.877050in}{0.745856in}}%
\pgfpathlineto{\pgfqpoint{2.866144in}{0.745856in}}%
\pgfpathlineto{\pgfqpoint{2.855238in}{0.745856in}}%
\pgfpathlineto{\pgfqpoint{2.844332in}{0.745856in}}%
\pgfpathlineto{\pgfqpoint{2.833426in}{0.745856in}}%
\pgfpathlineto{\pgfqpoint{2.822520in}{0.745856in}}%
\pgfpathlineto{\pgfqpoint{2.811614in}{0.745856in}}%
\pgfpathlineto{\pgfqpoint{2.800708in}{0.745856in}}%
\pgfpathlineto{\pgfqpoint{2.789802in}{0.745856in}}%
\pgfpathlineto{\pgfqpoint{2.778896in}{0.745856in}}%
\pgfpathlineto{\pgfqpoint{2.767990in}{0.745856in}}%
\pgfpathlineto{\pgfqpoint{2.757084in}{0.768188in}}%
\pgfpathlineto{\pgfqpoint{2.746178in}{0.768188in}}%
\pgfpathlineto{\pgfqpoint{2.735272in}{0.768188in}}%
\pgfpathlineto{\pgfqpoint{2.724366in}{0.768188in}}%
\pgfpathlineto{\pgfqpoint{2.713460in}{0.768188in}}%
\pgfpathlineto{\pgfqpoint{2.702554in}{0.768188in}}%
\pgfpathlineto{\pgfqpoint{2.691648in}{0.768188in}}%
\pgfpathlineto{\pgfqpoint{2.680742in}{0.768188in}}%
\pgfpathlineto{\pgfqpoint{2.669836in}{0.768188in}}%
\pgfpathlineto{\pgfqpoint{2.658930in}{0.768188in}}%
\pgfpathlineto{\pgfqpoint{2.648024in}{0.768188in}}%
\pgfpathlineto{\pgfqpoint{2.637118in}{0.768188in}}%
\pgfpathlineto{\pgfqpoint{2.626212in}{0.768188in}}%
\pgfpathlineto{\pgfqpoint{2.615306in}{0.768188in}}%
\pgfpathlineto{\pgfqpoint{2.604400in}{0.772234in}}%
\pgfpathlineto{\pgfqpoint{2.593494in}{0.772234in}}%
\pgfpathlineto{\pgfqpoint{2.582588in}{0.772234in}}%
\pgfpathlineto{\pgfqpoint{2.571682in}{0.772234in}}%
\pgfpathlineto{\pgfqpoint{2.560776in}{0.772234in}}%
\pgfpathlineto{\pgfqpoint{2.549870in}{0.772234in}}%
\pgfpathlineto{\pgfqpoint{2.538964in}{0.772234in}}%
\pgfpathlineto{\pgfqpoint{2.528058in}{0.772234in}}%
\pgfpathlineto{\pgfqpoint{2.517152in}{0.772234in}}%
\pgfpathlineto{\pgfqpoint{2.506246in}{0.772234in}}%
\pgfpathlineto{\pgfqpoint{2.495340in}{0.772234in}}%
\pgfpathlineto{\pgfqpoint{2.484434in}{0.772234in}}%
\pgfpathlineto{\pgfqpoint{2.473528in}{0.772234in}}%
\pgfpathlineto{\pgfqpoint{2.462623in}{0.772234in}}%
\pgfpathlineto{\pgfqpoint{2.451717in}{0.772234in}}%
\pgfpathlineto{\pgfqpoint{2.440811in}{0.772234in}}%
\pgfpathlineto{\pgfqpoint{2.429905in}{0.772234in}}%
\pgfpathlineto{\pgfqpoint{2.418999in}{0.772234in}}%
\pgfpathlineto{\pgfqpoint{2.408093in}{0.772234in}}%
\pgfpathlineto{\pgfqpoint{2.397187in}{0.772234in}}%
\pgfpathlineto{\pgfqpoint{2.386281in}{0.774387in}}%
\pgfpathlineto{\pgfqpoint{2.375375in}{0.774387in}}%
\pgfpathlineto{\pgfqpoint{2.364469in}{0.774387in}}%
\pgfpathlineto{\pgfqpoint{2.353563in}{0.774387in}}%
\pgfpathlineto{\pgfqpoint{2.342657in}{0.774387in}}%
\pgfpathlineto{\pgfqpoint{2.331751in}{0.774387in}}%
\pgfpathlineto{\pgfqpoint{2.320845in}{0.774387in}}%
\pgfpathlineto{\pgfqpoint{2.309939in}{0.774387in}}%
\pgfpathlineto{\pgfqpoint{2.299033in}{0.774387in}}%
\pgfpathlineto{\pgfqpoint{2.288127in}{0.774387in}}%
\pgfpathlineto{\pgfqpoint{2.277221in}{0.774387in}}%
\pgfpathlineto{\pgfqpoint{2.266315in}{0.774387in}}%
\pgfpathlineto{\pgfqpoint{2.255409in}{0.774387in}}%
\pgfpathlineto{\pgfqpoint{2.244503in}{0.774387in}}%
\pgfpathlineto{\pgfqpoint{2.233597in}{0.774387in}}%
\pgfpathlineto{\pgfqpoint{2.222691in}{0.780634in}}%
\pgfpathlineto{\pgfqpoint{2.211785in}{0.780634in}}%
\pgfpathlineto{\pgfqpoint{2.200879in}{0.780634in}}%
\pgfpathlineto{\pgfqpoint{2.189973in}{0.780634in}}%
\pgfpathlineto{\pgfqpoint{2.179067in}{0.780634in}}%
\pgfpathlineto{\pgfqpoint{2.168161in}{0.780634in}}%
\pgfpathlineto{\pgfqpoint{2.157255in}{0.780634in}}%
\pgfpathlineto{\pgfqpoint{2.146349in}{0.793288in}}%
\pgfpathlineto{\pgfqpoint{2.135443in}{0.793288in}}%
\pgfpathlineto{\pgfqpoint{2.124537in}{0.793288in}}%
\pgfpathlineto{\pgfqpoint{2.113631in}{0.793288in}}%
\pgfpathlineto{\pgfqpoint{2.102725in}{0.793288in}}%
\pgfpathlineto{\pgfqpoint{2.091819in}{0.793288in}}%
\pgfpathlineto{\pgfqpoint{2.080913in}{0.793288in}}%
\pgfpathlineto{\pgfqpoint{2.070007in}{0.793288in}}%
\pgfpathlineto{\pgfqpoint{2.059101in}{0.793288in}}%
\pgfpathlineto{\pgfqpoint{2.048195in}{0.793288in}}%
\pgfpathlineto{\pgfqpoint{2.037289in}{0.793288in}}%
\pgfpathlineto{\pgfqpoint{2.026383in}{0.793288in}}%
\pgfpathlineto{\pgfqpoint{2.015477in}{0.793288in}}%
\pgfpathlineto{\pgfqpoint{2.004571in}{0.793288in}}%
\pgfpathlineto{\pgfqpoint{1.993665in}{0.793288in}}%
\pgfpathlineto{\pgfqpoint{1.982759in}{0.793288in}}%
\pgfpathlineto{\pgfqpoint{1.971854in}{0.793288in}}%
\pgfpathlineto{\pgfqpoint{1.960948in}{0.793288in}}%
\pgfpathlineto{\pgfqpoint{1.950042in}{0.793288in}}%
\pgfpathlineto{\pgfqpoint{1.939136in}{0.793288in}}%
\pgfpathlineto{\pgfqpoint{1.928230in}{0.793288in}}%
\pgfpathlineto{\pgfqpoint{1.917324in}{0.793288in}}%
\pgfpathlineto{\pgfqpoint{1.906418in}{0.793288in}}%
\pgfpathlineto{\pgfqpoint{1.895512in}{0.793288in}}%
\pgfpathlineto{\pgfqpoint{1.884606in}{0.793288in}}%
\pgfpathlineto{\pgfqpoint{1.873700in}{0.793288in}}%
\pgfpathlineto{\pgfqpoint{1.862794in}{0.793288in}}%
\pgfpathlineto{\pgfqpoint{1.851888in}{0.793288in}}%
\pgfpathlineto{\pgfqpoint{1.840982in}{0.793288in}}%
\pgfpathlineto{\pgfqpoint{1.830076in}{0.793288in}}%
\pgfpathlineto{\pgfqpoint{1.819170in}{0.793288in}}%
\pgfpathlineto{\pgfqpoint{1.808264in}{0.793288in}}%
\pgfpathlineto{\pgfqpoint{1.797358in}{0.793288in}}%
\pgfpathlineto{\pgfqpoint{1.786452in}{0.793288in}}%
\pgfpathlineto{\pgfqpoint{1.775546in}{0.793288in}}%
\pgfpathlineto{\pgfqpoint{1.764640in}{0.793288in}}%
\pgfpathlineto{\pgfqpoint{1.753734in}{0.793288in}}%
\pgfpathlineto{\pgfqpoint{1.742828in}{0.793288in}}%
\pgfpathlineto{\pgfqpoint{1.731922in}{0.793288in}}%
\pgfpathlineto{\pgfqpoint{1.721016in}{0.793288in}}%
\pgfpathlineto{\pgfqpoint{1.710110in}{0.793288in}}%
\pgfpathlineto{\pgfqpoint{1.699204in}{0.793288in}}%
\pgfpathlineto{\pgfqpoint{1.688298in}{0.793288in}}%
\pgfpathlineto{\pgfqpoint{1.677392in}{0.793288in}}%
\pgfpathlineto{\pgfqpoint{1.666486in}{0.793288in}}%
\pgfpathlineto{\pgfqpoint{1.655580in}{0.793288in}}%
\pgfpathlineto{\pgfqpoint{1.644674in}{0.793288in}}%
\pgfpathlineto{\pgfqpoint{1.633768in}{0.793288in}}%
\pgfpathlineto{\pgfqpoint{1.622862in}{0.793288in}}%
\pgfpathlineto{\pgfqpoint{1.611956in}{0.793288in}}%
\pgfpathlineto{\pgfqpoint{1.601050in}{0.793288in}}%
\pgfpathlineto{\pgfqpoint{1.590144in}{0.793288in}}%
\pgfpathlineto{\pgfqpoint{1.579238in}{0.793288in}}%
\pgfpathlineto{\pgfqpoint{1.568332in}{0.793288in}}%
\pgfpathlineto{\pgfqpoint{1.557426in}{0.793288in}}%
\pgfpathlineto{\pgfqpoint{1.546520in}{0.797358in}}%
\pgfpathlineto{\pgfqpoint{1.535614in}{0.797358in}}%
\pgfpathlineto{\pgfqpoint{1.524708in}{0.797358in}}%
\pgfpathlineto{\pgfqpoint{1.513802in}{0.797358in}}%
\pgfpathlineto{\pgfqpoint{1.502896in}{0.823554in}}%
\pgfpathlineto{\pgfqpoint{1.491991in}{0.823554in}}%
\pgfpathlineto{\pgfqpoint{1.481085in}{0.823554in}}%
\pgfpathlineto{\pgfqpoint{1.470179in}{0.824272in}}%
\pgfpathlineto{\pgfqpoint{1.459273in}{0.824272in}}%
\pgfpathlineto{\pgfqpoint{1.448367in}{0.824272in}}%
\pgfpathlineto{\pgfqpoint{1.437461in}{0.824272in}}%
\pgfpathlineto{\pgfqpoint{1.426555in}{0.824272in}}%
\pgfpathlineto{\pgfqpoint{1.415649in}{0.824272in}}%
\pgfpathlineto{\pgfqpoint{1.404743in}{0.824272in}}%
\pgfpathlineto{\pgfqpoint{1.393837in}{0.824272in}}%
\pgfpathlineto{\pgfqpoint{1.382931in}{0.824272in}}%
\pgfpathlineto{\pgfqpoint{1.372025in}{0.824272in}}%
\pgfpathlineto{\pgfqpoint{1.361119in}{0.824272in}}%
\pgfpathlineto{\pgfqpoint{1.350213in}{0.824272in}}%
\pgfpathlineto{\pgfqpoint{1.339307in}{0.824272in}}%
\pgfpathlineto{\pgfqpoint{1.328401in}{0.824272in}}%
\pgfpathlineto{\pgfqpoint{1.317495in}{0.824272in}}%
\pgfpathlineto{\pgfqpoint{1.306589in}{0.824272in}}%
\pgfpathlineto{\pgfqpoint{1.295683in}{0.824272in}}%
\pgfpathlineto{\pgfqpoint{1.284777in}{0.824272in}}%
\pgfpathlineto{\pgfqpoint{1.273871in}{0.824272in}}%
\pgfpathlineto{\pgfqpoint{1.262965in}{0.827854in}}%
\pgfpathlineto{\pgfqpoint{1.252059in}{0.827854in}}%
\pgfpathlineto{\pgfqpoint{1.241153in}{0.827854in}}%
\pgfpathlineto{\pgfqpoint{1.230247in}{0.827854in}}%
\pgfpathlineto{\pgfqpoint{1.219341in}{0.834502in}}%
\pgfpathlineto{\pgfqpoint{1.208435in}{0.834502in}}%
\pgfpathlineto{\pgfqpoint{1.197529in}{0.834502in}}%
\pgfpathlineto{\pgfqpoint{1.186623in}{0.834502in}}%
\pgfpathlineto{\pgfqpoint{1.175717in}{0.834502in}}%
\pgfpathlineto{\pgfqpoint{1.164811in}{0.834502in}}%
\pgfpathlineto{\pgfqpoint{1.153905in}{0.834502in}}%
\pgfpathlineto{\pgfqpoint{1.142999in}{0.834502in}}%
\pgfpathlineto{\pgfqpoint{1.132093in}{0.834502in}}%
\pgfpathlineto{\pgfqpoint{1.121187in}{0.834502in}}%
\pgfpathlineto{\pgfqpoint{1.110281in}{0.842969in}}%
\pgfpathlineto{\pgfqpoint{1.099375in}{0.842969in}}%
\pgfpathlineto{\pgfqpoint{1.088469in}{0.842969in}}%
\pgfpathlineto{\pgfqpoint{1.077563in}{0.842969in}}%
\pgfpathlineto{\pgfqpoint{1.066657in}{0.842969in}}%
\pgfpathlineto{\pgfqpoint{1.055751in}{0.842969in}}%
\pgfpathlineto{\pgfqpoint{1.044845in}{0.842969in}}%
\pgfpathlineto{\pgfqpoint{1.033939in}{0.847100in}}%
\pgfpathlineto{\pgfqpoint{1.023033in}{0.874345in}}%
\pgfpathlineto{\pgfqpoint{1.012127in}{0.874345in}}%
\pgfpathlineto{\pgfqpoint{1.001222in}{0.874345in}}%
\pgfpathlineto{\pgfqpoint{0.990316in}{0.874345in}}%
\pgfpathlineto{\pgfqpoint{0.979410in}{0.877885in}}%
\pgfpathlineto{\pgfqpoint{0.968504in}{0.877885in}}%
\pgfpathlineto{\pgfqpoint{0.957598in}{0.877885in}}%
\pgfpathlineto{\pgfqpoint{0.946692in}{0.877885in}}%
\pgfpathlineto{\pgfqpoint{0.935786in}{0.877885in}}%
\pgfpathlineto{\pgfqpoint{0.924880in}{0.877885in}}%
\pgfpathlineto{\pgfqpoint{0.913974in}{0.877885in}}%
\pgfpathlineto{\pgfqpoint{0.903068in}{0.877885in}}%
\pgfpathlineto{\pgfqpoint{0.892162in}{0.893062in}}%
\pgfpathlineto{\pgfqpoint{0.881256in}{0.902153in}}%
\pgfpathlineto{\pgfqpoint{0.870350in}{0.825340in}}%
\pgfpathlineto{\pgfqpoint{0.859444in}{0.825340in}}%
\pgfpathlineto{\pgfqpoint{0.848538in}{0.847172in}}%
\pgfpathlineto{\pgfqpoint{0.837632in}{0.847172in}}%
\pgfpathlineto{\pgfqpoint{0.826726in}{0.847172in}}%
\pgfpathlineto{\pgfqpoint{0.815820in}{0.847172in}}%
\pgfpathlineto{\pgfqpoint{0.804914in}{0.856929in}}%
\pgfpathlineto{\pgfqpoint{0.794008in}{0.856929in}}%
\pgfpathlineto{\pgfqpoint{0.783102in}{0.892349in}}%
\pgfpathlineto{\pgfqpoint{0.772196in}{0.892349in}}%
\pgfpathlineto{\pgfqpoint{0.761290in}{0.892349in}}%
\pgfpathlineto{\pgfqpoint{0.750384in}{0.892349in}}%
\pgfpathlineto{\pgfqpoint{0.739478in}{0.892349in}}%
\pgfpathlineto{\pgfqpoint{0.728572in}{0.900546in}}%
\pgfpathlineto{\pgfqpoint{0.717666in}{0.900546in}}%
\pgfpathlineto{\pgfqpoint{0.706760in}{0.900546in}}%
\pgfpathlineto{\pgfqpoint{0.695854in}{1.028836in}}%
\pgfpathlineto{\pgfqpoint{0.684948in}{1.028836in}}%
\pgfpathlineto{\pgfqpoint{0.674042in}{1.069705in}}%
\pgfpathlineto{\pgfqpoint{0.663136in}{1.069705in}}%
\pgfpathlineto{\pgfqpoint{0.652230in}{1.059745in}}%
\pgfpathlineto{\pgfqpoint{0.641324in}{1.059745in}}%
\pgfpathlineto{\pgfqpoint{0.630418in}{1.074689in}}%
\pgfpathlineto{\pgfqpoint{0.619512in}{1.084154in}}%
\pgfpathlineto{\pgfqpoint{0.608606in}{1.084154in}}%
\pgfpathlineto{\pgfqpoint{0.597700in}{1.084154in}}%
\pgfpathlineto{\pgfqpoint{0.586794in}{1.184219in}}%
\pgfpathlineto{\pgfqpoint{0.575888in}{1.124408in}}%
\pgfpathlineto{\pgfqpoint{0.564982in}{1.267767in}}%
\pgfpathlineto{\pgfqpoint{0.554076in}{1.017486in}}%
\pgfpathlineto{\pgfqpoint{0.543170in}{1.446721in}}%
\pgfpathclose%
\pgfusepath{fill}%
\end{pgfscope}%
\begin{pgfscope}%
\pgfpathrectangle{\pgfqpoint{0.423750in}{0.261892in}}{\pgfqpoint{2.627250in}{1.581827in}}%
\pgfusepath{clip}%
\pgfsetbuttcap%
\pgfsetroundjoin%
\definecolor{currentfill}{rgb}{1.000000,0.498039,0.054902}%
\pgfsetfillcolor{currentfill}%
\pgfsetfillopacity{0.200000}%
\pgfsetlinewidth{0.000000pt}%
\definecolor{currentstroke}{rgb}{0.000000,0.000000,0.000000}%
\pgfsetstrokecolor{currentstroke}%
\pgfsetdash{}{0pt}%
\pgfpathmoveto{\pgfqpoint{0.543170in}{1.457971in}}%
\pgfpathlineto{\pgfqpoint{0.543170in}{1.637888in}}%
\pgfpathlineto{\pgfqpoint{0.554076in}{1.349156in}}%
\pgfpathlineto{\pgfqpoint{0.564982in}{1.349156in}}%
\pgfpathlineto{\pgfqpoint{0.575888in}{1.289453in}}%
\pgfpathlineto{\pgfqpoint{0.586794in}{1.263847in}}%
\pgfpathlineto{\pgfqpoint{0.597700in}{1.261071in}}%
\pgfpathlineto{\pgfqpoint{0.608606in}{1.261071in}}%
\pgfpathlineto{\pgfqpoint{0.619512in}{1.254395in}}%
\pgfpathlineto{\pgfqpoint{0.630418in}{1.254395in}}%
\pgfpathlineto{\pgfqpoint{0.641324in}{1.254395in}}%
\pgfpathlineto{\pgfqpoint{0.652230in}{1.254395in}}%
\pgfpathlineto{\pgfqpoint{0.663136in}{1.254395in}}%
\pgfpathlineto{\pgfqpoint{0.674042in}{1.254395in}}%
\pgfpathlineto{\pgfqpoint{0.684948in}{1.254395in}}%
\pgfpathlineto{\pgfqpoint{0.695854in}{1.229350in}}%
\pgfpathlineto{\pgfqpoint{0.706760in}{1.215987in}}%
\pgfpathlineto{\pgfqpoint{0.717666in}{1.215987in}}%
\pgfpathlineto{\pgfqpoint{0.728572in}{1.215987in}}%
\pgfpathlineto{\pgfqpoint{0.739478in}{1.210284in}}%
\pgfpathlineto{\pgfqpoint{0.750384in}{1.145200in}}%
\pgfpathlineto{\pgfqpoint{0.761290in}{1.145200in}}%
\pgfpathlineto{\pgfqpoint{0.772196in}{1.145200in}}%
\pgfpathlineto{\pgfqpoint{0.783102in}{1.145200in}}%
\pgfpathlineto{\pgfqpoint{0.794008in}{1.133082in}}%
\pgfpathlineto{\pgfqpoint{0.804914in}{1.133082in}}%
\pgfpathlineto{\pgfqpoint{0.815820in}{1.133082in}}%
\pgfpathlineto{\pgfqpoint{0.826726in}{1.115878in}}%
\pgfpathlineto{\pgfqpoint{0.837632in}{1.115878in}}%
\pgfpathlineto{\pgfqpoint{0.848538in}{1.115878in}}%
\pgfpathlineto{\pgfqpoint{0.859444in}{0.991942in}}%
\pgfpathlineto{\pgfqpoint{0.870350in}{0.977905in}}%
\pgfpathlineto{\pgfqpoint{0.881256in}{0.938928in}}%
\pgfpathlineto{\pgfqpoint{0.892162in}{0.927189in}}%
\pgfpathlineto{\pgfqpoint{0.903068in}{0.927189in}}%
\pgfpathlineto{\pgfqpoint{0.913974in}{0.927189in}}%
\pgfpathlineto{\pgfqpoint{0.924880in}{0.927189in}}%
\pgfpathlineto{\pgfqpoint{0.935786in}{0.927189in}}%
\pgfpathlineto{\pgfqpoint{0.946692in}{0.927189in}}%
\pgfpathlineto{\pgfqpoint{0.957598in}{0.927189in}}%
\pgfpathlineto{\pgfqpoint{0.968504in}{0.927189in}}%
\pgfpathlineto{\pgfqpoint{0.979410in}{0.922320in}}%
\pgfpathlineto{\pgfqpoint{0.990316in}{0.921739in}}%
\pgfpathlineto{\pgfqpoint{1.001222in}{0.917700in}}%
\pgfpathlineto{\pgfqpoint{1.012127in}{0.917700in}}%
\pgfpathlineto{\pgfqpoint{1.023033in}{0.917700in}}%
\pgfpathlineto{\pgfqpoint{1.033939in}{0.905835in}}%
\pgfpathlineto{\pgfqpoint{1.044845in}{0.885485in}}%
\pgfpathlineto{\pgfqpoint{1.055751in}{0.885485in}}%
\pgfpathlineto{\pgfqpoint{1.066657in}{0.837261in}}%
\pgfpathlineto{\pgfqpoint{1.077563in}{0.837261in}}%
\pgfpathlineto{\pgfqpoint{1.088469in}{0.789090in}}%
\pgfpathlineto{\pgfqpoint{1.099375in}{0.789090in}}%
\pgfpathlineto{\pgfqpoint{1.110281in}{0.789090in}}%
\pgfpathlineto{\pgfqpoint{1.121187in}{0.739003in}}%
\pgfpathlineto{\pgfqpoint{1.132093in}{0.739003in}}%
\pgfpathlineto{\pgfqpoint{1.142999in}{0.739003in}}%
\pgfpathlineto{\pgfqpoint{1.153905in}{0.736337in}}%
\pgfpathlineto{\pgfqpoint{1.164811in}{0.736337in}}%
\pgfpathlineto{\pgfqpoint{1.175717in}{0.736337in}}%
\pgfpathlineto{\pgfqpoint{1.186623in}{0.736337in}}%
\pgfpathlineto{\pgfqpoint{1.197529in}{0.736337in}}%
\pgfpathlineto{\pgfqpoint{1.208435in}{0.715800in}}%
\pgfpathlineto{\pgfqpoint{1.219341in}{0.715800in}}%
\pgfpathlineto{\pgfqpoint{1.230247in}{0.715800in}}%
\pgfpathlineto{\pgfqpoint{1.241153in}{0.715800in}}%
\pgfpathlineto{\pgfqpoint{1.252059in}{0.691000in}}%
\pgfpathlineto{\pgfqpoint{1.262965in}{0.683406in}}%
\pgfpathlineto{\pgfqpoint{1.273871in}{0.683406in}}%
\pgfpathlineto{\pgfqpoint{1.284777in}{0.683406in}}%
\pgfpathlineto{\pgfqpoint{1.295683in}{0.678752in}}%
\pgfpathlineto{\pgfqpoint{1.306589in}{0.678517in}}%
\pgfpathlineto{\pgfqpoint{1.317495in}{0.678517in}}%
\pgfpathlineto{\pgfqpoint{1.328401in}{0.678517in}}%
\pgfpathlineto{\pgfqpoint{1.339307in}{0.678517in}}%
\pgfpathlineto{\pgfqpoint{1.350213in}{0.678517in}}%
\pgfpathlineto{\pgfqpoint{1.361119in}{0.678517in}}%
\pgfpathlineto{\pgfqpoint{1.372025in}{0.649812in}}%
\pgfpathlineto{\pgfqpoint{1.382931in}{0.649812in}}%
\pgfpathlineto{\pgfqpoint{1.393837in}{0.649812in}}%
\pgfpathlineto{\pgfqpoint{1.404743in}{0.649812in}}%
\pgfpathlineto{\pgfqpoint{1.415649in}{0.649812in}}%
\pgfpathlineto{\pgfqpoint{1.426555in}{0.649812in}}%
\pgfpathlineto{\pgfqpoint{1.437461in}{0.649812in}}%
\pgfpathlineto{\pgfqpoint{1.448367in}{0.649812in}}%
\pgfpathlineto{\pgfqpoint{1.459273in}{0.649812in}}%
\pgfpathlineto{\pgfqpoint{1.470179in}{0.649812in}}%
\pgfpathlineto{\pgfqpoint{1.481085in}{0.649812in}}%
\pgfpathlineto{\pgfqpoint{1.491991in}{0.649812in}}%
\pgfpathlineto{\pgfqpoint{1.502896in}{0.649812in}}%
\pgfpathlineto{\pgfqpoint{1.513802in}{0.649812in}}%
\pgfpathlineto{\pgfqpoint{1.524708in}{0.649812in}}%
\pgfpathlineto{\pgfqpoint{1.535614in}{0.649812in}}%
\pgfpathlineto{\pgfqpoint{1.546520in}{0.649812in}}%
\pgfpathlineto{\pgfqpoint{1.557426in}{0.649812in}}%
\pgfpathlineto{\pgfqpoint{1.568332in}{0.649812in}}%
\pgfpathlineto{\pgfqpoint{1.579238in}{0.649812in}}%
\pgfpathlineto{\pgfqpoint{1.590144in}{0.649812in}}%
\pgfpathlineto{\pgfqpoint{1.601050in}{0.649812in}}%
\pgfpathlineto{\pgfqpoint{1.611956in}{0.649812in}}%
\pgfpathlineto{\pgfqpoint{1.622862in}{0.649812in}}%
\pgfpathlineto{\pgfqpoint{1.633768in}{0.649812in}}%
\pgfpathlineto{\pgfqpoint{1.644674in}{0.649812in}}%
\pgfpathlineto{\pgfqpoint{1.655580in}{0.649812in}}%
\pgfpathlineto{\pgfqpoint{1.666486in}{0.649812in}}%
\pgfpathlineto{\pgfqpoint{1.677392in}{0.649812in}}%
\pgfpathlineto{\pgfqpoint{1.688298in}{0.647413in}}%
\pgfpathlineto{\pgfqpoint{1.699204in}{0.647413in}}%
\pgfpathlineto{\pgfqpoint{1.710110in}{0.647413in}}%
\pgfpathlineto{\pgfqpoint{1.721016in}{0.647413in}}%
\pgfpathlineto{\pgfqpoint{1.731922in}{0.647413in}}%
\pgfpathlineto{\pgfqpoint{1.742828in}{0.647413in}}%
\pgfpathlineto{\pgfqpoint{1.753734in}{0.647413in}}%
\pgfpathlineto{\pgfqpoint{1.764640in}{0.647413in}}%
\pgfpathlineto{\pgfqpoint{1.775546in}{0.647413in}}%
\pgfpathlineto{\pgfqpoint{1.786452in}{0.647413in}}%
\pgfpathlineto{\pgfqpoint{1.797358in}{0.647413in}}%
\pgfpathlineto{\pgfqpoint{1.808264in}{0.647413in}}%
\pgfpathlineto{\pgfqpoint{1.819170in}{0.647413in}}%
\pgfpathlineto{\pgfqpoint{1.830076in}{0.642129in}}%
\pgfpathlineto{\pgfqpoint{1.840982in}{0.642129in}}%
\pgfpathlineto{\pgfqpoint{1.851888in}{0.642129in}}%
\pgfpathlineto{\pgfqpoint{1.862794in}{0.642129in}}%
\pgfpathlineto{\pgfqpoint{1.873700in}{0.642129in}}%
\pgfpathlineto{\pgfqpoint{1.884606in}{0.642129in}}%
\pgfpathlineto{\pgfqpoint{1.895512in}{0.642129in}}%
\pgfpathlineto{\pgfqpoint{1.906418in}{0.640383in}}%
\pgfpathlineto{\pgfqpoint{1.917324in}{0.638974in}}%
\pgfpathlineto{\pgfqpoint{1.928230in}{0.638974in}}%
\pgfpathlineto{\pgfqpoint{1.939136in}{0.638974in}}%
\pgfpathlineto{\pgfqpoint{1.950042in}{0.638974in}}%
\pgfpathlineto{\pgfqpoint{1.960948in}{0.638974in}}%
\pgfpathlineto{\pgfqpoint{1.971854in}{0.638974in}}%
\pgfpathlineto{\pgfqpoint{1.982759in}{0.638974in}}%
\pgfpathlineto{\pgfqpoint{1.993665in}{0.638974in}}%
\pgfpathlineto{\pgfqpoint{2.004571in}{0.638974in}}%
\pgfpathlineto{\pgfqpoint{2.015477in}{0.638974in}}%
\pgfpathlineto{\pgfqpoint{2.026383in}{0.615641in}}%
\pgfpathlineto{\pgfqpoint{2.037289in}{0.615641in}}%
\pgfpathlineto{\pgfqpoint{2.048195in}{0.615641in}}%
\pgfpathlineto{\pgfqpoint{2.059101in}{0.615641in}}%
\pgfpathlineto{\pgfqpoint{2.070007in}{0.615641in}}%
\pgfpathlineto{\pgfqpoint{2.080913in}{0.615641in}}%
\pgfpathlineto{\pgfqpoint{2.091819in}{0.615641in}}%
\pgfpathlineto{\pgfqpoint{2.102725in}{0.615641in}}%
\pgfpathlineto{\pgfqpoint{2.113631in}{0.615641in}}%
\pgfpathlineto{\pgfqpoint{2.124537in}{0.615641in}}%
\pgfpathlineto{\pgfqpoint{2.135443in}{0.615641in}}%
\pgfpathlineto{\pgfqpoint{2.146349in}{0.615641in}}%
\pgfpathlineto{\pgfqpoint{2.157255in}{0.615641in}}%
\pgfpathlineto{\pgfqpoint{2.168161in}{0.615641in}}%
\pgfpathlineto{\pgfqpoint{2.179067in}{0.612145in}}%
\pgfpathlineto{\pgfqpoint{2.189973in}{0.612145in}}%
\pgfpathlineto{\pgfqpoint{2.200879in}{0.612145in}}%
\pgfpathlineto{\pgfqpoint{2.211785in}{0.612145in}}%
\pgfpathlineto{\pgfqpoint{2.222691in}{0.612145in}}%
\pgfpathlineto{\pgfqpoint{2.233597in}{0.612145in}}%
\pgfpathlineto{\pgfqpoint{2.244503in}{0.612145in}}%
\pgfpathlineto{\pgfqpoint{2.255409in}{0.612145in}}%
\pgfpathlineto{\pgfqpoint{2.266315in}{0.612145in}}%
\pgfpathlineto{\pgfqpoint{2.277221in}{0.612145in}}%
\pgfpathlineto{\pgfqpoint{2.288127in}{0.612145in}}%
\pgfpathlineto{\pgfqpoint{2.299033in}{0.608924in}}%
\pgfpathlineto{\pgfqpoint{2.309939in}{0.608924in}}%
\pgfpathlineto{\pgfqpoint{2.320845in}{0.608924in}}%
\pgfpathlineto{\pgfqpoint{2.331751in}{0.608924in}}%
\pgfpathlineto{\pgfqpoint{2.342657in}{0.608924in}}%
\pgfpathlineto{\pgfqpoint{2.353563in}{0.608924in}}%
\pgfpathlineto{\pgfqpoint{2.364469in}{0.608924in}}%
\pgfpathlineto{\pgfqpoint{2.375375in}{0.608924in}}%
\pgfpathlineto{\pgfqpoint{2.386281in}{0.608924in}}%
\pgfpathlineto{\pgfqpoint{2.397187in}{0.608924in}}%
\pgfpathlineto{\pgfqpoint{2.408093in}{0.608924in}}%
\pgfpathlineto{\pgfqpoint{2.418999in}{0.585173in}}%
\pgfpathlineto{\pgfqpoint{2.429905in}{0.585173in}}%
\pgfpathlineto{\pgfqpoint{2.440811in}{0.585173in}}%
\pgfpathlineto{\pgfqpoint{2.451717in}{0.585173in}}%
\pgfpathlineto{\pgfqpoint{2.462623in}{0.585173in}}%
\pgfpathlineto{\pgfqpoint{2.473528in}{0.585173in}}%
\pgfpathlineto{\pgfqpoint{2.484434in}{0.585173in}}%
\pgfpathlineto{\pgfqpoint{2.495340in}{0.585173in}}%
\pgfpathlineto{\pgfqpoint{2.506246in}{0.585173in}}%
\pgfpathlineto{\pgfqpoint{2.517152in}{0.585173in}}%
\pgfpathlineto{\pgfqpoint{2.528058in}{0.585173in}}%
\pgfpathlineto{\pgfqpoint{2.538964in}{0.579322in}}%
\pgfpathlineto{\pgfqpoint{2.549870in}{0.579322in}}%
\pgfpathlineto{\pgfqpoint{2.560776in}{0.579322in}}%
\pgfpathlineto{\pgfqpoint{2.571682in}{0.579322in}}%
\pgfpathlineto{\pgfqpoint{2.582588in}{0.579322in}}%
\pgfpathlineto{\pgfqpoint{2.593494in}{0.579322in}}%
\pgfpathlineto{\pgfqpoint{2.604400in}{0.568884in}}%
\pgfpathlineto{\pgfqpoint{2.615306in}{0.568884in}}%
\pgfpathlineto{\pgfqpoint{2.626212in}{0.568884in}}%
\pgfpathlineto{\pgfqpoint{2.637118in}{0.568884in}}%
\pgfpathlineto{\pgfqpoint{2.648024in}{0.568473in}}%
\pgfpathlineto{\pgfqpoint{2.658930in}{0.568473in}}%
\pgfpathlineto{\pgfqpoint{2.669836in}{0.568473in}}%
\pgfpathlineto{\pgfqpoint{2.680742in}{0.568473in}}%
\pgfpathlineto{\pgfqpoint{2.691648in}{0.568473in}}%
\pgfpathlineto{\pgfqpoint{2.702554in}{0.568473in}}%
\pgfpathlineto{\pgfqpoint{2.713460in}{0.568473in}}%
\pgfpathlineto{\pgfqpoint{2.724366in}{0.568473in}}%
\pgfpathlineto{\pgfqpoint{2.735272in}{0.568473in}}%
\pgfpathlineto{\pgfqpoint{2.746178in}{0.568473in}}%
\pgfpathlineto{\pgfqpoint{2.757084in}{0.568473in}}%
\pgfpathlineto{\pgfqpoint{2.767990in}{0.568473in}}%
\pgfpathlineto{\pgfqpoint{2.778896in}{0.568473in}}%
\pgfpathlineto{\pgfqpoint{2.789802in}{0.568473in}}%
\pgfpathlineto{\pgfqpoint{2.800708in}{0.555906in}}%
\pgfpathlineto{\pgfqpoint{2.811614in}{0.555906in}}%
\pgfpathlineto{\pgfqpoint{2.822520in}{0.547500in}}%
\pgfpathlineto{\pgfqpoint{2.833426in}{0.547500in}}%
\pgfpathlineto{\pgfqpoint{2.844332in}{0.547500in}}%
\pgfpathlineto{\pgfqpoint{2.855238in}{0.547500in}}%
\pgfpathlineto{\pgfqpoint{2.866144in}{0.547500in}}%
\pgfpathlineto{\pgfqpoint{2.877050in}{0.547500in}}%
\pgfpathlineto{\pgfqpoint{2.887956in}{0.547500in}}%
\pgfpathlineto{\pgfqpoint{2.898862in}{0.547500in}}%
\pgfpathlineto{\pgfqpoint{2.909768in}{0.547500in}}%
\pgfpathlineto{\pgfqpoint{2.920674in}{0.547500in}}%
\pgfpathlineto{\pgfqpoint{2.931580in}{0.547500in}}%
\pgfpathlineto{\pgfqpoint{2.931580in}{0.333793in}}%
\pgfpathlineto{\pgfqpoint{2.931580in}{0.333793in}}%
\pgfpathlineto{\pgfqpoint{2.920674in}{0.333793in}}%
\pgfpathlineto{\pgfqpoint{2.909768in}{0.333793in}}%
\pgfpathlineto{\pgfqpoint{2.898862in}{0.333793in}}%
\pgfpathlineto{\pgfqpoint{2.887956in}{0.333793in}}%
\pgfpathlineto{\pgfqpoint{2.877050in}{0.333793in}}%
\pgfpathlineto{\pgfqpoint{2.866144in}{0.333793in}}%
\pgfpathlineto{\pgfqpoint{2.855238in}{0.333793in}}%
\pgfpathlineto{\pgfqpoint{2.844332in}{0.333793in}}%
\pgfpathlineto{\pgfqpoint{2.833426in}{0.333793in}}%
\pgfpathlineto{\pgfqpoint{2.822520in}{0.333793in}}%
\pgfpathlineto{\pgfqpoint{2.811614in}{0.370511in}}%
\pgfpathlineto{\pgfqpoint{2.800708in}{0.370511in}}%
\pgfpathlineto{\pgfqpoint{2.789802in}{0.370598in}}%
\pgfpathlineto{\pgfqpoint{2.778896in}{0.370598in}}%
\pgfpathlineto{\pgfqpoint{2.767990in}{0.370598in}}%
\pgfpathlineto{\pgfqpoint{2.757084in}{0.370598in}}%
\pgfpathlineto{\pgfqpoint{2.746178in}{0.370598in}}%
\pgfpathlineto{\pgfqpoint{2.735272in}{0.370598in}}%
\pgfpathlineto{\pgfqpoint{2.724366in}{0.370598in}}%
\pgfpathlineto{\pgfqpoint{2.713460in}{0.370598in}}%
\pgfpathlineto{\pgfqpoint{2.702554in}{0.370598in}}%
\pgfpathlineto{\pgfqpoint{2.691648in}{0.370598in}}%
\pgfpathlineto{\pgfqpoint{2.680742in}{0.370598in}}%
\pgfpathlineto{\pgfqpoint{2.669836in}{0.370598in}}%
\pgfpathlineto{\pgfqpoint{2.658930in}{0.370598in}}%
\pgfpathlineto{\pgfqpoint{2.648024in}{0.370598in}}%
\pgfpathlineto{\pgfqpoint{2.637118in}{0.372803in}}%
\pgfpathlineto{\pgfqpoint{2.626212in}{0.372803in}}%
\pgfpathlineto{\pgfqpoint{2.615306in}{0.372803in}}%
\pgfpathlineto{\pgfqpoint{2.604400in}{0.372803in}}%
\pgfpathlineto{\pgfqpoint{2.593494in}{0.410035in}}%
\pgfpathlineto{\pgfqpoint{2.582588in}{0.410035in}}%
\pgfpathlineto{\pgfqpoint{2.571682in}{0.410035in}}%
\pgfpathlineto{\pgfqpoint{2.560776in}{0.410035in}}%
\pgfpathlineto{\pgfqpoint{2.549870in}{0.410035in}}%
\pgfpathlineto{\pgfqpoint{2.538964in}{0.410035in}}%
\pgfpathlineto{\pgfqpoint{2.528058in}{0.433761in}}%
\pgfpathlineto{\pgfqpoint{2.517152in}{0.433761in}}%
\pgfpathlineto{\pgfqpoint{2.506246in}{0.433761in}}%
\pgfpathlineto{\pgfqpoint{2.495340in}{0.433761in}}%
\pgfpathlineto{\pgfqpoint{2.484434in}{0.433761in}}%
\pgfpathlineto{\pgfqpoint{2.473528in}{0.433761in}}%
\pgfpathlineto{\pgfqpoint{2.462623in}{0.433761in}}%
\pgfpathlineto{\pgfqpoint{2.451717in}{0.433761in}}%
\pgfpathlineto{\pgfqpoint{2.440811in}{0.433761in}}%
\pgfpathlineto{\pgfqpoint{2.429905in}{0.433761in}}%
\pgfpathlineto{\pgfqpoint{2.418999in}{0.433761in}}%
\pgfpathlineto{\pgfqpoint{2.408093in}{0.484080in}}%
\pgfpathlineto{\pgfqpoint{2.397187in}{0.484080in}}%
\pgfpathlineto{\pgfqpoint{2.386281in}{0.484080in}}%
\pgfpathlineto{\pgfqpoint{2.375375in}{0.484080in}}%
\pgfpathlineto{\pgfqpoint{2.364469in}{0.484080in}}%
\pgfpathlineto{\pgfqpoint{2.353563in}{0.484080in}}%
\pgfpathlineto{\pgfqpoint{2.342657in}{0.484080in}}%
\pgfpathlineto{\pgfqpoint{2.331751in}{0.484080in}}%
\pgfpathlineto{\pgfqpoint{2.320845in}{0.484080in}}%
\pgfpathlineto{\pgfqpoint{2.309939in}{0.484080in}}%
\pgfpathlineto{\pgfqpoint{2.299033in}{0.484080in}}%
\pgfpathlineto{\pgfqpoint{2.288127in}{0.498455in}}%
\pgfpathlineto{\pgfqpoint{2.277221in}{0.498455in}}%
\pgfpathlineto{\pgfqpoint{2.266315in}{0.498455in}}%
\pgfpathlineto{\pgfqpoint{2.255409in}{0.498455in}}%
\pgfpathlineto{\pgfqpoint{2.244503in}{0.498455in}}%
\pgfpathlineto{\pgfqpoint{2.233597in}{0.498455in}}%
\pgfpathlineto{\pgfqpoint{2.222691in}{0.498455in}}%
\pgfpathlineto{\pgfqpoint{2.211785in}{0.498455in}}%
\pgfpathlineto{\pgfqpoint{2.200879in}{0.498455in}}%
\pgfpathlineto{\pgfqpoint{2.189973in}{0.498455in}}%
\pgfpathlineto{\pgfqpoint{2.179067in}{0.498455in}}%
\pgfpathlineto{\pgfqpoint{2.168161in}{0.503006in}}%
\pgfpathlineto{\pgfqpoint{2.157255in}{0.503006in}}%
\pgfpathlineto{\pgfqpoint{2.146349in}{0.503006in}}%
\pgfpathlineto{\pgfqpoint{2.135443in}{0.503006in}}%
\pgfpathlineto{\pgfqpoint{2.124537in}{0.503006in}}%
\pgfpathlineto{\pgfqpoint{2.113631in}{0.503006in}}%
\pgfpathlineto{\pgfqpoint{2.102725in}{0.503006in}}%
\pgfpathlineto{\pgfqpoint{2.091819in}{0.503006in}}%
\pgfpathlineto{\pgfqpoint{2.080913in}{0.503006in}}%
\pgfpathlineto{\pgfqpoint{2.070007in}{0.503006in}}%
\pgfpathlineto{\pgfqpoint{2.059101in}{0.503006in}}%
\pgfpathlineto{\pgfqpoint{2.048195in}{0.503006in}}%
\pgfpathlineto{\pgfqpoint{2.037289in}{0.503006in}}%
\pgfpathlineto{\pgfqpoint{2.026383in}{0.503006in}}%
\pgfpathlineto{\pgfqpoint{2.015477in}{0.521565in}}%
\pgfpathlineto{\pgfqpoint{2.004571in}{0.521565in}}%
\pgfpathlineto{\pgfqpoint{1.993665in}{0.521565in}}%
\pgfpathlineto{\pgfqpoint{1.982759in}{0.521565in}}%
\pgfpathlineto{\pgfqpoint{1.971854in}{0.521565in}}%
\pgfpathlineto{\pgfqpoint{1.960948in}{0.521565in}}%
\pgfpathlineto{\pgfqpoint{1.950042in}{0.521565in}}%
\pgfpathlineto{\pgfqpoint{1.939136in}{0.521565in}}%
\pgfpathlineto{\pgfqpoint{1.928230in}{0.521565in}}%
\pgfpathlineto{\pgfqpoint{1.917324in}{0.521565in}}%
\pgfpathlineto{\pgfqpoint{1.906418in}{0.527071in}}%
\pgfpathlineto{\pgfqpoint{1.895512in}{0.537973in}}%
\pgfpathlineto{\pgfqpoint{1.884606in}{0.537973in}}%
\pgfpathlineto{\pgfqpoint{1.873700in}{0.537973in}}%
\pgfpathlineto{\pgfqpoint{1.862794in}{0.537973in}}%
\pgfpathlineto{\pgfqpoint{1.851888in}{0.537973in}}%
\pgfpathlineto{\pgfqpoint{1.840982in}{0.537973in}}%
\pgfpathlineto{\pgfqpoint{1.830076in}{0.537973in}}%
\pgfpathlineto{\pgfqpoint{1.819170in}{0.553406in}}%
\pgfpathlineto{\pgfqpoint{1.808264in}{0.553406in}}%
\pgfpathlineto{\pgfqpoint{1.797358in}{0.553406in}}%
\pgfpathlineto{\pgfqpoint{1.786452in}{0.553406in}}%
\pgfpathlineto{\pgfqpoint{1.775546in}{0.553406in}}%
\pgfpathlineto{\pgfqpoint{1.764640in}{0.553406in}}%
\pgfpathlineto{\pgfqpoint{1.753734in}{0.553406in}}%
\pgfpathlineto{\pgfqpoint{1.742828in}{0.553406in}}%
\pgfpathlineto{\pgfqpoint{1.731922in}{0.553406in}}%
\pgfpathlineto{\pgfqpoint{1.721016in}{0.553406in}}%
\pgfpathlineto{\pgfqpoint{1.710110in}{0.553406in}}%
\pgfpathlineto{\pgfqpoint{1.699204in}{0.553406in}}%
\pgfpathlineto{\pgfqpoint{1.688298in}{0.553406in}}%
\pgfpathlineto{\pgfqpoint{1.677392in}{0.558395in}}%
\pgfpathlineto{\pgfqpoint{1.666486in}{0.558395in}}%
\pgfpathlineto{\pgfqpoint{1.655580in}{0.558395in}}%
\pgfpathlineto{\pgfqpoint{1.644674in}{0.558395in}}%
\pgfpathlineto{\pgfqpoint{1.633768in}{0.558395in}}%
\pgfpathlineto{\pgfqpoint{1.622862in}{0.558395in}}%
\pgfpathlineto{\pgfqpoint{1.611956in}{0.558395in}}%
\pgfpathlineto{\pgfqpoint{1.601050in}{0.558395in}}%
\pgfpathlineto{\pgfqpoint{1.590144in}{0.558395in}}%
\pgfpathlineto{\pgfqpoint{1.579238in}{0.558395in}}%
\pgfpathlineto{\pgfqpoint{1.568332in}{0.558395in}}%
\pgfpathlineto{\pgfqpoint{1.557426in}{0.558395in}}%
\pgfpathlineto{\pgfqpoint{1.546520in}{0.558395in}}%
\pgfpathlineto{\pgfqpoint{1.535614in}{0.558395in}}%
\pgfpathlineto{\pgfqpoint{1.524708in}{0.558395in}}%
\pgfpathlineto{\pgfqpoint{1.513802in}{0.558395in}}%
\pgfpathlineto{\pgfqpoint{1.502896in}{0.558395in}}%
\pgfpathlineto{\pgfqpoint{1.491991in}{0.558395in}}%
\pgfpathlineto{\pgfqpoint{1.481085in}{0.558395in}}%
\pgfpathlineto{\pgfqpoint{1.470179in}{0.558395in}}%
\pgfpathlineto{\pgfqpoint{1.459273in}{0.558395in}}%
\pgfpathlineto{\pgfqpoint{1.448367in}{0.558395in}}%
\pgfpathlineto{\pgfqpoint{1.437461in}{0.558395in}}%
\pgfpathlineto{\pgfqpoint{1.426555in}{0.558395in}}%
\pgfpathlineto{\pgfqpoint{1.415649in}{0.558395in}}%
\pgfpathlineto{\pgfqpoint{1.404743in}{0.558395in}}%
\pgfpathlineto{\pgfqpoint{1.393837in}{0.558395in}}%
\pgfpathlineto{\pgfqpoint{1.382931in}{0.558395in}}%
\pgfpathlineto{\pgfqpoint{1.372025in}{0.558395in}}%
\pgfpathlineto{\pgfqpoint{1.361119in}{0.604624in}}%
\pgfpathlineto{\pgfqpoint{1.350213in}{0.604624in}}%
\pgfpathlineto{\pgfqpoint{1.339307in}{0.604624in}}%
\pgfpathlineto{\pgfqpoint{1.328401in}{0.604624in}}%
\pgfpathlineto{\pgfqpoint{1.317495in}{0.604624in}}%
\pgfpathlineto{\pgfqpoint{1.306589in}{0.604624in}}%
\pgfpathlineto{\pgfqpoint{1.295683in}{0.604749in}}%
\pgfpathlineto{\pgfqpoint{1.284777in}{0.632965in}}%
\pgfpathlineto{\pgfqpoint{1.273871in}{0.632965in}}%
\pgfpathlineto{\pgfqpoint{1.262965in}{0.632965in}}%
\pgfpathlineto{\pgfqpoint{1.252059in}{0.646459in}}%
\pgfpathlineto{\pgfqpoint{1.241153in}{0.672870in}}%
\pgfpathlineto{\pgfqpoint{1.230247in}{0.672870in}}%
\pgfpathlineto{\pgfqpoint{1.219341in}{0.672870in}}%
\pgfpathlineto{\pgfqpoint{1.208435in}{0.672870in}}%
\pgfpathlineto{\pgfqpoint{1.197529in}{0.678832in}}%
\pgfpathlineto{\pgfqpoint{1.186623in}{0.678832in}}%
\pgfpathlineto{\pgfqpoint{1.175717in}{0.678832in}}%
\pgfpathlineto{\pgfqpoint{1.164811in}{0.678832in}}%
\pgfpathlineto{\pgfqpoint{1.153905in}{0.678832in}}%
\pgfpathlineto{\pgfqpoint{1.142999in}{0.682775in}}%
\pgfpathlineto{\pgfqpoint{1.132093in}{0.682775in}}%
\pgfpathlineto{\pgfqpoint{1.121187in}{0.682775in}}%
\pgfpathlineto{\pgfqpoint{1.110281in}{0.724531in}}%
\pgfpathlineto{\pgfqpoint{1.099375in}{0.724531in}}%
\pgfpathlineto{\pgfqpoint{1.088469in}{0.724531in}}%
\pgfpathlineto{\pgfqpoint{1.077563in}{0.726645in}}%
\pgfpathlineto{\pgfqpoint{1.066657in}{0.726645in}}%
\pgfpathlineto{\pgfqpoint{1.055751in}{0.754618in}}%
\pgfpathlineto{\pgfqpoint{1.044845in}{0.754618in}}%
\pgfpathlineto{\pgfqpoint{1.033939in}{0.755156in}}%
\pgfpathlineto{\pgfqpoint{1.023033in}{0.789893in}}%
\pgfpathlineto{\pgfqpoint{1.012127in}{0.789893in}}%
\pgfpathlineto{\pgfqpoint{1.001222in}{0.789893in}}%
\pgfpathlineto{\pgfqpoint{0.990316in}{0.805442in}}%
\pgfpathlineto{\pgfqpoint{0.979410in}{0.806887in}}%
\pgfpathlineto{\pgfqpoint{0.968504in}{0.819726in}}%
\pgfpathlineto{\pgfqpoint{0.957598in}{0.819726in}}%
\pgfpathlineto{\pgfqpoint{0.946692in}{0.819726in}}%
\pgfpathlineto{\pgfqpoint{0.935786in}{0.819726in}}%
\pgfpathlineto{\pgfqpoint{0.924880in}{0.819726in}}%
\pgfpathlineto{\pgfqpoint{0.913974in}{0.819726in}}%
\pgfpathlineto{\pgfqpoint{0.903068in}{0.819726in}}%
\pgfpathlineto{\pgfqpoint{0.892162in}{0.819726in}}%
\pgfpathlineto{\pgfqpoint{0.881256in}{0.838101in}}%
\pgfpathlineto{\pgfqpoint{0.870350in}{0.861033in}}%
\pgfpathlineto{\pgfqpoint{0.859444in}{0.890702in}}%
\pgfpathlineto{\pgfqpoint{0.848538in}{1.024557in}}%
\pgfpathlineto{\pgfqpoint{0.837632in}{1.024557in}}%
\pgfpathlineto{\pgfqpoint{0.826726in}{1.024557in}}%
\pgfpathlineto{\pgfqpoint{0.815820in}{1.049694in}}%
\pgfpathlineto{\pgfqpoint{0.804914in}{1.049694in}}%
\pgfpathlineto{\pgfqpoint{0.794008in}{1.049694in}}%
\pgfpathlineto{\pgfqpoint{0.783102in}{1.068278in}}%
\pgfpathlineto{\pgfqpoint{0.772196in}{1.068278in}}%
\pgfpathlineto{\pgfqpoint{0.761290in}{1.068278in}}%
\pgfpathlineto{\pgfqpoint{0.750384in}{1.068278in}}%
\pgfpathlineto{\pgfqpoint{0.739478in}{1.090493in}}%
\pgfpathlineto{\pgfqpoint{0.728572in}{1.117509in}}%
\pgfpathlineto{\pgfqpoint{0.717666in}{1.117509in}}%
\pgfpathlineto{\pgfqpoint{0.706760in}{1.117509in}}%
\pgfpathlineto{\pgfqpoint{0.695854in}{1.136692in}}%
\pgfpathlineto{\pgfqpoint{0.684948in}{1.174358in}}%
\pgfpathlineto{\pgfqpoint{0.674042in}{1.174358in}}%
\pgfpathlineto{\pgfqpoint{0.663136in}{1.174358in}}%
\pgfpathlineto{\pgfqpoint{0.652230in}{1.174358in}}%
\pgfpathlineto{\pgfqpoint{0.641324in}{1.174358in}}%
\pgfpathlineto{\pgfqpoint{0.630418in}{1.174358in}}%
\pgfpathlineto{\pgfqpoint{0.619512in}{1.174358in}}%
\pgfpathlineto{\pgfqpoint{0.608606in}{1.176292in}}%
\pgfpathlineto{\pgfqpoint{0.597700in}{1.176292in}}%
\pgfpathlineto{\pgfqpoint{0.586794in}{1.179858in}}%
\pgfpathlineto{\pgfqpoint{0.575888in}{1.184916in}}%
\pgfpathlineto{\pgfqpoint{0.564982in}{1.246528in}}%
\pgfpathlineto{\pgfqpoint{0.554076in}{1.246528in}}%
\pgfpathlineto{\pgfqpoint{0.543170in}{1.457971in}}%
\pgfpathclose%
\pgfusepath{fill}%
\end{pgfscope}%
\begin{pgfscope}%
\pgfpathrectangle{\pgfqpoint{0.423750in}{0.261892in}}{\pgfqpoint{2.627250in}{1.581827in}}%
\pgfusepath{clip}%
\pgfsetbuttcap%
\pgfsetroundjoin%
\definecolor{currentfill}{rgb}{0.172549,0.627451,0.172549}%
\pgfsetfillcolor{currentfill}%
\pgfsetfillopacity{0.200000}%
\pgfsetlinewidth{0.000000pt}%
\definecolor{currentstroke}{rgb}{0.000000,0.000000,0.000000}%
\pgfsetstrokecolor{currentstroke}%
\pgfsetdash{}{0pt}%
\pgfpathmoveto{\pgfqpoint{0.543170in}{1.486607in}}%
\pgfpathlineto{\pgfqpoint{0.543170in}{1.651891in}}%
\pgfpathlineto{\pgfqpoint{0.554076in}{1.623290in}}%
\pgfpathlineto{\pgfqpoint{0.564982in}{1.544665in}}%
\pgfpathlineto{\pgfqpoint{0.575888in}{1.544665in}}%
\pgfpathlineto{\pgfqpoint{0.586794in}{1.484823in}}%
\pgfpathlineto{\pgfqpoint{0.597700in}{1.334443in}}%
\pgfpathlineto{\pgfqpoint{0.608606in}{1.287147in}}%
\pgfpathlineto{\pgfqpoint{0.619512in}{1.233999in}}%
\pgfpathlineto{\pgfqpoint{0.630418in}{1.233999in}}%
\pgfpathlineto{\pgfqpoint{0.641324in}{1.233999in}}%
\pgfpathlineto{\pgfqpoint{0.652230in}{1.233386in}}%
\pgfpathlineto{\pgfqpoint{0.663136in}{1.233386in}}%
\pgfpathlineto{\pgfqpoint{0.674042in}{1.233386in}}%
\pgfpathlineto{\pgfqpoint{0.684948in}{1.218744in}}%
\pgfpathlineto{\pgfqpoint{0.695854in}{1.218744in}}%
\pgfpathlineto{\pgfqpoint{0.706760in}{1.218744in}}%
\pgfpathlineto{\pgfqpoint{0.717666in}{1.216387in}}%
\pgfpathlineto{\pgfqpoint{0.728572in}{1.188644in}}%
\pgfpathlineto{\pgfqpoint{0.739478in}{1.188644in}}%
\pgfpathlineto{\pgfqpoint{0.750384in}{1.188644in}}%
\pgfpathlineto{\pgfqpoint{0.761290in}{1.188644in}}%
\pgfpathlineto{\pgfqpoint{0.772196in}{1.188644in}}%
\pgfpathlineto{\pgfqpoint{0.783102in}{1.188644in}}%
\pgfpathlineto{\pgfqpoint{0.794008in}{1.188644in}}%
\pgfpathlineto{\pgfqpoint{0.804914in}{1.188644in}}%
\pgfpathlineto{\pgfqpoint{0.815820in}{1.188644in}}%
\pgfpathlineto{\pgfqpoint{0.826726in}{1.188644in}}%
\pgfpathlineto{\pgfqpoint{0.837632in}{1.109741in}}%
\pgfpathlineto{\pgfqpoint{0.848538in}{1.109741in}}%
\pgfpathlineto{\pgfqpoint{0.859444in}{1.108591in}}%
\pgfpathlineto{\pgfqpoint{0.870350in}{1.108591in}}%
\pgfpathlineto{\pgfqpoint{0.881256in}{1.108591in}}%
\pgfpathlineto{\pgfqpoint{0.892162in}{1.108591in}}%
\pgfpathlineto{\pgfqpoint{0.903068in}{1.108591in}}%
\pgfpathlineto{\pgfqpoint{0.913974in}{1.108591in}}%
\pgfpathlineto{\pgfqpoint{0.924880in}{1.099673in}}%
\pgfpathlineto{\pgfqpoint{0.935786in}{1.074403in}}%
\pgfpathlineto{\pgfqpoint{0.946692in}{1.015583in}}%
\pgfpathlineto{\pgfqpoint{0.957598in}{0.998752in}}%
\pgfpathlineto{\pgfqpoint{0.968504in}{0.998752in}}%
\pgfpathlineto{\pgfqpoint{0.979410in}{0.998752in}}%
\pgfpathlineto{\pgfqpoint{0.990316in}{0.998752in}}%
\pgfpathlineto{\pgfqpoint{1.001222in}{0.998752in}}%
\pgfpathlineto{\pgfqpoint{1.012127in}{0.973235in}}%
\pgfpathlineto{\pgfqpoint{1.023033in}{0.973235in}}%
\pgfpathlineto{\pgfqpoint{1.033939in}{0.965346in}}%
\pgfpathlineto{\pgfqpoint{1.044845in}{0.898523in}}%
\pgfpathlineto{\pgfqpoint{1.055751in}{0.898523in}}%
\pgfpathlineto{\pgfqpoint{1.066657in}{0.898523in}}%
\pgfpathlineto{\pgfqpoint{1.077563in}{0.898523in}}%
\pgfpathlineto{\pgfqpoint{1.088469in}{0.898523in}}%
\pgfpathlineto{\pgfqpoint{1.099375in}{0.898523in}}%
\pgfpathlineto{\pgfqpoint{1.110281in}{0.898523in}}%
\pgfpathlineto{\pgfqpoint{1.121187in}{0.898523in}}%
\pgfpathlineto{\pgfqpoint{1.132093in}{0.898523in}}%
\pgfpathlineto{\pgfqpoint{1.142999in}{0.898523in}}%
\pgfpathlineto{\pgfqpoint{1.153905in}{0.898523in}}%
\pgfpathlineto{\pgfqpoint{1.164811in}{0.898523in}}%
\pgfpathlineto{\pgfqpoint{1.175717in}{0.898523in}}%
\pgfpathlineto{\pgfqpoint{1.186623in}{0.898523in}}%
\pgfpathlineto{\pgfqpoint{1.197529in}{0.898523in}}%
\pgfpathlineto{\pgfqpoint{1.208435in}{0.898523in}}%
\pgfpathlineto{\pgfqpoint{1.219341in}{0.898523in}}%
\pgfpathlineto{\pgfqpoint{1.230247in}{0.898523in}}%
\pgfpathlineto{\pgfqpoint{1.241153in}{0.898523in}}%
\pgfpathlineto{\pgfqpoint{1.252059in}{0.898523in}}%
\pgfpathlineto{\pgfqpoint{1.262965in}{0.898523in}}%
\pgfpathlineto{\pgfqpoint{1.273871in}{0.898523in}}%
\pgfpathlineto{\pgfqpoint{1.284777in}{0.898523in}}%
\pgfpathlineto{\pgfqpoint{1.295683in}{0.898523in}}%
\pgfpathlineto{\pgfqpoint{1.306589in}{0.898523in}}%
\pgfpathlineto{\pgfqpoint{1.317495in}{0.898523in}}%
\pgfpathlineto{\pgfqpoint{1.328401in}{0.898523in}}%
\pgfpathlineto{\pgfqpoint{1.339307in}{0.898523in}}%
\pgfpathlineto{\pgfqpoint{1.350213in}{0.898523in}}%
\pgfpathlineto{\pgfqpoint{1.361119in}{0.898523in}}%
\pgfpathlineto{\pgfqpoint{1.372025in}{0.898523in}}%
\pgfpathlineto{\pgfqpoint{1.382931in}{0.898523in}}%
\pgfpathlineto{\pgfqpoint{1.393837in}{0.898523in}}%
\pgfpathlineto{\pgfqpoint{1.404743in}{0.898523in}}%
\pgfpathlineto{\pgfqpoint{1.415649in}{0.898523in}}%
\pgfpathlineto{\pgfqpoint{1.426555in}{0.898523in}}%
\pgfpathlineto{\pgfqpoint{1.437461in}{0.898523in}}%
\pgfpathlineto{\pgfqpoint{1.448367in}{0.898523in}}%
\pgfpathlineto{\pgfqpoint{1.459273in}{0.831925in}}%
\pgfpathlineto{\pgfqpoint{1.470179in}{0.831925in}}%
\pgfpathlineto{\pgfqpoint{1.481085in}{0.831532in}}%
\pgfpathlineto{\pgfqpoint{1.491991in}{0.831532in}}%
\pgfpathlineto{\pgfqpoint{1.502896in}{0.831532in}}%
\pgfpathlineto{\pgfqpoint{1.513802in}{0.831532in}}%
\pgfpathlineto{\pgfqpoint{1.524708in}{0.831532in}}%
\pgfpathlineto{\pgfqpoint{1.535614in}{0.831532in}}%
\pgfpathlineto{\pgfqpoint{1.546520in}{0.831532in}}%
\pgfpathlineto{\pgfqpoint{1.557426in}{0.831532in}}%
\pgfpathlineto{\pgfqpoint{1.568332in}{0.812962in}}%
\pgfpathlineto{\pgfqpoint{1.579238in}{0.812962in}}%
\pgfpathlineto{\pgfqpoint{1.590144in}{0.812962in}}%
\pgfpathlineto{\pgfqpoint{1.601050in}{0.812962in}}%
\pgfpathlineto{\pgfqpoint{1.611956in}{0.812962in}}%
\pgfpathlineto{\pgfqpoint{1.622862in}{0.812962in}}%
\pgfpathlineto{\pgfqpoint{1.633768in}{0.812962in}}%
\pgfpathlineto{\pgfqpoint{1.644674in}{0.812962in}}%
\pgfpathlineto{\pgfqpoint{1.655580in}{0.812962in}}%
\pgfpathlineto{\pgfqpoint{1.666486in}{0.812962in}}%
\pgfpathlineto{\pgfqpoint{1.677392in}{0.812962in}}%
\pgfpathlineto{\pgfqpoint{1.688298in}{0.812962in}}%
\pgfpathlineto{\pgfqpoint{1.699204in}{0.806936in}}%
\pgfpathlineto{\pgfqpoint{1.710110in}{0.806936in}}%
\pgfpathlineto{\pgfqpoint{1.721016in}{0.806936in}}%
\pgfpathlineto{\pgfqpoint{1.731922in}{0.787600in}}%
\pgfpathlineto{\pgfqpoint{1.742828in}{0.787600in}}%
\pgfpathlineto{\pgfqpoint{1.753734in}{0.787600in}}%
\pgfpathlineto{\pgfqpoint{1.764640in}{0.787600in}}%
\pgfpathlineto{\pgfqpoint{1.775546in}{0.787600in}}%
\pgfpathlineto{\pgfqpoint{1.786452in}{0.787600in}}%
\pgfpathlineto{\pgfqpoint{1.797358in}{0.787600in}}%
\pgfpathlineto{\pgfqpoint{1.808264in}{0.787600in}}%
\pgfpathlineto{\pgfqpoint{1.819170in}{0.787600in}}%
\pgfpathlineto{\pgfqpoint{1.830076in}{0.787600in}}%
\pgfpathlineto{\pgfqpoint{1.840982in}{0.783777in}}%
\pgfpathlineto{\pgfqpoint{1.851888in}{0.783777in}}%
\pgfpathlineto{\pgfqpoint{1.862794in}{0.783777in}}%
\pgfpathlineto{\pgfqpoint{1.873700in}{0.783777in}}%
\pgfpathlineto{\pgfqpoint{1.884606in}{0.770268in}}%
\pgfpathlineto{\pgfqpoint{1.895512in}{0.751768in}}%
\pgfpathlineto{\pgfqpoint{1.906418in}{0.751768in}}%
\pgfpathlineto{\pgfqpoint{1.917324in}{0.751768in}}%
\pgfpathlineto{\pgfqpoint{1.928230in}{0.685618in}}%
\pgfpathlineto{\pgfqpoint{1.939136in}{0.685618in}}%
\pgfpathlineto{\pgfqpoint{1.950042in}{0.685618in}}%
\pgfpathlineto{\pgfqpoint{1.960948in}{0.685618in}}%
\pgfpathlineto{\pgfqpoint{1.971854in}{0.685618in}}%
\pgfpathlineto{\pgfqpoint{1.982759in}{0.665699in}}%
\pgfpathlineto{\pgfqpoint{1.993665in}{0.665699in}}%
\pgfpathlineto{\pgfqpoint{2.004571in}{0.646693in}}%
\pgfpathlineto{\pgfqpoint{2.015477in}{0.646693in}}%
\pgfpathlineto{\pgfqpoint{2.026383in}{0.646693in}}%
\pgfpathlineto{\pgfqpoint{2.037289in}{0.646693in}}%
\pgfpathlineto{\pgfqpoint{2.048195in}{0.646693in}}%
\pgfpathlineto{\pgfqpoint{2.059101in}{0.646693in}}%
\pgfpathlineto{\pgfqpoint{2.070007in}{0.646693in}}%
\pgfpathlineto{\pgfqpoint{2.080913in}{0.646693in}}%
\pgfpathlineto{\pgfqpoint{2.091819in}{0.646693in}}%
\pgfpathlineto{\pgfqpoint{2.102725in}{0.646693in}}%
\pgfpathlineto{\pgfqpoint{2.113631in}{0.646693in}}%
\pgfpathlineto{\pgfqpoint{2.124537in}{0.646693in}}%
\pgfpathlineto{\pgfqpoint{2.135443in}{0.646693in}}%
\pgfpathlineto{\pgfqpoint{2.146349in}{0.646693in}}%
\pgfpathlineto{\pgfqpoint{2.157255in}{0.646693in}}%
\pgfpathlineto{\pgfqpoint{2.168161in}{0.638679in}}%
\pgfpathlineto{\pgfqpoint{2.179067in}{0.638679in}}%
\pgfpathlineto{\pgfqpoint{2.189973in}{0.638679in}}%
\pgfpathlineto{\pgfqpoint{2.200879in}{0.638679in}}%
\pgfpathlineto{\pgfqpoint{2.211785in}{0.638679in}}%
\pgfpathlineto{\pgfqpoint{2.222691in}{0.638679in}}%
\pgfpathlineto{\pgfqpoint{2.233597in}{0.638679in}}%
\pgfpathlineto{\pgfqpoint{2.244503in}{0.638679in}}%
\pgfpathlineto{\pgfqpoint{2.255409in}{0.638679in}}%
\pgfpathlineto{\pgfqpoint{2.266315in}{0.638679in}}%
\pgfpathlineto{\pgfqpoint{2.277221in}{0.638679in}}%
\pgfpathlineto{\pgfqpoint{2.288127in}{0.638679in}}%
\pgfpathlineto{\pgfqpoint{2.299033in}{0.638679in}}%
\pgfpathlineto{\pgfqpoint{2.309939in}{0.638679in}}%
\pgfpathlineto{\pgfqpoint{2.320845in}{0.638679in}}%
\pgfpathlineto{\pgfqpoint{2.331751in}{0.638656in}}%
\pgfpathlineto{\pgfqpoint{2.342657in}{0.638656in}}%
\pgfpathlineto{\pgfqpoint{2.353563in}{0.638656in}}%
\pgfpathlineto{\pgfqpoint{2.364469in}{0.638656in}}%
\pgfpathlineto{\pgfqpoint{2.375375in}{0.638656in}}%
\pgfpathlineto{\pgfqpoint{2.386281in}{0.638656in}}%
\pgfpathlineto{\pgfqpoint{2.397187in}{0.638656in}}%
\pgfpathlineto{\pgfqpoint{2.408093in}{0.630109in}}%
\pgfpathlineto{\pgfqpoint{2.418999in}{0.630109in}}%
\pgfpathlineto{\pgfqpoint{2.429905in}{0.630109in}}%
\pgfpathlineto{\pgfqpoint{2.440811in}{0.630109in}}%
\pgfpathlineto{\pgfqpoint{2.451717in}{0.630109in}}%
\pgfpathlineto{\pgfqpoint{2.462623in}{0.629228in}}%
\pgfpathlineto{\pgfqpoint{2.473528in}{0.629228in}}%
\pgfpathlineto{\pgfqpoint{2.484434in}{0.629228in}}%
\pgfpathlineto{\pgfqpoint{2.495340in}{0.629228in}}%
\pgfpathlineto{\pgfqpoint{2.506246in}{0.629228in}}%
\pgfpathlineto{\pgfqpoint{2.517152in}{0.629228in}}%
\pgfpathlineto{\pgfqpoint{2.528058in}{0.629228in}}%
\pgfpathlineto{\pgfqpoint{2.538964in}{0.629228in}}%
\pgfpathlineto{\pgfqpoint{2.549870in}{0.629228in}}%
\pgfpathlineto{\pgfqpoint{2.560776in}{0.600299in}}%
\pgfpathlineto{\pgfqpoint{2.571682in}{0.600299in}}%
\pgfpathlineto{\pgfqpoint{2.582588in}{0.600299in}}%
\pgfpathlineto{\pgfqpoint{2.593494in}{0.600299in}}%
\pgfpathlineto{\pgfqpoint{2.604400in}{0.600299in}}%
\pgfpathlineto{\pgfqpoint{2.615306in}{0.600299in}}%
\pgfpathlineto{\pgfqpoint{2.626212in}{0.600299in}}%
\pgfpathlineto{\pgfqpoint{2.637118in}{0.600299in}}%
\pgfpathlineto{\pgfqpoint{2.648024in}{0.600299in}}%
\pgfpathlineto{\pgfqpoint{2.658930in}{0.600299in}}%
\pgfpathlineto{\pgfqpoint{2.669836in}{0.600299in}}%
\pgfpathlineto{\pgfqpoint{2.680742in}{0.600299in}}%
\pgfpathlineto{\pgfqpoint{2.691648in}{0.600299in}}%
\pgfpathlineto{\pgfqpoint{2.702554in}{0.600299in}}%
\pgfpathlineto{\pgfqpoint{2.713460in}{0.600299in}}%
\pgfpathlineto{\pgfqpoint{2.724366in}{0.600299in}}%
\pgfpathlineto{\pgfqpoint{2.735272in}{0.600299in}}%
\pgfpathlineto{\pgfqpoint{2.746178in}{0.600299in}}%
\pgfpathlineto{\pgfqpoint{2.757084in}{0.600299in}}%
\pgfpathlineto{\pgfqpoint{2.767990in}{0.600299in}}%
\pgfpathlineto{\pgfqpoint{2.778896in}{0.600299in}}%
\pgfpathlineto{\pgfqpoint{2.789802in}{0.600299in}}%
\pgfpathlineto{\pgfqpoint{2.800708in}{0.600299in}}%
\pgfpathlineto{\pgfqpoint{2.811614in}{0.600299in}}%
\pgfpathlineto{\pgfqpoint{2.822520in}{0.600299in}}%
\pgfpathlineto{\pgfqpoint{2.833426in}{0.600299in}}%
\pgfpathlineto{\pgfqpoint{2.844332in}{0.600299in}}%
\pgfpathlineto{\pgfqpoint{2.855238in}{0.600299in}}%
\pgfpathlineto{\pgfqpoint{2.866144in}{0.600299in}}%
\pgfpathlineto{\pgfqpoint{2.877050in}{0.600299in}}%
\pgfpathlineto{\pgfqpoint{2.887956in}{0.600299in}}%
\pgfpathlineto{\pgfqpoint{2.898862in}{0.600299in}}%
\pgfpathlineto{\pgfqpoint{2.909768in}{0.600299in}}%
\pgfpathlineto{\pgfqpoint{2.920674in}{0.600299in}}%
\pgfpathlineto{\pgfqpoint{2.931580in}{0.600299in}}%
\pgfpathlineto{\pgfqpoint{2.931580in}{0.516118in}}%
\pgfpathlineto{\pgfqpoint{2.931580in}{0.516118in}}%
\pgfpathlineto{\pgfqpoint{2.920674in}{0.516118in}}%
\pgfpathlineto{\pgfqpoint{2.909768in}{0.516118in}}%
\pgfpathlineto{\pgfqpoint{2.898862in}{0.516118in}}%
\pgfpathlineto{\pgfqpoint{2.887956in}{0.516118in}}%
\pgfpathlineto{\pgfqpoint{2.877050in}{0.516118in}}%
\pgfpathlineto{\pgfqpoint{2.866144in}{0.516118in}}%
\pgfpathlineto{\pgfqpoint{2.855238in}{0.516118in}}%
\pgfpathlineto{\pgfqpoint{2.844332in}{0.516118in}}%
\pgfpathlineto{\pgfqpoint{2.833426in}{0.516118in}}%
\pgfpathlineto{\pgfqpoint{2.822520in}{0.516118in}}%
\pgfpathlineto{\pgfqpoint{2.811614in}{0.516118in}}%
\pgfpathlineto{\pgfqpoint{2.800708in}{0.516118in}}%
\pgfpathlineto{\pgfqpoint{2.789802in}{0.516118in}}%
\pgfpathlineto{\pgfqpoint{2.778896in}{0.516118in}}%
\pgfpathlineto{\pgfqpoint{2.767990in}{0.516118in}}%
\pgfpathlineto{\pgfqpoint{2.757084in}{0.516118in}}%
\pgfpathlineto{\pgfqpoint{2.746178in}{0.516118in}}%
\pgfpathlineto{\pgfqpoint{2.735272in}{0.516118in}}%
\pgfpathlineto{\pgfqpoint{2.724366in}{0.516118in}}%
\pgfpathlineto{\pgfqpoint{2.713460in}{0.516118in}}%
\pgfpathlineto{\pgfqpoint{2.702554in}{0.516118in}}%
\pgfpathlineto{\pgfqpoint{2.691648in}{0.516118in}}%
\pgfpathlineto{\pgfqpoint{2.680742in}{0.516118in}}%
\pgfpathlineto{\pgfqpoint{2.669836in}{0.516118in}}%
\pgfpathlineto{\pgfqpoint{2.658930in}{0.516118in}}%
\pgfpathlineto{\pgfqpoint{2.648024in}{0.516118in}}%
\pgfpathlineto{\pgfqpoint{2.637118in}{0.516118in}}%
\pgfpathlineto{\pgfqpoint{2.626212in}{0.516118in}}%
\pgfpathlineto{\pgfqpoint{2.615306in}{0.516118in}}%
\pgfpathlineto{\pgfqpoint{2.604400in}{0.516118in}}%
\pgfpathlineto{\pgfqpoint{2.593494in}{0.516118in}}%
\pgfpathlineto{\pgfqpoint{2.582588in}{0.516118in}}%
\pgfpathlineto{\pgfqpoint{2.571682in}{0.516118in}}%
\pgfpathlineto{\pgfqpoint{2.560776in}{0.516118in}}%
\pgfpathlineto{\pgfqpoint{2.549870in}{0.574227in}}%
\pgfpathlineto{\pgfqpoint{2.538964in}{0.574227in}}%
\pgfpathlineto{\pgfqpoint{2.528058in}{0.574227in}}%
\pgfpathlineto{\pgfqpoint{2.517152in}{0.574227in}}%
\pgfpathlineto{\pgfqpoint{2.506246in}{0.574227in}}%
\pgfpathlineto{\pgfqpoint{2.495340in}{0.574227in}}%
\pgfpathlineto{\pgfqpoint{2.484434in}{0.574227in}}%
\pgfpathlineto{\pgfqpoint{2.473528in}{0.574227in}}%
\pgfpathlineto{\pgfqpoint{2.462623in}{0.574227in}}%
\pgfpathlineto{\pgfqpoint{2.451717in}{0.574804in}}%
\pgfpathlineto{\pgfqpoint{2.440811in}{0.574804in}}%
\pgfpathlineto{\pgfqpoint{2.429905in}{0.574804in}}%
\pgfpathlineto{\pgfqpoint{2.418999in}{0.574804in}}%
\pgfpathlineto{\pgfqpoint{2.408093in}{0.574804in}}%
\pgfpathlineto{\pgfqpoint{2.397187in}{0.584575in}}%
\pgfpathlineto{\pgfqpoint{2.386281in}{0.584575in}}%
\pgfpathlineto{\pgfqpoint{2.375375in}{0.584575in}}%
\pgfpathlineto{\pgfqpoint{2.364469in}{0.584575in}}%
\pgfpathlineto{\pgfqpoint{2.353563in}{0.584575in}}%
\pgfpathlineto{\pgfqpoint{2.342657in}{0.584575in}}%
\pgfpathlineto{\pgfqpoint{2.331751in}{0.584575in}}%
\pgfpathlineto{\pgfqpoint{2.320845in}{0.584645in}}%
\pgfpathlineto{\pgfqpoint{2.309939in}{0.584645in}}%
\pgfpathlineto{\pgfqpoint{2.299033in}{0.584645in}}%
\pgfpathlineto{\pgfqpoint{2.288127in}{0.584645in}}%
\pgfpathlineto{\pgfqpoint{2.277221in}{0.584645in}}%
\pgfpathlineto{\pgfqpoint{2.266315in}{0.584645in}}%
\pgfpathlineto{\pgfqpoint{2.255409in}{0.584645in}}%
\pgfpathlineto{\pgfqpoint{2.244503in}{0.584645in}}%
\pgfpathlineto{\pgfqpoint{2.233597in}{0.584645in}}%
\pgfpathlineto{\pgfqpoint{2.222691in}{0.584645in}}%
\pgfpathlineto{\pgfqpoint{2.211785in}{0.584645in}}%
\pgfpathlineto{\pgfqpoint{2.200879in}{0.584645in}}%
\pgfpathlineto{\pgfqpoint{2.189973in}{0.584645in}}%
\pgfpathlineto{\pgfqpoint{2.179067in}{0.584645in}}%
\pgfpathlineto{\pgfqpoint{2.168161in}{0.584645in}}%
\pgfpathlineto{\pgfqpoint{2.157255in}{0.589383in}}%
\pgfpathlineto{\pgfqpoint{2.146349in}{0.589383in}}%
\pgfpathlineto{\pgfqpoint{2.135443in}{0.589383in}}%
\pgfpathlineto{\pgfqpoint{2.124537in}{0.589383in}}%
\pgfpathlineto{\pgfqpoint{2.113631in}{0.589383in}}%
\pgfpathlineto{\pgfqpoint{2.102725in}{0.589383in}}%
\pgfpathlineto{\pgfqpoint{2.091819in}{0.589383in}}%
\pgfpathlineto{\pgfqpoint{2.080913in}{0.589383in}}%
\pgfpathlineto{\pgfqpoint{2.070007in}{0.589383in}}%
\pgfpathlineto{\pgfqpoint{2.059101in}{0.589383in}}%
\pgfpathlineto{\pgfqpoint{2.048195in}{0.589383in}}%
\pgfpathlineto{\pgfqpoint{2.037289in}{0.589383in}}%
\pgfpathlineto{\pgfqpoint{2.026383in}{0.589383in}}%
\pgfpathlineto{\pgfqpoint{2.015477in}{0.589383in}}%
\pgfpathlineto{\pgfqpoint{2.004571in}{0.589383in}}%
\pgfpathlineto{\pgfqpoint{1.993665in}{0.599831in}}%
\pgfpathlineto{\pgfqpoint{1.982759in}{0.599831in}}%
\pgfpathlineto{\pgfqpoint{1.971854in}{0.627633in}}%
\pgfpathlineto{\pgfqpoint{1.960948in}{0.627633in}}%
\pgfpathlineto{\pgfqpoint{1.950042in}{0.627633in}}%
\pgfpathlineto{\pgfqpoint{1.939136in}{0.627633in}}%
\pgfpathlineto{\pgfqpoint{1.928230in}{0.627633in}}%
\pgfpathlineto{\pgfqpoint{1.917324in}{0.639228in}}%
\pgfpathlineto{\pgfqpoint{1.906418in}{0.639228in}}%
\pgfpathlineto{\pgfqpoint{1.895512in}{0.639228in}}%
\pgfpathlineto{\pgfqpoint{1.884606in}{0.689135in}}%
\pgfpathlineto{\pgfqpoint{1.873700in}{0.707120in}}%
\pgfpathlineto{\pgfqpoint{1.862794in}{0.707120in}}%
\pgfpathlineto{\pgfqpoint{1.851888in}{0.707120in}}%
\pgfpathlineto{\pgfqpoint{1.840982in}{0.707120in}}%
\pgfpathlineto{\pgfqpoint{1.830076in}{0.717216in}}%
\pgfpathlineto{\pgfqpoint{1.819170in}{0.717216in}}%
\pgfpathlineto{\pgfqpoint{1.808264in}{0.717216in}}%
\pgfpathlineto{\pgfqpoint{1.797358in}{0.717216in}}%
\pgfpathlineto{\pgfqpoint{1.786452in}{0.717216in}}%
\pgfpathlineto{\pgfqpoint{1.775546in}{0.717216in}}%
\pgfpathlineto{\pgfqpoint{1.764640in}{0.717216in}}%
\pgfpathlineto{\pgfqpoint{1.753734in}{0.717216in}}%
\pgfpathlineto{\pgfqpoint{1.742828in}{0.717216in}}%
\pgfpathlineto{\pgfqpoint{1.731922in}{0.717216in}}%
\pgfpathlineto{\pgfqpoint{1.721016in}{0.738017in}}%
\pgfpathlineto{\pgfqpoint{1.710110in}{0.738017in}}%
\pgfpathlineto{\pgfqpoint{1.699204in}{0.738017in}}%
\pgfpathlineto{\pgfqpoint{1.688298in}{0.743789in}}%
\pgfpathlineto{\pgfqpoint{1.677392in}{0.743789in}}%
\pgfpathlineto{\pgfqpoint{1.666486in}{0.743789in}}%
\pgfpathlineto{\pgfqpoint{1.655580in}{0.743789in}}%
\pgfpathlineto{\pgfqpoint{1.644674in}{0.743789in}}%
\pgfpathlineto{\pgfqpoint{1.633768in}{0.743789in}}%
\pgfpathlineto{\pgfqpoint{1.622862in}{0.743789in}}%
\pgfpathlineto{\pgfqpoint{1.611956in}{0.743789in}}%
\pgfpathlineto{\pgfqpoint{1.601050in}{0.743789in}}%
\pgfpathlineto{\pgfqpoint{1.590144in}{0.743789in}}%
\pgfpathlineto{\pgfqpoint{1.579238in}{0.743789in}}%
\pgfpathlineto{\pgfqpoint{1.568332in}{0.743789in}}%
\pgfpathlineto{\pgfqpoint{1.557426in}{0.781462in}}%
\pgfpathlineto{\pgfqpoint{1.546520in}{0.781462in}}%
\pgfpathlineto{\pgfqpoint{1.535614in}{0.781462in}}%
\pgfpathlineto{\pgfqpoint{1.524708in}{0.781462in}}%
\pgfpathlineto{\pgfqpoint{1.513802in}{0.781462in}}%
\pgfpathlineto{\pgfqpoint{1.502896in}{0.781462in}}%
\pgfpathlineto{\pgfqpoint{1.491991in}{0.781462in}}%
\pgfpathlineto{\pgfqpoint{1.481085in}{0.781462in}}%
\pgfpathlineto{\pgfqpoint{1.470179in}{0.789702in}}%
\pgfpathlineto{\pgfqpoint{1.459273in}{0.789702in}}%
\pgfpathlineto{\pgfqpoint{1.448367in}{0.823509in}}%
\pgfpathlineto{\pgfqpoint{1.437461in}{0.823509in}}%
\pgfpathlineto{\pgfqpoint{1.426555in}{0.823509in}}%
\pgfpathlineto{\pgfqpoint{1.415649in}{0.823509in}}%
\pgfpathlineto{\pgfqpoint{1.404743in}{0.823509in}}%
\pgfpathlineto{\pgfqpoint{1.393837in}{0.823509in}}%
\pgfpathlineto{\pgfqpoint{1.382931in}{0.823509in}}%
\pgfpathlineto{\pgfqpoint{1.372025in}{0.823509in}}%
\pgfpathlineto{\pgfqpoint{1.361119in}{0.823509in}}%
\pgfpathlineto{\pgfqpoint{1.350213in}{0.823509in}}%
\pgfpathlineto{\pgfqpoint{1.339307in}{0.823509in}}%
\pgfpathlineto{\pgfqpoint{1.328401in}{0.823509in}}%
\pgfpathlineto{\pgfqpoint{1.317495in}{0.823509in}}%
\pgfpathlineto{\pgfqpoint{1.306589in}{0.823509in}}%
\pgfpathlineto{\pgfqpoint{1.295683in}{0.823509in}}%
\pgfpathlineto{\pgfqpoint{1.284777in}{0.823509in}}%
\pgfpathlineto{\pgfqpoint{1.273871in}{0.823509in}}%
\pgfpathlineto{\pgfqpoint{1.262965in}{0.823509in}}%
\pgfpathlineto{\pgfqpoint{1.252059in}{0.823509in}}%
\pgfpathlineto{\pgfqpoint{1.241153in}{0.823509in}}%
\pgfpathlineto{\pgfqpoint{1.230247in}{0.823509in}}%
\pgfpathlineto{\pgfqpoint{1.219341in}{0.823509in}}%
\pgfpathlineto{\pgfqpoint{1.208435in}{0.823509in}}%
\pgfpathlineto{\pgfqpoint{1.197529in}{0.823509in}}%
\pgfpathlineto{\pgfqpoint{1.186623in}{0.823509in}}%
\pgfpathlineto{\pgfqpoint{1.175717in}{0.823509in}}%
\pgfpathlineto{\pgfqpoint{1.164811in}{0.823509in}}%
\pgfpathlineto{\pgfqpoint{1.153905in}{0.823509in}}%
\pgfpathlineto{\pgfqpoint{1.142999in}{0.823509in}}%
\pgfpathlineto{\pgfqpoint{1.132093in}{0.823509in}}%
\pgfpathlineto{\pgfqpoint{1.121187in}{0.823509in}}%
\pgfpathlineto{\pgfqpoint{1.110281in}{0.823509in}}%
\pgfpathlineto{\pgfqpoint{1.099375in}{0.823509in}}%
\pgfpathlineto{\pgfqpoint{1.088469in}{0.823509in}}%
\pgfpathlineto{\pgfqpoint{1.077563in}{0.823509in}}%
\pgfpathlineto{\pgfqpoint{1.066657in}{0.823509in}}%
\pgfpathlineto{\pgfqpoint{1.055751in}{0.823509in}}%
\pgfpathlineto{\pgfqpoint{1.044845in}{0.823509in}}%
\pgfpathlineto{\pgfqpoint{1.033939in}{0.856711in}}%
\pgfpathlineto{\pgfqpoint{1.023033in}{0.877087in}}%
\pgfpathlineto{\pgfqpoint{1.012127in}{0.877087in}}%
\pgfpathlineto{\pgfqpoint{1.001222in}{0.913337in}}%
\pgfpathlineto{\pgfqpoint{0.990316in}{0.913337in}}%
\pgfpathlineto{\pgfqpoint{0.979410in}{0.913337in}}%
\pgfpathlineto{\pgfqpoint{0.968504in}{0.913337in}}%
\pgfpathlineto{\pgfqpoint{0.957598in}{0.913337in}}%
\pgfpathlineto{\pgfqpoint{0.946692in}{0.921305in}}%
\pgfpathlineto{\pgfqpoint{0.935786in}{0.965385in}}%
\pgfpathlineto{\pgfqpoint{0.924880in}{0.983896in}}%
\pgfpathlineto{\pgfqpoint{0.913974in}{0.997041in}}%
\pgfpathlineto{\pgfqpoint{0.903068in}{0.997041in}}%
\pgfpathlineto{\pgfqpoint{0.892162in}{0.997041in}}%
\pgfpathlineto{\pgfqpoint{0.881256in}{0.997041in}}%
\pgfpathlineto{\pgfqpoint{0.870350in}{0.997041in}}%
\pgfpathlineto{\pgfqpoint{0.859444in}{0.997041in}}%
\pgfpathlineto{\pgfqpoint{0.848538in}{1.004949in}}%
\pgfpathlineto{\pgfqpoint{0.837632in}{1.004949in}}%
\pgfpathlineto{\pgfqpoint{0.826726in}{1.037314in}}%
\pgfpathlineto{\pgfqpoint{0.815820in}{1.037314in}}%
\pgfpathlineto{\pgfqpoint{0.804914in}{1.037314in}}%
\pgfpathlineto{\pgfqpoint{0.794008in}{1.037314in}}%
\pgfpathlineto{\pgfqpoint{0.783102in}{1.037314in}}%
\pgfpathlineto{\pgfqpoint{0.772196in}{1.037314in}}%
\pgfpathlineto{\pgfqpoint{0.761290in}{1.037314in}}%
\pgfpathlineto{\pgfqpoint{0.750384in}{1.037314in}}%
\pgfpathlineto{\pgfqpoint{0.739478in}{1.037314in}}%
\pgfpathlineto{\pgfqpoint{0.728572in}{1.037314in}}%
\pgfpathlineto{\pgfqpoint{0.717666in}{1.104649in}}%
\pgfpathlineto{\pgfqpoint{0.706760in}{1.114412in}}%
\pgfpathlineto{\pgfqpoint{0.695854in}{1.114412in}}%
\pgfpathlineto{\pgfqpoint{0.684948in}{1.114412in}}%
\pgfpathlineto{\pgfqpoint{0.674042in}{1.117042in}}%
\pgfpathlineto{\pgfqpoint{0.663136in}{1.117042in}}%
\pgfpathlineto{\pgfqpoint{0.652230in}{1.117042in}}%
\pgfpathlineto{\pgfqpoint{0.641324in}{1.118390in}}%
\pgfpathlineto{\pgfqpoint{0.630418in}{1.118390in}}%
\pgfpathlineto{\pgfqpoint{0.619512in}{1.118390in}}%
\pgfpathlineto{\pgfqpoint{0.608606in}{1.185742in}}%
\pgfpathlineto{\pgfqpoint{0.597700in}{1.232284in}}%
\pgfpathlineto{\pgfqpoint{0.586794in}{1.291553in}}%
\pgfpathlineto{\pgfqpoint{0.575888in}{1.404531in}}%
\pgfpathlineto{\pgfqpoint{0.564982in}{1.404531in}}%
\pgfpathlineto{\pgfqpoint{0.554076in}{1.453945in}}%
\pgfpathlineto{\pgfqpoint{0.543170in}{1.486607in}}%
\pgfpathclose%
\pgfusepath{fill}%
\end{pgfscope}%
\begin{pgfscope}%
\pgfpathrectangle{\pgfqpoint{0.423750in}{0.261892in}}{\pgfqpoint{2.627250in}{1.581827in}}%
\pgfusepath{clip}%
\pgfsetbuttcap%
\pgfsetroundjoin%
\definecolor{currentfill}{rgb}{0.839216,0.152941,0.156863}%
\pgfsetfillcolor{currentfill}%
\pgfsetfillopacity{0.200000}%
\pgfsetlinewidth{0.000000pt}%
\definecolor{currentstroke}{rgb}{0.000000,0.000000,0.000000}%
\pgfsetstrokecolor{currentstroke}%
\pgfsetdash{}{0pt}%
\pgfpathmoveto{\pgfqpoint{0.543170in}{1.516591in}}%
\pgfpathlineto{\pgfqpoint{0.543170in}{1.771818in}}%
\pgfpathlineto{\pgfqpoint{0.554076in}{1.500388in}}%
\pgfpathlineto{\pgfqpoint{0.564982in}{1.450062in}}%
\pgfpathlineto{\pgfqpoint{0.575888in}{1.385361in}}%
\pgfpathlineto{\pgfqpoint{0.586794in}{1.384679in}}%
\pgfpathlineto{\pgfqpoint{0.597700in}{1.384679in}}%
\pgfpathlineto{\pgfqpoint{0.608606in}{1.304873in}}%
\pgfpathlineto{\pgfqpoint{0.619512in}{1.249875in}}%
\pgfpathlineto{\pgfqpoint{0.630418in}{1.249875in}}%
\pgfpathlineto{\pgfqpoint{0.641324in}{1.236491in}}%
\pgfpathlineto{\pgfqpoint{0.652230in}{1.236491in}}%
\pgfpathlineto{\pgfqpoint{0.663136in}{1.218366in}}%
\pgfpathlineto{\pgfqpoint{0.674042in}{1.176938in}}%
\pgfpathlineto{\pgfqpoint{0.684948in}{1.176938in}}%
\pgfpathlineto{\pgfqpoint{0.695854in}{1.175783in}}%
\pgfpathlineto{\pgfqpoint{0.706760in}{1.175783in}}%
\pgfpathlineto{\pgfqpoint{0.717666in}{1.156774in}}%
\pgfpathlineto{\pgfqpoint{0.728572in}{1.156774in}}%
\pgfpathlineto{\pgfqpoint{0.739478in}{1.155604in}}%
\pgfpathlineto{\pgfqpoint{0.750384in}{1.155604in}}%
\pgfpathlineto{\pgfqpoint{0.761290in}{1.155604in}}%
\pgfpathlineto{\pgfqpoint{0.772196in}{1.088533in}}%
\pgfpathlineto{\pgfqpoint{0.783102in}{1.050086in}}%
\pgfpathlineto{\pgfqpoint{0.794008in}{1.047809in}}%
\pgfpathlineto{\pgfqpoint{0.804914in}{1.047809in}}%
\pgfpathlineto{\pgfqpoint{0.815820in}{1.047809in}}%
\pgfpathlineto{\pgfqpoint{0.826726in}{1.047809in}}%
\pgfpathlineto{\pgfqpoint{0.837632in}{1.047809in}}%
\pgfpathlineto{\pgfqpoint{0.848538in}{1.047809in}}%
\pgfpathlineto{\pgfqpoint{0.859444in}{1.047809in}}%
\pgfpathlineto{\pgfqpoint{0.870350in}{1.047809in}}%
\pgfpathlineto{\pgfqpoint{0.881256in}{1.032214in}}%
\pgfpathlineto{\pgfqpoint{0.892162in}{1.032214in}}%
\pgfpathlineto{\pgfqpoint{0.903068in}{1.032214in}}%
\pgfpathlineto{\pgfqpoint{0.913974in}{1.018264in}}%
\pgfpathlineto{\pgfqpoint{0.924880in}{1.018264in}}%
\pgfpathlineto{\pgfqpoint{0.935786in}{1.002465in}}%
\pgfpathlineto{\pgfqpoint{0.946692in}{1.002465in}}%
\pgfpathlineto{\pgfqpoint{0.957598in}{1.002465in}}%
\pgfpathlineto{\pgfqpoint{0.968504in}{1.002465in}}%
\pgfpathlineto{\pgfqpoint{0.979410in}{0.996889in}}%
\pgfpathlineto{\pgfqpoint{0.990316in}{0.981937in}}%
\pgfpathlineto{\pgfqpoint{1.001222in}{0.981937in}}%
\pgfpathlineto{\pgfqpoint{1.012127in}{0.981937in}}%
\pgfpathlineto{\pgfqpoint{1.023033in}{0.981937in}}%
\pgfpathlineto{\pgfqpoint{1.033939in}{0.957777in}}%
\pgfpathlineto{\pgfqpoint{1.044845in}{0.957777in}}%
\pgfpathlineto{\pgfqpoint{1.055751in}{0.957777in}}%
\pgfpathlineto{\pgfqpoint{1.066657in}{0.957777in}}%
\pgfpathlineto{\pgfqpoint{1.077563in}{0.957777in}}%
\pgfpathlineto{\pgfqpoint{1.088469in}{0.875665in}}%
\pgfpathlineto{\pgfqpoint{1.099375in}{0.875665in}}%
\pgfpathlineto{\pgfqpoint{1.110281in}{0.845010in}}%
\pgfpathlineto{\pgfqpoint{1.121187in}{0.836232in}}%
\pgfpathlineto{\pgfqpoint{1.132093in}{0.836232in}}%
\pgfpathlineto{\pgfqpoint{1.142999in}{0.836232in}}%
\pgfpathlineto{\pgfqpoint{1.153905in}{0.836232in}}%
\pgfpathlineto{\pgfqpoint{1.164811in}{0.836232in}}%
\pgfpathlineto{\pgfqpoint{1.175717in}{0.836232in}}%
\pgfpathlineto{\pgfqpoint{1.186623in}{0.836232in}}%
\pgfpathlineto{\pgfqpoint{1.197529in}{0.769667in}}%
\pgfpathlineto{\pgfqpoint{1.208435in}{0.769667in}}%
\pgfpathlineto{\pgfqpoint{1.219341in}{0.769667in}}%
\pgfpathlineto{\pgfqpoint{1.230247in}{0.769667in}}%
\pgfpathlineto{\pgfqpoint{1.241153in}{0.769667in}}%
\pgfpathlineto{\pgfqpoint{1.252059in}{0.769667in}}%
\pgfpathlineto{\pgfqpoint{1.262965in}{0.735273in}}%
\pgfpathlineto{\pgfqpoint{1.273871in}{0.702165in}}%
\pgfpathlineto{\pgfqpoint{1.284777in}{0.679908in}}%
\pgfpathlineto{\pgfqpoint{1.295683in}{0.679908in}}%
\pgfpathlineto{\pgfqpoint{1.306589in}{0.679908in}}%
\pgfpathlineto{\pgfqpoint{1.317495in}{0.679908in}}%
\pgfpathlineto{\pgfqpoint{1.328401in}{0.679908in}}%
\pgfpathlineto{\pgfqpoint{1.339307in}{0.678407in}}%
\pgfpathlineto{\pgfqpoint{1.350213in}{0.678407in}}%
\pgfpathlineto{\pgfqpoint{1.361119in}{0.678407in}}%
\pgfpathlineto{\pgfqpoint{1.372025in}{0.652238in}}%
\pgfpathlineto{\pgfqpoint{1.382931in}{0.652238in}}%
\pgfpathlineto{\pgfqpoint{1.393837in}{0.651899in}}%
\pgfpathlineto{\pgfqpoint{1.404743in}{0.651899in}}%
\pgfpathlineto{\pgfqpoint{1.415649in}{0.651899in}}%
\pgfpathlineto{\pgfqpoint{1.426555in}{0.651899in}}%
\pgfpathlineto{\pgfqpoint{1.437461in}{0.641143in}}%
\pgfpathlineto{\pgfqpoint{1.448367in}{0.641143in}}%
\pgfpathlineto{\pgfqpoint{1.459273in}{0.641143in}}%
\pgfpathlineto{\pgfqpoint{1.470179in}{0.638265in}}%
\pgfpathlineto{\pgfqpoint{1.481085in}{0.627229in}}%
\pgfpathlineto{\pgfqpoint{1.491991in}{0.627229in}}%
\pgfpathlineto{\pgfqpoint{1.502896in}{0.627229in}}%
\pgfpathlineto{\pgfqpoint{1.513802in}{0.627229in}}%
\pgfpathlineto{\pgfqpoint{1.524708in}{0.627229in}}%
\pgfpathlineto{\pgfqpoint{1.535614in}{0.627229in}}%
\pgfpathlineto{\pgfqpoint{1.546520in}{0.627229in}}%
\pgfpathlineto{\pgfqpoint{1.557426in}{0.627229in}}%
\pgfpathlineto{\pgfqpoint{1.568332in}{0.627229in}}%
\pgfpathlineto{\pgfqpoint{1.579238in}{0.627229in}}%
\pgfpathlineto{\pgfqpoint{1.590144in}{0.627229in}}%
\pgfpathlineto{\pgfqpoint{1.601050in}{0.627229in}}%
\pgfpathlineto{\pgfqpoint{1.611956in}{0.627229in}}%
\pgfpathlineto{\pgfqpoint{1.622862in}{0.627229in}}%
\pgfpathlineto{\pgfqpoint{1.633768in}{0.627229in}}%
\pgfpathlineto{\pgfqpoint{1.644674in}{0.627229in}}%
\pgfpathlineto{\pgfqpoint{1.655580in}{0.627229in}}%
\pgfpathlineto{\pgfqpoint{1.666486in}{0.627229in}}%
\pgfpathlineto{\pgfqpoint{1.677392in}{0.627229in}}%
\pgfpathlineto{\pgfqpoint{1.688298in}{0.624656in}}%
\pgfpathlineto{\pgfqpoint{1.699204in}{0.624656in}}%
\pgfpathlineto{\pgfqpoint{1.710110in}{0.624656in}}%
\pgfpathlineto{\pgfqpoint{1.721016in}{0.624656in}}%
\pgfpathlineto{\pgfqpoint{1.731922in}{0.624656in}}%
\pgfpathlineto{\pgfqpoint{1.742828in}{0.624656in}}%
\pgfpathlineto{\pgfqpoint{1.753734in}{0.624656in}}%
\pgfpathlineto{\pgfqpoint{1.764640in}{0.624656in}}%
\pgfpathlineto{\pgfqpoint{1.775546in}{0.624656in}}%
\pgfpathlineto{\pgfqpoint{1.786452in}{0.624656in}}%
\pgfpathlineto{\pgfqpoint{1.797358in}{0.624656in}}%
\pgfpathlineto{\pgfqpoint{1.808264in}{0.624656in}}%
\pgfpathlineto{\pgfqpoint{1.819170in}{0.624656in}}%
\pgfpathlineto{\pgfqpoint{1.830076in}{0.624656in}}%
\pgfpathlineto{\pgfqpoint{1.840982in}{0.624656in}}%
\pgfpathlineto{\pgfqpoint{1.851888in}{0.624656in}}%
\pgfpathlineto{\pgfqpoint{1.862794in}{0.624656in}}%
\pgfpathlineto{\pgfqpoint{1.873700in}{0.624656in}}%
\pgfpathlineto{\pgfqpoint{1.884606in}{0.624656in}}%
\pgfpathlineto{\pgfqpoint{1.895512in}{0.624656in}}%
\pgfpathlineto{\pgfqpoint{1.906418in}{0.624656in}}%
\pgfpathlineto{\pgfqpoint{1.917324in}{0.624656in}}%
\pgfpathlineto{\pgfqpoint{1.928230in}{0.624656in}}%
\pgfpathlineto{\pgfqpoint{1.939136in}{0.624656in}}%
\pgfpathlineto{\pgfqpoint{1.950042in}{0.624656in}}%
\pgfpathlineto{\pgfqpoint{1.960948in}{0.624656in}}%
\pgfpathlineto{\pgfqpoint{1.971854in}{0.624656in}}%
\pgfpathlineto{\pgfqpoint{1.982759in}{0.624656in}}%
\pgfpathlineto{\pgfqpoint{1.993665in}{0.624656in}}%
\pgfpathlineto{\pgfqpoint{2.004571in}{0.624656in}}%
\pgfpathlineto{\pgfqpoint{2.015477in}{0.624656in}}%
\pgfpathlineto{\pgfqpoint{2.026383in}{0.624656in}}%
\pgfpathlineto{\pgfqpoint{2.037289in}{0.624656in}}%
\pgfpathlineto{\pgfqpoint{2.048195in}{0.624656in}}%
\pgfpathlineto{\pgfqpoint{2.059101in}{0.624656in}}%
\pgfpathlineto{\pgfqpoint{2.070007in}{0.624656in}}%
\pgfpathlineto{\pgfqpoint{2.080913in}{0.624656in}}%
\pgfpathlineto{\pgfqpoint{2.091819in}{0.624656in}}%
\pgfpathlineto{\pgfqpoint{2.102725in}{0.624656in}}%
\pgfpathlineto{\pgfqpoint{2.113631in}{0.624656in}}%
\pgfpathlineto{\pgfqpoint{2.124537in}{0.624656in}}%
\pgfpathlineto{\pgfqpoint{2.135443in}{0.624656in}}%
\pgfpathlineto{\pgfqpoint{2.146349in}{0.624656in}}%
\pgfpathlineto{\pgfqpoint{2.157255in}{0.624656in}}%
\pgfpathlineto{\pgfqpoint{2.168161in}{0.624656in}}%
\pgfpathlineto{\pgfqpoint{2.179067in}{0.624656in}}%
\pgfpathlineto{\pgfqpoint{2.189973in}{0.624656in}}%
\pgfpathlineto{\pgfqpoint{2.200879in}{0.624656in}}%
\pgfpathlineto{\pgfqpoint{2.211785in}{0.624656in}}%
\pgfpathlineto{\pgfqpoint{2.222691in}{0.624656in}}%
\pgfpathlineto{\pgfqpoint{2.233597in}{0.624656in}}%
\pgfpathlineto{\pgfqpoint{2.244503in}{0.624656in}}%
\pgfpathlineto{\pgfqpoint{2.255409in}{0.624656in}}%
\pgfpathlineto{\pgfqpoint{2.266315in}{0.624656in}}%
\pgfpathlineto{\pgfqpoint{2.277221in}{0.624656in}}%
\pgfpathlineto{\pgfqpoint{2.288127in}{0.624656in}}%
\pgfpathlineto{\pgfqpoint{2.299033in}{0.624656in}}%
\pgfpathlineto{\pgfqpoint{2.309939in}{0.624656in}}%
\pgfpathlineto{\pgfqpoint{2.320845in}{0.624656in}}%
\pgfpathlineto{\pgfqpoint{2.331751in}{0.624656in}}%
\pgfpathlineto{\pgfqpoint{2.342657in}{0.624656in}}%
\pgfpathlineto{\pgfqpoint{2.353563in}{0.622538in}}%
\pgfpathlineto{\pgfqpoint{2.364469in}{0.622538in}}%
\pgfpathlineto{\pgfqpoint{2.375375in}{0.622538in}}%
\pgfpathlineto{\pgfqpoint{2.386281in}{0.622538in}}%
\pgfpathlineto{\pgfqpoint{2.397187in}{0.622538in}}%
\pgfpathlineto{\pgfqpoint{2.408093in}{0.622538in}}%
\pgfpathlineto{\pgfqpoint{2.418999in}{0.622538in}}%
\pgfpathlineto{\pgfqpoint{2.429905in}{0.622538in}}%
\pgfpathlineto{\pgfqpoint{2.440811in}{0.622538in}}%
\pgfpathlineto{\pgfqpoint{2.451717in}{0.622538in}}%
\pgfpathlineto{\pgfqpoint{2.462623in}{0.622538in}}%
\pgfpathlineto{\pgfqpoint{2.473528in}{0.622538in}}%
\pgfpathlineto{\pgfqpoint{2.484434in}{0.622538in}}%
\pgfpathlineto{\pgfqpoint{2.495340in}{0.551834in}}%
\pgfpathlineto{\pgfqpoint{2.506246in}{0.551834in}}%
\pgfpathlineto{\pgfqpoint{2.517152in}{0.551834in}}%
\pgfpathlineto{\pgfqpoint{2.528058in}{0.551834in}}%
\pgfpathlineto{\pgfqpoint{2.538964in}{0.551834in}}%
\pgfpathlineto{\pgfqpoint{2.549870in}{0.551834in}}%
\pgfpathlineto{\pgfqpoint{2.560776in}{0.551834in}}%
\pgfpathlineto{\pgfqpoint{2.571682in}{0.551834in}}%
\pgfpathlineto{\pgfqpoint{2.582588in}{0.551834in}}%
\pgfpathlineto{\pgfqpoint{2.593494in}{0.551834in}}%
\pgfpathlineto{\pgfqpoint{2.604400in}{0.551834in}}%
\pgfpathlineto{\pgfqpoint{2.615306in}{0.551834in}}%
\pgfpathlineto{\pgfqpoint{2.626212in}{0.551834in}}%
\pgfpathlineto{\pgfqpoint{2.637118in}{0.551834in}}%
\pgfpathlineto{\pgfqpoint{2.648024in}{0.551834in}}%
\pgfpathlineto{\pgfqpoint{2.658930in}{0.551834in}}%
\pgfpathlineto{\pgfqpoint{2.669836in}{0.551834in}}%
\pgfpathlineto{\pgfqpoint{2.680742in}{0.551834in}}%
\pgfpathlineto{\pgfqpoint{2.691648in}{0.551834in}}%
\pgfpathlineto{\pgfqpoint{2.702554in}{0.551834in}}%
\pgfpathlineto{\pgfqpoint{2.713460in}{0.551834in}}%
\pgfpathlineto{\pgfqpoint{2.724366in}{0.551834in}}%
\pgfpathlineto{\pgfqpoint{2.735272in}{0.551834in}}%
\pgfpathlineto{\pgfqpoint{2.746178in}{0.551834in}}%
\pgfpathlineto{\pgfqpoint{2.757084in}{0.551834in}}%
\pgfpathlineto{\pgfqpoint{2.767990in}{0.551834in}}%
\pgfpathlineto{\pgfqpoint{2.778896in}{0.551834in}}%
\pgfpathlineto{\pgfqpoint{2.789802in}{0.551834in}}%
\pgfpathlineto{\pgfqpoint{2.800708in}{0.551834in}}%
\pgfpathlineto{\pgfqpoint{2.811614in}{0.551834in}}%
\pgfpathlineto{\pgfqpoint{2.822520in}{0.551834in}}%
\pgfpathlineto{\pgfqpoint{2.833426in}{0.548215in}}%
\pgfpathlineto{\pgfqpoint{2.844332in}{0.548215in}}%
\pgfpathlineto{\pgfqpoint{2.855238in}{0.548215in}}%
\pgfpathlineto{\pgfqpoint{2.866144in}{0.548215in}}%
\pgfpathlineto{\pgfqpoint{2.877050in}{0.548215in}}%
\pgfpathlineto{\pgfqpoint{2.887956in}{0.548215in}}%
\pgfpathlineto{\pgfqpoint{2.898862in}{0.548215in}}%
\pgfpathlineto{\pgfqpoint{2.909768in}{0.548215in}}%
\pgfpathlineto{\pgfqpoint{2.920674in}{0.548215in}}%
\pgfpathlineto{\pgfqpoint{2.931580in}{0.548215in}}%
\pgfpathlineto{\pgfqpoint{2.931580in}{0.507275in}}%
\pgfpathlineto{\pgfqpoint{2.931580in}{0.507275in}}%
\pgfpathlineto{\pgfqpoint{2.920674in}{0.507275in}}%
\pgfpathlineto{\pgfqpoint{2.909768in}{0.507275in}}%
\pgfpathlineto{\pgfqpoint{2.898862in}{0.507275in}}%
\pgfpathlineto{\pgfqpoint{2.887956in}{0.507275in}}%
\pgfpathlineto{\pgfqpoint{2.877050in}{0.507275in}}%
\pgfpathlineto{\pgfqpoint{2.866144in}{0.507275in}}%
\pgfpathlineto{\pgfqpoint{2.855238in}{0.507275in}}%
\pgfpathlineto{\pgfqpoint{2.844332in}{0.507275in}}%
\pgfpathlineto{\pgfqpoint{2.833426in}{0.507275in}}%
\pgfpathlineto{\pgfqpoint{2.822520in}{0.514038in}}%
\pgfpathlineto{\pgfqpoint{2.811614in}{0.514038in}}%
\pgfpathlineto{\pgfqpoint{2.800708in}{0.514038in}}%
\pgfpathlineto{\pgfqpoint{2.789802in}{0.514038in}}%
\pgfpathlineto{\pgfqpoint{2.778896in}{0.514038in}}%
\pgfpathlineto{\pgfqpoint{2.767990in}{0.514038in}}%
\pgfpathlineto{\pgfqpoint{2.757084in}{0.514038in}}%
\pgfpathlineto{\pgfqpoint{2.746178in}{0.514038in}}%
\pgfpathlineto{\pgfqpoint{2.735272in}{0.514038in}}%
\pgfpathlineto{\pgfqpoint{2.724366in}{0.514038in}}%
\pgfpathlineto{\pgfqpoint{2.713460in}{0.514038in}}%
\pgfpathlineto{\pgfqpoint{2.702554in}{0.514038in}}%
\pgfpathlineto{\pgfqpoint{2.691648in}{0.514038in}}%
\pgfpathlineto{\pgfqpoint{2.680742in}{0.514038in}}%
\pgfpathlineto{\pgfqpoint{2.669836in}{0.514038in}}%
\pgfpathlineto{\pgfqpoint{2.658930in}{0.514038in}}%
\pgfpathlineto{\pgfqpoint{2.648024in}{0.514038in}}%
\pgfpathlineto{\pgfqpoint{2.637118in}{0.514038in}}%
\pgfpathlineto{\pgfqpoint{2.626212in}{0.514038in}}%
\pgfpathlineto{\pgfqpoint{2.615306in}{0.514038in}}%
\pgfpathlineto{\pgfqpoint{2.604400in}{0.514038in}}%
\pgfpathlineto{\pgfqpoint{2.593494in}{0.514038in}}%
\pgfpathlineto{\pgfqpoint{2.582588in}{0.514038in}}%
\pgfpathlineto{\pgfqpoint{2.571682in}{0.514038in}}%
\pgfpathlineto{\pgfqpoint{2.560776in}{0.514038in}}%
\pgfpathlineto{\pgfqpoint{2.549870in}{0.514038in}}%
\pgfpathlineto{\pgfqpoint{2.538964in}{0.514038in}}%
\pgfpathlineto{\pgfqpoint{2.528058in}{0.514038in}}%
\pgfpathlineto{\pgfqpoint{2.517152in}{0.514038in}}%
\pgfpathlineto{\pgfqpoint{2.506246in}{0.514038in}}%
\pgfpathlineto{\pgfqpoint{2.495340in}{0.514038in}}%
\pgfpathlineto{\pgfqpoint{2.484434in}{0.518632in}}%
\pgfpathlineto{\pgfqpoint{2.473528in}{0.518632in}}%
\pgfpathlineto{\pgfqpoint{2.462623in}{0.518632in}}%
\pgfpathlineto{\pgfqpoint{2.451717in}{0.518632in}}%
\pgfpathlineto{\pgfqpoint{2.440811in}{0.518632in}}%
\pgfpathlineto{\pgfqpoint{2.429905in}{0.518632in}}%
\pgfpathlineto{\pgfqpoint{2.418999in}{0.518632in}}%
\pgfpathlineto{\pgfqpoint{2.408093in}{0.518632in}}%
\pgfpathlineto{\pgfqpoint{2.397187in}{0.518632in}}%
\pgfpathlineto{\pgfqpoint{2.386281in}{0.518632in}}%
\pgfpathlineto{\pgfqpoint{2.375375in}{0.518632in}}%
\pgfpathlineto{\pgfqpoint{2.364469in}{0.518632in}}%
\pgfpathlineto{\pgfqpoint{2.353563in}{0.518632in}}%
\pgfpathlineto{\pgfqpoint{2.342657in}{0.518662in}}%
\pgfpathlineto{\pgfqpoint{2.331751in}{0.518662in}}%
\pgfpathlineto{\pgfqpoint{2.320845in}{0.518662in}}%
\pgfpathlineto{\pgfqpoint{2.309939in}{0.518662in}}%
\pgfpathlineto{\pgfqpoint{2.299033in}{0.518662in}}%
\pgfpathlineto{\pgfqpoint{2.288127in}{0.518662in}}%
\pgfpathlineto{\pgfqpoint{2.277221in}{0.518662in}}%
\pgfpathlineto{\pgfqpoint{2.266315in}{0.518662in}}%
\pgfpathlineto{\pgfqpoint{2.255409in}{0.518662in}}%
\pgfpathlineto{\pgfqpoint{2.244503in}{0.518662in}}%
\pgfpathlineto{\pgfqpoint{2.233597in}{0.518662in}}%
\pgfpathlineto{\pgfqpoint{2.222691in}{0.518662in}}%
\pgfpathlineto{\pgfqpoint{2.211785in}{0.518662in}}%
\pgfpathlineto{\pgfqpoint{2.200879in}{0.518662in}}%
\pgfpathlineto{\pgfqpoint{2.189973in}{0.518662in}}%
\pgfpathlineto{\pgfqpoint{2.179067in}{0.518662in}}%
\pgfpathlineto{\pgfqpoint{2.168161in}{0.518662in}}%
\pgfpathlineto{\pgfqpoint{2.157255in}{0.518662in}}%
\pgfpathlineto{\pgfqpoint{2.146349in}{0.518662in}}%
\pgfpathlineto{\pgfqpoint{2.135443in}{0.518662in}}%
\pgfpathlineto{\pgfqpoint{2.124537in}{0.518662in}}%
\pgfpathlineto{\pgfqpoint{2.113631in}{0.518662in}}%
\pgfpathlineto{\pgfqpoint{2.102725in}{0.518662in}}%
\pgfpathlineto{\pgfqpoint{2.091819in}{0.518662in}}%
\pgfpathlineto{\pgfqpoint{2.080913in}{0.518662in}}%
\pgfpathlineto{\pgfqpoint{2.070007in}{0.518662in}}%
\pgfpathlineto{\pgfqpoint{2.059101in}{0.518662in}}%
\pgfpathlineto{\pgfqpoint{2.048195in}{0.518662in}}%
\pgfpathlineto{\pgfqpoint{2.037289in}{0.518662in}}%
\pgfpathlineto{\pgfqpoint{2.026383in}{0.518662in}}%
\pgfpathlineto{\pgfqpoint{2.015477in}{0.518662in}}%
\pgfpathlineto{\pgfqpoint{2.004571in}{0.518662in}}%
\pgfpathlineto{\pgfqpoint{1.993665in}{0.518662in}}%
\pgfpathlineto{\pgfqpoint{1.982759in}{0.518662in}}%
\pgfpathlineto{\pgfqpoint{1.971854in}{0.518662in}}%
\pgfpathlineto{\pgfqpoint{1.960948in}{0.518662in}}%
\pgfpathlineto{\pgfqpoint{1.950042in}{0.518662in}}%
\pgfpathlineto{\pgfqpoint{1.939136in}{0.518662in}}%
\pgfpathlineto{\pgfqpoint{1.928230in}{0.518662in}}%
\pgfpathlineto{\pgfqpoint{1.917324in}{0.518662in}}%
\pgfpathlineto{\pgfqpoint{1.906418in}{0.518662in}}%
\pgfpathlineto{\pgfqpoint{1.895512in}{0.518662in}}%
\pgfpathlineto{\pgfqpoint{1.884606in}{0.518662in}}%
\pgfpathlineto{\pgfqpoint{1.873700in}{0.518662in}}%
\pgfpathlineto{\pgfqpoint{1.862794in}{0.518662in}}%
\pgfpathlineto{\pgfqpoint{1.851888in}{0.518662in}}%
\pgfpathlineto{\pgfqpoint{1.840982in}{0.518662in}}%
\pgfpathlineto{\pgfqpoint{1.830076in}{0.518662in}}%
\pgfpathlineto{\pgfqpoint{1.819170in}{0.518662in}}%
\pgfpathlineto{\pgfqpoint{1.808264in}{0.518662in}}%
\pgfpathlineto{\pgfqpoint{1.797358in}{0.518662in}}%
\pgfpathlineto{\pgfqpoint{1.786452in}{0.518662in}}%
\pgfpathlineto{\pgfqpoint{1.775546in}{0.518662in}}%
\pgfpathlineto{\pgfqpoint{1.764640in}{0.518662in}}%
\pgfpathlineto{\pgfqpoint{1.753734in}{0.518662in}}%
\pgfpathlineto{\pgfqpoint{1.742828in}{0.518662in}}%
\pgfpathlineto{\pgfqpoint{1.731922in}{0.518662in}}%
\pgfpathlineto{\pgfqpoint{1.721016in}{0.518662in}}%
\pgfpathlineto{\pgfqpoint{1.710110in}{0.518662in}}%
\pgfpathlineto{\pgfqpoint{1.699204in}{0.518662in}}%
\pgfpathlineto{\pgfqpoint{1.688298in}{0.518662in}}%
\pgfpathlineto{\pgfqpoint{1.677392in}{0.523862in}}%
\pgfpathlineto{\pgfqpoint{1.666486in}{0.523862in}}%
\pgfpathlineto{\pgfqpoint{1.655580in}{0.523862in}}%
\pgfpathlineto{\pgfqpoint{1.644674in}{0.523862in}}%
\pgfpathlineto{\pgfqpoint{1.633768in}{0.523862in}}%
\pgfpathlineto{\pgfqpoint{1.622862in}{0.523862in}}%
\pgfpathlineto{\pgfqpoint{1.611956in}{0.523862in}}%
\pgfpathlineto{\pgfqpoint{1.601050in}{0.523862in}}%
\pgfpathlineto{\pgfqpoint{1.590144in}{0.523862in}}%
\pgfpathlineto{\pgfqpoint{1.579238in}{0.523862in}}%
\pgfpathlineto{\pgfqpoint{1.568332in}{0.523862in}}%
\pgfpathlineto{\pgfqpoint{1.557426in}{0.523862in}}%
\pgfpathlineto{\pgfqpoint{1.546520in}{0.523862in}}%
\pgfpathlineto{\pgfqpoint{1.535614in}{0.523862in}}%
\pgfpathlineto{\pgfqpoint{1.524708in}{0.523862in}}%
\pgfpathlineto{\pgfqpoint{1.513802in}{0.523862in}}%
\pgfpathlineto{\pgfqpoint{1.502896in}{0.523862in}}%
\pgfpathlineto{\pgfqpoint{1.491991in}{0.523862in}}%
\pgfpathlineto{\pgfqpoint{1.481085in}{0.523862in}}%
\pgfpathlineto{\pgfqpoint{1.470179in}{0.539062in}}%
\pgfpathlineto{\pgfqpoint{1.459273in}{0.549669in}}%
\pgfpathlineto{\pgfqpoint{1.448367in}{0.549669in}}%
\pgfpathlineto{\pgfqpoint{1.437461in}{0.549669in}}%
\pgfpathlineto{\pgfqpoint{1.426555in}{0.571083in}}%
\pgfpathlineto{\pgfqpoint{1.415649in}{0.571083in}}%
\pgfpathlineto{\pgfqpoint{1.404743in}{0.571083in}}%
\pgfpathlineto{\pgfqpoint{1.393837in}{0.571083in}}%
\pgfpathlineto{\pgfqpoint{1.382931in}{0.572844in}}%
\pgfpathlineto{\pgfqpoint{1.372025in}{0.572844in}}%
\pgfpathlineto{\pgfqpoint{1.361119in}{0.612939in}}%
\pgfpathlineto{\pgfqpoint{1.350213in}{0.612939in}}%
\pgfpathlineto{\pgfqpoint{1.339307in}{0.612939in}}%
\pgfpathlineto{\pgfqpoint{1.328401in}{0.615432in}}%
\pgfpathlineto{\pgfqpoint{1.317495in}{0.615432in}}%
\pgfpathlineto{\pgfqpoint{1.306589in}{0.615432in}}%
\pgfpathlineto{\pgfqpoint{1.295683in}{0.615432in}}%
\pgfpathlineto{\pgfqpoint{1.284777in}{0.615432in}}%
\pgfpathlineto{\pgfqpoint{1.273871in}{0.632851in}}%
\pgfpathlineto{\pgfqpoint{1.262965in}{0.640535in}}%
\pgfpathlineto{\pgfqpoint{1.252059in}{0.643667in}}%
\pgfpathlineto{\pgfqpoint{1.241153in}{0.643667in}}%
\pgfpathlineto{\pgfqpoint{1.230247in}{0.643667in}}%
\pgfpathlineto{\pgfqpoint{1.219341in}{0.643667in}}%
\pgfpathlineto{\pgfqpoint{1.208435in}{0.643667in}}%
\pgfpathlineto{\pgfqpoint{1.197529in}{0.643667in}}%
\pgfpathlineto{\pgfqpoint{1.186623in}{0.645959in}}%
\pgfpathlineto{\pgfqpoint{1.175717in}{0.645959in}}%
\pgfpathlineto{\pgfqpoint{1.164811in}{0.645959in}}%
\pgfpathlineto{\pgfqpoint{1.153905in}{0.645959in}}%
\pgfpathlineto{\pgfqpoint{1.142999in}{0.645959in}}%
\pgfpathlineto{\pgfqpoint{1.132093in}{0.645959in}}%
\pgfpathlineto{\pgfqpoint{1.121187in}{0.645959in}}%
\pgfpathlineto{\pgfqpoint{1.110281in}{0.672693in}}%
\pgfpathlineto{\pgfqpoint{1.099375in}{0.746545in}}%
\pgfpathlineto{\pgfqpoint{1.088469in}{0.746545in}}%
\pgfpathlineto{\pgfqpoint{1.077563in}{0.764210in}}%
\pgfpathlineto{\pgfqpoint{1.066657in}{0.764210in}}%
\pgfpathlineto{\pgfqpoint{1.055751in}{0.764210in}}%
\pgfpathlineto{\pgfqpoint{1.044845in}{0.764210in}}%
\pgfpathlineto{\pgfqpoint{1.033939in}{0.764210in}}%
\pgfpathlineto{\pgfqpoint{1.023033in}{0.850729in}}%
\pgfpathlineto{\pgfqpoint{1.012127in}{0.850729in}}%
\pgfpathlineto{\pgfqpoint{1.001222in}{0.850729in}}%
\pgfpathlineto{\pgfqpoint{0.990316in}{0.850729in}}%
\pgfpathlineto{\pgfqpoint{0.979410in}{0.876798in}}%
\pgfpathlineto{\pgfqpoint{0.968504in}{0.882781in}}%
\pgfpathlineto{\pgfqpoint{0.957598in}{0.882781in}}%
\pgfpathlineto{\pgfqpoint{0.946692in}{0.882781in}}%
\pgfpathlineto{\pgfqpoint{0.935786in}{0.882781in}}%
\pgfpathlineto{\pgfqpoint{0.924880in}{0.894570in}}%
\pgfpathlineto{\pgfqpoint{0.913974in}{0.894570in}}%
\pgfpathlineto{\pgfqpoint{0.903068in}{0.914586in}}%
\pgfpathlineto{\pgfqpoint{0.892162in}{0.914586in}}%
\pgfpathlineto{\pgfqpoint{0.881256in}{0.914586in}}%
\pgfpathlineto{\pgfqpoint{0.870350in}{0.933710in}}%
\pgfpathlineto{\pgfqpoint{0.859444in}{0.933710in}}%
\pgfpathlineto{\pgfqpoint{0.848538in}{0.933710in}}%
\pgfpathlineto{\pgfqpoint{0.837632in}{0.933710in}}%
\pgfpathlineto{\pgfqpoint{0.826726in}{0.933710in}}%
\pgfpathlineto{\pgfqpoint{0.815820in}{0.933710in}}%
\pgfpathlineto{\pgfqpoint{0.804914in}{0.933710in}}%
\pgfpathlineto{\pgfqpoint{0.794008in}{0.933710in}}%
\pgfpathlineto{\pgfqpoint{0.783102in}{0.939999in}}%
\pgfpathlineto{\pgfqpoint{0.772196in}{1.017509in}}%
\pgfpathlineto{\pgfqpoint{0.761290in}{1.068060in}}%
\pgfpathlineto{\pgfqpoint{0.750384in}{1.068060in}}%
\pgfpathlineto{\pgfqpoint{0.739478in}{1.068060in}}%
\pgfpathlineto{\pgfqpoint{0.728572in}{1.083310in}}%
\pgfpathlineto{\pgfqpoint{0.717666in}{1.083310in}}%
\pgfpathlineto{\pgfqpoint{0.706760in}{1.095505in}}%
\pgfpathlineto{\pgfqpoint{0.695854in}{1.095505in}}%
\pgfpathlineto{\pgfqpoint{0.684948in}{1.098030in}}%
\pgfpathlineto{\pgfqpoint{0.674042in}{1.098030in}}%
\pgfpathlineto{\pgfqpoint{0.663136in}{1.113545in}}%
\pgfpathlineto{\pgfqpoint{0.652230in}{1.161260in}}%
\pgfpathlineto{\pgfqpoint{0.641324in}{1.161260in}}%
\pgfpathlineto{\pgfqpoint{0.630418in}{1.162715in}}%
\pgfpathlineto{\pgfqpoint{0.619512in}{1.162715in}}%
\pgfpathlineto{\pgfqpoint{0.608606in}{1.191782in}}%
\pgfpathlineto{\pgfqpoint{0.597700in}{1.199663in}}%
\pgfpathlineto{\pgfqpoint{0.586794in}{1.199663in}}%
\pgfpathlineto{\pgfqpoint{0.575888in}{1.203151in}}%
\pgfpathlineto{\pgfqpoint{0.564982in}{1.304214in}}%
\pgfpathlineto{\pgfqpoint{0.554076in}{1.366846in}}%
\pgfpathlineto{\pgfqpoint{0.543170in}{1.516591in}}%
\pgfpathclose%
\pgfusepath{fill}%
\end{pgfscope}%
\begin{pgfscope}%
\pgfpathrectangle{\pgfqpoint{0.423750in}{0.261892in}}{\pgfqpoint{2.627250in}{1.581827in}}%
\pgfusepath{clip}%
\pgfsetroundcap%
\pgfsetroundjoin%
\pgfsetlinewidth{1.505625pt}%
\definecolor{currentstroke}{rgb}{0.121569,0.466667,0.705882}%
\pgfsetstrokecolor{currentstroke}%
\pgfsetdash{}{0pt}%
\pgfpathmoveto{\pgfqpoint{0.543170in}{1.639223in}}%
\pgfpathlineto{\pgfqpoint{0.554076in}{1.578825in}}%
\pgfpathlineto{\pgfqpoint{0.564982in}{1.443552in}}%
\pgfpathlineto{\pgfqpoint{0.575888in}{1.396367in}}%
\pgfpathlineto{\pgfqpoint{0.586794in}{1.327684in}}%
\pgfpathlineto{\pgfqpoint{0.597700in}{1.284174in}}%
\pgfpathlineto{\pgfqpoint{0.619512in}{1.284174in}}%
\pgfpathlineto{\pgfqpoint{0.630418in}{1.253956in}}%
\pgfpathlineto{\pgfqpoint{0.641324in}{1.245464in}}%
\pgfpathlineto{\pgfqpoint{0.652230in}{1.245464in}}%
\pgfpathlineto{\pgfqpoint{0.663136in}{1.231759in}}%
\pgfpathlineto{\pgfqpoint{0.674042in}{1.231759in}}%
\pgfpathlineto{\pgfqpoint{0.684948in}{1.161449in}}%
\pgfpathlineto{\pgfqpoint{0.695854in}{1.161449in}}%
\pgfpathlineto{\pgfqpoint{0.706760in}{1.084488in}}%
\pgfpathlineto{\pgfqpoint{0.728572in}{1.084488in}}%
\pgfpathlineto{\pgfqpoint{0.739478in}{1.082201in}}%
\pgfpathlineto{\pgfqpoint{0.783102in}{1.082201in}}%
\pgfpathlineto{\pgfqpoint{0.794008in}{1.070565in}}%
\pgfpathlineto{\pgfqpoint{0.804914in}{1.070565in}}%
\pgfpathlineto{\pgfqpoint{0.815820in}{1.068008in}}%
\pgfpathlineto{\pgfqpoint{0.848538in}{1.068008in}}%
\pgfpathlineto{\pgfqpoint{0.859444in}{1.062203in}}%
\pgfpathlineto{\pgfqpoint{0.870350in}{1.062203in}}%
\pgfpathlineto{\pgfqpoint{0.881256in}{0.952958in}}%
\pgfpathlineto{\pgfqpoint{0.903068in}{0.936139in}}%
\pgfpathlineto{\pgfqpoint{0.979410in}{0.936139in}}%
\pgfpathlineto{\pgfqpoint{0.990316in}{0.933856in}}%
\pgfpathlineto{\pgfqpoint{1.023033in}{0.933856in}}%
\pgfpathlineto{\pgfqpoint{1.033939in}{0.903931in}}%
\pgfpathlineto{\pgfqpoint{1.044845in}{0.901592in}}%
\pgfpathlineto{\pgfqpoint{1.110281in}{0.901592in}}%
\pgfpathlineto{\pgfqpoint{1.121187in}{0.894548in}}%
\pgfpathlineto{\pgfqpoint{1.219341in}{0.894548in}}%
\pgfpathlineto{\pgfqpoint{1.230247in}{0.881821in}}%
\pgfpathlineto{\pgfqpoint{1.262965in}{0.881821in}}%
\pgfpathlineto{\pgfqpoint{1.273871in}{0.879217in}}%
\pgfpathlineto{\pgfqpoint{1.470179in}{0.879217in}}%
\pgfpathlineto{\pgfqpoint{1.481085in}{0.877090in}}%
\pgfpathlineto{\pgfqpoint{1.502896in}{0.877090in}}%
\pgfpathlineto{\pgfqpoint{1.513802in}{0.850200in}}%
\pgfpathlineto{\pgfqpoint{1.546520in}{0.850200in}}%
\pgfpathlineto{\pgfqpoint{1.557426in}{0.836763in}}%
\pgfpathlineto{\pgfqpoint{2.146349in}{0.836763in}}%
\pgfpathlineto{\pgfqpoint{2.157255in}{0.815800in}}%
\pgfpathlineto{\pgfqpoint{2.222691in}{0.815800in}}%
\pgfpathlineto{\pgfqpoint{2.233597in}{0.811369in}}%
\pgfpathlineto{\pgfqpoint{2.386281in}{0.811369in}}%
\pgfpathlineto{\pgfqpoint{2.397187in}{0.809718in}}%
\pgfpathlineto{\pgfqpoint{2.604400in}{0.809718in}}%
\pgfpathlineto{\pgfqpoint{2.615306in}{0.804715in}}%
\pgfpathlineto{\pgfqpoint{2.757084in}{0.804715in}}%
\pgfpathlineto{\pgfqpoint{2.767990in}{0.780937in}}%
\pgfpathlineto{\pgfqpoint{2.931580in}{0.780937in}}%
\pgfpathlineto{\pgfqpoint{2.931580in}{0.780937in}}%
\pgfusepath{stroke}%
\end{pgfscope}%
\begin{pgfscope}%
\pgfpathrectangle{\pgfqpoint{0.423750in}{0.261892in}}{\pgfqpoint{2.627250in}{1.581827in}}%
\pgfusepath{clip}%
\pgfsetroundcap%
\pgfsetroundjoin%
\pgfsetlinewidth{1.505625pt}%
\definecolor{currentstroke}{rgb}{1.000000,0.498039,0.054902}%
\pgfsetstrokecolor{currentstroke}%
\pgfsetdash{}{0pt}%
\pgfpathmoveto{\pgfqpoint{0.543170in}{1.566193in}}%
\pgfpathlineto{\pgfqpoint{0.554076in}{1.302791in}}%
\pgfpathlineto{\pgfqpoint{0.564982in}{1.302791in}}%
\pgfpathlineto{\pgfqpoint{0.575888in}{1.242368in}}%
\pgfpathlineto{\pgfqpoint{0.586794in}{1.224782in}}%
\pgfpathlineto{\pgfqpoint{0.597700in}{1.221686in}}%
\pgfpathlineto{\pgfqpoint{0.608606in}{1.221686in}}%
\pgfpathlineto{\pgfqpoint{0.619512in}{1.216939in}}%
\pgfpathlineto{\pgfqpoint{0.684948in}{1.216939in}}%
\pgfpathlineto{\pgfqpoint{0.695854in}{1.186829in}}%
\pgfpathlineto{\pgfqpoint{0.706760in}{1.171205in}}%
\pgfpathlineto{\pgfqpoint{0.728572in}{1.171205in}}%
\pgfpathlineto{\pgfqpoint{0.739478in}{1.157621in}}%
\pgfpathlineto{\pgfqpoint{0.750384in}{1.109029in}}%
\pgfpathlineto{\pgfqpoint{0.783102in}{1.109029in}}%
\pgfpathlineto{\pgfqpoint{0.794008in}{1.094260in}}%
\pgfpathlineto{\pgfqpoint{0.815820in}{1.094260in}}%
\pgfpathlineto{\pgfqpoint{0.826726in}{1.073883in}}%
\pgfpathlineto{\pgfqpoint{0.848538in}{1.073883in}}%
\pgfpathlineto{\pgfqpoint{0.859444in}{0.946104in}}%
\pgfpathlineto{\pgfqpoint{0.870350in}{0.926285in}}%
\pgfpathlineto{\pgfqpoint{0.881256in}{0.893248in}}%
\pgfpathlineto{\pgfqpoint{0.892162in}{0.879009in}}%
\pgfpathlineto{\pgfqpoint{0.968504in}{0.879009in}}%
\pgfpathlineto{\pgfqpoint{0.979410in}{0.871218in}}%
\pgfpathlineto{\pgfqpoint{0.990316in}{0.870326in}}%
\pgfpathlineto{\pgfqpoint{1.001222in}{0.862232in}}%
\pgfpathlineto{\pgfqpoint{1.023033in}{0.862232in}}%
\pgfpathlineto{\pgfqpoint{1.033939in}{0.842824in}}%
\pgfpathlineto{\pgfqpoint{1.044845in}{0.828968in}}%
\pgfpathlineto{\pgfqpoint{1.055751in}{0.828968in}}%
\pgfpathlineto{\pgfqpoint{1.066657in}{0.787915in}}%
\pgfpathlineto{\pgfqpoint{1.077563in}{0.787915in}}%
\pgfpathlineto{\pgfqpoint{1.088469in}{0.758154in}}%
\pgfpathlineto{\pgfqpoint{1.110281in}{0.758154in}}%
\pgfpathlineto{\pgfqpoint{1.121187in}{0.711719in}}%
\pgfpathlineto{\pgfqpoint{1.142999in}{0.711719in}}%
\pgfpathlineto{\pgfqpoint{1.153905in}{0.708487in}}%
\pgfpathlineto{\pgfqpoint{1.197529in}{0.708487in}}%
\pgfpathlineto{\pgfqpoint{1.208435in}{0.694552in}}%
\pgfpathlineto{\pgfqpoint{1.241153in}{0.694552in}}%
\pgfpathlineto{\pgfqpoint{1.252059in}{0.669007in}}%
\pgfpathlineto{\pgfqpoint{1.262965in}{0.658717in}}%
\pgfpathlineto{\pgfqpoint{1.284777in}{0.658717in}}%
\pgfpathlineto{\pgfqpoint{1.295683in}{0.643798in}}%
\pgfpathlineto{\pgfqpoint{1.361119in}{0.643609in}}%
\pgfpathlineto{\pgfqpoint{1.372025in}{0.607779in}}%
\pgfpathlineto{\pgfqpoint{1.677392in}{0.607779in}}%
\pgfpathlineto{\pgfqpoint{1.688298in}{0.604364in}}%
\pgfpathlineto{\pgfqpoint{1.819170in}{0.604364in}}%
\pgfpathlineto{\pgfqpoint{1.830076in}{0.595188in}}%
\pgfpathlineto{\pgfqpoint{1.895512in}{0.595188in}}%
\pgfpathlineto{\pgfqpoint{1.906418in}{0.590051in}}%
\pgfpathlineto{\pgfqpoint{1.917324in}{0.587162in}}%
\pgfpathlineto{\pgfqpoint{2.015477in}{0.587162in}}%
\pgfpathlineto{\pgfqpoint{2.026383in}{0.565555in}}%
\pgfpathlineto{\pgfqpoint{2.168161in}{0.565555in}}%
\pgfpathlineto{\pgfqpoint{2.179067in}{0.561675in}}%
\pgfpathlineto{\pgfqpoint{2.288127in}{0.561675in}}%
\pgfpathlineto{\pgfqpoint{2.299033in}{0.554483in}}%
\pgfpathlineto{\pgfqpoint{2.408093in}{0.554483in}}%
\pgfpathlineto{\pgfqpoint{2.418999in}{0.521932in}}%
\pgfpathlineto{\pgfqpoint{2.528058in}{0.521932in}}%
\pgfpathlineto{\pgfqpoint{2.538964in}{0.510664in}}%
\pgfpathlineto{\pgfqpoint{2.593494in}{0.510664in}}%
\pgfpathlineto{\pgfqpoint{2.604400in}{0.492825in}}%
\pgfpathlineto{\pgfqpoint{2.789802in}{0.491949in}}%
\pgfpathlineto{\pgfqpoint{2.800708in}{0.482699in}}%
\pgfpathlineto{\pgfqpoint{2.811614in}{0.482699in}}%
\pgfpathlineto{\pgfqpoint{2.822520in}{0.467018in}}%
\pgfpathlineto{\pgfqpoint{2.931580in}{0.467018in}}%
\pgfpathlineto{\pgfqpoint{2.931580in}{0.467018in}}%
\pgfusepath{stroke}%
\end{pgfscope}%
\begin{pgfscope}%
\pgfpathrectangle{\pgfqpoint{0.423750in}{0.261892in}}{\pgfqpoint{2.627250in}{1.581827in}}%
\pgfusepath{clip}%
\pgfsetroundcap%
\pgfsetroundjoin%
\pgfsetlinewidth{1.505625pt}%
\definecolor{currentstroke}{rgb}{0.172549,0.627451,0.172549}%
\pgfsetstrokecolor{currentstroke}%
\pgfsetdash{}{0pt}%
\pgfpathmoveto{\pgfqpoint{0.543170in}{1.584412in}}%
\pgfpathlineto{\pgfqpoint{0.554076in}{1.554616in}}%
\pgfpathlineto{\pgfqpoint{0.564982in}{1.485048in}}%
\pgfpathlineto{\pgfqpoint{0.575888in}{1.485048in}}%
\pgfpathlineto{\pgfqpoint{0.586794in}{1.409502in}}%
\pgfpathlineto{\pgfqpoint{0.597700in}{1.288256in}}%
\pgfpathlineto{\pgfqpoint{0.608606in}{1.241246in}}%
\pgfpathlineto{\pgfqpoint{0.619512in}{1.182834in}}%
\pgfpathlineto{\pgfqpoint{0.674042in}{1.181956in}}%
\pgfpathlineto{\pgfqpoint{0.684948in}{1.171736in}}%
\pgfpathlineto{\pgfqpoint{0.706760in}{1.171736in}}%
\pgfpathlineto{\pgfqpoint{0.717666in}{1.166629in}}%
\pgfpathlineto{\pgfqpoint{0.728572in}{1.125429in}}%
\pgfpathlineto{\pgfqpoint{0.826726in}{1.125429in}}%
\pgfpathlineto{\pgfqpoint{0.837632in}{1.062560in}}%
\pgfpathlineto{\pgfqpoint{0.848538in}{1.062560in}}%
\pgfpathlineto{\pgfqpoint{0.859444in}{1.058903in}}%
\pgfpathlineto{\pgfqpoint{0.913974in}{1.058903in}}%
\pgfpathlineto{\pgfqpoint{0.924880in}{1.048447in}}%
\pgfpathlineto{\pgfqpoint{0.935786in}{1.025646in}}%
\pgfpathlineto{\pgfqpoint{0.946692in}{0.972428in}}%
\pgfpathlineto{\pgfqpoint{0.957598in}{0.959110in}}%
\pgfpathlineto{\pgfqpoint{1.001222in}{0.959110in}}%
\pgfpathlineto{\pgfqpoint{1.012127in}{0.929353in}}%
\pgfpathlineto{\pgfqpoint{1.023033in}{0.929353in}}%
\pgfpathlineto{\pgfqpoint{1.033939in}{0.916731in}}%
\pgfpathlineto{\pgfqpoint{1.044845in}{0.863146in}}%
\pgfpathlineto{\pgfqpoint{1.448367in}{0.863146in}}%
\pgfpathlineto{\pgfqpoint{1.459273in}{0.811005in}}%
\pgfpathlineto{\pgfqpoint{1.470179in}{0.811005in}}%
\pgfpathlineto{\pgfqpoint{1.481085in}{0.807011in}}%
\pgfpathlineto{\pgfqpoint{1.557426in}{0.807011in}}%
\pgfpathlineto{\pgfqpoint{1.568332in}{0.780047in}}%
\pgfpathlineto{\pgfqpoint{1.688298in}{0.780047in}}%
\pgfpathlineto{\pgfqpoint{1.699204in}{0.774129in}}%
\pgfpathlineto{\pgfqpoint{1.721016in}{0.774129in}}%
\pgfpathlineto{\pgfqpoint{1.731922in}{0.754170in}}%
\pgfpathlineto{\pgfqpoint{1.830076in}{0.754170in}}%
\pgfpathlineto{\pgfqpoint{1.840982in}{0.747716in}}%
\pgfpathlineto{\pgfqpoint{1.873700in}{0.747716in}}%
\pgfpathlineto{\pgfqpoint{1.884606in}{0.732364in}}%
\pgfpathlineto{\pgfqpoint{1.895512in}{0.701717in}}%
\pgfpathlineto{\pgfqpoint{1.917324in}{0.701717in}}%
\pgfpathlineto{\pgfqpoint{1.928230in}{0.657555in}}%
\pgfpathlineto{\pgfqpoint{1.971854in}{0.657555in}}%
\pgfpathlineto{\pgfqpoint{1.982759in}{0.634199in}}%
\pgfpathlineto{\pgfqpoint{1.993665in}{0.634199in}}%
\pgfpathlineto{\pgfqpoint{2.004571in}{0.618929in}}%
\pgfpathlineto{\pgfqpoint{2.157255in}{0.618929in}}%
\pgfpathlineto{\pgfqpoint{2.168161in}{0.612373in}}%
\pgfpathlineto{\pgfqpoint{2.397187in}{0.612329in}}%
\pgfpathlineto{\pgfqpoint{2.408093in}{0.603235in}}%
\pgfpathlineto{\pgfqpoint{2.549870in}{0.602490in}}%
\pgfpathlineto{\pgfqpoint{2.560776in}{0.561156in}}%
\pgfpathlineto{\pgfqpoint{2.931580in}{0.561156in}}%
\pgfpathlineto{\pgfqpoint{2.931580in}{0.561156in}}%
\pgfusepath{stroke}%
\end{pgfscope}%
\begin{pgfscope}%
\pgfpathrectangle{\pgfqpoint{0.423750in}{0.261892in}}{\pgfqpoint{2.627250in}{1.581827in}}%
\pgfusepath{clip}%
\pgfsetroundcap%
\pgfsetroundjoin%
\pgfsetlinewidth{1.505625pt}%
\definecolor{currentstroke}{rgb}{0.839216,0.152941,0.156863}%
\pgfsetstrokecolor{currentstroke}%
\pgfsetdash{}{0pt}%
\pgfpathmoveto{\pgfqpoint{0.543170in}{1.682200in}}%
\pgfpathlineto{\pgfqpoint{0.554076in}{1.442965in}}%
\pgfpathlineto{\pgfqpoint{0.564982in}{1.388589in}}%
\pgfpathlineto{\pgfqpoint{0.575888in}{1.313029in}}%
\pgfpathlineto{\pgfqpoint{0.597700in}{1.311575in}}%
\pgfpathlineto{\pgfqpoint{0.608606in}{1.254621in}}%
\pgfpathlineto{\pgfqpoint{0.619512in}{1.209533in}}%
\pgfpathlineto{\pgfqpoint{0.630418in}{1.209533in}}%
\pgfpathlineto{\pgfqpoint{0.641324in}{1.201023in}}%
\pgfpathlineto{\pgfqpoint{0.652230in}{1.201023in}}%
\pgfpathlineto{\pgfqpoint{0.674042in}{1.139947in}}%
\pgfpathlineto{\pgfqpoint{0.684948in}{1.139947in}}%
\pgfpathlineto{\pgfqpoint{0.695854in}{1.138228in}}%
\pgfpathlineto{\pgfqpoint{0.706760in}{1.138228in}}%
\pgfpathlineto{\pgfqpoint{0.717666in}{1.122045in}}%
\pgfpathlineto{\pgfqpoint{0.728572in}{1.122045in}}%
\pgfpathlineto{\pgfqpoint{0.739478in}{1.115108in}}%
\pgfpathlineto{\pgfqpoint{0.761290in}{1.115108in}}%
\pgfpathlineto{\pgfqpoint{0.783102in}{1.000935in}}%
\pgfpathlineto{\pgfqpoint{0.794008in}{0.997191in}}%
\pgfpathlineto{\pgfqpoint{0.870350in}{0.997191in}}%
\pgfpathlineto{\pgfqpoint{0.881256in}{0.980323in}}%
\pgfpathlineto{\pgfqpoint{0.903068in}{0.980323in}}%
\pgfpathlineto{\pgfqpoint{0.913974in}{0.964224in}}%
\pgfpathlineto{\pgfqpoint{0.924880in}{0.964224in}}%
\pgfpathlineto{\pgfqpoint{0.935786in}{0.949841in}}%
\pgfpathlineto{\pgfqpoint{0.968504in}{0.949841in}}%
\pgfpathlineto{\pgfqpoint{0.979410in}{0.944120in}}%
\pgfpathlineto{\pgfqpoint{0.990316in}{0.925305in}}%
\pgfpathlineto{\pgfqpoint{1.023033in}{0.925305in}}%
\pgfpathlineto{\pgfqpoint{1.033939in}{0.882377in}}%
\pgfpathlineto{\pgfqpoint{1.077563in}{0.882377in}}%
\pgfpathlineto{\pgfqpoint{1.088469in}{0.819746in}}%
\pgfpathlineto{\pgfqpoint{1.099375in}{0.819746in}}%
\pgfpathlineto{\pgfqpoint{1.110281in}{0.775473in}}%
\pgfpathlineto{\pgfqpoint{1.121187in}{0.761707in}}%
\pgfpathlineto{\pgfqpoint{1.186623in}{0.761707in}}%
\pgfpathlineto{\pgfqpoint{1.197529in}{0.714824in}}%
\pgfpathlineto{\pgfqpoint{1.252059in}{0.714824in}}%
\pgfpathlineto{\pgfqpoint{1.284777in}{0.649008in}}%
\pgfpathlineto{\pgfqpoint{1.328401in}{0.649008in}}%
\pgfpathlineto{\pgfqpoint{1.339307in}{0.647079in}}%
\pgfpathlineto{\pgfqpoint{1.361119in}{0.647079in}}%
\pgfpathlineto{\pgfqpoint{1.372025in}{0.615047in}}%
\pgfpathlineto{\pgfqpoint{1.426555in}{0.614125in}}%
\pgfpathlineto{\pgfqpoint{1.437461in}{0.599088in}}%
\pgfpathlineto{\pgfqpoint{1.459273in}{0.599088in}}%
\pgfpathlineto{\pgfqpoint{1.470179in}{0.593205in}}%
\pgfpathlineto{\pgfqpoint{1.481085in}{0.580585in}}%
\pgfpathlineto{\pgfqpoint{1.677392in}{0.580585in}}%
\pgfpathlineto{\pgfqpoint{1.688298in}{0.577024in}}%
\pgfpathlineto{\pgfqpoint{2.484434in}{0.575690in}}%
\pgfpathlineto{\pgfqpoint{2.495340in}{0.532985in}}%
\pgfpathlineto{\pgfqpoint{2.822520in}{0.532985in}}%
\pgfpathlineto{\pgfqpoint{2.833426in}{0.527892in}}%
\pgfpathlineto{\pgfqpoint{2.931580in}{0.527892in}}%
\pgfpathlineto{\pgfqpoint{2.931580in}{0.527892in}}%
\pgfusepath{stroke}%
\end{pgfscope}%
\begin{pgfscope}%
\pgfsetrectcap%
\pgfsetmiterjoin%
\pgfsetlinewidth{0.000000pt}%
\definecolor{currentstroke}{rgb}{1.000000,1.000000,1.000000}%
\pgfsetstrokecolor{currentstroke}%
\pgfsetdash{}{0pt}%
\pgfpathmoveto{\pgfqpoint{0.423750in}{0.261892in}}%
\pgfpathlineto{\pgfqpoint{0.423750in}{1.843719in}}%
\pgfusepath{}%
\end{pgfscope}%
\begin{pgfscope}%
\pgfsetrectcap%
\pgfsetmiterjoin%
\pgfsetlinewidth{0.000000pt}%
\definecolor{currentstroke}{rgb}{1.000000,1.000000,1.000000}%
\pgfsetstrokecolor{currentstroke}%
\pgfsetdash{}{0pt}%
\pgfpathmoveto{\pgfqpoint{3.051000in}{0.261892in}}%
\pgfpathlineto{\pgfqpoint{3.051000in}{1.843719in}}%
\pgfusepath{}%
\end{pgfscope}%
\begin{pgfscope}%
\pgfsetrectcap%
\pgfsetmiterjoin%
\pgfsetlinewidth{0.000000pt}%
\definecolor{currentstroke}{rgb}{1.000000,1.000000,1.000000}%
\pgfsetstrokecolor{currentstroke}%
\pgfsetdash{}{0pt}%
\pgfpathmoveto{\pgfqpoint{0.423750in}{0.261892in}}%
\pgfpathlineto{\pgfqpoint{3.051000in}{0.261892in}}%
\pgfusepath{}%
\end{pgfscope}%
\begin{pgfscope}%
\pgfsetrectcap%
\pgfsetmiterjoin%
\pgfsetlinewidth{0.000000pt}%
\definecolor{currentstroke}{rgb}{1.000000,1.000000,1.000000}%
\pgfsetstrokecolor{currentstroke}%
\pgfsetdash{}{0pt}%
\pgfpathmoveto{\pgfqpoint{0.423750in}{1.843719in}}%
\pgfpathlineto{\pgfqpoint{3.051000in}{1.843719in}}%
\pgfusepath{}%
\end{pgfscope}%
\begin{pgfscope}%
\definecolor{textcolor}{rgb}{0.150000,0.150000,0.150000}%
\pgfsetstrokecolor{textcolor}%
\pgfsetfillcolor{textcolor}%
\pgftext[x=1.737375in,y=1.927052in,,base]{\color{textcolor}\rmfamily\fontsize{8.000000}{9.600000}\selectfont Corana}%
\end{pgfscope}%
\begin{pgfscope}%
\pgfsetroundcap%
\pgfsetroundjoin%
\pgfsetlinewidth{1.505625pt}%
\definecolor{currentstroke}{rgb}{0.121569,0.466667,0.705882}%
\pgfsetstrokecolor{currentstroke}%
\pgfsetdash{}{0pt}%
\pgfpathmoveto{\pgfqpoint{1.320607in}{1.698189in}}%
\pgfpathlineto{\pgfqpoint{1.542829in}{1.698189in}}%
\pgfusepath{stroke}%
\end{pgfscope}%
\begin{pgfscope}%
\definecolor{textcolor}{rgb}{0.150000,0.150000,0.150000}%
\pgfsetstrokecolor{textcolor}%
\pgfsetfillcolor{textcolor}%
\pgftext[x=1.631718in,y=1.659301in,left,base]{\color{textcolor}\rmfamily\fontsize{8.000000}{9.600000}\selectfont random}%
\end{pgfscope}%
\begin{pgfscope}%
\pgfsetroundcap%
\pgfsetroundjoin%
\pgfsetlinewidth{1.505625pt}%
\definecolor{currentstroke}{rgb}{1.000000,0.498039,0.054902}%
\pgfsetstrokecolor{currentstroke}%
\pgfsetdash{}{0pt}%
\pgfpathmoveto{\pgfqpoint{1.320607in}{1.535104in}}%
\pgfpathlineto{\pgfqpoint{1.542829in}{1.535104in}}%
\pgfusepath{stroke}%
\end{pgfscope}%
\begin{pgfscope}%
\definecolor{textcolor}{rgb}{0.150000,0.150000,0.150000}%
\pgfsetstrokecolor{textcolor}%
\pgfsetfillcolor{textcolor}%
\pgftext[x=1.631718in,y=1.496215in,left,base]{\color{textcolor}\rmfamily\fontsize{8.000000}{9.600000}\selectfont 5 x DNGO retrain-reset}%
\end{pgfscope}%
\begin{pgfscope}%
\pgfsetroundcap%
\pgfsetroundjoin%
\pgfsetlinewidth{1.505625pt}%
\definecolor{currentstroke}{rgb}{0.172549,0.627451,0.172549}%
\pgfsetstrokecolor{currentstroke}%
\pgfsetdash{}{0pt}%
\pgfpathmoveto{\pgfqpoint{1.320607in}{1.372018in}}%
\pgfpathlineto{\pgfqpoint{1.542829in}{1.372018in}}%
\pgfusepath{stroke}%
\end{pgfscope}%
\begin{pgfscope}%
\definecolor{textcolor}{rgb}{0.150000,0.150000,0.150000}%
\pgfsetstrokecolor{textcolor}%
\pgfsetfillcolor{textcolor}%
\pgftext[x=1.631718in,y=1.333129in,left,base]{\color{textcolor}\rmfamily\fontsize{8.000000}{9.600000}\selectfont DNGO retrain-reset}%
\end{pgfscope}%
\begin{pgfscope}%
\pgfsetroundcap%
\pgfsetroundjoin%
\pgfsetlinewidth{1.505625pt}%
\definecolor{currentstroke}{rgb}{0.839216,0.152941,0.156863}%
\pgfsetstrokecolor{currentstroke}%
\pgfsetdash{}{0pt}%
\pgfpathmoveto{\pgfqpoint{1.320607in}{1.208932in}}%
\pgfpathlineto{\pgfqpoint{1.542829in}{1.208932in}}%
\pgfusepath{stroke}%
\end{pgfscope}%
\begin{pgfscope}%
\definecolor{textcolor}{rgb}{0.150000,0.150000,0.150000}%
\pgfsetstrokecolor{textcolor}%
\pgfsetfillcolor{textcolor}%
\pgftext[x=1.631718in,y=1.170043in,left,base]{\color{textcolor}\rmfamily\fontsize{8.000000}{9.600000}\selectfont GP}%
\end{pgfscope}%
\end{pgfpicture}%
\makeatother%
\endgroup%

            \caption{Example showing the faster convergence for ensembled DNGO over DNGO on nonsmooth function.}
            \end{subfigure}\qquad
            \begin{subfigure}[t]{0.45\textwidth}
                \centering
                % \resizebox{\linewidth}{!}{}
                %% Creator: Matplotlib, PGF backend
%%
%% To include the figure in your LaTeX document, write
%%   \input{<filename>.pgf}
%%
%% Make sure the required packages are loaded in your preamble
%%   \usepackage{pgf}
%%
%% Figures using additional raster images can only be included by \input if
%% they are in the same directory as the main LaTeX file. For loading figures
%% from other directories you can use the `import` package
%%   \usepackage{import}
%% and then include the figures with
%%   \import{<path to file>}{<filename>.pgf}
%%
%% Matplotlib used the following preamble
%%   \usepackage{gensymb}
%%   \usepackage{fontspec}
%%   \setmainfont{DejaVu Serif}
%%   \setsansfont{Arial}
%%   \setmonofont{DejaVu Sans Mono}
%%
\begingroup%
\makeatletter%
\begin{pgfpicture}%
\pgfpathrectangle{\pgfpointorigin}{\pgfqpoint{3.390000in}{2.095135in}}%
\pgfusepath{use as bounding box, clip}%
\begin{pgfscope}%
\pgfsetbuttcap%
\pgfsetmiterjoin%
\definecolor{currentfill}{rgb}{1.000000,1.000000,1.000000}%
\pgfsetfillcolor{currentfill}%
\pgfsetlinewidth{0.000000pt}%
\definecolor{currentstroke}{rgb}{1.000000,1.000000,1.000000}%
\pgfsetstrokecolor{currentstroke}%
\pgfsetdash{}{0pt}%
\pgfpathmoveto{\pgfqpoint{0.000000in}{0.000000in}}%
\pgfpathlineto{\pgfqpoint{3.390000in}{0.000000in}}%
\pgfpathlineto{\pgfqpoint{3.390000in}{2.095135in}}%
\pgfpathlineto{\pgfqpoint{0.000000in}{2.095135in}}%
\pgfpathclose%
\pgfusepath{fill}%
\end{pgfscope}%
\begin{pgfscope}%
\pgfsetbuttcap%
\pgfsetmiterjoin%
\definecolor{currentfill}{rgb}{0.917647,0.917647,0.949020}%
\pgfsetfillcolor{currentfill}%
\pgfsetlinewidth{0.000000pt}%
\definecolor{currentstroke}{rgb}{0.000000,0.000000,0.000000}%
\pgfsetstrokecolor{currentstroke}%
\pgfsetstrokeopacity{0.000000}%
\pgfsetdash{}{0pt}%
\pgfpathmoveto{\pgfqpoint{0.423750in}{0.261892in}}%
\pgfpathlineto{\pgfqpoint{3.051000in}{0.261892in}}%
\pgfpathlineto{\pgfqpoint{3.051000in}{1.843719in}}%
\pgfpathlineto{\pgfqpoint{0.423750in}{1.843719in}}%
\pgfpathclose%
\pgfusepath{fill}%
\end{pgfscope}%
\begin{pgfscope}%
\pgfpathrectangle{\pgfqpoint{0.423750in}{0.261892in}}{\pgfqpoint{2.627250in}{1.581827in}}%
\pgfusepath{clip}%
\pgfsetroundcap%
\pgfsetroundjoin%
\pgfsetlinewidth{0.803000pt}%
\definecolor{currentstroke}{rgb}{1.000000,1.000000,1.000000}%
\pgfsetstrokecolor{currentstroke}%
\pgfsetdash{}{0pt}%
\pgfpathmoveto{\pgfqpoint{0.543170in}{0.261892in}}%
\pgfpathlineto{\pgfqpoint{0.543170in}{1.843719in}}%
\pgfusepath{stroke}%
\end{pgfscope}%
\begin{pgfscope}%
\definecolor{textcolor}{rgb}{0.150000,0.150000,0.150000}%
\pgfsetstrokecolor{textcolor}%
\pgfsetfillcolor{textcolor}%
\pgftext[x=0.543170in,y=0.213281in,,top]{\color{textcolor}\rmfamily\fontsize{8.000000}{9.600000}\selectfont \(\displaystyle 0\)}%
\end{pgfscope}%
\begin{pgfscope}%
\pgfpathrectangle{\pgfqpoint{0.423750in}{0.261892in}}{\pgfqpoint{2.627250in}{1.581827in}}%
\pgfusepath{clip}%
\pgfsetroundcap%
\pgfsetroundjoin%
\pgfsetlinewidth{0.803000pt}%
\definecolor{currentstroke}{rgb}{1.000000,1.000000,1.000000}%
\pgfsetstrokecolor{currentstroke}%
\pgfsetdash{}{0pt}%
\pgfpathmoveto{\pgfqpoint{1.088469in}{0.261892in}}%
\pgfpathlineto{\pgfqpoint{1.088469in}{1.843719in}}%
\pgfusepath{stroke}%
\end{pgfscope}%
\begin{pgfscope}%
\definecolor{textcolor}{rgb}{0.150000,0.150000,0.150000}%
\pgfsetstrokecolor{textcolor}%
\pgfsetfillcolor{textcolor}%
\pgftext[x=1.088469in,y=0.213281in,,top]{\color{textcolor}\rmfamily\fontsize{8.000000}{9.600000}\selectfont \(\displaystyle 50\)}%
\end{pgfscope}%
\begin{pgfscope}%
\pgfpathrectangle{\pgfqpoint{0.423750in}{0.261892in}}{\pgfqpoint{2.627250in}{1.581827in}}%
\pgfusepath{clip}%
\pgfsetroundcap%
\pgfsetroundjoin%
\pgfsetlinewidth{0.803000pt}%
\definecolor{currentstroke}{rgb}{1.000000,1.000000,1.000000}%
\pgfsetstrokecolor{currentstroke}%
\pgfsetdash{}{0pt}%
\pgfpathmoveto{\pgfqpoint{1.633768in}{0.261892in}}%
\pgfpathlineto{\pgfqpoint{1.633768in}{1.843719in}}%
\pgfusepath{stroke}%
\end{pgfscope}%
\begin{pgfscope}%
\definecolor{textcolor}{rgb}{0.150000,0.150000,0.150000}%
\pgfsetstrokecolor{textcolor}%
\pgfsetfillcolor{textcolor}%
\pgftext[x=1.633768in,y=0.213281in,,top]{\color{textcolor}\rmfamily\fontsize{8.000000}{9.600000}\selectfont \(\displaystyle 100\)}%
\end{pgfscope}%
\begin{pgfscope}%
\pgfpathrectangle{\pgfqpoint{0.423750in}{0.261892in}}{\pgfqpoint{2.627250in}{1.581827in}}%
\pgfusepath{clip}%
\pgfsetroundcap%
\pgfsetroundjoin%
\pgfsetlinewidth{0.803000pt}%
\definecolor{currentstroke}{rgb}{1.000000,1.000000,1.000000}%
\pgfsetstrokecolor{currentstroke}%
\pgfsetdash{}{0pt}%
\pgfpathmoveto{\pgfqpoint{2.179067in}{0.261892in}}%
\pgfpathlineto{\pgfqpoint{2.179067in}{1.843719in}}%
\pgfusepath{stroke}%
\end{pgfscope}%
\begin{pgfscope}%
\definecolor{textcolor}{rgb}{0.150000,0.150000,0.150000}%
\pgfsetstrokecolor{textcolor}%
\pgfsetfillcolor{textcolor}%
\pgftext[x=2.179067in,y=0.213281in,,top]{\color{textcolor}\rmfamily\fontsize{8.000000}{9.600000}\selectfont \(\displaystyle 150\)}%
\end{pgfscope}%
\begin{pgfscope}%
\pgfpathrectangle{\pgfqpoint{0.423750in}{0.261892in}}{\pgfqpoint{2.627250in}{1.581827in}}%
\pgfusepath{clip}%
\pgfsetroundcap%
\pgfsetroundjoin%
\pgfsetlinewidth{0.803000pt}%
\definecolor{currentstroke}{rgb}{1.000000,1.000000,1.000000}%
\pgfsetstrokecolor{currentstroke}%
\pgfsetdash{}{0pt}%
\pgfpathmoveto{\pgfqpoint{2.724366in}{0.261892in}}%
\pgfpathlineto{\pgfqpoint{2.724366in}{1.843719in}}%
\pgfusepath{stroke}%
\end{pgfscope}%
\begin{pgfscope}%
\definecolor{textcolor}{rgb}{0.150000,0.150000,0.150000}%
\pgfsetstrokecolor{textcolor}%
\pgfsetfillcolor{textcolor}%
\pgftext[x=2.724366in,y=0.213281in,,top]{\color{textcolor}\rmfamily\fontsize{8.000000}{9.600000}\selectfont \(\displaystyle 200\)}%
\end{pgfscope}%
\begin{pgfscope}%
\definecolor{textcolor}{rgb}{0.150000,0.150000,0.150000}%
\pgfsetstrokecolor{textcolor}%
\pgfsetfillcolor{textcolor}%
\pgftext[x=1.737375in,y=0.050195in,,top]{\color{textcolor}\rmfamily\fontsize{8.000000}{9.600000}\selectfont Step}%
\end{pgfscope}%
\begin{pgfscope}%
\pgfpathrectangle{\pgfqpoint{0.423750in}{0.261892in}}{\pgfqpoint{2.627250in}{1.581827in}}%
\pgfusepath{clip}%
\pgfsetroundcap%
\pgfsetroundjoin%
\pgfsetlinewidth{0.803000pt}%
\definecolor{currentstroke}{rgb}{1.000000,1.000000,1.000000}%
\pgfsetstrokecolor{currentstroke}%
\pgfsetdash{}{0pt}%
\pgfpathmoveto{\pgfqpoint{0.423750in}{0.432158in}}%
\pgfpathlineto{\pgfqpoint{3.051000in}{0.432158in}}%
\pgfusepath{stroke}%
\end{pgfscope}%
\begin{pgfscope}%
\definecolor{textcolor}{rgb}{0.150000,0.150000,0.150000}%
\pgfsetstrokecolor{textcolor}%
\pgfsetfillcolor{textcolor}%
\pgftext[x=0.118966in,y=0.389949in,left,base]{\color{textcolor}\rmfamily\fontsize{8.000000}{9.600000}\selectfont \(\displaystyle 10^{-4}\)}%
\end{pgfscope}%
\begin{pgfscope}%
\pgfpathrectangle{\pgfqpoint{0.423750in}{0.261892in}}{\pgfqpoint{2.627250in}{1.581827in}}%
\pgfusepath{clip}%
\pgfsetroundcap%
\pgfsetroundjoin%
\pgfsetlinewidth{0.803000pt}%
\definecolor{currentstroke}{rgb}{1.000000,1.000000,1.000000}%
\pgfsetstrokecolor{currentstroke}%
\pgfsetdash{}{0pt}%
\pgfpathmoveto{\pgfqpoint{0.423750in}{0.766969in}}%
\pgfpathlineto{\pgfqpoint{3.051000in}{0.766969in}}%
\pgfusepath{stroke}%
\end{pgfscope}%
\begin{pgfscope}%
\definecolor{textcolor}{rgb}{0.150000,0.150000,0.150000}%
\pgfsetstrokecolor{textcolor}%
\pgfsetfillcolor{textcolor}%
\pgftext[x=0.118966in,y=0.724759in,left,base]{\color{textcolor}\rmfamily\fontsize{8.000000}{9.600000}\selectfont \(\displaystyle 10^{-3}\)}%
\end{pgfscope}%
\begin{pgfscope}%
\pgfpathrectangle{\pgfqpoint{0.423750in}{0.261892in}}{\pgfqpoint{2.627250in}{1.581827in}}%
\pgfusepath{clip}%
\pgfsetroundcap%
\pgfsetroundjoin%
\pgfsetlinewidth{0.803000pt}%
\definecolor{currentstroke}{rgb}{1.000000,1.000000,1.000000}%
\pgfsetstrokecolor{currentstroke}%
\pgfsetdash{}{0pt}%
\pgfpathmoveto{\pgfqpoint{0.423750in}{1.101779in}}%
\pgfpathlineto{\pgfqpoint{3.051000in}{1.101779in}}%
\pgfusepath{stroke}%
\end{pgfscope}%
\begin{pgfscope}%
\definecolor{textcolor}{rgb}{0.150000,0.150000,0.150000}%
\pgfsetstrokecolor{textcolor}%
\pgfsetfillcolor{textcolor}%
\pgftext[x=0.118966in,y=1.059570in,left,base]{\color{textcolor}\rmfamily\fontsize{8.000000}{9.600000}\selectfont \(\displaystyle 10^{-2}\)}%
\end{pgfscope}%
\begin{pgfscope}%
\pgfpathrectangle{\pgfqpoint{0.423750in}{0.261892in}}{\pgfqpoint{2.627250in}{1.581827in}}%
\pgfusepath{clip}%
\pgfsetroundcap%
\pgfsetroundjoin%
\pgfsetlinewidth{0.803000pt}%
\definecolor{currentstroke}{rgb}{1.000000,1.000000,1.000000}%
\pgfsetstrokecolor{currentstroke}%
\pgfsetdash{}{0pt}%
\pgfpathmoveto{\pgfqpoint{0.423750in}{1.436589in}}%
\pgfpathlineto{\pgfqpoint{3.051000in}{1.436589in}}%
\pgfusepath{stroke}%
\end{pgfscope}%
\begin{pgfscope}%
\definecolor{textcolor}{rgb}{0.150000,0.150000,0.150000}%
\pgfsetstrokecolor{textcolor}%
\pgfsetfillcolor{textcolor}%
\pgftext[x=0.118966in,y=1.394380in,left,base]{\color{textcolor}\rmfamily\fontsize{8.000000}{9.600000}\selectfont \(\displaystyle 10^{-1}\)}%
\end{pgfscope}%
\begin{pgfscope}%
\pgfpathrectangle{\pgfqpoint{0.423750in}{0.261892in}}{\pgfqpoint{2.627250in}{1.581827in}}%
\pgfusepath{clip}%
\pgfsetroundcap%
\pgfsetroundjoin%
\pgfsetlinewidth{0.803000pt}%
\definecolor{currentstroke}{rgb}{1.000000,1.000000,1.000000}%
\pgfsetstrokecolor{currentstroke}%
\pgfsetdash{}{0pt}%
\pgfpathmoveto{\pgfqpoint{0.423750in}{1.771400in}}%
\pgfpathlineto{\pgfqpoint{3.051000in}{1.771400in}}%
\pgfusepath{stroke}%
\end{pgfscope}%
\begin{pgfscope}%
\definecolor{textcolor}{rgb}{0.150000,0.150000,0.150000}%
\pgfsetstrokecolor{textcolor}%
\pgfsetfillcolor{textcolor}%
\pgftext[x=0.199212in,y=1.729191in,left,base]{\color{textcolor}\rmfamily\fontsize{8.000000}{9.600000}\selectfont \(\displaystyle 10^{0}\)}%
\end{pgfscope}%
\begin{pgfscope}%
\definecolor{textcolor}{rgb}{0.150000,0.150000,0.150000}%
\pgfsetstrokecolor{textcolor}%
\pgfsetfillcolor{textcolor}%
\pgftext[x=0.063410in,y=1.052805in,,bottom,rotate=90.000000]{\color{textcolor}\rmfamily\fontsize{8.000000}{9.600000}\selectfont Simple Regret}%
\end{pgfscope}%
\begin{pgfscope}%
\pgfpathrectangle{\pgfqpoint{0.423750in}{0.261892in}}{\pgfqpoint{2.627250in}{1.581827in}}%
\pgfusepath{clip}%
\pgfsetbuttcap%
\pgfsetroundjoin%
\definecolor{currentfill}{rgb}{0.121569,0.466667,0.705882}%
\pgfsetfillcolor{currentfill}%
\pgfsetfillopacity{0.200000}%
\pgfsetlinewidth{0.000000pt}%
\definecolor{currentstroke}{rgb}{0.000000,0.000000,0.000000}%
\pgfsetstrokecolor{currentstroke}%
\pgfsetdash{}{0pt}%
\pgfpathmoveto{\pgfqpoint{0.543170in}{1.764513in}}%
\pgfpathlineto{\pgfqpoint{0.543170in}{1.771818in}}%
\pgfpathlineto{\pgfqpoint{0.554076in}{1.769973in}}%
\pgfpathlineto{\pgfqpoint{0.564982in}{1.765852in}}%
\pgfpathlineto{\pgfqpoint{0.575888in}{1.765852in}}%
\pgfpathlineto{\pgfqpoint{0.586794in}{1.765771in}}%
\pgfpathlineto{\pgfqpoint{0.597700in}{1.765730in}}%
\pgfpathlineto{\pgfqpoint{0.608606in}{1.764950in}}%
\pgfpathlineto{\pgfqpoint{0.619512in}{1.764736in}}%
\pgfpathlineto{\pgfqpoint{0.630418in}{1.764736in}}%
\pgfpathlineto{\pgfqpoint{0.641324in}{1.766198in}}%
\pgfpathlineto{\pgfqpoint{0.652230in}{1.766198in}}%
\pgfpathlineto{\pgfqpoint{0.663136in}{1.765666in}}%
\pgfpathlineto{\pgfqpoint{0.674042in}{1.765666in}}%
\pgfpathlineto{\pgfqpoint{0.684948in}{1.765666in}}%
\pgfpathlineto{\pgfqpoint{0.695854in}{1.765666in}}%
\pgfpathlineto{\pgfqpoint{0.706760in}{1.763747in}}%
\pgfpathlineto{\pgfqpoint{0.717666in}{1.755655in}}%
\pgfpathlineto{\pgfqpoint{0.728572in}{1.750865in}}%
\pgfpathlineto{\pgfqpoint{0.739478in}{1.735729in}}%
\pgfpathlineto{\pgfqpoint{0.750384in}{1.735729in}}%
\pgfpathlineto{\pgfqpoint{0.761290in}{1.735729in}}%
\pgfpathlineto{\pgfqpoint{0.772196in}{1.726021in}}%
\pgfpathlineto{\pgfqpoint{0.783102in}{1.725826in}}%
\pgfpathlineto{\pgfqpoint{0.794008in}{1.716537in}}%
\pgfpathlineto{\pgfqpoint{0.804914in}{1.716165in}}%
\pgfpathlineto{\pgfqpoint{0.815820in}{1.708361in}}%
\pgfpathlineto{\pgfqpoint{0.826726in}{1.708361in}}%
\pgfpathlineto{\pgfqpoint{0.837632in}{1.701679in}}%
\pgfpathlineto{\pgfqpoint{0.848538in}{1.701679in}}%
\pgfpathlineto{\pgfqpoint{0.859444in}{1.698524in}}%
\pgfpathlineto{\pgfqpoint{0.870350in}{1.698524in}}%
\pgfpathlineto{\pgfqpoint{0.881256in}{1.698524in}}%
\pgfpathlineto{\pgfqpoint{0.892162in}{1.698524in}}%
\pgfpathlineto{\pgfqpoint{0.903068in}{1.698524in}}%
\pgfpathlineto{\pgfqpoint{0.913974in}{1.698524in}}%
\pgfpathlineto{\pgfqpoint{0.924880in}{1.698524in}}%
\pgfpathlineto{\pgfqpoint{0.935786in}{1.692914in}}%
\pgfpathlineto{\pgfqpoint{0.946692in}{1.692914in}}%
\pgfpathlineto{\pgfqpoint{0.957598in}{1.692914in}}%
\pgfpathlineto{\pgfqpoint{0.968504in}{1.692914in}}%
\pgfpathlineto{\pgfqpoint{0.979410in}{1.692914in}}%
\pgfpathlineto{\pgfqpoint{0.990316in}{1.692914in}}%
\pgfpathlineto{\pgfqpoint{1.001222in}{1.692244in}}%
\pgfpathlineto{\pgfqpoint{1.012127in}{1.692244in}}%
\pgfpathlineto{\pgfqpoint{1.023033in}{1.692244in}}%
\pgfpathlineto{\pgfqpoint{1.033939in}{1.692244in}}%
\pgfpathlineto{\pgfqpoint{1.044845in}{1.692244in}}%
\pgfpathlineto{\pgfqpoint{1.055751in}{1.692244in}}%
\pgfpathlineto{\pgfqpoint{1.066657in}{1.692244in}}%
\pgfpathlineto{\pgfqpoint{1.077563in}{1.692244in}}%
\pgfpathlineto{\pgfqpoint{1.088469in}{1.692244in}}%
\pgfpathlineto{\pgfqpoint{1.099375in}{1.678237in}}%
\pgfpathlineto{\pgfqpoint{1.110281in}{1.678237in}}%
\pgfpathlineto{\pgfqpoint{1.121187in}{1.678237in}}%
\pgfpathlineto{\pgfqpoint{1.132093in}{1.678237in}}%
\pgfpathlineto{\pgfqpoint{1.142999in}{1.678237in}}%
\pgfpathlineto{\pgfqpoint{1.153905in}{1.678237in}}%
\pgfpathlineto{\pgfqpoint{1.164811in}{1.678237in}}%
\pgfpathlineto{\pgfqpoint{1.175717in}{1.677531in}}%
\pgfpathlineto{\pgfqpoint{1.186623in}{1.677531in}}%
\pgfpathlineto{\pgfqpoint{1.197529in}{1.677531in}}%
\pgfpathlineto{\pgfqpoint{1.208435in}{1.677531in}}%
\pgfpathlineto{\pgfqpoint{1.219341in}{1.677531in}}%
\pgfpathlineto{\pgfqpoint{1.230247in}{1.677531in}}%
\pgfpathlineto{\pgfqpoint{1.241153in}{1.677531in}}%
\pgfpathlineto{\pgfqpoint{1.252059in}{1.677531in}}%
\pgfpathlineto{\pgfqpoint{1.262965in}{1.677531in}}%
\pgfpathlineto{\pgfqpoint{1.273871in}{1.677531in}}%
\pgfpathlineto{\pgfqpoint{1.284777in}{1.677531in}}%
\pgfpathlineto{\pgfqpoint{1.295683in}{1.677531in}}%
\pgfpathlineto{\pgfqpoint{1.306589in}{1.677531in}}%
\pgfpathlineto{\pgfqpoint{1.317495in}{1.677531in}}%
\pgfpathlineto{\pgfqpoint{1.328401in}{1.677531in}}%
\pgfpathlineto{\pgfqpoint{1.339307in}{1.677531in}}%
\pgfpathlineto{\pgfqpoint{1.350213in}{1.677531in}}%
\pgfpathlineto{\pgfqpoint{1.361119in}{1.677531in}}%
\pgfpathlineto{\pgfqpoint{1.372025in}{1.677531in}}%
\pgfpathlineto{\pgfqpoint{1.382931in}{1.677531in}}%
\pgfpathlineto{\pgfqpoint{1.393837in}{1.677531in}}%
\pgfpathlineto{\pgfqpoint{1.404743in}{1.677531in}}%
\pgfpathlineto{\pgfqpoint{1.415649in}{1.677531in}}%
\pgfpathlineto{\pgfqpoint{1.426555in}{1.677531in}}%
\pgfpathlineto{\pgfqpoint{1.437461in}{1.677531in}}%
\pgfpathlineto{\pgfqpoint{1.448367in}{1.677531in}}%
\pgfpathlineto{\pgfqpoint{1.459273in}{1.672842in}}%
\pgfpathlineto{\pgfqpoint{1.470179in}{1.672842in}}%
\pgfpathlineto{\pgfqpoint{1.481085in}{1.672842in}}%
\pgfpathlineto{\pgfqpoint{1.491991in}{1.672842in}}%
\pgfpathlineto{\pgfqpoint{1.502896in}{1.672842in}}%
\pgfpathlineto{\pgfqpoint{1.513802in}{1.672842in}}%
\pgfpathlineto{\pgfqpoint{1.524708in}{1.672785in}}%
\pgfpathlineto{\pgfqpoint{1.535614in}{1.672785in}}%
\pgfpathlineto{\pgfqpoint{1.546520in}{1.672785in}}%
\pgfpathlineto{\pgfqpoint{1.557426in}{1.672785in}}%
\pgfpathlineto{\pgfqpoint{1.568332in}{1.672785in}}%
\pgfpathlineto{\pgfqpoint{1.579238in}{1.672785in}}%
\pgfpathlineto{\pgfqpoint{1.590144in}{1.672785in}}%
\pgfpathlineto{\pgfqpoint{1.601050in}{1.672785in}}%
\pgfpathlineto{\pgfqpoint{1.611956in}{1.672785in}}%
\pgfpathlineto{\pgfqpoint{1.622862in}{1.672785in}}%
\pgfpathlineto{\pgfqpoint{1.633768in}{1.672785in}}%
\pgfpathlineto{\pgfqpoint{1.644674in}{1.672785in}}%
\pgfpathlineto{\pgfqpoint{1.655580in}{1.668485in}}%
\pgfpathlineto{\pgfqpoint{1.666486in}{1.668485in}}%
\pgfpathlineto{\pgfqpoint{1.677392in}{1.668485in}}%
\pgfpathlineto{\pgfqpoint{1.688298in}{1.668485in}}%
\pgfpathlineto{\pgfqpoint{1.699204in}{1.668485in}}%
\pgfpathlineto{\pgfqpoint{1.710110in}{1.668485in}}%
\pgfpathlineto{\pgfqpoint{1.721016in}{1.657085in}}%
\pgfpathlineto{\pgfqpoint{1.731922in}{1.657085in}}%
\pgfpathlineto{\pgfqpoint{1.742828in}{1.657085in}}%
\pgfpathlineto{\pgfqpoint{1.753734in}{1.657085in}}%
\pgfpathlineto{\pgfqpoint{1.764640in}{1.657085in}}%
\pgfpathlineto{\pgfqpoint{1.775546in}{1.645315in}}%
\pgfpathlineto{\pgfqpoint{1.786452in}{1.645315in}}%
\pgfpathlineto{\pgfqpoint{1.797358in}{1.645315in}}%
\pgfpathlineto{\pgfqpoint{1.808264in}{1.645315in}}%
\pgfpathlineto{\pgfqpoint{1.819170in}{1.645315in}}%
\pgfpathlineto{\pgfqpoint{1.830076in}{1.645315in}}%
\pgfpathlineto{\pgfqpoint{1.840982in}{1.645315in}}%
\pgfpathlineto{\pgfqpoint{1.851888in}{1.645315in}}%
\pgfpathlineto{\pgfqpoint{1.862794in}{1.645315in}}%
\pgfpathlineto{\pgfqpoint{1.873700in}{1.617936in}}%
\pgfpathlineto{\pgfqpoint{1.884606in}{1.617936in}}%
\pgfpathlineto{\pgfqpoint{1.895512in}{1.617936in}}%
\pgfpathlineto{\pgfqpoint{1.906418in}{1.617936in}}%
\pgfpathlineto{\pgfqpoint{1.917324in}{1.617936in}}%
\pgfpathlineto{\pgfqpoint{1.928230in}{1.617936in}}%
\pgfpathlineto{\pgfqpoint{1.939136in}{1.617936in}}%
\pgfpathlineto{\pgfqpoint{1.950042in}{1.617936in}}%
\pgfpathlineto{\pgfqpoint{1.960948in}{1.617936in}}%
\pgfpathlineto{\pgfqpoint{1.971854in}{1.617936in}}%
\pgfpathlineto{\pgfqpoint{1.982759in}{1.617936in}}%
\pgfpathlineto{\pgfqpoint{1.993665in}{1.617936in}}%
\pgfpathlineto{\pgfqpoint{2.004571in}{1.617936in}}%
\pgfpathlineto{\pgfqpoint{2.015477in}{1.617936in}}%
\pgfpathlineto{\pgfqpoint{2.026383in}{1.617936in}}%
\pgfpathlineto{\pgfqpoint{2.037289in}{1.617936in}}%
\pgfpathlineto{\pgfqpoint{2.048195in}{1.617936in}}%
\pgfpathlineto{\pgfqpoint{2.059101in}{1.617936in}}%
\pgfpathlineto{\pgfqpoint{2.070007in}{1.617936in}}%
\pgfpathlineto{\pgfqpoint{2.080913in}{1.617936in}}%
\pgfpathlineto{\pgfqpoint{2.091819in}{1.617936in}}%
\pgfpathlineto{\pgfqpoint{2.102725in}{1.617936in}}%
\pgfpathlineto{\pgfqpoint{2.113631in}{1.617936in}}%
\pgfpathlineto{\pgfqpoint{2.124537in}{1.617936in}}%
\pgfpathlineto{\pgfqpoint{2.135443in}{1.617936in}}%
\pgfpathlineto{\pgfqpoint{2.146349in}{1.617936in}}%
\pgfpathlineto{\pgfqpoint{2.157255in}{1.617936in}}%
\pgfpathlineto{\pgfqpoint{2.168161in}{1.617936in}}%
\pgfpathlineto{\pgfqpoint{2.179067in}{1.617936in}}%
\pgfpathlineto{\pgfqpoint{2.189973in}{1.617936in}}%
\pgfpathlineto{\pgfqpoint{2.200879in}{1.617936in}}%
\pgfpathlineto{\pgfqpoint{2.211785in}{1.617936in}}%
\pgfpathlineto{\pgfqpoint{2.222691in}{1.617936in}}%
\pgfpathlineto{\pgfqpoint{2.233597in}{1.617936in}}%
\pgfpathlineto{\pgfqpoint{2.244503in}{1.617936in}}%
\pgfpathlineto{\pgfqpoint{2.255409in}{1.617936in}}%
\pgfpathlineto{\pgfqpoint{2.266315in}{1.617936in}}%
\pgfpathlineto{\pgfqpoint{2.277221in}{1.617936in}}%
\pgfpathlineto{\pgfqpoint{2.288127in}{1.617936in}}%
\pgfpathlineto{\pgfqpoint{2.299033in}{1.617936in}}%
\pgfpathlineto{\pgfqpoint{2.309939in}{1.617936in}}%
\pgfpathlineto{\pgfqpoint{2.320845in}{1.617936in}}%
\pgfpathlineto{\pgfqpoint{2.331751in}{1.617936in}}%
\pgfpathlineto{\pgfqpoint{2.342657in}{1.617936in}}%
\pgfpathlineto{\pgfqpoint{2.353563in}{1.617936in}}%
\pgfpathlineto{\pgfqpoint{2.364469in}{1.617936in}}%
\pgfpathlineto{\pgfqpoint{2.375375in}{1.617936in}}%
\pgfpathlineto{\pgfqpoint{2.386281in}{1.617936in}}%
\pgfpathlineto{\pgfqpoint{2.397187in}{1.617936in}}%
\pgfpathlineto{\pgfqpoint{2.408093in}{1.617936in}}%
\pgfpathlineto{\pgfqpoint{2.418999in}{1.617936in}}%
\pgfpathlineto{\pgfqpoint{2.429905in}{1.617936in}}%
\pgfpathlineto{\pgfqpoint{2.440811in}{1.617936in}}%
\pgfpathlineto{\pgfqpoint{2.451717in}{1.617936in}}%
\pgfpathlineto{\pgfqpoint{2.462623in}{1.617936in}}%
\pgfpathlineto{\pgfqpoint{2.473528in}{1.601477in}}%
\pgfpathlineto{\pgfqpoint{2.484434in}{1.601477in}}%
\pgfpathlineto{\pgfqpoint{2.495340in}{1.601477in}}%
\pgfpathlineto{\pgfqpoint{2.506246in}{1.600522in}}%
\pgfpathlineto{\pgfqpoint{2.517152in}{1.600522in}}%
\pgfpathlineto{\pgfqpoint{2.528058in}{1.600522in}}%
\pgfpathlineto{\pgfqpoint{2.538964in}{1.600522in}}%
\pgfpathlineto{\pgfqpoint{2.549870in}{1.569102in}}%
\pgfpathlineto{\pgfqpoint{2.560776in}{1.569102in}}%
\pgfpathlineto{\pgfqpoint{2.571682in}{1.569102in}}%
\pgfpathlineto{\pgfqpoint{2.582588in}{1.569102in}}%
\pgfpathlineto{\pgfqpoint{2.593494in}{1.569102in}}%
\pgfpathlineto{\pgfqpoint{2.604400in}{1.569102in}}%
\pgfpathlineto{\pgfqpoint{2.615306in}{1.569102in}}%
\pgfpathlineto{\pgfqpoint{2.626212in}{1.569102in}}%
\pgfpathlineto{\pgfqpoint{2.637118in}{1.569102in}}%
\pgfpathlineto{\pgfqpoint{2.648024in}{1.569102in}}%
\pgfpathlineto{\pgfqpoint{2.658930in}{1.569102in}}%
\pgfpathlineto{\pgfqpoint{2.669836in}{1.569102in}}%
\pgfpathlineto{\pgfqpoint{2.680742in}{1.569102in}}%
\pgfpathlineto{\pgfqpoint{2.691648in}{1.569102in}}%
\pgfpathlineto{\pgfqpoint{2.702554in}{1.569102in}}%
\pgfpathlineto{\pgfqpoint{2.713460in}{1.569102in}}%
\pgfpathlineto{\pgfqpoint{2.724366in}{1.568932in}}%
\pgfpathlineto{\pgfqpoint{2.735272in}{1.568932in}}%
\pgfpathlineto{\pgfqpoint{2.746178in}{1.568932in}}%
\pgfpathlineto{\pgfqpoint{2.757084in}{1.568932in}}%
\pgfpathlineto{\pgfqpoint{2.767990in}{1.568932in}}%
\pgfpathlineto{\pgfqpoint{2.778896in}{1.568932in}}%
\pgfpathlineto{\pgfqpoint{2.789802in}{1.568932in}}%
\pgfpathlineto{\pgfqpoint{2.800708in}{1.568932in}}%
\pgfpathlineto{\pgfqpoint{2.811614in}{1.568932in}}%
\pgfpathlineto{\pgfqpoint{2.822520in}{1.516447in}}%
\pgfpathlineto{\pgfqpoint{2.833426in}{1.516447in}}%
\pgfpathlineto{\pgfqpoint{2.844332in}{1.516447in}}%
\pgfpathlineto{\pgfqpoint{2.855238in}{1.516447in}}%
\pgfpathlineto{\pgfqpoint{2.866144in}{1.516447in}}%
\pgfpathlineto{\pgfqpoint{2.877050in}{1.516447in}}%
\pgfpathlineto{\pgfqpoint{2.887956in}{1.516447in}}%
\pgfpathlineto{\pgfqpoint{2.898862in}{1.516447in}}%
\pgfpathlineto{\pgfqpoint{2.909768in}{1.446351in}}%
\pgfpathlineto{\pgfqpoint{2.920674in}{1.446351in}}%
\pgfpathlineto{\pgfqpoint{2.931580in}{1.446351in}}%
\pgfpathlineto{\pgfqpoint{2.931580in}{1.140108in}}%
\pgfpathlineto{\pgfqpoint{2.931580in}{1.140108in}}%
\pgfpathlineto{\pgfqpoint{2.920674in}{1.140108in}}%
\pgfpathlineto{\pgfqpoint{2.909768in}{1.140108in}}%
\pgfpathlineto{\pgfqpoint{2.898862in}{1.235463in}}%
\pgfpathlineto{\pgfqpoint{2.887956in}{1.235463in}}%
\pgfpathlineto{\pgfqpoint{2.877050in}{1.235463in}}%
\pgfpathlineto{\pgfqpoint{2.866144in}{1.235463in}}%
\pgfpathlineto{\pgfqpoint{2.855238in}{1.235463in}}%
\pgfpathlineto{\pgfqpoint{2.844332in}{1.235463in}}%
\pgfpathlineto{\pgfqpoint{2.833426in}{1.235463in}}%
\pgfpathlineto{\pgfqpoint{2.822520in}{1.235463in}}%
\pgfpathlineto{\pgfqpoint{2.811614in}{1.222648in}}%
\pgfpathlineto{\pgfqpoint{2.800708in}{1.222648in}}%
\pgfpathlineto{\pgfqpoint{2.789802in}{1.222648in}}%
\pgfpathlineto{\pgfqpoint{2.778896in}{1.222648in}}%
\pgfpathlineto{\pgfqpoint{2.767990in}{1.222648in}}%
\pgfpathlineto{\pgfqpoint{2.757084in}{1.222648in}}%
\pgfpathlineto{\pgfqpoint{2.746178in}{1.222648in}}%
\pgfpathlineto{\pgfqpoint{2.735272in}{1.222648in}}%
\pgfpathlineto{\pgfqpoint{2.724366in}{1.222648in}}%
\pgfpathlineto{\pgfqpoint{2.713460in}{1.225994in}}%
\pgfpathlineto{\pgfqpoint{2.702554in}{1.225994in}}%
\pgfpathlineto{\pgfqpoint{2.691648in}{1.225994in}}%
\pgfpathlineto{\pgfqpoint{2.680742in}{1.225994in}}%
\pgfpathlineto{\pgfqpoint{2.669836in}{1.225994in}}%
\pgfpathlineto{\pgfqpoint{2.658930in}{1.225994in}}%
\pgfpathlineto{\pgfqpoint{2.648024in}{1.225994in}}%
\pgfpathlineto{\pgfqpoint{2.637118in}{1.225994in}}%
\pgfpathlineto{\pgfqpoint{2.626212in}{1.225994in}}%
\pgfpathlineto{\pgfqpoint{2.615306in}{1.225994in}}%
\pgfpathlineto{\pgfqpoint{2.604400in}{1.225994in}}%
\pgfpathlineto{\pgfqpoint{2.593494in}{1.225994in}}%
\pgfpathlineto{\pgfqpoint{2.582588in}{1.225994in}}%
\pgfpathlineto{\pgfqpoint{2.571682in}{1.225994in}}%
\pgfpathlineto{\pgfqpoint{2.560776in}{1.225994in}}%
\pgfpathlineto{\pgfqpoint{2.549870in}{1.225994in}}%
\pgfpathlineto{\pgfqpoint{2.538964in}{1.367311in}}%
\pgfpathlineto{\pgfqpoint{2.528058in}{1.367311in}}%
\pgfpathlineto{\pgfqpoint{2.517152in}{1.367311in}}%
\pgfpathlineto{\pgfqpoint{2.506246in}{1.367311in}}%
\pgfpathlineto{\pgfqpoint{2.495340in}{1.376785in}}%
\pgfpathlineto{\pgfqpoint{2.484434in}{1.376785in}}%
\pgfpathlineto{\pgfqpoint{2.473528in}{1.376785in}}%
\pgfpathlineto{\pgfqpoint{2.462623in}{1.372452in}}%
\pgfpathlineto{\pgfqpoint{2.451717in}{1.372452in}}%
\pgfpathlineto{\pgfqpoint{2.440811in}{1.372452in}}%
\pgfpathlineto{\pgfqpoint{2.429905in}{1.372452in}}%
\pgfpathlineto{\pgfqpoint{2.418999in}{1.372452in}}%
\pgfpathlineto{\pgfqpoint{2.408093in}{1.372452in}}%
\pgfpathlineto{\pgfqpoint{2.397187in}{1.372452in}}%
\pgfpathlineto{\pgfqpoint{2.386281in}{1.372452in}}%
\pgfpathlineto{\pgfqpoint{2.375375in}{1.372452in}}%
\pgfpathlineto{\pgfqpoint{2.364469in}{1.372452in}}%
\pgfpathlineto{\pgfqpoint{2.353563in}{1.372452in}}%
\pgfpathlineto{\pgfqpoint{2.342657in}{1.372452in}}%
\pgfpathlineto{\pgfqpoint{2.331751in}{1.372452in}}%
\pgfpathlineto{\pgfqpoint{2.320845in}{1.372452in}}%
\pgfpathlineto{\pgfqpoint{2.309939in}{1.372452in}}%
\pgfpathlineto{\pgfqpoint{2.299033in}{1.372452in}}%
\pgfpathlineto{\pgfqpoint{2.288127in}{1.372452in}}%
\pgfpathlineto{\pgfqpoint{2.277221in}{1.372452in}}%
\pgfpathlineto{\pgfqpoint{2.266315in}{1.372452in}}%
\pgfpathlineto{\pgfqpoint{2.255409in}{1.372452in}}%
\pgfpathlineto{\pgfqpoint{2.244503in}{1.372452in}}%
\pgfpathlineto{\pgfqpoint{2.233597in}{1.372452in}}%
\pgfpathlineto{\pgfqpoint{2.222691in}{1.372452in}}%
\pgfpathlineto{\pgfqpoint{2.211785in}{1.372452in}}%
\pgfpathlineto{\pgfqpoint{2.200879in}{1.372452in}}%
\pgfpathlineto{\pgfqpoint{2.189973in}{1.372452in}}%
\pgfpathlineto{\pgfqpoint{2.179067in}{1.372452in}}%
\pgfpathlineto{\pgfqpoint{2.168161in}{1.372452in}}%
\pgfpathlineto{\pgfqpoint{2.157255in}{1.372452in}}%
\pgfpathlineto{\pgfqpoint{2.146349in}{1.372452in}}%
\pgfpathlineto{\pgfqpoint{2.135443in}{1.372452in}}%
\pgfpathlineto{\pgfqpoint{2.124537in}{1.372452in}}%
\pgfpathlineto{\pgfqpoint{2.113631in}{1.372452in}}%
\pgfpathlineto{\pgfqpoint{2.102725in}{1.372452in}}%
\pgfpathlineto{\pgfqpoint{2.091819in}{1.372452in}}%
\pgfpathlineto{\pgfqpoint{2.080913in}{1.372452in}}%
\pgfpathlineto{\pgfqpoint{2.070007in}{1.372452in}}%
\pgfpathlineto{\pgfqpoint{2.059101in}{1.372452in}}%
\pgfpathlineto{\pgfqpoint{2.048195in}{1.372452in}}%
\pgfpathlineto{\pgfqpoint{2.037289in}{1.372452in}}%
\pgfpathlineto{\pgfqpoint{2.026383in}{1.372452in}}%
\pgfpathlineto{\pgfqpoint{2.015477in}{1.372452in}}%
\pgfpathlineto{\pgfqpoint{2.004571in}{1.372452in}}%
\pgfpathlineto{\pgfqpoint{1.993665in}{1.372452in}}%
\pgfpathlineto{\pgfqpoint{1.982759in}{1.372452in}}%
\pgfpathlineto{\pgfqpoint{1.971854in}{1.372452in}}%
\pgfpathlineto{\pgfqpoint{1.960948in}{1.372452in}}%
\pgfpathlineto{\pgfqpoint{1.950042in}{1.372452in}}%
\pgfpathlineto{\pgfqpoint{1.939136in}{1.372452in}}%
\pgfpathlineto{\pgfqpoint{1.928230in}{1.372452in}}%
\pgfpathlineto{\pgfqpoint{1.917324in}{1.372452in}}%
\pgfpathlineto{\pgfqpoint{1.906418in}{1.372452in}}%
\pgfpathlineto{\pgfqpoint{1.895512in}{1.372452in}}%
\pgfpathlineto{\pgfqpoint{1.884606in}{1.372452in}}%
\pgfpathlineto{\pgfqpoint{1.873700in}{1.372452in}}%
\pgfpathlineto{\pgfqpoint{1.862794in}{1.461963in}}%
\pgfpathlineto{\pgfqpoint{1.851888in}{1.461963in}}%
\pgfpathlineto{\pgfqpoint{1.840982in}{1.461963in}}%
\pgfpathlineto{\pgfqpoint{1.830076in}{1.461963in}}%
\pgfpathlineto{\pgfqpoint{1.819170in}{1.461963in}}%
\pgfpathlineto{\pgfqpoint{1.808264in}{1.461963in}}%
\pgfpathlineto{\pgfqpoint{1.797358in}{1.461963in}}%
\pgfpathlineto{\pgfqpoint{1.786452in}{1.461963in}}%
\pgfpathlineto{\pgfqpoint{1.775546in}{1.461963in}}%
\pgfpathlineto{\pgfqpoint{1.764640in}{1.503018in}}%
\pgfpathlineto{\pgfqpoint{1.753734in}{1.503018in}}%
\pgfpathlineto{\pgfqpoint{1.742828in}{1.503018in}}%
\pgfpathlineto{\pgfqpoint{1.731922in}{1.503018in}}%
\pgfpathlineto{\pgfqpoint{1.721016in}{1.503018in}}%
\pgfpathlineto{\pgfqpoint{1.710110in}{1.540172in}}%
\pgfpathlineto{\pgfqpoint{1.699204in}{1.540172in}}%
\pgfpathlineto{\pgfqpoint{1.688298in}{1.540172in}}%
\pgfpathlineto{\pgfqpoint{1.677392in}{1.540172in}}%
\pgfpathlineto{\pgfqpoint{1.666486in}{1.540172in}}%
\pgfpathlineto{\pgfqpoint{1.655580in}{1.540172in}}%
\pgfpathlineto{\pgfqpoint{1.644674in}{1.545062in}}%
\pgfpathlineto{\pgfqpoint{1.633768in}{1.545062in}}%
\pgfpathlineto{\pgfqpoint{1.622862in}{1.545062in}}%
\pgfpathlineto{\pgfqpoint{1.611956in}{1.545062in}}%
\pgfpathlineto{\pgfqpoint{1.601050in}{1.545062in}}%
\pgfpathlineto{\pgfqpoint{1.590144in}{1.545062in}}%
\pgfpathlineto{\pgfqpoint{1.579238in}{1.545062in}}%
\pgfpathlineto{\pgfqpoint{1.568332in}{1.545062in}}%
\pgfpathlineto{\pgfqpoint{1.557426in}{1.545062in}}%
\pgfpathlineto{\pgfqpoint{1.546520in}{1.545062in}}%
\pgfpathlineto{\pgfqpoint{1.535614in}{1.545062in}}%
\pgfpathlineto{\pgfqpoint{1.524708in}{1.545062in}}%
\pgfpathlineto{\pgfqpoint{1.513802in}{1.545154in}}%
\pgfpathlineto{\pgfqpoint{1.502896in}{1.545154in}}%
\pgfpathlineto{\pgfqpoint{1.491991in}{1.545154in}}%
\pgfpathlineto{\pgfqpoint{1.481085in}{1.545154in}}%
\pgfpathlineto{\pgfqpoint{1.470179in}{1.545154in}}%
\pgfpathlineto{\pgfqpoint{1.459273in}{1.545154in}}%
\pgfpathlineto{\pgfqpoint{1.448367in}{1.551349in}}%
\pgfpathlineto{\pgfqpoint{1.437461in}{1.551349in}}%
\pgfpathlineto{\pgfqpoint{1.426555in}{1.551349in}}%
\pgfpathlineto{\pgfqpoint{1.415649in}{1.551349in}}%
\pgfpathlineto{\pgfqpoint{1.404743in}{1.551349in}}%
\pgfpathlineto{\pgfqpoint{1.393837in}{1.551349in}}%
\pgfpathlineto{\pgfqpoint{1.382931in}{1.551349in}}%
\pgfpathlineto{\pgfqpoint{1.372025in}{1.551349in}}%
\pgfpathlineto{\pgfqpoint{1.361119in}{1.551349in}}%
\pgfpathlineto{\pgfqpoint{1.350213in}{1.551349in}}%
\pgfpathlineto{\pgfqpoint{1.339307in}{1.551349in}}%
\pgfpathlineto{\pgfqpoint{1.328401in}{1.551349in}}%
\pgfpathlineto{\pgfqpoint{1.317495in}{1.551349in}}%
\pgfpathlineto{\pgfqpoint{1.306589in}{1.551349in}}%
\pgfpathlineto{\pgfqpoint{1.295683in}{1.551349in}}%
\pgfpathlineto{\pgfqpoint{1.284777in}{1.551349in}}%
\pgfpathlineto{\pgfqpoint{1.273871in}{1.551349in}}%
\pgfpathlineto{\pgfqpoint{1.262965in}{1.551349in}}%
\pgfpathlineto{\pgfqpoint{1.252059in}{1.551349in}}%
\pgfpathlineto{\pgfqpoint{1.241153in}{1.551349in}}%
\pgfpathlineto{\pgfqpoint{1.230247in}{1.551349in}}%
\pgfpathlineto{\pgfqpoint{1.219341in}{1.551349in}}%
\pgfpathlineto{\pgfqpoint{1.208435in}{1.551349in}}%
\pgfpathlineto{\pgfqpoint{1.197529in}{1.551349in}}%
\pgfpathlineto{\pgfqpoint{1.186623in}{1.551349in}}%
\pgfpathlineto{\pgfqpoint{1.175717in}{1.551349in}}%
\pgfpathlineto{\pgfqpoint{1.164811in}{1.552619in}}%
\pgfpathlineto{\pgfqpoint{1.153905in}{1.552619in}}%
\pgfpathlineto{\pgfqpoint{1.142999in}{1.552619in}}%
\pgfpathlineto{\pgfqpoint{1.132093in}{1.552619in}}%
\pgfpathlineto{\pgfqpoint{1.121187in}{1.552619in}}%
\pgfpathlineto{\pgfqpoint{1.110281in}{1.552619in}}%
\pgfpathlineto{\pgfqpoint{1.099375in}{1.552619in}}%
\pgfpathlineto{\pgfqpoint{1.088469in}{1.561442in}}%
\pgfpathlineto{\pgfqpoint{1.077563in}{1.561442in}}%
\pgfpathlineto{\pgfqpoint{1.066657in}{1.561442in}}%
\pgfpathlineto{\pgfqpoint{1.055751in}{1.561442in}}%
\pgfpathlineto{\pgfqpoint{1.044845in}{1.561442in}}%
\pgfpathlineto{\pgfqpoint{1.033939in}{1.561442in}}%
\pgfpathlineto{\pgfqpoint{1.023033in}{1.561442in}}%
\pgfpathlineto{\pgfqpoint{1.012127in}{1.561442in}}%
\pgfpathlineto{\pgfqpoint{1.001222in}{1.561442in}}%
\pgfpathlineto{\pgfqpoint{0.990316in}{1.561646in}}%
\pgfpathlineto{\pgfqpoint{0.979410in}{1.561646in}}%
\pgfpathlineto{\pgfqpoint{0.968504in}{1.561646in}}%
\pgfpathlineto{\pgfqpoint{0.957598in}{1.561646in}}%
\pgfpathlineto{\pgfqpoint{0.946692in}{1.561646in}}%
\pgfpathlineto{\pgfqpoint{0.935786in}{1.561646in}}%
\pgfpathlineto{\pgfqpoint{0.924880in}{1.567567in}}%
\pgfpathlineto{\pgfqpoint{0.913974in}{1.567567in}}%
\pgfpathlineto{\pgfqpoint{0.903068in}{1.567567in}}%
\pgfpathlineto{\pgfqpoint{0.892162in}{1.567567in}}%
\pgfpathlineto{\pgfqpoint{0.881256in}{1.567567in}}%
\pgfpathlineto{\pgfqpoint{0.870350in}{1.567567in}}%
\pgfpathlineto{\pgfqpoint{0.859444in}{1.567567in}}%
\pgfpathlineto{\pgfqpoint{0.848538in}{1.569163in}}%
\pgfpathlineto{\pgfqpoint{0.837632in}{1.569163in}}%
\pgfpathlineto{\pgfqpoint{0.826726in}{1.599323in}}%
\pgfpathlineto{\pgfqpoint{0.815820in}{1.599323in}}%
\pgfpathlineto{\pgfqpoint{0.804914in}{1.611420in}}%
\pgfpathlineto{\pgfqpoint{0.794008in}{1.611587in}}%
\pgfpathlineto{\pgfqpoint{0.783102in}{1.625405in}}%
\pgfpathlineto{\pgfqpoint{0.772196in}{1.625515in}}%
\pgfpathlineto{\pgfqpoint{0.761290in}{1.639419in}}%
\pgfpathlineto{\pgfqpoint{0.750384in}{1.639419in}}%
\pgfpathlineto{\pgfqpoint{0.739478in}{1.639419in}}%
\pgfpathlineto{\pgfqpoint{0.728572in}{1.674864in}}%
\pgfpathlineto{\pgfqpoint{0.717666in}{1.679912in}}%
\pgfpathlineto{\pgfqpoint{0.706760in}{1.711114in}}%
\pgfpathlineto{\pgfqpoint{0.695854in}{1.713357in}}%
\pgfpathlineto{\pgfqpoint{0.684948in}{1.713357in}}%
\pgfpathlineto{\pgfqpoint{0.674042in}{1.713357in}}%
\pgfpathlineto{\pgfqpoint{0.663136in}{1.713357in}}%
\pgfpathlineto{\pgfqpoint{0.652230in}{1.713783in}}%
\pgfpathlineto{\pgfqpoint{0.641324in}{1.713783in}}%
\pgfpathlineto{\pgfqpoint{0.630418in}{1.746697in}}%
\pgfpathlineto{\pgfqpoint{0.619512in}{1.746697in}}%
\pgfpathlineto{\pgfqpoint{0.608606in}{1.746798in}}%
\pgfpathlineto{\pgfqpoint{0.597700in}{1.747237in}}%
\pgfpathlineto{\pgfqpoint{0.586794in}{1.747256in}}%
\pgfpathlineto{\pgfqpoint{0.575888in}{1.747295in}}%
\pgfpathlineto{\pgfqpoint{0.564982in}{1.747295in}}%
\pgfpathlineto{\pgfqpoint{0.554076in}{1.752736in}}%
\pgfpathlineto{\pgfqpoint{0.543170in}{1.764513in}}%
\pgfpathclose%
\pgfusepath{fill}%
\end{pgfscope}%
\begin{pgfscope}%
\pgfpathrectangle{\pgfqpoint{0.423750in}{0.261892in}}{\pgfqpoint{2.627250in}{1.581827in}}%
\pgfusepath{clip}%
\pgfsetbuttcap%
\pgfsetroundjoin%
\definecolor{currentfill}{rgb}{1.000000,0.498039,0.054902}%
\pgfsetfillcolor{currentfill}%
\pgfsetfillopacity{0.200000}%
\pgfsetlinewidth{0.000000pt}%
\definecolor{currentstroke}{rgb}{0.000000,0.000000,0.000000}%
\pgfsetstrokecolor{currentstroke}%
\pgfsetdash{}{0pt}%
\pgfpathmoveto{\pgfqpoint{0.543170in}{1.739381in}}%
\pgfpathlineto{\pgfqpoint{0.543170in}{1.771159in}}%
\pgfpathlineto{\pgfqpoint{0.554076in}{1.771087in}}%
\pgfpathlineto{\pgfqpoint{0.564982in}{1.770778in}}%
\pgfpathlineto{\pgfqpoint{0.575888in}{1.770778in}}%
\pgfpathlineto{\pgfqpoint{0.586794in}{1.758733in}}%
\pgfpathlineto{\pgfqpoint{0.597700in}{1.758272in}}%
\pgfpathlineto{\pgfqpoint{0.608606in}{1.736419in}}%
\pgfpathlineto{\pgfqpoint{0.619512in}{1.732852in}}%
\pgfpathlineto{\pgfqpoint{0.630418in}{1.732629in}}%
\pgfpathlineto{\pgfqpoint{0.641324in}{1.732629in}}%
\pgfpathlineto{\pgfqpoint{0.652230in}{1.706115in}}%
\pgfpathlineto{\pgfqpoint{0.663136in}{1.706115in}}%
\pgfpathlineto{\pgfqpoint{0.674042in}{1.706115in}}%
\pgfpathlineto{\pgfqpoint{0.684948in}{1.706115in}}%
\pgfpathlineto{\pgfqpoint{0.695854in}{1.706115in}}%
\pgfpathlineto{\pgfqpoint{0.706760in}{1.706115in}}%
\pgfpathlineto{\pgfqpoint{0.717666in}{1.706115in}}%
\pgfpathlineto{\pgfqpoint{0.728572in}{1.706115in}}%
\pgfpathlineto{\pgfqpoint{0.739478in}{1.706115in}}%
\pgfpathlineto{\pgfqpoint{0.750384in}{1.683792in}}%
\pgfpathlineto{\pgfqpoint{0.761290in}{1.683792in}}%
\pgfpathlineto{\pgfqpoint{0.772196in}{1.683792in}}%
\pgfpathlineto{\pgfqpoint{0.783102in}{1.683792in}}%
\pgfpathlineto{\pgfqpoint{0.794008in}{1.683792in}}%
\pgfpathlineto{\pgfqpoint{0.804914in}{1.680695in}}%
\pgfpathlineto{\pgfqpoint{0.815820in}{1.680695in}}%
\pgfpathlineto{\pgfqpoint{0.826726in}{1.680695in}}%
\pgfpathlineto{\pgfqpoint{0.837632in}{1.680695in}}%
\pgfpathlineto{\pgfqpoint{0.848538in}{1.680695in}}%
\pgfpathlineto{\pgfqpoint{0.859444in}{1.680695in}}%
\pgfpathlineto{\pgfqpoint{0.870350in}{1.677802in}}%
\pgfpathlineto{\pgfqpoint{0.881256in}{1.650519in}}%
\pgfpathlineto{\pgfqpoint{0.892162in}{1.650519in}}%
\pgfpathlineto{\pgfqpoint{0.903068in}{1.643936in}}%
\pgfpathlineto{\pgfqpoint{0.913974in}{1.609340in}}%
\pgfpathlineto{\pgfqpoint{0.924880in}{1.609340in}}%
\pgfpathlineto{\pgfqpoint{0.935786in}{1.609340in}}%
\pgfpathlineto{\pgfqpoint{0.946692in}{1.609340in}}%
\pgfpathlineto{\pgfqpoint{0.957598in}{1.609340in}}%
\pgfpathlineto{\pgfqpoint{0.968504in}{1.609340in}}%
\pgfpathlineto{\pgfqpoint{0.979410in}{1.609340in}}%
\pgfpathlineto{\pgfqpoint{0.990316in}{1.609340in}}%
\pgfpathlineto{\pgfqpoint{1.001222in}{1.609340in}}%
\pgfpathlineto{\pgfqpoint{1.012127in}{1.609340in}}%
\pgfpathlineto{\pgfqpoint{1.023033in}{1.609340in}}%
\pgfpathlineto{\pgfqpoint{1.033939in}{1.609340in}}%
\pgfpathlineto{\pgfqpoint{1.044845in}{1.592492in}}%
\pgfpathlineto{\pgfqpoint{1.055751in}{1.592492in}}%
\pgfpathlineto{\pgfqpoint{1.066657in}{1.583271in}}%
\pgfpathlineto{\pgfqpoint{1.077563in}{1.537932in}}%
\pgfpathlineto{\pgfqpoint{1.088469in}{1.537932in}}%
\pgfpathlineto{\pgfqpoint{1.099375in}{1.537932in}}%
\pgfpathlineto{\pgfqpoint{1.110281in}{1.537932in}}%
\pgfpathlineto{\pgfqpoint{1.121187in}{1.537932in}}%
\pgfpathlineto{\pgfqpoint{1.132093in}{1.533637in}}%
\pgfpathlineto{\pgfqpoint{1.142999in}{1.533637in}}%
\pgfpathlineto{\pgfqpoint{1.153905in}{1.533637in}}%
\pgfpathlineto{\pgfqpoint{1.164811in}{1.289490in}}%
\pgfpathlineto{\pgfqpoint{1.175717in}{1.289490in}}%
\pgfpathlineto{\pgfqpoint{1.186623in}{1.289490in}}%
\pgfpathlineto{\pgfqpoint{1.197529in}{1.289490in}}%
\pgfpathlineto{\pgfqpoint{1.208435in}{1.289490in}}%
\pgfpathlineto{\pgfqpoint{1.219341in}{1.289490in}}%
\pgfpathlineto{\pgfqpoint{1.230247in}{1.289490in}}%
\pgfpathlineto{\pgfqpoint{1.241153in}{1.214856in}}%
\pgfpathlineto{\pgfqpoint{1.252059in}{1.214856in}}%
\pgfpathlineto{\pgfqpoint{1.262965in}{1.214856in}}%
\pgfpathlineto{\pgfqpoint{1.273871in}{1.214856in}}%
\pgfpathlineto{\pgfqpoint{1.284777in}{1.214856in}}%
\pgfpathlineto{\pgfqpoint{1.295683in}{1.214856in}}%
\pgfpathlineto{\pgfqpoint{1.306589in}{1.214856in}}%
\pgfpathlineto{\pgfqpoint{1.317495in}{1.214856in}}%
\pgfpathlineto{\pgfqpoint{1.328401in}{1.214856in}}%
\pgfpathlineto{\pgfqpoint{1.339307in}{1.214856in}}%
\pgfpathlineto{\pgfqpoint{1.350213in}{1.214856in}}%
\pgfpathlineto{\pgfqpoint{1.361119in}{1.214856in}}%
\pgfpathlineto{\pgfqpoint{1.372025in}{1.214856in}}%
\pgfpathlineto{\pgfqpoint{1.382931in}{1.214856in}}%
\pgfpathlineto{\pgfqpoint{1.393837in}{0.862393in}}%
\pgfpathlineto{\pgfqpoint{1.404743in}{0.862393in}}%
\pgfpathlineto{\pgfqpoint{1.415649in}{0.861985in}}%
\pgfpathlineto{\pgfqpoint{1.426555in}{0.852207in}}%
\pgfpathlineto{\pgfqpoint{1.437461in}{0.852207in}}%
\pgfpathlineto{\pgfqpoint{1.448367in}{0.852207in}}%
\pgfpathlineto{\pgfqpoint{1.459273in}{0.852207in}}%
\pgfpathlineto{\pgfqpoint{1.470179in}{0.852207in}}%
\pgfpathlineto{\pgfqpoint{1.481085in}{0.852207in}}%
\pgfpathlineto{\pgfqpoint{1.491991in}{0.852207in}}%
\pgfpathlineto{\pgfqpoint{1.502896in}{0.852207in}}%
\pgfpathlineto{\pgfqpoint{1.513802in}{0.852207in}}%
\pgfpathlineto{\pgfqpoint{1.524708in}{0.852207in}}%
\pgfpathlineto{\pgfqpoint{1.535614in}{0.852207in}}%
\pgfpathlineto{\pgfqpoint{1.546520in}{0.852207in}}%
\pgfpathlineto{\pgfqpoint{1.557426in}{0.852207in}}%
\pgfpathlineto{\pgfqpoint{1.568332in}{0.852207in}}%
\pgfpathlineto{\pgfqpoint{1.579238in}{0.852207in}}%
\pgfpathlineto{\pgfqpoint{1.590144in}{0.852207in}}%
\pgfpathlineto{\pgfqpoint{1.601050in}{0.852207in}}%
\pgfpathlineto{\pgfqpoint{1.611956in}{0.852207in}}%
\pgfpathlineto{\pgfqpoint{1.622862in}{0.852207in}}%
\pgfpathlineto{\pgfqpoint{1.633768in}{0.852207in}}%
\pgfpathlineto{\pgfqpoint{1.644674in}{0.852207in}}%
\pgfpathlineto{\pgfqpoint{1.655580in}{0.852207in}}%
\pgfpathlineto{\pgfqpoint{1.666486in}{0.852207in}}%
\pgfpathlineto{\pgfqpoint{1.677392in}{0.852207in}}%
\pgfpathlineto{\pgfqpoint{1.688298in}{0.852207in}}%
\pgfpathlineto{\pgfqpoint{1.699204in}{0.852207in}}%
\pgfpathlineto{\pgfqpoint{1.710110in}{0.852207in}}%
\pgfpathlineto{\pgfqpoint{1.721016in}{0.852207in}}%
\pgfpathlineto{\pgfqpoint{1.731922in}{0.852207in}}%
\pgfpathlineto{\pgfqpoint{1.742828in}{0.852207in}}%
\pgfpathlineto{\pgfqpoint{1.753734in}{0.852207in}}%
\pgfpathlineto{\pgfqpoint{1.764640in}{0.852207in}}%
\pgfpathlineto{\pgfqpoint{1.775546in}{0.852207in}}%
\pgfpathlineto{\pgfqpoint{1.786452in}{0.852207in}}%
\pgfpathlineto{\pgfqpoint{1.797358in}{0.852207in}}%
\pgfpathlineto{\pgfqpoint{1.808264in}{0.852207in}}%
\pgfpathlineto{\pgfqpoint{1.819170in}{0.852207in}}%
\pgfpathlineto{\pgfqpoint{1.830076in}{0.852207in}}%
\pgfpathlineto{\pgfqpoint{1.840982in}{0.852207in}}%
\pgfpathlineto{\pgfqpoint{1.851888in}{0.852207in}}%
\pgfpathlineto{\pgfqpoint{1.862794in}{0.852207in}}%
\pgfpathlineto{\pgfqpoint{1.873700in}{0.852207in}}%
\pgfpathlineto{\pgfqpoint{1.884606in}{0.852207in}}%
\pgfpathlineto{\pgfqpoint{1.895512in}{0.852207in}}%
\pgfpathlineto{\pgfqpoint{1.906418in}{0.852207in}}%
\pgfpathlineto{\pgfqpoint{1.917324in}{0.852207in}}%
\pgfpathlineto{\pgfqpoint{1.928230in}{0.852207in}}%
\pgfpathlineto{\pgfqpoint{1.939136in}{0.844433in}}%
\pgfpathlineto{\pgfqpoint{1.950042in}{0.844433in}}%
\pgfpathlineto{\pgfqpoint{1.960948in}{0.844433in}}%
\pgfpathlineto{\pgfqpoint{1.971854in}{0.844433in}}%
\pgfpathlineto{\pgfqpoint{1.982759in}{0.844433in}}%
\pgfpathlineto{\pgfqpoint{1.993665in}{0.844433in}}%
\pgfpathlineto{\pgfqpoint{2.004571in}{0.844433in}}%
\pgfpathlineto{\pgfqpoint{2.015477in}{0.844433in}}%
\pgfpathlineto{\pgfqpoint{2.026383in}{0.844433in}}%
\pgfpathlineto{\pgfqpoint{2.037289in}{0.844433in}}%
\pgfpathlineto{\pgfqpoint{2.048195in}{0.844433in}}%
\pgfpathlineto{\pgfqpoint{2.059101in}{0.844433in}}%
\pgfpathlineto{\pgfqpoint{2.070007in}{0.844433in}}%
\pgfpathlineto{\pgfqpoint{2.080913in}{0.844433in}}%
\pgfpathlineto{\pgfqpoint{2.091819in}{0.844433in}}%
\pgfpathlineto{\pgfqpoint{2.102725in}{0.844433in}}%
\pgfpathlineto{\pgfqpoint{2.113631in}{0.844433in}}%
\pgfpathlineto{\pgfqpoint{2.124537in}{0.844433in}}%
\pgfpathlineto{\pgfqpoint{2.135443in}{0.844433in}}%
\pgfpathlineto{\pgfqpoint{2.146349in}{0.844433in}}%
\pgfpathlineto{\pgfqpoint{2.157255in}{0.844433in}}%
\pgfpathlineto{\pgfqpoint{2.168161in}{0.844433in}}%
\pgfpathlineto{\pgfqpoint{2.179067in}{0.844433in}}%
\pgfpathlineto{\pgfqpoint{2.189973in}{0.844433in}}%
\pgfpathlineto{\pgfqpoint{2.200879in}{0.844433in}}%
\pgfpathlineto{\pgfqpoint{2.211785in}{0.844433in}}%
\pgfpathlineto{\pgfqpoint{2.222691in}{0.844433in}}%
\pgfpathlineto{\pgfqpoint{2.233597in}{0.844433in}}%
\pgfpathlineto{\pgfqpoint{2.244503in}{0.844433in}}%
\pgfpathlineto{\pgfqpoint{2.255409in}{0.844433in}}%
\pgfpathlineto{\pgfqpoint{2.266315in}{0.844433in}}%
\pgfpathlineto{\pgfqpoint{2.277221in}{0.844433in}}%
\pgfpathlineto{\pgfqpoint{2.288127in}{0.844433in}}%
\pgfpathlineto{\pgfqpoint{2.299033in}{0.844433in}}%
\pgfpathlineto{\pgfqpoint{2.309939in}{0.844433in}}%
\pgfpathlineto{\pgfqpoint{2.320845in}{0.844433in}}%
\pgfpathlineto{\pgfqpoint{2.331751in}{0.844433in}}%
\pgfpathlineto{\pgfqpoint{2.342657in}{0.844433in}}%
\pgfpathlineto{\pgfqpoint{2.353563in}{0.844433in}}%
\pgfpathlineto{\pgfqpoint{2.364469in}{0.844433in}}%
\pgfpathlineto{\pgfqpoint{2.375375in}{0.844433in}}%
\pgfpathlineto{\pgfqpoint{2.386281in}{0.844433in}}%
\pgfpathlineto{\pgfqpoint{2.397187in}{0.844433in}}%
\pgfpathlineto{\pgfqpoint{2.408093in}{0.844433in}}%
\pgfpathlineto{\pgfqpoint{2.418999in}{0.844433in}}%
\pgfpathlineto{\pgfqpoint{2.429905in}{0.844433in}}%
\pgfpathlineto{\pgfqpoint{2.440811in}{0.844433in}}%
\pgfpathlineto{\pgfqpoint{2.451717in}{0.844433in}}%
\pgfpathlineto{\pgfqpoint{2.462623in}{0.844433in}}%
\pgfpathlineto{\pgfqpoint{2.473528in}{0.844433in}}%
\pgfpathlineto{\pgfqpoint{2.484434in}{0.844433in}}%
\pgfpathlineto{\pgfqpoint{2.495340in}{0.844433in}}%
\pgfpathlineto{\pgfqpoint{2.506246in}{0.844433in}}%
\pgfpathlineto{\pgfqpoint{2.517152in}{0.844433in}}%
\pgfpathlineto{\pgfqpoint{2.528058in}{0.799147in}}%
\pgfpathlineto{\pgfqpoint{2.538964in}{0.799147in}}%
\pgfpathlineto{\pgfqpoint{2.549870in}{0.799147in}}%
\pgfpathlineto{\pgfqpoint{2.560776in}{0.799147in}}%
\pgfpathlineto{\pgfqpoint{2.571682in}{0.799147in}}%
\pgfpathlineto{\pgfqpoint{2.582588in}{0.799147in}}%
\pgfpathlineto{\pgfqpoint{2.593494in}{0.799147in}}%
\pgfpathlineto{\pgfqpoint{2.604400in}{0.799147in}}%
\pgfpathlineto{\pgfqpoint{2.615306in}{0.799147in}}%
\pgfpathlineto{\pgfqpoint{2.626212in}{0.799147in}}%
\pgfpathlineto{\pgfqpoint{2.637118in}{0.799147in}}%
\pgfpathlineto{\pgfqpoint{2.648024in}{0.799147in}}%
\pgfpathlineto{\pgfqpoint{2.658930in}{0.799147in}}%
\pgfpathlineto{\pgfqpoint{2.669836in}{0.799147in}}%
\pgfpathlineto{\pgfqpoint{2.680742in}{0.799147in}}%
\pgfpathlineto{\pgfqpoint{2.691648in}{0.799147in}}%
\pgfpathlineto{\pgfqpoint{2.702554in}{0.799147in}}%
\pgfpathlineto{\pgfqpoint{2.713460in}{0.799147in}}%
\pgfpathlineto{\pgfqpoint{2.724366in}{0.799147in}}%
\pgfpathlineto{\pgfqpoint{2.735272in}{0.799147in}}%
\pgfpathlineto{\pgfqpoint{2.746178in}{0.799147in}}%
\pgfpathlineto{\pgfqpoint{2.757084in}{0.799147in}}%
\pgfpathlineto{\pgfqpoint{2.767990in}{0.799147in}}%
\pgfpathlineto{\pgfqpoint{2.778896in}{0.799147in}}%
\pgfpathlineto{\pgfqpoint{2.789802in}{0.799147in}}%
\pgfpathlineto{\pgfqpoint{2.800708in}{0.799147in}}%
\pgfpathlineto{\pgfqpoint{2.811614in}{0.799147in}}%
\pgfpathlineto{\pgfqpoint{2.822520in}{0.799147in}}%
\pgfpathlineto{\pgfqpoint{2.833426in}{0.799147in}}%
\pgfpathlineto{\pgfqpoint{2.844332in}{0.799147in}}%
\pgfpathlineto{\pgfqpoint{2.855238in}{0.799147in}}%
\pgfpathlineto{\pgfqpoint{2.866144in}{0.485474in}}%
\pgfpathlineto{\pgfqpoint{2.877050in}{0.485474in}}%
\pgfpathlineto{\pgfqpoint{2.887956in}{0.485474in}}%
\pgfpathlineto{\pgfqpoint{2.898862in}{0.485474in}}%
\pgfpathlineto{\pgfqpoint{2.909768in}{0.485474in}}%
\pgfpathlineto{\pgfqpoint{2.920674in}{0.485474in}}%
\pgfpathlineto{\pgfqpoint{2.931580in}{0.485474in}}%
\pgfpathlineto{\pgfqpoint{2.931580in}{0.333793in}}%
\pgfpathlineto{\pgfqpoint{2.931580in}{0.333793in}}%
\pgfpathlineto{\pgfqpoint{2.920674in}{0.333793in}}%
\pgfpathlineto{\pgfqpoint{2.909768in}{0.333793in}}%
\pgfpathlineto{\pgfqpoint{2.898862in}{0.333793in}}%
\pgfpathlineto{\pgfqpoint{2.887956in}{0.333793in}}%
\pgfpathlineto{\pgfqpoint{2.877050in}{0.333793in}}%
\pgfpathlineto{\pgfqpoint{2.866144in}{0.333793in}}%
\pgfpathlineto{\pgfqpoint{2.855238in}{0.507329in}}%
\pgfpathlineto{\pgfqpoint{2.844332in}{0.507329in}}%
\pgfpathlineto{\pgfqpoint{2.833426in}{0.507329in}}%
\pgfpathlineto{\pgfqpoint{2.822520in}{0.507329in}}%
\pgfpathlineto{\pgfqpoint{2.811614in}{0.507329in}}%
\pgfpathlineto{\pgfqpoint{2.800708in}{0.507329in}}%
\pgfpathlineto{\pgfqpoint{2.789802in}{0.507329in}}%
\pgfpathlineto{\pgfqpoint{2.778896in}{0.507329in}}%
\pgfpathlineto{\pgfqpoint{2.767990in}{0.507329in}}%
\pgfpathlineto{\pgfqpoint{2.757084in}{0.507329in}}%
\pgfpathlineto{\pgfqpoint{2.746178in}{0.507329in}}%
\pgfpathlineto{\pgfqpoint{2.735272in}{0.507329in}}%
\pgfpathlineto{\pgfqpoint{2.724366in}{0.507329in}}%
\pgfpathlineto{\pgfqpoint{2.713460in}{0.507329in}}%
\pgfpathlineto{\pgfqpoint{2.702554in}{0.507329in}}%
\pgfpathlineto{\pgfqpoint{2.691648in}{0.507329in}}%
\pgfpathlineto{\pgfqpoint{2.680742in}{0.507329in}}%
\pgfpathlineto{\pgfqpoint{2.669836in}{0.507329in}}%
\pgfpathlineto{\pgfqpoint{2.658930in}{0.507329in}}%
\pgfpathlineto{\pgfqpoint{2.648024in}{0.507329in}}%
\pgfpathlineto{\pgfqpoint{2.637118in}{0.507329in}}%
\pgfpathlineto{\pgfqpoint{2.626212in}{0.507329in}}%
\pgfpathlineto{\pgfqpoint{2.615306in}{0.507329in}}%
\pgfpathlineto{\pgfqpoint{2.604400in}{0.507329in}}%
\pgfpathlineto{\pgfqpoint{2.593494in}{0.507329in}}%
\pgfpathlineto{\pgfqpoint{2.582588in}{0.507329in}}%
\pgfpathlineto{\pgfqpoint{2.571682in}{0.507329in}}%
\pgfpathlineto{\pgfqpoint{2.560776in}{0.507329in}}%
\pgfpathlineto{\pgfqpoint{2.549870in}{0.507329in}}%
\pgfpathlineto{\pgfqpoint{2.538964in}{0.507329in}}%
\pgfpathlineto{\pgfqpoint{2.528058in}{0.507329in}}%
\pgfpathlineto{\pgfqpoint{2.517152in}{0.667204in}}%
\pgfpathlineto{\pgfqpoint{2.506246in}{0.667204in}}%
\pgfpathlineto{\pgfqpoint{2.495340in}{0.667204in}}%
\pgfpathlineto{\pgfqpoint{2.484434in}{0.667204in}}%
\pgfpathlineto{\pgfqpoint{2.473528in}{0.667204in}}%
\pgfpathlineto{\pgfqpoint{2.462623in}{0.667204in}}%
\pgfpathlineto{\pgfqpoint{2.451717in}{0.667204in}}%
\pgfpathlineto{\pgfqpoint{2.440811in}{0.667204in}}%
\pgfpathlineto{\pgfqpoint{2.429905in}{0.667204in}}%
\pgfpathlineto{\pgfqpoint{2.418999in}{0.667204in}}%
\pgfpathlineto{\pgfqpoint{2.408093in}{0.667204in}}%
\pgfpathlineto{\pgfqpoint{2.397187in}{0.667204in}}%
\pgfpathlineto{\pgfqpoint{2.386281in}{0.667204in}}%
\pgfpathlineto{\pgfqpoint{2.375375in}{0.667204in}}%
\pgfpathlineto{\pgfqpoint{2.364469in}{0.667204in}}%
\pgfpathlineto{\pgfqpoint{2.353563in}{0.667204in}}%
\pgfpathlineto{\pgfqpoint{2.342657in}{0.667204in}}%
\pgfpathlineto{\pgfqpoint{2.331751in}{0.667204in}}%
\pgfpathlineto{\pgfqpoint{2.320845in}{0.667204in}}%
\pgfpathlineto{\pgfqpoint{2.309939in}{0.667204in}}%
\pgfpathlineto{\pgfqpoint{2.299033in}{0.667204in}}%
\pgfpathlineto{\pgfqpoint{2.288127in}{0.667204in}}%
\pgfpathlineto{\pgfqpoint{2.277221in}{0.667204in}}%
\pgfpathlineto{\pgfqpoint{2.266315in}{0.667204in}}%
\pgfpathlineto{\pgfqpoint{2.255409in}{0.667204in}}%
\pgfpathlineto{\pgfqpoint{2.244503in}{0.667204in}}%
\pgfpathlineto{\pgfqpoint{2.233597in}{0.667204in}}%
\pgfpathlineto{\pgfqpoint{2.222691in}{0.667204in}}%
\pgfpathlineto{\pgfqpoint{2.211785in}{0.667204in}}%
\pgfpathlineto{\pgfqpoint{2.200879in}{0.667204in}}%
\pgfpathlineto{\pgfqpoint{2.189973in}{0.667204in}}%
\pgfpathlineto{\pgfqpoint{2.179067in}{0.667204in}}%
\pgfpathlineto{\pgfqpoint{2.168161in}{0.667204in}}%
\pgfpathlineto{\pgfqpoint{2.157255in}{0.667204in}}%
\pgfpathlineto{\pgfqpoint{2.146349in}{0.667204in}}%
\pgfpathlineto{\pgfqpoint{2.135443in}{0.667204in}}%
\pgfpathlineto{\pgfqpoint{2.124537in}{0.667204in}}%
\pgfpathlineto{\pgfqpoint{2.113631in}{0.667204in}}%
\pgfpathlineto{\pgfqpoint{2.102725in}{0.667204in}}%
\pgfpathlineto{\pgfqpoint{2.091819in}{0.667204in}}%
\pgfpathlineto{\pgfqpoint{2.080913in}{0.667204in}}%
\pgfpathlineto{\pgfqpoint{2.070007in}{0.667204in}}%
\pgfpathlineto{\pgfqpoint{2.059101in}{0.667204in}}%
\pgfpathlineto{\pgfqpoint{2.048195in}{0.667204in}}%
\pgfpathlineto{\pgfqpoint{2.037289in}{0.667204in}}%
\pgfpathlineto{\pgfqpoint{2.026383in}{0.667204in}}%
\pgfpathlineto{\pgfqpoint{2.015477in}{0.667204in}}%
\pgfpathlineto{\pgfqpoint{2.004571in}{0.667204in}}%
\pgfpathlineto{\pgfqpoint{1.993665in}{0.667204in}}%
\pgfpathlineto{\pgfqpoint{1.982759in}{0.667204in}}%
\pgfpathlineto{\pgfqpoint{1.971854in}{0.667204in}}%
\pgfpathlineto{\pgfqpoint{1.960948in}{0.667204in}}%
\pgfpathlineto{\pgfqpoint{1.950042in}{0.667204in}}%
\pgfpathlineto{\pgfqpoint{1.939136in}{0.667204in}}%
\pgfpathlineto{\pgfqpoint{1.928230in}{0.671986in}}%
\pgfpathlineto{\pgfqpoint{1.917324in}{0.671986in}}%
\pgfpathlineto{\pgfqpoint{1.906418in}{0.671986in}}%
\pgfpathlineto{\pgfqpoint{1.895512in}{0.671986in}}%
\pgfpathlineto{\pgfqpoint{1.884606in}{0.671986in}}%
\pgfpathlineto{\pgfqpoint{1.873700in}{0.671986in}}%
\pgfpathlineto{\pgfqpoint{1.862794in}{0.671986in}}%
\pgfpathlineto{\pgfqpoint{1.851888in}{0.671986in}}%
\pgfpathlineto{\pgfqpoint{1.840982in}{0.671986in}}%
\pgfpathlineto{\pgfqpoint{1.830076in}{0.671986in}}%
\pgfpathlineto{\pgfqpoint{1.819170in}{0.671986in}}%
\pgfpathlineto{\pgfqpoint{1.808264in}{0.671986in}}%
\pgfpathlineto{\pgfqpoint{1.797358in}{0.671986in}}%
\pgfpathlineto{\pgfqpoint{1.786452in}{0.671986in}}%
\pgfpathlineto{\pgfqpoint{1.775546in}{0.671986in}}%
\pgfpathlineto{\pgfqpoint{1.764640in}{0.671986in}}%
\pgfpathlineto{\pgfqpoint{1.753734in}{0.671986in}}%
\pgfpathlineto{\pgfqpoint{1.742828in}{0.671986in}}%
\pgfpathlineto{\pgfqpoint{1.731922in}{0.671986in}}%
\pgfpathlineto{\pgfqpoint{1.721016in}{0.671986in}}%
\pgfpathlineto{\pgfqpoint{1.710110in}{0.671986in}}%
\pgfpathlineto{\pgfqpoint{1.699204in}{0.671986in}}%
\pgfpathlineto{\pgfqpoint{1.688298in}{0.671986in}}%
\pgfpathlineto{\pgfqpoint{1.677392in}{0.671986in}}%
\pgfpathlineto{\pgfqpoint{1.666486in}{0.671986in}}%
\pgfpathlineto{\pgfqpoint{1.655580in}{0.671986in}}%
\pgfpathlineto{\pgfqpoint{1.644674in}{0.671986in}}%
\pgfpathlineto{\pgfqpoint{1.633768in}{0.671986in}}%
\pgfpathlineto{\pgfqpoint{1.622862in}{0.671986in}}%
\pgfpathlineto{\pgfqpoint{1.611956in}{0.671986in}}%
\pgfpathlineto{\pgfqpoint{1.601050in}{0.671986in}}%
\pgfpathlineto{\pgfqpoint{1.590144in}{0.671986in}}%
\pgfpathlineto{\pgfqpoint{1.579238in}{0.671986in}}%
\pgfpathlineto{\pgfqpoint{1.568332in}{0.671986in}}%
\pgfpathlineto{\pgfqpoint{1.557426in}{0.671986in}}%
\pgfpathlineto{\pgfqpoint{1.546520in}{0.671986in}}%
\pgfpathlineto{\pgfqpoint{1.535614in}{0.671986in}}%
\pgfpathlineto{\pgfqpoint{1.524708in}{0.671986in}}%
\pgfpathlineto{\pgfqpoint{1.513802in}{0.671986in}}%
\pgfpathlineto{\pgfqpoint{1.502896in}{0.671986in}}%
\pgfpathlineto{\pgfqpoint{1.491991in}{0.671986in}}%
\pgfpathlineto{\pgfqpoint{1.481085in}{0.671986in}}%
\pgfpathlineto{\pgfqpoint{1.470179in}{0.671986in}}%
\pgfpathlineto{\pgfqpoint{1.459273in}{0.671986in}}%
\pgfpathlineto{\pgfqpoint{1.448367in}{0.671986in}}%
\pgfpathlineto{\pgfqpoint{1.437461in}{0.671986in}}%
\pgfpathlineto{\pgfqpoint{1.426555in}{0.671986in}}%
\pgfpathlineto{\pgfqpoint{1.415649in}{0.721820in}}%
\pgfpathlineto{\pgfqpoint{1.404743in}{0.724960in}}%
\pgfpathlineto{\pgfqpoint{1.393837in}{0.724960in}}%
\pgfpathlineto{\pgfqpoint{1.382931in}{0.906571in}}%
\pgfpathlineto{\pgfqpoint{1.372025in}{0.906571in}}%
\pgfpathlineto{\pgfqpoint{1.361119in}{0.906571in}}%
\pgfpathlineto{\pgfqpoint{1.350213in}{0.906571in}}%
\pgfpathlineto{\pgfqpoint{1.339307in}{0.906571in}}%
\pgfpathlineto{\pgfqpoint{1.328401in}{0.906571in}}%
\pgfpathlineto{\pgfqpoint{1.317495in}{0.906571in}}%
\pgfpathlineto{\pgfqpoint{1.306589in}{0.906571in}}%
\pgfpathlineto{\pgfqpoint{1.295683in}{0.906571in}}%
\pgfpathlineto{\pgfqpoint{1.284777in}{0.906571in}}%
\pgfpathlineto{\pgfqpoint{1.273871in}{0.906571in}}%
\pgfpathlineto{\pgfqpoint{1.262965in}{0.906571in}}%
\pgfpathlineto{\pgfqpoint{1.252059in}{0.906571in}}%
\pgfpathlineto{\pgfqpoint{1.241153in}{0.906571in}}%
\pgfpathlineto{\pgfqpoint{1.230247in}{1.113511in}}%
\pgfpathlineto{\pgfqpoint{1.219341in}{1.113511in}}%
\pgfpathlineto{\pgfqpoint{1.208435in}{1.113511in}}%
\pgfpathlineto{\pgfqpoint{1.197529in}{1.113511in}}%
\pgfpathlineto{\pgfqpoint{1.186623in}{1.113511in}}%
\pgfpathlineto{\pgfqpoint{1.175717in}{1.113511in}}%
\pgfpathlineto{\pgfqpoint{1.164811in}{1.113511in}}%
\pgfpathlineto{\pgfqpoint{1.153905in}{1.218711in}}%
\pgfpathlineto{\pgfqpoint{1.142999in}{1.218711in}}%
\pgfpathlineto{\pgfqpoint{1.132093in}{1.218711in}}%
\pgfpathlineto{\pgfqpoint{1.121187in}{1.269878in}}%
\pgfpathlineto{\pgfqpoint{1.110281in}{1.269878in}}%
\pgfpathlineto{\pgfqpoint{1.099375in}{1.269878in}}%
\pgfpathlineto{\pgfqpoint{1.088469in}{1.269878in}}%
\pgfpathlineto{\pgfqpoint{1.077563in}{1.269878in}}%
\pgfpathlineto{\pgfqpoint{1.066657in}{1.286448in}}%
\pgfpathlineto{\pgfqpoint{1.055751in}{1.368462in}}%
\pgfpathlineto{\pgfqpoint{1.044845in}{1.368462in}}%
\pgfpathlineto{\pgfqpoint{1.033939in}{1.372945in}}%
\pgfpathlineto{\pgfqpoint{1.023033in}{1.372945in}}%
\pgfpathlineto{\pgfqpoint{1.012127in}{1.372945in}}%
\pgfpathlineto{\pgfqpoint{1.001222in}{1.372945in}}%
\pgfpathlineto{\pgfqpoint{0.990316in}{1.372945in}}%
\pgfpathlineto{\pgfqpoint{0.979410in}{1.372945in}}%
\pgfpathlineto{\pgfqpoint{0.968504in}{1.372945in}}%
\pgfpathlineto{\pgfqpoint{0.957598in}{1.372945in}}%
\pgfpathlineto{\pgfqpoint{0.946692in}{1.372945in}}%
\pgfpathlineto{\pgfqpoint{0.935786in}{1.372945in}}%
\pgfpathlineto{\pgfqpoint{0.924880in}{1.372945in}}%
\pgfpathlineto{\pgfqpoint{0.913974in}{1.372945in}}%
\pgfpathlineto{\pgfqpoint{0.903068in}{1.489846in}}%
\pgfpathlineto{\pgfqpoint{0.892162in}{1.492548in}}%
\pgfpathlineto{\pgfqpoint{0.881256in}{1.492548in}}%
\pgfpathlineto{\pgfqpoint{0.870350in}{1.576710in}}%
\pgfpathlineto{\pgfqpoint{0.859444in}{1.580796in}}%
\pgfpathlineto{\pgfqpoint{0.848538in}{1.580796in}}%
\pgfpathlineto{\pgfqpoint{0.837632in}{1.580796in}}%
\pgfpathlineto{\pgfqpoint{0.826726in}{1.580796in}}%
\pgfpathlineto{\pgfqpoint{0.815820in}{1.580796in}}%
\pgfpathlineto{\pgfqpoint{0.804914in}{1.580796in}}%
\pgfpathlineto{\pgfqpoint{0.794008in}{1.582068in}}%
\pgfpathlineto{\pgfqpoint{0.783102in}{1.582068in}}%
\pgfpathlineto{\pgfqpoint{0.772196in}{1.582068in}}%
\pgfpathlineto{\pgfqpoint{0.761290in}{1.582068in}}%
\pgfpathlineto{\pgfqpoint{0.750384in}{1.582068in}}%
\pgfpathlineto{\pgfqpoint{0.739478in}{1.636366in}}%
\pgfpathlineto{\pgfqpoint{0.728572in}{1.636366in}}%
\pgfpathlineto{\pgfqpoint{0.717666in}{1.636366in}}%
\pgfpathlineto{\pgfqpoint{0.706760in}{1.636366in}}%
\pgfpathlineto{\pgfqpoint{0.695854in}{1.636366in}}%
\pgfpathlineto{\pgfqpoint{0.684948in}{1.636366in}}%
\pgfpathlineto{\pgfqpoint{0.674042in}{1.636366in}}%
\pgfpathlineto{\pgfqpoint{0.663136in}{1.636366in}}%
\pgfpathlineto{\pgfqpoint{0.652230in}{1.636366in}}%
\pgfpathlineto{\pgfqpoint{0.641324in}{1.658735in}}%
\pgfpathlineto{\pgfqpoint{0.630418in}{1.658735in}}%
\pgfpathlineto{\pgfqpoint{0.619512in}{1.658840in}}%
\pgfpathlineto{\pgfqpoint{0.608606in}{1.660933in}}%
\pgfpathlineto{\pgfqpoint{0.597700in}{1.716385in}}%
\pgfpathlineto{\pgfqpoint{0.586794in}{1.716663in}}%
\pgfpathlineto{\pgfqpoint{0.575888in}{1.739088in}}%
\pgfpathlineto{\pgfqpoint{0.564982in}{1.739088in}}%
\pgfpathlineto{\pgfqpoint{0.554076in}{1.739325in}}%
\pgfpathlineto{\pgfqpoint{0.543170in}{1.739381in}}%
\pgfpathclose%
\pgfusepath{fill}%
\end{pgfscope}%
\begin{pgfscope}%
\pgfpathrectangle{\pgfqpoint{0.423750in}{0.261892in}}{\pgfqpoint{2.627250in}{1.581827in}}%
\pgfusepath{clip}%
\pgfsetbuttcap%
\pgfsetroundjoin%
\definecolor{currentfill}{rgb}{0.172549,0.627451,0.172549}%
\pgfsetfillcolor{currentfill}%
\pgfsetfillopacity{0.200000}%
\pgfsetlinewidth{0.000000pt}%
\definecolor{currentstroke}{rgb}{0.000000,0.000000,0.000000}%
\pgfsetstrokecolor{currentstroke}%
\pgfsetdash{}{0pt}%
\pgfpathmoveto{\pgfqpoint{0.543170in}{1.770629in}}%
\pgfpathlineto{\pgfqpoint{0.543170in}{1.771252in}}%
\pgfpathlineto{\pgfqpoint{0.554076in}{1.770933in}}%
\pgfpathlineto{\pgfqpoint{0.564982in}{1.770929in}}%
\pgfpathlineto{\pgfqpoint{0.575888in}{1.769662in}}%
\pgfpathlineto{\pgfqpoint{0.586794in}{1.769662in}}%
\pgfpathlineto{\pgfqpoint{0.597700in}{1.767868in}}%
\pgfpathlineto{\pgfqpoint{0.608606in}{1.767868in}}%
\pgfpathlineto{\pgfqpoint{0.619512in}{1.767821in}}%
\pgfpathlineto{\pgfqpoint{0.630418in}{1.767821in}}%
\pgfpathlineto{\pgfqpoint{0.641324in}{1.767153in}}%
\pgfpathlineto{\pgfqpoint{0.652230in}{1.767153in}}%
\pgfpathlineto{\pgfqpoint{0.663136in}{1.767153in}}%
\pgfpathlineto{\pgfqpoint{0.674042in}{1.767153in}}%
\pgfpathlineto{\pgfqpoint{0.684948in}{1.767153in}}%
\pgfpathlineto{\pgfqpoint{0.695854in}{1.767153in}}%
\pgfpathlineto{\pgfqpoint{0.706760in}{1.743587in}}%
\pgfpathlineto{\pgfqpoint{0.717666in}{1.743587in}}%
\pgfpathlineto{\pgfqpoint{0.728572in}{1.743587in}}%
\pgfpathlineto{\pgfqpoint{0.739478in}{1.739237in}}%
\pgfpathlineto{\pgfqpoint{0.750384in}{1.739237in}}%
\pgfpathlineto{\pgfqpoint{0.761290in}{1.739237in}}%
\pgfpathlineto{\pgfqpoint{0.772196in}{1.739237in}}%
\pgfpathlineto{\pgfqpoint{0.783102in}{1.739237in}}%
\pgfpathlineto{\pgfqpoint{0.794008in}{1.739237in}}%
\pgfpathlineto{\pgfqpoint{0.804914in}{1.739237in}}%
\pgfpathlineto{\pgfqpoint{0.815820in}{1.739237in}}%
\pgfpathlineto{\pgfqpoint{0.826726in}{1.731196in}}%
\pgfpathlineto{\pgfqpoint{0.837632in}{1.731196in}}%
\pgfpathlineto{\pgfqpoint{0.848538in}{1.731196in}}%
\pgfpathlineto{\pgfqpoint{0.859444in}{1.731196in}}%
\pgfpathlineto{\pgfqpoint{0.870350in}{1.731196in}}%
\pgfpathlineto{\pgfqpoint{0.881256in}{1.731196in}}%
\pgfpathlineto{\pgfqpoint{0.892162in}{1.686265in}}%
\pgfpathlineto{\pgfqpoint{0.903068in}{1.686265in}}%
\pgfpathlineto{\pgfqpoint{0.913974in}{1.686265in}}%
\pgfpathlineto{\pgfqpoint{0.924880in}{1.651142in}}%
\pgfpathlineto{\pgfqpoint{0.935786in}{1.651142in}}%
\pgfpathlineto{\pgfqpoint{0.946692in}{1.619414in}}%
\pgfpathlineto{\pgfqpoint{0.957598in}{1.612428in}}%
\pgfpathlineto{\pgfqpoint{0.968504in}{1.612428in}}%
\pgfpathlineto{\pgfqpoint{0.979410in}{1.612428in}}%
\pgfpathlineto{\pgfqpoint{0.990316in}{1.612428in}}%
\pgfpathlineto{\pgfqpoint{1.001222in}{1.612428in}}%
\pgfpathlineto{\pgfqpoint{1.012127in}{1.612428in}}%
\pgfpathlineto{\pgfqpoint{1.023033in}{1.612428in}}%
\pgfpathlineto{\pgfqpoint{1.033939in}{1.612428in}}%
\pgfpathlineto{\pgfqpoint{1.044845in}{1.612428in}}%
\pgfpathlineto{\pgfqpoint{1.055751in}{1.612428in}}%
\pgfpathlineto{\pgfqpoint{1.066657in}{1.612428in}}%
\pgfpathlineto{\pgfqpoint{1.077563in}{1.612428in}}%
\pgfpathlineto{\pgfqpoint{1.088469in}{1.612428in}}%
\pgfpathlineto{\pgfqpoint{1.099375in}{1.612428in}}%
\pgfpathlineto{\pgfqpoint{1.110281in}{1.612428in}}%
\pgfpathlineto{\pgfqpoint{1.121187in}{1.612428in}}%
\pgfpathlineto{\pgfqpoint{1.132093in}{1.612428in}}%
\pgfpathlineto{\pgfqpoint{1.142999in}{1.612428in}}%
\pgfpathlineto{\pgfqpoint{1.153905in}{1.612428in}}%
\pgfpathlineto{\pgfqpoint{1.164811in}{1.612428in}}%
\pgfpathlineto{\pgfqpoint{1.175717in}{1.522322in}}%
\pgfpathlineto{\pgfqpoint{1.186623in}{1.522322in}}%
\pgfpathlineto{\pgfqpoint{1.197529in}{1.522322in}}%
\pgfpathlineto{\pgfqpoint{1.208435in}{1.522322in}}%
\pgfpathlineto{\pgfqpoint{1.219341in}{1.469900in}}%
\pgfpathlineto{\pgfqpoint{1.230247in}{1.469900in}}%
\pgfpathlineto{\pgfqpoint{1.241153in}{1.469900in}}%
\pgfpathlineto{\pgfqpoint{1.252059in}{1.469117in}}%
\pgfpathlineto{\pgfqpoint{1.262965in}{1.469117in}}%
\pgfpathlineto{\pgfqpoint{1.273871in}{1.469117in}}%
\pgfpathlineto{\pgfqpoint{1.284777in}{1.469117in}}%
\pgfpathlineto{\pgfqpoint{1.295683in}{1.468315in}}%
\pgfpathlineto{\pgfqpoint{1.306589in}{1.336612in}}%
\pgfpathlineto{\pgfqpoint{1.317495in}{1.336612in}}%
\pgfpathlineto{\pgfqpoint{1.328401in}{1.336612in}}%
\pgfpathlineto{\pgfqpoint{1.339307in}{1.301797in}}%
\pgfpathlineto{\pgfqpoint{1.350213in}{1.301797in}}%
\pgfpathlineto{\pgfqpoint{1.361119in}{1.301797in}}%
\pgfpathlineto{\pgfqpoint{1.372025in}{1.301797in}}%
\pgfpathlineto{\pgfqpoint{1.382931in}{1.301797in}}%
\pgfpathlineto{\pgfqpoint{1.393837in}{1.301797in}}%
\pgfpathlineto{\pgfqpoint{1.404743in}{1.301797in}}%
\pgfpathlineto{\pgfqpoint{1.415649in}{1.301797in}}%
\pgfpathlineto{\pgfqpoint{1.426555in}{1.301797in}}%
\pgfpathlineto{\pgfqpoint{1.437461in}{1.301797in}}%
\pgfpathlineto{\pgfqpoint{1.448367in}{1.301797in}}%
\pgfpathlineto{\pgfqpoint{1.459273in}{1.301797in}}%
\pgfpathlineto{\pgfqpoint{1.470179in}{1.301797in}}%
\pgfpathlineto{\pgfqpoint{1.481085in}{1.301797in}}%
\pgfpathlineto{\pgfqpoint{1.491991in}{1.301797in}}%
\pgfpathlineto{\pgfqpoint{1.502896in}{1.301797in}}%
\pgfpathlineto{\pgfqpoint{1.513802in}{1.301797in}}%
\pgfpathlineto{\pgfqpoint{1.524708in}{1.301797in}}%
\pgfpathlineto{\pgfqpoint{1.535614in}{1.301797in}}%
\pgfpathlineto{\pgfqpoint{1.546520in}{1.301797in}}%
\pgfpathlineto{\pgfqpoint{1.557426in}{1.301797in}}%
\pgfpathlineto{\pgfqpoint{1.568332in}{1.301797in}}%
\pgfpathlineto{\pgfqpoint{1.579238in}{1.301797in}}%
\pgfpathlineto{\pgfqpoint{1.590144in}{1.301797in}}%
\pgfpathlineto{\pgfqpoint{1.601050in}{1.301797in}}%
\pgfpathlineto{\pgfqpoint{1.611956in}{1.301797in}}%
\pgfpathlineto{\pgfqpoint{1.622862in}{1.301797in}}%
\pgfpathlineto{\pgfqpoint{1.633768in}{1.301797in}}%
\pgfpathlineto{\pgfqpoint{1.644674in}{1.301797in}}%
\pgfpathlineto{\pgfqpoint{1.655580in}{1.275947in}}%
\pgfpathlineto{\pgfqpoint{1.666486in}{1.275947in}}%
\pgfpathlineto{\pgfqpoint{1.677392in}{1.275947in}}%
\pgfpathlineto{\pgfqpoint{1.688298in}{1.264595in}}%
\pgfpathlineto{\pgfqpoint{1.699204in}{1.264595in}}%
\pgfpathlineto{\pgfqpoint{1.710110in}{1.264595in}}%
\pgfpathlineto{\pgfqpoint{1.721016in}{1.264595in}}%
\pgfpathlineto{\pgfqpoint{1.731922in}{1.264595in}}%
\pgfpathlineto{\pgfqpoint{1.742828in}{1.264595in}}%
\pgfpathlineto{\pgfqpoint{1.753734in}{1.264595in}}%
\pgfpathlineto{\pgfqpoint{1.764640in}{1.264595in}}%
\pgfpathlineto{\pgfqpoint{1.775546in}{1.264257in}}%
\pgfpathlineto{\pgfqpoint{1.786452in}{1.264257in}}%
\pgfpathlineto{\pgfqpoint{1.797358in}{1.264257in}}%
\pgfpathlineto{\pgfqpoint{1.808264in}{1.264257in}}%
\pgfpathlineto{\pgfqpoint{1.819170in}{1.264257in}}%
\pgfpathlineto{\pgfqpoint{1.830076in}{1.264257in}}%
\pgfpathlineto{\pgfqpoint{1.840982in}{1.264257in}}%
\pgfpathlineto{\pgfqpoint{1.851888in}{1.264257in}}%
\pgfpathlineto{\pgfqpoint{1.862794in}{1.264257in}}%
\pgfpathlineto{\pgfqpoint{1.873700in}{1.264257in}}%
\pgfpathlineto{\pgfqpoint{1.884606in}{1.264257in}}%
\pgfpathlineto{\pgfqpoint{1.895512in}{1.264257in}}%
\pgfpathlineto{\pgfqpoint{1.906418in}{1.264257in}}%
\pgfpathlineto{\pgfqpoint{1.917324in}{1.264257in}}%
\pgfpathlineto{\pgfqpoint{1.928230in}{1.264257in}}%
\pgfpathlineto{\pgfqpoint{1.939136in}{1.264257in}}%
\pgfpathlineto{\pgfqpoint{1.950042in}{1.264257in}}%
\pgfpathlineto{\pgfqpoint{1.960948in}{1.264257in}}%
\pgfpathlineto{\pgfqpoint{1.971854in}{1.264257in}}%
\pgfpathlineto{\pgfqpoint{1.982759in}{1.264257in}}%
\pgfpathlineto{\pgfqpoint{1.993665in}{1.264257in}}%
\pgfpathlineto{\pgfqpoint{2.004571in}{1.264257in}}%
\pgfpathlineto{\pgfqpoint{2.015477in}{1.264257in}}%
\pgfpathlineto{\pgfqpoint{2.026383in}{1.264257in}}%
\pgfpathlineto{\pgfqpoint{2.037289in}{1.264257in}}%
\pgfpathlineto{\pgfqpoint{2.048195in}{1.264257in}}%
\pgfpathlineto{\pgfqpoint{2.059101in}{1.224505in}}%
\pgfpathlineto{\pgfqpoint{2.070007in}{1.224505in}}%
\pgfpathlineto{\pgfqpoint{2.080913in}{1.224505in}}%
\pgfpathlineto{\pgfqpoint{2.091819in}{1.224505in}}%
\pgfpathlineto{\pgfqpoint{2.102725in}{1.224505in}}%
\pgfpathlineto{\pgfqpoint{2.113631in}{1.224505in}}%
\pgfpathlineto{\pgfqpoint{2.124537in}{1.224505in}}%
\pgfpathlineto{\pgfqpoint{2.135443in}{1.224505in}}%
\pgfpathlineto{\pgfqpoint{2.146349in}{1.224505in}}%
\pgfpathlineto{\pgfqpoint{2.157255in}{1.224505in}}%
\pgfpathlineto{\pgfqpoint{2.168161in}{1.224505in}}%
\pgfpathlineto{\pgfqpoint{2.179067in}{1.224505in}}%
\pgfpathlineto{\pgfqpoint{2.189973in}{1.224505in}}%
\pgfpathlineto{\pgfqpoint{2.200879in}{1.224505in}}%
\pgfpathlineto{\pgfqpoint{2.211785in}{1.224505in}}%
\pgfpathlineto{\pgfqpoint{2.222691in}{1.224505in}}%
\pgfpathlineto{\pgfqpoint{2.233597in}{1.224505in}}%
\pgfpathlineto{\pgfqpoint{2.244503in}{1.224505in}}%
\pgfpathlineto{\pgfqpoint{2.255409in}{1.224505in}}%
\pgfpathlineto{\pgfqpoint{2.266315in}{1.125165in}}%
\pgfpathlineto{\pgfqpoint{2.277221in}{1.125165in}}%
\pgfpathlineto{\pgfqpoint{2.288127in}{1.125165in}}%
\pgfpathlineto{\pgfqpoint{2.299033in}{1.125165in}}%
\pgfpathlineto{\pgfqpoint{2.309939in}{1.125165in}}%
\pgfpathlineto{\pgfqpoint{2.320845in}{1.125165in}}%
\pgfpathlineto{\pgfqpoint{2.331751in}{1.125165in}}%
\pgfpathlineto{\pgfqpoint{2.342657in}{1.125165in}}%
\pgfpathlineto{\pgfqpoint{2.353563in}{1.125165in}}%
\pgfpathlineto{\pgfqpoint{2.364469in}{1.125165in}}%
\pgfpathlineto{\pgfqpoint{2.375375in}{1.125165in}}%
\pgfpathlineto{\pgfqpoint{2.386281in}{1.125165in}}%
\pgfpathlineto{\pgfqpoint{2.397187in}{1.125165in}}%
\pgfpathlineto{\pgfqpoint{2.408093in}{1.125165in}}%
\pgfpathlineto{\pgfqpoint{2.418999in}{1.125165in}}%
\pgfpathlineto{\pgfqpoint{2.429905in}{1.125165in}}%
\pgfpathlineto{\pgfqpoint{2.440811in}{1.104886in}}%
\pgfpathlineto{\pgfqpoint{2.451717in}{1.104886in}}%
\pgfpathlineto{\pgfqpoint{2.462623in}{1.104886in}}%
\pgfpathlineto{\pgfqpoint{2.473528in}{1.104886in}}%
\pgfpathlineto{\pgfqpoint{2.484434in}{1.104886in}}%
\pgfpathlineto{\pgfqpoint{2.495340in}{1.104886in}}%
\pgfpathlineto{\pgfqpoint{2.506246in}{1.104886in}}%
\pgfpathlineto{\pgfqpoint{2.517152in}{1.104886in}}%
\pgfpathlineto{\pgfqpoint{2.528058in}{1.104886in}}%
\pgfpathlineto{\pgfqpoint{2.538964in}{1.104886in}}%
\pgfpathlineto{\pgfqpoint{2.549870in}{1.104886in}}%
\pgfpathlineto{\pgfqpoint{2.560776in}{1.104886in}}%
\pgfpathlineto{\pgfqpoint{2.571682in}{1.104886in}}%
\pgfpathlineto{\pgfqpoint{2.582588in}{1.104886in}}%
\pgfpathlineto{\pgfqpoint{2.593494in}{1.063576in}}%
\pgfpathlineto{\pgfqpoint{2.604400in}{1.063576in}}%
\pgfpathlineto{\pgfqpoint{2.615306in}{1.063576in}}%
\pgfpathlineto{\pgfqpoint{2.626212in}{1.063576in}}%
\pgfpathlineto{\pgfqpoint{2.637118in}{1.063576in}}%
\pgfpathlineto{\pgfqpoint{2.648024in}{1.063576in}}%
\pgfpathlineto{\pgfqpoint{2.658930in}{1.063576in}}%
\pgfpathlineto{\pgfqpoint{2.669836in}{1.063576in}}%
\pgfpathlineto{\pgfqpoint{2.680742in}{1.063576in}}%
\pgfpathlineto{\pgfqpoint{2.691648in}{1.063576in}}%
\pgfpathlineto{\pgfqpoint{2.702554in}{1.063576in}}%
\pgfpathlineto{\pgfqpoint{2.713460in}{1.062775in}}%
\pgfpathlineto{\pgfqpoint{2.724366in}{1.062775in}}%
\pgfpathlineto{\pgfqpoint{2.735272in}{1.062775in}}%
\pgfpathlineto{\pgfqpoint{2.746178in}{1.062775in}}%
\pgfpathlineto{\pgfqpoint{2.757084in}{1.062775in}}%
\pgfpathlineto{\pgfqpoint{2.767990in}{1.062775in}}%
\pgfpathlineto{\pgfqpoint{2.778896in}{1.062775in}}%
\pgfpathlineto{\pgfqpoint{2.789802in}{1.062775in}}%
\pgfpathlineto{\pgfqpoint{2.800708in}{1.062775in}}%
\pgfpathlineto{\pgfqpoint{2.811614in}{1.062775in}}%
\pgfpathlineto{\pgfqpoint{2.822520in}{1.062775in}}%
\pgfpathlineto{\pgfqpoint{2.833426in}{1.062775in}}%
\pgfpathlineto{\pgfqpoint{2.844332in}{1.062775in}}%
\pgfpathlineto{\pgfqpoint{2.855238in}{1.062775in}}%
\pgfpathlineto{\pgfqpoint{2.866144in}{1.062775in}}%
\pgfpathlineto{\pgfqpoint{2.877050in}{1.062775in}}%
\pgfpathlineto{\pgfqpoint{2.887956in}{1.062775in}}%
\pgfpathlineto{\pgfqpoint{2.898862in}{1.062775in}}%
\pgfpathlineto{\pgfqpoint{2.909768in}{1.062775in}}%
\pgfpathlineto{\pgfqpoint{2.920674in}{1.061863in}}%
\pgfpathlineto{\pgfqpoint{2.931580in}{1.061863in}}%
\pgfpathlineto{\pgfqpoint{2.931580in}{0.817191in}}%
\pgfpathlineto{\pgfqpoint{2.931580in}{0.817191in}}%
\pgfpathlineto{\pgfqpoint{2.920674in}{0.817191in}}%
\pgfpathlineto{\pgfqpoint{2.909768in}{0.824081in}}%
\pgfpathlineto{\pgfqpoint{2.898862in}{0.824081in}}%
\pgfpathlineto{\pgfqpoint{2.887956in}{0.824081in}}%
\pgfpathlineto{\pgfqpoint{2.877050in}{0.824081in}}%
\pgfpathlineto{\pgfqpoint{2.866144in}{0.824081in}}%
\pgfpathlineto{\pgfqpoint{2.855238in}{0.824081in}}%
\pgfpathlineto{\pgfqpoint{2.844332in}{0.824081in}}%
\pgfpathlineto{\pgfqpoint{2.833426in}{0.824081in}}%
\pgfpathlineto{\pgfqpoint{2.822520in}{0.824081in}}%
\pgfpathlineto{\pgfqpoint{2.811614in}{0.824081in}}%
\pgfpathlineto{\pgfqpoint{2.800708in}{0.824081in}}%
\pgfpathlineto{\pgfqpoint{2.789802in}{0.824081in}}%
\pgfpathlineto{\pgfqpoint{2.778896in}{0.824081in}}%
\pgfpathlineto{\pgfqpoint{2.767990in}{0.824081in}}%
\pgfpathlineto{\pgfqpoint{2.757084in}{0.824081in}}%
\pgfpathlineto{\pgfqpoint{2.746178in}{0.824081in}}%
\pgfpathlineto{\pgfqpoint{2.735272in}{0.824081in}}%
\pgfpathlineto{\pgfqpoint{2.724366in}{0.824081in}}%
\pgfpathlineto{\pgfqpoint{2.713460in}{0.824081in}}%
\pgfpathlineto{\pgfqpoint{2.702554in}{0.830575in}}%
\pgfpathlineto{\pgfqpoint{2.691648in}{0.830575in}}%
\pgfpathlineto{\pgfqpoint{2.680742in}{0.830575in}}%
\pgfpathlineto{\pgfqpoint{2.669836in}{0.830575in}}%
\pgfpathlineto{\pgfqpoint{2.658930in}{0.830575in}}%
\pgfpathlineto{\pgfqpoint{2.648024in}{0.830575in}}%
\pgfpathlineto{\pgfqpoint{2.637118in}{0.830575in}}%
\pgfpathlineto{\pgfqpoint{2.626212in}{0.830575in}}%
\pgfpathlineto{\pgfqpoint{2.615306in}{0.830575in}}%
\pgfpathlineto{\pgfqpoint{2.604400in}{0.830575in}}%
\pgfpathlineto{\pgfqpoint{2.593494in}{0.830575in}}%
\pgfpathlineto{\pgfqpoint{2.582588in}{0.842378in}}%
\pgfpathlineto{\pgfqpoint{2.571682in}{0.842378in}}%
\pgfpathlineto{\pgfqpoint{2.560776in}{0.842378in}}%
\pgfpathlineto{\pgfqpoint{2.549870in}{0.842378in}}%
\pgfpathlineto{\pgfqpoint{2.538964in}{0.842378in}}%
\pgfpathlineto{\pgfqpoint{2.528058in}{0.842378in}}%
\pgfpathlineto{\pgfqpoint{2.517152in}{0.842378in}}%
\pgfpathlineto{\pgfqpoint{2.506246in}{0.842378in}}%
\pgfpathlineto{\pgfqpoint{2.495340in}{0.842378in}}%
\pgfpathlineto{\pgfqpoint{2.484434in}{0.842378in}}%
\pgfpathlineto{\pgfqpoint{2.473528in}{0.842378in}}%
\pgfpathlineto{\pgfqpoint{2.462623in}{0.842378in}}%
\pgfpathlineto{\pgfqpoint{2.451717in}{0.842378in}}%
\pgfpathlineto{\pgfqpoint{2.440811in}{0.842378in}}%
\pgfpathlineto{\pgfqpoint{2.429905in}{0.940065in}}%
\pgfpathlineto{\pgfqpoint{2.418999in}{0.940065in}}%
\pgfpathlineto{\pgfqpoint{2.408093in}{0.940065in}}%
\pgfpathlineto{\pgfqpoint{2.397187in}{0.940065in}}%
\pgfpathlineto{\pgfqpoint{2.386281in}{0.940065in}}%
\pgfpathlineto{\pgfqpoint{2.375375in}{0.940065in}}%
\pgfpathlineto{\pgfqpoint{2.364469in}{0.940065in}}%
\pgfpathlineto{\pgfqpoint{2.353563in}{0.940065in}}%
\pgfpathlineto{\pgfqpoint{2.342657in}{0.940065in}}%
\pgfpathlineto{\pgfqpoint{2.331751in}{0.940065in}}%
\pgfpathlineto{\pgfqpoint{2.320845in}{0.940065in}}%
\pgfpathlineto{\pgfqpoint{2.309939in}{0.940065in}}%
\pgfpathlineto{\pgfqpoint{2.299033in}{0.940065in}}%
\pgfpathlineto{\pgfqpoint{2.288127in}{0.940065in}}%
\pgfpathlineto{\pgfqpoint{2.277221in}{0.940065in}}%
\pgfpathlineto{\pgfqpoint{2.266315in}{0.940065in}}%
\pgfpathlineto{\pgfqpoint{2.255409in}{0.970788in}}%
\pgfpathlineto{\pgfqpoint{2.244503in}{0.970788in}}%
\pgfpathlineto{\pgfqpoint{2.233597in}{0.970788in}}%
\pgfpathlineto{\pgfqpoint{2.222691in}{0.970788in}}%
\pgfpathlineto{\pgfqpoint{2.211785in}{0.970788in}}%
\pgfpathlineto{\pgfqpoint{2.200879in}{0.970788in}}%
\pgfpathlineto{\pgfqpoint{2.189973in}{0.970788in}}%
\pgfpathlineto{\pgfqpoint{2.179067in}{0.970788in}}%
\pgfpathlineto{\pgfqpoint{2.168161in}{0.970788in}}%
\pgfpathlineto{\pgfqpoint{2.157255in}{0.970788in}}%
\pgfpathlineto{\pgfqpoint{2.146349in}{0.970788in}}%
\pgfpathlineto{\pgfqpoint{2.135443in}{0.970788in}}%
\pgfpathlineto{\pgfqpoint{2.124537in}{0.970788in}}%
\pgfpathlineto{\pgfqpoint{2.113631in}{0.970788in}}%
\pgfpathlineto{\pgfqpoint{2.102725in}{0.970788in}}%
\pgfpathlineto{\pgfqpoint{2.091819in}{0.970788in}}%
\pgfpathlineto{\pgfqpoint{2.080913in}{0.970788in}}%
\pgfpathlineto{\pgfqpoint{2.070007in}{0.970788in}}%
\pgfpathlineto{\pgfqpoint{2.059101in}{0.970788in}}%
\pgfpathlineto{\pgfqpoint{2.048195in}{1.111526in}}%
\pgfpathlineto{\pgfqpoint{2.037289in}{1.111526in}}%
\pgfpathlineto{\pgfqpoint{2.026383in}{1.111526in}}%
\pgfpathlineto{\pgfqpoint{2.015477in}{1.111526in}}%
\pgfpathlineto{\pgfqpoint{2.004571in}{1.111526in}}%
\pgfpathlineto{\pgfqpoint{1.993665in}{1.111526in}}%
\pgfpathlineto{\pgfqpoint{1.982759in}{1.111526in}}%
\pgfpathlineto{\pgfqpoint{1.971854in}{1.111526in}}%
\pgfpathlineto{\pgfqpoint{1.960948in}{1.111526in}}%
\pgfpathlineto{\pgfqpoint{1.950042in}{1.111526in}}%
\pgfpathlineto{\pgfqpoint{1.939136in}{1.111526in}}%
\pgfpathlineto{\pgfqpoint{1.928230in}{1.111526in}}%
\pgfpathlineto{\pgfqpoint{1.917324in}{1.111526in}}%
\pgfpathlineto{\pgfqpoint{1.906418in}{1.111526in}}%
\pgfpathlineto{\pgfqpoint{1.895512in}{1.111526in}}%
\pgfpathlineto{\pgfqpoint{1.884606in}{1.111526in}}%
\pgfpathlineto{\pgfqpoint{1.873700in}{1.111526in}}%
\pgfpathlineto{\pgfqpoint{1.862794in}{1.111526in}}%
\pgfpathlineto{\pgfqpoint{1.851888in}{1.111526in}}%
\pgfpathlineto{\pgfqpoint{1.840982in}{1.111526in}}%
\pgfpathlineto{\pgfqpoint{1.830076in}{1.111526in}}%
\pgfpathlineto{\pgfqpoint{1.819170in}{1.111526in}}%
\pgfpathlineto{\pgfqpoint{1.808264in}{1.111526in}}%
\pgfpathlineto{\pgfqpoint{1.797358in}{1.111526in}}%
\pgfpathlineto{\pgfqpoint{1.786452in}{1.111526in}}%
\pgfpathlineto{\pgfqpoint{1.775546in}{1.111526in}}%
\pgfpathlineto{\pgfqpoint{1.764640in}{1.113923in}}%
\pgfpathlineto{\pgfqpoint{1.753734in}{1.113923in}}%
\pgfpathlineto{\pgfqpoint{1.742828in}{1.113923in}}%
\pgfpathlineto{\pgfqpoint{1.731922in}{1.113923in}}%
\pgfpathlineto{\pgfqpoint{1.721016in}{1.113923in}}%
\pgfpathlineto{\pgfqpoint{1.710110in}{1.113923in}}%
\pgfpathlineto{\pgfqpoint{1.699204in}{1.113923in}}%
\pgfpathlineto{\pgfqpoint{1.688298in}{1.113923in}}%
\pgfpathlineto{\pgfqpoint{1.677392in}{1.127028in}}%
\pgfpathlineto{\pgfqpoint{1.666486in}{1.127028in}}%
\pgfpathlineto{\pgfqpoint{1.655580in}{1.127028in}}%
\pgfpathlineto{\pgfqpoint{1.644674in}{1.140438in}}%
\pgfpathlineto{\pgfqpoint{1.633768in}{1.140438in}}%
\pgfpathlineto{\pgfqpoint{1.622862in}{1.140438in}}%
\pgfpathlineto{\pgfqpoint{1.611956in}{1.140438in}}%
\pgfpathlineto{\pgfqpoint{1.601050in}{1.140438in}}%
\pgfpathlineto{\pgfqpoint{1.590144in}{1.140438in}}%
\pgfpathlineto{\pgfqpoint{1.579238in}{1.140438in}}%
\pgfpathlineto{\pgfqpoint{1.568332in}{1.140438in}}%
\pgfpathlineto{\pgfqpoint{1.557426in}{1.140438in}}%
\pgfpathlineto{\pgfqpoint{1.546520in}{1.140438in}}%
\pgfpathlineto{\pgfqpoint{1.535614in}{1.140438in}}%
\pgfpathlineto{\pgfqpoint{1.524708in}{1.140438in}}%
\pgfpathlineto{\pgfqpoint{1.513802in}{1.140438in}}%
\pgfpathlineto{\pgfqpoint{1.502896in}{1.140438in}}%
\pgfpathlineto{\pgfqpoint{1.491991in}{1.140438in}}%
\pgfpathlineto{\pgfqpoint{1.481085in}{1.140438in}}%
\pgfpathlineto{\pgfqpoint{1.470179in}{1.140438in}}%
\pgfpathlineto{\pgfqpoint{1.459273in}{1.140438in}}%
\pgfpathlineto{\pgfqpoint{1.448367in}{1.140438in}}%
\pgfpathlineto{\pgfqpoint{1.437461in}{1.140438in}}%
\pgfpathlineto{\pgfqpoint{1.426555in}{1.140438in}}%
\pgfpathlineto{\pgfqpoint{1.415649in}{1.140438in}}%
\pgfpathlineto{\pgfqpoint{1.404743in}{1.140438in}}%
\pgfpathlineto{\pgfqpoint{1.393837in}{1.140438in}}%
\pgfpathlineto{\pgfqpoint{1.382931in}{1.140438in}}%
\pgfpathlineto{\pgfqpoint{1.372025in}{1.140438in}}%
\pgfpathlineto{\pgfqpoint{1.361119in}{1.140438in}}%
\pgfpathlineto{\pgfqpoint{1.350213in}{1.140438in}}%
\pgfpathlineto{\pgfqpoint{1.339307in}{1.140438in}}%
\pgfpathlineto{\pgfqpoint{1.328401in}{1.176806in}}%
\pgfpathlineto{\pgfqpoint{1.317495in}{1.176806in}}%
\pgfpathlineto{\pgfqpoint{1.306589in}{1.176806in}}%
\pgfpathlineto{\pgfqpoint{1.295683in}{1.231668in}}%
\pgfpathlineto{\pgfqpoint{1.284777in}{1.239418in}}%
\pgfpathlineto{\pgfqpoint{1.273871in}{1.239418in}}%
\pgfpathlineto{\pgfqpoint{1.262965in}{1.239418in}}%
\pgfpathlineto{\pgfqpoint{1.252059in}{1.239418in}}%
\pgfpathlineto{\pgfqpoint{1.241153in}{1.246385in}}%
\pgfpathlineto{\pgfqpoint{1.230247in}{1.246385in}}%
\pgfpathlineto{\pgfqpoint{1.219341in}{1.246385in}}%
\pgfpathlineto{\pgfqpoint{1.208435in}{1.386079in}}%
\pgfpathlineto{\pgfqpoint{1.197529in}{1.386079in}}%
\pgfpathlineto{\pgfqpoint{1.186623in}{1.386079in}}%
\pgfpathlineto{\pgfqpoint{1.175717in}{1.386079in}}%
\pgfpathlineto{\pgfqpoint{1.164811in}{1.428525in}}%
\pgfpathlineto{\pgfqpoint{1.153905in}{1.428525in}}%
\pgfpathlineto{\pgfqpoint{1.142999in}{1.428525in}}%
\pgfpathlineto{\pgfqpoint{1.132093in}{1.428525in}}%
\pgfpathlineto{\pgfqpoint{1.121187in}{1.428525in}}%
\pgfpathlineto{\pgfqpoint{1.110281in}{1.428525in}}%
\pgfpathlineto{\pgfqpoint{1.099375in}{1.428525in}}%
\pgfpathlineto{\pgfqpoint{1.088469in}{1.428525in}}%
\pgfpathlineto{\pgfqpoint{1.077563in}{1.428525in}}%
\pgfpathlineto{\pgfqpoint{1.066657in}{1.428525in}}%
\pgfpathlineto{\pgfqpoint{1.055751in}{1.428525in}}%
\pgfpathlineto{\pgfqpoint{1.044845in}{1.428525in}}%
\pgfpathlineto{\pgfqpoint{1.033939in}{1.428525in}}%
\pgfpathlineto{\pgfqpoint{1.023033in}{1.428525in}}%
\pgfpathlineto{\pgfqpoint{1.012127in}{1.428525in}}%
\pgfpathlineto{\pgfqpoint{1.001222in}{1.428525in}}%
\pgfpathlineto{\pgfqpoint{0.990316in}{1.428525in}}%
\pgfpathlineto{\pgfqpoint{0.979410in}{1.428525in}}%
\pgfpathlineto{\pgfqpoint{0.968504in}{1.428525in}}%
\pgfpathlineto{\pgfqpoint{0.957598in}{1.428525in}}%
\pgfpathlineto{\pgfqpoint{0.946692in}{1.445333in}}%
\pgfpathlineto{\pgfqpoint{0.935786in}{1.456166in}}%
\pgfpathlineto{\pgfqpoint{0.924880in}{1.456166in}}%
\pgfpathlineto{\pgfqpoint{0.913974in}{1.518545in}}%
\pgfpathlineto{\pgfqpoint{0.903068in}{1.518545in}}%
\pgfpathlineto{\pgfqpoint{0.892162in}{1.518545in}}%
\pgfpathlineto{\pgfqpoint{0.881256in}{1.615856in}}%
\pgfpathlineto{\pgfqpoint{0.870350in}{1.615856in}}%
\pgfpathlineto{\pgfqpoint{0.859444in}{1.615856in}}%
\pgfpathlineto{\pgfqpoint{0.848538in}{1.615856in}}%
\pgfpathlineto{\pgfqpoint{0.837632in}{1.615856in}}%
\pgfpathlineto{\pgfqpoint{0.826726in}{1.615856in}}%
\pgfpathlineto{\pgfqpoint{0.815820in}{1.624853in}}%
\pgfpathlineto{\pgfqpoint{0.804914in}{1.624853in}}%
\pgfpathlineto{\pgfqpoint{0.794008in}{1.624853in}}%
\pgfpathlineto{\pgfqpoint{0.783102in}{1.624853in}}%
\pgfpathlineto{\pgfqpoint{0.772196in}{1.624853in}}%
\pgfpathlineto{\pgfqpoint{0.761290in}{1.624853in}}%
\pgfpathlineto{\pgfqpoint{0.750384in}{1.624853in}}%
\pgfpathlineto{\pgfqpoint{0.739478in}{1.624853in}}%
\pgfpathlineto{\pgfqpoint{0.728572in}{1.629312in}}%
\pgfpathlineto{\pgfqpoint{0.717666in}{1.629312in}}%
\pgfpathlineto{\pgfqpoint{0.706760in}{1.629312in}}%
\pgfpathlineto{\pgfqpoint{0.695854in}{1.698479in}}%
\pgfpathlineto{\pgfqpoint{0.684948in}{1.698479in}}%
\pgfpathlineto{\pgfqpoint{0.674042in}{1.698479in}}%
\pgfpathlineto{\pgfqpoint{0.663136in}{1.698479in}}%
\pgfpathlineto{\pgfqpoint{0.652230in}{1.698479in}}%
\pgfpathlineto{\pgfqpoint{0.641324in}{1.698479in}}%
\pgfpathlineto{\pgfqpoint{0.630418in}{1.699168in}}%
\pgfpathlineto{\pgfqpoint{0.619512in}{1.699168in}}%
\pgfpathlineto{\pgfqpoint{0.608606in}{1.699213in}}%
\pgfpathlineto{\pgfqpoint{0.597700in}{1.699213in}}%
\pgfpathlineto{\pgfqpoint{0.586794in}{1.701033in}}%
\pgfpathlineto{\pgfqpoint{0.575888in}{1.701033in}}%
\pgfpathlineto{\pgfqpoint{0.564982in}{1.769202in}}%
\pgfpathlineto{\pgfqpoint{0.554076in}{1.769203in}}%
\pgfpathlineto{\pgfqpoint{0.543170in}{1.770629in}}%
\pgfpathclose%
\pgfusepath{fill}%
\end{pgfscope}%
\begin{pgfscope}%
\pgfpathrectangle{\pgfqpoint{0.423750in}{0.261892in}}{\pgfqpoint{2.627250in}{1.581827in}}%
\pgfusepath{clip}%
\pgfsetbuttcap%
\pgfsetroundjoin%
\definecolor{currentfill}{rgb}{0.839216,0.152941,0.156863}%
\pgfsetfillcolor{currentfill}%
\pgfsetfillopacity{0.200000}%
\pgfsetlinewidth{0.000000pt}%
\definecolor{currentstroke}{rgb}{0.000000,0.000000,0.000000}%
\pgfsetstrokecolor{currentstroke}%
\pgfsetdash{}{0pt}%
\pgfpathmoveto{\pgfqpoint{0.543170in}{1.771256in}}%
\pgfpathlineto{\pgfqpoint{0.543170in}{1.771351in}}%
\pgfpathlineto{\pgfqpoint{0.554076in}{1.771264in}}%
\pgfpathlineto{\pgfqpoint{0.564982in}{1.771264in}}%
\pgfpathlineto{\pgfqpoint{0.575888in}{1.771194in}}%
\pgfpathlineto{\pgfqpoint{0.586794in}{1.771158in}}%
\pgfpathlineto{\pgfqpoint{0.597700in}{1.770804in}}%
\pgfpathlineto{\pgfqpoint{0.608606in}{1.769892in}}%
\pgfpathlineto{\pgfqpoint{0.619512in}{1.769713in}}%
\pgfpathlineto{\pgfqpoint{0.630418in}{1.769713in}}%
\pgfpathlineto{\pgfqpoint{0.641324in}{1.769673in}}%
\pgfpathlineto{\pgfqpoint{0.652230in}{1.769669in}}%
\pgfpathlineto{\pgfqpoint{0.663136in}{1.764308in}}%
\pgfpathlineto{\pgfqpoint{0.674042in}{1.759767in}}%
\pgfpathlineto{\pgfqpoint{0.684948in}{1.759767in}}%
\pgfpathlineto{\pgfqpoint{0.695854in}{1.753550in}}%
\pgfpathlineto{\pgfqpoint{0.706760in}{1.753550in}}%
\pgfpathlineto{\pgfqpoint{0.717666in}{1.753550in}}%
\pgfpathlineto{\pgfqpoint{0.728572in}{1.753550in}}%
\pgfpathlineto{\pgfqpoint{0.739478in}{1.753550in}}%
\pgfpathlineto{\pgfqpoint{0.750384in}{1.753550in}}%
\pgfpathlineto{\pgfqpoint{0.761290in}{1.752715in}}%
\pgfpathlineto{\pgfqpoint{0.772196in}{1.752190in}}%
\pgfpathlineto{\pgfqpoint{0.783102in}{1.749356in}}%
\pgfpathlineto{\pgfqpoint{0.794008in}{1.743200in}}%
\pgfpathlineto{\pgfqpoint{0.804914in}{1.735225in}}%
\pgfpathlineto{\pgfqpoint{0.815820in}{1.729416in}}%
\pgfpathlineto{\pgfqpoint{0.826726in}{1.718146in}}%
\pgfpathlineto{\pgfqpoint{0.837632in}{1.710091in}}%
\pgfpathlineto{\pgfqpoint{0.848538in}{1.666447in}}%
\pgfpathlineto{\pgfqpoint{0.859444in}{1.654030in}}%
\pgfpathlineto{\pgfqpoint{0.870350in}{1.642009in}}%
\pgfpathlineto{\pgfqpoint{0.881256in}{1.636176in}}%
\pgfpathlineto{\pgfqpoint{0.892162in}{1.636176in}}%
\pgfpathlineto{\pgfqpoint{0.903068in}{1.599171in}}%
\pgfpathlineto{\pgfqpoint{0.913974in}{1.553164in}}%
\pgfpathlineto{\pgfqpoint{0.924880in}{1.553164in}}%
\pgfpathlineto{\pgfqpoint{0.935786in}{1.497915in}}%
\pgfpathlineto{\pgfqpoint{0.946692in}{1.497915in}}%
\pgfpathlineto{\pgfqpoint{0.957598in}{1.497915in}}%
\pgfpathlineto{\pgfqpoint{0.968504in}{1.482684in}}%
\pgfpathlineto{\pgfqpoint{0.979410in}{1.482684in}}%
\pgfpathlineto{\pgfqpoint{0.990316in}{1.402070in}}%
\pgfpathlineto{\pgfqpoint{1.001222in}{1.402070in}}%
\pgfpathlineto{\pgfqpoint{1.012127in}{1.402070in}}%
\pgfpathlineto{\pgfqpoint{1.023033in}{1.402070in}}%
\pgfpathlineto{\pgfqpoint{1.033939in}{1.400260in}}%
\pgfpathlineto{\pgfqpoint{1.044845in}{1.171128in}}%
\pgfpathlineto{\pgfqpoint{1.055751in}{1.171128in}}%
\pgfpathlineto{\pgfqpoint{1.066657in}{1.171128in}}%
\pgfpathlineto{\pgfqpoint{1.077563in}{1.149261in}}%
\pgfpathlineto{\pgfqpoint{1.088469in}{1.149261in}}%
\pgfpathlineto{\pgfqpoint{1.099375in}{1.149261in}}%
\pgfpathlineto{\pgfqpoint{1.110281in}{1.149261in}}%
\pgfpathlineto{\pgfqpoint{1.121187in}{1.149261in}}%
\pgfpathlineto{\pgfqpoint{1.132093in}{1.149261in}}%
\pgfpathlineto{\pgfqpoint{1.142999in}{1.149261in}}%
\pgfpathlineto{\pgfqpoint{1.153905in}{1.149261in}}%
\pgfpathlineto{\pgfqpoint{1.164811in}{1.149261in}}%
\pgfpathlineto{\pgfqpoint{1.175717in}{1.068636in}}%
\pgfpathlineto{\pgfqpoint{1.186623in}{1.068636in}}%
\pgfpathlineto{\pgfqpoint{1.197529in}{1.068636in}}%
\pgfpathlineto{\pgfqpoint{1.208435in}{1.068636in}}%
\pgfpathlineto{\pgfqpoint{1.219341in}{1.068636in}}%
\pgfpathlineto{\pgfqpoint{1.230247in}{1.068636in}}%
\pgfpathlineto{\pgfqpoint{1.241153in}{1.068636in}}%
\pgfpathlineto{\pgfqpoint{1.252059in}{1.068636in}}%
\pgfpathlineto{\pgfqpoint{1.262965in}{1.068636in}}%
\pgfpathlineto{\pgfqpoint{1.273871in}{1.068636in}}%
\pgfpathlineto{\pgfqpoint{1.284777in}{1.068636in}}%
\pgfpathlineto{\pgfqpoint{1.295683in}{1.068636in}}%
\pgfpathlineto{\pgfqpoint{1.306589in}{1.068636in}}%
\pgfpathlineto{\pgfqpoint{1.317495in}{1.068636in}}%
\pgfpathlineto{\pgfqpoint{1.328401in}{1.068636in}}%
\pgfpathlineto{\pgfqpoint{1.339307in}{1.068636in}}%
\pgfpathlineto{\pgfqpoint{1.350213in}{1.068636in}}%
\pgfpathlineto{\pgfqpoint{1.361119in}{1.068636in}}%
\pgfpathlineto{\pgfqpoint{1.372025in}{1.068636in}}%
\pgfpathlineto{\pgfqpoint{1.382931in}{1.068636in}}%
\pgfpathlineto{\pgfqpoint{1.393837in}{1.068636in}}%
\pgfpathlineto{\pgfqpoint{1.404743in}{1.068636in}}%
\pgfpathlineto{\pgfqpoint{1.415649in}{1.068636in}}%
\pgfpathlineto{\pgfqpoint{1.426555in}{1.068636in}}%
\pgfpathlineto{\pgfqpoint{1.437461in}{1.016326in}}%
\pgfpathlineto{\pgfqpoint{1.448367in}{1.016326in}}%
\pgfpathlineto{\pgfqpoint{1.459273in}{1.016326in}}%
\pgfpathlineto{\pgfqpoint{1.470179in}{1.016326in}}%
\pgfpathlineto{\pgfqpoint{1.481085in}{1.016326in}}%
\pgfpathlineto{\pgfqpoint{1.491991in}{1.016326in}}%
\pgfpathlineto{\pgfqpoint{1.502896in}{1.016326in}}%
\pgfpathlineto{\pgfqpoint{1.513802in}{1.016326in}}%
\pgfpathlineto{\pgfqpoint{1.524708in}{0.990943in}}%
\pgfpathlineto{\pgfqpoint{1.535614in}{0.990943in}}%
\pgfpathlineto{\pgfqpoint{1.546520in}{0.990943in}}%
\pgfpathlineto{\pgfqpoint{1.557426in}{0.990943in}}%
\pgfpathlineto{\pgfqpoint{1.568332in}{0.990943in}}%
\pgfpathlineto{\pgfqpoint{1.579238in}{0.990943in}}%
\pgfpathlineto{\pgfqpoint{1.590144in}{0.990943in}}%
\pgfpathlineto{\pgfqpoint{1.601050in}{0.990943in}}%
\pgfpathlineto{\pgfqpoint{1.611956in}{0.983762in}}%
\pgfpathlineto{\pgfqpoint{1.622862in}{0.983762in}}%
\pgfpathlineto{\pgfqpoint{1.633768in}{0.983762in}}%
\pgfpathlineto{\pgfqpoint{1.644674in}{0.983762in}}%
\pgfpathlineto{\pgfqpoint{1.655580in}{0.983762in}}%
\pgfpathlineto{\pgfqpoint{1.666486in}{0.983762in}}%
\pgfpathlineto{\pgfqpoint{1.677392in}{0.983762in}}%
\pgfpathlineto{\pgfqpoint{1.688298in}{0.983762in}}%
\pgfpathlineto{\pgfqpoint{1.699204in}{0.983762in}}%
\pgfpathlineto{\pgfqpoint{1.710110in}{0.983762in}}%
\pgfpathlineto{\pgfqpoint{1.721016in}{0.983762in}}%
\pgfpathlineto{\pgfqpoint{1.731922in}{0.983762in}}%
\pgfpathlineto{\pgfqpoint{1.742828in}{0.973305in}}%
\pgfpathlineto{\pgfqpoint{1.753734in}{0.973305in}}%
\pgfpathlineto{\pgfqpoint{1.764640in}{0.973305in}}%
\pgfpathlineto{\pgfqpoint{1.775546in}{0.973305in}}%
\pgfpathlineto{\pgfqpoint{1.786452in}{0.973305in}}%
\pgfpathlineto{\pgfqpoint{1.797358in}{0.973305in}}%
\pgfpathlineto{\pgfqpoint{1.808264in}{0.973305in}}%
\pgfpathlineto{\pgfqpoint{1.819170in}{0.973305in}}%
\pgfpathlineto{\pgfqpoint{1.830076in}{0.973305in}}%
\pgfpathlineto{\pgfqpoint{1.840982in}{0.951018in}}%
\pgfpathlineto{\pgfqpoint{1.851888in}{0.951018in}}%
\pgfpathlineto{\pgfqpoint{1.862794in}{0.951018in}}%
\pgfpathlineto{\pgfqpoint{1.873700in}{0.951018in}}%
\pgfpathlineto{\pgfqpoint{1.884606in}{0.951018in}}%
\pgfpathlineto{\pgfqpoint{1.895512in}{0.951018in}}%
\pgfpathlineto{\pgfqpoint{1.906418in}{0.951018in}}%
\pgfpathlineto{\pgfqpoint{1.917324in}{0.951018in}}%
\pgfpathlineto{\pgfqpoint{1.928230in}{0.951018in}}%
\pgfpathlineto{\pgfqpoint{1.939136in}{0.951018in}}%
\pgfpathlineto{\pgfqpoint{1.950042in}{0.951018in}}%
\pgfpathlineto{\pgfqpoint{1.960948in}{0.951018in}}%
\pgfpathlineto{\pgfqpoint{1.971854in}{0.951018in}}%
\pgfpathlineto{\pgfqpoint{1.982759in}{0.951018in}}%
\pgfpathlineto{\pgfqpoint{1.993665in}{0.951018in}}%
\pgfpathlineto{\pgfqpoint{2.004571in}{0.951018in}}%
\pgfpathlineto{\pgfqpoint{2.015477in}{0.951018in}}%
\pgfpathlineto{\pgfqpoint{2.026383in}{0.951018in}}%
\pgfpathlineto{\pgfqpoint{2.037289in}{0.951018in}}%
\pgfpathlineto{\pgfqpoint{2.048195in}{0.951018in}}%
\pgfpathlineto{\pgfqpoint{2.059101in}{0.951018in}}%
\pgfpathlineto{\pgfqpoint{2.070007in}{0.951018in}}%
\pgfpathlineto{\pgfqpoint{2.080913in}{0.951018in}}%
\pgfpathlineto{\pgfqpoint{2.091819in}{0.946834in}}%
\pgfpathlineto{\pgfqpoint{2.102725in}{0.946834in}}%
\pgfpathlineto{\pgfqpoint{2.113631in}{0.946834in}}%
\pgfpathlineto{\pgfqpoint{2.124537in}{0.946834in}}%
\pgfpathlineto{\pgfqpoint{2.135443in}{0.946834in}}%
\pgfpathlineto{\pgfqpoint{2.146349in}{0.946834in}}%
\pgfpathlineto{\pgfqpoint{2.157255in}{0.946006in}}%
\pgfpathlineto{\pgfqpoint{2.168161in}{0.946006in}}%
\pgfpathlineto{\pgfqpoint{2.179067in}{0.946006in}}%
\pgfpathlineto{\pgfqpoint{2.189973in}{0.946006in}}%
\pgfpathlineto{\pgfqpoint{2.200879in}{0.946006in}}%
\pgfpathlineto{\pgfqpoint{2.211785in}{0.946006in}}%
\pgfpathlineto{\pgfqpoint{2.222691in}{0.946006in}}%
\pgfpathlineto{\pgfqpoint{2.233597in}{0.946006in}}%
\pgfpathlineto{\pgfqpoint{2.244503in}{0.946006in}}%
\pgfpathlineto{\pgfqpoint{2.255409in}{0.946006in}}%
\pgfpathlineto{\pgfqpoint{2.266315in}{0.946006in}}%
\pgfpathlineto{\pgfqpoint{2.277221in}{0.946006in}}%
\pgfpathlineto{\pgfqpoint{2.288127in}{0.946006in}}%
\pgfpathlineto{\pgfqpoint{2.299033in}{0.946006in}}%
\pgfpathlineto{\pgfqpoint{2.309939in}{0.946006in}}%
\pgfpathlineto{\pgfqpoint{2.320845in}{0.946006in}}%
\pgfpathlineto{\pgfqpoint{2.331751in}{0.946006in}}%
\pgfpathlineto{\pgfqpoint{2.342657in}{0.946006in}}%
\pgfpathlineto{\pgfqpoint{2.353563in}{0.946006in}}%
\pgfpathlineto{\pgfqpoint{2.364469in}{0.946006in}}%
\pgfpathlineto{\pgfqpoint{2.375375in}{0.946006in}}%
\pgfpathlineto{\pgfqpoint{2.386281in}{0.946006in}}%
\pgfpathlineto{\pgfqpoint{2.397187in}{0.946006in}}%
\pgfpathlineto{\pgfqpoint{2.408093in}{0.946006in}}%
\pgfpathlineto{\pgfqpoint{2.418999in}{0.946006in}}%
\pgfpathlineto{\pgfqpoint{2.429905in}{0.946006in}}%
\pgfpathlineto{\pgfqpoint{2.440811in}{0.946006in}}%
\pgfpathlineto{\pgfqpoint{2.451717in}{0.946006in}}%
\pgfpathlineto{\pgfqpoint{2.462623in}{0.946006in}}%
\pgfpathlineto{\pgfqpoint{2.473528in}{0.946006in}}%
\pgfpathlineto{\pgfqpoint{2.484434in}{0.946006in}}%
\pgfpathlineto{\pgfqpoint{2.495340in}{0.946006in}}%
\pgfpathlineto{\pgfqpoint{2.506246in}{0.946006in}}%
\pgfpathlineto{\pgfqpoint{2.517152in}{0.946006in}}%
\pgfpathlineto{\pgfqpoint{2.528058in}{0.946006in}}%
\pgfpathlineto{\pgfqpoint{2.538964in}{0.946006in}}%
\pgfpathlineto{\pgfqpoint{2.549870in}{0.946006in}}%
\pgfpathlineto{\pgfqpoint{2.560776in}{0.946006in}}%
\pgfpathlineto{\pgfqpoint{2.571682in}{0.946006in}}%
\pgfpathlineto{\pgfqpoint{2.582588in}{0.946006in}}%
\pgfpathlineto{\pgfqpoint{2.593494in}{0.946006in}}%
\pgfpathlineto{\pgfqpoint{2.604400in}{0.889803in}}%
\pgfpathlineto{\pgfqpoint{2.615306in}{0.889803in}}%
\pgfpathlineto{\pgfqpoint{2.626212in}{0.889770in}}%
\pgfpathlineto{\pgfqpoint{2.637118in}{0.889770in}}%
\pgfpathlineto{\pgfqpoint{2.648024in}{0.889770in}}%
\pgfpathlineto{\pgfqpoint{2.658930in}{0.889770in}}%
\pgfpathlineto{\pgfqpoint{2.669836in}{0.889770in}}%
\pgfpathlineto{\pgfqpoint{2.680742in}{0.889770in}}%
\pgfpathlineto{\pgfqpoint{2.691648in}{0.889770in}}%
\pgfpathlineto{\pgfqpoint{2.702554in}{0.889770in}}%
\pgfpathlineto{\pgfqpoint{2.713460in}{0.889770in}}%
\pgfpathlineto{\pgfqpoint{2.724366in}{0.889770in}}%
\pgfpathlineto{\pgfqpoint{2.735272in}{0.889770in}}%
\pgfpathlineto{\pgfqpoint{2.746178in}{0.889770in}}%
\pgfpathlineto{\pgfqpoint{2.757084in}{0.889770in}}%
\pgfpathlineto{\pgfqpoint{2.767990in}{0.889770in}}%
\pgfpathlineto{\pgfqpoint{2.778896in}{0.889770in}}%
\pgfpathlineto{\pgfqpoint{2.789802in}{0.889770in}}%
\pgfpathlineto{\pgfqpoint{2.800708in}{0.889770in}}%
\pgfpathlineto{\pgfqpoint{2.811614in}{0.889770in}}%
\pgfpathlineto{\pgfqpoint{2.822520in}{0.889770in}}%
\pgfpathlineto{\pgfqpoint{2.833426in}{0.889770in}}%
\pgfpathlineto{\pgfqpoint{2.844332in}{0.889770in}}%
\pgfpathlineto{\pgfqpoint{2.855238in}{0.889770in}}%
\pgfpathlineto{\pgfqpoint{2.866144in}{0.889770in}}%
\pgfpathlineto{\pgfqpoint{2.877050in}{0.889770in}}%
\pgfpathlineto{\pgfqpoint{2.887956in}{0.889770in}}%
\pgfpathlineto{\pgfqpoint{2.898862in}{0.889770in}}%
\pgfpathlineto{\pgfqpoint{2.909768in}{0.889770in}}%
\pgfpathlineto{\pgfqpoint{2.920674in}{0.889770in}}%
\pgfpathlineto{\pgfqpoint{2.931580in}{0.889770in}}%
\pgfpathlineto{\pgfqpoint{2.931580in}{0.609303in}}%
\pgfpathlineto{\pgfqpoint{2.931580in}{0.609303in}}%
\pgfpathlineto{\pgfqpoint{2.920674in}{0.609303in}}%
\pgfpathlineto{\pgfqpoint{2.909768in}{0.609303in}}%
\pgfpathlineto{\pgfqpoint{2.898862in}{0.609303in}}%
\pgfpathlineto{\pgfqpoint{2.887956in}{0.609303in}}%
\pgfpathlineto{\pgfqpoint{2.877050in}{0.609303in}}%
\pgfpathlineto{\pgfqpoint{2.866144in}{0.609303in}}%
\pgfpathlineto{\pgfqpoint{2.855238in}{0.609303in}}%
\pgfpathlineto{\pgfqpoint{2.844332in}{0.609303in}}%
\pgfpathlineto{\pgfqpoint{2.833426in}{0.609303in}}%
\pgfpathlineto{\pgfqpoint{2.822520in}{0.609303in}}%
\pgfpathlineto{\pgfqpoint{2.811614in}{0.609303in}}%
\pgfpathlineto{\pgfqpoint{2.800708in}{0.609303in}}%
\pgfpathlineto{\pgfqpoint{2.789802in}{0.609303in}}%
\pgfpathlineto{\pgfqpoint{2.778896in}{0.609303in}}%
\pgfpathlineto{\pgfqpoint{2.767990in}{0.609303in}}%
\pgfpathlineto{\pgfqpoint{2.757084in}{0.609303in}}%
\pgfpathlineto{\pgfqpoint{2.746178in}{0.609303in}}%
\pgfpathlineto{\pgfqpoint{2.735272in}{0.609303in}}%
\pgfpathlineto{\pgfqpoint{2.724366in}{0.609303in}}%
\pgfpathlineto{\pgfqpoint{2.713460in}{0.609303in}}%
\pgfpathlineto{\pgfqpoint{2.702554in}{0.609303in}}%
\pgfpathlineto{\pgfqpoint{2.691648in}{0.609303in}}%
\pgfpathlineto{\pgfqpoint{2.680742in}{0.609303in}}%
\pgfpathlineto{\pgfqpoint{2.669836in}{0.609303in}}%
\pgfpathlineto{\pgfqpoint{2.658930in}{0.609303in}}%
\pgfpathlineto{\pgfqpoint{2.648024in}{0.609303in}}%
\pgfpathlineto{\pgfqpoint{2.637118in}{0.609303in}}%
\pgfpathlineto{\pgfqpoint{2.626212in}{0.609303in}}%
\pgfpathlineto{\pgfqpoint{2.615306in}{0.609709in}}%
\pgfpathlineto{\pgfqpoint{2.604400in}{0.609709in}}%
\pgfpathlineto{\pgfqpoint{2.593494in}{0.790103in}}%
\pgfpathlineto{\pgfqpoint{2.582588in}{0.790103in}}%
\pgfpathlineto{\pgfqpoint{2.571682in}{0.790103in}}%
\pgfpathlineto{\pgfqpoint{2.560776in}{0.790103in}}%
\pgfpathlineto{\pgfqpoint{2.549870in}{0.790103in}}%
\pgfpathlineto{\pgfqpoint{2.538964in}{0.790103in}}%
\pgfpathlineto{\pgfqpoint{2.528058in}{0.790103in}}%
\pgfpathlineto{\pgfqpoint{2.517152in}{0.790103in}}%
\pgfpathlineto{\pgfqpoint{2.506246in}{0.790103in}}%
\pgfpathlineto{\pgfqpoint{2.495340in}{0.790103in}}%
\pgfpathlineto{\pgfqpoint{2.484434in}{0.790103in}}%
\pgfpathlineto{\pgfqpoint{2.473528in}{0.790103in}}%
\pgfpathlineto{\pgfqpoint{2.462623in}{0.790103in}}%
\pgfpathlineto{\pgfqpoint{2.451717in}{0.790103in}}%
\pgfpathlineto{\pgfqpoint{2.440811in}{0.790103in}}%
\pgfpathlineto{\pgfqpoint{2.429905in}{0.790103in}}%
\pgfpathlineto{\pgfqpoint{2.418999in}{0.790103in}}%
\pgfpathlineto{\pgfqpoint{2.408093in}{0.790103in}}%
\pgfpathlineto{\pgfqpoint{2.397187in}{0.790103in}}%
\pgfpathlineto{\pgfqpoint{2.386281in}{0.790103in}}%
\pgfpathlineto{\pgfqpoint{2.375375in}{0.790103in}}%
\pgfpathlineto{\pgfqpoint{2.364469in}{0.790103in}}%
\pgfpathlineto{\pgfqpoint{2.353563in}{0.790103in}}%
\pgfpathlineto{\pgfqpoint{2.342657in}{0.790103in}}%
\pgfpathlineto{\pgfqpoint{2.331751in}{0.790103in}}%
\pgfpathlineto{\pgfqpoint{2.320845in}{0.790103in}}%
\pgfpathlineto{\pgfqpoint{2.309939in}{0.790103in}}%
\pgfpathlineto{\pgfqpoint{2.299033in}{0.790103in}}%
\pgfpathlineto{\pgfqpoint{2.288127in}{0.790103in}}%
\pgfpathlineto{\pgfqpoint{2.277221in}{0.790103in}}%
\pgfpathlineto{\pgfqpoint{2.266315in}{0.790103in}}%
\pgfpathlineto{\pgfqpoint{2.255409in}{0.790103in}}%
\pgfpathlineto{\pgfqpoint{2.244503in}{0.790103in}}%
\pgfpathlineto{\pgfqpoint{2.233597in}{0.790103in}}%
\pgfpathlineto{\pgfqpoint{2.222691in}{0.790103in}}%
\pgfpathlineto{\pgfqpoint{2.211785in}{0.790103in}}%
\pgfpathlineto{\pgfqpoint{2.200879in}{0.790103in}}%
\pgfpathlineto{\pgfqpoint{2.189973in}{0.790103in}}%
\pgfpathlineto{\pgfqpoint{2.179067in}{0.790103in}}%
\pgfpathlineto{\pgfqpoint{2.168161in}{0.790103in}}%
\pgfpathlineto{\pgfqpoint{2.157255in}{0.790103in}}%
\pgfpathlineto{\pgfqpoint{2.146349in}{0.790679in}}%
\pgfpathlineto{\pgfqpoint{2.135443in}{0.790679in}}%
\pgfpathlineto{\pgfqpoint{2.124537in}{0.790679in}}%
\pgfpathlineto{\pgfqpoint{2.113631in}{0.790679in}}%
\pgfpathlineto{\pgfqpoint{2.102725in}{0.790679in}}%
\pgfpathlineto{\pgfqpoint{2.091819in}{0.790679in}}%
\pgfpathlineto{\pgfqpoint{2.080913in}{0.808614in}}%
\pgfpathlineto{\pgfqpoint{2.070007in}{0.808614in}}%
\pgfpathlineto{\pgfqpoint{2.059101in}{0.808614in}}%
\pgfpathlineto{\pgfqpoint{2.048195in}{0.808614in}}%
\pgfpathlineto{\pgfqpoint{2.037289in}{0.808614in}}%
\pgfpathlineto{\pgfqpoint{2.026383in}{0.808614in}}%
\pgfpathlineto{\pgfqpoint{2.015477in}{0.808614in}}%
\pgfpathlineto{\pgfqpoint{2.004571in}{0.808614in}}%
\pgfpathlineto{\pgfqpoint{1.993665in}{0.808614in}}%
\pgfpathlineto{\pgfqpoint{1.982759in}{0.808614in}}%
\pgfpathlineto{\pgfqpoint{1.971854in}{0.808614in}}%
\pgfpathlineto{\pgfqpoint{1.960948in}{0.808614in}}%
\pgfpathlineto{\pgfqpoint{1.950042in}{0.808614in}}%
\pgfpathlineto{\pgfqpoint{1.939136in}{0.808614in}}%
\pgfpathlineto{\pgfqpoint{1.928230in}{0.808614in}}%
\pgfpathlineto{\pgfqpoint{1.917324in}{0.808614in}}%
\pgfpathlineto{\pgfqpoint{1.906418in}{0.808614in}}%
\pgfpathlineto{\pgfqpoint{1.895512in}{0.808614in}}%
\pgfpathlineto{\pgfqpoint{1.884606in}{0.808614in}}%
\pgfpathlineto{\pgfqpoint{1.873700in}{0.808614in}}%
\pgfpathlineto{\pgfqpoint{1.862794in}{0.808614in}}%
\pgfpathlineto{\pgfqpoint{1.851888in}{0.808614in}}%
\pgfpathlineto{\pgfqpoint{1.840982in}{0.808614in}}%
\pgfpathlineto{\pgfqpoint{1.830076in}{0.820831in}}%
\pgfpathlineto{\pgfqpoint{1.819170in}{0.820831in}}%
\pgfpathlineto{\pgfqpoint{1.808264in}{0.820831in}}%
\pgfpathlineto{\pgfqpoint{1.797358in}{0.820831in}}%
\pgfpathlineto{\pgfqpoint{1.786452in}{0.820831in}}%
\pgfpathlineto{\pgfqpoint{1.775546in}{0.820831in}}%
\pgfpathlineto{\pgfqpoint{1.764640in}{0.820831in}}%
\pgfpathlineto{\pgfqpoint{1.753734in}{0.820831in}}%
\pgfpathlineto{\pgfqpoint{1.742828in}{0.820831in}}%
\pgfpathlineto{\pgfqpoint{1.731922in}{0.871455in}}%
\pgfpathlineto{\pgfqpoint{1.721016in}{0.871455in}}%
\pgfpathlineto{\pgfqpoint{1.710110in}{0.871455in}}%
\pgfpathlineto{\pgfqpoint{1.699204in}{0.871455in}}%
\pgfpathlineto{\pgfqpoint{1.688298in}{0.871455in}}%
\pgfpathlineto{\pgfqpoint{1.677392in}{0.871455in}}%
\pgfpathlineto{\pgfqpoint{1.666486in}{0.871455in}}%
\pgfpathlineto{\pgfqpoint{1.655580in}{0.871455in}}%
\pgfpathlineto{\pgfqpoint{1.644674in}{0.871455in}}%
\pgfpathlineto{\pgfqpoint{1.633768in}{0.871455in}}%
\pgfpathlineto{\pgfqpoint{1.622862in}{0.871455in}}%
\pgfpathlineto{\pgfqpoint{1.611956in}{0.871455in}}%
\pgfpathlineto{\pgfqpoint{1.601050in}{0.888785in}}%
\pgfpathlineto{\pgfqpoint{1.590144in}{0.888785in}}%
\pgfpathlineto{\pgfqpoint{1.579238in}{0.888785in}}%
\pgfpathlineto{\pgfqpoint{1.568332in}{0.888785in}}%
\pgfpathlineto{\pgfqpoint{1.557426in}{0.888785in}}%
\pgfpathlineto{\pgfqpoint{1.546520in}{0.888785in}}%
\pgfpathlineto{\pgfqpoint{1.535614in}{0.888785in}}%
\pgfpathlineto{\pgfqpoint{1.524708in}{0.888785in}}%
\pgfpathlineto{\pgfqpoint{1.513802in}{0.958565in}}%
\pgfpathlineto{\pgfqpoint{1.502896in}{0.958565in}}%
\pgfpathlineto{\pgfqpoint{1.491991in}{0.958565in}}%
\pgfpathlineto{\pgfqpoint{1.481085in}{0.958565in}}%
\pgfpathlineto{\pgfqpoint{1.470179in}{0.958565in}}%
\pgfpathlineto{\pgfqpoint{1.459273in}{0.958565in}}%
\pgfpathlineto{\pgfqpoint{1.448367in}{0.958565in}}%
\pgfpathlineto{\pgfqpoint{1.437461in}{0.958565in}}%
\pgfpathlineto{\pgfqpoint{1.426555in}{0.993760in}}%
\pgfpathlineto{\pgfqpoint{1.415649in}{0.993760in}}%
\pgfpathlineto{\pgfqpoint{1.404743in}{0.993760in}}%
\pgfpathlineto{\pgfqpoint{1.393837in}{0.993760in}}%
\pgfpathlineto{\pgfqpoint{1.382931in}{0.993760in}}%
\pgfpathlineto{\pgfqpoint{1.372025in}{0.993760in}}%
\pgfpathlineto{\pgfqpoint{1.361119in}{0.993760in}}%
\pgfpathlineto{\pgfqpoint{1.350213in}{0.993760in}}%
\pgfpathlineto{\pgfqpoint{1.339307in}{0.993760in}}%
\pgfpathlineto{\pgfqpoint{1.328401in}{0.993760in}}%
\pgfpathlineto{\pgfqpoint{1.317495in}{0.993760in}}%
\pgfpathlineto{\pgfqpoint{1.306589in}{0.993760in}}%
\pgfpathlineto{\pgfqpoint{1.295683in}{0.993760in}}%
\pgfpathlineto{\pgfqpoint{1.284777in}{0.993760in}}%
\pgfpathlineto{\pgfqpoint{1.273871in}{0.993760in}}%
\pgfpathlineto{\pgfqpoint{1.262965in}{0.993760in}}%
\pgfpathlineto{\pgfqpoint{1.252059in}{0.993760in}}%
\pgfpathlineto{\pgfqpoint{1.241153in}{0.993760in}}%
\pgfpathlineto{\pgfqpoint{1.230247in}{0.993760in}}%
\pgfpathlineto{\pgfqpoint{1.219341in}{0.993760in}}%
\pgfpathlineto{\pgfqpoint{1.208435in}{0.993760in}}%
\pgfpathlineto{\pgfqpoint{1.197529in}{0.993760in}}%
\pgfpathlineto{\pgfqpoint{1.186623in}{0.993760in}}%
\pgfpathlineto{\pgfqpoint{1.175717in}{0.993760in}}%
\pgfpathlineto{\pgfqpoint{1.164811in}{1.042390in}}%
\pgfpathlineto{\pgfqpoint{1.153905in}{1.042390in}}%
\pgfpathlineto{\pgfqpoint{1.142999in}{1.042390in}}%
\pgfpathlineto{\pgfqpoint{1.132093in}{1.042390in}}%
\pgfpathlineto{\pgfqpoint{1.121187in}{1.042390in}}%
\pgfpathlineto{\pgfqpoint{1.110281in}{1.042390in}}%
\pgfpathlineto{\pgfqpoint{1.099375in}{1.042390in}}%
\pgfpathlineto{\pgfqpoint{1.088469in}{1.042390in}}%
\pgfpathlineto{\pgfqpoint{1.077563in}{1.042390in}}%
\pgfpathlineto{\pgfqpoint{1.066657in}{1.083657in}}%
\pgfpathlineto{\pgfqpoint{1.055751in}{1.083657in}}%
\pgfpathlineto{\pgfqpoint{1.044845in}{1.083657in}}%
\pgfpathlineto{\pgfqpoint{1.033939in}{1.198798in}}%
\pgfpathlineto{\pgfqpoint{1.023033in}{1.209477in}}%
\pgfpathlineto{\pgfqpoint{1.012127in}{1.209477in}}%
\pgfpathlineto{\pgfqpoint{1.001222in}{1.209477in}}%
\pgfpathlineto{\pgfqpoint{0.990316in}{1.209477in}}%
\pgfpathlineto{\pgfqpoint{0.979410in}{1.228891in}}%
\pgfpathlineto{\pgfqpoint{0.968504in}{1.228891in}}%
\pgfpathlineto{\pgfqpoint{0.957598in}{1.309780in}}%
\pgfpathlineto{\pgfqpoint{0.946692in}{1.309780in}}%
\pgfpathlineto{\pgfqpoint{0.935786in}{1.309780in}}%
\pgfpathlineto{\pgfqpoint{0.924880in}{1.391246in}}%
\pgfpathlineto{\pgfqpoint{0.913974in}{1.391246in}}%
\pgfpathlineto{\pgfqpoint{0.903068in}{1.442158in}}%
\pgfpathlineto{\pgfqpoint{0.892162in}{1.454968in}}%
\pgfpathlineto{\pgfqpoint{0.881256in}{1.454968in}}%
\pgfpathlineto{\pgfqpoint{0.870350in}{1.463868in}}%
\pgfpathlineto{\pgfqpoint{0.859444in}{1.483433in}}%
\pgfpathlineto{\pgfqpoint{0.848538in}{1.500259in}}%
\pgfpathlineto{\pgfqpoint{0.837632in}{1.599631in}}%
\pgfpathlineto{\pgfqpoint{0.826726in}{1.608394in}}%
\pgfpathlineto{\pgfqpoint{0.815820in}{1.635983in}}%
\pgfpathlineto{\pgfqpoint{0.804914in}{1.651798in}}%
\pgfpathlineto{\pgfqpoint{0.794008in}{1.670444in}}%
\pgfpathlineto{\pgfqpoint{0.783102in}{1.679221in}}%
\pgfpathlineto{\pgfqpoint{0.772196in}{1.682211in}}%
\pgfpathlineto{\pgfqpoint{0.761290in}{1.682711in}}%
\pgfpathlineto{\pgfqpoint{0.750384in}{1.685526in}}%
\pgfpathlineto{\pgfqpoint{0.739478in}{1.685526in}}%
\pgfpathlineto{\pgfqpoint{0.728572in}{1.685526in}}%
\pgfpathlineto{\pgfqpoint{0.717666in}{1.685526in}}%
\pgfpathlineto{\pgfqpoint{0.706760in}{1.685526in}}%
\pgfpathlineto{\pgfqpoint{0.695854in}{1.685526in}}%
\pgfpathlineto{\pgfqpoint{0.684948in}{1.691577in}}%
\pgfpathlineto{\pgfqpoint{0.674042in}{1.691577in}}%
\pgfpathlineto{\pgfqpoint{0.663136in}{1.746818in}}%
\pgfpathlineto{\pgfqpoint{0.652230in}{1.753110in}}%
\pgfpathlineto{\pgfqpoint{0.641324in}{1.753112in}}%
\pgfpathlineto{\pgfqpoint{0.630418in}{1.753144in}}%
\pgfpathlineto{\pgfqpoint{0.619512in}{1.753144in}}%
\pgfpathlineto{\pgfqpoint{0.608606in}{1.753262in}}%
\pgfpathlineto{\pgfqpoint{0.597700in}{1.769438in}}%
\pgfpathlineto{\pgfqpoint{0.586794in}{1.769950in}}%
\pgfpathlineto{\pgfqpoint{0.575888in}{1.769975in}}%
\pgfpathlineto{\pgfqpoint{0.564982in}{1.771180in}}%
\pgfpathlineto{\pgfqpoint{0.554076in}{1.771180in}}%
\pgfpathlineto{\pgfqpoint{0.543170in}{1.771256in}}%
\pgfpathclose%
\pgfusepath{fill}%
\end{pgfscope}%
\begin{pgfscope}%
\pgfpathrectangle{\pgfqpoint{0.423750in}{0.261892in}}{\pgfqpoint{2.627250in}{1.581827in}}%
\pgfusepath{clip}%
\pgfsetroundcap%
\pgfsetroundjoin%
\pgfsetlinewidth{1.505625pt}%
\definecolor{currentstroke}{rgb}{0.121569,0.466667,0.705882}%
\pgfsetstrokecolor{currentstroke}%
\pgfsetdash{}{0pt}%
\pgfpathmoveto{\pgfqpoint{0.543170in}{1.768211in}}%
\pgfpathlineto{\pgfqpoint{0.554076in}{1.761610in}}%
\pgfpathlineto{\pgfqpoint{0.564982in}{1.756870in}}%
\pgfpathlineto{\pgfqpoint{0.630418in}{1.755996in}}%
\pgfpathlineto{\pgfqpoint{0.641324in}{1.742340in}}%
\pgfpathlineto{\pgfqpoint{0.695854in}{1.741851in}}%
\pgfpathlineto{\pgfqpoint{0.706760in}{1.739799in}}%
\pgfpathlineto{\pgfqpoint{0.717666in}{1.722661in}}%
\pgfpathlineto{\pgfqpoint{0.728572in}{1.717775in}}%
\pgfpathlineto{\pgfqpoint{0.739478in}{1.695406in}}%
\pgfpathlineto{\pgfqpoint{0.761290in}{1.695406in}}%
\pgfpathlineto{\pgfqpoint{0.772196in}{1.684284in}}%
\pgfpathlineto{\pgfqpoint{0.783102in}{1.684118in}}%
\pgfpathlineto{\pgfqpoint{0.794008in}{1.673332in}}%
\pgfpathlineto{\pgfqpoint{0.804914in}{1.673027in}}%
\pgfpathlineto{\pgfqpoint{0.815820in}{1.663832in}}%
\pgfpathlineto{\pgfqpoint{0.826726in}{1.663832in}}%
\pgfpathlineto{\pgfqpoint{0.837632in}{1.650022in}}%
\pgfpathlineto{\pgfqpoint{0.848538in}{1.650022in}}%
\pgfpathlineto{\pgfqpoint{0.859444in}{1.647315in}}%
\pgfpathlineto{\pgfqpoint{0.924880in}{1.647315in}}%
\pgfpathlineto{\pgfqpoint{0.935786in}{1.641616in}}%
\pgfpathlineto{\pgfqpoint{1.088469in}{1.641080in}}%
\pgfpathlineto{\pgfqpoint{1.099375in}{1.628591in}}%
\pgfpathlineto{\pgfqpoint{1.448367in}{1.627719in}}%
\pgfpathlineto{\pgfqpoint{1.459273in}{1.622586in}}%
\pgfpathlineto{\pgfqpoint{1.644674in}{1.622519in}}%
\pgfpathlineto{\pgfqpoint{1.655580in}{1.618045in}}%
\pgfpathlineto{\pgfqpoint{1.710110in}{1.618045in}}%
\pgfpathlineto{\pgfqpoint{1.721016in}{1.599568in}}%
\pgfpathlineto{\pgfqpoint{1.764640in}{1.599568in}}%
\pgfpathlineto{\pgfqpoint{1.775546in}{1.580806in}}%
\pgfpathlineto{\pgfqpoint{1.862794in}{1.580806in}}%
\pgfpathlineto{\pgfqpoint{1.873700in}{1.541810in}}%
\pgfpathlineto{\pgfqpoint{2.462623in}{1.541810in}}%
\pgfpathlineto{\pgfqpoint{2.473528in}{1.528797in}}%
\pgfpathlineto{\pgfqpoint{2.495340in}{1.528797in}}%
\pgfpathlineto{\pgfqpoint{2.506246in}{1.526380in}}%
\pgfpathlineto{\pgfqpoint{2.538964in}{1.526380in}}%
\pgfpathlineto{\pgfqpoint{2.549870in}{1.481438in}}%
\pgfpathlineto{\pgfqpoint{2.811614in}{1.480997in}}%
\pgfpathlineto{\pgfqpoint{2.822520in}{1.435322in}}%
\pgfpathlineto{\pgfqpoint{2.898862in}{1.435322in}}%
\pgfpathlineto{\pgfqpoint{2.909768in}{1.362264in}}%
\pgfpathlineto{\pgfqpoint{2.931580in}{1.362264in}}%
\pgfpathlineto{\pgfqpoint{2.931580in}{1.362264in}}%
\pgfusepath{stroke}%
\end{pgfscope}%
\begin{pgfscope}%
\pgfpathrectangle{\pgfqpoint{0.423750in}{0.261892in}}{\pgfqpoint{2.627250in}{1.581827in}}%
\pgfusepath{clip}%
\pgfsetroundcap%
\pgfsetroundjoin%
\pgfsetlinewidth{1.505625pt}%
\definecolor{currentstroke}{rgb}{1.000000,0.498039,0.054902}%
\pgfsetstrokecolor{currentstroke}%
\pgfsetdash{}{0pt}%
\pgfpathmoveto{\pgfqpoint{0.543170in}{1.755301in}}%
\pgfpathlineto{\pgfqpoint{0.575888in}{1.754962in}}%
\pgfpathlineto{\pgfqpoint{0.586794in}{1.738110in}}%
\pgfpathlineto{\pgfqpoint{0.597700in}{1.737732in}}%
\pgfpathlineto{\pgfqpoint{0.608606in}{1.701565in}}%
\pgfpathlineto{\pgfqpoint{0.619512in}{1.698586in}}%
\pgfpathlineto{\pgfqpoint{0.641324in}{1.698410in}}%
\pgfpathlineto{\pgfqpoint{0.652230in}{1.673571in}}%
\pgfpathlineto{\pgfqpoint{0.739478in}{1.673571in}}%
\pgfpathlineto{\pgfqpoint{0.750384in}{1.639054in}}%
\pgfpathlineto{\pgfqpoint{0.794008in}{1.639054in}}%
\pgfpathlineto{\pgfqpoint{0.804914in}{1.636609in}}%
\pgfpathlineto{\pgfqpoint{0.859444in}{1.636609in}}%
\pgfpathlineto{\pgfqpoint{0.870350in}{1.633289in}}%
\pgfpathlineto{\pgfqpoint{0.881256in}{1.588166in}}%
\pgfpathlineto{\pgfqpoint{0.892162in}{1.588166in}}%
\pgfpathlineto{\pgfqpoint{0.903068in}{1.582651in}}%
\pgfpathlineto{\pgfqpoint{0.913974in}{1.529441in}}%
\pgfpathlineto{\pgfqpoint{1.033939in}{1.529441in}}%
\pgfpathlineto{\pgfqpoint{1.044845in}{1.514890in}}%
\pgfpathlineto{\pgfqpoint{1.055751in}{1.514890in}}%
\pgfpathlineto{\pgfqpoint{1.066657in}{1.494219in}}%
\pgfpathlineto{\pgfqpoint{1.077563in}{1.452837in}}%
\pgfpathlineto{\pgfqpoint{1.121187in}{1.452837in}}%
\pgfpathlineto{\pgfqpoint{1.132093in}{1.442423in}}%
\pgfpathlineto{\pgfqpoint{1.153905in}{1.442423in}}%
\pgfpathlineto{\pgfqpoint{1.164811in}{1.222443in}}%
\pgfpathlineto{\pgfqpoint{1.230247in}{1.222443in}}%
\pgfpathlineto{\pgfqpoint{1.241153in}{1.124407in}}%
\pgfpathlineto{\pgfqpoint{1.382931in}{1.124407in}}%
\pgfpathlineto{\pgfqpoint{1.393837in}{0.805934in}}%
\pgfpathlineto{\pgfqpoint{1.415649in}{0.804707in}}%
\pgfpathlineto{\pgfqpoint{1.426555in}{0.784117in}}%
\pgfpathlineto{\pgfqpoint{1.928230in}{0.784117in}}%
\pgfpathlineto{\pgfqpoint{1.939136in}{0.777076in}}%
\pgfpathlineto{\pgfqpoint{2.517152in}{0.777076in}}%
\pgfpathlineto{\pgfqpoint{2.528058in}{0.710736in}}%
\pgfpathlineto{\pgfqpoint{2.855238in}{0.710736in}}%
\pgfpathlineto{\pgfqpoint{2.866144in}{0.424863in}}%
\pgfpathlineto{\pgfqpoint{2.931580in}{0.424863in}}%
\pgfpathlineto{\pgfqpoint{2.931580in}{0.424863in}}%
\pgfusepath{stroke}%
\end{pgfscope}%
\begin{pgfscope}%
\pgfpathrectangle{\pgfqpoint{0.423750in}{0.261892in}}{\pgfqpoint{2.627250in}{1.581827in}}%
\pgfusepath{clip}%
\pgfsetroundcap%
\pgfsetroundjoin%
\pgfsetlinewidth{1.505625pt}%
\definecolor{currentstroke}{rgb}{0.172549,0.627451,0.172549}%
\pgfsetstrokecolor{currentstroke}%
\pgfsetdash{}{0pt}%
\pgfpathmoveto{\pgfqpoint{0.543170in}{1.770925in}}%
\pgfpathlineto{\pgfqpoint{0.564982in}{1.770023in}}%
\pgfpathlineto{\pgfqpoint{0.575888in}{1.737575in}}%
\pgfpathlineto{\pgfqpoint{0.586794in}{1.737575in}}%
\pgfpathlineto{\pgfqpoint{0.597700in}{1.735771in}}%
\pgfpathlineto{\pgfqpoint{0.695854in}{1.735048in}}%
\pgfpathlineto{\pgfqpoint{0.706760in}{1.694508in}}%
\pgfpathlineto{\pgfqpoint{0.728572in}{1.694508in}}%
\pgfpathlineto{\pgfqpoint{0.739478in}{1.690122in}}%
\pgfpathlineto{\pgfqpoint{0.815820in}{1.690122in}}%
\pgfpathlineto{\pgfqpoint{0.826726in}{1.681760in}}%
\pgfpathlineto{\pgfqpoint{0.881256in}{1.681760in}}%
\pgfpathlineto{\pgfqpoint{0.892162in}{1.621318in}}%
\pgfpathlineto{\pgfqpoint{0.913974in}{1.621318in}}%
\pgfpathlineto{\pgfqpoint{0.924880in}{1.579596in}}%
\pgfpathlineto{\pgfqpoint{0.935786in}{1.579596in}}%
\pgfpathlineto{\pgfqpoint{0.946692in}{1.552842in}}%
\pgfpathlineto{\pgfqpoint{0.957598in}{1.543449in}}%
\pgfpathlineto{\pgfqpoint{1.164811in}{1.543449in}}%
\pgfpathlineto{\pgfqpoint{1.175717in}{1.466223in}}%
\pgfpathlineto{\pgfqpoint{1.208435in}{1.466223in}}%
\pgfpathlineto{\pgfqpoint{1.219341in}{1.392397in}}%
\pgfpathlineto{\pgfqpoint{1.241153in}{1.392397in}}%
\pgfpathlineto{\pgfqpoint{1.252059in}{1.390442in}}%
\pgfpathlineto{\pgfqpoint{1.284777in}{1.390442in}}%
\pgfpathlineto{\pgfqpoint{1.295683in}{1.388371in}}%
\pgfpathlineto{\pgfqpoint{1.306589in}{1.273761in}}%
\pgfpathlineto{\pgfqpoint{1.328401in}{1.273761in}}%
\pgfpathlineto{\pgfqpoint{1.339307in}{1.238528in}}%
\pgfpathlineto{\pgfqpoint{1.644674in}{1.238528in}}%
\pgfpathlineto{\pgfqpoint{1.655580in}{1.216118in}}%
\pgfpathlineto{\pgfqpoint{1.677392in}{1.216118in}}%
\pgfpathlineto{\pgfqpoint{1.688298in}{1.204269in}}%
\pgfpathlineto{\pgfqpoint{2.048195in}{1.203352in}}%
\pgfpathlineto{\pgfqpoint{2.059101in}{1.141646in}}%
\pgfpathlineto{\pgfqpoint{2.255409in}{1.141646in}}%
\pgfpathlineto{\pgfqpoint{2.266315in}{1.055902in}}%
\pgfpathlineto{\pgfqpoint{2.429905in}{1.055902in}}%
\pgfpathlineto{\pgfqpoint{2.440811in}{1.020634in}}%
\pgfpathlineto{\pgfqpoint{2.582588in}{1.020634in}}%
\pgfpathlineto{\pgfqpoint{2.593494in}{0.984291in}}%
\pgfpathlineto{\pgfqpoint{2.702554in}{0.984291in}}%
\pgfpathlineto{\pgfqpoint{2.713460in}{0.982466in}}%
\pgfpathlineto{\pgfqpoint{2.909768in}{0.982466in}}%
\pgfpathlineto{\pgfqpoint{2.920674in}{0.980513in}}%
\pgfpathlineto{\pgfqpoint{2.931580in}{0.980513in}}%
\pgfpathlineto{\pgfqpoint{2.931580in}{0.980513in}}%
\pgfusepath{stroke}%
\end{pgfscope}%
\begin{pgfscope}%
\pgfpathrectangle{\pgfqpoint{0.423750in}{0.261892in}}{\pgfqpoint{2.627250in}{1.581827in}}%
\pgfusepath{clip}%
\pgfsetroundcap%
\pgfsetroundjoin%
\pgfsetlinewidth{1.505625pt}%
\definecolor{currentstroke}{rgb}{0.839216,0.152941,0.156863}%
\pgfsetstrokecolor{currentstroke}%
\pgfsetdash{}{0pt}%
\pgfpathmoveto{\pgfqpoint{0.543170in}{1.771301in}}%
\pgfpathlineto{\pgfqpoint{0.597700in}{1.770087in}}%
\pgfpathlineto{\pgfqpoint{0.608606in}{1.761377in}}%
\pgfpathlineto{\pgfqpoint{0.652230in}{1.761189in}}%
\pgfpathlineto{\pgfqpoint{0.663136in}{1.755365in}}%
\pgfpathlineto{\pgfqpoint{0.674042in}{1.727860in}}%
\pgfpathlineto{\pgfqpoint{0.684948in}{1.727860in}}%
\pgfpathlineto{\pgfqpoint{0.695854in}{1.721712in}}%
\pgfpathlineto{\pgfqpoint{0.750384in}{1.721712in}}%
\pgfpathlineto{\pgfqpoint{0.783102in}{1.716655in}}%
\pgfpathlineto{\pgfqpoint{0.794008in}{1.709438in}}%
\pgfpathlineto{\pgfqpoint{0.804914in}{1.697261in}}%
\pgfpathlineto{\pgfqpoint{0.815820in}{1.687682in}}%
\pgfpathlineto{\pgfqpoint{0.826726in}{1.670603in}}%
\pgfpathlineto{\pgfqpoint{0.837632in}{1.662306in}}%
\pgfpathlineto{\pgfqpoint{0.848538in}{1.601898in}}%
\pgfpathlineto{\pgfqpoint{0.870350in}{1.574428in}}%
\pgfpathlineto{\pgfqpoint{0.881256in}{1.567846in}}%
\pgfpathlineto{\pgfqpoint{0.892162in}{1.567846in}}%
\pgfpathlineto{\pgfqpoint{0.903068in}{1.537080in}}%
\pgfpathlineto{\pgfqpoint{0.913974in}{1.489745in}}%
\pgfpathlineto{\pgfqpoint{0.924880in}{1.489745in}}%
\pgfpathlineto{\pgfqpoint{0.935786in}{1.427937in}}%
\pgfpathlineto{\pgfqpoint{0.957598in}{1.427937in}}%
\pgfpathlineto{\pgfqpoint{0.968504in}{1.399813in}}%
\pgfpathlineto{\pgfqpoint{0.979410in}{1.399813in}}%
\pgfpathlineto{\pgfqpoint{0.990316in}{1.331063in}}%
\pgfpathlineto{\pgfqpoint{1.023033in}{1.331063in}}%
\pgfpathlineto{\pgfqpoint{1.033939in}{1.327278in}}%
\pgfpathlineto{\pgfqpoint{1.044845in}{1.131621in}}%
\pgfpathlineto{\pgfqpoint{1.066657in}{1.131621in}}%
\pgfpathlineto{\pgfqpoint{1.077563in}{1.102713in}}%
\pgfpathlineto{\pgfqpoint{1.164811in}{1.102713in}}%
\pgfpathlineto{\pgfqpoint{1.175717in}{1.034024in}}%
\pgfpathlineto{\pgfqpoint{1.426555in}{1.034024in}}%
\pgfpathlineto{\pgfqpoint{1.437461in}{0.988787in}}%
\pgfpathlineto{\pgfqpoint{1.513802in}{0.988787in}}%
\pgfpathlineto{\pgfqpoint{1.524708in}{0.946051in}}%
\pgfpathlineto{\pgfqpoint{1.601050in}{0.946051in}}%
\pgfpathlineto{\pgfqpoint{1.611956in}{0.935348in}}%
\pgfpathlineto{\pgfqpoint{1.731922in}{0.935348in}}%
\pgfpathlineto{\pgfqpoint{1.742828in}{0.912471in}}%
\pgfpathlineto{\pgfqpoint{1.830076in}{0.912471in}}%
\pgfpathlineto{\pgfqpoint{1.840982in}{0.893077in}}%
\pgfpathlineto{\pgfqpoint{2.080913in}{0.893077in}}%
\pgfpathlineto{\pgfqpoint{2.091819in}{0.884978in}}%
\pgfpathlineto{\pgfqpoint{2.593494in}{0.884219in}}%
\pgfpathlineto{\pgfqpoint{2.604400in}{0.802969in}}%
\pgfpathlineto{\pgfqpoint{2.931580in}{0.802885in}}%
\pgfpathlineto{\pgfqpoint{2.931580in}{0.802885in}}%
\pgfusepath{stroke}%
\end{pgfscope}%
\begin{pgfscope}%
\pgfsetrectcap%
\pgfsetmiterjoin%
\pgfsetlinewidth{0.000000pt}%
\definecolor{currentstroke}{rgb}{1.000000,1.000000,1.000000}%
\pgfsetstrokecolor{currentstroke}%
\pgfsetdash{}{0pt}%
\pgfpathmoveto{\pgfqpoint{0.423750in}{0.261892in}}%
\pgfpathlineto{\pgfqpoint{0.423750in}{1.843719in}}%
\pgfusepath{}%
\end{pgfscope}%
\begin{pgfscope}%
\pgfsetrectcap%
\pgfsetmiterjoin%
\pgfsetlinewidth{0.000000pt}%
\definecolor{currentstroke}{rgb}{1.000000,1.000000,1.000000}%
\pgfsetstrokecolor{currentstroke}%
\pgfsetdash{}{0pt}%
\pgfpathmoveto{\pgfqpoint{3.051000in}{0.261892in}}%
\pgfpathlineto{\pgfqpoint{3.051000in}{1.843719in}}%
\pgfusepath{}%
\end{pgfscope}%
\begin{pgfscope}%
\pgfsetrectcap%
\pgfsetmiterjoin%
\pgfsetlinewidth{0.000000pt}%
\definecolor{currentstroke}{rgb}{1.000000,1.000000,1.000000}%
\pgfsetstrokecolor{currentstroke}%
\pgfsetdash{}{0pt}%
\pgfpathmoveto{\pgfqpoint{0.423750in}{0.261892in}}%
\pgfpathlineto{\pgfqpoint{3.051000in}{0.261892in}}%
\pgfusepath{}%
\end{pgfscope}%
\begin{pgfscope}%
\pgfsetrectcap%
\pgfsetmiterjoin%
\pgfsetlinewidth{0.000000pt}%
\definecolor{currentstroke}{rgb}{1.000000,1.000000,1.000000}%
\pgfsetstrokecolor{currentstroke}%
\pgfsetdash{}{0pt}%
\pgfpathmoveto{\pgfqpoint{0.423750in}{1.843719in}}%
\pgfpathlineto{\pgfqpoint{3.051000in}{1.843719in}}%
\pgfusepath{}%
\end{pgfscope}%
\begin{pgfscope}%
\definecolor{textcolor}{rgb}{0.150000,0.150000,0.150000}%
\pgfsetstrokecolor{textcolor}%
\pgfsetfillcolor{textcolor}%
\pgftext[x=1.737375in,y=1.927052in,,base]{\color{textcolor}\rmfamily\fontsize{8.000000}{9.600000}\selectfont LennardJones6}%
\end{pgfscope}%
\begin{pgfscope}%
\pgfsetroundcap%
\pgfsetroundjoin%
\pgfsetlinewidth{1.505625pt}%
\definecolor{currentstroke}{rgb}{0.121569,0.466667,0.705882}%
\pgfsetstrokecolor{currentstroke}%
\pgfsetdash{}{0pt}%
\pgfpathmoveto{\pgfqpoint{0.523750in}{0.913150in}}%
\pgfpathlineto{\pgfqpoint{0.745972in}{0.913150in}}%
\pgfusepath{stroke}%
\end{pgfscope}%
\begin{pgfscope}%
\definecolor{textcolor}{rgb}{0.150000,0.150000,0.150000}%
\pgfsetstrokecolor{textcolor}%
\pgfsetfillcolor{textcolor}%
\pgftext[x=0.834861in,y=0.874261in,left,base]{\color{textcolor}\rmfamily\fontsize{8.000000}{9.600000}\selectfont random}%
\end{pgfscope}%
\begin{pgfscope}%
\pgfsetroundcap%
\pgfsetroundjoin%
\pgfsetlinewidth{1.505625pt}%
\definecolor{currentstroke}{rgb}{1.000000,0.498039,0.054902}%
\pgfsetstrokecolor{currentstroke}%
\pgfsetdash{}{0pt}%
\pgfpathmoveto{\pgfqpoint{0.523750in}{0.750064in}}%
\pgfpathlineto{\pgfqpoint{0.745972in}{0.750064in}}%
\pgfusepath{stroke}%
\end{pgfscope}%
\begin{pgfscope}%
\definecolor{textcolor}{rgb}{0.150000,0.150000,0.150000}%
\pgfsetstrokecolor{textcolor}%
\pgfsetfillcolor{textcolor}%
\pgftext[x=0.834861in,y=0.711175in,left,base]{\color{textcolor}\rmfamily\fontsize{8.000000}{9.600000}\selectfont 5 x DNGO retrain-reset}%
\end{pgfscope}%
\begin{pgfscope}%
\pgfsetroundcap%
\pgfsetroundjoin%
\pgfsetlinewidth{1.505625pt}%
\definecolor{currentstroke}{rgb}{0.172549,0.627451,0.172549}%
\pgfsetstrokecolor{currentstroke}%
\pgfsetdash{}{0pt}%
\pgfpathmoveto{\pgfqpoint{0.523750in}{0.586978in}}%
\pgfpathlineto{\pgfqpoint{0.745972in}{0.586978in}}%
\pgfusepath{stroke}%
\end{pgfscope}%
\begin{pgfscope}%
\definecolor{textcolor}{rgb}{0.150000,0.150000,0.150000}%
\pgfsetstrokecolor{textcolor}%
\pgfsetfillcolor{textcolor}%
\pgftext[x=0.834861in,y=0.548090in,left,base]{\color{textcolor}\rmfamily\fontsize{8.000000}{9.600000}\selectfont DNGO retrain-reset}%
\end{pgfscope}%
\begin{pgfscope}%
\pgfsetroundcap%
\pgfsetroundjoin%
\pgfsetlinewidth{1.505625pt}%
\definecolor{currentstroke}{rgb}{0.839216,0.152941,0.156863}%
\pgfsetstrokecolor{currentstroke}%
\pgfsetdash{}{0pt}%
\pgfpathmoveto{\pgfqpoint{0.523750in}{0.423893in}}%
\pgfpathlineto{\pgfqpoint{0.745972in}{0.423893in}}%
\pgfusepath{stroke}%
\end{pgfscope}%
\begin{pgfscope}%
\definecolor{textcolor}{rgb}{0.150000,0.150000,0.150000}%
\pgfsetstrokecolor{textcolor}%
\pgfsetfillcolor{textcolor}%
\pgftext[x=0.834861in,y=0.385004in,left,base]{\color{textcolor}\rmfamily\fontsize{8.000000}{9.600000}\selectfont GP}%
\end{pgfscope}%
\end{pgfpicture}%
\makeatother%
\endgroup%

                \caption{Example showing that an ensemble improves on mostly boring functions in sufficently high dimensions (i.e. above 6).}
            \end{subfigure}
            \caption{Both plots shows the average Simple Regret over 10 runs with a $1/4$ standard derivation confidence interval.}
            \label{fig:benchmark}
        \end{figure*}

    \subsection{Machine Learning Tasks}\label{sec:highdim}

        We tried the methods on three machine learning tasks from \parencite{eggensperger_hpolib2_2018}:
        
            \begin{itemize}
                \item \textbf{Logisitic Regression on MNIST} tuning learning rate, $l_2$ regularization, batch size and dropout rate of which one is discrete. 
                \item \textbf{Convolution Neural Network (CNN) for CIFAR-10} tuning learning rate, batch size and the number of units in each of the three layers. Four parameters are discrete.
                \item \textbf{Fully Connected Network for MNIST} tuning 10 parameters of which 3 are discrete.
            \end{itemize}
        
        As stated, some of the parameters are discrete.
        We deal with it by relaxing the problem to $\mathbb{R}$ in all dimensions and do rounding before evaluation of $f$.

        For both CNN and the fully connected network there was no noticeable different from random sampling for any of the tested methods.
        % TODO: Could be explained by being bad at discrete problems (does not seem to treat it much differently from GP. scalable shows how important choice of activation function is. Maybe this could improve)

        For the logistic regression all Bayesian Optimization models did on average better than random sampling.
        However, they all converged to the same suggested optimal value.
        Thus it did not provide any insight into the behavior of the ensemble.
        For further detail we refer to the appendix \cref{sec:appml}.

    % \subsection{Scaling}

    %     - Minibatch
    %     - Scaling to big N might be problematic (minibatch size start having influence on performance <= extrapolate batch size / observations ratio)

\section{Discussion}\label{sec:discussion}

    This section will start by making observation about the behavior of DNGO in \cref{sec:disc-characteristic}.
    Then \cref{sec:disc-diviations} follows up on results from \cref{sec:exp} by commenting on why they differ from the original paper.
    The motivation behind DNGO is subsequently discussed in \cref{sec:disc-extensions} and, finally, different approaches to performance evaluation is considered in \cref{sec:disc-evaluation}.
    
    
    \subsection{DNGO Characteristic}\label{sec:disc-characteristic}

        We note several characteristics of the DNGO model.
        First it seems generally worse at exploiting than GP.
        This is illustrated with the Branin function in \cref{fig:braningexploit}.
        This shows that when DNGO is exploiting (i.e. querying some local area) it does so in a relatively random fashion.

        This could be explained by the second behavior which was observed, namely sudden fluctuations.
        That is, high peaks in areas without observations which even had low variance.
        Because of these it was chosen to use \emph{mean} as ensemble aggregator as it would behave as a regularizer.
        \emph{Max} seemed to increase exploration but this was not thoroughly studied.
        However, investigating how different quantile might effect the exploration-exploitating trade-off could be an interesting future avenue to pursue.
        This is especially interesting considering the failure case of an ensemble in \cref{sec:expbenchmark}.

        Concerning embedding functions, they provided a promising case over GP and indicated a class in which an ensemble might be especially useful.
        However, the method is not competitive with specialized techniques like ARD covered in \cref{sec:ard}.
        Neither does it scale to the same level of dimensions like REMBO \parencite{wang_bayesian_2013} which specifically exploits problems with low effective dimensionality.
        Since similar assumption have not been made about the problem the hope is that ensembled DNGO usefulness will generalize to a broader class of problems.

        \begin{figure}
            % \resizebox{\linewidth}{!}{}
            %% Creator: Matplotlib, PGF backend
%%
%% To include the figure in your LaTeX document, write
%%   \input{<filename>.pgf}
%%
%% Make sure the required packages are loaded in your preamble
%%   \usepackage{pgf}
%%
%% Figures using additional raster images can only be included by \input if
%% they are in the same directory as the main LaTeX file. For loading figures
%% from other directories you can use the `import` package
%%   \usepackage{import}
%% and then include the figures with
%%   \import{<path to file>}{<filename>.pgf}
%%
%% Matplotlib used the following preamble
%%   \usepackage{gensymb}
%%   \usepackage{fontspec}
%%   \setmainfont{DejaVu Serif}
%%   \setsansfont{Arial}
%%   \setmonofont{DejaVu Sans Mono}
%%
\begingroup%
\makeatletter%
\begin{pgfpicture}%
\pgfpathrectangle{\pgfpointorigin}{\pgfqpoint{3.390000in}{3.390000in}}%
\pgfusepath{use as bounding box, clip}%
\begin{pgfscope}%
\pgfsetbuttcap%
\pgfsetmiterjoin%
\definecolor{currentfill}{rgb}{1.000000,1.000000,1.000000}%
\pgfsetfillcolor{currentfill}%
\pgfsetlinewidth{0.000000pt}%
\definecolor{currentstroke}{rgb}{1.000000,1.000000,1.000000}%
\pgfsetstrokecolor{currentstroke}%
\pgfsetdash{}{0pt}%
\pgfpathmoveto{\pgfqpoint{0.000000in}{0.000000in}}%
\pgfpathlineto{\pgfqpoint{3.390000in}{0.000000in}}%
\pgfpathlineto{\pgfqpoint{3.390000in}{3.390000in}}%
\pgfpathlineto{\pgfqpoint{0.000000in}{3.390000in}}%
\pgfpathclose%
\pgfusepath{fill}%
\end{pgfscope}%
\begin{pgfscope}%
\pgfsetbuttcap%
\pgfsetmiterjoin%
\definecolor{currentfill}{rgb}{0.917647,0.917647,0.949020}%
\pgfsetfillcolor{currentfill}%
\pgfsetlinewidth{0.000000pt}%
\definecolor{currentstroke}{rgb}{0.000000,0.000000,0.000000}%
\pgfsetstrokecolor{currentstroke}%
\pgfsetstrokeopacity{0.000000}%
\pgfsetdash{}{0pt}%
\pgfpathmoveto{\pgfqpoint{0.423750in}{1.819814in}}%
\pgfpathlineto{\pgfqpoint{1.617955in}{1.819814in}}%
\pgfpathlineto{\pgfqpoint{1.617955in}{2.983200in}}%
\pgfpathlineto{\pgfqpoint{0.423750in}{2.983200in}}%
\pgfpathclose%
\pgfusepath{fill}%
\end{pgfscope}%
\begin{pgfscope}%
\pgfpathrectangle{\pgfqpoint{0.423750in}{1.819814in}}{\pgfqpoint{1.194205in}{1.163386in}}%
\pgfusepath{clip}%
\pgfsetroundcap%
\pgfsetroundjoin%
\pgfsetlinewidth{0.803000pt}%
\definecolor{currentstroke}{rgb}{1.000000,1.000000,1.000000}%
\pgfsetstrokecolor{currentstroke}%
\pgfsetdash{}{0pt}%
\pgfpathmoveto{\pgfqpoint{0.423750in}{1.819814in}}%
\pgfpathlineto{\pgfqpoint{0.423750in}{2.983200in}}%
\pgfusepath{stroke}%
\end{pgfscope}%
\begin{pgfscope}%
\definecolor{textcolor}{rgb}{0.150000,0.150000,0.150000}%
\pgfsetstrokecolor{textcolor}%
\pgfsetfillcolor{textcolor}%
\pgftext[x=0.423750in,y=1.771203in,,top]{\color{textcolor}\rmfamily\fontsize{8.000000}{9.600000}\selectfont \(\displaystyle -5\)}%
\end{pgfscope}%
\begin{pgfscope}%
\pgfpathrectangle{\pgfqpoint{0.423750in}{1.819814in}}{\pgfqpoint{1.194205in}{1.163386in}}%
\pgfusepath{clip}%
\pgfsetroundcap%
\pgfsetroundjoin%
\pgfsetlinewidth{0.803000pt}%
\definecolor{currentstroke}{rgb}{1.000000,1.000000,1.000000}%
\pgfsetstrokecolor{currentstroke}%
\pgfsetdash{}{0pt}%
\pgfpathmoveto{\pgfqpoint{0.821818in}{1.819814in}}%
\pgfpathlineto{\pgfqpoint{0.821818in}{2.983200in}}%
\pgfusepath{stroke}%
\end{pgfscope}%
\begin{pgfscope}%
\definecolor{textcolor}{rgb}{0.150000,0.150000,0.150000}%
\pgfsetstrokecolor{textcolor}%
\pgfsetfillcolor{textcolor}%
\pgftext[x=0.821818in,y=1.771203in,,top]{\color{textcolor}\rmfamily\fontsize{8.000000}{9.600000}\selectfont \(\displaystyle 0\)}%
\end{pgfscope}%
\begin{pgfscope}%
\pgfpathrectangle{\pgfqpoint{0.423750in}{1.819814in}}{\pgfqpoint{1.194205in}{1.163386in}}%
\pgfusepath{clip}%
\pgfsetroundcap%
\pgfsetroundjoin%
\pgfsetlinewidth{0.803000pt}%
\definecolor{currentstroke}{rgb}{1.000000,1.000000,1.000000}%
\pgfsetstrokecolor{currentstroke}%
\pgfsetdash{}{0pt}%
\pgfpathmoveto{\pgfqpoint{1.219886in}{1.819814in}}%
\pgfpathlineto{\pgfqpoint{1.219886in}{2.983200in}}%
\pgfusepath{stroke}%
\end{pgfscope}%
\begin{pgfscope}%
\definecolor{textcolor}{rgb}{0.150000,0.150000,0.150000}%
\pgfsetstrokecolor{textcolor}%
\pgfsetfillcolor{textcolor}%
\pgftext[x=1.219886in,y=1.771203in,,top]{\color{textcolor}\rmfamily\fontsize{8.000000}{9.600000}\selectfont \(\displaystyle 5\)}%
\end{pgfscope}%
\begin{pgfscope}%
\pgfpathrectangle{\pgfqpoint{0.423750in}{1.819814in}}{\pgfqpoint{1.194205in}{1.163386in}}%
\pgfusepath{clip}%
\pgfsetroundcap%
\pgfsetroundjoin%
\pgfsetlinewidth{0.803000pt}%
\definecolor{currentstroke}{rgb}{1.000000,1.000000,1.000000}%
\pgfsetstrokecolor{currentstroke}%
\pgfsetdash{}{0pt}%
\pgfpathmoveto{\pgfqpoint{1.617955in}{1.819814in}}%
\pgfpathlineto{\pgfqpoint{1.617955in}{2.983200in}}%
\pgfusepath{stroke}%
\end{pgfscope}%
\begin{pgfscope}%
\definecolor{textcolor}{rgb}{0.150000,0.150000,0.150000}%
\pgfsetstrokecolor{textcolor}%
\pgfsetfillcolor{textcolor}%
\pgftext[x=1.617955in,y=1.771203in,,top]{\color{textcolor}\rmfamily\fontsize{8.000000}{9.600000}\selectfont \(\displaystyle 10\)}%
\end{pgfscope}%
\begin{pgfscope}%
\pgfpathrectangle{\pgfqpoint{0.423750in}{1.819814in}}{\pgfqpoint{1.194205in}{1.163386in}}%
\pgfusepath{clip}%
\pgfsetroundcap%
\pgfsetroundjoin%
\pgfsetlinewidth{0.803000pt}%
\definecolor{currentstroke}{rgb}{1.000000,1.000000,1.000000}%
\pgfsetstrokecolor{currentstroke}%
\pgfsetdash{}{0pt}%
\pgfpathmoveto{\pgfqpoint{0.423750in}{1.819814in}}%
\pgfpathlineto{\pgfqpoint{1.617955in}{1.819814in}}%
\pgfusepath{stroke}%
\end{pgfscope}%
\begin{pgfscope}%
\definecolor{textcolor}{rgb}{0.150000,0.150000,0.150000}%
\pgfsetstrokecolor{textcolor}%
\pgfsetfillcolor{textcolor}%
\pgftext[x=0.316110in,y=1.777604in,left,base]{\color{textcolor}\rmfamily\fontsize{8.000000}{9.600000}\selectfont \(\displaystyle 0\)}%
\end{pgfscope}%
\begin{pgfscope}%
\pgfpathrectangle{\pgfqpoint{0.423750in}{1.819814in}}{\pgfqpoint{1.194205in}{1.163386in}}%
\pgfusepath{clip}%
\pgfsetroundcap%
\pgfsetroundjoin%
\pgfsetlinewidth{0.803000pt}%
\definecolor{currentstroke}{rgb}{1.000000,1.000000,1.000000}%
\pgfsetstrokecolor{currentstroke}%
\pgfsetdash{}{0pt}%
\pgfpathmoveto{\pgfqpoint{0.423750in}{2.207609in}}%
\pgfpathlineto{\pgfqpoint{1.617955in}{2.207609in}}%
\pgfusepath{stroke}%
\end{pgfscope}%
\begin{pgfscope}%
\definecolor{textcolor}{rgb}{0.150000,0.150000,0.150000}%
\pgfsetstrokecolor{textcolor}%
\pgfsetfillcolor{textcolor}%
\pgftext[x=0.316110in,y=2.165400in,left,base]{\color{textcolor}\rmfamily\fontsize{8.000000}{9.600000}\selectfont \(\displaystyle 5\)}%
\end{pgfscope}%
\begin{pgfscope}%
\pgfpathrectangle{\pgfqpoint{0.423750in}{1.819814in}}{\pgfqpoint{1.194205in}{1.163386in}}%
\pgfusepath{clip}%
\pgfsetroundcap%
\pgfsetroundjoin%
\pgfsetlinewidth{0.803000pt}%
\definecolor{currentstroke}{rgb}{1.000000,1.000000,1.000000}%
\pgfsetstrokecolor{currentstroke}%
\pgfsetdash{}{0pt}%
\pgfpathmoveto{\pgfqpoint{0.423750in}{2.595405in}}%
\pgfpathlineto{\pgfqpoint{1.617955in}{2.595405in}}%
\pgfusepath{stroke}%
\end{pgfscope}%
\begin{pgfscope}%
\definecolor{textcolor}{rgb}{0.150000,0.150000,0.150000}%
\pgfsetstrokecolor{textcolor}%
\pgfsetfillcolor{textcolor}%
\pgftext[x=0.257082in,y=2.553195in,left,base]{\color{textcolor}\rmfamily\fontsize{8.000000}{9.600000}\selectfont \(\displaystyle 10\)}%
\end{pgfscope}%
\begin{pgfscope}%
\pgfpathrectangle{\pgfqpoint{0.423750in}{1.819814in}}{\pgfqpoint{1.194205in}{1.163386in}}%
\pgfusepath{clip}%
\pgfsetroundcap%
\pgfsetroundjoin%
\pgfsetlinewidth{0.803000pt}%
\definecolor{currentstroke}{rgb}{1.000000,1.000000,1.000000}%
\pgfsetstrokecolor{currentstroke}%
\pgfsetdash{}{0pt}%
\pgfpathmoveto{\pgfqpoint{0.423750in}{2.983200in}}%
\pgfpathlineto{\pgfqpoint{1.617955in}{2.983200in}}%
\pgfusepath{stroke}%
\end{pgfscope}%
\begin{pgfscope}%
\definecolor{textcolor}{rgb}{0.150000,0.150000,0.150000}%
\pgfsetstrokecolor{textcolor}%
\pgfsetfillcolor{textcolor}%
\pgftext[x=0.257082in,y=2.940991in,left,base]{\color{textcolor}\rmfamily\fontsize{8.000000}{9.600000}\selectfont \(\displaystyle 15\)}%
\end{pgfscope}%
\begin{pgfscope}%
\pgfpathrectangle{\pgfqpoint{0.423750in}{1.819814in}}{\pgfqpoint{1.194205in}{1.163386in}}%
\pgfusepath{clip}%
\pgfsetbuttcap%
\pgfsetroundjoin%
\definecolor{currentfill}{rgb}{0.038535,0.033658,0.122459}%
\pgfsetfillcolor{currentfill}%
\pgfsetlinewidth{0.000000pt}%
\definecolor{currentstroke}{rgb}{0.000000,0.000000,0.000000}%
\pgfsetstrokecolor{currentstroke}%
\pgfsetdash{}{0pt}%
\pgfpathmoveto{\pgfqpoint{0.425967in}{1.831565in}}%
\pgfpathlineto{\pgfqpoint{0.423750in}{1.838296in}}%
\pgfpathlineto{\pgfqpoint{0.423750in}{1.831565in}}%
\pgfpathlineto{\pgfqpoint{0.423750in}{1.819814in}}%
\pgfpathlineto{\pgfqpoint{0.429822in}{1.819814in}}%
\pgfpathclose%
\pgfusepath{fill}%
\end{pgfscope}%
\begin{pgfscope}%
\pgfpathrectangle{\pgfqpoint{0.423750in}{1.819814in}}{\pgfqpoint{1.194205in}{1.163386in}}%
\pgfusepath{clip}%
\pgfsetbuttcap%
\pgfsetroundjoin%
\definecolor{currentfill}{rgb}{0.106928,0.065172,0.170092}%
\pgfsetfillcolor{currentfill}%
\pgfsetlinewidth{0.000000pt}%
\definecolor{currentstroke}{rgb}{0.000000,0.000000,0.000000}%
\pgfsetstrokecolor{currentstroke}%
\pgfsetdash{}{0pt}%
\pgfpathmoveto{\pgfqpoint{0.435813in}{1.819814in}}%
\pgfpathlineto{\pgfqpoint{0.441208in}{1.819814in}}%
\pgfpathlineto{\pgfqpoint{0.437314in}{1.831565in}}%
\pgfpathlineto{\pgfqpoint{0.435813in}{1.836087in}}%
\pgfpathlineto{\pgfqpoint{0.433473in}{1.843316in}}%
\pgfpathlineto{\pgfqpoint{0.429654in}{1.855068in}}%
\pgfpathlineto{\pgfqpoint{0.425806in}{1.866819in}}%
\pgfpathlineto{\pgfqpoint{0.423750in}{1.873076in}}%
\pgfpathlineto{\pgfqpoint{0.423750in}{1.866819in}}%
\pgfpathlineto{\pgfqpoint{0.423750in}{1.855068in}}%
\pgfpathlineto{\pgfqpoint{0.423750in}{1.843316in}}%
\pgfpathlineto{\pgfqpoint{0.423750in}{1.838296in}}%
\pgfpathlineto{\pgfqpoint{0.425967in}{1.831565in}}%
\pgfpathlineto{\pgfqpoint{0.429822in}{1.819814in}}%
\pgfpathclose%
\pgfusepath{fill}%
\end{pgfscope}%
\begin{pgfscope}%
\pgfpathrectangle{\pgfqpoint{0.423750in}{1.819814in}}{\pgfqpoint{1.194205in}{1.163386in}}%
\pgfusepath{clip}%
\pgfsetbuttcap%
\pgfsetroundjoin%
\definecolor{currentfill}{rgb}{0.174895,0.088482,0.219029}%
\pgfsetfillcolor{currentfill}%
\pgfsetlinewidth{0.000000pt}%
\definecolor{currentstroke}{rgb}{0.000000,0.000000,0.000000}%
\pgfsetstrokecolor{currentstroke}%
\pgfsetdash{}{0pt}%
\pgfpathmoveto{\pgfqpoint{0.447875in}{1.819814in}}%
\pgfpathlineto{\pgfqpoint{0.452983in}{1.819814in}}%
\pgfpathlineto{\pgfqpoint{0.449046in}{1.831565in}}%
\pgfpathlineto{\pgfqpoint{0.447875in}{1.835055in}}%
\pgfpathlineto{\pgfqpoint{0.445180in}{1.843316in}}%
\pgfpathlineto{\pgfqpoint{0.441327in}{1.855068in}}%
\pgfpathlineto{\pgfqpoint{0.437444in}{1.866819in}}%
\pgfpathlineto{\pgfqpoint{0.435813in}{1.871744in}}%
\pgfpathlineto{\pgfqpoint{0.433610in}{1.878571in}}%
\pgfpathlineto{\pgfqpoint{0.429802in}{1.890322in}}%
\pgfpathlineto{\pgfqpoint{0.425965in}{1.902073in}}%
\pgfpathlineto{\pgfqpoint{0.423750in}{1.908828in}}%
\pgfpathlineto{\pgfqpoint{0.423750in}{1.902073in}}%
\pgfpathlineto{\pgfqpoint{0.423750in}{1.890322in}}%
\pgfpathlineto{\pgfqpoint{0.423750in}{1.878571in}}%
\pgfpathlineto{\pgfqpoint{0.423750in}{1.873076in}}%
\pgfpathlineto{\pgfqpoint{0.425806in}{1.866819in}}%
\pgfpathlineto{\pgfqpoint{0.429654in}{1.855068in}}%
\pgfpathlineto{\pgfqpoint{0.433473in}{1.843316in}}%
\pgfpathlineto{\pgfqpoint{0.435813in}{1.836087in}}%
\pgfpathlineto{\pgfqpoint{0.437314in}{1.831565in}}%
\pgfpathlineto{\pgfqpoint{0.441208in}{1.819814in}}%
\pgfpathclose%
\pgfusepath{fill}%
\end{pgfscope}%
\begin{pgfscope}%
\pgfpathrectangle{\pgfqpoint{0.423750in}{1.819814in}}{\pgfqpoint{1.194205in}{1.163386in}}%
\pgfusepath{clip}%
\pgfsetbuttcap%
\pgfsetroundjoin%
\definecolor{currentfill}{rgb}{0.245256,0.104974,0.263956}%
\pgfsetfillcolor{currentfill}%
\pgfsetlinewidth{0.000000pt}%
\definecolor{currentstroke}{rgb}{0.000000,0.000000,0.000000}%
\pgfsetstrokecolor{currentstroke}%
\pgfsetdash{}{0pt}%
\pgfpathmoveto{\pgfqpoint{0.459938in}{1.819814in}}%
\pgfpathlineto{\pgfqpoint{0.465208in}{1.819814in}}%
\pgfpathlineto{\pgfqpoint{0.461222in}{1.831565in}}%
\pgfpathlineto{\pgfqpoint{0.459938in}{1.835342in}}%
\pgfpathlineto{\pgfqpoint{0.457310in}{1.843316in}}%
\pgfpathlineto{\pgfqpoint{0.453417in}{1.855068in}}%
\pgfpathlineto{\pgfqpoint{0.449495in}{1.866819in}}%
\pgfpathlineto{\pgfqpoint{0.447875in}{1.871657in}}%
\pgfpathlineto{\pgfqpoint{0.445627in}{1.878571in}}%
\pgfpathlineto{\pgfqpoint{0.441788in}{1.890322in}}%
\pgfpathlineto{\pgfqpoint{0.437920in}{1.902073in}}%
\pgfpathlineto{\pgfqpoint{0.435813in}{1.908450in}}%
\pgfpathlineto{\pgfqpoint{0.434083in}{1.913825in}}%
\pgfpathlineto{\pgfqpoint{0.430290in}{1.925576in}}%
\pgfpathlineto{\pgfqpoint{0.426468in}{1.937327in}}%
\pgfpathlineto{\pgfqpoint{0.423750in}{1.945637in}}%
\pgfpathlineto{\pgfqpoint{0.423750in}{1.937327in}}%
\pgfpathlineto{\pgfqpoint{0.423750in}{1.925576in}}%
\pgfpathlineto{\pgfqpoint{0.423750in}{1.913825in}}%
\pgfpathlineto{\pgfqpoint{0.423750in}{1.908828in}}%
\pgfpathlineto{\pgfqpoint{0.425965in}{1.902073in}}%
\pgfpathlineto{\pgfqpoint{0.429802in}{1.890322in}}%
\pgfpathlineto{\pgfqpoint{0.433610in}{1.878571in}}%
\pgfpathlineto{\pgfqpoint{0.435813in}{1.871744in}}%
\pgfpathlineto{\pgfqpoint{0.437444in}{1.866819in}}%
\pgfpathlineto{\pgfqpoint{0.441327in}{1.855068in}}%
\pgfpathlineto{\pgfqpoint{0.445180in}{1.843316in}}%
\pgfpathlineto{\pgfqpoint{0.447875in}{1.835055in}}%
\pgfpathlineto{\pgfqpoint{0.449046in}{1.831565in}}%
\pgfpathlineto{\pgfqpoint{0.452983in}{1.819814in}}%
\pgfpathclose%
\pgfusepath{fill}%
\end{pgfscope}%
\begin{pgfscope}%
\pgfpathrectangle{\pgfqpoint{0.423750in}{1.819814in}}{\pgfqpoint{1.194205in}{1.163386in}}%
\pgfusepath{clip}%
\pgfsetbuttcap%
\pgfsetroundjoin%
\definecolor{currentfill}{rgb}{0.318263,0.115743,0.301001}%
\pgfsetfillcolor{currentfill}%
\pgfsetlinewidth{0.000000pt}%
\definecolor{currentstroke}{rgb}{0.000000,0.000000,0.000000}%
\pgfsetstrokecolor{currentstroke}%
\pgfsetdash{}{0pt}%
\pgfpathmoveto{\pgfqpoint{0.472001in}{1.819814in}}%
\pgfpathlineto{\pgfqpoint{0.477954in}{1.819814in}}%
\pgfpathlineto{\pgfqpoint{0.473912in}{1.831565in}}%
\pgfpathlineto{\pgfqpoint{0.472001in}{1.837105in}}%
\pgfpathlineto{\pgfqpoint{0.469927in}{1.843316in}}%
\pgfpathlineto{\pgfqpoint{0.465991in}{1.855068in}}%
\pgfpathlineto{\pgfqpoint{0.462022in}{1.866819in}}%
\pgfpathlineto{\pgfqpoint{0.459938in}{1.872968in}}%
\pgfpathlineto{\pgfqpoint{0.458097in}{1.878571in}}%
\pgfpathlineto{\pgfqpoint{0.454224in}{1.890322in}}%
\pgfpathlineto{\pgfqpoint{0.450319in}{1.902073in}}%
\pgfpathlineto{\pgfqpoint{0.447875in}{1.909395in}}%
\pgfpathlineto{\pgfqpoint{0.446439in}{1.913825in}}%
\pgfpathlineto{\pgfqpoint{0.442620in}{1.925576in}}%
\pgfpathlineto{\pgfqpoint{0.438770in}{1.937327in}}%
\pgfpathlineto{\pgfqpoint{0.435813in}{1.946301in}}%
\pgfpathlineto{\pgfqpoint{0.434921in}{1.949079in}}%
\pgfpathlineto{\pgfqpoint{0.431147in}{1.960830in}}%
\pgfpathlineto{\pgfqpoint{0.427343in}{1.972582in}}%
\pgfpathlineto{\pgfqpoint{0.423750in}{1.983597in}}%
\pgfpathlineto{\pgfqpoint{0.423750in}{1.972582in}}%
\pgfpathlineto{\pgfqpoint{0.423750in}{1.960830in}}%
\pgfpathlineto{\pgfqpoint{0.423750in}{1.949079in}}%
\pgfpathlineto{\pgfqpoint{0.423750in}{1.945637in}}%
\pgfpathlineto{\pgfqpoint{0.426468in}{1.937327in}}%
\pgfpathlineto{\pgfqpoint{0.430290in}{1.925576in}}%
\pgfpathlineto{\pgfqpoint{0.434083in}{1.913825in}}%
\pgfpathlineto{\pgfqpoint{0.435813in}{1.908450in}}%
\pgfpathlineto{\pgfqpoint{0.437920in}{1.902073in}}%
\pgfpathlineto{\pgfqpoint{0.441788in}{1.890322in}}%
\pgfpathlineto{\pgfqpoint{0.445627in}{1.878571in}}%
\pgfpathlineto{\pgfqpoint{0.447875in}{1.871657in}}%
\pgfpathlineto{\pgfqpoint{0.449495in}{1.866819in}}%
\pgfpathlineto{\pgfqpoint{0.453417in}{1.855068in}}%
\pgfpathlineto{\pgfqpoint{0.457310in}{1.843316in}}%
\pgfpathlineto{\pgfqpoint{0.459938in}{1.835342in}}%
\pgfpathlineto{\pgfqpoint{0.461222in}{1.831565in}}%
\pgfpathlineto{\pgfqpoint{0.465208in}{1.819814in}}%
\pgfpathclose%
\pgfusepath{fill}%
\end{pgfscope}%
\begin{pgfscope}%
\pgfpathrectangle{\pgfqpoint{0.423750in}{1.819814in}}{\pgfqpoint{1.194205in}{1.163386in}}%
\pgfusepath{clip}%
\pgfsetbuttcap%
\pgfsetroundjoin%
\definecolor{currentfill}{rgb}{0.400025,0.121350,0.331027}%
\pgfsetfillcolor{currentfill}%
\pgfsetlinewidth{0.000000pt}%
\definecolor{currentstroke}{rgb}{0.000000,0.000000,0.000000}%
\pgfsetstrokecolor{currentstroke}%
\pgfsetdash{}{0pt}%
\pgfpathmoveto{\pgfqpoint{0.484063in}{1.819814in}}%
\pgfpathlineto{\pgfqpoint{0.491310in}{1.819814in}}%
\pgfpathlineto{\pgfqpoint{0.487206in}{1.831565in}}%
\pgfpathlineto{\pgfqpoint{0.484063in}{1.840509in}}%
\pgfpathlineto{\pgfqpoint{0.483112in}{1.843316in}}%
\pgfpathlineto{\pgfqpoint{0.479125in}{1.855068in}}%
\pgfpathlineto{\pgfqpoint{0.475106in}{1.866819in}}%
\pgfpathlineto{\pgfqpoint{0.472001in}{1.875842in}}%
\pgfpathlineto{\pgfqpoint{0.471093in}{1.878571in}}%
\pgfpathlineto{\pgfqpoint{0.467179in}{1.890322in}}%
\pgfpathlineto{\pgfqpoint{0.463234in}{1.902073in}}%
\pgfpathlineto{\pgfqpoint{0.459938in}{1.911825in}}%
\pgfpathlineto{\pgfqpoint{0.459283in}{1.913825in}}%
\pgfpathlineto{\pgfqpoint{0.455433in}{1.925576in}}%
\pgfpathlineto{\pgfqpoint{0.451552in}{1.937327in}}%
\pgfpathlineto{\pgfqpoint{0.447875in}{1.948374in}}%
\pgfpathlineto{\pgfqpoint{0.447648in}{1.949079in}}%
\pgfpathlineto{\pgfqpoint{0.443852in}{1.960830in}}%
\pgfpathlineto{\pgfqpoint{0.440025in}{1.972582in}}%
\pgfpathlineto{\pgfqpoint{0.436167in}{1.984333in}}%
\pgfpathlineto{\pgfqpoint{0.435813in}{1.985412in}}%
\pgfpathlineto{\pgfqpoint{0.432404in}{1.996084in}}%
\pgfpathlineto{\pgfqpoint{0.428623in}{2.007836in}}%
\pgfpathlineto{\pgfqpoint{0.424810in}{2.019587in}}%
\pgfpathlineto{\pgfqpoint{0.423750in}{2.022849in}}%
\pgfpathlineto{\pgfqpoint{0.423750in}{2.019587in}}%
\pgfpathlineto{\pgfqpoint{0.423750in}{2.007836in}}%
\pgfpathlineto{\pgfqpoint{0.423750in}{1.996084in}}%
\pgfpathlineto{\pgfqpoint{0.423750in}{1.984333in}}%
\pgfpathlineto{\pgfqpoint{0.423750in}{1.983597in}}%
\pgfpathlineto{\pgfqpoint{0.427343in}{1.972582in}}%
\pgfpathlineto{\pgfqpoint{0.431147in}{1.960830in}}%
\pgfpathlineto{\pgfqpoint{0.434921in}{1.949079in}}%
\pgfpathlineto{\pgfqpoint{0.435813in}{1.946301in}}%
\pgfpathlineto{\pgfqpoint{0.438770in}{1.937327in}}%
\pgfpathlineto{\pgfqpoint{0.442620in}{1.925576in}}%
\pgfpathlineto{\pgfqpoint{0.446439in}{1.913825in}}%
\pgfpathlineto{\pgfqpoint{0.447875in}{1.909395in}}%
\pgfpathlineto{\pgfqpoint{0.450319in}{1.902073in}}%
\pgfpathlineto{\pgfqpoint{0.454224in}{1.890322in}}%
\pgfpathlineto{\pgfqpoint{0.458097in}{1.878571in}}%
\pgfpathlineto{\pgfqpoint{0.459938in}{1.872968in}}%
\pgfpathlineto{\pgfqpoint{0.462022in}{1.866819in}}%
\pgfpathlineto{\pgfqpoint{0.465991in}{1.855068in}}%
\pgfpathlineto{\pgfqpoint{0.469927in}{1.843316in}}%
\pgfpathlineto{\pgfqpoint{0.472001in}{1.837105in}}%
\pgfpathlineto{\pgfqpoint{0.473912in}{1.831565in}}%
\pgfpathlineto{\pgfqpoint{0.477954in}{1.819814in}}%
\pgfpathclose%
\pgfusepath{fill}%
\end{pgfscope}%
\begin{pgfscope}%
\pgfpathrectangle{\pgfqpoint{0.423750in}{1.819814in}}{\pgfqpoint{1.194205in}{1.163386in}}%
\pgfusepath{clip}%
\pgfsetbuttcap%
\pgfsetroundjoin%
\definecolor{currentfill}{rgb}{0.477697,0.120699,0.349023}%
\pgfsetfillcolor{currentfill}%
\pgfsetlinewidth{0.000000pt}%
\definecolor{currentstroke}{rgb}{0.000000,0.000000,0.000000}%
\pgfsetstrokecolor{currentstroke}%
\pgfsetdash{}{0pt}%
\pgfpathmoveto{\pgfqpoint{0.496126in}{1.819814in}}%
\pgfpathlineto{\pgfqpoint{0.505381in}{1.819814in}}%
\pgfpathlineto{\pgfqpoint{0.501210in}{1.831565in}}%
\pgfpathlineto{\pgfqpoint{0.497002in}{1.843316in}}%
\pgfpathlineto{\pgfqpoint{0.496126in}{1.845761in}}%
\pgfpathlineto{\pgfqpoint{0.492919in}{1.855068in}}%
\pgfpathlineto{\pgfqpoint{0.488843in}{1.866819in}}%
\pgfpathlineto{\pgfqpoint{0.484731in}{1.878571in}}%
\pgfpathlineto{\pgfqpoint{0.484063in}{1.880478in}}%
\pgfpathlineto{\pgfqpoint{0.480743in}{1.890322in}}%
\pgfpathlineto{\pgfqpoint{0.476751in}{1.902073in}}%
\pgfpathlineto{\pgfqpoint{0.472726in}{1.913825in}}%
\pgfpathlineto{\pgfqpoint{0.472001in}{1.915939in}}%
\pgfpathlineto{\pgfqpoint{0.468810in}{1.925576in}}%
\pgfpathlineto{\pgfqpoint{0.464893in}{1.937327in}}%
\pgfpathlineto{\pgfqpoint{0.460942in}{1.949079in}}%
\pgfpathlineto{\pgfqpoint{0.459938in}{1.952063in}}%
\pgfpathlineto{\pgfqpoint{0.457080in}{1.960830in}}%
\pgfpathlineto{\pgfqpoint{0.453228in}{1.972582in}}%
\pgfpathlineto{\pgfqpoint{0.449343in}{1.984333in}}%
\pgfpathlineto{\pgfqpoint{0.447875in}{1.988760in}}%
\pgfpathlineto{\pgfqpoint{0.445519in}{1.996084in}}%
\pgfpathlineto{\pgfqpoint{0.441721in}{2.007836in}}%
\pgfpathlineto{\pgfqpoint{0.437891in}{2.019587in}}%
\pgfpathlineto{\pgfqpoint{0.435813in}{2.025939in}}%
\pgfpathlineto{\pgfqpoint{0.434094in}{2.031338in}}%
\pgfpathlineto{\pgfqpoint{0.430342in}{2.043090in}}%
\pgfpathlineto{\pgfqpoint{0.426557in}{2.054841in}}%
\pgfpathlineto{\pgfqpoint{0.423750in}{2.063505in}}%
\pgfpathlineto{\pgfqpoint{0.423750in}{2.054841in}}%
\pgfpathlineto{\pgfqpoint{0.423750in}{2.043090in}}%
\pgfpathlineto{\pgfqpoint{0.423750in}{2.031338in}}%
\pgfpathlineto{\pgfqpoint{0.423750in}{2.022849in}}%
\pgfpathlineto{\pgfqpoint{0.424810in}{2.019587in}}%
\pgfpathlineto{\pgfqpoint{0.428623in}{2.007836in}}%
\pgfpathlineto{\pgfqpoint{0.432404in}{1.996084in}}%
\pgfpathlineto{\pgfqpoint{0.435813in}{1.985412in}}%
\pgfpathlineto{\pgfqpoint{0.436167in}{1.984333in}}%
\pgfpathlineto{\pgfqpoint{0.440025in}{1.972582in}}%
\pgfpathlineto{\pgfqpoint{0.443852in}{1.960830in}}%
\pgfpathlineto{\pgfqpoint{0.447648in}{1.949079in}}%
\pgfpathlineto{\pgfqpoint{0.447875in}{1.948374in}}%
\pgfpathlineto{\pgfqpoint{0.451552in}{1.937327in}}%
\pgfpathlineto{\pgfqpoint{0.455433in}{1.925576in}}%
\pgfpathlineto{\pgfqpoint{0.459283in}{1.913825in}}%
\pgfpathlineto{\pgfqpoint{0.459938in}{1.911825in}}%
\pgfpathlineto{\pgfqpoint{0.463234in}{1.902073in}}%
\pgfpathlineto{\pgfqpoint{0.467179in}{1.890322in}}%
\pgfpathlineto{\pgfqpoint{0.471093in}{1.878571in}}%
\pgfpathlineto{\pgfqpoint{0.472001in}{1.875842in}}%
\pgfpathlineto{\pgfqpoint{0.475106in}{1.866819in}}%
\pgfpathlineto{\pgfqpoint{0.479125in}{1.855068in}}%
\pgfpathlineto{\pgfqpoint{0.483112in}{1.843316in}}%
\pgfpathlineto{\pgfqpoint{0.484063in}{1.840509in}}%
\pgfpathlineto{\pgfqpoint{0.487206in}{1.831565in}}%
\pgfpathlineto{\pgfqpoint{0.491310in}{1.819814in}}%
\pgfpathclose%
\pgfusepath{fill}%
\end{pgfscope}%
\begin{pgfscope}%
\pgfpathrectangle{\pgfqpoint{0.423750in}{1.819814in}}{\pgfqpoint{1.194205in}{1.163386in}}%
\pgfusepath{clip}%
\pgfsetbuttcap%
\pgfsetroundjoin%
\definecolor{currentfill}{rgb}{0.477697,0.120699,0.349023}%
\pgfsetfillcolor{currentfill}%
\pgfsetlinewidth{0.000000pt}%
\definecolor{currentstroke}{rgb}{0.000000,0.000000,0.000000}%
\pgfsetstrokecolor{currentstroke}%
\pgfsetdash{}{0pt}%
\pgfpathmoveto{\pgfqpoint{1.280200in}{2.980421in}}%
\pgfpathlineto{\pgfqpoint{1.292262in}{2.977653in}}%
\pgfpathlineto{\pgfqpoint{1.304325in}{2.975923in}}%
\pgfpathlineto{\pgfqpoint{1.316388in}{2.975255in}}%
\pgfpathlineto{\pgfqpoint{1.328450in}{2.975662in}}%
\pgfpathlineto{\pgfqpoint{1.340513in}{2.977144in}}%
\pgfpathlineto{\pgfqpoint{1.352576in}{2.979691in}}%
\pgfpathlineto{\pgfqpoint{1.364374in}{2.983200in}}%
\pgfpathlineto{\pgfqpoint{1.352576in}{2.983200in}}%
\pgfpathlineto{\pgfqpoint{1.340513in}{2.983200in}}%
\pgfpathlineto{\pgfqpoint{1.328450in}{2.983200in}}%
\pgfpathlineto{\pgfqpoint{1.316388in}{2.983200in}}%
\pgfpathlineto{\pgfqpoint{1.304325in}{2.983200in}}%
\pgfpathlineto{\pgfqpoint{1.292262in}{2.983200in}}%
\pgfpathlineto{\pgfqpoint{1.280200in}{2.983200in}}%
\pgfpathlineto{\pgfqpoint{1.271305in}{2.983200in}}%
\pgfpathclose%
\pgfusepath{fill}%
\end{pgfscope}%
\begin{pgfscope}%
\pgfpathrectangle{\pgfqpoint{0.423750in}{1.819814in}}{\pgfqpoint{1.194205in}{1.163386in}}%
\pgfusepath{clip}%
\pgfsetbuttcap%
\pgfsetroundjoin%
\definecolor{currentfill}{rgb}{0.557345,0.113305,0.357751}%
\pgfsetfillcolor{currentfill}%
\pgfsetlinewidth{0.000000pt}%
\definecolor{currentstroke}{rgb}{0.000000,0.000000,0.000000}%
\pgfsetstrokecolor{currentstroke}%
\pgfsetdash{}{0pt}%
\pgfpathmoveto{\pgfqpoint{0.508189in}{1.819814in}}%
\pgfpathlineto{\pgfqpoint{0.520251in}{1.819814in}}%
\pgfpathlineto{\pgfqpoint{0.520303in}{1.819814in}}%
\pgfpathlineto{\pgfqpoint{0.520251in}{1.819949in}}%
\pgfpathlineto{\pgfqpoint{0.516056in}{1.831565in}}%
\pgfpathlineto{\pgfqpoint{0.511773in}{1.843316in}}%
\pgfpathlineto{\pgfqpoint{0.508189in}{1.853078in}}%
\pgfpathlineto{\pgfqpoint{0.507489in}{1.855068in}}%
\pgfpathlineto{\pgfqpoint{0.503351in}{1.866819in}}%
\pgfpathlineto{\pgfqpoint{0.499176in}{1.878571in}}%
\pgfpathlineto{\pgfqpoint{0.496126in}{1.887101in}}%
\pgfpathlineto{\pgfqpoint{0.495020in}{1.890322in}}%
\pgfpathlineto{\pgfqpoint{0.490978in}{1.902073in}}%
\pgfpathlineto{\pgfqpoint{0.486900in}{1.913825in}}%
\pgfpathlineto{\pgfqpoint{0.484063in}{1.921951in}}%
\pgfpathlineto{\pgfqpoint{0.482845in}{1.925576in}}%
\pgfpathlineto{\pgfqpoint{0.478888in}{1.937327in}}%
\pgfpathlineto{\pgfqpoint{0.474896in}{1.949079in}}%
\pgfpathlineto{\pgfqpoint{0.472001in}{1.957553in}}%
\pgfpathlineto{\pgfqpoint{0.470920in}{1.960830in}}%
\pgfpathlineto{\pgfqpoint{0.467037in}{1.972582in}}%
\pgfpathlineto{\pgfqpoint{0.463121in}{1.984333in}}%
\pgfpathlineto{\pgfqpoint{0.459938in}{1.993821in}}%
\pgfpathlineto{\pgfqpoint{0.459203in}{1.996084in}}%
\pgfpathlineto{\pgfqpoint{0.455386in}{2.007836in}}%
\pgfpathlineto{\pgfqpoint{0.451535in}{2.019587in}}%
\pgfpathlineto{\pgfqpoint{0.447875in}{2.030666in}}%
\pgfpathlineto{\pgfqpoint{0.447660in}{2.031338in}}%
\pgfpathlineto{\pgfqpoint{0.443897in}{2.043090in}}%
\pgfpathlineto{\pgfqpoint{0.440102in}{2.054841in}}%
\pgfpathlineto{\pgfqpoint{0.436273in}{2.066593in}}%
\pgfpathlineto{\pgfqpoint{0.435813in}{2.068006in}}%
\pgfpathlineto{\pgfqpoint{0.432539in}{2.078344in}}%
\pgfpathlineto{\pgfqpoint{0.428789in}{2.090095in}}%
\pgfpathlineto{\pgfqpoint{0.425006in}{2.101847in}}%
\pgfpathlineto{\pgfqpoint{0.423750in}{2.105741in}}%
\pgfpathlineto{\pgfqpoint{0.423750in}{2.101847in}}%
\pgfpathlineto{\pgfqpoint{0.423750in}{2.090095in}}%
\pgfpathlineto{\pgfqpoint{0.423750in}{2.078344in}}%
\pgfpathlineto{\pgfqpoint{0.423750in}{2.066593in}}%
\pgfpathlineto{\pgfqpoint{0.423750in}{2.063505in}}%
\pgfpathlineto{\pgfqpoint{0.426557in}{2.054841in}}%
\pgfpathlineto{\pgfqpoint{0.430342in}{2.043090in}}%
\pgfpathlineto{\pgfqpoint{0.434094in}{2.031338in}}%
\pgfpathlineto{\pgfqpoint{0.435813in}{2.025939in}}%
\pgfpathlineto{\pgfqpoint{0.437891in}{2.019587in}}%
\pgfpathlineto{\pgfqpoint{0.441721in}{2.007836in}}%
\pgfpathlineto{\pgfqpoint{0.445519in}{1.996084in}}%
\pgfpathlineto{\pgfqpoint{0.447875in}{1.988760in}}%
\pgfpathlineto{\pgfqpoint{0.449343in}{1.984333in}}%
\pgfpathlineto{\pgfqpoint{0.453228in}{1.972582in}}%
\pgfpathlineto{\pgfqpoint{0.457080in}{1.960830in}}%
\pgfpathlineto{\pgfqpoint{0.459938in}{1.952063in}}%
\pgfpathlineto{\pgfqpoint{0.460942in}{1.949079in}}%
\pgfpathlineto{\pgfqpoint{0.464893in}{1.937327in}}%
\pgfpathlineto{\pgfqpoint{0.468810in}{1.925576in}}%
\pgfpathlineto{\pgfqpoint{0.472001in}{1.915939in}}%
\pgfpathlineto{\pgfqpoint{0.472726in}{1.913825in}}%
\pgfpathlineto{\pgfqpoint{0.476751in}{1.902073in}}%
\pgfpathlineto{\pgfqpoint{0.480743in}{1.890322in}}%
\pgfpathlineto{\pgfqpoint{0.484063in}{1.880478in}}%
\pgfpathlineto{\pgfqpoint{0.484731in}{1.878571in}}%
\pgfpathlineto{\pgfqpoint{0.488843in}{1.866819in}}%
\pgfpathlineto{\pgfqpoint{0.492919in}{1.855068in}}%
\pgfpathlineto{\pgfqpoint{0.496126in}{1.845761in}}%
\pgfpathlineto{\pgfqpoint{0.497002in}{1.843316in}}%
\pgfpathlineto{\pgfqpoint{0.501210in}{1.831565in}}%
\pgfpathlineto{\pgfqpoint{0.505381in}{1.819814in}}%
\pgfpathclose%
\pgfusepath{fill}%
\end{pgfscope}%
\begin{pgfscope}%
\pgfpathrectangle{\pgfqpoint{0.423750in}{1.819814in}}{\pgfqpoint{1.194205in}{1.163386in}}%
\pgfusepath{clip}%
\pgfsetbuttcap%
\pgfsetroundjoin%
\definecolor{currentfill}{rgb}{0.557345,0.113305,0.357751}%
\pgfsetfillcolor{currentfill}%
\pgfsetlinewidth{0.000000pt}%
\definecolor{currentstroke}{rgb}{0.000000,0.000000,0.000000}%
\pgfsetstrokecolor{currentstroke}%
\pgfsetdash{}{0pt}%
\pgfpathmoveto{\pgfqpoint{1.292262in}{2.934719in}}%
\pgfpathlineto{\pgfqpoint{1.304325in}{2.932931in}}%
\pgfpathlineto{\pgfqpoint{1.316388in}{2.932236in}}%
\pgfpathlineto{\pgfqpoint{1.328450in}{2.932645in}}%
\pgfpathlineto{\pgfqpoint{1.340513in}{2.934160in}}%
\pgfpathlineto{\pgfqpoint{1.349925in}{2.936194in}}%
\pgfpathlineto{\pgfqpoint{1.352576in}{2.936762in}}%
\pgfpathlineto{\pgfqpoint{1.364638in}{2.940399in}}%
\pgfpathlineto{\pgfqpoint{1.376701in}{2.945059in}}%
\pgfpathlineto{\pgfqpoint{1.382887in}{2.947946in}}%
\pgfpathlineto{\pgfqpoint{1.388764in}{2.950665in}}%
\pgfpathlineto{\pgfqpoint{1.400826in}{2.957148in}}%
\pgfpathlineto{\pgfqpoint{1.405031in}{2.959697in}}%
\pgfpathlineto{\pgfqpoint{1.412889in}{2.964424in}}%
\pgfpathlineto{\pgfqpoint{1.423468in}{2.971449in}}%
\pgfpathlineto{\pgfqpoint{1.424952in}{2.972426in}}%
\pgfpathlineto{\pgfqpoint{1.437014in}{2.981031in}}%
\pgfpathlineto{\pgfqpoint{1.439868in}{2.983200in}}%
\pgfpathlineto{\pgfqpoint{1.437014in}{2.983200in}}%
\pgfpathlineto{\pgfqpoint{1.424952in}{2.983200in}}%
\pgfpathlineto{\pgfqpoint{1.412889in}{2.983200in}}%
\pgfpathlineto{\pgfqpoint{1.400826in}{2.983200in}}%
\pgfpathlineto{\pgfqpoint{1.388764in}{2.983200in}}%
\pgfpathlineto{\pgfqpoint{1.376701in}{2.983200in}}%
\pgfpathlineto{\pgfqpoint{1.364638in}{2.983200in}}%
\pgfpathlineto{\pgfqpoint{1.364374in}{2.983200in}}%
\pgfpathlineto{\pgfqpoint{1.352576in}{2.979691in}}%
\pgfpathlineto{\pgfqpoint{1.340513in}{2.977144in}}%
\pgfpathlineto{\pgfqpoint{1.328450in}{2.975662in}}%
\pgfpathlineto{\pgfqpoint{1.316388in}{2.975255in}}%
\pgfpathlineto{\pgfqpoint{1.304325in}{2.975923in}}%
\pgfpathlineto{\pgfqpoint{1.292262in}{2.977653in}}%
\pgfpathlineto{\pgfqpoint{1.280200in}{2.980421in}}%
\pgfpathlineto{\pgfqpoint{1.271305in}{2.983200in}}%
\pgfpathlineto{\pgfqpoint{1.268137in}{2.983200in}}%
\pgfpathlineto{\pgfqpoint{1.256074in}{2.983200in}}%
\pgfpathlineto{\pgfqpoint{1.244012in}{2.983200in}}%
\pgfpathlineto{\pgfqpoint{1.231949in}{2.983200in}}%
\pgfpathlineto{\pgfqpoint{1.219886in}{2.983200in}}%
\pgfpathlineto{\pgfqpoint{1.207824in}{2.983200in}}%
\pgfpathlineto{\pgfqpoint{1.195761in}{2.983200in}}%
\pgfpathlineto{\pgfqpoint{1.193787in}{2.983200in}}%
\pgfpathlineto{\pgfqpoint{1.195761in}{2.981755in}}%
\pgfpathlineto{\pgfqpoint{1.207824in}{2.973398in}}%
\pgfpathlineto{\pgfqpoint{1.210843in}{2.971449in}}%
\pgfpathlineto{\pgfqpoint{1.219886in}{2.965565in}}%
\pgfpathlineto{\pgfqpoint{1.229743in}{2.959697in}}%
\pgfpathlineto{\pgfqpoint{1.231949in}{2.958373in}}%
\pgfpathlineto{\pgfqpoint{1.244012in}{2.951880in}}%
\pgfpathlineto{\pgfqpoint{1.252395in}{2.947946in}}%
\pgfpathlineto{\pgfqpoint{1.256074in}{2.946203in}}%
\pgfpathlineto{\pgfqpoint{1.268137in}{2.941395in}}%
\pgfpathlineto{\pgfqpoint{1.280200in}{2.937556in}}%
\pgfpathlineto{\pgfqpoint{1.286020in}{2.936194in}}%
\pgfpathclose%
\pgfusepath{fill}%
\end{pgfscope}%
\begin{pgfscope}%
\pgfpathrectangle{\pgfqpoint{0.423750in}{1.819814in}}{\pgfqpoint{1.194205in}{1.163386in}}%
\pgfusepath{clip}%
\pgfsetbuttcap%
\pgfsetroundjoin%
\definecolor{currentfill}{rgb}{0.638121,0.099382,0.356038}%
\pgfsetfillcolor{currentfill}%
\pgfsetlinewidth{0.000000pt}%
\definecolor{currentstroke}{rgb}{0.000000,0.000000,0.000000}%
\pgfsetstrokecolor{currentstroke}%
\pgfsetdash{}{0pt}%
\pgfpathmoveto{\pgfqpoint{0.520251in}{1.819949in}}%
\pgfpathlineto{\pgfqpoint{0.520303in}{1.819814in}}%
\pgfpathlineto{\pgfqpoint{0.532314in}{1.819814in}}%
\pgfpathlineto{\pgfqpoint{0.536459in}{1.819814in}}%
\pgfpathlineto{\pgfqpoint{0.532314in}{1.830456in}}%
\pgfpathlineto{\pgfqpoint{0.531902in}{1.831565in}}%
\pgfpathlineto{\pgfqpoint{0.527541in}{1.843316in}}%
\pgfpathlineto{\pgfqpoint{0.523137in}{1.855068in}}%
\pgfpathlineto{\pgfqpoint{0.520251in}{1.862720in}}%
\pgfpathlineto{\pgfqpoint{0.518775in}{1.866819in}}%
\pgfpathlineto{\pgfqpoint{0.514533in}{1.878571in}}%
\pgfpathlineto{\pgfqpoint{0.510250in}{1.890322in}}%
\pgfpathlineto{\pgfqpoint{0.508189in}{1.895954in}}%
\pgfpathlineto{\pgfqpoint{0.506045in}{1.902073in}}%
\pgfpathlineto{\pgfqpoint{0.501911in}{1.913825in}}%
\pgfpathlineto{\pgfqpoint{0.497738in}{1.925576in}}%
\pgfpathlineto{\pgfqpoint{0.496126in}{1.930102in}}%
\pgfpathlineto{\pgfqpoint{0.493657in}{1.937327in}}%
\pgfpathlineto{\pgfqpoint{0.489620in}{1.949079in}}%
\pgfpathlineto{\pgfqpoint{0.485547in}{1.960830in}}%
\pgfpathlineto{\pgfqpoint{0.484063in}{1.965098in}}%
\pgfpathlineto{\pgfqpoint{0.481561in}{1.972582in}}%
\pgfpathlineto{\pgfqpoint{0.477611in}{1.984333in}}%
\pgfpathlineto{\pgfqpoint{0.473626in}{1.996084in}}%
\pgfpathlineto{\pgfqpoint{0.472001in}{2.000861in}}%
\pgfpathlineto{\pgfqpoint{0.469712in}{2.007836in}}%
\pgfpathlineto{\pgfqpoint{0.465839in}{2.019587in}}%
\pgfpathlineto{\pgfqpoint{0.461930in}{2.031338in}}%
\pgfpathlineto{\pgfqpoint{0.459938in}{2.037304in}}%
\pgfpathlineto{\pgfqpoint{0.458070in}{2.043090in}}%
\pgfpathlineto{\pgfqpoint{0.454262in}{2.054841in}}%
\pgfpathlineto{\pgfqpoint{0.450419in}{2.066593in}}%
\pgfpathlineto{\pgfqpoint{0.447875in}{2.074330in}}%
\pgfpathlineto{\pgfqpoint{0.446596in}{2.078344in}}%
\pgfpathlineto{\pgfqpoint{0.442843in}{2.090095in}}%
\pgfpathlineto{\pgfqpoint{0.439057in}{2.101847in}}%
\pgfpathlineto{\pgfqpoint{0.435813in}{2.111840in}}%
\pgfpathlineto{\pgfqpoint{0.435258in}{2.113598in}}%
\pgfpathlineto{\pgfqpoint{0.431551in}{2.125349in}}%
\pgfpathlineto{\pgfqpoint{0.427811in}{2.137101in}}%
\pgfpathlineto{\pgfqpoint{0.424036in}{2.148852in}}%
\pgfpathlineto{\pgfqpoint{0.423750in}{2.149741in}}%
\pgfpathlineto{\pgfqpoint{0.423750in}{2.148852in}}%
\pgfpathlineto{\pgfqpoint{0.423750in}{2.137101in}}%
\pgfpathlineto{\pgfqpoint{0.423750in}{2.125349in}}%
\pgfpathlineto{\pgfqpoint{0.423750in}{2.113598in}}%
\pgfpathlineto{\pgfqpoint{0.423750in}{2.105741in}}%
\pgfpathlineto{\pgfqpoint{0.425006in}{2.101847in}}%
\pgfpathlineto{\pgfqpoint{0.428789in}{2.090095in}}%
\pgfpathlineto{\pgfqpoint{0.432539in}{2.078344in}}%
\pgfpathlineto{\pgfqpoint{0.435813in}{2.068006in}}%
\pgfpathlineto{\pgfqpoint{0.436273in}{2.066593in}}%
\pgfpathlineto{\pgfqpoint{0.440102in}{2.054841in}}%
\pgfpathlineto{\pgfqpoint{0.443897in}{2.043090in}}%
\pgfpathlineto{\pgfqpoint{0.447660in}{2.031338in}}%
\pgfpathlineto{\pgfqpoint{0.447875in}{2.030666in}}%
\pgfpathlineto{\pgfqpoint{0.451535in}{2.019587in}}%
\pgfpathlineto{\pgfqpoint{0.455386in}{2.007836in}}%
\pgfpathlineto{\pgfqpoint{0.459203in}{1.996084in}}%
\pgfpathlineto{\pgfqpoint{0.459938in}{1.993821in}}%
\pgfpathlineto{\pgfqpoint{0.463121in}{1.984333in}}%
\pgfpathlineto{\pgfqpoint{0.467037in}{1.972582in}}%
\pgfpathlineto{\pgfqpoint{0.470920in}{1.960830in}}%
\pgfpathlineto{\pgfqpoint{0.472001in}{1.957553in}}%
\pgfpathlineto{\pgfqpoint{0.474896in}{1.949079in}}%
\pgfpathlineto{\pgfqpoint{0.478888in}{1.937327in}}%
\pgfpathlineto{\pgfqpoint{0.482845in}{1.925576in}}%
\pgfpathlineto{\pgfqpoint{0.484063in}{1.921951in}}%
\pgfpathlineto{\pgfqpoint{0.486900in}{1.913825in}}%
\pgfpathlineto{\pgfqpoint{0.490978in}{1.902073in}}%
\pgfpathlineto{\pgfqpoint{0.495020in}{1.890322in}}%
\pgfpathlineto{\pgfqpoint{0.496126in}{1.887101in}}%
\pgfpathlineto{\pgfqpoint{0.499176in}{1.878571in}}%
\pgfpathlineto{\pgfqpoint{0.503351in}{1.866819in}}%
\pgfpathlineto{\pgfqpoint{0.507489in}{1.855068in}}%
\pgfpathlineto{\pgfqpoint{0.508189in}{1.853078in}}%
\pgfpathlineto{\pgfqpoint{0.511773in}{1.843316in}}%
\pgfpathlineto{\pgfqpoint{0.516056in}{1.831565in}}%
\pgfpathclose%
\pgfusepath{fill}%
\end{pgfscope}%
\begin{pgfscope}%
\pgfpathrectangle{\pgfqpoint{0.423750in}{1.819814in}}{\pgfqpoint{1.194205in}{1.163386in}}%
\pgfusepath{clip}%
\pgfsetbuttcap%
\pgfsetroundjoin%
\definecolor{currentfill}{rgb}{0.638121,0.099382,0.356038}%
\pgfsetfillcolor{currentfill}%
\pgfsetlinewidth{0.000000pt}%
\definecolor{currentstroke}{rgb}{0.000000,0.000000,0.000000}%
\pgfsetstrokecolor{currentstroke}%
\pgfsetdash{}{0pt}%
\pgfpathmoveto{\pgfqpoint{1.304325in}{2.888067in}}%
\pgfpathlineto{\pgfqpoint{1.316388in}{2.887341in}}%
\pgfpathlineto{\pgfqpoint{1.328450in}{2.887752in}}%
\pgfpathlineto{\pgfqpoint{1.339642in}{2.889189in}}%
\pgfpathlineto{\pgfqpoint{1.340513in}{2.889300in}}%
\pgfpathlineto{\pgfqpoint{1.352576in}{2.891942in}}%
\pgfpathlineto{\pgfqpoint{1.364638in}{2.895672in}}%
\pgfpathlineto{\pgfqpoint{1.376701in}{2.900454in}}%
\pgfpathlineto{\pgfqpoint{1.377719in}{2.900940in}}%
\pgfpathlineto{\pgfqpoint{1.388764in}{2.906176in}}%
\pgfpathlineto{\pgfqpoint{1.400584in}{2.912692in}}%
\pgfpathlineto{\pgfqpoint{1.400826in}{2.912824in}}%
\pgfpathlineto{\pgfqpoint{1.412889in}{2.920253in}}%
\pgfpathlineto{\pgfqpoint{1.419058in}{2.924443in}}%
\pgfpathlineto{\pgfqpoint{1.424952in}{2.928417in}}%
\pgfpathlineto{\pgfqpoint{1.435608in}{2.936194in}}%
\pgfpathlineto{\pgfqpoint{1.437014in}{2.937214in}}%
\pgfpathlineto{\pgfqpoint{1.449077in}{2.946524in}}%
\pgfpathlineto{\pgfqpoint{1.450830in}{2.947946in}}%
\pgfpathlineto{\pgfqpoint{1.461140in}{2.956255in}}%
\pgfpathlineto{\pgfqpoint{1.465250in}{2.959697in}}%
\pgfpathlineto{\pgfqpoint{1.473202in}{2.966319in}}%
\pgfpathlineto{\pgfqpoint{1.479194in}{2.971449in}}%
\pgfpathlineto{\pgfqpoint{1.485265in}{2.976620in}}%
\pgfpathlineto{\pgfqpoint{1.492847in}{2.983200in}}%
\pgfpathlineto{\pgfqpoint{1.485265in}{2.983200in}}%
\pgfpathlineto{\pgfqpoint{1.473202in}{2.983200in}}%
\pgfpathlineto{\pgfqpoint{1.461140in}{2.983200in}}%
\pgfpathlineto{\pgfqpoint{1.449077in}{2.983200in}}%
\pgfpathlineto{\pgfqpoint{1.439868in}{2.983200in}}%
\pgfpathlineto{\pgfqpoint{1.437014in}{2.981031in}}%
\pgfpathlineto{\pgfqpoint{1.424952in}{2.972426in}}%
\pgfpathlineto{\pgfqpoint{1.423468in}{2.971449in}}%
\pgfpathlineto{\pgfqpoint{1.412889in}{2.964424in}}%
\pgfpathlineto{\pgfqpoint{1.405031in}{2.959697in}}%
\pgfpathlineto{\pgfqpoint{1.400826in}{2.957148in}}%
\pgfpathlineto{\pgfqpoint{1.388764in}{2.950665in}}%
\pgfpathlineto{\pgfqpoint{1.382887in}{2.947946in}}%
\pgfpathlineto{\pgfqpoint{1.376701in}{2.945059in}}%
\pgfpathlineto{\pgfqpoint{1.364638in}{2.940399in}}%
\pgfpathlineto{\pgfqpoint{1.352576in}{2.936762in}}%
\pgfpathlineto{\pgfqpoint{1.349925in}{2.936194in}}%
\pgfpathlineto{\pgfqpoint{1.340513in}{2.934160in}}%
\pgfpathlineto{\pgfqpoint{1.328450in}{2.932645in}}%
\pgfpathlineto{\pgfqpoint{1.316388in}{2.932236in}}%
\pgfpathlineto{\pgfqpoint{1.304325in}{2.932931in}}%
\pgfpathlineto{\pgfqpoint{1.292262in}{2.934719in}}%
\pgfpathlineto{\pgfqpoint{1.286020in}{2.936194in}}%
\pgfpathlineto{\pgfqpoint{1.280200in}{2.937556in}}%
\pgfpathlineto{\pgfqpoint{1.268137in}{2.941395in}}%
\pgfpathlineto{\pgfqpoint{1.256074in}{2.946203in}}%
\pgfpathlineto{\pgfqpoint{1.252395in}{2.947946in}}%
\pgfpathlineto{\pgfqpoint{1.244012in}{2.951880in}}%
\pgfpathlineto{\pgfqpoint{1.231949in}{2.958373in}}%
\pgfpathlineto{\pgfqpoint{1.229743in}{2.959697in}}%
\pgfpathlineto{\pgfqpoint{1.219886in}{2.965565in}}%
\pgfpathlineto{\pgfqpoint{1.210843in}{2.971449in}}%
\pgfpathlineto{\pgfqpoint{1.207824in}{2.973398in}}%
\pgfpathlineto{\pgfqpoint{1.195761in}{2.981755in}}%
\pgfpathlineto{\pgfqpoint{1.193787in}{2.983200in}}%
\pgfpathlineto{\pgfqpoint{1.183698in}{2.983200in}}%
\pgfpathlineto{\pgfqpoint{1.171636in}{2.983200in}}%
\pgfpathlineto{\pgfqpoint{1.159573in}{2.983200in}}%
\pgfpathlineto{\pgfqpoint{1.147510in}{2.983200in}}%
\pgfpathlineto{\pgfqpoint{1.138135in}{2.983200in}}%
\pgfpathlineto{\pgfqpoint{1.147510in}{2.975596in}}%
\pgfpathlineto{\pgfqpoint{1.152686in}{2.971449in}}%
\pgfpathlineto{\pgfqpoint{1.159573in}{2.965900in}}%
\pgfpathlineto{\pgfqpoint{1.167434in}{2.959697in}}%
\pgfpathlineto{\pgfqpoint{1.171636in}{2.956362in}}%
\pgfpathlineto{\pgfqpoint{1.182569in}{2.947946in}}%
\pgfpathlineto{\pgfqpoint{1.183698in}{2.947071in}}%
\pgfpathlineto{\pgfqpoint{1.195761in}{2.938100in}}%
\pgfpathlineto{\pgfqpoint{1.198468in}{2.936194in}}%
\pgfpathlineto{\pgfqpoint{1.207824in}{2.929558in}}%
\pgfpathlineto{\pgfqpoint{1.215559in}{2.924443in}}%
\pgfpathlineto{\pgfqpoint{1.219886in}{2.921559in}}%
\pgfpathlineto{\pgfqpoint{1.231949in}{2.914190in}}%
\pgfpathlineto{\pgfqpoint{1.234683in}{2.912692in}}%
\pgfpathlineto{\pgfqpoint{1.244012in}{2.907533in}}%
\pgfpathlineto{\pgfqpoint{1.256074in}{2.901715in}}%
\pgfpathlineto{\pgfqpoint{1.257981in}{2.900940in}}%
\pgfpathlineto{\pgfqpoint{1.268137in}{2.896773in}}%
\pgfpathlineto{\pgfqpoint{1.280200in}{2.892817in}}%
\pgfpathlineto{\pgfqpoint{1.292262in}{2.889907in}}%
\pgfpathlineto{\pgfqpoint{1.297003in}{2.889189in}}%
\pgfpathclose%
\pgfusepath{fill}%
\end{pgfscope}%
\begin{pgfscope}%
\pgfpathrectangle{\pgfqpoint{0.423750in}{1.819814in}}{\pgfqpoint{1.194205in}{1.163386in}}%
\pgfusepath{clip}%
\pgfsetbuttcap%
\pgfsetroundjoin%
\definecolor{currentfill}{rgb}{0.717551,0.087162,0.341300}%
\pgfsetfillcolor{currentfill}%
\pgfsetlinewidth{0.000000pt}%
\definecolor{currentstroke}{rgb}{0.000000,0.000000,0.000000}%
\pgfsetstrokecolor{currentstroke}%
\pgfsetdash{}{0pt}%
\pgfpathmoveto{\pgfqpoint{0.532314in}{1.830456in}}%
\pgfpathlineto{\pgfqpoint{0.536459in}{1.819814in}}%
\pgfpathlineto{\pgfqpoint{0.544377in}{1.819814in}}%
\pgfpathlineto{\pgfqpoint{0.553918in}{1.819814in}}%
\pgfpathlineto{\pgfqpoint{0.549240in}{1.831565in}}%
\pgfpathlineto{\pgfqpoint{0.544510in}{1.843316in}}%
\pgfpathlineto{\pgfqpoint{0.544377in}{1.843646in}}%
\pgfpathlineto{\pgfqpoint{0.540014in}{1.855068in}}%
\pgfpathlineto{\pgfqpoint{0.535480in}{1.866819in}}%
\pgfpathlineto{\pgfqpoint{0.532314in}{1.874965in}}%
\pgfpathlineto{\pgfqpoint{0.530981in}{1.878571in}}%
\pgfpathlineto{\pgfqpoint{0.526625in}{1.890322in}}%
\pgfpathlineto{\pgfqpoint{0.522226in}{1.902073in}}%
\pgfpathlineto{\pgfqpoint{0.520251in}{1.907325in}}%
\pgfpathlineto{\pgfqpoint{0.517921in}{1.913825in}}%
\pgfpathlineto{\pgfqpoint{0.513687in}{1.925576in}}%
\pgfpathlineto{\pgfqpoint{0.509411in}{1.937327in}}%
\pgfpathlineto{\pgfqpoint{0.508189in}{1.940681in}}%
\pgfpathlineto{\pgfqpoint{0.505261in}{1.949079in}}%
\pgfpathlineto{\pgfqpoint{0.501138in}{1.960830in}}%
\pgfpathlineto{\pgfqpoint{0.496975in}{1.972582in}}%
\pgfpathlineto{\pgfqpoint{0.496126in}{1.974976in}}%
\pgfpathlineto{\pgfqpoint{0.492945in}{1.984333in}}%
\pgfpathlineto{\pgfqpoint{0.488921in}{1.996084in}}%
\pgfpathlineto{\pgfqpoint{0.484860in}{2.007836in}}%
\pgfpathlineto{\pgfqpoint{0.484063in}{2.010138in}}%
\pgfpathlineto{\pgfqpoint{0.480922in}{2.019587in}}%
\pgfpathlineto{\pgfqpoint{0.476987in}{2.031338in}}%
\pgfpathlineto{\pgfqpoint{0.473015in}{2.043090in}}%
\pgfpathlineto{\pgfqpoint{0.472001in}{2.046086in}}%
\pgfpathlineto{\pgfqpoint{0.469145in}{2.054841in}}%
\pgfpathlineto{\pgfqpoint{0.465287in}{2.066593in}}%
\pgfpathlineto{\pgfqpoint{0.461393in}{2.078344in}}%
\pgfpathlineto{\pgfqpoint{0.459938in}{2.082725in}}%
\pgfpathlineto{\pgfqpoint{0.457572in}{2.090095in}}%
\pgfpathlineto{\pgfqpoint{0.453781in}{2.101847in}}%
\pgfpathlineto{\pgfqpoint{0.449955in}{2.113598in}}%
\pgfpathlineto{\pgfqpoint{0.447875in}{2.119956in}}%
\pgfpathlineto{\pgfqpoint{0.446167in}{2.125349in}}%
\pgfpathlineto{\pgfqpoint{0.442431in}{2.137101in}}%
\pgfpathlineto{\pgfqpoint{0.438661in}{2.148852in}}%
\pgfpathlineto{\pgfqpoint{0.435813in}{2.157673in}}%
\pgfpathlineto{\pgfqpoint{0.434894in}{2.160604in}}%
\pgfpathlineto{\pgfqpoint{0.431204in}{2.172355in}}%
\pgfpathlineto{\pgfqpoint{0.427481in}{2.184106in}}%
\pgfpathlineto{\pgfqpoint{0.423750in}{2.195769in}}%
\pgfpathlineto{\pgfqpoint{0.423750in}{2.184106in}}%
\pgfpathlineto{\pgfqpoint{0.423750in}{2.172355in}}%
\pgfpathlineto{\pgfqpoint{0.423750in}{2.160604in}}%
\pgfpathlineto{\pgfqpoint{0.423750in}{2.149741in}}%
\pgfpathlineto{\pgfqpoint{0.424036in}{2.148852in}}%
\pgfpathlineto{\pgfqpoint{0.427811in}{2.137101in}}%
\pgfpathlineto{\pgfqpoint{0.431551in}{2.125349in}}%
\pgfpathlineto{\pgfqpoint{0.435258in}{2.113598in}}%
\pgfpathlineto{\pgfqpoint{0.435813in}{2.111840in}}%
\pgfpathlineto{\pgfqpoint{0.439057in}{2.101847in}}%
\pgfpathlineto{\pgfqpoint{0.442843in}{2.090095in}}%
\pgfpathlineto{\pgfqpoint{0.446596in}{2.078344in}}%
\pgfpathlineto{\pgfqpoint{0.447875in}{2.074330in}}%
\pgfpathlineto{\pgfqpoint{0.450419in}{2.066593in}}%
\pgfpathlineto{\pgfqpoint{0.454262in}{2.054841in}}%
\pgfpathlineto{\pgfqpoint{0.458070in}{2.043090in}}%
\pgfpathlineto{\pgfqpoint{0.459938in}{2.037304in}}%
\pgfpathlineto{\pgfqpoint{0.461930in}{2.031338in}}%
\pgfpathlineto{\pgfqpoint{0.465839in}{2.019587in}}%
\pgfpathlineto{\pgfqpoint{0.469712in}{2.007836in}}%
\pgfpathlineto{\pgfqpoint{0.472001in}{2.000861in}}%
\pgfpathlineto{\pgfqpoint{0.473626in}{1.996084in}}%
\pgfpathlineto{\pgfqpoint{0.477611in}{1.984333in}}%
\pgfpathlineto{\pgfqpoint{0.481561in}{1.972582in}}%
\pgfpathlineto{\pgfqpoint{0.484063in}{1.965098in}}%
\pgfpathlineto{\pgfqpoint{0.485547in}{1.960830in}}%
\pgfpathlineto{\pgfqpoint{0.489620in}{1.949079in}}%
\pgfpathlineto{\pgfqpoint{0.493657in}{1.937327in}}%
\pgfpathlineto{\pgfqpoint{0.496126in}{1.930102in}}%
\pgfpathlineto{\pgfqpoint{0.497738in}{1.925576in}}%
\pgfpathlineto{\pgfqpoint{0.501911in}{1.913825in}}%
\pgfpathlineto{\pgfqpoint{0.506045in}{1.902073in}}%
\pgfpathlineto{\pgfqpoint{0.508189in}{1.895954in}}%
\pgfpathlineto{\pgfqpoint{0.510250in}{1.890322in}}%
\pgfpathlineto{\pgfqpoint{0.514533in}{1.878571in}}%
\pgfpathlineto{\pgfqpoint{0.518775in}{1.866819in}}%
\pgfpathlineto{\pgfqpoint{0.520251in}{1.862720in}}%
\pgfpathlineto{\pgfqpoint{0.523137in}{1.855068in}}%
\pgfpathlineto{\pgfqpoint{0.527541in}{1.843316in}}%
\pgfpathlineto{\pgfqpoint{0.531902in}{1.831565in}}%
\pgfpathclose%
\pgfusepath{fill}%
\end{pgfscope}%
\begin{pgfscope}%
\pgfpathrectangle{\pgfqpoint{0.423750in}{1.819814in}}{\pgfqpoint{1.194205in}{1.163386in}}%
\pgfusepath{clip}%
\pgfsetbuttcap%
\pgfsetroundjoin%
\definecolor{currentfill}{rgb}{0.717551,0.087162,0.341300}%
\pgfsetfillcolor{currentfill}%
\pgfsetlinewidth{0.000000pt}%
\definecolor{currentstroke}{rgb}{0.000000,0.000000,0.000000}%
\pgfsetstrokecolor{currentstroke}%
\pgfsetdash{}{0pt}%
\pgfpathmoveto{\pgfqpoint{1.304325in}{2.841046in}}%
\pgfpathlineto{\pgfqpoint{1.316388in}{2.840287in}}%
\pgfpathlineto{\pgfqpoint{1.328450in}{2.840699in}}%
\pgfpathlineto{\pgfqpoint{1.339752in}{2.842183in}}%
\pgfpathlineto{\pgfqpoint{1.340513in}{2.842282in}}%
\pgfpathlineto{\pgfqpoint{1.352576in}{2.844991in}}%
\pgfpathlineto{\pgfqpoint{1.364638in}{2.848820in}}%
\pgfpathlineto{\pgfqpoint{1.376701in}{2.853730in}}%
\pgfpathlineto{\pgfqpoint{1.377119in}{2.853935in}}%
\pgfpathlineto{\pgfqpoint{1.388764in}{2.859600in}}%
\pgfpathlineto{\pgfqpoint{1.399523in}{2.865686in}}%
\pgfpathlineto{\pgfqpoint{1.400826in}{2.866417in}}%
\pgfpathlineto{\pgfqpoint{1.412889in}{2.874036in}}%
\pgfpathlineto{\pgfqpoint{1.417774in}{2.877438in}}%
\pgfpathlineto{\pgfqpoint{1.424952in}{2.882395in}}%
\pgfpathlineto{\pgfqpoint{1.434043in}{2.889189in}}%
\pgfpathlineto{\pgfqpoint{1.437014in}{2.891393in}}%
\pgfpathlineto{\pgfqpoint{1.449077in}{2.900921in}}%
\pgfpathlineto{\pgfqpoint{1.449100in}{2.900940in}}%
\pgfpathlineto{\pgfqpoint{1.461140in}{2.910851in}}%
\pgfpathlineto{\pgfqpoint{1.463294in}{2.912692in}}%
\pgfpathlineto{\pgfqpoint{1.473202in}{2.921106in}}%
\pgfpathlineto{\pgfqpoint{1.477029in}{2.924443in}}%
\pgfpathlineto{\pgfqpoint{1.485265in}{2.931586in}}%
\pgfpathlineto{\pgfqpoint{1.490488in}{2.936194in}}%
\pgfpathlineto{\pgfqpoint{1.497328in}{2.942200in}}%
\pgfpathlineto{\pgfqpoint{1.503812in}{2.947946in}}%
\pgfpathlineto{\pgfqpoint{1.509390in}{2.952866in}}%
\pgfpathlineto{\pgfqpoint{1.517121in}{2.959697in}}%
\pgfpathlineto{\pgfqpoint{1.521453in}{2.963510in}}%
\pgfpathlineto{\pgfqpoint{1.530513in}{2.971449in}}%
\pgfpathlineto{\pgfqpoint{1.533516in}{2.974070in}}%
\pgfpathlineto{\pgfqpoint{1.544072in}{2.983200in}}%
\pgfpathlineto{\pgfqpoint{1.533516in}{2.983200in}}%
\pgfpathlineto{\pgfqpoint{1.521453in}{2.983200in}}%
\pgfpathlineto{\pgfqpoint{1.509390in}{2.983200in}}%
\pgfpathlineto{\pgfqpoint{1.497328in}{2.983200in}}%
\pgfpathlineto{\pgfqpoint{1.492847in}{2.983200in}}%
\pgfpathlineto{\pgfqpoint{1.485265in}{2.976620in}}%
\pgfpathlineto{\pgfqpoint{1.479194in}{2.971449in}}%
\pgfpathlineto{\pgfqpoint{1.473202in}{2.966319in}}%
\pgfpathlineto{\pgfqpoint{1.465250in}{2.959697in}}%
\pgfpathlineto{\pgfqpoint{1.461140in}{2.956255in}}%
\pgfpathlineto{\pgfqpoint{1.450830in}{2.947946in}}%
\pgfpathlineto{\pgfqpoint{1.449077in}{2.946524in}}%
\pgfpathlineto{\pgfqpoint{1.437014in}{2.937214in}}%
\pgfpathlineto{\pgfqpoint{1.435608in}{2.936194in}}%
\pgfpathlineto{\pgfqpoint{1.424952in}{2.928417in}}%
\pgfpathlineto{\pgfqpoint{1.419058in}{2.924443in}}%
\pgfpathlineto{\pgfqpoint{1.412889in}{2.920253in}}%
\pgfpathlineto{\pgfqpoint{1.400826in}{2.912824in}}%
\pgfpathlineto{\pgfqpoint{1.400584in}{2.912692in}}%
\pgfpathlineto{\pgfqpoint{1.388764in}{2.906176in}}%
\pgfpathlineto{\pgfqpoint{1.377719in}{2.900940in}}%
\pgfpathlineto{\pgfqpoint{1.376701in}{2.900454in}}%
\pgfpathlineto{\pgfqpoint{1.364638in}{2.895672in}}%
\pgfpathlineto{\pgfqpoint{1.352576in}{2.891942in}}%
\pgfpathlineto{\pgfqpoint{1.340513in}{2.889300in}}%
\pgfpathlineto{\pgfqpoint{1.339642in}{2.889189in}}%
\pgfpathlineto{\pgfqpoint{1.328450in}{2.887752in}}%
\pgfpathlineto{\pgfqpoint{1.316388in}{2.887341in}}%
\pgfpathlineto{\pgfqpoint{1.304325in}{2.888067in}}%
\pgfpathlineto{\pgfqpoint{1.297003in}{2.889189in}}%
\pgfpathlineto{\pgfqpoint{1.292262in}{2.889907in}}%
\pgfpathlineto{\pgfqpoint{1.280200in}{2.892817in}}%
\pgfpathlineto{\pgfqpoint{1.268137in}{2.896773in}}%
\pgfpathlineto{\pgfqpoint{1.257981in}{2.900940in}}%
\pgfpathlineto{\pgfqpoint{1.256074in}{2.901715in}}%
\pgfpathlineto{\pgfqpoint{1.244012in}{2.907533in}}%
\pgfpathlineto{\pgfqpoint{1.234683in}{2.912692in}}%
\pgfpathlineto{\pgfqpoint{1.231949in}{2.914190in}}%
\pgfpathlineto{\pgfqpoint{1.219886in}{2.921559in}}%
\pgfpathlineto{\pgfqpoint{1.215559in}{2.924443in}}%
\pgfpathlineto{\pgfqpoint{1.207824in}{2.929558in}}%
\pgfpathlineto{\pgfqpoint{1.198468in}{2.936194in}}%
\pgfpathlineto{\pgfqpoint{1.195761in}{2.938100in}}%
\pgfpathlineto{\pgfqpoint{1.183698in}{2.947071in}}%
\pgfpathlineto{\pgfqpoint{1.182569in}{2.947946in}}%
\pgfpathlineto{\pgfqpoint{1.171636in}{2.956362in}}%
\pgfpathlineto{\pgfqpoint{1.167434in}{2.959697in}}%
\pgfpathlineto{\pgfqpoint{1.159573in}{2.965900in}}%
\pgfpathlineto{\pgfqpoint{1.152686in}{2.971449in}}%
\pgfpathlineto{\pgfqpoint{1.147510in}{2.975596in}}%
\pgfpathlineto{\pgfqpoint{1.138135in}{2.983200in}}%
\pgfpathlineto{\pgfqpoint{1.135448in}{2.983200in}}%
\pgfpathlineto{\pgfqpoint{1.123385in}{2.983200in}}%
\pgfpathlineto{\pgfqpoint{1.111322in}{2.983200in}}%
\pgfpathlineto{\pgfqpoint{1.099260in}{2.983200in}}%
\pgfpathlineto{\pgfqpoint{1.087197in}{2.983200in}}%
\pgfpathlineto{\pgfqpoint{1.082477in}{2.983200in}}%
\pgfpathlineto{\pgfqpoint{1.087197in}{2.979547in}}%
\pgfpathlineto{\pgfqpoint{1.097478in}{2.971449in}}%
\pgfpathlineto{\pgfqpoint{1.099260in}{2.970040in}}%
\pgfpathlineto{\pgfqpoint{1.111322in}{2.960359in}}%
\pgfpathlineto{\pgfqpoint{1.112138in}{2.959697in}}%
\pgfpathlineto{\pgfqpoint{1.123385in}{2.950535in}}%
\pgfpathlineto{\pgfqpoint{1.126550in}{2.947946in}}%
\pgfpathlineto{\pgfqpoint{1.135448in}{2.940633in}}%
\pgfpathlineto{\pgfqpoint{1.140862in}{2.936194in}}%
\pgfpathlineto{\pgfqpoint{1.147510in}{2.930716in}}%
\pgfpathlineto{\pgfqpoint{1.155202in}{2.924443in}}%
\pgfpathlineto{\pgfqpoint{1.159573in}{2.920857in}}%
\pgfpathlineto{\pgfqpoint{1.169720in}{2.912692in}}%
\pgfpathlineto{\pgfqpoint{1.171636in}{2.911141in}}%
\pgfpathlineto{\pgfqpoint{1.183698in}{2.901650in}}%
\pgfpathlineto{\pgfqpoint{1.184637in}{2.900940in}}%
\pgfpathlineto{\pgfqpoint{1.195761in}{2.892472in}}%
\pgfpathlineto{\pgfqpoint{1.200313in}{2.889189in}}%
\pgfpathlineto{\pgfqpoint{1.207824in}{2.883729in}}%
\pgfpathlineto{\pgfqpoint{1.217097in}{2.877438in}}%
\pgfpathlineto{\pgfqpoint{1.219886in}{2.875530in}}%
\pgfpathlineto{\pgfqpoint{1.231949in}{2.867957in}}%
\pgfpathlineto{\pgfqpoint{1.235980in}{2.865686in}}%
\pgfpathlineto{\pgfqpoint{1.244012in}{2.861118in}}%
\pgfpathlineto{\pgfqpoint{1.256074in}{2.855128in}}%
\pgfpathlineto{\pgfqpoint{1.258925in}{2.853935in}}%
\pgfpathlineto{\pgfqpoint{1.268137in}{2.850039in}}%
\pgfpathlineto{\pgfqpoint{1.280200in}{2.845958in}}%
\pgfpathlineto{\pgfqpoint{1.292262in}{2.842952in}}%
\pgfpathlineto{\pgfqpoint{1.297160in}{2.842183in}}%
\pgfpathclose%
\pgfusepath{fill}%
\end{pgfscope}%
\begin{pgfscope}%
\pgfpathrectangle{\pgfqpoint{0.423750in}{1.819814in}}{\pgfqpoint{1.194205in}{1.163386in}}%
\pgfusepath{clip}%
\pgfsetbuttcap%
\pgfsetroundjoin%
\definecolor{currentfill}{rgb}{0.796501,0.105066,0.310630}%
\pgfsetfillcolor{currentfill}%
\pgfsetlinewidth{0.000000pt}%
\definecolor{currentstroke}{rgb}{0.000000,0.000000,0.000000}%
\pgfsetstrokecolor{currentstroke}%
\pgfsetdash{}{0pt}%
\pgfpathmoveto{\pgfqpoint{0.556439in}{1.819814in}}%
\pgfpathlineto{\pgfqpoint{0.568502in}{1.819814in}}%
\pgfpathlineto{\pgfqpoint{0.573262in}{1.819814in}}%
\pgfpathlineto{\pgfqpoint{0.568502in}{1.830754in}}%
\pgfpathlineto{\pgfqpoint{0.568169in}{1.831565in}}%
\pgfpathlineto{\pgfqpoint{0.563341in}{1.843316in}}%
\pgfpathlineto{\pgfqpoint{0.558456in}{1.855068in}}%
\pgfpathlineto{\pgfqpoint{0.556439in}{1.859894in}}%
\pgfpathlineto{\pgfqpoint{0.553704in}{1.866819in}}%
\pgfpathlineto{\pgfqpoint{0.549033in}{1.878571in}}%
\pgfpathlineto{\pgfqpoint{0.544377in}{1.890151in}}%
\pgfpathlineto{\pgfqpoint{0.544312in}{1.890322in}}%
\pgfpathlineto{\pgfqpoint{0.539837in}{1.902073in}}%
\pgfpathlineto{\pgfqpoint{0.535313in}{1.913825in}}%
\pgfpathlineto{\pgfqpoint{0.532314in}{1.921560in}}%
\pgfpathlineto{\pgfqpoint{0.530835in}{1.925576in}}%
\pgfpathlineto{\pgfqpoint{0.526495in}{1.937327in}}%
\pgfpathlineto{\pgfqpoint{0.522109in}{1.949079in}}%
\pgfpathlineto{\pgfqpoint{0.520251in}{1.954037in}}%
\pgfpathlineto{\pgfqpoint{0.517827in}{1.960830in}}%
\pgfpathlineto{\pgfqpoint{0.513612in}{1.972582in}}%
\pgfpathlineto{\pgfqpoint{0.509353in}{1.984333in}}%
\pgfpathlineto{\pgfqpoint{0.508189in}{1.987538in}}%
\pgfpathlineto{\pgfqpoint{0.505225in}{1.996084in}}%
\pgfpathlineto{\pgfqpoint{0.501123in}{2.007836in}}%
\pgfpathlineto{\pgfqpoint{0.496979in}{2.019587in}}%
\pgfpathlineto{\pgfqpoint{0.496126in}{2.022003in}}%
\pgfpathlineto{\pgfqpoint{0.492971in}{2.031338in}}%
\pgfpathlineto{\pgfqpoint{0.488970in}{2.043090in}}%
\pgfpathlineto{\pgfqpoint{0.484930in}{2.054841in}}%
\pgfpathlineto{\pgfqpoint{0.484063in}{2.057359in}}%
\pgfpathlineto{\pgfqpoint{0.481012in}{2.066593in}}%
\pgfpathlineto{\pgfqpoint{0.477101in}{2.078344in}}%
\pgfpathlineto{\pgfqpoint{0.473153in}{2.090095in}}%
\pgfpathlineto{\pgfqpoint{0.472001in}{2.093517in}}%
\pgfpathlineto{\pgfqpoint{0.469301in}{2.101847in}}%
\pgfpathlineto{\pgfqpoint{0.465469in}{2.113598in}}%
\pgfpathlineto{\pgfqpoint{0.461600in}{2.125349in}}%
\pgfpathlineto{\pgfqpoint{0.459938in}{2.130381in}}%
\pgfpathlineto{\pgfqpoint{0.457795in}{2.137101in}}%
\pgfpathlineto{\pgfqpoint{0.454030in}{2.148852in}}%
\pgfpathlineto{\pgfqpoint{0.450230in}{2.160604in}}%
\pgfpathlineto{\pgfqpoint{0.447875in}{2.167843in}}%
\pgfpathlineto{\pgfqpoint{0.446455in}{2.172355in}}%
\pgfpathlineto{\pgfqpoint{0.442747in}{2.184106in}}%
\pgfpathlineto{\pgfqpoint{0.439004in}{2.195858in}}%
\pgfpathlineto{\pgfqpoint{0.435813in}{2.205794in}}%
\pgfpathlineto{\pgfqpoint{0.435247in}{2.207609in}}%
\pgfpathlineto{\pgfqpoint{0.431585in}{2.219360in}}%
\pgfpathlineto{\pgfqpoint{0.427888in}{2.231112in}}%
\pgfpathlineto{\pgfqpoint{0.424154in}{2.242863in}}%
\pgfpathlineto{\pgfqpoint{0.423750in}{2.244136in}}%
\pgfpathlineto{\pgfqpoint{0.423750in}{2.242863in}}%
\pgfpathlineto{\pgfqpoint{0.423750in}{2.231112in}}%
\pgfpathlineto{\pgfqpoint{0.423750in}{2.219360in}}%
\pgfpathlineto{\pgfqpoint{0.423750in}{2.207609in}}%
\pgfpathlineto{\pgfqpoint{0.423750in}{2.195858in}}%
\pgfpathlineto{\pgfqpoint{0.423750in}{2.195769in}}%
\pgfpathlineto{\pgfqpoint{0.427481in}{2.184106in}}%
\pgfpathlineto{\pgfqpoint{0.431204in}{2.172355in}}%
\pgfpathlineto{\pgfqpoint{0.434894in}{2.160604in}}%
\pgfpathlineto{\pgfqpoint{0.435813in}{2.157673in}}%
\pgfpathlineto{\pgfqpoint{0.438661in}{2.148852in}}%
\pgfpathlineto{\pgfqpoint{0.442431in}{2.137101in}}%
\pgfpathlineto{\pgfqpoint{0.446167in}{2.125349in}}%
\pgfpathlineto{\pgfqpoint{0.447875in}{2.119956in}}%
\pgfpathlineto{\pgfqpoint{0.449955in}{2.113598in}}%
\pgfpathlineto{\pgfqpoint{0.453781in}{2.101847in}}%
\pgfpathlineto{\pgfqpoint{0.457572in}{2.090095in}}%
\pgfpathlineto{\pgfqpoint{0.459938in}{2.082725in}}%
\pgfpathlineto{\pgfqpoint{0.461393in}{2.078344in}}%
\pgfpathlineto{\pgfqpoint{0.465287in}{2.066593in}}%
\pgfpathlineto{\pgfqpoint{0.469145in}{2.054841in}}%
\pgfpathlineto{\pgfqpoint{0.472001in}{2.046086in}}%
\pgfpathlineto{\pgfqpoint{0.473015in}{2.043090in}}%
\pgfpathlineto{\pgfqpoint{0.476987in}{2.031338in}}%
\pgfpathlineto{\pgfqpoint{0.480922in}{2.019587in}}%
\pgfpathlineto{\pgfqpoint{0.484063in}{2.010138in}}%
\pgfpathlineto{\pgfqpoint{0.484860in}{2.007836in}}%
\pgfpathlineto{\pgfqpoint{0.488921in}{1.996084in}}%
\pgfpathlineto{\pgfqpoint{0.492945in}{1.984333in}}%
\pgfpathlineto{\pgfqpoint{0.496126in}{1.974976in}}%
\pgfpathlineto{\pgfqpoint{0.496975in}{1.972582in}}%
\pgfpathlineto{\pgfqpoint{0.501138in}{1.960830in}}%
\pgfpathlineto{\pgfqpoint{0.505261in}{1.949079in}}%
\pgfpathlineto{\pgfqpoint{0.508189in}{1.940681in}}%
\pgfpathlineto{\pgfqpoint{0.509411in}{1.937327in}}%
\pgfpathlineto{\pgfqpoint{0.513687in}{1.925576in}}%
\pgfpathlineto{\pgfqpoint{0.517921in}{1.913825in}}%
\pgfpathlineto{\pgfqpoint{0.520251in}{1.907325in}}%
\pgfpathlineto{\pgfqpoint{0.522226in}{1.902073in}}%
\pgfpathlineto{\pgfqpoint{0.526625in}{1.890322in}}%
\pgfpathlineto{\pgfqpoint{0.530981in}{1.878571in}}%
\pgfpathlineto{\pgfqpoint{0.532314in}{1.874965in}}%
\pgfpathlineto{\pgfqpoint{0.535480in}{1.866819in}}%
\pgfpathlineto{\pgfqpoint{0.540014in}{1.855068in}}%
\pgfpathlineto{\pgfqpoint{0.544377in}{1.843646in}}%
\pgfpathlineto{\pgfqpoint{0.544510in}{1.843316in}}%
\pgfpathlineto{\pgfqpoint{0.549240in}{1.831565in}}%
\pgfpathlineto{\pgfqpoint{0.553918in}{1.819814in}}%
\pgfpathclose%
\pgfusepath{fill}%
\end{pgfscope}%
\begin{pgfscope}%
\pgfpathrectangle{\pgfqpoint{0.423750in}{1.819814in}}{\pgfqpoint{1.194205in}{1.163386in}}%
\pgfusepath{clip}%
\pgfsetbuttcap%
\pgfsetroundjoin%
\definecolor{currentfill}{rgb}{0.796501,0.105066,0.310630}%
\pgfsetfillcolor{currentfill}%
\pgfsetlinewidth{0.000000pt}%
\definecolor{currentstroke}{rgb}{0.000000,0.000000,0.000000}%
\pgfsetstrokecolor{currentstroke}%
\pgfsetdash{}{0pt}%
\pgfpathmoveto{\pgfqpoint{1.292262in}{2.793512in}}%
\pgfpathlineto{\pgfqpoint{1.304325in}{2.791525in}}%
\pgfpathlineto{\pgfqpoint{1.316388in}{2.790730in}}%
\pgfpathlineto{\pgfqpoint{1.328450in}{2.791142in}}%
\pgfpathlineto{\pgfqpoint{1.340513in}{2.792763in}}%
\pgfpathlineto{\pgfqpoint{1.350861in}{2.795178in}}%
\pgfpathlineto{\pgfqpoint{1.352576in}{2.795574in}}%
\pgfpathlineto{\pgfqpoint{1.364638in}{2.799506in}}%
\pgfpathlineto{\pgfqpoint{1.376701in}{2.804552in}}%
\pgfpathlineto{\pgfqpoint{1.381406in}{2.806929in}}%
\pgfpathlineto{\pgfqpoint{1.388764in}{2.810610in}}%
\pgfpathlineto{\pgfqpoint{1.400826in}{2.817622in}}%
\pgfpathlineto{\pgfqpoint{1.402443in}{2.818681in}}%
\pgfpathlineto{\pgfqpoint{1.412889in}{2.825457in}}%
\pgfpathlineto{\pgfqpoint{1.419848in}{2.830432in}}%
\pgfpathlineto{\pgfqpoint{1.424952in}{2.834051in}}%
\pgfpathlineto{\pgfqpoint{1.435562in}{2.842183in}}%
\pgfpathlineto{\pgfqpoint{1.437014in}{2.843288in}}%
\pgfpathlineto{\pgfqpoint{1.449077in}{2.853043in}}%
\pgfpathlineto{\pgfqpoint{1.450129in}{2.853935in}}%
\pgfpathlineto{\pgfqpoint{1.461140in}{2.863208in}}%
\pgfpathlineto{\pgfqpoint{1.463978in}{2.865686in}}%
\pgfpathlineto{\pgfqpoint{1.473202in}{2.873688in}}%
\pgfpathlineto{\pgfqpoint{1.477418in}{2.877438in}}%
\pgfpathlineto{\pgfqpoint{1.485265in}{2.884378in}}%
\pgfpathlineto{\pgfqpoint{1.490620in}{2.889189in}}%
\pgfpathlineto{\pgfqpoint{1.497328in}{2.895183in}}%
\pgfpathlineto{\pgfqpoint{1.503723in}{2.900940in}}%
\pgfpathlineto{\pgfqpoint{1.509390in}{2.906018in}}%
\pgfpathlineto{\pgfqpoint{1.516841in}{2.912692in}}%
\pgfpathlineto{\pgfqpoint{1.521453in}{2.916805in}}%
\pgfpathlineto{\pgfqpoint{1.530074in}{2.924443in}}%
\pgfpathlineto{\pgfqpoint{1.533516in}{2.927481in}}%
\pgfpathlineto{\pgfqpoint{1.543507in}{2.936194in}}%
\pgfpathlineto{\pgfqpoint{1.545579in}{2.937995in}}%
\pgfpathlineto{\pgfqpoint{1.557216in}{2.947946in}}%
\pgfpathlineto{\pgfqpoint{1.557641in}{2.948309in}}%
\pgfpathlineto{\pgfqpoint{1.569704in}{2.958383in}}%
\pgfpathlineto{\pgfqpoint{1.571313in}{2.959697in}}%
\pgfpathlineto{\pgfqpoint{1.581767in}{2.968218in}}%
\pgfpathlineto{\pgfqpoint{1.585821in}{2.971449in}}%
\pgfpathlineto{\pgfqpoint{1.593829in}{2.977823in}}%
\pgfpathlineto{\pgfqpoint{1.600730in}{2.983200in}}%
\pgfpathlineto{\pgfqpoint{1.593829in}{2.983200in}}%
\pgfpathlineto{\pgfqpoint{1.581767in}{2.983200in}}%
\pgfpathlineto{\pgfqpoint{1.569704in}{2.983200in}}%
\pgfpathlineto{\pgfqpoint{1.557641in}{2.983200in}}%
\pgfpathlineto{\pgfqpoint{1.545579in}{2.983200in}}%
\pgfpathlineto{\pgfqpoint{1.544072in}{2.983200in}}%
\pgfpathlineto{\pgfqpoint{1.533516in}{2.974070in}}%
\pgfpathlineto{\pgfqpoint{1.530513in}{2.971449in}}%
\pgfpathlineto{\pgfqpoint{1.521453in}{2.963510in}}%
\pgfpathlineto{\pgfqpoint{1.517121in}{2.959697in}}%
\pgfpathlineto{\pgfqpoint{1.509390in}{2.952866in}}%
\pgfpathlineto{\pgfqpoint{1.503812in}{2.947946in}}%
\pgfpathlineto{\pgfqpoint{1.497328in}{2.942200in}}%
\pgfpathlineto{\pgfqpoint{1.490488in}{2.936194in}}%
\pgfpathlineto{\pgfqpoint{1.485265in}{2.931586in}}%
\pgfpathlineto{\pgfqpoint{1.477029in}{2.924443in}}%
\pgfpathlineto{\pgfqpoint{1.473202in}{2.921106in}}%
\pgfpathlineto{\pgfqpoint{1.463294in}{2.912692in}}%
\pgfpathlineto{\pgfqpoint{1.461140in}{2.910851in}}%
\pgfpathlineto{\pgfqpoint{1.449100in}{2.900940in}}%
\pgfpathlineto{\pgfqpoint{1.449077in}{2.900921in}}%
\pgfpathlineto{\pgfqpoint{1.437014in}{2.891393in}}%
\pgfpathlineto{\pgfqpoint{1.434043in}{2.889189in}}%
\pgfpathlineto{\pgfqpoint{1.424952in}{2.882395in}}%
\pgfpathlineto{\pgfqpoint{1.417774in}{2.877438in}}%
\pgfpathlineto{\pgfqpoint{1.412889in}{2.874036in}}%
\pgfpathlineto{\pgfqpoint{1.400826in}{2.866417in}}%
\pgfpathlineto{\pgfqpoint{1.399523in}{2.865686in}}%
\pgfpathlineto{\pgfqpoint{1.388764in}{2.859600in}}%
\pgfpathlineto{\pgfqpoint{1.377119in}{2.853935in}}%
\pgfpathlineto{\pgfqpoint{1.376701in}{2.853730in}}%
\pgfpathlineto{\pgfqpoint{1.364638in}{2.848820in}}%
\pgfpathlineto{\pgfqpoint{1.352576in}{2.844991in}}%
\pgfpathlineto{\pgfqpoint{1.340513in}{2.842282in}}%
\pgfpathlineto{\pgfqpoint{1.339752in}{2.842183in}}%
\pgfpathlineto{\pgfqpoint{1.328450in}{2.840699in}}%
\pgfpathlineto{\pgfqpoint{1.316388in}{2.840287in}}%
\pgfpathlineto{\pgfqpoint{1.304325in}{2.841046in}}%
\pgfpathlineto{\pgfqpoint{1.297160in}{2.842183in}}%
\pgfpathlineto{\pgfqpoint{1.292262in}{2.842952in}}%
\pgfpathlineto{\pgfqpoint{1.280200in}{2.845958in}}%
\pgfpathlineto{\pgfqpoint{1.268137in}{2.850039in}}%
\pgfpathlineto{\pgfqpoint{1.258925in}{2.853935in}}%
\pgfpathlineto{\pgfqpoint{1.256074in}{2.855128in}}%
\pgfpathlineto{\pgfqpoint{1.244012in}{2.861118in}}%
\pgfpathlineto{\pgfqpoint{1.235980in}{2.865686in}}%
\pgfpathlineto{\pgfqpoint{1.231949in}{2.867957in}}%
\pgfpathlineto{\pgfqpoint{1.219886in}{2.875530in}}%
\pgfpathlineto{\pgfqpoint{1.217097in}{2.877438in}}%
\pgfpathlineto{\pgfqpoint{1.207824in}{2.883729in}}%
\pgfpathlineto{\pgfqpoint{1.200313in}{2.889189in}}%
\pgfpathlineto{\pgfqpoint{1.195761in}{2.892472in}}%
\pgfpathlineto{\pgfqpoint{1.184637in}{2.900940in}}%
\pgfpathlineto{\pgfqpoint{1.183698in}{2.901650in}}%
\pgfpathlineto{\pgfqpoint{1.171636in}{2.911141in}}%
\pgfpathlineto{\pgfqpoint{1.169720in}{2.912692in}}%
\pgfpathlineto{\pgfqpoint{1.159573in}{2.920857in}}%
\pgfpathlineto{\pgfqpoint{1.155202in}{2.924443in}}%
\pgfpathlineto{\pgfqpoint{1.147510in}{2.930716in}}%
\pgfpathlineto{\pgfqpoint{1.140862in}{2.936194in}}%
\pgfpathlineto{\pgfqpoint{1.135448in}{2.940633in}}%
\pgfpathlineto{\pgfqpoint{1.126550in}{2.947946in}}%
\pgfpathlineto{\pgfqpoint{1.123385in}{2.950535in}}%
\pgfpathlineto{\pgfqpoint{1.112138in}{2.959697in}}%
\pgfpathlineto{\pgfqpoint{1.111322in}{2.960359in}}%
\pgfpathlineto{\pgfqpoint{1.099260in}{2.970040in}}%
\pgfpathlineto{\pgfqpoint{1.097478in}{2.971449in}}%
\pgfpathlineto{\pgfqpoint{1.087197in}{2.979547in}}%
\pgfpathlineto{\pgfqpoint{1.082477in}{2.983200in}}%
\pgfpathlineto{\pgfqpoint{1.075134in}{2.983200in}}%
\pgfpathlineto{\pgfqpoint{1.063072in}{2.983200in}}%
\pgfpathlineto{\pgfqpoint{1.051009in}{2.983200in}}%
\pgfpathlineto{\pgfqpoint{1.038946in}{2.983200in}}%
\pgfpathlineto{\pgfqpoint{1.026884in}{2.983200in}}%
\pgfpathlineto{\pgfqpoint{1.018727in}{2.983200in}}%
\pgfpathlineto{\pgfqpoint{1.026884in}{2.977558in}}%
\pgfpathlineto{\pgfqpoint{1.035589in}{2.971449in}}%
\pgfpathlineto{\pgfqpoint{1.038946in}{2.969090in}}%
\pgfpathlineto{\pgfqpoint{1.051009in}{2.960439in}}%
\pgfpathlineto{\pgfqpoint{1.052017in}{2.959697in}}%
\pgfpathlineto{\pgfqpoint{1.063072in}{2.951553in}}%
\pgfpathlineto{\pgfqpoint{1.067854in}{2.947946in}}%
\pgfpathlineto{\pgfqpoint{1.075134in}{2.942441in}}%
\pgfpathlineto{\pgfqpoint{1.083209in}{2.936194in}}%
\pgfpathlineto{\pgfqpoint{1.087197in}{2.933100in}}%
\pgfpathlineto{\pgfqpoint{1.098120in}{2.924443in}}%
\pgfpathlineto{\pgfqpoint{1.099260in}{2.923537in}}%
\pgfpathlineto{\pgfqpoint{1.111322in}{2.913763in}}%
\pgfpathlineto{\pgfqpoint{1.112628in}{2.912692in}}%
\pgfpathlineto{\pgfqpoint{1.123385in}{2.903822in}}%
\pgfpathlineto{\pgfqpoint{1.126857in}{2.900940in}}%
\pgfpathlineto{\pgfqpoint{1.135448in}{2.893775in}}%
\pgfpathlineto{\pgfqpoint{1.140949in}{2.889189in}}%
\pgfpathlineto{\pgfqpoint{1.147510in}{2.883688in}}%
\pgfpathlineto{\pgfqpoint{1.155028in}{2.877438in}}%
\pgfpathlineto{\pgfqpoint{1.159573in}{2.873636in}}%
\pgfpathlineto{\pgfqpoint{1.169245in}{2.865686in}}%
\pgfpathlineto{\pgfqpoint{1.171636in}{2.863708in}}%
\pgfpathlineto{\pgfqpoint{1.183698in}{2.854000in}}%
\pgfpathlineto{\pgfqpoint{1.183782in}{2.853935in}}%
\pgfpathlineto{\pgfqpoint{1.195761in}{2.844587in}}%
\pgfpathlineto{\pgfqpoint{1.199008in}{2.842183in}}%
\pgfpathlineto{\pgfqpoint{1.207824in}{2.835603in}}%
\pgfpathlineto{\pgfqpoint{1.215240in}{2.830432in}}%
\pgfpathlineto{\pgfqpoint{1.219886in}{2.827164in}}%
\pgfpathlineto{\pgfqpoint{1.231949in}{2.819375in}}%
\pgfpathlineto{\pgfqpoint{1.233145in}{2.818681in}}%
\pgfpathlineto{\pgfqpoint{1.244012in}{2.812311in}}%
\pgfpathlineto{\pgfqpoint{1.254523in}{2.806929in}}%
\pgfpathlineto{\pgfqpoint{1.256074in}{2.806127in}}%
\pgfpathlineto{\pgfqpoint{1.268137in}{2.800859in}}%
\pgfpathlineto{\pgfqpoint{1.280200in}{2.796643in}}%
\pgfpathlineto{\pgfqpoint{1.285881in}{2.795178in}}%
\pgfpathclose%
\pgfusepath{fill}%
\end{pgfscope}%
\begin{pgfscope}%
\pgfpathrectangle{\pgfqpoint{0.423750in}{1.819814in}}{\pgfqpoint{1.194205in}{1.163386in}}%
\pgfusepath{clip}%
\pgfsetbuttcap%
\pgfsetroundjoin%
\definecolor{currentfill}{rgb}{0.857426,0.162258,0.276275}%
\pgfsetfillcolor{currentfill}%
\pgfsetlinewidth{0.000000pt}%
\definecolor{currentstroke}{rgb}{0.000000,0.000000,0.000000}%
\pgfsetstrokecolor{currentstroke}%
\pgfsetdash{}{0pt}%
\pgfpathmoveto{\pgfqpoint{0.568502in}{1.830754in}}%
\pgfpathlineto{\pgfqpoint{0.573262in}{1.819814in}}%
\pgfpathlineto{\pgfqpoint{0.580565in}{1.819814in}}%
\pgfpathlineto{\pgfqpoint{0.592627in}{1.819814in}}%
\pgfpathlineto{\pgfqpoint{0.595025in}{1.819814in}}%
\pgfpathlineto{\pgfqpoint{0.592627in}{1.824854in}}%
\pgfpathlineto{\pgfqpoint{0.589631in}{1.831565in}}%
\pgfpathlineto{\pgfqpoint{0.584341in}{1.843316in}}%
\pgfpathlineto{\pgfqpoint{0.580565in}{1.851620in}}%
\pgfpathlineto{\pgfqpoint{0.579091in}{1.855068in}}%
\pgfpathlineto{\pgfqpoint{0.574048in}{1.866819in}}%
\pgfpathlineto{\pgfqpoint{0.568938in}{1.878571in}}%
\pgfpathlineto{\pgfqpoint{0.568502in}{1.879571in}}%
\pgfpathlineto{\pgfqpoint{0.564095in}{1.890322in}}%
\pgfpathlineto{\pgfqpoint{0.559223in}{1.902073in}}%
\pgfpathlineto{\pgfqpoint{0.556439in}{1.908740in}}%
\pgfpathlineto{\pgfqpoint{0.554436in}{1.913825in}}%
\pgfpathlineto{\pgfqpoint{0.549784in}{1.925576in}}%
\pgfpathlineto{\pgfqpoint{0.545076in}{1.937327in}}%
\pgfpathlineto{\pgfqpoint{0.544377in}{1.939070in}}%
\pgfpathlineto{\pgfqpoint{0.540580in}{1.949079in}}%
\pgfpathlineto{\pgfqpoint{0.536079in}{1.960830in}}%
\pgfpathlineto{\pgfqpoint{0.532314in}{1.970568in}}%
\pgfpathlineto{\pgfqpoint{0.531576in}{1.972582in}}%
\pgfpathlineto{\pgfqpoint{0.527264in}{1.984333in}}%
\pgfpathlineto{\pgfqpoint{0.522904in}{1.996084in}}%
\pgfpathlineto{\pgfqpoint{0.520251in}{2.003188in}}%
\pgfpathlineto{\pgfqpoint{0.518602in}{2.007836in}}%
\pgfpathlineto{\pgfqpoint{0.514416in}{2.019587in}}%
\pgfpathlineto{\pgfqpoint{0.510186in}{2.031338in}}%
\pgfpathlineto{\pgfqpoint{0.508189in}{2.036863in}}%
\pgfpathlineto{\pgfqpoint{0.506042in}{2.043090in}}%
\pgfpathlineto{\pgfqpoint{0.501972in}{2.054841in}}%
\pgfpathlineto{\pgfqpoint{0.497860in}{2.066593in}}%
\pgfpathlineto{\pgfqpoint{0.496126in}{2.071529in}}%
\pgfpathlineto{\pgfqpoint{0.493838in}{2.078344in}}%
\pgfpathlineto{\pgfqpoint{0.489871in}{2.090095in}}%
\pgfpathlineto{\pgfqpoint{0.485865in}{2.101847in}}%
\pgfpathlineto{\pgfqpoint{0.484063in}{2.107110in}}%
\pgfpathlineto{\pgfqpoint{0.481934in}{2.113598in}}%
\pgfpathlineto{\pgfqpoint{0.478059in}{2.125349in}}%
\pgfpathlineto{\pgfqpoint{0.474146in}{2.137101in}}%
\pgfpathlineto{\pgfqpoint{0.472001in}{2.143511in}}%
\pgfpathlineto{\pgfqpoint{0.470282in}{2.148852in}}%
\pgfpathlineto{\pgfqpoint{0.466487in}{2.160604in}}%
\pgfpathlineto{\pgfqpoint{0.462655in}{2.172355in}}%
\pgfpathlineto{\pgfqpoint{0.459938in}{2.180631in}}%
\pgfpathlineto{\pgfqpoint{0.458838in}{2.184106in}}%
\pgfpathlineto{\pgfqpoint{0.455112in}{2.195858in}}%
\pgfpathlineto{\pgfqpoint{0.451349in}{2.207609in}}%
\pgfpathlineto{\pgfqpoint{0.447875in}{2.218358in}}%
\pgfpathlineto{\pgfqpoint{0.447562in}{2.219360in}}%
\pgfpathlineto{\pgfqpoint{0.443893in}{2.231112in}}%
\pgfpathlineto{\pgfqpoint{0.440188in}{2.242863in}}%
\pgfpathlineto{\pgfqpoint{0.436445in}{2.254615in}}%
\pgfpathlineto{\pgfqpoint{0.435813in}{2.256601in}}%
\pgfpathlineto{\pgfqpoint{0.432797in}{2.266366in}}%
\pgfpathlineto{\pgfqpoint{0.429139in}{2.278117in}}%
\pgfpathlineto{\pgfqpoint{0.425443in}{2.289869in}}%
\pgfpathlineto{\pgfqpoint{0.423750in}{2.295232in}}%
\pgfpathlineto{\pgfqpoint{0.423750in}{2.289869in}}%
\pgfpathlineto{\pgfqpoint{0.423750in}{2.278117in}}%
\pgfpathlineto{\pgfqpoint{0.423750in}{2.266366in}}%
\pgfpathlineto{\pgfqpoint{0.423750in}{2.254615in}}%
\pgfpathlineto{\pgfqpoint{0.423750in}{2.244136in}}%
\pgfpathlineto{\pgfqpoint{0.424154in}{2.242863in}}%
\pgfpathlineto{\pgfqpoint{0.427888in}{2.231112in}}%
\pgfpathlineto{\pgfqpoint{0.431585in}{2.219360in}}%
\pgfpathlineto{\pgfqpoint{0.435247in}{2.207609in}}%
\pgfpathlineto{\pgfqpoint{0.435813in}{2.205794in}}%
\pgfpathlineto{\pgfqpoint{0.439004in}{2.195858in}}%
\pgfpathlineto{\pgfqpoint{0.442747in}{2.184106in}}%
\pgfpathlineto{\pgfqpoint{0.446455in}{2.172355in}}%
\pgfpathlineto{\pgfqpoint{0.447875in}{2.167843in}}%
\pgfpathlineto{\pgfqpoint{0.450230in}{2.160604in}}%
\pgfpathlineto{\pgfqpoint{0.454030in}{2.148852in}}%
\pgfpathlineto{\pgfqpoint{0.457795in}{2.137101in}}%
\pgfpathlineto{\pgfqpoint{0.459938in}{2.130381in}}%
\pgfpathlineto{\pgfqpoint{0.461600in}{2.125349in}}%
\pgfpathlineto{\pgfqpoint{0.465469in}{2.113598in}}%
\pgfpathlineto{\pgfqpoint{0.469301in}{2.101847in}}%
\pgfpathlineto{\pgfqpoint{0.472001in}{2.093517in}}%
\pgfpathlineto{\pgfqpoint{0.473153in}{2.090095in}}%
\pgfpathlineto{\pgfqpoint{0.477101in}{2.078344in}}%
\pgfpathlineto{\pgfqpoint{0.481012in}{2.066593in}}%
\pgfpathlineto{\pgfqpoint{0.484063in}{2.057359in}}%
\pgfpathlineto{\pgfqpoint{0.484930in}{2.054841in}}%
\pgfpathlineto{\pgfqpoint{0.488970in}{2.043090in}}%
\pgfpathlineto{\pgfqpoint{0.492971in}{2.031338in}}%
\pgfpathlineto{\pgfqpoint{0.496126in}{2.022003in}}%
\pgfpathlineto{\pgfqpoint{0.496979in}{2.019587in}}%
\pgfpathlineto{\pgfqpoint{0.501123in}{2.007836in}}%
\pgfpathlineto{\pgfqpoint{0.505225in}{1.996084in}}%
\pgfpathlineto{\pgfqpoint{0.508189in}{1.987538in}}%
\pgfpathlineto{\pgfqpoint{0.509353in}{1.984333in}}%
\pgfpathlineto{\pgfqpoint{0.513612in}{1.972582in}}%
\pgfpathlineto{\pgfqpoint{0.517827in}{1.960830in}}%
\pgfpathlineto{\pgfqpoint{0.520251in}{1.954037in}}%
\pgfpathlineto{\pgfqpoint{0.522109in}{1.949079in}}%
\pgfpathlineto{\pgfqpoint{0.526495in}{1.937327in}}%
\pgfpathlineto{\pgfqpoint{0.530835in}{1.925576in}}%
\pgfpathlineto{\pgfqpoint{0.532314in}{1.921560in}}%
\pgfpathlineto{\pgfqpoint{0.535313in}{1.913825in}}%
\pgfpathlineto{\pgfqpoint{0.539837in}{1.902073in}}%
\pgfpathlineto{\pgfqpoint{0.544312in}{1.890322in}}%
\pgfpathlineto{\pgfqpoint{0.544377in}{1.890151in}}%
\pgfpathlineto{\pgfqpoint{0.549033in}{1.878571in}}%
\pgfpathlineto{\pgfqpoint{0.553704in}{1.866819in}}%
\pgfpathlineto{\pgfqpoint{0.556439in}{1.859894in}}%
\pgfpathlineto{\pgfqpoint{0.558456in}{1.855068in}}%
\pgfpathlineto{\pgfqpoint{0.563341in}{1.843316in}}%
\pgfpathlineto{\pgfqpoint{0.568169in}{1.831565in}}%
\pgfpathclose%
\pgfusepath{fill}%
\end{pgfscope}%
\begin{pgfscope}%
\pgfpathrectangle{\pgfqpoint{0.423750in}{1.819814in}}{\pgfqpoint{1.194205in}{1.163386in}}%
\pgfusepath{clip}%
\pgfsetbuttcap%
\pgfsetroundjoin%
\definecolor{currentfill}{rgb}{0.857426,0.162258,0.276275}%
\pgfsetfillcolor{currentfill}%
\pgfsetlinewidth{0.000000pt}%
\definecolor{currentstroke}{rgb}{0.000000,0.000000,0.000000}%
\pgfsetstrokecolor{currentstroke}%
\pgfsetdash{}{0pt}%
\pgfpathmoveto{\pgfqpoint{1.280200in}{2.744413in}}%
\pgfpathlineto{\pgfqpoint{1.292262in}{2.741148in}}%
\pgfpathlineto{\pgfqpoint{1.304325in}{2.739082in}}%
\pgfpathlineto{\pgfqpoint{1.316388in}{2.738247in}}%
\pgfpathlineto{\pgfqpoint{1.328450in}{2.738658in}}%
\pgfpathlineto{\pgfqpoint{1.340513in}{2.740316in}}%
\pgfpathlineto{\pgfqpoint{1.352576in}{2.743209in}}%
\pgfpathlineto{\pgfqpoint{1.364638in}{2.747307in}}%
\pgfpathlineto{\pgfqpoint{1.366626in}{2.748172in}}%
\pgfpathlineto{\pgfqpoint{1.376701in}{2.752509in}}%
\pgfpathlineto{\pgfqpoint{1.388764in}{2.758791in}}%
\pgfpathlineto{\pgfqpoint{1.390639in}{2.759924in}}%
\pgfpathlineto{\pgfqpoint{1.400826in}{2.766015in}}%
\pgfpathlineto{\pgfqpoint{1.409220in}{2.771675in}}%
\pgfpathlineto{\pgfqpoint{1.412889in}{2.774126in}}%
\pgfpathlineto{\pgfqpoint{1.424952in}{2.782998in}}%
\pgfpathlineto{\pgfqpoint{1.425492in}{2.783427in}}%
\pgfpathlineto{\pgfqpoint{1.437014in}{2.792492in}}%
\pgfpathlineto{\pgfqpoint{1.440227in}{2.795178in}}%
\pgfpathlineto{\pgfqpoint{1.449077in}{2.802521in}}%
\pgfpathlineto{\pgfqpoint{1.454147in}{2.806929in}}%
\pgfpathlineto{\pgfqpoint{1.461140in}{2.812965in}}%
\pgfpathlineto{\pgfqpoint{1.467536in}{2.818681in}}%
\pgfpathlineto{\pgfqpoint{1.473202in}{2.823710in}}%
\pgfpathlineto{\pgfqpoint{1.480598in}{2.830432in}}%
\pgfpathlineto{\pgfqpoint{1.485265in}{2.834649in}}%
\pgfpathlineto{\pgfqpoint{1.493490in}{2.842183in}}%
\pgfpathlineto{\pgfqpoint{1.497328in}{2.845680in}}%
\pgfpathlineto{\pgfqpoint{1.506342in}{2.853935in}}%
\pgfpathlineto{\pgfqpoint{1.509390in}{2.856713in}}%
\pgfpathlineto{\pgfqpoint{1.519261in}{2.865686in}}%
\pgfpathlineto{\pgfqpoint{1.521453in}{2.867670in}}%
\pgfpathlineto{\pgfqpoint{1.532346in}{2.877438in}}%
\pgfpathlineto{\pgfqpoint{1.533516in}{2.878483in}}%
\pgfpathlineto{\pgfqpoint{1.545579in}{2.889098in}}%
\pgfpathlineto{\pgfqpoint{1.545684in}{2.889189in}}%
\pgfpathlineto{\pgfqpoint{1.557641in}{2.899461in}}%
\pgfpathlineto{\pgfqpoint{1.559405in}{2.900940in}}%
\pgfpathlineto{\pgfqpoint{1.569704in}{2.909560in}}%
\pgfpathlineto{\pgfqpoint{1.573543in}{2.912692in}}%
\pgfpathlineto{\pgfqpoint{1.581767in}{2.919387in}}%
\pgfpathlineto{\pgfqpoint{1.588141in}{2.924443in}}%
\pgfpathlineto{\pgfqpoint{1.593829in}{2.928948in}}%
\pgfpathlineto{\pgfqpoint{1.603211in}{2.936194in}}%
\pgfpathlineto{\pgfqpoint{1.605892in}{2.938264in}}%
\pgfpathlineto{\pgfqpoint{1.617955in}{2.947363in}}%
\pgfpathlineto{\pgfqpoint{1.617955in}{2.947946in}}%
\pgfpathlineto{\pgfqpoint{1.617955in}{2.959697in}}%
\pgfpathlineto{\pgfqpoint{1.617955in}{2.971449in}}%
\pgfpathlineto{\pgfqpoint{1.617955in}{2.983200in}}%
\pgfpathlineto{\pgfqpoint{1.605892in}{2.983200in}}%
\pgfpathlineto{\pgfqpoint{1.600730in}{2.983200in}}%
\pgfpathlineto{\pgfqpoint{1.593829in}{2.977823in}}%
\pgfpathlineto{\pgfqpoint{1.585821in}{2.971449in}}%
\pgfpathlineto{\pgfqpoint{1.581767in}{2.968218in}}%
\pgfpathlineto{\pgfqpoint{1.571313in}{2.959697in}}%
\pgfpathlineto{\pgfqpoint{1.569704in}{2.958383in}}%
\pgfpathlineto{\pgfqpoint{1.557641in}{2.948309in}}%
\pgfpathlineto{\pgfqpoint{1.557216in}{2.947946in}}%
\pgfpathlineto{\pgfqpoint{1.545579in}{2.937995in}}%
\pgfpathlineto{\pgfqpoint{1.543507in}{2.936194in}}%
\pgfpathlineto{\pgfqpoint{1.533516in}{2.927481in}}%
\pgfpathlineto{\pgfqpoint{1.530074in}{2.924443in}}%
\pgfpathlineto{\pgfqpoint{1.521453in}{2.916805in}}%
\pgfpathlineto{\pgfqpoint{1.516841in}{2.912692in}}%
\pgfpathlineto{\pgfqpoint{1.509390in}{2.906018in}}%
\pgfpathlineto{\pgfqpoint{1.503723in}{2.900940in}}%
\pgfpathlineto{\pgfqpoint{1.497328in}{2.895183in}}%
\pgfpathlineto{\pgfqpoint{1.490620in}{2.889189in}}%
\pgfpathlineto{\pgfqpoint{1.485265in}{2.884378in}}%
\pgfpathlineto{\pgfqpoint{1.477418in}{2.877438in}}%
\pgfpathlineto{\pgfqpoint{1.473202in}{2.873688in}}%
\pgfpathlineto{\pgfqpoint{1.463978in}{2.865686in}}%
\pgfpathlineto{\pgfqpoint{1.461140in}{2.863208in}}%
\pgfpathlineto{\pgfqpoint{1.450129in}{2.853935in}}%
\pgfpathlineto{\pgfqpoint{1.449077in}{2.853043in}}%
\pgfpathlineto{\pgfqpoint{1.437014in}{2.843288in}}%
\pgfpathlineto{\pgfqpoint{1.435562in}{2.842183in}}%
\pgfpathlineto{\pgfqpoint{1.424952in}{2.834051in}}%
\pgfpathlineto{\pgfqpoint{1.419848in}{2.830432in}}%
\pgfpathlineto{\pgfqpoint{1.412889in}{2.825457in}}%
\pgfpathlineto{\pgfqpoint{1.402443in}{2.818681in}}%
\pgfpathlineto{\pgfqpoint{1.400826in}{2.817622in}}%
\pgfpathlineto{\pgfqpoint{1.388764in}{2.810610in}}%
\pgfpathlineto{\pgfqpoint{1.381406in}{2.806929in}}%
\pgfpathlineto{\pgfqpoint{1.376701in}{2.804552in}}%
\pgfpathlineto{\pgfqpoint{1.364638in}{2.799506in}}%
\pgfpathlineto{\pgfqpoint{1.352576in}{2.795574in}}%
\pgfpathlineto{\pgfqpoint{1.350861in}{2.795178in}}%
\pgfpathlineto{\pgfqpoint{1.340513in}{2.792763in}}%
\pgfpathlineto{\pgfqpoint{1.328450in}{2.791142in}}%
\pgfpathlineto{\pgfqpoint{1.316388in}{2.790730in}}%
\pgfpathlineto{\pgfqpoint{1.304325in}{2.791525in}}%
\pgfpathlineto{\pgfqpoint{1.292262in}{2.793512in}}%
\pgfpathlineto{\pgfqpoint{1.285881in}{2.795178in}}%
\pgfpathlineto{\pgfqpoint{1.280200in}{2.796643in}}%
\pgfpathlineto{\pgfqpoint{1.268137in}{2.800859in}}%
\pgfpathlineto{\pgfqpoint{1.256074in}{2.806127in}}%
\pgfpathlineto{\pgfqpoint{1.254523in}{2.806929in}}%
\pgfpathlineto{\pgfqpoint{1.244012in}{2.812311in}}%
\pgfpathlineto{\pgfqpoint{1.233145in}{2.818681in}}%
\pgfpathlineto{\pgfqpoint{1.231949in}{2.819375in}}%
\pgfpathlineto{\pgfqpoint{1.219886in}{2.827164in}}%
\pgfpathlineto{\pgfqpoint{1.215240in}{2.830432in}}%
\pgfpathlineto{\pgfqpoint{1.207824in}{2.835603in}}%
\pgfpathlineto{\pgfqpoint{1.199008in}{2.842183in}}%
\pgfpathlineto{\pgfqpoint{1.195761in}{2.844587in}}%
\pgfpathlineto{\pgfqpoint{1.183782in}{2.853935in}}%
\pgfpathlineto{\pgfqpoint{1.183698in}{2.854000in}}%
\pgfpathlineto{\pgfqpoint{1.171636in}{2.863708in}}%
\pgfpathlineto{\pgfqpoint{1.169245in}{2.865686in}}%
\pgfpathlineto{\pgfqpoint{1.159573in}{2.873636in}}%
\pgfpathlineto{\pgfqpoint{1.155028in}{2.877438in}}%
\pgfpathlineto{\pgfqpoint{1.147510in}{2.883688in}}%
\pgfpathlineto{\pgfqpoint{1.140949in}{2.889189in}}%
\pgfpathlineto{\pgfqpoint{1.135448in}{2.893775in}}%
\pgfpathlineto{\pgfqpoint{1.126857in}{2.900940in}}%
\pgfpathlineto{\pgfqpoint{1.123385in}{2.903822in}}%
\pgfpathlineto{\pgfqpoint{1.112628in}{2.912692in}}%
\pgfpathlineto{\pgfqpoint{1.111322in}{2.913763in}}%
\pgfpathlineto{\pgfqpoint{1.099260in}{2.923537in}}%
\pgfpathlineto{\pgfqpoint{1.098120in}{2.924443in}}%
\pgfpathlineto{\pgfqpoint{1.087197in}{2.933100in}}%
\pgfpathlineto{\pgfqpoint{1.083209in}{2.936194in}}%
\pgfpathlineto{\pgfqpoint{1.075134in}{2.942441in}}%
\pgfpathlineto{\pgfqpoint{1.067854in}{2.947946in}}%
\pgfpathlineto{\pgfqpoint{1.063072in}{2.951553in}}%
\pgfpathlineto{\pgfqpoint{1.052017in}{2.959697in}}%
\pgfpathlineto{\pgfqpoint{1.051009in}{2.960439in}}%
\pgfpathlineto{\pgfqpoint{1.038946in}{2.969090in}}%
\pgfpathlineto{\pgfqpoint{1.035589in}{2.971449in}}%
\pgfpathlineto{\pgfqpoint{1.026884in}{2.977558in}}%
\pgfpathlineto{\pgfqpoint{1.018727in}{2.983200in}}%
\pgfpathlineto{\pgfqpoint{1.014821in}{2.983200in}}%
\pgfpathlineto{\pgfqpoint{1.002758in}{2.983200in}}%
\pgfpathlineto{\pgfqpoint{0.990696in}{2.983200in}}%
\pgfpathlineto{\pgfqpoint{0.978633in}{2.983200in}}%
\pgfpathlineto{\pgfqpoint{0.966570in}{2.983200in}}%
\pgfpathlineto{\pgfqpoint{0.954508in}{2.983200in}}%
\pgfpathlineto{\pgfqpoint{0.946815in}{2.983200in}}%
\pgfpathlineto{\pgfqpoint{0.954508in}{2.977568in}}%
\pgfpathlineto{\pgfqpoint{0.963217in}{2.971449in}}%
\pgfpathlineto{\pgfqpoint{0.966570in}{2.969097in}}%
\pgfpathlineto{\pgfqpoint{0.978633in}{2.960874in}}%
\pgfpathlineto{\pgfqpoint{0.980385in}{2.959697in}}%
\pgfpathlineto{\pgfqpoint{0.990696in}{2.952777in}}%
\pgfpathlineto{\pgfqpoint{0.997950in}{2.947946in}}%
\pgfpathlineto{\pgfqpoint{1.002758in}{2.944744in}}%
\pgfpathlineto{\pgfqpoint{1.014821in}{2.936684in}}%
\pgfpathlineto{\pgfqpoint{1.015541in}{2.936194in}}%
\pgfpathlineto{\pgfqpoint{1.026884in}{2.928486in}}%
\pgfpathlineto{\pgfqpoint{1.032723in}{2.924443in}}%
\pgfpathlineto{\pgfqpoint{1.038946in}{2.920129in}}%
\pgfpathlineto{\pgfqpoint{1.049425in}{2.912692in}}%
\pgfpathlineto{\pgfqpoint{1.051009in}{2.911565in}}%
\pgfpathlineto{\pgfqpoint{1.063072in}{2.902734in}}%
\pgfpathlineto{\pgfqpoint{1.065452in}{2.900940in}}%
\pgfpathlineto{\pgfqpoint{1.075134in}{2.893626in}}%
\pgfpathlineto{\pgfqpoint{1.080853in}{2.889189in}}%
\pgfpathlineto{\pgfqpoint{1.087197in}{2.884251in}}%
\pgfpathlineto{\pgfqpoint{1.095737in}{2.877438in}}%
\pgfpathlineto{\pgfqpoint{1.099260in}{2.874617in}}%
\pgfpathlineto{\pgfqpoint{1.110183in}{2.865686in}}%
\pgfpathlineto{\pgfqpoint{1.111322in}{2.864750in}}%
\pgfpathlineto{\pgfqpoint{1.123385in}{2.854680in}}%
\pgfpathlineto{\pgfqpoint{1.124269in}{2.853935in}}%
\pgfpathlineto{\pgfqpoint{1.135448in}{2.844460in}}%
\pgfpathlineto{\pgfqpoint{1.138129in}{2.842183in}}%
\pgfpathlineto{\pgfqpoint{1.147510in}{2.834170in}}%
\pgfpathlineto{\pgfqpoint{1.151914in}{2.830432in}}%
\pgfpathlineto{\pgfqpoint{1.159573in}{2.823890in}}%
\pgfpathlineto{\pgfqpoint{1.165767in}{2.818681in}}%
\pgfpathlineto{\pgfqpoint{1.171636in}{2.813710in}}%
\pgfpathlineto{\pgfqpoint{1.179853in}{2.806929in}}%
\pgfpathlineto{\pgfqpoint{1.183698in}{2.803732in}}%
\pgfpathlineto{\pgfqpoint{1.194383in}{2.795178in}}%
\pgfpathlineto{\pgfqpoint{1.195761in}{2.794066in}}%
\pgfpathlineto{\pgfqpoint{1.207824in}{2.784808in}}%
\pgfpathlineto{\pgfqpoint{1.209748in}{2.783427in}}%
\pgfpathlineto{\pgfqpoint{1.219886in}{2.776077in}}%
\pgfpathlineto{\pgfqpoint{1.226497in}{2.771675in}}%
\pgfpathlineto{\pgfqpoint{1.231949in}{2.768009in}}%
\pgfpathlineto{\pgfqpoint{1.244012in}{2.760716in}}%
\pgfpathlineto{\pgfqpoint{1.245509in}{2.759924in}}%
\pgfpathlineto{\pgfqpoint{1.256074in}{2.754269in}}%
\pgfpathlineto{\pgfqpoint{1.268137in}{2.748825in}}%
\pgfpathlineto{\pgfqpoint{1.269941in}{2.748172in}}%
\pgfpathclose%
\pgfusepath{fill}%
\end{pgfscope}%
\begin{pgfscope}%
\pgfpathrectangle{\pgfqpoint{0.423750in}{1.819814in}}{\pgfqpoint{1.194205in}{1.163386in}}%
\pgfusepath{clip}%
\pgfsetbuttcap%
\pgfsetroundjoin%
\definecolor{currentfill}{rgb}{0.905301,0.238545,0.247481}%
\pgfsetfillcolor{currentfill}%
\pgfsetlinewidth{0.000000pt}%
\definecolor{currentstroke}{rgb}{0.000000,0.000000,0.000000}%
\pgfsetstrokecolor{currentstroke}%
\pgfsetdash{}{0pt}%
\pgfpathmoveto{\pgfqpoint{0.592627in}{1.824854in}}%
\pgfpathlineto{\pgfqpoint{0.595025in}{1.819814in}}%
\pgfpathlineto{\pgfqpoint{0.604690in}{1.819814in}}%
\pgfpathlineto{\pgfqpoint{0.616753in}{1.819814in}}%
\pgfpathlineto{\pgfqpoint{0.620384in}{1.819814in}}%
\pgfpathlineto{\pgfqpoint{0.616753in}{1.826727in}}%
\pgfpathlineto{\pgfqpoint{0.614376in}{1.831565in}}%
\pgfpathlineto{\pgfqpoint{0.608559in}{1.843316in}}%
\pgfpathlineto{\pgfqpoint{0.604690in}{1.851042in}}%
\pgfpathlineto{\pgfqpoint{0.602804in}{1.855068in}}%
\pgfpathlineto{\pgfqpoint{0.597266in}{1.866819in}}%
\pgfpathlineto{\pgfqpoint{0.592627in}{1.876532in}}%
\pgfpathlineto{\pgfqpoint{0.591716in}{1.878571in}}%
\pgfpathlineto{\pgfqpoint{0.586447in}{1.890322in}}%
\pgfpathlineto{\pgfqpoint{0.581098in}{1.902073in}}%
\pgfpathlineto{\pgfqpoint{0.580565in}{1.903242in}}%
\pgfpathlineto{\pgfqpoint{0.576042in}{1.913825in}}%
\pgfpathlineto{\pgfqpoint{0.570956in}{1.925576in}}%
\pgfpathlineto{\pgfqpoint{0.568502in}{1.931206in}}%
\pgfpathlineto{\pgfqpoint{0.565997in}{1.937327in}}%
\pgfpathlineto{\pgfqpoint{0.561155in}{1.949079in}}%
\pgfpathlineto{\pgfqpoint{0.556439in}{1.960379in}}%
\pgfpathlineto{\pgfqpoint{0.556262in}{1.960830in}}%
\pgfpathlineto{\pgfqpoint{0.551646in}{1.972582in}}%
\pgfpathlineto{\pgfqpoint{0.546972in}{1.984333in}}%
\pgfpathlineto{\pgfqpoint{0.544377in}{1.990814in}}%
\pgfpathlineto{\pgfqpoint{0.542385in}{1.996084in}}%
\pgfpathlineto{\pgfqpoint{0.537924in}{2.007836in}}%
\pgfpathlineto{\pgfqpoint{0.533408in}{2.019587in}}%
\pgfpathlineto{\pgfqpoint{0.532314in}{2.022427in}}%
\pgfpathlineto{\pgfqpoint{0.529067in}{2.031338in}}%
\pgfpathlineto{\pgfqpoint{0.524749in}{2.043090in}}%
\pgfpathlineto{\pgfqpoint{0.520381in}{2.054841in}}%
\pgfpathlineto{\pgfqpoint{0.520251in}{2.055189in}}%
\pgfpathlineto{\pgfqpoint{0.516233in}{2.066593in}}%
\pgfpathlineto{\pgfqpoint{0.512048in}{2.078344in}}%
\pgfpathlineto{\pgfqpoint{0.508189in}{2.089068in}}%
\pgfpathlineto{\pgfqpoint{0.507837in}{2.090095in}}%
\pgfpathlineto{\pgfqpoint{0.503815in}{2.101847in}}%
\pgfpathlineto{\pgfqpoint{0.499750in}{2.113598in}}%
\pgfpathlineto{\pgfqpoint{0.496126in}{2.123974in}}%
\pgfpathlineto{\pgfqpoint{0.495667in}{2.125349in}}%
\pgfpathlineto{\pgfqpoint{0.491751in}{2.137101in}}%
\pgfpathlineto{\pgfqpoint{0.487793in}{2.148852in}}%
\pgfpathlineto{\pgfqpoint{0.484063in}{2.159817in}}%
\pgfpathlineto{\pgfqpoint{0.483807in}{2.160604in}}%
\pgfpathlineto{\pgfqpoint{0.479984in}{2.172355in}}%
\pgfpathlineto{\pgfqpoint{0.476121in}{2.184106in}}%
\pgfpathlineto{\pgfqpoint{0.472218in}{2.195858in}}%
\pgfpathlineto{\pgfqpoint{0.472001in}{2.196511in}}%
\pgfpathlineto{\pgfqpoint{0.468465in}{2.207609in}}%
\pgfpathlineto{\pgfqpoint{0.464684in}{2.219360in}}%
\pgfpathlineto{\pgfqpoint{0.460865in}{2.231112in}}%
\pgfpathlineto{\pgfqpoint{0.459938in}{2.233959in}}%
\pgfpathlineto{\pgfqpoint{0.457147in}{2.242863in}}%
\pgfpathlineto{\pgfqpoint{0.453438in}{2.254615in}}%
\pgfpathlineto{\pgfqpoint{0.449689in}{2.266366in}}%
\pgfpathlineto{\pgfqpoint{0.447875in}{2.272029in}}%
\pgfpathlineto{\pgfqpoint{0.445992in}{2.278117in}}%
\pgfpathlineto{\pgfqpoint{0.442341in}{2.289869in}}%
\pgfpathlineto{\pgfqpoint{0.438652in}{2.301620in}}%
\pgfpathlineto{\pgfqpoint{0.435813in}{2.310596in}}%
\pgfpathlineto{\pgfqpoint{0.434962in}{2.313371in}}%
\pgfpathlineto{\pgfqpoint{0.431359in}{2.325123in}}%
\pgfpathlineto{\pgfqpoint{0.427718in}{2.336874in}}%
\pgfpathlineto{\pgfqpoint{0.424037in}{2.348626in}}%
\pgfpathlineto{\pgfqpoint{0.423750in}{2.349543in}}%
\pgfpathlineto{\pgfqpoint{0.423750in}{2.348626in}}%
\pgfpathlineto{\pgfqpoint{0.423750in}{2.336874in}}%
\pgfpathlineto{\pgfqpoint{0.423750in}{2.325123in}}%
\pgfpathlineto{\pgfqpoint{0.423750in}{2.313371in}}%
\pgfpathlineto{\pgfqpoint{0.423750in}{2.301620in}}%
\pgfpathlineto{\pgfqpoint{0.423750in}{2.295232in}}%
\pgfpathlineto{\pgfqpoint{0.425443in}{2.289869in}}%
\pgfpathlineto{\pgfqpoint{0.429139in}{2.278117in}}%
\pgfpathlineto{\pgfqpoint{0.432797in}{2.266366in}}%
\pgfpathlineto{\pgfqpoint{0.435813in}{2.256601in}}%
\pgfpathlineto{\pgfqpoint{0.436445in}{2.254615in}}%
\pgfpathlineto{\pgfqpoint{0.440188in}{2.242863in}}%
\pgfpathlineto{\pgfqpoint{0.443893in}{2.231112in}}%
\pgfpathlineto{\pgfqpoint{0.447562in}{2.219360in}}%
\pgfpathlineto{\pgfqpoint{0.447875in}{2.218358in}}%
\pgfpathlineto{\pgfqpoint{0.451349in}{2.207609in}}%
\pgfpathlineto{\pgfqpoint{0.455112in}{2.195858in}}%
\pgfpathlineto{\pgfqpoint{0.458838in}{2.184106in}}%
\pgfpathlineto{\pgfqpoint{0.459938in}{2.180631in}}%
\pgfpathlineto{\pgfqpoint{0.462655in}{2.172355in}}%
\pgfpathlineto{\pgfqpoint{0.466487in}{2.160604in}}%
\pgfpathlineto{\pgfqpoint{0.470282in}{2.148852in}}%
\pgfpathlineto{\pgfqpoint{0.472001in}{2.143511in}}%
\pgfpathlineto{\pgfqpoint{0.474146in}{2.137101in}}%
\pgfpathlineto{\pgfqpoint{0.478059in}{2.125349in}}%
\pgfpathlineto{\pgfqpoint{0.481934in}{2.113598in}}%
\pgfpathlineto{\pgfqpoint{0.484063in}{2.107110in}}%
\pgfpathlineto{\pgfqpoint{0.485865in}{2.101847in}}%
\pgfpathlineto{\pgfqpoint{0.489871in}{2.090095in}}%
\pgfpathlineto{\pgfqpoint{0.493838in}{2.078344in}}%
\pgfpathlineto{\pgfqpoint{0.496126in}{2.071529in}}%
\pgfpathlineto{\pgfqpoint{0.497860in}{2.066593in}}%
\pgfpathlineto{\pgfqpoint{0.501972in}{2.054841in}}%
\pgfpathlineto{\pgfqpoint{0.506042in}{2.043090in}}%
\pgfpathlineto{\pgfqpoint{0.508189in}{2.036863in}}%
\pgfpathlineto{\pgfqpoint{0.510186in}{2.031338in}}%
\pgfpathlineto{\pgfqpoint{0.514416in}{2.019587in}}%
\pgfpathlineto{\pgfqpoint{0.518602in}{2.007836in}}%
\pgfpathlineto{\pgfqpoint{0.520251in}{2.003188in}}%
\pgfpathlineto{\pgfqpoint{0.522904in}{1.996084in}}%
\pgfpathlineto{\pgfqpoint{0.527264in}{1.984333in}}%
\pgfpathlineto{\pgfqpoint{0.531576in}{1.972582in}}%
\pgfpathlineto{\pgfqpoint{0.532314in}{1.970568in}}%
\pgfpathlineto{\pgfqpoint{0.536079in}{1.960830in}}%
\pgfpathlineto{\pgfqpoint{0.540580in}{1.949079in}}%
\pgfpathlineto{\pgfqpoint{0.544377in}{1.939070in}}%
\pgfpathlineto{\pgfqpoint{0.545076in}{1.937327in}}%
\pgfpathlineto{\pgfqpoint{0.549784in}{1.925576in}}%
\pgfpathlineto{\pgfqpoint{0.554436in}{1.913825in}}%
\pgfpathlineto{\pgfqpoint{0.556439in}{1.908740in}}%
\pgfpathlineto{\pgfqpoint{0.559223in}{1.902073in}}%
\pgfpathlineto{\pgfqpoint{0.564095in}{1.890322in}}%
\pgfpathlineto{\pgfqpoint{0.568502in}{1.879571in}}%
\pgfpathlineto{\pgfqpoint{0.568938in}{1.878571in}}%
\pgfpathlineto{\pgfqpoint{0.574048in}{1.866819in}}%
\pgfpathlineto{\pgfqpoint{0.579091in}{1.855068in}}%
\pgfpathlineto{\pgfqpoint{0.580565in}{1.851620in}}%
\pgfpathlineto{\pgfqpoint{0.584341in}{1.843316in}}%
\pgfpathlineto{\pgfqpoint{0.589631in}{1.831565in}}%
\pgfpathclose%
\pgfusepath{fill}%
\end{pgfscope}%
\begin{pgfscope}%
\pgfpathrectangle{\pgfqpoint{0.423750in}{1.819814in}}{\pgfqpoint{1.194205in}{1.163386in}}%
\pgfusepath{clip}%
\pgfsetbuttcap%
\pgfsetroundjoin%
\definecolor{currentfill}{rgb}{0.905301,0.238545,0.247481}%
\pgfsetfillcolor{currentfill}%
\pgfsetlinewidth{0.000000pt}%
\definecolor{currentstroke}{rgb}{0.000000,0.000000,0.000000}%
\pgfsetstrokecolor{currentstroke}%
\pgfsetdash{}{0pt}%
\pgfpathmoveto{\pgfqpoint{1.280200in}{2.688727in}}%
\pgfpathlineto{\pgfqpoint{1.292262in}{2.685285in}}%
\pgfpathlineto{\pgfqpoint{1.304325in}{2.683102in}}%
\pgfpathlineto{\pgfqpoint{1.316388in}{2.682209in}}%
\pgfpathlineto{\pgfqpoint{1.328450in}{2.682623in}}%
\pgfpathlineto{\pgfqpoint{1.340513in}{2.684346in}}%
\pgfpathlineto{\pgfqpoint{1.352576in}{2.687363in}}%
\pgfpathlineto{\pgfqpoint{1.358365in}{2.689416in}}%
\pgfpathlineto{\pgfqpoint{1.364638in}{2.691612in}}%
\pgfpathlineto{\pgfqpoint{1.376701in}{2.697032in}}%
\pgfpathlineto{\pgfqpoint{1.384310in}{2.701167in}}%
\pgfpathlineto{\pgfqpoint{1.388764in}{2.703560in}}%
\pgfpathlineto{\pgfqpoint{1.400826in}{2.711084in}}%
\pgfpathlineto{\pgfqpoint{1.403438in}{2.712918in}}%
\pgfpathlineto{\pgfqpoint{1.412889in}{2.719487in}}%
\pgfpathlineto{\pgfqpoint{1.419663in}{2.724670in}}%
\pgfpathlineto{\pgfqpoint{1.424952in}{2.728677in}}%
\pgfpathlineto{\pgfqpoint{1.434421in}{2.736421in}}%
\pgfpathlineto{\pgfqpoint{1.437014in}{2.738523in}}%
\pgfpathlineto{\pgfqpoint{1.448226in}{2.748172in}}%
\pgfpathlineto{\pgfqpoint{1.449077in}{2.748899in}}%
\pgfpathlineto{\pgfqpoint{1.461140in}{2.759674in}}%
\pgfpathlineto{\pgfqpoint{1.461410in}{2.759924in}}%
\pgfpathlineto{\pgfqpoint{1.473202in}{2.770727in}}%
\pgfpathlineto{\pgfqpoint{1.474215in}{2.771675in}}%
\pgfpathlineto{\pgfqpoint{1.485265in}{2.781953in}}%
\pgfpathlineto{\pgfqpoint{1.486831in}{2.783427in}}%
\pgfpathlineto{\pgfqpoint{1.497328in}{2.793246in}}%
\pgfpathlineto{\pgfqpoint{1.499388in}{2.795178in}}%
\pgfpathlineto{\pgfqpoint{1.509390in}{2.804509in}}%
\pgfpathlineto{\pgfqpoint{1.511999in}{2.806929in}}%
\pgfpathlineto{\pgfqpoint{1.521453in}{2.815659in}}%
\pgfpathlineto{\pgfqpoint{1.524767in}{2.818681in}}%
\pgfpathlineto{\pgfqpoint{1.533516in}{2.826626in}}%
\pgfpathlineto{\pgfqpoint{1.537786in}{2.830432in}}%
\pgfpathlineto{\pgfqpoint{1.545579in}{2.837352in}}%
\pgfpathlineto{\pgfqpoint{1.551148in}{2.842183in}}%
\pgfpathlineto{\pgfqpoint{1.557641in}{2.847798in}}%
\pgfpathlineto{\pgfqpoint{1.564935in}{2.853935in}}%
\pgfpathlineto{\pgfqpoint{1.569704in}{2.857937in}}%
\pgfpathlineto{\pgfqpoint{1.579214in}{2.865686in}}%
\pgfpathlineto{\pgfqpoint{1.581767in}{2.867761in}}%
\pgfpathlineto{\pgfqpoint{1.593829in}{2.877274in}}%
\pgfpathlineto{\pgfqpoint{1.594043in}{2.877438in}}%
\pgfpathlineto{\pgfqpoint{1.605892in}{2.886468in}}%
\pgfpathlineto{\pgfqpoint{1.609561in}{2.889189in}}%
\pgfpathlineto{\pgfqpoint{1.617955in}{2.895410in}}%
\pgfpathlineto{\pgfqpoint{1.617955in}{2.900940in}}%
\pgfpathlineto{\pgfqpoint{1.617955in}{2.912692in}}%
\pgfpathlineto{\pgfqpoint{1.617955in}{2.924443in}}%
\pgfpathlineto{\pgfqpoint{1.617955in}{2.936194in}}%
\pgfpathlineto{\pgfqpoint{1.617955in}{2.947363in}}%
\pgfpathlineto{\pgfqpoint{1.605892in}{2.938264in}}%
\pgfpathlineto{\pgfqpoint{1.603211in}{2.936194in}}%
\pgfpathlineto{\pgfqpoint{1.593829in}{2.928948in}}%
\pgfpathlineto{\pgfqpoint{1.588141in}{2.924443in}}%
\pgfpathlineto{\pgfqpoint{1.581767in}{2.919387in}}%
\pgfpathlineto{\pgfqpoint{1.573543in}{2.912692in}}%
\pgfpathlineto{\pgfqpoint{1.569704in}{2.909560in}}%
\pgfpathlineto{\pgfqpoint{1.559405in}{2.900940in}}%
\pgfpathlineto{\pgfqpoint{1.557641in}{2.899461in}}%
\pgfpathlineto{\pgfqpoint{1.545684in}{2.889189in}}%
\pgfpathlineto{\pgfqpoint{1.545579in}{2.889098in}}%
\pgfpathlineto{\pgfqpoint{1.533516in}{2.878483in}}%
\pgfpathlineto{\pgfqpoint{1.532346in}{2.877438in}}%
\pgfpathlineto{\pgfqpoint{1.521453in}{2.867670in}}%
\pgfpathlineto{\pgfqpoint{1.519261in}{2.865686in}}%
\pgfpathlineto{\pgfqpoint{1.509390in}{2.856713in}}%
\pgfpathlineto{\pgfqpoint{1.506342in}{2.853935in}}%
\pgfpathlineto{\pgfqpoint{1.497328in}{2.845680in}}%
\pgfpathlineto{\pgfqpoint{1.493490in}{2.842183in}}%
\pgfpathlineto{\pgfqpoint{1.485265in}{2.834649in}}%
\pgfpathlineto{\pgfqpoint{1.480598in}{2.830432in}}%
\pgfpathlineto{\pgfqpoint{1.473202in}{2.823710in}}%
\pgfpathlineto{\pgfqpoint{1.467536in}{2.818681in}}%
\pgfpathlineto{\pgfqpoint{1.461140in}{2.812965in}}%
\pgfpathlineto{\pgfqpoint{1.454147in}{2.806929in}}%
\pgfpathlineto{\pgfqpoint{1.449077in}{2.802521in}}%
\pgfpathlineto{\pgfqpoint{1.440227in}{2.795178in}}%
\pgfpathlineto{\pgfqpoint{1.437014in}{2.792492in}}%
\pgfpathlineto{\pgfqpoint{1.425492in}{2.783427in}}%
\pgfpathlineto{\pgfqpoint{1.424952in}{2.782998in}}%
\pgfpathlineto{\pgfqpoint{1.412889in}{2.774126in}}%
\pgfpathlineto{\pgfqpoint{1.409220in}{2.771675in}}%
\pgfpathlineto{\pgfqpoint{1.400826in}{2.766015in}}%
\pgfpathlineto{\pgfqpoint{1.390639in}{2.759924in}}%
\pgfpathlineto{\pgfqpoint{1.388764in}{2.758791in}}%
\pgfpathlineto{\pgfqpoint{1.376701in}{2.752509in}}%
\pgfpathlineto{\pgfqpoint{1.366626in}{2.748172in}}%
\pgfpathlineto{\pgfqpoint{1.364638in}{2.747307in}}%
\pgfpathlineto{\pgfqpoint{1.352576in}{2.743209in}}%
\pgfpathlineto{\pgfqpoint{1.340513in}{2.740316in}}%
\pgfpathlineto{\pgfqpoint{1.328450in}{2.738658in}}%
\pgfpathlineto{\pgfqpoint{1.316388in}{2.738247in}}%
\pgfpathlineto{\pgfqpoint{1.304325in}{2.739082in}}%
\pgfpathlineto{\pgfqpoint{1.292262in}{2.741148in}}%
\pgfpathlineto{\pgfqpoint{1.280200in}{2.744413in}}%
\pgfpathlineto{\pgfqpoint{1.269941in}{2.748172in}}%
\pgfpathlineto{\pgfqpoint{1.268137in}{2.748825in}}%
\pgfpathlineto{\pgfqpoint{1.256074in}{2.754269in}}%
\pgfpathlineto{\pgfqpoint{1.245509in}{2.759924in}}%
\pgfpathlineto{\pgfqpoint{1.244012in}{2.760716in}}%
\pgfpathlineto{\pgfqpoint{1.231949in}{2.768009in}}%
\pgfpathlineto{\pgfqpoint{1.226497in}{2.771675in}}%
\pgfpathlineto{\pgfqpoint{1.219886in}{2.776077in}}%
\pgfpathlineto{\pgfqpoint{1.209748in}{2.783427in}}%
\pgfpathlineto{\pgfqpoint{1.207824in}{2.784808in}}%
\pgfpathlineto{\pgfqpoint{1.195761in}{2.794066in}}%
\pgfpathlineto{\pgfqpoint{1.194383in}{2.795178in}}%
\pgfpathlineto{\pgfqpoint{1.183698in}{2.803732in}}%
\pgfpathlineto{\pgfqpoint{1.179853in}{2.806929in}}%
\pgfpathlineto{\pgfqpoint{1.171636in}{2.813710in}}%
\pgfpathlineto{\pgfqpoint{1.165767in}{2.818681in}}%
\pgfpathlineto{\pgfqpoint{1.159573in}{2.823890in}}%
\pgfpathlineto{\pgfqpoint{1.151914in}{2.830432in}}%
\pgfpathlineto{\pgfqpoint{1.147510in}{2.834170in}}%
\pgfpathlineto{\pgfqpoint{1.138129in}{2.842183in}}%
\pgfpathlineto{\pgfqpoint{1.135448in}{2.844460in}}%
\pgfpathlineto{\pgfqpoint{1.124269in}{2.853935in}}%
\pgfpathlineto{\pgfqpoint{1.123385in}{2.854680in}}%
\pgfpathlineto{\pgfqpoint{1.111322in}{2.864750in}}%
\pgfpathlineto{\pgfqpoint{1.110183in}{2.865686in}}%
\pgfpathlineto{\pgfqpoint{1.099260in}{2.874617in}}%
\pgfpathlineto{\pgfqpoint{1.095737in}{2.877438in}}%
\pgfpathlineto{\pgfqpoint{1.087197in}{2.884251in}}%
\pgfpathlineto{\pgfqpoint{1.080853in}{2.889189in}}%
\pgfpathlineto{\pgfqpoint{1.075134in}{2.893626in}}%
\pgfpathlineto{\pgfqpoint{1.065452in}{2.900940in}}%
\pgfpathlineto{\pgfqpoint{1.063072in}{2.902734in}}%
\pgfpathlineto{\pgfqpoint{1.051009in}{2.911565in}}%
\pgfpathlineto{\pgfqpoint{1.049425in}{2.912692in}}%
\pgfpathlineto{\pgfqpoint{1.038946in}{2.920129in}}%
\pgfpathlineto{\pgfqpoint{1.032723in}{2.924443in}}%
\pgfpathlineto{\pgfqpoint{1.026884in}{2.928486in}}%
\pgfpathlineto{\pgfqpoint{1.015541in}{2.936194in}}%
\pgfpathlineto{\pgfqpoint{1.014821in}{2.936684in}}%
\pgfpathlineto{\pgfqpoint{1.002758in}{2.944744in}}%
\pgfpathlineto{\pgfqpoint{0.997950in}{2.947946in}}%
\pgfpathlineto{\pgfqpoint{0.990696in}{2.952777in}}%
\pgfpathlineto{\pgfqpoint{0.980385in}{2.959697in}}%
\pgfpathlineto{\pgfqpoint{0.978633in}{2.960874in}}%
\pgfpathlineto{\pgfqpoint{0.966570in}{2.969097in}}%
\pgfpathlineto{\pgfqpoint{0.963217in}{2.971449in}}%
\pgfpathlineto{\pgfqpoint{0.954508in}{2.977568in}}%
\pgfpathlineto{\pgfqpoint{0.946815in}{2.983200in}}%
\pgfpathlineto{\pgfqpoint{0.942445in}{2.983200in}}%
\pgfpathlineto{\pgfqpoint{0.930382in}{2.983200in}}%
\pgfpathlineto{\pgfqpoint{0.918320in}{2.983200in}}%
\pgfpathlineto{\pgfqpoint{0.906257in}{2.983200in}}%
\pgfpathlineto{\pgfqpoint{0.894194in}{2.983200in}}%
\pgfpathlineto{\pgfqpoint{0.883549in}{2.983200in}}%
\pgfpathlineto{\pgfqpoint{0.894194in}{2.972554in}}%
\pgfpathlineto{\pgfqpoint{0.895387in}{2.971449in}}%
\pgfpathlineto{\pgfqpoint{0.906257in}{2.961428in}}%
\pgfpathlineto{\pgfqpoint{0.908277in}{2.959697in}}%
\pgfpathlineto{\pgfqpoint{0.918320in}{2.951136in}}%
\pgfpathlineto{\pgfqpoint{0.922329in}{2.947946in}}%
\pgfpathlineto{\pgfqpoint{0.930382in}{2.941563in}}%
\pgfpathlineto{\pgfqpoint{0.937592in}{2.936194in}}%
\pgfpathlineto{\pgfqpoint{0.942445in}{2.932592in}}%
\pgfpathlineto{\pgfqpoint{0.954024in}{2.924443in}}%
\pgfpathlineto{\pgfqpoint{0.954508in}{2.924104in}}%
\pgfpathlineto{\pgfqpoint{0.966570in}{2.915933in}}%
\pgfpathlineto{\pgfqpoint{0.971497in}{2.912692in}}%
\pgfpathlineto{\pgfqpoint{0.978633in}{2.908004in}}%
\pgfpathlineto{\pgfqpoint{0.989571in}{2.900940in}}%
\pgfpathlineto{\pgfqpoint{0.990696in}{2.900215in}}%
\pgfpathlineto{\pgfqpoint{1.002758in}{2.892418in}}%
\pgfpathlineto{\pgfqpoint{1.007716in}{2.889189in}}%
\pgfpathlineto{\pgfqpoint{1.014821in}{2.884560in}}%
\pgfpathlineto{\pgfqpoint{1.025577in}{2.877438in}}%
\pgfpathlineto{\pgfqpoint{1.026884in}{2.876572in}}%
\pgfpathlineto{\pgfqpoint{1.038946in}{2.868348in}}%
\pgfpathlineto{\pgfqpoint{1.042737in}{2.865686in}}%
\pgfpathlineto{\pgfqpoint{1.051009in}{2.859866in}}%
\pgfpathlineto{\pgfqpoint{1.059175in}{2.853935in}}%
\pgfpathlineto{\pgfqpoint{1.063072in}{2.851098in}}%
\pgfpathlineto{\pgfqpoint{1.074921in}{2.842183in}}%
\pgfpathlineto{\pgfqpoint{1.075134in}{2.842023in}}%
\pgfpathlineto{\pgfqpoint{1.087197in}{2.832603in}}%
\pgfpathlineto{\pgfqpoint{1.089898in}{2.830432in}}%
\pgfpathlineto{\pgfqpoint{1.099260in}{2.822879in}}%
\pgfpathlineto{\pgfqpoint{1.104339in}{2.818681in}}%
\pgfpathlineto{\pgfqpoint{1.111322in}{2.812881in}}%
\pgfpathlineto{\pgfqpoint{1.118354in}{2.806929in}}%
\pgfpathlineto{\pgfqpoint{1.123385in}{2.802649in}}%
\pgfpathlineto{\pgfqpoint{1.132056in}{2.795178in}}%
\pgfpathlineto{\pgfqpoint{1.135448in}{2.792239in}}%
\pgfpathlineto{\pgfqpoint{1.145566in}{2.783427in}}%
\pgfpathlineto{\pgfqpoint{1.147510in}{2.781722in}}%
\pgfpathlineto{\pgfqpoint{1.159014in}{2.771675in}}%
\pgfpathlineto{\pgfqpoint{1.159573in}{2.771184in}}%
\pgfpathlineto{\pgfqpoint{1.171636in}{2.760709in}}%
\pgfpathlineto{\pgfqpoint{1.172562in}{2.759924in}}%
\pgfpathlineto{\pgfqpoint{1.183698in}{2.750406in}}%
\pgfpathlineto{\pgfqpoint{1.186408in}{2.748172in}}%
\pgfpathlineto{\pgfqpoint{1.195761in}{2.740397in}}%
\pgfpathlineto{\pgfqpoint{1.200788in}{2.736421in}}%
\pgfpathlineto{\pgfqpoint{1.207824in}{2.730804in}}%
\pgfpathlineto{\pgfqpoint{1.216027in}{2.724670in}}%
\pgfpathlineto{\pgfqpoint{1.219886in}{2.721755in}}%
\pgfpathlineto{\pgfqpoint{1.231949in}{2.713373in}}%
\pgfpathlineto{\pgfqpoint{1.232675in}{2.712918in}}%
\pgfpathlineto{\pgfqpoint{1.244012in}{2.705742in}}%
\pgfpathlineto{\pgfqpoint{1.252272in}{2.701167in}}%
\pgfpathlineto{\pgfqpoint{1.256074in}{2.699035in}}%
\pgfpathlineto{\pgfqpoint{1.268137in}{2.693320in}}%
\pgfpathlineto{\pgfqpoint{1.278411in}{2.689416in}}%
\pgfpathclose%
\pgfusepath{fill}%
\end{pgfscope}%
\begin{pgfscope}%
\pgfpathrectangle{\pgfqpoint{0.423750in}{1.819814in}}{\pgfqpoint{1.194205in}{1.163386in}}%
\pgfusepath{clip}%
\pgfsetbuttcap%
\pgfsetroundjoin%
\definecolor{currentfill}{rgb}{0.934351,0.329284,0.247753}%
\pgfsetfillcolor{currentfill}%
\pgfsetlinewidth{0.000000pt}%
\definecolor{currentstroke}{rgb}{0.000000,0.000000,0.000000}%
\pgfsetstrokecolor{currentstroke}%
\pgfsetdash{}{0pt}%
\pgfpathmoveto{\pgfqpoint{0.616753in}{1.826727in}}%
\pgfpathlineto{\pgfqpoint{0.620384in}{1.819814in}}%
\pgfpathlineto{\pgfqpoint{0.628815in}{1.819814in}}%
\pgfpathlineto{\pgfqpoint{0.640878in}{1.819814in}}%
\pgfpathlineto{\pgfqpoint{0.651003in}{1.819814in}}%
\pgfpathlineto{\pgfqpoint{0.644209in}{1.831565in}}%
\pgfpathlineto{\pgfqpoint{0.640878in}{1.837256in}}%
\pgfpathlineto{\pgfqpoint{0.637566in}{1.843316in}}%
\pgfpathlineto{\pgfqpoint{0.631068in}{1.855068in}}%
\pgfpathlineto{\pgfqpoint{0.628815in}{1.859105in}}%
\pgfpathlineto{\pgfqpoint{0.624804in}{1.866819in}}%
\pgfpathlineto{\pgfqpoint{0.618610in}{1.878571in}}%
\pgfpathlineto{\pgfqpoint{0.616753in}{1.882068in}}%
\pgfpathlineto{\pgfqpoint{0.612670in}{1.890322in}}%
\pgfpathlineto{\pgfqpoint{0.606781in}{1.902073in}}%
\pgfpathlineto{\pgfqpoint{0.604690in}{1.906215in}}%
\pgfpathlineto{\pgfqpoint{0.601110in}{1.913825in}}%
\pgfpathlineto{\pgfqpoint{0.595518in}{1.925576in}}%
\pgfpathlineto{\pgfqpoint{0.592627in}{1.931594in}}%
\pgfpathlineto{\pgfqpoint{0.590059in}{1.937327in}}%
\pgfpathlineto{\pgfqpoint{0.584752in}{1.949079in}}%
\pgfpathlineto{\pgfqpoint{0.580565in}{1.958235in}}%
\pgfpathlineto{\pgfqpoint{0.579456in}{1.960830in}}%
\pgfpathlineto{\pgfqpoint{0.574418in}{1.972582in}}%
\pgfpathlineto{\pgfqpoint{0.569303in}{1.984333in}}%
\pgfpathlineto{\pgfqpoint{0.568502in}{1.986169in}}%
\pgfpathlineto{\pgfqpoint{0.564453in}{1.996084in}}%
\pgfpathlineto{\pgfqpoint{0.559598in}{2.007836in}}%
\pgfpathlineto{\pgfqpoint{0.556439in}{2.015409in}}%
\pgfpathlineto{\pgfqpoint{0.554805in}{2.019587in}}%
\pgfpathlineto{\pgfqpoint{0.550187in}{2.031338in}}%
\pgfpathlineto{\pgfqpoint{0.545506in}{2.043090in}}%
\pgfpathlineto{\pgfqpoint{0.544377in}{2.045918in}}%
\pgfpathlineto{\pgfqpoint{0.541024in}{2.054841in}}%
\pgfpathlineto{\pgfqpoint{0.536567in}{2.066593in}}%
\pgfpathlineto{\pgfqpoint{0.532314in}{2.077672in}}%
\pgfpathlineto{\pgfqpoint{0.532071in}{2.078344in}}%
\pgfpathlineto{\pgfqpoint{0.527816in}{2.090095in}}%
\pgfpathlineto{\pgfqpoint{0.523509in}{2.101847in}}%
\pgfpathlineto{\pgfqpoint{0.520251in}{2.110656in}}%
\pgfpathlineto{\pgfqpoint{0.519221in}{2.113598in}}%
\pgfpathlineto{\pgfqpoint{0.515101in}{2.125349in}}%
\pgfpathlineto{\pgfqpoint{0.510932in}{2.137101in}}%
\pgfpathlineto{\pgfqpoint{0.508189in}{2.144779in}}%
\pgfpathlineto{\pgfqpoint{0.506806in}{2.148852in}}%
\pgfpathlineto{\pgfqpoint{0.502808in}{2.160604in}}%
\pgfpathlineto{\pgfqpoint{0.498764in}{2.172355in}}%
\pgfpathlineto{\pgfqpoint{0.496126in}{2.179969in}}%
\pgfpathlineto{\pgfqpoint{0.494760in}{2.184106in}}%
\pgfpathlineto{\pgfqpoint{0.490871in}{2.195858in}}%
\pgfpathlineto{\pgfqpoint{0.486939in}{2.207609in}}%
\pgfpathlineto{\pgfqpoint{0.484063in}{2.216139in}}%
\pgfpathlineto{\pgfqpoint{0.483025in}{2.219360in}}%
\pgfpathlineto{\pgfqpoint{0.479233in}{2.231112in}}%
\pgfpathlineto{\pgfqpoint{0.475399in}{2.242863in}}%
\pgfpathlineto{\pgfqpoint{0.472001in}{2.253183in}}%
\pgfpathlineto{\pgfqpoint{0.471549in}{2.254615in}}%
\pgfpathlineto{\pgfqpoint{0.467841in}{2.266366in}}%
\pgfpathlineto{\pgfqpoint{0.464093in}{2.278117in}}%
\pgfpathlineto{\pgfqpoint{0.460303in}{2.289869in}}%
\pgfpathlineto{\pgfqpoint{0.459938in}{2.291002in}}%
\pgfpathlineto{\pgfqpoint{0.456648in}{2.301620in}}%
\pgfpathlineto{\pgfqpoint{0.452973in}{2.313371in}}%
\pgfpathlineto{\pgfqpoint{0.449257in}{2.325123in}}%
\pgfpathlineto{\pgfqpoint{0.447875in}{2.329482in}}%
\pgfpathlineto{\pgfqpoint{0.445615in}{2.336874in}}%
\pgfpathlineto{\pgfqpoint{0.442000in}{2.348626in}}%
\pgfpathlineto{\pgfqpoint{0.438345in}{2.360377in}}%
\pgfpathlineto{\pgfqpoint{0.435813in}{2.368464in}}%
\pgfpathlineto{\pgfqpoint{0.434703in}{2.372128in}}%
\pgfpathlineto{\pgfqpoint{0.431137in}{2.383880in}}%
\pgfpathlineto{\pgfqpoint{0.427531in}{2.395631in}}%
\pgfpathlineto{\pgfqpoint{0.423884in}{2.407383in}}%
\pgfpathlineto{\pgfqpoint{0.423750in}{2.407816in}}%
\pgfpathlineto{\pgfqpoint{0.423750in}{2.407383in}}%
\pgfpathlineto{\pgfqpoint{0.423750in}{2.395631in}}%
\pgfpathlineto{\pgfqpoint{0.423750in}{2.383880in}}%
\pgfpathlineto{\pgfqpoint{0.423750in}{2.372128in}}%
\pgfpathlineto{\pgfqpoint{0.423750in}{2.360377in}}%
\pgfpathlineto{\pgfqpoint{0.423750in}{2.349543in}}%
\pgfpathlineto{\pgfqpoint{0.424037in}{2.348626in}}%
\pgfpathlineto{\pgfqpoint{0.427718in}{2.336874in}}%
\pgfpathlineto{\pgfqpoint{0.431359in}{2.325123in}}%
\pgfpathlineto{\pgfqpoint{0.434962in}{2.313371in}}%
\pgfpathlineto{\pgfqpoint{0.435813in}{2.310596in}}%
\pgfpathlineto{\pgfqpoint{0.438652in}{2.301620in}}%
\pgfpathlineto{\pgfqpoint{0.442341in}{2.289869in}}%
\pgfpathlineto{\pgfqpoint{0.445992in}{2.278117in}}%
\pgfpathlineto{\pgfqpoint{0.447875in}{2.272029in}}%
\pgfpathlineto{\pgfqpoint{0.449689in}{2.266366in}}%
\pgfpathlineto{\pgfqpoint{0.453438in}{2.254615in}}%
\pgfpathlineto{\pgfqpoint{0.457147in}{2.242863in}}%
\pgfpathlineto{\pgfqpoint{0.459938in}{2.233959in}}%
\pgfpathlineto{\pgfqpoint{0.460865in}{2.231112in}}%
\pgfpathlineto{\pgfqpoint{0.464684in}{2.219360in}}%
\pgfpathlineto{\pgfqpoint{0.468465in}{2.207609in}}%
\pgfpathlineto{\pgfqpoint{0.472001in}{2.196511in}}%
\pgfpathlineto{\pgfqpoint{0.472218in}{2.195858in}}%
\pgfpathlineto{\pgfqpoint{0.476121in}{2.184106in}}%
\pgfpathlineto{\pgfqpoint{0.479984in}{2.172355in}}%
\pgfpathlineto{\pgfqpoint{0.483807in}{2.160604in}}%
\pgfpathlineto{\pgfqpoint{0.484063in}{2.159817in}}%
\pgfpathlineto{\pgfqpoint{0.487793in}{2.148852in}}%
\pgfpathlineto{\pgfqpoint{0.491751in}{2.137101in}}%
\pgfpathlineto{\pgfqpoint{0.495667in}{2.125349in}}%
\pgfpathlineto{\pgfqpoint{0.496126in}{2.123974in}}%
\pgfpathlineto{\pgfqpoint{0.499750in}{2.113598in}}%
\pgfpathlineto{\pgfqpoint{0.503815in}{2.101847in}}%
\pgfpathlineto{\pgfqpoint{0.507837in}{2.090095in}}%
\pgfpathlineto{\pgfqpoint{0.508189in}{2.089068in}}%
\pgfpathlineto{\pgfqpoint{0.512048in}{2.078344in}}%
\pgfpathlineto{\pgfqpoint{0.516233in}{2.066593in}}%
\pgfpathlineto{\pgfqpoint{0.520251in}{2.055189in}}%
\pgfpathlineto{\pgfqpoint{0.520381in}{2.054841in}}%
\pgfpathlineto{\pgfqpoint{0.524749in}{2.043090in}}%
\pgfpathlineto{\pgfqpoint{0.529067in}{2.031338in}}%
\pgfpathlineto{\pgfqpoint{0.532314in}{2.022427in}}%
\pgfpathlineto{\pgfqpoint{0.533408in}{2.019587in}}%
\pgfpathlineto{\pgfqpoint{0.537924in}{2.007836in}}%
\pgfpathlineto{\pgfqpoint{0.542385in}{1.996084in}}%
\pgfpathlineto{\pgfqpoint{0.544377in}{1.990814in}}%
\pgfpathlineto{\pgfqpoint{0.546972in}{1.984333in}}%
\pgfpathlineto{\pgfqpoint{0.551646in}{1.972582in}}%
\pgfpathlineto{\pgfqpoint{0.556262in}{1.960830in}}%
\pgfpathlineto{\pgfqpoint{0.556439in}{1.960379in}}%
\pgfpathlineto{\pgfqpoint{0.561155in}{1.949079in}}%
\pgfpathlineto{\pgfqpoint{0.565997in}{1.937327in}}%
\pgfpathlineto{\pgfqpoint{0.568502in}{1.931206in}}%
\pgfpathlineto{\pgfqpoint{0.570956in}{1.925576in}}%
\pgfpathlineto{\pgfqpoint{0.576042in}{1.913825in}}%
\pgfpathlineto{\pgfqpoint{0.580565in}{1.903242in}}%
\pgfpathlineto{\pgfqpoint{0.581098in}{1.902073in}}%
\pgfpathlineto{\pgfqpoint{0.586447in}{1.890322in}}%
\pgfpathlineto{\pgfqpoint{0.591716in}{1.878571in}}%
\pgfpathlineto{\pgfqpoint{0.592627in}{1.876532in}}%
\pgfpathlineto{\pgfqpoint{0.597266in}{1.866819in}}%
\pgfpathlineto{\pgfqpoint{0.602804in}{1.855068in}}%
\pgfpathlineto{\pgfqpoint{0.604690in}{1.851042in}}%
\pgfpathlineto{\pgfqpoint{0.608559in}{1.843316in}}%
\pgfpathlineto{\pgfqpoint{0.614376in}{1.831565in}}%
\pgfpathclose%
\pgfusepath{fill}%
\end{pgfscope}%
\begin{pgfscope}%
\pgfpathrectangle{\pgfqpoint{0.423750in}{1.819814in}}{\pgfqpoint{1.194205in}{1.163386in}}%
\pgfusepath{clip}%
\pgfsetbuttcap%
\pgfsetroundjoin%
\definecolor{currentfill}{rgb}{0.934351,0.329284,0.247753}%
\pgfsetfillcolor{currentfill}%
\pgfsetlinewidth{0.000000pt}%
\definecolor{currentstroke}{rgb}{0.000000,0.000000,0.000000}%
\pgfsetstrokecolor{currentstroke}%
\pgfsetdash{}{0pt}%
\pgfpathmoveto{\pgfqpoint{1.280200in}{2.628736in}}%
\pgfpathlineto{\pgfqpoint{1.292262in}{2.625096in}}%
\pgfpathlineto{\pgfqpoint{1.304325in}{2.622779in}}%
\pgfpathlineto{\pgfqpoint{1.316388in}{2.621819in}}%
\pgfpathlineto{\pgfqpoint{1.328450in}{2.622234in}}%
\pgfpathlineto{\pgfqpoint{1.340513in}{2.624026in}}%
\pgfpathlineto{\pgfqpoint{1.352576in}{2.627181in}}%
\pgfpathlineto{\pgfqpoint{1.361935in}{2.630659in}}%
\pgfpathlineto{\pgfqpoint{1.364638in}{2.631649in}}%
\pgfpathlineto{\pgfqpoint{1.376701in}{2.637324in}}%
\pgfpathlineto{\pgfqpoint{1.385632in}{2.642410in}}%
\pgfpathlineto{\pgfqpoint{1.388764in}{2.644172in}}%
\pgfpathlineto{\pgfqpoint{1.400826in}{2.652045in}}%
\pgfpathlineto{\pgfqpoint{1.403707in}{2.654161in}}%
\pgfpathlineto{\pgfqpoint{1.412889in}{2.660832in}}%
\pgfpathlineto{\pgfqpoint{1.419247in}{2.665913in}}%
\pgfpathlineto{\pgfqpoint{1.424952in}{2.670424in}}%
\pgfpathlineto{\pgfqpoint{1.433443in}{2.677664in}}%
\pgfpathlineto{\pgfqpoint{1.437014in}{2.680680in}}%
\pgfpathlineto{\pgfqpoint{1.446770in}{2.689416in}}%
\pgfpathlineto{\pgfqpoint{1.449077in}{2.691463in}}%
\pgfpathlineto{\pgfqpoint{1.459539in}{2.701167in}}%
\pgfpathlineto{\pgfqpoint{1.461140in}{2.702640in}}%
\pgfpathlineto{\pgfqpoint{1.471968in}{2.712918in}}%
\pgfpathlineto{\pgfqpoint{1.473202in}{2.714082in}}%
\pgfpathlineto{\pgfqpoint{1.484220in}{2.724670in}}%
\pgfpathlineto{\pgfqpoint{1.485265in}{2.725667in}}%
\pgfpathlineto{\pgfqpoint{1.496424in}{2.736421in}}%
\pgfpathlineto{\pgfqpoint{1.497328in}{2.737286in}}%
\pgfpathlineto{\pgfqpoint{1.508693in}{2.748172in}}%
\pgfpathlineto{\pgfqpoint{1.509390in}{2.748837in}}%
\pgfpathlineto{\pgfqpoint{1.521124in}{2.759924in}}%
\pgfpathlineto{\pgfqpoint{1.521453in}{2.760233in}}%
\pgfpathlineto{\pgfqpoint{1.533516in}{2.771396in}}%
\pgfpathlineto{\pgfqpoint{1.533824in}{2.771675in}}%
\pgfpathlineto{\pgfqpoint{1.545579in}{2.782263in}}%
\pgfpathlineto{\pgfqpoint{1.546908in}{2.783427in}}%
\pgfpathlineto{\pgfqpoint{1.557641in}{2.792794in}}%
\pgfpathlineto{\pgfqpoint{1.560462in}{2.795178in}}%
\pgfpathlineto{\pgfqpoint{1.569704in}{2.802965in}}%
\pgfpathlineto{\pgfqpoint{1.574578in}{2.806929in}}%
\pgfpathlineto{\pgfqpoint{1.581767in}{2.812764in}}%
\pgfpathlineto{\pgfqpoint{1.589327in}{2.818681in}}%
\pgfpathlineto{\pgfqpoint{1.593829in}{2.822199in}}%
\pgfpathlineto{\pgfqpoint{1.604745in}{2.830432in}}%
\pgfpathlineto{\pgfqpoint{1.605892in}{2.831296in}}%
\pgfpathlineto{\pgfqpoint{1.617955in}{2.840063in}}%
\pgfpathlineto{\pgfqpoint{1.617955in}{2.842183in}}%
\pgfpathlineto{\pgfqpoint{1.617955in}{2.853935in}}%
\pgfpathlineto{\pgfqpoint{1.617955in}{2.865686in}}%
\pgfpathlineto{\pgfqpoint{1.617955in}{2.877438in}}%
\pgfpathlineto{\pgfqpoint{1.617955in}{2.889189in}}%
\pgfpathlineto{\pgfqpoint{1.617955in}{2.895410in}}%
\pgfpathlineto{\pgfqpoint{1.609561in}{2.889189in}}%
\pgfpathlineto{\pgfqpoint{1.605892in}{2.886468in}}%
\pgfpathlineto{\pgfqpoint{1.594043in}{2.877438in}}%
\pgfpathlineto{\pgfqpoint{1.593829in}{2.877274in}}%
\pgfpathlineto{\pgfqpoint{1.581767in}{2.867761in}}%
\pgfpathlineto{\pgfqpoint{1.579214in}{2.865686in}}%
\pgfpathlineto{\pgfqpoint{1.569704in}{2.857937in}}%
\pgfpathlineto{\pgfqpoint{1.564935in}{2.853935in}}%
\pgfpathlineto{\pgfqpoint{1.557641in}{2.847798in}}%
\pgfpathlineto{\pgfqpoint{1.551148in}{2.842183in}}%
\pgfpathlineto{\pgfqpoint{1.545579in}{2.837352in}}%
\pgfpathlineto{\pgfqpoint{1.537786in}{2.830432in}}%
\pgfpathlineto{\pgfqpoint{1.533516in}{2.826626in}}%
\pgfpathlineto{\pgfqpoint{1.524767in}{2.818681in}}%
\pgfpathlineto{\pgfqpoint{1.521453in}{2.815659in}}%
\pgfpathlineto{\pgfqpoint{1.511999in}{2.806929in}}%
\pgfpathlineto{\pgfqpoint{1.509390in}{2.804509in}}%
\pgfpathlineto{\pgfqpoint{1.499388in}{2.795178in}}%
\pgfpathlineto{\pgfqpoint{1.497328in}{2.793246in}}%
\pgfpathlineto{\pgfqpoint{1.486831in}{2.783427in}}%
\pgfpathlineto{\pgfqpoint{1.485265in}{2.781953in}}%
\pgfpathlineto{\pgfqpoint{1.474215in}{2.771675in}}%
\pgfpathlineto{\pgfqpoint{1.473202in}{2.770727in}}%
\pgfpathlineto{\pgfqpoint{1.461410in}{2.759924in}}%
\pgfpathlineto{\pgfqpoint{1.461140in}{2.759674in}}%
\pgfpathlineto{\pgfqpoint{1.449077in}{2.748899in}}%
\pgfpathlineto{\pgfqpoint{1.448226in}{2.748172in}}%
\pgfpathlineto{\pgfqpoint{1.437014in}{2.738523in}}%
\pgfpathlineto{\pgfqpoint{1.434421in}{2.736421in}}%
\pgfpathlineto{\pgfqpoint{1.424952in}{2.728677in}}%
\pgfpathlineto{\pgfqpoint{1.419663in}{2.724670in}}%
\pgfpathlineto{\pgfqpoint{1.412889in}{2.719487in}}%
\pgfpathlineto{\pgfqpoint{1.403438in}{2.712918in}}%
\pgfpathlineto{\pgfqpoint{1.400826in}{2.711084in}}%
\pgfpathlineto{\pgfqpoint{1.388764in}{2.703560in}}%
\pgfpathlineto{\pgfqpoint{1.384310in}{2.701167in}}%
\pgfpathlineto{\pgfqpoint{1.376701in}{2.697032in}}%
\pgfpathlineto{\pgfqpoint{1.364638in}{2.691612in}}%
\pgfpathlineto{\pgfqpoint{1.358365in}{2.689416in}}%
\pgfpathlineto{\pgfqpoint{1.352576in}{2.687363in}}%
\pgfpathlineto{\pgfqpoint{1.340513in}{2.684346in}}%
\pgfpathlineto{\pgfqpoint{1.328450in}{2.682623in}}%
\pgfpathlineto{\pgfqpoint{1.316388in}{2.682209in}}%
\pgfpathlineto{\pgfqpoint{1.304325in}{2.683102in}}%
\pgfpathlineto{\pgfqpoint{1.292262in}{2.685285in}}%
\pgfpathlineto{\pgfqpoint{1.280200in}{2.688727in}}%
\pgfpathlineto{\pgfqpoint{1.278411in}{2.689416in}}%
\pgfpathlineto{\pgfqpoint{1.268137in}{2.693320in}}%
\pgfpathlineto{\pgfqpoint{1.256074in}{2.699035in}}%
\pgfpathlineto{\pgfqpoint{1.252272in}{2.701167in}}%
\pgfpathlineto{\pgfqpoint{1.244012in}{2.705742in}}%
\pgfpathlineto{\pgfqpoint{1.232675in}{2.712918in}}%
\pgfpathlineto{\pgfqpoint{1.231949in}{2.713373in}}%
\pgfpathlineto{\pgfqpoint{1.219886in}{2.721755in}}%
\pgfpathlineto{\pgfqpoint{1.216027in}{2.724670in}}%
\pgfpathlineto{\pgfqpoint{1.207824in}{2.730804in}}%
\pgfpathlineto{\pgfqpoint{1.200788in}{2.736421in}}%
\pgfpathlineto{\pgfqpoint{1.195761in}{2.740397in}}%
\pgfpathlineto{\pgfqpoint{1.186408in}{2.748172in}}%
\pgfpathlineto{\pgfqpoint{1.183698in}{2.750406in}}%
\pgfpathlineto{\pgfqpoint{1.172562in}{2.759924in}}%
\pgfpathlineto{\pgfqpoint{1.171636in}{2.760709in}}%
\pgfpathlineto{\pgfqpoint{1.159573in}{2.771184in}}%
\pgfpathlineto{\pgfqpoint{1.159014in}{2.771675in}}%
\pgfpathlineto{\pgfqpoint{1.147510in}{2.781722in}}%
\pgfpathlineto{\pgfqpoint{1.145566in}{2.783427in}}%
\pgfpathlineto{\pgfqpoint{1.135448in}{2.792239in}}%
\pgfpathlineto{\pgfqpoint{1.132056in}{2.795178in}}%
\pgfpathlineto{\pgfqpoint{1.123385in}{2.802649in}}%
\pgfpathlineto{\pgfqpoint{1.118354in}{2.806929in}}%
\pgfpathlineto{\pgfqpoint{1.111322in}{2.812881in}}%
\pgfpathlineto{\pgfqpoint{1.104339in}{2.818681in}}%
\pgfpathlineto{\pgfqpoint{1.099260in}{2.822879in}}%
\pgfpathlineto{\pgfqpoint{1.089898in}{2.830432in}}%
\pgfpathlineto{\pgfqpoint{1.087197in}{2.832603in}}%
\pgfpathlineto{\pgfqpoint{1.075134in}{2.842023in}}%
\pgfpathlineto{\pgfqpoint{1.074921in}{2.842183in}}%
\pgfpathlineto{\pgfqpoint{1.063072in}{2.851098in}}%
\pgfpathlineto{\pgfqpoint{1.059175in}{2.853935in}}%
\pgfpathlineto{\pgfqpoint{1.051009in}{2.859866in}}%
\pgfpathlineto{\pgfqpoint{1.042737in}{2.865686in}}%
\pgfpathlineto{\pgfqpoint{1.038946in}{2.868348in}}%
\pgfpathlineto{\pgfqpoint{1.026884in}{2.876572in}}%
\pgfpathlineto{\pgfqpoint{1.025577in}{2.877438in}}%
\pgfpathlineto{\pgfqpoint{1.014821in}{2.884560in}}%
\pgfpathlineto{\pgfqpoint{1.007716in}{2.889189in}}%
\pgfpathlineto{\pgfqpoint{1.002758in}{2.892418in}}%
\pgfpathlineto{\pgfqpoint{0.990696in}{2.900215in}}%
\pgfpathlineto{\pgfqpoint{0.989571in}{2.900940in}}%
\pgfpathlineto{\pgfqpoint{0.978633in}{2.908004in}}%
\pgfpathlineto{\pgfqpoint{0.971497in}{2.912692in}}%
\pgfpathlineto{\pgfqpoint{0.966570in}{2.915933in}}%
\pgfpathlineto{\pgfqpoint{0.954508in}{2.924104in}}%
\pgfpathlineto{\pgfqpoint{0.954024in}{2.924443in}}%
\pgfpathlineto{\pgfqpoint{0.942445in}{2.932592in}}%
\pgfpathlineto{\pgfqpoint{0.937592in}{2.936194in}}%
\pgfpathlineto{\pgfqpoint{0.930382in}{2.941563in}}%
\pgfpathlineto{\pgfqpoint{0.922329in}{2.947946in}}%
\pgfpathlineto{\pgfqpoint{0.918320in}{2.951136in}}%
\pgfpathlineto{\pgfqpoint{0.908277in}{2.959697in}}%
\pgfpathlineto{\pgfqpoint{0.906257in}{2.961428in}}%
\pgfpathlineto{\pgfqpoint{0.895387in}{2.971449in}}%
\pgfpathlineto{\pgfqpoint{0.894194in}{2.972554in}}%
\pgfpathlineto{\pgfqpoint{0.883549in}{2.983200in}}%
\pgfpathlineto{\pgfqpoint{0.882132in}{2.983200in}}%
\pgfpathlineto{\pgfqpoint{0.870069in}{2.983200in}}%
\pgfpathlineto{\pgfqpoint{0.858006in}{2.983200in}}%
\pgfpathlineto{\pgfqpoint{0.845944in}{2.983200in}}%
\pgfpathlineto{\pgfqpoint{0.834373in}{2.983200in}}%
\pgfpathlineto{\pgfqpoint{0.842867in}{2.971449in}}%
\pgfpathlineto{\pgfqpoint{0.845944in}{2.967225in}}%
\pgfpathlineto{\pgfqpoint{0.851852in}{2.959697in}}%
\pgfpathlineto{\pgfqpoint{0.858006in}{2.951919in}}%
\pgfpathlineto{\pgfqpoint{0.861406in}{2.947946in}}%
\pgfpathlineto{\pgfqpoint{0.870069in}{2.937897in}}%
\pgfpathlineto{\pgfqpoint{0.871660in}{2.936194in}}%
\pgfpathlineto{\pgfqpoint{0.882132in}{2.925067in}}%
\pgfpathlineto{\pgfqpoint{0.882769in}{2.924443in}}%
\pgfpathlineto{\pgfqpoint{0.894194in}{2.913328in}}%
\pgfpathlineto{\pgfqpoint{0.894904in}{2.912692in}}%
\pgfpathlineto{\pgfqpoint{0.906257in}{2.902567in}}%
\pgfpathlineto{\pgfqpoint{0.908231in}{2.900940in}}%
\pgfpathlineto{\pgfqpoint{0.918320in}{2.892667in}}%
\pgfpathlineto{\pgfqpoint{0.922887in}{2.889189in}}%
\pgfpathlineto{\pgfqpoint{0.930382in}{2.883504in}}%
\pgfpathlineto{\pgfqpoint{0.938928in}{2.877438in}}%
\pgfpathlineto{\pgfqpoint{0.942445in}{2.874950in}}%
\pgfpathlineto{\pgfqpoint{0.954508in}{2.866861in}}%
\pgfpathlineto{\pgfqpoint{0.956326in}{2.865686in}}%
\pgfpathlineto{\pgfqpoint{0.966570in}{2.859084in}}%
\pgfpathlineto{\pgfqpoint{0.974796in}{2.853935in}}%
\pgfpathlineto{\pgfqpoint{0.978633in}{2.851537in}}%
\pgfpathlineto{\pgfqpoint{0.990696in}{2.844076in}}%
\pgfpathlineto{\pgfqpoint{0.993743in}{2.842183in}}%
\pgfpathlineto{\pgfqpoint{1.002758in}{2.836587in}}%
\pgfpathlineto{\pgfqpoint{1.012556in}{2.830432in}}%
\pgfpathlineto{\pgfqpoint{1.014821in}{2.829009in}}%
\pgfpathlineto{\pgfqpoint{1.026884in}{2.821220in}}%
\pgfpathlineto{\pgfqpoint{1.030692in}{2.818681in}}%
\pgfpathlineto{\pgfqpoint{1.038946in}{2.813169in}}%
\pgfpathlineto{\pgfqpoint{1.047968in}{2.806929in}}%
\pgfpathlineto{\pgfqpoint{1.051009in}{2.804821in}}%
\pgfpathlineto{\pgfqpoint{1.063072in}{2.796121in}}%
\pgfpathlineto{\pgfqpoint{1.064327in}{2.795178in}}%
\pgfpathlineto{\pgfqpoint{1.075134in}{2.787033in}}%
\pgfpathlineto{\pgfqpoint{1.079746in}{2.783427in}}%
\pgfpathlineto{\pgfqpoint{1.087197in}{2.777579in}}%
\pgfpathlineto{\pgfqpoint{1.094470in}{2.771675in}}%
\pgfpathlineto{\pgfqpoint{1.099260in}{2.767770in}}%
\pgfpathlineto{\pgfqpoint{1.108606in}{2.759924in}}%
\pgfpathlineto{\pgfqpoint{1.111322in}{2.757632in}}%
\pgfpathlineto{\pgfqpoint{1.122275in}{2.748172in}}%
\pgfpathlineto{\pgfqpoint{1.123385in}{2.747208in}}%
\pgfpathlineto{\pgfqpoint{1.135448in}{2.736556in}}%
\pgfpathlineto{\pgfqpoint{1.135600in}{2.736421in}}%
\pgfpathlineto{\pgfqpoint{1.147510in}{2.725742in}}%
\pgfpathlineto{\pgfqpoint{1.148707in}{2.724670in}}%
\pgfpathlineto{\pgfqpoint{1.159573in}{2.714866in}}%
\pgfpathlineto{\pgfqpoint{1.161755in}{2.712918in}}%
\pgfpathlineto{\pgfqpoint{1.171636in}{2.704031in}}%
\pgfpathlineto{\pgfqpoint{1.174890in}{2.701167in}}%
\pgfpathlineto{\pgfqpoint{1.183698in}{2.693348in}}%
\pgfpathlineto{\pgfqpoint{1.188283in}{2.689416in}}%
\pgfpathlineto{\pgfqpoint{1.195761in}{2.682941in}}%
\pgfpathlineto{\pgfqpoint{1.202155in}{2.677664in}}%
\pgfpathlineto{\pgfqpoint{1.207824in}{2.672939in}}%
\pgfpathlineto{\pgfqpoint{1.216809in}{2.665913in}}%
\pgfpathlineto{\pgfqpoint{1.219886in}{2.663480in}}%
\pgfpathlineto{\pgfqpoint{1.231949in}{2.654694in}}%
\pgfpathlineto{\pgfqpoint{1.232760in}{2.654161in}}%
\pgfpathlineto{\pgfqpoint{1.244012in}{2.646682in}}%
\pgfpathlineto{\pgfqpoint{1.251345in}{2.642410in}}%
\pgfpathlineto{\pgfqpoint{1.256074in}{2.639620in}}%
\pgfpathlineto{\pgfqpoint{1.268137in}{2.633600in}}%
\pgfpathlineto{\pgfqpoint{1.275472in}{2.630659in}}%
\pgfpathclose%
\pgfusepath{fill}%
\end{pgfscope}%
\begin{pgfscope}%
\pgfpathrectangle{\pgfqpoint{0.423750in}{1.819814in}}{\pgfqpoint{1.194205in}{1.163386in}}%
\pgfusepath{clip}%
\pgfsetbuttcap%
\pgfsetroundjoin%
\definecolor{currentfill}{rgb}{0.949145,0.420383,0.287810}%
\pgfsetfillcolor{currentfill}%
\pgfsetlinewidth{0.000000pt}%
\definecolor{currentstroke}{rgb}{0.000000,0.000000,0.000000}%
\pgfsetstrokecolor{currentstroke}%
\pgfsetdash{}{0pt}%
\pgfpathmoveto{\pgfqpoint{0.652941in}{1.819814in}}%
\pgfpathlineto{\pgfqpoint{0.665003in}{1.819814in}}%
\pgfpathlineto{\pgfqpoint{0.677066in}{1.819814in}}%
\pgfpathlineto{\pgfqpoint{0.689129in}{1.819814in}}%
\pgfpathlineto{\pgfqpoint{0.690010in}{1.819814in}}%
\pgfpathlineto{\pgfqpoint{0.689129in}{1.821058in}}%
\pgfpathlineto{\pgfqpoint{0.682083in}{1.831565in}}%
\pgfpathlineto{\pgfqpoint{0.677066in}{1.838918in}}%
\pgfpathlineto{\pgfqpoint{0.674243in}{1.843316in}}%
\pgfpathlineto{\pgfqpoint{0.666600in}{1.855068in}}%
\pgfpathlineto{\pgfqpoint{0.665003in}{1.857499in}}%
\pgfpathlineto{\pgfqpoint{0.659279in}{1.866819in}}%
\pgfpathlineto{\pgfqpoint{0.652941in}{1.876950in}}%
\pgfpathlineto{\pgfqpoint{0.651996in}{1.878571in}}%
\pgfpathlineto{\pgfqpoint{0.645099in}{1.890322in}}%
\pgfpathlineto{\pgfqpoint{0.640878in}{1.897399in}}%
\pgfpathlineto{\pgfqpoint{0.638287in}{1.902073in}}%
\pgfpathlineto{\pgfqpoint{0.631704in}{1.913825in}}%
\pgfpathlineto{\pgfqpoint{0.628815in}{1.918919in}}%
\pgfpathlineto{\pgfqpoint{0.625315in}{1.925576in}}%
\pgfpathlineto{\pgfqpoint{0.619055in}{1.937327in}}%
\pgfpathlineto{\pgfqpoint{0.616753in}{1.941609in}}%
\pgfpathlineto{\pgfqpoint{0.613029in}{1.949079in}}%
\pgfpathlineto{\pgfqpoint{0.607092in}{1.960830in}}%
\pgfpathlineto{\pgfqpoint{0.604690in}{1.965542in}}%
\pgfpathlineto{\pgfqpoint{0.601362in}{1.972582in}}%
\pgfpathlineto{\pgfqpoint{0.595741in}{1.984333in}}%
\pgfpathlineto{\pgfqpoint{0.592627in}{1.990772in}}%
\pgfpathlineto{\pgfqpoint{0.590242in}{1.996084in}}%
\pgfpathlineto{\pgfqpoint{0.584923in}{2.007836in}}%
\pgfpathlineto{\pgfqpoint{0.580565in}{2.017328in}}%
\pgfpathlineto{\pgfqpoint{0.579600in}{2.019587in}}%
\pgfpathlineto{\pgfqpoint{0.574564in}{2.031338in}}%
\pgfpathlineto{\pgfqpoint{0.569446in}{2.043090in}}%
\pgfpathlineto{\pgfqpoint{0.568502in}{2.045249in}}%
\pgfpathlineto{\pgfqpoint{0.564595in}{2.054841in}}%
\pgfpathlineto{\pgfqpoint{0.559750in}{2.066593in}}%
\pgfpathlineto{\pgfqpoint{0.556439in}{2.074539in}}%
\pgfpathlineto{\pgfqpoint{0.554957in}{2.078344in}}%
\pgfpathlineto{\pgfqpoint{0.550362in}{2.090095in}}%
\pgfpathlineto{\pgfqpoint{0.545699in}{2.101847in}}%
\pgfpathlineto{\pgfqpoint{0.544377in}{2.105168in}}%
\pgfpathlineto{\pgfqpoint{0.541230in}{2.113598in}}%
\pgfpathlineto{\pgfqpoint{0.536802in}{2.125349in}}%
\pgfpathlineto{\pgfqpoint{0.532314in}{2.137097in}}%
\pgfpathlineto{\pgfqpoint{0.532313in}{2.137101in}}%
\pgfpathlineto{\pgfqpoint{0.528096in}{2.148852in}}%
\pgfpathlineto{\pgfqpoint{0.523824in}{2.160604in}}%
\pgfpathlineto{\pgfqpoint{0.520251in}{2.170328in}}%
\pgfpathlineto{\pgfqpoint{0.519548in}{2.172355in}}%
\pgfpathlineto{\pgfqpoint{0.515471in}{2.184106in}}%
\pgfpathlineto{\pgfqpoint{0.511343in}{2.195858in}}%
\pgfpathlineto{\pgfqpoint{0.508189in}{2.204755in}}%
\pgfpathlineto{\pgfqpoint{0.507230in}{2.207609in}}%
\pgfpathlineto{\pgfqpoint{0.503280in}{2.219360in}}%
\pgfpathlineto{\pgfqpoint{0.499282in}{2.231112in}}%
\pgfpathlineto{\pgfqpoint{0.496126in}{2.240304in}}%
\pgfpathlineto{\pgfqpoint{0.495291in}{2.242863in}}%
\pgfpathlineto{\pgfqpoint{0.491454in}{2.254615in}}%
\pgfpathlineto{\pgfqpoint{0.487572in}{2.266366in}}%
\pgfpathlineto{\pgfqpoint{0.484063in}{2.276877in}}%
\pgfpathlineto{\pgfqpoint{0.483668in}{2.278117in}}%
\pgfpathlineto{\pgfqpoint{0.479932in}{2.289869in}}%
\pgfpathlineto{\pgfqpoint{0.476152in}{2.301620in}}%
\pgfpathlineto{\pgfqpoint{0.472328in}{2.313371in}}%
\pgfpathlineto{\pgfqpoint{0.472001in}{2.314378in}}%
\pgfpathlineto{\pgfqpoint{0.468660in}{2.325123in}}%
\pgfpathlineto{\pgfqpoint{0.464969in}{2.336874in}}%
\pgfpathlineto{\pgfqpoint{0.461235in}{2.348626in}}%
\pgfpathlineto{\pgfqpoint{0.459938in}{2.352698in}}%
\pgfpathlineto{\pgfqpoint{0.457590in}{2.360377in}}%
\pgfpathlineto{\pgfqpoint{0.453975in}{2.372128in}}%
\pgfpathlineto{\pgfqpoint{0.450318in}{2.383880in}}%
\pgfpathlineto{\pgfqpoint{0.447875in}{2.391673in}}%
\pgfpathlineto{\pgfqpoint{0.446681in}{2.395631in}}%
\pgfpathlineto{\pgfqpoint{0.443128in}{2.407383in}}%
\pgfpathlineto{\pgfqpoint{0.439534in}{2.419134in}}%
\pgfpathlineto{\pgfqpoint{0.435898in}{2.430885in}}%
\pgfpathlineto{\pgfqpoint{0.435813in}{2.431160in}}%
\pgfpathlineto{\pgfqpoint{0.432391in}{2.442637in}}%
\pgfpathlineto{\pgfqpoint{0.428848in}{2.454388in}}%
\pgfpathlineto{\pgfqpoint{0.425262in}{2.466139in}}%
\pgfpathlineto{\pgfqpoint{0.423750in}{2.471076in}}%
\pgfpathlineto{\pgfqpoint{0.423750in}{2.466139in}}%
\pgfpathlineto{\pgfqpoint{0.423750in}{2.454388in}}%
\pgfpathlineto{\pgfqpoint{0.423750in}{2.442637in}}%
\pgfpathlineto{\pgfqpoint{0.423750in}{2.430885in}}%
\pgfpathlineto{\pgfqpoint{0.423750in}{2.419134in}}%
\pgfpathlineto{\pgfqpoint{0.423750in}{2.407816in}}%
\pgfpathlineto{\pgfqpoint{0.423884in}{2.407383in}}%
\pgfpathlineto{\pgfqpoint{0.427531in}{2.395631in}}%
\pgfpathlineto{\pgfqpoint{0.431137in}{2.383880in}}%
\pgfpathlineto{\pgfqpoint{0.434703in}{2.372128in}}%
\pgfpathlineto{\pgfqpoint{0.435813in}{2.368464in}}%
\pgfpathlineto{\pgfqpoint{0.438345in}{2.360377in}}%
\pgfpathlineto{\pgfqpoint{0.442000in}{2.348626in}}%
\pgfpathlineto{\pgfqpoint{0.445615in}{2.336874in}}%
\pgfpathlineto{\pgfqpoint{0.447875in}{2.329482in}}%
\pgfpathlineto{\pgfqpoint{0.449257in}{2.325123in}}%
\pgfpathlineto{\pgfqpoint{0.452973in}{2.313371in}}%
\pgfpathlineto{\pgfqpoint{0.456648in}{2.301620in}}%
\pgfpathlineto{\pgfqpoint{0.459938in}{2.291002in}}%
\pgfpathlineto{\pgfqpoint{0.460303in}{2.289869in}}%
\pgfpathlineto{\pgfqpoint{0.464093in}{2.278117in}}%
\pgfpathlineto{\pgfqpoint{0.467841in}{2.266366in}}%
\pgfpathlineto{\pgfqpoint{0.471549in}{2.254615in}}%
\pgfpathlineto{\pgfqpoint{0.472001in}{2.253183in}}%
\pgfpathlineto{\pgfqpoint{0.475399in}{2.242863in}}%
\pgfpathlineto{\pgfqpoint{0.479233in}{2.231112in}}%
\pgfpathlineto{\pgfqpoint{0.483025in}{2.219360in}}%
\pgfpathlineto{\pgfqpoint{0.484063in}{2.216139in}}%
\pgfpathlineto{\pgfqpoint{0.486939in}{2.207609in}}%
\pgfpathlineto{\pgfqpoint{0.490871in}{2.195858in}}%
\pgfpathlineto{\pgfqpoint{0.494760in}{2.184106in}}%
\pgfpathlineto{\pgfqpoint{0.496126in}{2.179969in}}%
\pgfpathlineto{\pgfqpoint{0.498764in}{2.172355in}}%
\pgfpathlineto{\pgfqpoint{0.502808in}{2.160604in}}%
\pgfpathlineto{\pgfqpoint{0.506806in}{2.148852in}}%
\pgfpathlineto{\pgfqpoint{0.508189in}{2.144779in}}%
\pgfpathlineto{\pgfqpoint{0.510932in}{2.137101in}}%
\pgfpathlineto{\pgfqpoint{0.515101in}{2.125349in}}%
\pgfpathlineto{\pgfqpoint{0.519221in}{2.113598in}}%
\pgfpathlineto{\pgfqpoint{0.520251in}{2.110656in}}%
\pgfpathlineto{\pgfqpoint{0.523509in}{2.101847in}}%
\pgfpathlineto{\pgfqpoint{0.527816in}{2.090095in}}%
\pgfpathlineto{\pgfqpoint{0.532071in}{2.078344in}}%
\pgfpathlineto{\pgfqpoint{0.532314in}{2.077672in}}%
\pgfpathlineto{\pgfqpoint{0.536567in}{2.066593in}}%
\pgfpathlineto{\pgfqpoint{0.541024in}{2.054841in}}%
\pgfpathlineto{\pgfqpoint{0.544377in}{2.045918in}}%
\pgfpathlineto{\pgfqpoint{0.545506in}{2.043090in}}%
\pgfpathlineto{\pgfqpoint{0.550187in}{2.031338in}}%
\pgfpathlineto{\pgfqpoint{0.554805in}{2.019587in}}%
\pgfpathlineto{\pgfqpoint{0.556439in}{2.015409in}}%
\pgfpathlineto{\pgfqpoint{0.559598in}{2.007836in}}%
\pgfpathlineto{\pgfqpoint{0.564453in}{1.996084in}}%
\pgfpathlineto{\pgfqpoint{0.568502in}{1.986169in}}%
\pgfpathlineto{\pgfqpoint{0.569303in}{1.984333in}}%
\pgfpathlineto{\pgfqpoint{0.574418in}{1.972582in}}%
\pgfpathlineto{\pgfqpoint{0.579456in}{1.960830in}}%
\pgfpathlineto{\pgfqpoint{0.580565in}{1.958235in}}%
\pgfpathlineto{\pgfqpoint{0.584752in}{1.949079in}}%
\pgfpathlineto{\pgfqpoint{0.590059in}{1.937327in}}%
\pgfpathlineto{\pgfqpoint{0.592627in}{1.931594in}}%
\pgfpathlineto{\pgfqpoint{0.595518in}{1.925576in}}%
\pgfpathlineto{\pgfqpoint{0.601110in}{1.913825in}}%
\pgfpathlineto{\pgfqpoint{0.604690in}{1.906215in}}%
\pgfpathlineto{\pgfqpoint{0.606781in}{1.902073in}}%
\pgfpathlineto{\pgfqpoint{0.612670in}{1.890322in}}%
\pgfpathlineto{\pgfqpoint{0.616753in}{1.882068in}}%
\pgfpathlineto{\pgfqpoint{0.618610in}{1.878571in}}%
\pgfpathlineto{\pgfqpoint{0.624804in}{1.866819in}}%
\pgfpathlineto{\pgfqpoint{0.628815in}{1.859105in}}%
\pgfpathlineto{\pgfqpoint{0.631068in}{1.855068in}}%
\pgfpathlineto{\pgfqpoint{0.637566in}{1.843316in}}%
\pgfpathlineto{\pgfqpoint{0.640878in}{1.837256in}}%
\pgfpathlineto{\pgfqpoint{0.644209in}{1.831565in}}%
\pgfpathlineto{\pgfqpoint{0.651003in}{1.819814in}}%
\pgfpathclose%
\pgfusepath{fill}%
\end{pgfscope}%
\begin{pgfscope}%
\pgfpathrectangle{\pgfqpoint{0.423750in}{1.819814in}}{\pgfqpoint{1.194205in}{1.163386in}}%
\pgfusepath{clip}%
\pgfsetbuttcap%
\pgfsetroundjoin%
\definecolor{currentfill}{rgb}{0.949145,0.420383,0.287810}%
\pgfsetfillcolor{currentfill}%
\pgfsetlinewidth{0.000000pt}%
\definecolor{currentstroke}{rgb}{0.000000,0.000000,0.000000}%
\pgfsetstrokecolor{currentstroke}%
\pgfsetdash{}{0pt}%
\pgfpathmoveto{\pgfqpoint{1.292262in}{2.559411in}}%
\pgfpathlineto{\pgfqpoint{1.304325in}{2.556896in}}%
\pgfpathlineto{\pgfqpoint{1.316388in}{2.555839in}}%
\pgfpathlineto{\pgfqpoint{1.328450in}{2.556260in}}%
\pgfpathlineto{\pgfqpoint{1.340513in}{2.558161in}}%
\pgfpathlineto{\pgfqpoint{1.347652in}{2.560150in}}%
\pgfpathlineto{\pgfqpoint{1.352576in}{2.561501in}}%
\pgfpathlineto{\pgfqpoint{1.364638in}{2.566210in}}%
\pgfpathlineto{\pgfqpoint{1.375963in}{2.571902in}}%
\pgfpathlineto{\pgfqpoint{1.376701in}{2.572267in}}%
\pgfpathlineto{\pgfqpoint{1.388764in}{2.579489in}}%
\pgfpathlineto{\pgfqpoint{1.394755in}{2.583653in}}%
\pgfpathlineto{\pgfqpoint{1.400826in}{2.587820in}}%
\pgfpathlineto{\pgfqpoint{1.410642in}{2.595405in}}%
\pgfpathlineto{\pgfqpoint{1.412889in}{2.597120in}}%
\pgfpathlineto{\pgfqpoint{1.424852in}{2.607156in}}%
\pgfpathlineto{\pgfqpoint{1.424952in}{2.607239in}}%
\pgfpathlineto{\pgfqpoint{1.437014in}{2.618010in}}%
\pgfpathlineto{\pgfqpoint{1.437964in}{2.618907in}}%
\pgfpathlineto{\pgfqpoint{1.449077in}{2.629308in}}%
\pgfpathlineto{\pgfqpoint{1.450461in}{2.630659in}}%
\pgfpathlineto{\pgfqpoint{1.461140in}{2.640989in}}%
\pgfpathlineto{\pgfqpoint{1.462567in}{2.642410in}}%
\pgfpathlineto{\pgfqpoint{1.473202in}{2.652913in}}%
\pgfpathlineto{\pgfqpoint{1.474446in}{2.654161in}}%
\pgfpathlineto{\pgfqpoint{1.485265in}{2.664947in}}%
\pgfpathlineto{\pgfqpoint{1.486228in}{2.665913in}}%
\pgfpathlineto{\pgfqpoint{1.497328in}{2.676972in}}%
\pgfpathlineto{\pgfqpoint{1.498025in}{2.677664in}}%
\pgfpathlineto{\pgfqpoint{1.509390in}{2.688879in}}%
\pgfpathlineto{\pgfqpoint{1.509941in}{2.689416in}}%
\pgfpathlineto{\pgfqpoint{1.521453in}{2.700573in}}%
\pgfpathlineto{\pgfqpoint{1.522079in}{2.701167in}}%
\pgfpathlineto{\pgfqpoint{1.533516in}{2.711975in}}%
\pgfpathlineto{\pgfqpoint{1.534542in}{2.712918in}}%
\pgfpathlineto{\pgfqpoint{1.545579in}{2.723022in}}%
\pgfpathlineto{\pgfqpoint{1.547440in}{2.724670in}}%
\pgfpathlineto{\pgfqpoint{1.557641in}{2.733668in}}%
\pgfpathlineto{\pgfqpoint{1.560884in}{2.736421in}}%
\pgfpathlineto{\pgfqpoint{1.569704in}{2.743886in}}%
\pgfpathlineto{\pgfqpoint{1.574984in}{2.748172in}}%
\pgfpathlineto{\pgfqpoint{1.581767in}{2.753666in}}%
\pgfpathlineto{\pgfqpoint{1.589835in}{2.759924in}}%
\pgfpathlineto{\pgfqpoint{1.593829in}{2.763016in}}%
\pgfpathlineto{\pgfqpoint{1.605502in}{2.771675in}}%
\pgfpathlineto{\pgfqpoint{1.605892in}{2.771964in}}%
\pgfpathlineto{\pgfqpoint{1.617955in}{2.780506in}}%
\pgfpathlineto{\pgfqpoint{1.617955in}{2.783427in}}%
\pgfpathlineto{\pgfqpoint{1.617955in}{2.795178in}}%
\pgfpathlineto{\pgfqpoint{1.617955in}{2.806929in}}%
\pgfpathlineto{\pgfqpoint{1.617955in}{2.818681in}}%
\pgfpathlineto{\pgfqpoint{1.617955in}{2.830432in}}%
\pgfpathlineto{\pgfqpoint{1.617955in}{2.840063in}}%
\pgfpathlineto{\pgfqpoint{1.605892in}{2.831296in}}%
\pgfpathlineto{\pgfqpoint{1.604745in}{2.830432in}}%
\pgfpathlineto{\pgfqpoint{1.593829in}{2.822199in}}%
\pgfpathlineto{\pgfqpoint{1.589327in}{2.818681in}}%
\pgfpathlineto{\pgfqpoint{1.581767in}{2.812764in}}%
\pgfpathlineto{\pgfqpoint{1.574578in}{2.806929in}}%
\pgfpathlineto{\pgfqpoint{1.569704in}{2.802965in}}%
\pgfpathlineto{\pgfqpoint{1.560462in}{2.795178in}}%
\pgfpathlineto{\pgfqpoint{1.557641in}{2.792794in}}%
\pgfpathlineto{\pgfqpoint{1.546908in}{2.783427in}}%
\pgfpathlineto{\pgfqpoint{1.545579in}{2.782263in}}%
\pgfpathlineto{\pgfqpoint{1.533824in}{2.771675in}}%
\pgfpathlineto{\pgfqpoint{1.533516in}{2.771396in}}%
\pgfpathlineto{\pgfqpoint{1.521453in}{2.760233in}}%
\pgfpathlineto{\pgfqpoint{1.521124in}{2.759924in}}%
\pgfpathlineto{\pgfqpoint{1.509390in}{2.748837in}}%
\pgfpathlineto{\pgfqpoint{1.508693in}{2.748172in}}%
\pgfpathlineto{\pgfqpoint{1.497328in}{2.737286in}}%
\pgfpathlineto{\pgfqpoint{1.496424in}{2.736421in}}%
\pgfpathlineto{\pgfqpoint{1.485265in}{2.725667in}}%
\pgfpathlineto{\pgfqpoint{1.484220in}{2.724670in}}%
\pgfpathlineto{\pgfqpoint{1.473202in}{2.714082in}}%
\pgfpathlineto{\pgfqpoint{1.471968in}{2.712918in}}%
\pgfpathlineto{\pgfqpoint{1.461140in}{2.702640in}}%
\pgfpathlineto{\pgfqpoint{1.459539in}{2.701167in}}%
\pgfpathlineto{\pgfqpoint{1.449077in}{2.691463in}}%
\pgfpathlineto{\pgfqpoint{1.446770in}{2.689416in}}%
\pgfpathlineto{\pgfqpoint{1.437014in}{2.680680in}}%
\pgfpathlineto{\pgfqpoint{1.433443in}{2.677664in}}%
\pgfpathlineto{\pgfqpoint{1.424952in}{2.670424in}}%
\pgfpathlineto{\pgfqpoint{1.419247in}{2.665913in}}%
\pgfpathlineto{\pgfqpoint{1.412889in}{2.660832in}}%
\pgfpathlineto{\pgfqpoint{1.403707in}{2.654161in}}%
\pgfpathlineto{\pgfqpoint{1.400826in}{2.652045in}}%
\pgfpathlineto{\pgfqpoint{1.388764in}{2.644172in}}%
\pgfpathlineto{\pgfqpoint{1.385632in}{2.642410in}}%
\pgfpathlineto{\pgfqpoint{1.376701in}{2.637324in}}%
\pgfpathlineto{\pgfqpoint{1.364638in}{2.631649in}}%
\pgfpathlineto{\pgfqpoint{1.361935in}{2.630659in}}%
\pgfpathlineto{\pgfqpoint{1.352576in}{2.627181in}}%
\pgfpathlineto{\pgfqpoint{1.340513in}{2.624026in}}%
\pgfpathlineto{\pgfqpoint{1.328450in}{2.622234in}}%
\pgfpathlineto{\pgfqpoint{1.316388in}{2.621819in}}%
\pgfpathlineto{\pgfqpoint{1.304325in}{2.622779in}}%
\pgfpathlineto{\pgfqpoint{1.292262in}{2.625096in}}%
\pgfpathlineto{\pgfqpoint{1.280200in}{2.628736in}}%
\pgfpathlineto{\pgfqpoint{1.275472in}{2.630659in}}%
\pgfpathlineto{\pgfqpoint{1.268137in}{2.633600in}}%
\pgfpathlineto{\pgfqpoint{1.256074in}{2.639620in}}%
\pgfpathlineto{\pgfqpoint{1.251345in}{2.642410in}}%
\pgfpathlineto{\pgfqpoint{1.244012in}{2.646682in}}%
\pgfpathlineto{\pgfqpoint{1.232760in}{2.654161in}}%
\pgfpathlineto{\pgfqpoint{1.231949in}{2.654694in}}%
\pgfpathlineto{\pgfqpoint{1.219886in}{2.663480in}}%
\pgfpathlineto{\pgfqpoint{1.216809in}{2.665913in}}%
\pgfpathlineto{\pgfqpoint{1.207824in}{2.672939in}}%
\pgfpathlineto{\pgfqpoint{1.202155in}{2.677664in}}%
\pgfpathlineto{\pgfqpoint{1.195761in}{2.682941in}}%
\pgfpathlineto{\pgfqpoint{1.188283in}{2.689416in}}%
\pgfpathlineto{\pgfqpoint{1.183698in}{2.693348in}}%
\pgfpathlineto{\pgfqpoint{1.174890in}{2.701167in}}%
\pgfpathlineto{\pgfqpoint{1.171636in}{2.704031in}}%
\pgfpathlineto{\pgfqpoint{1.161755in}{2.712918in}}%
\pgfpathlineto{\pgfqpoint{1.159573in}{2.714866in}}%
\pgfpathlineto{\pgfqpoint{1.148707in}{2.724670in}}%
\pgfpathlineto{\pgfqpoint{1.147510in}{2.725742in}}%
\pgfpathlineto{\pgfqpoint{1.135600in}{2.736421in}}%
\pgfpathlineto{\pgfqpoint{1.135448in}{2.736556in}}%
\pgfpathlineto{\pgfqpoint{1.123385in}{2.747208in}}%
\pgfpathlineto{\pgfqpoint{1.122275in}{2.748172in}}%
\pgfpathlineto{\pgfqpoint{1.111322in}{2.757632in}}%
\pgfpathlineto{\pgfqpoint{1.108606in}{2.759924in}}%
\pgfpathlineto{\pgfqpoint{1.099260in}{2.767770in}}%
\pgfpathlineto{\pgfqpoint{1.094470in}{2.771675in}}%
\pgfpathlineto{\pgfqpoint{1.087197in}{2.777579in}}%
\pgfpathlineto{\pgfqpoint{1.079746in}{2.783427in}}%
\pgfpathlineto{\pgfqpoint{1.075134in}{2.787033in}}%
\pgfpathlineto{\pgfqpoint{1.064327in}{2.795178in}}%
\pgfpathlineto{\pgfqpoint{1.063072in}{2.796121in}}%
\pgfpathlineto{\pgfqpoint{1.051009in}{2.804821in}}%
\pgfpathlineto{\pgfqpoint{1.047968in}{2.806929in}}%
\pgfpathlineto{\pgfqpoint{1.038946in}{2.813169in}}%
\pgfpathlineto{\pgfqpoint{1.030692in}{2.818681in}}%
\pgfpathlineto{\pgfqpoint{1.026884in}{2.821220in}}%
\pgfpathlineto{\pgfqpoint{1.014821in}{2.829009in}}%
\pgfpathlineto{\pgfqpoint{1.012556in}{2.830432in}}%
\pgfpathlineto{\pgfqpoint{1.002758in}{2.836587in}}%
\pgfpathlineto{\pgfqpoint{0.993743in}{2.842183in}}%
\pgfpathlineto{\pgfqpoint{0.990696in}{2.844076in}}%
\pgfpathlineto{\pgfqpoint{0.978633in}{2.851537in}}%
\pgfpathlineto{\pgfqpoint{0.974796in}{2.853935in}}%
\pgfpathlineto{\pgfqpoint{0.966570in}{2.859084in}}%
\pgfpathlineto{\pgfqpoint{0.956326in}{2.865686in}}%
\pgfpathlineto{\pgfqpoint{0.954508in}{2.866861in}}%
\pgfpathlineto{\pgfqpoint{0.942445in}{2.874950in}}%
\pgfpathlineto{\pgfqpoint{0.938928in}{2.877438in}}%
\pgfpathlineto{\pgfqpoint{0.930382in}{2.883504in}}%
\pgfpathlineto{\pgfqpoint{0.922887in}{2.889189in}}%
\pgfpathlineto{\pgfqpoint{0.918320in}{2.892667in}}%
\pgfpathlineto{\pgfqpoint{0.908231in}{2.900940in}}%
\pgfpathlineto{\pgfqpoint{0.906257in}{2.902567in}}%
\pgfpathlineto{\pgfqpoint{0.894904in}{2.912692in}}%
\pgfpathlineto{\pgfqpoint{0.894194in}{2.913328in}}%
\pgfpathlineto{\pgfqpoint{0.882769in}{2.924443in}}%
\pgfpathlineto{\pgfqpoint{0.882132in}{2.925067in}}%
\pgfpathlineto{\pgfqpoint{0.871660in}{2.936194in}}%
\pgfpathlineto{\pgfqpoint{0.870069in}{2.937897in}}%
\pgfpathlineto{\pgfqpoint{0.861406in}{2.947946in}}%
\pgfpathlineto{\pgfqpoint{0.858006in}{2.951919in}}%
\pgfpathlineto{\pgfqpoint{0.851852in}{2.959697in}}%
\pgfpathlineto{\pgfqpoint{0.845944in}{2.967225in}}%
\pgfpathlineto{\pgfqpoint{0.842867in}{2.971449in}}%
\pgfpathlineto{\pgfqpoint{0.834373in}{2.983200in}}%
\pgfpathlineto{\pgfqpoint{0.833881in}{2.983200in}}%
\pgfpathlineto{\pgfqpoint{0.821818in}{2.983200in}}%
\pgfpathlineto{\pgfqpoint{0.809756in}{2.983200in}}%
\pgfpathlineto{\pgfqpoint{0.797693in}{2.983200in}}%
\pgfpathlineto{\pgfqpoint{0.793909in}{2.983200in}}%
\pgfpathlineto{\pgfqpoint{0.797693in}{2.976235in}}%
\pgfpathlineto{\pgfqpoint{0.800441in}{2.971449in}}%
\pgfpathlineto{\pgfqpoint{0.807223in}{2.959697in}}%
\pgfpathlineto{\pgfqpoint{0.809756in}{2.955332in}}%
\pgfpathlineto{\pgfqpoint{0.814314in}{2.947946in}}%
\pgfpathlineto{\pgfqpoint{0.821643in}{2.936194in}}%
\pgfpathlineto{\pgfqpoint{0.821818in}{2.935913in}}%
\pgfpathlineto{\pgfqpoint{0.829481in}{2.924443in}}%
\pgfpathlineto{\pgfqpoint{0.833881in}{2.917925in}}%
\pgfpathlineto{\pgfqpoint{0.837686in}{2.912692in}}%
\pgfpathlineto{\pgfqpoint{0.845944in}{2.901445in}}%
\pgfpathlineto{\pgfqpoint{0.846344in}{2.900940in}}%
\pgfpathlineto{\pgfqpoint{0.855721in}{2.889189in}}%
\pgfpathlineto{\pgfqpoint{0.858006in}{2.886349in}}%
\pgfpathlineto{\pgfqpoint{0.865811in}{2.877438in}}%
\pgfpathlineto{\pgfqpoint{0.870069in}{2.872616in}}%
\pgfpathlineto{\pgfqpoint{0.876754in}{2.865686in}}%
\pgfpathlineto{\pgfqpoint{0.882132in}{2.860152in}}%
\pgfpathlineto{\pgfqpoint{0.888744in}{2.853935in}}%
\pgfpathlineto{\pgfqpoint{0.894194in}{2.848844in}}%
\pgfpathlineto{\pgfqpoint{0.901997in}{2.842183in}}%
\pgfpathlineto{\pgfqpoint{0.906257in}{2.838569in}}%
\pgfpathlineto{\pgfqpoint{0.916720in}{2.830432in}}%
\pgfpathlineto{\pgfqpoint{0.918320in}{2.829195in}}%
\pgfpathlineto{\pgfqpoint{0.930382in}{2.820553in}}%
\pgfpathlineto{\pgfqpoint{0.933181in}{2.818681in}}%
\pgfpathlineto{\pgfqpoint{0.942445in}{2.812506in}}%
\pgfpathlineto{\pgfqpoint{0.951326in}{2.806929in}}%
\pgfpathlineto{\pgfqpoint{0.954508in}{2.804937in}}%
\pgfpathlineto{\pgfqpoint{0.966570in}{2.797669in}}%
\pgfpathlineto{\pgfqpoint{0.970802in}{2.795178in}}%
\pgfpathlineto{\pgfqpoint{0.978633in}{2.790575in}}%
\pgfpathlineto{\pgfqpoint{0.990696in}{2.783563in}}%
\pgfpathlineto{\pgfqpoint{0.990926in}{2.783427in}}%
\pgfpathlineto{\pgfqpoint{1.002758in}{2.776444in}}%
\pgfpathlineto{\pgfqpoint{1.010691in}{2.771675in}}%
\pgfpathlineto{\pgfqpoint{1.014821in}{2.769192in}}%
\pgfpathlineto{\pgfqpoint{1.026884in}{2.761686in}}%
\pgfpathlineto{\pgfqpoint{1.029597in}{2.759924in}}%
\pgfpathlineto{\pgfqpoint{1.038946in}{2.753843in}}%
\pgfpathlineto{\pgfqpoint{1.047293in}{2.748172in}}%
\pgfpathlineto{\pgfqpoint{1.051009in}{2.745642in}}%
\pgfpathlineto{\pgfqpoint{1.063072in}{2.737030in}}%
\pgfpathlineto{\pgfqpoint{1.063883in}{2.736421in}}%
\pgfpathlineto{\pgfqpoint{1.075134in}{2.727955in}}%
\pgfpathlineto{\pgfqpoint{1.079314in}{2.724670in}}%
\pgfpathlineto{\pgfqpoint{1.087197in}{2.718449in}}%
\pgfpathlineto{\pgfqpoint{1.093930in}{2.712918in}}%
\pgfpathlineto{\pgfqpoint{1.099260in}{2.708521in}}%
\pgfpathlineto{\pgfqpoint{1.107866in}{2.701167in}}%
\pgfpathlineto{\pgfqpoint{1.111322in}{2.698198in}}%
\pgfpathlineto{\pgfqpoint{1.121261in}{2.689416in}}%
\pgfpathlineto{\pgfqpoint{1.123385in}{2.687527in}}%
\pgfpathlineto{\pgfqpoint{1.134250in}{2.677664in}}%
\pgfpathlineto{\pgfqpoint{1.135448in}{2.676570in}}%
\pgfpathlineto{\pgfqpoint{1.146967in}{2.665913in}}%
\pgfpathlineto{\pgfqpoint{1.147510in}{2.665406in}}%
\pgfpathlineto{\pgfqpoint{1.159541in}{2.654161in}}%
\pgfpathlineto{\pgfqpoint{1.159573in}{2.654131in}}%
\pgfpathlineto{\pgfqpoint{1.171636in}{2.642847in}}%
\pgfpathlineto{\pgfqpoint{1.172112in}{2.642410in}}%
\pgfpathlineto{\pgfqpoint{1.183698in}{2.631679in}}%
\pgfpathlineto{\pgfqpoint{1.184836in}{2.630659in}}%
\pgfpathlineto{\pgfqpoint{1.195761in}{2.620760in}}%
\pgfpathlineto{\pgfqpoint{1.197903in}{2.618907in}}%
\pgfpathlineto{\pgfqpoint{1.207824in}{2.610229in}}%
\pgfpathlineto{\pgfqpoint{1.211562in}{2.607156in}}%
\pgfpathlineto{\pgfqpoint{1.219886in}{2.600232in}}%
\pgfpathlineto{\pgfqpoint{1.226178in}{2.595405in}}%
\pgfpathlineto{\pgfqpoint{1.231949in}{2.590921in}}%
\pgfpathlineto{\pgfqpoint{1.242317in}{2.583653in}}%
\pgfpathlineto{\pgfqpoint{1.244012in}{2.582449in}}%
\pgfpathlineto{\pgfqpoint{1.256074in}{2.574919in}}%
\pgfpathlineto{\pgfqpoint{1.261785in}{2.571902in}}%
\pgfpathlineto{\pgfqpoint{1.268137in}{2.568496in}}%
\pgfpathlineto{\pgfqpoint{1.280200in}{2.563289in}}%
\pgfpathlineto{\pgfqpoint{1.289992in}{2.560150in}}%
\pgfpathclose%
\pgfusepath{fill}%
\end{pgfscope}%
\begin{pgfscope}%
\pgfpathrectangle{\pgfqpoint{0.423750in}{1.819814in}}{\pgfqpoint{1.194205in}{1.163386in}}%
\pgfusepath{clip}%
\pgfsetbuttcap%
\pgfsetroundjoin%
\definecolor{currentfill}{rgb}{0.957344,0.505732,0.351309}%
\pgfsetfillcolor{currentfill}%
\pgfsetlinewidth{0.000000pt}%
\definecolor{currentstroke}{rgb}{0.000000,0.000000,0.000000}%
\pgfsetstrokecolor{currentstroke}%
\pgfsetdash{}{0pt}%
\pgfpathmoveto{\pgfqpoint{0.689129in}{1.821058in}}%
\pgfpathlineto{\pgfqpoint{0.690010in}{1.819814in}}%
\pgfpathlineto{\pgfqpoint{0.701191in}{1.819814in}}%
\pgfpathlineto{\pgfqpoint{0.713254in}{1.819814in}}%
\pgfpathlineto{\pgfqpoint{0.725317in}{1.819814in}}%
\pgfpathlineto{\pgfqpoint{0.737379in}{1.819814in}}%
\pgfpathlineto{\pgfqpoint{0.741475in}{1.819814in}}%
\pgfpathlineto{\pgfqpoint{0.737379in}{1.825063in}}%
\pgfpathlineto{\pgfqpoint{0.732401in}{1.831565in}}%
\pgfpathlineto{\pgfqpoint{0.725317in}{1.840660in}}%
\pgfpathlineto{\pgfqpoint{0.723308in}{1.843316in}}%
\pgfpathlineto{\pgfqpoint{0.714294in}{1.855068in}}%
\pgfpathlineto{\pgfqpoint{0.713254in}{1.856408in}}%
\pgfpathlineto{\pgfqpoint{0.705500in}{1.866819in}}%
\pgfpathlineto{\pgfqpoint{0.701191in}{1.872499in}}%
\pgfpathlineto{\pgfqpoint{0.696815in}{1.878571in}}%
\pgfpathlineto{\pgfqpoint{0.689129in}{1.889036in}}%
\pgfpathlineto{\pgfqpoint{0.688239in}{1.890322in}}%
\pgfpathlineto{\pgfqpoint{0.680026in}{1.902073in}}%
\pgfpathlineto{\pgfqpoint{0.677066in}{1.906239in}}%
\pgfpathlineto{\pgfqpoint{0.672030in}{1.913825in}}%
\pgfpathlineto{\pgfqpoint{0.665003in}{1.924199in}}%
\pgfpathlineto{\pgfqpoint{0.664137in}{1.925576in}}%
\pgfpathlineto{\pgfqpoint{0.656677in}{1.937327in}}%
\pgfpathlineto{\pgfqpoint{0.652941in}{1.943109in}}%
\pgfpathlineto{\pgfqpoint{0.649372in}{1.949079in}}%
\pgfpathlineto{\pgfqpoint{0.642232in}{1.960830in}}%
\pgfpathlineto{\pgfqpoint{0.640878in}{1.963037in}}%
\pgfpathlineto{\pgfqpoint{0.635479in}{1.972582in}}%
\pgfpathlineto{\pgfqpoint{0.628815in}{1.984113in}}%
\pgfpathlineto{\pgfqpoint{0.628698in}{1.984333in}}%
\pgfpathlineto{\pgfqpoint{0.622418in}{1.996084in}}%
\pgfpathlineto{\pgfqpoint{0.616753in}{2.006458in}}%
\pgfpathlineto{\pgfqpoint{0.616060in}{2.007836in}}%
\pgfpathlineto{\pgfqpoint{0.610120in}{2.019587in}}%
\pgfpathlineto{\pgfqpoint{0.604690in}{2.030115in}}%
\pgfpathlineto{\pgfqpoint{0.604108in}{2.031338in}}%
\pgfpathlineto{\pgfqpoint{0.598501in}{2.043090in}}%
\pgfpathlineto{\pgfqpoint{0.592776in}{2.054841in}}%
\pgfpathlineto{\pgfqpoint{0.592627in}{2.055146in}}%
\pgfpathlineto{\pgfqpoint{0.587471in}{2.066593in}}%
\pgfpathlineto{\pgfqpoint{0.582078in}{2.078344in}}%
\pgfpathlineto{\pgfqpoint{0.580565in}{2.081622in}}%
\pgfpathlineto{\pgfqpoint{0.576946in}{2.090095in}}%
\pgfpathlineto{\pgfqpoint{0.571863in}{2.101847in}}%
\pgfpathlineto{\pgfqpoint{0.568502in}{2.109526in}}%
\pgfpathlineto{\pgfqpoint{0.566849in}{2.113598in}}%
\pgfpathlineto{\pgfqpoint{0.562051in}{2.125349in}}%
\pgfpathlineto{\pgfqpoint{0.557174in}{2.137101in}}%
\pgfpathlineto{\pgfqpoint{0.556439in}{2.138865in}}%
\pgfpathlineto{\pgfqpoint{0.552574in}{2.148852in}}%
\pgfpathlineto{\pgfqpoint{0.547967in}{2.160604in}}%
\pgfpathlineto{\pgfqpoint{0.544377in}{2.169653in}}%
\pgfpathlineto{\pgfqpoint{0.543376in}{2.172355in}}%
\pgfpathlineto{\pgfqpoint{0.539013in}{2.184106in}}%
\pgfpathlineto{\pgfqpoint{0.534586in}{2.195858in}}%
\pgfpathlineto{\pgfqpoint{0.532314in}{2.201845in}}%
\pgfpathlineto{\pgfqpoint{0.530265in}{2.207609in}}%
\pgfpathlineto{\pgfqpoint{0.526064in}{2.219360in}}%
\pgfpathlineto{\pgfqpoint{0.521804in}{2.231112in}}%
\pgfpathlineto{\pgfqpoint{0.520251in}{2.235378in}}%
\pgfpathlineto{\pgfqpoint{0.517690in}{2.242863in}}%
\pgfpathlineto{\pgfqpoint{0.513638in}{2.254615in}}%
\pgfpathlineto{\pgfqpoint{0.509531in}{2.266366in}}%
\pgfpathlineto{\pgfqpoint{0.508189in}{2.270195in}}%
\pgfpathlineto{\pgfqpoint{0.505568in}{2.278117in}}%
\pgfpathlineto{\pgfqpoint{0.501651in}{2.289869in}}%
\pgfpathlineto{\pgfqpoint{0.497683in}{2.301620in}}%
\pgfpathlineto{\pgfqpoint{0.496126in}{2.306214in}}%
\pgfpathlineto{\pgfqpoint{0.493828in}{2.313371in}}%
\pgfpathlineto{\pgfqpoint{0.490032in}{2.325123in}}%
\pgfpathlineto{\pgfqpoint{0.486187in}{2.336874in}}%
\pgfpathlineto{\pgfqpoint{0.484063in}{2.343329in}}%
\pgfpathlineto{\pgfqpoint{0.482406in}{2.348626in}}%
\pgfpathlineto{\pgfqpoint{0.478716in}{2.360377in}}%
\pgfpathlineto{\pgfqpoint{0.474980in}{2.372128in}}%
\pgfpathlineto{\pgfqpoint{0.472001in}{2.381414in}}%
\pgfpathlineto{\pgfqpoint{0.471245in}{2.383880in}}%
\pgfpathlineto{\pgfqpoint{0.467647in}{2.395631in}}%
\pgfpathlineto{\pgfqpoint{0.464006in}{2.407383in}}%
\pgfpathlineto{\pgfqpoint{0.460319in}{2.419134in}}%
\pgfpathlineto{\pgfqpoint{0.459938in}{2.420349in}}%
\pgfpathlineto{\pgfqpoint{0.456777in}{2.430885in}}%
\pgfpathlineto{\pgfqpoint{0.453215in}{2.442637in}}%
\pgfpathlineto{\pgfqpoint{0.449610in}{2.454388in}}%
\pgfpathlineto{\pgfqpoint{0.447875in}{2.460016in}}%
\pgfpathlineto{\pgfqpoint{0.446062in}{2.466139in}}%
\pgfpathlineto{\pgfqpoint{0.442567in}{2.477891in}}%
\pgfpathlineto{\pgfqpoint{0.439028in}{2.489642in}}%
\pgfpathlineto{\pgfqpoint{0.435813in}{2.500203in}}%
\pgfpathlineto{\pgfqpoint{0.435463in}{2.501394in}}%
\pgfpathlineto{\pgfqpoint{0.432020in}{2.513145in}}%
\pgfpathlineto{\pgfqpoint{0.428535in}{2.524896in}}%
\pgfpathlineto{\pgfqpoint{0.425005in}{2.536648in}}%
\pgfpathlineto{\pgfqpoint{0.423750in}{2.540816in}}%
\pgfpathlineto{\pgfqpoint{0.423750in}{2.536648in}}%
\pgfpathlineto{\pgfqpoint{0.423750in}{2.524896in}}%
\pgfpathlineto{\pgfqpoint{0.423750in}{2.513145in}}%
\pgfpathlineto{\pgfqpoint{0.423750in}{2.501394in}}%
\pgfpathlineto{\pgfqpoint{0.423750in}{2.489642in}}%
\pgfpathlineto{\pgfqpoint{0.423750in}{2.477891in}}%
\pgfpathlineto{\pgfqpoint{0.423750in}{2.471076in}}%
\pgfpathlineto{\pgfqpoint{0.425262in}{2.466139in}}%
\pgfpathlineto{\pgfqpoint{0.428848in}{2.454388in}}%
\pgfpathlineto{\pgfqpoint{0.432391in}{2.442637in}}%
\pgfpathlineto{\pgfqpoint{0.435813in}{2.431160in}}%
\pgfpathlineto{\pgfqpoint{0.435898in}{2.430885in}}%
\pgfpathlineto{\pgfqpoint{0.439534in}{2.419134in}}%
\pgfpathlineto{\pgfqpoint{0.443128in}{2.407383in}}%
\pgfpathlineto{\pgfqpoint{0.446681in}{2.395631in}}%
\pgfpathlineto{\pgfqpoint{0.447875in}{2.391673in}}%
\pgfpathlineto{\pgfqpoint{0.450318in}{2.383880in}}%
\pgfpathlineto{\pgfqpoint{0.453975in}{2.372128in}}%
\pgfpathlineto{\pgfqpoint{0.457590in}{2.360377in}}%
\pgfpathlineto{\pgfqpoint{0.459938in}{2.352698in}}%
\pgfpathlineto{\pgfqpoint{0.461235in}{2.348626in}}%
\pgfpathlineto{\pgfqpoint{0.464969in}{2.336874in}}%
\pgfpathlineto{\pgfqpoint{0.468660in}{2.325123in}}%
\pgfpathlineto{\pgfqpoint{0.472001in}{2.314378in}}%
\pgfpathlineto{\pgfqpoint{0.472328in}{2.313371in}}%
\pgfpathlineto{\pgfqpoint{0.476152in}{2.301620in}}%
\pgfpathlineto{\pgfqpoint{0.479932in}{2.289869in}}%
\pgfpathlineto{\pgfqpoint{0.483668in}{2.278117in}}%
\pgfpathlineto{\pgfqpoint{0.484063in}{2.276877in}}%
\pgfpathlineto{\pgfqpoint{0.487572in}{2.266366in}}%
\pgfpathlineto{\pgfqpoint{0.491454in}{2.254615in}}%
\pgfpathlineto{\pgfqpoint{0.495291in}{2.242863in}}%
\pgfpathlineto{\pgfqpoint{0.496126in}{2.240304in}}%
\pgfpathlineto{\pgfqpoint{0.499282in}{2.231112in}}%
\pgfpathlineto{\pgfqpoint{0.503280in}{2.219360in}}%
\pgfpathlineto{\pgfqpoint{0.507230in}{2.207609in}}%
\pgfpathlineto{\pgfqpoint{0.508189in}{2.204755in}}%
\pgfpathlineto{\pgfqpoint{0.511343in}{2.195858in}}%
\pgfpathlineto{\pgfqpoint{0.515471in}{2.184106in}}%
\pgfpathlineto{\pgfqpoint{0.519548in}{2.172355in}}%
\pgfpathlineto{\pgfqpoint{0.520251in}{2.170328in}}%
\pgfpathlineto{\pgfqpoint{0.523824in}{2.160604in}}%
\pgfpathlineto{\pgfqpoint{0.528096in}{2.148852in}}%
\pgfpathlineto{\pgfqpoint{0.532313in}{2.137101in}}%
\pgfpathlineto{\pgfqpoint{0.532314in}{2.137097in}}%
\pgfpathlineto{\pgfqpoint{0.536802in}{2.125349in}}%
\pgfpathlineto{\pgfqpoint{0.541230in}{2.113598in}}%
\pgfpathlineto{\pgfqpoint{0.544377in}{2.105168in}}%
\pgfpathlineto{\pgfqpoint{0.545699in}{2.101847in}}%
\pgfpathlineto{\pgfqpoint{0.550362in}{2.090095in}}%
\pgfpathlineto{\pgfqpoint{0.554957in}{2.078344in}}%
\pgfpathlineto{\pgfqpoint{0.556439in}{2.074539in}}%
\pgfpathlineto{\pgfqpoint{0.559750in}{2.066593in}}%
\pgfpathlineto{\pgfqpoint{0.564595in}{2.054841in}}%
\pgfpathlineto{\pgfqpoint{0.568502in}{2.045249in}}%
\pgfpathlineto{\pgfqpoint{0.569446in}{2.043090in}}%
\pgfpathlineto{\pgfqpoint{0.574564in}{2.031338in}}%
\pgfpathlineto{\pgfqpoint{0.579600in}{2.019587in}}%
\pgfpathlineto{\pgfqpoint{0.580565in}{2.017328in}}%
\pgfpathlineto{\pgfqpoint{0.584923in}{2.007836in}}%
\pgfpathlineto{\pgfqpoint{0.590242in}{1.996084in}}%
\pgfpathlineto{\pgfqpoint{0.592627in}{1.990772in}}%
\pgfpathlineto{\pgfqpoint{0.595741in}{1.984333in}}%
\pgfpathlineto{\pgfqpoint{0.601362in}{1.972582in}}%
\pgfpathlineto{\pgfqpoint{0.604690in}{1.965542in}}%
\pgfpathlineto{\pgfqpoint{0.607092in}{1.960830in}}%
\pgfpathlineto{\pgfqpoint{0.613029in}{1.949079in}}%
\pgfpathlineto{\pgfqpoint{0.616753in}{1.941609in}}%
\pgfpathlineto{\pgfqpoint{0.619055in}{1.937327in}}%
\pgfpathlineto{\pgfqpoint{0.625315in}{1.925576in}}%
\pgfpathlineto{\pgfqpoint{0.628815in}{1.918919in}}%
\pgfpathlineto{\pgfqpoint{0.631704in}{1.913825in}}%
\pgfpathlineto{\pgfqpoint{0.638287in}{1.902073in}}%
\pgfpathlineto{\pgfqpoint{0.640878in}{1.897399in}}%
\pgfpathlineto{\pgfqpoint{0.645099in}{1.890322in}}%
\pgfpathlineto{\pgfqpoint{0.651996in}{1.878571in}}%
\pgfpathlineto{\pgfqpoint{0.652941in}{1.876950in}}%
\pgfpathlineto{\pgfqpoint{0.659279in}{1.866819in}}%
\pgfpathlineto{\pgfqpoint{0.665003in}{1.857499in}}%
\pgfpathlineto{\pgfqpoint{0.666600in}{1.855068in}}%
\pgfpathlineto{\pgfqpoint{0.674243in}{1.843316in}}%
\pgfpathlineto{\pgfqpoint{0.677066in}{1.838918in}}%
\pgfpathlineto{\pgfqpoint{0.682083in}{1.831565in}}%
\pgfpathclose%
\pgfusepath{fill}%
\end{pgfscope}%
\begin{pgfscope}%
\pgfpathrectangle{\pgfqpoint{0.423750in}{1.819814in}}{\pgfqpoint{1.194205in}{1.163386in}}%
\pgfusepath{clip}%
\pgfsetbuttcap%
\pgfsetroundjoin%
\definecolor{currentfill}{rgb}{0.957344,0.505732,0.351309}%
\pgfsetfillcolor{currentfill}%
\pgfsetlinewidth{0.000000pt}%
\definecolor{currentstroke}{rgb}{0.000000,0.000000,0.000000}%
\pgfsetstrokecolor{currentstroke}%
\pgfsetdash{}{0pt}%
\pgfpathmoveto{\pgfqpoint{1.292262in}{2.486303in}}%
\pgfpathlineto{\pgfqpoint{1.304325in}{2.483549in}}%
\pgfpathlineto{\pgfqpoint{1.316388in}{2.482372in}}%
\pgfpathlineto{\pgfqpoint{1.328450in}{2.482795in}}%
\pgfpathlineto{\pgfqpoint{1.340513in}{2.484820in}}%
\pgfpathlineto{\pgfqpoint{1.352576in}{2.488431in}}%
\pgfpathlineto{\pgfqpoint{1.355415in}{2.489642in}}%
\pgfpathlineto{\pgfqpoint{1.364638in}{2.493510in}}%
\pgfpathlineto{\pgfqpoint{1.376701in}{2.500028in}}%
\pgfpathlineto{\pgfqpoint{1.378791in}{2.501394in}}%
\pgfpathlineto{\pgfqpoint{1.388764in}{2.507808in}}%
\pgfpathlineto{\pgfqpoint{1.395911in}{2.513145in}}%
\pgfpathlineto{\pgfqpoint{1.400826in}{2.516764in}}%
\pgfpathlineto{\pgfqpoint{1.410646in}{2.524896in}}%
\pgfpathlineto{\pgfqpoint{1.412889in}{2.526729in}}%
\pgfpathlineto{\pgfqpoint{1.423948in}{2.536648in}}%
\pgfpathlineto{\pgfqpoint{1.424952in}{2.537537in}}%
\pgfpathlineto{\pgfqpoint{1.436355in}{2.548399in}}%
\pgfpathlineto{\pgfqpoint{1.437014in}{2.549020in}}%
\pgfpathlineto{\pgfqpoint{1.448199in}{2.560150in}}%
\pgfpathlineto{\pgfqpoint{1.449077in}{2.561015in}}%
\pgfpathlineto{\pgfqpoint{1.459698in}{2.571902in}}%
\pgfpathlineto{\pgfqpoint{1.461140in}{2.573365in}}%
\pgfpathlineto{\pgfqpoint{1.471010in}{2.583653in}}%
\pgfpathlineto{\pgfqpoint{1.473202in}{2.585918in}}%
\pgfpathlineto{\pgfqpoint{1.482255in}{2.595405in}}%
\pgfpathlineto{\pgfqpoint{1.485265in}{2.598534in}}%
\pgfpathlineto{\pgfqpoint{1.493533in}{2.607156in}}%
\pgfpathlineto{\pgfqpoint{1.497328in}{2.611083in}}%
\pgfpathlineto{\pgfqpoint{1.504940in}{2.618907in}}%
\pgfpathlineto{\pgfqpoint{1.509390in}{2.623450in}}%
\pgfpathlineto{\pgfqpoint{1.516568in}{2.630659in}}%
\pgfpathlineto{\pgfqpoint{1.521453in}{2.635535in}}%
\pgfpathlineto{\pgfqpoint{1.528516in}{2.642410in}}%
\pgfpathlineto{\pgfqpoint{1.533516in}{2.647252in}}%
\pgfpathlineto{\pgfqpoint{1.540890in}{2.654161in}}%
\pgfpathlineto{\pgfqpoint{1.545579in}{2.658535in}}%
\pgfpathlineto{\pgfqpoint{1.553808in}{2.665913in}}%
\pgfpathlineto{\pgfqpoint{1.557641in}{2.669336in}}%
\pgfpathlineto{\pgfqpoint{1.567397in}{2.677664in}}%
\pgfpathlineto{\pgfqpoint{1.569704in}{2.679627in}}%
\pgfpathlineto{\pgfqpoint{1.581767in}{2.689397in}}%
\pgfpathlineto{\pgfqpoint{1.581791in}{2.689416in}}%
\pgfpathlineto{\pgfqpoint{1.593829in}{2.698613in}}%
\pgfpathlineto{\pgfqpoint{1.597357in}{2.701167in}}%
\pgfpathlineto{\pgfqpoint{1.605892in}{2.707340in}}%
\pgfpathlineto{\pgfqpoint{1.614013in}{2.712918in}}%
\pgfpathlineto{\pgfqpoint{1.617955in}{2.715625in}}%
\pgfpathlineto{\pgfqpoint{1.617955in}{2.724670in}}%
\pgfpathlineto{\pgfqpoint{1.617955in}{2.736421in}}%
\pgfpathlineto{\pgfqpoint{1.617955in}{2.748172in}}%
\pgfpathlineto{\pgfqpoint{1.617955in}{2.759924in}}%
\pgfpathlineto{\pgfqpoint{1.617955in}{2.771675in}}%
\pgfpathlineto{\pgfqpoint{1.617955in}{2.780506in}}%
\pgfpathlineto{\pgfqpoint{1.605892in}{2.771964in}}%
\pgfpathlineto{\pgfqpoint{1.605502in}{2.771675in}}%
\pgfpathlineto{\pgfqpoint{1.593829in}{2.763016in}}%
\pgfpathlineto{\pgfqpoint{1.589835in}{2.759924in}}%
\pgfpathlineto{\pgfqpoint{1.581767in}{2.753666in}}%
\pgfpathlineto{\pgfqpoint{1.574984in}{2.748172in}}%
\pgfpathlineto{\pgfqpoint{1.569704in}{2.743886in}}%
\pgfpathlineto{\pgfqpoint{1.560884in}{2.736421in}}%
\pgfpathlineto{\pgfqpoint{1.557641in}{2.733668in}}%
\pgfpathlineto{\pgfqpoint{1.547440in}{2.724670in}}%
\pgfpathlineto{\pgfqpoint{1.545579in}{2.723022in}}%
\pgfpathlineto{\pgfqpoint{1.534542in}{2.712918in}}%
\pgfpathlineto{\pgfqpoint{1.533516in}{2.711975in}}%
\pgfpathlineto{\pgfqpoint{1.522079in}{2.701167in}}%
\pgfpathlineto{\pgfqpoint{1.521453in}{2.700573in}}%
\pgfpathlineto{\pgfqpoint{1.509941in}{2.689416in}}%
\pgfpathlineto{\pgfqpoint{1.509390in}{2.688879in}}%
\pgfpathlineto{\pgfqpoint{1.498025in}{2.677664in}}%
\pgfpathlineto{\pgfqpoint{1.497328in}{2.676972in}}%
\pgfpathlineto{\pgfqpoint{1.486228in}{2.665913in}}%
\pgfpathlineto{\pgfqpoint{1.485265in}{2.664947in}}%
\pgfpathlineto{\pgfqpoint{1.474446in}{2.654161in}}%
\pgfpathlineto{\pgfqpoint{1.473202in}{2.652913in}}%
\pgfpathlineto{\pgfqpoint{1.462567in}{2.642410in}}%
\pgfpathlineto{\pgfqpoint{1.461140in}{2.640989in}}%
\pgfpathlineto{\pgfqpoint{1.450461in}{2.630659in}}%
\pgfpathlineto{\pgfqpoint{1.449077in}{2.629308in}}%
\pgfpathlineto{\pgfqpoint{1.437964in}{2.618907in}}%
\pgfpathlineto{\pgfqpoint{1.437014in}{2.618010in}}%
\pgfpathlineto{\pgfqpoint{1.424952in}{2.607239in}}%
\pgfpathlineto{\pgfqpoint{1.424852in}{2.607156in}}%
\pgfpathlineto{\pgfqpoint{1.412889in}{2.597120in}}%
\pgfpathlineto{\pgfqpoint{1.410642in}{2.595405in}}%
\pgfpathlineto{\pgfqpoint{1.400826in}{2.587820in}}%
\pgfpathlineto{\pgfqpoint{1.394755in}{2.583653in}}%
\pgfpathlineto{\pgfqpoint{1.388764in}{2.579489in}}%
\pgfpathlineto{\pgfqpoint{1.376701in}{2.572267in}}%
\pgfpathlineto{\pgfqpoint{1.375963in}{2.571902in}}%
\pgfpathlineto{\pgfqpoint{1.364638in}{2.566210in}}%
\pgfpathlineto{\pgfqpoint{1.352576in}{2.561501in}}%
\pgfpathlineto{\pgfqpoint{1.347652in}{2.560150in}}%
\pgfpathlineto{\pgfqpoint{1.340513in}{2.558161in}}%
\pgfpathlineto{\pgfqpoint{1.328450in}{2.556260in}}%
\pgfpathlineto{\pgfqpoint{1.316388in}{2.555839in}}%
\pgfpathlineto{\pgfqpoint{1.304325in}{2.556896in}}%
\pgfpathlineto{\pgfqpoint{1.292262in}{2.559411in}}%
\pgfpathlineto{\pgfqpoint{1.289992in}{2.560150in}}%
\pgfpathlineto{\pgfqpoint{1.280200in}{2.563289in}}%
\pgfpathlineto{\pgfqpoint{1.268137in}{2.568496in}}%
\pgfpathlineto{\pgfqpoint{1.261785in}{2.571902in}}%
\pgfpathlineto{\pgfqpoint{1.256074in}{2.574919in}}%
\pgfpathlineto{\pgfqpoint{1.244012in}{2.582449in}}%
\pgfpathlineto{\pgfqpoint{1.242317in}{2.583653in}}%
\pgfpathlineto{\pgfqpoint{1.231949in}{2.590921in}}%
\pgfpathlineto{\pgfqpoint{1.226178in}{2.595405in}}%
\pgfpathlineto{\pgfqpoint{1.219886in}{2.600232in}}%
\pgfpathlineto{\pgfqpoint{1.211562in}{2.607156in}}%
\pgfpathlineto{\pgfqpoint{1.207824in}{2.610229in}}%
\pgfpathlineto{\pgfqpoint{1.197903in}{2.618907in}}%
\pgfpathlineto{\pgfqpoint{1.195761in}{2.620760in}}%
\pgfpathlineto{\pgfqpoint{1.184836in}{2.630659in}}%
\pgfpathlineto{\pgfqpoint{1.183698in}{2.631679in}}%
\pgfpathlineto{\pgfqpoint{1.172112in}{2.642410in}}%
\pgfpathlineto{\pgfqpoint{1.171636in}{2.642847in}}%
\pgfpathlineto{\pgfqpoint{1.159573in}{2.654131in}}%
\pgfpathlineto{\pgfqpoint{1.159541in}{2.654161in}}%
\pgfpathlineto{\pgfqpoint{1.147510in}{2.665406in}}%
\pgfpathlineto{\pgfqpoint{1.146967in}{2.665913in}}%
\pgfpathlineto{\pgfqpoint{1.135448in}{2.676570in}}%
\pgfpathlineto{\pgfqpoint{1.134250in}{2.677664in}}%
\pgfpathlineto{\pgfqpoint{1.123385in}{2.687527in}}%
\pgfpathlineto{\pgfqpoint{1.121261in}{2.689416in}}%
\pgfpathlineto{\pgfqpoint{1.111322in}{2.698198in}}%
\pgfpathlineto{\pgfqpoint{1.107866in}{2.701167in}}%
\pgfpathlineto{\pgfqpoint{1.099260in}{2.708521in}}%
\pgfpathlineto{\pgfqpoint{1.093930in}{2.712918in}}%
\pgfpathlineto{\pgfqpoint{1.087197in}{2.718449in}}%
\pgfpathlineto{\pgfqpoint{1.079314in}{2.724670in}}%
\pgfpathlineto{\pgfqpoint{1.075134in}{2.727955in}}%
\pgfpathlineto{\pgfqpoint{1.063883in}{2.736421in}}%
\pgfpathlineto{\pgfqpoint{1.063072in}{2.737030in}}%
\pgfpathlineto{\pgfqpoint{1.051009in}{2.745642in}}%
\pgfpathlineto{\pgfqpoint{1.047293in}{2.748172in}}%
\pgfpathlineto{\pgfqpoint{1.038946in}{2.753843in}}%
\pgfpathlineto{\pgfqpoint{1.029597in}{2.759924in}}%
\pgfpathlineto{\pgfqpoint{1.026884in}{2.761686in}}%
\pgfpathlineto{\pgfqpoint{1.014821in}{2.769192in}}%
\pgfpathlineto{\pgfqpoint{1.010691in}{2.771675in}}%
\pgfpathlineto{\pgfqpoint{1.002758in}{2.776444in}}%
\pgfpathlineto{\pgfqpoint{0.990926in}{2.783427in}}%
\pgfpathlineto{\pgfqpoint{0.990696in}{2.783563in}}%
\pgfpathlineto{\pgfqpoint{0.978633in}{2.790575in}}%
\pgfpathlineto{\pgfqpoint{0.970802in}{2.795178in}}%
\pgfpathlineto{\pgfqpoint{0.966570in}{2.797669in}}%
\pgfpathlineto{\pgfqpoint{0.954508in}{2.804937in}}%
\pgfpathlineto{\pgfqpoint{0.951326in}{2.806929in}}%
\pgfpathlineto{\pgfqpoint{0.942445in}{2.812506in}}%
\pgfpathlineto{\pgfqpoint{0.933181in}{2.818681in}}%
\pgfpathlineto{\pgfqpoint{0.930382in}{2.820553in}}%
\pgfpathlineto{\pgfqpoint{0.918320in}{2.829195in}}%
\pgfpathlineto{\pgfqpoint{0.916720in}{2.830432in}}%
\pgfpathlineto{\pgfqpoint{0.906257in}{2.838569in}}%
\pgfpathlineto{\pgfqpoint{0.901997in}{2.842183in}}%
\pgfpathlineto{\pgfqpoint{0.894194in}{2.848844in}}%
\pgfpathlineto{\pgfqpoint{0.888744in}{2.853935in}}%
\pgfpathlineto{\pgfqpoint{0.882132in}{2.860152in}}%
\pgfpathlineto{\pgfqpoint{0.876754in}{2.865686in}}%
\pgfpathlineto{\pgfqpoint{0.870069in}{2.872616in}}%
\pgfpathlineto{\pgfqpoint{0.865811in}{2.877438in}}%
\pgfpathlineto{\pgfqpoint{0.858006in}{2.886349in}}%
\pgfpathlineto{\pgfqpoint{0.855721in}{2.889189in}}%
\pgfpathlineto{\pgfqpoint{0.846344in}{2.900940in}}%
\pgfpathlineto{\pgfqpoint{0.845944in}{2.901445in}}%
\pgfpathlineto{\pgfqpoint{0.837686in}{2.912692in}}%
\pgfpathlineto{\pgfqpoint{0.833881in}{2.917925in}}%
\pgfpathlineto{\pgfqpoint{0.829481in}{2.924443in}}%
\pgfpathlineto{\pgfqpoint{0.821818in}{2.935913in}}%
\pgfpathlineto{\pgfqpoint{0.821643in}{2.936194in}}%
\pgfpathlineto{\pgfqpoint{0.814314in}{2.947946in}}%
\pgfpathlineto{\pgfqpoint{0.809756in}{2.955332in}}%
\pgfpathlineto{\pgfqpoint{0.807223in}{2.959697in}}%
\pgfpathlineto{\pgfqpoint{0.800441in}{2.971449in}}%
\pgfpathlineto{\pgfqpoint{0.797693in}{2.976235in}}%
\pgfpathlineto{\pgfqpoint{0.793909in}{2.983200in}}%
\pgfpathlineto{\pgfqpoint{0.785630in}{2.983200in}}%
\pgfpathlineto{\pgfqpoint{0.773567in}{2.983200in}}%
\pgfpathlineto{\pgfqpoint{0.761505in}{2.983200in}}%
\pgfpathlineto{\pgfqpoint{0.757867in}{2.983200in}}%
\pgfpathlineto{\pgfqpoint{0.761505in}{2.975202in}}%
\pgfpathlineto{\pgfqpoint{0.763275in}{2.971449in}}%
\pgfpathlineto{\pgfqpoint{0.768825in}{2.959697in}}%
\pgfpathlineto{\pgfqpoint{0.773567in}{2.949758in}}%
\pgfpathlineto{\pgfqpoint{0.774470in}{2.947946in}}%
\pgfpathlineto{\pgfqpoint{0.780322in}{2.936194in}}%
\pgfpathlineto{\pgfqpoint{0.785630in}{2.925664in}}%
\pgfpathlineto{\pgfqpoint{0.786278in}{2.924443in}}%
\pgfpathlineto{\pgfqpoint{0.792506in}{2.912692in}}%
\pgfpathlineto{\pgfqpoint{0.797693in}{2.903022in}}%
\pgfpathlineto{\pgfqpoint{0.798877in}{2.900940in}}%
\pgfpathlineto{\pgfqpoint{0.805571in}{2.889189in}}%
\pgfpathlineto{\pgfqpoint{0.809756in}{2.881916in}}%
\pgfpathlineto{\pgfqpoint{0.812509in}{2.877438in}}%
\pgfpathlineto{\pgfqpoint{0.819783in}{2.865686in}}%
\pgfpathlineto{\pgfqpoint{0.821818in}{2.862415in}}%
\pgfpathlineto{\pgfqpoint{0.827499in}{2.853935in}}%
\pgfpathlineto{\pgfqpoint{0.833881in}{2.844520in}}%
\pgfpathlineto{\pgfqpoint{0.835597in}{2.842183in}}%
\pgfpathlineto{\pgfqpoint{0.844289in}{2.830432in}}%
\pgfpathlineto{\pgfqpoint{0.845944in}{2.828210in}}%
\pgfpathlineto{\pgfqpoint{0.853692in}{2.818681in}}%
\pgfpathlineto{\pgfqpoint{0.858006in}{2.813429in}}%
\pgfpathlineto{\pgfqpoint{0.863870in}{2.806929in}}%
\pgfpathlineto{\pgfqpoint{0.870069in}{2.800121in}}%
\pgfpathlineto{\pgfqpoint{0.875034in}{2.795178in}}%
\pgfpathlineto{\pgfqpoint{0.882132in}{2.788171in}}%
\pgfpathlineto{\pgfqpoint{0.887447in}{2.783427in}}%
\pgfpathlineto{\pgfqpoint{0.894194in}{2.777449in}}%
\pgfpathlineto{\pgfqpoint{0.901406in}{2.771675in}}%
\pgfpathlineto{\pgfqpoint{0.906257in}{2.767818in}}%
\pgfpathlineto{\pgfqpoint{0.917213in}{2.759924in}}%
\pgfpathlineto{\pgfqpoint{0.918320in}{2.759131in}}%
\pgfpathlineto{\pgfqpoint{0.930382in}{2.751180in}}%
\pgfpathlineto{\pgfqpoint{0.935315in}{2.748172in}}%
\pgfpathlineto{\pgfqpoint{0.942445in}{2.743842in}}%
\pgfpathlineto{\pgfqpoint{0.954508in}{2.736970in}}%
\pgfpathlineto{\pgfqpoint{0.955504in}{2.736421in}}%
\pgfpathlineto{\pgfqpoint{0.966570in}{2.730339in}}%
\pgfpathlineto{\pgfqpoint{0.977166in}{2.724670in}}%
\pgfpathlineto{\pgfqpoint{0.978633in}{2.723886in}}%
\pgfpathlineto{\pgfqpoint{0.990696in}{2.717395in}}%
\pgfpathlineto{\pgfqpoint{0.998891in}{2.712918in}}%
\pgfpathlineto{\pgfqpoint{1.002758in}{2.710807in}}%
\pgfpathlineto{\pgfqpoint{1.014821in}{2.703973in}}%
\pgfpathlineto{\pgfqpoint{1.019551in}{2.701167in}}%
\pgfpathlineto{\pgfqpoint{1.026884in}{2.696813in}}%
\pgfpathlineto{\pgfqpoint{1.038728in}{2.689416in}}%
\pgfpathlineto{\pgfqpoint{1.038946in}{2.689279in}}%
\pgfpathlineto{\pgfqpoint{1.051009in}{2.681242in}}%
\pgfpathlineto{\pgfqpoint{1.056083in}{2.677664in}}%
\pgfpathlineto{\pgfqpoint{1.063072in}{2.672722in}}%
\pgfpathlineto{\pgfqpoint{1.072174in}{2.665913in}}%
\pgfpathlineto{\pgfqpoint{1.075134in}{2.663690in}}%
\pgfpathlineto{\pgfqpoint{1.087168in}{2.654161in}}%
\pgfpathlineto{\pgfqpoint{1.087197in}{2.654138in}}%
\pgfpathlineto{\pgfqpoint{1.099260in}{2.644044in}}%
\pgfpathlineto{\pgfqpoint{1.101134in}{2.642410in}}%
\pgfpathlineto{\pgfqpoint{1.111322in}{2.633475in}}%
\pgfpathlineto{\pgfqpoint{1.114427in}{2.630659in}}%
\pgfpathlineto{\pgfqpoint{1.123385in}{2.622478in}}%
\pgfpathlineto{\pgfqpoint{1.127196in}{2.618907in}}%
\pgfpathlineto{\pgfqpoint{1.135448in}{2.611120in}}%
\pgfpathlineto{\pgfqpoint{1.139579in}{2.607156in}}%
\pgfpathlineto{\pgfqpoint{1.147510in}{2.599485in}}%
\pgfpathlineto{\pgfqpoint{1.151701in}{2.595405in}}%
\pgfpathlineto{\pgfqpoint{1.159573in}{2.587673in}}%
\pgfpathlineto{\pgfqpoint{1.163683in}{2.583653in}}%
\pgfpathlineto{\pgfqpoint{1.171636in}{2.575801in}}%
\pgfpathlineto{\pgfqpoint{1.175649in}{2.571902in}}%
\pgfpathlineto{\pgfqpoint{1.183698in}{2.563999in}}%
\pgfpathlineto{\pgfqpoint{1.187735in}{2.560150in}}%
\pgfpathlineto{\pgfqpoint{1.195761in}{2.552411in}}%
\pgfpathlineto{\pgfqpoint{1.200106in}{2.548399in}}%
\pgfpathlineto{\pgfqpoint{1.207824in}{2.541187in}}%
\pgfpathlineto{\pgfqpoint{1.212981in}{2.536648in}}%
\pgfpathlineto{\pgfqpoint{1.219886in}{2.530490in}}%
\pgfpathlineto{\pgfqpoint{1.226673in}{2.524896in}}%
\pgfpathlineto{\pgfqpoint{1.231949in}{2.520487in}}%
\pgfpathlineto{\pgfqpoint{1.241668in}{2.513145in}}%
\pgfpathlineto{\pgfqpoint{1.244012in}{2.511348in}}%
\pgfpathlineto{\pgfqpoint{1.256074in}{2.503212in}}%
\pgfpathlineto{\pgfqpoint{1.259253in}{2.501394in}}%
\pgfpathlineto{\pgfqpoint{1.268137in}{2.496226in}}%
\pgfpathlineto{\pgfqpoint{1.280200in}{2.490575in}}%
\pgfpathlineto{\pgfqpoint{1.282872in}{2.489642in}}%
\pgfpathclose%
\pgfusepath{fill}%
\end{pgfscope}%
\begin{pgfscope}%
\pgfpathrectangle{\pgfqpoint{0.423750in}{1.819814in}}{\pgfqpoint{1.194205in}{1.163386in}}%
\pgfusepath{clip}%
\pgfsetbuttcap%
\pgfsetroundjoin%
\definecolor{currentfill}{rgb}{0.962283,0.593046,0.431453}%
\pgfsetfillcolor{currentfill}%
\pgfsetlinewidth{0.000000pt}%
\definecolor{currentstroke}{rgb}{0.000000,0.000000,0.000000}%
\pgfsetstrokecolor{currentstroke}%
\pgfsetdash{}{0pt}%
\pgfpathmoveto{\pgfqpoint{0.737379in}{1.825063in}}%
\pgfpathlineto{\pgfqpoint{0.741475in}{1.819814in}}%
\pgfpathlineto{\pgfqpoint{0.749442in}{1.819814in}}%
\pgfpathlineto{\pgfqpoint{0.761505in}{1.819814in}}%
\pgfpathlineto{\pgfqpoint{0.773567in}{1.819814in}}%
\pgfpathlineto{\pgfqpoint{0.785630in}{1.819814in}}%
\pgfpathlineto{\pgfqpoint{0.797693in}{1.819814in}}%
\pgfpathlineto{\pgfqpoint{0.803343in}{1.819814in}}%
\pgfpathlineto{\pgfqpoint{0.797693in}{1.827901in}}%
\pgfpathlineto{\pgfqpoint{0.795057in}{1.831565in}}%
\pgfpathlineto{\pgfqpoint{0.786489in}{1.843316in}}%
\pgfpathlineto{\pgfqpoint{0.785630in}{1.844486in}}%
\pgfpathlineto{\pgfqpoint{0.777659in}{1.855068in}}%
\pgfpathlineto{\pgfqpoint{0.773567in}{1.860408in}}%
\pgfpathlineto{\pgfqpoint{0.768571in}{1.866819in}}%
\pgfpathlineto{\pgfqpoint{0.761505in}{1.875727in}}%
\pgfpathlineto{\pgfqpoint{0.759234in}{1.878571in}}%
\pgfpathlineto{\pgfqpoint{0.749685in}{1.890322in}}%
\pgfpathlineto{\pgfqpoint{0.749442in}{1.890617in}}%
\pgfpathlineto{\pgfqpoint{0.740077in}{1.902073in}}%
\pgfpathlineto{\pgfqpoint{0.737379in}{1.905309in}}%
\pgfpathlineto{\pgfqpoint{0.730411in}{1.913825in}}%
\pgfpathlineto{\pgfqpoint{0.725317in}{1.919927in}}%
\pgfpathlineto{\pgfqpoint{0.720750in}{1.925576in}}%
\pgfpathlineto{\pgfqpoint{0.713254in}{1.934663in}}%
\pgfpathlineto{\pgfqpoint{0.711153in}{1.937327in}}%
\pgfpathlineto{\pgfqpoint{0.701711in}{1.949079in}}%
\pgfpathlineto{\pgfqpoint{0.701191in}{1.949716in}}%
\pgfpathlineto{\pgfqpoint{0.692652in}{1.960830in}}%
\pgfpathlineto{\pgfqpoint{0.689129in}{1.965318in}}%
\pgfpathlineto{\pgfqpoint{0.683804in}{1.972582in}}%
\pgfpathlineto{\pgfqpoint{0.677066in}{1.981575in}}%
\pgfpathlineto{\pgfqpoint{0.675152in}{1.984333in}}%
\pgfpathlineto{\pgfqpoint{0.666868in}{1.996084in}}%
\pgfpathlineto{\pgfqpoint{0.665003in}{1.998688in}}%
\pgfpathlineto{\pgfqpoint{0.658981in}{2.007836in}}%
\pgfpathlineto{\pgfqpoint{0.652941in}{2.016803in}}%
\pgfpathlineto{\pgfqpoint{0.651223in}{2.019587in}}%
\pgfpathlineto{\pgfqpoint{0.643880in}{2.031338in}}%
\pgfpathlineto{\pgfqpoint{0.640878in}{2.036059in}}%
\pgfpathlineto{\pgfqpoint{0.636796in}{2.043090in}}%
\pgfpathlineto{\pgfqpoint{0.629834in}{2.054841in}}%
\pgfpathlineto{\pgfqpoint{0.628815in}{2.056545in}}%
\pgfpathlineto{\pgfqpoint{0.623339in}{2.066593in}}%
\pgfpathlineto{\pgfqpoint{0.616774in}{2.078344in}}%
\pgfpathlineto{\pgfqpoint{0.616753in}{2.078382in}}%
\pgfpathlineto{\pgfqpoint{0.610761in}{2.090095in}}%
\pgfpathlineto{\pgfqpoint{0.604690in}{2.101667in}}%
\pgfpathlineto{\pgfqpoint{0.604604in}{2.101847in}}%
\pgfpathlineto{\pgfqpoint{0.598955in}{2.113598in}}%
\pgfpathlineto{\pgfqpoint{0.593172in}{2.125349in}}%
\pgfpathlineto{\pgfqpoint{0.592627in}{2.126452in}}%
\pgfpathlineto{\pgfqpoint{0.587811in}{2.137101in}}%
\pgfpathlineto{\pgfqpoint{0.582392in}{2.148852in}}%
\pgfpathlineto{\pgfqpoint{0.580565in}{2.152781in}}%
\pgfpathlineto{\pgfqpoint{0.577226in}{2.160604in}}%
\pgfpathlineto{\pgfqpoint{0.572143in}{2.172355in}}%
\pgfpathlineto{\pgfqpoint{0.568502in}{2.180654in}}%
\pgfpathlineto{\pgfqpoint{0.567106in}{2.184106in}}%
\pgfpathlineto{\pgfqpoint{0.562332in}{2.195858in}}%
\pgfpathlineto{\pgfqpoint{0.557469in}{2.207609in}}%
\pgfpathlineto{\pgfqpoint{0.556439in}{2.210088in}}%
\pgfpathlineto{\pgfqpoint{0.552880in}{2.219360in}}%
\pgfpathlineto{\pgfqpoint{0.548309in}{2.231112in}}%
\pgfpathlineto{\pgfqpoint{0.544377in}{2.241073in}}%
\pgfpathlineto{\pgfqpoint{0.543721in}{2.242863in}}%
\pgfpathlineto{\pgfqpoint{0.539411in}{2.254615in}}%
\pgfpathlineto{\pgfqpoint{0.535032in}{2.266366in}}%
\pgfpathlineto{\pgfqpoint{0.532314in}{2.273591in}}%
\pgfpathlineto{\pgfqpoint{0.530727in}{2.278117in}}%
\pgfpathlineto{\pgfqpoint{0.526590in}{2.289869in}}%
\pgfpathlineto{\pgfqpoint{0.522390in}{2.301620in}}%
\pgfpathlineto{\pgfqpoint{0.520251in}{2.307562in}}%
\pgfpathlineto{\pgfqpoint{0.518295in}{2.313371in}}%
\pgfpathlineto{\pgfqpoint{0.514316in}{2.325123in}}%
\pgfpathlineto{\pgfqpoint{0.510279in}{2.336874in}}%
\pgfpathlineto{\pgfqpoint{0.508189in}{2.342922in}}%
\pgfpathlineto{\pgfqpoint{0.506336in}{2.348626in}}%
\pgfpathlineto{\pgfqpoint{0.502501in}{2.360377in}}%
\pgfpathlineto{\pgfqpoint{0.498612in}{2.372128in}}%
\pgfpathlineto{\pgfqpoint{0.496126in}{2.379580in}}%
\pgfpathlineto{\pgfqpoint{0.494773in}{2.383880in}}%
\pgfpathlineto{\pgfqpoint{0.491066in}{2.395631in}}%
\pgfpathlineto{\pgfqpoint{0.487309in}{2.407383in}}%
\pgfpathlineto{\pgfqpoint{0.484063in}{2.417420in}}%
\pgfpathlineto{\pgfqpoint{0.483538in}{2.419134in}}%
\pgfpathlineto{\pgfqpoint{0.479944in}{2.430885in}}%
\pgfpathlineto{\pgfqpoint{0.476303in}{2.442637in}}%
\pgfpathlineto{\pgfqpoint{0.472613in}{2.454388in}}%
\pgfpathlineto{\pgfqpoint{0.472001in}{2.456339in}}%
\pgfpathlineto{\pgfqpoint{0.469077in}{2.466139in}}%
\pgfpathlineto{\pgfqpoint{0.465536in}{2.477891in}}%
\pgfpathlineto{\pgfqpoint{0.461949in}{2.489642in}}%
\pgfpathlineto{\pgfqpoint{0.459938in}{2.496192in}}%
\pgfpathlineto{\pgfqpoint{0.458413in}{2.501394in}}%
\pgfpathlineto{\pgfqpoint{0.454958in}{2.513145in}}%
\pgfpathlineto{\pgfqpoint{0.451458in}{2.524896in}}%
\pgfpathlineto{\pgfqpoint{0.447911in}{2.536648in}}%
\pgfpathlineto{\pgfqpoint{0.447875in}{2.536766in}}%
\pgfpathlineto{\pgfqpoint{0.444525in}{2.548399in}}%
\pgfpathlineto{\pgfqpoint{0.441097in}{2.560150in}}%
\pgfpathlineto{\pgfqpoint{0.437623in}{2.571902in}}%
\pgfpathlineto{\pgfqpoint{0.435813in}{2.577994in}}%
\pgfpathlineto{\pgfqpoint{0.434196in}{2.583653in}}%
\pgfpathlineto{\pgfqpoint{0.430826in}{2.595405in}}%
\pgfpathlineto{\pgfqpoint{0.427410in}{2.607156in}}%
\pgfpathlineto{\pgfqpoint{0.423947in}{2.618907in}}%
\pgfpathlineto{\pgfqpoint{0.423750in}{2.619578in}}%
\pgfpathlineto{\pgfqpoint{0.423750in}{2.618907in}}%
\pgfpathlineto{\pgfqpoint{0.423750in}{2.607156in}}%
\pgfpathlineto{\pgfqpoint{0.423750in}{2.595405in}}%
\pgfpathlineto{\pgfqpoint{0.423750in}{2.583653in}}%
\pgfpathlineto{\pgfqpoint{0.423750in}{2.571902in}}%
\pgfpathlineto{\pgfqpoint{0.423750in}{2.560150in}}%
\pgfpathlineto{\pgfqpoint{0.423750in}{2.548399in}}%
\pgfpathlineto{\pgfqpoint{0.423750in}{2.540816in}}%
\pgfpathlineto{\pgfqpoint{0.425005in}{2.536648in}}%
\pgfpathlineto{\pgfqpoint{0.428535in}{2.524896in}}%
\pgfpathlineto{\pgfqpoint{0.432020in}{2.513145in}}%
\pgfpathlineto{\pgfqpoint{0.435463in}{2.501394in}}%
\pgfpathlineto{\pgfqpoint{0.435813in}{2.500203in}}%
\pgfpathlineto{\pgfqpoint{0.439028in}{2.489642in}}%
\pgfpathlineto{\pgfqpoint{0.442567in}{2.477891in}}%
\pgfpathlineto{\pgfqpoint{0.446062in}{2.466139in}}%
\pgfpathlineto{\pgfqpoint{0.447875in}{2.460016in}}%
\pgfpathlineto{\pgfqpoint{0.449610in}{2.454388in}}%
\pgfpathlineto{\pgfqpoint{0.453215in}{2.442637in}}%
\pgfpathlineto{\pgfqpoint{0.456777in}{2.430885in}}%
\pgfpathlineto{\pgfqpoint{0.459938in}{2.420349in}}%
\pgfpathlineto{\pgfqpoint{0.460319in}{2.419134in}}%
\pgfpathlineto{\pgfqpoint{0.464006in}{2.407383in}}%
\pgfpathlineto{\pgfqpoint{0.467647in}{2.395631in}}%
\pgfpathlineto{\pgfqpoint{0.471245in}{2.383880in}}%
\pgfpathlineto{\pgfqpoint{0.472001in}{2.381414in}}%
\pgfpathlineto{\pgfqpoint{0.474980in}{2.372128in}}%
\pgfpathlineto{\pgfqpoint{0.478716in}{2.360377in}}%
\pgfpathlineto{\pgfqpoint{0.482406in}{2.348626in}}%
\pgfpathlineto{\pgfqpoint{0.484063in}{2.343329in}}%
\pgfpathlineto{\pgfqpoint{0.486187in}{2.336874in}}%
\pgfpathlineto{\pgfqpoint{0.490032in}{2.325123in}}%
\pgfpathlineto{\pgfqpoint{0.493828in}{2.313371in}}%
\pgfpathlineto{\pgfqpoint{0.496126in}{2.306214in}}%
\pgfpathlineto{\pgfqpoint{0.497683in}{2.301620in}}%
\pgfpathlineto{\pgfqpoint{0.501651in}{2.289869in}}%
\pgfpathlineto{\pgfqpoint{0.505568in}{2.278117in}}%
\pgfpathlineto{\pgfqpoint{0.508189in}{2.270195in}}%
\pgfpathlineto{\pgfqpoint{0.509531in}{2.266366in}}%
\pgfpathlineto{\pgfqpoint{0.513638in}{2.254615in}}%
\pgfpathlineto{\pgfqpoint{0.517690in}{2.242863in}}%
\pgfpathlineto{\pgfqpoint{0.520251in}{2.235378in}}%
\pgfpathlineto{\pgfqpoint{0.521804in}{2.231112in}}%
\pgfpathlineto{\pgfqpoint{0.526064in}{2.219360in}}%
\pgfpathlineto{\pgfqpoint{0.530265in}{2.207609in}}%
\pgfpathlineto{\pgfqpoint{0.532314in}{2.201845in}}%
\pgfpathlineto{\pgfqpoint{0.534586in}{2.195858in}}%
\pgfpathlineto{\pgfqpoint{0.539013in}{2.184106in}}%
\pgfpathlineto{\pgfqpoint{0.543376in}{2.172355in}}%
\pgfpathlineto{\pgfqpoint{0.544377in}{2.169653in}}%
\pgfpathlineto{\pgfqpoint{0.547967in}{2.160604in}}%
\pgfpathlineto{\pgfqpoint{0.552574in}{2.148852in}}%
\pgfpathlineto{\pgfqpoint{0.556439in}{2.138865in}}%
\pgfpathlineto{\pgfqpoint{0.557174in}{2.137101in}}%
\pgfpathlineto{\pgfqpoint{0.562051in}{2.125349in}}%
\pgfpathlineto{\pgfqpoint{0.566849in}{2.113598in}}%
\pgfpathlineto{\pgfqpoint{0.568502in}{2.109526in}}%
\pgfpathlineto{\pgfqpoint{0.571863in}{2.101847in}}%
\pgfpathlineto{\pgfqpoint{0.576946in}{2.090095in}}%
\pgfpathlineto{\pgfqpoint{0.580565in}{2.081622in}}%
\pgfpathlineto{\pgfqpoint{0.582078in}{2.078344in}}%
\pgfpathlineto{\pgfqpoint{0.587471in}{2.066593in}}%
\pgfpathlineto{\pgfqpoint{0.592627in}{2.055146in}}%
\pgfpathlineto{\pgfqpoint{0.592776in}{2.054841in}}%
\pgfpathlineto{\pgfqpoint{0.598501in}{2.043090in}}%
\pgfpathlineto{\pgfqpoint{0.604108in}{2.031338in}}%
\pgfpathlineto{\pgfqpoint{0.604690in}{2.030115in}}%
\pgfpathlineto{\pgfqpoint{0.610120in}{2.019587in}}%
\pgfpathlineto{\pgfqpoint{0.616060in}{2.007836in}}%
\pgfpathlineto{\pgfqpoint{0.616753in}{2.006458in}}%
\pgfpathlineto{\pgfqpoint{0.622418in}{1.996084in}}%
\pgfpathlineto{\pgfqpoint{0.628698in}{1.984333in}}%
\pgfpathlineto{\pgfqpoint{0.628815in}{1.984113in}}%
\pgfpathlineto{\pgfqpoint{0.635479in}{1.972582in}}%
\pgfpathlineto{\pgfqpoint{0.640878in}{1.963037in}}%
\pgfpathlineto{\pgfqpoint{0.642232in}{1.960830in}}%
\pgfpathlineto{\pgfqpoint{0.649372in}{1.949079in}}%
\pgfpathlineto{\pgfqpoint{0.652941in}{1.943109in}}%
\pgfpathlineto{\pgfqpoint{0.656677in}{1.937327in}}%
\pgfpathlineto{\pgfqpoint{0.664137in}{1.925576in}}%
\pgfpathlineto{\pgfqpoint{0.665003in}{1.924199in}}%
\pgfpathlineto{\pgfqpoint{0.672030in}{1.913825in}}%
\pgfpathlineto{\pgfqpoint{0.677066in}{1.906239in}}%
\pgfpathlineto{\pgfqpoint{0.680026in}{1.902073in}}%
\pgfpathlineto{\pgfqpoint{0.688239in}{1.890322in}}%
\pgfpathlineto{\pgfqpoint{0.689129in}{1.889036in}}%
\pgfpathlineto{\pgfqpoint{0.696815in}{1.878571in}}%
\pgfpathlineto{\pgfqpoint{0.701191in}{1.872499in}}%
\pgfpathlineto{\pgfqpoint{0.705500in}{1.866819in}}%
\pgfpathlineto{\pgfqpoint{0.713254in}{1.856408in}}%
\pgfpathlineto{\pgfqpoint{0.714294in}{1.855068in}}%
\pgfpathlineto{\pgfqpoint{0.723308in}{1.843316in}}%
\pgfpathlineto{\pgfqpoint{0.725317in}{1.840660in}}%
\pgfpathlineto{\pgfqpoint{0.732401in}{1.831565in}}%
\pgfpathclose%
\pgfusepath{fill}%
\end{pgfscope}%
\begin{pgfscope}%
\pgfpathrectangle{\pgfqpoint{0.423750in}{1.819814in}}{\pgfqpoint{1.194205in}{1.163386in}}%
\pgfusepath{clip}%
\pgfsetbuttcap%
\pgfsetroundjoin%
\definecolor{currentfill}{rgb}{0.962283,0.593046,0.431453}%
\pgfsetfillcolor{currentfill}%
\pgfsetlinewidth{0.000000pt}%
\definecolor{currentstroke}{rgb}{0.000000,0.000000,0.000000}%
\pgfsetstrokecolor{currentstroke}%
\pgfsetdash{}{0pt}%
\pgfpathmoveto{\pgfqpoint{1.292262in}{2.402569in}}%
\pgfpathlineto{\pgfqpoint{1.304325in}{2.399448in}}%
\pgfpathlineto{\pgfqpoint{1.316388in}{2.398089in}}%
\pgfpathlineto{\pgfqpoint{1.328450in}{2.398516in}}%
\pgfpathlineto{\pgfqpoint{1.340513in}{2.400733in}}%
\pgfpathlineto{\pgfqpoint{1.352576in}{2.404722in}}%
\pgfpathlineto{\pgfqpoint{1.358197in}{2.407383in}}%
\pgfpathlineto{\pgfqpoint{1.364638in}{2.410371in}}%
\pgfpathlineto{\pgfqpoint{1.376701in}{2.417588in}}%
\pgfpathlineto{\pgfqpoint{1.378837in}{2.419134in}}%
\pgfpathlineto{\pgfqpoint{1.388764in}{2.426191in}}%
\pgfpathlineto{\pgfqpoint{1.394454in}{2.430885in}}%
\pgfpathlineto{\pgfqpoint{1.400826in}{2.436056in}}%
\pgfpathlineto{\pgfqpoint{1.408042in}{2.442637in}}%
\pgfpathlineto{\pgfqpoint{1.412889in}{2.446989in}}%
\pgfpathlineto{\pgfqpoint{1.420408in}{2.454388in}}%
\pgfpathlineto{\pgfqpoint{1.424952in}{2.458796in}}%
\pgfpathlineto{\pgfqpoint{1.432004in}{2.466139in}}%
\pgfpathlineto{\pgfqpoint{1.437014in}{2.471287in}}%
\pgfpathlineto{\pgfqpoint{1.443110in}{2.477891in}}%
\pgfpathlineto{\pgfqpoint{1.449077in}{2.484276in}}%
\pgfpathlineto{\pgfqpoint{1.453908in}{2.489642in}}%
\pgfpathlineto{\pgfqpoint{1.461140in}{2.497584in}}%
\pgfpathlineto{\pgfqpoint{1.464528in}{2.501394in}}%
\pgfpathlineto{\pgfqpoint{1.473202in}{2.511044in}}%
\pgfpathlineto{\pgfqpoint{1.475071in}{2.513145in}}%
\pgfpathlineto{\pgfqpoint{1.485265in}{2.524496in}}%
\pgfpathlineto{\pgfqpoint{1.485625in}{2.524896in}}%
\pgfpathlineto{\pgfqpoint{1.496295in}{2.536648in}}%
\pgfpathlineto{\pgfqpoint{1.497328in}{2.537775in}}%
\pgfpathlineto{\pgfqpoint{1.507174in}{2.548399in}}%
\pgfpathlineto{\pgfqpoint{1.509390in}{2.550773in}}%
\pgfpathlineto{\pgfqpoint{1.518342in}{2.560150in}}%
\pgfpathlineto{\pgfqpoint{1.521453in}{2.563388in}}%
\pgfpathlineto{\pgfqpoint{1.529896in}{2.571902in}}%
\pgfpathlineto{\pgfqpoint{1.533516in}{2.575531in}}%
\pgfpathlineto{\pgfqpoint{1.541951in}{2.583653in}}%
\pgfpathlineto{\pgfqpoint{1.545579in}{2.587128in}}%
\pgfpathlineto{\pgfqpoint{1.554644in}{2.595405in}}%
\pgfpathlineto{\pgfqpoint{1.557641in}{2.598129in}}%
\pgfpathlineto{\pgfqpoint{1.568131in}{2.607156in}}%
\pgfpathlineto{\pgfqpoint{1.569704in}{2.608505in}}%
\pgfpathlineto{\pgfqpoint{1.581767in}{2.618231in}}%
\pgfpathlineto{\pgfqpoint{1.582665in}{2.618907in}}%
\pgfpathlineto{\pgfqpoint{1.593829in}{2.627295in}}%
\pgfpathlineto{\pgfqpoint{1.598620in}{2.630659in}}%
\pgfpathlineto{\pgfqpoint{1.605892in}{2.635757in}}%
\pgfpathlineto{\pgfqpoint{1.616032in}{2.642410in}}%
\pgfpathlineto{\pgfqpoint{1.617955in}{2.643671in}}%
\pgfpathlineto{\pgfqpoint{1.617955in}{2.654161in}}%
\pgfpathlineto{\pgfqpoint{1.617955in}{2.665913in}}%
\pgfpathlineto{\pgfqpoint{1.617955in}{2.677664in}}%
\pgfpathlineto{\pgfqpoint{1.617955in}{2.689416in}}%
\pgfpathlineto{\pgfqpoint{1.617955in}{2.701167in}}%
\pgfpathlineto{\pgfqpoint{1.617955in}{2.712918in}}%
\pgfpathlineto{\pgfqpoint{1.617955in}{2.715625in}}%
\pgfpathlineto{\pgfqpoint{1.614013in}{2.712918in}}%
\pgfpathlineto{\pgfqpoint{1.605892in}{2.707340in}}%
\pgfpathlineto{\pgfqpoint{1.597357in}{2.701167in}}%
\pgfpathlineto{\pgfqpoint{1.593829in}{2.698613in}}%
\pgfpathlineto{\pgfqpoint{1.581791in}{2.689416in}}%
\pgfpathlineto{\pgfqpoint{1.581767in}{2.689397in}}%
\pgfpathlineto{\pgfqpoint{1.569704in}{2.679627in}}%
\pgfpathlineto{\pgfqpoint{1.567397in}{2.677664in}}%
\pgfpathlineto{\pgfqpoint{1.557641in}{2.669336in}}%
\pgfpathlineto{\pgfqpoint{1.553808in}{2.665913in}}%
\pgfpathlineto{\pgfqpoint{1.545579in}{2.658535in}}%
\pgfpathlineto{\pgfqpoint{1.540890in}{2.654161in}}%
\pgfpathlineto{\pgfqpoint{1.533516in}{2.647252in}}%
\pgfpathlineto{\pgfqpoint{1.528516in}{2.642410in}}%
\pgfpathlineto{\pgfqpoint{1.521453in}{2.635535in}}%
\pgfpathlineto{\pgfqpoint{1.516568in}{2.630659in}}%
\pgfpathlineto{\pgfqpoint{1.509390in}{2.623450in}}%
\pgfpathlineto{\pgfqpoint{1.504940in}{2.618907in}}%
\pgfpathlineto{\pgfqpoint{1.497328in}{2.611083in}}%
\pgfpathlineto{\pgfqpoint{1.493533in}{2.607156in}}%
\pgfpathlineto{\pgfqpoint{1.485265in}{2.598534in}}%
\pgfpathlineto{\pgfqpoint{1.482255in}{2.595405in}}%
\pgfpathlineto{\pgfqpoint{1.473202in}{2.585918in}}%
\pgfpathlineto{\pgfqpoint{1.471010in}{2.583653in}}%
\pgfpathlineto{\pgfqpoint{1.461140in}{2.573365in}}%
\pgfpathlineto{\pgfqpoint{1.459698in}{2.571902in}}%
\pgfpathlineto{\pgfqpoint{1.449077in}{2.561015in}}%
\pgfpathlineto{\pgfqpoint{1.448199in}{2.560150in}}%
\pgfpathlineto{\pgfqpoint{1.437014in}{2.549020in}}%
\pgfpathlineto{\pgfqpoint{1.436355in}{2.548399in}}%
\pgfpathlineto{\pgfqpoint{1.424952in}{2.537537in}}%
\pgfpathlineto{\pgfqpoint{1.423948in}{2.536648in}}%
\pgfpathlineto{\pgfqpoint{1.412889in}{2.526729in}}%
\pgfpathlineto{\pgfqpoint{1.410646in}{2.524896in}}%
\pgfpathlineto{\pgfqpoint{1.400826in}{2.516764in}}%
\pgfpathlineto{\pgfqpoint{1.395911in}{2.513145in}}%
\pgfpathlineto{\pgfqpoint{1.388764in}{2.507808in}}%
\pgfpathlineto{\pgfqpoint{1.378791in}{2.501394in}}%
\pgfpathlineto{\pgfqpoint{1.376701in}{2.500028in}}%
\pgfpathlineto{\pgfqpoint{1.364638in}{2.493510in}}%
\pgfpathlineto{\pgfqpoint{1.355415in}{2.489642in}}%
\pgfpathlineto{\pgfqpoint{1.352576in}{2.488431in}}%
\pgfpathlineto{\pgfqpoint{1.340513in}{2.484820in}}%
\pgfpathlineto{\pgfqpoint{1.328450in}{2.482795in}}%
\pgfpathlineto{\pgfqpoint{1.316388in}{2.482372in}}%
\pgfpathlineto{\pgfqpoint{1.304325in}{2.483549in}}%
\pgfpathlineto{\pgfqpoint{1.292262in}{2.486303in}}%
\pgfpathlineto{\pgfqpoint{1.282872in}{2.489642in}}%
\pgfpathlineto{\pgfqpoint{1.280200in}{2.490575in}}%
\pgfpathlineto{\pgfqpoint{1.268137in}{2.496226in}}%
\pgfpathlineto{\pgfqpoint{1.259253in}{2.501394in}}%
\pgfpathlineto{\pgfqpoint{1.256074in}{2.503212in}}%
\pgfpathlineto{\pgfqpoint{1.244012in}{2.511348in}}%
\pgfpathlineto{\pgfqpoint{1.241668in}{2.513145in}}%
\pgfpathlineto{\pgfqpoint{1.231949in}{2.520487in}}%
\pgfpathlineto{\pgfqpoint{1.226673in}{2.524896in}}%
\pgfpathlineto{\pgfqpoint{1.219886in}{2.530490in}}%
\pgfpathlineto{\pgfqpoint{1.212981in}{2.536648in}}%
\pgfpathlineto{\pgfqpoint{1.207824in}{2.541187in}}%
\pgfpathlineto{\pgfqpoint{1.200106in}{2.548399in}}%
\pgfpathlineto{\pgfqpoint{1.195761in}{2.552411in}}%
\pgfpathlineto{\pgfqpoint{1.187735in}{2.560150in}}%
\pgfpathlineto{\pgfqpoint{1.183698in}{2.563999in}}%
\pgfpathlineto{\pgfqpoint{1.175649in}{2.571902in}}%
\pgfpathlineto{\pgfqpoint{1.171636in}{2.575801in}}%
\pgfpathlineto{\pgfqpoint{1.163683in}{2.583653in}}%
\pgfpathlineto{\pgfqpoint{1.159573in}{2.587673in}}%
\pgfpathlineto{\pgfqpoint{1.151701in}{2.595405in}}%
\pgfpathlineto{\pgfqpoint{1.147510in}{2.599485in}}%
\pgfpathlineto{\pgfqpoint{1.139579in}{2.607156in}}%
\pgfpathlineto{\pgfqpoint{1.135448in}{2.611120in}}%
\pgfpathlineto{\pgfqpoint{1.127196in}{2.618907in}}%
\pgfpathlineto{\pgfqpoint{1.123385in}{2.622478in}}%
\pgfpathlineto{\pgfqpoint{1.114427in}{2.630659in}}%
\pgfpathlineto{\pgfqpoint{1.111322in}{2.633475in}}%
\pgfpathlineto{\pgfqpoint{1.101134in}{2.642410in}}%
\pgfpathlineto{\pgfqpoint{1.099260in}{2.644044in}}%
\pgfpathlineto{\pgfqpoint{1.087197in}{2.654138in}}%
\pgfpathlineto{\pgfqpoint{1.087168in}{2.654161in}}%
\pgfpathlineto{\pgfqpoint{1.075134in}{2.663690in}}%
\pgfpathlineto{\pgfqpoint{1.072174in}{2.665913in}}%
\pgfpathlineto{\pgfqpoint{1.063072in}{2.672722in}}%
\pgfpathlineto{\pgfqpoint{1.056083in}{2.677664in}}%
\pgfpathlineto{\pgfqpoint{1.051009in}{2.681242in}}%
\pgfpathlineto{\pgfqpoint{1.038946in}{2.689279in}}%
\pgfpathlineto{\pgfqpoint{1.038728in}{2.689416in}}%
\pgfpathlineto{\pgfqpoint{1.026884in}{2.696813in}}%
\pgfpathlineto{\pgfqpoint{1.019551in}{2.701167in}}%
\pgfpathlineto{\pgfqpoint{1.014821in}{2.703973in}}%
\pgfpathlineto{\pgfqpoint{1.002758in}{2.710807in}}%
\pgfpathlineto{\pgfqpoint{0.998891in}{2.712918in}}%
\pgfpathlineto{\pgfqpoint{0.990696in}{2.717395in}}%
\pgfpathlineto{\pgfqpoint{0.978633in}{2.723886in}}%
\pgfpathlineto{\pgfqpoint{0.977166in}{2.724670in}}%
\pgfpathlineto{\pgfqpoint{0.966570in}{2.730339in}}%
\pgfpathlineto{\pgfqpoint{0.955504in}{2.736421in}}%
\pgfpathlineto{\pgfqpoint{0.954508in}{2.736970in}}%
\pgfpathlineto{\pgfqpoint{0.942445in}{2.743842in}}%
\pgfpathlineto{\pgfqpoint{0.935315in}{2.748172in}}%
\pgfpathlineto{\pgfqpoint{0.930382in}{2.751180in}}%
\pgfpathlineto{\pgfqpoint{0.918320in}{2.759131in}}%
\pgfpathlineto{\pgfqpoint{0.917213in}{2.759924in}}%
\pgfpathlineto{\pgfqpoint{0.906257in}{2.767818in}}%
\pgfpathlineto{\pgfqpoint{0.901406in}{2.771675in}}%
\pgfpathlineto{\pgfqpoint{0.894194in}{2.777449in}}%
\pgfpathlineto{\pgfqpoint{0.887447in}{2.783427in}}%
\pgfpathlineto{\pgfqpoint{0.882132in}{2.788171in}}%
\pgfpathlineto{\pgfqpoint{0.875034in}{2.795178in}}%
\pgfpathlineto{\pgfqpoint{0.870069in}{2.800121in}}%
\pgfpathlineto{\pgfqpoint{0.863870in}{2.806929in}}%
\pgfpathlineto{\pgfqpoint{0.858006in}{2.813429in}}%
\pgfpathlineto{\pgfqpoint{0.853692in}{2.818681in}}%
\pgfpathlineto{\pgfqpoint{0.845944in}{2.828210in}}%
\pgfpathlineto{\pgfqpoint{0.844289in}{2.830432in}}%
\pgfpathlineto{\pgfqpoint{0.835597in}{2.842183in}}%
\pgfpathlineto{\pgfqpoint{0.833881in}{2.844520in}}%
\pgfpathlineto{\pgfqpoint{0.827499in}{2.853935in}}%
\pgfpathlineto{\pgfqpoint{0.821818in}{2.862415in}}%
\pgfpathlineto{\pgfqpoint{0.819783in}{2.865686in}}%
\pgfpathlineto{\pgfqpoint{0.812509in}{2.877438in}}%
\pgfpathlineto{\pgfqpoint{0.809756in}{2.881916in}}%
\pgfpathlineto{\pgfqpoint{0.805571in}{2.889189in}}%
\pgfpathlineto{\pgfqpoint{0.798877in}{2.900940in}}%
\pgfpathlineto{\pgfqpoint{0.797693in}{2.903022in}}%
\pgfpathlineto{\pgfqpoint{0.792506in}{2.912692in}}%
\pgfpathlineto{\pgfqpoint{0.786278in}{2.924443in}}%
\pgfpathlineto{\pgfqpoint{0.785630in}{2.925664in}}%
\pgfpathlineto{\pgfqpoint{0.780322in}{2.936194in}}%
\pgfpathlineto{\pgfqpoint{0.774470in}{2.947946in}}%
\pgfpathlineto{\pgfqpoint{0.773567in}{2.949758in}}%
\pgfpathlineto{\pgfqpoint{0.768825in}{2.959697in}}%
\pgfpathlineto{\pgfqpoint{0.763275in}{2.971449in}}%
\pgfpathlineto{\pgfqpoint{0.761505in}{2.975202in}}%
\pgfpathlineto{\pgfqpoint{0.757867in}{2.983200in}}%
\pgfpathlineto{\pgfqpoint{0.749442in}{2.983200in}}%
\pgfpathlineto{\pgfqpoint{0.737379in}{2.983200in}}%
\pgfpathlineto{\pgfqpoint{0.725317in}{2.983200in}}%
\pgfpathlineto{\pgfqpoint{0.723686in}{2.983200in}}%
\pgfpathlineto{\pgfqpoint{0.725317in}{2.979100in}}%
\pgfpathlineto{\pgfqpoint{0.728405in}{2.971449in}}%
\pgfpathlineto{\pgfqpoint{0.733173in}{2.959697in}}%
\pgfpathlineto{\pgfqpoint{0.737379in}{2.949421in}}%
\pgfpathlineto{\pgfqpoint{0.737997in}{2.947946in}}%
\pgfpathlineto{\pgfqpoint{0.742899in}{2.936194in}}%
\pgfpathlineto{\pgfqpoint{0.747858in}{2.924443in}}%
\pgfpathlineto{\pgfqpoint{0.749442in}{2.920689in}}%
\pgfpathlineto{\pgfqpoint{0.752920in}{2.912692in}}%
\pgfpathlineto{\pgfqpoint{0.758065in}{2.900940in}}%
\pgfpathlineto{\pgfqpoint{0.761505in}{2.893134in}}%
\pgfpathlineto{\pgfqpoint{0.763311in}{2.889189in}}%
\pgfpathlineto{\pgfqpoint{0.768696in}{2.877438in}}%
\pgfpathlineto{\pgfqpoint{0.773567in}{2.866933in}}%
\pgfpathlineto{\pgfqpoint{0.774173in}{2.865686in}}%
\pgfpathlineto{\pgfqpoint{0.779868in}{2.853935in}}%
\pgfpathlineto{\pgfqpoint{0.785630in}{2.842220in}}%
\pgfpathlineto{\pgfqpoint{0.785649in}{2.842183in}}%
\pgfpathlineto{\pgfqpoint{0.791735in}{2.830432in}}%
\pgfpathlineto{\pgfqpoint{0.797693in}{2.819112in}}%
\pgfpathlineto{\pgfqpoint{0.797935in}{2.818681in}}%
\pgfpathlineto{\pgfqpoint{0.804515in}{2.806929in}}%
\pgfpathlineto{\pgfqpoint{0.809756in}{2.797709in}}%
\pgfpathlineto{\pgfqpoint{0.811302in}{2.795178in}}%
\pgfpathlineto{\pgfqpoint{0.818514in}{2.783427in}}%
\pgfpathlineto{\pgfqpoint{0.821818in}{2.778094in}}%
\pgfpathlineto{\pgfqpoint{0.826137in}{2.771675in}}%
\pgfpathlineto{\pgfqpoint{0.833881in}{2.760322in}}%
\pgfpathlineto{\pgfqpoint{0.834178in}{2.759924in}}%
\pgfpathlineto{\pgfqpoint{0.842987in}{2.748172in}}%
\pgfpathlineto{\pgfqpoint{0.845944in}{2.744274in}}%
\pgfpathlineto{\pgfqpoint{0.852531in}{2.736421in}}%
\pgfpathlineto{\pgfqpoint{0.858006in}{2.729969in}}%
\pgfpathlineto{\pgfqpoint{0.863017in}{2.724670in}}%
\pgfpathlineto{\pgfqpoint{0.870069in}{2.717291in}}%
\pgfpathlineto{\pgfqpoint{0.874756in}{2.712918in}}%
\pgfpathlineto{\pgfqpoint{0.882132in}{2.706102in}}%
\pgfpathlineto{\pgfqpoint{0.888149in}{2.701167in}}%
\pgfpathlineto{\pgfqpoint{0.894194in}{2.696251in}}%
\pgfpathlineto{\pgfqpoint{0.903685in}{2.689416in}}%
\pgfpathlineto{\pgfqpoint{0.906257in}{2.687577in}}%
\pgfpathlineto{\pgfqpoint{0.918320in}{2.679864in}}%
\pgfpathlineto{\pgfqpoint{0.922130in}{2.677664in}}%
\pgfpathlineto{\pgfqpoint{0.930382in}{2.672928in}}%
\pgfpathlineto{\pgfqpoint{0.942445in}{2.666622in}}%
\pgfpathlineto{\pgfqpoint{0.943881in}{2.665913in}}%
\pgfpathlineto{\pgfqpoint{0.954508in}{2.660681in}}%
\pgfpathlineto{\pgfqpoint{0.966570in}{2.655020in}}%
\pgfpathlineto{\pgfqpoint{0.968404in}{2.654161in}}%
\pgfpathlineto{\pgfqpoint{0.978633in}{2.649385in}}%
\pgfpathlineto{\pgfqpoint{0.990696in}{2.643709in}}%
\pgfpathlineto{\pgfqpoint{0.993339in}{2.642410in}}%
\pgfpathlineto{\pgfqpoint{1.002758in}{2.637784in}}%
\pgfpathlineto{\pgfqpoint{1.014821in}{2.631565in}}%
\pgfpathlineto{\pgfqpoint{1.016456in}{2.630659in}}%
\pgfpathlineto{\pgfqpoint{1.026884in}{2.624874in}}%
\pgfpathlineto{\pgfqpoint{1.036933in}{2.618907in}}%
\pgfpathlineto{\pgfqpoint{1.038946in}{2.617710in}}%
\pgfpathlineto{\pgfqpoint{1.051009in}{2.609944in}}%
\pgfpathlineto{\pgfqpoint{1.055031in}{2.607156in}}%
\pgfpathlineto{\pgfqpoint{1.063072in}{2.601564in}}%
\pgfpathlineto{\pgfqpoint{1.071333in}{2.595405in}}%
\pgfpathlineto{\pgfqpoint{1.075134in}{2.592558in}}%
\pgfpathlineto{\pgfqpoint{1.086278in}{2.583653in}}%
\pgfpathlineto{\pgfqpoint{1.087197in}{2.582915in}}%
\pgfpathlineto{\pgfqpoint{1.099260in}{2.572627in}}%
\pgfpathlineto{\pgfqpoint{1.100068in}{2.571902in}}%
\pgfpathlineto{\pgfqpoint{1.111322in}{2.561736in}}%
\pgfpathlineto{\pgfqpoint{1.113006in}{2.560150in}}%
\pgfpathlineto{\pgfqpoint{1.123385in}{2.550306in}}%
\pgfpathlineto{\pgfqpoint{1.125332in}{2.548399in}}%
\pgfpathlineto{\pgfqpoint{1.135448in}{2.538409in}}%
\pgfpathlineto{\pgfqpoint{1.137191in}{2.536648in}}%
\pgfpathlineto{\pgfqpoint{1.147510in}{2.526134in}}%
\pgfpathlineto{\pgfqpoint{1.148711in}{2.524896in}}%
\pgfpathlineto{\pgfqpoint{1.159573in}{2.513590in}}%
\pgfpathlineto{\pgfqpoint{1.160001in}{2.513145in}}%
\pgfpathlineto{\pgfqpoint{1.171166in}{2.501394in}}%
\pgfpathlineto{\pgfqpoint{1.171636in}{2.500894in}}%
\pgfpathlineto{\pgfqpoint{1.182332in}{2.489642in}}%
\pgfpathlineto{\pgfqpoint{1.183698in}{2.488188in}}%
\pgfpathlineto{\pgfqpoint{1.193620in}{2.477891in}}%
\pgfpathlineto{\pgfqpoint{1.195761in}{2.475640in}}%
\pgfpathlineto{\pgfqpoint{1.205169in}{2.466139in}}%
\pgfpathlineto{\pgfqpoint{1.207824in}{2.463421in}}%
\pgfpathlineto{\pgfqpoint{1.217163in}{2.454388in}}%
\pgfpathlineto{\pgfqpoint{1.219886in}{2.451714in}}%
\pgfpathlineto{\pgfqpoint{1.229864in}{2.442637in}}%
\pgfpathlineto{\pgfqpoint{1.231949in}{2.440709in}}%
\pgfpathlineto{\pgfqpoint{1.243679in}{2.430885in}}%
\pgfpathlineto{\pgfqpoint{1.244012in}{2.430602in}}%
\pgfpathlineto{\pgfqpoint{1.256074in}{2.421538in}}%
\pgfpathlineto{\pgfqpoint{1.259842in}{2.419134in}}%
\pgfpathlineto{\pgfqpoint{1.268137in}{2.413737in}}%
\pgfpathlineto{\pgfqpoint{1.280200in}{2.407405in}}%
\pgfpathlineto{\pgfqpoint{1.280256in}{2.407383in}}%
\pgfpathclose%
\pgfusepath{fill}%
\end{pgfscope}%
\begin{pgfscope}%
\pgfpathrectangle{\pgfqpoint{0.423750in}{1.819814in}}{\pgfqpoint{1.194205in}{1.163386in}}%
\pgfusepath{clip}%
\pgfsetbuttcap%
\pgfsetroundjoin%
\definecolor{currentfill}{rgb}{0.964433,0.670254,0.515093}%
\pgfsetfillcolor{currentfill}%
\pgfsetlinewidth{0.000000pt}%
\definecolor{currentstroke}{rgb}{0.000000,0.000000,0.000000}%
\pgfsetstrokecolor{currentstroke}%
\pgfsetdash{}{0pt}%
\pgfpathmoveto{\pgfqpoint{0.797693in}{1.827901in}}%
\pgfpathlineto{\pgfqpoint{0.803343in}{1.819814in}}%
\pgfpathlineto{\pgfqpoint{0.809756in}{1.819814in}}%
\pgfpathlineto{\pgfqpoint{0.821818in}{1.819814in}}%
\pgfpathlineto{\pgfqpoint{0.833881in}{1.819814in}}%
\pgfpathlineto{\pgfqpoint{0.845944in}{1.819814in}}%
\pgfpathlineto{\pgfqpoint{0.858006in}{1.819814in}}%
\pgfpathlineto{\pgfqpoint{0.864993in}{1.819814in}}%
\pgfpathlineto{\pgfqpoint{0.858690in}{1.831565in}}%
\pgfpathlineto{\pgfqpoint{0.858006in}{1.832845in}}%
\pgfpathlineto{\pgfqpoint{0.852209in}{1.843316in}}%
\pgfpathlineto{\pgfqpoint{0.845944in}{1.854469in}}%
\pgfpathlineto{\pgfqpoint{0.845594in}{1.855068in}}%
\pgfpathlineto{\pgfqpoint{0.838736in}{1.866819in}}%
\pgfpathlineto{\pgfqpoint{0.833881in}{1.875018in}}%
\pgfpathlineto{\pgfqpoint{0.831680in}{1.878571in}}%
\pgfpathlineto{\pgfqpoint{0.824347in}{1.890322in}}%
\pgfpathlineto{\pgfqpoint{0.821818in}{1.894339in}}%
\pgfpathlineto{\pgfqpoint{0.816708in}{1.902073in}}%
\pgfpathlineto{\pgfqpoint{0.809756in}{1.912405in}}%
\pgfpathlineto{\pgfqpoint{0.808754in}{1.913825in}}%
\pgfpathlineto{\pgfqpoint{0.800392in}{1.925576in}}%
\pgfpathlineto{\pgfqpoint{0.797693in}{1.929311in}}%
\pgfpathlineto{\pgfqpoint{0.791624in}{1.937327in}}%
\pgfpathlineto{\pgfqpoint{0.785630in}{1.945085in}}%
\pgfpathlineto{\pgfqpoint{0.782419in}{1.949079in}}%
\pgfpathlineto{\pgfqpoint{0.773567in}{1.959859in}}%
\pgfpathlineto{\pgfqpoint{0.772746in}{1.960830in}}%
\pgfpathlineto{\pgfqpoint{0.762634in}{1.972582in}}%
\pgfpathlineto{\pgfqpoint{0.761505in}{1.973869in}}%
\pgfpathlineto{\pgfqpoint{0.752163in}{1.984333in}}%
\pgfpathlineto{\pgfqpoint{0.749442in}{1.987310in}}%
\pgfpathlineto{\pgfqpoint{0.741415in}{1.996084in}}%
\pgfpathlineto{\pgfqpoint{0.737379in}{2.000393in}}%
\pgfpathlineto{\pgfqpoint{0.730527in}{2.007836in}}%
\pgfpathlineto{\pgfqpoint{0.725317in}{2.013360in}}%
\pgfpathlineto{\pgfqpoint{0.719653in}{2.019587in}}%
\pgfpathlineto{\pgfqpoint{0.713254in}{2.026451in}}%
\pgfpathlineto{\pgfqpoint{0.708939in}{2.031338in}}%
\pgfpathlineto{\pgfqpoint{0.701191in}{2.039897in}}%
\pgfpathlineto{\pgfqpoint{0.698497in}{2.043090in}}%
\pgfpathlineto{\pgfqpoint{0.689129in}{2.053911in}}%
\pgfpathlineto{\pgfqpoint{0.688387in}{2.054841in}}%
\pgfpathlineto{\pgfqpoint{0.678849in}{2.066593in}}%
\pgfpathlineto{\pgfqpoint{0.677066in}{2.068739in}}%
\pgfpathlineto{\pgfqpoint{0.669811in}{2.078344in}}%
\pgfpathlineto{\pgfqpoint{0.665003in}{2.084540in}}%
\pgfpathlineto{\pgfqpoint{0.661108in}{2.090095in}}%
\pgfpathlineto{\pgfqpoint{0.652941in}{2.101433in}}%
\pgfpathlineto{\pgfqpoint{0.652672in}{2.101847in}}%
\pgfpathlineto{\pgfqpoint{0.644966in}{2.113598in}}%
\pgfpathlineto{\pgfqpoint{0.640878in}{2.119669in}}%
\pgfpathlineto{\pgfqpoint{0.637442in}{2.125349in}}%
\pgfpathlineto{\pgfqpoint{0.630175in}{2.137101in}}%
\pgfpathlineto{\pgfqpoint{0.628815in}{2.139270in}}%
\pgfpathlineto{\pgfqpoint{0.623426in}{2.148852in}}%
\pgfpathlineto{\pgfqpoint{0.616753in}{2.160381in}}%
\pgfpathlineto{\pgfqpoint{0.616637in}{2.160604in}}%
\pgfpathlineto{\pgfqpoint{0.610487in}{2.172355in}}%
\pgfpathlineto{\pgfqpoint{0.604690in}{2.183118in}}%
\pgfpathlineto{\pgfqpoint{0.604210in}{2.184106in}}%
\pgfpathlineto{\pgfqpoint{0.598473in}{2.195858in}}%
\pgfpathlineto{\pgfqpoint{0.592627in}{2.207501in}}%
\pgfpathlineto{\pgfqpoint{0.592578in}{2.207609in}}%
\pgfpathlineto{\pgfqpoint{0.587230in}{2.219360in}}%
\pgfpathlineto{\pgfqpoint{0.581745in}{2.231112in}}%
\pgfpathlineto{\pgfqpoint{0.580565in}{2.233622in}}%
\pgfpathlineto{\pgfqpoint{0.576623in}{2.242863in}}%
\pgfpathlineto{\pgfqpoint{0.571519in}{2.254615in}}%
\pgfpathlineto{\pgfqpoint{0.568502in}{2.261466in}}%
\pgfpathlineto{\pgfqpoint{0.566534in}{2.266366in}}%
\pgfpathlineto{\pgfqpoint{0.561776in}{2.278117in}}%
\pgfpathlineto{\pgfqpoint{0.556915in}{2.289869in}}%
\pgfpathlineto{\pgfqpoint{0.556439in}{2.291017in}}%
\pgfpathlineto{\pgfqpoint{0.552417in}{2.301620in}}%
\pgfpathlineto{\pgfqpoint{0.547882in}{2.313371in}}%
\pgfpathlineto{\pgfqpoint{0.544377in}{2.322323in}}%
\pgfpathlineto{\pgfqpoint{0.543367in}{2.325123in}}%
\pgfpathlineto{\pgfqpoint{0.539121in}{2.336874in}}%
\pgfpathlineto{\pgfqpoint{0.534798in}{2.348626in}}%
\pgfpathlineto{\pgfqpoint{0.532314in}{2.355312in}}%
\pgfpathlineto{\pgfqpoint{0.530574in}{2.360377in}}%
\pgfpathlineto{\pgfqpoint{0.526517in}{2.372128in}}%
\pgfpathlineto{\pgfqpoint{0.522391in}{2.383880in}}%
\pgfpathlineto{\pgfqpoint{0.520251in}{2.389926in}}%
\pgfpathlineto{\pgfqpoint{0.518376in}{2.395631in}}%
\pgfpathlineto{\pgfqpoint{0.514492in}{2.407383in}}%
\pgfpathlineto{\pgfqpoint{0.510546in}{2.419134in}}%
\pgfpathlineto{\pgfqpoint{0.508189in}{2.426095in}}%
\pgfpathlineto{\pgfqpoint{0.506674in}{2.430885in}}%
\pgfpathlineto{\pgfqpoint{0.502947in}{2.442637in}}%
\pgfpathlineto{\pgfqpoint{0.499164in}{2.454388in}}%
\pgfpathlineto{\pgfqpoint{0.496126in}{2.463714in}}%
\pgfpathlineto{\pgfqpoint{0.495385in}{2.466139in}}%
\pgfpathlineto{\pgfqpoint{0.491798in}{2.477891in}}%
\pgfpathlineto{\pgfqpoint{0.488160in}{2.489642in}}%
\pgfpathlineto{\pgfqpoint{0.484467in}{2.501394in}}%
\pgfpathlineto{\pgfqpoint{0.484063in}{2.502680in}}%
\pgfpathlineto{\pgfqpoint{0.480973in}{2.513145in}}%
\pgfpathlineto{\pgfqpoint{0.477463in}{2.524896in}}%
\pgfpathlineto{\pgfqpoint{0.473901in}{2.536648in}}%
\pgfpathlineto{\pgfqpoint{0.472001in}{2.542884in}}%
\pgfpathlineto{\pgfqpoint{0.470410in}{2.548399in}}%
\pgfpathlineto{\pgfqpoint{0.467011in}{2.560150in}}%
\pgfpathlineto{\pgfqpoint{0.463564in}{2.571902in}}%
\pgfpathlineto{\pgfqpoint{0.460066in}{2.583653in}}%
\pgfpathlineto{\pgfqpoint{0.459938in}{2.584086in}}%
\pgfpathlineto{\pgfqpoint{0.456753in}{2.595405in}}%
\pgfpathlineto{\pgfqpoint{0.453403in}{2.607156in}}%
\pgfpathlineto{\pgfqpoint{0.450005in}{2.618907in}}%
\pgfpathlineto{\pgfqpoint{0.447875in}{2.626222in}}%
\pgfpathlineto{\pgfqpoint{0.446643in}{2.630659in}}%
\pgfpathlineto{\pgfqpoint{0.443374in}{2.642410in}}%
\pgfpathlineto{\pgfqpoint{0.440060in}{2.654161in}}%
\pgfpathlineto{\pgfqpoint{0.436696in}{2.665913in}}%
\pgfpathlineto{\pgfqpoint{0.435813in}{2.669000in}}%
\pgfpathlineto{\pgfqpoint{0.433439in}{2.677664in}}%
\pgfpathlineto{\pgfqpoint{0.430191in}{2.689416in}}%
\pgfpathlineto{\pgfqpoint{0.426896in}{2.701167in}}%
\pgfpathlineto{\pgfqpoint{0.423750in}{2.712228in}}%
\pgfpathlineto{\pgfqpoint{0.423750in}{2.701167in}}%
\pgfpathlineto{\pgfqpoint{0.423750in}{2.689416in}}%
\pgfpathlineto{\pgfqpoint{0.423750in}{2.677664in}}%
\pgfpathlineto{\pgfqpoint{0.423750in}{2.665913in}}%
\pgfpathlineto{\pgfqpoint{0.423750in}{2.654161in}}%
\pgfpathlineto{\pgfqpoint{0.423750in}{2.642410in}}%
\pgfpathlineto{\pgfqpoint{0.423750in}{2.630659in}}%
\pgfpathlineto{\pgfqpoint{0.423750in}{2.619578in}}%
\pgfpathlineto{\pgfqpoint{0.423947in}{2.618907in}}%
\pgfpathlineto{\pgfqpoint{0.427410in}{2.607156in}}%
\pgfpathlineto{\pgfqpoint{0.430826in}{2.595405in}}%
\pgfpathlineto{\pgfqpoint{0.434196in}{2.583653in}}%
\pgfpathlineto{\pgfqpoint{0.435813in}{2.577994in}}%
\pgfpathlineto{\pgfqpoint{0.437623in}{2.571902in}}%
\pgfpathlineto{\pgfqpoint{0.441097in}{2.560150in}}%
\pgfpathlineto{\pgfqpoint{0.444525in}{2.548399in}}%
\pgfpathlineto{\pgfqpoint{0.447875in}{2.536766in}}%
\pgfpathlineto{\pgfqpoint{0.447911in}{2.536648in}}%
\pgfpathlineto{\pgfqpoint{0.451458in}{2.524896in}}%
\pgfpathlineto{\pgfqpoint{0.454958in}{2.513145in}}%
\pgfpathlineto{\pgfqpoint{0.458413in}{2.501394in}}%
\pgfpathlineto{\pgfqpoint{0.459938in}{2.496192in}}%
\pgfpathlineto{\pgfqpoint{0.461949in}{2.489642in}}%
\pgfpathlineto{\pgfqpoint{0.465536in}{2.477891in}}%
\pgfpathlineto{\pgfqpoint{0.469077in}{2.466139in}}%
\pgfpathlineto{\pgfqpoint{0.472001in}{2.456339in}}%
\pgfpathlineto{\pgfqpoint{0.472613in}{2.454388in}}%
\pgfpathlineto{\pgfqpoint{0.476303in}{2.442637in}}%
\pgfpathlineto{\pgfqpoint{0.479944in}{2.430885in}}%
\pgfpathlineto{\pgfqpoint{0.483538in}{2.419134in}}%
\pgfpathlineto{\pgfqpoint{0.484063in}{2.417420in}}%
\pgfpathlineto{\pgfqpoint{0.487309in}{2.407383in}}%
\pgfpathlineto{\pgfqpoint{0.491066in}{2.395631in}}%
\pgfpathlineto{\pgfqpoint{0.494773in}{2.383880in}}%
\pgfpathlineto{\pgfqpoint{0.496126in}{2.379580in}}%
\pgfpathlineto{\pgfqpoint{0.498612in}{2.372128in}}%
\pgfpathlineto{\pgfqpoint{0.502501in}{2.360377in}}%
\pgfpathlineto{\pgfqpoint{0.506336in}{2.348626in}}%
\pgfpathlineto{\pgfqpoint{0.508189in}{2.342922in}}%
\pgfpathlineto{\pgfqpoint{0.510279in}{2.336874in}}%
\pgfpathlineto{\pgfqpoint{0.514316in}{2.325123in}}%
\pgfpathlineto{\pgfqpoint{0.518295in}{2.313371in}}%
\pgfpathlineto{\pgfqpoint{0.520251in}{2.307562in}}%
\pgfpathlineto{\pgfqpoint{0.522390in}{2.301620in}}%
\pgfpathlineto{\pgfqpoint{0.526590in}{2.289869in}}%
\pgfpathlineto{\pgfqpoint{0.530727in}{2.278117in}}%
\pgfpathlineto{\pgfqpoint{0.532314in}{2.273591in}}%
\pgfpathlineto{\pgfqpoint{0.535032in}{2.266366in}}%
\pgfpathlineto{\pgfqpoint{0.539411in}{2.254615in}}%
\pgfpathlineto{\pgfqpoint{0.543721in}{2.242863in}}%
\pgfpathlineto{\pgfqpoint{0.544377in}{2.241073in}}%
\pgfpathlineto{\pgfqpoint{0.548309in}{2.231112in}}%
\pgfpathlineto{\pgfqpoint{0.552880in}{2.219360in}}%
\pgfpathlineto{\pgfqpoint{0.556439in}{2.210088in}}%
\pgfpathlineto{\pgfqpoint{0.557469in}{2.207609in}}%
\pgfpathlineto{\pgfqpoint{0.562332in}{2.195858in}}%
\pgfpathlineto{\pgfqpoint{0.567106in}{2.184106in}}%
\pgfpathlineto{\pgfqpoint{0.568502in}{2.180654in}}%
\pgfpathlineto{\pgfqpoint{0.572143in}{2.172355in}}%
\pgfpathlineto{\pgfqpoint{0.577226in}{2.160604in}}%
\pgfpathlineto{\pgfqpoint{0.580565in}{2.152781in}}%
\pgfpathlineto{\pgfqpoint{0.582392in}{2.148852in}}%
\pgfpathlineto{\pgfqpoint{0.587811in}{2.137101in}}%
\pgfpathlineto{\pgfqpoint{0.592627in}{2.126452in}}%
\pgfpathlineto{\pgfqpoint{0.593172in}{2.125349in}}%
\pgfpathlineto{\pgfqpoint{0.598955in}{2.113598in}}%
\pgfpathlineto{\pgfqpoint{0.604604in}{2.101847in}}%
\pgfpathlineto{\pgfqpoint{0.604690in}{2.101667in}}%
\pgfpathlineto{\pgfqpoint{0.610761in}{2.090095in}}%
\pgfpathlineto{\pgfqpoint{0.616753in}{2.078382in}}%
\pgfpathlineto{\pgfqpoint{0.616774in}{2.078344in}}%
\pgfpathlineto{\pgfqpoint{0.623339in}{2.066593in}}%
\pgfpathlineto{\pgfqpoint{0.628815in}{2.056545in}}%
\pgfpathlineto{\pgfqpoint{0.629834in}{2.054841in}}%
\pgfpathlineto{\pgfqpoint{0.636796in}{2.043090in}}%
\pgfpathlineto{\pgfqpoint{0.640878in}{2.036059in}}%
\pgfpathlineto{\pgfqpoint{0.643880in}{2.031338in}}%
\pgfpathlineto{\pgfqpoint{0.651223in}{2.019587in}}%
\pgfpathlineto{\pgfqpoint{0.652941in}{2.016803in}}%
\pgfpathlineto{\pgfqpoint{0.658981in}{2.007836in}}%
\pgfpathlineto{\pgfqpoint{0.665003in}{1.998688in}}%
\pgfpathlineto{\pgfqpoint{0.666868in}{1.996084in}}%
\pgfpathlineto{\pgfqpoint{0.675152in}{1.984333in}}%
\pgfpathlineto{\pgfqpoint{0.677066in}{1.981575in}}%
\pgfpathlineto{\pgfqpoint{0.683804in}{1.972582in}}%
\pgfpathlineto{\pgfqpoint{0.689129in}{1.965318in}}%
\pgfpathlineto{\pgfqpoint{0.692652in}{1.960830in}}%
\pgfpathlineto{\pgfqpoint{0.701191in}{1.949716in}}%
\pgfpathlineto{\pgfqpoint{0.701711in}{1.949079in}}%
\pgfpathlineto{\pgfqpoint{0.711153in}{1.937327in}}%
\pgfpathlineto{\pgfqpoint{0.713254in}{1.934663in}}%
\pgfpathlineto{\pgfqpoint{0.720750in}{1.925576in}}%
\pgfpathlineto{\pgfqpoint{0.725317in}{1.919927in}}%
\pgfpathlineto{\pgfqpoint{0.730411in}{1.913825in}}%
\pgfpathlineto{\pgfqpoint{0.737379in}{1.905309in}}%
\pgfpathlineto{\pgfqpoint{0.740077in}{1.902073in}}%
\pgfpathlineto{\pgfqpoint{0.749442in}{1.890617in}}%
\pgfpathlineto{\pgfqpoint{0.749685in}{1.890322in}}%
\pgfpathlineto{\pgfqpoint{0.759234in}{1.878571in}}%
\pgfpathlineto{\pgfqpoint{0.761505in}{1.875727in}}%
\pgfpathlineto{\pgfqpoint{0.768571in}{1.866819in}}%
\pgfpathlineto{\pgfqpoint{0.773567in}{1.860408in}}%
\pgfpathlineto{\pgfqpoint{0.777659in}{1.855068in}}%
\pgfpathlineto{\pgfqpoint{0.785630in}{1.844486in}}%
\pgfpathlineto{\pgfqpoint{0.786489in}{1.843316in}}%
\pgfpathlineto{\pgfqpoint{0.795057in}{1.831565in}}%
\pgfpathclose%
\pgfusepath{fill}%
\end{pgfscope}%
\begin{pgfscope}%
\pgfpathrectangle{\pgfqpoint{0.423750in}{1.819814in}}{\pgfqpoint{1.194205in}{1.163386in}}%
\pgfusepath{clip}%
\pgfsetbuttcap%
\pgfsetroundjoin%
\definecolor{currentfill}{rgb}{0.964433,0.670254,0.515093}%
\pgfsetfillcolor{currentfill}%
\pgfsetlinewidth{0.000000pt}%
\definecolor{currentstroke}{rgb}{0.000000,0.000000,0.000000}%
\pgfsetstrokecolor{currentstroke}%
\pgfsetdash{}{0pt}%
\pgfpathmoveto{\pgfqpoint{1.292262in}{2.301553in}}%
\pgfpathlineto{\pgfqpoint{1.304325in}{2.297724in}}%
\pgfpathlineto{\pgfqpoint{1.316388in}{2.296015in}}%
\pgfpathlineto{\pgfqpoint{1.328450in}{2.296457in}}%
\pgfpathlineto{\pgfqpoint{1.340513in}{2.299054in}}%
\pgfpathlineto{\pgfqpoint{1.347064in}{2.301620in}}%
\pgfpathlineto{\pgfqpoint{1.352576in}{2.303723in}}%
\pgfpathlineto{\pgfqpoint{1.364638in}{2.310345in}}%
\pgfpathlineto{\pgfqpoint{1.368908in}{2.313371in}}%
\pgfpathlineto{\pgfqpoint{1.376701in}{2.318767in}}%
\pgfpathlineto{\pgfqpoint{1.384274in}{2.325123in}}%
\pgfpathlineto{\pgfqpoint{1.388764in}{2.328808in}}%
\pgfpathlineto{\pgfqpoint{1.397234in}{2.336874in}}%
\pgfpathlineto{\pgfqpoint{1.400826in}{2.340226in}}%
\pgfpathlineto{\pgfqpoint{1.408846in}{2.348626in}}%
\pgfpathlineto{\pgfqpoint{1.412889in}{2.352781in}}%
\pgfpathlineto{\pgfqpoint{1.419644in}{2.360377in}}%
\pgfpathlineto{\pgfqpoint{1.424952in}{2.366241in}}%
\pgfpathlineto{\pgfqpoint{1.429928in}{2.372128in}}%
\pgfpathlineto{\pgfqpoint{1.437014in}{2.380378in}}%
\pgfpathlineto{\pgfqpoint{1.439876in}{2.383880in}}%
\pgfpathlineto{\pgfqpoint{1.449077in}{2.394973in}}%
\pgfpathlineto{\pgfqpoint{1.449605in}{2.395631in}}%
\pgfpathlineto{\pgfqpoint{1.459177in}{2.407383in}}%
\pgfpathlineto{\pgfqpoint{1.461140in}{2.409758in}}%
\pgfpathlineto{\pgfqpoint{1.468735in}{2.419134in}}%
\pgfpathlineto{\pgfqpoint{1.473202in}{2.424582in}}%
\pgfpathlineto{\pgfqpoint{1.478341in}{2.430885in}}%
\pgfpathlineto{\pgfqpoint{1.485265in}{2.439284in}}%
\pgfpathlineto{\pgfqpoint{1.488048in}{2.442637in}}%
\pgfpathlineto{\pgfqpoint{1.497328in}{2.453704in}}%
\pgfpathlineto{\pgfqpoint{1.497913in}{2.454388in}}%
\pgfpathlineto{\pgfqpoint{1.508063in}{2.466139in}}%
\pgfpathlineto{\pgfqpoint{1.509390in}{2.467662in}}%
\pgfpathlineto{\pgfqpoint{1.518576in}{2.477891in}}%
\pgfpathlineto{\pgfqpoint{1.521453in}{2.481070in}}%
\pgfpathlineto{\pgfqpoint{1.529530in}{2.489642in}}%
\pgfpathlineto{\pgfqpoint{1.533516in}{2.493843in}}%
\pgfpathlineto{\pgfqpoint{1.541051in}{2.501394in}}%
\pgfpathlineto{\pgfqpoint{1.545579in}{2.505904in}}%
\pgfpathlineto{\pgfqpoint{1.553299in}{2.513145in}}%
\pgfpathlineto{\pgfqpoint{1.557641in}{2.517197in}}%
\pgfpathlineto{\pgfqpoint{1.566482in}{2.524896in}}%
\pgfpathlineto{\pgfqpoint{1.569704in}{2.527691in}}%
\pgfpathlineto{\pgfqpoint{1.580856in}{2.536648in}}%
\pgfpathlineto{\pgfqpoint{1.581767in}{2.537377in}}%
\pgfpathlineto{\pgfqpoint{1.593829in}{2.546222in}}%
\pgfpathlineto{\pgfqpoint{1.597084in}{2.548399in}}%
\pgfpathlineto{\pgfqpoint{1.605892in}{2.554280in}}%
\pgfpathlineto{\pgfqpoint{1.615524in}{2.560150in}}%
\pgfpathlineto{\pgfqpoint{1.617955in}{2.561630in}}%
\pgfpathlineto{\pgfqpoint{1.617955in}{2.571902in}}%
\pgfpathlineto{\pgfqpoint{1.617955in}{2.583653in}}%
\pgfpathlineto{\pgfqpoint{1.617955in}{2.595405in}}%
\pgfpathlineto{\pgfqpoint{1.617955in}{2.607156in}}%
\pgfpathlineto{\pgfqpoint{1.617955in}{2.618907in}}%
\pgfpathlineto{\pgfqpoint{1.617955in}{2.630659in}}%
\pgfpathlineto{\pgfqpoint{1.617955in}{2.642410in}}%
\pgfpathlineto{\pgfqpoint{1.617955in}{2.643671in}}%
\pgfpathlineto{\pgfqpoint{1.616032in}{2.642410in}}%
\pgfpathlineto{\pgfqpoint{1.605892in}{2.635757in}}%
\pgfpathlineto{\pgfqpoint{1.598620in}{2.630659in}}%
\pgfpathlineto{\pgfqpoint{1.593829in}{2.627295in}}%
\pgfpathlineto{\pgfqpoint{1.582665in}{2.618907in}}%
\pgfpathlineto{\pgfqpoint{1.581767in}{2.618231in}}%
\pgfpathlineto{\pgfqpoint{1.569704in}{2.608505in}}%
\pgfpathlineto{\pgfqpoint{1.568131in}{2.607156in}}%
\pgfpathlineto{\pgfqpoint{1.557641in}{2.598129in}}%
\pgfpathlineto{\pgfqpoint{1.554644in}{2.595405in}}%
\pgfpathlineto{\pgfqpoint{1.545579in}{2.587128in}}%
\pgfpathlineto{\pgfqpoint{1.541951in}{2.583653in}}%
\pgfpathlineto{\pgfqpoint{1.533516in}{2.575531in}}%
\pgfpathlineto{\pgfqpoint{1.529896in}{2.571902in}}%
\pgfpathlineto{\pgfqpoint{1.521453in}{2.563388in}}%
\pgfpathlineto{\pgfqpoint{1.518342in}{2.560150in}}%
\pgfpathlineto{\pgfqpoint{1.509390in}{2.550773in}}%
\pgfpathlineto{\pgfqpoint{1.507174in}{2.548399in}}%
\pgfpathlineto{\pgfqpoint{1.497328in}{2.537775in}}%
\pgfpathlineto{\pgfqpoint{1.496295in}{2.536648in}}%
\pgfpathlineto{\pgfqpoint{1.485625in}{2.524896in}}%
\pgfpathlineto{\pgfqpoint{1.485265in}{2.524496in}}%
\pgfpathlineto{\pgfqpoint{1.475071in}{2.513145in}}%
\pgfpathlineto{\pgfqpoint{1.473202in}{2.511044in}}%
\pgfpathlineto{\pgfqpoint{1.464528in}{2.501394in}}%
\pgfpathlineto{\pgfqpoint{1.461140in}{2.497584in}}%
\pgfpathlineto{\pgfqpoint{1.453908in}{2.489642in}}%
\pgfpathlineto{\pgfqpoint{1.449077in}{2.484276in}}%
\pgfpathlineto{\pgfqpoint{1.443110in}{2.477891in}}%
\pgfpathlineto{\pgfqpoint{1.437014in}{2.471287in}}%
\pgfpathlineto{\pgfqpoint{1.432004in}{2.466139in}}%
\pgfpathlineto{\pgfqpoint{1.424952in}{2.458796in}}%
\pgfpathlineto{\pgfqpoint{1.420408in}{2.454388in}}%
\pgfpathlineto{\pgfqpoint{1.412889in}{2.446989in}}%
\pgfpathlineto{\pgfqpoint{1.408042in}{2.442637in}}%
\pgfpathlineto{\pgfqpoint{1.400826in}{2.436056in}}%
\pgfpathlineto{\pgfqpoint{1.394454in}{2.430885in}}%
\pgfpathlineto{\pgfqpoint{1.388764in}{2.426191in}}%
\pgfpathlineto{\pgfqpoint{1.378837in}{2.419134in}}%
\pgfpathlineto{\pgfqpoint{1.376701in}{2.417588in}}%
\pgfpathlineto{\pgfqpoint{1.364638in}{2.410371in}}%
\pgfpathlineto{\pgfqpoint{1.358197in}{2.407383in}}%
\pgfpathlineto{\pgfqpoint{1.352576in}{2.404722in}}%
\pgfpathlineto{\pgfqpoint{1.340513in}{2.400733in}}%
\pgfpathlineto{\pgfqpoint{1.328450in}{2.398516in}}%
\pgfpathlineto{\pgfqpoint{1.316388in}{2.398089in}}%
\pgfpathlineto{\pgfqpoint{1.304325in}{2.399448in}}%
\pgfpathlineto{\pgfqpoint{1.292262in}{2.402569in}}%
\pgfpathlineto{\pgfqpoint{1.280256in}{2.407383in}}%
\pgfpathlineto{\pgfqpoint{1.280200in}{2.407405in}}%
\pgfpathlineto{\pgfqpoint{1.268137in}{2.413737in}}%
\pgfpathlineto{\pgfqpoint{1.259842in}{2.419134in}}%
\pgfpathlineto{\pgfqpoint{1.256074in}{2.421538in}}%
\pgfpathlineto{\pgfqpoint{1.244012in}{2.430602in}}%
\pgfpathlineto{\pgfqpoint{1.243679in}{2.430885in}}%
\pgfpathlineto{\pgfqpoint{1.231949in}{2.440709in}}%
\pgfpathlineto{\pgfqpoint{1.229864in}{2.442637in}}%
\pgfpathlineto{\pgfqpoint{1.219886in}{2.451714in}}%
\pgfpathlineto{\pgfqpoint{1.217163in}{2.454388in}}%
\pgfpathlineto{\pgfqpoint{1.207824in}{2.463421in}}%
\pgfpathlineto{\pgfqpoint{1.205169in}{2.466139in}}%
\pgfpathlineto{\pgfqpoint{1.195761in}{2.475640in}}%
\pgfpathlineto{\pgfqpoint{1.193620in}{2.477891in}}%
\pgfpathlineto{\pgfqpoint{1.183698in}{2.488188in}}%
\pgfpathlineto{\pgfqpoint{1.182332in}{2.489642in}}%
\pgfpathlineto{\pgfqpoint{1.171636in}{2.500894in}}%
\pgfpathlineto{\pgfqpoint{1.171166in}{2.501394in}}%
\pgfpathlineto{\pgfqpoint{1.160001in}{2.513145in}}%
\pgfpathlineto{\pgfqpoint{1.159573in}{2.513590in}}%
\pgfpathlineto{\pgfqpoint{1.148711in}{2.524896in}}%
\pgfpathlineto{\pgfqpoint{1.147510in}{2.526134in}}%
\pgfpathlineto{\pgfqpoint{1.137191in}{2.536648in}}%
\pgfpathlineto{\pgfqpoint{1.135448in}{2.538409in}}%
\pgfpathlineto{\pgfqpoint{1.125332in}{2.548399in}}%
\pgfpathlineto{\pgfqpoint{1.123385in}{2.550306in}}%
\pgfpathlineto{\pgfqpoint{1.113006in}{2.560150in}}%
\pgfpathlineto{\pgfqpoint{1.111322in}{2.561736in}}%
\pgfpathlineto{\pgfqpoint{1.100068in}{2.571902in}}%
\pgfpathlineto{\pgfqpoint{1.099260in}{2.572627in}}%
\pgfpathlineto{\pgfqpoint{1.087197in}{2.582915in}}%
\pgfpathlineto{\pgfqpoint{1.086278in}{2.583653in}}%
\pgfpathlineto{\pgfqpoint{1.075134in}{2.592558in}}%
\pgfpathlineto{\pgfqpoint{1.071333in}{2.595405in}}%
\pgfpathlineto{\pgfqpoint{1.063072in}{2.601564in}}%
\pgfpathlineto{\pgfqpoint{1.055031in}{2.607156in}}%
\pgfpathlineto{\pgfqpoint{1.051009in}{2.609944in}}%
\pgfpathlineto{\pgfqpoint{1.038946in}{2.617710in}}%
\pgfpathlineto{\pgfqpoint{1.036933in}{2.618907in}}%
\pgfpathlineto{\pgfqpoint{1.026884in}{2.624874in}}%
\pgfpathlineto{\pgfqpoint{1.016456in}{2.630659in}}%
\pgfpathlineto{\pgfqpoint{1.014821in}{2.631565in}}%
\pgfpathlineto{\pgfqpoint{1.002758in}{2.637784in}}%
\pgfpathlineto{\pgfqpoint{0.993339in}{2.642410in}}%
\pgfpathlineto{\pgfqpoint{0.990696in}{2.643709in}}%
\pgfpathlineto{\pgfqpoint{0.978633in}{2.649385in}}%
\pgfpathlineto{\pgfqpoint{0.968404in}{2.654161in}}%
\pgfpathlineto{\pgfqpoint{0.966570in}{2.655020in}}%
\pgfpathlineto{\pgfqpoint{0.954508in}{2.660681in}}%
\pgfpathlineto{\pgfqpoint{0.943881in}{2.665913in}}%
\pgfpathlineto{\pgfqpoint{0.942445in}{2.666622in}}%
\pgfpathlineto{\pgfqpoint{0.930382in}{2.672928in}}%
\pgfpathlineto{\pgfqpoint{0.922130in}{2.677664in}}%
\pgfpathlineto{\pgfqpoint{0.918320in}{2.679864in}}%
\pgfpathlineto{\pgfqpoint{0.906257in}{2.687577in}}%
\pgfpathlineto{\pgfqpoint{0.903685in}{2.689416in}}%
\pgfpathlineto{\pgfqpoint{0.894194in}{2.696251in}}%
\pgfpathlineto{\pgfqpoint{0.888149in}{2.701167in}}%
\pgfpathlineto{\pgfqpoint{0.882132in}{2.706102in}}%
\pgfpathlineto{\pgfqpoint{0.874756in}{2.712918in}}%
\pgfpathlineto{\pgfqpoint{0.870069in}{2.717291in}}%
\pgfpathlineto{\pgfqpoint{0.863017in}{2.724670in}}%
\pgfpathlineto{\pgfqpoint{0.858006in}{2.729969in}}%
\pgfpathlineto{\pgfqpoint{0.852531in}{2.736421in}}%
\pgfpathlineto{\pgfqpoint{0.845944in}{2.744274in}}%
\pgfpathlineto{\pgfqpoint{0.842987in}{2.748172in}}%
\pgfpathlineto{\pgfqpoint{0.834178in}{2.759924in}}%
\pgfpathlineto{\pgfqpoint{0.833881in}{2.760322in}}%
\pgfpathlineto{\pgfqpoint{0.826137in}{2.771675in}}%
\pgfpathlineto{\pgfqpoint{0.821818in}{2.778094in}}%
\pgfpathlineto{\pgfqpoint{0.818514in}{2.783427in}}%
\pgfpathlineto{\pgfqpoint{0.811302in}{2.795178in}}%
\pgfpathlineto{\pgfqpoint{0.809756in}{2.797709in}}%
\pgfpathlineto{\pgfqpoint{0.804515in}{2.806929in}}%
\pgfpathlineto{\pgfqpoint{0.797935in}{2.818681in}}%
\pgfpathlineto{\pgfqpoint{0.797693in}{2.819112in}}%
\pgfpathlineto{\pgfqpoint{0.791735in}{2.830432in}}%
\pgfpathlineto{\pgfqpoint{0.785649in}{2.842183in}}%
\pgfpathlineto{\pgfqpoint{0.785630in}{2.842220in}}%
\pgfpathlineto{\pgfqpoint{0.779868in}{2.853935in}}%
\pgfpathlineto{\pgfqpoint{0.774173in}{2.865686in}}%
\pgfpathlineto{\pgfqpoint{0.773567in}{2.866933in}}%
\pgfpathlineto{\pgfqpoint{0.768696in}{2.877438in}}%
\pgfpathlineto{\pgfqpoint{0.763311in}{2.889189in}}%
\pgfpathlineto{\pgfqpoint{0.761505in}{2.893134in}}%
\pgfpathlineto{\pgfqpoint{0.758065in}{2.900940in}}%
\pgfpathlineto{\pgfqpoint{0.752920in}{2.912692in}}%
\pgfpathlineto{\pgfqpoint{0.749442in}{2.920689in}}%
\pgfpathlineto{\pgfqpoint{0.747858in}{2.924443in}}%
\pgfpathlineto{\pgfqpoint{0.742899in}{2.936194in}}%
\pgfpathlineto{\pgfqpoint{0.737997in}{2.947946in}}%
\pgfpathlineto{\pgfqpoint{0.737379in}{2.949421in}}%
\pgfpathlineto{\pgfqpoint{0.733173in}{2.959697in}}%
\pgfpathlineto{\pgfqpoint{0.728405in}{2.971449in}}%
\pgfpathlineto{\pgfqpoint{0.725317in}{2.979100in}}%
\pgfpathlineto{\pgfqpoint{0.723686in}{2.983200in}}%
\pgfpathlineto{\pgfqpoint{0.713254in}{2.983200in}}%
\pgfpathlineto{\pgfqpoint{0.701191in}{2.983200in}}%
\pgfpathlineto{\pgfqpoint{0.689260in}{2.983200in}}%
\pgfpathlineto{\pgfqpoint{0.693510in}{2.971449in}}%
\pgfpathlineto{\pgfqpoint{0.697807in}{2.959697in}}%
\pgfpathlineto{\pgfqpoint{0.701191in}{2.950505in}}%
\pgfpathlineto{\pgfqpoint{0.702132in}{2.947946in}}%
\pgfpathlineto{\pgfqpoint{0.706437in}{2.936194in}}%
\pgfpathlineto{\pgfqpoint{0.710787in}{2.924443in}}%
\pgfpathlineto{\pgfqpoint{0.713254in}{2.917799in}}%
\pgfpathlineto{\pgfqpoint{0.715162in}{2.912692in}}%
\pgfpathlineto{\pgfqpoint{0.719553in}{2.900940in}}%
\pgfpathlineto{\pgfqpoint{0.723991in}{2.889189in}}%
\pgfpathlineto{\pgfqpoint{0.725317in}{2.885671in}}%
\pgfpathlineto{\pgfqpoint{0.728464in}{2.877438in}}%
\pgfpathlineto{\pgfqpoint{0.732979in}{2.865686in}}%
\pgfpathlineto{\pgfqpoint{0.737379in}{2.854358in}}%
\pgfpathlineto{\pgfqpoint{0.737547in}{2.853935in}}%
\pgfpathlineto{\pgfqpoint{0.742182in}{2.842183in}}%
\pgfpathlineto{\pgfqpoint{0.746872in}{2.830432in}}%
\pgfpathlineto{\pgfqpoint{0.749442in}{2.824012in}}%
\pgfpathlineto{\pgfqpoint{0.751645in}{2.818681in}}%
\pgfpathlineto{\pgfqpoint{0.756508in}{2.806929in}}%
\pgfpathlineto{\pgfqpoint{0.761435in}{2.795178in}}%
\pgfpathlineto{\pgfqpoint{0.761505in}{2.795011in}}%
\pgfpathlineto{\pgfqpoint{0.766530in}{2.783427in}}%
\pgfpathlineto{\pgfqpoint{0.771700in}{2.771675in}}%
\pgfpathlineto{\pgfqpoint{0.773567in}{2.767435in}}%
\pgfpathlineto{\pgfqpoint{0.777049in}{2.759924in}}%
\pgfpathlineto{\pgfqpoint{0.782540in}{2.748172in}}%
\pgfpathlineto{\pgfqpoint{0.785630in}{2.741605in}}%
\pgfpathlineto{\pgfqpoint{0.788224in}{2.736421in}}%
\pgfpathlineto{\pgfqpoint{0.794138in}{2.724670in}}%
\pgfpathlineto{\pgfqpoint{0.797693in}{2.717673in}}%
\pgfpathlineto{\pgfqpoint{0.800291in}{2.712918in}}%
\pgfpathlineto{\pgfqpoint{0.806760in}{2.701167in}}%
\pgfpathlineto{\pgfqpoint{0.809756in}{2.695772in}}%
\pgfpathlineto{\pgfqpoint{0.813599in}{2.689416in}}%
\pgfpathlineto{\pgfqpoint{0.820805in}{2.677664in}}%
\pgfpathlineto{\pgfqpoint{0.821818in}{2.676016in}}%
\pgfpathlineto{\pgfqpoint{0.828674in}{2.665913in}}%
\pgfpathlineto{\pgfqpoint{0.833881in}{2.658370in}}%
\pgfpathlineto{\pgfqpoint{0.837124in}{2.654161in}}%
\pgfpathlineto{\pgfqpoint{0.845944in}{2.642897in}}%
\pgfpathlineto{\pgfqpoint{0.846374in}{2.642410in}}%
\pgfpathlineto{\pgfqpoint{0.856879in}{2.630659in}}%
\pgfpathlineto{\pgfqpoint{0.858006in}{2.629414in}}%
\pgfpathlineto{\pgfqpoint{0.868940in}{2.618907in}}%
\pgfpathlineto{\pgfqpoint{0.870069in}{2.617837in}}%
\pgfpathlineto{\pgfqpoint{0.882132in}{2.607973in}}%
\pgfpathlineto{\pgfqpoint{0.883295in}{2.607156in}}%
\pgfpathlineto{\pgfqpoint{0.894194in}{2.599580in}}%
\pgfpathlineto{\pgfqpoint{0.901274in}{2.595405in}}%
\pgfpathlineto{\pgfqpoint{0.906257in}{2.592492in}}%
\pgfpathlineto{\pgfqpoint{0.918320in}{2.586429in}}%
\pgfpathlineto{\pgfqpoint{0.924653in}{2.583653in}}%
\pgfpathlineto{\pgfqpoint{0.930382in}{2.581159in}}%
\pgfpathlineto{\pgfqpoint{0.942445in}{2.576436in}}%
\pgfpathlineto{\pgfqpoint{0.954508in}{2.572097in}}%
\pgfpathlineto{\pgfqpoint{0.955058in}{2.571902in}}%
\pgfpathlineto{\pgfqpoint{0.966570in}{2.567830in}}%
\pgfpathlineto{\pgfqpoint{0.978633in}{2.563547in}}%
\pgfpathlineto{\pgfqpoint{0.987765in}{2.560150in}}%
\pgfpathlineto{\pgfqpoint{0.990696in}{2.559062in}}%
\pgfpathlineto{\pgfqpoint{1.002758in}{2.554174in}}%
\pgfpathlineto{\pgfqpoint{1.014821in}{2.548849in}}%
\pgfpathlineto{\pgfqpoint{1.015730in}{2.548399in}}%
\pgfpathlineto{\pgfqpoint{1.026884in}{2.542872in}}%
\pgfpathlineto{\pgfqpoint{1.038265in}{2.536648in}}%
\pgfpathlineto{\pgfqpoint{1.038946in}{2.536274in}}%
\pgfpathlineto{\pgfqpoint{1.051009in}{2.528877in}}%
\pgfpathlineto{\pgfqpoint{1.056906in}{2.524896in}}%
\pgfpathlineto{\pgfqpoint{1.063072in}{2.520717in}}%
\pgfpathlineto{\pgfqpoint{1.073279in}{2.513145in}}%
\pgfpathlineto{\pgfqpoint{1.075134in}{2.511762in}}%
\pgfpathlineto{\pgfqpoint{1.087197in}{2.501979in}}%
\pgfpathlineto{\pgfqpoint{1.087867in}{2.501394in}}%
\pgfpathlineto{\pgfqpoint{1.099260in}{2.491376in}}%
\pgfpathlineto{\pgfqpoint{1.101112in}{2.489642in}}%
\pgfpathlineto{\pgfqpoint{1.111322in}{2.480014in}}%
\pgfpathlineto{\pgfqpoint{1.113459in}{2.477891in}}%
\pgfpathlineto{\pgfqpoint{1.123385in}{2.467945in}}%
\pgfpathlineto{\pgfqpoint{1.125114in}{2.466139in}}%
\pgfpathlineto{\pgfqpoint{1.135448in}{2.455245in}}%
\pgfpathlineto{\pgfqpoint{1.136237in}{2.454388in}}%
\pgfpathlineto{\pgfqpoint{1.146936in}{2.442637in}}%
\pgfpathlineto{\pgfqpoint{1.147510in}{2.441999in}}%
\pgfpathlineto{\pgfqpoint{1.157324in}{2.430885in}}%
\pgfpathlineto{\pgfqpoint{1.159573in}{2.428309in}}%
\pgfpathlineto{\pgfqpoint{1.167530in}{2.419134in}}%
\pgfpathlineto{\pgfqpoint{1.171636in}{2.414338in}}%
\pgfpathlineto{\pgfqpoint{1.177632in}{2.407383in}}%
\pgfpathlineto{\pgfqpoint{1.183698in}{2.400247in}}%
\pgfpathlineto{\pgfqpoint{1.187706in}{2.395631in}}%
\pgfpathlineto{\pgfqpoint{1.195761in}{2.386214in}}%
\pgfpathlineto{\pgfqpoint{1.197832in}{2.383880in}}%
\pgfpathlineto{\pgfqpoint{1.207824in}{2.372432in}}%
\pgfpathlineto{\pgfqpoint{1.208103in}{2.372128in}}%
\pgfpathlineto{\pgfqpoint{1.218729in}{2.360377in}}%
\pgfpathlineto{\pgfqpoint{1.219886in}{2.359073in}}%
\pgfpathlineto{\pgfqpoint{1.229864in}{2.348626in}}%
\pgfpathlineto{\pgfqpoint{1.231949in}{2.346399in}}%
\pgfpathlineto{\pgfqpoint{1.241768in}{2.336874in}}%
\pgfpathlineto{\pgfqpoint{1.244012in}{2.334652in}}%
\pgfpathlineto{\pgfqpoint{1.254903in}{2.325123in}}%
\pgfpathlineto{\pgfqpoint{1.256074in}{2.324075in}}%
\pgfpathlineto{\pgfqpoint{1.268137in}{2.314876in}}%
\pgfpathlineto{\pgfqpoint{1.270573in}{2.313371in}}%
\pgfpathlineto{\pgfqpoint{1.280200in}{2.307275in}}%
\pgfpathlineto{\pgfqpoint{1.292124in}{2.301620in}}%
\pgfpathclose%
\pgfusepath{fill}%
\end{pgfscope}%
\begin{pgfscope}%
\pgfpathrectangle{\pgfqpoint{0.423750in}{1.819814in}}{\pgfqpoint{1.194205in}{1.163386in}}%
\pgfusepath{clip}%
\pgfsetbuttcap%
\pgfsetroundjoin%
\definecolor{currentfill}{rgb}{0.966120,0.744512,0.608720}%
\pgfsetfillcolor{currentfill}%
\pgfsetlinewidth{0.000000pt}%
\definecolor{currentstroke}{rgb}{0.000000,0.000000,0.000000}%
\pgfsetstrokecolor{currentstroke}%
\pgfsetdash{}{0pt}%
\pgfpathmoveto{\pgfqpoint{0.870069in}{1.819814in}}%
\pgfpathlineto{\pgfqpoint{0.882132in}{1.819814in}}%
\pgfpathlineto{\pgfqpoint{0.894194in}{1.819814in}}%
\pgfpathlineto{\pgfqpoint{0.906257in}{1.819814in}}%
\pgfpathlineto{\pgfqpoint{0.918320in}{1.819814in}}%
\pgfpathlineto{\pgfqpoint{0.924478in}{1.819814in}}%
\pgfpathlineto{\pgfqpoint{0.919373in}{1.831565in}}%
\pgfpathlineto{\pgfqpoint{0.918320in}{1.834023in}}%
\pgfpathlineto{\pgfqpoint{0.914318in}{1.843316in}}%
\pgfpathlineto{\pgfqpoint{0.909263in}{1.855068in}}%
\pgfpathlineto{\pgfqpoint{0.906257in}{1.862091in}}%
\pgfpathlineto{\pgfqpoint{0.904204in}{1.866819in}}%
\pgfpathlineto{\pgfqpoint{0.899141in}{1.878571in}}%
\pgfpathlineto{\pgfqpoint{0.894194in}{1.890005in}}%
\pgfpathlineto{\pgfqpoint{0.894054in}{1.890322in}}%
\pgfpathlineto{\pgfqpoint{0.888924in}{1.902073in}}%
\pgfpathlineto{\pgfqpoint{0.883757in}{1.913825in}}%
\pgfpathlineto{\pgfqpoint{0.882132in}{1.917556in}}%
\pgfpathlineto{\pgfqpoint{0.878508in}{1.925576in}}%
\pgfpathlineto{\pgfqpoint{0.873188in}{1.937327in}}%
\pgfpathlineto{\pgfqpoint{0.870069in}{1.944220in}}%
\pgfpathlineto{\pgfqpoint{0.867763in}{1.949079in}}%
\pgfpathlineto{\pgfqpoint{0.862201in}{1.960830in}}%
\pgfpathlineto{\pgfqpoint{0.858006in}{1.969621in}}%
\pgfpathlineto{\pgfqpoint{0.856510in}{1.972582in}}%
\pgfpathlineto{\pgfqpoint{0.850590in}{1.984333in}}%
\pgfpathlineto{\pgfqpoint{0.845944in}{1.993427in}}%
\pgfpathlineto{\pgfqpoint{0.844489in}{1.996084in}}%
\pgfpathlineto{\pgfqpoint{0.838058in}{2.007836in}}%
\pgfpathlineto{\pgfqpoint{0.833881in}{2.015344in}}%
\pgfpathlineto{\pgfqpoint{0.831325in}{2.019587in}}%
\pgfpathlineto{\pgfqpoint{0.824159in}{2.031338in}}%
\pgfpathlineto{\pgfqpoint{0.821818in}{2.035133in}}%
\pgfpathlineto{\pgfqpoint{0.816448in}{2.043090in}}%
\pgfpathlineto{\pgfqpoint{0.809756in}{2.052727in}}%
\pgfpathlineto{\pgfqpoint{0.808144in}{2.054841in}}%
\pgfpathlineto{\pgfqpoint{0.798990in}{2.066593in}}%
\pgfpathlineto{\pgfqpoint{0.797693in}{2.068224in}}%
\pgfpathlineto{\pgfqpoint{0.788804in}{2.078344in}}%
\pgfpathlineto{\pgfqpoint{0.785630in}{2.081844in}}%
\pgfpathlineto{\pgfqpoint{0.777415in}{2.090095in}}%
\pgfpathlineto{\pgfqpoint{0.773567in}{2.093836in}}%
\pgfpathlineto{\pgfqpoint{0.764646in}{2.101847in}}%
\pgfpathlineto{\pgfqpoint{0.761505in}{2.104575in}}%
\pgfpathlineto{\pgfqpoint{0.750515in}{2.113598in}}%
\pgfpathlineto{\pgfqpoint{0.749442in}{2.114449in}}%
\pgfpathlineto{\pgfqpoint{0.737379in}{2.123902in}}%
\pgfpathlineto{\pgfqpoint{0.735570in}{2.125349in}}%
\pgfpathlineto{\pgfqpoint{0.725317in}{2.133281in}}%
\pgfpathlineto{\pgfqpoint{0.720620in}{2.137101in}}%
\pgfpathlineto{\pgfqpoint{0.713254in}{2.142894in}}%
\pgfpathlineto{\pgfqpoint{0.706278in}{2.148852in}}%
\pgfpathlineto{\pgfqpoint{0.701191in}{2.153052in}}%
\pgfpathlineto{\pgfqpoint{0.692988in}{2.160604in}}%
\pgfpathlineto{\pgfqpoint{0.689129in}{2.164037in}}%
\pgfpathlineto{\pgfqpoint{0.680892in}{2.172355in}}%
\pgfpathlineto{\pgfqpoint{0.677066in}{2.176089in}}%
\pgfpathlineto{\pgfqpoint{0.669906in}{2.184106in}}%
\pgfpathlineto{\pgfqpoint{0.665003in}{2.189410in}}%
\pgfpathlineto{\pgfqpoint{0.659832in}{2.195858in}}%
\pgfpathlineto{\pgfqpoint{0.652941in}{2.204160in}}%
\pgfpathlineto{\pgfqpoint{0.650456in}{2.207609in}}%
\pgfpathlineto{\pgfqpoint{0.641732in}{2.219360in}}%
\pgfpathlineto{\pgfqpoint{0.640878in}{2.220486in}}%
\pgfpathlineto{\pgfqpoint{0.633895in}{2.231112in}}%
\pgfpathlineto{\pgfqpoint{0.628815in}{2.238574in}}%
\pgfpathlineto{\pgfqpoint{0.626270in}{2.242863in}}%
\pgfpathlineto{\pgfqpoint{0.619135in}{2.254615in}}%
\pgfpathlineto{\pgfqpoint{0.616753in}{2.258452in}}%
\pgfpathlineto{\pgfqpoint{0.612455in}{2.266366in}}%
\pgfpathlineto{\pgfqpoint{0.605881in}{2.278117in}}%
\pgfpathlineto{\pgfqpoint{0.604690in}{2.280218in}}%
\pgfpathlineto{\pgfqpoint{0.599880in}{2.289869in}}%
\pgfpathlineto{\pgfqpoint{0.593843in}{2.301620in}}%
\pgfpathlineto{\pgfqpoint{0.592627in}{2.303959in}}%
\pgfpathlineto{\pgfqpoint{0.588302in}{2.313371in}}%
\pgfpathlineto{\pgfqpoint{0.582760in}{2.325123in}}%
\pgfpathlineto{\pgfqpoint{0.580565in}{2.329712in}}%
\pgfpathlineto{\pgfqpoint{0.577513in}{2.336874in}}%
\pgfpathlineto{\pgfqpoint{0.572420in}{2.348626in}}%
\pgfpathlineto{\pgfqpoint{0.568502in}{2.357469in}}%
\pgfpathlineto{\pgfqpoint{0.567346in}{2.360377in}}%
\pgfpathlineto{\pgfqpoint{0.562653in}{2.372128in}}%
\pgfpathlineto{\pgfqpoint{0.557837in}{2.383880in}}%
\pgfpathlineto{\pgfqpoint{0.556439in}{2.387267in}}%
\pgfpathlineto{\pgfqpoint{0.553327in}{2.395631in}}%
\pgfpathlineto{\pgfqpoint{0.548887in}{2.407383in}}%
\pgfpathlineto{\pgfqpoint{0.544377in}{2.419043in}}%
\pgfpathlineto{\pgfqpoint{0.544345in}{2.419134in}}%
\pgfpathlineto{\pgfqpoint{0.540231in}{2.430885in}}%
\pgfpathlineto{\pgfqpoint{0.536032in}{2.442637in}}%
\pgfpathlineto{\pgfqpoint{0.532314in}{2.452852in}}%
\pgfpathlineto{\pgfqpoint{0.531803in}{2.454388in}}%
\pgfpathlineto{\pgfqpoint{0.527903in}{2.466139in}}%
\pgfpathlineto{\pgfqpoint{0.523928in}{2.477891in}}%
\pgfpathlineto{\pgfqpoint{0.520251in}{2.488568in}}%
\pgfpathlineto{\pgfqpoint{0.519911in}{2.489642in}}%
\pgfpathlineto{\pgfqpoint{0.516207in}{2.501394in}}%
\pgfpathlineto{\pgfqpoint{0.512437in}{2.513145in}}%
\pgfpathlineto{\pgfqpoint{0.508597in}{2.524896in}}%
\pgfpathlineto{\pgfqpoint{0.508189in}{2.526149in}}%
\pgfpathlineto{\pgfqpoint{0.505030in}{2.536648in}}%
\pgfpathlineto{\pgfqpoint{0.501446in}{2.548399in}}%
\pgfpathlineto{\pgfqpoint{0.497800in}{2.560150in}}%
\pgfpathlineto{\pgfqpoint{0.496126in}{2.565520in}}%
\pgfpathlineto{\pgfqpoint{0.494279in}{2.571902in}}%
\pgfpathlineto{\pgfqpoint{0.490861in}{2.583653in}}%
\pgfpathlineto{\pgfqpoint{0.487388in}{2.595405in}}%
\pgfpathlineto{\pgfqpoint{0.484063in}{2.606482in}}%
\pgfpathlineto{\pgfqpoint{0.483875in}{2.607156in}}%
\pgfpathlineto{\pgfqpoint{0.480604in}{2.618907in}}%
\pgfpathlineto{\pgfqpoint{0.477284in}{2.630659in}}%
\pgfpathlineto{\pgfqpoint{0.473911in}{2.642410in}}%
\pgfpathlineto{\pgfqpoint{0.472001in}{2.649028in}}%
\pgfpathlineto{\pgfqpoint{0.470611in}{2.654161in}}%
\pgfpathlineto{\pgfqpoint{0.467424in}{2.665913in}}%
\pgfpathlineto{\pgfqpoint{0.464191in}{2.677664in}}%
\pgfpathlineto{\pgfqpoint{0.460906in}{2.689416in}}%
\pgfpathlineto{\pgfqpoint{0.459938in}{2.692884in}}%
\pgfpathlineto{\pgfqpoint{0.457759in}{2.701167in}}%
\pgfpathlineto{\pgfqpoint{0.454645in}{2.712918in}}%
\pgfpathlineto{\pgfqpoint{0.451484in}{2.724670in}}%
\pgfpathlineto{\pgfqpoint{0.448274in}{2.736421in}}%
\pgfpathlineto{\pgfqpoint{0.447875in}{2.737887in}}%
\pgfpathlineto{\pgfqpoint{0.445231in}{2.748172in}}%
\pgfpathlineto{\pgfqpoint{0.442176in}{2.759924in}}%
\pgfpathlineto{\pgfqpoint{0.439074in}{2.771675in}}%
\pgfpathlineto{\pgfqpoint{0.435921in}{2.783427in}}%
\pgfpathlineto{\pgfqpoint{0.435813in}{2.783835in}}%
\pgfpathlineto{\pgfqpoint{0.432946in}{2.795178in}}%
\pgfpathlineto{\pgfqpoint{0.429934in}{2.806929in}}%
\pgfpathlineto{\pgfqpoint{0.426874in}{2.818681in}}%
\pgfpathlineto{\pgfqpoint{0.423762in}{2.830432in}}%
\pgfpathlineto{\pgfqpoint{0.423750in}{2.830478in}}%
\pgfpathlineto{\pgfqpoint{0.423750in}{2.830432in}}%
\pgfpathlineto{\pgfqpoint{0.423750in}{2.818681in}}%
\pgfpathlineto{\pgfqpoint{0.423750in}{2.806929in}}%
\pgfpathlineto{\pgfqpoint{0.423750in}{2.795178in}}%
\pgfpathlineto{\pgfqpoint{0.423750in}{2.783427in}}%
\pgfpathlineto{\pgfqpoint{0.423750in}{2.771675in}}%
\pgfpathlineto{\pgfqpoint{0.423750in}{2.759924in}}%
\pgfpathlineto{\pgfqpoint{0.423750in}{2.748172in}}%
\pgfpathlineto{\pgfqpoint{0.423750in}{2.736421in}}%
\pgfpathlineto{\pgfqpoint{0.423750in}{2.724670in}}%
\pgfpathlineto{\pgfqpoint{0.423750in}{2.712918in}}%
\pgfpathlineto{\pgfqpoint{0.423750in}{2.712228in}}%
\pgfpathlineto{\pgfqpoint{0.426896in}{2.701167in}}%
\pgfpathlineto{\pgfqpoint{0.430191in}{2.689416in}}%
\pgfpathlineto{\pgfqpoint{0.433439in}{2.677664in}}%
\pgfpathlineto{\pgfqpoint{0.435813in}{2.669000in}}%
\pgfpathlineto{\pgfqpoint{0.436696in}{2.665913in}}%
\pgfpathlineto{\pgfqpoint{0.440060in}{2.654161in}}%
\pgfpathlineto{\pgfqpoint{0.443374in}{2.642410in}}%
\pgfpathlineto{\pgfqpoint{0.446643in}{2.630659in}}%
\pgfpathlineto{\pgfqpoint{0.447875in}{2.626222in}}%
\pgfpathlineto{\pgfqpoint{0.450005in}{2.618907in}}%
\pgfpathlineto{\pgfqpoint{0.453403in}{2.607156in}}%
\pgfpathlineto{\pgfqpoint{0.456753in}{2.595405in}}%
\pgfpathlineto{\pgfqpoint{0.459938in}{2.584086in}}%
\pgfpathlineto{\pgfqpoint{0.460066in}{2.583653in}}%
\pgfpathlineto{\pgfqpoint{0.463564in}{2.571902in}}%
\pgfpathlineto{\pgfqpoint{0.467011in}{2.560150in}}%
\pgfpathlineto{\pgfqpoint{0.470410in}{2.548399in}}%
\pgfpathlineto{\pgfqpoint{0.472001in}{2.542884in}}%
\pgfpathlineto{\pgfqpoint{0.473901in}{2.536648in}}%
\pgfpathlineto{\pgfqpoint{0.477463in}{2.524896in}}%
\pgfpathlineto{\pgfqpoint{0.480973in}{2.513145in}}%
\pgfpathlineto{\pgfqpoint{0.484063in}{2.502680in}}%
\pgfpathlineto{\pgfqpoint{0.484467in}{2.501394in}}%
\pgfpathlineto{\pgfqpoint{0.488160in}{2.489642in}}%
\pgfpathlineto{\pgfqpoint{0.491798in}{2.477891in}}%
\pgfpathlineto{\pgfqpoint{0.495385in}{2.466139in}}%
\pgfpathlineto{\pgfqpoint{0.496126in}{2.463714in}}%
\pgfpathlineto{\pgfqpoint{0.499164in}{2.454388in}}%
\pgfpathlineto{\pgfqpoint{0.502947in}{2.442637in}}%
\pgfpathlineto{\pgfqpoint{0.506674in}{2.430885in}}%
\pgfpathlineto{\pgfqpoint{0.508189in}{2.426095in}}%
\pgfpathlineto{\pgfqpoint{0.510546in}{2.419134in}}%
\pgfpathlineto{\pgfqpoint{0.514492in}{2.407383in}}%
\pgfpathlineto{\pgfqpoint{0.518376in}{2.395631in}}%
\pgfpathlineto{\pgfqpoint{0.520251in}{2.389926in}}%
\pgfpathlineto{\pgfqpoint{0.522391in}{2.383880in}}%
\pgfpathlineto{\pgfqpoint{0.526517in}{2.372128in}}%
\pgfpathlineto{\pgfqpoint{0.530574in}{2.360377in}}%
\pgfpathlineto{\pgfqpoint{0.532314in}{2.355312in}}%
\pgfpathlineto{\pgfqpoint{0.534798in}{2.348626in}}%
\pgfpathlineto{\pgfqpoint{0.539121in}{2.336874in}}%
\pgfpathlineto{\pgfqpoint{0.543367in}{2.325123in}}%
\pgfpathlineto{\pgfqpoint{0.544377in}{2.322323in}}%
\pgfpathlineto{\pgfqpoint{0.547882in}{2.313371in}}%
\pgfpathlineto{\pgfqpoint{0.552417in}{2.301620in}}%
\pgfpathlineto{\pgfqpoint{0.556439in}{2.291017in}}%
\pgfpathlineto{\pgfqpoint{0.556915in}{2.289869in}}%
\pgfpathlineto{\pgfqpoint{0.561776in}{2.278117in}}%
\pgfpathlineto{\pgfqpoint{0.566534in}{2.266366in}}%
\pgfpathlineto{\pgfqpoint{0.568502in}{2.261466in}}%
\pgfpathlineto{\pgfqpoint{0.571519in}{2.254615in}}%
\pgfpathlineto{\pgfqpoint{0.576623in}{2.242863in}}%
\pgfpathlineto{\pgfqpoint{0.580565in}{2.233622in}}%
\pgfpathlineto{\pgfqpoint{0.581745in}{2.231112in}}%
\pgfpathlineto{\pgfqpoint{0.587230in}{2.219360in}}%
\pgfpathlineto{\pgfqpoint{0.592578in}{2.207609in}}%
\pgfpathlineto{\pgfqpoint{0.592627in}{2.207501in}}%
\pgfpathlineto{\pgfqpoint{0.598473in}{2.195858in}}%
\pgfpathlineto{\pgfqpoint{0.604210in}{2.184106in}}%
\pgfpathlineto{\pgfqpoint{0.604690in}{2.183118in}}%
\pgfpathlineto{\pgfqpoint{0.610487in}{2.172355in}}%
\pgfpathlineto{\pgfqpoint{0.616637in}{2.160604in}}%
\pgfpathlineto{\pgfqpoint{0.616753in}{2.160381in}}%
\pgfpathlineto{\pgfqpoint{0.623426in}{2.148852in}}%
\pgfpathlineto{\pgfqpoint{0.628815in}{2.139270in}}%
\pgfpathlineto{\pgfqpoint{0.630175in}{2.137101in}}%
\pgfpathlineto{\pgfqpoint{0.637442in}{2.125349in}}%
\pgfpathlineto{\pgfqpoint{0.640878in}{2.119669in}}%
\pgfpathlineto{\pgfqpoint{0.644966in}{2.113598in}}%
\pgfpathlineto{\pgfqpoint{0.652672in}{2.101847in}}%
\pgfpathlineto{\pgfqpoint{0.652941in}{2.101433in}}%
\pgfpathlineto{\pgfqpoint{0.661108in}{2.090095in}}%
\pgfpathlineto{\pgfqpoint{0.665003in}{2.084540in}}%
\pgfpathlineto{\pgfqpoint{0.669811in}{2.078344in}}%
\pgfpathlineto{\pgfqpoint{0.677066in}{2.068739in}}%
\pgfpathlineto{\pgfqpoint{0.678849in}{2.066593in}}%
\pgfpathlineto{\pgfqpoint{0.688387in}{2.054841in}}%
\pgfpathlineto{\pgfqpoint{0.689129in}{2.053911in}}%
\pgfpathlineto{\pgfqpoint{0.698497in}{2.043090in}}%
\pgfpathlineto{\pgfqpoint{0.701191in}{2.039897in}}%
\pgfpathlineto{\pgfqpoint{0.708939in}{2.031338in}}%
\pgfpathlineto{\pgfqpoint{0.713254in}{2.026451in}}%
\pgfpathlineto{\pgfqpoint{0.719653in}{2.019587in}}%
\pgfpathlineto{\pgfqpoint{0.725317in}{2.013360in}}%
\pgfpathlineto{\pgfqpoint{0.730527in}{2.007836in}}%
\pgfpathlineto{\pgfqpoint{0.737379in}{2.000393in}}%
\pgfpathlineto{\pgfqpoint{0.741415in}{1.996084in}}%
\pgfpathlineto{\pgfqpoint{0.749442in}{1.987310in}}%
\pgfpathlineto{\pgfqpoint{0.752163in}{1.984333in}}%
\pgfpathlineto{\pgfqpoint{0.761505in}{1.973869in}}%
\pgfpathlineto{\pgfqpoint{0.762634in}{1.972582in}}%
\pgfpathlineto{\pgfqpoint{0.772746in}{1.960830in}}%
\pgfpathlineto{\pgfqpoint{0.773567in}{1.959859in}}%
\pgfpathlineto{\pgfqpoint{0.782419in}{1.949079in}}%
\pgfpathlineto{\pgfqpoint{0.785630in}{1.945085in}}%
\pgfpathlineto{\pgfqpoint{0.791624in}{1.937327in}}%
\pgfpathlineto{\pgfqpoint{0.797693in}{1.929311in}}%
\pgfpathlineto{\pgfqpoint{0.800392in}{1.925576in}}%
\pgfpathlineto{\pgfqpoint{0.808754in}{1.913825in}}%
\pgfpathlineto{\pgfqpoint{0.809756in}{1.912405in}}%
\pgfpathlineto{\pgfqpoint{0.816708in}{1.902073in}}%
\pgfpathlineto{\pgfqpoint{0.821818in}{1.894339in}}%
\pgfpathlineto{\pgfqpoint{0.824347in}{1.890322in}}%
\pgfpathlineto{\pgfqpoint{0.831680in}{1.878571in}}%
\pgfpathlineto{\pgfqpoint{0.833881in}{1.875018in}}%
\pgfpathlineto{\pgfqpoint{0.838736in}{1.866819in}}%
\pgfpathlineto{\pgfqpoint{0.845594in}{1.855068in}}%
\pgfpathlineto{\pgfqpoint{0.845944in}{1.854469in}}%
\pgfpathlineto{\pgfqpoint{0.852209in}{1.843316in}}%
\pgfpathlineto{\pgfqpoint{0.858006in}{1.832845in}}%
\pgfpathlineto{\pgfqpoint{0.858690in}{1.831565in}}%
\pgfpathlineto{\pgfqpoint{0.864993in}{1.819814in}}%
\pgfpathclose%
\pgfusepath{fill}%
\end{pgfscope}%
\begin{pgfscope}%
\pgfpathrectangle{\pgfqpoint{0.423750in}{1.819814in}}{\pgfqpoint{1.194205in}{1.163386in}}%
\pgfusepath{clip}%
\pgfsetbuttcap%
\pgfsetroundjoin%
\definecolor{currentfill}{rgb}{0.966120,0.744512,0.608720}%
\pgfsetfillcolor{currentfill}%
\pgfsetlinewidth{0.000000pt}%
\definecolor{currentstroke}{rgb}{0.000000,0.000000,0.000000}%
\pgfsetstrokecolor{currentstroke}%
\pgfsetdash{}{0pt}%
\pgfpathmoveto{\pgfqpoint{1.304325in}{2.157956in}}%
\pgfpathlineto{\pgfqpoint{1.316388in}{2.155322in}}%
\pgfpathlineto{\pgfqpoint{1.328450in}{2.155794in}}%
\pgfpathlineto{\pgfqpoint{1.340513in}{2.159384in}}%
\pgfpathlineto{\pgfqpoint{1.342720in}{2.160604in}}%
\pgfpathlineto{\pgfqpoint{1.352576in}{2.165829in}}%
\pgfpathlineto{\pgfqpoint{1.361053in}{2.172355in}}%
\pgfpathlineto{\pgfqpoint{1.364638in}{2.175013in}}%
\pgfpathlineto{\pgfqpoint{1.374102in}{2.184106in}}%
\pgfpathlineto{\pgfqpoint{1.376701in}{2.186518in}}%
\pgfpathlineto{\pgfqpoint{1.385000in}{2.195858in}}%
\pgfpathlineto{\pgfqpoint{1.388764in}{2.199959in}}%
\pgfpathlineto{\pgfqpoint{1.394820in}{2.207609in}}%
\pgfpathlineto{\pgfqpoint{1.400826in}{2.214975in}}%
\pgfpathlineto{\pgfqpoint{1.404018in}{2.219360in}}%
\pgfpathlineto{\pgfqpoint{1.412806in}{2.231112in}}%
\pgfpathlineto{\pgfqpoint{1.412889in}{2.231220in}}%
\pgfpathlineto{\pgfqpoint{1.421105in}{2.242863in}}%
\pgfpathlineto{\pgfqpoint{1.424952in}{2.248185in}}%
\pgfpathlineto{\pgfqpoint{1.429307in}{2.254615in}}%
\pgfpathlineto{\pgfqpoint{1.437014in}{2.265746in}}%
\pgfpathlineto{\pgfqpoint{1.437425in}{2.266366in}}%
\pgfpathlineto{\pgfqpoint{1.445393in}{2.278117in}}%
\pgfpathlineto{\pgfqpoint{1.449077in}{2.283442in}}%
\pgfpathlineto{\pgfqpoint{1.453397in}{2.289869in}}%
\pgfpathlineto{\pgfqpoint{1.461140in}{2.301183in}}%
\pgfpathlineto{\pgfqpoint{1.461435in}{2.301620in}}%
\pgfpathlineto{\pgfqpoint{1.469543in}{2.313371in}}%
\pgfpathlineto{\pgfqpoint{1.473202in}{2.318591in}}%
\pgfpathlineto{\pgfqpoint{1.477788in}{2.325123in}}%
\pgfpathlineto{\pgfqpoint{1.485265in}{2.335623in}}%
\pgfpathlineto{\pgfqpoint{1.486170in}{2.336874in}}%
\pgfpathlineto{\pgfqpoint{1.494809in}{2.348626in}}%
\pgfpathlineto{\pgfqpoint{1.497328in}{2.352005in}}%
\pgfpathlineto{\pgfqpoint{1.503744in}{2.360377in}}%
\pgfpathlineto{\pgfqpoint{1.509390in}{2.367662in}}%
\pgfpathlineto{\pgfqpoint{1.512997in}{2.372128in}}%
\pgfpathlineto{\pgfqpoint{1.521453in}{2.382495in}}%
\pgfpathlineto{\pgfqpoint{1.522646in}{2.383880in}}%
\pgfpathlineto{\pgfqpoint{1.532867in}{2.395631in}}%
\pgfpathlineto{\pgfqpoint{1.533516in}{2.396370in}}%
\pgfpathlineto{\pgfqpoint{1.543850in}{2.407383in}}%
\pgfpathlineto{\pgfqpoint{1.545579in}{2.409211in}}%
\pgfpathlineto{\pgfqpoint{1.555722in}{2.419134in}}%
\pgfpathlineto{\pgfqpoint{1.557641in}{2.420999in}}%
\pgfpathlineto{\pgfqpoint{1.568778in}{2.430885in}}%
\pgfpathlineto{\pgfqpoint{1.569704in}{2.431703in}}%
\pgfpathlineto{\pgfqpoint{1.581767in}{2.441281in}}%
\pgfpathlineto{\pgfqpoint{1.583696in}{2.442637in}}%
\pgfpathlineto{\pgfqpoint{1.593829in}{2.449738in}}%
\pgfpathlineto{\pgfqpoint{1.601394in}{2.454388in}}%
\pgfpathlineto{\pgfqpoint{1.605892in}{2.457147in}}%
\pgfpathlineto{\pgfqpoint{1.617955in}{2.463512in}}%
\pgfpathlineto{\pgfqpoint{1.617955in}{2.466139in}}%
\pgfpathlineto{\pgfqpoint{1.617955in}{2.477891in}}%
\pgfpathlineto{\pgfqpoint{1.617955in}{2.489642in}}%
\pgfpathlineto{\pgfqpoint{1.617955in}{2.501394in}}%
\pgfpathlineto{\pgfqpoint{1.617955in}{2.513145in}}%
\pgfpathlineto{\pgfqpoint{1.617955in}{2.524896in}}%
\pgfpathlineto{\pgfqpoint{1.617955in}{2.536648in}}%
\pgfpathlineto{\pgfqpoint{1.617955in}{2.548399in}}%
\pgfpathlineto{\pgfqpoint{1.617955in}{2.560150in}}%
\pgfpathlineto{\pgfqpoint{1.617955in}{2.561630in}}%
\pgfpathlineto{\pgfqpoint{1.615524in}{2.560150in}}%
\pgfpathlineto{\pgfqpoint{1.605892in}{2.554280in}}%
\pgfpathlineto{\pgfqpoint{1.597084in}{2.548399in}}%
\pgfpathlineto{\pgfqpoint{1.593829in}{2.546222in}}%
\pgfpathlineto{\pgfqpoint{1.581767in}{2.537377in}}%
\pgfpathlineto{\pgfqpoint{1.580856in}{2.536648in}}%
\pgfpathlineto{\pgfqpoint{1.569704in}{2.527691in}}%
\pgfpathlineto{\pgfqpoint{1.566482in}{2.524896in}}%
\pgfpathlineto{\pgfqpoint{1.557641in}{2.517197in}}%
\pgfpathlineto{\pgfqpoint{1.553299in}{2.513145in}}%
\pgfpathlineto{\pgfqpoint{1.545579in}{2.505904in}}%
\pgfpathlineto{\pgfqpoint{1.541051in}{2.501394in}}%
\pgfpathlineto{\pgfqpoint{1.533516in}{2.493843in}}%
\pgfpathlineto{\pgfqpoint{1.529530in}{2.489642in}}%
\pgfpathlineto{\pgfqpoint{1.521453in}{2.481070in}}%
\pgfpathlineto{\pgfqpoint{1.518576in}{2.477891in}}%
\pgfpathlineto{\pgfqpoint{1.509390in}{2.467662in}}%
\pgfpathlineto{\pgfqpoint{1.508063in}{2.466139in}}%
\pgfpathlineto{\pgfqpoint{1.497913in}{2.454388in}}%
\pgfpathlineto{\pgfqpoint{1.497328in}{2.453704in}}%
\pgfpathlineto{\pgfqpoint{1.488048in}{2.442637in}}%
\pgfpathlineto{\pgfqpoint{1.485265in}{2.439284in}}%
\pgfpathlineto{\pgfqpoint{1.478341in}{2.430885in}}%
\pgfpathlineto{\pgfqpoint{1.473202in}{2.424582in}}%
\pgfpathlineto{\pgfqpoint{1.468735in}{2.419134in}}%
\pgfpathlineto{\pgfqpoint{1.461140in}{2.409758in}}%
\pgfpathlineto{\pgfqpoint{1.459177in}{2.407383in}}%
\pgfpathlineto{\pgfqpoint{1.449605in}{2.395631in}}%
\pgfpathlineto{\pgfqpoint{1.449077in}{2.394973in}}%
\pgfpathlineto{\pgfqpoint{1.439876in}{2.383880in}}%
\pgfpathlineto{\pgfqpoint{1.437014in}{2.380378in}}%
\pgfpathlineto{\pgfqpoint{1.429928in}{2.372128in}}%
\pgfpathlineto{\pgfqpoint{1.424952in}{2.366241in}}%
\pgfpathlineto{\pgfqpoint{1.419644in}{2.360377in}}%
\pgfpathlineto{\pgfqpoint{1.412889in}{2.352781in}}%
\pgfpathlineto{\pgfqpoint{1.408846in}{2.348626in}}%
\pgfpathlineto{\pgfqpoint{1.400826in}{2.340226in}}%
\pgfpathlineto{\pgfqpoint{1.397234in}{2.336874in}}%
\pgfpathlineto{\pgfqpoint{1.388764in}{2.328808in}}%
\pgfpathlineto{\pgfqpoint{1.384274in}{2.325123in}}%
\pgfpathlineto{\pgfqpoint{1.376701in}{2.318767in}}%
\pgfpathlineto{\pgfqpoint{1.368908in}{2.313371in}}%
\pgfpathlineto{\pgfqpoint{1.364638in}{2.310345in}}%
\pgfpathlineto{\pgfqpoint{1.352576in}{2.303723in}}%
\pgfpathlineto{\pgfqpoint{1.347064in}{2.301620in}}%
\pgfpathlineto{\pgfqpoint{1.340513in}{2.299054in}}%
\pgfpathlineto{\pgfqpoint{1.328450in}{2.296457in}}%
\pgfpathlineto{\pgfqpoint{1.316388in}{2.296015in}}%
\pgfpathlineto{\pgfqpoint{1.304325in}{2.297724in}}%
\pgfpathlineto{\pgfqpoint{1.292262in}{2.301553in}}%
\pgfpathlineto{\pgfqpoint{1.292124in}{2.301620in}}%
\pgfpathlineto{\pgfqpoint{1.280200in}{2.307275in}}%
\pgfpathlineto{\pgfqpoint{1.270573in}{2.313371in}}%
\pgfpathlineto{\pgfqpoint{1.268137in}{2.314876in}}%
\pgfpathlineto{\pgfqpoint{1.256074in}{2.324075in}}%
\pgfpathlineto{\pgfqpoint{1.254903in}{2.325123in}}%
\pgfpathlineto{\pgfqpoint{1.244012in}{2.334652in}}%
\pgfpathlineto{\pgfqpoint{1.241768in}{2.336874in}}%
\pgfpathlineto{\pgfqpoint{1.231949in}{2.346399in}}%
\pgfpathlineto{\pgfqpoint{1.229864in}{2.348626in}}%
\pgfpathlineto{\pgfqpoint{1.219886in}{2.359073in}}%
\pgfpathlineto{\pgfqpoint{1.218729in}{2.360377in}}%
\pgfpathlineto{\pgfqpoint{1.208103in}{2.372128in}}%
\pgfpathlineto{\pgfqpoint{1.207824in}{2.372432in}}%
\pgfpathlineto{\pgfqpoint{1.197832in}{2.383880in}}%
\pgfpathlineto{\pgfqpoint{1.195761in}{2.386214in}}%
\pgfpathlineto{\pgfqpoint{1.187706in}{2.395631in}}%
\pgfpathlineto{\pgfqpoint{1.183698in}{2.400247in}}%
\pgfpathlineto{\pgfqpoint{1.177632in}{2.407383in}}%
\pgfpathlineto{\pgfqpoint{1.171636in}{2.414338in}}%
\pgfpathlineto{\pgfqpoint{1.167530in}{2.419134in}}%
\pgfpathlineto{\pgfqpoint{1.159573in}{2.428309in}}%
\pgfpathlineto{\pgfqpoint{1.157324in}{2.430885in}}%
\pgfpathlineto{\pgfqpoint{1.147510in}{2.441999in}}%
\pgfpathlineto{\pgfqpoint{1.146936in}{2.442637in}}%
\pgfpathlineto{\pgfqpoint{1.136237in}{2.454388in}}%
\pgfpathlineto{\pgfqpoint{1.135448in}{2.455245in}}%
\pgfpathlineto{\pgfqpoint{1.125114in}{2.466139in}}%
\pgfpathlineto{\pgfqpoint{1.123385in}{2.467945in}}%
\pgfpathlineto{\pgfqpoint{1.113459in}{2.477891in}}%
\pgfpathlineto{\pgfqpoint{1.111322in}{2.480014in}}%
\pgfpathlineto{\pgfqpoint{1.101112in}{2.489642in}}%
\pgfpathlineto{\pgfqpoint{1.099260in}{2.491376in}}%
\pgfpathlineto{\pgfqpoint{1.087867in}{2.501394in}}%
\pgfpathlineto{\pgfqpoint{1.087197in}{2.501979in}}%
\pgfpathlineto{\pgfqpoint{1.075134in}{2.511762in}}%
\pgfpathlineto{\pgfqpoint{1.073279in}{2.513145in}}%
\pgfpathlineto{\pgfqpoint{1.063072in}{2.520717in}}%
\pgfpathlineto{\pgfqpoint{1.056906in}{2.524896in}}%
\pgfpathlineto{\pgfqpoint{1.051009in}{2.528877in}}%
\pgfpathlineto{\pgfqpoint{1.038946in}{2.536274in}}%
\pgfpathlineto{\pgfqpoint{1.038265in}{2.536648in}}%
\pgfpathlineto{\pgfqpoint{1.026884in}{2.542872in}}%
\pgfpathlineto{\pgfqpoint{1.015730in}{2.548399in}}%
\pgfpathlineto{\pgfqpoint{1.014821in}{2.548849in}}%
\pgfpathlineto{\pgfqpoint{1.002758in}{2.554174in}}%
\pgfpathlineto{\pgfqpoint{0.990696in}{2.559062in}}%
\pgfpathlineto{\pgfqpoint{0.987765in}{2.560150in}}%
\pgfpathlineto{\pgfqpoint{0.978633in}{2.563547in}}%
\pgfpathlineto{\pgfqpoint{0.966570in}{2.567830in}}%
\pgfpathlineto{\pgfqpoint{0.955058in}{2.571902in}}%
\pgfpathlineto{\pgfqpoint{0.954508in}{2.572097in}}%
\pgfpathlineto{\pgfqpoint{0.942445in}{2.576436in}}%
\pgfpathlineto{\pgfqpoint{0.930382in}{2.581159in}}%
\pgfpathlineto{\pgfqpoint{0.924653in}{2.583653in}}%
\pgfpathlineto{\pgfqpoint{0.918320in}{2.586429in}}%
\pgfpathlineto{\pgfqpoint{0.906257in}{2.592492in}}%
\pgfpathlineto{\pgfqpoint{0.901274in}{2.595405in}}%
\pgfpathlineto{\pgfqpoint{0.894194in}{2.599580in}}%
\pgfpathlineto{\pgfqpoint{0.883295in}{2.607156in}}%
\pgfpathlineto{\pgfqpoint{0.882132in}{2.607973in}}%
\pgfpathlineto{\pgfqpoint{0.870069in}{2.617837in}}%
\pgfpathlineto{\pgfqpoint{0.868940in}{2.618907in}}%
\pgfpathlineto{\pgfqpoint{0.858006in}{2.629414in}}%
\pgfpathlineto{\pgfqpoint{0.856879in}{2.630659in}}%
\pgfpathlineto{\pgfqpoint{0.846374in}{2.642410in}}%
\pgfpathlineto{\pgfqpoint{0.845944in}{2.642897in}}%
\pgfpathlineto{\pgfqpoint{0.837124in}{2.654161in}}%
\pgfpathlineto{\pgfqpoint{0.833881in}{2.658370in}}%
\pgfpathlineto{\pgfqpoint{0.828674in}{2.665913in}}%
\pgfpathlineto{\pgfqpoint{0.821818in}{2.676016in}}%
\pgfpathlineto{\pgfqpoint{0.820805in}{2.677664in}}%
\pgfpathlineto{\pgfqpoint{0.813599in}{2.689416in}}%
\pgfpathlineto{\pgfqpoint{0.809756in}{2.695772in}}%
\pgfpathlineto{\pgfqpoint{0.806760in}{2.701167in}}%
\pgfpathlineto{\pgfqpoint{0.800291in}{2.712918in}}%
\pgfpathlineto{\pgfqpoint{0.797693in}{2.717673in}}%
\pgfpathlineto{\pgfqpoint{0.794138in}{2.724670in}}%
\pgfpathlineto{\pgfqpoint{0.788224in}{2.736421in}}%
\pgfpathlineto{\pgfqpoint{0.785630in}{2.741605in}}%
\pgfpathlineto{\pgfqpoint{0.782540in}{2.748172in}}%
\pgfpathlineto{\pgfqpoint{0.777049in}{2.759924in}}%
\pgfpathlineto{\pgfqpoint{0.773567in}{2.767435in}}%
\pgfpathlineto{\pgfqpoint{0.771700in}{2.771675in}}%
\pgfpathlineto{\pgfqpoint{0.766530in}{2.783427in}}%
\pgfpathlineto{\pgfqpoint{0.761505in}{2.795011in}}%
\pgfpathlineto{\pgfqpoint{0.761435in}{2.795178in}}%
\pgfpathlineto{\pgfqpoint{0.756508in}{2.806929in}}%
\pgfpathlineto{\pgfqpoint{0.751645in}{2.818681in}}%
\pgfpathlineto{\pgfqpoint{0.749442in}{2.824012in}}%
\pgfpathlineto{\pgfqpoint{0.746872in}{2.830432in}}%
\pgfpathlineto{\pgfqpoint{0.742182in}{2.842183in}}%
\pgfpathlineto{\pgfqpoint{0.737547in}{2.853935in}}%
\pgfpathlineto{\pgfqpoint{0.737379in}{2.854358in}}%
\pgfpathlineto{\pgfqpoint{0.732979in}{2.865686in}}%
\pgfpathlineto{\pgfqpoint{0.728464in}{2.877438in}}%
\pgfpathlineto{\pgfqpoint{0.725317in}{2.885671in}}%
\pgfpathlineto{\pgfqpoint{0.723991in}{2.889189in}}%
\pgfpathlineto{\pgfqpoint{0.719553in}{2.900940in}}%
\pgfpathlineto{\pgfqpoint{0.715162in}{2.912692in}}%
\pgfpathlineto{\pgfqpoint{0.713254in}{2.917799in}}%
\pgfpathlineto{\pgfqpoint{0.710787in}{2.924443in}}%
\pgfpathlineto{\pgfqpoint{0.706437in}{2.936194in}}%
\pgfpathlineto{\pgfqpoint{0.702132in}{2.947946in}}%
\pgfpathlineto{\pgfqpoint{0.701191in}{2.950505in}}%
\pgfpathlineto{\pgfqpoint{0.697807in}{2.959697in}}%
\pgfpathlineto{\pgfqpoint{0.693510in}{2.971449in}}%
\pgfpathlineto{\pgfqpoint{0.689260in}{2.983200in}}%
\pgfpathlineto{\pgfqpoint{0.689129in}{2.983200in}}%
\pgfpathlineto{\pgfqpoint{0.677066in}{2.983200in}}%
\pgfpathlineto{\pgfqpoint{0.665003in}{2.983200in}}%
\pgfpathlineto{\pgfqpoint{0.652941in}{2.983200in}}%
\pgfpathlineto{\pgfqpoint{0.651818in}{2.983200in}}%
\pgfpathlineto{\pgfqpoint{0.652941in}{2.980025in}}%
\pgfpathlineto{\pgfqpoint{0.655861in}{2.971449in}}%
\pgfpathlineto{\pgfqpoint{0.659888in}{2.959697in}}%
\pgfpathlineto{\pgfqpoint{0.663969in}{2.947946in}}%
\pgfpathlineto{\pgfqpoint{0.665003in}{2.944957in}}%
\pgfpathlineto{\pgfqpoint{0.667946in}{2.936194in}}%
\pgfpathlineto{\pgfqpoint{0.671915in}{2.924443in}}%
\pgfpathlineto{\pgfqpoint{0.675933in}{2.912692in}}%
\pgfpathlineto{\pgfqpoint{0.677066in}{2.909364in}}%
\pgfpathlineto{\pgfqpoint{0.679871in}{2.900940in}}%
\pgfpathlineto{\pgfqpoint{0.683802in}{2.889189in}}%
\pgfpathlineto{\pgfqpoint{0.687776in}{2.877438in}}%
\pgfpathlineto{\pgfqpoint{0.689129in}{2.873424in}}%
\pgfpathlineto{\pgfqpoint{0.691699in}{2.865686in}}%
\pgfpathlineto{\pgfqpoint{0.695612in}{2.853935in}}%
\pgfpathlineto{\pgfqpoint{0.699567in}{2.842183in}}%
\pgfpathlineto{\pgfqpoint{0.701191in}{2.837342in}}%
\pgfpathlineto{\pgfqpoint{0.703497in}{2.830432in}}%
\pgfpathlineto{\pgfqpoint{0.707418in}{2.818681in}}%
\pgfpathlineto{\pgfqpoint{0.711379in}{2.806929in}}%
\pgfpathlineto{\pgfqpoint{0.713254in}{2.801357in}}%
\pgfpathlineto{\pgfqpoint{0.715339in}{2.795178in}}%
\pgfpathlineto{\pgfqpoint{0.719299in}{2.783427in}}%
\pgfpathlineto{\pgfqpoint{0.723299in}{2.771675in}}%
\pgfpathlineto{\pgfqpoint{0.725317in}{2.765737in}}%
\pgfpathlineto{\pgfqpoint{0.727317in}{2.759924in}}%
\pgfpathlineto{\pgfqpoint{0.731352in}{2.748172in}}%
\pgfpathlineto{\pgfqpoint{0.735429in}{2.736421in}}%
\pgfpathlineto{\pgfqpoint{0.737379in}{2.730784in}}%
\pgfpathlineto{\pgfqpoint{0.739545in}{2.724670in}}%
\pgfpathlineto{\pgfqpoint{0.743701in}{2.712918in}}%
\pgfpathlineto{\pgfqpoint{0.747904in}{2.701167in}}%
\pgfpathlineto{\pgfqpoint{0.749442in}{2.696844in}}%
\pgfpathlineto{\pgfqpoint{0.752178in}{2.689416in}}%
\pgfpathlineto{\pgfqpoint{0.756516in}{2.677664in}}%
\pgfpathlineto{\pgfqpoint{0.760910in}{2.665913in}}%
\pgfpathlineto{\pgfqpoint{0.761505in}{2.664305in}}%
\pgfpathlineto{\pgfqpoint{0.765438in}{2.654161in}}%
\pgfpathlineto{\pgfqpoint{0.770044in}{2.642410in}}%
\pgfpathlineto{\pgfqpoint{0.773567in}{2.633493in}}%
\pgfpathlineto{\pgfqpoint{0.774757in}{2.630659in}}%
\pgfpathlineto{\pgfqpoint{0.779663in}{2.618907in}}%
\pgfpathlineto{\pgfqpoint{0.784662in}{2.607156in}}%
\pgfpathlineto{\pgfqpoint{0.785630in}{2.604864in}}%
\pgfpathlineto{\pgfqpoint{0.789947in}{2.595405in}}%
\pgfpathlineto{\pgfqpoint{0.795400in}{2.583653in}}%
\pgfpathlineto{\pgfqpoint{0.797693in}{2.578731in}}%
\pgfpathlineto{\pgfqpoint{0.801190in}{2.571902in}}%
\pgfpathlineto{\pgfqpoint{0.807294in}{2.560150in}}%
\pgfpathlineto{\pgfqpoint{0.809756in}{2.555450in}}%
\pgfpathlineto{\pgfqpoint{0.813899in}{2.548399in}}%
\pgfpathlineto{\pgfqpoint{0.820966in}{2.536648in}}%
\pgfpathlineto{\pgfqpoint{0.821818in}{2.535234in}}%
\pgfpathlineto{\pgfqpoint{0.828994in}{2.524896in}}%
\pgfpathlineto{\pgfqpoint{0.833881in}{2.518043in}}%
\pgfpathlineto{\pgfqpoint{0.838003in}{2.513145in}}%
\pgfpathlineto{\pgfqpoint{0.845944in}{2.503939in}}%
\pgfpathlineto{\pgfqpoint{0.848614in}{2.501394in}}%
\pgfpathlineto{\pgfqpoint{0.858006in}{2.492637in}}%
\pgfpathlineto{\pgfqpoint{0.862065in}{2.489642in}}%
\pgfpathlineto{\pgfqpoint{0.870069in}{2.483852in}}%
\pgfpathlineto{\pgfqpoint{0.881015in}{2.477891in}}%
\pgfpathlineto{\pgfqpoint{0.882132in}{2.477294in}}%
\pgfpathlineto{\pgfqpoint{0.894194in}{2.472405in}}%
\pgfpathlineto{\pgfqpoint{0.906257in}{2.468959in}}%
\pgfpathlineto{\pgfqpoint{0.918320in}{2.466538in}}%
\pgfpathlineto{\pgfqpoint{0.920965in}{2.466139in}}%
\pgfpathlineto{\pgfqpoint{0.930382in}{2.464735in}}%
\pgfpathlineto{\pgfqpoint{0.942445in}{2.463296in}}%
\pgfpathlineto{\pgfqpoint{0.954508in}{2.461959in}}%
\pgfpathlineto{\pgfqpoint{0.966570in}{2.460486in}}%
\pgfpathlineto{\pgfqpoint{0.978633in}{2.458674in}}%
\pgfpathlineto{\pgfqpoint{0.990696in}{2.456354in}}%
\pgfpathlineto{\pgfqpoint{0.998608in}{2.454388in}}%
\pgfpathlineto{\pgfqpoint{1.002758in}{2.453358in}}%
\pgfpathlineto{\pgfqpoint{1.014821in}{2.449528in}}%
\pgfpathlineto{\pgfqpoint{1.026884in}{2.444844in}}%
\pgfpathlineto{\pgfqpoint{1.031574in}{2.442637in}}%
\pgfpathlineto{\pgfqpoint{1.038946in}{2.439157in}}%
\pgfpathlineto{\pgfqpoint{1.051009in}{2.432455in}}%
\pgfpathlineto{\pgfqpoint{1.053439in}{2.430885in}}%
\pgfpathlineto{\pgfqpoint{1.063072in}{2.424634in}}%
\pgfpathlineto{\pgfqpoint{1.070545in}{2.419134in}}%
\pgfpathlineto{\pgfqpoint{1.075134in}{2.415737in}}%
\pgfpathlineto{\pgfqpoint{1.085224in}{2.407383in}}%
\pgfpathlineto{\pgfqpoint{1.087197in}{2.405738in}}%
\pgfpathlineto{\pgfqpoint{1.098187in}{2.395631in}}%
\pgfpathlineto{\pgfqpoint{1.099260in}{2.394637in}}%
\pgfpathlineto{\pgfqpoint{1.109930in}{2.383880in}}%
\pgfpathlineto{\pgfqpoint{1.111322in}{2.382463in}}%
\pgfpathlineto{\pgfqpoint{1.120793in}{2.372128in}}%
\pgfpathlineto{\pgfqpoint{1.123385in}{2.369270in}}%
\pgfpathlineto{\pgfqpoint{1.131011in}{2.360377in}}%
\pgfpathlineto{\pgfqpoint{1.135448in}{2.355141in}}%
\pgfpathlineto{\pgfqpoint{1.140742in}{2.348626in}}%
\pgfpathlineto{\pgfqpoint{1.147510in}{2.340185in}}%
\pgfpathlineto{\pgfqpoint{1.150092in}{2.336874in}}%
\pgfpathlineto{\pgfqpoint{1.159117in}{2.325123in}}%
\pgfpathlineto{\pgfqpoint{1.159573in}{2.324518in}}%
\pgfpathlineto{\pgfqpoint{1.167852in}{2.313371in}}%
\pgfpathlineto{\pgfqpoint{1.171636in}{2.308192in}}%
\pgfpathlineto{\pgfqpoint{1.176436in}{2.301620in}}%
\pgfpathlineto{\pgfqpoint{1.183698in}{2.291496in}}%
\pgfpathlineto{\pgfqpoint{1.184883in}{2.289869in}}%
\pgfpathlineto{\pgfqpoint{1.193251in}{2.278117in}}%
\pgfpathlineto{\pgfqpoint{1.195761in}{2.274516in}}%
\pgfpathlineto{\pgfqpoint{1.201619in}{2.266366in}}%
\pgfpathlineto{\pgfqpoint{1.207824in}{2.257540in}}%
\pgfpathlineto{\pgfqpoint{1.209983in}{2.254615in}}%
\pgfpathlineto{\pgfqpoint{1.218440in}{2.242863in}}%
\pgfpathlineto{\pgfqpoint{1.219886in}{2.240799in}}%
\pgfpathlineto{\pgfqpoint{1.227155in}{2.231112in}}%
\pgfpathlineto{\pgfqpoint{1.231949in}{2.224549in}}%
\pgfpathlineto{\pgfqpoint{1.236104in}{2.219360in}}%
\pgfpathlineto{\pgfqpoint{1.244012in}{2.209194in}}%
\pgfpathlineto{\pgfqpoint{1.245403in}{2.207609in}}%
\pgfpathlineto{\pgfqpoint{1.255389in}{2.195858in}}%
\pgfpathlineto{\pgfqpoint{1.256074in}{2.195024in}}%
\pgfpathlineto{\pgfqpoint{1.266566in}{2.184106in}}%
\pgfpathlineto{\pgfqpoint{1.268137in}{2.182412in}}%
\pgfpathlineto{\pgfqpoint{1.279605in}{2.172355in}}%
\pgfpathlineto{\pgfqpoint{1.280200in}{2.171812in}}%
\pgfpathlineto{\pgfqpoint{1.292262in}{2.163515in}}%
\pgfpathlineto{\pgfqpoint{1.298712in}{2.160604in}}%
\pgfpathclose%
\pgfusepath{fill}%
\end{pgfscope}%
\begin{pgfscope}%
\pgfpathrectangle{\pgfqpoint{0.423750in}{1.819814in}}{\pgfqpoint{1.194205in}{1.163386in}}%
\pgfusepath{clip}%
\pgfsetbuttcap%
\pgfsetroundjoin%
\definecolor{currentfill}{rgb}{0.970255,0.815666,0.711203}%
\pgfsetfillcolor{currentfill}%
\pgfsetlinewidth{0.000000pt}%
\definecolor{currentstroke}{rgb}{0.000000,0.000000,0.000000}%
\pgfsetstrokecolor{currentstroke}%
\pgfsetdash{}{0pt}%
\pgfpathmoveto{\pgfqpoint{0.930382in}{1.819814in}}%
\pgfpathlineto{\pgfqpoint{0.942445in}{1.819814in}}%
\pgfpathlineto{\pgfqpoint{0.954508in}{1.819814in}}%
\pgfpathlineto{\pgfqpoint{0.966570in}{1.819814in}}%
\pgfpathlineto{\pgfqpoint{0.978633in}{1.819814in}}%
\pgfpathlineto{\pgfqpoint{0.990696in}{1.819814in}}%
\pgfpathlineto{\pgfqpoint{0.992107in}{1.819814in}}%
\pgfpathlineto{\pgfqpoint{0.990696in}{1.822928in}}%
\pgfpathlineto{\pgfqpoint{0.987044in}{1.831565in}}%
\pgfpathlineto{\pgfqpoint{0.982135in}{1.843316in}}%
\pgfpathlineto{\pgfqpoint{0.978633in}{1.851809in}}%
\pgfpathlineto{\pgfqpoint{0.977360in}{1.855068in}}%
\pgfpathlineto{\pgfqpoint{0.972879in}{1.866819in}}%
\pgfpathlineto{\pgfqpoint{0.968443in}{1.878571in}}%
\pgfpathlineto{\pgfqpoint{0.966570in}{1.883648in}}%
\pgfpathlineto{\pgfqpoint{0.964205in}{1.890322in}}%
\pgfpathlineto{\pgfqpoint{0.960131in}{1.902073in}}%
\pgfpathlineto{\pgfqpoint{0.956114in}{1.913825in}}%
\pgfpathlineto{\pgfqpoint{0.954508in}{1.918661in}}%
\pgfpathlineto{\pgfqpoint{0.952268in}{1.925576in}}%
\pgfpathlineto{\pgfqpoint{0.948563in}{1.937327in}}%
\pgfpathlineto{\pgfqpoint{0.944926in}{1.949079in}}%
\pgfpathlineto{\pgfqpoint{0.942445in}{1.957324in}}%
\pgfpathlineto{\pgfqpoint{0.941401in}{1.960830in}}%
\pgfpathlineto{\pgfqpoint{0.938040in}{1.972582in}}%
\pgfpathlineto{\pgfqpoint{0.934763in}{1.984333in}}%
\pgfpathlineto{\pgfqpoint{0.931576in}{1.996084in}}%
\pgfpathlineto{\pgfqpoint{0.930382in}{2.000700in}}%
\pgfpathlineto{\pgfqpoint{0.928527in}{2.007836in}}%
\pgfpathlineto{\pgfqpoint{0.925607in}{2.019587in}}%
\pgfpathlineto{\pgfqpoint{0.922799in}{2.031338in}}%
\pgfpathlineto{\pgfqpoint{0.920111in}{2.043090in}}%
\pgfpathlineto{\pgfqpoint{0.918320in}{2.051410in}}%
\pgfpathlineto{\pgfqpoint{0.917562in}{2.054841in}}%
\pgfpathlineto{\pgfqpoint{0.915168in}{2.066593in}}%
\pgfpathlineto{\pgfqpoint{0.912929in}{2.078344in}}%
\pgfpathlineto{\pgfqpoint{0.910864in}{2.090095in}}%
\pgfpathlineto{\pgfqpoint{0.908990in}{2.101847in}}%
\pgfpathlineto{\pgfqpoint{0.907330in}{2.113598in}}%
\pgfpathlineto{\pgfqpoint{0.906257in}{2.122564in}}%
\pgfpathlineto{\pgfqpoint{0.905905in}{2.125349in}}%
\pgfpathlineto{\pgfqpoint{0.904736in}{2.137101in}}%
\pgfpathlineto{\pgfqpoint{0.903873in}{2.148852in}}%
\pgfpathlineto{\pgfqpoint{0.903360in}{2.160604in}}%
\pgfpathlineto{\pgfqpoint{0.903251in}{2.172355in}}%
\pgfpathlineto{\pgfqpoint{0.903610in}{2.184106in}}%
\pgfpathlineto{\pgfqpoint{0.904516in}{2.195858in}}%
\pgfpathlineto{\pgfqpoint{0.906066in}{2.207609in}}%
\pgfpathlineto{\pgfqpoint{0.906257in}{2.208635in}}%
\pgfpathlineto{\pgfqpoint{0.908306in}{2.219360in}}%
\pgfpathlineto{\pgfqpoint{0.911407in}{2.231112in}}%
\pgfpathlineto{\pgfqpoint{0.915548in}{2.242863in}}%
\pgfpathlineto{\pgfqpoint{0.918320in}{2.249093in}}%
\pgfpathlineto{\pgfqpoint{0.920897in}{2.254615in}}%
\pgfpathlineto{\pgfqpoint{0.927676in}{2.266366in}}%
\pgfpathlineto{\pgfqpoint{0.930382in}{2.270255in}}%
\pgfpathlineto{\pgfqpoint{0.936314in}{2.278117in}}%
\pgfpathlineto{\pgfqpoint{0.942445in}{2.284942in}}%
\pgfpathlineto{\pgfqpoint{0.947449in}{2.289869in}}%
\pgfpathlineto{\pgfqpoint{0.954508in}{2.295872in}}%
\pgfpathlineto{\pgfqpoint{0.962621in}{2.301620in}}%
\pgfpathlineto{\pgfqpoint{0.966570in}{2.304089in}}%
\pgfpathlineto{\pgfqpoint{0.978633in}{2.309995in}}%
\pgfpathlineto{\pgfqpoint{0.988816in}{2.313371in}}%
\pgfpathlineto{\pgfqpoint{0.990696in}{2.313934in}}%
\pgfpathlineto{\pgfqpoint{1.002758in}{2.315896in}}%
\pgfpathlineto{\pgfqpoint{1.014821in}{2.316059in}}%
\pgfpathlineto{\pgfqpoint{1.026884in}{2.314449in}}%
\pgfpathlineto{\pgfqpoint{1.030677in}{2.313371in}}%
\pgfpathlineto{\pgfqpoint{1.038946in}{2.311014in}}%
\pgfpathlineto{\pgfqpoint{1.051009in}{2.305804in}}%
\pgfpathlineto{\pgfqpoint{1.058281in}{2.301620in}}%
\pgfpathlineto{\pgfqpoint{1.063072in}{2.298845in}}%
\pgfpathlineto{\pgfqpoint{1.075134in}{2.290134in}}%
\pgfpathlineto{\pgfqpoint{1.075441in}{2.289869in}}%
\pgfpathlineto{\pgfqpoint{1.087197in}{2.279589in}}%
\pgfpathlineto{\pgfqpoint{1.088656in}{2.278117in}}%
\pgfpathlineto{\pgfqpoint{1.099260in}{2.267296in}}%
\pgfpathlineto{\pgfqpoint{1.100069in}{2.266366in}}%
\pgfpathlineto{\pgfqpoint{1.110137in}{2.254615in}}%
\pgfpathlineto{\pgfqpoint{1.111322in}{2.253210in}}%
\pgfpathlineto{\pgfqpoint{1.119217in}{2.242863in}}%
\pgfpathlineto{\pgfqpoint{1.123385in}{2.237316in}}%
\pgfpathlineto{\pgfqpoint{1.127687in}{2.231112in}}%
\pgfpathlineto{\pgfqpoint{1.135448in}{2.219724in}}%
\pgfpathlineto{\pgfqpoint{1.135681in}{2.219360in}}%
\pgfpathlineto{\pgfqpoint{1.142998in}{2.207609in}}%
\pgfpathlineto{\pgfqpoint{1.147510in}{2.200210in}}%
\pgfpathlineto{\pgfqpoint{1.150046in}{2.195858in}}%
\pgfpathlineto{\pgfqpoint{1.156701in}{2.184106in}}%
\pgfpathlineto{\pgfqpoint{1.159573in}{2.178884in}}%
\pgfpathlineto{\pgfqpoint{1.163059in}{2.172355in}}%
\pgfpathlineto{\pgfqpoint{1.169139in}{2.160604in}}%
\pgfpathlineto{\pgfqpoint{1.171636in}{2.155607in}}%
\pgfpathlineto{\pgfqpoint{1.174965in}{2.148852in}}%
\pgfpathlineto{\pgfqpoint{1.180545in}{2.137101in}}%
\pgfpathlineto{\pgfqpoint{1.183698in}{2.130195in}}%
\pgfpathlineto{\pgfqpoint{1.185917in}{2.125349in}}%
\pgfpathlineto{\pgfqpoint{1.191059in}{2.113598in}}%
\pgfpathlineto{\pgfqpoint{1.195761in}{2.102382in}}%
\pgfpathlineto{\pgfqpoint{1.195990in}{2.101847in}}%
\pgfpathlineto{\pgfqpoint{1.200740in}{2.090095in}}%
\pgfpathlineto{\pgfqpoint{1.205252in}{2.078344in}}%
\pgfpathlineto{\pgfqpoint{1.207824in}{2.071232in}}%
\pgfpathlineto{\pgfqpoint{1.209567in}{2.066593in}}%
\pgfpathlineto{\pgfqpoint{1.213692in}{2.054841in}}%
\pgfpathlineto{\pgfqpoint{1.217542in}{2.043090in}}%
\pgfpathlineto{\pgfqpoint{1.219886in}{2.035341in}}%
\pgfpathlineto{\pgfqpoint{1.221171in}{2.031338in}}%
\pgfpathlineto{\pgfqpoint{1.224610in}{2.019587in}}%
\pgfpathlineto{\pgfqpoint{1.227731in}{2.007836in}}%
\pgfpathlineto{\pgfqpoint{1.230523in}{1.996084in}}%
\pgfpathlineto{\pgfqpoint{1.231949in}{1.989238in}}%
\pgfpathlineto{\pgfqpoint{1.233058in}{1.984333in}}%
\pgfpathlineto{\pgfqpoint{1.235326in}{1.972582in}}%
\pgfpathlineto{\pgfqpoint{1.237210in}{1.960830in}}%
\pgfpathlineto{\pgfqpoint{1.238699in}{1.949079in}}%
\pgfpathlineto{\pgfqpoint{1.239784in}{1.937327in}}%
\pgfpathlineto{\pgfqpoint{1.240453in}{1.925576in}}%
\pgfpathlineto{\pgfqpoint{1.240696in}{1.913825in}}%
\pgfpathlineto{\pgfqpoint{1.240502in}{1.902073in}}%
\pgfpathlineto{\pgfqpoint{1.239859in}{1.890322in}}%
\pgfpathlineto{\pgfqpoint{1.238755in}{1.878571in}}%
\pgfpathlineto{\pgfqpoint{1.237177in}{1.866819in}}%
\pgfpathlineto{\pgfqpoint{1.235112in}{1.855068in}}%
\pgfpathlineto{\pgfqpoint{1.232548in}{1.843316in}}%
\pgfpathlineto{\pgfqpoint{1.231949in}{1.841015in}}%
\pgfpathlineto{\pgfqpoint{1.229621in}{1.831565in}}%
\pgfpathlineto{\pgfqpoint{1.226230in}{1.819814in}}%
\pgfpathlineto{\pgfqpoint{1.231949in}{1.819814in}}%
\pgfpathlineto{\pgfqpoint{1.244012in}{1.819814in}}%
\pgfpathlineto{\pgfqpoint{1.256074in}{1.819814in}}%
\pgfpathlineto{\pgfqpoint{1.268137in}{1.819814in}}%
\pgfpathlineto{\pgfqpoint{1.280200in}{1.819814in}}%
\pgfpathlineto{\pgfqpoint{1.292262in}{1.819814in}}%
\pgfpathlineto{\pgfqpoint{1.304325in}{1.819814in}}%
\pgfpathlineto{\pgfqpoint{1.316388in}{1.819814in}}%
\pgfpathlineto{\pgfqpoint{1.328450in}{1.819814in}}%
\pgfpathlineto{\pgfqpoint{1.340513in}{1.819814in}}%
\pgfpathlineto{\pgfqpoint{1.352576in}{1.819814in}}%
\pgfpathlineto{\pgfqpoint{1.364638in}{1.819814in}}%
\pgfpathlineto{\pgfqpoint{1.376701in}{1.819814in}}%
\pgfpathlineto{\pgfqpoint{1.388764in}{1.819814in}}%
\pgfpathlineto{\pgfqpoint{1.400826in}{1.819814in}}%
\pgfpathlineto{\pgfqpoint{1.412889in}{1.819814in}}%
\pgfpathlineto{\pgfqpoint{1.419812in}{1.819814in}}%
\pgfpathlineto{\pgfqpoint{1.416100in}{1.831565in}}%
\pgfpathlineto{\pgfqpoint{1.412923in}{1.843316in}}%
\pgfpathlineto{\pgfqpoint{1.412889in}{1.843465in}}%
\pgfpathlineto{\pgfqpoint{1.410093in}{1.855068in}}%
\pgfpathlineto{\pgfqpoint{1.407784in}{1.866819in}}%
\pgfpathlineto{\pgfqpoint{1.405982in}{1.878571in}}%
\pgfpathlineto{\pgfqpoint{1.404668in}{1.890322in}}%
\pgfpathlineto{\pgfqpoint{1.403827in}{1.902073in}}%
\pgfpathlineto{\pgfqpoint{1.403442in}{1.913825in}}%
\pgfpathlineto{\pgfqpoint{1.403499in}{1.925576in}}%
\pgfpathlineto{\pgfqpoint{1.403982in}{1.937327in}}%
\pgfpathlineto{\pgfqpoint{1.404879in}{1.949079in}}%
\pgfpathlineto{\pgfqpoint{1.406175in}{1.960830in}}%
\pgfpathlineto{\pgfqpoint{1.407859in}{1.972582in}}%
\pgfpathlineto{\pgfqpoint{1.409918in}{1.984333in}}%
\pgfpathlineto{\pgfqpoint{1.412341in}{1.996084in}}%
\pgfpathlineto{\pgfqpoint{1.412889in}{1.998388in}}%
\pgfpathlineto{\pgfqpoint{1.414965in}{2.007836in}}%
\pgfpathlineto{\pgfqpoint{1.417868in}{2.019587in}}%
\pgfpathlineto{\pgfqpoint{1.421076in}{2.031338in}}%
\pgfpathlineto{\pgfqpoint{1.424581in}{2.043090in}}%
\pgfpathlineto{\pgfqpoint{1.424952in}{2.044228in}}%
\pgfpathlineto{\pgfqpoint{1.428216in}{2.054841in}}%
\pgfpathlineto{\pgfqpoint{1.432095in}{2.066593in}}%
\pgfpathlineto{\pgfqpoint{1.436226in}{2.078344in}}%
\pgfpathlineto{\pgfqpoint{1.437014in}{2.080444in}}%
\pgfpathlineto{\pgfqpoint{1.440507in}{2.090095in}}%
\pgfpathlineto{\pgfqpoint{1.444992in}{2.101847in}}%
\pgfpathlineto{\pgfqpoint{1.449077in}{2.112042in}}%
\pgfpathlineto{\pgfqpoint{1.449689in}{2.113598in}}%
\pgfpathlineto{\pgfqpoint{1.454559in}{2.125349in}}%
\pgfpathlineto{\pgfqpoint{1.459620in}{2.137101in}}%
\pgfpathlineto{\pgfqpoint{1.461140in}{2.140476in}}%
\pgfpathlineto{\pgfqpoint{1.464902in}{2.148852in}}%
\pgfpathlineto{\pgfqpoint{1.470369in}{2.160604in}}%
\pgfpathlineto{\pgfqpoint{1.473202in}{2.166482in}}%
\pgfpathlineto{\pgfqpoint{1.476074in}{2.172355in}}%
\pgfpathlineto{\pgfqpoint{1.482003in}{2.184106in}}%
\pgfpathlineto{\pgfqpoint{1.485265in}{2.190383in}}%
\pgfpathlineto{\pgfqpoint{1.488196in}{2.195858in}}%
\pgfpathlineto{\pgfqpoint{1.494658in}{2.207609in}}%
\pgfpathlineto{\pgfqpoint{1.497328in}{2.212338in}}%
\pgfpathlineto{\pgfqpoint{1.501475in}{2.219360in}}%
\pgfpathlineto{\pgfqpoint{1.508551in}{2.231112in}}%
\pgfpathlineto{\pgfqpoint{1.509390in}{2.232470in}}%
\pgfpathlineto{\pgfqpoint{1.516215in}{2.242863in}}%
\pgfpathlineto{\pgfqpoint{1.521453in}{2.250724in}}%
\pgfpathlineto{\pgfqpoint{1.524257in}{2.254615in}}%
\pgfpathlineto{\pgfqpoint{1.532851in}{2.266366in}}%
\pgfpathlineto{\pgfqpoint{1.533516in}{2.267258in}}%
\pgfpathlineto{\pgfqpoint{1.542414in}{2.278117in}}%
\pgfpathlineto{\pgfqpoint{1.545579in}{2.281938in}}%
\pgfpathlineto{\pgfqpoint{1.552949in}{2.289869in}}%
\pgfpathlineto{\pgfqpoint{1.557641in}{2.294872in}}%
\pgfpathlineto{\pgfqpoint{1.564910in}{2.301620in}}%
\pgfpathlineto{\pgfqpoint{1.569704in}{2.306038in}}%
\pgfpathlineto{\pgfqpoint{1.579097in}{2.313371in}}%
\pgfpathlineto{\pgfqpoint{1.581767in}{2.315444in}}%
\pgfpathlineto{\pgfqpoint{1.593829in}{2.323053in}}%
\pgfpathlineto{\pgfqpoint{1.598116in}{2.325123in}}%
\pgfpathlineto{\pgfqpoint{1.605892in}{2.328867in}}%
\pgfpathlineto{\pgfqpoint{1.617955in}{2.332919in}}%
\pgfpathlineto{\pgfqpoint{1.617955in}{2.336874in}}%
\pgfpathlineto{\pgfqpoint{1.617955in}{2.348626in}}%
\pgfpathlineto{\pgfqpoint{1.617955in}{2.360377in}}%
\pgfpathlineto{\pgfqpoint{1.617955in}{2.372128in}}%
\pgfpathlineto{\pgfqpoint{1.617955in}{2.383880in}}%
\pgfpathlineto{\pgfqpoint{1.617955in}{2.395631in}}%
\pgfpathlineto{\pgfqpoint{1.617955in}{2.407383in}}%
\pgfpathlineto{\pgfqpoint{1.617955in}{2.419134in}}%
\pgfpathlineto{\pgfqpoint{1.617955in}{2.430885in}}%
\pgfpathlineto{\pgfqpoint{1.617955in}{2.442637in}}%
\pgfpathlineto{\pgfqpoint{1.617955in}{2.454388in}}%
\pgfpathlineto{\pgfqpoint{1.617955in}{2.463512in}}%
\pgfpathlineto{\pgfqpoint{1.605892in}{2.457147in}}%
\pgfpathlineto{\pgfqpoint{1.601394in}{2.454388in}}%
\pgfpathlineto{\pgfqpoint{1.593829in}{2.449738in}}%
\pgfpathlineto{\pgfqpoint{1.583696in}{2.442637in}}%
\pgfpathlineto{\pgfqpoint{1.581767in}{2.441281in}}%
\pgfpathlineto{\pgfqpoint{1.569704in}{2.431703in}}%
\pgfpathlineto{\pgfqpoint{1.568778in}{2.430885in}}%
\pgfpathlineto{\pgfqpoint{1.557641in}{2.420999in}}%
\pgfpathlineto{\pgfqpoint{1.555722in}{2.419134in}}%
\pgfpathlineto{\pgfqpoint{1.545579in}{2.409211in}}%
\pgfpathlineto{\pgfqpoint{1.543850in}{2.407383in}}%
\pgfpathlineto{\pgfqpoint{1.533516in}{2.396370in}}%
\pgfpathlineto{\pgfqpoint{1.532867in}{2.395631in}}%
\pgfpathlineto{\pgfqpoint{1.522646in}{2.383880in}}%
\pgfpathlineto{\pgfqpoint{1.521453in}{2.382495in}}%
\pgfpathlineto{\pgfqpoint{1.512997in}{2.372128in}}%
\pgfpathlineto{\pgfqpoint{1.509390in}{2.367662in}}%
\pgfpathlineto{\pgfqpoint{1.503744in}{2.360377in}}%
\pgfpathlineto{\pgfqpoint{1.497328in}{2.352005in}}%
\pgfpathlineto{\pgfqpoint{1.494809in}{2.348626in}}%
\pgfpathlineto{\pgfqpoint{1.486170in}{2.336874in}}%
\pgfpathlineto{\pgfqpoint{1.485265in}{2.335623in}}%
\pgfpathlineto{\pgfqpoint{1.477788in}{2.325123in}}%
\pgfpathlineto{\pgfqpoint{1.473202in}{2.318591in}}%
\pgfpathlineto{\pgfqpoint{1.469543in}{2.313371in}}%
\pgfpathlineto{\pgfqpoint{1.461435in}{2.301620in}}%
\pgfpathlineto{\pgfqpoint{1.461140in}{2.301183in}}%
\pgfpathlineto{\pgfqpoint{1.453397in}{2.289869in}}%
\pgfpathlineto{\pgfqpoint{1.449077in}{2.283442in}}%
\pgfpathlineto{\pgfqpoint{1.445393in}{2.278117in}}%
\pgfpathlineto{\pgfqpoint{1.437425in}{2.266366in}}%
\pgfpathlineto{\pgfqpoint{1.437014in}{2.265746in}}%
\pgfpathlineto{\pgfqpoint{1.429307in}{2.254615in}}%
\pgfpathlineto{\pgfqpoint{1.424952in}{2.248185in}}%
\pgfpathlineto{\pgfqpoint{1.421105in}{2.242863in}}%
\pgfpathlineto{\pgfqpoint{1.412889in}{2.231220in}}%
\pgfpathlineto{\pgfqpoint{1.412806in}{2.231112in}}%
\pgfpathlineto{\pgfqpoint{1.404018in}{2.219360in}}%
\pgfpathlineto{\pgfqpoint{1.400826in}{2.214975in}}%
\pgfpathlineto{\pgfqpoint{1.394820in}{2.207609in}}%
\pgfpathlineto{\pgfqpoint{1.388764in}{2.199959in}}%
\pgfpathlineto{\pgfqpoint{1.385000in}{2.195858in}}%
\pgfpathlineto{\pgfqpoint{1.376701in}{2.186518in}}%
\pgfpathlineto{\pgfqpoint{1.374102in}{2.184106in}}%
\pgfpathlineto{\pgfqpoint{1.364638in}{2.175013in}}%
\pgfpathlineto{\pgfqpoint{1.361053in}{2.172355in}}%
\pgfpathlineto{\pgfqpoint{1.352576in}{2.165829in}}%
\pgfpathlineto{\pgfqpoint{1.342720in}{2.160604in}}%
\pgfpathlineto{\pgfqpoint{1.340513in}{2.159384in}}%
\pgfpathlineto{\pgfqpoint{1.328450in}{2.155794in}}%
\pgfpathlineto{\pgfqpoint{1.316388in}{2.155322in}}%
\pgfpathlineto{\pgfqpoint{1.304325in}{2.157956in}}%
\pgfpathlineto{\pgfqpoint{1.298712in}{2.160604in}}%
\pgfpathlineto{\pgfqpoint{1.292262in}{2.163515in}}%
\pgfpathlineto{\pgfqpoint{1.280200in}{2.171812in}}%
\pgfpathlineto{\pgfqpoint{1.279605in}{2.172355in}}%
\pgfpathlineto{\pgfqpoint{1.268137in}{2.182412in}}%
\pgfpathlineto{\pgfqpoint{1.266566in}{2.184106in}}%
\pgfpathlineto{\pgfqpoint{1.256074in}{2.195024in}}%
\pgfpathlineto{\pgfqpoint{1.255389in}{2.195858in}}%
\pgfpathlineto{\pgfqpoint{1.245403in}{2.207609in}}%
\pgfpathlineto{\pgfqpoint{1.244012in}{2.209194in}}%
\pgfpathlineto{\pgfqpoint{1.236104in}{2.219360in}}%
\pgfpathlineto{\pgfqpoint{1.231949in}{2.224549in}}%
\pgfpathlineto{\pgfqpoint{1.227155in}{2.231112in}}%
\pgfpathlineto{\pgfqpoint{1.219886in}{2.240799in}}%
\pgfpathlineto{\pgfqpoint{1.218440in}{2.242863in}}%
\pgfpathlineto{\pgfqpoint{1.209983in}{2.254615in}}%
\pgfpathlineto{\pgfqpoint{1.207824in}{2.257540in}}%
\pgfpathlineto{\pgfqpoint{1.201619in}{2.266366in}}%
\pgfpathlineto{\pgfqpoint{1.195761in}{2.274516in}}%
\pgfpathlineto{\pgfqpoint{1.193251in}{2.278117in}}%
\pgfpathlineto{\pgfqpoint{1.184883in}{2.289869in}}%
\pgfpathlineto{\pgfqpoint{1.183698in}{2.291496in}}%
\pgfpathlineto{\pgfqpoint{1.176436in}{2.301620in}}%
\pgfpathlineto{\pgfqpoint{1.171636in}{2.308192in}}%
\pgfpathlineto{\pgfqpoint{1.167852in}{2.313371in}}%
\pgfpathlineto{\pgfqpoint{1.159573in}{2.324518in}}%
\pgfpathlineto{\pgfqpoint{1.159117in}{2.325123in}}%
\pgfpathlineto{\pgfqpoint{1.150092in}{2.336874in}}%
\pgfpathlineto{\pgfqpoint{1.147510in}{2.340185in}}%
\pgfpathlineto{\pgfqpoint{1.140742in}{2.348626in}}%
\pgfpathlineto{\pgfqpoint{1.135448in}{2.355141in}}%
\pgfpathlineto{\pgfqpoint{1.131011in}{2.360377in}}%
\pgfpathlineto{\pgfqpoint{1.123385in}{2.369270in}}%
\pgfpathlineto{\pgfqpoint{1.120793in}{2.372128in}}%
\pgfpathlineto{\pgfqpoint{1.111322in}{2.382463in}}%
\pgfpathlineto{\pgfqpoint{1.109930in}{2.383880in}}%
\pgfpathlineto{\pgfqpoint{1.099260in}{2.394637in}}%
\pgfpathlineto{\pgfqpoint{1.098187in}{2.395631in}}%
\pgfpathlineto{\pgfqpoint{1.087197in}{2.405738in}}%
\pgfpathlineto{\pgfqpoint{1.085224in}{2.407383in}}%
\pgfpathlineto{\pgfqpoint{1.075134in}{2.415737in}}%
\pgfpathlineto{\pgfqpoint{1.070545in}{2.419134in}}%
\pgfpathlineto{\pgfqpoint{1.063072in}{2.424634in}}%
\pgfpathlineto{\pgfqpoint{1.053439in}{2.430885in}}%
\pgfpathlineto{\pgfqpoint{1.051009in}{2.432455in}}%
\pgfpathlineto{\pgfqpoint{1.038946in}{2.439157in}}%
\pgfpathlineto{\pgfqpoint{1.031574in}{2.442637in}}%
\pgfpathlineto{\pgfqpoint{1.026884in}{2.444844in}}%
\pgfpathlineto{\pgfqpoint{1.014821in}{2.449528in}}%
\pgfpathlineto{\pgfqpoint{1.002758in}{2.453358in}}%
\pgfpathlineto{\pgfqpoint{0.998608in}{2.454388in}}%
\pgfpathlineto{\pgfqpoint{0.990696in}{2.456354in}}%
\pgfpathlineto{\pgfqpoint{0.978633in}{2.458674in}}%
\pgfpathlineto{\pgfqpoint{0.966570in}{2.460486in}}%
\pgfpathlineto{\pgfqpoint{0.954508in}{2.461959in}}%
\pgfpathlineto{\pgfqpoint{0.942445in}{2.463296in}}%
\pgfpathlineto{\pgfqpoint{0.930382in}{2.464735in}}%
\pgfpathlineto{\pgfqpoint{0.920965in}{2.466139in}}%
\pgfpathlineto{\pgfqpoint{0.918320in}{2.466538in}}%
\pgfpathlineto{\pgfqpoint{0.906257in}{2.468959in}}%
\pgfpathlineto{\pgfqpoint{0.894194in}{2.472405in}}%
\pgfpathlineto{\pgfqpoint{0.882132in}{2.477294in}}%
\pgfpathlineto{\pgfqpoint{0.881015in}{2.477891in}}%
\pgfpathlineto{\pgfqpoint{0.870069in}{2.483852in}}%
\pgfpathlineto{\pgfqpoint{0.862065in}{2.489642in}}%
\pgfpathlineto{\pgfqpoint{0.858006in}{2.492637in}}%
\pgfpathlineto{\pgfqpoint{0.848614in}{2.501394in}}%
\pgfpathlineto{\pgfqpoint{0.845944in}{2.503939in}}%
\pgfpathlineto{\pgfqpoint{0.838003in}{2.513145in}}%
\pgfpathlineto{\pgfqpoint{0.833881in}{2.518043in}}%
\pgfpathlineto{\pgfqpoint{0.828994in}{2.524896in}}%
\pgfpathlineto{\pgfqpoint{0.821818in}{2.535234in}}%
\pgfpathlineto{\pgfqpoint{0.820966in}{2.536648in}}%
\pgfpathlineto{\pgfqpoint{0.813899in}{2.548399in}}%
\pgfpathlineto{\pgfqpoint{0.809756in}{2.555450in}}%
\pgfpathlineto{\pgfqpoint{0.807294in}{2.560150in}}%
\pgfpathlineto{\pgfqpoint{0.801190in}{2.571902in}}%
\pgfpathlineto{\pgfqpoint{0.797693in}{2.578731in}}%
\pgfpathlineto{\pgfqpoint{0.795400in}{2.583653in}}%
\pgfpathlineto{\pgfqpoint{0.789947in}{2.595405in}}%
\pgfpathlineto{\pgfqpoint{0.785630in}{2.604864in}}%
\pgfpathlineto{\pgfqpoint{0.784662in}{2.607156in}}%
\pgfpathlineto{\pgfqpoint{0.779663in}{2.618907in}}%
\pgfpathlineto{\pgfqpoint{0.774757in}{2.630659in}}%
\pgfpathlineto{\pgfqpoint{0.773567in}{2.633493in}}%
\pgfpathlineto{\pgfqpoint{0.770044in}{2.642410in}}%
\pgfpathlineto{\pgfqpoint{0.765438in}{2.654161in}}%
\pgfpathlineto{\pgfqpoint{0.761505in}{2.664305in}}%
\pgfpathlineto{\pgfqpoint{0.760910in}{2.665913in}}%
\pgfpathlineto{\pgfqpoint{0.756516in}{2.677664in}}%
\pgfpathlineto{\pgfqpoint{0.752178in}{2.689416in}}%
\pgfpathlineto{\pgfqpoint{0.749442in}{2.696844in}}%
\pgfpathlineto{\pgfqpoint{0.747904in}{2.701167in}}%
\pgfpathlineto{\pgfqpoint{0.743701in}{2.712918in}}%
\pgfpathlineto{\pgfqpoint{0.739545in}{2.724670in}}%
\pgfpathlineto{\pgfqpoint{0.737379in}{2.730784in}}%
\pgfpathlineto{\pgfqpoint{0.735429in}{2.736421in}}%
\pgfpathlineto{\pgfqpoint{0.731352in}{2.748172in}}%
\pgfpathlineto{\pgfqpoint{0.727317in}{2.759924in}}%
\pgfpathlineto{\pgfqpoint{0.725317in}{2.765737in}}%
\pgfpathlineto{\pgfqpoint{0.723299in}{2.771675in}}%
\pgfpathlineto{\pgfqpoint{0.719299in}{2.783427in}}%
\pgfpathlineto{\pgfqpoint{0.715339in}{2.795178in}}%
\pgfpathlineto{\pgfqpoint{0.713254in}{2.801357in}}%
\pgfpathlineto{\pgfqpoint{0.711379in}{2.806929in}}%
\pgfpathlineto{\pgfqpoint{0.707418in}{2.818681in}}%
\pgfpathlineto{\pgfqpoint{0.703497in}{2.830432in}}%
\pgfpathlineto{\pgfqpoint{0.701191in}{2.837342in}}%
\pgfpathlineto{\pgfqpoint{0.699567in}{2.842183in}}%
\pgfpathlineto{\pgfqpoint{0.695612in}{2.853935in}}%
\pgfpathlineto{\pgfqpoint{0.691699in}{2.865686in}}%
\pgfpathlineto{\pgfqpoint{0.689129in}{2.873424in}}%
\pgfpathlineto{\pgfqpoint{0.687776in}{2.877438in}}%
\pgfpathlineto{\pgfqpoint{0.683802in}{2.889189in}}%
\pgfpathlineto{\pgfqpoint{0.679871in}{2.900940in}}%
\pgfpathlineto{\pgfqpoint{0.677066in}{2.909364in}}%
\pgfpathlineto{\pgfqpoint{0.675933in}{2.912692in}}%
\pgfpathlineto{\pgfqpoint{0.671915in}{2.924443in}}%
\pgfpathlineto{\pgfqpoint{0.667946in}{2.936194in}}%
\pgfpathlineto{\pgfqpoint{0.665003in}{2.944957in}}%
\pgfpathlineto{\pgfqpoint{0.663969in}{2.947946in}}%
\pgfpathlineto{\pgfqpoint{0.659888in}{2.959697in}}%
\pgfpathlineto{\pgfqpoint{0.655861in}{2.971449in}}%
\pgfpathlineto{\pgfqpoint{0.652941in}{2.980025in}}%
\pgfpathlineto{\pgfqpoint{0.651818in}{2.983200in}}%
\pgfpathlineto{\pgfqpoint{0.640878in}{2.983200in}}%
\pgfpathlineto{\pgfqpoint{0.628815in}{2.983200in}}%
\pgfpathlineto{\pgfqpoint{0.616753in}{2.983200in}}%
\pgfpathlineto{\pgfqpoint{0.605088in}{2.983200in}}%
\pgfpathlineto{\pgfqpoint{0.609175in}{2.971449in}}%
\pgfpathlineto{\pgfqpoint{0.613357in}{2.959697in}}%
\pgfpathlineto{\pgfqpoint{0.616753in}{2.950328in}}%
\pgfpathlineto{\pgfqpoint{0.617547in}{2.947946in}}%
\pgfpathlineto{\pgfqpoint{0.621445in}{2.936194in}}%
\pgfpathlineto{\pgfqpoint{0.625420in}{2.924443in}}%
\pgfpathlineto{\pgfqpoint{0.628815in}{2.914562in}}%
\pgfpathlineto{\pgfqpoint{0.629412in}{2.912692in}}%
\pgfpathlineto{\pgfqpoint{0.633127in}{2.900940in}}%
\pgfpathlineto{\pgfqpoint{0.636902in}{2.889189in}}%
\pgfpathlineto{\pgfqpoint{0.640743in}{2.877438in}}%
\pgfpathlineto{\pgfqpoint{0.640878in}{2.877018in}}%
\pgfpathlineto{\pgfqpoint{0.644296in}{2.865686in}}%
\pgfpathlineto{\pgfqpoint{0.647882in}{2.853935in}}%
\pgfpathlineto{\pgfqpoint{0.651517in}{2.842183in}}%
\pgfpathlineto{\pgfqpoint{0.652941in}{2.837561in}}%
\pgfpathlineto{\pgfqpoint{0.655013in}{2.830432in}}%
\pgfpathlineto{\pgfqpoint{0.658422in}{2.818681in}}%
\pgfpathlineto{\pgfqpoint{0.661866in}{2.806929in}}%
\pgfpathlineto{\pgfqpoint{0.665003in}{2.796306in}}%
\pgfpathlineto{\pgfqpoint{0.665321in}{2.795178in}}%
\pgfpathlineto{\pgfqpoint{0.668566in}{2.783427in}}%
\pgfpathlineto{\pgfqpoint{0.671832in}{2.771675in}}%
\pgfpathlineto{\pgfqpoint{0.675120in}{2.759924in}}%
\pgfpathlineto{\pgfqpoint{0.677066in}{2.752934in}}%
\pgfpathlineto{\pgfqpoint{0.678340in}{2.748172in}}%
\pgfpathlineto{\pgfqpoint{0.681440in}{2.736421in}}%
\pgfpathlineto{\pgfqpoint{0.684549in}{2.724670in}}%
\pgfpathlineto{\pgfqpoint{0.687668in}{2.712918in}}%
\pgfpathlineto{\pgfqpoint{0.689129in}{2.707336in}}%
\pgfpathlineto{\pgfqpoint{0.690696in}{2.701167in}}%
\pgfpathlineto{\pgfqpoint{0.693635in}{2.689416in}}%
\pgfpathlineto{\pgfqpoint{0.696569in}{2.677664in}}%
\pgfpathlineto{\pgfqpoint{0.699497in}{2.665913in}}%
\pgfpathlineto{\pgfqpoint{0.701191in}{2.659001in}}%
\pgfpathlineto{\pgfqpoint{0.702355in}{2.654161in}}%
\pgfpathlineto{\pgfqpoint{0.705108in}{2.642410in}}%
\pgfpathlineto{\pgfqpoint{0.707839in}{2.630659in}}%
\pgfpathlineto{\pgfqpoint{0.710545in}{2.618907in}}%
\pgfpathlineto{\pgfqpoint{0.713224in}{2.607156in}}%
\pgfpathlineto{\pgfqpoint{0.713254in}{2.607015in}}%
\pgfpathlineto{\pgfqpoint{0.715752in}{2.595405in}}%
\pgfpathlineto{\pgfqpoint{0.718233in}{2.583653in}}%
\pgfpathlineto{\pgfqpoint{0.720662in}{2.571902in}}%
\pgfpathlineto{\pgfqpoint{0.723033in}{2.560150in}}%
\pgfpathlineto{\pgfqpoint{0.725317in}{2.548507in}}%
\pgfpathlineto{\pgfqpoint{0.725338in}{2.548399in}}%
\pgfpathlineto{\pgfqpoint{0.727486in}{2.536648in}}%
\pgfpathlineto{\pgfqpoint{0.729541in}{2.524896in}}%
\pgfpathlineto{\pgfqpoint{0.731489in}{2.513145in}}%
\pgfpathlineto{\pgfqpoint{0.733316in}{2.501394in}}%
\pgfpathlineto{\pgfqpoint{0.735003in}{2.489642in}}%
\pgfpathlineto{\pgfqpoint{0.736528in}{2.477891in}}%
\pgfpathlineto{\pgfqpoint{0.737379in}{2.470239in}}%
\pgfpathlineto{\pgfqpoint{0.737847in}{2.466139in}}%
\pgfpathlineto{\pgfqpoint{0.738919in}{2.454388in}}%
\pgfpathlineto{\pgfqpoint{0.739725in}{2.442637in}}%
\pgfpathlineto{\pgfqpoint{0.740213in}{2.430885in}}%
\pgfpathlineto{\pgfqpoint{0.740318in}{2.419134in}}%
\pgfpathlineto{\pgfqpoint{0.739955in}{2.407383in}}%
\pgfpathlineto{\pgfqpoint{0.739012in}{2.395631in}}%
\pgfpathlineto{\pgfqpoint{0.737379in}{2.384151in}}%
\pgfpathlineto{\pgfqpoint{0.737335in}{2.383880in}}%
\pgfpathlineto{\pgfqpoint{0.734528in}{2.372128in}}%
\pgfpathlineto{\pgfqpoint{0.730322in}{2.360377in}}%
\pgfpathlineto{\pgfqpoint{0.725317in}{2.350610in}}%
\pgfpathlineto{\pgfqpoint{0.723948in}{2.348626in}}%
\pgfpathlineto{\pgfqpoint{0.713254in}{2.337246in}}%
\pgfpathlineto{\pgfqpoint{0.712642in}{2.336874in}}%
\pgfpathlineto{\pgfqpoint{0.701191in}{2.331399in}}%
\pgfpathlineto{\pgfqpoint{0.689129in}{2.330125in}}%
\pgfpathlineto{\pgfqpoint{0.677066in}{2.332514in}}%
\pgfpathlineto{\pgfqpoint{0.667687in}{2.336874in}}%
\pgfpathlineto{\pgfqpoint{0.665003in}{2.338051in}}%
\pgfpathlineto{\pgfqpoint{0.652941in}{2.346595in}}%
\pgfpathlineto{\pgfqpoint{0.650876in}{2.348626in}}%
\pgfpathlineto{\pgfqpoint{0.640878in}{2.357946in}}%
\pgfpathlineto{\pgfqpoint{0.638880in}{2.360377in}}%
\pgfpathlineto{\pgfqpoint{0.628815in}{2.372003in}}%
\pgfpathlineto{\pgfqpoint{0.628729in}{2.372128in}}%
\pgfpathlineto{\pgfqpoint{0.620459in}{2.383880in}}%
\pgfpathlineto{\pgfqpoint{0.616753in}{2.388876in}}%
\pgfpathlineto{\pgfqpoint{0.612702in}{2.395631in}}%
\pgfpathlineto{\pgfqpoint{0.605313in}{2.407383in}}%
\pgfpathlineto{\pgfqpoint{0.604690in}{2.408352in}}%
\pgfpathlineto{\pgfqpoint{0.598993in}{2.419134in}}%
\pgfpathlineto{\pgfqpoint{0.592627in}{2.430555in}}%
\pgfpathlineto{\pgfqpoint{0.592473in}{2.430885in}}%
\pgfpathlineto{\pgfqpoint{0.586930in}{2.442637in}}%
\pgfpathlineto{\pgfqpoint{0.581111in}{2.454388in}}%
\pgfpathlineto{\pgfqpoint{0.580565in}{2.455481in}}%
\pgfpathlineto{\pgfqpoint{0.576036in}{2.466139in}}%
\pgfpathlineto{\pgfqpoint{0.570857in}{2.477891in}}%
\pgfpathlineto{\pgfqpoint{0.568502in}{2.483128in}}%
\pgfpathlineto{\pgfqpoint{0.565977in}{2.489642in}}%
\pgfpathlineto{\pgfqpoint{0.561341in}{2.501394in}}%
\pgfpathlineto{\pgfqpoint{0.556536in}{2.513145in}}%
\pgfpathlineto{\pgfqpoint{0.556439in}{2.513380in}}%
\pgfpathlineto{\pgfqpoint{0.552344in}{2.524896in}}%
\pgfpathlineto{\pgfqpoint{0.548043in}{2.536648in}}%
\pgfpathlineto{\pgfqpoint{0.544377in}{2.546412in}}%
\pgfpathlineto{\pgfqpoint{0.543719in}{2.548399in}}%
\pgfpathlineto{\pgfqpoint{0.539838in}{2.560150in}}%
\pgfpathlineto{\pgfqpoint{0.535854in}{2.571902in}}%
\pgfpathlineto{\pgfqpoint{0.532314in}{2.582105in}}%
\pgfpathlineto{\pgfqpoint{0.531837in}{2.583653in}}%
\pgfpathlineto{\pgfqpoint{0.528231in}{2.595405in}}%
\pgfpathlineto{\pgfqpoint{0.524543in}{2.607156in}}%
\pgfpathlineto{\pgfqpoint{0.520763in}{2.618907in}}%
\pgfpathlineto{\pgfqpoint{0.520251in}{2.620506in}}%
\pgfpathlineto{\pgfqpoint{0.517343in}{2.630659in}}%
\pgfpathlineto{\pgfqpoint{0.513925in}{2.642410in}}%
\pgfpathlineto{\pgfqpoint{0.510434in}{2.654161in}}%
\pgfpathlineto{\pgfqpoint{0.508189in}{2.661644in}}%
\pgfpathlineto{\pgfqpoint{0.507033in}{2.665913in}}%
\pgfpathlineto{\pgfqpoint{0.503860in}{2.677664in}}%
\pgfpathlineto{\pgfqpoint{0.500629in}{2.689416in}}%
\pgfpathlineto{\pgfqpoint{0.497336in}{2.701167in}}%
\pgfpathlineto{\pgfqpoint{0.496126in}{2.705487in}}%
\pgfpathlineto{\pgfqpoint{0.494238in}{2.712918in}}%
\pgfpathlineto{\pgfqpoint{0.491241in}{2.724670in}}%
\pgfpathlineto{\pgfqpoint{0.488196in}{2.736421in}}%
\pgfpathlineto{\pgfqpoint{0.485097in}{2.748172in}}%
\pgfpathlineto{\pgfqpoint{0.484063in}{2.752108in}}%
\pgfpathlineto{\pgfqpoint{0.482190in}{2.759924in}}%
\pgfpathlineto{\pgfqpoint{0.479367in}{2.771675in}}%
\pgfpathlineto{\pgfqpoint{0.476503in}{2.783427in}}%
\pgfpathlineto{\pgfqpoint{0.473595in}{2.795178in}}%
\pgfpathlineto{\pgfqpoint{0.472001in}{2.801615in}}%
\pgfpathlineto{\pgfqpoint{0.470793in}{2.806929in}}%
\pgfpathlineto{\pgfqpoint{0.468139in}{2.818681in}}%
\pgfpathlineto{\pgfqpoint{0.465454in}{2.830432in}}%
\pgfpathlineto{\pgfqpoint{0.462733in}{2.842183in}}%
\pgfpathlineto{\pgfqpoint{0.459973in}{2.853935in}}%
\pgfpathlineto{\pgfqpoint{0.459938in}{2.854088in}}%
\pgfpathlineto{\pgfqpoint{0.457484in}{2.865686in}}%
\pgfpathlineto{\pgfqpoint{0.454976in}{2.877438in}}%
\pgfpathlineto{\pgfqpoint{0.452442in}{2.889189in}}%
\pgfpathlineto{\pgfqpoint{0.449879in}{2.900940in}}%
\pgfpathlineto{\pgfqpoint{0.447875in}{2.910093in}}%
\pgfpathlineto{\pgfqpoint{0.447349in}{2.912692in}}%
\pgfpathlineto{\pgfqpoint{0.445019in}{2.924443in}}%
\pgfpathlineto{\pgfqpoint{0.442675in}{2.936194in}}%
\pgfpathlineto{\pgfqpoint{0.440313in}{2.947946in}}%
\pgfpathlineto{\pgfqpoint{0.437932in}{2.959697in}}%
\pgfpathlineto{\pgfqpoint{0.435813in}{2.970111in}}%
\pgfpathlineto{\pgfqpoint{0.435560in}{2.971449in}}%
\pgfpathlineto{\pgfqpoint{0.433414in}{2.983200in}}%
\pgfpathlineto{\pgfqpoint{0.423750in}{2.983200in}}%
\pgfpathlineto{\pgfqpoint{0.423750in}{2.971449in}}%
\pgfpathlineto{\pgfqpoint{0.423750in}{2.959697in}}%
\pgfpathlineto{\pgfqpoint{0.423750in}{2.947946in}}%
\pgfpathlineto{\pgfqpoint{0.423750in}{2.936194in}}%
\pgfpathlineto{\pgfqpoint{0.423750in}{2.924443in}}%
\pgfpathlineto{\pgfqpoint{0.423750in}{2.912692in}}%
\pgfpathlineto{\pgfqpoint{0.423750in}{2.900940in}}%
\pgfpathlineto{\pgfqpoint{0.423750in}{2.889189in}}%
\pgfpathlineto{\pgfqpoint{0.423750in}{2.877438in}}%
\pgfpathlineto{\pgfqpoint{0.423750in}{2.865686in}}%
\pgfpathlineto{\pgfqpoint{0.423750in}{2.853935in}}%
\pgfpathlineto{\pgfqpoint{0.423750in}{2.842183in}}%
\pgfpathlineto{\pgfqpoint{0.423750in}{2.830478in}}%
\pgfpathlineto{\pgfqpoint{0.423762in}{2.830432in}}%
\pgfpathlineto{\pgfqpoint{0.426874in}{2.818681in}}%
\pgfpathlineto{\pgfqpoint{0.429934in}{2.806929in}}%
\pgfpathlineto{\pgfqpoint{0.432946in}{2.795178in}}%
\pgfpathlineto{\pgfqpoint{0.435813in}{2.783835in}}%
\pgfpathlineto{\pgfqpoint{0.435921in}{2.783427in}}%
\pgfpathlineto{\pgfqpoint{0.439074in}{2.771675in}}%
\pgfpathlineto{\pgfqpoint{0.442176in}{2.759924in}}%
\pgfpathlineto{\pgfqpoint{0.445231in}{2.748172in}}%
\pgfpathlineto{\pgfqpoint{0.447875in}{2.737887in}}%
\pgfpathlineto{\pgfqpoint{0.448274in}{2.736421in}}%
\pgfpathlineto{\pgfqpoint{0.451484in}{2.724670in}}%
\pgfpathlineto{\pgfqpoint{0.454645in}{2.712918in}}%
\pgfpathlineto{\pgfqpoint{0.457759in}{2.701167in}}%
\pgfpathlineto{\pgfqpoint{0.459938in}{2.692884in}}%
\pgfpathlineto{\pgfqpoint{0.460906in}{2.689416in}}%
\pgfpathlineto{\pgfqpoint{0.464191in}{2.677664in}}%
\pgfpathlineto{\pgfqpoint{0.467424in}{2.665913in}}%
\pgfpathlineto{\pgfqpoint{0.470611in}{2.654161in}}%
\pgfpathlineto{\pgfqpoint{0.472001in}{2.649028in}}%
\pgfpathlineto{\pgfqpoint{0.473911in}{2.642410in}}%
\pgfpathlineto{\pgfqpoint{0.477284in}{2.630659in}}%
\pgfpathlineto{\pgfqpoint{0.480604in}{2.618907in}}%
\pgfpathlineto{\pgfqpoint{0.483875in}{2.607156in}}%
\pgfpathlineto{\pgfqpoint{0.484063in}{2.606482in}}%
\pgfpathlineto{\pgfqpoint{0.487388in}{2.595405in}}%
\pgfpathlineto{\pgfqpoint{0.490861in}{2.583653in}}%
\pgfpathlineto{\pgfqpoint{0.494279in}{2.571902in}}%
\pgfpathlineto{\pgfqpoint{0.496126in}{2.565520in}}%
\pgfpathlineto{\pgfqpoint{0.497800in}{2.560150in}}%
\pgfpathlineto{\pgfqpoint{0.501446in}{2.548399in}}%
\pgfpathlineto{\pgfqpoint{0.505030in}{2.536648in}}%
\pgfpathlineto{\pgfqpoint{0.508189in}{2.526149in}}%
\pgfpathlineto{\pgfqpoint{0.508597in}{2.524896in}}%
\pgfpathlineto{\pgfqpoint{0.512437in}{2.513145in}}%
\pgfpathlineto{\pgfqpoint{0.516207in}{2.501394in}}%
\pgfpathlineto{\pgfqpoint{0.519911in}{2.489642in}}%
\pgfpathlineto{\pgfqpoint{0.520251in}{2.488568in}}%
\pgfpathlineto{\pgfqpoint{0.523928in}{2.477891in}}%
\pgfpathlineto{\pgfqpoint{0.527903in}{2.466139in}}%
\pgfpathlineto{\pgfqpoint{0.531803in}{2.454388in}}%
\pgfpathlineto{\pgfqpoint{0.532314in}{2.452852in}}%
\pgfpathlineto{\pgfqpoint{0.536032in}{2.442637in}}%
\pgfpathlineto{\pgfqpoint{0.540231in}{2.430885in}}%
\pgfpathlineto{\pgfqpoint{0.544345in}{2.419134in}}%
\pgfpathlineto{\pgfqpoint{0.544377in}{2.419043in}}%
\pgfpathlineto{\pgfqpoint{0.548887in}{2.407383in}}%
\pgfpathlineto{\pgfqpoint{0.553327in}{2.395631in}}%
\pgfpathlineto{\pgfqpoint{0.556439in}{2.387267in}}%
\pgfpathlineto{\pgfqpoint{0.557837in}{2.383880in}}%
\pgfpathlineto{\pgfqpoint{0.562653in}{2.372128in}}%
\pgfpathlineto{\pgfqpoint{0.567346in}{2.360377in}}%
\pgfpathlineto{\pgfqpoint{0.568502in}{2.357469in}}%
\pgfpathlineto{\pgfqpoint{0.572420in}{2.348626in}}%
\pgfpathlineto{\pgfqpoint{0.577513in}{2.336874in}}%
\pgfpathlineto{\pgfqpoint{0.580565in}{2.329712in}}%
\pgfpathlineto{\pgfqpoint{0.582760in}{2.325123in}}%
\pgfpathlineto{\pgfqpoint{0.588302in}{2.313371in}}%
\pgfpathlineto{\pgfqpoint{0.592627in}{2.303959in}}%
\pgfpathlineto{\pgfqpoint{0.593843in}{2.301620in}}%
\pgfpathlineto{\pgfqpoint{0.599880in}{2.289869in}}%
\pgfpathlineto{\pgfqpoint{0.604690in}{2.280218in}}%
\pgfpathlineto{\pgfqpoint{0.605881in}{2.278117in}}%
\pgfpathlineto{\pgfqpoint{0.612455in}{2.266366in}}%
\pgfpathlineto{\pgfqpoint{0.616753in}{2.258452in}}%
\pgfpathlineto{\pgfqpoint{0.619135in}{2.254615in}}%
\pgfpathlineto{\pgfqpoint{0.626270in}{2.242863in}}%
\pgfpathlineto{\pgfqpoint{0.628815in}{2.238574in}}%
\pgfpathlineto{\pgfqpoint{0.633895in}{2.231112in}}%
\pgfpathlineto{\pgfqpoint{0.640878in}{2.220486in}}%
\pgfpathlineto{\pgfqpoint{0.641732in}{2.219360in}}%
\pgfpathlineto{\pgfqpoint{0.650456in}{2.207609in}}%
\pgfpathlineto{\pgfqpoint{0.652941in}{2.204160in}}%
\pgfpathlineto{\pgfqpoint{0.659832in}{2.195858in}}%
\pgfpathlineto{\pgfqpoint{0.665003in}{2.189410in}}%
\pgfpathlineto{\pgfqpoint{0.669906in}{2.184106in}}%
\pgfpathlineto{\pgfqpoint{0.677066in}{2.176089in}}%
\pgfpathlineto{\pgfqpoint{0.680892in}{2.172355in}}%
\pgfpathlineto{\pgfqpoint{0.689129in}{2.164037in}}%
\pgfpathlineto{\pgfqpoint{0.692988in}{2.160604in}}%
\pgfpathlineto{\pgfqpoint{0.701191in}{2.153052in}}%
\pgfpathlineto{\pgfqpoint{0.706278in}{2.148852in}}%
\pgfpathlineto{\pgfqpoint{0.713254in}{2.142894in}}%
\pgfpathlineto{\pgfqpoint{0.720620in}{2.137101in}}%
\pgfpathlineto{\pgfqpoint{0.725317in}{2.133281in}}%
\pgfpathlineto{\pgfqpoint{0.735570in}{2.125349in}}%
\pgfpathlineto{\pgfqpoint{0.737379in}{2.123902in}}%
\pgfpathlineto{\pgfqpoint{0.749442in}{2.114449in}}%
\pgfpathlineto{\pgfqpoint{0.750515in}{2.113598in}}%
\pgfpathlineto{\pgfqpoint{0.761505in}{2.104575in}}%
\pgfpathlineto{\pgfqpoint{0.764646in}{2.101847in}}%
\pgfpathlineto{\pgfqpoint{0.773567in}{2.093836in}}%
\pgfpathlineto{\pgfqpoint{0.777415in}{2.090095in}}%
\pgfpathlineto{\pgfqpoint{0.785630in}{2.081844in}}%
\pgfpathlineto{\pgfqpoint{0.788804in}{2.078344in}}%
\pgfpathlineto{\pgfqpoint{0.797693in}{2.068224in}}%
\pgfpathlineto{\pgfqpoint{0.798990in}{2.066593in}}%
\pgfpathlineto{\pgfqpoint{0.808144in}{2.054841in}}%
\pgfpathlineto{\pgfqpoint{0.809756in}{2.052727in}}%
\pgfpathlineto{\pgfqpoint{0.816448in}{2.043090in}}%
\pgfpathlineto{\pgfqpoint{0.821818in}{2.035133in}}%
\pgfpathlineto{\pgfqpoint{0.824159in}{2.031338in}}%
\pgfpathlineto{\pgfqpoint{0.831325in}{2.019587in}}%
\pgfpathlineto{\pgfqpoint{0.833881in}{2.015344in}}%
\pgfpathlineto{\pgfqpoint{0.838058in}{2.007836in}}%
\pgfpathlineto{\pgfqpoint{0.844489in}{1.996084in}}%
\pgfpathlineto{\pgfqpoint{0.845944in}{1.993427in}}%
\pgfpathlineto{\pgfqpoint{0.850590in}{1.984333in}}%
\pgfpathlineto{\pgfqpoint{0.856510in}{1.972582in}}%
\pgfpathlineto{\pgfqpoint{0.858006in}{1.969621in}}%
\pgfpathlineto{\pgfqpoint{0.862201in}{1.960830in}}%
\pgfpathlineto{\pgfqpoint{0.867763in}{1.949079in}}%
\pgfpathlineto{\pgfqpoint{0.870069in}{1.944220in}}%
\pgfpathlineto{\pgfqpoint{0.873188in}{1.937327in}}%
\pgfpathlineto{\pgfqpoint{0.878508in}{1.925576in}}%
\pgfpathlineto{\pgfqpoint{0.882132in}{1.917556in}}%
\pgfpathlineto{\pgfqpoint{0.883757in}{1.913825in}}%
\pgfpathlineto{\pgfqpoint{0.888924in}{1.902073in}}%
\pgfpathlineto{\pgfqpoint{0.894054in}{1.890322in}}%
\pgfpathlineto{\pgfqpoint{0.894194in}{1.890005in}}%
\pgfpathlineto{\pgfqpoint{0.899141in}{1.878571in}}%
\pgfpathlineto{\pgfqpoint{0.904204in}{1.866819in}}%
\pgfpathlineto{\pgfqpoint{0.906257in}{1.862091in}}%
\pgfpathlineto{\pgfqpoint{0.909263in}{1.855068in}}%
\pgfpathlineto{\pgfqpoint{0.914318in}{1.843316in}}%
\pgfpathlineto{\pgfqpoint{0.918320in}{1.834023in}}%
\pgfpathlineto{\pgfqpoint{0.919373in}{1.831565in}}%
\pgfpathlineto{\pgfqpoint{0.924478in}{1.819814in}}%
\pgfpathclose%
\pgfusepath{fill}%
\end{pgfscope}%
\begin{pgfscope}%
\pgfpathrectangle{\pgfqpoint{0.423750in}{1.819814in}}{\pgfqpoint{1.194205in}{1.163386in}}%
\pgfusepath{clip}%
\pgfsetbuttcap%
\pgfsetroundjoin%
\definecolor{currentfill}{rgb}{0.976961,0.885681,0.814303}%
\pgfsetfillcolor{currentfill}%
\pgfsetlinewidth{0.000000pt}%
\definecolor{currentstroke}{rgb}{0.000000,0.000000,0.000000}%
\pgfsetstrokecolor{currentstroke}%
\pgfsetdash{}{0pt}%
\pgfpathmoveto{\pgfqpoint{0.990696in}{1.822928in}}%
\pgfpathlineto{\pgfqpoint{0.992107in}{1.819814in}}%
\pgfpathlineto{\pgfqpoint{1.002758in}{1.819814in}}%
\pgfpathlineto{\pgfqpoint{1.014821in}{1.819814in}}%
\pgfpathlineto{\pgfqpoint{1.026884in}{1.819814in}}%
\pgfpathlineto{\pgfqpoint{1.038946in}{1.819814in}}%
\pgfpathlineto{\pgfqpoint{1.051009in}{1.819814in}}%
\pgfpathlineto{\pgfqpoint{1.063072in}{1.819814in}}%
\pgfpathlineto{\pgfqpoint{1.075134in}{1.819814in}}%
\pgfpathlineto{\pgfqpoint{1.087197in}{1.819814in}}%
\pgfpathlineto{\pgfqpoint{1.099260in}{1.819814in}}%
\pgfpathlineto{\pgfqpoint{1.111322in}{1.819814in}}%
\pgfpathlineto{\pgfqpoint{1.123385in}{1.819814in}}%
\pgfpathlineto{\pgfqpoint{1.135448in}{1.819814in}}%
\pgfpathlineto{\pgfqpoint{1.147510in}{1.819814in}}%
\pgfpathlineto{\pgfqpoint{1.159573in}{1.819814in}}%
\pgfpathlineto{\pgfqpoint{1.171636in}{1.819814in}}%
\pgfpathlineto{\pgfqpoint{1.183698in}{1.819814in}}%
\pgfpathlineto{\pgfqpoint{1.195761in}{1.819814in}}%
\pgfpathlineto{\pgfqpoint{1.207824in}{1.819814in}}%
\pgfpathlineto{\pgfqpoint{1.219886in}{1.819814in}}%
\pgfpathlineto{\pgfqpoint{1.226230in}{1.819814in}}%
\pgfpathlineto{\pgfqpoint{1.229621in}{1.831565in}}%
\pgfpathlineto{\pgfqpoint{1.231949in}{1.841015in}}%
\pgfpathlineto{\pgfqpoint{1.232548in}{1.843316in}}%
\pgfpathlineto{\pgfqpoint{1.235112in}{1.855068in}}%
\pgfpathlineto{\pgfqpoint{1.237177in}{1.866819in}}%
\pgfpathlineto{\pgfqpoint{1.238755in}{1.878571in}}%
\pgfpathlineto{\pgfqpoint{1.239859in}{1.890322in}}%
\pgfpathlineto{\pgfqpoint{1.240502in}{1.902073in}}%
\pgfpathlineto{\pgfqpoint{1.240696in}{1.913825in}}%
\pgfpathlineto{\pgfqpoint{1.240453in}{1.925576in}}%
\pgfpathlineto{\pgfqpoint{1.239784in}{1.937327in}}%
\pgfpathlineto{\pgfqpoint{1.238699in}{1.949079in}}%
\pgfpathlineto{\pgfqpoint{1.237210in}{1.960830in}}%
\pgfpathlineto{\pgfqpoint{1.235326in}{1.972582in}}%
\pgfpathlineto{\pgfqpoint{1.233058in}{1.984333in}}%
\pgfpathlineto{\pgfqpoint{1.231949in}{1.989238in}}%
\pgfpathlineto{\pgfqpoint{1.230523in}{1.996084in}}%
\pgfpathlineto{\pgfqpoint{1.227731in}{2.007836in}}%
\pgfpathlineto{\pgfqpoint{1.224610in}{2.019587in}}%
\pgfpathlineto{\pgfqpoint{1.221171in}{2.031338in}}%
\pgfpathlineto{\pgfqpoint{1.219886in}{2.035341in}}%
\pgfpathlineto{\pgfqpoint{1.217542in}{2.043090in}}%
\pgfpathlineto{\pgfqpoint{1.213692in}{2.054841in}}%
\pgfpathlineto{\pgfqpoint{1.209567in}{2.066593in}}%
\pgfpathlineto{\pgfqpoint{1.207824in}{2.071232in}}%
\pgfpathlineto{\pgfqpoint{1.205252in}{2.078344in}}%
\pgfpathlineto{\pgfqpoint{1.200740in}{2.090095in}}%
\pgfpathlineto{\pgfqpoint{1.195990in}{2.101847in}}%
\pgfpathlineto{\pgfqpoint{1.195761in}{2.102382in}}%
\pgfpathlineto{\pgfqpoint{1.191059in}{2.113598in}}%
\pgfpathlineto{\pgfqpoint{1.185917in}{2.125349in}}%
\pgfpathlineto{\pgfqpoint{1.183698in}{2.130195in}}%
\pgfpathlineto{\pgfqpoint{1.180545in}{2.137101in}}%
\pgfpathlineto{\pgfqpoint{1.174965in}{2.148852in}}%
\pgfpathlineto{\pgfqpoint{1.171636in}{2.155607in}}%
\pgfpathlineto{\pgfqpoint{1.169139in}{2.160604in}}%
\pgfpathlineto{\pgfqpoint{1.163059in}{2.172355in}}%
\pgfpathlineto{\pgfqpoint{1.159573in}{2.178884in}}%
\pgfpathlineto{\pgfqpoint{1.156701in}{2.184106in}}%
\pgfpathlineto{\pgfqpoint{1.150046in}{2.195858in}}%
\pgfpathlineto{\pgfqpoint{1.147510in}{2.200210in}}%
\pgfpathlineto{\pgfqpoint{1.142998in}{2.207609in}}%
\pgfpathlineto{\pgfqpoint{1.135681in}{2.219360in}}%
\pgfpathlineto{\pgfqpoint{1.135448in}{2.219724in}}%
\pgfpathlineto{\pgfqpoint{1.127687in}{2.231112in}}%
\pgfpathlineto{\pgfqpoint{1.123385in}{2.237316in}}%
\pgfpathlineto{\pgfqpoint{1.119217in}{2.242863in}}%
\pgfpathlineto{\pgfqpoint{1.111322in}{2.253210in}}%
\pgfpathlineto{\pgfqpoint{1.110137in}{2.254615in}}%
\pgfpathlineto{\pgfqpoint{1.100069in}{2.266366in}}%
\pgfpathlineto{\pgfqpoint{1.099260in}{2.267296in}}%
\pgfpathlineto{\pgfqpoint{1.088656in}{2.278117in}}%
\pgfpathlineto{\pgfqpoint{1.087197in}{2.279589in}}%
\pgfpathlineto{\pgfqpoint{1.075441in}{2.289869in}}%
\pgfpathlineto{\pgfqpoint{1.075134in}{2.290134in}}%
\pgfpathlineto{\pgfqpoint{1.063072in}{2.298845in}}%
\pgfpathlineto{\pgfqpoint{1.058281in}{2.301620in}}%
\pgfpathlineto{\pgfqpoint{1.051009in}{2.305804in}}%
\pgfpathlineto{\pgfqpoint{1.038946in}{2.311014in}}%
\pgfpathlineto{\pgfqpoint{1.030677in}{2.313371in}}%
\pgfpathlineto{\pgfqpoint{1.026884in}{2.314449in}}%
\pgfpathlineto{\pgfqpoint{1.014821in}{2.316059in}}%
\pgfpathlineto{\pgfqpoint{1.002758in}{2.315896in}}%
\pgfpathlineto{\pgfqpoint{0.990696in}{2.313934in}}%
\pgfpathlineto{\pgfqpoint{0.988816in}{2.313371in}}%
\pgfpathlineto{\pgfqpoint{0.978633in}{2.309995in}}%
\pgfpathlineto{\pgfqpoint{0.966570in}{2.304089in}}%
\pgfpathlineto{\pgfqpoint{0.962621in}{2.301620in}}%
\pgfpathlineto{\pgfqpoint{0.954508in}{2.295872in}}%
\pgfpathlineto{\pgfqpoint{0.947449in}{2.289869in}}%
\pgfpathlineto{\pgfqpoint{0.942445in}{2.284942in}}%
\pgfpathlineto{\pgfqpoint{0.936314in}{2.278117in}}%
\pgfpathlineto{\pgfqpoint{0.930382in}{2.270255in}}%
\pgfpathlineto{\pgfqpoint{0.927676in}{2.266366in}}%
\pgfpathlineto{\pgfqpoint{0.920897in}{2.254615in}}%
\pgfpathlineto{\pgfqpoint{0.918320in}{2.249093in}}%
\pgfpathlineto{\pgfqpoint{0.915548in}{2.242863in}}%
\pgfpathlineto{\pgfqpoint{0.911407in}{2.231112in}}%
\pgfpathlineto{\pgfqpoint{0.908306in}{2.219360in}}%
\pgfpathlineto{\pgfqpoint{0.906257in}{2.208635in}}%
\pgfpathlineto{\pgfqpoint{0.906066in}{2.207609in}}%
\pgfpathlineto{\pgfqpoint{0.904516in}{2.195858in}}%
\pgfpathlineto{\pgfqpoint{0.903610in}{2.184106in}}%
\pgfpathlineto{\pgfqpoint{0.903251in}{2.172355in}}%
\pgfpathlineto{\pgfqpoint{0.903360in}{2.160604in}}%
\pgfpathlineto{\pgfqpoint{0.903873in}{2.148852in}}%
\pgfpathlineto{\pgfqpoint{0.904736in}{2.137101in}}%
\pgfpathlineto{\pgfqpoint{0.905905in}{2.125349in}}%
\pgfpathlineto{\pgfqpoint{0.906257in}{2.122564in}}%
\pgfpathlineto{\pgfqpoint{0.907330in}{2.113598in}}%
\pgfpathlineto{\pgfqpoint{0.908990in}{2.101847in}}%
\pgfpathlineto{\pgfqpoint{0.910864in}{2.090095in}}%
\pgfpathlineto{\pgfqpoint{0.912929in}{2.078344in}}%
\pgfpathlineto{\pgfqpoint{0.915168in}{2.066593in}}%
\pgfpathlineto{\pgfqpoint{0.917562in}{2.054841in}}%
\pgfpathlineto{\pgfqpoint{0.918320in}{2.051410in}}%
\pgfpathlineto{\pgfqpoint{0.920111in}{2.043090in}}%
\pgfpathlineto{\pgfqpoint{0.922799in}{2.031338in}}%
\pgfpathlineto{\pgfqpoint{0.925607in}{2.019587in}}%
\pgfpathlineto{\pgfqpoint{0.928527in}{2.007836in}}%
\pgfpathlineto{\pgfqpoint{0.930382in}{2.000700in}}%
\pgfpathlineto{\pgfqpoint{0.931576in}{1.996084in}}%
\pgfpathlineto{\pgfqpoint{0.934763in}{1.984333in}}%
\pgfpathlineto{\pgfqpoint{0.938040in}{1.972582in}}%
\pgfpathlineto{\pgfqpoint{0.941401in}{1.960830in}}%
\pgfpathlineto{\pgfqpoint{0.942445in}{1.957324in}}%
\pgfpathlineto{\pgfqpoint{0.944926in}{1.949079in}}%
\pgfpathlineto{\pgfqpoint{0.948563in}{1.937327in}}%
\pgfpathlineto{\pgfqpoint{0.952268in}{1.925576in}}%
\pgfpathlineto{\pgfqpoint{0.954508in}{1.918661in}}%
\pgfpathlineto{\pgfqpoint{0.956114in}{1.913825in}}%
\pgfpathlineto{\pgfqpoint{0.960131in}{1.902073in}}%
\pgfpathlineto{\pgfqpoint{0.964205in}{1.890322in}}%
\pgfpathlineto{\pgfqpoint{0.966570in}{1.883648in}}%
\pgfpathlineto{\pgfqpoint{0.968443in}{1.878571in}}%
\pgfpathlineto{\pgfqpoint{0.972879in}{1.866819in}}%
\pgfpathlineto{\pgfqpoint{0.977360in}{1.855068in}}%
\pgfpathlineto{\pgfqpoint{0.978633in}{1.851809in}}%
\pgfpathlineto{\pgfqpoint{0.982135in}{1.843316in}}%
\pgfpathlineto{\pgfqpoint{0.987044in}{1.831565in}}%
\pgfpathclose%
\pgfusepath{fill}%
\end{pgfscope}%
\begin{pgfscope}%
\pgfpathrectangle{\pgfqpoint{0.423750in}{1.819814in}}{\pgfqpoint{1.194205in}{1.163386in}}%
\pgfusepath{clip}%
\pgfsetbuttcap%
\pgfsetroundjoin%
\definecolor{currentfill}{rgb}{0.976961,0.885681,0.814303}%
\pgfsetfillcolor{currentfill}%
\pgfsetlinewidth{0.000000pt}%
\definecolor{currentstroke}{rgb}{0.000000,0.000000,0.000000}%
\pgfsetstrokecolor{currentstroke}%
\pgfsetdash{}{0pt}%
\pgfpathmoveto{\pgfqpoint{1.424952in}{1.819814in}}%
\pgfpathlineto{\pgfqpoint{1.437014in}{1.819814in}}%
\pgfpathlineto{\pgfqpoint{1.449077in}{1.819814in}}%
\pgfpathlineto{\pgfqpoint{1.461140in}{1.819814in}}%
\pgfpathlineto{\pgfqpoint{1.473202in}{1.819814in}}%
\pgfpathlineto{\pgfqpoint{1.485265in}{1.819814in}}%
\pgfpathlineto{\pgfqpoint{1.497328in}{1.819814in}}%
\pgfpathlineto{\pgfqpoint{1.509390in}{1.819814in}}%
\pgfpathlineto{\pgfqpoint{1.521453in}{1.819814in}}%
\pgfpathlineto{\pgfqpoint{1.533516in}{1.819814in}}%
\pgfpathlineto{\pgfqpoint{1.545579in}{1.819814in}}%
\pgfpathlineto{\pgfqpoint{1.557641in}{1.819814in}}%
\pgfpathlineto{\pgfqpoint{1.569704in}{1.819814in}}%
\pgfpathlineto{\pgfqpoint{1.581767in}{1.819814in}}%
\pgfpathlineto{\pgfqpoint{1.593829in}{1.819814in}}%
\pgfpathlineto{\pgfqpoint{1.605892in}{1.819814in}}%
\pgfpathlineto{\pgfqpoint{1.617955in}{1.819814in}}%
\pgfpathlineto{\pgfqpoint{1.617955in}{1.831565in}}%
\pgfpathlineto{\pgfqpoint{1.617955in}{1.843316in}}%
\pgfpathlineto{\pgfqpoint{1.617955in}{1.855068in}}%
\pgfpathlineto{\pgfqpoint{1.617955in}{1.866819in}}%
\pgfpathlineto{\pgfqpoint{1.617955in}{1.878571in}}%
\pgfpathlineto{\pgfqpoint{1.617955in}{1.890322in}}%
\pgfpathlineto{\pgfqpoint{1.617955in}{1.902073in}}%
\pgfpathlineto{\pgfqpoint{1.617955in}{1.913825in}}%
\pgfpathlineto{\pgfqpoint{1.617955in}{1.925576in}}%
\pgfpathlineto{\pgfqpoint{1.617955in}{1.937327in}}%
\pgfpathlineto{\pgfqpoint{1.617955in}{1.949079in}}%
\pgfpathlineto{\pgfqpoint{1.617955in}{1.960830in}}%
\pgfpathlineto{\pgfqpoint{1.617955in}{1.972582in}}%
\pgfpathlineto{\pgfqpoint{1.617955in}{1.984333in}}%
\pgfpathlineto{\pgfqpoint{1.617955in}{1.996084in}}%
\pgfpathlineto{\pgfqpoint{1.617955in}{2.007836in}}%
\pgfpathlineto{\pgfqpoint{1.617955in}{2.019587in}}%
\pgfpathlineto{\pgfqpoint{1.617955in}{2.031338in}}%
\pgfpathlineto{\pgfqpoint{1.617955in}{2.043090in}}%
\pgfpathlineto{\pgfqpoint{1.617955in}{2.054841in}}%
\pgfpathlineto{\pgfqpoint{1.617955in}{2.066593in}}%
\pgfpathlineto{\pgfqpoint{1.617955in}{2.078344in}}%
\pgfpathlineto{\pgfqpoint{1.617955in}{2.090095in}}%
\pgfpathlineto{\pgfqpoint{1.617955in}{2.101847in}}%
\pgfpathlineto{\pgfqpoint{1.617955in}{2.113598in}}%
\pgfpathlineto{\pgfqpoint{1.617955in}{2.125349in}}%
\pgfpathlineto{\pgfqpoint{1.617955in}{2.137101in}}%
\pgfpathlineto{\pgfqpoint{1.617955in}{2.148852in}}%
\pgfpathlineto{\pgfqpoint{1.617955in}{2.160604in}}%
\pgfpathlineto{\pgfqpoint{1.617955in}{2.172355in}}%
\pgfpathlineto{\pgfqpoint{1.617955in}{2.184106in}}%
\pgfpathlineto{\pgfqpoint{1.617955in}{2.195858in}}%
\pgfpathlineto{\pgfqpoint{1.617955in}{2.207609in}}%
\pgfpathlineto{\pgfqpoint{1.617955in}{2.219360in}}%
\pgfpathlineto{\pgfqpoint{1.617955in}{2.231112in}}%
\pgfpathlineto{\pgfqpoint{1.617955in}{2.242863in}}%
\pgfpathlineto{\pgfqpoint{1.617955in}{2.254615in}}%
\pgfpathlineto{\pgfqpoint{1.617955in}{2.266366in}}%
\pgfpathlineto{\pgfqpoint{1.617955in}{2.278117in}}%
\pgfpathlineto{\pgfqpoint{1.617955in}{2.289869in}}%
\pgfpathlineto{\pgfqpoint{1.617955in}{2.301620in}}%
\pgfpathlineto{\pgfqpoint{1.617955in}{2.313371in}}%
\pgfpathlineto{\pgfqpoint{1.617955in}{2.325123in}}%
\pgfpathlineto{\pgfqpoint{1.617955in}{2.332919in}}%
\pgfpathlineto{\pgfqpoint{1.605892in}{2.328867in}}%
\pgfpathlineto{\pgfqpoint{1.598116in}{2.325123in}}%
\pgfpathlineto{\pgfqpoint{1.593829in}{2.323053in}}%
\pgfpathlineto{\pgfqpoint{1.581767in}{2.315444in}}%
\pgfpathlineto{\pgfqpoint{1.579097in}{2.313371in}}%
\pgfpathlineto{\pgfqpoint{1.569704in}{2.306038in}}%
\pgfpathlineto{\pgfqpoint{1.564910in}{2.301620in}}%
\pgfpathlineto{\pgfqpoint{1.557641in}{2.294872in}}%
\pgfpathlineto{\pgfqpoint{1.552949in}{2.289869in}}%
\pgfpathlineto{\pgfqpoint{1.545579in}{2.281938in}}%
\pgfpathlineto{\pgfqpoint{1.542414in}{2.278117in}}%
\pgfpathlineto{\pgfqpoint{1.533516in}{2.267258in}}%
\pgfpathlineto{\pgfqpoint{1.532851in}{2.266366in}}%
\pgfpathlineto{\pgfqpoint{1.524257in}{2.254615in}}%
\pgfpathlineto{\pgfqpoint{1.521453in}{2.250724in}}%
\pgfpathlineto{\pgfqpoint{1.516215in}{2.242863in}}%
\pgfpathlineto{\pgfqpoint{1.509390in}{2.232470in}}%
\pgfpathlineto{\pgfqpoint{1.508551in}{2.231112in}}%
\pgfpathlineto{\pgfqpoint{1.501475in}{2.219360in}}%
\pgfpathlineto{\pgfqpoint{1.497328in}{2.212338in}}%
\pgfpathlineto{\pgfqpoint{1.494658in}{2.207609in}}%
\pgfpathlineto{\pgfqpoint{1.488196in}{2.195858in}}%
\pgfpathlineto{\pgfqpoint{1.485265in}{2.190383in}}%
\pgfpathlineto{\pgfqpoint{1.482003in}{2.184106in}}%
\pgfpathlineto{\pgfqpoint{1.476074in}{2.172355in}}%
\pgfpathlineto{\pgfqpoint{1.473202in}{2.166482in}}%
\pgfpathlineto{\pgfqpoint{1.470369in}{2.160604in}}%
\pgfpathlineto{\pgfqpoint{1.464902in}{2.148852in}}%
\pgfpathlineto{\pgfqpoint{1.461140in}{2.140476in}}%
\pgfpathlineto{\pgfqpoint{1.459620in}{2.137101in}}%
\pgfpathlineto{\pgfqpoint{1.454559in}{2.125349in}}%
\pgfpathlineto{\pgfqpoint{1.449689in}{2.113598in}}%
\pgfpathlineto{\pgfqpoint{1.449077in}{2.112042in}}%
\pgfpathlineto{\pgfqpoint{1.444992in}{2.101847in}}%
\pgfpathlineto{\pgfqpoint{1.440507in}{2.090095in}}%
\pgfpathlineto{\pgfqpoint{1.437014in}{2.080444in}}%
\pgfpathlineto{\pgfqpoint{1.436226in}{2.078344in}}%
\pgfpathlineto{\pgfqpoint{1.432095in}{2.066593in}}%
\pgfpathlineto{\pgfqpoint{1.428216in}{2.054841in}}%
\pgfpathlineto{\pgfqpoint{1.424952in}{2.044228in}}%
\pgfpathlineto{\pgfqpoint{1.424581in}{2.043090in}}%
\pgfpathlineto{\pgfqpoint{1.421076in}{2.031338in}}%
\pgfpathlineto{\pgfqpoint{1.417868in}{2.019587in}}%
\pgfpathlineto{\pgfqpoint{1.414965in}{2.007836in}}%
\pgfpathlineto{\pgfqpoint{1.412889in}{1.998388in}}%
\pgfpathlineto{\pgfqpoint{1.412341in}{1.996084in}}%
\pgfpathlineto{\pgfqpoint{1.409918in}{1.984333in}}%
\pgfpathlineto{\pgfqpoint{1.407859in}{1.972582in}}%
\pgfpathlineto{\pgfqpoint{1.406175in}{1.960830in}}%
\pgfpathlineto{\pgfqpoint{1.404879in}{1.949079in}}%
\pgfpathlineto{\pgfqpoint{1.403982in}{1.937327in}}%
\pgfpathlineto{\pgfqpoint{1.403499in}{1.925576in}}%
\pgfpathlineto{\pgfqpoint{1.403442in}{1.913825in}}%
\pgfpathlineto{\pgfqpoint{1.403827in}{1.902073in}}%
\pgfpathlineto{\pgfqpoint{1.404668in}{1.890322in}}%
\pgfpathlineto{\pgfqpoint{1.405982in}{1.878571in}}%
\pgfpathlineto{\pgfqpoint{1.407784in}{1.866819in}}%
\pgfpathlineto{\pgfqpoint{1.410093in}{1.855068in}}%
\pgfpathlineto{\pgfqpoint{1.412889in}{1.843465in}}%
\pgfpathlineto{\pgfqpoint{1.412923in}{1.843316in}}%
\pgfpathlineto{\pgfqpoint{1.416100in}{1.831565in}}%
\pgfpathlineto{\pgfqpoint{1.419812in}{1.819814in}}%
\pgfpathclose%
\pgfusepath{fill}%
\end{pgfscope}%
\begin{pgfscope}%
\pgfpathrectangle{\pgfqpoint{0.423750in}{1.819814in}}{\pgfqpoint{1.194205in}{1.163386in}}%
\pgfusepath{clip}%
\pgfsetbuttcap%
\pgfsetroundjoin%
\definecolor{currentfill}{rgb}{0.976961,0.885681,0.814303}%
\pgfsetfillcolor{currentfill}%
\pgfsetlinewidth{0.000000pt}%
\definecolor{currentstroke}{rgb}{0.000000,0.000000,0.000000}%
\pgfsetstrokecolor{currentstroke}%
\pgfsetdash{}{0pt}%
\pgfpathmoveto{\pgfqpoint{0.677066in}{2.332514in}}%
\pgfpathlineto{\pgfqpoint{0.689129in}{2.330125in}}%
\pgfpathlineto{\pgfqpoint{0.701191in}{2.331399in}}%
\pgfpathlineto{\pgfqpoint{0.712642in}{2.336874in}}%
\pgfpathlineto{\pgfqpoint{0.713254in}{2.337246in}}%
\pgfpathlineto{\pgfqpoint{0.723948in}{2.348626in}}%
\pgfpathlineto{\pgfqpoint{0.725317in}{2.350610in}}%
\pgfpathlineto{\pgfqpoint{0.730322in}{2.360377in}}%
\pgfpathlineto{\pgfqpoint{0.734528in}{2.372128in}}%
\pgfpathlineto{\pgfqpoint{0.737335in}{2.383880in}}%
\pgfpathlineto{\pgfqpoint{0.737379in}{2.384151in}}%
\pgfpathlineto{\pgfqpoint{0.739012in}{2.395631in}}%
\pgfpathlineto{\pgfqpoint{0.739955in}{2.407383in}}%
\pgfpathlineto{\pgfqpoint{0.740318in}{2.419134in}}%
\pgfpathlineto{\pgfqpoint{0.740213in}{2.430885in}}%
\pgfpathlineto{\pgfqpoint{0.739725in}{2.442637in}}%
\pgfpathlineto{\pgfqpoint{0.738919in}{2.454388in}}%
\pgfpathlineto{\pgfqpoint{0.737847in}{2.466139in}}%
\pgfpathlineto{\pgfqpoint{0.737379in}{2.470239in}}%
\pgfpathlineto{\pgfqpoint{0.736528in}{2.477891in}}%
\pgfpathlineto{\pgfqpoint{0.735003in}{2.489642in}}%
\pgfpathlineto{\pgfqpoint{0.733316in}{2.501394in}}%
\pgfpathlineto{\pgfqpoint{0.731489in}{2.513145in}}%
\pgfpathlineto{\pgfqpoint{0.729541in}{2.524896in}}%
\pgfpathlineto{\pgfqpoint{0.727486in}{2.536648in}}%
\pgfpathlineto{\pgfqpoint{0.725338in}{2.548399in}}%
\pgfpathlineto{\pgfqpoint{0.725317in}{2.548507in}}%
\pgfpathlineto{\pgfqpoint{0.723033in}{2.560150in}}%
\pgfpathlineto{\pgfqpoint{0.720662in}{2.571902in}}%
\pgfpathlineto{\pgfqpoint{0.718233in}{2.583653in}}%
\pgfpathlineto{\pgfqpoint{0.715752in}{2.595405in}}%
\pgfpathlineto{\pgfqpoint{0.713254in}{2.607015in}}%
\pgfpathlineto{\pgfqpoint{0.713224in}{2.607156in}}%
\pgfpathlineto{\pgfqpoint{0.710545in}{2.618907in}}%
\pgfpathlineto{\pgfqpoint{0.707839in}{2.630659in}}%
\pgfpathlineto{\pgfqpoint{0.705108in}{2.642410in}}%
\pgfpathlineto{\pgfqpoint{0.702355in}{2.654161in}}%
\pgfpathlineto{\pgfqpoint{0.701191in}{2.659001in}}%
\pgfpathlineto{\pgfqpoint{0.699497in}{2.665913in}}%
\pgfpathlineto{\pgfqpoint{0.696569in}{2.677664in}}%
\pgfpathlineto{\pgfqpoint{0.693635in}{2.689416in}}%
\pgfpathlineto{\pgfqpoint{0.690696in}{2.701167in}}%
\pgfpathlineto{\pgfqpoint{0.689129in}{2.707336in}}%
\pgfpathlineto{\pgfqpoint{0.687668in}{2.712918in}}%
\pgfpathlineto{\pgfqpoint{0.684549in}{2.724670in}}%
\pgfpathlineto{\pgfqpoint{0.681440in}{2.736421in}}%
\pgfpathlineto{\pgfqpoint{0.678340in}{2.748172in}}%
\pgfpathlineto{\pgfqpoint{0.677066in}{2.752934in}}%
\pgfpathlineto{\pgfqpoint{0.675120in}{2.759924in}}%
\pgfpathlineto{\pgfqpoint{0.671832in}{2.771675in}}%
\pgfpathlineto{\pgfqpoint{0.668566in}{2.783427in}}%
\pgfpathlineto{\pgfqpoint{0.665321in}{2.795178in}}%
\pgfpathlineto{\pgfqpoint{0.665003in}{2.796306in}}%
\pgfpathlineto{\pgfqpoint{0.661866in}{2.806929in}}%
\pgfpathlineto{\pgfqpoint{0.658422in}{2.818681in}}%
\pgfpathlineto{\pgfqpoint{0.655013in}{2.830432in}}%
\pgfpathlineto{\pgfqpoint{0.652941in}{2.837561in}}%
\pgfpathlineto{\pgfqpoint{0.651517in}{2.842183in}}%
\pgfpathlineto{\pgfqpoint{0.647882in}{2.853935in}}%
\pgfpathlineto{\pgfqpoint{0.644296in}{2.865686in}}%
\pgfpathlineto{\pgfqpoint{0.640878in}{2.877018in}}%
\pgfpathlineto{\pgfqpoint{0.640743in}{2.877438in}}%
\pgfpathlineto{\pgfqpoint{0.636902in}{2.889189in}}%
\pgfpathlineto{\pgfqpoint{0.633127in}{2.900940in}}%
\pgfpathlineto{\pgfqpoint{0.629412in}{2.912692in}}%
\pgfpathlineto{\pgfqpoint{0.628815in}{2.914562in}}%
\pgfpathlineto{\pgfqpoint{0.625420in}{2.924443in}}%
\pgfpathlineto{\pgfqpoint{0.621445in}{2.936194in}}%
\pgfpathlineto{\pgfqpoint{0.617547in}{2.947946in}}%
\pgfpathlineto{\pgfqpoint{0.616753in}{2.950328in}}%
\pgfpathlineto{\pgfqpoint{0.613357in}{2.959697in}}%
\pgfpathlineto{\pgfqpoint{0.609175in}{2.971449in}}%
\pgfpathlineto{\pgfqpoint{0.605088in}{2.983200in}}%
\pgfpathlineto{\pgfqpoint{0.604690in}{2.983200in}}%
\pgfpathlineto{\pgfqpoint{0.592627in}{2.983200in}}%
\pgfpathlineto{\pgfqpoint{0.580565in}{2.983200in}}%
\pgfpathlineto{\pgfqpoint{0.568502in}{2.983200in}}%
\pgfpathlineto{\pgfqpoint{0.556439in}{2.983200in}}%
\pgfpathlineto{\pgfqpoint{0.544377in}{2.983200in}}%
\pgfpathlineto{\pgfqpoint{0.532314in}{2.983200in}}%
\pgfpathlineto{\pgfqpoint{0.520251in}{2.983200in}}%
\pgfpathlineto{\pgfqpoint{0.508189in}{2.983200in}}%
\pgfpathlineto{\pgfqpoint{0.496126in}{2.983200in}}%
\pgfpathlineto{\pgfqpoint{0.484063in}{2.983200in}}%
\pgfpathlineto{\pgfqpoint{0.472001in}{2.983200in}}%
\pgfpathlineto{\pgfqpoint{0.459938in}{2.983200in}}%
\pgfpathlineto{\pgfqpoint{0.447875in}{2.983200in}}%
\pgfpathlineto{\pgfqpoint{0.435813in}{2.983200in}}%
\pgfpathlineto{\pgfqpoint{0.433414in}{2.983200in}}%
\pgfpathlineto{\pgfqpoint{0.435560in}{2.971449in}}%
\pgfpathlineto{\pgfqpoint{0.435813in}{2.970111in}}%
\pgfpathlineto{\pgfqpoint{0.437932in}{2.959697in}}%
\pgfpathlineto{\pgfqpoint{0.440313in}{2.947946in}}%
\pgfpathlineto{\pgfqpoint{0.442675in}{2.936194in}}%
\pgfpathlineto{\pgfqpoint{0.445019in}{2.924443in}}%
\pgfpathlineto{\pgfqpoint{0.447349in}{2.912692in}}%
\pgfpathlineto{\pgfqpoint{0.447875in}{2.910093in}}%
\pgfpathlineto{\pgfqpoint{0.449879in}{2.900940in}}%
\pgfpathlineto{\pgfqpoint{0.452442in}{2.889189in}}%
\pgfpathlineto{\pgfqpoint{0.454976in}{2.877438in}}%
\pgfpathlineto{\pgfqpoint{0.457484in}{2.865686in}}%
\pgfpathlineto{\pgfqpoint{0.459938in}{2.854088in}}%
\pgfpathlineto{\pgfqpoint{0.459973in}{2.853935in}}%
\pgfpathlineto{\pgfqpoint{0.462733in}{2.842183in}}%
\pgfpathlineto{\pgfqpoint{0.465454in}{2.830432in}}%
\pgfpathlineto{\pgfqpoint{0.468139in}{2.818681in}}%
\pgfpathlineto{\pgfqpoint{0.470793in}{2.806929in}}%
\pgfpathlineto{\pgfqpoint{0.472001in}{2.801615in}}%
\pgfpathlineto{\pgfqpoint{0.473595in}{2.795178in}}%
\pgfpathlineto{\pgfqpoint{0.476503in}{2.783427in}}%
\pgfpathlineto{\pgfqpoint{0.479367in}{2.771675in}}%
\pgfpathlineto{\pgfqpoint{0.482190in}{2.759924in}}%
\pgfpathlineto{\pgfqpoint{0.484063in}{2.752108in}}%
\pgfpathlineto{\pgfqpoint{0.485097in}{2.748172in}}%
\pgfpathlineto{\pgfqpoint{0.488196in}{2.736421in}}%
\pgfpathlineto{\pgfqpoint{0.491241in}{2.724670in}}%
\pgfpathlineto{\pgfqpoint{0.494238in}{2.712918in}}%
\pgfpathlineto{\pgfqpoint{0.496126in}{2.705487in}}%
\pgfpathlineto{\pgfqpoint{0.497336in}{2.701167in}}%
\pgfpathlineto{\pgfqpoint{0.500629in}{2.689416in}}%
\pgfpathlineto{\pgfqpoint{0.503860in}{2.677664in}}%
\pgfpathlineto{\pgfqpoint{0.507033in}{2.665913in}}%
\pgfpathlineto{\pgfqpoint{0.508189in}{2.661644in}}%
\pgfpathlineto{\pgfqpoint{0.510434in}{2.654161in}}%
\pgfpathlineto{\pgfqpoint{0.513925in}{2.642410in}}%
\pgfpathlineto{\pgfqpoint{0.517343in}{2.630659in}}%
\pgfpathlineto{\pgfqpoint{0.520251in}{2.620506in}}%
\pgfpathlineto{\pgfqpoint{0.520763in}{2.618907in}}%
\pgfpathlineto{\pgfqpoint{0.524543in}{2.607156in}}%
\pgfpathlineto{\pgfqpoint{0.528231in}{2.595405in}}%
\pgfpathlineto{\pgfqpoint{0.531837in}{2.583653in}}%
\pgfpathlineto{\pgfqpoint{0.532314in}{2.582105in}}%
\pgfpathlineto{\pgfqpoint{0.535854in}{2.571902in}}%
\pgfpathlineto{\pgfqpoint{0.539838in}{2.560150in}}%
\pgfpathlineto{\pgfqpoint{0.543719in}{2.548399in}}%
\pgfpathlineto{\pgfqpoint{0.544377in}{2.546412in}}%
\pgfpathlineto{\pgfqpoint{0.548043in}{2.536648in}}%
\pgfpathlineto{\pgfqpoint{0.552344in}{2.524896in}}%
\pgfpathlineto{\pgfqpoint{0.556439in}{2.513380in}}%
\pgfpathlineto{\pgfqpoint{0.556536in}{2.513145in}}%
\pgfpathlineto{\pgfqpoint{0.561341in}{2.501394in}}%
\pgfpathlineto{\pgfqpoint{0.565977in}{2.489642in}}%
\pgfpathlineto{\pgfqpoint{0.568502in}{2.483128in}}%
\pgfpathlineto{\pgfqpoint{0.570857in}{2.477891in}}%
\pgfpathlineto{\pgfqpoint{0.576036in}{2.466139in}}%
\pgfpathlineto{\pgfqpoint{0.580565in}{2.455481in}}%
\pgfpathlineto{\pgfqpoint{0.581111in}{2.454388in}}%
\pgfpathlineto{\pgfqpoint{0.586930in}{2.442637in}}%
\pgfpathlineto{\pgfqpoint{0.592473in}{2.430885in}}%
\pgfpathlineto{\pgfqpoint{0.592627in}{2.430555in}}%
\pgfpathlineto{\pgfqpoint{0.598993in}{2.419134in}}%
\pgfpathlineto{\pgfqpoint{0.604690in}{2.408352in}}%
\pgfpathlineto{\pgfqpoint{0.605313in}{2.407383in}}%
\pgfpathlineto{\pgfqpoint{0.612702in}{2.395631in}}%
\pgfpathlineto{\pgfqpoint{0.616753in}{2.388876in}}%
\pgfpathlineto{\pgfqpoint{0.620459in}{2.383880in}}%
\pgfpathlineto{\pgfqpoint{0.628729in}{2.372128in}}%
\pgfpathlineto{\pgfqpoint{0.628815in}{2.372003in}}%
\pgfpathlineto{\pgfqpoint{0.638880in}{2.360377in}}%
\pgfpathlineto{\pgfqpoint{0.640878in}{2.357946in}}%
\pgfpathlineto{\pgfqpoint{0.650876in}{2.348626in}}%
\pgfpathlineto{\pgfqpoint{0.652941in}{2.346595in}}%
\pgfpathlineto{\pgfqpoint{0.665003in}{2.338051in}}%
\pgfpathlineto{\pgfqpoint{0.667687in}{2.336874in}}%
\pgfpathclose%
\pgfusepath{fill}%
\end{pgfscope}%
\begin{pgfscope}%
\pgfsetrectcap%
\pgfsetmiterjoin%
\pgfsetlinewidth{0.000000pt}%
\definecolor{currentstroke}{rgb}{1.000000,1.000000,1.000000}%
\pgfsetstrokecolor{currentstroke}%
\pgfsetdash{}{0pt}%
\pgfpathmoveto{\pgfqpoint{0.423750in}{1.819814in}}%
\pgfpathlineto{\pgfqpoint{0.423750in}{2.983200in}}%
\pgfusepath{}%
\end{pgfscope}%
\begin{pgfscope}%
\pgfsetrectcap%
\pgfsetmiterjoin%
\pgfsetlinewidth{0.000000pt}%
\definecolor{currentstroke}{rgb}{1.000000,1.000000,1.000000}%
\pgfsetstrokecolor{currentstroke}%
\pgfsetdash{}{0pt}%
\pgfpathmoveto{\pgfqpoint{1.617955in}{1.819814in}}%
\pgfpathlineto{\pgfqpoint{1.617955in}{2.983200in}}%
\pgfusepath{}%
\end{pgfscope}%
\begin{pgfscope}%
\pgfsetrectcap%
\pgfsetmiterjoin%
\pgfsetlinewidth{0.000000pt}%
\definecolor{currentstroke}{rgb}{1.000000,1.000000,1.000000}%
\pgfsetstrokecolor{currentstroke}%
\pgfsetdash{}{0pt}%
\pgfpathmoveto{\pgfqpoint{0.423750in}{1.819814in}}%
\pgfpathlineto{\pgfqpoint{1.617955in}{1.819814in}}%
\pgfusepath{}%
\end{pgfscope}%
\begin{pgfscope}%
\pgfsetrectcap%
\pgfsetmiterjoin%
\pgfsetlinewidth{0.000000pt}%
\definecolor{currentstroke}{rgb}{1.000000,1.000000,1.000000}%
\pgfsetstrokecolor{currentstroke}%
\pgfsetdash{}{0pt}%
\pgfpathmoveto{\pgfqpoint{0.423750in}{2.983200in}}%
\pgfpathlineto{\pgfqpoint{1.617955in}{2.983200in}}%
\pgfusepath{}%
\end{pgfscope}%
\begin{pgfscope}%
\pgfsetbuttcap%
\pgfsetmiterjoin%
\definecolor{currentfill}{rgb}{0.917647,0.917647,0.949020}%
\pgfsetfillcolor{currentfill}%
\pgfsetlinewidth{0.000000pt}%
\definecolor{currentstroke}{rgb}{0.000000,0.000000,0.000000}%
\pgfsetstrokecolor{currentstroke}%
\pgfsetstrokeopacity{0.000000}%
\pgfsetdash{}{0pt}%
\pgfpathmoveto{\pgfqpoint{1.856795in}{1.819814in}}%
\pgfpathlineto{\pgfqpoint{3.051000in}{1.819814in}}%
\pgfpathlineto{\pgfqpoint{3.051000in}{2.983200in}}%
\pgfpathlineto{\pgfqpoint{1.856795in}{2.983200in}}%
\pgfpathclose%
\pgfusepath{fill}%
\end{pgfscope}%
\begin{pgfscope}%
\pgfpathrectangle{\pgfqpoint{1.856795in}{1.819814in}}{\pgfqpoint{1.194205in}{1.163386in}}%
\pgfusepath{clip}%
\pgfsetroundcap%
\pgfsetroundjoin%
\pgfsetlinewidth{0.803000pt}%
\definecolor{currentstroke}{rgb}{1.000000,1.000000,1.000000}%
\pgfsetstrokecolor{currentstroke}%
\pgfsetdash{}{0pt}%
\pgfpathmoveto{\pgfqpoint{2.279376in}{1.819814in}}%
\pgfpathlineto{\pgfqpoint{2.279376in}{2.983200in}}%
\pgfusepath{stroke}%
\end{pgfscope}%
\begin{pgfscope}%
\definecolor{textcolor}{rgb}{0.150000,0.150000,0.150000}%
\pgfsetstrokecolor{textcolor}%
\pgfsetfillcolor{textcolor}%
\pgftext[x=2.279376in,y=1.771203in,,top]{\color{textcolor}\rmfamily\fontsize{8.000000}{9.600000}\selectfont \(\displaystyle 0\)}%
\end{pgfscope}%
\begin{pgfscope}%
\pgfpathrectangle{\pgfqpoint{1.856795in}{1.819814in}}{\pgfqpoint{1.194205in}{1.163386in}}%
\pgfusepath{clip}%
\pgfsetroundcap%
\pgfsetroundjoin%
\pgfsetlinewidth{0.803000pt}%
\definecolor{currentstroke}{rgb}{1.000000,1.000000,1.000000}%
\pgfsetstrokecolor{currentstroke}%
\pgfsetdash{}{0pt}%
\pgfpathmoveto{\pgfqpoint{2.977462in}{1.819814in}}%
\pgfpathlineto{\pgfqpoint{2.977462in}{2.983200in}}%
\pgfusepath{stroke}%
\end{pgfscope}%
\begin{pgfscope}%
\definecolor{textcolor}{rgb}{0.150000,0.150000,0.150000}%
\pgfsetstrokecolor{textcolor}%
\pgfsetfillcolor{textcolor}%
\pgftext[x=2.977462in,y=1.771203in,,top]{\color{textcolor}\rmfamily\fontsize{8.000000}{9.600000}\selectfont \(\displaystyle 10\)}%
\end{pgfscope}%
\begin{pgfscope}%
\pgfpathrectangle{\pgfqpoint{1.856795in}{1.819814in}}{\pgfqpoint{1.194205in}{1.163386in}}%
\pgfusepath{clip}%
\pgfsetroundcap%
\pgfsetroundjoin%
\pgfsetlinewidth{0.803000pt}%
\definecolor{currentstroke}{rgb}{1.000000,1.000000,1.000000}%
\pgfsetstrokecolor{currentstroke}%
\pgfsetdash{}{0pt}%
\pgfpathmoveto{\pgfqpoint{1.856795in}{1.891932in}}%
\pgfpathlineto{\pgfqpoint{3.051000in}{1.891932in}}%
\pgfusepath{stroke}%
\end{pgfscope}%
\begin{pgfscope}%
\definecolor{textcolor}{rgb}{0.150000,0.150000,0.150000}%
\pgfsetstrokecolor{textcolor}%
\pgfsetfillcolor{textcolor}%
\pgftext[x=1.749156in,y=1.849723in,left,base]{\color{textcolor}\rmfamily\fontsize{8.000000}{9.600000}\selectfont \(\displaystyle 0\)}%
\end{pgfscope}%
\begin{pgfscope}%
\pgfpathrectangle{\pgfqpoint{1.856795in}{1.819814in}}{\pgfqpoint{1.194205in}{1.163386in}}%
\pgfusepath{clip}%
\pgfsetroundcap%
\pgfsetroundjoin%
\pgfsetlinewidth{0.803000pt}%
\definecolor{currentstroke}{rgb}{1.000000,1.000000,1.000000}%
\pgfsetstrokecolor{currentstroke}%
\pgfsetdash{}{0pt}%
\pgfpathmoveto{\pgfqpoint{1.856795in}{2.231649in}}%
\pgfpathlineto{\pgfqpoint{3.051000in}{2.231649in}}%
\pgfusepath{stroke}%
\end{pgfscope}%
\begin{pgfscope}%
\definecolor{textcolor}{rgb}{0.150000,0.150000,0.150000}%
\pgfsetstrokecolor{textcolor}%
\pgfsetfillcolor{textcolor}%
\pgftext[x=1.749156in,y=2.189439in,left,base]{\color{textcolor}\rmfamily\fontsize{8.000000}{9.600000}\selectfont \(\displaystyle 5\)}%
\end{pgfscope}%
\begin{pgfscope}%
\pgfpathrectangle{\pgfqpoint{1.856795in}{1.819814in}}{\pgfqpoint{1.194205in}{1.163386in}}%
\pgfusepath{clip}%
\pgfsetroundcap%
\pgfsetroundjoin%
\pgfsetlinewidth{0.803000pt}%
\definecolor{currentstroke}{rgb}{1.000000,1.000000,1.000000}%
\pgfsetstrokecolor{currentstroke}%
\pgfsetdash{}{0pt}%
\pgfpathmoveto{\pgfqpoint{1.856795in}{2.571365in}}%
\pgfpathlineto{\pgfqpoint{3.051000in}{2.571365in}}%
\pgfusepath{stroke}%
\end{pgfscope}%
\begin{pgfscope}%
\definecolor{textcolor}{rgb}{0.150000,0.150000,0.150000}%
\pgfsetstrokecolor{textcolor}%
\pgfsetfillcolor{textcolor}%
\pgftext[x=1.690127in,y=2.529156in,left,base]{\color{textcolor}\rmfamily\fontsize{8.000000}{9.600000}\selectfont \(\displaystyle 10\)}%
\end{pgfscope}%
\begin{pgfscope}%
\pgfpathrectangle{\pgfqpoint{1.856795in}{1.819814in}}{\pgfqpoint{1.194205in}{1.163386in}}%
\pgfusepath{clip}%
\pgfsetroundcap%
\pgfsetroundjoin%
\pgfsetlinewidth{0.803000pt}%
\definecolor{currentstroke}{rgb}{1.000000,1.000000,1.000000}%
\pgfsetstrokecolor{currentstroke}%
\pgfsetdash{}{0pt}%
\pgfpathmoveto{\pgfqpoint{1.856795in}{2.911081in}}%
\pgfpathlineto{\pgfqpoint{3.051000in}{2.911081in}}%
\pgfusepath{stroke}%
\end{pgfscope}%
\begin{pgfscope}%
\definecolor{textcolor}{rgb}{0.150000,0.150000,0.150000}%
\pgfsetstrokecolor{textcolor}%
\pgfsetfillcolor{textcolor}%
\pgftext[x=1.690127in,y=2.868872in,left,base]{\color{textcolor}\rmfamily\fontsize{8.000000}{9.600000}\selectfont \(\displaystyle 15\)}%
\end{pgfscope}%
\begin{pgfscope}%
\pgfpathrectangle{\pgfqpoint{1.856795in}{1.819814in}}{\pgfqpoint{1.194205in}{1.163386in}}%
\pgfusepath{clip}%
\pgfsetbuttcap%
\pgfsetroundjoin%
\definecolor{currentfill}{rgb}{0.032852,0.030888,0.118630}%
\pgfsetfillcolor{currentfill}%
\pgfsetlinewidth{0.000000pt}%
\definecolor{currentstroke}{rgb}{0.000000,0.000000,0.000000}%
\pgfsetstrokecolor{currentstroke}%
\pgfsetdash{}{0pt}%
\pgfpathmoveto{\pgfqpoint{2.681305in}{2.891677in}}%
\pgfpathlineto{\pgfqpoint{2.691882in}{2.894918in}}%
\pgfpathlineto{\pgfqpoint{2.702459in}{2.898905in}}%
\pgfpathlineto{\pgfqpoint{2.706825in}{2.900787in}}%
\pgfpathlineto{\pgfqpoint{2.713036in}{2.903959in}}%
\pgfpathlineto{\pgfqpoint{2.723613in}{2.910003in}}%
\pgfpathlineto{\pgfqpoint{2.725322in}{2.911081in}}%
\pgfpathlineto{\pgfqpoint{2.723613in}{2.911081in}}%
\pgfpathlineto{\pgfqpoint{2.713036in}{2.911081in}}%
\pgfpathlineto{\pgfqpoint{2.702459in}{2.911081in}}%
\pgfpathlineto{\pgfqpoint{2.691882in}{2.911081in}}%
\pgfpathlineto{\pgfqpoint{2.681305in}{2.911081in}}%
\pgfpathlineto{\pgfqpoint{2.670727in}{2.911081in}}%
\pgfpathlineto{\pgfqpoint{2.670693in}{2.911081in}}%
\pgfpathlineto{\pgfqpoint{2.670727in}{2.910808in}}%
\pgfpathlineto{\pgfqpoint{2.672207in}{2.900787in}}%
\pgfpathclose%
\pgfusepath{fill}%
\end{pgfscope}%
\begin{pgfscope}%
\pgfpathrectangle{\pgfqpoint{1.856795in}{1.819814in}}{\pgfqpoint{1.194205in}{1.163386in}}%
\pgfusepath{clip}%
\pgfsetbuttcap%
\pgfsetroundjoin%
\definecolor{currentfill}{rgb}{0.095724,0.060501,0.162005}%
\pgfsetfillcolor{currentfill}%
\pgfsetlinewidth{0.000000pt}%
\definecolor{currentstroke}{rgb}{0.000000,0.000000,0.000000}%
\pgfsetstrokecolor{currentstroke}%
\pgfsetdash{}{0pt}%
\pgfpathmoveto{\pgfqpoint{1.930700in}{1.943405in}}%
\pgfpathlineto{\pgfqpoint{1.930455in}{1.953699in}}%
\pgfpathlineto{\pgfqpoint{1.930333in}{1.954256in}}%
\pgfpathlineto{\pgfqpoint{1.930333in}{1.953699in}}%
\pgfpathlineto{\pgfqpoint{1.930333in}{1.943405in}}%
\pgfpathlineto{\pgfqpoint{1.930333in}{1.941483in}}%
\pgfpathclose%
\pgfusepath{fill}%
\end{pgfscope}%
\begin{pgfscope}%
\pgfpathrectangle{\pgfqpoint{1.856795in}{1.819814in}}{\pgfqpoint{1.194205in}{1.163386in}}%
\pgfusepath{clip}%
\pgfsetbuttcap%
\pgfsetroundjoin%
\definecolor{currentfill}{rgb}{0.095724,0.060501,0.162005}%
\pgfsetfillcolor{currentfill}%
\pgfsetlinewidth{0.000000pt}%
\definecolor{currentstroke}{rgb}{0.000000,0.000000,0.000000}%
\pgfsetstrokecolor{currentstroke}%
\pgfsetdash{}{0pt}%
\pgfpathmoveto{\pgfqpoint{2.660150in}{2.875392in}}%
\pgfpathlineto{\pgfqpoint{2.670727in}{2.871584in}}%
\pgfpathlineto{\pgfqpoint{2.681305in}{2.873364in}}%
\pgfpathlineto{\pgfqpoint{2.691882in}{2.875819in}}%
\pgfpathlineto{\pgfqpoint{2.702459in}{2.878799in}}%
\pgfpathlineto{\pgfqpoint{2.706750in}{2.880198in}}%
\pgfpathlineto{\pgfqpoint{2.713036in}{2.882417in}}%
\pgfpathlineto{\pgfqpoint{2.723613in}{2.886689in}}%
\pgfpathlineto{\pgfqpoint{2.731935in}{2.890492in}}%
\pgfpathlineto{\pgfqpoint{2.734190in}{2.891649in}}%
\pgfpathlineto{\pgfqpoint{2.744767in}{2.897628in}}%
\pgfpathlineto{\pgfqpoint{2.749851in}{2.900787in}}%
\pgfpathlineto{\pgfqpoint{2.755344in}{2.904827in}}%
\pgfpathlineto{\pgfqpoint{2.763168in}{2.911081in}}%
\pgfpathlineto{\pgfqpoint{2.755344in}{2.911081in}}%
\pgfpathlineto{\pgfqpoint{2.744767in}{2.911081in}}%
\pgfpathlineto{\pgfqpoint{2.734190in}{2.911081in}}%
\pgfpathlineto{\pgfqpoint{2.725322in}{2.911081in}}%
\pgfpathlineto{\pgfqpoint{2.723613in}{2.910003in}}%
\pgfpathlineto{\pgfqpoint{2.713036in}{2.903959in}}%
\pgfpathlineto{\pgfqpoint{2.706825in}{2.900787in}}%
\pgfpathlineto{\pgfqpoint{2.702459in}{2.898905in}}%
\pgfpathlineto{\pgfqpoint{2.691882in}{2.894918in}}%
\pgfpathlineto{\pgfqpoint{2.681305in}{2.891677in}}%
\pgfpathlineto{\pgfqpoint{2.672207in}{2.900787in}}%
\pgfpathlineto{\pgfqpoint{2.670727in}{2.910808in}}%
\pgfpathlineto{\pgfqpoint{2.670693in}{2.911081in}}%
\pgfpathlineto{\pgfqpoint{2.660150in}{2.911081in}}%
\pgfpathlineto{\pgfqpoint{2.651274in}{2.911081in}}%
\pgfpathlineto{\pgfqpoint{2.652382in}{2.900787in}}%
\pgfpathlineto{\pgfqpoint{2.654490in}{2.890492in}}%
\pgfpathlineto{\pgfqpoint{2.657702in}{2.880198in}}%
\pgfpathclose%
\pgfusepath{fill}%
\end{pgfscope}%
\begin{pgfscope}%
\pgfpathrectangle{\pgfqpoint{1.856795in}{1.819814in}}{\pgfqpoint{1.194205in}{1.163386in}}%
\pgfusepath{clip}%
\pgfsetbuttcap%
\pgfsetroundjoin%
\definecolor{currentfill}{rgb}{0.157709,0.083393,0.206797}%
\pgfsetfillcolor{currentfill}%
\pgfsetlinewidth{0.000000pt}%
\definecolor{currentstroke}{rgb}{0.000000,0.000000,0.000000}%
\pgfsetstrokecolor{currentstroke}%
\pgfsetdash{}{0pt}%
\pgfpathmoveto{\pgfqpoint{1.938590in}{1.902227in}}%
\pgfpathlineto{\pgfqpoint{1.940910in}{1.904753in}}%
\pgfpathlineto{\pgfqpoint{1.946744in}{1.912521in}}%
\pgfpathlineto{\pgfqpoint{1.951487in}{1.921505in}}%
\pgfpathlineto{\pgfqpoint{1.952080in}{1.922816in}}%
\pgfpathlineto{\pgfqpoint{1.954923in}{1.933110in}}%
\pgfpathlineto{\pgfqpoint{1.955967in}{1.943405in}}%
\pgfpathlineto{\pgfqpoint{1.955292in}{1.953699in}}%
\pgfpathlineto{\pgfqpoint{1.953018in}{1.963993in}}%
\pgfpathlineto{\pgfqpoint{1.951487in}{1.968248in}}%
\pgfpathlineto{\pgfqpoint{1.949138in}{1.974288in}}%
\pgfpathlineto{\pgfqpoint{1.943976in}{1.984582in}}%
\pgfpathlineto{\pgfqpoint{1.940910in}{1.989872in}}%
\pgfpathlineto{\pgfqpoint{1.937801in}{1.994877in}}%
\pgfpathlineto{\pgfqpoint{1.930971in}{2.005171in}}%
\pgfpathlineto{\pgfqpoint{1.930333in}{2.006097in}}%
\pgfpathlineto{\pgfqpoint{1.930333in}{2.005171in}}%
\pgfpathlineto{\pgfqpoint{1.930333in}{1.994877in}}%
\pgfpathlineto{\pgfqpoint{1.930333in}{1.984582in}}%
\pgfpathlineto{\pgfqpoint{1.930333in}{1.974288in}}%
\pgfpathlineto{\pgfqpoint{1.930333in}{1.963993in}}%
\pgfpathlineto{\pgfqpoint{1.930333in}{1.954256in}}%
\pgfpathlineto{\pgfqpoint{1.930455in}{1.953699in}}%
\pgfpathlineto{\pgfqpoint{1.930700in}{1.943405in}}%
\pgfpathlineto{\pgfqpoint{1.930333in}{1.941483in}}%
\pgfpathlineto{\pgfqpoint{1.930333in}{1.933110in}}%
\pgfpathlineto{\pgfqpoint{1.930333in}{1.922816in}}%
\pgfpathlineto{\pgfqpoint{1.930333in}{1.912521in}}%
\pgfpathlineto{\pgfqpoint{1.930333in}{1.902227in}}%
\pgfpathlineto{\pgfqpoint{1.930333in}{1.895291in}}%
\pgfpathclose%
\pgfusepath{fill}%
\end{pgfscope}%
\begin{pgfscope}%
\pgfpathrectangle{\pgfqpoint{1.856795in}{1.819814in}}{\pgfqpoint{1.194205in}{1.163386in}}%
\pgfusepath{clip}%
\pgfsetbuttcap%
\pgfsetroundjoin%
\definecolor{currentfill}{rgb}{0.157709,0.083393,0.206797}%
\pgfsetfillcolor{currentfill}%
\pgfsetlinewidth{0.000000pt}%
\definecolor{currentstroke}{rgb}{0.000000,0.000000,0.000000}%
\pgfsetstrokecolor{currentstroke}%
\pgfsetdash{}{0pt}%
\pgfpathmoveto{\pgfqpoint{2.649573in}{2.855410in}}%
\pgfpathlineto{\pgfqpoint{2.660150in}{2.853709in}}%
\pgfpathlineto{\pgfqpoint{2.670727in}{2.854707in}}%
\pgfpathlineto{\pgfqpoint{2.681305in}{2.856222in}}%
\pgfpathlineto{\pgfqpoint{2.691882in}{2.858170in}}%
\pgfpathlineto{\pgfqpoint{2.698317in}{2.859609in}}%
\pgfpathlineto{\pgfqpoint{2.702459in}{2.860580in}}%
\pgfpathlineto{\pgfqpoint{2.713036in}{2.863493in}}%
\pgfpathlineto{\pgfqpoint{2.723613in}{2.866889in}}%
\pgfpathlineto{\pgfqpoint{2.731857in}{2.869903in}}%
\pgfpathlineto{\pgfqpoint{2.734190in}{2.870814in}}%
\pgfpathlineto{\pgfqpoint{2.744767in}{2.875391in}}%
\pgfpathlineto{\pgfqpoint{2.754739in}{2.880198in}}%
\pgfpathlineto{\pgfqpoint{2.755344in}{2.880517in}}%
\pgfpathlineto{\pgfqpoint{2.765921in}{2.886532in}}%
\pgfpathlineto{\pgfqpoint{2.772356in}{2.890492in}}%
\pgfpathlineto{\pgfqpoint{2.776498in}{2.893392in}}%
\pgfpathlineto{\pgfqpoint{2.786383in}{2.900787in}}%
\pgfpathlineto{\pgfqpoint{2.787075in}{2.901411in}}%
\pgfpathlineto{\pgfqpoint{2.797221in}{2.911081in}}%
\pgfpathlineto{\pgfqpoint{2.787075in}{2.911081in}}%
\pgfpathlineto{\pgfqpoint{2.776498in}{2.911081in}}%
\pgfpathlineto{\pgfqpoint{2.765921in}{2.911081in}}%
\pgfpathlineto{\pgfqpoint{2.763168in}{2.911081in}}%
\pgfpathlineto{\pgfqpoint{2.755344in}{2.904827in}}%
\pgfpathlineto{\pgfqpoint{2.749851in}{2.900787in}}%
\pgfpathlineto{\pgfqpoint{2.744767in}{2.897628in}}%
\pgfpathlineto{\pgfqpoint{2.734190in}{2.891649in}}%
\pgfpathlineto{\pgfqpoint{2.731935in}{2.890492in}}%
\pgfpathlineto{\pgfqpoint{2.723613in}{2.886689in}}%
\pgfpathlineto{\pgfqpoint{2.713036in}{2.882417in}}%
\pgfpathlineto{\pgfqpoint{2.706750in}{2.880198in}}%
\pgfpathlineto{\pgfqpoint{2.702459in}{2.878799in}}%
\pgfpathlineto{\pgfqpoint{2.691882in}{2.875819in}}%
\pgfpathlineto{\pgfqpoint{2.681305in}{2.873364in}}%
\pgfpathlineto{\pgfqpoint{2.670727in}{2.871584in}}%
\pgfpathlineto{\pgfqpoint{2.660150in}{2.875392in}}%
\pgfpathlineto{\pgfqpoint{2.657702in}{2.880198in}}%
\pgfpathlineto{\pgfqpoint{2.654490in}{2.890492in}}%
\pgfpathlineto{\pgfqpoint{2.652382in}{2.900787in}}%
\pgfpathlineto{\pgfqpoint{2.651274in}{2.911081in}}%
\pgfpathlineto{\pgfqpoint{2.649573in}{2.911081in}}%
\pgfpathlineto{\pgfqpoint{2.638996in}{2.911081in}}%
\pgfpathlineto{\pgfqpoint{2.633285in}{2.911081in}}%
\pgfpathlineto{\pgfqpoint{2.634042in}{2.900787in}}%
\pgfpathlineto{\pgfqpoint{2.635631in}{2.890492in}}%
\pgfpathlineto{\pgfqpoint{2.638039in}{2.880198in}}%
\pgfpathlineto{\pgfqpoint{2.638996in}{2.877083in}}%
\pgfpathlineto{\pgfqpoint{2.641359in}{2.869903in}}%
\pgfpathlineto{\pgfqpoint{2.645886in}{2.859609in}}%
\pgfpathclose%
\pgfusepath{fill}%
\end{pgfscope}%
\begin{pgfscope}%
\pgfpathrectangle{\pgfqpoint{1.856795in}{1.819814in}}{\pgfqpoint{1.194205in}{1.163386in}}%
\pgfusepath{clip}%
\pgfsetbuttcap%
\pgfsetroundjoin%
\definecolor{currentfill}{rgb}{0.221500,0.100139,0.249787}%
\pgfsetfillcolor{currentfill}%
\pgfsetlinewidth{0.000000pt}%
\definecolor{currentstroke}{rgb}{0.000000,0.000000,0.000000}%
\pgfsetstrokecolor{currentstroke}%
\pgfsetdash{}{0pt}%
\pgfpathmoveto{\pgfqpoint{1.940910in}{1.891932in}}%
\pgfpathlineto{\pgfqpoint{1.951487in}{1.891932in}}%
\pgfpathlineto{\pgfqpoint{1.955791in}{1.891932in}}%
\pgfpathlineto{\pgfqpoint{1.962064in}{1.899686in}}%
\pgfpathlineto{\pgfqpoint{1.963849in}{1.902227in}}%
\pgfpathlineto{\pgfqpoint{1.969377in}{1.912521in}}%
\pgfpathlineto{\pgfqpoint{1.972641in}{1.921287in}}%
\pgfpathlineto{\pgfqpoint{1.973153in}{1.922816in}}%
\pgfpathlineto{\pgfqpoint{1.975205in}{1.933110in}}%
\pgfpathlineto{\pgfqpoint{1.975860in}{1.943405in}}%
\pgfpathlineto{\pgfqpoint{1.975132in}{1.953699in}}%
\pgfpathlineto{\pgfqpoint{1.973022in}{1.963993in}}%
\pgfpathlineto{\pgfqpoint{1.972641in}{1.965133in}}%
\pgfpathlineto{\pgfqpoint{1.969388in}{1.974288in}}%
\pgfpathlineto{\pgfqpoint{1.964592in}{1.984582in}}%
\pgfpathlineto{\pgfqpoint{1.962064in}{1.989194in}}%
\pgfpathlineto{\pgfqpoint{1.958803in}{1.994877in}}%
\pgfpathlineto{\pgfqpoint{1.952428in}{2.005171in}}%
\pgfpathlineto{\pgfqpoint{1.951487in}{2.006630in}}%
\pgfpathlineto{\pgfqpoint{1.945529in}{2.015466in}}%
\pgfpathlineto{\pgfqpoint{1.940910in}{2.022371in}}%
\pgfpathlineto{\pgfqpoint{1.938520in}{2.025760in}}%
\pgfpathlineto{\pgfqpoint{1.931467in}{2.036054in}}%
\pgfpathlineto{\pgfqpoint{1.930333in}{2.037780in}}%
\pgfpathlineto{\pgfqpoint{1.930333in}{2.036054in}}%
\pgfpathlineto{\pgfqpoint{1.930333in}{2.025760in}}%
\pgfpathlineto{\pgfqpoint{1.930333in}{2.015466in}}%
\pgfpathlineto{\pgfqpoint{1.930333in}{2.006097in}}%
\pgfpathlineto{\pgfqpoint{1.930971in}{2.005171in}}%
\pgfpathlineto{\pgfqpoint{1.937801in}{1.994877in}}%
\pgfpathlineto{\pgfqpoint{1.940910in}{1.989872in}}%
\pgfpathlineto{\pgfqpoint{1.943976in}{1.984582in}}%
\pgfpathlineto{\pgfqpoint{1.949138in}{1.974288in}}%
\pgfpathlineto{\pgfqpoint{1.951487in}{1.968248in}}%
\pgfpathlineto{\pgfqpoint{1.953018in}{1.963993in}}%
\pgfpathlineto{\pgfqpoint{1.955292in}{1.953699in}}%
\pgfpathlineto{\pgfqpoint{1.955967in}{1.943405in}}%
\pgfpathlineto{\pgfqpoint{1.954923in}{1.933110in}}%
\pgfpathlineto{\pgfqpoint{1.952080in}{1.922816in}}%
\pgfpathlineto{\pgfqpoint{1.951487in}{1.921505in}}%
\pgfpathlineto{\pgfqpoint{1.946744in}{1.912521in}}%
\pgfpathlineto{\pgfqpoint{1.940910in}{1.904753in}}%
\pgfpathlineto{\pgfqpoint{1.938590in}{1.902227in}}%
\pgfpathlineto{\pgfqpoint{1.930333in}{1.895291in}}%
\pgfpathlineto{\pgfqpoint{1.930333in}{1.891932in}}%
\pgfpathclose%
\pgfusepath{fill}%
\end{pgfscope}%
\begin{pgfscope}%
\pgfpathrectangle{\pgfqpoint{1.856795in}{1.819814in}}{\pgfqpoint{1.194205in}{1.163386in}}%
\pgfusepath{clip}%
\pgfsetbuttcap%
\pgfsetroundjoin%
\definecolor{currentfill}{rgb}{0.221500,0.100139,0.249787}%
\pgfsetfillcolor{currentfill}%
\pgfsetlinewidth{0.000000pt}%
\definecolor{currentstroke}{rgb}{0.000000,0.000000,0.000000}%
\pgfsetstrokecolor{currentstroke}%
\pgfsetdash{}{0pt}%
\pgfpathmoveto{\pgfqpoint{2.638996in}{2.837343in}}%
\pgfpathlineto{\pgfqpoint{2.649573in}{2.837238in}}%
\pgfpathlineto{\pgfqpoint{2.660150in}{2.837701in}}%
\pgfpathlineto{\pgfqpoint{2.670727in}{2.838506in}}%
\pgfpathlineto{\pgfqpoint{2.675475in}{2.839020in}}%
\pgfpathlineto{\pgfqpoint{2.681305in}{2.839673in}}%
\pgfpathlineto{\pgfqpoint{2.691882in}{2.841224in}}%
\pgfpathlineto{\pgfqpoint{2.702459in}{2.843178in}}%
\pgfpathlineto{\pgfqpoint{2.713036in}{2.845558in}}%
\pgfpathlineto{\pgfqpoint{2.723613in}{2.848381in}}%
\pgfpathlineto{\pgfqpoint{2.726673in}{2.849315in}}%
\pgfpathlineto{\pgfqpoint{2.734190in}{2.851702in}}%
\pgfpathlineto{\pgfqpoint{2.744767in}{2.855512in}}%
\pgfpathlineto{\pgfqpoint{2.754905in}{2.859609in}}%
\pgfpathlineto{\pgfqpoint{2.755344in}{2.859796in}}%
\pgfpathlineto{\pgfqpoint{2.765921in}{2.864680in}}%
\pgfpathlineto{\pgfqpoint{2.776276in}{2.869903in}}%
\pgfpathlineto{\pgfqpoint{2.776498in}{2.870024in}}%
\pgfpathlineto{\pgfqpoint{2.787075in}{2.876083in}}%
\pgfpathlineto{\pgfqpoint{2.793850in}{2.880198in}}%
\pgfpathlineto{\pgfqpoint{2.797652in}{2.882747in}}%
\pgfpathlineto{\pgfqpoint{2.808229in}{2.890162in}}%
\pgfpathlineto{\pgfqpoint{2.808688in}{2.890492in}}%
\pgfpathlineto{\pgfqpoint{2.818806in}{2.898936in}}%
\pgfpathlineto{\pgfqpoint{2.820970in}{2.900787in}}%
\pgfpathlineto{\pgfqpoint{2.829383in}{2.909815in}}%
\pgfpathlineto{\pgfqpoint{2.830541in}{2.911081in}}%
\pgfpathlineto{\pgfqpoint{2.829383in}{2.911081in}}%
\pgfpathlineto{\pgfqpoint{2.818806in}{2.911081in}}%
\pgfpathlineto{\pgfqpoint{2.808229in}{2.911081in}}%
\pgfpathlineto{\pgfqpoint{2.797652in}{2.911081in}}%
\pgfpathlineto{\pgfqpoint{2.797221in}{2.911081in}}%
\pgfpathlineto{\pgfqpoint{2.787075in}{2.901411in}}%
\pgfpathlineto{\pgfqpoint{2.786383in}{2.900787in}}%
\pgfpathlineto{\pgfqpoint{2.776498in}{2.893392in}}%
\pgfpathlineto{\pgfqpoint{2.772356in}{2.890492in}}%
\pgfpathlineto{\pgfqpoint{2.765921in}{2.886532in}}%
\pgfpathlineto{\pgfqpoint{2.755344in}{2.880517in}}%
\pgfpathlineto{\pgfqpoint{2.754739in}{2.880198in}}%
\pgfpathlineto{\pgfqpoint{2.744767in}{2.875391in}}%
\pgfpathlineto{\pgfqpoint{2.734190in}{2.870814in}}%
\pgfpathlineto{\pgfqpoint{2.731857in}{2.869903in}}%
\pgfpathlineto{\pgfqpoint{2.723613in}{2.866889in}}%
\pgfpathlineto{\pgfqpoint{2.713036in}{2.863493in}}%
\pgfpathlineto{\pgfqpoint{2.702459in}{2.860580in}}%
\pgfpathlineto{\pgfqpoint{2.698317in}{2.859609in}}%
\pgfpathlineto{\pgfqpoint{2.691882in}{2.858170in}}%
\pgfpathlineto{\pgfqpoint{2.681305in}{2.856222in}}%
\pgfpathlineto{\pgfqpoint{2.670727in}{2.854707in}}%
\pgfpathlineto{\pgfqpoint{2.660150in}{2.853709in}}%
\pgfpathlineto{\pgfqpoint{2.649573in}{2.855410in}}%
\pgfpathlineto{\pgfqpoint{2.645886in}{2.859609in}}%
\pgfpathlineto{\pgfqpoint{2.641359in}{2.869903in}}%
\pgfpathlineto{\pgfqpoint{2.638996in}{2.877083in}}%
\pgfpathlineto{\pgfqpoint{2.638039in}{2.880198in}}%
\pgfpathlineto{\pgfqpoint{2.635631in}{2.890492in}}%
\pgfpathlineto{\pgfqpoint{2.634042in}{2.900787in}}%
\pgfpathlineto{\pgfqpoint{2.633285in}{2.911081in}}%
\pgfpathlineto{\pgfqpoint{2.628419in}{2.911081in}}%
\pgfpathlineto{\pgfqpoint{2.617842in}{2.911081in}}%
\pgfpathlineto{\pgfqpoint{2.615983in}{2.911081in}}%
\pgfpathlineto{\pgfqpoint{2.616409in}{2.900787in}}%
\pgfpathlineto{\pgfqpoint{2.617570in}{2.890492in}}%
\pgfpathlineto{\pgfqpoint{2.617842in}{2.888970in}}%
\pgfpathlineto{\pgfqpoint{2.619468in}{2.880198in}}%
\pgfpathlineto{\pgfqpoint{2.622086in}{2.869903in}}%
\pgfpathlineto{\pgfqpoint{2.625419in}{2.859609in}}%
\pgfpathlineto{\pgfqpoint{2.628419in}{2.851929in}}%
\pgfpathlineto{\pgfqpoint{2.629520in}{2.849315in}}%
\pgfpathlineto{\pgfqpoint{2.636056in}{2.839020in}}%
\pgfpathclose%
\pgfusepath{fill}%
\end{pgfscope}%
\begin{pgfscope}%
\pgfpathrectangle{\pgfqpoint{1.856795in}{1.819814in}}{\pgfqpoint{1.194205in}{1.163386in}}%
\pgfusepath{clip}%
\pgfsetbuttcap%
\pgfsetroundjoin%
\definecolor{currentfill}{rgb}{0.293648,0.112759,0.289608}%
\pgfsetfillcolor{currentfill}%
\pgfsetlinewidth{0.000000pt}%
\definecolor{currentstroke}{rgb}{0.000000,0.000000,0.000000}%
\pgfsetstrokecolor{currentstroke}%
\pgfsetdash{}{0pt}%
\pgfpathmoveto{\pgfqpoint{1.962064in}{1.891932in}}%
\pgfpathlineto{\pgfqpoint{1.972641in}{1.891932in}}%
\pgfpathlineto{\pgfqpoint{1.977754in}{1.891932in}}%
\pgfpathlineto{\pgfqpoint{1.983218in}{1.901018in}}%
\pgfpathlineto{\pgfqpoint{1.983877in}{1.902227in}}%
\pgfpathlineto{\pgfqpoint{1.988142in}{1.912521in}}%
\pgfpathlineto{\pgfqpoint{1.991053in}{1.922816in}}%
\pgfpathlineto{\pgfqpoint{1.992713in}{1.933110in}}%
\pgfpathlineto{\pgfqpoint{1.993180in}{1.943405in}}%
\pgfpathlineto{\pgfqpoint{1.992452in}{1.953699in}}%
\pgfpathlineto{\pgfqpoint{1.990478in}{1.963993in}}%
\pgfpathlineto{\pgfqpoint{1.987216in}{1.974288in}}%
\pgfpathlineto{\pgfqpoint{1.983218in}{1.983509in}}%
\pgfpathlineto{\pgfqpoint{1.982732in}{1.984582in}}%
\pgfpathlineto{\pgfqpoint{1.977137in}{1.994877in}}%
\pgfpathlineto{\pgfqpoint{1.972641in}{2.002432in}}%
\pgfpathlineto{\pgfqpoint{1.970962in}{2.005171in}}%
\pgfpathlineto{\pgfqpoint{1.964400in}{2.015466in}}%
\pgfpathlineto{\pgfqpoint{1.962064in}{2.019117in}}%
\pgfpathlineto{\pgfqpoint{1.957676in}{2.025760in}}%
\pgfpathlineto{\pgfqpoint{1.951487in}{2.035388in}}%
\pgfpathlineto{\pgfqpoint{1.951042in}{2.036054in}}%
\pgfpathlineto{\pgfqpoint{1.944394in}{2.046349in}}%
\pgfpathlineto{\pgfqpoint{1.940910in}{2.052079in}}%
\pgfpathlineto{\pgfqpoint{1.938011in}{2.056643in}}%
\pgfpathlineto{\pgfqpoint{1.931940in}{2.066938in}}%
\pgfpathlineto{\pgfqpoint{1.930333in}{2.069896in}}%
\pgfpathlineto{\pgfqpoint{1.930333in}{2.066938in}}%
\pgfpathlineto{\pgfqpoint{1.930333in}{2.056643in}}%
\pgfpathlineto{\pgfqpoint{1.930333in}{2.046349in}}%
\pgfpathlineto{\pgfqpoint{1.930333in}{2.037780in}}%
\pgfpathlineto{\pgfqpoint{1.931467in}{2.036054in}}%
\pgfpathlineto{\pgfqpoint{1.938520in}{2.025760in}}%
\pgfpathlineto{\pgfqpoint{1.940910in}{2.022371in}}%
\pgfpathlineto{\pgfqpoint{1.945529in}{2.015466in}}%
\pgfpathlineto{\pgfqpoint{1.951487in}{2.006630in}}%
\pgfpathlineto{\pgfqpoint{1.952428in}{2.005171in}}%
\pgfpathlineto{\pgfqpoint{1.958803in}{1.994877in}}%
\pgfpathlineto{\pgfqpoint{1.962064in}{1.989194in}}%
\pgfpathlineto{\pgfqpoint{1.964592in}{1.984582in}}%
\pgfpathlineto{\pgfqpoint{1.969388in}{1.974288in}}%
\pgfpathlineto{\pgfqpoint{1.972641in}{1.965133in}}%
\pgfpathlineto{\pgfqpoint{1.973022in}{1.963993in}}%
\pgfpathlineto{\pgfqpoint{1.975132in}{1.953699in}}%
\pgfpathlineto{\pgfqpoint{1.975860in}{1.943405in}}%
\pgfpathlineto{\pgfqpoint{1.975205in}{1.933110in}}%
\pgfpathlineto{\pgfqpoint{1.973153in}{1.922816in}}%
\pgfpathlineto{\pgfqpoint{1.972641in}{1.921287in}}%
\pgfpathlineto{\pgfqpoint{1.969377in}{1.912521in}}%
\pgfpathlineto{\pgfqpoint{1.963849in}{1.902227in}}%
\pgfpathlineto{\pgfqpoint{1.962064in}{1.899686in}}%
\pgfpathlineto{\pgfqpoint{1.955791in}{1.891932in}}%
\pgfpathclose%
\pgfusepath{fill}%
\end{pgfscope}%
\begin{pgfscope}%
\pgfpathrectangle{\pgfqpoint{1.856795in}{1.819814in}}{\pgfqpoint{1.194205in}{1.163386in}}%
\pgfusepath{clip}%
\pgfsetbuttcap%
\pgfsetroundjoin%
\definecolor{currentfill}{rgb}{0.293648,0.112759,0.289608}%
\pgfsetfillcolor{currentfill}%
\pgfsetlinewidth{0.000000pt}%
\definecolor{currentstroke}{rgb}{0.000000,0.000000,0.000000}%
\pgfsetstrokecolor{currentstroke}%
\pgfsetdash{}{0pt}%
\pgfpathmoveto{\pgfqpoint{2.628419in}{2.822521in}}%
\pgfpathlineto{\pgfqpoint{2.638996in}{2.822149in}}%
\pgfpathlineto{\pgfqpoint{2.649573in}{2.822042in}}%
\pgfpathlineto{\pgfqpoint{2.660150in}{2.822205in}}%
\pgfpathlineto{\pgfqpoint{2.670727in}{2.822665in}}%
\pgfpathlineto{\pgfqpoint{2.681305in}{2.823451in}}%
\pgfpathlineto{\pgfqpoint{2.691882in}{2.824592in}}%
\pgfpathlineto{\pgfqpoint{2.702459in}{2.826118in}}%
\pgfpathlineto{\pgfqpoint{2.713036in}{2.828056in}}%
\pgfpathlineto{\pgfqpoint{2.716076in}{2.828726in}}%
\pgfpathlineto{\pgfqpoint{2.723613in}{2.830431in}}%
\pgfpathlineto{\pgfqpoint{2.734190in}{2.833258in}}%
\pgfpathlineto{\pgfqpoint{2.744767in}{2.836548in}}%
\pgfpathlineto{\pgfqpoint{2.751782in}{2.839020in}}%
\pgfpathlineto{\pgfqpoint{2.755344in}{2.840313in}}%
\pgfpathlineto{\pgfqpoint{2.765921in}{2.844558in}}%
\pgfpathlineto{\pgfqpoint{2.776498in}{2.849240in}}%
\pgfpathlineto{\pgfqpoint{2.776655in}{2.849315in}}%
\pgfpathlineto{\pgfqpoint{2.787075in}{2.854428in}}%
\pgfpathlineto{\pgfqpoint{2.796948in}{2.859609in}}%
\pgfpathlineto{\pgfqpoint{2.797652in}{2.859998in}}%
\pgfpathlineto{\pgfqpoint{2.808229in}{2.866085in}}%
\pgfpathlineto{\pgfqpoint{2.814627in}{2.869903in}}%
\pgfpathlineto{\pgfqpoint{2.818806in}{2.872583in}}%
\pgfpathlineto{\pgfqpoint{2.829383in}{2.879514in}}%
\pgfpathlineto{\pgfqpoint{2.830419in}{2.880198in}}%
\pgfpathlineto{\pgfqpoint{2.839960in}{2.887203in}}%
\pgfpathlineto{\pgfqpoint{2.844444in}{2.890492in}}%
\pgfpathlineto{\pgfqpoint{2.850538in}{2.895758in}}%
\pgfpathlineto{\pgfqpoint{2.856406in}{2.900787in}}%
\pgfpathlineto{\pgfqpoint{2.861115in}{2.906071in}}%
\pgfpathlineto{\pgfqpoint{2.865618in}{2.911081in}}%
\pgfpathlineto{\pgfqpoint{2.861115in}{2.911081in}}%
\pgfpathlineto{\pgfqpoint{2.850538in}{2.911081in}}%
\pgfpathlineto{\pgfqpoint{2.839960in}{2.911081in}}%
\pgfpathlineto{\pgfqpoint{2.830541in}{2.911081in}}%
\pgfpathlineto{\pgfqpoint{2.829383in}{2.909815in}}%
\pgfpathlineto{\pgfqpoint{2.820970in}{2.900787in}}%
\pgfpathlineto{\pgfqpoint{2.818806in}{2.898936in}}%
\pgfpathlineto{\pgfqpoint{2.808688in}{2.890492in}}%
\pgfpathlineto{\pgfqpoint{2.808229in}{2.890162in}}%
\pgfpathlineto{\pgfqpoint{2.797652in}{2.882747in}}%
\pgfpathlineto{\pgfqpoint{2.793850in}{2.880198in}}%
\pgfpathlineto{\pgfqpoint{2.787075in}{2.876083in}}%
\pgfpathlineto{\pgfqpoint{2.776498in}{2.870024in}}%
\pgfpathlineto{\pgfqpoint{2.776276in}{2.869903in}}%
\pgfpathlineto{\pgfqpoint{2.765921in}{2.864680in}}%
\pgfpathlineto{\pgfqpoint{2.755344in}{2.859796in}}%
\pgfpathlineto{\pgfqpoint{2.754905in}{2.859609in}}%
\pgfpathlineto{\pgfqpoint{2.744767in}{2.855512in}}%
\pgfpathlineto{\pgfqpoint{2.734190in}{2.851702in}}%
\pgfpathlineto{\pgfqpoint{2.726673in}{2.849315in}}%
\pgfpathlineto{\pgfqpoint{2.723613in}{2.848381in}}%
\pgfpathlineto{\pgfqpoint{2.713036in}{2.845558in}}%
\pgfpathlineto{\pgfqpoint{2.702459in}{2.843178in}}%
\pgfpathlineto{\pgfqpoint{2.691882in}{2.841224in}}%
\pgfpathlineto{\pgfqpoint{2.681305in}{2.839673in}}%
\pgfpathlineto{\pgfqpoint{2.675475in}{2.839020in}}%
\pgfpathlineto{\pgfqpoint{2.670727in}{2.838506in}}%
\pgfpathlineto{\pgfqpoint{2.660150in}{2.837701in}}%
\pgfpathlineto{\pgfqpoint{2.649573in}{2.837238in}}%
\pgfpathlineto{\pgfqpoint{2.638996in}{2.837343in}}%
\pgfpathlineto{\pgfqpoint{2.636056in}{2.839020in}}%
\pgfpathlineto{\pgfqpoint{2.629520in}{2.849315in}}%
\pgfpathlineto{\pgfqpoint{2.628419in}{2.851929in}}%
\pgfpathlineto{\pgfqpoint{2.625419in}{2.859609in}}%
\pgfpathlineto{\pgfqpoint{2.622086in}{2.869903in}}%
\pgfpathlineto{\pgfqpoint{2.619468in}{2.880198in}}%
\pgfpathlineto{\pgfqpoint{2.617842in}{2.888970in}}%
\pgfpathlineto{\pgfqpoint{2.617570in}{2.890492in}}%
\pgfpathlineto{\pgfqpoint{2.616409in}{2.900787in}}%
\pgfpathlineto{\pgfqpoint{2.615983in}{2.911081in}}%
\pgfpathlineto{\pgfqpoint{2.607265in}{2.911081in}}%
\pgfpathlineto{\pgfqpoint{2.599007in}{2.911081in}}%
\pgfpathlineto{\pgfqpoint{2.599110in}{2.900787in}}%
\pgfpathlineto{\pgfqpoint{2.599888in}{2.890492in}}%
\pgfpathlineto{\pgfqpoint{2.601322in}{2.880198in}}%
\pgfpathlineto{\pgfqpoint{2.603400in}{2.869903in}}%
\pgfpathlineto{\pgfqpoint{2.606114in}{2.859609in}}%
\pgfpathlineto{\pgfqpoint{2.607265in}{2.855999in}}%
\pgfpathlineto{\pgfqpoint{2.609477in}{2.849315in}}%
\pgfpathlineto{\pgfqpoint{2.613494in}{2.839020in}}%
\pgfpathlineto{\pgfqpoint{2.617842in}{2.829435in}}%
\pgfpathlineto{\pgfqpoint{2.618218in}{2.828726in}}%
\pgfpathclose%
\pgfusepath{fill}%
\end{pgfscope}%
\begin{pgfscope}%
\pgfpathrectangle{\pgfqpoint{1.856795in}{1.819814in}}{\pgfqpoint{1.194205in}{1.163386in}}%
\pgfusepath{clip}%
\pgfsetbuttcap%
\pgfsetroundjoin%
\definecolor{currentfill}{rgb}{0.361966,0.119541,0.318520}%
\pgfsetfillcolor{currentfill}%
\pgfsetlinewidth{0.000000pt}%
\definecolor{currentstroke}{rgb}{0.000000,0.000000,0.000000}%
\pgfsetstrokecolor{currentstroke}%
\pgfsetdash{}{0pt}%
\pgfpathmoveto{\pgfqpoint{1.983218in}{1.891932in}}%
\pgfpathlineto{\pgfqpoint{1.993796in}{1.891932in}}%
\pgfpathlineto{\pgfqpoint{1.996627in}{1.891932in}}%
\pgfpathlineto{\pgfqpoint{2.001551in}{1.902227in}}%
\pgfpathlineto{\pgfqpoint{2.004373in}{1.910268in}}%
\pgfpathlineto{\pgfqpoint{2.005114in}{1.912521in}}%
\pgfpathlineto{\pgfqpoint{2.007419in}{1.922816in}}%
\pgfpathlineto{\pgfqpoint{2.008670in}{1.933110in}}%
\pgfpathlineto{\pgfqpoint{2.008937in}{1.943405in}}%
\pgfpathlineto{\pgfqpoint{2.008241in}{1.953699in}}%
\pgfpathlineto{\pgfqpoint{2.006533in}{1.963993in}}%
\pgfpathlineto{\pgfqpoint{2.004373in}{1.971824in}}%
\pgfpathlineto{\pgfqpoint{2.003655in}{1.974288in}}%
\pgfpathlineto{\pgfqpoint{1.999417in}{1.984582in}}%
\pgfpathlineto{\pgfqpoint{1.994101in}{1.994877in}}%
\pgfpathlineto{\pgfqpoint{1.993796in}{1.995399in}}%
\pgfpathlineto{\pgfqpoint{1.987981in}{2.005171in}}%
\pgfpathlineto{\pgfqpoint{1.983218in}{2.012851in}}%
\pgfpathlineto{\pgfqpoint{1.981567in}{2.015466in}}%
\pgfpathlineto{\pgfqpoint{1.975025in}{2.025760in}}%
\pgfpathlineto{\pgfqpoint{1.972641in}{2.029581in}}%
\pgfpathlineto{\pgfqpoint{1.968510in}{2.036054in}}%
\pgfpathlineto{\pgfqpoint{1.962198in}{2.046349in}}%
\pgfpathlineto{\pgfqpoint{1.962064in}{2.046576in}}%
\pgfpathlineto{\pgfqpoint{1.955989in}{2.056643in}}%
\pgfpathlineto{\pgfqpoint{1.951487in}{2.064625in}}%
\pgfpathlineto{\pgfqpoint{1.950141in}{2.066938in}}%
\pgfpathlineto{\pgfqpoint{1.944569in}{2.077232in}}%
\pgfpathlineto{\pgfqpoint{1.940910in}{2.084643in}}%
\pgfpathlineto{\pgfqpoint{1.939439in}{2.087527in}}%
\pgfpathlineto{\pgfqpoint{1.934684in}{2.097821in}}%
\pgfpathlineto{\pgfqpoint{1.930496in}{2.108116in}}%
\pgfpathlineto{\pgfqpoint{1.930333in}{2.108565in}}%
\pgfpathlineto{\pgfqpoint{1.930333in}{2.108116in}}%
\pgfpathlineto{\pgfqpoint{1.930333in}{2.097821in}}%
\pgfpathlineto{\pgfqpoint{1.930333in}{2.087527in}}%
\pgfpathlineto{\pgfqpoint{1.930333in}{2.077232in}}%
\pgfpathlineto{\pgfqpoint{1.930333in}{2.069896in}}%
\pgfpathlineto{\pgfqpoint{1.931940in}{2.066938in}}%
\pgfpathlineto{\pgfqpoint{1.938011in}{2.056643in}}%
\pgfpathlineto{\pgfqpoint{1.940910in}{2.052079in}}%
\pgfpathlineto{\pgfqpoint{1.944394in}{2.046349in}}%
\pgfpathlineto{\pgfqpoint{1.951042in}{2.036054in}}%
\pgfpathlineto{\pgfqpoint{1.951487in}{2.035388in}}%
\pgfpathlineto{\pgfqpoint{1.957676in}{2.025760in}}%
\pgfpathlineto{\pgfqpoint{1.962064in}{2.019117in}}%
\pgfpathlineto{\pgfqpoint{1.964400in}{2.015466in}}%
\pgfpathlineto{\pgfqpoint{1.970962in}{2.005171in}}%
\pgfpathlineto{\pgfqpoint{1.972641in}{2.002432in}}%
\pgfpathlineto{\pgfqpoint{1.977137in}{1.994877in}}%
\pgfpathlineto{\pgfqpoint{1.982732in}{1.984582in}}%
\pgfpathlineto{\pgfqpoint{1.983218in}{1.983509in}}%
\pgfpathlineto{\pgfqpoint{1.987216in}{1.974288in}}%
\pgfpathlineto{\pgfqpoint{1.990478in}{1.963993in}}%
\pgfpathlineto{\pgfqpoint{1.992452in}{1.953699in}}%
\pgfpathlineto{\pgfqpoint{1.993180in}{1.943405in}}%
\pgfpathlineto{\pgfqpoint{1.992713in}{1.933110in}}%
\pgfpathlineto{\pgfqpoint{1.991053in}{1.922816in}}%
\pgfpathlineto{\pgfqpoint{1.988142in}{1.912521in}}%
\pgfpathlineto{\pgfqpoint{1.983877in}{1.902227in}}%
\pgfpathlineto{\pgfqpoint{1.983218in}{1.901018in}}%
\pgfpathlineto{\pgfqpoint{1.977754in}{1.891932in}}%
\pgfpathclose%
\pgfusepath{fill}%
\end{pgfscope}%
\begin{pgfscope}%
\pgfpathrectangle{\pgfqpoint{1.856795in}{1.819814in}}{\pgfqpoint{1.194205in}{1.163386in}}%
\pgfusepath{clip}%
\pgfsetbuttcap%
\pgfsetroundjoin%
\definecolor{currentfill}{rgb}{0.361966,0.119541,0.318520}%
\pgfsetfillcolor{currentfill}%
\pgfsetlinewidth{0.000000pt}%
\definecolor{currentstroke}{rgb}{0.000000,0.000000,0.000000}%
\pgfsetstrokecolor{currentstroke}%
\pgfsetdash{}{0pt}%
\pgfpathmoveto{\pgfqpoint{2.628419in}{2.808059in}}%
\pgfpathlineto{\pgfqpoint{2.638996in}{2.807391in}}%
\pgfpathlineto{\pgfqpoint{2.649573in}{2.806955in}}%
\pgfpathlineto{\pgfqpoint{2.660150in}{2.806768in}}%
\pgfpathlineto{\pgfqpoint{2.670727in}{2.806858in}}%
\pgfpathlineto{\pgfqpoint{2.681305in}{2.807256in}}%
\pgfpathlineto{\pgfqpoint{2.691882in}{2.807996in}}%
\pgfpathlineto{\pgfqpoint{2.693252in}{2.808137in}}%
\pgfpathlineto{\pgfqpoint{2.702459in}{2.809101in}}%
\pgfpathlineto{\pgfqpoint{2.713036in}{2.810607in}}%
\pgfpathlineto{\pgfqpoint{2.723613in}{2.812548in}}%
\pgfpathlineto{\pgfqpoint{2.734190in}{2.814947in}}%
\pgfpathlineto{\pgfqpoint{2.744767in}{2.817826in}}%
\pgfpathlineto{\pgfqpoint{2.746702in}{2.818431in}}%
\pgfpathlineto{\pgfqpoint{2.755344in}{2.821172in}}%
\pgfpathlineto{\pgfqpoint{2.765921in}{2.824994in}}%
\pgfpathlineto{\pgfqpoint{2.775144in}{2.828726in}}%
\pgfpathlineto{\pgfqpoint{2.776498in}{2.829283in}}%
\pgfpathlineto{\pgfqpoint{2.787075in}{2.834006in}}%
\pgfpathlineto{\pgfqpoint{2.797445in}{2.839020in}}%
\pgfpathlineto{\pgfqpoint{2.797652in}{2.839123in}}%
\pgfpathlineto{\pgfqpoint{2.808229in}{2.844628in}}%
\pgfpathlineto{\pgfqpoint{2.816823in}{2.849315in}}%
\pgfpathlineto{\pgfqpoint{2.818806in}{2.850429in}}%
\pgfpathlineto{\pgfqpoint{2.829383in}{2.856540in}}%
\pgfpathlineto{\pgfqpoint{2.834607in}{2.859609in}}%
\pgfpathlineto{\pgfqpoint{2.839960in}{2.862895in}}%
\pgfpathlineto{\pgfqpoint{2.850538in}{2.869398in}}%
\pgfpathlineto{\pgfqpoint{2.851368in}{2.869903in}}%
\pgfpathlineto{\pgfqpoint{2.861115in}{2.876252in}}%
\pgfpathlineto{\pgfqpoint{2.867338in}{2.880198in}}%
\pgfpathlineto{\pgfqpoint{2.871692in}{2.883269in}}%
\pgfpathlineto{\pgfqpoint{2.882269in}{2.890425in}}%
\pgfpathlineto{\pgfqpoint{2.882373in}{2.890492in}}%
\pgfpathlineto{\pgfqpoint{2.892846in}{2.898561in}}%
\pgfpathlineto{\pgfqpoint{2.895948in}{2.900787in}}%
\pgfpathlineto{\pgfqpoint{2.903423in}{2.908405in}}%
\pgfpathlineto{\pgfqpoint{2.906217in}{2.911081in}}%
\pgfpathlineto{\pgfqpoint{2.903423in}{2.911081in}}%
\pgfpathlineto{\pgfqpoint{2.892846in}{2.911081in}}%
\pgfpathlineto{\pgfqpoint{2.882269in}{2.911081in}}%
\pgfpathlineto{\pgfqpoint{2.871692in}{2.911081in}}%
\pgfpathlineto{\pgfqpoint{2.865618in}{2.911081in}}%
\pgfpathlineto{\pgfqpoint{2.861115in}{2.906071in}}%
\pgfpathlineto{\pgfqpoint{2.856406in}{2.900787in}}%
\pgfpathlineto{\pgfqpoint{2.850538in}{2.895758in}}%
\pgfpathlineto{\pgfqpoint{2.844444in}{2.890492in}}%
\pgfpathlineto{\pgfqpoint{2.839960in}{2.887203in}}%
\pgfpathlineto{\pgfqpoint{2.830419in}{2.880198in}}%
\pgfpathlineto{\pgfqpoint{2.829383in}{2.879514in}}%
\pgfpathlineto{\pgfqpoint{2.818806in}{2.872583in}}%
\pgfpathlineto{\pgfqpoint{2.814627in}{2.869903in}}%
\pgfpathlineto{\pgfqpoint{2.808229in}{2.866085in}}%
\pgfpathlineto{\pgfqpoint{2.797652in}{2.859998in}}%
\pgfpathlineto{\pgfqpoint{2.796948in}{2.859609in}}%
\pgfpathlineto{\pgfqpoint{2.787075in}{2.854428in}}%
\pgfpathlineto{\pgfqpoint{2.776655in}{2.849315in}}%
\pgfpathlineto{\pgfqpoint{2.776498in}{2.849240in}}%
\pgfpathlineto{\pgfqpoint{2.765921in}{2.844558in}}%
\pgfpathlineto{\pgfqpoint{2.755344in}{2.840313in}}%
\pgfpathlineto{\pgfqpoint{2.751782in}{2.839020in}}%
\pgfpathlineto{\pgfqpoint{2.744767in}{2.836548in}}%
\pgfpathlineto{\pgfqpoint{2.734190in}{2.833258in}}%
\pgfpathlineto{\pgfqpoint{2.723613in}{2.830431in}}%
\pgfpathlineto{\pgfqpoint{2.716076in}{2.828726in}}%
\pgfpathlineto{\pgfqpoint{2.713036in}{2.828056in}}%
\pgfpathlineto{\pgfqpoint{2.702459in}{2.826118in}}%
\pgfpathlineto{\pgfqpoint{2.691882in}{2.824592in}}%
\pgfpathlineto{\pgfqpoint{2.681305in}{2.823451in}}%
\pgfpathlineto{\pgfqpoint{2.670727in}{2.822665in}}%
\pgfpathlineto{\pgfqpoint{2.660150in}{2.822205in}}%
\pgfpathlineto{\pgfqpoint{2.649573in}{2.822042in}}%
\pgfpathlineto{\pgfqpoint{2.638996in}{2.822149in}}%
\pgfpathlineto{\pgfqpoint{2.628419in}{2.822521in}}%
\pgfpathlineto{\pgfqpoint{2.618218in}{2.828726in}}%
\pgfpathlineto{\pgfqpoint{2.617842in}{2.829435in}}%
\pgfpathlineto{\pgfqpoint{2.613494in}{2.839020in}}%
\pgfpathlineto{\pgfqpoint{2.609477in}{2.849315in}}%
\pgfpathlineto{\pgfqpoint{2.607265in}{2.855999in}}%
\pgfpathlineto{\pgfqpoint{2.606114in}{2.859609in}}%
\pgfpathlineto{\pgfqpoint{2.603400in}{2.869903in}}%
\pgfpathlineto{\pgfqpoint{2.601322in}{2.880198in}}%
\pgfpathlineto{\pgfqpoint{2.599888in}{2.890492in}}%
\pgfpathlineto{\pgfqpoint{2.599110in}{2.900787in}}%
\pgfpathlineto{\pgfqpoint{2.599007in}{2.911081in}}%
\pgfpathlineto{\pgfqpoint{2.596688in}{2.911081in}}%
\pgfpathlineto{\pgfqpoint{2.586111in}{2.911081in}}%
\pgfpathlineto{\pgfqpoint{2.582026in}{2.911081in}}%
\pgfpathlineto{\pgfqpoint{2.581797in}{2.900787in}}%
\pgfpathlineto{\pgfqpoint{2.582199in}{2.890492in}}%
\pgfpathlineto{\pgfqpoint{2.583213in}{2.880198in}}%
\pgfpathlineto{\pgfqpoint{2.584826in}{2.869903in}}%
\pgfpathlineto{\pgfqpoint{2.586111in}{2.863827in}}%
\pgfpathlineto{\pgfqpoint{2.587021in}{2.859609in}}%
\pgfpathlineto{\pgfqpoint{2.589790in}{2.849315in}}%
\pgfpathlineto{\pgfqpoint{2.593144in}{2.839020in}}%
\pgfpathlineto{\pgfqpoint{2.596688in}{2.829770in}}%
\pgfpathlineto{\pgfqpoint{2.597102in}{2.828726in}}%
\pgfpathlineto{\pgfqpoint{2.601769in}{2.818431in}}%
\pgfpathlineto{\pgfqpoint{2.607265in}{2.812151in}}%
\pgfpathlineto{\pgfqpoint{2.617842in}{2.808974in}}%
\pgfpathlineto{\pgfqpoint{2.627525in}{2.808137in}}%
\pgfpathclose%
\pgfusepath{fill}%
\end{pgfscope}%
\begin{pgfscope}%
\pgfpathrectangle{\pgfqpoint{1.856795in}{1.819814in}}{\pgfqpoint{1.194205in}{1.163386in}}%
\pgfusepath{clip}%
\pgfsetbuttcap%
\pgfsetroundjoin%
\definecolor{currentfill}{rgb}{0.432143,0.121800,0.339663}%
\pgfsetfillcolor{currentfill}%
\pgfsetlinewidth{0.000000pt}%
\definecolor{currentstroke}{rgb}{0.000000,0.000000,0.000000}%
\pgfsetstrokecolor{currentstroke}%
\pgfsetdash{}{0pt}%
\pgfpathmoveto{\pgfqpoint{2.004373in}{1.891932in}}%
\pgfpathlineto{\pgfqpoint{2.014010in}{1.891932in}}%
\pgfpathlineto{\pgfqpoint{2.014950in}{1.894173in}}%
\pgfpathlineto{\pgfqpoint{2.018145in}{1.902227in}}%
\pgfpathlineto{\pgfqpoint{2.021075in}{1.912521in}}%
\pgfpathlineto{\pgfqpoint{2.022950in}{1.922816in}}%
\pgfpathlineto{\pgfqpoint{2.023874in}{1.933110in}}%
\pgfpathlineto{\pgfqpoint{2.023929in}{1.943405in}}%
\pgfpathlineto{\pgfqpoint{2.023164in}{1.953699in}}%
\pgfpathlineto{\pgfqpoint{2.021571in}{1.963993in}}%
\pgfpathlineto{\pgfqpoint{2.019042in}{1.974288in}}%
\pgfpathlineto{\pgfqpoint{2.015333in}{1.984582in}}%
\pgfpathlineto{\pgfqpoint{2.014950in}{1.985380in}}%
\pgfpathlineto{\pgfqpoint{2.010218in}{1.994877in}}%
\pgfpathlineto{\pgfqpoint{2.004373in}{2.004908in}}%
\pgfpathlineto{\pgfqpoint{2.004218in}{2.005171in}}%
\pgfpathlineto{\pgfqpoint{1.997798in}{2.015466in}}%
\pgfpathlineto{\pgfqpoint{1.993796in}{2.021857in}}%
\pgfpathlineto{\pgfqpoint{1.991327in}{2.025760in}}%
\pgfpathlineto{\pgfqpoint{1.984937in}{2.036054in}}%
\pgfpathlineto{\pgfqpoint{1.983218in}{2.038913in}}%
\pgfpathlineto{\pgfqpoint{1.978680in}{2.046349in}}%
\pgfpathlineto{\pgfqpoint{1.972690in}{2.056643in}}%
\pgfpathlineto{\pgfqpoint{1.972641in}{2.056731in}}%
\pgfpathlineto{\pgfqpoint{1.966893in}{2.066938in}}%
\pgfpathlineto{\pgfqpoint{1.962064in}{2.076132in}}%
\pgfpathlineto{\pgfqpoint{1.961475in}{2.077232in}}%
\pgfpathlineto{\pgfqpoint{1.956345in}{2.087527in}}%
\pgfpathlineto{\pgfqpoint{1.951679in}{2.097821in}}%
\pgfpathlineto{\pgfqpoint{1.951487in}{2.098284in}}%
\pgfpathlineto{\pgfqpoint{1.947344in}{2.108116in}}%
\pgfpathlineto{\pgfqpoint{1.943497in}{2.118410in}}%
\pgfpathlineto{\pgfqpoint{1.940910in}{2.126278in}}%
\pgfpathlineto{\pgfqpoint{1.940098in}{2.128704in}}%
\pgfpathlineto{\pgfqpoint{1.937076in}{2.138999in}}%
\pgfpathlineto{\pgfqpoint{1.934502in}{2.149293in}}%
\pgfpathlineto{\pgfqpoint{1.932331in}{2.159588in}}%
\pgfpathlineto{\pgfqpoint{1.930518in}{2.169882in}}%
\pgfpathlineto{\pgfqpoint{1.930333in}{2.171116in}}%
\pgfpathlineto{\pgfqpoint{1.930333in}{2.169882in}}%
\pgfpathlineto{\pgfqpoint{1.930333in}{2.159588in}}%
\pgfpathlineto{\pgfqpoint{1.930333in}{2.149293in}}%
\pgfpathlineto{\pgfqpoint{1.930333in}{2.138999in}}%
\pgfpathlineto{\pgfqpoint{1.930333in}{2.128704in}}%
\pgfpathlineto{\pgfqpoint{1.930333in}{2.118410in}}%
\pgfpathlineto{\pgfqpoint{1.930333in}{2.108565in}}%
\pgfpathlineto{\pgfqpoint{1.930496in}{2.108116in}}%
\pgfpathlineto{\pgfqpoint{1.934684in}{2.097821in}}%
\pgfpathlineto{\pgfqpoint{1.939439in}{2.087527in}}%
\pgfpathlineto{\pgfqpoint{1.940910in}{2.084643in}}%
\pgfpathlineto{\pgfqpoint{1.944569in}{2.077232in}}%
\pgfpathlineto{\pgfqpoint{1.950141in}{2.066938in}}%
\pgfpathlineto{\pgfqpoint{1.951487in}{2.064625in}}%
\pgfpathlineto{\pgfqpoint{1.955989in}{2.056643in}}%
\pgfpathlineto{\pgfqpoint{1.962064in}{2.046576in}}%
\pgfpathlineto{\pgfqpoint{1.962198in}{2.046349in}}%
\pgfpathlineto{\pgfqpoint{1.968510in}{2.036054in}}%
\pgfpathlineto{\pgfqpoint{1.972641in}{2.029581in}}%
\pgfpathlineto{\pgfqpoint{1.975025in}{2.025760in}}%
\pgfpathlineto{\pgfqpoint{1.981567in}{2.015466in}}%
\pgfpathlineto{\pgfqpoint{1.983218in}{2.012851in}}%
\pgfpathlineto{\pgfqpoint{1.987981in}{2.005171in}}%
\pgfpathlineto{\pgfqpoint{1.993796in}{1.995399in}}%
\pgfpathlineto{\pgfqpoint{1.994101in}{1.994877in}}%
\pgfpathlineto{\pgfqpoint{1.999417in}{1.984582in}}%
\pgfpathlineto{\pgfqpoint{2.003655in}{1.974288in}}%
\pgfpathlineto{\pgfqpoint{2.004373in}{1.971824in}}%
\pgfpathlineto{\pgfqpoint{2.006533in}{1.963993in}}%
\pgfpathlineto{\pgfqpoint{2.008241in}{1.953699in}}%
\pgfpathlineto{\pgfqpoint{2.008937in}{1.943405in}}%
\pgfpathlineto{\pgfqpoint{2.008670in}{1.933110in}}%
\pgfpathlineto{\pgfqpoint{2.007419in}{1.922816in}}%
\pgfpathlineto{\pgfqpoint{2.005114in}{1.912521in}}%
\pgfpathlineto{\pgfqpoint{2.004373in}{1.910268in}}%
\pgfpathlineto{\pgfqpoint{2.001551in}{1.902227in}}%
\pgfpathlineto{\pgfqpoint{1.996627in}{1.891932in}}%
\pgfpathclose%
\pgfusepath{fill}%
\end{pgfscope}%
\begin{pgfscope}%
\pgfpathrectangle{\pgfqpoint{1.856795in}{1.819814in}}{\pgfqpoint{1.194205in}{1.163386in}}%
\pgfusepath{clip}%
\pgfsetbuttcap%
\pgfsetroundjoin%
\definecolor{currentfill}{rgb}{0.432143,0.121800,0.339663}%
\pgfsetfillcolor{currentfill}%
\pgfsetlinewidth{0.000000pt}%
\definecolor{currentstroke}{rgb}{0.000000,0.000000,0.000000}%
\pgfsetstrokecolor{currentstroke}%
\pgfsetdash{}{0pt}%
\pgfpathmoveto{\pgfqpoint{2.596688in}{2.797604in}}%
\pgfpathlineto{\pgfqpoint{2.607265in}{2.795868in}}%
\pgfpathlineto{\pgfqpoint{2.617842in}{2.794518in}}%
\pgfpathlineto{\pgfqpoint{2.628419in}{2.793369in}}%
\pgfpathlineto{\pgfqpoint{2.638996in}{2.792404in}}%
\pgfpathlineto{\pgfqpoint{2.649573in}{2.791635in}}%
\pgfpathlineto{\pgfqpoint{2.660150in}{2.791086in}}%
\pgfpathlineto{\pgfqpoint{2.670727in}{2.790788in}}%
\pgfpathlineto{\pgfqpoint{2.681305in}{2.790775in}}%
\pgfpathlineto{\pgfqpoint{2.691882in}{2.791083in}}%
\pgfpathlineto{\pgfqpoint{2.702459in}{2.791754in}}%
\pgfpathlineto{\pgfqpoint{2.713036in}{2.792826in}}%
\pgfpathlineto{\pgfqpoint{2.723613in}{2.794341in}}%
\pgfpathlineto{\pgfqpoint{2.734190in}{2.796336in}}%
\pgfpathlineto{\pgfqpoint{2.740613in}{2.797842in}}%
\pgfpathlineto{\pgfqpoint{2.744767in}{2.798820in}}%
\pgfpathlineto{\pgfqpoint{2.755344in}{2.801791in}}%
\pgfpathlineto{\pgfqpoint{2.765921in}{2.805292in}}%
\pgfpathlineto{\pgfqpoint{2.773427in}{2.808137in}}%
\pgfpathlineto{\pgfqpoint{2.776498in}{2.809302in}}%
\pgfpathlineto{\pgfqpoint{2.787075in}{2.813758in}}%
\pgfpathlineto{\pgfqpoint{2.797132in}{2.818431in}}%
\pgfpathlineto{\pgfqpoint{2.797652in}{2.818674in}}%
\pgfpathlineto{\pgfqpoint{2.808229in}{2.823942in}}%
\pgfpathlineto{\pgfqpoint{2.817263in}{2.828726in}}%
\pgfpathlineto{\pgfqpoint{2.818806in}{2.829546in}}%
\pgfpathlineto{\pgfqpoint{2.829383in}{2.835402in}}%
\pgfpathlineto{\pgfqpoint{2.835729in}{2.839020in}}%
\pgfpathlineto{\pgfqpoint{2.839960in}{2.841453in}}%
\pgfpathlineto{\pgfqpoint{2.850538in}{2.847622in}}%
\pgfpathlineto{\pgfqpoint{2.853424in}{2.849315in}}%
\pgfpathlineto{\pgfqpoint{2.861115in}{2.853895in}}%
\pgfpathlineto{\pgfqpoint{2.870855in}{2.859609in}}%
\pgfpathlineto{\pgfqpoint{2.871692in}{2.860116in}}%
\pgfpathlineto{\pgfqpoint{2.882269in}{2.866402in}}%
\pgfpathlineto{\pgfqpoint{2.888411in}{2.869903in}}%
\pgfpathlineto{\pgfqpoint{2.892846in}{2.872565in}}%
\pgfpathlineto{\pgfqpoint{2.903423in}{2.878572in}}%
\pgfpathlineto{\pgfqpoint{2.906462in}{2.880198in}}%
\pgfpathlineto{\pgfqpoint{2.914000in}{2.884580in}}%
\pgfpathlineto{\pgfqpoint{2.924577in}{2.890096in}}%
\pgfpathlineto{\pgfqpoint{2.925397in}{2.890492in}}%
\pgfpathlineto{\pgfqpoint{2.935154in}{2.895960in}}%
\pgfpathlineto{\pgfqpoint{2.945709in}{2.900787in}}%
\pgfpathlineto{\pgfqpoint{2.945731in}{2.900802in}}%
\pgfpathlineto{\pgfqpoint{2.956308in}{2.907228in}}%
\pgfpathlineto{\pgfqpoint{2.965890in}{2.911081in}}%
\pgfpathlineto{\pgfqpoint{2.956308in}{2.911081in}}%
\pgfpathlineto{\pgfqpoint{2.945731in}{2.911081in}}%
\pgfpathlineto{\pgfqpoint{2.935154in}{2.911081in}}%
\pgfpathlineto{\pgfqpoint{2.924577in}{2.911081in}}%
\pgfpathlineto{\pgfqpoint{2.914000in}{2.911081in}}%
\pgfpathlineto{\pgfqpoint{2.906217in}{2.911081in}}%
\pgfpathlineto{\pgfqpoint{2.903423in}{2.908405in}}%
\pgfpathlineto{\pgfqpoint{2.895948in}{2.900787in}}%
\pgfpathlineto{\pgfqpoint{2.892846in}{2.898561in}}%
\pgfpathlineto{\pgfqpoint{2.882373in}{2.890492in}}%
\pgfpathlineto{\pgfqpoint{2.882269in}{2.890425in}}%
\pgfpathlineto{\pgfqpoint{2.871692in}{2.883269in}}%
\pgfpathlineto{\pgfqpoint{2.867338in}{2.880198in}}%
\pgfpathlineto{\pgfqpoint{2.861115in}{2.876252in}}%
\pgfpathlineto{\pgfqpoint{2.851368in}{2.869903in}}%
\pgfpathlineto{\pgfqpoint{2.850538in}{2.869398in}}%
\pgfpathlineto{\pgfqpoint{2.839960in}{2.862895in}}%
\pgfpathlineto{\pgfqpoint{2.834607in}{2.859609in}}%
\pgfpathlineto{\pgfqpoint{2.829383in}{2.856540in}}%
\pgfpathlineto{\pgfqpoint{2.818806in}{2.850429in}}%
\pgfpathlineto{\pgfqpoint{2.816823in}{2.849315in}}%
\pgfpathlineto{\pgfqpoint{2.808229in}{2.844628in}}%
\pgfpathlineto{\pgfqpoint{2.797652in}{2.839123in}}%
\pgfpathlineto{\pgfqpoint{2.797445in}{2.839020in}}%
\pgfpathlineto{\pgfqpoint{2.787075in}{2.834006in}}%
\pgfpathlineto{\pgfqpoint{2.776498in}{2.829283in}}%
\pgfpathlineto{\pgfqpoint{2.775144in}{2.828726in}}%
\pgfpathlineto{\pgfqpoint{2.765921in}{2.824994in}}%
\pgfpathlineto{\pgfqpoint{2.755344in}{2.821172in}}%
\pgfpathlineto{\pgfqpoint{2.746702in}{2.818431in}}%
\pgfpathlineto{\pgfqpoint{2.744767in}{2.817826in}}%
\pgfpathlineto{\pgfqpoint{2.734190in}{2.814947in}}%
\pgfpathlineto{\pgfqpoint{2.723613in}{2.812548in}}%
\pgfpathlineto{\pgfqpoint{2.713036in}{2.810607in}}%
\pgfpathlineto{\pgfqpoint{2.702459in}{2.809101in}}%
\pgfpathlineto{\pgfqpoint{2.693252in}{2.808137in}}%
\pgfpathlineto{\pgfqpoint{2.691882in}{2.807996in}}%
\pgfpathlineto{\pgfqpoint{2.681305in}{2.807256in}}%
\pgfpathlineto{\pgfqpoint{2.670727in}{2.806858in}}%
\pgfpathlineto{\pgfqpoint{2.660150in}{2.806768in}}%
\pgfpathlineto{\pgfqpoint{2.649573in}{2.806955in}}%
\pgfpathlineto{\pgfqpoint{2.638996in}{2.807391in}}%
\pgfpathlineto{\pgfqpoint{2.628419in}{2.808059in}}%
\pgfpathlineto{\pgfqpoint{2.627525in}{2.808137in}}%
\pgfpathlineto{\pgfqpoint{2.617842in}{2.808974in}}%
\pgfpathlineto{\pgfqpoint{2.607265in}{2.812151in}}%
\pgfpathlineto{\pgfqpoint{2.601769in}{2.818431in}}%
\pgfpathlineto{\pgfqpoint{2.597102in}{2.828726in}}%
\pgfpathlineto{\pgfqpoint{2.596688in}{2.829770in}}%
\pgfpathlineto{\pgfqpoint{2.593144in}{2.839020in}}%
\pgfpathlineto{\pgfqpoint{2.589790in}{2.849315in}}%
\pgfpathlineto{\pgfqpoint{2.587021in}{2.859609in}}%
\pgfpathlineto{\pgfqpoint{2.586111in}{2.863827in}}%
\pgfpathlineto{\pgfqpoint{2.584826in}{2.869903in}}%
\pgfpathlineto{\pgfqpoint{2.583213in}{2.880198in}}%
\pgfpathlineto{\pgfqpoint{2.582199in}{2.890492in}}%
\pgfpathlineto{\pgfqpoint{2.581797in}{2.900787in}}%
\pgfpathlineto{\pgfqpoint{2.582026in}{2.911081in}}%
\pgfpathlineto{\pgfqpoint{2.575534in}{2.911081in}}%
\pgfpathlineto{\pgfqpoint{2.564957in}{2.911081in}}%
\pgfpathlineto{\pgfqpoint{2.564797in}{2.911081in}}%
\pgfpathlineto{\pgfqpoint{2.564209in}{2.900787in}}%
\pgfpathlineto{\pgfqpoint{2.564224in}{2.890492in}}%
\pgfpathlineto{\pgfqpoint{2.564825in}{2.880198in}}%
\pgfpathlineto{\pgfqpoint{2.564957in}{2.879010in}}%
\pgfpathlineto{\pgfqpoint{2.565977in}{2.869903in}}%
\pgfpathlineto{\pgfqpoint{2.567677in}{2.859609in}}%
\pgfpathlineto{\pgfqpoint{2.569921in}{2.849315in}}%
\pgfpathlineto{\pgfqpoint{2.572707in}{2.839020in}}%
\pgfpathlineto{\pgfqpoint{2.575534in}{2.830249in}}%
\pgfpathlineto{\pgfqpoint{2.576035in}{2.828726in}}%
\pgfpathlineto{\pgfqpoint{2.579900in}{2.818431in}}%
\pgfpathlineto{\pgfqpoint{2.584487in}{2.808137in}}%
\pgfpathlineto{\pgfqpoint{2.586111in}{2.805564in}}%
\pgfpathlineto{\pgfqpoint{2.595939in}{2.797842in}}%
\pgfpathclose%
\pgfusepath{fill}%
\end{pgfscope}%
\begin{pgfscope}%
\pgfpathrectangle{\pgfqpoint{1.856795in}{1.819814in}}{\pgfqpoint{1.194205in}{1.163386in}}%
\pgfusepath{clip}%
\pgfsetbuttcap%
\pgfsetroundjoin%
\definecolor{currentfill}{rgb}{0.504033,0.119071,0.352952}%
\pgfsetfillcolor{currentfill}%
\pgfsetlinewidth{0.000000pt}%
\definecolor{currentstroke}{rgb}{0.000000,0.000000,0.000000}%
\pgfsetstrokecolor{currentstroke}%
\pgfsetdash{}{0pt}%
\pgfpathmoveto{\pgfqpoint{2.014950in}{1.891932in}}%
\pgfpathlineto{\pgfqpoint{2.025527in}{1.891932in}}%
\pgfpathlineto{\pgfqpoint{2.030695in}{1.891932in}}%
\pgfpathlineto{\pgfqpoint{2.034242in}{1.902227in}}%
\pgfpathlineto{\pgfqpoint{2.036104in}{1.910007in}}%
\pgfpathlineto{\pgfqpoint{2.036689in}{1.912521in}}%
\pgfpathlineto{\pgfqpoint{2.038155in}{1.922816in}}%
\pgfpathlineto{\pgfqpoint{2.038742in}{1.933110in}}%
\pgfpathlineto{\pgfqpoint{2.038537in}{1.943405in}}%
\pgfpathlineto{\pgfqpoint{2.037613in}{1.953699in}}%
\pgfpathlineto{\pgfqpoint{2.036104in}{1.963460in}}%
\pgfpathlineto{\pgfqpoint{2.036021in}{1.963993in}}%
\pgfpathlineto{\pgfqpoint{2.033716in}{1.974288in}}%
\pgfpathlineto{\pgfqpoint{2.030523in}{1.984582in}}%
\pgfpathlineto{\pgfqpoint{2.025973in}{1.994877in}}%
\pgfpathlineto{\pgfqpoint{2.025527in}{1.995658in}}%
\pgfpathlineto{\pgfqpoint{2.020020in}{2.005171in}}%
\pgfpathlineto{\pgfqpoint{2.014950in}{2.013263in}}%
\pgfpathlineto{\pgfqpoint{2.013573in}{2.015466in}}%
\pgfpathlineto{\pgfqpoint{2.007093in}{2.025760in}}%
\pgfpathlineto{\pgfqpoint{2.004373in}{2.030175in}}%
\pgfpathlineto{\pgfqpoint{2.000731in}{2.036054in}}%
\pgfpathlineto{\pgfqpoint{1.994568in}{2.046349in}}%
\pgfpathlineto{\pgfqpoint{1.993796in}{2.047691in}}%
\pgfpathlineto{\pgfqpoint{1.988602in}{2.056643in}}%
\pgfpathlineto{\pgfqpoint{1.983218in}{2.066405in}}%
\pgfpathlineto{\pgfqpoint{1.982922in}{2.066938in}}%
\pgfpathlineto{\pgfqpoint{1.977497in}{2.077232in}}%
\pgfpathlineto{\pgfqpoint{1.972641in}{2.087100in}}%
\pgfpathlineto{\pgfqpoint{1.972430in}{2.087527in}}%
\pgfpathlineto{\pgfqpoint{1.967670in}{2.097821in}}%
\pgfpathlineto{\pgfqpoint{1.963323in}{2.108116in}}%
\pgfpathlineto{\pgfqpoint{1.962064in}{2.111374in}}%
\pgfpathlineto{\pgfqpoint{1.959334in}{2.118410in}}%
\pgfpathlineto{\pgfqpoint{1.955750in}{2.128704in}}%
\pgfpathlineto{\pgfqpoint{1.952579in}{2.138999in}}%
\pgfpathlineto{\pgfqpoint{1.951487in}{2.142991in}}%
\pgfpathlineto{\pgfqpoint{1.949758in}{2.149293in}}%
\pgfpathlineto{\pgfqpoint{1.947284in}{2.159588in}}%
\pgfpathlineto{\pgfqpoint{1.945147in}{2.169882in}}%
\pgfpathlineto{\pgfqpoint{1.943306in}{2.180177in}}%
\pgfpathlineto{\pgfqpoint{1.941716in}{2.190471in}}%
\pgfpathlineto{\pgfqpoint{1.940910in}{2.196399in}}%
\pgfpathlineto{\pgfqpoint{1.940313in}{2.200765in}}%
\pgfpathlineto{\pgfqpoint{1.939045in}{2.211060in}}%
\pgfpathlineto{\pgfqpoint{1.937897in}{2.221354in}}%
\pgfpathlineto{\pgfqpoint{1.936825in}{2.231649in}}%
\pgfpathlineto{\pgfqpoint{1.935793in}{2.241943in}}%
\pgfpathlineto{\pgfqpoint{1.934764in}{2.252238in}}%
\pgfpathlineto{\pgfqpoint{1.933708in}{2.262532in}}%
\pgfpathlineto{\pgfqpoint{1.932601in}{2.272826in}}%
\pgfpathlineto{\pgfqpoint{1.931428in}{2.283121in}}%
\pgfpathlineto{\pgfqpoint{1.930333in}{2.292258in}}%
\pgfpathlineto{\pgfqpoint{1.930333in}{2.283121in}}%
\pgfpathlineto{\pgfqpoint{1.930333in}{2.272826in}}%
\pgfpathlineto{\pgfqpoint{1.930333in}{2.262532in}}%
\pgfpathlineto{\pgfqpoint{1.930333in}{2.252238in}}%
\pgfpathlineto{\pgfqpoint{1.930333in}{2.241943in}}%
\pgfpathlineto{\pgfqpoint{1.930333in}{2.231649in}}%
\pgfpathlineto{\pgfqpoint{1.930333in}{2.221354in}}%
\pgfpathlineto{\pgfqpoint{1.930333in}{2.211060in}}%
\pgfpathlineto{\pgfqpoint{1.930333in}{2.200765in}}%
\pgfpathlineto{\pgfqpoint{1.930333in}{2.190471in}}%
\pgfpathlineto{\pgfqpoint{1.930333in}{2.180177in}}%
\pgfpathlineto{\pgfqpoint{1.930333in}{2.171116in}}%
\pgfpathlineto{\pgfqpoint{1.930518in}{2.169882in}}%
\pgfpathlineto{\pgfqpoint{1.932331in}{2.159588in}}%
\pgfpathlineto{\pgfqpoint{1.934502in}{2.149293in}}%
\pgfpathlineto{\pgfqpoint{1.937076in}{2.138999in}}%
\pgfpathlineto{\pgfqpoint{1.940098in}{2.128704in}}%
\pgfpathlineto{\pgfqpoint{1.940910in}{2.126278in}}%
\pgfpathlineto{\pgfqpoint{1.943497in}{2.118410in}}%
\pgfpathlineto{\pgfqpoint{1.947344in}{2.108116in}}%
\pgfpathlineto{\pgfqpoint{1.951487in}{2.098284in}}%
\pgfpathlineto{\pgfqpoint{1.951679in}{2.097821in}}%
\pgfpathlineto{\pgfqpoint{1.956345in}{2.087527in}}%
\pgfpathlineto{\pgfqpoint{1.961475in}{2.077232in}}%
\pgfpathlineto{\pgfqpoint{1.962064in}{2.076132in}}%
\pgfpathlineto{\pgfqpoint{1.966893in}{2.066938in}}%
\pgfpathlineto{\pgfqpoint{1.972641in}{2.056731in}}%
\pgfpathlineto{\pgfqpoint{1.972690in}{2.056643in}}%
\pgfpathlineto{\pgfqpoint{1.978680in}{2.046349in}}%
\pgfpathlineto{\pgfqpoint{1.983218in}{2.038913in}}%
\pgfpathlineto{\pgfqpoint{1.984937in}{2.036054in}}%
\pgfpathlineto{\pgfqpoint{1.991327in}{2.025760in}}%
\pgfpathlineto{\pgfqpoint{1.993796in}{2.021857in}}%
\pgfpathlineto{\pgfqpoint{1.997798in}{2.015466in}}%
\pgfpathlineto{\pgfqpoint{2.004218in}{2.005171in}}%
\pgfpathlineto{\pgfqpoint{2.004373in}{2.004908in}}%
\pgfpathlineto{\pgfqpoint{2.010218in}{1.994877in}}%
\pgfpathlineto{\pgfqpoint{2.014950in}{1.985380in}}%
\pgfpathlineto{\pgfqpoint{2.015333in}{1.984582in}}%
\pgfpathlineto{\pgfqpoint{2.019042in}{1.974288in}}%
\pgfpathlineto{\pgfqpoint{2.021571in}{1.963993in}}%
\pgfpathlineto{\pgfqpoint{2.023164in}{1.953699in}}%
\pgfpathlineto{\pgfqpoint{2.023929in}{1.943405in}}%
\pgfpathlineto{\pgfqpoint{2.023874in}{1.933110in}}%
\pgfpathlineto{\pgfqpoint{2.022950in}{1.922816in}}%
\pgfpathlineto{\pgfqpoint{2.021075in}{1.912521in}}%
\pgfpathlineto{\pgfqpoint{2.018145in}{1.902227in}}%
\pgfpathlineto{\pgfqpoint{2.014950in}{1.894173in}}%
\pgfpathlineto{\pgfqpoint{2.014010in}{1.891932in}}%
\pgfpathclose%
\pgfusepath{fill}%
\end{pgfscope}%
\begin{pgfscope}%
\pgfpathrectangle{\pgfqpoint{1.856795in}{1.819814in}}{\pgfqpoint{1.194205in}{1.163386in}}%
\pgfusepath{clip}%
\pgfsetbuttcap%
\pgfsetroundjoin%
\definecolor{currentfill}{rgb}{0.504033,0.119071,0.352952}%
\pgfsetfillcolor{currentfill}%
\pgfsetlinewidth{0.000000pt}%
\definecolor{currentstroke}{rgb}{0.000000,0.000000,0.000000}%
\pgfsetstrokecolor{currentstroke}%
\pgfsetdash{}{0pt}%
\pgfpathmoveto{\pgfqpoint{2.638996in}{2.776937in}}%
\pgfpathlineto{\pgfqpoint{2.649573in}{2.775786in}}%
\pgfpathlineto{\pgfqpoint{2.660150in}{2.774830in}}%
\pgfpathlineto{\pgfqpoint{2.670727in}{2.774103in}}%
\pgfpathlineto{\pgfqpoint{2.681305in}{2.773641in}}%
\pgfpathlineto{\pgfqpoint{2.691882in}{2.773487in}}%
\pgfpathlineto{\pgfqpoint{2.702459in}{2.773688in}}%
\pgfpathlineto{\pgfqpoint{2.713036in}{2.774292in}}%
\pgfpathlineto{\pgfqpoint{2.723613in}{2.775351in}}%
\pgfpathlineto{\pgfqpoint{2.734190in}{2.776918in}}%
\pgfpathlineto{\pgfqpoint{2.735900in}{2.777254in}}%
\pgfpathlineto{\pgfqpoint{2.744767in}{2.778974in}}%
\pgfpathlineto{\pgfqpoint{2.755344in}{2.781590in}}%
\pgfpathlineto{\pgfqpoint{2.765921in}{2.784809in}}%
\pgfpathlineto{\pgfqpoint{2.773512in}{2.787548in}}%
\pgfpathlineto{\pgfqpoint{2.776498in}{2.788612in}}%
\pgfpathlineto{\pgfqpoint{2.787075in}{2.792911in}}%
\pgfpathlineto{\pgfqpoint{2.797652in}{2.797779in}}%
\pgfpathlineto{\pgfqpoint{2.797778in}{2.797842in}}%
\pgfpathlineto{\pgfqpoint{2.808229in}{2.803015in}}%
\pgfpathlineto{\pgfqpoint{2.817780in}{2.808137in}}%
\pgfpathlineto{\pgfqpoint{2.818806in}{2.808679in}}%
\pgfpathlineto{\pgfqpoint{2.829383in}{2.814588in}}%
\pgfpathlineto{\pgfqpoint{2.835958in}{2.818431in}}%
\pgfpathlineto{\pgfqpoint{2.839960in}{2.820735in}}%
\pgfpathlineto{\pgfqpoint{2.850538in}{2.827004in}}%
\pgfpathlineto{\pgfqpoint{2.853382in}{2.828726in}}%
\pgfpathlineto{\pgfqpoint{2.861115in}{2.833341in}}%
\pgfpathlineto{\pgfqpoint{2.870610in}{2.839020in}}%
\pgfpathlineto{\pgfqpoint{2.871692in}{2.839663in}}%
\pgfpathlineto{\pgfqpoint{2.882269in}{2.845948in}}%
\pgfpathlineto{\pgfqpoint{2.888028in}{2.849315in}}%
\pgfpathlineto{\pgfqpoint{2.892846in}{2.852126in}}%
\pgfpathlineto{\pgfqpoint{2.903423in}{2.858140in}}%
\pgfpathlineto{\pgfqpoint{2.906053in}{2.859609in}}%
\pgfpathlineto{\pgfqpoint{2.914000in}{2.864054in}}%
\pgfpathlineto{\pgfqpoint{2.924577in}{2.869681in}}%
\pgfpathlineto{\pgfqpoint{2.924996in}{2.869903in}}%
\pgfpathlineto{\pgfqpoint{2.935154in}{2.875308in}}%
\pgfpathlineto{\pgfqpoint{2.944862in}{2.880198in}}%
\pgfpathlineto{\pgfqpoint{2.945731in}{2.880642in}}%
\pgfpathlineto{\pgfqpoint{2.956308in}{2.886130in}}%
\pgfpathlineto{\pgfqpoint{2.964505in}{2.890492in}}%
\pgfpathlineto{\pgfqpoint{2.966885in}{2.891725in}}%
\pgfpathlineto{\pgfqpoint{2.977462in}{2.898326in}}%
\pgfpathlineto{\pgfqpoint{2.977462in}{2.900787in}}%
\pgfpathlineto{\pgfqpoint{2.977462in}{2.911081in}}%
\pgfpathlineto{\pgfqpoint{2.966885in}{2.911081in}}%
\pgfpathlineto{\pgfqpoint{2.965890in}{2.911081in}}%
\pgfpathlineto{\pgfqpoint{2.956308in}{2.907228in}}%
\pgfpathlineto{\pgfqpoint{2.945731in}{2.900802in}}%
\pgfpathlineto{\pgfqpoint{2.945709in}{2.900787in}}%
\pgfpathlineto{\pgfqpoint{2.935154in}{2.895960in}}%
\pgfpathlineto{\pgfqpoint{2.925397in}{2.890492in}}%
\pgfpathlineto{\pgfqpoint{2.924577in}{2.890096in}}%
\pgfpathlineto{\pgfqpoint{2.914000in}{2.884580in}}%
\pgfpathlineto{\pgfqpoint{2.906462in}{2.880198in}}%
\pgfpathlineto{\pgfqpoint{2.903423in}{2.878572in}}%
\pgfpathlineto{\pgfqpoint{2.892846in}{2.872565in}}%
\pgfpathlineto{\pgfqpoint{2.888411in}{2.869903in}}%
\pgfpathlineto{\pgfqpoint{2.882269in}{2.866402in}}%
\pgfpathlineto{\pgfqpoint{2.871692in}{2.860116in}}%
\pgfpathlineto{\pgfqpoint{2.870855in}{2.859609in}}%
\pgfpathlineto{\pgfqpoint{2.861115in}{2.853895in}}%
\pgfpathlineto{\pgfqpoint{2.853424in}{2.849315in}}%
\pgfpathlineto{\pgfqpoint{2.850538in}{2.847622in}}%
\pgfpathlineto{\pgfqpoint{2.839960in}{2.841453in}}%
\pgfpathlineto{\pgfqpoint{2.835729in}{2.839020in}}%
\pgfpathlineto{\pgfqpoint{2.829383in}{2.835402in}}%
\pgfpathlineto{\pgfqpoint{2.818806in}{2.829546in}}%
\pgfpathlineto{\pgfqpoint{2.817263in}{2.828726in}}%
\pgfpathlineto{\pgfqpoint{2.808229in}{2.823942in}}%
\pgfpathlineto{\pgfqpoint{2.797652in}{2.818674in}}%
\pgfpathlineto{\pgfqpoint{2.797132in}{2.818431in}}%
\pgfpathlineto{\pgfqpoint{2.787075in}{2.813758in}}%
\pgfpathlineto{\pgfqpoint{2.776498in}{2.809302in}}%
\pgfpathlineto{\pgfqpoint{2.773427in}{2.808137in}}%
\pgfpathlineto{\pgfqpoint{2.765921in}{2.805292in}}%
\pgfpathlineto{\pgfqpoint{2.755344in}{2.801791in}}%
\pgfpathlineto{\pgfqpoint{2.744767in}{2.798820in}}%
\pgfpathlineto{\pgfqpoint{2.740613in}{2.797842in}}%
\pgfpathlineto{\pgfqpoint{2.734190in}{2.796336in}}%
\pgfpathlineto{\pgfqpoint{2.723613in}{2.794341in}}%
\pgfpathlineto{\pgfqpoint{2.713036in}{2.792826in}}%
\pgfpathlineto{\pgfqpoint{2.702459in}{2.791754in}}%
\pgfpathlineto{\pgfqpoint{2.691882in}{2.791083in}}%
\pgfpathlineto{\pgfqpoint{2.681305in}{2.790775in}}%
\pgfpathlineto{\pgfqpoint{2.670727in}{2.790788in}}%
\pgfpathlineto{\pgfqpoint{2.660150in}{2.791086in}}%
\pgfpathlineto{\pgfqpoint{2.649573in}{2.791635in}}%
\pgfpathlineto{\pgfqpoint{2.638996in}{2.792404in}}%
\pgfpathlineto{\pgfqpoint{2.628419in}{2.793369in}}%
\pgfpathlineto{\pgfqpoint{2.617842in}{2.794518in}}%
\pgfpathlineto{\pgfqpoint{2.607265in}{2.795868in}}%
\pgfpathlineto{\pgfqpoint{2.596688in}{2.797604in}}%
\pgfpathlineto{\pgfqpoint{2.595939in}{2.797842in}}%
\pgfpathlineto{\pgfqpoint{2.586111in}{2.805564in}}%
\pgfpathlineto{\pgfqpoint{2.584487in}{2.808137in}}%
\pgfpathlineto{\pgfqpoint{2.579900in}{2.818431in}}%
\pgfpathlineto{\pgfqpoint{2.576035in}{2.828726in}}%
\pgfpathlineto{\pgfqpoint{2.575534in}{2.830249in}}%
\pgfpathlineto{\pgfqpoint{2.572707in}{2.839020in}}%
\pgfpathlineto{\pgfqpoint{2.569921in}{2.849315in}}%
\pgfpathlineto{\pgfqpoint{2.567677in}{2.859609in}}%
\pgfpathlineto{\pgfqpoint{2.565977in}{2.869903in}}%
\pgfpathlineto{\pgfqpoint{2.564957in}{2.879010in}}%
\pgfpathlineto{\pgfqpoint{2.564825in}{2.880198in}}%
\pgfpathlineto{\pgfqpoint{2.564224in}{2.890492in}}%
\pgfpathlineto{\pgfqpoint{2.564209in}{2.900787in}}%
\pgfpathlineto{\pgfqpoint{2.564797in}{2.911081in}}%
\pgfpathlineto{\pgfqpoint{2.554380in}{2.911081in}}%
\pgfpathlineto{\pgfqpoint{2.546965in}{2.911081in}}%
\pgfpathlineto{\pgfqpoint{2.545994in}{2.900787in}}%
\pgfpathlineto{\pgfqpoint{2.545594in}{2.890492in}}%
\pgfpathlineto{\pgfqpoint{2.545749in}{2.880198in}}%
\pgfpathlineto{\pgfqpoint{2.546445in}{2.869903in}}%
\pgfpathlineto{\pgfqpoint{2.547672in}{2.859609in}}%
\pgfpathlineto{\pgfqpoint{2.549419in}{2.849315in}}%
\pgfpathlineto{\pgfqpoint{2.551680in}{2.839020in}}%
\pgfpathlineto{\pgfqpoint{2.554380in}{2.828980in}}%
\pgfpathlineto{\pgfqpoint{2.554449in}{2.828726in}}%
\pgfpathlineto{\pgfqpoint{2.557659in}{2.818431in}}%
\pgfpathlineto{\pgfqpoint{2.561402in}{2.808137in}}%
\pgfpathlineto{\pgfqpoint{2.564957in}{2.799746in}}%
\pgfpathlineto{\pgfqpoint{2.565805in}{2.797842in}}%
\pgfpathlineto{\pgfqpoint{2.574496in}{2.787548in}}%
\pgfpathlineto{\pgfqpoint{2.575534in}{2.787101in}}%
\pgfpathlineto{\pgfqpoint{2.586111in}{2.784663in}}%
\pgfpathlineto{\pgfqpoint{2.596688in}{2.782817in}}%
\pgfpathlineto{\pgfqpoint{2.607265in}{2.781157in}}%
\pgfpathlineto{\pgfqpoint{2.617842in}{2.779625in}}%
\pgfpathlineto{\pgfqpoint{2.628419in}{2.778218in}}%
\pgfpathlineto{\pgfqpoint{2.636486in}{2.777254in}}%
\pgfpathclose%
\pgfusepath{fill}%
\end{pgfscope}%
\begin{pgfscope}%
\pgfpathrectangle{\pgfqpoint{1.856795in}{1.819814in}}{\pgfqpoint{1.194205in}{1.163386in}}%
\pgfusepath{clip}%
\pgfsetbuttcap%
\pgfsetroundjoin%
\definecolor{currentfill}{rgb}{0.577499,0.110312,0.358417}%
\pgfsetfillcolor{currentfill}%
\pgfsetlinewidth{0.000000pt}%
\definecolor{currentstroke}{rgb}{0.000000,0.000000,0.000000}%
\pgfsetstrokecolor{currentstroke}%
\pgfsetdash{}{0pt}%
\pgfpathmoveto{\pgfqpoint{2.036104in}{1.891932in}}%
\pgfpathlineto{\pgfqpoint{2.046681in}{1.891932in}}%
\pgfpathlineto{\pgfqpoint{2.047237in}{1.891932in}}%
\pgfpathlineto{\pgfqpoint{2.050308in}{1.902227in}}%
\pgfpathlineto{\pgfqpoint{2.052334in}{1.912521in}}%
\pgfpathlineto{\pgfqpoint{2.053424in}{1.922816in}}%
\pgfpathlineto{\pgfqpoint{2.053674in}{1.933110in}}%
\pgfpathlineto{\pgfqpoint{2.053171in}{1.943405in}}%
\pgfpathlineto{\pgfqpoint{2.051990in}{1.953699in}}%
\pgfpathlineto{\pgfqpoint{2.050201in}{1.963993in}}%
\pgfpathlineto{\pgfqpoint{2.047869in}{1.974288in}}%
\pgfpathlineto{\pgfqpoint{2.046681in}{1.978601in}}%
\pgfpathlineto{\pgfqpoint{2.045004in}{1.984582in}}%
\pgfpathlineto{\pgfqpoint{2.041308in}{1.994877in}}%
\pgfpathlineto{\pgfqpoint{2.036104in}{2.004556in}}%
\pgfpathlineto{\pgfqpoint{2.035756in}{2.005171in}}%
\pgfpathlineto{\pgfqpoint{2.029155in}{2.015466in}}%
\pgfpathlineto{\pgfqpoint{2.025527in}{2.021170in}}%
\pgfpathlineto{\pgfqpoint{2.022618in}{2.025760in}}%
\pgfpathlineto{\pgfqpoint{2.016272in}{2.036054in}}%
\pgfpathlineto{\pgfqpoint{2.014950in}{2.038278in}}%
\pgfpathlineto{\pgfqpoint{2.010136in}{2.046349in}}%
\pgfpathlineto{\pgfqpoint{2.004373in}{2.056396in}}%
\pgfpathlineto{\pgfqpoint{2.004230in}{2.056643in}}%
\pgfpathlineto{\pgfqpoint{1.998565in}{2.066938in}}%
\pgfpathlineto{\pgfqpoint{1.993796in}{2.076031in}}%
\pgfpathlineto{\pgfqpoint{1.993165in}{2.077232in}}%
\pgfpathlineto{\pgfqpoint{1.988050in}{2.087527in}}%
\pgfpathlineto{\pgfqpoint{1.983254in}{2.097821in}}%
\pgfpathlineto{\pgfqpoint{1.983218in}{2.097901in}}%
\pgfpathlineto{\pgfqpoint{1.978776in}{2.108116in}}%
\pgfpathlineto{\pgfqpoint{1.974660in}{2.118410in}}%
\pgfpathlineto{\pgfqpoint{1.972641in}{2.123909in}}%
\pgfpathlineto{\pgfqpoint{1.970893in}{2.128704in}}%
\pgfpathlineto{\pgfqpoint{1.967475in}{2.138999in}}%
\pgfpathlineto{\pgfqpoint{1.964415in}{2.149293in}}%
\pgfpathlineto{\pgfqpoint{1.962064in}{2.158151in}}%
\pgfpathlineto{\pgfqpoint{1.961686in}{2.159588in}}%
\pgfpathlineto{\pgfqpoint{1.959231in}{2.169882in}}%
\pgfpathlineto{\pgfqpoint{1.957056in}{2.180177in}}%
\pgfpathlineto{\pgfqpoint{1.955124in}{2.190471in}}%
\pgfpathlineto{\pgfqpoint{1.953396in}{2.200765in}}%
\pgfpathlineto{\pgfqpoint{1.951831in}{2.211060in}}%
\pgfpathlineto{\pgfqpoint{1.951487in}{2.213455in}}%
\pgfpathlineto{\pgfqpoint{1.950357in}{2.221354in}}%
\pgfpathlineto{\pgfqpoint{1.948958in}{2.231649in}}%
\pgfpathlineto{\pgfqpoint{1.947607in}{2.241943in}}%
\pgfpathlineto{\pgfqpoint{1.946272in}{2.252238in}}%
\pgfpathlineto{\pgfqpoint{1.944919in}{2.262532in}}%
\pgfpathlineto{\pgfqpoint{1.943518in}{2.272826in}}%
\pgfpathlineto{\pgfqpoint{1.942027in}{2.283121in}}%
\pgfpathlineto{\pgfqpoint{1.940910in}{2.289839in}}%
\pgfpathlineto{\pgfqpoint{1.940307in}{2.293415in}}%
\pgfpathlineto{\pgfqpoint{1.938156in}{2.303710in}}%
\pgfpathlineto{\pgfqpoint{1.934392in}{2.314004in}}%
\pgfpathlineto{\pgfqpoint{1.930333in}{2.322941in}}%
\pgfpathlineto{\pgfqpoint{1.930333in}{2.314004in}}%
\pgfpathlineto{\pgfqpoint{1.930333in}{2.303710in}}%
\pgfpathlineto{\pgfqpoint{1.930333in}{2.293415in}}%
\pgfpathlineto{\pgfqpoint{1.930333in}{2.292258in}}%
\pgfpathlineto{\pgfqpoint{1.931428in}{2.283121in}}%
\pgfpathlineto{\pgfqpoint{1.932601in}{2.272826in}}%
\pgfpathlineto{\pgfqpoint{1.933708in}{2.262532in}}%
\pgfpathlineto{\pgfqpoint{1.934764in}{2.252238in}}%
\pgfpathlineto{\pgfqpoint{1.935793in}{2.241943in}}%
\pgfpathlineto{\pgfqpoint{1.936825in}{2.231649in}}%
\pgfpathlineto{\pgfqpoint{1.937897in}{2.221354in}}%
\pgfpathlineto{\pgfqpoint{1.939045in}{2.211060in}}%
\pgfpathlineto{\pgfqpoint{1.940313in}{2.200765in}}%
\pgfpathlineto{\pgfqpoint{1.940910in}{2.196399in}}%
\pgfpathlineto{\pgfqpoint{1.941716in}{2.190471in}}%
\pgfpathlineto{\pgfqpoint{1.943306in}{2.180177in}}%
\pgfpathlineto{\pgfqpoint{1.945147in}{2.169882in}}%
\pgfpathlineto{\pgfqpoint{1.947284in}{2.159588in}}%
\pgfpathlineto{\pgfqpoint{1.949758in}{2.149293in}}%
\pgfpathlineto{\pgfqpoint{1.951487in}{2.142991in}}%
\pgfpathlineto{\pgfqpoint{1.952579in}{2.138999in}}%
\pgfpathlineto{\pgfqpoint{1.955750in}{2.128704in}}%
\pgfpathlineto{\pgfqpoint{1.959334in}{2.118410in}}%
\pgfpathlineto{\pgfqpoint{1.962064in}{2.111374in}}%
\pgfpathlineto{\pgfqpoint{1.963323in}{2.108116in}}%
\pgfpathlineto{\pgfqpoint{1.967670in}{2.097821in}}%
\pgfpathlineto{\pgfqpoint{1.972430in}{2.087527in}}%
\pgfpathlineto{\pgfqpoint{1.972641in}{2.087100in}}%
\pgfpathlineto{\pgfqpoint{1.977497in}{2.077232in}}%
\pgfpathlineto{\pgfqpoint{1.982922in}{2.066938in}}%
\pgfpathlineto{\pgfqpoint{1.983218in}{2.066405in}}%
\pgfpathlineto{\pgfqpoint{1.988602in}{2.056643in}}%
\pgfpathlineto{\pgfqpoint{1.993796in}{2.047691in}}%
\pgfpathlineto{\pgfqpoint{1.994568in}{2.046349in}}%
\pgfpathlineto{\pgfqpoint{2.000731in}{2.036054in}}%
\pgfpathlineto{\pgfqpoint{2.004373in}{2.030175in}}%
\pgfpathlineto{\pgfqpoint{2.007093in}{2.025760in}}%
\pgfpathlineto{\pgfqpoint{2.013573in}{2.015466in}}%
\pgfpathlineto{\pgfqpoint{2.014950in}{2.013263in}}%
\pgfpathlineto{\pgfqpoint{2.020020in}{2.005171in}}%
\pgfpathlineto{\pgfqpoint{2.025527in}{1.995658in}}%
\pgfpathlineto{\pgfqpoint{2.025973in}{1.994877in}}%
\pgfpathlineto{\pgfqpoint{2.030523in}{1.984582in}}%
\pgfpathlineto{\pgfqpoint{2.033716in}{1.974288in}}%
\pgfpathlineto{\pgfqpoint{2.036021in}{1.963993in}}%
\pgfpathlineto{\pgfqpoint{2.036104in}{1.963460in}}%
\pgfpathlineto{\pgfqpoint{2.037613in}{1.953699in}}%
\pgfpathlineto{\pgfqpoint{2.038537in}{1.943405in}}%
\pgfpathlineto{\pgfqpoint{2.038742in}{1.933110in}}%
\pgfpathlineto{\pgfqpoint{2.038155in}{1.922816in}}%
\pgfpathlineto{\pgfqpoint{2.036689in}{1.912521in}}%
\pgfpathlineto{\pgfqpoint{2.036104in}{1.910007in}}%
\pgfpathlineto{\pgfqpoint{2.034242in}{1.902227in}}%
\pgfpathlineto{\pgfqpoint{2.030695in}{1.891932in}}%
\pgfpathclose%
\pgfusepath{fill}%
\end{pgfscope}%
\begin{pgfscope}%
\pgfpathrectangle{\pgfqpoint{1.856795in}{1.819814in}}{\pgfqpoint{1.194205in}{1.163386in}}%
\pgfusepath{clip}%
\pgfsetbuttcap%
\pgfsetroundjoin%
\definecolor{currentfill}{rgb}{0.577499,0.110312,0.358417}%
\pgfsetfillcolor{currentfill}%
\pgfsetlinewidth{0.000000pt}%
\definecolor{currentstroke}{rgb}{0.000000,0.000000,0.000000}%
\pgfsetstrokecolor{currentstroke}%
\pgfsetdash{}{0pt}%
\pgfpathmoveto{\pgfqpoint{2.670727in}{2.756356in}}%
\pgfpathlineto{\pgfqpoint{2.681305in}{2.755355in}}%
\pgfpathlineto{\pgfqpoint{2.691882in}{2.754650in}}%
\pgfpathlineto{\pgfqpoint{2.702459in}{2.754295in}}%
\pgfpathlineto{\pgfqpoint{2.713036in}{2.754346in}}%
\pgfpathlineto{\pgfqpoint{2.723613in}{2.754870in}}%
\pgfpathlineto{\pgfqpoint{2.734190in}{2.755936in}}%
\pgfpathlineto{\pgfqpoint{2.738866in}{2.756665in}}%
\pgfpathlineto{\pgfqpoint{2.744767in}{2.757561in}}%
\pgfpathlineto{\pgfqpoint{2.755344in}{2.759786in}}%
\pgfpathlineto{\pgfqpoint{2.765921in}{2.762709in}}%
\pgfpathlineto{\pgfqpoint{2.776498in}{2.766374in}}%
\pgfpathlineto{\pgfqpoint{2.777922in}{2.766959in}}%
\pgfpathlineto{\pgfqpoint{2.787075in}{2.770604in}}%
\pgfpathlineto{\pgfqpoint{2.797652in}{2.775520in}}%
\pgfpathlineto{\pgfqpoint{2.800954in}{2.777254in}}%
\pgfpathlineto{\pgfqpoint{2.808229in}{2.780943in}}%
\pgfpathlineto{\pgfqpoint{2.818806in}{2.786878in}}%
\pgfpathlineto{\pgfqpoint{2.819909in}{2.787548in}}%
\pgfpathlineto{\pgfqpoint{2.829383in}{2.793096in}}%
\pgfpathlineto{\pgfqpoint{2.836997in}{2.797842in}}%
\pgfpathlineto{\pgfqpoint{2.839960in}{2.799627in}}%
\pgfpathlineto{\pgfqpoint{2.850538in}{2.806291in}}%
\pgfpathlineto{\pgfqpoint{2.853363in}{2.808137in}}%
\pgfpathlineto{\pgfqpoint{2.861115in}{2.813023in}}%
\pgfpathlineto{\pgfqpoint{2.869530in}{2.818431in}}%
\pgfpathlineto{\pgfqpoint{2.871692in}{2.819778in}}%
\pgfpathlineto{\pgfqpoint{2.882269in}{2.826456in}}%
\pgfpathlineto{\pgfqpoint{2.885839in}{2.828726in}}%
\pgfpathlineto{\pgfqpoint{2.892846in}{2.833046in}}%
\pgfpathlineto{\pgfqpoint{2.902616in}{2.839020in}}%
\pgfpathlineto{\pgfqpoint{2.903423in}{2.839502in}}%
\pgfpathlineto{\pgfqpoint{2.914000in}{2.845830in}}%
\pgfpathlineto{\pgfqpoint{2.919878in}{2.849315in}}%
\pgfpathlineto{\pgfqpoint{2.924577in}{2.852029in}}%
\pgfpathlineto{\pgfqpoint{2.935154in}{2.858114in}}%
\pgfpathlineto{\pgfqpoint{2.937692in}{2.859609in}}%
\pgfpathlineto{\pgfqpoint{2.945731in}{2.864200in}}%
\pgfpathlineto{\pgfqpoint{2.955541in}{2.869903in}}%
\pgfpathlineto{\pgfqpoint{2.956308in}{2.870336in}}%
\pgfpathlineto{\pgfqpoint{2.966885in}{2.876809in}}%
\pgfpathlineto{\pgfqpoint{2.971794in}{2.880198in}}%
\pgfpathlineto{\pgfqpoint{2.977462in}{2.884011in}}%
\pgfpathlineto{\pgfqpoint{2.977462in}{2.890492in}}%
\pgfpathlineto{\pgfqpoint{2.977462in}{2.898326in}}%
\pgfpathlineto{\pgfqpoint{2.966885in}{2.891725in}}%
\pgfpathlineto{\pgfqpoint{2.964505in}{2.890492in}}%
\pgfpathlineto{\pgfqpoint{2.956308in}{2.886130in}}%
\pgfpathlineto{\pgfqpoint{2.945731in}{2.880642in}}%
\pgfpathlineto{\pgfqpoint{2.944862in}{2.880198in}}%
\pgfpathlineto{\pgfqpoint{2.935154in}{2.875308in}}%
\pgfpathlineto{\pgfqpoint{2.924996in}{2.869903in}}%
\pgfpathlineto{\pgfqpoint{2.924577in}{2.869681in}}%
\pgfpathlineto{\pgfqpoint{2.914000in}{2.864054in}}%
\pgfpathlineto{\pgfqpoint{2.906053in}{2.859609in}}%
\pgfpathlineto{\pgfqpoint{2.903423in}{2.858140in}}%
\pgfpathlineto{\pgfqpoint{2.892846in}{2.852126in}}%
\pgfpathlineto{\pgfqpoint{2.888028in}{2.849315in}}%
\pgfpathlineto{\pgfqpoint{2.882269in}{2.845948in}}%
\pgfpathlineto{\pgfqpoint{2.871692in}{2.839663in}}%
\pgfpathlineto{\pgfqpoint{2.870610in}{2.839020in}}%
\pgfpathlineto{\pgfqpoint{2.861115in}{2.833341in}}%
\pgfpathlineto{\pgfqpoint{2.853382in}{2.828726in}}%
\pgfpathlineto{\pgfqpoint{2.850538in}{2.827004in}}%
\pgfpathlineto{\pgfqpoint{2.839960in}{2.820735in}}%
\pgfpathlineto{\pgfqpoint{2.835958in}{2.818431in}}%
\pgfpathlineto{\pgfqpoint{2.829383in}{2.814588in}}%
\pgfpathlineto{\pgfqpoint{2.818806in}{2.808679in}}%
\pgfpathlineto{\pgfqpoint{2.817780in}{2.808137in}}%
\pgfpathlineto{\pgfqpoint{2.808229in}{2.803015in}}%
\pgfpathlineto{\pgfqpoint{2.797778in}{2.797842in}}%
\pgfpathlineto{\pgfqpoint{2.797652in}{2.797779in}}%
\pgfpathlineto{\pgfqpoint{2.787075in}{2.792911in}}%
\pgfpathlineto{\pgfqpoint{2.776498in}{2.788612in}}%
\pgfpathlineto{\pgfqpoint{2.773512in}{2.787548in}}%
\pgfpathlineto{\pgfqpoint{2.765921in}{2.784809in}}%
\pgfpathlineto{\pgfqpoint{2.755344in}{2.781590in}}%
\pgfpathlineto{\pgfqpoint{2.744767in}{2.778974in}}%
\pgfpathlineto{\pgfqpoint{2.735900in}{2.777254in}}%
\pgfpathlineto{\pgfqpoint{2.734190in}{2.776918in}}%
\pgfpathlineto{\pgfqpoint{2.723613in}{2.775351in}}%
\pgfpathlineto{\pgfqpoint{2.713036in}{2.774292in}}%
\pgfpathlineto{\pgfqpoint{2.702459in}{2.773688in}}%
\pgfpathlineto{\pgfqpoint{2.691882in}{2.773487in}}%
\pgfpathlineto{\pgfqpoint{2.681305in}{2.773641in}}%
\pgfpathlineto{\pgfqpoint{2.670727in}{2.774103in}}%
\pgfpathlineto{\pgfqpoint{2.660150in}{2.774830in}}%
\pgfpathlineto{\pgfqpoint{2.649573in}{2.775786in}}%
\pgfpathlineto{\pgfqpoint{2.638996in}{2.776937in}}%
\pgfpathlineto{\pgfqpoint{2.636486in}{2.777254in}}%
\pgfpathlineto{\pgfqpoint{2.628419in}{2.778218in}}%
\pgfpathlineto{\pgfqpoint{2.617842in}{2.779625in}}%
\pgfpathlineto{\pgfqpoint{2.607265in}{2.781157in}}%
\pgfpathlineto{\pgfqpoint{2.596688in}{2.782817in}}%
\pgfpathlineto{\pgfqpoint{2.586111in}{2.784663in}}%
\pgfpathlineto{\pgfqpoint{2.575534in}{2.787101in}}%
\pgfpathlineto{\pgfqpoint{2.574496in}{2.787548in}}%
\pgfpathlineto{\pgfqpoint{2.565805in}{2.797842in}}%
\pgfpathlineto{\pgfqpoint{2.564957in}{2.799746in}}%
\pgfpathlineto{\pgfqpoint{2.561402in}{2.808137in}}%
\pgfpathlineto{\pgfqpoint{2.557659in}{2.818431in}}%
\pgfpathlineto{\pgfqpoint{2.554449in}{2.828726in}}%
\pgfpathlineto{\pgfqpoint{2.554380in}{2.828980in}}%
\pgfpathlineto{\pgfqpoint{2.551680in}{2.839020in}}%
\pgfpathlineto{\pgfqpoint{2.549419in}{2.849315in}}%
\pgfpathlineto{\pgfqpoint{2.547672in}{2.859609in}}%
\pgfpathlineto{\pgfqpoint{2.546445in}{2.869903in}}%
\pgfpathlineto{\pgfqpoint{2.545749in}{2.880198in}}%
\pgfpathlineto{\pgfqpoint{2.545594in}{2.890492in}}%
\pgfpathlineto{\pgfqpoint{2.545994in}{2.900787in}}%
\pgfpathlineto{\pgfqpoint{2.546965in}{2.911081in}}%
\pgfpathlineto{\pgfqpoint{2.543803in}{2.911081in}}%
\pgfpathlineto{\pgfqpoint{2.533226in}{2.911081in}}%
\pgfpathlineto{\pgfqpoint{2.528247in}{2.911081in}}%
\pgfpathlineto{\pgfqpoint{2.526807in}{2.900787in}}%
\pgfpathlineto{\pgfqpoint{2.525929in}{2.890492in}}%
\pgfpathlineto{\pgfqpoint{2.525600in}{2.880198in}}%
\pgfpathlineto{\pgfqpoint{2.525804in}{2.869903in}}%
\pgfpathlineto{\pgfqpoint{2.526531in}{2.859609in}}%
\pgfpathlineto{\pgfqpoint{2.527770in}{2.849315in}}%
\pgfpathlineto{\pgfqpoint{2.529510in}{2.839020in}}%
\pgfpathlineto{\pgfqpoint{2.531742in}{2.828726in}}%
\pgfpathlineto{\pgfqpoint{2.533226in}{2.823018in}}%
\pgfpathlineto{\pgfqpoint{2.534416in}{2.818431in}}%
\pgfpathlineto{\pgfqpoint{2.537507in}{2.808137in}}%
\pgfpathlineto{\pgfqpoint{2.541073in}{2.797842in}}%
\pgfpathlineto{\pgfqpoint{2.543803in}{2.790869in}}%
\pgfpathlineto{\pgfqpoint{2.545145in}{2.787548in}}%
\pgfpathlineto{\pgfqpoint{2.551915in}{2.777254in}}%
\pgfpathlineto{\pgfqpoint{2.554380in}{2.776123in}}%
\pgfpathlineto{\pgfqpoint{2.564957in}{2.773573in}}%
\pgfpathlineto{\pgfqpoint{2.575534in}{2.771507in}}%
\pgfpathlineto{\pgfqpoint{2.586111in}{2.769551in}}%
\pgfpathlineto{\pgfqpoint{2.596688in}{2.767657in}}%
\pgfpathlineto{\pgfqpoint{2.600731in}{2.766959in}}%
\pgfpathlineto{\pgfqpoint{2.607265in}{2.765767in}}%
\pgfpathlineto{\pgfqpoint{2.617842in}{2.763910in}}%
\pgfpathlineto{\pgfqpoint{2.628419in}{2.762137in}}%
\pgfpathlineto{\pgfqpoint{2.638996in}{2.760469in}}%
\pgfpathlineto{\pgfqpoint{2.649573in}{2.758931in}}%
\pgfpathlineto{\pgfqpoint{2.660150in}{2.757554in}}%
\pgfpathlineto{\pgfqpoint{2.668157in}{2.756665in}}%
\pgfpathclose%
\pgfusepath{fill}%
\end{pgfscope}%
\begin{pgfscope}%
\pgfpathrectangle{\pgfqpoint{1.856795in}{1.819814in}}{\pgfqpoint{1.194205in}{1.163386in}}%
\pgfusepath{clip}%
\pgfsetbuttcap%
\pgfsetroundjoin%
\definecolor{currentfill}{rgb}{0.651546,0.096802,0.354541}%
\pgfsetfillcolor{currentfill}%
\pgfsetlinewidth{0.000000pt}%
\definecolor{currentstroke}{rgb}{0.000000,0.000000,0.000000}%
\pgfsetstrokecolor{currentstroke}%
\pgfsetdash{}{0pt}%
\pgfpathmoveto{\pgfqpoint{2.057258in}{1.891932in}}%
\pgfpathlineto{\pgfqpoint{2.064067in}{1.891932in}}%
\pgfpathlineto{\pgfqpoint{2.066700in}{1.902227in}}%
\pgfpathlineto{\pgfqpoint{2.067835in}{1.909344in}}%
\pgfpathlineto{\pgfqpoint{2.068344in}{1.912521in}}%
\pgfpathlineto{\pgfqpoint{2.069096in}{1.922816in}}%
\pgfpathlineto{\pgfqpoint{2.069010in}{1.933110in}}%
\pgfpathlineto{\pgfqpoint{2.068163in}{1.943405in}}%
\pgfpathlineto{\pgfqpoint{2.067835in}{1.945571in}}%
\pgfpathlineto{\pgfqpoint{2.066667in}{1.953699in}}%
\pgfpathlineto{\pgfqpoint{2.064580in}{1.963993in}}%
\pgfpathlineto{\pgfqpoint{2.061954in}{1.974288in}}%
\pgfpathlineto{\pgfqpoint{2.058860in}{1.984582in}}%
\pgfpathlineto{\pgfqpoint{2.057258in}{1.989221in}}%
\pgfpathlineto{\pgfqpoint{2.055353in}{1.994877in}}%
\pgfpathlineto{\pgfqpoint{2.050875in}{2.005171in}}%
\pgfpathlineto{\pgfqpoint{2.046681in}{2.012267in}}%
\pgfpathlineto{\pgfqpoint{2.044549in}{2.015466in}}%
\pgfpathlineto{\pgfqpoint{2.038023in}{2.025760in}}%
\pgfpathlineto{\pgfqpoint{2.036104in}{2.028944in}}%
\pgfpathlineto{\pgfqpoint{2.031740in}{2.036054in}}%
\pgfpathlineto{\pgfqpoint{2.025672in}{2.046349in}}%
\pgfpathlineto{\pgfqpoint{2.025527in}{2.046606in}}%
\pgfpathlineto{\pgfqpoint{2.019827in}{2.056643in}}%
\pgfpathlineto{\pgfqpoint{2.014950in}{2.065530in}}%
\pgfpathlineto{\pgfqpoint{2.014179in}{2.066938in}}%
\pgfpathlineto{\pgfqpoint{2.008776in}{2.077232in}}%
\pgfpathlineto{\pgfqpoint{2.004373in}{2.085973in}}%
\pgfpathlineto{\pgfqpoint{2.003596in}{2.087527in}}%
\pgfpathlineto{\pgfqpoint{1.998703in}{2.097821in}}%
\pgfpathlineto{\pgfqpoint{1.994082in}{2.108116in}}%
\pgfpathlineto{\pgfqpoint{1.993796in}{2.108789in}}%
\pgfpathlineto{\pgfqpoint{1.989777in}{2.118410in}}%
\pgfpathlineto{\pgfqpoint{1.985788in}{2.128704in}}%
\pgfpathlineto{\pgfqpoint{1.983218in}{2.135876in}}%
\pgfpathlineto{\pgfqpoint{1.982119in}{2.138999in}}%
\pgfpathlineto{\pgfqpoint{1.978765in}{2.149293in}}%
\pgfpathlineto{\pgfqpoint{1.975719in}{2.159588in}}%
\pgfpathlineto{\pgfqpoint{1.972963in}{2.169882in}}%
\pgfpathlineto{\pgfqpoint{1.972641in}{2.171179in}}%
\pgfpathlineto{\pgfqpoint{1.970448in}{2.180177in}}%
\pgfpathlineto{\pgfqpoint{1.968164in}{2.190471in}}%
\pgfpathlineto{\pgfqpoint{1.966080in}{2.200765in}}%
\pgfpathlineto{\pgfqpoint{1.964159in}{2.211060in}}%
\pgfpathlineto{\pgfqpoint{1.962363in}{2.221354in}}%
\pgfpathlineto{\pgfqpoint{1.962064in}{2.223106in}}%
\pgfpathlineto{\pgfqpoint{1.960630in}{2.231649in}}%
\pgfpathlineto{\pgfqpoint{1.958946in}{2.241943in}}%
\pgfpathlineto{\pgfqpoint{1.957283in}{2.252238in}}%
\pgfpathlineto{\pgfqpoint{1.955602in}{2.262532in}}%
\pgfpathlineto{\pgfqpoint{1.953856in}{2.272826in}}%
\pgfpathlineto{\pgfqpoint{1.951979in}{2.283121in}}%
\pgfpathlineto{\pgfqpoint{1.951487in}{2.285470in}}%
\pgfpathlineto{\pgfqpoint{1.949885in}{2.293415in}}%
\pgfpathlineto{\pgfqpoint{1.947457in}{2.303710in}}%
\pgfpathlineto{\pgfqpoint{1.944348in}{2.314004in}}%
\pgfpathlineto{\pgfqpoint{1.940910in}{2.323137in}}%
\pgfpathlineto{\pgfqpoint{1.940423in}{2.324299in}}%
\pgfpathlineto{\pgfqpoint{1.935696in}{2.334593in}}%
\pgfpathlineto{\pgfqpoint{1.930619in}{2.344887in}}%
\pgfpathlineto{\pgfqpoint{1.930333in}{2.345455in}}%
\pgfpathlineto{\pgfqpoint{1.930333in}{2.344887in}}%
\pgfpathlineto{\pgfqpoint{1.930333in}{2.334593in}}%
\pgfpathlineto{\pgfqpoint{1.930333in}{2.324299in}}%
\pgfpathlineto{\pgfqpoint{1.930333in}{2.322941in}}%
\pgfpathlineto{\pgfqpoint{1.934392in}{2.314004in}}%
\pgfpathlineto{\pgfqpoint{1.938156in}{2.303710in}}%
\pgfpathlineto{\pgfqpoint{1.940307in}{2.293415in}}%
\pgfpathlineto{\pgfqpoint{1.940910in}{2.289839in}}%
\pgfpathlineto{\pgfqpoint{1.942027in}{2.283121in}}%
\pgfpathlineto{\pgfqpoint{1.943518in}{2.272826in}}%
\pgfpathlineto{\pgfqpoint{1.944919in}{2.262532in}}%
\pgfpathlineto{\pgfqpoint{1.946272in}{2.252238in}}%
\pgfpathlineto{\pgfqpoint{1.947607in}{2.241943in}}%
\pgfpathlineto{\pgfqpoint{1.948958in}{2.231649in}}%
\pgfpathlineto{\pgfqpoint{1.950357in}{2.221354in}}%
\pgfpathlineto{\pgfqpoint{1.951487in}{2.213455in}}%
\pgfpathlineto{\pgfqpoint{1.951831in}{2.211060in}}%
\pgfpathlineto{\pgfqpoint{1.953396in}{2.200765in}}%
\pgfpathlineto{\pgfqpoint{1.955124in}{2.190471in}}%
\pgfpathlineto{\pgfqpoint{1.957056in}{2.180177in}}%
\pgfpathlineto{\pgfqpoint{1.959231in}{2.169882in}}%
\pgfpathlineto{\pgfqpoint{1.961686in}{2.159588in}}%
\pgfpathlineto{\pgfqpoint{1.962064in}{2.158151in}}%
\pgfpathlineto{\pgfqpoint{1.964415in}{2.149293in}}%
\pgfpathlineto{\pgfqpoint{1.967475in}{2.138999in}}%
\pgfpathlineto{\pgfqpoint{1.970893in}{2.128704in}}%
\pgfpathlineto{\pgfqpoint{1.972641in}{2.123909in}}%
\pgfpathlineto{\pgfqpoint{1.974660in}{2.118410in}}%
\pgfpathlineto{\pgfqpoint{1.978776in}{2.108116in}}%
\pgfpathlineto{\pgfqpoint{1.983218in}{2.097901in}}%
\pgfpathlineto{\pgfqpoint{1.983254in}{2.097821in}}%
\pgfpathlineto{\pgfqpoint{1.988050in}{2.087527in}}%
\pgfpathlineto{\pgfqpoint{1.993165in}{2.077232in}}%
\pgfpathlineto{\pgfqpoint{1.993796in}{2.076031in}}%
\pgfpathlineto{\pgfqpoint{1.998565in}{2.066938in}}%
\pgfpathlineto{\pgfqpoint{2.004230in}{2.056643in}}%
\pgfpathlineto{\pgfqpoint{2.004373in}{2.056396in}}%
\pgfpathlineto{\pgfqpoint{2.010136in}{2.046349in}}%
\pgfpathlineto{\pgfqpoint{2.014950in}{2.038278in}}%
\pgfpathlineto{\pgfqpoint{2.016272in}{2.036054in}}%
\pgfpathlineto{\pgfqpoint{2.022618in}{2.025760in}}%
\pgfpathlineto{\pgfqpoint{2.025527in}{2.021170in}}%
\pgfpathlineto{\pgfqpoint{2.029155in}{2.015466in}}%
\pgfpathlineto{\pgfqpoint{2.035756in}{2.005171in}}%
\pgfpathlineto{\pgfqpoint{2.036104in}{2.004556in}}%
\pgfpathlineto{\pgfqpoint{2.041308in}{1.994877in}}%
\pgfpathlineto{\pgfqpoint{2.045004in}{1.984582in}}%
\pgfpathlineto{\pgfqpoint{2.046681in}{1.978601in}}%
\pgfpathlineto{\pgfqpoint{2.047869in}{1.974288in}}%
\pgfpathlineto{\pgfqpoint{2.050201in}{1.963993in}}%
\pgfpathlineto{\pgfqpoint{2.051990in}{1.953699in}}%
\pgfpathlineto{\pgfqpoint{2.053171in}{1.943405in}}%
\pgfpathlineto{\pgfqpoint{2.053674in}{1.933110in}}%
\pgfpathlineto{\pgfqpoint{2.053424in}{1.922816in}}%
\pgfpathlineto{\pgfqpoint{2.052334in}{1.912521in}}%
\pgfpathlineto{\pgfqpoint{2.050308in}{1.902227in}}%
\pgfpathlineto{\pgfqpoint{2.047237in}{1.891932in}}%
\pgfpathclose%
\pgfusepath{fill}%
\end{pgfscope}%
\begin{pgfscope}%
\pgfpathrectangle{\pgfqpoint{1.856795in}{1.819814in}}{\pgfqpoint{1.194205in}{1.163386in}}%
\pgfusepath{clip}%
\pgfsetbuttcap%
\pgfsetroundjoin%
\definecolor{currentfill}{rgb}{0.651546,0.096802,0.354541}%
\pgfsetfillcolor{currentfill}%
\pgfsetlinewidth{0.000000pt}%
\definecolor{currentstroke}{rgb}{0.000000,0.000000,0.000000}%
\pgfsetstrokecolor{currentstroke}%
\pgfsetdash{}{0pt}%
\pgfpathmoveto{\pgfqpoint{2.681305in}{2.734976in}}%
\pgfpathlineto{\pgfqpoint{2.691882in}{2.733550in}}%
\pgfpathlineto{\pgfqpoint{2.702459in}{2.732472in}}%
\pgfpathlineto{\pgfqpoint{2.713036in}{2.731810in}}%
\pgfpathlineto{\pgfqpoint{2.723613in}{2.731641in}}%
\pgfpathlineto{\pgfqpoint{2.734190in}{2.732055in}}%
\pgfpathlineto{\pgfqpoint{2.744767in}{2.733147in}}%
\pgfpathlineto{\pgfqpoint{2.755344in}{2.735023in}}%
\pgfpathlineto{\pgfqpoint{2.759456in}{2.736076in}}%
\pgfpathlineto{\pgfqpoint{2.765921in}{2.737652in}}%
\pgfpathlineto{\pgfqpoint{2.776498in}{2.741101in}}%
\pgfpathlineto{\pgfqpoint{2.787075in}{2.745519in}}%
\pgfpathlineto{\pgfqpoint{2.788767in}{2.746370in}}%
\pgfpathlineto{\pgfqpoint{2.797652in}{2.750611in}}%
\pgfpathlineto{\pgfqpoint{2.808229in}{2.756560in}}%
\pgfpathlineto{\pgfqpoint{2.808393in}{2.756665in}}%
\pgfpathlineto{\pgfqpoint{2.818806in}{2.762941in}}%
\pgfpathlineto{\pgfqpoint{2.824806in}{2.766959in}}%
\pgfpathlineto{\pgfqpoint{2.829383in}{2.769862in}}%
\pgfpathlineto{\pgfqpoint{2.839960in}{2.777130in}}%
\pgfpathlineto{\pgfqpoint{2.840128in}{2.777254in}}%
\pgfpathlineto{\pgfqpoint{2.850538in}{2.784478in}}%
\pgfpathlineto{\pgfqpoint{2.854750in}{2.787548in}}%
\pgfpathlineto{\pgfqpoint{2.861115in}{2.791944in}}%
\pgfpathlineto{\pgfqpoint{2.869367in}{2.797842in}}%
\pgfpathlineto{\pgfqpoint{2.871692in}{2.799423in}}%
\pgfpathlineto{\pgfqpoint{2.882269in}{2.806798in}}%
\pgfpathlineto{\pgfqpoint{2.884147in}{2.808137in}}%
\pgfpathlineto{\pgfqpoint{2.892846in}{2.814036in}}%
\pgfpathlineto{\pgfqpoint{2.899265in}{2.818431in}}%
\pgfpathlineto{\pgfqpoint{2.903423in}{2.821152in}}%
\pgfpathlineto{\pgfqpoint{2.914000in}{2.828116in}}%
\pgfpathlineto{\pgfqpoint{2.914909in}{2.828726in}}%
\pgfpathlineto{\pgfqpoint{2.924577in}{2.834930in}}%
\pgfpathlineto{\pgfqpoint{2.930897in}{2.839020in}}%
\pgfpathlineto{\pgfqpoint{2.935154in}{2.841664in}}%
\pgfpathlineto{\pgfqpoint{2.945731in}{2.848363in}}%
\pgfpathlineto{\pgfqpoint{2.947165in}{2.849315in}}%
\pgfpathlineto{\pgfqpoint{2.956308in}{2.855140in}}%
\pgfpathlineto{\pgfqpoint{2.962943in}{2.859609in}}%
\pgfpathlineto{\pgfqpoint{2.966885in}{2.862170in}}%
\pgfpathlineto{\pgfqpoint{2.977462in}{2.869731in}}%
\pgfpathlineto{\pgfqpoint{2.977462in}{2.869903in}}%
\pgfpathlineto{\pgfqpoint{2.977462in}{2.880198in}}%
\pgfpathlineto{\pgfqpoint{2.977462in}{2.884011in}}%
\pgfpathlineto{\pgfqpoint{2.971794in}{2.880198in}}%
\pgfpathlineto{\pgfqpoint{2.966885in}{2.876809in}}%
\pgfpathlineto{\pgfqpoint{2.956308in}{2.870336in}}%
\pgfpathlineto{\pgfqpoint{2.955541in}{2.869903in}}%
\pgfpathlineto{\pgfqpoint{2.945731in}{2.864200in}}%
\pgfpathlineto{\pgfqpoint{2.937692in}{2.859609in}}%
\pgfpathlineto{\pgfqpoint{2.935154in}{2.858114in}}%
\pgfpathlineto{\pgfqpoint{2.924577in}{2.852029in}}%
\pgfpathlineto{\pgfqpoint{2.919878in}{2.849315in}}%
\pgfpathlineto{\pgfqpoint{2.914000in}{2.845830in}}%
\pgfpathlineto{\pgfqpoint{2.903423in}{2.839502in}}%
\pgfpathlineto{\pgfqpoint{2.902616in}{2.839020in}}%
\pgfpathlineto{\pgfqpoint{2.892846in}{2.833046in}}%
\pgfpathlineto{\pgfqpoint{2.885839in}{2.828726in}}%
\pgfpathlineto{\pgfqpoint{2.882269in}{2.826456in}}%
\pgfpathlineto{\pgfqpoint{2.871692in}{2.819778in}}%
\pgfpathlineto{\pgfqpoint{2.869530in}{2.818431in}}%
\pgfpathlineto{\pgfqpoint{2.861115in}{2.813023in}}%
\pgfpathlineto{\pgfqpoint{2.853363in}{2.808137in}}%
\pgfpathlineto{\pgfqpoint{2.850538in}{2.806291in}}%
\pgfpathlineto{\pgfqpoint{2.839960in}{2.799627in}}%
\pgfpathlineto{\pgfqpoint{2.836997in}{2.797842in}}%
\pgfpathlineto{\pgfqpoint{2.829383in}{2.793096in}}%
\pgfpathlineto{\pgfqpoint{2.819909in}{2.787548in}}%
\pgfpathlineto{\pgfqpoint{2.818806in}{2.786878in}}%
\pgfpathlineto{\pgfqpoint{2.808229in}{2.780943in}}%
\pgfpathlineto{\pgfqpoint{2.800954in}{2.777254in}}%
\pgfpathlineto{\pgfqpoint{2.797652in}{2.775520in}}%
\pgfpathlineto{\pgfqpoint{2.787075in}{2.770604in}}%
\pgfpathlineto{\pgfqpoint{2.777922in}{2.766959in}}%
\pgfpathlineto{\pgfqpoint{2.776498in}{2.766374in}}%
\pgfpathlineto{\pgfqpoint{2.765921in}{2.762709in}}%
\pgfpathlineto{\pgfqpoint{2.755344in}{2.759786in}}%
\pgfpathlineto{\pgfqpoint{2.744767in}{2.757561in}}%
\pgfpathlineto{\pgfqpoint{2.738866in}{2.756665in}}%
\pgfpathlineto{\pgfqpoint{2.734190in}{2.755936in}}%
\pgfpathlineto{\pgfqpoint{2.723613in}{2.754870in}}%
\pgfpathlineto{\pgfqpoint{2.713036in}{2.754346in}}%
\pgfpathlineto{\pgfqpoint{2.702459in}{2.754295in}}%
\pgfpathlineto{\pgfqpoint{2.691882in}{2.754650in}}%
\pgfpathlineto{\pgfqpoint{2.681305in}{2.755355in}}%
\pgfpathlineto{\pgfqpoint{2.670727in}{2.756356in}}%
\pgfpathlineto{\pgfqpoint{2.668157in}{2.756665in}}%
\pgfpathlineto{\pgfqpoint{2.660150in}{2.757554in}}%
\pgfpathlineto{\pgfqpoint{2.649573in}{2.758931in}}%
\pgfpathlineto{\pgfqpoint{2.638996in}{2.760469in}}%
\pgfpathlineto{\pgfqpoint{2.628419in}{2.762137in}}%
\pgfpathlineto{\pgfqpoint{2.617842in}{2.763910in}}%
\pgfpathlineto{\pgfqpoint{2.607265in}{2.765767in}}%
\pgfpathlineto{\pgfqpoint{2.600731in}{2.766959in}}%
\pgfpathlineto{\pgfqpoint{2.596688in}{2.767657in}}%
\pgfpathlineto{\pgfqpoint{2.586111in}{2.769551in}}%
\pgfpathlineto{\pgfqpoint{2.575534in}{2.771507in}}%
\pgfpathlineto{\pgfqpoint{2.564957in}{2.773573in}}%
\pgfpathlineto{\pgfqpoint{2.554380in}{2.776123in}}%
\pgfpathlineto{\pgfqpoint{2.551915in}{2.777254in}}%
\pgfpathlineto{\pgfqpoint{2.545145in}{2.787548in}}%
\pgfpathlineto{\pgfqpoint{2.543803in}{2.790869in}}%
\pgfpathlineto{\pgfqpoint{2.541073in}{2.797842in}}%
\pgfpathlineto{\pgfqpoint{2.537507in}{2.808137in}}%
\pgfpathlineto{\pgfqpoint{2.534416in}{2.818431in}}%
\pgfpathlineto{\pgfqpoint{2.533226in}{2.823018in}}%
\pgfpathlineto{\pgfqpoint{2.531742in}{2.828726in}}%
\pgfpathlineto{\pgfqpoint{2.529510in}{2.839020in}}%
\pgfpathlineto{\pgfqpoint{2.527770in}{2.849315in}}%
\pgfpathlineto{\pgfqpoint{2.526531in}{2.859609in}}%
\pgfpathlineto{\pgfqpoint{2.525804in}{2.869903in}}%
\pgfpathlineto{\pgfqpoint{2.525600in}{2.880198in}}%
\pgfpathlineto{\pgfqpoint{2.525929in}{2.890492in}}%
\pgfpathlineto{\pgfqpoint{2.526807in}{2.900787in}}%
\pgfpathlineto{\pgfqpoint{2.528247in}{2.911081in}}%
\pgfpathlineto{\pgfqpoint{2.522649in}{2.911081in}}%
\pgfpathlineto{\pgfqpoint{2.512072in}{2.911081in}}%
\pgfpathlineto{\pgfqpoint{2.508208in}{2.911081in}}%
\pgfpathlineto{\pgfqpoint{2.506191in}{2.900787in}}%
\pgfpathlineto{\pgfqpoint{2.504739in}{2.890492in}}%
\pgfpathlineto{\pgfqpoint{2.503838in}{2.880198in}}%
\pgfpathlineto{\pgfqpoint{2.503476in}{2.869903in}}%
\pgfpathlineto{\pgfqpoint{2.503642in}{2.859609in}}%
\pgfpathlineto{\pgfqpoint{2.504326in}{2.849315in}}%
\pgfpathlineto{\pgfqpoint{2.505518in}{2.839020in}}%
\pgfpathlineto{\pgfqpoint{2.507204in}{2.828726in}}%
\pgfpathlineto{\pgfqpoint{2.509372in}{2.818431in}}%
\pgfpathlineto{\pgfqpoint{2.512003in}{2.808137in}}%
\pgfpathlineto{\pgfqpoint{2.512072in}{2.807902in}}%
\pgfpathlineto{\pgfqpoint{2.514970in}{2.797842in}}%
\pgfpathlineto{\pgfqpoint{2.518359in}{2.787548in}}%
\pgfpathlineto{\pgfqpoint{2.522152in}{2.777254in}}%
\pgfpathlineto{\pgfqpoint{2.522649in}{2.776016in}}%
\pgfpathlineto{\pgfqpoint{2.527127in}{2.766959in}}%
\pgfpathlineto{\pgfqpoint{2.533226in}{2.764232in}}%
\pgfpathlineto{\pgfqpoint{2.543803in}{2.762135in}}%
\pgfpathlineto{\pgfqpoint{2.554380in}{2.760044in}}%
\pgfpathlineto{\pgfqpoint{2.564957in}{2.757948in}}%
\pgfpathlineto{\pgfqpoint{2.571429in}{2.756665in}}%
\pgfpathlineto{\pgfqpoint{2.575534in}{2.755805in}}%
\pgfpathlineto{\pgfqpoint{2.586111in}{2.753585in}}%
\pgfpathlineto{\pgfqpoint{2.596688in}{2.751364in}}%
\pgfpathlineto{\pgfqpoint{2.607265in}{2.749154in}}%
\pgfpathlineto{\pgfqpoint{2.617842in}{2.746972in}}%
\pgfpathlineto{\pgfqpoint{2.620854in}{2.746370in}}%
\pgfpathlineto{\pgfqpoint{2.628419in}{2.744731in}}%
\pgfpathlineto{\pgfqpoint{2.638996in}{2.742522in}}%
\pgfpathlineto{\pgfqpoint{2.649573in}{2.740414in}}%
\pgfpathlineto{\pgfqpoint{2.660150in}{2.738442in}}%
\pgfpathlineto{\pgfqpoint{2.670727in}{2.736642in}}%
\pgfpathlineto{\pgfqpoint{2.674571in}{2.736076in}}%
\pgfpathclose%
\pgfusepath{fill}%
\end{pgfscope}%
\begin{pgfscope}%
\pgfpathrectangle{\pgfqpoint{1.856795in}{1.819814in}}{\pgfqpoint{1.194205in}{1.163386in}}%
\pgfusepath{clip}%
\pgfsetbuttcap%
\pgfsetroundjoin%
\definecolor{currentfill}{rgb}{0.723982,0.086887,0.339440}%
\pgfsetfillcolor{currentfill}%
\pgfsetlinewidth{0.000000pt}%
\definecolor{currentstroke}{rgb}{0.000000,0.000000,0.000000}%
\pgfsetstrokecolor{currentstroke}%
\pgfsetdash{}{0pt}%
\pgfpathmoveto{\pgfqpoint{2.067835in}{1.891932in}}%
\pgfpathlineto{\pgfqpoint{2.078412in}{1.891932in}}%
\pgfpathlineto{\pgfqpoint{2.081562in}{1.891932in}}%
\pgfpathlineto{\pgfqpoint{2.083872in}{1.902227in}}%
\pgfpathlineto{\pgfqpoint{2.085187in}{1.912521in}}%
\pgfpathlineto{\pgfqpoint{2.085600in}{1.922816in}}%
\pgfpathlineto{\pgfqpoint{2.085187in}{1.933110in}}%
\pgfpathlineto{\pgfqpoint{2.084015in}{1.943405in}}%
\pgfpathlineto{\pgfqpoint{2.082135in}{1.953699in}}%
\pgfpathlineto{\pgfqpoint{2.079591in}{1.963993in}}%
\pgfpathlineto{\pgfqpoint{2.078412in}{1.967776in}}%
\pgfpathlineto{\pgfqpoint{2.076517in}{1.974288in}}%
\pgfpathlineto{\pgfqpoint{2.072941in}{1.984582in}}%
\pgfpathlineto{\pgfqpoint{2.068793in}{1.994877in}}%
\pgfpathlineto{\pgfqpoint{2.067835in}{1.996895in}}%
\pgfpathlineto{\pgfqpoint{2.064187in}{2.005171in}}%
\pgfpathlineto{\pgfqpoint{2.058790in}{2.015466in}}%
\pgfpathlineto{\pgfqpoint{2.057258in}{2.018275in}}%
\pgfpathlineto{\pgfqpoint{2.052908in}{2.025760in}}%
\pgfpathlineto{\pgfqpoint{2.047084in}{2.036054in}}%
\pgfpathlineto{\pgfqpoint{2.046681in}{2.036803in}}%
\pgfpathlineto{\pgfqpoint{2.041271in}{2.046349in}}%
\pgfpathlineto{\pgfqpoint{2.036104in}{2.055733in}}%
\pgfpathlineto{\pgfqpoint{2.035595in}{2.056643in}}%
\pgfpathlineto{\pgfqpoint{2.030051in}{2.066938in}}%
\pgfpathlineto{\pgfqpoint{2.025527in}{2.075489in}}%
\pgfpathlineto{\pgfqpoint{2.024613in}{2.077232in}}%
\pgfpathlineto{\pgfqpoint{2.019381in}{2.087527in}}%
\pgfpathlineto{\pgfqpoint{2.014950in}{2.096494in}}%
\pgfpathlineto{\pgfqpoint{2.014307in}{2.097821in}}%
\pgfpathlineto{\pgfqpoint{2.009508in}{2.108116in}}%
\pgfpathlineto{\pgfqpoint{2.004939in}{2.118410in}}%
\pgfpathlineto{\pgfqpoint{2.004373in}{2.119745in}}%
\pgfpathlineto{\pgfqpoint{2.000676in}{2.128704in}}%
\pgfpathlineto{\pgfqpoint{1.996697in}{2.138999in}}%
\pgfpathlineto{\pgfqpoint{1.993796in}{2.147055in}}%
\pgfpathlineto{\pgfqpoint{1.993011in}{2.149293in}}%
\pgfpathlineto{\pgfqpoint{1.989619in}{2.159588in}}%
\pgfpathlineto{\pgfqpoint{1.986500in}{2.169882in}}%
\pgfpathlineto{\pgfqpoint{1.983635in}{2.180177in}}%
\pgfpathlineto{\pgfqpoint{1.983218in}{2.181763in}}%
\pgfpathlineto{\pgfqpoint{1.980991in}{2.190471in}}%
\pgfpathlineto{\pgfqpoint{1.978546in}{2.200765in}}%
\pgfpathlineto{\pgfqpoint{1.976269in}{2.211060in}}%
\pgfpathlineto{\pgfqpoint{1.974120in}{2.221354in}}%
\pgfpathlineto{\pgfqpoint{1.972641in}{2.228683in}}%
\pgfpathlineto{\pgfqpoint{1.972059in}{2.231649in}}%
\pgfpathlineto{\pgfqpoint{1.970058in}{2.241943in}}%
\pgfpathlineto{\pgfqpoint{1.968086in}{2.252238in}}%
\pgfpathlineto{\pgfqpoint{1.966100in}{2.262532in}}%
\pgfpathlineto{\pgfqpoint{1.964049in}{2.272826in}}%
\pgfpathlineto{\pgfqpoint{1.962064in}{2.282188in}}%
\pgfpathlineto{\pgfqpoint{1.961874in}{2.283121in}}%
\pgfpathlineto{\pgfqpoint{1.959574in}{2.293415in}}%
\pgfpathlineto{\pgfqpoint{1.957014in}{2.303710in}}%
\pgfpathlineto{\pgfqpoint{1.954052in}{2.314004in}}%
\pgfpathlineto{\pgfqpoint{1.951487in}{2.321693in}}%
\pgfpathlineto{\pgfqpoint{1.950594in}{2.324299in}}%
\pgfpathlineto{\pgfqpoint{1.946588in}{2.334593in}}%
\pgfpathlineto{\pgfqpoint{1.942109in}{2.344887in}}%
\pgfpathlineto{\pgfqpoint{1.940910in}{2.347511in}}%
\pgfpathlineto{\pgfqpoint{1.937165in}{2.355182in}}%
\pgfpathlineto{\pgfqpoint{1.931982in}{2.365476in}}%
\pgfpathlineto{\pgfqpoint{1.930333in}{2.368722in}}%
\pgfpathlineto{\pgfqpoint{1.930333in}{2.365476in}}%
\pgfpathlineto{\pgfqpoint{1.930333in}{2.355182in}}%
\pgfpathlineto{\pgfqpoint{1.930333in}{2.345455in}}%
\pgfpathlineto{\pgfqpoint{1.930619in}{2.344887in}}%
\pgfpathlineto{\pgfqpoint{1.935696in}{2.334593in}}%
\pgfpathlineto{\pgfqpoint{1.940423in}{2.324299in}}%
\pgfpathlineto{\pgfqpoint{1.940910in}{2.323137in}}%
\pgfpathlineto{\pgfqpoint{1.944348in}{2.314004in}}%
\pgfpathlineto{\pgfqpoint{1.947457in}{2.303710in}}%
\pgfpathlineto{\pgfqpoint{1.949885in}{2.293415in}}%
\pgfpathlineto{\pgfqpoint{1.951487in}{2.285470in}}%
\pgfpathlineto{\pgfqpoint{1.951979in}{2.283121in}}%
\pgfpathlineto{\pgfqpoint{1.953856in}{2.272826in}}%
\pgfpathlineto{\pgfqpoint{1.955602in}{2.262532in}}%
\pgfpathlineto{\pgfqpoint{1.957283in}{2.252238in}}%
\pgfpathlineto{\pgfqpoint{1.958946in}{2.241943in}}%
\pgfpathlineto{\pgfqpoint{1.960630in}{2.231649in}}%
\pgfpathlineto{\pgfqpoint{1.962064in}{2.223106in}}%
\pgfpathlineto{\pgfqpoint{1.962363in}{2.221354in}}%
\pgfpathlineto{\pgfqpoint{1.964159in}{2.211060in}}%
\pgfpathlineto{\pgfqpoint{1.966080in}{2.200765in}}%
\pgfpathlineto{\pgfqpoint{1.968164in}{2.190471in}}%
\pgfpathlineto{\pgfqpoint{1.970448in}{2.180177in}}%
\pgfpathlineto{\pgfqpoint{1.972641in}{2.171179in}}%
\pgfpathlineto{\pgfqpoint{1.972963in}{2.169882in}}%
\pgfpathlineto{\pgfqpoint{1.975719in}{2.159588in}}%
\pgfpathlineto{\pgfqpoint{1.978765in}{2.149293in}}%
\pgfpathlineto{\pgfqpoint{1.982119in}{2.138999in}}%
\pgfpathlineto{\pgfqpoint{1.983218in}{2.135876in}}%
\pgfpathlineto{\pgfqpoint{1.985788in}{2.128704in}}%
\pgfpathlineto{\pgfqpoint{1.989777in}{2.118410in}}%
\pgfpathlineto{\pgfqpoint{1.993796in}{2.108789in}}%
\pgfpathlineto{\pgfqpoint{1.994082in}{2.108116in}}%
\pgfpathlineto{\pgfqpoint{1.998703in}{2.097821in}}%
\pgfpathlineto{\pgfqpoint{2.003596in}{2.087527in}}%
\pgfpathlineto{\pgfqpoint{2.004373in}{2.085973in}}%
\pgfpathlineto{\pgfqpoint{2.008776in}{2.077232in}}%
\pgfpathlineto{\pgfqpoint{2.014179in}{2.066938in}}%
\pgfpathlineto{\pgfqpoint{2.014950in}{2.065530in}}%
\pgfpathlineto{\pgfqpoint{2.019827in}{2.056643in}}%
\pgfpathlineto{\pgfqpoint{2.025527in}{2.046606in}}%
\pgfpathlineto{\pgfqpoint{2.025672in}{2.046349in}}%
\pgfpathlineto{\pgfqpoint{2.031740in}{2.036054in}}%
\pgfpathlineto{\pgfqpoint{2.036104in}{2.028944in}}%
\pgfpathlineto{\pgfqpoint{2.038023in}{2.025760in}}%
\pgfpathlineto{\pgfqpoint{2.044549in}{2.015466in}}%
\pgfpathlineto{\pgfqpoint{2.046681in}{2.012267in}}%
\pgfpathlineto{\pgfqpoint{2.050875in}{2.005171in}}%
\pgfpathlineto{\pgfqpoint{2.055353in}{1.994877in}}%
\pgfpathlineto{\pgfqpoint{2.057258in}{1.989221in}}%
\pgfpathlineto{\pgfqpoint{2.058860in}{1.984582in}}%
\pgfpathlineto{\pgfqpoint{2.061954in}{1.974288in}}%
\pgfpathlineto{\pgfqpoint{2.064580in}{1.963993in}}%
\pgfpathlineto{\pgfqpoint{2.066667in}{1.953699in}}%
\pgfpathlineto{\pgfqpoint{2.067835in}{1.945571in}}%
\pgfpathlineto{\pgfqpoint{2.068163in}{1.943405in}}%
\pgfpathlineto{\pgfqpoint{2.069010in}{1.933110in}}%
\pgfpathlineto{\pgfqpoint{2.069096in}{1.922816in}}%
\pgfpathlineto{\pgfqpoint{2.068344in}{1.912521in}}%
\pgfpathlineto{\pgfqpoint{2.067835in}{1.909344in}}%
\pgfpathlineto{\pgfqpoint{2.066700in}{1.902227in}}%
\pgfpathlineto{\pgfqpoint{2.064067in}{1.891932in}}%
\pgfpathclose%
\pgfusepath{fill}%
\end{pgfscope}%
\begin{pgfscope}%
\pgfpathrectangle{\pgfqpoint{1.856795in}{1.819814in}}{\pgfqpoint{1.194205in}{1.163386in}}%
\pgfusepath{clip}%
\pgfsetbuttcap%
\pgfsetroundjoin%
\definecolor{currentfill}{rgb}{0.723982,0.086887,0.339440}%
\pgfsetfillcolor{currentfill}%
\pgfsetlinewidth{0.000000pt}%
\definecolor{currentstroke}{rgb}{0.000000,0.000000,0.000000}%
\pgfsetstrokecolor{currentstroke}%
\pgfsetdash{}{0pt}%
\pgfpathmoveto{\pgfqpoint{2.713036in}{2.704233in}}%
\pgfpathlineto{\pgfqpoint{2.723613in}{2.702936in}}%
\pgfpathlineto{\pgfqpoint{2.734190in}{2.702355in}}%
\pgfpathlineto{\pgfqpoint{2.744767in}{2.702632in}}%
\pgfpathlineto{\pgfqpoint{2.755344in}{2.703928in}}%
\pgfpathlineto{\pgfqpoint{2.760785in}{2.705193in}}%
\pgfpathlineto{\pgfqpoint{2.765921in}{2.706287in}}%
\pgfpathlineto{\pgfqpoint{2.776498in}{2.709737in}}%
\pgfpathlineto{\pgfqpoint{2.787075in}{2.714579in}}%
\pgfpathlineto{\pgfqpoint{2.788620in}{2.715487in}}%
\pgfpathlineto{\pgfqpoint{2.797652in}{2.720368in}}%
\pgfpathlineto{\pgfqpoint{2.805764in}{2.725781in}}%
\pgfpathlineto{\pgfqpoint{2.808229in}{2.727302in}}%
\pgfpathlineto{\pgfqpoint{2.818806in}{2.734962in}}%
\pgfpathlineto{\pgfqpoint{2.820154in}{2.736076in}}%
\pgfpathlineto{\pgfqpoint{2.829383in}{2.743107in}}%
\pgfpathlineto{\pgfqpoint{2.833238in}{2.746370in}}%
\pgfpathlineto{\pgfqpoint{2.839960in}{2.751631in}}%
\pgfpathlineto{\pgfqpoint{2.845889in}{2.756665in}}%
\pgfpathlineto{\pgfqpoint{2.850538in}{2.760326in}}%
\pgfpathlineto{\pgfqpoint{2.858448in}{2.766959in}}%
\pgfpathlineto{\pgfqpoint{2.861115in}{2.769041in}}%
\pgfpathlineto{\pgfqpoint{2.871150in}{2.777254in}}%
\pgfpathlineto{\pgfqpoint{2.871692in}{2.777668in}}%
\pgfpathlineto{\pgfqpoint{2.882269in}{2.786070in}}%
\pgfpathlineto{\pgfqpoint{2.884074in}{2.787548in}}%
\pgfpathlineto{\pgfqpoint{2.892846in}{2.794268in}}%
\pgfpathlineto{\pgfqpoint{2.897412in}{2.797842in}}%
\pgfpathlineto{\pgfqpoint{2.903423in}{2.802268in}}%
\pgfpathlineto{\pgfqpoint{2.911268in}{2.808137in}}%
\pgfpathlineto{\pgfqpoint{2.914000in}{2.810068in}}%
\pgfpathlineto{\pgfqpoint{2.924577in}{2.817669in}}%
\pgfpathlineto{\pgfqpoint{2.925611in}{2.818431in}}%
\pgfpathlineto{\pgfqpoint{2.935154in}{2.825095in}}%
\pgfpathlineto{\pgfqpoint{2.940234in}{2.828726in}}%
\pgfpathlineto{\pgfqpoint{2.945731in}{2.832465in}}%
\pgfpathlineto{\pgfqpoint{2.955050in}{2.839020in}}%
\pgfpathlineto{\pgfqpoint{2.956308in}{2.839867in}}%
\pgfpathlineto{\pgfqpoint{2.966885in}{2.847399in}}%
\pgfpathlineto{\pgfqpoint{2.969393in}{2.849315in}}%
\pgfpathlineto{\pgfqpoint{2.977462in}{2.855296in}}%
\pgfpathlineto{\pgfqpoint{2.977462in}{2.859609in}}%
\pgfpathlineto{\pgfqpoint{2.977462in}{2.869731in}}%
\pgfpathlineto{\pgfqpoint{2.966885in}{2.862170in}}%
\pgfpathlineto{\pgfqpoint{2.962943in}{2.859609in}}%
\pgfpathlineto{\pgfqpoint{2.956308in}{2.855140in}}%
\pgfpathlineto{\pgfqpoint{2.947165in}{2.849315in}}%
\pgfpathlineto{\pgfqpoint{2.945731in}{2.848363in}}%
\pgfpathlineto{\pgfqpoint{2.935154in}{2.841664in}}%
\pgfpathlineto{\pgfqpoint{2.930897in}{2.839020in}}%
\pgfpathlineto{\pgfqpoint{2.924577in}{2.834930in}}%
\pgfpathlineto{\pgfqpoint{2.914909in}{2.828726in}}%
\pgfpathlineto{\pgfqpoint{2.914000in}{2.828116in}}%
\pgfpathlineto{\pgfqpoint{2.903423in}{2.821152in}}%
\pgfpathlineto{\pgfqpoint{2.899265in}{2.818431in}}%
\pgfpathlineto{\pgfqpoint{2.892846in}{2.814036in}}%
\pgfpathlineto{\pgfqpoint{2.884147in}{2.808137in}}%
\pgfpathlineto{\pgfqpoint{2.882269in}{2.806798in}}%
\pgfpathlineto{\pgfqpoint{2.871692in}{2.799423in}}%
\pgfpathlineto{\pgfqpoint{2.869367in}{2.797842in}}%
\pgfpathlineto{\pgfqpoint{2.861115in}{2.791944in}}%
\pgfpathlineto{\pgfqpoint{2.854750in}{2.787548in}}%
\pgfpathlineto{\pgfqpoint{2.850538in}{2.784478in}}%
\pgfpathlineto{\pgfqpoint{2.840128in}{2.777254in}}%
\pgfpathlineto{\pgfqpoint{2.839960in}{2.777130in}}%
\pgfpathlineto{\pgfqpoint{2.829383in}{2.769862in}}%
\pgfpathlineto{\pgfqpoint{2.824806in}{2.766959in}}%
\pgfpathlineto{\pgfqpoint{2.818806in}{2.762941in}}%
\pgfpathlineto{\pgfqpoint{2.808393in}{2.756665in}}%
\pgfpathlineto{\pgfqpoint{2.808229in}{2.756560in}}%
\pgfpathlineto{\pgfqpoint{2.797652in}{2.750611in}}%
\pgfpathlineto{\pgfqpoint{2.788767in}{2.746370in}}%
\pgfpathlineto{\pgfqpoint{2.787075in}{2.745519in}}%
\pgfpathlineto{\pgfqpoint{2.776498in}{2.741101in}}%
\pgfpathlineto{\pgfqpoint{2.765921in}{2.737652in}}%
\pgfpathlineto{\pgfqpoint{2.759456in}{2.736076in}}%
\pgfpathlineto{\pgfqpoint{2.755344in}{2.735023in}}%
\pgfpathlineto{\pgfqpoint{2.744767in}{2.733147in}}%
\pgfpathlineto{\pgfqpoint{2.734190in}{2.732055in}}%
\pgfpathlineto{\pgfqpoint{2.723613in}{2.731641in}}%
\pgfpathlineto{\pgfqpoint{2.713036in}{2.731810in}}%
\pgfpathlineto{\pgfqpoint{2.702459in}{2.732472in}}%
\pgfpathlineto{\pgfqpoint{2.691882in}{2.733550in}}%
\pgfpathlineto{\pgfqpoint{2.681305in}{2.734976in}}%
\pgfpathlineto{\pgfqpoint{2.674571in}{2.736076in}}%
\pgfpathlineto{\pgfqpoint{2.670727in}{2.736642in}}%
\pgfpathlineto{\pgfqpoint{2.660150in}{2.738442in}}%
\pgfpathlineto{\pgfqpoint{2.649573in}{2.740414in}}%
\pgfpathlineto{\pgfqpoint{2.638996in}{2.742522in}}%
\pgfpathlineto{\pgfqpoint{2.628419in}{2.744731in}}%
\pgfpathlineto{\pgfqpoint{2.620854in}{2.746370in}}%
\pgfpathlineto{\pgfqpoint{2.617842in}{2.746972in}}%
\pgfpathlineto{\pgfqpoint{2.607265in}{2.749154in}}%
\pgfpathlineto{\pgfqpoint{2.596688in}{2.751364in}}%
\pgfpathlineto{\pgfqpoint{2.586111in}{2.753585in}}%
\pgfpathlineto{\pgfqpoint{2.575534in}{2.755805in}}%
\pgfpathlineto{\pgfqpoint{2.571429in}{2.756665in}}%
\pgfpathlineto{\pgfqpoint{2.564957in}{2.757948in}}%
\pgfpathlineto{\pgfqpoint{2.554380in}{2.760044in}}%
\pgfpathlineto{\pgfqpoint{2.543803in}{2.762135in}}%
\pgfpathlineto{\pgfqpoint{2.533226in}{2.764232in}}%
\pgfpathlineto{\pgfqpoint{2.527127in}{2.766959in}}%
\pgfpathlineto{\pgfqpoint{2.522649in}{2.776016in}}%
\pgfpathlineto{\pgfqpoint{2.522152in}{2.777254in}}%
\pgfpathlineto{\pgfqpoint{2.518359in}{2.787548in}}%
\pgfpathlineto{\pgfqpoint{2.514970in}{2.797842in}}%
\pgfpathlineto{\pgfqpoint{2.512072in}{2.807902in}}%
\pgfpathlineto{\pgfqpoint{2.512003in}{2.808137in}}%
\pgfpathlineto{\pgfqpoint{2.509372in}{2.818431in}}%
\pgfpathlineto{\pgfqpoint{2.507204in}{2.828726in}}%
\pgfpathlineto{\pgfqpoint{2.505518in}{2.839020in}}%
\pgfpathlineto{\pgfqpoint{2.504326in}{2.849315in}}%
\pgfpathlineto{\pgfqpoint{2.503642in}{2.859609in}}%
\pgfpathlineto{\pgfqpoint{2.503476in}{2.869903in}}%
\pgfpathlineto{\pgfqpoint{2.503838in}{2.880198in}}%
\pgfpathlineto{\pgfqpoint{2.504739in}{2.890492in}}%
\pgfpathlineto{\pgfqpoint{2.506191in}{2.900787in}}%
\pgfpathlineto{\pgfqpoint{2.508208in}{2.911081in}}%
\pgfpathlineto{\pgfqpoint{2.501495in}{2.911081in}}%
\pgfpathlineto{\pgfqpoint{2.490917in}{2.911081in}}%
\pgfpathlineto{\pgfqpoint{2.486227in}{2.911081in}}%
\pgfpathlineto{\pgfqpoint{2.483486in}{2.900787in}}%
\pgfpathlineto{\pgfqpoint{2.481322in}{2.890492in}}%
\pgfpathlineto{\pgfqpoint{2.480340in}{2.884114in}}%
\pgfpathlineto{\pgfqpoint{2.479687in}{2.880198in}}%
\pgfpathlineto{\pgfqpoint{2.478586in}{2.869903in}}%
\pgfpathlineto{\pgfqpoint{2.478068in}{2.859609in}}%
\pgfpathlineto{\pgfqpoint{2.478127in}{2.849315in}}%
\pgfpathlineto{\pgfqpoint{2.478751in}{2.839020in}}%
\pgfpathlineto{\pgfqpoint{2.479930in}{2.828726in}}%
\pgfpathlineto{\pgfqpoint{2.480340in}{2.826222in}}%
\pgfpathlineto{\pgfqpoint{2.481582in}{2.818431in}}%
\pgfpathlineto{\pgfqpoint{2.483715in}{2.808137in}}%
\pgfpathlineto{\pgfqpoint{2.486336in}{2.797842in}}%
\pgfpathlineto{\pgfqpoint{2.489434in}{2.787548in}}%
\pgfpathlineto{\pgfqpoint{2.490917in}{2.783267in}}%
\pgfpathlineto{\pgfqpoint{2.492951in}{2.777254in}}%
\pgfpathlineto{\pgfqpoint{2.497071in}{2.766959in}}%
\pgfpathlineto{\pgfqpoint{2.501495in}{2.760198in}}%
\pgfpathlineto{\pgfqpoint{2.504451in}{2.756665in}}%
\pgfpathlineto{\pgfqpoint{2.512072in}{2.753348in}}%
\pgfpathlineto{\pgfqpoint{2.522649in}{2.750589in}}%
\pgfpathlineto{\pgfqpoint{2.533226in}{2.748277in}}%
\pgfpathlineto{\pgfqpoint{2.542189in}{2.746370in}}%
\pgfpathlineto{\pgfqpoint{2.543803in}{2.746005in}}%
\pgfpathlineto{\pgfqpoint{2.554380in}{2.743610in}}%
\pgfpathlineto{\pgfqpoint{2.564957in}{2.741178in}}%
\pgfpathlineto{\pgfqpoint{2.575534in}{2.738703in}}%
\pgfpathlineto{\pgfqpoint{2.586111in}{2.736185in}}%
\pgfpathlineto{\pgfqpoint{2.586568in}{2.736076in}}%
\pgfpathlineto{\pgfqpoint{2.596688in}{2.733445in}}%
\pgfpathlineto{\pgfqpoint{2.607265in}{2.730664in}}%
\pgfpathlineto{\pgfqpoint{2.617842in}{2.727871in}}%
\pgfpathlineto{\pgfqpoint{2.625813in}{2.725781in}}%
\pgfpathlineto{\pgfqpoint{2.628419in}{2.725022in}}%
\pgfpathlineto{\pgfqpoint{2.638996in}{2.722016in}}%
\pgfpathlineto{\pgfqpoint{2.649573in}{2.719077in}}%
\pgfpathlineto{\pgfqpoint{2.660150in}{2.716247in}}%
\pgfpathlineto{\pgfqpoint{2.663201in}{2.715487in}}%
\pgfpathlineto{\pgfqpoint{2.670727in}{2.713355in}}%
\pgfpathlineto{\pgfqpoint{2.681305in}{2.710602in}}%
\pgfpathlineto{\pgfqpoint{2.691882in}{2.708134in}}%
\pgfpathlineto{\pgfqpoint{2.702459in}{2.706021in}}%
\pgfpathlineto{\pgfqpoint{2.707756in}{2.705193in}}%
\pgfpathclose%
\pgfusepath{fill}%
\end{pgfscope}%
\begin{pgfscope}%
\pgfpathrectangle{\pgfqpoint{1.856795in}{1.819814in}}{\pgfqpoint{1.194205in}{1.163386in}}%
\pgfusepath{clip}%
\pgfsetbuttcap%
\pgfsetroundjoin%
\definecolor{currentfill}{rgb}{0.796501,0.105066,0.310630}%
\pgfsetfillcolor{currentfill}%
\pgfsetlinewidth{0.000000pt}%
\definecolor{currentstroke}{rgb}{0.000000,0.000000,0.000000}%
\pgfsetstrokecolor{currentstroke}%
\pgfsetdash{}{0pt}%
\pgfpathmoveto{\pgfqpoint{2.088989in}{1.891932in}}%
\pgfpathlineto{\pgfqpoint{2.099566in}{1.891932in}}%
\pgfpathlineto{\pgfqpoint{2.100143in}{1.891932in}}%
\pgfpathlineto{\pgfqpoint{2.102181in}{1.902227in}}%
\pgfpathlineto{\pgfqpoint{2.103212in}{1.912521in}}%
\pgfpathlineto{\pgfqpoint{2.103321in}{1.922816in}}%
\pgfpathlineto{\pgfqpoint{2.102574in}{1.933110in}}%
\pgfpathlineto{\pgfqpoint{2.101016in}{1.943405in}}%
\pgfpathlineto{\pgfqpoint{2.099566in}{1.949766in}}%
\pgfpathlineto{\pgfqpoint{2.098731in}{1.953699in}}%
\pgfpathlineto{\pgfqpoint{2.095822in}{1.963993in}}%
\pgfpathlineto{\pgfqpoint{2.092206in}{1.974288in}}%
\pgfpathlineto{\pgfqpoint{2.088989in}{1.981892in}}%
\pgfpathlineto{\pgfqpoint{2.087939in}{1.984582in}}%
\pgfpathlineto{\pgfqpoint{2.083231in}{1.994877in}}%
\pgfpathlineto{\pgfqpoint{2.078412in}{2.004209in}}%
\pgfpathlineto{\pgfqpoint{2.077951in}{2.005171in}}%
\pgfpathlineto{\pgfqpoint{2.072593in}{2.015466in}}%
\pgfpathlineto{\pgfqpoint{2.067835in}{2.024336in}}%
\pgfpathlineto{\pgfqpoint{2.067081in}{2.025760in}}%
\pgfpathlineto{\pgfqpoint{2.061758in}{2.036054in}}%
\pgfpathlineto{\pgfqpoint{2.057258in}{2.045083in}}%
\pgfpathlineto{\pgfqpoint{2.056591in}{2.046349in}}%
\pgfpathlineto{\pgfqpoint{2.051447in}{2.056643in}}%
\pgfpathlineto{\pgfqpoint{2.046681in}{2.066231in}}%
\pgfpathlineto{\pgfqpoint{2.046313in}{2.066938in}}%
\pgfpathlineto{\pgfqpoint{2.041036in}{2.077232in}}%
\pgfpathlineto{\pgfqpoint{2.036104in}{2.086634in}}%
\pgfpathlineto{\pgfqpoint{2.035645in}{2.087527in}}%
\pgfpathlineto{\pgfqpoint{2.030395in}{2.097821in}}%
\pgfpathlineto{\pgfqpoint{2.025527in}{2.107536in}}%
\pgfpathlineto{\pgfqpoint{2.025246in}{2.108116in}}%
\pgfpathlineto{\pgfqpoint{2.020372in}{2.118410in}}%
\pgfpathlineto{\pgfqpoint{2.015720in}{2.128704in}}%
\pgfpathlineto{\pgfqpoint{2.014950in}{2.130478in}}%
\pgfpathlineto{\pgfqpoint{2.011374in}{2.138999in}}%
\pgfpathlineto{\pgfqpoint{2.007306in}{2.149293in}}%
\pgfpathlineto{\pgfqpoint{2.004373in}{2.157208in}}%
\pgfpathlineto{\pgfqpoint{2.003520in}{2.159588in}}%
\pgfpathlineto{\pgfqpoint{2.000024in}{2.169882in}}%
\pgfpathlineto{\pgfqpoint{1.996779in}{2.180177in}}%
\pgfpathlineto{\pgfqpoint{1.993796in}{2.190340in}}%
\pgfpathlineto{\pgfqpoint{1.993758in}{2.190471in}}%
\pgfpathlineto{\pgfqpoint{1.990963in}{2.200765in}}%
\pgfpathlineto{\pgfqpoint{1.988342in}{2.211060in}}%
\pgfpathlineto{\pgfqpoint{1.985858in}{2.221354in}}%
\pgfpathlineto{\pgfqpoint{1.983469in}{2.231649in}}%
\pgfpathlineto{\pgfqpoint{1.983218in}{2.232735in}}%
\pgfpathlineto{\pgfqpoint{1.981180in}{2.241943in}}%
\pgfpathlineto{\pgfqpoint{1.978935in}{2.252238in}}%
\pgfpathlineto{\pgfqpoint{1.976686in}{2.262532in}}%
\pgfpathlineto{\pgfqpoint{1.974385in}{2.272826in}}%
\pgfpathlineto{\pgfqpoint{1.972641in}{2.280272in}}%
\pgfpathlineto{\pgfqpoint{1.972004in}{2.283121in}}%
\pgfpathlineto{\pgfqpoint{1.969554in}{2.293415in}}%
\pgfpathlineto{\pgfqpoint{1.966908in}{2.303710in}}%
\pgfpathlineto{\pgfqpoint{1.963986in}{2.314004in}}%
\pgfpathlineto{\pgfqpoint{1.962064in}{2.320122in}}%
\pgfpathlineto{\pgfqpoint{1.960762in}{2.324299in}}%
\pgfpathlineto{\pgfqpoint{1.957184in}{2.334593in}}%
\pgfpathlineto{\pgfqpoint{1.953184in}{2.344887in}}%
\pgfpathlineto{\pgfqpoint{1.951487in}{2.348972in}}%
\pgfpathlineto{\pgfqpoint{1.948783in}{2.355182in}}%
\pgfpathlineto{\pgfqpoint{1.944067in}{2.365476in}}%
\pgfpathlineto{\pgfqpoint{1.940910in}{2.372132in}}%
\pgfpathlineto{\pgfqpoint{1.939105in}{2.375771in}}%
\pgfpathlineto{\pgfqpoint{1.933976in}{2.386065in}}%
\pgfpathlineto{\pgfqpoint{1.930333in}{2.393354in}}%
\pgfpathlineto{\pgfqpoint{1.930333in}{2.386065in}}%
\pgfpathlineto{\pgfqpoint{1.930333in}{2.375771in}}%
\pgfpathlineto{\pgfqpoint{1.930333in}{2.368722in}}%
\pgfpathlineto{\pgfqpoint{1.931982in}{2.365476in}}%
\pgfpathlineto{\pgfqpoint{1.937165in}{2.355182in}}%
\pgfpathlineto{\pgfqpoint{1.940910in}{2.347511in}}%
\pgfpathlineto{\pgfqpoint{1.942109in}{2.344887in}}%
\pgfpathlineto{\pgfqpoint{1.946588in}{2.334593in}}%
\pgfpathlineto{\pgfqpoint{1.950594in}{2.324299in}}%
\pgfpathlineto{\pgfqpoint{1.951487in}{2.321693in}}%
\pgfpathlineto{\pgfqpoint{1.954052in}{2.314004in}}%
\pgfpathlineto{\pgfqpoint{1.957014in}{2.303710in}}%
\pgfpathlineto{\pgfqpoint{1.959574in}{2.293415in}}%
\pgfpathlineto{\pgfqpoint{1.961874in}{2.283121in}}%
\pgfpathlineto{\pgfqpoint{1.962064in}{2.282188in}}%
\pgfpathlineto{\pgfqpoint{1.964049in}{2.272826in}}%
\pgfpathlineto{\pgfqpoint{1.966100in}{2.262532in}}%
\pgfpathlineto{\pgfqpoint{1.968086in}{2.252238in}}%
\pgfpathlineto{\pgfqpoint{1.970058in}{2.241943in}}%
\pgfpathlineto{\pgfqpoint{1.972059in}{2.231649in}}%
\pgfpathlineto{\pgfqpoint{1.972641in}{2.228683in}}%
\pgfpathlineto{\pgfqpoint{1.974120in}{2.221354in}}%
\pgfpathlineto{\pgfqpoint{1.976269in}{2.211060in}}%
\pgfpathlineto{\pgfqpoint{1.978546in}{2.200765in}}%
\pgfpathlineto{\pgfqpoint{1.980991in}{2.190471in}}%
\pgfpathlineto{\pgfqpoint{1.983218in}{2.181763in}}%
\pgfpathlineto{\pgfqpoint{1.983635in}{2.180177in}}%
\pgfpathlineto{\pgfqpoint{1.986500in}{2.169882in}}%
\pgfpathlineto{\pgfqpoint{1.989619in}{2.159588in}}%
\pgfpathlineto{\pgfqpoint{1.993011in}{2.149293in}}%
\pgfpathlineto{\pgfqpoint{1.993796in}{2.147055in}}%
\pgfpathlineto{\pgfqpoint{1.996697in}{2.138999in}}%
\pgfpathlineto{\pgfqpoint{2.000676in}{2.128704in}}%
\pgfpathlineto{\pgfqpoint{2.004373in}{2.119745in}}%
\pgfpathlineto{\pgfqpoint{2.004939in}{2.118410in}}%
\pgfpathlineto{\pgfqpoint{2.009508in}{2.108116in}}%
\pgfpathlineto{\pgfqpoint{2.014307in}{2.097821in}}%
\pgfpathlineto{\pgfqpoint{2.014950in}{2.096494in}}%
\pgfpathlineto{\pgfqpoint{2.019381in}{2.087527in}}%
\pgfpathlineto{\pgfqpoint{2.024613in}{2.077232in}}%
\pgfpathlineto{\pgfqpoint{2.025527in}{2.075489in}}%
\pgfpathlineto{\pgfqpoint{2.030051in}{2.066938in}}%
\pgfpathlineto{\pgfqpoint{2.035595in}{2.056643in}}%
\pgfpathlineto{\pgfqpoint{2.036104in}{2.055733in}}%
\pgfpathlineto{\pgfqpoint{2.041271in}{2.046349in}}%
\pgfpathlineto{\pgfqpoint{2.046681in}{2.036803in}}%
\pgfpathlineto{\pgfqpoint{2.047084in}{2.036054in}}%
\pgfpathlineto{\pgfqpoint{2.052908in}{2.025760in}}%
\pgfpathlineto{\pgfqpoint{2.057258in}{2.018275in}}%
\pgfpathlineto{\pgfqpoint{2.058790in}{2.015466in}}%
\pgfpathlineto{\pgfqpoint{2.064187in}{2.005171in}}%
\pgfpathlineto{\pgfqpoint{2.067835in}{1.996895in}}%
\pgfpathlineto{\pgfqpoint{2.068793in}{1.994877in}}%
\pgfpathlineto{\pgfqpoint{2.072941in}{1.984582in}}%
\pgfpathlineto{\pgfqpoint{2.076517in}{1.974288in}}%
\pgfpathlineto{\pgfqpoint{2.078412in}{1.967776in}}%
\pgfpathlineto{\pgfqpoint{2.079591in}{1.963993in}}%
\pgfpathlineto{\pgfqpoint{2.082135in}{1.953699in}}%
\pgfpathlineto{\pgfqpoint{2.084015in}{1.943405in}}%
\pgfpathlineto{\pgfqpoint{2.085187in}{1.933110in}}%
\pgfpathlineto{\pgfqpoint{2.085600in}{1.922816in}}%
\pgfpathlineto{\pgfqpoint{2.085187in}{1.912521in}}%
\pgfpathlineto{\pgfqpoint{2.083872in}{1.902227in}}%
\pgfpathlineto{\pgfqpoint{2.081562in}{1.891932in}}%
\pgfpathclose%
\pgfusepath{fill}%
\end{pgfscope}%
\begin{pgfscope}%
\pgfpathrectangle{\pgfqpoint{1.856795in}{1.819814in}}{\pgfqpoint{1.194205in}{1.163386in}}%
\pgfusepath{clip}%
\pgfsetbuttcap%
\pgfsetroundjoin%
\definecolor{currentfill}{rgb}{0.796501,0.105066,0.310630}%
\pgfsetfillcolor{currentfill}%
\pgfsetlinewidth{0.000000pt}%
\definecolor{currentstroke}{rgb}{0.000000,0.000000,0.000000}%
\pgfsetstrokecolor{currentstroke}%
\pgfsetdash{}{0pt}%
\pgfpathmoveto{\pgfqpoint{2.734190in}{2.652654in}}%
\pgfpathlineto{\pgfqpoint{2.744767in}{2.650421in}}%
\pgfpathlineto{\pgfqpoint{2.755344in}{2.650557in}}%
\pgfpathlineto{\pgfqpoint{2.765921in}{2.653460in}}%
\pgfpathlineto{\pgfqpoint{2.766377in}{2.653720in}}%
\pgfpathlineto{\pgfqpoint{2.776498in}{2.658419in}}%
\pgfpathlineto{\pgfqpoint{2.783897in}{2.664015in}}%
\pgfpathlineto{\pgfqpoint{2.787075in}{2.666039in}}%
\pgfpathlineto{\pgfqpoint{2.796226in}{2.674309in}}%
\pgfpathlineto{\pgfqpoint{2.797652in}{2.675420in}}%
\pgfpathlineto{\pgfqpoint{2.806703in}{2.684604in}}%
\pgfpathlineto{\pgfqpoint{2.808229in}{2.685957in}}%
\pgfpathlineto{\pgfqpoint{2.816437in}{2.694898in}}%
\pgfpathlineto{\pgfqpoint{2.818806in}{2.697176in}}%
\pgfpathlineto{\pgfqpoint{2.825901in}{2.705193in}}%
\pgfpathlineto{\pgfqpoint{2.829383in}{2.708691in}}%
\pgfpathlineto{\pgfqpoint{2.835339in}{2.715487in}}%
\pgfpathlineto{\pgfqpoint{2.839960in}{2.720207in}}%
\pgfpathlineto{\pgfqpoint{2.844898in}{2.725781in}}%
\pgfpathlineto{\pgfqpoint{2.850538in}{2.731513in}}%
\pgfpathlineto{\pgfqpoint{2.854685in}{2.736076in}}%
\pgfpathlineto{\pgfqpoint{2.861115in}{2.742479in}}%
\pgfpathlineto{\pgfqpoint{2.864787in}{2.746370in}}%
\pgfpathlineto{\pgfqpoint{2.871692in}{2.753032in}}%
\pgfpathlineto{\pgfqpoint{2.875279in}{2.756665in}}%
\pgfpathlineto{\pgfqpoint{2.882269in}{2.763146in}}%
\pgfpathlineto{\pgfqpoint{2.886229in}{2.766959in}}%
\pgfpathlineto{\pgfqpoint{2.892846in}{2.772827in}}%
\pgfpathlineto{\pgfqpoint{2.897691in}{2.777254in}}%
\pgfpathlineto{\pgfqpoint{2.903423in}{2.782103in}}%
\pgfpathlineto{\pgfqpoint{2.909706in}{2.787548in}}%
\pgfpathlineto{\pgfqpoint{2.914000in}{2.791014in}}%
\pgfpathlineto{\pgfqpoint{2.922282in}{2.797842in}}%
\pgfpathlineto{\pgfqpoint{2.924577in}{2.799615in}}%
\pgfpathlineto{\pgfqpoint{2.935154in}{2.807960in}}%
\pgfpathlineto{\pgfqpoint{2.935371in}{2.808137in}}%
\pgfpathlineto{\pgfqpoint{2.945731in}{2.816072in}}%
\pgfpathlineto{\pgfqpoint{2.948716in}{2.818431in}}%
\pgfpathlineto{\pgfqpoint{2.956308in}{2.824124in}}%
\pgfpathlineto{\pgfqpoint{2.962193in}{2.828726in}}%
\pgfpathlineto{\pgfqpoint{2.966885in}{2.832234in}}%
\pgfpathlineto{\pgfqpoint{2.975451in}{2.839020in}}%
\pgfpathlineto{\pgfqpoint{2.977462in}{2.840560in}}%
\pgfpathlineto{\pgfqpoint{2.977462in}{2.849315in}}%
\pgfpathlineto{\pgfqpoint{2.977462in}{2.855296in}}%
\pgfpathlineto{\pgfqpoint{2.969393in}{2.849315in}}%
\pgfpathlineto{\pgfqpoint{2.966885in}{2.847399in}}%
\pgfpathlineto{\pgfqpoint{2.956308in}{2.839867in}}%
\pgfpathlineto{\pgfqpoint{2.955050in}{2.839020in}}%
\pgfpathlineto{\pgfqpoint{2.945731in}{2.832465in}}%
\pgfpathlineto{\pgfqpoint{2.940234in}{2.828726in}}%
\pgfpathlineto{\pgfqpoint{2.935154in}{2.825095in}}%
\pgfpathlineto{\pgfqpoint{2.925611in}{2.818431in}}%
\pgfpathlineto{\pgfqpoint{2.924577in}{2.817669in}}%
\pgfpathlineto{\pgfqpoint{2.914000in}{2.810068in}}%
\pgfpathlineto{\pgfqpoint{2.911268in}{2.808137in}}%
\pgfpathlineto{\pgfqpoint{2.903423in}{2.802268in}}%
\pgfpathlineto{\pgfqpoint{2.897412in}{2.797842in}}%
\pgfpathlineto{\pgfqpoint{2.892846in}{2.794268in}}%
\pgfpathlineto{\pgfqpoint{2.884074in}{2.787548in}}%
\pgfpathlineto{\pgfqpoint{2.882269in}{2.786070in}}%
\pgfpathlineto{\pgfqpoint{2.871692in}{2.777668in}}%
\pgfpathlineto{\pgfqpoint{2.871150in}{2.777254in}}%
\pgfpathlineto{\pgfqpoint{2.861115in}{2.769041in}}%
\pgfpathlineto{\pgfqpoint{2.858448in}{2.766959in}}%
\pgfpathlineto{\pgfqpoint{2.850538in}{2.760326in}}%
\pgfpathlineto{\pgfqpoint{2.845889in}{2.756665in}}%
\pgfpathlineto{\pgfqpoint{2.839960in}{2.751631in}}%
\pgfpathlineto{\pgfqpoint{2.833238in}{2.746370in}}%
\pgfpathlineto{\pgfqpoint{2.829383in}{2.743107in}}%
\pgfpathlineto{\pgfqpoint{2.820154in}{2.736076in}}%
\pgfpathlineto{\pgfqpoint{2.818806in}{2.734962in}}%
\pgfpathlineto{\pgfqpoint{2.808229in}{2.727302in}}%
\pgfpathlineto{\pgfqpoint{2.805764in}{2.725781in}}%
\pgfpathlineto{\pgfqpoint{2.797652in}{2.720368in}}%
\pgfpathlineto{\pgfqpoint{2.788620in}{2.715487in}}%
\pgfpathlineto{\pgfqpoint{2.787075in}{2.714579in}}%
\pgfpathlineto{\pgfqpoint{2.776498in}{2.709737in}}%
\pgfpathlineto{\pgfqpoint{2.765921in}{2.706287in}}%
\pgfpathlineto{\pgfqpoint{2.760785in}{2.705193in}}%
\pgfpathlineto{\pgfqpoint{2.755344in}{2.703928in}}%
\pgfpathlineto{\pgfqpoint{2.744767in}{2.702632in}}%
\pgfpathlineto{\pgfqpoint{2.734190in}{2.702355in}}%
\pgfpathlineto{\pgfqpoint{2.723613in}{2.702936in}}%
\pgfpathlineto{\pgfqpoint{2.713036in}{2.704233in}}%
\pgfpathlineto{\pgfqpoint{2.707756in}{2.705193in}}%
\pgfpathlineto{\pgfqpoint{2.702459in}{2.706021in}}%
\pgfpathlineto{\pgfqpoint{2.691882in}{2.708134in}}%
\pgfpathlineto{\pgfqpoint{2.681305in}{2.710602in}}%
\pgfpathlineto{\pgfqpoint{2.670727in}{2.713355in}}%
\pgfpathlineto{\pgfqpoint{2.663201in}{2.715487in}}%
\pgfpathlineto{\pgfqpoint{2.660150in}{2.716247in}}%
\pgfpathlineto{\pgfqpoint{2.649573in}{2.719077in}}%
\pgfpathlineto{\pgfqpoint{2.638996in}{2.722016in}}%
\pgfpathlineto{\pgfqpoint{2.628419in}{2.725022in}}%
\pgfpathlineto{\pgfqpoint{2.625813in}{2.725781in}}%
\pgfpathlineto{\pgfqpoint{2.617842in}{2.727871in}}%
\pgfpathlineto{\pgfqpoint{2.607265in}{2.730664in}}%
\pgfpathlineto{\pgfqpoint{2.596688in}{2.733445in}}%
\pgfpathlineto{\pgfqpoint{2.586568in}{2.736076in}}%
\pgfpathlineto{\pgfqpoint{2.586111in}{2.736185in}}%
\pgfpathlineto{\pgfqpoint{2.575534in}{2.738703in}}%
\pgfpathlineto{\pgfqpoint{2.564957in}{2.741178in}}%
\pgfpathlineto{\pgfqpoint{2.554380in}{2.743610in}}%
\pgfpathlineto{\pgfqpoint{2.543803in}{2.746005in}}%
\pgfpathlineto{\pgfqpoint{2.542189in}{2.746370in}}%
\pgfpathlineto{\pgfqpoint{2.533226in}{2.748277in}}%
\pgfpathlineto{\pgfqpoint{2.522649in}{2.750589in}}%
\pgfpathlineto{\pgfqpoint{2.512072in}{2.753348in}}%
\pgfpathlineto{\pgfqpoint{2.504451in}{2.756665in}}%
\pgfpathlineto{\pgfqpoint{2.501495in}{2.760198in}}%
\pgfpathlineto{\pgfqpoint{2.497071in}{2.766959in}}%
\pgfpathlineto{\pgfqpoint{2.492951in}{2.777254in}}%
\pgfpathlineto{\pgfqpoint{2.490917in}{2.783267in}}%
\pgfpathlineto{\pgfqpoint{2.489434in}{2.787548in}}%
\pgfpathlineto{\pgfqpoint{2.486336in}{2.797842in}}%
\pgfpathlineto{\pgfqpoint{2.483715in}{2.808137in}}%
\pgfpathlineto{\pgfqpoint{2.481582in}{2.818431in}}%
\pgfpathlineto{\pgfqpoint{2.480340in}{2.826222in}}%
\pgfpathlineto{\pgfqpoint{2.479930in}{2.828726in}}%
\pgfpathlineto{\pgfqpoint{2.478751in}{2.839020in}}%
\pgfpathlineto{\pgfqpoint{2.478127in}{2.849315in}}%
\pgfpathlineto{\pgfqpoint{2.478068in}{2.859609in}}%
\pgfpathlineto{\pgfqpoint{2.478586in}{2.869903in}}%
\pgfpathlineto{\pgfqpoint{2.479687in}{2.880198in}}%
\pgfpathlineto{\pgfqpoint{2.480340in}{2.884114in}}%
\pgfpathlineto{\pgfqpoint{2.481322in}{2.890492in}}%
\pgfpathlineto{\pgfqpoint{2.483486in}{2.900787in}}%
\pgfpathlineto{\pgfqpoint{2.486227in}{2.911081in}}%
\pgfpathlineto{\pgfqpoint{2.480340in}{2.911081in}}%
\pgfpathlineto{\pgfqpoint{2.469763in}{2.911081in}}%
\pgfpathlineto{\pgfqpoint{2.461438in}{2.911081in}}%
\pgfpathlineto{\pgfqpoint{2.459186in}{2.904679in}}%
\pgfpathlineto{\pgfqpoint{2.457697in}{2.900787in}}%
\pgfpathlineto{\pgfqpoint{2.454452in}{2.890492in}}%
\pgfpathlineto{\pgfqpoint{2.451842in}{2.880198in}}%
\pgfpathlineto{\pgfqpoint{2.449868in}{2.869903in}}%
\pgfpathlineto{\pgfqpoint{2.448609in}{2.860200in}}%
\pgfpathlineto{\pgfqpoint{2.448526in}{2.859609in}}%
\pgfpathlineto{\pgfqpoint{2.447783in}{2.849315in}}%
\pgfpathlineto{\pgfqpoint{2.447717in}{2.839020in}}%
\pgfpathlineto{\pgfqpoint{2.448322in}{2.828726in}}%
\pgfpathlineto{\pgfqpoint{2.448609in}{2.826362in}}%
\pgfpathlineto{\pgfqpoint{2.449537in}{2.818431in}}%
\pgfpathlineto{\pgfqpoint{2.451356in}{2.808137in}}%
\pgfpathlineto{\pgfqpoint{2.453789in}{2.797842in}}%
\pgfpathlineto{\pgfqpoint{2.456841in}{2.787548in}}%
\pgfpathlineto{\pgfqpoint{2.459186in}{2.781046in}}%
\pgfpathlineto{\pgfqpoint{2.460503in}{2.777254in}}%
\pgfpathlineto{\pgfqpoint{2.464904in}{2.766959in}}%
\pgfpathlineto{\pgfqpoint{2.469763in}{2.758559in}}%
\pgfpathlineto{\pgfqpoint{2.470840in}{2.756665in}}%
\pgfpathlineto{\pgfqpoint{2.480340in}{2.747628in}}%
\pgfpathlineto{\pgfqpoint{2.482026in}{2.746370in}}%
\pgfpathlineto{\pgfqpoint{2.490917in}{2.742393in}}%
\pgfpathlineto{\pgfqpoint{2.501495in}{2.738937in}}%
\pgfpathlineto{\pgfqpoint{2.512072in}{2.736157in}}%
\pgfpathlineto{\pgfqpoint{2.512409in}{2.736076in}}%
\pgfpathlineto{\pgfqpoint{2.522649in}{2.733465in}}%
\pgfpathlineto{\pgfqpoint{2.533226in}{2.730833in}}%
\pgfpathlineto{\pgfqpoint{2.543803in}{2.728192in}}%
\pgfpathlineto{\pgfqpoint{2.553283in}{2.725781in}}%
\pgfpathlineto{\pgfqpoint{2.554380in}{2.725478in}}%
\pgfpathlineto{\pgfqpoint{2.564957in}{2.722491in}}%
\pgfpathlineto{\pgfqpoint{2.575534in}{2.719407in}}%
\pgfpathlineto{\pgfqpoint{2.586111in}{2.716226in}}%
\pgfpathlineto{\pgfqpoint{2.588528in}{2.715487in}}%
\pgfpathlineto{\pgfqpoint{2.596688in}{2.712704in}}%
\pgfpathlineto{\pgfqpoint{2.607265in}{2.709012in}}%
\pgfpathlineto{\pgfqpoint{2.617842in}{2.705246in}}%
\pgfpathlineto{\pgfqpoint{2.617992in}{2.705193in}}%
\pgfpathlineto{\pgfqpoint{2.628419in}{2.700976in}}%
\pgfpathlineto{\pgfqpoint{2.638996in}{2.696672in}}%
\pgfpathlineto{\pgfqpoint{2.643435in}{2.694898in}}%
\pgfpathlineto{\pgfqpoint{2.649573in}{2.692023in}}%
\pgfpathlineto{\pgfqpoint{2.660150in}{2.687186in}}%
\pgfpathlineto{\pgfqpoint{2.666037in}{2.684604in}}%
\pgfpathlineto{\pgfqpoint{2.670727in}{2.682124in}}%
\pgfpathlineto{\pgfqpoint{2.681305in}{2.676864in}}%
\pgfpathlineto{\pgfqpoint{2.686887in}{2.674309in}}%
\pgfpathlineto{\pgfqpoint{2.691882in}{2.671471in}}%
\pgfpathlineto{\pgfqpoint{2.702459in}{2.666116in}}%
\pgfpathlineto{\pgfqpoint{2.707332in}{2.664015in}}%
\pgfpathlineto{\pgfqpoint{2.713036in}{2.660858in}}%
\pgfpathlineto{\pgfqpoint{2.723613in}{2.656205in}}%
\pgfpathlineto{\pgfqpoint{2.731606in}{2.653720in}}%
\pgfpathclose%
\pgfusepath{fill}%
\end{pgfscope}%
\begin{pgfscope}%
\pgfpathrectangle{\pgfqpoint{1.856795in}{1.819814in}}{\pgfqpoint{1.194205in}{1.163386in}}%
\pgfusepath{clip}%
\pgfsetbuttcap%
\pgfsetroundjoin%
\definecolor{currentfill}{rgb}{0.852817,0.156578,0.279098}%
\pgfsetfillcolor{currentfill}%
\pgfsetlinewidth{0.000000pt}%
\definecolor{currentstroke}{rgb}{0.000000,0.000000,0.000000}%
\pgfsetstrokecolor{currentstroke}%
\pgfsetdash{}{0pt}%
\pgfpathmoveto{\pgfqpoint{2.110143in}{1.891932in}}%
\pgfpathlineto{\pgfqpoint{2.120440in}{1.891932in}}%
\pgfpathlineto{\pgfqpoint{2.120720in}{1.893606in}}%
\pgfpathlineto{\pgfqpoint{2.122217in}{1.902227in}}%
\pgfpathlineto{\pgfqpoint{2.122985in}{1.912521in}}%
\pgfpathlineto{\pgfqpoint{2.122804in}{1.922816in}}%
\pgfpathlineto{\pgfqpoint{2.121727in}{1.933110in}}%
\pgfpathlineto{\pgfqpoint{2.120720in}{1.938421in}}%
\pgfpathlineto{\pgfqpoint{2.119841in}{1.943405in}}%
\pgfpathlineto{\pgfqpoint{2.117221in}{1.953699in}}%
\pgfpathlineto{\pgfqpoint{2.113781in}{1.963993in}}%
\pgfpathlineto{\pgfqpoint{2.110143in}{1.972683in}}%
\pgfpathlineto{\pgfqpoint{2.109522in}{1.974288in}}%
\pgfpathlineto{\pgfqpoint{2.104703in}{1.984582in}}%
\pgfpathlineto{\pgfqpoint{2.099566in}{1.993995in}}%
\pgfpathlineto{\pgfqpoint{2.099124in}{1.994877in}}%
\pgfpathlineto{\pgfqpoint{2.093376in}{2.005171in}}%
\pgfpathlineto{\pgfqpoint{2.088989in}{2.012622in}}%
\pgfpathlineto{\pgfqpoint{2.087430in}{2.015466in}}%
\pgfpathlineto{\pgfqpoint{2.081751in}{2.025760in}}%
\pgfpathlineto{\pgfqpoint{2.078412in}{2.031971in}}%
\pgfpathlineto{\pgfqpoint{2.076323in}{2.036054in}}%
\pgfpathlineto{\pgfqpoint{2.071317in}{2.046349in}}%
\pgfpathlineto{\pgfqpoint{2.067835in}{2.054018in}}%
\pgfpathlineto{\pgfqpoint{2.066638in}{2.056643in}}%
\pgfpathlineto{\pgfqpoint{2.062254in}{2.066938in}}%
\pgfpathlineto{\pgfqpoint{2.058024in}{2.077232in}}%
\pgfpathlineto{\pgfqpoint{2.057258in}{2.078939in}}%
\pgfpathlineto{\pgfqpoint{2.052851in}{2.087527in}}%
\pgfpathlineto{\pgfqpoint{2.047079in}{2.097821in}}%
\pgfpathlineto{\pgfqpoint{2.046681in}{2.098534in}}%
\pgfpathlineto{\pgfqpoint{2.041497in}{2.108116in}}%
\pgfpathlineto{\pgfqpoint{2.036125in}{2.118410in}}%
\pgfpathlineto{\pgfqpoint{2.036104in}{2.118452in}}%
\pgfpathlineto{\pgfqpoint{2.031055in}{2.128704in}}%
\pgfpathlineto{\pgfqpoint{2.026252in}{2.138999in}}%
\pgfpathlineto{\pgfqpoint{2.025527in}{2.140616in}}%
\pgfpathlineto{\pgfqpoint{2.021775in}{2.149293in}}%
\pgfpathlineto{\pgfqpoint{2.017585in}{2.159588in}}%
\pgfpathlineto{\pgfqpoint{2.014950in}{2.166435in}}%
\pgfpathlineto{\pgfqpoint{2.013675in}{2.169882in}}%
\pgfpathlineto{\pgfqpoint{2.010064in}{2.180177in}}%
\pgfpathlineto{\pgfqpoint{2.006682in}{2.190471in}}%
\pgfpathlineto{\pgfqpoint{2.004373in}{2.197886in}}%
\pgfpathlineto{\pgfqpoint{2.003514in}{2.200765in}}%
\pgfpathlineto{\pgfqpoint{2.000577in}{2.211060in}}%
\pgfpathlineto{\pgfqpoint{1.997787in}{2.221354in}}%
\pgfpathlineto{\pgfqpoint{1.995101in}{2.231649in}}%
\pgfpathlineto{\pgfqpoint{1.993796in}{2.236747in}}%
\pgfpathlineto{\pgfqpoint{1.992527in}{2.241943in}}%
\pgfpathlineto{\pgfqpoint{1.990046in}{2.252238in}}%
\pgfpathlineto{\pgfqpoint{1.987574in}{2.262532in}}%
\pgfpathlineto{\pgfqpoint{1.985067in}{2.272826in}}%
\pgfpathlineto{\pgfqpoint{1.983218in}{2.280180in}}%
\pgfpathlineto{\pgfqpoint{1.982516in}{2.283121in}}%
\pgfpathlineto{\pgfqpoint{1.979948in}{2.293415in}}%
\pgfpathlineto{\pgfqpoint{1.977231in}{2.303710in}}%
\pgfpathlineto{\pgfqpoint{1.974310in}{2.314004in}}%
\pgfpathlineto{\pgfqpoint{1.972641in}{2.319466in}}%
\pgfpathlineto{\pgfqpoint{1.971202in}{2.324299in}}%
\pgfpathlineto{\pgfqpoint{1.967864in}{2.334593in}}%
\pgfpathlineto{\pgfqpoint{1.964191in}{2.344887in}}%
\pgfpathlineto{\pgfqpoint{1.962064in}{2.350437in}}%
\pgfpathlineto{\pgfqpoint{1.960204in}{2.355182in}}%
\pgfpathlineto{\pgfqpoint{1.955925in}{2.365476in}}%
\pgfpathlineto{\pgfqpoint{1.951487in}{2.375529in}}%
\pgfpathlineto{\pgfqpoint{1.951377in}{2.375771in}}%
\pgfpathlineto{\pgfqpoint{1.946621in}{2.386065in}}%
\pgfpathlineto{\pgfqpoint{1.941749in}{2.396360in}}%
\pgfpathlineto{\pgfqpoint{1.940910in}{2.398141in}}%
\pgfpathlineto{\pgfqpoint{1.936812in}{2.406654in}}%
\pgfpathlineto{\pgfqpoint{1.931902in}{2.416948in}}%
\pgfpathlineto{\pgfqpoint{1.930333in}{2.420321in}}%
\pgfpathlineto{\pgfqpoint{1.930333in}{2.416948in}}%
\pgfpathlineto{\pgfqpoint{1.930333in}{2.406654in}}%
\pgfpathlineto{\pgfqpoint{1.930333in}{2.396360in}}%
\pgfpathlineto{\pgfqpoint{1.930333in}{2.393354in}}%
\pgfpathlineto{\pgfqpoint{1.933976in}{2.386065in}}%
\pgfpathlineto{\pgfqpoint{1.939105in}{2.375771in}}%
\pgfpathlineto{\pgfqpoint{1.940910in}{2.372132in}}%
\pgfpathlineto{\pgfqpoint{1.944067in}{2.365476in}}%
\pgfpathlineto{\pgfqpoint{1.948783in}{2.355182in}}%
\pgfpathlineto{\pgfqpoint{1.951487in}{2.348972in}}%
\pgfpathlineto{\pgfqpoint{1.953184in}{2.344887in}}%
\pgfpathlineto{\pgfqpoint{1.957184in}{2.334593in}}%
\pgfpathlineto{\pgfqpoint{1.960762in}{2.324299in}}%
\pgfpathlineto{\pgfqpoint{1.962064in}{2.320122in}}%
\pgfpathlineto{\pgfqpoint{1.963986in}{2.314004in}}%
\pgfpathlineto{\pgfqpoint{1.966908in}{2.303710in}}%
\pgfpathlineto{\pgfqpoint{1.969554in}{2.293415in}}%
\pgfpathlineto{\pgfqpoint{1.972004in}{2.283121in}}%
\pgfpathlineto{\pgfqpoint{1.972641in}{2.280272in}}%
\pgfpathlineto{\pgfqpoint{1.974385in}{2.272826in}}%
\pgfpathlineto{\pgfqpoint{1.976686in}{2.262532in}}%
\pgfpathlineto{\pgfqpoint{1.978935in}{2.252238in}}%
\pgfpathlineto{\pgfqpoint{1.981180in}{2.241943in}}%
\pgfpathlineto{\pgfqpoint{1.983218in}{2.232735in}}%
\pgfpathlineto{\pgfqpoint{1.983469in}{2.231649in}}%
\pgfpathlineto{\pgfqpoint{1.985858in}{2.221354in}}%
\pgfpathlineto{\pgfqpoint{1.988342in}{2.211060in}}%
\pgfpathlineto{\pgfqpoint{1.990963in}{2.200765in}}%
\pgfpathlineto{\pgfqpoint{1.993758in}{2.190471in}}%
\pgfpathlineto{\pgfqpoint{1.993796in}{2.190340in}}%
\pgfpathlineto{\pgfqpoint{1.996779in}{2.180177in}}%
\pgfpathlineto{\pgfqpoint{2.000024in}{2.169882in}}%
\pgfpathlineto{\pgfqpoint{2.003520in}{2.159588in}}%
\pgfpathlineto{\pgfqpoint{2.004373in}{2.157208in}}%
\pgfpathlineto{\pgfqpoint{2.007306in}{2.149293in}}%
\pgfpathlineto{\pgfqpoint{2.011374in}{2.138999in}}%
\pgfpathlineto{\pgfqpoint{2.014950in}{2.130478in}}%
\pgfpathlineto{\pgfqpoint{2.015720in}{2.128704in}}%
\pgfpathlineto{\pgfqpoint{2.020372in}{2.118410in}}%
\pgfpathlineto{\pgfqpoint{2.025246in}{2.108116in}}%
\pgfpathlineto{\pgfqpoint{2.025527in}{2.107536in}}%
\pgfpathlineto{\pgfqpoint{2.030395in}{2.097821in}}%
\pgfpathlineto{\pgfqpoint{2.035645in}{2.087527in}}%
\pgfpathlineto{\pgfqpoint{2.036104in}{2.086634in}}%
\pgfpathlineto{\pgfqpoint{2.041036in}{2.077232in}}%
\pgfpathlineto{\pgfqpoint{2.046313in}{2.066938in}}%
\pgfpathlineto{\pgfqpoint{2.046681in}{2.066231in}}%
\pgfpathlineto{\pgfqpoint{2.051447in}{2.056643in}}%
\pgfpathlineto{\pgfqpoint{2.056591in}{2.046349in}}%
\pgfpathlineto{\pgfqpoint{2.057258in}{2.045083in}}%
\pgfpathlineto{\pgfqpoint{2.061758in}{2.036054in}}%
\pgfpathlineto{\pgfqpoint{2.067081in}{2.025760in}}%
\pgfpathlineto{\pgfqpoint{2.067835in}{2.024336in}}%
\pgfpathlineto{\pgfqpoint{2.072593in}{2.015466in}}%
\pgfpathlineto{\pgfqpoint{2.077951in}{2.005171in}}%
\pgfpathlineto{\pgfqpoint{2.078412in}{2.004209in}}%
\pgfpathlineto{\pgfqpoint{2.083231in}{1.994877in}}%
\pgfpathlineto{\pgfqpoint{2.087939in}{1.984582in}}%
\pgfpathlineto{\pgfqpoint{2.088989in}{1.981892in}}%
\pgfpathlineto{\pgfqpoint{2.092206in}{1.974288in}}%
\pgfpathlineto{\pgfqpoint{2.095822in}{1.963993in}}%
\pgfpathlineto{\pgfqpoint{2.098731in}{1.953699in}}%
\pgfpathlineto{\pgfqpoint{2.099566in}{1.949766in}}%
\pgfpathlineto{\pgfqpoint{2.101016in}{1.943405in}}%
\pgfpathlineto{\pgfqpoint{2.102574in}{1.933110in}}%
\pgfpathlineto{\pgfqpoint{2.103321in}{1.922816in}}%
\pgfpathlineto{\pgfqpoint{2.103212in}{1.912521in}}%
\pgfpathlineto{\pgfqpoint{2.102181in}{1.902227in}}%
\pgfpathlineto{\pgfqpoint{2.100143in}{1.891932in}}%
\pgfpathclose%
\pgfusepath{fill}%
\end{pgfscope}%
\begin{pgfscope}%
\pgfpathrectangle{\pgfqpoint{1.856795in}{1.819814in}}{\pgfqpoint{1.194205in}{1.163386in}}%
\pgfusepath{clip}%
\pgfsetbuttcap%
\pgfsetroundjoin%
\definecolor{currentfill}{rgb}{0.852817,0.156578,0.279098}%
\pgfsetfillcolor{currentfill}%
\pgfsetlinewidth{0.000000pt}%
\definecolor{currentstroke}{rgb}{0.000000,0.000000,0.000000}%
\pgfsetstrokecolor{currentstroke}%
\pgfsetdash{}{0pt}%
\pgfpathmoveto{\pgfqpoint{2.691882in}{2.550343in}}%
\pgfpathlineto{\pgfqpoint{2.702459in}{2.549653in}}%
\pgfpathlineto{\pgfqpoint{2.712190in}{2.550776in}}%
\pgfpathlineto{\pgfqpoint{2.713036in}{2.550870in}}%
\pgfpathlineto{\pgfqpoint{2.723613in}{2.553379in}}%
\pgfpathlineto{\pgfqpoint{2.734190in}{2.556842in}}%
\pgfpathlineto{\pgfqpoint{2.744767in}{2.560975in}}%
\pgfpathlineto{\pgfqpoint{2.744985in}{2.561071in}}%
\pgfpathlineto{\pgfqpoint{2.755344in}{2.565562in}}%
\pgfpathlineto{\pgfqpoint{2.765921in}{2.570376in}}%
\pgfpathlineto{\pgfqpoint{2.767982in}{2.571365in}}%
\pgfpathlineto{\pgfqpoint{2.776498in}{2.575346in}}%
\pgfpathlineto{\pgfqpoint{2.787075in}{2.580334in}}%
\pgfpathlineto{\pgfqpoint{2.789724in}{2.581659in}}%
\pgfpathlineto{\pgfqpoint{2.797652in}{2.585449in}}%
\pgfpathlineto{\pgfqpoint{2.808229in}{2.591387in}}%
\pgfpathlineto{\pgfqpoint{2.809081in}{2.591954in}}%
\pgfpathlineto{\pgfqpoint{2.817491in}{2.602248in}}%
\pgfpathlineto{\pgfqpoint{2.818806in}{2.606577in}}%
\pgfpathlineto{\pgfqpoint{2.820502in}{2.612543in}}%
\pgfpathlineto{\pgfqpoint{2.822750in}{2.622837in}}%
\pgfpathlineto{\pgfqpoint{2.825311in}{2.633132in}}%
\pgfpathlineto{\pgfqpoint{2.828479in}{2.643426in}}%
\pgfpathlineto{\pgfqpoint{2.829383in}{2.645812in}}%
\pgfpathlineto{\pgfqpoint{2.831984in}{2.653720in}}%
\pgfpathlineto{\pgfqpoint{2.836145in}{2.664015in}}%
\pgfpathlineto{\pgfqpoint{2.839960in}{2.671831in}}%
\pgfpathlineto{\pgfqpoint{2.841038in}{2.674309in}}%
\pgfpathlineto{\pgfqpoint{2.846344in}{2.684604in}}%
\pgfpathlineto{\pgfqpoint{2.850538in}{2.691553in}}%
\pgfpathlineto{\pgfqpoint{2.852378in}{2.694898in}}%
\pgfpathlineto{\pgfqpoint{2.858922in}{2.705193in}}%
\pgfpathlineto{\pgfqpoint{2.861115in}{2.708217in}}%
\pgfpathlineto{\pgfqpoint{2.866019in}{2.715487in}}%
\pgfpathlineto{\pgfqpoint{2.871692in}{2.722911in}}%
\pgfpathlineto{\pgfqpoint{2.873762in}{2.725781in}}%
\pgfpathlineto{\pgfqpoint{2.882100in}{2.736076in}}%
\pgfpathlineto{\pgfqpoint{2.882269in}{2.736264in}}%
\pgfpathlineto{\pgfqpoint{2.890957in}{2.746370in}}%
\pgfpathlineto{\pgfqpoint{2.892846in}{2.748359in}}%
\pgfpathlineto{\pgfqpoint{2.900459in}{2.756665in}}%
\pgfpathlineto{\pgfqpoint{2.903423in}{2.759613in}}%
\pgfpathlineto{\pgfqpoint{2.910597in}{2.766959in}}%
\pgfpathlineto{\pgfqpoint{2.914000in}{2.770159in}}%
\pgfpathlineto{\pgfqpoint{2.921360in}{2.777254in}}%
\pgfpathlineto{\pgfqpoint{2.924577in}{2.780120in}}%
\pgfpathlineto{\pgfqpoint{2.932719in}{2.787548in}}%
\pgfpathlineto{\pgfqpoint{2.935154in}{2.789616in}}%
\pgfpathlineto{\pgfqpoint{2.944602in}{2.797842in}}%
\pgfpathlineto{\pgfqpoint{2.945731in}{2.798764in}}%
\pgfpathlineto{\pgfqpoint{2.956308in}{2.807662in}}%
\pgfpathlineto{\pgfqpoint{2.956852in}{2.808137in}}%
\pgfpathlineto{\pgfqpoint{2.966885in}{2.816425in}}%
\pgfpathlineto{\pgfqpoint{2.969209in}{2.818431in}}%
\pgfpathlineto{\pgfqpoint{2.977462in}{2.825249in}}%
\pgfpathlineto{\pgfqpoint{2.977462in}{2.828726in}}%
\pgfpathlineto{\pgfqpoint{2.977462in}{2.839020in}}%
\pgfpathlineto{\pgfqpoint{2.977462in}{2.840560in}}%
\pgfpathlineto{\pgfqpoint{2.975451in}{2.839020in}}%
\pgfpathlineto{\pgfqpoint{2.966885in}{2.832234in}}%
\pgfpathlineto{\pgfqpoint{2.962193in}{2.828726in}}%
\pgfpathlineto{\pgfqpoint{2.956308in}{2.824124in}}%
\pgfpathlineto{\pgfqpoint{2.948716in}{2.818431in}}%
\pgfpathlineto{\pgfqpoint{2.945731in}{2.816072in}}%
\pgfpathlineto{\pgfqpoint{2.935371in}{2.808137in}}%
\pgfpathlineto{\pgfqpoint{2.935154in}{2.807960in}}%
\pgfpathlineto{\pgfqpoint{2.924577in}{2.799615in}}%
\pgfpathlineto{\pgfqpoint{2.922282in}{2.797842in}}%
\pgfpathlineto{\pgfqpoint{2.914000in}{2.791014in}}%
\pgfpathlineto{\pgfqpoint{2.909706in}{2.787548in}}%
\pgfpathlineto{\pgfqpoint{2.903423in}{2.782103in}}%
\pgfpathlineto{\pgfqpoint{2.897691in}{2.777254in}}%
\pgfpathlineto{\pgfqpoint{2.892846in}{2.772827in}}%
\pgfpathlineto{\pgfqpoint{2.886229in}{2.766959in}}%
\pgfpathlineto{\pgfqpoint{2.882269in}{2.763146in}}%
\pgfpathlineto{\pgfqpoint{2.875279in}{2.756665in}}%
\pgfpathlineto{\pgfqpoint{2.871692in}{2.753032in}}%
\pgfpathlineto{\pgfqpoint{2.864787in}{2.746370in}}%
\pgfpathlineto{\pgfqpoint{2.861115in}{2.742479in}}%
\pgfpathlineto{\pgfqpoint{2.854685in}{2.736076in}}%
\pgfpathlineto{\pgfqpoint{2.850538in}{2.731513in}}%
\pgfpathlineto{\pgfqpoint{2.844898in}{2.725781in}}%
\pgfpathlineto{\pgfqpoint{2.839960in}{2.720207in}}%
\pgfpathlineto{\pgfqpoint{2.835339in}{2.715487in}}%
\pgfpathlineto{\pgfqpoint{2.829383in}{2.708691in}}%
\pgfpathlineto{\pgfqpoint{2.825901in}{2.705193in}}%
\pgfpathlineto{\pgfqpoint{2.818806in}{2.697176in}}%
\pgfpathlineto{\pgfqpoint{2.816437in}{2.694898in}}%
\pgfpathlineto{\pgfqpoint{2.808229in}{2.685957in}}%
\pgfpathlineto{\pgfqpoint{2.806703in}{2.684604in}}%
\pgfpathlineto{\pgfqpoint{2.797652in}{2.675420in}}%
\pgfpathlineto{\pgfqpoint{2.796226in}{2.674309in}}%
\pgfpathlineto{\pgfqpoint{2.787075in}{2.666039in}}%
\pgfpathlineto{\pgfqpoint{2.783897in}{2.664015in}}%
\pgfpathlineto{\pgfqpoint{2.776498in}{2.658419in}}%
\pgfpathlineto{\pgfqpoint{2.766377in}{2.653720in}}%
\pgfpathlineto{\pgfqpoint{2.765921in}{2.653460in}}%
\pgfpathlineto{\pgfqpoint{2.755344in}{2.650557in}}%
\pgfpathlineto{\pgfqpoint{2.744767in}{2.650421in}}%
\pgfpathlineto{\pgfqpoint{2.734190in}{2.652654in}}%
\pgfpathlineto{\pgfqpoint{2.731606in}{2.653720in}}%
\pgfpathlineto{\pgfqpoint{2.723613in}{2.656205in}}%
\pgfpathlineto{\pgfqpoint{2.713036in}{2.660858in}}%
\pgfpathlineto{\pgfqpoint{2.707332in}{2.664015in}}%
\pgfpathlineto{\pgfqpoint{2.702459in}{2.666116in}}%
\pgfpathlineto{\pgfqpoint{2.691882in}{2.671471in}}%
\pgfpathlineto{\pgfqpoint{2.686887in}{2.674309in}}%
\pgfpathlineto{\pgfqpoint{2.681305in}{2.676864in}}%
\pgfpathlineto{\pgfqpoint{2.670727in}{2.682124in}}%
\pgfpathlineto{\pgfqpoint{2.666037in}{2.684604in}}%
\pgfpathlineto{\pgfqpoint{2.660150in}{2.687186in}}%
\pgfpathlineto{\pgfqpoint{2.649573in}{2.692023in}}%
\pgfpathlineto{\pgfqpoint{2.643435in}{2.694898in}}%
\pgfpathlineto{\pgfqpoint{2.638996in}{2.696672in}}%
\pgfpathlineto{\pgfqpoint{2.628419in}{2.700976in}}%
\pgfpathlineto{\pgfqpoint{2.617992in}{2.705193in}}%
\pgfpathlineto{\pgfqpoint{2.617842in}{2.705246in}}%
\pgfpathlineto{\pgfqpoint{2.607265in}{2.709012in}}%
\pgfpathlineto{\pgfqpoint{2.596688in}{2.712704in}}%
\pgfpathlineto{\pgfqpoint{2.588528in}{2.715487in}}%
\pgfpathlineto{\pgfqpoint{2.586111in}{2.716226in}}%
\pgfpathlineto{\pgfqpoint{2.575534in}{2.719407in}}%
\pgfpathlineto{\pgfqpoint{2.564957in}{2.722491in}}%
\pgfpathlineto{\pgfqpoint{2.554380in}{2.725478in}}%
\pgfpathlineto{\pgfqpoint{2.553283in}{2.725781in}}%
\pgfpathlineto{\pgfqpoint{2.543803in}{2.728192in}}%
\pgfpathlineto{\pgfqpoint{2.533226in}{2.730833in}}%
\pgfpathlineto{\pgfqpoint{2.522649in}{2.733465in}}%
\pgfpathlineto{\pgfqpoint{2.512409in}{2.736076in}}%
\pgfpathlineto{\pgfqpoint{2.512072in}{2.736157in}}%
\pgfpathlineto{\pgfqpoint{2.501495in}{2.738937in}}%
\pgfpathlineto{\pgfqpoint{2.490917in}{2.742393in}}%
\pgfpathlineto{\pgfqpoint{2.482026in}{2.746370in}}%
\pgfpathlineto{\pgfqpoint{2.480340in}{2.747628in}}%
\pgfpathlineto{\pgfqpoint{2.470840in}{2.756665in}}%
\pgfpathlineto{\pgfqpoint{2.469763in}{2.758559in}}%
\pgfpathlineto{\pgfqpoint{2.464904in}{2.766959in}}%
\pgfpathlineto{\pgfqpoint{2.460503in}{2.777254in}}%
\pgfpathlineto{\pgfqpoint{2.459186in}{2.781046in}}%
\pgfpathlineto{\pgfqpoint{2.456841in}{2.787548in}}%
\pgfpathlineto{\pgfqpoint{2.453789in}{2.797842in}}%
\pgfpathlineto{\pgfqpoint{2.451356in}{2.808137in}}%
\pgfpathlineto{\pgfqpoint{2.449537in}{2.818431in}}%
\pgfpathlineto{\pgfqpoint{2.448609in}{2.826362in}}%
\pgfpathlineto{\pgfqpoint{2.448322in}{2.828726in}}%
\pgfpathlineto{\pgfqpoint{2.447717in}{2.839020in}}%
\pgfpathlineto{\pgfqpoint{2.447783in}{2.849315in}}%
\pgfpathlineto{\pgfqpoint{2.448526in}{2.859609in}}%
\pgfpathlineto{\pgfqpoint{2.448609in}{2.860200in}}%
\pgfpathlineto{\pgfqpoint{2.449868in}{2.869903in}}%
\pgfpathlineto{\pgfqpoint{2.451842in}{2.880198in}}%
\pgfpathlineto{\pgfqpoint{2.454452in}{2.890492in}}%
\pgfpathlineto{\pgfqpoint{2.457697in}{2.900787in}}%
\pgfpathlineto{\pgfqpoint{2.459186in}{2.904679in}}%
\pgfpathlineto{\pgfqpoint{2.461438in}{2.911081in}}%
\pgfpathlineto{\pgfqpoint{2.459186in}{2.911081in}}%
\pgfpathlineto{\pgfqpoint{2.448609in}{2.911081in}}%
\pgfpathlineto{\pgfqpoint{2.438032in}{2.911081in}}%
\pgfpathlineto{\pgfqpoint{2.432361in}{2.911081in}}%
\pgfpathlineto{\pgfqpoint{2.427455in}{2.901134in}}%
\pgfpathlineto{\pgfqpoint{2.427269in}{2.900787in}}%
\pgfpathlineto{\pgfqpoint{2.422599in}{2.890492in}}%
\pgfpathlineto{\pgfqpoint{2.418684in}{2.880198in}}%
\pgfpathlineto{\pgfqpoint{2.416878in}{2.874250in}}%
\pgfpathlineto{\pgfqpoint{2.415455in}{2.869903in}}%
\pgfpathlineto{\pgfqpoint{2.412951in}{2.859609in}}%
\pgfpathlineto{\pgfqpoint{2.411291in}{2.849315in}}%
\pgfpathlineto{\pgfqpoint{2.410484in}{2.839020in}}%
\pgfpathlineto{\pgfqpoint{2.410533in}{2.828726in}}%
\pgfpathlineto{\pgfqpoint{2.411435in}{2.818431in}}%
\pgfpathlineto{\pgfqpoint{2.413182in}{2.808137in}}%
\pgfpathlineto{\pgfqpoint{2.415763in}{2.797842in}}%
\pgfpathlineto{\pgfqpoint{2.416878in}{2.794487in}}%
\pgfpathlineto{\pgfqpoint{2.419098in}{2.787548in}}%
\pgfpathlineto{\pgfqpoint{2.423253in}{2.777254in}}%
\pgfpathlineto{\pgfqpoint{2.427455in}{2.768871in}}%
\pgfpathlineto{\pgfqpoint{2.428376in}{2.766959in}}%
\pgfpathlineto{\pgfqpoint{2.434712in}{2.756665in}}%
\pgfpathlineto{\pgfqpoint{2.438032in}{2.752688in}}%
\pgfpathlineto{\pgfqpoint{2.443339in}{2.746370in}}%
\pgfpathlineto{\pgfqpoint{2.448609in}{2.742122in}}%
\pgfpathlineto{\pgfqpoint{2.457174in}{2.736076in}}%
\pgfpathlineto{\pgfqpoint{2.459186in}{2.735020in}}%
\pgfpathlineto{\pgfqpoint{2.469763in}{2.730244in}}%
\pgfpathlineto{\pgfqpoint{2.480340in}{2.726434in}}%
\pgfpathlineto{\pgfqpoint{2.482441in}{2.725781in}}%
\pgfpathlineto{\pgfqpoint{2.490917in}{2.723132in}}%
\pgfpathlineto{\pgfqpoint{2.501495in}{2.720092in}}%
\pgfpathlineto{\pgfqpoint{2.512072in}{2.717189in}}%
\pgfpathlineto{\pgfqpoint{2.518352in}{2.715487in}}%
\pgfpathlineto{\pgfqpoint{2.522649in}{2.714227in}}%
\pgfpathlineto{\pgfqpoint{2.533226in}{2.711079in}}%
\pgfpathlineto{\pgfqpoint{2.543803in}{2.707833in}}%
\pgfpathlineto{\pgfqpoint{2.552092in}{2.705193in}}%
\pgfpathlineto{\pgfqpoint{2.554380in}{2.704383in}}%
\pgfpathlineto{\pgfqpoint{2.564957in}{2.700495in}}%
\pgfpathlineto{\pgfqpoint{2.575534in}{2.696409in}}%
\pgfpathlineto{\pgfqpoint{2.579310in}{2.694898in}}%
\pgfpathlineto{\pgfqpoint{2.586111in}{2.691789in}}%
\pgfpathlineto{\pgfqpoint{2.596688in}{2.686741in}}%
\pgfpathlineto{\pgfqpoint{2.601035in}{2.684604in}}%
\pgfpathlineto{\pgfqpoint{2.607265in}{2.681005in}}%
\pgfpathlineto{\pgfqpoint{2.617842in}{2.674682in}}%
\pgfpathlineto{\pgfqpoint{2.618461in}{2.674309in}}%
\pgfpathlineto{\pgfqpoint{2.628419in}{2.667016in}}%
\pgfpathlineto{\pgfqpoint{2.632468in}{2.664015in}}%
\pgfpathlineto{\pgfqpoint{2.638996in}{2.657917in}}%
\pgfpathlineto{\pgfqpoint{2.643506in}{2.653720in}}%
\pgfpathlineto{\pgfqpoint{2.649573in}{2.646262in}}%
\pgfpathlineto{\pgfqpoint{2.651933in}{2.643426in}}%
\pgfpathlineto{\pgfqpoint{2.658091in}{2.633132in}}%
\pgfpathlineto{\pgfqpoint{2.660150in}{2.627960in}}%
\pgfpathlineto{\pgfqpoint{2.662290in}{2.622837in}}%
\pgfpathlineto{\pgfqpoint{2.664918in}{2.612543in}}%
\pgfpathlineto{\pgfqpoint{2.666324in}{2.602248in}}%
\pgfpathlineto{\pgfqpoint{2.667092in}{2.591954in}}%
\pgfpathlineto{\pgfqpoint{2.667904in}{2.581659in}}%
\pgfpathlineto{\pgfqpoint{2.669674in}{2.571365in}}%
\pgfpathlineto{\pgfqpoint{2.670727in}{2.568831in}}%
\pgfpathlineto{\pgfqpoint{2.674394in}{2.561071in}}%
\pgfpathlineto{\pgfqpoint{2.681305in}{2.554896in}}%
\pgfpathlineto{\pgfqpoint{2.690380in}{2.550776in}}%
\pgfpathclose%
\pgfusepath{fill}%
\end{pgfscope}%
\begin{pgfscope}%
\pgfpathrectangle{\pgfqpoint{1.856795in}{1.819814in}}{\pgfqpoint{1.194205in}{1.163386in}}%
\pgfusepath{clip}%
\pgfsetbuttcap%
\pgfsetroundjoin%
\definecolor{currentfill}{rgb}{0.898503,0.224633,0.251087}%
\pgfsetfillcolor{currentfill}%
\pgfsetlinewidth{0.000000pt}%
\definecolor{currentstroke}{rgb}{0.000000,0.000000,0.000000}%
\pgfsetstrokecolor{currentstroke}%
\pgfsetdash{}{0pt}%
\pgfpathmoveto{\pgfqpoint{2.120720in}{1.891932in}}%
\pgfpathlineto{\pgfqpoint{2.131297in}{1.891932in}}%
\pgfpathlineto{\pgfqpoint{2.141874in}{1.891932in}}%
\pgfpathlineto{\pgfqpoint{2.143216in}{1.891932in}}%
\pgfpathlineto{\pgfqpoint{2.144777in}{1.902227in}}%
\pgfpathlineto{\pgfqpoint{2.145302in}{1.912521in}}%
\pgfpathlineto{\pgfqpoint{2.144861in}{1.922816in}}%
\pgfpathlineto{\pgfqpoint{2.143494in}{1.933110in}}%
\pgfpathlineto{\pgfqpoint{2.141874in}{1.940418in}}%
\pgfpathlineto{\pgfqpoint{2.141257in}{1.943405in}}%
\pgfpathlineto{\pgfqpoint{2.138258in}{1.953699in}}%
\pgfpathlineto{\pgfqpoint{2.134320in}{1.963993in}}%
\pgfpathlineto{\pgfqpoint{2.131297in}{1.970243in}}%
\pgfpathlineto{\pgfqpoint{2.129481in}{1.974288in}}%
\pgfpathlineto{\pgfqpoint{2.123835in}{1.984582in}}%
\pgfpathlineto{\pgfqpoint{2.120720in}{1.989419in}}%
\pgfpathlineto{\pgfqpoint{2.117488in}{1.994877in}}%
\pgfpathlineto{\pgfqpoint{2.110741in}{2.005171in}}%
\pgfpathlineto{\pgfqpoint{2.110143in}{2.006047in}}%
\pgfpathlineto{\pgfqpoint{2.104221in}{2.015466in}}%
\pgfpathlineto{\pgfqpoint{2.099566in}{2.022847in}}%
\pgfpathlineto{\pgfqpoint{2.097865in}{2.025760in}}%
\pgfpathlineto{\pgfqpoint{2.092025in}{2.036054in}}%
\pgfpathlineto{\pgfqpoint{2.088989in}{2.041656in}}%
\pgfpathlineto{\pgfqpoint{2.086618in}{2.046349in}}%
\pgfpathlineto{\pgfqpoint{2.081722in}{2.056643in}}%
\pgfpathlineto{\pgfqpoint{2.078412in}{2.064057in}}%
\pgfpathlineto{\pgfqpoint{2.077202in}{2.066938in}}%
\pgfpathlineto{\pgfqpoint{2.073140in}{2.077232in}}%
\pgfpathlineto{\pgfqpoint{2.068952in}{2.087527in}}%
\pgfpathlineto{\pgfqpoint{2.067835in}{2.089808in}}%
\pgfpathlineto{\pgfqpoint{2.063571in}{2.097821in}}%
\pgfpathlineto{\pgfqpoint{2.057857in}{2.108116in}}%
\pgfpathlineto{\pgfqpoint{2.057258in}{2.109198in}}%
\pgfpathlineto{\pgfqpoint{2.052129in}{2.118410in}}%
\pgfpathlineto{\pgfqpoint{2.046681in}{2.128626in}}%
\pgfpathlineto{\pgfqpoint{2.046640in}{2.128704in}}%
\pgfpathlineto{\pgfqpoint{2.041464in}{2.138999in}}%
\pgfpathlineto{\pgfqpoint{2.036545in}{2.149293in}}%
\pgfpathlineto{\pgfqpoint{2.036104in}{2.150252in}}%
\pgfpathlineto{\pgfqpoint{2.031986in}{2.159588in}}%
\pgfpathlineto{\pgfqpoint{2.027693in}{2.169882in}}%
\pgfpathlineto{\pgfqpoint{2.025527in}{2.175330in}}%
\pgfpathlineto{\pgfqpoint{2.023688in}{2.180177in}}%
\pgfpathlineto{\pgfqpoint{2.019982in}{2.190471in}}%
\pgfpathlineto{\pgfqpoint{2.016470in}{2.200765in}}%
\pgfpathlineto{\pgfqpoint{2.014950in}{2.205411in}}%
\pgfpathlineto{\pgfqpoint{2.013196in}{2.211060in}}%
\pgfpathlineto{\pgfqpoint{2.010134in}{2.221354in}}%
\pgfpathlineto{\pgfqpoint{2.007184in}{2.231649in}}%
\pgfpathlineto{\pgfqpoint{2.004373in}{2.241712in}}%
\pgfpathlineto{\pgfqpoint{2.004312in}{2.241943in}}%
\pgfpathlineto{\pgfqpoint{2.001625in}{2.252238in}}%
\pgfpathlineto{\pgfqpoint{1.998960in}{2.262532in}}%
\pgfpathlineto{\pgfqpoint{1.996278in}{2.272826in}}%
\pgfpathlineto{\pgfqpoint{1.993796in}{2.282153in}}%
\pgfpathlineto{\pgfqpoint{1.993552in}{2.283121in}}%
\pgfpathlineto{\pgfqpoint{1.990885in}{2.293415in}}%
\pgfpathlineto{\pgfqpoint{1.988103in}{2.303710in}}%
\pgfpathlineto{\pgfqpoint{1.985163in}{2.314004in}}%
\pgfpathlineto{\pgfqpoint{1.983218in}{2.320427in}}%
\pgfpathlineto{\pgfqpoint{1.982089in}{2.324299in}}%
\pgfpathlineto{\pgfqpoint{1.978890in}{2.334593in}}%
\pgfpathlineto{\pgfqpoint{1.975428in}{2.344887in}}%
\pgfpathlineto{\pgfqpoint{1.972641in}{2.352608in}}%
\pgfpathlineto{\pgfqpoint{1.971713in}{2.355182in}}%
\pgfpathlineto{\pgfqpoint{1.967794in}{2.365476in}}%
\pgfpathlineto{\pgfqpoint{1.963600in}{2.375771in}}%
\pgfpathlineto{\pgfqpoint{1.962064in}{2.379414in}}%
\pgfpathlineto{\pgfqpoint{1.959206in}{2.386065in}}%
\pgfpathlineto{\pgfqpoint{1.954655in}{2.396360in}}%
\pgfpathlineto{\pgfqpoint{1.951487in}{2.403400in}}%
\pgfpathlineto{\pgfqpoint{1.949997in}{2.406654in}}%
\pgfpathlineto{\pgfqpoint{1.945322in}{2.416948in}}%
\pgfpathlineto{\pgfqpoint{1.940910in}{2.426705in}}%
\pgfpathlineto{\pgfqpoint{1.940664in}{2.427243in}}%
\pgfpathlineto{\pgfqpoint{1.936120in}{2.437537in}}%
\pgfpathlineto{\pgfqpoint{1.931708in}{2.447832in}}%
\pgfpathlineto{\pgfqpoint{1.930333in}{2.451186in}}%
\pgfpathlineto{\pgfqpoint{1.930333in}{2.447832in}}%
\pgfpathlineto{\pgfqpoint{1.930333in}{2.437537in}}%
\pgfpathlineto{\pgfqpoint{1.930333in}{2.427243in}}%
\pgfpathlineto{\pgfqpoint{1.930333in}{2.420321in}}%
\pgfpathlineto{\pgfqpoint{1.931902in}{2.416948in}}%
\pgfpathlineto{\pgfqpoint{1.936812in}{2.406654in}}%
\pgfpathlineto{\pgfqpoint{1.940910in}{2.398141in}}%
\pgfpathlineto{\pgfqpoint{1.941749in}{2.396360in}}%
\pgfpathlineto{\pgfqpoint{1.946621in}{2.386065in}}%
\pgfpathlineto{\pgfqpoint{1.951377in}{2.375771in}}%
\pgfpathlineto{\pgfqpoint{1.951487in}{2.375529in}}%
\pgfpathlineto{\pgfqpoint{1.955925in}{2.365476in}}%
\pgfpathlineto{\pgfqpoint{1.960204in}{2.355182in}}%
\pgfpathlineto{\pgfqpoint{1.962064in}{2.350437in}}%
\pgfpathlineto{\pgfqpoint{1.964191in}{2.344887in}}%
\pgfpathlineto{\pgfqpoint{1.967864in}{2.334593in}}%
\pgfpathlineto{\pgfqpoint{1.971202in}{2.324299in}}%
\pgfpathlineto{\pgfqpoint{1.972641in}{2.319466in}}%
\pgfpathlineto{\pgfqpoint{1.974310in}{2.314004in}}%
\pgfpathlineto{\pgfqpoint{1.977231in}{2.303710in}}%
\pgfpathlineto{\pgfqpoint{1.979948in}{2.293415in}}%
\pgfpathlineto{\pgfqpoint{1.982516in}{2.283121in}}%
\pgfpathlineto{\pgfqpoint{1.983218in}{2.280180in}}%
\pgfpathlineto{\pgfqpoint{1.985067in}{2.272826in}}%
\pgfpathlineto{\pgfqpoint{1.987574in}{2.262532in}}%
\pgfpathlineto{\pgfqpoint{1.990046in}{2.252238in}}%
\pgfpathlineto{\pgfqpoint{1.992527in}{2.241943in}}%
\pgfpathlineto{\pgfqpoint{1.993796in}{2.236747in}}%
\pgfpathlineto{\pgfqpoint{1.995101in}{2.231649in}}%
\pgfpathlineto{\pgfqpoint{1.997787in}{2.221354in}}%
\pgfpathlineto{\pgfqpoint{2.000577in}{2.211060in}}%
\pgfpathlineto{\pgfqpoint{2.003514in}{2.200765in}}%
\pgfpathlineto{\pgfqpoint{2.004373in}{2.197886in}}%
\pgfpathlineto{\pgfqpoint{2.006682in}{2.190471in}}%
\pgfpathlineto{\pgfqpoint{2.010064in}{2.180177in}}%
\pgfpathlineto{\pgfqpoint{2.013675in}{2.169882in}}%
\pgfpathlineto{\pgfqpoint{2.014950in}{2.166435in}}%
\pgfpathlineto{\pgfqpoint{2.017585in}{2.159588in}}%
\pgfpathlineto{\pgfqpoint{2.021775in}{2.149293in}}%
\pgfpathlineto{\pgfqpoint{2.025527in}{2.140616in}}%
\pgfpathlineto{\pgfqpoint{2.026252in}{2.138999in}}%
\pgfpathlineto{\pgfqpoint{2.031055in}{2.128704in}}%
\pgfpathlineto{\pgfqpoint{2.036104in}{2.118452in}}%
\pgfpathlineto{\pgfqpoint{2.036125in}{2.118410in}}%
\pgfpathlineto{\pgfqpoint{2.041497in}{2.108116in}}%
\pgfpathlineto{\pgfqpoint{2.046681in}{2.098534in}}%
\pgfpathlineto{\pgfqpoint{2.047079in}{2.097821in}}%
\pgfpathlineto{\pgfqpoint{2.052851in}{2.087527in}}%
\pgfpathlineto{\pgfqpoint{2.057258in}{2.078939in}}%
\pgfpathlineto{\pgfqpoint{2.058024in}{2.077232in}}%
\pgfpathlineto{\pgfqpoint{2.062254in}{2.066938in}}%
\pgfpathlineto{\pgfqpoint{2.066638in}{2.056643in}}%
\pgfpathlineto{\pgfqpoint{2.067835in}{2.054018in}}%
\pgfpathlineto{\pgfqpoint{2.071317in}{2.046349in}}%
\pgfpathlineto{\pgfqpoint{2.076323in}{2.036054in}}%
\pgfpathlineto{\pgfqpoint{2.078412in}{2.031971in}}%
\pgfpathlineto{\pgfqpoint{2.081751in}{2.025760in}}%
\pgfpathlineto{\pgfqpoint{2.087430in}{2.015466in}}%
\pgfpathlineto{\pgfqpoint{2.088989in}{2.012622in}}%
\pgfpathlineto{\pgfqpoint{2.093376in}{2.005171in}}%
\pgfpathlineto{\pgfqpoint{2.099124in}{1.994877in}}%
\pgfpathlineto{\pgfqpoint{2.099566in}{1.993995in}}%
\pgfpathlineto{\pgfqpoint{2.104703in}{1.984582in}}%
\pgfpathlineto{\pgfqpoint{2.109522in}{1.974288in}}%
\pgfpathlineto{\pgfqpoint{2.110143in}{1.972683in}}%
\pgfpathlineto{\pgfqpoint{2.113781in}{1.963993in}}%
\pgfpathlineto{\pgfqpoint{2.117221in}{1.953699in}}%
\pgfpathlineto{\pgfqpoint{2.119841in}{1.943405in}}%
\pgfpathlineto{\pgfqpoint{2.120720in}{1.938421in}}%
\pgfpathlineto{\pgfqpoint{2.121727in}{1.933110in}}%
\pgfpathlineto{\pgfqpoint{2.122804in}{1.922816in}}%
\pgfpathlineto{\pgfqpoint{2.122985in}{1.912521in}}%
\pgfpathlineto{\pgfqpoint{2.122217in}{1.902227in}}%
\pgfpathlineto{\pgfqpoint{2.120720in}{1.893606in}}%
\pgfpathlineto{\pgfqpoint{2.120440in}{1.891932in}}%
\pgfpathclose%
\pgfusepath{fill}%
\end{pgfscope}%
\begin{pgfscope}%
\pgfpathrectangle{\pgfqpoint{1.856795in}{1.819814in}}{\pgfqpoint{1.194205in}{1.163386in}}%
\pgfusepath{clip}%
\pgfsetbuttcap%
\pgfsetroundjoin%
\definecolor{currentfill}{rgb}{0.898503,0.224633,0.251087}%
\pgfsetfillcolor{currentfill}%
\pgfsetlinewidth{0.000000pt}%
\definecolor{currentstroke}{rgb}{0.000000,0.000000,0.000000}%
\pgfsetstrokecolor{currentstroke}%
\pgfsetdash{}{0pt}%
\pgfpathmoveto{\pgfqpoint{2.660150in}{2.529891in}}%
\pgfpathlineto{\pgfqpoint{2.670727in}{2.525256in}}%
\pgfpathlineto{\pgfqpoint{2.681305in}{2.522667in}}%
\pgfpathlineto{\pgfqpoint{2.691882in}{2.521754in}}%
\pgfpathlineto{\pgfqpoint{2.702459in}{2.522218in}}%
\pgfpathlineto{\pgfqpoint{2.713036in}{2.523853in}}%
\pgfpathlineto{\pgfqpoint{2.723613in}{2.526511in}}%
\pgfpathlineto{\pgfqpoint{2.734190in}{2.530065in}}%
\pgfpathlineto{\pgfqpoint{2.734491in}{2.530187in}}%
\pgfpathlineto{\pgfqpoint{2.744767in}{2.534285in}}%
\pgfpathlineto{\pgfqpoint{2.755344in}{2.539125in}}%
\pgfpathlineto{\pgfqpoint{2.758029in}{2.540482in}}%
\pgfpathlineto{\pgfqpoint{2.765921in}{2.544361in}}%
\pgfpathlineto{\pgfqpoint{2.776498in}{2.549926in}}%
\pgfpathlineto{\pgfqpoint{2.778029in}{2.550776in}}%
\pgfpathlineto{\pgfqpoint{2.787075in}{2.555615in}}%
\pgfpathlineto{\pgfqpoint{2.796901in}{2.561071in}}%
\pgfpathlineto{\pgfqpoint{2.797652in}{2.561471in}}%
\pgfpathlineto{\pgfqpoint{2.808229in}{2.567355in}}%
\pgfpathlineto{\pgfqpoint{2.815074in}{2.571365in}}%
\pgfpathlineto{\pgfqpoint{2.818806in}{2.573535in}}%
\pgfpathlineto{\pgfqpoint{2.829383in}{2.580389in}}%
\pgfpathlineto{\pgfqpoint{2.831116in}{2.581659in}}%
\pgfpathlineto{\pgfqpoint{2.839960in}{2.589774in}}%
\pgfpathlineto{\pgfqpoint{2.841954in}{2.591954in}}%
\pgfpathlineto{\pgfqpoint{2.847909in}{2.602248in}}%
\pgfpathlineto{\pgfqpoint{2.850538in}{2.609547in}}%
\pgfpathlineto{\pgfqpoint{2.851560in}{2.612543in}}%
\pgfpathlineto{\pgfqpoint{2.854259in}{2.622837in}}%
\pgfpathlineto{\pgfqpoint{2.856795in}{2.633132in}}%
\pgfpathlineto{\pgfqpoint{2.859508in}{2.643426in}}%
\pgfpathlineto{\pgfqpoint{2.861115in}{2.648802in}}%
\pgfpathlineto{\pgfqpoint{2.862489in}{2.653720in}}%
\pgfpathlineto{\pgfqpoint{2.865828in}{2.664015in}}%
\pgfpathlineto{\pgfqpoint{2.869759in}{2.674309in}}%
\pgfpathlineto{\pgfqpoint{2.871692in}{2.678634in}}%
\pgfpathlineto{\pgfqpoint{2.874202in}{2.684604in}}%
\pgfpathlineto{\pgfqpoint{2.879237in}{2.694898in}}%
\pgfpathlineto{\pgfqpoint{2.882269in}{2.700273in}}%
\pgfpathlineto{\pgfqpoint{2.884910in}{2.705193in}}%
\pgfpathlineto{\pgfqpoint{2.891227in}{2.715487in}}%
\pgfpathlineto{\pgfqpoint{2.892846in}{2.717819in}}%
\pgfpathlineto{\pgfqpoint{2.898156in}{2.725781in}}%
\pgfpathlineto{\pgfqpoint{2.903423in}{2.732802in}}%
\pgfpathlineto{\pgfqpoint{2.905800in}{2.736076in}}%
\pgfpathlineto{\pgfqpoint{2.914000in}{2.746204in}}%
\pgfpathlineto{\pgfqpoint{2.914131in}{2.746370in}}%
\pgfpathlineto{\pgfqpoint{2.923075in}{2.756665in}}%
\pgfpathlineto{\pgfqpoint{2.924577in}{2.758240in}}%
\pgfpathlineto{\pgfqpoint{2.932685in}{2.766959in}}%
\pgfpathlineto{\pgfqpoint{2.935154in}{2.769397in}}%
\pgfpathlineto{\pgfqpoint{2.942917in}{2.777254in}}%
\pgfpathlineto{\pgfqpoint{2.945731in}{2.779891in}}%
\pgfpathlineto{\pgfqpoint{2.953688in}{2.787548in}}%
\pgfpathlineto{\pgfqpoint{2.956308in}{2.789902in}}%
\pgfpathlineto{\pgfqpoint{2.964873in}{2.797842in}}%
\pgfpathlineto{\pgfqpoint{2.966885in}{2.799600in}}%
\pgfpathlineto{\pgfqpoint{2.976287in}{2.808137in}}%
\pgfpathlineto{\pgfqpoint{2.977462in}{2.809152in}}%
\pgfpathlineto{\pgfqpoint{2.977462in}{2.818431in}}%
\pgfpathlineto{\pgfqpoint{2.977462in}{2.825249in}}%
\pgfpathlineto{\pgfqpoint{2.969209in}{2.818431in}}%
\pgfpathlineto{\pgfqpoint{2.966885in}{2.816425in}}%
\pgfpathlineto{\pgfqpoint{2.956852in}{2.808137in}}%
\pgfpathlineto{\pgfqpoint{2.956308in}{2.807662in}}%
\pgfpathlineto{\pgfqpoint{2.945731in}{2.798764in}}%
\pgfpathlineto{\pgfqpoint{2.944602in}{2.797842in}}%
\pgfpathlineto{\pgfqpoint{2.935154in}{2.789616in}}%
\pgfpathlineto{\pgfqpoint{2.932719in}{2.787548in}}%
\pgfpathlineto{\pgfqpoint{2.924577in}{2.780120in}}%
\pgfpathlineto{\pgfqpoint{2.921360in}{2.777254in}}%
\pgfpathlineto{\pgfqpoint{2.914000in}{2.770159in}}%
\pgfpathlineto{\pgfqpoint{2.910597in}{2.766959in}}%
\pgfpathlineto{\pgfqpoint{2.903423in}{2.759613in}}%
\pgfpathlineto{\pgfqpoint{2.900459in}{2.756665in}}%
\pgfpathlineto{\pgfqpoint{2.892846in}{2.748359in}}%
\pgfpathlineto{\pgfqpoint{2.890957in}{2.746370in}}%
\pgfpathlineto{\pgfqpoint{2.882269in}{2.736264in}}%
\pgfpathlineto{\pgfqpoint{2.882100in}{2.736076in}}%
\pgfpathlineto{\pgfqpoint{2.873762in}{2.725781in}}%
\pgfpathlineto{\pgfqpoint{2.871692in}{2.722911in}}%
\pgfpathlineto{\pgfqpoint{2.866019in}{2.715487in}}%
\pgfpathlineto{\pgfqpoint{2.861115in}{2.708217in}}%
\pgfpathlineto{\pgfqpoint{2.858922in}{2.705193in}}%
\pgfpathlineto{\pgfqpoint{2.852378in}{2.694898in}}%
\pgfpathlineto{\pgfqpoint{2.850538in}{2.691553in}}%
\pgfpathlineto{\pgfqpoint{2.846344in}{2.684604in}}%
\pgfpathlineto{\pgfqpoint{2.841038in}{2.674309in}}%
\pgfpathlineto{\pgfqpoint{2.839960in}{2.671831in}}%
\pgfpathlineto{\pgfqpoint{2.836145in}{2.664015in}}%
\pgfpathlineto{\pgfqpoint{2.831984in}{2.653720in}}%
\pgfpathlineto{\pgfqpoint{2.829383in}{2.645812in}}%
\pgfpathlineto{\pgfqpoint{2.828479in}{2.643426in}}%
\pgfpathlineto{\pgfqpoint{2.825311in}{2.633132in}}%
\pgfpathlineto{\pgfqpoint{2.822750in}{2.622837in}}%
\pgfpathlineto{\pgfqpoint{2.820502in}{2.612543in}}%
\pgfpathlineto{\pgfqpoint{2.818806in}{2.606577in}}%
\pgfpathlineto{\pgfqpoint{2.817491in}{2.602248in}}%
\pgfpathlineto{\pgfqpoint{2.809081in}{2.591954in}}%
\pgfpathlineto{\pgfqpoint{2.808229in}{2.591387in}}%
\pgfpathlineto{\pgfqpoint{2.797652in}{2.585449in}}%
\pgfpathlineto{\pgfqpoint{2.789724in}{2.581659in}}%
\pgfpathlineto{\pgfqpoint{2.787075in}{2.580334in}}%
\pgfpathlineto{\pgfqpoint{2.776498in}{2.575346in}}%
\pgfpathlineto{\pgfqpoint{2.767982in}{2.571365in}}%
\pgfpathlineto{\pgfqpoint{2.765921in}{2.570376in}}%
\pgfpathlineto{\pgfqpoint{2.755344in}{2.565562in}}%
\pgfpathlineto{\pgfqpoint{2.744985in}{2.561071in}}%
\pgfpathlineto{\pgfqpoint{2.744767in}{2.560975in}}%
\pgfpathlineto{\pgfqpoint{2.734190in}{2.556842in}}%
\pgfpathlineto{\pgfqpoint{2.723613in}{2.553379in}}%
\pgfpathlineto{\pgfqpoint{2.713036in}{2.550870in}}%
\pgfpathlineto{\pgfqpoint{2.712190in}{2.550776in}}%
\pgfpathlineto{\pgfqpoint{2.702459in}{2.549653in}}%
\pgfpathlineto{\pgfqpoint{2.691882in}{2.550343in}}%
\pgfpathlineto{\pgfqpoint{2.690380in}{2.550776in}}%
\pgfpathlineto{\pgfqpoint{2.681305in}{2.554896in}}%
\pgfpathlineto{\pgfqpoint{2.674394in}{2.561071in}}%
\pgfpathlineto{\pgfqpoint{2.670727in}{2.568831in}}%
\pgfpathlineto{\pgfqpoint{2.669674in}{2.571365in}}%
\pgfpathlineto{\pgfqpoint{2.667904in}{2.581659in}}%
\pgfpathlineto{\pgfqpoint{2.667092in}{2.591954in}}%
\pgfpathlineto{\pgfqpoint{2.666324in}{2.602248in}}%
\pgfpathlineto{\pgfqpoint{2.664918in}{2.612543in}}%
\pgfpathlineto{\pgfqpoint{2.662290in}{2.622837in}}%
\pgfpathlineto{\pgfqpoint{2.660150in}{2.627960in}}%
\pgfpathlineto{\pgfqpoint{2.658091in}{2.633132in}}%
\pgfpathlineto{\pgfqpoint{2.651933in}{2.643426in}}%
\pgfpathlineto{\pgfqpoint{2.649573in}{2.646262in}}%
\pgfpathlineto{\pgfqpoint{2.643506in}{2.653720in}}%
\pgfpathlineto{\pgfqpoint{2.638996in}{2.657917in}}%
\pgfpathlineto{\pgfqpoint{2.632468in}{2.664015in}}%
\pgfpathlineto{\pgfqpoint{2.628419in}{2.667016in}}%
\pgfpathlineto{\pgfqpoint{2.618461in}{2.674309in}}%
\pgfpathlineto{\pgfqpoint{2.617842in}{2.674682in}}%
\pgfpathlineto{\pgfqpoint{2.607265in}{2.681005in}}%
\pgfpathlineto{\pgfqpoint{2.601035in}{2.684604in}}%
\pgfpathlineto{\pgfqpoint{2.596688in}{2.686741in}}%
\pgfpathlineto{\pgfqpoint{2.586111in}{2.691789in}}%
\pgfpathlineto{\pgfqpoint{2.579310in}{2.694898in}}%
\pgfpathlineto{\pgfqpoint{2.575534in}{2.696409in}}%
\pgfpathlineto{\pgfqpoint{2.564957in}{2.700495in}}%
\pgfpathlineto{\pgfqpoint{2.554380in}{2.704383in}}%
\pgfpathlineto{\pgfqpoint{2.552092in}{2.705193in}}%
\pgfpathlineto{\pgfqpoint{2.543803in}{2.707833in}}%
\pgfpathlineto{\pgfqpoint{2.533226in}{2.711079in}}%
\pgfpathlineto{\pgfqpoint{2.522649in}{2.714227in}}%
\pgfpathlineto{\pgfqpoint{2.518352in}{2.715487in}}%
\pgfpathlineto{\pgfqpoint{2.512072in}{2.717189in}}%
\pgfpathlineto{\pgfqpoint{2.501495in}{2.720092in}}%
\pgfpathlineto{\pgfqpoint{2.490917in}{2.723132in}}%
\pgfpathlineto{\pgfqpoint{2.482441in}{2.725781in}}%
\pgfpathlineto{\pgfqpoint{2.480340in}{2.726434in}}%
\pgfpathlineto{\pgfqpoint{2.469763in}{2.730244in}}%
\pgfpathlineto{\pgfqpoint{2.459186in}{2.735020in}}%
\pgfpathlineto{\pgfqpoint{2.457174in}{2.736076in}}%
\pgfpathlineto{\pgfqpoint{2.448609in}{2.742122in}}%
\pgfpathlineto{\pgfqpoint{2.443339in}{2.746370in}}%
\pgfpathlineto{\pgfqpoint{2.438032in}{2.752688in}}%
\pgfpathlineto{\pgfqpoint{2.434712in}{2.756665in}}%
\pgfpathlineto{\pgfqpoint{2.428376in}{2.766959in}}%
\pgfpathlineto{\pgfqpoint{2.427455in}{2.768871in}}%
\pgfpathlineto{\pgfqpoint{2.423253in}{2.777254in}}%
\pgfpathlineto{\pgfqpoint{2.419098in}{2.787548in}}%
\pgfpathlineto{\pgfqpoint{2.416878in}{2.794487in}}%
\pgfpathlineto{\pgfqpoint{2.415763in}{2.797842in}}%
\pgfpathlineto{\pgfqpoint{2.413182in}{2.808137in}}%
\pgfpathlineto{\pgfqpoint{2.411435in}{2.818431in}}%
\pgfpathlineto{\pgfqpoint{2.410533in}{2.828726in}}%
\pgfpathlineto{\pgfqpoint{2.410484in}{2.839020in}}%
\pgfpathlineto{\pgfqpoint{2.411291in}{2.849315in}}%
\pgfpathlineto{\pgfqpoint{2.412951in}{2.859609in}}%
\pgfpathlineto{\pgfqpoint{2.415455in}{2.869903in}}%
\pgfpathlineto{\pgfqpoint{2.416878in}{2.874250in}}%
\pgfpathlineto{\pgfqpoint{2.418684in}{2.880198in}}%
\pgfpathlineto{\pgfqpoint{2.422599in}{2.890492in}}%
\pgfpathlineto{\pgfqpoint{2.427269in}{2.900787in}}%
\pgfpathlineto{\pgfqpoint{2.427455in}{2.901134in}}%
\pgfpathlineto{\pgfqpoint{2.432361in}{2.911081in}}%
\pgfpathlineto{\pgfqpoint{2.427455in}{2.911081in}}%
\pgfpathlineto{\pgfqpoint{2.416878in}{2.911081in}}%
\pgfpathlineto{\pgfqpoint{2.406301in}{2.911081in}}%
\pgfpathlineto{\pgfqpoint{2.397077in}{2.911081in}}%
\pgfpathlineto{\pgfqpoint{2.395724in}{2.908997in}}%
\pgfpathlineto{\pgfqpoint{2.390022in}{2.900787in}}%
\pgfpathlineto{\pgfqpoint{2.385147in}{2.892665in}}%
\pgfpathlineto{\pgfqpoint{2.383769in}{2.890492in}}%
\pgfpathlineto{\pgfqpoint{2.378306in}{2.880198in}}%
\pgfpathlineto{\pgfqpoint{2.374570in}{2.871520in}}%
\pgfpathlineto{\pgfqpoint{2.373849in}{2.869903in}}%
\pgfpathlineto{\pgfqpoint{2.370366in}{2.859609in}}%
\pgfpathlineto{\pgfqpoint{2.367972in}{2.849315in}}%
\pgfpathlineto{\pgfqpoint{2.366667in}{2.839020in}}%
\pgfpathlineto{\pgfqpoint{2.366444in}{2.828726in}}%
\pgfpathlineto{\pgfqpoint{2.367293in}{2.818431in}}%
\pgfpathlineto{\pgfqpoint{2.369193in}{2.808137in}}%
\pgfpathlineto{\pgfqpoint{2.372118in}{2.797842in}}%
\pgfpathlineto{\pgfqpoint{2.374570in}{2.791441in}}%
\pgfpathlineto{\pgfqpoint{2.376037in}{2.787548in}}%
\pgfpathlineto{\pgfqpoint{2.380943in}{2.777254in}}%
\pgfpathlineto{\pgfqpoint{2.385147in}{2.770011in}}%
\pgfpathlineto{\pgfqpoint{2.386879in}{2.766959in}}%
\pgfpathlineto{\pgfqpoint{2.393971in}{2.756665in}}%
\pgfpathlineto{\pgfqpoint{2.395724in}{2.754571in}}%
\pgfpathlineto{\pgfqpoint{2.402553in}{2.746370in}}%
\pgfpathlineto{\pgfqpoint{2.406301in}{2.742755in}}%
\pgfpathlineto{\pgfqpoint{2.413437in}{2.736076in}}%
\pgfpathlineto{\pgfqpoint{2.416878in}{2.733504in}}%
\pgfpathlineto{\pgfqpoint{2.427455in}{2.726295in}}%
\pgfpathlineto{\pgfqpoint{2.428287in}{2.725781in}}%
\pgfpathlineto{\pgfqpoint{2.438032in}{2.720701in}}%
\pgfpathlineto{\pgfqpoint{2.448609in}{2.716055in}}%
\pgfpathlineto{\pgfqpoint{2.450068in}{2.715487in}}%
\pgfpathlineto{\pgfqpoint{2.459186in}{2.712054in}}%
\pgfpathlineto{\pgfqpoint{2.469763in}{2.708494in}}%
\pgfpathlineto{\pgfqpoint{2.480340in}{2.705232in}}%
\pgfpathlineto{\pgfqpoint{2.480472in}{2.705193in}}%
\pgfpathlineto{\pgfqpoint{2.490917in}{2.701892in}}%
\pgfpathlineto{\pgfqpoint{2.501495in}{2.698589in}}%
\pgfpathlineto{\pgfqpoint{2.512072in}{2.695244in}}%
\pgfpathlineto{\pgfqpoint{2.513135in}{2.694898in}}%
\pgfpathlineto{\pgfqpoint{2.522649in}{2.691489in}}%
\pgfpathlineto{\pgfqpoint{2.533226in}{2.687501in}}%
\pgfpathlineto{\pgfqpoint{2.540495in}{2.684604in}}%
\pgfpathlineto{\pgfqpoint{2.543803in}{2.683117in}}%
\pgfpathlineto{\pgfqpoint{2.554380in}{2.678068in}}%
\pgfpathlineto{\pgfqpoint{2.561743in}{2.674309in}}%
\pgfpathlineto{\pgfqpoint{2.564957in}{2.672408in}}%
\pgfpathlineto{\pgfqpoint{2.575534in}{2.665737in}}%
\pgfpathlineto{\pgfqpoint{2.578118in}{2.664015in}}%
\pgfpathlineto{\pgfqpoint{2.586111in}{2.657631in}}%
\pgfpathlineto{\pgfqpoint{2.590733in}{2.653720in}}%
\pgfpathlineto{\pgfqpoint{2.596688in}{2.647504in}}%
\pgfpathlineto{\pgfqpoint{2.600412in}{2.643426in}}%
\pgfpathlineto{\pgfqpoint{2.607265in}{2.633831in}}%
\pgfpathlineto{\pgfqpoint{2.607747in}{2.633132in}}%
\pgfpathlineto{\pgfqpoint{2.613308in}{2.622837in}}%
\pgfpathlineto{\pgfqpoint{2.617463in}{2.612543in}}%
\pgfpathlineto{\pgfqpoint{2.617842in}{2.611385in}}%
\pgfpathlineto{\pgfqpoint{2.620768in}{2.602248in}}%
\pgfpathlineto{\pgfqpoint{2.623561in}{2.591954in}}%
\pgfpathlineto{\pgfqpoint{2.626261in}{2.581659in}}%
\pgfpathlineto{\pgfqpoint{2.628419in}{2.574435in}}%
\pgfpathlineto{\pgfqpoint{2.629336in}{2.571365in}}%
\pgfpathlineto{\pgfqpoint{2.633315in}{2.561071in}}%
\pgfpathlineto{\pgfqpoint{2.638677in}{2.550776in}}%
\pgfpathlineto{\pgfqpoint{2.638996in}{2.550314in}}%
\pgfpathlineto{\pgfqpoint{2.646710in}{2.540482in}}%
\pgfpathlineto{\pgfqpoint{2.649573in}{2.537713in}}%
\pgfpathlineto{\pgfqpoint{2.659671in}{2.530187in}}%
\pgfpathclose%
\pgfusepath{fill}%
\end{pgfscope}%
\begin{pgfscope}%
\pgfpathrectangle{\pgfqpoint{1.856795in}{1.819814in}}{\pgfqpoint{1.194205in}{1.163386in}}%
\pgfusepath{clip}%
\pgfsetbuttcap%
\pgfsetroundjoin%
\definecolor{currentfill}{rgb}{0.928830,0.305981,0.243623}%
\pgfsetfillcolor{currentfill}%
\pgfsetlinewidth{0.000000pt}%
\definecolor{currentstroke}{rgb}{0.000000,0.000000,0.000000}%
\pgfsetstrokecolor{currentstroke}%
\pgfsetdash{}{0pt}%
\pgfpathmoveto{\pgfqpoint{2.152451in}{1.891932in}}%
\pgfpathlineto{\pgfqpoint{2.163029in}{1.891932in}}%
\pgfpathlineto{\pgfqpoint{2.169468in}{1.891932in}}%
\pgfpathlineto{\pgfqpoint{2.170788in}{1.902227in}}%
\pgfpathlineto{\pgfqpoint{2.171086in}{1.912521in}}%
\pgfpathlineto{\pgfqpoint{2.170433in}{1.922816in}}%
\pgfpathlineto{\pgfqpoint{2.168863in}{1.933110in}}%
\pgfpathlineto{\pgfqpoint{2.166373in}{1.943405in}}%
\pgfpathlineto{\pgfqpoint{2.163029in}{1.953332in}}%
\pgfpathlineto{\pgfqpoint{2.162912in}{1.953699in}}%
\pgfpathlineto{\pgfqpoint{2.158654in}{1.963993in}}%
\pgfpathlineto{\pgfqpoint{2.153150in}{1.974288in}}%
\pgfpathlineto{\pgfqpoint{2.152451in}{1.975333in}}%
\pgfpathlineto{\pgfqpoint{2.146679in}{1.984582in}}%
\pgfpathlineto{\pgfqpoint{2.141874in}{1.990996in}}%
\pgfpathlineto{\pgfqpoint{2.139190in}{1.994877in}}%
\pgfpathlineto{\pgfqpoint{2.131297in}{2.005130in}}%
\pgfpathlineto{\pgfqpoint{2.131268in}{2.005171in}}%
\pgfpathlineto{\pgfqpoint{2.123699in}{2.015466in}}%
\pgfpathlineto{\pgfqpoint{2.120720in}{2.019510in}}%
\pgfpathlineto{\pgfqpoint{2.116495in}{2.025760in}}%
\pgfpathlineto{\pgfqpoint{2.110143in}{2.035379in}}%
\pgfpathlineto{\pgfqpoint{2.109733in}{2.036054in}}%
\pgfpathlineto{\pgfqpoint{2.103761in}{2.046349in}}%
\pgfpathlineto{\pgfqpoint{2.099566in}{2.053921in}}%
\pgfpathlineto{\pgfqpoint{2.098180in}{2.056643in}}%
\pgfpathlineto{\pgfqpoint{2.093197in}{2.066938in}}%
\pgfpathlineto{\pgfqpoint{2.088989in}{2.075896in}}%
\pgfpathlineto{\pgfqpoint{2.088413in}{2.077232in}}%
\pgfpathlineto{\pgfqpoint{2.083936in}{2.087527in}}%
\pgfpathlineto{\pgfqpoint{2.079045in}{2.097821in}}%
\pgfpathlineto{\pgfqpoint{2.078412in}{2.099050in}}%
\pgfpathlineto{\pgfqpoint{2.073811in}{2.108116in}}%
\pgfpathlineto{\pgfqpoint{2.068198in}{2.118410in}}%
\pgfpathlineto{\pgfqpoint{2.067835in}{2.119069in}}%
\pgfpathlineto{\pgfqpoint{2.062625in}{2.128704in}}%
\pgfpathlineto{\pgfqpoint{2.057258in}{2.138735in}}%
\pgfpathlineto{\pgfqpoint{2.057121in}{2.138999in}}%
\pgfpathlineto{\pgfqpoint{2.051955in}{2.149293in}}%
\pgfpathlineto{\pgfqpoint{2.046963in}{2.159588in}}%
\pgfpathlineto{\pgfqpoint{2.046681in}{2.160192in}}%
\pgfpathlineto{\pgfqpoint{2.042374in}{2.169882in}}%
\pgfpathlineto{\pgfqpoint{2.037999in}{2.180177in}}%
\pgfpathlineto{\pgfqpoint{2.036104in}{2.184822in}}%
\pgfpathlineto{\pgfqpoint{2.033927in}{2.190471in}}%
\pgfpathlineto{\pgfqpoint{2.030138in}{2.200765in}}%
\pgfpathlineto{\pgfqpoint{2.026505in}{2.211060in}}%
\pgfpathlineto{\pgfqpoint{2.025527in}{2.213931in}}%
\pgfpathlineto{\pgfqpoint{2.023151in}{2.221354in}}%
\pgfpathlineto{\pgfqpoint{2.019970in}{2.231649in}}%
\pgfpathlineto{\pgfqpoint{2.016870in}{2.241943in}}%
\pgfpathlineto{\pgfqpoint{2.014950in}{2.248421in}}%
\pgfpathlineto{\pgfqpoint{2.013889in}{2.252238in}}%
\pgfpathlineto{\pgfqpoint{2.011056in}{2.262532in}}%
\pgfpathlineto{\pgfqpoint{2.008219in}{2.272826in}}%
\pgfpathlineto{\pgfqpoint{2.005342in}{2.283121in}}%
\pgfpathlineto{\pgfqpoint{2.004373in}{2.286523in}}%
\pgfpathlineto{\pgfqpoint{2.002524in}{2.293415in}}%
\pgfpathlineto{\pgfqpoint{1.999679in}{2.303710in}}%
\pgfpathlineto{\pgfqpoint{1.996708in}{2.314004in}}%
\pgfpathlineto{\pgfqpoint{1.993796in}{2.323597in}}%
\pgfpathlineto{\pgfqpoint{1.993592in}{2.324299in}}%
\pgfpathlineto{\pgfqpoint{1.990476in}{2.334593in}}%
\pgfpathlineto{\pgfqpoint{1.987149in}{2.344887in}}%
\pgfpathlineto{\pgfqpoint{1.983589in}{2.355182in}}%
\pgfpathlineto{\pgfqpoint{1.983218in}{2.356212in}}%
\pgfpathlineto{\pgfqpoint{1.979930in}{2.365476in}}%
\pgfpathlineto{\pgfqpoint{1.976036in}{2.375771in}}%
\pgfpathlineto{\pgfqpoint{1.972641in}{2.384276in}}%
\pgfpathlineto{\pgfqpoint{1.971923in}{2.386065in}}%
\pgfpathlineto{\pgfqpoint{1.967697in}{2.396360in}}%
\pgfpathlineto{\pgfqpoint{1.963309in}{2.406654in}}%
\pgfpathlineto{\pgfqpoint{1.962064in}{2.409554in}}%
\pgfpathlineto{\pgfqpoint{1.958873in}{2.416948in}}%
\pgfpathlineto{\pgfqpoint{1.954419in}{2.427243in}}%
\pgfpathlineto{\pgfqpoint{1.951487in}{2.434075in}}%
\pgfpathlineto{\pgfqpoint{1.950001in}{2.437537in}}%
\pgfpathlineto{\pgfqpoint{1.945709in}{2.447832in}}%
\pgfpathlineto{\pgfqpoint{1.941533in}{2.458126in}}%
\pgfpathlineto{\pgfqpoint{1.940910in}{2.459733in}}%
\pgfpathlineto{\pgfqpoint{1.937579in}{2.468421in}}%
\pgfpathlineto{\pgfqpoint{1.933831in}{2.478715in}}%
\pgfpathlineto{\pgfqpoint{1.930333in}{2.488907in}}%
\pgfpathlineto{\pgfqpoint{1.930333in}{2.478715in}}%
\pgfpathlineto{\pgfqpoint{1.930333in}{2.468421in}}%
\pgfpathlineto{\pgfqpoint{1.930333in}{2.458126in}}%
\pgfpathlineto{\pgfqpoint{1.930333in}{2.451186in}}%
\pgfpathlineto{\pgfqpoint{1.931708in}{2.447832in}}%
\pgfpathlineto{\pgfqpoint{1.936120in}{2.437537in}}%
\pgfpathlineto{\pgfqpoint{1.940664in}{2.427243in}}%
\pgfpathlineto{\pgfqpoint{1.940910in}{2.426705in}}%
\pgfpathlineto{\pgfqpoint{1.945322in}{2.416948in}}%
\pgfpathlineto{\pgfqpoint{1.949997in}{2.406654in}}%
\pgfpathlineto{\pgfqpoint{1.951487in}{2.403400in}}%
\pgfpathlineto{\pgfqpoint{1.954655in}{2.396360in}}%
\pgfpathlineto{\pgfqpoint{1.959206in}{2.386065in}}%
\pgfpathlineto{\pgfqpoint{1.962064in}{2.379414in}}%
\pgfpathlineto{\pgfqpoint{1.963600in}{2.375771in}}%
\pgfpathlineto{\pgfqpoint{1.967794in}{2.365476in}}%
\pgfpathlineto{\pgfqpoint{1.971713in}{2.355182in}}%
\pgfpathlineto{\pgfqpoint{1.972641in}{2.352608in}}%
\pgfpathlineto{\pgfqpoint{1.975428in}{2.344887in}}%
\pgfpathlineto{\pgfqpoint{1.978890in}{2.334593in}}%
\pgfpathlineto{\pgfqpoint{1.982089in}{2.324299in}}%
\pgfpathlineto{\pgfqpoint{1.983218in}{2.320427in}}%
\pgfpathlineto{\pgfqpoint{1.985163in}{2.314004in}}%
\pgfpathlineto{\pgfqpoint{1.988103in}{2.303710in}}%
\pgfpathlineto{\pgfqpoint{1.990885in}{2.293415in}}%
\pgfpathlineto{\pgfqpoint{1.993552in}{2.283121in}}%
\pgfpathlineto{\pgfqpoint{1.993796in}{2.282153in}}%
\pgfpathlineto{\pgfqpoint{1.996278in}{2.272826in}}%
\pgfpathlineto{\pgfqpoint{1.998960in}{2.262532in}}%
\pgfpathlineto{\pgfqpoint{2.001625in}{2.252238in}}%
\pgfpathlineto{\pgfqpoint{2.004312in}{2.241943in}}%
\pgfpathlineto{\pgfqpoint{2.004373in}{2.241712in}}%
\pgfpathlineto{\pgfqpoint{2.007184in}{2.231649in}}%
\pgfpathlineto{\pgfqpoint{2.010134in}{2.221354in}}%
\pgfpathlineto{\pgfqpoint{2.013196in}{2.211060in}}%
\pgfpathlineto{\pgfqpoint{2.014950in}{2.205411in}}%
\pgfpathlineto{\pgfqpoint{2.016470in}{2.200765in}}%
\pgfpathlineto{\pgfqpoint{2.019982in}{2.190471in}}%
\pgfpathlineto{\pgfqpoint{2.023688in}{2.180177in}}%
\pgfpathlineto{\pgfqpoint{2.025527in}{2.175330in}}%
\pgfpathlineto{\pgfqpoint{2.027693in}{2.169882in}}%
\pgfpathlineto{\pgfqpoint{2.031986in}{2.159588in}}%
\pgfpathlineto{\pgfqpoint{2.036104in}{2.150252in}}%
\pgfpathlineto{\pgfqpoint{2.036545in}{2.149293in}}%
\pgfpathlineto{\pgfqpoint{2.041464in}{2.138999in}}%
\pgfpathlineto{\pgfqpoint{2.046640in}{2.128704in}}%
\pgfpathlineto{\pgfqpoint{2.046681in}{2.128626in}}%
\pgfpathlineto{\pgfqpoint{2.052129in}{2.118410in}}%
\pgfpathlineto{\pgfqpoint{2.057258in}{2.109198in}}%
\pgfpathlineto{\pgfqpoint{2.057857in}{2.108116in}}%
\pgfpathlineto{\pgfqpoint{2.063571in}{2.097821in}}%
\pgfpathlineto{\pgfqpoint{2.067835in}{2.089808in}}%
\pgfpathlineto{\pgfqpoint{2.068952in}{2.087527in}}%
\pgfpathlineto{\pgfqpoint{2.073140in}{2.077232in}}%
\pgfpathlineto{\pgfqpoint{2.077202in}{2.066938in}}%
\pgfpathlineto{\pgfqpoint{2.078412in}{2.064057in}}%
\pgfpathlineto{\pgfqpoint{2.081722in}{2.056643in}}%
\pgfpathlineto{\pgfqpoint{2.086618in}{2.046349in}}%
\pgfpathlineto{\pgfqpoint{2.088989in}{2.041656in}}%
\pgfpathlineto{\pgfqpoint{2.092025in}{2.036054in}}%
\pgfpathlineto{\pgfqpoint{2.097865in}{2.025760in}}%
\pgfpathlineto{\pgfqpoint{2.099566in}{2.022847in}}%
\pgfpathlineto{\pgfqpoint{2.104221in}{2.015466in}}%
\pgfpathlineto{\pgfqpoint{2.110143in}{2.006047in}}%
\pgfpathlineto{\pgfqpoint{2.110741in}{2.005171in}}%
\pgfpathlineto{\pgfqpoint{2.117488in}{1.994877in}}%
\pgfpathlineto{\pgfqpoint{2.120720in}{1.989419in}}%
\pgfpathlineto{\pgfqpoint{2.123835in}{1.984582in}}%
\pgfpathlineto{\pgfqpoint{2.129481in}{1.974288in}}%
\pgfpathlineto{\pgfqpoint{2.131297in}{1.970243in}}%
\pgfpathlineto{\pgfqpoint{2.134320in}{1.963993in}}%
\pgfpathlineto{\pgfqpoint{2.138258in}{1.953699in}}%
\pgfpathlineto{\pgfqpoint{2.141257in}{1.943405in}}%
\pgfpathlineto{\pgfqpoint{2.141874in}{1.940418in}}%
\pgfpathlineto{\pgfqpoint{2.143494in}{1.933110in}}%
\pgfpathlineto{\pgfqpoint{2.144861in}{1.922816in}}%
\pgfpathlineto{\pgfqpoint{2.145302in}{1.912521in}}%
\pgfpathlineto{\pgfqpoint{2.144777in}{1.902227in}}%
\pgfpathlineto{\pgfqpoint{2.143216in}{1.891932in}}%
\pgfpathclose%
\pgfusepath{fill}%
\end{pgfscope}%
\begin{pgfscope}%
\pgfpathrectangle{\pgfqpoint{1.856795in}{1.819814in}}{\pgfqpoint{1.194205in}{1.163386in}}%
\pgfusepath{clip}%
\pgfsetbuttcap%
\pgfsetroundjoin%
\definecolor{currentfill}{rgb}{0.928830,0.305981,0.243623}%
\pgfsetfillcolor{currentfill}%
\pgfsetlinewidth{0.000000pt}%
\definecolor{currentstroke}{rgb}{0.000000,0.000000,0.000000}%
\pgfsetstrokecolor{currentstroke}%
\pgfsetdash{}{0pt}%
\pgfpathmoveto{\pgfqpoint{2.660150in}{2.497957in}}%
\pgfpathlineto{\pgfqpoint{2.670727in}{2.494883in}}%
\pgfpathlineto{\pgfqpoint{2.681305in}{2.493175in}}%
\pgfpathlineto{\pgfqpoint{2.691882in}{2.492707in}}%
\pgfpathlineto{\pgfqpoint{2.702459in}{2.493379in}}%
\pgfpathlineto{\pgfqpoint{2.713036in}{2.495113in}}%
\pgfpathlineto{\pgfqpoint{2.723613in}{2.497846in}}%
\pgfpathlineto{\pgfqpoint{2.727840in}{2.499304in}}%
\pgfpathlineto{\pgfqpoint{2.734190in}{2.501442in}}%
\pgfpathlineto{\pgfqpoint{2.744767in}{2.505819in}}%
\pgfpathlineto{\pgfqpoint{2.752577in}{2.509598in}}%
\pgfpathlineto{\pgfqpoint{2.755344in}{2.510904in}}%
\pgfpathlineto{\pgfqpoint{2.765921in}{2.516491in}}%
\pgfpathlineto{\pgfqpoint{2.771815in}{2.519893in}}%
\pgfpathlineto{\pgfqpoint{2.776498in}{2.522514in}}%
\pgfpathlineto{\pgfqpoint{2.787075in}{2.528821in}}%
\pgfpathlineto{\pgfqpoint{2.789261in}{2.530187in}}%
\pgfpathlineto{\pgfqpoint{2.797652in}{2.535242in}}%
\pgfpathlineto{\pgfqpoint{2.806052in}{2.540482in}}%
\pgfpathlineto{\pgfqpoint{2.808229in}{2.541788in}}%
\pgfpathlineto{\pgfqpoint{2.818806in}{2.548318in}}%
\pgfpathlineto{\pgfqpoint{2.822679in}{2.550776in}}%
\pgfpathlineto{\pgfqpoint{2.829383in}{2.554921in}}%
\pgfpathlineto{\pgfqpoint{2.838850in}{2.561071in}}%
\pgfpathlineto{\pgfqpoint{2.839960in}{2.561800in}}%
\pgfpathlineto{\pgfqpoint{2.850538in}{2.569396in}}%
\pgfpathlineto{\pgfqpoint{2.852978in}{2.571365in}}%
\pgfpathlineto{\pgfqpoint{2.861115in}{2.579250in}}%
\pgfpathlineto{\pgfqpoint{2.863226in}{2.581659in}}%
\pgfpathlineto{\pgfqpoint{2.869683in}{2.591954in}}%
\pgfpathlineto{\pgfqpoint{2.871692in}{2.596720in}}%
\pgfpathlineto{\pgfqpoint{2.873844in}{2.602248in}}%
\pgfpathlineto{\pgfqpoint{2.876850in}{2.612543in}}%
\pgfpathlineto{\pgfqpoint{2.879391in}{2.622837in}}%
\pgfpathlineto{\pgfqpoint{2.881832in}{2.633132in}}%
\pgfpathlineto{\pgfqpoint{2.882269in}{2.634886in}}%
\pgfpathlineto{\pgfqpoint{2.884311in}{2.643426in}}%
\pgfpathlineto{\pgfqpoint{2.887041in}{2.653720in}}%
\pgfpathlineto{\pgfqpoint{2.890169in}{2.664015in}}%
\pgfpathlineto{\pgfqpoint{2.892846in}{2.671609in}}%
\pgfpathlineto{\pgfqpoint{2.893761in}{2.674309in}}%
\pgfpathlineto{\pgfqpoint{2.897805in}{2.684604in}}%
\pgfpathlineto{\pgfqpoint{2.902491in}{2.694898in}}%
\pgfpathlineto{\pgfqpoint{2.903423in}{2.696682in}}%
\pgfpathlineto{\pgfqpoint{2.907722in}{2.705193in}}%
\pgfpathlineto{\pgfqpoint{2.913664in}{2.715487in}}%
\pgfpathlineto{\pgfqpoint{2.914000in}{2.716003in}}%
\pgfpathlineto{\pgfqpoint{2.920193in}{2.725781in}}%
\pgfpathlineto{\pgfqpoint{2.924577in}{2.731944in}}%
\pgfpathlineto{\pgfqpoint{2.927445in}{2.736076in}}%
\pgfpathlineto{\pgfqpoint{2.935154in}{2.746053in}}%
\pgfpathlineto{\pgfqpoint{2.935393in}{2.746370in}}%
\pgfpathlineto{\pgfqpoint{2.943955in}{2.756665in}}%
\pgfpathlineto{\pgfqpoint{2.945731in}{2.758615in}}%
\pgfpathlineto{\pgfqpoint{2.953148in}{2.766959in}}%
\pgfpathlineto{\pgfqpoint{2.956308in}{2.770236in}}%
\pgfpathlineto{\pgfqpoint{2.962900in}{2.777254in}}%
\pgfpathlineto{\pgfqpoint{2.966885in}{2.781201in}}%
\pgfpathlineto{\pgfqpoint{2.973098in}{2.787548in}}%
\pgfpathlineto{\pgfqpoint{2.977462in}{2.791739in}}%
\pgfpathlineto{\pgfqpoint{2.977462in}{2.797842in}}%
\pgfpathlineto{\pgfqpoint{2.977462in}{2.808137in}}%
\pgfpathlineto{\pgfqpoint{2.977462in}{2.809152in}}%
\pgfpathlineto{\pgfqpoint{2.976287in}{2.808137in}}%
\pgfpathlineto{\pgfqpoint{2.966885in}{2.799600in}}%
\pgfpathlineto{\pgfqpoint{2.964873in}{2.797842in}}%
\pgfpathlineto{\pgfqpoint{2.956308in}{2.789902in}}%
\pgfpathlineto{\pgfqpoint{2.953688in}{2.787548in}}%
\pgfpathlineto{\pgfqpoint{2.945731in}{2.779891in}}%
\pgfpathlineto{\pgfqpoint{2.942917in}{2.777254in}}%
\pgfpathlineto{\pgfqpoint{2.935154in}{2.769397in}}%
\pgfpathlineto{\pgfqpoint{2.932685in}{2.766959in}}%
\pgfpathlineto{\pgfqpoint{2.924577in}{2.758240in}}%
\pgfpathlineto{\pgfqpoint{2.923075in}{2.756665in}}%
\pgfpathlineto{\pgfqpoint{2.914131in}{2.746370in}}%
\pgfpathlineto{\pgfqpoint{2.914000in}{2.746204in}}%
\pgfpathlineto{\pgfqpoint{2.905800in}{2.736076in}}%
\pgfpathlineto{\pgfqpoint{2.903423in}{2.732802in}}%
\pgfpathlineto{\pgfqpoint{2.898156in}{2.725781in}}%
\pgfpathlineto{\pgfqpoint{2.892846in}{2.717819in}}%
\pgfpathlineto{\pgfqpoint{2.891227in}{2.715487in}}%
\pgfpathlineto{\pgfqpoint{2.884910in}{2.705193in}}%
\pgfpathlineto{\pgfqpoint{2.882269in}{2.700273in}}%
\pgfpathlineto{\pgfqpoint{2.879237in}{2.694898in}}%
\pgfpathlineto{\pgfqpoint{2.874202in}{2.684604in}}%
\pgfpathlineto{\pgfqpoint{2.871692in}{2.678634in}}%
\pgfpathlineto{\pgfqpoint{2.869759in}{2.674309in}}%
\pgfpathlineto{\pgfqpoint{2.865828in}{2.664015in}}%
\pgfpathlineto{\pgfqpoint{2.862489in}{2.653720in}}%
\pgfpathlineto{\pgfqpoint{2.861115in}{2.648802in}}%
\pgfpathlineto{\pgfqpoint{2.859508in}{2.643426in}}%
\pgfpathlineto{\pgfqpoint{2.856795in}{2.633132in}}%
\pgfpathlineto{\pgfqpoint{2.854259in}{2.622837in}}%
\pgfpathlineto{\pgfqpoint{2.851560in}{2.612543in}}%
\pgfpathlineto{\pgfqpoint{2.850538in}{2.609547in}}%
\pgfpathlineto{\pgfqpoint{2.847909in}{2.602248in}}%
\pgfpathlineto{\pgfqpoint{2.841954in}{2.591954in}}%
\pgfpathlineto{\pgfqpoint{2.839960in}{2.589774in}}%
\pgfpathlineto{\pgfqpoint{2.831116in}{2.581659in}}%
\pgfpathlineto{\pgfqpoint{2.829383in}{2.580389in}}%
\pgfpathlineto{\pgfqpoint{2.818806in}{2.573535in}}%
\pgfpathlineto{\pgfqpoint{2.815074in}{2.571365in}}%
\pgfpathlineto{\pgfqpoint{2.808229in}{2.567355in}}%
\pgfpathlineto{\pgfqpoint{2.797652in}{2.561471in}}%
\pgfpathlineto{\pgfqpoint{2.796901in}{2.561071in}}%
\pgfpathlineto{\pgfqpoint{2.787075in}{2.555615in}}%
\pgfpathlineto{\pgfqpoint{2.778029in}{2.550776in}}%
\pgfpathlineto{\pgfqpoint{2.776498in}{2.549926in}}%
\pgfpathlineto{\pgfqpoint{2.765921in}{2.544361in}}%
\pgfpathlineto{\pgfqpoint{2.758029in}{2.540482in}}%
\pgfpathlineto{\pgfqpoint{2.755344in}{2.539125in}}%
\pgfpathlineto{\pgfqpoint{2.744767in}{2.534285in}}%
\pgfpathlineto{\pgfqpoint{2.734491in}{2.530187in}}%
\pgfpathlineto{\pgfqpoint{2.734190in}{2.530065in}}%
\pgfpathlineto{\pgfqpoint{2.723613in}{2.526511in}}%
\pgfpathlineto{\pgfqpoint{2.713036in}{2.523853in}}%
\pgfpathlineto{\pgfqpoint{2.702459in}{2.522218in}}%
\pgfpathlineto{\pgfqpoint{2.691882in}{2.521754in}}%
\pgfpathlineto{\pgfqpoint{2.681305in}{2.522667in}}%
\pgfpathlineto{\pgfqpoint{2.670727in}{2.525256in}}%
\pgfpathlineto{\pgfqpoint{2.660150in}{2.529891in}}%
\pgfpathlineto{\pgfqpoint{2.659671in}{2.530187in}}%
\pgfpathlineto{\pgfqpoint{2.649573in}{2.537713in}}%
\pgfpathlineto{\pgfqpoint{2.646710in}{2.540482in}}%
\pgfpathlineto{\pgfqpoint{2.638996in}{2.550314in}}%
\pgfpathlineto{\pgfqpoint{2.638677in}{2.550776in}}%
\pgfpathlineto{\pgfqpoint{2.633315in}{2.561071in}}%
\pgfpathlineto{\pgfqpoint{2.629336in}{2.571365in}}%
\pgfpathlineto{\pgfqpoint{2.628419in}{2.574435in}}%
\pgfpathlineto{\pgfqpoint{2.626261in}{2.581659in}}%
\pgfpathlineto{\pgfqpoint{2.623561in}{2.591954in}}%
\pgfpathlineto{\pgfqpoint{2.620768in}{2.602248in}}%
\pgfpathlineto{\pgfqpoint{2.617842in}{2.611385in}}%
\pgfpathlineto{\pgfqpoint{2.617463in}{2.612543in}}%
\pgfpathlineto{\pgfqpoint{2.613308in}{2.622837in}}%
\pgfpathlineto{\pgfqpoint{2.607747in}{2.633132in}}%
\pgfpathlineto{\pgfqpoint{2.607265in}{2.633831in}}%
\pgfpathlineto{\pgfqpoint{2.600412in}{2.643426in}}%
\pgfpathlineto{\pgfqpoint{2.596688in}{2.647504in}}%
\pgfpathlineto{\pgfqpoint{2.590733in}{2.653720in}}%
\pgfpathlineto{\pgfqpoint{2.586111in}{2.657631in}}%
\pgfpathlineto{\pgfqpoint{2.578118in}{2.664015in}}%
\pgfpathlineto{\pgfqpoint{2.575534in}{2.665737in}}%
\pgfpathlineto{\pgfqpoint{2.564957in}{2.672408in}}%
\pgfpathlineto{\pgfqpoint{2.561743in}{2.674309in}}%
\pgfpathlineto{\pgfqpoint{2.554380in}{2.678068in}}%
\pgfpathlineto{\pgfqpoint{2.543803in}{2.683117in}}%
\pgfpathlineto{\pgfqpoint{2.540495in}{2.684604in}}%
\pgfpathlineto{\pgfqpoint{2.533226in}{2.687501in}}%
\pgfpathlineto{\pgfqpoint{2.522649in}{2.691489in}}%
\pgfpathlineto{\pgfqpoint{2.513135in}{2.694898in}}%
\pgfpathlineto{\pgfqpoint{2.512072in}{2.695244in}}%
\pgfpathlineto{\pgfqpoint{2.501495in}{2.698589in}}%
\pgfpathlineto{\pgfqpoint{2.490917in}{2.701892in}}%
\pgfpathlineto{\pgfqpoint{2.480472in}{2.705193in}}%
\pgfpathlineto{\pgfqpoint{2.480340in}{2.705232in}}%
\pgfpathlineto{\pgfqpoint{2.469763in}{2.708494in}}%
\pgfpathlineto{\pgfqpoint{2.459186in}{2.712054in}}%
\pgfpathlineto{\pgfqpoint{2.450068in}{2.715487in}}%
\pgfpathlineto{\pgfqpoint{2.448609in}{2.716055in}}%
\pgfpathlineto{\pgfqpoint{2.438032in}{2.720701in}}%
\pgfpathlineto{\pgfqpoint{2.428287in}{2.725781in}}%
\pgfpathlineto{\pgfqpoint{2.427455in}{2.726295in}}%
\pgfpathlineto{\pgfqpoint{2.416878in}{2.733504in}}%
\pgfpathlineto{\pgfqpoint{2.413437in}{2.736076in}}%
\pgfpathlineto{\pgfqpoint{2.406301in}{2.742755in}}%
\pgfpathlineto{\pgfqpoint{2.402553in}{2.746370in}}%
\pgfpathlineto{\pgfqpoint{2.395724in}{2.754571in}}%
\pgfpathlineto{\pgfqpoint{2.393971in}{2.756665in}}%
\pgfpathlineto{\pgfqpoint{2.386879in}{2.766959in}}%
\pgfpathlineto{\pgfqpoint{2.385147in}{2.770011in}}%
\pgfpathlineto{\pgfqpoint{2.380943in}{2.777254in}}%
\pgfpathlineto{\pgfqpoint{2.376037in}{2.787548in}}%
\pgfpathlineto{\pgfqpoint{2.374570in}{2.791441in}}%
\pgfpathlineto{\pgfqpoint{2.372118in}{2.797842in}}%
\pgfpathlineto{\pgfqpoint{2.369193in}{2.808137in}}%
\pgfpathlineto{\pgfqpoint{2.367293in}{2.818431in}}%
\pgfpathlineto{\pgfqpoint{2.366444in}{2.828726in}}%
\pgfpathlineto{\pgfqpoint{2.366667in}{2.839020in}}%
\pgfpathlineto{\pgfqpoint{2.367972in}{2.849315in}}%
\pgfpathlineto{\pgfqpoint{2.370366in}{2.859609in}}%
\pgfpathlineto{\pgfqpoint{2.373849in}{2.869903in}}%
\pgfpathlineto{\pgfqpoint{2.374570in}{2.871520in}}%
\pgfpathlineto{\pgfqpoint{2.378306in}{2.880198in}}%
\pgfpathlineto{\pgfqpoint{2.383769in}{2.890492in}}%
\pgfpathlineto{\pgfqpoint{2.385147in}{2.892665in}}%
\pgfpathlineto{\pgfqpoint{2.390022in}{2.900787in}}%
\pgfpathlineto{\pgfqpoint{2.395724in}{2.908997in}}%
\pgfpathlineto{\pgfqpoint{2.397077in}{2.911081in}}%
\pgfpathlineto{\pgfqpoint{2.395724in}{2.911081in}}%
\pgfpathlineto{\pgfqpoint{2.385147in}{2.911081in}}%
\pgfpathlineto{\pgfqpoint{2.374570in}{2.911081in}}%
\pgfpathlineto{\pgfqpoint{2.363993in}{2.911081in}}%
\pgfpathlineto{\pgfqpoint{2.353954in}{2.911081in}}%
\pgfpathlineto{\pgfqpoint{2.353416in}{2.910409in}}%
\pgfpathlineto{\pgfqpoint{2.345581in}{2.900787in}}%
\pgfpathlineto{\pgfqpoint{2.342839in}{2.896831in}}%
\pgfpathlineto{\pgfqpoint{2.338457in}{2.890492in}}%
\pgfpathlineto{\pgfqpoint{2.332596in}{2.880198in}}%
\pgfpathlineto{\pgfqpoint{2.332262in}{2.879459in}}%
\pgfpathlineto{\pgfqpoint{2.328033in}{2.869903in}}%
\pgfpathlineto{\pgfqpoint{2.324666in}{2.859609in}}%
\pgfpathlineto{\pgfqpoint{2.322451in}{2.849315in}}%
\pgfpathlineto{\pgfqpoint{2.321684in}{2.842136in}}%
\pgfpathlineto{\pgfqpoint{2.321365in}{2.839020in}}%
\pgfpathlineto{\pgfqpoint{2.321365in}{2.828726in}}%
\pgfpathlineto{\pgfqpoint{2.321684in}{2.825566in}}%
\pgfpathlineto{\pgfqpoint{2.322425in}{2.818431in}}%
\pgfpathlineto{\pgfqpoint{2.324548in}{2.808137in}}%
\pgfpathlineto{\pgfqpoint{2.327695in}{2.797842in}}%
\pgfpathlineto{\pgfqpoint{2.331837in}{2.787548in}}%
\pgfpathlineto{\pgfqpoint{2.332262in}{2.786701in}}%
\pgfpathlineto{\pgfqpoint{2.337075in}{2.777254in}}%
\pgfpathlineto{\pgfqpoint{2.342839in}{2.767701in}}%
\pgfpathlineto{\pgfqpoint{2.343291in}{2.766959in}}%
\pgfpathlineto{\pgfqpoint{2.350599in}{2.756665in}}%
\pgfpathlineto{\pgfqpoint{2.353416in}{2.753184in}}%
\pgfpathlineto{\pgfqpoint{2.359018in}{2.746370in}}%
\pgfpathlineto{\pgfqpoint{2.363993in}{2.741041in}}%
\pgfpathlineto{\pgfqpoint{2.368759in}{2.736076in}}%
\pgfpathlineto{\pgfqpoint{2.374570in}{2.730740in}}%
\pgfpathlineto{\pgfqpoint{2.380265in}{2.725781in}}%
\pgfpathlineto{\pgfqpoint{2.385147in}{2.722006in}}%
\pgfpathlineto{\pgfqpoint{2.394392in}{2.715487in}}%
\pgfpathlineto{\pgfqpoint{2.395724in}{2.714633in}}%
\pgfpathlineto{\pgfqpoint{2.406301in}{2.708471in}}%
\pgfpathlineto{\pgfqpoint{2.412754in}{2.705193in}}%
\pgfpathlineto{\pgfqpoint{2.416878in}{2.703202in}}%
\pgfpathlineto{\pgfqpoint{2.427455in}{2.698638in}}%
\pgfpathlineto{\pgfqpoint{2.437360in}{2.694898in}}%
\pgfpathlineto{\pgfqpoint{2.438032in}{2.694643in}}%
\pgfpathlineto{\pgfqpoint{2.448609in}{2.690859in}}%
\pgfpathlineto{\pgfqpoint{2.459186in}{2.687339in}}%
\pgfpathlineto{\pgfqpoint{2.467720in}{2.684604in}}%
\pgfpathlineto{\pgfqpoint{2.469763in}{2.683908in}}%
\pgfpathlineto{\pgfqpoint{2.480340in}{2.680271in}}%
\pgfpathlineto{\pgfqpoint{2.490917in}{2.676550in}}%
\pgfpathlineto{\pgfqpoint{2.497016in}{2.674309in}}%
\pgfpathlineto{\pgfqpoint{2.501495in}{2.672497in}}%
\pgfpathlineto{\pgfqpoint{2.512072in}{2.667929in}}%
\pgfpathlineto{\pgfqpoint{2.520462in}{2.664015in}}%
\pgfpathlineto{\pgfqpoint{2.522649in}{2.662865in}}%
\pgfpathlineto{\pgfqpoint{2.533226in}{2.656824in}}%
\pgfpathlineto{\pgfqpoint{2.538201in}{2.653720in}}%
\pgfpathlineto{\pgfqpoint{2.543803in}{2.649679in}}%
\pgfpathlineto{\pgfqpoint{2.551700in}{2.643426in}}%
\pgfpathlineto{\pgfqpoint{2.554380in}{2.640930in}}%
\pgfpathlineto{\pgfqpoint{2.562048in}{2.633132in}}%
\pgfpathlineto{\pgfqpoint{2.564957in}{2.629599in}}%
\pgfpathlineto{\pgfqpoint{2.570098in}{2.622837in}}%
\pgfpathlineto{\pgfqpoint{2.575534in}{2.614206in}}%
\pgfpathlineto{\pgfqpoint{2.576511in}{2.612543in}}%
\pgfpathlineto{\pgfqpoint{2.581740in}{2.602248in}}%
\pgfpathlineto{\pgfqpoint{2.586111in}{2.592350in}}%
\pgfpathlineto{\pgfqpoint{2.586277in}{2.591954in}}%
\pgfpathlineto{\pgfqpoint{2.590434in}{2.581659in}}%
\pgfpathlineto{\pgfqpoint{2.594574in}{2.571365in}}%
\pgfpathlineto{\pgfqpoint{2.596688in}{2.566497in}}%
\pgfpathlineto{\pgfqpoint{2.598991in}{2.561071in}}%
\pgfpathlineto{\pgfqpoint{2.603971in}{2.550776in}}%
\pgfpathlineto{\pgfqpoint{2.607265in}{2.544936in}}%
\pgfpathlineto{\pgfqpoint{2.609846in}{2.540482in}}%
\pgfpathlineto{\pgfqpoint{2.616977in}{2.530187in}}%
\pgfpathlineto{\pgfqpoint{2.617842in}{2.529112in}}%
\pgfpathlineto{\pgfqpoint{2.626061in}{2.519893in}}%
\pgfpathlineto{\pgfqpoint{2.628419in}{2.517594in}}%
\pgfpathlineto{\pgfqpoint{2.638119in}{2.509598in}}%
\pgfpathlineto{\pgfqpoint{2.638996in}{2.508945in}}%
\pgfpathlineto{\pgfqpoint{2.649573in}{2.502567in}}%
\pgfpathlineto{\pgfqpoint{2.656973in}{2.499304in}}%
\pgfpathclose%
\pgfusepath{fill}%
\end{pgfscope}%
\begin{pgfscope}%
\pgfpathrectangle{\pgfqpoint{1.856795in}{1.819814in}}{\pgfqpoint{1.194205in}{1.163386in}}%
\pgfusepath{clip}%
\pgfsetbuttcap%
\pgfsetroundjoin%
\definecolor{currentfill}{rgb}{0.945204,0.390623,0.270949}%
\pgfsetfillcolor{currentfill}%
\pgfsetlinewidth{0.000000pt}%
\definecolor{currentstroke}{rgb}{0.000000,0.000000,0.000000}%
\pgfsetstrokecolor{currentstroke}%
\pgfsetdash{}{0pt}%
\pgfpathmoveto{\pgfqpoint{2.173606in}{1.891932in}}%
\pgfpathlineto{\pgfqpoint{2.184183in}{1.891932in}}%
\pgfpathlineto{\pgfqpoint{2.194760in}{1.891932in}}%
\pgfpathlineto{\pgfqpoint{2.200316in}{1.891932in}}%
\pgfpathlineto{\pgfqpoint{2.201401in}{1.902227in}}%
\pgfpathlineto{\pgfqpoint{2.201476in}{1.912521in}}%
\pgfpathlineto{\pgfqpoint{2.200617in}{1.922816in}}%
\pgfpathlineto{\pgfqpoint{2.198863in}{1.933110in}}%
\pgfpathlineto{\pgfqpoint{2.196213in}{1.943405in}}%
\pgfpathlineto{\pgfqpoint{2.194760in}{1.947597in}}%
\pgfpathlineto{\pgfqpoint{2.192720in}{1.953699in}}%
\pgfpathlineto{\pgfqpoint{2.188239in}{1.963993in}}%
\pgfpathlineto{\pgfqpoint{2.184183in}{1.971137in}}%
\pgfpathlineto{\pgfqpoint{2.182450in}{1.974288in}}%
\pgfpathlineto{\pgfqpoint{2.175071in}{1.984582in}}%
\pgfpathlineto{\pgfqpoint{2.173606in}{1.986211in}}%
\pgfpathlineto{\pgfqpoint{2.166249in}{1.994877in}}%
\pgfpathlineto{\pgfqpoint{2.163029in}{1.998209in}}%
\pgfpathlineto{\pgfqpoint{2.156785in}{2.005171in}}%
\pgfpathlineto{\pgfqpoint{2.152451in}{2.009778in}}%
\pgfpathlineto{\pgfqpoint{2.147527in}{2.015466in}}%
\pgfpathlineto{\pgfqpoint{2.141874in}{2.021977in}}%
\pgfpathlineto{\pgfqpoint{2.138864in}{2.025760in}}%
\pgfpathlineto{\pgfqpoint{2.131297in}{2.035478in}}%
\pgfpathlineto{\pgfqpoint{2.130887in}{2.036054in}}%
\pgfpathlineto{\pgfqpoint{2.123852in}{2.046349in}}%
\pgfpathlineto{\pgfqpoint{2.120720in}{2.051120in}}%
\pgfpathlineto{\pgfqpoint{2.117417in}{2.056643in}}%
\pgfpathlineto{\pgfqpoint{2.111504in}{2.066938in}}%
\pgfpathlineto{\pgfqpoint{2.110143in}{2.069379in}}%
\pgfpathlineto{\pgfqpoint{2.106160in}{2.077232in}}%
\pgfpathlineto{\pgfqpoint{2.100941in}{2.087527in}}%
\pgfpathlineto{\pgfqpoint{2.099566in}{2.090170in}}%
\pgfpathlineto{\pgfqpoint{2.095890in}{2.097821in}}%
\pgfpathlineto{\pgfqpoint{2.090575in}{2.108116in}}%
\pgfpathlineto{\pgfqpoint{2.088989in}{2.111016in}}%
\pgfpathlineto{\pgfqpoint{2.085125in}{2.118410in}}%
\pgfpathlineto{\pgfqpoint{2.079448in}{2.128704in}}%
\pgfpathlineto{\pgfqpoint{2.078412in}{2.130558in}}%
\pgfpathlineto{\pgfqpoint{2.073889in}{2.138999in}}%
\pgfpathlineto{\pgfqpoint{2.068344in}{2.149293in}}%
\pgfpathlineto{\pgfqpoint{2.067835in}{2.150256in}}%
\pgfpathlineto{\pgfqpoint{2.063160in}{2.159588in}}%
\pgfpathlineto{\pgfqpoint{2.058105in}{2.169882in}}%
\pgfpathlineto{\pgfqpoint{2.057258in}{2.171662in}}%
\pgfpathlineto{\pgfqpoint{2.053453in}{2.180177in}}%
\pgfpathlineto{\pgfqpoint{2.049004in}{2.190471in}}%
\pgfpathlineto{\pgfqpoint{2.046681in}{2.196031in}}%
\pgfpathlineto{\pgfqpoint{2.044832in}{2.200765in}}%
\pgfpathlineto{\pgfqpoint{2.040965in}{2.211060in}}%
\pgfpathlineto{\pgfqpoint{2.037218in}{2.221354in}}%
\pgfpathlineto{\pgfqpoint{2.036104in}{2.224506in}}%
\pgfpathlineto{\pgfqpoint{2.033756in}{2.231649in}}%
\pgfpathlineto{\pgfqpoint{2.030460in}{2.241943in}}%
\pgfpathlineto{\pgfqpoint{2.027217in}{2.252238in}}%
\pgfpathlineto{\pgfqpoint{2.025527in}{2.257656in}}%
\pgfpathlineto{\pgfqpoint{2.024112in}{2.262532in}}%
\pgfpathlineto{\pgfqpoint{2.021139in}{2.272826in}}%
\pgfpathlineto{\pgfqpoint{2.018141in}{2.283121in}}%
\pgfpathlineto{\pgfqpoint{2.015083in}{2.293415in}}%
\pgfpathlineto{\pgfqpoint{2.014950in}{2.293857in}}%
\pgfpathlineto{\pgfqpoint{2.012169in}{2.303710in}}%
\pgfpathlineto{\pgfqpoint{2.009163in}{2.314004in}}%
\pgfpathlineto{\pgfqpoint{2.006025in}{2.324299in}}%
\pgfpathlineto{\pgfqpoint{2.004373in}{2.329511in}}%
\pgfpathlineto{\pgfqpoint{2.002844in}{2.334593in}}%
\pgfpathlineto{\pgfqpoint{1.999607in}{2.344887in}}%
\pgfpathlineto{\pgfqpoint{1.996176in}{2.355182in}}%
\pgfpathlineto{\pgfqpoint{1.993796in}{2.361975in}}%
\pgfpathlineto{\pgfqpoint{1.992605in}{2.365476in}}%
\pgfpathlineto{\pgfqpoint{1.988951in}{2.375771in}}%
\pgfpathlineto{\pgfqpoint{1.985078in}{2.386065in}}%
\pgfpathlineto{\pgfqpoint{1.983218in}{2.390837in}}%
\pgfpathlineto{\pgfqpoint{1.981084in}{2.396360in}}%
\pgfpathlineto{\pgfqpoint{1.976984in}{2.406654in}}%
\pgfpathlineto{\pgfqpoint{1.972734in}{2.416948in}}%
\pgfpathlineto{\pgfqpoint{1.972641in}{2.417172in}}%
\pgfpathlineto{\pgfqpoint{1.968514in}{2.427243in}}%
\pgfpathlineto{\pgfqpoint{1.964251in}{2.437537in}}%
\pgfpathlineto{\pgfqpoint{1.962064in}{2.442869in}}%
\pgfpathlineto{\pgfqpoint{1.960050in}{2.447832in}}%
\pgfpathlineto{\pgfqpoint{1.955973in}{2.458126in}}%
\pgfpathlineto{\pgfqpoint{1.951995in}{2.468421in}}%
\pgfpathlineto{\pgfqpoint{1.951487in}{2.469794in}}%
\pgfpathlineto{\pgfqpoint{1.948257in}{2.478715in}}%
\pgfpathlineto{\pgfqpoint{1.944706in}{2.489009in}}%
\pgfpathlineto{\pgfqpoint{1.941345in}{2.499304in}}%
\pgfpathlineto{\pgfqpoint{1.940910in}{2.500725in}}%
\pgfpathlineto{\pgfqpoint{1.938271in}{2.509598in}}%
\pgfpathlineto{\pgfqpoint{1.935420in}{2.519893in}}%
\pgfpathlineto{\pgfqpoint{1.932775in}{2.530187in}}%
\pgfpathlineto{\pgfqpoint{1.930333in}{2.540455in}}%
\pgfpathlineto{\pgfqpoint{1.930333in}{2.530187in}}%
\pgfpathlineto{\pgfqpoint{1.930333in}{2.519893in}}%
\pgfpathlineto{\pgfqpoint{1.930333in}{2.509598in}}%
\pgfpathlineto{\pgfqpoint{1.930333in}{2.499304in}}%
\pgfpathlineto{\pgfqpoint{1.930333in}{2.489009in}}%
\pgfpathlineto{\pgfqpoint{1.930333in}{2.488907in}}%
\pgfpathlineto{\pgfqpoint{1.933831in}{2.478715in}}%
\pgfpathlineto{\pgfqpoint{1.937579in}{2.468421in}}%
\pgfpathlineto{\pgfqpoint{1.940910in}{2.459733in}}%
\pgfpathlineto{\pgfqpoint{1.941533in}{2.458126in}}%
\pgfpathlineto{\pgfqpoint{1.945709in}{2.447832in}}%
\pgfpathlineto{\pgfqpoint{1.950001in}{2.437537in}}%
\pgfpathlineto{\pgfqpoint{1.951487in}{2.434075in}}%
\pgfpathlineto{\pgfqpoint{1.954419in}{2.427243in}}%
\pgfpathlineto{\pgfqpoint{1.958873in}{2.416948in}}%
\pgfpathlineto{\pgfqpoint{1.962064in}{2.409554in}}%
\pgfpathlineto{\pgfqpoint{1.963309in}{2.406654in}}%
\pgfpathlineto{\pgfqpoint{1.967697in}{2.396360in}}%
\pgfpathlineto{\pgfqpoint{1.971923in}{2.386065in}}%
\pgfpathlineto{\pgfqpoint{1.972641in}{2.384276in}}%
\pgfpathlineto{\pgfqpoint{1.976036in}{2.375771in}}%
\pgfpathlineto{\pgfqpoint{1.979930in}{2.365476in}}%
\pgfpathlineto{\pgfqpoint{1.983218in}{2.356212in}}%
\pgfpathlineto{\pgfqpoint{1.983589in}{2.355182in}}%
\pgfpathlineto{\pgfqpoint{1.987149in}{2.344887in}}%
\pgfpathlineto{\pgfqpoint{1.990476in}{2.334593in}}%
\pgfpathlineto{\pgfqpoint{1.993592in}{2.324299in}}%
\pgfpathlineto{\pgfqpoint{1.993796in}{2.323597in}}%
\pgfpathlineto{\pgfqpoint{1.996708in}{2.314004in}}%
\pgfpathlineto{\pgfqpoint{1.999679in}{2.303710in}}%
\pgfpathlineto{\pgfqpoint{2.002524in}{2.293415in}}%
\pgfpathlineto{\pgfqpoint{2.004373in}{2.286523in}}%
\pgfpathlineto{\pgfqpoint{2.005342in}{2.283121in}}%
\pgfpathlineto{\pgfqpoint{2.008219in}{2.272826in}}%
\pgfpathlineto{\pgfqpoint{2.011056in}{2.262532in}}%
\pgfpathlineto{\pgfqpoint{2.013889in}{2.252238in}}%
\pgfpathlineto{\pgfqpoint{2.014950in}{2.248421in}}%
\pgfpathlineto{\pgfqpoint{2.016870in}{2.241943in}}%
\pgfpathlineto{\pgfqpoint{2.019970in}{2.231649in}}%
\pgfpathlineto{\pgfqpoint{2.023151in}{2.221354in}}%
\pgfpathlineto{\pgfqpoint{2.025527in}{2.213931in}}%
\pgfpathlineto{\pgfqpoint{2.026505in}{2.211060in}}%
\pgfpathlineto{\pgfqpoint{2.030138in}{2.200765in}}%
\pgfpathlineto{\pgfqpoint{2.033927in}{2.190471in}}%
\pgfpathlineto{\pgfqpoint{2.036104in}{2.184822in}}%
\pgfpathlineto{\pgfqpoint{2.037999in}{2.180177in}}%
\pgfpathlineto{\pgfqpoint{2.042374in}{2.169882in}}%
\pgfpathlineto{\pgfqpoint{2.046681in}{2.160192in}}%
\pgfpathlineto{\pgfqpoint{2.046963in}{2.159588in}}%
\pgfpathlineto{\pgfqpoint{2.051955in}{2.149293in}}%
\pgfpathlineto{\pgfqpoint{2.057121in}{2.138999in}}%
\pgfpathlineto{\pgfqpoint{2.057258in}{2.138735in}}%
\pgfpathlineto{\pgfqpoint{2.062625in}{2.128704in}}%
\pgfpathlineto{\pgfqpoint{2.067835in}{2.119069in}}%
\pgfpathlineto{\pgfqpoint{2.068198in}{2.118410in}}%
\pgfpathlineto{\pgfqpoint{2.073811in}{2.108116in}}%
\pgfpathlineto{\pgfqpoint{2.078412in}{2.099050in}}%
\pgfpathlineto{\pgfqpoint{2.079045in}{2.097821in}}%
\pgfpathlineto{\pgfqpoint{2.083936in}{2.087527in}}%
\pgfpathlineto{\pgfqpoint{2.088413in}{2.077232in}}%
\pgfpathlineto{\pgfqpoint{2.088989in}{2.075896in}}%
\pgfpathlineto{\pgfqpoint{2.093197in}{2.066938in}}%
\pgfpathlineto{\pgfqpoint{2.098180in}{2.056643in}}%
\pgfpathlineto{\pgfqpoint{2.099566in}{2.053921in}}%
\pgfpathlineto{\pgfqpoint{2.103761in}{2.046349in}}%
\pgfpathlineto{\pgfqpoint{2.109733in}{2.036054in}}%
\pgfpathlineto{\pgfqpoint{2.110143in}{2.035379in}}%
\pgfpathlineto{\pgfqpoint{2.116495in}{2.025760in}}%
\pgfpathlineto{\pgfqpoint{2.120720in}{2.019510in}}%
\pgfpathlineto{\pgfqpoint{2.123699in}{2.015466in}}%
\pgfpathlineto{\pgfqpoint{2.131268in}{2.005171in}}%
\pgfpathlineto{\pgfqpoint{2.131297in}{2.005130in}}%
\pgfpathlineto{\pgfqpoint{2.139190in}{1.994877in}}%
\pgfpathlineto{\pgfqpoint{2.141874in}{1.990996in}}%
\pgfpathlineto{\pgfqpoint{2.146679in}{1.984582in}}%
\pgfpathlineto{\pgfqpoint{2.152451in}{1.975333in}}%
\pgfpathlineto{\pgfqpoint{2.153150in}{1.974288in}}%
\pgfpathlineto{\pgfqpoint{2.158654in}{1.963993in}}%
\pgfpathlineto{\pgfqpoint{2.162912in}{1.953699in}}%
\pgfpathlineto{\pgfqpoint{2.163029in}{1.953332in}}%
\pgfpathlineto{\pgfqpoint{2.166373in}{1.943405in}}%
\pgfpathlineto{\pgfqpoint{2.168863in}{1.933110in}}%
\pgfpathlineto{\pgfqpoint{2.170433in}{1.922816in}}%
\pgfpathlineto{\pgfqpoint{2.171086in}{1.912521in}}%
\pgfpathlineto{\pgfqpoint{2.170788in}{1.902227in}}%
\pgfpathlineto{\pgfqpoint{2.169468in}{1.891932in}}%
\pgfpathclose%
\pgfusepath{fill}%
\end{pgfscope}%
\begin{pgfscope}%
\pgfpathrectangle{\pgfqpoint{1.856795in}{1.819814in}}{\pgfqpoint{1.194205in}{1.163386in}}%
\pgfusepath{clip}%
\pgfsetbuttcap%
\pgfsetroundjoin%
\definecolor{currentfill}{rgb}{0.945204,0.390623,0.270949}%
\pgfsetfillcolor{currentfill}%
\pgfsetlinewidth{0.000000pt}%
\definecolor{currentstroke}{rgb}{0.000000,0.000000,0.000000}%
\pgfsetstrokecolor{currentstroke}%
\pgfsetdash{}{0pt}%
\pgfpathmoveto{\pgfqpoint{2.660150in}{2.465866in}}%
\pgfpathlineto{\pgfqpoint{2.670727in}{2.463124in}}%
\pgfpathlineto{\pgfqpoint{2.681305in}{2.461552in}}%
\pgfpathlineto{\pgfqpoint{2.691882in}{2.461099in}}%
\pgfpathlineto{\pgfqpoint{2.702459in}{2.461730in}}%
\pgfpathlineto{\pgfqpoint{2.713036in}{2.463419in}}%
\pgfpathlineto{\pgfqpoint{2.723613in}{2.466144in}}%
\pgfpathlineto{\pgfqpoint{2.730093in}{2.468421in}}%
\pgfpathlineto{\pgfqpoint{2.734190in}{2.469816in}}%
\pgfpathlineto{\pgfqpoint{2.744767in}{2.474316in}}%
\pgfpathlineto{\pgfqpoint{2.753427in}{2.478715in}}%
\pgfpathlineto{\pgfqpoint{2.755344in}{2.479660in}}%
\pgfpathlineto{\pgfqpoint{2.765921in}{2.485590in}}%
\pgfpathlineto{\pgfqpoint{2.771407in}{2.489009in}}%
\pgfpathlineto{\pgfqpoint{2.776498in}{2.492081in}}%
\pgfpathlineto{\pgfqpoint{2.787075in}{2.498993in}}%
\pgfpathlineto{\pgfqpoint{2.787524in}{2.499304in}}%
\pgfpathlineto{\pgfqpoint{2.797652in}{2.506056in}}%
\pgfpathlineto{\pgfqpoint{2.802740in}{2.509598in}}%
\pgfpathlineto{\pgfqpoint{2.808229in}{2.513274in}}%
\pgfpathlineto{\pgfqpoint{2.817836in}{2.519893in}}%
\pgfpathlineto{\pgfqpoint{2.818806in}{2.520536in}}%
\pgfpathlineto{\pgfqpoint{2.829383in}{2.527659in}}%
\pgfpathlineto{\pgfqpoint{2.833089in}{2.530187in}}%
\pgfpathlineto{\pgfqpoint{2.839960in}{2.534724in}}%
\pgfpathlineto{\pgfqpoint{2.848511in}{2.540482in}}%
\pgfpathlineto{\pgfqpoint{2.850538in}{2.541826in}}%
\pgfpathlineto{\pgfqpoint{2.861115in}{2.549115in}}%
\pgfpathlineto{\pgfqpoint{2.863398in}{2.550776in}}%
\pgfpathlineto{\pgfqpoint{2.871692in}{2.557231in}}%
\pgfpathlineto{\pgfqpoint{2.876053in}{2.561071in}}%
\pgfpathlineto{\pgfqpoint{2.882269in}{2.567742in}}%
\pgfpathlineto{\pgfqpoint{2.885134in}{2.571365in}}%
\pgfpathlineto{\pgfqpoint{2.891046in}{2.581659in}}%
\pgfpathlineto{\pgfqpoint{2.892846in}{2.586129in}}%
\pgfpathlineto{\pgfqpoint{2.894980in}{2.591954in}}%
\pgfpathlineto{\pgfqpoint{2.897844in}{2.602248in}}%
\pgfpathlineto{\pgfqpoint{2.900189in}{2.612543in}}%
\pgfpathlineto{\pgfqpoint{2.902334in}{2.622837in}}%
\pgfpathlineto{\pgfqpoint{2.903423in}{2.628006in}}%
\pgfpathlineto{\pgfqpoint{2.904470in}{2.633132in}}%
\pgfpathlineto{\pgfqpoint{2.906740in}{2.643426in}}%
\pgfpathlineto{\pgfqpoint{2.909285in}{2.653720in}}%
\pgfpathlineto{\pgfqpoint{2.912211in}{2.664015in}}%
\pgfpathlineto{\pgfqpoint{2.914000in}{2.669435in}}%
\pgfpathlineto{\pgfqpoint{2.915567in}{2.674309in}}%
\pgfpathlineto{\pgfqpoint{2.919408in}{2.684604in}}%
\pgfpathlineto{\pgfqpoint{2.923847in}{2.694898in}}%
\pgfpathlineto{\pgfqpoint{2.924577in}{2.696372in}}%
\pgfpathlineto{\pgfqpoint{2.928843in}{2.705193in}}%
\pgfpathlineto{\pgfqpoint{2.934516in}{2.715487in}}%
\pgfpathlineto{\pgfqpoint{2.935154in}{2.716512in}}%
\pgfpathlineto{\pgfqpoint{2.940799in}{2.725781in}}%
\pgfpathlineto{\pgfqpoint{2.945731in}{2.733012in}}%
\pgfpathlineto{\pgfqpoint{2.947779in}{2.736076in}}%
\pgfpathlineto{\pgfqpoint{2.955410in}{2.746370in}}%
\pgfpathlineto{\pgfqpoint{2.956308in}{2.747473in}}%
\pgfpathlineto{\pgfqpoint{2.963636in}{2.756665in}}%
\pgfpathlineto{\pgfqpoint{2.966885in}{2.760404in}}%
\pgfpathlineto{\pgfqpoint{2.972439in}{2.766959in}}%
\pgfpathlineto{\pgfqpoint{2.977462in}{2.772456in}}%
\pgfpathlineto{\pgfqpoint{2.977462in}{2.777254in}}%
\pgfpathlineto{\pgfqpoint{2.977462in}{2.787548in}}%
\pgfpathlineto{\pgfqpoint{2.977462in}{2.791739in}}%
\pgfpathlineto{\pgfqpoint{2.973098in}{2.787548in}}%
\pgfpathlineto{\pgfqpoint{2.966885in}{2.781201in}}%
\pgfpathlineto{\pgfqpoint{2.962900in}{2.777254in}}%
\pgfpathlineto{\pgfqpoint{2.956308in}{2.770236in}}%
\pgfpathlineto{\pgfqpoint{2.953148in}{2.766959in}}%
\pgfpathlineto{\pgfqpoint{2.945731in}{2.758615in}}%
\pgfpathlineto{\pgfqpoint{2.943955in}{2.756665in}}%
\pgfpathlineto{\pgfqpoint{2.935393in}{2.746370in}}%
\pgfpathlineto{\pgfqpoint{2.935154in}{2.746053in}}%
\pgfpathlineto{\pgfqpoint{2.927445in}{2.736076in}}%
\pgfpathlineto{\pgfqpoint{2.924577in}{2.731944in}}%
\pgfpathlineto{\pgfqpoint{2.920193in}{2.725781in}}%
\pgfpathlineto{\pgfqpoint{2.914000in}{2.716003in}}%
\pgfpathlineto{\pgfqpoint{2.913664in}{2.715487in}}%
\pgfpathlineto{\pgfqpoint{2.907722in}{2.705193in}}%
\pgfpathlineto{\pgfqpoint{2.903423in}{2.696682in}}%
\pgfpathlineto{\pgfqpoint{2.902491in}{2.694898in}}%
\pgfpathlineto{\pgfqpoint{2.897805in}{2.684604in}}%
\pgfpathlineto{\pgfqpoint{2.893761in}{2.674309in}}%
\pgfpathlineto{\pgfqpoint{2.892846in}{2.671609in}}%
\pgfpathlineto{\pgfqpoint{2.890169in}{2.664015in}}%
\pgfpathlineto{\pgfqpoint{2.887041in}{2.653720in}}%
\pgfpathlineto{\pgfqpoint{2.884311in}{2.643426in}}%
\pgfpathlineto{\pgfqpoint{2.882269in}{2.634886in}}%
\pgfpathlineto{\pgfqpoint{2.881832in}{2.633132in}}%
\pgfpathlineto{\pgfqpoint{2.879391in}{2.622837in}}%
\pgfpathlineto{\pgfqpoint{2.876850in}{2.612543in}}%
\pgfpathlineto{\pgfqpoint{2.873844in}{2.602248in}}%
\pgfpathlineto{\pgfqpoint{2.871692in}{2.596720in}}%
\pgfpathlineto{\pgfqpoint{2.869683in}{2.591954in}}%
\pgfpathlineto{\pgfqpoint{2.863226in}{2.581659in}}%
\pgfpathlineto{\pgfqpoint{2.861115in}{2.579250in}}%
\pgfpathlineto{\pgfqpoint{2.852978in}{2.571365in}}%
\pgfpathlineto{\pgfqpoint{2.850538in}{2.569396in}}%
\pgfpathlineto{\pgfqpoint{2.839960in}{2.561800in}}%
\pgfpathlineto{\pgfqpoint{2.838850in}{2.561071in}}%
\pgfpathlineto{\pgfqpoint{2.829383in}{2.554921in}}%
\pgfpathlineto{\pgfqpoint{2.822679in}{2.550776in}}%
\pgfpathlineto{\pgfqpoint{2.818806in}{2.548318in}}%
\pgfpathlineto{\pgfqpoint{2.808229in}{2.541788in}}%
\pgfpathlineto{\pgfqpoint{2.806052in}{2.540482in}}%
\pgfpathlineto{\pgfqpoint{2.797652in}{2.535242in}}%
\pgfpathlineto{\pgfqpoint{2.789261in}{2.530187in}}%
\pgfpathlineto{\pgfqpoint{2.787075in}{2.528821in}}%
\pgfpathlineto{\pgfqpoint{2.776498in}{2.522514in}}%
\pgfpathlineto{\pgfqpoint{2.771815in}{2.519893in}}%
\pgfpathlineto{\pgfqpoint{2.765921in}{2.516491in}}%
\pgfpathlineto{\pgfqpoint{2.755344in}{2.510904in}}%
\pgfpathlineto{\pgfqpoint{2.752577in}{2.509598in}}%
\pgfpathlineto{\pgfqpoint{2.744767in}{2.505819in}}%
\pgfpathlineto{\pgfqpoint{2.734190in}{2.501442in}}%
\pgfpathlineto{\pgfqpoint{2.727840in}{2.499304in}}%
\pgfpathlineto{\pgfqpoint{2.723613in}{2.497846in}}%
\pgfpathlineto{\pgfqpoint{2.713036in}{2.495113in}}%
\pgfpathlineto{\pgfqpoint{2.702459in}{2.493379in}}%
\pgfpathlineto{\pgfqpoint{2.691882in}{2.492707in}}%
\pgfpathlineto{\pgfqpoint{2.681305in}{2.493175in}}%
\pgfpathlineto{\pgfqpoint{2.670727in}{2.494883in}}%
\pgfpathlineto{\pgfqpoint{2.660150in}{2.497957in}}%
\pgfpathlineto{\pgfqpoint{2.656973in}{2.499304in}}%
\pgfpathlineto{\pgfqpoint{2.649573in}{2.502567in}}%
\pgfpathlineto{\pgfqpoint{2.638996in}{2.508945in}}%
\pgfpathlineto{\pgfqpoint{2.638119in}{2.509598in}}%
\pgfpathlineto{\pgfqpoint{2.628419in}{2.517594in}}%
\pgfpathlineto{\pgfqpoint{2.626061in}{2.519893in}}%
\pgfpathlineto{\pgfqpoint{2.617842in}{2.529112in}}%
\pgfpathlineto{\pgfqpoint{2.616977in}{2.530187in}}%
\pgfpathlineto{\pgfqpoint{2.609846in}{2.540482in}}%
\pgfpathlineto{\pgfqpoint{2.607265in}{2.544936in}}%
\pgfpathlineto{\pgfqpoint{2.603971in}{2.550776in}}%
\pgfpathlineto{\pgfqpoint{2.598991in}{2.561071in}}%
\pgfpathlineto{\pgfqpoint{2.596688in}{2.566497in}}%
\pgfpathlineto{\pgfqpoint{2.594574in}{2.571365in}}%
\pgfpathlineto{\pgfqpoint{2.590434in}{2.581659in}}%
\pgfpathlineto{\pgfqpoint{2.586277in}{2.591954in}}%
\pgfpathlineto{\pgfqpoint{2.586111in}{2.592350in}}%
\pgfpathlineto{\pgfqpoint{2.581740in}{2.602248in}}%
\pgfpathlineto{\pgfqpoint{2.576511in}{2.612543in}}%
\pgfpathlineto{\pgfqpoint{2.575534in}{2.614206in}}%
\pgfpathlineto{\pgfqpoint{2.570098in}{2.622837in}}%
\pgfpathlineto{\pgfqpoint{2.564957in}{2.629599in}}%
\pgfpathlineto{\pgfqpoint{2.562048in}{2.633132in}}%
\pgfpathlineto{\pgfqpoint{2.554380in}{2.640930in}}%
\pgfpathlineto{\pgfqpoint{2.551700in}{2.643426in}}%
\pgfpathlineto{\pgfqpoint{2.543803in}{2.649679in}}%
\pgfpathlineto{\pgfqpoint{2.538201in}{2.653720in}}%
\pgfpathlineto{\pgfqpoint{2.533226in}{2.656824in}}%
\pgfpathlineto{\pgfqpoint{2.522649in}{2.662865in}}%
\pgfpathlineto{\pgfqpoint{2.520462in}{2.664015in}}%
\pgfpathlineto{\pgfqpoint{2.512072in}{2.667929in}}%
\pgfpathlineto{\pgfqpoint{2.501495in}{2.672497in}}%
\pgfpathlineto{\pgfqpoint{2.497016in}{2.674309in}}%
\pgfpathlineto{\pgfqpoint{2.490917in}{2.676550in}}%
\pgfpathlineto{\pgfqpoint{2.480340in}{2.680271in}}%
\pgfpathlineto{\pgfqpoint{2.469763in}{2.683908in}}%
\pgfpathlineto{\pgfqpoint{2.467720in}{2.684604in}}%
\pgfpathlineto{\pgfqpoint{2.459186in}{2.687339in}}%
\pgfpathlineto{\pgfqpoint{2.448609in}{2.690859in}}%
\pgfpathlineto{\pgfqpoint{2.438032in}{2.694643in}}%
\pgfpathlineto{\pgfqpoint{2.437360in}{2.694898in}}%
\pgfpathlineto{\pgfqpoint{2.427455in}{2.698638in}}%
\pgfpathlineto{\pgfqpoint{2.416878in}{2.703202in}}%
\pgfpathlineto{\pgfqpoint{2.412754in}{2.705193in}}%
\pgfpathlineto{\pgfqpoint{2.406301in}{2.708471in}}%
\pgfpathlineto{\pgfqpoint{2.395724in}{2.714633in}}%
\pgfpathlineto{\pgfqpoint{2.394392in}{2.715487in}}%
\pgfpathlineto{\pgfqpoint{2.385147in}{2.722006in}}%
\pgfpathlineto{\pgfqpoint{2.380265in}{2.725781in}}%
\pgfpathlineto{\pgfqpoint{2.374570in}{2.730740in}}%
\pgfpathlineto{\pgfqpoint{2.368759in}{2.736076in}}%
\pgfpathlineto{\pgfqpoint{2.363993in}{2.741041in}}%
\pgfpathlineto{\pgfqpoint{2.359018in}{2.746370in}}%
\pgfpathlineto{\pgfqpoint{2.353416in}{2.753184in}}%
\pgfpathlineto{\pgfqpoint{2.350599in}{2.756665in}}%
\pgfpathlineto{\pgfqpoint{2.343291in}{2.766959in}}%
\pgfpathlineto{\pgfqpoint{2.342839in}{2.767701in}}%
\pgfpathlineto{\pgfqpoint{2.337075in}{2.777254in}}%
\pgfpathlineto{\pgfqpoint{2.332262in}{2.786701in}}%
\pgfpathlineto{\pgfqpoint{2.331837in}{2.787548in}}%
\pgfpathlineto{\pgfqpoint{2.327695in}{2.797842in}}%
\pgfpathlineto{\pgfqpoint{2.324548in}{2.808137in}}%
\pgfpathlineto{\pgfqpoint{2.322425in}{2.818431in}}%
\pgfpathlineto{\pgfqpoint{2.321684in}{2.825566in}}%
\pgfpathlineto{\pgfqpoint{2.321365in}{2.828726in}}%
\pgfpathlineto{\pgfqpoint{2.321365in}{2.839020in}}%
\pgfpathlineto{\pgfqpoint{2.321684in}{2.842136in}}%
\pgfpathlineto{\pgfqpoint{2.322451in}{2.849315in}}%
\pgfpathlineto{\pgfqpoint{2.324666in}{2.859609in}}%
\pgfpathlineto{\pgfqpoint{2.328033in}{2.869903in}}%
\pgfpathlineto{\pgfqpoint{2.332262in}{2.879459in}}%
\pgfpathlineto{\pgfqpoint{2.332596in}{2.880198in}}%
\pgfpathlineto{\pgfqpoint{2.338457in}{2.890492in}}%
\pgfpathlineto{\pgfqpoint{2.342839in}{2.896831in}}%
\pgfpathlineto{\pgfqpoint{2.345581in}{2.900787in}}%
\pgfpathlineto{\pgfqpoint{2.353416in}{2.910409in}}%
\pgfpathlineto{\pgfqpoint{2.353954in}{2.911081in}}%
\pgfpathlineto{\pgfqpoint{2.353416in}{2.911081in}}%
\pgfpathlineto{\pgfqpoint{2.342839in}{2.911081in}}%
\pgfpathlineto{\pgfqpoint{2.332262in}{2.911081in}}%
\pgfpathlineto{\pgfqpoint{2.321684in}{2.911081in}}%
\pgfpathlineto{\pgfqpoint{2.311107in}{2.911081in}}%
\pgfpathlineto{\pgfqpoint{2.306718in}{2.911081in}}%
\pgfpathlineto{\pgfqpoint{2.300530in}{2.902517in}}%
\pgfpathlineto{\pgfqpoint{2.299349in}{2.900787in}}%
\pgfpathlineto{\pgfqpoint{2.293513in}{2.890492in}}%
\pgfpathlineto{\pgfqpoint{2.289953in}{2.882579in}}%
\pgfpathlineto{\pgfqpoint{2.288948in}{2.880198in}}%
\pgfpathlineto{\pgfqpoint{2.285608in}{2.869903in}}%
\pgfpathlineto{\pgfqpoint{2.283271in}{2.859609in}}%
\pgfpathlineto{\pgfqpoint{2.281874in}{2.849315in}}%
\pgfpathlineto{\pgfqpoint{2.281378in}{2.839020in}}%
\pgfpathlineto{\pgfqpoint{2.281760in}{2.828726in}}%
\pgfpathlineto{\pgfqpoint{2.283012in}{2.818431in}}%
\pgfpathlineto{\pgfqpoint{2.285135in}{2.808137in}}%
\pgfpathlineto{\pgfqpoint{2.288131in}{2.797842in}}%
\pgfpathlineto{\pgfqpoint{2.289953in}{2.793008in}}%
\pgfpathlineto{\pgfqpoint{2.292089in}{2.787548in}}%
\pgfpathlineto{\pgfqpoint{2.296999in}{2.777254in}}%
\pgfpathlineto{\pgfqpoint{2.300530in}{2.770911in}}%
\pgfpathlineto{\pgfqpoint{2.302807in}{2.766959in}}%
\pgfpathlineto{\pgfqpoint{2.309506in}{2.756665in}}%
\pgfpathlineto{\pgfqpoint{2.311107in}{2.754420in}}%
\pgfpathlineto{\pgfqpoint{2.317073in}{2.746370in}}%
\pgfpathlineto{\pgfqpoint{2.321684in}{2.740584in}}%
\pgfpathlineto{\pgfqpoint{2.325435in}{2.736076in}}%
\pgfpathlineto{\pgfqpoint{2.332262in}{2.728357in}}%
\pgfpathlineto{\pgfqpoint{2.334663in}{2.725781in}}%
\pgfpathlineto{\pgfqpoint{2.342839in}{2.717474in}}%
\pgfpathlineto{\pgfqpoint{2.344935in}{2.715487in}}%
\pgfpathlineto{\pgfqpoint{2.353416in}{2.707837in}}%
\pgfpathlineto{\pgfqpoint{2.356638in}{2.705193in}}%
\pgfpathlineto{\pgfqpoint{2.363993in}{2.699411in}}%
\pgfpathlineto{\pgfqpoint{2.370495in}{2.694898in}}%
\pgfpathlineto{\pgfqpoint{2.374570in}{2.692162in}}%
\pgfpathlineto{\pgfqpoint{2.385147in}{2.685981in}}%
\pgfpathlineto{\pgfqpoint{2.387840in}{2.684604in}}%
\pgfpathlineto{\pgfqpoint{2.395724in}{2.680629in}}%
\pgfpathlineto{\pgfqpoint{2.406301in}{2.676036in}}%
\pgfpathlineto{\pgfqpoint{2.410768in}{2.674309in}}%
\pgfpathlineto{\pgfqpoint{2.416878in}{2.671911in}}%
\pgfpathlineto{\pgfqpoint{2.427455in}{2.668110in}}%
\pgfpathlineto{\pgfqpoint{2.438032in}{2.664579in}}%
\pgfpathlineto{\pgfqpoint{2.439745in}{2.664015in}}%
\pgfpathlineto{\pgfqpoint{2.448609in}{2.660936in}}%
\pgfpathlineto{\pgfqpoint{2.459186in}{2.657225in}}%
\pgfpathlineto{\pgfqpoint{2.468796in}{2.653720in}}%
\pgfpathlineto{\pgfqpoint{2.469763in}{2.653340in}}%
\pgfpathlineto{\pgfqpoint{2.480340in}{2.648847in}}%
\pgfpathlineto{\pgfqpoint{2.490917in}{2.643933in}}%
\pgfpathlineto{\pgfqpoint{2.491911in}{2.643426in}}%
\pgfpathlineto{\pgfqpoint{2.501495in}{2.637991in}}%
\pgfpathlineto{\pgfqpoint{2.509132in}{2.633132in}}%
\pgfpathlineto{\pgfqpoint{2.512072in}{2.631037in}}%
\pgfpathlineto{\pgfqpoint{2.522286in}{2.622837in}}%
\pgfpathlineto{\pgfqpoint{2.522649in}{2.622510in}}%
\pgfpathlineto{\pgfqpoint{2.532482in}{2.612543in}}%
\pgfpathlineto{\pgfqpoint{2.533226in}{2.611695in}}%
\pgfpathlineto{\pgfqpoint{2.540683in}{2.602248in}}%
\pgfpathlineto{\pgfqpoint{2.543803in}{2.597856in}}%
\pgfpathlineto{\pgfqpoint{2.547615in}{2.591954in}}%
\pgfpathlineto{\pgfqpoint{2.553759in}{2.581659in}}%
\pgfpathlineto{\pgfqpoint{2.554380in}{2.580591in}}%
\pgfpathlineto{\pgfqpoint{2.559350in}{2.571365in}}%
\pgfpathlineto{\pgfqpoint{2.564872in}{2.561071in}}%
\pgfpathlineto{\pgfqpoint{2.564957in}{2.560919in}}%
\pgfpathlineto{\pgfqpoint{2.570386in}{2.550776in}}%
\pgfpathlineto{\pgfqpoint{2.575534in}{2.541806in}}%
\pgfpathlineto{\pgfqpoint{2.576276in}{2.540482in}}%
\pgfpathlineto{\pgfqpoint{2.582635in}{2.530187in}}%
\pgfpathlineto{\pgfqpoint{2.586111in}{2.525111in}}%
\pgfpathlineto{\pgfqpoint{2.589750in}{2.519893in}}%
\pgfpathlineto{\pgfqpoint{2.596688in}{2.510985in}}%
\pgfpathlineto{\pgfqpoint{2.597829in}{2.509598in}}%
\pgfpathlineto{\pgfqpoint{2.607209in}{2.499304in}}%
\pgfpathlineto{\pgfqpoint{2.607265in}{2.499247in}}%
\pgfpathlineto{\pgfqpoint{2.617842in}{2.489537in}}%
\pgfpathlineto{\pgfqpoint{2.618503in}{2.489009in}}%
\pgfpathlineto{\pgfqpoint{2.628419in}{2.481516in}}%
\pgfpathlineto{\pgfqpoint{2.632879in}{2.478715in}}%
\pgfpathlineto{\pgfqpoint{2.638996in}{2.474972in}}%
\pgfpathlineto{\pgfqpoint{2.649573in}{2.469803in}}%
\pgfpathlineto{\pgfqpoint{2.653312in}{2.468421in}}%
\pgfpathclose%
\pgfusepath{fill}%
\end{pgfscope}%
\begin{pgfscope}%
\pgfpathrectangle{\pgfqpoint{1.856795in}{1.819814in}}{\pgfqpoint{1.194205in}{1.163386in}}%
\pgfusepath{clip}%
\pgfsetbuttcap%
\pgfsetroundjoin%
\definecolor{currentfill}{rgb}{0.954476,0.470822,0.323110}%
\pgfsetfillcolor{currentfill}%
\pgfsetlinewidth{0.000000pt}%
\definecolor{currentstroke}{rgb}{0.000000,0.000000,0.000000}%
\pgfsetstrokecolor{currentstroke}%
\pgfsetdash{}{0pt}%
\pgfpathmoveto{\pgfqpoint{2.205337in}{1.891932in}}%
\pgfpathlineto{\pgfqpoint{2.215914in}{1.891932in}}%
\pgfpathlineto{\pgfqpoint{2.226491in}{1.891932in}}%
\pgfpathlineto{\pgfqpoint{2.236361in}{1.891932in}}%
\pgfpathlineto{\pgfqpoint{2.237023in}{1.902227in}}%
\pgfpathlineto{\pgfqpoint{2.236783in}{1.912521in}}%
\pgfpathlineto{\pgfqpoint{2.235726in}{1.922816in}}%
\pgfpathlineto{\pgfqpoint{2.233903in}{1.933110in}}%
\pgfpathlineto{\pgfqpoint{2.231335in}{1.943405in}}%
\pgfpathlineto{\pgfqpoint{2.228004in}{1.953699in}}%
\pgfpathlineto{\pgfqpoint{2.226491in}{1.957483in}}%
\pgfpathlineto{\pgfqpoint{2.223889in}{1.963993in}}%
\pgfpathlineto{\pgfqpoint{2.218665in}{1.974288in}}%
\pgfpathlineto{\pgfqpoint{2.215914in}{1.978311in}}%
\pgfpathlineto{\pgfqpoint{2.211536in}{1.984582in}}%
\pgfpathlineto{\pgfqpoint{2.205337in}{1.990940in}}%
\pgfpathlineto{\pgfqpoint{2.201550in}{1.994877in}}%
\pgfpathlineto{\pgfqpoint{2.194760in}{2.000809in}}%
\pgfpathlineto{\pgfqpoint{2.189982in}{2.005171in}}%
\pgfpathlineto{\pgfqpoint{2.184183in}{2.010193in}}%
\pgfpathlineto{\pgfqpoint{2.178440in}{2.015466in}}%
\pgfpathlineto{\pgfqpoint{2.173606in}{2.019894in}}%
\pgfpathlineto{\pgfqpoint{2.167641in}{2.025760in}}%
\pgfpathlineto{\pgfqpoint{2.163029in}{2.030388in}}%
\pgfpathlineto{\pgfqpoint{2.157827in}{2.036054in}}%
\pgfpathlineto{\pgfqpoint{2.152451in}{2.042098in}}%
\pgfpathlineto{\pgfqpoint{2.148998in}{2.046349in}}%
\pgfpathlineto{\pgfqpoint{2.141874in}{2.055441in}}%
\pgfpathlineto{\pgfqpoint{2.141019in}{2.056643in}}%
\pgfpathlineto{\pgfqpoint{2.133989in}{2.066938in}}%
\pgfpathlineto{\pgfqpoint{2.131297in}{2.070991in}}%
\pgfpathlineto{\pgfqpoint{2.127546in}{2.077232in}}%
\pgfpathlineto{\pgfqpoint{2.121415in}{2.087527in}}%
\pgfpathlineto{\pgfqpoint{2.120720in}{2.088690in}}%
\pgfpathlineto{\pgfqpoint{2.115756in}{2.097821in}}%
\pgfpathlineto{\pgfqpoint{2.110143in}{2.107685in}}%
\pgfpathlineto{\pgfqpoint{2.109917in}{2.108116in}}%
\pgfpathlineto{\pgfqpoint{2.104297in}{2.118410in}}%
\pgfpathlineto{\pgfqpoint{2.099566in}{2.126548in}}%
\pgfpathlineto{\pgfqpoint{2.098388in}{2.128704in}}%
\pgfpathlineto{\pgfqpoint{2.092606in}{2.138999in}}%
\pgfpathlineto{\pgfqpoint{2.088989in}{2.145261in}}%
\pgfpathlineto{\pgfqpoint{2.086798in}{2.149293in}}%
\pgfpathlineto{\pgfqpoint{2.081202in}{2.159588in}}%
\pgfpathlineto{\pgfqpoint{2.078412in}{2.164737in}}%
\pgfpathlineto{\pgfqpoint{2.075810in}{2.169882in}}%
\pgfpathlineto{\pgfqpoint{2.070697in}{2.180177in}}%
\pgfpathlineto{\pgfqpoint{2.067835in}{2.186048in}}%
\pgfpathlineto{\pgfqpoint{2.065836in}{2.190471in}}%
\pgfpathlineto{\pgfqpoint{2.061320in}{2.200765in}}%
\pgfpathlineto{\pgfqpoint{2.057258in}{2.210228in}}%
\pgfpathlineto{\pgfqpoint{2.056928in}{2.211060in}}%
\pgfpathlineto{\pgfqpoint{2.052988in}{2.221354in}}%
\pgfpathlineto{\pgfqpoint{2.049133in}{2.231649in}}%
\pgfpathlineto{\pgfqpoint{2.046681in}{2.238332in}}%
\pgfpathlineto{\pgfqpoint{2.045460in}{2.241943in}}%
\pgfpathlineto{\pgfqpoint{2.042056in}{2.252238in}}%
\pgfpathlineto{\pgfqpoint{2.038678in}{2.262532in}}%
\pgfpathlineto{\pgfqpoint{2.036104in}{2.270386in}}%
\pgfpathlineto{\pgfqpoint{2.035366in}{2.272826in}}%
\pgfpathlineto{\pgfqpoint{2.032263in}{2.283121in}}%
\pgfpathlineto{\pgfqpoint{2.029113in}{2.293415in}}%
\pgfpathlineto{\pgfqpoint{2.025887in}{2.303710in}}%
\pgfpathlineto{\pgfqpoint{2.025527in}{2.304840in}}%
\pgfpathlineto{\pgfqpoint{2.022817in}{2.314004in}}%
\pgfpathlineto{\pgfqpoint{2.019667in}{2.324299in}}%
\pgfpathlineto{\pgfqpoint{2.016380in}{2.334593in}}%
\pgfpathlineto{\pgfqpoint{2.014950in}{2.338920in}}%
\pgfpathlineto{\pgfqpoint{2.013094in}{2.344887in}}%
\pgfpathlineto{\pgfqpoint{2.009758in}{2.355182in}}%
\pgfpathlineto{\pgfqpoint{2.006241in}{2.365476in}}%
\pgfpathlineto{\pgfqpoint{2.004373in}{2.370732in}}%
\pgfpathlineto{\pgfqpoint{2.002654in}{2.375771in}}%
\pgfpathlineto{\pgfqpoint{1.999000in}{2.386065in}}%
\pgfpathlineto{\pgfqpoint{1.995153in}{2.396360in}}%
\pgfpathlineto{\pgfqpoint{1.993796in}{2.399896in}}%
\pgfpathlineto{\pgfqpoint{1.991259in}{2.406654in}}%
\pgfpathlineto{\pgfqpoint{1.987274in}{2.416948in}}%
\pgfpathlineto{\pgfqpoint{1.983218in}{2.427080in}}%
\pgfpathlineto{\pgfqpoint{1.983154in}{2.427243in}}%
\pgfpathlineto{\pgfqpoint{1.979115in}{2.437537in}}%
\pgfpathlineto{\pgfqpoint{1.975023in}{2.447832in}}%
\pgfpathlineto{\pgfqpoint{1.972641in}{2.453861in}}%
\pgfpathlineto{\pgfqpoint{1.970991in}{2.458126in}}%
\pgfpathlineto{\pgfqpoint{1.967100in}{2.468421in}}%
\pgfpathlineto{\pgfqpoint{1.963289in}{2.478715in}}%
\pgfpathlineto{\pgfqpoint{1.962064in}{2.482152in}}%
\pgfpathlineto{\pgfqpoint{1.959697in}{2.489009in}}%
\pgfpathlineto{\pgfqpoint{1.956304in}{2.499304in}}%
\pgfpathlineto{\pgfqpoint{1.953075in}{2.509598in}}%
\pgfpathlineto{\pgfqpoint{1.951487in}{2.514958in}}%
\pgfpathlineto{\pgfqpoint{1.950082in}{2.519893in}}%
\pgfpathlineto{\pgfqpoint{1.947338in}{2.530187in}}%
\pgfpathlineto{\pgfqpoint{1.944771in}{2.540482in}}%
\pgfpathlineto{\pgfqpoint{1.942373in}{2.550776in}}%
\pgfpathlineto{\pgfqpoint{1.940910in}{2.557477in}}%
\pgfpathlineto{\pgfqpoint{1.940159in}{2.561071in}}%
\pgfpathlineto{\pgfqpoint{1.938139in}{2.571365in}}%
\pgfpathlineto{\pgfqpoint{1.936232in}{2.581659in}}%
\pgfpathlineto{\pgfqpoint{1.934414in}{2.591954in}}%
\pgfpathlineto{\pgfqpoint{1.932659in}{2.602248in}}%
\pgfpathlineto{\pgfqpoint{1.930947in}{2.612543in}}%
\pgfpathlineto{\pgfqpoint{1.930333in}{2.614761in}}%
\pgfpathlineto{\pgfqpoint{1.930333in}{2.612543in}}%
\pgfpathlineto{\pgfqpoint{1.930333in}{2.602248in}}%
\pgfpathlineto{\pgfqpoint{1.930333in}{2.591954in}}%
\pgfpathlineto{\pgfqpoint{1.930333in}{2.581659in}}%
\pgfpathlineto{\pgfqpoint{1.930333in}{2.571365in}}%
\pgfpathlineto{\pgfqpoint{1.930333in}{2.561071in}}%
\pgfpathlineto{\pgfqpoint{1.930333in}{2.550776in}}%
\pgfpathlineto{\pgfqpoint{1.930333in}{2.540482in}}%
\pgfpathlineto{\pgfqpoint{1.930333in}{2.540455in}}%
\pgfpathlineto{\pgfqpoint{1.932775in}{2.530187in}}%
\pgfpathlineto{\pgfqpoint{1.935420in}{2.519893in}}%
\pgfpathlineto{\pgfqpoint{1.938271in}{2.509598in}}%
\pgfpathlineto{\pgfqpoint{1.940910in}{2.500725in}}%
\pgfpathlineto{\pgfqpoint{1.941345in}{2.499304in}}%
\pgfpathlineto{\pgfqpoint{1.944706in}{2.489009in}}%
\pgfpathlineto{\pgfqpoint{1.948257in}{2.478715in}}%
\pgfpathlineto{\pgfqpoint{1.951487in}{2.469794in}}%
\pgfpathlineto{\pgfqpoint{1.951995in}{2.468421in}}%
\pgfpathlineto{\pgfqpoint{1.955973in}{2.458126in}}%
\pgfpathlineto{\pgfqpoint{1.960050in}{2.447832in}}%
\pgfpathlineto{\pgfqpoint{1.962064in}{2.442869in}}%
\pgfpathlineto{\pgfqpoint{1.964251in}{2.437537in}}%
\pgfpathlineto{\pgfqpoint{1.968514in}{2.427243in}}%
\pgfpathlineto{\pgfqpoint{1.972641in}{2.417172in}}%
\pgfpathlineto{\pgfqpoint{1.972734in}{2.416948in}}%
\pgfpathlineto{\pgfqpoint{1.976984in}{2.406654in}}%
\pgfpathlineto{\pgfqpoint{1.981084in}{2.396360in}}%
\pgfpathlineto{\pgfqpoint{1.983218in}{2.390837in}}%
\pgfpathlineto{\pgfqpoint{1.985078in}{2.386065in}}%
\pgfpathlineto{\pgfqpoint{1.988951in}{2.375771in}}%
\pgfpathlineto{\pgfqpoint{1.992605in}{2.365476in}}%
\pgfpathlineto{\pgfqpoint{1.993796in}{2.361975in}}%
\pgfpathlineto{\pgfqpoint{1.996176in}{2.355182in}}%
\pgfpathlineto{\pgfqpoint{1.999607in}{2.344887in}}%
\pgfpathlineto{\pgfqpoint{2.002844in}{2.334593in}}%
\pgfpathlineto{\pgfqpoint{2.004373in}{2.329511in}}%
\pgfpathlineto{\pgfqpoint{2.006025in}{2.324299in}}%
\pgfpathlineto{\pgfqpoint{2.009163in}{2.314004in}}%
\pgfpathlineto{\pgfqpoint{2.012169in}{2.303710in}}%
\pgfpathlineto{\pgfqpoint{2.014950in}{2.293857in}}%
\pgfpathlineto{\pgfqpoint{2.015083in}{2.293415in}}%
\pgfpathlineto{\pgfqpoint{2.018141in}{2.283121in}}%
\pgfpathlineto{\pgfqpoint{2.021139in}{2.272826in}}%
\pgfpathlineto{\pgfqpoint{2.024112in}{2.262532in}}%
\pgfpathlineto{\pgfqpoint{2.025527in}{2.257656in}}%
\pgfpathlineto{\pgfqpoint{2.027217in}{2.252238in}}%
\pgfpathlineto{\pgfqpoint{2.030460in}{2.241943in}}%
\pgfpathlineto{\pgfqpoint{2.033756in}{2.231649in}}%
\pgfpathlineto{\pgfqpoint{2.036104in}{2.224506in}}%
\pgfpathlineto{\pgfqpoint{2.037218in}{2.221354in}}%
\pgfpathlineto{\pgfqpoint{2.040965in}{2.211060in}}%
\pgfpathlineto{\pgfqpoint{2.044832in}{2.200765in}}%
\pgfpathlineto{\pgfqpoint{2.046681in}{2.196031in}}%
\pgfpathlineto{\pgfqpoint{2.049004in}{2.190471in}}%
\pgfpathlineto{\pgfqpoint{2.053453in}{2.180177in}}%
\pgfpathlineto{\pgfqpoint{2.057258in}{2.171662in}}%
\pgfpathlineto{\pgfqpoint{2.058105in}{2.169882in}}%
\pgfpathlineto{\pgfqpoint{2.063160in}{2.159588in}}%
\pgfpathlineto{\pgfqpoint{2.067835in}{2.150256in}}%
\pgfpathlineto{\pgfqpoint{2.068344in}{2.149293in}}%
\pgfpathlineto{\pgfqpoint{2.073889in}{2.138999in}}%
\pgfpathlineto{\pgfqpoint{2.078412in}{2.130558in}}%
\pgfpathlineto{\pgfqpoint{2.079448in}{2.128704in}}%
\pgfpathlineto{\pgfqpoint{2.085125in}{2.118410in}}%
\pgfpathlineto{\pgfqpoint{2.088989in}{2.111016in}}%
\pgfpathlineto{\pgfqpoint{2.090575in}{2.108116in}}%
\pgfpathlineto{\pgfqpoint{2.095890in}{2.097821in}}%
\pgfpathlineto{\pgfqpoint{2.099566in}{2.090170in}}%
\pgfpathlineto{\pgfqpoint{2.100941in}{2.087527in}}%
\pgfpathlineto{\pgfqpoint{2.106160in}{2.077232in}}%
\pgfpathlineto{\pgfqpoint{2.110143in}{2.069379in}}%
\pgfpathlineto{\pgfqpoint{2.111504in}{2.066938in}}%
\pgfpathlineto{\pgfqpoint{2.117417in}{2.056643in}}%
\pgfpathlineto{\pgfqpoint{2.120720in}{2.051120in}}%
\pgfpathlineto{\pgfqpoint{2.123852in}{2.046349in}}%
\pgfpathlineto{\pgfqpoint{2.130887in}{2.036054in}}%
\pgfpathlineto{\pgfqpoint{2.131297in}{2.035478in}}%
\pgfpathlineto{\pgfqpoint{2.138864in}{2.025760in}}%
\pgfpathlineto{\pgfqpoint{2.141874in}{2.021977in}}%
\pgfpathlineto{\pgfqpoint{2.147527in}{2.015466in}}%
\pgfpathlineto{\pgfqpoint{2.152451in}{2.009778in}}%
\pgfpathlineto{\pgfqpoint{2.156785in}{2.005171in}}%
\pgfpathlineto{\pgfqpoint{2.163029in}{1.998209in}}%
\pgfpathlineto{\pgfqpoint{2.166249in}{1.994877in}}%
\pgfpathlineto{\pgfqpoint{2.173606in}{1.986211in}}%
\pgfpathlineto{\pgfqpoint{2.175071in}{1.984582in}}%
\pgfpathlineto{\pgfqpoint{2.182450in}{1.974288in}}%
\pgfpathlineto{\pgfqpoint{2.184183in}{1.971137in}}%
\pgfpathlineto{\pgfqpoint{2.188239in}{1.963993in}}%
\pgfpathlineto{\pgfqpoint{2.192720in}{1.953699in}}%
\pgfpathlineto{\pgfqpoint{2.194760in}{1.947597in}}%
\pgfpathlineto{\pgfqpoint{2.196213in}{1.943405in}}%
\pgfpathlineto{\pgfqpoint{2.198863in}{1.933110in}}%
\pgfpathlineto{\pgfqpoint{2.200617in}{1.922816in}}%
\pgfpathlineto{\pgfqpoint{2.201476in}{1.912521in}}%
\pgfpathlineto{\pgfqpoint{2.201401in}{1.902227in}}%
\pgfpathlineto{\pgfqpoint{2.200316in}{1.891932in}}%
\pgfpathclose%
\pgfusepath{fill}%
\end{pgfscope}%
\begin{pgfscope}%
\pgfpathrectangle{\pgfqpoint{1.856795in}{1.819814in}}{\pgfqpoint{1.194205in}{1.163386in}}%
\pgfusepath{clip}%
\pgfsetbuttcap%
\pgfsetroundjoin%
\definecolor{currentfill}{rgb}{0.954476,0.470822,0.323110}%
\pgfsetfillcolor{currentfill}%
\pgfsetlinewidth{0.000000pt}%
\definecolor{currentstroke}{rgb}{0.000000,0.000000,0.000000}%
\pgfsetstrokecolor{currentstroke}%
\pgfsetdash{}{0pt}%
\pgfpathmoveto{\pgfqpoint{2.670727in}{2.427237in}}%
\pgfpathlineto{\pgfqpoint{2.681305in}{2.425283in}}%
\pgfpathlineto{\pgfqpoint{2.691882in}{2.424492in}}%
\pgfpathlineto{\pgfqpoint{2.702459in}{2.424871in}}%
\pgfpathlineto{\pgfqpoint{2.713036in}{2.426433in}}%
\pgfpathlineto{\pgfqpoint{2.716179in}{2.427243in}}%
\pgfpathlineto{\pgfqpoint{2.723613in}{2.429073in}}%
\pgfpathlineto{\pgfqpoint{2.734190in}{2.432791in}}%
\pgfpathlineto{\pgfqpoint{2.744547in}{2.437537in}}%
\pgfpathlineto{\pgfqpoint{2.744767in}{2.437634in}}%
\pgfpathlineto{\pgfqpoint{2.755344in}{2.443239in}}%
\pgfpathlineto{\pgfqpoint{2.762732in}{2.447832in}}%
\pgfpathlineto{\pgfqpoint{2.765921in}{2.449739in}}%
\pgfpathlineto{\pgfqpoint{2.776498in}{2.456868in}}%
\pgfpathlineto{\pgfqpoint{2.778199in}{2.458126in}}%
\pgfpathlineto{\pgfqpoint{2.787075in}{2.464438in}}%
\pgfpathlineto{\pgfqpoint{2.792278in}{2.468421in}}%
\pgfpathlineto{\pgfqpoint{2.797652in}{2.472374in}}%
\pgfpathlineto{\pgfqpoint{2.805827in}{2.478715in}}%
\pgfpathlineto{\pgfqpoint{2.808229in}{2.480504in}}%
\pgfpathlineto{\pgfqpoint{2.818806in}{2.488659in}}%
\pgfpathlineto{\pgfqpoint{2.819252in}{2.489009in}}%
\pgfpathlineto{\pgfqpoint{2.829383in}{2.496634in}}%
\pgfpathlineto{\pgfqpoint{2.832901in}{2.499304in}}%
\pgfpathlineto{\pgfqpoint{2.839960in}{2.504442in}}%
\pgfpathlineto{\pgfqpoint{2.847047in}{2.509598in}}%
\pgfpathlineto{\pgfqpoint{2.850538in}{2.512044in}}%
\pgfpathlineto{\pgfqpoint{2.861115in}{2.519432in}}%
\pgfpathlineto{\pgfqpoint{2.861782in}{2.519893in}}%
\pgfpathlineto{\pgfqpoint{2.871692in}{2.526625in}}%
\pgfpathlineto{\pgfqpoint{2.876853in}{2.530187in}}%
\pgfpathlineto{\pgfqpoint{2.882269in}{2.534009in}}%
\pgfpathlineto{\pgfqpoint{2.890844in}{2.540482in}}%
\pgfpathlineto{\pgfqpoint{2.892846in}{2.542160in}}%
\pgfpathlineto{\pgfqpoint{2.901755in}{2.550776in}}%
\pgfpathlineto{\pgfqpoint{2.903423in}{2.552848in}}%
\pgfpathlineto{\pgfqpoint{2.908948in}{2.561071in}}%
\pgfpathlineto{\pgfqpoint{2.913608in}{2.571365in}}%
\pgfpathlineto{\pgfqpoint{2.914000in}{2.572556in}}%
\pgfpathlineto{\pgfqpoint{2.916709in}{2.581659in}}%
\pgfpathlineto{\pgfqpoint{2.919013in}{2.591954in}}%
\pgfpathlineto{\pgfqpoint{2.920896in}{2.602248in}}%
\pgfpathlineto{\pgfqpoint{2.922600in}{2.612543in}}%
\pgfpathlineto{\pgfqpoint{2.924292in}{2.622837in}}%
\pgfpathlineto{\pgfqpoint{2.924577in}{2.624439in}}%
\pgfpathlineto{\pgfqpoint{2.926088in}{2.633132in}}%
\pgfpathlineto{\pgfqpoint{2.928101in}{2.643426in}}%
\pgfpathlineto{\pgfqpoint{2.930422in}{2.653720in}}%
\pgfpathlineto{\pgfqpoint{2.933134in}{2.664015in}}%
\pgfpathlineto{\pgfqpoint{2.935154in}{2.670547in}}%
\pgfpathlineto{\pgfqpoint{2.936297in}{2.674309in}}%
\pgfpathlineto{\pgfqpoint{2.939952in}{2.684604in}}%
\pgfpathlineto{\pgfqpoint{2.944183in}{2.694898in}}%
\pgfpathlineto{\pgfqpoint{2.945731in}{2.698168in}}%
\pgfpathlineto{\pgfqpoint{2.948998in}{2.705193in}}%
\pgfpathlineto{\pgfqpoint{2.954446in}{2.715487in}}%
\pgfpathlineto{\pgfqpoint{2.956308in}{2.718603in}}%
\pgfpathlineto{\pgfqpoint{2.960522in}{2.725781in}}%
\pgfpathlineto{\pgfqpoint{2.966885in}{2.735497in}}%
\pgfpathlineto{\pgfqpoint{2.967258in}{2.736076in}}%
\pgfpathlineto{\pgfqpoint{2.974577in}{2.746370in}}%
\pgfpathlineto{\pgfqpoint{2.977462in}{2.750075in}}%
\pgfpathlineto{\pgfqpoint{2.977462in}{2.756665in}}%
\pgfpathlineto{\pgfqpoint{2.977462in}{2.766959in}}%
\pgfpathlineto{\pgfqpoint{2.977462in}{2.772456in}}%
\pgfpathlineto{\pgfqpoint{2.972439in}{2.766959in}}%
\pgfpathlineto{\pgfqpoint{2.966885in}{2.760404in}}%
\pgfpathlineto{\pgfqpoint{2.963636in}{2.756665in}}%
\pgfpathlineto{\pgfqpoint{2.956308in}{2.747473in}}%
\pgfpathlineto{\pgfqpoint{2.955410in}{2.746370in}}%
\pgfpathlineto{\pgfqpoint{2.947779in}{2.736076in}}%
\pgfpathlineto{\pgfqpoint{2.945731in}{2.733012in}}%
\pgfpathlineto{\pgfqpoint{2.940799in}{2.725781in}}%
\pgfpathlineto{\pgfqpoint{2.935154in}{2.716512in}}%
\pgfpathlineto{\pgfqpoint{2.934516in}{2.715487in}}%
\pgfpathlineto{\pgfqpoint{2.928843in}{2.705193in}}%
\pgfpathlineto{\pgfqpoint{2.924577in}{2.696372in}}%
\pgfpathlineto{\pgfqpoint{2.923847in}{2.694898in}}%
\pgfpathlineto{\pgfqpoint{2.919408in}{2.684604in}}%
\pgfpathlineto{\pgfqpoint{2.915567in}{2.674309in}}%
\pgfpathlineto{\pgfqpoint{2.914000in}{2.669435in}}%
\pgfpathlineto{\pgfqpoint{2.912211in}{2.664015in}}%
\pgfpathlineto{\pgfqpoint{2.909285in}{2.653720in}}%
\pgfpathlineto{\pgfqpoint{2.906740in}{2.643426in}}%
\pgfpathlineto{\pgfqpoint{2.904470in}{2.633132in}}%
\pgfpathlineto{\pgfqpoint{2.903423in}{2.628006in}}%
\pgfpathlineto{\pgfqpoint{2.902334in}{2.622837in}}%
\pgfpathlineto{\pgfqpoint{2.900189in}{2.612543in}}%
\pgfpathlineto{\pgfqpoint{2.897844in}{2.602248in}}%
\pgfpathlineto{\pgfqpoint{2.894980in}{2.591954in}}%
\pgfpathlineto{\pgfqpoint{2.892846in}{2.586129in}}%
\pgfpathlineto{\pgfqpoint{2.891046in}{2.581659in}}%
\pgfpathlineto{\pgfqpoint{2.885134in}{2.571365in}}%
\pgfpathlineto{\pgfqpoint{2.882269in}{2.567742in}}%
\pgfpathlineto{\pgfqpoint{2.876053in}{2.561071in}}%
\pgfpathlineto{\pgfqpoint{2.871692in}{2.557231in}}%
\pgfpathlineto{\pgfqpoint{2.863398in}{2.550776in}}%
\pgfpathlineto{\pgfqpoint{2.861115in}{2.549115in}}%
\pgfpathlineto{\pgfqpoint{2.850538in}{2.541826in}}%
\pgfpathlineto{\pgfqpoint{2.848511in}{2.540482in}}%
\pgfpathlineto{\pgfqpoint{2.839960in}{2.534724in}}%
\pgfpathlineto{\pgfqpoint{2.833089in}{2.530187in}}%
\pgfpathlineto{\pgfqpoint{2.829383in}{2.527659in}}%
\pgfpathlineto{\pgfqpoint{2.818806in}{2.520536in}}%
\pgfpathlineto{\pgfqpoint{2.817836in}{2.519893in}}%
\pgfpathlineto{\pgfqpoint{2.808229in}{2.513274in}}%
\pgfpathlineto{\pgfqpoint{2.802740in}{2.509598in}}%
\pgfpathlineto{\pgfqpoint{2.797652in}{2.506056in}}%
\pgfpathlineto{\pgfqpoint{2.787524in}{2.499304in}}%
\pgfpathlineto{\pgfqpoint{2.787075in}{2.498993in}}%
\pgfpathlineto{\pgfqpoint{2.776498in}{2.492081in}}%
\pgfpathlineto{\pgfqpoint{2.771407in}{2.489009in}}%
\pgfpathlineto{\pgfqpoint{2.765921in}{2.485590in}}%
\pgfpathlineto{\pgfqpoint{2.755344in}{2.479660in}}%
\pgfpathlineto{\pgfqpoint{2.753427in}{2.478715in}}%
\pgfpathlineto{\pgfqpoint{2.744767in}{2.474316in}}%
\pgfpathlineto{\pgfqpoint{2.734190in}{2.469816in}}%
\pgfpathlineto{\pgfqpoint{2.730093in}{2.468421in}}%
\pgfpathlineto{\pgfqpoint{2.723613in}{2.466144in}}%
\pgfpathlineto{\pgfqpoint{2.713036in}{2.463419in}}%
\pgfpathlineto{\pgfqpoint{2.702459in}{2.461730in}}%
\pgfpathlineto{\pgfqpoint{2.691882in}{2.461099in}}%
\pgfpathlineto{\pgfqpoint{2.681305in}{2.461552in}}%
\pgfpathlineto{\pgfqpoint{2.670727in}{2.463124in}}%
\pgfpathlineto{\pgfqpoint{2.660150in}{2.465866in}}%
\pgfpathlineto{\pgfqpoint{2.653312in}{2.468421in}}%
\pgfpathlineto{\pgfqpoint{2.649573in}{2.469803in}}%
\pgfpathlineto{\pgfqpoint{2.638996in}{2.474972in}}%
\pgfpathlineto{\pgfqpoint{2.632879in}{2.478715in}}%
\pgfpathlineto{\pgfqpoint{2.628419in}{2.481516in}}%
\pgfpathlineto{\pgfqpoint{2.618503in}{2.489009in}}%
\pgfpathlineto{\pgfqpoint{2.617842in}{2.489537in}}%
\pgfpathlineto{\pgfqpoint{2.607265in}{2.499247in}}%
\pgfpathlineto{\pgfqpoint{2.607209in}{2.499304in}}%
\pgfpathlineto{\pgfqpoint{2.597829in}{2.509598in}}%
\pgfpathlineto{\pgfqpoint{2.596688in}{2.510985in}}%
\pgfpathlineto{\pgfqpoint{2.589750in}{2.519893in}}%
\pgfpathlineto{\pgfqpoint{2.586111in}{2.525111in}}%
\pgfpathlineto{\pgfqpoint{2.582635in}{2.530187in}}%
\pgfpathlineto{\pgfqpoint{2.576276in}{2.540482in}}%
\pgfpathlineto{\pgfqpoint{2.575534in}{2.541806in}}%
\pgfpathlineto{\pgfqpoint{2.570386in}{2.550776in}}%
\pgfpathlineto{\pgfqpoint{2.564957in}{2.560919in}}%
\pgfpathlineto{\pgfqpoint{2.564872in}{2.561071in}}%
\pgfpathlineto{\pgfqpoint{2.559350in}{2.571365in}}%
\pgfpathlineto{\pgfqpoint{2.554380in}{2.580591in}}%
\pgfpathlineto{\pgfqpoint{2.553759in}{2.581659in}}%
\pgfpathlineto{\pgfqpoint{2.547615in}{2.591954in}}%
\pgfpathlineto{\pgfqpoint{2.543803in}{2.597856in}}%
\pgfpathlineto{\pgfqpoint{2.540683in}{2.602248in}}%
\pgfpathlineto{\pgfqpoint{2.533226in}{2.611695in}}%
\pgfpathlineto{\pgfqpoint{2.532482in}{2.612543in}}%
\pgfpathlineto{\pgfqpoint{2.522649in}{2.622510in}}%
\pgfpathlineto{\pgfqpoint{2.522286in}{2.622837in}}%
\pgfpathlineto{\pgfqpoint{2.512072in}{2.631037in}}%
\pgfpathlineto{\pgfqpoint{2.509132in}{2.633132in}}%
\pgfpathlineto{\pgfqpoint{2.501495in}{2.637991in}}%
\pgfpathlineto{\pgfqpoint{2.491911in}{2.643426in}}%
\pgfpathlineto{\pgfqpoint{2.490917in}{2.643933in}}%
\pgfpathlineto{\pgfqpoint{2.480340in}{2.648847in}}%
\pgfpathlineto{\pgfqpoint{2.469763in}{2.653340in}}%
\pgfpathlineto{\pgfqpoint{2.468796in}{2.653720in}}%
\pgfpathlineto{\pgfqpoint{2.459186in}{2.657225in}}%
\pgfpathlineto{\pgfqpoint{2.448609in}{2.660936in}}%
\pgfpathlineto{\pgfqpoint{2.439745in}{2.664015in}}%
\pgfpathlineto{\pgfqpoint{2.438032in}{2.664579in}}%
\pgfpathlineto{\pgfqpoint{2.427455in}{2.668110in}}%
\pgfpathlineto{\pgfqpoint{2.416878in}{2.671911in}}%
\pgfpathlineto{\pgfqpoint{2.410768in}{2.674309in}}%
\pgfpathlineto{\pgfqpoint{2.406301in}{2.676036in}}%
\pgfpathlineto{\pgfqpoint{2.395724in}{2.680629in}}%
\pgfpathlineto{\pgfqpoint{2.387840in}{2.684604in}}%
\pgfpathlineto{\pgfqpoint{2.385147in}{2.685981in}}%
\pgfpathlineto{\pgfqpoint{2.374570in}{2.692162in}}%
\pgfpathlineto{\pgfqpoint{2.370495in}{2.694898in}}%
\pgfpathlineto{\pgfqpoint{2.363993in}{2.699411in}}%
\pgfpathlineto{\pgfqpoint{2.356638in}{2.705193in}}%
\pgfpathlineto{\pgfqpoint{2.353416in}{2.707837in}}%
\pgfpathlineto{\pgfqpoint{2.344935in}{2.715487in}}%
\pgfpathlineto{\pgfqpoint{2.342839in}{2.717474in}}%
\pgfpathlineto{\pgfqpoint{2.334663in}{2.725781in}}%
\pgfpathlineto{\pgfqpoint{2.332262in}{2.728357in}}%
\pgfpathlineto{\pgfqpoint{2.325435in}{2.736076in}}%
\pgfpathlineto{\pgfqpoint{2.321684in}{2.740584in}}%
\pgfpathlineto{\pgfqpoint{2.317073in}{2.746370in}}%
\pgfpathlineto{\pgfqpoint{2.311107in}{2.754420in}}%
\pgfpathlineto{\pgfqpoint{2.309506in}{2.756665in}}%
\pgfpathlineto{\pgfqpoint{2.302807in}{2.766959in}}%
\pgfpathlineto{\pgfqpoint{2.300530in}{2.770911in}}%
\pgfpathlineto{\pgfqpoint{2.296999in}{2.777254in}}%
\pgfpathlineto{\pgfqpoint{2.292089in}{2.787548in}}%
\pgfpathlineto{\pgfqpoint{2.289953in}{2.793008in}}%
\pgfpathlineto{\pgfqpoint{2.288131in}{2.797842in}}%
\pgfpathlineto{\pgfqpoint{2.285135in}{2.808137in}}%
\pgfpathlineto{\pgfqpoint{2.283012in}{2.818431in}}%
\pgfpathlineto{\pgfqpoint{2.281760in}{2.828726in}}%
\pgfpathlineto{\pgfqpoint{2.281378in}{2.839020in}}%
\pgfpathlineto{\pgfqpoint{2.281874in}{2.849315in}}%
\pgfpathlineto{\pgfqpoint{2.283271in}{2.859609in}}%
\pgfpathlineto{\pgfqpoint{2.285608in}{2.869903in}}%
\pgfpathlineto{\pgfqpoint{2.288948in}{2.880198in}}%
\pgfpathlineto{\pgfqpoint{2.289953in}{2.882579in}}%
\pgfpathlineto{\pgfqpoint{2.293513in}{2.890492in}}%
\pgfpathlineto{\pgfqpoint{2.299349in}{2.900787in}}%
\pgfpathlineto{\pgfqpoint{2.300530in}{2.902517in}}%
\pgfpathlineto{\pgfqpoint{2.306718in}{2.911081in}}%
\pgfpathlineto{\pgfqpoint{2.300530in}{2.911081in}}%
\pgfpathlineto{\pgfqpoint{2.289953in}{2.911081in}}%
\pgfpathlineto{\pgfqpoint{2.279376in}{2.911081in}}%
\pgfpathlineto{\pgfqpoint{2.268799in}{2.911081in}}%
\pgfpathlineto{\pgfqpoint{2.262968in}{2.911081in}}%
\pgfpathlineto{\pgfqpoint{2.258222in}{2.901397in}}%
\pgfpathlineto{\pgfqpoint{2.257946in}{2.900787in}}%
\pgfpathlineto{\pgfqpoint{2.254163in}{2.890492in}}%
\pgfpathlineto{\pgfqpoint{2.251301in}{2.880198in}}%
\pgfpathlineto{\pgfqpoint{2.249250in}{2.869903in}}%
\pgfpathlineto{\pgfqpoint{2.247935in}{2.859609in}}%
\pgfpathlineto{\pgfqpoint{2.247645in}{2.854890in}}%
\pgfpathlineto{\pgfqpoint{2.247322in}{2.849315in}}%
\pgfpathlineto{\pgfqpoint{2.247355in}{2.839020in}}%
\pgfpathlineto{\pgfqpoint{2.247645in}{2.834528in}}%
\pgfpathlineto{\pgfqpoint{2.248032in}{2.828726in}}%
\pgfpathlineto{\pgfqpoint{2.249389in}{2.818431in}}%
\pgfpathlineto{\pgfqpoint{2.251442in}{2.808137in}}%
\pgfpathlineto{\pgfqpoint{2.254227in}{2.797842in}}%
\pgfpathlineto{\pgfqpoint{2.257769in}{2.787548in}}%
\pgfpathlineto{\pgfqpoint{2.258222in}{2.786462in}}%
\pgfpathlineto{\pgfqpoint{2.262215in}{2.777254in}}%
\pgfpathlineto{\pgfqpoint{2.267383in}{2.766959in}}%
\pgfpathlineto{\pgfqpoint{2.268799in}{2.764423in}}%
\pgfpathlineto{\pgfqpoint{2.273309in}{2.756665in}}%
\pgfpathlineto{\pgfqpoint{2.279376in}{2.746964in}}%
\pgfpathlineto{\pgfqpoint{2.279764in}{2.746370in}}%
\pgfpathlineto{\pgfqpoint{2.286790in}{2.736076in}}%
\pgfpathlineto{\pgfqpoint{2.289953in}{2.731543in}}%
\pgfpathlineto{\pgfqpoint{2.294198in}{2.725781in}}%
\pgfpathlineto{\pgfqpoint{2.300530in}{2.717271in}}%
\pgfpathlineto{\pgfqpoint{2.301944in}{2.715487in}}%
\pgfpathlineto{\pgfqpoint{2.310096in}{2.705193in}}%
\pgfpathlineto{\pgfqpoint{2.311107in}{2.703897in}}%
\pgfpathlineto{\pgfqpoint{2.318762in}{2.694898in}}%
\pgfpathlineto{\pgfqpoint{2.321684in}{2.691439in}}%
\pgfpathlineto{\pgfqpoint{2.328098in}{2.684604in}}%
\pgfpathlineto{\pgfqpoint{2.332262in}{2.680164in}}%
\pgfpathlineto{\pgfqpoint{2.338512in}{2.674309in}}%
\pgfpathlineto{\pgfqpoint{2.342839in}{2.670283in}}%
\pgfpathlineto{\pgfqpoint{2.350736in}{2.664015in}}%
\pgfpathlineto{\pgfqpoint{2.353416in}{2.661913in}}%
\pgfpathlineto{\pgfqpoint{2.363993in}{2.654952in}}%
\pgfpathlineto{\pgfqpoint{2.366224in}{2.653720in}}%
\pgfpathlineto{\pgfqpoint{2.374570in}{2.649148in}}%
\pgfpathlineto{\pgfqpoint{2.385147in}{2.644383in}}%
\pgfpathlineto{\pgfqpoint{2.387595in}{2.643426in}}%
\pgfpathlineto{\pgfqpoint{2.395724in}{2.640221in}}%
\pgfpathlineto{\pgfqpoint{2.406301in}{2.636536in}}%
\pgfpathlineto{\pgfqpoint{2.416878in}{2.633153in}}%
\pgfpathlineto{\pgfqpoint{2.416945in}{2.633132in}}%
\pgfpathlineto{\pgfqpoint{2.427455in}{2.629655in}}%
\pgfpathlineto{\pgfqpoint{2.438032in}{2.626070in}}%
\pgfpathlineto{\pgfqpoint{2.446944in}{2.622837in}}%
\pgfpathlineto{\pgfqpoint{2.448609in}{2.622200in}}%
\pgfpathlineto{\pgfqpoint{2.459186in}{2.617673in}}%
\pgfpathlineto{\pgfqpoint{2.469763in}{2.612543in}}%
\pgfpathlineto{\pgfqpoint{2.469764in}{2.612543in}}%
\pgfpathlineto{\pgfqpoint{2.480340in}{2.606255in}}%
\pgfpathlineto{\pgfqpoint{2.486175in}{2.602248in}}%
\pgfpathlineto{\pgfqpoint{2.490917in}{2.598756in}}%
\pgfpathlineto{\pgfqpoint{2.498924in}{2.591954in}}%
\pgfpathlineto{\pgfqpoint{2.501495in}{2.589631in}}%
\pgfpathlineto{\pgfqpoint{2.509211in}{2.581659in}}%
\pgfpathlineto{\pgfqpoint{2.512072in}{2.578561in}}%
\pgfpathlineto{\pgfqpoint{2.517962in}{2.571365in}}%
\pgfpathlineto{\pgfqpoint{2.522649in}{2.565482in}}%
\pgfpathlineto{\pgfqpoint{2.525816in}{2.561071in}}%
\pgfpathlineto{\pgfqpoint{2.533194in}{2.550776in}}%
\pgfpathlineto{\pgfqpoint{2.533226in}{2.550733in}}%
\pgfpathlineto{\pgfqpoint{2.540165in}{2.540482in}}%
\pgfpathlineto{\pgfqpoint{2.543803in}{2.535316in}}%
\pgfpathlineto{\pgfqpoint{2.547214in}{2.530187in}}%
\pgfpathlineto{\pgfqpoint{2.554380in}{2.520036in}}%
\pgfpathlineto{\pgfqpoint{2.554478in}{2.519893in}}%
\pgfpathlineto{\pgfqpoint{2.561982in}{2.509598in}}%
\pgfpathlineto{\pgfqpoint{2.564957in}{2.505789in}}%
\pgfpathlineto{\pgfqpoint{2.570019in}{2.499304in}}%
\pgfpathlineto{\pgfqpoint{2.575534in}{2.492713in}}%
\pgfpathlineto{\pgfqpoint{2.578714in}{2.489009in}}%
\pgfpathlineto{\pgfqpoint{2.586111in}{2.480917in}}%
\pgfpathlineto{\pgfqpoint{2.588235in}{2.478715in}}%
\pgfpathlineto{\pgfqpoint{2.596688in}{2.470375in}}%
\pgfpathlineto{\pgfqpoint{2.598839in}{2.468421in}}%
\pgfpathlineto{\pgfqpoint{2.607265in}{2.461006in}}%
\pgfpathlineto{\pgfqpoint{2.610919in}{2.458126in}}%
\pgfpathlineto{\pgfqpoint{2.617842in}{2.452740in}}%
\pgfpathlineto{\pgfqpoint{2.625083in}{2.447832in}}%
\pgfpathlineto{\pgfqpoint{2.628419in}{2.445556in}}%
\pgfpathlineto{\pgfqpoint{2.638996in}{2.439407in}}%
\pgfpathlineto{\pgfqpoint{2.642922in}{2.437537in}}%
\pgfpathlineto{\pgfqpoint{2.649573in}{2.434266in}}%
\pgfpathlineto{\pgfqpoint{2.660150in}{2.430189in}}%
\pgfpathlineto{\pgfqpoint{2.670706in}{2.427243in}}%
\pgfpathclose%
\pgfusepath{fill}%
\end{pgfscope}%
\begin{pgfscope}%
\pgfpathrectangle{\pgfqpoint{1.856795in}{1.819814in}}{\pgfqpoint{1.194205in}{1.163386in}}%
\pgfusepath{clip}%
\pgfsetbuttcap%
\pgfsetroundjoin%
\definecolor{currentfill}{rgb}{0.960043,0.546576,0.387029}%
\pgfsetfillcolor{currentfill}%
\pgfsetlinewidth{0.000000pt}%
\definecolor{currentstroke}{rgb}{0.000000,0.000000,0.000000}%
\pgfsetstrokecolor{currentstroke}%
\pgfsetdash{}{0pt}%
\pgfpathmoveto{\pgfqpoint{2.237068in}{1.891932in}}%
\pgfpathlineto{\pgfqpoint{2.247645in}{1.891932in}}%
\pgfpathlineto{\pgfqpoint{2.258222in}{1.891932in}}%
\pgfpathlineto{\pgfqpoint{2.268799in}{1.891932in}}%
\pgfpathlineto{\pgfqpoint{2.276175in}{1.891932in}}%
\pgfpathlineto{\pgfqpoint{2.276171in}{1.902227in}}%
\pgfpathlineto{\pgfqpoint{2.275462in}{1.912521in}}%
\pgfpathlineto{\pgfqpoint{2.274129in}{1.922816in}}%
\pgfpathlineto{\pgfqpoint{2.272228in}{1.933110in}}%
\pgfpathlineto{\pgfqpoint{2.269798in}{1.943405in}}%
\pgfpathlineto{\pgfqpoint{2.268799in}{1.946924in}}%
\pgfpathlineto{\pgfqpoint{2.266860in}{1.953699in}}%
\pgfpathlineto{\pgfqpoint{2.263408in}{1.963993in}}%
\pgfpathlineto{\pgfqpoint{2.259397in}{1.974288in}}%
\pgfpathlineto{\pgfqpoint{2.258222in}{1.976853in}}%
\pgfpathlineto{\pgfqpoint{2.254459in}{1.984582in}}%
\pgfpathlineto{\pgfqpoint{2.247645in}{1.992914in}}%
\pgfpathlineto{\pgfqpoint{2.245922in}{1.994877in}}%
\pgfpathlineto{\pgfqpoint{2.237068in}{2.002326in}}%
\pgfpathlineto{\pgfqpoint{2.233619in}{2.005171in}}%
\pgfpathlineto{\pgfqpoint{2.226491in}{2.010791in}}%
\pgfpathlineto{\pgfqpoint{2.220489in}{2.015466in}}%
\pgfpathlineto{\pgfqpoint{2.215914in}{2.019042in}}%
\pgfpathlineto{\pgfqpoint{2.207371in}{2.025760in}}%
\pgfpathlineto{\pgfqpoint{2.205337in}{2.027392in}}%
\pgfpathlineto{\pgfqpoint{2.194842in}{2.036054in}}%
\pgfpathlineto{\pgfqpoint{2.194760in}{2.036125in}}%
\pgfpathlineto{\pgfqpoint{2.184183in}{2.045612in}}%
\pgfpathlineto{\pgfqpoint{2.183415in}{2.046349in}}%
\pgfpathlineto{\pgfqpoint{2.173606in}{2.056082in}}%
\pgfpathlineto{\pgfqpoint{2.173084in}{2.056643in}}%
\pgfpathlineto{\pgfqpoint{2.163876in}{2.066938in}}%
\pgfpathlineto{\pgfqpoint{2.163029in}{2.067915in}}%
\pgfpathlineto{\pgfqpoint{2.155692in}{2.077232in}}%
\pgfpathlineto{\pgfqpoint{2.152451in}{2.081420in}}%
\pgfpathlineto{\pgfqpoint{2.148181in}{2.087527in}}%
\pgfpathlineto{\pgfqpoint{2.141874in}{2.096532in}}%
\pgfpathlineto{\pgfqpoint{2.141058in}{2.097821in}}%
\pgfpathlineto{\pgfqpoint{2.134484in}{2.108116in}}%
\pgfpathlineto{\pgfqpoint{2.131297in}{2.112936in}}%
\pgfpathlineto{\pgfqpoint{2.127992in}{2.118410in}}%
\pgfpathlineto{\pgfqpoint{2.121464in}{2.128704in}}%
\pgfpathlineto{\pgfqpoint{2.120720in}{2.129842in}}%
\pgfpathlineto{\pgfqpoint{2.115208in}{2.138999in}}%
\pgfpathlineto{\pgfqpoint{2.110143in}{2.147023in}}%
\pgfpathlineto{\pgfqpoint{2.108817in}{2.149293in}}%
\pgfpathlineto{\pgfqpoint{2.102723in}{2.159588in}}%
\pgfpathlineto{\pgfqpoint{2.099566in}{2.164828in}}%
\pgfpathlineto{\pgfqpoint{2.096764in}{2.169882in}}%
\pgfpathlineto{\pgfqpoint{2.091049in}{2.180177in}}%
\pgfpathlineto{\pgfqpoint{2.088989in}{2.183914in}}%
\pgfpathlineto{\pgfqpoint{2.085684in}{2.190471in}}%
\pgfpathlineto{\pgfqpoint{2.080550in}{2.200765in}}%
\pgfpathlineto{\pgfqpoint{2.078412in}{2.205125in}}%
\pgfpathlineto{\pgfqpoint{2.075762in}{2.211060in}}%
\pgfpathlineto{\pgfqpoint{2.071255in}{2.221354in}}%
\pgfpathlineto{\pgfqpoint{2.067835in}{2.229271in}}%
\pgfpathlineto{\pgfqpoint{2.066901in}{2.231649in}}%
\pgfpathlineto{\pgfqpoint{2.062958in}{2.241943in}}%
\pgfpathlineto{\pgfqpoint{2.059043in}{2.252238in}}%
\pgfpathlineto{\pgfqpoint{2.057258in}{2.256980in}}%
\pgfpathlineto{\pgfqpoint{2.055359in}{2.262532in}}%
\pgfpathlineto{\pgfqpoint{2.051861in}{2.272826in}}%
\pgfpathlineto{\pgfqpoint{2.048332in}{2.283121in}}%
\pgfpathlineto{\pgfqpoint{2.046681in}{2.287903in}}%
\pgfpathlineto{\pgfqpoint{2.044944in}{2.293415in}}%
\pgfpathlineto{\pgfqpoint{2.041660in}{2.303710in}}%
\pgfpathlineto{\pgfqpoint{2.038283in}{2.314004in}}%
\pgfpathlineto{\pgfqpoint{2.036104in}{2.320471in}}%
\pgfpathlineto{\pgfqpoint{2.034917in}{2.324299in}}%
\pgfpathlineto{\pgfqpoint{2.031643in}{2.334593in}}%
\pgfpathlineto{\pgfqpoint{2.028227in}{2.344887in}}%
\pgfpathlineto{\pgfqpoint{2.025527in}{2.352705in}}%
\pgfpathlineto{\pgfqpoint{2.024730in}{2.355182in}}%
\pgfpathlineto{\pgfqpoint{2.021318in}{2.365476in}}%
\pgfpathlineto{\pgfqpoint{2.017736in}{2.375771in}}%
\pgfpathlineto{\pgfqpoint{2.014950in}{2.383454in}}%
\pgfpathlineto{\pgfqpoint{2.014052in}{2.386065in}}%
\pgfpathlineto{\pgfqpoint{2.010415in}{2.396360in}}%
\pgfpathlineto{\pgfqpoint{2.006606in}{2.406654in}}%
\pgfpathlineto{\pgfqpoint{2.004373in}{2.412514in}}%
\pgfpathlineto{\pgfqpoint{2.002742in}{2.416948in}}%
\pgfpathlineto{\pgfqpoint{1.998877in}{2.427243in}}%
\pgfpathlineto{\pgfqpoint{1.994884in}{2.437537in}}%
\pgfpathlineto{\pgfqpoint{1.993796in}{2.440328in}}%
\pgfpathlineto{\pgfqpoint{1.990955in}{2.447832in}}%
\pgfpathlineto{\pgfqpoint{1.987028in}{2.458126in}}%
\pgfpathlineto{\pgfqpoint{1.983218in}{2.468048in}}%
\pgfpathlineto{\pgfqpoint{1.983080in}{2.468421in}}%
\pgfpathlineto{\pgfqpoint{1.979355in}{2.478715in}}%
\pgfpathlineto{\pgfqpoint{1.975688in}{2.489009in}}%
\pgfpathlineto{\pgfqpoint{1.972641in}{2.497783in}}%
\pgfpathlineto{\pgfqpoint{1.972135in}{2.499304in}}%
\pgfpathlineto{\pgfqpoint{1.968866in}{2.509598in}}%
\pgfpathlineto{\pgfqpoint{1.965734in}{2.519893in}}%
\pgfpathlineto{\pgfqpoint{1.962748in}{2.530187in}}%
\pgfpathlineto{\pgfqpoint{1.962064in}{2.532678in}}%
\pgfpathlineto{\pgfqpoint{1.960035in}{2.540482in}}%
\pgfpathlineto{\pgfqpoint{1.957509in}{2.550776in}}%
\pgfpathlineto{\pgfqpoint{1.955124in}{2.561071in}}%
\pgfpathlineto{\pgfqpoint{1.952863in}{2.571365in}}%
\pgfpathlineto{\pgfqpoint{1.951487in}{2.577933in}}%
\pgfpathlineto{\pgfqpoint{1.950750in}{2.581659in}}%
\pgfpathlineto{\pgfqpoint{1.948794in}{2.591954in}}%
\pgfpathlineto{\pgfqpoint{1.946884in}{2.602248in}}%
\pgfpathlineto{\pgfqpoint{1.944940in}{2.612543in}}%
\pgfpathlineto{\pgfqpoint{1.942491in}{2.622837in}}%
\pgfpathlineto{\pgfqpoint{1.940910in}{2.627975in}}%
\pgfpathlineto{\pgfqpoint{1.939301in}{2.633132in}}%
\pgfpathlineto{\pgfqpoint{1.935952in}{2.643426in}}%
\pgfpathlineto{\pgfqpoint{1.932522in}{2.653720in}}%
\pgfpathlineto{\pgfqpoint{1.930333in}{2.660242in}}%
\pgfpathlineto{\pgfqpoint{1.930333in}{2.653720in}}%
\pgfpathlineto{\pgfqpoint{1.930333in}{2.643426in}}%
\pgfpathlineto{\pgfqpoint{1.930333in}{2.633132in}}%
\pgfpathlineto{\pgfqpoint{1.930333in}{2.622837in}}%
\pgfpathlineto{\pgfqpoint{1.930333in}{2.614761in}}%
\pgfpathlineto{\pgfqpoint{1.930947in}{2.612543in}}%
\pgfpathlineto{\pgfqpoint{1.932659in}{2.602248in}}%
\pgfpathlineto{\pgfqpoint{1.934414in}{2.591954in}}%
\pgfpathlineto{\pgfqpoint{1.936232in}{2.581659in}}%
\pgfpathlineto{\pgfqpoint{1.938139in}{2.571365in}}%
\pgfpathlineto{\pgfqpoint{1.940159in}{2.561071in}}%
\pgfpathlineto{\pgfqpoint{1.940910in}{2.557477in}}%
\pgfpathlineto{\pgfqpoint{1.942373in}{2.550776in}}%
\pgfpathlineto{\pgfqpoint{1.944771in}{2.540482in}}%
\pgfpathlineto{\pgfqpoint{1.947338in}{2.530187in}}%
\pgfpathlineto{\pgfqpoint{1.950082in}{2.519893in}}%
\pgfpathlineto{\pgfqpoint{1.951487in}{2.514958in}}%
\pgfpathlineto{\pgfqpoint{1.953075in}{2.509598in}}%
\pgfpathlineto{\pgfqpoint{1.956304in}{2.499304in}}%
\pgfpathlineto{\pgfqpoint{1.959697in}{2.489009in}}%
\pgfpathlineto{\pgfqpoint{1.962064in}{2.482152in}}%
\pgfpathlineto{\pgfqpoint{1.963289in}{2.478715in}}%
\pgfpathlineto{\pgfqpoint{1.967100in}{2.468421in}}%
\pgfpathlineto{\pgfqpoint{1.970991in}{2.458126in}}%
\pgfpathlineto{\pgfqpoint{1.972641in}{2.453861in}}%
\pgfpathlineto{\pgfqpoint{1.975023in}{2.447832in}}%
\pgfpathlineto{\pgfqpoint{1.979115in}{2.437537in}}%
\pgfpathlineto{\pgfqpoint{1.983154in}{2.427243in}}%
\pgfpathlineto{\pgfqpoint{1.983218in}{2.427080in}}%
\pgfpathlineto{\pgfqpoint{1.987274in}{2.416948in}}%
\pgfpathlineto{\pgfqpoint{1.991259in}{2.406654in}}%
\pgfpathlineto{\pgfqpoint{1.993796in}{2.399896in}}%
\pgfpathlineto{\pgfqpoint{1.995153in}{2.396360in}}%
\pgfpathlineto{\pgfqpoint{1.999000in}{2.386065in}}%
\pgfpathlineto{\pgfqpoint{2.002654in}{2.375771in}}%
\pgfpathlineto{\pgfqpoint{2.004373in}{2.370732in}}%
\pgfpathlineto{\pgfqpoint{2.006241in}{2.365476in}}%
\pgfpathlineto{\pgfqpoint{2.009758in}{2.355182in}}%
\pgfpathlineto{\pgfqpoint{2.013094in}{2.344887in}}%
\pgfpathlineto{\pgfqpoint{2.014950in}{2.338920in}}%
\pgfpathlineto{\pgfqpoint{2.016380in}{2.334593in}}%
\pgfpathlineto{\pgfqpoint{2.019667in}{2.324299in}}%
\pgfpathlineto{\pgfqpoint{2.022817in}{2.314004in}}%
\pgfpathlineto{\pgfqpoint{2.025527in}{2.304840in}}%
\pgfpathlineto{\pgfqpoint{2.025887in}{2.303710in}}%
\pgfpathlineto{\pgfqpoint{2.029113in}{2.293415in}}%
\pgfpathlineto{\pgfqpoint{2.032263in}{2.283121in}}%
\pgfpathlineto{\pgfqpoint{2.035366in}{2.272826in}}%
\pgfpathlineto{\pgfqpoint{2.036104in}{2.270386in}}%
\pgfpathlineto{\pgfqpoint{2.038678in}{2.262532in}}%
\pgfpathlineto{\pgfqpoint{2.042056in}{2.252238in}}%
\pgfpathlineto{\pgfqpoint{2.045460in}{2.241943in}}%
\pgfpathlineto{\pgfqpoint{2.046681in}{2.238332in}}%
\pgfpathlineto{\pgfqpoint{2.049133in}{2.231649in}}%
\pgfpathlineto{\pgfqpoint{2.052988in}{2.221354in}}%
\pgfpathlineto{\pgfqpoint{2.056928in}{2.211060in}}%
\pgfpathlineto{\pgfqpoint{2.057258in}{2.210228in}}%
\pgfpathlineto{\pgfqpoint{2.061320in}{2.200765in}}%
\pgfpathlineto{\pgfqpoint{2.065836in}{2.190471in}}%
\pgfpathlineto{\pgfqpoint{2.067835in}{2.186048in}}%
\pgfpathlineto{\pgfqpoint{2.070697in}{2.180177in}}%
\pgfpathlineto{\pgfqpoint{2.075810in}{2.169882in}}%
\pgfpathlineto{\pgfqpoint{2.078412in}{2.164737in}}%
\pgfpathlineto{\pgfqpoint{2.081202in}{2.159588in}}%
\pgfpathlineto{\pgfqpoint{2.086798in}{2.149293in}}%
\pgfpathlineto{\pgfqpoint{2.088989in}{2.145261in}}%
\pgfpathlineto{\pgfqpoint{2.092606in}{2.138999in}}%
\pgfpathlineto{\pgfqpoint{2.098388in}{2.128704in}}%
\pgfpathlineto{\pgfqpoint{2.099566in}{2.126548in}}%
\pgfpathlineto{\pgfqpoint{2.104297in}{2.118410in}}%
\pgfpathlineto{\pgfqpoint{2.109917in}{2.108116in}}%
\pgfpathlineto{\pgfqpoint{2.110143in}{2.107685in}}%
\pgfpathlineto{\pgfqpoint{2.115756in}{2.097821in}}%
\pgfpathlineto{\pgfqpoint{2.120720in}{2.088690in}}%
\pgfpathlineto{\pgfqpoint{2.121415in}{2.087527in}}%
\pgfpathlineto{\pgfqpoint{2.127546in}{2.077232in}}%
\pgfpathlineto{\pgfqpoint{2.131297in}{2.070991in}}%
\pgfpathlineto{\pgfqpoint{2.133989in}{2.066938in}}%
\pgfpathlineto{\pgfqpoint{2.141019in}{2.056643in}}%
\pgfpathlineto{\pgfqpoint{2.141874in}{2.055441in}}%
\pgfpathlineto{\pgfqpoint{2.148998in}{2.046349in}}%
\pgfpathlineto{\pgfqpoint{2.152451in}{2.042098in}}%
\pgfpathlineto{\pgfqpoint{2.157827in}{2.036054in}}%
\pgfpathlineto{\pgfqpoint{2.163029in}{2.030388in}}%
\pgfpathlineto{\pgfqpoint{2.167641in}{2.025760in}}%
\pgfpathlineto{\pgfqpoint{2.173606in}{2.019894in}}%
\pgfpathlineto{\pgfqpoint{2.178440in}{2.015466in}}%
\pgfpathlineto{\pgfqpoint{2.184183in}{2.010193in}}%
\pgfpathlineto{\pgfqpoint{2.189982in}{2.005171in}}%
\pgfpathlineto{\pgfqpoint{2.194760in}{2.000809in}}%
\pgfpathlineto{\pgfqpoint{2.201550in}{1.994877in}}%
\pgfpathlineto{\pgfqpoint{2.205337in}{1.990940in}}%
\pgfpathlineto{\pgfqpoint{2.211536in}{1.984582in}}%
\pgfpathlineto{\pgfqpoint{2.215914in}{1.978311in}}%
\pgfpathlineto{\pgfqpoint{2.218665in}{1.974288in}}%
\pgfpathlineto{\pgfqpoint{2.223889in}{1.963993in}}%
\pgfpathlineto{\pgfqpoint{2.226491in}{1.957483in}}%
\pgfpathlineto{\pgfqpoint{2.228004in}{1.953699in}}%
\pgfpathlineto{\pgfqpoint{2.231335in}{1.943405in}}%
\pgfpathlineto{\pgfqpoint{2.233903in}{1.933110in}}%
\pgfpathlineto{\pgfqpoint{2.235726in}{1.922816in}}%
\pgfpathlineto{\pgfqpoint{2.236783in}{1.912521in}}%
\pgfpathlineto{\pgfqpoint{2.237023in}{1.902227in}}%
\pgfpathlineto{\pgfqpoint{2.236361in}{1.891932in}}%
\pgfpathclose%
\pgfusepath{fill}%
\end{pgfscope}%
\begin{pgfscope}%
\pgfpathrectangle{\pgfqpoint{1.856795in}{1.819814in}}{\pgfqpoint{1.194205in}{1.163386in}}%
\pgfusepath{clip}%
\pgfsetbuttcap%
\pgfsetroundjoin%
\definecolor{currentfill}{rgb}{0.960043,0.546576,0.387029}%
\pgfsetfillcolor{currentfill}%
\pgfsetlinewidth{0.000000pt}%
\definecolor{currentstroke}{rgb}{0.000000,0.000000,0.000000}%
\pgfsetstrokecolor{currentstroke}%
\pgfsetdash{}{0pt}%
\pgfpathmoveto{\pgfqpoint{2.670727in}{2.382162in}}%
\pgfpathlineto{\pgfqpoint{2.681305in}{2.379119in}}%
\pgfpathlineto{\pgfqpoint{2.691882in}{2.377399in}}%
\pgfpathlineto{\pgfqpoint{2.702459in}{2.377055in}}%
\pgfpathlineto{\pgfqpoint{2.713036in}{2.378144in}}%
\pgfpathlineto{\pgfqpoint{2.723613in}{2.380720in}}%
\pgfpathlineto{\pgfqpoint{2.734190in}{2.384834in}}%
\pgfpathlineto{\pgfqpoint{2.736523in}{2.386065in}}%
\pgfpathlineto{\pgfqpoint{2.744767in}{2.390170in}}%
\pgfpathlineto{\pgfqpoint{2.754555in}{2.396360in}}%
\pgfpathlineto{\pgfqpoint{2.755344in}{2.396832in}}%
\pgfpathlineto{\pgfqpoint{2.765921in}{2.404364in}}%
\pgfpathlineto{\pgfqpoint{2.768713in}{2.406654in}}%
\pgfpathlineto{\pgfqpoint{2.776498in}{2.412716in}}%
\pgfpathlineto{\pgfqpoint{2.781360in}{2.416948in}}%
\pgfpathlineto{\pgfqpoint{2.787075in}{2.421680in}}%
\pgfpathlineto{\pgfqpoint{2.793242in}{2.427243in}}%
\pgfpathlineto{\pgfqpoint{2.797652in}{2.431032in}}%
\pgfpathlineto{\pgfqpoint{2.804756in}{2.437537in}}%
\pgfpathlineto{\pgfqpoint{2.808229in}{2.440571in}}%
\pgfpathlineto{\pgfqpoint{2.816184in}{2.447832in}}%
\pgfpathlineto{\pgfqpoint{2.818806in}{2.450114in}}%
\pgfpathlineto{\pgfqpoint{2.827766in}{2.458126in}}%
\pgfpathlineto{\pgfqpoint{2.829383in}{2.459506in}}%
\pgfpathlineto{\pgfqpoint{2.839724in}{2.468421in}}%
\pgfpathlineto{\pgfqpoint{2.839960in}{2.468615in}}%
\pgfpathlineto{\pgfqpoint{2.850538in}{2.477262in}}%
\pgfpathlineto{\pgfqpoint{2.852346in}{2.478715in}}%
\pgfpathlineto{\pgfqpoint{2.861115in}{2.485445in}}%
\pgfpathlineto{\pgfqpoint{2.865890in}{2.489009in}}%
\pgfpathlineto{\pgfqpoint{2.871692in}{2.493171in}}%
\pgfpathlineto{\pgfqpoint{2.880528in}{2.499304in}}%
\pgfpathlineto{\pgfqpoint{2.882269in}{2.500479in}}%
\pgfpathlineto{\pgfqpoint{2.892846in}{2.507383in}}%
\pgfpathlineto{\pgfqpoint{2.896330in}{2.509598in}}%
\pgfpathlineto{\pgfqpoint{2.903423in}{2.514207in}}%
\pgfpathlineto{\pgfqpoint{2.911803in}{2.519893in}}%
\pgfpathlineto{\pgfqpoint{2.914000in}{2.521549in}}%
\pgfpathlineto{\pgfqpoint{2.923772in}{2.530187in}}%
\pgfpathlineto{\pgfqpoint{2.924577in}{2.531115in}}%
\pgfpathlineto{\pgfqpoint{2.930953in}{2.540482in}}%
\pgfpathlineto{\pgfqpoint{2.935154in}{2.550364in}}%
\pgfpathlineto{\pgfqpoint{2.935297in}{2.550776in}}%
\pgfpathlineto{\pgfqpoint{2.937953in}{2.561071in}}%
\pgfpathlineto{\pgfqpoint{2.939824in}{2.571365in}}%
\pgfpathlineto{\pgfqpoint{2.941242in}{2.581659in}}%
\pgfpathlineto{\pgfqpoint{2.942421in}{2.591954in}}%
\pgfpathlineto{\pgfqpoint{2.943508in}{2.602248in}}%
\pgfpathlineto{\pgfqpoint{2.944620in}{2.612543in}}%
\pgfpathlineto{\pgfqpoint{2.945731in}{2.621802in}}%
\pgfpathlineto{\pgfqpoint{2.945853in}{2.622837in}}%
\pgfpathlineto{\pgfqpoint{2.947289in}{2.633132in}}%
\pgfpathlineto{\pgfqpoint{2.949004in}{2.643426in}}%
\pgfpathlineto{\pgfqpoint{2.951068in}{2.653720in}}%
\pgfpathlineto{\pgfqpoint{2.953546in}{2.664015in}}%
\pgfpathlineto{\pgfqpoint{2.956308in}{2.673635in}}%
\pgfpathlineto{\pgfqpoint{2.956499in}{2.674309in}}%
\pgfpathlineto{\pgfqpoint{2.959961in}{2.684604in}}%
\pgfpathlineto{\pgfqpoint{2.963990in}{2.694898in}}%
\pgfpathlineto{\pgfqpoint{2.966885in}{2.701309in}}%
\pgfpathlineto{\pgfqpoint{2.968614in}{2.705193in}}%
\pgfpathlineto{\pgfqpoint{2.973843in}{2.715487in}}%
\pgfpathlineto{\pgfqpoint{2.977462in}{2.721811in}}%
\pgfpathlineto{\pgfqpoint{2.977462in}{2.725781in}}%
\pgfpathlineto{\pgfqpoint{2.977462in}{2.736076in}}%
\pgfpathlineto{\pgfqpoint{2.977462in}{2.746370in}}%
\pgfpathlineto{\pgfqpoint{2.977462in}{2.750075in}}%
\pgfpathlineto{\pgfqpoint{2.974577in}{2.746370in}}%
\pgfpathlineto{\pgfqpoint{2.967258in}{2.736076in}}%
\pgfpathlineto{\pgfqpoint{2.966885in}{2.735497in}}%
\pgfpathlineto{\pgfqpoint{2.960522in}{2.725781in}}%
\pgfpathlineto{\pgfqpoint{2.956308in}{2.718603in}}%
\pgfpathlineto{\pgfqpoint{2.954446in}{2.715487in}}%
\pgfpathlineto{\pgfqpoint{2.948998in}{2.705193in}}%
\pgfpathlineto{\pgfqpoint{2.945731in}{2.698168in}}%
\pgfpathlineto{\pgfqpoint{2.944183in}{2.694898in}}%
\pgfpathlineto{\pgfqpoint{2.939952in}{2.684604in}}%
\pgfpathlineto{\pgfqpoint{2.936297in}{2.674309in}}%
\pgfpathlineto{\pgfqpoint{2.935154in}{2.670547in}}%
\pgfpathlineto{\pgfqpoint{2.933134in}{2.664015in}}%
\pgfpathlineto{\pgfqpoint{2.930422in}{2.653720in}}%
\pgfpathlineto{\pgfqpoint{2.928101in}{2.643426in}}%
\pgfpathlineto{\pgfqpoint{2.926088in}{2.633132in}}%
\pgfpathlineto{\pgfqpoint{2.924577in}{2.624439in}}%
\pgfpathlineto{\pgfqpoint{2.924292in}{2.622837in}}%
\pgfpathlineto{\pgfqpoint{2.922600in}{2.612543in}}%
\pgfpathlineto{\pgfqpoint{2.920896in}{2.602248in}}%
\pgfpathlineto{\pgfqpoint{2.919013in}{2.591954in}}%
\pgfpathlineto{\pgfqpoint{2.916709in}{2.581659in}}%
\pgfpathlineto{\pgfqpoint{2.914000in}{2.572556in}}%
\pgfpathlineto{\pgfqpoint{2.913608in}{2.571365in}}%
\pgfpathlineto{\pgfqpoint{2.908948in}{2.561071in}}%
\pgfpathlineto{\pgfqpoint{2.903423in}{2.552848in}}%
\pgfpathlineto{\pgfqpoint{2.901755in}{2.550776in}}%
\pgfpathlineto{\pgfqpoint{2.892846in}{2.542160in}}%
\pgfpathlineto{\pgfqpoint{2.890844in}{2.540482in}}%
\pgfpathlineto{\pgfqpoint{2.882269in}{2.534009in}}%
\pgfpathlineto{\pgfqpoint{2.876853in}{2.530187in}}%
\pgfpathlineto{\pgfqpoint{2.871692in}{2.526625in}}%
\pgfpathlineto{\pgfqpoint{2.861782in}{2.519893in}}%
\pgfpathlineto{\pgfqpoint{2.861115in}{2.519432in}}%
\pgfpathlineto{\pgfqpoint{2.850538in}{2.512044in}}%
\pgfpathlineto{\pgfqpoint{2.847047in}{2.509598in}}%
\pgfpathlineto{\pgfqpoint{2.839960in}{2.504442in}}%
\pgfpathlineto{\pgfqpoint{2.832901in}{2.499304in}}%
\pgfpathlineto{\pgfqpoint{2.829383in}{2.496634in}}%
\pgfpathlineto{\pgfqpoint{2.819252in}{2.489009in}}%
\pgfpathlineto{\pgfqpoint{2.818806in}{2.488659in}}%
\pgfpathlineto{\pgfqpoint{2.808229in}{2.480504in}}%
\pgfpathlineto{\pgfqpoint{2.805827in}{2.478715in}}%
\pgfpathlineto{\pgfqpoint{2.797652in}{2.472374in}}%
\pgfpathlineto{\pgfqpoint{2.792278in}{2.468421in}}%
\pgfpathlineto{\pgfqpoint{2.787075in}{2.464438in}}%
\pgfpathlineto{\pgfqpoint{2.778199in}{2.458126in}}%
\pgfpathlineto{\pgfqpoint{2.776498in}{2.456868in}}%
\pgfpathlineto{\pgfqpoint{2.765921in}{2.449739in}}%
\pgfpathlineto{\pgfqpoint{2.762732in}{2.447832in}}%
\pgfpathlineto{\pgfqpoint{2.755344in}{2.443239in}}%
\pgfpathlineto{\pgfqpoint{2.744767in}{2.437634in}}%
\pgfpathlineto{\pgfqpoint{2.744547in}{2.437537in}}%
\pgfpathlineto{\pgfqpoint{2.734190in}{2.432791in}}%
\pgfpathlineto{\pgfqpoint{2.723613in}{2.429073in}}%
\pgfpathlineto{\pgfqpoint{2.716179in}{2.427243in}}%
\pgfpathlineto{\pgfqpoint{2.713036in}{2.426433in}}%
\pgfpathlineto{\pgfqpoint{2.702459in}{2.424871in}}%
\pgfpathlineto{\pgfqpoint{2.691882in}{2.424492in}}%
\pgfpathlineto{\pgfqpoint{2.681305in}{2.425283in}}%
\pgfpathlineto{\pgfqpoint{2.670727in}{2.427237in}}%
\pgfpathlineto{\pgfqpoint{2.670706in}{2.427243in}}%
\pgfpathlineto{\pgfqpoint{2.660150in}{2.430189in}}%
\pgfpathlineto{\pgfqpoint{2.649573in}{2.434266in}}%
\pgfpathlineto{\pgfqpoint{2.642922in}{2.437537in}}%
\pgfpathlineto{\pgfqpoint{2.638996in}{2.439407in}}%
\pgfpathlineto{\pgfqpoint{2.628419in}{2.445556in}}%
\pgfpathlineto{\pgfqpoint{2.625083in}{2.447832in}}%
\pgfpathlineto{\pgfqpoint{2.617842in}{2.452740in}}%
\pgfpathlineto{\pgfqpoint{2.610919in}{2.458126in}}%
\pgfpathlineto{\pgfqpoint{2.607265in}{2.461006in}}%
\pgfpathlineto{\pgfqpoint{2.598839in}{2.468421in}}%
\pgfpathlineto{\pgfqpoint{2.596688in}{2.470375in}}%
\pgfpathlineto{\pgfqpoint{2.588235in}{2.478715in}}%
\pgfpathlineto{\pgfqpoint{2.586111in}{2.480917in}}%
\pgfpathlineto{\pgfqpoint{2.578714in}{2.489009in}}%
\pgfpathlineto{\pgfqpoint{2.575534in}{2.492713in}}%
\pgfpathlineto{\pgfqpoint{2.570019in}{2.499304in}}%
\pgfpathlineto{\pgfqpoint{2.564957in}{2.505789in}}%
\pgfpathlineto{\pgfqpoint{2.561982in}{2.509598in}}%
\pgfpathlineto{\pgfqpoint{2.554478in}{2.519893in}}%
\pgfpathlineto{\pgfqpoint{2.554380in}{2.520036in}}%
\pgfpathlineto{\pgfqpoint{2.547214in}{2.530187in}}%
\pgfpathlineto{\pgfqpoint{2.543803in}{2.535316in}}%
\pgfpathlineto{\pgfqpoint{2.540165in}{2.540482in}}%
\pgfpathlineto{\pgfqpoint{2.533226in}{2.550733in}}%
\pgfpathlineto{\pgfqpoint{2.533194in}{2.550776in}}%
\pgfpathlineto{\pgfqpoint{2.525816in}{2.561071in}}%
\pgfpathlineto{\pgfqpoint{2.522649in}{2.565482in}}%
\pgfpathlineto{\pgfqpoint{2.517962in}{2.571365in}}%
\pgfpathlineto{\pgfqpoint{2.512072in}{2.578561in}}%
\pgfpathlineto{\pgfqpoint{2.509211in}{2.581659in}}%
\pgfpathlineto{\pgfqpoint{2.501495in}{2.589631in}}%
\pgfpathlineto{\pgfqpoint{2.498924in}{2.591954in}}%
\pgfpathlineto{\pgfqpoint{2.490917in}{2.598756in}}%
\pgfpathlineto{\pgfqpoint{2.486175in}{2.602248in}}%
\pgfpathlineto{\pgfqpoint{2.480340in}{2.606255in}}%
\pgfpathlineto{\pgfqpoint{2.469764in}{2.612543in}}%
\pgfpathlineto{\pgfqpoint{2.469763in}{2.612543in}}%
\pgfpathlineto{\pgfqpoint{2.459186in}{2.617673in}}%
\pgfpathlineto{\pgfqpoint{2.448609in}{2.622200in}}%
\pgfpathlineto{\pgfqpoint{2.446944in}{2.622837in}}%
\pgfpathlineto{\pgfqpoint{2.438032in}{2.626070in}}%
\pgfpathlineto{\pgfqpoint{2.427455in}{2.629655in}}%
\pgfpathlineto{\pgfqpoint{2.416945in}{2.633132in}}%
\pgfpathlineto{\pgfqpoint{2.416878in}{2.633153in}}%
\pgfpathlineto{\pgfqpoint{2.406301in}{2.636536in}}%
\pgfpathlineto{\pgfqpoint{2.395724in}{2.640221in}}%
\pgfpathlineto{\pgfqpoint{2.387595in}{2.643426in}}%
\pgfpathlineto{\pgfqpoint{2.385147in}{2.644383in}}%
\pgfpathlineto{\pgfqpoint{2.374570in}{2.649148in}}%
\pgfpathlineto{\pgfqpoint{2.366224in}{2.653720in}}%
\pgfpathlineto{\pgfqpoint{2.363993in}{2.654952in}}%
\pgfpathlineto{\pgfqpoint{2.353416in}{2.661913in}}%
\pgfpathlineto{\pgfqpoint{2.350736in}{2.664015in}}%
\pgfpathlineto{\pgfqpoint{2.342839in}{2.670283in}}%
\pgfpathlineto{\pgfqpoint{2.338512in}{2.674309in}}%
\pgfpathlineto{\pgfqpoint{2.332262in}{2.680164in}}%
\pgfpathlineto{\pgfqpoint{2.328098in}{2.684604in}}%
\pgfpathlineto{\pgfqpoint{2.321684in}{2.691439in}}%
\pgfpathlineto{\pgfqpoint{2.318762in}{2.694898in}}%
\pgfpathlineto{\pgfqpoint{2.311107in}{2.703897in}}%
\pgfpathlineto{\pgfqpoint{2.310096in}{2.705193in}}%
\pgfpathlineto{\pgfqpoint{2.301944in}{2.715487in}}%
\pgfpathlineto{\pgfqpoint{2.300530in}{2.717271in}}%
\pgfpathlineto{\pgfqpoint{2.294198in}{2.725781in}}%
\pgfpathlineto{\pgfqpoint{2.289953in}{2.731543in}}%
\pgfpathlineto{\pgfqpoint{2.286790in}{2.736076in}}%
\pgfpathlineto{\pgfqpoint{2.279764in}{2.746370in}}%
\pgfpathlineto{\pgfqpoint{2.279376in}{2.746964in}}%
\pgfpathlineto{\pgfqpoint{2.273309in}{2.756665in}}%
\pgfpathlineto{\pgfqpoint{2.268799in}{2.764423in}}%
\pgfpathlineto{\pgfqpoint{2.267383in}{2.766959in}}%
\pgfpathlineto{\pgfqpoint{2.262215in}{2.777254in}}%
\pgfpathlineto{\pgfqpoint{2.258222in}{2.786462in}}%
\pgfpathlineto{\pgfqpoint{2.257769in}{2.787548in}}%
\pgfpathlineto{\pgfqpoint{2.254227in}{2.797842in}}%
\pgfpathlineto{\pgfqpoint{2.251442in}{2.808137in}}%
\pgfpathlineto{\pgfqpoint{2.249389in}{2.818431in}}%
\pgfpathlineto{\pgfqpoint{2.248032in}{2.828726in}}%
\pgfpathlineto{\pgfqpoint{2.247645in}{2.834528in}}%
\pgfpathlineto{\pgfqpoint{2.247355in}{2.839020in}}%
\pgfpathlineto{\pgfqpoint{2.247322in}{2.849315in}}%
\pgfpathlineto{\pgfqpoint{2.247645in}{2.854890in}}%
\pgfpathlineto{\pgfqpoint{2.247935in}{2.859609in}}%
\pgfpathlineto{\pgfqpoint{2.249250in}{2.869903in}}%
\pgfpathlineto{\pgfqpoint{2.251301in}{2.880198in}}%
\pgfpathlineto{\pgfqpoint{2.254163in}{2.890492in}}%
\pgfpathlineto{\pgfqpoint{2.257946in}{2.900787in}}%
\pgfpathlineto{\pgfqpoint{2.258222in}{2.901397in}}%
\pgfpathlineto{\pgfqpoint{2.262968in}{2.911081in}}%
\pgfpathlineto{\pgfqpoint{2.258222in}{2.911081in}}%
\pgfpathlineto{\pgfqpoint{2.247645in}{2.911081in}}%
\pgfpathlineto{\pgfqpoint{2.237068in}{2.911081in}}%
\pgfpathlineto{\pgfqpoint{2.226491in}{2.911081in}}%
\pgfpathlineto{\pgfqpoint{2.224572in}{2.911081in}}%
\pgfpathlineto{\pgfqpoint{2.221804in}{2.900787in}}%
\pgfpathlineto{\pgfqpoint{2.219753in}{2.890492in}}%
\pgfpathlineto{\pgfqpoint{2.218309in}{2.880198in}}%
\pgfpathlineto{\pgfqpoint{2.217398in}{2.869903in}}%
\pgfpathlineto{\pgfqpoint{2.216970in}{2.859609in}}%
\pgfpathlineto{\pgfqpoint{2.216996in}{2.849315in}}%
\pgfpathlineto{\pgfqpoint{2.217465in}{2.839020in}}%
\pgfpathlineto{\pgfqpoint{2.218386in}{2.828726in}}%
\pgfpathlineto{\pgfqpoint{2.219786in}{2.818431in}}%
\pgfpathlineto{\pgfqpoint{2.221712in}{2.808137in}}%
\pgfpathlineto{\pgfqpoint{2.224233in}{2.797842in}}%
\pgfpathlineto{\pgfqpoint{2.226491in}{2.790533in}}%
\pgfpathlineto{\pgfqpoint{2.227443in}{2.787548in}}%
\pgfpathlineto{\pgfqpoint{2.231416in}{2.777254in}}%
\pgfpathlineto{\pgfqpoint{2.236026in}{2.766959in}}%
\pgfpathlineto{\pgfqpoint{2.237068in}{2.764851in}}%
\pgfpathlineto{\pgfqpoint{2.241256in}{2.756665in}}%
\pgfpathlineto{\pgfqpoint{2.246858in}{2.746370in}}%
\pgfpathlineto{\pgfqpoint{2.247645in}{2.744957in}}%
\pgfpathlineto{\pgfqpoint{2.252800in}{2.736076in}}%
\pgfpathlineto{\pgfqpoint{2.258222in}{2.726803in}}%
\pgfpathlineto{\pgfqpoint{2.258849in}{2.725781in}}%
\pgfpathlineto{\pgfqpoint{2.265045in}{2.715487in}}%
\pgfpathlineto{\pgfqpoint{2.268799in}{2.709094in}}%
\pgfpathlineto{\pgfqpoint{2.271226in}{2.705193in}}%
\pgfpathlineto{\pgfqpoint{2.277400in}{2.694898in}}%
\pgfpathlineto{\pgfqpoint{2.279376in}{2.691447in}}%
\pgfpathlineto{\pgfqpoint{2.283583in}{2.684604in}}%
\pgfpathlineto{\pgfqpoint{2.289687in}{2.674309in}}%
\pgfpathlineto{\pgfqpoint{2.289953in}{2.673836in}}%
\pgfpathlineto{\pgfqpoint{2.295995in}{2.664015in}}%
\pgfpathlineto{\pgfqpoint{2.300530in}{2.656485in}}%
\pgfpathlineto{\pgfqpoint{2.302396in}{2.653720in}}%
\pgfpathlineto{\pgfqpoint{2.309254in}{2.643426in}}%
\pgfpathlineto{\pgfqpoint{2.311107in}{2.640648in}}%
\pgfpathlineto{\pgfqpoint{2.316973in}{2.633132in}}%
\pgfpathlineto{\pgfqpoint{2.321684in}{2.627328in}}%
\pgfpathlineto{\pgfqpoint{2.326101in}{2.622837in}}%
\pgfpathlineto{\pgfqpoint{2.332262in}{2.616906in}}%
\pgfpathlineto{\pgfqpoint{2.337941in}{2.612543in}}%
\pgfpathlineto{\pgfqpoint{2.342839in}{2.608988in}}%
\pgfpathlineto{\pgfqpoint{2.353416in}{2.602976in}}%
\pgfpathlineto{\pgfqpoint{2.355016in}{2.602248in}}%
\pgfpathlineto{\pgfqpoint{2.363993in}{2.598313in}}%
\pgfpathlineto{\pgfqpoint{2.374570in}{2.594604in}}%
\pgfpathlineto{\pgfqpoint{2.383613in}{2.591954in}}%
\pgfpathlineto{\pgfqpoint{2.385147in}{2.591512in}}%
\pgfpathlineto{\pgfqpoint{2.395724in}{2.588701in}}%
\pgfpathlineto{\pgfqpoint{2.406301in}{2.585979in}}%
\pgfpathlineto{\pgfqpoint{2.416878in}{2.583132in}}%
\pgfpathlineto{\pgfqpoint{2.421760in}{2.581659in}}%
\pgfpathlineto{\pgfqpoint{2.427455in}{2.579918in}}%
\pgfpathlineto{\pgfqpoint{2.438032in}{2.576138in}}%
\pgfpathlineto{\pgfqpoint{2.448609in}{2.571653in}}%
\pgfpathlineto{\pgfqpoint{2.449177in}{2.571365in}}%
\pgfpathlineto{\pgfqpoint{2.459186in}{2.566182in}}%
\pgfpathlineto{\pgfqpoint{2.467460in}{2.561071in}}%
\pgfpathlineto{\pgfqpoint{2.469763in}{2.559626in}}%
\pgfpathlineto{\pgfqpoint{2.480340in}{2.551799in}}%
\pgfpathlineto{\pgfqpoint{2.481528in}{2.550776in}}%
\pgfpathlineto{\pgfqpoint{2.490917in}{2.542659in}}%
\pgfpathlineto{\pgfqpoint{2.493118in}{2.540482in}}%
\pgfpathlineto{\pgfqpoint{2.501495in}{2.532273in}}%
\pgfpathlineto{\pgfqpoint{2.503392in}{2.530187in}}%
\pgfpathlineto{\pgfqpoint{2.512072in}{2.520861in}}%
\pgfpathlineto{\pgfqpoint{2.512892in}{2.519893in}}%
\pgfpathlineto{\pgfqpoint{2.521913in}{2.509598in}}%
\pgfpathlineto{\pgfqpoint{2.522649in}{2.508788in}}%
\pgfpathlineto{\pgfqpoint{2.530703in}{2.499304in}}%
\pgfpathlineto{\pgfqpoint{2.533226in}{2.496458in}}%
\pgfpathlineto{\pgfqpoint{2.539532in}{2.489009in}}%
\pgfpathlineto{\pgfqpoint{2.543803in}{2.484188in}}%
\pgfpathlineto{\pgfqpoint{2.548532in}{2.478715in}}%
\pgfpathlineto{\pgfqpoint{2.554380in}{2.472233in}}%
\pgfpathlineto{\pgfqpoint{2.557808in}{2.468421in}}%
\pgfpathlineto{\pgfqpoint{2.564957in}{2.460760in}}%
\pgfpathlineto{\pgfqpoint{2.567458in}{2.458126in}}%
\pgfpathlineto{\pgfqpoint{2.575534in}{2.449856in}}%
\pgfpathlineto{\pgfqpoint{2.577588in}{2.447832in}}%
\pgfpathlineto{\pgfqpoint{2.586111in}{2.439558in}}%
\pgfpathlineto{\pgfqpoint{2.588317in}{2.437537in}}%
\pgfpathlineto{\pgfqpoint{2.596688in}{2.429879in}}%
\pgfpathlineto{\pgfqpoint{2.599799in}{2.427243in}}%
\pgfpathlineto{\pgfqpoint{2.607265in}{2.420829in}}%
\pgfpathlineto{\pgfqpoint{2.612233in}{2.416948in}}%
\pgfpathlineto{\pgfqpoint{2.617842in}{2.412437in}}%
\pgfpathlineto{\pgfqpoint{2.625898in}{2.406654in}}%
\pgfpathlineto{\pgfqpoint{2.628419in}{2.404762in}}%
\pgfpathlineto{\pgfqpoint{2.638996in}{2.397808in}}%
\pgfpathlineto{\pgfqpoint{2.641593in}{2.396360in}}%
\pgfpathlineto{\pgfqpoint{2.649573in}{2.391611in}}%
\pgfpathlineto{\pgfqpoint{2.660150in}{2.386448in}}%
\pgfpathlineto{\pgfqpoint{2.661168in}{2.386065in}}%
\pgfpathclose%
\pgfusepath{fill}%
\end{pgfscope}%
\begin{pgfscope}%
\pgfpathrectangle{\pgfqpoint{1.856795in}{1.819814in}}{\pgfqpoint{1.194205in}{1.163386in}}%
\pgfusepath{clip}%
\pgfsetbuttcap%
\pgfsetroundjoin%
\definecolor{currentfill}{rgb}{0.963190,0.619109,0.458249}%
\pgfsetfillcolor{currentfill}%
\pgfsetlinewidth{0.000000pt}%
\definecolor{currentstroke}{rgb}{0.000000,0.000000,0.000000}%
\pgfsetstrokecolor{currentstroke}%
\pgfsetdash{}{0pt}%
\pgfpathmoveto{\pgfqpoint{2.279376in}{1.891932in}}%
\pgfpathlineto{\pgfqpoint{2.289953in}{1.891932in}}%
\pgfpathlineto{\pgfqpoint{2.300530in}{1.891932in}}%
\pgfpathlineto{\pgfqpoint{2.311107in}{1.891932in}}%
\pgfpathlineto{\pgfqpoint{2.316673in}{1.891932in}}%
\pgfpathlineto{\pgfqpoint{2.315873in}{1.902227in}}%
\pgfpathlineto{\pgfqpoint{2.314617in}{1.912521in}}%
\pgfpathlineto{\pgfqpoint{2.312964in}{1.922816in}}%
\pgfpathlineto{\pgfqpoint{2.311107in}{1.932342in}}%
\pgfpathlineto{\pgfqpoint{2.310958in}{1.933110in}}%
\pgfpathlineto{\pgfqpoint{2.308613in}{1.943405in}}%
\pgfpathlineto{\pgfqpoint{2.305960in}{1.953699in}}%
\pgfpathlineto{\pgfqpoint{2.303013in}{1.963993in}}%
\pgfpathlineto{\pgfqpoint{2.300530in}{1.971919in}}%
\pgfpathlineto{\pgfqpoint{2.299773in}{1.974288in}}%
\pgfpathlineto{\pgfqpoint{2.296211in}{1.984582in}}%
\pgfpathlineto{\pgfqpoint{2.291478in}{1.994877in}}%
\pgfpathlineto{\pgfqpoint{2.289953in}{1.996683in}}%
\pgfpathlineto{\pgfqpoint{2.281905in}{2.005171in}}%
\pgfpathlineto{\pgfqpoint{2.279376in}{2.007823in}}%
\pgfpathlineto{\pgfqpoint{2.271416in}{2.015466in}}%
\pgfpathlineto{\pgfqpoint{2.268799in}{2.018000in}}%
\pgfpathlineto{\pgfqpoint{2.260092in}{2.025760in}}%
\pgfpathlineto{\pgfqpoint{2.258222in}{2.027453in}}%
\pgfpathlineto{\pgfqpoint{2.248000in}{2.036054in}}%
\pgfpathlineto{\pgfqpoint{2.247645in}{2.036361in}}%
\pgfpathlineto{\pgfqpoint{2.237068in}{2.045039in}}%
\pgfpathlineto{\pgfqpoint{2.235416in}{2.046349in}}%
\pgfpathlineto{\pgfqpoint{2.226491in}{2.053640in}}%
\pgfpathlineto{\pgfqpoint{2.222776in}{2.056643in}}%
\pgfpathlineto{\pgfqpoint{2.215914in}{2.062374in}}%
\pgfpathlineto{\pgfqpoint{2.210546in}{2.066938in}}%
\pgfpathlineto{\pgfqpoint{2.205337in}{2.071509in}}%
\pgfpathlineto{\pgfqpoint{2.199104in}{2.077232in}}%
\pgfpathlineto{\pgfqpoint{2.194760in}{2.081329in}}%
\pgfpathlineto{\pgfqpoint{2.188621in}{2.087527in}}%
\pgfpathlineto{\pgfqpoint{2.184183in}{2.092079in}}%
\pgfpathlineto{\pgfqpoint{2.179041in}{2.097821in}}%
\pgfpathlineto{\pgfqpoint{2.173606in}{2.103893in}}%
\pgfpathlineto{\pgfqpoint{2.170167in}{2.108116in}}%
\pgfpathlineto{\pgfqpoint{2.163029in}{2.116720in}}%
\pgfpathlineto{\pgfqpoint{2.161757in}{2.118410in}}%
\pgfpathlineto{\pgfqpoint{2.153808in}{2.128704in}}%
\pgfpathlineto{\pgfqpoint{2.152451in}{2.130401in}}%
\pgfpathlineto{\pgfqpoint{2.146220in}{2.138999in}}%
\pgfpathlineto{\pgfqpoint{2.141874in}{2.144701in}}%
\pgfpathlineto{\pgfqpoint{2.138698in}{2.149293in}}%
\pgfpathlineto{\pgfqpoint{2.131297in}{2.159496in}}%
\pgfpathlineto{\pgfqpoint{2.131237in}{2.159588in}}%
\pgfpathlineto{\pgfqpoint{2.124377in}{2.169882in}}%
\pgfpathlineto{\pgfqpoint{2.120720in}{2.175188in}}%
\pgfpathlineto{\pgfqpoint{2.117622in}{2.180177in}}%
\pgfpathlineto{\pgfqpoint{2.111117in}{2.190471in}}%
\pgfpathlineto{\pgfqpoint{2.110143in}{2.192015in}}%
\pgfpathlineto{\pgfqpoint{2.105207in}{2.200765in}}%
\pgfpathlineto{\pgfqpoint{2.099566in}{2.210632in}}%
\pgfpathlineto{\pgfqpoint{2.099347in}{2.211060in}}%
\pgfpathlineto{\pgfqpoint{2.094195in}{2.221354in}}%
\pgfpathlineto{\pgfqpoint{2.088999in}{2.231649in}}%
\pgfpathlineto{\pgfqpoint{2.088989in}{2.231669in}}%
\pgfpathlineto{\pgfqpoint{2.084472in}{2.241943in}}%
\pgfpathlineto{\pgfqpoint{2.079925in}{2.252238in}}%
\pgfpathlineto{\pgfqpoint{2.078412in}{2.255690in}}%
\pgfpathlineto{\pgfqpoint{2.075741in}{2.262532in}}%
\pgfpathlineto{\pgfqpoint{2.071720in}{2.272826in}}%
\pgfpathlineto{\pgfqpoint{2.067835in}{2.282634in}}%
\pgfpathlineto{\pgfqpoint{2.067662in}{2.283121in}}%
\pgfpathlineto{\pgfqpoint{2.064025in}{2.293415in}}%
\pgfpathlineto{\pgfqpoint{2.060301in}{2.303710in}}%
\pgfpathlineto{\pgfqpoint{2.057258in}{2.311905in}}%
\pgfpathlineto{\pgfqpoint{2.056556in}{2.314004in}}%
\pgfpathlineto{\pgfqpoint{2.053061in}{2.324299in}}%
\pgfpathlineto{\pgfqpoint{2.049431in}{2.334593in}}%
\pgfpathlineto{\pgfqpoint{2.046681in}{2.342115in}}%
\pgfpathlineto{\pgfqpoint{2.045759in}{2.344887in}}%
\pgfpathlineto{\pgfqpoint{2.042249in}{2.355182in}}%
\pgfpathlineto{\pgfqpoint{2.038574in}{2.365476in}}%
\pgfpathlineto{\pgfqpoint{2.036104in}{2.372146in}}%
\pgfpathlineto{\pgfqpoint{2.034866in}{2.375771in}}%
\pgfpathlineto{\pgfqpoint{2.031253in}{2.386065in}}%
\pgfpathlineto{\pgfqpoint{2.027471in}{2.396360in}}%
\pgfpathlineto{\pgfqpoint{2.025527in}{2.401502in}}%
\pgfpathlineto{\pgfqpoint{2.023703in}{2.406654in}}%
\pgfpathlineto{\pgfqpoint{2.019958in}{2.416948in}}%
\pgfpathlineto{\pgfqpoint{2.016062in}{2.427243in}}%
\pgfpathlineto{\pgfqpoint{2.014950in}{2.430141in}}%
\pgfpathlineto{\pgfqpoint{2.012251in}{2.437537in}}%
\pgfpathlineto{\pgfqpoint{2.008406in}{2.447832in}}%
\pgfpathlineto{\pgfqpoint{2.004451in}{2.458126in}}%
\pgfpathlineto{\pgfqpoint{2.004373in}{2.458331in}}%
\pgfpathlineto{\pgfqpoint{2.000692in}{2.468421in}}%
\pgfpathlineto{\pgfqpoint{1.996892in}{2.478715in}}%
\pgfpathlineto{\pgfqpoint{1.993796in}{2.487089in}}%
\pgfpathlineto{\pgfqpoint{1.993120in}{2.489009in}}%
\pgfpathlineto{\pgfqpoint{1.989585in}{2.499304in}}%
\pgfpathlineto{\pgfqpoint{1.986105in}{2.509598in}}%
\pgfpathlineto{\pgfqpoint{1.983218in}{2.518346in}}%
\pgfpathlineto{\pgfqpoint{1.982738in}{2.519893in}}%
\pgfpathlineto{\pgfqpoint{1.979677in}{2.530187in}}%
\pgfpathlineto{\pgfqpoint{1.976733in}{2.540482in}}%
\pgfpathlineto{\pgfqpoint{1.973910in}{2.550776in}}%
\pgfpathlineto{\pgfqpoint{1.972641in}{2.555614in}}%
\pgfpathlineto{\pgfqpoint{1.971308in}{2.561071in}}%
\pgfpathlineto{\pgfqpoint{1.968906in}{2.571365in}}%
\pgfpathlineto{\pgfqpoint{1.966600in}{2.581659in}}%
\pgfpathlineto{\pgfqpoint{1.964361in}{2.591954in}}%
\pgfpathlineto{\pgfqpoint{1.962147in}{2.602248in}}%
\pgfpathlineto{\pgfqpoint{1.962064in}{2.602627in}}%
\pgfpathlineto{\pgfqpoint{1.960045in}{2.612543in}}%
\pgfpathlineto{\pgfqpoint{1.957736in}{2.622837in}}%
\pgfpathlineto{\pgfqpoint{1.955003in}{2.633132in}}%
\pgfpathlineto{\pgfqpoint{1.951941in}{2.643426in}}%
\pgfpathlineto{\pgfqpoint{1.951487in}{2.644912in}}%
\pgfpathlineto{\pgfqpoint{1.948857in}{2.653720in}}%
\pgfpathlineto{\pgfqpoint{1.945710in}{2.664015in}}%
\pgfpathlineto{\pgfqpoint{1.942498in}{2.674309in}}%
\pgfpathlineto{\pgfqpoint{1.940910in}{2.679382in}}%
\pgfpathlineto{\pgfqpoint{1.939303in}{2.684604in}}%
\pgfpathlineto{\pgfqpoint{1.936129in}{2.694898in}}%
\pgfpathlineto{\pgfqpoint{1.932899in}{2.705193in}}%
\pgfpathlineto{\pgfqpoint{1.930333in}{2.713258in}}%
\pgfpathlineto{\pgfqpoint{1.930333in}{2.705193in}}%
\pgfpathlineto{\pgfqpoint{1.930333in}{2.694898in}}%
\pgfpathlineto{\pgfqpoint{1.930333in}{2.684604in}}%
\pgfpathlineto{\pgfqpoint{1.930333in}{2.674309in}}%
\pgfpathlineto{\pgfqpoint{1.930333in}{2.664015in}}%
\pgfpathlineto{\pgfqpoint{1.930333in}{2.660242in}}%
\pgfpathlineto{\pgfqpoint{1.932522in}{2.653720in}}%
\pgfpathlineto{\pgfqpoint{1.935952in}{2.643426in}}%
\pgfpathlineto{\pgfqpoint{1.939301in}{2.633132in}}%
\pgfpathlineto{\pgfqpoint{1.940910in}{2.627975in}}%
\pgfpathlineto{\pgfqpoint{1.942491in}{2.622837in}}%
\pgfpathlineto{\pgfqpoint{1.944940in}{2.612543in}}%
\pgfpathlineto{\pgfqpoint{1.946884in}{2.602248in}}%
\pgfpathlineto{\pgfqpoint{1.948794in}{2.591954in}}%
\pgfpathlineto{\pgfqpoint{1.950750in}{2.581659in}}%
\pgfpathlineto{\pgfqpoint{1.951487in}{2.577933in}}%
\pgfpathlineto{\pgfqpoint{1.952863in}{2.571365in}}%
\pgfpathlineto{\pgfqpoint{1.955124in}{2.561071in}}%
\pgfpathlineto{\pgfqpoint{1.957509in}{2.550776in}}%
\pgfpathlineto{\pgfqpoint{1.960035in}{2.540482in}}%
\pgfpathlineto{\pgfqpoint{1.962064in}{2.532678in}}%
\pgfpathlineto{\pgfqpoint{1.962748in}{2.530187in}}%
\pgfpathlineto{\pgfqpoint{1.965734in}{2.519893in}}%
\pgfpathlineto{\pgfqpoint{1.968866in}{2.509598in}}%
\pgfpathlineto{\pgfqpoint{1.972135in}{2.499304in}}%
\pgfpathlineto{\pgfqpoint{1.972641in}{2.497783in}}%
\pgfpathlineto{\pgfqpoint{1.975688in}{2.489009in}}%
\pgfpathlineto{\pgfqpoint{1.979355in}{2.478715in}}%
\pgfpathlineto{\pgfqpoint{1.983080in}{2.468421in}}%
\pgfpathlineto{\pgfqpoint{1.983218in}{2.468048in}}%
\pgfpathlineto{\pgfqpoint{1.987028in}{2.458126in}}%
\pgfpathlineto{\pgfqpoint{1.990955in}{2.447832in}}%
\pgfpathlineto{\pgfqpoint{1.993796in}{2.440328in}}%
\pgfpathlineto{\pgfqpoint{1.994884in}{2.437537in}}%
\pgfpathlineto{\pgfqpoint{1.998877in}{2.427243in}}%
\pgfpathlineto{\pgfqpoint{2.002742in}{2.416948in}}%
\pgfpathlineto{\pgfqpoint{2.004373in}{2.412514in}}%
\pgfpathlineto{\pgfqpoint{2.006606in}{2.406654in}}%
\pgfpathlineto{\pgfqpoint{2.010415in}{2.396360in}}%
\pgfpathlineto{\pgfqpoint{2.014052in}{2.386065in}}%
\pgfpathlineto{\pgfqpoint{2.014950in}{2.383454in}}%
\pgfpathlineto{\pgfqpoint{2.017736in}{2.375771in}}%
\pgfpathlineto{\pgfqpoint{2.021318in}{2.365476in}}%
\pgfpathlineto{\pgfqpoint{2.024730in}{2.355182in}}%
\pgfpathlineto{\pgfqpoint{2.025527in}{2.352705in}}%
\pgfpathlineto{\pgfqpoint{2.028227in}{2.344887in}}%
\pgfpathlineto{\pgfqpoint{2.031643in}{2.334593in}}%
\pgfpathlineto{\pgfqpoint{2.034917in}{2.324299in}}%
\pgfpathlineto{\pgfqpoint{2.036104in}{2.320471in}}%
\pgfpathlineto{\pgfqpoint{2.038283in}{2.314004in}}%
\pgfpathlineto{\pgfqpoint{2.041660in}{2.303710in}}%
\pgfpathlineto{\pgfqpoint{2.044944in}{2.293415in}}%
\pgfpathlineto{\pgfqpoint{2.046681in}{2.287903in}}%
\pgfpathlineto{\pgfqpoint{2.048332in}{2.283121in}}%
\pgfpathlineto{\pgfqpoint{2.051861in}{2.272826in}}%
\pgfpathlineto{\pgfqpoint{2.055359in}{2.262532in}}%
\pgfpathlineto{\pgfqpoint{2.057258in}{2.256980in}}%
\pgfpathlineto{\pgfqpoint{2.059043in}{2.252238in}}%
\pgfpathlineto{\pgfqpoint{2.062958in}{2.241943in}}%
\pgfpathlineto{\pgfqpoint{2.066901in}{2.231649in}}%
\pgfpathlineto{\pgfqpoint{2.067835in}{2.229271in}}%
\pgfpathlineto{\pgfqpoint{2.071255in}{2.221354in}}%
\pgfpathlineto{\pgfqpoint{2.075762in}{2.211060in}}%
\pgfpathlineto{\pgfqpoint{2.078412in}{2.205125in}}%
\pgfpathlineto{\pgfqpoint{2.080550in}{2.200765in}}%
\pgfpathlineto{\pgfqpoint{2.085684in}{2.190471in}}%
\pgfpathlineto{\pgfqpoint{2.088989in}{2.183914in}}%
\pgfpathlineto{\pgfqpoint{2.091049in}{2.180177in}}%
\pgfpathlineto{\pgfqpoint{2.096764in}{2.169882in}}%
\pgfpathlineto{\pgfqpoint{2.099566in}{2.164828in}}%
\pgfpathlineto{\pgfqpoint{2.102723in}{2.159588in}}%
\pgfpathlineto{\pgfqpoint{2.108817in}{2.149293in}}%
\pgfpathlineto{\pgfqpoint{2.110143in}{2.147023in}}%
\pgfpathlineto{\pgfqpoint{2.115208in}{2.138999in}}%
\pgfpathlineto{\pgfqpoint{2.120720in}{2.129842in}}%
\pgfpathlineto{\pgfqpoint{2.121464in}{2.128704in}}%
\pgfpathlineto{\pgfqpoint{2.127992in}{2.118410in}}%
\pgfpathlineto{\pgfqpoint{2.131297in}{2.112936in}}%
\pgfpathlineto{\pgfqpoint{2.134484in}{2.108116in}}%
\pgfpathlineto{\pgfqpoint{2.141058in}{2.097821in}}%
\pgfpathlineto{\pgfqpoint{2.141874in}{2.096532in}}%
\pgfpathlineto{\pgfqpoint{2.148181in}{2.087527in}}%
\pgfpathlineto{\pgfqpoint{2.152451in}{2.081420in}}%
\pgfpathlineto{\pgfqpoint{2.155692in}{2.077232in}}%
\pgfpathlineto{\pgfqpoint{2.163029in}{2.067915in}}%
\pgfpathlineto{\pgfqpoint{2.163876in}{2.066938in}}%
\pgfpathlineto{\pgfqpoint{2.173084in}{2.056643in}}%
\pgfpathlineto{\pgfqpoint{2.173606in}{2.056082in}}%
\pgfpathlineto{\pgfqpoint{2.183415in}{2.046349in}}%
\pgfpathlineto{\pgfqpoint{2.184183in}{2.045612in}}%
\pgfpathlineto{\pgfqpoint{2.194760in}{2.036125in}}%
\pgfpathlineto{\pgfqpoint{2.194842in}{2.036054in}}%
\pgfpathlineto{\pgfqpoint{2.205337in}{2.027392in}}%
\pgfpathlineto{\pgfqpoint{2.207371in}{2.025760in}}%
\pgfpathlineto{\pgfqpoint{2.215914in}{2.019042in}}%
\pgfpathlineto{\pgfqpoint{2.220489in}{2.015466in}}%
\pgfpathlineto{\pgfqpoint{2.226491in}{2.010791in}}%
\pgfpathlineto{\pgfqpoint{2.233619in}{2.005171in}}%
\pgfpathlineto{\pgfqpoint{2.237068in}{2.002326in}}%
\pgfpathlineto{\pgfqpoint{2.245922in}{1.994877in}}%
\pgfpathlineto{\pgfqpoint{2.247645in}{1.992914in}}%
\pgfpathlineto{\pgfqpoint{2.254459in}{1.984582in}}%
\pgfpathlineto{\pgfqpoint{2.258222in}{1.976853in}}%
\pgfpathlineto{\pgfqpoint{2.259397in}{1.974288in}}%
\pgfpathlineto{\pgfqpoint{2.263408in}{1.963993in}}%
\pgfpathlineto{\pgfqpoint{2.266860in}{1.953699in}}%
\pgfpathlineto{\pgfqpoint{2.268799in}{1.946924in}}%
\pgfpathlineto{\pgfqpoint{2.269798in}{1.943405in}}%
\pgfpathlineto{\pgfqpoint{2.272228in}{1.933110in}}%
\pgfpathlineto{\pgfqpoint{2.274129in}{1.922816in}}%
\pgfpathlineto{\pgfqpoint{2.275462in}{1.912521in}}%
\pgfpathlineto{\pgfqpoint{2.276171in}{1.902227in}}%
\pgfpathlineto{\pgfqpoint{2.276175in}{1.891932in}}%
\pgfpathclose%
\pgfusepath{fill}%
\end{pgfscope}%
\begin{pgfscope}%
\pgfpathrectangle{\pgfqpoint{1.856795in}{1.819814in}}{\pgfqpoint{1.194205in}{1.163386in}}%
\pgfusepath{clip}%
\pgfsetbuttcap%
\pgfsetroundjoin%
\definecolor{currentfill}{rgb}{0.963190,0.619109,0.458249}%
\pgfsetfillcolor{currentfill}%
\pgfsetlinewidth{0.000000pt}%
\definecolor{currentstroke}{rgb}{0.000000,0.000000,0.000000}%
\pgfsetstrokecolor{currentstroke}%
\pgfsetdash{}{0pt}%
\pgfpathmoveto{\pgfqpoint{2.691882in}{2.302340in}}%
\pgfpathlineto{\pgfqpoint{2.702459in}{2.299529in}}%
\pgfpathlineto{\pgfqpoint{2.713036in}{2.299276in}}%
\pgfpathlineto{\pgfqpoint{2.723613in}{2.301712in}}%
\pgfpathlineto{\pgfqpoint{2.727708in}{2.303710in}}%
\pgfpathlineto{\pgfqpoint{2.734190in}{2.306743in}}%
\pgfpathlineto{\pgfqpoint{2.744301in}{2.314004in}}%
\pgfpathlineto{\pgfqpoint{2.744767in}{2.314323in}}%
\pgfpathlineto{\pgfqpoint{2.755344in}{2.323945in}}%
\pgfpathlineto{\pgfqpoint{2.755659in}{2.324299in}}%
\pgfpathlineto{\pgfqpoint{2.765348in}{2.334593in}}%
\pgfpathlineto{\pgfqpoint{2.765921in}{2.335166in}}%
\pgfpathlineto{\pgfqpoint{2.774209in}{2.344887in}}%
\pgfpathlineto{\pgfqpoint{2.776498in}{2.347409in}}%
\pgfpathlineto{\pgfqpoint{2.782734in}{2.355182in}}%
\pgfpathlineto{\pgfqpoint{2.787075in}{2.360259in}}%
\pgfpathlineto{\pgfqpoint{2.791131in}{2.365476in}}%
\pgfpathlineto{\pgfqpoint{2.797652in}{2.373349in}}%
\pgfpathlineto{\pgfqpoint{2.799520in}{2.375771in}}%
\pgfpathlineto{\pgfqpoint{2.807975in}{2.386065in}}%
\pgfpathlineto{\pgfqpoint{2.808229in}{2.386355in}}%
\pgfpathlineto{\pgfqpoint{2.816558in}{2.396360in}}%
\pgfpathlineto{\pgfqpoint{2.818806in}{2.398904in}}%
\pgfpathlineto{\pgfqpoint{2.825436in}{2.406654in}}%
\pgfpathlineto{\pgfqpoint{2.829383in}{2.411009in}}%
\pgfpathlineto{\pgfqpoint{2.834689in}{2.416948in}}%
\pgfpathlineto{\pgfqpoint{2.839960in}{2.422523in}}%
\pgfpathlineto{\pgfqpoint{2.844435in}{2.427243in}}%
\pgfpathlineto{\pgfqpoint{2.850538in}{2.433331in}}%
\pgfpathlineto{\pgfqpoint{2.854838in}{2.437537in}}%
\pgfpathlineto{\pgfqpoint{2.861115in}{2.443352in}}%
\pgfpathlineto{\pgfqpoint{2.866132in}{2.447832in}}%
\pgfpathlineto{\pgfqpoint{2.871692in}{2.452541in}}%
\pgfpathlineto{\pgfqpoint{2.878652in}{2.458126in}}%
\pgfpathlineto{\pgfqpoint{2.882269in}{2.460890in}}%
\pgfpathlineto{\pgfqpoint{2.892843in}{2.468421in}}%
\pgfpathlineto{\pgfqpoint{2.892846in}{2.468423in}}%
\pgfpathlineto{\pgfqpoint{2.903423in}{2.474996in}}%
\pgfpathlineto{\pgfqpoint{2.910052in}{2.478715in}}%
\pgfpathlineto{\pgfqpoint{2.914000in}{2.480893in}}%
\pgfpathlineto{\pgfqpoint{2.924577in}{2.486152in}}%
\pgfpathlineto{\pgfqpoint{2.930765in}{2.489009in}}%
\pgfpathlineto{\pgfqpoint{2.935154in}{2.491197in}}%
\pgfpathlineto{\pgfqpoint{2.945731in}{2.497220in}}%
\pgfpathlineto{\pgfqpoint{2.948202in}{2.499304in}}%
\pgfpathlineto{\pgfqpoint{2.955984in}{2.509598in}}%
\pgfpathlineto{\pgfqpoint{2.956308in}{2.510298in}}%
\pgfpathlineto{\pgfqpoint{2.959330in}{2.519893in}}%
\pgfpathlineto{\pgfqpoint{2.961410in}{2.530187in}}%
\pgfpathlineto{\pgfqpoint{2.962779in}{2.540482in}}%
\pgfpathlineto{\pgfqpoint{2.963698in}{2.550776in}}%
\pgfpathlineto{\pgfqpoint{2.964328in}{2.561071in}}%
\pgfpathlineto{\pgfqpoint{2.964785in}{2.571365in}}%
\pgfpathlineto{\pgfqpoint{2.965160in}{2.581659in}}%
\pgfpathlineto{\pgfqpoint{2.965533in}{2.591954in}}%
\pgfpathlineto{\pgfqpoint{2.965977in}{2.602248in}}%
\pgfpathlineto{\pgfqpoint{2.966560in}{2.612543in}}%
\pgfpathlineto{\pgfqpoint{2.966885in}{2.616736in}}%
\pgfpathlineto{\pgfqpoint{2.967350in}{2.622837in}}%
\pgfpathlineto{\pgfqpoint{2.968411in}{2.633132in}}%
\pgfpathlineto{\pgfqpoint{2.969803in}{2.643426in}}%
\pgfpathlineto{\pgfqpoint{2.971581in}{2.653720in}}%
\pgfpathlineto{\pgfqpoint{2.973801in}{2.664015in}}%
\pgfpathlineto{\pgfqpoint{2.976510in}{2.674309in}}%
\pgfpathlineto{\pgfqpoint{2.977462in}{2.677306in}}%
\pgfpathlineto{\pgfqpoint{2.977462in}{2.684604in}}%
\pgfpathlineto{\pgfqpoint{2.977462in}{2.694898in}}%
\pgfpathlineto{\pgfqpoint{2.977462in}{2.705193in}}%
\pgfpathlineto{\pgfqpoint{2.977462in}{2.715487in}}%
\pgfpathlineto{\pgfqpoint{2.977462in}{2.721811in}}%
\pgfpathlineto{\pgfqpoint{2.973843in}{2.715487in}}%
\pgfpathlineto{\pgfqpoint{2.968614in}{2.705193in}}%
\pgfpathlineto{\pgfqpoint{2.966885in}{2.701309in}}%
\pgfpathlineto{\pgfqpoint{2.963990in}{2.694898in}}%
\pgfpathlineto{\pgfqpoint{2.959961in}{2.684604in}}%
\pgfpathlineto{\pgfqpoint{2.956499in}{2.674309in}}%
\pgfpathlineto{\pgfqpoint{2.956308in}{2.673635in}}%
\pgfpathlineto{\pgfqpoint{2.953546in}{2.664015in}}%
\pgfpathlineto{\pgfqpoint{2.951068in}{2.653720in}}%
\pgfpathlineto{\pgfqpoint{2.949004in}{2.643426in}}%
\pgfpathlineto{\pgfqpoint{2.947289in}{2.633132in}}%
\pgfpathlineto{\pgfqpoint{2.945853in}{2.622837in}}%
\pgfpathlineto{\pgfqpoint{2.945731in}{2.621802in}}%
\pgfpathlineto{\pgfqpoint{2.944620in}{2.612543in}}%
\pgfpathlineto{\pgfqpoint{2.943508in}{2.602248in}}%
\pgfpathlineto{\pgfqpoint{2.942421in}{2.591954in}}%
\pgfpathlineto{\pgfqpoint{2.941242in}{2.581659in}}%
\pgfpathlineto{\pgfqpoint{2.939824in}{2.571365in}}%
\pgfpathlineto{\pgfqpoint{2.937953in}{2.561071in}}%
\pgfpathlineto{\pgfqpoint{2.935297in}{2.550776in}}%
\pgfpathlineto{\pgfqpoint{2.935154in}{2.550364in}}%
\pgfpathlineto{\pgfqpoint{2.930953in}{2.540482in}}%
\pgfpathlineto{\pgfqpoint{2.924577in}{2.531115in}}%
\pgfpathlineto{\pgfqpoint{2.923772in}{2.530187in}}%
\pgfpathlineto{\pgfqpoint{2.914000in}{2.521549in}}%
\pgfpathlineto{\pgfqpoint{2.911803in}{2.519893in}}%
\pgfpathlineto{\pgfqpoint{2.903423in}{2.514207in}}%
\pgfpathlineto{\pgfqpoint{2.896330in}{2.509598in}}%
\pgfpathlineto{\pgfqpoint{2.892846in}{2.507383in}}%
\pgfpathlineto{\pgfqpoint{2.882269in}{2.500479in}}%
\pgfpathlineto{\pgfqpoint{2.880528in}{2.499304in}}%
\pgfpathlineto{\pgfqpoint{2.871692in}{2.493171in}}%
\pgfpathlineto{\pgfqpoint{2.865890in}{2.489009in}}%
\pgfpathlineto{\pgfqpoint{2.861115in}{2.485445in}}%
\pgfpathlineto{\pgfqpoint{2.852346in}{2.478715in}}%
\pgfpathlineto{\pgfqpoint{2.850538in}{2.477262in}}%
\pgfpathlineto{\pgfqpoint{2.839960in}{2.468615in}}%
\pgfpathlineto{\pgfqpoint{2.839724in}{2.468421in}}%
\pgfpathlineto{\pgfqpoint{2.829383in}{2.459506in}}%
\pgfpathlineto{\pgfqpoint{2.827766in}{2.458126in}}%
\pgfpathlineto{\pgfqpoint{2.818806in}{2.450114in}}%
\pgfpathlineto{\pgfqpoint{2.816184in}{2.447832in}}%
\pgfpathlineto{\pgfqpoint{2.808229in}{2.440571in}}%
\pgfpathlineto{\pgfqpoint{2.804756in}{2.437537in}}%
\pgfpathlineto{\pgfqpoint{2.797652in}{2.431032in}}%
\pgfpathlineto{\pgfqpoint{2.793242in}{2.427243in}}%
\pgfpathlineto{\pgfqpoint{2.787075in}{2.421680in}}%
\pgfpathlineto{\pgfqpoint{2.781360in}{2.416948in}}%
\pgfpathlineto{\pgfqpoint{2.776498in}{2.412716in}}%
\pgfpathlineto{\pgfqpoint{2.768713in}{2.406654in}}%
\pgfpathlineto{\pgfqpoint{2.765921in}{2.404364in}}%
\pgfpathlineto{\pgfqpoint{2.755344in}{2.396832in}}%
\pgfpathlineto{\pgfqpoint{2.754555in}{2.396360in}}%
\pgfpathlineto{\pgfqpoint{2.744767in}{2.390170in}}%
\pgfpathlineto{\pgfqpoint{2.736523in}{2.386065in}}%
\pgfpathlineto{\pgfqpoint{2.734190in}{2.384834in}}%
\pgfpathlineto{\pgfqpoint{2.723613in}{2.380720in}}%
\pgfpathlineto{\pgfqpoint{2.713036in}{2.378144in}}%
\pgfpathlineto{\pgfqpoint{2.702459in}{2.377055in}}%
\pgfpathlineto{\pgfqpoint{2.691882in}{2.377399in}}%
\pgfpathlineto{\pgfqpoint{2.681305in}{2.379119in}}%
\pgfpathlineto{\pgfqpoint{2.670727in}{2.382162in}}%
\pgfpathlineto{\pgfqpoint{2.661168in}{2.386065in}}%
\pgfpathlineto{\pgfqpoint{2.660150in}{2.386448in}}%
\pgfpathlineto{\pgfqpoint{2.649573in}{2.391611in}}%
\pgfpathlineto{\pgfqpoint{2.641593in}{2.396360in}}%
\pgfpathlineto{\pgfqpoint{2.638996in}{2.397808in}}%
\pgfpathlineto{\pgfqpoint{2.628419in}{2.404762in}}%
\pgfpathlineto{\pgfqpoint{2.625898in}{2.406654in}}%
\pgfpathlineto{\pgfqpoint{2.617842in}{2.412437in}}%
\pgfpathlineto{\pgfqpoint{2.612233in}{2.416948in}}%
\pgfpathlineto{\pgfqpoint{2.607265in}{2.420829in}}%
\pgfpathlineto{\pgfqpoint{2.599799in}{2.427243in}}%
\pgfpathlineto{\pgfqpoint{2.596688in}{2.429879in}}%
\pgfpathlineto{\pgfqpoint{2.588317in}{2.437537in}}%
\pgfpathlineto{\pgfqpoint{2.586111in}{2.439558in}}%
\pgfpathlineto{\pgfqpoint{2.577588in}{2.447832in}}%
\pgfpathlineto{\pgfqpoint{2.575534in}{2.449856in}}%
\pgfpathlineto{\pgfqpoint{2.567458in}{2.458126in}}%
\pgfpathlineto{\pgfqpoint{2.564957in}{2.460760in}}%
\pgfpathlineto{\pgfqpoint{2.557808in}{2.468421in}}%
\pgfpathlineto{\pgfqpoint{2.554380in}{2.472233in}}%
\pgfpathlineto{\pgfqpoint{2.548532in}{2.478715in}}%
\pgfpathlineto{\pgfqpoint{2.543803in}{2.484188in}}%
\pgfpathlineto{\pgfqpoint{2.539532in}{2.489009in}}%
\pgfpathlineto{\pgfqpoint{2.533226in}{2.496458in}}%
\pgfpathlineto{\pgfqpoint{2.530703in}{2.499304in}}%
\pgfpathlineto{\pgfqpoint{2.522649in}{2.508788in}}%
\pgfpathlineto{\pgfqpoint{2.521913in}{2.509598in}}%
\pgfpathlineto{\pgfqpoint{2.512892in}{2.519893in}}%
\pgfpathlineto{\pgfqpoint{2.512072in}{2.520861in}}%
\pgfpathlineto{\pgfqpoint{2.503392in}{2.530187in}}%
\pgfpathlineto{\pgfqpoint{2.501495in}{2.532273in}}%
\pgfpathlineto{\pgfqpoint{2.493118in}{2.540482in}}%
\pgfpathlineto{\pgfqpoint{2.490917in}{2.542659in}}%
\pgfpathlineto{\pgfqpoint{2.481528in}{2.550776in}}%
\pgfpathlineto{\pgfqpoint{2.480340in}{2.551799in}}%
\pgfpathlineto{\pgfqpoint{2.469763in}{2.559626in}}%
\pgfpathlineto{\pgfqpoint{2.467460in}{2.561071in}}%
\pgfpathlineto{\pgfqpoint{2.459186in}{2.566182in}}%
\pgfpathlineto{\pgfqpoint{2.449177in}{2.571365in}}%
\pgfpathlineto{\pgfqpoint{2.448609in}{2.571653in}}%
\pgfpathlineto{\pgfqpoint{2.438032in}{2.576138in}}%
\pgfpathlineto{\pgfqpoint{2.427455in}{2.579918in}}%
\pgfpathlineto{\pgfqpoint{2.421760in}{2.581659in}}%
\pgfpathlineto{\pgfqpoint{2.416878in}{2.583132in}}%
\pgfpathlineto{\pgfqpoint{2.406301in}{2.585979in}}%
\pgfpathlineto{\pgfqpoint{2.395724in}{2.588701in}}%
\pgfpathlineto{\pgfqpoint{2.385147in}{2.591512in}}%
\pgfpathlineto{\pgfqpoint{2.383613in}{2.591954in}}%
\pgfpathlineto{\pgfqpoint{2.374570in}{2.594604in}}%
\pgfpathlineto{\pgfqpoint{2.363993in}{2.598313in}}%
\pgfpathlineto{\pgfqpoint{2.355016in}{2.602248in}}%
\pgfpathlineto{\pgfqpoint{2.353416in}{2.602976in}}%
\pgfpathlineto{\pgfqpoint{2.342839in}{2.608988in}}%
\pgfpathlineto{\pgfqpoint{2.337941in}{2.612543in}}%
\pgfpathlineto{\pgfqpoint{2.332262in}{2.616906in}}%
\pgfpathlineto{\pgfqpoint{2.326101in}{2.622837in}}%
\pgfpathlineto{\pgfqpoint{2.321684in}{2.627328in}}%
\pgfpathlineto{\pgfqpoint{2.316973in}{2.633132in}}%
\pgfpathlineto{\pgfqpoint{2.311107in}{2.640648in}}%
\pgfpathlineto{\pgfqpoint{2.309254in}{2.643426in}}%
\pgfpathlineto{\pgfqpoint{2.302396in}{2.653720in}}%
\pgfpathlineto{\pgfqpoint{2.300530in}{2.656485in}}%
\pgfpathlineto{\pgfqpoint{2.295995in}{2.664015in}}%
\pgfpathlineto{\pgfqpoint{2.289953in}{2.673836in}}%
\pgfpathlineto{\pgfqpoint{2.289687in}{2.674309in}}%
\pgfpathlineto{\pgfqpoint{2.283583in}{2.684604in}}%
\pgfpathlineto{\pgfqpoint{2.279376in}{2.691447in}}%
\pgfpathlineto{\pgfqpoint{2.277400in}{2.694898in}}%
\pgfpathlineto{\pgfqpoint{2.271226in}{2.705193in}}%
\pgfpathlineto{\pgfqpoint{2.268799in}{2.709094in}}%
\pgfpathlineto{\pgfqpoint{2.265045in}{2.715487in}}%
\pgfpathlineto{\pgfqpoint{2.258849in}{2.725781in}}%
\pgfpathlineto{\pgfqpoint{2.258222in}{2.726803in}}%
\pgfpathlineto{\pgfqpoint{2.252800in}{2.736076in}}%
\pgfpathlineto{\pgfqpoint{2.247645in}{2.744957in}}%
\pgfpathlineto{\pgfqpoint{2.246858in}{2.746370in}}%
\pgfpathlineto{\pgfqpoint{2.241256in}{2.756665in}}%
\pgfpathlineto{\pgfqpoint{2.237068in}{2.764851in}}%
\pgfpathlineto{\pgfqpoint{2.236026in}{2.766959in}}%
\pgfpathlineto{\pgfqpoint{2.231416in}{2.777254in}}%
\pgfpathlineto{\pgfqpoint{2.227443in}{2.787548in}}%
\pgfpathlineto{\pgfqpoint{2.226491in}{2.790533in}}%
\pgfpathlineto{\pgfqpoint{2.224233in}{2.797842in}}%
\pgfpathlineto{\pgfqpoint{2.221712in}{2.808137in}}%
\pgfpathlineto{\pgfqpoint{2.219786in}{2.818431in}}%
\pgfpathlineto{\pgfqpoint{2.218386in}{2.828726in}}%
\pgfpathlineto{\pgfqpoint{2.217465in}{2.839020in}}%
\pgfpathlineto{\pgfqpoint{2.216996in}{2.849315in}}%
\pgfpathlineto{\pgfqpoint{2.216970in}{2.859609in}}%
\pgfpathlineto{\pgfqpoint{2.217398in}{2.869903in}}%
\pgfpathlineto{\pgfqpoint{2.218309in}{2.880198in}}%
\pgfpathlineto{\pgfqpoint{2.219753in}{2.890492in}}%
\pgfpathlineto{\pgfqpoint{2.221804in}{2.900787in}}%
\pgfpathlineto{\pgfqpoint{2.224572in}{2.911081in}}%
\pgfpathlineto{\pgfqpoint{2.215914in}{2.911081in}}%
\pgfpathlineto{\pgfqpoint{2.205337in}{2.911081in}}%
\pgfpathlineto{\pgfqpoint{2.194760in}{2.911081in}}%
\pgfpathlineto{\pgfqpoint{2.189468in}{2.911081in}}%
\pgfpathlineto{\pgfqpoint{2.188563in}{2.900787in}}%
\pgfpathlineto{\pgfqpoint{2.188041in}{2.890492in}}%
\pgfpathlineto{\pgfqpoint{2.187843in}{2.880198in}}%
\pgfpathlineto{\pgfqpoint{2.187931in}{2.869903in}}%
\pgfpathlineto{\pgfqpoint{2.188286in}{2.859609in}}%
\pgfpathlineto{\pgfqpoint{2.188898in}{2.849315in}}%
\pgfpathlineto{\pgfqpoint{2.189771in}{2.839020in}}%
\pgfpathlineto{\pgfqpoint{2.190921in}{2.828726in}}%
\pgfpathlineto{\pgfqpoint{2.192380in}{2.818431in}}%
\pgfpathlineto{\pgfqpoint{2.194206in}{2.808137in}}%
\pgfpathlineto{\pgfqpoint{2.194760in}{2.805629in}}%
\pgfpathlineto{\pgfqpoint{2.196508in}{2.797842in}}%
\pgfpathlineto{\pgfqpoint{2.199388in}{2.787548in}}%
\pgfpathlineto{\pgfqpoint{2.202924in}{2.777254in}}%
\pgfpathlineto{\pgfqpoint{2.205337in}{2.771257in}}%
\pgfpathlineto{\pgfqpoint{2.207104in}{2.766959in}}%
\pgfpathlineto{\pgfqpoint{2.211776in}{2.756665in}}%
\pgfpathlineto{\pgfqpoint{2.215914in}{2.748035in}}%
\pgfpathlineto{\pgfqpoint{2.216734in}{2.746370in}}%
\pgfpathlineto{\pgfqpoint{2.221863in}{2.736076in}}%
\pgfpathlineto{\pgfqpoint{2.226491in}{2.726766in}}%
\pgfpathlineto{\pgfqpoint{2.226998in}{2.725781in}}%
\pgfpathlineto{\pgfqpoint{2.232123in}{2.715487in}}%
\pgfpathlineto{\pgfqpoint{2.237068in}{2.705243in}}%
\pgfpathlineto{\pgfqpoint{2.237093in}{2.705193in}}%
\pgfpathlineto{\pgfqpoint{2.241973in}{2.694898in}}%
\pgfpathlineto{\pgfqpoint{2.246618in}{2.684604in}}%
\pgfpathlineto{\pgfqpoint{2.247645in}{2.682160in}}%
\pgfpathlineto{\pgfqpoint{2.251114in}{2.674309in}}%
\pgfpathlineto{\pgfqpoint{2.255386in}{2.664015in}}%
\pgfpathlineto{\pgfqpoint{2.258222in}{2.656721in}}%
\pgfpathlineto{\pgfqpoint{2.259463in}{2.653720in}}%
\pgfpathlineto{\pgfqpoint{2.263448in}{2.643426in}}%
\pgfpathlineto{\pgfqpoint{2.267268in}{2.633132in}}%
\pgfpathlineto{\pgfqpoint{2.268799in}{2.628834in}}%
\pgfpathlineto{\pgfqpoint{2.271136in}{2.622837in}}%
\pgfpathlineto{\pgfqpoint{2.275133in}{2.612543in}}%
\pgfpathlineto{\pgfqpoint{2.279296in}{2.602248in}}%
\pgfpathlineto{\pgfqpoint{2.279376in}{2.602057in}}%
\pgfpathlineto{\pgfqpoint{2.284300in}{2.591954in}}%
\pgfpathlineto{\pgfqpoint{2.289953in}{2.581818in}}%
\pgfpathlineto{\pgfqpoint{2.290059in}{2.581659in}}%
\pgfpathlineto{\pgfqpoint{2.297758in}{2.571365in}}%
\pgfpathlineto{\pgfqpoint{2.300530in}{2.568105in}}%
\pgfpathlineto{\pgfqpoint{2.307875in}{2.561071in}}%
\pgfpathlineto{\pgfqpoint{2.311107in}{2.558244in}}%
\pgfpathlineto{\pgfqpoint{2.321684in}{2.550848in}}%
\pgfpathlineto{\pgfqpoint{2.321813in}{2.550776in}}%
\pgfpathlineto{\pgfqpoint{2.332262in}{2.545250in}}%
\pgfpathlineto{\pgfqpoint{2.342839in}{2.540972in}}%
\pgfpathlineto{\pgfqpoint{2.344399in}{2.540482in}}%
\pgfpathlineto{\pgfqpoint{2.353416in}{2.537715in}}%
\pgfpathlineto{\pgfqpoint{2.363993in}{2.535204in}}%
\pgfpathlineto{\pgfqpoint{2.374570in}{2.533191in}}%
\pgfpathlineto{\pgfqpoint{2.385147in}{2.531438in}}%
\pgfpathlineto{\pgfqpoint{2.392815in}{2.530187in}}%
\pgfpathlineto{\pgfqpoint{2.395724in}{2.529715in}}%
\pgfpathlineto{\pgfqpoint{2.406301in}{2.527801in}}%
\pgfpathlineto{\pgfqpoint{2.416878in}{2.525513in}}%
\pgfpathlineto{\pgfqpoint{2.427455in}{2.522670in}}%
\pgfpathlineto{\pgfqpoint{2.435692in}{2.519893in}}%
\pgfpathlineto{\pgfqpoint{2.438032in}{2.519107in}}%
\pgfpathlineto{\pgfqpoint{2.448609in}{2.514674in}}%
\pgfpathlineto{\pgfqpoint{2.458575in}{2.509598in}}%
\pgfpathlineto{\pgfqpoint{2.459186in}{2.509290in}}%
\pgfpathlineto{\pgfqpoint{2.469763in}{2.502887in}}%
\pgfpathlineto{\pgfqpoint{2.474833in}{2.499304in}}%
\pgfpathlineto{\pgfqpoint{2.480340in}{2.495477in}}%
\pgfpathlineto{\pgfqpoint{2.488495in}{2.489009in}}%
\pgfpathlineto{\pgfqpoint{2.490917in}{2.487128in}}%
\pgfpathlineto{\pgfqpoint{2.500615in}{2.478715in}}%
\pgfpathlineto{\pgfqpoint{2.501495in}{2.477970in}}%
\pgfpathlineto{\pgfqpoint{2.511795in}{2.468421in}}%
\pgfpathlineto{\pgfqpoint{2.512072in}{2.468170in}}%
\pgfpathlineto{\pgfqpoint{2.522418in}{2.458126in}}%
\pgfpathlineto{\pgfqpoint{2.522649in}{2.457908in}}%
\pgfpathlineto{\pgfqpoint{2.532732in}{2.447832in}}%
\pgfpathlineto{\pgfqpoint{2.533226in}{2.447347in}}%
\pgfpathlineto{\pgfqpoint{2.542892in}{2.437537in}}%
\pgfpathlineto{\pgfqpoint{2.543803in}{2.436625in}}%
\pgfpathlineto{\pgfqpoint{2.552999in}{2.427243in}}%
\pgfpathlineto{\pgfqpoint{2.554380in}{2.425841in}}%
\pgfpathlineto{\pgfqpoint{2.563113in}{2.416948in}}%
\pgfpathlineto{\pgfqpoint{2.564957in}{2.415061in}}%
\pgfpathlineto{\pgfqpoint{2.573265in}{2.406654in}}%
\pgfpathlineto{\pgfqpoint{2.575534in}{2.404320in}}%
\pgfpathlineto{\pgfqpoint{2.583470in}{2.396360in}}%
\pgfpathlineto{\pgfqpoint{2.586111in}{2.393633in}}%
\pgfpathlineto{\pgfqpoint{2.593730in}{2.386065in}}%
\pgfpathlineto{\pgfqpoint{2.596688in}{2.383003in}}%
\pgfpathlineto{\pgfqpoint{2.604047in}{2.375771in}}%
\pgfpathlineto{\pgfqpoint{2.607265in}{2.372434in}}%
\pgfpathlineto{\pgfqpoint{2.614430in}{2.365476in}}%
\pgfpathlineto{\pgfqpoint{2.617842in}{2.361939in}}%
\pgfpathlineto{\pgfqpoint{2.624909in}{2.355182in}}%
\pgfpathlineto{\pgfqpoint{2.628419in}{2.351560in}}%
\pgfpathlineto{\pgfqpoint{2.635558in}{2.344887in}}%
\pgfpathlineto{\pgfqpoint{2.638996in}{2.341388in}}%
\pgfpathlineto{\pgfqpoint{2.646530in}{2.334593in}}%
\pgfpathlineto{\pgfqpoint{2.649573in}{2.331585in}}%
\pgfpathlineto{\pgfqpoint{2.658133in}{2.324299in}}%
\pgfpathlineto{\pgfqpoint{2.660150in}{2.322411in}}%
\pgfpathlineto{\pgfqpoint{2.670727in}{2.314224in}}%
\pgfpathlineto{\pgfqpoint{2.671096in}{2.314004in}}%
\pgfpathlineto{\pgfqpoint{2.681305in}{2.307301in}}%
\pgfpathlineto{\pgfqpoint{2.689141in}{2.303710in}}%
\pgfpathclose%
\pgfusepath{fill}%
\end{pgfscope}%
\begin{pgfscope}%
\pgfpathrectangle{\pgfqpoint{1.856795in}{1.819814in}}{\pgfqpoint{1.194205in}{1.163386in}}%
\pgfusepath{clip}%
\pgfsetbuttcap%
\pgfsetroundjoin%
\definecolor{currentfill}{rgb}{0.964920,0.695342,0.545192}%
\pgfsetfillcolor{currentfill}%
\pgfsetlinewidth{0.000000pt}%
\definecolor{currentstroke}{rgb}{0.000000,0.000000,0.000000}%
\pgfsetstrokecolor{currentstroke}%
\pgfsetdash{}{0pt}%
\pgfpathmoveto{\pgfqpoint{2.321684in}{1.891932in}}%
\pgfpathlineto{\pgfqpoint{2.332262in}{1.891932in}}%
\pgfpathlineto{\pgfqpoint{2.342839in}{1.891932in}}%
\pgfpathlineto{\pgfqpoint{2.353416in}{1.891932in}}%
\pgfpathlineto{\pgfqpoint{2.356858in}{1.891932in}}%
\pgfpathlineto{\pgfqpoint{2.355218in}{1.902227in}}%
\pgfpathlineto{\pgfqpoint{2.353416in}{1.912129in}}%
\pgfpathlineto{\pgfqpoint{2.353346in}{1.912521in}}%
\pgfpathlineto{\pgfqpoint{2.351299in}{1.922816in}}%
\pgfpathlineto{\pgfqpoint{2.349082in}{1.933110in}}%
\pgfpathlineto{\pgfqpoint{2.346708in}{1.943405in}}%
\pgfpathlineto{\pgfqpoint{2.344183in}{1.953699in}}%
\pgfpathlineto{\pgfqpoint{2.342839in}{1.958880in}}%
\pgfpathlineto{\pgfqpoint{2.341509in}{1.963993in}}%
\pgfpathlineto{\pgfqpoint{2.338676in}{1.974288in}}%
\pgfpathlineto{\pgfqpoint{2.335636in}{1.984582in}}%
\pgfpathlineto{\pgfqpoint{2.332262in}{1.994276in}}%
\pgfpathlineto{\pgfqpoint{2.332047in}{1.994877in}}%
\pgfpathlineto{\pgfqpoint{2.326579in}{2.005171in}}%
\pgfpathlineto{\pgfqpoint{2.321684in}{2.012982in}}%
\pgfpathlineto{\pgfqpoint{2.320000in}{2.015466in}}%
\pgfpathlineto{\pgfqpoint{2.312992in}{2.025760in}}%
\pgfpathlineto{\pgfqpoint{2.311107in}{2.028570in}}%
\pgfpathlineto{\pgfqpoint{2.305619in}{2.036054in}}%
\pgfpathlineto{\pgfqpoint{2.300530in}{2.043077in}}%
\pgfpathlineto{\pgfqpoint{2.297917in}{2.046349in}}%
\pgfpathlineto{\pgfqpoint{2.289953in}{2.056525in}}%
\pgfpathlineto{\pgfqpoint{2.289851in}{2.056643in}}%
\pgfpathlineto{\pgfqpoint{2.281291in}{2.066938in}}%
\pgfpathlineto{\pgfqpoint{2.279376in}{2.069317in}}%
\pgfpathlineto{\pgfqpoint{2.272274in}{2.077232in}}%
\pgfpathlineto{\pgfqpoint{2.268799in}{2.081241in}}%
\pgfpathlineto{\pgfqpoint{2.262747in}{2.087527in}}%
\pgfpathlineto{\pgfqpoint{2.258222in}{2.092389in}}%
\pgfpathlineto{\pgfqpoint{2.252662in}{2.097821in}}%
\pgfpathlineto{\pgfqpoint{2.247645in}{2.102867in}}%
\pgfpathlineto{\pgfqpoint{2.242002in}{2.108116in}}%
\pgfpathlineto{\pgfqpoint{2.237068in}{2.112792in}}%
\pgfpathlineto{\pgfqpoint{2.230809in}{2.118410in}}%
\pgfpathlineto{\pgfqpoint{2.226491in}{2.122298in}}%
\pgfpathlineto{\pgfqpoint{2.219194in}{2.128704in}}%
\pgfpathlineto{\pgfqpoint{2.215914in}{2.131540in}}%
\pgfpathlineto{\pgfqpoint{2.207342in}{2.138999in}}%
\pgfpathlineto{\pgfqpoint{2.205337in}{2.140685in}}%
\pgfpathlineto{\pgfqpoint{2.195488in}{2.149293in}}%
\pgfpathlineto{\pgfqpoint{2.194760in}{2.149899in}}%
\pgfpathlineto{\pgfqpoint{2.184183in}{2.159360in}}%
\pgfpathlineto{\pgfqpoint{2.183952in}{2.159588in}}%
\pgfpathlineto{\pgfqpoint{2.173606in}{2.169236in}}%
\pgfpathlineto{\pgfqpoint{2.172984in}{2.169882in}}%
\pgfpathlineto{\pgfqpoint{2.163029in}{2.179685in}}%
\pgfpathlineto{\pgfqpoint{2.162587in}{2.180177in}}%
\pgfpathlineto{\pgfqpoint{2.152925in}{2.190471in}}%
\pgfpathlineto{\pgfqpoint{2.152451in}{2.190955in}}%
\pgfpathlineto{\pgfqpoint{2.144156in}{2.200765in}}%
\pgfpathlineto{\pgfqpoint{2.141874in}{2.203360in}}%
\pgfpathlineto{\pgfqpoint{2.136058in}{2.211060in}}%
\pgfpathlineto{\pgfqpoint{2.131297in}{2.217163in}}%
\pgfpathlineto{\pgfqpoint{2.128489in}{2.221354in}}%
\pgfpathlineto{\pgfqpoint{2.121444in}{2.231649in}}%
\pgfpathlineto{\pgfqpoint{2.120720in}{2.232701in}}%
\pgfpathlineto{\pgfqpoint{2.115293in}{2.241943in}}%
\pgfpathlineto{\pgfqpoint{2.110143in}{2.250483in}}%
\pgfpathlineto{\pgfqpoint{2.109233in}{2.252238in}}%
\pgfpathlineto{\pgfqpoint{2.103915in}{2.262532in}}%
\pgfpathlineto{\pgfqpoint{2.099566in}{2.270739in}}%
\pgfpathlineto{\pgfqpoint{2.098612in}{2.272826in}}%
\pgfpathlineto{\pgfqpoint{2.093901in}{2.283121in}}%
\pgfpathlineto{\pgfqpoint{2.089025in}{2.293415in}}%
\pgfpathlineto{\pgfqpoint{2.088989in}{2.293491in}}%
\pgfpathlineto{\pgfqpoint{2.084775in}{2.303710in}}%
\pgfpathlineto{\pgfqpoint{2.080363in}{2.314004in}}%
\pgfpathlineto{\pgfqpoint{2.078412in}{2.318436in}}%
\pgfpathlineto{\pgfqpoint{2.076156in}{2.324299in}}%
\pgfpathlineto{\pgfqpoint{2.072069in}{2.334593in}}%
\pgfpathlineto{\pgfqpoint{2.067835in}{2.344753in}}%
\pgfpathlineto{\pgfqpoint{2.067785in}{2.344887in}}%
\pgfpathlineto{\pgfqpoint{2.063923in}{2.355182in}}%
\pgfpathlineto{\pgfqpoint{2.059864in}{2.365476in}}%
\pgfpathlineto{\pgfqpoint{2.057258in}{2.371836in}}%
\pgfpathlineto{\pgfqpoint{2.055816in}{2.375771in}}%
\pgfpathlineto{\pgfqpoint{2.051933in}{2.386065in}}%
\pgfpathlineto{\pgfqpoint{2.047861in}{2.396360in}}%
\pgfpathlineto{\pgfqpoint{2.046681in}{2.399278in}}%
\pgfpathlineto{\pgfqpoint{2.043976in}{2.406654in}}%
\pgfpathlineto{\pgfqpoint{2.040073in}{2.416948in}}%
\pgfpathlineto{\pgfqpoint{2.036104in}{2.427009in}}%
\pgfpathlineto{\pgfqpoint{2.036019in}{2.427243in}}%
\pgfpathlineto{\pgfqpoint{2.032257in}{2.437537in}}%
\pgfpathlineto{\pgfqpoint{2.028359in}{2.447832in}}%
\pgfpathlineto{\pgfqpoint{2.025527in}{2.455137in}}%
\pgfpathlineto{\pgfqpoint{2.024443in}{2.458126in}}%
\pgfpathlineto{\pgfqpoint{2.020685in}{2.468421in}}%
\pgfpathlineto{\pgfqpoint{2.016822in}{2.478715in}}%
\pgfpathlineto{\pgfqpoint{2.014950in}{2.483661in}}%
\pgfpathlineto{\pgfqpoint{2.013049in}{2.489009in}}%
\pgfpathlineto{\pgfqpoint{2.009386in}{2.499304in}}%
\pgfpathlineto{\pgfqpoint{2.005687in}{2.509598in}}%
\pgfpathlineto{\pgfqpoint{2.004373in}{2.513296in}}%
\pgfpathlineto{\pgfqpoint{2.002197in}{2.519893in}}%
\pgfpathlineto{\pgfqpoint{1.998862in}{2.530187in}}%
\pgfpathlineto{\pgfqpoint{1.995586in}{2.540482in}}%
\pgfpathlineto{\pgfqpoint{1.993796in}{2.546266in}}%
\pgfpathlineto{\pgfqpoint{1.992512in}{2.550776in}}%
\pgfpathlineto{\pgfqpoint{1.989691in}{2.561071in}}%
\pgfpathlineto{\pgfqpoint{1.986960in}{2.571365in}}%
\pgfpathlineto{\pgfqpoint{1.984308in}{2.581659in}}%
\pgfpathlineto{\pgfqpoint{1.983218in}{2.586000in}}%
\pgfpathlineto{\pgfqpoint{1.981851in}{2.591954in}}%
\pgfpathlineto{\pgfqpoint{1.979529in}{2.602248in}}%
\pgfpathlineto{\pgfqpoint{1.977189in}{2.612543in}}%
\pgfpathlineto{\pgfqpoint{1.974746in}{2.622837in}}%
\pgfpathlineto{\pgfqpoint{1.972641in}{2.631080in}}%
\pgfpathlineto{\pgfqpoint{1.972151in}{2.633132in}}%
\pgfpathlineto{\pgfqpoint{1.969509in}{2.643426in}}%
\pgfpathlineto{\pgfqpoint{1.966679in}{2.653720in}}%
\pgfpathlineto{\pgfqpoint{1.963720in}{2.664015in}}%
\pgfpathlineto{\pgfqpoint{1.962064in}{2.669687in}}%
\pgfpathlineto{\pgfqpoint{1.960769in}{2.674309in}}%
\pgfpathlineto{\pgfqpoint{1.957873in}{2.684604in}}%
\pgfpathlineto{\pgfqpoint{1.954925in}{2.694898in}}%
\pgfpathlineto{\pgfqpoint{1.951925in}{2.705193in}}%
\pgfpathlineto{\pgfqpoint{1.951487in}{2.706708in}}%
\pgfpathlineto{\pgfqpoint{1.949031in}{2.715487in}}%
\pgfpathlineto{\pgfqpoint{1.946108in}{2.725781in}}%
\pgfpathlineto{\pgfqpoint{1.943118in}{2.736076in}}%
\pgfpathlineto{\pgfqpoint{1.940910in}{2.743531in}}%
\pgfpathlineto{\pgfqpoint{1.940084in}{2.746370in}}%
\pgfpathlineto{\pgfqpoint{1.937055in}{2.756665in}}%
\pgfpathlineto{\pgfqpoint{1.933894in}{2.766959in}}%
\pgfpathlineto{\pgfqpoint{1.930566in}{2.777254in}}%
\pgfpathlineto{\pgfqpoint{1.930333in}{2.777960in}}%
\pgfpathlineto{\pgfqpoint{1.930333in}{2.777254in}}%
\pgfpathlineto{\pgfqpoint{1.930333in}{2.766959in}}%
\pgfpathlineto{\pgfqpoint{1.930333in}{2.756665in}}%
\pgfpathlineto{\pgfqpoint{1.930333in}{2.746370in}}%
\pgfpathlineto{\pgfqpoint{1.930333in}{2.736076in}}%
\pgfpathlineto{\pgfqpoint{1.930333in}{2.725781in}}%
\pgfpathlineto{\pgfqpoint{1.930333in}{2.715487in}}%
\pgfpathlineto{\pgfqpoint{1.930333in}{2.713258in}}%
\pgfpathlineto{\pgfqpoint{1.932899in}{2.705193in}}%
\pgfpathlineto{\pgfqpoint{1.936129in}{2.694898in}}%
\pgfpathlineto{\pgfqpoint{1.939303in}{2.684604in}}%
\pgfpathlineto{\pgfqpoint{1.940910in}{2.679382in}}%
\pgfpathlineto{\pgfqpoint{1.942498in}{2.674309in}}%
\pgfpathlineto{\pgfqpoint{1.945710in}{2.664015in}}%
\pgfpathlineto{\pgfqpoint{1.948857in}{2.653720in}}%
\pgfpathlineto{\pgfqpoint{1.951487in}{2.644912in}}%
\pgfpathlineto{\pgfqpoint{1.951941in}{2.643426in}}%
\pgfpathlineto{\pgfqpoint{1.955003in}{2.633132in}}%
\pgfpathlineto{\pgfqpoint{1.957736in}{2.622837in}}%
\pgfpathlineto{\pgfqpoint{1.960045in}{2.612543in}}%
\pgfpathlineto{\pgfqpoint{1.962064in}{2.602627in}}%
\pgfpathlineto{\pgfqpoint{1.962147in}{2.602248in}}%
\pgfpathlineto{\pgfqpoint{1.964361in}{2.591954in}}%
\pgfpathlineto{\pgfqpoint{1.966600in}{2.581659in}}%
\pgfpathlineto{\pgfqpoint{1.968906in}{2.571365in}}%
\pgfpathlineto{\pgfqpoint{1.971308in}{2.561071in}}%
\pgfpathlineto{\pgfqpoint{1.972641in}{2.555614in}}%
\pgfpathlineto{\pgfqpoint{1.973910in}{2.550776in}}%
\pgfpathlineto{\pgfqpoint{1.976733in}{2.540482in}}%
\pgfpathlineto{\pgfqpoint{1.979677in}{2.530187in}}%
\pgfpathlineto{\pgfqpoint{1.982738in}{2.519893in}}%
\pgfpathlineto{\pgfqpoint{1.983218in}{2.518346in}}%
\pgfpathlineto{\pgfqpoint{1.986105in}{2.509598in}}%
\pgfpathlineto{\pgfqpoint{1.989585in}{2.499304in}}%
\pgfpathlineto{\pgfqpoint{1.993120in}{2.489009in}}%
\pgfpathlineto{\pgfqpoint{1.993796in}{2.487089in}}%
\pgfpathlineto{\pgfqpoint{1.996892in}{2.478715in}}%
\pgfpathlineto{\pgfqpoint{2.000692in}{2.468421in}}%
\pgfpathlineto{\pgfqpoint{2.004373in}{2.458331in}}%
\pgfpathlineto{\pgfqpoint{2.004451in}{2.458126in}}%
\pgfpathlineto{\pgfqpoint{2.008406in}{2.447832in}}%
\pgfpathlineto{\pgfqpoint{2.012251in}{2.437537in}}%
\pgfpathlineto{\pgfqpoint{2.014950in}{2.430141in}}%
\pgfpathlineto{\pgfqpoint{2.016062in}{2.427243in}}%
\pgfpathlineto{\pgfqpoint{2.019958in}{2.416948in}}%
\pgfpathlineto{\pgfqpoint{2.023703in}{2.406654in}}%
\pgfpathlineto{\pgfqpoint{2.025527in}{2.401502in}}%
\pgfpathlineto{\pgfqpoint{2.027471in}{2.396360in}}%
\pgfpathlineto{\pgfqpoint{2.031253in}{2.386065in}}%
\pgfpathlineto{\pgfqpoint{2.034866in}{2.375771in}}%
\pgfpathlineto{\pgfqpoint{2.036104in}{2.372146in}}%
\pgfpathlineto{\pgfqpoint{2.038574in}{2.365476in}}%
\pgfpathlineto{\pgfqpoint{2.042249in}{2.355182in}}%
\pgfpathlineto{\pgfqpoint{2.045759in}{2.344887in}}%
\pgfpathlineto{\pgfqpoint{2.046681in}{2.342115in}}%
\pgfpathlineto{\pgfqpoint{2.049431in}{2.334593in}}%
\pgfpathlineto{\pgfqpoint{2.053061in}{2.324299in}}%
\pgfpathlineto{\pgfqpoint{2.056556in}{2.314004in}}%
\pgfpathlineto{\pgfqpoint{2.057258in}{2.311905in}}%
\pgfpathlineto{\pgfqpoint{2.060301in}{2.303710in}}%
\pgfpathlineto{\pgfqpoint{2.064025in}{2.293415in}}%
\pgfpathlineto{\pgfqpoint{2.067662in}{2.283121in}}%
\pgfpathlineto{\pgfqpoint{2.067835in}{2.282634in}}%
\pgfpathlineto{\pgfqpoint{2.071720in}{2.272826in}}%
\pgfpathlineto{\pgfqpoint{2.075741in}{2.262532in}}%
\pgfpathlineto{\pgfqpoint{2.078412in}{2.255690in}}%
\pgfpathlineto{\pgfqpoint{2.079925in}{2.252238in}}%
\pgfpathlineto{\pgfqpoint{2.084472in}{2.241943in}}%
\pgfpathlineto{\pgfqpoint{2.088989in}{2.231669in}}%
\pgfpathlineto{\pgfqpoint{2.088999in}{2.231649in}}%
\pgfpathlineto{\pgfqpoint{2.094195in}{2.221354in}}%
\pgfpathlineto{\pgfqpoint{2.099347in}{2.211060in}}%
\pgfpathlineto{\pgfqpoint{2.099566in}{2.210632in}}%
\pgfpathlineto{\pgfqpoint{2.105207in}{2.200765in}}%
\pgfpathlineto{\pgfqpoint{2.110143in}{2.192015in}}%
\pgfpathlineto{\pgfqpoint{2.111117in}{2.190471in}}%
\pgfpathlineto{\pgfqpoint{2.117622in}{2.180177in}}%
\pgfpathlineto{\pgfqpoint{2.120720in}{2.175188in}}%
\pgfpathlineto{\pgfqpoint{2.124377in}{2.169882in}}%
\pgfpathlineto{\pgfqpoint{2.131237in}{2.159588in}}%
\pgfpathlineto{\pgfqpoint{2.131297in}{2.159496in}}%
\pgfpathlineto{\pgfqpoint{2.138698in}{2.149293in}}%
\pgfpathlineto{\pgfqpoint{2.141874in}{2.144701in}}%
\pgfpathlineto{\pgfqpoint{2.146220in}{2.138999in}}%
\pgfpathlineto{\pgfqpoint{2.152451in}{2.130401in}}%
\pgfpathlineto{\pgfqpoint{2.153808in}{2.128704in}}%
\pgfpathlineto{\pgfqpoint{2.161757in}{2.118410in}}%
\pgfpathlineto{\pgfqpoint{2.163029in}{2.116720in}}%
\pgfpathlineto{\pgfqpoint{2.170167in}{2.108116in}}%
\pgfpathlineto{\pgfqpoint{2.173606in}{2.103893in}}%
\pgfpathlineto{\pgfqpoint{2.179041in}{2.097821in}}%
\pgfpathlineto{\pgfqpoint{2.184183in}{2.092079in}}%
\pgfpathlineto{\pgfqpoint{2.188621in}{2.087527in}}%
\pgfpathlineto{\pgfqpoint{2.194760in}{2.081329in}}%
\pgfpathlineto{\pgfqpoint{2.199104in}{2.077232in}}%
\pgfpathlineto{\pgfqpoint{2.205337in}{2.071509in}}%
\pgfpathlineto{\pgfqpoint{2.210546in}{2.066938in}}%
\pgfpathlineto{\pgfqpoint{2.215914in}{2.062374in}}%
\pgfpathlineto{\pgfqpoint{2.222776in}{2.056643in}}%
\pgfpathlineto{\pgfqpoint{2.226491in}{2.053640in}}%
\pgfpathlineto{\pgfqpoint{2.235416in}{2.046349in}}%
\pgfpathlineto{\pgfqpoint{2.237068in}{2.045039in}}%
\pgfpathlineto{\pgfqpoint{2.247645in}{2.036361in}}%
\pgfpathlineto{\pgfqpoint{2.248000in}{2.036054in}}%
\pgfpathlineto{\pgfqpoint{2.258222in}{2.027453in}}%
\pgfpathlineto{\pgfqpoint{2.260092in}{2.025760in}}%
\pgfpathlineto{\pgfqpoint{2.268799in}{2.018000in}}%
\pgfpathlineto{\pgfqpoint{2.271416in}{2.015466in}}%
\pgfpathlineto{\pgfqpoint{2.279376in}{2.007823in}}%
\pgfpathlineto{\pgfqpoint{2.281905in}{2.005171in}}%
\pgfpathlineto{\pgfqpoint{2.289953in}{1.996683in}}%
\pgfpathlineto{\pgfqpoint{2.291478in}{1.994877in}}%
\pgfpathlineto{\pgfqpoint{2.296211in}{1.984582in}}%
\pgfpathlineto{\pgfqpoint{2.299773in}{1.974288in}}%
\pgfpathlineto{\pgfqpoint{2.300530in}{1.971919in}}%
\pgfpathlineto{\pgfqpoint{2.303013in}{1.963993in}}%
\pgfpathlineto{\pgfqpoint{2.305960in}{1.953699in}}%
\pgfpathlineto{\pgfqpoint{2.308613in}{1.943405in}}%
\pgfpathlineto{\pgfqpoint{2.310958in}{1.933110in}}%
\pgfpathlineto{\pgfqpoint{2.311107in}{1.932342in}}%
\pgfpathlineto{\pgfqpoint{2.312964in}{1.922816in}}%
\pgfpathlineto{\pgfqpoint{2.314617in}{1.912521in}}%
\pgfpathlineto{\pgfqpoint{2.315873in}{1.902227in}}%
\pgfpathlineto{\pgfqpoint{2.316673in}{1.891932in}}%
\pgfpathclose%
\pgfusepath{fill}%
\end{pgfscope}%
\begin{pgfscope}%
\pgfpathrectangle{\pgfqpoint{1.856795in}{1.819814in}}{\pgfqpoint{1.194205in}{1.163386in}}%
\pgfusepath{clip}%
\pgfsetbuttcap%
\pgfsetroundjoin%
\definecolor{currentfill}{rgb}{0.964920,0.695342,0.545192}%
\pgfsetfillcolor{currentfill}%
\pgfsetlinewidth{0.000000pt}%
\definecolor{currentstroke}{rgb}{0.000000,0.000000,0.000000}%
\pgfsetstrokecolor{currentstroke}%
\pgfsetdash{}{0pt}%
\pgfpathmoveto{\pgfqpoint{2.681305in}{2.199873in}}%
\pgfpathlineto{\pgfqpoint{2.691882in}{2.195090in}}%
\pgfpathlineto{\pgfqpoint{2.702459in}{2.192228in}}%
\pgfpathlineto{\pgfqpoint{2.713036in}{2.191357in}}%
\pgfpathlineto{\pgfqpoint{2.723613in}{2.192523in}}%
\pgfpathlineto{\pgfqpoint{2.734190in}{2.195744in}}%
\pgfpathlineto{\pgfqpoint{2.744294in}{2.200765in}}%
\pgfpathlineto{\pgfqpoint{2.744767in}{2.201009in}}%
\pgfpathlineto{\pgfqpoint{2.755344in}{2.208481in}}%
\pgfpathlineto{\pgfqpoint{2.758227in}{2.211060in}}%
\pgfpathlineto{\pgfqpoint{2.765921in}{2.218516in}}%
\pgfpathlineto{\pgfqpoint{2.768351in}{2.221354in}}%
\pgfpathlineto{\pgfqpoint{2.776349in}{2.231649in}}%
\pgfpathlineto{\pgfqpoint{2.776498in}{2.231857in}}%
\pgfpathlineto{\pgfqpoint{2.782741in}{2.241943in}}%
\pgfpathlineto{\pgfqpoint{2.787075in}{2.249405in}}%
\pgfpathlineto{\pgfqpoint{2.788544in}{2.252238in}}%
\pgfpathlineto{\pgfqpoint{2.793722in}{2.262532in}}%
\pgfpathlineto{\pgfqpoint{2.797652in}{2.270449in}}%
\pgfpathlineto{\pgfqpoint{2.798739in}{2.272826in}}%
\pgfpathlineto{\pgfqpoint{2.803495in}{2.283121in}}%
\pgfpathlineto{\pgfqpoint{2.808229in}{2.293139in}}%
\pgfpathlineto{\pgfqpoint{2.808353in}{2.293415in}}%
\pgfpathlineto{\pgfqpoint{2.813132in}{2.303710in}}%
\pgfpathlineto{\pgfqpoint{2.818144in}{2.314004in}}%
\pgfpathlineto{\pgfqpoint{2.818806in}{2.315290in}}%
\pgfpathlineto{\pgfqpoint{2.823289in}{2.324299in}}%
\pgfpathlineto{\pgfqpoint{2.828737in}{2.334593in}}%
\pgfpathlineto{\pgfqpoint{2.829383in}{2.335736in}}%
\pgfpathlineto{\pgfqpoint{2.834481in}{2.344887in}}%
\pgfpathlineto{\pgfqpoint{2.839960in}{2.354090in}}%
\pgfpathlineto{\pgfqpoint{2.840613in}{2.355182in}}%
\pgfpathlineto{\pgfqpoint{2.847205in}{2.365476in}}%
\pgfpathlineto{\pgfqpoint{2.850538in}{2.370341in}}%
\pgfpathlineto{\pgfqpoint{2.854339in}{2.375771in}}%
\pgfpathlineto{\pgfqpoint{2.861115in}{2.384830in}}%
\pgfpathlineto{\pgfqpoint{2.862077in}{2.386065in}}%
\pgfpathlineto{\pgfqpoint{2.870647in}{2.396360in}}%
\pgfpathlineto{\pgfqpoint{2.871692in}{2.397537in}}%
\pgfpathlineto{\pgfqpoint{2.880303in}{2.406654in}}%
\pgfpathlineto{\pgfqpoint{2.882269in}{2.408612in}}%
\pgfpathlineto{\pgfqpoint{2.891401in}{2.416948in}}%
\pgfpathlineto{\pgfqpoint{2.892846in}{2.418192in}}%
\pgfpathlineto{\pgfqpoint{2.903423in}{2.426288in}}%
\pgfpathlineto{\pgfqpoint{2.904886in}{2.427243in}}%
\pgfpathlineto{\pgfqpoint{2.914000in}{2.432881in}}%
\pgfpathlineto{\pgfqpoint{2.923284in}{2.437537in}}%
\pgfpathlineto{\pgfqpoint{2.924577in}{2.438157in}}%
\pgfpathlineto{\pgfqpoint{2.935154in}{2.441887in}}%
\pgfpathlineto{\pgfqpoint{2.945731in}{2.444215in}}%
\pgfpathlineto{\pgfqpoint{2.956308in}{2.445070in}}%
\pgfpathlineto{\pgfqpoint{2.966885in}{2.444745in}}%
\pgfpathlineto{\pgfqpoint{2.972055in}{2.447832in}}%
\pgfpathlineto{\pgfqpoint{2.977462in}{2.453233in}}%
\pgfpathlineto{\pgfqpoint{2.977462in}{2.458126in}}%
\pgfpathlineto{\pgfqpoint{2.977462in}{2.468421in}}%
\pgfpathlineto{\pgfqpoint{2.977462in}{2.478715in}}%
\pgfpathlineto{\pgfqpoint{2.977462in}{2.489009in}}%
\pgfpathlineto{\pgfqpoint{2.977462in}{2.499304in}}%
\pgfpathlineto{\pgfqpoint{2.977462in}{2.509598in}}%
\pgfpathlineto{\pgfqpoint{2.977462in}{2.519893in}}%
\pgfpathlineto{\pgfqpoint{2.977462in}{2.530187in}}%
\pgfpathlineto{\pgfqpoint{2.977462in}{2.540482in}}%
\pgfpathlineto{\pgfqpoint{2.977462in}{2.550776in}}%
\pgfpathlineto{\pgfqpoint{2.977462in}{2.561071in}}%
\pgfpathlineto{\pgfqpoint{2.977462in}{2.571365in}}%
\pgfpathlineto{\pgfqpoint{2.977462in}{2.581659in}}%
\pgfpathlineto{\pgfqpoint{2.977462in}{2.591954in}}%
\pgfpathlineto{\pgfqpoint{2.977462in}{2.602248in}}%
\pgfpathlineto{\pgfqpoint{2.977462in}{2.612543in}}%
\pgfpathlineto{\pgfqpoint{2.977462in}{2.622837in}}%
\pgfpathlineto{\pgfqpoint{2.977462in}{2.633132in}}%
\pgfpathlineto{\pgfqpoint{2.977462in}{2.643426in}}%
\pgfpathlineto{\pgfqpoint{2.977462in}{2.653720in}}%
\pgfpathlineto{\pgfqpoint{2.977462in}{2.664015in}}%
\pgfpathlineto{\pgfqpoint{2.977462in}{2.674309in}}%
\pgfpathlineto{\pgfqpoint{2.977462in}{2.677306in}}%
\pgfpathlineto{\pgfqpoint{2.976510in}{2.674309in}}%
\pgfpathlineto{\pgfqpoint{2.973801in}{2.664015in}}%
\pgfpathlineto{\pgfqpoint{2.971581in}{2.653720in}}%
\pgfpathlineto{\pgfqpoint{2.969803in}{2.643426in}}%
\pgfpathlineto{\pgfqpoint{2.968411in}{2.633132in}}%
\pgfpathlineto{\pgfqpoint{2.967350in}{2.622837in}}%
\pgfpathlineto{\pgfqpoint{2.966885in}{2.616736in}}%
\pgfpathlineto{\pgfqpoint{2.966560in}{2.612543in}}%
\pgfpathlineto{\pgfqpoint{2.965977in}{2.602248in}}%
\pgfpathlineto{\pgfqpoint{2.965533in}{2.591954in}}%
\pgfpathlineto{\pgfqpoint{2.965160in}{2.581659in}}%
\pgfpathlineto{\pgfqpoint{2.964785in}{2.571365in}}%
\pgfpathlineto{\pgfqpoint{2.964328in}{2.561071in}}%
\pgfpathlineto{\pgfqpoint{2.963698in}{2.550776in}}%
\pgfpathlineto{\pgfqpoint{2.962779in}{2.540482in}}%
\pgfpathlineto{\pgfqpoint{2.961410in}{2.530187in}}%
\pgfpathlineto{\pgfqpoint{2.959330in}{2.519893in}}%
\pgfpathlineto{\pgfqpoint{2.956308in}{2.510298in}}%
\pgfpathlineto{\pgfqpoint{2.955984in}{2.509598in}}%
\pgfpathlineto{\pgfqpoint{2.948202in}{2.499304in}}%
\pgfpathlineto{\pgfqpoint{2.945731in}{2.497220in}}%
\pgfpathlineto{\pgfqpoint{2.935154in}{2.491197in}}%
\pgfpathlineto{\pgfqpoint{2.930765in}{2.489009in}}%
\pgfpathlineto{\pgfqpoint{2.924577in}{2.486152in}}%
\pgfpathlineto{\pgfqpoint{2.914000in}{2.480893in}}%
\pgfpathlineto{\pgfqpoint{2.910052in}{2.478715in}}%
\pgfpathlineto{\pgfqpoint{2.903423in}{2.474996in}}%
\pgfpathlineto{\pgfqpoint{2.892846in}{2.468423in}}%
\pgfpathlineto{\pgfqpoint{2.892843in}{2.468421in}}%
\pgfpathlineto{\pgfqpoint{2.882269in}{2.460890in}}%
\pgfpathlineto{\pgfqpoint{2.878652in}{2.458126in}}%
\pgfpathlineto{\pgfqpoint{2.871692in}{2.452541in}}%
\pgfpathlineto{\pgfqpoint{2.866132in}{2.447832in}}%
\pgfpathlineto{\pgfqpoint{2.861115in}{2.443352in}}%
\pgfpathlineto{\pgfqpoint{2.854838in}{2.437537in}}%
\pgfpathlineto{\pgfqpoint{2.850538in}{2.433331in}}%
\pgfpathlineto{\pgfqpoint{2.844435in}{2.427243in}}%
\pgfpathlineto{\pgfqpoint{2.839960in}{2.422523in}}%
\pgfpathlineto{\pgfqpoint{2.834689in}{2.416948in}}%
\pgfpathlineto{\pgfqpoint{2.829383in}{2.411009in}}%
\pgfpathlineto{\pgfqpoint{2.825436in}{2.406654in}}%
\pgfpathlineto{\pgfqpoint{2.818806in}{2.398904in}}%
\pgfpathlineto{\pgfqpoint{2.816558in}{2.396360in}}%
\pgfpathlineto{\pgfqpoint{2.808229in}{2.386355in}}%
\pgfpathlineto{\pgfqpoint{2.807975in}{2.386065in}}%
\pgfpathlineto{\pgfqpoint{2.799520in}{2.375771in}}%
\pgfpathlineto{\pgfqpoint{2.797652in}{2.373349in}}%
\pgfpathlineto{\pgfqpoint{2.791131in}{2.365476in}}%
\pgfpathlineto{\pgfqpoint{2.787075in}{2.360259in}}%
\pgfpathlineto{\pgfqpoint{2.782734in}{2.355182in}}%
\pgfpathlineto{\pgfqpoint{2.776498in}{2.347409in}}%
\pgfpathlineto{\pgfqpoint{2.774209in}{2.344887in}}%
\pgfpathlineto{\pgfqpoint{2.765921in}{2.335166in}}%
\pgfpathlineto{\pgfqpoint{2.765348in}{2.334593in}}%
\pgfpathlineto{\pgfqpoint{2.755659in}{2.324299in}}%
\pgfpathlineto{\pgfqpoint{2.755344in}{2.323945in}}%
\pgfpathlineto{\pgfqpoint{2.744767in}{2.314323in}}%
\pgfpathlineto{\pgfqpoint{2.744301in}{2.314004in}}%
\pgfpathlineto{\pgfqpoint{2.734190in}{2.306743in}}%
\pgfpathlineto{\pgfqpoint{2.727708in}{2.303710in}}%
\pgfpathlineto{\pgfqpoint{2.723613in}{2.301712in}}%
\pgfpathlineto{\pgfqpoint{2.713036in}{2.299276in}}%
\pgfpathlineto{\pgfqpoint{2.702459in}{2.299529in}}%
\pgfpathlineto{\pgfqpoint{2.691882in}{2.302340in}}%
\pgfpathlineto{\pgfqpoint{2.689141in}{2.303710in}}%
\pgfpathlineto{\pgfqpoint{2.681305in}{2.307301in}}%
\pgfpathlineto{\pgfqpoint{2.671096in}{2.314004in}}%
\pgfpathlineto{\pgfqpoint{2.670727in}{2.314224in}}%
\pgfpathlineto{\pgfqpoint{2.660150in}{2.322411in}}%
\pgfpathlineto{\pgfqpoint{2.658133in}{2.324299in}}%
\pgfpathlineto{\pgfqpoint{2.649573in}{2.331585in}}%
\pgfpathlineto{\pgfqpoint{2.646530in}{2.334593in}}%
\pgfpathlineto{\pgfqpoint{2.638996in}{2.341388in}}%
\pgfpathlineto{\pgfqpoint{2.635558in}{2.344887in}}%
\pgfpathlineto{\pgfqpoint{2.628419in}{2.351560in}}%
\pgfpathlineto{\pgfqpoint{2.624909in}{2.355182in}}%
\pgfpathlineto{\pgfqpoint{2.617842in}{2.361939in}}%
\pgfpathlineto{\pgfqpoint{2.614430in}{2.365476in}}%
\pgfpathlineto{\pgfqpoint{2.607265in}{2.372434in}}%
\pgfpathlineto{\pgfqpoint{2.604047in}{2.375771in}}%
\pgfpathlineto{\pgfqpoint{2.596688in}{2.383003in}}%
\pgfpathlineto{\pgfqpoint{2.593730in}{2.386065in}}%
\pgfpathlineto{\pgfqpoint{2.586111in}{2.393633in}}%
\pgfpathlineto{\pgfqpoint{2.583470in}{2.396360in}}%
\pgfpathlineto{\pgfqpoint{2.575534in}{2.404320in}}%
\pgfpathlineto{\pgfqpoint{2.573265in}{2.406654in}}%
\pgfpathlineto{\pgfqpoint{2.564957in}{2.415061in}}%
\pgfpathlineto{\pgfqpoint{2.563113in}{2.416948in}}%
\pgfpathlineto{\pgfqpoint{2.554380in}{2.425841in}}%
\pgfpathlineto{\pgfqpoint{2.552999in}{2.427243in}}%
\pgfpathlineto{\pgfqpoint{2.543803in}{2.436625in}}%
\pgfpathlineto{\pgfqpoint{2.542892in}{2.437537in}}%
\pgfpathlineto{\pgfqpoint{2.533226in}{2.447347in}}%
\pgfpathlineto{\pgfqpoint{2.532732in}{2.447832in}}%
\pgfpathlineto{\pgfqpoint{2.522649in}{2.457908in}}%
\pgfpathlineto{\pgfqpoint{2.522418in}{2.458126in}}%
\pgfpathlineto{\pgfqpoint{2.512072in}{2.468170in}}%
\pgfpathlineto{\pgfqpoint{2.511795in}{2.468421in}}%
\pgfpathlineto{\pgfqpoint{2.501495in}{2.477970in}}%
\pgfpathlineto{\pgfqpoint{2.500615in}{2.478715in}}%
\pgfpathlineto{\pgfqpoint{2.490917in}{2.487128in}}%
\pgfpathlineto{\pgfqpoint{2.488495in}{2.489009in}}%
\pgfpathlineto{\pgfqpoint{2.480340in}{2.495477in}}%
\pgfpathlineto{\pgfqpoint{2.474833in}{2.499304in}}%
\pgfpathlineto{\pgfqpoint{2.469763in}{2.502887in}}%
\pgfpathlineto{\pgfqpoint{2.459186in}{2.509290in}}%
\pgfpathlineto{\pgfqpoint{2.458575in}{2.509598in}}%
\pgfpathlineto{\pgfqpoint{2.448609in}{2.514674in}}%
\pgfpathlineto{\pgfqpoint{2.438032in}{2.519107in}}%
\pgfpathlineto{\pgfqpoint{2.435692in}{2.519893in}}%
\pgfpathlineto{\pgfqpoint{2.427455in}{2.522670in}}%
\pgfpathlineto{\pgfqpoint{2.416878in}{2.525513in}}%
\pgfpathlineto{\pgfqpoint{2.406301in}{2.527801in}}%
\pgfpathlineto{\pgfqpoint{2.395724in}{2.529715in}}%
\pgfpathlineto{\pgfqpoint{2.392815in}{2.530187in}}%
\pgfpathlineto{\pgfqpoint{2.385147in}{2.531438in}}%
\pgfpathlineto{\pgfqpoint{2.374570in}{2.533191in}}%
\pgfpathlineto{\pgfqpoint{2.363993in}{2.535204in}}%
\pgfpathlineto{\pgfqpoint{2.353416in}{2.537715in}}%
\pgfpathlineto{\pgfqpoint{2.344399in}{2.540482in}}%
\pgfpathlineto{\pgfqpoint{2.342839in}{2.540972in}}%
\pgfpathlineto{\pgfqpoint{2.332262in}{2.545250in}}%
\pgfpathlineto{\pgfqpoint{2.321813in}{2.550776in}}%
\pgfpathlineto{\pgfqpoint{2.321684in}{2.550848in}}%
\pgfpathlineto{\pgfqpoint{2.311107in}{2.558244in}}%
\pgfpathlineto{\pgfqpoint{2.307875in}{2.561071in}}%
\pgfpathlineto{\pgfqpoint{2.300530in}{2.568105in}}%
\pgfpathlineto{\pgfqpoint{2.297758in}{2.571365in}}%
\pgfpathlineto{\pgfqpoint{2.290059in}{2.581659in}}%
\pgfpathlineto{\pgfqpoint{2.289953in}{2.581818in}}%
\pgfpathlineto{\pgfqpoint{2.284300in}{2.591954in}}%
\pgfpathlineto{\pgfqpoint{2.279376in}{2.602057in}}%
\pgfpathlineto{\pgfqpoint{2.279296in}{2.602248in}}%
\pgfpathlineto{\pgfqpoint{2.275133in}{2.612543in}}%
\pgfpathlineto{\pgfqpoint{2.271136in}{2.622837in}}%
\pgfpathlineto{\pgfqpoint{2.268799in}{2.628834in}}%
\pgfpathlineto{\pgfqpoint{2.267268in}{2.633132in}}%
\pgfpathlineto{\pgfqpoint{2.263448in}{2.643426in}}%
\pgfpathlineto{\pgfqpoint{2.259463in}{2.653720in}}%
\pgfpathlineto{\pgfqpoint{2.258222in}{2.656721in}}%
\pgfpathlineto{\pgfqpoint{2.255386in}{2.664015in}}%
\pgfpathlineto{\pgfqpoint{2.251114in}{2.674309in}}%
\pgfpathlineto{\pgfqpoint{2.247645in}{2.682160in}}%
\pgfpathlineto{\pgfqpoint{2.246618in}{2.684604in}}%
\pgfpathlineto{\pgfqpoint{2.241973in}{2.694898in}}%
\pgfpathlineto{\pgfqpoint{2.237093in}{2.705193in}}%
\pgfpathlineto{\pgfqpoint{2.237068in}{2.705243in}}%
\pgfpathlineto{\pgfqpoint{2.232123in}{2.715487in}}%
\pgfpathlineto{\pgfqpoint{2.226998in}{2.725781in}}%
\pgfpathlineto{\pgfqpoint{2.226491in}{2.726766in}}%
\pgfpathlineto{\pgfqpoint{2.221863in}{2.736076in}}%
\pgfpathlineto{\pgfqpoint{2.216734in}{2.746370in}}%
\pgfpathlineto{\pgfqpoint{2.215914in}{2.748035in}}%
\pgfpathlineto{\pgfqpoint{2.211776in}{2.756665in}}%
\pgfpathlineto{\pgfqpoint{2.207104in}{2.766959in}}%
\pgfpathlineto{\pgfqpoint{2.205337in}{2.771257in}}%
\pgfpathlineto{\pgfqpoint{2.202924in}{2.777254in}}%
\pgfpathlineto{\pgfqpoint{2.199388in}{2.787548in}}%
\pgfpathlineto{\pgfqpoint{2.196508in}{2.797842in}}%
\pgfpathlineto{\pgfqpoint{2.194760in}{2.805629in}}%
\pgfpathlineto{\pgfqpoint{2.194206in}{2.808137in}}%
\pgfpathlineto{\pgfqpoint{2.192380in}{2.818431in}}%
\pgfpathlineto{\pgfqpoint{2.190921in}{2.828726in}}%
\pgfpathlineto{\pgfqpoint{2.189771in}{2.839020in}}%
\pgfpathlineto{\pgfqpoint{2.188898in}{2.849315in}}%
\pgfpathlineto{\pgfqpoint{2.188286in}{2.859609in}}%
\pgfpathlineto{\pgfqpoint{2.187931in}{2.869903in}}%
\pgfpathlineto{\pgfqpoint{2.187843in}{2.880198in}}%
\pgfpathlineto{\pgfqpoint{2.188041in}{2.890492in}}%
\pgfpathlineto{\pgfqpoint{2.188563in}{2.900787in}}%
\pgfpathlineto{\pgfqpoint{2.189468in}{2.911081in}}%
\pgfpathlineto{\pgfqpoint{2.184183in}{2.911081in}}%
\pgfpathlineto{\pgfqpoint{2.173606in}{2.911081in}}%
\pgfpathlineto{\pgfqpoint{2.163029in}{2.911081in}}%
\pgfpathlineto{\pgfqpoint{2.154824in}{2.911081in}}%
\pgfpathlineto{\pgfqpoint{2.155758in}{2.900787in}}%
\pgfpathlineto{\pgfqpoint{2.156750in}{2.890492in}}%
\pgfpathlineto{\pgfqpoint{2.157792in}{2.880198in}}%
\pgfpathlineto{\pgfqpoint{2.158890in}{2.869903in}}%
\pgfpathlineto{\pgfqpoint{2.160056in}{2.859609in}}%
\pgfpathlineto{\pgfqpoint{2.161304in}{2.849315in}}%
\pgfpathlineto{\pgfqpoint{2.162653in}{2.839020in}}%
\pgfpathlineto{\pgfqpoint{2.163029in}{2.836422in}}%
\pgfpathlineto{\pgfqpoint{2.164108in}{2.828726in}}%
\pgfpathlineto{\pgfqpoint{2.165709in}{2.818431in}}%
\pgfpathlineto{\pgfqpoint{2.167510in}{2.808137in}}%
\pgfpathlineto{\pgfqpoint{2.169594in}{2.797842in}}%
\pgfpathlineto{\pgfqpoint{2.172108in}{2.787548in}}%
\pgfpathlineto{\pgfqpoint{2.173606in}{2.782595in}}%
\pgfpathlineto{\pgfqpoint{2.175236in}{2.777254in}}%
\pgfpathlineto{\pgfqpoint{2.179018in}{2.766959in}}%
\pgfpathlineto{\pgfqpoint{2.183278in}{2.756665in}}%
\pgfpathlineto{\pgfqpoint{2.184183in}{2.754560in}}%
\pgfpathlineto{\pgfqpoint{2.187737in}{2.746370in}}%
\pgfpathlineto{\pgfqpoint{2.192249in}{2.736076in}}%
\pgfpathlineto{\pgfqpoint{2.194760in}{2.730250in}}%
\pgfpathlineto{\pgfqpoint{2.196719in}{2.725781in}}%
\pgfpathlineto{\pgfqpoint{2.201069in}{2.715487in}}%
\pgfpathlineto{\pgfqpoint{2.205268in}{2.705193in}}%
\pgfpathlineto{\pgfqpoint{2.205337in}{2.705014in}}%
\pgfpathlineto{\pgfqpoint{2.209280in}{2.694898in}}%
\pgfpathlineto{\pgfqpoint{2.213079in}{2.684604in}}%
\pgfpathlineto{\pgfqpoint{2.215914in}{2.676412in}}%
\pgfpathlineto{\pgfqpoint{2.216660in}{2.674309in}}%
\pgfpathlineto{\pgfqpoint{2.220029in}{2.664015in}}%
\pgfpathlineto{\pgfqpoint{2.223177in}{2.653720in}}%
\pgfpathlineto{\pgfqpoint{2.226121in}{2.643426in}}%
\pgfpathlineto{\pgfqpoint{2.226491in}{2.642030in}}%
\pgfpathlineto{\pgfqpoint{2.228921in}{2.633132in}}%
\pgfpathlineto{\pgfqpoint{2.231576in}{2.622837in}}%
\pgfpathlineto{\pgfqpoint{2.234117in}{2.612543in}}%
\pgfpathlineto{\pgfqpoint{2.236580in}{2.602248in}}%
\pgfpathlineto{\pgfqpoint{2.237068in}{2.600156in}}%
\pgfpathlineto{\pgfqpoint{2.239080in}{2.591954in}}%
\pgfpathlineto{\pgfqpoint{2.241630in}{2.581659in}}%
\pgfpathlineto{\pgfqpoint{2.244293in}{2.571365in}}%
\pgfpathlineto{\pgfqpoint{2.247270in}{2.561071in}}%
\pgfpathlineto{\pgfqpoint{2.247645in}{2.560068in}}%
\pgfpathlineto{\pgfqpoint{2.251664in}{2.550776in}}%
\pgfpathlineto{\pgfqpoint{2.257638in}{2.540482in}}%
\pgfpathlineto{\pgfqpoint{2.258222in}{2.539561in}}%
\pgfpathlineto{\pgfqpoint{2.264829in}{2.530187in}}%
\pgfpathlineto{\pgfqpoint{2.268799in}{2.524728in}}%
\pgfpathlineto{\pgfqpoint{2.272793in}{2.519893in}}%
\pgfpathlineto{\pgfqpoint{2.279376in}{2.512027in}}%
\pgfpathlineto{\pgfqpoint{2.281738in}{2.509598in}}%
\pgfpathlineto{\pgfqpoint{2.289953in}{2.501154in}}%
\pgfpathlineto{\pgfqpoint{2.292100in}{2.499304in}}%
\pgfpathlineto{\pgfqpoint{2.300530in}{2.491965in}}%
\pgfpathlineto{\pgfqpoint{2.304709in}{2.489009in}}%
\pgfpathlineto{\pgfqpoint{2.311107in}{2.484400in}}%
\pgfpathlineto{\pgfqpoint{2.321231in}{2.478715in}}%
\pgfpathlineto{\pgfqpoint{2.321684in}{2.478454in}}%
\pgfpathlineto{\pgfqpoint{2.332262in}{2.473887in}}%
\pgfpathlineto{\pgfqpoint{2.342839in}{2.470694in}}%
\pgfpathlineto{\pgfqpoint{2.353416in}{2.468603in}}%
\pgfpathlineto{\pgfqpoint{2.354893in}{2.468421in}}%
\pgfpathlineto{\pgfqpoint{2.363993in}{2.467272in}}%
\pgfpathlineto{\pgfqpoint{2.374570in}{2.466475in}}%
\pgfpathlineto{\pgfqpoint{2.385147in}{2.465928in}}%
\pgfpathlineto{\pgfqpoint{2.395724in}{2.465361in}}%
\pgfpathlineto{\pgfqpoint{2.406301in}{2.464540in}}%
\pgfpathlineto{\pgfqpoint{2.416878in}{2.463263in}}%
\pgfpathlineto{\pgfqpoint{2.427455in}{2.461364in}}%
\pgfpathlineto{\pgfqpoint{2.438032in}{2.458714in}}%
\pgfpathlineto{\pgfqpoint{2.439772in}{2.458126in}}%
\pgfpathlineto{\pgfqpoint{2.448609in}{2.455154in}}%
\pgfpathlineto{\pgfqpoint{2.459186in}{2.450687in}}%
\pgfpathlineto{\pgfqpoint{2.464734in}{2.447832in}}%
\pgfpathlineto{\pgfqpoint{2.469763in}{2.445266in}}%
\pgfpathlineto{\pgfqpoint{2.480340in}{2.438913in}}%
\pgfpathlineto{\pgfqpoint{2.482325in}{2.437537in}}%
\pgfpathlineto{\pgfqpoint{2.490917in}{2.431639in}}%
\pgfpathlineto{\pgfqpoint{2.496646in}{2.427243in}}%
\pgfpathlineto{\pgfqpoint{2.501495in}{2.423551in}}%
\pgfpathlineto{\pgfqpoint{2.509389in}{2.416948in}}%
\pgfpathlineto{\pgfqpoint{2.512072in}{2.414713in}}%
\pgfpathlineto{\pgfqpoint{2.521021in}{2.406654in}}%
\pgfpathlineto{\pgfqpoint{2.522649in}{2.405185in}}%
\pgfpathlineto{\pgfqpoint{2.531833in}{2.396360in}}%
\pgfpathlineto{\pgfqpoint{2.533226in}{2.395009in}}%
\pgfpathlineto{\pgfqpoint{2.542004in}{2.386065in}}%
\pgfpathlineto{\pgfqpoint{2.543803in}{2.384199in}}%
\pgfpathlineto{\pgfqpoint{2.551640in}{2.375771in}}%
\pgfpathlineto{\pgfqpoint{2.554380in}{2.372742in}}%
\pgfpathlineto{\pgfqpoint{2.560798in}{2.365476in}}%
\pgfpathlineto{\pgfqpoint{2.564957in}{2.360588in}}%
\pgfpathlineto{\pgfqpoint{2.569504in}{2.355182in}}%
\pgfpathlineto{\pgfqpoint{2.575534in}{2.347663in}}%
\pgfpathlineto{\pgfqpoint{2.577761in}{2.344887in}}%
\pgfpathlineto{\pgfqpoint{2.585559in}{2.334593in}}%
\pgfpathlineto{\pgfqpoint{2.586111in}{2.333817in}}%
\pgfpathlineto{\pgfqpoint{2.592952in}{2.324299in}}%
\pgfpathlineto{\pgfqpoint{2.596688in}{2.318731in}}%
\pgfpathlineto{\pgfqpoint{2.599937in}{2.314004in}}%
\pgfpathlineto{\pgfqpoint{2.606510in}{2.303710in}}%
\pgfpathlineto{\pgfqpoint{2.607265in}{2.302440in}}%
\pgfpathlineto{\pgfqpoint{2.612841in}{2.293415in}}%
\pgfpathlineto{\pgfqpoint{2.617842in}{2.284701in}}%
\pgfpathlineto{\pgfqpoint{2.618803in}{2.283121in}}%
\pgfpathlineto{\pgfqpoint{2.624670in}{2.272826in}}%
\pgfpathlineto{\pgfqpoint{2.628419in}{2.265843in}}%
\pgfpathlineto{\pgfqpoint{2.630349in}{2.262532in}}%
\pgfpathlineto{\pgfqpoint{2.636095in}{2.252238in}}%
\pgfpathlineto{\pgfqpoint{2.638996in}{2.246895in}}%
\pgfpathlineto{\pgfqpoint{2.642023in}{2.241943in}}%
\pgfpathlineto{\pgfqpoint{2.648355in}{2.231649in}}%
\pgfpathlineto{\pgfqpoint{2.649573in}{2.229746in}}%
\pgfpathlineto{\pgfqpoint{2.655967in}{2.221354in}}%
\pgfpathlineto{\pgfqpoint{2.660150in}{2.216390in}}%
\pgfpathlineto{\pgfqpoint{2.665751in}{2.211060in}}%
\pgfpathlineto{\pgfqpoint{2.670727in}{2.206806in}}%
\pgfpathlineto{\pgfqpoint{2.679884in}{2.200765in}}%
\pgfpathclose%
\pgfusepath{fill}%
\end{pgfscope}%
\begin{pgfscope}%
\pgfpathrectangle{\pgfqpoint{1.856795in}{1.819814in}}{\pgfqpoint{1.194205in}{1.163386in}}%
\pgfusepath{clip}%
\pgfsetbuttcap%
\pgfsetroundjoin%
\definecolor{currentfill}{rgb}{0.966812,0.762584,0.633643}%
\pgfsetfillcolor{currentfill}%
\pgfsetlinewidth{0.000000pt}%
\definecolor{currentstroke}{rgb}{0.000000,0.000000,0.000000}%
\pgfsetstrokecolor{currentstroke}%
\pgfsetdash{}{0pt}%
\pgfpathmoveto{\pgfqpoint{2.363993in}{1.891932in}}%
\pgfpathlineto{\pgfqpoint{2.374570in}{1.891932in}}%
\pgfpathlineto{\pgfqpoint{2.385147in}{1.891932in}}%
\pgfpathlineto{\pgfqpoint{2.395724in}{1.891932in}}%
\pgfpathlineto{\pgfqpoint{2.399201in}{1.891932in}}%
\pgfpathlineto{\pgfqpoint{2.396411in}{1.902227in}}%
\pgfpathlineto{\pgfqpoint{2.395724in}{1.904726in}}%
\pgfpathlineto{\pgfqpoint{2.393707in}{1.912521in}}%
\pgfpathlineto{\pgfqpoint{2.391044in}{1.922816in}}%
\pgfpathlineto{\pgfqpoint{2.388391in}{1.933110in}}%
\pgfpathlineto{\pgfqpoint{2.385740in}{1.943405in}}%
\pgfpathlineto{\pgfqpoint{2.385147in}{1.945700in}}%
\pgfpathlineto{\pgfqpoint{2.383148in}{1.953699in}}%
\pgfpathlineto{\pgfqpoint{2.380551in}{1.963993in}}%
\pgfpathlineto{\pgfqpoint{2.377913in}{1.974288in}}%
\pgfpathlineto{\pgfqpoint{2.375201in}{1.984582in}}%
\pgfpathlineto{\pgfqpoint{2.374570in}{1.986865in}}%
\pgfpathlineto{\pgfqpoint{2.372366in}{1.994877in}}%
\pgfpathlineto{\pgfqpoint{2.368901in}{2.005171in}}%
\pgfpathlineto{\pgfqpoint{2.364668in}{2.015466in}}%
\pgfpathlineto{\pgfqpoint{2.363993in}{2.017105in}}%
\pgfpathlineto{\pgfqpoint{2.360297in}{2.025760in}}%
\pgfpathlineto{\pgfqpoint{2.355962in}{2.036054in}}%
\pgfpathlineto{\pgfqpoint{2.353416in}{2.042363in}}%
\pgfpathlineto{\pgfqpoint{2.351721in}{2.046349in}}%
\pgfpathlineto{\pgfqpoint{2.347606in}{2.056643in}}%
\pgfpathlineto{\pgfqpoint{2.343687in}{2.066938in}}%
\pgfpathlineto{\pgfqpoint{2.342839in}{2.069384in}}%
\pgfpathlineto{\pgfqpoint{2.339921in}{2.077232in}}%
\pgfpathlineto{\pgfqpoint{2.336395in}{2.087527in}}%
\pgfpathlineto{\pgfqpoint{2.333170in}{2.097821in}}%
\pgfpathlineto{\pgfqpoint{2.332262in}{2.101128in}}%
\pgfpathlineto{\pgfqpoint{2.330165in}{2.108116in}}%
\pgfpathlineto{\pgfqpoint{2.327455in}{2.118410in}}%
\pgfpathlineto{\pgfqpoint{2.325098in}{2.128704in}}%
\pgfpathlineto{\pgfqpoint{2.323095in}{2.138999in}}%
\pgfpathlineto{\pgfqpoint{2.321684in}{2.147763in}}%
\pgfpathlineto{\pgfqpoint{2.321408in}{2.149293in}}%
\pgfpathlineto{\pgfqpoint{2.319897in}{2.159588in}}%
\pgfpathlineto{\pgfqpoint{2.318580in}{2.169882in}}%
\pgfpathlineto{\pgfqpoint{2.317184in}{2.180177in}}%
\pgfpathlineto{\pgfqpoint{2.315225in}{2.190471in}}%
\pgfpathlineto{\pgfqpoint{2.312117in}{2.200765in}}%
\pgfpathlineto{\pgfqpoint{2.311107in}{2.203335in}}%
\pgfpathlineto{\pgfqpoint{2.307838in}{2.211060in}}%
\pgfpathlineto{\pgfqpoint{2.302744in}{2.221354in}}%
\pgfpathlineto{\pgfqpoint{2.300530in}{2.225435in}}%
\pgfpathlineto{\pgfqpoint{2.296726in}{2.231649in}}%
\pgfpathlineto{\pgfqpoint{2.289953in}{2.241492in}}%
\pgfpathlineto{\pgfqpoint{2.289593in}{2.241943in}}%
\pgfpathlineto{\pgfqpoint{2.280796in}{2.252238in}}%
\pgfpathlineto{\pgfqpoint{2.279376in}{2.253761in}}%
\pgfpathlineto{\pgfqpoint{2.269602in}{2.262532in}}%
\pgfpathlineto{\pgfqpoint{2.268799in}{2.263204in}}%
\pgfpathlineto{\pgfqpoint{2.258222in}{2.270599in}}%
\pgfpathlineto{\pgfqpoint{2.254100in}{2.272826in}}%
\pgfpathlineto{\pgfqpoint{2.247645in}{2.276270in}}%
\pgfpathlineto{\pgfqpoint{2.237068in}{2.280003in}}%
\pgfpathlineto{\pgfqpoint{2.226491in}{2.280872in}}%
\pgfpathlineto{\pgfqpoint{2.215914in}{2.277703in}}%
\pgfpathlineto{\pgfqpoint{2.205397in}{2.272826in}}%
\pgfpathlineto{\pgfqpoint{2.205337in}{2.272800in}}%
\pgfpathlineto{\pgfqpoint{2.194760in}{2.269199in}}%
\pgfpathlineto{\pgfqpoint{2.184183in}{2.267943in}}%
\pgfpathlineto{\pgfqpoint{2.173606in}{2.269236in}}%
\pgfpathlineto{\pgfqpoint{2.163632in}{2.272826in}}%
\pgfpathlineto{\pgfqpoint{2.163029in}{2.273031in}}%
\pgfpathlineto{\pgfqpoint{2.152451in}{2.279463in}}%
\pgfpathlineto{\pgfqpoint{2.148251in}{2.283121in}}%
\pgfpathlineto{\pgfqpoint{2.141874in}{2.288376in}}%
\pgfpathlineto{\pgfqpoint{2.137364in}{2.293415in}}%
\pgfpathlineto{\pgfqpoint{2.131297in}{2.299815in}}%
\pgfpathlineto{\pgfqpoint{2.128461in}{2.303710in}}%
\pgfpathlineto{\pgfqpoint{2.120720in}{2.313713in}}%
\pgfpathlineto{\pgfqpoint{2.120541in}{2.314004in}}%
\pgfpathlineto{\pgfqpoint{2.114023in}{2.324299in}}%
\pgfpathlineto{\pgfqpoint{2.110143in}{2.330024in}}%
\pgfpathlineto{\pgfqpoint{2.107642in}{2.334593in}}%
\pgfpathlineto{\pgfqpoint{2.101698in}{2.344887in}}%
\pgfpathlineto{\pgfqpoint{2.099566in}{2.348394in}}%
\pgfpathlineto{\pgfqpoint{2.096170in}{2.355182in}}%
\pgfpathlineto{\pgfqpoint{2.090702in}{2.365476in}}%
\pgfpathlineto{\pgfqpoint{2.088989in}{2.368561in}}%
\pgfpathlineto{\pgfqpoint{2.085635in}{2.375771in}}%
\pgfpathlineto{\pgfqpoint{2.080577in}{2.386065in}}%
\pgfpathlineto{\pgfqpoint{2.078412in}{2.390291in}}%
\pgfpathlineto{\pgfqpoint{2.075761in}{2.396360in}}%
\pgfpathlineto{\pgfqpoint{2.071076in}{2.406654in}}%
\pgfpathlineto{\pgfqpoint{2.067835in}{2.413481in}}%
\pgfpathlineto{\pgfqpoint{2.066405in}{2.416948in}}%
\pgfpathlineto{\pgfqpoint{2.062069in}{2.427243in}}%
\pgfpathlineto{\pgfqpoint{2.057529in}{2.437537in}}%
\pgfpathlineto{\pgfqpoint{2.057258in}{2.438147in}}%
\pgfpathlineto{\pgfqpoint{2.053467in}{2.447832in}}%
\pgfpathlineto{\pgfqpoint{2.049293in}{2.458126in}}%
\pgfpathlineto{\pgfqpoint{2.046681in}{2.464428in}}%
\pgfpathlineto{\pgfqpoint{2.045192in}{2.468421in}}%
\pgfpathlineto{\pgfqpoint{2.041309in}{2.478715in}}%
\pgfpathlineto{\pgfqpoint{2.037285in}{2.489009in}}%
\pgfpathlineto{\pgfqpoint{2.036104in}{2.491998in}}%
\pgfpathlineto{\pgfqpoint{2.033470in}{2.499304in}}%
\pgfpathlineto{\pgfqpoint{2.029668in}{2.509598in}}%
\pgfpathlineto{\pgfqpoint{2.025728in}{2.519893in}}%
\pgfpathlineto{\pgfqpoint{2.025527in}{2.520419in}}%
\pgfpathlineto{\pgfqpoint{2.022140in}{2.530187in}}%
\pgfpathlineto{\pgfqpoint{2.018527in}{2.540482in}}%
\pgfpathlineto{\pgfqpoint{2.014950in}{2.550622in}}%
\pgfpathlineto{\pgfqpoint{2.014901in}{2.550776in}}%
\pgfpathlineto{\pgfqpoint{2.011730in}{2.561071in}}%
\pgfpathlineto{\pgfqpoint{2.008614in}{2.571365in}}%
\pgfpathlineto{\pgfqpoint{2.005555in}{2.581659in}}%
\pgfpathlineto{\pgfqpoint{2.004373in}{2.585745in}}%
\pgfpathlineto{\pgfqpoint{2.002765in}{2.591954in}}%
\pgfpathlineto{\pgfqpoint{2.000160in}{2.602248in}}%
\pgfpathlineto{\pgfqpoint{1.997568in}{2.612543in}}%
\pgfpathlineto{\pgfqpoint{1.994944in}{2.622837in}}%
\pgfpathlineto{\pgfqpoint{1.993796in}{2.627264in}}%
\pgfpathlineto{\pgfqpoint{1.992415in}{2.633132in}}%
\pgfpathlineto{\pgfqpoint{1.989899in}{2.643426in}}%
\pgfpathlineto{\pgfqpoint{1.987241in}{2.653720in}}%
\pgfpathlineto{\pgfqpoint{1.984448in}{2.664015in}}%
\pgfpathlineto{\pgfqpoint{1.983218in}{2.668458in}}%
\pgfpathlineto{\pgfqpoint{1.981713in}{2.674309in}}%
\pgfpathlineto{\pgfqpoint{1.979029in}{2.684604in}}%
\pgfpathlineto{\pgfqpoint{1.976285in}{2.694898in}}%
\pgfpathlineto{\pgfqpoint{1.973489in}{2.705193in}}%
\pgfpathlineto{\pgfqpoint{1.972641in}{2.708335in}}%
\pgfpathlineto{\pgfqpoint{1.970829in}{2.715487in}}%
\pgfpathlineto{\pgfqpoint{1.968209in}{2.725781in}}%
\pgfpathlineto{\pgfqpoint{1.965549in}{2.736076in}}%
\pgfpathlineto{\pgfqpoint{1.962839in}{2.746370in}}%
\pgfpathlineto{\pgfqpoint{1.962064in}{2.749321in}}%
\pgfpathlineto{\pgfqpoint{1.960223in}{2.756665in}}%
\pgfpathlineto{\pgfqpoint{1.957590in}{2.766959in}}%
\pgfpathlineto{\pgfqpoint{1.954853in}{2.777254in}}%
\pgfpathlineto{\pgfqpoint{1.951978in}{2.787548in}}%
\pgfpathlineto{\pgfqpoint{1.951487in}{2.789275in}}%
\pgfpathlineto{\pgfqpoint{1.949114in}{2.797842in}}%
\pgfpathlineto{\pgfqpoint{1.946128in}{2.808137in}}%
\pgfpathlineto{\pgfqpoint{1.943001in}{2.818431in}}%
\pgfpathlineto{\pgfqpoint{1.940910in}{2.825227in}}%
\pgfpathlineto{\pgfqpoint{1.939840in}{2.828726in}}%
\pgfpathlineto{\pgfqpoint{1.936755in}{2.839020in}}%
\pgfpathlineto{\pgfqpoint{1.933714in}{2.849315in}}%
\pgfpathlineto{\pgfqpoint{1.930819in}{2.859609in}}%
\pgfpathlineto{\pgfqpoint{1.930333in}{2.861617in}}%
\pgfpathlineto{\pgfqpoint{1.930333in}{2.859609in}}%
\pgfpathlineto{\pgfqpoint{1.930333in}{2.849315in}}%
\pgfpathlineto{\pgfqpoint{1.930333in}{2.839020in}}%
\pgfpathlineto{\pgfqpoint{1.930333in}{2.828726in}}%
\pgfpathlineto{\pgfqpoint{1.930333in}{2.818431in}}%
\pgfpathlineto{\pgfqpoint{1.930333in}{2.808137in}}%
\pgfpathlineto{\pgfqpoint{1.930333in}{2.797842in}}%
\pgfpathlineto{\pgfqpoint{1.930333in}{2.787548in}}%
\pgfpathlineto{\pgfqpoint{1.930333in}{2.777960in}}%
\pgfpathlineto{\pgfqpoint{1.930566in}{2.777254in}}%
\pgfpathlineto{\pgfqpoint{1.933894in}{2.766959in}}%
\pgfpathlineto{\pgfqpoint{1.937055in}{2.756665in}}%
\pgfpathlineto{\pgfqpoint{1.940084in}{2.746370in}}%
\pgfpathlineto{\pgfqpoint{1.940910in}{2.743531in}}%
\pgfpathlineto{\pgfqpoint{1.943118in}{2.736076in}}%
\pgfpathlineto{\pgfqpoint{1.946108in}{2.725781in}}%
\pgfpathlineto{\pgfqpoint{1.949031in}{2.715487in}}%
\pgfpathlineto{\pgfqpoint{1.951487in}{2.706708in}}%
\pgfpathlineto{\pgfqpoint{1.951925in}{2.705193in}}%
\pgfpathlineto{\pgfqpoint{1.954925in}{2.694898in}}%
\pgfpathlineto{\pgfqpoint{1.957873in}{2.684604in}}%
\pgfpathlineto{\pgfqpoint{1.960769in}{2.674309in}}%
\pgfpathlineto{\pgfqpoint{1.962064in}{2.669687in}}%
\pgfpathlineto{\pgfqpoint{1.963720in}{2.664015in}}%
\pgfpathlineto{\pgfqpoint{1.966679in}{2.653720in}}%
\pgfpathlineto{\pgfqpoint{1.969509in}{2.643426in}}%
\pgfpathlineto{\pgfqpoint{1.972151in}{2.633132in}}%
\pgfpathlineto{\pgfqpoint{1.972641in}{2.631080in}}%
\pgfpathlineto{\pgfqpoint{1.974746in}{2.622837in}}%
\pgfpathlineto{\pgfqpoint{1.977189in}{2.612543in}}%
\pgfpathlineto{\pgfqpoint{1.979529in}{2.602248in}}%
\pgfpathlineto{\pgfqpoint{1.981851in}{2.591954in}}%
\pgfpathlineto{\pgfqpoint{1.983218in}{2.586000in}}%
\pgfpathlineto{\pgfqpoint{1.984308in}{2.581659in}}%
\pgfpathlineto{\pgfqpoint{1.986960in}{2.571365in}}%
\pgfpathlineto{\pgfqpoint{1.989691in}{2.561071in}}%
\pgfpathlineto{\pgfqpoint{1.992512in}{2.550776in}}%
\pgfpathlineto{\pgfqpoint{1.993796in}{2.546266in}}%
\pgfpathlineto{\pgfqpoint{1.995586in}{2.540482in}}%
\pgfpathlineto{\pgfqpoint{1.998862in}{2.530187in}}%
\pgfpathlineto{\pgfqpoint{2.002197in}{2.519893in}}%
\pgfpathlineto{\pgfqpoint{2.004373in}{2.513296in}}%
\pgfpathlineto{\pgfqpoint{2.005687in}{2.509598in}}%
\pgfpathlineto{\pgfqpoint{2.009386in}{2.499304in}}%
\pgfpathlineto{\pgfqpoint{2.013049in}{2.489009in}}%
\pgfpathlineto{\pgfqpoint{2.014950in}{2.483661in}}%
\pgfpathlineto{\pgfqpoint{2.016822in}{2.478715in}}%
\pgfpathlineto{\pgfqpoint{2.020685in}{2.468421in}}%
\pgfpathlineto{\pgfqpoint{2.024443in}{2.458126in}}%
\pgfpathlineto{\pgfqpoint{2.025527in}{2.455137in}}%
\pgfpathlineto{\pgfqpoint{2.028359in}{2.447832in}}%
\pgfpathlineto{\pgfqpoint{2.032257in}{2.437537in}}%
\pgfpathlineto{\pgfqpoint{2.036019in}{2.427243in}}%
\pgfpathlineto{\pgfqpoint{2.036104in}{2.427009in}}%
\pgfpathlineto{\pgfqpoint{2.040073in}{2.416948in}}%
\pgfpathlineto{\pgfqpoint{2.043976in}{2.406654in}}%
\pgfpathlineto{\pgfqpoint{2.046681in}{2.399278in}}%
\pgfpathlineto{\pgfqpoint{2.047861in}{2.396360in}}%
\pgfpathlineto{\pgfqpoint{2.051933in}{2.386065in}}%
\pgfpathlineto{\pgfqpoint{2.055816in}{2.375771in}}%
\pgfpathlineto{\pgfqpoint{2.057258in}{2.371836in}}%
\pgfpathlineto{\pgfqpoint{2.059864in}{2.365476in}}%
\pgfpathlineto{\pgfqpoint{2.063923in}{2.355182in}}%
\pgfpathlineto{\pgfqpoint{2.067785in}{2.344887in}}%
\pgfpathlineto{\pgfqpoint{2.067835in}{2.344753in}}%
\pgfpathlineto{\pgfqpoint{2.072069in}{2.334593in}}%
\pgfpathlineto{\pgfqpoint{2.076156in}{2.324299in}}%
\pgfpathlineto{\pgfqpoint{2.078412in}{2.318436in}}%
\pgfpathlineto{\pgfqpoint{2.080363in}{2.314004in}}%
\pgfpathlineto{\pgfqpoint{2.084775in}{2.303710in}}%
\pgfpathlineto{\pgfqpoint{2.088989in}{2.293491in}}%
\pgfpathlineto{\pgfqpoint{2.089025in}{2.293415in}}%
\pgfpathlineto{\pgfqpoint{2.093901in}{2.283121in}}%
\pgfpathlineto{\pgfqpoint{2.098612in}{2.272826in}}%
\pgfpathlineto{\pgfqpoint{2.099566in}{2.270739in}}%
\pgfpathlineto{\pgfqpoint{2.103915in}{2.262532in}}%
\pgfpathlineto{\pgfqpoint{2.109233in}{2.252238in}}%
\pgfpathlineto{\pgfqpoint{2.110143in}{2.250483in}}%
\pgfpathlineto{\pgfqpoint{2.115293in}{2.241943in}}%
\pgfpathlineto{\pgfqpoint{2.120720in}{2.232701in}}%
\pgfpathlineto{\pgfqpoint{2.121444in}{2.231649in}}%
\pgfpathlineto{\pgfqpoint{2.128489in}{2.221354in}}%
\pgfpathlineto{\pgfqpoint{2.131297in}{2.217163in}}%
\pgfpathlineto{\pgfqpoint{2.136058in}{2.211060in}}%
\pgfpathlineto{\pgfqpoint{2.141874in}{2.203360in}}%
\pgfpathlineto{\pgfqpoint{2.144156in}{2.200765in}}%
\pgfpathlineto{\pgfqpoint{2.152451in}{2.190955in}}%
\pgfpathlineto{\pgfqpoint{2.152925in}{2.190471in}}%
\pgfpathlineto{\pgfqpoint{2.162587in}{2.180177in}}%
\pgfpathlineto{\pgfqpoint{2.163029in}{2.179685in}}%
\pgfpathlineto{\pgfqpoint{2.172984in}{2.169882in}}%
\pgfpathlineto{\pgfqpoint{2.173606in}{2.169236in}}%
\pgfpathlineto{\pgfqpoint{2.183952in}{2.159588in}}%
\pgfpathlineto{\pgfqpoint{2.184183in}{2.159360in}}%
\pgfpathlineto{\pgfqpoint{2.194760in}{2.149899in}}%
\pgfpathlineto{\pgfqpoint{2.195488in}{2.149293in}}%
\pgfpathlineto{\pgfqpoint{2.205337in}{2.140685in}}%
\pgfpathlineto{\pgfqpoint{2.207342in}{2.138999in}}%
\pgfpathlineto{\pgfqpoint{2.215914in}{2.131540in}}%
\pgfpathlineto{\pgfqpoint{2.219194in}{2.128704in}}%
\pgfpathlineto{\pgfqpoint{2.226491in}{2.122298in}}%
\pgfpathlineto{\pgfqpoint{2.230809in}{2.118410in}}%
\pgfpathlineto{\pgfqpoint{2.237068in}{2.112792in}}%
\pgfpathlineto{\pgfqpoint{2.242002in}{2.108116in}}%
\pgfpathlineto{\pgfqpoint{2.247645in}{2.102867in}}%
\pgfpathlineto{\pgfqpoint{2.252662in}{2.097821in}}%
\pgfpathlineto{\pgfqpoint{2.258222in}{2.092389in}}%
\pgfpathlineto{\pgfqpoint{2.262747in}{2.087527in}}%
\pgfpathlineto{\pgfqpoint{2.268799in}{2.081241in}}%
\pgfpathlineto{\pgfqpoint{2.272274in}{2.077232in}}%
\pgfpathlineto{\pgfqpoint{2.279376in}{2.069317in}}%
\pgfpathlineto{\pgfqpoint{2.281291in}{2.066938in}}%
\pgfpathlineto{\pgfqpoint{2.289851in}{2.056643in}}%
\pgfpathlineto{\pgfqpoint{2.289953in}{2.056525in}}%
\pgfpathlineto{\pgfqpoint{2.297917in}{2.046349in}}%
\pgfpathlineto{\pgfqpoint{2.300530in}{2.043077in}}%
\pgfpathlineto{\pgfqpoint{2.305619in}{2.036054in}}%
\pgfpathlineto{\pgfqpoint{2.311107in}{2.028570in}}%
\pgfpathlineto{\pgfqpoint{2.312992in}{2.025760in}}%
\pgfpathlineto{\pgfqpoint{2.320000in}{2.015466in}}%
\pgfpathlineto{\pgfqpoint{2.321684in}{2.012982in}}%
\pgfpathlineto{\pgfqpoint{2.326579in}{2.005171in}}%
\pgfpathlineto{\pgfqpoint{2.332047in}{1.994877in}}%
\pgfpathlineto{\pgfqpoint{2.332262in}{1.994276in}}%
\pgfpathlineto{\pgfqpoint{2.335636in}{1.984582in}}%
\pgfpathlineto{\pgfqpoint{2.338676in}{1.974288in}}%
\pgfpathlineto{\pgfqpoint{2.341509in}{1.963993in}}%
\pgfpathlineto{\pgfqpoint{2.342839in}{1.958880in}}%
\pgfpathlineto{\pgfqpoint{2.344183in}{1.953699in}}%
\pgfpathlineto{\pgfqpoint{2.346708in}{1.943405in}}%
\pgfpathlineto{\pgfqpoint{2.349082in}{1.933110in}}%
\pgfpathlineto{\pgfqpoint{2.351299in}{1.922816in}}%
\pgfpathlineto{\pgfqpoint{2.353346in}{1.912521in}}%
\pgfpathlineto{\pgfqpoint{2.353416in}{1.912129in}}%
\pgfpathlineto{\pgfqpoint{2.355218in}{1.902227in}}%
\pgfpathlineto{\pgfqpoint{2.356858in}{1.891932in}}%
\pgfpathclose%
\pgfusepath{fill}%
\end{pgfscope}%
\begin{pgfscope}%
\pgfpathrectangle{\pgfqpoint{1.856795in}{1.819814in}}{\pgfqpoint{1.194205in}{1.163386in}}%
\pgfusepath{clip}%
\pgfsetbuttcap%
\pgfsetroundjoin%
\definecolor{currentfill}{rgb}{0.966812,0.762584,0.633643}%
\pgfsetfillcolor{currentfill}%
\pgfsetlinewidth{0.000000pt}%
\definecolor{currentstroke}{rgb}{0.000000,0.000000,0.000000}%
\pgfsetstrokecolor{currentstroke}%
\pgfsetdash{}{0pt}%
\pgfpathmoveto{\pgfqpoint{2.713036in}{1.900897in}}%
\pgfpathlineto{\pgfqpoint{2.723613in}{1.901905in}}%
\pgfpathlineto{\pgfqpoint{2.723898in}{1.902227in}}%
\pgfpathlineto{\pgfqpoint{2.731038in}{1.912521in}}%
\pgfpathlineto{\pgfqpoint{2.734190in}{1.918090in}}%
\pgfpathlineto{\pgfqpoint{2.735583in}{1.922816in}}%
\pgfpathlineto{\pgfqpoint{2.738342in}{1.933110in}}%
\pgfpathlineto{\pgfqpoint{2.741023in}{1.943405in}}%
\pgfpathlineto{\pgfqpoint{2.743795in}{1.953699in}}%
\pgfpathlineto{\pgfqpoint{2.744767in}{1.957042in}}%
\pgfpathlineto{\pgfqpoint{2.746156in}{1.963993in}}%
\pgfpathlineto{\pgfqpoint{2.748202in}{1.974288in}}%
\pgfpathlineto{\pgfqpoint{2.747837in}{1.984582in}}%
\pgfpathlineto{\pgfqpoint{2.747949in}{1.994877in}}%
\pgfpathlineto{\pgfqpoint{2.748742in}{2.005171in}}%
\pgfpathlineto{\pgfqpoint{2.750095in}{2.015466in}}%
\pgfpathlineto{\pgfqpoint{2.751887in}{2.025760in}}%
\pgfpathlineto{\pgfqpoint{2.754017in}{2.036054in}}%
\pgfpathlineto{\pgfqpoint{2.755344in}{2.041703in}}%
\pgfpathlineto{\pgfqpoint{2.756197in}{2.046349in}}%
\pgfpathlineto{\pgfqpoint{2.758317in}{2.056643in}}%
\pgfpathlineto{\pgfqpoint{2.760700in}{2.066938in}}%
\pgfpathlineto{\pgfqpoint{2.763415in}{2.077232in}}%
\pgfpathlineto{\pgfqpoint{2.765921in}{2.085440in}}%
\pgfpathlineto{\pgfqpoint{2.766458in}{2.087527in}}%
\pgfpathlineto{\pgfqpoint{2.769566in}{2.097821in}}%
\pgfpathlineto{\pgfqpoint{2.773243in}{2.108116in}}%
\pgfpathlineto{\pgfqpoint{2.776498in}{2.115807in}}%
\pgfpathlineto{\pgfqpoint{2.777471in}{2.118410in}}%
\pgfpathlineto{\pgfqpoint{2.782010in}{2.128704in}}%
\pgfpathlineto{\pgfqpoint{2.787075in}{2.138494in}}%
\pgfpathlineto{\pgfqpoint{2.787314in}{2.138999in}}%
\pgfpathlineto{\pgfqpoint{2.792934in}{2.149293in}}%
\pgfpathlineto{\pgfqpoint{2.797652in}{2.156873in}}%
\pgfpathlineto{\pgfqpoint{2.799232in}{2.159588in}}%
\pgfpathlineto{\pgfqpoint{2.805891in}{2.169882in}}%
\pgfpathlineto{\pgfqpoint{2.808229in}{2.173250in}}%
\pgfpathlineto{\pgfqpoint{2.812786in}{2.180177in}}%
\pgfpathlineto{\pgfqpoint{2.818806in}{2.189324in}}%
\pgfpathlineto{\pgfqpoint{2.819521in}{2.190471in}}%
\pgfpathlineto{\pgfqpoint{2.825411in}{2.200765in}}%
\pgfpathlineto{\pgfqpoint{2.829383in}{2.208732in}}%
\pgfpathlineto{\pgfqpoint{2.830471in}{2.211060in}}%
\pgfpathlineto{\pgfqpoint{2.834893in}{2.221354in}}%
\pgfpathlineto{\pgfqpoint{2.839057in}{2.231649in}}%
\pgfpathlineto{\pgfqpoint{2.839960in}{2.233937in}}%
\pgfpathlineto{\pgfqpoint{2.843074in}{2.241943in}}%
\pgfpathlineto{\pgfqpoint{2.847059in}{2.252238in}}%
\pgfpathlineto{\pgfqpoint{2.850538in}{2.261149in}}%
\pgfpathlineto{\pgfqpoint{2.851084in}{2.262532in}}%
\pgfpathlineto{\pgfqpoint{2.855239in}{2.272826in}}%
\pgfpathlineto{\pgfqpoint{2.859528in}{2.283121in}}%
\pgfpathlineto{\pgfqpoint{2.861115in}{2.286787in}}%
\pgfpathlineto{\pgfqpoint{2.864096in}{2.293415in}}%
\pgfpathlineto{\pgfqpoint{2.868954in}{2.303710in}}%
\pgfpathlineto{\pgfqpoint{2.871692in}{2.309211in}}%
\pgfpathlineto{\pgfqpoint{2.874233in}{2.314004in}}%
\pgfpathlineto{\pgfqpoint{2.880024in}{2.324299in}}%
\pgfpathlineto{\pgfqpoint{2.882269in}{2.328051in}}%
\pgfpathlineto{\pgfqpoint{2.886558in}{2.334593in}}%
\pgfpathlineto{\pgfqpoint{2.892846in}{2.343575in}}%
\pgfpathlineto{\pgfqpoint{2.893888in}{2.344887in}}%
\pgfpathlineto{\pgfqpoint{2.902606in}{2.355182in}}%
\pgfpathlineto{\pgfqpoint{2.903423in}{2.356086in}}%
\pgfpathlineto{\pgfqpoint{2.913489in}{2.365476in}}%
\pgfpathlineto{\pgfqpoint{2.914000in}{2.365924in}}%
\pgfpathlineto{\pgfqpoint{2.924577in}{2.373242in}}%
\pgfpathlineto{\pgfqpoint{2.929748in}{2.375771in}}%
\pgfpathlineto{\pgfqpoint{2.935154in}{2.378259in}}%
\pgfpathlineto{\pgfqpoint{2.945731in}{2.380967in}}%
\pgfpathlineto{\pgfqpoint{2.956308in}{2.381404in}}%
\pgfpathlineto{\pgfqpoint{2.966885in}{2.379593in}}%
\pgfpathlineto{\pgfqpoint{2.977462in}{2.377649in}}%
\pgfpathlineto{\pgfqpoint{2.977462in}{2.386065in}}%
\pgfpathlineto{\pgfqpoint{2.977462in}{2.396360in}}%
\pgfpathlineto{\pgfqpoint{2.977462in}{2.406654in}}%
\pgfpathlineto{\pgfqpoint{2.977462in}{2.416948in}}%
\pgfpathlineto{\pgfqpoint{2.977462in}{2.427243in}}%
\pgfpathlineto{\pgfqpoint{2.977462in}{2.437537in}}%
\pgfpathlineto{\pgfqpoint{2.977462in}{2.447832in}}%
\pgfpathlineto{\pgfqpoint{2.977462in}{2.453233in}}%
\pgfpathlineto{\pgfqpoint{2.972055in}{2.447832in}}%
\pgfpathlineto{\pgfqpoint{2.966885in}{2.444745in}}%
\pgfpathlineto{\pgfqpoint{2.956308in}{2.445070in}}%
\pgfpathlineto{\pgfqpoint{2.945731in}{2.444215in}}%
\pgfpathlineto{\pgfqpoint{2.935154in}{2.441887in}}%
\pgfpathlineto{\pgfqpoint{2.924577in}{2.438157in}}%
\pgfpathlineto{\pgfqpoint{2.923284in}{2.437537in}}%
\pgfpathlineto{\pgfqpoint{2.914000in}{2.432881in}}%
\pgfpathlineto{\pgfqpoint{2.904886in}{2.427243in}}%
\pgfpathlineto{\pgfqpoint{2.903423in}{2.426288in}}%
\pgfpathlineto{\pgfqpoint{2.892846in}{2.418192in}}%
\pgfpathlineto{\pgfqpoint{2.891401in}{2.416948in}}%
\pgfpathlineto{\pgfqpoint{2.882269in}{2.408612in}}%
\pgfpathlineto{\pgfqpoint{2.880303in}{2.406654in}}%
\pgfpathlineto{\pgfqpoint{2.871692in}{2.397537in}}%
\pgfpathlineto{\pgfqpoint{2.870647in}{2.396360in}}%
\pgfpathlineto{\pgfqpoint{2.862077in}{2.386065in}}%
\pgfpathlineto{\pgfqpoint{2.861115in}{2.384830in}}%
\pgfpathlineto{\pgfqpoint{2.854339in}{2.375771in}}%
\pgfpathlineto{\pgfqpoint{2.850538in}{2.370341in}}%
\pgfpathlineto{\pgfqpoint{2.847205in}{2.365476in}}%
\pgfpathlineto{\pgfqpoint{2.840613in}{2.355182in}}%
\pgfpathlineto{\pgfqpoint{2.839960in}{2.354090in}}%
\pgfpathlineto{\pgfqpoint{2.834481in}{2.344887in}}%
\pgfpathlineto{\pgfqpoint{2.829383in}{2.335736in}}%
\pgfpathlineto{\pgfqpoint{2.828737in}{2.334593in}}%
\pgfpathlineto{\pgfqpoint{2.823289in}{2.324299in}}%
\pgfpathlineto{\pgfqpoint{2.818806in}{2.315290in}}%
\pgfpathlineto{\pgfqpoint{2.818144in}{2.314004in}}%
\pgfpathlineto{\pgfqpoint{2.813132in}{2.303710in}}%
\pgfpathlineto{\pgfqpoint{2.808353in}{2.293415in}}%
\pgfpathlineto{\pgfqpoint{2.808229in}{2.293139in}}%
\pgfpathlineto{\pgfqpoint{2.803495in}{2.283121in}}%
\pgfpathlineto{\pgfqpoint{2.798739in}{2.272826in}}%
\pgfpathlineto{\pgfqpoint{2.797652in}{2.270449in}}%
\pgfpathlineto{\pgfqpoint{2.793722in}{2.262532in}}%
\pgfpathlineto{\pgfqpoint{2.788544in}{2.252238in}}%
\pgfpathlineto{\pgfqpoint{2.787075in}{2.249405in}}%
\pgfpathlineto{\pgfqpoint{2.782741in}{2.241943in}}%
\pgfpathlineto{\pgfqpoint{2.776498in}{2.231857in}}%
\pgfpathlineto{\pgfqpoint{2.776349in}{2.231649in}}%
\pgfpathlineto{\pgfqpoint{2.768351in}{2.221354in}}%
\pgfpathlineto{\pgfqpoint{2.765921in}{2.218516in}}%
\pgfpathlineto{\pgfqpoint{2.758227in}{2.211060in}}%
\pgfpathlineto{\pgfqpoint{2.755344in}{2.208481in}}%
\pgfpathlineto{\pgfqpoint{2.744767in}{2.201009in}}%
\pgfpathlineto{\pgfqpoint{2.744294in}{2.200765in}}%
\pgfpathlineto{\pgfqpoint{2.734190in}{2.195744in}}%
\pgfpathlineto{\pgfqpoint{2.723613in}{2.192523in}}%
\pgfpathlineto{\pgfqpoint{2.713036in}{2.191357in}}%
\pgfpathlineto{\pgfqpoint{2.702459in}{2.192228in}}%
\pgfpathlineto{\pgfqpoint{2.691882in}{2.195090in}}%
\pgfpathlineto{\pgfqpoint{2.681305in}{2.199873in}}%
\pgfpathlineto{\pgfqpoint{2.679884in}{2.200765in}}%
\pgfpathlineto{\pgfqpoint{2.670727in}{2.206806in}}%
\pgfpathlineto{\pgfqpoint{2.665751in}{2.211060in}}%
\pgfpathlineto{\pgfqpoint{2.660150in}{2.216390in}}%
\pgfpathlineto{\pgfqpoint{2.655967in}{2.221354in}}%
\pgfpathlineto{\pgfqpoint{2.649573in}{2.229746in}}%
\pgfpathlineto{\pgfqpoint{2.648355in}{2.231649in}}%
\pgfpathlineto{\pgfqpoint{2.642023in}{2.241943in}}%
\pgfpathlineto{\pgfqpoint{2.638996in}{2.246895in}}%
\pgfpathlineto{\pgfqpoint{2.636095in}{2.252238in}}%
\pgfpathlineto{\pgfqpoint{2.630349in}{2.262532in}}%
\pgfpathlineto{\pgfqpoint{2.628419in}{2.265843in}}%
\pgfpathlineto{\pgfqpoint{2.624670in}{2.272826in}}%
\pgfpathlineto{\pgfqpoint{2.618803in}{2.283121in}}%
\pgfpathlineto{\pgfqpoint{2.617842in}{2.284701in}}%
\pgfpathlineto{\pgfqpoint{2.612841in}{2.293415in}}%
\pgfpathlineto{\pgfqpoint{2.607265in}{2.302440in}}%
\pgfpathlineto{\pgfqpoint{2.606510in}{2.303710in}}%
\pgfpathlineto{\pgfqpoint{2.599937in}{2.314004in}}%
\pgfpathlineto{\pgfqpoint{2.596688in}{2.318731in}}%
\pgfpathlineto{\pgfqpoint{2.592952in}{2.324299in}}%
\pgfpathlineto{\pgfqpoint{2.586111in}{2.333817in}}%
\pgfpathlineto{\pgfqpoint{2.585559in}{2.334593in}}%
\pgfpathlineto{\pgfqpoint{2.577761in}{2.344887in}}%
\pgfpathlineto{\pgfqpoint{2.575534in}{2.347663in}}%
\pgfpathlineto{\pgfqpoint{2.569504in}{2.355182in}}%
\pgfpathlineto{\pgfqpoint{2.564957in}{2.360588in}}%
\pgfpathlineto{\pgfqpoint{2.560798in}{2.365476in}}%
\pgfpathlineto{\pgfqpoint{2.554380in}{2.372742in}}%
\pgfpathlineto{\pgfqpoint{2.551640in}{2.375771in}}%
\pgfpathlineto{\pgfqpoint{2.543803in}{2.384199in}}%
\pgfpathlineto{\pgfqpoint{2.542004in}{2.386065in}}%
\pgfpathlineto{\pgfqpoint{2.533226in}{2.395009in}}%
\pgfpathlineto{\pgfqpoint{2.531833in}{2.396360in}}%
\pgfpathlineto{\pgfqpoint{2.522649in}{2.405185in}}%
\pgfpathlineto{\pgfqpoint{2.521021in}{2.406654in}}%
\pgfpathlineto{\pgfqpoint{2.512072in}{2.414713in}}%
\pgfpathlineto{\pgfqpoint{2.509389in}{2.416948in}}%
\pgfpathlineto{\pgfqpoint{2.501495in}{2.423551in}}%
\pgfpathlineto{\pgfqpoint{2.496646in}{2.427243in}}%
\pgfpathlineto{\pgfqpoint{2.490917in}{2.431639in}}%
\pgfpathlineto{\pgfqpoint{2.482325in}{2.437537in}}%
\pgfpathlineto{\pgfqpoint{2.480340in}{2.438913in}}%
\pgfpathlineto{\pgfqpoint{2.469763in}{2.445266in}}%
\pgfpathlineto{\pgfqpoint{2.464734in}{2.447832in}}%
\pgfpathlineto{\pgfqpoint{2.459186in}{2.450687in}}%
\pgfpathlineto{\pgfqpoint{2.448609in}{2.455154in}}%
\pgfpathlineto{\pgfqpoint{2.439772in}{2.458126in}}%
\pgfpathlineto{\pgfqpoint{2.438032in}{2.458714in}}%
\pgfpathlineto{\pgfqpoint{2.427455in}{2.461364in}}%
\pgfpathlineto{\pgfqpoint{2.416878in}{2.463263in}}%
\pgfpathlineto{\pgfqpoint{2.406301in}{2.464540in}}%
\pgfpathlineto{\pgfqpoint{2.395724in}{2.465361in}}%
\pgfpathlineto{\pgfqpoint{2.385147in}{2.465928in}}%
\pgfpathlineto{\pgfqpoint{2.374570in}{2.466475in}}%
\pgfpathlineto{\pgfqpoint{2.363993in}{2.467272in}}%
\pgfpathlineto{\pgfqpoint{2.354893in}{2.468421in}}%
\pgfpathlineto{\pgfqpoint{2.353416in}{2.468603in}}%
\pgfpathlineto{\pgfqpoint{2.342839in}{2.470694in}}%
\pgfpathlineto{\pgfqpoint{2.332262in}{2.473887in}}%
\pgfpathlineto{\pgfqpoint{2.321684in}{2.478454in}}%
\pgfpathlineto{\pgfqpoint{2.321231in}{2.478715in}}%
\pgfpathlineto{\pgfqpoint{2.311107in}{2.484400in}}%
\pgfpathlineto{\pgfqpoint{2.304709in}{2.489009in}}%
\pgfpathlineto{\pgfqpoint{2.300530in}{2.491965in}}%
\pgfpathlineto{\pgfqpoint{2.292100in}{2.499304in}}%
\pgfpathlineto{\pgfqpoint{2.289953in}{2.501154in}}%
\pgfpathlineto{\pgfqpoint{2.281738in}{2.509598in}}%
\pgfpathlineto{\pgfqpoint{2.279376in}{2.512027in}}%
\pgfpathlineto{\pgfqpoint{2.272793in}{2.519893in}}%
\pgfpathlineto{\pgfqpoint{2.268799in}{2.524728in}}%
\pgfpathlineto{\pgfqpoint{2.264829in}{2.530187in}}%
\pgfpathlineto{\pgfqpoint{2.258222in}{2.539561in}}%
\pgfpathlineto{\pgfqpoint{2.257638in}{2.540482in}}%
\pgfpathlineto{\pgfqpoint{2.251664in}{2.550776in}}%
\pgfpathlineto{\pgfqpoint{2.247645in}{2.560068in}}%
\pgfpathlineto{\pgfqpoint{2.247270in}{2.561071in}}%
\pgfpathlineto{\pgfqpoint{2.244293in}{2.571365in}}%
\pgfpathlineto{\pgfqpoint{2.241630in}{2.581659in}}%
\pgfpathlineto{\pgfqpoint{2.239080in}{2.591954in}}%
\pgfpathlineto{\pgfqpoint{2.237068in}{2.600156in}}%
\pgfpathlineto{\pgfqpoint{2.236580in}{2.602248in}}%
\pgfpathlineto{\pgfqpoint{2.234117in}{2.612543in}}%
\pgfpathlineto{\pgfqpoint{2.231576in}{2.622837in}}%
\pgfpathlineto{\pgfqpoint{2.228921in}{2.633132in}}%
\pgfpathlineto{\pgfqpoint{2.226491in}{2.642030in}}%
\pgfpathlineto{\pgfqpoint{2.226121in}{2.643426in}}%
\pgfpathlineto{\pgfqpoint{2.223177in}{2.653720in}}%
\pgfpathlineto{\pgfqpoint{2.220029in}{2.664015in}}%
\pgfpathlineto{\pgfqpoint{2.216660in}{2.674309in}}%
\pgfpathlineto{\pgfqpoint{2.215914in}{2.676412in}}%
\pgfpathlineto{\pgfqpoint{2.213079in}{2.684604in}}%
\pgfpathlineto{\pgfqpoint{2.209280in}{2.694898in}}%
\pgfpathlineto{\pgfqpoint{2.205337in}{2.705014in}}%
\pgfpathlineto{\pgfqpoint{2.205268in}{2.705193in}}%
\pgfpathlineto{\pgfqpoint{2.201069in}{2.715487in}}%
\pgfpathlineto{\pgfqpoint{2.196719in}{2.725781in}}%
\pgfpathlineto{\pgfqpoint{2.194760in}{2.730250in}}%
\pgfpathlineto{\pgfqpoint{2.192249in}{2.736076in}}%
\pgfpathlineto{\pgfqpoint{2.187737in}{2.746370in}}%
\pgfpathlineto{\pgfqpoint{2.184183in}{2.754560in}}%
\pgfpathlineto{\pgfqpoint{2.183278in}{2.756665in}}%
\pgfpathlineto{\pgfqpoint{2.179018in}{2.766959in}}%
\pgfpathlineto{\pgfqpoint{2.175236in}{2.777254in}}%
\pgfpathlineto{\pgfqpoint{2.173606in}{2.782595in}}%
\pgfpathlineto{\pgfqpoint{2.172108in}{2.787548in}}%
\pgfpathlineto{\pgfqpoint{2.169594in}{2.797842in}}%
\pgfpathlineto{\pgfqpoint{2.167510in}{2.808137in}}%
\pgfpathlineto{\pgfqpoint{2.165709in}{2.818431in}}%
\pgfpathlineto{\pgfqpoint{2.164108in}{2.828726in}}%
\pgfpathlineto{\pgfqpoint{2.163029in}{2.836422in}}%
\pgfpathlineto{\pgfqpoint{2.162653in}{2.839020in}}%
\pgfpathlineto{\pgfqpoint{2.161304in}{2.849315in}}%
\pgfpathlineto{\pgfqpoint{2.160056in}{2.859609in}}%
\pgfpathlineto{\pgfqpoint{2.158890in}{2.869903in}}%
\pgfpathlineto{\pgfqpoint{2.157792in}{2.880198in}}%
\pgfpathlineto{\pgfqpoint{2.156750in}{2.890492in}}%
\pgfpathlineto{\pgfqpoint{2.155758in}{2.900787in}}%
\pgfpathlineto{\pgfqpoint{2.154824in}{2.911081in}}%
\pgfpathlineto{\pgfqpoint{2.152451in}{2.911081in}}%
\pgfpathlineto{\pgfqpoint{2.141874in}{2.911081in}}%
\pgfpathlineto{\pgfqpoint{2.131297in}{2.911081in}}%
\pgfpathlineto{\pgfqpoint{2.120720in}{2.911081in}}%
\pgfpathlineto{\pgfqpoint{2.116545in}{2.911081in}}%
\pgfpathlineto{\pgfqpoint{2.119974in}{2.900787in}}%
\pgfpathlineto{\pgfqpoint{2.120720in}{2.898375in}}%
\pgfpathlineto{\pgfqpoint{2.122888in}{2.890492in}}%
\pgfpathlineto{\pgfqpoint{2.125478in}{2.880198in}}%
\pgfpathlineto{\pgfqpoint{2.127858in}{2.869903in}}%
\pgfpathlineto{\pgfqpoint{2.130084in}{2.859609in}}%
\pgfpathlineto{\pgfqpoint{2.131297in}{2.853775in}}%
\pgfpathlineto{\pgfqpoint{2.132154in}{2.849315in}}%
\pgfpathlineto{\pgfqpoint{2.134095in}{2.839020in}}%
\pgfpathlineto{\pgfqpoint{2.136013in}{2.828726in}}%
\pgfpathlineto{\pgfqpoint{2.137943in}{2.818431in}}%
\pgfpathlineto{\pgfqpoint{2.139918in}{2.808137in}}%
\pgfpathlineto{\pgfqpoint{2.141874in}{2.798362in}}%
\pgfpathlineto{\pgfqpoint{2.141974in}{2.797842in}}%
\pgfpathlineto{\pgfqpoint{2.144108in}{2.787548in}}%
\pgfpathlineto{\pgfqpoint{2.146636in}{2.777254in}}%
\pgfpathlineto{\pgfqpoint{2.149980in}{2.766959in}}%
\pgfpathlineto{\pgfqpoint{2.152451in}{2.760552in}}%
\pgfpathlineto{\pgfqpoint{2.153925in}{2.756665in}}%
\pgfpathlineto{\pgfqpoint{2.157961in}{2.746370in}}%
\pgfpathlineto{\pgfqpoint{2.162023in}{2.736076in}}%
\pgfpathlineto{\pgfqpoint{2.163029in}{2.733440in}}%
\pgfpathlineto{\pgfqpoint{2.165924in}{2.725781in}}%
\pgfpathlineto{\pgfqpoint{2.169686in}{2.715487in}}%
\pgfpathlineto{\pgfqpoint{2.173320in}{2.705193in}}%
\pgfpathlineto{\pgfqpoint{2.173606in}{2.704325in}}%
\pgfpathlineto{\pgfqpoint{2.176691in}{2.694898in}}%
\pgfpathlineto{\pgfqpoint{2.179879in}{2.684604in}}%
\pgfpathlineto{\pgfqpoint{2.182888in}{2.674309in}}%
\pgfpathlineto{\pgfqpoint{2.184183in}{2.669542in}}%
\pgfpathlineto{\pgfqpoint{2.185675in}{2.664015in}}%
\pgfpathlineto{\pgfqpoint{2.188253in}{2.653720in}}%
\pgfpathlineto{\pgfqpoint{2.190672in}{2.643426in}}%
\pgfpathlineto{\pgfqpoint{2.192953in}{2.633132in}}%
\pgfpathlineto{\pgfqpoint{2.194760in}{2.624531in}}%
\pgfpathlineto{\pgfqpoint{2.195113in}{2.622837in}}%
\pgfpathlineto{\pgfqpoint{2.197145in}{2.612543in}}%
\pgfpathlineto{\pgfqpoint{2.199119in}{2.602248in}}%
\pgfpathlineto{\pgfqpoint{2.201069in}{2.591954in}}%
\pgfpathlineto{\pgfqpoint{2.203045in}{2.581659in}}%
\pgfpathlineto{\pgfqpoint{2.205133in}{2.571365in}}%
\pgfpathlineto{\pgfqpoint{2.205337in}{2.570476in}}%
\pgfpathlineto{\pgfqpoint{2.207452in}{2.561071in}}%
\pgfpathlineto{\pgfqpoint{2.210375in}{2.550776in}}%
\pgfpathlineto{\pgfqpoint{2.214113in}{2.540482in}}%
\pgfpathlineto{\pgfqpoint{2.215914in}{2.536049in}}%
\pgfpathlineto{\pgfqpoint{2.218380in}{2.530187in}}%
\pgfpathlineto{\pgfqpoint{2.223009in}{2.519893in}}%
\pgfpathlineto{\pgfqpoint{2.226491in}{2.512563in}}%
\pgfpathlineto{\pgfqpoint{2.227971in}{2.509598in}}%
\pgfpathlineto{\pgfqpoint{2.233202in}{2.499304in}}%
\pgfpathlineto{\pgfqpoint{2.237068in}{2.491972in}}%
\pgfpathlineto{\pgfqpoint{2.238727in}{2.489009in}}%
\pgfpathlineto{\pgfqpoint{2.244515in}{2.478715in}}%
\pgfpathlineto{\pgfqpoint{2.247645in}{2.473230in}}%
\pgfpathlineto{\pgfqpoint{2.250589in}{2.468421in}}%
\pgfpathlineto{\pgfqpoint{2.256902in}{2.458126in}}%
\pgfpathlineto{\pgfqpoint{2.258222in}{2.455942in}}%
\pgfpathlineto{\pgfqpoint{2.263532in}{2.447832in}}%
\pgfpathlineto{\pgfqpoint{2.268799in}{2.439764in}}%
\pgfpathlineto{\pgfqpoint{2.270401in}{2.437537in}}%
\pgfpathlineto{\pgfqpoint{2.277587in}{2.427243in}}%
\pgfpathlineto{\pgfqpoint{2.279376in}{2.424606in}}%
\pgfpathlineto{\pgfqpoint{2.285191in}{2.416948in}}%
\pgfpathlineto{\pgfqpoint{2.289953in}{2.410513in}}%
\pgfpathlineto{\pgfqpoint{2.293230in}{2.406654in}}%
\pgfpathlineto{\pgfqpoint{2.300530in}{2.397754in}}%
\pgfpathlineto{\pgfqpoint{2.301889in}{2.396360in}}%
\pgfpathlineto{\pgfqpoint{2.311107in}{2.386488in}}%
\pgfpathlineto{\pgfqpoint{2.311602in}{2.386065in}}%
\pgfpathlineto{\pgfqpoint{2.321684in}{2.377004in}}%
\pgfpathlineto{\pgfqpoint{2.323568in}{2.375771in}}%
\pgfpathlineto{\pgfqpoint{2.332262in}{2.369745in}}%
\pgfpathlineto{\pgfqpoint{2.342578in}{2.365476in}}%
\pgfpathlineto{\pgfqpoint{2.342839in}{2.365361in}}%
\pgfpathlineto{\pgfqpoint{2.353416in}{2.364281in}}%
\pgfpathlineto{\pgfqpoint{2.358612in}{2.365476in}}%
\pgfpathlineto{\pgfqpoint{2.363993in}{2.366601in}}%
\pgfpathlineto{\pgfqpoint{2.374570in}{2.370876in}}%
\pgfpathlineto{\pgfqpoint{2.385147in}{2.375763in}}%
\pgfpathlineto{\pgfqpoint{2.385163in}{2.375771in}}%
\pgfpathlineto{\pgfqpoint{2.395724in}{2.379979in}}%
\pgfpathlineto{\pgfqpoint{2.406301in}{2.383421in}}%
\pgfpathlineto{\pgfqpoint{2.416878in}{2.385813in}}%
\pgfpathlineto{\pgfqpoint{2.419007in}{2.386065in}}%
\pgfpathlineto{\pgfqpoint{2.427455in}{2.386956in}}%
\pgfpathlineto{\pgfqpoint{2.438032in}{2.386887in}}%
\pgfpathlineto{\pgfqpoint{2.444590in}{2.386065in}}%
\pgfpathlineto{\pgfqpoint{2.448609in}{2.385563in}}%
\pgfpathlineto{\pgfqpoint{2.459186in}{2.382921in}}%
\pgfpathlineto{\pgfqpoint{2.469763in}{2.379048in}}%
\pgfpathlineto{\pgfqpoint{2.476544in}{2.375771in}}%
\pgfpathlineto{\pgfqpoint{2.480340in}{2.373929in}}%
\pgfpathlineto{\pgfqpoint{2.490917in}{2.367534in}}%
\pgfpathlineto{\pgfqpoint{2.493777in}{2.365476in}}%
\pgfpathlineto{\pgfqpoint{2.501495in}{2.359859in}}%
\pgfpathlineto{\pgfqpoint{2.507076in}{2.355182in}}%
\pgfpathlineto{\pgfqpoint{2.512072in}{2.350920in}}%
\pgfpathlineto{\pgfqpoint{2.518336in}{2.344887in}}%
\pgfpathlineto{\pgfqpoint{2.522649in}{2.340629in}}%
\pgfpathlineto{\pgfqpoint{2.528162in}{2.334593in}}%
\pgfpathlineto{\pgfqpoint{2.533226in}{2.328865in}}%
\pgfpathlineto{\pgfqpoint{2.536930in}{2.324299in}}%
\pgfpathlineto{\pgfqpoint{2.543803in}{2.315479in}}%
\pgfpathlineto{\pgfqpoint{2.544875in}{2.314004in}}%
\pgfpathlineto{\pgfqpoint{2.551997in}{2.303710in}}%
\pgfpathlineto{\pgfqpoint{2.554380in}{2.300096in}}%
\pgfpathlineto{\pgfqpoint{2.558536in}{2.293415in}}%
\pgfpathlineto{\pgfqpoint{2.564618in}{2.283121in}}%
\pgfpathlineto{\pgfqpoint{2.564957in}{2.282518in}}%
\pgfpathlineto{\pgfqpoint{2.570164in}{2.272826in}}%
\pgfpathlineto{\pgfqpoint{2.575421in}{2.262532in}}%
\pgfpathlineto{\pgfqpoint{2.575534in}{2.262300in}}%
\pgfpathlineto{\pgfqpoint{2.580309in}{2.252238in}}%
\pgfpathlineto{\pgfqpoint{2.584984in}{2.241943in}}%
\pgfpathlineto{\pgfqpoint{2.586111in}{2.239369in}}%
\pgfpathlineto{\pgfqpoint{2.589463in}{2.231649in}}%
\pgfpathlineto{\pgfqpoint{2.593792in}{2.221354in}}%
\pgfpathlineto{\pgfqpoint{2.596688in}{2.214282in}}%
\pgfpathlineto{\pgfqpoint{2.598034in}{2.211060in}}%
\pgfpathlineto{\pgfqpoint{2.602283in}{2.200765in}}%
\pgfpathlineto{\pgfqpoint{2.606658in}{2.190471in}}%
\pgfpathlineto{\pgfqpoint{2.607265in}{2.189371in}}%
\pgfpathlineto{\pgfqpoint{2.613409in}{2.180177in}}%
\pgfpathlineto{\pgfqpoint{2.617842in}{2.174246in}}%
\pgfpathlineto{\pgfqpoint{2.621233in}{2.169882in}}%
\pgfpathlineto{\pgfqpoint{2.628419in}{2.159618in}}%
\pgfpathlineto{\pgfqpoint{2.628441in}{2.159588in}}%
\pgfpathlineto{\pgfqpoint{2.635206in}{2.149293in}}%
\pgfpathlineto{\pgfqpoint{2.638996in}{2.142643in}}%
\pgfpathlineto{\pgfqpoint{2.641262in}{2.138999in}}%
\pgfpathlineto{\pgfqpoint{2.646772in}{2.128704in}}%
\pgfpathlineto{\pgfqpoint{2.649573in}{2.122587in}}%
\pgfpathlineto{\pgfqpoint{2.651718in}{2.118410in}}%
\pgfpathlineto{\pgfqpoint{2.656226in}{2.108116in}}%
\pgfpathlineto{\pgfqpoint{2.660059in}{2.097821in}}%
\pgfpathlineto{\pgfqpoint{2.660150in}{2.097537in}}%
\pgfpathlineto{\pgfqpoint{2.663877in}{2.087527in}}%
\pgfpathlineto{\pgfqpoint{2.667182in}{2.077232in}}%
\pgfpathlineto{\pgfqpoint{2.670077in}{2.066938in}}%
\pgfpathlineto{\pgfqpoint{2.670727in}{2.064362in}}%
\pgfpathlineto{\pgfqpoint{2.673106in}{2.056643in}}%
\pgfpathlineto{\pgfqpoint{2.675989in}{2.046349in}}%
\pgfpathlineto{\pgfqpoint{2.678634in}{2.036054in}}%
\pgfpathlineto{\pgfqpoint{2.681041in}{2.025760in}}%
\pgfpathlineto{\pgfqpoint{2.681305in}{2.024486in}}%
\pgfpathlineto{\pgfqpoint{2.683740in}{2.015466in}}%
\pgfpathlineto{\pgfqpoint{2.685995in}{2.005171in}}%
\pgfpathlineto{\pgfqpoint{2.687586in}{1.994877in}}%
\pgfpathlineto{\pgfqpoint{2.688395in}{1.984582in}}%
\pgfpathlineto{\pgfqpoint{2.688322in}{1.974288in}}%
\pgfpathlineto{\pgfqpoint{2.687955in}{1.963993in}}%
\pgfpathlineto{\pgfqpoint{2.690059in}{1.953699in}}%
\pgfpathlineto{\pgfqpoint{2.691882in}{1.944532in}}%
\pgfpathlineto{\pgfqpoint{2.692227in}{1.943405in}}%
\pgfpathlineto{\pgfqpoint{2.695336in}{1.933110in}}%
\pgfpathlineto{\pgfqpoint{2.698631in}{1.922816in}}%
\pgfpathlineto{\pgfqpoint{2.702353in}{1.912521in}}%
\pgfpathlineto{\pgfqpoint{2.702459in}{1.912275in}}%
\pgfpathlineto{\pgfqpoint{2.711572in}{1.902227in}}%
\pgfpathclose%
\pgfusepath{fill}%
\end{pgfscope}%
\begin{pgfscope}%
\pgfpathrectangle{\pgfqpoint{1.856795in}{1.819814in}}{\pgfqpoint{1.194205in}{1.163386in}}%
\pgfusepath{clip}%
\pgfsetbuttcap%
\pgfsetroundjoin%
\definecolor{currentfill}{rgb}{0.971202,0.827364,0.728520}%
\pgfsetfillcolor{currentfill}%
\pgfsetlinewidth{0.000000pt}%
\definecolor{currentstroke}{rgb}{0.000000,0.000000,0.000000}%
\pgfsetstrokecolor{currentstroke}%
\pgfsetdash{}{0pt}%
\pgfpathmoveto{\pgfqpoint{2.406301in}{1.891932in}}%
\pgfpathlineto{\pgfqpoint{2.416878in}{1.891932in}}%
\pgfpathlineto{\pgfqpoint{2.427455in}{1.891932in}}%
\pgfpathlineto{\pgfqpoint{2.438032in}{1.891932in}}%
\pgfpathlineto{\pgfqpoint{2.448609in}{1.891932in}}%
\pgfpathlineto{\pgfqpoint{2.454454in}{1.891932in}}%
\pgfpathlineto{\pgfqpoint{2.448609in}{1.901668in}}%
\pgfpathlineto{\pgfqpoint{2.448326in}{1.902227in}}%
\pgfpathlineto{\pgfqpoint{2.443390in}{1.912521in}}%
\pgfpathlineto{\pgfqpoint{2.438860in}{1.922816in}}%
\pgfpathlineto{\pgfqpoint{2.438032in}{1.924843in}}%
\pgfpathlineto{\pgfqpoint{2.435032in}{1.933110in}}%
\pgfpathlineto{\pgfqpoint{2.431534in}{1.943405in}}%
\pgfpathlineto{\pgfqpoint{2.428219in}{1.953699in}}%
\pgfpathlineto{\pgfqpoint{2.427455in}{1.956192in}}%
\pgfpathlineto{\pgfqpoint{2.425264in}{1.963993in}}%
\pgfpathlineto{\pgfqpoint{2.422475in}{1.974288in}}%
\pgfpathlineto{\pgfqpoint{2.419756in}{1.984582in}}%
\pgfpathlineto{\pgfqpoint{2.417083in}{1.994877in}}%
\pgfpathlineto{\pgfqpoint{2.416878in}{1.995681in}}%
\pgfpathlineto{\pgfqpoint{2.414586in}{2.005171in}}%
\pgfpathlineto{\pgfqpoint{2.411676in}{2.015466in}}%
\pgfpathlineto{\pgfqpoint{2.408526in}{2.025760in}}%
\pgfpathlineto{\pgfqpoint{2.406301in}{2.033492in}}%
\pgfpathlineto{\pgfqpoint{2.405582in}{2.036054in}}%
\pgfpathlineto{\pgfqpoint{2.402975in}{2.046349in}}%
\pgfpathlineto{\pgfqpoint{2.400604in}{2.056643in}}%
\pgfpathlineto{\pgfqpoint{2.398492in}{2.066938in}}%
\pgfpathlineto{\pgfqpoint{2.396653in}{2.077232in}}%
\pgfpathlineto{\pgfqpoint{2.395724in}{2.083434in}}%
\pgfpathlineto{\pgfqpoint{2.395118in}{2.087527in}}%
\pgfpathlineto{\pgfqpoint{2.393880in}{2.097821in}}%
\pgfpathlineto{\pgfqpoint{2.392880in}{2.108116in}}%
\pgfpathlineto{\pgfqpoint{2.392072in}{2.118410in}}%
\pgfpathlineto{\pgfqpoint{2.391395in}{2.128704in}}%
\pgfpathlineto{\pgfqpoint{2.390815in}{2.138999in}}%
\pgfpathlineto{\pgfqpoint{2.390353in}{2.149293in}}%
\pgfpathlineto{\pgfqpoint{2.390073in}{2.159588in}}%
\pgfpathlineto{\pgfqpoint{2.390067in}{2.169882in}}%
\pgfpathlineto{\pgfqpoint{2.390450in}{2.180177in}}%
\pgfpathlineto{\pgfqpoint{2.391380in}{2.190471in}}%
\pgfpathlineto{\pgfqpoint{2.393086in}{2.200765in}}%
\pgfpathlineto{\pgfqpoint{2.395724in}{2.210408in}}%
\pgfpathlineto{\pgfqpoint{2.395937in}{2.211060in}}%
\pgfpathlineto{\pgfqpoint{2.400731in}{2.221354in}}%
\pgfpathlineto{\pgfqpoint{2.406301in}{2.229899in}}%
\pgfpathlineto{\pgfqpoint{2.407320in}{2.231649in}}%
\pgfpathlineto{\pgfqpoint{2.413162in}{2.241943in}}%
\pgfpathlineto{\pgfqpoint{2.416878in}{2.249013in}}%
\pgfpathlineto{\pgfqpoint{2.418780in}{2.252238in}}%
\pgfpathlineto{\pgfqpoint{2.425434in}{2.262532in}}%
\pgfpathlineto{\pgfqpoint{2.427455in}{2.265336in}}%
\pgfpathlineto{\pgfqpoint{2.434516in}{2.272826in}}%
\pgfpathlineto{\pgfqpoint{2.438032in}{2.276117in}}%
\pgfpathlineto{\pgfqpoint{2.448609in}{2.282589in}}%
\pgfpathlineto{\pgfqpoint{2.450399in}{2.283121in}}%
\pgfpathlineto{\pgfqpoint{2.459186in}{2.285455in}}%
\pgfpathlineto{\pgfqpoint{2.469763in}{2.285339in}}%
\pgfpathlineto{\pgfqpoint{2.478095in}{2.283121in}}%
\pgfpathlineto{\pgfqpoint{2.480340in}{2.282519in}}%
\pgfpathlineto{\pgfqpoint{2.490917in}{2.277012in}}%
\pgfpathlineto{\pgfqpoint{2.496461in}{2.272826in}}%
\pgfpathlineto{\pgfqpoint{2.501495in}{2.268962in}}%
\pgfpathlineto{\pgfqpoint{2.507877in}{2.262532in}}%
\pgfpathlineto{\pgfqpoint{2.512072in}{2.258222in}}%
\pgfpathlineto{\pgfqpoint{2.516789in}{2.252238in}}%
\pgfpathlineto{\pgfqpoint{2.522649in}{2.244644in}}%
\pgfpathlineto{\pgfqpoint{2.524413in}{2.241943in}}%
\pgfpathlineto{\pgfqpoint{2.530963in}{2.231649in}}%
\pgfpathlineto{\pgfqpoint{2.533226in}{2.228018in}}%
\pgfpathlineto{\pgfqpoint{2.536850in}{2.221354in}}%
\pgfpathlineto{\pgfqpoint{2.542352in}{2.211060in}}%
\pgfpathlineto{\pgfqpoint{2.543803in}{2.208300in}}%
\pgfpathlineto{\pgfqpoint{2.547353in}{2.200765in}}%
\pgfpathlineto{\pgfqpoint{2.552167in}{2.190471in}}%
\pgfpathlineto{\pgfqpoint{2.554380in}{2.185730in}}%
\pgfpathlineto{\pgfqpoint{2.556766in}{2.180177in}}%
\pgfpathlineto{\pgfqpoint{2.561265in}{2.169882in}}%
\pgfpathlineto{\pgfqpoint{2.564957in}{2.161835in}}%
\pgfpathlineto{\pgfqpoint{2.565971in}{2.159588in}}%
\pgfpathlineto{\pgfqpoint{2.570798in}{2.149293in}}%
\pgfpathlineto{\pgfqpoint{2.575534in}{2.139185in}}%
\pgfpathlineto{\pgfqpoint{2.575621in}{2.138999in}}%
\pgfpathlineto{\pgfqpoint{2.579962in}{2.128704in}}%
\pgfpathlineto{\pgfqpoint{2.583843in}{2.118410in}}%
\pgfpathlineto{\pgfqpoint{2.586111in}{2.111589in}}%
\pgfpathlineto{\pgfqpoint{2.587243in}{2.108116in}}%
\pgfpathlineto{\pgfqpoint{2.590169in}{2.097821in}}%
\pgfpathlineto{\pgfqpoint{2.592731in}{2.087527in}}%
\pgfpathlineto{\pgfqpoint{2.594986in}{2.077232in}}%
\pgfpathlineto{\pgfqpoint{2.596688in}{2.068487in}}%
\pgfpathlineto{\pgfqpoint{2.596986in}{2.066938in}}%
\pgfpathlineto{\pgfqpoint{2.598769in}{2.056643in}}%
\pgfpathlineto{\pgfqpoint{2.600406in}{2.046349in}}%
\pgfpathlineto{\pgfqpoint{2.601909in}{2.036054in}}%
\pgfpathlineto{\pgfqpoint{2.603241in}{2.025760in}}%
\pgfpathlineto{\pgfqpoint{2.604334in}{2.015466in}}%
\pgfpathlineto{\pgfqpoint{2.605122in}{2.005171in}}%
\pgfpathlineto{\pgfqpoint{2.605552in}{1.994877in}}%
\pgfpathlineto{\pgfqpoint{2.605577in}{1.984582in}}%
\pgfpathlineto{\pgfqpoint{2.605164in}{1.974288in}}%
\pgfpathlineto{\pgfqpoint{2.604361in}{1.963993in}}%
\pgfpathlineto{\pgfqpoint{2.603716in}{1.953699in}}%
\pgfpathlineto{\pgfqpoint{2.603243in}{1.943405in}}%
\pgfpathlineto{\pgfqpoint{2.602684in}{1.933110in}}%
\pgfpathlineto{\pgfqpoint{2.602050in}{1.922816in}}%
\pgfpathlineto{\pgfqpoint{2.601401in}{1.912521in}}%
\pgfpathlineto{\pgfqpoint{2.600826in}{1.902227in}}%
\pgfpathlineto{\pgfqpoint{2.600444in}{1.891932in}}%
\pgfpathlineto{\pgfqpoint{2.607265in}{1.891932in}}%
\pgfpathlineto{\pgfqpoint{2.617842in}{1.891932in}}%
\pgfpathlineto{\pgfqpoint{2.628419in}{1.891932in}}%
\pgfpathlineto{\pgfqpoint{2.638996in}{1.891932in}}%
\pgfpathlineto{\pgfqpoint{2.649573in}{1.891932in}}%
\pgfpathlineto{\pgfqpoint{2.660150in}{1.891932in}}%
\pgfpathlineto{\pgfqpoint{2.670727in}{1.891932in}}%
\pgfpathlineto{\pgfqpoint{2.681305in}{1.891932in}}%
\pgfpathlineto{\pgfqpoint{2.691882in}{1.891932in}}%
\pgfpathlineto{\pgfqpoint{2.702459in}{1.891932in}}%
\pgfpathlineto{\pgfqpoint{2.713036in}{1.891932in}}%
\pgfpathlineto{\pgfqpoint{2.723613in}{1.891932in}}%
\pgfpathlineto{\pgfqpoint{2.734190in}{1.891932in}}%
\pgfpathlineto{\pgfqpoint{2.744767in}{1.891932in}}%
\pgfpathlineto{\pgfqpoint{2.755344in}{1.891932in}}%
\pgfpathlineto{\pgfqpoint{2.765921in}{1.891932in}}%
\pgfpathlineto{\pgfqpoint{2.776498in}{1.891932in}}%
\pgfpathlineto{\pgfqpoint{2.787075in}{1.891932in}}%
\pgfpathlineto{\pgfqpoint{2.797652in}{1.891932in}}%
\pgfpathlineto{\pgfqpoint{2.808229in}{1.891932in}}%
\pgfpathlineto{\pgfqpoint{2.818806in}{1.891932in}}%
\pgfpathlineto{\pgfqpoint{2.829383in}{1.891932in}}%
\pgfpathlineto{\pgfqpoint{2.839960in}{1.891932in}}%
\pgfpathlineto{\pgfqpoint{2.843341in}{1.891932in}}%
\pgfpathlineto{\pgfqpoint{2.840418in}{1.902227in}}%
\pgfpathlineto{\pgfqpoint{2.839960in}{1.903977in}}%
\pgfpathlineto{\pgfqpoint{2.837976in}{1.912521in}}%
\pgfpathlineto{\pgfqpoint{2.835960in}{1.922816in}}%
\pgfpathlineto{\pgfqpoint{2.834364in}{1.933110in}}%
\pgfpathlineto{\pgfqpoint{2.833164in}{1.943405in}}%
\pgfpathlineto{\pgfqpoint{2.832316in}{1.953699in}}%
\pgfpathlineto{\pgfqpoint{2.831764in}{1.963993in}}%
\pgfpathlineto{\pgfqpoint{2.831417in}{1.974288in}}%
\pgfpathlineto{\pgfqpoint{2.831028in}{1.984582in}}%
\pgfpathlineto{\pgfqpoint{2.830604in}{1.994877in}}%
\pgfpathlineto{\pgfqpoint{2.830433in}{2.005171in}}%
\pgfpathlineto{\pgfqpoint{2.830566in}{2.015466in}}%
\pgfpathlineto{\pgfqpoint{2.830981in}{2.025760in}}%
\pgfpathlineto{\pgfqpoint{2.831646in}{2.036054in}}%
\pgfpathlineto{\pgfqpoint{2.832532in}{2.046349in}}%
\pgfpathlineto{\pgfqpoint{2.833631in}{2.056643in}}%
\pgfpathlineto{\pgfqpoint{2.834956in}{2.066938in}}%
\pgfpathlineto{\pgfqpoint{2.836540in}{2.077232in}}%
\pgfpathlineto{\pgfqpoint{2.838428in}{2.087527in}}%
\pgfpathlineto{\pgfqpoint{2.839960in}{2.094539in}}%
\pgfpathlineto{\pgfqpoint{2.840692in}{2.097821in}}%
\pgfpathlineto{\pgfqpoint{2.843410in}{2.108116in}}%
\pgfpathlineto{\pgfqpoint{2.846610in}{2.118410in}}%
\pgfpathlineto{\pgfqpoint{2.850346in}{2.128704in}}%
\pgfpathlineto{\pgfqpoint{2.850538in}{2.129160in}}%
\pgfpathlineto{\pgfqpoint{2.854805in}{2.138999in}}%
\pgfpathlineto{\pgfqpoint{2.859875in}{2.149293in}}%
\pgfpathlineto{\pgfqpoint{2.861115in}{2.151549in}}%
\pgfpathlineto{\pgfqpoint{2.865713in}{2.159588in}}%
\pgfpathlineto{\pgfqpoint{2.871369in}{2.169882in}}%
\pgfpathlineto{\pgfqpoint{2.871692in}{2.170605in}}%
\pgfpathlineto{\pgfqpoint{2.875705in}{2.180177in}}%
\pgfpathlineto{\pgfqpoint{2.879794in}{2.190471in}}%
\pgfpathlineto{\pgfqpoint{2.882269in}{2.196720in}}%
\pgfpathlineto{\pgfqpoint{2.883985in}{2.200765in}}%
\pgfpathlineto{\pgfqpoint{2.888397in}{2.211060in}}%
\pgfpathlineto{\pgfqpoint{2.892846in}{2.221340in}}%
\pgfpathlineto{\pgfqpoint{2.892853in}{2.221354in}}%
\pgfpathlineto{\pgfqpoint{2.897912in}{2.231649in}}%
\pgfpathlineto{\pgfqpoint{2.903078in}{2.241943in}}%
\pgfpathlineto{\pgfqpoint{2.903423in}{2.242610in}}%
\pgfpathlineto{\pgfqpoint{2.909201in}{2.252238in}}%
\pgfpathlineto{\pgfqpoint{2.914000in}{2.259952in}}%
\pgfpathlineto{\pgfqpoint{2.915975in}{2.262532in}}%
\pgfpathlineto{\pgfqpoint{2.924210in}{2.272826in}}%
\pgfpathlineto{\pgfqpoint{2.924577in}{2.273265in}}%
\pgfpathlineto{\pgfqpoint{2.935154in}{2.282581in}}%
\pgfpathlineto{\pgfqpoint{2.936143in}{2.283121in}}%
\pgfpathlineto{\pgfqpoint{2.945731in}{2.288073in}}%
\pgfpathlineto{\pgfqpoint{2.956308in}{2.289997in}}%
\pgfpathlineto{\pgfqpoint{2.966885in}{2.288373in}}%
\pgfpathlineto{\pgfqpoint{2.977462in}{2.284435in}}%
\pgfpathlineto{\pgfqpoint{2.977462in}{2.293415in}}%
\pgfpathlineto{\pgfqpoint{2.977462in}{2.303710in}}%
\pgfpathlineto{\pgfqpoint{2.977462in}{2.314004in}}%
\pgfpathlineto{\pgfqpoint{2.977462in}{2.324299in}}%
\pgfpathlineto{\pgfqpoint{2.977462in}{2.334593in}}%
\pgfpathlineto{\pgfqpoint{2.977462in}{2.344887in}}%
\pgfpathlineto{\pgfqpoint{2.977462in}{2.355182in}}%
\pgfpathlineto{\pgfqpoint{2.977462in}{2.365476in}}%
\pgfpathlineto{\pgfqpoint{2.977462in}{2.375771in}}%
\pgfpathlineto{\pgfqpoint{2.977462in}{2.377649in}}%
\pgfpathlineto{\pgfqpoint{2.966885in}{2.379593in}}%
\pgfpathlineto{\pgfqpoint{2.956308in}{2.381404in}}%
\pgfpathlineto{\pgfqpoint{2.945731in}{2.380967in}}%
\pgfpathlineto{\pgfqpoint{2.935154in}{2.378259in}}%
\pgfpathlineto{\pgfqpoint{2.929748in}{2.375771in}}%
\pgfpathlineto{\pgfqpoint{2.924577in}{2.373242in}}%
\pgfpathlineto{\pgfqpoint{2.914000in}{2.365924in}}%
\pgfpathlineto{\pgfqpoint{2.913489in}{2.365476in}}%
\pgfpathlineto{\pgfqpoint{2.903423in}{2.356086in}}%
\pgfpathlineto{\pgfqpoint{2.902606in}{2.355182in}}%
\pgfpathlineto{\pgfqpoint{2.893888in}{2.344887in}}%
\pgfpathlineto{\pgfqpoint{2.892846in}{2.343575in}}%
\pgfpathlineto{\pgfqpoint{2.886558in}{2.334593in}}%
\pgfpathlineto{\pgfqpoint{2.882269in}{2.328051in}}%
\pgfpathlineto{\pgfqpoint{2.880024in}{2.324299in}}%
\pgfpathlineto{\pgfqpoint{2.874233in}{2.314004in}}%
\pgfpathlineto{\pgfqpoint{2.871692in}{2.309211in}}%
\pgfpathlineto{\pgfqpoint{2.868954in}{2.303710in}}%
\pgfpathlineto{\pgfqpoint{2.864096in}{2.293415in}}%
\pgfpathlineto{\pgfqpoint{2.861115in}{2.286787in}}%
\pgfpathlineto{\pgfqpoint{2.859528in}{2.283121in}}%
\pgfpathlineto{\pgfqpoint{2.855239in}{2.272826in}}%
\pgfpathlineto{\pgfqpoint{2.851084in}{2.262532in}}%
\pgfpathlineto{\pgfqpoint{2.850538in}{2.261149in}}%
\pgfpathlineto{\pgfqpoint{2.847059in}{2.252238in}}%
\pgfpathlineto{\pgfqpoint{2.843074in}{2.241943in}}%
\pgfpathlineto{\pgfqpoint{2.839960in}{2.233937in}}%
\pgfpathlineto{\pgfqpoint{2.839057in}{2.231649in}}%
\pgfpathlineto{\pgfqpoint{2.834893in}{2.221354in}}%
\pgfpathlineto{\pgfqpoint{2.830471in}{2.211060in}}%
\pgfpathlineto{\pgfqpoint{2.829383in}{2.208732in}}%
\pgfpathlineto{\pgfqpoint{2.825411in}{2.200765in}}%
\pgfpathlineto{\pgfqpoint{2.819521in}{2.190471in}}%
\pgfpathlineto{\pgfqpoint{2.818806in}{2.189324in}}%
\pgfpathlineto{\pgfqpoint{2.812786in}{2.180177in}}%
\pgfpathlineto{\pgfqpoint{2.808229in}{2.173250in}}%
\pgfpathlineto{\pgfqpoint{2.805891in}{2.169882in}}%
\pgfpathlineto{\pgfqpoint{2.799232in}{2.159588in}}%
\pgfpathlineto{\pgfqpoint{2.797652in}{2.156873in}}%
\pgfpathlineto{\pgfqpoint{2.792934in}{2.149293in}}%
\pgfpathlineto{\pgfqpoint{2.787314in}{2.138999in}}%
\pgfpathlineto{\pgfqpoint{2.787075in}{2.138494in}}%
\pgfpathlineto{\pgfqpoint{2.782010in}{2.128704in}}%
\pgfpathlineto{\pgfqpoint{2.777471in}{2.118410in}}%
\pgfpathlineto{\pgfqpoint{2.776498in}{2.115807in}}%
\pgfpathlineto{\pgfqpoint{2.773243in}{2.108116in}}%
\pgfpathlineto{\pgfqpoint{2.769566in}{2.097821in}}%
\pgfpathlineto{\pgfqpoint{2.766458in}{2.087527in}}%
\pgfpathlineto{\pgfqpoint{2.765921in}{2.085440in}}%
\pgfpathlineto{\pgfqpoint{2.763415in}{2.077232in}}%
\pgfpathlineto{\pgfqpoint{2.760700in}{2.066938in}}%
\pgfpathlineto{\pgfqpoint{2.758317in}{2.056643in}}%
\pgfpathlineto{\pgfqpoint{2.756197in}{2.046349in}}%
\pgfpathlineto{\pgfqpoint{2.755344in}{2.041703in}}%
\pgfpathlineto{\pgfqpoint{2.754017in}{2.036054in}}%
\pgfpathlineto{\pgfqpoint{2.751887in}{2.025760in}}%
\pgfpathlineto{\pgfqpoint{2.750095in}{2.015466in}}%
\pgfpathlineto{\pgfqpoint{2.748742in}{2.005171in}}%
\pgfpathlineto{\pgfqpoint{2.747949in}{1.994877in}}%
\pgfpathlineto{\pgfqpoint{2.747837in}{1.984582in}}%
\pgfpathlineto{\pgfqpoint{2.748202in}{1.974288in}}%
\pgfpathlineto{\pgfqpoint{2.746156in}{1.963993in}}%
\pgfpathlineto{\pgfqpoint{2.744767in}{1.957042in}}%
\pgfpathlineto{\pgfqpoint{2.743795in}{1.953699in}}%
\pgfpathlineto{\pgfqpoint{2.741023in}{1.943405in}}%
\pgfpathlineto{\pgfqpoint{2.738342in}{1.933110in}}%
\pgfpathlineto{\pgfqpoint{2.735583in}{1.922816in}}%
\pgfpathlineto{\pgfqpoint{2.734190in}{1.918090in}}%
\pgfpathlineto{\pgfqpoint{2.731038in}{1.912521in}}%
\pgfpathlineto{\pgfqpoint{2.723898in}{1.902227in}}%
\pgfpathlineto{\pgfqpoint{2.723613in}{1.901905in}}%
\pgfpathlineto{\pgfqpoint{2.713036in}{1.900897in}}%
\pgfpathlineto{\pgfqpoint{2.711572in}{1.902227in}}%
\pgfpathlineto{\pgfqpoint{2.702459in}{1.912275in}}%
\pgfpathlineto{\pgfqpoint{2.702353in}{1.912521in}}%
\pgfpathlineto{\pgfqpoint{2.698631in}{1.922816in}}%
\pgfpathlineto{\pgfqpoint{2.695336in}{1.933110in}}%
\pgfpathlineto{\pgfqpoint{2.692227in}{1.943405in}}%
\pgfpathlineto{\pgfqpoint{2.691882in}{1.944532in}}%
\pgfpathlineto{\pgfqpoint{2.690059in}{1.953699in}}%
\pgfpathlineto{\pgfqpoint{2.687955in}{1.963993in}}%
\pgfpathlineto{\pgfqpoint{2.688322in}{1.974288in}}%
\pgfpathlineto{\pgfqpoint{2.688395in}{1.984582in}}%
\pgfpathlineto{\pgfqpoint{2.687586in}{1.994877in}}%
\pgfpathlineto{\pgfqpoint{2.685995in}{2.005171in}}%
\pgfpathlineto{\pgfqpoint{2.683740in}{2.015466in}}%
\pgfpathlineto{\pgfqpoint{2.681305in}{2.024486in}}%
\pgfpathlineto{\pgfqpoint{2.681041in}{2.025760in}}%
\pgfpathlineto{\pgfqpoint{2.678634in}{2.036054in}}%
\pgfpathlineto{\pgfqpoint{2.675989in}{2.046349in}}%
\pgfpathlineto{\pgfqpoint{2.673106in}{2.056643in}}%
\pgfpathlineto{\pgfqpoint{2.670727in}{2.064362in}}%
\pgfpathlineto{\pgfqpoint{2.670077in}{2.066938in}}%
\pgfpathlineto{\pgfqpoint{2.667182in}{2.077232in}}%
\pgfpathlineto{\pgfqpoint{2.663877in}{2.087527in}}%
\pgfpathlineto{\pgfqpoint{2.660150in}{2.097537in}}%
\pgfpathlineto{\pgfqpoint{2.660059in}{2.097821in}}%
\pgfpathlineto{\pgfqpoint{2.656226in}{2.108116in}}%
\pgfpathlineto{\pgfqpoint{2.651718in}{2.118410in}}%
\pgfpathlineto{\pgfqpoint{2.649573in}{2.122587in}}%
\pgfpathlineto{\pgfqpoint{2.646772in}{2.128704in}}%
\pgfpathlineto{\pgfqpoint{2.641262in}{2.138999in}}%
\pgfpathlineto{\pgfqpoint{2.638996in}{2.142643in}}%
\pgfpathlineto{\pgfqpoint{2.635206in}{2.149293in}}%
\pgfpathlineto{\pgfqpoint{2.628441in}{2.159588in}}%
\pgfpathlineto{\pgfqpoint{2.628419in}{2.159618in}}%
\pgfpathlineto{\pgfqpoint{2.621233in}{2.169882in}}%
\pgfpathlineto{\pgfqpoint{2.617842in}{2.174246in}}%
\pgfpathlineto{\pgfqpoint{2.613409in}{2.180177in}}%
\pgfpathlineto{\pgfqpoint{2.607265in}{2.189371in}}%
\pgfpathlineto{\pgfqpoint{2.606658in}{2.190471in}}%
\pgfpathlineto{\pgfqpoint{2.602283in}{2.200765in}}%
\pgfpathlineto{\pgfqpoint{2.598034in}{2.211060in}}%
\pgfpathlineto{\pgfqpoint{2.596688in}{2.214282in}}%
\pgfpathlineto{\pgfqpoint{2.593792in}{2.221354in}}%
\pgfpathlineto{\pgfqpoint{2.589463in}{2.231649in}}%
\pgfpathlineto{\pgfqpoint{2.586111in}{2.239369in}}%
\pgfpathlineto{\pgfqpoint{2.584984in}{2.241943in}}%
\pgfpathlineto{\pgfqpoint{2.580309in}{2.252238in}}%
\pgfpathlineto{\pgfqpoint{2.575534in}{2.262300in}}%
\pgfpathlineto{\pgfqpoint{2.575421in}{2.262532in}}%
\pgfpathlineto{\pgfqpoint{2.570164in}{2.272826in}}%
\pgfpathlineto{\pgfqpoint{2.564957in}{2.282518in}}%
\pgfpathlineto{\pgfqpoint{2.564618in}{2.283121in}}%
\pgfpathlineto{\pgfqpoint{2.558536in}{2.293415in}}%
\pgfpathlineto{\pgfqpoint{2.554380in}{2.300096in}}%
\pgfpathlineto{\pgfqpoint{2.551997in}{2.303710in}}%
\pgfpathlineto{\pgfqpoint{2.544875in}{2.314004in}}%
\pgfpathlineto{\pgfqpoint{2.543803in}{2.315479in}}%
\pgfpathlineto{\pgfqpoint{2.536930in}{2.324299in}}%
\pgfpathlineto{\pgfqpoint{2.533226in}{2.328865in}}%
\pgfpathlineto{\pgfqpoint{2.528162in}{2.334593in}}%
\pgfpathlineto{\pgfqpoint{2.522649in}{2.340629in}}%
\pgfpathlineto{\pgfqpoint{2.518336in}{2.344887in}}%
\pgfpathlineto{\pgfqpoint{2.512072in}{2.350920in}}%
\pgfpathlineto{\pgfqpoint{2.507076in}{2.355182in}}%
\pgfpathlineto{\pgfqpoint{2.501495in}{2.359859in}}%
\pgfpathlineto{\pgfqpoint{2.493777in}{2.365476in}}%
\pgfpathlineto{\pgfqpoint{2.490917in}{2.367534in}}%
\pgfpathlineto{\pgfqpoint{2.480340in}{2.373929in}}%
\pgfpathlineto{\pgfqpoint{2.476544in}{2.375771in}}%
\pgfpathlineto{\pgfqpoint{2.469763in}{2.379048in}}%
\pgfpathlineto{\pgfqpoint{2.459186in}{2.382921in}}%
\pgfpathlineto{\pgfqpoint{2.448609in}{2.385563in}}%
\pgfpathlineto{\pgfqpoint{2.444590in}{2.386065in}}%
\pgfpathlineto{\pgfqpoint{2.438032in}{2.386887in}}%
\pgfpathlineto{\pgfqpoint{2.427455in}{2.386956in}}%
\pgfpathlineto{\pgfqpoint{2.419007in}{2.386065in}}%
\pgfpathlineto{\pgfqpoint{2.416878in}{2.385813in}}%
\pgfpathlineto{\pgfqpoint{2.406301in}{2.383421in}}%
\pgfpathlineto{\pgfqpoint{2.395724in}{2.379979in}}%
\pgfpathlineto{\pgfqpoint{2.385163in}{2.375771in}}%
\pgfpathlineto{\pgfqpoint{2.385147in}{2.375763in}}%
\pgfpathlineto{\pgfqpoint{2.374570in}{2.370876in}}%
\pgfpathlineto{\pgfqpoint{2.363993in}{2.366601in}}%
\pgfpathlineto{\pgfqpoint{2.358612in}{2.365476in}}%
\pgfpathlineto{\pgfqpoint{2.353416in}{2.364281in}}%
\pgfpathlineto{\pgfqpoint{2.342839in}{2.365361in}}%
\pgfpathlineto{\pgfqpoint{2.342578in}{2.365476in}}%
\pgfpathlineto{\pgfqpoint{2.332262in}{2.369745in}}%
\pgfpathlineto{\pgfqpoint{2.323568in}{2.375771in}}%
\pgfpathlineto{\pgfqpoint{2.321684in}{2.377004in}}%
\pgfpathlineto{\pgfqpoint{2.311602in}{2.386065in}}%
\pgfpathlineto{\pgfqpoint{2.311107in}{2.386488in}}%
\pgfpathlineto{\pgfqpoint{2.301889in}{2.396360in}}%
\pgfpathlineto{\pgfqpoint{2.300530in}{2.397754in}}%
\pgfpathlineto{\pgfqpoint{2.293230in}{2.406654in}}%
\pgfpathlineto{\pgfqpoint{2.289953in}{2.410513in}}%
\pgfpathlineto{\pgfqpoint{2.285191in}{2.416948in}}%
\pgfpathlineto{\pgfqpoint{2.279376in}{2.424606in}}%
\pgfpathlineto{\pgfqpoint{2.277587in}{2.427243in}}%
\pgfpathlineto{\pgfqpoint{2.270401in}{2.437537in}}%
\pgfpathlineto{\pgfqpoint{2.268799in}{2.439764in}}%
\pgfpathlineto{\pgfqpoint{2.263532in}{2.447832in}}%
\pgfpathlineto{\pgfqpoint{2.258222in}{2.455942in}}%
\pgfpathlineto{\pgfqpoint{2.256902in}{2.458126in}}%
\pgfpathlineto{\pgfqpoint{2.250589in}{2.468421in}}%
\pgfpathlineto{\pgfqpoint{2.247645in}{2.473230in}}%
\pgfpathlineto{\pgfqpoint{2.244515in}{2.478715in}}%
\pgfpathlineto{\pgfqpoint{2.238727in}{2.489009in}}%
\pgfpathlineto{\pgfqpoint{2.237068in}{2.491972in}}%
\pgfpathlineto{\pgfqpoint{2.233202in}{2.499304in}}%
\pgfpathlineto{\pgfqpoint{2.227971in}{2.509598in}}%
\pgfpathlineto{\pgfqpoint{2.226491in}{2.512563in}}%
\pgfpathlineto{\pgfqpoint{2.223009in}{2.519893in}}%
\pgfpathlineto{\pgfqpoint{2.218380in}{2.530187in}}%
\pgfpathlineto{\pgfqpoint{2.215914in}{2.536049in}}%
\pgfpathlineto{\pgfqpoint{2.214113in}{2.540482in}}%
\pgfpathlineto{\pgfqpoint{2.210375in}{2.550776in}}%
\pgfpathlineto{\pgfqpoint{2.207452in}{2.561071in}}%
\pgfpathlineto{\pgfqpoint{2.205337in}{2.570476in}}%
\pgfpathlineto{\pgfqpoint{2.205133in}{2.571365in}}%
\pgfpathlineto{\pgfqpoint{2.203045in}{2.581659in}}%
\pgfpathlineto{\pgfqpoint{2.201069in}{2.591954in}}%
\pgfpathlineto{\pgfqpoint{2.199119in}{2.602248in}}%
\pgfpathlineto{\pgfqpoint{2.197145in}{2.612543in}}%
\pgfpathlineto{\pgfqpoint{2.195113in}{2.622837in}}%
\pgfpathlineto{\pgfqpoint{2.194760in}{2.624531in}}%
\pgfpathlineto{\pgfqpoint{2.192953in}{2.633132in}}%
\pgfpathlineto{\pgfqpoint{2.190672in}{2.643426in}}%
\pgfpathlineto{\pgfqpoint{2.188253in}{2.653720in}}%
\pgfpathlineto{\pgfqpoint{2.185675in}{2.664015in}}%
\pgfpathlineto{\pgfqpoint{2.184183in}{2.669542in}}%
\pgfpathlineto{\pgfqpoint{2.182888in}{2.674309in}}%
\pgfpathlineto{\pgfqpoint{2.179879in}{2.684604in}}%
\pgfpathlineto{\pgfqpoint{2.176691in}{2.694898in}}%
\pgfpathlineto{\pgfqpoint{2.173606in}{2.704325in}}%
\pgfpathlineto{\pgfqpoint{2.173320in}{2.705193in}}%
\pgfpathlineto{\pgfqpoint{2.169686in}{2.715487in}}%
\pgfpathlineto{\pgfqpoint{2.165924in}{2.725781in}}%
\pgfpathlineto{\pgfqpoint{2.163029in}{2.733440in}}%
\pgfpathlineto{\pgfqpoint{2.162023in}{2.736076in}}%
\pgfpathlineto{\pgfqpoint{2.157961in}{2.746370in}}%
\pgfpathlineto{\pgfqpoint{2.153925in}{2.756665in}}%
\pgfpathlineto{\pgfqpoint{2.152451in}{2.760552in}}%
\pgfpathlineto{\pgfqpoint{2.149980in}{2.766959in}}%
\pgfpathlineto{\pgfqpoint{2.146636in}{2.777254in}}%
\pgfpathlineto{\pgfqpoint{2.144108in}{2.787548in}}%
\pgfpathlineto{\pgfqpoint{2.141974in}{2.797842in}}%
\pgfpathlineto{\pgfqpoint{2.141874in}{2.798362in}}%
\pgfpathlineto{\pgfqpoint{2.139918in}{2.808137in}}%
\pgfpathlineto{\pgfqpoint{2.137943in}{2.818431in}}%
\pgfpathlineto{\pgfqpoint{2.136013in}{2.828726in}}%
\pgfpathlineto{\pgfqpoint{2.134095in}{2.839020in}}%
\pgfpathlineto{\pgfqpoint{2.132154in}{2.849315in}}%
\pgfpathlineto{\pgfqpoint{2.131297in}{2.853775in}}%
\pgfpathlineto{\pgfqpoint{2.130084in}{2.859609in}}%
\pgfpathlineto{\pgfqpoint{2.127858in}{2.869903in}}%
\pgfpathlineto{\pgfqpoint{2.125478in}{2.880198in}}%
\pgfpathlineto{\pgfqpoint{2.122888in}{2.890492in}}%
\pgfpathlineto{\pgfqpoint{2.120720in}{2.898375in}}%
\pgfpathlineto{\pgfqpoint{2.119974in}{2.900787in}}%
\pgfpathlineto{\pgfqpoint{2.116545in}{2.911081in}}%
\pgfpathlineto{\pgfqpoint{2.110143in}{2.911081in}}%
\pgfpathlineto{\pgfqpoint{2.099566in}{2.911081in}}%
\pgfpathlineto{\pgfqpoint{2.088989in}{2.911081in}}%
\pgfpathlineto{\pgfqpoint{2.078412in}{2.911081in}}%
\pgfpathlineto{\pgfqpoint{2.070011in}{2.911081in}}%
\pgfpathlineto{\pgfqpoint{2.073350in}{2.900787in}}%
\pgfpathlineto{\pgfqpoint{2.078412in}{2.890510in}}%
\pgfpathlineto{\pgfqpoint{2.078423in}{2.890492in}}%
\pgfpathlineto{\pgfqpoint{2.084107in}{2.880198in}}%
\pgfpathlineto{\pgfqpoint{2.088989in}{2.870201in}}%
\pgfpathlineto{\pgfqpoint{2.089114in}{2.869903in}}%
\pgfpathlineto{\pgfqpoint{2.093036in}{2.859609in}}%
\pgfpathlineto{\pgfqpoint{2.096604in}{2.849315in}}%
\pgfpathlineto{\pgfqpoint{2.099566in}{2.840129in}}%
\pgfpathlineto{\pgfqpoint{2.099878in}{2.839020in}}%
\pgfpathlineto{\pgfqpoint{2.102607in}{2.828726in}}%
\pgfpathlineto{\pgfqpoint{2.105213in}{2.818431in}}%
\pgfpathlineto{\pgfqpoint{2.107748in}{2.808137in}}%
\pgfpathlineto{\pgfqpoint{2.110143in}{2.798283in}}%
\pgfpathlineto{\pgfqpoint{2.110239in}{2.797842in}}%
\pgfpathlineto{\pgfqpoint{2.112439in}{2.787548in}}%
\pgfpathlineto{\pgfqpoint{2.114648in}{2.777254in}}%
\pgfpathlineto{\pgfqpoint{2.116911in}{2.766959in}}%
\pgfpathlineto{\pgfqpoint{2.120369in}{2.756665in}}%
\pgfpathlineto{\pgfqpoint{2.120720in}{2.755712in}}%
\pgfpathlineto{\pgfqpoint{2.123990in}{2.746370in}}%
\pgfpathlineto{\pgfqpoint{2.127566in}{2.736076in}}%
\pgfpathlineto{\pgfqpoint{2.131114in}{2.725781in}}%
\pgfpathlineto{\pgfqpoint{2.131297in}{2.725216in}}%
\pgfpathlineto{\pgfqpoint{2.134281in}{2.715487in}}%
\pgfpathlineto{\pgfqpoint{2.137328in}{2.705193in}}%
\pgfpathlineto{\pgfqpoint{2.140255in}{2.694898in}}%
\pgfpathlineto{\pgfqpoint{2.141874in}{2.688838in}}%
\pgfpathlineto{\pgfqpoint{2.142945in}{2.684604in}}%
\pgfpathlineto{\pgfqpoint{2.145348in}{2.674309in}}%
\pgfpathlineto{\pgfqpoint{2.147603in}{2.664015in}}%
\pgfpathlineto{\pgfqpoint{2.149716in}{2.653720in}}%
\pgfpathlineto{\pgfqpoint{2.151700in}{2.643426in}}%
\pgfpathlineto{\pgfqpoint{2.152451in}{2.639209in}}%
\pgfpathlineto{\pgfqpoint{2.153469in}{2.633132in}}%
\pgfpathlineto{\pgfqpoint{2.155069in}{2.622837in}}%
\pgfpathlineto{\pgfqpoint{2.156585in}{2.612543in}}%
\pgfpathlineto{\pgfqpoint{2.158041in}{2.602248in}}%
\pgfpathlineto{\pgfqpoint{2.159466in}{2.591954in}}%
\pgfpathlineto{\pgfqpoint{2.160907in}{2.581659in}}%
\pgfpathlineto{\pgfqpoint{2.162439in}{2.571365in}}%
\pgfpathlineto{\pgfqpoint{2.163029in}{2.567747in}}%
\pgfpathlineto{\pgfqpoint{2.164047in}{2.561071in}}%
\pgfpathlineto{\pgfqpoint{2.165832in}{2.550776in}}%
\pgfpathlineto{\pgfqpoint{2.167865in}{2.540482in}}%
\pgfpathlineto{\pgfqpoint{2.170098in}{2.530187in}}%
\pgfpathlineto{\pgfqpoint{2.172484in}{2.519893in}}%
\pgfpathlineto{\pgfqpoint{2.173606in}{2.515020in}}%
\pgfpathlineto{\pgfqpoint{2.174807in}{2.509598in}}%
\pgfpathlineto{\pgfqpoint{2.176999in}{2.499304in}}%
\pgfpathlineto{\pgfqpoint{2.179120in}{2.489009in}}%
\pgfpathlineto{\pgfqpoint{2.181068in}{2.478715in}}%
\pgfpathlineto{\pgfqpoint{2.182683in}{2.468421in}}%
\pgfpathlineto{\pgfqpoint{2.183682in}{2.458126in}}%
\pgfpathlineto{\pgfqpoint{2.183514in}{2.447832in}}%
\pgfpathlineto{\pgfqpoint{2.180920in}{2.437537in}}%
\pgfpathlineto{\pgfqpoint{2.173606in}{2.428368in}}%
\pgfpathlineto{\pgfqpoint{2.169147in}{2.427243in}}%
\pgfpathlineto{\pgfqpoint{2.163029in}{2.426343in}}%
\pgfpathlineto{\pgfqpoint{2.158909in}{2.427243in}}%
\pgfpathlineto{\pgfqpoint{2.152451in}{2.428531in}}%
\pgfpathlineto{\pgfqpoint{2.149865in}{2.427243in}}%
\pgfpathlineto{\pgfqpoint{2.141874in}{2.421880in}}%
\pgfpathlineto{\pgfqpoint{2.131297in}{2.419938in}}%
\pgfpathlineto{\pgfqpoint{2.120720in}{2.423742in}}%
\pgfpathlineto{\pgfqpoint{2.116449in}{2.427243in}}%
\pgfpathlineto{\pgfqpoint{2.110143in}{2.432012in}}%
\pgfpathlineto{\pgfqpoint{2.105488in}{2.437537in}}%
\pgfpathlineto{\pgfqpoint{2.099566in}{2.444082in}}%
\pgfpathlineto{\pgfqpoint{2.097125in}{2.447832in}}%
\pgfpathlineto{\pgfqpoint{2.090048in}{2.458126in}}%
\pgfpathlineto{\pgfqpoint{2.088989in}{2.459613in}}%
\pgfpathlineto{\pgfqpoint{2.084214in}{2.468421in}}%
\pgfpathlineto{\pgfqpoint{2.078412in}{2.478477in}}%
\pgfpathlineto{\pgfqpoint{2.078302in}{2.478715in}}%
\pgfpathlineto{\pgfqpoint{2.073526in}{2.489009in}}%
\pgfpathlineto{\pgfqpoint{2.068472in}{2.499304in}}%
\pgfpathlineto{\pgfqpoint{2.067835in}{2.500590in}}%
\pgfpathlineto{\pgfqpoint{2.064130in}{2.509598in}}%
\pgfpathlineto{\pgfqpoint{2.059679in}{2.519893in}}%
\pgfpathlineto{\pgfqpoint{2.057258in}{2.525246in}}%
\pgfpathlineto{\pgfqpoint{2.055327in}{2.530187in}}%
\pgfpathlineto{\pgfqpoint{2.051066in}{2.540482in}}%
\pgfpathlineto{\pgfqpoint{2.046681in}{2.550126in}}%
\pgfpathlineto{\pgfqpoint{2.046428in}{2.550776in}}%
\pgfpathlineto{\pgfqpoint{2.042393in}{2.561071in}}%
\pgfpathlineto{\pgfqpoint{2.038285in}{2.571365in}}%
\pgfpathlineto{\pgfqpoint{2.036104in}{2.576894in}}%
\pgfpathlineto{\pgfqpoint{2.034503in}{2.581659in}}%
\pgfpathlineto{\pgfqpoint{2.031144in}{2.591954in}}%
\pgfpathlineto{\pgfqpoint{2.027822in}{2.602248in}}%
\pgfpathlineto{\pgfqpoint{2.025527in}{2.609474in}}%
\pgfpathlineto{\pgfqpoint{2.024690in}{2.612543in}}%
\pgfpathlineto{\pgfqpoint{2.021965in}{2.622837in}}%
\pgfpathlineto{\pgfqpoint{2.019228in}{2.633132in}}%
\pgfpathlineto{\pgfqpoint{2.016438in}{2.643426in}}%
\pgfpathlineto{\pgfqpoint{2.014950in}{2.648833in}}%
\pgfpathlineto{\pgfqpoint{2.013775in}{2.653720in}}%
\pgfpathlineto{\pgfqpoint{2.011251in}{2.664015in}}%
\pgfpathlineto{\pgfqpoint{2.008615in}{2.674309in}}%
\pgfpathlineto{\pgfqpoint{2.005865in}{2.684604in}}%
\pgfpathlineto{\pgfqpoint{2.004373in}{2.690102in}}%
\pgfpathlineto{\pgfqpoint{2.003220in}{2.694898in}}%
\pgfpathlineto{\pgfqpoint{2.000742in}{2.705193in}}%
\pgfpathlineto{\pgfqpoint{1.998220in}{2.715487in}}%
\pgfpathlineto{\pgfqpoint{1.995660in}{2.725781in}}%
\pgfpathlineto{\pgfqpoint{1.993796in}{2.733267in}}%
\pgfpathlineto{\pgfqpoint{1.993169in}{2.736076in}}%
\pgfpathlineto{\pgfqpoint{1.990927in}{2.746370in}}%
\pgfpathlineto{\pgfqpoint{1.988678in}{2.756665in}}%
\pgfpathlineto{\pgfqpoint{1.986425in}{2.766959in}}%
\pgfpathlineto{\pgfqpoint{1.984176in}{2.777254in}}%
\pgfpathlineto{\pgfqpoint{1.983218in}{2.781819in}}%
\pgfpathlineto{\pgfqpoint{1.982094in}{2.787548in}}%
\pgfpathlineto{\pgfqpoint{1.979899in}{2.797842in}}%
\pgfpathlineto{\pgfqpoint{1.977308in}{2.808137in}}%
\pgfpathlineto{\pgfqpoint{1.974719in}{2.818431in}}%
\pgfpathlineto{\pgfqpoint{1.972641in}{2.826462in}}%
\pgfpathlineto{\pgfqpoint{1.972075in}{2.828726in}}%
\pgfpathlineto{\pgfqpoint{1.969636in}{2.839020in}}%
\pgfpathlineto{\pgfqpoint{1.967324in}{2.849315in}}%
\pgfpathlineto{\pgfqpoint{1.965070in}{2.859609in}}%
\pgfpathlineto{\pgfqpoint{1.962781in}{2.869903in}}%
\pgfpathlineto{\pgfqpoint{1.962064in}{2.873029in}}%
\pgfpathlineto{\pgfqpoint{1.960612in}{2.880198in}}%
\pgfpathlineto{\pgfqpoint{1.958491in}{2.890492in}}%
\pgfpathlineto{\pgfqpoint{1.956204in}{2.900787in}}%
\pgfpathlineto{\pgfqpoint{1.953675in}{2.911081in}}%
\pgfpathlineto{\pgfqpoint{1.951487in}{2.911081in}}%
\pgfpathlineto{\pgfqpoint{1.940910in}{2.911081in}}%
\pgfpathlineto{\pgfqpoint{1.930333in}{2.911081in}}%
\pgfpathlineto{\pgfqpoint{1.930333in}{2.900787in}}%
\pgfpathlineto{\pgfqpoint{1.930333in}{2.890492in}}%
\pgfpathlineto{\pgfqpoint{1.930333in}{2.880198in}}%
\pgfpathlineto{\pgfqpoint{1.930333in}{2.869903in}}%
\pgfpathlineto{\pgfqpoint{1.930333in}{2.861617in}}%
\pgfpathlineto{\pgfqpoint{1.930819in}{2.859609in}}%
\pgfpathlineto{\pgfqpoint{1.933714in}{2.849315in}}%
\pgfpathlineto{\pgfqpoint{1.936755in}{2.839020in}}%
\pgfpathlineto{\pgfqpoint{1.939840in}{2.828726in}}%
\pgfpathlineto{\pgfqpoint{1.940910in}{2.825227in}}%
\pgfpathlineto{\pgfqpoint{1.943001in}{2.818431in}}%
\pgfpathlineto{\pgfqpoint{1.946128in}{2.808137in}}%
\pgfpathlineto{\pgfqpoint{1.949114in}{2.797842in}}%
\pgfpathlineto{\pgfqpoint{1.951487in}{2.789275in}}%
\pgfpathlineto{\pgfqpoint{1.951978in}{2.787548in}}%
\pgfpathlineto{\pgfqpoint{1.954853in}{2.777254in}}%
\pgfpathlineto{\pgfqpoint{1.957590in}{2.766959in}}%
\pgfpathlineto{\pgfqpoint{1.960223in}{2.756665in}}%
\pgfpathlineto{\pgfqpoint{1.962064in}{2.749321in}}%
\pgfpathlineto{\pgfqpoint{1.962839in}{2.746370in}}%
\pgfpathlineto{\pgfqpoint{1.965549in}{2.736076in}}%
\pgfpathlineto{\pgfqpoint{1.968209in}{2.725781in}}%
\pgfpathlineto{\pgfqpoint{1.970829in}{2.715487in}}%
\pgfpathlineto{\pgfqpoint{1.972641in}{2.708335in}}%
\pgfpathlineto{\pgfqpoint{1.973489in}{2.705193in}}%
\pgfpathlineto{\pgfqpoint{1.976285in}{2.694898in}}%
\pgfpathlineto{\pgfqpoint{1.979029in}{2.684604in}}%
\pgfpathlineto{\pgfqpoint{1.981713in}{2.674309in}}%
\pgfpathlineto{\pgfqpoint{1.983218in}{2.668458in}}%
\pgfpathlineto{\pgfqpoint{1.984448in}{2.664015in}}%
\pgfpathlineto{\pgfqpoint{1.987241in}{2.653720in}}%
\pgfpathlineto{\pgfqpoint{1.989899in}{2.643426in}}%
\pgfpathlineto{\pgfqpoint{1.992415in}{2.633132in}}%
\pgfpathlineto{\pgfqpoint{1.993796in}{2.627264in}}%
\pgfpathlineto{\pgfqpoint{1.994944in}{2.622837in}}%
\pgfpathlineto{\pgfqpoint{1.997568in}{2.612543in}}%
\pgfpathlineto{\pgfqpoint{2.000160in}{2.602248in}}%
\pgfpathlineto{\pgfqpoint{2.002765in}{2.591954in}}%
\pgfpathlineto{\pgfqpoint{2.004373in}{2.585745in}}%
\pgfpathlineto{\pgfqpoint{2.005555in}{2.581659in}}%
\pgfpathlineto{\pgfqpoint{2.008614in}{2.571365in}}%
\pgfpathlineto{\pgfqpoint{2.011730in}{2.561071in}}%
\pgfpathlineto{\pgfqpoint{2.014901in}{2.550776in}}%
\pgfpathlineto{\pgfqpoint{2.014950in}{2.550622in}}%
\pgfpathlineto{\pgfqpoint{2.018527in}{2.540482in}}%
\pgfpathlineto{\pgfqpoint{2.022140in}{2.530187in}}%
\pgfpathlineto{\pgfqpoint{2.025527in}{2.520419in}}%
\pgfpathlineto{\pgfqpoint{2.025728in}{2.519893in}}%
\pgfpathlineto{\pgfqpoint{2.029668in}{2.509598in}}%
\pgfpathlineto{\pgfqpoint{2.033470in}{2.499304in}}%
\pgfpathlineto{\pgfqpoint{2.036104in}{2.491998in}}%
\pgfpathlineto{\pgfqpoint{2.037285in}{2.489009in}}%
\pgfpathlineto{\pgfqpoint{2.041309in}{2.478715in}}%
\pgfpathlineto{\pgfqpoint{2.045192in}{2.468421in}}%
\pgfpathlineto{\pgfqpoint{2.046681in}{2.464428in}}%
\pgfpathlineto{\pgfqpoint{2.049293in}{2.458126in}}%
\pgfpathlineto{\pgfqpoint{2.053467in}{2.447832in}}%
\pgfpathlineto{\pgfqpoint{2.057258in}{2.438147in}}%
\pgfpathlineto{\pgfqpoint{2.057529in}{2.437537in}}%
\pgfpathlineto{\pgfqpoint{2.062069in}{2.427243in}}%
\pgfpathlineto{\pgfqpoint{2.066405in}{2.416948in}}%
\pgfpathlineto{\pgfqpoint{2.067835in}{2.413481in}}%
\pgfpathlineto{\pgfqpoint{2.071076in}{2.406654in}}%
\pgfpathlineto{\pgfqpoint{2.075761in}{2.396360in}}%
\pgfpathlineto{\pgfqpoint{2.078412in}{2.390291in}}%
\pgfpathlineto{\pgfqpoint{2.080577in}{2.386065in}}%
\pgfpathlineto{\pgfqpoint{2.085635in}{2.375771in}}%
\pgfpathlineto{\pgfqpoint{2.088989in}{2.368561in}}%
\pgfpathlineto{\pgfqpoint{2.090702in}{2.365476in}}%
\pgfpathlineto{\pgfqpoint{2.096170in}{2.355182in}}%
\pgfpathlineto{\pgfqpoint{2.099566in}{2.348394in}}%
\pgfpathlineto{\pgfqpoint{2.101698in}{2.344887in}}%
\pgfpathlineto{\pgfqpoint{2.107642in}{2.334593in}}%
\pgfpathlineto{\pgfqpoint{2.110143in}{2.330024in}}%
\pgfpathlineto{\pgfqpoint{2.114023in}{2.324299in}}%
\pgfpathlineto{\pgfqpoint{2.120541in}{2.314004in}}%
\pgfpathlineto{\pgfqpoint{2.120720in}{2.313713in}}%
\pgfpathlineto{\pgfqpoint{2.128461in}{2.303710in}}%
\pgfpathlineto{\pgfqpoint{2.131297in}{2.299815in}}%
\pgfpathlineto{\pgfqpoint{2.137364in}{2.293415in}}%
\pgfpathlineto{\pgfqpoint{2.141874in}{2.288376in}}%
\pgfpathlineto{\pgfqpoint{2.148251in}{2.283121in}}%
\pgfpathlineto{\pgfqpoint{2.152451in}{2.279463in}}%
\pgfpathlineto{\pgfqpoint{2.163029in}{2.273031in}}%
\pgfpathlineto{\pgfqpoint{2.163632in}{2.272826in}}%
\pgfpathlineto{\pgfqpoint{2.173606in}{2.269236in}}%
\pgfpathlineto{\pgfqpoint{2.184183in}{2.267943in}}%
\pgfpathlineto{\pgfqpoint{2.194760in}{2.269199in}}%
\pgfpathlineto{\pgfqpoint{2.205337in}{2.272800in}}%
\pgfpathlineto{\pgfqpoint{2.205397in}{2.272826in}}%
\pgfpathlineto{\pgfqpoint{2.215914in}{2.277703in}}%
\pgfpathlineto{\pgfqpoint{2.226491in}{2.280872in}}%
\pgfpathlineto{\pgfqpoint{2.237068in}{2.280003in}}%
\pgfpathlineto{\pgfqpoint{2.247645in}{2.276270in}}%
\pgfpathlineto{\pgfqpoint{2.254100in}{2.272826in}}%
\pgfpathlineto{\pgfqpoint{2.258222in}{2.270599in}}%
\pgfpathlineto{\pgfqpoint{2.268799in}{2.263204in}}%
\pgfpathlineto{\pgfqpoint{2.269602in}{2.262532in}}%
\pgfpathlineto{\pgfqpoint{2.279376in}{2.253761in}}%
\pgfpathlineto{\pgfqpoint{2.280796in}{2.252238in}}%
\pgfpathlineto{\pgfqpoint{2.289593in}{2.241943in}}%
\pgfpathlineto{\pgfqpoint{2.289953in}{2.241492in}}%
\pgfpathlineto{\pgfqpoint{2.296726in}{2.231649in}}%
\pgfpathlineto{\pgfqpoint{2.300530in}{2.225435in}}%
\pgfpathlineto{\pgfqpoint{2.302744in}{2.221354in}}%
\pgfpathlineto{\pgfqpoint{2.307838in}{2.211060in}}%
\pgfpathlineto{\pgfqpoint{2.311107in}{2.203335in}}%
\pgfpathlineto{\pgfqpoint{2.312117in}{2.200765in}}%
\pgfpathlineto{\pgfqpoint{2.315225in}{2.190471in}}%
\pgfpathlineto{\pgfqpoint{2.317184in}{2.180177in}}%
\pgfpathlineto{\pgfqpoint{2.318580in}{2.169882in}}%
\pgfpathlineto{\pgfqpoint{2.319897in}{2.159588in}}%
\pgfpathlineto{\pgfqpoint{2.321408in}{2.149293in}}%
\pgfpathlineto{\pgfqpoint{2.321684in}{2.147763in}}%
\pgfpathlineto{\pgfqpoint{2.323095in}{2.138999in}}%
\pgfpathlineto{\pgfqpoint{2.325098in}{2.128704in}}%
\pgfpathlineto{\pgfqpoint{2.327455in}{2.118410in}}%
\pgfpathlineto{\pgfqpoint{2.330165in}{2.108116in}}%
\pgfpathlineto{\pgfqpoint{2.332262in}{2.101128in}}%
\pgfpathlineto{\pgfqpoint{2.333170in}{2.097821in}}%
\pgfpathlineto{\pgfqpoint{2.336395in}{2.087527in}}%
\pgfpathlineto{\pgfqpoint{2.339921in}{2.077232in}}%
\pgfpathlineto{\pgfqpoint{2.342839in}{2.069384in}}%
\pgfpathlineto{\pgfqpoint{2.343687in}{2.066938in}}%
\pgfpathlineto{\pgfqpoint{2.347606in}{2.056643in}}%
\pgfpathlineto{\pgfqpoint{2.351721in}{2.046349in}}%
\pgfpathlineto{\pgfqpoint{2.353416in}{2.042363in}}%
\pgfpathlineto{\pgfqpoint{2.355962in}{2.036054in}}%
\pgfpathlineto{\pgfqpoint{2.360297in}{2.025760in}}%
\pgfpathlineto{\pgfqpoint{2.363993in}{2.017105in}}%
\pgfpathlineto{\pgfqpoint{2.364668in}{2.015466in}}%
\pgfpathlineto{\pgfqpoint{2.368901in}{2.005171in}}%
\pgfpathlineto{\pgfqpoint{2.372366in}{1.994877in}}%
\pgfpathlineto{\pgfqpoint{2.374570in}{1.986865in}}%
\pgfpathlineto{\pgfqpoint{2.375201in}{1.984582in}}%
\pgfpathlineto{\pgfqpoint{2.377913in}{1.974288in}}%
\pgfpathlineto{\pgfqpoint{2.380551in}{1.963993in}}%
\pgfpathlineto{\pgfqpoint{2.383148in}{1.953699in}}%
\pgfpathlineto{\pgfqpoint{2.385147in}{1.945700in}}%
\pgfpathlineto{\pgfqpoint{2.385740in}{1.943405in}}%
\pgfpathlineto{\pgfqpoint{2.388391in}{1.933110in}}%
\pgfpathlineto{\pgfqpoint{2.391044in}{1.922816in}}%
\pgfpathlineto{\pgfqpoint{2.393707in}{1.912521in}}%
\pgfpathlineto{\pgfqpoint{2.395724in}{1.904726in}}%
\pgfpathlineto{\pgfqpoint{2.396411in}{1.902227in}}%
\pgfpathlineto{\pgfqpoint{2.399201in}{1.891932in}}%
\pgfpathclose%
\pgfusepath{fill}%
\end{pgfscope}%
\begin{pgfscope}%
\pgfpathrectangle{\pgfqpoint{1.856795in}{1.819814in}}{\pgfqpoint{1.194205in}{1.163386in}}%
\pgfusepath{clip}%
\pgfsetbuttcap%
\pgfsetroundjoin%
\definecolor{currentfill}{rgb}{0.971202,0.827364,0.728520}%
\pgfsetfillcolor{currentfill}%
\pgfsetlinewidth{0.000000pt}%
\definecolor{currentstroke}{rgb}{0.000000,0.000000,0.000000}%
\pgfsetstrokecolor{currentstroke}%
\pgfsetdash{}{0pt}%
\pgfpathmoveto{\pgfqpoint{2.977462in}{1.891932in}}%
\pgfpathlineto{\pgfqpoint{2.977462in}{1.902227in}}%
\pgfpathlineto{\pgfqpoint{2.977462in}{1.906735in}}%
\pgfpathlineto{\pgfqpoint{2.974567in}{1.902227in}}%
\pgfpathlineto{\pgfqpoint{2.968727in}{1.891932in}}%
\pgfpathclose%
\pgfusepath{fill}%
\end{pgfscope}%
\begin{pgfscope}%
\pgfpathrectangle{\pgfqpoint{1.856795in}{1.819814in}}{\pgfqpoint{1.194205in}{1.163386in}}%
\pgfusepath{clip}%
\pgfsetbuttcap%
\pgfsetroundjoin%
\definecolor{currentfill}{rgb}{0.977657,0.891500,0.822809}%
\pgfsetfillcolor{currentfill}%
\pgfsetlinewidth{0.000000pt}%
\definecolor{currentstroke}{rgb}{0.000000,0.000000,0.000000}%
\pgfsetstrokecolor{currentstroke}%
\pgfsetdash{}{0pt}%
\pgfpathmoveto{\pgfqpoint{2.448609in}{1.901668in}}%
\pgfpathlineto{\pgfqpoint{2.454454in}{1.891932in}}%
\pgfpathlineto{\pgfqpoint{2.459186in}{1.891932in}}%
\pgfpathlineto{\pgfqpoint{2.469763in}{1.891932in}}%
\pgfpathlineto{\pgfqpoint{2.480340in}{1.891932in}}%
\pgfpathlineto{\pgfqpoint{2.490917in}{1.891932in}}%
\pgfpathlineto{\pgfqpoint{2.501495in}{1.891932in}}%
\pgfpathlineto{\pgfqpoint{2.512072in}{1.891932in}}%
\pgfpathlineto{\pgfqpoint{2.522649in}{1.891932in}}%
\pgfpathlineto{\pgfqpoint{2.533226in}{1.891932in}}%
\pgfpathlineto{\pgfqpoint{2.543803in}{1.891932in}}%
\pgfpathlineto{\pgfqpoint{2.554380in}{1.891932in}}%
\pgfpathlineto{\pgfqpoint{2.564957in}{1.891932in}}%
\pgfpathlineto{\pgfqpoint{2.575534in}{1.891932in}}%
\pgfpathlineto{\pgfqpoint{2.586111in}{1.891932in}}%
\pgfpathlineto{\pgfqpoint{2.596688in}{1.891932in}}%
\pgfpathlineto{\pgfqpoint{2.600444in}{1.891932in}}%
\pgfpathlineto{\pgfqpoint{2.600826in}{1.902227in}}%
\pgfpathlineto{\pgfqpoint{2.601401in}{1.912521in}}%
\pgfpathlineto{\pgfqpoint{2.602050in}{1.922816in}}%
\pgfpathlineto{\pgfqpoint{2.602684in}{1.933110in}}%
\pgfpathlineto{\pgfqpoint{2.603243in}{1.943405in}}%
\pgfpathlineto{\pgfqpoint{2.603716in}{1.953699in}}%
\pgfpathlineto{\pgfqpoint{2.604361in}{1.963993in}}%
\pgfpathlineto{\pgfqpoint{2.605164in}{1.974288in}}%
\pgfpathlineto{\pgfqpoint{2.605577in}{1.984582in}}%
\pgfpathlineto{\pgfqpoint{2.605552in}{1.994877in}}%
\pgfpathlineto{\pgfqpoint{2.605122in}{2.005171in}}%
\pgfpathlineto{\pgfqpoint{2.604334in}{2.015466in}}%
\pgfpathlineto{\pgfqpoint{2.603241in}{2.025760in}}%
\pgfpathlineto{\pgfqpoint{2.601909in}{2.036054in}}%
\pgfpathlineto{\pgfqpoint{2.600406in}{2.046349in}}%
\pgfpathlineto{\pgfqpoint{2.598769in}{2.056643in}}%
\pgfpathlineto{\pgfqpoint{2.596986in}{2.066938in}}%
\pgfpathlineto{\pgfqpoint{2.596688in}{2.068487in}}%
\pgfpathlineto{\pgfqpoint{2.594986in}{2.077232in}}%
\pgfpathlineto{\pgfqpoint{2.592731in}{2.087527in}}%
\pgfpathlineto{\pgfqpoint{2.590169in}{2.097821in}}%
\pgfpathlineto{\pgfqpoint{2.587243in}{2.108116in}}%
\pgfpathlineto{\pgfqpoint{2.586111in}{2.111589in}}%
\pgfpathlineto{\pgfqpoint{2.583843in}{2.118410in}}%
\pgfpathlineto{\pgfqpoint{2.579962in}{2.128704in}}%
\pgfpathlineto{\pgfqpoint{2.575621in}{2.138999in}}%
\pgfpathlineto{\pgfqpoint{2.575534in}{2.139185in}}%
\pgfpathlineto{\pgfqpoint{2.570798in}{2.149293in}}%
\pgfpathlineto{\pgfqpoint{2.565971in}{2.159588in}}%
\pgfpathlineto{\pgfqpoint{2.564957in}{2.161835in}}%
\pgfpathlineto{\pgfqpoint{2.561265in}{2.169882in}}%
\pgfpathlineto{\pgfqpoint{2.556766in}{2.180177in}}%
\pgfpathlineto{\pgfqpoint{2.554380in}{2.185730in}}%
\pgfpathlineto{\pgfqpoint{2.552167in}{2.190471in}}%
\pgfpathlineto{\pgfqpoint{2.547353in}{2.200765in}}%
\pgfpathlineto{\pgfqpoint{2.543803in}{2.208300in}}%
\pgfpathlineto{\pgfqpoint{2.542352in}{2.211060in}}%
\pgfpathlineto{\pgfqpoint{2.536850in}{2.221354in}}%
\pgfpathlineto{\pgfqpoint{2.533226in}{2.228018in}}%
\pgfpathlineto{\pgfqpoint{2.530963in}{2.231649in}}%
\pgfpathlineto{\pgfqpoint{2.524413in}{2.241943in}}%
\pgfpathlineto{\pgfqpoint{2.522649in}{2.244644in}}%
\pgfpathlineto{\pgfqpoint{2.516789in}{2.252238in}}%
\pgfpathlineto{\pgfqpoint{2.512072in}{2.258222in}}%
\pgfpathlineto{\pgfqpoint{2.507877in}{2.262532in}}%
\pgfpathlineto{\pgfqpoint{2.501495in}{2.268962in}}%
\pgfpathlineto{\pgfqpoint{2.496461in}{2.272826in}}%
\pgfpathlineto{\pgfqpoint{2.490917in}{2.277012in}}%
\pgfpathlineto{\pgfqpoint{2.480340in}{2.282519in}}%
\pgfpathlineto{\pgfqpoint{2.478095in}{2.283121in}}%
\pgfpathlineto{\pgfqpoint{2.469763in}{2.285339in}}%
\pgfpathlineto{\pgfqpoint{2.459186in}{2.285455in}}%
\pgfpathlineto{\pgfqpoint{2.450399in}{2.283121in}}%
\pgfpathlineto{\pgfqpoint{2.448609in}{2.282589in}}%
\pgfpathlineto{\pgfqpoint{2.438032in}{2.276117in}}%
\pgfpathlineto{\pgfqpoint{2.434516in}{2.272826in}}%
\pgfpathlineto{\pgfqpoint{2.427455in}{2.265336in}}%
\pgfpathlineto{\pgfqpoint{2.425434in}{2.262532in}}%
\pgfpathlineto{\pgfqpoint{2.418780in}{2.252238in}}%
\pgfpathlineto{\pgfqpoint{2.416878in}{2.249013in}}%
\pgfpathlineto{\pgfqpoint{2.413162in}{2.241943in}}%
\pgfpathlineto{\pgfqpoint{2.407320in}{2.231649in}}%
\pgfpathlineto{\pgfqpoint{2.406301in}{2.229899in}}%
\pgfpathlineto{\pgfqpoint{2.400731in}{2.221354in}}%
\pgfpathlineto{\pgfqpoint{2.395937in}{2.211060in}}%
\pgfpathlineto{\pgfqpoint{2.395724in}{2.210408in}}%
\pgfpathlineto{\pgfqpoint{2.393086in}{2.200765in}}%
\pgfpathlineto{\pgfqpoint{2.391380in}{2.190471in}}%
\pgfpathlineto{\pgfqpoint{2.390450in}{2.180177in}}%
\pgfpathlineto{\pgfqpoint{2.390067in}{2.169882in}}%
\pgfpathlineto{\pgfqpoint{2.390073in}{2.159588in}}%
\pgfpathlineto{\pgfqpoint{2.390353in}{2.149293in}}%
\pgfpathlineto{\pgfqpoint{2.390815in}{2.138999in}}%
\pgfpathlineto{\pgfqpoint{2.391395in}{2.128704in}}%
\pgfpathlineto{\pgfqpoint{2.392072in}{2.118410in}}%
\pgfpathlineto{\pgfqpoint{2.392880in}{2.108116in}}%
\pgfpathlineto{\pgfqpoint{2.393880in}{2.097821in}}%
\pgfpathlineto{\pgfqpoint{2.395118in}{2.087527in}}%
\pgfpathlineto{\pgfqpoint{2.395724in}{2.083434in}}%
\pgfpathlineto{\pgfqpoint{2.396653in}{2.077232in}}%
\pgfpathlineto{\pgfqpoint{2.398492in}{2.066938in}}%
\pgfpathlineto{\pgfqpoint{2.400604in}{2.056643in}}%
\pgfpathlineto{\pgfqpoint{2.402975in}{2.046349in}}%
\pgfpathlineto{\pgfqpoint{2.405582in}{2.036054in}}%
\pgfpathlineto{\pgfqpoint{2.406301in}{2.033492in}}%
\pgfpathlineto{\pgfqpoint{2.408526in}{2.025760in}}%
\pgfpathlineto{\pgfqpoint{2.411676in}{2.015466in}}%
\pgfpathlineto{\pgfqpoint{2.414586in}{2.005171in}}%
\pgfpathlineto{\pgfqpoint{2.416878in}{1.995681in}}%
\pgfpathlineto{\pgfqpoint{2.417083in}{1.994877in}}%
\pgfpathlineto{\pgfqpoint{2.419756in}{1.984582in}}%
\pgfpathlineto{\pgfqpoint{2.422475in}{1.974288in}}%
\pgfpathlineto{\pgfqpoint{2.425264in}{1.963993in}}%
\pgfpathlineto{\pgfqpoint{2.427455in}{1.956192in}}%
\pgfpathlineto{\pgfqpoint{2.428219in}{1.953699in}}%
\pgfpathlineto{\pgfqpoint{2.431534in}{1.943405in}}%
\pgfpathlineto{\pgfqpoint{2.435032in}{1.933110in}}%
\pgfpathlineto{\pgfqpoint{2.438032in}{1.924843in}}%
\pgfpathlineto{\pgfqpoint{2.438860in}{1.922816in}}%
\pgfpathlineto{\pgfqpoint{2.443390in}{1.912521in}}%
\pgfpathlineto{\pgfqpoint{2.448326in}{1.902227in}}%
\pgfpathclose%
\pgfusepath{fill}%
\end{pgfscope}%
\begin{pgfscope}%
\pgfpathrectangle{\pgfqpoint{1.856795in}{1.819814in}}{\pgfqpoint{1.194205in}{1.163386in}}%
\pgfusepath{clip}%
\pgfsetbuttcap%
\pgfsetroundjoin%
\definecolor{currentfill}{rgb}{0.977657,0.891500,0.822809}%
\pgfsetfillcolor{currentfill}%
\pgfsetlinewidth{0.000000pt}%
\definecolor{currentstroke}{rgb}{0.000000,0.000000,0.000000}%
\pgfsetstrokecolor{currentstroke}%
\pgfsetdash{}{0pt}%
\pgfpathmoveto{\pgfqpoint{2.850538in}{1.891932in}}%
\pgfpathlineto{\pgfqpoint{2.861115in}{1.891932in}}%
\pgfpathlineto{\pgfqpoint{2.871692in}{1.891932in}}%
\pgfpathlineto{\pgfqpoint{2.882269in}{1.891932in}}%
\pgfpathlineto{\pgfqpoint{2.892846in}{1.891932in}}%
\pgfpathlineto{\pgfqpoint{2.903423in}{1.891932in}}%
\pgfpathlineto{\pgfqpoint{2.914000in}{1.891932in}}%
\pgfpathlineto{\pgfqpoint{2.924577in}{1.891932in}}%
\pgfpathlineto{\pgfqpoint{2.935154in}{1.891932in}}%
\pgfpathlineto{\pgfqpoint{2.945731in}{1.891932in}}%
\pgfpathlineto{\pgfqpoint{2.956308in}{1.891932in}}%
\pgfpathlineto{\pgfqpoint{2.966885in}{1.891932in}}%
\pgfpathlineto{\pgfqpoint{2.968727in}{1.891932in}}%
\pgfpathlineto{\pgfqpoint{2.974567in}{1.902227in}}%
\pgfpathlineto{\pgfqpoint{2.977462in}{1.906735in}}%
\pgfpathlineto{\pgfqpoint{2.977462in}{1.912521in}}%
\pgfpathlineto{\pgfqpoint{2.977462in}{1.922816in}}%
\pgfpathlineto{\pgfqpoint{2.977462in}{1.933110in}}%
\pgfpathlineto{\pgfqpoint{2.977462in}{1.943405in}}%
\pgfpathlineto{\pgfqpoint{2.977462in}{1.953699in}}%
\pgfpathlineto{\pgfqpoint{2.977462in}{1.963993in}}%
\pgfpathlineto{\pgfqpoint{2.977462in}{1.974288in}}%
\pgfpathlineto{\pgfqpoint{2.977462in}{1.984582in}}%
\pgfpathlineto{\pgfqpoint{2.977462in}{1.994877in}}%
\pgfpathlineto{\pgfqpoint{2.977462in}{2.005171in}}%
\pgfpathlineto{\pgfqpoint{2.977462in}{2.015466in}}%
\pgfpathlineto{\pgfqpoint{2.977462in}{2.025760in}}%
\pgfpathlineto{\pgfqpoint{2.977462in}{2.036054in}}%
\pgfpathlineto{\pgfqpoint{2.977462in}{2.046349in}}%
\pgfpathlineto{\pgfqpoint{2.977462in}{2.056643in}}%
\pgfpathlineto{\pgfqpoint{2.977462in}{2.066938in}}%
\pgfpathlineto{\pgfqpoint{2.977462in}{2.077232in}}%
\pgfpathlineto{\pgfqpoint{2.977462in}{2.087527in}}%
\pgfpathlineto{\pgfqpoint{2.977462in}{2.097821in}}%
\pgfpathlineto{\pgfqpoint{2.977462in}{2.108116in}}%
\pgfpathlineto{\pgfqpoint{2.977462in}{2.118410in}}%
\pgfpathlineto{\pgfqpoint{2.977462in}{2.128704in}}%
\pgfpathlineto{\pgfqpoint{2.977462in}{2.138999in}}%
\pgfpathlineto{\pgfqpoint{2.977462in}{2.149293in}}%
\pgfpathlineto{\pgfqpoint{2.977462in}{2.159588in}}%
\pgfpathlineto{\pgfqpoint{2.977462in}{2.169882in}}%
\pgfpathlineto{\pgfqpoint{2.977462in}{2.180177in}}%
\pgfpathlineto{\pgfqpoint{2.977462in}{2.190471in}}%
\pgfpathlineto{\pgfqpoint{2.977462in}{2.200765in}}%
\pgfpathlineto{\pgfqpoint{2.977462in}{2.211060in}}%
\pgfpathlineto{\pgfqpoint{2.977462in}{2.221354in}}%
\pgfpathlineto{\pgfqpoint{2.977462in}{2.231649in}}%
\pgfpathlineto{\pgfqpoint{2.977462in}{2.241943in}}%
\pgfpathlineto{\pgfqpoint{2.977462in}{2.252238in}}%
\pgfpathlineto{\pgfqpoint{2.977462in}{2.262532in}}%
\pgfpathlineto{\pgfqpoint{2.977462in}{2.272826in}}%
\pgfpathlineto{\pgfqpoint{2.977462in}{2.283121in}}%
\pgfpathlineto{\pgfqpoint{2.977462in}{2.284435in}}%
\pgfpathlineto{\pgfqpoint{2.966885in}{2.288373in}}%
\pgfpathlineto{\pgfqpoint{2.956308in}{2.289997in}}%
\pgfpathlineto{\pgfqpoint{2.945731in}{2.288073in}}%
\pgfpathlineto{\pgfqpoint{2.936143in}{2.283121in}}%
\pgfpathlineto{\pgfqpoint{2.935154in}{2.282581in}}%
\pgfpathlineto{\pgfqpoint{2.924577in}{2.273265in}}%
\pgfpathlineto{\pgfqpoint{2.924210in}{2.272826in}}%
\pgfpathlineto{\pgfqpoint{2.915975in}{2.262532in}}%
\pgfpathlineto{\pgfqpoint{2.914000in}{2.259952in}}%
\pgfpathlineto{\pgfqpoint{2.909201in}{2.252238in}}%
\pgfpathlineto{\pgfqpoint{2.903423in}{2.242610in}}%
\pgfpathlineto{\pgfqpoint{2.903078in}{2.241943in}}%
\pgfpathlineto{\pgfqpoint{2.897912in}{2.231649in}}%
\pgfpathlineto{\pgfqpoint{2.892853in}{2.221354in}}%
\pgfpathlineto{\pgfqpoint{2.892846in}{2.221340in}}%
\pgfpathlineto{\pgfqpoint{2.888397in}{2.211060in}}%
\pgfpathlineto{\pgfqpoint{2.883985in}{2.200765in}}%
\pgfpathlineto{\pgfqpoint{2.882269in}{2.196720in}}%
\pgfpathlineto{\pgfqpoint{2.879794in}{2.190471in}}%
\pgfpathlineto{\pgfqpoint{2.875705in}{2.180177in}}%
\pgfpathlineto{\pgfqpoint{2.871692in}{2.170605in}}%
\pgfpathlineto{\pgfqpoint{2.871369in}{2.169882in}}%
\pgfpathlineto{\pgfqpoint{2.865713in}{2.159588in}}%
\pgfpathlineto{\pgfqpoint{2.861115in}{2.151549in}}%
\pgfpathlineto{\pgfqpoint{2.859875in}{2.149293in}}%
\pgfpathlineto{\pgfqpoint{2.854805in}{2.138999in}}%
\pgfpathlineto{\pgfqpoint{2.850538in}{2.129160in}}%
\pgfpathlineto{\pgfqpoint{2.850346in}{2.128704in}}%
\pgfpathlineto{\pgfqpoint{2.846610in}{2.118410in}}%
\pgfpathlineto{\pgfqpoint{2.843410in}{2.108116in}}%
\pgfpathlineto{\pgfqpoint{2.840692in}{2.097821in}}%
\pgfpathlineto{\pgfqpoint{2.839960in}{2.094539in}}%
\pgfpathlineto{\pgfqpoint{2.838428in}{2.087527in}}%
\pgfpathlineto{\pgfqpoint{2.836540in}{2.077232in}}%
\pgfpathlineto{\pgfqpoint{2.834956in}{2.066938in}}%
\pgfpathlineto{\pgfqpoint{2.833631in}{2.056643in}}%
\pgfpathlineto{\pgfqpoint{2.832532in}{2.046349in}}%
\pgfpathlineto{\pgfqpoint{2.831646in}{2.036054in}}%
\pgfpathlineto{\pgfqpoint{2.830981in}{2.025760in}}%
\pgfpathlineto{\pgfqpoint{2.830566in}{2.015466in}}%
\pgfpathlineto{\pgfqpoint{2.830433in}{2.005171in}}%
\pgfpathlineto{\pgfqpoint{2.830604in}{1.994877in}}%
\pgfpathlineto{\pgfqpoint{2.831028in}{1.984582in}}%
\pgfpathlineto{\pgfqpoint{2.831417in}{1.974288in}}%
\pgfpathlineto{\pgfqpoint{2.831764in}{1.963993in}}%
\pgfpathlineto{\pgfqpoint{2.832316in}{1.953699in}}%
\pgfpathlineto{\pgfqpoint{2.833164in}{1.943405in}}%
\pgfpathlineto{\pgfqpoint{2.834364in}{1.933110in}}%
\pgfpathlineto{\pgfqpoint{2.835960in}{1.922816in}}%
\pgfpathlineto{\pgfqpoint{2.837976in}{1.912521in}}%
\pgfpathlineto{\pgfqpoint{2.839960in}{1.903977in}}%
\pgfpathlineto{\pgfqpoint{2.840418in}{1.902227in}}%
\pgfpathlineto{\pgfqpoint{2.843341in}{1.891932in}}%
\pgfpathclose%
\pgfusepath{fill}%
\end{pgfscope}%
\begin{pgfscope}%
\pgfpathrectangle{\pgfqpoint{1.856795in}{1.819814in}}{\pgfqpoint{1.194205in}{1.163386in}}%
\pgfusepath{clip}%
\pgfsetbuttcap%
\pgfsetroundjoin%
\definecolor{currentfill}{rgb}{0.977657,0.891500,0.822809}%
\pgfsetfillcolor{currentfill}%
\pgfsetlinewidth{0.000000pt}%
\definecolor{currentstroke}{rgb}{0.000000,0.000000,0.000000}%
\pgfsetstrokecolor{currentstroke}%
\pgfsetdash{}{0pt}%
\pgfpathmoveto{\pgfqpoint{2.120720in}{2.423742in}}%
\pgfpathlineto{\pgfqpoint{2.131297in}{2.419938in}}%
\pgfpathlineto{\pgfqpoint{2.141874in}{2.421880in}}%
\pgfpathlineto{\pgfqpoint{2.149865in}{2.427243in}}%
\pgfpathlineto{\pgfqpoint{2.152451in}{2.428531in}}%
\pgfpathlineto{\pgfqpoint{2.158909in}{2.427243in}}%
\pgfpathlineto{\pgfqpoint{2.163029in}{2.426343in}}%
\pgfpathlineto{\pgfqpoint{2.169147in}{2.427243in}}%
\pgfpathlineto{\pgfqpoint{2.173606in}{2.428368in}}%
\pgfpathlineto{\pgfqpoint{2.180920in}{2.437537in}}%
\pgfpathlineto{\pgfqpoint{2.183514in}{2.447832in}}%
\pgfpathlineto{\pgfqpoint{2.183682in}{2.458126in}}%
\pgfpathlineto{\pgfqpoint{2.182683in}{2.468421in}}%
\pgfpathlineto{\pgfqpoint{2.181068in}{2.478715in}}%
\pgfpathlineto{\pgfqpoint{2.179120in}{2.489009in}}%
\pgfpathlineto{\pgfqpoint{2.176999in}{2.499304in}}%
\pgfpathlineto{\pgfqpoint{2.174807in}{2.509598in}}%
\pgfpathlineto{\pgfqpoint{2.173606in}{2.515020in}}%
\pgfpathlineto{\pgfqpoint{2.172484in}{2.519893in}}%
\pgfpathlineto{\pgfqpoint{2.170098in}{2.530187in}}%
\pgfpathlineto{\pgfqpoint{2.167865in}{2.540482in}}%
\pgfpathlineto{\pgfqpoint{2.165832in}{2.550776in}}%
\pgfpathlineto{\pgfqpoint{2.164047in}{2.561071in}}%
\pgfpathlineto{\pgfqpoint{2.163029in}{2.567747in}}%
\pgfpathlineto{\pgfqpoint{2.162439in}{2.571365in}}%
\pgfpathlineto{\pgfqpoint{2.160907in}{2.581659in}}%
\pgfpathlineto{\pgfqpoint{2.159466in}{2.591954in}}%
\pgfpathlineto{\pgfqpoint{2.158041in}{2.602248in}}%
\pgfpathlineto{\pgfqpoint{2.156585in}{2.612543in}}%
\pgfpathlineto{\pgfqpoint{2.155069in}{2.622837in}}%
\pgfpathlineto{\pgfqpoint{2.153469in}{2.633132in}}%
\pgfpathlineto{\pgfqpoint{2.152451in}{2.639209in}}%
\pgfpathlineto{\pgfqpoint{2.151700in}{2.643426in}}%
\pgfpathlineto{\pgfqpoint{2.149716in}{2.653720in}}%
\pgfpathlineto{\pgfqpoint{2.147603in}{2.664015in}}%
\pgfpathlineto{\pgfqpoint{2.145348in}{2.674309in}}%
\pgfpathlineto{\pgfqpoint{2.142945in}{2.684604in}}%
\pgfpathlineto{\pgfqpoint{2.141874in}{2.688838in}}%
\pgfpathlineto{\pgfqpoint{2.140255in}{2.694898in}}%
\pgfpathlineto{\pgfqpoint{2.137328in}{2.705193in}}%
\pgfpathlineto{\pgfqpoint{2.134281in}{2.715487in}}%
\pgfpathlineto{\pgfqpoint{2.131297in}{2.725216in}}%
\pgfpathlineto{\pgfqpoint{2.131114in}{2.725781in}}%
\pgfpathlineto{\pgfqpoint{2.127566in}{2.736076in}}%
\pgfpathlineto{\pgfqpoint{2.123990in}{2.746370in}}%
\pgfpathlineto{\pgfqpoint{2.120720in}{2.755712in}}%
\pgfpathlineto{\pgfqpoint{2.120369in}{2.756665in}}%
\pgfpathlineto{\pgfqpoint{2.116911in}{2.766959in}}%
\pgfpathlineto{\pgfqpoint{2.114648in}{2.777254in}}%
\pgfpathlineto{\pgfqpoint{2.112439in}{2.787548in}}%
\pgfpathlineto{\pgfqpoint{2.110239in}{2.797842in}}%
\pgfpathlineto{\pgfqpoint{2.110143in}{2.798283in}}%
\pgfpathlineto{\pgfqpoint{2.107748in}{2.808137in}}%
\pgfpathlineto{\pgfqpoint{2.105213in}{2.818431in}}%
\pgfpathlineto{\pgfqpoint{2.102607in}{2.828726in}}%
\pgfpathlineto{\pgfqpoint{2.099878in}{2.839020in}}%
\pgfpathlineto{\pgfqpoint{2.099566in}{2.840129in}}%
\pgfpathlineto{\pgfqpoint{2.096604in}{2.849315in}}%
\pgfpathlineto{\pgfqpoint{2.093036in}{2.859609in}}%
\pgfpathlineto{\pgfqpoint{2.089114in}{2.869903in}}%
\pgfpathlineto{\pgfqpoint{2.088989in}{2.870201in}}%
\pgfpathlineto{\pgfqpoint{2.084107in}{2.880198in}}%
\pgfpathlineto{\pgfqpoint{2.078423in}{2.890492in}}%
\pgfpathlineto{\pgfqpoint{2.078412in}{2.890510in}}%
\pgfpathlineto{\pgfqpoint{2.073350in}{2.900787in}}%
\pgfpathlineto{\pgfqpoint{2.070011in}{2.911081in}}%
\pgfpathlineto{\pgfqpoint{2.067835in}{2.911081in}}%
\pgfpathlineto{\pgfqpoint{2.057258in}{2.911081in}}%
\pgfpathlineto{\pgfqpoint{2.046681in}{2.911081in}}%
\pgfpathlineto{\pgfqpoint{2.036104in}{2.911081in}}%
\pgfpathlineto{\pgfqpoint{2.025527in}{2.911081in}}%
\pgfpathlineto{\pgfqpoint{2.014950in}{2.911081in}}%
\pgfpathlineto{\pgfqpoint{2.004373in}{2.911081in}}%
\pgfpathlineto{\pgfqpoint{1.993796in}{2.911081in}}%
\pgfpathlineto{\pgfqpoint{1.983218in}{2.911081in}}%
\pgfpathlineto{\pgfqpoint{1.972641in}{2.911081in}}%
\pgfpathlineto{\pgfqpoint{1.962064in}{2.911081in}}%
\pgfpathlineto{\pgfqpoint{1.953675in}{2.911081in}}%
\pgfpathlineto{\pgfqpoint{1.956204in}{2.900787in}}%
\pgfpathlineto{\pgfqpoint{1.958491in}{2.890492in}}%
\pgfpathlineto{\pgfqpoint{1.960612in}{2.880198in}}%
\pgfpathlineto{\pgfqpoint{1.962064in}{2.873029in}}%
\pgfpathlineto{\pgfqpoint{1.962781in}{2.869903in}}%
\pgfpathlineto{\pgfqpoint{1.965070in}{2.859609in}}%
\pgfpathlineto{\pgfqpoint{1.967324in}{2.849315in}}%
\pgfpathlineto{\pgfqpoint{1.969636in}{2.839020in}}%
\pgfpathlineto{\pgfqpoint{1.972075in}{2.828726in}}%
\pgfpathlineto{\pgfqpoint{1.972641in}{2.826462in}}%
\pgfpathlineto{\pgfqpoint{1.974719in}{2.818431in}}%
\pgfpathlineto{\pgfqpoint{1.977308in}{2.808137in}}%
\pgfpathlineto{\pgfqpoint{1.979899in}{2.797842in}}%
\pgfpathlineto{\pgfqpoint{1.982094in}{2.787548in}}%
\pgfpathlineto{\pgfqpoint{1.983218in}{2.781819in}}%
\pgfpathlineto{\pgfqpoint{1.984176in}{2.777254in}}%
\pgfpathlineto{\pgfqpoint{1.986425in}{2.766959in}}%
\pgfpathlineto{\pgfqpoint{1.988678in}{2.756665in}}%
\pgfpathlineto{\pgfqpoint{1.990927in}{2.746370in}}%
\pgfpathlineto{\pgfqpoint{1.993169in}{2.736076in}}%
\pgfpathlineto{\pgfqpoint{1.993796in}{2.733267in}}%
\pgfpathlineto{\pgfqpoint{1.995660in}{2.725781in}}%
\pgfpathlineto{\pgfqpoint{1.998220in}{2.715487in}}%
\pgfpathlineto{\pgfqpoint{2.000742in}{2.705193in}}%
\pgfpathlineto{\pgfqpoint{2.003220in}{2.694898in}}%
\pgfpathlineto{\pgfqpoint{2.004373in}{2.690102in}}%
\pgfpathlineto{\pgfqpoint{2.005865in}{2.684604in}}%
\pgfpathlineto{\pgfqpoint{2.008615in}{2.674309in}}%
\pgfpathlineto{\pgfqpoint{2.011251in}{2.664015in}}%
\pgfpathlineto{\pgfqpoint{2.013775in}{2.653720in}}%
\pgfpathlineto{\pgfqpoint{2.014950in}{2.648833in}}%
\pgfpathlineto{\pgfqpoint{2.016438in}{2.643426in}}%
\pgfpathlineto{\pgfqpoint{2.019228in}{2.633132in}}%
\pgfpathlineto{\pgfqpoint{2.021965in}{2.622837in}}%
\pgfpathlineto{\pgfqpoint{2.024690in}{2.612543in}}%
\pgfpathlineto{\pgfqpoint{2.025527in}{2.609474in}}%
\pgfpathlineto{\pgfqpoint{2.027822in}{2.602248in}}%
\pgfpathlineto{\pgfqpoint{2.031144in}{2.591954in}}%
\pgfpathlineto{\pgfqpoint{2.034503in}{2.581659in}}%
\pgfpathlineto{\pgfqpoint{2.036104in}{2.576894in}}%
\pgfpathlineto{\pgfqpoint{2.038285in}{2.571365in}}%
\pgfpathlineto{\pgfqpoint{2.042393in}{2.561071in}}%
\pgfpathlineto{\pgfqpoint{2.046428in}{2.550776in}}%
\pgfpathlineto{\pgfqpoint{2.046681in}{2.550126in}}%
\pgfpathlineto{\pgfqpoint{2.051066in}{2.540482in}}%
\pgfpathlineto{\pgfqpoint{2.055327in}{2.530187in}}%
\pgfpathlineto{\pgfqpoint{2.057258in}{2.525246in}}%
\pgfpathlineto{\pgfqpoint{2.059679in}{2.519893in}}%
\pgfpathlineto{\pgfqpoint{2.064130in}{2.509598in}}%
\pgfpathlineto{\pgfqpoint{2.067835in}{2.500590in}}%
\pgfpathlineto{\pgfqpoint{2.068472in}{2.499304in}}%
\pgfpathlineto{\pgfqpoint{2.073526in}{2.489009in}}%
\pgfpathlineto{\pgfqpoint{2.078302in}{2.478715in}}%
\pgfpathlineto{\pgfqpoint{2.078412in}{2.478477in}}%
\pgfpathlineto{\pgfqpoint{2.084214in}{2.468421in}}%
\pgfpathlineto{\pgfqpoint{2.088989in}{2.459613in}}%
\pgfpathlineto{\pgfqpoint{2.090048in}{2.458126in}}%
\pgfpathlineto{\pgfqpoint{2.097125in}{2.447832in}}%
\pgfpathlineto{\pgfqpoint{2.099566in}{2.444082in}}%
\pgfpathlineto{\pgfqpoint{2.105488in}{2.437537in}}%
\pgfpathlineto{\pgfqpoint{2.110143in}{2.432012in}}%
\pgfpathlineto{\pgfqpoint{2.116449in}{2.427243in}}%
\pgfpathclose%
\pgfusepath{fill}%
\end{pgfscope}%
\begin{pgfscope}%
\pgfpathrectangle{\pgfqpoint{1.856795in}{1.819814in}}{\pgfqpoint{1.194205in}{1.163386in}}%
\pgfusepath{clip}%
\pgfsetbuttcap%
\pgfsetroundjoin%
\definecolor{currentfill}{rgb}{0.121569,0.466667,0.705882}%
\pgfsetfillcolor{currentfill}%
\pgfsetlinewidth{1.003750pt}%
\definecolor{currentstroke}{rgb}{0.121569,0.466667,0.705882}%
\pgfsetstrokecolor{currentstroke}%
\pgfsetdash{}{0pt}%
\pgfsys@defobject{currentmarker}{\pgfqpoint{-0.021960in}{-0.021960in}}{\pgfqpoint{0.021960in}{0.021960in}}{%
\pgfpathmoveto{\pgfqpoint{0.000000in}{-0.021960in}}%
\pgfpathcurveto{\pgfqpoint{0.005824in}{-0.021960in}}{\pgfqpoint{0.011410in}{-0.019646in}}{\pgfqpoint{0.015528in}{-0.015528in}}%
\pgfpathcurveto{\pgfqpoint{0.019646in}{-0.011410in}}{\pgfqpoint{0.021960in}{-0.005824in}}{\pgfqpoint{0.021960in}{0.000000in}}%
\pgfpathcurveto{\pgfqpoint{0.021960in}{0.005824in}}{\pgfqpoint{0.019646in}{0.011410in}}{\pgfqpoint{0.015528in}{0.015528in}}%
\pgfpathcurveto{\pgfqpoint{0.011410in}{0.019646in}}{\pgfqpoint{0.005824in}{0.021960in}}{\pgfqpoint{0.000000in}{0.021960in}}%
\pgfpathcurveto{\pgfqpoint{-0.005824in}{0.021960in}}{\pgfqpoint{-0.011410in}{0.019646in}}{\pgfqpoint{-0.015528in}{0.015528in}}%
\pgfpathcurveto{\pgfqpoint{-0.019646in}{0.011410in}}{\pgfqpoint{-0.021960in}{0.005824in}}{\pgfqpoint{-0.021960in}{0.000000in}}%
\pgfpathcurveto{\pgfqpoint{-0.021960in}{-0.005824in}}{\pgfqpoint{-0.019646in}{-0.011410in}}{\pgfqpoint{-0.015528in}{-0.015528in}}%
\pgfpathcurveto{\pgfqpoint{-0.011410in}{-0.019646in}}{\pgfqpoint{-0.005824in}{-0.021960in}}{\pgfqpoint{0.000000in}{-0.021960in}}%
\pgfpathclose%
\pgfusepath{stroke,fill}%
}%
\begin{pgfscope}%
\pgfsys@transformshift{2.795891in}{2.594949in}%
\pgfsys@useobject{currentmarker}{}%
\end{pgfscope}%
\begin{pgfscope}%
\pgfsys@transformshift{2.688224in}{1.965815in}%
\pgfsys@useobject{currentmarker}{}%
\end{pgfscope}%
\begin{pgfscope}%
\pgfsys@transformshift{2.521252in}{2.015193in}%
\pgfsys@useobject{currentmarker}{}%
\end{pgfscope}%
\begin{pgfscope}%
\pgfsys@transformshift{2.884259in}{2.155797in}%
\pgfsys@useobject{currentmarker}{}%
\end{pgfscope}%
\begin{pgfscope}%
\pgfsys@transformshift{2.062646in}{2.084147in}%
\pgfsys@useobject{currentmarker}{}%
\end{pgfscope}%
\begin{pgfscope}%
\pgfsys@transformshift{2.235176in}{2.552559in}%
\pgfsys@useobject{currentmarker}{}%
\end{pgfscope}%
\begin{pgfscope}%
\pgfsys@transformshift{2.598124in}{2.180910in}%
\pgfsys@useobject{currentmarker}{}%
\end{pgfscope}%
\begin{pgfscope}%
\pgfsys@transformshift{2.284982in}{1.992641in}%
\pgfsys@useobject{currentmarker}{}%
\end{pgfscope}%
\begin{pgfscope}%
\pgfsys@transformshift{2.677683in}{2.894100in}%
\pgfsys@useobject{currentmarker}{}%
\end{pgfscope}%
\begin{pgfscope}%
\pgfsys@transformshift{1.982906in}{2.795447in}%
\pgfsys@useobject{currentmarker}{}%
\end{pgfscope}%
\begin{pgfscope}%
\pgfsys@transformshift{2.050202in}{2.003040in}%
\pgfsys@useobject{currentmarker}{}%
\end{pgfscope}%
\begin{pgfscope}%
\pgfsys@transformshift{2.153502in}{2.421561in}%
\pgfsys@useobject{currentmarker}{}%
\end{pgfscope}%
\begin{pgfscope}%
\pgfsys@transformshift{2.554402in}{1.976800in}%
\pgfsys@useobject{currentmarker}{}%
\end{pgfscope}%
\begin{pgfscope}%
\pgfsys@transformshift{2.073216in}{2.548584in}%
\pgfsys@useobject{currentmarker}{}%
\end{pgfscope}%
\begin{pgfscope}%
\pgfsys@transformshift{2.527843in}{2.765856in}%
\pgfsys@useobject{currentmarker}{}%
\end{pgfscope}%
\begin{pgfscope}%
\pgfsys@transformshift{2.794435in}{1.978013in}%
\pgfsys@useobject{currentmarker}{}%
\end{pgfscope}%
\begin{pgfscope}%
\pgfsys@transformshift{2.624996in}{2.826418in}%
\pgfsys@useobject{currentmarker}{}%
\end{pgfscope}%
\begin{pgfscope}%
\pgfsys@transformshift{2.712742in}{1.970090in}%
\pgfsys@useobject{currentmarker}{}%
\end{pgfscope}%
\begin{pgfscope}%
\pgfsys@transformshift{2.430687in}{2.014256in}%
\pgfsys@useobject{currentmarker}{}%
\end{pgfscope}%
\begin{pgfscope}%
\pgfsys@transformshift{2.717601in}{2.570458in}%
\pgfsys@useobject{currentmarker}{}%
\end{pgfscope}%
\begin{pgfscope}%
\pgfsys@transformshift{1.930333in}{2.616055in}%
\pgfsys@useobject{currentmarker}{}%
\end{pgfscope}%
\begin{pgfscope}%
\pgfsys@transformshift{2.077288in}{2.911081in}%
\pgfsys@useobject{currentmarker}{}%
\end{pgfscope}%
\begin{pgfscope}%
\pgfsys@transformshift{2.977462in}{1.891932in}%
\pgfsys@useobject{currentmarker}{}%
\end{pgfscope}%
\begin{pgfscope}%
\pgfsys@transformshift{2.977462in}{2.205333in}%
\pgfsys@useobject{currentmarker}{}%
\end{pgfscope}%
\begin{pgfscope}%
\pgfsys@transformshift{2.392457in}{2.262726in}%
\pgfsys@useobject{currentmarker}{}%
\end{pgfscope}%
\begin{pgfscope}%
\pgfsys@transformshift{2.977462in}{2.141266in}%
\pgfsys@useobject{currentmarker}{}%
\end{pgfscope}%
\begin{pgfscope}%
\pgfsys@transformshift{1.930333in}{2.911081in}%
\pgfsys@useobject{currentmarker}{}%
\end{pgfscope}%
\begin{pgfscope}%
\pgfsys@transformshift{2.977462in}{2.137527in}%
\pgfsys@useobject{currentmarker}{}%
\end{pgfscope}%
\begin{pgfscope}%
\pgfsys@transformshift{2.977462in}{2.227623in}%
\pgfsys@useobject{currentmarker}{}%
\end{pgfscope}%
\begin{pgfscope}%
\pgfsys@transformshift{2.977462in}{2.084657in}%
\pgfsys@useobject{currentmarker}{}%
\end{pgfscope}%
\begin{pgfscope}%
\pgfsys@transformshift{2.977462in}{2.091865in}%
\pgfsys@useobject{currentmarker}{}%
\end{pgfscope}%
\begin{pgfscope}%
\pgfsys@transformshift{2.977462in}{2.094395in}%
\pgfsys@useobject{currentmarker}{}%
\end{pgfscope}%
\begin{pgfscope}%
\pgfsys@transformshift{2.977462in}{2.094437in}%
\pgfsys@useobject{currentmarker}{}%
\end{pgfscope}%
\begin{pgfscope}%
\pgfsys@transformshift{2.465306in}{2.099244in}%
\pgfsys@useobject{currentmarker}{}%
\end{pgfscope}%
\begin{pgfscope}%
\pgfsys@transformshift{2.117997in}{2.764263in}%
\pgfsys@useobject{currentmarker}{}%
\end{pgfscope}%
\begin{pgfscope}%
\pgfsys@transformshift{2.468882in}{2.091641in}%
\pgfsys@useobject{currentmarker}{}%
\end{pgfscope}%
\begin{pgfscope}%
\pgfsys@transformshift{2.470005in}{2.087495in}%
\pgfsys@useobject{currentmarker}{}%
\end{pgfscope}%
\begin{pgfscope}%
\pgfsys@transformshift{2.470203in}{2.084763in}%
\pgfsys@useobject{currentmarker}{}%
\end{pgfscope}%
\begin{pgfscope}%
\pgfsys@transformshift{2.469889in}{2.082897in}%
\pgfsys@useobject{currentmarker}{}%
\end{pgfscope}%
\begin{pgfscope}%
\pgfsys@transformshift{2.469250in}{2.081582in}%
\pgfsys@useobject{currentmarker}{}%
\end{pgfscope}%
\begin{pgfscope}%
\pgfsys@transformshift{2.468348in}{2.080621in}%
\pgfsys@useobject{currentmarker}{}%
\end{pgfscope}%
\begin{pgfscope}%
\pgfsys@transformshift{2.467102in}{2.079872in}%
\pgfsys@useobject{currentmarker}{}%
\end{pgfscope}%
\begin{pgfscope}%
\pgfsys@transformshift{2.477105in}{1.891932in}%
\pgfsys@useobject{currentmarker}{}%
\end{pgfscope}%
\begin{pgfscope}%
\pgfsys@transformshift{2.977462in}{2.415564in}%
\pgfsys@useobject{currentmarker}{}%
\end{pgfscope}%
\begin{pgfscope}%
\pgfsys@transformshift{2.466305in}{2.073827in}%
\pgfsys@useobject{currentmarker}{}%
\end{pgfscope}%
\begin{pgfscope}%
\pgfsys@transformshift{2.083965in}{2.659611in}%
\pgfsys@useobject{currentmarker}{}%
\end{pgfscope}%
\begin{pgfscope}%
\pgfsys@transformshift{2.289209in}{2.210970in}%
\pgfsys@useobject{currentmarker}{}%
\end{pgfscope}%
\begin{pgfscope}%
\pgfsys@transformshift{2.929358in}{2.053942in}%
\pgfsys@useobject{currentmarker}{}%
\end{pgfscope}%
\begin{pgfscope}%
\pgfsys@transformshift{1.930333in}{2.303933in}%
\pgfsys@useobject{currentmarker}{}%
\end{pgfscope}%
\begin{pgfscope}%
\pgfsys@transformshift{2.117276in}{2.570841in}%
\pgfsys@useobject{currentmarker}{}%
\end{pgfscope}%
\begin{pgfscope}%
\pgfsys@transformshift{2.527234in}{2.060122in}%
\pgfsys@useobject{currentmarker}{}%
\end{pgfscope}%
\begin{pgfscope}%
\pgfsys@transformshift{2.929836in}{2.065195in}%
\pgfsys@useobject{currentmarker}{}%
\end{pgfscope}%
\begin{pgfscope}%
\pgfsys@transformshift{2.944510in}{2.025968in}%
\pgfsys@useobject{currentmarker}{}%
\end{pgfscope}%
\begin{pgfscope}%
\pgfsys@transformshift{2.057333in}{2.710800in}%
\pgfsys@useobject{currentmarker}{}%
\end{pgfscope}%
\begin{pgfscope}%
\pgfsys@transformshift{2.047037in}{2.787578in}%
\pgfsys@useobject{currentmarker}{}%
\end{pgfscope}%
\begin{pgfscope}%
\pgfsys@transformshift{2.056128in}{2.724322in}%
\pgfsys@useobject{currentmarker}{}%
\end{pgfscope}%
\begin{pgfscope}%
\pgfsys@transformshift{2.885671in}{1.891932in}%
\pgfsys@useobject{currentmarker}{}%
\end{pgfscope}%
\begin{pgfscope}%
\pgfsys@transformshift{2.936541in}{2.072334in}%
\pgfsys@useobject{currentmarker}{}%
\end{pgfscope}%
\begin{pgfscope}%
\pgfsys@transformshift{2.055893in}{2.732438in}%
\pgfsys@useobject{currentmarker}{}%
\end{pgfscope}%
\begin{pgfscope}%
\pgfsys@transformshift{2.500289in}{2.049090in}%
\pgfsys@useobject{currentmarker}{}%
\end{pgfscope}%
\begin{pgfscope}%
\pgfsys@transformshift{2.056835in}{2.731722in}%
\pgfsys@useobject{currentmarker}{}%
\end{pgfscope}%
\begin{pgfscope}%
\pgfsys@transformshift{2.937419in}{2.062125in}%
\pgfsys@useobject{currentmarker}{}%
\end{pgfscope}%
\begin{pgfscope}%
\pgfsys@transformshift{2.499069in}{2.048381in}%
\pgfsys@useobject{currentmarker}{}%
\end{pgfscope}%
\begin{pgfscope}%
\pgfsys@transformshift{2.937567in}{2.060370in}%
\pgfsys@useobject{currentmarker}{}%
\end{pgfscope}%
\begin{pgfscope}%
\pgfsys@transformshift{2.498778in}{2.048353in}%
\pgfsys@useobject{currentmarker}{}%
\end{pgfscope}%
\begin{pgfscope}%
\pgfsys@transformshift{2.059423in}{2.729310in}%
\pgfsys@useobject{currentmarker}{}%
\end{pgfscope}%
\begin{pgfscope}%
\pgfsys@transformshift{2.937768in}{2.059912in}%
\pgfsys@useobject{currentmarker}{}%
\end{pgfscope}%
\begin{pgfscope}%
\pgfsys@transformshift{2.498691in}{2.048006in}%
\pgfsys@useobject{currentmarker}{}%
\end{pgfscope}%
\begin{pgfscope}%
\pgfsys@transformshift{2.060141in}{2.727728in}%
\pgfsys@useobject{currentmarker}{}%
\end{pgfscope}%
\begin{pgfscope}%
\pgfsys@transformshift{2.937789in}{2.059578in}%
\pgfsys@useobject{currentmarker}{}%
\end{pgfscope}%
\begin{pgfscope}%
\pgfsys@transformshift{2.937707in}{2.059288in}%
\pgfsys@useobject{currentmarker}{}%
\end{pgfscope}%
\begin{pgfscope}%
\pgfsys@transformshift{2.060521in}{2.726394in}%
\pgfsys@useobject{currentmarker}{}%
\end{pgfscope}%
\begin{pgfscope}%
\pgfsys@transformshift{2.937651in}{2.059170in}%
\pgfsys@useobject{currentmarker}{}%
\end{pgfscope}%
\begin{pgfscope}%
\pgfsys@transformshift{2.060567in}{2.725632in}%
\pgfsys@useobject{currentmarker}{}%
\end{pgfscope}%
\begin{pgfscope}%
\pgfsys@transformshift{2.937597in}{2.059193in}%
\pgfsys@useobject{currentmarker}{}%
\end{pgfscope}%
\begin{pgfscope}%
\pgfsys@transformshift{2.060571in}{2.725202in}%
\pgfsys@useobject{currentmarker}{}%
\end{pgfscope}%
\begin{pgfscope}%
\pgfsys@transformshift{2.937553in}{2.059318in}%
\pgfsys@useobject{currentmarker}{}%
\end{pgfscope}%
\begin{pgfscope}%
\pgfsys@transformshift{2.060561in}{2.724977in}%
\pgfsys@useobject{currentmarker}{}%
\end{pgfscope}%
\begin{pgfscope}%
\pgfsys@transformshift{2.937514in}{2.059483in}%
\pgfsys@useobject{currentmarker}{}%
\end{pgfscope}%
\begin{pgfscope}%
\pgfsys@transformshift{2.060525in}{2.724913in}%
\pgfsys@useobject{currentmarker}{}%
\end{pgfscope}%
\begin{pgfscope}%
\pgfsys@transformshift{2.937477in}{2.059658in}%
\pgfsys@useobject{currentmarker}{}%
\end{pgfscope}%
\begin{pgfscope}%
\pgfsys@transformshift{2.060452in}{2.725020in}%
\pgfsys@useobject{currentmarker}{}%
\end{pgfscope}%
\begin{pgfscope}%
\pgfsys@transformshift{2.937454in}{2.059750in}%
\pgfsys@useobject{currentmarker}{}%
\end{pgfscope}%
\begin{pgfscope}%
\pgfsys@transformshift{2.060394in}{2.725112in}%
\pgfsys@useobject{currentmarker}{}%
\end{pgfscope}%
\begin{pgfscope}%
\pgfsys@transformshift{2.498709in}{2.046805in}%
\pgfsys@useobject{currentmarker}{}%
\end{pgfscope}%
\begin{pgfscope}%
\pgfsys@transformshift{2.937417in}{2.059870in}%
\pgfsys@useobject{currentmarker}{}%
\end{pgfscope}%
\begin{pgfscope}%
\pgfsys@transformshift{2.060316in}{2.725301in}%
\pgfsys@useobject{currentmarker}{}%
\end{pgfscope}%
\begin{pgfscope}%
\pgfsys@transformshift{2.498704in}{2.046774in}%
\pgfsys@useobject{currentmarker}{}%
\end{pgfscope}%
\begin{pgfscope}%
\pgfsys@transformshift{2.937407in}{2.059933in}%
\pgfsys@useobject{currentmarker}{}%
\end{pgfscope}%
\begin{pgfscope}%
\pgfsys@transformshift{2.060273in}{2.725408in}%
\pgfsys@useobject{currentmarker}{}%
\end{pgfscope}%
\begin{pgfscope}%
\pgfsys@transformshift{2.937398in}{2.059987in}%
\pgfsys@useobject{currentmarker}{}%
\end{pgfscope}%
\begin{pgfscope}%
\pgfsys@transformshift{2.498698in}{2.046747in}%
\pgfsys@useobject{currentmarker}{}%
\end{pgfscope}%
\begin{pgfscope}%
\pgfsys@transformshift{2.060237in}{2.725465in}%
\pgfsys@useobject{currentmarker}{}%
\end{pgfscope}%
\begin{pgfscope}%
\pgfsys@transformshift{2.937397in}{2.060006in}%
\pgfsys@useobject{currentmarker}{}%
\end{pgfscope}%
\begin{pgfscope}%
\pgfsys@transformshift{2.060208in}{2.725522in}%
\pgfsys@useobject{currentmarker}{}%
\end{pgfscope}%
\begin{pgfscope}%
\pgfsys@transformshift{2.498692in}{2.046674in}%
\pgfsys@useobject{currentmarker}{}%
\end{pgfscope}%
\begin{pgfscope}%
\pgfsys@transformshift{2.937389in}{2.060056in}%
\pgfsys@useobject{currentmarker}{}%
\end{pgfscope}%
\begin{pgfscope}%
\pgfsys@transformshift{2.060193in}{2.725566in}%
\pgfsys@useobject{currentmarker}{}%
\end{pgfscope}%
\begin{pgfscope}%
\pgfsys@transformshift{2.498698in}{2.046702in}%
\pgfsys@useobject{currentmarker}{}%
\end{pgfscope}%
\begin{pgfscope}%
\pgfsys@transformshift{2.937375in}{2.060032in}%
\pgfsys@useobject{currentmarker}{}%
\end{pgfscope}%
\begin{pgfscope}%
\pgfsys@transformshift{2.060184in}{2.725544in}%
\pgfsys@useobject{currentmarker}{}%
\end{pgfscope}%
\begin{pgfscope}%
\pgfsys@transformshift{2.977462in}{2.911081in}%
\pgfsys@useobject{currentmarker}{}%
\end{pgfscope}%
\begin{pgfscope}%
\pgfsys@transformshift{2.937378in}{2.060022in}%
\pgfsys@useobject{currentmarker}{}%
\end{pgfscope}%
\begin{pgfscope}%
\pgfsys@transformshift{2.498694in}{2.046690in}%
\pgfsys@useobject{currentmarker}{}%
\end{pgfscope}%
\begin{pgfscope}%
\pgfsys@transformshift{2.937369in}{2.060046in}%
\pgfsys@useobject{currentmarker}{}%
\end{pgfscope}%
\begin{pgfscope}%
\pgfsys@transformshift{2.060171in}{2.725574in}%
\pgfsys@useobject{currentmarker}{}%
\end{pgfscope}%
\begin{pgfscope}%
\pgfsys@transformshift{2.498694in}{2.046678in}%
\pgfsys@useobject{currentmarker}{}%
\end{pgfscope}%
\begin{pgfscope}%
\pgfsys@transformshift{2.937374in}{2.060056in}%
\pgfsys@useobject{currentmarker}{}%
\end{pgfscope}%
\begin{pgfscope}%
\pgfsys@transformshift{2.060193in}{2.725551in}%
\pgfsys@useobject{currentmarker}{}%
\end{pgfscope}%
\begin{pgfscope}%
\pgfsys@transformshift{2.498693in}{2.046673in}%
\pgfsys@useobject{currentmarker}{}%
\end{pgfscope}%
\begin{pgfscope}%
\pgfsys@transformshift{2.937370in}{2.060060in}%
\pgfsys@useobject{currentmarker}{}%
\end{pgfscope}%
\begin{pgfscope}%
\pgfsys@transformshift{2.060164in}{2.725608in}%
\pgfsys@useobject{currentmarker}{}%
\end{pgfscope}%
\begin{pgfscope}%
\pgfsys@transformshift{2.937353in}{2.060057in}%
\pgfsys@useobject{currentmarker}{}%
\end{pgfscope}%
\begin{pgfscope}%
\pgfsys@transformshift{2.498703in}{2.046645in}%
\pgfsys@useobject{currentmarker}{}%
\end{pgfscope}%
\begin{pgfscope}%
\pgfsys@transformshift{2.060192in}{2.725540in}%
\pgfsys@useobject{currentmarker}{}%
\end{pgfscope}%
\begin{pgfscope}%
\pgfsys@transformshift{2.937371in}{2.060054in}%
\pgfsys@useobject{currentmarker}{}%
\end{pgfscope}%
\begin{pgfscope}%
\pgfsys@transformshift{2.498690in}{2.046656in}%
\pgfsys@useobject{currentmarker}{}%
\end{pgfscope}%
\begin{pgfscope}%
\pgfsys@transformshift{2.060162in}{2.725624in}%
\pgfsys@useobject{currentmarker}{}%
\end{pgfscope}%
\begin{pgfscope}%
\pgfsys@transformshift{2.937346in}{2.060097in}%
\pgfsys@useobject{currentmarker}{}%
\end{pgfscope}%
\begin{pgfscope}%
\pgfsys@transformshift{2.057760in}{2.723999in}%
\pgfsys@useobject{currentmarker}{}%
\end{pgfscope}%
\begin{pgfscope}%
\pgfsys@transformshift{2.498693in}{2.046632in}%
\pgfsys@useobject{currentmarker}{}%
\end{pgfscope}%
\begin{pgfscope}%
\pgfsys@transformshift{2.937350in}{2.060053in}%
\pgfsys@useobject{currentmarker}{}%
\end{pgfscope}%
\begin{pgfscope}%
\pgfsys@transformshift{2.060107in}{2.725822in}%
\pgfsys@useobject{currentmarker}{}%
\end{pgfscope}%
\begin{pgfscope}%
\pgfsys@transformshift{2.498678in}{2.046657in}%
\pgfsys@useobject{currentmarker}{}%
\end{pgfscope}%
\begin{pgfscope}%
\pgfsys@transformshift{2.937365in}{2.060086in}%
\pgfsys@useobject{currentmarker}{}%
\end{pgfscope}%
\begin{pgfscope}%
\pgfsys@transformshift{2.060119in}{2.725691in}%
\pgfsys@useobject{currentmarker}{}%
\end{pgfscope}%
\begin{pgfscope}%
\pgfsys@transformshift{2.937369in}{2.060114in}%
\pgfsys@useobject{currentmarker}{}%
\end{pgfscope}%
\begin{pgfscope}%
\pgfsys@transformshift{2.060141in}{2.725623in}%
\pgfsys@useobject{currentmarker}{}%
\end{pgfscope}%
\begin{pgfscope}%
\pgfsys@transformshift{2.498694in}{2.046596in}%
\pgfsys@useobject{currentmarker}{}%
\end{pgfscope}%
\begin{pgfscope}%
\pgfsys@transformshift{2.937362in}{2.060076in}%
\pgfsys@useobject{currentmarker}{}%
\end{pgfscope}%
\begin{pgfscope}%
\pgfsys@transformshift{2.060143in}{2.725616in}%
\pgfsys@useobject{currentmarker}{}%
\end{pgfscope}%
\begin{pgfscope}%
\pgfsys@transformshift{2.498700in}{2.046616in}%
\pgfsys@useobject{currentmarker}{}%
\end{pgfscope}%
\begin{pgfscope}%
\pgfsys@transformshift{2.937364in}{2.060087in}%
\pgfsys@useobject{currentmarker}{}%
\end{pgfscope}%
\begin{pgfscope}%
\pgfsys@transformshift{2.060125in}{2.725659in}%
\pgfsys@useobject{currentmarker}{}%
\end{pgfscope}%
\begin{pgfscope}%
\pgfsys@transformshift{2.498694in}{2.046643in}%
\pgfsys@useobject{currentmarker}{}%
\end{pgfscope}%
\begin{pgfscope}%
\pgfsys@transformshift{2.937349in}{2.060025in}%
\pgfsys@useobject{currentmarker}{}%
\end{pgfscope}%
\begin{pgfscope}%
\pgfsys@transformshift{2.060150in}{2.725556in}%
\pgfsys@useobject{currentmarker}{}%
\end{pgfscope}%
\begin{pgfscope}%
\pgfsys@transformshift{2.498694in}{2.046648in}%
\pgfsys@useobject{currentmarker}{}%
\end{pgfscope}%
\begin{pgfscope}%
\pgfsys@transformshift{2.937378in}{2.060116in}%
\pgfsys@useobject{currentmarker}{}%
\end{pgfscope}%
\begin{pgfscope}%
\pgfsys@transformshift{2.060146in}{2.725609in}%
\pgfsys@useobject{currentmarker}{}%
\end{pgfscope}%
\begin{pgfscope}%
\pgfsys@transformshift{2.498698in}{2.046601in}%
\pgfsys@useobject{currentmarker}{}%
\end{pgfscope}%
\begin{pgfscope}%
\pgfsys@transformshift{2.937335in}{2.060135in}%
\pgfsys@useobject{currentmarker}{}%
\end{pgfscope}%
\begin{pgfscope}%
\pgfsys@transformshift{2.060132in}{2.725603in}%
\pgfsys@useobject{currentmarker}{}%
\end{pgfscope}%
\begin{pgfscope}%
\pgfsys@transformshift{2.937362in}{2.060107in}%
\pgfsys@useobject{currentmarker}{}%
\end{pgfscope}%
\begin{pgfscope}%
\pgfsys@transformshift{2.498695in}{2.046652in}%
\pgfsys@useobject{currentmarker}{}%
\end{pgfscope}%
\begin{pgfscope}%
\pgfsys@transformshift{2.060178in}{2.725490in}%
\pgfsys@useobject{currentmarker}{}%
\end{pgfscope}%
\begin{pgfscope}%
\pgfsys@transformshift{2.937371in}{2.060130in}%
\pgfsys@useobject{currentmarker}{}%
\end{pgfscope}%
\begin{pgfscope}%
\pgfsys@transformshift{2.498703in}{2.046635in}%
\pgfsys@useobject{currentmarker}{}%
\end{pgfscope}%
\begin{pgfscope}%
\pgfsys@transformshift{2.060115in}{2.725682in}%
\pgfsys@useobject{currentmarker}{}%
\end{pgfscope}%
\begin{pgfscope}%
\pgfsys@transformshift{2.937362in}{2.060069in}%
\pgfsys@useobject{currentmarker}{}%
\end{pgfscope}%
\begin{pgfscope}%
\pgfsys@transformshift{2.498689in}{2.046619in}%
\pgfsys@useobject{currentmarker}{}%
\end{pgfscope}%
\begin{pgfscope}%
\pgfsys@transformshift{2.060118in}{2.725704in}%
\pgfsys@useobject{currentmarker}{}%
\end{pgfscope}%
\begin{pgfscope}%
\pgfsys@transformshift{2.937350in}{2.060132in}%
\pgfsys@useobject{currentmarker}{}%
\end{pgfscope}%
\begin{pgfscope}%
\pgfsys@transformshift{2.060115in}{2.725690in}%
\pgfsys@useobject{currentmarker}{}%
\end{pgfscope}%
\begin{pgfscope}%
\pgfsys@transformshift{2.498677in}{2.046657in}%
\pgfsys@useobject{currentmarker}{}%
\end{pgfscope}%
\begin{pgfscope}%
\pgfsys@transformshift{2.937356in}{2.060095in}%
\pgfsys@useobject{currentmarker}{}%
\end{pgfscope}%
\begin{pgfscope}%
\pgfsys@transformshift{2.937364in}{2.060134in}%
\pgfsys@useobject{currentmarker}{}%
\end{pgfscope}%
\begin{pgfscope}%
\pgfsys@transformshift{2.060142in}{2.725714in}%
\pgfsys@useobject{currentmarker}{}%
\end{pgfscope}%
\begin{pgfscope}%
\pgfsys@transformshift{2.498696in}{2.046617in}%
\pgfsys@useobject{currentmarker}{}%
\end{pgfscope}%
\begin{pgfscope}%
\pgfsys@transformshift{2.937347in}{2.060067in}%
\pgfsys@useobject{currentmarker}{}%
\end{pgfscope}%
\begin{pgfscope}%
\pgfsys@transformshift{2.060061in}{2.725856in}%
\pgfsys@useobject{currentmarker}{}%
\end{pgfscope}%
\begin{pgfscope}%
\pgfsys@transformshift{2.498702in}{2.046641in}%
\pgfsys@useobject{currentmarker}{}%
\end{pgfscope}%
\begin{pgfscope}%
\pgfsys@transformshift{2.060139in}{2.725622in}%
\pgfsys@useobject{currentmarker}{}%
\end{pgfscope}%
\begin{pgfscope}%
\pgfsys@transformshift{2.937339in}{2.060132in}%
\pgfsys@useobject{currentmarker}{}%
\end{pgfscope}%
\begin{pgfscope}%
\pgfsys@transformshift{2.498674in}{2.046629in}%
\pgfsys@useobject{currentmarker}{}%
\end{pgfscope}%
\begin{pgfscope}%
\pgfsys@transformshift{2.937337in}{2.060094in}%
\pgfsys@useobject{currentmarker}{}%
\end{pgfscope}%
\begin{pgfscope}%
\pgfsys@transformshift{2.498680in}{2.046619in}%
\pgfsys@useobject{currentmarker}{}%
\end{pgfscope}%
\begin{pgfscope}%
\pgfsys@transformshift{2.060055in}{2.725834in}%
\pgfsys@useobject{currentmarker}{}%
\end{pgfscope}%
\begin{pgfscope}%
\pgfsys@transformshift{2.060109in}{2.725695in}%
\pgfsys@useobject{currentmarker}{}%
\end{pgfscope}%
\begin{pgfscope}%
\pgfsys@transformshift{2.937317in}{2.060084in}%
\pgfsys@useobject{currentmarker}{}%
\end{pgfscope}%
\end{pgfscope}%
\begin{pgfscope}%
\pgfsetrectcap%
\pgfsetmiterjoin%
\pgfsetlinewidth{0.000000pt}%
\definecolor{currentstroke}{rgb}{1.000000,1.000000,1.000000}%
\pgfsetstrokecolor{currentstroke}%
\pgfsetdash{}{0pt}%
\pgfpathmoveto{\pgfqpoint{1.856795in}{1.819814in}}%
\pgfpathlineto{\pgfqpoint{1.856795in}{2.983200in}}%
\pgfusepath{}%
\end{pgfscope}%
\begin{pgfscope}%
\pgfsetrectcap%
\pgfsetmiterjoin%
\pgfsetlinewidth{0.000000pt}%
\definecolor{currentstroke}{rgb}{1.000000,1.000000,1.000000}%
\pgfsetstrokecolor{currentstroke}%
\pgfsetdash{}{0pt}%
\pgfpathmoveto{\pgfqpoint{3.051000in}{1.819814in}}%
\pgfpathlineto{\pgfqpoint{3.051000in}{2.983200in}}%
\pgfusepath{}%
\end{pgfscope}%
\begin{pgfscope}%
\pgfsetrectcap%
\pgfsetmiterjoin%
\pgfsetlinewidth{0.000000pt}%
\definecolor{currentstroke}{rgb}{1.000000,1.000000,1.000000}%
\pgfsetstrokecolor{currentstroke}%
\pgfsetdash{}{0pt}%
\pgfpathmoveto{\pgfqpoint{1.856795in}{1.819814in}}%
\pgfpathlineto{\pgfqpoint{3.051000in}{1.819814in}}%
\pgfusepath{}%
\end{pgfscope}%
\begin{pgfscope}%
\pgfsetrectcap%
\pgfsetmiterjoin%
\pgfsetlinewidth{0.000000pt}%
\definecolor{currentstroke}{rgb}{1.000000,1.000000,1.000000}%
\pgfsetstrokecolor{currentstroke}%
\pgfsetdash{}{0pt}%
\pgfpathmoveto{\pgfqpoint{1.856795in}{2.983200in}}%
\pgfpathlineto{\pgfqpoint{3.051000in}{2.983200in}}%
\pgfusepath{}%
\end{pgfscope}%
\begin{pgfscope}%
\pgfsetbuttcap%
\pgfsetmiterjoin%
\definecolor{currentfill}{rgb}{0.917647,0.917647,0.949020}%
\pgfsetfillcolor{currentfill}%
\pgfsetlinewidth{0.000000pt}%
\definecolor{currentstroke}{rgb}{0.000000,0.000000,0.000000}%
\pgfsetstrokecolor{currentstroke}%
\pgfsetstrokeopacity{0.000000}%
\pgfsetdash{}{0pt}%
\pgfpathmoveto{\pgfqpoint{0.423750in}{0.423750in}}%
\pgfpathlineto{\pgfqpoint{1.617955in}{0.423750in}}%
\pgfpathlineto{\pgfqpoint{1.617955in}{1.587136in}}%
\pgfpathlineto{\pgfqpoint{0.423750in}{1.587136in}}%
\pgfpathclose%
\pgfusepath{fill}%
\end{pgfscope}%
\begin{pgfscope}%
\pgfpathrectangle{\pgfqpoint{0.423750in}{0.423750in}}{\pgfqpoint{1.194205in}{1.163386in}}%
\pgfusepath{clip}%
\pgfsetroundcap%
\pgfsetroundjoin%
\pgfsetlinewidth{0.803000pt}%
\definecolor{currentstroke}{rgb}{1.000000,1.000000,1.000000}%
\pgfsetstrokecolor{currentstroke}%
\pgfsetdash{}{0pt}%
\pgfpathmoveto{\pgfqpoint{0.846331in}{0.423750in}}%
\pgfpathlineto{\pgfqpoint{0.846331in}{1.587136in}}%
\pgfusepath{stroke}%
\end{pgfscope}%
\begin{pgfscope}%
\definecolor{textcolor}{rgb}{0.150000,0.150000,0.150000}%
\pgfsetstrokecolor{textcolor}%
\pgfsetfillcolor{textcolor}%
\pgftext[x=0.846331in,y=0.375139in,,top]{\color{textcolor}\rmfamily\fontsize{8.000000}{9.600000}\selectfont \(\displaystyle 0\)}%
\end{pgfscope}%
\begin{pgfscope}%
\pgfpathrectangle{\pgfqpoint{0.423750in}{0.423750in}}{\pgfqpoint{1.194205in}{1.163386in}}%
\pgfusepath{clip}%
\pgfsetroundcap%
\pgfsetroundjoin%
\pgfsetlinewidth{0.803000pt}%
\definecolor{currentstroke}{rgb}{1.000000,1.000000,1.000000}%
\pgfsetstrokecolor{currentstroke}%
\pgfsetdash{}{0pt}%
\pgfpathmoveto{\pgfqpoint{1.544417in}{0.423750in}}%
\pgfpathlineto{\pgfqpoint{1.544417in}{1.587136in}}%
\pgfusepath{stroke}%
\end{pgfscope}%
\begin{pgfscope}%
\definecolor{textcolor}{rgb}{0.150000,0.150000,0.150000}%
\pgfsetstrokecolor{textcolor}%
\pgfsetfillcolor{textcolor}%
\pgftext[x=1.544417in,y=0.375139in,,top]{\color{textcolor}\rmfamily\fontsize{8.000000}{9.600000}\selectfont \(\displaystyle 10\)}%
\end{pgfscope}%
\begin{pgfscope}%
\pgfpathrectangle{\pgfqpoint{0.423750in}{0.423750in}}{\pgfqpoint{1.194205in}{1.163386in}}%
\pgfusepath{clip}%
\pgfsetroundcap%
\pgfsetroundjoin%
\pgfsetlinewidth{0.803000pt}%
\definecolor{currentstroke}{rgb}{1.000000,1.000000,1.000000}%
\pgfsetstrokecolor{currentstroke}%
\pgfsetdash{}{0pt}%
\pgfpathmoveto{\pgfqpoint{0.423750in}{0.495869in}}%
\pgfpathlineto{\pgfqpoint{1.617955in}{0.495869in}}%
\pgfusepath{stroke}%
\end{pgfscope}%
\begin{pgfscope}%
\definecolor{textcolor}{rgb}{0.150000,0.150000,0.150000}%
\pgfsetstrokecolor{textcolor}%
\pgfsetfillcolor{textcolor}%
\pgftext[x=0.316110in,y=0.453660in,left,base]{\color{textcolor}\rmfamily\fontsize{8.000000}{9.600000}\selectfont \(\displaystyle 0\)}%
\end{pgfscope}%
\begin{pgfscope}%
\pgfpathrectangle{\pgfqpoint{0.423750in}{0.423750in}}{\pgfqpoint{1.194205in}{1.163386in}}%
\pgfusepath{clip}%
\pgfsetroundcap%
\pgfsetroundjoin%
\pgfsetlinewidth{0.803000pt}%
\definecolor{currentstroke}{rgb}{1.000000,1.000000,1.000000}%
\pgfsetstrokecolor{currentstroke}%
\pgfsetdash{}{0pt}%
\pgfpathmoveto{\pgfqpoint{0.423750in}{0.835585in}}%
\pgfpathlineto{\pgfqpoint{1.617955in}{0.835585in}}%
\pgfusepath{stroke}%
\end{pgfscope}%
\begin{pgfscope}%
\definecolor{textcolor}{rgb}{0.150000,0.150000,0.150000}%
\pgfsetstrokecolor{textcolor}%
\pgfsetfillcolor{textcolor}%
\pgftext[x=0.316110in,y=0.793376in,left,base]{\color{textcolor}\rmfamily\fontsize{8.000000}{9.600000}\selectfont \(\displaystyle 5\)}%
\end{pgfscope}%
\begin{pgfscope}%
\pgfpathrectangle{\pgfqpoint{0.423750in}{0.423750in}}{\pgfqpoint{1.194205in}{1.163386in}}%
\pgfusepath{clip}%
\pgfsetroundcap%
\pgfsetroundjoin%
\pgfsetlinewidth{0.803000pt}%
\definecolor{currentstroke}{rgb}{1.000000,1.000000,1.000000}%
\pgfsetstrokecolor{currentstroke}%
\pgfsetdash{}{0pt}%
\pgfpathmoveto{\pgfqpoint{0.423750in}{1.175301in}}%
\pgfpathlineto{\pgfqpoint{1.617955in}{1.175301in}}%
\pgfusepath{stroke}%
\end{pgfscope}%
\begin{pgfscope}%
\definecolor{textcolor}{rgb}{0.150000,0.150000,0.150000}%
\pgfsetstrokecolor{textcolor}%
\pgfsetfillcolor{textcolor}%
\pgftext[x=0.257082in,y=1.133092in,left,base]{\color{textcolor}\rmfamily\fontsize{8.000000}{9.600000}\selectfont \(\displaystyle 10\)}%
\end{pgfscope}%
\begin{pgfscope}%
\pgfpathrectangle{\pgfqpoint{0.423750in}{0.423750in}}{\pgfqpoint{1.194205in}{1.163386in}}%
\pgfusepath{clip}%
\pgfsetroundcap%
\pgfsetroundjoin%
\pgfsetlinewidth{0.803000pt}%
\definecolor{currentstroke}{rgb}{1.000000,1.000000,1.000000}%
\pgfsetstrokecolor{currentstroke}%
\pgfsetdash{}{0pt}%
\pgfpathmoveto{\pgfqpoint{0.423750in}{1.515018in}}%
\pgfpathlineto{\pgfqpoint{1.617955in}{1.515018in}}%
\pgfusepath{stroke}%
\end{pgfscope}%
\begin{pgfscope}%
\definecolor{textcolor}{rgb}{0.150000,0.150000,0.150000}%
\pgfsetstrokecolor{textcolor}%
\pgfsetfillcolor{textcolor}%
\pgftext[x=0.257082in,y=1.472808in,left,base]{\color{textcolor}\rmfamily\fontsize{8.000000}{9.600000}\selectfont \(\displaystyle 15\)}%
\end{pgfscope}%
\begin{pgfscope}%
\pgfpathrectangle{\pgfqpoint{0.423750in}{0.423750in}}{\pgfqpoint{1.194205in}{1.163386in}}%
\pgfusepath{clip}%
\pgfsetbuttcap%
\pgfsetroundjoin%
\definecolor{currentfill}{rgb}{0.038535,0.033658,0.122459}%
\pgfsetfillcolor{currentfill}%
\pgfsetlinewidth{0.000000pt}%
\definecolor{currentstroke}{rgb}{0.000000,0.000000,0.000000}%
\pgfsetstrokecolor{currentstroke}%
\pgfsetdash{}{0pt}%
\pgfpathmoveto{\pgfqpoint{0.497288in}{0.501515in}}%
\pgfpathlineto{\pgfqpoint{0.497288in}{0.495869in}}%
\pgfpathlineto{\pgfqpoint{0.501974in}{0.495869in}}%
\pgfpathclose%
\pgfusepath{fill}%
\end{pgfscope}%
\begin{pgfscope}%
\pgfpathrectangle{\pgfqpoint{0.423750in}{0.423750in}}{\pgfqpoint{1.194205in}{1.163386in}}%
\pgfusepath{clip}%
\pgfsetbuttcap%
\pgfsetroundjoin%
\definecolor{currentfill}{rgb}{0.106928,0.065172,0.170092}%
\pgfsetfillcolor{currentfill}%
\pgfsetlinewidth{0.000000pt}%
\definecolor{currentstroke}{rgb}{0.000000,0.000000,0.000000}%
\pgfsetstrokecolor{currentstroke}%
\pgfsetdash{}{0pt}%
\pgfpathmoveto{\pgfqpoint{0.507865in}{0.495869in}}%
\pgfpathlineto{\pgfqpoint{0.518442in}{0.495869in}}%
\pgfpathlineto{\pgfqpoint{0.529019in}{0.495869in}}%
\pgfpathlineto{\pgfqpoint{0.539596in}{0.495869in}}%
\pgfpathlineto{\pgfqpoint{0.545774in}{0.495869in}}%
\pgfpathlineto{\pgfqpoint{0.540408in}{0.506163in}}%
\pgfpathlineto{\pgfqpoint{0.539596in}{0.507610in}}%
\pgfpathlineto{\pgfqpoint{0.533688in}{0.516458in}}%
\pgfpathlineto{\pgfqpoint{0.529019in}{0.523310in}}%
\pgfpathlineto{\pgfqpoint{0.526243in}{0.526752in}}%
\pgfpathlineto{\pgfqpoint{0.518442in}{0.536672in}}%
\pgfpathlineto{\pgfqpoint{0.518109in}{0.537047in}}%
\pgfpathlineto{\pgfqpoint{0.509326in}{0.547341in}}%
\pgfpathlineto{\pgfqpoint{0.507865in}{0.549098in}}%
\pgfpathlineto{\pgfqpoint{0.500173in}{0.557635in}}%
\pgfpathlineto{\pgfqpoint{0.497288in}{0.560932in}}%
\pgfpathlineto{\pgfqpoint{0.497288in}{0.557635in}}%
\pgfpathlineto{\pgfqpoint{0.497288in}{0.547341in}}%
\pgfpathlineto{\pgfqpoint{0.497288in}{0.537047in}}%
\pgfpathlineto{\pgfqpoint{0.497288in}{0.526752in}}%
\pgfpathlineto{\pgfqpoint{0.497288in}{0.516458in}}%
\pgfpathlineto{\pgfqpoint{0.497288in}{0.506163in}}%
\pgfpathlineto{\pgfqpoint{0.497288in}{0.501515in}}%
\pgfpathlineto{\pgfqpoint{0.501974in}{0.495869in}}%
\pgfpathclose%
\pgfusepath{fill}%
\end{pgfscope}%
\begin{pgfscope}%
\pgfpathrectangle{\pgfqpoint{0.423750in}{0.423750in}}{\pgfqpoint{1.194205in}{1.163386in}}%
\pgfusepath{clip}%
\pgfsetbuttcap%
\pgfsetroundjoin%
\definecolor{currentfill}{rgb}{0.174895,0.088482,0.219029}%
\pgfsetfillcolor{currentfill}%
\pgfsetlinewidth{0.000000pt}%
\definecolor{currentstroke}{rgb}{0.000000,0.000000,0.000000}%
\pgfsetstrokecolor{currentstroke}%
\pgfsetdash{}{0pt}%
\pgfpathmoveto{\pgfqpoint{0.550173in}{0.495869in}}%
\pgfpathlineto{\pgfqpoint{0.560750in}{0.495869in}}%
\pgfpathlineto{\pgfqpoint{0.570977in}{0.495869in}}%
\pgfpathlineto{\pgfqpoint{0.565613in}{0.506163in}}%
\pgfpathlineto{\pgfqpoint{0.560750in}{0.515591in}}%
\pgfpathlineto{\pgfqpoint{0.560286in}{0.516458in}}%
\pgfpathlineto{\pgfqpoint{0.554456in}{0.526752in}}%
\pgfpathlineto{\pgfqpoint{0.550173in}{0.534135in}}%
\pgfpathlineto{\pgfqpoint{0.548361in}{0.537047in}}%
\pgfpathlineto{\pgfqpoint{0.541983in}{0.547341in}}%
\pgfpathlineto{\pgfqpoint{0.539596in}{0.551207in}}%
\pgfpathlineto{\pgfqpoint{0.535189in}{0.557635in}}%
\pgfpathlineto{\pgfqpoint{0.529019in}{0.566871in}}%
\pgfpathlineto{\pgfqpoint{0.528217in}{0.567930in}}%
\pgfpathlineto{\pgfqpoint{0.520543in}{0.578224in}}%
\pgfpathlineto{\pgfqpoint{0.518442in}{0.581098in}}%
\pgfpathlineto{\pgfqpoint{0.512458in}{0.588519in}}%
\pgfpathlineto{\pgfqpoint{0.507865in}{0.594403in}}%
\pgfpathlineto{\pgfqpoint{0.504160in}{0.598813in}}%
\pgfpathlineto{\pgfqpoint{0.497288in}{0.607248in}}%
\pgfpathlineto{\pgfqpoint{0.497288in}{0.598813in}}%
\pgfpathlineto{\pgfqpoint{0.497288in}{0.588519in}}%
\pgfpathlineto{\pgfqpoint{0.497288in}{0.578224in}}%
\pgfpathlineto{\pgfqpoint{0.497288in}{0.567930in}}%
\pgfpathlineto{\pgfqpoint{0.497288in}{0.560932in}}%
\pgfpathlineto{\pgfqpoint{0.500173in}{0.557635in}}%
\pgfpathlineto{\pgfqpoint{0.507865in}{0.549098in}}%
\pgfpathlineto{\pgfqpoint{0.509326in}{0.547341in}}%
\pgfpathlineto{\pgfqpoint{0.518109in}{0.537047in}}%
\pgfpathlineto{\pgfqpoint{0.518442in}{0.536672in}}%
\pgfpathlineto{\pgfqpoint{0.526243in}{0.526752in}}%
\pgfpathlineto{\pgfqpoint{0.529019in}{0.523310in}}%
\pgfpathlineto{\pgfqpoint{0.533688in}{0.516458in}}%
\pgfpathlineto{\pgfqpoint{0.539596in}{0.507610in}}%
\pgfpathlineto{\pgfqpoint{0.540408in}{0.506163in}}%
\pgfpathlineto{\pgfqpoint{0.545774in}{0.495869in}}%
\pgfpathclose%
\pgfusepath{fill}%
\end{pgfscope}%
\begin{pgfscope}%
\pgfpathrectangle{\pgfqpoint{0.423750in}{0.423750in}}{\pgfqpoint{1.194205in}{1.163386in}}%
\pgfusepath{clip}%
\pgfsetbuttcap%
\pgfsetroundjoin%
\definecolor{currentfill}{rgb}{0.245256,0.104974,0.263956}%
\pgfsetfillcolor{currentfill}%
\pgfsetlinewidth{0.000000pt}%
\definecolor{currentstroke}{rgb}{0.000000,0.000000,0.000000}%
\pgfsetstrokecolor{currentstroke}%
\pgfsetdash{}{0pt}%
\pgfpathmoveto{\pgfqpoint{0.571327in}{0.495869in}}%
\pgfpathlineto{\pgfqpoint{0.581904in}{0.495869in}}%
\pgfpathlineto{\pgfqpoint{0.592481in}{0.495869in}}%
\pgfpathlineto{\pgfqpoint{0.593009in}{0.495869in}}%
\pgfpathlineto{\pgfqpoint{0.592481in}{0.496844in}}%
\pgfpathlineto{\pgfqpoint{0.587395in}{0.506163in}}%
\pgfpathlineto{\pgfqpoint{0.581932in}{0.516458in}}%
\pgfpathlineto{\pgfqpoint{0.581904in}{0.516512in}}%
\pgfpathlineto{\pgfqpoint{0.576637in}{0.526752in}}%
\pgfpathlineto{\pgfqpoint{0.571327in}{0.536276in}}%
\pgfpathlineto{\pgfqpoint{0.570882in}{0.537047in}}%
\pgfpathlineto{\pgfqpoint{0.564878in}{0.547341in}}%
\pgfpathlineto{\pgfqpoint{0.560750in}{0.554409in}}%
\pgfpathlineto{\pgfqpoint{0.558813in}{0.557635in}}%
\pgfpathlineto{\pgfqpoint{0.552604in}{0.567930in}}%
\pgfpathlineto{\pgfqpoint{0.550173in}{0.571946in}}%
\pgfpathlineto{\pgfqpoint{0.546196in}{0.578224in}}%
\pgfpathlineto{\pgfqpoint{0.539711in}{0.588519in}}%
\pgfpathlineto{\pgfqpoint{0.539596in}{0.588701in}}%
\pgfpathlineto{\pgfqpoint{0.532724in}{0.598813in}}%
\pgfpathlineto{\pgfqpoint{0.529019in}{0.604278in}}%
\pgfpathlineto{\pgfqpoint{0.525482in}{0.609108in}}%
\pgfpathlineto{\pgfqpoint{0.518442in}{0.618953in}}%
\pgfpathlineto{\pgfqpoint{0.518098in}{0.619402in}}%
\pgfpathlineto{\pgfqpoint{0.510429in}{0.629696in}}%
\pgfpathlineto{\pgfqpoint{0.507865in}{0.633216in}}%
\pgfpathlineto{\pgfqpoint{0.502620in}{0.639991in}}%
\pgfpathlineto{\pgfqpoint{0.497288in}{0.647053in}}%
\pgfpathlineto{\pgfqpoint{0.497288in}{0.639991in}}%
\pgfpathlineto{\pgfqpoint{0.497288in}{0.629696in}}%
\pgfpathlineto{\pgfqpoint{0.497288in}{0.619402in}}%
\pgfpathlineto{\pgfqpoint{0.497288in}{0.609108in}}%
\pgfpathlineto{\pgfqpoint{0.497288in}{0.607248in}}%
\pgfpathlineto{\pgfqpoint{0.504160in}{0.598813in}}%
\pgfpathlineto{\pgfqpoint{0.507865in}{0.594403in}}%
\pgfpathlineto{\pgfqpoint{0.512458in}{0.588519in}}%
\pgfpathlineto{\pgfqpoint{0.518442in}{0.581098in}}%
\pgfpathlineto{\pgfqpoint{0.520543in}{0.578224in}}%
\pgfpathlineto{\pgfqpoint{0.528217in}{0.567930in}}%
\pgfpathlineto{\pgfqpoint{0.529019in}{0.566871in}}%
\pgfpathlineto{\pgfqpoint{0.535189in}{0.557635in}}%
\pgfpathlineto{\pgfqpoint{0.539596in}{0.551207in}}%
\pgfpathlineto{\pgfqpoint{0.541983in}{0.547341in}}%
\pgfpathlineto{\pgfqpoint{0.548361in}{0.537047in}}%
\pgfpathlineto{\pgfqpoint{0.550173in}{0.534135in}}%
\pgfpathlineto{\pgfqpoint{0.554456in}{0.526752in}}%
\pgfpathlineto{\pgfqpoint{0.560286in}{0.516458in}}%
\pgfpathlineto{\pgfqpoint{0.560750in}{0.515591in}}%
\pgfpathlineto{\pgfqpoint{0.565613in}{0.506163in}}%
\pgfpathlineto{\pgfqpoint{0.570977in}{0.495869in}}%
\pgfpathclose%
\pgfusepath{fill}%
\end{pgfscope}%
\begin{pgfscope}%
\pgfpathrectangle{\pgfqpoint{0.423750in}{0.423750in}}{\pgfqpoint{1.194205in}{1.163386in}}%
\pgfusepath{clip}%
\pgfsetbuttcap%
\pgfsetroundjoin%
\definecolor{currentfill}{rgb}{0.318263,0.115743,0.301001}%
\pgfsetfillcolor{currentfill}%
\pgfsetlinewidth{0.000000pt}%
\definecolor{currentstroke}{rgb}{0.000000,0.000000,0.000000}%
\pgfsetstrokecolor{currentstroke}%
\pgfsetdash{}{0pt}%
\pgfpathmoveto{\pgfqpoint{0.592481in}{0.496844in}}%
\pgfpathlineto{\pgfqpoint{0.593009in}{0.495869in}}%
\pgfpathlineto{\pgfqpoint{0.603058in}{0.495869in}}%
\pgfpathlineto{\pgfqpoint{0.613362in}{0.495869in}}%
\pgfpathlineto{\pgfqpoint{0.607939in}{0.506163in}}%
\pgfpathlineto{\pgfqpoint{0.603058in}{0.515273in}}%
\pgfpathlineto{\pgfqpoint{0.602414in}{0.516458in}}%
\pgfpathlineto{\pgfqpoint{0.597133in}{0.526752in}}%
\pgfpathlineto{\pgfqpoint{0.592481in}{0.535756in}}%
\pgfpathlineto{\pgfqpoint{0.591794in}{0.537047in}}%
\pgfpathlineto{\pgfqpoint{0.585851in}{0.547341in}}%
\pgfpathlineto{\pgfqpoint{0.581904in}{0.554147in}}%
\pgfpathlineto{\pgfqpoint{0.579860in}{0.557635in}}%
\pgfpathlineto{\pgfqpoint{0.573796in}{0.567930in}}%
\pgfpathlineto{\pgfqpoint{0.571327in}{0.572103in}}%
\pgfpathlineto{\pgfqpoint{0.567649in}{0.578224in}}%
\pgfpathlineto{\pgfqpoint{0.561453in}{0.588519in}}%
\pgfpathlineto{\pgfqpoint{0.560750in}{0.589678in}}%
\pgfpathlineto{\pgfqpoint{0.555080in}{0.598813in}}%
\pgfpathlineto{\pgfqpoint{0.550173in}{0.606727in}}%
\pgfpathlineto{\pgfqpoint{0.548641in}{0.609108in}}%
\pgfpathlineto{\pgfqpoint{0.541943in}{0.619402in}}%
\pgfpathlineto{\pgfqpoint{0.539596in}{0.623013in}}%
\pgfpathlineto{\pgfqpoint{0.535047in}{0.629696in}}%
\pgfpathlineto{\pgfqpoint{0.529019in}{0.638677in}}%
\pgfpathlineto{\pgfqpoint{0.528093in}{0.639991in}}%
\pgfpathlineto{\pgfqpoint{0.520885in}{0.650285in}}%
\pgfpathlineto{\pgfqpoint{0.518442in}{0.653811in}}%
\pgfpathlineto{\pgfqpoint{0.513512in}{0.660580in}}%
\pgfpathlineto{\pgfqpoint{0.507865in}{0.668459in}}%
\pgfpathlineto{\pgfqpoint{0.506047in}{0.670874in}}%
\pgfpathlineto{\pgfqpoint{0.498634in}{0.681169in}}%
\pgfpathlineto{\pgfqpoint{0.497288in}{0.683129in}}%
\pgfpathlineto{\pgfqpoint{0.497288in}{0.681169in}}%
\pgfpathlineto{\pgfqpoint{0.497288in}{0.670874in}}%
\pgfpathlineto{\pgfqpoint{0.497288in}{0.660580in}}%
\pgfpathlineto{\pgfqpoint{0.497288in}{0.650285in}}%
\pgfpathlineto{\pgfqpoint{0.497288in}{0.647053in}}%
\pgfpathlineto{\pgfqpoint{0.502620in}{0.639991in}}%
\pgfpathlineto{\pgfqpoint{0.507865in}{0.633216in}}%
\pgfpathlineto{\pgfqpoint{0.510429in}{0.629696in}}%
\pgfpathlineto{\pgfqpoint{0.518098in}{0.619402in}}%
\pgfpathlineto{\pgfqpoint{0.518442in}{0.618953in}}%
\pgfpathlineto{\pgfqpoint{0.525482in}{0.609108in}}%
\pgfpathlineto{\pgfqpoint{0.529019in}{0.604278in}}%
\pgfpathlineto{\pgfqpoint{0.532724in}{0.598813in}}%
\pgfpathlineto{\pgfqpoint{0.539596in}{0.588701in}}%
\pgfpathlineto{\pgfqpoint{0.539711in}{0.588519in}}%
\pgfpathlineto{\pgfqpoint{0.546196in}{0.578224in}}%
\pgfpathlineto{\pgfqpoint{0.550173in}{0.571946in}}%
\pgfpathlineto{\pgfqpoint{0.552604in}{0.567930in}}%
\pgfpathlineto{\pgfqpoint{0.558813in}{0.557635in}}%
\pgfpathlineto{\pgfqpoint{0.560750in}{0.554409in}}%
\pgfpathlineto{\pgfqpoint{0.564878in}{0.547341in}}%
\pgfpathlineto{\pgfqpoint{0.570882in}{0.537047in}}%
\pgfpathlineto{\pgfqpoint{0.571327in}{0.536276in}}%
\pgfpathlineto{\pgfqpoint{0.576637in}{0.526752in}}%
\pgfpathlineto{\pgfqpoint{0.581904in}{0.516512in}}%
\pgfpathlineto{\pgfqpoint{0.581932in}{0.516458in}}%
\pgfpathlineto{\pgfqpoint{0.587395in}{0.506163in}}%
\pgfpathclose%
\pgfusepath{fill}%
\end{pgfscope}%
\begin{pgfscope}%
\pgfpathrectangle{\pgfqpoint{0.423750in}{0.423750in}}{\pgfqpoint{1.194205in}{1.163386in}}%
\pgfusepath{clip}%
\pgfsetbuttcap%
\pgfsetroundjoin%
\definecolor{currentfill}{rgb}{0.400025,0.121350,0.331027}%
\pgfsetfillcolor{currentfill}%
\pgfsetlinewidth{0.000000pt}%
\definecolor{currentstroke}{rgb}{0.000000,0.000000,0.000000}%
\pgfsetstrokecolor{currentstroke}%
\pgfsetdash{}{0pt}%
\pgfpathmoveto{\pgfqpoint{0.613635in}{0.495869in}}%
\pgfpathlineto{\pgfqpoint{0.624212in}{0.495869in}}%
\pgfpathlineto{\pgfqpoint{0.632527in}{0.495869in}}%
\pgfpathlineto{\pgfqpoint{0.627211in}{0.506163in}}%
\pgfpathlineto{\pgfqpoint{0.624212in}{0.512242in}}%
\pgfpathlineto{\pgfqpoint{0.622061in}{0.516458in}}%
\pgfpathlineto{\pgfqpoint{0.616855in}{0.526752in}}%
\pgfpathlineto{\pgfqpoint{0.613635in}{0.532969in}}%
\pgfpathlineto{\pgfqpoint{0.611512in}{0.537047in}}%
\pgfpathlineto{\pgfqpoint{0.605942in}{0.547341in}}%
\pgfpathlineto{\pgfqpoint{0.603058in}{0.552318in}}%
\pgfpathlineto{\pgfqpoint{0.599942in}{0.557635in}}%
\pgfpathlineto{\pgfqpoint{0.593892in}{0.567930in}}%
\pgfpathlineto{\pgfqpoint{0.592481in}{0.570314in}}%
\pgfpathlineto{\pgfqpoint{0.587770in}{0.578224in}}%
\pgfpathlineto{\pgfqpoint{0.581904in}{0.588066in}}%
\pgfpathlineto{\pgfqpoint{0.581632in}{0.588519in}}%
\pgfpathlineto{\pgfqpoint{0.575386in}{0.598813in}}%
\pgfpathlineto{\pgfqpoint{0.571327in}{0.605494in}}%
\pgfpathlineto{\pgfqpoint{0.569101in}{0.609108in}}%
\pgfpathlineto{\pgfqpoint{0.562732in}{0.619402in}}%
\pgfpathlineto{\pgfqpoint{0.560750in}{0.622610in}}%
\pgfpathlineto{\pgfqpoint{0.556284in}{0.629696in}}%
\pgfpathlineto{\pgfqpoint{0.550173in}{0.639343in}}%
\pgfpathlineto{\pgfqpoint{0.549745in}{0.639991in}}%
\pgfpathlineto{\pgfqpoint{0.542929in}{0.650285in}}%
\pgfpathlineto{\pgfqpoint{0.539596in}{0.655340in}}%
\pgfpathlineto{\pgfqpoint{0.536037in}{0.660580in}}%
\pgfpathlineto{\pgfqpoint{0.529089in}{0.670874in}}%
\pgfpathlineto{\pgfqpoint{0.529019in}{0.670977in}}%
\pgfpathlineto{\pgfqpoint{0.521876in}{0.681169in}}%
\pgfpathlineto{\pgfqpoint{0.518442in}{0.686152in}}%
\pgfpathlineto{\pgfqpoint{0.514738in}{0.691463in}}%
\pgfpathlineto{\pgfqpoint{0.507915in}{0.701757in}}%
\pgfpathlineto{\pgfqpoint{0.507865in}{0.701834in}}%
\pgfpathlineto{\pgfqpoint{0.500833in}{0.712052in}}%
\pgfpathlineto{\pgfqpoint{0.497288in}{0.717275in}}%
\pgfpathlineto{\pgfqpoint{0.497288in}{0.712052in}}%
\pgfpathlineto{\pgfqpoint{0.497288in}{0.701757in}}%
\pgfpathlineto{\pgfqpoint{0.497288in}{0.691463in}}%
\pgfpathlineto{\pgfqpoint{0.497288in}{0.683129in}}%
\pgfpathlineto{\pgfqpoint{0.498634in}{0.681169in}}%
\pgfpathlineto{\pgfqpoint{0.506047in}{0.670874in}}%
\pgfpathlineto{\pgfqpoint{0.507865in}{0.668459in}}%
\pgfpathlineto{\pgfqpoint{0.513512in}{0.660580in}}%
\pgfpathlineto{\pgfqpoint{0.518442in}{0.653811in}}%
\pgfpathlineto{\pgfqpoint{0.520885in}{0.650285in}}%
\pgfpathlineto{\pgfqpoint{0.528093in}{0.639991in}}%
\pgfpathlineto{\pgfqpoint{0.529019in}{0.638677in}}%
\pgfpathlineto{\pgfqpoint{0.535047in}{0.629696in}}%
\pgfpathlineto{\pgfqpoint{0.539596in}{0.623013in}}%
\pgfpathlineto{\pgfqpoint{0.541943in}{0.619402in}}%
\pgfpathlineto{\pgfqpoint{0.548641in}{0.609108in}}%
\pgfpathlineto{\pgfqpoint{0.550173in}{0.606727in}}%
\pgfpathlineto{\pgfqpoint{0.555080in}{0.598813in}}%
\pgfpathlineto{\pgfqpoint{0.560750in}{0.589678in}}%
\pgfpathlineto{\pgfqpoint{0.561453in}{0.588519in}}%
\pgfpathlineto{\pgfqpoint{0.567649in}{0.578224in}}%
\pgfpathlineto{\pgfqpoint{0.571327in}{0.572103in}}%
\pgfpathlineto{\pgfqpoint{0.573796in}{0.567930in}}%
\pgfpathlineto{\pgfqpoint{0.579860in}{0.557635in}}%
\pgfpathlineto{\pgfqpoint{0.581904in}{0.554147in}}%
\pgfpathlineto{\pgfqpoint{0.585851in}{0.547341in}}%
\pgfpathlineto{\pgfqpoint{0.591794in}{0.537047in}}%
\pgfpathlineto{\pgfqpoint{0.592481in}{0.535756in}}%
\pgfpathlineto{\pgfqpoint{0.597133in}{0.526752in}}%
\pgfpathlineto{\pgfqpoint{0.602414in}{0.516458in}}%
\pgfpathlineto{\pgfqpoint{0.603058in}{0.515273in}}%
\pgfpathlineto{\pgfqpoint{0.607939in}{0.506163in}}%
\pgfpathlineto{\pgfqpoint{0.613362in}{0.495869in}}%
\pgfpathclose%
\pgfusepath{fill}%
\end{pgfscope}%
\begin{pgfscope}%
\pgfpathrectangle{\pgfqpoint{0.423750in}{0.423750in}}{\pgfqpoint{1.194205in}{1.163386in}}%
\pgfusepath{clip}%
\pgfsetbuttcap%
\pgfsetroundjoin%
\definecolor{currentfill}{rgb}{0.477697,0.120699,0.349023}%
\pgfsetfillcolor{currentfill}%
\pgfsetlinewidth{0.000000pt}%
\definecolor{currentstroke}{rgb}{0.000000,0.000000,0.000000}%
\pgfsetstrokecolor{currentstroke}%
\pgfsetdash{}{0pt}%
\pgfpathmoveto{\pgfqpoint{0.634790in}{0.495869in}}%
\pgfpathlineto{\pgfqpoint{0.645367in}{0.495869in}}%
\pgfpathlineto{\pgfqpoint{0.650890in}{0.495869in}}%
\pgfpathlineto{\pgfqpoint{0.645700in}{0.506163in}}%
\pgfpathlineto{\pgfqpoint{0.645367in}{0.506866in}}%
\pgfpathlineto{\pgfqpoint{0.640745in}{0.516458in}}%
\pgfpathlineto{\pgfqpoint{0.635763in}{0.526752in}}%
\pgfpathlineto{\pgfqpoint{0.634790in}{0.528743in}}%
\pgfpathlineto{\pgfqpoint{0.630591in}{0.537047in}}%
\pgfpathlineto{\pgfqpoint{0.625175in}{0.547341in}}%
\pgfpathlineto{\pgfqpoint{0.624212in}{0.549087in}}%
\pgfpathlineto{\pgfqpoint{0.619376in}{0.557635in}}%
\pgfpathlineto{\pgfqpoint{0.613635in}{0.567375in}}%
\pgfpathlineto{\pgfqpoint{0.613306in}{0.567930in}}%
\pgfpathlineto{\pgfqpoint{0.607156in}{0.578224in}}%
\pgfpathlineto{\pgfqpoint{0.603058in}{0.585076in}}%
\pgfpathlineto{\pgfqpoint{0.600989in}{0.588519in}}%
\pgfpathlineto{\pgfqpoint{0.594770in}{0.598813in}}%
\pgfpathlineto{\pgfqpoint{0.592481in}{0.602586in}}%
\pgfpathlineto{\pgfqpoint{0.588498in}{0.609108in}}%
\pgfpathlineto{\pgfqpoint{0.582201in}{0.619402in}}%
\pgfpathlineto{\pgfqpoint{0.581904in}{0.619884in}}%
\pgfpathlineto{\pgfqpoint{0.575813in}{0.629696in}}%
\pgfpathlineto{\pgfqpoint{0.571327in}{0.636991in}}%
\pgfpathlineto{\pgfqpoint{0.569459in}{0.639991in}}%
\pgfpathlineto{\pgfqpoint{0.563031in}{0.650285in}}%
\pgfpathlineto{\pgfqpoint{0.560750in}{0.653879in}}%
\pgfpathlineto{\pgfqpoint{0.556352in}{0.660580in}}%
\pgfpathlineto{\pgfqpoint{0.550173in}{0.670015in}}%
\pgfpathlineto{\pgfqpoint{0.549599in}{0.670874in}}%
\pgfpathlineto{\pgfqpoint{0.542696in}{0.681169in}}%
\pgfpathlineto{\pgfqpoint{0.539596in}{0.685788in}}%
\pgfpathlineto{\pgfqpoint{0.535698in}{0.691463in}}%
\pgfpathlineto{\pgfqpoint{0.529019in}{0.701401in}}%
\pgfpathlineto{\pgfqpoint{0.528779in}{0.701757in}}%
\pgfpathlineto{\pgfqpoint{0.522097in}{0.712052in}}%
\pgfpathlineto{\pgfqpoint{0.518442in}{0.717712in}}%
\pgfpathlineto{\pgfqpoint{0.515345in}{0.722346in}}%
\pgfpathlineto{\pgfqpoint{0.508513in}{0.732641in}}%
\pgfpathlineto{\pgfqpoint{0.507865in}{0.733600in}}%
\pgfpathlineto{\pgfqpoint{0.501189in}{0.742935in}}%
\pgfpathlineto{\pgfqpoint{0.497288in}{0.748370in}}%
\pgfpathlineto{\pgfqpoint{0.497288in}{0.742935in}}%
\pgfpathlineto{\pgfqpoint{0.497288in}{0.732641in}}%
\pgfpathlineto{\pgfqpoint{0.497288in}{0.722346in}}%
\pgfpathlineto{\pgfqpoint{0.497288in}{0.717275in}}%
\pgfpathlineto{\pgfqpoint{0.500833in}{0.712052in}}%
\pgfpathlineto{\pgfqpoint{0.507865in}{0.701834in}}%
\pgfpathlineto{\pgfqpoint{0.507915in}{0.701757in}}%
\pgfpathlineto{\pgfqpoint{0.514738in}{0.691463in}}%
\pgfpathlineto{\pgfqpoint{0.518442in}{0.686152in}}%
\pgfpathlineto{\pgfqpoint{0.521876in}{0.681169in}}%
\pgfpathlineto{\pgfqpoint{0.529019in}{0.670977in}}%
\pgfpathlineto{\pgfqpoint{0.529089in}{0.670874in}}%
\pgfpathlineto{\pgfqpoint{0.536037in}{0.660580in}}%
\pgfpathlineto{\pgfqpoint{0.539596in}{0.655340in}}%
\pgfpathlineto{\pgfqpoint{0.542929in}{0.650285in}}%
\pgfpathlineto{\pgfqpoint{0.549745in}{0.639991in}}%
\pgfpathlineto{\pgfqpoint{0.550173in}{0.639343in}}%
\pgfpathlineto{\pgfqpoint{0.556284in}{0.629696in}}%
\pgfpathlineto{\pgfqpoint{0.560750in}{0.622610in}}%
\pgfpathlineto{\pgfqpoint{0.562732in}{0.619402in}}%
\pgfpathlineto{\pgfqpoint{0.569101in}{0.609108in}}%
\pgfpathlineto{\pgfqpoint{0.571327in}{0.605494in}}%
\pgfpathlineto{\pgfqpoint{0.575386in}{0.598813in}}%
\pgfpathlineto{\pgfqpoint{0.581632in}{0.588519in}}%
\pgfpathlineto{\pgfqpoint{0.581904in}{0.588066in}}%
\pgfpathlineto{\pgfqpoint{0.587770in}{0.578224in}}%
\pgfpathlineto{\pgfqpoint{0.592481in}{0.570314in}}%
\pgfpathlineto{\pgfqpoint{0.593892in}{0.567930in}}%
\pgfpathlineto{\pgfqpoint{0.599942in}{0.557635in}}%
\pgfpathlineto{\pgfqpoint{0.603058in}{0.552318in}}%
\pgfpathlineto{\pgfqpoint{0.605942in}{0.547341in}}%
\pgfpathlineto{\pgfqpoint{0.611512in}{0.537047in}}%
\pgfpathlineto{\pgfqpoint{0.613635in}{0.532969in}}%
\pgfpathlineto{\pgfqpoint{0.616855in}{0.526752in}}%
\pgfpathlineto{\pgfqpoint{0.622061in}{0.516458in}}%
\pgfpathlineto{\pgfqpoint{0.624212in}{0.512242in}}%
\pgfpathlineto{\pgfqpoint{0.627211in}{0.506163in}}%
\pgfpathlineto{\pgfqpoint{0.632527in}{0.495869in}}%
\pgfpathclose%
\pgfusepath{fill}%
\end{pgfscope}%
\begin{pgfscope}%
\pgfpathrectangle{\pgfqpoint{0.423750in}{0.423750in}}{\pgfqpoint{1.194205in}{1.163386in}}%
\pgfusepath{clip}%
\pgfsetbuttcap%
\pgfsetroundjoin%
\definecolor{currentfill}{rgb}{0.477697,0.120699,0.349023}%
\pgfsetfillcolor{currentfill}%
\pgfsetlinewidth{0.000000pt}%
\definecolor{currentstroke}{rgb}{0.000000,0.000000,0.000000}%
\pgfsetstrokecolor{currentstroke}%
\pgfsetdash{}{0pt}%
\pgfpathmoveto{\pgfqpoint{1.279990in}{1.460787in}}%
\pgfpathlineto{\pgfqpoint{1.290567in}{1.456922in}}%
\pgfpathlineto{\pgfqpoint{1.301144in}{1.454629in}}%
\pgfpathlineto{\pgfqpoint{1.311721in}{1.456609in}}%
\pgfpathlineto{\pgfqpoint{1.322299in}{1.458828in}}%
\pgfpathlineto{\pgfqpoint{1.332876in}{1.461547in}}%
\pgfpathlineto{\pgfqpoint{1.338835in}{1.463545in}}%
\pgfpathlineto{\pgfqpoint{1.343453in}{1.465022in}}%
\pgfpathlineto{\pgfqpoint{1.354030in}{1.469960in}}%
\pgfpathlineto{\pgfqpoint{1.362094in}{1.473840in}}%
\pgfpathlineto{\pgfqpoint{1.364607in}{1.475054in}}%
\pgfpathlineto{\pgfqpoint{1.375184in}{1.480323in}}%
\pgfpathlineto{\pgfqpoint{1.382687in}{1.484134in}}%
\pgfpathlineto{\pgfqpoint{1.385761in}{1.485708in}}%
\pgfpathlineto{\pgfqpoint{1.396338in}{1.491542in}}%
\pgfpathlineto{\pgfqpoint{1.401080in}{1.494429in}}%
\pgfpathlineto{\pgfqpoint{1.406915in}{1.497968in}}%
\pgfpathlineto{\pgfqpoint{1.417462in}{1.504723in}}%
\pgfpathlineto{\pgfqpoint{1.417492in}{1.504742in}}%
\pgfpathlineto{\pgfqpoint{1.428069in}{1.512320in}}%
\pgfpathlineto{\pgfqpoint{1.431785in}{1.515018in}}%
\pgfpathlineto{\pgfqpoint{1.428069in}{1.515018in}}%
\pgfpathlineto{\pgfqpoint{1.417492in}{1.515018in}}%
\pgfpathlineto{\pgfqpoint{1.406915in}{1.515018in}}%
\pgfpathlineto{\pgfqpoint{1.396338in}{1.515018in}}%
\pgfpathlineto{\pgfqpoint{1.385761in}{1.515018in}}%
\pgfpathlineto{\pgfqpoint{1.375184in}{1.515018in}}%
\pgfpathlineto{\pgfqpoint{1.364607in}{1.515018in}}%
\pgfpathlineto{\pgfqpoint{1.354030in}{1.515018in}}%
\pgfpathlineto{\pgfqpoint{1.343453in}{1.515018in}}%
\pgfpathlineto{\pgfqpoint{1.332876in}{1.515018in}}%
\pgfpathlineto{\pgfqpoint{1.322299in}{1.515018in}}%
\pgfpathlineto{\pgfqpoint{1.311721in}{1.515018in}}%
\pgfpathlineto{\pgfqpoint{1.301144in}{1.515018in}}%
\pgfpathlineto{\pgfqpoint{1.290567in}{1.515018in}}%
\pgfpathlineto{\pgfqpoint{1.279990in}{1.515018in}}%
\pgfpathlineto{\pgfqpoint{1.269413in}{1.515018in}}%
\pgfpathlineto{\pgfqpoint{1.258836in}{1.515018in}}%
\pgfpathlineto{\pgfqpoint{1.248259in}{1.515018in}}%
\pgfpathlineto{\pgfqpoint{1.237682in}{1.515018in}}%
\pgfpathlineto{\pgfqpoint{1.227105in}{1.515018in}}%
\pgfpathlineto{\pgfqpoint{1.216528in}{1.515018in}}%
\pgfpathlineto{\pgfqpoint{1.205951in}{1.515018in}}%
\pgfpathlineto{\pgfqpoint{1.195374in}{1.515018in}}%
\pgfpathlineto{\pgfqpoint{1.184797in}{1.515018in}}%
\pgfpathlineto{\pgfqpoint{1.174220in}{1.515018in}}%
\pgfpathlineto{\pgfqpoint{1.163643in}{1.515018in}}%
\pgfpathlineto{\pgfqpoint{1.157598in}{1.515018in}}%
\pgfpathlineto{\pgfqpoint{1.163643in}{1.509913in}}%
\pgfpathlineto{\pgfqpoint{1.171365in}{1.504723in}}%
\pgfpathlineto{\pgfqpoint{1.174220in}{1.502892in}}%
\pgfpathlineto{\pgfqpoint{1.184797in}{1.497394in}}%
\pgfpathlineto{\pgfqpoint{1.193750in}{1.494429in}}%
\pgfpathlineto{\pgfqpoint{1.195374in}{1.493938in}}%
\pgfpathlineto{\pgfqpoint{1.205951in}{1.490964in}}%
\pgfpathlineto{\pgfqpoint{1.216528in}{1.488162in}}%
\pgfpathlineto{\pgfqpoint{1.227105in}{1.485504in}}%
\pgfpathlineto{\pgfqpoint{1.229802in}{1.484134in}}%
\pgfpathlineto{\pgfqpoint{1.237682in}{1.479887in}}%
\pgfpathlineto{\pgfqpoint{1.248259in}{1.474103in}}%
\pgfpathlineto{\pgfqpoint{1.248794in}{1.473840in}}%
\pgfpathlineto{\pgfqpoint{1.258836in}{1.469044in}}%
\pgfpathlineto{\pgfqpoint{1.269413in}{1.464820in}}%
\pgfpathlineto{\pgfqpoint{1.272697in}{1.463545in}}%
\pgfpathclose%
\pgfusepath{fill}%
\end{pgfscope}%
\begin{pgfscope}%
\pgfpathrectangle{\pgfqpoint{0.423750in}{0.423750in}}{\pgfqpoint{1.194205in}{1.163386in}}%
\pgfusepath{clip}%
\pgfsetbuttcap%
\pgfsetroundjoin%
\definecolor{currentfill}{rgb}{0.557345,0.113305,0.357751}%
\pgfsetfillcolor{currentfill}%
\pgfsetlinewidth{0.000000pt}%
\definecolor{currentstroke}{rgb}{0.000000,0.000000,0.000000}%
\pgfsetstrokecolor{currentstroke}%
\pgfsetdash{}{0pt}%
\pgfpathmoveto{\pgfqpoint{0.655944in}{0.495869in}}%
\pgfpathlineto{\pgfqpoint{0.666521in}{0.495869in}}%
\pgfpathlineto{\pgfqpoint{0.668544in}{0.495869in}}%
\pgfpathlineto{\pgfqpoint{0.666521in}{0.500185in}}%
\pgfpathlineto{\pgfqpoint{0.663629in}{0.506163in}}%
\pgfpathlineto{\pgfqpoint{0.658756in}{0.516458in}}%
\pgfpathlineto{\pgfqpoint{0.655944in}{0.522351in}}%
\pgfpathlineto{\pgfqpoint{0.653811in}{0.526752in}}%
\pgfpathlineto{\pgfqpoint{0.648780in}{0.537047in}}%
\pgfpathlineto{\pgfqpoint{0.645367in}{0.543980in}}%
\pgfpathlineto{\pgfqpoint{0.643660in}{0.547341in}}%
\pgfpathlineto{\pgfqpoint{0.638151in}{0.557635in}}%
\pgfpathlineto{\pgfqpoint{0.634790in}{0.563479in}}%
\pgfpathlineto{\pgfqpoint{0.632176in}{0.567930in}}%
\pgfpathlineto{\pgfqpoint{0.626003in}{0.578224in}}%
\pgfpathlineto{\pgfqpoint{0.624212in}{0.581194in}}%
\pgfpathlineto{\pgfqpoint{0.619782in}{0.588519in}}%
\pgfpathlineto{\pgfqpoint{0.613635in}{0.598674in}}%
\pgfpathlineto{\pgfqpoint{0.613551in}{0.598813in}}%
\pgfpathlineto{\pgfqpoint{0.607244in}{0.609108in}}%
\pgfpathlineto{\pgfqpoint{0.603058in}{0.615932in}}%
\pgfpathlineto{\pgfqpoint{0.600920in}{0.619402in}}%
\pgfpathlineto{\pgfqpoint{0.594544in}{0.629696in}}%
\pgfpathlineto{\pgfqpoint{0.592481in}{0.633044in}}%
\pgfpathlineto{\pgfqpoint{0.588179in}{0.639991in}}%
\pgfpathlineto{\pgfqpoint{0.581904in}{0.650136in}}%
\pgfpathlineto{\pgfqpoint{0.581811in}{0.650285in}}%
\pgfpathlineto{\pgfqpoint{0.575352in}{0.660580in}}%
\pgfpathlineto{\pgfqpoint{0.571327in}{0.667000in}}%
\pgfpathlineto{\pgfqpoint{0.568826in}{0.670874in}}%
\pgfpathlineto{\pgfqpoint{0.562065in}{0.681169in}}%
\pgfpathlineto{\pgfqpoint{0.560750in}{0.683160in}}%
\pgfpathlineto{\pgfqpoint{0.555195in}{0.691463in}}%
\pgfpathlineto{\pgfqpoint{0.550173in}{0.698964in}}%
\pgfpathlineto{\pgfqpoint{0.548271in}{0.701757in}}%
\pgfpathlineto{\pgfqpoint{0.541420in}{0.712052in}}%
\pgfpathlineto{\pgfqpoint{0.539596in}{0.714923in}}%
\pgfpathlineto{\pgfqpoint{0.534839in}{0.722346in}}%
\pgfpathlineto{\pgfqpoint{0.529019in}{0.731445in}}%
\pgfpathlineto{\pgfqpoint{0.528235in}{0.732641in}}%
\pgfpathlineto{\pgfqpoint{0.521458in}{0.742935in}}%
\pgfpathlineto{\pgfqpoint{0.518442in}{0.747465in}}%
\pgfpathlineto{\pgfqpoint{0.514405in}{0.753230in}}%
\pgfpathlineto{\pgfqpoint{0.507865in}{0.762444in}}%
\pgfpathlineto{\pgfqpoint{0.507081in}{0.763524in}}%
\pgfpathlineto{\pgfqpoint{0.499630in}{0.773818in}}%
\pgfpathlineto{\pgfqpoint{0.497288in}{0.777078in}}%
\pgfpathlineto{\pgfqpoint{0.497288in}{0.773818in}}%
\pgfpathlineto{\pgfqpoint{0.497288in}{0.763524in}}%
\pgfpathlineto{\pgfqpoint{0.497288in}{0.753230in}}%
\pgfpathlineto{\pgfqpoint{0.497288in}{0.748370in}}%
\pgfpathlineto{\pgfqpoint{0.501189in}{0.742935in}}%
\pgfpathlineto{\pgfqpoint{0.507865in}{0.733600in}}%
\pgfpathlineto{\pgfqpoint{0.508513in}{0.732641in}}%
\pgfpathlineto{\pgfqpoint{0.515345in}{0.722346in}}%
\pgfpathlineto{\pgfqpoint{0.518442in}{0.717712in}}%
\pgfpathlineto{\pgfqpoint{0.522097in}{0.712052in}}%
\pgfpathlineto{\pgfqpoint{0.528779in}{0.701757in}}%
\pgfpathlineto{\pgfqpoint{0.529019in}{0.701401in}}%
\pgfpathlineto{\pgfqpoint{0.535698in}{0.691463in}}%
\pgfpathlineto{\pgfqpoint{0.539596in}{0.685788in}}%
\pgfpathlineto{\pgfqpoint{0.542696in}{0.681169in}}%
\pgfpathlineto{\pgfqpoint{0.549599in}{0.670874in}}%
\pgfpathlineto{\pgfqpoint{0.550173in}{0.670015in}}%
\pgfpathlineto{\pgfqpoint{0.556352in}{0.660580in}}%
\pgfpathlineto{\pgfqpoint{0.560750in}{0.653879in}}%
\pgfpathlineto{\pgfqpoint{0.563031in}{0.650285in}}%
\pgfpathlineto{\pgfqpoint{0.569459in}{0.639991in}}%
\pgfpathlineto{\pgfqpoint{0.571327in}{0.636991in}}%
\pgfpathlineto{\pgfqpoint{0.575813in}{0.629696in}}%
\pgfpathlineto{\pgfqpoint{0.581904in}{0.619884in}}%
\pgfpathlineto{\pgfqpoint{0.582201in}{0.619402in}}%
\pgfpathlineto{\pgfqpoint{0.588498in}{0.609108in}}%
\pgfpathlineto{\pgfqpoint{0.592481in}{0.602586in}}%
\pgfpathlineto{\pgfqpoint{0.594770in}{0.598813in}}%
\pgfpathlineto{\pgfqpoint{0.600989in}{0.588519in}}%
\pgfpathlineto{\pgfqpoint{0.603058in}{0.585076in}}%
\pgfpathlineto{\pgfqpoint{0.607156in}{0.578224in}}%
\pgfpathlineto{\pgfqpoint{0.613306in}{0.567930in}}%
\pgfpathlineto{\pgfqpoint{0.613635in}{0.567375in}}%
\pgfpathlineto{\pgfqpoint{0.619376in}{0.557635in}}%
\pgfpathlineto{\pgfqpoint{0.624212in}{0.549087in}}%
\pgfpathlineto{\pgfqpoint{0.625175in}{0.547341in}}%
\pgfpathlineto{\pgfqpoint{0.630591in}{0.537047in}}%
\pgfpathlineto{\pgfqpoint{0.634790in}{0.528743in}}%
\pgfpathlineto{\pgfqpoint{0.635763in}{0.526752in}}%
\pgfpathlineto{\pgfqpoint{0.640745in}{0.516458in}}%
\pgfpathlineto{\pgfqpoint{0.645367in}{0.506866in}}%
\pgfpathlineto{\pgfqpoint{0.645700in}{0.506163in}}%
\pgfpathlineto{\pgfqpoint{0.650890in}{0.495869in}}%
\pgfpathclose%
\pgfusepath{fill}%
\end{pgfscope}%
\begin{pgfscope}%
\pgfpathrectangle{\pgfqpoint{0.423750in}{0.423750in}}{\pgfqpoint{1.194205in}{1.163386in}}%
\pgfusepath{clip}%
\pgfsetbuttcap%
\pgfsetroundjoin%
\definecolor{currentfill}{rgb}{0.557345,0.113305,0.357751}%
\pgfsetfillcolor{currentfill}%
\pgfsetlinewidth{0.000000pt}%
\definecolor{currentstroke}{rgb}{0.000000,0.000000,0.000000}%
\pgfsetstrokecolor{currentstroke}%
\pgfsetdash{}{0pt}%
\pgfpathmoveto{\pgfqpoint{1.279990in}{1.401328in}}%
\pgfpathlineto{\pgfqpoint{1.290567in}{1.397855in}}%
\pgfpathlineto{\pgfqpoint{1.301144in}{1.394565in}}%
\pgfpathlineto{\pgfqpoint{1.311721in}{1.397582in}}%
\pgfpathlineto{\pgfqpoint{1.321493in}{1.401779in}}%
\pgfpathlineto{\pgfqpoint{1.322299in}{1.402094in}}%
\pgfpathlineto{\pgfqpoint{1.332876in}{1.407329in}}%
\pgfpathlineto{\pgfqpoint{1.342106in}{1.412073in}}%
\pgfpathlineto{\pgfqpoint{1.343453in}{1.412760in}}%
\pgfpathlineto{\pgfqpoint{1.354030in}{1.418538in}}%
\pgfpathlineto{\pgfqpoint{1.360532in}{1.422368in}}%
\pgfpathlineto{\pgfqpoint{1.364607in}{1.424739in}}%
\pgfpathlineto{\pgfqpoint{1.375184in}{1.431397in}}%
\pgfpathlineto{\pgfqpoint{1.377086in}{1.432662in}}%
\pgfpathlineto{\pgfqpoint{1.385761in}{1.438445in}}%
\pgfpathlineto{\pgfqpoint{1.392467in}{1.442957in}}%
\pgfpathlineto{\pgfqpoint{1.396338in}{1.445567in}}%
\pgfpathlineto{\pgfqpoint{1.406915in}{1.452916in}}%
\pgfpathlineto{\pgfqpoint{1.407386in}{1.453251in}}%
\pgfpathlineto{\pgfqpoint{1.417492in}{1.460462in}}%
\pgfpathlineto{\pgfqpoint{1.421764in}{1.463545in}}%
\pgfpathlineto{\pgfqpoint{1.428069in}{1.468107in}}%
\pgfpathlineto{\pgfqpoint{1.435899in}{1.473840in}}%
\pgfpathlineto{\pgfqpoint{1.438646in}{1.475857in}}%
\pgfpathlineto{\pgfqpoint{1.449223in}{1.483729in}}%
\pgfpathlineto{\pgfqpoint{1.449759in}{1.484134in}}%
\pgfpathlineto{\pgfqpoint{1.459800in}{1.491765in}}%
\pgfpathlineto{\pgfqpoint{1.463263in}{1.494429in}}%
\pgfpathlineto{\pgfqpoint{1.470377in}{1.499923in}}%
\pgfpathlineto{\pgfqpoint{1.476521in}{1.504723in}}%
\pgfpathlineto{\pgfqpoint{1.480954in}{1.508202in}}%
\pgfpathlineto{\pgfqpoint{1.489461in}{1.515018in}}%
\pgfpathlineto{\pgfqpoint{1.480954in}{1.515018in}}%
\pgfpathlineto{\pgfqpoint{1.470377in}{1.515018in}}%
\pgfpathlineto{\pgfqpoint{1.459800in}{1.515018in}}%
\pgfpathlineto{\pgfqpoint{1.449223in}{1.515018in}}%
\pgfpathlineto{\pgfqpoint{1.438646in}{1.515018in}}%
\pgfpathlineto{\pgfqpoint{1.431785in}{1.515018in}}%
\pgfpathlineto{\pgfqpoint{1.428069in}{1.512320in}}%
\pgfpathlineto{\pgfqpoint{1.417492in}{1.504742in}}%
\pgfpathlineto{\pgfqpoint{1.417462in}{1.504723in}}%
\pgfpathlineto{\pgfqpoint{1.406915in}{1.497968in}}%
\pgfpathlineto{\pgfqpoint{1.401080in}{1.494429in}}%
\pgfpathlineto{\pgfqpoint{1.396338in}{1.491542in}}%
\pgfpathlineto{\pgfqpoint{1.385761in}{1.485708in}}%
\pgfpathlineto{\pgfqpoint{1.382687in}{1.484134in}}%
\pgfpathlineto{\pgfqpoint{1.375184in}{1.480323in}}%
\pgfpathlineto{\pgfqpoint{1.364607in}{1.475054in}}%
\pgfpathlineto{\pgfqpoint{1.362094in}{1.473840in}}%
\pgfpathlineto{\pgfqpoint{1.354030in}{1.469960in}}%
\pgfpathlineto{\pgfqpoint{1.343453in}{1.465022in}}%
\pgfpathlineto{\pgfqpoint{1.338835in}{1.463545in}}%
\pgfpathlineto{\pgfqpoint{1.332876in}{1.461547in}}%
\pgfpathlineto{\pgfqpoint{1.322299in}{1.458828in}}%
\pgfpathlineto{\pgfqpoint{1.311721in}{1.456609in}}%
\pgfpathlineto{\pgfqpoint{1.301144in}{1.454629in}}%
\pgfpathlineto{\pgfqpoint{1.290567in}{1.456922in}}%
\pgfpathlineto{\pgfqpoint{1.279990in}{1.460787in}}%
\pgfpathlineto{\pgfqpoint{1.272697in}{1.463545in}}%
\pgfpathlineto{\pgfqpoint{1.269413in}{1.464820in}}%
\pgfpathlineto{\pgfqpoint{1.258836in}{1.469044in}}%
\pgfpathlineto{\pgfqpoint{1.248794in}{1.473840in}}%
\pgfpathlineto{\pgfqpoint{1.248259in}{1.474103in}}%
\pgfpathlineto{\pgfqpoint{1.237682in}{1.479887in}}%
\pgfpathlineto{\pgfqpoint{1.229802in}{1.484134in}}%
\pgfpathlineto{\pgfqpoint{1.227105in}{1.485504in}}%
\pgfpathlineto{\pgfqpoint{1.216528in}{1.488162in}}%
\pgfpathlineto{\pgfqpoint{1.205951in}{1.490964in}}%
\pgfpathlineto{\pgfqpoint{1.195374in}{1.493938in}}%
\pgfpathlineto{\pgfqpoint{1.193750in}{1.494429in}}%
\pgfpathlineto{\pgfqpoint{1.184797in}{1.497394in}}%
\pgfpathlineto{\pgfqpoint{1.174220in}{1.502892in}}%
\pgfpathlineto{\pgfqpoint{1.171365in}{1.504723in}}%
\pgfpathlineto{\pgfqpoint{1.163643in}{1.509913in}}%
\pgfpathlineto{\pgfqpoint{1.157598in}{1.515018in}}%
\pgfpathlineto{\pgfqpoint{1.153066in}{1.515018in}}%
\pgfpathlineto{\pgfqpoint{1.142488in}{1.515018in}}%
\pgfpathlineto{\pgfqpoint{1.131911in}{1.515018in}}%
\pgfpathlineto{\pgfqpoint{1.121334in}{1.515018in}}%
\pgfpathlineto{\pgfqpoint{1.110757in}{1.515018in}}%
\pgfpathlineto{\pgfqpoint{1.103845in}{1.515018in}}%
\pgfpathlineto{\pgfqpoint{1.110240in}{1.504723in}}%
\pgfpathlineto{\pgfqpoint{1.110757in}{1.503901in}}%
\pgfpathlineto{\pgfqpoint{1.116780in}{1.494429in}}%
\pgfpathlineto{\pgfqpoint{1.121334in}{1.487337in}}%
\pgfpathlineto{\pgfqpoint{1.123423in}{1.484134in}}%
\pgfpathlineto{\pgfqpoint{1.130242in}{1.473840in}}%
\pgfpathlineto{\pgfqpoint{1.131911in}{1.471360in}}%
\pgfpathlineto{\pgfqpoint{1.137381in}{1.463545in}}%
\pgfpathlineto{\pgfqpoint{1.142488in}{1.456437in}}%
\pgfpathlineto{\pgfqpoint{1.147670in}{1.453251in}}%
\pgfpathlineto{\pgfqpoint{1.153066in}{1.450599in}}%
\pgfpathlineto{\pgfqpoint{1.163643in}{1.446025in}}%
\pgfpathlineto{\pgfqpoint{1.171762in}{1.442957in}}%
\pgfpathlineto{\pgfqpoint{1.174220in}{1.442036in}}%
\pgfpathlineto{\pgfqpoint{1.184797in}{1.438538in}}%
\pgfpathlineto{\pgfqpoint{1.195374in}{1.435357in}}%
\pgfpathlineto{\pgfqpoint{1.205116in}{1.432662in}}%
\pgfpathlineto{\pgfqpoint{1.205951in}{1.432436in}}%
\pgfpathlineto{\pgfqpoint{1.216528in}{1.428922in}}%
\pgfpathlineto{\pgfqpoint{1.227105in}{1.423331in}}%
\pgfpathlineto{\pgfqpoint{1.228965in}{1.422368in}}%
\pgfpathlineto{\pgfqpoint{1.237682in}{1.417925in}}%
\pgfpathlineto{\pgfqpoint{1.248259in}{1.412635in}}%
\pgfpathlineto{\pgfqpoint{1.249727in}{1.412073in}}%
\pgfpathlineto{\pgfqpoint{1.258836in}{1.408684in}}%
\pgfpathlineto{\pgfqpoint{1.269413in}{1.404946in}}%
\pgfpathlineto{\pgfqpoint{1.278657in}{1.401779in}}%
\pgfpathclose%
\pgfusepath{fill}%
\end{pgfscope}%
\begin{pgfscope}%
\pgfpathrectangle{\pgfqpoint{0.423750in}{0.423750in}}{\pgfqpoint{1.194205in}{1.163386in}}%
\pgfusepath{clip}%
\pgfsetbuttcap%
\pgfsetroundjoin%
\definecolor{currentfill}{rgb}{0.638121,0.099382,0.356038}%
\pgfsetfillcolor{currentfill}%
\pgfsetlinewidth{0.000000pt}%
\definecolor{currentstroke}{rgb}{0.000000,0.000000,0.000000}%
\pgfsetstrokecolor{currentstroke}%
\pgfsetdash{}{0pt}%
\pgfpathmoveto{\pgfqpoint{0.666521in}{0.500185in}}%
\pgfpathlineto{\pgfqpoint{0.668544in}{0.495869in}}%
\pgfpathlineto{\pgfqpoint{0.677098in}{0.495869in}}%
\pgfpathlineto{\pgfqpoint{0.685264in}{0.495869in}}%
\pgfpathlineto{\pgfqpoint{0.680793in}{0.506163in}}%
\pgfpathlineto{\pgfqpoint{0.677098in}{0.514356in}}%
\pgfpathlineto{\pgfqpoint{0.676110in}{0.516458in}}%
\pgfpathlineto{\pgfqpoint{0.671221in}{0.526752in}}%
\pgfpathlineto{\pgfqpoint{0.666521in}{0.536581in}}%
\pgfpathlineto{\pgfqpoint{0.666294in}{0.537047in}}%
\pgfpathlineto{\pgfqpoint{0.661227in}{0.547341in}}%
\pgfpathlineto{\pgfqpoint{0.656127in}{0.557635in}}%
\pgfpathlineto{\pgfqpoint{0.655944in}{0.557989in}}%
\pgfpathlineto{\pgfqpoint{0.650504in}{0.567930in}}%
\pgfpathlineto{\pgfqpoint{0.645367in}{0.576601in}}%
\pgfpathlineto{\pgfqpoint{0.644393in}{0.578224in}}%
\pgfpathlineto{\pgfqpoint{0.638123in}{0.588519in}}%
\pgfpathlineto{\pgfqpoint{0.634790in}{0.593977in}}%
\pgfpathlineto{\pgfqpoint{0.631828in}{0.598813in}}%
\pgfpathlineto{\pgfqpoint{0.625504in}{0.609108in}}%
\pgfpathlineto{\pgfqpoint{0.624212in}{0.611196in}}%
\pgfpathlineto{\pgfqpoint{0.619125in}{0.619402in}}%
\pgfpathlineto{\pgfqpoint{0.613635in}{0.628242in}}%
\pgfpathlineto{\pgfqpoint{0.612729in}{0.629696in}}%
\pgfpathlineto{\pgfqpoint{0.606317in}{0.639991in}}%
\pgfpathlineto{\pgfqpoint{0.603058in}{0.645235in}}%
\pgfpathlineto{\pgfqpoint{0.599907in}{0.650285in}}%
\pgfpathlineto{\pgfqpoint{0.593475in}{0.660580in}}%
\pgfpathlineto{\pgfqpoint{0.592481in}{0.662161in}}%
\pgfpathlineto{\pgfqpoint{0.586979in}{0.670874in}}%
\pgfpathlineto{\pgfqpoint{0.581904in}{0.678910in}}%
\pgfpathlineto{\pgfqpoint{0.580455in}{0.681169in}}%
\pgfpathlineto{\pgfqpoint{0.573736in}{0.691463in}}%
\pgfpathlineto{\pgfqpoint{0.571327in}{0.695097in}}%
\pgfpathlineto{\pgfqpoint{0.566866in}{0.701757in}}%
\pgfpathlineto{\pgfqpoint{0.560750in}{0.710877in}}%
\pgfpathlineto{\pgfqpoint{0.559953in}{0.712052in}}%
\pgfpathlineto{\pgfqpoint{0.553084in}{0.722346in}}%
\pgfpathlineto{\pgfqpoint{0.550173in}{0.726931in}}%
\pgfpathlineto{\pgfqpoint{0.546522in}{0.732641in}}%
\pgfpathlineto{\pgfqpoint{0.539926in}{0.742935in}}%
\pgfpathlineto{\pgfqpoint{0.539596in}{0.743446in}}%
\pgfpathlineto{\pgfqpoint{0.533170in}{0.753230in}}%
\pgfpathlineto{\pgfqpoint{0.529019in}{0.759507in}}%
\pgfpathlineto{\pgfqpoint{0.526244in}{0.763524in}}%
\pgfpathlineto{\pgfqpoint{0.518961in}{0.773818in}}%
\pgfpathlineto{\pgfqpoint{0.518442in}{0.774550in}}%
\pgfpathlineto{\pgfqpoint{0.511555in}{0.784113in}}%
\pgfpathlineto{\pgfqpoint{0.507865in}{0.789271in}}%
\pgfpathlineto{\pgfqpoint{0.504131in}{0.794407in}}%
\pgfpathlineto{\pgfqpoint{0.497288in}{0.803898in}}%
\pgfpathlineto{\pgfqpoint{0.497288in}{0.794407in}}%
\pgfpathlineto{\pgfqpoint{0.497288in}{0.784113in}}%
\pgfpathlineto{\pgfqpoint{0.497288in}{0.777078in}}%
\pgfpathlineto{\pgfqpoint{0.499630in}{0.773818in}}%
\pgfpathlineto{\pgfqpoint{0.507081in}{0.763524in}}%
\pgfpathlineto{\pgfqpoint{0.507865in}{0.762444in}}%
\pgfpathlineto{\pgfqpoint{0.514405in}{0.753230in}}%
\pgfpathlineto{\pgfqpoint{0.518442in}{0.747465in}}%
\pgfpathlineto{\pgfqpoint{0.521458in}{0.742935in}}%
\pgfpathlineto{\pgfqpoint{0.528235in}{0.732641in}}%
\pgfpathlineto{\pgfqpoint{0.529019in}{0.731445in}}%
\pgfpathlineto{\pgfqpoint{0.534839in}{0.722346in}}%
\pgfpathlineto{\pgfqpoint{0.539596in}{0.714923in}}%
\pgfpathlineto{\pgfqpoint{0.541420in}{0.712052in}}%
\pgfpathlineto{\pgfqpoint{0.548271in}{0.701757in}}%
\pgfpathlineto{\pgfqpoint{0.550173in}{0.698964in}}%
\pgfpathlineto{\pgfqpoint{0.555195in}{0.691463in}}%
\pgfpathlineto{\pgfqpoint{0.560750in}{0.683160in}}%
\pgfpathlineto{\pgfqpoint{0.562065in}{0.681169in}}%
\pgfpathlineto{\pgfqpoint{0.568826in}{0.670874in}}%
\pgfpathlineto{\pgfqpoint{0.571327in}{0.667000in}}%
\pgfpathlineto{\pgfqpoint{0.575352in}{0.660580in}}%
\pgfpathlineto{\pgfqpoint{0.581811in}{0.650285in}}%
\pgfpathlineto{\pgfqpoint{0.581904in}{0.650136in}}%
\pgfpathlineto{\pgfqpoint{0.588179in}{0.639991in}}%
\pgfpathlineto{\pgfqpoint{0.592481in}{0.633044in}}%
\pgfpathlineto{\pgfqpoint{0.594544in}{0.629696in}}%
\pgfpathlineto{\pgfqpoint{0.600920in}{0.619402in}}%
\pgfpathlineto{\pgfqpoint{0.603058in}{0.615932in}}%
\pgfpathlineto{\pgfqpoint{0.607244in}{0.609108in}}%
\pgfpathlineto{\pgfqpoint{0.613551in}{0.598813in}}%
\pgfpathlineto{\pgfqpoint{0.613635in}{0.598674in}}%
\pgfpathlineto{\pgfqpoint{0.619782in}{0.588519in}}%
\pgfpathlineto{\pgfqpoint{0.624212in}{0.581194in}}%
\pgfpathlineto{\pgfqpoint{0.626003in}{0.578224in}}%
\pgfpathlineto{\pgfqpoint{0.632176in}{0.567930in}}%
\pgfpathlineto{\pgfqpoint{0.634790in}{0.563479in}}%
\pgfpathlineto{\pgfqpoint{0.638151in}{0.557635in}}%
\pgfpathlineto{\pgfqpoint{0.643660in}{0.547341in}}%
\pgfpathlineto{\pgfqpoint{0.645367in}{0.543980in}}%
\pgfpathlineto{\pgfqpoint{0.648780in}{0.537047in}}%
\pgfpathlineto{\pgfqpoint{0.653811in}{0.526752in}}%
\pgfpathlineto{\pgfqpoint{0.655944in}{0.522351in}}%
\pgfpathlineto{\pgfqpoint{0.658756in}{0.516458in}}%
\pgfpathlineto{\pgfqpoint{0.663629in}{0.506163in}}%
\pgfpathclose%
\pgfusepath{fill}%
\end{pgfscope}%
\begin{pgfscope}%
\pgfpathrectangle{\pgfqpoint{0.423750in}{0.423750in}}{\pgfqpoint{1.194205in}{1.163386in}}%
\pgfusepath{clip}%
\pgfsetbuttcap%
\pgfsetroundjoin%
\definecolor{currentfill}{rgb}{0.638121,0.099382,0.356038}%
\pgfsetfillcolor{currentfill}%
\pgfsetlinewidth{0.000000pt}%
\definecolor{currentstroke}{rgb}{0.000000,0.000000,0.000000}%
\pgfsetstrokecolor{currentstroke}%
\pgfsetdash{}{0pt}%
\pgfpathmoveto{\pgfqpoint{1.279990in}{1.346876in}}%
\pgfpathlineto{\pgfqpoint{1.290567in}{1.344397in}}%
\pgfpathlineto{\pgfqpoint{1.301144in}{1.344001in}}%
\pgfpathlineto{\pgfqpoint{1.311721in}{1.348456in}}%
\pgfpathlineto{\pgfqpoint{1.314926in}{1.350307in}}%
\pgfpathlineto{\pgfqpoint{1.322299in}{1.354437in}}%
\pgfpathlineto{\pgfqpoint{1.331997in}{1.360601in}}%
\pgfpathlineto{\pgfqpoint{1.332876in}{1.361137in}}%
\pgfpathlineto{\pgfqpoint{1.343453in}{1.368179in}}%
\pgfpathlineto{\pgfqpoint{1.347481in}{1.370896in}}%
\pgfpathlineto{\pgfqpoint{1.354030in}{1.375308in}}%
\pgfpathlineto{\pgfqpoint{1.362672in}{1.381190in}}%
\pgfpathlineto{\pgfqpoint{1.364607in}{1.382505in}}%
\pgfpathlineto{\pgfqpoint{1.375184in}{1.389808in}}%
\pgfpathlineto{\pgfqpoint{1.377576in}{1.391484in}}%
\pgfpathlineto{\pgfqpoint{1.385761in}{1.397211in}}%
\pgfpathlineto{\pgfqpoint{1.392220in}{1.401779in}}%
\pgfpathlineto{\pgfqpoint{1.396338in}{1.404672in}}%
\pgfpathlineto{\pgfqpoint{1.406470in}{1.412073in}}%
\pgfpathlineto{\pgfqpoint{1.406915in}{1.412393in}}%
\pgfpathlineto{\pgfqpoint{1.417492in}{1.420162in}}%
\pgfpathlineto{\pgfqpoint{1.420459in}{1.422368in}}%
\pgfpathlineto{\pgfqpoint{1.428069in}{1.428022in}}%
\pgfpathlineto{\pgfqpoint{1.434260in}{1.432662in}}%
\pgfpathlineto{\pgfqpoint{1.438646in}{1.435958in}}%
\pgfpathlineto{\pgfqpoint{1.447896in}{1.442957in}}%
\pgfpathlineto{\pgfqpoint{1.449223in}{1.443966in}}%
\pgfpathlineto{\pgfqpoint{1.459800in}{1.452061in}}%
\pgfpathlineto{\pgfqpoint{1.461345in}{1.453251in}}%
\pgfpathlineto{\pgfqpoint{1.470377in}{1.460252in}}%
\pgfpathlineto{\pgfqpoint{1.474593in}{1.463545in}}%
\pgfpathlineto{\pgfqpoint{1.480954in}{1.468545in}}%
\pgfpathlineto{\pgfqpoint{1.487640in}{1.473840in}}%
\pgfpathlineto{\pgfqpoint{1.491532in}{1.476943in}}%
\pgfpathlineto{\pgfqpoint{1.500478in}{1.484134in}}%
\pgfpathlineto{\pgfqpoint{1.502109in}{1.485454in}}%
\pgfpathlineto{\pgfqpoint{1.512686in}{1.494089in}}%
\pgfpathlineto{\pgfqpoint{1.513098in}{1.494429in}}%
\pgfpathlineto{\pgfqpoint{1.523263in}{1.502859in}}%
\pgfpathlineto{\pgfqpoint{1.525492in}{1.504723in}}%
\pgfpathlineto{\pgfqpoint{1.533840in}{1.511751in}}%
\pgfpathlineto{\pgfqpoint{1.537688in}{1.515018in}}%
\pgfpathlineto{\pgfqpoint{1.533840in}{1.515018in}}%
\pgfpathlineto{\pgfqpoint{1.523263in}{1.515018in}}%
\pgfpathlineto{\pgfqpoint{1.512686in}{1.515018in}}%
\pgfpathlineto{\pgfqpoint{1.502109in}{1.515018in}}%
\pgfpathlineto{\pgfqpoint{1.491532in}{1.515018in}}%
\pgfpathlineto{\pgfqpoint{1.489461in}{1.515018in}}%
\pgfpathlineto{\pgfqpoint{1.480954in}{1.508202in}}%
\pgfpathlineto{\pgfqpoint{1.476521in}{1.504723in}}%
\pgfpathlineto{\pgfqpoint{1.470377in}{1.499923in}}%
\pgfpathlineto{\pgfqpoint{1.463263in}{1.494429in}}%
\pgfpathlineto{\pgfqpoint{1.459800in}{1.491765in}}%
\pgfpathlineto{\pgfqpoint{1.449759in}{1.484134in}}%
\pgfpathlineto{\pgfqpoint{1.449223in}{1.483729in}}%
\pgfpathlineto{\pgfqpoint{1.438646in}{1.475857in}}%
\pgfpathlineto{\pgfqpoint{1.435899in}{1.473840in}}%
\pgfpathlineto{\pgfqpoint{1.428069in}{1.468107in}}%
\pgfpathlineto{\pgfqpoint{1.421764in}{1.463545in}}%
\pgfpathlineto{\pgfqpoint{1.417492in}{1.460462in}}%
\pgfpathlineto{\pgfqpoint{1.407386in}{1.453251in}}%
\pgfpathlineto{\pgfqpoint{1.406915in}{1.452916in}}%
\pgfpathlineto{\pgfqpoint{1.396338in}{1.445567in}}%
\pgfpathlineto{\pgfqpoint{1.392467in}{1.442957in}}%
\pgfpathlineto{\pgfqpoint{1.385761in}{1.438445in}}%
\pgfpathlineto{\pgfqpoint{1.377086in}{1.432662in}}%
\pgfpathlineto{\pgfqpoint{1.375184in}{1.431397in}}%
\pgfpathlineto{\pgfqpoint{1.364607in}{1.424739in}}%
\pgfpathlineto{\pgfqpoint{1.360532in}{1.422368in}}%
\pgfpathlineto{\pgfqpoint{1.354030in}{1.418538in}}%
\pgfpathlineto{\pgfqpoint{1.343453in}{1.412760in}}%
\pgfpathlineto{\pgfqpoint{1.342106in}{1.412073in}}%
\pgfpathlineto{\pgfqpoint{1.332876in}{1.407329in}}%
\pgfpathlineto{\pgfqpoint{1.322299in}{1.402094in}}%
\pgfpathlineto{\pgfqpoint{1.321493in}{1.401779in}}%
\pgfpathlineto{\pgfqpoint{1.311721in}{1.397582in}}%
\pgfpathlineto{\pgfqpoint{1.301144in}{1.394565in}}%
\pgfpathlineto{\pgfqpoint{1.290567in}{1.397855in}}%
\pgfpathlineto{\pgfqpoint{1.279990in}{1.401328in}}%
\pgfpathlineto{\pgfqpoint{1.278657in}{1.401779in}}%
\pgfpathlineto{\pgfqpoint{1.269413in}{1.404946in}}%
\pgfpathlineto{\pgfqpoint{1.258836in}{1.408684in}}%
\pgfpathlineto{\pgfqpoint{1.249727in}{1.412073in}}%
\pgfpathlineto{\pgfqpoint{1.248259in}{1.412635in}}%
\pgfpathlineto{\pgfqpoint{1.237682in}{1.417925in}}%
\pgfpathlineto{\pgfqpoint{1.228965in}{1.422368in}}%
\pgfpathlineto{\pgfqpoint{1.227105in}{1.423331in}}%
\pgfpathlineto{\pgfqpoint{1.216528in}{1.428922in}}%
\pgfpathlineto{\pgfqpoint{1.205951in}{1.432436in}}%
\pgfpathlineto{\pgfqpoint{1.205116in}{1.432662in}}%
\pgfpathlineto{\pgfqpoint{1.195374in}{1.435357in}}%
\pgfpathlineto{\pgfqpoint{1.184797in}{1.438538in}}%
\pgfpathlineto{\pgfqpoint{1.174220in}{1.442036in}}%
\pgfpathlineto{\pgfqpoint{1.171762in}{1.442957in}}%
\pgfpathlineto{\pgfqpoint{1.163643in}{1.446025in}}%
\pgfpathlineto{\pgfqpoint{1.153066in}{1.450599in}}%
\pgfpathlineto{\pgfqpoint{1.147670in}{1.453251in}}%
\pgfpathlineto{\pgfqpoint{1.142488in}{1.456437in}}%
\pgfpathlineto{\pgfqpoint{1.137381in}{1.463545in}}%
\pgfpathlineto{\pgfqpoint{1.131911in}{1.471360in}}%
\pgfpathlineto{\pgfqpoint{1.130242in}{1.473840in}}%
\pgfpathlineto{\pgfqpoint{1.123423in}{1.484134in}}%
\pgfpathlineto{\pgfqpoint{1.121334in}{1.487337in}}%
\pgfpathlineto{\pgfqpoint{1.116780in}{1.494429in}}%
\pgfpathlineto{\pgfqpoint{1.110757in}{1.503901in}}%
\pgfpathlineto{\pgfqpoint{1.110240in}{1.504723in}}%
\pgfpathlineto{\pgfqpoint{1.103845in}{1.515018in}}%
\pgfpathlineto{\pgfqpoint{1.100180in}{1.515018in}}%
\pgfpathlineto{\pgfqpoint{1.089603in}{1.515018in}}%
\pgfpathlineto{\pgfqpoint{1.079026in}{1.515018in}}%
\pgfpathlineto{\pgfqpoint{1.068449in}{1.515018in}}%
\pgfpathlineto{\pgfqpoint{1.059317in}{1.515018in}}%
\pgfpathlineto{\pgfqpoint{1.065685in}{1.504723in}}%
\pgfpathlineto{\pgfqpoint{1.068449in}{1.500276in}}%
\pgfpathlineto{\pgfqpoint{1.072110in}{1.494429in}}%
\pgfpathlineto{\pgfqpoint{1.078589in}{1.484134in}}%
\pgfpathlineto{\pgfqpoint{1.079026in}{1.483446in}}%
\pgfpathlineto{\pgfqpoint{1.085187in}{1.473840in}}%
\pgfpathlineto{\pgfqpoint{1.089603in}{1.467006in}}%
\pgfpathlineto{\pgfqpoint{1.091863in}{1.463545in}}%
\pgfpathlineto{\pgfqpoint{1.098661in}{1.453251in}}%
\pgfpathlineto{\pgfqpoint{1.100180in}{1.450977in}}%
\pgfpathlineto{\pgfqpoint{1.105611in}{1.442957in}}%
\pgfpathlineto{\pgfqpoint{1.110757in}{1.435440in}}%
\pgfpathlineto{\pgfqpoint{1.112688in}{1.432662in}}%
\pgfpathlineto{\pgfqpoint{1.120427in}{1.422368in}}%
\pgfpathlineto{\pgfqpoint{1.121334in}{1.421303in}}%
\pgfpathlineto{\pgfqpoint{1.130809in}{1.412073in}}%
\pgfpathlineto{\pgfqpoint{1.131911in}{1.411102in}}%
\pgfpathlineto{\pgfqpoint{1.142488in}{1.404175in}}%
\pgfpathlineto{\pgfqpoint{1.147178in}{1.401779in}}%
\pgfpathlineto{\pgfqpoint{1.153066in}{1.398776in}}%
\pgfpathlineto{\pgfqpoint{1.163643in}{1.393886in}}%
\pgfpathlineto{\pgfqpoint{1.169762in}{1.391484in}}%
\pgfpathlineto{\pgfqpoint{1.174220in}{1.389742in}}%
\pgfpathlineto{\pgfqpoint{1.184797in}{1.386138in}}%
\pgfpathlineto{\pgfqpoint{1.195374in}{1.382888in}}%
\pgfpathlineto{\pgfqpoint{1.199731in}{1.381190in}}%
\pgfpathlineto{\pgfqpoint{1.205951in}{1.378592in}}%
\pgfpathlineto{\pgfqpoint{1.216528in}{1.373004in}}%
\pgfpathlineto{\pgfqpoint{1.220597in}{1.370896in}}%
\pgfpathlineto{\pgfqpoint{1.227105in}{1.367563in}}%
\pgfpathlineto{\pgfqpoint{1.237682in}{1.362251in}}%
\pgfpathlineto{\pgfqpoint{1.241388in}{1.360601in}}%
\pgfpathlineto{\pgfqpoint{1.248259in}{1.357615in}}%
\pgfpathlineto{\pgfqpoint{1.258836in}{1.353905in}}%
\pgfpathlineto{\pgfqpoint{1.269413in}{1.350309in}}%
\pgfpathlineto{\pgfqpoint{1.269421in}{1.350307in}}%
\pgfpathclose%
\pgfusepath{fill}%
\end{pgfscope}%
\begin{pgfscope}%
\pgfpathrectangle{\pgfqpoint{0.423750in}{0.423750in}}{\pgfqpoint{1.194205in}{1.163386in}}%
\pgfusepath{clip}%
\pgfsetbuttcap%
\pgfsetroundjoin%
\definecolor{currentfill}{rgb}{0.717551,0.087162,0.341300}%
\pgfsetfillcolor{currentfill}%
\pgfsetlinewidth{0.000000pt}%
\definecolor{currentstroke}{rgb}{0.000000,0.000000,0.000000}%
\pgfsetstrokecolor{currentstroke}%
\pgfsetdash{}{0pt}%
\pgfpathmoveto{\pgfqpoint{0.687675in}{0.495869in}}%
\pgfpathlineto{\pgfqpoint{0.698252in}{0.495869in}}%
\pgfpathlineto{\pgfqpoint{0.701254in}{0.495869in}}%
\pgfpathlineto{\pgfqpoint{0.698252in}{0.502989in}}%
\pgfpathlineto{\pgfqpoint{0.696906in}{0.506163in}}%
\pgfpathlineto{\pgfqpoint{0.692513in}{0.516458in}}%
\pgfpathlineto{\pgfqpoint{0.687990in}{0.526752in}}%
\pgfpathlineto{\pgfqpoint{0.687675in}{0.527448in}}%
\pgfpathlineto{\pgfqpoint{0.683125in}{0.537047in}}%
\pgfpathlineto{\pgfqpoint{0.678203in}{0.547341in}}%
\pgfpathlineto{\pgfqpoint{0.677098in}{0.549624in}}%
\pgfpathlineto{\pgfqpoint{0.673143in}{0.557635in}}%
\pgfpathlineto{\pgfqpoint{0.668013in}{0.567930in}}%
\pgfpathlineto{\pgfqpoint{0.666521in}{0.570735in}}%
\pgfpathlineto{\pgfqpoint{0.662309in}{0.578224in}}%
\pgfpathlineto{\pgfqpoint{0.656079in}{0.588519in}}%
\pgfpathlineto{\pgfqpoint{0.655944in}{0.588738in}}%
\pgfpathlineto{\pgfqpoint{0.649717in}{0.598813in}}%
\pgfpathlineto{\pgfqpoint{0.645367in}{0.605841in}}%
\pgfpathlineto{\pgfqpoint{0.643341in}{0.609108in}}%
\pgfpathlineto{\pgfqpoint{0.636923in}{0.619402in}}%
\pgfpathlineto{\pgfqpoint{0.634790in}{0.622805in}}%
\pgfpathlineto{\pgfqpoint{0.630461in}{0.629696in}}%
\pgfpathlineto{\pgfqpoint{0.624212in}{0.639671in}}%
\pgfpathlineto{\pgfqpoint{0.624012in}{0.639991in}}%
\pgfpathlineto{\pgfqpoint{0.617548in}{0.650285in}}%
\pgfpathlineto{\pgfqpoint{0.613635in}{0.656512in}}%
\pgfpathlineto{\pgfqpoint{0.611072in}{0.660580in}}%
\pgfpathlineto{\pgfqpoint{0.604566in}{0.670874in}}%
\pgfpathlineto{\pgfqpoint{0.603058in}{0.673247in}}%
\pgfpathlineto{\pgfqpoint{0.598008in}{0.681169in}}%
\pgfpathlineto{\pgfqpoint{0.592481in}{0.689826in}}%
\pgfpathlineto{\pgfqpoint{0.591432in}{0.691463in}}%
\pgfpathlineto{\pgfqpoint{0.584739in}{0.701757in}}%
\pgfpathlineto{\pgfqpoint{0.581904in}{0.706046in}}%
\pgfpathlineto{\pgfqpoint{0.577864in}{0.712052in}}%
\pgfpathlineto{\pgfqpoint{0.571327in}{0.721690in}}%
\pgfpathlineto{\pgfqpoint{0.570878in}{0.722346in}}%
\pgfpathlineto{\pgfqpoint{0.563972in}{0.732641in}}%
\pgfpathlineto{\pgfqpoint{0.560750in}{0.737689in}}%
\pgfpathlineto{\pgfqpoint{0.557375in}{0.742935in}}%
\pgfpathlineto{\pgfqpoint{0.550721in}{0.753230in}}%
\pgfpathlineto{\pgfqpoint{0.550173in}{0.754068in}}%
\pgfpathlineto{\pgfqpoint{0.543940in}{0.763524in}}%
\pgfpathlineto{\pgfqpoint{0.539596in}{0.770092in}}%
\pgfpathlineto{\pgfqpoint{0.537038in}{0.773818in}}%
\pgfpathlineto{\pgfqpoint{0.529752in}{0.784113in}}%
\pgfpathlineto{\pgfqpoint{0.529019in}{0.785145in}}%
\pgfpathlineto{\pgfqpoint{0.522371in}{0.794407in}}%
\pgfpathlineto{\pgfqpoint{0.518442in}{0.799902in}}%
\pgfpathlineto{\pgfqpoint{0.514970in}{0.804702in}}%
\pgfpathlineto{\pgfqpoint{0.507865in}{0.814576in}}%
\pgfpathlineto{\pgfqpoint{0.507558in}{0.814996in}}%
\pgfpathlineto{\pgfqpoint{0.500298in}{0.825291in}}%
\pgfpathlineto{\pgfqpoint{0.497288in}{0.829883in}}%
\pgfpathlineto{\pgfqpoint{0.497288in}{0.825291in}}%
\pgfpathlineto{\pgfqpoint{0.497288in}{0.814996in}}%
\pgfpathlineto{\pgfqpoint{0.497288in}{0.804702in}}%
\pgfpathlineto{\pgfqpoint{0.497288in}{0.803898in}}%
\pgfpathlineto{\pgfqpoint{0.504131in}{0.794407in}}%
\pgfpathlineto{\pgfqpoint{0.507865in}{0.789271in}}%
\pgfpathlineto{\pgfqpoint{0.511555in}{0.784113in}}%
\pgfpathlineto{\pgfqpoint{0.518442in}{0.774550in}}%
\pgfpathlineto{\pgfqpoint{0.518961in}{0.773818in}}%
\pgfpathlineto{\pgfqpoint{0.526244in}{0.763524in}}%
\pgfpathlineto{\pgfqpoint{0.529019in}{0.759507in}}%
\pgfpathlineto{\pgfqpoint{0.533170in}{0.753230in}}%
\pgfpathlineto{\pgfqpoint{0.539596in}{0.743446in}}%
\pgfpathlineto{\pgfqpoint{0.539926in}{0.742935in}}%
\pgfpathlineto{\pgfqpoint{0.546522in}{0.732641in}}%
\pgfpathlineto{\pgfqpoint{0.550173in}{0.726931in}}%
\pgfpathlineto{\pgfqpoint{0.553084in}{0.722346in}}%
\pgfpathlineto{\pgfqpoint{0.559953in}{0.712052in}}%
\pgfpathlineto{\pgfqpoint{0.560750in}{0.710877in}}%
\pgfpathlineto{\pgfqpoint{0.566866in}{0.701757in}}%
\pgfpathlineto{\pgfqpoint{0.571327in}{0.695097in}}%
\pgfpathlineto{\pgfqpoint{0.573736in}{0.691463in}}%
\pgfpathlineto{\pgfqpoint{0.580455in}{0.681169in}}%
\pgfpathlineto{\pgfqpoint{0.581904in}{0.678910in}}%
\pgfpathlineto{\pgfqpoint{0.586979in}{0.670874in}}%
\pgfpathlineto{\pgfqpoint{0.592481in}{0.662161in}}%
\pgfpathlineto{\pgfqpoint{0.593475in}{0.660580in}}%
\pgfpathlineto{\pgfqpoint{0.599907in}{0.650285in}}%
\pgfpathlineto{\pgfqpoint{0.603058in}{0.645235in}}%
\pgfpathlineto{\pgfqpoint{0.606317in}{0.639991in}}%
\pgfpathlineto{\pgfqpoint{0.612729in}{0.629696in}}%
\pgfpathlineto{\pgfqpoint{0.613635in}{0.628242in}}%
\pgfpathlineto{\pgfqpoint{0.619125in}{0.619402in}}%
\pgfpathlineto{\pgfqpoint{0.624212in}{0.611196in}}%
\pgfpathlineto{\pgfqpoint{0.625504in}{0.609108in}}%
\pgfpathlineto{\pgfqpoint{0.631828in}{0.598813in}}%
\pgfpathlineto{\pgfqpoint{0.634790in}{0.593977in}}%
\pgfpathlineto{\pgfqpoint{0.638123in}{0.588519in}}%
\pgfpathlineto{\pgfqpoint{0.644393in}{0.578224in}}%
\pgfpathlineto{\pgfqpoint{0.645367in}{0.576601in}}%
\pgfpathlineto{\pgfqpoint{0.650504in}{0.567930in}}%
\pgfpathlineto{\pgfqpoint{0.655944in}{0.557989in}}%
\pgfpathlineto{\pgfqpoint{0.656127in}{0.557635in}}%
\pgfpathlineto{\pgfqpoint{0.661227in}{0.547341in}}%
\pgfpathlineto{\pgfqpoint{0.666294in}{0.537047in}}%
\pgfpathlineto{\pgfqpoint{0.666521in}{0.536581in}}%
\pgfpathlineto{\pgfqpoint{0.671221in}{0.526752in}}%
\pgfpathlineto{\pgfqpoint{0.676110in}{0.516458in}}%
\pgfpathlineto{\pgfqpoint{0.677098in}{0.514356in}}%
\pgfpathlineto{\pgfqpoint{0.680793in}{0.506163in}}%
\pgfpathlineto{\pgfqpoint{0.685264in}{0.495869in}}%
\pgfpathclose%
\pgfusepath{fill}%
\end{pgfscope}%
\begin{pgfscope}%
\pgfpathrectangle{\pgfqpoint{0.423750in}{0.423750in}}{\pgfqpoint{1.194205in}{1.163386in}}%
\pgfusepath{clip}%
\pgfsetbuttcap%
\pgfsetroundjoin%
\definecolor{currentfill}{rgb}{0.717551,0.087162,0.341300}%
\pgfsetfillcolor{currentfill}%
\pgfsetlinewidth{0.000000pt}%
\definecolor{currentstroke}{rgb}{0.000000,0.000000,0.000000}%
\pgfsetstrokecolor{currentstroke}%
\pgfsetdash{}{0pt}%
\pgfpathmoveto{\pgfqpoint{1.279990in}{1.298558in}}%
\pgfpathlineto{\pgfqpoint{1.290567in}{1.298801in}}%
\pgfpathlineto{\pgfqpoint{1.290767in}{1.298834in}}%
\pgfpathlineto{\pgfqpoint{1.301144in}{1.300563in}}%
\pgfpathlineto{\pgfqpoint{1.311721in}{1.306169in}}%
\pgfpathlineto{\pgfqpoint{1.315723in}{1.309129in}}%
\pgfpathlineto{\pgfqpoint{1.322299in}{1.313798in}}%
\pgfpathlineto{\pgfqpoint{1.330516in}{1.319423in}}%
\pgfpathlineto{\pgfqpoint{1.332876in}{1.321034in}}%
\pgfpathlineto{\pgfqpoint{1.343453in}{1.328385in}}%
\pgfpathlineto{\pgfqpoint{1.345333in}{1.329718in}}%
\pgfpathlineto{\pgfqpoint{1.354030in}{1.335855in}}%
\pgfpathlineto{\pgfqpoint{1.359838in}{1.340012in}}%
\pgfpathlineto{\pgfqpoint{1.364607in}{1.343410in}}%
\pgfpathlineto{\pgfqpoint{1.374174in}{1.350307in}}%
\pgfpathlineto{\pgfqpoint{1.375184in}{1.351032in}}%
\pgfpathlineto{\pgfqpoint{1.385761in}{1.358708in}}%
\pgfpathlineto{\pgfqpoint{1.388356in}{1.360601in}}%
\pgfpathlineto{\pgfqpoint{1.396338in}{1.366430in}}%
\pgfpathlineto{\pgfqpoint{1.402312in}{1.370896in}}%
\pgfpathlineto{\pgfqpoint{1.406915in}{1.374345in}}%
\pgfpathlineto{\pgfqpoint{1.416044in}{1.381190in}}%
\pgfpathlineto{\pgfqpoint{1.417492in}{1.382279in}}%
\pgfpathlineto{\pgfqpoint{1.428069in}{1.390249in}}%
\pgfpathlineto{\pgfqpoint{1.429705in}{1.391484in}}%
\pgfpathlineto{\pgfqpoint{1.438646in}{1.398258in}}%
\pgfpathlineto{\pgfqpoint{1.443286in}{1.401779in}}%
\pgfpathlineto{\pgfqpoint{1.449223in}{1.406303in}}%
\pgfpathlineto{\pgfqpoint{1.456780in}{1.412073in}}%
\pgfpathlineto{\pgfqpoint{1.459800in}{1.414375in}}%
\pgfpathlineto{\pgfqpoint{1.470222in}{1.422368in}}%
\pgfpathlineto{\pgfqpoint{1.470377in}{1.422486in}}%
\pgfpathlineto{\pgfqpoint{1.480954in}{1.430623in}}%
\pgfpathlineto{\pgfqpoint{1.483592in}{1.432662in}}%
\pgfpathlineto{\pgfqpoint{1.491532in}{1.438843in}}%
\pgfpathlineto{\pgfqpoint{1.496784in}{1.442957in}}%
\pgfpathlineto{\pgfqpoint{1.502109in}{1.447158in}}%
\pgfpathlineto{\pgfqpoint{1.509781in}{1.453251in}}%
\pgfpathlineto{\pgfqpoint{1.512686in}{1.455576in}}%
\pgfpathlineto{\pgfqpoint{1.522571in}{1.463545in}}%
\pgfpathlineto{\pgfqpoint{1.523263in}{1.464108in}}%
\pgfpathlineto{\pgfqpoint{1.533840in}{1.472769in}}%
\pgfpathlineto{\pgfqpoint{1.535139in}{1.473840in}}%
\pgfpathlineto{\pgfqpoint{1.544417in}{1.481553in}}%
\pgfpathlineto{\pgfqpoint{1.544417in}{1.484134in}}%
\pgfpathlineto{\pgfqpoint{1.544417in}{1.494429in}}%
\pgfpathlineto{\pgfqpoint{1.544417in}{1.504723in}}%
\pgfpathlineto{\pgfqpoint{1.544417in}{1.515018in}}%
\pgfpathlineto{\pgfqpoint{1.537688in}{1.515018in}}%
\pgfpathlineto{\pgfqpoint{1.533840in}{1.511751in}}%
\pgfpathlineto{\pgfqpoint{1.525492in}{1.504723in}}%
\pgfpathlineto{\pgfqpoint{1.523263in}{1.502859in}}%
\pgfpathlineto{\pgfqpoint{1.513098in}{1.494429in}}%
\pgfpathlineto{\pgfqpoint{1.512686in}{1.494089in}}%
\pgfpathlineto{\pgfqpoint{1.502109in}{1.485454in}}%
\pgfpathlineto{\pgfqpoint{1.500478in}{1.484134in}}%
\pgfpathlineto{\pgfqpoint{1.491532in}{1.476943in}}%
\pgfpathlineto{\pgfqpoint{1.487640in}{1.473840in}}%
\pgfpathlineto{\pgfqpoint{1.480954in}{1.468545in}}%
\pgfpathlineto{\pgfqpoint{1.474593in}{1.463545in}}%
\pgfpathlineto{\pgfqpoint{1.470377in}{1.460252in}}%
\pgfpathlineto{\pgfqpoint{1.461345in}{1.453251in}}%
\pgfpathlineto{\pgfqpoint{1.459800in}{1.452061in}}%
\pgfpathlineto{\pgfqpoint{1.449223in}{1.443966in}}%
\pgfpathlineto{\pgfqpoint{1.447896in}{1.442957in}}%
\pgfpathlineto{\pgfqpoint{1.438646in}{1.435958in}}%
\pgfpathlineto{\pgfqpoint{1.434260in}{1.432662in}}%
\pgfpathlineto{\pgfqpoint{1.428069in}{1.428022in}}%
\pgfpathlineto{\pgfqpoint{1.420459in}{1.422368in}}%
\pgfpathlineto{\pgfqpoint{1.417492in}{1.420162in}}%
\pgfpathlineto{\pgfqpoint{1.406915in}{1.412393in}}%
\pgfpathlineto{\pgfqpoint{1.406470in}{1.412073in}}%
\pgfpathlineto{\pgfqpoint{1.396338in}{1.404672in}}%
\pgfpathlineto{\pgfqpoint{1.392220in}{1.401779in}}%
\pgfpathlineto{\pgfqpoint{1.385761in}{1.397211in}}%
\pgfpathlineto{\pgfqpoint{1.377576in}{1.391484in}}%
\pgfpathlineto{\pgfqpoint{1.375184in}{1.389808in}}%
\pgfpathlineto{\pgfqpoint{1.364607in}{1.382505in}}%
\pgfpathlineto{\pgfqpoint{1.362672in}{1.381190in}}%
\pgfpathlineto{\pgfqpoint{1.354030in}{1.375308in}}%
\pgfpathlineto{\pgfqpoint{1.347481in}{1.370896in}}%
\pgfpathlineto{\pgfqpoint{1.343453in}{1.368179in}}%
\pgfpathlineto{\pgfqpoint{1.332876in}{1.361137in}}%
\pgfpathlineto{\pgfqpoint{1.331997in}{1.360601in}}%
\pgfpathlineto{\pgfqpoint{1.322299in}{1.354437in}}%
\pgfpathlineto{\pgfqpoint{1.314926in}{1.350307in}}%
\pgfpathlineto{\pgfqpoint{1.311721in}{1.348456in}}%
\pgfpathlineto{\pgfqpoint{1.301144in}{1.344001in}}%
\pgfpathlineto{\pgfqpoint{1.290567in}{1.344397in}}%
\pgfpathlineto{\pgfqpoint{1.279990in}{1.346876in}}%
\pgfpathlineto{\pgfqpoint{1.269421in}{1.350307in}}%
\pgfpathlineto{\pgfqpoint{1.269413in}{1.350309in}}%
\pgfpathlineto{\pgfqpoint{1.258836in}{1.353905in}}%
\pgfpathlineto{\pgfqpoint{1.248259in}{1.357615in}}%
\pgfpathlineto{\pgfqpoint{1.241388in}{1.360601in}}%
\pgfpathlineto{\pgfqpoint{1.237682in}{1.362251in}}%
\pgfpathlineto{\pgfqpoint{1.227105in}{1.367563in}}%
\pgfpathlineto{\pgfqpoint{1.220597in}{1.370896in}}%
\pgfpathlineto{\pgfqpoint{1.216528in}{1.373004in}}%
\pgfpathlineto{\pgfqpoint{1.205951in}{1.378592in}}%
\pgfpathlineto{\pgfqpoint{1.199731in}{1.381190in}}%
\pgfpathlineto{\pgfqpoint{1.195374in}{1.382888in}}%
\pgfpathlineto{\pgfqpoint{1.184797in}{1.386138in}}%
\pgfpathlineto{\pgfqpoint{1.174220in}{1.389742in}}%
\pgfpathlineto{\pgfqpoint{1.169762in}{1.391484in}}%
\pgfpathlineto{\pgfqpoint{1.163643in}{1.393886in}}%
\pgfpathlineto{\pgfqpoint{1.153066in}{1.398776in}}%
\pgfpathlineto{\pgfqpoint{1.147178in}{1.401779in}}%
\pgfpathlineto{\pgfqpoint{1.142488in}{1.404175in}}%
\pgfpathlineto{\pgfqpoint{1.131911in}{1.411102in}}%
\pgfpathlineto{\pgfqpoint{1.130809in}{1.412073in}}%
\pgfpathlineto{\pgfqpoint{1.121334in}{1.421303in}}%
\pgfpathlineto{\pgfqpoint{1.120427in}{1.422368in}}%
\pgfpathlineto{\pgfqpoint{1.112688in}{1.432662in}}%
\pgfpathlineto{\pgfqpoint{1.110757in}{1.435440in}}%
\pgfpathlineto{\pgfqpoint{1.105611in}{1.442957in}}%
\pgfpathlineto{\pgfqpoint{1.100180in}{1.450977in}}%
\pgfpathlineto{\pgfqpoint{1.098661in}{1.453251in}}%
\pgfpathlineto{\pgfqpoint{1.091863in}{1.463545in}}%
\pgfpathlineto{\pgfqpoint{1.089603in}{1.467006in}}%
\pgfpathlineto{\pgfqpoint{1.085187in}{1.473840in}}%
\pgfpathlineto{\pgfqpoint{1.079026in}{1.483446in}}%
\pgfpathlineto{\pgfqpoint{1.078589in}{1.484134in}}%
\pgfpathlineto{\pgfqpoint{1.072110in}{1.494429in}}%
\pgfpathlineto{\pgfqpoint{1.068449in}{1.500276in}}%
\pgfpathlineto{\pgfqpoint{1.065685in}{1.504723in}}%
\pgfpathlineto{\pgfqpoint{1.059317in}{1.515018in}}%
\pgfpathlineto{\pgfqpoint{1.057872in}{1.515018in}}%
\pgfpathlineto{\pgfqpoint{1.047295in}{1.515018in}}%
\pgfpathlineto{\pgfqpoint{1.036718in}{1.515018in}}%
\pgfpathlineto{\pgfqpoint{1.026141in}{1.515018in}}%
\pgfpathlineto{\pgfqpoint{1.017630in}{1.515018in}}%
\pgfpathlineto{\pgfqpoint{1.023945in}{1.504723in}}%
\pgfpathlineto{\pgfqpoint{1.026141in}{1.501150in}}%
\pgfpathlineto{\pgfqpoint{1.030308in}{1.494429in}}%
\pgfpathlineto{\pgfqpoint{1.036695in}{1.484134in}}%
\pgfpathlineto{\pgfqpoint{1.036718in}{1.484098in}}%
\pgfpathlineto{\pgfqpoint{1.043182in}{1.473840in}}%
\pgfpathlineto{\pgfqpoint{1.047295in}{1.467330in}}%
\pgfpathlineto{\pgfqpoint{1.049709in}{1.463545in}}%
\pgfpathlineto{\pgfqpoint{1.056316in}{1.453251in}}%
\pgfpathlineto{\pgfqpoint{1.057872in}{1.450845in}}%
\pgfpathlineto{\pgfqpoint{1.063028in}{1.442957in}}%
\pgfpathlineto{\pgfqpoint{1.068449in}{1.434719in}}%
\pgfpathlineto{\pgfqpoint{1.069819in}{1.432662in}}%
\pgfpathlineto{\pgfqpoint{1.076739in}{1.422368in}}%
\pgfpathlineto{\pgfqpoint{1.079026in}{1.418996in}}%
\pgfpathlineto{\pgfqpoint{1.083781in}{1.412073in}}%
\pgfpathlineto{\pgfqpoint{1.089603in}{1.403668in}}%
\pgfpathlineto{\pgfqpoint{1.090960in}{1.401779in}}%
\pgfpathlineto{\pgfqpoint{1.099338in}{1.391484in}}%
\pgfpathlineto{\pgfqpoint{1.100180in}{1.390552in}}%
\pgfpathlineto{\pgfqpoint{1.109163in}{1.381190in}}%
\pgfpathlineto{\pgfqpoint{1.110757in}{1.379619in}}%
\pgfpathlineto{\pgfqpoint{1.120684in}{1.370896in}}%
\pgfpathlineto{\pgfqpoint{1.121334in}{1.370345in}}%
\pgfpathlineto{\pgfqpoint{1.131911in}{1.362708in}}%
\pgfpathlineto{\pgfqpoint{1.135244in}{1.360601in}}%
\pgfpathlineto{\pgfqpoint{1.142488in}{1.356020in}}%
\pgfpathlineto{\pgfqpoint{1.153066in}{1.350453in}}%
\pgfpathlineto{\pgfqpoint{1.153348in}{1.350307in}}%
\pgfpathlineto{\pgfqpoint{1.163643in}{1.344966in}}%
\pgfpathlineto{\pgfqpoint{1.174220in}{1.340157in}}%
\pgfpathlineto{\pgfqpoint{1.174593in}{1.340012in}}%
\pgfpathlineto{\pgfqpoint{1.184797in}{1.336056in}}%
\pgfpathlineto{\pgfqpoint{1.195374in}{1.331419in}}%
\pgfpathlineto{\pgfqpoint{1.198624in}{1.329718in}}%
\pgfpathlineto{\pgfqpoint{1.205951in}{1.325917in}}%
\pgfpathlineto{\pgfqpoint{1.216528in}{1.320538in}}%
\pgfpathlineto{\pgfqpoint{1.218766in}{1.319423in}}%
\pgfpathlineto{\pgfqpoint{1.227105in}{1.315157in}}%
\pgfpathlineto{\pgfqpoint{1.237682in}{1.309605in}}%
\pgfpathlineto{\pgfqpoint{1.238838in}{1.309129in}}%
\pgfpathlineto{\pgfqpoint{1.248259in}{1.305388in}}%
\pgfpathlineto{\pgfqpoint{1.258836in}{1.302129in}}%
\pgfpathlineto{\pgfqpoint{1.269413in}{1.299444in}}%
\pgfpathlineto{\pgfqpoint{1.276457in}{1.298834in}}%
\pgfpathclose%
\pgfusepath{fill}%
\end{pgfscope}%
\begin{pgfscope}%
\pgfpathrectangle{\pgfqpoint{0.423750in}{0.423750in}}{\pgfqpoint{1.194205in}{1.163386in}}%
\pgfusepath{clip}%
\pgfsetbuttcap%
\pgfsetroundjoin%
\definecolor{currentfill}{rgb}{0.796501,0.105066,0.310630}%
\pgfsetfillcolor{currentfill}%
\pgfsetlinewidth{0.000000pt}%
\definecolor{currentstroke}{rgb}{0.000000,0.000000,0.000000}%
\pgfsetstrokecolor{currentstroke}%
\pgfsetdash{}{0pt}%
\pgfpathmoveto{\pgfqpoint{0.698252in}{0.502989in}}%
\pgfpathlineto{\pgfqpoint{0.701254in}{0.495869in}}%
\pgfpathlineto{\pgfqpoint{0.708829in}{0.495869in}}%
\pgfpathlineto{\pgfqpoint{0.717353in}{0.495869in}}%
\pgfpathlineto{\pgfqpoint{0.712760in}{0.506163in}}%
\pgfpathlineto{\pgfqpoint{0.708829in}{0.514778in}}%
\pgfpathlineto{\pgfqpoint{0.708108in}{0.516458in}}%
\pgfpathlineto{\pgfqpoint{0.703665in}{0.526752in}}%
\pgfpathlineto{\pgfqpoint{0.699211in}{0.537047in}}%
\pgfpathlineto{\pgfqpoint{0.698252in}{0.539215in}}%
\pgfpathlineto{\pgfqpoint{0.694485in}{0.547341in}}%
\pgfpathlineto{\pgfqpoint{0.689592in}{0.557635in}}%
\pgfpathlineto{\pgfqpoint{0.687675in}{0.561615in}}%
\pgfpathlineto{\pgfqpoint{0.684558in}{0.567930in}}%
\pgfpathlineto{\pgfqpoint{0.679349in}{0.578224in}}%
\pgfpathlineto{\pgfqpoint{0.677098in}{0.582384in}}%
\pgfpathlineto{\pgfqpoint{0.673596in}{0.588519in}}%
\pgfpathlineto{\pgfqpoint{0.667268in}{0.598813in}}%
\pgfpathlineto{\pgfqpoint{0.666521in}{0.600007in}}%
\pgfpathlineto{\pgfqpoint{0.660822in}{0.609108in}}%
\pgfpathlineto{\pgfqpoint{0.655944in}{0.616879in}}%
\pgfpathlineto{\pgfqpoint{0.654358in}{0.619402in}}%
\pgfpathlineto{\pgfqpoint{0.647845in}{0.629696in}}%
\pgfpathlineto{\pgfqpoint{0.645367in}{0.633598in}}%
\pgfpathlineto{\pgfqpoint{0.641305in}{0.639991in}}%
\pgfpathlineto{\pgfqpoint{0.634822in}{0.650285in}}%
\pgfpathlineto{\pgfqpoint{0.634790in}{0.650337in}}%
\pgfpathlineto{\pgfqpoint{0.628283in}{0.660580in}}%
\pgfpathlineto{\pgfqpoint{0.624212in}{0.666978in}}%
\pgfpathlineto{\pgfqpoint{0.621730in}{0.670874in}}%
\pgfpathlineto{\pgfqpoint{0.615142in}{0.681169in}}%
\pgfpathlineto{\pgfqpoint{0.613635in}{0.683507in}}%
\pgfpathlineto{\pgfqpoint{0.608501in}{0.691463in}}%
\pgfpathlineto{\pgfqpoint{0.603058in}{0.699906in}}%
\pgfpathlineto{\pgfqpoint{0.601862in}{0.701757in}}%
\pgfpathlineto{\pgfqpoint{0.595171in}{0.712052in}}%
\pgfpathlineto{\pgfqpoint{0.592481in}{0.716106in}}%
\pgfpathlineto{\pgfqpoint{0.588239in}{0.722346in}}%
\pgfpathlineto{\pgfqpoint{0.581904in}{0.731553in}}%
\pgfpathlineto{\pgfqpoint{0.581151in}{0.732641in}}%
\pgfpathlineto{\pgfqpoint{0.574229in}{0.742935in}}%
\pgfpathlineto{\pgfqpoint{0.571327in}{0.747420in}}%
\pgfpathlineto{\pgfqpoint{0.567548in}{0.753230in}}%
\pgfpathlineto{\pgfqpoint{0.560821in}{0.763524in}}%
\pgfpathlineto{\pgfqpoint{0.560750in}{0.763631in}}%
\pgfpathlineto{\pgfqpoint{0.553983in}{0.773818in}}%
\pgfpathlineto{\pgfqpoint{0.550173in}{0.779542in}}%
\pgfpathlineto{\pgfqpoint{0.547039in}{0.784113in}}%
\pgfpathlineto{\pgfqpoint{0.539767in}{0.794407in}}%
\pgfpathlineto{\pgfqpoint{0.539596in}{0.794661in}}%
\pgfpathlineto{\pgfqpoint{0.532592in}{0.804702in}}%
\pgfpathlineto{\pgfqpoint{0.529019in}{0.810043in}}%
\pgfpathlineto{\pgfqpoint{0.525644in}{0.814996in}}%
\pgfpathlineto{\pgfqpoint{0.518786in}{0.825291in}}%
\pgfpathlineto{\pgfqpoint{0.518442in}{0.825828in}}%
\pgfpathlineto{\pgfqpoint{0.512131in}{0.835585in}}%
\pgfpathlineto{\pgfqpoint{0.507865in}{0.842221in}}%
\pgfpathlineto{\pgfqpoint{0.505488in}{0.845879in}}%
\pgfpathlineto{\pgfqpoint{0.499197in}{0.856174in}}%
\pgfpathlineto{\pgfqpoint{0.497288in}{0.859632in}}%
\pgfpathlineto{\pgfqpoint{0.497288in}{0.856174in}}%
\pgfpathlineto{\pgfqpoint{0.497288in}{0.845879in}}%
\pgfpathlineto{\pgfqpoint{0.497288in}{0.835585in}}%
\pgfpathlineto{\pgfqpoint{0.497288in}{0.829883in}}%
\pgfpathlineto{\pgfqpoint{0.500298in}{0.825291in}}%
\pgfpathlineto{\pgfqpoint{0.507558in}{0.814996in}}%
\pgfpathlineto{\pgfqpoint{0.507865in}{0.814576in}}%
\pgfpathlineto{\pgfqpoint{0.514970in}{0.804702in}}%
\pgfpathlineto{\pgfqpoint{0.518442in}{0.799902in}}%
\pgfpathlineto{\pgfqpoint{0.522371in}{0.794407in}}%
\pgfpathlineto{\pgfqpoint{0.529019in}{0.785145in}}%
\pgfpathlineto{\pgfqpoint{0.529752in}{0.784113in}}%
\pgfpathlineto{\pgfqpoint{0.537038in}{0.773818in}}%
\pgfpathlineto{\pgfqpoint{0.539596in}{0.770092in}}%
\pgfpathlineto{\pgfqpoint{0.543940in}{0.763524in}}%
\pgfpathlineto{\pgfqpoint{0.550173in}{0.754068in}}%
\pgfpathlineto{\pgfqpoint{0.550721in}{0.753230in}}%
\pgfpathlineto{\pgfqpoint{0.557375in}{0.742935in}}%
\pgfpathlineto{\pgfqpoint{0.560750in}{0.737689in}}%
\pgfpathlineto{\pgfqpoint{0.563972in}{0.732641in}}%
\pgfpathlineto{\pgfqpoint{0.570878in}{0.722346in}}%
\pgfpathlineto{\pgfqpoint{0.571327in}{0.721690in}}%
\pgfpathlineto{\pgfqpoint{0.577864in}{0.712052in}}%
\pgfpathlineto{\pgfqpoint{0.581904in}{0.706046in}}%
\pgfpathlineto{\pgfqpoint{0.584739in}{0.701757in}}%
\pgfpathlineto{\pgfqpoint{0.591432in}{0.691463in}}%
\pgfpathlineto{\pgfqpoint{0.592481in}{0.689826in}}%
\pgfpathlineto{\pgfqpoint{0.598008in}{0.681169in}}%
\pgfpathlineto{\pgfqpoint{0.603058in}{0.673247in}}%
\pgfpathlineto{\pgfqpoint{0.604566in}{0.670874in}}%
\pgfpathlineto{\pgfqpoint{0.611072in}{0.660580in}}%
\pgfpathlineto{\pgfqpoint{0.613635in}{0.656512in}}%
\pgfpathlineto{\pgfqpoint{0.617548in}{0.650285in}}%
\pgfpathlineto{\pgfqpoint{0.624012in}{0.639991in}}%
\pgfpathlineto{\pgfqpoint{0.624212in}{0.639671in}}%
\pgfpathlineto{\pgfqpoint{0.630461in}{0.629696in}}%
\pgfpathlineto{\pgfqpoint{0.634790in}{0.622805in}}%
\pgfpathlineto{\pgfqpoint{0.636923in}{0.619402in}}%
\pgfpathlineto{\pgfqpoint{0.643341in}{0.609108in}}%
\pgfpathlineto{\pgfqpoint{0.645367in}{0.605841in}}%
\pgfpathlineto{\pgfqpoint{0.649717in}{0.598813in}}%
\pgfpathlineto{\pgfqpoint{0.655944in}{0.588738in}}%
\pgfpathlineto{\pgfqpoint{0.656079in}{0.588519in}}%
\pgfpathlineto{\pgfqpoint{0.662309in}{0.578224in}}%
\pgfpathlineto{\pgfqpoint{0.666521in}{0.570735in}}%
\pgfpathlineto{\pgfqpoint{0.668013in}{0.567930in}}%
\pgfpathlineto{\pgfqpoint{0.673143in}{0.557635in}}%
\pgfpathlineto{\pgfqpoint{0.677098in}{0.549624in}}%
\pgfpathlineto{\pgfqpoint{0.678203in}{0.547341in}}%
\pgfpathlineto{\pgfqpoint{0.683125in}{0.537047in}}%
\pgfpathlineto{\pgfqpoint{0.687675in}{0.527448in}}%
\pgfpathlineto{\pgfqpoint{0.687990in}{0.526752in}}%
\pgfpathlineto{\pgfqpoint{0.692513in}{0.516458in}}%
\pgfpathlineto{\pgfqpoint{0.696906in}{0.506163in}}%
\pgfpathclose%
\pgfusepath{fill}%
\end{pgfscope}%
\begin{pgfscope}%
\pgfpathrectangle{\pgfqpoint{0.423750in}{0.423750in}}{\pgfqpoint{1.194205in}{1.163386in}}%
\pgfusepath{clip}%
\pgfsetbuttcap%
\pgfsetroundjoin%
\definecolor{currentfill}{rgb}{0.796501,0.105066,0.310630}%
\pgfsetfillcolor{currentfill}%
\pgfsetlinewidth{0.000000pt}%
\definecolor{currentstroke}{rgb}{0.000000,0.000000,0.000000}%
\pgfsetstrokecolor{currentstroke}%
\pgfsetdash{}{0pt}%
\pgfpathmoveto{\pgfqpoint{1.248259in}{1.257355in}}%
\pgfpathlineto{\pgfqpoint{1.258836in}{1.255686in}}%
\pgfpathlineto{\pgfqpoint{1.269413in}{1.256465in}}%
\pgfpathlineto{\pgfqpoint{1.276043in}{1.257657in}}%
\pgfpathlineto{\pgfqpoint{1.279990in}{1.258369in}}%
\pgfpathlineto{\pgfqpoint{1.290567in}{1.260470in}}%
\pgfpathlineto{\pgfqpoint{1.301144in}{1.262751in}}%
\pgfpathlineto{\pgfqpoint{1.311721in}{1.267679in}}%
\pgfpathlineto{\pgfqpoint{1.312064in}{1.267951in}}%
\pgfpathlineto{\pgfqpoint{1.322299in}{1.276074in}}%
\pgfpathlineto{\pgfqpoint{1.325015in}{1.278246in}}%
\pgfpathlineto{\pgfqpoint{1.332876in}{1.284290in}}%
\pgfpathlineto{\pgfqpoint{1.338963in}{1.288540in}}%
\pgfpathlineto{\pgfqpoint{1.343453in}{1.291726in}}%
\pgfpathlineto{\pgfqpoint{1.353675in}{1.298834in}}%
\pgfpathlineto{\pgfqpoint{1.354030in}{1.299091in}}%
\pgfpathlineto{\pgfqpoint{1.364607in}{1.306784in}}%
\pgfpathlineto{\pgfqpoint{1.367902in}{1.309129in}}%
\pgfpathlineto{\pgfqpoint{1.375184in}{1.314418in}}%
\pgfpathlineto{\pgfqpoint{1.382080in}{1.319423in}}%
\pgfpathlineto{\pgfqpoint{1.385761in}{1.322102in}}%
\pgfpathlineto{\pgfqpoint{1.396159in}{1.329718in}}%
\pgfpathlineto{\pgfqpoint{1.396338in}{1.329849in}}%
\pgfpathlineto{\pgfqpoint{1.406915in}{1.337767in}}%
\pgfpathlineto{\pgfqpoint{1.409916in}{1.340012in}}%
\pgfpathlineto{\pgfqpoint{1.417492in}{1.345696in}}%
\pgfpathlineto{\pgfqpoint{1.423643in}{1.350307in}}%
\pgfpathlineto{\pgfqpoint{1.428069in}{1.353635in}}%
\pgfpathlineto{\pgfqpoint{1.437338in}{1.360601in}}%
\pgfpathlineto{\pgfqpoint{1.438646in}{1.361588in}}%
\pgfpathlineto{\pgfqpoint{1.449223in}{1.369574in}}%
\pgfpathlineto{\pgfqpoint{1.450972in}{1.370896in}}%
\pgfpathlineto{\pgfqpoint{1.459800in}{1.377596in}}%
\pgfpathlineto{\pgfqpoint{1.464534in}{1.381190in}}%
\pgfpathlineto{\pgfqpoint{1.470377in}{1.385648in}}%
\pgfpathlineto{\pgfqpoint{1.478016in}{1.391484in}}%
\pgfpathlineto{\pgfqpoint{1.480954in}{1.393742in}}%
\pgfpathlineto{\pgfqpoint{1.491388in}{1.401779in}}%
\pgfpathlineto{\pgfqpoint{1.491532in}{1.401890in}}%
\pgfpathlineto{\pgfqpoint{1.502109in}{1.410125in}}%
\pgfpathlineto{\pgfqpoint{1.504603in}{1.412073in}}%
\pgfpathlineto{\pgfqpoint{1.512686in}{1.418434in}}%
\pgfpathlineto{\pgfqpoint{1.517665in}{1.422368in}}%
\pgfpathlineto{\pgfqpoint{1.523263in}{1.426792in}}%
\pgfpathlineto{\pgfqpoint{1.530640in}{1.432662in}}%
\pgfpathlineto{\pgfqpoint{1.533840in}{1.435218in}}%
\pgfpathlineto{\pgfqpoint{1.543482in}{1.442957in}}%
\pgfpathlineto{\pgfqpoint{1.544417in}{1.443715in}}%
\pgfpathlineto{\pgfqpoint{1.544417in}{1.453251in}}%
\pgfpathlineto{\pgfqpoint{1.544417in}{1.463545in}}%
\pgfpathlineto{\pgfqpoint{1.544417in}{1.473840in}}%
\pgfpathlineto{\pgfqpoint{1.544417in}{1.481553in}}%
\pgfpathlineto{\pgfqpoint{1.535139in}{1.473840in}}%
\pgfpathlineto{\pgfqpoint{1.533840in}{1.472769in}}%
\pgfpathlineto{\pgfqpoint{1.523263in}{1.464108in}}%
\pgfpathlineto{\pgfqpoint{1.522571in}{1.463545in}}%
\pgfpathlineto{\pgfqpoint{1.512686in}{1.455576in}}%
\pgfpathlineto{\pgfqpoint{1.509781in}{1.453251in}}%
\pgfpathlineto{\pgfqpoint{1.502109in}{1.447158in}}%
\pgfpathlineto{\pgfqpoint{1.496784in}{1.442957in}}%
\pgfpathlineto{\pgfqpoint{1.491532in}{1.438843in}}%
\pgfpathlineto{\pgfqpoint{1.483592in}{1.432662in}}%
\pgfpathlineto{\pgfqpoint{1.480954in}{1.430623in}}%
\pgfpathlineto{\pgfqpoint{1.470377in}{1.422486in}}%
\pgfpathlineto{\pgfqpoint{1.470222in}{1.422368in}}%
\pgfpathlineto{\pgfqpoint{1.459800in}{1.414375in}}%
\pgfpathlineto{\pgfqpoint{1.456780in}{1.412073in}}%
\pgfpathlineto{\pgfqpoint{1.449223in}{1.406303in}}%
\pgfpathlineto{\pgfqpoint{1.443286in}{1.401779in}}%
\pgfpathlineto{\pgfqpoint{1.438646in}{1.398258in}}%
\pgfpathlineto{\pgfqpoint{1.429705in}{1.391484in}}%
\pgfpathlineto{\pgfqpoint{1.428069in}{1.390249in}}%
\pgfpathlineto{\pgfqpoint{1.417492in}{1.382279in}}%
\pgfpathlineto{\pgfqpoint{1.416044in}{1.381190in}}%
\pgfpathlineto{\pgfqpoint{1.406915in}{1.374345in}}%
\pgfpathlineto{\pgfqpoint{1.402312in}{1.370896in}}%
\pgfpathlineto{\pgfqpoint{1.396338in}{1.366430in}}%
\pgfpathlineto{\pgfqpoint{1.388356in}{1.360601in}}%
\pgfpathlineto{\pgfqpoint{1.385761in}{1.358708in}}%
\pgfpathlineto{\pgfqpoint{1.375184in}{1.351032in}}%
\pgfpathlineto{\pgfqpoint{1.374174in}{1.350307in}}%
\pgfpathlineto{\pgfqpoint{1.364607in}{1.343410in}}%
\pgfpathlineto{\pgfqpoint{1.359838in}{1.340012in}}%
\pgfpathlineto{\pgfqpoint{1.354030in}{1.335855in}}%
\pgfpathlineto{\pgfqpoint{1.345333in}{1.329718in}}%
\pgfpathlineto{\pgfqpoint{1.343453in}{1.328385in}}%
\pgfpathlineto{\pgfqpoint{1.332876in}{1.321034in}}%
\pgfpathlineto{\pgfqpoint{1.330516in}{1.319423in}}%
\pgfpathlineto{\pgfqpoint{1.322299in}{1.313798in}}%
\pgfpathlineto{\pgfqpoint{1.315723in}{1.309129in}}%
\pgfpathlineto{\pgfqpoint{1.311721in}{1.306169in}}%
\pgfpathlineto{\pgfqpoint{1.301144in}{1.300563in}}%
\pgfpathlineto{\pgfqpoint{1.290767in}{1.298834in}}%
\pgfpathlineto{\pgfqpoint{1.290567in}{1.298801in}}%
\pgfpathlineto{\pgfqpoint{1.279990in}{1.298558in}}%
\pgfpathlineto{\pgfqpoint{1.276457in}{1.298834in}}%
\pgfpathlineto{\pgfqpoint{1.269413in}{1.299444in}}%
\pgfpathlineto{\pgfqpoint{1.258836in}{1.302129in}}%
\pgfpathlineto{\pgfqpoint{1.248259in}{1.305388in}}%
\pgfpathlineto{\pgfqpoint{1.238838in}{1.309129in}}%
\pgfpathlineto{\pgfqpoint{1.237682in}{1.309605in}}%
\pgfpathlineto{\pgfqpoint{1.227105in}{1.315157in}}%
\pgfpathlineto{\pgfqpoint{1.218766in}{1.319423in}}%
\pgfpathlineto{\pgfqpoint{1.216528in}{1.320538in}}%
\pgfpathlineto{\pgfqpoint{1.205951in}{1.325917in}}%
\pgfpathlineto{\pgfqpoint{1.198624in}{1.329718in}}%
\pgfpathlineto{\pgfqpoint{1.195374in}{1.331419in}}%
\pgfpathlineto{\pgfqpoint{1.184797in}{1.336056in}}%
\pgfpathlineto{\pgfqpoint{1.174593in}{1.340012in}}%
\pgfpathlineto{\pgfqpoint{1.174220in}{1.340157in}}%
\pgfpathlineto{\pgfqpoint{1.163643in}{1.344966in}}%
\pgfpathlineto{\pgfqpoint{1.153348in}{1.350307in}}%
\pgfpathlineto{\pgfqpoint{1.153066in}{1.350453in}}%
\pgfpathlineto{\pgfqpoint{1.142488in}{1.356020in}}%
\pgfpathlineto{\pgfqpoint{1.135244in}{1.360601in}}%
\pgfpathlineto{\pgfqpoint{1.131911in}{1.362708in}}%
\pgfpathlineto{\pgfqpoint{1.121334in}{1.370345in}}%
\pgfpathlineto{\pgfqpoint{1.120684in}{1.370896in}}%
\pgfpathlineto{\pgfqpoint{1.110757in}{1.379619in}}%
\pgfpathlineto{\pgfqpoint{1.109163in}{1.381190in}}%
\pgfpathlineto{\pgfqpoint{1.100180in}{1.390552in}}%
\pgfpathlineto{\pgfqpoint{1.099338in}{1.391484in}}%
\pgfpathlineto{\pgfqpoint{1.090960in}{1.401779in}}%
\pgfpathlineto{\pgfqpoint{1.089603in}{1.403668in}}%
\pgfpathlineto{\pgfqpoint{1.083781in}{1.412073in}}%
\pgfpathlineto{\pgfqpoint{1.079026in}{1.418996in}}%
\pgfpathlineto{\pgfqpoint{1.076739in}{1.422368in}}%
\pgfpathlineto{\pgfqpoint{1.069819in}{1.432662in}}%
\pgfpathlineto{\pgfqpoint{1.068449in}{1.434719in}}%
\pgfpathlineto{\pgfqpoint{1.063028in}{1.442957in}}%
\pgfpathlineto{\pgfqpoint{1.057872in}{1.450845in}}%
\pgfpathlineto{\pgfqpoint{1.056316in}{1.453251in}}%
\pgfpathlineto{\pgfqpoint{1.049709in}{1.463545in}}%
\pgfpathlineto{\pgfqpoint{1.047295in}{1.467330in}}%
\pgfpathlineto{\pgfqpoint{1.043182in}{1.473840in}}%
\pgfpathlineto{\pgfqpoint{1.036718in}{1.484098in}}%
\pgfpathlineto{\pgfqpoint{1.036695in}{1.484134in}}%
\pgfpathlineto{\pgfqpoint{1.030308in}{1.494429in}}%
\pgfpathlineto{\pgfqpoint{1.026141in}{1.501150in}}%
\pgfpathlineto{\pgfqpoint{1.023945in}{1.504723in}}%
\pgfpathlineto{\pgfqpoint{1.017630in}{1.515018in}}%
\pgfpathlineto{\pgfqpoint{1.015564in}{1.515018in}}%
\pgfpathlineto{\pgfqpoint{1.004987in}{1.515018in}}%
\pgfpathlineto{\pgfqpoint{0.994410in}{1.515018in}}%
\pgfpathlineto{\pgfqpoint{0.983833in}{1.515018in}}%
\pgfpathlineto{\pgfqpoint{0.973255in}{1.515018in}}%
\pgfpathlineto{\pgfqpoint{0.966137in}{1.515018in}}%
\pgfpathlineto{\pgfqpoint{0.972032in}{1.504723in}}%
\pgfpathlineto{\pgfqpoint{0.973255in}{1.502609in}}%
\pgfpathlineto{\pgfqpoint{0.978138in}{1.494429in}}%
\pgfpathlineto{\pgfqpoint{0.983833in}{1.486214in}}%
\pgfpathlineto{\pgfqpoint{0.985657in}{1.484134in}}%
\pgfpathlineto{\pgfqpoint{0.994410in}{1.474345in}}%
\pgfpathlineto{\pgfqpoint{0.994864in}{1.473840in}}%
\pgfpathlineto{\pgfqpoint{1.004175in}{1.463545in}}%
\pgfpathlineto{\pgfqpoint{1.004987in}{1.462654in}}%
\pgfpathlineto{\pgfqpoint{1.013606in}{1.453251in}}%
\pgfpathlineto{\pgfqpoint{1.015564in}{1.451129in}}%
\pgfpathlineto{\pgfqpoint{1.021945in}{1.442957in}}%
\pgfpathlineto{\pgfqpoint{1.026141in}{1.438177in}}%
\pgfpathlineto{\pgfqpoint{1.029687in}{1.432662in}}%
\pgfpathlineto{\pgfqpoint{1.036329in}{1.422368in}}%
\pgfpathlineto{\pgfqpoint{1.036718in}{1.421777in}}%
\pgfpathlineto{\pgfqpoint{1.043111in}{1.412073in}}%
\pgfpathlineto{\pgfqpoint{1.047295in}{1.405756in}}%
\pgfpathlineto{\pgfqpoint{1.049965in}{1.401779in}}%
\pgfpathlineto{\pgfqpoint{1.056918in}{1.391484in}}%
\pgfpathlineto{\pgfqpoint{1.057872in}{1.390071in}}%
\pgfpathlineto{\pgfqpoint{1.064005in}{1.381190in}}%
\pgfpathlineto{\pgfqpoint{1.068449in}{1.374725in}}%
\pgfpathlineto{\pgfqpoint{1.071360in}{1.370896in}}%
\pgfpathlineto{\pgfqpoint{1.079026in}{1.361819in}}%
\pgfpathlineto{\pgfqpoint{1.080112in}{1.360601in}}%
\pgfpathlineto{\pgfqpoint{1.089359in}{1.350307in}}%
\pgfpathlineto{\pgfqpoint{1.089603in}{1.350044in}}%
\pgfpathlineto{\pgfqpoint{1.100047in}{1.340012in}}%
\pgfpathlineto{\pgfqpoint{1.100180in}{1.339889in}}%
\pgfpathlineto{\pgfqpoint{1.110757in}{1.330642in}}%
\pgfpathlineto{\pgfqpoint{1.112010in}{1.329718in}}%
\pgfpathlineto{\pgfqpoint{1.121334in}{1.322879in}}%
\pgfpathlineto{\pgfqpoint{1.126283in}{1.319423in}}%
\pgfpathlineto{\pgfqpoint{1.131911in}{1.315513in}}%
\pgfpathlineto{\pgfqpoint{1.142046in}{1.309129in}}%
\pgfpathlineto{\pgfqpoint{1.142488in}{1.308852in}}%
\pgfpathlineto{\pgfqpoint{1.153066in}{1.303227in}}%
\pgfpathlineto{\pgfqpoint{1.161838in}{1.298834in}}%
\pgfpathlineto{\pgfqpoint{1.163643in}{1.297942in}}%
\pgfpathlineto{\pgfqpoint{1.174220in}{1.292763in}}%
\pgfpathlineto{\pgfqpoint{1.182400in}{1.288540in}}%
\pgfpathlineto{\pgfqpoint{1.184797in}{1.287199in}}%
\pgfpathlineto{\pgfqpoint{1.195374in}{1.280777in}}%
\pgfpathlineto{\pgfqpoint{1.199744in}{1.278246in}}%
\pgfpathlineto{\pgfqpoint{1.205951in}{1.274636in}}%
\pgfpathlineto{\pgfqpoint{1.216528in}{1.268806in}}%
\pgfpathlineto{\pgfqpoint{1.218245in}{1.267951in}}%
\pgfpathlineto{\pgfqpoint{1.227105in}{1.263844in}}%
\pgfpathlineto{\pgfqpoint{1.237682in}{1.260031in}}%
\pgfpathlineto{\pgfqpoint{1.247037in}{1.257657in}}%
\pgfpathclose%
\pgfusepath{fill}%
\end{pgfscope}%
\begin{pgfscope}%
\pgfpathrectangle{\pgfqpoint{0.423750in}{0.423750in}}{\pgfqpoint{1.194205in}{1.163386in}}%
\pgfusepath{clip}%
\pgfsetbuttcap%
\pgfsetroundjoin%
\definecolor{currentfill}{rgb}{0.857426,0.162258,0.276275}%
\pgfsetfillcolor{currentfill}%
\pgfsetlinewidth{0.000000pt}%
\definecolor{currentstroke}{rgb}{0.000000,0.000000,0.000000}%
\pgfsetstrokecolor{currentstroke}%
\pgfsetdash{}{0pt}%
\pgfpathmoveto{\pgfqpoint{0.719406in}{0.495869in}}%
\pgfpathlineto{\pgfqpoint{0.729983in}{0.495869in}}%
\pgfpathlineto{\pgfqpoint{0.734591in}{0.495869in}}%
\pgfpathlineto{\pgfqpoint{0.729983in}{0.505649in}}%
\pgfpathlineto{\pgfqpoint{0.729740in}{0.506163in}}%
\pgfpathlineto{\pgfqpoint{0.724830in}{0.516458in}}%
\pgfpathlineto{\pgfqpoint{0.719960in}{0.526752in}}%
\pgfpathlineto{\pgfqpoint{0.719406in}{0.527917in}}%
\pgfpathlineto{\pgfqpoint{0.715090in}{0.537047in}}%
\pgfpathlineto{\pgfqpoint{0.710136in}{0.547341in}}%
\pgfpathlineto{\pgfqpoint{0.708829in}{0.550053in}}%
\pgfpathlineto{\pgfqpoint{0.705435in}{0.557635in}}%
\pgfpathlineto{\pgfqpoint{0.700567in}{0.567930in}}%
\pgfpathlineto{\pgfqpoint{0.698252in}{0.572608in}}%
\pgfpathlineto{\pgfqpoint{0.695494in}{0.578224in}}%
\pgfpathlineto{\pgfqpoint{0.690162in}{0.588519in}}%
\pgfpathlineto{\pgfqpoint{0.687675in}{0.593096in}}%
\pgfpathlineto{\pgfqpoint{0.684399in}{0.598813in}}%
\pgfpathlineto{\pgfqpoint{0.678003in}{0.609108in}}%
\pgfpathlineto{\pgfqpoint{0.677098in}{0.610535in}}%
\pgfpathlineto{\pgfqpoint{0.671469in}{0.619402in}}%
\pgfpathlineto{\pgfqpoint{0.666521in}{0.627171in}}%
\pgfpathlineto{\pgfqpoint{0.664911in}{0.629696in}}%
\pgfpathlineto{\pgfqpoint{0.658301in}{0.639991in}}%
\pgfpathlineto{\pgfqpoint{0.655944in}{0.643691in}}%
\pgfpathlineto{\pgfqpoint{0.651743in}{0.650285in}}%
\pgfpathlineto{\pgfqpoint{0.645367in}{0.660291in}}%
\pgfpathlineto{\pgfqpoint{0.645182in}{0.660580in}}%
\pgfpathlineto{\pgfqpoint{0.638564in}{0.670874in}}%
\pgfpathlineto{\pgfqpoint{0.634790in}{0.676726in}}%
\pgfpathlineto{\pgfqpoint{0.631922in}{0.681169in}}%
\pgfpathlineto{\pgfqpoint{0.625243in}{0.691463in}}%
\pgfpathlineto{\pgfqpoint{0.624212in}{0.693046in}}%
\pgfpathlineto{\pgfqpoint{0.618535in}{0.701757in}}%
\pgfpathlineto{\pgfqpoint{0.613635in}{0.709249in}}%
\pgfpathlineto{\pgfqpoint{0.611801in}{0.712052in}}%
\pgfpathlineto{\pgfqpoint{0.605006in}{0.722346in}}%
\pgfpathlineto{\pgfqpoint{0.603058in}{0.725272in}}%
\pgfpathlineto{\pgfqpoint{0.598048in}{0.732641in}}%
\pgfpathlineto{\pgfqpoint{0.592481in}{0.740620in}}%
\pgfpathlineto{\pgfqpoint{0.590864in}{0.742935in}}%
\pgfpathlineto{\pgfqpoint{0.583937in}{0.753230in}}%
\pgfpathlineto{\pgfqpoint{0.581904in}{0.756330in}}%
\pgfpathlineto{\pgfqpoint{0.577168in}{0.763524in}}%
\pgfpathlineto{\pgfqpoint{0.571327in}{0.772359in}}%
\pgfpathlineto{\pgfqpoint{0.570358in}{0.773818in}}%
\pgfpathlineto{\pgfqpoint{0.563809in}{0.784113in}}%
\pgfpathlineto{\pgfqpoint{0.560750in}{0.789019in}}%
\pgfpathlineto{\pgfqpoint{0.557313in}{0.794407in}}%
\pgfpathlineto{\pgfqpoint{0.550546in}{0.804702in}}%
\pgfpathlineto{\pgfqpoint{0.550173in}{0.805265in}}%
\pgfpathlineto{\pgfqpoint{0.543679in}{0.814996in}}%
\pgfpathlineto{\pgfqpoint{0.539596in}{0.821125in}}%
\pgfpathlineto{\pgfqpoint{0.536800in}{0.825291in}}%
\pgfpathlineto{\pgfqpoint{0.530450in}{0.835585in}}%
\pgfpathlineto{\pgfqpoint{0.529019in}{0.838193in}}%
\pgfpathlineto{\pgfqpoint{0.524768in}{0.845879in}}%
\pgfpathlineto{\pgfqpoint{0.519103in}{0.856174in}}%
\pgfpathlineto{\pgfqpoint{0.518442in}{0.857359in}}%
\pgfpathlineto{\pgfqpoint{0.513375in}{0.866468in}}%
\pgfpathlineto{\pgfqpoint{0.507865in}{0.876422in}}%
\pgfpathlineto{\pgfqpoint{0.507674in}{0.876763in}}%
\pgfpathlineto{\pgfqpoint{0.502799in}{0.887057in}}%
\pgfpathlineto{\pgfqpoint{0.497620in}{0.897352in}}%
\pgfpathlineto{\pgfqpoint{0.497288in}{0.898223in}}%
\pgfpathlineto{\pgfqpoint{0.497288in}{0.897352in}}%
\pgfpathlineto{\pgfqpoint{0.497288in}{0.887057in}}%
\pgfpathlineto{\pgfqpoint{0.497288in}{0.876763in}}%
\pgfpathlineto{\pgfqpoint{0.497288in}{0.866468in}}%
\pgfpathlineto{\pgfqpoint{0.497288in}{0.859632in}}%
\pgfpathlineto{\pgfqpoint{0.499197in}{0.856174in}}%
\pgfpathlineto{\pgfqpoint{0.505488in}{0.845879in}}%
\pgfpathlineto{\pgfqpoint{0.507865in}{0.842221in}}%
\pgfpathlineto{\pgfqpoint{0.512131in}{0.835585in}}%
\pgfpathlineto{\pgfqpoint{0.518442in}{0.825828in}}%
\pgfpathlineto{\pgfqpoint{0.518786in}{0.825291in}}%
\pgfpathlineto{\pgfqpoint{0.525644in}{0.814996in}}%
\pgfpathlineto{\pgfqpoint{0.529019in}{0.810043in}}%
\pgfpathlineto{\pgfqpoint{0.532592in}{0.804702in}}%
\pgfpathlineto{\pgfqpoint{0.539596in}{0.794661in}}%
\pgfpathlineto{\pgfqpoint{0.539767in}{0.794407in}}%
\pgfpathlineto{\pgfqpoint{0.547039in}{0.784113in}}%
\pgfpathlineto{\pgfqpoint{0.550173in}{0.779542in}}%
\pgfpathlineto{\pgfqpoint{0.553983in}{0.773818in}}%
\pgfpathlineto{\pgfqpoint{0.560750in}{0.763631in}}%
\pgfpathlineto{\pgfqpoint{0.560821in}{0.763524in}}%
\pgfpathlineto{\pgfqpoint{0.567548in}{0.753230in}}%
\pgfpathlineto{\pgfqpoint{0.571327in}{0.747420in}}%
\pgfpathlineto{\pgfqpoint{0.574229in}{0.742935in}}%
\pgfpathlineto{\pgfqpoint{0.581151in}{0.732641in}}%
\pgfpathlineto{\pgfqpoint{0.581904in}{0.731553in}}%
\pgfpathlineto{\pgfqpoint{0.588239in}{0.722346in}}%
\pgfpathlineto{\pgfqpoint{0.592481in}{0.716106in}}%
\pgfpathlineto{\pgfqpoint{0.595171in}{0.712052in}}%
\pgfpathlineto{\pgfqpoint{0.601862in}{0.701757in}}%
\pgfpathlineto{\pgfqpoint{0.603058in}{0.699906in}}%
\pgfpathlineto{\pgfqpoint{0.608501in}{0.691463in}}%
\pgfpathlineto{\pgfqpoint{0.613635in}{0.683507in}}%
\pgfpathlineto{\pgfqpoint{0.615142in}{0.681169in}}%
\pgfpathlineto{\pgfqpoint{0.621730in}{0.670874in}}%
\pgfpathlineto{\pgfqpoint{0.624212in}{0.666978in}}%
\pgfpathlineto{\pgfqpoint{0.628283in}{0.660580in}}%
\pgfpathlineto{\pgfqpoint{0.634790in}{0.650337in}}%
\pgfpathlineto{\pgfqpoint{0.634822in}{0.650285in}}%
\pgfpathlineto{\pgfqpoint{0.641305in}{0.639991in}}%
\pgfpathlineto{\pgfqpoint{0.645367in}{0.633598in}}%
\pgfpathlineto{\pgfqpoint{0.647845in}{0.629696in}}%
\pgfpathlineto{\pgfqpoint{0.654358in}{0.619402in}}%
\pgfpathlineto{\pgfqpoint{0.655944in}{0.616879in}}%
\pgfpathlineto{\pgfqpoint{0.660822in}{0.609108in}}%
\pgfpathlineto{\pgfqpoint{0.666521in}{0.600007in}}%
\pgfpathlineto{\pgfqpoint{0.667268in}{0.598813in}}%
\pgfpathlineto{\pgfqpoint{0.673596in}{0.588519in}}%
\pgfpathlineto{\pgfqpoint{0.677098in}{0.582384in}}%
\pgfpathlineto{\pgfqpoint{0.679349in}{0.578224in}}%
\pgfpathlineto{\pgfqpoint{0.684558in}{0.567930in}}%
\pgfpathlineto{\pgfqpoint{0.687675in}{0.561615in}}%
\pgfpathlineto{\pgfqpoint{0.689592in}{0.557635in}}%
\pgfpathlineto{\pgfqpoint{0.694485in}{0.547341in}}%
\pgfpathlineto{\pgfqpoint{0.698252in}{0.539215in}}%
\pgfpathlineto{\pgfqpoint{0.699211in}{0.537047in}}%
\pgfpathlineto{\pgfqpoint{0.703665in}{0.526752in}}%
\pgfpathlineto{\pgfqpoint{0.708108in}{0.516458in}}%
\pgfpathlineto{\pgfqpoint{0.708829in}{0.514778in}}%
\pgfpathlineto{\pgfqpoint{0.712760in}{0.506163in}}%
\pgfpathlineto{\pgfqpoint{0.717353in}{0.495869in}}%
\pgfpathclose%
\pgfusepath{fill}%
\end{pgfscope}%
\begin{pgfscope}%
\pgfpathrectangle{\pgfqpoint{0.423750in}{0.423750in}}{\pgfqpoint{1.194205in}{1.163386in}}%
\pgfusepath{clip}%
\pgfsetbuttcap%
\pgfsetroundjoin%
\definecolor{currentfill}{rgb}{0.857426,0.162258,0.276275}%
\pgfsetfillcolor{currentfill}%
\pgfsetlinewidth{0.000000pt}%
\definecolor{currentstroke}{rgb}{0.000000,0.000000,0.000000}%
\pgfsetstrokecolor{currentstroke}%
\pgfsetdash{}{0pt}%
\pgfpathmoveto{\pgfqpoint{1.237682in}{1.215648in}}%
\pgfpathlineto{\pgfqpoint{1.248259in}{1.216031in}}%
\pgfpathlineto{\pgfqpoint{1.250755in}{1.216479in}}%
\pgfpathlineto{\pgfqpoint{1.258836in}{1.217975in}}%
\pgfpathlineto{\pgfqpoint{1.269413in}{1.220173in}}%
\pgfpathlineto{\pgfqpoint{1.279990in}{1.222558in}}%
\pgfpathlineto{\pgfqpoint{1.290567in}{1.224918in}}%
\pgfpathlineto{\pgfqpoint{1.298585in}{1.226773in}}%
\pgfpathlineto{\pgfqpoint{1.301144in}{1.227369in}}%
\pgfpathlineto{\pgfqpoint{1.311721in}{1.231593in}}%
\pgfpathlineto{\pgfqpoint{1.318613in}{1.237068in}}%
\pgfpathlineto{\pgfqpoint{1.322299in}{1.240006in}}%
\pgfpathlineto{\pgfqpoint{1.331485in}{1.247362in}}%
\pgfpathlineto{\pgfqpoint{1.332876in}{1.248479in}}%
\pgfpathlineto{\pgfqpoint{1.343453in}{1.257018in}}%
\pgfpathlineto{\pgfqpoint{1.344312in}{1.257657in}}%
\pgfpathlineto{\pgfqpoint{1.354030in}{1.264571in}}%
\pgfpathlineto{\pgfqpoint{1.358871in}{1.267951in}}%
\pgfpathlineto{\pgfqpoint{1.364607in}{1.271968in}}%
\pgfpathlineto{\pgfqpoint{1.373587in}{1.278246in}}%
\pgfpathlineto{\pgfqpoint{1.375184in}{1.279366in}}%
\pgfpathlineto{\pgfqpoint{1.385761in}{1.286791in}}%
\pgfpathlineto{\pgfqpoint{1.388223in}{1.288540in}}%
\pgfpathlineto{\pgfqpoint{1.396338in}{1.294320in}}%
\pgfpathlineto{\pgfqpoint{1.402609in}{1.298834in}}%
\pgfpathlineto{\pgfqpoint{1.406915in}{1.301946in}}%
\pgfpathlineto{\pgfqpoint{1.416884in}{1.309129in}}%
\pgfpathlineto{\pgfqpoint{1.417492in}{1.309572in}}%
\pgfpathlineto{\pgfqpoint{1.428069in}{1.316657in}}%
\pgfpathlineto{\pgfqpoint{1.432346in}{1.319423in}}%
\pgfpathlineto{\pgfqpoint{1.438646in}{1.323723in}}%
\pgfpathlineto{\pgfqpoint{1.448075in}{1.329718in}}%
\pgfpathlineto{\pgfqpoint{1.449223in}{1.330497in}}%
\pgfpathlineto{\pgfqpoint{1.459800in}{1.337014in}}%
\pgfpathlineto{\pgfqpoint{1.464752in}{1.340012in}}%
\pgfpathlineto{\pgfqpoint{1.470377in}{1.343446in}}%
\pgfpathlineto{\pgfqpoint{1.480954in}{1.350014in}}%
\pgfpathlineto{\pgfqpoint{1.481409in}{1.350307in}}%
\pgfpathlineto{\pgfqpoint{1.491532in}{1.356793in}}%
\pgfpathlineto{\pgfqpoint{1.497297in}{1.360601in}}%
\pgfpathlineto{\pgfqpoint{1.502109in}{1.363767in}}%
\pgfpathlineto{\pgfqpoint{1.512593in}{1.370896in}}%
\pgfpathlineto{\pgfqpoint{1.512686in}{1.370958in}}%
\pgfpathlineto{\pgfqpoint{1.523263in}{1.378420in}}%
\pgfpathlineto{\pgfqpoint{1.527064in}{1.381190in}}%
\pgfpathlineto{\pgfqpoint{1.533840in}{1.386115in}}%
\pgfpathlineto{\pgfqpoint{1.541006in}{1.391484in}}%
\pgfpathlineto{\pgfqpoint{1.544417in}{1.394036in}}%
\pgfpathlineto{\pgfqpoint{1.544417in}{1.401779in}}%
\pgfpathlineto{\pgfqpoint{1.544417in}{1.412073in}}%
\pgfpathlineto{\pgfqpoint{1.544417in}{1.422368in}}%
\pgfpathlineto{\pgfqpoint{1.544417in}{1.432662in}}%
\pgfpathlineto{\pgfqpoint{1.544417in}{1.442957in}}%
\pgfpathlineto{\pgfqpoint{1.544417in}{1.443715in}}%
\pgfpathlineto{\pgfqpoint{1.543482in}{1.442957in}}%
\pgfpathlineto{\pgfqpoint{1.533840in}{1.435218in}}%
\pgfpathlineto{\pgfqpoint{1.530640in}{1.432662in}}%
\pgfpathlineto{\pgfqpoint{1.523263in}{1.426792in}}%
\pgfpathlineto{\pgfqpoint{1.517665in}{1.422368in}}%
\pgfpathlineto{\pgfqpoint{1.512686in}{1.418434in}}%
\pgfpathlineto{\pgfqpoint{1.504603in}{1.412073in}}%
\pgfpathlineto{\pgfqpoint{1.502109in}{1.410125in}}%
\pgfpathlineto{\pgfqpoint{1.491532in}{1.401890in}}%
\pgfpathlineto{\pgfqpoint{1.491388in}{1.401779in}}%
\pgfpathlineto{\pgfqpoint{1.480954in}{1.393742in}}%
\pgfpathlineto{\pgfqpoint{1.478016in}{1.391484in}}%
\pgfpathlineto{\pgfqpoint{1.470377in}{1.385648in}}%
\pgfpathlineto{\pgfqpoint{1.464534in}{1.381190in}}%
\pgfpathlineto{\pgfqpoint{1.459800in}{1.377596in}}%
\pgfpathlineto{\pgfqpoint{1.450972in}{1.370896in}}%
\pgfpathlineto{\pgfqpoint{1.449223in}{1.369574in}}%
\pgfpathlineto{\pgfqpoint{1.438646in}{1.361588in}}%
\pgfpathlineto{\pgfqpoint{1.437338in}{1.360601in}}%
\pgfpathlineto{\pgfqpoint{1.428069in}{1.353635in}}%
\pgfpathlineto{\pgfqpoint{1.423643in}{1.350307in}}%
\pgfpathlineto{\pgfqpoint{1.417492in}{1.345696in}}%
\pgfpathlineto{\pgfqpoint{1.409916in}{1.340012in}}%
\pgfpathlineto{\pgfqpoint{1.406915in}{1.337767in}}%
\pgfpathlineto{\pgfqpoint{1.396338in}{1.329849in}}%
\pgfpathlineto{\pgfqpoint{1.396159in}{1.329718in}}%
\pgfpathlineto{\pgfqpoint{1.385761in}{1.322102in}}%
\pgfpathlineto{\pgfqpoint{1.382080in}{1.319423in}}%
\pgfpathlineto{\pgfqpoint{1.375184in}{1.314418in}}%
\pgfpathlineto{\pgfqpoint{1.367902in}{1.309129in}}%
\pgfpathlineto{\pgfqpoint{1.364607in}{1.306784in}}%
\pgfpathlineto{\pgfqpoint{1.354030in}{1.299091in}}%
\pgfpathlineto{\pgfqpoint{1.353675in}{1.298834in}}%
\pgfpathlineto{\pgfqpoint{1.343453in}{1.291726in}}%
\pgfpathlineto{\pgfqpoint{1.338963in}{1.288540in}}%
\pgfpathlineto{\pgfqpoint{1.332876in}{1.284290in}}%
\pgfpathlineto{\pgfqpoint{1.325015in}{1.278246in}}%
\pgfpathlineto{\pgfqpoint{1.322299in}{1.276074in}}%
\pgfpathlineto{\pgfqpoint{1.312064in}{1.267951in}}%
\pgfpathlineto{\pgfqpoint{1.311721in}{1.267679in}}%
\pgfpathlineto{\pgfqpoint{1.301144in}{1.262751in}}%
\pgfpathlineto{\pgfqpoint{1.290567in}{1.260470in}}%
\pgfpathlineto{\pgfqpoint{1.279990in}{1.258369in}}%
\pgfpathlineto{\pgfqpoint{1.276043in}{1.257657in}}%
\pgfpathlineto{\pgfqpoint{1.269413in}{1.256465in}}%
\pgfpathlineto{\pgfqpoint{1.258836in}{1.255686in}}%
\pgfpathlineto{\pgfqpoint{1.248259in}{1.257355in}}%
\pgfpathlineto{\pgfqpoint{1.247037in}{1.257657in}}%
\pgfpathlineto{\pgfqpoint{1.237682in}{1.260031in}}%
\pgfpathlineto{\pgfqpoint{1.227105in}{1.263844in}}%
\pgfpathlineto{\pgfqpoint{1.218245in}{1.267951in}}%
\pgfpathlineto{\pgfqpoint{1.216528in}{1.268806in}}%
\pgfpathlineto{\pgfqpoint{1.205951in}{1.274636in}}%
\pgfpathlineto{\pgfqpoint{1.199744in}{1.278246in}}%
\pgfpathlineto{\pgfqpoint{1.195374in}{1.280777in}}%
\pgfpathlineto{\pgfqpoint{1.184797in}{1.287199in}}%
\pgfpathlineto{\pgfqpoint{1.182400in}{1.288540in}}%
\pgfpathlineto{\pgfqpoint{1.174220in}{1.292763in}}%
\pgfpathlineto{\pgfqpoint{1.163643in}{1.297942in}}%
\pgfpathlineto{\pgfqpoint{1.161838in}{1.298834in}}%
\pgfpathlineto{\pgfqpoint{1.153066in}{1.303227in}}%
\pgfpathlineto{\pgfqpoint{1.142488in}{1.308852in}}%
\pgfpathlineto{\pgfqpoint{1.142046in}{1.309129in}}%
\pgfpathlineto{\pgfqpoint{1.131911in}{1.315513in}}%
\pgfpathlineto{\pgfqpoint{1.126283in}{1.319423in}}%
\pgfpathlineto{\pgfqpoint{1.121334in}{1.322879in}}%
\pgfpathlineto{\pgfqpoint{1.112010in}{1.329718in}}%
\pgfpathlineto{\pgfqpoint{1.110757in}{1.330642in}}%
\pgfpathlineto{\pgfqpoint{1.100180in}{1.339889in}}%
\pgfpathlineto{\pgfqpoint{1.100047in}{1.340012in}}%
\pgfpathlineto{\pgfqpoint{1.089603in}{1.350044in}}%
\pgfpathlineto{\pgfqpoint{1.089359in}{1.350307in}}%
\pgfpathlineto{\pgfqpoint{1.080112in}{1.360601in}}%
\pgfpathlineto{\pgfqpoint{1.079026in}{1.361819in}}%
\pgfpathlineto{\pgfqpoint{1.071360in}{1.370896in}}%
\pgfpathlineto{\pgfqpoint{1.068449in}{1.374725in}}%
\pgfpathlineto{\pgfqpoint{1.064005in}{1.381190in}}%
\pgfpathlineto{\pgfqpoint{1.057872in}{1.390071in}}%
\pgfpathlineto{\pgfqpoint{1.056918in}{1.391484in}}%
\pgfpathlineto{\pgfqpoint{1.049965in}{1.401779in}}%
\pgfpathlineto{\pgfqpoint{1.047295in}{1.405756in}}%
\pgfpathlineto{\pgfqpoint{1.043111in}{1.412073in}}%
\pgfpathlineto{\pgfqpoint{1.036718in}{1.421777in}}%
\pgfpathlineto{\pgfqpoint{1.036329in}{1.422368in}}%
\pgfpathlineto{\pgfqpoint{1.029687in}{1.432662in}}%
\pgfpathlineto{\pgfqpoint{1.026141in}{1.438177in}}%
\pgfpathlineto{\pgfqpoint{1.021945in}{1.442957in}}%
\pgfpathlineto{\pgfqpoint{1.015564in}{1.451129in}}%
\pgfpathlineto{\pgfqpoint{1.013606in}{1.453251in}}%
\pgfpathlineto{\pgfqpoint{1.004987in}{1.462654in}}%
\pgfpathlineto{\pgfqpoint{1.004175in}{1.463545in}}%
\pgfpathlineto{\pgfqpoint{0.994864in}{1.473840in}}%
\pgfpathlineto{\pgfqpoint{0.994410in}{1.474345in}}%
\pgfpathlineto{\pgfqpoint{0.985657in}{1.484134in}}%
\pgfpathlineto{\pgfqpoint{0.983833in}{1.486214in}}%
\pgfpathlineto{\pgfqpoint{0.978138in}{1.494429in}}%
\pgfpathlineto{\pgfqpoint{0.973255in}{1.502609in}}%
\pgfpathlineto{\pgfqpoint{0.972032in}{1.504723in}}%
\pgfpathlineto{\pgfqpoint{0.966137in}{1.515018in}}%
\pgfpathlineto{\pgfqpoint{0.962678in}{1.515018in}}%
\pgfpathlineto{\pgfqpoint{0.952101in}{1.515018in}}%
\pgfpathlineto{\pgfqpoint{0.941524in}{1.515018in}}%
\pgfpathlineto{\pgfqpoint{0.930947in}{1.515018in}}%
\pgfpathlineto{\pgfqpoint{0.921909in}{1.515018in}}%
\pgfpathlineto{\pgfqpoint{0.926835in}{1.504723in}}%
\pgfpathlineto{\pgfqpoint{0.930947in}{1.496439in}}%
\pgfpathlineto{\pgfqpoint{0.932006in}{1.494429in}}%
\pgfpathlineto{\pgfqpoint{0.937575in}{1.484134in}}%
\pgfpathlineto{\pgfqpoint{0.941524in}{1.476875in}}%
\pgfpathlineto{\pgfqpoint{0.943219in}{1.473840in}}%
\pgfpathlineto{\pgfqpoint{0.948933in}{1.463545in}}%
\pgfpathlineto{\pgfqpoint{0.952101in}{1.457836in}}%
\pgfpathlineto{\pgfqpoint{0.954737in}{1.453251in}}%
\pgfpathlineto{\pgfqpoint{0.961521in}{1.442957in}}%
\pgfpathlineto{\pgfqpoint{0.962678in}{1.441433in}}%
\pgfpathlineto{\pgfqpoint{0.970518in}{1.432662in}}%
\pgfpathlineto{\pgfqpoint{0.973255in}{1.429616in}}%
\pgfpathlineto{\pgfqpoint{0.979813in}{1.422368in}}%
\pgfpathlineto{\pgfqpoint{0.983833in}{1.417951in}}%
\pgfpathlineto{\pgfqpoint{0.989223in}{1.412073in}}%
\pgfpathlineto{\pgfqpoint{0.994410in}{1.406455in}}%
\pgfpathlineto{\pgfqpoint{0.998765in}{1.401779in}}%
\pgfpathlineto{\pgfqpoint{1.004987in}{1.395145in}}%
\pgfpathlineto{\pgfqpoint{1.008452in}{1.391484in}}%
\pgfpathlineto{\pgfqpoint{1.015564in}{1.384026in}}%
\pgfpathlineto{\pgfqpoint{1.018295in}{1.381190in}}%
\pgfpathlineto{\pgfqpoint{1.026141in}{1.373139in}}%
\pgfpathlineto{\pgfqpoint{1.028123in}{1.370896in}}%
\pgfpathlineto{\pgfqpoint{1.036718in}{1.362144in}}%
\pgfpathlineto{\pgfqpoint{1.037755in}{1.360601in}}%
\pgfpathlineto{\pgfqpoint{1.044508in}{1.350307in}}%
\pgfpathlineto{\pgfqpoint{1.047295in}{1.346515in}}%
\pgfpathlineto{\pgfqpoint{1.052375in}{1.340012in}}%
\pgfpathlineto{\pgfqpoint{1.057872in}{1.333408in}}%
\pgfpathlineto{\pgfqpoint{1.061053in}{1.329718in}}%
\pgfpathlineto{\pgfqpoint{1.068449in}{1.321246in}}%
\pgfpathlineto{\pgfqpoint{1.070114in}{1.319423in}}%
\pgfpathlineto{\pgfqpoint{1.079026in}{1.310143in}}%
\pgfpathlineto{\pgfqpoint{1.080102in}{1.309129in}}%
\pgfpathlineto{\pgfqpoint{1.089603in}{1.300232in}}%
\pgfpathlineto{\pgfqpoint{1.091151in}{1.298834in}}%
\pgfpathlineto{\pgfqpoint{1.100180in}{1.290993in}}%
\pgfpathlineto{\pgfqpoint{1.103465in}{1.288540in}}%
\pgfpathlineto{\pgfqpoint{1.110757in}{1.283272in}}%
\pgfpathlineto{\pgfqpoint{1.117650in}{1.278246in}}%
\pgfpathlineto{\pgfqpoint{1.121334in}{1.275606in}}%
\pgfpathlineto{\pgfqpoint{1.131911in}{1.268374in}}%
\pgfpathlineto{\pgfqpoint{1.132560in}{1.267951in}}%
\pgfpathlineto{\pgfqpoint{1.142488in}{1.261380in}}%
\pgfpathlineto{\pgfqpoint{1.148785in}{1.257657in}}%
\pgfpathlineto{\pgfqpoint{1.153066in}{1.255174in}}%
\pgfpathlineto{\pgfqpoint{1.163643in}{1.249610in}}%
\pgfpathlineto{\pgfqpoint{1.167268in}{1.247362in}}%
\pgfpathlineto{\pgfqpoint{1.174220in}{1.243122in}}%
\pgfpathlineto{\pgfqpoint{1.184712in}{1.237068in}}%
\pgfpathlineto{\pgfqpoint{1.184797in}{1.237019in}}%
\pgfpathlineto{\pgfqpoint{1.195374in}{1.231549in}}%
\pgfpathlineto{\pgfqpoint{1.205951in}{1.226995in}}%
\pgfpathlineto{\pgfqpoint{1.206491in}{1.226773in}}%
\pgfpathlineto{\pgfqpoint{1.216528in}{1.222697in}}%
\pgfpathlineto{\pgfqpoint{1.227105in}{1.218582in}}%
\pgfpathlineto{\pgfqpoint{1.234482in}{1.216479in}}%
\pgfpathclose%
\pgfusepath{fill}%
\end{pgfscope}%
\begin{pgfscope}%
\pgfpathrectangle{\pgfqpoint{0.423750in}{0.423750in}}{\pgfqpoint{1.194205in}{1.163386in}}%
\pgfusepath{clip}%
\pgfsetbuttcap%
\pgfsetroundjoin%
\definecolor{currentfill}{rgb}{0.905301,0.238545,0.247481}%
\pgfsetfillcolor{currentfill}%
\pgfsetlinewidth{0.000000pt}%
\definecolor{currentstroke}{rgb}{0.000000,0.000000,0.000000}%
\pgfsetstrokecolor{currentstroke}%
\pgfsetdash{}{0pt}%
\pgfpathmoveto{\pgfqpoint{0.729983in}{0.505649in}}%
\pgfpathlineto{\pgfqpoint{0.734591in}{0.495869in}}%
\pgfpathlineto{\pgfqpoint{0.740560in}{0.495869in}}%
\pgfpathlineto{\pgfqpoint{0.751137in}{0.495869in}}%
\pgfpathlineto{\pgfqpoint{0.756517in}{0.495869in}}%
\pgfpathlineto{\pgfqpoint{0.751137in}{0.503618in}}%
\pgfpathlineto{\pgfqpoint{0.749603in}{0.506163in}}%
\pgfpathlineto{\pgfqpoint{0.742656in}{0.516458in}}%
\pgfpathlineto{\pgfqpoint{0.740560in}{0.519339in}}%
\pgfpathlineto{\pgfqpoint{0.737063in}{0.526752in}}%
\pgfpathlineto{\pgfqpoint{0.732125in}{0.537047in}}%
\pgfpathlineto{\pgfqpoint{0.729983in}{0.541141in}}%
\pgfpathlineto{\pgfqpoint{0.726988in}{0.547341in}}%
\pgfpathlineto{\pgfqpoint{0.721989in}{0.557635in}}%
\pgfpathlineto{\pgfqpoint{0.719406in}{0.562921in}}%
\pgfpathlineto{\pgfqpoint{0.716943in}{0.567930in}}%
\pgfpathlineto{\pgfqpoint{0.711774in}{0.578224in}}%
\pgfpathlineto{\pgfqpoint{0.708829in}{0.583957in}}%
\pgfpathlineto{\pgfqpoint{0.706509in}{0.588519in}}%
\pgfpathlineto{\pgfqpoint{0.700743in}{0.598813in}}%
\pgfpathlineto{\pgfqpoint{0.698252in}{0.603001in}}%
\pgfpathlineto{\pgfqpoint{0.694738in}{0.609108in}}%
\pgfpathlineto{\pgfqpoint{0.688309in}{0.619402in}}%
\pgfpathlineto{\pgfqpoint{0.687675in}{0.620388in}}%
\pgfpathlineto{\pgfqpoint{0.681681in}{0.629696in}}%
\pgfpathlineto{\pgfqpoint{0.677098in}{0.636783in}}%
\pgfpathlineto{\pgfqpoint{0.675022in}{0.639991in}}%
\pgfpathlineto{\pgfqpoint{0.668394in}{0.650285in}}%
\pgfpathlineto{\pgfqpoint{0.666521in}{0.653200in}}%
\pgfpathlineto{\pgfqpoint{0.661769in}{0.660580in}}%
\pgfpathlineto{\pgfqpoint{0.655944in}{0.669606in}}%
\pgfpathlineto{\pgfqpoint{0.655124in}{0.670874in}}%
\pgfpathlineto{\pgfqpoint{0.648422in}{0.681169in}}%
\pgfpathlineto{\pgfqpoint{0.645367in}{0.685834in}}%
\pgfpathlineto{\pgfqpoint{0.641679in}{0.691463in}}%
\pgfpathlineto{\pgfqpoint{0.634937in}{0.701757in}}%
\pgfpathlineto{\pgfqpoint{0.634790in}{0.701981in}}%
\pgfpathlineto{\pgfqpoint{0.628124in}{0.712052in}}%
\pgfpathlineto{\pgfqpoint{0.624212in}{0.717926in}}%
\pgfpathlineto{\pgfqpoint{0.621269in}{0.722346in}}%
\pgfpathlineto{\pgfqpoint{0.614366in}{0.732641in}}%
\pgfpathlineto{\pgfqpoint{0.613635in}{0.733723in}}%
\pgfpathlineto{\pgfqpoint{0.607346in}{0.742935in}}%
\pgfpathlineto{\pgfqpoint{0.603058in}{0.749241in}}%
\pgfpathlineto{\pgfqpoint{0.600278in}{0.753230in}}%
\pgfpathlineto{\pgfqpoint{0.593694in}{0.763524in}}%
\pgfpathlineto{\pgfqpoint{0.592481in}{0.765496in}}%
\pgfpathlineto{\pgfqpoint{0.587343in}{0.773818in}}%
\pgfpathlineto{\pgfqpoint{0.581904in}{0.782595in}}%
\pgfpathlineto{\pgfqpoint{0.580960in}{0.784113in}}%
\pgfpathlineto{\pgfqpoint{0.574523in}{0.794407in}}%
\pgfpathlineto{\pgfqpoint{0.571327in}{0.799505in}}%
\pgfpathlineto{\pgfqpoint{0.568013in}{0.804702in}}%
\pgfpathlineto{\pgfqpoint{0.561351in}{0.814996in}}%
\pgfpathlineto{\pgfqpoint{0.560750in}{0.816070in}}%
\pgfpathlineto{\pgfqpoint{0.555675in}{0.825291in}}%
\pgfpathlineto{\pgfqpoint{0.550173in}{0.835401in}}%
\pgfpathlineto{\pgfqpoint{0.550074in}{0.835585in}}%
\pgfpathlineto{\pgfqpoint{0.546052in}{0.845879in}}%
\pgfpathlineto{\pgfqpoint{0.541900in}{0.856174in}}%
\pgfpathlineto{\pgfqpoint{0.539596in}{0.862371in}}%
\pgfpathlineto{\pgfqpoint{0.538065in}{0.866468in}}%
\pgfpathlineto{\pgfqpoint{0.534194in}{0.876763in}}%
\pgfpathlineto{\pgfqpoint{0.530305in}{0.887057in}}%
\pgfpathlineto{\pgfqpoint{0.529019in}{0.890442in}}%
\pgfpathlineto{\pgfqpoint{0.526381in}{0.897352in}}%
\pgfpathlineto{\pgfqpoint{0.522434in}{0.907646in}}%
\pgfpathlineto{\pgfqpoint{0.518473in}{0.917941in}}%
\pgfpathlineto{\pgfqpoint{0.518442in}{0.918022in}}%
\pgfpathlineto{\pgfqpoint{0.514474in}{0.928235in}}%
\pgfpathlineto{\pgfqpoint{0.510465in}{0.938529in}}%
\pgfpathlineto{\pgfqpoint{0.507865in}{0.945185in}}%
\pgfpathlineto{\pgfqpoint{0.506438in}{0.948824in}}%
\pgfpathlineto{\pgfqpoint{0.502376in}{0.959118in}}%
\pgfpathlineto{\pgfqpoint{0.498180in}{0.969413in}}%
\pgfpathlineto{\pgfqpoint{0.497288in}{0.971552in}}%
\pgfpathlineto{\pgfqpoint{0.497288in}{0.969413in}}%
\pgfpathlineto{\pgfqpoint{0.497288in}{0.959118in}}%
\pgfpathlineto{\pgfqpoint{0.497288in}{0.948824in}}%
\pgfpathlineto{\pgfqpoint{0.497288in}{0.938529in}}%
\pgfpathlineto{\pgfqpoint{0.497288in}{0.928235in}}%
\pgfpathlineto{\pgfqpoint{0.497288in}{0.917941in}}%
\pgfpathlineto{\pgfqpoint{0.497288in}{0.907646in}}%
\pgfpathlineto{\pgfqpoint{0.497288in}{0.898223in}}%
\pgfpathlineto{\pgfqpoint{0.497620in}{0.897352in}}%
\pgfpathlineto{\pgfqpoint{0.502799in}{0.887057in}}%
\pgfpathlineto{\pgfqpoint{0.507674in}{0.876763in}}%
\pgfpathlineto{\pgfqpoint{0.507865in}{0.876422in}}%
\pgfpathlineto{\pgfqpoint{0.513375in}{0.866468in}}%
\pgfpathlineto{\pgfqpoint{0.518442in}{0.857359in}}%
\pgfpathlineto{\pgfqpoint{0.519103in}{0.856174in}}%
\pgfpathlineto{\pgfqpoint{0.524768in}{0.845879in}}%
\pgfpathlineto{\pgfqpoint{0.529019in}{0.838193in}}%
\pgfpathlineto{\pgfqpoint{0.530450in}{0.835585in}}%
\pgfpathlineto{\pgfqpoint{0.536800in}{0.825291in}}%
\pgfpathlineto{\pgfqpoint{0.539596in}{0.821125in}}%
\pgfpathlineto{\pgfqpoint{0.543679in}{0.814996in}}%
\pgfpathlineto{\pgfqpoint{0.550173in}{0.805265in}}%
\pgfpathlineto{\pgfqpoint{0.550546in}{0.804702in}}%
\pgfpathlineto{\pgfqpoint{0.557313in}{0.794407in}}%
\pgfpathlineto{\pgfqpoint{0.560750in}{0.789019in}}%
\pgfpathlineto{\pgfqpoint{0.563809in}{0.784113in}}%
\pgfpathlineto{\pgfqpoint{0.570358in}{0.773818in}}%
\pgfpathlineto{\pgfqpoint{0.571327in}{0.772359in}}%
\pgfpathlineto{\pgfqpoint{0.577168in}{0.763524in}}%
\pgfpathlineto{\pgfqpoint{0.581904in}{0.756330in}}%
\pgfpathlineto{\pgfqpoint{0.583937in}{0.753230in}}%
\pgfpathlineto{\pgfqpoint{0.590864in}{0.742935in}}%
\pgfpathlineto{\pgfqpoint{0.592481in}{0.740620in}}%
\pgfpathlineto{\pgfqpoint{0.598048in}{0.732641in}}%
\pgfpathlineto{\pgfqpoint{0.603058in}{0.725272in}}%
\pgfpathlineto{\pgfqpoint{0.605006in}{0.722346in}}%
\pgfpathlineto{\pgfqpoint{0.611801in}{0.712052in}}%
\pgfpathlineto{\pgfqpoint{0.613635in}{0.709249in}}%
\pgfpathlineto{\pgfqpoint{0.618535in}{0.701757in}}%
\pgfpathlineto{\pgfqpoint{0.624212in}{0.693046in}}%
\pgfpathlineto{\pgfqpoint{0.625243in}{0.691463in}}%
\pgfpathlineto{\pgfqpoint{0.631922in}{0.681169in}}%
\pgfpathlineto{\pgfqpoint{0.634790in}{0.676726in}}%
\pgfpathlineto{\pgfqpoint{0.638564in}{0.670874in}}%
\pgfpathlineto{\pgfqpoint{0.645182in}{0.660580in}}%
\pgfpathlineto{\pgfqpoint{0.645367in}{0.660291in}}%
\pgfpathlineto{\pgfqpoint{0.651743in}{0.650285in}}%
\pgfpathlineto{\pgfqpoint{0.655944in}{0.643691in}}%
\pgfpathlineto{\pgfqpoint{0.658301in}{0.639991in}}%
\pgfpathlineto{\pgfqpoint{0.664911in}{0.629696in}}%
\pgfpathlineto{\pgfqpoint{0.666521in}{0.627171in}}%
\pgfpathlineto{\pgfqpoint{0.671469in}{0.619402in}}%
\pgfpathlineto{\pgfqpoint{0.677098in}{0.610535in}}%
\pgfpathlineto{\pgfqpoint{0.678003in}{0.609108in}}%
\pgfpathlineto{\pgfqpoint{0.684399in}{0.598813in}}%
\pgfpathlineto{\pgfqpoint{0.687675in}{0.593096in}}%
\pgfpathlineto{\pgfqpoint{0.690162in}{0.588519in}}%
\pgfpathlineto{\pgfqpoint{0.695494in}{0.578224in}}%
\pgfpathlineto{\pgfqpoint{0.698252in}{0.572608in}}%
\pgfpathlineto{\pgfqpoint{0.700567in}{0.567930in}}%
\pgfpathlineto{\pgfqpoint{0.705435in}{0.557635in}}%
\pgfpathlineto{\pgfqpoint{0.708829in}{0.550053in}}%
\pgfpathlineto{\pgfqpoint{0.710136in}{0.547341in}}%
\pgfpathlineto{\pgfqpoint{0.715090in}{0.537047in}}%
\pgfpathlineto{\pgfqpoint{0.719406in}{0.527917in}}%
\pgfpathlineto{\pgfqpoint{0.719960in}{0.526752in}}%
\pgfpathlineto{\pgfqpoint{0.724830in}{0.516458in}}%
\pgfpathlineto{\pgfqpoint{0.729740in}{0.506163in}}%
\pgfpathclose%
\pgfusepath{fill}%
\end{pgfscope}%
\begin{pgfscope}%
\pgfpathrectangle{\pgfqpoint{0.423750in}{0.423750in}}{\pgfqpoint{1.194205in}{1.163386in}}%
\pgfusepath{clip}%
\pgfsetbuttcap%
\pgfsetroundjoin%
\definecolor{currentfill}{rgb}{0.905301,0.238545,0.247481}%
\pgfsetfillcolor{currentfill}%
\pgfsetlinewidth{0.000000pt}%
\definecolor{currentstroke}{rgb}{0.000000,0.000000,0.000000}%
\pgfsetstrokecolor{currentstroke}%
\pgfsetdash{}{0pt}%
\pgfpathmoveto{\pgfqpoint{1.205951in}{1.184266in}}%
\pgfpathlineto{\pgfqpoint{1.216528in}{1.180096in}}%
\pgfpathlineto{\pgfqpoint{1.227105in}{1.177654in}}%
\pgfpathlineto{\pgfqpoint{1.237682in}{1.179290in}}%
\pgfpathlineto{\pgfqpoint{1.248259in}{1.181242in}}%
\pgfpathlineto{\pgfqpoint{1.258836in}{1.183322in}}%
\pgfpathlineto{\pgfqpoint{1.269413in}{1.185515in}}%
\pgfpathlineto{\pgfqpoint{1.269782in}{1.185596in}}%
\pgfpathlineto{\pgfqpoint{1.279990in}{1.187848in}}%
\pgfpathlineto{\pgfqpoint{1.290567in}{1.190076in}}%
\pgfpathlineto{\pgfqpoint{1.301144in}{1.192005in}}%
\pgfpathlineto{\pgfqpoint{1.311721in}{1.194531in}}%
\pgfpathlineto{\pgfqpoint{1.313692in}{1.195890in}}%
\pgfpathlineto{\pgfqpoint{1.322299in}{1.202077in}}%
\pgfpathlineto{\pgfqpoint{1.327985in}{1.206185in}}%
\pgfpathlineto{\pgfqpoint{1.332876in}{1.209733in}}%
\pgfpathlineto{\pgfqpoint{1.342123in}{1.216479in}}%
\pgfpathlineto{\pgfqpoint{1.343453in}{1.217453in}}%
\pgfpathlineto{\pgfqpoint{1.354030in}{1.225080in}}%
\pgfpathlineto{\pgfqpoint{1.356541in}{1.226773in}}%
\pgfpathlineto{\pgfqpoint{1.364607in}{1.231999in}}%
\pgfpathlineto{\pgfqpoint{1.374320in}{1.237068in}}%
\pgfpathlineto{\pgfqpoint{1.375184in}{1.237526in}}%
\pgfpathlineto{\pgfqpoint{1.385761in}{1.243001in}}%
\pgfpathlineto{\pgfqpoint{1.393906in}{1.247362in}}%
\pgfpathlineto{\pgfqpoint{1.396338in}{1.248665in}}%
\pgfpathlineto{\pgfqpoint{1.406915in}{1.254473in}}%
\pgfpathlineto{\pgfqpoint{1.412649in}{1.257657in}}%
\pgfpathlineto{\pgfqpoint{1.417492in}{1.260347in}}%
\pgfpathlineto{\pgfqpoint{1.428069in}{1.266305in}}%
\pgfpathlineto{\pgfqpoint{1.430939in}{1.267951in}}%
\pgfpathlineto{\pgfqpoint{1.438646in}{1.272368in}}%
\pgfpathlineto{\pgfqpoint{1.448726in}{1.278246in}}%
\pgfpathlineto{\pgfqpoint{1.449223in}{1.278535in}}%
\pgfpathlineto{\pgfqpoint{1.459800in}{1.284851in}}%
\pgfpathlineto{\pgfqpoint{1.465833in}{1.288540in}}%
\pgfpathlineto{\pgfqpoint{1.470377in}{1.291312in}}%
\pgfpathlineto{\pgfqpoint{1.480954in}{1.297939in}}%
\pgfpathlineto{\pgfqpoint{1.482342in}{1.298834in}}%
\pgfpathlineto{\pgfqpoint{1.491532in}{1.304749in}}%
\pgfpathlineto{\pgfqpoint{1.498155in}{1.309129in}}%
\pgfpathlineto{\pgfqpoint{1.502109in}{1.311739in}}%
\pgfpathlineto{\pgfqpoint{1.512686in}{1.318911in}}%
\pgfpathlineto{\pgfqpoint{1.513424in}{1.319423in}}%
\pgfpathlineto{\pgfqpoint{1.523263in}{1.326253in}}%
\pgfpathlineto{\pgfqpoint{1.528149in}{1.329718in}}%
\pgfpathlineto{\pgfqpoint{1.533840in}{1.333755in}}%
\pgfpathlineto{\pgfqpoint{1.542487in}{1.340012in}}%
\pgfpathlineto{\pgfqpoint{1.544417in}{1.341410in}}%
\pgfpathlineto{\pgfqpoint{1.544417in}{1.350307in}}%
\pgfpathlineto{\pgfqpoint{1.544417in}{1.360601in}}%
\pgfpathlineto{\pgfqpoint{1.544417in}{1.370896in}}%
\pgfpathlineto{\pgfqpoint{1.544417in}{1.381190in}}%
\pgfpathlineto{\pgfqpoint{1.544417in}{1.391484in}}%
\pgfpathlineto{\pgfqpoint{1.544417in}{1.394036in}}%
\pgfpathlineto{\pgfqpoint{1.541006in}{1.391484in}}%
\pgfpathlineto{\pgfqpoint{1.533840in}{1.386115in}}%
\pgfpathlineto{\pgfqpoint{1.527064in}{1.381190in}}%
\pgfpathlineto{\pgfqpoint{1.523263in}{1.378420in}}%
\pgfpathlineto{\pgfqpoint{1.512686in}{1.370958in}}%
\pgfpathlineto{\pgfqpoint{1.512593in}{1.370896in}}%
\pgfpathlineto{\pgfqpoint{1.502109in}{1.363767in}}%
\pgfpathlineto{\pgfqpoint{1.497297in}{1.360601in}}%
\pgfpathlineto{\pgfqpoint{1.491532in}{1.356793in}}%
\pgfpathlineto{\pgfqpoint{1.481409in}{1.350307in}}%
\pgfpathlineto{\pgfqpoint{1.480954in}{1.350014in}}%
\pgfpathlineto{\pgfqpoint{1.470377in}{1.343446in}}%
\pgfpathlineto{\pgfqpoint{1.464752in}{1.340012in}}%
\pgfpathlineto{\pgfqpoint{1.459800in}{1.337014in}}%
\pgfpathlineto{\pgfqpoint{1.449223in}{1.330497in}}%
\pgfpathlineto{\pgfqpoint{1.448075in}{1.329718in}}%
\pgfpathlineto{\pgfqpoint{1.438646in}{1.323723in}}%
\pgfpathlineto{\pgfqpoint{1.432346in}{1.319423in}}%
\pgfpathlineto{\pgfqpoint{1.428069in}{1.316657in}}%
\pgfpathlineto{\pgfqpoint{1.417492in}{1.309572in}}%
\pgfpathlineto{\pgfqpoint{1.416884in}{1.309129in}}%
\pgfpathlineto{\pgfqpoint{1.406915in}{1.301946in}}%
\pgfpathlineto{\pgfqpoint{1.402609in}{1.298834in}}%
\pgfpathlineto{\pgfqpoint{1.396338in}{1.294320in}}%
\pgfpathlineto{\pgfqpoint{1.388223in}{1.288540in}}%
\pgfpathlineto{\pgfqpoint{1.385761in}{1.286791in}}%
\pgfpathlineto{\pgfqpoint{1.375184in}{1.279366in}}%
\pgfpathlineto{\pgfqpoint{1.373587in}{1.278246in}}%
\pgfpathlineto{\pgfqpoint{1.364607in}{1.271968in}}%
\pgfpathlineto{\pgfqpoint{1.358871in}{1.267951in}}%
\pgfpathlineto{\pgfqpoint{1.354030in}{1.264571in}}%
\pgfpathlineto{\pgfqpoint{1.344312in}{1.257657in}}%
\pgfpathlineto{\pgfqpoint{1.343453in}{1.257018in}}%
\pgfpathlineto{\pgfqpoint{1.332876in}{1.248479in}}%
\pgfpathlineto{\pgfqpoint{1.331485in}{1.247362in}}%
\pgfpathlineto{\pgfqpoint{1.322299in}{1.240006in}}%
\pgfpathlineto{\pgfqpoint{1.318613in}{1.237068in}}%
\pgfpathlineto{\pgfqpoint{1.311721in}{1.231593in}}%
\pgfpathlineto{\pgfqpoint{1.301144in}{1.227369in}}%
\pgfpathlineto{\pgfqpoint{1.298585in}{1.226773in}}%
\pgfpathlineto{\pgfqpoint{1.290567in}{1.224918in}}%
\pgfpathlineto{\pgfqpoint{1.279990in}{1.222558in}}%
\pgfpathlineto{\pgfqpoint{1.269413in}{1.220173in}}%
\pgfpathlineto{\pgfqpoint{1.258836in}{1.217975in}}%
\pgfpathlineto{\pgfqpoint{1.250755in}{1.216479in}}%
\pgfpathlineto{\pgfqpoint{1.248259in}{1.216031in}}%
\pgfpathlineto{\pgfqpoint{1.237682in}{1.215648in}}%
\pgfpathlineto{\pgfqpoint{1.234482in}{1.216479in}}%
\pgfpathlineto{\pgfqpoint{1.227105in}{1.218582in}}%
\pgfpathlineto{\pgfqpoint{1.216528in}{1.222697in}}%
\pgfpathlineto{\pgfqpoint{1.206491in}{1.226773in}}%
\pgfpathlineto{\pgfqpoint{1.205951in}{1.226995in}}%
\pgfpathlineto{\pgfqpoint{1.195374in}{1.231549in}}%
\pgfpathlineto{\pgfqpoint{1.184797in}{1.237019in}}%
\pgfpathlineto{\pgfqpoint{1.184712in}{1.237068in}}%
\pgfpathlineto{\pgfqpoint{1.174220in}{1.243122in}}%
\pgfpathlineto{\pgfqpoint{1.167268in}{1.247362in}}%
\pgfpathlineto{\pgfqpoint{1.163643in}{1.249610in}}%
\pgfpathlineto{\pgfqpoint{1.153066in}{1.255174in}}%
\pgfpathlineto{\pgfqpoint{1.148785in}{1.257657in}}%
\pgfpathlineto{\pgfqpoint{1.142488in}{1.261380in}}%
\pgfpathlineto{\pgfqpoint{1.132560in}{1.267951in}}%
\pgfpathlineto{\pgfqpoint{1.131911in}{1.268374in}}%
\pgfpathlineto{\pgfqpoint{1.121334in}{1.275606in}}%
\pgfpathlineto{\pgfqpoint{1.117650in}{1.278246in}}%
\pgfpathlineto{\pgfqpoint{1.110757in}{1.283272in}}%
\pgfpathlineto{\pgfqpoint{1.103465in}{1.288540in}}%
\pgfpathlineto{\pgfqpoint{1.100180in}{1.290993in}}%
\pgfpathlineto{\pgfqpoint{1.091151in}{1.298834in}}%
\pgfpathlineto{\pgfqpoint{1.089603in}{1.300232in}}%
\pgfpathlineto{\pgfqpoint{1.080102in}{1.309129in}}%
\pgfpathlineto{\pgfqpoint{1.079026in}{1.310143in}}%
\pgfpathlineto{\pgfqpoint{1.070114in}{1.319423in}}%
\pgfpathlineto{\pgfqpoint{1.068449in}{1.321246in}}%
\pgfpathlineto{\pgfqpoint{1.061053in}{1.329718in}}%
\pgfpathlineto{\pgfqpoint{1.057872in}{1.333408in}}%
\pgfpathlineto{\pgfqpoint{1.052375in}{1.340012in}}%
\pgfpathlineto{\pgfqpoint{1.047295in}{1.346515in}}%
\pgfpathlineto{\pgfqpoint{1.044508in}{1.350307in}}%
\pgfpathlineto{\pgfqpoint{1.037755in}{1.360601in}}%
\pgfpathlineto{\pgfqpoint{1.036718in}{1.362144in}}%
\pgfpathlineto{\pgfqpoint{1.028123in}{1.370896in}}%
\pgfpathlineto{\pgfqpoint{1.026141in}{1.373139in}}%
\pgfpathlineto{\pgfqpoint{1.018295in}{1.381190in}}%
\pgfpathlineto{\pgfqpoint{1.015564in}{1.384026in}}%
\pgfpathlineto{\pgfqpoint{1.008452in}{1.391484in}}%
\pgfpathlineto{\pgfqpoint{1.004987in}{1.395145in}}%
\pgfpathlineto{\pgfqpoint{0.998765in}{1.401779in}}%
\pgfpathlineto{\pgfqpoint{0.994410in}{1.406455in}}%
\pgfpathlineto{\pgfqpoint{0.989223in}{1.412073in}}%
\pgfpathlineto{\pgfqpoint{0.983833in}{1.417951in}}%
\pgfpathlineto{\pgfqpoint{0.979813in}{1.422368in}}%
\pgfpathlineto{\pgfqpoint{0.973255in}{1.429616in}}%
\pgfpathlineto{\pgfqpoint{0.970518in}{1.432662in}}%
\pgfpathlineto{\pgfqpoint{0.962678in}{1.441433in}}%
\pgfpathlineto{\pgfqpoint{0.961521in}{1.442957in}}%
\pgfpathlineto{\pgfqpoint{0.954737in}{1.453251in}}%
\pgfpathlineto{\pgfqpoint{0.952101in}{1.457836in}}%
\pgfpathlineto{\pgfqpoint{0.948933in}{1.463545in}}%
\pgfpathlineto{\pgfqpoint{0.943219in}{1.473840in}}%
\pgfpathlineto{\pgfqpoint{0.941524in}{1.476875in}}%
\pgfpathlineto{\pgfqpoint{0.937575in}{1.484134in}}%
\pgfpathlineto{\pgfqpoint{0.932006in}{1.494429in}}%
\pgfpathlineto{\pgfqpoint{0.930947in}{1.496439in}}%
\pgfpathlineto{\pgfqpoint{0.926835in}{1.504723in}}%
\pgfpathlineto{\pgfqpoint{0.921909in}{1.515018in}}%
\pgfpathlineto{\pgfqpoint{0.920370in}{1.515018in}}%
\pgfpathlineto{\pgfqpoint{0.909793in}{1.515018in}}%
\pgfpathlineto{\pgfqpoint{0.899216in}{1.515018in}}%
\pgfpathlineto{\pgfqpoint{0.888639in}{1.515018in}}%
\pgfpathlineto{\pgfqpoint{0.882839in}{1.515018in}}%
\pgfpathlineto{\pgfqpoint{0.887898in}{1.504723in}}%
\pgfpathlineto{\pgfqpoint{0.888639in}{1.503154in}}%
\pgfpathlineto{\pgfqpoint{0.892490in}{1.494429in}}%
\pgfpathlineto{\pgfqpoint{0.897044in}{1.484134in}}%
\pgfpathlineto{\pgfqpoint{0.899216in}{1.479324in}}%
\pgfpathlineto{\pgfqpoint{0.901746in}{1.473840in}}%
\pgfpathlineto{\pgfqpoint{0.906849in}{1.463545in}}%
\pgfpathlineto{\pgfqpoint{0.909793in}{1.457620in}}%
\pgfpathlineto{\pgfqpoint{0.912000in}{1.453251in}}%
\pgfpathlineto{\pgfqpoint{0.917223in}{1.442957in}}%
\pgfpathlineto{\pgfqpoint{0.920370in}{1.436792in}}%
\pgfpathlineto{\pgfqpoint{0.922526in}{1.432662in}}%
\pgfpathlineto{\pgfqpoint{0.927951in}{1.422368in}}%
\pgfpathlineto{\pgfqpoint{0.930947in}{1.416754in}}%
\pgfpathlineto{\pgfqpoint{0.933623in}{1.412073in}}%
\pgfpathlineto{\pgfqpoint{0.940180in}{1.401779in}}%
\pgfpathlineto{\pgfqpoint{0.941524in}{1.399933in}}%
\pgfpathlineto{\pgfqpoint{0.949085in}{1.391484in}}%
\pgfpathlineto{\pgfqpoint{0.952101in}{1.388126in}}%
\pgfpathlineto{\pgfqpoint{0.958394in}{1.381190in}}%
\pgfpathlineto{\pgfqpoint{0.962678in}{1.376487in}}%
\pgfpathlineto{\pgfqpoint{0.967827in}{1.370896in}}%
\pgfpathlineto{\pgfqpoint{0.973255in}{1.365040in}}%
\pgfpathlineto{\pgfqpoint{0.977422in}{1.360601in}}%
\pgfpathlineto{\pgfqpoint{0.983833in}{1.353836in}}%
\pgfpathlineto{\pgfqpoint{0.987213in}{1.350307in}}%
\pgfpathlineto{\pgfqpoint{0.994410in}{1.342844in}}%
\pgfpathlineto{\pgfqpoint{0.997165in}{1.340012in}}%
\pgfpathlineto{\pgfqpoint{1.004987in}{1.332238in}}%
\pgfpathlineto{\pgfqpoint{1.007471in}{1.329718in}}%
\pgfpathlineto{\pgfqpoint{1.015564in}{1.321998in}}%
\pgfpathlineto{\pgfqpoint{1.018285in}{1.319423in}}%
\pgfpathlineto{\pgfqpoint{1.026141in}{1.312962in}}%
\pgfpathlineto{\pgfqpoint{1.031576in}{1.309129in}}%
\pgfpathlineto{\pgfqpoint{1.036718in}{1.305522in}}%
\pgfpathlineto{\pgfqpoint{1.044007in}{1.298834in}}%
\pgfpathlineto{\pgfqpoint{1.047295in}{1.295954in}}%
\pgfpathlineto{\pgfqpoint{1.053928in}{1.288540in}}%
\pgfpathlineto{\pgfqpoint{1.057872in}{1.284155in}}%
\pgfpathlineto{\pgfqpoint{1.063796in}{1.278246in}}%
\pgfpathlineto{\pgfqpoint{1.068449in}{1.273739in}}%
\pgfpathlineto{\pgfqpoint{1.074654in}{1.267951in}}%
\pgfpathlineto{\pgfqpoint{1.079026in}{1.263824in}}%
\pgfpathlineto{\pgfqpoint{1.085938in}{1.257657in}}%
\pgfpathlineto{\pgfqpoint{1.089603in}{1.254269in}}%
\pgfpathlineto{\pgfqpoint{1.098003in}{1.247362in}}%
\pgfpathlineto{\pgfqpoint{1.100180in}{1.245543in}}%
\pgfpathlineto{\pgfqpoint{1.110567in}{1.237068in}}%
\pgfpathlineto{\pgfqpoint{1.110757in}{1.236913in}}%
\pgfpathlineto{\pgfqpoint{1.121334in}{1.228968in}}%
\pgfpathlineto{\pgfqpoint{1.124657in}{1.226773in}}%
\pgfpathlineto{\pgfqpoint{1.131911in}{1.222270in}}%
\pgfpathlineto{\pgfqpoint{1.142488in}{1.216607in}}%
\pgfpathlineto{\pgfqpoint{1.142759in}{1.216479in}}%
\pgfpathlineto{\pgfqpoint{1.153066in}{1.211523in}}%
\pgfpathlineto{\pgfqpoint{1.161842in}{1.206185in}}%
\pgfpathlineto{\pgfqpoint{1.163643in}{1.205051in}}%
\pgfpathlineto{\pgfqpoint{1.174220in}{1.198773in}}%
\pgfpathlineto{\pgfqpoint{1.180110in}{1.195890in}}%
\pgfpathlineto{\pgfqpoint{1.184797in}{1.193549in}}%
\pgfpathlineto{\pgfqpoint{1.195374in}{1.188829in}}%
\pgfpathlineto{\pgfqpoint{1.202822in}{1.185596in}}%
\pgfpathclose%
\pgfusepath{fill}%
\end{pgfscope}%
\begin{pgfscope}%
\pgfpathrectangle{\pgfqpoint{0.423750in}{0.423750in}}{\pgfqpoint{1.194205in}{1.163386in}}%
\pgfusepath{clip}%
\pgfsetbuttcap%
\pgfsetroundjoin%
\definecolor{currentfill}{rgb}{0.934351,0.329284,0.247753}%
\pgfsetfillcolor{currentfill}%
\pgfsetlinewidth{0.000000pt}%
\definecolor{currentstroke}{rgb}{0.000000,0.000000,0.000000}%
\pgfsetstrokecolor{currentstroke}%
\pgfsetdash{}{0pt}%
\pgfpathmoveto{\pgfqpoint{0.751137in}{0.503618in}}%
\pgfpathlineto{\pgfqpoint{0.756517in}{0.495869in}}%
\pgfpathlineto{\pgfqpoint{0.761714in}{0.495869in}}%
\pgfpathlineto{\pgfqpoint{0.772291in}{0.495869in}}%
\pgfpathlineto{\pgfqpoint{0.782868in}{0.495869in}}%
\pgfpathlineto{\pgfqpoint{0.784609in}{0.495869in}}%
\pgfpathlineto{\pgfqpoint{0.782868in}{0.498355in}}%
\pgfpathlineto{\pgfqpoint{0.777412in}{0.506163in}}%
\pgfpathlineto{\pgfqpoint{0.772291in}{0.513447in}}%
\pgfpathlineto{\pgfqpoint{0.770175in}{0.516458in}}%
\pgfpathlineto{\pgfqpoint{0.762878in}{0.526752in}}%
\pgfpathlineto{\pgfqpoint{0.761714in}{0.528372in}}%
\pgfpathlineto{\pgfqpoint{0.755469in}{0.537047in}}%
\pgfpathlineto{\pgfqpoint{0.751137in}{0.543037in}}%
\pgfpathlineto{\pgfqpoint{0.748049in}{0.547341in}}%
\pgfpathlineto{\pgfqpoint{0.740650in}{0.557635in}}%
\pgfpathlineto{\pgfqpoint{0.740560in}{0.557759in}}%
\pgfpathlineto{\pgfqpoint{0.734674in}{0.567930in}}%
\pgfpathlineto{\pgfqpoint{0.729983in}{0.575034in}}%
\pgfpathlineto{\pgfqpoint{0.728398in}{0.578224in}}%
\pgfpathlineto{\pgfqpoint{0.723217in}{0.588519in}}%
\pgfpathlineto{\pgfqpoint{0.719406in}{0.595848in}}%
\pgfpathlineto{\pgfqpoint{0.717799in}{0.598813in}}%
\pgfpathlineto{\pgfqpoint{0.711893in}{0.609108in}}%
\pgfpathlineto{\pgfqpoint{0.708829in}{0.614211in}}%
\pgfpathlineto{\pgfqpoint{0.705533in}{0.619402in}}%
\pgfpathlineto{\pgfqpoint{0.698309in}{0.629696in}}%
\pgfpathlineto{\pgfqpoint{0.698252in}{0.629775in}}%
\pgfpathlineto{\pgfqpoint{0.691567in}{0.639991in}}%
\pgfpathlineto{\pgfqpoint{0.687675in}{0.645806in}}%
\pgfpathlineto{\pgfqpoint{0.684759in}{0.650285in}}%
\pgfpathlineto{\pgfqpoint{0.678107in}{0.660580in}}%
\pgfpathlineto{\pgfqpoint{0.677098in}{0.662130in}}%
\pgfpathlineto{\pgfqpoint{0.671399in}{0.670874in}}%
\pgfpathlineto{\pgfqpoint{0.666521in}{0.678332in}}%
\pgfpathlineto{\pgfqpoint{0.664663in}{0.681169in}}%
\pgfpathlineto{\pgfqpoint{0.657869in}{0.691463in}}%
\pgfpathlineto{\pgfqpoint{0.655944in}{0.694374in}}%
\pgfpathlineto{\pgfqpoint{0.651057in}{0.701757in}}%
\pgfpathlineto{\pgfqpoint{0.645367in}{0.710306in}}%
\pgfpathlineto{\pgfqpoint{0.644204in}{0.712052in}}%
\pgfpathlineto{\pgfqpoint{0.637375in}{0.722346in}}%
\pgfpathlineto{\pgfqpoint{0.634790in}{0.726348in}}%
\pgfpathlineto{\pgfqpoint{0.630651in}{0.732641in}}%
\pgfpathlineto{\pgfqpoint{0.624212in}{0.742663in}}%
\pgfpathlineto{\pgfqpoint{0.624035in}{0.742935in}}%
\pgfpathlineto{\pgfqpoint{0.617438in}{0.753230in}}%
\pgfpathlineto{\pgfqpoint{0.613635in}{0.759053in}}%
\pgfpathlineto{\pgfqpoint{0.610690in}{0.763524in}}%
\pgfpathlineto{\pgfqpoint{0.604531in}{0.773818in}}%
\pgfpathlineto{\pgfqpoint{0.603058in}{0.776447in}}%
\pgfpathlineto{\pgfqpoint{0.598913in}{0.784113in}}%
\pgfpathlineto{\pgfqpoint{0.593017in}{0.794407in}}%
\pgfpathlineto{\pgfqpoint{0.592481in}{0.795825in}}%
\pgfpathlineto{\pgfqpoint{0.588931in}{0.804702in}}%
\pgfpathlineto{\pgfqpoint{0.585327in}{0.814996in}}%
\pgfpathlineto{\pgfqpoint{0.581904in}{0.824415in}}%
\pgfpathlineto{\pgfqpoint{0.581575in}{0.825291in}}%
\pgfpathlineto{\pgfqpoint{0.577649in}{0.835585in}}%
\pgfpathlineto{\pgfqpoint{0.573897in}{0.845879in}}%
\pgfpathlineto{\pgfqpoint{0.571327in}{0.852875in}}%
\pgfpathlineto{\pgfqpoint{0.570110in}{0.856174in}}%
\pgfpathlineto{\pgfqpoint{0.566283in}{0.866468in}}%
\pgfpathlineto{\pgfqpoint{0.562430in}{0.876763in}}%
\pgfpathlineto{\pgfqpoint{0.560750in}{0.881218in}}%
\pgfpathlineto{\pgfqpoint{0.558541in}{0.887057in}}%
\pgfpathlineto{\pgfqpoint{0.554623in}{0.897352in}}%
\pgfpathlineto{\pgfqpoint{0.550684in}{0.907646in}}%
\pgfpathlineto{\pgfqpoint{0.550173in}{0.908973in}}%
\pgfpathlineto{\pgfqpoint{0.546712in}{0.917941in}}%
\pgfpathlineto{\pgfqpoint{0.542721in}{0.928235in}}%
\pgfpathlineto{\pgfqpoint{0.539596in}{0.936259in}}%
\pgfpathlineto{\pgfqpoint{0.538710in}{0.938529in}}%
\pgfpathlineto{\pgfqpoint{0.534675in}{0.948824in}}%
\pgfpathlineto{\pgfqpoint{0.530629in}{0.959118in}}%
\pgfpathlineto{\pgfqpoint{0.529019in}{0.963174in}}%
\pgfpathlineto{\pgfqpoint{0.526530in}{0.969413in}}%
\pgfpathlineto{\pgfqpoint{0.522297in}{0.979707in}}%
\pgfpathlineto{\pgfqpoint{0.518442in}{0.988860in}}%
\pgfpathlineto{\pgfqpoint{0.517959in}{0.990002in}}%
\pgfpathlineto{\pgfqpoint{0.513516in}{1.000296in}}%
\pgfpathlineto{\pgfqpoint{0.508993in}{1.010590in}}%
\pgfpathlineto{\pgfqpoint{0.507865in}{1.013934in}}%
\pgfpathlineto{\pgfqpoint{0.505548in}{1.020885in}}%
\pgfpathlineto{\pgfqpoint{0.502089in}{1.031179in}}%
\pgfpathlineto{\pgfqpoint{0.498595in}{1.041474in}}%
\pgfpathlineto{\pgfqpoint{0.497288in}{1.045345in}}%
\pgfpathlineto{\pgfqpoint{0.497288in}{1.041474in}}%
\pgfpathlineto{\pgfqpoint{0.497288in}{1.031179in}}%
\pgfpathlineto{\pgfqpoint{0.497288in}{1.020885in}}%
\pgfpathlineto{\pgfqpoint{0.497288in}{1.010590in}}%
\pgfpathlineto{\pgfqpoint{0.497288in}{1.000296in}}%
\pgfpathlineto{\pgfqpoint{0.497288in}{0.990002in}}%
\pgfpathlineto{\pgfqpoint{0.497288in}{0.979707in}}%
\pgfpathlineto{\pgfqpoint{0.497288in}{0.971552in}}%
\pgfpathlineto{\pgfqpoint{0.498180in}{0.969413in}}%
\pgfpathlineto{\pgfqpoint{0.502376in}{0.959118in}}%
\pgfpathlineto{\pgfqpoint{0.506438in}{0.948824in}}%
\pgfpathlineto{\pgfqpoint{0.507865in}{0.945185in}}%
\pgfpathlineto{\pgfqpoint{0.510465in}{0.938529in}}%
\pgfpathlineto{\pgfqpoint{0.514474in}{0.928235in}}%
\pgfpathlineto{\pgfqpoint{0.518442in}{0.918022in}}%
\pgfpathlineto{\pgfqpoint{0.518473in}{0.917941in}}%
\pgfpathlineto{\pgfqpoint{0.522434in}{0.907646in}}%
\pgfpathlineto{\pgfqpoint{0.526381in}{0.897352in}}%
\pgfpathlineto{\pgfqpoint{0.529019in}{0.890442in}}%
\pgfpathlineto{\pgfqpoint{0.530305in}{0.887057in}}%
\pgfpathlineto{\pgfqpoint{0.534194in}{0.876763in}}%
\pgfpathlineto{\pgfqpoint{0.538065in}{0.866468in}}%
\pgfpathlineto{\pgfqpoint{0.539596in}{0.862371in}}%
\pgfpathlineto{\pgfqpoint{0.541900in}{0.856174in}}%
\pgfpathlineto{\pgfqpoint{0.546052in}{0.845879in}}%
\pgfpathlineto{\pgfqpoint{0.550074in}{0.835585in}}%
\pgfpathlineto{\pgfqpoint{0.550173in}{0.835401in}}%
\pgfpathlineto{\pgfqpoint{0.555675in}{0.825291in}}%
\pgfpathlineto{\pgfqpoint{0.560750in}{0.816070in}}%
\pgfpathlineto{\pgfqpoint{0.561351in}{0.814996in}}%
\pgfpathlineto{\pgfqpoint{0.568013in}{0.804702in}}%
\pgfpathlineto{\pgfqpoint{0.571327in}{0.799505in}}%
\pgfpathlineto{\pgfqpoint{0.574523in}{0.794407in}}%
\pgfpathlineto{\pgfqpoint{0.580960in}{0.784113in}}%
\pgfpathlineto{\pgfqpoint{0.581904in}{0.782595in}}%
\pgfpathlineto{\pgfqpoint{0.587343in}{0.773818in}}%
\pgfpathlineto{\pgfqpoint{0.592481in}{0.765496in}}%
\pgfpathlineto{\pgfqpoint{0.593694in}{0.763524in}}%
\pgfpathlineto{\pgfqpoint{0.600278in}{0.753230in}}%
\pgfpathlineto{\pgfqpoint{0.603058in}{0.749241in}}%
\pgfpathlineto{\pgfqpoint{0.607346in}{0.742935in}}%
\pgfpathlineto{\pgfqpoint{0.613635in}{0.733723in}}%
\pgfpathlineto{\pgfqpoint{0.614366in}{0.732641in}}%
\pgfpathlineto{\pgfqpoint{0.621269in}{0.722346in}}%
\pgfpathlineto{\pgfqpoint{0.624212in}{0.717926in}}%
\pgfpathlineto{\pgfqpoint{0.628124in}{0.712052in}}%
\pgfpathlineto{\pgfqpoint{0.634790in}{0.701981in}}%
\pgfpathlineto{\pgfqpoint{0.634937in}{0.701757in}}%
\pgfpathlineto{\pgfqpoint{0.641679in}{0.691463in}}%
\pgfpathlineto{\pgfqpoint{0.645367in}{0.685834in}}%
\pgfpathlineto{\pgfqpoint{0.648422in}{0.681169in}}%
\pgfpathlineto{\pgfqpoint{0.655124in}{0.670874in}}%
\pgfpathlineto{\pgfqpoint{0.655944in}{0.669606in}}%
\pgfpathlineto{\pgfqpoint{0.661769in}{0.660580in}}%
\pgfpathlineto{\pgfqpoint{0.666521in}{0.653200in}}%
\pgfpathlineto{\pgfqpoint{0.668394in}{0.650285in}}%
\pgfpathlineto{\pgfqpoint{0.675022in}{0.639991in}}%
\pgfpathlineto{\pgfqpoint{0.677098in}{0.636783in}}%
\pgfpathlineto{\pgfqpoint{0.681681in}{0.629696in}}%
\pgfpathlineto{\pgfqpoint{0.687675in}{0.620388in}}%
\pgfpathlineto{\pgfqpoint{0.688309in}{0.619402in}}%
\pgfpathlineto{\pgfqpoint{0.694738in}{0.609108in}}%
\pgfpathlineto{\pgfqpoint{0.698252in}{0.603001in}}%
\pgfpathlineto{\pgfqpoint{0.700743in}{0.598813in}}%
\pgfpathlineto{\pgfqpoint{0.706509in}{0.588519in}}%
\pgfpathlineto{\pgfqpoint{0.708829in}{0.583957in}}%
\pgfpathlineto{\pgfqpoint{0.711774in}{0.578224in}}%
\pgfpathlineto{\pgfqpoint{0.716943in}{0.567930in}}%
\pgfpathlineto{\pgfqpoint{0.719406in}{0.562921in}}%
\pgfpathlineto{\pgfqpoint{0.721989in}{0.557635in}}%
\pgfpathlineto{\pgfqpoint{0.726988in}{0.547341in}}%
\pgfpathlineto{\pgfqpoint{0.729983in}{0.541141in}}%
\pgfpathlineto{\pgfqpoint{0.732125in}{0.537047in}}%
\pgfpathlineto{\pgfqpoint{0.737063in}{0.526752in}}%
\pgfpathlineto{\pgfqpoint{0.740560in}{0.519339in}}%
\pgfpathlineto{\pgfqpoint{0.742656in}{0.516458in}}%
\pgfpathlineto{\pgfqpoint{0.749603in}{0.506163in}}%
\pgfpathclose%
\pgfusepath{fill}%
\end{pgfscope}%
\begin{pgfscope}%
\pgfpathrectangle{\pgfqpoint{0.423750in}{0.423750in}}{\pgfqpoint{1.194205in}{1.163386in}}%
\pgfusepath{clip}%
\pgfsetbuttcap%
\pgfsetroundjoin%
\definecolor{currentfill}{rgb}{0.934351,0.329284,0.247753}%
\pgfsetfillcolor{currentfill}%
\pgfsetlinewidth{0.000000pt}%
\definecolor{currentstroke}{rgb}{0.000000,0.000000,0.000000}%
\pgfsetstrokecolor{currentstroke}%
\pgfsetdash{}{0pt}%
\pgfpathmoveto{\pgfqpoint{1.205951in}{1.144036in}}%
\pgfpathlineto{\pgfqpoint{1.216528in}{1.143055in}}%
\pgfpathlineto{\pgfqpoint{1.227105in}{1.144365in}}%
\pgfpathlineto{\pgfqpoint{1.227378in}{1.144418in}}%
\pgfpathlineto{\pgfqpoint{1.237682in}{1.146356in}}%
\pgfpathlineto{\pgfqpoint{1.248259in}{1.147227in}}%
\pgfpathlineto{\pgfqpoint{1.258836in}{1.147886in}}%
\pgfpathlineto{\pgfqpoint{1.269413in}{1.148550in}}%
\pgfpathlineto{\pgfqpoint{1.279990in}{1.148772in}}%
\pgfpathlineto{\pgfqpoint{1.290567in}{1.148484in}}%
\pgfpathlineto{\pgfqpoint{1.301144in}{1.147871in}}%
\pgfpathlineto{\pgfqpoint{1.311721in}{1.147272in}}%
\pgfpathlineto{\pgfqpoint{1.322299in}{1.152810in}}%
\pgfpathlineto{\pgfqpoint{1.325622in}{1.154712in}}%
\pgfpathlineto{\pgfqpoint{1.332876in}{1.158908in}}%
\pgfpathlineto{\pgfqpoint{1.343342in}{1.165007in}}%
\pgfpathlineto{\pgfqpoint{1.343453in}{1.165072in}}%
\pgfpathlineto{\pgfqpoint{1.354030in}{1.171304in}}%
\pgfpathlineto{\pgfqpoint{1.360507in}{1.175301in}}%
\pgfpathlineto{\pgfqpoint{1.364607in}{1.178042in}}%
\pgfpathlineto{\pgfqpoint{1.375184in}{1.185184in}}%
\pgfpathlineto{\pgfqpoint{1.375787in}{1.185596in}}%
\pgfpathlineto{\pgfqpoint{1.385761in}{1.192038in}}%
\pgfpathlineto{\pgfqpoint{1.392770in}{1.195890in}}%
\pgfpathlineto{\pgfqpoint{1.396338in}{1.197841in}}%
\pgfpathlineto{\pgfqpoint{1.406915in}{1.203712in}}%
\pgfpathlineto{\pgfqpoint{1.411314in}{1.206185in}}%
\pgfpathlineto{\pgfqpoint{1.417492in}{1.209667in}}%
\pgfpathlineto{\pgfqpoint{1.428069in}{1.215714in}}%
\pgfpathlineto{\pgfqpoint{1.429384in}{1.216479in}}%
\pgfpathlineto{\pgfqpoint{1.438646in}{1.221880in}}%
\pgfpathlineto{\pgfqpoint{1.446893in}{1.226773in}}%
\pgfpathlineto{\pgfqpoint{1.449223in}{1.228158in}}%
\pgfpathlineto{\pgfqpoint{1.459800in}{1.234572in}}%
\pgfpathlineto{\pgfqpoint{1.463833in}{1.237068in}}%
\pgfpathlineto{\pgfqpoint{1.470377in}{1.241124in}}%
\pgfpathlineto{\pgfqpoint{1.480238in}{1.247362in}}%
\pgfpathlineto{\pgfqpoint{1.480954in}{1.247816in}}%
\pgfpathlineto{\pgfqpoint{1.491532in}{1.254652in}}%
\pgfpathlineto{\pgfqpoint{1.496094in}{1.257657in}}%
\pgfpathlineto{\pgfqpoint{1.502109in}{1.261627in}}%
\pgfpathlineto{\pgfqpoint{1.511513in}{1.267951in}}%
\pgfpathlineto{\pgfqpoint{1.512686in}{1.268742in}}%
\pgfpathlineto{\pgfqpoint{1.523263in}{1.275992in}}%
\pgfpathlineto{\pgfqpoint{1.526502in}{1.278246in}}%
\pgfpathlineto{\pgfqpoint{1.533840in}{1.283367in}}%
\pgfpathlineto{\pgfqpoint{1.541144in}{1.288540in}}%
\pgfpathlineto{\pgfqpoint{1.544417in}{1.290865in}}%
\pgfpathlineto{\pgfqpoint{1.544417in}{1.298834in}}%
\pgfpathlineto{\pgfqpoint{1.544417in}{1.309129in}}%
\pgfpathlineto{\pgfqpoint{1.544417in}{1.319423in}}%
\pgfpathlineto{\pgfqpoint{1.544417in}{1.329718in}}%
\pgfpathlineto{\pgfqpoint{1.544417in}{1.340012in}}%
\pgfpathlineto{\pgfqpoint{1.544417in}{1.341410in}}%
\pgfpathlineto{\pgfqpoint{1.542487in}{1.340012in}}%
\pgfpathlineto{\pgfqpoint{1.533840in}{1.333755in}}%
\pgfpathlineto{\pgfqpoint{1.528149in}{1.329718in}}%
\pgfpathlineto{\pgfqpoint{1.523263in}{1.326253in}}%
\pgfpathlineto{\pgfqpoint{1.513424in}{1.319423in}}%
\pgfpathlineto{\pgfqpoint{1.512686in}{1.318911in}}%
\pgfpathlineto{\pgfqpoint{1.502109in}{1.311739in}}%
\pgfpathlineto{\pgfqpoint{1.498155in}{1.309129in}}%
\pgfpathlineto{\pgfqpoint{1.491532in}{1.304749in}}%
\pgfpathlineto{\pgfqpoint{1.482342in}{1.298834in}}%
\pgfpathlineto{\pgfqpoint{1.480954in}{1.297939in}}%
\pgfpathlineto{\pgfqpoint{1.470377in}{1.291312in}}%
\pgfpathlineto{\pgfqpoint{1.465833in}{1.288540in}}%
\pgfpathlineto{\pgfqpoint{1.459800in}{1.284851in}}%
\pgfpathlineto{\pgfqpoint{1.449223in}{1.278535in}}%
\pgfpathlineto{\pgfqpoint{1.448726in}{1.278246in}}%
\pgfpathlineto{\pgfqpoint{1.438646in}{1.272368in}}%
\pgfpathlineto{\pgfqpoint{1.430939in}{1.267951in}}%
\pgfpathlineto{\pgfqpoint{1.428069in}{1.266305in}}%
\pgfpathlineto{\pgfqpoint{1.417492in}{1.260347in}}%
\pgfpathlineto{\pgfqpoint{1.412649in}{1.257657in}}%
\pgfpathlineto{\pgfqpoint{1.406915in}{1.254473in}}%
\pgfpathlineto{\pgfqpoint{1.396338in}{1.248665in}}%
\pgfpathlineto{\pgfqpoint{1.393906in}{1.247362in}}%
\pgfpathlineto{\pgfqpoint{1.385761in}{1.243001in}}%
\pgfpathlineto{\pgfqpoint{1.375184in}{1.237526in}}%
\pgfpathlineto{\pgfqpoint{1.374320in}{1.237068in}}%
\pgfpathlineto{\pgfqpoint{1.364607in}{1.231999in}}%
\pgfpathlineto{\pgfqpoint{1.356541in}{1.226773in}}%
\pgfpathlineto{\pgfqpoint{1.354030in}{1.225080in}}%
\pgfpathlineto{\pgfqpoint{1.343453in}{1.217453in}}%
\pgfpathlineto{\pgfqpoint{1.342123in}{1.216479in}}%
\pgfpathlineto{\pgfqpoint{1.332876in}{1.209733in}}%
\pgfpathlineto{\pgfqpoint{1.327985in}{1.206185in}}%
\pgfpathlineto{\pgfqpoint{1.322299in}{1.202077in}}%
\pgfpathlineto{\pgfqpoint{1.313692in}{1.195890in}}%
\pgfpathlineto{\pgfqpoint{1.311721in}{1.194531in}}%
\pgfpathlineto{\pgfqpoint{1.301144in}{1.192005in}}%
\pgfpathlineto{\pgfqpoint{1.290567in}{1.190076in}}%
\pgfpathlineto{\pgfqpoint{1.279990in}{1.187848in}}%
\pgfpathlineto{\pgfqpoint{1.269782in}{1.185596in}}%
\pgfpathlineto{\pgfqpoint{1.269413in}{1.185515in}}%
\pgfpathlineto{\pgfqpoint{1.258836in}{1.183322in}}%
\pgfpathlineto{\pgfqpoint{1.248259in}{1.181242in}}%
\pgfpathlineto{\pgfqpoint{1.237682in}{1.179290in}}%
\pgfpathlineto{\pgfqpoint{1.227105in}{1.177654in}}%
\pgfpathlineto{\pgfqpoint{1.216528in}{1.180096in}}%
\pgfpathlineto{\pgfqpoint{1.205951in}{1.184266in}}%
\pgfpathlineto{\pgfqpoint{1.202822in}{1.185596in}}%
\pgfpathlineto{\pgfqpoint{1.195374in}{1.188829in}}%
\pgfpathlineto{\pgfqpoint{1.184797in}{1.193549in}}%
\pgfpathlineto{\pgfqpoint{1.180110in}{1.195890in}}%
\pgfpathlineto{\pgfqpoint{1.174220in}{1.198773in}}%
\pgfpathlineto{\pgfqpoint{1.163643in}{1.205051in}}%
\pgfpathlineto{\pgfqpoint{1.161842in}{1.206185in}}%
\pgfpathlineto{\pgfqpoint{1.153066in}{1.211523in}}%
\pgfpathlineto{\pgfqpoint{1.142759in}{1.216479in}}%
\pgfpathlineto{\pgfqpoint{1.142488in}{1.216607in}}%
\pgfpathlineto{\pgfqpoint{1.131911in}{1.222270in}}%
\pgfpathlineto{\pgfqpoint{1.124657in}{1.226773in}}%
\pgfpathlineto{\pgfqpoint{1.121334in}{1.228968in}}%
\pgfpathlineto{\pgfqpoint{1.110757in}{1.236913in}}%
\pgfpathlineto{\pgfqpoint{1.110567in}{1.237068in}}%
\pgfpathlineto{\pgfqpoint{1.100180in}{1.245543in}}%
\pgfpathlineto{\pgfqpoint{1.098003in}{1.247362in}}%
\pgfpathlineto{\pgfqpoint{1.089603in}{1.254269in}}%
\pgfpathlineto{\pgfqpoint{1.085938in}{1.257657in}}%
\pgfpathlineto{\pgfqpoint{1.079026in}{1.263824in}}%
\pgfpathlineto{\pgfqpoint{1.074654in}{1.267951in}}%
\pgfpathlineto{\pgfqpoint{1.068449in}{1.273739in}}%
\pgfpathlineto{\pgfqpoint{1.063796in}{1.278246in}}%
\pgfpathlineto{\pgfqpoint{1.057872in}{1.284155in}}%
\pgfpathlineto{\pgfqpoint{1.053928in}{1.288540in}}%
\pgfpathlineto{\pgfqpoint{1.047295in}{1.295954in}}%
\pgfpathlineto{\pgfqpoint{1.044007in}{1.298834in}}%
\pgfpathlineto{\pgfqpoint{1.036718in}{1.305522in}}%
\pgfpathlineto{\pgfqpoint{1.031576in}{1.309129in}}%
\pgfpathlineto{\pgfqpoint{1.026141in}{1.312962in}}%
\pgfpathlineto{\pgfqpoint{1.018285in}{1.319423in}}%
\pgfpathlineto{\pgfqpoint{1.015564in}{1.321998in}}%
\pgfpathlineto{\pgfqpoint{1.007471in}{1.329718in}}%
\pgfpathlineto{\pgfqpoint{1.004987in}{1.332238in}}%
\pgfpathlineto{\pgfqpoint{0.997165in}{1.340012in}}%
\pgfpathlineto{\pgfqpoint{0.994410in}{1.342844in}}%
\pgfpathlineto{\pgfqpoint{0.987213in}{1.350307in}}%
\pgfpathlineto{\pgfqpoint{0.983833in}{1.353836in}}%
\pgfpathlineto{\pgfqpoint{0.977422in}{1.360601in}}%
\pgfpathlineto{\pgfqpoint{0.973255in}{1.365040in}}%
\pgfpathlineto{\pgfqpoint{0.967827in}{1.370896in}}%
\pgfpathlineto{\pgfqpoint{0.962678in}{1.376487in}}%
\pgfpathlineto{\pgfqpoint{0.958394in}{1.381190in}}%
\pgfpathlineto{\pgfqpoint{0.952101in}{1.388126in}}%
\pgfpathlineto{\pgfqpoint{0.949085in}{1.391484in}}%
\pgfpathlineto{\pgfqpoint{0.941524in}{1.399933in}}%
\pgfpathlineto{\pgfqpoint{0.940180in}{1.401779in}}%
\pgfpathlineto{\pgfqpoint{0.933623in}{1.412073in}}%
\pgfpathlineto{\pgfqpoint{0.930947in}{1.416754in}}%
\pgfpathlineto{\pgfqpoint{0.927951in}{1.422368in}}%
\pgfpathlineto{\pgfqpoint{0.922526in}{1.432662in}}%
\pgfpathlineto{\pgfqpoint{0.920370in}{1.436792in}}%
\pgfpathlineto{\pgfqpoint{0.917223in}{1.442957in}}%
\pgfpathlineto{\pgfqpoint{0.912000in}{1.453251in}}%
\pgfpathlineto{\pgfqpoint{0.909793in}{1.457620in}}%
\pgfpathlineto{\pgfqpoint{0.906849in}{1.463545in}}%
\pgfpathlineto{\pgfqpoint{0.901746in}{1.473840in}}%
\pgfpathlineto{\pgfqpoint{0.899216in}{1.479324in}}%
\pgfpathlineto{\pgfqpoint{0.897044in}{1.484134in}}%
\pgfpathlineto{\pgfqpoint{0.892490in}{1.494429in}}%
\pgfpathlineto{\pgfqpoint{0.888639in}{1.503154in}}%
\pgfpathlineto{\pgfqpoint{0.887898in}{1.504723in}}%
\pgfpathlineto{\pgfqpoint{0.882839in}{1.515018in}}%
\pgfpathlineto{\pgfqpoint{0.878062in}{1.515018in}}%
\pgfpathlineto{\pgfqpoint{0.867485in}{1.515018in}}%
\pgfpathlineto{\pgfqpoint{0.856908in}{1.515018in}}%
\pgfpathlineto{\pgfqpoint{0.846331in}{1.515018in}}%
\pgfpathlineto{\pgfqpoint{0.842979in}{1.515018in}}%
\pgfpathlineto{\pgfqpoint{0.846331in}{1.507978in}}%
\pgfpathlineto{\pgfqpoint{0.847892in}{1.504723in}}%
\pgfpathlineto{\pgfqpoint{0.852873in}{1.494429in}}%
\pgfpathlineto{\pgfqpoint{0.856908in}{1.486163in}}%
\pgfpathlineto{\pgfqpoint{0.857906in}{1.484134in}}%
\pgfpathlineto{\pgfqpoint{0.863135in}{1.473840in}}%
\pgfpathlineto{\pgfqpoint{0.867485in}{1.465387in}}%
\pgfpathlineto{\pgfqpoint{0.868466in}{1.463545in}}%
\pgfpathlineto{\pgfqpoint{0.874212in}{1.453251in}}%
\pgfpathlineto{\pgfqpoint{0.878062in}{1.445945in}}%
\pgfpathlineto{\pgfqpoint{0.879590in}{1.442957in}}%
\pgfpathlineto{\pgfqpoint{0.884787in}{1.432662in}}%
\pgfpathlineto{\pgfqpoint{0.888639in}{1.425133in}}%
\pgfpathlineto{\pgfqpoint{0.890083in}{1.422368in}}%
\pgfpathlineto{\pgfqpoint{0.895422in}{1.412073in}}%
\pgfpathlineto{\pgfqpoint{0.899216in}{1.404809in}}%
\pgfpathlineto{\pgfqpoint{0.900825in}{1.401779in}}%
\pgfpathlineto{\pgfqpoint{0.906331in}{1.391484in}}%
\pgfpathlineto{\pgfqpoint{0.909793in}{1.385070in}}%
\pgfpathlineto{\pgfqpoint{0.911939in}{1.381190in}}%
\pgfpathlineto{\pgfqpoint{0.917768in}{1.370896in}}%
\pgfpathlineto{\pgfqpoint{0.920370in}{1.366946in}}%
\pgfpathlineto{\pgfqpoint{0.925566in}{1.360601in}}%
\pgfpathlineto{\pgfqpoint{0.930947in}{1.354393in}}%
\pgfpathlineto{\pgfqpoint{0.934542in}{1.350307in}}%
\pgfpathlineto{\pgfqpoint{0.941524in}{1.342495in}}%
\pgfpathlineto{\pgfqpoint{0.943869in}{1.340012in}}%
\pgfpathlineto{\pgfqpoint{0.952101in}{1.331266in}}%
\pgfpathlineto{\pgfqpoint{0.953613in}{1.329718in}}%
\pgfpathlineto{\pgfqpoint{0.962678in}{1.320483in}}%
\pgfpathlineto{\pgfqpoint{0.963732in}{1.319423in}}%
\pgfpathlineto{\pgfqpoint{0.973255in}{1.309881in}}%
\pgfpathlineto{\pgfqpoint{0.974032in}{1.309129in}}%
\pgfpathlineto{\pgfqpoint{0.983833in}{1.299456in}}%
\pgfpathlineto{\pgfqpoint{0.984483in}{1.298834in}}%
\pgfpathlineto{\pgfqpoint{0.994410in}{1.289170in}}%
\pgfpathlineto{\pgfqpoint{0.995146in}{1.288540in}}%
\pgfpathlineto{\pgfqpoint{1.004987in}{1.280911in}}%
\pgfpathlineto{\pgfqpoint{1.008649in}{1.278246in}}%
\pgfpathlineto{\pgfqpoint{1.015564in}{1.273229in}}%
\pgfpathlineto{\pgfqpoint{1.022984in}{1.267951in}}%
\pgfpathlineto{\pgfqpoint{1.026141in}{1.265713in}}%
\pgfpathlineto{\pgfqpoint{1.036718in}{1.258332in}}%
\pgfpathlineto{\pgfqpoint{1.037627in}{1.257657in}}%
\pgfpathlineto{\pgfqpoint{1.047295in}{1.251015in}}%
\pgfpathlineto{\pgfqpoint{1.050973in}{1.247362in}}%
\pgfpathlineto{\pgfqpoint{1.057872in}{1.240536in}}%
\pgfpathlineto{\pgfqpoint{1.061469in}{1.237068in}}%
\pgfpathlineto{\pgfqpoint{1.068449in}{1.230343in}}%
\pgfpathlineto{\pgfqpoint{1.072235in}{1.226773in}}%
\pgfpathlineto{\pgfqpoint{1.079026in}{1.220561in}}%
\pgfpathlineto{\pgfqpoint{1.083822in}{1.216479in}}%
\pgfpathlineto{\pgfqpoint{1.089603in}{1.211694in}}%
\pgfpathlineto{\pgfqpoint{1.096208in}{1.206185in}}%
\pgfpathlineto{\pgfqpoint{1.100180in}{1.202923in}}%
\pgfpathlineto{\pgfqpoint{1.109462in}{1.195890in}}%
\pgfpathlineto{\pgfqpoint{1.110757in}{1.194900in}}%
\pgfpathlineto{\pgfqpoint{1.121334in}{1.187549in}}%
\pgfpathlineto{\pgfqpoint{1.124429in}{1.185596in}}%
\pgfpathlineto{\pgfqpoint{1.131911in}{1.181032in}}%
\pgfpathlineto{\pgfqpoint{1.140530in}{1.175301in}}%
\pgfpathlineto{\pgfqpoint{1.142488in}{1.173975in}}%
\pgfpathlineto{\pgfqpoint{1.153066in}{1.168031in}}%
\pgfpathlineto{\pgfqpoint{1.159498in}{1.165007in}}%
\pgfpathlineto{\pgfqpoint{1.163643in}{1.163069in}}%
\pgfpathlineto{\pgfqpoint{1.174220in}{1.157443in}}%
\pgfpathlineto{\pgfqpoint{1.179543in}{1.154712in}}%
\pgfpathlineto{\pgfqpoint{1.184797in}{1.152036in}}%
\pgfpathlineto{\pgfqpoint{1.195374in}{1.147245in}}%
\pgfpathlineto{\pgfqpoint{1.204576in}{1.144418in}}%
\pgfpathclose%
\pgfusepath{fill}%
\end{pgfscope}%
\begin{pgfscope}%
\pgfpathrectangle{\pgfqpoint{0.423750in}{0.423750in}}{\pgfqpoint{1.194205in}{1.163386in}}%
\pgfusepath{clip}%
\pgfsetbuttcap%
\pgfsetroundjoin%
\definecolor{currentfill}{rgb}{0.949145,0.420383,0.287810}%
\pgfsetfillcolor{currentfill}%
\pgfsetlinewidth{0.000000pt}%
\definecolor{currentstroke}{rgb}{0.000000,0.000000,0.000000}%
\pgfsetstrokecolor{currentstroke}%
\pgfsetdash{}{0pt}%
\pgfpathmoveto{\pgfqpoint{0.782868in}{0.498355in}}%
\pgfpathlineto{\pgfqpoint{0.784609in}{0.495869in}}%
\pgfpathlineto{\pgfqpoint{0.793445in}{0.495869in}}%
\pgfpathlineto{\pgfqpoint{0.804023in}{0.495869in}}%
\pgfpathlineto{\pgfqpoint{0.813321in}{0.495869in}}%
\pgfpathlineto{\pgfqpoint{0.806357in}{0.506163in}}%
\pgfpathlineto{\pgfqpoint{0.804023in}{0.509507in}}%
\pgfpathlineto{\pgfqpoint{0.799026in}{0.516458in}}%
\pgfpathlineto{\pgfqpoint{0.793445in}{0.524192in}}%
\pgfpathlineto{\pgfqpoint{0.791591in}{0.526752in}}%
\pgfpathlineto{\pgfqpoint{0.784088in}{0.537047in}}%
\pgfpathlineto{\pgfqpoint{0.782868in}{0.538697in}}%
\pgfpathlineto{\pgfqpoint{0.776432in}{0.547341in}}%
\pgfpathlineto{\pgfqpoint{0.772291in}{0.552873in}}%
\pgfpathlineto{\pgfqpoint{0.768720in}{0.557635in}}%
\pgfpathlineto{\pgfqpoint{0.761714in}{0.566982in}}%
\pgfpathlineto{\pgfqpoint{0.761019in}{0.567930in}}%
\pgfpathlineto{\pgfqpoint{0.753236in}{0.578224in}}%
\pgfpathlineto{\pgfqpoint{0.751137in}{0.580981in}}%
\pgfpathlineto{\pgfqpoint{0.745480in}{0.588519in}}%
\pgfpathlineto{\pgfqpoint{0.740560in}{0.594898in}}%
\pgfpathlineto{\pgfqpoint{0.737642in}{0.598813in}}%
\pgfpathlineto{\pgfqpoint{0.729983in}{0.608084in}}%
\pgfpathlineto{\pgfqpoint{0.729436in}{0.609108in}}%
\pgfpathlineto{\pgfqpoint{0.723523in}{0.619402in}}%
\pgfpathlineto{\pgfqpoint{0.719406in}{0.625190in}}%
\pgfpathlineto{\pgfqpoint{0.716530in}{0.629696in}}%
\pgfpathlineto{\pgfqpoint{0.709350in}{0.639991in}}%
\pgfpathlineto{\pgfqpoint{0.708829in}{0.640685in}}%
\pgfpathlineto{\pgfqpoint{0.701734in}{0.650285in}}%
\pgfpathlineto{\pgfqpoint{0.698252in}{0.655080in}}%
\pgfpathlineto{\pgfqpoint{0.694544in}{0.660580in}}%
\pgfpathlineto{\pgfqpoint{0.687675in}{0.670475in}}%
\pgfpathlineto{\pgfqpoint{0.687413in}{0.670874in}}%
\pgfpathlineto{\pgfqpoint{0.680615in}{0.681169in}}%
\pgfpathlineto{\pgfqpoint{0.677098in}{0.686476in}}%
\pgfpathlineto{\pgfqpoint{0.673783in}{0.691463in}}%
\pgfpathlineto{\pgfqpoint{0.667049in}{0.701757in}}%
\pgfpathlineto{\pgfqpoint{0.666521in}{0.702613in}}%
\pgfpathlineto{\pgfqpoint{0.660895in}{0.712052in}}%
\pgfpathlineto{\pgfqpoint{0.655944in}{0.719826in}}%
\pgfpathlineto{\pgfqpoint{0.654392in}{0.722346in}}%
\pgfpathlineto{\pgfqpoint{0.648705in}{0.732641in}}%
\pgfpathlineto{\pgfqpoint{0.645367in}{0.739415in}}%
\pgfpathlineto{\pgfqpoint{0.643691in}{0.742935in}}%
\pgfpathlineto{\pgfqpoint{0.638526in}{0.753230in}}%
\pgfpathlineto{\pgfqpoint{0.634790in}{0.760754in}}%
\pgfpathlineto{\pgfqpoint{0.633410in}{0.763524in}}%
\pgfpathlineto{\pgfqpoint{0.628263in}{0.773818in}}%
\pgfpathlineto{\pgfqpoint{0.624212in}{0.782050in}}%
\pgfpathlineto{\pgfqpoint{0.623179in}{0.784113in}}%
\pgfpathlineto{\pgfqpoint{0.619338in}{0.794407in}}%
\pgfpathlineto{\pgfqpoint{0.615848in}{0.804702in}}%
\pgfpathlineto{\pgfqpoint{0.613635in}{0.811149in}}%
\pgfpathlineto{\pgfqpoint{0.612306in}{0.814996in}}%
\pgfpathlineto{\pgfqpoint{0.608713in}{0.825291in}}%
\pgfpathlineto{\pgfqpoint{0.605103in}{0.835585in}}%
\pgfpathlineto{\pgfqpoint{0.603058in}{0.841762in}}%
\pgfpathlineto{\pgfqpoint{0.601656in}{0.845879in}}%
\pgfpathlineto{\pgfqpoint{0.597903in}{0.856174in}}%
\pgfpathlineto{\pgfqpoint{0.594120in}{0.866468in}}%
\pgfpathlineto{\pgfqpoint{0.592481in}{0.870892in}}%
\pgfpathlineto{\pgfqpoint{0.590296in}{0.876763in}}%
\pgfpathlineto{\pgfqpoint{0.586436in}{0.887057in}}%
\pgfpathlineto{\pgfqpoint{0.582547in}{0.897352in}}%
\pgfpathlineto{\pgfqpoint{0.581904in}{0.899042in}}%
\pgfpathlineto{\pgfqpoint{0.578621in}{0.907646in}}%
\pgfpathlineto{\pgfqpoint{0.574669in}{0.917941in}}%
\pgfpathlineto{\pgfqpoint{0.571327in}{0.926594in}}%
\pgfpathlineto{\pgfqpoint{0.570692in}{0.928235in}}%
\pgfpathlineto{\pgfqpoint{0.566687in}{0.938529in}}%
\pgfpathlineto{\pgfqpoint{0.562664in}{0.948824in}}%
\pgfpathlineto{\pgfqpoint{0.560750in}{0.953700in}}%
\pgfpathlineto{\pgfqpoint{0.558620in}{0.959118in}}%
\pgfpathlineto{\pgfqpoint{0.554559in}{0.969413in}}%
\pgfpathlineto{\pgfqpoint{0.550433in}{0.979707in}}%
\pgfpathlineto{\pgfqpoint{0.550173in}{0.980339in}}%
\pgfpathlineto{\pgfqpoint{0.546174in}{0.990002in}}%
\pgfpathlineto{\pgfqpoint{0.541798in}{1.000296in}}%
\pgfpathlineto{\pgfqpoint{0.539596in}{1.005360in}}%
\pgfpathlineto{\pgfqpoint{0.537316in}{1.010590in}}%
\pgfpathlineto{\pgfqpoint{0.533460in}{1.020885in}}%
\pgfpathlineto{\pgfqpoint{0.530022in}{1.031179in}}%
\pgfpathlineto{\pgfqpoint{0.529019in}{1.034152in}}%
\pgfpathlineto{\pgfqpoint{0.526546in}{1.041474in}}%
\pgfpathlineto{\pgfqpoint{0.523030in}{1.051768in}}%
\pgfpathlineto{\pgfqpoint{0.519543in}{1.062063in}}%
\pgfpathlineto{\pgfqpoint{0.518442in}{1.065273in}}%
\pgfpathlineto{\pgfqpoint{0.516010in}{1.072357in}}%
\pgfpathlineto{\pgfqpoint{0.512420in}{1.082651in}}%
\pgfpathlineto{\pgfqpoint{0.508769in}{1.092946in}}%
\pgfpathlineto{\pgfqpoint{0.507865in}{1.095446in}}%
\pgfpathlineto{\pgfqpoint{0.505043in}{1.103240in}}%
\pgfpathlineto{\pgfqpoint{0.501244in}{1.113535in}}%
\pgfpathlineto{\pgfqpoint{0.497378in}{1.123829in}}%
\pgfpathlineto{\pgfqpoint{0.497288in}{1.124067in}}%
\pgfpathlineto{\pgfqpoint{0.497288in}{1.123829in}}%
\pgfpathlineto{\pgfqpoint{0.497288in}{1.113535in}}%
\pgfpathlineto{\pgfqpoint{0.497288in}{1.103240in}}%
\pgfpathlineto{\pgfqpoint{0.497288in}{1.092946in}}%
\pgfpathlineto{\pgfqpoint{0.497288in}{1.082651in}}%
\pgfpathlineto{\pgfqpoint{0.497288in}{1.072357in}}%
\pgfpathlineto{\pgfqpoint{0.497288in}{1.062063in}}%
\pgfpathlineto{\pgfqpoint{0.497288in}{1.051768in}}%
\pgfpathlineto{\pgfqpoint{0.497288in}{1.045345in}}%
\pgfpathlineto{\pgfqpoint{0.498595in}{1.041474in}}%
\pgfpathlineto{\pgfqpoint{0.502089in}{1.031179in}}%
\pgfpathlineto{\pgfqpoint{0.505548in}{1.020885in}}%
\pgfpathlineto{\pgfqpoint{0.507865in}{1.013934in}}%
\pgfpathlineto{\pgfqpoint{0.508993in}{1.010590in}}%
\pgfpathlineto{\pgfqpoint{0.513516in}{1.000296in}}%
\pgfpathlineto{\pgfqpoint{0.517959in}{0.990002in}}%
\pgfpathlineto{\pgfqpoint{0.518442in}{0.988860in}}%
\pgfpathlineto{\pgfqpoint{0.522297in}{0.979707in}}%
\pgfpathlineto{\pgfqpoint{0.526530in}{0.969413in}}%
\pgfpathlineto{\pgfqpoint{0.529019in}{0.963174in}}%
\pgfpathlineto{\pgfqpoint{0.530629in}{0.959118in}}%
\pgfpathlineto{\pgfqpoint{0.534675in}{0.948824in}}%
\pgfpathlineto{\pgfqpoint{0.538710in}{0.938529in}}%
\pgfpathlineto{\pgfqpoint{0.539596in}{0.936259in}}%
\pgfpathlineto{\pgfqpoint{0.542721in}{0.928235in}}%
\pgfpathlineto{\pgfqpoint{0.546712in}{0.917941in}}%
\pgfpathlineto{\pgfqpoint{0.550173in}{0.908973in}}%
\pgfpathlineto{\pgfqpoint{0.550684in}{0.907646in}}%
\pgfpathlineto{\pgfqpoint{0.554623in}{0.897352in}}%
\pgfpathlineto{\pgfqpoint{0.558541in}{0.887057in}}%
\pgfpathlineto{\pgfqpoint{0.560750in}{0.881218in}}%
\pgfpathlineto{\pgfqpoint{0.562430in}{0.876763in}}%
\pgfpathlineto{\pgfqpoint{0.566283in}{0.866468in}}%
\pgfpathlineto{\pgfqpoint{0.570110in}{0.856174in}}%
\pgfpathlineto{\pgfqpoint{0.571327in}{0.852875in}}%
\pgfpathlineto{\pgfqpoint{0.573897in}{0.845879in}}%
\pgfpathlineto{\pgfqpoint{0.577649in}{0.835585in}}%
\pgfpathlineto{\pgfqpoint{0.581575in}{0.825291in}}%
\pgfpathlineto{\pgfqpoint{0.581904in}{0.824415in}}%
\pgfpathlineto{\pgfqpoint{0.585327in}{0.814996in}}%
\pgfpathlineto{\pgfqpoint{0.588931in}{0.804702in}}%
\pgfpathlineto{\pgfqpoint{0.592481in}{0.795825in}}%
\pgfpathlineto{\pgfqpoint{0.593017in}{0.794407in}}%
\pgfpathlineto{\pgfqpoint{0.598913in}{0.784113in}}%
\pgfpathlineto{\pgfqpoint{0.603058in}{0.776447in}}%
\pgfpathlineto{\pgfqpoint{0.604531in}{0.773818in}}%
\pgfpathlineto{\pgfqpoint{0.610690in}{0.763524in}}%
\pgfpathlineto{\pgfqpoint{0.613635in}{0.759053in}}%
\pgfpathlineto{\pgfqpoint{0.617438in}{0.753230in}}%
\pgfpathlineto{\pgfqpoint{0.624035in}{0.742935in}}%
\pgfpathlineto{\pgfqpoint{0.624212in}{0.742663in}}%
\pgfpathlineto{\pgfqpoint{0.630651in}{0.732641in}}%
\pgfpathlineto{\pgfqpoint{0.634790in}{0.726348in}}%
\pgfpathlineto{\pgfqpoint{0.637375in}{0.722346in}}%
\pgfpathlineto{\pgfqpoint{0.644204in}{0.712052in}}%
\pgfpathlineto{\pgfqpoint{0.645367in}{0.710306in}}%
\pgfpathlineto{\pgfqpoint{0.651057in}{0.701757in}}%
\pgfpathlineto{\pgfqpoint{0.655944in}{0.694374in}}%
\pgfpathlineto{\pgfqpoint{0.657869in}{0.691463in}}%
\pgfpathlineto{\pgfqpoint{0.664663in}{0.681169in}}%
\pgfpathlineto{\pgfqpoint{0.666521in}{0.678332in}}%
\pgfpathlineto{\pgfqpoint{0.671399in}{0.670874in}}%
\pgfpathlineto{\pgfqpoint{0.677098in}{0.662130in}}%
\pgfpathlineto{\pgfqpoint{0.678107in}{0.660580in}}%
\pgfpathlineto{\pgfqpoint{0.684759in}{0.650285in}}%
\pgfpathlineto{\pgfqpoint{0.687675in}{0.645806in}}%
\pgfpathlineto{\pgfqpoint{0.691567in}{0.639991in}}%
\pgfpathlineto{\pgfqpoint{0.698252in}{0.629775in}}%
\pgfpathlineto{\pgfqpoint{0.698309in}{0.629696in}}%
\pgfpathlineto{\pgfqpoint{0.705533in}{0.619402in}}%
\pgfpathlineto{\pgfqpoint{0.708829in}{0.614211in}}%
\pgfpathlineto{\pgfqpoint{0.711893in}{0.609108in}}%
\pgfpathlineto{\pgfqpoint{0.717799in}{0.598813in}}%
\pgfpathlineto{\pgfqpoint{0.719406in}{0.595848in}}%
\pgfpathlineto{\pgfqpoint{0.723217in}{0.588519in}}%
\pgfpathlineto{\pgfqpoint{0.728398in}{0.578224in}}%
\pgfpathlineto{\pgfqpoint{0.729983in}{0.575034in}}%
\pgfpathlineto{\pgfqpoint{0.734674in}{0.567930in}}%
\pgfpathlineto{\pgfqpoint{0.740560in}{0.557759in}}%
\pgfpathlineto{\pgfqpoint{0.740650in}{0.557635in}}%
\pgfpathlineto{\pgfqpoint{0.748049in}{0.547341in}}%
\pgfpathlineto{\pgfqpoint{0.751137in}{0.543037in}}%
\pgfpathlineto{\pgfqpoint{0.755469in}{0.537047in}}%
\pgfpathlineto{\pgfqpoint{0.761714in}{0.528372in}}%
\pgfpathlineto{\pgfqpoint{0.762878in}{0.526752in}}%
\pgfpathlineto{\pgfqpoint{0.770175in}{0.516458in}}%
\pgfpathlineto{\pgfqpoint{0.772291in}{0.513447in}}%
\pgfpathlineto{\pgfqpoint{0.777412in}{0.506163in}}%
\pgfpathclose%
\pgfusepath{fill}%
\end{pgfscope}%
\begin{pgfscope}%
\pgfpathrectangle{\pgfqpoint{0.423750in}{0.423750in}}{\pgfqpoint{1.194205in}{1.163386in}}%
\pgfusepath{clip}%
\pgfsetbuttcap%
\pgfsetroundjoin%
\definecolor{currentfill}{rgb}{0.949145,0.420383,0.287810}%
\pgfsetfillcolor{currentfill}%
\pgfsetlinewidth{0.000000pt}%
\definecolor{currentstroke}{rgb}{0.000000,0.000000,0.000000}%
\pgfsetstrokecolor{currentstroke}%
\pgfsetdash{}{0pt}%
\pgfpathmoveto{\pgfqpoint{1.279990in}{1.102688in}}%
\pgfpathlineto{\pgfqpoint{1.290567in}{1.102106in}}%
\pgfpathlineto{\pgfqpoint{1.301144in}{1.101578in}}%
\pgfpathlineto{\pgfqpoint{1.311721in}{1.101129in}}%
\pgfpathlineto{\pgfqpoint{1.317179in}{1.103240in}}%
\pgfpathlineto{\pgfqpoint{1.322299in}{1.105293in}}%
\pgfpathlineto{\pgfqpoint{1.332876in}{1.111211in}}%
\pgfpathlineto{\pgfqpoint{1.337007in}{1.113535in}}%
\pgfpathlineto{\pgfqpoint{1.343453in}{1.117193in}}%
\pgfpathlineto{\pgfqpoint{1.354030in}{1.123231in}}%
\pgfpathlineto{\pgfqpoint{1.355071in}{1.123829in}}%
\pgfpathlineto{\pgfqpoint{1.364607in}{1.129368in}}%
\pgfpathlineto{\pgfqpoint{1.372734in}{1.134124in}}%
\pgfpathlineto{\pgfqpoint{1.375184in}{1.135573in}}%
\pgfpathlineto{\pgfqpoint{1.385761in}{1.141879in}}%
\pgfpathlineto{\pgfqpoint{1.389931in}{1.144418in}}%
\pgfpathlineto{\pgfqpoint{1.396338in}{1.148319in}}%
\pgfpathlineto{\pgfqpoint{1.406915in}{1.154678in}}%
\pgfpathlineto{\pgfqpoint{1.406980in}{1.154712in}}%
\pgfpathlineto{\pgfqpoint{1.417492in}{1.160563in}}%
\pgfpathlineto{\pgfqpoint{1.425389in}{1.165007in}}%
\pgfpathlineto{\pgfqpoint{1.428069in}{1.166524in}}%
\pgfpathlineto{\pgfqpoint{1.438646in}{1.172582in}}%
\pgfpathlineto{\pgfqpoint{1.443334in}{1.175301in}}%
\pgfpathlineto{\pgfqpoint{1.449223in}{1.178739in}}%
\pgfpathlineto{\pgfqpoint{1.459800in}{1.185000in}}%
\pgfpathlineto{\pgfqpoint{1.460795in}{1.185596in}}%
\pgfpathlineto{\pgfqpoint{1.470377in}{1.191377in}}%
\pgfpathlineto{\pgfqpoint{1.477748in}{1.195890in}}%
\pgfpathlineto{\pgfqpoint{1.480954in}{1.197866in}}%
\pgfpathlineto{\pgfqpoint{1.491532in}{1.204473in}}%
\pgfpathlineto{\pgfqpoint{1.494239in}{1.206185in}}%
\pgfpathlineto{\pgfqpoint{1.502109in}{1.211195in}}%
\pgfpathlineto{\pgfqpoint{1.510293in}{1.216479in}}%
\pgfpathlineto{\pgfqpoint{1.512686in}{1.218034in}}%
\pgfpathlineto{\pgfqpoint{1.523263in}{1.224985in}}%
\pgfpathlineto{\pgfqpoint{1.525953in}{1.226773in}}%
\pgfpathlineto{\pgfqpoint{1.533840in}{1.232055in}}%
\pgfpathlineto{\pgfqpoint{1.541237in}{1.237068in}}%
\pgfpathlineto{\pgfqpoint{1.544417in}{1.239237in}}%
\pgfpathlineto{\pgfqpoint{1.544417in}{1.247362in}}%
\pgfpathlineto{\pgfqpoint{1.544417in}{1.257657in}}%
\pgfpathlineto{\pgfqpoint{1.544417in}{1.267951in}}%
\pgfpathlineto{\pgfqpoint{1.544417in}{1.278246in}}%
\pgfpathlineto{\pgfqpoint{1.544417in}{1.288540in}}%
\pgfpathlineto{\pgfqpoint{1.544417in}{1.290865in}}%
\pgfpathlineto{\pgfqpoint{1.541144in}{1.288540in}}%
\pgfpathlineto{\pgfqpoint{1.533840in}{1.283367in}}%
\pgfpathlineto{\pgfqpoint{1.526502in}{1.278246in}}%
\pgfpathlineto{\pgfqpoint{1.523263in}{1.275992in}}%
\pgfpathlineto{\pgfqpoint{1.512686in}{1.268742in}}%
\pgfpathlineto{\pgfqpoint{1.511513in}{1.267951in}}%
\pgfpathlineto{\pgfqpoint{1.502109in}{1.261627in}}%
\pgfpathlineto{\pgfqpoint{1.496094in}{1.257657in}}%
\pgfpathlineto{\pgfqpoint{1.491532in}{1.254652in}}%
\pgfpathlineto{\pgfqpoint{1.480954in}{1.247816in}}%
\pgfpathlineto{\pgfqpoint{1.480238in}{1.247362in}}%
\pgfpathlineto{\pgfqpoint{1.470377in}{1.241124in}}%
\pgfpathlineto{\pgfqpoint{1.463833in}{1.237068in}}%
\pgfpathlineto{\pgfqpoint{1.459800in}{1.234572in}}%
\pgfpathlineto{\pgfqpoint{1.449223in}{1.228158in}}%
\pgfpathlineto{\pgfqpoint{1.446893in}{1.226773in}}%
\pgfpathlineto{\pgfqpoint{1.438646in}{1.221880in}}%
\pgfpathlineto{\pgfqpoint{1.429384in}{1.216479in}}%
\pgfpathlineto{\pgfqpoint{1.428069in}{1.215714in}}%
\pgfpathlineto{\pgfqpoint{1.417492in}{1.209667in}}%
\pgfpathlineto{\pgfqpoint{1.411314in}{1.206185in}}%
\pgfpathlineto{\pgfqpoint{1.406915in}{1.203712in}}%
\pgfpathlineto{\pgfqpoint{1.396338in}{1.197841in}}%
\pgfpathlineto{\pgfqpoint{1.392770in}{1.195890in}}%
\pgfpathlineto{\pgfqpoint{1.385761in}{1.192038in}}%
\pgfpathlineto{\pgfqpoint{1.375787in}{1.185596in}}%
\pgfpathlineto{\pgfqpoint{1.375184in}{1.185184in}}%
\pgfpathlineto{\pgfqpoint{1.364607in}{1.178042in}}%
\pgfpathlineto{\pgfqpoint{1.360507in}{1.175301in}}%
\pgfpathlineto{\pgfqpoint{1.354030in}{1.171304in}}%
\pgfpathlineto{\pgfqpoint{1.343453in}{1.165072in}}%
\pgfpathlineto{\pgfqpoint{1.343342in}{1.165007in}}%
\pgfpathlineto{\pgfqpoint{1.332876in}{1.158908in}}%
\pgfpathlineto{\pgfqpoint{1.325622in}{1.154712in}}%
\pgfpathlineto{\pgfqpoint{1.322299in}{1.152810in}}%
\pgfpathlineto{\pgfqpoint{1.311721in}{1.147272in}}%
\pgfpathlineto{\pgfqpoint{1.301144in}{1.147871in}}%
\pgfpathlineto{\pgfqpoint{1.290567in}{1.148484in}}%
\pgfpathlineto{\pgfqpoint{1.279990in}{1.148772in}}%
\pgfpathlineto{\pgfqpoint{1.269413in}{1.148550in}}%
\pgfpathlineto{\pgfqpoint{1.258836in}{1.147886in}}%
\pgfpathlineto{\pgfqpoint{1.248259in}{1.147227in}}%
\pgfpathlineto{\pgfqpoint{1.237682in}{1.146356in}}%
\pgfpathlineto{\pgfqpoint{1.227378in}{1.144418in}}%
\pgfpathlineto{\pgfqpoint{1.227105in}{1.144365in}}%
\pgfpathlineto{\pgfqpoint{1.216528in}{1.143055in}}%
\pgfpathlineto{\pgfqpoint{1.205951in}{1.144036in}}%
\pgfpathlineto{\pgfqpoint{1.204576in}{1.144418in}}%
\pgfpathlineto{\pgfqpoint{1.195374in}{1.147245in}}%
\pgfpathlineto{\pgfqpoint{1.184797in}{1.152036in}}%
\pgfpathlineto{\pgfqpoint{1.179543in}{1.154712in}}%
\pgfpathlineto{\pgfqpoint{1.174220in}{1.157443in}}%
\pgfpathlineto{\pgfqpoint{1.163643in}{1.163069in}}%
\pgfpathlineto{\pgfqpoint{1.159498in}{1.165007in}}%
\pgfpathlineto{\pgfqpoint{1.153066in}{1.168031in}}%
\pgfpathlineto{\pgfqpoint{1.142488in}{1.173975in}}%
\pgfpathlineto{\pgfqpoint{1.140530in}{1.175301in}}%
\pgfpathlineto{\pgfqpoint{1.131911in}{1.181032in}}%
\pgfpathlineto{\pgfqpoint{1.124429in}{1.185596in}}%
\pgfpathlineto{\pgfqpoint{1.121334in}{1.187549in}}%
\pgfpathlineto{\pgfqpoint{1.110757in}{1.194900in}}%
\pgfpathlineto{\pgfqpoint{1.109462in}{1.195890in}}%
\pgfpathlineto{\pgfqpoint{1.100180in}{1.202923in}}%
\pgfpathlineto{\pgfqpoint{1.096208in}{1.206185in}}%
\pgfpathlineto{\pgfqpoint{1.089603in}{1.211694in}}%
\pgfpathlineto{\pgfqpoint{1.083822in}{1.216479in}}%
\pgfpathlineto{\pgfqpoint{1.079026in}{1.220561in}}%
\pgfpathlineto{\pgfqpoint{1.072235in}{1.226773in}}%
\pgfpathlineto{\pgfqpoint{1.068449in}{1.230343in}}%
\pgfpathlineto{\pgfqpoint{1.061469in}{1.237068in}}%
\pgfpathlineto{\pgfqpoint{1.057872in}{1.240536in}}%
\pgfpathlineto{\pgfqpoint{1.050973in}{1.247362in}}%
\pgfpathlineto{\pgfqpoint{1.047295in}{1.251015in}}%
\pgfpathlineto{\pgfqpoint{1.037627in}{1.257657in}}%
\pgfpathlineto{\pgfqpoint{1.036718in}{1.258332in}}%
\pgfpathlineto{\pgfqpoint{1.026141in}{1.265713in}}%
\pgfpathlineto{\pgfqpoint{1.022984in}{1.267951in}}%
\pgfpathlineto{\pgfqpoint{1.015564in}{1.273229in}}%
\pgfpathlineto{\pgfqpoint{1.008649in}{1.278246in}}%
\pgfpathlineto{\pgfqpoint{1.004987in}{1.280911in}}%
\pgfpathlineto{\pgfqpoint{0.995146in}{1.288540in}}%
\pgfpathlineto{\pgfqpoint{0.994410in}{1.289170in}}%
\pgfpathlineto{\pgfqpoint{0.984483in}{1.298834in}}%
\pgfpathlineto{\pgfqpoint{0.983833in}{1.299456in}}%
\pgfpathlineto{\pgfqpoint{0.974032in}{1.309129in}}%
\pgfpathlineto{\pgfqpoint{0.973255in}{1.309881in}}%
\pgfpathlineto{\pgfqpoint{0.963732in}{1.319423in}}%
\pgfpathlineto{\pgfqpoint{0.962678in}{1.320483in}}%
\pgfpathlineto{\pgfqpoint{0.953613in}{1.329718in}}%
\pgfpathlineto{\pgfqpoint{0.952101in}{1.331266in}}%
\pgfpathlineto{\pgfqpoint{0.943869in}{1.340012in}}%
\pgfpathlineto{\pgfqpoint{0.941524in}{1.342495in}}%
\pgfpathlineto{\pgfqpoint{0.934542in}{1.350307in}}%
\pgfpathlineto{\pgfqpoint{0.930947in}{1.354393in}}%
\pgfpathlineto{\pgfqpoint{0.925566in}{1.360601in}}%
\pgfpathlineto{\pgfqpoint{0.920370in}{1.366946in}}%
\pgfpathlineto{\pgfqpoint{0.917768in}{1.370896in}}%
\pgfpathlineto{\pgfqpoint{0.911939in}{1.381190in}}%
\pgfpathlineto{\pgfqpoint{0.909793in}{1.385070in}}%
\pgfpathlineto{\pgfqpoint{0.906331in}{1.391484in}}%
\pgfpathlineto{\pgfqpoint{0.900825in}{1.401779in}}%
\pgfpathlineto{\pgfqpoint{0.899216in}{1.404809in}}%
\pgfpathlineto{\pgfqpoint{0.895422in}{1.412073in}}%
\pgfpathlineto{\pgfqpoint{0.890083in}{1.422368in}}%
\pgfpathlineto{\pgfqpoint{0.888639in}{1.425133in}}%
\pgfpathlineto{\pgfqpoint{0.884787in}{1.432662in}}%
\pgfpathlineto{\pgfqpoint{0.879590in}{1.442957in}}%
\pgfpathlineto{\pgfqpoint{0.878062in}{1.445945in}}%
\pgfpathlineto{\pgfqpoint{0.874212in}{1.453251in}}%
\pgfpathlineto{\pgfqpoint{0.868466in}{1.463545in}}%
\pgfpathlineto{\pgfqpoint{0.867485in}{1.465387in}}%
\pgfpathlineto{\pgfqpoint{0.863135in}{1.473840in}}%
\pgfpathlineto{\pgfqpoint{0.857906in}{1.484134in}}%
\pgfpathlineto{\pgfqpoint{0.856908in}{1.486163in}}%
\pgfpathlineto{\pgfqpoint{0.852873in}{1.494429in}}%
\pgfpathlineto{\pgfqpoint{0.847892in}{1.504723in}}%
\pgfpathlineto{\pgfqpoint{0.846331in}{1.507978in}}%
\pgfpathlineto{\pgfqpoint{0.842979in}{1.515018in}}%
\pgfpathlineto{\pgfqpoint{0.835754in}{1.515018in}}%
\pgfpathlineto{\pgfqpoint{0.825177in}{1.515018in}}%
\pgfpathlineto{\pgfqpoint{0.814600in}{1.515018in}}%
\pgfpathlineto{\pgfqpoint{0.804887in}{1.515018in}}%
\pgfpathlineto{\pgfqpoint{0.810171in}{1.504723in}}%
\pgfpathlineto{\pgfqpoint{0.814600in}{1.496263in}}%
\pgfpathlineto{\pgfqpoint{0.815557in}{1.494429in}}%
\pgfpathlineto{\pgfqpoint{0.820964in}{1.484134in}}%
\pgfpathlineto{\pgfqpoint{0.825177in}{1.476158in}}%
\pgfpathlineto{\pgfqpoint{0.826353in}{1.473840in}}%
\pgfpathlineto{\pgfqpoint{0.831909in}{1.463545in}}%
\pgfpathlineto{\pgfqpoint{0.835754in}{1.456648in}}%
\pgfpathlineto{\pgfqpoint{0.837647in}{1.453251in}}%
\pgfpathlineto{\pgfqpoint{0.843405in}{1.442957in}}%
\pgfpathlineto{\pgfqpoint{0.846331in}{1.437745in}}%
\pgfpathlineto{\pgfqpoint{0.849209in}{1.432662in}}%
\pgfpathlineto{\pgfqpoint{0.855061in}{1.422368in}}%
\pgfpathlineto{\pgfqpoint{0.856908in}{1.419132in}}%
\pgfpathlineto{\pgfqpoint{0.860903in}{1.412073in}}%
\pgfpathlineto{\pgfqpoint{0.866551in}{1.401779in}}%
\pgfpathlineto{\pgfqpoint{0.867485in}{1.400088in}}%
\pgfpathlineto{\pgfqpoint{0.871969in}{1.391484in}}%
\pgfpathlineto{\pgfqpoint{0.877581in}{1.381190in}}%
\pgfpathlineto{\pgfqpoint{0.878062in}{1.380315in}}%
\pgfpathlineto{\pgfqpoint{0.883306in}{1.370896in}}%
\pgfpathlineto{\pgfqpoint{0.888639in}{1.361413in}}%
\pgfpathlineto{\pgfqpoint{0.889103in}{1.360601in}}%
\pgfpathlineto{\pgfqpoint{0.895065in}{1.350307in}}%
\pgfpathlineto{\pgfqpoint{0.899216in}{1.343296in}}%
\pgfpathlineto{\pgfqpoint{0.901315in}{1.340012in}}%
\pgfpathlineto{\pgfqpoint{0.909005in}{1.329718in}}%
\pgfpathlineto{\pgfqpoint{0.909793in}{1.328774in}}%
\pgfpathlineto{\pgfqpoint{0.918081in}{1.319423in}}%
\pgfpathlineto{\pgfqpoint{0.920370in}{1.316853in}}%
\pgfpathlineto{\pgfqpoint{0.927331in}{1.309129in}}%
\pgfpathlineto{\pgfqpoint{0.930947in}{1.305138in}}%
\pgfpathlineto{\pgfqpoint{0.936733in}{1.298834in}}%
\pgfpathlineto{\pgfqpoint{0.941524in}{1.293646in}}%
\pgfpathlineto{\pgfqpoint{0.946302in}{1.288540in}}%
\pgfpathlineto{\pgfqpoint{0.952101in}{1.282379in}}%
\pgfpathlineto{\pgfqpoint{0.956223in}{1.278246in}}%
\pgfpathlineto{\pgfqpoint{0.962678in}{1.271743in}}%
\pgfpathlineto{\pgfqpoint{0.966545in}{1.267951in}}%
\pgfpathlineto{\pgfqpoint{0.973255in}{1.261632in}}%
\pgfpathlineto{\pgfqpoint{0.978041in}{1.257657in}}%
\pgfpathlineto{\pgfqpoint{0.983833in}{1.253119in}}%
\pgfpathlineto{\pgfqpoint{0.991257in}{1.247362in}}%
\pgfpathlineto{\pgfqpoint{0.994410in}{1.244928in}}%
\pgfpathlineto{\pgfqpoint{1.004710in}{1.237068in}}%
\pgfpathlineto{\pgfqpoint{1.004987in}{1.236858in}}%
\pgfpathlineto{\pgfqpoint{1.015564in}{1.228908in}}%
\pgfpathlineto{\pgfqpoint{1.018427in}{1.226773in}}%
\pgfpathlineto{\pgfqpoint{1.026141in}{1.221029in}}%
\pgfpathlineto{\pgfqpoint{1.032702in}{1.216479in}}%
\pgfpathlineto{\pgfqpoint{1.036718in}{1.213765in}}%
\pgfpathlineto{\pgfqpoint{1.047295in}{1.206831in}}%
\pgfpathlineto{\pgfqpoint{1.048073in}{1.206185in}}%
\pgfpathlineto{\pgfqpoint{1.057872in}{1.198661in}}%
\pgfpathlineto{\pgfqpoint{1.060880in}{1.195890in}}%
\pgfpathlineto{\pgfqpoint{1.068449in}{1.188631in}}%
\pgfpathlineto{\pgfqpoint{1.072144in}{1.185596in}}%
\pgfpathlineto{\pgfqpoint{1.079026in}{1.179664in}}%
\pgfpathlineto{\pgfqpoint{1.083987in}{1.175301in}}%
\pgfpathlineto{\pgfqpoint{1.089603in}{1.170551in}}%
\pgfpathlineto{\pgfqpoint{1.096824in}{1.165007in}}%
\pgfpathlineto{\pgfqpoint{1.100180in}{1.162569in}}%
\pgfpathlineto{\pgfqpoint{1.110757in}{1.155699in}}%
\pgfpathlineto{\pgfqpoint{1.112345in}{1.154712in}}%
\pgfpathlineto{\pgfqpoint{1.121334in}{1.148921in}}%
\pgfpathlineto{\pgfqpoint{1.128657in}{1.144418in}}%
\pgfpathlineto{\pgfqpoint{1.131911in}{1.142574in}}%
\pgfpathlineto{\pgfqpoint{1.142488in}{1.136737in}}%
\pgfpathlineto{\pgfqpoint{1.147263in}{1.134124in}}%
\pgfpathlineto{\pgfqpoint{1.153066in}{1.130974in}}%
\pgfpathlineto{\pgfqpoint{1.163643in}{1.125275in}}%
\pgfpathlineto{\pgfqpoint{1.166262in}{1.123829in}}%
\pgfpathlineto{\pgfqpoint{1.174220in}{1.119092in}}%
\pgfpathlineto{\pgfqpoint{1.184797in}{1.113747in}}%
\pgfpathlineto{\pgfqpoint{1.185368in}{1.113535in}}%
\pgfpathlineto{\pgfqpoint{1.195374in}{1.109000in}}%
\pgfpathlineto{\pgfqpoint{1.205951in}{1.106615in}}%
\pgfpathlineto{\pgfqpoint{1.216528in}{1.105956in}}%
\pgfpathlineto{\pgfqpoint{1.227105in}{1.105763in}}%
\pgfpathlineto{\pgfqpoint{1.237682in}{1.105351in}}%
\pgfpathlineto{\pgfqpoint{1.248259in}{1.104705in}}%
\pgfpathlineto{\pgfqpoint{1.258836in}{1.104019in}}%
\pgfpathlineto{\pgfqpoint{1.269413in}{1.103324in}}%
\pgfpathlineto{\pgfqpoint{1.270785in}{1.103240in}}%
\pgfpathclose%
\pgfusepath{fill}%
\end{pgfscope}%
\begin{pgfscope}%
\pgfpathrectangle{\pgfqpoint{0.423750in}{0.423750in}}{\pgfqpoint{1.194205in}{1.163386in}}%
\pgfusepath{clip}%
\pgfsetbuttcap%
\pgfsetroundjoin%
\definecolor{currentfill}{rgb}{0.957344,0.505732,0.351309}%
\pgfsetfillcolor{currentfill}%
\pgfsetlinewidth{0.000000pt}%
\definecolor{currentstroke}{rgb}{0.000000,0.000000,0.000000}%
\pgfsetstrokecolor{currentstroke}%
\pgfsetdash{}{0pt}%
\pgfpathmoveto{\pgfqpoint{0.814600in}{0.495869in}}%
\pgfpathlineto{\pgfqpoint{0.825177in}{0.495869in}}%
\pgfpathlineto{\pgfqpoint{0.835754in}{0.495869in}}%
\pgfpathlineto{\pgfqpoint{0.840798in}{0.495869in}}%
\pgfpathlineto{\pgfqpoint{0.835754in}{0.503665in}}%
\pgfpathlineto{\pgfqpoint{0.834099in}{0.506163in}}%
\pgfpathlineto{\pgfqpoint{0.827154in}{0.516458in}}%
\pgfpathlineto{\pgfqpoint{0.825177in}{0.519314in}}%
\pgfpathlineto{\pgfqpoint{0.819989in}{0.526752in}}%
\pgfpathlineto{\pgfqpoint{0.814600in}{0.534290in}}%
\pgfpathlineto{\pgfqpoint{0.812579in}{0.537047in}}%
\pgfpathlineto{\pgfqpoint{0.805049in}{0.547341in}}%
\pgfpathlineto{\pgfqpoint{0.804023in}{0.548672in}}%
\pgfpathlineto{\pgfqpoint{0.797281in}{0.557635in}}%
\pgfpathlineto{\pgfqpoint{0.793445in}{0.562721in}}%
\pgfpathlineto{\pgfqpoint{0.789486in}{0.567930in}}%
\pgfpathlineto{\pgfqpoint{0.782868in}{0.576636in}}%
\pgfpathlineto{\pgfqpoint{0.781655in}{0.578224in}}%
\pgfpathlineto{\pgfqpoint{0.774255in}{0.588519in}}%
\pgfpathlineto{\pgfqpoint{0.772291in}{0.591093in}}%
\pgfpathlineto{\pgfqpoint{0.766784in}{0.598813in}}%
\pgfpathlineto{\pgfqpoint{0.761714in}{0.605240in}}%
\pgfpathlineto{\pgfqpoint{0.758850in}{0.609108in}}%
\pgfpathlineto{\pgfqpoint{0.751137in}{0.618184in}}%
\pgfpathlineto{\pgfqpoint{0.750133in}{0.619402in}}%
\pgfpathlineto{\pgfqpoint{0.741519in}{0.629696in}}%
\pgfpathlineto{\pgfqpoint{0.740560in}{0.630852in}}%
\pgfpathlineto{\pgfqpoint{0.732481in}{0.639991in}}%
\pgfpathlineto{\pgfqpoint{0.729983in}{0.642203in}}%
\pgfpathlineto{\pgfqpoint{0.722433in}{0.650285in}}%
\pgfpathlineto{\pgfqpoint{0.719406in}{0.652560in}}%
\pgfpathlineto{\pgfqpoint{0.714217in}{0.660580in}}%
\pgfpathlineto{\pgfqpoint{0.708829in}{0.668254in}}%
\pgfpathlineto{\pgfqpoint{0.707152in}{0.670874in}}%
\pgfpathlineto{\pgfqpoint{0.700434in}{0.681169in}}%
\pgfpathlineto{\pgfqpoint{0.698252in}{0.684894in}}%
\pgfpathlineto{\pgfqpoint{0.694773in}{0.691463in}}%
\pgfpathlineto{\pgfqpoint{0.689020in}{0.701757in}}%
\pgfpathlineto{\pgfqpoint{0.687675in}{0.704081in}}%
\pgfpathlineto{\pgfqpoint{0.683971in}{0.712052in}}%
\pgfpathlineto{\pgfqpoint{0.679009in}{0.722346in}}%
\pgfpathlineto{\pgfqpoint{0.677098in}{0.726153in}}%
\pgfpathlineto{\pgfqpoint{0.673959in}{0.732641in}}%
\pgfpathlineto{\pgfqpoint{0.668954in}{0.742935in}}%
\pgfpathlineto{\pgfqpoint{0.666521in}{0.747918in}}%
\pgfpathlineto{\pgfqpoint{0.663909in}{0.753230in}}%
\pgfpathlineto{\pgfqpoint{0.658908in}{0.763524in}}%
\pgfpathlineto{\pgfqpoint{0.655944in}{0.771046in}}%
\pgfpathlineto{\pgfqpoint{0.654778in}{0.773818in}}%
\pgfpathlineto{\pgfqpoint{0.651071in}{0.784113in}}%
\pgfpathlineto{\pgfqpoint{0.647340in}{0.794407in}}%
\pgfpathlineto{\pgfqpoint{0.645367in}{0.799823in}}%
\pgfpathlineto{\pgfqpoint{0.643552in}{0.804702in}}%
\pgfpathlineto{\pgfqpoint{0.639551in}{0.814996in}}%
\pgfpathlineto{\pgfqpoint{0.635472in}{0.825291in}}%
\pgfpathlineto{\pgfqpoint{0.634790in}{0.827118in}}%
\pgfpathlineto{\pgfqpoint{0.631857in}{0.835585in}}%
\pgfpathlineto{\pgfqpoint{0.628437in}{0.845879in}}%
\pgfpathlineto{\pgfqpoint{0.625050in}{0.856174in}}%
\pgfpathlineto{\pgfqpoint{0.624212in}{0.858701in}}%
\pgfpathlineto{\pgfqpoint{0.621575in}{0.866468in}}%
\pgfpathlineto{\pgfqpoint{0.617844in}{0.876763in}}%
\pgfpathlineto{\pgfqpoint{0.614049in}{0.887057in}}%
\pgfpathlineto{\pgfqpoint{0.613635in}{0.888170in}}%
\pgfpathlineto{\pgfqpoint{0.610205in}{0.897352in}}%
\pgfpathlineto{\pgfqpoint{0.606330in}{0.907646in}}%
\pgfpathlineto{\pgfqpoint{0.603058in}{0.916276in}}%
\pgfpathlineto{\pgfqpoint{0.602424in}{0.917941in}}%
\pgfpathlineto{\pgfqpoint{0.598480in}{0.928235in}}%
\pgfpathlineto{\pgfqpoint{0.594512in}{0.938529in}}%
\pgfpathlineto{\pgfqpoint{0.592481in}{0.943769in}}%
\pgfpathlineto{\pgfqpoint{0.590516in}{0.948824in}}%
\pgfpathlineto{\pgfqpoint{0.586496in}{0.959118in}}%
\pgfpathlineto{\pgfqpoint{0.582457in}{0.969413in}}%
\pgfpathlineto{\pgfqpoint{0.581904in}{0.970816in}}%
\pgfpathlineto{\pgfqpoint{0.578396in}{0.979707in}}%
\pgfpathlineto{\pgfqpoint{0.574270in}{0.990002in}}%
\pgfpathlineto{\pgfqpoint{0.571327in}{0.997114in}}%
\pgfpathlineto{\pgfqpoint{0.570003in}{1.000296in}}%
\pgfpathlineto{\pgfqpoint{0.565603in}{1.010590in}}%
\pgfpathlineto{\pgfqpoint{0.561354in}{1.020885in}}%
\pgfpathlineto{\pgfqpoint{0.560750in}{1.022695in}}%
\pgfpathlineto{\pgfqpoint{0.557924in}{1.031179in}}%
\pgfpathlineto{\pgfqpoint{0.554451in}{1.041474in}}%
\pgfpathlineto{\pgfqpoint{0.550927in}{1.051768in}}%
\pgfpathlineto{\pgfqpoint{0.550173in}{1.053943in}}%
\pgfpathlineto{\pgfqpoint{0.547371in}{1.062063in}}%
\pgfpathlineto{\pgfqpoint{0.543823in}{1.072357in}}%
\pgfpathlineto{\pgfqpoint{0.540224in}{1.082651in}}%
\pgfpathlineto{\pgfqpoint{0.539596in}{1.084474in}}%
\pgfpathlineto{\pgfqpoint{0.536688in}{1.092946in}}%
\pgfpathlineto{\pgfqpoint{0.533094in}{1.103240in}}%
\pgfpathlineto{\pgfqpoint{0.529455in}{1.113535in}}%
\pgfpathlineto{\pgfqpoint{0.529019in}{1.114744in}}%
\pgfpathlineto{\pgfqpoint{0.525747in}{1.123829in}}%
\pgfpathlineto{\pgfqpoint{0.521966in}{1.134124in}}%
\pgfpathlineto{\pgfqpoint{0.518442in}{1.143539in}}%
\pgfpathlineto{\pgfqpoint{0.518113in}{1.144418in}}%
\pgfpathlineto{\pgfqpoint{0.514309in}{1.154712in}}%
\pgfpathlineto{\pgfqpoint{0.511153in}{1.165007in}}%
\pgfpathlineto{\pgfqpoint{0.508087in}{1.175301in}}%
\pgfpathlineto{\pgfqpoint{0.507865in}{1.176040in}}%
\pgfpathlineto{\pgfqpoint{0.504991in}{1.185596in}}%
\pgfpathlineto{\pgfqpoint{0.501874in}{1.195890in}}%
\pgfpathlineto{\pgfqpoint{0.498734in}{1.206185in}}%
\pgfpathlineto{\pgfqpoint{0.497288in}{1.210877in}}%
\pgfpathlineto{\pgfqpoint{0.497288in}{1.206185in}}%
\pgfpathlineto{\pgfqpoint{0.497288in}{1.195890in}}%
\pgfpathlineto{\pgfqpoint{0.497288in}{1.185596in}}%
\pgfpathlineto{\pgfqpoint{0.497288in}{1.175301in}}%
\pgfpathlineto{\pgfqpoint{0.497288in}{1.165007in}}%
\pgfpathlineto{\pgfqpoint{0.497288in}{1.154712in}}%
\pgfpathlineto{\pgfqpoint{0.497288in}{1.144418in}}%
\pgfpathlineto{\pgfqpoint{0.497288in}{1.134124in}}%
\pgfpathlineto{\pgfqpoint{0.497288in}{1.124067in}}%
\pgfpathlineto{\pgfqpoint{0.497378in}{1.123829in}}%
\pgfpathlineto{\pgfqpoint{0.501244in}{1.113535in}}%
\pgfpathlineto{\pgfqpoint{0.505043in}{1.103240in}}%
\pgfpathlineto{\pgfqpoint{0.507865in}{1.095446in}}%
\pgfpathlineto{\pgfqpoint{0.508769in}{1.092946in}}%
\pgfpathlineto{\pgfqpoint{0.512420in}{1.082651in}}%
\pgfpathlineto{\pgfqpoint{0.516010in}{1.072357in}}%
\pgfpathlineto{\pgfqpoint{0.518442in}{1.065273in}}%
\pgfpathlineto{\pgfqpoint{0.519543in}{1.062063in}}%
\pgfpathlineto{\pgfqpoint{0.523030in}{1.051768in}}%
\pgfpathlineto{\pgfqpoint{0.526546in}{1.041474in}}%
\pgfpathlineto{\pgfqpoint{0.529019in}{1.034152in}}%
\pgfpathlineto{\pgfqpoint{0.530022in}{1.031179in}}%
\pgfpathlineto{\pgfqpoint{0.533460in}{1.020885in}}%
\pgfpathlineto{\pgfqpoint{0.537316in}{1.010590in}}%
\pgfpathlineto{\pgfqpoint{0.539596in}{1.005360in}}%
\pgfpathlineto{\pgfqpoint{0.541798in}{1.000296in}}%
\pgfpathlineto{\pgfqpoint{0.546174in}{0.990002in}}%
\pgfpathlineto{\pgfqpoint{0.550173in}{0.980339in}}%
\pgfpathlineto{\pgfqpoint{0.550433in}{0.979707in}}%
\pgfpathlineto{\pgfqpoint{0.554559in}{0.969413in}}%
\pgfpathlineto{\pgfqpoint{0.558620in}{0.959118in}}%
\pgfpathlineto{\pgfqpoint{0.560750in}{0.953700in}}%
\pgfpathlineto{\pgfqpoint{0.562664in}{0.948824in}}%
\pgfpathlineto{\pgfqpoint{0.566687in}{0.938529in}}%
\pgfpathlineto{\pgfqpoint{0.570692in}{0.928235in}}%
\pgfpathlineto{\pgfqpoint{0.571327in}{0.926594in}}%
\pgfpathlineto{\pgfqpoint{0.574669in}{0.917941in}}%
\pgfpathlineto{\pgfqpoint{0.578621in}{0.907646in}}%
\pgfpathlineto{\pgfqpoint{0.581904in}{0.899042in}}%
\pgfpathlineto{\pgfqpoint{0.582547in}{0.897352in}}%
\pgfpathlineto{\pgfqpoint{0.586436in}{0.887057in}}%
\pgfpathlineto{\pgfqpoint{0.590296in}{0.876763in}}%
\pgfpathlineto{\pgfqpoint{0.592481in}{0.870892in}}%
\pgfpathlineto{\pgfqpoint{0.594120in}{0.866468in}}%
\pgfpathlineto{\pgfqpoint{0.597903in}{0.856174in}}%
\pgfpathlineto{\pgfqpoint{0.601656in}{0.845879in}}%
\pgfpathlineto{\pgfqpoint{0.603058in}{0.841762in}}%
\pgfpathlineto{\pgfqpoint{0.605103in}{0.835585in}}%
\pgfpathlineto{\pgfqpoint{0.608713in}{0.825291in}}%
\pgfpathlineto{\pgfqpoint{0.612306in}{0.814996in}}%
\pgfpathlineto{\pgfqpoint{0.613635in}{0.811149in}}%
\pgfpathlineto{\pgfqpoint{0.615848in}{0.804702in}}%
\pgfpathlineto{\pgfqpoint{0.619338in}{0.794407in}}%
\pgfpathlineto{\pgfqpoint{0.623179in}{0.784113in}}%
\pgfpathlineto{\pgfqpoint{0.624212in}{0.782050in}}%
\pgfpathlineto{\pgfqpoint{0.628263in}{0.773818in}}%
\pgfpathlineto{\pgfqpoint{0.633410in}{0.763524in}}%
\pgfpathlineto{\pgfqpoint{0.634790in}{0.760754in}}%
\pgfpathlineto{\pgfqpoint{0.638526in}{0.753230in}}%
\pgfpathlineto{\pgfqpoint{0.643691in}{0.742935in}}%
\pgfpathlineto{\pgfqpoint{0.645367in}{0.739415in}}%
\pgfpathlineto{\pgfqpoint{0.648705in}{0.732641in}}%
\pgfpathlineto{\pgfqpoint{0.654392in}{0.722346in}}%
\pgfpathlineto{\pgfqpoint{0.655944in}{0.719826in}}%
\pgfpathlineto{\pgfqpoint{0.660895in}{0.712052in}}%
\pgfpathlineto{\pgfqpoint{0.666521in}{0.702613in}}%
\pgfpathlineto{\pgfqpoint{0.667049in}{0.701757in}}%
\pgfpathlineto{\pgfqpoint{0.673783in}{0.691463in}}%
\pgfpathlineto{\pgfqpoint{0.677098in}{0.686476in}}%
\pgfpathlineto{\pgfqpoint{0.680615in}{0.681169in}}%
\pgfpathlineto{\pgfqpoint{0.687413in}{0.670874in}}%
\pgfpathlineto{\pgfqpoint{0.687675in}{0.670475in}}%
\pgfpathlineto{\pgfqpoint{0.694544in}{0.660580in}}%
\pgfpathlineto{\pgfqpoint{0.698252in}{0.655080in}}%
\pgfpathlineto{\pgfqpoint{0.701734in}{0.650285in}}%
\pgfpathlineto{\pgfqpoint{0.708829in}{0.640685in}}%
\pgfpathlineto{\pgfqpoint{0.709350in}{0.639991in}}%
\pgfpathlineto{\pgfqpoint{0.716530in}{0.629696in}}%
\pgfpathlineto{\pgfqpoint{0.719406in}{0.625190in}}%
\pgfpathlineto{\pgfqpoint{0.723523in}{0.619402in}}%
\pgfpathlineto{\pgfqpoint{0.729436in}{0.609108in}}%
\pgfpathlineto{\pgfqpoint{0.729983in}{0.608084in}}%
\pgfpathlineto{\pgfqpoint{0.737642in}{0.598813in}}%
\pgfpathlineto{\pgfqpoint{0.740560in}{0.594898in}}%
\pgfpathlineto{\pgfqpoint{0.745480in}{0.588519in}}%
\pgfpathlineto{\pgfqpoint{0.751137in}{0.580981in}}%
\pgfpathlineto{\pgfqpoint{0.753236in}{0.578224in}}%
\pgfpathlineto{\pgfqpoint{0.761019in}{0.567930in}}%
\pgfpathlineto{\pgfqpoint{0.761714in}{0.566982in}}%
\pgfpathlineto{\pgfqpoint{0.768720in}{0.557635in}}%
\pgfpathlineto{\pgfqpoint{0.772291in}{0.552873in}}%
\pgfpathlineto{\pgfqpoint{0.776432in}{0.547341in}}%
\pgfpathlineto{\pgfqpoint{0.782868in}{0.538697in}}%
\pgfpathlineto{\pgfqpoint{0.784088in}{0.537047in}}%
\pgfpathlineto{\pgfqpoint{0.791591in}{0.526752in}}%
\pgfpathlineto{\pgfqpoint{0.793445in}{0.524192in}}%
\pgfpathlineto{\pgfqpoint{0.799026in}{0.516458in}}%
\pgfpathlineto{\pgfqpoint{0.804023in}{0.509507in}}%
\pgfpathlineto{\pgfqpoint{0.806357in}{0.506163in}}%
\pgfpathlineto{\pgfqpoint{0.813321in}{0.495869in}}%
\pgfpathclose%
\pgfusepath{fill}%
\end{pgfscope}%
\begin{pgfscope}%
\pgfpathrectangle{\pgfqpoint{0.423750in}{0.423750in}}{\pgfqpoint{1.194205in}{1.163386in}}%
\pgfusepath{clip}%
\pgfsetbuttcap%
\pgfsetroundjoin%
\definecolor{currentfill}{rgb}{0.957344,0.505732,0.351309}%
\pgfsetfillcolor{currentfill}%
\pgfsetlinewidth{0.000000pt}%
\definecolor{currentstroke}{rgb}{0.000000,0.000000,0.000000}%
\pgfsetstrokecolor{currentstroke}%
\pgfsetdash{}{0pt}%
\pgfpathmoveto{\pgfqpoint{1.269413in}{1.040754in}}%
\pgfpathlineto{\pgfqpoint{1.279990in}{1.039767in}}%
\pgfpathlineto{\pgfqpoint{1.290567in}{1.038845in}}%
\pgfpathlineto{\pgfqpoint{1.301144in}{1.037988in}}%
\pgfpathlineto{\pgfqpoint{1.311721in}{1.037138in}}%
\pgfpathlineto{\pgfqpoint{1.322299in}{1.040937in}}%
\pgfpathlineto{\pgfqpoint{1.322897in}{1.041474in}}%
\pgfpathlineto{\pgfqpoint{1.332876in}{1.050063in}}%
\pgfpathlineto{\pgfqpoint{1.334844in}{1.051768in}}%
\pgfpathlineto{\pgfqpoint{1.343453in}{1.059292in}}%
\pgfpathlineto{\pgfqpoint{1.346602in}{1.062063in}}%
\pgfpathlineto{\pgfqpoint{1.354030in}{1.068489in}}%
\pgfpathlineto{\pgfqpoint{1.358696in}{1.072357in}}%
\pgfpathlineto{\pgfqpoint{1.364607in}{1.076668in}}%
\pgfpathlineto{\pgfqpoint{1.372764in}{1.082651in}}%
\pgfpathlineto{\pgfqpoint{1.375184in}{1.084454in}}%
\pgfpathlineto{\pgfqpoint{1.385761in}{1.092541in}}%
\pgfpathlineto{\pgfqpoint{1.386274in}{1.092946in}}%
\pgfpathlineto{\pgfqpoint{1.396338in}{1.100951in}}%
\pgfpathlineto{\pgfqpoint{1.399163in}{1.103240in}}%
\pgfpathlineto{\pgfqpoint{1.406915in}{1.107681in}}%
\pgfpathlineto{\pgfqpoint{1.415592in}{1.113535in}}%
\pgfpathlineto{\pgfqpoint{1.417492in}{1.114700in}}%
\pgfpathlineto{\pgfqpoint{1.428069in}{1.120348in}}%
\pgfpathlineto{\pgfqpoint{1.435656in}{1.123829in}}%
\pgfpathlineto{\pgfqpoint{1.438646in}{1.125216in}}%
\pgfpathlineto{\pgfqpoint{1.449223in}{1.130364in}}%
\pgfpathlineto{\pgfqpoint{1.456353in}{1.134124in}}%
\pgfpathlineto{\pgfqpoint{1.459800in}{1.136067in}}%
\pgfpathlineto{\pgfqpoint{1.470377in}{1.142073in}}%
\pgfpathlineto{\pgfqpoint{1.474477in}{1.144418in}}%
\pgfpathlineto{\pgfqpoint{1.480954in}{1.148164in}}%
\pgfpathlineto{\pgfqpoint{1.491532in}{1.154340in}}%
\pgfpathlineto{\pgfqpoint{1.492166in}{1.154712in}}%
\pgfpathlineto{\pgfqpoint{1.502109in}{1.160616in}}%
\pgfpathlineto{\pgfqpoint{1.509426in}{1.165007in}}%
\pgfpathlineto{\pgfqpoint{1.512686in}{1.166985in}}%
\pgfpathlineto{\pgfqpoint{1.523263in}{1.173455in}}%
\pgfpathlineto{\pgfqpoint{1.526260in}{1.175301in}}%
\pgfpathlineto{\pgfqpoint{1.533840in}{1.180026in}}%
\pgfpathlineto{\pgfqpoint{1.542676in}{1.185596in}}%
\pgfpathlineto{\pgfqpoint{1.544417in}{1.186706in}}%
\pgfpathlineto{\pgfqpoint{1.544417in}{1.195890in}}%
\pgfpathlineto{\pgfqpoint{1.544417in}{1.206185in}}%
\pgfpathlineto{\pgfqpoint{1.544417in}{1.216479in}}%
\pgfpathlineto{\pgfqpoint{1.544417in}{1.226773in}}%
\pgfpathlineto{\pgfqpoint{1.544417in}{1.237068in}}%
\pgfpathlineto{\pgfqpoint{1.544417in}{1.239237in}}%
\pgfpathlineto{\pgfqpoint{1.541237in}{1.237068in}}%
\pgfpathlineto{\pgfqpoint{1.533840in}{1.232055in}}%
\pgfpathlineto{\pgfqpoint{1.525953in}{1.226773in}}%
\pgfpathlineto{\pgfqpoint{1.523263in}{1.224985in}}%
\pgfpathlineto{\pgfqpoint{1.512686in}{1.218034in}}%
\pgfpathlineto{\pgfqpoint{1.510293in}{1.216479in}}%
\pgfpathlineto{\pgfqpoint{1.502109in}{1.211195in}}%
\pgfpathlineto{\pgfqpoint{1.494239in}{1.206185in}}%
\pgfpathlineto{\pgfqpoint{1.491532in}{1.204473in}}%
\pgfpathlineto{\pgfqpoint{1.480954in}{1.197866in}}%
\pgfpathlineto{\pgfqpoint{1.477748in}{1.195890in}}%
\pgfpathlineto{\pgfqpoint{1.470377in}{1.191377in}}%
\pgfpathlineto{\pgfqpoint{1.460795in}{1.185596in}}%
\pgfpathlineto{\pgfqpoint{1.459800in}{1.185000in}}%
\pgfpathlineto{\pgfqpoint{1.449223in}{1.178739in}}%
\pgfpathlineto{\pgfqpoint{1.443334in}{1.175301in}}%
\pgfpathlineto{\pgfqpoint{1.438646in}{1.172582in}}%
\pgfpathlineto{\pgfqpoint{1.428069in}{1.166524in}}%
\pgfpathlineto{\pgfqpoint{1.425389in}{1.165007in}}%
\pgfpathlineto{\pgfqpoint{1.417492in}{1.160563in}}%
\pgfpathlineto{\pgfqpoint{1.406980in}{1.154712in}}%
\pgfpathlineto{\pgfqpoint{1.406915in}{1.154678in}}%
\pgfpathlineto{\pgfqpoint{1.396338in}{1.148319in}}%
\pgfpathlineto{\pgfqpoint{1.389931in}{1.144418in}}%
\pgfpathlineto{\pgfqpoint{1.385761in}{1.141879in}}%
\pgfpathlineto{\pgfqpoint{1.375184in}{1.135573in}}%
\pgfpathlineto{\pgfqpoint{1.372734in}{1.134124in}}%
\pgfpathlineto{\pgfqpoint{1.364607in}{1.129368in}}%
\pgfpathlineto{\pgfqpoint{1.355071in}{1.123829in}}%
\pgfpathlineto{\pgfqpoint{1.354030in}{1.123231in}}%
\pgfpathlineto{\pgfqpoint{1.343453in}{1.117193in}}%
\pgfpathlineto{\pgfqpoint{1.337007in}{1.113535in}}%
\pgfpathlineto{\pgfqpoint{1.332876in}{1.111211in}}%
\pgfpathlineto{\pgfqpoint{1.322299in}{1.105293in}}%
\pgfpathlineto{\pgfqpoint{1.317179in}{1.103240in}}%
\pgfpathlineto{\pgfqpoint{1.311721in}{1.101129in}}%
\pgfpathlineto{\pgfqpoint{1.301144in}{1.101578in}}%
\pgfpathlineto{\pgfqpoint{1.290567in}{1.102106in}}%
\pgfpathlineto{\pgfqpoint{1.279990in}{1.102688in}}%
\pgfpathlineto{\pgfqpoint{1.270785in}{1.103240in}}%
\pgfpathlineto{\pgfqpoint{1.269413in}{1.103324in}}%
\pgfpathlineto{\pgfqpoint{1.258836in}{1.104019in}}%
\pgfpathlineto{\pgfqpoint{1.248259in}{1.104705in}}%
\pgfpathlineto{\pgfqpoint{1.237682in}{1.105351in}}%
\pgfpathlineto{\pgfqpoint{1.227105in}{1.105763in}}%
\pgfpathlineto{\pgfqpoint{1.216528in}{1.105956in}}%
\pgfpathlineto{\pgfqpoint{1.205951in}{1.106615in}}%
\pgfpathlineto{\pgfqpoint{1.195374in}{1.109000in}}%
\pgfpathlineto{\pgfqpoint{1.185368in}{1.113535in}}%
\pgfpathlineto{\pgfqpoint{1.184797in}{1.113747in}}%
\pgfpathlineto{\pgfqpoint{1.174220in}{1.119092in}}%
\pgfpathlineto{\pgfqpoint{1.166262in}{1.123829in}}%
\pgfpathlineto{\pgfqpoint{1.163643in}{1.125275in}}%
\pgfpathlineto{\pgfqpoint{1.153066in}{1.130974in}}%
\pgfpathlineto{\pgfqpoint{1.147263in}{1.134124in}}%
\pgfpathlineto{\pgfqpoint{1.142488in}{1.136737in}}%
\pgfpathlineto{\pgfqpoint{1.131911in}{1.142574in}}%
\pgfpathlineto{\pgfqpoint{1.128657in}{1.144418in}}%
\pgfpathlineto{\pgfqpoint{1.121334in}{1.148921in}}%
\pgfpathlineto{\pgfqpoint{1.112345in}{1.154712in}}%
\pgfpathlineto{\pgfqpoint{1.110757in}{1.155699in}}%
\pgfpathlineto{\pgfqpoint{1.100180in}{1.162569in}}%
\pgfpathlineto{\pgfqpoint{1.096824in}{1.165007in}}%
\pgfpathlineto{\pgfqpoint{1.089603in}{1.170551in}}%
\pgfpathlineto{\pgfqpoint{1.083987in}{1.175301in}}%
\pgfpathlineto{\pgfqpoint{1.079026in}{1.179664in}}%
\pgfpathlineto{\pgfqpoint{1.072144in}{1.185596in}}%
\pgfpathlineto{\pgfqpoint{1.068449in}{1.188631in}}%
\pgfpathlineto{\pgfqpoint{1.060880in}{1.195890in}}%
\pgfpathlineto{\pgfqpoint{1.057872in}{1.198661in}}%
\pgfpathlineto{\pgfqpoint{1.048073in}{1.206185in}}%
\pgfpathlineto{\pgfqpoint{1.047295in}{1.206831in}}%
\pgfpathlineto{\pgfqpoint{1.036718in}{1.213765in}}%
\pgfpathlineto{\pgfqpoint{1.032702in}{1.216479in}}%
\pgfpathlineto{\pgfqpoint{1.026141in}{1.221029in}}%
\pgfpathlineto{\pgfqpoint{1.018427in}{1.226773in}}%
\pgfpathlineto{\pgfqpoint{1.015564in}{1.228908in}}%
\pgfpathlineto{\pgfqpoint{1.004987in}{1.236858in}}%
\pgfpathlineto{\pgfqpoint{1.004710in}{1.237068in}}%
\pgfpathlineto{\pgfqpoint{0.994410in}{1.244928in}}%
\pgfpathlineto{\pgfqpoint{0.991257in}{1.247362in}}%
\pgfpathlineto{\pgfqpoint{0.983833in}{1.253119in}}%
\pgfpathlineto{\pgfqpoint{0.978041in}{1.257657in}}%
\pgfpathlineto{\pgfqpoint{0.973255in}{1.261632in}}%
\pgfpathlineto{\pgfqpoint{0.966545in}{1.267951in}}%
\pgfpathlineto{\pgfqpoint{0.962678in}{1.271743in}}%
\pgfpathlineto{\pgfqpoint{0.956223in}{1.278246in}}%
\pgfpathlineto{\pgfqpoint{0.952101in}{1.282379in}}%
\pgfpathlineto{\pgfqpoint{0.946302in}{1.288540in}}%
\pgfpathlineto{\pgfqpoint{0.941524in}{1.293646in}}%
\pgfpathlineto{\pgfqpoint{0.936733in}{1.298834in}}%
\pgfpathlineto{\pgfqpoint{0.930947in}{1.305138in}}%
\pgfpathlineto{\pgfqpoint{0.927331in}{1.309129in}}%
\pgfpathlineto{\pgfqpoint{0.920370in}{1.316853in}}%
\pgfpathlineto{\pgfqpoint{0.918081in}{1.319423in}}%
\pgfpathlineto{\pgfqpoint{0.909793in}{1.328774in}}%
\pgfpathlineto{\pgfqpoint{0.909005in}{1.329718in}}%
\pgfpathlineto{\pgfqpoint{0.901315in}{1.340012in}}%
\pgfpathlineto{\pgfqpoint{0.899216in}{1.343296in}}%
\pgfpathlineto{\pgfqpoint{0.895065in}{1.350307in}}%
\pgfpathlineto{\pgfqpoint{0.889103in}{1.360601in}}%
\pgfpathlineto{\pgfqpoint{0.888639in}{1.361413in}}%
\pgfpathlineto{\pgfqpoint{0.883306in}{1.370896in}}%
\pgfpathlineto{\pgfqpoint{0.878062in}{1.380315in}}%
\pgfpathlineto{\pgfqpoint{0.877581in}{1.381190in}}%
\pgfpathlineto{\pgfqpoint{0.871969in}{1.391484in}}%
\pgfpathlineto{\pgfqpoint{0.867485in}{1.400088in}}%
\pgfpathlineto{\pgfqpoint{0.866551in}{1.401779in}}%
\pgfpathlineto{\pgfqpoint{0.860903in}{1.412073in}}%
\pgfpathlineto{\pgfqpoint{0.856908in}{1.419132in}}%
\pgfpathlineto{\pgfqpoint{0.855061in}{1.422368in}}%
\pgfpathlineto{\pgfqpoint{0.849209in}{1.432662in}}%
\pgfpathlineto{\pgfqpoint{0.846331in}{1.437745in}}%
\pgfpathlineto{\pgfqpoint{0.843405in}{1.442957in}}%
\pgfpathlineto{\pgfqpoint{0.837647in}{1.453251in}}%
\pgfpathlineto{\pgfqpoint{0.835754in}{1.456648in}}%
\pgfpathlineto{\pgfqpoint{0.831909in}{1.463545in}}%
\pgfpathlineto{\pgfqpoint{0.826353in}{1.473840in}}%
\pgfpathlineto{\pgfqpoint{0.825177in}{1.476158in}}%
\pgfpathlineto{\pgfqpoint{0.820964in}{1.484134in}}%
\pgfpathlineto{\pgfqpoint{0.815557in}{1.494429in}}%
\pgfpathlineto{\pgfqpoint{0.814600in}{1.496263in}}%
\pgfpathlineto{\pgfqpoint{0.810171in}{1.504723in}}%
\pgfpathlineto{\pgfqpoint{0.804887in}{1.515018in}}%
\pgfpathlineto{\pgfqpoint{0.804023in}{1.515018in}}%
\pgfpathlineto{\pgfqpoint{0.793445in}{1.515018in}}%
\pgfpathlineto{\pgfqpoint{0.782868in}{1.515018in}}%
\pgfpathlineto{\pgfqpoint{0.772291in}{1.515018in}}%
\pgfpathlineto{\pgfqpoint{0.765198in}{1.515018in}}%
\pgfpathlineto{\pgfqpoint{0.770733in}{1.504723in}}%
\pgfpathlineto{\pgfqpoint{0.772291in}{1.501846in}}%
\pgfpathlineto{\pgfqpoint{0.776322in}{1.494429in}}%
\pgfpathlineto{\pgfqpoint{0.781971in}{1.484134in}}%
\pgfpathlineto{\pgfqpoint{0.782868in}{1.482613in}}%
\pgfpathlineto{\pgfqpoint{0.788071in}{1.473840in}}%
\pgfpathlineto{\pgfqpoint{0.793445in}{1.464821in}}%
\pgfpathlineto{\pgfqpoint{0.794187in}{1.463545in}}%
\pgfpathlineto{\pgfqpoint{0.800311in}{1.453251in}}%
\pgfpathlineto{\pgfqpoint{0.804023in}{1.447167in}}%
\pgfpathlineto{\pgfqpoint{0.806470in}{1.442957in}}%
\pgfpathlineto{\pgfqpoint{0.812482in}{1.432662in}}%
\pgfpathlineto{\pgfqpoint{0.814600in}{1.429130in}}%
\pgfpathlineto{\pgfqpoint{0.818595in}{1.422368in}}%
\pgfpathlineto{\pgfqpoint{0.824849in}{1.412073in}}%
\pgfpathlineto{\pgfqpoint{0.825177in}{1.411561in}}%
\pgfpathlineto{\pgfqpoint{0.831154in}{1.401779in}}%
\pgfpathlineto{\pgfqpoint{0.835754in}{1.394187in}}%
\pgfpathlineto{\pgfqpoint{0.837394in}{1.391484in}}%
\pgfpathlineto{\pgfqpoint{0.843671in}{1.381190in}}%
\pgfpathlineto{\pgfqpoint{0.846331in}{1.376844in}}%
\pgfpathlineto{\pgfqpoint{0.849883in}{1.370896in}}%
\pgfpathlineto{\pgfqpoint{0.855786in}{1.360601in}}%
\pgfpathlineto{\pgfqpoint{0.856908in}{1.358590in}}%
\pgfpathlineto{\pgfqpoint{0.861476in}{1.350307in}}%
\pgfpathlineto{\pgfqpoint{0.867209in}{1.340012in}}%
\pgfpathlineto{\pgfqpoint{0.867485in}{1.339531in}}%
\pgfpathlineto{\pgfqpoint{0.873210in}{1.329718in}}%
\pgfpathlineto{\pgfqpoint{0.878062in}{1.321597in}}%
\pgfpathlineto{\pgfqpoint{0.879384in}{1.319423in}}%
\pgfpathlineto{\pgfqpoint{0.885752in}{1.309129in}}%
\pgfpathlineto{\pgfqpoint{0.888639in}{1.304660in}}%
\pgfpathlineto{\pgfqpoint{0.893386in}{1.298834in}}%
\pgfpathlineto{\pgfqpoint{0.899216in}{1.291858in}}%
\pgfpathlineto{\pgfqpoint{0.902138in}{1.288540in}}%
\pgfpathlineto{\pgfqpoint{0.909793in}{1.279439in}}%
\pgfpathlineto{\pgfqpoint{0.910857in}{1.278246in}}%
\pgfpathlineto{\pgfqpoint{0.919572in}{1.267951in}}%
\pgfpathlineto{\pgfqpoint{0.920370in}{1.267011in}}%
\pgfpathlineto{\pgfqpoint{0.928431in}{1.257657in}}%
\pgfpathlineto{\pgfqpoint{0.930947in}{1.254755in}}%
\pgfpathlineto{\pgfqpoint{0.937459in}{1.247362in}}%
\pgfpathlineto{\pgfqpoint{0.941524in}{1.242776in}}%
\pgfpathlineto{\pgfqpoint{0.946697in}{1.237068in}}%
\pgfpathlineto{\pgfqpoint{0.952101in}{1.231769in}}%
\pgfpathlineto{\pgfqpoint{0.957915in}{1.226773in}}%
\pgfpathlineto{\pgfqpoint{0.962678in}{1.222712in}}%
\pgfpathlineto{\pgfqpoint{0.970540in}{1.216479in}}%
\pgfpathlineto{\pgfqpoint{0.973255in}{1.214322in}}%
\pgfpathlineto{\pgfqpoint{0.983748in}{1.206185in}}%
\pgfpathlineto{\pgfqpoint{0.983833in}{1.206119in}}%
\pgfpathlineto{\pgfqpoint{0.994410in}{1.197985in}}%
\pgfpathlineto{\pgfqpoint{0.997165in}{1.195890in}}%
\pgfpathlineto{\pgfqpoint{1.004987in}{1.190113in}}%
\pgfpathlineto{\pgfqpoint{1.011363in}{1.185596in}}%
\pgfpathlineto{\pgfqpoint{1.015564in}{1.182599in}}%
\pgfpathlineto{\pgfqpoint{1.026141in}{1.175498in}}%
\pgfpathlineto{\pgfqpoint{1.026438in}{1.175301in}}%
\pgfpathlineto{\pgfqpoint{1.036718in}{1.168577in}}%
\pgfpathlineto{\pgfqpoint{1.042277in}{1.165007in}}%
\pgfpathlineto{\pgfqpoint{1.047295in}{1.161810in}}%
\pgfpathlineto{\pgfqpoint{1.057872in}{1.155183in}}%
\pgfpathlineto{\pgfqpoint{1.058449in}{1.154712in}}%
\pgfpathlineto{\pgfqpoint{1.068449in}{1.147566in}}%
\pgfpathlineto{\pgfqpoint{1.072273in}{1.144418in}}%
\pgfpathlineto{\pgfqpoint{1.079026in}{1.139034in}}%
\pgfpathlineto{\pgfqpoint{1.085824in}{1.134124in}}%
\pgfpathlineto{\pgfqpoint{1.089603in}{1.130729in}}%
\pgfpathlineto{\pgfqpoint{1.097688in}{1.123829in}}%
\pgfpathlineto{\pgfqpoint{1.100180in}{1.121290in}}%
\pgfpathlineto{\pgfqpoint{1.109229in}{1.113535in}}%
\pgfpathlineto{\pgfqpoint{1.110757in}{1.112335in}}%
\pgfpathlineto{\pgfqpoint{1.121334in}{1.104203in}}%
\pgfpathlineto{\pgfqpoint{1.122658in}{1.103240in}}%
\pgfpathlineto{\pgfqpoint{1.131911in}{1.096093in}}%
\pgfpathlineto{\pgfqpoint{1.136123in}{1.092946in}}%
\pgfpathlineto{\pgfqpoint{1.142488in}{1.087841in}}%
\pgfpathlineto{\pgfqpoint{1.149559in}{1.082651in}}%
\pgfpathlineto{\pgfqpoint{1.153066in}{1.080007in}}%
\pgfpathlineto{\pgfqpoint{1.163643in}{1.074251in}}%
\pgfpathlineto{\pgfqpoint{1.167935in}{1.072357in}}%
\pgfpathlineto{\pgfqpoint{1.174220in}{1.069788in}}%
\pgfpathlineto{\pgfqpoint{1.184797in}{1.066100in}}%
\pgfpathlineto{\pgfqpoint{1.195374in}{1.062300in}}%
\pgfpathlineto{\pgfqpoint{1.195969in}{1.062063in}}%
\pgfpathlineto{\pgfqpoint{1.205951in}{1.056458in}}%
\pgfpathlineto{\pgfqpoint{1.215770in}{1.051768in}}%
\pgfpathlineto{\pgfqpoint{1.216528in}{1.051420in}}%
\pgfpathlineto{\pgfqpoint{1.227105in}{1.048421in}}%
\pgfpathlineto{\pgfqpoint{1.237682in}{1.045540in}}%
\pgfpathlineto{\pgfqpoint{1.248259in}{1.042814in}}%
\pgfpathlineto{\pgfqpoint{1.258836in}{1.041777in}}%
\pgfpathlineto{\pgfqpoint{1.261969in}{1.041474in}}%
\pgfpathclose%
\pgfusepath{fill}%
\end{pgfscope}%
\begin{pgfscope}%
\pgfpathrectangle{\pgfqpoint{0.423750in}{0.423750in}}{\pgfqpoint{1.194205in}{1.163386in}}%
\pgfusepath{clip}%
\pgfsetbuttcap%
\pgfsetroundjoin%
\definecolor{currentfill}{rgb}{0.962283,0.593046,0.431453}%
\pgfsetfillcolor{currentfill}%
\pgfsetlinewidth{0.000000pt}%
\definecolor{currentstroke}{rgb}{0.000000,0.000000,0.000000}%
\pgfsetstrokecolor{currentstroke}%
\pgfsetdash{}{0pt}%
\pgfpathmoveto{\pgfqpoint{0.835754in}{0.503665in}}%
\pgfpathlineto{\pgfqpoint{0.840798in}{0.495869in}}%
\pgfpathlineto{\pgfqpoint{0.846331in}{0.495869in}}%
\pgfpathlineto{\pgfqpoint{0.856908in}{0.495869in}}%
\pgfpathlineto{\pgfqpoint{0.867485in}{0.495869in}}%
\pgfpathlineto{\pgfqpoint{0.875387in}{0.495869in}}%
\pgfpathlineto{\pgfqpoint{0.872042in}{0.506163in}}%
\pgfpathlineto{\pgfqpoint{0.868461in}{0.516458in}}%
\pgfpathlineto{\pgfqpoint{0.867485in}{0.518995in}}%
\pgfpathlineto{\pgfqpoint{0.864392in}{0.526752in}}%
\pgfpathlineto{\pgfqpoint{0.860288in}{0.537047in}}%
\pgfpathlineto{\pgfqpoint{0.856908in}{0.545261in}}%
\pgfpathlineto{\pgfqpoint{0.856036in}{0.547341in}}%
\pgfpathlineto{\pgfqpoint{0.851545in}{0.557635in}}%
\pgfpathlineto{\pgfqpoint{0.846839in}{0.567930in}}%
\pgfpathlineto{\pgfqpoint{0.846331in}{0.568970in}}%
\pgfpathlineto{\pgfqpoint{0.841540in}{0.578224in}}%
\pgfpathlineto{\pgfqpoint{0.835850in}{0.588519in}}%
\pgfpathlineto{\pgfqpoint{0.835754in}{0.588690in}}%
\pgfpathlineto{\pgfqpoint{0.830008in}{0.598813in}}%
\pgfpathlineto{\pgfqpoint{0.825177in}{0.607163in}}%
\pgfpathlineto{\pgfqpoint{0.824042in}{0.609108in}}%
\pgfpathlineto{\pgfqpoint{0.817950in}{0.619402in}}%
\pgfpathlineto{\pgfqpoint{0.814600in}{0.625046in}}%
\pgfpathlineto{\pgfqpoint{0.811831in}{0.629696in}}%
\pgfpathlineto{\pgfqpoint{0.805156in}{0.639991in}}%
\pgfpathlineto{\pgfqpoint{0.804023in}{0.641534in}}%
\pgfpathlineto{\pgfqpoint{0.797488in}{0.650285in}}%
\pgfpathlineto{\pgfqpoint{0.793445in}{0.655747in}}%
\pgfpathlineto{\pgfqpoint{0.789694in}{0.660580in}}%
\pgfpathlineto{\pgfqpoint{0.782868in}{0.670435in}}%
\pgfpathlineto{\pgfqpoint{0.782544in}{0.670874in}}%
\pgfpathlineto{\pgfqpoint{0.775792in}{0.681169in}}%
\pgfpathlineto{\pgfqpoint{0.772291in}{0.686439in}}%
\pgfpathlineto{\pgfqpoint{0.767400in}{0.691463in}}%
\pgfpathlineto{\pgfqpoint{0.761714in}{0.695325in}}%
\pgfpathlineto{\pgfqpoint{0.751137in}{0.700176in}}%
\pgfpathlineto{\pgfqpoint{0.747345in}{0.701757in}}%
\pgfpathlineto{\pgfqpoint{0.740560in}{0.704571in}}%
\pgfpathlineto{\pgfqpoint{0.729983in}{0.708829in}}%
\pgfpathlineto{\pgfqpoint{0.724148in}{0.712052in}}%
\pgfpathlineto{\pgfqpoint{0.719406in}{0.714424in}}%
\pgfpathlineto{\pgfqpoint{0.711410in}{0.722346in}}%
\pgfpathlineto{\pgfqpoint{0.708829in}{0.724423in}}%
\pgfpathlineto{\pgfqpoint{0.704056in}{0.732641in}}%
\pgfpathlineto{\pgfqpoint{0.698252in}{0.742614in}}%
\pgfpathlineto{\pgfqpoint{0.698061in}{0.742935in}}%
\pgfpathlineto{\pgfqpoint{0.692704in}{0.753230in}}%
\pgfpathlineto{\pgfqpoint{0.688514in}{0.763524in}}%
\pgfpathlineto{\pgfqpoint{0.687675in}{0.765574in}}%
\pgfpathlineto{\pgfqpoint{0.684335in}{0.773818in}}%
\pgfpathlineto{\pgfqpoint{0.679912in}{0.784113in}}%
\pgfpathlineto{\pgfqpoint{0.677098in}{0.790732in}}%
\pgfpathlineto{\pgfqpoint{0.675832in}{0.794407in}}%
\pgfpathlineto{\pgfqpoint{0.671992in}{0.804702in}}%
\pgfpathlineto{\pgfqpoint{0.667975in}{0.814996in}}%
\pgfpathlineto{\pgfqpoint{0.666521in}{0.818578in}}%
\pgfpathlineto{\pgfqpoint{0.664128in}{0.825291in}}%
\pgfpathlineto{\pgfqpoint{0.660229in}{0.835585in}}%
\pgfpathlineto{\pgfqpoint{0.656049in}{0.845879in}}%
\pgfpathlineto{\pgfqpoint{0.655944in}{0.846160in}}%
\pgfpathlineto{\pgfqpoint{0.652387in}{0.856174in}}%
\pgfpathlineto{\pgfqpoint{0.648575in}{0.866468in}}%
\pgfpathlineto{\pgfqpoint{0.645367in}{0.874818in}}%
\pgfpathlineto{\pgfqpoint{0.644720in}{0.876763in}}%
\pgfpathlineto{\pgfqpoint{0.641288in}{0.887057in}}%
\pgfpathlineto{\pgfqpoint{0.637664in}{0.897352in}}%
\pgfpathlineto{\pgfqpoint{0.634790in}{0.904823in}}%
\pgfpathlineto{\pgfqpoint{0.633744in}{0.907646in}}%
\pgfpathlineto{\pgfqpoint{0.629906in}{0.917941in}}%
\pgfpathlineto{\pgfqpoint{0.626041in}{0.928235in}}%
\pgfpathlineto{\pgfqpoint{0.624212in}{0.933077in}}%
\pgfpathlineto{\pgfqpoint{0.622139in}{0.938529in}}%
\pgfpathlineto{\pgfqpoint{0.618203in}{0.948824in}}%
\pgfpathlineto{\pgfqpoint{0.614244in}{0.959118in}}%
\pgfpathlineto{\pgfqpoint{0.613635in}{0.960693in}}%
\pgfpathlineto{\pgfqpoint{0.610251in}{0.969413in}}%
\pgfpathlineto{\pgfqpoint{0.606236in}{0.979707in}}%
\pgfpathlineto{\pgfqpoint{0.603058in}{0.987872in}}%
\pgfpathlineto{\pgfqpoint{0.602226in}{0.990002in}}%
\pgfpathlineto{\pgfqpoint{0.598137in}{1.000296in}}%
\pgfpathlineto{\pgfqpoint{0.593895in}{1.010590in}}%
\pgfpathlineto{\pgfqpoint{0.592481in}{1.013917in}}%
\pgfpathlineto{\pgfqpoint{0.589499in}{1.020885in}}%
\pgfpathlineto{\pgfqpoint{0.585954in}{1.031179in}}%
\pgfpathlineto{\pgfqpoint{0.582517in}{1.041474in}}%
\pgfpathlineto{\pgfqpoint{0.581904in}{1.043350in}}%
\pgfpathlineto{\pgfqpoint{0.579152in}{1.051768in}}%
\pgfpathlineto{\pgfqpoint{0.575733in}{1.062063in}}%
\pgfpathlineto{\pgfqpoint{0.572302in}{1.072357in}}%
\pgfpathlineto{\pgfqpoint{0.571327in}{1.075271in}}%
\pgfpathlineto{\pgfqpoint{0.568862in}{1.082651in}}%
\pgfpathlineto{\pgfqpoint{0.565352in}{1.092946in}}%
\pgfpathlineto{\pgfqpoint{0.561762in}{1.103240in}}%
\pgfpathlineto{\pgfqpoint{0.560750in}{1.106072in}}%
\pgfpathlineto{\pgfqpoint{0.558092in}{1.113535in}}%
\pgfpathlineto{\pgfqpoint{0.554334in}{1.123829in}}%
\pgfpathlineto{\pgfqpoint{0.550491in}{1.134124in}}%
\pgfpathlineto{\pgfqpoint{0.550173in}{1.134998in}}%
\pgfpathlineto{\pgfqpoint{0.546697in}{1.144418in}}%
\pgfpathlineto{\pgfqpoint{0.543534in}{1.154712in}}%
\pgfpathlineto{\pgfqpoint{0.540400in}{1.165007in}}%
\pgfpathlineto{\pgfqpoint{0.539596in}{1.167621in}}%
\pgfpathlineto{\pgfqpoint{0.537242in}{1.175301in}}%
\pgfpathlineto{\pgfqpoint{0.534058in}{1.185596in}}%
\pgfpathlineto{\pgfqpoint{0.530847in}{1.195890in}}%
\pgfpathlineto{\pgfqpoint{0.529019in}{1.201684in}}%
\pgfpathlineto{\pgfqpoint{0.527603in}{1.206185in}}%
\pgfpathlineto{\pgfqpoint{0.524435in}{1.216479in}}%
\pgfpathlineto{\pgfqpoint{0.521270in}{1.226773in}}%
\pgfpathlineto{\pgfqpoint{0.518442in}{1.235842in}}%
\pgfpathlineto{\pgfqpoint{0.518060in}{1.237068in}}%
\pgfpathlineto{\pgfqpoint{0.514694in}{1.247362in}}%
\pgfpathlineto{\pgfqpoint{0.510999in}{1.257657in}}%
\pgfpathlineto{\pgfqpoint{0.507865in}{1.266306in}}%
\pgfpathlineto{\pgfqpoint{0.507258in}{1.267951in}}%
\pgfpathlineto{\pgfqpoint{0.503650in}{1.278246in}}%
\pgfpathlineto{\pgfqpoint{0.500064in}{1.288540in}}%
\pgfpathlineto{\pgfqpoint{0.497288in}{1.296337in}}%
\pgfpathlineto{\pgfqpoint{0.497288in}{1.288540in}}%
\pgfpathlineto{\pgfqpoint{0.497288in}{1.278246in}}%
\pgfpathlineto{\pgfqpoint{0.497288in}{1.267951in}}%
\pgfpathlineto{\pgfqpoint{0.497288in}{1.257657in}}%
\pgfpathlineto{\pgfqpoint{0.497288in}{1.247362in}}%
\pgfpathlineto{\pgfqpoint{0.497288in}{1.237068in}}%
\pgfpathlineto{\pgfqpoint{0.497288in}{1.226773in}}%
\pgfpathlineto{\pgfqpoint{0.497288in}{1.216479in}}%
\pgfpathlineto{\pgfqpoint{0.497288in}{1.210877in}}%
\pgfpathlineto{\pgfqpoint{0.498734in}{1.206185in}}%
\pgfpathlineto{\pgfqpoint{0.501874in}{1.195890in}}%
\pgfpathlineto{\pgfqpoint{0.504991in}{1.185596in}}%
\pgfpathlineto{\pgfqpoint{0.507865in}{1.176040in}}%
\pgfpathlineto{\pgfqpoint{0.508087in}{1.175301in}}%
\pgfpathlineto{\pgfqpoint{0.511153in}{1.165007in}}%
\pgfpathlineto{\pgfqpoint{0.514309in}{1.154712in}}%
\pgfpathlineto{\pgfqpoint{0.518113in}{1.144418in}}%
\pgfpathlineto{\pgfqpoint{0.518442in}{1.143539in}}%
\pgfpathlineto{\pgfqpoint{0.521966in}{1.134124in}}%
\pgfpathlineto{\pgfqpoint{0.525747in}{1.123829in}}%
\pgfpathlineto{\pgfqpoint{0.529019in}{1.114744in}}%
\pgfpathlineto{\pgfqpoint{0.529455in}{1.113535in}}%
\pgfpathlineto{\pgfqpoint{0.533094in}{1.103240in}}%
\pgfpathlineto{\pgfqpoint{0.536688in}{1.092946in}}%
\pgfpathlineto{\pgfqpoint{0.539596in}{1.084474in}}%
\pgfpathlineto{\pgfqpoint{0.540224in}{1.082651in}}%
\pgfpathlineto{\pgfqpoint{0.543823in}{1.072357in}}%
\pgfpathlineto{\pgfqpoint{0.547371in}{1.062063in}}%
\pgfpathlineto{\pgfqpoint{0.550173in}{1.053943in}}%
\pgfpathlineto{\pgfqpoint{0.550927in}{1.051768in}}%
\pgfpathlineto{\pgfqpoint{0.554451in}{1.041474in}}%
\pgfpathlineto{\pgfqpoint{0.557924in}{1.031179in}}%
\pgfpathlineto{\pgfqpoint{0.560750in}{1.022695in}}%
\pgfpathlineto{\pgfqpoint{0.561354in}{1.020885in}}%
\pgfpathlineto{\pgfqpoint{0.565603in}{1.010590in}}%
\pgfpathlineto{\pgfqpoint{0.570003in}{1.000296in}}%
\pgfpathlineto{\pgfqpoint{0.571327in}{0.997114in}}%
\pgfpathlineto{\pgfqpoint{0.574270in}{0.990002in}}%
\pgfpathlineto{\pgfqpoint{0.578396in}{0.979707in}}%
\pgfpathlineto{\pgfqpoint{0.581904in}{0.970816in}}%
\pgfpathlineto{\pgfqpoint{0.582457in}{0.969413in}}%
\pgfpathlineto{\pgfqpoint{0.586496in}{0.959118in}}%
\pgfpathlineto{\pgfqpoint{0.590516in}{0.948824in}}%
\pgfpathlineto{\pgfqpoint{0.592481in}{0.943769in}}%
\pgfpathlineto{\pgfqpoint{0.594512in}{0.938529in}}%
\pgfpathlineto{\pgfqpoint{0.598480in}{0.928235in}}%
\pgfpathlineto{\pgfqpoint{0.602424in}{0.917941in}}%
\pgfpathlineto{\pgfqpoint{0.603058in}{0.916276in}}%
\pgfpathlineto{\pgfqpoint{0.606330in}{0.907646in}}%
\pgfpathlineto{\pgfqpoint{0.610205in}{0.897352in}}%
\pgfpathlineto{\pgfqpoint{0.613635in}{0.888170in}}%
\pgfpathlineto{\pgfqpoint{0.614049in}{0.887057in}}%
\pgfpathlineto{\pgfqpoint{0.617844in}{0.876763in}}%
\pgfpathlineto{\pgfqpoint{0.621575in}{0.866468in}}%
\pgfpathlineto{\pgfqpoint{0.624212in}{0.858701in}}%
\pgfpathlineto{\pgfqpoint{0.625050in}{0.856174in}}%
\pgfpathlineto{\pgfqpoint{0.628437in}{0.845879in}}%
\pgfpathlineto{\pgfqpoint{0.631857in}{0.835585in}}%
\pgfpathlineto{\pgfqpoint{0.634790in}{0.827118in}}%
\pgfpathlineto{\pgfqpoint{0.635472in}{0.825291in}}%
\pgfpathlineto{\pgfqpoint{0.639551in}{0.814996in}}%
\pgfpathlineto{\pgfqpoint{0.643552in}{0.804702in}}%
\pgfpathlineto{\pgfqpoint{0.645367in}{0.799823in}}%
\pgfpathlineto{\pgfqpoint{0.647340in}{0.794407in}}%
\pgfpathlineto{\pgfqpoint{0.651071in}{0.784113in}}%
\pgfpathlineto{\pgfqpoint{0.654778in}{0.773818in}}%
\pgfpathlineto{\pgfqpoint{0.655944in}{0.771046in}}%
\pgfpathlineto{\pgfqpoint{0.658908in}{0.763524in}}%
\pgfpathlineto{\pgfqpoint{0.663909in}{0.753230in}}%
\pgfpathlineto{\pgfqpoint{0.666521in}{0.747918in}}%
\pgfpathlineto{\pgfqpoint{0.668954in}{0.742935in}}%
\pgfpathlineto{\pgfqpoint{0.673959in}{0.732641in}}%
\pgfpathlineto{\pgfqpoint{0.677098in}{0.726153in}}%
\pgfpathlineto{\pgfqpoint{0.679009in}{0.722346in}}%
\pgfpathlineto{\pgfqpoint{0.683971in}{0.712052in}}%
\pgfpathlineto{\pgfqpoint{0.687675in}{0.704081in}}%
\pgfpathlineto{\pgfqpoint{0.689020in}{0.701757in}}%
\pgfpathlineto{\pgfqpoint{0.694773in}{0.691463in}}%
\pgfpathlineto{\pgfqpoint{0.698252in}{0.684894in}}%
\pgfpathlineto{\pgfqpoint{0.700434in}{0.681169in}}%
\pgfpathlineto{\pgfqpoint{0.707152in}{0.670874in}}%
\pgfpathlineto{\pgfqpoint{0.708829in}{0.668254in}}%
\pgfpathlineto{\pgfqpoint{0.714217in}{0.660580in}}%
\pgfpathlineto{\pgfqpoint{0.719406in}{0.652560in}}%
\pgfpathlineto{\pgfqpoint{0.722433in}{0.650285in}}%
\pgfpathlineto{\pgfqpoint{0.729983in}{0.642203in}}%
\pgfpathlineto{\pgfqpoint{0.732481in}{0.639991in}}%
\pgfpathlineto{\pgfqpoint{0.740560in}{0.630852in}}%
\pgfpathlineto{\pgfqpoint{0.741519in}{0.629696in}}%
\pgfpathlineto{\pgfqpoint{0.750133in}{0.619402in}}%
\pgfpathlineto{\pgfqpoint{0.751137in}{0.618184in}}%
\pgfpathlineto{\pgfqpoint{0.758850in}{0.609108in}}%
\pgfpathlineto{\pgfqpoint{0.761714in}{0.605240in}}%
\pgfpathlineto{\pgfqpoint{0.766784in}{0.598813in}}%
\pgfpathlineto{\pgfqpoint{0.772291in}{0.591093in}}%
\pgfpathlineto{\pgfqpoint{0.774255in}{0.588519in}}%
\pgfpathlineto{\pgfqpoint{0.781655in}{0.578224in}}%
\pgfpathlineto{\pgfqpoint{0.782868in}{0.576636in}}%
\pgfpathlineto{\pgfqpoint{0.789486in}{0.567930in}}%
\pgfpathlineto{\pgfqpoint{0.793445in}{0.562721in}}%
\pgfpathlineto{\pgfqpoint{0.797281in}{0.557635in}}%
\pgfpathlineto{\pgfqpoint{0.804023in}{0.548672in}}%
\pgfpathlineto{\pgfqpoint{0.805049in}{0.547341in}}%
\pgfpathlineto{\pgfqpoint{0.812579in}{0.537047in}}%
\pgfpathlineto{\pgfqpoint{0.814600in}{0.534290in}}%
\pgfpathlineto{\pgfqpoint{0.819989in}{0.526752in}}%
\pgfpathlineto{\pgfqpoint{0.825177in}{0.519314in}}%
\pgfpathlineto{\pgfqpoint{0.827154in}{0.516458in}}%
\pgfpathlineto{\pgfqpoint{0.834099in}{0.506163in}}%
\pgfpathclose%
\pgfusepath{fill}%
\end{pgfscope}%
\begin{pgfscope}%
\pgfpathrectangle{\pgfqpoint{0.423750in}{0.423750in}}{\pgfqpoint{1.194205in}{1.163386in}}%
\pgfusepath{clip}%
\pgfsetbuttcap%
\pgfsetroundjoin%
\definecolor{currentfill}{rgb}{0.962283,0.593046,0.431453}%
\pgfsetfillcolor{currentfill}%
\pgfsetlinewidth{0.000000pt}%
\definecolor{currentstroke}{rgb}{0.000000,0.000000,0.000000}%
\pgfsetstrokecolor{currentstroke}%
\pgfsetdash{}{0pt}%
\pgfpathmoveto{\pgfqpoint{1.322299in}{0.915134in}}%
\pgfpathlineto{\pgfqpoint{1.324616in}{0.917941in}}%
\pgfpathlineto{\pgfqpoint{1.332160in}{0.928235in}}%
\pgfpathlineto{\pgfqpoint{1.332876in}{0.929777in}}%
\pgfpathlineto{\pgfqpoint{1.335692in}{0.938529in}}%
\pgfpathlineto{\pgfqpoint{1.338618in}{0.948824in}}%
\pgfpathlineto{\pgfqpoint{1.341388in}{0.959118in}}%
\pgfpathlineto{\pgfqpoint{1.343453in}{0.963026in}}%
\pgfpathlineto{\pgfqpoint{1.347853in}{0.969413in}}%
\pgfpathlineto{\pgfqpoint{1.354030in}{0.976662in}}%
\pgfpathlineto{\pgfqpoint{1.356620in}{0.979707in}}%
\pgfpathlineto{\pgfqpoint{1.364607in}{0.989246in}}%
\pgfpathlineto{\pgfqpoint{1.365238in}{0.990002in}}%
\pgfpathlineto{\pgfqpoint{1.373684in}{1.000296in}}%
\pgfpathlineto{\pgfqpoint{1.375184in}{1.002125in}}%
\pgfpathlineto{\pgfqpoint{1.381733in}{1.010590in}}%
\pgfpathlineto{\pgfqpoint{1.385761in}{1.014343in}}%
\pgfpathlineto{\pgfqpoint{1.392537in}{1.020885in}}%
\pgfpathlineto{\pgfqpoint{1.396338in}{1.024341in}}%
\pgfpathlineto{\pgfqpoint{1.403842in}{1.031179in}}%
\pgfpathlineto{\pgfqpoint{1.406915in}{1.034024in}}%
\pgfpathlineto{\pgfqpoint{1.414940in}{1.041474in}}%
\pgfpathlineto{\pgfqpoint{1.417492in}{1.043882in}}%
\pgfpathlineto{\pgfqpoint{1.425825in}{1.051768in}}%
\pgfpathlineto{\pgfqpoint{1.428069in}{1.053613in}}%
\pgfpathlineto{\pgfqpoint{1.438646in}{1.061931in}}%
\pgfpathlineto{\pgfqpoint{1.438817in}{1.062063in}}%
\pgfpathlineto{\pgfqpoint{1.449223in}{1.070002in}}%
\pgfpathlineto{\pgfqpoint{1.453538in}{1.072357in}}%
\pgfpathlineto{\pgfqpoint{1.459800in}{1.075733in}}%
\pgfpathlineto{\pgfqpoint{1.470377in}{1.081427in}}%
\pgfpathlineto{\pgfqpoint{1.472680in}{1.082651in}}%
\pgfpathlineto{\pgfqpoint{1.480954in}{1.087098in}}%
\pgfpathlineto{\pgfqpoint{1.491532in}{1.092920in}}%
\pgfpathlineto{\pgfqpoint{1.491577in}{1.092946in}}%
\pgfpathlineto{\pgfqpoint{1.502109in}{1.098959in}}%
\pgfpathlineto{\pgfqpoint{1.509280in}{1.103240in}}%
\pgfpathlineto{\pgfqpoint{1.512686in}{1.105496in}}%
\pgfpathlineto{\pgfqpoint{1.523263in}{1.112875in}}%
\pgfpathlineto{\pgfqpoint{1.524200in}{1.113535in}}%
\pgfpathlineto{\pgfqpoint{1.533840in}{1.120412in}}%
\pgfpathlineto{\pgfqpoint{1.538578in}{1.123829in}}%
\pgfpathlineto{\pgfqpoint{1.544417in}{1.128091in}}%
\pgfpathlineto{\pgfqpoint{1.544417in}{1.134124in}}%
\pgfpathlineto{\pgfqpoint{1.544417in}{1.144418in}}%
\pgfpathlineto{\pgfqpoint{1.544417in}{1.154712in}}%
\pgfpathlineto{\pgfqpoint{1.544417in}{1.165007in}}%
\pgfpathlineto{\pgfqpoint{1.544417in}{1.175301in}}%
\pgfpathlineto{\pgfqpoint{1.544417in}{1.185596in}}%
\pgfpathlineto{\pgfqpoint{1.544417in}{1.186706in}}%
\pgfpathlineto{\pgfqpoint{1.542676in}{1.185596in}}%
\pgfpathlineto{\pgfqpoint{1.533840in}{1.180026in}}%
\pgfpathlineto{\pgfqpoint{1.526260in}{1.175301in}}%
\pgfpathlineto{\pgfqpoint{1.523263in}{1.173455in}}%
\pgfpathlineto{\pgfqpoint{1.512686in}{1.166985in}}%
\pgfpathlineto{\pgfqpoint{1.509426in}{1.165007in}}%
\pgfpathlineto{\pgfqpoint{1.502109in}{1.160616in}}%
\pgfpathlineto{\pgfqpoint{1.492166in}{1.154712in}}%
\pgfpathlineto{\pgfqpoint{1.491532in}{1.154340in}}%
\pgfpathlineto{\pgfqpoint{1.480954in}{1.148164in}}%
\pgfpathlineto{\pgfqpoint{1.474477in}{1.144418in}}%
\pgfpathlineto{\pgfqpoint{1.470377in}{1.142073in}}%
\pgfpathlineto{\pgfqpoint{1.459800in}{1.136067in}}%
\pgfpathlineto{\pgfqpoint{1.456353in}{1.134124in}}%
\pgfpathlineto{\pgfqpoint{1.449223in}{1.130364in}}%
\pgfpathlineto{\pgfqpoint{1.438646in}{1.125216in}}%
\pgfpathlineto{\pgfqpoint{1.435656in}{1.123829in}}%
\pgfpathlineto{\pgfqpoint{1.428069in}{1.120348in}}%
\pgfpathlineto{\pgfqpoint{1.417492in}{1.114700in}}%
\pgfpathlineto{\pgfqpoint{1.415592in}{1.113535in}}%
\pgfpathlineto{\pgfqpoint{1.406915in}{1.107681in}}%
\pgfpathlineto{\pgfqpoint{1.399163in}{1.103240in}}%
\pgfpathlineto{\pgfqpoint{1.396338in}{1.100951in}}%
\pgfpathlineto{\pgfqpoint{1.386274in}{1.092946in}}%
\pgfpathlineto{\pgfqpoint{1.385761in}{1.092541in}}%
\pgfpathlineto{\pgfqpoint{1.375184in}{1.084454in}}%
\pgfpathlineto{\pgfqpoint{1.372764in}{1.082651in}}%
\pgfpathlineto{\pgfqpoint{1.364607in}{1.076668in}}%
\pgfpathlineto{\pgfqpoint{1.358696in}{1.072357in}}%
\pgfpathlineto{\pgfqpoint{1.354030in}{1.068489in}}%
\pgfpathlineto{\pgfqpoint{1.346602in}{1.062063in}}%
\pgfpathlineto{\pgfqpoint{1.343453in}{1.059292in}}%
\pgfpathlineto{\pgfqpoint{1.334844in}{1.051768in}}%
\pgfpathlineto{\pgfqpoint{1.332876in}{1.050063in}}%
\pgfpathlineto{\pgfqpoint{1.322897in}{1.041474in}}%
\pgfpathlineto{\pgfqpoint{1.322299in}{1.040937in}}%
\pgfpathlineto{\pgfqpoint{1.311721in}{1.037138in}}%
\pgfpathlineto{\pgfqpoint{1.301144in}{1.037988in}}%
\pgfpathlineto{\pgfqpoint{1.290567in}{1.038845in}}%
\pgfpathlineto{\pgfqpoint{1.279990in}{1.039767in}}%
\pgfpathlineto{\pgfqpoint{1.269413in}{1.040754in}}%
\pgfpathlineto{\pgfqpoint{1.261969in}{1.041474in}}%
\pgfpathlineto{\pgfqpoint{1.258836in}{1.041777in}}%
\pgfpathlineto{\pgfqpoint{1.248259in}{1.042814in}}%
\pgfpathlineto{\pgfqpoint{1.237682in}{1.045540in}}%
\pgfpathlineto{\pgfqpoint{1.227105in}{1.048421in}}%
\pgfpathlineto{\pgfqpoint{1.216528in}{1.051420in}}%
\pgfpathlineto{\pgfqpoint{1.215770in}{1.051768in}}%
\pgfpathlineto{\pgfqpoint{1.205951in}{1.056458in}}%
\pgfpathlineto{\pgfqpoint{1.195969in}{1.062063in}}%
\pgfpathlineto{\pgfqpoint{1.195374in}{1.062300in}}%
\pgfpathlineto{\pgfqpoint{1.184797in}{1.066100in}}%
\pgfpathlineto{\pgfqpoint{1.174220in}{1.069788in}}%
\pgfpathlineto{\pgfqpoint{1.167935in}{1.072357in}}%
\pgfpathlineto{\pgfqpoint{1.163643in}{1.074251in}}%
\pgfpathlineto{\pgfqpoint{1.153066in}{1.080007in}}%
\pgfpathlineto{\pgfqpoint{1.149559in}{1.082651in}}%
\pgfpathlineto{\pgfqpoint{1.142488in}{1.087841in}}%
\pgfpathlineto{\pgfqpoint{1.136123in}{1.092946in}}%
\pgfpathlineto{\pgfqpoint{1.131911in}{1.096093in}}%
\pgfpathlineto{\pgfqpoint{1.122658in}{1.103240in}}%
\pgfpathlineto{\pgfqpoint{1.121334in}{1.104203in}}%
\pgfpathlineto{\pgfqpoint{1.110757in}{1.112335in}}%
\pgfpathlineto{\pgfqpoint{1.109229in}{1.113535in}}%
\pgfpathlineto{\pgfqpoint{1.100180in}{1.121290in}}%
\pgfpathlineto{\pgfqpoint{1.097688in}{1.123829in}}%
\pgfpathlineto{\pgfqpoint{1.089603in}{1.130729in}}%
\pgfpathlineto{\pgfqpoint{1.085824in}{1.134124in}}%
\pgfpathlineto{\pgfqpoint{1.079026in}{1.139034in}}%
\pgfpathlineto{\pgfqpoint{1.072273in}{1.144418in}}%
\pgfpathlineto{\pgfqpoint{1.068449in}{1.147566in}}%
\pgfpathlineto{\pgfqpoint{1.058449in}{1.154712in}}%
\pgfpathlineto{\pgfqpoint{1.057872in}{1.155183in}}%
\pgfpathlineto{\pgfqpoint{1.047295in}{1.161810in}}%
\pgfpathlineto{\pgfqpoint{1.042277in}{1.165007in}}%
\pgfpathlineto{\pgfqpoint{1.036718in}{1.168577in}}%
\pgfpathlineto{\pgfqpoint{1.026438in}{1.175301in}}%
\pgfpathlineto{\pgfqpoint{1.026141in}{1.175498in}}%
\pgfpathlineto{\pgfqpoint{1.015564in}{1.182599in}}%
\pgfpathlineto{\pgfqpoint{1.011363in}{1.185596in}}%
\pgfpathlineto{\pgfqpoint{1.004987in}{1.190113in}}%
\pgfpathlineto{\pgfqpoint{0.997165in}{1.195890in}}%
\pgfpathlineto{\pgfqpoint{0.994410in}{1.197985in}}%
\pgfpathlineto{\pgfqpoint{0.983833in}{1.206119in}}%
\pgfpathlineto{\pgfqpoint{0.983748in}{1.206185in}}%
\pgfpathlineto{\pgfqpoint{0.973255in}{1.214322in}}%
\pgfpathlineto{\pgfqpoint{0.970540in}{1.216479in}}%
\pgfpathlineto{\pgfqpoint{0.962678in}{1.222712in}}%
\pgfpathlineto{\pgfqpoint{0.957915in}{1.226773in}}%
\pgfpathlineto{\pgfqpoint{0.952101in}{1.231769in}}%
\pgfpathlineto{\pgfqpoint{0.946697in}{1.237068in}}%
\pgfpathlineto{\pgfqpoint{0.941524in}{1.242776in}}%
\pgfpathlineto{\pgfqpoint{0.937459in}{1.247362in}}%
\pgfpathlineto{\pgfqpoint{0.930947in}{1.254755in}}%
\pgfpathlineto{\pgfqpoint{0.928431in}{1.257657in}}%
\pgfpathlineto{\pgfqpoint{0.920370in}{1.267011in}}%
\pgfpathlineto{\pgfqpoint{0.919572in}{1.267951in}}%
\pgfpathlineto{\pgfqpoint{0.910857in}{1.278246in}}%
\pgfpathlineto{\pgfqpoint{0.909793in}{1.279439in}}%
\pgfpathlineto{\pgfqpoint{0.902138in}{1.288540in}}%
\pgfpathlineto{\pgfqpoint{0.899216in}{1.291858in}}%
\pgfpathlineto{\pgfqpoint{0.893386in}{1.298834in}}%
\pgfpathlineto{\pgfqpoint{0.888639in}{1.304660in}}%
\pgfpathlineto{\pgfqpoint{0.885752in}{1.309129in}}%
\pgfpathlineto{\pgfqpoint{0.879384in}{1.319423in}}%
\pgfpathlineto{\pgfqpoint{0.878062in}{1.321597in}}%
\pgfpathlineto{\pgfqpoint{0.873210in}{1.329718in}}%
\pgfpathlineto{\pgfqpoint{0.867485in}{1.339531in}}%
\pgfpathlineto{\pgfqpoint{0.867209in}{1.340012in}}%
\pgfpathlineto{\pgfqpoint{0.861476in}{1.350307in}}%
\pgfpathlineto{\pgfqpoint{0.856908in}{1.358590in}}%
\pgfpathlineto{\pgfqpoint{0.855786in}{1.360601in}}%
\pgfpathlineto{\pgfqpoint{0.849883in}{1.370896in}}%
\pgfpathlineto{\pgfqpoint{0.846331in}{1.376844in}}%
\pgfpathlineto{\pgfqpoint{0.843671in}{1.381190in}}%
\pgfpathlineto{\pgfqpoint{0.837394in}{1.391484in}}%
\pgfpathlineto{\pgfqpoint{0.835754in}{1.394187in}}%
\pgfpathlineto{\pgfqpoint{0.831154in}{1.401779in}}%
\pgfpathlineto{\pgfqpoint{0.825177in}{1.411561in}}%
\pgfpathlineto{\pgfqpoint{0.824849in}{1.412073in}}%
\pgfpathlineto{\pgfqpoint{0.818595in}{1.422368in}}%
\pgfpathlineto{\pgfqpoint{0.814600in}{1.429130in}}%
\pgfpathlineto{\pgfqpoint{0.812482in}{1.432662in}}%
\pgfpathlineto{\pgfqpoint{0.806470in}{1.442957in}}%
\pgfpathlineto{\pgfqpoint{0.804023in}{1.447167in}}%
\pgfpathlineto{\pgfqpoint{0.800311in}{1.453251in}}%
\pgfpathlineto{\pgfqpoint{0.794187in}{1.463545in}}%
\pgfpathlineto{\pgfqpoint{0.793445in}{1.464821in}}%
\pgfpathlineto{\pgfqpoint{0.788071in}{1.473840in}}%
\pgfpathlineto{\pgfqpoint{0.782868in}{1.482613in}}%
\pgfpathlineto{\pgfqpoint{0.781971in}{1.484134in}}%
\pgfpathlineto{\pgfqpoint{0.776322in}{1.494429in}}%
\pgfpathlineto{\pgfqpoint{0.772291in}{1.501846in}}%
\pgfpathlineto{\pgfqpoint{0.770733in}{1.504723in}}%
\pgfpathlineto{\pgfqpoint{0.765198in}{1.515018in}}%
\pgfpathlineto{\pgfqpoint{0.761714in}{1.515018in}}%
\pgfpathlineto{\pgfqpoint{0.751137in}{1.515018in}}%
\pgfpathlineto{\pgfqpoint{0.740560in}{1.515018in}}%
\pgfpathlineto{\pgfqpoint{0.729983in}{1.515018in}}%
\pgfpathlineto{\pgfqpoint{0.724904in}{1.515018in}}%
\pgfpathlineto{\pgfqpoint{0.729983in}{1.505589in}}%
\pgfpathlineto{\pgfqpoint{0.730449in}{1.504723in}}%
\pgfpathlineto{\pgfqpoint{0.736378in}{1.494429in}}%
\pgfpathlineto{\pgfqpoint{0.740560in}{1.487219in}}%
\pgfpathlineto{\pgfqpoint{0.742353in}{1.484134in}}%
\pgfpathlineto{\pgfqpoint{0.748367in}{1.473840in}}%
\pgfpathlineto{\pgfqpoint{0.751137in}{1.469128in}}%
\pgfpathlineto{\pgfqpoint{0.754425in}{1.463545in}}%
\pgfpathlineto{\pgfqpoint{0.760526in}{1.453251in}}%
\pgfpathlineto{\pgfqpoint{0.761714in}{1.451263in}}%
\pgfpathlineto{\pgfqpoint{0.766684in}{1.442957in}}%
\pgfpathlineto{\pgfqpoint{0.772291in}{1.434149in}}%
\pgfpathlineto{\pgfqpoint{0.773215in}{1.432662in}}%
\pgfpathlineto{\pgfqpoint{0.779925in}{1.422368in}}%
\pgfpathlineto{\pgfqpoint{0.782868in}{1.417893in}}%
\pgfpathlineto{\pgfqpoint{0.786697in}{1.412073in}}%
\pgfpathlineto{\pgfqpoint{0.793445in}{1.401863in}}%
\pgfpathlineto{\pgfqpoint{0.793501in}{1.401779in}}%
\pgfpathlineto{\pgfqpoint{0.800307in}{1.391484in}}%
\pgfpathlineto{\pgfqpoint{0.804023in}{1.385971in}}%
\pgfpathlineto{\pgfqpoint{0.807225in}{1.381190in}}%
\pgfpathlineto{\pgfqpoint{0.814185in}{1.370896in}}%
\pgfpathlineto{\pgfqpoint{0.814600in}{1.370280in}}%
\pgfpathlineto{\pgfqpoint{0.820892in}{1.360601in}}%
\pgfpathlineto{\pgfqpoint{0.825177in}{1.354128in}}%
\pgfpathlineto{\pgfqpoint{0.827468in}{1.350307in}}%
\pgfpathlineto{\pgfqpoint{0.833533in}{1.340012in}}%
\pgfpathlineto{\pgfqpoint{0.835754in}{1.335919in}}%
\pgfpathlineto{\pgfqpoint{0.838888in}{1.329718in}}%
\pgfpathlineto{\pgfqpoint{0.844078in}{1.319423in}}%
\pgfpathlineto{\pgfqpoint{0.846331in}{1.314870in}}%
\pgfpathlineto{\pgfqpoint{0.849184in}{1.309129in}}%
\pgfpathlineto{\pgfqpoint{0.854449in}{1.298834in}}%
\pgfpathlineto{\pgfqpoint{0.856908in}{1.294065in}}%
\pgfpathlineto{\pgfqpoint{0.859779in}{1.288540in}}%
\pgfpathlineto{\pgfqpoint{0.865321in}{1.278246in}}%
\pgfpathlineto{\pgfqpoint{0.867485in}{1.274295in}}%
\pgfpathlineto{\pgfqpoint{0.871514in}{1.267951in}}%
\pgfpathlineto{\pgfqpoint{0.878062in}{1.259233in}}%
\pgfpathlineto{\pgfqpoint{0.879343in}{1.257657in}}%
\pgfpathlineto{\pgfqpoint{0.887771in}{1.247362in}}%
\pgfpathlineto{\pgfqpoint{0.888639in}{1.246275in}}%
\pgfpathlineto{\pgfqpoint{0.896237in}{1.237068in}}%
\pgfpathlineto{\pgfqpoint{0.899216in}{1.233397in}}%
\pgfpathlineto{\pgfqpoint{0.904748in}{1.226773in}}%
\pgfpathlineto{\pgfqpoint{0.909793in}{1.220681in}}%
\pgfpathlineto{\pgfqpoint{0.913375in}{1.216479in}}%
\pgfpathlineto{\pgfqpoint{0.920370in}{1.208218in}}%
\pgfpathlineto{\pgfqpoint{0.922326in}{1.206185in}}%
\pgfpathlineto{\pgfqpoint{0.930947in}{1.197858in}}%
\pgfpathlineto{\pgfqpoint{0.933265in}{1.195890in}}%
\pgfpathlineto{\pgfqpoint{0.941524in}{1.188475in}}%
\pgfpathlineto{\pgfqpoint{0.944999in}{1.185596in}}%
\pgfpathlineto{\pgfqpoint{0.952101in}{1.179113in}}%
\pgfpathlineto{\pgfqpoint{0.956473in}{1.175301in}}%
\pgfpathlineto{\pgfqpoint{0.962678in}{1.169767in}}%
\pgfpathlineto{\pgfqpoint{0.968031in}{1.165007in}}%
\pgfpathlineto{\pgfqpoint{0.973255in}{1.159332in}}%
\pgfpathlineto{\pgfqpoint{0.978228in}{1.154712in}}%
\pgfpathlineto{\pgfqpoint{0.983833in}{1.147999in}}%
\pgfpathlineto{\pgfqpoint{0.986904in}{1.144418in}}%
\pgfpathlineto{\pgfqpoint{0.994410in}{1.136110in}}%
\pgfpathlineto{\pgfqpoint{0.996383in}{1.134124in}}%
\pgfpathlineto{\pgfqpoint{1.004987in}{1.125596in}}%
\pgfpathlineto{\pgfqpoint{1.006823in}{1.123829in}}%
\pgfpathlineto{\pgfqpoint{1.015564in}{1.115467in}}%
\pgfpathlineto{\pgfqpoint{1.017646in}{1.113535in}}%
\pgfpathlineto{\pgfqpoint{1.026141in}{1.105698in}}%
\pgfpathlineto{\pgfqpoint{1.029448in}{1.103240in}}%
\pgfpathlineto{\pgfqpoint{1.036718in}{1.098925in}}%
\pgfpathlineto{\pgfqpoint{1.047295in}{1.095729in}}%
\pgfpathlineto{\pgfqpoint{1.057872in}{1.093816in}}%
\pgfpathlineto{\pgfqpoint{1.059427in}{1.092946in}}%
\pgfpathlineto{\pgfqpoint{1.068449in}{1.087343in}}%
\pgfpathlineto{\pgfqpoint{1.072581in}{1.082651in}}%
\pgfpathlineto{\pgfqpoint{1.079026in}{1.076252in}}%
\pgfpathlineto{\pgfqpoint{1.083340in}{1.072357in}}%
\pgfpathlineto{\pgfqpoint{1.089603in}{1.066453in}}%
\pgfpathlineto{\pgfqpoint{1.094120in}{1.062063in}}%
\pgfpathlineto{\pgfqpoint{1.100180in}{1.056065in}}%
\pgfpathlineto{\pgfqpoint{1.104639in}{1.051768in}}%
\pgfpathlineto{\pgfqpoint{1.110757in}{1.045925in}}%
\pgfpathlineto{\pgfqpoint{1.115531in}{1.041474in}}%
\pgfpathlineto{\pgfqpoint{1.121334in}{1.036111in}}%
\pgfpathlineto{\pgfqpoint{1.127165in}{1.031179in}}%
\pgfpathlineto{\pgfqpoint{1.131911in}{1.027432in}}%
\pgfpathlineto{\pgfqpoint{1.140444in}{1.020885in}}%
\pgfpathlineto{\pgfqpoint{1.142488in}{1.019208in}}%
\pgfpathlineto{\pgfqpoint{1.153066in}{1.011278in}}%
\pgfpathlineto{\pgfqpoint{1.154023in}{1.010590in}}%
\pgfpathlineto{\pgfqpoint{1.163643in}{1.003257in}}%
\pgfpathlineto{\pgfqpoint{1.167840in}{1.000296in}}%
\pgfpathlineto{\pgfqpoint{1.174220in}{0.995714in}}%
\pgfpathlineto{\pgfqpoint{1.181566in}{0.990002in}}%
\pgfpathlineto{\pgfqpoint{1.184797in}{0.987490in}}%
\pgfpathlineto{\pgfqpoint{1.194918in}{0.979707in}}%
\pgfpathlineto{\pgfqpoint{1.195374in}{0.979358in}}%
\pgfpathlineto{\pgfqpoint{1.205951in}{0.971560in}}%
\pgfpathlineto{\pgfqpoint{1.210543in}{0.969413in}}%
\pgfpathlineto{\pgfqpoint{1.216528in}{0.966696in}}%
\pgfpathlineto{\pgfqpoint{1.227105in}{0.962304in}}%
\pgfpathlineto{\pgfqpoint{1.235915in}{0.959118in}}%
\pgfpathlineto{\pgfqpoint{1.237682in}{0.958452in}}%
\pgfpathlineto{\pgfqpoint{1.248259in}{0.955831in}}%
\pgfpathlineto{\pgfqpoint{1.258836in}{0.951911in}}%
\pgfpathlineto{\pgfqpoint{1.264375in}{0.948824in}}%
\pgfpathlineto{\pgfqpoint{1.269413in}{0.945677in}}%
\pgfpathlineto{\pgfqpoint{1.279990in}{0.939190in}}%
\pgfpathlineto{\pgfqpoint{1.281101in}{0.938529in}}%
\pgfpathlineto{\pgfqpoint{1.290567in}{0.932986in}}%
\pgfpathlineto{\pgfqpoint{1.298859in}{0.928235in}}%
\pgfpathlineto{\pgfqpoint{1.301144in}{0.926890in}}%
\pgfpathlineto{\pgfqpoint{1.311721in}{0.921011in}}%
\pgfpathlineto{\pgfqpoint{1.317299in}{0.917941in}}%
\pgfpathclose%
\pgfusepath{fill}%
\end{pgfscope}%
\begin{pgfscope}%
\pgfpathrectangle{\pgfqpoint{0.423750in}{0.423750in}}{\pgfqpoint{1.194205in}{1.163386in}}%
\pgfusepath{clip}%
\pgfsetbuttcap%
\pgfsetroundjoin%
\definecolor{currentfill}{rgb}{0.964433,0.670254,0.515093}%
\pgfsetfillcolor{currentfill}%
\pgfsetlinewidth{0.000000pt}%
\definecolor{currentstroke}{rgb}{0.000000,0.000000,0.000000}%
\pgfsetstrokecolor{currentstroke}%
\pgfsetdash{}{0pt}%
\pgfpathmoveto{\pgfqpoint{0.878062in}{0.495869in}}%
\pgfpathlineto{\pgfqpoint{0.888639in}{0.495869in}}%
\pgfpathlineto{\pgfqpoint{0.899216in}{0.495869in}}%
\pgfpathlineto{\pgfqpoint{0.909793in}{0.495869in}}%
\pgfpathlineto{\pgfqpoint{0.920370in}{0.495869in}}%
\pgfpathlineto{\pgfqpoint{0.930947in}{0.495869in}}%
\pgfpathlineto{\pgfqpoint{0.941524in}{0.495869in}}%
\pgfpathlineto{\pgfqpoint{0.942041in}{0.495869in}}%
\pgfpathlineto{\pgfqpoint{0.941524in}{0.497011in}}%
\pgfpathlineto{\pgfqpoint{0.937358in}{0.506163in}}%
\pgfpathlineto{\pgfqpoint{0.932570in}{0.516458in}}%
\pgfpathlineto{\pgfqpoint{0.930947in}{0.519877in}}%
\pgfpathlineto{\pgfqpoint{0.927648in}{0.526752in}}%
\pgfpathlineto{\pgfqpoint{0.922602in}{0.537047in}}%
\pgfpathlineto{\pgfqpoint{0.920370in}{0.541510in}}%
\pgfpathlineto{\pgfqpoint{0.917671in}{0.547341in}}%
\pgfpathlineto{\pgfqpoint{0.913233in}{0.557635in}}%
\pgfpathlineto{\pgfqpoint{0.909793in}{0.566591in}}%
\pgfpathlineto{\pgfqpoint{0.909319in}{0.567930in}}%
\pgfpathlineto{\pgfqpoint{0.905392in}{0.578224in}}%
\pgfpathlineto{\pgfqpoint{0.901108in}{0.588519in}}%
\pgfpathlineto{\pgfqpoint{0.899216in}{0.592795in}}%
\pgfpathlineto{\pgfqpoint{0.896472in}{0.598813in}}%
\pgfpathlineto{\pgfqpoint{0.891518in}{0.609108in}}%
\pgfpathlineto{\pgfqpoint{0.888639in}{0.614759in}}%
\pgfpathlineto{\pgfqpoint{0.886232in}{0.619402in}}%
\pgfpathlineto{\pgfqpoint{0.880657in}{0.629696in}}%
\pgfpathlineto{\pgfqpoint{0.878062in}{0.634369in}}%
\pgfpathlineto{\pgfqpoint{0.874785in}{0.639991in}}%
\pgfpathlineto{\pgfqpoint{0.870438in}{0.650285in}}%
\pgfpathlineto{\pgfqpoint{0.867485in}{0.657220in}}%
\pgfpathlineto{\pgfqpoint{0.866015in}{0.660580in}}%
\pgfpathlineto{\pgfqpoint{0.860333in}{0.670874in}}%
\pgfpathlineto{\pgfqpoint{0.856908in}{0.676770in}}%
\pgfpathlineto{\pgfqpoint{0.854340in}{0.681169in}}%
\pgfpathlineto{\pgfqpoint{0.848705in}{0.691463in}}%
\pgfpathlineto{\pgfqpoint{0.846331in}{0.696154in}}%
\pgfpathlineto{\pgfqpoint{0.843423in}{0.701757in}}%
\pgfpathlineto{\pgfqpoint{0.837706in}{0.712052in}}%
\pgfpathlineto{\pgfqpoint{0.835754in}{0.715466in}}%
\pgfpathlineto{\pgfqpoint{0.831776in}{0.722346in}}%
\pgfpathlineto{\pgfqpoint{0.825668in}{0.732641in}}%
\pgfpathlineto{\pgfqpoint{0.825177in}{0.735492in}}%
\pgfpathlineto{\pgfqpoint{0.823859in}{0.742935in}}%
\pgfpathlineto{\pgfqpoint{0.821187in}{0.753230in}}%
\pgfpathlineto{\pgfqpoint{0.815232in}{0.763524in}}%
\pgfpathlineto{\pgfqpoint{0.814600in}{0.764278in}}%
\pgfpathlineto{\pgfqpoint{0.804023in}{0.772287in}}%
\pgfpathlineto{\pgfqpoint{0.801683in}{0.773818in}}%
\pgfpathlineto{\pgfqpoint{0.793445in}{0.779170in}}%
\pgfpathlineto{\pgfqpoint{0.785554in}{0.784113in}}%
\pgfpathlineto{\pgfqpoint{0.782868in}{0.785782in}}%
\pgfpathlineto{\pgfqpoint{0.772291in}{0.792091in}}%
\pgfpathlineto{\pgfqpoint{0.768654in}{0.794407in}}%
\pgfpathlineto{\pgfqpoint{0.761714in}{0.798521in}}%
\pgfpathlineto{\pgfqpoint{0.753091in}{0.804702in}}%
\pgfpathlineto{\pgfqpoint{0.751137in}{0.806083in}}%
\pgfpathlineto{\pgfqpoint{0.740560in}{0.813429in}}%
\pgfpathlineto{\pgfqpoint{0.738247in}{0.814996in}}%
\pgfpathlineto{\pgfqpoint{0.729983in}{0.820542in}}%
\pgfpathlineto{\pgfqpoint{0.722718in}{0.825291in}}%
\pgfpathlineto{\pgfqpoint{0.719406in}{0.827438in}}%
\pgfpathlineto{\pgfqpoint{0.708829in}{0.834101in}}%
\pgfpathlineto{\pgfqpoint{0.706400in}{0.835585in}}%
\pgfpathlineto{\pgfqpoint{0.698252in}{0.840529in}}%
\pgfpathlineto{\pgfqpoint{0.695361in}{0.845879in}}%
\pgfpathlineto{\pgfqpoint{0.689165in}{0.856174in}}%
\pgfpathlineto{\pgfqpoint{0.687675in}{0.858164in}}%
\pgfpathlineto{\pgfqpoint{0.684463in}{0.866468in}}%
\pgfpathlineto{\pgfqpoint{0.680105in}{0.876763in}}%
\pgfpathlineto{\pgfqpoint{0.677098in}{0.883168in}}%
\pgfpathlineto{\pgfqpoint{0.675427in}{0.887057in}}%
\pgfpathlineto{\pgfqpoint{0.670785in}{0.897352in}}%
\pgfpathlineto{\pgfqpoint{0.666521in}{0.906085in}}%
\pgfpathlineto{\pgfqpoint{0.665864in}{0.907646in}}%
\pgfpathlineto{\pgfqpoint{0.661278in}{0.917941in}}%
\pgfpathlineto{\pgfqpoint{0.656326in}{0.928235in}}%
\pgfpathlineto{\pgfqpoint{0.655944in}{0.929021in}}%
\pgfpathlineto{\pgfqpoint{0.652160in}{0.938529in}}%
\pgfpathlineto{\pgfqpoint{0.648049in}{0.948824in}}%
\pgfpathlineto{\pgfqpoint{0.645367in}{0.955467in}}%
\pgfpathlineto{\pgfqpoint{0.643880in}{0.959118in}}%
\pgfpathlineto{\pgfqpoint{0.639670in}{0.969413in}}%
\pgfpathlineto{\pgfqpoint{0.635453in}{0.979707in}}%
\pgfpathlineto{\pgfqpoint{0.634790in}{0.981347in}}%
\pgfpathlineto{\pgfqpoint{0.631261in}{0.990002in}}%
\pgfpathlineto{\pgfqpoint{0.627124in}{1.000296in}}%
\pgfpathlineto{\pgfqpoint{0.624212in}{1.007497in}}%
\pgfpathlineto{\pgfqpoint{0.623041in}{1.010590in}}%
\pgfpathlineto{\pgfqpoint{0.618923in}{1.020885in}}%
\pgfpathlineto{\pgfqpoint{0.614694in}{1.031179in}}%
\pgfpathlineto{\pgfqpoint{0.613635in}{1.034234in}}%
\pgfpathlineto{\pgfqpoint{0.611335in}{1.041474in}}%
\pgfpathlineto{\pgfqpoint{0.608016in}{1.051768in}}%
\pgfpathlineto{\pgfqpoint{0.604636in}{1.062063in}}%
\pgfpathlineto{\pgfqpoint{0.603058in}{1.066770in}}%
\pgfpathlineto{\pgfqpoint{0.601181in}{1.072357in}}%
\pgfpathlineto{\pgfqpoint{0.597714in}{1.082651in}}%
\pgfpathlineto{\pgfqpoint{0.594185in}{1.092946in}}%
\pgfpathlineto{\pgfqpoint{0.592481in}{1.097778in}}%
\pgfpathlineto{\pgfqpoint{0.590556in}{1.103240in}}%
\pgfpathlineto{\pgfqpoint{0.586818in}{1.113535in}}%
\pgfpathlineto{\pgfqpoint{0.582978in}{1.123829in}}%
\pgfpathlineto{\pgfqpoint{0.581904in}{1.126885in}}%
\pgfpathlineto{\pgfqpoint{0.579294in}{1.134124in}}%
\pgfpathlineto{\pgfqpoint{0.576030in}{1.144418in}}%
\pgfpathlineto{\pgfqpoint{0.572817in}{1.154712in}}%
\pgfpathlineto{\pgfqpoint{0.571327in}{1.159433in}}%
\pgfpathlineto{\pgfqpoint{0.569573in}{1.165007in}}%
\pgfpathlineto{\pgfqpoint{0.566296in}{1.175301in}}%
\pgfpathlineto{\pgfqpoint{0.562986in}{1.185596in}}%
\pgfpathlineto{\pgfqpoint{0.560750in}{1.192653in}}%
\pgfpathlineto{\pgfqpoint{0.559719in}{1.195890in}}%
\pgfpathlineto{\pgfqpoint{0.556485in}{1.206185in}}%
\pgfpathlineto{\pgfqpoint{0.553214in}{1.216479in}}%
\pgfpathlineto{\pgfqpoint{0.550173in}{1.225937in}}%
\pgfpathlineto{\pgfqpoint{0.549905in}{1.226773in}}%
\pgfpathlineto{\pgfqpoint{0.546536in}{1.237068in}}%
\pgfpathlineto{\pgfqpoint{0.542950in}{1.247362in}}%
\pgfpathlineto{\pgfqpoint{0.539596in}{1.256518in}}%
\pgfpathlineto{\pgfqpoint{0.539179in}{1.257657in}}%
\pgfpathlineto{\pgfqpoint{0.535541in}{1.267951in}}%
\pgfpathlineto{\pgfqpoint{0.532078in}{1.278246in}}%
\pgfpathlineto{\pgfqpoint{0.529019in}{1.287250in}}%
\pgfpathlineto{\pgfqpoint{0.528580in}{1.288540in}}%
\pgfpathlineto{\pgfqpoint{0.524984in}{1.298834in}}%
\pgfpathlineto{\pgfqpoint{0.521316in}{1.309129in}}%
\pgfpathlineto{\pgfqpoint{0.518442in}{1.317103in}}%
\pgfpathlineto{\pgfqpoint{0.517595in}{1.319423in}}%
\pgfpathlineto{\pgfqpoint{0.513885in}{1.329718in}}%
\pgfpathlineto{\pgfqpoint{0.510219in}{1.340012in}}%
\pgfpathlineto{\pgfqpoint{0.507865in}{1.347042in}}%
\pgfpathlineto{\pgfqpoint{0.506748in}{1.350307in}}%
\pgfpathlineto{\pgfqpoint{0.503564in}{1.360601in}}%
\pgfpathlineto{\pgfqpoint{0.500565in}{1.370896in}}%
\pgfpathlineto{\pgfqpoint{0.497551in}{1.381190in}}%
\pgfpathlineto{\pgfqpoint{0.497288in}{1.382081in}}%
\pgfpathlineto{\pgfqpoint{0.497288in}{1.381190in}}%
\pgfpathlineto{\pgfqpoint{0.497288in}{1.370896in}}%
\pgfpathlineto{\pgfqpoint{0.497288in}{1.360601in}}%
\pgfpathlineto{\pgfqpoint{0.497288in}{1.350307in}}%
\pgfpathlineto{\pgfqpoint{0.497288in}{1.340012in}}%
\pgfpathlineto{\pgfqpoint{0.497288in}{1.329718in}}%
\pgfpathlineto{\pgfqpoint{0.497288in}{1.319423in}}%
\pgfpathlineto{\pgfqpoint{0.497288in}{1.309129in}}%
\pgfpathlineto{\pgfqpoint{0.497288in}{1.298834in}}%
\pgfpathlineto{\pgfqpoint{0.497288in}{1.296337in}}%
\pgfpathlineto{\pgfqpoint{0.500064in}{1.288540in}}%
\pgfpathlineto{\pgfqpoint{0.503650in}{1.278246in}}%
\pgfpathlineto{\pgfqpoint{0.507258in}{1.267951in}}%
\pgfpathlineto{\pgfqpoint{0.507865in}{1.266306in}}%
\pgfpathlineto{\pgfqpoint{0.510999in}{1.257657in}}%
\pgfpathlineto{\pgfqpoint{0.514694in}{1.247362in}}%
\pgfpathlineto{\pgfqpoint{0.518060in}{1.237068in}}%
\pgfpathlineto{\pgfqpoint{0.518442in}{1.235842in}}%
\pgfpathlineto{\pgfqpoint{0.521270in}{1.226773in}}%
\pgfpathlineto{\pgfqpoint{0.524435in}{1.216479in}}%
\pgfpathlineto{\pgfqpoint{0.527603in}{1.206185in}}%
\pgfpathlineto{\pgfqpoint{0.529019in}{1.201684in}}%
\pgfpathlineto{\pgfqpoint{0.530847in}{1.195890in}}%
\pgfpathlineto{\pgfqpoint{0.534058in}{1.185596in}}%
\pgfpathlineto{\pgfqpoint{0.537242in}{1.175301in}}%
\pgfpathlineto{\pgfqpoint{0.539596in}{1.167621in}}%
\pgfpathlineto{\pgfqpoint{0.540400in}{1.165007in}}%
\pgfpathlineto{\pgfqpoint{0.543534in}{1.154712in}}%
\pgfpathlineto{\pgfqpoint{0.546697in}{1.144418in}}%
\pgfpathlineto{\pgfqpoint{0.550173in}{1.134998in}}%
\pgfpathlineto{\pgfqpoint{0.550491in}{1.134124in}}%
\pgfpathlineto{\pgfqpoint{0.554334in}{1.123829in}}%
\pgfpathlineto{\pgfqpoint{0.558092in}{1.113535in}}%
\pgfpathlineto{\pgfqpoint{0.560750in}{1.106072in}}%
\pgfpathlineto{\pgfqpoint{0.561762in}{1.103240in}}%
\pgfpathlineto{\pgfqpoint{0.565352in}{1.092946in}}%
\pgfpathlineto{\pgfqpoint{0.568862in}{1.082651in}}%
\pgfpathlineto{\pgfqpoint{0.571327in}{1.075271in}}%
\pgfpathlineto{\pgfqpoint{0.572302in}{1.072357in}}%
\pgfpathlineto{\pgfqpoint{0.575733in}{1.062063in}}%
\pgfpathlineto{\pgfqpoint{0.579152in}{1.051768in}}%
\pgfpathlineto{\pgfqpoint{0.581904in}{1.043350in}}%
\pgfpathlineto{\pgfqpoint{0.582517in}{1.041474in}}%
\pgfpathlineto{\pgfqpoint{0.585954in}{1.031179in}}%
\pgfpathlineto{\pgfqpoint{0.589499in}{1.020885in}}%
\pgfpathlineto{\pgfqpoint{0.592481in}{1.013917in}}%
\pgfpathlineto{\pgfqpoint{0.593895in}{1.010590in}}%
\pgfpathlineto{\pgfqpoint{0.598137in}{1.000296in}}%
\pgfpathlineto{\pgfqpoint{0.602226in}{0.990002in}}%
\pgfpathlineto{\pgfqpoint{0.603058in}{0.987872in}}%
\pgfpathlineto{\pgfqpoint{0.606236in}{0.979707in}}%
\pgfpathlineto{\pgfqpoint{0.610251in}{0.969413in}}%
\pgfpathlineto{\pgfqpoint{0.613635in}{0.960693in}}%
\pgfpathlineto{\pgfqpoint{0.614244in}{0.959118in}}%
\pgfpathlineto{\pgfqpoint{0.618203in}{0.948824in}}%
\pgfpathlineto{\pgfqpoint{0.622139in}{0.938529in}}%
\pgfpathlineto{\pgfqpoint{0.624212in}{0.933077in}}%
\pgfpathlineto{\pgfqpoint{0.626041in}{0.928235in}}%
\pgfpathlineto{\pgfqpoint{0.629906in}{0.917941in}}%
\pgfpathlineto{\pgfqpoint{0.633744in}{0.907646in}}%
\pgfpathlineto{\pgfqpoint{0.634790in}{0.904823in}}%
\pgfpathlineto{\pgfqpoint{0.637664in}{0.897352in}}%
\pgfpathlineto{\pgfqpoint{0.641288in}{0.887057in}}%
\pgfpathlineto{\pgfqpoint{0.644720in}{0.876763in}}%
\pgfpathlineto{\pgfqpoint{0.645367in}{0.874818in}}%
\pgfpathlineto{\pgfqpoint{0.648575in}{0.866468in}}%
\pgfpathlineto{\pgfqpoint{0.652387in}{0.856174in}}%
\pgfpathlineto{\pgfqpoint{0.655944in}{0.846160in}}%
\pgfpathlineto{\pgfqpoint{0.656049in}{0.845879in}}%
\pgfpathlineto{\pgfqpoint{0.660229in}{0.835585in}}%
\pgfpathlineto{\pgfqpoint{0.664128in}{0.825291in}}%
\pgfpathlineto{\pgfqpoint{0.666521in}{0.818578in}}%
\pgfpathlineto{\pgfqpoint{0.667975in}{0.814996in}}%
\pgfpathlineto{\pgfqpoint{0.671992in}{0.804702in}}%
\pgfpathlineto{\pgfqpoint{0.675832in}{0.794407in}}%
\pgfpathlineto{\pgfqpoint{0.677098in}{0.790732in}}%
\pgfpathlineto{\pgfqpoint{0.679912in}{0.784113in}}%
\pgfpathlineto{\pgfqpoint{0.684335in}{0.773818in}}%
\pgfpathlineto{\pgfqpoint{0.687675in}{0.765574in}}%
\pgfpathlineto{\pgfqpoint{0.688514in}{0.763524in}}%
\pgfpathlineto{\pgfqpoint{0.692704in}{0.753230in}}%
\pgfpathlineto{\pgfqpoint{0.698061in}{0.742935in}}%
\pgfpathlineto{\pgfqpoint{0.698252in}{0.742614in}}%
\pgfpathlineto{\pgfqpoint{0.704056in}{0.732641in}}%
\pgfpathlineto{\pgfqpoint{0.708829in}{0.724423in}}%
\pgfpathlineto{\pgfqpoint{0.711410in}{0.722346in}}%
\pgfpathlineto{\pgfqpoint{0.719406in}{0.714424in}}%
\pgfpathlineto{\pgfqpoint{0.724148in}{0.712052in}}%
\pgfpathlineto{\pgfqpoint{0.729983in}{0.708829in}}%
\pgfpathlineto{\pgfqpoint{0.740560in}{0.704571in}}%
\pgfpathlineto{\pgfqpoint{0.747345in}{0.701757in}}%
\pgfpathlineto{\pgfqpoint{0.751137in}{0.700176in}}%
\pgfpathlineto{\pgfqpoint{0.761714in}{0.695325in}}%
\pgfpathlineto{\pgfqpoint{0.767400in}{0.691463in}}%
\pgfpathlineto{\pgfqpoint{0.772291in}{0.686439in}}%
\pgfpathlineto{\pgfqpoint{0.775792in}{0.681169in}}%
\pgfpathlineto{\pgfqpoint{0.782544in}{0.670874in}}%
\pgfpathlineto{\pgfqpoint{0.782868in}{0.670435in}}%
\pgfpathlineto{\pgfqpoint{0.789694in}{0.660580in}}%
\pgfpathlineto{\pgfqpoint{0.793445in}{0.655747in}}%
\pgfpathlineto{\pgfqpoint{0.797488in}{0.650285in}}%
\pgfpathlineto{\pgfqpoint{0.804023in}{0.641534in}}%
\pgfpathlineto{\pgfqpoint{0.805156in}{0.639991in}}%
\pgfpathlineto{\pgfqpoint{0.811831in}{0.629696in}}%
\pgfpathlineto{\pgfqpoint{0.814600in}{0.625046in}}%
\pgfpathlineto{\pgfqpoint{0.817950in}{0.619402in}}%
\pgfpathlineto{\pgfqpoint{0.824042in}{0.609108in}}%
\pgfpathlineto{\pgfqpoint{0.825177in}{0.607163in}}%
\pgfpathlineto{\pgfqpoint{0.830008in}{0.598813in}}%
\pgfpathlineto{\pgfqpoint{0.835754in}{0.588690in}}%
\pgfpathlineto{\pgfqpoint{0.835850in}{0.588519in}}%
\pgfpathlineto{\pgfqpoint{0.841540in}{0.578224in}}%
\pgfpathlineto{\pgfqpoint{0.846331in}{0.568970in}}%
\pgfpathlineto{\pgfqpoint{0.846839in}{0.567930in}}%
\pgfpathlineto{\pgfqpoint{0.851545in}{0.557635in}}%
\pgfpathlineto{\pgfqpoint{0.856036in}{0.547341in}}%
\pgfpathlineto{\pgfqpoint{0.856908in}{0.545261in}}%
\pgfpathlineto{\pgfqpoint{0.860288in}{0.537047in}}%
\pgfpathlineto{\pgfqpoint{0.864392in}{0.526752in}}%
\pgfpathlineto{\pgfqpoint{0.867485in}{0.518995in}}%
\pgfpathlineto{\pgfqpoint{0.868461in}{0.516458in}}%
\pgfpathlineto{\pgfqpoint{0.872042in}{0.506163in}}%
\pgfpathlineto{\pgfqpoint{0.875387in}{0.495869in}}%
\pgfpathclose%
\pgfusepath{fill}%
\end{pgfscope}%
\begin{pgfscope}%
\pgfpathrectangle{\pgfqpoint{0.423750in}{0.423750in}}{\pgfqpoint{1.194205in}{1.163386in}}%
\pgfusepath{clip}%
\pgfsetbuttcap%
\pgfsetroundjoin%
\definecolor{currentfill}{rgb}{0.964433,0.670254,0.515093}%
\pgfsetfillcolor{currentfill}%
\pgfsetlinewidth{0.000000pt}%
\definecolor{currentstroke}{rgb}{0.000000,0.000000,0.000000}%
\pgfsetstrokecolor{currentstroke}%
\pgfsetdash{}{0pt}%
\pgfpathmoveto{\pgfqpoint{1.322299in}{0.810743in}}%
\pgfpathlineto{\pgfqpoint{1.332876in}{0.810975in}}%
\pgfpathlineto{\pgfqpoint{1.338639in}{0.814996in}}%
\pgfpathlineto{\pgfqpoint{1.343453in}{0.818944in}}%
\pgfpathlineto{\pgfqpoint{1.349981in}{0.825291in}}%
\pgfpathlineto{\pgfqpoint{1.354030in}{0.829301in}}%
\pgfpathlineto{\pgfqpoint{1.360350in}{0.835585in}}%
\pgfpathlineto{\pgfqpoint{1.364607in}{0.839839in}}%
\pgfpathlineto{\pgfqpoint{1.370465in}{0.845879in}}%
\pgfpathlineto{\pgfqpoint{1.375184in}{0.850857in}}%
\pgfpathlineto{\pgfqpoint{1.380071in}{0.856174in}}%
\pgfpathlineto{\pgfqpoint{1.385761in}{0.864134in}}%
\pgfpathlineto{\pgfqpoint{1.386962in}{0.866468in}}%
\pgfpathlineto{\pgfqpoint{1.391112in}{0.876763in}}%
\pgfpathlineto{\pgfqpoint{1.395102in}{0.887057in}}%
\pgfpathlineto{\pgfqpoint{1.396338in}{0.890352in}}%
\pgfpathlineto{\pgfqpoint{1.398967in}{0.897352in}}%
\pgfpathlineto{\pgfqpoint{1.402706in}{0.907646in}}%
\pgfpathlineto{\pgfqpoint{1.406915in}{0.917036in}}%
\pgfpathlineto{\pgfqpoint{1.407433in}{0.917941in}}%
\pgfpathlineto{\pgfqpoint{1.413661in}{0.928235in}}%
\pgfpathlineto{\pgfqpoint{1.417492in}{0.934712in}}%
\pgfpathlineto{\pgfqpoint{1.419757in}{0.938529in}}%
\pgfpathlineto{\pgfqpoint{1.426562in}{0.948824in}}%
\pgfpathlineto{\pgfqpoint{1.428069in}{0.950744in}}%
\pgfpathlineto{\pgfqpoint{1.434890in}{0.959118in}}%
\pgfpathlineto{\pgfqpoint{1.438646in}{0.963696in}}%
\pgfpathlineto{\pgfqpoint{1.443339in}{0.969413in}}%
\pgfpathlineto{\pgfqpoint{1.449223in}{0.976724in}}%
\pgfpathlineto{\pgfqpoint{1.451625in}{0.979707in}}%
\pgfpathlineto{\pgfqpoint{1.459800in}{0.989278in}}%
\pgfpathlineto{\pgfqpoint{1.460441in}{0.990002in}}%
\pgfpathlineto{\pgfqpoint{1.470377in}{1.000295in}}%
\pgfpathlineto{\pgfqpoint{1.470379in}{1.000296in}}%
\pgfpathlineto{\pgfqpoint{1.480954in}{1.007677in}}%
\pgfpathlineto{\pgfqpoint{1.485106in}{1.010590in}}%
\pgfpathlineto{\pgfqpoint{1.491532in}{1.015177in}}%
\pgfpathlineto{\pgfqpoint{1.499478in}{1.020885in}}%
\pgfpathlineto{\pgfqpoint{1.502109in}{1.022659in}}%
\pgfpathlineto{\pgfqpoint{1.512686in}{1.030091in}}%
\pgfpathlineto{\pgfqpoint{1.514186in}{1.031179in}}%
\pgfpathlineto{\pgfqpoint{1.523263in}{1.036105in}}%
\pgfpathlineto{\pgfqpoint{1.533771in}{1.041474in}}%
\pgfpathlineto{\pgfqpoint{1.533840in}{1.041509in}}%
\pgfpathlineto{\pgfqpoint{1.544417in}{1.047105in}}%
\pgfpathlineto{\pgfqpoint{1.544417in}{1.051768in}}%
\pgfpathlineto{\pgfqpoint{1.544417in}{1.062063in}}%
\pgfpathlineto{\pgfqpoint{1.544417in}{1.072357in}}%
\pgfpathlineto{\pgfqpoint{1.544417in}{1.082651in}}%
\pgfpathlineto{\pgfqpoint{1.544417in}{1.092946in}}%
\pgfpathlineto{\pgfqpoint{1.544417in}{1.103240in}}%
\pgfpathlineto{\pgfqpoint{1.544417in}{1.113535in}}%
\pgfpathlineto{\pgfqpoint{1.544417in}{1.123829in}}%
\pgfpathlineto{\pgfqpoint{1.544417in}{1.128091in}}%
\pgfpathlineto{\pgfqpoint{1.538578in}{1.123829in}}%
\pgfpathlineto{\pgfqpoint{1.533840in}{1.120412in}}%
\pgfpathlineto{\pgfqpoint{1.524200in}{1.113535in}}%
\pgfpathlineto{\pgfqpoint{1.523263in}{1.112875in}}%
\pgfpathlineto{\pgfqpoint{1.512686in}{1.105496in}}%
\pgfpathlineto{\pgfqpoint{1.509280in}{1.103240in}}%
\pgfpathlineto{\pgfqpoint{1.502109in}{1.098959in}}%
\pgfpathlineto{\pgfqpoint{1.491577in}{1.092946in}}%
\pgfpathlineto{\pgfqpoint{1.491532in}{1.092920in}}%
\pgfpathlineto{\pgfqpoint{1.480954in}{1.087098in}}%
\pgfpathlineto{\pgfqpoint{1.472680in}{1.082651in}}%
\pgfpathlineto{\pgfqpoint{1.470377in}{1.081427in}}%
\pgfpathlineto{\pgfqpoint{1.459800in}{1.075733in}}%
\pgfpathlineto{\pgfqpoint{1.453538in}{1.072357in}}%
\pgfpathlineto{\pgfqpoint{1.449223in}{1.070002in}}%
\pgfpathlineto{\pgfqpoint{1.438817in}{1.062063in}}%
\pgfpathlineto{\pgfqpoint{1.438646in}{1.061931in}}%
\pgfpathlineto{\pgfqpoint{1.428069in}{1.053613in}}%
\pgfpathlineto{\pgfqpoint{1.425825in}{1.051768in}}%
\pgfpathlineto{\pgfqpoint{1.417492in}{1.043882in}}%
\pgfpathlineto{\pgfqpoint{1.414940in}{1.041474in}}%
\pgfpathlineto{\pgfqpoint{1.406915in}{1.034024in}}%
\pgfpathlineto{\pgfqpoint{1.403842in}{1.031179in}}%
\pgfpathlineto{\pgfqpoint{1.396338in}{1.024341in}}%
\pgfpathlineto{\pgfqpoint{1.392537in}{1.020885in}}%
\pgfpathlineto{\pgfqpoint{1.385761in}{1.014343in}}%
\pgfpathlineto{\pgfqpoint{1.381733in}{1.010590in}}%
\pgfpathlineto{\pgfqpoint{1.375184in}{1.002125in}}%
\pgfpathlineto{\pgfqpoint{1.373684in}{1.000296in}}%
\pgfpathlineto{\pgfqpoint{1.365238in}{0.990002in}}%
\pgfpathlineto{\pgfqpoint{1.364607in}{0.989246in}}%
\pgfpathlineto{\pgfqpoint{1.356620in}{0.979707in}}%
\pgfpathlineto{\pgfqpoint{1.354030in}{0.976662in}}%
\pgfpathlineto{\pgfqpoint{1.347853in}{0.969413in}}%
\pgfpathlineto{\pgfqpoint{1.343453in}{0.963026in}}%
\pgfpathlineto{\pgfqpoint{1.341388in}{0.959118in}}%
\pgfpathlineto{\pgfqpoint{1.338618in}{0.948824in}}%
\pgfpathlineto{\pgfqpoint{1.335692in}{0.938529in}}%
\pgfpathlineto{\pgfqpoint{1.332876in}{0.929777in}}%
\pgfpathlineto{\pgfqpoint{1.332160in}{0.928235in}}%
\pgfpathlineto{\pgfqpoint{1.324616in}{0.917941in}}%
\pgfpathlineto{\pgfqpoint{1.322299in}{0.915134in}}%
\pgfpathlineto{\pgfqpoint{1.317299in}{0.917941in}}%
\pgfpathlineto{\pgfqpoint{1.311721in}{0.921011in}}%
\pgfpathlineto{\pgfqpoint{1.301144in}{0.926890in}}%
\pgfpathlineto{\pgfqpoint{1.298859in}{0.928235in}}%
\pgfpathlineto{\pgfqpoint{1.290567in}{0.932986in}}%
\pgfpathlineto{\pgfqpoint{1.281101in}{0.938529in}}%
\pgfpathlineto{\pgfqpoint{1.279990in}{0.939190in}}%
\pgfpathlineto{\pgfqpoint{1.269413in}{0.945677in}}%
\pgfpathlineto{\pgfqpoint{1.264375in}{0.948824in}}%
\pgfpathlineto{\pgfqpoint{1.258836in}{0.951911in}}%
\pgfpathlineto{\pgfqpoint{1.248259in}{0.955831in}}%
\pgfpathlineto{\pgfqpoint{1.237682in}{0.958452in}}%
\pgfpathlineto{\pgfqpoint{1.235915in}{0.959118in}}%
\pgfpathlineto{\pgfqpoint{1.227105in}{0.962304in}}%
\pgfpathlineto{\pgfqpoint{1.216528in}{0.966696in}}%
\pgfpathlineto{\pgfqpoint{1.210543in}{0.969413in}}%
\pgfpathlineto{\pgfqpoint{1.205951in}{0.971560in}}%
\pgfpathlineto{\pgfqpoint{1.195374in}{0.979358in}}%
\pgfpathlineto{\pgfqpoint{1.194918in}{0.979707in}}%
\pgfpathlineto{\pgfqpoint{1.184797in}{0.987490in}}%
\pgfpathlineto{\pgfqpoint{1.181566in}{0.990002in}}%
\pgfpathlineto{\pgfqpoint{1.174220in}{0.995714in}}%
\pgfpathlineto{\pgfqpoint{1.167840in}{1.000296in}}%
\pgfpathlineto{\pgfqpoint{1.163643in}{1.003257in}}%
\pgfpathlineto{\pgfqpoint{1.154023in}{1.010590in}}%
\pgfpathlineto{\pgfqpoint{1.153066in}{1.011278in}}%
\pgfpathlineto{\pgfqpoint{1.142488in}{1.019208in}}%
\pgfpathlineto{\pgfqpoint{1.140444in}{1.020885in}}%
\pgfpathlineto{\pgfqpoint{1.131911in}{1.027432in}}%
\pgfpathlineto{\pgfqpoint{1.127165in}{1.031179in}}%
\pgfpathlineto{\pgfqpoint{1.121334in}{1.036111in}}%
\pgfpathlineto{\pgfqpoint{1.115531in}{1.041474in}}%
\pgfpathlineto{\pgfqpoint{1.110757in}{1.045925in}}%
\pgfpathlineto{\pgfqpoint{1.104639in}{1.051768in}}%
\pgfpathlineto{\pgfqpoint{1.100180in}{1.056065in}}%
\pgfpathlineto{\pgfqpoint{1.094120in}{1.062063in}}%
\pgfpathlineto{\pgfqpoint{1.089603in}{1.066453in}}%
\pgfpathlineto{\pgfqpoint{1.083340in}{1.072357in}}%
\pgfpathlineto{\pgfqpoint{1.079026in}{1.076252in}}%
\pgfpathlineto{\pgfqpoint{1.072581in}{1.082651in}}%
\pgfpathlineto{\pgfqpoint{1.068449in}{1.087343in}}%
\pgfpathlineto{\pgfqpoint{1.059427in}{1.092946in}}%
\pgfpathlineto{\pgfqpoint{1.057872in}{1.093816in}}%
\pgfpathlineto{\pgfqpoint{1.047295in}{1.095729in}}%
\pgfpathlineto{\pgfqpoint{1.036718in}{1.098925in}}%
\pgfpathlineto{\pgfqpoint{1.029448in}{1.103240in}}%
\pgfpathlineto{\pgfqpoint{1.026141in}{1.105698in}}%
\pgfpathlineto{\pgfqpoint{1.017646in}{1.113535in}}%
\pgfpathlineto{\pgfqpoint{1.015564in}{1.115467in}}%
\pgfpathlineto{\pgfqpoint{1.006823in}{1.123829in}}%
\pgfpathlineto{\pgfqpoint{1.004987in}{1.125596in}}%
\pgfpathlineto{\pgfqpoint{0.996383in}{1.134124in}}%
\pgfpathlineto{\pgfqpoint{0.994410in}{1.136110in}}%
\pgfpathlineto{\pgfqpoint{0.986904in}{1.144418in}}%
\pgfpathlineto{\pgfqpoint{0.983833in}{1.147999in}}%
\pgfpathlineto{\pgfqpoint{0.978228in}{1.154712in}}%
\pgfpathlineto{\pgfqpoint{0.973255in}{1.159332in}}%
\pgfpathlineto{\pgfqpoint{0.968031in}{1.165007in}}%
\pgfpathlineto{\pgfqpoint{0.962678in}{1.169767in}}%
\pgfpathlineto{\pgfqpoint{0.956473in}{1.175301in}}%
\pgfpathlineto{\pgfqpoint{0.952101in}{1.179113in}}%
\pgfpathlineto{\pgfqpoint{0.944999in}{1.185596in}}%
\pgfpathlineto{\pgfqpoint{0.941524in}{1.188475in}}%
\pgfpathlineto{\pgfqpoint{0.933265in}{1.195890in}}%
\pgfpathlineto{\pgfqpoint{0.930947in}{1.197858in}}%
\pgfpathlineto{\pgfqpoint{0.922326in}{1.206185in}}%
\pgfpathlineto{\pgfqpoint{0.920370in}{1.208218in}}%
\pgfpathlineto{\pgfqpoint{0.913375in}{1.216479in}}%
\pgfpathlineto{\pgfqpoint{0.909793in}{1.220681in}}%
\pgfpathlineto{\pgfqpoint{0.904748in}{1.226773in}}%
\pgfpathlineto{\pgfqpoint{0.899216in}{1.233397in}}%
\pgfpathlineto{\pgfqpoint{0.896237in}{1.237068in}}%
\pgfpathlineto{\pgfqpoint{0.888639in}{1.246275in}}%
\pgfpathlineto{\pgfqpoint{0.887771in}{1.247362in}}%
\pgfpathlineto{\pgfqpoint{0.879343in}{1.257657in}}%
\pgfpathlineto{\pgfqpoint{0.878062in}{1.259233in}}%
\pgfpathlineto{\pgfqpoint{0.871514in}{1.267951in}}%
\pgfpathlineto{\pgfqpoint{0.867485in}{1.274295in}}%
\pgfpathlineto{\pgfqpoint{0.865321in}{1.278246in}}%
\pgfpathlineto{\pgfqpoint{0.859779in}{1.288540in}}%
\pgfpathlineto{\pgfqpoint{0.856908in}{1.294065in}}%
\pgfpathlineto{\pgfqpoint{0.854449in}{1.298834in}}%
\pgfpathlineto{\pgfqpoint{0.849184in}{1.309129in}}%
\pgfpathlineto{\pgfqpoint{0.846331in}{1.314870in}}%
\pgfpathlineto{\pgfqpoint{0.844078in}{1.319423in}}%
\pgfpathlineto{\pgfqpoint{0.838888in}{1.329718in}}%
\pgfpathlineto{\pgfqpoint{0.835754in}{1.335919in}}%
\pgfpathlineto{\pgfqpoint{0.833533in}{1.340012in}}%
\pgfpathlineto{\pgfqpoint{0.827468in}{1.350307in}}%
\pgfpathlineto{\pgfqpoint{0.825177in}{1.354128in}}%
\pgfpathlineto{\pgfqpoint{0.820892in}{1.360601in}}%
\pgfpathlineto{\pgfqpoint{0.814600in}{1.370280in}}%
\pgfpathlineto{\pgfqpoint{0.814185in}{1.370896in}}%
\pgfpathlineto{\pgfqpoint{0.807225in}{1.381190in}}%
\pgfpathlineto{\pgfqpoint{0.804023in}{1.385971in}}%
\pgfpathlineto{\pgfqpoint{0.800307in}{1.391484in}}%
\pgfpathlineto{\pgfqpoint{0.793501in}{1.401779in}}%
\pgfpathlineto{\pgfqpoint{0.793445in}{1.401863in}}%
\pgfpathlineto{\pgfqpoint{0.786697in}{1.412073in}}%
\pgfpathlineto{\pgfqpoint{0.782868in}{1.417893in}}%
\pgfpathlineto{\pgfqpoint{0.779925in}{1.422368in}}%
\pgfpathlineto{\pgfqpoint{0.773215in}{1.432662in}}%
\pgfpathlineto{\pgfqpoint{0.772291in}{1.434149in}}%
\pgfpathlineto{\pgfqpoint{0.766684in}{1.442957in}}%
\pgfpathlineto{\pgfqpoint{0.761714in}{1.451263in}}%
\pgfpathlineto{\pgfqpoint{0.760526in}{1.453251in}}%
\pgfpathlineto{\pgfqpoint{0.754425in}{1.463545in}}%
\pgfpathlineto{\pgfqpoint{0.751137in}{1.469128in}}%
\pgfpathlineto{\pgfqpoint{0.748367in}{1.473840in}}%
\pgfpathlineto{\pgfqpoint{0.742353in}{1.484134in}}%
\pgfpathlineto{\pgfqpoint{0.740560in}{1.487219in}}%
\pgfpathlineto{\pgfqpoint{0.736378in}{1.494429in}}%
\pgfpathlineto{\pgfqpoint{0.730449in}{1.504723in}}%
\pgfpathlineto{\pgfqpoint{0.729983in}{1.505589in}}%
\pgfpathlineto{\pgfqpoint{0.724904in}{1.515018in}}%
\pgfpathlineto{\pgfqpoint{0.719406in}{1.515018in}}%
\pgfpathlineto{\pgfqpoint{0.708829in}{1.515018in}}%
\pgfpathlineto{\pgfqpoint{0.698252in}{1.515018in}}%
\pgfpathlineto{\pgfqpoint{0.687675in}{1.515018in}}%
\pgfpathlineto{\pgfqpoint{0.684698in}{1.515018in}}%
\pgfpathlineto{\pgfqpoint{0.687675in}{1.509672in}}%
\pgfpathlineto{\pgfqpoint{0.690427in}{1.504723in}}%
\pgfpathlineto{\pgfqpoint{0.696197in}{1.494429in}}%
\pgfpathlineto{\pgfqpoint{0.698252in}{1.490792in}}%
\pgfpathlineto{\pgfqpoint{0.702010in}{1.484134in}}%
\pgfpathlineto{\pgfqpoint{0.707868in}{1.473840in}}%
\pgfpathlineto{\pgfqpoint{0.708829in}{1.472242in}}%
\pgfpathlineto{\pgfqpoint{0.713923in}{1.463545in}}%
\pgfpathlineto{\pgfqpoint{0.719406in}{1.454826in}}%
\pgfpathlineto{\pgfqpoint{0.720389in}{1.453251in}}%
\pgfpathlineto{\pgfqpoint{0.726948in}{1.442957in}}%
\pgfpathlineto{\pgfqpoint{0.729983in}{1.438223in}}%
\pgfpathlineto{\pgfqpoint{0.733551in}{1.432662in}}%
\pgfpathlineto{\pgfqpoint{0.740197in}{1.422368in}}%
\pgfpathlineto{\pgfqpoint{0.740560in}{1.421809in}}%
\pgfpathlineto{\pgfqpoint{0.746903in}{1.412073in}}%
\pgfpathlineto{\pgfqpoint{0.751137in}{1.405376in}}%
\pgfpathlineto{\pgfqpoint{0.753466in}{1.401779in}}%
\pgfpathlineto{\pgfqpoint{0.759704in}{1.391484in}}%
\pgfpathlineto{\pgfqpoint{0.761714in}{1.388079in}}%
\pgfpathlineto{\pgfqpoint{0.765795in}{1.381190in}}%
\pgfpathlineto{\pgfqpoint{0.771909in}{1.370896in}}%
\pgfpathlineto{\pgfqpoint{0.772291in}{1.370259in}}%
\pgfpathlineto{\pgfqpoint{0.778083in}{1.360601in}}%
\pgfpathlineto{\pgfqpoint{0.782868in}{1.352633in}}%
\pgfpathlineto{\pgfqpoint{0.784272in}{1.350307in}}%
\pgfpathlineto{\pgfqpoint{0.789920in}{1.340012in}}%
\pgfpathlineto{\pgfqpoint{0.793445in}{1.333495in}}%
\pgfpathlineto{\pgfqpoint{0.795465in}{1.329718in}}%
\pgfpathlineto{\pgfqpoint{0.801097in}{1.319423in}}%
\pgfpathlineto{\pgfqpoint{0.804023in}{1.314055in}}%
\pgfpathlineto{\pgfqpoint{0.806701in}{1.309129in}}%
\pgfpathlineto{\pgfqpoint{0.812226in}{1.298834in}}%
\pgfpathlineto{\pgfqpoint{0.814600in}{1.294160in}}%
\pgfpathlineto{\pgfqpoint{0.817133in}{1.288540in}}%
\pgfpathlineto{\pgfqpoint{0.821534in}{1.278246in}}%
\pgfpathlineto{\pgfqpoint{0.825177in}{1.269086in}}%
\pgfpathlineto{\pgfqpoint{0.825619in}{1.267951in}}%
\pgfpathlineto{\pgfqpoint{0.829112in}{1.257657in}}%
\pgfpathlineto{\pgfqpoint{0.832892in}{1.247362in}}%
\pgfpathlineto{\pgfqpoint{0.835754in}{1.238939in}}%
\pgfpathlineto{\pgfqpoint{0.836374in}{1.237068in}}%
\pgfpathlineto{\pgfqpoint{0.840023in}{1.226773in}}%
\pgfpathlineto{\pgfqpoint{0.843674in}{1.216479in}}%
\pgfpathlineto{\pgfqpoint{0.846331in}{1.210763in}}%
\pgfpathlineto{\pgfqpoint{0.848980in}{1.206185in}}%
\pgfpathlineto{\pgfqpoint{0.855242in}{1.195890in}}%
\pgfpathlineto{\pgfqpoint{0.856908in}{1.193185in}}%
\pgfpathlineto{\pgfqpoint{0.861742in}{1.185596in}}%
\pgfpathlineto{\pgfqpoint{0.867485in}{1.175519in}}%
\pgfpathlineto{\pgfqpoint{0.867611in}{1.175301in}}%
\pgfpathlineto{\pgfqpoint{0.873461in}{1.165007in}}%
\pgfpathlineto{\pgfqpoint{0.878062in}{1.157315in}}%
\pgfpathlineto{\pgfqpoint{0.879846in}{1.154712in}}%
\pgfpathlineto{\pgfqpoint{0.888420in}{1.144418in}}%
\pgfpathlineto{\pgfqpoint{0.888639in}{1.144159in}}%
\pgfpathlineto{\pgfqpoint{0.897204in}{1.134124in}}%
\pgfpathlineto{\pgfqpoint{0.899216in}{1.131703in}}%
\pgfpathlineto{\pgfqpoint{0.905968in}{1.123829in}}%
\pgfpathlineto{\pgfqpoint{0.909793in}{1.119199in}}%
\pgfpathlineto{\pgfqpoint{0.914553in}{1.113535in}}%
\pgfpathlineto{\pgfqpoint{0.920370in}{1.106628in}}%
\pgfpathlineto{\pgfqpoint{0.923271in}{1.103240in}}%
\pgfpathlineto{\pgfqpoint{0.930947in}{1.093907in}}%
\pgfpathlineto{\pgfqpoint{0.931758in}{1.092946in}}%
\pgfpathlineto{\pgfqpoint{0.940418in}{1.082651in}}%
\pgfpathlineto{\pgfqpoint{0.941524in}{1.081363in}}%
\pgfpathlineto{\pgfqpoint{0.949305in}{1.072357in}}%
\pgfpathlineto{\pgfqpoint{0.952101in}{1.069172in}}%
\pgfpathlineto{\pgfqpoint{0.959049in}{1.062063in}}%
\pgfpathlineto{\pgfqpoint{0.962678in}{1.058492in}}%
\pgfpathlineto{\pgfqpoint{0.969292in}{1.051768in}}%
\pgfpathlineto{\pgfqpoint{0.973255in}{1.047398in}}%
\pgfpathlineto{\pgfqpoint{0.978225in}{1.041474in}}%
\pgfpathlineto{\pgfqpoint{0.983833in}{1.035941in}}%
\pgfpathlineto{\pgfqpoint{0.991118in}{1.031179in}}%
\pgfpathlineto{\pgfqpoint{0.994410in}{1.029201in}}%
\pgfpathlineto{\pgfqpoint{1.004987in}{1.023907in}}%
\pgfpathlineto{\pgfqpoint{1.011458in}{1.020885in}}%
\pgfpathlineto{\pgfqpoint{1.015564in}{1.019011in}}%
\pgfpathlineto{\pgfqpoint{1.026141in}{1.015720in}}%
\pgfpathlineto{\pgfqpoint{1.036718in}{1.014016in}}%
\pgfpathlineto{\pgfqpoint{1.047295in}{1.012430in}}%
\pgfpathlineto{\pgfqpoint{1.054408in}{1.010590in}}%
\pgfpathlineto{\pgfqpoint{1.057872in}{1.009689in}}%
\pgfpathlineto{\pgfqpoint{1.068449in}{1.006260in}}%
\pgfpathlineto{\pgfqpoint{1.074161in}{1.000296in}}%
\pgfpathlineto{\pgfqpoint{1.079026in}{0.995471in}}%
\pgfpathlineto{\pgfqpoint{1.083909in}{0.990002in}}%
\pgfpathlineto{\pgfqpoint{1.089603in}{0.983737in}}%
\pgfpathlineto{\pgfqpoint{1.093475in}{0.979707in}}%
\pgfpathlineto{\pgfqpoint{1.100180in}{0.972995in}}%
\pgfpathlineto{\pgfqpoint{1.103972in}{0.969413in}}%
\pgfpathlineto{\pgfqpoint{1.110757in}{0.963140in}}%
\pgfpathlineto{\pgfqpoint{1.115411in}{0.959118in}}%
\pgfpathlineto{\pgfqpoint{1.121334in}{0.954157in}}%
\pgfpathlineto{\pgfqpoint{1.127840in}{0.948824in}}%
\pgfpathlineto{\pgfqpoint{1.131911in}{0.945542in}}%
\pgfpathlineto{\pgfqpoint{1.141531in}{0.938529in}}%
\pgfpathlineto{\pgfqpoint{1.142488in}{0.937855in}}%
\pgfpathlineto{\pgfqpoint{1.153066in}{0.930520in}}%
\pgfpathlineto{\pgfqpoint{1.156425in}{0.928235in}}%
\pgfpathlineto{\pgfqpoint{1.163643in}{0.923382in}}%
\pgfpathlineto{\pgfqpoint{1.171856in}{0.917941in}}%
\pgfpathlineto{\pgfqpoint{1.174220in}{0.916393in}}%
\pgfpathlineto{\pgfqpoint{1.184797in}{0.908813in}}%
\pgfpathlineto{\pgfqpoint{1.186191in}{0.907646in}}%
\pgfpathlineto{\pgfqpoint{1.194632in}{0.897352in}}%
\pgfpathlineto{\pgfqpoint{1.195374in}{0.896368in}}%
\pgfpathlineto{\pgfqpoint{1.203678in}{0.887057in}}%
\pgfpathlineto{\pgfqpoint{1.205951in}{0.884620in}}%
\pgfpathlineto{\pgfqpoint{1.214479in}{0.876763in}}%
\pgfpathlineto{\pgfqpoint{1.216528in}{0.874925in}}%
\pgfpathlineto{\pgfqpoint{1.226868in}{0.866468in}}%
\pgfpathlineto{\pgfqpoint{1.227105in}{0.866276in}}%
\pgfpathlineto{\pgfqpoint{1.237682in}{0.858203in}}%
\pgfpathlineto{\pgfqpoint{1.240528in}{0.856174in}}%
\pgfpathlineto{\pgfqpoint{1.248259in}{0.850783in}}%
\pgfpathlineto{\pgfqpoint{1.255461in}{0.845879in}}%
\pgfpathlineto{\pgfqpoint{1.258836in}{0.843859in}}%
\pgfpathlineto{\pgfqpoint{1.269413in}{0.838182in}}%
\pgfpathlineto{\pgfqpoint{1.274379in}{0.835585in}}%
\pgfpathlineto{\pgfqpoint{1.279990in}{0.832696in}}%
\pgfpathlineto{\pgfqpoint{1.290567in}{0.827101in}}%
\pgfpathlineto{\pgfqpoint{1.293941in}{0.825291in}}%
\pgfpathlineto{\pgfqpoint{1.301144in}{0.821479in}}%
\pgfpathlineto{\pgfqpoint{1.311721in}{0.816013in}}%
\pgfpathlineto{\pgfqpoint{1.313743in}{0.814996in}}%
\pgfpathclose%
\pgfusepath{fill}%
\end{pgfscope}%
\begin{pgfscope}%
\pgfpathrectangle{\pgfqpoint{0.423750in}{0.423750in}}{\pgfqpoint{1.194205in}{1.163386in}}%
\pgfusepath{clip}%
\pgfsetbuttcap%
\pgfsetroundjoin%
\definecolor{currentfill}{rgb}{0.966120,0.744512,0.608720}%
\pgfsetfillcolor{currentfill}%
\pgfsetlinewidth{0.000000pt}%
\definecolor{currentstroke}{rgb}{0.000000,0.000000,0.000000}%
\pgfsetstrokecolor{currentstroke}%
\pgfsetdash{}{0pt}%
\pgfpathmoveto{\pgfqpoint{0.941524in}{0.497011in}}%
\pgfpathlineto{\pgfqpoint{0.942041in}{0.495869in}}%
\pgfpathlineto{\pgfqpoint{0.952101in}{0.495869in}}%
\pgfpathlineto{\pgfqpoint{0.962678in}{0.495869in}}%
\pgfpathlineto{\pgfqpoint{0.973255in}{0.495869in}}%
\pgfpathlineto{\pgfqpoint{0.983833in}{0.495869in}}%
\pgfpathlineto{\pgfqpoint{0.994410in}{0.495869in}}%
\pgfpathlineto{\pgfqpoint{1.004987in}{0.495869in}}%
\pgfpathlineto{\pgfqpoint{1.015019in}{0.495869in}}%
\pgfpathlineto{\pgfqpoint{1.010379in}{0.506163in}}%
\pgfpathlineto{\pgfqpoint{1.005621in}{0.516458in}}%
\pgfpathlineto{\pgfqpoint{1.004987in}{0.517799in}}%
\pgfpathlineto{\pgfqpoint{1.000723in}{0.526752in}}%
\pgfpathlineto{\pgfqpoint{0.995758in}{0.537047in}}%
\pgfpathlineto{\pgfqpoint{0.994410in}{0.539805in}}%
\pgfpathlineto{\pgfqpoint{0.990712in}{0.547341in}}%
\pgfpathlineto{\pgfqpoint{0.985608in}{0.557635in}}%
\pgfpathlineto{\pgfqpoint{0.983833in}{0.561322in}}%
\pgfpathlineto{\pgfqpoint{0.980698in}{0.567930in}}%
\pgfpathlineto{\pgfqpoint{0.975645in}{0.578224in}}%
\pgfpathlineto{\pgfqpoint{0.973255in}{0.582995in}}%
\pgfpathlineto{\pgfqpoint{0.970457in}{0.588519in}}%
\pgfpathlineto{\pgfqpoint{0.964898in}{0.598813in}}%
\pgfpathlineto{\pgfqpoint{0.962678in}{0.602578in}}%
\pgfpathlineto{\pgfqpoint{0.958753in}{0.609108in}}%
\pgfpathlineto{\pgfqpoint{0.953421in}{0.619402in}}%
\pgfpathlineto{\pgfqpoint{0.952101in}{0.622705in}}%
\pgfpathlineto{\pgfqpoint{0.949226in}{0.629696in}}%
\pgfpathlineto{\pgfqpoint{0.944830in}{0.639991in}}%
\pgfpathlineto{\pgfqpoint{0.941524in}{0.647580in}}%
\pgfpathlineto{\pgfqpoint{0.940642in}{0.650285in}}%
\pgfpathlineto{\pgfqpoint{0.937783in}{0.660580in}}%
\pgfpathlineto{\pgfqpoint{0.934854in}{0.670874in}}%
\pgfpathlineto{\pgfqpoint{0.931811in}{0.681169in}}%
\pgfpathlineto{\pgfqpoint{0.930947in}{0.683914in}}%
\pgfpathlineto{\pgfqpoint{0.928324in}{0.691463in}}%
\pgfpathlineto{\pgfqpoint{0.923079in}{0.701757in}}%
\pgfpathlineto{\pgfqpoint{0.920370in}{0.708276in}}%
\pgfpathlineto{\pgfqpoint{0.918587in}{0.712052in}}%
\pgfpathlineto{\pgfqpoint{0.916829in}{0.722346in}}%
\pgfpathlineto{\pgfqpoint{0.915274in}{0.732641in}}%
\pgfpathlineto{\pgfqpoint{0.913963in}{0.742935in}}%
\pgfpathlineto{\pgfqpoint{0.912690in}{0.753230in}}%
\pgfpathlineto{\pgfqpoint{0.911319in}{0.763524in}}%
\pgfpathlineto{\pgfqpoint{0.909814in}{0.773818in}}%
\pgfpathlineto{\pgfqpoint{0.909793in}{0.773953in}}%
\pgfpathlineto{\pgfqpoint{0.908169in}{0.784113in}}%
\pgfpathlineto{\pgfqpoint{0.906437in}{0.794407in}}%
\pgfpathlineto{\pgfqpoint{0.904554in}{0.804702in}}%
\pgfpathlineto{\pgfqpoint{0.902508in}{0.814996in}}%
\pgfpathlineto{\pgfqpoint{0.900519in}{0.825291in}}%
\pgfpathlineto{\pgfqpoint{0.899677in}{0.835585in}}%
\pgfpathlineto{\pgfqpoint{0.899216in}{0.841600in}}%
\pgfpathlineto{\pgfqpoint{0.898894in}{0.845879in}}%
\pgfpathlineto{\pgfqpoint{0.899216in}{0.849789in}}%
\pgfpathlineto{\pgfqpoint{0.899859in}{0.856174in}}%
\pgfpathlineto{\pgfqpoint{0.900499in}{0.866468in}}%
\pgfpathlineto{\pgfqpoint{0.900210in}{0.876763in}}%
\pgfpathlineto{\pgfqpoint{0.899568in}{0.887057in}}%
\pgfpathlineto{\pgfqpoint{0.909793in}{0.895237in}}%
\pgfpathlineto{\pgfqpoint{0.911339in}{0.897352in}}%
\pgfpathlineto{\pgfqpoint{0.917946in}{0.907646in}}%
\pgfpathlineto{\pgfqpoint{0.920370in}{0.909416in}}%
\pgfpathlineto{\pgfqpoint{0.930947in}{0.913850in}}%
\pgfpathlineto{\pgfqpoint{0.941524in}{0.916578in}}%
\pgfpathlineto{\pgfqpoint{0.952101in}{0.916553in}}%
\pgfpathlineto{\pgfqpoint{0.962678in}{0.916978in}}%
\pgfpathlineto{\pgfqpoint{0.973255in}{0.916558in}}%
\pgfpathlineto{\pgfqpoint{0.983833in}{0.915104in}}%
\pgfpathlineto{\pgfqpoint{0.994410in}{0.913824in}}%
\pgfpathlineto{\pgfqpoint{1.004987in}{0.912873in}}%
\pgfpathlineto{\pgfqpoint{1.015564in}{0.912573in}}%
\pgfpathlineto{\pgfqpoint{1.026141in}{0.912908in}}%
\pgfpathlineto{\pgfqpoint{1.036718in}{0.912658in}}%
\pgfpathlineto{\pgfqpoint{1.047295in}{0.912429in}}%
\pgfpathlineto{\pgfqpoint{1.057872in}{0.910501in}}%
\pgfpathlineto{\pgfqpoint{1.065577in}{0.907646in}}%
\pgfpathlineto{\pgfqpoint{1.068449in}{0.906555in}}%
\pgfpathlineto{\pgfqpoint{1.079026in}{0.900465in}}%
\pgfpathlineto{\pgfqpoint{1.082299in}{0.897352in}}%
\pgfpathlineto{\pgfqpoint{1.089603in}{0.889759in}}%
\pgfpathlineto{\pgfqpoint{1.092274in}{0.887057in}}%
\pgfpathlineto{\pgfqpoint{1.100180in}{0.879228in}}%
\pgfpathlineto{\pgfqpoint{1.102847in}{0.876763in}}%
\pgfpathlineto{\pgfqpoint{1.110757in}{0.869709in}}%
\pgfpathlineto{\pgfqpoint{1.114390in}{0.866468in}}%
\pgfpathlineto{\pgfqpoint{1.121334in}{0.859661in}}%
\pgfpathlineto{\pgfqpoint{1.124355in}{0.856174in}}%
\pgfpathlineto{\pgfqpoint{1.130733in}{0.845879in}}%
\pgfpathlineto{\pgfqpoint{1.131911in}{0.843646in}}%
\pgfpathlineto{\pgfqpoint{1.136201in}{0.835585in}}%
\pgfpathlineto{\pgfqpoint{1.141975in}{0.825291in}}%
\pgfpathlineto{\pgfqpoint{1.142488in}{0.824417in}}%
\pgfpathlineto{\pgfqpoint{1.148295in}{0.814996in}}%
\pgfpathlineto{\pgfqpoint{1.153066in}{0.808244in}}%
\pgfpathlineto{\pgfqpoint{1.155653in}{0.804702in}}%
\pgfpathlineto{\pgfqpoint{1.163643in}{0.795458in}}%
\pgfpathlineto{\pgfqpoint{1.164688in}{0.794407in}}%
\pgfpathlineto{\pgfqpoint{1.174220in}{0.786004in}}%
\pgfpathlineto{\pgfqpoint{1.176444in}{0.784113in}}%
\pgfpathlineto{\pgfqpoint{1.184797in}{0.777158in}}%
\pgfpathlineto{\pgfqpoint{1.189774in}{0.773818in}}%
\pgfpathlineto{\pgfqpoint{1.195374in}{0.770138in}}%
\pgfpathlineto{\pgfqpoint{1.205951in}{0.764825in}}%
\pgfpathlineto{\pgfqpoint{1.208627in}{0.763524in}}%
\pgfpathlineto{\pgfqpoint{1.216528in}{0.759679in}}%
\pgfpathlineto{\pgfqpoint{1.227105in}{0.754707in}}%
\pgfpathlineto{\pgfqpoint{1.230348in}{0.753230in}}%
\pgfpathlineto{\pgfqpoint{1.237682in}{0.749542in}}%
\pgfpathlineto{\pgfqpoint{1.248259in}{0.744291in}}%
\pgfpathlineto{\pgfqpoint{1.250991in}{0.742935in}}%
\pgfpathlineto{\pgfqpoint{1.258836in}{0.737014in}}%
\pgfpathlineto{\pgfqpoint{1.264747in}{0.732641in}}%
\pgfpathlineto{\pgfqpoint{1.269413in}{0.729379in}}%
\pgfpathlineto{\pgfqpoint{1.279690in}{0.722346in}}%
\pgfpathlineto{\pgfqpoint{1.279990in}{0.722151in}}%
\pgfpathlineto{\pgfqpoint{1.290567in}{0.715450in}}%
\pgfpathlineto{\pgfqpoint{1.296049in}{0.712052in}}%
\pgfpathlineto{\pgfqpoint{1.301144in}{0.708863in}}%
\pgfpathlineto{\pgfqpoint{1.311721in}{0.702535in}}%
\pgfpathlineto{\pgfqpoint{1.313078in}{0.701757in}}%
\pgfpathlineto{\pgfqpoint{1.322299in}{0.694639in}}%
\pgfpathlineto{\pgfqpoint{1.326469in}{0.691463in}}%
\pgfpathlineto{\pgfqpoint{1.332876in}{0.684651in}}%
\pgfpathlineto{\pgfqpoint{1.335723in}{0.691463in}}%
\pgfpathlineto{\pgfqpoint{1.342530in}{0.701757in}}%
\pgfpathlineto{\pgfqpoint{1.343453in}{0.702808in}}%
\pgfpathlineto{\pgfqpoint{1.352018in}{0.712052in}}%
\pgfpathlineto{\pgfqpoint{1.354030in}{0.714286in}}%
\pgfpathlineto{\pgfqpoint{1.361281in}{0.722346in}}%
\pgfpathlineto{\pgfqpoint{1.364607in}{0.726089in}}%
\pgfpathlineto{\pgfqpoint{1.370356in}{0.732641in}}%
\pgfpathlineto{\pgfqpoint{1.375184in}{0.738848in}}%
\pgfpathlineto{\pgfqpoint{1.378673in}{0.742935in}}%
\pgfpathlineto{\pgfqpoint{1.385761in}{0.749100in}}%
\pgfpathlineto{\pgfqpoint{1.389390in}{0.753230in}}%
\pgfpathlineto{\pgfqpoint{1.396338in}{0.760814in}}%
\pgfpathlineto{\pgfqpoint{1.398929in}{0.763524in}}%
\pgfpathlineto{\pgfqpoint{1.406915in}{0.771987in}}%
\pgfpathlineto{\pgfqpoint{1.408634in}{0.773818in}}%
\pgfpathlineto{\pgfqpoint{1.417492in}{0.783330in}}%
\pgfpathlineto{\pgfqpoint{1.418246in}{0.784113in}}%
\pgfpathlineto{\pgfqpoint{1.428069in}{0.794123in}}%
\pgfpathlineto{\pgfqpoint{1.428317in}{0.794407in}}%
\pgfpathlineto{\pgfqpoint{1.435069in}{0.804702in}}%
\pgfpathlineto{\pgfqpoint{1.438646in}{0.813242in}}%
\pgfpathlineto{\pgfqpoint{1.439384in}{0.814996in}}%
\pgfpathlineto{\pgfqpoint{1.443725in}{0.825291in}}%
\pgfpathlineto{\pgfqpoint{1.447695in}{0.835585in}}%
\pgfpathlineto{\pgfqpoint{1.449223in}{0.839470in}}%
\pgfpathlineto{\pgfqpoint{1.451786in}{0.845879in}}%
\pgfpathlineto{\pgfqpoint{1.455738in}{0.856174in}}%
\pgfpathlineto{\pgfqpoint{1.459687in}{0.866468in}}%
\pgfpathlineto{\pgfqpoint{1.459800in}{0.866702in}}%
\pgfpathlineto{\pgfqpoint{1.465719in}{0.876763in}}%
\pgfpathlineto{\pgfqpoint{1.470377in}{0.883062in}}%
\pgfpathlineto{\pgfqpoint{1.474303in}{0.887057in}}%
\pgfpathlineto{\pgfqpoint{1.480954in}{0.894412in}}%
\pgfpathlineto{\pgfqpoint{1.484059in}{0.897352in}}%
\pgfpathlineto{\pgfqpoint{1.491532in}{0.904488in}}%
\pgfpathlineto{\pgfqpoint{1.494819in}{0.907646in}}%
\pgfpathlineto{\pgfqpoint{1.502109in}{0.914698in}}%
\pgfpathlineto{\pgfqpoint{1.505433in}{0.917941in}}%
\pgfpathlineto{\pgfqpoint{1.512686in}{0.925051in}}%
\pgfpathlineto{\pgfqpoint{1.515902in}{0.928235in}}%
\pgfpathlineto{\pgfqpoint{1.523263in}{0.934615in}}%
\pgfpathlineto{\pgfqpoint{1.527932in}{0.938529in}}%
\pgfpathlineto{\pgfqpoint{1.533840in}{0.943509in}}%
\pgfpathlineto{\pgfqpoint{1.540135in}{0.948824in}}%
\pgfpathlineto{\pgfqpoint{1.544417in}{0.952051in}}%
\pgfpathlineto{\pgfqpoint{1.544417in}{0.959118in}}%
\pgfpathlineto{\pgfqpoint{1.544417in}{0.969413in}}%
\pgfpathlineto{\pgfqpoint{1.544417in}{0.979707in}}%
\pgfpathlineto{\pgfqpoint{1.544417in}{0.990002in}}%
\pgfpathlineto{\pgfqpoint{1.544417in}{1.000296in}}%
\pgfpathlineto{\pgfqpoint{1.544417in}{1.010590in}}%
\pgfpathlineto{\pgfqpoint{1.544417in}{1.020885in}}%
\pgfpathlineto{\pgfqpoint{1.544417in}{1.031179in}}%
\pgfpathlineto{\pgfqpoint{1.544417in}{1.041474in}}%
\pgfpathlineto{\pgfqpoint{1.544417in}{1.047105in}}%
\pgfpathlineto{\pgfqpoint{1.533840in}{1.041509in}}%
\pgfpathlineto{\pgfqpoint{1.533771in}{1.041474in}}%
\pgfpathlineto{\pgfqpoint{1.523263in}{1.036105in}}%
\pgfpathlineto{\pgfqpoint{1.514186in}{1.031179in}}%
\pgfpathlineto{\pgfqpoint{1.512686in}{1.030091in}}%
\pgfpathlineto{\pgfqpoint{1.502109in}{1.022659in}}%
\pgfpathlineto{\pgfqpoint{1.499478in}{1.020885in}}%
\pgfpathlineto{\pgfqpoint{1.491532in}{1.015177in}}%
\pgfpathlineto{\pgfqpoint{1.485106in}{1.010590in}}%
\pgfpathlineto{\pgfqpoint{1.480954in}{1.007677in}}%
\pgfpathlineto{\pgfqpoint{1.470379in}{1.000296in}}%
\pgfpathlineto{\pgfqpoint{1.470377in}{1.000295in}}%
\pgfpathlineto{\pgfqpoint{1.460441in}{0.990002in}}%
\pgfpathlineto{\pgfqpoint{1.459800in}{0.989278in}}%
\pgfpathlineto{\pgfqpoint{1.451625in}{0.979707in}}%
\pgfpathlineto{\pgfqpoint{1.449223in}{0.976724in}}%
\pgfpathlineto{\pgfqpoint{1.443339in}{0.969413in}}%
\pgfpathlineto{\pgfqpoint{1.438646in}{0.963696in}}%
\pgfpathlineto{\pgfqpoint{1.434890in}{0.959118in}}%
\pgfpathlineto{\pgfqpoint{1.428069in}{0.950744in}}%
\pgfpathlineto{\pgfqpoint{1.426562in}{0.948824in}}%
\pgfpathlineto{\pgfqpoint{1.419757in}{0.938529in}}%
\pgfpathlineto{\pgfqpoint{1.417492in}{0.934712in}}%
\pgfpathlineto{\pgfqpoint{1.413661in}{0.928235in}}%
\pgfpathlineto{\pgfqpoint{1.407433in}{0.917941in}}%
\pgfpathlineto{\pgfqpoint{1.406915in}{0.917036in}}%
\pgfpathlineto{\pgfqpoint{1.402706in}{0.907646in}}%
\pgfpathlineto{\pgfqpoint{1.398967in}{0.897352in}}%
\pgfpathlineto{\pgfqpoint{1.396338in}{0.890352in}}%
\pgfpathlineto{\pgfqpoint{1.395102in}{0.887057in}}%
\pgfpathlineto{\pgfqpoint{1.391112in}{0.876763in}}%
\pgfpathlineto{\pgfqpoint{1.386962in}{0.866468in}}%
\pgfpathlineto{\pgfqpoint{1.385761in}{0.864134in}}%
\pgfpathlineto{\pgfqpoint{1.380071in}{0.856174in}}%
\pgfpathlineto{\pgfqpoint{1.375184in}{0.850857in}}%
\pgfpathlineto{\pgfqpoint{1.370465in}{0.845879in}}%
\pgfpathlineto{\pgfqpoint{1.364607in}{0.839839in}}%
\pgfpathlineto{\pgfqpoint{1.360350in}{0.835585in}}%
\pgfpathlineto{\pgfqpoint{1.354030in}{0.829301in}}%
\pgfpathlineto{\pgfqpoint{1.349981in}{0.825291in}}%
\pgfpathlineto{\pgfqpoint{1.343453in}{0.818944in}}%
\pgfpathlineto{\pgfqpoint{1.338639in}{0.814996in}}%
\pgfpathlineto{\pgfqpoint{1.332876in}{0.810975in}}%
\pgfpathlineto{\pgfqpoint{1.322299in}{0.810743in}}%
\pgfpathlineto{\pgfqpoint{1.313743in}{0.814996in}}%
\pgfpathlineto{\pgfqpoint{1.311721in}{0.816013in}}%
\pgfpathlineto{\pgfqpoint{1.301144in}{0.821479in}}%
\pgfpathlineto{\pgfqpoint{1.293941in}{0.825291in}}%
\pgfpathlineto{\pgfqpoint{1.290567in}{0.827101in}}%
\pgfpathlineto{\pgfqpoint{1.279990in}{0.832696in}}%
\pgfpathlineto{\pgfqpoint{1.274379in}{0.835585in}}%
\pgfpathlineto{\pgfqpoint{1.269413in}{0.838182in}}%
\pgfpathlineto{\pgfqpoint{1.258836in}{0.843859in}}%
\pgfpathlineto{\pgfqpoint{1.255461in}{0.845879in}}%
\pgfpathlineto{\pgfqpoint{1.248259in}{0.850783in}}%
\pgfpathlineto{\pgfqpoint{1.240528in}{0.856174in}}%
\pgfpathlineto{\pgfqpoint{1.237682in}{0.858203in}}%
\pgfpathlineto{\pgfqpoint{1.227105in}{0.866276in}}%
\pgfpathlineto{\pgfqpoint{1.226868in}{0.866468in}}%
\pgfpathlineto{\pgfqpoint{1.216528in}{0.874925in}}%
\pgfpathlineto{\pgfqpoint{1.214479in}{0.876763in}}%
\pgfpathlineto{\pgfqpoint{1.205951in}{0.884620in}}%
\pgfpathlineto{\pgfqpoint{1.203678in}{0.887057in}}%
\pgfpathlineto{\pgfqpoint{1.195374in}{0.896368in}}%
\pgfpathlineto{\pgfqpoint{1.194632in}{0.897352in}}%
\pgfpathlineto{\pgfqpoint{1.186191in}{0.907646in}}%
\pgfpathlineto{\pgfqpoint{1.184797in}{0.908813in}}%
\pgfpathlineto{\pgfqpoint{1.174220in}{0.916393in}}%
\pgfpathlineto{\pgfqpoint{1.171856in}{0.917941in}}%
\pgfpathlineto{\pgfqpoint{1.163643in}{0.923382in}}%
\pgfpathlineto{\pgfqpoint{1.156425in}{0.928235in}}%
\pgfpathlineto{\pgfqpoint{1.153066in}{0.930520in}}%
\pgfpathlineto{\pgfqpoint{1.142488in}{0.937855in}}%
\pgfpathlineto{\pgfqpoint{1.141531in}{0.938529in}}%
\pgfpathlineto{\pgfqpoint{1.131911in}{0.945542in}}%
\pgfpathlineto{\pgfqpoint{1.127840in}{0.948824in}}%
\pgfpathlineto{\pgfqpoint{1.121334in}{0.954157in}}%
\pgfpathlineto{\pgfqpoint{1.115411in}{0.959118in}}%
\pgfpathlineto{\pgfqpoint{1.110757in}{0.963140in}}%
\pgfpathlineto{\pgfqpoint{1.103972in}{0.969413in}}%
\pgfpathlineto{\pgfqpoint{1.100180in}{0.972995in}}%
\pgfpathlineto{\pgfqpoint{1.093475in}{0.979707in}}%
\pgfpathlineto{\pgfqpoint{1.089603in}{0.983737in}}%
\pgfpathlineto{\pgfqpoint{1.083909in}{0.990002in}}%
\pgfpathlineto{\pgfqpoint{1.079026in}{0.995471in}}%
\pgfpathlineto{\pgfqpoint{1.074161in}{1.000296in}}%
\pgfpathlineto{\pgfqpoint{1.068449in}{1.006260in}}%
\pgfpathlineto{\pgfqpoint{1.057872in}{1.009689in}}%
\pgfpathlineto{\pgfqpoint{1.054408in}{1.010590in}}%
\pgfpathlineto{\pgfqpoint{1.047295in}{1.012430in}}%
\pgfpathlineto{\pgfqpoint{1.036718in}{1.014016in}}%
\pgfpathlineto{\pgfqpoint{1.026141in}{1.015720in}}%
\pgfpathlineto{\pgfqpoint{1.015564in}{1.019011in}}%
\pgfpathlineto{\pgfqpoint{1.011458in}{1.020885in}}%
\pgfpathlineto{\pgfqpoint{1.004987in}{1.023907in}}%
\pgfpathlineto{\pgfqpoint{0.994410in}{1.029201in}}%
\pgfpathlineto{\pgfqpoint{0.991118in}{1.031179in}}%
\pgfpathlineto{\pgfqpoint{0.983833in}{1.035941in}}%
\pgfpathlineto{\pgfqpoint{0.978225in}{1.041474in}}%
\pgfpathlineto{\pgfqpoint{0.973255in}{1.047398in}}%
\pgfpathlineto{\pgfqpoint{0.969292in}{1.051768in}}%
\pgfpathlineto{\pgfqpoint{0.962678in}{1.058492in}}%
\pgfpathlineto{\pgfqpoint{0.959049in}{1.062063in}}%
\pgfpathlineto{\pgfqpoint{0.952101in}{1.069172in}}%
\pgfpathlineto{\pgfqpoint{0.949305in}{1.072357in}}%
\pgfpathlineto{\pgfqpoint{0.941524in}{1.081363in}}%
\pgfpathlineto{\pgfqpoint{0.940418in}{1.082651in}}%
\pgfpathlineto{\pgfqpoint{0.931758in}{1.092946in}}%
\pgfpathlineto{\pgfqpoint{0.930947in}{1.093907in}}%
\pgfpathlineto{\pgfqpoint{0.923271in}{1.103240in}}%
\pgfpathlineto{\pgfqpoint{0.920370in}{1.106628in}}%
\pgfpathlineto{\pgfqpoint{0.914553in}{1.113535in}}%
\pgfpathlineto{\pgfqpoint{0.909793in}{1.119199in}}%
\pgfpathlineto{\pgfqpoint{0.905968in}{1.123829in}}%
\pgfpathlineto{\pgfqpoint{0.899216in}{1.131703in}}%
\pgfpathlineto{\pgfqpoint{0.897204in}{1.134124in}}%
\pgfpathlineto{\pgfqpoint{0.888639in}{1.144159in}}%
\pgfpathlineto{\pgfqpoint{0.888420in}{1.144418in}}%
\pgfpathlineto{\pgfqpoint{0.879846in}{1.154712in}}%
\pgfpathlineto{\pgfqpoint{0.878062in}{1.157315in}}%
\pgfpathlineto{\pgfqpoint{0.873461in}{1.165007in}}%
\pgfpathlineto{\pgfqpoint{0.867611in}{1.175301in}}%
\pgfpathlineto{\pgfqpoint{0.867485in}{1.175519in}}%
\pgfpathlineto{\pgfqpoint{0.861742in}{1.185596in}}%
\pgfpathlineto{\pgfqpoint{0.856908in}{1.193185in}}%
\pgfpathlineto{\pgfqpoint{0.855242in}{1.195890in}}%
\pgfpathlineto{\pgfqpoint{0.848980in}{1.206185in}}%
\pgfpathlineto{\pgfqpoint{0.846331in}{1.210763in}}%
\pgfpathlineto{\pgfqpoint{0.843674in}{1.216479in}}%
\pgfpathlineto{\pgfqpoint{0.840023in}{1.226773in}}%
\pgfpathlineto{\pgfqpoint{0.836374in}{1.237068in}}%
\pgfpathlineto{\pgfqpoint{0.835754in}{1.238939in}}%
\pgfpathlineto{\pgfqpoint{0.832892in}{1.247362in}}%
\pgfpathlineto{\pgfqpoint{0.829112in}{1.257657in}}%
\pgfpathlineto{\pgfqpoint{0.825619in}{1.267951in}}%
\pgfpathlineto{\pgfqpoint{0.825177in}{1.269086in}}%
\pgfpathlineto{\pgfqpoint{0.821534in}{1.278246in}}%
\pgfpathlineto{\pgfqpoint{0.817133in}{1.288540in}}%
\pgfpathlineto{\pgfqpoint{0.814600in}{1.294160in}}%
\pgfpathlineto{\pgfqpoint{0.812226in}{1.298834in}}%
\pgfpathlineto{\pgfqpoint{0.806701in}{1.309129in}}%
\pgfpathlineto{\pgfqpoint{0.804023in}{1.314055in}}%
\pgfpathlineto{\pgfqpoint{0.801097in}{1.319423in}}%
\pgfpathlineto{\pgfqpoint{0.795465in}{1.329718in}}%
\pgfpathlineto{\pgfqpoint{0.793445in}{1.333495in}}%
\pgfpathlineto{\pgfqpoint{0.789920in}{1.340012in}}%
\pgfpathlineto{\pgfqpoint{0.784272in}{1.350307in}}%
\pgfpathlineto{\pgfqpoint{0.782868in}{1.352633in}}%
\pgfpathlineto{\pgfqpoint{0.778083in}{1.360601in}}%
\pgfpathlineto{\pgfqpoint{0.772291in}{1.370259in}}%
\pgfpathlineto{\pgfqpoint{0.771909in}{1.370896in}}%
\pgfpathlineto{\pgfqpoint{0.765795in}{1.381190in}}%
\pgfpathlineto{\pgfqpoint{0.761714in}{1.388079in}}%
\pgfpathlineto{\pgfqpoint{0.759704in}{1.391484in}}%
\pgfpathlineto{\pgfqpoint{0.753466in}{1.401779in}}%
\pgfpathlineto{\pgfqpoint{0.751137in}{1.405376in}}%
\pgfpathlineto{\pgfqpoint{0.746903in}{1.412073in}}%
\pgfpathlineto{\pgfqpoint{0.740560in}{1.421809in}}%
\pgfpathlineto{\pgfqpoint{0.740197in}{1.422368in}}%
\pgfpathlineto{\pgfqpoint{0.733551in}{1.432662in}}%
\pgfpathlineto{\pgfqpoint{0.729983in}{1.438223in}}%
\pgfpathlineto{\pgfqpoint{0.726948in}{1.442957in}}%
\pgfpathlineto{\pgfqpoint{0.720389in}{1.453251in}}%
\pgfpathlineto{\pgfqpoint{0.719406in}{1.454826in}}%
\pgfpathlineto{\pgfqpoint{0.713923in}{1.463545in}}%
\pgfpathlineto{\pgfqpoint{0.708829in}{1.472242in}}%
\pgfpathlineto{\pgfqpoint{0.707868in}{1.473840in}}%
\pgfpathlineto{\pgfqpoint{0.702010in}{1.484134in}}%
\pgfpathlineto{\pgfqpoint{0.698252in}{1.490792in}}%
\pgfpathlineto{\pgfqpoint{0.696197in}{1.494429in}}%
\pgfpathlineto{\pgfqpoint{0.690427in}{1.504723in}}%
\pgfpathlineto{\pgfqpoint{0.687675in}{1.509672in}}%
\pgfpathlineto{\pgfqpoint{0.684698in}{1.515018in}}%
\pgfpathlineto{\pgfqpoint{0.677098in}{1.515018in}}%
\pgfpathlineto{\pgfqpoint{0.666521in}{1.515018in}}%
\pgfpathlineto{\pgfqpoint{0.655944in}{1.515018in}}%
\pgfpathlineto{\pgfqpoint{0.645367in}{1.515018in}}%
\pgfpathlineto{\pgfqpoint{0.643314in}{1.515018in}}%
\pgfpathlineto{\pgfqpoint{0.645367in}{1.511402in}}%
\pgfpathlineto{\pgfqpoint{0.649171in}{1.504723in}}%
\pgfpathlineto{\pgfqpoint{0.655075in}{1.494429in}}%
\pgfpathlineto{\pgfqpoint{0.655944in}{1.493034in}}%
\pgfpathlineto{\pgfqpoint{0.661395in}{1.484134in}}%
\pgfpathlineto{\pgfqpoint{0.666521in}{1.476050in}}%
\pgfpathlineto{\pgfqpoint{0.667926in}{1.473840in}}%
\pgfpathlineto{\pgfqpoint{0.673991in}{1.463545in}}%
\pgfpathlineto{\pgfqpoint{0.677098in}{1.457897in}}%
\pgfpathlineto{\pgfqpoint{0.679699in}{1.453251in}}%
\pgfpathlineto{\pgfqpoint{0.684684in}{1.442957in}}%
\pgfpathlineto{\pgfqpoint{0.687675in}{1.436260in}}%
\pgfpathlineto{\pgfqpoint{0.689232in}{1.432662in}}%
\pgfpathlineto{\pgfqpoint{0.693557in}{1.422368in}}%
\pgfpathlineto{\pgfqpoint{0.697576in}{1.412073in}}%
\pgfpathlineto{\pgfqpoint{0.698252in}{1.410371in}}%
\pgfpathlineto{\pgfqpoint{0.701745in}{1.401779in}}%
\pgfpathlineto{\pgfqpoint{0.705895in}{1.391484in}}%
\pgfpathlineto{\pgfqpoint{0.708829in}{1.384224in}}%
\pgfpathlineto{\pgfqpoint{0.710080in}{1.381190in}}%
\pgfpathlineto{\pgfqpoint{0.714275in}{1.370896in}}%
\pgfpathlineto{\pgfqpoint{0.718536in}{1.360601in}}%
\pgfpathlineto{\pgfqpoint{0.719406in}{1.358450in}}%
\pgfpathlineto{\pgfqpoint{0.722692in}{1.350307in}}%
\pgfpathlineto{\pgfqpoint{0.726994in}{1.340012in}}%
\pgfpathlineto{\pgfqpoint{0.729983in}{1.332557in}}%
\pgfpathlineto{\pgfqpoint{0.731122in}{1.329718in}}%
\pgfpathlineto{\pgfqpoint{0.735414in}{1.319423in}}%
\pgfpathlineto{\pgfqpoint{0.739685in}{1.309129in}}%
\pgfpathlineto{\pgfqpoint{0.740560in}{1.307155in}}%
\pgfpathlineto{\pgfqpoint{0.744172in}{1.298834in}}%
\pgfpathlineto{\pgfqpoint{0.748441in}{1.288540in}}%
\pgfpathlineto{\pgfqpoint{0.751137in}{1.282640in}}%
\pgfpathlineto{\pgfqpoint{0.753077in}{1.278246in}}%
\pgfpathlineto{\pgfqpoint{0.757264in}{1.267951in}}%
\pgfpathlineto{\pgfqpoint{0.761714in}{1.258478in}}%
\pgfpathlineto{\pgfqpoint{0.762080in}{1.257657in}}%
\pgfpathlineto{\pgfqpoint{0.765624in}{1.247362in}}%
\pgfpathlineto{\pgfqpoint{0.768996in}{1.237068in}}%
\pgfpathlineto{\pgfqpoint{0.772291in}{1.228815in}}%
\pgfpathlineto{\pgfqpoint{0.772967in}{1.226773in}}%
\pgfpathlineto{\pgfqpoint{0.775568in}{1.216479in}}%
\pgfpathlineto{\pgfqpoint{0.779197in}{1.206185in}}%
\pgfpathlineto{\pgfqpoint{0.781043in}{1.195890in}}%
\pgfpathlineto{\pgfqpoint{0.782868in}{1.187129in}}%
\pgfpathlineto{\pgfqpoint{0.783133in}{1.185596in}}%
\pgfpathlineto{\pgfqpoint{0.784493in}{1.175301in}}%
\pgfpathlineto{\pgfqpoint{0.786554in}{1.165007in}}%
\pgfpathlineto{\pgfqpoint{0.788894in}{1.154712in}}%
\pgfpathlineto{\pgfqpoint{0.789546in}{1.144418in}}%
\pgfpathlineto{\pgfqpoint{0.789463in}{1.134124in}}%
\pgfpathlineto{\pgfqpoint{0.790376in}{1.123829in}}%
\pgfpathlineto{\pgfqpoint{0.791191in}{1.113535in}}%
\pgfpathlineto{\pgfqpoint{0.791550in}{1.103240in}}%
\pgfpathlineto{\pgfqpoint{0.792157in}{1.092946in}}%
\pgfpathlineto{\pgfqpoint{0.791002in}{1.082651in}}%
\pgfpathlineto{\pgfqpoint{0.789250in}{1.072357in}}%
\pgfpathlineto{\pgfqpoint{0.786624in}{1.062063in}}%
\pgfpathlineto{\pgfqpoint{0.782981in}{1.051768in}}%
\pgfpathlineto{\pgfqpoint{0.782868in}{1.051196in}}%
\pgfpathlineto{\pgfqpoint{0.780731in}{1.041474in}}%
\pgfpathlineto{\pgfqpoint{0.782868in}{1.033172in}}%
\pgfpathlineto{\pgfqpoint{0.783816in}{1.031179in}}%
\pgfpathlineto{\pgfqpoint{0.789293in}{1.020885in}}%
\pgfpathlineto{\pgfqpoint{0.793445in}{1.012802in}}%
\pgfpathlineto{\pgfqpoint{0.794571in}{1.010590in}}%
\pgfpathlineto{\pgfqpoint{0.801174in}{1.000296in}}%
\pgfpathlineto{\pgfqpoint{0.804023in}{0.996000in}}%
\pgfpathlineto{\pgfqpoint{0.807706in}{0.990002in}}%
\pgfpathlineto{\pgfqpoint{0.814600in}{0.981997in}}%
\pgfpathlineto{\pgfqpoint{0.816247in}{0.979707in}}%
\pgfpathlineto{\pgfqpoint{0.823891in}{0.969413in}}%
\pgfpathlineto{\pgfqpoint{0.825177in}{0.968028in}}%
\pgfpathlineto{\pgfqpoint{0.833037in}{0.959118in}}%
\pgfpathlineto{\pgfqpoint{0.835754in}{0.956148in}}%
\pgfpathlineto{\pgfqpoint{0.843183in}{0.948824in}}%
\pgfpathlineto{\pgfqpoint{0.846331in}{0.945813in}}%
\pgfpathlineto{\pgfqpoint{0.854801in}{0.938529in}}%
\pgfpathlineto{\pgfqpoint{0.856908in}{0.936763in}}%
\pgfpathlineto{\pgfqpoint{0.867485in}{0.928826in}}%
\pgfpathlineto{\pgfqpoint{0.868377in}{0.928235in}}%
\pgfpathlineto{\pgfqpoint{0.878062in}{0.921961in}}%
\pgfpathlineto{\pgfqpoint{0.888639in}{0.918098in}}%
\pgfpathlineto{\pgfqpoint{0.899216in}{0.918095in}}%
\pgfpathlineto{\pgfqpoint{0.899291in}{0.917941in}}%
\pgfpathlineto{\pgfqpoint{0.899216in}{0.917439in}}%
\pgfpathlineto{\pgfqpoint{0.897011in}{0.907646in}}%
\pgfpathlineto{\pgfqpoint{0.897846in}{0.897352in}}%
\pgfpathlineto{\pgfqpoint{0.898627in}{0.887057in}}%
\pgfpathlineto{\pgfqpoint{0.888639in}{0.882921in}}%
\pgfpathlineto{\pgfqpoint{0.878062in}{0.881260in}}%
\pgfpathlineto{\pgfqpoint{0.872275in}{0.887057in}}%
\pgfpathlineto{\pgfqpoint{0.867485in}{0.891249in}}%
\pgfpathlineto{\pgfqpoint{0.857945in}{0.897352in}}%
\pgfpathlineto{\pgfqpoint{0.856908in}{0.898015in}}%
\pgfpathlineto{\pgfqpoint{0.846331in}{0.904031in}}%
\pgfpathlineto{\pgfqpoint{0.840532in}{0.907646in}}%
\pgfpathlineto{\pgfqpoint{0.835754in}{0.910426in}}%
\pgfpathlineto{\pgfqpoint{0.825177in}{0.916690in}}%
\pgfpathlineto{\pgfqpoint{0.823457in}{0.917941in}}%
\pgfpathlineto{\pgfqpoint{0.814600in}{0.923277in}}%
\pgfpathlineto{\pgfqpoint{0.807673in}{0.928235in}}%
\pgfpathlineto{\pgfqpoint{0.804023in}{0.930664in}}%
\pgfpathlineto{\pgfqpoint{0.793445in}{0.938015in}}%
\pgfpathlineto{\pgfqpoint{0.792517in}{0.938529in}}%
\pgfpathlineto{\pgfqpoint{0.782868in}{0.944827in}}%
\pgfpathlineto{\pgfqpoint{0.774689in}{0.948824in}}%
\pgfpathlineto{\pgfqpoint{0.772291in}{0.950155in}}%
\pgfpathlineto{\pgfqpoint{0.761714in}{0.958028in}}%
\pgfpathlineto{\pgfqpoint{0.760188in}{0.959118in}}%
\pgfpathlineto{\pgfqpoint{0.751137in}{0.966427in}}%
\pgfpathlineto{\pgfqpoint{0.746569in}{0.969413in}}%
\pgfpathlineto{\pgfqpoint{0.740560in}{0.973928in}}%
\pgfpathlineto{\pgfqpoint{0.730540in}{0.979707in}}%
\pgfpathlineto{\pgfqpoint{0.729983in}{0.980035in}}%
\pgfpathlineto{\pgfqpoint{0.719406in}{0.983545in}}%
\pgfpathlineto{\pgfqpoint{0.708829in}{0.985129in}}%
\pgfpathlineto{\pgfqpoint{0.698252in}{0.986096in}}%
\pgfpathlineto{\pgfqpoint{0.687675in}{0.986427in}}%
\pgfpathlineto{\pgfqpoint{0.677098in}{0.986138in}}%
\pgfpathlineto{\pgfqpoint{0.674064in}{0.990002in}}%
\pgfpathlineto{\pgfqpoint{0.666521in}{0.997560in}}%
\pgfpathlineto{\pgfqpoint{0.665111in}{1.000296in}}%
\pgfpathlineto{\pgfqpoint{0.659798in}{1.010590in}}%
\pgfpathlineto{\pgfqpoint{0.655944in}{1.018247in}}%
\pgfpathlineto{\pgfqpoint{0.654694in}{1.020885in}}%
\pgfpathlineto{\pgfqpoint{0.649030in}{1.031179in}}%
\pgfpathlineto{\pgfqpoint{0.645367in}{1.037865in}}%
\pgfpathlineto{\pgfqpoint{0.644058in}{1.041474in}}%
\pgfpathlineto{\pgfqpoint{0.640606in}{1.051768in}}%
\pgfpathlineto{\pgfqpoint{0.636343in}{1.062063in}}%
\pgfpathlineto{\pgfqpoint{0.634790in}{1.065437in}}%
\pgfpathlineto{\pgfqpoint{0.632349in}{1.072357in}}%
\pgfpathlineto{\pgfqpoint{0.628653in}{1.082651in}}%
\pgfpathlineto{\pgfqpoint{0.624486in}{1.092946in}}%
\pgfpathlineto{\pgfqpoint{0.624212in}{1.093560in}}%
\pgfpathlineto{\pgfqpoint{0.620382in}{1.103240in}}%
\pgfpathlineto{\pgfqpoint{0.616221in}{1.113535in}}%
\pgfpathlineto{\pgfqpoint{0.613635in}{1.120423in}}%
\pgfpathlineto{\pgfqpoint{0.612418in}{1.123829in}}%
\pgfpathlineto{\pgfqpoint{0.608923in}{1.134124in}}%
\pgfpathlineto{\pgfqpoint{0.605417in}{1.144418in}}%
\pgfpathlineto{\pgfqpoint{0.603058in}{1.151040in}}%
\pgfpathlineto{\pgfqpoint{0.601872in}{1.154712in}}%
\pgfpathlineto{\pgfqpoint{0.598618in}{1.165007in}}%
\pgfpathlineto{\pgfqpoint{0.595243in}{1.175301in}}%
\pgfpathlineto{\pgfqpoint{0.592481in}{1.183544in}}%
\pgfpathlineto{\pgfqpoint{0.591819in}{1.185596in}}%
\pgfpathlineto{\pgfqpoint{0.588441in}{1.195890in}}%
\pgfpathlineto{\pgfqpoint{0.585036in}{1.206185in}}%
\pgfpathlineto{\pgfqpoint{0.581904in}{1.215576in}}%
\pgfpathlineto{\pgfqpoint{0.581602in}{1.216479in}}%
\pgfpathlineto{\pgfqpoint{0.578130in}{1.226773in}}%
\pgfpathlineto{\pgfqpoint{0.574763in}{1.237068in}}%
\pgfpathlineto{\pgfqpoint{0.571327in}{1.246166in}}%
\pgfpathlineto{\pgfqpoint{0.570892in}{1.247362in}}%
\pgfpathlineto{\pgfqpoint{0.567176in}{1.257657in}}%
\pgfpathlineto{\pgfqpoint{0.564065in}{1.267951in}}%
\pgfpathlineto{\pgfqpoint{0.560750in}{1.276341in}}%
\pgfpathlineto{\pgfqpoint{0.560105in}{1.278246in}}%
\pgfpathlineto{\pgfqpoint{0.557343in}{1.288540in}}%
\pgfpathlineto{\pgfqpoint{0.554025in}{1.298834in}}%
\pgfpathlineto{\pgfqpoint{0.550665in}{1.309129in}}%
\pgfpathlineto{\pgfqpoint{0.550173in}{1.310682in}}%
\pgfpathlineto{\pgfqpoint{0.547741in}{1.319423in}}%
\pgfpathlineto{\pgfqpoint{0.545171in}{1.329718in}}%
\pgfpathlineto{\pgfqpoint{0.542203in}{1.340012in}}%
\pgfpathlineto{\pgfqpoint{0.539788in}{1.350307in}}%
\pgfpathlineto{\pgfqpoint{0.539596in}{1.351389in}}%
\pgfpathlineto{\pgfqpoint{0.538183in}{1.360601in}}%
\pgfpathlineto{\pgfqpoint{0.536292in}{1.370896in}}%
\pgfpathlineto{\pgfqpoint{0.534410in}{1.381190in}}%
\pgfpathlineto{\pgfqpoint{0.532695in}{1.391484in}}%
\pgfpathlineto{\pgfqpoint{0.530942in}{1.401779in}}%
\pgfpathlineto{\pgfqpoint{0.529135in}{1.412073in}}%
\pgfpathlineto{\pgfqpoint{0.529019in}{1.412840in}}%
\pgfpathlineto{\pgfqpoint{0.527535in}{1.422368in}}%
\pgfpathlineto{\pgfqpoint{0.525894in}{1.432662in}}%
\pgfpathlineto{\pgfqpoint{0.524241in}{1.442957in}}%
\pgfpathlineto{\pgfqpoint{0.522550in}{1.453251in}}%
\pgfpathlineto{\pgfqpoint{0.520838in}{1.463545in}}%
\pgfpathlineto{\pgfqpoint{0.518980in}{1.473840in}}%
\pgfpathlineto{\pgfqpoint{0.518442in}{1.476202in}}%
\pgfpathlineto{\pgfqpoint{0.516413in}{1.484134in}}%
\pgfpathlineto{\pgfqpoint{0.513664in}{1.494429in}}%
\pgfpathlineto{\pgfqpoint{0.510695in}{1.504723in}}%
\pgfpathlineto{\pgfqpoint{0.507865in}{1.513788in}}%
\pgfpathlineto{\pgfqpoint{0.507469in}{1.515018in}}%
\pgfpathlineto{\pgfqpoint{0.497288in}{1.515018in}}%
\pgfpathlineto{\pgfqpoint{0.497288in}{1.504723in}}%
\pgfpathlineto{\pgfqpoint{0.497288in}{1.494429in}}%
\pgfpathlineto{\pgfqpoint{0.497288in}{1.484134in}}%
\pgfpathlineto{\pgfqpoint{0.497288in}{1.473840in}}%
\pgfpathlineto{\pgfqpoint{0.497288in}{1.463545in}}%
\pgfpathlineto{\pgfqpoint{0.497288in}{1.453251in}}%
\pgfpathlineto{\pgfqpoint{0.497288in}{1.442957in}}%
\pgfpathlineto{\pgfqpoint{0.497288in}{1.432662in}}%
\pgfpathlineto{\pgfqpoint{0.497288in}{1.422368in}}%
\pgfpathlineto{\pgfqpoint{0.497288in}{1.412073in}}%
\pgfpathlineto{\pgfqpoint{0.497288in}{1.401779in}}%
\pgfpathlineto{\pgfqpoint{0.497288in}{1.391484in}}%
\pgfpathlineto{\pgfqpoint{0.497288in}{1.382081in}}%
\pgfpathlineto{\pgfqpoint{0.497551in}{1.381190in}}%
\pgfpathlineto{\pgfqpoint{0.500565in}{1.370896in}}%
\pgfpathlineto{\pgfqpoint{0.503564in}{1.360601in}}%
\pgfpathlineto{\pgfqpoint{0.506748in}{1.350307in}}%
\pgfpathlineto{\pgfqpoint{0.507865in}{1.347042in}}%
\pgfpathlineto{\pgfqpoint{0.510219in}{1.340012in}}%
\pgfpathlineto{\pgfqpoint{0.513885in}{1.329718in}}%
\pgfpathlineto{\pgfqpoint{0.517595in}{1.319423in}}%
\pgfpathlineto{\pgfqpoint{0.518442in}{1.317103in}}%
\pgfpathlineto{\pgfqpoint{0.521316in}{1.309129in}}%
\pgfpathlineto{\pgfqpoint{0.524984in}{1.298834in}}%
\pgfpathlineto{\pgfqpoint{0.528580in}{1.288540in}}%
\pgfpathlineto{\pgfqpoint{0.529019in}{1.287250in}}%
\pgfpathlineto{\pgfqpoint{0.532078in}{1.278246in}}%
\pgfpathlineto{\pgfqpoint{0.535541in}{1.267951in}}%
\pgfpathlineto{\pgfqpoint{0.539179in}{1.257657in}}%
\pgfpathlineto{\pgfqpoint{0.539596in}{1.256518in}}%
\pgfpathlineto{\pgfqpoint{0.542950in}{1.247362in}}%
\pgfpathlineto{\pgfqpoint{0.546536in}{1.237068in}}%
\pgfpathlineto{\pgfqpoint{0.549905in}{1.226773in}}%
\pgfpathlineto{\pgfqpoint{0.550173in}{1.225937in}}%
\pgfpathlineto{\pgfqpoint{0.553214in}{1.216479in}}%
\pgfpathlineto{\pgfqpoint{0.556485in}{1.206185in}}%
\pgfpathlineto{\pgfqpoint{0.559719in}{1.195890in}}%
\pgfpathlineto{\pgfqpoint{0.560750in}{1.192653in}}%
\pgfpathlineto{\pgfqpoint{0.562986in}{1.185596in}}%
\pgfpathlineto{\pgfqpoint{0.566296in}{1.175301in}}%
\pgfpathlineto{\pgfqpoint{0.569573in}{1.165007in}}%
\pgfpathlineto{\pgfqpoint{0.571327in}{1.159433in}}%
\pgfpathlineto{\pgfqpoint{0.572817in}{1.154712in}}%
\pgfpathlineto{\pgfqpoint{0.576030in}{1.144418in}}%
\pgfpathlineto{\pgfqpoint{0.579294in}{1.134124in}}%
\pgfpathlineto{\pgfqpoint{0.581904in}{1.126885in}}%
\pgfpathlineto{\pgfqpoint{0.582978in}{1.123829in}}%
\pgfpathlineto{\pgfqpoint{0.586818in}{1.113535in}}%
\pgfpathlineto{\pgfqpoint{0.590556in}{1.103240in}}%
\pgfpathlineto{\pgfqpoint{0.592481in}{1.097778in}}%
\pgfpathlineto{\pgfqpoint{0.594185in}{1.092946in}}%
\pgfpathlineto{\pgfqpoint{0.597714in}{1.082651in}}%
\pgfpathlineto{\pgfqpoint{0.601181in}{1.072357in}}%
\pgfpathlineto{\pgfqpoint{0.603058in}{1.066770in}}%
\pgfpathlineto{\pgfqpoint{0.604636in}{1.062063in}}%
\pgfpathlineto{\pgfqpoint{0.608016in}{1.051768in}}%
\pgfpathlineto{\pgfqpoint{0.611335in}{1.041474in}}%
\pgfpathlineto{\pgfqpoint{0.613635in}{1.034234in}}%
\pgfpathlineto{\pgfqpoint{0.614694in}{1.031179in}}%
\pgfpathlineto{\pgfqpoint{0.618923in}{1.020885in}}%
\pgfpathlineto{\pgfqpoint{0.623041in}{1.010590in}}%
\pgfpathlineto{\pgfqpoint{0.624212in}{1.007497in}}%
\pgfpathlineto{\pgfqpoint{0.627124in}{1.000296in}}%
\pgfpathlineto{\pgfqpoint{0.631261in}{0.990002in}}%
\pgfpathlineto{\pgfqpoint{0.634790in}{0.981347in}}%
\pgfpathlineto{\pgfqpoint{0.635453in}{0.979707in}}%
\pgfpathlineto{\pgfqpoint{0.639670in}{0.969413in}}%
\pgfpathlineto{\pgfqpoint{0.643880in}{0.959118in}}%
\pgfpathlineto{\pgfqpoint{0.645367in}{0.955467in}}%
\pgfpathlineto{\pgfqpoint{0.648049in}{0.948824in}}%
\pgfpathlineto{\pgfqpoint{0.652160in}{0.938529in}}%
\pgfpathlineto{\pgfqpoint{0.655944in}{0.929021in}}%
\pgfpathlineto{\pgfqpoint{0.656326in}{0.928235in}}%
\pgfpathlineto{\pgfqpoint{0.661278in}{0.917941in}}%
\pgfpathlineto{\pgfqpoint{0.665864in}{0.907646in}}%
\pgfpathlineto{\pgfqpoint{0.666521in}{0.906085in}}%
\pgfpathlineto{\pgfqpoint{0.670785in}{0.897352in}}%
\pgfpathlineto{\pgfqpoint{0.675427in}{0.887057in}}%
\pgfpathlineto{\pgfqpoint{0.677098in}{0.883168in}}%
\pgfpathlineto{\pgfqpoint{0.680105in}{0.876763in}}%
\pgfpathlineto{\pgfqpoint{0.684463in}{0.866468in}}%
\pgfpathlineto{\pgfqpoint{0.687675in}{0.858164in}}%
\pgfpathlineto{\pgfqpoint{0.689165in}{0.856174in}}%
\pgfpathlineto{\pgfqpoint{0.695361in}{0.845879in}}%
\pgfpathlineto{\pgfqpoint{0.698252in}{0.840529in}}%
\pgfpathlineto{\pgfqpoint{0.706400in}{0.835585in}}%
\pgfpathlineto{\pgfqpoint{0.708829in}{0.834101in}}%
\pgfpathlineto{\pgfqpoint{0.719406in}{0.827438in}}%
\pgfpathlineto{\pgfqpoint{0.722718in}{0.825291in}}%
\pgfpathlineto{\pgfqpoint{0.729983in}{0.820542in}}%
\pgfpathlineto{\pgfqpoint{0.738247in}{0.814996in}}%
\pgfpathlineto{\pgfqpoint{0.740560in}{0.813429in}}%
\pgfpathlineto{\pgfqpoint{0.751137in}{0.806083in}}%
\pgfpathlineto{\pgfqpoint{0.753091in}{0.804702in}}%
\pgfpathlineto{\pgfqpoint{0.761714in}{0.798521in}}%
\pgfpathlineto{\pgfqpoint{0.768654in}{0.794407in}}%
\pgfpathlineto{\pgfqpoint{0.772291in}{0.792091in}}%
\pgfpathlineto{\pgfqpoint{0.782868in}{0.785782in}}%
\pgfpathlineto{\pgfqpoint{0.785554in}{0.784113in}}%
\pgfpathlineto{\pgfqpoint{0.793445in}{0.779170in}}%
\pgfpathlineto{\pgfqpoint{0.801683in}{0.773818in}}%
\pgfpathlineto{\pgfqpoint{0.804023in}{0.772287in}}%
\pgfpathlineto{\pgfqpoint{0.814600in}{0.764278in}}%
\pgfpathlineto{\pgfqpoint{0.815232in}{0.763524in}}%
\pgfpathlineto{\pgfqpoint{0.821187in}{0.753230in}}%
\pgfpathlineto{\pgfqpoint{0.823859in}{0.742935in}}%
\pgfpathlineto{\pgfqpoint{0.825177in}{0.735492in}}%
\pgfpathlineto{\pgfqpoint{0.825668in}{0.732641in}}%
\pgfpathlineto{\pgfqpoint{0.831776in}{0.722346in}}%
\pgfpathlineto{\pgfqpoint{0.835754in}{0.715466in}}%
\pgfpathlineto{\pgfqpoint{0.837706in}{0.712052in}}%
\pgfpathlineto{\pgfqpoint{0.843423in}{0.701757in}}%
\pgfpathlineto{\pgfqpoint{0.846331in}{0.696154in}}%
\pgfpathlineto{\pgfqpoint{0.848705in}{0.691463in}}%
\pgfpathlineto{\pgfqpoint{0.854340in}{0.681169in}}%
\pgfpathlineto{\pgfqpoint{0.856908in}{0.676770in}}%
\pgfpathlineto{\pgfqpoint{0.860333in}{0.670874in}}%
\pgfpathlineto{\pgfqpoint{0.866015in}{0.660580in}}%
\pgfpathlineto{\pgfqpoint{0.867485in}{0.657220in}}%
\pgfpathlineto{\pgfqpoint{0.870438in}{0.650285in}}%
\pgfpathlineto{\pgfqpoint{0.874785in}{0.639991in}}%
\pgfpathlineto{\pgfqpoint{0.878062in}{0.634369in}}%
\pgfpathlineto{\pgfqpoint{0.880657in}{0.629696in}}%
\pgfpathlineto{\pgfqpoint{0.886232in}{0.619402in}}%
\pgfpathlineto{\pgfqpoint{0.888639in}{0.614759in}}%
\pgfpathlineto{\pgfqpoint{0.891518in}{0.609108in}}%
\pgfpathlineto{\pgfqpoint{0.896472in}{0.598813in}}%
\pgfpathlineto{\pgfqpoint{0.899216in}{0.592795in}}%
\pgfpathlineto{\pgfqpoint{0.901108in}{0.588519in}}%
\pgfpathlineto{\pgfqpoint{0.905392in}{0.578224in}}%
\pgfpathlineto{\pgfqpoint{0.909319in}{0.567930in}}%
\pgfpathlineto{\pgfqpoint{0.909793in}{0.566591in}}%
\pgfpathlineto{\pgfqpoint{0.913233in}{0.557635in}}%
\pgfpathlineto{\pgfqpoint{0.917671in}{0.547341in}}%
\pgfpathlineto{\pgfqpoint{0.920370in}{0.541510in}}%
\pgfpathlineto{\pgfqpoint{0.922602in}{0.537047in}}%
\pgfpathlineto{\pgfqpoint{0.927648in}{0.526752in}}%
\pgfpathlineto{\pgfqpoint{0.930947in}{0.519877in}}%
\pgfpathlineto{\pgfqpoint{0.932570in}{0.516458in}}%
\pgfpathlineto{\pgfqpoint{0.937358in}{0.506163in}}%
\pgfpathclose%
\pgfusepath{fill}%
\end{pgfscope}%
\begin{pgfscope}%
\pgfpathrectangle{\pgfqpoint{0.423750in}{0.423750in}}{\pgfqpoint{1.194205in}{1.163386in}}%
\pgfusepath{clip}%
\pgfsetbuttcap%
\pgfsetroundjoin%
\definecolor{currentfill}{rgb}{0.966120,0.744512,0.608720}%
\pgfsetfillcolor{currentfill}%
\pgfsetlinewidth{0.000000pt}%
\definecolor{currentstroke}{rgb}{0.000000,0.000000,0.000000}%
\pgfsetstrokecolor{currentstroke}%
\pgfsetdash{}{0pt}%
\pgfpathmoveto{\pgfqpoint{1.237682in}{0.495869in}}%
\pgfpathlineto{\pgfqpoint{1.248259in}{0.495869in}}%
\pgfpathlineto{\pgfqpoint{1.258836in}{0.495869in}}%
\pgfpathlineto{\pgfqpoint{1.269413in}{0.495869in}}%
\pgfpathlineto{\pgfqpoint{1.279990in}{0.495869in}}%
\pgfpathlineto{\pgfqpoint{1.290567in}{0.495869in}}%
\pgfpathlineto{\pgfqpoint{1.301144in}{0.495869in}}%
\pgfpathlineto{\pgfqpoint{1.311721in}{0.495869in}}%
\pgfpathlineto{\pgfqpoint{1.322299in}{0.495869in}}%
\pgfpathlineto{\pgfqpoint{1.332876in}{0.495869in}}%
\pgfpathlineto{\pgfqpoint{1.343453in}{0.495869in}}%
\pgfpathlineto{\pgfqpoint{1.354030in}{0.495869in}}%
\pgfpathlineto{\pgfqpoint{1.364607in}{0.495869in}}%
\pgfpathlineto{\pgfqpoint{1.375184in}{0.495869in}}%
\pgfpathlineto{\pgfqpoint{1.385761in}{0.495869in}}%
\pgfpathlineto{\pgfqpoint{1.396338in}{0.495869in}}%
\pgfpathlineto{\pgfqpoint{1.406915in}{0.495869in}}%
\pgfpathlineto{\pgfqpoint{1.417492in}{0.495869in}}%
\pgfpathlineto{\pgfqpoint{1.428069in}{0.495869in}}%
\pgfpathlineto{\pgfqpoint{1.438646in}{0.495869in}}%
\pgfpathlineto{\pgfqpoint{1.444254in}{0.495869in}}%
\pgfpathlineto{\pgfqpoint{1.438646in}{0.504416in}}%
\pgfpathlineto{\pgfqpoint{1.437505in}{0.506163in}}%
\pgfpathlineto{\pgfqpoint{1.430663in}{0.516458in}}%
\pgfpathlineto{\pgfqpoint{1.428069in}{0.520294in}}%
\pgfpathlineto{\pgfqpoint{1.423715in}{0.526752in}}%
\pgfpathlineto{\pgfqpoint{1.417492in}{0.535823in}}%
\pgfpathlineto{\pgfqpoint{1.416654in}{0.537047in}}%
\pgfpathlineto{\pgfqpoint{1.409479in}{0.547341in}}%
\pgfpathlineto{\pgfqpoint{1.406915in}{0.550957in}}%
\pgfpathlineto{\pgfqpoint{1.402181in}{0.557635in}}%
\pgfpathlineto{\pgfqpoint{1.396338in}{0.565742in}}%
\pgfpathlineto{\pgfqpoint{1.394853in}{0.567930in}}%
\pgfpathlineto{\pgfqpoint{1.387209in}{0.578224in}}%
\pgfpathlineto{\pgfqpoint{1.385761in}{0.580167in}}%
\pgfpathlineto{\pgfqpoint{1.380200in}{0.588519in}}%
\pgfpathlineto{\pgfqpoint{1.375184in}{0.595246in}}%
\pgfpathlineto{\pgfqpoint{1.372716in}{0.598813in}}%
\pgfpathlineto{\pgfqpoint{1.365523in}{0.609108in}}%
\pgfpathlineto{\pgfqpoint{1.364607in}{0.610363in}}%
\pgfpathlineto{\pgfqpoint{1.358060in}{0.619402in}}%
\pgfpathlineto{\pgfqpoint{1.354030in}{0.624881in}}%
\pgfpathlineto{\pgfqpoint{1.350365in}{0.629696in}}%
\pgfpathlineto{\pgfqpoint{1.343453in}{0.638675in}}%
\pgfpathlineto{\pgfqpoint{1.341777in}{0.639991in}}%
\pgfpathlineto{\pgfqpoint{1.335698in}{0.650285in}}%
\pgfpathlineto{\pgfqpoint{1.332876in}{0.657133in}}%
\pgfpathlineto{\pgfqpoint{1.327452in}{0.650285in}}%
\pgfpathlineto{\pgfqpoint{1.322299in}{0.644206in}}%
\pgfpathlineto{\pgfqpoint{1.318928in}{0.639991in}}%
\pgfpathlineto{\pgfqpoint{1.311721in}{0.634684in}}%
\pgfpathlineto{\pgfqpoint{1.306489in}{0.629696in}}%
\pgfpathlineto{\pgfqpoint{1.301144in}{0.625722in}}%
\pgfpathlineto{\pgfqpoint{1.293166in}{0.619402in}}%
\pgfpathlineto{\pgfqpoint{1.290567in}{0.617289in}}%
\pgfpathlineto{\pgfqpoint{1.281150in}{0.609108in}}%
\pgfpathlineto{\pgfqpoint{1.279990in}{0.608069in}}%
\pgfpathlineto{\pgfqpoint{1.270329in}{0.598813in}}%
\pgfpathlineto{\pgfqpoint{1.269413in}{0.597906in}}%
\pgfpathlineto{\pgfqpoint{1.262026in}{0.588519in}}%
\pgfpathlineto{\pgfqpoint{1.258836in}{0.583616in}}%
\pgfpathlineto{\pgfqpoint{1.256240in}{0.578224in}}%
\pgfpathlineto{\pgfqpoint{1.253236in}{0.567930in}}%
\pgfpathlineto{\pgfqpoint{1.250525in}{0.557635in}}%
\pgfpathlineto{\pgfqpoint{1.248259in}{0.548571in}}%
\pgfpathlineto{\pgfqpoint{1.247957in}{0.547341in}}%
\pgfpathlineto{\pgfqpoint{1.245562in}{0.537047in}}%
\pgfpathlineto{\pgfqpoint{1.243309in}{0.526752in}}%
\pgfpathlineto{\pgfqpoint{1.241196in}{0.516458in}}%
\pgfpathlineto{\pgfqpoint{1.239219in}{0.506163in}}%
\pgfpathlineto{\pgfqpoint{1.237682in}{0.497592in}}%
\pgfpathlineto{\pgfqpoint{1.237378in}{0.495869in}}%
\pgfpathclose%
\pgfusepath{fill}%
\end{pgfscope}%
\begin{pgfscope}%
\pgfpathrectangle{\pgfqpoint{0.423750in}{0.423750in}}{\pgfqpoint{1.194205in}{1.163386in}}%
\pgfusepath{clip}%
\pgfsetbuttcap%
\pgfsetroundjoin%
\definecolor{currentfill}{rgb}{0.970255,0.815666,0.711203}%
\pgfsetfillcolor{currentfill}%
\pgfsetlinewidth{0.000000pt}%
\definecolor{currentstroke}{rgb}{0.000000,0.000000,0.000000}%
\pgfsetstrokecolor{currentstroke}%
\pgfsetdash{}{0pt}%
\pgfpathmoveto{\pgfqpoint{1.015564in}{0.495869in}}%
\pgfpathlineto{\pgfqpoint{1.026141in}{0.495869in}}%
\pgfpathlineto{\pgfqpoint{1.036718in}{0.495869in}}%
\pgfpathlineto{\pgfqpoint{1.047295in}{0.495869in}}%
\pgfpathlineto{\pgfqpoint{1.057872in}{0.495869in}}%
\pgfpathlineto{\pgfqpoint{1.068449in}{0.495869in}}%
\pgfpathlineto{\pgfqpoint{1.079026in}{0.495869in}}%
\pgfpathlineto{\pgfqpoint{1.089603in}{0.495869in}}%
\pgfpathlineto{\pgfqpoint{1.100180in}{0.495869in}}%
\pgfpathlineto{\pgfqpoint{1.110757in}{0.495869in}}%
\pgfpathlineto{\pgfqpoint{1.121334in}{0.495869in}}%
\pgfpathlineto{\pgfqpoint{1.131911in}{0.495869in}}%
\pgfpathlineto{\pgfqpoint{1.142488in}{0.495869in}}%
\pgfpathlineto{\pgfqpoint{1.153066in}{0.495869in}}%
\pgfpathlineto{\pgfqpoint{1.163643in}{0.495869in}}%
\pgfpathlineto{\pgfqpoint{1.174220in}{0.495869in}}%
\pgfpathlineto{\pgfqpoint{1.184797in}{0.495869in}}%
\pgfpathlineto{\pgfqpoint{1.195374in}{0.495869in}}%
\pgfpathlineto{\pgfqpoint{1.205951in}{0.495869in}}%
\pgfpathlineto{\pgfqpoint{1.216528in}{0.495869in}}%
\pgfpathlineto{\pgfqpoint{1.227105in}{0.495869in}}%
\pgfpathlineto{\pgfqpoint{1.237378in}{0.495869in}}%
\pgfpathlineto{\pgfqpoint{1.237682in}{0.497592in}}%
\pgfpathlineto{\pgfqpoint{1.239219in}{0.506163in}}%
\pgfpathlineto{\pgfqpoint{1.241196in}{0.516458in}}%
\pgfpathlineto{\pgfqpoint{1.243309in}{0.526752in}}%
\pgfpathlineto{\pgfqpoint{1.245562in}{0.537047in}}%
\pgfpathlineto{\pgfqpoint{1.247957in}{0.547341in}}%
\pgfpathlineto{\pgfqpoint{1.248259in}{0.548571in}}%
\pgfpathlineto{\pgfqpoint{1.250525in}{0.557635in}}%
\pgfpathlineto{\pgfqpoint{1.253236in}{0.567930in}}%
\pgfpathlineto{\pgfqpoint{1.256240in}{0.578224in}}%
\pgfpathlineto{\pgfqpoint{1.258836in}{0.583616in}}%
\pgfpathlineto{\pgfqpoint{1.262026in}{0.588519in}}%
\pgfpathlineto{\pgfqpoint{1.269413in}{0.597906in}}%
\pgfpathlineto{\pgfqpoint{1.270329in}{0.598813in}}%
\pgfpathlineto{\pgfqpoint{1.279990in}{0.608069in}}%
\pgfpathlineto{\pgfqpoint{1.281150in}{0.609108in}}%
\pgfpathlineto{\pgfqpoint{1.290567in}{0.617289in}}%
\pgfpathlineto{\pgfqpoint{1.293166in}{0.619402in}}%
\pgfpathlineto{\pgfqpoint{1.301144in}{0.625722in}}%
\pgfpathlineto{\pgfqpoint{1.306489in}{0.629696in}}%
\pgfpathlineto{\pgfqpoint{1.311721in}{0.634684in}}%
\pgfpathlineto{\pgfqpoint{1.318928in}{0.639991in}}%
\pgfpathlineto{\pgfqpoint{1.322299in}{0.644206in}}%
\pgfpathlineto{\pgfqpoint{1.327452in}{0.650285in}}%
\pgfpathlineto{\pgfqpoint{1.332876in}{0.657133in}}%
\pgfpathlineto{\pgfqpoint{1.335698in}{0.650285in}}%
\pgfpathlineto{\pgfqpoint{1.341777in}{0.639991in}}%
\pgfpathlineto{\pgfqpoint{1.343453in}{0.638675in}}%
\pgfpathlineto{\pgfqpoint{1.350365in}{0.629696in}}%
\pgfpathlineto{\pgfqpoint{1.354030in}{0.624881in}}%
\pgfpathlineto{\pgfqpoint{1.358060in}{0.619402in}}%
\pgfpathlineto{\pgfqpoint{1.364607in}{0.610363in}}%
\pgfpathlineto{\pgfqpoint{1.365523in}{0.609108in}}%
\pgfpathlineto{\pgfqpoint{1.372716in}{0.598813in}}%
\pgfpathlineto{\pgfqpoint{1.375184in}{0.595246in}}%
\pgfpathlineto{\pgfqpoint{1.380200in}{0.588519in}}%
\pgfpathlineto{\pgfqpoint{1.385761in}{0.580167in}}%
\pgfpathlineto{\pgfqpoint{1.387209in}{0.578224in}}%
\pgfpathlineto{\pgfqpoint{1.394853in}{0.567930in}}%
\pgfpathlineto{\pgfqpoint{1.396338in}{0.565742in}}%
\pgfpathlineto{\pgfqpoint{1.402181in}{0.557635in}}%
\pgfpathlineto{\pgfqpoint{1.406915in}{0.550957in}}%
\pgfpathlineto{\pgfqpoint{1.409479in}{0.547341in}}%
\pgfpathlineto{\pgfqpoint{1.416654in}{0.537047in}}%
\pgfpathlineto{\pgfqpoint{1.417492in}{0.535823in}}%
\pgfpathlineto{\pgfqpoint{1.423715in}{0.526752in}}%
\pgfpathlineto{\pgfqpoint{1.428069in}{0.520294in}}%
\pgfpathlineto{\pgfqpoint{1.430663in}{0.516458in}}%
\pgfpathlineto{\pgfqpoint{1.437505in}{0.506163in}}%
\pgfpathlineto{\pgfqpoint{1.438646in}{0.504416in}}%
\pgfpathlineto{\pgfqpoint{1.444254in}{0.495869in}}%
\pgfpathlineto{\pgfqpoint{1.449223in}{0.495869in}}%
\pgfpathlineto{\pgfqpoint{1.459800in}{0.495869in}}%
\pgfpathlineto{\pgfqpoint{1.470377in}{0.495869in}}%
\pgfpathlineto{\pgfqpoint{1.480954in}{0.495869in}}%
\pgfpathlineto{\pgfqpoint{1.491532in}{0.495869in}}%
\pgfpathlineto{\pgfqpoint{1.502109in}{0.495869in}}%
\pgfpathlineto{\pgfqpoint{1.512686in}{0.495869in}}%
\pgfpathlineto{\pgfqpoint{1.523263in}{0.495869in}}%
\pgfpathlineto{\pgfqpoint{1.533840in}{0.495869in}}%
\pgfpathlineto{\pgfqpoint{1.544417in}{0.495869in}}%
\pgfpathlineto{\pgfqpoint{1.544417in}{0.506163in}}%
\pgfpathlineto{\pgfqpoint{1.544417in}{0.516458in}}%
\pgfpathlineto{\pgfqpoint{1.544417in}{0.526752in}}%
\pgfpathlineto{\pgfqpoint{1.544417in}{0.537047in}}%
\pgfpathlineto{\pgfqpoint{1.544417in}{0.547341in}}%
\pgfpathlineto{\pgfqpoint{1.544417in}{0.557635in}}%
\pgfpathlineto{\pgfqpoint{1.544417in}{0.567930in}}%
\pgfpathlineto{\pgfqpoint{1.544417in}{0.578224in}}%
\pgfpathlineto{\pgfqpoint{1.544417in}{0.588519in}}%
\pgfpathlineto{\pgfqpoint{1.544417in}{0.598813in}}%
\pgfpathlineto{\pgfqpoint{1.544417in}{0.609108in}}%
\pgfpathlineto{\pgfqpoint{1.544417in}{0.619402in}}%
\pgfpathlineto{\pgfqpoint{1.544417in}{0.629696in}}%
\pgfpathlineto{\pgfqpoint{1.544417in}{0.639991in}}%
\pgfpathlineto{\pgfqpoint{1.544417in}{0.650285in}}%
\pgfpathlineto{\pgfqpoint{1.544417in}{0.660580in}}%
\pgfpathlineto{\pgfqpoint{1.544417in}{0.670874in}}%
\pgfpathlineto{\pgfqpoint{1.544417in}{0.681169in}}%
\pgfpathlineto{\pgfqpoint{1.544417in}{0.691463in}}%
\pgfpathlineto{\pgfqpoint{1.544417in}{0.701757in}}%
\pgfpathlineto{\pgfqpoint{1.544417in}{0.712052in}}%
\pgfpathlineto{\pgfqpoint{1.544417in}{0.722346in}}%
\pgfpathlineto{\pgfqpoint{1.544417in}{0.732641in}}%
\pgfpathlineto{\pgfqpoint{1.544417in}{0.742935in}}%
\pgfpathlineto{\pgfqpoint{1.544417in}{0.753230in}}%
\pgfpathlineto{\pgfqpoint{1.544417in}{0.763524in}}%
\pgfpathlineto{\pgfqpoint{1.544417in}{0.773818in}}%
\pgfpathlineto{\pgfqpoint{1.544417in}{0.784113in}}%
\pgfpathlineto{\pgfqpoint{1.544417in}{0.794407in}}%
\pgfpathlineto{\pgfqpoint{1.544417in}{0.804702in}}%
\pgfpathlineto{\pgfqpoint{1.544417in}{0.814996in}}%
\pgfpathlineto{\pgfqpoint{1.544417in}{0.825291in}}%
\pgfpathlineto{\pgfqpoint{1.544417in}{0.835585in}}%
\pgfpathlineto{\pgfqpoint{1.544417in}{0.845879in}}%
\pgfpathlineto{\pgfqpoint{1.544417in}{0.856174in}}%
\pgfpathlineto{\pgfqpoint{1.544417in}{0.866468in}}%
\pgfpathlineto{\pgfqpoint{1.544417in}{0.876763in}}%
\pgfpathlineto{\pgfqpoint{1.544417in}{0.887057in}}%
\pgfpathlineto{\pgfqpoint{1.544417in}{0.897352in}}%
\pgfpathlineto{\pgfqpoint{1.544417in}{0.907646in}}%
\pgfpathlineto{\pgfqpoint{1.544417in}{0.917941in}}%
\pgfpathlineto{\pgfqpoint{1.544417in}{0.928235in}}%
\pgfpathlineto{\pgfqpoint{1.544417in}{0.938529in}}%
\pgfpathlineto{\pgfqpoint{1.544417in}{0.948824in}}%
\pgfpathlineto{\pgfqpoint{1.544417in}{0.952051in}}%
\pgfpathlineto{\pgfqpoint{1.540135in}{0.948824in}}%
\pgfpathlineto{\pgfqpoint{1.533840in}{0.943509in}}%
\pgfpathlineto{\pgfqpoint{1.527932in}{0.938529in}}%
\pgfpathlineto{\pgfqpoint{1.523263in}{0.934615in}}%
\pgfpathlineto{\pgfqpoint{1.515902in}{0.928235in}}%
\pgfpathlineto{\pgfqpoint{1.512686in}{0.925051in}}%
\pgfpathlineto{\pgfqpoint{1.505433in}{0.917941in}}%
\pgfpathlineto{\pgfqpoint{1.502109in}{0.914698in}}%
\pgfpathlineto{\pgfqpoint{1.494819in}{0.907646in}}%
\pgfpathlineto{\pgfqpoint{1.491532in}{0.904488in}}%
\pgfpathlineto{\pgfqpoint{1.484059in}{0.897352in}}%
\pgfpathlineto{\pgfqpoint{1.480954in}{0.894412in}}%
\pgfpathlineto{\pgfqpoint{1.474303in}{0.887057in}}%
\pgfpathlineto{\pgfqpoint{1.470377in}{0.883062in}}%
\pgfpathlineto{\pgfqpoint{1.465719in}{0.876763in}}%
\pgfpathlineto{\pgfqpoint{1.459800in}{0.866702in}}%
\pgfpathlineto{\pgfqpoint{1.459687in}{0.866468in}}%
\pgfpathlineto{\pgfqpoint{1.455738in}{0.856174in}}%
\pgfpathlineto{\pgfqpoint{1.451786in}{0.845879in}}%
\pgfpathlineto{\pgfqpoint{1.449223in}{0.839470in}}%
\pgfpathlineto{\pgfqpoint{1.447695in}{0.835585in}}%
\pgfpathlineto{\pgfqpoint{1.443725in}{0.825291in}}%
\pgfpathlineto{\pgfqpoint{1.439384in}{0.814996in}}%
\pgfpathlineto{\pgfqpoint{1.438646in}{0.813242in}}%
\pgfpathlineto{\pgfqpoint{1.435069in}{0.804702in}}%
\pgfpathlineto{\pgfqpoint{1.428317in}{0.794407in}}%
\pgfpathlineto{\pgfqpoint{1.428069in}{0.794123in}}%
\pgfpathlineto{\pgfqpoint{1.418246in}{0.784113in}}%
\pgfpathlineto{\pgfqpoint{1.417492in}{0.783330in}}%
\pgfpathlineto{\pgfqpoint{1.408634in}{0.773818in}}%
\pgfpathlineto{\pgfqpoint{1.406915in}{0.771987in}}%
\pgfpathlineto{\pgfqpoint{1.398929in}{0.763524in}}%
\pgfpathlineto{\pgfqpoint{1.396338in}{0.760814in}}%
\pgfpathlineto{\pgfqpoint{1.389390in}{0.753230in}}%
\pgfpathlineto{\pgfqpoint{1.385761in}{0.749100in}}%
\pgfpathlineto{\pgfqpoint{1.378673in}{0.742935in}}%
\pgfpathlineto{\pgfqpoint{1.375184in}{0.738848in}}%
\pgfpathlineto{\pgfqpoint{1.370356in}{0.732641in}}%
\pgfpathlineto{\pgfqpoint{1.364607in}{0.726089in}}%
\pgfpathlineto{\pgfqpoint{1.361281in}{0.722346in}}%
\pgfpathlineto{\pgfqpoint{1.354030in}{0.714286in}}%
\pgfpathlineto{\pgfqpoint{1.352018in}{0.712052in}}%
\pgfpathlineto{\pgfqpoint{1.343453in}{0.702808in}}%
\pgfpathlineto{\pgfqpoint{1.342530in}{0.701757in}}%
\pgfpathlineto{\pgfqpoint{1.335723in}{0.691463in}}%
\pgfpathlineto{\pgfqpoint{1.332876in}{0.684651in}}%
\pgfpathlineto{\pgfqpoint{1.326469in}{0.691463in}}%
\pgfpathlineto{\pgfqpoint{1.322299in}{0.694639in}}%
\pgfpathlineto{\pgfqpoint{1.313078in}{0.701757in}}%
\pgfpathlineto{\pgfqpoint{1.311721in}{0.702535in}}%
\pgfpathlineto{\pgfqpoint{1.301144in}{0.708863in}}%
\pgfpathlineto{\pgfqpoint{1.296049in}{0.712052in}}%
\pgfpathlineto{\pgfqpoint{1.290567in}{0.715450in}}%
\pgfpathlineto{\pgfqpoint{1.279990in}{0.722151in}}%
\pgfpathlineto{\pgfqpoint{1.279690in}{0.722346in}}%
\pgfpathlineto{\pgfqpoint{1.269413in}{0.729379in}}%
\pgfpathlineto{\pgfqpoint{1.264747in}{0.732641in}}%
\pgfpathlineto{\pgfqpoint{1.258836in}{0.737014in}}%
\pgfpathlineto{\pgfqpoint{1.250991in}{0.742935in}}%
\pgfpathlineto{\pgfqpoint{1.248259in}{0.744291in}}%
\pgfpathlineto{\pgfqpoint{1.237682in}{0.749542in}}%
\pgfpathlineto{\pgfqpoint{1.230348in}{0.753230in}}%
\pgfpathlineto{\pgfqpoint{1.227105in}{0.754707in}}%
\pgfpathlineto{\pgfqpoint{1.216528in}{0.759679in}}%
\pgfpathlineto{\pgfqpoint{1.208627in}{0.763524in}}%
\pgfpathlineto{\pgfqpoint{1.205951in}{0.764825in}}%
\pgfpathlineto{\pgfqpoint{1.195374in}{0.770138in}}%
\pgfpathlineto{\pgfqpoint{1.189774in}{0.773818in}}%
\pgfpathlineto{\pgfqpoint{1.184797in}{0.777158in}}%
\pgfpathlineto{\pgfqpoint{1.176444in}{0.784113in}}%
\pgfpathlineto{\pgfqpoint{1.174220in}{0.786004in}}%
\pgfpathlineto{\pgfqpoint{1.164688in}{0.794407in}}%
\pgfpathlineto{\pgfqpoint{1.163643in}{0.795458in}}%
\pgfpathlineto{\pgfqpoint{1.155653in}{0.804702in}}%
\pgfpathlineto{\pgfqpoint{1.153066in}{0.808244in}}%
\pgfpathlineto{\pgfqpoint{1.148295in}{0.814996in}}%
\pgfpathlineto{\pgfqpoint{1.142488in}{0.824417in}}%
\pgfpathlineto{\pgfqpoint{1.141975in}{0.825291in}}%
\pgfpathlineto{\pgfqpoint{1.136201in}{0.835585in}}%
\pgfpathlineto{\pgfqpoint{1.131911in}{0.843646in}}%
\pgfpathlineto{\pgfqpoint{1.130733in}{0.845879in}}%
\pgfpathlineto{\pgfqpoint{1.124355in}{0.856174in}}%
\pgfpathlineto{\pgfqpoint{1.121334in}{0.859661in}}%
\pgfpathlineto{\pgfqpoint{1.114390in}{0.866468in}}%
\pgfpathlineto{\pgfqpoint{1.110757in}{0.869709in}}%
\pgfpathlineto{\pgfqpoint{1.102847in}{0.876763in}}%
\pgfpathlineto{\pgfqpoint{1.100180in}{0.879228in}}%
\pgfpathlineto{\pgfqpoint{1.092274in}{0.887057in}}%
\pgfpathlineto{\pgfqpoint{1.089603in}{0.889759in}}%
\pgfpathlineto{\pgfqpoint{1.082299in}{0.897352in}}%
\pgfpathlineto{\pgfqpoint{1.079026in}{0.900465in}}%
\pgfpathlineto{\pgfqpoint{1.068449in}{0.906555in}}%
\pgfpathlineto{\pgfqpoint{1.065577in}{0.907646in}}%
\pgfpathlineto{\pgfqpoint{1.057872in}{0.910501in}}%
\pgfpathlineto{\pgfqpoint{1.047295in}{0.912429in}}%
\pgfpathlineto{\pgfqpoint{1.036718in}{0.912658in}}%
\pgfpathlineto{\pgfqpoint{1.026141in}{0.912908in}}%
\pgfpathlineto{\pgfqpoint{1.015564in}{0.912573in}}%
\pgfpathlineto{\pgfqpoint{1.004987in}{0.912873in}}%
\pgfpathlineto{\pgfqpoint{0.994410in}{0.913824in}}%
\pgfpathlineto{\pgfqpoint{0.983833in}{0.915104in}}%
\pgfpathlineto{\pgfqpoint{0.973255in}{0.916558in}}%
\pgfpathlineto{\pgfqpoint{0.962678in}{0.916978in}}%
\pgfpathlineto{\pgfqpoint{0.952101in}{0.916553in}}%
\pgfpathlineto{\pgfqpoint{0.941524in}{0.916578in}}%
\pgfpathlineto{\pgfqpoint{0.930947in}{0.913850in}}%
\pgfpathlineto{\pgfqpoint{0.920370in}{0.909416in}}%
\pgfpathlineto{\pgfqpoint{0.917946in}{0.907646in}}%
\pgfpathlineto{\pgfqpoint{0.911339in}{0.897352in}}%
\pgfpathlineto{\pgfqpoint{0.909793in}{0.895237in}}%
\pgfpathlineto{\pgfqpoint{0.899568in}{0.887057in}}%
\pgfpathlineto{\pgfqpoint{0.900210in}{0.876763in}}%
\pgfpathlineto{\pgfqpoint{0.900499in}{0.866468in}}%
\pgfpathlineto{\pgfqpoint{0.899859in}{0.856174in}}%
\pgfpathlineto{\pgfqpoint{0.899216in}{0.849789in}}%
\pgfpathlineto{\pgfqpoint{0.898894in}{0.845879in}}%
\pgfpathlineto{\pgfqpoint{0.899216in}{0.841600in}}%
\pgfpathlineto{\pgfqpoint{0.899677in}{0.835585in}}%
\pgfpathlineto{\pgfqpoint{0.900519in}{0.825291in}}%
\pgfpathlineto{\pgfqpoint{0.902508in}{0.814996in}}%
\pgfpathlineto{\pgfqpoint{0.904554in}{0.804702in}}%
\pgfpathlineto{\pgfqpoint{0.906437in}{0.794407in}}%
\pgfpathlineto{\pgfqpoint{0.908169in}{0.784113in}}%
\pgfpathlineto{\pgfqpoint{0.909793in}{0.773953in}}%
\pgfpathlineto{\pgfqpoint{0.909814in}{0.773818in}}%
\pgfpathlineto{\pgfqpoint{0.911319in}{0.763524in}}%
\pgfpathlineto{\pgfqpoint{0.912690in}{0.753230in}}%
\pgfpathlineto{\pgfqpoint{0.913963in}{0.742935in}}%
\pgfpathlineto{\pgfqpoint{0.915274in}{0.732641in}}%
\pgfpathlineto{\pgfqpoint{0.916829in}{0.722346in}}%
\pgfpathlineto{\pgfqpoint{0.918587in}{0.712052in}}%
\pgfpathlineto{\pgfqpoint{0.920370in}{0.708276in}}%
\pgfpathlineto{\pgfqpoint{0.923079in}{0.701757in}}%
\pgfpathlineto{\pgfqpoint{0.928324in}{0.691463in}}%
\pgfpathlineto{\pgfqpoint{0.930947in}{0.683914in}}%
\pgfpathlineto{\pgfqpoint{0.931811in}{0.681169in}}%
\pgfpathlineto{\pgfqpoint{0.934854in}{0.670874in}}%
\pgfpathlineto{\pgfqpoint{0.937783in}{0.660580in}}%
\pgfpathlineto{\pgfqpoint{0.940642in}{0.650285in}}%
\pgfpathlineto{\pgfqpoint{0.941524in}{0.647580in}}%
\pgfpathlineto{\pgfqpoint{0.944830in}{0.639991in}}%
\pgfpathlineto{\pgfqpoint{0.949226in}{0.629696in}}%
\pgfpathlineto{\pgfqpoint{0.952101in}{0.622705in}}%
\pgfpathlineto{\pgfqpoint{0.953421in}{0.619402in}}%
\pgfpathlineto{\pgfqpoint{0.958753in}{0.609108in}}%
\pgfpathlineto{\pgfqpoint{0.962678in}{0.602578in}}%
\pgfpathlineto{\pgfqpoint{0.964898in}{0.598813in}}%
\pgfpathlineto{\pgfqpoint{0.970457in}{0.588519in}}%
\pgfpathlineto{\pgfqpoint{0.973255in}{0.582995in}}%
\pgfpathlineto{\pgfqpoint{0.975645in}{0.578224in}}%
\pgfpathlineto{\pgfqpoint{0.980698in}{0.567930in}}%
\pgfpathlineto{\pgfqpoint{0.983833in}{0.561322in}}%
\pgfpathlineto{\pgfqpoint{0.985608in}{0.557635in}}%
\pgfpathlineto{\pgfqpoint{0.990712in}{0.547341in}}%
\pgfpathlineto{\pgfqpoint{0.994410in}{0.539805in}}%
\pgfpathlineto{\pgfqpoint{0.995758in}{0.537047in}}%
\pgfpathlineto{\pgfqpoint{1.000723in}{0.526752in}}%
\pgfpathlineto{\pgfqpoint{1.004987in}{0.517799in}}%
\pgfpathlineto{\pgfqpoint{1.005621in}{0.516458in}}%
\pgfpathlineto{\pgfqpoint{1.010379in}{0.506163in}}%
\pgfpathlineto{\pgfqpoint{1.015019in}{0.495869in}}%
\pgfpathclose%
\pgfpathmoveto{\pgfqpoint{1.067107in}{0.567930in}}%
\pgfpathlineto{\pgfqpoint{1.063085in}{0.578224in}}%
\pgfpathlineto{\pgfqpoint{1.058893in}{0.588519in}}%
\pgfpathlineto{\pgfqpoint{1.057872in}{0.590481in}}%
\pgfpathlineto{\pgfqpoint{1.055554in}{0.598813in}}%
\pgfpathlineto{\pgfqpoint{1.052216in}{0.609108in}}%
\pgfpathlineto{\pgfqpoint{1.048186in}{0.619402in}}%
\pgfpathlineto{\pgfqpoint{1.047295in}{0.621830in}}%
\pgfpathlineto{\pgfqpoint{1.045063in}{0.629696in}}%
\pgfpathlineto{\pgfqpoint{1.043055in}{0.639991in}}%
\pgfpathlineto{\pgfqpoint{1.041380in}{0.650285in}}%
\pgfpathlineto{\pgfqpoint{1.039378in}{0.660580in}}%
\pgfpathlineto{\pgfqpoint{1.038436in}{0.670874in}}%
\pgfpathlineto{\pgfqpoint{1.038446in}{0.681169in}}%
\pgfpathlineto{\pgfqpoint{1.038734in}{0.691463in}}%
\pgfpathlineto{\pgfqpoint{1.039451in}{0.701757in}}%
\pgfpathlineto{\pgfqpoint{1.042453in}{0.712052in}}%
\pgfpathlineto{\pgfqpoint{1.047295in}{0.716952in}}%
\pgfpathlineto{\pgfqpoint{1.057872in}{0.715869in}}%
\pgfpathlineto{\pgfqpoint{1.068449in}{0.712629in}}%
\pgfpathlineto{\pgfqpoint{1.069015in}{0.712052in}}%
\pgfpathlineto{\pgfqpoint{1.076763in}{0.701757in}}%
\pgfpathlineto{\pgfqpoint{1.079026in}{0.698776in}}%
\pgfpathlineto{\pgfqpoint{1.080996in}{0.691463in}}%
\pgfpathlineto{\pgfqpoint{1.083677in}{0.681169in}}%
\pgfpathlineto{\pgfqpoint{1.086445in}{0.670874in}}%
\pgfpathlineto{\pgfqpoint{1.088326in}{0.660580in}}%
\pgfpathlineto{\pgfqpoint{1.089603in}{0.654458in}}%
\pgfpathlineto{\pgfqpoint{1.090280in}{0.650285in}}%
\pgfpathlineto{\pgfqpoint{1.091369in}{0.639991in}}%
\pgfpathlineto{\pgfqpoint{1.091455in}{0.629696in}}%
\pgfpathlineto{\pgfqpoint{1.091197in}{0.619402in}}%
\pgfpathlineto{\pgfqpoint{1.091002in}{0.609108in}}%
\pgfpathlineto{\pgfqpoint{1.090390in}{0.598813in}}%
\pgfpathlineto{\pgfqpoint{1.089603in}{0.593100in}}%
\pgfpathlineto{\pgfqpoint{1.086345in}{0.588519in}}%
\pgfpathlineto{\pgfqpoint{1.079962in}{0.578224in}}%
\pgfpathlineto{\pgfqpoint{1.079026in}{0.576506in}}%
\pgfpathlineto{\pgfqpoint{1.074169in}{0.567930in}}%
\pgfpathlineto{\pgfqpoint{1.068449in}{0.566204in}}%
\pgfpathclose%
\pgfusepath{fill}%
\end{pgfscope}%
\begin{pgfscope}%
\pgfpathrectangle{\pgfqpoint{0.423750in}{0.423750in}}{\pgfqpoint{1.194205in}{1.163386in}}%
\pgfusepath{clip}%
\pgfsetbuttcap%
\pgfsetroundjoin%
\definecolor{currentfill}{rgb}{0.970255,0.815666,0.711203}%
\pgfsetfillcolor{currentfill}%
\pgfsetlinewidth{0.000000pt}%
\definecolor{currentstroke}{rgb}{0.000000,0.000000,0.000000}%
\pgfsetstrokecolor{currentstroke}%
\pgfsetdash{}{0pt}%
\pgfpathmoveto{\pgfqpoint{0.878062in}{0.881260in}}%
\pgfpathlineto{\pgfqpoint{0.888639in}{0.882921in}}%
\pgfpathlineto{\pgfqpoint{0.898627in}{0.887057in}}%
\pgfpathlineto{\pgfqpoint{0.897846in}{0.897352in}}%
\pgfpathlineto{\pgfqpoint{0.897011in}{0.907646in}}%
\pgfpathlineto{\pgfqpoint{0.899216in}{0.917439in}}%
\pgfpathlineto{\pgfqpoint{0.899291in}{0.917941in}}%
\pgfpathlineto{\pgfqpoint{0.899216in}{0.918095in}}%
\pgfpathlineto{\pgfqpoint{0.888639in}{0.918098in}}%
\pgfpathlineto{\pgfqpoint{0.878062in}{0.921961in}}%
\pgfpathlineto{\pgfqpoint{0.868377in}{0.928235in}}%
\pgfpathlineto{\pgfqpoint{0.867485in}{0.928826in}}%
\pgfpathlineto{\pgfqpoint{0.856908in}{0.936763in}}%
\pgfpathlineto{\pgfqpoint{0.854801in}{0.938529in}}%
\pgfpathlineto{\pgfqpoint{0.846331in}{0.945813in}}%
\pgfpathlineto{\pgfqpoint{0.843183in}{0.948824in}}%
\pgfpathlineto{\pgfqpoint{0.835754in}{0.956148in}}%
\pgfpathlineto{\pgfqpoint{0.833037in}{0.959118in}}%
\pgfpathlineto{\pgfqpoint{0.825177in}{0.968028in}}%
\pgfpathlineto{\pgfqpoint{0.823891in}{0.969413in}}%
\pgfpathlineto{\pgfqpoint{0.816247in}{0.979707in}}%
\pgfpathlineto{\pgfqpoint{0.814600in}{0.981997in}}%
\pgfpathlineto{\pgfqpoint{0.807706in}{0.990002in}}%
\pgfpathlineto{\pgfqpoint{0.804023in}{0.996000in}}%
\pgfpathlineto{\pgfqpoint{0.801174in}{1.000296in}}%
\pgfpathlineto{\pgfqpoint{0.794571in}{1.010590in}}%
\pgfpathlineto{\pgfqpoint{0.793445in}{1.012802in}}%
\pgfpathlineto{\pgfqpoint{0.789293in}{1.020885in}}%
\pgfpathlineto{\pgfqpoint{0.783816in}{1.031179in}}%
\pgfpathlineto{\pgfqpoint{0.782868in}{1.033172in}}%
\pgfpathlineto{\pgfqpoint{0.780731in}{1.041474in}}%
\pgfpathlineto{\pgfqpoint{0.782868in}{1.051196in}}%
\pgfpathlineto{\pgfqpoint{0.782981in}{1.051768in}}%
\pgfpathlineto{\pgfqpoint{0.786624in}{1.062063in}}%
\pgfpathlineto{\pgfqpoint{0.789250in}{1.072357in}}%
\pgfpathlineto{\pgfqpoint{0.791002in}{1.082651in}}%
\pgfpathlineto{\pgfqpoint{0.792157in}{1.092946in}}%
\pgfpathlineto{\pgfqpoint{0.791550in}{1.103240in}}%
\pgfpathlineto{\pgfqpoint{0.791191in}{1.113535in}}%
\pgfpathlineto{\pgfqpoint{0.790376in}{1.123829in}}%
\pgfpathlineto{\pgfqpoint{0.789463in}{1.134124in}}%
\pgfpathlineto{\pgfqpoint{0.789546in}{1.144418in}}%
\pgfpathlineto{\pgfqpoint{0.788894in}{1.154712in}}%
\pgfpathlineto{\pgfqpoint{0.786554in}{1.165007in}}%
\pgfpathlineto{\pgfqpoint{0.784493in}{1.175301in}}%
\pgfpathlineto{\pgfqpoint{0.783133in}{1.185596in}}%
\pgfpathlineto{\pgfqpoint{0.782868in}{1.187129in}}%
\pgfpathlineto{\pgfqpoint{0.781043in}{1.195890in}}%
\pgfpathlineto{\pgfqpoint{0.779197in}{1.206185in}}%
\pgfpathlineto{\pgfqpoint{0.775568in}{1.216479in}}%
\pgfpathlineto{\pgfqpoint{0.772967in}{1.226773in}}%
\pgfpathlineto{\pgfqpoint{0.772291in}{1.228815in}}%
\pgfpathlineto{\pgfqpoint{0.768996in}{1.237068in}}%
\pgfpathlineto{\pgfqpoint{0.765624in}{1.247362in}}%
\pgfpathlineto{\pgfqpoint{0.762080in}{1.257657in}}%
\pgfpathlineto{\pgfqpoint{0.761714in}{1.258478in}}%
\pgfpathlineto{\pgfqpoint{0.757264in}{1.267951in}}%
\pgfpathlineto{\pgfqpoint{0.753077in}{1.278246in}}%
\pgfpathlineto{\pgfqpoint{0.751137in}{1.282640in}}%
\pgfpathlineto{\pgfqpoint{0.748441in}{1.288540in}}%
\pgfpathlineto{\pgfqpoint{0.744172in}{1.298834in}}%
\pgfpathlineto{\pgfqpoint{0.740560in}{1.307155in}}%
\pgfpathlineto{\pgfqpoint{0.739685in}{1.309129in}}%
\pgfpathlineto{\pgfqpoint{0.735414in}{1.319423in}}%
\pgfpathlineto{\pgfqpoint{0.731122in}{1.329718in}}%
\pgfpathlineto{\pgfqpoint{0.729983in}{1.332557in}}%
\pgfpathlineto{\pgfqpoint{0.726994in}{1.340012in}}%
\pgfpathlineto{\pgfqpoint{0.722692in}{1.350307in}}%
\pgfpathlineto{\pgfqpoint{0.719406in}{1.358450in}}%
\pgfpathlineto{\pgfqpoint{0.718536in}{1.360601in}}%
\pgfpathlineto{\pgfqpoint{0.714275in}{1.370896in}}%
\pgfpathlineto{\pgfqpoint{0.710080in}{1.381190in}}%
\pgfpathlineto{\pgfqpoint{0.708829in}{1.384224in}}%
\pgfpathlineto{\pgfqpoint{0.705895in}{1.391484in}}%
\pgfpathlineto{\pgfqpoint{0.701745in}{1.401779in}}%
\pgfpathlineto{\pgfqpoint{0.698252in}{1.410371in}}%
\pgfpathlineto{\pgfqpoint{0.697576in}{1.412073in}}%
\pgfpathlineto{\pgfqpoint{0.693557in}{1.422368in}}%
\pgfpathlineto{\pgfqpoint{0.689232in}{1.432662in}}%
\pgfpathlineto{\pgfqpoint{0.687675in}{1.436260in}}%
\pgfpathlineto{\pgfqpoint{0.684684in}{1.442957in}}%
\pgfpathlineto{\pgfqpoint{0.679699in}{1.453251in}}%
\pgfpathlineto{\pgfqpoint{0.677098in}{1.457897in}}%
\pgfpathlineto{\pgfqpoint{0.673991in}{1.463545in}}%
\pgfpathlineto{\pgfqpoint{0.667926in}{1.473840in}}%
\pgfpathlineto{\pgfqpoint{0.666521in}{1.476050in}}%
\pgfpathlineto{\pgfqpoint{0.661395in}{1.484134in}}%
\pgfpathlineto{\pgfqpoint{0.655944in}{1.493034in}}%
\pgfpathlineto{\pgfqpoint{0.655075in}{1.494429in}}%
\pgfpathlineto{\pgfqpoint{0.649171in}{1.504723in}}%
\pgfpathlineto{\pgfqpoint{0.645367in}{1.511402in}}%
\pgfpathlineto{\pgfqpoint{0.643314in}{1.515018in}}%
\pgfpathlineto{\pgfqpoint{0.634790in}{1.515018in}}%
\pgfpathlineto{\pgfqpoint{0.624212in}{1.515018in}}%
\pgfpathlineto{\pgfqpoint{0.613635in}{1.515018in}}%
\pgfpathlineto{\pgfqpoint{0.603058in}{1.515018in}}%
\pgfpathlineto{\pgfqpoint{0.592481in}{1.515018in}}%
\pgfpathlineto{\pgfqpoint{0.581904in}{1.515018in}}%
\pgfpathlineto{\pgfqpoint{0.571327in}{1.515018in}}%
\pgfpathlineto{\pgfqpoint{0.560750in}{1.515018in}}%
\pgfpathlineto{\pgfqpoint{0.550173in}{1.515018in}}%
\pgfpathlineto{\pgfqpoint{0.539596in}{1.515018in}}%
\pgfpathlineto{\pgfqpoint{0.529019in}{1.515018in}}%
\pgfpathlineto{\pgfqpoint{0.518442in}{1.515018in}}%
\pgfpathlineto{\pgfqpoint{0.507865in}{1.515018in}}%
\pgfpathlineto{\pgfqpoint{0.507469in}{1.515018in}}%
\pgfpathlineto{\pgfqpoint{0.507865in}{1.513788in}}%
\pgfpathlineto{\pgfqpoint{0.510695in}{1.504723in}}%
\pgfpathlineto{\pgfqpoint{0.513664in}{1.494429in}}%
\pgfpathlineto{\pgfqpoint{0.516413in}{1.484134in}}%
\pgfpathlineto{\pgfqpoint{0.518442in}{1.476202in}}%
\pgfpathlineto{\pgfqpoint{0.518980in}{1.473840in}}%
\pgfpathlineto{\pgfqpoint{0.520838in}{1.463545in}}%
\pgfpathlineto{\pgfqpoint{0.522550in}{1.453251in}}%
\pgfpathlineto{\pgfqpoint{0.524241in}{1.442957in}}%
\pgfpathlineto{\pgfqpoint{0.525894in}{1.432662in}}%
\pgfpathlineto{\pgfqpoint{0.527535in}{1.422368in}}%
\pgfpathlineto{\pgfqpoint{0.529019in}{1.412840in}}%
\pgfpathlineto{\pgfqpoint{0.529135in}{1.412073in}}%
\pgfpathlineto{\pgfqpoint{0.530942in}{1.401779in}}%
\pgfpathlineto{\pgfqpoint{0.532695in}{1.391484in}}%
\pgfpathlineto{\pgfqpoint{0.534410in}{1.381190in}}%
\pgfpathlineto{\pgfqpoint{0.536292in}{1.370896in}}%
\pgfpathlineto{\pgfqpoint{0.538183in}{1.360601in}}%
\pgfpathlineto{\pgfqpoint{0.539596in}{1.351389in}}%
\pgfpathlineto{\pgfqpoint{0.539788in}{1.350307in}}%
\pgfpathlineto{\pgfqpoint{0.542203in}{1.340012in}}%
\pgfpathlineto{\pgfqpoint{0.545171in}{1.329718in}}%
\pgfpathlineto{\pgfqpoint{0.547741in}{1.319423in}}%
\pgfpathlineto{\pgfqpoint{0.550173in}{1.310682in}}%
\pgfpathlineto{\pgfqpoint{0.550665in}{1.309129in}}%
\pgfpathlineto{\pgfqpoint{0.554025in}{1.298834in}}%
\pgfpathlineto{\pgfqpoint{0.557343in}{1.288540in}}%
\pgfpathlineto{\pgfqpoint{0.560105in}{1.278246in}}%
\pgfpathlineto{\pgfqpoint{0.560750in}{1.276341in}}%
\pgfpathlineto{\pgfqpoint{0.564065in}{1.267951in}}%
\pgfpathlineto{\pgfqpoint{0.567176in}{1.257657in}}%
\pgfpathlineto{\pgfqpoint{0.570892in}{1.247362in}}%
\pgfpathlineto{\pgfqpoint{0.571327in}{1.246166in}}%
\pgfpathlineto{\pgfqpoint{0.574763in}{1.237068in}}%
\pgfpathlineto{\pgfqpoint{0.578130in}{1.226773in}}%
\pgfpathlineto{\pgfqpoint{0.581602in}{1.216479in}}%
\pgfpathlineto{\pgfqpoint{0.581904in}{1.215576in}}%
\pgfpathlineto{\pgfqpoint{0.585036in}{1.206185in}}%
\pgfpathlineto{\pgfqpoint{0.588441in}{1.195890in}}%
\pgfpathlineto{\pgfqpoint{0.591819in}{1.185596in}}%
\pgfpathlineto{\pgfqpoint{0.592481in}{1.183544in}}%
\pgfpathlineto{\pgfqpoint{0.595243in}{1.175301in}}%
\pgfpathlineto{\pgfqpoint{0.598618in}{1.165007in}}%
\pgfpathlineto{\pgfqpoint{0.601872in}{1.154712in}}%
\pgfpathlineto{\pgfqpoint{0.603058in}{1.151040in}}%
\pgfpathlineto{\pgfqpoint{0.605417in}{1.144418in}}%
\pgfpathlineto{\pgfqpoint{0.608923in}{1.134124in}}%
\pgfpathlineto{\pgfqpoint{0.612418in}{1.123829in}}%
\pgfpathlineto{\pgfqpoint{0.613635in}{1.120423in}}%
\pgfpathlineto{\pgfqpoint{0.616221in}{1.113535in}}%
\pgfpathlineto{\pgfqpoint{0.620382in}{1.103240in}}%
\pgfpathlineto{\pgfqpoint{0.624212in}{1.093560in}}%
\pgfpathlineto{\pgfqpoint{0.624486in}{1.092946in}}%
\pgfpathlineto{\pgfqpoint{0.628653in}{1.082651in}}%
\pgfpathlineto{\pgfqpoint{0.632349in}{1.072357in}}%
\pgfpathlineto{\pgfqpoint{0.634790in}{1.065437in}}%
\pgfpathlineto{\pgfqpoint{0.636343in}{1.062063in}}%
\pgfpathlineto{\pgfqpoint{0.640606in}{1.051768in}}%
\pgfpathlineto{\pgfqpoint{0.644058in}{1.041474in}}%
\pgfpathlineto{\pgfqpoint{0.645367in}{1.037865in}}%
\pgfpathlineto{\pgfqpoint{0.649030in}{1.031179in}}%
\pgfpathlineto{\pgfqpoint{0.654694in}{1.020885in}}%
\pgfpathlineto{\pgfqpoint{0.655944in}{1.018247in}}%
\pgfpathlineto{\pgfqpoint{0.659798in}{1.010590in}}%
\pgfpathlineto{\pgfqpoint{0.665111in}{1.000296in}}%
\pgfpathlineto{\pgfqpoint{0.666521in}{0.997560in}}%
\pgfpathlineto{\pgfqpoint{0.674064in}{0.990002in}}%
\pgfpathlineto{\pgfqpoint{0.677098in}{0.986138in}}%
\pgfpathlineto{\pgfqpoint{0.687675in}{0.986427in}}%
\pgfpathlineto{\pgfqpoint{0.698252in}{0.986096in}}%
\pgfpathlineto{\pgfqpoint{0.708829in}{0.985129in}}%
\pgfpathlineto{\pgfqpoint{0.719406in}{0.983545in}}%
\pgfpathlineto{\pgfqpoint{0.729983in}{0.980035in}}%
\pgfpathlineto{\pgfqpoint{0.730540in}{0.979707in}}%
\pgfpathlineto{\pgfqpoint{0.740560in}{0.973928in}}%
\pgfpathlineto{\pgfqpoint{0.746569in}{0.969413in}}%
\pgfpathlineto{\pgfqpoint{0.751137in}{0.966427in}}%
\pgfpathlineto{\pgfqpoint{0.760188in}{0.959118in}}%
\pgfpathlineto{\pgfqpoint{0.761714in}{0.958028in}}%
\pgfpathlineto{\pgfqpoint{0.772291in}{0.950155in}}%
\pgfpathlineto{\pgfqpoint{0.774689in}{0.948824in}}%
\pgfpathlineto{\pgfqpoint{0.782868in}{0.944827in}}%
\pgfpathlineto{\pgfqpoint{0.792517in}{0.938529in}}%
\pgfpathlineto{\pgfqpoint{0.793445in}{0.938015in}}%
\pgfpathlineto{\pgfqpoint{0.804023in}{0.930664in}}%
\pgfpathlineto{\pgfqpoint{0.807673in}{0.928235in}}%
\pgfpathlineto{\pgfqpoint{0.814600in}{0.923277in}}%
\pgfpathlineto{\pgfqpoint{0.823457in}{0.917941in}}%
\pgfpathlineto{\pgfqpoint{0.825177in}{0.916690in}}%
\pgfpathlineto{\pgfqpoint{0.835754in}{0.910426in}}%
\pgfpathlineto{\pgfqpoint{0.840532in}{0.907646in}}%
\pgfpathlineto{\pgfqpoint{0.846331in}{0.904031in}}%
\pgfpathlineto{\pgfqpoint{0.856908in}{0.898015in}}%
\pgfpathlineto{\pgfqpoint{0.857945in}{0.897352in}}%
\pgfpathlineto{\pgfqpoint{0.867485in}{0.891249in}}%
\pgfpathlineto{\pgfqpoint{0.872275in}{0.887057in}}%
\pgfpathclose%
\pgfpathmoveto{\pgfqpoint{0.644919in}{1.237068in}}%
\pgfpathlineto{\pgfqpoint{0.634790in}{1.241531in}}%
\pgfpathlineto{\pgfqpoint{0.632334in}{1.247362in}}%
\pgfpathlineto{\pgfqpoint{0.629167in}{1.257657in}}%
\pgfpathlineto{\pgfqpoint{0.624888in}{1.267951in}}%
\pgfpathlineto{\pgfqpoint{0.624212in}{1.269784in}}%
\pgfpathlineto{\pgfqpoint{0.622218in}{1.278246in}}%
\pgfpathlineto{\pgfqpoint{0.619246in}{1.288540in}}%
\pgfpathlineto{\pgfqpoint{0.616508in}{1.298834in}}%
\pgfpathlineto{\pgfqpoint{0.613635in}{1.308867in}}%
\pgfpathlineto{\pgfqpoint{0.613571in}{1.309129in}}%
\pgfpathlineto{\pgfqpoint{0.611304in}{1.319423in}}%
\pgfpathlineto{\pgfqpoint{0.609076in}{1.329718in}}%
\pgfpathlineto{\pgfqpoint{0.607110in}{1.340012in}}%
\pgfpathlineto{\pgfqpoint{0.604637in}{1.350307in}}%
\pgfpathlineto{\pgfqpoint{0.603058in}{1.354831in}}%
\pgfpathlineto{\pgfqpoint{0.602026in}{1.360601in}}%
\pgfpathlineto{\pgfqpoint{0.601347in}{1.370896in}}%
\pgfpathlineto{\pgfqpoint{0.600832in}{1.381190in}}%
\pgfpathlineto{\pgfqpoint{0.599725in}{1.391484in}}%
\pgfpathlineto{\pgfqpoint{0.600156in}{1.401779in}}%
\pgfpathlineto{\pgfqpoint{0.602284in}{1.412073in}}%
\pgfpathlineto{\pgfqpoint{0.603058in}{1.414676in}}%
\pgfpathlineto{\pgfqpoint{0.604544in}{1.412073in}}%
\pgfpathlineto{\pgfqpoint{0.613635in}{1.408812in}}%
\pgfpathlineto{\pgfqpoint{0.615718in}{1.401779in}}%
\pgfpathlineto{\pgfqpoint{0.617605in}{1.391484in}}%
\pgfpathlineto{\pgfqpoint{0.620351in}{1.381190in}}%
\pgfpathlineto{\pgfqpoint{0.624212in}{1.373949in}}%
\pgfpathlineto{\pgfqpoint{0.625146in}{1.370896in}}%
\pgfpathlineto{\pgfqpoint{0.627335in}{1.360601in}}%
\pgfpathlineto{\pgfqpoint{0.629558in}{1.350307in}}%
\pgfpathlineto{\pgfqpoint{0.632718in}{1.340012in}}%
\pgfpathlineto{\pgfqpoint{0.634790in}{1.334894in}}%
\pgfpathlineto{\pgfqpoint{0.636093in}{1.329718in}}%
\pgfpathlineto{\pgfqpoint{0.637705in}{1.319423in}}%
\pgfpathlineto{\pgfqpoint{0.639748in}{1.309129in}}%
\pgfpathlineto{\pgfqpoint{0.643300in}{1.298834in}}%
\pgfpathlineto{\pgfqpoint{0.645367in}{1.293735in}}%
\pgfpathlineto{\pgfqpoint{0.646538in}{1.288540in}}%
\pgfpathlineto{\pgfqpoint{0.647733in}{1.278246in}}%
\pgfpathlineto{\pgfqpoint{0.649789in}{1.267951in}}%
\pgfpathlineto{\pgfqpoint{0.653293in}{1.257657in}}%
\pgfpathlineto{\pgfqpoint{0.655944in}{1.251680in}}%
\pgfpathlineto{\pgfqpoint{0.656621in}{1.247362in}}%
\pgfpathlineto{\pgfqpoint{0.656280in}{1.237068in}}%
\pgfpathlineto{\pgfqpoint{0.655944in}{1.236445in}}%
\pgfpathlineto{\pgfqpoint{0.645367in}{1.236832in}}%
\pgfpathclose%
\pgfusepath{fill}%
\end{pgfscope}%
\begin{pgfscope}%
\pgfpathrectangle{\pgfqpoint{0.423750in}{0.423750in}}{\pgfqpoint{1.194205in}{1.163386in}}%
\pgfusepath{clip}%
\pgfsetbuttcap%
\pgfsetroundjoin%
\definecolor{currentfill}{rgb}{0.976961,0.885681,0.814303}%
\pgfsetfillcolor{currentfill}%
\pgfsetlinewidth{0.000000pt}%
\definecolor{currentstroke}{rgb}{0.000000,0.000000,0.000000}%
\pgfsetstrokecolor{currentstroke}%
\pgfsetdash{}{0pt}%
\pgfpathmoveto{\pgfqpoint{1.068449in}{0.566204in}}%
\pgfpathlineto{\pgfqpoint{1.074169in}{0.567930in}}%
\pgfpathlineto{\pgfqpoint{1.079026in}{0.576506in}}%
\pgfpathlineto{\pgfqpoint{1.079962in}{0.578224in}}%
\pgfpathlineto{\pgfqpoint{1.086345in}{0.588519in}}%
\pgfpathlineto{\pgfqpoint{1.089603in}{0.593100in}}%
\pgfpathlineto{\pgfqpoint{1.090390in}{0.598813in}}%
\pgfpathlineto{\pgfqpoint{1.091002in}{0.609108in}}%
\pgfpathlineto{\pgfqpoint{1.091197in}{0.619402in}}%
\pgfpathlineto{\pgfqpoint{1.091455in}{0.629696in}}%
\pgfpathlineto{\pgfqpoint{1.091369in}{0.639991in}}%
\pgfpathlineto{\pgfqpoint{1.090280in}{0.650285in}}%
\pgfpathlineto{\pgfqpoint{1.089603in}{0.654458in}}%
\pgfpathlineto{\pgfqpoint{1.088326in}{0.660580in}}%
\pgfpathlineto{\pgfqpoint{1.086445in}{0.670874in}}%
\pgfpathlineto{\pgfqpoint{1.083677in}{0.681169in}}%
\pgfpathlineto{\pgfqpoint{1.080996in}{0.691463in}}%
\pgfpathlineto{\pgfqpoint{1.079026in}{0.698776in}}%
\pgfpathlineto{\pgfqpoint{1.076763in}{0.701757in}}%
\pgfpathlineto{\pgfqpoint{1.069015in}{0.712052in}}%
\pgfpathlineto{\pgfqpoint{1.068449in}{0.712629in}}%
\pgfpathlineto{\pgfqpoint{1.057872in}{0.715869in}}%
\pgfpathlineto{\pgfqpoint{1.047295in}{0.716952in}}%
\pgfpathlineto{\pgfqpoint{1.042453in}{0.712052in}}%
\pgfpathlineto{\pgfqpoint{1.039451in}{0.701757in}}%
\pgfpathlineto{\pgfqpoint{1.038734in}{0.691463in}}%
\pgfpathlineto{\pgfqpoint{1.038446in}{0.681169in}}%
\pgfpathlineto{\pgfqpoint{1.038436in}{0.670874in}}%
\pgfpathlineto{\pgfqpoint{1.039378in}{0.660580in}}%
\pgfpathlineto{\pgfqpoint{1.041380in}{0.650285in}}%
\pgfpathlineto{\pgfqpoint{1.043055in}{0.639991in}}%
\pgfpathlineto{\pgfqpoint{1.045063in}{0.629696in}}%
\pgfpathlineto{\pgfqpoint{1.047295in}{0.621830in}}%
\pgfpathlineto{\pgfqpoint{1.048186in}{0.619402in}}%
\pgfpathlineto{\pgfqpoint{1.052216in}{0.609108in}}%
\pgfpathlineto{\pgfqpoint{1.055554in}{0.598813in}}%
\pgfpathlineto{\pgfqpoint{1.057872in}{0.590481in}}%
\pgfpathlineto{\pgfqpoint{1.058893in}{0.588519in}}%
\pgfpathlineto{\pgfqpoint{1.063085in}{0.578224in}}%
\pgfpathlineto{\pgfqpoint{1.067107in}{0.567930in}}%
\pgfpathclose%
\pgfusepath{fill}%
\end{pgfscope}%
\begin{pgfscope}%
\pgfpathrectangle{\pgfqpoint{0.423750in}{0.423750in}}{\pgfqpoint{1.194205in}{1.163386in}}%
\pgfusepath{clip}%
\pgfsetbuttcap%
\pgfsetroundjoin%
\definecolor{currentfill}{rgb}{0.976961,0.885681,0.814303}%
\pgfsetfillcolor{currentfill}%
\pgfsetlinewidth{0.000000pt}%
\definecolor{currentstroke}{rgb}{0.000000,0.000000,0.000000}%
\pgfsetstrokecolor{currentstroke}%
\pgfsetdash{}{0pt}%
\pgfpathmoveto{\pgfqpoint{0.645367in}{1.236832in}}%
\pgfpathlineto{\pgfqpoint{0.655944in}{1.236445in}}%
\pgfpathlineto{\pgfqpoint{0.656280in}{1.237068in}}%
\pgfpathlineto{\pgfqpoint{0.656621in}{1.247362in}}%
\pgfpathlineto{\pgfqpoint{0.655944in}{1.251680in}}%
\pgfpathlineto{\pgfqpoint{0.653293in}{1.257657in}}%
\pgfpathlineto{\pgfqpoint{0.649789in}{1.267951in}}%
\pgfpathlineto{\pgfqpoint{0.647733in}{1.278246in}}%
\pgfpathlineto{\pgfqpoint{0.646538in}{1.288540in}}%
\pgfpathlineto{\pgfqpoint{0.645367in}{1.293735in}}%
\pgfpathlineto{\pgfqpoint{0.643300in}{1.298834in}}%
\pgfpathlineto{\pgfqpoint{0.639748in}{1.309129in}}%
\pgfpathlineto{\pgfqpoint{0.637705in}{1.319423in}}%
\pgfpathlineto{\pgfqpoint{0.636093in}{1.329718in}}%
\pgfpathlineto{\pgfqpoint{0.634790in}{1.334894in}}%
\pgfpathlineto{\pgfqpoint{0.632718in}{1.340012in}}%
\pgfpathlineto{\pgfqpoint{0.629558in}{1.350307in}}%
\pgfpathlineto{\pgfqpoint{0.627335in}{1.360601in}}%
\pgfpathlineto{\pgfqpoint{0.625146in}{1.370896in}}%
\pgfpathlineto{\pgfqpoint{0.624212in}{1.373949in}}%
\pgfpathlineto{\pgfqpoint{0.620351in}{1.381190in}}%
\pgfpathlineto{\pgfqpoint{0.617605in}{1.391484in}}%
\pgfpathlineto{\pgfqpoint{0.615718in}{1.401779in}}%
\pgfpathlineto{\pgfqpoint{0.613635in}{1.408812in}}%
\pgfpathlineto{\pgfqpoint{0.604544in}{1.412073in}}%
\pgfpathlineto{\pgfqpoint{0.603058in}{1.414676in}}%
\pgfpathlineto{\pgfqpoint{0.602284in}{1.412073in}}%
\pgfpathlineto{\pgfqpoint{0.600156in}{1.401779in}}%
\pgfpathlineto{\pgfqpoint{0.599725in}{1.391484in}}%
\pgfpathlineto{\pgfqpoint{0.600832in}{1.381190in}}%
\pgfpathlineto{\pgfqpoint{0.601347in}{1.370896in}}%
\pgfpathlineto{\pgfqpoint{0.602026in}{1.360601in}}%
\pgfpathlineto{\pgfqpoint{0.603058in}{1.354831in}}%
\pgfpathlineto{\pgfqpoint{0.604637in}{1.350307in}}%
\pgfpathlineto{\pgfqpoint{0.607110in}{1.340012in}}%
\pgfpathlineto{\pgfqpoint{0.609076in}{1.329718in}}%
\pgfpathlineto{\pgfqpoint{0.611304in}{1.319423in}}%
\pgfpathlineto{\pgfqpoint{0.613571in}{1.309129in}}%
\pgfpathlineto{\pgfqpoint{0.613635in}{1.308867in}}%
\pgfpathlineto{\pgfqpoint{0.616508in}{1.298834in}}%
\pgfpathlineto{\pgfqpoint{0.619246in}{1.288540in}}%
\pgfpathlineto{\pgfqpoint{0.622218in}{1.278246in}}%
\pgfpathlineto{\pgfqpoint{0.624212in}{1.269784in}}%
\pgfpathlineto{\pgfqpoint{0.624888in}{1.267951in}}%
\pgfpathlineto{\pgfqpoint{0.629167in}{1.257657in}}%
\pgfpathlineto{\pgfqpoint{0.632334in}{1.247362in}}%
\pgfpathlineto{\pgfqpoint{0.634790in}{1.241531in}}%
\pgfpathlineto{\pgfqpoint{0.644919in}{1.237068in}}%
\pgfpathclose%
\pgfusepath{fill}%
\end{pgfscope}%
\begin{pgfscope}%
\pgfpathrectangle{\pgfqpoint{0.423750in}{0.423750in}}{\pgfqpoint{1.194205in}{1.163386in}}%
\pgfusepath{clip}%
\pgfsetbuttcap%
\pgfsetroundjoin%
\definecolor{currentfill}{rgb}{0.121569,0.466667,0.705882}%
\pgfsetfillcolor{currentfill}%
\pgfsetlinewidth{1.003750pt}%
\definecolor{currentstroke}{rgb}{0.121569,0.466667,0.705882}%
\pgfsetstrokecolor{currentstroke}%
\pgfsetdash{}{0pt}%
\pgfsys@defobject{currentmarker}{\pgfqpoint{-0.021960in}{-0.021960in}}{\pgfqpoint{0.021960in}{0.021960in}}{%
\pgfpathmoveto{\pgfqpoint{0.000000in}{-0.021960in}}%
\pgfpathcurveto{\pgfqpoint{0.005824in}{-0.021960in}}{\pgfqpoint{0.011410in}{-0.019646in}}{\pgfqpoint{0.015528in}{-0.015528in}}%
\pgfpathcurveto{\pgfqpoint{0.019646in}{-0.011410in}}{\pgfqpoint{0.021960in}{-0.005824in}}{\pgfqpoint{0.021960in}{0.000000in}}%
\pgfpathcurveto{\pgfqpoint{0.021960in}{0.005824in}}{\pgfqpoint{0.019646in}{0.011410in}}{\pgfqpoint{0.015528in}{0.015528in}}%
\pgfpathcurveto{\pgfqpoint{0.011410in}{0.019646in}}{\pgfqpoint{0.005824in}{0.021960in}}{\pgfqpoint{0.000000in}{0.021960in}}%
\pgfpathcurveto{\pgfqpoint{-0.005824in}{0.021960in}}{\pgfqpoint{-0.011410in}{0.019646in}}{\pgfqpoint{-0.015528in}{0.015528in}}%
\pgfpathcurveto{\pgfqpoint{-0.019646in}{0.011410in}}{\pgfqpoint{-0.021960in}{0.005824in}}{\pgfqpoint{-0.021960in}{0.000000in}}%
\pgfpathcurveto{\pgfqpoint{-0.021960in}{-0.005824in}}{\pgfqpoint{-0.019646in}{-0.011410in}}{\pgfqpoint{-0.015528in}{-0.015528in}}%
\pgfpathcurveto{\pgfqpoint{-0.011410in}{-0.019646in}}{\pgfqpoint{-0.005824in}{-0.021960in}}{\pgfqpoint{0.000000in}{-0.021960in}}%
\pgfpathclose%
\pgfusepath{stroke,fill}%
}%
\begin{pgfscope}%
\pgfsys@transformshift{1.374715in}{0.898984in}%
\pgfsys@useobject{currentmarker}{}%
\end{pgfscope}%
\begin{pgfscope}%
\pgfsys@transformshift{1.071801in}{0.958587in}%
\pgfsys@useobject{currentmarker}{}%
\end{pgfscope}%
\begin{pgfscope}%
\pgfsys@transformshift{1.467443in}{0.874090in}%
\pgfsys@useobject{currentmarker}{}%
\end{pgfscope}%
\begin{pgfscope}%
\pgfsys@transformshift{1.222816in}{1.172572in}%
\pgfsys@useobject{currentmarker}{}%
\end{pgfscope}%
\begin{pgfscope}%
\pgfsys@transformshift{1.060594in}{1.132075in}%
\pgfsys@useobject{currentmarker}{}%
\end{pgfscope}%
\begin{pgfscope}%
\pgfsys@transformshift{0.995668in}{1.209548in}%
\pgfsys@useobject{currentmarker}{}%
\end{pgfscope}%
\begin{pgfscope}%
\pgfsys@transformshift{0.979767in}{1.057501in}%
\pgfsys@useobject{currentmarker}{}%
\end{pgfscope}%
\begin{pgfscope}%
\pgfsys@transformshift{1.470397in}{1.334294in}%
\pgfsys@useobject{currentmarker}{}%
\end{pgfscope}%
\begin{pgfscope}%
\pgfsys@transformshift{1.383427in}{0.785418in}%
\pgfsys@useobject{currentmarker}{}%
\end{pgfscope}%
\begin{pgfscope}%
\pgfsys@transformshift{1.141316in}{1.450533in}%
\pgfsys@useobject{currentmarker}{}%
\end{pgfscope}%
\begin{pgfscope}%
\pgfsys@transformshift{1.077692in}{0.901561in}%
\pgfsys@useobject{currentmarker}{}%
\end{pgfscope}%
\begin{pgfscope}%
\pgfsys@transformshift{0.685350in}{0.689794in}%
\pgfsys@useobject{currentmarker}{}%
\end{pgfscope}%
\begin{pgfscope}%
\pgfsys@transformshift{1.453383in}{1.204143in}%
\pgfsys@useobject{currentmarker}{}%
\end{pgfscope}%
\begin{pgfscope}%
\pgfsys@transformshift{0.870774in}{0.830355in}%
\pgfsys@useobject{currentmarker}{}%
\end{pgfscope}%
\begin{pgfscope}%
\pgfsys@transformshift{1.367689in}{0.869496in}%
\pgfsys@useobject{currentmarker}{}%
\end{pgfscope}%
\begin{pgfscope}%
\pgfsys@transformshift{1.454864in}{1.473475in}%
\pgfsys@useobject{currentmarker}{}%
\end{pgfscope}%
\begin{pgfscope}%
\pgfsys@transformshift{1.028580in}{1.407201in}%
\pgfsys@useobject{currentmarker}{}%
\end{pgfscope}%
\begin{pgfscope}%
\pgfsys@transformshift{1.484099in}{0.854246in}%
\pgfsys@useobject{currentmarker}{}%
\end{pgfscope}%
\begin{pgfscope}%
\pgfsys@transformshift{0.942739in}{1.360525in}%
\pgfsys@useobject{currentmarker}{}%
\end{pgfscope}%
\begin{pgfscope}%
\pgfsys@transformshift{0.789646in}{0.560542in}%
\pgfsys@useobject{currentmarker}{}%
\end{pgfscope}%
\begin{pgfscope}%
\pgfsys@transformshift{1.530187in}{0.641160in}%
\pgfsys@useobject{currentmarker}{}%
\end{pgfscope}%
\begin{pgfscope}%
\pgfsys@transformshift{0.497288in}{1.257460in}%
\pgfsys@useobject{currentmarker}{}%
\end{pgfscope}%
\begin{pgfscope}%
\pgfsys@transformshift{1.307068in}{0.495869in}%
\pgfsys@useobject{currentmarker}{}%
\end{pgfscope}%
\begin{pgfscope}%
\pgfsys@transformshift{1.544417in}{0.882860in}%
\pgfsys@useobject{currentmarker}{}%
\end{pgfscope}%
\begin{pgfscope}%
\pgfsys@transformshift{1.528907in}{0.791009in}%
\pgfsys@useobject{currentmarker}{}%
\end{pgfscope}%
\begin{pgfscope}%
\pgfsys@transformshift{0.556068in}{1.484792in}%
\pgfsys@useobject{currentmarker}{}%
\end{pgfscope}%
\begin{pgfscope}%
\pgfsys@transformshift{1.544417in}{0.650160in}%
\pgfsys@useobject{currentmarker}{}%
\end{pgfscope}%
\begin{pgfscope}%
\pgfsys@transformshift{1.544417in}{0.534099in}%
\pgfsys@useobject{currentmarker}{}%
\end{pgfscope}%
\begin{pgfscope}%
\pgfsys@transformshift{0.613148in}{1.515018in}%
\pgfsys@useobject{currentmarker}{}%
\end{pgfscope}%
\begin{pgfscope}%
\pgfsys@transformshift{1.544417in}{0.749672in}%
\pgfsys@useobject{currentmarker}{}%
\end{pgfscope}%
\begin{pgfscope}%
\pgfsys@transformshift{0.600477in}{1.504191in}%
\pgfsys@useobject{currentmarker}{}%
\end{pgfscope}%
\begin{pgfscope}%
\pgfsys@transformshift{0.536807in}{0.515624in}%
\pgfsys@useobject{currentmarker}{}%
\end{pgfscope}%
\begin{pgfscope}%
\pgfsys@transformshift{0.549093in}{1.515018in}%
\pgfsys@useobject{currentmarker}{}%
\end{pgfscope}%
\begin{pgfscope}%
\pgfsys@transformshift{1.542261in}{0.662108in}%
\pgfsys@useobject{currentmarker}{}%
\end{pgfscope}%
\begin{pgfscope}%
\pgfsys@transformshift{0.571898in}{1.429788in}%
\pgfsys@useobject{currentmarker}{}%
\end{pgfscope}%
\begin{pgfscope}%
\pgfsys@transformshift{1.538057in}{0.636523in}%
\pgfsys@useobject{currentmarker}{}%
\end{pgfscope}%
\begin{pgfscope}%
\pgfsys@transformshift{1.534991in}{1.027105in}%
\pgfsys@useobject{currentmarker}{}%
\end{pgfscope}%
\begin{pgfscope}%
\pgfsys@transformshift{0.974735in}{0.698532in}%
\pgfsys@useobject{currentmarker}{}%
\end{pgfscope}%
\begin{pgfscope}%
\pgfsys@transformshift{1.015667in}{0.716623in}%
\pgfsys@useobject{currentmarker}{}%
\end{pgfscope}%
\begin{pgfscope}%
\pgfsys@transformshift{1.046856in}{0.720659in}%
\pgfsys@useobject{currentmarker}{}%
\end{pgfscope}%
\begin{pgfscope}%
\pgfsys@transformshift{0.618433in}{1.388616in}%
\pgfsys@useobject{currentmarker}{}%
\end{pgfscope}%
\begin{pgfscope}%
\pgfsys@transformshift{0.698165in}{1.315625in}%
\pgfsys@useobject{currentmarker}{}%
\end{pgfscope}%
\begin{pgfscope}%
\pgfsys@transformshift{1.075249in}{0.638294in}%
\pgfsys@useobject{currentmarker}{}%
\end{pgfscope}%
\begin{pgfscope}%
\pgfsys@transformshift{1.070617in}{0.692234in}%
\pgfsys@useobject{currentmarker}{}%
\end{pgfscope}%
\begin{pgfscope}%
\pgfsys@transformshift{1.037980in}{0.670141in}%
\pgfsys@useobject{currentmarker}{}%
\end{pgfscope}%
\begin{pgfscope}%
\pgfsys@transformshift{1.140895in}{0.670083in}%
\pgfsys@useobject{currentmarker}{}%
\end{pgfscope}%
\begin{pgfscope}%
\pgfsys@transformshift{1.062433in}{0.672697in}%
\pgfsys@useobject{currentmarker}{}%
\end{pgfscope}%
\begin{pgfscope}%
\pgfsys@transformshift{1.109856in}{0.771144in}%
\pgfsys@useobject{currentmarker}{}%
\end{pgfscope}%
\begin{pgfscope}%
\pgfsys@transformshift{1.058202in}{0.676075in}%
\pgfsys@useobject{currentmarker}{}%
\end{pgfscope}%
\begin{pgfscope}%
\pgfsys@transformshift{0.598260in}{1.400232in}%
\pgfsys@useobject{currentmarker}{}%
\end{pgfscope}%
\begin{pgfscope}%
\pgfsys@transformshift{1.082922in}{0.693371in}%
\pgfsys@useobject{currentmarker}{}%
\end{pgfscope}%
\begin{pgfscope}%
\pgfsys@transformshift{1.048815in}{0.597462in}%
\pgfsys@useobject{currentmarker}{}%
\end{pgfscope}%
\begin{pgfscope}%
\pgfsys@transformshift{1.042362in}{0.590931in}%
\pgfsys@useobject{currentmarker}{}%
\end{pgfscope}%
\begin{pgfscope}%
\pgfsys@transformshift{1.052867in}{0.705123in}%
\pgfsys@useobject{currentmarker}{}%
\end{pgfscope}%
\begin{pgfscope}%
\pgfsys@transformshift{1.056306in}{0.619419in}%
\pgfsys@useobject{currentmarker}{}%
\end{pgfscope}%
\begin{pgfscope}%
\pgfsys@transformshift{0.497288in}{1.503511in}%
\pgfsys@useobject{currentmarker}{}%
\end{pgfscope}%
\begin{pgfscope}%
\pgfsys@transformshift{0.603229in}{1.419869in}%
\pgfsys@useobject{currentmarker}{}%
\end{pgfscope}%
\begin{pgfscope}%
\pgfsys@transformshift{0.747324in}{1.118155in}%
\pgfsys@useobject{currentmarker}{}%
\end{pgfscope}%
\begin{pgfscope}%
\pgfsys@transformshift{1.057118in}{0.690542in}%
\pgfsys@useobject{currentmarker}{}%
\end{pgfscope}%
\begin{pgfscope}%
\pgfsys@transformshift{0.643654in}{1.250698in}%
\pgfsys@useobject{currentmarker}{}%
\end{pgfscope}%
\begin{pgfscope}%
\pgfsys@transformshift{1.544417in}{0.671840in}%
\pgfsys@useobject{currentmarker}{}%
\end{pgfscope}%
\begin{pgfscope}%
\pgfsys@transformshift{1.072461in}{0.613659in}%
\pgfsys@useobject{currentmarker}{}%
\end{pgfscope}%
\begin{pgfscope}%
\pgfsys@transformshift{1.076339in}{0.678900in}%
\pgfsys@useobject{currentmarker}{}%
\end{pgfscope}%
\begin{pgfscope}%
\pgfsys@transformshift{0.623641in}{1.297821in}%
\pgfsys@useobject{currentmarker}{}%
\end{pgfscope}%
\begin{pgfscope}%
\pgfsys@transformshift{1.060015in}{0.648013in}%
\pgfsys@useobject{currentmarker}{}%
\end{pgfscope}%
\begin{pgfscope}%
\pgfsys@transformshift{1.049103in}{0.666788in}%
\pgfsys@useobject{currentmarker}{}%
\end{pgfscope}%
\begin{pgfscope}%
\pgfsys@transformshift{1.069242in}{0.632039in}%
\pgfsys@useobject{currentmarker}{}%
\end{pgfscope}%
\begin{pgfscope}%
\pgfsys@transformshift{1.076230in}{0.662584in}%
\pgfsys@useobject{currentmarker}{}%
\end{pgfscope}%
\begin{pgfscope}%
\pgfsys@transformshift{1.090223in}{0.637983in}%
\pgfsys@useobject{currentmarker}{}%
\end{pgfscope}%
\begin{pgfscope}%
\pgfsys@transformshift{0.673388in}{1.132677in}%
\pgfsys@useobject{currentmarker}{}%
\end{pgfscope}%
\begin{pgfscope}%
\pgfsys@transformshift{0.641571in}{1.192633in}%
\pgfsys@useobject{currentmarker}{}%
\end{pgfscope}%
\begin{pgfscope}%
\pgfsys@transformshift{1.042567in}{0.602172in}%
\pgfsys@useobject{currentmarker}{}%
\end{pgfscope}%
\begin{pgfscope}%
\pgfsys@transformshift{0.612554in}{1.364788in}%
\pgfsys@useobject{currentmarker}{}%
\end{pgfscope}%
\begin{pgfscope}%
\pgfsys@transformshift{1.067338in}{0.629527in}%
\pgfsys@useobject{currentmarker}{}%
\end{pgfscope}%
\begin{pgfscope}%
\pgfsys@transformshift{1.055616in}{0.665334in}%
\pgfsys@useobject{currentmarker}{}%
\end{pgfscope}%
\begin{pgfscope}%
\pgfsys@transformshift{1.544417in}{0.816147in}%
\pgfsys@useobject{currentmarker}{}%
\end{pgfscope}%
\begin{pgfscope}%
\pgfsys@transformshift{1.088063in}{0.664237in}%
\pgfsys@useobject{currentmarker}{}%
\end{pgfscope}%
\begin{pgfscope}%
\pgfsys@transformshift{0.534788in}{0.744318in}%
\pgfsys@useobject{currentmarker}{}%
\end{pgfscope}%
\begin{pgfscope}%
\pgfsys@transformshift{1.543453in}{0.708971in}%
\pgfsys@useobject{currentmarker}{}%
\end{pgfscope}%
\begin{pgfscope}%
\pgfsys@transformshift{0.593408in}{1.413208in}%
\pgfsys@useobject{currentmarker}{}%
\end{pgfscope}%
\begin{pgfscope}%
\pgfsys@transformshift{1.073938in}{0.622501in}%
\pgfsys@useobject{currentmarker}{}%
\end{pgfscope}%
\begin{pgfscope}%
\pgfsys@transformshift{1.078447in}{0.610862in}%
\pgfsys@useobject{currentmarker}{}%
\end{pgfscope}%
\begin{pgfscope}%
\pgfsys@transformshift{0.606295in}{1.339367in}%
\pgfsys@useobject{currentmarker}{}%
\end{pgfscope}%
\begin{pgfscope}%
\pgfsys@transformshift{0.618913in}{1.383813in}%
\pgfsys@useobject{currentmarker}{}%
\end{pgfscope}%
\begin{pgfscope}%
\pgfsys@transformshift{0.612032in}{1.309626in}%
\pgfsys@useobject{currentmarker}{}%
\end{pgfscope}%
\begin{pgfscope}%
\pgfsys@transformshift{0.581575in}{1.463385in}%
\pgfsys@useobject{currentmarker}{}%
\end{pgfscope}%
\begin{pgfscope}%
\pgfsys@transformshift{1.064807in}{0.673728in}%
\pgfsys@useobject{currentmarker}{}%
\end{pgfscope}%
\begin{pgfscope}%
\pgfsys@transformshift{1.542811in}{1.426802in}%
\pgfsys@useobject{currentmarker}{}%
\end{pgfscope}%
\begin{pgfscope}%
\pgfsys@transformshift{0.628453in}{1.264464in}%
\pgfsys@useobject{currentmarker}{}%
\end{pgfscope}%
\begin{pgfscope}%
\pgfsys@transformshift{0.596098in}{1.464605in}%
\pgfsys@useobject{currentmarker}{}%
\end{pgfscope}%
\begin{pgfscope}%
\pgfsys@transformshift{0.634952in}{1.299461in}%
\pgfsys@useobject{currentmarker}{}%
\end{pgfscope}%
\begin{pgfscope}%
\pgfsys@transformshift{1.069840in}{0.626787in}%
\pgfsys@useobject{currentmarker}{}%
\end{pgfscope}%
\begin{pgfscope}%
\pgfsys@transformshift{1.031528in}{0.611412in}%
\pgfsys@useobject{currentmarker}{}%
\end{pgfscope}%
\begin{pgfscope}%
\pgfsys@transformshift{0.599729in}{1.376997in}%
\pgfsys@useobject{currentmarker}{}%
\end{pgfscope}%
\begin{pgfscope}%
\pgfsys@transformshift{1.070115in}{0.599867in}%
\pgfsys@useobject{currentmarker}{}%
\end{pgfscope}%
\begin{pgfscope}%
\pgfsys@transformshift{1.066006in}{0.672578in}%
\pgfsys@useobject{currentmarker}{}%
\end{pgfscope}%
\begin{pgfscope}%
\pgfsys@transformshift{1.075198in}{0.598650in}%
\pgfsys@useobject{currentmarker}{}%
\end{pgfscope}%
\begin{pgfscope}%
\pgfsys@transformshift{1.066028in}{0.646194in}%
\pgfsys@useobject{currentmarker}{}%
\end{pgfscope}%
\begin{pgfscope}%
\pgfsys@transformshift{1.079618in}{0.670928in}%
\pgfsys@useobject{currentmarker}{}%
\end{pgfscope}%
\begin{pgfscope}%
\pgfsys@transformshift{1.075407in}{0.668515in}%
\pgfsys@useobject{currentmarker}{}%
\end{pgfscope}%
\begin{pgfscope}%
\pgfsys@transformshift{1.065899in}{0.643713in}%
\pgfsys@useobject{currentmarker}{}%
\end{pgfscope}%
\begin{pgfscope}%
\pgfsys@transformshift{1.060674in}{0.671209in}%
\pgfsys@useobject{currentmarker}{}%
\end{pgfscope}%
\begin{pgfscope}%
\pgfsys@transformshift{0.624467in}{1.136363in}%
\pgfsys@useobject{currentmarker}{}%
\end{pgfscope}%
\begin{pgfscope}%
\pgfsys@transformshift{0.632293in}{1.290295in}%
\pgfsys@useobject{currentmarker}{}%
\end{pgfscope}%
\begin{pgfscope}%
\pgfsys@transformshift{0.615208in}{1.344549in}%
\pgfsys@useobject{currentmarker}{}%
\end{pgfscope}%
\begin{pgfscope}%
\pgfsys@transformshift{1.091060in}{0.543778in}%
\pgfsys@useobject{currentmarker}{}%
\end{pgfscope}%
\begin{pgfscope}%
\pgfsys@transformshift{1.084898in}{0.634081in}%
\pgfsys@useobject{currentmarker}{}%
\end{pgfscope}%
\begin{pgfscope}%
\pgfsys@transformshift{0.624308in}{1.320034in}%
\pgfsys@useobject{currentmarker}{}%
\end{pgfscope}%
\begin{pgfscope}%
\pgfsys@transformshift{1.043589in}{0.649697in}%
\pgfsys@useobject{currentmarker}{}%
\end{pgfscope}%
\begin{pgfscope}%
\pgfsys@transformshift{0.604494in}{1.406939in}%
\pgfsys@useobject{currentmarker}{}%
\end{pgfscope}%
\begin{pgfscope}%
\pgfsys@transformshift{0.682367in}{0.812883in}%
\pgfsys@useobject{currentmarker}{}%
\end{pgfscope}%
\begin{pgfscope}%
\pgfsys@transformshift{0.599179in}{1.453496in}%
\pgfsys@useobject{currentmarker}{}%
\end{pgfscope}%
\begin{pgfscope}%
\pgfsys@transformshift{1.062710in}{0.657890in}%
\pgfsys@useobject{currentmarker}{}%
\end{pgfscope}%
\begin{pgfscope}%
\pgfsys@transformshift{1.072924in}{0.643751in}%
\pgfsys@useobject{currentmarker}{}%
\end{pgfscope}%
\begin{pgfscope}%
\pgfsys@transformshift{1.049104in}{0.691538in}%
\pgfsys@useobject{currentmarker}{}%
\end{pgfscope}%
\begin{pgfscope}%
\pgfsys@transformshift{0.585282in}{1.415314in}%
\pgfsys@useobject{currentmarker}{}%
\end{pgfscope}%
\begin{pgfscope}%
\pgfsys@transformshift{1.066088in}{0.626992in}%
\pgfsys@useobject{currentmarker}{}%
\end{pgfscope}%
\begin{pgfscope}%
\pgfsys@transformshift{1.056628in}{0.624781in}%
\pgfsys@useobject{currentmarker}{}%
\end{pgfscope}%
\begin{pgfscope}%
\pgfsys@transformshift{1.059903in}{0.654395in}%
\pgfsys@useobject{currentmarker}{}%
\end{pgfscope}%
\begin{pgfscope}%
\pgfsys@transformshift{1.048660in}{0.663188in}%
\pgfsys@useobject{currentmarker}{}%
\end{pgfscope}%
\begin{pgfscope}%
\pgfsys@transformshift{0.636634in}{1.267592in}%
\pgfsys@useobject{currentmarker}{}%
\end{pgfscope}%
\begin{pgfscope}%
\pgfsys@transformshift{1.074957in}{0.619743in}%
\pgfsys@useobject{currentmarker}{}%
\end{pgfscope}%
\begin{pgfscope}%
\pgfsys@transformshift{0.603799in}{1.265763in}%
\pgfsys@useobject{currentmarker}{}%
\end{pgfscope}%
\begin{pgfscope}%
\pgfsys@transformshift{0.602431in}{1.228495in}%
\pgfsys@useobject{currentmarker}{}%
\end{pgfscope}%
\begin{pgfscope}%
\pgfsys@transformshift{1.071637in}{0.646921in}%
\pgfsys@useobject{currentmarker}{}%
\end{pgfscope}%
\begin{pgfscope}%
\pgfsys@transformshift{0.624089in}{1.326784in}%
\pgfsys@useobject{currentmarker}{}%
\end{pgfscope}%
\begin{pgfscope}%
\pgfsys@transformshift{0.612159in}{1.349915in}%
\pgfsys@useobject{currentmarker}{}%
\end{pgfscope}%
\begin{pgfscope}%
\pgfsys@transformshift{0.609175in}{1.359427in}%
\pgfsys@useobject{currentmarker}{}%
\end{pgfscope}%
\begin{pgfscope}%
\pgfsys@transformshift{1.079088in}{0.632964in}%
\pgfsys@useobject{currentmarker}{}%
\end{pgfscope}%
\begin{pgfscope}%
\pgfsys@transformshift{1.067542in}{0.629997in}%
\pgfsys@useobject{currentmarker}{}%
\end{pgfscope}%
\begin{pgfscope}%
\pgfsys@transformshift{1.544417in}{0.643573in}%
\pgfsys@useobject{currentmarker}{}%
\end{pgfscope}%
\begin{pgfscope}%
\pgfsys@transformshift{1.058277in}{0.630550in}%
\pgfsys@useobject{currentmarker}{}%
\end{pgfscope}%
\begin{pgfscope}%
\pgfsys@transformshift{0.617461in}{1.344235in}%
\pgfsys@useobject{currentmarker}{}%
\end{pgfscope}%
\begin{pgfscope}%
\pgfsys@transformshift{0.622868in}{1.363997in}%
\pgfsys@useobject{currentmarker}{}%
\end{pgfscope}%
\begin{pgfscope}%
\pgfsys@transformshift{0.617947in}{1.381284in}%
\pgfsys@useobject{currentmarker}{}%
\end{pgfscope}%
\begin{pgfscope}%
\pgfsys@transformshift{1.042110in}{0.670309in}%
\pgfsys@useobject{currentmarker}{}%
\end{pgfscope}%
\begin{pgfscope}%
\pgfsys@transformshift{1.058384in}{0.686101in}%
\pgfsys@useobject{currentmarker}{}%
\end{pgfscope}%
\begin{pgfscope}%
\pgfsys@transformshift{1.544417in}{0.655238in}%
\pgfsys@useobject{currentmarker}{}%
\end{pgfscope}%
\begin{pgfscope}%
\pgfsys@transformshift{0.579911in}{1.454690in}%
\pgfsys@useobject{currentmarker}{}%
\end{pgfscope}%
\begin{pgfscope}%
\pgfsys@transformshift{1.077051in}{0.685075in}%
\pgfsys@useobject{currentmarker}{}%
\end{pgfscope}%
\begin{pgfscope}%
\pgfsys@transformshift{1.067556in}{0.662340in}%
\pgfsys@useobject{currentmarker}{}%
\end{pgfscope}%
\begin{pgfscope}%
\pgfsys@transformshift{1.079006in}{0.585113in}%
\pgfsys@useobject{currentmarker}{}%
\end{pgfscope}%
\begin{pgfscope}%
\pgfsys@transformshift{1.049097in}{0.703037in}%
\pgfsys@useobject{currentmarker}{}%
\end{pgfscope}%
\begin{pgfscope}%
\pgfsys@transformshift{0.623818in}{1.301979in}%
\pgfsys@useobject{currentmarker}{}%
\end{pgfscope}%
\begin{pgfscope}%
\pgfsys@transformshift{1.082780in}{0.584875in}%
\pgfsys@useobject{currentmarker}{}%
\end{pgfscope}%
\begin{pgfscope}%
\pgfsys@transformshift{1.049573in}{0.612631in}%
\pgfsys@useobject{currentmarker}{}%
\end{pgfscope}%
\begin{pgfscope}%
\pgfsys@transformshift{0.616382in}{1.318755in}%
\pgfsys@useobject{currentmarker}{}%
\end{pgfscope}%
\begin{pgfscope}%
\pgfsys@transformshift{0.625874in}{1.327043in}%
\pgfsys@useobject{currentmarker}{}%
\end{pgfscope}%
\begin{pgfscope}%
\pgfsys@transformshift{1.071418in}{0.606795in}%
\pgfsys@useobject{currentmarker}{}%
\end{pgfscope}%
\begin{pgfscope}%
\pgfsys@transformshift{1.065015in}{0.680566in}%
\pgfsys@useobject{currentmarker}{}%
\end{pgfscope}%
\begin{pgfscope}%
\pgfsys@transformshift{0.932919in}{0.517135in}%
\pgfsys@useobject{currentmarker}{}%
\end{pgfscope}%
\begin{pgfscope}%
\pgfsys@transformshift{0.598591in}{1.356896in}%
\pgfsys@useobject{currentmarker}{}%
\end{pgfscope}%
\begin{pgfscope}%
\pgfsys@transformshift{1.063639in}{0.758283in}%
\pgfsys@useobject{currentmarker}{}%
\end{pgfscope}%
\begin{pgfscope}%
\pgfsys@transformshift{1.071613in}{0.658104in}%
\pgfsys@useobject{currentmarker}{}%
\end{pgfscope}%
\begin{pgfscope}%
\pgfsys@transformshift{1.095839in}{0.590529in}%
\pgfsys@useobject{currentmarker}{}%
\end{pgfscope}%
\begin{pgfscope}%
\pgfsys@transformshift{1.049088in}{0.619638in}%
\pgfsys@useobject{currentmarker}{}%
\end{pgfscope}%
\begin{pgfscope}%
\pgfsys@transformshift{1.070748in}{0.661293in}%
\pgfsys@useobject{currentmarker}{}%
\end{pgfscope}%
\begin{pgfscope}%
\pgfsys@transformshift{0.620358in}{1.319338in}%
\pgfsys@useobject{currentmarker}{}%
\end{pgfscope}%
\begin{pgfscope}%
\pgfsys@transformshift{1.054073in}{0.653377in}%
\pgfsys@useobject{currentmarker}{}%
\end{pgfscope}%
\begin{pgfscope}%
\pgfsys@transformshift{1.058128in}{0.674871in}%
\pgfsys@useobject{currentmarker}{}%
\end{pgfscope}%
\begin{pgfscope}%
\pgfsys@transformshift{1.084037in}{0.598792in}%
\pgfsys@useobject{currentmarker}{}%
\end{pgfscope}%
\begin{pgfscope}%
\pgfsys@transformshift{1.020508in}{0.768066in}%
\pgfsys@useobject{currentmarker}{}%
\end{pgfscope}%
\begin{pgfscope}%
\pgfsys@transformshift{1.053814in}{0.687720in}%
\pgfsys@useobject{currentmarker}{}%
\end{pgfscope}%
\begin{pgfscope}%
\pgfsys@transformshift{1.068671in}{0.627655in}%
\pgfsys@useobject{currentmarker}{}%
\end{pgfscope}%
\begin{pgfscope}%
\pgfsys@transformshift{0.614564in}{1.341334in}%
\pgfsys@useobject{currentmarker}{}%
\end{pgfscope}%
\begin{pgfscope}%
\pgfsys@transformshift{0.599691in}{1.396636in}%
\pgfsys@useobject{currentmarker}{}%
\end{pgfscope}%
\begin{pgfscope}%
\pgfsys@transformshift{0.632646in}{1.300447in}%
\pgfsys@useobject{currentmarker}{}%
\end{pgfscope}%
\begin{pgfscope}%
\pgfsys@transformshift{0.628176in}{1.315279in}%
\pgfsys@useobject{currentmarker}{}%
\end{pgfscope}%
\begin{pgfscope}%
\pgfsys@transformshift{0.612347in}{1.319545in}%
\pgfsys@useobject{currentmarker}{}%
\end{pgfscope}%
\begin{pgfscope}%
\pgfsys@transformshift{1.070109in}{0.704176in}%
\pgfsys@useobject{currentmarker}{}%
\end{pgfscope}%
\end{pgfscope}%
\begin{pgfscope}%
\pgfsetrectcap%
\pgfsetmiterjoin%
\pgfsetlinewidth{0.000000pt}%
\definecolor{currentstroke}{rgb}{1.000000,1.000000,1.000000}%
\pgfsetstrokecolor{currentstroke}%
\pgfsetdash{}{0pt}%
\pgfpathmoveto{\pgfqpoint{0.423750in}{0.423750in}}%
\pgfpathlineto{\pgfqpoint{0.423750in}{1.587136in}}%
\pgfusepath{}%
\end{pgfscope}%
\begin{pgfscope}%
\pgfsetrectcap%
\pgfsetmiterjoin%
\pgfsetlinewidth{0.000000pt}%
\definecolor{currentstroke}{rgb}{1.000000,1.000000,1.000000}%
\pgfsetstrokecolor{currentstroke}%
\pgfsetdash{}{0pt}%
\pgfpathmoveto{\pgfqpoint{1.617955in}{0.423750in}}%
\pgfpathlineto{\pgfqpoint{1.617955in}{1.587136in}}%
\pgfusepath{}%
\end{pgfscope}%
\begin{pgfscope}%
\pgfsetrectcap%
\pgfsetmiterjoin%
\pgfsetlinewidth{0.000000pt}%
\definecolor{currentstroke}{rgb}{1.000000,1.000000,1.000000}%
\pgfsetstrokecolor{currentstroke}%
\pgfsetdash{}{0pt}%
\pgfpathmoveto{\pgfqpoint{0.423750in}{0.423750in}}%
\pgfpathlineto{\pgfqpoint{1.617955in}{0.423750in}}%
\pgfusepath{}%
\end{pgfscope}%
\begin{pgfscope}%
\pgfsetrectcap%
\pgfsetmiterjoin%
\pgfsetlinewidth{0.000000pt}%
\definecolor{currentstroke}{rgb}{1.000000,1.000000,1.000000}%
\pgfsetstrokecolor{currentstroke}%
\pgfsetdash{}{0pt}%
\pgfpathmoveto{\pgfqpoint{0.423750in}{1.587136in}}%
\pgfpathlineto{\pgfqpoint{1.617955in}{1.587136in}}%
\pgfusepath{}%
\end{pgfscope}%
\begin{pgfscope}%
\pgfsetbuttcap%
\pgfsetmiterjoin%
\definecolor{currentfill}{rgb}{0.917647,0.917647,0.949020}%
\pgfsetfillcolor{currentfill}%
\pgfsetlinewidth{0.000000pt}%
\definecolor{currentstroke}{rgb}{0.000000,0.000000,0.000000}%
\pgfsetstrokecolor{currentstroke}%
\pgfsetstrokeopacity{0.000000}%
\pgfsetdash{}{0pt}%
\pgfpathmoveto{\pgfqpoint{1.856795in}{0.423750in}}%
\pgfpathlineto{\pgfqpoint{3.051000in}{0.423750in}}%
\pgfpathlineto{\pgfqpoint{3.051000in}{1.587136in}}%
\pgfpathlineto{\pgfqpoint{1.856795in}{1.587136in}}%
\pgfpathclose%
\pgfusepath{fill}%
\end{pgfscope}%
\begin{pgfscope}%
\pgfpathrectangle{\pgfqpoint{1.856795in}{0.423750in}}{\pgfqpoint{1.194205in}{1.163386in}}%
\pgfusepath{clip}%
\pgfsetroundcap%
\pgfsetroundjoin%
\pgfsetlinewidth{0.803000pt}%
\definecolor{currentstroke}{rgb}{1.000000,1.000000,1.000000}%
\pgfsetstrokecolor{currentstroke}%
\pgfsetdash{}{0pt}%
\pgfpathmoveto{\pgfqpoint{2.229062in}{0.423750in}}%
\pgfpathlineto{\pgfqpoint{2.229062in}{1.587136in}}%
\pgfusepath{stroke}%
\end{pgfscope}%
\begin{pgfscope}%
\definecolor{textcolor}{rgb}{0.150000,0.150000,0.150000}%
\pgfsetstrokecolor{textcolor}%
\pgfsetfillcolor{textcolor}%
\pgftext[x=2.229062in,y=0.375139in,,top]{\color{textcolor}\rmfamily\fontsize{8.000000}{9.600000}\selectfont \(\displaystyle 0\)}%
\end{pgfscope}%
\begin{pgfscope}%
\pgfpathrectangle{\pgfqpoint{1.856795in}{0.423750in}}{\pgfqpoint{1.194205in}{1.163386in}}%
\pgfusepath{clip}%
\pgfsetroundcap%
\pgfsetroundjoin%
\pgfsetlinewidth{0.803000pt}%
\definecolor{currentstroke}{rgb}{1.000000,1.000000,1.000000}%
\pgfsetstrokecolor{currentstroke}%
\pgfsetdash{}{0pt}%
\pgfpathmoveto{\pgfqpoint{2.973596in}{0.423750in}}%
\pgfpathlineto{\pgfqpoint{2.973596in}{1.587136in}}%
\pgfusepath{stroke}%
\end{pgfscope}%
\begin{pgfscope}%
\definecolor{textcolor}{rgb}{0.150000,0.150000,0.150000}%
\pgfsetstrokecolor{textcolor}%
\pgfsetfillcolor{textcolor}%
\pgftext[x=2.973596in,y=0.375139in,,top]{\color{textcolor}\rmfamily\fontsize{8.000000}{9.600000}\selectfont \(\displaystyle 10\)}%
\end{pgfscope}%
\begin{pgfscope}%
\pgfpathrectangle{\pgfqpoint{1.856795in}{0.423750in}}{\pgfqpoint{1.194205in}{1.163386in}}%
\pgfusepath{clip}%
\pgfsetroundcap%
\pgfsetroundjoin%
\pgfsetlinewidth{0.803000pt}%
\definecolor{currentstroke}{rgb}{1.000000,1.000000,1.000000}%
\pgfsetstrokecolor{currentstroke}%
\pgfsetdash{}{0pt}%
\pgfpathmoveto{\pgfqpoint{1.856795in}{0.491252in}}%
\pgfpathlineto{\pgfqpoint{3.051000in}{0.491252in}}%
\pgfusepath{stroke}%
\end{pgfscope}%
\begin{pgfscope}%
\definecolor{textcolor}{rgb}{0.150000,0.150000,0.150000}%
\pgfsetstrokecolor{textcolor}%
\pgfsetfillcolor{textcolor}%
\pgftext[x=1.749156in,y=0.449043in,left,base]{\color{textcolor}\rmfamily\fontsize{8.000000}{9.600000}\selectfont \(\displaystyle 0\)}%
\end{pgfscope}%
\begin{pgfscope}%
\pgfpathrectangle{\pgfqpoint{1.856795in}{0.423750in}}{\pgfqpoint{1.194205in}{1.163386in}}%
\pgfusepath{clip}%
\pgfsetroundcap%
\pgfsetroundjoin%
\pgfsetlinewidth{0.803000pt}%
\definecolor{currentstroke}{rgb}{1.000000,1.000000,1.000000}%
\pgfsetstrokecolor{currentstroke}%
\pgfsetdash{}{0pt}%
\pgfpathmoveto{\pgfqpoint{1.856795in}{0.856547in}}%
\pgfpathlineto{\pgfqpoint{3.051000in}{0.856547in}}%
\pgfusepath{stroke}%
\end{pgfscope}%
\begin{pgfscope}%
\definecolor{textcolor}{rgb}{0.150000,0.150000,0.150000}%
\pgfsetstrokecolor{textcolor}%
\pgfsetfillcolor{textcolor}%
\pgftext[x=1.749156in,y=0.814337in,left,base]{\color{textcolor}\rmfamily\fontsize{8.000000}{9.600000}\selectfont \(\displaystyle 5\)}%
\end{pgfscope}%
\begin{pgfscope}%
\pgfpathrectangle{\pgfqpoint{1.856795in}{0.423750in}}{\pgfqpoint{1.194205in}{1.163386in}}%
\pgfusepath{clip}%
\pgfsetroundcap%
\pgfsetroundjoin%
\pgfsetlinewidth{0.803000pt}%
\definecolor{currentstroke}{rgb}{1.000000,1.000000,1.000000}%
\pgfsetstrokecolor{currentstroke}%
\pgfsetdash{}{0pt}%
\pgfpathmoveto{\pgfqpoint{1.856795in}{1.221842in}}%
\pgfpathlineto{\pgfqpoint{3.051000in}{1.221842in}}%
\pgfusepath{stroke}%
\end{pgfscope}%
\begin{pgfscope}%
\definecolor{textcolor}{rgb}{0.150000,0.150000,0.150000}%
\pgfsetstrokecolor{textcolor}%
\pgfsetfillcolor{textcolor}%
\pgftext[x=1.690127in,y=1.179632in,left,base]{\color{textcolor}\rmfamily\fontsize{8.000000}{9.600000}\selectfont \(\displaystyle 10\)}%
\end{pgfscope}%
\begin{pgfscope}%
\pgfpathrectangle{\pgfqpoint{1.856795in}{0.423750in}}{\pgfqpoint{1.194205in}{1.163386in}}%
\pgfusepath{clip}%
\pgfsetroundcap%
\pgfsetroundjoin%
\pgfsetlinewidth{0.803000pt}%
\definecolor{currentstroke}{rgb}{1.000000,1.000000,1.000000}%
\pgfsetstrokecolor{currentstroke}%
\pgfsetdash{}{0pt}%
\pgfpathmoveto{\pgfqpoint{1.856795in}{1.587136in}}%
\pgfpathlineto{\pgfqpoint{3.051000in}{1.587136in}}%
\pgfusepath{stroke}%
\end{pgfscope}%
\begin{pgfscope}%
\definecolor{textcolor}{rgb}{0.150000,0.150000,0.150000}%
\pgfsetstrokecolor{textcolor}%
\pgfsetfillcolor{textcolor}%
\pgftext[x=1.690127in,y=1.544927in,left,base]{\color{textcolor}\rmfamily\fontsize{8.000000}{9.600000}\selectfont \(\displaystyle 15\)}%
\end{pgfscope}%
\begin{pgfscope}%
\pgfpathrectangle{\pgfqpoint{1.856795in}{0.423750in}}{\pgfqpoint{1.194205in}{1.163386in}}%
\pgfusepath{clip}%
\pgfsetbuttcap%
\pgfsetroundjoin%
\definecolor{currentfill}{rgb}{0.032852,0.030888,0.118630}%
\pgfsetfillcolor{currentfill}%
\pgfsetlinewidth{0.000000pt}%
\definecolor{currentstroke}{rgb}{0.000000,0.000000,0.000000}%
\pgfsetstrokecolor{currentstroke}%
\pgfsetdash{}{0pt}%
\pgfpathmoveto{\pgfqpoint{1.868076in}{0.491252in}}%
\pgfpathlineto{\pgfqpoint{1.879357in}{0.491252in}}%
\pgfpathlineto{\pgfqpoint{1.890638in}{0.491252in}}%
\pgfpathlineto{\pgfqpoint{1.901919in}{0.491252in}}%
\pgfpathlineto{\pgfqpoint{1.913200in}{0.491252in}}%
\pgfpathlineto{\pgfqpoint{1.913445in}{0.491252in}}%
\pgfpathlineto{\pgfqpoint{1.913200in}{0.491467in}}%
\pgfpathlineto{\pgfqpoint{1.901919in}{0.501258in}}%
\pgfpathlineto{\pgfqpoint{1.900694in}{0.502321in}}%
\pgfpathlineto{\pgfqpoint{1.890638in}{0.510247in}}%
\pgfpathlineto{\pgfqpoint{1.886635in}{0.513391in}}%
\pgfpathlineto{\pgfqpoint{1.879357in}{0.518920in}}%
\pgfpathlineto{\pgfqpoint{1.871909in}{0.524460in}}%
\pgfpathlineto{\pgfqpoint{1.868076in}{0.527250in}}%
\pgfpathlineto{\pgfqpoint{1.856795in}{0.535258in}}%
\pgfpathlineto{\pgfqpoint{1.856795in}{0.524460in}}%
\pgfpathlineto{\pgfqpoint{1.856795in}{0.513391in}}%
\pgfpathlineto{\pgfqpoint{1.856795in}{0.502321in}}%
\pgfpathlineto{\pgfqpoint{1.856795in}{0.491252in}}%
\pgfpathclose%
\pgfusepath{fill}%
\end{pgfscope}%
\begin{pgfscope}%
\pgfpathrectangle{\pgfqpoint{1.856795in}{0.423750in}}{\pgfqpoint{1.194205in}{1.163386in}}%
\pgfusepath{clip}%
\pgfsetbuttcap%
\pgfsetroundjoin%
\definecolor{currentfill}{rgb}{0.101323,0.062868,0.166043}%
\pgfsetfillcolor{currentfill}%
\pgfsetlinewidth{0.000000pt}%
\definecolor{currentstroke}{rgb}{0.000000,0.000000,0.000000}%
\pgfsetstrokecolor{currentstroke}%
\pgfsetdash{}{0pt}%
\pgfpathmoveto{\pgfqpoint{1.901919in}{0.501258in}}%
\pgfpathlineto{\pgfqpoint{1.913200in}{0.491467in}}%
\pgfpathlineto{\pgfqpoint{1.913445in}{0.491252in}}%
\pgfpathlineto{\pgfqpoint{1.924480in}{0.491252in}}%
\pgfpathlineto{\pgfqpoint{1.935761in}{0.491252in}}%
\pgfpathlineto{\pgfqpoint{1.947042in}{0.491252in}}%
\pgfpathlineto{\pgfqpoint{1.958323in}{0.491252in}}%
\pgfpathlineto{\pgfqpoint{1.965015in}{0.491252in}}%
\pgfpathlineto{\pgfqpoint{1.958323in}{0.497204in}}%
\pgfpathlineto{\pgfqpoint{1.952571in}{0.502321in}}%
\pgfpathlineto{\pgfqpoint{1.947042in}{0.507112in}}%
\pgfpathlineto{\pgfqpoint{1.939741in}{0.513391in}}%
\pgfpathlineto{\pgfqpoint{1.935761in}{0.516711in}}%
\pgfpathlineto{\pgfqpoint{1.926342in}{0.524460in}}%
\pgfpathlineto{\pgfqpoint{1.924480in}{0.525958in}}%
\pgfpathlineto{\pgfqpoint{1.913200in}{0.534980in}}%
\pgfpathlineto{\pgfqpoint{1.912508in}{0.535530in}}%
\pgfpathlineto{\pgfqpoint{1.901919in}{0.543892in}}%
\pgfpathlineto{\pgfqpoint{1.898470in}{0.546600in}}%
\pgfpathlineto{\pgfqpoint{1.890638in}{0.552703in}}%
\pgfpathlineto{\pgfqpoint{1.884195in}{0.557669in}}%
\pgfpathlineto{\pgfqpoint{1.879357in}{0.561378in}}%
\pgfpathlineto{\pgfqpoint{1.869628in}{0.568739in}}%
\pgfpathlineto{\pgfqpoint{1.868076in}{0.569907in}}%
\pgfpathlineto{\pgfqpoint{1.856795in}{0.578245in}}%
\pgfpathlineto{\pgfqpoint{1.856795in}{0.568739in}}%
\pgfpathlineto{\pgfqpoint{1.856795in}{0.557669in}}%
\pgfpathlineto{\pgfqpoint{1.856795in}{0.546600in}}%
\pgfpathlineto{\pgfqpoint{1.856795in}{0.535530in}}%
\pgfpathlineto{\pgfqpoint{1.856795in}{0.535258in}}%
\pgfpathlineto{\pgfqpoint{1.868076in}{0.527250in}}%
\pgfpathlineto{\pgfqpoint{1.871909in}{0.524460in}}%
\pgfpathlineto{\pgfqpoint{1.879357in}{0.518920in}}%
\pgfpathlineto{\pgfqpoint{1.886635in}{0.513391in}}%
\pgfpathlineto{\pgfqpoint{1.890638in}{0.510247in}}%
\pgfpathlineto{\pgfqpoint{1.900694in}{0.502321in}}%
\pgfpathclose%
\pgfusepath{fill}%
\end{pgfscope}%
\begin{pgfscope}%
\pgfpathrectangle{\pgfqpoint{1.856795in}{0.423750in}}{\pgfqpoint{1.194205in}{1.163386in}}%
\pgfusepath{clip}%
\pgfsetbuttcap%
\pgfsetroundjoin%
\definecolor{currentfill}{rgb}{0.169154,0.086830,0.214981}%
\pgfsetfillcolor{currentfill}%
\pgfsetlinewidth{0.000000pt}%
\definecolor{currentstroke}{rgb}{0.000000,0.000000,0.000000}%
\pgfsetstrokecolor{currentstroke}%
\pgfsetdash{}{0pt}%
\pgfpathmoveto{\pgfqpoint{1.958323in}{0.497204in}}%
\pgfpathlineto{\pgfqpoint{1.965015in}{0.491252in}}%
\pgfpathlineto{\pgfqpoint{1.969604in}{0.491252in}}%
\pgfpathlineto{\pgfqpoint{1.980884in}{0.491252in}}%
\pgfpathlineto{\pgfqpoint{1.992165in}{0.491252in}}%
\pgfpathlineto{\pgfqpoint{2.003446in}{0.491252in}}%
\pgfpathlineto{\pgfqpoint{2.008165in}{0.491252in}}%
\pgfpathlineto{\pgfqpoint{2.003446in}{0.495763in}}%
\pgfpathlineto{\pgfqpoint{1.996501in}{0.502321in}}%
\pgfpathlineto{\pgfqpoint{1.992165in}{0.506202in}}%
\pgfpathlineto{\pgfqpoint{1.984029in}{0.513391in}}%
\pgfpathlineto{\pgfqpoint{1.980884in}{0.516132in}}%
\pgfpathlineto{\pgfqpoint{1.971331in}{0.524460in}}%
\pgfpathlineto{\pgfqpoint{1.969604in}{0.525937in}}%
\pgfpathlineto{\pgfqpoint{1.958323in}{0.535420in}}%
\pgfpathlineto{\pgfqpoint{1.958191in}{0.535530in}}%
\pgfpathlineto{\pgfqpoint{1.947042in}{0.544658in}}%
\pgfpathlineto{\pgfqpoint{1.944656in}{0.546600in}}%
\pgfpathlineto{\pgfqpoint{1.935761in}{0.553799in}}%
\pgfpathlineto{\pgfqpoint{1.930956in}{0.557669in}}%
\pgfpathlineto{\pgfqpoint{1.924480in}{0.562870in}}%
\pgfpathlineto{\pgfqpoint{1.917178in}{0.568739in}}%
\pgfpathlineto{\pgfqpoint{1.913200in}{0.571930in}}%
\pgfpathlineto{\pgfqpoint{1.903304in}{0.579808in}}%
\pgfpathlineto{\pgfqpoint{1.901919in}{0.580907in}}%
\pgfpathlineto{\pgfqpoint{1.890638in}{0.589750in}}%
\pgfpathlineto{\pgfqpoint{1.889181in}{0.590878in}}%
\pgfpathlineto{\pgfqpoint{1.879357in}{0.598458in}}%
\pgfpathlineto{\pgfqpoint{1.874783in}{0.601947in}}%
\pgfpathlineto{\pgfqpoint{1.868076in}{0.607049in}}%
\pgfpathlineto{\pgfqpoint{1.860137in}{0.613017in}}%
\pgfpathlineto{\pgfqpoint{1.856795in}{0.615520in}}%
\pgfpathlineto{\pgfqpoint{1.856795in}{0.613017in}}%
\pgfpathlineto{\pgfqpoint{1.856795in}{0.601947in}}%
\pgfpathlineto{\pgfqpoint{1.856795in}{0.590878in}}%
\pgfpathlineto{\pgfqpoint{1.856795in}{0.579808in}}%
\pgfpathlineto{\pgfqpoint{1.856795in}{0.578245in}}%
\pgfpathlineto{\pgfqpoint{1.868076in}{0.569907in}}%
\pgfpathlineto{\pgfqpoint{1.869628in}{0.568739in}}%
\pgfpathlineto{\pgfqpoint{1.879357in}{0.561378in}}%
\pgfpathlineto{\pgfqpoint{1.884195in}{0.557669in}}%
\pgfpathlineto{\pgfqpoint{1.890638in}{0.552703in}}%
\pgfpathlineto{\pgfqpoint{1.898470in}{0.546600in}}%
\pgfpathlineto{\pgfqpoint{1.901919in}{0.543892in}}%
\pgfpathlineto{\pgfqpoint{1.912508in}{0.535530in}}%
\pgfpathlineto{\pgfqpoint{1.913200in}{0.534980in}}%
\pgfpathlineto{\pgfqpoint{1.924480in}{0.525958in}}%
\pgfpathlineto{\pgfqpoint{1.926342in}{0.524460in}}%
\pgfpathlineto{\pgfqpoint{1.935761in}{0.516711in}}%
\pgfpathlineto{\pgfqpoint{1.939741in}{0.513391in}}%
\pgfpathlineto{\pgfqpoint{1.947042in}{0.507112in}}%
\pgfpathlineto{\pgfqpoint{1.952571in}{0.502321in}}%
\pgfpathclose%
\pgfusepath{fill}%
\end{pgfscope}%
\begin{pgfscope}%
\pgfpathrectangle{\pgfqpoint{1.856795in}{0.423750in}}{\pgfqpoint{1.194205in}{1.163386in}}%
\pgfusepath{clip}%
\pgfsetbuttcap%
\pgfsetroundjoin%
\definecolor{currentfill}{rgb}{0.233340,0.102637,0.256977}%
\pgfsetfillcolor{currentfill}%
\pgfsetlinewidth{0.000000pt}%
\definecolor{currentstroke}{rgb}{0.000000,0.000000,0.000000}%
\pgfsetstrokecolor{currentstroke}%
\pgfsetdash{}{0pt}%
\pgfpathmoveto{\pgfqpoint{2.003446in}{0.495763in}}%
\pgfpathlineto{\pgfqpoint{2.008165in}{0.491252in}}%
\pgfpathlineto{\pgfqpoint{2.014727in}{0.491252in}}%
\pgfpathlineto{\pgfqpoint{2.026008in}{0.491252in}}%
\pgfpathlineto{\pgfqpoint{2.037289in}{0.491252in}}%
\pgfpathlineto{\pgfqpoint{2.047454in}{0.491252in}}%
\pgfpathlineto{\pgfqpoint{2.037289in}{0.500712in}}%
\pgfpathlineto{\pgfqpoint{2.035604in}{0.502321in}}%
\pgfpathlineto{\pgfqpoint{2.026008in}{0.511267in}}%
\pgfpathlineto{\pgfqpoint{2.023738in}{0.513391in}}%
\pgfpathlineto{\pgfqpoint{2.014727in}{0.521512in}}%
\pgfpathlineto{\pgfqpoint{2.011400in}{0.524460in}}%
\pgfpathlineto{\pgfqpoint{2.003446in}{0.531433in}}%
\pgfpathlineto{\pgfqpoint{1.998756in}{0.535530in}}%
\pgfpathlineto{\pgfqpoint{1.992165in}{0.541200in}}%
\pgfpathlineto{\pgfqpoint{1.985825in}{0.546600in}}%
\pgfpathlineto{\pgfqpoint{1.980884in}{0.550767in}}%
\pgfpathlineto{\pgfqpoint{1.972625in}{0.557669in}}%
\pgfpathlineto{\pgfqpoint{1.969604in}{0.560180in}}%
\pgfpathlineto{\pgfqpoint{1.959253in}{0.568739in}}%
\pgfpathlineto{\pgfqpoint{1.958323in}{0.569506in}}%
\pgfpathlineto{\pgfqpoint{1.947042in}{0.578745in}}%
\pgfpathlineto{\pgfqpoint{1.945737in}{0.579808in}}%
\pgfpathlineto{\pgfqpoint{1.935761in}{0.587936in}}%
\pgfpathlineto{\pgfqpoint{1.932145in}{0.590878in}}%
\pgfpathlineto{\pgfqpoint{1.924480in}{0.597111in}}%
\pgfpathlineto{\pgfqpoint{1.918496in}{0.601947in}}%
\pgfpathlineto{\pgfqpoint{1.913200in}{0.606223in}}%
\pgfpathlineto{\pgfqpoint{1.904699in}{0.613017in}}%
\pgfpathlineto{\pgfqpoint{1.901919in}{0.615236in}}%
\pgfpathlineto{\pgfqpoint{1.890716in}{0.624086in}}%
\pgfpathlineto{\pgfqpoint{1.890638in}{0.624148in}}%
\pgfpathlineto{\pgfqpoint{1.879357in}{0.632911in}}%
\pgfpathlineto{\pgfqpoint{1.876432in}{0.635156in}}%
\pgfpathlineto{\pgfqpoint{1.868076in}{0.641552in}}%
\pgfpathlineto{\pgfqpoint{1.861910in}{0.646225in}}%
\pgfpathlineto{\pgfqpoint{1.856795in}{0.650086in}}%
\pgfpathlineto{\pgfqpoint{1.856795in}{0.646225in}}%
\pgfpathlineto{\pgfqpoint{1.856795in}{0.635156in}}%
\pgfpathlineto{\pgfqpoint{1.856795in}{0.624086in}}%
\pgfpathlineto{\pgfqpoint{1.856795in}{0.615520in}}%
\pgfpathlineto{\pgfqpoint{1.860137in}{0.613017in}}%
\pgfpathlineto{\pgfqpoint{1.868076in}{0.607049in}}%
\pgfpathlineto{\pgfqpoint{1.874783in}{0.601947in}}%
\pgfpathlineto{\pgfqpoint{1.879357in}{0.598458in}}%
\pgfpathlineto{\pgfqpoint{1.889181in}{0.590878in}}%
\pgfpathlineto{\pgfqpoint{1.890638in}{0.589750in}}%
\pgfpathlineto{\pgfqpoint{1.901919in}{0.580907in}}%
\pgfpathlineto{\pgfqpoint{1.903304in}{0.579808in}}%
\pgfpathlineto{\pgfqpoint{1.913200in}{0.571930in}}%
\pgfpathlineto{\pgfqpoint{1.917178in}{0.568739in}}%
\pgfpathlineto{\pgfqpoint{1.924480in}{0.562870in}}%
\pgfpathlineto{\pgfqpoint{1.930956in}{0.557669in}}%
\pgfpathlineto{\pgfqpoint{1.935761in}{0.553799in}}%
\pgfpathlineto{\pgfqpoint{1.944656in}{0.546600in}}%
\pgfpathlineto{\pgfqpoint{1.947042in}{0.544658in}}%
\pgfpathlineto{\pgfqpoint{1.958191in}{0.535530in}}%
\pgfpathlineto{\pgfqpoint{1.958323in}{0.535420in}}%
\pgfpathlineto{\pgfqpoint{1.969604in}{0.525937in}}%
\pgfpathlineto{\pgfqpoint{1.971331in}{0.524460in}}%
\pgfpathlineto{\pgfqpoint{1.980884in}{0.516132in}}%
\pgfpathlineto{\pgfqpoint{1.984029in}{0.513391in}}%
\pgfpathlineto{\pgfqpoint{1.992165in}{0.506202in}}%
\pgfpathlineto{\pgfqpoint{1.996501in}{0.502321in}}%
\pgfpathclose%
\pgfusepath{fill}%
\end{pgfscope}%
\begin{pgfscope}%
\pgfpathrectangle{\pgfqpoint{1.856795in}{0.423750in}}{\pgfqpoint{1.194205in}{1.163386in}}%
\pgfusepath{clip}%
\pgfsetbuttcap%
\pgfsetroundjoin%
\definecolor{currentfill}{rgb}{0.305922,0.114326,0.295427}%
\pgfsetfillcolor{currentfill}%
\pgfsetlinewidth{0.000000pt}%
\definecolor{currentstroke}{rgb}{0.000000,0.000000,0.000000}%
\pgfsetstrokecolor{currentstroke}%
\pgfsetdash{}{0pt}%
\pgfpathmoveto{\pgfqpoint{2.037289in}{0.500712in}}%
\pgfpathlineto{\pgfqpoint{2.047454in}{0.491252in}}%
\pgfpathlineto{\pgfqpoint{2.048569in}{0.491252in}}%
\pgfpathlineto{\pgfqpoint{2.059850in}{0.491252in}}%
\pgfpathlineto{\pgfqpoint{2.071131in}{0.491252in}}%
\pgfpathlineto{\pgfqpoint{2.082412in}{0.491252in}}%
\pgfpathlineto{\pgfqpoint{2.084462in}{0.491252in}}%
\pgfpathlineto{\pgfqpoint{2.082412in}{0.493301in}}%
\pgfpathlineto{\pgfqpoint{2.073143in}{0.502321in}}%
\pgfpathlineto{\pgfqpoint{2.071131in}{0.504192in}}%
\pgfpathlineto{\pgfqpoint{2.061133in}{0.513391in}}%
\pgfpathlineto{\pgfqpoint{2.059850in}{0.514557in}}%
\pgfpathlineto{\pgfqpoint{2.048911in}{0.524460in}}%
\pgfpathlineto{\pgfqpoint{2.048569in}{0.524763in}}%
\pgfpathlineto{\pgfqpoint{2.037289in}{0.534535in}}%
\pgfpathlineto{\pgfqpoint{2.036153in}{0.535530in}}%
\pgfpathlineto{\pgfqpoint{2.026008in}{0.544360in}}%
\pgfpathlineto{\pgfqpoint{2.023463in}{0.546600in}}%
\pgfpathlineto{\pgfqpoint{2.014727in}{0.554151in}}%
\pgfpathlineto{\pgfqpoint{2.010640in}{0.557669in}}%
\pgfpathlineto{\pgfqpoint{2.003446in}{0.563857in}}%
\pgfpathlineto{\pgfqpoint{1.997710in}{0.568739in}}%
\pgfpathlineto{\pgfqpoint{1.992165in}{0.573425in}}%
\pgfpathlineto{\pgfqpoint{1.984576in}{0.579808in}}%
\pgfpathlineto{\pgfqpoint{1.980884in}{0.582911in}}%
\pgfpathlineto{\pgfqpoint{1.971353in}{0.590878in}}%
\pgfpathlineto{\pgfqpoint{1.969604in}{0.592337in}}%
\pgfpathlineto{\pgfqpoint{1.958323in}{0.601702in}}%
\pgfpathlineto{\pgfqpoint{1.958026in}{0.601947in}}%
\pgfpathlineto{\pgfqpoint{1.947042in}{0.611034in}}%
\pgfpathlineto{\pgfqpoint{1.944642in}{0.613017in}}%
\pgfpathlineto{\pgfqpoint{1.935761in}{0.620327in}}%
\pgfpathlineto{\pgfqpoint{1.931172in}{0.624086in}}%
\pgfpathlineto{\pgfqpoint{1.924480in}{0.629562in}}%
\pgfpathlineto{\pgfqpoint{1.917581in}{0.635156in}}%
\pgfpathlineto{\pgfqpoint{1.913200in}{0.638709in}}%
\pgfpathlineto{\pgfqpoint{1.903829in}{0.646225in}}%
\pgfpathlineto{\pgfqpoint{1.901919in}{0.647759in}}%
\pgfpathlineto{\pgfqpoint{1.890638in}{0.656709in}}%
\pgfpathlineto{\pgfqpoint{1.889889in}{0.657295in}}%
\pgfpathlineto{\pgfqpoint{1.879357in}{0.665531in}}%
\pgfpathlineto{\pgfqpoint{1.875687in}{0.668364in}}%
\pgfpathlineto{\pgfqpoint{1.868076in}{0.674238in}}%
\pgfpathlineto{\pgfqpoint{1.861257in}{0.679434in}}%
\pgfpathlineto{\pgfqpoint{1.856795in}{0.682838in}}%
\pgfpathlineto{\pgfqpoint{1.856795in}{0.679434in}}%
\pgfpathlineto{\pgfqpoint{1.856795in}{0.668364in}}%
\pgfpathlineto{\pgfqpoint{1.856795in}{0.657295in}}%
\pgfpathlineto{\pgfqpoint{1.856795in}{0.650086in}}%
\pgfpathlineto{\pgfqpoint{1.861910in}{0.646225in}}%
\pgfpathlineto{\pgfqpoint{1.868076in}{0.641552in}}%
\pgfpathlineto{\pgfqpoint{1.876432in}{0.635156in}}%
\pgfpathlineto{\pgfqpoint{1.879357in}{0.632911in}}%
\pgfpathlineto{\pgfqpoint{1.890638in}{0.624148in}}%
\pgfpathlineto{\pgfqpoint{1.890716in}{0.624086in}}%
\pgfpathlineto{\pgfqpoint{1.901919in}{0.615236in}}%
\pgfpathlineto{\pgfqpoint{1.904699in}{0.613017in}}%
\pgfpathlineto{\pgfqpoint{1.913200in}{0.606223in}}%
\pgfpathlineto{\pgfqpoint{1.918496in}{0.601947in}}%
\pgfpathlineto{\pgfqpoint{1.924480in}{0.597111in}}%
\pgfpathlineto{\pgfqpoint{1.932145in}{0.590878in}}%
\pgfpathlineto{\pgfqpoint{1.935761in}{0.587936in}}%
\pgfpathlineto{\pgfqpoint{1.945737in}{0.579808in}}%
\pgfpathlineto{\pgfqpoint{1.947042in}{0.578745in}}%
\pgfpathlineto{\pgfqpoint{1.958323in}{0.569506in}}%
\pgfpathlineto{\pgfqpoint{1.959253in}{0.568739in}}%
\pgfpathlineto{\pgfqpoint{1.969604in}{0.560180in}}%
\pgfpathlineto{\pgfqpoint{1.972625in}{0.557669in}}%
\pgfpathlineto{\pgfqpoint{1.980884in}{0.550767in}}%
\pgfpathlineto{\pgfqpoint{1.985825in}{0.546600in}}%
\pgfpathlineto{\pgfqpoint{1.992165in}{0.541200in}}%
\pgfpathlineto{\pgfqpoint{1.998756in}{0.535530in}}%
\pgfpathlineto{\pgfqpoint{2.003446in}{0.531433in}}%
\pgfpathlineto{\pgfqpoint{2.011400in}{0.524460in}}%
\pgfpathlineto{\pgfqpoint{2.014727in}{0.521512in}}%
\pgfpathlineto{\pgfqpoint{2.023738in}{0.513391in}}%
\pgfpathlineto{\pgfqpoint{2.026008in}{0.511267in}}%
\pgfpathlineto{\pgfqpoint{2.035604in}{0.502321in}}%
\pgfpathclose%
\pgfusepath{fill}%
\end{pgfscope}%
\begin{pgfscope}%
\pgfpathrectangle{\pgfqpoint{1.856795in}{0.423750in}}{\pgfqpoint{1.194205in}{1.163386in}}%
\pgfusepath{clip}%
\pgfsetbuttcap%
\pgfsetroundjoin%
\definecolor{currentfill}{rgb}{0.380929,0.120615,0.325065}%
\pgfsetfillcolor{currentfill}%
\pgfsetlinewidth{0.000000pt}%
\definecolor{currentstroke}{rgb}{0.000000,0.000000,0.000000}%
\pgfsetstrokecolor{currentstroke}%
\pgfsetdash{}{0pt}%
\pgfpathmoveto{\pgfqpoint{2.082412in}{0.493301in}}%
\pgfpathlineto{\pgfqpoint{2.084462in}{0.491252in}}%
\pgfpathlineto{\pgfqpoint{2.093693in}{0.491252in}}%
\pgfpathlineto{\pgfqpoint{2.104973in}{0.491252in}}%
\pgfpathlineto{\pgfqpoint{2.116254in}{0.491252in}}%
\pgfpathlineto{\pgfqpoint{2.118291in}{0.491252in}}%
\pgfpathlineto{\pgfqpoint{2.116254in}{0.493391in}}%
\pgfpathlineto{\pgfqpoint{2.107612in}{0.502321in}}%
\pgfpathlineto{\pgfqpoint{2.104973in}{0.504984in}}%
\pgfpathlineto{\pgfqpoint{2.096479in}{0.513391in}}%
\pgfpathlineto{\pgfqpoint{2.093693in}{0.516024in}}%
\pgfpathlineto{\pgfqpoint{2.084527in}{0.524460in}}%
\pgfpathlineto{\pgfqpoint{2.082412in}{0.526401in}}%
\pgfpathlineto{\pgfqpoint{2.072424in}{0.535530in}}%
\pgfpathlineto{\pgfqpoint{2.071131in}{0.536682in}}%
\pgfpathlineto{\pgfqpoint{2.059850in}{0.546512in}}%
\pgfpathlineto{\pgfqpoint{2.059748in}{0.546600in}}%
\pgfpathlineto{\pgfqpoint{2.048569in}{0.556156in}}%
\pgfpathlineto{\pgfqpoint{2.046780in}{0.557669in}}%
\pgfpathlineto{\pgfqpoint{2.037289in}{0.565491in}}%
\pgfpathlineto{\pgfqpoint{2.033465in}{0.568739in}}%
\pgfpathlineto{\pgfqpoint{2.026008in}{0.575068in}}%
\pgfpathlineto{\pgfqpoint{2.020524in}{0.579808in}}%
\pgfpathlineto{\pgfqpoint{2.014727in}{0.584791in}}%
\pgfpathlineto{\pgfqpoint{2.007590in}{0.590878in}}%
\pgfpathlineto{\pgfqpoint{2.003446in}{0.594411in}}%
\pgfpathlineto{\pgfqpoint{1.994562in}{0.601947in}}%
\pgfpathlineto{\pgfqpoint{1.992165in}{0.603982in}}%
\pgfpathlineto{\pgfqpoint{1.981461in}{0.613017in}}%
\pgfpathlineto{\pgfqpoint{1.980884in}{0.613505in}}%
\pgfpathlineto{\pgfqpoint{1.969604in}{0.622985in}}%
\pgfpathlineto{\pgfqpoint{1.968290in}{0.624086in}}%
\pgfpathlineto{\pgfqpoint{1.958323in}{0.632471in}}%
\pgfpathlineto{\pgfqpoint{1.955108in}{0.635156in}}%
\pgfpathlineto{\pgfqpoint{1.947042in}{0.641914in}}%
\pgfpathlineto{\pgfqpoint{1.941866in}{0.646225in}}%
\pgfpathlineto{\pgfqpoint{1.935761in}{0.651316in}}%
\pgfpathlineto{\pgfqpoint{1.928519in}{0.657295in}}%
\pgfpathlineto{\pgfqpoint{1.924480in}{0.660633in}}%
\pgfpathlineto{\pgfqpoint{1.915041in}{0.668364in}}%
\pgfpathlineto{\pgfqpoint{1.913200in}{0.669875in}}%
\pgfpathlineto{\pgfqpoint{1.901919in}{0.679013in}}%
\pgfpathlineto{\pgfqpoint{1.901392in}{0.679434in}}%
\pgfpathlineto{\pgfqpoint{1.890638in}{0.688020in}}%
\pgfpathlineto{\pgfqpoint{1.887489in}{0.690504in}}%
\pgfpathlineto{\pgfqpoint{1.879357in}{0.696925in}}%
\pgfpathlineto{\pgfqpoint{1.873398in}{0.701573in}}%
\pgfpathlineto{\pgfqpoint{1.868076in}{0.705707in}}%
\pgfpathlineto{\pgfqpoint{1.859050in}{0.712643in}}%
\pgfpathlineto{\pgfqpoint{1.856795in}{0.714382in}}%
\pgfpathlineto{\pgfqpoint{1.856795in}{0.712643in}}%
\pgfpathlineto{\pgfqpoint{1.856795in}{0.701573in}}%
\pgfpathlineto{\pgfqpoint{1.856795in}{0.690504in}}%
\pgfpathlineto{\pgfqpoint{1.856795in}{0.682838in}}%
\pgfpathlineto{\pgfqpoint{1.861257in}{0.679434in}}%
\pgfpathlineto{\pgfqpoint{1.868076in}{0.674238in}}%
\pgfpathlineto{\pgfqpoint{1.875687in}{0.668364in}}%
\pgfpathlineto{\pgfqpoint{1.879357in}{0.665531in}}%
\pgfpathlineto{\pgfqpoint{1.889889in}{0.657295in}}%
\pgfpathlineto{\pgfqpoint{1.890638in}{0.656709in}}%
\pgfpathlineto{\pgfqpoint{1.901919in}{0.647759in}}%
\pgfpathlineto{\pgfqpoint{1.903829in}{0.646225in}}%
\pgfpathlineto{\pgfqpoint{1.913200in}{0.638709in}}%
\pgfpathlineto{\pgfqpoint{1.917581in}{0.635156in}}%
\pgfpathlineto{\pgfqpoint{1.924480in}{0.629562in}}%
\pgfpathlineto{\pgfqpoint{1.931172in}{0.624086in}}%
\pgfpathlineto{\pgfqpoint{1.935761in}{0.620327in}}%
\pgfpathlineto{\pgfqpoint{1.944642in}{0.613017in}}%
\pgfpathlineto{\pgfqpoint{1.947042in}{0.611034in}}%
\pgfpathlineto{\pgfqpoint{1.958026in}{0.601947in}}%
\pgfpathlineto{\pgfqpoint{1.958323in}{0.601702in}}%
\pgfpathlineto{\pgfqpoint{1.969604in}{0.592337in}}%
\pgfpathlineto{\pgfqpoint{1.971353in}{0.590878in}}%
\pgfpathlineto{\pgfqpoint{1.980884in}{0.582911in}}%
\pgfpathlineto{\pgfqpoint{1.984576in}{0.579808in}}%
\pgfpathlineto{\pgfqpoint{1.992165in}{0.573425in}}%
\pgfpathlineto{\pgfqpoint{1.997710in}{0.568739in}}%
\pgfpathlineto{\pgfqpoint{2.003446in}{0.563857in}}%
\pgfpathlineto{\pgfqpoint{2.010640in}{0.557669in}}%
\pgfpathlineto{\pgfqpoint{2.014727in}{0.554151in}}%
\pgfpathlineto{\pgfqpoint{2.023463in}{0.546600in}}%
\pgfpathlineto{\pgfqpoint{2.026008in}{0.544360in}}%
\pgfpathlineto{\pgfqpoint{2.036153in}{0.535530in}}%
\pgfpathlineto{\pgfqpoint{2.037289in}{0.534535in}}%
\pgfpathlineto{\pgfqpoint{2.048569in}{0.524763in}}%
\pgfpathlineto{\pgfqpoint{2.048911in}{0.524460in}}%
\pgfpathlineto{\pgfqpoint{2.059850in}{0.514557in}}%
\pgfpathlineto{\pgfqpoint{2.061133in}{0.513391in}}%
\pgfpathlineto{\pgfqpoint{2.071131in}{0.504192in}}%
\pgfpathlineto{\pgfqpoint{2.073143in}{0.502321in}}%
\pgfpathclose%
\pgfusepath{fill}%
\end{pgfscope}%
\begin{pgfscope}%
\pgfpathrectangle{\pgfqpoint{1.856795in}{0.423750in}}{\pgfqpoint{1.194205in}{1.163386in}}%
\pgfusepath{clip}%
\pgfsetbuttcap%
\pgfsetroundjoin%
\definecolor{currentfill}{rgb}{0.451583,0.121586,0.344063}%
\pgfsetfillcolor{currentfill}%
\pgfsetlinewidth{0.000000pt}%
\definecolor{currentstroke}{rgb}{0.000000,0.000000,0.000000}%
\pgfsetstrokecolor{currentstroke}%
\pgfsetdash{}{0pt}%
\pgfpathmoveto{\pgfqpoint{2.116254in}{0.493391in}}%
\pgfpathlineto{\pgfqpoint{2.118291in}{0.491252in}}%
\pgfpathlineto{\pgfqpoint{2.127535in}{0.491252in}}%
\pgfpathlineto{\pgfqpoint{2.138816in}{0.491252in}}%
\pgfpathlineto{\pgfqpoint{2.149999in}{0.491252in}}%
\pgfpathlineto{\pgfqpoint{2.139691in}{0.502321in}}%
\pgfpathlineto{\pgfqpoint{2.138816in}{0.503240in}}%
\pgfpathlineto{\pgfqpoint{2.128978in}{0.513391in}}%
\pgfpathlineto{\pgfqpoint{2.127535in}{0.514844in}}%
\pgfpathlineto{\pgfqpoint{2.117924in}{0.524460in}}%
\pgfpathlineto{\pgfqpoint{2.116254in}{0.526094in}}%
\pgfpathlineto{\pgfqpoint{2.106270in}{0.535530in}}%
\pgfpathlineto{\pgfqpoint{2.104973in}{0.536719in}}%
\pgfpathlineto{\pgfqpoint{2.094140in}{0.546600in}}%
\pgfpathlineto{\pgfqpoint{2.093693in}{0.546998in}}%
\pgfpathlineto{\pgfqpoint{2.082412in}{0.556876in}}%
\pgfpathlineto{\pgfqpoint{2.081493in}{0.557669in}}%
\pgfpathlineto{\pgfqpoint{2.071131in}{0.566548in}}%
\pgfpathlineto{\pgfqpoint{2.068548in}{0.568739in}}%
\pgfpathlineto{\pgfqpoint{2.059850in}{0.575973in}}%
\pgfpathlineto{\pgfqpoint{2.055231in}{0.579808in}}%
\pgfpathlineto{\pgfqpoint{2.048569in}{0.585373in}}%
\pgfpathlineto{\pgfqpoint{2.041988in}{0.590878in}}%
\pgfpathlineto{\pgfqpoint{2.037289in}{0.594791in}}%
\pgfpathlineto{\pgfqpoint{2.028807in}{0.601947in}}%
\pgfpathlineto{\pgfqpoint{2.026008in}{0.604304in}}%
\pgfpathlineto{\pgfqpoint{2.015903in}{0.613017in}}%
\pgfpathlineto{\pgfqpoint{2.014727in}{0.614033in}}%
\pgfpathlineto{\pgfqpoint{2.003446in}{0.623723in}}%
\pgfpathlineto{\pgfqpoint{2.003021in}{0.624086in}}%
\pgfpathlineto{\pgfqpoint{1.992165in}{0.633390in}}%
\pgfpathlineto{\pgfqpoint{1.990093in}{0.635156in}}%
\pgfpathlineto{\pgfqpoint{1.980884in}{0.643025in}}%
\pgfpathlineto{\pgfqpoint{1.977131in}{0.646225in}}%
\pgfpathlineto{\pgfqpoint{1.969604in}{0.652665in}}%
\pgfpathlineto{\pgfqpoint{1.964157in}{0.657295in}}%
\pgfpathlineto{\pgfqpoint{1.958323in}{0.662269in}}%
\pgfpathlineto{\pgfqpoint{1.951129in}{0.668364in}}%
\pgfpathlineto{\pgfqpoint{1.947042in}{0.671837in}}%
\pgfpathlineto{\pgfqpoint{1.938029in}{0.679434in}}%
\pgfpathlineto{\pgfqpoint{1.935761in}{0.681349in}}%
\pgfpathlineto{\pgfqpoint{1.924795in}{0.690504in}}%
\pgfpathlineto{\pgfqpoint{1.924480in}{0.690767in}}%
\pgfpathlineto{\pgfqpoint{1.913200in}{0.700075in}}%
\pgfpathlineto{\pgfqpoint{1.911363in}{0.701573in}}%
\pgfpathlineto{\pgfqpoint{1.901919in}{0.709264in}}%
\pgfpathlineto{\pgfqpoint{1.897717in}{0.712643in}}%
\pgfpathlineto{\pgfqpoint{1.890638in}{0.718332in}}%
\pgfpathlineto{\pgfqpoint{1.883855in}{0.723712in}}%
\pgfpathlineto{\pgfqpoint{1.879357in}{0.727282in}}%
\pgfpathlineto{\pgfqpoint{1.869853in}{0.734782in}}%
\pgfpathlineto{\pgfqpoint{1.868076in}{0.736199in}}%
\pgfpathlineto{\pgfqpoint{1.856795in}{0.745071in}}%
\pgfpathlineto{\pgfqpoint{1.856795in}{0.734782in}}%
\pgfpathlineto{\pgfqpoint{1.856795in}{0.723712in}}%
\pgfpathlineto{\pgfqpoint{1.856795in}{0.714382in}}%
\pgfpathlineto{\pgfqpoint{1.859050in}{0.712643in}}%
\pgfpathlineto{\pgfqpoint{1.868076in}{0.705707in}}%
\pgfpathlineto{\pgfqpoint{1.873398in}{0.701573in}}%
\pgfpathlineto{\pgfqpoint{1.879357in}{0.696925in}}%
\pgfpathlineto{\pgfqpoint{1.887489in}{0.690504in}}%
\pgfpathlineto{\pgfqpoint{1.890638in}{0.688020in}}%
\pgfpathlineto{\pgfqpoint{1.901392in}{0.679434in}}%
\pgfpathlineto{\pgfqpoint{1.901919in}{0.679013in}}%
\pgfpathlineto{\pgfqpoint{1.913200in}{0.669875in}}%
\pgfpathlineto{\pgfqpoint{1.915041in}{0.668364in}}%
\pgfpathlineto{\pgfqpoint{1.924480in}{0.660633in}}%
\pgfpathlineto{\pgfqpoint{1.928519in}{0.657295in}}%
\pgfpathlineto{\pgfqpoint{1.935761in}{0.651316in}}%
\pgfpathlineto{\pgfqpoint{1.941866in}{0.646225in}}%
\pgfpathlineto{\pgfqpoint{1.947042in}{0.641914in}}%
\pgfpathlineto{\pgfqpoint{1.955108in}{0.635156in}}%
\pgfpathlineto{\pgfqpoint{1.958323in}{0.632471in}}%
\pgfpathlineto{\pgfqpoint{1.968290in}{0.624086in}}%
\pgfpathlineto{\pgfqpoint{1.969604in}{0.622985in}}%
\pgfpathlineto{\pgfqpoint{1.980884in}{0.613505in}}%
\pgfpathlineto{\pgfqpoint{1.981461in}{0.613017in}}%
\pgfpathlineto{\pgfqpoint{1.992165in}{0.603982in}}%
\pgfpathlineto{\pgfqpoint{1.994562in}{0.601947in}}%
\pgfpathlineto{\pgfqpoint{2.003446in}{0.594411in}}%
\pgfpathlineto{\pgfqpoint{2.007590in}{0.590878in}}%
\pgfpathlineto{\pgfqpoint{2.014727in}{0.584791in}}%
\pgfpathlineto{\pgfqpoint{2.020524in}{0.579808in}}%
\pgfpathlineto{\pgfqpoint{2.026008in}{0.575068in}}%
\pgfpathlineto{\pgfqpoint{2.033465in}{0.568739in}}%
\pgfpathlineto{\pgfqpoint{2.037289in}{0.565491in}}%
\pgfpathlineto{\pgfqpoint{2.046780in}{0.557669in}}%
\pgfpathlineto{\pgfqpoint{2.048569in}{0.556156in}}%
\pgfpathlineto{\pgfqpoint{2.059748in}{0.546600in}}%
\pgfpathlineto{\pgfqpoint{2.059850in}{0.546512in}}%
\pgfpathlineto{\pgfqpoint{2.071131in}{0.536682in}}%
\pgfpathlineto{\pgfqpoint{2.072424in}{0.535530in}}%
\pgfpathlineto{\pgfqpoint{2.082412in}{0.526401in}}%
\pgfpathlineto{\pgfqpoint{2.084527in}{0.524460in}}%
\pgfpathlineto{\pgfqpoint{2.093693in}{0.516024in}}%
\pgfpathlineto{\pgfqpoint{2.096479in}{0.513391in}}%
\pgfpathlineto{\pgfqpoint{2.104973in}{0.504984in}}%
\pgfpathlineto{\pgfqpoint{2.107612in}{0.502321in}}%
\pgfpathclose%
\pgfusepath{fill}%
\end{pgfscope}%
\begin{pgfscope}%
\pgfpathrectangle{\pgfqpoint{1.856795in}{0.423750in}}{\pgfqpoint{1.194205in}{1.163386in}}%
\pgfusepath{clip}%
\pgfsetbuttcap%
\pgfsetroundjoin%
\definecolor{currentfill}{rgb}{0.530589,0.116624,0.355860}%
\pgfsetfillcolor{currentfill}%
\pgfsetlinewidth{0.000000pt}%
\definecolor{currentstroke}{rgb}{0.000000,0.000000,0.000000}%
\pgfsetstrokecolor{currentstroke}%
\pgfsetdash{}{0pt}%
\pgfpathmoveto{\pgfqpoint{2.150097in}{0.491252in}}%
\pgfpathlineto{\pgfqpoint{2.161377in}{0.491252in}}%
\pgfpathlineto{\pgfqpoint{2.172658in}{0.491252in}}%
\pgfpathlineto{\pgfqpoint{2.179951in}{0.491252in}}%
\pgfpathlineto{\pgfqpoint{2.172658in}{0.499505in}}%
\pgfpathlineto{\pgfqpoint{2.170084in}{0.502321in}}%
\pgfpathlineto{\pgfqpoint{2.161377in}{0.511668in}}%
\pgfpathlineto{\pgfqpoint{2.159761in}{0.513391in}}%
\pgfpathlineto{\pgfqpoint{2.150097in}{0.523284in}}%
\pgfpathlineto{\pgfqpoint{2.148933in}{0.524460in}}%
\pgfpathlineto{\pgfqpoint{2.138816in}{0.534616in}}%
\pgfpathlineto{\pgfqpoint{2.137896in}{0.535530in}}%
\pgfpathlineto{\pgfqpoint{2.127535in}{0.545626in}}%
\pgfpathlineto{\pgfqpoint{2.126502in}{0.546600in}}%
\pgfpathlineto{\pgfqpoint{2.116254in}{0.555954in}}%
\pgfpathlineto{\pgfqpoint{2.114356in}{0.557669in}}%
\pgfpathlineto{\pgfqpoint{2.104973in}{0.565956in}}%
\pgfpathlineto{\pgfqpoint{2.101780in}{0.568739in}}%
\pgfpathlineto{\pgfqpoint{2.093693in}{0.575709in}}%
\pgfpathlineto{\pgfqpoint{2.088887in}{0.579808in}}%
\pgfpathlineto{\pgfqpoint{2.082412in}{0.585239in}}%
\pgfpathlineto{\pgfqpoint{2.075676in}{0.590878in}}%
\pgfpathlineto{\pgfqpoint{2.071131in}{0.594696in}}%
\pgfpathlineto{\pgfqpoint{2.062540in}{0.601947in}}%
\pgfpathlineto{\pgfqpoint{2.059850in}{0.604218in}}%
\pgfpathlineto{\pgfqpoint{2.049362in}{0.613017in}}%
\pgfpathlineto{\pgfqpoint{2.048569in}{0.613681in}}%
\pgfpathlineto{\pgfqpoint{2.037289in}{0.623087in}}%
\pgfpathlineto{\pgfqpoint{2.036103in}{0.624086in}}%
\pgfpathlineto{\pgfqpoint{2.026008in}{0.632598in}}%
\pgfpathlineto{\pgfqpoint{2.023065in}{0.635156in}}%
\pgfpathlineto{\pgfqpoint{2.014727in}{0.642437in}}%
\pgfpathlineto{\pgfqpoint{2.010364in}{0.646225in}}%
\pgfpathlineto{\pgfqpoint{2.003446in}{0.652237in}}%
\pgfpathlineto{\pgfqpoint{1.997595in}{0.657295in}}%
\pgfpathlineto{\pgfqpoint{1.992165in}{0.662016in}}%
\pgfpathlineto{\pgfqpoint{1.984855in}{0.668364in}}%
\pgfpathlineto{\pgfqpoint{1.980884in}{0.671818in}}%
\pgfpathlineto{\pgfqpoint{1.972061in}{0.679434in}}%
\pgfpathlineto{\pgfqpoint{1.969604in}{0.681565in}}%
\pgfpathlineto{\pgfqpoint{1.959265in}{0.690504in}}%
\pgfpathlineto{\pgfqpoint{1.958323in}{0.691319in}}%
\pgfpathlineto{\pgfqpoint{1.947042in}{0.700982in}}%
\pgfpathlineto{\pgfqpoint{1.946344in}{0.701573in}}%
\pgfpathlineto{\pgfqpoint{1.935761in}{0.710549in}}%
\pgfpathlineto{\pgfqpoint{1.933266in}{0.712643in}}%
\pgfpathlineto{\pgfqpoint{1.924480in}{0.720017in}}%
\pgfpathlineto{\pgfqpoint{1.920027in}{0.723712in}}%
\pgfpathlineto{\pgfqpoint{1.913200in}{0.729376in}}%
\pgfpathlineto{\pgfqpoint{1.906607in}{0.734782in}}%
\pgfpathlineto{\pgfqpoint{1.901919in}{0.738653in}}%
\pgfpathlineto{\pgfqpoint{1.893084in}{0.745851in}}%
\pgfpathlineto{\pgfqpoint{1.890638in}{0.747883in}}%
\pgfpathlineto{\pgfqpoint{1.879662in}{0.756921in}}%
\pgfpathlineto{\pgfqpoint{1.879357in}{0.757174in}}%
\pgfpathlineto{\pgfqpoint{1.868076in}{0.766341in}}%
\pgfpathlineto{\pgfqpoint{1.866006in}{0.767990in}}%
\pgfpathlineto{\pgfqpoint{1.856795in}{0.775281in}}%
\pgfpathlineto{\pgfqpoint{1.856795in}{0.767990in}}%
\pgfpathlineto{\pgfqpoint{1.856795in}{0.756921in}}%
\pgfpathlineto{\pgfqpoint{1.856795in}{0.745851in}}%
\pgfpathlineto{\pgfqpoint{1.856795in}{0.745071in}}%
\pgfpathlineto{\pgfqpoint{1.868076in}{0.736199in}}%
\pgfpathlineto{\pgfqpoint{1.869853in}{0.734782in}}%
\pgfpathlineto{\pgfqpoint{1.879357in}{0.727282in}}%
\pgfpathlineto{\pgfqpoint{1.883855in}{0.723712in}}%
\pgfpathlineto{\pgfqpoint{1.890638in}{0.718332in}}%
\pgfpathlineto{\pgfqpoint{1.897717in}{0.712643in}}%
\pgfpathlineto{\pgfqpoint{1.901919in}{0.709264in}}%
\pgfpathlineto{\pgfqpoint{1.911363in}{0.701573in}}%
\pgfpathlineto{\pgfqpoint{1.913200in}{0.700075in}}%
\pgfpathlineto{\pgfqpoint{1.924480in}{0.690767in}}%
\pgfpathlineto{\pgfqpoint{1.924795in}{0.690504in}}%
\pgfpathlineto{\pgfqpoint{1.935761in}{0.681349in}}%
\pgfpathlineto{\pgfqpoint{1.938029in}{0.679434in}}%
\pgfpathlineto{\pgfqpoint{1.947042in}{0.671837in}}%
\pgfpathlineto{\pgfqpoint{1.951129in}{0.668364in}}%
\pgfpathlineto{\pgfqpoint{1.958323in}{0.662269in}}%
\pgfpathlineto{\pgfqpoint{1.964157in}{0.657295in}}%
\pgfpathlineto{\pgfqpoint{1.969604in}{0.652665in}}%
\pgfpathlineto{\pgfqpoint{1.977131in}{0.646225in}}%
\pgfpathlineto{\pgfqpoint{1.980884in}{0.643025in}}%
\pgfpathlineto{\pgfqpoint{1.990093in}{0.635156in}}%
\pgfpathlineto{\pgfqpoint{1.992165in}{0.633390in}}%
\pgfpathlineto{\pgfqpoint{2.003021in}{0.624086in}}%
\pgfpathlineto{\pgfqpoint{2.003446in}{0.623723in}}%
\pgfpathlineto{\pgfqpoint{2.014727in}{0.614033in}}%
\pgfpathlineto{\pgfqpoint{2.015903in}{0.613017in}}%
\pgfpathlineto{\pgfqpoint{2.026008in}{0.604304in}}%
\pgfpathlineto{\pgfqpoint{2.028807in}{0.601947in}}%
\pgfpathlineto{\pgfqpoint{2.037289in}{0.594791in}}%
\pgfpathlineto{\pgfqpoint{2.041988in}{0.590878in}}%
\pgfpathlineto{\pgfqpoint{2.048569in}{0.585373in}}%
\pgfpathlineto{\pgfqpoint{2.055231in}{0.579808in}}%
\pgfpathlineto{\pgfqpoint{2.059850in}{0.575973in}}%
\pgfpathlineto{\pgfqpoint{2.068548in}{0.568739in}}%
\pgfpathlineto{\pgfqpoint{2.071131in}{0.566548in}}%
\pgfpathlineto{\pgfqpoint{2.081493in}{0.557669in}}%
\pgfpathlineto{\pgfqpoint{2.082412in}{0.556876in}}%
\pgfpathlineto{\pgfqpoint{2.093693in}{0.546998in}}%
\pgfpathlineto{\pgfqpoint{2.094140in}{0.546600in}}%
\pgfpathlineto{\pgfqpoint{2.104973in}{0.536719in}}%
\pgfpathlineto{\pgfqpoint{2.106270in}{0.535530in}}%
\pgfpathlineto{\pgfqpoint{2.116254in}{0.526094in}}%
\pgfpathlineto{\pgfqpoint{2.117924in}{0.524460in}}%
\pgfpathlineto{\pgfqpoint{2.127535in}{0.514844in}}%
\pgfpathlineto{\pgfqpoint{2.128978in}{0.513391in}}%
\pgfpathlineto{\pgfqpoint{2.138816in}{0.503240in}}%
\pgfpathlineto{\pgfqpoint{2.139691in}{0.502321in}}%
\pgfpathlineto{\pgfqpoint{2.149999in}{0.491252in}}%
\pgfpathclose%
\pgfusepath{fill}%
\end{pgfscope}%
\begin{pgfscope}%
\pgfpathrectangle{\pgfqpoint{1.856795in}{0.423750in}}{\pgfqpoint{1.194205in}{1.163386in}}%
\pgfusepath{clip}%
\pgfsetbuttcap%
\pgfsetroundjoin%
\definecolor{currentfill}{rgb}{0.604442,0.105739,0.358205}%
\pgfsetfillcolor{currentfill}%
\pgfsetlinewidth{0.000000pt}%
\definecolor{currentstroke}{rgb}{0.000000,0.000000,0.000000}%
\pgfsetstrokecolor{currentstroke}%
\pgfsetdash{}{0pt}%
\pgfpathmoveto{\pgfqpoint{2.172658in}{0.499505in}}%
\pgfpathlineto{\pgfqpoint{2.179951in}{0.491252in}}%
\pgfpathlineto{\pgfqpoint{2.183939in}{0.491252in}}%
\pgfpathlineto{\pgfqpoint{2.195220in}{0.491252in}}%
\pgfpathlineto{\pgfqpoint{2.206501in}{0.491252in}}%
\pgfpathlineto{\pgfqpoint{2.212057in}{0.491252in}}%
\pgfpathlineto{\pgfqpoint{2.206501in}{0.497373in}}%
\pgfpathlineto{\pgfqpoint{2.201986in}{0.502321in}}%
\pgfpathlineto{\pgfqpoint{2.195220in}{0.509427in}}%
\pgfpathlineto{\pgfqpoint{2.191262in}{0.513391in}}%
\pgfpathlineto{\pgfqpoint{2.183939in}{0.520359in}}%
\pgfpathlineto{\pgfqpoint{2.179666in}{0.524460in}}%
\pgfpathlineto{\pgfqpoint{2.172658in}{0.531017in}}%
\pgfpathlineto{\pgfqpoint{2.168233in}{0.535530in}}%
\pgfpathlineto{\pgfqpoint{2.161377in}{0.541942in}}%
\pgfpathlineto{\pgfqpoint{2.156724in}{0.546600in}}%
\pgfpathlineto{\pgfqpoint{2.150097in}{0.553139in}}%
\pgfpathlineto{\pgfqpoint{2.145438in}{0.557669in}}%
\pgfpathlineto{\pgfqpoint{2.138816in}{0.563871in}}%
\pgfpathlineto{\pgfqpoint{2.133461in}{0.568739in}}%
\pgfpathlineto{\pgfqpoint{2.127535in}{0.574001in}}%
\pgfpathlineto{\pgfqpoint{2.120923in}{0.579808in}}%
\pgfpathlineto{\pgfqpoint{2.116254in}{0.583856in}}%
\pgfpathlineto{\pgfqpoint{2.108062in}{0.590878in}}%
\pgfpathlineto{\pgfqpoint{2.104973in}{0.593490in}}%
\pgfpathlineto{\pgfqpoint{2.094953in}{0.601947in}}%
\pgfpathlineto{\pgfqpoint{2.093693in}{0.603011in}}%
\pgfpathlineto{\pgfqpoint{2.082412in}{0.612566in}}%
\pgfpathlineto{\pgfqpoint{2.081881in}{0.613017in}}%
\pgfpathlineto{\pgfqpoint{2.071131in}{0.622133in}}%
\pgfpathlineto{\pgfqpoint{2.068817in}{0.624086in}}%
\pgfpathlineto{\pgfqpoint{2.059850in}{0.631651in}}%
\pgfpathlineto{\pgfqpoint{2.055688in}{0.635156in}}%
\pgfpathlineto{\pgfqpoint{2.048569in}{0.641164in}}%
\pgfpathlineto{\pgfqpoint{2.042549in}{0.646225in}}%
\pgfpathlineto{\pgfqpoint{2.037289in}{0.650662in}}%
\pgfpathlineto{\pgfqpoint{2.029482in}{0.657295in}}%
\pgfpathlineto{\pgfqpoint{2.026008in}{0.660242in}}%
\pgfpathlineto{\pgfqpoint{2.016761in}{0.668364in}}%
\pgfpathlineto{\pgfqpoint{2.014727in}{0.670156in}}%
\pgfpathlineto{\pgfqpoint{2.004177in}{0.679434in}}%
\pgfpathlineto{\pgfqpoint{2.003446in}{0.680080in}}%
\pgfpathlineto{\pgfqpoint{1.992165in}{0.689999in}}%
\pgfpathlineto{\pgfqpoint{1.991588in}{0.690504in}}%
\pgfpathlineto{\pgfqpoint{1.980884in}{0.699892in}}%
\pgfpathlineto{\pgfqpoint{1.978962in}{0.701573in}}%
\pgfpathlineto{\pgfqpoint{1.969604in}{0.709752in}}%
\pgfpathlineto{\pgfqpoint{1.966271in}{0.712643in}}%
\pgfpathlineto{\pgfqpoint{1.958323in}{0.719550in}}%
\pgfpathlineto{\pgfqpoint{1.953490in}{0.723712in}}%
\pgfpathlineto{\pgfqpoint{1.947042in}{0.729265in}}%
\pgfpathlineto{\pgfqpoint{1.940574in}{0.734782in}}%
\pgfpathlineto{\pgfqpoint{1.935761in}{0.738955in}}%
\pgfpathlineto{\pgfqpoint{1.927691in}{0.745851in}}%
\pgfpathlineto{\pgfqpoint{1.924480in}{0.748641in}}%
\pgfpathlineto{\pgfqpoint{1.914833in}{0.756921in}}%
\pgfpathlineto{\pgfqpoint{1.913200in}{0.758322in}}%
\pgfpathlineto{\pgfqpoint{1.901919in}{0.767868in}}%
\pgfpathlineto{\pgfqpoint{1.901774in}{0.767990in}}%
\pgfpathlineto{\pgfqpoint{1.890638in}{0.777368in}}%
\pgfpathlineto{\pgfqpoint{1.888591in}{0.779060in}}%
\pgfpathlineto{\pgfqpoint{1.879357in}{0.786680in}}%
\pgfpathlineto{\pgfqpoint{1.875089in}{0.790129in}}%
\pgfpathlineto{\pgfqpoint{1.868076in}{0.795761in}}%
\pgfpathlineto{\pgfqpoint{1.861152in}{0.801199in}}%
\pgfpathlineto{\pgfqpoint{1.856795in}{0.804586in}}%
\pgfpathlineto{\pgfqpoint{1.856795in}{0.801199in}}%
\pgfpathlineto{\pgfqpoint{1.856795in}{0.790129in}}%
\pgfpathlineto{\pgfqpoint{1.856795in}{0.779060in}}%
\pgfpathlineto{\pgfqpoint{1.856795in}{0.775281in}}%
\pgfpathlineto{\pgfqpoint{1.866006in}{0.767990in}}%
\pgfpathlineto{\pgfqpoint{1.868076in}{0.766341in}}%
\pgfpathlineto{\pgfqpoint{1.879357in}{0.757174in}}%
\pgfpathlineto{\pgfqpoint{1.879662in}{0.756921in}}%
\pgfpathlineto{\pgfqpoint{1.890638in}{0.747883in}}%
\pgfpathlineto{\pgfqpoint{1.893084in}{0.745851in}}%
\pgfpathlineto{\pgfqpoint{1.901919in}{0.738653in}}%
\pgfpathlineto{\pgfqpoint{1.906607in}{0.734782in}}%
\pgfpathlineto{\pgfqpoint{1.913200in}{0.729376in}}%
\pgfpathlineto{\pgfqpoint{1.920027in}{0.723712in}}%
\pgfpathlineto{\pgfqpoint{1.924480in}{0.720017in}}%
\pgfpathlineto{\pgfqpoint{1.933266in}{0.712643in}}%
\pgfpathlineto{\pgfqpoint{1.935761in}{0.710549in}}%
\pgfpathlineto{\pgfqpoint{1.946344in}{0.701573in}}%
\pgfpathlineto{\pgfqpoint{1.947042in}{0.700982in}}%
\pgfpathlineto{\pgfqpoint{1.958323in}{0.691319in}}%
\pgfpathlineto{\pgfqpoint{1.959265in}{0.690504in}}%
\pgfpathlineto{\pgfqpoint{1.969604in}{0.681565in}}%
\pgfpathlineto{\pgfqpoint{1.972061in}{0.679434in}}%
\pgfpathlineto{\pgfqpoint{1.980884in}{0.671818in}}%
\pgfpathlineto{\pgfqpoint{1.984855in}{0.668364in}}%
\pgfpathlineto{\pgfqpoint{1.992165in}{0.662016in}}%
\pgfpathlineto{\pgfqpoint{1.997595in}{0.657295in}}%
\pgfpathlineto{\pgfqpoint{2.003446in}{0.652237in}}%
\pgfpathlineto{\pgfqpoint{2.010364in}{0.646225in}}%
\pgfpathlineto{\pgfqpoint{2.014727in}{0.642437in}}%
\pgfpathlineto{\pgfqpoint{2.023065in}{0.635156in}}%
\pgfpathlineto{\pgfqpoint{2.026008in}{0.632598in}}%
\pgfpathlineto{\pgfqpoint{2.036103in}{0.624086in}}%
\pgfpathlineto{\pgfqpoint{2.037289in}{0.623087in}}%
\pgfpathlineto{\pgfqpoint{2.048569in}{0.613681in}}%
\pgfpathlineto{\pgfqpoint{2.049362in}{0.613017in}}%
\pgfpathlineto{\pgfqpoint{2.059850in}{0.604218in}}%
\pgfpathlineto{\pgfqpoint{2.062540in}{0.601947in}}%
\pgfpathlineto{\pgfqpoint{2.071131in}{0.594696in}}%
\pgfpathlineto{\pgfqpoint{2.075676in}{0.590878in}}%
\pgfpathlineto{\pgfqpoint{2.082412in}{0.585239in}}%
\pgfpathlineto{\pgfqpoint{2.088887in}{0.579808in}}%
\pgfpathlineto{\pgfqpoint{2.093693in}{0.575709in}}%
\pgfpathlineto{\pgfqpoint{2.101780in}{0.568739in}}%
\pgfpathlineto{\pgfqpoint{2.104973in}{0.565956in}}%
\pgfpathlineto{\pgfqpoint{2.114356in}{0.557669in}}%
\pgfpathlineto{\pgfqpoint{2.116254in}{0.555954in}}%
\pgfpathlineto{\pgfqpoint{2.126502in}{0.546600in}}%
\pgfpathlineto{\pgfqpoint{2.127535in}{0.545626in}}%
\pgfpathlineto{\pgfqpoint{2.137896in}{0.535530in}}%
\pgfpathlineto{\pgfqpoint{2.138816in}{0.534616in}}%
\pgfpathlineto{\pgfqpoint{2.148933in}{0.524460in}}%
\pgfpathlineto{\pgfqpoint{2.150097in}{0.523284in}}%
\pgfpathlineto{\pgfqpoint{2.159761in}{0.513391in}}%
\pgfpathlineto{\pgfqpoint{2.161377in}{0.511668in}}%
\pgfpathlineto{\pgfqpoint{2.170084in}{0.502321in}}%
\pgfpathclose%
\pgfusepath{fill}%
\end{pgfscope}%
\begin{pgfscope}%
\pgfpathrectangle{\pgfqpoint{1.856795in}{0.423750in}}{\pgfqpoint{1.194205in}{1.163386in}}%
\pgfusepath{clip}%
\pgfsetbuttcap%
\pgfsetroundjoin%
\definecolor{currentfill}{rgb}{0.604442,0.105739,0.358205}%
\pgfsetfillcolor{currentfill}%
\pgfsetlinewidth{0.000000pt}%
\definecolor{currentstroke}{rgb}{0.000000,0.000000,0.000000}%
\pgfsetstrokecolor{currentstroke}%
\pgfsetdash{}{0pt}%
\pgfpathmoveto{\pgfqpoint{2.714137in}{1.519492in}}%
\pgfpathlineto{\pgfqpoint{2.725418in}{1.519548in}}%
\pgfpathlineto{\pgfqpoint{2.736699in}{1.519925in}}%
\pgfpathlineto{\pgfqpoint{2.747980in}{1.519962in}}%
\pgfpathlineto{\pgfqpoint{2.755595in}{1.520719in}}%
\pgfpathlineto{\pgfqpoint{2.759261in}{1.521087in}}%
\pgfpathlineto{\pgfqpoint{2.770542in}{1.523151in}}%
\pgfpathlineto{\pgfqpoint{2.781822in}{1.525998in}}%
\pgfpathlineto{\pgfqpoint{2.793103in}{1.530003in}}%
\pgfpathlineto{\pgfqpoint{2.797812in}{1.531789in}}%
\pgfpathlineto{\pgfqpoint{2.804384in}{1.534302in}}%
\pgfpathlineto{\pgfqpoint{2.815665in}{1.538944in}}%
\pgfpathlineto{\pgfqpoint{2.824182in}{1.542858in}}%
\pgfpathlineto{\pgfqpoint{2.826946in}{1.544109in}}%
\pgfpathlineto{\pgfqpoint{2.838226in}{1.550694in}}%
\pgfpathlineto{\pgfqpoint{2.843924in}{1.553928in}}%
\pgfpathlineto{\pgfqpoint{2.849507in}{1.557127in}}%
\pgfpathlineto{\pgfqpoint{2.860788in}{1.563899in}}%
\pgfpathlineto{\pgfqpoint{2.862524in}{1.564997in}}%
\pgfpathlineto{\pgfqpoint{2.872069in}{1.571082in}}%
\pgfpathlineto{\pgfqpoint{2.879511in}{1.576067in}}%
\pgfpathlineto{\pgfqpoint{2.883350in}{1.578630in}}%
\pgfpathlineto{\pgfqpoint{2.894631in}{1.586972in}}%
\pgfpathlineto{\pgfqpoint{2.894838in}{1.587136in}}%
\pgfpathlineto{\pgfqpoint{2.894631in}{1.587136in}}%
\pgfpathlineto{\pgfqpoint{2.883350in}{1.587136in}}%
\pgfpathlineto{\pgfqpoint{2.872069in}{1.587136in}}%
\pgfpathlineto{\pgfqpoint{2.860788in}{1.587136in}}%
\pgfpathlineto{\pgfqpoint{2.849507in}{1.587136in}}%
\pgfpathlineto{\pgfqpoint{2.838226in}{1.587136in}}%
\pgfpathlineto{\pgfqpoint{2.826946in}{1.587136in}}%
\pgfpathlineto{\pgfqpoint{2.815665in}{1.587136in}}%
\pgfpathlineto{\pgfqpoint{2.804384in}{1.587136in}}%
\pgfpathlineto{\pgfqpoint{2.793103in}{1.587136in}}%
\pgfpathlineto{\pgfqpoint{2.781822in}{1.587136in}}%
\pgfpathlineto{\pgfqpoint{2.770542in}{1.587136in}}%
\pgfpathlineto{\pgfqpoint{2.759261in}{1.587136in}}%
\pgfpathlineto{\pgfqpoint{2.747980in}{1.587136in}}%
\pgfpathlineto{\pgfqpoint{2.736699in}{1.587136in}}%
\pgfpathlineto{\pgfqpoint{2.725418in}{1.587136in}}%
\pgfpathlineto{\pgfqpoint{2.714137in}{1.587136in}}%
\pgfpathlineto{\pgfqpoint{2.702857in}{1.587136in}}%
\pgfpathlineto{\pgfqpoint{2.691576in}{1.587136in}}%
\pgfpathlineto{\pgfqpoint{2.680295in}{1.587136in}}%
\pgfpathlineto{\pgfqpoint{2.669014in}{1.587136in}}%
\pgfpathlineto{\pgfqpoint{2.657733in}{1.587136in}}%
\pgfpathlineto{\pgfqpoint{2.646453in}{1.587136in}}%
\pgfpathlineto{\pgfqpoint{2.635172in}{1.587136in}}%
\pgfpathlineto{\pgfqpoint{2.623891in}{1.587136in}}%
\pgfpathlineto{\pgfqpoint{2.612610in}{1.587136in}}%
\pgfpathlineto{\pgfqpoint{2.601329in}{1.587136in}}%
\pgfpathlineto{\pgfqpoint{2.590048in}{1.587136in}}%
\pgfpathlineto{\pgfqpoint{2.578768in}{1.587136in}}%
\pgfpathlineto{\pgfqpoint{2.568565in}{1.587136in}}%
\pgfpathlineto{\pgfqpoint{2.578768in}{1.580551in}}%
\pgfpathlineto{\pgfqpoint{2.586313in}{1.576067in}}%
\pgfpathlineto{\pgfqpoint{2.590048in}{1.573884in}}%
\pgfpathlineto{\pgfqpoint{2.601329in}{1.567435in}}%
\pgfpathlineto{\pgfqpoint{2.605705in}{1.564997in}}%
\pgfpathlineto{\pgfqpoint{2.612610in}{1.561198in}}%
\pgfpathlineto{\pgfqpoint{2.623891in}{1.554988in}}%
\pgfpathlineto{\pgfqpoint{2.625512in}{1.553928in}}%
\pgfpathlineto{\pgfqpoint{2.635172in}{1.547781in}}%
\pgfpathlineto{\pgfqpoint{2.643169in}{1.542858in}}%
\pgfpathlineto{\pgfqpoint{2.646453in}{1.540863in}}%
\pgfpathlineto{\pgfqpoint{2.657733in}{1.534947in}}%
\pgfpathlineto{\pgfqpoint{2.665295in}{1.531789in}}%
\pgfpathlineto{\pgfqpoint{2.669014in}{1.530268in}}%
\pgfpathlineto{\pgfqpoint{2.680295in}{1.526688in}}%
\pgfpathlineto{\pgfqpoint{2.691576in}{1.523831in}}%
\pgfpathlineto{\pgfqpoint{2.702857in}{1.521484in}}%
\pgfpathlineto{\pgfqpoint{2.707104in}{1.520719in}}%
\pgfpathclose%
\pgfusepath{fill}%
\end{pgfscope}%
\begin{pgfscope}%
\pgfpathrectangle{\pgfqpoint{1.856795in}{0.423750in}}{\pgfqpoint{1.194205in}{1.163386in}}%
\pgfusepath{clip}%
\pgfsetbuttcap%
\pgfsetroundjoin%
\definecolor{currentfill}{rgb}{0.684863,0.090856,0.349141}%
\pgfsetfillcolor{currentfill}%
\pgfsetlinewidth{0.000000pt}%
\definecolor{currentstroke}{rgb}{0.000000,0.000000,0.000000}%
\pgfsetstrokecolor{currentstroke}%
\pgfsetdash{}{0pt}%
\pgfpathmoveto{\pgfqpoint{2.206501in}{0.497373in}}%
\pgfpathlineto{\pgfqpoint{2.212057in}{0.491252in}}%
\pgfpathlineto{\pgfqpoint{2.217782in}{0.491252in}}%
\pgfpathlineto{\pgfqpoint{2.229062in}{0.491252in}}%
\pgfpathlineto{\pgfqpoint{2.240343in}{0.491252in}}%
\pgfpathlineto{\pgfqpoint{2.246934in}{0.491252in}}%
\pgfpathlineto{\pgfqpoint{2.240343in}{0.498104in}}%
\pgfpathlineto{\pgfqpoint{2.236325in}{0.502321in}}%
\pgfpathlineto{\pgfqpoint{2.229062in}{0.509589in}}%
\pgfpathlineto{\pgfqpoint{2.225407in}{0.513391in}}%
\pgfpathlineto{\pgfqpoint{2.217782in}{0.520882in}}%
\pgfpathlineto{\pgfqpoint{2.214189in}{0.524460in}}%
\pgfpathlineto{\pgfqpoint{2.206501in}{0.531630in}}%
\pgfpathlineto{\pgfqpoint{2.202440in}{0.535530in}}%
\pgfpathlineto{\pgfqpoint{2.195220in}{0.542351in}}%
\pgfpathlineto{\pgfqpoint{2.190675in}{0.546600in}}%
\pgfpathlineto{\pgfqpoint{2.183939in}{0.552804in}}%
\pgfpathlineto{\pgfqpoint{2.178581in}{0.557669in}}%
\pgfpathlineto{\pgfqpoint{2.172658in}{0.562949in}}%
\pgfpathlineto{\pgfqpoint{2.166072in}{0.568739in}}%
\pgfpathlineto{\pgfqpoint{2.161377in}{0.572775in}}%
\pgfpathlineto{\pgfqpoint{2.152967in}{0.579808in}}%
\pgfpathlineto{\pgfqpoint{2.150097in}{0.582121in}}%
\pgfpathlineto{\pgfqpoint{2.139190in}{0.590878in}}%
\pgfpathlineto{\pgfqpoint{2.138816in}{0.591170in}}%
\pgfpathlineto{\pgfqpoint{2.127535in}{0.600842in}}%
\pgfpathlineto{\pgfqpoint{2.126236in}{0.601947in}}%
\pgfpathlineto{\pgfqpoint{2.116254in}{0.610420in}}%
\pgfpathlineto{\pgfqpoint{2.113190in}{0.613017in}}%
\pgfpathlineto{\pgfqpoint{2.104973in}{0.619976in}}%
\pgfpathlineto{\pgfqpoint{2.100151in}{0.624086in}}%
\pgfpathlineto{\pgfqpoint{2.093693in}{0.629610in}}%
\pgfpathlineto{\pgfqpoint{2.087185in}{0.635156in}}%
\pgfpathlineto{\pgfqpoint{2.082412in}{0.639217in}}%
\pgfpathlineto{\pgfqpoint{2.074178in}{0.646225in}}%
\pgfpathlineto{\pgfqpoint{2.071131in}{0.648834in}}%
\pgfpathlineto{\pgfqpoint{2.061223in}{0.657295in}}%
\pgfpathlineto{\pgfqpoint{2.059850in}{0.658468in}}%
\pgfpathlineto{\pgfqpoint{2.048569in}{0.668058in}}%
\pgfpathlineto{\pgfqpoint{2.048207in}{0.668364in}}%
\pgfpathlineto{\pgfqpoint{2.037289in}{0.677595in}}%
\pgfpathlineto{\pgfqpoint{2.035126in}{0.679434in}}%
\pgfpathlineto{\pgfqpoint{2.026008in}{0.687189in}}%
\pgfpathlineto{\pgfqpoint{2.022264in}{0.690504in}}%
\pgfpathlineto{\pgfqpoint{2.014727in}{0.697195in}}%
\pgfpathlineto{\pgfqpoint{2.009794in}{0.701573in}}%
\pgfpathlineto{\pgfqpoint{2.003446in}{0.707212in}}%
\pgfpathlineto{\pgfqpoint{1.997312in}{0.712643in}}%
\pgfpathlineto{\pgfqpoint{1.992165in}{0.717203in}}%
\pgfpathlineto{\pgfqpoint{1.984797in}{0.723712in}}%
\pgfpathlineto{\pgfqpoint{1.980884in}{0.727172in}}%
\pgfpathlineto{\pgfqpoint{1.972223in}{0.734782in}}%
\pgfpathlineto{\pgfqpoint{1.969604in}{0.737144in}}%
\pgfpathlineto{\pgfqpoint{1.959854in}{0.745851in}}%
\pgfpathlineto{\pgfqpoint{1.958323in}{0.747226in}}%
\pgfpathlineto{\pgfqpoint{1.947440in}{0.756921in}}%
\pgfpathlineto{\pgfqpoint{1.947042in}{0.757275in}}%
\pgfpathlineto{\pgfqpoint{1.935761in}{0.767214in}}%
\pgfpathlineto{\pgfqpoint{1.934870in}{0.767990in}}%
\pgfpathlineto{\pgfqpoint{1.924480in}{0.777004in}}%
\pgfpathlineto{\pgfqpoint{1.922080in}{0.779060in}}%
\pgfpathlineto{\pgfqpoint{1.913200in}{0.786697in}}%
\pgfpathlineto{\pgfqpoint{1.909205in}{0.790129in}}%
\pgfpathlineto{\pgfqpoint{1.901919in}{0.796421in}}%
\pgfpathlineto{\pgfqpoint{1.896266in}{0.801199in}}%
\pgfpathlineto{\pgfqpoint{1.890638in}{0.805924in}}%
\pgfpathlineto{\pgfqpoint{1.882894in}{0.812269in}}%
\pgfpathlineto{\pgfqpoint{1.879357in}{0.815149in}}%
\pgfpathlineto{\pgfqpoint{1.869010in}{0.823338in}}%
\pgfpathlineto{\pgfqpoint{1.868076in}{0.824069in}}%
\pgfpathlineto{\pgfqpoint{1.856795in}{0.832875in}}%
\pgfpathlineto{\pgfqpoint{1.856795in}{0.823338in}}%
\pgfpathlineto{\pgfqpoint{1.856795in}{0.812269in}}%
\pgfpathlineto{\pgfqpoint{1.856795in}{0.804586in}}%
\pgfpathlineto{\pgfqpoint{1.861152in}{0.801199in}}%
\pgfpathlineto{\pgfqpoint{1.868076in}{0.795761in}}%
\pgfpathlineto{\pgfqpoint{1.875089in}{0.790129in}}%
\pgfpathlineto{\pgfqpoint{1.879357in}{0.786680in}}%
\pgfpathlineto{\pgfqpoint{1.888591in}{0.779060in}}%
\pgfpathlineto{\pgfqpoint{1.890638in}{0.777368in}}%
\pgfpathlineto{\pgfqpoint{1.901774in}{0.767990in}}%
\pgfpathlineto{\pgfqpoint{1.901919in}{0.767868in}}%
\pgfpathlineto{\pgfqpoint{1.913200in}{0.758322in}}%
\pgfpathlineto{\pgfqpoint{1.914833in}{0.756921in}}%
\pgfpathlineto{\pgfqpoint{1.924480in}{0.748641in}}%
\pgfpathlineto{\pgfqpoint{1.927691in}{0.745851in}}%
\pgfpathlineto{\pgfqpoint{1.935761in}{0.738955in}}%
\pgfpathlineto{\pgfqpoint{1.940574in}{0.734782in}}%
\pgfpathlineto{\pgfqpoint{1.947042in}{0.729265in}}%
\pgfpathlineto{\pgfqpoint{1.953490in}{0.723712in}}%
\pgfpathlineto{\pgfqpoint{1.958323in}{0.719550in}}%
\pgfpathlineto{\pgfqpoint{1.966271in}{0.712643in}}%
\pgfpathlineto{\pgfqpoint{1.969604in}{0.709752in}}%
\pgfpathlineto{\pgfqpoint{1.978962in}{0.701573in}}%
\pgfpathlineto{\pgfqpoint{1.980884in}{0.699892in}}%
\pgfpathlineto{\pgfqpoint{1.991588in}{0.690504in}}%
\pgfpathlineto{\pgfqpoint{1.992165in}{0.689999in}}%
\pgfpathlineto{\pgfqpoint{2.003446in}{0.680080in}}%
\pgfpathlineto{\pgfqpoint{2.004177in}{0.679434in}}%
\pgfpathlineto{\pgfqpoint{2.014727in}{0.670156in}}%
\pgfpathlineto{\pgfqpoint{2.016761in}{0.668364in}}%
\pgfpathlineto{\pgfqpoint{2.026008in}{0.660242in}}%
\pgfpathlineto{\pgfqpoint{2.029482in}{0.657295in}}%
\pgfpathlineto{\pgfqpoint{2.037289in}{0.650662in}}%
\pgfpathlineto{\pgfqpoint{2.042549in}{0.646225in}}%
\pgfpathlineto{\pgfqpoint{2.048569in}{0.641164in}}%
\pgfpathlineto{\pgfqpoint{2.055688in}{0.635156in}}%
\pgfpathlineto{\pgfqpoint{2.059850in}{0.631651in}}%
\pgfpathlineto{\pgfqpoint{2.068817in}{0.624086in}}%
\pgfpathlineto{\pgfqpoint{2.071131in}{0.622133in}}%
\pgfpathlineto{\pgfqpoint{2.081881in}{0.613017in}}%
\pgfpathlineto{\pgfqpoint{2.082412in}{0.612566in}}%
\pgfpathlineto{\pgfqpoint{2.093693in}{0.603011in}}%
\pgfpathlineto{\pgfqpoint{2.094953in}{0.601947in}}%
\pgfpathlineto{\pgfqpoint{2.104973in}{0.593490in}}%
\pgfpathlineto{\pgfqpoint{2.108062in}{0.590878in}}%
\pgfpathlineto{\pgfqpoint{2.116254in}{0.583856in}}%
\pgfpathlineto{\pgfqpoint{2.120923in}{0.579808in}}%
\pgfpathlineto{\pgfqpoint{2.127535in}{0.574001in}}%
\pgfpathlineto{\pgfqpoint{2.133461in}{0.568739in}}%
\pgfpathlineto{\pgfqpoint{2.138816in}{0.563871in}}%
\pgfpathlineto{\pgfqpoint{2.145438in}{0.557669in}}%
\pgfpathlineto{\pgfqpoint{2.150097in}{0.553139in}}%
\pgfpathlineto{\pgfqpoint{2.156724in}{0.546600in}}%
\pgfpathlineto{\pgfqpoint{2.161377in}{0.541942in}}%
\pgfpathlineto{\pgfqpoint{2.168233in}{0.535530in}}%
\pgfpathlineto{\pgfqpoint{2.172658in}{0.531017in}}%
\pgfpathlineto{\pgfqpoint{2.179666in}{0.524460in}}%
\pgfpathlineto{\pgfqpoint{2.183939in}{0.520359in}}%
\pgfpathlineto{\pgfqpoint{2.191262in}{0.513391in}}%
\pgfpathlineto{\pgfqpoint{2.195220in}{0.509427in}}%
\pgfpathlineto{\pgfqpoint{2.201986in}{0.502321in}}%
\pgfpathclose%
\pgfusepath{fill}%
\end{pgfscope}%
\begin{pgfscope}%
\pgfpathrectangle{\pgfqpoint{1.856795in}{0.423750in}}{\pgfqpoint{1.194205in}{1.163386in}}%
\pgfusepath{clip}%
\pgfsetbuttcap%
\pgfsetroundjoin%
\definecolor{currentfill}{rgb}{0.684863,0.090856,0.349141}%
\pgfsetfillcolor{currentfill}%
\pgfsetlinewidth{0.000000pt}%
\definecolor{currentstroke}{rgb}{0.000000,0.000000,0.000000}%
\pgfsetstrokecolor{currentstroke}%
\pgfsetdash{}{0pt}%
\pgfpathmoveto{\pgfqpoint{2.691576in}{1.441349in}}%
\pgfpathlineto{\pgfqpoint{2.702857in}{1.438971in}}%
\pgfpathlineto{\pgfqpoint{2.714137in}{1.437207in}}%
\pgfpathlineto{\pgfqpoint{2.725418in}{1.437332in}}%
\pgfpathlineto{\pgfqpoint{2.736699in}{1.437614in}}%
\pgfpathlineto{\pgfqpoint{2.747980in}{1.438604in}}%
\pgfpathlineto{\pgfqpoint{2.759261in}{1.441081in}}%
\pgfpathlineto{\pgfqpoint{2.764829in}{1.443232in}}%
\pgfpathlineto{\pgfqpoint{2.770542in}{1.445395in}}%
\pgfpathlineto{\pgfqpoint{2.781822in}{1.450269in}}%
\pgfpathlineto{\pgfqpoint{2.790376in}{1.454302in}}%
\pgfpathlineto{\pgfqpoint{2.793103in}{1.455595in}}%
\pgfpathlineto{\pgfqpoint{2.804384in}{1.461343in}}%
\pgfpathlineto{\pgfqpoint{2.811076in}{1.465371in}}%
\pgfpathlineto{\pgfqpoint{2.815665in}{1.468114in}}%
\pgfpathlineto{\pgfqpoint{2.826946in}{1.475227in}}%
\pgfpathlineto{\pgfqpoint{2.828816in}{1.476441in}}%
\pgfpathlineto{\pgfqpoint{2.838226in}{1.482489in}}%
\pgfpathlineto{\pgfqpoint{2.845581in}{1.487510in}}%
\pgfpathlineto{\pgfqpoint{2.849507in}{1.490163in}}%
\pgfpathlineto{\pgfqpoint{2.860788in}{1.498244in}}%
\pgfpathlineto{\pgfqpoint{2.861237in}{1.498580in}}%
\pgfpathlineto{\pgfqpoint{2.872069in}{1.506703in}}%
\pgfpathlineto{\pgfqpoint{2.875872in}{1.509650in}}%
\pgfpathlineto{\pgfqpoint{2.883350in}{1.515439in}}%
\pgfpathlineto{\pgfqpoint{2.889913in}{1.520719in}}%
\pgfpathlineto{\pgfqpoint{2.894631in}{1.524504in}}%
\pgfpathlineto{\pgfqpoint{2.903318in}{1.531789in}}%
\pgfpathlineto{\pgfqpoint{2.905911in}{1.533954in}}%
\pgfpathlineto{\pgfqpoint{2.916073in}{1.542858in}}%
\pgfpathlineto{\pgfqpoint{2.917192in}{1.543834in}}%
\pgfpathlineto{\pgfqpoint{2.928165in}{1.553928in}}%
\pgfpathlineto{\pgfqpoint{2.928473in}{1.554208in}}%
\pgfpathlineto{\pgfqpoint{2.939667in}{1.564997in}}%
\pgfpathlineto{\pgfqpoint{2.939754in}{1.565080in}}%
\pgfpathlineto{\pgfqpoint{2.950835in}{1.576067in}}%
\pgfpathlineto{\pgfqpoint{2.951035in}{1.576264in}}%
\pgfpathlineto{\pgfqpoint{2.961938in}{1.587136in}}%
\pgfpathlineto{\pgfqpoint{2.951035in}{1.587136in}}%
\pgfpathlineto{\pgfqpoint{2.939754in}{1.587136in}}%
\pgfpathlineto{\pgfqpoint{2.928473in}{1.587136in}}%
\pgfpathlineto{\pgfqpoint{2.917192in}{1.587136in}}%
\pgfpathlineto{\pgfqpoint{2.905911in}{1.587136in}}%
\pgfpathlineto{\pgfqpoint{2.894838in}{1.587136in}}%
\pgfpathlineto{\pgfqpoint{2.894631in}{1.586972in}}%
\pgfpathlineto{\pgfqpoint{2.883350in}{1.578630in}}%
\pgfpathlineto{\pgfqpoint{2.879511in}{1.576067in}}%
\pgfpathlineto{\pgfqpoint{2.872069in}{1.571082in}}%
\pgfpathlineto{\pgfqpoint{2.862524in}{1.564997in}}%
\pgfpathlineto{\pgfqpoint{2.860788in}{1.563899in}}%
\pgfpathlineto{\pgfqpoint{2.849507in}{1.557127in}}%
\pgfpathlineto{\pgfqpoint{2.843924in}{1.553928in}}%
\pgfpathlineto{\pgfqpoint{2.838226in}{1.550694in}}%
\pgfpathlineto{\pgfqpoint{2.826946in}{1.544109in}}%
\pgfpathlineto{\pgfqpoint{2.824182in}{1.542858in}}%
\pgfpathlineto{\pgfqpoint{2.815665in}{1.538944in}}%
\pgfpathlineto{\pgfqpoint{2.804384in}{1.534302in}}%
\pgfpathlineto{\pgfqpoint{2.797812in}{1.531789in}}%
\pgfpathlineto{\pgfqpoint{2.793103in}{1.530003in}}%
\pgfpathlineto{\pgfqpoint{2.781822in}{1.525998in}}%
\pgfpathlineto{\pgfqpoint{2.770542in}{1.523151in}}%
\pgfpathlineto{\pgfqpoint{2.759261in}{1.521087in}}%
\pgfpathlineto{\pgfqpoint{2.755595in}{1.520719in}}%
\pgfpathlineto{\pgfqpoint{2.747980in}{1.519962in}}%
\pgfpathlineto{\pgfqpoint{2.736699in}{1.519925in}}%
\pgfpathlineto{\pgfqpoint{2.725418in}{1.519548in}}%
\pgfpathlineto{\pgfqpoint{2.714137in}{1.519492in}}%
\pgfpathlineto{\pgfqpoint{2.707104in}{1.520719in}}%
\pgfpathlineto{\pgfqpoint{2.702857in}{1.521484in}}%
\pgfpathlineto{\pgfqpoint{2.691576in}{1.523831in}}%
\pgfpathlineto{\pgfqpoint{2.680295in}{1.526688in}}%
\pgfpathlineto{\pgfqpoint{2.669014in}{1.530268in}}%
\pgfpathlineto{\pgfqpoint{2.665295in}{1.531789in}}%
\pgfpathlineto{\pgfqpoint{2.657733in}{1.534947in}}%
\pgfpathlineto{\pgfqpoint{2.646453in}{1.540863in}}%
\pgfpathlineto{\pgfqpoint{2.643169in}{1.542858in}}%
\pgfpathlineto{\pgfqpoint{2.635172in}{1.547781in}}%
\pgfpathlineto{\pgfqpoint{2.625512in}{1.553928in}}%
\pgfpathlineto{\pgfqpoint{2.623891in}{1.554988in}}%
\pgfpathlineto{\pgfqpoint{2.612610in}{1.561198in}}%
\pgfpathlineto{\pgfqpoint{2.605705in}{1.564997in}}%
\pgfpathlineto{\pgfqpoint{2.601329in}{1.567435in}}%
\pgfpathlineto{\pgfqpoint{2.590048in}{1.573884in}}%
\pgfpathlineto{\pgfqpoint{2.586313in}{1.576067in}}%
\pgfpathlineto{\pgfqpoint{2.578768in}{1.580551in}}%
\pgfpathlineto{\pgfqpoint{2.568565in}{1.587136in}}%
\pgfpathlineto{\pgfqpoint{2.567487in}{1.587136in}}%
\pgfpathlineto{\pgfqpoint{2.556206in}{1.587136in}}%
\pgfpathlineto{\pgfqpoint{2.544925in}{1.587136in}}%
\pgfpathlineto{\pgfqpoint{2.533644in}{1.587136in}}%
\pgfpathlineto{\pgfqpoint{2.522364in}{1.587136in}}%
\pgfpathlineto{\pgfqpoint{2.511083in}{1.587136in}}%
\pgfpathlineto{\pgfqpoint{2.499802in}{1.587136in}}%
\pgfpathlineto{\pgfqpoint{2.488521in}{1.587136in}}%
\pgfpathlineto{\pgfqpoint{2.477240in}{1.587136in}}%
\pgfpathlineto{\pgfqpoint{2.474368in}{1.587136in}}%
\pgfpathlineto{\pgfqpoint{2.477240in}{1.583781in}}%
\pgfpathlineto{\pgfqpoint{2.483940in}{1.576067in}}%
\pgfpathlineto{\pgfqpoint{2.488521in}{1.570830in}}%
\pgfpathlineto{\pgfqpoint{2.493831in}{1.564997in}}%
\pgfpathlineto{\pgfqpoint{2.499802in}{1.558435in}}%
\pgfpathlineto{\pgfqpoint{2.504049in}{1.553928in}}%
\pgfpathlineto{\pgfqpoint{2.511083in}{1.546598in}}%
\pgfpathlineto{\pgfqpoint{2.514876in}{1.542858in}}%
\pgfpathlineto{\pgfqpoint{2.522364in}{1.535656in}}%
\pgfpathlineto{\pgfqpoint{2.526482in}{1.531789in}}%
\pgfpathlineto{\pgfqpoint{2.533644in}{1.525146in}}%
\pgfpathlineto{\pgfqpoint{2.538963in}{1.520719in}}%
\pgfpathlineto{\pgfqpoint{2.544925in}{1.515673in}}%
\pgfpathlineto{\pgfqpoint{2.552634in}{1.509650in}}%
\pgfpathlineto{\pgfqpoint{2.556206in}{1.506933in}}%
\pgfpathlineto{\pgfqpoint{2.567487in}{1.499894in}}%
\pgfpathlineto{\pgfqpoint{2.569694in}{1.498580in}}%
\pgfpathlineto{\pgfqpoint{2.578768in}{1.493230in}}%
\pgfpathlineto{\pgfqpoint{2.588951in}{1.487510in}}%
\pgfpathlineto{\pgfqpoint{2.590048in}{1.486916in}}%
\pgfpathlineto{\pgfqpoint{2.601329in}{1.481362in}}%
\pgfpathlineto{\pgfqpoint{2.612145in}{1.476441in}}%
\pgfpathlineto{\pgfqpoint{2.612610in}{1.476232in}}%
\pgfpathlineto{\pgfqpoint{2.623891in}{1.471326in}}%
\pgfpathlineto{\pgfqpoint{2.635172in}{1.466169in}}%
\pgfpathlineto{\pgfqpoint{2.636606in}{1.465371in}}%
\pgfpathlineto{\pgfqpoint{2.646453in}{1.459958in}}%
\pgfpathlineto{\pgfqpoint{2.657138in}{1.454302in}}%
\pgfpathlineto{\pgfqpoint{2.657733in}{1.453990in}}%
\pgfpathlineto{\pgfqpoint{2.669014in}{1.449023in}}%
\pgfpathlineto{\pgfqpoint{2.680295in}{1.444636in}}%
\pgfpathlineto{\pgfqpoint{2.685110in}{1.443232in}}%
\pgfpathclose%
\pgfusepath{fill}%
\end{pgfscope}%
\begin{pgfscope}%
\pgfpathrectangle{\pgfqpoint{1.856795in}{0.423750in}}{\pgfqpoint{1.194205in}{1.163386in}}%
\pgfusepath{clip}%
\pgfsetbuttcap%
\pgfsetroundjoin%
\definecolor{currentfill}{rgb}{0.761456,0.090507,0.326371}%
\pgfsetfillcolor{currentfill}%
\pgfsetlinewidth{0.000000pt}%
\definecolor{currentstroke}{rgb}{0.000000,0.000000,0.000000}%
\pgfsetstrokecolor{currentstroke}%
\pgfsetdash{}{0pt}%
\pgfpathmoveto{\pgfqpoint{2.240343in}{0.498104in}}%
\pgfpathlineto{\pgfqpoint{2.246934in}{0.491252in}}%
\pgfpathlineto{\pgfqpoint{2.251624in}{0.491252in}}%
\pgfpathlineto{\pgfqpoint{2.262905in}{0.491252in}}%
\pgfpathlineto{\pgfqpoint{2.274186in}{0.491252in}}%
\pgfpathlineto{\pgfqpoint{2.284016in}{0.491252in}}%
\pgfpathlineto{\pgfqpoint{2.274186in}{0.501602in}}%
\pgfpathlineto{\pgfqpoint{2.273522in}{0.502321in}}%
\pgfpathlineto{\pgfqpoint{2.263120in}{0.513391in}}%
\pgfpathlineto{\pgfqpoint{2.262905in}{0.513609in}}%
\pgfpathlineto{\pgfqpoint{2.251944in}{0.524460in}}%
\pgfpathlineto{\pgfqpoint{2.251624in}{0.524767in}}%
\pgfpathlineto{\pgfqpoint{2.240343in}{0.535482in}}%
\pgfpathlineto{\pgfqpoint{2.240293in}{0.535530in}}%
\pgfpathlineto{\pgfqpoint{2.229062in}{0.545756in}}%
\pgfpathlineto{\pgfqpoint{2.228109in}{0.546600in}}%
\pgfpathlineto{\pgfqpoint{2.217782in}{0.555669in}}%
\pgfpathlineto{\pgfqpoint{2.215437in}{0.557669in}}%
\pgfpathlineto{\pgfqpoint{2.206501in}{0.565107in}}%
\pgfpathlineto{\pgfqpoint{2.202082in}{0.568739in}}%
\pgfpathlineto{\pgfqpoint{2.195220in}{0.574224in}}%
\pgfpathlineto{\pgfqpoint{2.188791in}{0.579808in}}%
\pgfpathlineto{\pgfqpoint{2.183939in}{0.583771in}}%
\pgfpathlineto{\pgfqpoint{2.175572in}{0.590878in}}%
\pgfpathlineto{\pgfqpoint{2.172658in}{0.593190in}}%
\pgfpathlineto{\pgfqpoint{2.161745in}{0.601947in}}%
\pgfpathlineto{\pgfqpoint{2.161377in}{0.602234in}}%
\pgfpathlineto{\pgfqpoint{2.150097in}{0.610915in}}%
\pgfpathlineto{\pgfqpoint{2.147342in}{0.613017in}}%
\pgfpathlineto{\pgfqpoint{2.138816in}{0.619503in}}%
\pgfpathlineto{\pgfqpoint{2.132784in}{0.624086in}}%
\pgfpathlineto{\pgfqpoint{2.127535in}{0.628063in}}%
\pgfpathlineto{\pgfqpoint{2.118213in}{0.635156in}}%
\pgfpathlineto{\pgfqpoint{2.116254in}{0.636649in}}%
\pgfpathlineto{\pgfqpoint{2.104973in}{0.645909in}}%
\pgfpathlineto{\pgfqpoint{2.104603in}{0.646225in}}%
\pgfpathlineto{\pgfqpoint{2.093693in}{0.655561in}}%
\pgfpathlineto{\pgfqpoint{2.091671in}{0.657295in}}%
\pgfpathlineto{\pgfqpoint{2.082412in}{0.665283in}}%
\pgfpathlineto{\pgfqpoint{2.078834in}{0.668364in}}%
\pgfpathlineto{\pgfqpoint{2.071131in}{0.674979in}}%
\pgfpathlineto{\pgfqpoint{2.065918in}{0.679434in}}%
\pgfpathlineto{\pgfqpoint{2.059850in}{0.684610in}}%
\pgfpathlineto{\pgfqpoint{2.052934in}{0.690504in}}%
\pgfpathlineto{\pgfqpoint{2.048569in}{0.694225in}}%
\pgfpathlineto{\pgfqpoint{2.039933in}{0.701573in}}%
\pgfpathlineto{\pgfqpoint{2.037289in}{0.703809in}}%
\pgfpathlineto{\pgfqpoint{2.026885in}{0.712643in}}%
\pgfpathlineto{\pgfqpoint{2.026008in}{0.713392in}}%
\pgfpathlineto{\pgfqpoint{2.014727in}{0.723517in}}%
\pgfpathlineto{\pgfqpoint{2.014508in}{0.723712in}}%
\pgfpathlineto{\pgfqpoint{2.003446in}{0.733653in}}%
\pgfpathlineto{\pgfqpoint{2.002180in}{0.734782in}}%
\pgfpathlineto{\pgfqpoint{1.992165in}{0.743891in}}%
\pgfpathlineto{\pgfqpoint{1.990009in}{0.745851in}}%
\pgfpathlineto{\pgfqpoint{1.980884in}{0.754162in}}%
\pgfpathlineto{\pgfqpoint{1.977840in}{0.756921in}}%
\pgfpathlineto{\pgfqpoint{1.969604in}{0.764364in}}%
\pgfpathlineto{\pgfqpoint{1.965569in}{0.767990in}}%
\pgfpathlineto{\pgfqpoint{1.958323in}{0.774475in}}%
\pgfpathlineto{\pgfqpoint{1.953166in}{0.779060in}}%
\pgfpathlineto{\pgfqpoint{1.947042in}{0.784470in}}%
\pgfpathlineto{\pgfqpoint{1.940594in}{0.790129in}}%
\pgfpathlineto{\pgfqpoint{1.935761in}{0.794375in}}%
\pgfpathlineto{\pgfqpoint{1.927979in}{0.801199in}}%
\pgfpathlineto{\pgfqpoint{1.924480in}{0.804302in}}%
\pgfpathlineto{\pgfqpoint{1.915451in}{0.812269in}}%
\pgfpathlineto{\pgfqpoint{1.913200in}{0.814264in}}%
\pgfpathlineto{\pgfqpoint{1.902726in}{0.823338in}}%
\pgfpathlineto{\pgfqpoint{1.901919in}{0.824028in}}%
\pgfpathlineto{\pgfqpoint{1.890638in}{0.833409in}}%
\pgfpathlineto{\pgfqpoint{1.889402in}{0.834408in}}%
\pgfpathlineto{\pgfqpoint{1.879357in}{0.842246in}}%
\pgfpathlineto{\pgfqpoint{1.875227in}{0.845477in}}%
\pgfpathlineto{\pgfqpoint{1.868076in}{0.851307in}}%
\pgfpathlineto{\pgfqpoint{1.861261in}{0.856547in}}%
\pgfpathlineto{\pgfqpoint{1.856795in}{0.860159in}}%
\pgfpathlineto{\pgfqpoint{1.856795in}{0.856547in}}%
\pgfpathlineto{\pgfqpoint{1.856795in}{0.845477in}}%
\pgfpathlineto{\pgfqpoint{1.856795in}{0.834408in}}%
\pgfpathlineto{\pgfqpoint{1.856795in}{0.832875in}}%
\pgfpathlineto{\pgfqpoint{1.868076in}{0.824069in}}%
\pgfpathlineto{\pgfqpoint{1.869010in}{0.823338in}}%
\pgfpathlineto{\pgfqpoint{1.879357in}{0.815149in}}%
\pgfpathlineto{\pgfqpoint{1.882894in}{0.812269in}}%
\pgfpathlineto{\pgfqpoint{1.890638in}{0.805924in}}%
\pgfpathlineto{\pgfqpoint{1.896266in}{0.801199in}}%
\pgfpathlineto{\pgfqpoint{1.901919in}{0.796421in}}%
\pgfpathlineto{\pgfqpoint{1.909205in}{0.790129in}}%
\pgfpathlineto{\pgfqpoint{1.913200in}{0.786697in}}%
\pgfpathlineto{\pgfqpoint{1.922080in}{0.779060in}}%
\pgfpathlineto{\pgfqpoint{1.924480in}{0.777004in}}%
\pgfpathlineto{\pgfqpoint{1.934870in}{0.767990in}}%
\pgfpathlineto{\pgfqpoint{1.935761in}{0.767214in}}%
\pgfpathlineto{\pgfqpoint{1.947042in}{0.757275in}}%
\pgfpathlineto{\pgfqpoint{1.947440in}{0.756921in}}%
\pgfpathlineto{\pgfqpoint{1.958323in}{0.747226in}}%
\pgfpathlineto{\pgfqpoint{1.959854in}{0.745851in}}%
\pgfpathlineto{\pgfqpoint{1.969604in}{0.737144in}}%
\pgfpathlineto{\pgfqpoint{1.972223in}{0.734782in}}%
\pgfpathlineto{\pgfqpoint{1.980884in}{0.727172in}}%
\pgfpathlineto{\pgfqpoint{1.984797in}{0.723712in}}%
\pgfpathlineto{\pgfqpoint{1.992165in}{0.717203in}}%
\pgfpathlineto{\pgfqpoint{1.997312in}{0.712643in}}%
\pgfpathlineto{\pgfqpoint{2.003446in}{0.707212in}}%
\pgfpathlineto{\pgfqpoint{2.009794in}{0.701573in}}%
\pgfpathlineto{\pgfqpoint{2.014727in}{0.697195in}}%
\pgfpathlineto{\pgfqpoint{2.022264in}{0.690504in}}%
\pgfpathlineto{\pgfqpoint{2.026008in}{0.687189in}}%
\pgfpathlineto{\pgfqpoint{2.035126in}{0.679434in}}%
\pgfpathlineto{\pgfqpoint{2.037289in}{0.677595in}}%
\pgfpathlineto{\pgfqpoint{2.048207in}{0.668364in}}%
\pgfpathlineto{\pgfqpoint{2.048569in}{0.668058in}}%
\pgfpathlineto{\pgfqpoint{2.059850in}{0.658468in}}%
\pgfpathlineto{\pgfqpoint{2.061223in}{0.657295in}}%
\pgfpathlineto{\pgfqpoint{2.071131in}{0.648834in}}%
\pgfpathlineto{\pgfqpoint{2.074178in}{0.646225in}}%
\pgfpathlineto{\pgfqpoint{2.082412in}{0.639217in}}%
\pgfpathlineto{\pgfqpoint{2.087185in}{0.635156in}}%
\pgfpathlineto{\pgfqpoint{2.093693in}{0.629610in}}%
\pgfpathlineto{\pgfqpoint{2.100151in}{0.624086in}}%
\pgfpathlineto{\pgfqpoint{2.104973in}{0.619976in}}%
\pgfpathlineto{\pgfqpoint{2.113190in}{0.613017in}}%
\pgfpathlineto{\pgfqpoint{2.116254in}{0.610420in}}%
\pgfpathlineto{\pgfqpoint{2.126236in}{0.601947in}}%
\pgfpathlineto{\pgfqpoint{2.127535in}{0.600842in}}%
\pgfpathlineto{\pgfqpoint{2.138816in}{0.591170in}}%
\pgfpathlineto{\pgfqpoint{2.139190in}{0.590878in}}%
\pgfpathlineto{\pgfqpoint{2.150097in}{0.582121in}}%
\pgfpathlineto{\pgfqpoint{2.152967in}{0.579808in}}%
\pgfpathlineto{\pgfqpoint{2.161377in}{0.572775in}}%
\pgfpathlineto{\pgfqpoint{2.166072in}{0.568739in}}%
\pgfpathlineto{\pgfqpoint{2.172658in}{0.562949in}}%
\pgfpathlineto{\pgfqpoint{2.178581in}{0.557669in}}%
\pgfpathlineto{\pgfqpoint{2.183939in}{0.552804in}}%
\pgfpathlineto{\pgfqpoint{2.190675in}{0.546600in}}%
\pgfpathlineto{\pgfqpoint{2.195220in}{0.542351in}}%
\pgfpathlineto{\pgfqpoint{2.202440in}{0.535530in}}%
\pgfpathlineto{\pgfqpoint{2.206501in}{0.531630in}}%
\pgfpathlineto{\pgfqpoint{2.214189in}{0.524460in}}%
\pgfpathlineto{\pgfqpoint{2.217782in}{0.520882in}}%
\pgfpathlineto{\pgfqpoint{2.225407in}{0.513391in}}%
\pgfpathlineto{\pgfqpoint{2.229062in}{0.509589in}}%
\pgfpathlineto{\pgfqpoint{2.236325in}{0.502321in}}%
\pgfpathclose%
\pgfusepath{fill}%
\end{pgfscope}%
\begin{pgfscope}%
\pgfpathrectangle{\pgfqpoint{1.856795in}{0.423750in}}{\pgfqpoint{1.194205in}{1.163386in}}%
\pgfusepath{clip}%
\pgfsetbuttcap%
\pgfsetroundjoin%
\definecolor{currentfill}{rgb}{0.761456,0.090507,0.326371}%
\pgfsetfillcolor{currentfill}%
\pgfsetlinewidth{0.000000pt}%
\definecolor{currentstroke}{rgb}{0.000000,0.000000,0.000000}%
\pgfsetstrokecolor{currentstroke}%
\pgfsetdash{}{0pt}%
\pgfpathmoveto{\pgfqpoint{2.680295in}{1.376614in}}%
\pgfpathlineto{\pgfqpoint{2.691576in}{1.373032in}}%
\pgfpathlineto{\pgfqpoint{2.702857in}{1.370318in}}%
\pgfpathlineto{\pgfqpoint{2.714137in}{1.368492in}}%
\pgfpathlineto{\pgfqpoint{2.725418in}{1.368252in}}%
\pgfpathlineto{\pgfqpoint{2.736699in}{1.368996in}}%
\pgfpathlineto{\pgfqpoint{2.747980in}{1.372259in}}%
\pgfpathlineto{\pgfqpoint{2.756991in}{1.376815in}}%
\pgfpathlineto{\pgfqpoint{2.759261in}{1.377947in}}%
\pgfpathlineto{\pgfqpoint{2.770542in}{1.383560in}}%
\pgfpathlineto{\pgfqpoint{2.778719in}{1.387885in}}%
\pgfpathlineto{\pgfqpoint{2.781822in}{1.389499in}}%
\pgfpathlineto{\pgfqpoint{2.793103in}{1.396523in}}%
\pgfpathlineto{\pgfqpoint{2.796582in}{1.398954in}}%
\pgfpathlineto{\pgfqpoint{2.804384in}{1.404330in}}%
\pgfpathlineto{\pgfqpoint{2.812255in}{1.410024in}}%
\pgfpathlineto{\pgfqpoint{2.815665in}{1.412477in}}%
\pgfpathlineto{\pgfqpoint{2.826946in}{1.420598in}}%
\pgfpathlineto{\pgfqpoint{2.827636in}{1.421093in}}%
\pgfpathlineto{\pgfqpoint{2.838226in}{1.428653in}}%
\pgfpathlineto{\pgfqpoint{2.842955in}{1.432163in}}%
\pgfpathlineto{\pgfqpoint{2.849507in}{1.436996in}}%
\pgfpathlineto{\pgfqpoint{2.857657in}{1.443232in}}%
\pgfpathlineto{\pgfqpoint{2.860788in}{1.445614in}}%
\pgfpathlineto{\pgfqpoint{2.871797in}{1.454302in}}%
\pgfpathlineto{\pgfqpoint{2.872069in}{1.454515in}}%
\pgfpathlineto{\pgfqpoint{2.883350in}{1.463837in}}%
\pgfpathlineto{\pgfqpoint{2.885128in}{1.465371in}}%
\pgfpathlineto{\pgfqpoint{2.894631in}{1.473509in}}%
\pgfpathlineto{\pgfqpoint{2.897933in}{1.476441in}}%
\pgfpathlineto{\pgfqpoint{2.905911in}{1.483502in}}%
\pgfpathlineto{\pgfqpoint{2.910290in}{1.487510in}}%
\pgfpathlineto{\pgfqpoint{2.917192in}{1.493815in}}%
\pgfpathlineto{\pgfqpoint{2.922253in}{1.498580in}}%
\pgfpathlineto{\pgfqpoint{2.928473in}{1.504453in}}%
\pgfpathlineto{\pgfqpoint{2.933831in}{1.509650in}}%
\pgfpathlineto{\pgfqpoint{2.939754in}{1.515414in}}%
\pgfpathlineto{\pgfqpoint{2.945135in}{1.520719in}}%
\pgfpathlineto{\pgfqpoint{2.951035in}{1.526548in}}%
\pgfpathlineto{\pgfqpoint{2.956353in}{1.531789in}}%
\pgfpathlineto{\pgfqpoint{2.962315in}{1.537677in}}%
\pgfpathlineto{\pgfqpoint{2.967464in}{1.542858in}}%
\pgfpathlineto{\pgfqpoint{2.973596in}{1.549047in}}%
\pgfpathlineto{\pgfqpoint{2.973596in}{1.553928in}}%
\pgfpathlineto{\pgfqpoint{2.973596in}{1.564997in}}%
\pgfpathlineto{\pgfqpoint{2.973596in}{1.576067in}}%
\pgfpathlineto{\pgfqpoint{2.973596in}{1.587136in}}%
\pgfpathlineto{\pgfqpoint{2.962315in}{1.587136in}}%
\pgfpathlineto{\pgfqpoint{2.961938in}{1.587136in}}%
\pgfpathlineto{\pgfqpoint{2.951035in}{1.576264in}}%
\pgfpathlineto{\pgfqpoint{2.950835in}{1.576067in}}%
\pgfpathlineto{\pgfqpoint{2.939754in}{1.565080in}}%
\pgfpathlineto{\pgfqpoint{2.939667in}{1.564997in}}%
\pgfpathlineto{\pgfqpoint{2.928473in}{1.554208in}}%
\pgfpathlineto{\pgfqpoint{2.928165in}{1.553928in}}%
\pgfpathlineto{\pgfqpoint{2.917192in}{1.543834in}}%
\pgfpathlineto{\pgfqpoint{2.916073in}{1.542858in}}%
\pgfpathlineto{\pgfqpoint{2.905911in}{1.533954in}}%
\pgfpathlineto{\pgfqpoint{2.903318in}{1.531789in}}%
\pgfpathlineto{\pgfqpoint{2.894631in}{1.524504in}}%
\pgfpathlineto{\pgfqpoint{2.889913in}{1.520719in}}%
\pgfpathlineto{\pgfqpoint{2.883350in}{1.515439in}}%
\pgfpathlineto{\pgfqpoint{2.875872in}{1.509650in}}%
\pgfpathlineto{\pgfqpoint{2.872069in}{1.506703in}}%
\pgfpathlineto{\pgfqpoint{2.861237in}{1.498580in}}%
\pgfpathlineto{\pgfqpoint{2.860788in}{1.498244in}}%
\pgfpathlineto{\pgfqpoint{2.849507in}{1.490163in}}%
\pgfpathlineto{\pgfqpoint{2.845581in}{1.487510in}}%
\pgfpathlineto{\pgfqpoint{2.838226in}{1.482489in}}%
\pgfpathlineto{\pgfqpoint{2.828816in}{1.476441in}}%
\pgfpathlineto{\pgfqpoint{2.826946in}{1.475227in}}%
\pgfpathlineto{\pgfqpoint{2.815665in}{1.468114in}}%
\pgfpathlineto{\pgfqpoint{2.811076in}{1.465371in}}%
\pgfpathlineto{\pgfqpoint{2.804384in}{1.461343in}}%
\pgfpathlineto{\pgfqpoint{2.793103in}{1.455595in}}%
\pgfpathlineto{\pgfqpoint{2.790376in}{1.454302in}}%
\pgfpathlineto{\pgfqpoint{2.781822in}{1.450269in}}%
\pgfpathlineto{\pgfqpoint{2.770542in}{1.445395in}}%
\pgfpathlineto{\pgfqpoint{2.764829in}{1.443232in}}%
\pgfpathlineto{\pgfqpoint{2.759261in}{1.441081in}}%
\pgfpathlineto{\pgfqpoint{2.747980in}{1.438604in}}%
\pgfpathlineto{\pgfqpoint{2.736699in}{1.437614in}}%
\pgfpathlineto{\pgfqpoint{2.725418in}{1.437332in}}%
\pgfpathlineto{\pgfqpoint{2.714137in}{1.437207in}}%
\pgfpathlineto{\pgfqpoint{2.702857in}{1.438971in}}%
\pgfpathlineto{\pgfqpoint{2.691576in}{1.441349in}}%
\pgfpathlineto{\pgfqpoint{2.685110in}{1.443232in}}%
\pgfpathlineto{\pgfqpoint{2.680295in}{1.444636in}}%
\pgfpathlineto{\pgfqpoint{2.669014in}{1.449023in}}%
\pgfpathlineto{\pgfqpoint{2.657733in}{1.453990in}}%
\pgfpathlineto{\pgfqpoint{2.657138in}{1.454302in}}%
\pgfpathlineto{\pgfqpoint{2.646453in}{1.459958in}}%
\pgfpathlineto{\pgfqpoint{2.636606in}{1.465371in}}%
\pgfpathlineto{\pgfqpoint{2.635172in}{1.466169in}}%
\pgfpathlineto{\pgfqpoint{2.623891in}{1.471326in}}%
\pgfpathlineto{\pgfqpoint{2.612610in}{1.476232in}}%
\pgfpathlineto{\pgfqpoint{2.612145in}{1.476441in}}%
\pgfpathlineto{\pgfqpoint{2.601329in}{1.481362in}}%
\pgfpathlineto{\pgfqpoint{2.590048in}{1.486916in}}%
\pgfpathlineto{\pgfqpoint{2.588951in}{1.487510in}}%
\pgfpathlineto{\pgfqpoint{2.578768in}{1.493230in}}%
\pgfpathlineto{\pgfqpoint{2.569694in}{1.498580in}}%
\pgfpathlineto{\pgfqpoint{2.567487in}{1.499894in}}%
\pgfpathlineto{\pgfqpoint{2.556206in}{1.506933in}}%
\pgfpathlineto{\pgfqpoint{2.552634in}{1.509650in}}%
\pgfpathlineto{\pgfqpoint{2.544925in}{1.515673in}}%
\pgfpathlineto{\pgfqpoint{2.538963in}{1.520719in}}%
\pgfpathlineto{\pgfqpoint{2.533644in}{1.525146in}}%
\pgfpathlineto{\pgfqpoint{2.526482in}{1.531789in}}%
\pgfpathlineto{\pgfqpoint{2.522364in}{1.535656in}}%
\pgfpathlineto{\pgfqpoint{2.514876in}{1.542858in}}%
\pgfpathlineto{\pgfqpoint{2.511083in}{1.546598in}}%
\pgfpathlineto{\pgfqpoint{2.504049in}{1.553928in}}%
\pgfpathlineto{\pgfqpoint{2.499802in}{1.558435in}}%
\pgfpathlineto{\pgfqpoint{2.493831in}{1.564997in}}%
\pgfpathlineto{\pgfqpoint{2.488521in}{1.570830in}}%
\pgfpathlineto{\pgfqpoint{2.483940in}{1.576067in}}%
\pgfpathlineto{\pgfqpoint{2.477240in}{1.583781in}}%
\pgfpathlineto{\pgfqpoint{2.474368in}{1.587136in}}%
\pgfpathlineto{\pgfqpoint{2.465960in}{1.587136in}}%
\pgfpathlineto{\pgfqpoint{2.454679in}{1.587136in}}%
\pgfpathlineto{\pgfqpoint{2.443398in}{1.587136in}}%
\pgfpathlineto{\pgfqpoint{2.432117in}{1.587136in}}%
\pgfpathlineto{\pgfqpoint{2.420836in}{1.587136in}}%
\pgfpathlineto{\pgfqpoint{2.409555in}{1.587136in}}%
\pgfpathlineto{\pgfqpoint{2.406990in}{1.587136in}}%
\pgfpathlineto{\pgfqpoint{2.409555in}{1.583965in}}%
\pgfpathlineto{\pgfqpoint{2.416005in}{1.576067in}}%
\pgfpathlineto{\pgfqpoint{2.420836in}{1.570188in}}%
\pgfpathlineto{\pgfqpoint{2.425133in}{1.564997in}}%
\pgfpathlineto{\pgfqpoint{2.432117in}{1.556556in}}%
\pgfpathlineto{\pgfqpoint{2.434296in}{1.553928in}}%
\pgfpathlineto{\pgfqpoint{2.443398in}{1.542964in}}%
\pgfpathlineto{\pgfqpoint{2.443487in}{1.542858in}}%
\pgfpathlineto{\pgfqpoint{2.452944in}{1.531789in}}%
\pgfpathlineto{\pgfqpoint{2.454679in}{1.529761in}}%
\pgfpathlineto{\pgfqpoint{2.462616in}{1.520719in}}%
\pgfpathlineto{\pgfqpoint{2.465960in}{1.516921in}}%
\pgfpathlineto{\pgfqpoint{2.472443in}{1.509650in}}%
\pgfpathlineto{\pgfqpoint{2.477240in}{1.504426in}}%
\pgfpathlineto{\pgfqpoint{2.482918in}{1.498580in}}%
\pgfpathlineto{\pgfqpoint{2.488521in}{1.492912in}}%
\pgfpathlineto{\pgfqpoint{2.494037in}{1.487510in}}%
\pgfpathlineto{\pgfqpoint{2.499802in}{1.481924in}}%
\pgfpathlineto{\pgfqpoint{2.505561in}{1.476441in}}%
\pgfpathlineto{\pgfqpoint{2.511083in}{1.471233in}}%
\pgfpathlineto{\pgfqpoint{2.517379in}{1.465371in}}%
\pgfpathlineto{\pgfqpoint{2.522364in}{1.460769in}}%
\pgfpathlineto{\pgfqpoint{2.529729in}{1.454302in}}%
\pgfpathlineto{\pgfqpoint{2.533644in}{1.450852in}}%
\pgfpathlineto{\pgfqpoint{2.543149in}{1.443232in}}%
\pgfpathlineto{\pgfqpoint{2.544925in}{1.441886in}}%
\pgfpathlineto{\pgfqpoint{2.556206in}{1.434882in}}%
\pgfpathlineto{\pgfqpoint{2.560930in}{1.432163in}}%
\pgfpathlineto{\pgfqpoint{2.567487in}{1.428433in}}%
\pgfpathlineto{\pgfqpoint{2.578768in}{1.422193in}}%
\pgfpathlineto{\pgfqpoint{2.580971in}{1.421093in}}%
\pgfpathlineto{\pgfqpoint{2.590048in}{1.416595in}}%
\pgfpathlineto{\pgfqpoint{2.601329in}{1.411237in}}%
\pgfpathlineto{\pgfqpoint{2.603999in}{1.410024in}}%
\pgfpathlineto{\pgfqpoint{2.612610in}{1.406158in}}%
\pgfpathlineto{\pgfqpoint{2.623891in}{1.401256in}}%
\pgfpathlineto{\pgfqpoint{2.629351in}{1.398954in}}%
\pgfpathlineto{\pgfqpoint{2.635172in}{1.396536in}}%
\pgfpathlineto{\pgfqpoint{2.646453in}{1.391266in}}%
\pgfpathlineto{\pgfqpoint{2.653231in}{1.387885in}}%
\pgfpathlineto{\pgfqpoint{2.657733in}{1.385695in}}%
\pgfpathlineto{\pgfqpoint{2.669014in}{1.380696in}}%
\pgfpathlineto{\pgfqpoint{2.679734in}{1.376815in}}%
\pgfpathclose%
\pgfusepath{fill}%
\end{pgfscope}%
\begin{pgfscope}%
\pgfpathrectangle{\pgfqpoint{1.856795in}{0.423750in}}{\pgfqpoint{1.194205in}{1.163386in}}%
\pgfusepath{clip}%
\pgfsetbuttcap%
\pgfsetroundjoin%
\definecolor{currentfill}{rgb}{0.823415,0.125353,0.296370}%
\pgfsetfillcolor{currentfill}%
\pgfsetlinewidth{0.000000pt}%
\definecolor{currentstroke}{rgb}{0.000000,0.000000,0.000000}%
\pgfsetstrokecolor{currentstroke}%
\pgfsetdash{}{0pt}%
\pgfpathmoveto{\pgfqpoint{2.274186in}{0.501602in}}%
\pgfpathlineto{\pgfqpoint{2.284016in}{0.491252in}}%
\pgfpathlineto{\pgfqpoint{2.285466in}{0.491252in}}%
\pgfpathlineto{\pgfqpoint{2.296747in}{0.491252in}}%
\pgfpathlineto{\pgfqpoint{2.308028in}{0.491252in}}%
\pgfpathlineto{\pgfqpoint{2.319309in}{0.491252in}}%
\pgfpathlineto{\pgfqpoint{2.323440in}{0.491252in}}%
\pgfpathlineto{\pgfqpoint{2.319309in}{0.495992in}}%
\pgfpathlineto{\pgfqpoint{2.313861in}{0.502321in}}%
\pgfpathlineto{\pgfqpoint{2.308028in}{0.508836in}}%
\pgfpathlineto{\pgfqpoint{2.303871in}{0.513391in}}%
\pgfpathlineto{\pgfqpoint{2.296747in}{0.520884in}}%
\pgfpathlineto{\pgfqpoint{2.293277in}{0.524460in}}%
\pgfpathlineto{\pgfqpoint{2.285466in}{0.532330in}}%
\pgfpathlineto{\pgfqpoint{2.282248in}{0.535530in}}%
\pgfpathlineto{\pgfqpoint{2.274186in}{0.543328in}}%
\pgfpathlineto{\pgfqpoint{2.270673in}{0.546600in}}%
\pgfpathlineto{\pgfqpoint{2.262905in}{0.553700in}}%
\pgfpathlineto{\pgfqpoint{2.258549in}{0.557669in}}%
\pgfpathlineto{\pgfqpoint{2.251624in}{0.563630in}}%
\pgfpathlineto{\pgfqpoint{2.245564in}{0.568739in}}%
\pgfpathlineto{\pgfqpoint{2.240343in}{0.573045in}}%
\pgfpathlineto{\pgfqpoint{2.231895in}{0.579808in}}%
\pgfpathlineto{\pgfqpoint{2.229062in}{0.581995in}}%
\pgfpathlineto{\pgfqpoint{2.217782in}{0.590513in}}%
\pgfpathlineto{\pgfqpoint{2.217295in}{0.590878in}}%
\pgfpathlineto{\pgfqpoint{2.206501in}{0.598836in}}%
\pgfpathlineto{\pgfqpoint{2.202159in}{0.601947in}}%
\pgfpathlineto{\pgfqpoint{2.195220in}{0.606662in}}%
\pgfpathlineto{\pgfqpoint{2.185947in}{0.613017in}}%
\pgfpathlineto{\pgfqpoint{2.183939in}{0.614328in}}%
\pgfpathlineto{\pgfqpoint{2.172658in}{0.622369in}}%
\pgfpathlineto{\pgfqpoint{2.170294in}{0.624086in}}%
\pgfpathlineto{\pgfqpoint{2.161377in}{0.630537in}}%
\pgfpathlineto{\pgfqpoint{2.155077in}{0.635156in}}%
\pgfpathlineto{\pgfqpoint{2.150097in}{0.638754in}}%
\pgfpathlineto{\pgfqpoint{2.140111in}{0.646225in}}%
\pgfpathlineto{\pgfqpoint{2.138816in}{0.647168in}}%
\pgfpathlineto{\pgfqpoint{2.127535in}{0.655764in}}%
\pgfpathlineto{\pgfqpoint{2.125538in}{0.657295in}}%
\pgfpathlineto{\pgfqpoint{2.116254in}{0.664409in}}%
\pgfpathlineto{\pgfqpoint{2.111127in}{0.668364in}}%
\pgfpathlineto{\pgfqpoint{2.104973in}{0.673141in}}%
\pgfpathlineto{\pgfqpoint{2.096897in}{0.679434in}}%
\pgfpathlineto{\pgfqpoint{2.093693in}{0.681911in}}%
\pgfpathlineto{\pgfqpoint{2.083219in}{0.690504in}}%
\pgfpathlineto{\pgfqpoint{2.082412in}{0.691134in}}%
\pgfpathlineto{\pgfqpoint{2.071131in}{0.700400in}}%
\pgfpathlineto{\pgfqpoint{2.069760in}{0.701573in}}%
\pgfpathlineto{\pgfqpoint{2.059850in}{0.710012in}}%
\pgfpathlineto{\pgfqpoint{2.056757in}{0.712643in}}%
\pgfpathlineto{\pgfqpoint{2.048569in}{0.719587in}}%
\pgfpathlineto{\pgfqpoint{2.043722in}{0.723712in}}%
\pgfpathlineto{\pgfqpoint{2.037289in}{0.729337in}}%
\pgfpathlineto{\pgfqpoint{2.031045in}{0.734782in}}%
\pgfpathlineto{\pgfqpoint{2.026008in}{0.739180in}}%
\pgfpathlineto{\pgfqpoint{2.018716in}{0.745851in}}%
\pgfpathlineto{\pgfqpoint{2.014727in}{0.749509in}}%
\pgfpathlineto{\pgfqpoint{2.006626in}{0.756921in}}%
\pgfpathlineto{\pgfqpoint{2.003446in}{0.759826in}}%
\pgfpathlineto{\pgfqpoint{1.994506in}{0.767990in}}%
\pgfpathlineto{\pgfqpoint{1.992165in}{0.770118in}}%
\pgfpathlineto{\pgfqpoint{1.982286in}{0.779060in}}%
\pgfpathlineto{\pgfqpoint{1.980884in}{0.780318in}}%
\pgfpathlineto{\pgfqpoint{1.969916in}{0.790129in}}%
\pgfpathlineto{\pgfqpoint{1.969604in}{0.790407in}}%
\pgfpathlineto{\pgfqpoint{1.958323in}{0.800429in}}%
\pgfpathlineto{\pgfqpoint{1.957452in}{0.801199in}}%
\pgfpathlineto{\pgfqpoint{1.947042in}{0.810494in}}%
\pgfpathlineto{\pgfqpoint{1.945044in}{0.812269in}}%
\pgfpathlineto{\pgfqpoint{1.935761in}{0.820575in}}%
\pgfpathlineto{\pgfqpoint{1.932693in}{0.823338in}}%
\pgfpathlineto{\pgfqpoint{1.924480in}{0.830598in}}%
\pgfpathlineto{\pgfqpoint{1.920132in}{0.834408in}}%
\pgfpathlineto{\pgfqpoint{1.913200in}{0.840658in}}%
\pgfpathlineto{\pgfqpoint{1.907580in}{0.845477in}}%
\pgfpathlineto{\pgfqpoint{1.901919in}{0.850405in}}%
\pgfpathlineto{\pgfqpoint{1.894689in}{0.856547in}}%
\pgfpathlineto{\pgfqpoint{1.890638in}{0.860022in}}%
\pgfpathlineto{\pgfqpoint{1.881743in}{0.867616in}}%
\pgfpathlineto{\pgfqpoint{1.879357in}{0.869631in}}%
\pgfpathlineto{\pgfqpoint{1.868180in}{0.878686in}}%
\pgfpathlineto{\pgfqpoint{1.868076in}{0.878771in}}%
\pgfpathlineto{\pgfqpoint{1.856795in}{0.887935in}}%
\pgfpathlineto{\pgfqpoint{1.856795in}{0.878686in}}%
\pgfpathlineto{\pgfqpoint{1.856795in}{0.867616in}}%
\pgfpathlineto{\pgfqpoint{1.856795in}{0.860159in}}%
\pgfpathlineto{\pgfqpoint{1.861261in}{0.856547in}}%
\pgfpathlineto{\pgfqpoint{1.868076in}{0.851307in}}%
\pgfpathlineto{\pgfqpoint{1.875227in}{0.845477in}}%
\pgfpathlineto{\pgfqpoint{1.879357in}{0.842246in}}%
\pgfpathlineto{\pgfqpoint{1.889402in}{0.834408in}}%
\pgfpathlineto{\pgfqpoint{1.890638in}{0.833409in}}%
\pgfpathlineto{\pgfqpoint{1.901919in}{0.824028in}}%
\pgfpathlineto{\pgfqpoint{1.902726in}{0.823338in}}%
\pgfpathlineto{\pgfqpoint{1.913200in}{0.814264in}}%
\pgfpathlineto{\pgfqpoint{1.915451in}{0.812269in}}%
\pgfpathlineto{\pgfqpoint{1.924480in}{0.804302in}}%
\pgfpathlineto{\pgfqpoint{1.927979in}{0.801199in}}%
\pgfpathlineto{\pgfqpoint{1.935761in}{0.794375in}}%
\pgfpathlineto{\pgfqpoint{1.940594in}{0.790129in}}%
\pgfpathlineto{\pgfqpoint{1.947042in}{0.784470in}}%
\pgfpathlineto{\pgfqpoint{1.953166in}{0.779060in}}%
\pgfpathlineto{\pgfqpoint{1.958323in}{0.774475in}}%
\pgfpathlineto{\pgfqpoint{1.965569in}{0.767990in}}%
\pgfpathlineto{\pgfqpoint{1.969604in}{0.764364in}}%
\pgfpathlineto{\pgfqpoint{1.977840in}{0.756921in}}%
\pgfpathlineto{\pgfqpoint{1.980884in}{0.754162in}}%
\pgfpathlineto{\pgfqpoint{1.990009in}{0.745851in}}%
\pgfpathlineto{\pgfqpoint{1.992165in}{0.743891in}}%
\pgfpathlineto{\pgfqpoint{2.002180in}{0.734782in}}%
\pgfpathlineto{\pgfqpoint{2.003446in}{0.733653in}}%
\pgfpathlineto{\pgfqpoint{2.014508in}{0.723712in}}%
\pgfpathlineto{\pgfqpoint{2.014727in}{0.723517in}}%
\pgfpathlineto{\pgfqpoint{2.026008in}{0.713392in}}%
\pgfpathlineto{\pgfqpoint{2.026885in}{0.712643in}}%
\pgfpathlineto{\pgfqpoint{2.037289in}{0.703809in}}%
\pgfpathlineto{\pgfqpoint{2.039933in}{0.701573in}}%
\pgfpathlineto{\pgfqpoint{2.048569in}{0.694225in}}%
\pgfpathlineto{\pgfqpoint{2.052934in}{0.690504in}}%
\pgfpathlineto{\pgfqpoint{2.059850in}{0.684610in}}%
\pgfpathlineto{\pgfqpoint{2.065918in}{0.679434in}}%
\pgfpathlineto{\pgfqpoint{2.071131in}{0.674979in}}%
\pgfpathlineto{\pgfqpoint{2.078834in}{0.668364in}}%
\pgfpathlineto{\pgfqpoint{2.082412in}{0.665283in}}%
\pgfpathlineto{\pgfqpoint{2.091671in}{0.657295in}}%
\pgfpathlineto{\pgfqpoint{2.093693in}{0.655561in}}%
\pgfpathlineto{\pgfqpoint{2.104603in}{0.646225in}}%
\pgfpathlineto{\pgfqpoint{2.104973in}{0.645909in}}%
\pgfpathlineto{\pgfqpoint{2.116254in}{0.636649in}}%
\pgfpathlineto{\pgfqpoint{2.118213in}{0.635156in}}%
\pgfpathlineto{\pgfqpoint{2.127535in}{0.628063in}}%
\pgfpathlineto{\pgfqpoint{2.132784in}{0.624086in}}%
\pgfpathlineto{\pgfqpoint{2.138816in}{0.619503in}}%
\pgfpathlineto{\pgfqpoint{2.147342in}{0.613017in}}%
\pgfpathlineto{\pgfqpoint{2.150097in}{0.610915in}}%
\pgfpathlineto{\pgfqpoint{2.161377in}{0.602234in}}%
\pgfpathlineto{\pgfqpoint{2.161745in}{0.601947in}}%
\pgfpathlineto{\pgfqpoint{2.172658in}{0.593190in}}%
\pgfpathlineto{\pgfqpoint{2.175572in}{0.590878in}}%
\pgfpathlineto{\pgfqpoint{2.183939in}{0.583771in}}%
\pgfpathlineto{\pgfqpoint{2.188791in}{0.579808in}}%
\pgfpathlineto{\pgfqpoint{2.195220in}{0.574224in}}%
\pgfpathlineto{\pgfqpoint{2.202082in}{0.568739in}}%
\pgfpathlineto{\pgfqpoint{2.206501in}{0.565107in}}%
\pgfpathlineto{\pgfqpoint{2.215437in}{0.557669in}}%
\pgfpathlineto{\pgfqpoint{2.217782in}{0.555669in}}%
\pgfpathlineto{\pgfqpoint{2.228109in}{0.546600in}}%
\pgfpathlineto{\pgfqpoint{2.229062in}{0.545756in}}%
\pgfpathlineto{\pgfqpoint{2.240293in}{0.535530in}}%
\pgfpathlineto{\pgfqpoint{2.240343in}{0.535482in}}%
\pgfpathlineto{\pgfqpoint{2.251624in}{0.524767in}}%
\pgfpathlineto{\pgfqpoint{2.251944in}{0.524460in}}%
\pgfpathlineto{\pgfqpoint{2.262905in}{0.513609in}}%
\pgfpathlineto{\pgfqpoint{2.263120in}{0.513391in}}%
\pgfpathlineto{\pgfqpoint{2.273522in}{0.502321in}}%
\pgfpathclose%
\pgfusepath{fill}%
\end{pgfscope}%
\begin{pgfscope}%
\pgfpathrectangle{\pgfqpoint{1.856795in}{0.423750in}}{\pgfqpoint{1.194205in}{1.163386in}}%
\pgfusepath{clip}%
\pgfsetbuttcap%
\pgfsetroundjoin%
\definecolor{currentfill}{rgb}{0.823415,0.125353,0.296370}%
\pgfsetfillcolor{currentfill}%
\pgfsetlinewidth{0.000000pt}%
\definecolor{currentstroke}{rgb}{0.000000,0.000000,0.000000}%
\pgfsetstrokecolor{currentstroke}%
\pgfsetdash{}{0pt}%
\pgfpathmoveto{\pgfqpoint{2.691576in}{1.310263in}}%
\pgfpathlineto{\pgfqpoint{2.702857in}{1.307635in}}%
\pgfpathlineto{\pgfqpoint{2.714137in}{1.305782in}}%
\pgfpathlineto{\pgfqpoint{2.725418in}{1.306419in}}%
\pgfpathlineto{\pgfqpoint{2.736699in}{1.310085in}}%
\pgfpathlineto{\pgfqpoint{2.737217in}{1.310398in}}%
\pgfpathlineto{\pgfqpoint{2.747980in}{1.316726in}}%
\pgfpathlineto{\pgfqpoint{2.755552in}{1.321467in}}%
\pgfpathlineto{\pgfqpoint{2.759261in}{1.323799in}}%
\pgfpathlineto{\pgfqpoint{2.770542in}{1.331238in}}%
\pgfpathlineto{\pgfqpoint{2.772296in}{1.332537in}}%
\pgfpathlineto{\pgfqpoint{2.781822in}{1.339471in}}%
\pgfpathlineto{\pgfqpoint{2.787388in}{1.343606in}}%
\pgfpathlineto{\pgfqpoint{2.793103in}{1.347678in}}%
\pgfpathlineto{\pgfqpoint{2.802696in}{1.354676in}}%
\pgfpathlineto{\pgfqpoint{2.804384in}{1.355909in}}%
\pgfpathlineto{\pgfqpoint{2.815665in}{1.364365in}}%
\pgfpathlineto{\pgfqpoint{2.817580in}{1.365746in}}%
\pgfpathlineto{\pgfqpoint{2.826946in}{1.372510in}}%
\pgfpathlineto{\pgfqpoint{2.832712in}{1.376815in}}%
\pgfpathlineto{\pgfqpoint{2.838226in}{1.380924in}}%
\pgfpathlineto{\pgfqpoint{2.847243in}{1.387885in}}%
\pgfpathlineto{\pgfqpoint{2.849507in}{1.389633in}}%
\pgfpathlineto{\pgfqpoint{2.860788in}{1.398634in}}%
\pgfpathlineto{\pgfqpoint{2.861175in}{1.398954in}}%
\pgfpathlineto{\pgfqpoint{2.872069in}{1.407960in}}%
\pgfpathlineto{\pgfqpoint{2.874489in}{1.410024in}}%
\pgfpathlineto{\pgfqpoint{2.883350in}{1.417577in}}%
\pgfpathlineto{\pgfqpoint{2.887357in}{1.421093in}}%
\pgfpathlineto{\pgfqpoint{2.894631in}{1.427472in}}%
\pgfpathlineto{\pgfqpoint{2.899835in}{1.432163in}}%
\pgfpathlineto{\pgfqpoint{2.905911in}{1.437637in}}%
\pgfpathlineto{\pgfqpoint{2.911966in}{1.443232in}}%
\pgfpathlineto{\pgfqpoint{2.917192in}{1.448062in}}%
\pgfpathlineto{\pgfqpoint{2.923788in}{1.454302in}}%
\pgfpathlineto{\pgfqpoint{2.928473in}{1.458735in}}%
\pgfpathlineto{\pgfqpoint{2.935336in}{1.465371in}}%
\pgfpathlineto{\pgfqpoint{2.939754in}{1.469646in}}%
\pgfpathlineto{\pgfqpoint{2.946727in}{1.476441in}}%
\pgfpathlineto{\pgfqpoint{2.951035in}{1.480648in}}%
\pgfpathlineto{\pgfqpoint{2.958063in}{1.487510in}}%
\pgfpathlineto{\pgfqpoint{2.962315in}{1.491666in}}%
\pgfpathlineto{\pgfqpoint{2.969270in}{1.498580in}}%
\pgfpathlineto{\pgfqpoint{2.973596in}{1.502884in}}%
\pgfpathlineto{\pgfqpoint{2.973596in}{1.509650in}}%
\pgfpathlineto{\pgfqpoint{2.973596in}{1.520719in}}%
\pgfpathlineto{\pgfqpoint{2.973596in}{1.531789in}}%
\pgfpathlineto{\pgfqpoint{2.973596in}{1.542858in}}%
\pgfpathlineto{\pgfqpoint{2.973596in}{1.549047in}}%
\pgfpathlineto{\pgfqpoint{2.967464in}{1.542858in}}%
\pgfpathlineto{\pgfqpoint{2.962315in}{1.537677in}}%
\pgfpathlineto{\pgfqpoint{2.956353in}{1.531789in}}%
\pgfpathlineto{\pgfqpoint{2.951035in}{1.526548in}}%
\pgfpathlineto{\pgfqpoint{2.945135in}{1.520719in}}%
\pgfpathlineto{\pgfqpoint{2.939754in}{1.515414in}}%
\pgfpathlineto{\pgfqpoint{2.933831in}{1.509650in}}%
\pgfpathlineto{\pgfqpoint{2.928473in}{1.504453in}}%
\pgfpathlineto{\pgfqpoint{2.922253in}{1.498580in}}%
\pgfpathlineto{\pgfqpoint{2.917192in}{1.493815in}}%
\pgfpathlineto{\pgfqpoint{2.910290in}{1.487510in}}%
\pgfpathlineto{\pgfqpoint{2.905911in}{1.483502in}}%
\pgfpathlineto{\pgfqpoint{2.897933in}{1.476441in}}%
\pgfpathlineto{\pgfqpoint{2.894631in}{1.473509in}}%
\pgfpathlineto{\pgfqpoint{2.885128in}{1.465371in}}%
\pgfpathlineto{\pgfqpoint{2.883350in}{1.463837in}}%
\pgfpathlineto{\pgfqpoint{2.872069in}{1.454515in}}%
\pgfpathlineto{\pgfqpoint{2.871797in}{1.454302in}}%
\pgfpathlineto{\pgfqpoint{2.860788in}{1.445614in}}%
\pgfpathlineto{\pgfqpoint{2.857657in}{1.443232in}}%
\pgfpathlineto{\pgfqpoint{2.849507in}{1.436996in}}%
\pgfpathlineto{\pgfqpoint{2.842955in}{1.432163in}}%
\pgfpathlineto{\pgfqpoint{2.838226in}{1.428653in}}%
\pgfpathlineto{\pgfqpoint{2.827636in}{1.421093in}}%
\pgfpathlineto{\pgfqpoint{2.826946in}{1.420598in}}%
\pgfpathlineto{\pgfqpoint{2.815665in}{1.412477in}}%
\pgfpathlineto{\pgfqpoint{2.812255in}{1.410024in}}%
\pgfpathlineto{\pgfqpoint{2.804384in}{1.404330in}}%
\pgfpathlineto{\pgfqpoint{2.796582in}{1.398954in}}%
\pgfpathlineto{\pgfqpoint{2.793103in}{1.396523in}}%
\pgfpathlineto{\pgfqpoint{2.781822in}{1.389499in}}%
\pgfpathlineto{\pgfqpoint{2.778719in}{1.387885in}}%
\pgfpathlineto{\pgfqpoint{2.770542in}{1.383560in}}%
\pgfpathlineto{\pgfqpoint{2.759261in}{1.377947in}}%
\pgfpathlineto{\pgfqpoint{2.756991in}{1.376815in}}%
\pgfpathlineto{\pgfqpoint{2.747980in}{1.372259in}}%
\pgfpathlineto{\pgfqpoint{2.736699in}{1.368996in}}%
\pgfpathlineto{\pgfqpoint{2.725418in}{1.368252in}}%
\pgfpathlineto{\pgfqpoint{2.714137in}{1.368492in}}%
\pgfpathlineto{\pgfqpoint{2.702857in}{1.370318in}}%
\pgfpathlineto{\pgfqpoint{2.691576in}{1.373032in}}%
\pgfpathlineto{\pgfqpoint{2.680295in}{1.376614in}}%
\pgfpathlineto{\pgfqpoint{2.679734in}{1.376815in}}%
\pgfpathlineto{\pgfqpoint{2.669014in}{1.380696in}}%
\pgfpathlineto{\pgfqpoint{2.657733in}{1.385695in}}%
\pgfpathlineto{\pgfqpoint{2.653231in}{1.387885in}}%
\pgfpathlineto{\pgfqpoint{2.646453in}{1.391266in}}%
\pgfpathlineto{\pgfqpoint{2.635172in}{1.396536in}}%
\pgfpathlineto{\pgfqpoint{2.629351in}{1.398954in}}%
\pgfpathlineto{\pgfqpoint{2.623891in}{1.401256in}}%
\pgfpathlineto{\pgfqpoint{2.612610in}{1.406158in}}%
\pgfpathlineto{\pgfqpoint{2.603999in}{1.410024in}}%
\pgfpathlineto{\pgfqpoint{2.601329in}{1.411237in}}%
\pgfpathlineto{\pgfqpoint{2.590048in}{1.416595in}}%
\pgfpathlineto{\pgfqpoint{2.580971in}{1.421093in}}%
\pgfpathlineto{\pgfqpoint{2.578768in}{1.422193in}}%
\pgfpathlineto{\pgfqpoint{2.567487in}{1.428433in}}%
\pgfpathlineto{\pgfqpoint{2.560930in}{1.432163in}}%
\pgfpathlineto{\pgfqpoint{2.556206in}{1.434882in}}%
\pgfpathlineto{\pgfqpoint{2.544925in}{1.441886in}}%
\pgfpathlineto{\pgfqpoint{2.543149in}{1.443232in}}%
\pgfpathlineto{\pgfqpoint{2.533644in}{1.450852in}}%
\pgfpathlineto{\pgfqpoint{2.529729in}{1.454302in}}%
\pgfpathlineto{\pgfqpoint{2.522364in}{1.460769in}}%
\pgfpathlineto{\pgfqpoint{2.517379in}{1.465371in}}%
\pgfpathlineto{\pgfqpoint{2.511083in}{1.471233in}}%
\pgfpathlineto{\pgfqpoint{2.505561in}{1.476441in}}%
\pgfpathlineto{\pgfqpoint{2.499802in}{1.481924in}}%
\pgfpathlineto{\pgfqpoint{2.494037in}{1.487510in}}%
\pgfpathlineto{\pgfqpoint{2.488521in}{1.492912in}}%
\pgfpathlineto{\pgfqpoint{2.482918in}{1.498580in}}%
\pgfpathlineto{\pgfqpoint{2.477240in}{1.504426in}}%
\pgfpathlineto{\pgfqpoint{2.472443in}{1.509650in}}%
\pgfpathlineto{\pgfqpoint{2.465960in}{1.516921in}}%
\pgfpathlineto{\pgfqpoint{2.462616in}{1.520719in}}%
\pgfpathlineto{\pgfqpoint{2.454679in}{1.529761in}}%
\pgfpathlineto{\pgfqpoint{2.452944in}{1.531789in}}%
\pgfpathlineto{\pgfqpoint{2.443487in}{1.542858in}}%
\pgfpathlineto{\pgfqpoint{2.443398in}{1.542964in}}%
\pgfpathlineto{\pgfqpoint{2.434296in}{1.553928in}}%
\pgfpathlineto{\pgfqpoint{2.432117in}{1.556556in}}%
\pgfpathlineto{\pgfqpoint{2.425133in}{1.564997in}}%
\pgfpathlineto{\pgfqpoint{2.420836in}{1.570188in}}%
\pgfpathlineto{\pgfqpoint{2.416005in}{1.576067in}}%
\pgfpathlineto{\pgfqpoint{2.409555in}{1.583965in}}%
\pgfpathlineto{\pgfqpoint{2.406990in}{1.587136in}}%
\pgfpathlineto{\pgfqpoint{2.398275in}{1.587136in}}%
\pgfpathlineto{\pgfqpoint{2.386994in}{1.587136in}}%
\pgfpathlineto{\pgfqpoint{2.375713in}{1.587136in}}%
\pgfpathlineto{\pgfqpoint{2.364432in}{1.587136in}}%
\pgfpathlineto{\pgfqpoint{2.353151in}{1.587136in}}%
\pgfpathlineto{\pgfqpoint{2.347574in}{1.587136in}}%
\pgfpathlineto{\pgfqpoint{2.353151in}{1.579841in}}%
\pgfpathlineto{\pgfqpoint{2.356139in}{1.576067in}}%
\pgfpathlineto{\pgfqpoint{2.364432in}{1.565692in}}%
\pgfpathlineto{\pgfqpoint{2.365000in}{1.564997in}}%
\pgfpathlineto{\pgfqpoint{2.374130in}{1.553928in}}%
\pgfpathlineto{\pgfqpoint{2.375713in}{1.552014in}}%
\pgfpathlineto{\pgfqpoint{2.383316in}{1.542858in}}%
\pgfpathlineto{\pgfqpoint{2.386994in}{1.538488in}}%
\pgfpathlineto{\pgfqpoint{2.392596in}{1.531789in}}%
\pgfpathlineto{\pgfqpoint{2.398275in}{1.525063in}}%
\pgfpathlineto{\pgfqpoint{2.401955in}{1.520719in}}%
\pgfpathlineto{\pgfqpoint{2.409555in}{1.511826in}}%
\pgfpathlineto{\pgfqpoint{2.411414in}{1.509650in}}%
\pgfpathlineto{\pgfqpoint{2.420836in}{1.498658in}}%
\pgfpathlineto{\pgfqpoint{2.420903in}{1.498580in}}%
\pgfpathlineto{\pgfqpoint{2.430459in}{1.487510in}}%
\pgfpathlineto{\pgfqpoint{2.432117in}{1.485610in}}%
\pgfpathlineto{\pgfqpoint{2.440182in}{1.476441in}}%
\pgfpathlineto{\pgfqpoint{2.443398in}{1.472824in}}%
\pgfpathlineto{\pgfqpoint{2.450295in}{1.465371in}}%
\pgfpathlineto{\pgfqpoint{2.454679in}{1.460861in}}%
\pgfpathlineto{\pgfqpoint{2.461224in}{1.454302in}}%
\pgfpathlineto{\pgfqpoint{2.465960in}{1.449625in}}%
\pgfpathlineto{\pgfqpoint{2.472565in}{1.443232in}}%
\pgfpathlineto{\pgfqpoint{2.477240in}{1.438744in}}%
\pgfpathlineto{\pgfqpoint{2.484155in}{1.432163in}}%
\pgfpathlineto{\pgfqpoint{2.488521in}{1.428061in}}%
\pgfpathlineto{\pgfqpoint{2.495981in}{1.421093in}}%
\pgfpathlineto{\pgfqpoint{2.499802in}{1.417508in}}%
\pgfpathlineto{\pgfqpoint{2.507925in}{1.410024in}}%
\pgfpathlineto{\pgfqpoint{2.511083in}{1.407154in}}%
\pgfpathlineto{\pgfqpoint{2.520489in}{1.398954in}}%
\pgfpathlineto{\pgfqpoint{2.522364in}{1.397354in}}%
\pgfpathlineto{\pgfqpoint{2.533644in}{1.388458in}}%
\pgfpathlineto{\pgfqpoint{2.534465in}{1.387885in}}%
\pgfpathlineto{\pgfqpoint{2.544925in}{1.380528in}}%
\pgfpathlineto{\pgfqpoint{2.550763in}{1.376815in}}%
\pgfpathlineto{\pgfqpoint{2.556206in}{1.373317in}}%
\pgfpathlineto{\pgfqpoint{2.567487in}{1.366914in}}%
\pgfpathlineto{\pgfqpoint{2.569606in}{1.365746in}}%
\pgfpathlineto{\pgfqpoint{2.578768in}{1.360735in}}%
\pgfpathlineto{\pgfqpoint{2.590043in}{1.354676in}}%
\pgfpathlineto{\pgfqpoint{2.590048in}{1.354673in}}%
\pgfpathlineto{\pgfqpoint{2.601329in}{1.348816in}}%
\pgfpathlineto{\pgfqpoint{2.611527in}{1.343606in}}%
\pgfpathlineto{\pgfqpoint{2.612610in}{1.343059in}}%
\pgfpathlineto{\pgfqpoint{2.623891in}{1.337846in}}%
\pgfpathlineto{\pgfqpoint{2.635172in}{1.333069in}}%
\pgfpathlineto{\pgfqpoint{2.636507in}{1.332537in}}%
\pgfpathlineto{\pgfqpoint{2.646453in}{1.328610in}}%
\pgfpathlineto{\pgfqpoint{2.657733in}{1.323215in}}%
\pgfpathlineto{\pgfqpoint{2.661484in}{1.321467in}}%
\pgfpathlineto{\pgfqpoint{2.669014in}{1.318007in}}%
\pgfpathlineto{\pgfqpoint{2.680295in}{1.313823in}}%
\pgfpathlineto{\pgfqpoint{2.691141in}{1.310398in}}%
\pgfpathclose%
\pgfusepath{fill}%
\end{pgfscope}%
\begin{pgfscope}%
\pgfpathrectangle{\pgfqpoint{1.856795in}{0.423750in}}{\pgfqpoint{1.194205in}{1.163386in}}%
\pgfusepath{clip}%
\pgfsetbuttcap%
\pgfsetroundjoin%
\definecolor{currentfill}{rgb}{0.879259,0.192033,0.262681}%
\pgfsetfillcolor{currentfill}%
\pgfsetlinewidth{0.000000pt}%
\definecolor{currentstroke}{rgb}{0.000000,0.000000,0.000000}%
\pgfsetstrokecolor{currentstroke}%
\pgfsetdash{}{0pt}%
\pgfpathmoveto{\pgfqpoint{2.319309in}{0.495992in}}%
\pgfpathlineto{\pgfqpoint{2.323440in}{0.491252in}}%
\pgfpathlineto{\pgfqpoint{2.330590in}{0.491252in}}%
\pgfpathlineto{\pgfqpoint{2.341871in}{0.491252in}}%
\pgfpathlineto{\pgfqpoint{2.353151in}{0.491252in}}%
\pgfpathlineto{\pgfqpoint{2.362159in}{0.491252in}}%
\pgfpathlineto{\pgfqpoint{2.353523in}{0.502321in}}%
\pgfpathlineto{\pgfqpoint{2.353151in}{0.502792in}}%
\pgfpathlineto{\pgfqpoint{2.344846in}{0.513391in}}%
\pgfpathlineto{\pgfqpoint{2.341871in}{0.517011in}}%
\pgfpathlineto{\pgfqpoint{2.335330in}{0.524460in}}%
\pgfpathlineto{\pgfqpoint{2.330590in}{0.529543in}}%
\pgfpathlineto{\pgfqpoint{2.324934in}{0.535530in}}%
\pgfpathlineto{\pgfqpoint{2.319309in}{0.541484in}}%
\pgfpathlineto{\pgfqpoint{2.314371in}{0.546600in}}%
\pgfpathlineto{\pgfqpoint{2.308028in}{0.553033in}}%
\pgfpathlineto{\pgfqpoint{2.303367in}{0.557669in}}%
\pgfpathlineto{\pgfqpoint{2.296747in}{0.564084in}}%
\pgfpathlineto{\pgfqpoint{2.291741in}{0.568739in}}%
\pgfpathlineto{\pgfqpoint{2.285466in}{0.574345in}}%
\pgfpathlineto{\pgfqpoint{2.279229in}{0.579808in}}%
\pgfpathlineto{\pgfqpoint{2.274186in}{0.584058in}}%
\pgfpathlineto{\pgfqpoint{2.265758in}{0.590878in}}%
\pgfpathlineto{\pgfqpoint{2.262905in}{0.593106in}}%
\pgfpathlineto{\pgfqpoint{2.251624in}{0.601646in}}%
\pgfpathlineto{\pgfqpoint{2.251210in}{0.601947in}}%
\pgfpathlineto{\pgfqpoint{2.240343in}{0.609609in}}%
\pgfpathlineto{\pgfqpoint{2.235610in}{0.613017in}}%
\pgfpathlineto{\pgfqpoint{2.229062in}{0.617458in}}%
\pgfpathlineto{\pgfqpoint{2.219100in}{0.624086in}}%
\pgfpathlineto{\pgfqpoint{2.217782in}{0.624929in}}%
\pgfpathlineto{\pgfqpoint{2.206501in}{0.632053in}}%
\pgfpathlineto{\pgfqpoint{2.201641in}{0.635156in}}%
\pgfpathlineto{\pgfqpoint{2.195220in}{0.639186in}}%
\pgfpathlineto{\pgfqpoint{2.183939in}{0.646211in}}%
\pgfpathlineto{\pgfqpoint{2.183917in}{0.646225in}}%
\pgfpathlineto{\pgfqpoint{2.172658in}{0.653160in}}%
\pgfpathlineto{\pgfqpoint{2.166170in}{0.657295in}}%
\pgfpathlineto{\pgfqpoint{2.161377in}{0.660292in}}%
\pgfpathlineto{\pgfqpoint{2.150097in}{0.667707in}}%
\pgfpathlineto{\pgfqpoint{2.149190in}{0.668364in}}%
\pgfpathlineto{\pgfqpoint{2.138816in}{0.675844in}}%
\pgfpathlineto{\pgfqpoint{2.133838in}{0.679434in}}%
\pgfpathlineto{\pgfqpoint{2.127535in}{0.683960in}}%
\pgfpathlineto{\pgfqpoint{2.118470in}{0.690504in}}%
\pgfpathlineto{\pgfqpoint{2.116254in}{0.692086in}}%
\pgfpathlineto{\pgfqpoint{2.104973in}{0.700748in}}%
\pgfpathlineto{\pgfqpoint{2.103905in}{0.701573in}}%
\pgfpathlineto{\pgfqpoint{2.093693in}{0.709357in}}%
\pgfpathlineto{\pgfqpoint{2.089342in}{0.712643in}}%
\pgfpathlineto{\pgfqpoint{2.082412in}{0.717847in}}%
\pgfpathlineto{\pgfqpoint{2.074624in}{0.723712in}}%
\pgfpathlineto{\pgfqpoint{2.071131in}{0.726359in}}%
\pgfpathlineto{\pgfqpoint{2.060775in}{0.734782in}}%
\pgfpathlineto{\pgfqpoint{2.059850in}{0.735500in}}%
\pgfpathlineto{\pgfqpoint{2.048569in}{0.744958in}}%
\pgfpathlineto{\pgfqpoint{2.047552in}{0.745851in}}%
\pgfpathlineto{\pgfqpoint{2.037289in}{0.754862in}}%
\pgfpathlineto{\pgfqpoint{2.034943in}{0.756921in}}%
\pgfpathlineto{\pgfqpoint{2.026008in}{0.764695in}}%
\pgfpathlineto{\pgfqpoint{2.022387in}{0.767990in}}%
\pgfpathlineto{\pgfqpoint{2.014727in}{0.774878in}}%
\pgfpathlineto{\pgfqpoint{2.010125in}{0.779060in}}%
\pgfpathlineto{\pgfqpoint{2.003446in}{0.785064in}}%
\pgfpathlineto{\pgfqpoint{1.997820in}{0.790129in}}%
\pgfpathlineto{\pgfqpoint{1.992165in}{0.795156in}}%
\pgfpathlineto{\pgfqpoint{1.985377in}{0.801199in}}%
\pgfpathlineto{\pgfqpoint{1.980884in}{0.805203in}}%
\pgfpathlineto{\pgfqpoint{1.973006in}{0.812269in}}%
\pgfpathlineto{\pgfqpoint{1.969604in}{0.815284in}}%
\pgfpathlineto{\pgfqpoint{1.960534in}{0.823338in}}%
\pgfpathlineto{\pgfqpoint{1.958323in}{0.825359in}}%
\pgfpathlineto{\pgfqpoint{1.948301in}{0.834408in}}%
\pgfpathlineto{\pgfqpoint{1.947042in}{0.835565in}}%
\pgfpathlineto{\pgfqpoint{1.936355in}{0.845477in}}%
\pgfpathlineto{\pgfqpoint{1.935761in}{0.846018in}}%
\pgfpathlineto{\pgfqpoint{1.924480in}{0.856169in}}%
\pgfpathlineto{\pgfqpoint{1.924057in}{0.856547in}}%
\pgfpathlineto{\pgfqpoint{1.913200in}{0.866073in}}%
\pgfpathlineto{\pgfqpoint{1.911425in}{0.867616in}}%
\pgfpathlineto{\pgfqpoint{1.901919in}{0.875840in}}%
\pgfpathlineto{\pgfqpoint{1.898673in}{0.878686in}}%
\pgfpathlineto{\pgfqpoint{1.890638in}{0.885551in}}%
\pgfpathlineto{\pgfqpoint{1.885624in}{0.889755in}}%
\pgfpathlineto{\pgfqpoint{1.879357in}{0.895050in}}%
\pgfpathlineto{\pgfqpoint{1.872345in}{0.900825in}}%
\pgfpathlineto{\pgfqpoint{1.868076in}{0.904510in}}%
\pgfpathlineto{\pgfqpoint{1.859161in}{0.911894in}}%
\pgfpathlineto{\pgfqpoint{1.856795in}{0.913929in}}%
\pgfpathlineto{\pgfqpoint{1.856795in}{0.911894in}}%
\pgfpathlineto{\pgfqpoint{1.856795in}{0.900825in}}%
\pgfpathlineto{\pgfqpoint{1.856795in}{0.889755in}}%
\pgfpathlineto{\pgfqpoint{1.856795in}{0.887935in}}%
\pgfpathlineto{\pgfqpoint{1.868076in}{0.878771in}}%
\pgfpathlineto{\pgfqpoint{1.868180in}{0.878686in}}%
\pgfpathlineto{\pgfqpoint{1.879357in}{0.869631in}}%
\pgfpathlineto{\pgfqpoint{1.881743in}{0.867616in}}%
\pgfpathlineto{\pgfqpoint{1.890638in}{0.860022in}}%
\pgfpathlineto{\pgfqpoint{1.894689in}{0.856547in}}%
\pgfpathlineto{\pgfqpoint{1.901919in}{0.850405in}}%
\pgfpathlineto{\pgfqpoint{1.907580in}{0.845477in}}%
\pgfpathlineto{\pgfqpoint{1.913200in}{0.840658in}}%
\pgfpathlineto{\pgfqpoint{1.920132in}{0.834408in}}%
\pgfpathlineto{\pgfqpoint{1.924480in}{0.830598in}}%
\pgfpathlineto{\pgfqpoint{1.932693in}{0.823338in}}%
\pgfpathlineto{\pgfqpoint{1.935761in}{0.820575in}}%
\pgfpathlineto{\pgfqpoint{1.945044in}{0.812269in}}%
\pgfpathlineto{\pgfqpoint{1.947042in}{0.810494in}}%
\pgfpathlineto{\pgfqpoint{1.957452in}{0.801199in}}%
\pgfpathlineto{\pgfqpoint{1.958323in}{0.800429in}}%
\pgfpathlineto{\pgfqpoint{1.969604in}{0.790407in}}%
\pgfpathlineto{\pgfqpoint{1.969916in}{0.790129in}}%
\pgfpathlineto{\pgfqpoint{1.980884in}{0.780318in}}%
\pgfpathlineto{\pgfqpoint{1.982286in}{0.779060in}}%
\pgfpathlineto{\pgfqpoint{1.992165in}{0.770118in}}%
\pgfpathlineto{\pgfqpoint{1.994506in}{0.767990in}}%
\pgfpathlineto{\pgfqpoint{2.003446in}{0.759826in}}%
\pgfpathlineto{\pgfqpoint{2.006626in}{0.756921in}}%
\pgfpathlineto{\pgfqpoint{2.014727in}{0.749509in}}%
\pgfpathlineto{\pgfqpoint{2.018716in}{0.745851in}}%
\pgfpathlineto{\pgfqpoint{2.026008in}{0.739180in}}%
\pgfpathlineto{\pgfqpoint{2.031045in}{0.734782in}}%
\pgfpathlineto{\pgfqpoint{2.037289in}{0.729337in}}%
\pgfpathlineto{\pgfqpoint{2.043722in}{0.723712in}}%
\pgfpathlineto{\pgfqpoint{2.048569in}{0.719587in}}%
\pgfpathlineto{\pgfqpoint{2.056757in}{0.712643in}}%
\pgfpathlineto{\pgfqpoint{2.059850in}{0.710012in}}%
\pgfpathlineto{\pgfqpoint{2.069760in}{0.701573in}}%
\pgfpathlineto{\pgfqpoint{2.071131in}{0.700400in}}%
\pgfpathlineto{\pgfqpoint{2.082412in}{0.691134in}}%
\pgfpathlineto{\pgfqpoint{2.083219in}{0.690504in}}%
\pgfpathlineto{\pgfqpoint{2.093693in}{0.681911in}}%
\pgfpathlineto{\pgfqpoint{2.096897in}{0.679434in}}%
\pgfpathlineto{\pgfqpoint{2.104973in}{0.673141in}}%
\pgfpathlineto{\pgfqpoint{2.111127in}{0.668364in}}%
\pgfpathlineto{\pgfqpoint{2.116254in}{0.664409in}}%
\pgfpathlineto{\pgfqpoint{2.125538in}{0.657295in}}%
\pgfpathlineto{\pgfqpoint{2.127535in}{0.655764in}}%
\pgfpathlineto{\pgfqpoint{2.138816in}{0.647168in}}%
\pgfpathlineto{\pgfqpoint{2.140111in}{0.646225in}}%
\pgfpathlineto{\pgfqpoint{2.150097in}{0.638754in}}%
\pgfpathlineto{\pgfqpoint{2.155077in}{0.635156in}}%
\pgfpathlineto{\pgfqpoint{2.161377in}{0.630537in}}%
\pgfpathlineto{\pgfqpoint{2.170294in}{0.624086in}}%
\pgfpathlineto{\pgfqpoint{2.172658in}{0.622369in}}%
\pgfpathlineto{\pgfqpoint{2.183939in}{0.614328in}}%
\pgfpathlineto{\pgfqpoint{2.185947in}{0.613017in}}%
\pgfpathlineto{\pgfqpoint{2.195220in}{0.606662in}}%
\pgfpathlineto{\pgfqpoint{2.202159in}{0.601947in}}%
\pgfpathlineto{\pgfqpoint{2.206501in}{0.598836in}}%
\pgfpathlineto{\pgfqpoint{2.217295in}{0.590878in}}%
\pgfpathlineto{\pgfqpoint{2.217782in}{0.590513in}}%
\pgfpathlineto{\pgfqpoint{2.229062in}{0.581995in}}%
\pgfpathlineto{\pgfqpoint{2.231895in}{0.579808in}}%
\pgfpathlineto{\pgfqpoint{2.240343in}{0.573045in}}%
\pgfpathlineto{\pgfqpoint{2.245564in}{0.568739in}}%
\pgfpathlineto{\pgfqpoint{2.251624in}{0.563630in}}%
\pgfpathlineto{\pgfqpoint{2.258549in}{0.557669in}}%
\pgfpathlineto{\pgfqpoint{2.262905in}{0.553700in}}%
\pgfpathlineto{\pgfqpoint{2.270673in}{0.546600in}}%
\pgfpathlineto{\pgfqpoint{2.274186in}{0.543328in}}%
\pgfpathlineto{\pgfqpoint{2.282248in}{0.535530in}}%
\pgfpathlineto{\pgfqpoint{2.285466in}{0.532330in}}%
\pgfpathlineto{\pgfqpoint{2.293277in}{0.524460in}}%
\pgfpathlineto{\pgfqpoint{2.296747in}{0.520884in}}%
\pgfpathlineto{\pgfqpoint{2.303871in}{0.513391in}}%
\pgfpathlineto{\pgfqpoint{2.308028in}{0.508836in}}%
\pgfpathlineto{\pgfqpoint{2.313861in}{0.502321in}}%
\pgfpathclose%
\pgfusepath{fill}%
\end{pgfscope}%
\begin{pgfscope}%
\pgfpathrectangle{\pgfqpoint{1.856795in}{0.423750in}}{\pgfqpoint{1.194205in}{1.163386in}}%
\pgfusepath{clip}%
\pgfsetbuttcap%
\pgfsetroundjoin%
\definecolor{currentfill}{rgb}{0.879259,0.192033,0.262681}%
\pgfsetfillcolor{currentfill}%
\pgfsetlinewidth{0.000000pt}%
\definecolor{currentstroke}{rgb}{0.000000,0.000000,0.000000}%
\pgfsetstrokecolor{currentstroke}%
\pgfsetdash{}{0pt}%
\pgfpathmoveto{\pgfqpoint{2.691576in}{1.253994in}}%
\pgfpathlineto{\pgfqpoint{2.702857in}{1.252590in}}%
\pgfpathlineto{\pgfqpoint{2.714137in}{1.251806in}}%
\pgfpathlineto{\pgfqpoint{2.725418in}{1.254655in}}%
\pgfpathlineto{\pgfqpoint{2.726096in}{1.255050in}}%
\pgfpathlineto{\pgfqpoint{2.736699in}{1.261081in}}%
\pgfpathlineto{\pgfqpoint{2.744282in}{1.266120in}}%
\pgfpathlineto{\pgfqpoint{2.747980in}{1.268628in}}%
\pgfpathlineto{\pgfqpoint{2.759261in}{1.276128in}}%
\pgfpathlineto{\pgfqpoint{2.760721in}{1.277189in}}%
\pgfpathlineto{\pgfqpoint{2.770542in}{1.284368in}}%
\pgfpathlineto{\pgfqpoint{2.775699in}{1.288259in}}%
\pgfpathlineto{\pgfqpoint{2.781822in}{1.292923in}}%
\pgfpathlineto{\pgfqpoint{2.789957in}{1.299328in}}%
\pgfpathlineto{\pgfqpoint{2.793103in}{1.301812in}}%
\pgfpathlineto{\pgfqpoint{2.803611in}{1.310398in}}%
\pgfpathlineto{\pgfqpoint{2.804384in}{1.311020in}}%
\pgfpathlineto{\pgfqpoint{2.815665in}{1.319931in}}%
\pgfpathlineto{\pgfqpoint{2.817622in}{1.321467in}}%
\pgfpathlineto{\pgfqpoint{2.826946in}{1.328758in}}%
\pgfpathlineto{\pgfqpoint{2.831712in}{1.332537in}}%
\pgfpathlineto{\pgfqpoint{2.838226in}{1.337481in}}%
\pgfpathlineto{\pgfqpoint{2.846129in}{1.343606in}}%
\pgfpathlineto{\pgfqpoint{2.849507in}{1.346240in}}%
\pgfpathlineto{\pgfqpoint{2.860041in}{1.354676in}}%
\pgfpathlineto{\pgfqpoint{2.860788in}{1.355277in}}%
\pgfpathlineto{\pgfqpoint{2.872069in}{1.364614in}}%
\pgfpathlineto{\pgfqpoint{2.873401in}{1.365746in}}%
\pgfpathlineto{\pgfqpoint{2.883350in}{1.374243in}}%
\pgfpathlineto{\pgfqpoint{2.886290in}{1.376815in}}%
\pgfpathlineto{\pgfqpoint{2.894631in}{1.384143in}}%
\pgfpathlineto{\pgfqpoint{2.898796in}{1.387885in}}%
\pgfpathlineto{\pgfqpoint{2.905911in}{1.394297in}}%
\pgfpathlineto{\pgfqpoint{2.910973in}{1.398954in}}%
\pgfpathlineto{\pgfqpoint{2.917192in}{1.404693in}}%
\pgfpathlineto{\pgfqpoint{2.922859in}{1.410024in}}%
\pgfpathlineto{\pgfqpoint{2.928473in}{1.415320in}}%
\pgfpathlineto{\pgfqpoint{2.934484in}{1.421093in}}%
\pgfpathlineto{\pgfqpoint{2.939754in}{1.426167in}}%
\pgfpathlineto{\pgfqpoint{2.945970in}{1.432163in}}%
\pgfpathlineto{\pgfqpoint{2.951035in}{1.437062in}}%
\pgfpathlineto{\pgfqpoint{2.957411in}{1.443232in}}%
\pgfpathlineto{\pgfqpoint{2.962315in}{1.447992in}}%
\pgfpathlineto{\pgfqpoint{2.968710in}{1.454302in}}%
\pgfpathlineto{\pgfqpoint{2.973596in}{1.459133in}}%
\pgfpathlineto{\pgfqpoint{2.973596in}{1.465371in}}%
\pgfpathlineto{\pgfqpoint{2.973596in}{1.476441in}}%
\pgfpathlineto{\pgfqpoint{2.973596in}{1.487510in}}%
\pgfpathlineto{\pgfqpoint{2.973596in}{1.498580in}}%
\pgfpathlineto{\pgfqpoint{2.973596in}{1.502884in}}%
\pgfpathlineto{\pgfqpoint{2.969270in}{1.498580in}}%
\pgfpathlineto{\pgfqpoint{2.962315in}{1.491666in}}%
\pgfpathlineto{\pgfqpoint{2.958063in}{1.487510in}}%
\pgfpathlineto{\pgfqpoint{2.951035in}{1.480648in}}%
\pgfpathlineto{\pgfqpoint{2.946727in}{1.476441in}}%
\pgfpathlineto{\pgfqpoint{2.939754in}{1.469646in}}%
\pgfpathlineto{\pgfqpoint{2.935336in}{1.465371in}}%
\pgfpathlineto{\pgfqpoint{2.928473in}{1.458735in}}%
\pgfpathlineto{\pgfqpoint{2.923788in}{1.454302in}}%
\pgfpathlineto{\pgfqpoint{2.917192in}{1.448062in}}%
\pgfpathlineto{\pgfqpoint{2.911966in}{1.443232in}}%
\pgfpathlineto{\pgfqpoint{2.905911in}{1.437637in}}%
\pgfpathlineto{\pgfqpoint{2.899835in}{1.432163in}}%
\pgfpathlineto{\pgfqpoint{2.894631in}{1.427472in}}%
\pgfpathlineto{\pgfqpoint{2.887357in}{1.421093in}}%
\pgfpathlineto{\pgfqpoint{2.883350in}{1.417577in}}%
\pgfpathlineto{\pgfqpoint{2.874489in}{1.410024in}}%
\pgfpathlineto{\pgfqpoint{2.872069in}{1.407960in}}%
\pgfpathlineto{\pgfqpoint{2.861175in}{1.398954in}}%
\pgfpathlineto{\pgfqpoint{2.860788in}{1.398634in}}%
\pgfpathlineto{\pgfqpoint{2.849507in}{1.389633in}}%
\pgfpathlineto{\pgfqpoint{2.847243in}{1.387885in}}%
\pgfpathlineto{\pgfqpoint{2.838226in}{1.380924in}}%
\pgfpathlineto{\pgfqpoint{2.832712in}{1.376815in}}%
\pgfpathlineto{\pgfqpoint{2.826946in}{1.372510in}}%
\pgfpathlineto{\pgfqpoint{2.817580in}{1.365746in}}%
\pgfpathlineto{\pgfqpoint{2.815665in}{1.364365in}}%
\pgfpathlineto{\pgfqpoint{2.804384in}{1.355909in}}%
\pgfpathlineto{\pgfqpoint{2.802696in}{1.354676in}}%
\pgfpathlineto{\pgfqpoint{2.793103in}{1.347678in}}%
\pgfpathlineto{\pgfqpoint{2.787388in}{1.343606in}}%
\pgfpathlineto{\pgfqpoint{2.781822in}{1.339471in}}%
\pgfpathlineto{\pgfqpoint{2.772296in}{1.332537in}}%
\pgfpathlineto{\pgfqpoint{2.770542in}{1.331238in}}%
\pgfpathlineto{\pgfqpoint{2.759261in}{1.323799in}}%
\pgfpathlineto{\pgfqpoint{2.755552in}{1.321467in}}%
\pgfpathlineto{\pgfqpoint{2.747980in}{1.316726in}}%
\pgfpathlineto{\pgfqpoint{2.737217in}{1.310398in}}%
\pgfpathlineto{\pgfqpoint{2.736699in}{1.310085in}}%
\pgfpathlineto{\pgfqpoint{2.725418in}{1.306419in}}%
\pgfpathlineto{\pgfqpoint{2.714137in}{1.305782in}}%
\pgfpathlineto{\pgfqpoint{2.702857in}{1.307635in}}%
\pgfpathlineto{\pgfqpoint{2.691576in}{1.310263in}}%
\pgfpathlineto{\pgfqpoint{2.691141in}{1.310398in}}%
\pgfpathlineto{\pgfqpoint{2.680295in}{1.313823in}}%
\pgfpathlineto{\pgfqpoint{2.669014in}{1.318007in}}%
\pgfpathlineto{\pgfqpoint{2.661484in}{1.321467in}}%
\pgfpathlineto{\pgfqpoint{2.657733in}{1.323215in}}%
\pgfpathlineto{\pgfqpoint{2.646453in}{1.328610in}}%
\pgfpathlineto{\pgfqpoint{2.636507in}{1.332537in}}%
\pgfpathlineto{\pgfqpoint{2.635172in}{1.333069in}}%
\pgfpathlineto{\pgfqpoint{2.623891in}{1.337846in}}%
\pgfpathlineto{\pgfqpoint{2.612610in}{1.343059in}}%
\pgfpathlineto{\pgfqpoint{2.611527in}{1.343606in}}%
\pgfpathlineto{\pgfqpoint{2.601329in}{1.348816in}}%
\pgfpathlineto{\pgfqpoint{2.590048in}{1.354673in}}%
\pgfpathlineto{\pgfqpoint{2.590043in}{1.354676in}}%
\pgfpathlineto{\pgfqpoint{2.578768in}{1.360735in}}%
\pgfpathlineto{\pgfqpoint{2.569606in}{1.365746in}}%
\pgfpathlineto{\pgfqpoint{2.567487in}{1.366914in}}%
\pgfpathlineto{\pgfqpoint{2.556206in}{1.373317in}}%
\pgfpathlineto{\pgfqpoint{2.550763in}{1.376815in}}%
\pgfpathlineto{\pgfqpoint{2.544925in}{1.380528in}}%
\pgfpathlineto{\pgfqpoint{2.534465in}{1.387885in}}%
\pgfpathlineto{\pgfqpoint{2.533644in}{1.388458in}}%
\pgfpathlineto{\pgfqpoint{2.522364in}{1.397354in}}%
\pgfpathlineto{\pgfqpoint{2.520489in}{1.398954in}}%
\pgfpathlineto{\pgfqpoint{2.511083in}{1.407154in}}%
\pgfpathlineto{\pgfqpoint{2.507925in}{1.410024in}}%
\pgfpathlineto{\pgfqpoint{2.499802in}{1.417508in}}%
\pgfpathlineto{\pgfqpoint{2.495981in}{1.421093in}}%
\pgfpathlineto{\pgfqpoint{2.488521in}{1.428061in}}%
\pgfpathlineto{\pgfqpoint{2.484155in}{1.432163in}}%
\pgfpathlineto{\pgfqpoint{2.477240in}{1.438744in}}%
\pgfpathlineto{\pgfqpoint{2.472565in}{1.443232in}}%
\pgfpathlineto{\pgfqpoint{2.465960in}{1.449625in}}%
\pgfpathlineto{\pgfqpoint{2.461224in}{1.454302in}}%
\pgfpathlineto{\pgfqpoint{2.454679in}{1.460861in}}%
\pgfpathlineto{\pgfqpoint{2.450295in}{1.465371in}}%
\pgfpathlineto{\pgfqpoint{2.443398in}{1.472824in}}%
\pgfpathlineto{\pgfqpoint{2.440182in}{1.476441in}}%
\pgfpathlineto{\pgfqpoint{2.432117in}{1.485610in}}%
\pgfpathlineto{\pgfqpoint{2.430459in}{1.487510in}}%
\pgfpathlineto{\pgfqpoint{2.420903in}{1.498580in}}%
\pgfpathlineto{\pgfqpoint{2.420836in}{1.498658in}}%
\pgfpathlineto{\pgfqpoint{2.411414in}{1.509650in}}%
\pgfpathlineto{\pgfqpoint{2.409555in}{1.511826in}}%
\pgfpathlineto{\pgfqpoint{2.401955in}{1.520719in}}%
\pgfpathlineto{\pgfqpoint{2.398275in}{1.525063in}}%
\pgfpathlineto{\pgfqpoint{2.392596in}{1.531789in}}%
\pgfpathlineto{\pgfqpoint{2.386994in}{1.538488in}}%
\pgfpathlineto{\pgfqpoint{2.383316in}{1.542858in}}%
\pgfpathlineto{\pgfqpoint{2.375713in}{1.552014in}}%
\pgfpathlineto{\pgfqpoint{2.374130in}{1.553928in}}%
\pgfpathlineto{\pgfqpoint{2.365000in}{1.564997in}}%
\pgfpathlineto{\pgfqpoint{2.364432in}{1.565692in}}%
\pgfpathlineto{\pgfqpoint{2.356139in}{1.576067in}}%
\pgfpathlineto{\pgfqpoint{2.353151in}{1.579841in}}%
\pgfpathlineto{\pgfqpoint{2.347574in}{1.587136in}}%
\pgfpathlineto{\pgfqpoint{2.341871in}{1.587136in}}%
\pgfpathlineto{\pgfqpoint{2.330590in}{1.587136in}}%
\pgfpathlineto{\pgfqpoint{2.319309in}{1.587136in}}%
\pgfpathlineto{\pgfqpoint{2.308028in}{1.587136in}}%
\pgfpathlineto{\pgfqpoint{2.296747in}{1.587136in}}%
\pgfpathlineto{\pgfqpoint{2.293615in}{1.587136in}}%
\pgfpathlineto{\pgfqpoint{2.296747in}{1.582874in}}%
\pgfpathlineto{\pgfqpoint{2.301769in}{1.576067in}}%
\pgfpathlineto{\pgfqpoint{2.308028in}{1.567673in}}%
\pgfpathlineto{\pgfqpoint{2.310029in}{1.564997in}}%
\pgfpathlineto{\pgfqpoint{2.318401in}{1.553928in}}%
\pgfpathlineto{\pgfqpoint{2.319309in}{1.552742in}}%
\pgfpathlineto{\pgfqpoint{2.326902in}{1.542858in}}%
\pgfpathlineto{\pgfqpoint{2.330590in}{1.538130in}}%
\pgfpathlineto{\pgfqpoint{2.335710in}{1.531789in}}%
\pgfpathlineto{\pgfqpoint{2.341871in}{1.524364in}}%
\pgfpathlineto{\pgfqpoint{2.344949in}{1.520719in}}%
\pgfpathlineto{\pgfqpoint{2.353151in}{1.511117in}}%
\pgfpathlineto{\pgfqpoint{2.354407in}{1.509650in}}%
\pgfpathlineto{\pgfqpoint{2.363979in}{1.498580in}}%
\pgfpathlineto{\pgfqpoint{2.364432in}{1.498062in}}%
\pgfpathlineto{\pgfqpoint{2.373641in}{1.487510in}}%
\pgfpathlineto{\pgfqpoint{2.375713in}{1.485167in}}%
\pgfpathlineto{\pgfqpoint{2.383426in}{1.476441in}}%
\pgfpathlineto{\pgfqpoint{2.386994in}{1.472453in}}%
\pgfpathlineto{\pgfqpoint{2.393339in}{1.465371in}}%
\pgfpathlineto{\pgfqpoint{2.398275in}{1.459934in}}%
\pgfpathlineto{\pgfqpoint{2.403373in}{1.454302in}}%
\pgfpathlineto{\pgfqpoint{2.409555in}{1.447508in}}%
\pgfpathlineto{\pgfqpoint{2.413444in}{1.443232in}}%
\pgfpathlineto{\pgfqpoint{2.420836in}{1.435193in}}%
\pgfpathlineto{\pgfqpoint{2.423763in}{1.432163in}}%
\pgfpathlineto{\pgfqpoint{2.432117in}{1.423731in}}%
\pgfpathlineto{\pgfqpoint{2.434750in}{1.421093in}}%
\pgfpathlineto{\pgfqpoint{2.443398in}{1.412537in}}%
\pgfpathlineto{\pgfqpoint{2.445951in}{1.410024in}}%
\pgfpathlineto{\pgfqpoint{2.454679in}{1.401455in}}%
\pgfpathlineto{\pgfqpoint{2.457246in}{1.398954in}}%
\pgfpathlineto{\pgfqpoint{2.465960in}{1.390547in}}%
\pgfpathlineto{\pgfqpoint{2.468752in}{1.387885in}}%
\pgfpathlineto{\pgfqpoint{2.477240in}{1.379959in}}%
\pgfpathlineto{\pgfqpoint{2.480726in}{1.376815in}}%
\pgfpathlineto{\pgfqpoint{2.488521in}{1.369995in}}%
\pgfpathlineto{\pgfqpoint{2.493427in}{1.365746in}}%
\pgfpathlineto{\pgfqpoint{2.499802in}{1.360355in}}%
\pgfpathlineto{\pgfqpoint{2.506630in}{1.354676in}}%
\pgfpathlineto{\pgfqpoint{2.511083in}{1.351016in}}%
\pgfpathlineto{\pgfqpoint{2.520382in}{1.343606in}}%
\pgfpathlineto{\pgfqpoint{2.522364in}{1.342066in}}%
\pgfpathlineto{\pgfqpoint{2.533644in}{1.333994in}}%
\pgfpathlineto{\pgfqpoint{2.535691in}{1.332537in}}%
\pgfpathlineto{\pgfqpoint{2.544925in}{1.326005in}}%
\pgfpathlineto{\pgfqpoint{2.551631in}{1.321467in}}%
\pgfpathlineto{\pgfqpoint{2.556206in}{1.318307in}}%
\pgfpathlineto{\pgfqpoint{2.567487in}{1.310960in}}%
\pgfpathlineto{\pgfqpoint{2.568438in}{1.310398in}}%
\pgfpathlineto{\pgfqpoint{2.578768in}{1.304235in}}%
\pgfpathlineto{\pgfqpoint{2.587497in}{1.299328in}}%
\pgfpathlineto{\pgfqpoint{2.590048in}{1.297905in}}%
\pgfpathlineto{\pgfqpoint{2.601329in}{1.291820in}}%
\pgfpathlineto{\pgfqpoint{2.608304in}{1.288259in}}%
\pgfpathlineto{\pgfqpoint{2.612610in}{1.286077in}}%
\pgfpathlineto{\pgfqpoint{2.623891in}{1.280593in}}%
\pgfpathlineto{\pgfqpoint{2.631086in}{1.277189in}}%
\pgfpathlineto{\pgfqpoint{2.635172in}{1.275268in}}%
\pgfpathlineto{\pgfqpoint{2.646453in}{1.270185in}}%
\pgfpathlineto{\pgfqpoint{2.656018in}{1.266120in}}%
\pgfpathlineto{\pgfqpoint{2.657733in}{1.265406in}}%
\pgfpathlineto{\pgfqpoint{2.669014in}{1.260379in}}%
\pgfpathlineto{\pgfqpoint{2.680295in}{1.256304in}}%
\pgfpathlineto{\pgfqpoint{2.686251in}{1.255050in}}%
\pgfpathclose%
\pgfusepath{fill}%
\end{pgfscope}%
\begin{pgfscope}%
\pgfpathrectangle{\pgfqpoint{1.856795in}{0.423750in}}{\pgfqpoint{1.194205in}{1.163386in}}%
\pgfusepath{clip}%
\pgfsetbuttcap%
\pgfsetroundjoin%
\definecolor{currentfill}{rgb}{0.919781,0.275262,0.242460}%
\pgfsetfillcolor{currentfill}%
\pgfsetlinewidth{0.000000pt}%
\definecolor{currentstroke}{rgb}{0.000000,0.000000,0.000000}%
\pgfsetstrokecolor{currentstroke}%
\pgfsetdash{}{0pt}%
\pgfpathmoveto{\pgfqpoint{2.364432in}{0.491252in}}%
\pgfpathlineto{\pgfqpoint{2.375713in}{0.491252in}}%
\pgfpathlineto{\pgfqpoint{2.386994in}{0.491252in}}%
\pgfpathlineto{\pgfqpoint{2.398275in}{0.491252in}}%
\pgfpathlineto{\pgfqpoint{2.399962in}{0.491252in}}%
\pgfpathlineto{\pgfqpoint{2.398275in}{0.493719in}}%
\pgfpathlineto{\pgfqpoint{2.392375in}{0.502321in}}%
\pgfpathlineto{\pgfqpoint{2.386994in}{0.509643in}}%
\pgfpathlineto{\pgfqpoint{2.384120in}{0.513391in}}%
\pgfpathlineto{\pgfqpoint{2.375713in}{0.524074in}}%
\pgfpathlineto{\pgfqpoint{2.375406in}{0.524460in}}%
\pgfpathlineto{\pgfqpoint{2.366267in}{0.535530in}}%
\pgfpathlineto{\pgfqpoint{2.364432in}{0.537715in}}%
\pgfpathlineto{\pgfqpoint{2.356709in}{0.546600in}}%
\pgfpathlineto{\pgfqpoint{2.353151in}{0.550589in}}%
\pgfpathlineto{\pgfqpoint{2.346513in}{0.557669in}}%
\pgfpathlineto{\pgfqpoint{2.341871in}{0.562528in}}%
\pgfpathlineto{\pgfqpoint{2.335871in}{0.568739in}}%
\pgfpathlineto{\pgfqpoint{2.330590in}{0.574074in}}%
\pgfpathlineto{\pgfqpoint{2.324815in}{0.579808in}}%
\pgfpathlineto{\pgfqpoint{2.319309in}{0.585022in}}%
\pgfpathlineto{\pgfqpoint{2.312997in}{0.590878in}}%
\pgfpathlineto{\pgfqpoint{2.308028in}{0.595288in}}%
\pgfpathlineto{\pgfqpoint{2.300292in}{0.601947in}}%
\pgfpathlineto{\pgfqpoint{2.296747in}{0.604886in}}%
\pgfpathlineto{\pgfqpoint{2.286519in}{0.613017in}}%
\pgfpathlineto{\pgfqpoint{2.285466in}{0.613835in}}%
\pgfpathlineto{\pgfqpoint{2.274186in}{0.622398in}}%
\pgfpathlineto{\pgfqpoint{2.271858in}{0.624086in}}%
\pgfpathlineto{\pgfqpoint{2.262905in}{0.630257in}}%
\pgfpathlineto{\pgfqpoint{2.255351in}{0.635156in}}%
\pgfpathlineto{\pgfqpoint{2.251624in}{0.637451in}}%
\pgfpathlineto{\pgfqpoint{2.240343in}{0.644278in}}%
\pgfpathlineto{\pgfqpoint{2.237094in}{0.646225in}}%
\pgfpathlineto{\pgfqpoint{2.229062in}{0.650961in}}%
\pgfpathlineto{\pgfqpoint{2.218572in}{0.657295in}}%
\pgfpathlineto{\pgfqpoint{2.217782in}{0.657762in}}%
\pgfpathlineto{\pgfqpoint{2.206501in}{0.664582in}}%
\pgfpathlineto{\pgfqpoint{2.200200in}{0.668364in}}%
\pgfpathlineto{\pgfqpoint{2.195220in}{0.671324in}}%
\pgfpathlineto{\pgfqpoint{2.183939in}{0.678029in}}%
\pgfpathlineto{\pgfqpoint{2.181573in}{0.679434in}}%
\pgfpathlineto{\pgfqpoint{2.172658in}{0.684679in}}%
\pgfpathlineto{\pgfqpoint{2.163020in}{0.690504in}}%
\pgfpathlineto{\pgfqpoint{2.161377in}{0.691476in}}%
\pgfpathlineto{\pgfqpoint{2.150097in}{0.698487in}}%
\pgfpathlineto{\pgfqpoint{2.145245in}{0.701573in}}%
\pgfpathlineto{\pgfqpoint{2.138816in}{0.705681in}}%
\pgfpathlineto{\pgfqpoint{2.127810in}{0.712643in}}%
\pgfpathlineto{\pgfqpoint{2.127535in}{0.712816in}}%
\pgfpathlineto{\pgfqpoint{2.116254in}{0.720507in}}%
\pgfpathlineto{\pgfqpoint{2.111698in}{0.723712in}}%
\pgfpathlineto{\pgfqpoint{2.104973in}{0.728381in}}%
\pgfpathlineto{\pgfqpoint{2.095801in}{0.734782in}}%
\pgfpathlineto{\pgfqpoint{2.093693in}{0.736233in}}%
\pgfpathlineto{\pgfqpoint{2.082412in}{0.744719in}}%
\pgfpathlineto{\pgfqpoint{2.080927in}{0.745851in}}%
\pgfpathlineto{\pgfqpoint{2.071131in}{0.753318in}}%
\pgfpathlineto{\pgfqpoint{2.066397in}{0.756921in}}%
\pgfpathlineto{\pgfqpoint{2.059850in}{0.761815in}}%
\pgfpathlineto{\pgfqpoint{2.051896in}{0.767990in}}%
\pgfpathlineto{\pgfqpoint{2.048569in}{0.770455in}}%
\pgfpathlineto{\pgfqpoint{2.037930in}{0.779060in}}%
\pgfpathlineto{\pgfqpoint{2.037289in}{0.779564in}}%
\pgfpathlineto{\pgfqpoint{2.026008in}{0.788907in}}%
\pgfpathlineto{\pgfqpoint{2.024632in}{0.790129in}}%
\pgfpathlineto{\pgfqpoint{2.014727in}{0.798831in}}%
\pgfpathlineto{\pgfqpoint{2.012070in}{0.801199in}}%
\pgfpathlineto{\pgfqpoint{2.003446in}{0.808805in}}%
\pgfpathlineto{\pgfqpoint{1.999548in}{0.812269in}}%
\pgfpathlineto{\pgfqpoint{1.992165in}{0.818991in}}%
\pgfpathlineto{\pgfqpoint{1.987300in}{0.823338in}}%
\pgfpathlineto{\pgfqpoint{1.980884in}{0.829252in}}%
\pgfpathlineto{\pgfqpoint{1.975301in}{0.834408in}}%
\pgfpathlineto{\pgfqpoint{1.969604in}{0.839606in}}%
\pgfpathlineto{\pgfqpoint{1.963155in}{0.845477in}}%
\pgfpathlineto{\pgfqpoint{1.958323in}{0.849811in}}%
\pgfpathlineto{\pgfqpoint{1.950967in}{0.856547in}}%
\pgfpathlineto{\pgfqpoint{1.947042in}{0.860002in}}%
\pgfpathlineto{\pgfqpoint{1.938369in}{0.867616in}}%
\pgfpathlineto{\pgfqpoint{1.935761in}{0.869904in}}%
\pgfpathlineto{\pgfqpoint{1.925471in}{0.878686in}}%
\pgfpathlineto{\pgfqpoint{1.924480in}{0.879554in}}%
\pgfpathlineto{\pgfqpoint{1.913200in}{0.889596in}}%
\pgfpathlineto{\pgfqpoint{1.913019in}{0.889755in}}%
\pgfpathlineto{\pgfqpoint{1.901919in}{0.899836in}}%
\pgfpathlineto{\pgfqpoint{1.900788in}{0.900825in}}%
\pgfpathlineto{\pgfqpoint{1.890638in}{0.910048in}}%
\pgfpathlineto{\pgfqpoint{1.888562in}{0.911894in}}%
\pgfpathlineto{\pgfqpoint{1.879357in}{0.920363in}}%
\pgfpathlineto{\pgfqpoint{1.876437in}{0.922964in}}%
\pgfpathlineto{\pgfqpoint{1.868076in}{0.930782in}}%
\pgfpathlineto{\pgfqpoint{1.864383in}{0.934033in}}%
\pgfpathlineto{\pgfqpoint{1.856795in}{0.942483in}}%
\pgfpathlineto{\pgfqpoint{1.856795in}{0.934033in}}%
\pgfpathlineto{\pgfqpoint{1.856795in}{0.922964in}}%
\pgfpathlineto{\pgfqpoint{1.856795in}{0.913929in}}%
\pgfpathlineto{\pgfqpoint{1.859161in}{0.911894in}}%
\pgfpathlineto{\pgfqpoint{1.868076in}{0.904510in}}%
\pgfpathlineto{\pgfqpoint{1.872345in}{0.900825in}}%
\pgfpathlineto{\pgfqpoint{1.879357in}{0.895050in}}%
\pgfpathlineto{\pgfqpoint{1.885624in}{0.889755in}}%
\pgfpathlineto{\pgfqpoint{1.890638in}{0.885551in}}%
\pgfpathlineto{\pgfqpoint{1.898673in}{0.878686in}}%
\pgfpathlineto{\pgfqpoint{1.901919in}{0.875840in}}%
\pgfpathlineto{\pgfqpoint{1.911425in}{0.867616in}}%
\pgfpathlineto{\pgfqpoint{1.913200in}{0.866073in}}%
\pgfpathlineto{\pgfqpoint{1.924057in}{0.856547in}}%
\pgfpathlineto{\pgfqpoint{1.924480in}{0.856169in}}%
\pgfpathlineto{\pgfqpoint{1.935761in}{0.846018in}}%
\pgfpathlineto{\pgfqpoint{1.936355in}{0.845477in}}%
\pgfpathlineto{\pgfqpoint{1.947042in}{0.835565in}}%
\pgfpathlineto{\pgfqpoint{1.948301in}{0.834408in}}%
\pgfpathlineto{\pgfqpoint{1.958323in}{0.825359in}}%
\pgfpathlineto{\pgfqpoint{1.960534in}{0.823338in}}%
\pgfpathlineto{\pgfqpoint{1.969604in}{0.815284in}}%
\pgfpathlineto{\pgfqpoint{1.973006in}{0.812269in}}%
\pgfpathlineto{\pgfqpoint{1.980884in}{0.805203in}}%
\pgfpathlineto{\pgfqpoint{1.985377in}{0.801199in}}%
\pgfpathlineto{\pgfqpoint{1.992165in}{0.795156in}}%
\pgfpathlineto{\pgfqpoint{1.997820in}{0.790129in}}%
\pgfpathlineto{\pgfqpoint{2.003446in}{0.785064in}}%
\pgfpathlineto{\pgfqpoint{2.010125in}{0.779060in}}%
\pgfpathlineto{\pgfqpoint{2.014727in}{0.774878in}}%
\pgfpathlineto{\pgfqpoint{2.022387in}{0.767990in}}%
\pgfpathlineto{\pgfqpoint{2.026008in}{0.764695in}}%
\pgfpathlineto{\pgfqpoint{2.034943in}{0.756921in}}%
\pgfpathlineto{\pgfqpoint{2.037289in}{0.754862in}}%
\pgfpathlineto{\pgfqpoint{2.047552in}{0.745851in}}%
\pgfpathlineto{\pgfqpoint{2.048569in}{0.744958in}}%
\pgfpathlineto{\pgfqpoint{2.059850in}{0.735500in}}%
\pgfpathlineto{\pgfqpoint{2.060775in}{0.734782in}}%
\pgfpathlineto{\pgfqpoint{2.071131in}{0.726359in}}%
\pgfpathlineto{\pgfqpoint{2.074624in}{0.723712in}}%
\pgfpathlineto{\pgfqpoint{2.082412in}{0.717847in}}%
\pgfpathlineto{\pgfqpoint{2.089342in}{0.712643in}}%
\pgfpathlineto{\pgfqpoint{2.093693in}{0.709357in}}%
\pgfpathlineto{\pgfqpoint{2.103905in}{0.701573in}}%
\pgfpathlineto{\pgfqpoint{2.104973in}{0.700748in}}%
\pgfpathlineto{\pgfqpoint{2.116254in}{0.692086in}}%
\pgfpathlineto{\pgfqpoint{2.118470in}{0.690504in}}%
\pgfpathlineto{\pgfqpoint{2.127535in}{0.683960in}}%
\pgfpathlineto{\pgfqpoint{2.133838in}{0.679434in}}%
\pgfpathlineto{\pgfqpoint{2.138816in}{0.675844in}}%
\pgfpathlineto{\pgfqpoint{2.149190in}{0.668364in}}%
\pgfpathlineto{\pgfqpoint{2.150097in}{0.667707in}}%
\pgfpathlineto{\pgfqpoint{2.161377in}{0.660292in}}%
\pgfpathlineto{\pgfqpoint{2.166170in}{0.657295in}}%
\pgfpathlineto{\pgfqpoint{2.172658in}{0.653160in}}%
\pgfpathlineto{\pgfqpoint{2.183917in}{0.646225in}}%
\pgfpathlineto{\pgfqpoint{2.183939in}{0.646211in}}%
\pgfpathlineto{\pgfqpoint{2.195220in}{0.639186in}}%
\pgfpathlineto{\pgfqpoint{2.201641in}{0.635156in}}%
\pgfpathlineto{\pgfqpoint{2.206501in}{0.632053in}}%
\pgfpathlineto{\pgfqpoint{2.217782in}{0.624929in}}%
\pgfpathlineto{\pgfqpoint{2.219100in}{0.624086in}}%
\pgfpathlineto{\pgfqpoint{2.229062in}{0.617458in}}%
\pgfpathlineto{\pgfqpoint{2.235610in}{0.613017in}}%
\pgfpathlineto{\pgfqpoint{2.240343in}{0.609609in}}%
\pgfpathlineto{\pgfqpoint{2.251210in}{0.601947in}}%
\pgfpathlineto{\pgfqpoint{2.251624in}{0.601646in}}%
\pgfpathlineto{\pgfqpoint{2.262905in}{0.593106in}}%
\pgfpathlineto{\pgfqpoint{2.265758in}{0.590878in}}%
\pgfpathlineto{\pgfqpoint{2.274186in}{0.584058in}}%
\pgfpathlineto{\pgfqpoint{2.279229in}{0.579808in}}%
\pgfpathlineto{\pgfqpoint{2.285466in}{0.574345in}}%
\pgfpathlineto{\pgfqpoint{2.291741in}{0.568739in}}%
\pgfpathlineto{\pgfqpoint{2.296747in}{0.564084in}}%
\pgfpathlineto{\pgfqpoint{2.303367in}{0.557669in}}%
\pgfpathlineto{\pgfqpoint{2.308028in}{0.553033in}}%
\pgfpathlineto{\pgfqpoint{2.314371in}{0.546600in}}%
\pgfpathlineto{\pgfqpoint{2.319309in}{0.541484in}}%
\pgfpathlineto{\pgfqpoint{2.324934in}{0.535530in}}%
\pgfpathlineto{\pgfqpoint{2.330590in}{0.529543in}}%
\pgfpathlineto{\pgfqpoint{2.335330in}{0.524460in}}%
\pgfpathlineto{\pgfqpoint{2.341871in}{0.517011in}}%
\pgfpathlineto{\pgfqpoint{2.344846in}{0.513391in}}%
\pgfpathlineto{\pgfqpoint{2.353151in}{0.502792in}}%
\pgfpathlineto{\pgfqpoint{2.353523in}{0.502321in}}%
\pgfpathlineto{\pgfqpoint{2.362159in}{0.491252in}}%
\pgfpathclose%
\pgfusepath{fill}%
\end{pgfscope}%
\begin{pgfscope}%
\pgfpathrectangle{\pgfqpoint{1.856795in}{0.423750in}}{\pgfqpoint{1.194205in}{1.163386in}}%
\pgfusepath{clip}%
\pgfsetbuttcap%
\pgfsetroundjoin%
\definecolor{currentfill}{rgb}{0.919781,0.275262,0.242460}%
\pgfsetfillcolor{currentfill}%
\pgfsetlinewidth{0.000000pt}%
\definecolor{currentstroke}{rgb}{0.000000,0.000000,0.000000}%
\pgfsetstrokecolor{currentstroke}%
\pgfsetdash{}{0pt}%
\pgfpathmoveto{\pgfqpoint{2.657733in}{1.208838in}}%
\pgfpathlineto{\pgfqpoint{2.669014in}{1.205062in}}%
\pgfpathlineto{\pgfqpoint{2.680295in}{1.202103in}}%
\pgfpathlineto{\pgfqpoint{2.691576in}{1.200732in}}%
\pgfpathlineto{\pgfqpoint{2.702857in}{1.200790in}}%
\pgfpathlineto{\pgfqpoint{2.714137in}{1.202128in}}%
\pgfpathlineto{\pgfqpoint{2.725418in}{1.206952in}}%
\pgfpathlineto{\pgfqpoint{2.730856in}{1.210772in}}%
\pgfpathlineto{\pgfqpoint{2.736699in}{1.214853in}}%
\pgfpathlineto{\pgfqpoint{2.746558in}{1.221842in}}%
\pgfpathlineto{\pgfqpoint{2.747980in}{1.222802in}}%
\pgfpathlineto{\pgfqpoint{2.759261in}{1.231264in}}%
\pgfpathlineto{\pgfqpoint{2.761355in}{1.232911in}}%
\pgfpathlineto{\pgfqpoint{2.770542in}{1.239820in}}%
\pgfpathlineto{\pgfqpoint{2.775743in}{1.243981in}}%
\pgfpathlineto{\pgfqpoint{2.781822in}{1.248790in}}%
\pgfpathlineto{\pgfqpoint{2.789577in}{1.255050in}}%
\pgfpathlineto{\pgfqpoint{2.793103in}{1.257953in}}%
\pgfpathlineto{\pgfqpoint{2.802836in}{1.266120in}}%
\pgfpathlineto{\pgfqpoint{2.804384in}{1.267389in}}%
\pgfpathlineto{\pgfqpoint{2.815665in}{1.276592in}}%
\pgfpathlineto{\pgfqpoint{2.816395in}{1.277189in}}%
\pgfpathlineto{\pgfqpoint{2.826946in}{1.285731in}}%
\pgfpathlineto{\pgfqpoint{2.829933in}{1.288259in}}%
\pgfpathlineto{\pgfqpoint{2.838226in}{1.294926in}}%
\pgfpathlineto{\pgfqpoint{2.843257in}{1.299328in}}%
\pgfpathlineto{\pgfqpoint{2.849507in}{1.304541in}}%
\pgfpathlineto{\pgfqpoint{2.856345in}{1.310398in}}%
\pgfpathlineto{\pgfqpoint{2.860788in}{1.314201in}}%
\pgfpathlineto{\pgfqpoint{2.869054in}{1.321467in}}%
\pgfpathlineto{\pgfqpoint{2.872069in}{1.323972in}}%
\pgfpathlineto{\pgfqpoint{2.882303in}{1.332537in}}%
\pgfpathlineto{\pgfqpoint{2.883350in}{1.333418in}}%
\pgfpathlineto{\pgfqpoint{2.894631in}{1.343126in}}%
\pgfpathlineto{\pgfqpoint{2.895178in}{1.343606in}}%
\pgfpathlineto{\pgfqpoint{2.905911in}{1.353112in}}%
\pgfpathlineto{\pgfqpoint{2.907647in}{1.354676in}}%
\pgfpathlineto{\pgfqpoint{2.917192in}{1.363345in}}%
\pgfpathlineto{\pgfqpoint{2.919795in}{1.365746in}}%
\pgfpathlineto{\pgfqpoint{2.928473in}{1.373814in}}%
\pgfpathlineto{\pgfqpoint{2.931654in}{1.376815in}}%
\pgfpathlineto{\pgfqpoint{2.939754in}{1.384508in}}%
\pgfpathlineto{\pgfqpoint{2.943317in}{1.387885in}}%
\pgfpathlineto{\pgfqpoint{2.951035in}{1.395242in}}%
\pgfpathlineto{\pgfqpoint{2.954927in}{1.398954in}}%
\pgfpathlineto{\pgfqpoint{2.962315in}{1.406035in}}%
\pgfpathlineto{\pgfqpoint{2.966428in}{1.410024in}}%
\pgfpathlineto{\pgfqpoint{2.973596in}{1.417017in}}%
\pgfpathlineto{\pgfqpoint{2.973596in}{1.421093in}}%
\pgfpathlineto{\pgfqpoint{2.973596in}{1.432163in}}%
\pgfpathlineto{\pgfqpoint{2.973596in}{1.443232in}}%
\pgfpathlineto{\pgfqpoint{2.973596in}{1.454302in}}%
\pgfpathlineto{\pgfqpoint{2.973596in}{1.459133in}}%
\pgfpathlineto{\pgfqpoint{2.968710in}{1.454302in}}%
\pgfpathlineto{\pgfqpoint{2.962315in}{1.447992in}}%
\pgfpathlineto{\pgfqpoint{2.957411in}{1.443232in}}%
\pgfpathlineto{\pgfqpoint{2.951035in}{1.437062in}}%
\pgfpathlineto{\pgfqpoint{2.945970in}{1.432163in}}%
\pgfpathlineto{\pgfqpoint{2.939754in}{1.426167in}}%
\pgfpathlineto{\pgfqpoint{2.934484in}{1.421093in}}%
\pgfpathlineto{\pgfqpoint{2.928473in}{1.415320in}}%
\pgfpathlineto{\pgfqpoint{2.922859in}{1.410024in}}%
\pgfpathlineto{\pgfqpoint{2.917192in}{1.404693in}}%
\pgfpathlineto{\pgfqpoint{2.910973in}{1.398954in}}%
\pgfpathlineto{\pgfqpoint{2.905911in}{1.394297in}}%
\pgfpathlineto{\pgfqpoint{2.898796in}{1.387885in}}%
\pgfpathlineto{\pgfqpoint{2.894631in}{1.384143in}}%
\pgfpathlineto{\pgfqpoint{2.886290in}{1.376815in}}%
\pgfpathlineto{\pgfqpoint{2.883350in}{1.374243in}}%
\pgfpathlineto{\pgfqpoint{2.873401in}{1.365746in}}%
\pgfpathlineto{\pgfqpoint{2.872069in}{1.364614in}}%
\pgfpathlineto{\pgfqpoint{2.860788in}{1.355277in}}%
\pgfpathlineto{\pgfqpoint{2.860041in}{1.354676in}}%
\pgfpathlineto{\pgfqpoint{2.849507in}{1.346240in}}%
\pgfpathlineto{\pgfqpoint{2.846129in}{1.343606in}}%
\pgfpathlineto{\pgfqpoint{2.838226in}{1.337481in}}%
\pgfpathlineto{\pgfqpoint{2.831712in}{1.332537in}}%
\pgfpathlineto{\pgfqpoint{2.826946in}{1.328758in}}%
\pgfpathlineto{\pgfqpoint{2.817622in}{1.321467in}}%
\pgfpathlineto{\pgfqpoint{2.815665in}{1.319931in}}%
\pgfpathlineto{\pgfqpoint{2.804384in}{1.311020in}}%
\pgfpathlineto{\pgfqpoint{2.803611in}{1.310398in}}%
\pgfpathlineto{\pgfqpoint{2.793103in}{1.301812in}}%
\pgfpathlineto{\pgfqpoint{2.789957in}{1.299328in}}%
\pgfpathlineto{\pgfqpoint{2.781822in}{1.292923in}}%
\pgfpathlineto{\pgfqpoint{2.775699in}{1.288259in}}%
\pgfpathlineto{\pgfqpoint{2.770542in}{1.284368in}}%
\pgfpathlineto{\pgfqpoint{2.760721in}{1.277189in}}%
\pgfpathlineto{\pgfqpoint{2.759261in}{1.276128in}}%
\pgfpathlineto{\pgfqpoint{2.747980in}{1.268628in}}%
\pgfpathlineto{\pgfqpoint{2.744282in}{1.266120in}}%
\pgfpathlineto{\pgfqpoint{2.736699in}{1.261081in}}%
\pgfpathlineto{\pgfqpoint{2.726096in}{1.255050in}}%
\pgfpathlineto{\pgfqpoint{2.725418in}{1.254655in}}%
\pgfpathlineto{\pgfqpoint{2.714137in}{1.251806in}}%
\pgfpathlineto{\pgfqpoint{2.702857in}{1.252590in}}%
\pgfpathlineto{\pgfqpoint{2.691576in}{1.253994in}}%
\pgfpathlineto{\pgfqpoint{2.686251in}{1.255050in}}%
\pgfpathlineto{\pgfqpoint{2.680295in}{1.256304in}}%
\pgfpathlineto{\pgfqpoint{2.669014in}{1.260379in}}%
\pgfpathlineto{\pgfqpoint{2.657733in}{1.265406in}}%
\pgfpathlineto{\pgfqpoint{2.656018in}{1.266120in}}%
\pgfpathlineto{\pgfqpoint{2.646453in}{1.270185in}}%
\pgfpathlineto{\pgfqpoint{2.635172in}{1.275268in}}%
\pgfpathlineto{\pgfqpoint{2.631086in}{1.277189in}}%
\pgfpathlineto{\pgfqpoint{2.623891in}{1.280593in}}%
\pgfpathlineto{\pgfqpoint{2.612610in}{1.286077in}}%
\pgfpathlineto{\pgfqpoint{2.608304in}{1.288259in}}%
\pgfpathlineto{\pgfqpoint{2.601329in}{1.291820in}}%
\pgfpathlineto{\pgfqpoint{2.590048in}{1.297905in}}%
\pgfpathlineto{\pgfqpoint{2.587497in}{1.299328in}}%
\pgfpathlineto{\pgfqpoint{2.578768in}{1.304235in}}%
\pgfpathlineto{\pgfqpoint{2.568438in}{1.310398in}}%
\pgfpathlineto{\pgfqpoint{2.567487in}{1.310960in}}%
\pgfpathlineto{\pgfqpoint{2.556206in}{1.318307in}}%
\pgfpathlineto{\pgfqpoint{2.551631in}{1.321467in}}%
\pgfpathlineto{\pgfqpoint{2.544925in}{1.326005in}}%
\pgfpathlineto{\pgfqpoint{2.535691in}{1.332537in}}%
\pgfpathlineto{\pgfqpoint{2.533644in}{1.333994in}}%
\pgfpathlineto{\pgfqpoint{2.522364in}{1.342066in}}%
\pgfpathlineto{\pgfqpoint{2.520382in}{1.343606in}}%
\pgfpathlineto{\pgfqpoint{2.511083in}{1.351016in}}%
\pgfpathlineto{\pgfqpoint{2.506630in}{1.354676in}}%
\pgfpathlineto{\pgfqpoint{2.499802in}{1.360355in}}%
\pgfpathlineto{\pgfqpoint{2.493427in}{1.365746in}}%
\pgfpathlineto{\pgfqpoint{2.488521in}{1.369995in}}%
\pgfpathlineto{\pgfqpoint{2.480726in}{1.376815in}}%
\pgfpathlineto{\pgfqpoint{2.477240in}{1.379959in}}%
\pgfpathlineto{\pgfqpoint{2.468752in}{1.387885in}}%
\pgfpathlineto{\pgfqpoint{2.465960in}{1.390547in}}%
\pgfpathlineto{\pgfqpoint{2.457246in}{1.398954in}}%
\pgfpathlineto{\pgfqpoint{2.454679in}{1.401455in}}%
\pgfpathlineto{\pgfqpoint{2.445951in}{1.410024in}}%
\pgfpathlineto{\pgfqpoint{2.443398in}{1.412537in}}%
\pgfpathlineto{\pgfqpoint{2.434750in}{1.421093in}}%
\pgfpathlineto{\pgfqpoint{2.432117in}{1.423731in}}%
\pgfpathlineto{\pgfqpoint{2.423763in}{1.432163in}}%
\pgfpathlineto{\pgfqpoint{2.420836in}{1.435193in}}%
\pgfpathlineto{\pgfqpoint{2.413444in}{1.443232in}}%
\pgfpathlineto{\pgfqpoint{2.409555in}{1.447508in}}%
\pgfpathlineto{\pgfqpoint{2.403373in}{1.454302in}}%
\pgfpathlineto{\pgfqpoint{2.398275in}{1.459934in}}%
\pgfpathlineto{\pgfqpoint{2.393339in}{1.465371in}}%
\pgfpathlineto{\pgfqpoint{2.386994in}{1.472453in}}%
\pgfpathlineto{\pgfqpoint{2.383426in}{1.476441in}}%
\pgfpathlineto{\pgfqpoint{2.375713in}{1.485167in}}%
\pgfpathlineto{\pgfqpoint{2.373641in}{1.487510in}}%
\pgfpathlineto{\pgfqpoint{2.364432in}{1.498062in}}%
\pgfpathlineto{\pgfqpoint{2.363979in}{1.498580in}}%
\pgfpathlineto{\pgfqpoint{2.354407in}{1.509650in}}%
\pgfpathlineto{\pgfqpoint{2.353151in}{1.511117in}}%
\pgfpathlineto{\pgfqpoint{2.344949in}{1.520719in}}%
\pgfpathlineto{\pgfqpoint{2.341871in}{1.524364in}}%
\pgfpathlineto{\pgfqpoint{2.335710in}{1.531789in}}%
\pgfpathlineto{\pgfqpoint{2.330590in}{1.538130in}}%
\pgfpathlineto{\pgfqpoint{2.326902in}{1.542858in}}%
\pgfpathlineto{\pgfqpoint{2.319309in}{1.552742in}}%
\pgfpathlineto{\pgfqpoint{2.318401in}{1.553928in}}%
\pgfpathlineto{\pgfqpoint{2.310029in}{1.564997in}}%
\pgfpathlineto{\pgfqpoint{2.308028in}{1.567673in}}%
\pgfpathlineto{\pgfqpoint{2.301769in}{1.576067in}}%
\pgfpathlineto{\pgfqpoint{2.296747in}{1.582874in}}%
\pgfpathlineto{\pgfqpoint{2.293615in}{1.587136in}}%
\pgfpathlineto{\pgfqpoint{2.285466in}{1.587136in}}%
\pgfpathlineto{\pgfqpoint{2.274186in}{1.587136in}}%
\pgfpathlineto{\pgfqpoint{2.262905in}{1.587136in}}%
\pgfpathlineto{\pgfqpoint{2.251624in}{1.587136in}}%
\pgfpathlineto{\pgfqpoint{2.241442in}{1.587136in}}%
\pgfpathlineto{\pgfqpoint{2.249606in}{1.576067in}}%
\pgfpathlineto{\pgfqpoint{2.251624in}{1.573364in}}%
\pgfpathlineto{\pgfqpoint{2.257865in}{1.564997in}}%
\pgfpathlineto{\pgfqpoint{2.262905in}{1.558334in}}%
\pgfpathlineto{\pgfqpoint{2.266226in}{1.553928in}}%
\pgfpathlineto{\pgfqpoint{2.274186in}{1.543536in}}%
\pgfpathlineto{\pgfqpoint{2.274703in}{1.542858in}}%
\pgfpathlineto{\pgfqpoint{2.283280in}{1.531789in}}%
\pgfpathlineto{\pgfqpoint{2.285466in}{1.529014in}}%
\pgfpathlineto{\pgfqpoint{2.291975in}{1.520719in}}%
\pgfpathlineto{\pgfqpoint{2.296747in}{1.514729in}}%
\pgfpathlineto{\pgfqpoint{2.300792in}{1.509650in}}%
\pgfpathlineto{\pgfqpoint{2.308028in}{1.500692in}}%
\pgfpathlineto{\pgfqpoint{2.309746in}{1.498580in}}%
\pgfpathlineto{\pgfqpoint{2.318996in}{1.487510in}}%
\pgfpathlineto{\pgfqpoint{2.319309in}{1.487123in}}%
\pgfpathlineto{\pgfqpoint{2.328595in}{1.476441in}}%
\pgfpathlineto{\pgfqpoint{2.330590in}{1.474159in}}%
\pgfpathlineto{\pgfqpoint{2.338268in}{1.465371in}}%
\pgfpathlineto{\pgfqpoint{2.341871in}{1.461304in}}%
\pgfpathlineto{\pgfqpoint{2.348073in}{1.454302in}}%
\pgfpathlineto{\pgfqpoint{2.353151in}{1.448477in}}%
\pgfpathlineto{\pgfqpoint{2.357764in}{1.443232in}}%
\pgfpathlineto{\pgfqpoint{2.364432in}{1.435650in}}%
\pgfpathlineto{\pgfqpoint{2.367515in}{1.432163in}}%
\pgfpathlineto{\pgfqpoint{2.375713in}{1.422929in}}%
\pgfpathlineto{\pgfqpoint{2.377357in}{1.421093in}}%
\pgfpathlineto{\pgfqpoint{2.386994in}{1.410207in}}%
\pgfpathlineto{\pgfqpoint{2.387156in}{1.410024in}}%
\pgfpathlineto{\pgfqpoint{2.396963in}{1.398954in}}%
\pgfpathlineto{\pgfqpoint{2.398275in}{1.397549in}}%
\pgfpathlineto{\pgfqpoint{2.407771in}{1.387885in}}%
\pgfpathlineto{\pgfqpoint{2.409555in}{1.386067in}}%
\pgfpathlineto{\pgfqpoint{2.418806in}{1.376815in}}%
\pgfpathlineto{\pgfqpoint{2.420836in}{1.374760in}}%
\pgfpathlineto{\pgfqpoint{2.429835in}{1.365746in}}%
\pgfpathlineto{\pgfqpoint{2.432117in}{1.363465in}}%
\pgfpathlineto{\pgfqpoint{2.440889in}{1.354676in}}%
\pgfpathlineto{\pgfqpoint{2.443398in}{1.352250in}}%
\pgfpathlineto{\pgfqpoint{2.452610in}{1.343606in}}%
\pgfpathlineto{\pgfqpoint{2.454679in}{1.341735in}}%
\pgfpathlineto{\pgfqpoint{2.464861in}{1.332537in}}%
\pgfpathlineto{\pgfqpoint{2.465960in}{1.331558in}}%
\pgfpathlineto{\pgfqpoint{2.477240in}{1.321730in}}%
\pgfpathlineto{\pgfqpoint{2.477552in}{1.321467in}}%
\pgfpathlineto{\pgfqpoint{2.488521in}{1.312133in}}%
\pgfpathlineto{\pgfqpoint{2.490602in}{1.310398in}}%
\pgfpathlineto{\pgfqpoint{2.499802in}{1.302569in}}%
\pgfpathlineto{\pgfqpoint{2.503681in}{1.299328in}}%
\pgfpathlineto{\pgfqpoint{2.511083in}{1.293322in}}%
\pgfpathlineto{\pgfqpoint{2.517926in}{1.288259in}}%
\pgfpathlineto{\pgfqpoint{2.522364in}{1.285054in}}%
\pgfpathlineto{\pgfqpoint{2.533317in}{1.277189in}}%
\pgfpathlineto{\pgfqpoint{2.533644in}{1.276959in}}%
\pgfpathlineto{\pgfqpoint{2.544925in}{1.268903in}}%
\pgfpathlineto{\pgfqpoint{2.548879in}{1.266120in}}%
\pgfpathlineto{\pgfqpoint{2.556206in}{1.261044in}}%
\pgfpathlineto{\pgfqpoint{2.565030in}{1.255050in}}%
\pgfpathlineto{\pgfqpoint{2.567487in}{1.253410in}}%
\pgfpathlineto{\pgfqpoint{2.578768in}{1.246130in}}%
\pgfpathlineto{\pgfqpoint{2.582531in}{1.243981in}}%
\pgfpathlineto{\pgfqpoint{2.590048in}{1.239743in}}%
\pgfpathlineto{\pgfqpoint{2.601329in}{1.233717in}}%
\pgfpathlineto{\pgfqpoint{2.602976in}{1.232911in}}%
\pgfpathlineto{\pgfqpoint{2.612610in}{1.228139in}}%
\pgfpathlineto{\pgfqpoint{2.623891in}{1.222898in}}%
\pgfpathlineto{\pgfqpoint{2.626346in}{1.221842in}}%
\pgfpathlineto{\pgfqpoint{2.635172in}{1.218018in}}%
\pgfpathlineto{\pgfqpoint{2.646453in}{1.213268in}}%
\pgfpathlineto{\pgfqpoint{2.652783in}{1.210772in}}%
\pgfpathclose%
\pgfusepath{fill}%
\end{pgfscope}%
\begin{pgfscope}%
\pgfpathrectangle{\pgfqpoint{1.856795in}{0.423750in}}{\pgfqpoint{1.194205in}{1.163386in}}%
\pgfusepath{clip}%
\pgfsetbuttcap%
\pgfsetroundjoin%
\definecolor{currentfill}{rgb}{0.940366,0.360209,0.257347}%
\pgfsetfillcolor{currentfill}%
\pgfsetlinewidth{0.000000pt}%
\definecolor{currentstroke}{rgb}{0.000000,0.000000,0.000000}%
\pgfsetstrokecolor{currentstroke}%
\pgfsetdash{}{0pt}%
\pgfpathmoveto{\pgfqpoint{2.398275in}{0.493719in}}%
\pgfpathlineto{\pgfqpoint{2.399962in}{0.491252in}}%
\pgfpathlineto{\pgfqpoint{2.409555in}{0.491252in}}%
\pgfpathlineto{\pgfqpoint{2.420836in}{0.491252in}}%
\pgfpathlineto{\pgfqpoint{2.432117in}{0.491252in}}%
\pgfpathlineto{\pgfqpoint{2.436880in}{0.491252in}}%
\pgfpathlineto{\pgfqpoint{2.432117in}{0.498744in}}%
\pgfpathlineto{\pgfqpoint{2.429838in}{0.502321in}}%
\pgfpathlineto{\pgfqpoint{2.422403in}{0.513391in}}%
\pgfpathlineto{\pgfqpoint{2.420836in}{0.515685in}}%
\pgfpathlineto{\pgfqpoint{2.414710in}{0.524460in}}%
\pgfpathlineto{\pgfqpoint{2.409555in}{0.531158in}}%
\pgfpathlineto{\pgfqpoint{2.406141in}{0.535530in}}%
\pgfpathlineto{\pgfqpoint{2.398275in}{0.545296in}}%
\pgfpathlineto{\pgfqpoint{2.397220in}{0.546600in}}%
\pgfpathlineto{\pgfqpoint{2.388075in}{0.557669in}}%
\pgfpathlineto{\pgfqpoint{2.386994in}{0.558947in}}%
\pgfpathlineto{\pgfqpoint{2.378574in}{0.568739in}}%
\pgfpathlineto{\pgfqpoint{2.375713in}{0.571991in}}%
\pgfpathlineto{\pgfqpoint{2.368636in}{0.579808in}}%
\pgfpathlineto{\pgfqpoint{2.364432in}{0.584381in}}%
\pgfpathlineto{\pgfqpoint{2.358258in}{0.590878in}}%
\pgfpathlineto{\pgfqpoint{2.353151in}{0.596113in}}%
\pgfpathlineto{\pgfqpoint{2.347159in}{0.601947in}}%
\pgfpathlineto{\pgfqpoint{2.341871in}{0.606705in}}%
\pgfpathlineto{\pgfqpoint{2.334575in}{0.613017in}}%
\pgfpathlineto{\pgfqpoint{2.330590in}{0.616374in}}%
\pgfpathlineto{\pgfqpoint{2.321251in}{0.624086in}}%
\pgfpathlineto{\pgfqpoint{2.319309in}{0.625607in}}%
\pgfpathlineto{\pgfqpoint{2.308028in}{0.634048in}}%
\pgfpathlineto{\pgfqpoint{2.306544in}{0.635156in}}%
\pgfpathlineto{\pgfqpoint{2.296747in}{0.642199in}}%
\pgfpathlineto{\pgfqpoint{2.291021in}{0.646225in}}%
\pgfpathlineto{\pgfqpoint{2.285466in}{0.649852in}}%
\pgfpathlineto{\pgfqpoint{2.274186in}{0.656784in}}%
\pgfpathlineto{\pgfqpoint{2.273335in}{0.657295in}}%
\pgfpathlineto{\pgfqpoint{2.262905in}{0.663430in}}%
\pgfpathlineto{\pgfqpoint{2.254473in}{0.668364in}}%
\pgfpathlineto{\pgfqpoint{2.251624in}{0.670004in}}%
\pgfpathlineto{\pgfqpoint{2.240343in}{0.676476in}}%
\pgfpathlineto{\pgfqpoint{2.235200in}{0.679434in}}%
\pgfpathlineto{\pgfqpoint{2.229062in}{0.682903in}}%
\pgfpathlineto{\pgfqpoint{2.217782in}{0.689244in}}%
\pgfpathlineto{\pgfqpoint{2.215612in}{0.690504in}}%
\pgfpathlineto{\pgfqpoint{2.206501in}{0.695763in}}%
\pgfpathlineto{\pgfqpoint{2.196742in}{0.701573in}}%
\pgfpathlineto{\pgfqpoint{2.195220in}{0.702485in}}%
\pgfpathlineto{\pgfqpoint{2.183939in}{0.709232in}}%
\pgfpathlineto{\pgfqpoint{2.178259in}{0.712643in}}%
\pgfpathlineto{\pgfqpoint{2.172658in}{0.715942in}}%
\pgfpathlineto{\pgfqpoint{2.161377in}{0.722596in}}%
\pgfpathlineto{\pgfqpoint{2.159481in}{0.723712in}}%
\pgfpathlineto{\pgfqpoint{2.150097in}{0.729145in}}%
\pgfpathlineto{\pgfqpoint{2.140623in}{0.734782in}}%
\pgfpathlineto{\pgfqpoint{2.138816in}{0.735820in}}%
\pgfpathlineto{\pgfqpoint{2.127535in}{0.742744in}}%
\pgfpathlineto{\pgfqpoint{2.122454in}{0.745851in}}%
\pgfpathlineto{\pgfqpoint{2.116254in}{0.749612in}}%
\pgfpathlineto{\pgfqpoint{2.104973in}{0.756488in}}%
\pgfpathlineto{\pgfqpoint{2.104330in}{0.756921in}}%
\pgfpathlineto{\pgfqpoint{2.093693in}{0.763550in}}%
\pgfpathlineto{\pgfqpoint{2.087167in}{0.767990in}}%
\pgfpathlineto{\pgfqpoint{2.082412in}{0.771139in}}%
\pgfpathlineto{\pgfqpoint{2.071131in}{0.778968in}}%
\pgfpathlineto{\pgfqpoint{2.071006in}{0.779060in}}%
\pgfpathlineto{\pgfqpoint{2.059850in}{0.787121in}}%
\pgfpathlineto{\pgfqpoint{2.055699in}{0.790129in}}%
\pgfpathlineto{\pgfqpoint{2.048569in}{0.795215in}}%
\pgfpathlineto{\pgfqpoint{2.040412in}{0.801199in}}%
\pgfpathlineto{\pgfqpoint{2.037289in}{0.803520in}}%
\pgfpathlineto{\pgfqpoint{2.026089in}{0.812269in}}%
\pgfpathlineto{\pgfqpoint{2.026008in}{0.812334in}}%
\pgfpathlineto{\pgfqpoint{2.014727in}{0.822118in}}%
\pgfpathlineto{\pgfqpoint{2.013404in}{0.823338in}}%
\pgfpathlineto{\pgfqpoint{2.003446in}{0.832602in}}%
\pgfpathlineto{\pgfqpoint{2.001510in}{0.834408in}}%
\pgfpathlineto{\pgfqpoint{1.992165in}{0.842799in}}%
\pgfpathlineto{\pgfqpoint{1.989223in}{0.845477in}}%
\pgfpathlineto{\pgfqpoint{1.980884in}{0.852904in}}%
\pgfpathlineto{\pgfqpoint{1.976736in}{0.856547in}}%
\pgfpathlineto{\pgfqpoint{1.969604in}{0.862889in}}%
\pgfpathlineto{\pgfqpoint{1.964339in}{0.867616in}}%
\pgfpathlineto{\pgfqpoint{1.958323in}{0.872992in}}%
\pgfpathlineto{\pgfqpoint{1.951888in}{0.878686in}}%
\pgfpathlineto{\pgfqpoint{1.947042in}{0.883113in}}%
\pgfpathlineto{\pgfqpoint{1.939522in}{0.889755in}}%
\pgfpathlineto{\pgfqpoint{1.935761in}{0.893256in}}%
\pgfpathlineto{\pgfqpoint{1.927776in}{0.900825in}}%
\pgfpathlineto{\pgfqpoint{1.924480in}{0.904079in}}%
\pgfpathlineto{\pgfqpoint{1.916382in}{0.911894in}}%
\pgfpathlineto{\pgfqpoint{1.913200in}{0.915108in}}%
\pgfpathlineto{\pgfqpoint{1.904900in}{0.922964in}}%
\pgfpathlineto{\pgfqpoint{1.901919in}{0.926627in}}%
\pgfpathlineto{\pgfqpoint{1.895413in}{0.934033in}}%
\pgfpathlineto{\pgfqpoint{1.890638in}{0.940420in}}%
\pgfpathlineto{\pgfqpoint{1.887022in}{0.945103in}}%
\pgfpathlineto{\pgfqpoint{1.879357in}{0.955463in}}%
\pgfpathlineto{\pgfqpoint{1.878829in}{0.956173in}}%
\pgfpathlineto{\pgfqpoint{1.870658in}{0.967242in}}%
\pgfpathlineto{\pgfqpoint{1.868076in}{0.970593in}}%
\pgfpathlineto{\pgfqpoint{1.861989in}{0.978312in}}%
\pgfpathlineto{\pgfqpoint{1.856795in}{0.986015in}}%
\pgfpathlineto{\pgfqpoint{1.856795in}{0.978312in}}%
\pgfpathlineto{\pgfqpoint{1.856795in}{0.967242in}}%
\pgfpathlineto{\pgfqpoint{1.856795in}{0.956173in}}%
\pgfpathlineto{\pgfqpoint{1.856795in}{0.945103in}}%
\pgfpathlineto{\pgfqpoint{1.856795in}{0.942483in}}%
\pgfpathlineto{\pgfqpoint{1.864383in}{0.934033in}}%
\pgfpathlineto{\pgfqpoint{1.868076in}{0.930782in}}%
\pgfpathlineto{\pgfqpoint{1.876437in}{0.922964in}}%
\pgfpathlineto{\pgfqpoint{1.879357in}{0.920363in}}%
\pgfpathlineto{\pgfqpoint{1.888562in}{0.911894in}}%
\pgfpathlineto{\pgfqpoint{1.890638in}{0.910048in}}%
\pgfpathlineto{\pgfqpoint{1.900788in}{0.900825in}}%
\pgfpathlineto{\pgfqpoint{1.901919in}{0.899836in}}%
\pgfpathlineto{\pgfqpoint{1.913019in}{0.889755in}}%
\pgfpathlineto{\pgfqpoint{1.913200in}{0.889596in}}%
\pgfpathlineto{\pgfqpoint{1.924480in}{0.879554in}}%
\pgfpathlineto{\pgfqpoint{1.925471in}{0.878686in}}%
\pgfpathlineto{\pgfqpoint{1.935761in}{0.869904in}}%
\pgfpathlineto{\pgfqpoint{1.938369in}{0.867616in}}%
\pgfpathlineto{\pgfqpoint{1.947042in}{0.860002in}}%
\pgfpathlineto{\pgfqpoint{1.950967in}{0.856547in}}%
\pgfpathlineto{\pgfqpoint{1.958323in}{0.849811in}}%
\pgfpathlineto{\pgfqpoint{1.963155in}{0.845477in}}%
\pgfpathlineto{\pgfqpoint{1.969604in}{0.839606in}}%
\pgfpathlineto{\pgfqpoint{1.975301in}{0.834408in}}%
\pgfpathlineto{\pgfqpoint{1.980884in}{0.829252in}}%
\pgfpathlineto{\pgfqpoint{1.987300in}{0.823338in}}%
\pgfpathlineto{\pgfqpoint{1.992165in}{0.818991in}}%
\pgfpathlineto{\pgfqpoint{1.999548in}{0.812269in}}%
\pgfpathlineto{\pgfqpoint{2.003446in}{0.808805in}}%
\pgfpathlineto{\pgfqpoint{2.012070in}{0.801199in}}%
\pgfpathlineto{\pgfqpoint{2.014727in}{0.798831in}}%
\pgfpathlineto{\pgfqpoint{2.024632in}{0.790129in}}%
\pgfpathlineto{\pgfqpoint{2.026008in}{0.788907in}}%
\pgfpathlineto{\pgfqpoint{2.037289in}{0.779564in}}%
\pgfpathlineto{\pgfqpoint{2.037930in}{0.779060in}}%
\pgfpathlineto{\pgfqpoint{2.048569in}{0.770455in}}%
\pgfpathlineto{\pgfqpoint{2.051896in}{0.767990in}}%
\pgfpathlineto{\pgfqpoint{2.059850in}{0.761815in}}%
\pgfpathlineto{\pgfqpoint{2.066397in}{0.756921in}}%
\pgfpathlineto{\pgfqpoint{2.071131in}{0.753318in}}%
\pgfpathlineto{\pgfqpoint{2.080927in}{0.745851in}}%
\pgfpathlineto{\pgfqpoint{2.082412in}{0.744719in}}%
\pgfpathlineto{\pgfqpoint{2.093693in}{0.736233in}}%
\pgfpathlineto{\pgfqpoint{2.095801in}{0.734782in}}%
\pgfpathlineto{\pgfqpoint{2.104973in}{0.728381in}}%
\pgfpathlineto{\pgfqpoint{2.111698in}{0.723712in}}%
\pgfpathlineto{\pgfqpoint{2.116254in}{0.720507in}}%
\pgfpathlineto{\pgfqpoint{2.127535in}{0.712816in}}%
\pgfpathlineto{\pgfqpoint{2.127810in}{0.712643in}}%
\pgfpathlineto{\pgfqpoint{2.138816in}{0.705681in}}%
\pgfpathlineto{\pgfqpoint{2.145245in}{0.701573in}}%
\pgfpathlineto{\pgfqpoint{2.150097in}{0.698487in}}%
\pgfpathlineto{\pgfqpoint{2.161377in}{0.691476in}}%
\pgfpathlineto{\pgfqpoint{2.163020in}{0.690504in}}%
\pgfpathlineto{\pgfqpoint{2.172658in}{0.684679in}}%
\pgfpathlineto{\pgfqpoint{2.181573in}{0.679434in}}%
\pgfpathlineto{\pgfqpoint{2.183939in}{0.678029in}}%
\pgfpathlineto{\pgfqpoint{2.195220in}{0.671324in}}%
\pgfpathlineto{\pgfqpoint{2.200200in}{0.668364in}}%
\pgfpathlineto{\pgfqpoint{2.206501in}{0.664582in}}%
\pgfpathlineto{\pgfqpoint{2.217782in}{0.657762in}}%
\pgfpathlineto{\pgfqpoint{2.218572in}{0.657295in}}%
\pgfpathlineto{\pgfqpoint{2.229062in}{0.650961in}}%
\pgfpathlineto{\pgfqpoint{2.237094in}{0.646225in}}%
\pgfpathlineto{\pgfqpoint{2.240343in}{0.644278in}}%
\pgfpathlineto{\pgfqpoint{2.251624in}{0.637451in}}%
\pgfpathlineto{\pgfqpoint{2.255351in}{0.635156in}}%
\pgfpathlineto{\pgfqpoint{2.262905in}{0.630257in}}%
\pgfpathlineto{\pgfqpoint{2.271858in}{0.624086in}}%
\pgfpathlineto{\pgfqpoint{2.274186in}{0.622398in}}%
\pgfpathlineto{\pgfqpoint{2.285466in}{0.613835in}}%
\pgfpathlineto{\pgfqpoint{2.286519in}{0.613017in}}%
\pgfpathlineto{\pgfqpoint{2.296747in}{0.604886in}}%
\pgfpathlineto{\pgfqpoint{2.300292in}{0.601947in}}%
\pgfpathlineto{\pgfqpoint{2.308028in}{0.595288in}}%
\pgfpathlineto{\pgfqpoint{2.312997in}{0.590878in}}%
\pgfpathlineto{\pgfqpoint{2.319309in}{0.585022in}}%
\pgfpathlineto{\pgfqpoint{2.324815in}{0.579808in}}%
\pgfpathlineto{\pgfqpoint{2.330590in}{0.574074in}}%
\pgfpathlineto{\pgfqpoint{2.335871in}{0.568739in}}%
\pgfpathlineto{\pgfqpoint{2.341871in}{0.562528in}}%
\pgfpathlineto{\pgfqpoint{2.346513in}{0.557669in}}%
\pgfpathlineto{\pgfqpoint{2.353151in}{0.550589in}}%
\pgfpathlineto{\pgfqpoint{2.356709in}{0.546600in}}%
\pgfpathlineto{\pgfqpoint{2.364432in}{0.537715in}}%
\pgfpathlineto{\pgfqpoint{2.366267in}{0.535530in}}%
\pgfpathlineto{\pgfqpoint{2.375406in}{0.524460in}}%
\pgfpathlineto{\pgfqpoint{2.375713in}{0.524074in}}%
\pgfpathlineto{\pgfqpoint{2.384120in}{0.513391in}}%
\pgfpathlineto{\pgfqpoint{2.386994in}{0.509643in}}%
\pgfpathlineto{\pgfqpoint{2.392375in}{0.502321in}}%
\pgfpathclose%
\pgfusepath{fill}%
\end{pgfscope}%
\begin{pgfscope}%
\pgfpathrectangle{\pgfqpoint{1.856795in}{0.423750in}}{\pgfqpoint{1.194205in}{1.163386in}}%
\pgfusepath{clip}%
\pgfsetbuttcap%
\pgfsetroundjoin%
\definecolor{currentfill}{rgb}{0.940366,0.360209,0.257347}%
\pgfsetfillcolor{currentfill}%
\pgfsetlinewidth{0.000000pt}%
\definecolor{currentstroke}{rgb}{0.000000,0.000000,0.000000}%
\pgfsetstrokecolor{currentstroke}%
\pgfsetdash{}{0pt}%
\pgfpathmoveto{\pgfqpoint{2.657733in}{1.154234in}}%
\pgfpathlineto{\pgfqpoint{2.669014in}{1.150726in}}%
\pgfpathlineto{\pgfqpoint{2.680295in}{1.148169in}}%
\pgfpathlineto{\pgfqpoint{2.691576in}{1.146928in}}%
\pgfpathlineto{\pgfqpoint{2.702857in}{1.147055in}}%
\pgfpathlineto{\pgfqpoint{2.714137in}{1.149150in}}%
\pgfpathlineto{\pgfqpoint{2.725418in}{1.155176in}}%
\pgfpathlineto{\pgfqpoint{2.725794in}{1.155424in}}%
\pgfpathlineto{\pgfqpoint{2.736699in}{1.162920in}}%
\pgfpathlineto{\pgfqpoint{2.741613in}{1.166494in}}%
\pgfpathlineto{\pgfqpoint{2.747980in}{1.171197in}}%
\pgfpathlineto{\pgfqpoint{2.756235in}{1.177563in}}%
\pgfpathlineto{\pgfqpoint{2.759261in}{1.179917in}}%
\pgfpathlineto{\pgfqpoint{2.770129in}{1.188633in}}%
\pgfpathlineto{\pgfqpoint{2.770542in}{1.188961in}}%
\pgfpathlineto{\pgfqpoint{2.781822in}{1.198121in}}%
\pgfpathlineto{\pgfqpoint{2.783708in}{1.199702in}}%
\pgfpathlineto{\pgfqpoint{2.793103in}{1.207672in}}%
\pgfpathlineto{\pgfqpoint{2.796642in}{1.210772in}}%
\pgfpathlineto{\pgfqpoint{2.804384in}{1.217634in}}%
\pgfpathlineto{\pgfqpoint{2.809401in}{1.221842in}}%
\pgfpathlineto{\pgfqpoint{2.815665in}{1.226815in}}%
\pgfpathlineto{\pgfqpoint{2.822783in}{1.232911in}}%
\pgfpathlineto{\pgfqpoint{2.826946in}{1.236378in}}%
\pgfpathlineto{\pgfqpoint{2.835892in}{1.243981in}}%
\pgfpathlineto{\pgfqpoint{2.838226in}{1.245988in}}%
\pgfpathlineto{\pgfqpoint{2.848444in}{1.255050in}}%
\pgfpathlineto{\pgfqpoint{2.849507in}{1.255975in}}%
\pgfpathlineto{\pgfqpoint{2.860764in}{1.266120in}}%
\pgfpathlineto{\pgfqpoint{2.860788in}{1.266140in}}%
\pgfpathlineto{\pgfqpoint{2.872069in}{1.276297in}}%
\pgfpathlineto{\pgfqpoint{2.873016in}{1.277189in}}%
\pgfpathlineto{\pgfqpoint{2.883350in}{1.286515in}}%
\pgfpathlineto{\pgfqpoint{2.885220in}{1.288259in}}%
\pgfpathlineto{\pgfqpoint{2.894631in}{1.297084in}}%
\pgfpathlineto{\pgfqpoint{2.896979in}{1.299328in}}%
\pgfpathlineto{\pgfqpoint{2.905911in}{1.308040in}}%
\pgfpathlineto{\pgfqpoint{2.908291in}{1.310398in}}%
\pgfpathlineto{\pgfqpoint{2.917192in}{1.319022in}}%
\pgfpathlineto{\pgfqpoint{2.919709in}{1.321467in}}%
\pgfpathlineto{\pgfqpoint{2.928473in}{1.329888in}}%
\pgfpathlineto{\pgfqpoint{2.931190in}{1.332537in}}%
\pgfpathlineto{\pgfqpoint{2.939754in}{1.340942in}}%
\pgfpathlineto{\pgfqpoint{2.942479in}{1.343606in}}%
\pgfpathlineto{\pgfqpoint{2.951035in}{1.351777in}}%
\pgfpathlineto{\pgfqpoint{2.953997in}{1.354676in}}%
\pgfpathlineto{\pgfqpoint{2.962315in}{1.362683in}}%
\pgfpathlineto{\pgfqpoint{2.965468in}{1.365746in}}%
\pgfpathlineto{\pgfqpoint{2.973596in}{1.373600in}}%
\pgfpathlineto{\pgfqpoint{2.973596in}{1.376815in}}%
\pgfpathlineto{\pgfqpoint{2.973596in}{1.387885in}}%
\pgfpathlineto{\pgfqpoint{2.973596in}{1.398954in}}%
\pgfpathlineto{\pgfqpoint{2.973596in}{1.410024in}}%
\pgfpathlineto{\pgfqpoint{2.973596in}{1.417017in}}%
\pgfpathlineto{\pgfqpoint{2.966428in}{1.410024in}}%
\pgfpathlineto{\pgfqpoint{2.962315in}{1.406035in}}%
\pgfpathlineto{\pgfqpoint{2.954927in}{1.398954in}}%
\pgfpathlineto{\pgfqpoint{2.951035in}{1.395242in}}%
\pgfpathlineto{\pgfqpoint{2.943317in}{1.387885in}}%
\pgfpathlineto{\pgfqpoint{2.939754in}{1.384508in}}%
\pgfpathlineto{\pgfqpoint{2.931654in}{1.376815in}}%
\pgfpathlineto{\pgfqpoint{2.928473in}{1.373814in}}%
\pgfpathlineto{\pgfqpoint{2.919795in}{1.365746in}}%
\pgfpathlineto{\pgfqpoint{2.917192in}{1.363345in}}%
\pgfpathlineto{\pgfqpoint{2.907647in}{1.354676in}}%
\pgfpathlineto{\pgfqpoint{2.905911in}{1.353112in}}%
\pgfpathlineto{\pgfqpoint{2.895178in}{1.343606in}}%
\pgfpathlineto{\pgfqpoint{2.894631in}{1.343126in}}%
\pgfpathlineto{\pgfqpoint{2.883350in}{1.333418in}}%
\pgfpathlineto{\pgfqpoint{2.882303in}{1.332537in}}%
\pgfpathlineto{\pgfqpoint{2.872069in}{1.323972in}}%
\pgfpathlineto{\pgfqpoint{2.869054in}{1.321467in}}%
\pgfpathlineto{\pgfqpoint{2.860788in}{1.314201in}}%
\pgfpathlineto{\pgfqpoint{2.856345in}{1.310398in}}%
\pgfpathlineto{\pgfqpoint{2.849507in}{1.304541in}}%
\pgfpathlineto{\pgfqpoint{2.843257in}{1.299328in}}%
\pgfpathlineto{\pgfqpoint{2.838226in}{1.294926in}}%
\pgfpathlineto{\pgfqpoint{2.829933in}{1.288259in}}%
\pgfpathlineto{\pgfqpoint{2.826946in}{1.285731in}}%
\pgfpathlineto{\pgfqpoint{2.816395in}{1.277189in}}%
\pgfpathlineto{\pgfqpoint{2.815665in}{1.276592in}}%
\pgfpathlineto{\pgfqpoint{2.804384in}{1.267389in}}%
\pgfpathlineto{\pgfqpoint{2.802836in}{1.266120in}}%
\pgfpathlineto{\pgfqpoint{2.793103in}{1.257953in}}%
\pgfpathlineto{\pgfqpoint{2.789577in}{1.255050in}}%
\pgfpathlineto{\pgfqpoint{2.781822in}{1.248790in}}%
\pgfpathlineto{\pgfqpoint{2.775743in}{1.243981in}}%
\pgfpathlineto{\pgfqpoint{2.770542in}{1.239820in}}%
\pgfpathlineto{\pgfqpoint{2.761355in}{1.232911in}}%
\pgfpathlineto{\pgfqpoint{2.759261in}{1.231264in}}%
\pgfpathlineto{\pgfqpoint{2.747980in}{1.222802in}}%
\pgfpathlineto{\pgfqpoint{2.746558in}{1.221842in}}%
\pgfpathlineto{\pgfqpoint{2.736699in}{1.214853in}}%
\pgfpathlineto{\pgfqpoint{2.730856in}{1.210772in}}%
\pgfpathlineto{\pgfqpoint{2.725418in}{1.206952in}}%
\pgfpathlineto{\pgfqpoint{2.714137in}{1.202128in}}%
\pgfpathlineto{\pgfqpoint{2.702857in}{1.200790in}}%
\pgfpathlineto{\pgfqpoint{2.691576in}{1.200732in}}%
\pgfpathlineto{\pgfqpoint{2.680295in}{1.202103in}}%
\pgfpathlineto{\pgfqpoint{2.669014in}{1.205062in}}%
\pgfpathlineto{\pgfqpoint{2.657733in}{1.208838in}}%
\pgfpathlineto{\pgfqpoint{2.652783in}{1.210772in}}%
\pgfpathlineto{\pgfqpoint{2.646453in}{1.213268in}}%
\pgfpathlineto{\pgfqpoint{2.635172in}{1.218018in}}%
\pgfpathlineto{\pgfqpoint{2.626346in}{1.221842in}}%
\pgfpathlineto{\pgfqpoint{2.623891in}{1.222898in}}%
\pgfpathlineto{\pgfqpoint{2.612610in}{1.228139in}}%
\pgfpathlineto{\pgfqpoint{2.602976in}{1.232911in}}%
\pgfpathlineto{\pgfqpoint{2.601329in}{1.233717in}}%
\pgfpathlineto{\pgfqpoint{2.590048in}{1.239743in}}%
\pgfpathlineto{\pgfqpoint{2.582531in}{1.243981in}}%
\pgfpathlineto{\pgfqpoint{2.578768in}{1.246130in}}%
\pgfpathlineto{\pgfqpoint{2.567487in}{1.253410in}}%
\pgfpathlineto{\pgfqpoint{2.565030in}{1.255050in}}%
\pgfpathlineto{\pgfqpoint{2.556206in}{1.261044in}}%
\pgfpathlineto{\pgfqpoint{2.548879in}{1.266120in}}%
\pgfpathlineto{\pgfqpoint{2.544925in}{1.268903in}}%
\pgfpathlineto{\pgfqpoint{2.533644in}{1.276959in}}%
\pgfpathlineto{\pgfqpoint{2.533317in}{1.277189in}}%
\pgfpathlineto{\pgfqpoint{2.522364in}{1.285054in}}%
\pgfpathlineto{\pgfqpoint{2.517926in}{1.288259in}}%
\pgfpathlineto{\pgfqpoint{2.511083in}{1.293322in}}%
\pgfpathlineto{\pgfqpoint{2.503681in}{1.299328in}}%
\pgfpathlineto{\pgfqpoint{2.499802in}{1.302569in}}%
\pgfpathlineto{\pgfqpoint{2.490602in}{1.310398in}}%
\pgfpathlineto{\pgfqpoint{2.488521in}{1.312133in}}%
\pgfpathlineto{\pgfqpoint{2.477552in}{1.321467in}}%
\pgfpathlineto{\pgfqpoint{2.477240in}{1.321730in}}%
\pgfpathlineto{\pgfqpoint{2.465960in}{1.331558in}}%
\pgfpathlineto{\pgfqpoint{2.464861in}{1.332537in}}%
\pgfpathlineto{\pgfqpoint{2.454679in}{1.341735in}}%
\pgfpathlineto{\pgfqpoint{2.452610in}{1.343606in}}%
\pgfpathlineto{\pgfqpoint{2.443398in}{1.352250in}}%
\pgfpathlineto{\pgfqpoint{2.440889in}{1.354676in}}%
\pgfpathlineto{\pgfqpoint{2.432117in}{1.363465in}}%
\pgfpathlineto{\pgfqpoint{2.429835in}{1.365746in}}%
\pgfpathlineto{\pgfqpoint{2.420836in}{1.374760in}}%
\pgfpathlineto{\pgfqpoint{2.418806in}{1.376815in}}%
\pgfpathlineto{\pgfqpoint{2.409555in}{1.386067in}}%
\pgfpathlineto{\pgfqpoint{2.407771in}{1.387885in}}%
\pgfpathlineto{\pgfqpoint{2.398275in}{1.397549in}}%
\pgfpathlineto{\pgfqpoint{2.396963in}{1.398954in}}%
\pgfpathlineto{\pgfqpoint{2.387156in}{1.410024in}}%
\pgfpathlineto{\pgfqpoint{2.386994in}{1.410207in}}%
\pgfpathlineto{\pgfqpoint{2.377357in}{1.421093in}}%
\pgfpathlineto{\pgfqpoint{2.375713in}{1.422929in}}%
\pgfpathlineto{\pgfqpoint{2.367515in}{1.432163in}}%
\pgfpathlineto{\pgfqpoint{2.364432in}{1.435650in}}%
\pgfpathlineto{\pgfqpoint{2.357764in}{1.443232in}}%
\pgfpathlineto{\pgfqpoint{2.353151in}{1.448477in}}%
\pgfpathlineto{\pgfqpoint{2.348073in}{1.454302in}}%
\pgfpathlineto{\pgfqpoint{2.341871in}{1.461304in}}%
\pgfpathlineto{\pgfqpoint{2.338268in}{1.465371in}}%
\pgfpathlineto{\pgfqpoint{2.330590in}{1.474159in}}%
\pgfpathlineto{\pgfqpoint{2.328595in}{1.476441in}}%
\pgfpathlineto{\pgfqpoint{2.319309in}{1.487123in}}%
\pgfpathlineto{\pgfqpoint{2.318996in}{1.487510in}}%
\pgfpathlineto{\pgfqpoint{2.309746in}{1.498580in}}%
\pgfpathlineto{\pgfqpoint{2.308028in}{1.500692in}}%
\pgfpathlineto{\pgfqpoint{2.300792in}{1.509650in}}%
\pgfpathlineto{\pgfqpoint{2.296747in}{1.514729in}}%
\pgfpathlineto{\pgfqpoint{2.291975in}{1.520719in}}%
\pgfpathlineto{\pgfqpoint{2.285466in}{1.529014in}}%
\pgfpathlineto{\pgfqpoint{2.283280in}{1.531789in}}%
\pgfpathlineto{\pgfqpoint{2.274703in}{1.542858in}}%
\pgfpathlineto{\pgfqpoint{2.274186in}{1.543536in}}%
\pgfpathlineto{\pgfqpoint{2.266226in}{1.553928in}}%
\pgfpathlineto{\pgfqpoint{2.262905in}{1.558334in}}%
\pgfpathlineto{\pgfqpoint{2.257865in}{1.564997in}}%
\pgfpathlineto{\pgfqpoint{2.251624in}{1.573364in}}%
\pgfpathlineto{\pgfqpoint{2.249606in}{1.576067in}}%
\pgfpathlineto{\pgfqpoint{2.241442in}{1.587136in}}%
\pgfpathlineto{\pgfqpoint{2.240343in}{1.587136in}}%
\pgfpathlineto{\pgfqpoint{2.229062in}{1.587136in}}%
\pgfpathlineto{\pgfqpoint{2.217782in}{1.587136in}}%
\pgfpathlineto{\pgfqpoint{2.206501in}{1.587136in}}%
\pgfpathlineto{\pgfqpoint{2.195220in}{1.587136in}}%
\pgfpathlineto{\pgfqpoint{2.189745in}{1.587136in}}%
\pgfpathlineto{\pgfqpoint{2.195220in}{1.579760in}}%
\pgfpathlineto{\pgfqpoint{2.197962in}{1.576067in}}%
\pgfpathlineto{\pgfqpoint{2.206246in}{1.564997in}}%
\pgfpathlineto{\pgfqpoint{2.206501in}{1.564657in}}%
\pgfpathlineto{\pgfqpoint{2.214512in}{1.553928in}}%
\pgfpathlineto{\pgfqpoint{2.217782in}{1.549617in}}%
\pgfpathlineto{\pgfqpoint{2.222899in}{1.542858in}}%
\pgfpathlineto{\pgfqpoint{2.229062in}{1.534820in}}%
\pgfpathlineto{\pgfqpoint{2.231390in}{1.531789in}}%
\pgfpathlineto{\pgfqpoint{2.239985in}{1.520719in}}%
\pgfpathlineto{\pgfqpoint{2.240343in}{1.520247in}}%
\pgfpathlineto{\pgfqpoint{2.248497in}{1.509650in}}%
\pgfpathlineto{\pgfqpoint{2.251624in}{1.505392in}}%
\pgfpathlineto{\pgfqpoint{2.256741in}{1.498580in}}%
\pgfpathlineto{\pgfqpoint{2.262905in}{1.489981in}}%
\pgfpathlineto{\pgfqpoint{2.264678in}{1.487510in}}%
\pgfpathlineto{\pgfqpoint{2.272804in}{1.476441in}}%
\pgfpathlineto{\pgfqpoint{2.274186in}{1.474562in}}%
\pgfpathlineto{\pgfqpoint{2.281020in}{1.465371in}}%
\pgfpathlineto{\pgfqpoint{2.285466in}{1.459270in}}%
\pgfpathlineto{\pgfqpoint{2.289114in}{1.454302in}}%
\pgfpathlineto{\pgfqpoint{2.296747in}{1.444063in}}%
\pgfpathlineto{\pgfqpoint{2.297397in}{1.443232in}}%
\pgfpathlineto{\pgfqpoint{2.306174in}{1.432163in}}%
\pgfpathlineto{\pgfqpoint{2.308028in}{1.429934in}}%
\pgfpathlineto{\pgfqpoint{2.315407in}{1.421093in}}%
\pgfpathlineto{\pgfqpoint{2.319309in}{1.416499in}}%
\pgfpathlineto{\pgfqpoint{2.324838in}{1.410024in}}%
\pgfpathlineto{\pgfqpoint{2.330590in}{1.403370in}}%
\pgfpathlineto{\pgfqpoint{2.334411in}{1.398954in}}%
\pgfpathlineto{\pgfqpoint{2.341871in}{1.390543in}}%
\pgfpathlineto{\pgfqpoint{2.344220in}{1.387885in}}%
\pgfpathlineto{\pgfqpoint{2.353151in}{1.378035in}}%
\pgfpathlineto{\pgfqpoint{2.354255in}{1.376815in}}%
\pgfpathlineto{\pgfqpoint{2.364294in}{1.365746in}}%
\pgfpathlineto{\pgfqpoint{2.364432in}{1.365598in}}%
\pgfpathlineto{\pgfqpoint{2.375074in}{1.354676in}}%
\pgfpathlineto{\pgfqpoint{2.375713in}{1.354045in}}%
\pgfpathlineto{\pgfqpoint{2.386237in}{1.343606in}}%
\pgfpathlineto{\pgfqpoint{2.386994in}{1.342866in}}%
\pgfpathlineto{\pgfqpoint{2.397385in}{1.332537in}}%
\pgfpathlineto{\pgfqpoint{2.398275in}{1.331668in}}%
\pgfpathlineto{\pgfqpoint{2.408657in}{1.321467in}}%
\pgfpathlineto{\pgfqpoint{2.409555in}{1.320609in}}%
\pgfpathlineto{\pgfqpoint{2.420679in}{1.310398in}}%
\pgfpathlineto{\pgfqpoint{2.420836in}{1.310254in}}%
\pgfpathlineto{\pgfqpoint{2.432117in}{1.300069in}}%
\pgfpathlineto{\pgfqpoint{2.432955in}{1.299328in}}%
\pgfpathlineto{\pgfqpoint{2.443398in}{1.289995in}}%
\pgfpathlineto{\pgfqpoint{2.445437in}{1.288259in}}%
\pgfpathlineto{\pgfqpoint{2.454679in}{1.280341in}}%
\pgfpathlineto{\pgfqpoint{2.458347in}{1.277189in}}%
\pgfpathlineto{\pgfqpoint{2.465960in}{1.270734in}}%
\pgfpathlineto{\pgfqpoint{2.471385in}{1.266120in}}%
\pgfpathlineto{\pgfqpoint{2.477240in}{1.261172in}}%
\pgfpathlineto{\pgfqpoint{2.484577in}{1.255050in}}%
\pgfpathlineto{\pgfqpoint{2.488521in}{1.251766in}}%
\pgfpathlineto{\pgfqpoint{2.498244in}{1.243981in}}%
\pgfpathlineto{\pgfqpoint{2.499802in}{1.242747in}}%
\pgfpathlineto{\pgfqpoint{2.511083in}{1.234885in}}%
\pgfpathlineto{\pgfqpoint{2.514003in}{1.232911in}}%
\pgfpathlineto{\pgfqpoint{2.522364in}{1.227250in}}%
\pgfpathlineto{\pgfqpoint{2.530340in}{1.221842in}}%
\pgfpathlineto{\pgfqpoint{2.533644in}{1.219595in}}%
\pgfpathlineto{\pgfqpoint{2.544925in}{1.211959in}}%
\pgfpathlineto{\pgfqpoint{2.546714in}{1.210772in}}%
\pgfpathlineto{\pgfqpoint{2.556206in}{1.204466in}}%
\pgfpathlineto{\pgfqpoint{2.564075in}{1.199702in}}%
\pgfpathlineto{\pgfqpoint{2.567487in}{1.197646in}}%
\pgfpathlineto{\pgfqpoint{2.578768in}{1.190879in}}%
\pgfpathlineto{\pgfqpoint{2.582864in}{1.188633in}}%
\pgfpathlineto{\pgfqpoint{2.590048in}{1.184614in}}%
\pgfpathlineto{\pgfqpoint{2.601329in}{1.178549in}}%
\pgfpathlineto{\pgfqpoint{2.603242in}{1.177563in}}%
\pgfpathlineto{\pgfqpoint{2.612610in}{1.172692in}}%
\pgfpathlineto{\pgfqpoint{2.623891in}{1.167148in}}%
\pgfpathlineto{\pgfqpoint{2.625346in}{1.166494in}}%
\pgfpathlineto{\pgfqpoint{2.635172in}{1.162250in}}%
\pgfpathlineto{\pgfqpoint{2.646453in}{1.158045in}}%
\pgfpathlineto{\pgfqpoint{2.654287in}{1.155424in}}%
\pgfpathclose%
\pgfusepath{fill}%
\end{pgfscope}%
\begin{pgfscope}%
\pgfpathrectangle{\pgfqpoint{1.856795in}{0.423750in}}{\pgfqpoint{1.194205in}{1.163386in}}%
\pgfusepath{clip}%
\pgfsetbuttcap%
\pgfsetroundjoin%
\definecolor{currentfill}{rgb}{0.952404,0.449449,0.307210}%
\pgfsetfillcolor{currentfill}%
\pgfsetlinewidth{0.000000pt}%
\definecolor{currentstroke}{rgb}{0.000000,0.000000,0.000000}%
\pgfsetstrokecolor{currentstroke}%
\pgfsetdash{}{0pt}%
\pgfpathmoveto{\pgfqpoint{2.432117in}{0.498744in}}%
\pgfpathlineto{\pgfqpoint{2.436880in}{0.491252in}}%
\pgfpathlineto{\pgfqpoint{2.443398in}{0.491252in}}%
\pgfpathlineto{\pgfqpoint{2.454679in}{0.491252in}}%
\pgfpathlineto{\pgfqpoint{2.465960in}{0.491252in}}%
\pgfpathlineto{\pgfqpoint{2.477240in}{0.491252in}}%
\pgfpathlineto{\pgfqpoint{2.481272in}{0.491252in}}%
\pgfpathlineto{\pgfqpoint{2.477240in}{0.497273in}}%
\pgfpathlineto{\pgfqpoint{2.473843in}{0.502321in}}%
\pgfpathlineto{\pgfqpoint{2.466112in}{0.513391in}}%
\pgfpathlineto{\pgfqpoint{2.465960in}{0.513605in}}%
\pgfpathlineto{\pgfqpoint{2.458141in}{0.524460in}}%
\pgfpathlineto{\pgfqpoint{2.454679in}{0.529036in}}%
\pgfpathlineto{\pgfqpoint{2.449727in}{0.535530in}}%
\pgfpathlineto{\pgfqpoint{2.443398in}{0.542993in}}%
\pgfpathlineto{\pgfqpoint{2.440424in}{0.546600in}}%
\pgfpathlineto{\pgfqpoint{2.432117in}{0.556237in}}%
\pgfpathlineto{\pgfqpoint{2.430981in}{0.557669in}}%
\pgfpathlineto{\pgfqpoint{2.421666in}{0.568739in}}%
\pgfpathlineto{\pgfqpoint{2.420836in}{0.569663in}}%
\pgfpathlineto{\pgfqpoint{2.412347in}{0.579808in}}%
\pgfpathlineto{\pgfqpoint{2.409555in}{0.583065in}}%
\pgfpathlineto{\pgfqpoint{2.402774in}{0.590878in}}%
\pgfpathlineto{\pgfqpoint{2.398275in}{0.595951in}}%
\pgfpathlineto{\pgfqpoint{2.392868in}{0.601947in}}%
\pgfpathlineto{\pgfqpoint{2.386994in}{0.608308in}}%
\pgfpathlineto{\pgfqpoint{2.382555in}{0.613017in}}%
\pgfpathlineto{\pgfqpoint{2.375713in}{0.620028in}}%
\pgfpathlineto{\pgfqpoint{2.371709in}{0.624086in}}%
\pgfpathlineto{\pgfqpoint{2.364432in}{0.630970in}}%
\pgfpathlineto{\pgfqpoint{2.359447in}{0.635156in}}%
\pgfpathlineto{\pgfqpoint{2.353151in}{0.640248in}}%
\pgfpathlineto{\pgfqpoint{2.345562in}{0.646225in}}%
\pgfpathlineto{\pgfqpoint{2.341871in}{0.649020in}}%
\pgfpathlineto{\pgfqpoint{2.330590in}{0.657058in}}%
\pgfpathlineto{\pgfqpoint{2.330245in}{0.657295in}}%
\pgfpathlineto{\pgfqpoint{2.319309in}{0.664342in}}%
\pgfpathlineto{\pgfqpoint{2.312955in}{0.668364in}}%
\pgfpathlineto{\pgfqpoint{2.308028in}{0.671371in}}%
\pgfpathlineto{\pgfqpoint{2.296747in}{0.677981in}}%
\pgfpathlineto{\pgfqpoint{2.294151in}{0.679434in}}%
\pgfpathlineto{\pgfqpoint{2.285466in}{0.684354in}}%
\pgfpathlineto{\pgfqpoint{2.274297in}{0.690504in}}%
\pgfpathlineto{\pgfqpoint{2.274186in}{0.690564in}}%
\pgfpathlineto{\pgfqpoint{2.262905in}{0.696682in}}%
\pgfpathlineto{\pgfqpoint{2.253964in}{0.701573in}}%
\pgfpathlineto{\pgfqpoint{2.251624in}{0.702842in}}%
\pgfpathlineto{\pgfqpoint{2.240343in}{0.708823in}}%
\pgfpathlineto{\pgfqpoint{2.233073in}{0.712643in}}%
\pgfpathlineto{\pgfqpoint{2.229062in}{0.714724in}}%
\pgfpathlineto{\pgfqpoint{2.217782in}{0.720615in}}%
\pgfpathlineto{\pgfqpoint{2.212105in}{0.723712in}}%
\pgfpathlineto{\pgfqpoint{2.206501in}{0.726637in}}%
\pgfpathlineto{\pgfqpoint{2.195220in}{0.732826in}}%
\pgfpathlineto{\pgfqpoint{2.191814in}{0.734782in}}%
\pgfpathlineto{\pgfqpoint{2.183939in}{0.739287in}}%
\pgfpathlineto{\pgfqpoint{2.172658in}{0.745736in}}%
\pgfpathlineto{\pgfqpoint{2.172457in}{0.745851in}}%
\pgfpathlineto{\pgfqpoint{2.161377in}{0.752159in}}%
\pgfpathlineto{\pgfqpoint{2.152939in}{0.756921in}}%
\pgfpathlineto{\pgfqpoint{2.150097in}{0.758511in}}%
\pgfpathlineto{\pgfqpoint{2.138816in}{0.764865in}}%
\pgfpathlineto{\pgfqpoint{2.133342in}{0.767990in}}%
\pgfpathlineto{\pgfqpoint{2.127535in}{0.771220in}}%
\pgfpathlineto{\pgfqpoint{2.116254in}{0.777739in}}%
\pgfpathlineto{\pgfqpoint{2.114015in}{0.779060in}}%
\pgfpathlineto{\pgfqpoint{2.104973in}{0.784401in}}%
\pgfpathlineto{\pgfqpoint{2.095134in}{0.790129in}}%
\pgfpathlineto{\pgfqpoint{2.093693in}{0.791035in}}%
\pgfpathlineto{\pgfqpoint{2.082412in}{0.797799in}}%
\pgfpathlineto{\pgfqpoint{2.076758in}{0.801199in}}%
\pgfpathlineto{\pgfqpoint{2.071131in}{0.804649in}}%
\pgfpathlineto{\pgfqpoint{2.059850in}{0.812123in}}%
\pgfpathlineto{\pgfqpoint{2.059655in}{0.812269in}}%
\pgfpathlineto{\pgfqpoint{2.048569in}{0.820585in}}%
\pgfpathlineto{\pgfqpoint{2.044918in}{0.823338in}}%
\pgfpathlineto{\pgfqpoint{2.037289in}{0.828772in}}%
\pgfpathlineto{\pgfqpoint{2.029650in}{0.834408in}}%
\pgfpathlineto{\pgfqpoint{2.026008in}{0.836979in}}%
\pgfpathlineto{\pgfqpoint{2.015324in}{0.845477in}}%
\pgfpathlineto{\pgfqpoint{2.014727in}{0.845935in}}%
\pgfpathlineto{\pgfqpoint{2.003446in}{0.855479in}}%
\pgfpathlineto{\pgfqpoint{2.002251in}{0.856547in}}%
\pgfpathlineto{\pgfqpoint{1.992165in}{0.865621in}}%
\pgfpathlineto{\pgfqpoint{1.989908in}{0.867616in}}%
\pgfpathlineto{\pgfqpoint{1.980884in}{0.875956in}}%
\pgfpathlineto{\pgfqpoint{1.977906in}{0.878686in}}%
\pgfpathlineto{\pgfqpoint{1.969604in}{0.886438in}}%
\pgfpathlineto{\pgfqpoint{1.966015in}{0.889755in}}%
\pgfpathlineto{\pgfqpoint{1.958323in}{0.897016in}}%
\pgfpathlineto{\pgfqpoint{1.954216in}{0.900825in}}%
\pgfpathlineto{\pgfqpoint{1.947042in}{0.907954in}}%
\pgfpathlineto{\pgfqpoint{1.943076in}{0.911894in}}%
\pgfpathlineto{\pgfqpoint{1.935761in}{0.920955in}}%
\pgfpathlineto{\pgfqpoint{1.933997in}{0.922964in}}%
\pgfpathlineto{\pgfqpoint{1.925941in}{0.934033in}}%
\pgfpathlineto{\pgfqpoint{1.924480in}{0.936116in}}%
\pgfpathlineto{\pgfqpoint{1.918065in}{0.945103in}}%
\pgfpathlineto{\pgfqpoint{1.913200in}{0.951980in}}%
\pgfpathlineto{\pgfqpoint{1.910169in}{0.956173in}}%
\pgfpathlineto{\pgfqpoint{1.902234in}{0.967242in}}%
\pgfpathlineto{\pgfqpoint{1.901919in}{0.967757in}}%
\pgfpathlineto{\pgfqpoint{1.894890in}{0.978312in}}%
\pgfpathlineto{\pgfqpoint{1.890638in}{0.986405in}}%
\pgfpathlineto{\pgfqpoint{1.889029in}{0.989381in}}%
\pgfpathlineto{\pgfqpoint{1.885375in}{1.000451in}}%
\pgfpathlineto{\pgfqpoint{1.882256in}{1.011520in}}%
\pgfpathlineto{\pgfqpoint{1.879357in}{1.021466in}}%
\pgfpathlineto{\pgfqpoint{1.879016in}{1.022590in}}%
\pgfpathlineto{\pgfqpoint{1.875778in}{1.033659in}}%
\pgfpathlineto{\pgfqpoint{1.873959in}{1.044729in}}%
\pgfpathlineto{\pgfqpoint{1.874321in}{1.055798in}}%
\pgfpathlineto{\pgfqpoint{1.874868in}{1.066868in}}%
\pgfpathlineto{\pgfqpoint{1.874921in}{1.077937in}}%
\pgfpathlineto{\pgfqpoint{1.874112in}{1.089007in}}%
\pgfpathlineto{\pgfqpoint{1.868237in}{1.100077in}}%
\pgfpathlineto{\pgfqpoint{1.868076in}{1.100283in}}%
\pgfpathlineto{\pgfqpoint{1.859557in}{1.111146in}}%
\pgfpathlineto{\pgfqpoint{1.856795in}{1.114492in}}%
\pgfpathlineto{\pgfqpoint{1.856795in}{1.111146in}}%
\pgfpathlineto{\pgfqpoint{1.856795in}{1.100077in}}%
\pgfpathlineto{\pgfqpoint{1.856795in}{1.089007in}}%
\pgfpathlineto{\pgfqpoint{1.856795in}{1.077937in}}%
\pgfpathlineto{\pgfqpoint{1.856795in}{1.066868in}}%
\pgfpathlineto{\pgfqpoint{1.856795in}{1.055798in}}%
\pgfpathlineto{\pgfqpoint{1.856795in}{1.044729in}}%
\pgfpathlineto{\pgfqpoint{1.856795in}{1.033659in}}%
\pgfpathlineto{\pgfqpoint{1.856795in}{1.022590in}}%
\pgfpathlineto{\pgfqpoint{1.856795in}{1.011520in}}%
\pgfpathlineto{\pgfqpoint{1.856795in}{1.000451in}}%
\pgfpathlineto{\pgfqpoint{1.856795in}{0.989381in}}%
\pgfpathlineto{\pgfqpoint{1.856795in}{0.986015in}}%
\pgfpathlineto{\pgfqpoint{1.861989in}{0.978312in}}%
\pgfpathlineto{\pgfqpoint{1.868076in}{0.970593in}}%
\pgfpathlineto{\pgfqpoint{1.870658in}{0.967242in}}%
\pgfpathlineto{\pgfqpoint{1.878829in}{0.956173in}}%
\pgfpathlineto{\pgfqpoint{1.879357in}{0.955463in}}%
\pgfpathlineto{\pgfqpoint{1.887022in}{0.945103in}}%
\pgfpathlineto{\pgfqpoint{1.890638in}{0.940420in}}%
\pgfpathlineto{\pgfqpoint{1.895413in}{0.934033in}}%
\pgfpathlineto{\pgfqpoint{1.901919in}{0.926627in}}%
\pgfpathlineto{\pgfqpoint{1.904900in}{0.922964in}}%
\pgfpathlineto{\pgfqpoint{1.913200in}{0.915108in}}%
\pgfpathlineto{\pgfqpoint{1.916382in}{0.911894in}}%
\pgfpathlineto{\pgfqpoint{1.924480in}{0.904079in}}%
\pgfpathlineto{\pgfqpoint{1.927776in}{0.900825in}}%
\pgfpathlineto{\pgfqpoint{1.935761in}{0.893256in}}%
\pgfpathlineto{\pgfqpoint{1.939522in}{0.889755in}}%
\pgfpathlineto{\pgfqpoint{1.947042in}{0.883113in}}%
\pgfpathlineto{\pgfqpoint{1.951888in}{0.878686in}}%
\pgfpathlineto{\pgfqpoint{1.958323in}{0.872992in}}%
\pgfpathlineto{\pgfqpoint{1.964339in}{0.867616in}}%
\pgfpathlineto{\pgfqpoint{1.969604in}{0.862889in}}%
\pgfpathlineto{\pgfqpoint{1.976736in}{0.856547in}}%
\pgfpathlineto{\pgfqpoint{1.980884in}{0.852904in}}%
\pgfpathlineto{\pgfqpoint{1.989223in}{0.845477in}}%
\pgfpathlineto{\pgfqpoint{1.992165in}{0.842799in}}%
\pgfpathlineto{\pgfqpoint{2.001510in}{0.834408in}}%
\pgfpathlineto{\pgfqpoint{2.003446in}{0.832602in}}%
\pgfpathlineto{\pgfqpoint{2.013404in}{0.823338in}}%
\pgfpathlineto{\pgfqpoint{2.014727in}{0.822118in}}%
\pgfpathlineto{\pgfqpoint{2.026008in}{0.812334in}}%
\pgfpathlineto{\pgfqpoint{2.026089in}{0.812269in}}%
\pgfpathlineto{\pgfqpoint{2.037289in}{0.803520in}}%
\pgfpathlineto{\pgfqpoint{2.040412in}{0.801199in}}%
\pgfpathlineto{\pgfqpoint{2.048569in}{0.795215in}}%
\pgfpathlineto{\pgfqpoint{2.055699in}{0.790129in}}%
\pgfpathlineto{\pgfqpoint{2.059850in}{0.787121in}}%
\pgfpathlineto{\pgfqpoint{2.071006in}{0.779060in}}%
\pgfpathlineto{\pgfqpoint{2.071131in}{0.778968in}}%
\pgfpathlineto{\pgfqpoint{2.082412in}{0.771139in}}%
\pgfpathlineto{\pgfqpoint{2.087167in}{0.767990in}}%
\pgfpathlineto{\pgfqpoint{2.093693in}{0.763550in}}%
\pgfpathlineto{\pgfqpoint{2.104330in}{0.756921in}}%
\pgfpathlineto{\pgfqpoint{2.104973in}{0.756488in}}%
\pgfpathlineto{\pgfqpoint{2.116254in}{0.749612in}}%
\pgfpathlineto{\pgfqpoint{2.122454in}{0.745851in}}%
\pgfpathlineto{\pgfqpoint{2.127535in}{0.742744in}}%
\pgfpathlineto{\pgfqpoint{2.138816in}{0.735820in}}%
\pgfpathlineto{\pgfqpoint{2.140623in}{0.734782in}}%
\pgfpathlineto{\pgfqpoint{2.150097in}{0.729145in}}%
\pgfpathlineto{\pgfqpoint{2.159481in}{0.723712in}}%
\pgfpathlineto{\pgfqpoint{2.161377in}{0.722596in}}%
\pgfpathlineto{\pgfqpoint{2.172658in}{0.715942in}}%
\pgfpathlineto{\pgfqpoint{2.178259in}{0.712643in}}%
\pgfpathlineto{\pgfqpoint{2.183939in}{0.709232in}}%
\pgfpathlineto{\pgfqpoint{2.195220in}{0.702485in}}%
\pgfpathlineto{\pgfqpoint{2.196742in}{0.701573in}}%
\pgfpathlineto{\pgfqpoint{2.206501in}{0.695763in}}%
\pgfpathlineto{\pgfqpoint{2.215612in}{0.690504in}}%
\pgfpathlineto{\pgfqpoint{2.217782in}{0.689244in}}%
\pgfpathlineto{\pgfqpoint{2.229062in}{0.682903in}}%
\pgfpathlineto{\pgfqpoint{2.235200in}{0.679434in}}%
\pgfpathlineto{\pgfqpoint{2.240343in}{0.676476in}}%
\pgfpathlineto{\pgfqpoint{2.251624in}{0.670004in}}%
\pgfpathlineto{\pgfqpoint{2.254473in}{0.668364in}}%
\pgfpathlineto{\pgfqpoint{2.262905in}{0.663430in}}%
\pgfpathlineto{\pgfqpoint{2.273335in}{0.657295in}}%
\pgfpathlineto{\pgfqpoint{2.274186in}{0.656784in}}%
\pgfpathlineto{\pgfqpoint{2.285466in}{0.649852in}}%
\pgfpathlineto{\pgfqpoint{2.291021in}{0.646225in}}%
\pgfpathlineto{\pgfqpoint{2.296747in}{0.642199in}}%
\pgfpathlineto{\pgfqpoint{2.306544in}{0.635156in}}%
\pgfpathlineto{\pgfqpoint{2.308028in}{0.634048in}}%
\pgfpathlineto{\pgfqpoint{2.319309in}{0.625607in}}%
\pgfpathlineto{\pgfqpoint{2.321251in}{0.624086in}}%
\pgfpathlineto{\pgfqpoint{2.330590in}{0.616374in}}%
\pgfpathlineto{\pgfqpoint{2.334575in}{0.613017in}}%
\pgfpathlineto{\pgfqpoint{2.341871in}{0.606705in}}%
\pgfpathlineto{\pgfqpoint{2.347159in}{0.601947in}}%
\pgfpathlineto{\pgfqpoint{2.353151in}{0.596113in}}%
\pgfpathlineto{\pgfqpoint{2.358258in}{0.590878in}}%
\pgfpathlineto{\pgfqpoint{2.364432in}{0.584381in}}%
\pgfpathlineto{\pgfqpoint{2.368636in}{0.579808in}}%
\pgfpathlineto{\pgfqpoint{2.375713in}{0.571991in}}%
\pgfpathlineto{\pgfqpoint{2.378574in}{0.568739in}}%
\pgfpathlineto{\pgfqpoint{2.386994in}{0.558947in}}%
\pgfpathlineto{\pgfqpoint{2.388075in}{0.557669in}}%
\pgfpathlineto{\pgfqpoint{2.397220in}{0.546600in}}%
\pgfpathlineto{\pgfqpoint{2.398275in}{0.545296in}}%
\pgfpathlineto{\pgfqpoint{2.406141in}{0.535530in}}%
\pgfpathlineto{\pgfqpoint{2.409555in}{0.531158in}}%
\pgfpathlineto{\pgfqpoint{2.414710in}{0.524460in}}%
\pgfpathlineto{\pgfqpoint{2.420836in}{0.515685in}}%
\pgfpathlineto{\pgfqpoint{2.422403in}{0.513391in}}%
\pgfpathlineto{\pgfqpoint{2.429838in}{0.502321in}}%
\pgfpathclose%
\pgfusepath{fill}%
\end{pgfscope}%
\begin{pgfscope}%
\pgfpathrectangle{\pgfqpoint{1.856795in}{0.423750in}}{\pgfqpoint{1.194205in}{1.163386in}}%
\pgfusepath{clip}%
\pgfsetbuttcap%
\pgfsetroundjoin%
\definecolor{currentfill}{rgb}{0.952404,0.449449,0.307210}%
\pgfsetfillcolor{currentfill}%
\pgfsetlinewidth{0.000000pt}%
\definecolor{currentstroke}{rgb}{0.000000,0.000000,0.000000}%
\pgfsetstrokecolor{currentstroke}%
\pgfsetdash{}{0pt}%
\pgfpathmoveto{\pgfqpoint{2.657733in}{1.086632in}}%
\pgfpathlineto{\pgfqpoint{2.669014in}{1.083220in}}%
\pgfpathlineto{\pgfqpoint{2.680295in}{1.080372in}}%
\pgfpathlineto{\pgfqpoint{2.691576in}{1.078549in}}%
\pgfpathlineto{\pgfqpoint{2.702857in}{1.079151in}}%
\pgfpathlineto{\pgfqpoint{2.714137in}{1.083778in}}%
\pgfpathlineto{\pgfqpoint{2.722538in}{1.089007in}}%
\pgfpathlineto{\pgfqpoint{2.725418in}{1.090710in}}%
\pgfpathlineto{\pgfqpoint{2.736699in}{1.097844in}}%
\pgfpathlineto{\pgfqpoint{2.739895in}{1.100077in}}%
\pgfpathlineto{\pgfqpoint{2.747980in}{1.105812in}}%
\pgfpathlineto{\pgfqpoint{2.754751in}{1.111146in}}%
\pgfpathlineto{\pgfqpoint{2.759261in}{1.114698in}}%
\pgfpathlineto{\pgfqpoint{2.768358in}{1.122216in}}%
\pgfpathlineto{\pgfqpoint{2.770542in}{1.124076in}}%
\pgfpathlineto{\pgfqpoint{2.780904in}{1.133285in}}%
\pgfpathlineto{\pgfqpoint{2.781822in}{1.134109in}}%
\pgfpathlineto{\pgfqpoint{2.792678in}{1.144355in}}%
\pgfpathlineto{\pgfqpoint{2.793103in}{1.144760in}}%
\pgfpathlineto{\pgfqpoint{2.804259in}{1.155424in}}%
\pgfpathlineto{\pgfqpoint{2.804384in}{1.155544in}}%
\pgfpathlineto{\pgfqpoint{2.815665in}{1.165887in}}%
\pgfpathlineto{\pgfqpoint{2.816318in}{1.166494in}}%
\pgfpathlineto{\pgfqpoint{2.826946in}{1.176169in}}%
\pgfpathlineto{\pgfqpoint{2.828432in}{1.177563in}}%
\pgfpathlineto{\pgfqpoint{2.838226in}{1.186834in}}%
\pgfpathlineto{\pgfqpoint{2.840087in}{1.188633in}}%
\pgfpathlineto{\pgfqpoint{2.849507in}{1.197882in}}%
\pgfpathlineto{\pgfqpoint{2.851315in}{1.199702in}}%
\pgfpathlineto{\pgfqpoint{2.860788in}{1.209382in}}%
\pgfpathlineto{\pgfqpoint{2.862115in}{1.210772in}}%
\pgfpathlineto{\pgfqpoint{2.872069in}{1.220453in}}%
\pgfpathlineto{\pgfqpoint{2.873354in}{1.221842in}}%
\pgfpathlineto{\pgfqpoint{2.883350in}{1.231646in}}%
\pgfpathlineto{\pgfqpoint{2.884573in}{1.232911in}}%
\pgfpathlineto{\pgfqpoint{2.894631in}{1.243124in}}%
\pgfpathlineto{\pgfqpoint{2.895471in}{1.243981in}}%
\pgfpathlineto{\pgfqpoint{2.905911in}{1.254831in}}%
\pgfpathlineto{\pgfqpoint{2.906120in}{1.255050in}}%
\pgfpathlineto{\pgfqpoint{2.916446in}{1.266120in}}%
\pgfpathlineto{\pgfqpoint{2.917192in}{1.266887in}}%
\pgfpathlineto{\pgfqpoint{2.926947in}{1.277189in}}%
\pgfpathlineto{\pgfqpoint{2.928473in}{1.278789in}}%
\pgfpathlineto{\pgfqpoint{2.937382in}{1.288259in}}%
\pgfpathlineto{\pgfqpoint{2.939754in}{1.290758in}}%
\pgfpathlineto{\pgfqpoint{2.948009in}{1.299328in}}%
\pgfpathlineto{\pgfqpoint{2.951035in}{1.302244in}}%
\pgfpathlineto{\pgfqpoint{2.959302in}{1.310398in}}%
\pgfpathlineto{\pgfqpoint{2.962315in}{1.313254in}}%
\pgfpathlineto{\pgfqpoint{2.970931in}{1.321467in}}%
\pgfpathlineto{\pgfqpoint{2.973596in}{1.324016in}}%
\pgfpathlineto{\pgfqpoint{2.973596in}{1.332537in}}%
\pgfpathlineto{\pgfqpoint{2.973596in}{1.343606in}}%
\pgfpathlineto{\pgfqpoint{2.973596in}{1.354676in}}%
\pgfpathlineto{\pgfqpoint{2.973596in}{1.365746in}}%
\pgfpathlineto{\pgfqpoint{2.973596in}{1.373600in}}%
\pgfpathlineto{\pgfqpoint{2.965468in}{1.365746in}}%
\pgfpathlineto{\pgfqpoint{2.962315in}{1.362683in}}%
\pgfpathlineto{\pgfqpoint{2.953997in}{1.354676in}}%
\pgfpathlineto{\pgfqpoint{2.951035in}{1.351777in}}%
\pgfpathlineto{\pgfqpoint{2.942479in}{1.343606in}}%
\pgfpathlineto{\pgfqpoint{2.939754in}{1.340942in}}%
\pgfpathlineto{\pgfqpoint{2.931190in}{1.332537in}}%
\pgfpathlineto{\pgfqpoint{2.928473in}{1.329888in}}%
\pgfpathlineto{\pgfqpoint{2.919709in}{1.321467in}}%
\pgfpathlineto{\pgfqpoint{2.917192in}{1.319022in}}%
\pgfpathlineto{\pgfqpoint{2.908291in}{1.310398in}}%
\pgfpathlineto{\pgfqpoint{2.905911in}{1.308040in}}%
\pgfpathlineto{\pgfqpoint{2.896979in}{1.299328in}}%
\pgfpathlineto{\pgfqpoint{2.894631in}{1.297084in}}%
\pgfpathlineto{\pgfqpoint{2.885220in}{1.288259in}}%
\pgfpathlineto{\pgfqpoint{2.883350in}{1.286515in}}%
\pgfpathlineto{\pgfqpoint{2.873016in}{1.277189in}}%
\pgfpathlineto{\pgfqpoint{2.872069in}{1.276297in}}%
\pgfpathlineto{\pgfqpoint{2.860788in}{1.266140in}}%
\pgfpathlineto{\pgfqpoint{2.860764in}{1.266120in}}%
\pgfpathlineto{\pgfqpoint{2.849507in}{1.255975in}}%
\pgfpathlineto{\pgfqpoint{2.848444in}{1.255050in}}%
\pgfpathlineto{\pgfqpoint{2.838226in}{1.245988in}}%
\pgfpathlineto{\pgfqpoint{2.835892in}{1.243981in}}%
\pgfpathlineto{\pgfqpoint{2.826946in}{1.236378in}}%
\pgfpathlineto{\pgfqpoint{2.822783in}{1.232911in}}%
\pgfpathlineto{\pgfqpoint{2.815665in}{1.226815in}}%
\pgfpathlineto{\pgfqpoint{2.809401in}{1.221842in}}%
\pgfpathlineto{\pgfqpoint{2.804384in}{1.217634in}}%
\pgfpathlineto{\pgfqpoint{2.796642in}{1.210772in}}%
\pgfpathlineto{\pgfqpoint{2.793103in}{1.207672in}}%
\pgfpathlineto{\pgfqpoint{2.783708in}{1.199702in}}%
\pgfpathlineto{\pgfqpoint{2.781822in}{1.198121in}}%
\pgfpathlineto{\pgfqpoint{2.770542in}{1.188961in}}%
\pgfpathlineto{\pgfqpoint{2.770129in}{1.188633in}}%
\pgfpathlineto{\pgfqpoint{2.759261in}{1.179917in}}%
\pgfpathlineto{\pgfqpoint{2.756235in}{1.177563in}}%
\pgfpathlineto{\pgfqpoint{2.747980in}{1.171197in}}%
\pgfpathlineto{\pgfqpoint{2.741613in}{1.166494in}}%
\pgfpathlineto{\pgfqpoint{2.736699in}{1.162920in}}%
\pgfpathlineto{\pgfqpoint{2.725794in}{1.155424in}}%
\pgfpathlineto{\pgfqpoint{2.725418in}{1.155176in}}%
\pgfpathlineto{\pgfqpoint{2.714137in}{1.149150in}}%
\pgfpathlineto{\pgfqpoint{2.702857in}{1.147055in}}%
\pgfpathlineto{\pgfqpoint{2.691576in}{1.146928in}}%
\pgfpathlineto{\pgfqpoint{2.680295in}{1.148169in}}%
\pgfpathlineto{\pgfqpoint{2.669014in}{1.150726in}}%
\pgfpathlineto{\pgfqpoint{2.657733in}{1.154234in}}%
\pgfpathlineto{\pgfqpoint{2.654287in}{1.155424in}}%
\pgfpathlineto{\pgfqpoint{2.646453in}{1.158045in}}%
\pgfpathlineto{\pgfqpoint{2.635172in}{1.162250in}}%
\pgfpathlineto{\pgfqpoint{2.625346in}{1.166494in}}%
\pgfpathlineto{\pgfqpoint{2.623891in}{1.167148in}}%
\pgfpathlineto{\pgfqpoint{2.612610in}{1.172692in}}%
\pgfpathlineto{\pgfqpoint{2.603242in}{1.177563in}}%
\pgfpathlineto{\pgfqpoint{2.601329in}{1.178549in}}%
\pgfpathlineto{\pgfqpoint{2.590048in}{1.184614in}}%
\pgfpathlineto{\pgfqpoint{2.582864in}{1.188633in}}%
\pgfpathlineto{\pgfqpoint{2.578768in}{1.190879in}}%
\pgfpathlineto{\pgfqpoint{2.567487in}{1.197646in}}%
\pgfpathlineto{\pgfqpoint{2.564075in}{1.199702in}}%
\pgfpathlineto{\pgfqpoint{2.556206in}{1.204466in}}%
\pgfpathlineto{\pgfqpoint{2.546714in}{1.210772in}}%
\pgfpathlineto{\pgfqpoint{2.544925in}{1.211959in}}%
\pgfpathlineto{\pgfqpoint{2.533644in}{1.219595in}}%
\pgfpathlineto{\pgfqpoint{2.530340in}{1.221842in}}%
\pgfpathlineto{\pgfqpoint{2.522364in}{1.227250in}}%
\pgfpathlineto{\pgfqpoint{2.514003in}{1.232911in}}%
\pgfpathlineto{\pgfqpoint{2.511083in}{1.234885in}}%
\pgfpathlineto{\pgfqpoint{2.499802in}{1.242747in}}%
\pgfpathlineto{\pgfqpoint{2.498244in}{1.243981in}}%
\pgfpathlineto{\pgfqpoint{2.488521in}{1.251766in}}%
\pgfpathlineto{\pgfqpoint{2.484577in}{1.255050in}}%
\pgfpathlineto{\pgfqpoint{2.477240in}{1.261172in}}%
\pgfpathlineto{\pgfqpoint{2.471385in}{1.266120in}}%
\pgfpathlineto{\pgfqpoint{2.465960in}{1.270734in}}%
\pgfpathlineto{\pgfqpoint{2.458347in}{1.277189in}}%
\pgfpathlineto{\pgfqpoint{2.454679in}{1.280341in}}%
\pgfpathlineto{\pgfqpoint{2.445437in}{1.288259in}}%
\pgfpathlineto{\pgfqpoint{2.443398in}{1.289995in}}%
\pgfpathlineto{\pgfqpoint{2.432955in}{1.299328in}}%
\pgfpathlineto{\pgfqpoint{2.432117in}{1.300069in}}%
\pgfpathlineto{\pgfqpoint{2.420836in}{1.310254in}}%
\pgfpathlineto{\pgfqpoint{2.420679in}{1.310398in}}%
\pgfpathlineto{\pgfqpoint{2.409555in}{1.320609in}}%
\pgfpathlineto{\pgfqpoint{2.408657in}{1.321467in}}%
\pgfpathlineto{\pgfqpoint{2.398275in}{1.331668in}}%
\pgfpathlineto{\pgfqpoint{2.397385in}{1.332537in}}%
\pgfpathlineto{\pgfqpoint{2.386994in}{1.342866in}}%
\pgfpathlineto{\pgfqpoint{2.386237in}{1.343606in}}%
\pgfpathlineto{\pgfqpoint{2.375713in}{1.354045in}}%
\pgfpathlineto{\pgfqpoint{2.375074in}{1.354676in}}%
\pgfpathlineto{\pgfqpoint{2.364432in}{1.365598in}}%
\pgfpathlineto{\pgfqpoint{2.364294in}{1.365746in}}%
\pgfpathlineto{\pgfqpoint{2.354255in}{1.376815in}}%
\pgfpathlineto{\pgfqpoint{2.353151in}{1.378035in}}%
\pgfpathlineto{\pgfqpoint{2.344220in}{1.387885in}}%
\pgfpathlineto{\pgfqpoint{2.341871in}{1.390543in}}%
\pgfpathlineto{\pgfqpoint{2.334411in}{1.398954in}}%
\pgfpathlineto{\pgfqpoint{2.330590in}{1.403370in}}%
\pgfpathlineto{\pgfqpoint{2.324838in}{1.410024in}}%
\pgfpathlineto{\pgfqpoint{2.319309in}{1.416499in}}%
\pgfpathlineto{\pgfqpoint{2.315407in}{1.421093in}}%
\pgfpathlineto{\pgfqpoint{2.308028in}{1.429934in}}%
\pgfpathlineto{\pgfqpoint{2.306174in}{1.432163in}}%
\pgfpathlineto{\pgfqpoint{2.297397in}{1.443232in}}%
\pgfpathlineto{\pgfqpoint{2.296747in}{1.444063in}}%
\pgfpathlineto{\pgfqpoint{2.289114in}{1.454302in}}%
\pgfpathlineto{\pgfqpoint{2.285466in}{1.459270in}}%
\pgfpathlineto{\pgfqpoint{2.281020in}{1.465371in}}%
\pgfpathlineto{\pgfqpoint{2.274186in}{1.474562in}}%
\pgfpathlineto{\pgfqpoint{2.272804in}{1.476441in}}%
\pgfpathlineto{\pgfqpoint{2.264678in}{1.487510in}}%
\pgfpathlineto{\pgfqpoint{2.262905in}{1.489981in}}%
\pgfpathlineto{\pgfqpoint{2.256741in}{1.498580in}}%
\pgfpathlineto{\pgfqpoint{2.251624in}{1.505392in}}%
\pgfpathlineto{\pgfqpoint{2.248497in}{1.509650in}}%
\pgfpathlineto{\pgfqpoint{2.240343in}{1.520247in}}%
\pgfpathlineto{\pgfqpoint{2.239985in}{1.520719in}}%
\pgfpathlineto{\pgfqpoint{2.231390in}{1.531789in}}%
\pgfpathlineto{\pgfqpoint{2.229062in}{1.534820in}}%
\pgfpathlineto{\pgfqpoint{2.222899in}{1.542858in}}%
\pgfpathlineto{\pgfqpoint{2.217782in}{1.549617in}}%
\pgfpathlineto{\pgfqpoint{2.214512in}{1.553928in}}%
\pgfpathlineto{\pgfqpoint{2.206501in}{1.564657in}}%
\pgfpathlineto{\pgfqpoint{2.206246in}{1.564997in}}%
\pgfpathlineto{\pgfqpoint{2.197962in}{1.576067in}}%
\pgfpathlineto{\pgfqpoint{2.195220in}{1.579760in}}%
\pgfpathlineto{\pgfqpoint{2.189745in}{1.587136in}}%
\pgfpathlineto{\pgfqpoint{2.183939in}{1.587136in}}%
\pgfpathlineto{\pgfqpoint{2.172658in}{1.587136in}}%
\pgfpathlineto{\pgfqpoint{2.161377in}{1.587136in}}%
\pgfpathlineto{\pgfqpoint{2.150097in}{1.587136in}}%
\pgfpathlineto{\pgfqpoint{2.138816in}{1.587136in}}%
\pgfpathlineto{\pgfqpoint{2.136373in}{1.587136in}}%
\pgfpathlineto{\pgfqpoint{2.138816in}{1.583666in}}%
\pgfpathlineto{\pgfqpoint{2.144144in}{1.576067in}}%
\pgfpathlineto{\pgfqpoint{2.150097in}{1.567737in}}%
\pgfpathlineto{\pgfqpoint{2.152032in}{1.564997in}}%
\pgfpathlineto{\pgfqpoint{2.159980in}{1.553928in}}%
\pgfpathlineto{\pgfqpoint{2.161377in}{1.552017in}}%
\pgfpathlineto{\pgfqpoint{2.168063in}{1.542858in}}%
\pgfpathlineto{\pgfqpoint{2.172658in}{1.536618in}}%
\pgfpathlineto{\pgfqpoint{2.176227in}{1.531789in}}%
\pgfpathlineto{\pgfqpoint{2.183939in}{1.521423in}}%
\pgfpathlineto{\pgfqpoint{2.184466in}{1.520719in}}%
\pgfpathlineto{\pgfqpoint{2.192135in}{1.509650in}}%
\pgfpathlineto{\pgfqpoint{2.195220in}{1.505268in}}%
\pgfpathlineto{\pgfqpoint{2.199926in}{1.498580in}}%
\pgfpathlineto{\pgfqpoint{2.206501in}{1.489355in}}%
\pgfpathlineto{\pgfqpoint{2.207820in}{1.487510in}}%
\pgfpathlineto{\pgfqpoint{2.215589in}{1.476441in}}%
\pgfpathlineto{\pgfqpoint{2.217782in}{1.473399in}}%
\pgfpathlineto{\pgfqpoint{2.223535in}{1.465371in}}%
\pgfpathlineto{\pgfqpoint{2.229062in}{1.457743in}}%
\pgfpathlineto{\pgfqpoint{2.231572in}{1.454302in}}%
\pgfpathlineto{\pgfqpoint{2.239540in}{1.443232in}}%
\pgfpathlineto{\pgfqpoint{2.240343in}{1.442104in}}%
\pgfpathlineto{\pgfqpoint{2.247560in}{1.432163in}}%
\pgfpathlineto{\pgfqpoint{2.251624in}{1.426556in}}%
\pgfpathlineto{\pgfqpoint{2.255586in}{1.421093in}}%
\pgfpathlineto{\pgfqpoint{2.262905in}{1.411230in}}%
\pgfpathlineto{\pgfqpoint{2.263799in}{1.410024in}}%
\pgfpathlineto{\pgfqpoint{2.272230in}{1.398954in}}%
\pgfpathlineto{\pgfqpoint{2.274186in}{1.396421in}}%
\pgfpathlineto{\pgfqpoint{2.281171in}{1.387885in}}%
\pgfpathlineto{\pgfqpoint{2.285466in}{1.382834in}}%
\pgfpathlineto{\pgfqpoint{2.290742in}{1.376815in}}%
\pgfpathlineto{\pgfqpoint{2.296747in}{1.370097in}}%
\pgfpathlineto{\pgfqpoint{2.300641in}{1.365746in}}%
\pgfpathlineto{\pgfqpoint{2.308028in}{1.357628in}}%
\pgfpathlineto{\pgfqpoint{2.310713in}{1.354676in}}%
\pgfpathlineto{\pgfqpoint{2.319309in}{1.345392in}}%
\pgfpathlineto{\pgfqpoint{2.320956in}{1.343606in}}%
\pgfpathlineto{\pgfqpoint{2.330590in}{1.333297in}}%
\pgfpathlineto{\pgfqpoint{2.331303in}{1.332537in}}%
\pgfpathlineto{\pgfqpoint{2.341871in}{1.321522in}}%
\pgfpathlineto{\pgfqpoint{2.341926in}{1.321467in}}%
\pgfpathlineto{\pgfqpoint{2.353151in}{1.310432in}}%
\pgfpathlineto{\pgfqpoint{2.353187in}{1.310398in}}%
\pgfpathlineto{\pgfqpoint{2.364432in}{1.299496in}}%
\pgfpathlineto{\pgfqpoint{2.364606in}{1.299328in}}%
\pgfpathlineto{\pgfqpoint{2.375713in}{1.288681in}}%
\pgfpathlineto{\pgfqpoint{2.376165in}{1.288259in}}%
\pgfpathlineto{\pgfqpoint{2.386994in}{1.278507in}}%
\pgfpathlineto{\pgfqpoint{2.388445in}{1.277189in}}%
\pgfpathlineto{\pgfqpoint{2.398275in}{1.268331in}}%
\pgfpathlineto{\pgfqpoint{2.400734in}{1.266120in}}%
\pgfpathlineto{\pgfqpoint{2.409555in}{1.258160in}}%
\pgfpathlineto{\pgfqpoint{2.413058in}{1.255050in}}%
\pgfpathlineto{\pgfqpoint{2.420836in}{1.248191in}}%
\pgfpathlineto{\pgfqpoint{2.425674in}{1.243981in}}%
\pgfpathlineto{\pgfqpoint{2.432117in}{1.238403in}}%
\pgfpathlineto{\pgfqpoint{2.438547in}{1.232911in}}%
\pgfpathlineto{\pgfqpoint{2.443398in}{1.228729in}}%
\pgfpathlineto{\pgfqpoint{2.451572in}{1.221842in}}%
\pgfpathlineto{\pgfqpoint{2.454679in}{1.219209in}}%
\pgfpathlineto{\pgfqpoint{2.464782in}{1.210772in}}%
\pgfpathlineto{\pgfqpoint{2.465960in}{1.209799in}}%
\pgfpathlineto{\pgfqpoint{2.477240in}{1.200503in}}%
\pgfpathlineto{\pgfqpoint{2.478211in}{1.199702in}}%
\pgfpathlineto{\pgfqpoint{2.488521in}{1.191483in}}%
\pgfpathlineto{\pgfqpoint{2.492681in}{1.188633in}}%
\pgfpathlineto{\pgfqpoint{2.499802in}{1.183846in}}%
\pgfpathlineto{\pgfqpoint{2.509407in}{1.177563in}}%
\pgfpathlineto{\pgfqpoint{2.511083in}{1.176351in}}%
\pgfpathlineto{\pgfqpoint{2.522364in}{1.168681in}}%
\pgfpathlineto{\pgfqpoint{2.525782in}{1.166494in}}%
\pgfpathlineto{\pgfqpoint{2.533644in}{1.160609in}}%
\pgfpathlineto{\pgfqpoint{2.540895in}{1.155424in}}%
\pgfpathlineto{\pgfqpoint{2.544925in}{1.152479in}}%
\pgfpathlineto{\pgfqpoint{2.556206in}{1.144500in}}%
\pgfpathlineto{\pgfqpoint{2.556419in}{1.144355in}}%
\pgfpathlineto{\pgfqpoint{2.567487in}{1.136792in}}%
\pgfpathlineto{\pgfqpoint{2.572806in}{1.133285in}}%
\pgfpathlineto{\pgfqpoint{2.578768in}{1.129392in}}%
\pgfpathlineto{\pgfqpoint{2.589747in}{1.122216in}}%
\pgfpathlineto{\pgfqpoint{2.590048in}{1.122018in}}%
\pgfpathlineto{\pgfqpoint{2.601329in}{1.114806in}}%
\pgfpathlineto{\pgfqpoint{2.607432in}{1.111146in}}%
\pgfpathlineto{\pgfqpoint{2.612610in}{1.108031in}}%
\pgfpathlineto{\pgfqpoint{2.623891in}{1.101631in}}%
\pgfpathlineto{\pgfqpoint{2.626780in}{1.100077in}}%
\pgfpathlineto{\pgfqpoint{2.635172in}{1.095615in}}%
\pgfpathlineto{\pgfqpoint{2.646453in}{1.090567in}}%
\pgfpathlineto{\pgfqpoint{2.650952in}{1.089007in}}%
\pgfpathclose%
\pgfusepath{fill}%
\end{pgfscope}%
\begin{pgfscope}%
\pgfpathrectangle{\pgfqpoint{1.856795in}{0.423750in}}{\pgfqpoint{1.194205in}{1.163386in}}%
\pgfusepath{clip}%
\pgfsetbuttcap%
\pgfsetroundjoin%
\definecolor{currentfill}{rgb}{0.959229,0.533075,0.374889}%
\pgfsetfillcolor{currentfill}%
\pgfsetlinewidth{0.000000pt}%
\definecolor{currentstroke}{rgb}{0.000000,0.000000,0.000000}%
\pgfsetstrokecolor{currentstroke}%
\pgfsetdash{}{0pt}%
\pgfpathmoveto{\pgfqpoint{2.477240in}{0.497273in}}%
\pgfpathlineto{\pgfqpoint{2.481272in}{0.491252in}}%
\pgfpathlineto{\pgfqpoint{2.488521in}{0.491252in}}%
\pgfpathlineto{\pgfqpoint{2.499802in}{0.491252in}}%
\pgfpathlineto{\pgfqpoint{2.511083in}{0.491252in}}%
\pgfpathlineto{\pgfqpoint{2.522364in}{0.491252in}}%
\pgfpathlineto{\pgfqpoint{2.531741in}{0.491252in}}%
\pgfpathlineto{\pgfqpoint{2.523963in}{0.502321in}}%
\pgfpathlineto{\pgfqpoint{2.522364in}{0.504512in}}%
\pgfpathlineto{\pgfqpoint{2.515971in}{0.513391in}}%
\pgfpathlineto{\pgfqpoint{2.511083in}{0.520042in}}%
\pgfpathlineto{\pgfqpoint{2.507819in}{0.524460in}}%
\pgfpathlineto{\pgfqpoint{2.499802in}{0.534937in}}%
\pgfpathlineto{\pgfqpoint{2.499345in}{0.535530in}}%
\pgfpathlineto{\pgfqpoint{2.490270in}{0.546600in}}%
\pgfpathlineto{\pgfqpoint{2.488521in}{0.548677in}}%
\pgfpathlineto{\pgfqpoint{2.481197in}{0.557669in}}%
\pgfpathlineto{\pgfqpoint{2.477240in}{0.562093in}}%
\pgfpathlineto{\pgfqpoint{2.471343in}{0.568739in}}%
\pgfpathlineto{\pgfqpoint{2.465960in}{0.574469in}}%
\pgfpathlineto{\pgfqpoint{2.460889in}{0.579808in}}%
\pgfpathlineto{\pgfqpoint{2.454679in}{0.586236in}}%
\pgfpathlineto{\pgfqpoint{2.450148in}{0.590878in}}%
\pgfpathlineto{\pgfqpoint{2.443398in}{0.597544in}}%
\pgfpathlineto{\pgfqpoint{2.438856in}{0.601947in}}%
\pgfpathlineto{\pgfqpoint{2.432117in}{0.608264in}}%
\pgfpathlineto{\pgfqpoint{2.427385in}{0.613017in}}%
\pgfpathlineto{\pgfqpoint{2.420836in}{0.619391in}}%
\pgfpathlineto{\pgfqpoint{2.416108in}{0.624086in}}%
\pgfpathlineto{\pgfqpoint{2.409555in}{0.630287in}}%
\pgfpathlineto{\pgfqpoint{2.404868in}{0.635156in}}%
\pgfpathlineto{\pgfqpoint{2.398275in}{0.641634in}}%
\pgfpathlineto{\pgfqpoint{2.393547in}{0.646225in}}%
\pgfpathlineto{\pgfqpoint{2.386994in}{0.652317in}}%
\pgfpathlineto{\pgfqpoint{2.381543in}{0.657295in}}%
\pgfpathlineto{\pgfqpoint{2.375713in}{0.662495in}}%
\pgfpathlineto{\pgfqpoint{2.368360in}{0.668364in}}%
\pgfpathlineto{\pgfqpoint{2.364432in}{0.671403in}}%
\pgfpathlineto{\pgfqpoint{2.353172in}{0.679434in}}%
\pgfpathlineto{\pgfqpoint{2.353151in}{0.679448in}}%
\pgfpathlineto{\pgfqpoint{2.341871in}{0.686522in}}%
\pgfpathlineto{\pgfqpoint{2.335437in}{0.690504in}}%
\pgfpathlineto{\pgfqpoint{2.330590in}{0.693427in}}%
\pgfpathlineto{\pgfqpoint{2.319309in}{0.700131in}}%
\pgfpathlineto{\pgfqpoint{2.316793in}{0.701573in}}%
\pgfpathlineto{\pgfqpoint{2.308028in}{0.706406in}}%
\pgfpathlineto{\pgfqpoint{2.296747in}{0.712502in}}%
\pgfpathlineto{\pgfqpoint{2.296488in}{0.712643in}}%
\pgfpathlineto{\pgfqpoint{2.285466in}{0.718466in}}%
\pgfpathlineto{\pgfqpoint{2.275721in}{0.723712in}}%
\pgfpathlineto{\pgfqpoint{2.274186in}{0.724505in}}%
\pgfpathlineto{\pgfqpoint{2.262905in}{0.730313in}}%
\pgfpathlineto{\pgfqpoint{2.254063in}{0.734782in}}%
\pgfpathlineto{\pgfqpoint{2.251624in}{0.735960in}}%
\pgfpathlineto{\pgfqpoint{2.240343in}{0.741447in}}%
\pgfpathlineto{\pgfqpoint{2.231452in}{0.745851in}}%
\pgfpathlineto{\pgfqpoint{2.229062in}{0.747028in}}%
\pgfpathlineto{\pgfqpoint{2.217782in}{0.752501in}}%
\pgfpathlineto{\pgfqpoint{2.208723in}{0.756921in}}%
\pgfpathlineto{\pgfqpoint{2.206501in}{0.758073in}}%
\pgfpathlineto{\pgfqpoint{2.195220in}{0.763658in}}%
\pgfpathlineto{\pgfqpoint{2.186079in}{0.767990in}}%
\pgfpathlineto{\pgfqpoint{2.183939in}{0.769110in}}%
\pgfpathlineto{\pgfqpoint{2.172658in}{0.775661in}}%
\pgfpathlineto{\pgfqpoint{2.166719in}{0.779060in}}%
\pgfpathlineto{\pgfqpoint{2.161377in}{0.782245in}}%
\pgfpathlineto{\pgfqpoint{2.150097in}{0.788903in}}%
\pgfpathlineto{\pgfqpoint{2.147994in}{0.790129in}}%
\pgfpathlineto{\pgfqpoint{2.138816in}{0.795372in}}%
\pgfpathlineto{\pgfqpoint{2.128541in}{0.801199in}}%
\pgfpathlineto{\pgfqpoint{2.127535in}{0.801743in}}%
\pgfpathlineto{\pgfqpoint{2.116254in}{0.807916in}}%
\pgfpathlineto{\pgfqpoint{2.108575in}{0.812269in}}%
\pgfpathlineto{\pgfqpoint{2.104973in}{0.814230in}}%
\pgfpathlineto{\pgfqpoint{2.093693in}{0.820385in}}%
\pgfpathlineto{\pgfqpoint{2.088431in}{0.823338in}}%
\pgfpathlineto{\pgfqpoint{2.082412in}{0.826650in}}%
\pgfpathlineto{\pgfqpoint{2.071131in}{0.832857in}}%
\pgfpathlineto{\pgfqpoint{2.068266in}{0.834408in}}%
\pgfpathlineto{\pgfqpoint{2.059850in}{0.838946in}}%
\pgfpathlineto{\pgfqpoint{2.048569in}{0.845278in}}%
\pgfpathlineto{\pgfqpoint{2.048279in}{0.845477in}}%
\pgfpathlineto{\pgfqpoint{2.037289in}{0.852931in}}%
\pgfpathlineto{\pgfqpoint{2.032246in}{0.856547in}}%
\pgfpathlineto{\pgfqpoint{2.026008in}{0.861136in}}%
\pgfpathlineto{\pgfqpoint{2.017314in}{0.867616in}}%
\pgfpathlineto{\pgfqpoint{2.014727in}{0.869577in}}%
\pgfpathlineto{\pgfqpoint{2.003818in}{0.878686in}}%
\pgfpathlineto{\pgfqpoint{2.003446in}{0.879003in}}%
\pgfpathlineto{\pgfqpoint{1.992197in}{0.889755in}}%
\pgfpathlineto{\pgfqpoint{1.992165in}{0.889787in}}%
\pgfpathlineto{\pgfqpoint{1.981042in}{0.900825in}}%
\pgfpathlineto{\pgfqpoint{1.980884in}{0.901008in}}%
\pgfpathlineto{\pgfqpoint{1.970936in}{0.911894in}}%
\pgfpathlineto{\pgfqpoint{1.969604in}{0.913616in}}%
\pgfpathlineto{\pgfqpoint{1.962433in}{0.922964in}}%
\pgfpathlineto{\pgfqpoint{1.958323in}{0.929063in}}%
\pgfpathlineto{\pgfqpoint{1.954950in}{0.934033in}}%
\pgfpathlineto{\pgfqpoint{1.947421in}{0.945103in}}%
\pgfpathlineto{\pgfqpoint{1.947042in}{0.945670in}}%
\pgfpathlineto{\pgfqpoint{1.939828in}{0.956173in}}%
\pgfpathlineto{\pgfqpoint{1.935761in}{0.963127in}}%
\pgfpathlineto{\pgfqpoint{1.933158in}{0.967242in}}%
\pgfpathlineto{\pgfqpoint{1.927912in}{0.978312in}}%
\pgfpathlineto{\pgfqpoint{1.924480in}{0.987230in}}%
\pgfpathlineto{\pgfqpoint{1.923639in}{0.989381in}}%
\pgfpathlineto{\pgfqpoint{1.919597in}{1.000451in}}%
\pgfpathlineto{\pgfqpoint{1.916900in}{1.011520in}}%
\pgfpathlineto{\pgfqpoint{1.914787in}{1.022590in}}%
\pgfpathlineto{\pgfqpoint{1.914025in}{1.033659in}}%
\pgfpathlineto{\pgfqpoint{1.915381in}{1.044729in}}%
\pgfpathlineto{\pgfqpoint{1.916361in}{1.055798in}}%
\pgfpathlineto{\pgfqpoint{1.916841in}{1.066868in}}%
\pgfpathlineto{\pgfqpoint{1.916780in}{1.077937in}}%
\pgfpathlineto{\pgfqpoint{1.913405in}{1.089007in}}%
\pgfpathlineto{\pgfqpoint{1.913200in}{1.089312in}}%
\pgfpathlineto{\pgfqpoint{1.905853in}{1.100077in}}%
\pgfpathlineto{\pgfqpoint{1.901919in}{1.105301in}}%
\pgfpathlineto{\pgfqpoint{1.897421in}{1.111146in}}%
\pgfpathlineto{\pgfqpoint{1.890638in}{1.119772in}}%
\pgfpathlineto{\pgfqpoint{1.888679in}{1.122216in}}%
\pgfpathlineto{\pgfqpoint{1.879747in}{1.133285in}}%
\pgfpathlineto{\pgfqpoint{1.879357in}{1.133761in}}%
\pgfpathlineto{\pgfqpoint{1.870534in}{1.144355in}}%
\pgfpathlineto{\pgfqpoint{1.868076in}{1.147284in}}%
\pgfpathlineto{\pgfqpoint{1.861120in}{1.155424in}}%
\pgfpathlineto{\pgfqpoint{1.856795in}{1.160478in}}%
\pgfpathlineto{\pgfqpoint{1.856795in}{1.155424in}}%
\pgfpathlineto{\pgfqpoint{1.856795in}{1.144355in}}%
\pgfpathlineto{\pgfqpoint{1.856795in}{1.133285in}}%
\pgfpathlineto{\pgfqpoint{1.856795in}{1.122216in}}%
\pgfpathlineto{\pgfqpoint{1.856795in}{1.114492in}}%
\pgfpathlineto{\pgfqpoint{1.859557in}{1.111146in}}%
\pgfpathlineto{\pgfqpoint{1.868076in}{1.100283in}}%
\pgfpathlineto{\pgfqpoint{1.868237in}{1.100077in}}%
\pgfpathlineto{\pgfqpoint{1.874112in}{1.089007in}}%
\pgfpathlineto{\pgfqpoint{1.874921in}{1.077937in}}%
\pgfpathlineto{\pgfqpoint{1.874868in}{1.066868in}}%
\pgfpathlineto{\pgfqpoint{1.874321in}{1.055798in}}%
\pgfpathlineto{\pgfqpoint{1.873959in}{1.044729in}}%
\pgfpathlineto{\pgfqpoint{1.875778in}{1.033659in}}%
\pgfpathlineto{\pgfqpoint{1.879016in}{1.022590in}}%
\pgfpathlineto{\pgfqpoint{1.879357in}{1.021466in}}%
\pgfpathlineto{\pgfqpoint{1.882256in}{1.011520in}}%
\pgfpathlineto{\pgfqpoint{1.885375in}{1.000451in}}%
\pgfpathlineto{\pgfqpoint{1.889029in}{0.989381in}}%
\pgfpathlineto{\pgfqpoint{1.890638in}{0.986405in}}%
\pgfpathlineto{\pgfqpoint{1.894890in}{0.978312in}}%
\pgfpathlineto{\pgfqpoint{1.901919in}{0.967757in}}%
\pgfpathlineto{\pgfqpoint{1.902234in}{0.967242in}}%
\pgfpathlineto{\pgfqpoint{1.910169in}{0.956173in}}%
\pgfpathlineto{\pgfqpoint{1.913200in}{0.951980in}}%
\pgfpathlineto{\pgfqpoint{1.918065in}{0.945103in}}%
\pgfpathlineto{\pgfqpoint{1.924480in}{0.936116in}}%
\pgfpathlineto{\pgfqpoint{1.925941in}{0.934033in}}%
\pgfpathlineto{\pgfqpoint{1.933997in}{0.922964in}}%
\pgfpathlineto{\pgfqpoint{1.935761in}{0.920955in}}%
\pgfpathlineto{\pgfqpoint{1.943076in}{0.911894in}}%
\pgfpathlineto{\pgfqpoint{1.947042in}{0.907954in}}%
\pgfpathlineto{\pgfqpoint{1.954216in}{0.900825in}}%
\pgfpathlineto{\pgfqpoint{1.958323in}{0.897016in}}%
\pgfpathlineto{\pgfqpoint{1.966015in}{0.889755in}}%
\pgfpathlineto{\pgfqpoint{1.969604in}{0.886438in}}%
\pgfpathlineto{\pgfqpoint{1.977906in}{0.878686in}}%
\pgfpathlineto{\pgfqpoint{1.980884in}{0.875956in}}%
\pgfpathlineto{\pgfqpoint{1.989908in}{0.867616in}}%
\pgfpathlineto{\pgfqpoint{1.992165in}{0.865621in}}%
\pgfpathlineto{\pgfqpoint{2.002251in}{0.856547in}}%
\pgfpathlineto{\pgfqpoint{2.003446in}{0.855479in}}%
\pgfpathlineto{\pgfqpoint{2.014727in}{0.845935in}}%
\pgfpathlineto{\pgfqpoint{2.015324in}{0.845477in}}%
\pgfpathlineto{\pgfqpoint{2.026008in}{0.836979in}}%
\pgfpathlineto{\pgfqpoint{2.029650in}{0.834408in}}%
\pgfpathlineto{\pgfqpoint{2.037289in}{0.828772in}}%
\pgfpathlineto{\pgfqpoint{2.044918in}{0.823338in}}%
\pgfpathlineto{\pgfqpoint{2.048569in}{0.820585in}}%
\pgfpathlineto{\pgfqpoint{2.059655in}{0.812269in}}%
\pgfpathlineto{\pgfqpoint{2.059850in}{0.812123in}}%
\pgfpathlineto{\pgfqpoint{2.071131in}{0.804649in}}%
\pgfpathlineto{\pgfqpoint{2.076758in}{0.801199in}}%
\pgfpathlineto{\pgfqpoint{2.082412in}{0.797799in}}%
\pgfpathlineto{\pgfqpoint{2.093693in}{0.791035in}}%
\pgfpathlineto{\pgfqpoint{2.095134in}{0.790129in}}%
\pgfpathlineto{\pgfqpoint{2.104973in}{0.784401in}}%
\pgfpathlineto{\pgfqpoint{2.114015in}{0.779060in}}%
\pgfpathlineto{\pgfqpoint{2.116254in}{0.777739in}}%
\pgfpathlineto{\pgfqpoint{2.127535in}{0.771220in}}%
\pgfpathlineto{\pgfqpoint{2.133342in}{0.767990in}}%
\pgfpathlineto{\pgfqpoint{2.138816in}{0.764865in}}%
\pgfpathlineto{\pgfqpoint{2.150097in}{0.758511in}}%
\pgfpathlineto{\pgfqpoint{2.152939in}{0.756921in}}%
\pgfpathlineto{\pgfqpoint{2.161377in}{0.752159in}}%
\pgfpathlineto{\pgfqpoint{2.172457in}{0.745851in}}%
\pgfpathlineto{\pgfqpoint{2.172658in}{0.745736in}}%
\pgfpathlineto{\pgfqpoint{2.183939in}{0.739287in}}%
\pgfpathlineto{\pgfqpoint{2.191814in}{0.734782in}}%
\pgfpathlineto{\pgfqpoint{2.195220in}{0.732826in}}%
\pgfpathlineto{\pgfqpoint{2.206501in}{0.726637in}}%
\pgfpathlineto{\pgfqpoint{2.212105in}{0.723712in}}%
\pgfpathlineto{\pgfqpoint{2.217782in}{0.720615in}}%
\pgfpathlineto{\pgfqpoint{2.229062in}{0.714724in}}%
\pgfpathlineto{\pgfqpoint{2.233073in}{0.712643in}}%
\pgfpathlineto{\pgfqpoint{2.240343in}{0.708823in}}%
\pgfpathlineto{\pgfqpoint{2.251624in}{0.702842in}}%
\pgfpathlineto{\pgfqpoint{2.253964in}{0.701573in}}%
\pgfpathlineto{\pgfqpoint{2.262905in}{0.696682in}}%
\pgfpathlineto{\pgfqpoint{2.274186in}{0.690564in}}%
\pgfpathlineto{\pgfqpoint{2.274297in}{0.690504in}}%
\pgfpathlineto{\pgfqpoint{2.285466in}{0.684354in}}%
\pgfpathlineto{\pgfqpoint{2.294151in}{0.679434in}}%
\pgfpathlineto{\pgfqpoint{2.296747in}{0.677981in}}%
\pgfpathlineto{\pgfqpoint{2.308028in}{0.671371in}}%
\pgfpathlineto{\pgfqpoint{2.312955in}{0.668364in}}%
\pgfpathlineto{\pgfqpoint{2.319309in}{0.664342in}}%
\pgfpathlineto{\pgfqpoint{2.330245in}{0.657295in}}%
\pgfpathlineto{\pgfqpoint{2.330590in}{0.657058in}}%
\pgfpathlineto{\pgfqpoint{2.341871in}{0.649020in}}%
\pgfpathlineto{\pgfqpoint{2.345562in}{0.646225in}}%
\pgfpathlineto{\pgfqpoint{2.353151in}{0.640248in}}%
\pgfpathlineto{\pgfqpoint{2.359447in}{0.635156in}}%
\pgfpathlineto{\pgfqpoint{2.364432in}{0.630970in}}%
\pgfpathlineto{\pgfqpoint{2.371709in}{0.624086in}}%
\pgfpathlineto{\pgfqpoint{2.375713in}{0.620028in}}%
\pgfpathlineto{\pgfqpoint{2.382555in}{0.613017in}}%
\pgfpathlineto{\pgfqpoint{2.386994in}{0.608308in}}%
\pgfpathlineto{\pgfqpoint{2.392868in}{0.601947in}}%
\pgfpathlineto{\pgfqpoint{2.398275in}{0.595951in}}%
\pgfpathlineto{\pgfqpoint{2.402774in}{0.590878in}}%
\pgfpathlineto{\pgfqpoint{2.409555in}{0.583065in}}%
\pgfpathlineto{\pgfqpoint{2.412347in}{0.579808in}}%
\pgfpathlineto{\pgfqpoint{2.420836in}{0.569663in}}%
\pgfpathlineto{\pgfqpoint{2.421666in}{0.568739in}}%
\pgfpathlineto{\pgfqpoint{2.430981in}{0.557669in}}%
\pgfpathlineto{\pgfqpoint{2.432117in}{0.556237in}}%
\pgfpathlineto{\pgfqpoint{2.440424in}{0.546600in}}%
\pgfpathlineto{\pgfqpoint{2.443398in}{0.542993in}}%
\pgfpathlineto{\pgfqpoint{2.449727in}{0.535530in}}%
\pgfpathlineto{\pgfqpoint{2.454679in}{0.529036in}}%
\pgfpathlineto{\pgfqpoint{2.458141in}{0.524460in}}%
\pgfpathlineto{\pgfqpoint{2.465960in}{0.513605in}}%
\pgfpathlineto{\pgfqpoint{2.466112in}{0.513391in}}%
\pgfpathlineto{\pgfqpoint{2.473843in}{0.502321in}}%
\pgfpathclose%
\pgfusepath{fill}%
\end{pgfscope}%
\begin{pgfscope}%
\pgfpathrectangle{\pgfqpoint{1.856795in}{0.423750in}}{\pgfqpoint{1.194205in}{1.163386in}}%
\pgfusepath{clip}%
\pgfsetbuttcap%
\pgfsetroundjoin%
\definecolor{currentfill}{rgb}{0.959229,0.533075,0.374889}%
\pgfsetfillcolor{currentfill}%
\pgfsetlinewidth{0.000000pt}%
\definecolor{currentstroke}{rgb}{0.000000,0.000000,0.000000}%
\pgfsetstrokecolor{currentstroke}%
\pgfsetdash{}{0pt}%
\pgfpathmoveto{\pgfqpoint{2.669014in}{1.019119in}}%
\pgfpathlineto{\pgfqpoint{2.680295in}{1.016060in}}%
\pgfpathlineto{\pgfqpoint{2.691576in}{1.014381in}}%
\pgfpathlineto{\pgfqpoint{2.702857in}{1.016611in}}%
\pgfpathlineto{\pgfqpoint{2.714137in}{1.020815in}}%
\pgfpathlineto{\pgfqpoint{2.717788in}{1.022590in}}%
\pgfpathlineto{\pgfqpoint{2.725418in}{1.026579in}}%
\pgfpathlineto{\pgfqpoint{2.736380in}{1.033659in}}%
\pgfpathlineto{\pgfqpoint{2.736699in}{1.033861in}}%
\pgfpathlineto{\pgfqpoint{2.747980in}{1.041702in}}%
\pgfpathlineto{\pgfqpoint{2.752035in}{1.044729in}}%
\pgfpathlineto{\pgfqpoint{2.759261in}{1.050046in}}%
\pgfpathlineto{\pgfqpoint{2.766004in}{1.055798in}}%
\pgfpathlineto{\pgfqpoint{2.770542in}{1.059777in}}%
\pgfpathlineto{\pgfqpoint{2.778210in}{1.066868in}}%
\pgfpathlineto{\pgfqpoint{2.781822in}{1.070244in}}%
\pgfpathlineto{\pgfqpoint{2.789765in}{1.077937in}}%
\pgfpathlineto{\pgfqpoint{2.793103in}{1.081166in}}%
\pgfpathlineto{\pgfqpoint{2.802027in}{1.089007in}}%
\pgfpathlineto{\pgfqpoint{2.804384in}{1.091051in}}%
\pgfpathlineto{\pgfqpoint{2.815271in}{1.100077in}}%
\pgfpathlineto{\pgfqpoint{2.815665in}{1.100409in}}%
\pgfpathlineto{\pgfqpoint{2.826946in}{1.109975in}}%
\pgfpathlineto{\pgfqpoint{2.828283in}{1.111146in}}%
\pgfpathlineto{\pgfqpoint{2.838226in}{1.119829in}}%
\pgfpathlineto{\pgfqpoint{2.840913in}{1.122216in}}%
\pgfpathlineto{\pgfqpoint{2.849507in}{1.130016in}}%
\pgfpathlineto{\pgfqpoint{2.853045in}{1.133285in}}%
\pgfpathlineto{\pgfqpoint{2.860788in}{1.140439in}}%
\pgfpathlineto{\pgfqpoint{2.865010in}{1.144355in}}%
\pgfpathlineto{\pgfqpoint{2.872069in}{1.151008in}}%
\pgfpathlineto{\pgfqpoint{2.876636in}{1.155424in}}%
\pgfpathlineto{\pgfqpoint{2.883350in}{1.162103in}}%
\pgfpathlineto{\pgfqpoint{2.887651in}{1.166494in}}%
\pgfpathlineto{\pgfqpoint{2.894631in}{1.173803in}}%
\pgfpathlineto{\pgfqpoint{2.898128in}{1.177563in}}%
\pgfpathlineto{\pgfqpoint{2.905911in}{1.186104in}}%
\pgfpathlineto{\pgfqpoint{2.908169in}{1.188633in}}%
\pgfpathlineto{\pgfqpoint{2.917192in}{1.198998in}}%
\pgfpathlineto{\pgfqpoint{2.917805in}{1.199702in}}%
\pgfpathlineto{\pgfqpoint{2.927140in}{1.210772in}}%
\pgfpathlineto{\pgfqpoint{2.928473in}{1.212332in}}%
\pgfpathlineto{\pgfqpoint{2.936616in}{1.221842in}}%
\pgfpathlineto{\pgfqpoint{2.939754in}{1.225665in}}%
\pgfpathlineto{\pgfqpoint{2.945897in}{1.232911in}}%
\pgfpathlineto{\pgfqpoint{2.951035in}{1.239061in}}%
\pgfpathlineto{\pgfqpoint{2.955089in}{1.243981in}}%
\pgfpathlineto{\pgfqpoint{2.962315in}{1.252860in}}%
\pgfpathlineto{\pgfqpoint{2.964066in}{1.255050in}}%
\pgfpathlineto{\pgfqpoint{2.972817in}{1.266120in}}%
\pgfpathlineto{\pgfqpoint{2.973596in}{1.267043in}}%
\pgfpathlineto{\pgfqpoint{2.973596in}{1.277189in}}%
\pgfpathlineto{\pgfqpoint{2.973596in}{1.288259in}}%
\pgfpathlineto{\pgfqpoint{2.973596in}{1.299328in}}%
\pgfpathlineto{\pgfqpoint{2.973596in}{1.310398in}}%
\pgfpathlineto{\pgfqpoint{2.973596in}{1.321467in}}%
\pgfpathlineto{\pgfqpoint{2.973596in}{1.324016in}}%
\pgfpathlineto{\pgfqpoint{2.970931in}{1.321467in}}%
\pgfpathlineto{\pgfqpoint{2.962315in}{1.313254in}}%
\pgfpathlineto{\pgfqpoint{2.959302in}{1.310398in}}%
\pgfpathlineto{\pgfqpoint{2.951035in}{1.302244in}}%
\pgfpathlineto{\pgfqpoint{2.948009in}{1.299328in}}%
\pgfpathlineto{\pgfqpoint{2.939754in}{1.290758in}}%
\pgfpathlineto{\pgfqpoint{2.937382in}{1.288259in}}%
\pgfpathlineto{\pgfqpoint{2.928473in}{1.278789in}}%
\pgfpathlineto{\pgfqpoint{2.926947in}{1.277189in}}%
\pgfpathlineto{\pgfqpoint{2.917192in}{1.266887in}}%
\pgfpathlineto{\pgfqpoint{2.916446in}{1.266120in}}%
\pgfpathlineto{\pgfqpoint{2.906120in}{1.255050in}}%
\pgfpathlineto{\pgfqpoint{2.905911in}{1.254831in}}%
\pgfpathlineto{\pgfqpoint{2.895471in}{1.243981in}}%
\pgfpathlineto{\pgfqpoint{2.894631in}{1.243124in}}%
\pgfpathlineto{\pgfqpoint{2.884573in}{1.232911in}}%
\pgfpathlineto{\pgfqpoint{2.883350in}{1.231646in}}%
\pgfpathlineto{\pgfqpoint{2.873354in}{1.221842in}}%
\pgfpathlineto{\pgfqpoint{2.872069in}{1.220453in}}%
\pgfpathlineto{\pgfqpoint{2.862115in}{1.210772in}}%
\pgfpathlineto{\pgfqpoint{2.860788in}{1.209382in}}%
\pgfpathlineto{\pgfqpoint{2.851315in}{1.199702in}}%
\pgfpathlineto{\pgfqpoint{2.849507in}{1.197882in}}%
\pgfpathlineto{\pgfqpoint{2.840087in}{1.188633in}}%
\pgfpathlineto{\pgfqpoint{2.838226in}{1.186834in}}%
\pgfpathlineto{\pgfqpoint{2.828432in}{1.177563in}}%
\pgfpathlineto{\pgfqpoint{2.826946in}{1.176169in}}%
\pgfpathlineto{\pgfqpoint{2.816318in}{1.166494in}}%
\pgfpathlineto{\pgfqpoint{2.815665in}{1.165887in}}%
\pgfpathlineto{\pgfqpoint{2.804384in}{1.155544in}}%
\pgfpathlineto{\pgfqpoint{2.804259in}{1.155424in}}%
\pgfpathlineto{\pgfqpoint{2.793103in}{1.144760in}}%
\pgfpathlineto{\pgfqpoint{2.792678in}{1.144355in}}%
\pgfpathlineto{\pgfqpoint{2.781822in}{1.134109in}}%
\pgfpathlineto{\pgfqpoint{2.780904in}{1.133285in}}%
\pgfpathlineto{\pgfqpoint{2.770542in}{1.124076in}}%
\pgfpathlineto{\pgfqpoint{2.768358in}{1.122216in}}%
\pgfpathlineto{\pgfqpoint{2.759261in}{1.114698in}}%
\pgfpathlineto{\pgfqpoint{2.754751in}{1.111146in}}%
\pgfpathlineto{\pgfqpoint{2.747980in}{1.105812in}}%
\pgfpathlineto{\pgfqpoint{2.739895in}{1.100077in}}%
\pgfpathlineto{\pgfqpoint{2.736699in}{1.097844in}}%
\pgfpathlineto{\pgfqpoint{2.725418in}{1.090710in}}%
\pgfpathlineto{\pgfqpoint{2.722538in}{1.089007in}}%
\pgfpathlineto{\pgfqpoint{2.714137in}{1.083778in}}%
\pgfpathlineto{\pgfqpoint{2.702857in}{1.079151in}}%
\pgfpathlineto{\pgfqpoint{2.691576in}{1.078549in}}%
\pgfpathlineto{\pgfqpoint{2.680295in}{1.080372in}}%
\pgfpathlineto{\pgfqpoint{2.669014in}{1.083220in}}%
\pgfpathlineto{\pgfqpoint{2.657733in}{1.086632in}}%
\pgfpathlineto{\pgfqpoint{2.650952in}{1.089007in}}%
\pgfpathlineto{\pgfqpoint{2.646453in}{1.090567in}}%
\pgfpathlineto{\pgfqpoint{2.635172in}{1.095615in}}%
\pgfpathlineto{\pgfqpoint{2.626780in}{1.100077in}}%
\pgfpathlineto{\pgfqpoint{2.623891in}{1.101631in}}%
\pgfpathlineto{\pgfqpoint{2.612610in}{1.108031in}}%
\pgfpathlineto{\pgfqpoint{2.607432in}{1.111146in}}%
\pgfpathlineto{\pgfqpoint{2.601329in}{1.114806in}}%
\pgfpathlineto{\pgfqpoint{2.590048in}{1.122018in}}%
\pgfpathlineto{\pgfqpoint{2.589747in}{1.122216in}}%
\pgfpathlineto{\pgfqpoint{2.578768in}{1.129392in}}%
\pgfpathlineto{\pgfqpoint{2.572806in}{1.133285in}}%
\pgfpathlineto{\pgfqpoint{2.567487in}{1.136792in}}%
\pgfpathlineto{\pgfqpoint{2.556419in}{1.144355in}}%
\pgfpathlineto{\pgfqpoint{2.556206in}{1.144500in}}%
\pgfpathlineto{\pgfqpoint{2.544925in}{1.152479in}}%
\pgfpathlineto{\pgfqpoint{2.540895in}{1.155424in}}%
\pgfpathlineto{\pgfqpoint{2.533644in}{1.160609in}}%
\pgfpathlineto{\pgfqpoint{2.525782in}{1.166494in}}%
\pgfpathlineto{\pgfqpoint{2.522364in}{1.168681in}}%
\pgfpathlineto{\pgfqpoint{2.511083in}{1.176351in}}%
\pgfpathlineto{\pgfqpoint{2.509407in}{1.177563in}}%
\pgfpathlineto{\pgfqpoint{2.499802in}{1.183846in}}%
\pgfpathlineto{\pgfqpoint{2.492681in}{1.188633in}}%
\pgfpathlineto{\pgfqpoint{2.488521in}{1.191483in}}%
\pgfpathlineto{\pgfqpoint{2.478211in}{1.199702in}}%
\pgfpathlineto{\pgfqpoint{2.477240in}{1.200503in}}%
\pgfpathlineto{\pgfqpoint{2.465960in}{1.209799in}}%
\pgfpathlineto{\pgfqpoint{2.464782in}{1.210772in}}%
\pgfpathlineto{\pgfqpoint{2.454679in}{1.219209in}}%
\pgfpathlineto{\pgfqpoint{2.451572in}{1.221842in}}%
\pgfpathlineto{\pgfqpoint{2.443398in}{1.228729in}}%
\pgfpathlineto{\pgfqpoint{2.438547in}{1.232911in}}%
\pgfpathlineto{\pgfqpoint{2.432117in}{1.238403in}}%
\pgfpathlineto{\pgfqpoint{2.425674in}{1.243981in}}%
\pgfpathlineto{\pgfqpoint{2.420836in}{1.248191in}}%
\pgfpathlineto{\pgfqpoint{2.413058in}{1.255050in}}%
\pgfpathlineto{\pgfqpoint{2.409555in}{1.258160in}}%
\pgfpathlineto{\pgfqpoint{2.400734in}{1.266120in}}%
\pgfpathlineto{\pgfqpoint{2.398275in}{1.268331in}}%
\pgfpathlineto{\pgfqpoint{2.388445in}{1.277189in}}%
\pgfpathlineto{\pgfqpoint{2.386994in}{1.278507in}}%
\pgfpathlineto{\pgfqpoint{2.376165in}{1.288259in}}%
\pgfpathlineto{\pgfqpoint{2.375713in}{1.288681in}}%
\pgfpathlineto{\pgfqpoint{2.364606in}{1.299328in}}%
\pgfpathlineto{\pgfqpoint{2.364432in}{1.299496in}}%
\pgfpathlineto{\pgfqpoint{2.353187in}{1.310398in}}%
\pgfpathlineto{\pgfqpoint{2.353151in}{1.310432in}}%
\pgfpathlineto{\pgfqpoint{2.341926in}{1.321467in}}%
\pgfpathlineto{\pgfqpoint{2.341871in}{1.321522in}}%
\pgfpathlineto{\pgfqpoint{2.331303in}{1.332537in}}%
\pgfpathlineto{\pgfqpoint{2.330590in}{1.333297in}}%
\pgfpathlineto{\pgfqpoint{2.320956in}{1.343606in}}%
\pgfpathlineto{\pgfqpoint{2.319309in}{1.345392in}}%
\pgfpathlineto{\pgfqpoint{2.310713in}{1.354676in}}%
\pgfpathlineto{\pgfqpoint{2.308028in}{1.357628in}}%
\pgfpathlineto{\pgfqpoint{2.300641in}{1.365746in}}%
\pgfpathlineto{\pgfqpoint{2.296747in}{1.370097in}}%
\pgfpathlineto{\pgfqpoint{2.290742in}{1.376815in}}%
\pgfpathlineto{\pgfqpoint{2.285466in}{1.382834in}}%
\pgfpathlineto{\pgfqpoint{2.281171in}{1.387885in}}%
\pgfpathlineto{\pgfqpoint{2.274186in}{1.396421in}}%
\pgfpathlineto{\pgfqpoint{2.272230in}{1.398954in}}%
\pgfpathlineto{\pgfqpoint{2.263799in}{1.410024in}}%
\pgfpathlineto{\pgfqpoint{2.262905in}{1.411230in}}%
\pgfpathlineto{\pgfqpoint{2.255586in}{1.421093in}}%
\pgfpathlineto{\pgfqpoint{2.251624in}{1.426556in}}%
\pgfpathlineto{\pgfqpoint{2.247560in}{1.432163in}}%
\pgfpathlineto{\pgfqpoint{2.240343in}{1.442104in}}%
\pgfpathlineto{\pgfqpoint{2.239540in}{1.443232in}}%
\pgfpathlineto{\pgfqpoint{2.231572in}{1.454302in}}%
\pgfpathlineto{\pgfqpoint{2.229062in}{1.457743in}}%
\pgfpathlineto{\pgfqpoint{2.223535in}{1.465371in}}%
\pgfpathlineto{\pgfqpoint{2.217782in}{1.473399in}}%
\pgfpathlineto{\pgfqpoint{2.215589in}{1.476441in}}%
\pgfpathlineto{\pgfqpoint{2.207820in}{1.487510in}}%
\pgfpathlineto{\pgfqpoint{2.206501in}{1.489355in}}%
\pgfpathlineto{\pgfqpoint{2.199926in}{1.498580in}}%
\pgfpathlineto{\pgfqpoint{2.195220in}{1.505268in}}%
\pgfpathlineto{\pgfqpoint{2.192135in}{1.509650in}}%
\pgfpathlineto{\pgfqpoint{2.184466in}{1.520719in}}%
\pgfpathlineto{\pgfqpoint{2.183939in}{1.521423in}}%
\pgfpathlineto{\pgfqpoint{2.176227in}{1.531789in}}%
\pgfpathlineto{\pgfqpoint{2.172658in}{1.536618in}}%
\pgfpathlineto{\pgfqpoint{2.168063in}{1.542858in}}%
\pgfpathlineto{\pgfqpoint{2.161377in}{1.552017in}}%
\pgfpathlineto{\pgfqpoint{2.159980in}{1.553928in}}%
\pgfpathlineto{\pgfqpoint{2.152032in}{1.564997in}}%
\pgfpathlineto{\pgfqpoint{2.150097in}{1.567737in}}%
\pgfpathlineto{\pgfqpoint{2.144144in}{1.576067in}}%
\pgfpathlineto{\pgfqpoint{2.138816in}{1.583666in}}%
\pgfpathlineto{\pgfqpoint{2.136373in}{1.587136in}}%
\pgfpathlineto{\pgfqpoint{2.127535in}{1.587136in}}%
\pgfpathlineto{\pgfqpoint{2.116254in}{1.587136in}}%
\pgfpathlineto{\pgfqpoint{2.104973in}{1.587136in}}%
\pgfpathlineto{\pgfqpoint{2.093693in}{1.587136in}}%
\pgfpathlineto{\pgfqpoint{2.082412in}{1.587136in}}%
\pgfpathlineto{\pgfqpoint{2.080407in}{1.587136in}}%
\pgfpathlineto{\pgfqpoint{2.082412in}{1.584283in}}%
\pgfpathlineto{\pgfqpoint{2.088180in}{1.576067in}}%
\pgfpathlineto{\pgfqpoint{2.093693in}{1.568347in}}%
\pgfpathlineto{\pgfqpoint{2.096084in}{1.564997in}}%
\pgfpathlineto{\pgfqpoint{2.104173in}{1.553928in}}%
\pgfpathlineto{\pgfqpoint{2.104973in}{1.552857in}}%
\pgfpathlineto{\pgfqpoint{2.112455in}{1.542858in}}%
\pgfpathlineto{\pgfqpoint{2.116254in}{1.537577in}}%
\pgfpathlineto{\pgfqpoint{2.120497in}{1.531789in}}%
\pgfpathlineto{\pgfqpoint{2.127535in}{1.521974in}}%
\pgfpathlineto{\pgfqpoint{2.128435in}{1.520719in}}%
\pgfpathlineto{\pgfqpoint{2.136579in}{1.509650in}}%
\pgfpathlineto{\pgfqpoint{2.138816in}{1.506608in}}%
\pgfpathlineto{\pgfqpoint{2.144796in}{1.498580in}}%
\pgfpathlineto{\pgfqpoint{2.150097in}{1.491276in}}%
\pgfpathlineto{\pgfqpoint{2.152870in}{1.487510in}}%
\pgfpathlineto{\pgfqpoint{2.160754in}{1.476441in}}%
\pgfpathlineto{\pgfqpoint{2.161377in}{1.475550in}}%
\pgfpathlineto{\pgfqpoint{2.168358in}{1.465371in}}%
\pgfpathlineto{\pgfqpoint{2.172658in}{1.459224in}}%
\pgfpathlineto{\pgfqpoint{2.176076in}{1.454302in}}%
\pgfpathlineto{\pgfqpoint{2.183929in}{1.443232in}}%
\pgfpathlineto{\pgfqpoint{2.183939in}{1.443218in}}%
\pgfpathlineto{\pgfqpoint{2.191879in}{1.432163in}}%
\pgfpathlineto{\pgfqpoint{2.195220in}{1.427586in}}%
\pgfpathlineto{\pgfqpoint{2.199919in}{1.421093in}}%
\pgfpathlineto{\pgfqpoint{2.206501in}{1.412250in}}%
\pgfpathlineto{\pgfqpoint{2.208147in}{1.410024in}}%
\pgfpathlineto{\pgfqpoint{2.216358in}{1.398954in}}%
\pgfpathlineto{\pgfqpoint{2.217782in}{1.397062in}}%
\pgfpathlineto{\pgfqpoint{2.224623in}{1.387885in}}%
\pgfpathlineto{\pgfqpoint{2.229062in}{1.382071in}}%
\pgfpathlineto{\pgfqpoint{2.233040in}{1.376815in}}%
\pgfpathlineto{\pgfqpoint{2.240343in}{1.367320in}}%
\pgfpathlineto{\pgfqpoint{2.241546in}{1.365746in}}%
\pgfpathlineto{\pgfqpoint{2.250167in}{1.354676in}}%
\pgfpathlineto{\pgfqpoint{2.251624in}{1.352807in}}%
\pgfpathlineto{\pgfqpoint{2.258942in}{1.343606in}}%
\pgfpathlineto{\pgfqpoint{2.262905in}{1.338946in}}%
\pgfpathlineto{\pgfqpoint{2.268569in}{1.332537in}}%
\pgfpathlineto{\pgfqpoint{2.274186in}{1.326259in}}%
\pgfpathlineto{\pgfqpoint{2.278487in}{1.321467in}}%
\pgfpathlineto{\pgfqpoint{2.285466in}{1.313782in}}%
\pgfpathlineto{\pgfqpoint{2.288550in}{1.310398in}}%
\pgfpathlineto{\pgfqpoint{2.296747in}{1.301333in}}%
\pgfpathlineto{\pgfqpoint{2.298598in}{1.299328in}}%
\pgfpathlineto{\pgfqpoint{2.308028in}{1.288520in}}%
\pgfpathlineto{\pgfqpoint{2.308281in}{1.288259in}}%
\pgfpathlineto{\pgfqpoint{2.318581in}{1.277189in}}%
\pgfpathlineto{\pgfqpoint{2.319309in}{1.276411in}}%
\pgfpathlineto{\pgfqpoint{2.328942in}{1.266120in}}%
\pgfpathlineto{\pgfqpoint{2.330590in}{1.264395in}}%
\pgfpathlineto{\pgfqpoint{2.339632in}{1.255050in}}%
\pgfpathlineto{\pgfqpoint{2.341871in}{1.252822in}}%
\pgfpathlineto{\pgfqpoint{2.350967in}{1.243981in}}%
\pgfpathlineto{\pgfqpoint{2.353151in}{1.241880in}}%
\pgfpathlineto{\pgfqpoint{2.362468in}{1.232911in}}%
\pgfpathlineto{\pgfqpoint{2.364432in}{1.231144in}}%
\pgfpathlineto{\pgfqpoint{2.374589in}{1.221842in}}%
\pgfpathlineto{\pgfqpoint{2.375713in}{1.220846in}}%
\pgfpathlineto{\pgfqpoint{2.386994in}{1.210917in}}%
\pgfpathlineto{\pgfqpoint{2.387160in}{1.210772in}}%
\pgfpathlineto{\pgfqpoint{2.398275in}{1.201197in}}%
\pgfpathlineto{\pgfqpoint{2.400005in}{1.199702in}}%
\pgfpathlineto{\pgfqpoint{2.409555in}{1.191549in}}%
\pgfpathlineto{\pgfqpoint{2.412960in}{1.188633in}}%
\pgfpathlineto{\pgfqpoint{2.420836in}{1.181933in}}%
\pgfpathlineto{\pgfqpoint{2.425973in}{1.177563in}}%
\pgfpathlineto{\pgfqpoint{2.432117in}{1.172524in}}%
\pgfpathlineto{\pgfqpoint{2.439323in}{1.166494in}}%
\pgfpathlineto{\pgfqpoint{2.443398in}{1.163212in}}%
\pgfpathlineto{\pgfqpoint{2.453153in}{1.155424in}}%
\pgfpathlineto{\pgfqpoint{2.454679in}{1.154210in}}%
\pgfpathlineto{\pgfqpoint{2.465960in}{1.145213in}}%
\pgfpathlineto{\pgfqpoint{2.467043in}{1.144355in}}%
\pgfpathlineto{\pgfqpoint{2.477240in}{1.136368in}}%
\pgfpathlineto{\pgfqpoint{2.481284in}{1.133285in}}%
\pgfpathlineto{\pgfqpoint{2.488521in}{1.127945in}}%
\pgfpathlineto{\pgfqpoint{2.496347in}{1.122216in}}%
\pgfpathlineto{\pgfqpoint{2.499802in}{1.119735in}}%
\pgfpathlineto{\pgfqpoint{2.511083in}{1.111742in}}%
\pgfpathlineto{\pgfqpoint{2.511922in}{1.111146in}}%
\pgfpathlineto{\pgfqpoint{2.522364in}{1.103548in}}%
\pgfpathlineto{\pgfqpoint{2.527224in}{1.100077in}}%
\pgfpathlineto{\pgfqpoint{2.533644in}{1.095253in}}%
\pgfpathlineto{\pgfqpoint{2.542096in}{1.089007in}}%
\pgfpathlineto{\pgfqpoint{2.544925in}{1.086927in}}%
\pgfpathlineto{\pgfqpoint{2.556206in}{1.078600in}}%
\pgfpathlineto{\pgfqpoint{2.557116in}{1.077937in}}%
\pgfpathlineto{\pgfqpoint{2.567487in}{1.070724in}}%
\pgfpathlineto{\pgfqpoint{2.573203in}{1.066868in}}%
\pgfpathlineto{\pgfqpoint{2.578768in}{1.063186in}}%
\pgfpathlineto{\pgfqpoint{2.590048in}{1.055915in}}%
\pgfpathlineto{\pgfqpoint{2.590234in}{1.055798in}}%
\pgfpathlineto{\pgfqpoint{2.601329in}{1.048939in}}%
\pgfpathlineto{\pgfqpoint{2.608505in}{1.044729in}}%
\pgfpathlineto{\pgfqpoint{2.612610in}{1.042395in}}%
\pgfpathlineto{\pgfqpoint{2.623891in}{1.036915in}}%
\pgfpathlineto{\pgfqpoint{2.631464in}{1.033659in}}%
\pgfpathlineto{\pgfqpoint{2.635172in}{1.032014in}}%
\pgfpathlineto{\pgfqpoint{2.646453in}{1.027308in}}%
\pgfpathlineto{\pgfqpoint{2.657733in}{1.022949in}}%
\pgfpathlineto{\pgfqpoint{2.658770in}{1.022590in}}%
\pgfpathclose%
\pgfusepath{fill}%
\end{pgfscope}%
\begin{pgfscope}%
\pgfpathrectangle{\pgfqpoint{1.856795in}{0.423750in}}{\pgfqpoint{1.194205in}{1.163386in}}%
\pgfusepath{clip}%
\pgfsetbuttcap%
\pgfsetroundjoin%
\definecolor{currentfill}{rgb}{0.962765,0.606121,0.444717}%
\pgfsetfillcolor{currentfill}%
\pgfsetlinewidth{0.000000pt}%
\definecolor{currentstroke}{rgb}{0.000000,0.000000,0.000000}%
\pgfsetstrokecolor{currentstroke}%
\pgfsetdash{}{0pt}%
\pgfpathmoveto{\pgfqpoint{2.533644in}{0.491252in}}%
\pgfpathlineto{\pgfqpoint{2.544925in}{0.491252in}}%
\pgfpathlineto{\pgfqpoint{2.556206in}{0.491252in}}%
\pgfpathlineto{\pgfqpoint{2.567487in}{0.491252in}}%
\pgfpathlineto{\pgfqpoint{2.578768in}{0.491252in}}%
\pgfpathlineto{\pgfqpoint{2.589368in}{0.491252in}}%
\pgfpathlineto{\pgfqpoint{2.581245in}{0.502321in}}%
\pgfpathlineto{\pgfqpoint{2.578768in}{0.505472in}}%
\pgfpathlineto{\pgfqpoint{2.573050in}{0.513391in}}%
\pgfpathlineto{\pgfqpoint{2.567487in}{0.520491in}}%
\pgfpathlineto{\pgfqpoint{2.564479in}{0.524460in}}%
\pgfpathlineto{\pgfqpoint{2.556206in}{0.534999in}}%
\pgfpathlineto{\pgfqpoint{2.555773in}{0.535530in}}%
\pgfpathlineto{\pgfqpoint{2.546435in}{0.546600in}}%
\pgfpathlineto{\pgfqpoint{2.544925in}{0.548294in}}%
\pgfpathlineto{\pgfqpoint{2.536465in}{0.557669in}}%
\pgfpathlineto{\pgfqpoint{2.533644in}{0.560706in}}%
\pgfpathlineto{\pgfqpoint{2.526751in}{0.568739in}}%
\pgfpathlineto{\pgfqpoint{2.522364in}{0.573557in}}%
\pgfpathlineto{\pgfqpoint{2.516874in}{0.579808in}}%
\pgfpathlineto{\pgfqpoint{2.511083in}{0.586194in}}%
\pgfpathlineto{\pgfqpoint{2.506709in}{0.590878in}}%
\pgfpathlineto{\pgfqpoint{2.499802in}{0.597815in}}%
\pgfpathlineto{\pgfqpoint{2.495461in}{0.601947in}}%
\pgfpathlineto{\pgfqpoint{2.488521in}{0.608483in}}%
\pgfpathlineto{\pgfqpoint{2.483687in}{0.613017in}}%
\pgfpathlineto{\pgfqpoint{2.477240in}{0.618940in}}%
\pgfpathlineto{\pgfqpoint{2.471484in}{0.624086in}}%
\pgfpathlineto{\pgfqpoint{2.465960in}{0.628874in}}%
\pgfpathlineto{\pgfqpoint{2.458925in}{0.635156in}}%
\pgfpathlineto{\pgfqpoint{2.454679in}{0.638717in}}%
\pgfpathlineto{\pgfqpoint{2.446228in}{0.646225in}}%
\pgfpathlineto{\pgfqpoint{2.443398in}{0.648709in}}%
\pgfpathlineto{\pgfqpoint{2.433315in}{0.657295in}}%
\pgfpathlineto{\pgfqpoint{2.432117in}{0.658310in}}%
\pgfpathlineto{\pgfqpoint{2.420836in}{0.667521in}}%
\pgfpathlineto{\pgfqpoint{2.419796in}{0.668364in}}%
\pgfpathlineto{\pgfqpoint{2.409555in}{0.676247in}}%
\pgfpathlineto{\pgfqpoint{2.405626in}{0.679434in}}%
\pgfpathlineto{\pgfqpoint{2.398275in}{0.685171in}}%
\pgfpathlineto{\pgfqpoint{2.391536in}{0.690504in}}%
\pgfpathlineto{\pgfqpoint{2.386994in}{0.693812in}}%
\pgfpathlineto{\pgfqpoint{2.375895in}{0.701573in}}%
\pgfpathlineto{\pgfqpoint{2.375713in}{0.701692in}}%
\pgfpathlineto{\pgfqpoint{2.364432in}{0.708763in}}%
\pgfpathlineto{\pgfqpoint{2.357948in}{0.712643in}}%
\pgfpathlineto{\pgfqpoint{2.353151in}{0.715505in}}%
\pgfpathlineto{\pgfqpoint{2.341871in}{0.722178in}}%
\pgfpathlineto{\pgfqpoint{2.339163in}{0.723712in}}%
\pgfpathlineto{\pgfqpoint{2.330590in}{0.728470in}}%
\pgfpathlineto{\pgfqpoint{2.319309in}{0.734441in}}%
\pgfpathlineto{\pgfqpoint{2.318642in}{0.734782in}}%
\pgfpathlineto{\pgfqpoint{2.308028in}{0.740324in}}%
\pgfpathlineto{\pgfqpoint{2.296923in}{0.745851in}}%
\pgfpathlineto{\pgfqpoint{2.296747in}{0.745946in}}%
\pgfpathlineto{\pgfqpoint{2.285466in}{0.752189in}}%
\pgfpathlineto{\pgfqpoint{2.276483in}{0.756921in}}%
\pgfpathlineto{\pgfqpoint{2.274186in}{0.758139in}}%
\pgfpathlineto{\pgfqpoint{2.262905in}{0.764244in}}%
\pgfpathlineto{\pgfqpoint{2.256141in}{0.767990in}}%
\pgfpathlineto{\pgfqpoint{2.251624in}{0.770383in}}%
\pgfpathlineto{\pgfqpoint{2.240343in}{0.776316in}}%
\pgfpathlineto{\pgfqpoint{2.235249in}{0.779060in}}%
\pgfpathlineto{\pgfqpoint{2.229062in}{0.782227in}}%
\pgfpathlineto{\pgfqpoint{2.217782in}{0.787759in}}%
\pgfpathlineto{\pgfqpoint{2.213068in}{0.790129in}}%
\pgfpathlineto{\pgfqpoint{2.206501in}{0.793422in}}%
\pgfpathlineto{\pgfqpoint{2.195220in}{0.798867in}}%
\pgfpathlineto{\pgfqpoint{2.190294in}{0.801199in}}%
\pgfpathlineto{\pgfqpoint{2.183939in}{0.804223in}}%
\pgfpathlineto{\pgfqpoint{2.172658in}{0.809526in}}%
\pgfpathlineto{\pgfqpoint{2.166810in}{0.812269in}}%
\pgfpathlineto{\pgfqpoint{2.161377in}{0.814804in}}%
\pgfpathlineto{\pgfqpoint{2.150097in}{0.820245in}}%
\pgfpathlineto{\pgfqpoint{2.143946in}{0.823338in}}%
\pgfpathlineto{\pgfqpoint{2.138816in}{0.825850in}}%
\pgfpathlineto{\pgfqpoint{2.127535in}{0.831772in}}%
\pgfpathlineto{\pgfqpoint{2.122464in}{0.834408in}}%
\pgfpathlineto{\pgfqpoint{2.116254in}{0.837653in}}%
\pgfpathlineto{\pgfqpoint{2.104973in}{0.843559in}}%
\pgfpathlineto{\pgfqpoint{2.101384in}{0.845477in}}%
\pgfpathlineto{\pgfqpoint{2.093693in}{0.849606in}}%
\pgfpathlineto{\pgfqpoint{2.082412in}{0.855564in}}%
\pgfpathlineto{\pgfqpoint{2.080575in}{0.856547in}}%
\pgfpathlineto{\pgfqpoint{2.071131in}{0.861634in}}%
\pgfpathlineto{\pgfqpoint{2.060048in}{0.867616in}}%
\pgfpathlineto{\pgfqpoint{2.059850in}{0.867729in}}%
\pgfpathlineto{\pgfqpoint{2.048569in}{0.873967in}}%
\pgfpathlineto{\pgfqpoint{2.040298in}{0.878686in}}%
\pgfpathlineto{\pgfqpoint{2.037289in}{0.880443in}}%
\pgfpathlineto{\pgfqpoint{2.026008in}{0.887925in}}%
\pgfpathlineto{\pgfqpoint{2.023660in}{0.889755in}}%
\pgfpathlineto{\pgfqpoint{2.014727in}{0.897221in}}%
\pgfpathlineto{\pgfqpoint{2.010326in}{0.900825in}}%
\pgfpathlineto{\pgfqpoint{2.003446in}{0.907598in}}%
\pgfpathlineto{\pgfqpoint{1.999830in}{0.911894in}}%
\pgfpathlineto{\pgfqpoint{1.992165in}{0.921272in}}%
\pgfpathlineto{\pgfqpoint{1.990892in}{0.922964in}}%
\pgfpathlineto{\pgfqpoint{1.982473in}{0.934033in}}%
\pgfpathlineto{\pgfqpoint{1.980884in}{0.936375in}}%
\pgfpathlineto{\pgfqpoint{1.975121in}{0.945103in}}%
\pgfpathlineto{\pgfqpoint{1.969604in}{0.954989in}}%
\pgfpathlineto{\pgfqpoint{1.968900in}{0.956173in}}%
\pgfpathlineto{\pgfqpoint{1.964254in}{0.967242in}}%
\pgfpathlineto{\pgfqpoint{1.960018in}{0.978312in}}%
\pgfpathlineto{\pgfqpoint{1.958323in}{0.983105in}}%
\pgfpathlineto{\pgfqpoint{1.956020in}{0.989381in}}%
\pgfpathlineto{\pgfqpoint{1.952653in}{1.000451in}}%
\pgfpathlineto{\pgfqpoint{1.950191in}{1.011520in}}%
\pgfpathlineto{\pgfqpoint{1.949573in}{1.022590in}}%
\pgfpathlineto{\pgfqpoint{1.951369in}{1.033659in}}%
\pgfpathlineto{\pgfqpoint{1.952826in}{1.044729in}}%
\pgfpathlineto{\pgfqpoint{1.953824in}{1.055798in}}%
\pgfpathlineto{\pgfqpoint{1.954187in}{1.066868in}}%
\pgfpathlineto{\pgfqpoint{1.952868in}{1.077937in}}%
\pgfpathlineto{\pgfqpoint{1.947449in}{1.089007in}}%
\pgfpathlineto{\pgfqpoint{1.947042in}{1.089641in}}%
\pgfpathlineto{\pgfqpoint{1.940179in}{1.100077in}}%
\pgfpathlineto{\pgfqpoint{1.935761in}{1.106371in}}%
\pgfpathlineto{\pgfqpoint{1.932322in}{1.111146in}}%
\pgfpathlineto{\pgfqpoint{1.924480in}{1.121758in}}%
\pgfpathlineto{\pgfqpoint{1.924134in}{1.122216in}}%
\pgfpathlineto{\pgfqpoint{1.915721in}{1.133285in}}%
\pgfpathlineto{\pgfqpoint{1.913200in}{1.136546in}}%
\pgfpathlineto{\pgfqpoint{1.907052in}{1.144355in}}%
\pgfpathlineto{\pgfqpoint{1.901919in}{1.150762in}}%
\pgfpathlineto{\pgfqpoint{1.898122in}{1.155424in}}%
\pgfpathlineto{\pgfqpoint{1.890638in}{1.164492in}}%
\pgfpathlineto{\pgfqpoint{1.888961in}{1.166494in}}%
\pgfpathlineto{\pgfqpoint{1.879604in}{1.177563in}}%
\pgfpathlineto{\pgfqpoint{1.879357in}{1.177861in}}%
\pgfpathlineto{\pgfqpoint{1.870272in}{1.188633in}}%
\pgfpathlineto{\pgfqpoint{1.868076in}{1.191241in}}%
\pgfpathlineto{\pgfqpoint{1.860799in}{1.199702in}}%
\pgfpathlineto{\pgfqpoint{1.856795in}{1.204355in}}%
\pgfpathlineto{\pgfqpoint{1.856795in}{1.199702in}}%
\pgfpathlineto{\pgfqpoint{1.856795in}{1.188633in}}%
\pgfpathlineto{\pgfqpoint{1.856795in}{1.177563in}}%
\pgfpathlineto{\pgfqpoint{1.856795in}{1.166494in}}%
\pgfpathlineto{\pgfqpoint{1.856795in}{1.160478in}}%
\pgfpathlineto{\pgfqpoint{1.861120in}{1.155424in}}%
\pgfpathlineto{\pgfqpoint{1.868076in}{1.147284in}}%
\pgfpathlineto{\pgfqpoint{1.870534in}{1.144355in}}%
\pgfpathlineto{\pgfqpoint{1.879357in}{1.133761in}}%
\pgfpathlineto{\pgfqpoint{1.879747in}{1.133285in}}%
\pgfpathlineto{\pgfqpoint{1.888679in}{1.122216in}}%
\pgfpathlineto{\pgfqpoint{1.890638in}{1.119772in}}%
\pgfpathlineto{\pgfqpoint{1.897421in}{1.111146in}}%
\pgfpathlineto{\pgfqpoint{1.901919in}{1.105301in}}%
\pgfpathlineto{\pgfqpoint{1.905853in}{1.100077in}}%
\pgfpathlineto{\pgfqpoint{1.913200in}{1.089312in}}%
\pgfpathlineto{\pgfqpoint{1.913405in}{1.089007in}}%
\pgfpathlineto{\pgfqpoint{1.916780in}{1.077937in}}%
\pgfpathlineto{\pgfqpoint{1.916841in}{1.066868in}}%
\pgfpathlineto{\pgfqpoint{1.916361in}{1.055798in}}%
\pgfpathlineto{\pgfqpoint{1.915381in}{1.044729in}}%
\pgfpathlineto{\pgfqpoint{1.914025in}{1.033659in}}%
\pgfpathlineto{\pgfqpoint{1.914787in}{1.022590in}}%
\pgfpathlineto{\pgfqpoint{1.916900in}{1.011520in}}%
\pgfpathlineto{\pgfqpoint{1.919597in}{1.000451in}}%
\pgfpathlineto{\pgfqpoint{1.923639in}{0.989381in}}%
\pgfpathlineto{\pgfqpoint{1.924480in}{0.987230in}}%
\pgfpathlineto{\pgfqpoint{1.927912in}{0.978312in}}%
\pgfpathlineto{\pgfqpoint{1.933158in}{0.967242in}}%
\pgfpathlineto{\pgfqpoint{1.935761in}{0.963127in}}%
\pgfpathlineto{\pgfqpoint{1.939828in}{0.956173in}}%
\pgfpathlineto{\pgfqpoint{1.947042in}{0.945670in}}%
\pgfpathlineto{\pgfqpoint{1.947421in}{0.945103in}}%
\pgfpathlineto{\pgfqpoint{1.954950in}{0.934033in}}%
\pgfpathlineto{\pgfqpoint{1.958323in}{0.929063in}}%
\pgfpathlineto{\pgfqpoint{1.962433in}{0.922964in}}%
\pgfpathlineto{\pgfqpoint{1.969604in}{0.913616in}}%
\pgfpathlineto{\pgfqpoint{1.970936in}{0.911894in}}%
\pgfpathlineto{\pgfqpoint{1.980884in}{0.901008in}}%
\pgfpathlineto{\pgfqpoint{1.981042in}{0.900825in}}%
\pgfpathlineto{\pgfqpoint{1.992165in}{0.889787in}}%
\pgfpathlineto{\pgfqpoint{1.992197in}{0.889755in}}%
\pgfpathlineto{\pgfqpoint{2.003446in}{0.879003in}}%
\pgfpathlineto{\pgfqpoint{2.003818in}{0.878686in}}%
\pgfpathlineto{\pgfqpoint{2.014727in}{0.869577in}}%
\pgfpathlineto{\pgfqpoint{2.017314in}{0.867616in}}%
\pgfpathlineto{\pgfqpoint{2.026008in}{0.861136in}}%
\pgfpathlineto{\pgfqpoint{2.032246in}{0.856547in}}%
\pgfpathlineto{\pgfqpoint{2.037289in}{0.852931in}}%
\pgfpathlineto{\pgfqpoint{2.048279in}{0.845477in}}%
\pgfpathlineto{\pgfqpoint{2.048569in}{0.845278in}}%
\pgfpathlineto{\pgfqpoint{2.059850in}{0.838946in}}%
\pgfpathlineto{\pgfqpoint{2.068266in}{0.834408in}}%
\pgfpathlineto{\pgfqpoint{2.071131in}{0.832857in}}%
\pgfpathlineto{\pgfqpoint{2.082412in}{0.826650in}}%
\pgfpathlineto{\pgfqpoint{2.088431in}{0.823338in}}%
\pgfpathlineto{\pgfqpoint{2.093693in}{0.820385in}}%
\pgfpathlineto{\pgfqpoint{2.104973in}{0.814230in}}%
\pgfpathlineto{\pgfqpoint{2.108575in}{0.812269in}}%
\pgfpathlineto{\pgfqpoint{2.116254in}{0.807916in}}%
\pgfpathlineto{\pgfqpoint{2.127535in}{0.801743in}}%
\pgfpathlineto{\pgfqpoint{2.128541in}{0.801199in}}%
\pgfpathlineto{\pgfqpoint{2.138816in}{0.795372in}}%
\pgfpathlineto{\pgfqpoint{2.147994in}{0.790129in}}%
\pgfpathlineto{\pgfqpoint{2.150097in}{0.788903in}}%
\pgfpathlineto{\pgfqpoint{2.161377in}{0.782245in}}%
\pgfpathlineto{\pgfqpoint{2.166719in}{0.779060in}}%
\pgfpathlineto{\pgfqpoint{2.172658in}{0.775661in}}%
\pgfpathlineto{\pgfqpoint{2.183939in}{0.769110in}}%
\pgfpathlineto{\pgfqpoint{2.186079in}{0.767990in}}%
\pgfpathlineto{\pgfqpoint{2.195220in}{0.763658in}}%
\pgfpathlineto{\pgfqpoint{2.206501in}{0.758073in}}%
\pgfpathlineto{\pgfqpoint{2.208723in}{0.756921in}}%
\pgfpathlineto{\pgfqpoint{2.217782in}{0.752501in}}%
\pgfpathlineto{\pgfqpoint{2.229062in}{0.747028in}}%
\pgfpathlineto{\pgfqpoint{2.231452in}{0.745851in}}%
\pgfpathlineto{\pgfqpoint{2.240343in}{0.741447in}}%
\pgfpathlineto{\pgfqpoint{2.251624in}{0.735960in}}%
\pgfpathlineto{\pgfqpoint{2.254063in}{0.734782in}}%
\pgfpathlineto{\pgfqpoint{2.262905in}{0.730313in}}%
\pgfpathlineto{\pgfqpoint{2.274186in}{0.724505in}}%
\pgfpathlineto{\pgfqpoint{2.275721in}{0.723712in}}%
\pgfpathlineto{\pgfqpoint{2.285466in}{0.718466in}}%
\pgfpathlineto{\pgfqpoint{2.296488in}{0.712643in}}%
\pgfpathlineto{\pgfqpoint{2.296747in}{0.712502in}}%
\pgfpathlineto{\pgfqpoint{2.308028in}{0.706406in}}%
\pgfpathlineto{\pgfqpoint{2.316793in}{0.701573in}}%
\pgfpathlineto{\pgfqpoint{2.319309in}{0.700131in}}%
\pgfpathlineto{\pgfqpoint{2.330590in}{0.693427in}}%
\pgfpathlineto{\pgfqpoint{2.335437in}{0.690504in}}%
\pgfpathlineto{\pgfqpoint{2.341871in}{0.686522in}}%
\pgfpathlineto{\pgfqpoint{2.353151in}{0.679448in}}%
\pgfpathlineto{\pgfqpoint{2.353172in}{0.679434in}}%
\pgfpathlineto{\pgfqpoint{2.364432in}{0.671403in}}%
\pgfpathlineto{\pgfqpoint{2.368360in}{0.668364in}}%
\pgfpathlineto{\pgfqpoint{2.375713in}{0.662495in}}%
\pgfpathlineto{\pgfqpoint{2.381543in}{0.657295in}}%
\pgfpathlineto{\pgfqpoint{2.386994in}{0.652317in}}%
\pgfpathlineto{\pgfqpoint{2.393547in}{0.646225in}}%
\pgfpathlineto{\pgfqpoint{2.398275in}{0.641634in}}%
\pgfpathlineto{\pgfqpoint{2.404868in}{0.635156in}}%
\pgfpathlineto{\pgfqpoint{2.409555in}{0.630287in}}%
\pgfpathlineto{\pgfqpoint{2.416108in}{0.624086in}}%
\pgfpathlineto{\pgfqpoint{2.420836in}{0.619391in}}%
\pgfpathlineto{\pgfqpoint{2.427385in}{0.613017in}}%
\pgfpathlineto{\pgfqpoint{2.432117in}{0.608264in}}%
\pgfpathlineto{\pgfqpoint{2.438856in}{0.601947in}}%
\pgfpathlineto{\pgfqpoint{2.443398in}{0.597544in}}%
\pgfpathlineto{\pgfqpoint{2.450148in}{0.590878in}}%
\pgfpathlineto{\pgfqpoint{2.454679in}{0.586236in}}%
\pgfpathlineto{\pgfqpoint{2.460889in}{0.579808in}}%
\pgfpathlineto{\pgfqpoint{2.465960in}{0.574469in}}%
\pgfpathlineto{\pgfqpoint{2.471343in}{0.568739in}}%
\pgfpathlineto{\pgfqpoint{2.477240in}{0.562093in}}%
\pgfpathlineto{\pgfqpoint{2.481197in}{0.557669in}}%
\pgfpathlineto{\pgfqpoint{2.488521in}{0.548677in}}%
\pgfpathlineto{\pgfqpoint{2.490270in}{0.546600in}}%
\pgfpathlineto{\pgfqpoint{2.499345in}{0.535530in}}%
\pgfpathlineto{\pgfqpoint{2.499802in}{0.534937in}}%
\pgfpathlineto{\pgfqpoint{2.507819in}{0.524460in}}%
\pgfpathlineto{\pgfqpoint{2.511083in}{0.520042in}}%
\pgfpathlineto{\pgfqpoint{2.515971in}{0.513391in}}%
\pgfpathlineto{\pgfqpoint{2.522364in}{0.504512in}}%
\pgfpathlineto{\pgfqpoint{2.523963in}{0.502321in}}%
\pgfpathlineto{\pgfqpoint{2.531741in}{0.491252in}}%
\pgfpathclose%
\pgfusepath{fill}%
\end{pgfscope}%
\begin{pgfscope}%
\pgfpathrectangle{\pgfqpoint{1.856795in}{0.423750in}}{\pgfqpoint{1.194205in}{1.163386in}}%
\pgfusepath{clip}%
\pgfsetbuttcap%
\pgfsetroundjoin%
\definecolor{currentfill}{rgb}{0.962765,0.606121,0.444717}%
\pgfsetfillcolor{currentfill}%
\pgfsetlinewidth{0.000000pt}%
\definecolor{currentstroke}{rgb}{0.000000,0.000000,0.000000}%
\pgfsetstrokecolor{currentstroke}%
\pgfsetdash{}{0pt}%
\pgfpathmoveto{\pgfqpoint{2.657733in}{0.966102in}}%
\pgfpathlineto{\pgfqpoint{2.669014in}{0.963987in}}%
\pgfpathlineto{\pgfqpoint{2.680295in}{0.962389in}}%
\pgfpathlineto{\pgfqpoint{2.691576in}{0.963120in}}%
\pgfpathlineto{\pgfqpoint{2.702857in}{0.964383in}}%
\pgfpathlineto{\pgfqpoint{2.714137in}{0.967118in}}%
\pgfpathlineto{\pgfqpoint{2.714533in}{0.967242in}}%
\pgfpathlineto{\pgfqpoint{2.725418in}{0.970478in}}%
\pgfpathlineto{\pgfqpoint{2.736699in}{0.976031in}}%
\pgfpathlineto{\pgfqpoint{2.740467in}{0.978312in}}%
\pgfpathlineto{\pgfqpoint{2.747980in}{0.983122in}}%
\pgfpathlineto{\pgfqpoint{2.756675in}{0.989381in}}%
\pgfpathlineto{\pgfqpoint{2.759261in}{0.991144in}}%
\pgfpathlineto{\pgfqpoint{2.770542in}{0.999538in}}%
\pgfpathlineto{\pgfqpoint{2.771609in}{1.000451in}}%
\pgfpathlineto{\pgfqpoint{2.781822in}{1.008580in}}%
\pgfpathlineto{\pgfqpoint{2.785307in}{1.011520in}}%
\pgfpathlineto{\pgfqpoint{2.793103in}{1.018158in}}%
\pgfpathlineto{\pgfqpoint{2.799146in}{1.022590in}}%
\pgfpathlineto{\pgfqpoint{2.804384in}{1.026462in}}%
\pgfpathlineto{\pgfqpoint{2.813840in}{1.033659in}}%
\pgfpathlineto{\pgfqpoint{2.815665in}{1.035147in}}%
\pgfpathlineto{\pgfqpoint{2.826946in}{1.044461in}}%
\pgfpathlineto{\pgfqpoint{2.827266in}{1.044729in}}%
\pgfpathlineto{\pgfqpoint{2.838226in}{1.054205in}}%
\pgfpathlineto{\pgfqpoint{2.840052in}{1.055798in}}%
\pgfpathlineto{\pgfqpoint{2.849507in}{1.063970in}}%
\pgfpathlineto{\pgfqpoint{2.852898in}{1.066868in}}%
\pgfpathlineto{\pgfqpoint{2.860788in}{1.073843in}}%
\pgfpathlineto{\pgfqpoint{2.865506in}{1.077937in}}%
\pgfpathlineto{\pgfqpoint{2.872069in}{1.084051in}}%
\pgfpathlineto{\pgfqpoint{2.877250in}{1.089007in}}%
\pgfpathlineto{\pgfqpoint{2.883350in}{1.095010in}}%
\pgfpathlineto{\pgfqpoint{2.888372in}{1.100077in}}%
\pgfpathlineto{\pgfqpoint{2.894631in}{1.106392in}}%
\pgfpathlineto{\pgfqpoint{2.899224in}{1.111146in}}%
\pgfpathlineto{\pgfqpoint{2.905911in}{1.117405in}}%
\pgfpathlineto{\pgfqpoint{2.910793in}{1.122216in}}%
\pgfpathlineto{\pgfqpoint{2.917192in}{1.128342in}}%
\pgfpathlineto{\pgfqpoint{2.922231in}{1.133285in}}%
\pgfpathlineto{\pgfqpoint{2.928473in}{1.139620in}}%
\pgfpathlineto{\pgfqpoint{2.933021in}{1.144355in}}%
\pgfpathlineto{\pgfqpoint{2.939754in}{1.151589in}}%
\pgfpathlineto{\pgfqpoint{2.943386in}{1.155424in}}%
\pgfpathlineto{\pgfqpoint{2.951035in}{1.163714in}}%
\pgfpathlineto{\pgfqpoint{2.953632in}{1.166494in}}%
\pgfpathlineto{\pgfqpoint{2.962315in}{1.176503in}}%
\pgfpathlineto{\pgfqpoint{2.963275in}{1.177563in}}%
\pgfpathlineto{\pgfqpoint{2.973139in}{1.188633in}}%
\pgfpathlineto{\pgfqpoint{2.973596in}{1.189179in}}%
\pgfpathlineto{\pgfqpoint{2.973596in}{1.199702in}}%
\pgfpathlineto{\pgfqpoint{2.973596in}{1.210772in}}%
\pgfpathlineto{\pgfqpoint{2.973596in}{1.221842in}}%
\pgfpathlineto{\pgfqpoint{2.973596in}{1.232911in}}%
\pgfpathlineto{\pgfqpoint{2.973596in}{1.243981in}}%
\pgfpathlineto{\pgfqpoint{2.973596in}{1.255050in}}%
\pgfpathlineto{\pgfqpoint{2.973596in}{1.266120in}}%
\pgfpathlineto{\pgfqpoint{2.973596in}{1.267043in}}%
\pgfpathlineto{\pgfqpoint{2.972817in}{1.266120in}}%
\pgfpathlineto{\pgfqpoint{2.964066in}{1.255050in}}%
\pgfpathlineto{\pgfqpoint{2.962315in}{1.252860in}}%
\pgfpathlineto{\pgfqpoint{2.955089in}{1.243981in}}%
\pgfpathlineto{\pgfqpoint{2.951035in}{1.239061in}}%
\pgfpathlineto{\pgfqpoint{2.945897in}{1.232911in}}%
\pgfpathlineto{\pgfqpoint{2.939754in}{1.225665in}}%
\pgfpathlineto{\pgfqpoint{2.936616in}{1.221842in}}%
\pgfpathlineto{\pgfqpoint{2.928473in}{1.212332in}}%
\pgfpathlineto{\pgfqpoint{2.927140in}{1.210772in}}%
\pgfpathlineto{\pgfqpoint{2.917805in}{1.199702in}}%
\pgfpathlineto{\pgfqpoint{2.917192in}{1.198998in}}%
\pgfpathlineto{\pgfqpoint{2.908169in}{1.188633in}}%
\pgfpathlineto{\pgfqpoint{2.905911in}{1.186104in}}%
\pgfpathlineto{\pgfqpoint{2.898128in}{1.177563in}}%
\pgfpathlineto{\pgfqpoint{2.894631in}{1.173803in}}%
\pgfpathlineto{\pgfqpoint{2.887651in}{1.166494in}}%
\pgfpathlineto{\pgfqpoint{2.883350in}{1.162103in}}%
\pgfpathlineto{\pgfqpoint{2.876636in}{1.155424in}}%
\pgfpathlineto{\pgfqpoint{2.872069in}{1.151008in}}%
\pgfpathlineto{\pgfqpoint{2.865010in}{1.144355in}}%
\pgfpathlineto{\pgfqpoint{2.860788in}{1.140439in}}%
\pgfpathlineto{\pgfqpoint{2.853045in}{1.133285in}}%
\pgfpathlineto{\pgfqpoint{2.849507in}{1.130016in}}%
\pgfpathlineto{\pgfqpoint{2.840913in}{1.122216in}}%
\pgfpathlineto{\pgfqpoint{2.838226in}{1.119829in}}%
\pgfpathlineto{\pgfqpoint{2.828283in}{1.111146in}}%
\pgfpathlineto{\pgfqpoint{2.826946in}{1.109975in}}%
\pgfpathlineto{\pgfqpoint{2.815665in}{1.100409in}}%
\pgfpathlineto{\pgfqpoint{2.815271in}{1.100077in}}%
\pgfpathlineto{\pgfqpoint{2.804384in}{1.091051in}}%
\pgfpathlineto{\pgfqpoint{2.802027in}{1.089007in}}%
\pgfpathlineto{\pgfqpoint{2.793103in}{1.081166in}}%
\pgfpathlineto{\pgfqpoint{2.789765in}{1.077937in}}%
\pgfpathlineto{\pgfqpoint{2.781822in}{1.070244in}}%
\pgfpathlineto{\pgfqpoint{2.778210in}{1.066868in}}%
\pgfpathlineto{\pgfqpoint{2.770542in}{1.059777in}}%
\pgfpathlineto{\pgfqpoint{2.766004in}{1.055798in}}%
\pgfpathlineto{\pgfqpoint{2.759261in}{1.050046in}}%
\pgfpathlineto{\pgfqpoint{2.752035in}{1.044729in}}%
\pgfpathlineto{\pgfqpoint{2.747980in}{1.041702in}}%
\pgfpathlineto{\pgfqpoint{2.736699in}{1.033861in}}%
\pgfpathlineto{\pgfqpoint{2.736380in}{1.033659in}}%
\pgfpathlineto{\pgfqpoint{2.725418in}{1.026579in}}%
\pgfpathlineto{\pgfqpoint{2.717788in}{1.022590in}}%
\pgfpathlineto{\pgfqpoint{2.714137in}{1.020815in}}%
\pgfpathlineto{\pgfqpoint{2.702857in}{1.016611in}}%
\pgfpathlineto{\pgfqpoint{2.691576in}{1.014381in}}%
\pgfpathlineto{\pgfqpoint{2.680295in}{1.016060in}}%
\pgfpathlineto{\pgfqpoint{2.669014in}{1.019119in}}%
\pgfpathlineto{\pgfqpoint{2.658770in}{1.022590in}}%
\pgfpathlineto{\pgfqpoint{2.657733in}{1.022949in}}%
\pgfpathlineto{\pgfqpoint{2.646453in}{1.027308in}}%
\pgfpathlineto{\pgfqpoint{2.635172in}{1.032014in}}%
\pgfpathlineto{\pgfqpoint{2.631464in}{1.033659in}}%
\pgfpathlineto{\pgfqpoint{2.623891in}{1.036915in}}%
\pgfpathlineto{\pgfqpoint{2.612610in}{1.042395in}}%
\pgfpathlineto{\pgfqpoint{2.608505in}{1.044729in}}%
\pgfpathlineto{\pgfqpoint{2.601329in}{1.048939in}}%
\pgfpathlineto{\pgfqpoint{2.590234in}{1.055798in}}%
\pgfpathlineto{\pgfqpoint{2.590048in}{1.055915in}}%
\pgfpathlineto{\pgfqpoint{2.578768in}{1.063186in}}%
\pgfpathlineto{\pgfqpoint{2.573203in}{1.066868in}}%
\pgfpathlineto{\pgfqpoint{2.567487in}{1.070724in}}%
\pgfpathlineto{\pgfqpoint{2.557116in}{1.077937in}}%
\pgfpathlineto{\pgfqpoint{2.556206in}{1.078600in}}%
\pgfpathlineto{\pgfqpoint{2.544925in}{1.086927in}}%
\pgfpathlineto{\pgfqpoint{2.542096in}{1.089007in}}%
\pgfpathlineto{\pgfqpoint{2.533644in}{1.095253in}}%
\pgfpathlineto{\pgfqpoint{2.527224in}{1.100077in}}%
\pgfpathlineto{\pgfqpoint{2.522364in}{1.103548in}}%
\pgfpathlineto{\pgfqpoint{2.511922in}{1.111146in}}%
\pgfpathlineto{\pgfqpoint{2.511083in}{1.111742in}}%
\pgfpathlineto{\pgfqpoint{2.499802in}{1.119735in}}%
\pgfpathlineto{\pgfqpoint{2.496347in}{1.122216in}}%
\pgfpathlineto{\pgfqpoint{2.488521in}{1.127945in}}%
\pgfpathlineto{\pgfqpoint{2.481284in}{1.133285in}}%
\pgfpathlineto{\pgfqpoint{2.477240in}{1.136368in}}%
\pgfpathlineto{\pgfqpoint{2.467043in}{1.144355in}}%
\pgfpathlineto{\pgfqpoint{2.465960in}{1.145213in}}%
\pgfpathlineto{\pgfqpoint{2.454679in}{1.154210in}}%
\pgfpathlineto{\pgfqpoint{2.453153in}{1.155424in}}%
\pgfpathlineto{\pgfqpoint{2.443398in}{1.163212in}}%
\pgfpathlineto{\pgfqpoint{2.439323in}{1.166494in}}%
\pgfpathlineto{\pgfqpoint{2.432117in}{1.172524in}}%
\pgfpathlineto{\pgfqpoint{2.425973in}{1.177563in}}%
\pgfpathlineto{\pgfqpoint{2.420836in}{1.181933in}}%
\pgfpathlineto{\pgfqpoint{2.412960in}{1.188633in}}%
\pgfpathlineto{\pgfqpoint{2.409555in}{1.191549in}}%
\pgfpathlineto{\pgfqpoint{2.400005in}{1.199702in}}%
\pgfpathlineto{\pgfqpoint{2.398275in}{1.201197in}}%
\pgfpathlineto{\pgfqpoint{2.387160in}{1.210772in}}%
\pgfpathlineto{\pgfqpoint{2.386994in}{1.210917in}}%
\pgfpathlineto{\pgfqpoint{2.375713in}{1.220846in}}%
\pgfpathlineto{\pgfqpoint{2.374589in}{1.221842in}}%
\pgfpathlineto{\pgfqpoint{2.364432in}{1.231144in}}%
\pgfpathlineto{\pgfqpoint{2.362468in}{1.232911in}}%
\pgfpathlineto{\pgfqpoint{2.353151in}{1.241880in}}%
\pgfpathlineto{\pgfqpoint{2.350967in}{1.243981in}}%
\pgfpathlineto{\pgfqpoint{2.341871in}{1.252822in}}%
\pgfpathlineto{\pgfqpoint{2.339632in}{1.255050in}}%
\pgfpathlineto{\pgfqpoint{2.330590in}{1.264395in}}%
\pgfpathlineto{\pgfqpoint{2.328942in}{1.266120in}}%
\pgfpathlineto{\pgfqpoint{2.319309in}{1.276411in}}%
\pgfpathlineto{\pgfqpoint{2.318581in}{1.277189in}}%
\pgfpathlineto{\pgfqpoint{2.308281in}{1.288259in}}%
\pgfpathlineto{\pgfqpoint{2.308028in}{1.288520in}}%
\pgfpathlineto{\pgfqpoint{2.298598in}{1.299328in}}%
\pgfpathlineto{\pgfqpoint{2.296747in}{1.301333in}}%
\pgfpathlineto{\pgfqpoint{2.288550in}{1.310398in}}%
\pgfpathlineto{\pgfqpoint{2.285466in}{1.313782in}}%
\pgfpathlineto{\pgfqpoint{2.278487in}{1.321467in}}%
\pgfpathlineto{\pgfqpoint{2.274186in}{1.326259in}}%
\pgfpathlineto{\pgfqpoint{2.268569in}{1.332537in}}%
\pgfpathlineto{\pgfqpoint{2.262905in}{1.338946in}}%
\pgfpathlineto{\pgfqpoint{2.258942in}{1.343606in}}%
\pgfpathlineto{\pgfqpoint{2.251624in}{1.352807in}}%
\pgfpathlineto{\pgfqpoint{2.250167in}{1.354676in}}%
\pgfpathlineto{\pgfqpoint{2.241546in}{1.365746in}}%
\pgfpathlineto{\pgfqpoint{2.240343in}{1.367320in}}%
\pgfpathlineto{\pgfqpoint{2.233040in}{1.376815in}}%
\pgfpathlineto{\pgfqpoint{2.229062in}{1.382071in}}%
\pgfpathlineto{\pgfqpoint{2.224623in}{1.387885in}}%
\pgfpathlineto{\pgfqpoint{2.217782in}{1.397062in}}%
\pgfpathlineto{\pgfqpoint{2.216358in}{1.398954in}}%
\pgfpathlineto{\pgfqpoint{2.208147in}{1.410024in}}%
\pgfpathlineto{\pgfqpoint{2.206501in}{1.412250in}}%
\pgfpathlineto{\pgfqpoint{2.199919in}{1.421093in}}%
\pgfpathlineto{\pgfqpoint{2.195220in}{1.427586in}}%
\pgfpathlineto{\pgfqpoint{2.191879in}{1.432163in}}%
\pgfpathlineto{\pgfqpoint{2.183939in}{1.443218in}}%
\pgfpathlineto{\pgfqpoint{2.183929in}{1.443232in}}%
\pgfpathlineto{\pgfqpoint{2.176076in}{1.454302in}}%
\pgfpathlineto{\pgfqpoint{2.172658in}{1.459224in}}%
\pgfpathlineto{\pgfqpoint{2.168358in}{1.465371in}}%
\pgfpathlineto{\pgfqpoint{2.161377in}{1.475550in}}%
\pgfpathlineto{\pgfqpoint{2.160754in}{1.476441in}}%
\pgfpathlineto{\pgfqpoint{2.152870in}{1.487510in}}%
\pgfpathlineto{\pgfqpoint{2.150097in}{1.491276in}}%
\pgfpathlineto{\pgfqpoint{2.144796in}{1.498580in}}%
\pgfpathlineto{\pgfqpoint{2.138816in}{1.506608in}}%
\pgfpathlineto{\pgfqpoint{2.136579in}{1.509650in}}%
\pgfpathlineto{\pgfqpoint{2.128435in}{1.520719in}}%
\pgfpathlineto{\pgfqpoint{2.127535in}{1.521974in}}%
\pgfpathlineto{\pgfqpoint{2.120497in}{1.531789in}}%
\pgfpathlineto{\pgfqpoint{2.116254in}{1.537577in}}%
\pgfpathlineto{\pgfqpoint{2.112455in}{1.542858in}}%
\pgfpathlineto{\pgfqpoint{2.104973in}{1.552857in}}%
\pgfpathlineto{\pgfqpoint{2.104173in}{1.553928in}}%
\pgfpathlineto{\pgfqpoint{2.096084in}{1.564997in}}%
\pgfpathlineto{\pgfqpoint{2.093693in}{1.568347in}}%
\pgfpathlineto{\pgfqpoint{2.088180in}{1.576067in}}%
\pgfpathlineto{\pgfqpoint{2.082412in}{1.584283in}}%
\pgfpathlineto{\pgfqpoint{2.080407in}{1.587136in}}%
\pgfpathlineto{\pgfqpoint{2.071131in}{1.587136in}}%
\pgfpathlineto{\pgfqpoint{2.059850in}{1.587136in}}%
\pgfpathlineto{\pgfqpoint{2.048569in}{1.587136in}}%
\pgfpathlineto{\pgfqpoint{2.037289in}{1.587136in}}%
\pgfpathlineto{\pgfqpoint{2.026300in}{1.587136in}}%
\pgfpathlineto{\pgfqpoint{2.034326in}{1.576067in}}%
\pgfpathlineto{\pgfqpoint{2.037289in}{1.572067in}}%
\pgfpathlineto{\pgfqpoint{2.042527in}{1.564997in}}%
\pgfpathlineto{\pgfqpoint{2.048569in}{1.556842in}}%
\pgfpathlineto{\pgfqpoint{2.050755in}{1.553928in}}%
\pgfpathlineto{\pgfqpoint{2.058658in}{1.542858in}}%
\pgfpathlineto{\pgfqpoint{2.059850in}{1.541172in}}%
\pgfpathlineto{\pgfqpoint{2.066496in}{1.531789in}}%
\pgfpathlineto{\pgfqpoint{2.071131in}{1.525358in}}%
\pgfpathlineto{\pgfqpoint{2.074471in}{1.520719in}}%
\pgfpathlineto{\pgfqpoint{2.082170in}{1.509650in}}%
\pgfpathlineto{\pgfqpoint{2.082412in}{1.509280in}}%
\pgfpathlineto{\pgfqpoint{2.089488in}{1.498580in}}%
\pgfpathlineto{\pgfqpoint{2.093693in}{1.492235in}}%
\pgfpathlineto{\pgfqpoint{2.096815in}{1.487510in}}%
\pgfpathlineto{\pgfqpoint{2.104300in}{1.476441in}}%
\pgfpathlineto{\pgfqpoint{2.104973in}{1.475461in}}%
\pgfpathlineto{\pgfqpoint{2.111909in}{1.465371in}}%
\pgfpathlineto{\pgfqpoint{2.116254in}{1.459122in}}%
\pgfpathlineto{\pgfqpoint{2.119614in}{1.454302in}}%
\pgfpathlineto{\pgfqpoint{2.127299in}{1.443232in}}%
\pgfpathlineto{\pgfqpoint{2.127535in}{1.442884in}}%
\pgfpathlineto{\pgfqpoint{2.134871in}{1.432163in}}%
\pgfpathlineto{\pgfqpoint{2.138816in}{1.426421in}}%
\pgfpathlineto{\pgfqpoint{2.142488in}{1.421093in}}%
\pgfpathlineto{\pgfqpoint{2.150097in}{1.410174in}}%
\pgfpathlineto{\pgfqpoint{2.150201in}{1.410024in}}%
\pgfpathlineto{\pgfqpoint{2.158070in}{1.398954in}}%
\pgfpathlineto{\pgfqpoint{2.161377in}{1.394322in}}%
\pgfpathlineto{\pgfqpoint{2.165966in}{1.387885in}}%
\pgfpathlineto{\pgfqpoint{2.172658in}{1.377984in}}%
\pgfpathlineto{\pgfqpoint{2.173503in}{1.376815in}}%
\pgfpathlineto{\pgfqpoint{2.180564in}{1.365746in}}%
\pgfpathlineto{\pgfqpoint{2.183939in}{1.360281in}}%
\pgfpathlineto{\pgfqpoint{2.187481in}{1.354676in}}%
\pgfpathlineto{\pgfqpoint{2.194508in}{1.343606in}}%
\pgfpathlineto{\pgfqpoint{2.195220in}{1.342487in}}%
\pgfpathlineto{\pgfqpoint{2.201665in}{1.332537in}}%
\pgfpathlineto{\pgfqpoint{2.206501in}{1.325116in}}%
\pgfpathlineto{\pgfqpoint{2.208911in}{1.321467in}}%
\pgfpathlineto{\pgfqpoint{2.216332in}{1.310398in}}%
\pgfpathlineto{\pgfqpoint{2.217782in}{1.308264in}}%
\pgfpathlineto{\pgfqpoint{2.223948in}{1.299328in}}%
\pgfpathlineto{\pgfqpoint{2.229062in}{1.292137in}}%
\pgfpathlineto{\pgfqpoint{2.231977in}{1.288259in}}%
\pgfpathlineto{\pgfqpoint{2.240343in}{1.277788in}}%
\pgfpathlineto{\pgfqpoint{2.240846in}{1.277189in}}%
\pgfpathlineto{\pgfqpoint{2.250252in}{1.266120in}}%
\pgfpathlineto{\pgfqpoint{2.251624in}{1.264586in}}%
\pgfpathlineto{\pgfqpoint{2.260126in}{1.255050in}}%
\pgfpathlineto{\pgfqpoint{2.262905in}{1.252027in}}%
\pgfpathlineto{\pgfqpoint{2.270475in}{1.243981in}}%
\pgfpathlineto{\pgfqpoint{2.274186in}{1.240175in}}%
\pgfpathlineto{\pgfqpoint{2.281770in}{1.232911in}}%
\pgfpathlineto{\pgfqpoint{2.285466in}{1.229460in}}%
\pgfpathlineto{\pgfqpoint{2.293407in}{1.221842in}}%
\pgfpathlineto{\pgfqpoint{2.296747in}{1.218865in}}%
\pgfpathlineto{\pgfqpoint{2.306062in}{1.210772in}}%
\pgfpathlineto{\pgfqpoint{2.308028in}{1.209088in}}%
\pgfpathlineto{\pgfqpoint{2.319064in}{1.199702in}}%
\pgfpathlineto{\pgfqpoint{2.319309in}{1.199495in}}%
\pgfpathlineto{\pgfqpoint{2.330590in}{1.189885in}}%
\pgfpathlineto{\pgfqpoint{2.332046in}{1.188633in}}%
\pgfpathlineto{\pgfqpoint{2.341871in}{1.180222in}}%
\pgfpathlineto{\pgfqpoint{2.344996in}{1.177563in}}%
\pgfpathlineto{\pgfqpoint{2.353151in}{1.170712in}}%
\pgfpathlineto{\pgfqpoint{2.358191in}{1.166494in}}%
\pgfpathlineto{\pgfqpoint{2.364432in}{1.161482in}}%
\pgfpathlineto{\pgfqpoint{2.371781in}{1.155424in}}%
\pgfpathlineto{\pgfqpoint{2.375713in}{1.152282in}}%
\pgfpathlineto{\pgfqpoint{2.385590in}{1.144355in}}%
\pgfpathlineto{\pgfqpoint{2.386994in}{1.143243in}}%
\pgfpathlineto{\pgfqpoint{2.398275in}{1.134085in}}%
\pgfpathlineto{\pgfqpoint{2.399253in}{1.133285in}}%
\pgfpathlineto{\pgfqpoint{2.409555in}{1.125049in}}%
\pgfpathlineto{\pgfqpoint{2.413121in}{1.122216in}}%
\pgfpathlineto{\pgfqpoint{2.420836in}{1.116296in}}%
\pgfpathlineto{\pgfqpoint{2.427596in}{1.111146in}}%
\pgfpathlineto{\pgfqpoint{2.432117in}{1.107543in}}%
\pgfpathlineto{\pgfqpoint{2.441538in}{1.100077in}}%
\pgfpathlineto{\pgfqpoint{2.443398in}{1.098544in}}%
\pgfpathlineto{\pgfqpoint{2.454679in}{1.089348in}}%
\pgfpathlineto{\pgfqpoint{2.455097in}{1.089007in}}%
\pgfpathlineto{\pgfqpoint{2.465960in}{1.080371in}}%
\pgfpathlineto{\pgfqpoint{2.469105in}{1.077937in}}%
\pgfpathlineto{\pgfqpoint{2.477240in}{1.071602in}}%
\pgfpathlineto{\pgfqpoint{2.483445in}{1.066868in}}%
\pgfpathlineto{\pgfqpoint{2.488521in}{1.062952in}}%
\pgfpathlineto{\pgfqpoint{2.498079in}{1.055798in}}%
\pgfpathlineto{\pgfqpoint{2.499802in}{1.054524in}}%
\pgfpathlineto{\pgfqpoint{2.511083in}{1.046461in}}%
\pgfpathlineto{\pgfqpoint{2.513504in}{1.044729in}}%
\pgfpathlineto{\pgfqpoint{2.522364in}{1.038147in}}%
\pgfpathlineto{\pgfqpoint{2.528743in}{1.033659in}}%
\pgfpathlineto{\pgfqpoint{2.533644in}{1.029888in}}%
\pgfpathlineto{\pgfqpoint{2.543149in}{1.022590in}}%
\pgfpathlineto{\pgfqpoint{2.544925in}{1.021236in}}%
\pgfpathlineto{\pgfqpoint{2.556206in}{1.012816in}}%
\pgfpathlineto{\pgfqpoint{2.558034in}{1.011520in}}%
\pgfpathlineto{\pgfqpoint{2.567487in}{1.005091in}}%
\pgfpathlineto{\pgfqpoint{2.574632in}{1.000451in}}%
\pgfpathlineto{\pgfqpoint{2.578768in}{0.997962in}}%
\pgfpathlineto{\pgfqpoint{2.590048in}{0.991511in}}%
\pgfpathlineto{\pgfqpoint{2.594209in}{0.989381in}}%
\pgfpathlineto{\pgfqpoint{2.601329in}{0.986002in}}%
\pgfpathlineto{\pgfqpoint{2.612610in}{0.981032in}}%
\pgfpathlineto{\pgfqpoint{2.619478in}{0.978312in}}%
\pgfpathlineto{\pgfqpoint{2.623891in}{0.976664in}}%
\pgfpathlineto{\pgfqpoint{2.635172in}{0.972590in}}%
\pgfpathlineto{\pgfqpoint{2.646453in}{0.968967in}}%
\pgfpathlineto{\pgfqpoint{2.653070in}{0.967242in}}%
\pgfpathclose%
\pgfusepath{fill}%
\end{pgfscope}%
\begin{pgfscope}%
\pgfpathrectangle{\pgfqpoint{1.856795in}{0.423750in}}{\pgfqpoint{1.194205in}{1.163386in}}%
\pgfusepath{clip}%
\pgfsetbuttcap%
\pgfsetroundjoin%
\definecolor{currentfill}{rgb}{0.964679,0.682838,0.530002}%
\pgfsetfillcolor{currentfill}%
\pgfsetlinewidth{0.000000pt}%
\definecolor{currentstroke}{rgb}{0.000000,0.000000,0.000000}%
\pgfsetstrokecolor{currentstroke}%
\pgfsetdash{}{0pt}%
\pgfpathmoveto{\pgfqpoint{2.590048in}{0.491252in}}%
\pgfpathlineto{\pgfqpoint{2.601329in}{0.491252in}}%
\pgfpathlineto{\pgfqpoint{2.612610in}{0.491252in}}%
\pgfpathlineto{\pgfqpoint{2.623891in}{0.491252in}}%
\pgfpathlineto{\pgfqpoint{2.635172in}{0.491252in}}%
\pgfpathlineto{\pgfqpoint{2.646453in}{0.491252in}}%
\pgfpathlineto{\pgfqpoint{2.654626in}{0.491252in}}%
\pgfpathlineto{\pgfqpoint{2.647043in}{0.502321in}}%
\pgfpathlineto{\pgfqpoint{2.646453in}{0.503103in}}%
\pgfpathlineto{\pgfqpoint{2.638759in}{0.513391in}}%
\pgfpathlineto{\pgfqpoint{2.635172in}{0.518167in}}%
\pgfpathlineto{\pgfqpoint{2.630447in}{0.524460in}}%
\pgfpathlineto{\pgfqpoint{2.623891in}{0.532558in}}%
\pgfpathlineto{\pgfqpoint{2.621361in}{0.535530in}}%
\pgfpathlineto{\pgfqpoint{2.612610in}{0.545758in}}%
\pgfpathlineto{\pgfqpoint{2.611901in}{0.546600in}}%
\pgfpathlineto{\pgfqpoint{2.602342in}{0.557669in}}%
\pgfpathlineto{\pgfqpoint{2.601329in}{0.558794in}}%
\pgfpathlineto{\pgfqpoint{2.592062in}{0.568739in}}%
\pgfpathlineto{\pgfqpoint{2.590048in}{0.570757in}}%
\pgfpathlineto{\pgfqpoint{2.581670in}{0.579808in}}%
\pgfpathlineto{\pgfqpoint{2.578768in}{0.582574in}}%
\pgfpathlineto{\pgfqpoint{2.570299in}{0.590878in}}%
\pgfpathlineto{\pgfqpoint{2.567487in}{0.593383in}}%
\pgfpathlineto{\pgfqpoint{2.558220in}{0.601947in}}%
\pgfpathlineto{\pgfqpoint{2.556206in}{0.603727in}}%
\pgfpathlineto{\pgfqpoint{2.545726in}{0.613017in}}%
\pgfpathlineto{\pgfqpoint{2.544925in}{0.613691in}}%
\pgfpathlineto{\pgfqpoint{2.533644in}{0.623061in}}%
\pgfpathlineto{\pgfqpoint{2.532380in}{0.624086in}}%
\pgfpathlineto{\pgfqpoint{2.522364in}{0.631411in}}%
\pgfpathlineto{\pgfqpoint{2.517697in}{0.635156in}}%
\pgfpathlineto{\pgfqpoint{2.511083in}{0.640345in}}%
\pgfpathlineto{\pgfqpoint{2.504398in}{0.646225in}}%
\pgfpathlineto{\pgfqpoint{2.499802in}{0.650381in}}%
\pgfpathlineto{\pgfqpoint{2.492382in}{0.657295in}}%
\pgfpathlineto{\pgfqpoint{2.488521in}{0.661002in}}%
\pgfpathlineto{\pgfqpoint{2.480176in}{0.668364in}}%
\pgfpathlineto{\pgfqpoint{2.477240in}{0.671215in}}%
\pgfpathlineto{\pgfqpoint{2.467393in}{0.679434in}}%
\pgfpathlineto{\pgfqpoint{2.465960in}{0.680742in}}%
\pgfpathlineto{\pgfqpoint{2.454679in}{0.690412in}}%
\pgfpathlineto{\pgfqpoint{2.454554in}{0.690504in}}%
\pgfpathlineto{\pgfqpoint{2.443398in}{0.699318in}}%
\pgfpathlineto{\pgfqpoint{2.440526in}{0.701573in}}%
\pgfpathlineto{\pgfqpoint{2.432117in}{0.707947in}}%
\pgfpathlineto{\pgfqpoint{2.425872in}{0.712643in}}%
\pgfpathlineto{\pgfqpoint{2.420836in}{0.716298in}}%
\pgfpathlineto{\pgfqpoint{2.410588in}{0.723712in}}%
\pgfpathlineto{\pgfqpoint{2.409555in}{0.724411in}}%
\pgfpathlineto{\pgfqpoint{2.398275in}{0.731226in}}%
\pgfpathlineto{\pgfqpoint{2.392067in}{0.734782in}}%
\pgfpathlineto{\pgfqpoint{2.386994in}{0.737506in}}%
\pgfpathlineto{\pgfqpoint{2.375713in}{0.743044in}}%
\pgfpathlineto{\pgfqpoint{2.370006in}{0.745851in}}%
\pgfpathlineto{\pgfqpoint{2.364432in}{0.748485in}}%
\pgfpathlineto{\pgfqpoint{2.353151in}{0.754400in}}%
\pgfpathlineto{\pgfqpoint{2.348427in}{0.756921in}}%
\pgfpathlineto{\pgfqpoint{2.341871in}{0.760318in}}%
\pgfpathlineto{\pgfqpoint{2.330590in}{0.766242in}}%
\pgfpathlineto{\pgfqpoint{2.327297in}{0.767990in}}%
\pgfpathlineto{\pgfqpoint{2.319309in}{0.772117in}}%
\pgfpathlineto{\pgfqpoint{2.308028in}{0.777994in}}%
\pgfpathlineto{\pgfqpoint{2.306008in}{0.779060in}}%
\pgfpathlineto{\pgfqpoint{2.296747in}{0.783809in}}%
\pgfpathlineto{\pgfqpoint{2.285466in}{0.789464in}}%
\pgfpathlineto{\pgfqpoint{2.284161in}{0.790129in}}%
\pgfpathlineto{\pgfqpoint{2.274186in}{0.795154in}}%
\pgfpathlineto{\pgfqpoint{2.262905in}{0.800592in}}%
\pgfpathlineto{\pgfqpoint{2.261656in}{0.801199in}}%
\pgfpathlineto{\pgfqpoint{2.251624in}{0.806084in}}%
\pgfpathlineto{\pgfqpoint{2.240343in}{0.811418in}}%
\pgfpathlineto{\pgfqpoint{2.238562in}{0.812269in}}%
\pgfpathlineto{\pgfqpoint{2.229062in}{0.816919in}}%
\pgfpathlineto{\pgfqpoint{2.217782in}{0.822180in}}%
\pgfpathlineto{\pgfqpoint{2.215238in}{0.823338in}}%
\pgfpathlineto{\pgfqpoint{2.206501in}{0.827561in}}%
\pgfpathlineto{\pgfqpoint{2.195220in}{0.832888in}}%
\pgfpathlineto{\pgfqpoint{2.191809in}{0.834408in}}%
\pgfpathlineto{\pgfqpoint{2.183939in}{0.838146in}}%
\pgfpathlineto{\pgfqpoint{2.172658in}{0.843396in}}%
\pgfpathlineto{\pgfqpoint{2.168057in}{0.845477in}}%
\pgfpathlineto{\pgfqpoint{2.161377in}{0.848585in}}%
\pgfpathlineto{\pgfqpoint{2.150097in}{0.854039in}}%
\pgfpathlineto{\pgfqpoint{2.144998in}{0.856547in}}%
\pgfpathlineto{\pgfqpoint{2.138816in}{0.859545in}}%
\pgfpathlineto{\pgfqpoint{2.127535in}{0.864893in}}%
\pgfpathlineto{\pgfqpoint{2.121732in}{0.867616in}}%
\pgfpathlineto{\pgfqpoint{2.116254in}{0.870520in}}%
\pgfpathlineto{\pgfqpoint{2.104973in}{0.876333in}}%
\pgfpathlineto{\pgfqpoint{2.100471in}{0.878686in}}%
\pgfpathlineto{\pgfqpoint{2.093693in}{0.883212in}}%
\pgfpathlineto{\pgfqpoint{2.083660in}{0.889755in}}%
\pgfpathlineto{\pgfqpoint{2.082412in}{0.890628in}}%
\pgfpathlineto{\pgfqpoint{2.071131in}{0.898500in}}%
\pgfpathlineto{\pgfqpoint{2.067688in}{0.900825in}}%
\pgfpathlineto{\pgfqpoint{2.059850in}{0.906106in}}%
\pgfpathlineto{\pgfqpoint{2.051130in}{0.911894in}}%
\pgfpathlineto{\pgfqpoint{2.048569in}{0.913674in}}%
\pgfpathlineto{\pgfqpoint{2.037289in}{0.921607in}}%
\pgfpathlineto{\pgfqpoint{2.035365in}{0.922964in}}%
\pgfpathlineto{\pgfqpoint{2.026008in}{0.930069in}}%
\pgfpathlineto{\pgfqpoint{2.020631in}{0.934033in}}%
\pgfpathlineto{\pgfqpoint{2.014727in}{0.939472in}}%
\pgfpathlineto{\pgfqpoint{2.009512in}{0.945103in}}%
\pgfpathlineto{\pgfqpoint{2.003446in}{0.952659in}}%
\pgfpathlineto{\pgfqpoint{2.001373in}{0.956173in}}%
\pgfpathlineto{\pgfqpoint{1.995338in}{0.967242in}}%
\pgfpathlineto{\pgfqpoint{1.992165in}{0.974034in}}%
\pgfpathlineto{\pgfqpoint{1.990683in}{0.978312in}}%
\pgfpathlineto{\pgfqpoint{1.987740in}{0.989381in}}%
\pgfpathlineto{\pgfqpoint{1.985798in}{1.000451in}}%
\pgfpathlineto{\pgfqpoint{1.984296in}{1.011520in}}%
\pgfpathlineto{\pgfqpoint{1.984417in}{1.022590in}}%
\pgfpathlineto{\pgfqpoint{1.985450in}{1.033659in}}%
\pgfpathlineto{\pgfqpoint{1.986842in}{1.044729in}}%
\pgfpathlineto{\pgfqpoint{1.987573in}{1.055798in}}%
\pgfpathlineto{\pgfqpoint{1.987018in}{1.066868in}}%
\pgfpathlineto{\pgfqpoint{1.984075in}{1.077937in}}%
\pgfpathlineto{\pgfqpoint{1.980884in}{1.084535in}}%
\pgfpathlineto{\pgfqpoint{1.978667in}{1.089007in}}%
\pgfpathlineto{\pgfqpoint{1.971844in}{1.100077in}}%
\pgfpathlineto{\pgfqpoint{1.969604in}{1.103547in}}%
\pgfpathlineto{\pgfqpoint{1.964544in}{1.111146in}}%
\pgfpathlineto{\pgfqpoint{1.958323in}{1.120251in}}%
\pgfpathlineto{\pgfqpoint{1.956943in}{1.122216in}}%
\pgfpathlineto{\pgfqpoint{1.949050in}{1.133285in}}%
\pgfpathlineto{\pgfqpoint{1.947042in}{1.136046in}}%
\pgfpathlineto{\pgfqpoint{1.940856in}{1.144355in}}%
\pgfpathlineto{\pgfqpoint{1.935761in}{1.151200in}}%
\pgfpathlineto{\pgfqpoint{1.932498in}{1.155424in}}%
\pgfpathlineto{\pgfqpoint{1.924480in}{1.166091in}}%
\pgfpathlineto{\pgfqpoint{1.924167in}{1.166494in}}%
\pgfpathlineto{\pgfqpoint{1.915836in}{1.177563in}}%
\pgfpathlineto{\pgfqpoint{1.913200in}{1.181464in}}%
\pgfpathlineto{\pgfqpoint{1.907982in}{1.188633in}}%
\pgfpathlineto{\pgfqpoint{1.901919in}{1.197561in}}%
\pgfpathlineto{\pgfqpoint{1.900383in}{1.199702in}}%
\pgfpathlineto{\pgfqpoint{1.892970in}{1.210772in}}%
\pgfpathlineto{\pgfqpoint{1.890638in}{1.214288in}}%
\pgfpathlineto{\pgfqpoint{1.885504in}{1.221842in}}%
\pgfpathlineto{\pgfqpoint{1.879357in}{1.230867in}}%
\pgfpathlineto{\pgfqpoint{1.877933in}{1.232911in}}%
\pgfpathlineto{\pgfqpoint{1.870302in}{1.243981in}}%
\pgfpathlineto{\pgfqpoint{1.868076in}{1.247246in}}%
\pgfpathlineto{\pgfqpoint{1.862621in}{1.255050in}}%
\pgfpathlineto{\pgfqpoint{1.856795in}{1.263410in}}%
\pgfpathlineto{\pgfqpoint{1.856795in}{1.255050in}}%
\pgfpathlineto{\pgfqpoint{1.856795in}{1.243981in}}%
\pgfpathlineto{\pgfqpoint{1.856795in}{1.232911in}}%
\pgfpathlineto{\pgfqpoint{1.856795in}{1.221842in}}%
\pgfpathlineto{\pgfqpoint{1.856795in}{1.210772in}}%
\pgfpathlineto{\pgfqpoint{1.856795in}{1.204355in}}%
\pgfpathlineto{\pgfqpoint{1.860799in}{1.199702in}}%
\pgfpathlineto{\pgfqpoint{1.868076in}{1.191241in}}%
\pgfpathlineto{\pgfqpoint{1.870272in}{1.188633in}}%
\pgfpathlineto{\pgfqpoint{1.879357in}{1.177861in}}%
\pgfpathlineto{\pgfqpoint{1.879604in}{1.177563in}}%
\pgfpathlineto{\pgfqpoint{1.888961in}{1.166494in}}%
\pgfpathlineto{\pgfqpoint{1.890638in}{1.164492in}}%
\pgfpathlineto{\pgfqpoint{1.898122in}{1.155424in}}%
\pgfpathlineto{\pgfqpoint{1.901919in}{1.150762in}}%
\pgfpathlineto{\pgfqpoint{1.907052in}{1.144355in}}%
\pgfpathlineto{\pgfqpoint{1.913200in}{1.136546in}}%
\pgfpathlineto{\pgfqpoint{1.915721in}{1.133285in}}%
\pgfpathlineto{\pgfqpoint{1.924134in}{1.122216in}}%
\pgfpathlineto{\pgfqpoint{1.924480in}{1.121758in}}%
\pgfpathlineto{\pgfqpoint{1.932322in}{1.111146in}}%
\pgfpathlineto{\pgfqpoint{1.935761in}{1.106371in}}%
\pgfpathlineto{\pgfqpoint{1.940179in}{1.100077in}}%
\pgfpathlineto{\pgfqpoint{1.947042in}{1.089641in}}%
\pgfpathlineto{\pgfqpoint{1.947449in}{1.089007in}}%
\pgfpathlineto{\pgfqpoint{1.952868in}{1.077937in}}%
\pgfpathlineto{\pgfqpoint{1.954187in}{1.066868in}}%
\pgfpathlineto{\pgfqpoint{1.953824in}{1.055798in}}%
\pgfpathlineto{\pgfqpoint{1.952826in}{1.044729in}}%
\pgfpathlineto{\pgfqpoint{1.951369in}{1.033659in}}%
\pgfpathlineto{\pgfqpoint{1.949573in}{1.022590in}}%
\pgfpathlineto{\pgfqpoint{1.950191in}{1.011520in}}%
\pgfpathlineto{\pgfqpoint{1.952653in}{1.000451in}}%
\pgfpathlineto{\pgfqpoint{1.956020in}{0.989381in}}%
\pgfpathlineto{\pgfqpoint{1.958323in}{0.983105in}}%
\pgfpathlineto{\pgfqpoint{1.960018in}{0.978312in}}%
\pgfpathlineto{\pgfqpoint{1.964254in}{0.967242in}}%
\pgfpathlineto{\pgfqpoint{1.968900in}{0.956173in}}%
\pgfpathlineto{\pgfqpoint{1.969604in}{0.954989in}}%
\pgfpathlineto{\pgfqpoint{1.975121in}{0.945103in}}%
\pgfpathlineto{\pgfqpoint{1.980884in}{0.936375in}}%
\pgfpathlineto{\pgfqpoint{1.982473in}{0.934033in}}%
\pgfpathlineto{\pgfqpoint{1.990892in}{0.922964in}}%
\pgfpathlineto{\pgfqpoint{1.992165in}{0.921272in}}%
\pgfpathlineto{\pgfqpoint{1.999830in}{0.911894in}}%
\pgfpathlineto{\pgfqpoint{2.003446in}{0.907598in}}%
\pgfpathlineto{\pgfqpoint{2.010326in}{0.900825in}}%
\pgfpathlineto{\pgfqpoint{2.014727in}{0.897221in}}%
\pgfpathlineto{\pgfqpoint{2.023660in}{0.889755in}}%
\pgfpathlineto{\pgfqpoint{2.026008in}{0.887925in}}%
\pgfpathlineto{\pgfqpoint{2.037289in}{0.880443in}}%
\pgfpathlineto{\pgfqpoint{2.040298in}{0.878686in}}%
\pgfpathlineto{\pgfqpoint{2.048569in}{0.873967in}}%
\pgfpathlineto{\pgfqpoint{2.059850in}{0.867729in}}%
\pgfpathlineto{\pgfqpoint{2.060048in}{0.867616in}}%
\pgfpathlineto{\pgfqpoint{2.071131in}{0.861634in}}%
\pgfpathlineto{\pgfqpoint{2.080575in}{0.856547in}}%
\pgfpathlineto{\pgfqpoint{2.082412in}{0.855564in}}%
\pgfpathlineto{\pgfqpoint{2.093693in}{0.849606in}}%
\pgfpathlineto{\pgfqpoint{2.101384in}{0.845477in}}%
\pgfpathlineto{\pgfqpoint{2.104973in}{0.843559in}}%
\pgfpathlineto{\pgfqpoint{2.116254in}{0.837653in}}%
\pgfpathlineto{\pgfqpoint{2.122464in}{0.834408in}}%
\pgfpathlineto{\pgfqpoint{2.127535in}{0.831772in}}%
\pgfpathlineto{\pgfqpoint{2.138816in}{0.825850in}}%
\pgfpathlineto{\pgfqpoint{2.143946in}{0.823338in}}%
\pgfpathlineto{\pgfqpoint{2.150097in}{0.820245in}}%
\pgfpathlineto{\pgfqpoint{2.161377in}{0.814804in}}%
\pgfpathlineto{\pgfqpoint{2.166810in}{0.812269in}}%
\pgfpathlineto{\pgfqpoint{2.172658in}{0.809526in}}%
\pgfpathlineto{\pgfqpoint{2.183939in}{0.804223in}}%
\pgfpathlineto{\pgfqpoint{2.190294in}{0.801199in}}%
\pgfpathlineto{\pgfqpoint{2.195220in}{0.798867in}}%
\pgfpathlineto{\pgfqpoint{2.206501in}{0.793422in}}%
\pgfpathlineto{\pgfqpoint{2.213068in}{0.790129in}}%
\pgfpathlineto{\pgfqpoint{2.217782in}{0.787759in}}%
\pgfpathlineto{\pgfqpoint{2.229062in}{0.782227in}}%
\pgfpathlineto{\pgfqpoint{2.235249in}{0.779060in}}%
\pgfpathlineto{\pgfqpoint{2.240343in}{0.776316in}}%
\pgfpathlineto{\pgfqpoint{2.251624in}{0.770383in}}%
\pgfpathlineto{\pgfqpoint{2.256141in}{0.767990in}}%
\pgfpathlineto{\pgfqpoint{2.262905in}{0.764244in}}%
\pgfpathlineto{\pgfqpoint{2.274186in}{0.758139in}}%
\pgfpathlineto{\pgfqpoint{2.276483in}{0.756921in}}%
\pgfpathlineto{\pgfqpoint{2.285466in}{0.752189in}}%
\pgfpathlineto{\pgfqpoint{2.296747in}{0.745946in}}%
\pgfpathlineto{\pgfqpoint{2.296923in}{0.745851in}}%
\pgfpathlineto{\pgfqpoint{2.308028in}{0.740324in}}%
\pgfpathlineto{\pgfqpoint{2.318642in}{0.734782in}}%
\pgfpathlineto{\pgfqpoint{2.319309in}{0.734441in}}%
\pgfpathlineto{\pgfqpoint{2.330590in}{0.728470in}}%
\pgfpathlineto{\pgfqpoint{2.339163in}{0.723712in}}%
\pgfpathlineto{\pgfqpoint{2.341871in}{0.722178in}}%
\pgfpathlineto{\pgfqpoint{2.353151in}{0.715505in}}%
\pgfpathlineto{\pgfqpoint{2.357948in}{0.712643in}}%
\pgfpathlineto{\pgfqpoint{2.364432in}{0.708763in}}%
\pgfpathlineto{\pgfqpoint{2.375713in}{0.701692in}}%
\pgfpathlineto{\pgfqpoint{2.375895in}{0.701573in}}%
\pgfpathlineto{\pgfqpoint{2.386994in}{0.693812in}}%
\pgfpathlineto{\pgfqpoint{2.391536in}{0.690504in}}%
\pgfpathlineto{\pgfqpoint{2.398275in}{0.685171in}}%
\pgfpathlineto{\pgfqpoint{2.405626in}{0.679434in}}%
\pgfpathlineto{\pgfqpoint{2.409555in}{0.676247in}}%
\pgfpathlineto{\pgfqpoint{2.419796in}{0.668364in}}%
\pgfpathlineto{\pgfqpoint{2.420836in}{0.667521in}}%
\pgfpathlineto{\pgfqpoint{2.432117in}{0.658310in}}%
\pgfpathlineto{\pgfqpoint{2.433315in}{0.657295in}}%
\pgfpathlineto{\pgfqpoint{2.443398in}{0.648709in}}%
\pgfpathlineto{\pgfqpoint{2.446228in}{0.646225in}}%
\pgfpathlineto{\pgfqpoint{2.454679in}{0.638717in}}%
\pgfpathlineto{\pgfqpoint{2.458925in}{0.635156in}}%
\pgfpathlineto{\pgfqpoint{2.465960in}{0.628874in}}%
\pgfpathlineto{\pgfqpoint{2.471484in}{0.624086in}}%
\pgfpathlineto{\pgfqpoint{2.477240in}{0.618940in}}%
\pgfpathlineto{\pgfqpoint{2.483687in}{0.613017in}}%
\pgfpathlineto{\pgfqpoint{2.488521in}{0.608483in}}%
\pgfpathlineto{\pgfqpoint{2.495461in}{0.601947in}}%
\pgfpathlineto{\pgfqpoint{2.499802in}{0.597815in}}%
\pgfpathlineto{\pgfqpoint{2.506709in}{0.590878in}}%
\pgfpathlineto{\pgfqpoint{2.511083in}{0.586194in}}%
\pgfpathlineto{\pgfqpoint{2.516874in}{0.579808in}}%
\pgfpathlineto{\pgfqpoint{2.522364in}{0.573557in}}%
\pgfpathlineto{\pgfqpoint{2.526751in}{0.568739in}}%
\pgfpathlineto{\pgfqpoint{2.533644in}{0.560706in}}%
\pgfpathlineto{\pgfqpoint{2.536465in}{0.557669in}}%
\pgfpathlineto{\pgfqpoint{2.544925in}{0.548294in}}%
\pgfpathlineto{\pgfqpoint{2.546435in}{0.546600in}}%
\pgfpathlineto{\pgfqpoint{2.555773in}{0.535530in}}%
\pgfpathlineto{\pgfqpoint{2.556206in}{0.534999in}}%
\pgfpathlineto{\pgfqpoint{2.564479in}{0.524460in}}%
\pgfpathlineto{\pgfqpoint{2.567487in}{0.520491in}}%
\pgfpathlineto{\pgfqpoint{2.573050in}{0.513391in}}%
\pgfpathlineto{\pgfqpoint{2.578768in}{0.505472in}}%
\pgfpathlineto{\pgfqpoint{2.581245in}{0.502321in}}%
\pgfpathlineto{\pgfqpoint{2.589368in}{0.491252in}}%
\pgfpathclose%
\pgfusepath{fill}%
\end{pgfscope}%
\begin{pgfscope}%
\pgfpathrectangle{\pgfqpoint{1.856795in}{0.423750in}}{\pgfqpoint{1.194205in}{1.163386in}}%
\pgfusepath{clip}%
\pgfsetbuttcap%
\pgfsetroundjoin%
\definecolor{currentfill}{rgb}{0.964679,0.682838,0.530002}%
\pgfsetfillcolor{currentfill}%
\pgfsetlinewidth{0.000000pt}%
\definecolor{currentstroke}{rgb}{0.000000,0.000000,0.000000}%
\pgfsetstrokecolor{currentstroke}%
\pgfsetdash{}{0pt}%
\pgfpathmoveto{\pgfqpoint{2.669014in}{0.911174in}}%
\pgfpathlineto{\pgfqpoint{2.680295in}{0.911219in}}%
\pgfpathlineto{\pgfqpoint{2.691576in}{0.911749in}}%
\pgfpathlineto{\pgfqpoint{2.693924in}{0.911894in}}%
\pgfpathlineto{\pgfqpoint{2.702857in}{0.912388in}}%
\pgfpathlineto{\pgfqpoint{2.714137in}{0.913826in}}%
\pgfpathlineto{\pgfqpoint{2.725418in}{0.916335in}}%
\pgfpathlineto{\pgfqpoint{2.736699in}{0.920652in}}%
\pgfpathlineto{\pgfqpoint{2.741653in}{0.922964in}}%
\pgfpathlineto{\pgfqpoint{2.747980in}{0.925267in}}%
\pgfpathlineto{\pgfqpoint{2.759261in}{0.930087in}}%
\pgfpathlineto{\pgfqpoint{2.766102in}{0.934033in}}%
\pgfpathlineto{\pgfqpoint{2.770542in}{0.936839in}}%
\pgfpathlineto{\pgfqpoint{2.781059in}{0.945103in}}%
\pgfpathlineto{\pgfqpoint{2.781822in}{0.945703in}}%
\pgfpathlineto{\pgfqpoint{2.793103in}{0.955232in}}%
\pgfpathlineto{\pgfqpoint{2.794403in}{0.956173in}}%
\pgfpathlineto{\pgfqpoint{2.804384in}{0.963493in}}%
\pgfpathlineto{\pgfqpoint{2.809428in}{0.967242in}}%
\pgfpathlineto{\pgfqpoint{2.815665in}{0.971846in}}%
\pgfpathlineto{\pgfqpoint{2.824187in}{0.978312in}}%
\pgfpathlineto{\pgfqpoint{2.826946in}{0.980493in}}%
\pgfpathlineto{\pgfqpoint{2.838226in}{0.989305in}}%
\pgfpathlineto{\pgfqpoint{2.838323in}{0.989381in}}%
\pgfpathlineto{\pgfqpoint{2.849507in}{0.998101in}}%
\pgfpathlineto{\pgfqpoint{2.852420in}{1.000451in}}%
\pgfpathlineto{\pgfqpoint{2.860788in}{1.006982in}}%
\pgfpathlineto{\pgfqpoint{2.866472in}{1.011520in}}%
\pgfpathlineto{\pgfqpoint{2.872069in}{1.016243in}}%
\pgfpathlineto{\pgfqpoint{2.879392in}{1.022590in}}%
\pgfpathlineto{\pgfqpoint{2.883350in}{1.026093in}}%
\pgfpathlineto{\pgfqpoint{2.891880in}{1.033659in}}%
\pgfpathlineto{\pgfqpoint{2.894631in}{1.036024in}}%
\pgfpathlineto{\pgfqpoint{2.904384in}{1.044729in}}%
\pgfpathlineto{\pgfqpoint{2.905911in}{1.046103in}}%
\pgfpathlineto{\pgfqpoint{2.916907in}{1.055798in}}%
\pgfpathlineto{\pgfqpoint{2.917192in}{1.056073in}}%
\pgfpathlineto{\pgfqpoint{2.928473in}{1.066722in}}%
\pgfpathlineto{\pgfqpoint{2.928624in}{1.066868in}}%
\pgfpathlineto{\pgfqpoint{2.939272in}{1.077937in}}%
\pgfpathlineto{\pgfqpoint{2.939754in}{1.078428in}}%
\pgfpathlineto{\pgfqpoint{2.949812in}{1.089007in}}%
\pgfpathlineto{\pgfqpoint{2.951035in}{1.090247in}}%
\pgfpathlineto{\pgfqpoint{2.960343in}{1.100077in}}%
\pgfpathlineto{\pgfqpoint{2.962315in}{1.102181in}}%
\pgfpathlineto{\pgfqpoint{2.970485in}{1.111146in}}%
\pgfpathlineto{\pgfqpoint{2.973596in}{1.114552in}}%
\pgfpathlineto{\pgfqpoint{2.973596in}{1.122216in}}%
\pgfpathlineto{\pgfqpoint{2.973596in}{1.133285in}}%
\pgfpathlineto{\pgfqpoint{2.973596in}{1.144355in}}%
\pgfpathlineto{\pgfqpoint{2.973596in}{1.155424in}}%
\pgfpathlineto{\pgfqpoint{2.973596in}{1.166494in}}%
\pgfpathlineto{\pgfqpoint{2.973596in}{1.177563in}}%
\pgfpathlineto{\pgfqpoint{2.973596in}{1.188633in}}%
\pgfpathlineto{\pgfqpoint{2.973596in}{1.189179in}}%
\pgfpathlineto{\pgfqpoint{2.973139in}{1.188633in}}%
\pgfpathlineto{\pgfqpoint{2.963275in}{1.177563in}}%
\pgfpathlineto{\pgfqpoint{2.962315in}{1.176503in}}%
\pgfpathlineto{\pgfqpoint{2.953632in}{1.166494in}}%
\pgfpathlineto{\pgfqpoint{2.951035in}{1.163714in}}%
\pgfpathlineto{\pgfqpoint{2.943386in}{1.155424in}}%
\pgfpathlineto{\pgfqpoint{2.939754in}{1.151589in}}%
\pgfpathlineto{\pgfqpoint{2.933021in}{1.144355in}}%
\pgfpathlineto{\pgfqpoint{2.928473in}{1.139620in}}%
\pgfpathlineto{\pgfqpoint{2.922231in}{1.133285in}}%
\pgfpathlineto{\pgfqpoint{2.917192in}{1.128342in}}%
\pgfpathlineto{\pgfqpoint{2.910793in}{1.122216in}}%
\pgfpathlineto{\pgfqpoint{2.905911in}{1.117405in}}%
\pgfpathlineto{\pgfqpoint{2.899224in}{1.111146in}}%
\pgfpathlineto{\pgfqpoint{2.894631in}{1.106392in}}%
\pgfpathlineto{\pgfqpoint{2.888372in}{1.100077in}}%
\pgfpathlineto{\pgfqpoint{2.883350in}{1.095010in}}%
\pgfpathlineto{\pgfqpoint{2.877250in}{1.089007in}}%
\pgfpathlineto{\pgfqpoint{2.872069in}{1.084051in}}%
\pgfpathlineto{\pgfqpoint{2.865506in}{1.077937in}}%
\pgfpathlineto{\pgfqpoint{2.860788in}{1.073843in}}%
\pgfpathlineto{\pgfqpoint{2.852898in}{1.066868in}}%
\pgfpathlineto{\pgfqpoint{2.849507in}{1.063970in}}%
\pgfpathlineto{\pgfqpoint{2.840052in}{1.055798in}}%
\pgfpathlineto{\pgfqpoint{2.838226in}{1.054205in}}%
\pgfpathlineto{\pgfqpoint{2.827266in}{1.044729in}}%
\pgfpathlineto{\pgfqpoint{2.826946in}{1.044461in}}%
\pgfpathlineto{\pgfqpoint{2.815665in}{1.035147in}}%
\pgfpathlineto{\pgfqpoint{2.813840in}{1.033659in}}%
\pgfpathlineto{\pgfqpoint{2.804384in}{1.026462in}}%
\pgfpathlineto{\pgfqpoint{2.799146in}{1.022590in}}%
\pgfpathlineto{\pgfqpoint{2.793103in}{1.018158in}}%
\pgfpathlineto{\pgfqpoint{2.785307in}{1.011520in}}%
\pgfpathlineto{\pgfqpoint{2.781822in}{1.008580in}}%
\pgfpathlineto{\pgfqpoint{2.771609in}{1.000451in}}%
\pgfpathlineto{\pgfqpoint{2.770542in}{0.999538in}}%
\pgfpathlineto{\pgfqpoint{2.759261in}{0.991144in}}%
\pgfpathlineto{\pgfqpoint{2.756675in}{0.989381in}}%
\pgfpathlineto{\pgfqpoint{2.747980in}{0.983122in}}%
\pgfpathlineto{\pgfqpoint{2.740467in}{0.978312in}}%
\pgfpathlineto{\pgfqpoint{2.736699in}{0.976031in}}%
\pgfpathlineto{\pgfqpoint{2.725418in}{0.970478in}}%
\pgfpathlineto{\pgfqpoint{2.714533in}{0.967242in}}%
\pgfpathlineto{\pgfqpoint{2.714137in}{0.967118in}}%
\pgfpathlineto{\pgfqpoint{2.702857in}{0.964383in}}%
\pgfpathlineto{\pgfqpoint{2.691576in}{0.963120in}}%
\pgfpathlineto{\pgfqpoint{2.680295in}{0.962389in}}%
\pgfpathlineto{\pgfqpoint{2.669014in}{0.963987in}}%
\pgfpathlineto{\pgfqpoint{2.657733in}{0.966102in}}%
\pgfpathlineto{\pgfqpoint{2.653070in}{0.967242in}}%
\pgfpathlineto{\pgfqpoint{2.646453in}{0.968967in}}%
\pgfpathlineto{\pgfqpoint{2.635172in}{0.972590in}}%
\pgfpathlineto{\pgfqpoint{2.623891in}{0.976664in}}%
\pgfpathlineto{\pgfqpoint{2.619478in}{0.978312in}}%
\pgfpathlineto{\pgfqpoint{2.612610in}{0.981032in}}%
\pgfpathlineto{\pgfqpoint{2.601329in}{0.986002in}}%
\pgfpathlineto{\pgfqpoint{2.594209in}{0.989381in}}%
\pgfpathlineto{\pgfqpoint{2.590048in}{0.991511in}}%
\pgfpathlineto{\pgfqpoint{2.578768in}{0.997962in}}%
\pgfpathlineto{\pgfqpoint{2.574632in}{1.000451in}}%
\pgfpathlineto{\pgfqpoint{2.567487in}{1.005091in}}%
\pgfpathlineto{\pgfqpoint{2.558034in}{1.011520in}}%
\pgfpathlineto{\pgfqpoint{2.556206in}{1.012816in}}%
\pgfpathlineto{\pgfqpoint{2.544925in}{1.021236in}}%
\pgfpathlineto{\pgfqpoint{2.543149in}{1.022590in}}%
\pgfpathlineto{\pgfqpoint{2.533644in}{1.029888in}}%
\pgfpathlineto{\pgfqpoint{2.528743in}{1.033659in}}%
\pgfpathlineto{\pgfqpoint{2.522364in}{1.038147in}}%
\pgfpathlineto{\pgfqpoint{2.513504in}{1.044729in}}%
\pgfpathlineto{\pgfqpoint{2.511083in}{1.046461in}}%
\pgfpathlineto{\pgfqpoint{2.499802in}{1.054524in}}%
\pgfpathlineto{\pgfqpoint{2.498079in}{1.055798in}}%
\pgfpathlineto{\pgfqpoint{2.488521in}{1.062952in}}%
\pgfpathlineto{\pgfqpoint{2.483445in}{1.066868in}}%
\pgfpathlineto{\pgfqpoint{2.477240in}{1.071602in}}%
\pgfpathlineto{\pgfqpoint{2.469105in}{1.077937in}}%
\pgfpathlineto{\pgfqpoint{2.465960in}{1.080371in}}%
\pgfpathlineto{\pgfqpoint{2.455097in}{1.089007in}}%
\pgfpathlineto{\pgfqpoint{2.454679in}{1.089348in}}%
\pgfpathlineto{\pgfqpoint{2.443398in}{1.098544in}}%
\pgfpathlineto{\pgfqpoint{2.441538in}{1.100077in}}%
\pgfpathlineto{\pgfqpoint{2.432117in}{1.107543in}}%
\pgfpathlineto{\pgfqpoint{2.427596in}{1.111146in}}%
\pgfpathlineto{\pgfqpoint{2.420836in}{1.116296in}}%
\pgfpathlineto{\pgfqpoint{2.413121in}{1.122216in}}%
\pgfpathlineto{\pgfqpoint{2.409555in}{1.125049in}}%
\pgfpathlineto{\pgfqpoint{2.399253in}{1.133285in}}%
\pgfpathlineto{\pgfqpoint{2.398275in}{1.134085in}}%
\pgfpathlineto{\pgfqpoint{2.386994in}{1.143243in}}%
\pgfpathlineto{\pgfqpoint{2.385590in}{1.144355in}}%
\pgfpathlineto{\pgfqpoint{2.375713in}{1.152282in}}%
\pgfpathlineto{\pgfqpoint{2.371781in}{1.155424in}}%
\pgfpathlineto{\pgfqpoint{2.364432in}{1.161482in}}%
\pgfpathlineto{\pgfqpoint{2.358191in}{1.166494in}}%
\pgfpathlineto{\pgfqpoint{2.353151in}{1.170712in}}%
\pgfpathlineto{\pgfqpoint{2.344996in}{1.177563in}}%
\pgfpathlineto{\pgfqpoint{2.341871in}{1.180222in}}%
\pgfpathlineto{\pgfqpoint{2.332046in}{1.188633in}}%
\pgfpathlineto{\pgfqpoint{2.330590in}{1.189885in}}%
\pgfpathlineto{\pgfqpoint{2.319309in}{1.199495in}}%
\pgfpathlineto{\pgfqpoint{2.319064in}{1.199702in}}%
\pgfpathlineto{\pgfqpoint{2.308028in}{1.209088in}}%
\pgfpathlineto{\pgfqpoint{2.306062in}{1.210772in}}%
\pgfpathlineto{\pgfqpoint{2.296747in}{1.218865in}}%
\pgfpathlineto{\pgfqpoint{2.293407in}{1.221842in}}%
\pgfpathlineto{\pgfqpoint{2.285466in}{1.229460in}}%
\pgfpathlineto{\pgfqpoint{2.281770in}{1.232911in}}%
\pgfpathlineto{\pgfqpoint{2.274186in}{1.240175in}}%
\pgfpathlineto{\pgfqpoint{2.270475in}{1.243981in}}%
\pgfpathlineto{\pgfqpoint{2.262905in}{1.252027in}}%
\pgfpathlineto{\pgfqpoint{2.260126in}{1.255050in}}%
\pgfpathlineto{\pgfqpoint{2.251624in}{1.264586in}}%
\pgfpathlineto{\pgfqpoint{2.250252in}{1.266120in}}%
\pgfpathlineto{\pgfqpoint{2.240846in}{1.277189in}}%
\pgfpathlineto{\pgfqpoint{2.240343in}{1.277788in}}%
\pgfpathlineto{\pgfqpoint{2.231977in}{1.288259in}}%
\pgfpathlineto{\pgfqpoint{2.229062in}{1.292137in}}%
\pgfpathlineto{\pgfqpoint{2.223948in}{1.299328in}}%
\pgfpathlineto{\pgfqpoint{2.217782in}{1.308264in}}%
\pgfpathlineto{\pgfqpoint{2.216332in}{1.310398in}}%
\pgfpathlineto{\pgfqpoint{2.208911in}{1.321467in}}%
\pgfpathlineto{\pgfqpoint{2.206501in}{1.325116in}}%
\pgfpathlineto{\pgfqpoint{2.201665in}{1.332537in}}%
\pgfpathlineto{\pgfqpoint{2.195220in}{1.342487in}}%
\pgfpathlineto{\pgfqpoint{2.194508in}{1.343606in}}%
\pgfpathlineto{\pgfqpoint{2.187481in}{1.354676in}}%
\pgfpathlineto{\pgfqpoint{2.183939in}{1.360281in}}%
\pgfpathlineto{\pgfqpoint{2.180564in}{1.365746in}}%
\pgfpathlineto{\pgfqpoint{2.173503in}{1.376815in}}%
\pgfpathlineto{\pgfqpoint{2.172658in}{1.377984in}}%
\pgfpathlineto{\pgfqpoint{2.165966in}{1.387885in}}%
\pgfpathlineto{\pgfqpoint{2.161377in}{1.394322in}}%
\pgfpathlineto{\pgfqpoint{2.158070in}{1.398954in}}%
\pgfpathlineto{\pgfqpoint{2.150201in}{1.410024in}}%
\pgfpathlineto{\pgfqpoint{2.150097in}{1.410174in}}%
\pgfpathlineto{\pgfqpoint{2.142488in}{1.421093in}}%
\pgfpathlineto{\pgfqpoint{2.138816in}{1.426421in}}%
\pgfpathlineto{\pgfqpoint{2.134871in}{1.432163in}}%
\pgfpathlineto{\pgfqpoint{2.127535in}{1.442884in}}%
\pgfpathlineto{\pgfqpoint{2.127299in}{1.443232in}}%
\pgfpathlineto{\pgfqpoint{2.119614in}{1.454302in}}%
\pgfpathlineto{\pgfqpoint{2.116254in}{1.459122in}}%
\pgfpathlineto{\pgfqpoint{2.111909in}{1.465371in}}%
\pgfpathlineto{\pgfqpoint{2.104973in}{1.475461in}}%
\pgfpathlineto{\pgfqpoint{2.104300in}{1.476441in}}%
\pgfpathlineto{\pgfqpoint{2.096815in}{1.487510in}}%
\pgfpathlineto{\pgfqpoint{2.093693in}{1.492235in}}%
\pgfpathlineto{\pgfqpoint{2.089488in}{1.498580in}}%
\pgfpathlineto{\pgfqpoint{2.082412in}{1.509280in}}%
\pgfpathlineto{\pgfqpoint{2.082170in}{1.509650in}}%
\pgfpathlineto{\pgfqpoint{2.074471in}{1.520719in}}%
\pgfpathlineto{\pgfqpoint{2.071131in}{1.525358in}}%
\pgfpathlineto{\pgfqpoint{2.066496in}{1.531789in}}%
\pgfpathlineto{\pgfqpoint{2.059850in}{1.541172in}}%
\pgfpathlineto{\pgfqpoint{2.058658in}{1.542858in}}%
\pgfpathlineto{\pgfqpoint{2.050755in}{1.553928in}}%
\pgfpathlineto{\pgfqpoint{2.048569in}{1.556842in}}%
\pgfpathlineto{\pgfqpoint{2.042527in}{1.564997in}}%
\pgfpathlineto{\pgfqpoint{2.037289in}{1.572067in}}%
\pgfpathlineto{\pgfqpoint{2.034326in}{1.576067in}}%
\pgfpathlineto{\pgfqpoint{2.026300in}{1.587136in}}%
\pgfpathlineto{\pgfqpoint{2.026008in}{1.587136in}}%
\pgfpathlineto{\pgfqpoint{2.014727in}{1.587136in}}%
\pgfpathlineto{\pgfqpoint{2.003446in}{1.587136in}}%
\pgfpathlineto{\pgfqpoint{1.992165in}{1.587136in}}%
\pgfpathlineto{\pgfqpoint{1.980884in}{1.587136in}}%
\pgfpathlineto{\pgfqpoint{1.972974in}{1.587136in}}%
\pgfpathlineto{\pgfqpoint{1.980884in}{1.576472in}}%
\pgfpathlineto{\pgfqpoint{1.981183in}{1.576067in}}%
\pgfpathlineto{\pgfqpoint{1.989169in}{1.564997in}}%
\pgfpathlineto{\pgfqpoint{1.992165in}{1.560811in}}%
\pgfpathlineto{\pgfqpoint{1.997021in}{1.553928in}}%
\pgfpathlineto{\pgfqpoint{2.003446in}{1.544848in}}%
\pgfpathlineto{\pgfqpoint{2.004839in}{1.542858in}}%
\pgfpathlineto{\pgfqpoint{2.012504in}{1.531789in}}%
\pgfpathlineto{\pgfqpoint{2.014727in}{1.528355in}}%
\pgfpathlineto{\pgfqpoint{2.019718in}{1.520719in}}%
\pgfpathlineto{\pgfqpoint{2.026008in}{1.511015in}}%
\pgfpathlineto{\pgfqpoint{2.026892in}{1.509650in}}%
\pgfpathlineto{\pgfqpoint{2.034021in}{1.498580in}}%
\pgfpathlineto{\pgfqpoint{2.037289in}{1.493566in}}%
\pgfpathlineto{\pgfqpoint{2.041214in}{1.487510in}}%
\pgfpathlineto{\pgfqpoint{2.048569in}{1.476441in}}%
\pgfpathlineto{\pgfqpoint{2.048569in}{1.476440in}}%
\pgfpathlineto{\pgfqpoint{2.055992in}{1.465371in}}%
\pgfpathlineto{\pgfqpoint{2.059850in}{1.459577in}}%
\pgfpathlineto{\pgfqpoint{2.063364in}{1.454302in}}%
\pgfpathlineto{\pgfqpoint{2.070107in}{1.443232in}}%
\pgfpathlineto{\pgfqpoint{2.071131in}{1.441413in}}%
\pgfpathlineto{\pgfqpoint{2.076353in}{1.432163in}}%
\pgfpathlineto{\pgfqpoint{2.082412in}{1.421479in}}%
\pgfpathlineto{\pgfqpoint{2.082631in}{1.421093in}}%
\pgfpathlineto{\pgfqpoint{2.088971in}{1.410024in}}%
\pgfpathlineto{\pgfqpoint{2.093693in}{1.401896in}}%
\pgfpathlineto{\pgfqpoint{2.095401in}{1.398954in}}%
\pgfpathlineto{\pgfqpoint{2.101918in}{1.387885in}}%
\pgfpathlineto{\pgfqpoint{2.104973in}{1.382751in}}%
\pgfpathlineto{\pgfqpoint{2.108527in}{1.376815in}}%
\pgfpathlineto{\pgfqpoint{2.115167in}{1.365746in}}%
\pgfpathlineto{\pgfqpoint{2.116254in}{1.363934in}}%
\pgfpathlineto{\pgfqpoint{2.121814in}{1.354676in}}%
\pgfpathlineto{\pgfqpoint{2.127535in}{1.345130in}}%
\pgfpathlineto{\pgfqpoint{2.128446in}{1.343606in}}%
\pgfpathlineto{\pgfqpoint{2.134740in}{1.332537in}}%
\pgfpathlineto{\pgfqpoint{2.138816in}{1.325357in}}%
\pgfpathlineto{\pgfqpoint{2.141030in}{1.321467in}}%
\pgfpathlineto{\pgfqpoint{2.147390in}{1.310398in}}%
\pgfpathlineto{\pgfqpoint{2.150097in}{1.305828in}}%
\pgfpathlineto{\pgfqpoint{2.153941in}{1.299328in}}%
\pgfpathlineto{\pgfqpoint{2.160778in}{1.288259in}}%
\pgfpathlineto{\pgfqpoint{2.161377in}{1.287431in}}%
\pgfpathlineto{\pgfqpoint{2.168707in}{1.277189in}}%
\pgfpathlineto{\pgfqpoint{2.172658in}{1.271924in}}%
\pgfpathlineto{\pgfqpoint{2.176894in}{1.266120in}}%
\pgfpathlineto{\pgfqpoint{2.183939in}{1.257403in}}%
\pgfpathlineto{\pgfqpoint{2.185785in}{1.255050in}}%
\pgfpathlineto{\pgfqpoint{2.194949in}{1.243981in}}%
\pgfpathlineto{\pgfqpoint{2.195220in}{1.243658in}}%
\pgfpathlineto{\pgfqpoint{2.204382in}{1.232911in}}%
\pgfpathlineto{\pgfqpoint{2.206501in}{1.230425in}}%
\pgfpathlineto{\pgfqpoint{2.214627in}{1.221842in}}%
\pgfpathlineto{\pgfqpoint{2.217782in}{1.218625in}}%
\pgfpathlineto{\pgfqpoint{2.225982in}{1.210772in}}%
\pgfpathlineto{\pgfqpoint{2.229062in}{1.207849in}}%
\pgfpathlineto{\pgfqpoint{2.237729in}{1.199702in}}%
\pgfpathlineto{\pgfqpoint{2.240343in}{1.197354in}}%
\pgfpathlineto{\pgfqpoint{2.250485in}{1.188633in}}%
\pgfpathlineto{\pgfqpoint{2.251624in}{1.187651in}}%
\pgfpathlineto{\pgfqpoint{2.262905in}{1.178102in}}%
\pgfpathlineto{\pgfqpoint{2.263554in}{1.177563in}}%
\pgfpathlineto{\pgfqpoint{2.274186in}{1.168725in}}%
\pgfpathlineto{\pgfqpoint{2.276863in}{1.166494in}}%
\pgfpathlineto{\pgfqpoint{2.285466in}{1.159423in}}%
\pgfpathlineto{\pgfqpoint{2.290319in}{1.155424in}}%
\pgfpathlineto{\pgfqpoint{2.296747in}{1.150201in}}%
\pgfpathlineto{\pgfqpoint{2.303923in}{1.144355in}}%
\pgfpathlineto{\pgfqpoint{2.308028in}{1.141057in}}%
\pgfpathlineto{\pgfqpoint{2.317675in}{1.133285in}}%
\pgfpathlineto{\pgfqpoint{2.319309in}{1.131998in}}%
\pgfpathlineto{\pgfqpoint{2.330590in}{1.123238in}}%
\pgfpathlineto{\pgfqpoint{2.331928in}{1.122216in}}%
\pgfpathlineto{\pgfqpoint{2.341871in}{1.114575in}}%
\pgfpathlineto{\pgfqpoint{2.346519in}{1.111146in}}%
\pgfpathlineto{\pgfqpoint{2.353151in}{1.105959in}}%
\pgfpathlineto{\pgfqpoint{2.360621in}{1.100077in}}%
\pgfpathlineto{\pgfqpoint{2.364432in}{1.096879in}}%
\pgfpathlineto{\pgfqpoint{2.374170in}{1.089007in}}%
\pgfpathlineto{\pgfqpoint{2.375713in}{1.087664in}}%
\pgfpathlineto{\pgfqpoint{2.386614in}{1.077937in}}%
\pgfpathlineto{\pgfqpoint{2.386994in}{1.077599in}}%
\pgfpathlineto{\pgfqpoint{2.398275in}{1.067412in}}%
\pgfpathlineto{\pgfqpoint{2.398877in}{1.066868in}}%
\pgfpathlineto{\pgfqpoint{2.409555in}{1.057184in}}%
\pgfpathlineto{\pgfqpoint{2.411087in}{1.055798in}}%
\pgfpathlineto{\pgfqpoint{2.420836in}{1.046797in}}%
\pgfpathlineto{\pgfqpoint{2.423128in}{1.044729in}}%
\pgfpathlineto{\pgfqpoint{2.432117in}{1.036775in}}%
\pgfpathlineto{\pgfqpoint{2.435604in}{1.033659in}}%
\pgfpathlineto{\pgfqpoint{2.443398in}{1.027062in}}%
\pgfpathlineto{\pgfqpoint{2.448779in}{1.022590in}}%
\pgfpathlineto{\pgfqpoint{2.454679in}{1.017788in}}%
\pgfpathlineto{\pgfqpoint{2.462979in}{1.011520in}}%
\pgfpathlineto{\pgfqpoint{2.465960in}{1.009138in}}%
\pgfpathlineto{\pgfqpoint{2.477240in}{1.000550in}}%
\pgfpathlineto{\pgfqpoint{2.477371in}{1.000451in}}%
\pgfpathlineto{\pgfqpoint{2.488521in}{0.991808in}}%
\pgfpathlineto{\pgfqpoint{2.491722in}{0.989381in}}%
\pgfpathlineto{\pgfqpoint{2.499802in}{0.983195in}}%
\pgfpathlineto{\pgfqpoint{2.506846in}{0.978312in}}%
\pgfpathlineto{\pgfqpoint{2.511083in}{0.975658in}}%
\pgfpathlineto{\pgfqpoint{2.522364in}{0.968449in}}%
\pgfpathlineto{\pgfqpoint{2.524378in}{0.967242in}}%
\pgfpathlineto{\pgfqpoint{2.533644in}{0.961678in}}%
\pgfpathlineto{\pgfqpoint{2.543705in}{0.956173in}}%
\pgfpathlineto{\pgfqpoint{2.544925in}{0.955508in}}%
\pgfpathlineto{\pgfqpoint{2.556206in}{0.949427in}}%
\pgfpathlineto{\pgfqpoint{2.565294in}{0.945103in}}%
\pgfpathlineto{\pgfqpoint{2.567487in}{0.944011in}}%
\pgfpathlineto{\pgfqpoint{2.578768in}{0.938763in}}%
\pgfpathlineto{\pgfqpoint{2.590048in}{0.934489in}}%
\pgfpathlineto{\pgfqpoint{2.591186in}{0.934033in}}%
\pgfpathlineto{\pgfqpoint{2.601329in}{0.929853in}}%
\pgfpathlineto{\pgfqpoint{2.612610in}{0.926033in}}%
\pgfpathlineto{\pgfqpoint{2.622329in}{0.922964in}}%
\pgfpathlineto{\pgfqpoint{2.623891in}{0.922404in}}%
\pgfpathlineto{\pgfqpoint{2.635172in}{0.918760in}}%
\pgfpathlineto{\pgfqpoint{2.646453in}{0.915281in}}%
\pgfpathlineto{\pgfqpoint{2.657733in}{0.912714in}}%
\pgfpathlineto{\pgfqpoint{2.663990in}{0.911894in}}%
\pgfpathclose%
\pgfusepath{fill}%
\end{pgfscope}%
\begin{pgfscope}%
\pgfpathrectangle{\pgfqpoint{1.856795in}{0.423750in}}{\pgfqpoint{1.194205in}{1.163386in}}%
\pgfusepath{clip}%
\pgfsetbuttcap%
\pgfsetroundjoin%
\definecolor{currentfill}{rgb}{0.966328,0.750560,0.616961}%
\pgfsetfillcolor{currentfill}%
\pgfsetlinewidth{0.000000pt}%
\definecolor{currentstroke}{rgb}{0.000000,0.000000,0.000000}%
\pgfsetstrokecolor{currentstroke}%
\pgfsetdash{}{0pt}%
\pgfpathmoveto{\pgfqpoint{2.657733in}{0.491252in}}%
\pgfpathlineto{\pgfqpoint{2.669014in}{0.491252in}}%
\pgfpathlineto{\pgfqpoint{2.680295in}{0.491252in}}%
\pgfpathlineto{\pgfqpoint{2.691576in}{0.491252in}}%
\pgfpathlineto{\pgfqpoint{2.702857in}{0.491252in}}%
\pgfpathlineto{\pgfqpoint{2.714137in}{0.491252in}}%
\pgfpathlineto{\pgfqpoint{2.725418in}{0.491252in}}%
\pgfpathlineto{\pgfqpoint{2.736699in}{0.491252in}}%
\pgfpathlineto{\pgfqpoint{2.746410in}{0.491252in}}%
\pgfpathlineto{\pgfqpoint{2.738958in}{0.502321in}}%
\pgfpathlineto{\pgfqpoint{2.736699in}{0.505277in}}%
\pgfpathlineto{\pgfqpoint{2.730215in}{0.513391in}}%
\pgfpathlineto{\pgfqpoint{2.725418in}{0.519007in}}%
\pgfpathlineto{\pgfqpoint{2.720745in}{0.524460in}}%
\pgfpathlineto{\pgfqpoint{2.714137in}{0.532228in}}%
\pgfpathlineto{\pgfqpoint{2.711466in}{0.535530in}}%
\pgfpathlineto{\pgfqpoint{2.702857in}{0.545679in}}%
\pgfpathlineto{\pgfqpoint{2.702133in}{0.546600in}}%
\pgfpathlineto{\pgfqpoint{2.692955in}{0.557669in}}%
\pgfpathlineto{\pgfqpoint{2.691576in}{0.559280in}}%
\pgfpathlineto{\pgfqpoint{2.683673in}{0.568739in}}%
\pgfpathlineto{\pgfqpoint{2.680295in}{0.572337in}}%
\pgfpathlineto{\pgfqpoint{2.673859in}{0.579808in}}%
\pgfpathlineto{\pgfqpoint{2.669014in}{0.585302in}}%
\pgfpathlineto{\pgfqpoint{2.663935in}{0.590878in}}%
\pgfpathlineto{\pgfqpoint{2.657733in}{0.597083in}}%
\pgfpathlineto{\pgfqpoint{2.652527in}{0.601947in}}%
\pgfpathlineto{\pgfqpoint{2.646453in}{0.607454in}}%
\pgfpathlineto{\pgfqpoint{2.640484in}{0.613017in}}%
\pgfpathlineto{\pgfqpoint{2.635172in}{0.618412in}}%
\pgfpathlineto{\pgfqpoint{2.629059in}{0.624086in}}%
\pgfpathlineto{\pgfqpoint{2.623891in}{0.629274in}}%
\pgfpathlineto{\pgfqpoint{2.617614in}{0.635156in}}%
\pgfpathlineto{\pgfqpoint{2.612610in}{0.639789in}}%
\pgfpathlineto{\pgfqpoint{2.605148in}{0.646225in}}%
\pgfpathlineto{\pgfqpoint{2.601329in}{0.649461in}}%
\pgfpathlineto{\pgfqpoint{2.590816in}{0.657295in}}%
\pgfpathlineto{\pgfqpoint{2.590048in}{0.657831in}}%
\pgfpathlineto{\pgfqpoint{2.578768in}{0.665424in}}%
\pgfpathlineto{\pgfqpoint{2.574167in}{0.668364in}}%
\pgfpathlineto{\pgfqpoint{2.567487in}{0.672411in}}%
\pgfpathlineto{\pgfqpoint{2.556555in}{0.679434in}}%
\pgfpathlineto{\pgfqpoint{2.556206in}{0.679681in}}%
\pgfpathlineto{\pgfqpoint{2.544925in}{0.687806in}}%
\pgfpathlineto{\pgfqpoint{2.541542in}{0.690504in}}%
\pgfpathlineto{\pgfqpoint{2.533644in}{0.696815in}}%
\pgfpathlineto{\pgfqpoint{2.527840in}{0.701573in}}%
\pgfpathlineto{\pgfqpoint{2.522364in}{0.705573in}}%
\pgfpathlineto{\pgfqpoint{2.513410in}{0.712643in}}%
\pgfpathlineto{\pgfqpoint{2.511083in}{0.714336in}}%
\pgfpathlineto{\pgfqpoint{2.499802in}{0.721674in}}%
\pgfpathlineto{\pgfqpoint{2.496459in}{0.723712in}}%
\pgfpathlineto{\pgfqpoint{2.488521in}{0.728211in}}%
\pgfpathlineto{\pgfqpoint{2.477240in}{0.734467in}}%
\pgfpathlineto{\pgfqpoint{2.476671in}{0.734782in}}%
\pgfpathlineto{\pgfqpoint{2.465960in}{0.740476in}}%
\pgfpathlineto{\pgfqpoint{2.457358in}{0.745851in}}%
\pgfpathlineto{\pgfqpoint{2.454679in}{0.747486in}}%
\pgfpathlineto{\pgfqpoint{2.443398in}{0.754537in}}%
\pgfpathlineto{\pgfqpoint{2.439246in}{0.756921in}}%
\pgfpathlineto{\pgfqpoint{2.432117in}{0.761078in}}%
\pgfpathlineto{\pgfqpoint{2.420836in}{0.766976in}}%
\pgfpathlineto{\pgfqpoint{2.419029in}{0.767990in}}%
\pgfpathlineto{\pgfqpoint{2.409555in}{0.773475in}}%
\pgfpathlineto{\pgfqpoint{2.399469in}{0.779060in}}%
\pgfpathlineto{\pgfqpoint{2.398275in}{0.779786in}}%
\pgfpathlineto{\pgfqpoint{2.386994in}{0.785885in}}%
\pgfpathlineto{\pgfqpoint{2.378447in}{0.790129in}}%
\pgfpathlineto{\pgfqpoint{2.375713in}{0.791540in}}%
\pgfpathlineto{\pgfqpoint{2.364432in}{0.797122in}}%
\pgfpathlineto{\pgfqpoint{2.354876in}{0.801199in}}%
\pgfpathlineto{\pgfqpoint{2.353151in}{0.801964in}}%
\pgfpathlineto{\pgfqpoint{2.341871in}{0.806988in}}%
\pgfpathlineto{\pgfqpoint{2.330590in}{0.811516in}}%
\pgfpathlineto{\pgfqpoint{2.328619in}{0.812269in}}%
\pgfpathlineto{\pgfqpoint{2.319309in}{0.815893in}}%
\pgfpathlineto{\pgfqpoint{2.308028in}{0.820692in}}%
\pgfpathlineto{\pgfqpoint{2.301816in}{0.823338in}}%
\pgfpathlineto{\pgfqpoint{2.296747in}{0.825950in}}%
\pgfpathlineto{\pgfqpoint{2.285466in}{0.830895in}}%
\pgfpathlineto{\pgfqpoint{2.276441in}{0.834408in}}%
\pgfpathlineto{\pgfqpoint{2.274186in}{0.835857in}}%
\pgfpathlineto{\pgfqpoint{2.262905in}{0.842079in}}%
\pgfpathlineto{\pgfqpoint{2.254671in}{0.845477in}}%
\pgfpathlineto{\pgfqpoint{2.251624in}{0.847415in}}%
\pgfpathlineto{\pgfqpoint{2.240343in}{0.853504in}}%
\pgfpathlineto{\pgfqpoint{2.234530in}{0.856547in}}%
\pgfpathlineto{\pgfqpoint{2.229062in}{0.860456in}}%
\pgfpathlineto{\pgfqpoint{2.217782in}{0.866851in}}%
\pgfpathlineto{\pgfqpoint{2.216347in}{0.867616in}}%
\pgfpathlineto{\pgfqpoint{2.206501in}{0.874888in}}%
\pgfpathlineto{\pgfqpoint{2.200775in}{0.878686in}}%
\pgfpathlineto{\pgfqpoint{2.195220in}{0.883088in}}%
\pgfpathlineto{\pgfqpoint{2.185207in}{0.889755in}}%
\pgfpathlineto{\pgfqpoint{2.183939in}{0.890821in}}%
\pgfpathlineto{\pgfqpoint{2.172658in}{0.900134in}}%
\pgfpathlineto{\pgfqpoint{2.171783in}{0.900825in}}%
\pgfpathlineto{\pgfqpoint{2.161377in}{0.909338in}}%
\pgfpathlineto{\pgfqpoint{2.158164in}{0.911894in}}%
\pgfpathlineto{\pgfqpoint{2.150097in}{0.918926in}}%
\pgfpathlineto{\pgfqpoint{2.144698in}{0.922964in}}%
\pgfpathlineto{\pgfqpoint{2.138816in}{0.927987in}}%
\pgfpathlineto{\pgfqpoint{2.131467in}{0.934033in}}%
\pgfpathlineto{\pgfqpoint{2.127535in}{0.937396in}}%
\pgfpathlineto{\pgfqpoint{2.119043in}{0.945103in}}%
\pgfpathlineto{\pgfqpoint{2.116254in}{0.948005in}}%
\pgfpathlineto{\pgfqpoint{2.107565in}{0.956173in}}%
\pgfpathlineto{\pgfqpoint{2.104973in}{0.959458in}}%
\pgfpathlineto{\pgfqpoint{2.098956in}{0.967242in}}%
\pgfpathlineto{\pgfqpoint{2.093693in}{0.975345in}}%
\pgfpathlineto{\pgfqpoint{2.091797in}{0.978312in}}%
\pgfpathlineto{\pgfqpoint{2.086314in}{0.989381in}}%
\pgfpathlineto{\pgfqpoint{2.086146in}{1.000451in}}%
\pgfpathlineto{\pgfqpoint{2.085399in}{1.011520in}}%
\pgfpathlineto{\pgfqpoint{2.082476in}{1.022590in}}%
\pgfpathlineto{\pgfqpoint{2.082412in}{1.022650in}}%
\pgfpathlineto{\pgfqpoint{2.073028in}{1.033659in}}%
\pgfpathlineto{\pgfqpoint{2.071131in}{1.034975in}}%
\pgfpathlineto{\pgfqpoint{2.062002in}{1.044729in}}%
\pgfpathlineto{\pgfqpoint{2.059850in}{1.046746in}}%
\pgfpathlineto{\pgfqpoint{2.052759in}{1.055798in}}%
\pgfpathlineto{\pgfqpoint{2.048569in}{1.058728in}}%
\pgfpathlineto{\pgfqpoint{2.040848in}{1.066868in}}%
\pgfpathlineto{\pgfqpoint{2.037289in}{1.069241in}}%
\pgfpathlineto{\pgfqpoint{2.026008in}{1.077616in}}%
\pgfpathlineto{\pgfqpoint{2.025771in}{1.077937in}}%
\pgfpathlineto{\pgfqpoint{2.014727in}{1.087219in}}%
\pgfpathlineto{\pgfqpoint{2.013284in}{1.089007in}}%
\pgfpathlineto{\pgfqpoint{2.003446in}{1.099653in}}%
\pgfpathlineto{\pgfqpoint{2.003168in}{1.100077in}}%
\pgfpathlineto{\pgfqpoint{1.995364in}{1.111146in}}%
\pgfpathlineto{\pgfqpoint{1.992165in}{1.115302in}}%
\pgfpathlineto{\pgfqpoint{1.987489in}{1.122216in}}%
\pgfpathlineto{\pgfqpoint{1.980884in}{1.131521in}}%
\pgfpathlineto{\pgfqpoint{1.979702in}{1.133285in}}%
\pgfpathlineto{\pgfqpoint{1.973129in}{1.144355in}}%
\pgfpathlineto{\pgfqpoint{1.969604in}{1.150692in}}%
\pgfpathlineto{\pgfqpoint{1.966933in}{1.155424in}}%
\pgfpathlineto{\pgfqpoint{1.961258in}{1.166494in}}%
\pgfpathlineto{\pgfqpoint{1.958323in}{1.172189in}}%
\pgfpathlineto{\pgfqpoint{1.955471in}{1.177563in}}%
\pgfpathlineto{\pgfqpoint{1.949479in}{1.188633in}}%
\pgfpathlineto{\pgfqpoint{1.947042in}{1.193230in}}%
\pgfpathlineto{\pgfqpoint{1.943505in}{1.199702in}}%
\pgfpathlineto{\pgfqpoint{1.937297in}{1.210772in}}%
\pgfpathlineto{\pgfqpoint{1.935761in}{1.213526in}}%
\pgfpathlineto{\pgfqpoint{1.930918in}{1.221842in}}%
\pgfpathlineto{\pgfqpoint{1.924496in}{1.232911in}}%
\pgfpathlineto{\pgfqpoint{1.924480in}{1.232937in}}%
\pgfpathlineto{\pgfqpoint{1.917771in}{1.243981in}}%
\pgfpathlineto{\pgfqpoint{1.913200in}{1.251712in}}%
\pgfpathlineto{\pgfqpoint{1.911130in}{1.255050in}}%
\pgfpathlineto{\pgfqpoint{1.905448in}{1.266120in}}%
\pgfpathlineto{\pgfqpoint{1.901919in}{1.275206in}}%
\pgfpathlineto{\pgfqpoint{1.901101in}{1.277189in}}%
\pgfpathlineto{\pgfqpoint{1.896604in}{1.288259in}}%
\pgfpathlineto{\pgfqpoint{1.892113in}{1.299328in}}%
\pgfpathlineto{\pgfqpoint{1.890638in}{1.302752in}}%
\pgfpathlineto{\pgfqpoint{1.887251in}{1.310398in}}%
\pgfpathlineto{\pgfqpoint{1.881955in}{1.321467in}}%
\pgfpathlineto{\pgfqpoint{1.879357in}{1.327064in}}%
\pgfpathlineto{\pgfqpoint{1.876667in}{1.332537in}}%
\pgfpathlineto{\pgfqpoint{1.871887in}{1.343606in}}%
\pgfpathlineto{\pgfqpoint{1.868076in}{1.353558in}}%
\pgfpathlineto{\pgfqpoint{1.867628in}{1.354676in}}%
\pgfpathlineto{\pgfqpoint{1.863716in}{1.365746in}}%
\pgfpathlineto{\pgfqpoint{1.859900in}{1.376815in}}%
\pgfpathlineto{\pgfqpoint{1.856795in}{1.384945in}}%
\pgfpathlineto{\pgfqpoint{1.856795in}{1.376815in}}%
\pgfpathlineto{\pgfqpoint{1.856795in}{1.365746in}}%
\pgfpathlineto{\pgfqpoint{1.856795in}{1.354676in}}%
\pgfpathlineto{\pgfqpoint{1.856795in}{1.343606in}}%
\pgfpathlineto{\pgfqpoint{1.856795in}{1.332537in}}%
\pgfpathlineto{\pgfqpoint{1.856795in}{1.321467in}}%
\pgfpathlineto{\pgfqpoint{1.856795in}{1.310398in}}%
\pgfpathlineto{\pgfqpoint{1.856795in}{1.299328in}}%
\pgfpathlineto{\pgfqpoint{1.856795in}{1.288259in}}%
\pgfpathlineto{\pgfqpoint{1.856795in}{1.277189in}}%
\pgfpathlineto{\pgfqpoint{1.856795in}{1.266120in}}%
\pgfpathlineto{\pgfqpoint{1.856795in}{1.263410in}}%
\pgfpathlineto{\pgfqpoint{1.862621in}{1.255050in}}%
\pgfpathlineto{\pgfqpoint{1.868076in}{1.247246in}}%
\pgfpathlineto{\pgfqpoint{1.870302in}{1.243981in}}%
\pgfpathlineto{\pgfqpoint{1.877933in}{1.232911in}}%
\pgfpathlineto{\pgfqpoint{1.879357in}{1.230867in}}%
\pgfpathlineto{\pgfqpoint{1.885504in}{1.221842in}}%
\pgfpathlineto{\pgfqpoint{1.890638in}{1.214288in}}%
\pgfpathlineto{\pgfqpoint{1.892970in}{1.210772in}}%
\pgfpathlineto{\pgfqpoint{1.900383in}{1.199702in}}%
\pgfpathlineto{\pgfqpoint{1.901919in}{1.197561in}}%
\pgfpathlineto{\pgfqpoint{1.907982in}{1.188633in}}%
\pgfpathlineto{\pgfqpoint{1.913200in}{1.181464in}}%
\pgfpathlineto{\pgfqpoint{1.915836in}{1.177563in}}%
\pgfpathlineto{\pgfqpoint{1.924167in}{1.166494in}}%
\pgfpathlineto{\pgfqpoint{1.924480in}{1.166091in}}%
\pgfpathlineto{\pgfqpoint{1.932498in}{1.155424in}}%
\pgfpathlineto{\pgfqpoint{1.935761in}{1.151200in}}%
\pgfpathlineto{\pgfqpoint{1.940856in}{1.144355in}}%
\pgfpathlineto{\pgfqpoint{1.947042in}{1.136046in}}%
\pgfpathlineto{\pgfqpoint{1.949050in}{1.133285in}}%
\pgfpathlineto{\pgfqpoint{1.956943in}{1.122216in}}%
\pgfpathlineto{\pgfqpoint{1.958323in}{1.120251in}}%
\pgfpathlineto{\pgfqpoint{1.964544in}{1.111146in}}%
\pgfpathlineto{\pgfqpoint{1.969604in}{1.103547in}}%
\pgfpathlineto{\pgfqpoint{1.971844in}{1.100077in}}%
\pgfpathlineto{\pgfqpoint{1.978667in}{1.089007in}}%
\pgfpathlineto{\pgfqpoint{1.980884in}{1.084535in}}%
\pgfpathlineto{\pgfqpoint{1.984075in}{1.077937in}}%
\pgfpathlineto{\pgfqpoint{1.987018in}{1.066868in}}%
\pgfpathlineto{\pgfqpoint{1.987573in}{1.055798in}}%
\pgfpathlineto{\pgfqpoint{1.986842in}{1.044729in}}%
\pgfpathlineto{\pgfqpoint{1.985450in}{1.033659in}}%
\pgfpathlineto{\pgfqpoint{1.984417in}{1.022590in}}%
\pgfpathlineto{\pgfqpoint{1.984296in}{1.011520in}}%
\pgfpathlineto{\pgfqpoint{1.985798in}{1.000451in}}%
\pgfpathlineto{\pgfqpoint{1.987740in}{0.989381in}}%
\pgfpathlineto{\pgfqpoint{1.990683in}{0.978312in}}%
\pgfpathlineto{\pgfqpoint{1.992165in}{0.974034in}}%
\pgfpathlineto{\pgfqpoint{1.995338in}{0.967242in}}%
\pgfpathlineto{\pgfqpoint{2.001373in}{0.956173in}}%
\pgfpathlineto{\pgfqpoint{2.003446in}{0.952659in}}%
\pgfpathlineto{\pgfqpoint{2.009512in}{0.945103in}}%
\pgfpathlineto{\pgfqpoint{2.014727in}{0.939472in}}%
\pgfpathlineto{\pgfqpoint{2.020631in}{0.934033in}}%
\pgfpathlineto{\pgfqpoint{2.026008in}{0.930069in}}%
\pgfpathlineto{\pgfqpoint{2.035365in}{0.922964in}}%
\pgfpathlineto{\pgfqpoint{2.037289in}{0.921607in}}%
\pgfpathlineto{\pgfqpoint{2.048569in}{0.913674in}}%
\pgfpathlineto{\pgfqpoint{2.051130in}{0.911894in}}%
\pgfpathlineto{\pgfqpoint{2.059850in}{0.906106in}}%
\pgfpathlineto{\pgfqpoint{2.067688in}{0.900825in}}%
\pgfpathlineto{\pgfqpoint{2.071131in}{0.898500in}}%
\pgfpathlineto{\pgfqpoint{2.082412in}{0.890628in}}%
\pgfpathlineto{\pgfqpoint{2.083660in}{0.889755in}}%
\pgfpathlineto{\pgfqpoint{2.093693in}{0.883212in}}%
\pgfpathlineto{\pgfqpoint{2.100471in}{0.878686in}}%
\pgfpathlineto{\pgfqpoint{2.104973in}{0.876333in}}%
\pgfpathlineto{\pgfqpoint{2.116254in}{0.870520in}}%
\pgfpathlineto{\pgfqpoint{2.121732in}{0.867616in}}%
\pgfpathlineto{\pgfqpoint{2.127535in}{0.864893in}}%
\pgfpathlineto{\pgfqpoint{2.138816in}{0.859545in}}%
\pgfpathlineto{\pgfqpoint{2.144998in}{0.856547in}}%
\pgfpathlineto{\pgfqpoint{2.150097in}{0.854039in}}%
\pgfpathlineto{\pgfqpoint{2.161377in}{0.848585in}}%
\pgfpathlineto{\pgfqpoint{2.168057in}{0.845477in}}%
\pgfpathlineto{\pgfqpoint{2.172658in}{0.843396in}}%
\pgfpathlineto{\pgfqpoint{2.183939in}{0.838146in}}%
\pgfpathlineto{\pgfqpoint{2.191809in}{0.834408in}}%
\pgfpathlineto{\pgfqpoint{2.195220in}{0.832888in}}%
\pgfpathlineto{\pgfqpoint{2.206501in}{0.827561in}}%
\pgfpathlineto{\pgfqpoint{2.215238in}{0.823338in}}%
\pgfpathlineto{\pgfqpoint{2.217782in}{0.822180in}}%
\pgfpathlineto{\pgfqpoint{2.229062in}{0.816919in}}%
\pgfpathlineto{\pgfqpoint{2.238562in}{0.812269in}}%
\pgfpathlineto{\pgfqpoint{2.240343in}{0.811418in}}%
\pgfpathlineto{\pgfqpoint{2.251624in}{0.806084in}}%
\pgfpathlineto{\pgfqpoint{2.261656in}{0.801199in}}%
\pgfpathlineto{\pgfqpoint{2.262905in}{0.800592in}}%
\pgfpathlineto{\pgfqpoint{2.274186in}{0.795154in}}%
\pgfpathlineto{\pgfqpoint{2.284161in}{0.790129in}}%
\pgfpathlineto{\pgfqpoint{2.285466in}{0.789464in}}%
\pgfpathlineto{\pgfqpoint{2.296747in}{0.783809in}}%
\pgfpathlineto{\pgfqpoint{2.306008in}{0.779060in}}%
\pgfpathlineto{\pgfqpoint{2.308028in}{0.777994in}}%
\pgfpathlineto{\pgfqpoint{2.319309in}{0.772117in}}%
\pgfpathlineto{\pgfqpoint{2.327297in}{0.767990in}}%
\pgfpathlineto{\pgfqpoint{2.330590in}{0.766242in}}%
\pgfpathlineto{\pgfqpoint{2.341871in}{0.760318in}}%
\pgfpathlineto{\pgfqpoint{2.348427in}{0.756921in}}%
\pgfpathlineto{\pgfqpoint{2.353151in}{0.754400in}}%
\pgfpathlineto{\pgfqpoint{2.364432in}{0.748485in}}%
\pgfpathlineto{\pgfqpoint{2.370006in}{0.745851in}}%
\pgfpathlineto{\pgfqpoint{2.375713in}{0.743044in}}%
\pgfpathlineto{\pgfqpoint{2.386994in}{0.737506in}}%
\pgfpathlineto{\pgfqpoint{2.392067in}{0.734782in}}%
\pgfpathlineto{\pgfqpoint{2.398275in}{0.731226in}}%
\pgfpathlineto{\pgfqpoint{2.409555in}{0.724411in}}%
\pgfpathlineto{\pgfqpoint{2.410588in}{0.723712in}}%
\pgfpathlineto{\pgfqpoint{2.420836in}{0.716298in}}%
\pgfpathlineto{\pgfqpoint{2.425872in}{0.712643in}}%
\pgfpathlineto{\pgfqpoint{2.432117in}{0.707947in}}%
\pgfpathlineto{\pgfqpoint{2.440526in}{0.701573in}}%
\pgfpathlineto{\pgfqpoint{2.443398in}{0.699318in}}%
\pgfpathlineto{\pgfqpoint{2.454554in}{0.690504in}}%
\pgfpathlineto{\pgfqpoint{2.454679in}{0.690412in}}%
\pgfpathlineto{\pgfqpoint{2.465960in}{0.680742in}}%
\pgfpathlineto{\pgfqpoint{2.467393in}{0.679434in}}%
\pgfpathlineto{\pgfqpoint{2.477240in}{0.671215in}}%
\pgfpathlineto{\pgfqpoint{2.480176in}{0.668364in}}%
\pgfpathlineto{\pgfqpoint{2.488521in}{0.661002in}}%
\pgfpathlineto{\pgfqpoint{2.492382in}{0.657295in}}%
\pgfpathlineto{\pgfqpoint{2.499802in}{0.650381in}}%
\pgfpathlineto{\pgfqpoint{2.504398in}{0.646225in}}%
\pgfpathlineto{\pgfqpoint{2.511083in}{0.640345in}}%
\pgfpathlineto{\pgfqpoint{2.517697in}{0.635156in}}%
\pgfpathlineto{\pgfqpoint{2.522364in}{0.631411in}}%
\pgfpathlineto{\pgfqpoint{2.532380in}{0.624086in}}%
\pgfpathlineto{\pgfqpoint{2.533644in}{0.623061in}}%
\pgfpathlineto{\pgfqpoint{2.544925in}{0.613691in}}%
\pgfpathlineto{\pgfqpoint{2.545726in}{0.613017in}}%
\pgfpathlineto{\pgfqpoint{2.556206in}{0.603727in}}%
\pgfpathlineto{\pgfqpoint{2.558220in}{0.601947in}}%
\pgfpathlineto{\pgfqpoint{2.567487in}{0.593383in}}%
\pgfpathlineto{\pgfqpoint{2.570299in}{0.590878in}}%
\pgfpathlineto{\pgfqpoint{2.578768in}{0.582574in}}%
\pgfpathlineto{\pgfqpoint{2.581670in}{0.579808in}}%
\pgfpathlineto{\pgfqpoint{2.590048in}{0.570757in}}%
\pgfpathlineto{\pgfqpoint{2.592062in}{0.568739in}}%
\pgfpathlineto{\pgfqpoint{2.601329in}{0.558794in}}%
\pgfpathlineto{\pgfqpoint{2.602342in}{0.557669in}}%
\pgfpathlineto{\pgfqpoint{2.611901in}{0.546600in}}%
\pgfpathlineto{\pgfqpoint{2.612610in}{0.545758in}}%
\pgfpathlineto{\pgfqpoint{2.621361in}{0.535530in}}%
\pgfpathlineto{\pgfqpoint{2.623891in}{0.532558in}}%
\pgfpathlineto{\pgfqpoint{2.630447in}{0.524460in}}%
\pgfpathlineto{\pgfqpoint{2.635172in}{0.518167in}}%
\pgfpathlineto{\pgfqpoint{2.638759in}{0.513391in}}%
\pgfpathlineto{\pgfqpoint{2.646453in}{0.503103in}}%
\pgfpathlineto{\pgfqpoint{2.647043in}{0.502321in}}%
\pgfpathlineto{\pgfqpoint{2.654626in}{0.491252in}}%
\pgfpathclose%
\pgfusepath{fill}%
\end{pgfscope}%
\begin{pgfscope}%
\pgfpathrectangle{\pgfqpoint{1.856795in}{0.423750in}}{\pgfqpoint{1.194205in}{1.163386in}}%
\pgfusepath{clip}%
\pgfsetbuttcap%
\pgfsetroundjoin%
\definecolor{currentfill}{rgb}{0.966328,0.750560,0.616961}%
\pgfsetfillcolor{currentfill}%
\pgfsetlinewidth{0.000000pt}%
\definecolor{currentstroke}{rgb}{0.000000,0.000000,0.000000}%
\pgfsetstrokecolor{currentstroke}%
\pgfsetdash{}{0pt}%
\pgfpathmoveto{\pgfqpoint{2.635172in}{0.844939in}}%
\pgfpathlineto{\pgfqpoint{2.646453in}{0.844110in}}%
\pgfpathlineto{\pgfqpoint{2.657733in}{0.843235in}}%
\pgfpathlineto{\pgfqpoint{2.669014in}{0.842518in}}%
\pgfpathlineto{\pgfqpoint{2.680295in}{0.842672in}}%
\pgfpathlineto{\pgfqpoint{2.691576in}{0.841043in}}%
\pgfpathlineto{\pgfqpoint{2.702857in}{0.839182in}}%
\pgfpathlineto{\pgfqpoint{2.714137in}{0.838106in}}%
\pgfpathlineto{\pgfqpoint{2.725418in}{0.838387in}}%
\pgfpathlineto{\pgfqpoint{2.736699in}{0.838743in}}%
\pgfpathlineto{\pgfqpoint{2.747980in}{0.839560in}}%
\pgfpathlineto{\pgfqpoint{2.759261in}{0.841081in}}%
\pgfpathlineto{\pgfqpoint{2.770542in}{0.843664in}}%
\pgfpathlineto{\pgfqpoint{2.775662in}{0.845477in}}%
\pgfpathlineto{\pgfqpoint{2.781822in}{0.847762in}}%
\pgfpathlineto{\pgfqpoint{2.793103in}{0.851502in}}%
\pgfpathlineto{\pgfqpoint{2.804359in}{0.856547in}}%
\pgfpathlineto{\pgfqpoint{2.804384in}{0.856558in}}%
\pgfpathlineto{\pgfqpoint{2.815665in}{0.862039in}}%
\pgfpathlineto{\pgfqpoint{2.824588in}{0.867616in}}%
\pgfpathlineto{\pgfqpoint{2.826946in}{0.869505in}}%
\pgfpathlineto{\pgfqpoint{2.836218in}{0.878686in}}%
\pgfpathlineto{\pgfqpoint{2.838226in}{0.881235in}}%
\pgfpathlineto{\pgfqpoint{2.844825in}{0.889755in}}%
\pgfpathlineto{\pgfqpoint{2.849507in}{0.894572in}}%
\pgfpathlineto{\pgfqpoint{2.854615in}{0.900825in}}%
\pgfpathlineto{\pgfqpoint{2.860788in}{0.908003in}}%
\pgfpathlineto{\pgfqpoint{2.863555in}{0.911894in}}%
\pgfpathlineto{\pgfqpoint{2.872069in}{0.920710in}}%
\pgfpathlineto{\pgfqpoint{2.874277in}{0.922964in}}%
\pgfpathlineto{\pgfqpoint{2.883350in}{0.931235in}}%
\pgfpathlineto{\pgfqpoint{2.886267in}{0.934033in}}%
\pgfpathlineto{\pgfqpoint{2.894631in}{0.942158in}}%
\pgfpathlineto{\pgfqpoint{2.897596in}{0.945103in}}%
\pgfpathlineto{\pgfqpoint{2.905911in}{0.953397in}}%
\pgfpathlineto{\pgfqpoint{2.908611in}{0.956173in}}%
\pgfpathlineto{\pgfqpoint{2.917192in}{0.964966in}}%
\pgfpathlineto{\pgfqpoint{2.919306in}{0.967242in}}%
\pgfpathlineto{\pgfqpoint{2.928473in}{0.977617in}}%
\pgfpathlineto{\pgfqpoint{2.929065in}{0.978312in}}%
\pgfpathlineto{\pgfqpoint{2.938555in}{0.989381in}}%
\pgfpathlineto{\pgfqpoint{2.939754in}{0.990521in}}%
\pgfpathlineto{\pgfqpoint{2.948666in}{1.000451in}}%
\pgfpathlineto{\pgfqpoint{2.951035in}{1.002709in}}%
\pgfpathlineto{\pgfqpoint{2.959819in}{1.011520in}}%
\pgfpathlineto{\pgfqpoint{2.962315in}{1.014061in}}%
\pgfpathlineto{\pgfqpoint{2.970312in}{1.022590in}}%
\pgfpathlineto{\pgfqpoint{2.973596in}{1.026112in}}%
\pgfpathlineto{\pgfqpoint{2.973596in}{1.033659in}}%
\pgfpathlineto{\pgfqpoint{2.973596in}{1.044729in}}%
\pgfpathlineto{\pgfqpoint{2.973596in}{1.055798in}}%
\pgfpathlineto{\pgfqpoint{2.973596in}{1.066868in}}%
\pgfpathlineto{\pgfqpoint{2.973596in}{1.077937in}}%
\pgfpathlineto{\pgfqpoint{2.973596in}{1.089007in}}%
\pgfpathlineto{\pgfqpoint{2.973596in}{1.100077in}}%
\pgfpathlineto{\pgfqpoint{2.973596in}{1.111146in}}%
\pgfpathlineto{\pgfqpoint{2.973596in}{1.114552in}}%
\pgfpathlineto{\pgfqpoint{2.970485in}{1.111146in}}%
\pgfpathlineto{\pgfqpoint{2.962315in}{1.102181in}}%
\pgfpathlineto{\pgfqpoint{2.960343in}{1.100077in}}%
\pgfpathlineto{\pgfqpoint{2.951035in}{1.090247in}}%
\pgfpathlineto{\pgfqpoint{2.949812in}{1.089007in}}%
\pgfpathlineto{\pgfqpoint{2.939754in}{1.078428in}}%
\pgfpathlineto{\pgfqpoint{2.939272in}{1.077937in}}%
\pgfpathlineto{\pgfqpoint{2.928624in}{1.066868in}}%
\pgfpathlineto{\pgfqpoint{2.928473in}{1.066722in}}%
\pgfpathlineto{\pgfqpoint{2.917192in}{1.056073in}}%
\pgfpathlineto{\pgfqpoint{2.916907in}{1.055798in}}%
\pgfpathlineto{\pgfqpoint{2.905911in}{1.046103in}}%
\pgfpathlineto{\pgfqpoint{2.904384in}{1.044729in}}%
\pgfpathlineto{\pgfqpoint{2.894631in}{1.036024in}}%
\pgfpathlineto{\pgfqpoint{2.891880in}{1.033659in}}%
\pgfpathlineto{\pgfqpoint{2.883350in}{1.026093in}}%
\pgfpathlineto{\pgfqpoint{2.879392in}{1.022590in}}%
\pgfpathlineto{\pgfqpoint{2.872069in}{1.016243in}}%
\pgfpathlineto{\pgfqpoint{2.866472in}{1.011520in}}%
\pgfpathlineto{\pgfqpoint{2.860788in}{1.006982in}}%
\pgfpathlineto{\pgfqpoint{2.852420in}{1.000451in}}%
\pgfpathlineto{\pgfqpoint{2.849507in}{0.998101in}}%
\pgfpathlineto{\pgfqpoint{2.838323in}{0.989381in}}%
\pgfpathlineto{\pgfqpoint{2.838226in}{0.989305in}}%
\pgfpathlineto{\pgfqpoint{2.826946in}{0.980493in}}%
\pgfpathlineto{\pgfqpoint{2.824187in}{0.978312in}}%
\pgfpathlineto{\pgfqpoint{2.815665in}{0.971846in}}%
\pgfpathlineto{\pgfqpoint{2.809428in}{0.967242in}}%
\pgfpathlineto{\pgfqpoint{2.804384in}{0.963493in}}%
\pgfpathlineto{\pgfqpoint{2.794403in}{0.956173in}}%
\pgfpathlineto{\pgfqpoint{2.793103in}{0.955232in}}%
\pgfpathlineto{\pgfqpoint{2.781822in}{0.945703in}}%
\pgfpathlineto{\pgfqpoint{2.781059in}{0.945103in}}%
\pgfpathlineto{\pgfqpoint{2.770542in}{0.936839in}}%
\pgfpathlineto{\pgfqpoint{2.766102in}{0.934033in}}%
\pgfpathlineto{\pgfqpoint{2.759261in}{0.930087in}}%
\pgfpathlineto{\pgfqpoint{2.747980in}{0.925267in}}%
\pgfpathlineto{\pgfqpoint{2.741653in}{0.922964in}}%
\pgfpathlineto{\pgfqpoint{2.736699in}{0.920652in}}%
\pgfpathlineto{\pgfqpoint{2.725418in}{0.916335in}}%
\pgfpathlineto{\pgfqpoint{2.714137in}{0.913826in}}%
\pgfpathlineto{\pgfqpoint{2.702857in}{0.912388in}}%
\pgfpathlineto{\pgfqpoint{2.693924in}{0.911894in}}%
\pgfpathlineto{\pgfqpoint{2.691576in}{0.911749in}}%
\pgfpathlineto{\pgfqpoint{2.680295in}{0.911219in}}%
\pgfpathlineto{\pgfqpoint{2.669014in}{0.911174in}}%
\pgfpathlineto{\pgfqpoint{2.663990in}{0.911894in}}%
\pgfpathlineto{\pgfqpoint{2.657733in}{0.912714in}}%
\pgfpathlineto{\pgfqpoint{2.646453in}{0.915281in}}%
\pgfpathlineto{\pgfqpoint{2.635172in}{0.918760in}}%
\pgfpathlineto{\pgfqpoint{2.623891in}{0.922404in}}%
\pgfpathlineto{\pgfqpoint{2.622329in}{0.922964in}}%
\pgfpathlineto{\pgfqpoint{2.612610in}{0.926033in}}%
\pgfpathlineto{\pgfqpoint{2.601329in}{0.929853in}}%
\pgfpathlineto{\pgfqpoint{2.591186in}{0.934033in}}%
\pgfpathlineto{\pgfqpoint{2.590048in}{0.934489in}}%
\pgfpathlineto{\pgfqpoint{2.578768in}{0.938763in}}%
\pgfpathlineto{\pgfqpoint{2.567487in}{0.944011in}}%
\pgfpathlineto{\pgfqpoint{2.565294in}{0.945103in}}%
\pgfpathlineto{\pgfqpoint{2.556206in}{0.949427in}}%
\pgfpathlineto{\pgfqpoint{2.544925in}{0.955508in}}%
\pgfpathlineto{\pgfqpoint{2.543705in}{0.956173in}}%
\pgfpathlineto{\pgfqpoint{2.533644in}{0.961678in}}%
\pgfpathlineto{\pgfqpoint{2.524378in}{0.967242in}}%
\pgfpathlineto{\pgfqpoint{2.522364in}{0.968449in}}%
\pgfpathlineto{\pgfqpoint{2.511083in}{0.975658in}}%
\pgfpathlineto{\pgfqpoint{2.506846in}{0.978312in}}%
\pgfpathlineto{\pgfqpoint{2.499802in}{0.983195in}}%
\pgfpathlineto{\pgfqpoint{2.491722in}{0.989381in}}%
\pgfpathlineto{\pgfqpoint{2.488521in}{0.991808in}}%
\pgfpathlineto{\pgfqpoint{2.477371in}{1.000451in}}%
\pgfpathlineto{\pgfqpoint{2.477240in}{1.000550in}}%
\pgfpathlineto{\pgfqpoint{2.465960in}{1.009138in}}%
\pgfpathlineto{\pgfqpoint{2.462979in}{1.011520in}}%
\pgfpathlineto{\pgfqpoint{2.454679in}{1.017788in}}%
\pgfpathlineto{\pgfqpoint{2.448779in}{1.022590in}}%
\pgfpathlineto{\pgfqpoint{2.443398in}{1.027062in}}%
\pgfpathlineto{\pgfqpoint{2.435604in}{1.033659in}}%
\pgfpathlineto{\pgfqpoint{2.432117in}{1.036775in}}%
\pgfpathlineto{\pgfqpoint{2.423128in}{1.044729in}}%
\pgfpathlineto{\pgfqpoint{2.420836in}{1.046797in}}%
\pgfpathlineto{\pgfqpoint{2.411087in}{1.055798in}}%
\pgfpathlineto{\pgfqpoint{2.409555in}{1.057184in}}%
\pgfpathlineto{\pgfqpoint{2.398877in}{1.066868in}}%
\pgfpathlineto{\pgfqpoint{2.398275in}{1.067412in}}%
\pgfpathlineto{\pgfqpoint{2.386994in}{1.077599in}}%
\pgfpathlineto{\pgfqpoint{2.386614in}{1.077937in}}%
\pgfpathlineto{\pgfqpoint{2.375713in}{1.087664in}}%
\pgfpathlineto{\pgfqpoint{2.374170in}{1.089007in}}%
\pgfpathlineto{\pgfqpoint{2.364432in}{1.096879in}}%
\pgfpathlineto{\pgfqpoint{2.360621in}{1.100077in}}%
\pgfpathlineto{\pgfqpoint{2.353151in}{1.105959in}}%
\pgfpathlineto{\pgfqpoint{2.346519in}{1.111146in}}%
\pgfpathlineto{\pgfqpoint{2.341871in}{1.114575in}}%
\pgfpathlineto{\pgfqpoint{2.331928in}{1.122216in}}%
\pgfpathlineto{\pgfqpoint{2.330590in}{1.123238in}}%
\pgfpathlineto{\pgfqpoint{2.319309in}{1.131998in}}%
\pgfpathlineto{\pgfqpoint{2.317675in}{1.133285in}}%
\pgfpathlineto{\pgfqpoint{2.308028in}{1.141057in}}%
\pgfpathlineto{\pgfqpoint{2.303923in}{1.144355in}}%
\pgfpathlineto{\pgfqpoint{2.296747in}{1.150201in}}%
\pgfpathlineto{\pgfqpoint{2.290319in}{1.155424in}}%
\pgfpathlineto{\pgfqpoint{2.285466in}{1.159423in}}%
\pgfpathlineto{\pgfqpoint{2.276863in}{1.166494in}}%
\pgfpathlineto{\pgfqpoint{2.274186in}{1.168725in}}%
\pgfpathlineto{\pgfqpoint{2.263554in}{1.177563in}}%
\pgfpathlineto{\pgfqpoint{2.262905in}{1.178102in}}%
\pgfpathlineto{\pgfqpoint{2.251624in}{1.187651in}}%
\pgfpathlineto{\pgfqpoint{2.250485in}{1.188633in}}%
\pgfpathlineto{\pgfqpoint{2.240343in}{1.197354in}}%
\pgfpathlineto{\pgfqpoint{2.237729in}{1.199702in}}%
\pgfpathlineto{\pgfqpoint{2.229062in}{1.207849in}}%
\pgfpathlineto{\pgfqpoint{2.225982in}{1.210772in}}%
\pgfpathlineto{\pgfqpoint{2.217782in}{1.218625in}}%
\pgfpathlineto{\pgfqpoint{2.214627in}{1.221842in}}%
\pgfpathlineto{\pgfqpoint{2.206501in}{1.230425in}}%
\pgfpathlineto{\pgfqpoint{2.204382in}{1.232911in}}%
\pgfpathlineto{\pgfqpoint{2.195220in}{1.243658in}}%
\pgfpathlineto{\pgfqpoint{2.194949in}{1.243981in}}%
\pgfpathlineto{\pgfqpoint{2.185785in}{1.255050in}}%
\pgfpathlineto{\pgfqpoint{2.183939in}{1.257403in}}%
\pgfpathlineto{\pgfqpoint{2.176894in}{1.266120in}}%
\pgfpathlineto{\pgfqpoint{2.172658in}{1.271924in}}%
\pgfpathlineto{\pgfqpoint{2.168707in}{1.277189in}}%
\pgfpathlineto{\pgfqpoint{2.161377in}{1.287431in}}%
\pgfpathlineto{\pgfqpoint{2.160778in}{1.288259in}}%
\pgfpathlineto{\pgfqpoint{2.153941in}{1.299328in}}%
\pgfpathlineto{\pgfqpoint{2.150097in}{1.305828in}}%
\pgfpathlineto{\pgfqpoint{2.147390in}{1.310398in}}%
\pgfpathlineto{\pgfqpoint{2.141030in}{1.321467in}}%
\pgfpathlineto{\pgfqpoint{2.138816in}{1.325357in}}%
\pgfpathlineto{\pgfqpoint{2.134740in}{1.332537in}}%
\pgfpathlineto{\pgfqpoint{2.128446in}{1.343606in}}%
\pgfpathlineto{\pgfqpoint{2.127535in}{1.345130in}}%
\pgfpathlineto{\pgfqpoint{2.121814in}{1.354676in}}%
\pgfpathlineto{\pgfqpoint{2.116254in}{1.363934in}}%
\pgfpathlineto{\pgfqpoint{2.115167in}{1.365746in}}%
\pgfpathlineto{\pgfqpoint{2.108527in}{1.376815in}}%
\pgfpathlineto{\pgfqpoint{2.104973in}{1.382751in}}%
\pgfpathlineto{\pgfqpoint{2.101918in}{1.387885in}}%
\pgfpathlineto{\pgfqpoint{2.095401in}{1.398954in}}%
\pgfpathlineto{\pgfqpoint{2.093693in}{1.401896in}}%
\pgfpathlineto{\pgfqpoint{2.088971in}{1.410024in}}%
\pgfpathlineto{\pgfqpoint{2.082631in}{1.421093in}}%
\pgfpathlineto{\pgfqpoint{2.082412in}{1.421479in}}%
\pgfpathlineto{\pgfqpoint{2.076353in}{1.432163in}}%
\pgfpathlineto{\pgfqpoint{2.071131in}{1.441413in}}%
\pgfpathlineto{\pgfqpoint{2.070107in}{1.443232in}}%
\pgfpathlineto{\pgfqpoint{2.063364in}{1.454302in}}%
\pgfpathlineto{\pgfqpoint{2.059850in}{1.459577in}}%
\pgfpathlineto{\pgfqpoint{2.055992in}{1.465371in}}%
\pgfpathlineto{\pgfqpoint{2.048569in}{1.476440in}}%
\pgfpathlineto{\pgfqpoint{2.048569in}{1.476441in}}%
\pgfpathlineto{\pgfqpoint{2.041214in}{1.487510in}}%
\pgfpathlineto{\pgfqpoint{2.037289in}{1.493566in}}%
\pgfpathlineto{\pgfqpoint{2.034021in}{1.498580in}}%
\pgfpathlineto{\pgfqpoint{2.026892in}{1.509650in}}%
\pgfpathlineto{\pgfqpoint{2.026008in}{1.511015in}}%
\pgfpathlineto{\pgfqpoint{2.019718in}{1.520719in}}%
\pgfpathlineto{\pgfqpoint{2.014727in}{1.528355in}}%
\pgfpathlineto{\pgfqpoint{2.012504in}{1.531789in}}%
\pgfpathlineto{\pgfqpoint{2.004839in}{1.542858in}}%
\pgfpathlineto{\pgfqpoint{2.003446in}{1.544848in}}%
\pgfpathlineto{\pgfqpoint{1.997021in}{1.553928in}}%
\pgfpathlineto{\pgfqpoint{1.992165in}{1.560811in}}%
\pgfpathlineto{\pgfqpoint{1.989169in}{1.564997in}}%
\pgfpathlineto{\pgfqpoint{1.981183in}{1.576067in}}%
\pgfpathlineto{\pgfqpoint{1.980884in}{1.576472in}}%
\pgfpathlineto{\pgfqpoint{1.972974in}{1.587136in}}%
\pgfpathlineto{\pgfqpoint{1.969604in}{1.587136in}}%
\pgfpathlineto{\pgfqpoint{1.958323in}{1.587136in}}%
\pgfpathlineto{\pgfqpoint{1.947042in}{1.587136in}}%
\pgfpathlineto{\pgfqpoint{1.935761in}{1.587136in}}%
\pgfpathlineto{\pgfqpoint{1.924480in}{1.587136in}}%
\pgfpathlineto{\pgfqpoint{1.917465in}{1.587136in}}%
\pgfpathlineto{\pgfqpoint{1.924480in}{1.576957in}}%
\pgfpathlineto{\pgfqpoint{1.925101in}{1.576067in}}%
\pgfpathlineto{\pgfqpoint{1.932686in}{1.564997in}}%
\pgfpathlineto{\pgfqpoint{1.935761in}{1.560541in}}%
\pgfpathlineto{\pgfqpoint{1.940327in}{1.553928in}}%
\pgfpathlineto{\pgfqpoint{1.947042in}{1.543449in}}%
\pgfpathlineto{\pgfqpoint{1.947428in}{1.542858in}}%
\pgfpathlineto{\pgfqpoint{1.954419in}{1.531789in}}%
\pgfpathlineto{\pgfqpoint{1.958323in}{1.525808in}}%
\pgfpathlineto{\pgfqpoint{1.961606in}{1.520719in}}%
\pgfpathlineto{\pgfqpoint{1.968177in}{1.509650in}}%
\pgfpathlineto{\pgfqpoint{1.969604in}{1.507039in}}%
\pgfpathlineto{\pgfqpoint{1.974179in}{1.498580in}}%
\pgfpathlineto{\pgfqpoint{1.980298in}{1.487510in}}%
\pgfpathlineto{\pgfqpoint{1.980884in}{1.486347in}}%
\pgfpathlineto{\pgfqpoint{1.985888in}{1.476441in}}%
\pgfpathlineto{\pgfqpoint{1.991531in}{1.465371in}}%
\pgfpathlineto{\pgfqpoint{1.992165in}{1.464142in}}%
\pgfpathlineto{\pgfqpoint{1.997192in}{1.454302in}}%
\pgfpathlineto{\pgfqpoint{2.002933in}{1.443232in}}%
\pgfpathlineto{\pgfqpoint{2.003446in}{1.442262in}}%
\pgfpathlineto{\pgfqpoint{2.008617in}{1.432163in}}%
\pgfpathlineto{\pgfqpoint{2.014346in}{1.421093in}}%
\pgfpathlineto{\pgfqpoint{2.014727in}{1.420374in}}%
\pgfpathlineto{\pgfqpoint{2.020195in}{1.410024in}}%
\pgfpathlineto{\pgfqpoint{2.026008in}{1.399139in}}%
\pgfpathlineto{\pgfqpoint{2.026107in}{1.398954in}}%
\pgfpathlineto{\pgfqpoint{2.032101in}{1.387885in}}%
\pgfpathlineto{\pgfqpoint{2.037289in}{1.378635in}}%
\pgfpathlineto{\pgfqpoint{2.038301in}{1.376815in}}%
\pgfpathlineto{\pgfqpoint{2.044277in}{1.365746in}}%
\pgfpathlineto{\pgfqpoint{2.048569in}{1.357903in}}%
\pgfpathlineto{\pgfqpoint{2.050312in}{1.354676in}}%
\pgfpathlineto{\pgfqpoint{2.056363in}{1.343606in}}%
\pgfpathlineto{\pgfqpoint{2.059850in}{1.337348in}}%
\pgfpathlineto{\pgfqpoint{2.062507in}{1.332537in}}%
\pgfpathlineto{\pgfqpoint{2.068679in}{1.321467in}}%
\pgfpathlineto{\pgfqpoint{2.071131in}{1.317608in}}%
\pgfpathlineto{\pgfqpoint{2.075487in}{1.310398in}}%
\pgfpathlineto{\pgfqpoint{2.082412in}{1.300811in}}%
\pgfpathlineto{\pgfqpoint{2.083459in}{1.299328in}}%
\pgfpathlineto{\pgfqpoint{2.091973in}{1.288259in}}%
\pgfpathlineto{\pgfqpoint{2.093693in}{1.286040in}}%
\pgfpathlineto{\pgfqpoint{2.100530in}{1.277189in}}%
\pgfpathlineto{\pgfqpoint{2.104973in}{1.271686in}}%
\pgfpathlineto{\pgfqpoint{2.109400in}{1.266120in}}%
\pgfpathlineto{\pgfqpoint{2.116254in}{1.257859in}}%
\pgfpathlineto{\pgfqpoint{2.118570in}{1.255050in}}%
\pgfpathlineto{\pgfqpoint{2.127535in}{1.244397in}}%
\pgfpathlineto{\pgfqpoint{2.127882in}{1.243981in}}%
\pgfpathlineto{\pgfqpoint{2.137197in}{1.232911in}}%
\pgfpathlineto{\pgfqpoint{2.138816in}{1.231041in}}%
\pgfpathlineto{\pgfqpoint{2.146663in}{1.221842in}}%
\pgfpathlineto{\pgfqpoint{2.150097in}{1.217902in}}%
\pgfpathlineto{\pgfqpoint{2.156262in}{1.210772in}}%
\pgfpathlineto{\pgfqpoint{2.161377in}{1.204935in}}%
\pgfpathlineto{\pgfqpoint{2.165862in}{1.199702in}}%
\pgfpathlineto{\pgfqpoint{2.172658in}{1.192009in}}%
\pgfpathlineto{\pgfqpoint{2.175606in}{1.188633in}}%
\pgfpathlineto{\pgfqpoint{2.183939in}{1.179308in}}%
\pgfpathlineto{\pgfqpoint{2.185561in}{1.177563in}}%
\pgfpathlineto{\pgfqpoint{2.195220in}{1.167687in}}%
\pgfpathlineto{\pgfqpoint{2.196398in}{1.166494in}}%
\pgfpathlineto{\pgfqpoint{2.206501in}{1.156305in}}%
\pgfpathlineto{\pgfqpoint{2.207435in}{1.155424in}}%
\pgfpathlineto{\pgfqpoint{2.217782in}{1.144852in}}%
\pgfpathlineto{\pgfqpoint{2.218382in}{1.144355in}}%
\pgfpathlineto{\pgfqpoint{2.229062in}{1.133870in}}%
\pgfpathlineto{\pgfqpoint{2.229709in}{1.133285in}}%
\pgfpathlineto{\pgfqpoint{2.239451in}{1.122216in}}%
\pgfpathlineto{\pgfqpoint{2.240343in}{1.120904in}}%
\pgfpathlineto{\pgfqpoint{2.248196in}{1.111146in}}%
\pgfpathlineto{\pgfqpoint{2.251624in}{1.106613in}}%
\pgfpathlineto{\pgfqpoint{2.256956in}{1.100077in}}%
\pgfpathlineto{\pgfqpoint{2.262905in}{1.091951in}}%
\pgfpathlineto{\pgfqpoint{2.265191in}{1.089007in}}%
\pgfpathlineto{\pgfqpoint{2.273466in}{1.077937in}}%
\pgfpathlineto{\pgfqpoint{2.274186in}{1.077033in}}%
\pgfpathlineto{\pgfqpoint{2.282622in}{1.066868in}}%
\pgfpathlineto{\pgfqpoint{2.285466in}{1.063739in}}%
\pgfpathlineto{\pgfqpoint{2.292760in}{1.055798in}}%
\pgfpathlineto{\pgfqpoint{2.296747in}{1.051358in}}%
\pgfpathlineto{\pgfqpoint{2.302351in}{1.044729in}}%
\pgfpathlineto{\pgfqpoint{2.308028in}{1.038536in}}%
\pgfpathlineto{\pgfqpoint{2.312878in}{1.033659in}}%
\pgfpathlineto{\pgfqpoint{2.319309in}{1.027245in}}%
\pgfpathlineto{\pgfqpoint{2.324235in}{1.022590in}}%
\pgfpathlineto{\pgfqpoint{2.330590in}{1.016728in}}%
\pgfpathlineto{\pgfqpoint{2.336141in}{1.011520in}}%
\pgfpathlineto{\pgfqpoint{2.341871in}{1.007703in}}%
\pgfpathlineto{\pgfqpoint{2.351622in}{1.000451in}}%
\pgfpathlineto{\pgfqpoint{2.353151in}{0.999433in}}%
\pgfpathlineto{\pgfqpoint{2.364432in}{0.991556in}}%
\pgfpathlineto{\pgfqpoint{2.367403in}{0.989381in}}%
\pgfpathlineto{\pgfqpoint{2.375713in}{0.983508in}}%
\pgfpathlineto{\pgfqpoint{2.382544in}{0.978312in}}%
\pgfpathlineto{\pgfqpoint{2.386994in}{0.975029in}}%
\pgfpathlineto{\pgfqpoint{2.397217in}{0.967242in}}%
\pgfpathlineto{\pgfqpoint{2.398275in}{0.966479in}}%
\pgfpathlineto{\pgfqpoint{2.409555in}{0.958612in}}%
\pgfpathlineto{\pgfqpoint{2.412991in}{0.956173in}}%
\pgfpathlineto{\pgfqpoint{2.420836in}{0.950798in}}%
\pgfpathlineto{\pgfqpoint{2.429061in}{0.945103in}}%
\pgfpathlineto{\pgfqpoint{2.432117in}{0.943088in}}%
\pgfpathlineto{\pgfqpoint{2.443398in}{0.936697in}}%
\pgfpathlineto{\pgfqpoint{2.448307in}{0.934033in}}%
\pgfpathlineto{\pgfqpoint{2.454679in}{0.930537in}}%
\pgfpathlineto{\pgfqpoint{2.465960in}{0.924337in}}%
\pgfpathlineto{\pgfqpoint{2.468512in}{0.922964in}}%
\pgfpathlineto{\pgfqpoint{2.477240in}{0.918067in}}%
\pgfpathlineto{\pgfqpoint{2.488521in}{0.911973in}}%
\pgfpathlineto{\pgfqpoint{2.488656in}{0.911894in}}%
\pgfpathlineto{\pgfqpoint{2.499802in}{0.904075in}}%
\pgfpathlineto{\pgfqpoint{2.504340in}{0.900825in}}%
\pgfpathlineto{\pgfqpoint{2.511083in}{0.896003in}}%
\pgfpathlineto{\pgfqpoint{2.521485in}{0.889755in}}%
\pgfpathlineto{\pgfqpoint{2.522364in}{0.889203in}}%
\pgfpathlineto{\pgfqpoint{2.533644in}{0.882585in}}%
\pgfpathlineto{\pgfqpoint{2.541676in}{0.878686in}}%
\pgfpathlineto{\pgfqpoint{2.544925in}{0.876965in}}%
\pgfpathlineto{\pgfqpoint{2.556206in}{0.870993in}}%
\pgfpathlineto{\pgfqpoint{2.563590in}{0.867616in}}%
\pgfpathlineto{\pgfqpoint{2.567487in}{0.865614in}}%
\pgfpathlineto{\pgfqpoint{2.578768in}{0.859639in}}%
\pgfpathlineto{\pgfqpoint{2.585136in}{0.856547in}}%
\pgfpathlineto{\pgfqpoint{2.590048in}{0.854225in}}%
\pgfpathlineto{\pgfqpoint{2.601329in}{0.850325in}}%
\pgfpathlineto{\pgfqpoint{2.612610in}{0.847785in}}%
\pgfpathlineto{\pgfqpoint{2.623891in}{0.846053in}}%
\pgfpathlineto{\pgfqpoint{2.629694in}{0.845477in}}%
\pgfpathclose%
\pgfusepath{fill}%
\end{pgfscope}%
\begin{pgfscope}%
\pgfpathrectangle{\pgfqpoint{1.856795in}{0.423750in}}{\pgfqpoint{1.194205in}{1.163386in}}%
\pgfusepath{clip}%
\pgfsetbuttcap%
\pgfsetroundjoin%
\definecolor{currentfill}{rgb}{0.970718,0.821518,0.719872}%
\pgfsetfillcolor{currentfill}%
\pgfsetlinewidth{0.000000pt}%
\definecolor{currentstroke}{rgb}{0.000000,0.000000,0.000000}%
\pgfsetstrokecolor{currentstroke}%
\pgfsetdash{}{0pt}%
\pgfpathmoveto{\pgfqpoint{2.747980in}{0.491252in}}%
\pgfpathlineto{\pgfqpoint{2.759261in}{0.491252in}}%
\pgfpathlineto{\pgfqpoint{2.770542in}{0.491252in}}%
\pgfpathlineto{\pgfqpoint{2.781822in}{0.491252in}}%
\pgfpathlineto{\pgfqpoint{2.793103in}{0.491252in}}%
\pgfpathlineto{\pgfqpoint{2.804384in}{0.491252in}}%
\pgfpathlineto{\pgfqpoint{2.815665in}{0.491252in}}%
\pgfpathlineto{\pgfqpoint{2.826946in}{0.491252in}}%
\pgfpathlineto{\pgfqpoint{2.838226in}{0.491252in}}%
\pgfpathlineto{\pgfqpoint{2.849507in}{0.491252in}}%
\pgfpathlineto{\pgfqpoint{2.860788in}{0.491252in}}%
\pgfpathlineto{\pgfqpoint{2.872069in}{0.491252in}}%
\pgfpathlineto{\pgfqpoint{2.883350in}{0.491252in}}%
\pgfpathlineto{\pgfqpoint{2.894631in}{0.491252in}}%
\pgfpathlineto{\pgfqpoint{2.905911in}{0.491252in}}%
\pgfpathlineto{\pgfqpoint{2.917192in}{0.491252in}}%
\pgfpathlineto{\pgfqpoint{2.928473in}{0.491252in}}%
\pgfpathlineto{\pgfqpoint{2.939754in}{0.491252in}}%
\pgfpathlineto{\pgfqpoint{2.951035in}{0.491252in}}%
\pgfpathlineto{\pgfqpoint{2.962315in}{0.491252in}}%
\pgfpathlineto{\pgfqpoint{2.973596in}{0.491252in}}%
\pgfpathlineto{\pgfqpoint{2.973596in}{0.502321in}}%
\pgfpathlineto{\pgfqpoint{2.973596in}{0.513391in}}%
\pgfpathlineto{\pgfqpoint{2.973596in}{0.524460in}}%
\pgfpathlineto{\pgfqpoint{2.973596in}{0.535530in}}%
\pgfpathlineto{\pgfqpoint{2.973596in}{0.546600in}}%
\pgfpathlineto{\pgfqpoint{2.973596in}{0.557669in}}%
\pgfpathlineto{\pgfqpoint{2.973596in}{0.568739in}}%
\pgfpathlineto{\pgfqpoint{2.973596in}{0.579808in}}%
\pgfpathlineto{\pgfqpoint{2.973596in}{0.590878in}}%
\pgfpathlineto{\pgfqpoint{2.973596in}{0.600160in}}%
\pgfpathlineto{\pgfqpoint{2.967599in}{0.590878in}}%
\pgfpathlineto{\pgfqpoint{2.962315in}{0.582748in}}%
\pgfpathlineto{\pgfqpoint{2.960273in}{0.579808in}}%
\pgfpathlineto{\pgfqpoint{2.951035in}{0.571304in}}%
\pgfpathlineto{\pgfqpoint{2.947517in}{0.568739in}}%
\pgfpathlineto{\pgfqpoint{2.939754in}{0.564076in}}%
\pgfpathlineto{\pgfqpoint{2.929217in}{0.557669in}}%
\pgfpathlineto{\pgfqpoint{2.928473in}{0.557227in}}%
\pgfpathlineto{\pgfqpoint{2.917192in}{0.551880in}}%
\pgfpathlineto{\pgfqpoint{2.908071in}{0.546600in}}%
\pgfpathlineto{\pgfqpoint{2.905911in}{0.545511in}}%
\pgfpathlineto{\pgfqpoint{2.894631in}{0.545414in}}%
\pgfpathlineto{\pgfqpoint{2.892786in}{0.546600in}}%
\pgfpathlineto{\pgfqpoint{2.883350in}{0.553504in}}%
\pgfpathlineto{\pgfqpoint{2.878005in}{0.557669in}}%
\pgfpathlineto{\pgfqpoint{2.872069in}{0.563129in}}%
\pgfpathlineto{\pgfqpoint{2.867204in}{0.568739in}}%
\pgfpathlineto{\pgfqpoint{2.860788in}{0.578826in}}%
\pgfpathlineto{\pgfqpoint{2.860101in}{0.579808in}}%
\pgfpathlineto{\pgfqpoint{2.855549in}{0.590878in}}%
\pgfpathlineto{\pgfqpoint{2.852753in}{0.601947in}}%
\pgfpathlineto{\pgfqpoint{2.851829in}{0.613017in}}%
\pgfpathlineto{\pgfqpoint{2.851663in}{0.624086in}}%
\pgfpathlineto{\pgfqpoint{2.853670in}{0.635156in}}%
\pgfpathlineto{\pgfqpoint{2.856559in}{0.646225in}}%
\pgfpathlineto{\pgfqpoint{2.860099in}{0.657295in}}%
\pgfpathlineto{\pgfqpoint{2.860788in}{0.659570in}}%
\pgfpathlineto{\pgfqpoint{2.863131in}{0.668364in}}%
\pgfpathlineto{\pgfqpoint{2.865590in}{0.679434in}}%
\pgfpathlineto{\pgfqpoint{2.867690in}{0.690504in}}%
\pgfpathlineto{\pgfqpoint{2.870559in}{0.701573in}}%
\pgfpathlineto{\pgfqpoint{2.872069in}{0.704497in}}%
\pgfpathlineto{\pgfqpoint{2.876483in}{0.712643in}}%
\pgfpathlineto{\pgfqpoint{2.882815in}{0.723712in}}%
\pgfpathlineto{\pgfqpoint{2.883350in}{0.724318in}}%
\pgfpathlineto{\pgfqpoint{2.891589in}{0.734782in}}%
\pgfpathlineto{\pgfqpoint{2.894631in}{0.737851in}}%
\pgfpathlineto{\pgfqpoint{2.902500in}{0.745851in}}%
\pgfpathlineto{\pgfqpoint{2.905911in}{0.748552in}}%
\pgfpathlineto{\pgfqpoint{2.917192in}{0.756905in}}%
\pgfpathlineto{\pgfqpoint{2.917214in}{0.756921in}}%
\pgfpathlineto{\pgfqpoint{2.928473in}{0.765122in}}%
\pgfpathlineto{\pgfqpoint{2.932332in}{0.767990in}}%
\pgfpathlineto{\pgfqpoint{2.939754in}{0.773756in}}%
\pgfpathlineto{\pgfqpoint{2.944708in}{0.779060in}}%
\pgfpathlineto{\pgfqpoint{2.951035in}{0.787233in}}%
\pgfpathlineto{\pgfqpoint{2.953282in}{0.790129in}}%
\pgfpathlineto{\pgfqpoint{2.960447in}{0.801199in}}%
\pgfpathlineto{\pgfqpoint{2.962315in}{0.804399in}}%
\pgfpathlineto{\pgfqpoint{2.967951in}{0.812269in}}%
\pgfpathlineto{\pgfqpoint{2.972540in}{0.823338in}}%
\pgfpathlineto{\pgfqpoint{2.973596in}{0.825323in}}%
\pgfpathlineto{\pgfqpoint{2.973596in}{0.834408in}}%
\pgfpathlineto{\pgfqpoint{2.973596in}{0.845477in}}%
\pgfpathlineto{\pgfqpoint{2.973596in}{0.856547in}}%
\pgfpathlineto{\pgfqpoint{2.973596in}{0.867616in}}%
\pgfpathlineto{\pgfqpoint{2.973596in}{0.878686in}}%
\pgfpathlineto{\pgfqpoint{2.973596in}{0.889755in}}%
\pgfpathlineto{\pgfqpoint{2.973596in}{0.900825in}}%
\pgfpathlineto{\pgfqpoint{2.973596in}{0.911894in}}%
\pgfpathlineto{\pgfqpoint{2.973596in}{0.922964in}}%
\pgfpathlineto{\pgfqpoint{2.973596in}{0.934033in}}%
\pgfpathlineto{\pgfqpoint{2.973596in}{0.945103in}}%
\pgfpathlineto{\pgfqpoint{2.973596in}{0.956173in}}%
\pgfpathlineto{\pgfqpoint{2.973596in}{0.967242in}}%
\pgfpathlineto{\pgfqpoint{2.973596in}{0.978312in}}%
\pgfpathlineto{\pgfqpoint{2.973596in}{0.989381in}}%
\pgfpathlineto{\pgfqpoint{2.973596in}{1.000451in}}%
\pgfpathlineto{\pgfqpoint{2.973596in}{1.011520in}}%
\pgfpathlineto{\pgfqpoint{2.973596in}{1.022590in}}%
\pgfpathlineto{\pgfqpoint{2.973596in}{1.026112in}}%
\pgfpathlineto{\pgfqpoint{2.970312in}{1.022590in}}%
\pgfpathlineto{\pgfqpoint{2.962315in}{1.014061in}}%
\pgfpathlineto{\pgfqpoint{2.959819in}{1.011520in}}%
\pgfpathlineto{\pgfqpoint{2.951035in}{1.002709in}}%
\pgfpathlineto{\pgfqpoint{2.948666in}{1.000451in}}%
\pgfpathlineto{\pgfqpoint{2.939754in}{0.990521in}}%
\pgfpathlineto{\pgfqpoint{2.938555in}{0.989381in}}%
\pgfpathlineto{\pgfqpoint{2.929065in}{0.978312in}}%
\pgfpathlineto{\pgfqpoint{2.928473in}{0.977617in}}%
\pgfpathlineto{\pgfqpoint{2.919306in}{0.967242in}}%
\pgfpathlineto{\pgfqpoint{2.917192in}{0.964966in}}%
\pgfpathlineto{\pgfqpoint{2.908611in}{0.956173in}}%
\pgfpathlineto{\pgfqpoint{2.905911in}{0.953397in}}%
\pgfpathlineto{\pgfqpoint{2.897596in}{0.945103in}}%
\pgfpathlineto{\pgfqpoint{2.894631in}{0.942158in}}%
\pgfpathlineto{\pgfqpoint{2.886267in}{0.934033in}}%
\pgfpathlineto{\pgfqpoint{2.883350in}{0.931235in}}%
\pgfpathlineto{\pgfqpoint{2.874277in}{0.922964in}}%
\pgfpathlineto{\pgfqpoint{2.872069in}{0.920710in}}%
\pgfpathlineto{\pgfqpoint{2.863555in}{0.911894in}}%
\pgfpathlineto{\pgfqpoint{2.860788in}{0.908003in}}%
\pgfpathlineto{\pgfqpoint{2.854615in}{0.900825in}}%
\pgfpathlineto{\pgfqpoint{2.849507in}{0.894572in}}%
\pgfpathlineto{\pgfqpoint{2.844825in}{0.889755in}}%
\pgfpathlineto{\pgfqpoint{2.838226in}{0.881235in}}%
\pgfpathlineto{\pgfqpoint{2.836218in}{0.878686in}}%
\pgfpathlineto{\pgfqpoint{2.826946in}{0.869505in}}%
\pgfpathlineto{\pgfqpoint{2.824588in}{0.867616in}}%
\pgfpathlineto{\pgfqpoint{2.815665in}{0.862039in}}%
\pgfpathlineto{\pgfqpoint{2.804384in}{0.856558in}}%
\pgfpathlineto{\pgfqpoint{2.804359in}{0.856547in}}%
\pgfpathlineto{\pgfqpoint{2.793103in}{0.851502in}}%
\pgfpathlineto{\pgfqpoint{2.781822in}{0.847762in}}%
\pgfpathlineto{\pgfqpoint{2.775662in}{0.845477in}}%
\pgfpathlineto{\pgfqpoint{2.770542in}{0.843664in}}%
\pgfpathlineto{\pgfqpoint{2.759261in}{0.841081in}}%
\pgfpathlineto{\pgfqpoint{2.747980in}{0.839560in}}%
\pgfpathlineto{\pgfqpoint{2.736699in}{0.838743in}}%
\pgfpathlineto{\pgfqpoint{2.725418in}{0.838387in}}%
\pgfpathlineto{\pgfqpoint{2.714137in}{0.838106in}}%
\pgfpathlineto{\pgfqpoint{2.702857in}{0.839182in}}%
\pgfpathlineto{\pgfqpoint{2.691576in}{0.841043in}}%
\pgfpathlineto{\pgfqpoint{2.680295in}{0.842672in}}%
\pgfpathlineto{\pgfqpoint{2.669014in}{0.842518in}}%
\pgfpathlineto{\pgfqpoint{2.657733in}{0.843235in}}%
\pgfpathlineto{\pgfqpoint{2.646453in}{0.844110in}}%
\pgfpathlineto{\pgfqpoint{2.635172in}{0.844939in}}%
\pgfpathlineto{\pgfqpoint{2.629694in}{0.845477in}}%
\pgfpathlineto{\pgfqpoint{2.623891in}{0.846053in}}%
\pgfpathlineto{\pgfqpoint{2.612610in}{0.847785in}}%
\pgfpathlineto{\pgfqpoint{2.601329in}{0.850325in}}%
\pgfpathlineto{\pgfqpoint{2.590048in}{0.854225in}}%
\pgfpathlineto{\pgfqpoint{2.585136in}{0.856547in}}%
\pgfpathlineto{\pgfqpoint{2.578768in}{0.859639in}}%
\pgfpathlineto{\pgfqpoint{2.567487in}{0.865614in}}%
\pgfpathlineto{\pgfqpoint{2.563590in}{0.867616in}}%
\pgfpathlineto{\pgfqpoint{2.556206in}{0.870993in}}%
\pgfpathlineto{\pgfqpoint{2.544925in}{0.876965in}}%
\pgfpathlineto{\pgfqpoint{2.541676in}{0.878686in}}%
\pgfpathlineto{\pgfqpoint{2.533644in}{0.882585in}}%
\pgfpathlineto{\pgfqpoint{2.522364in}{0.889203in}}%
\pgfpathlineto{\pgfqpoint{2.521485in}{0.889755in}}%
\pgfpathlineto{\pgfqpoint{2.511083in}{0.896003in}}%
\pgfpathlineto{\pgfqpoint{2.504340in}{0.900825in}}%
\pgfpathlineto{\pgfqpoint{2.499802in}{0.904075in}}%
\pgfpathlineto{\pgfqpoint{2.488656in}{0.911894in}}%
\pgfpathlineto{\pgfqpoint{2.488521in}{0.911973in}}%
\pgfpathlineto{\pgfqpoint{2.477240in}{0.918067in}}%
\pgfpathlineto{\pgfqpoint{2.468512in}{0.922964in}}%
\pgfpathlineto{\pgfqpoint{2.465960in}{0.924337in}}%
\pgfpathlineto{\pgfqpoint{2.454679in}{0.930537in}}%
\pgfpathlineto{\pgfqpoint{2.448307in}{0.934033in}}%
\pgfpathlineto{\pgfqpoint{2.443398in}{0.936697in}}%
\pgfpathlineto{\pgfqpoint{2.432117in}{0.943088in}}%
\pgfpathlineto{\pgfqpoint{2.429061in}{0.945103in}}%
\pgfpathlineto{\pgfqpoint{2.420836in}{0.950798in}}%
\pgfpathlineto{\pgfqpoint{2.412991in}{0.956173in}}%
\pgfpathlineto{\pgfqpoint{2.409555in}{0.958612in}}%
\pgfpathlineto{\pgfqpoint{2.398275in}{0.966479in}}%
\pgfpathlineto{\pgfqpoint{2.397217in}{0.967242in}}%
\pgfpathlineto{\pgfqpoint{2.386994in}{0.975029in}}%
\pgfpathlineto{\pgfqpoint{2.382544in}{0.978312in}}%
\pgfpathlineto{\pgfqpoint{2.375713in}{0.983508in}}%
\pgfpathlineto{\pgfqpoint{2.367403in}{0.989381in}}%
\pgfpathlineto{\pgfqpoint{2.364432in}{0.991556in}}%
\pgfpathlineto{\pgfqpoint{2.353151in}{0.999433in}}%
\pgfpathlineto{\pgfqpoint{2.351622in}{1.000451in}}%
\pgfpathlineto{\pgfqpoint{2.341871in}{1.007703in}}%
\pgfpathlineto{\pgfqpoint{2.336141in}{1.011520in}}%
\pgfpathlineto{\pgfqpoint{2.330590in}{1.016728in}}%
\pgfpathlineto{\pgfqpoint{2.324235in}{1.022590in}}%
\pgfpathlineto{\pgfqpoint{2.319309in}{1.027245in}}%
\pgfpathlineto{\pgfqpoint{2.312878in}{1.033659in}}%
\pgfpathlineto{\pgfqpoint{2.308028in}{1.038536in}}%
\pgfpathlineto{\pgfqpoint{2.302351in}{1.044729in}}%
\pgfpathlineto{\pgfqpoint{2.296747in}{1.051358in}}%
\pgfpathlineto{\pgfqpoint{2.292760in}{1.055798in}}%
\pgfpathlineto{\pgfqpoint{2.285466in}{1.063739in}}%
\pgfpathlineto{\pgfqpoint{2.282622in}{1.066868in}}%
\pgfpathlineto{\pgfqpoint{2.274186in}{1.077033in}}%
\pgfpathlineto{\pgfqpoint{2.273466in}{1.077937in}}%
\pgfpathlineto{\pgfqpoint{2.265191in}{1.089007in}}%
\pgfpathlineto{\pgfqpoint{2.262905in}{1.091951in}}%
\pgfpathlineto{\pgfqpoint{2.256956in}{1.100077in}}%
\pgfpathlineto{\pgfqpoint{2.251624in}{1.106613in}}%
\pgfpathlineto{\pgfqpoint{2.248196in}{1.111146in}}%
\pgfpathlineto{\pgfqpoint{2.240343in}{1.120904in}}%
\pgfpathlineto{\pgfqpoint{2.239451in}{1.122216in}}%
\pgfpathlineto{\pgfqpoint{2.229709in}{1.133285in}}%
\pgfpathlineto{\pgfqpoint{2.229062in}{1.133870in}}%
\pgfpathlineto{\pgfqpoint{2.218382in}{1.144355in}}%
\pgfpathlineto{\pgfqpoint{2.217782in}{1.144852in}}%
\pgfpathlineto{\pgfqpoint{2.207435in}{1.155424in}}%
\pgfpathlineto{\pgfqpoint{2.206501in}{1.156305in}}%
\pgfpathlineto{\pgfqpoint{2.196398in}{1.166494in}}%
\pgfpathlineto{\pgfqpoint{2.195220in}{1.167687in}}%
\pgfpathlineto{\pgfqpoint{2.185561in}{1.177563in}}%
\pgfpathlineto{\pgfqpoint{2.183939in}{1.179308in}}%
\pgfpathlineto{\pgfqpoint{2.175606in}{1.188633in}}%
\pgfpathlineto{\pgfqpoint{2.172658in}{1.192009in}}%
\pgfpathlineto{\pgfqpoint{2.165862in}{1.199702in}}%
\pgfpathlineto{\pgfqpoint{2.161377in}{1.204935in}}%
\pgfpathlineto{\pgfqpoint{2.156262in}{1.210772in}}%
\pgfpathlineto{\pgfqpoint{2.150097in}{1.217902in}}%
\pgfpathlineto{\pgfqpoint{2.146663in}{1.221842in}}%
\pgfpathlineto{\pgfqpoint{2.138816in}{1.231041in}}%
\pgfpathlineto{\pgfqpoint{2.137197in}{1.232911in}}%
\pgfpathlineto{\pgfqpoint{2.127882in}{1.243981in}}%
\pgfpathlineto{\pgfqpoint{2.127535in}{1.244397in}}%
\pgfpathlineto{\pgfqpoint{2.118570in}{1.255050in}}%
\pgfpathlineto{\pgfqpoint{2.116254in}{1.257859in}}%
\pgfpathlineto{\pgfqpoint{2.109400in}{1.266120in}}%
\pgfpathlineto{\pgfqpoint{2.104973in}{1.271686in}}%
\pgfpathlineto{\pgfqpoint{2.100530in}{1.277189in}}%
\pgfpathlineto{\pgfqpoint{2.093693in}{1.286040in}}%
\pgfpathlineto{\pgfqpoint{2.091973in}{1.288259in}}%
\pgfpathlineto{\pgfqpoint{2.083459in}{1.299328in}}%
\pgfpathlineto{\pgfqpoint{2.082412in}{1.300811in}}%
\pgfpathlineto{\pgfqpoint{2.075487in}{1.310398in}}%
\pgfpathlineto{\pgfqpoint{2.071131in}{1.317608in}}%
\pgfpathlineto{\pgfqpoint{2.068679in}{1.321467in}}%
\pgfpathlineto{\pgfqpoint{2.062507in}{1.332537in}}%
\pgfpathlineto{\pgfqpoint{2.059850in}{1.337348in}}%
\pgfpathlineto{\pgfqpoint{2.056363in}{1.343606in}}%
\pgfpathlineto{\pgfqpoint{2.050312in}{1.354676in}}%
\pgfpathlineto{\pgfqpoint{2.048569in}{1.357903in}}%
\pgfpathlineto{\pgfqpoint{2.044277in}{1.365746in}}%
\pgfpathlineto{\pgfqpoint{2.038301in}{1.376815in}}%
\pgfpathlineto{\pgfqpoint{2.037289in}{1.378635in}}%
\pgfpathlineto{\pgfqpoint{2.032101in}{1.387885in}}%
\pgfpathlineto{\pgfqpoint{2.026107in}{1.398954in}}%
\pgfpathlineto{\pgfqpoint{2.026008in}{1.399139in}}%
\pgfpathlineto{\pgfqpoint{2.020195in}{1.410024in}}%
\pgfpathlineto{\pgfqpoint{2.014727in}{1.420374in}}%
\pgfpathlineto{\pgfqpoint{2.014346in}{1.421093in}}%
\pgfpathlineto{\pgfqpoint{2.008617in}{1.432163in}}%
\pgfpathlineto{\pgfqpoint{2.003446in}{1.442262in}}%
\pgfpathlineto{\pgfqpoint{2.002933in}{1.443232in}}%
\pgfpathlineto{\pgfqpoint{1.997192in}{1.454302in}}%
\pgfpathlineto{\pgfqpoint{1.992165in}{1.464142in}}%
\pgfpathlineto{\pgfqpoint{1.991531in}{1.465371in}}%
\pgfpathlineto{\pgfqpoint{1.985888in}{1.476441in}}%
\pgfpathlineto{\pgfqpoint{1.980884in}{1.486347in}}%
\pgfpathlineto{\pgfqpoint{1.980298in}{1.487510in}}%
\pgfpathlineto{\pgfqpoint{1.974179in}{1.498580in}}%
\pgfpathlineto{\pgfqpoint{1.969604in}{1.507039in}}%
\pgfpathlineto{\pgfqpoint{1.968177in}{1.509650in}}%
\pgfpathlineto{\pgfqpoint{1.961606in}{1.520719in}}%
\pgfpathlineto{\pgfqpoint{1.958323in}{1.525808in}}%
\pgfpathlineto{\pgfqpoint{1.954419in}{1.531789in}}%
\pgfpathlineto{\pgfqpoint{1.947428in}{1.542858in}}%
\pgfpathlineto{\pgfqpoint{1.947042in}{1.543449in}}%
\pgfpathlineto{\pgfqpoint{1.940327in}{1.553928in}}%
\pgfpathlineto{\pgfqpoint{1.935761in}{1.560541in}}%
\pgfpathlineto{\pgfqpoint{1.932686in}{1.564997in}}%
\pgfpathlineto{\pgfqpoint{1.925101in}{1.576067in}}%
\pgfpathlineto{\pgfqpoint{1.924480in}{1.576957in}}%
\pgfpathlineto{\pgfqpoint{1.917465in}{1.587136in}}%
\pgfpathlineto{\pgfqpoint{1.913200in}{1.587136in}}%
\pgfpathlineto{\pgfqpoint{1.901919in}{1.587136in}}%
\pgfpathlineto{\pgfqpoint{1.890638in}{1.587136in}}%
\pgfpathlineto{\pgfqpoint{1.879357in}{1.587136in}}%
\pgfpathlineto{\pgfqpoint{1.868076in}{1.587136in}}%
\pgfpathlineto{\pgfqpoint{1.856795in}{1.587136in}}%
\pgfpathlineto{\pgfqpoint{1.856795in}{1.576067in}}%
\pgfpathlineto{\pgfqpoint{1.856795in}{1.564997in}}%
\pgfpathlineto{\pgfqpoint{1.856795in}{1.553928in}}%
\pgfpathlineto{\pgfqpoint{1.856795in}{1.542858in}}%
\pgfpathlineto{\pgfqpoint{1.856795in}{1.531789in}}%
\pgfpathlineto{\pgfqpoint{1.856795in}{1.520719in}}%
\pgfpathlineto{\pgfqpoint{1.856795in}{1.509650in}}%
\pgfpathlineto{\pgfqpoint{1.856795in}{1.498580in}}%
\pgfpathlineto{\pgfqpoint{1.856795in}{1.487510in}}%
\pgfpathlineto{\pgfqpoint{1.856795in}{1.476441in}}%
\pgfpathlineto{\pgfqpoint{1.856795in}{1.465371in}}%
\pgfpathlineto{\pgfqpoint{1.856795in}{1.454302in}}%
\pgfpathlineto{\pgfqpoint{1.856795in}{1.443232in}}%
\pgfpathlineto{\pgfqpoint{1.856795in}{1.432163in}}%
\pgfpathlineto{\pgfqpoint{1.856795in}{1.421093in}}%
\pgfpathlineto{\pgfqpoint{1.856795in}{1.410024in}}%
\pgfpathlineto{\pgfqpoint{1.856795in}{1.398954in}}%
\pgfpathlineto{\pgfqpoint{1.856795in}{1.387885in}}%
\pgfpathlineto{\pgfqpoint{1.856795in}{1.384945in}}%
\pgfpathlineto{\pgfqpoint{1.859900in}{1.376815in}}%
\pgfpathlineto{\pgfqpoint{1.863716in}{1.365746in}}%
\pgfpathlineto{\pgfqpoint{1.867628in}{1.354676in}}%
\pgfpathlineto{\pgfqpoint{1.868076in}{1.353558in}}%
\pgfpathlineto{\pgfqpoint{1.871887in}{1.343606in}}%
\pgfpathlineto{\pgfqpoint{1.876667in}{1.332537in}}%
\pgfpathlineto{\pgfqpoint{1.879357in}{1.327064in}}%
\pgfpathlineto{\pgfqpoint{1.881955in}{1.321467in}}%
\pgfpathlineto{\pgfqpoint{1.887251in}{1.310398in}}%
\pgfpathlineto{\pgfqpoint{1.890638in}{1.302752in}}%
\pgfpathlineto{\pgfqpoint{1.892113in}{1.299328in}}%
\pgfpathlineto{\pgfqpoint{1.896604in}{1.288259in}}%
\pgfpathlineto{\pgfqpoint{1.901101in}{1.277189in}}%
\pgfpathlineto{\pgfqpoint{1.901919in}{1.275206in}}%
\pgfpathlineto{\pgfqpoint{1.905448in}{1.266120in}}%
\pgfpathlineto{\pgfqpoint{1.911130in}{1.255050in}}%
\pgfpathlineto{\pgfqpoint{1.913200in}{1.251712in}}%
\pgfpathlineto{\pgfqpoint{1.917771in}{1.243981in}}%
\pgfpathlineto{\pgfqpoint{1.924480in}{1.232937in}}%
\pgfpathlineto{\pgfqpoint{1.924496in}{1.232911in}}%
\pgfpathlineto{\pgfqpoint{1.930918in}{1.221842in}}%
\pgfpathlineto{\pgfqpoint{1.935761in}{1.213526in}}%
\pgfpathlineto{\pgfqpoint{1.937297in}{1.210772in}}%
\pgfpathlineto{\pgfqpoint{1.943505in}{1.199702in}}%
\pgfpathlineto{\pgfqpoint{1.947042in}{1.193230in}}%
\pgfpathlineto{\pgfqpoint{1.949479in}{1.188633in}}%
\pgfpathlineto{\pgfqpoint{1.955471in}{1.177563in}}%
\pgfpathlineto{\pgfqpoint{1.958323in}{1.172189in}}%
\pgfpathlineto{\pgfqpoint{1.961258in}{1.166494in}}%
\pgfpathlineto{\pgfqpoint{1.966933in}{1.155424in}}%
\pgfpathlineto{\pgfqpoint{1.969604in}{1.150692in}}%
\pgfpathlineto{\pgfqpoint{1.973129in}{1.144355in}}%
\pgfpathlineto{\pgfqpoint{1.979702in}{1.133285in}}%
\pgfpathlineto{\pgfqpoint{1.980884in}{1.131521in}}%
\pgfpathlineto{\pgfqpoint{1.987489in}{1.122216in}}%
\pgfpathlineto{\pgfqpoint{1.992165in}{1.115302in}}%
\pgfpathlineto{\pgfqpoint{1.995364in}{1.111146in}}%
\pgfpathlineto{\pgfqpoint{2.003168in}{1.100077in}}%
\pgfpathlineto{\pgfqpoint{2.003446in}{1.099653in}}%
\pgfpathlineto{\pgfqpoint{2.013284in}{1.089007in}}%
\pgfpathlineto{\pgfqpoint{2.014727in}{1.087219in}}%
\pgfpathlineto{\pgfqpoint{2.025771in}{1.077937in}}%
\pgfpathlineto{\pgfqpoint{2.026008in}{1.077616in}}%
\pgfpathlineto{\pgfqpoint{2.037289in}{1.069241in}}%
\pgfpathlineto{\pgfqpoint{2.040848in}{1.066868in}}%
\pgfpathlineto{\pgfqpoint{2.048569in}{1.058728in}}%
\pgfpathlineto{\pgfqpoint{2.052759in}{1.055798in}}%
\pgfpathlineto{\pgfqpoint{2.059850in}{1.046746in}}%
\pgfpathlineto{\pgfqpoint{2.062002in}{1.044729in}}%
\pgfpathlineto{\pgfqpoint{2.071131in}{1.034975in}}%
\pgfpathlineto{\pgfqpoint{2.073028in}{1.033659in}}%
\pgfpathlineto{\pgfqpoint{2.082412in}{1.022650in}}%
\pgfpathlineto{\pgfqpoint{2.082476in}{1.022590in}}%
\pgfpathlineto{\pgfqpoint{2.085399in}{1.011520in}}%
\pgfpathlineto{\pgfqpoint{2.086146in}{1.000451in}}%
\pgfpathlineto{\pgfqpoint{2.086314in}{0.989381in}}%
\pgfpathlineto{\pgfqpoint{2.091797in}{0.978312in}}%
\pgfpathlineto{\pgfqpoint{2.093693in}{0.975345in}}%
\pgfpathlineto{\pgfqpoint{2.098956in}{0.967242in}}%
\pgfpathlineto{\pgfqpoint{2.104973in}{0.959458in}}%
\pgfpathlineto{\pgfqpoint{2.107565in}{0.956173in}}%
\pgfpathlineto{\pgfqpoint{2.116254in}{0.948005in}}%
\pgfpathlineto{\pgfqpoint{2.119043in}{0.945103in}}%
\pgfpathlineto{\pgfqpoint{2.127535in}{0.937396in}}%
\pgfpathlineto{\pgfqpoint{2.131467in}{0.934033in}}%
\pgfpathlineto{\pgfqpoint{2.138816in}{0.927987in}}%
\pgfpathlineto{\pgfqpoint{2.144698in}{0.922964in}}%
\pgfpathlineto{\pgfqpoint{2.150097in}{0.918926in}}%
\pgfpathlineto{\pgfqpoint{2.158164in}{0.911894in}}%
\pgfpathlineto{\pgfqpoint{2.161377in}{0.909338in}}%
\pgfpathlineto{\pgfqpoint{2.171783in}{0.900825in}}%
\pgfpathlineto{\pgfqpoint{2.172658in}{0.900134in}}%
\pgfpathlineto{\pgfqpoint{2.183939in}{0.890821in}}%
\pgfpathlineto{\pgfqpoint{2.185207in}{0.889755in}}%
\pgfpathlineto{\pgfqpoint{2.195220in}{0.883088in}}%
\pgfpathlineto{\pgfqpoint{2.200775in}{0.878686in}}%
\pgfpathlineto{\pgfqpoint{2.206501in}{0.874888in}}%
\pgfpathlineto{\pgfqpoint{2.216347in}{0.867616in}}%
\pgfpathlineto{\pgfqpoint{2.217782in}{0.866851in}}%
\pgfpathlineto{\pgfqpoint{2.229062in}{0.860456in}}%
\pgfpathlineto{\pgfqpoint{2.234530in}{0.856547in}}%
\pgfpathlineto{\pgfqpoint{2.240343in}{0.853504in}}%
\pgfpathlineto{\pgfqpoint{2.251624in}{0.847415in}}%
\pgfpathlineto{\pgfqpoint{2.254671in}{0.845477in}}%
\pgfpathlineto{\pgfqpoint{2.262905in}{0.842079in}}%
\pgfpathlineto{\pgfqpoint{2.274186in}{0.835857in}}%
\pgfpathlineto{\pgfqpoint{2.276441in}{0.834408in}}%
\pgfpathlineto{\pgfqpoint{2.285466in}{0.830895in}}%
\pgfpathlineto{\pgfqpoint{2.296747in}{0.825950in}}%
\pgfpathlineto{\pgfqpoint{2.301816in}{0.823338in}}%
\pgfpathlineto{\pgfqpoint{2.308028in}{0.820692in}}%
\pgfpathlineto{\pgfqpoint{2.319309in}{0.815893in}}%
\pgfpathlineto{\pgfqpoint{2.328619in}{0.812269in}}%
\pgfpathlineto{\pgfqpoint{2.330590in}{0.811516in}}%
\pgfpathlineto{\pgfqpoint{2.341871in}{0.806988in}}%
\pgfpathlineto{\pgfqpoint{2.353151in}{0.801964in}}%
\pgfpathlineto{\pgfqpoint{2.354876in}{0.801199in}}%
\pgfpathlineto{\pgfqpoint{2.364432in}{0.797122in}}%
\pgfpathlineto{\pgfqpoint{2.375713in}{0.791540in}}%
\pgfpathlineto{\pgfqpoint{2.378447in}{0.790129in}}%
\pgfpathlineto{\pgfqpoint{2.386994in}{0.785885in}}%
\pgfpathlineto{\pgfqpoint{2.398275in}{0.779786in}}%
\pgfpathlineto{\pgfqpoint{2.399469in}{0.779060in}}%
\pgfpathlineto{\pgfqpoint{2.409555in}{0.773475in}}%
\pgfpathlineto{\pgfqpoint{2.419029in}{0.767990in}}%
\pgfpathlineto{\pgfqpoint{2.420836in}{0.766976in}}%
\pgfpathlineto{\pgfqpoint{2.432117in}{0.761078in}}%
\pgfpathlineto{\pgfqpoint{2.439246in}{0.756921in}}%
\pgfpathlineto{\pgfqpoint{2.443398in}{0.754537in}}%
\pgfpathlineto{\pgfqpoint{2.454679in}{0.747486in}}%
\pgfpathlineto{\pgfqpoint{2.457358in}{0.745851in}}%
\pgfpathlineto{\pgfqpoint{2.465960in}{0.740476in}}%
\pgfpathlineto{\pgfqpoint{2.476671in}{0.734782in}}%
\pgfpathlineto{\pgfqpoint{2.477240in}{0.734467in}}%
\pgfpathlineto{\pgfqpoint{2.488521in}{0.728211in}}%
\pgfpathlineto{\pgfqpoint{2.496459in}{0.723712in}}%
\pgfpathlineto{\pgfqpoint{2.499802in}{0.721674in}}%
\pgfpathlineto{\pgfqpoint{2.511083in}{0.714336in}}%
\pgfpathlineto{\pgfqpoint{2.513410in}{0.712643in}}%
\pgfpathlineto{\pgfqpoint{2.522364in}{0.705573in}}%
\pgfpathlineto{\pgfqpoint{2.527840in}{0.701573in}}%
\pgfpathlineto{\pgfqpoint{2.533644in}{0.696815in}}%
\pgfpathlineto{\pgfqpoint{2.541542in}{0.690504in}}%
\pgfpathlineto{\pgfqpoint{2.544925in}{0.687806in}}%
\pgfpathlineto{\pgfqpoint{2.556206in}{0.679681in}}%
\pgfpathlineto{\pgfqpoint{2.556555in}{0.679434in}}%
\pgfpathlineto{\pgfqpoint{2.567487in}{0.672411in}}%
\pgfpathlineto{\pgfqpoint{2.574167in}{0.668364in}}%
\pgfpathlineto{\pgfqpoint{2.578768in}{0.665424in}}%
\pgfpathlineto{\pgfqpoint{2.590048in}{0.657831in}}%
\pgfpathlineto{\pgfqpoint{2.590816in}{0.657295in}}%
\pgfpathlineto{\pgfqpoint{2.601329in}{0.649461in}}%
\pgfpathlineto{\pgfqpoint{2.605148in}{0.646225in}}%
\pgfpathlineto{\pgfqpoint{2.612610in}{0.639789in}}%
\pgfpathlineto{\pgfqpoint{2.617614in}{0.635156in}}%
\pgfpathlineto{\pgfqpoint{2.623891in}{0.629274in}}%
\pgfpathlineto{\pgfqpoint{2.629059in}{0.624086in}}%
\pgfpathlineto{\pgfqpoint{2.635172in}{0.618412in}}%
\pgfpathlineto{\pgfqpoint{2.640484in}{0.613017in}}%
\pgfpathlineto{\pgfqpoint{2.646453in}{0.607454in}}%
\pgfpathlineto{\pgfqpoint{2.652527in}{0.601947in}}%
\pgfpathlineto{\pgfqpoint{2.657733in}{0.597083in}}%
\pgfpathlineto{\pgfqpoint{2.663935in}{0.590878in}}%
\pgfpathlineto{\pgfqpoint{2.669014in}{0.585302in}}%
\pgfpathlineto{\pgfqpoint{2.673859in}{0.579808in}}%
\pgfpathlineto{\pgfqpoint{2.680295in}{0.572337in}}%
\pgfpathlineto{\pgfqpoint{2.683673in}{0.568739in}}%
\pgfpathlineto{\pgfqpoint{2.691576in}{0.559280in}}%
\pgfpathlineto{\pgfqpoint{2.692955in}{0.557669in}}%
\pgfpathlineto{\pgfqpoint{2.702133in}{0.546600in}}%
\pgfpathlineto{\pgfqpoint{2.702857in}{0.545679in}}%
\pgfpathlineto{\pgfqpoint{2.711466in}{0.535530in}}%
\pgfpathlineto{\pgfqpoint{2.714137in}{0.532228in}}%
\pgfpathlineto{\pgfqpoint{2.720745in}{0.524460in}}%
\pgfpathlineto{\pgfqpoint{2.725418in}{0.519007in}}%
\pgfpathlineto{\pgfqpoint{2.730215in}{0.513391in}}%
\pgfpathlineto{\pgfqpoint{2.736699in}{0.505277in}}%
\pgfpathlineto{\pgfqpoint{2.738958in}{0.502321in}}%
\pgfpathlineto{\pgfqpoint{2.746410in}{0.491252in}}%
\pgfpathclose%
\pgfpathmoveto{\pgfqpoint{2.014258in}{1.210772in}}%
\pgfpathlineto{\pgfqpoint{2.003446in}{1.215102in}}%
\pgfpathlineto{\pgfqpoint{1.992165in}{1.220947in}}%
\pgfpathlineto{\pgfqpoint{1.991470in}{1.221842in}}%
\pgfpathlineto{\pgfqpoint{1.983073in}{1.232911in}}%
\pgfpathlineto{\pgfqpoint{1.980884in}{1.239432in}}%
\pgfpathlineto{\pgfqpoint{1.979843in}{1.243981in}}%
\pgfpathlineto{\pgfqpoint{1.977505in}{1.255050in}}%
\pgfpathlineto{\pgfqpoint{1.974544in}{1.266120in}}%
\pgfpathlineto{\pgfqpoint{1.971407in}{1.277189in}}%
\pgfpathlineto{\pgfqpoint{1.980884in}{1.287589in}}%
\pgfpathlineto{\pgfqpoint{1.992165in}{1.281123in}}%
\pgfpathlineto{\pgfqpoint{1.999003in}{1.277189in}}%
\pgfpathlineto{\pgfqpoint{2.003446in}{1.274659in}}%
\pgfpathlineto{\pgfqpoint{2.014727in}{1.267469in}}%
\pgfpathlineto{\pgfqpoint{2.016042in}{1.266120in}}%
\pgfpathlineto{\pgfqpoint{2.026008in}{1.257849in}}%
\pgfpathlineto{\pgfqpoint{2.028609in}{1.255050in}}%
\pgfpathlineto{\pgfqpoint{2.037289in}{1.246336in}}%
\pgfpathlineto{\pgfqpoint{2.039371in}{1.243981in}}%
\pgfpathlineto{\pgfqpoint{2.048569in}{1.233295in}}%
\pgfpathlineto{\pgfqpoint{2.048908in}{1.232911in}}%
\pgfpathlineto{\pgfqpoint{2.053839in}{1.221842in}}%
\pgfpathlineto{\pgfqpoint{2.056474in}{1.210772in}}%
\pgfpathlineto{\pgfqpoint{2.048569in}{1.201528in}}%
\pgfpathlineto{\pgfqpoint{2.037289in}{1.202512in}}%
\pgfpathlineto{\pgfqpoint{2.026008in}{1.206484in}}%
\pgfpathlineto{\pgfqpoint{2.014727in}{1.210589in}}%
\pgfpathclose%
\pgfusepath{fill}%
\end{pgfscope}%
\begin{pgfscope}%
\pgfpathrectangle{\pgfqpoint{1.856795in}{0.423750in}}{\pgfqpoint{1.194205in}{1.163386in}}%
\pgfusepath{clip}%
\pgfsetbuttcap%
\pgfsetroundjoin%
\definecolor{currentfill}{rgb}{0.977657,0.891500,0.822809}%
\pgfsetfillcolor{currentfill}%
\pgfsetlinewidth{0.000000pt}%
\definecolor{currentstroke}{rgb}{0.000000,0.000000,0.000000}%
\pgfsetstrokecolor{currentstroke}%
\pgfsetdash{}{0pt}%
\pgfpathmoveto{\pgfqpoint{2.894631in}{0.545414in}}%
\pgfpathlineto{\pgfqpoint{2.905911in}{0.545511in}}%
\pgfpathlineto{\pgfqpoint{2.908071in}{0.546600in}}%
\pgfpathlineto{\pgfqpoint{2.917192in}{0.551880in}}%
\pgfpathlineto{\pgfqpoint{2.928473in}{0.557227in}}%
\pgfpathlineto{\pgfqpoint{2.929217in}{0.557669in}}%
\pgfpathlineto{\pgfqpoint{2.939754in}{0.564076in}}%
\pgfpathlineto{\pgfqpoint{2.947517in}{0.568739in}}%
\pgfpathlineto{\pgfqpoint{2.951035in}{0.571304in}}%
\pgfpathlineto{\pgfqpoint{2.960273in}{0.579808in}}%
\pgfpathlineto{\pgfqpoint{2.962315in}{0.582748in}}%
\pgfpathlineto{\pgfqpoint{2.967599in}{0.590878in}}%
\pgfpathlineto{\pgfqpoint{2.973596in}{0.600160in}}%
\pgfpathlineto{\pgfqpoint{2.973596in}{0.601947in}}%
\pgfpathlineto{\pgfqpoint{2.973596in}{0.613017in}}%
\pgfpathlineto{\pgfqpoint{2.973596in}{0.624086in}}%
\pgfpathlineto{\pgfqpoint{2.973596in}{0.635156in}}%
\pgfpathlineto{\pgfqpoint{2.973596in}{0.646225in}}%
\pgfpathlineto{\pgfqpoint{2.973596in}{0.657295in}}%
\pgfpathlineto{\pgfqpoint{2.973596in}{0.668364in}}%
\pgfpathlineto{\pgfqpoint{2.973596in}{0.679434in}}%
\pgfpathlineto{\pgfqpoint{2.973596in}{0.690504in}}%
\pgfpathlineto{\pgfqpoint{2.973596in}{0.701573in}}%
\pgfpathlineto{\pgfqpoint{2.973596in}{0.712643in}}%
\pgfpathlineto{\pgfqpoint{2.973596in}{0.723712in}}%
\pgfpathlineto{\pgfqpoint{2.973596in}{0.734782in}}%
\pgfpathlineto{\pgfqpoint{2.973596in}{0.745851in}}%
\pgfpathlineto{\pgfqpoint{2.973596in}{0.756921in}}%
\pgfpathlineto{\pgfqpoint{2.973596in}{0.767990in}}%
\pgfpathlineto{\pgfqpoint{2.973596in}{0.779060in}}%
\pgfpathlineto{\pgfqpoint{2.973596in}{0.790129in}}%
\pgfpathlineto{\pgfqpoint{2.973596in}{0.801199in}}%
\pgfpathlineto{\pgfqpoint{2.973596in}{0.812269in}}%
\pgfpathlineto{\pgfqpoint{2.973596in}{0.823338in}}%
\pgfpathlineto{\pgfqpoint{2.973596in}{0.825323in}}%
\pgfpathlineto{\pgfqpoint{2.972540in}{0.823338in}}%
\pgfpathlineto{\pgfqpoint{2.967951in}{0.812269in}}%
\pgfpathlineto{\pgfqpoint{2.962315in}{0.804399in}}%
\pgfpathlineto{\pgfqpoint{2.960447in}{0.801199in}}%
\pgfpathlineto{\pgfqpoint{2.953282in}{0.790129in}}%
\pgfpathlineto{\pgfqpoint{2.951035in}{0.787233in}}%
\pgfpathlineto{\pgfqpoint{2.944708in}{0.779060in}}%
\pgfpathlineto{\pgfqpoint{2.939754in}{0.773756in}}%
\pgfpathlineto{\pgfqpoint{2.932332in}{0.767990in}}%
\pgfpathlineto{\pgfqpoint{2.928473in}{0.765122in}}%
\pgfpathlineto{\pgfqpoint{2.917214in}{0.756921in}}%
\pgfpathlineto{\pgfqpoint{2.917192in}{0.756905in}}%
\pgfpathlineto{\pgfqpoint{2.905911in}{0.748552in}}%
\pgfpathlineto{\pgfqpoint{2.902500in}{0.745851in}}%
\pgfpathlineto{\pgfqpoint{2.894631in}{0.737851in}}%
\pgfpathlineto{\pgfqpoint{2.891589in}{0.734782in}}%
\pgfpathlineto{\pgfqpoint{2.883350in}{0.724318in}}%
\pgfpathlineto{\pgfqpoint{2.882815in}{0.723712in}}%
\pgfpathlineto{\pgfqpoint{2.876483in}{0.712643in}}%
\pgfpathlineto{\pgfqpoint{2.872069in}{0.704497in}}%
\pgfpathlineto{\pgfqpoint{2.870559in}{0.701573in}}%
\pgfpathlineto{\pgfqpoint{2.867690in}{0.690504in}}%
\pgfpathlineto{\pgfqpoint{2.865590in}{0.679434in}}%
\pgfpathlineto{\pgfqpoint{2.863131in}{0.668364in}}%
\pgfpathlineto{\pgfqpoint{2.860788in}{0.659570in}}%
\pgfpathlineto{\pgfqpoint{2.860099in}{0.657295in}}%
\pgfpathlineto{\pgfqpoint{2.856559in}{0.646225in}}%
\pgfpathlineto{\pgfqpoint{2.853670in}{0.635156in}}%
\pgfpathlineto{\pgfqpoint{2.851663in}{0.624086in}}%
\pgfpathlineto{\pgfqpoint{2.851829in}{0.613017in}}%
\pgfpathlineto{\pgfqpoint{2.852753in}{0.601947in}}%
\pgfpathlineto{\pgfqpoint{2.855549in}{0.590878in}}%
\pgfpathlineto{\pgfqpoint{2.860101in}{0.579808in}}%
\pgfpathlineto{\pgfqpoint{2.860788in}{0.578826in}}%
\pgfpathlineto{\pgfqpoint{2.867204in}{0.568739in}}%
\pgfpathlineto{\pgfqpoint{2.872069in}{0.563129in}}%
\pgfpathlineto{\pgfqpoint{2.878005in}{0.557669in}}%
\pgfpathlineto{\pgfqpoint{2.883350in}{0.553504in}}%
\pgfpathlineto{\pgfqpoint{2.892786in}{0.546600in}}%
\pgfpathclose%
\pgfusepath{fill}%
\end{pgfscope}%
\begin{pgfscope}%
\pgfpathrectangle{\pgfqpoint{1.856795in}{0.423750in}}{\pgfqpoint{1.194205in}{1.163386in}}%
\pgfusepath{clip}%
\pgfsetbuttcap%
\pgfsetroundjoin%
\definecolor{currentfill}{rgb}{0.977657,0.891500,0.822809}%
\pgfsetfillcolor{currentfill}%
\pgfsetlinewidth{0.000000pt}%
\definecolor{currentstroke}{rgb}{0.000000,0.000000,0.000000}%
\pgfsetstrokecolor{currentstroke}%
\pgfsetdash{}{0pt}%
\pgfpathmoveto{\pgfqpoint{2.014727in}{1.210589in}}%
\pgfpathlineto{\pgfqpoint{2.026008in}{1.206484in}}%
\pgfpathlineto{\pgfqpoint{2.037289in}{1.202512in}}%
\pgfpathlineto{\pgfqpoint{2.048569in}{1.201528in}}%
\pgfpathlineto{\pgfqpoint{2.056474in}{1.210772in}}%
\pgfpathlineto{\pgfqpoint{2.053839in}{1.221842in}}%
\pgfpathlineto{\pgfqpoint{2.048908in}{1.232911in}}%
\pgfpathlineto{\pgfqpoint{2.048569in}{1.233295in}}%
\pgfpathlineto{\pgfqpoint{2.039371in}{1.243981in}}%
\pgfpathlineto{\pgfqpoint{2.037289in}{1.246336in}}%
\pgfpathlineto{\pgfqpoint{2.028609in}{1.255050in}}%
\pgfpathlineto{\pgfqpoint{2.026008in}{1.257849in}}%
\pgfpathlineto{\pgfqpoint{2.016042in}{1.266120in}}%
\pgfpathlineto{\pgfqpoint{2.014727in}{1.267469in}}%
\pgfpathlineto{\pgfqpoint{2.003446in}{1.274659in}}%
\pgfpathlineto{\pgfqpoint{1.999003in}{1.277189in}}%
\pgfpathlineto{\pgfqpoint{1.992165in}{1.281123in}}%
\pgfpathlineto{\pgfqpoint{1.980884in}{1.287589in}}%
\pgfpathlineto{\pgfqpoint{1.971407in}{1.277189in}}%
\pgfpathlineto{\pgfqpoint{1.974544in}{1.266120in}}%
\pgfpathlineto{\pgfqpoint{1.977505in}{1.255050in}}%
\pgfpathlineto{\pgfqpoint{1.979843in}{1.243981in}}%
\pgfpathlineto{\pgfqpoint{1.980884in}{1.239432in}}%
\pgfpathlineto{\pgfqpoint{1.983073in}{1.232911in}}%
\pgfpathlineto{\pgfqpoint{1.991470in}{1.221842in}}%
\pgfpathlineto{\pgfqpoint{1.992165in}{1.220947in}}%
\pgfpathlineto{\pgfqpoint{2.003446in}{1.215102in}}%
\pgfpathlineto{\pgfqpoint{2.014258in}{1.210772in}}%
\pgfpathclose%
\pgfusepath{fill}%
\end{pgfscope}%
\begin{pgfscope}%
\pgfpathrectangle{\pgfqpoint{1.856795in}{0.423750in}}{\pgfqpoint{1.194205in}{1.163386in}}%
\pgfusepath{clip}%
\pgfsetbuttcap%
\pgfsetroundjoin%
\definecolor{currentfill}{rgb}{0.121569,0.466667,0.705882}%
\pgfsetfillcolor{currentfill}%
\pgfsetlinewidth{1.003750pt}%
\definecolor{currentstroke}{rgb}{0.121569,0.466667,0.705882}%
\pgfsetstrokecolor{currentstroke}%
\pgfsetdash{}{0pt}%
\pgfsys@defobject{currentmarker}{\pgfqpoint{-0.021960in}{-0.021960in}}{\pgfqpoint{0.021960in}{0.021960in}}{%
\pgfpathmoveto{\pgfqpoint{0.000000in}{-0.021960in}}%
\pgfpathcurveto{\pgfqpoint{0.005824in}{-0.021960in}}{\pgfqpoint{0.011410in}{-0.019646in}}{\pgfqpoint{0.015528in}{-0.015528in}}%
\pgfpathcurveto{\pgfqpoint{0.019646in}{-0.011410in}}{\pgfqpoint{0.021960in}{-0.005824in}}{\pgfqpoint{0.021960in}{0.000000in}}%
\pgfpathcurveto{\pgfqpoint{0.021960in}{0.005824in}}{\pgfqpoint{0.019646in}{0.011410in}}{\pgfqpoint{0.015528in}{0.015528in}}%
\pgfpathcurveto{\pgfqpoint{0.011410in}{0.019646in}}{\pgfqpoint{0.005824in}{0.021960in}}{\pgfqpoint{0.000000in}{0.021960in}}%
\pgfpathcurveto{\pgfqpoint{-0.005824in}{0.021960in}}{\pgfqpoint{-0.011410in}{0.019646in}}{\pgfqpoint{-0.015528in}{0.015528in}}%
\pgfpathcurveto{\pgfqpoint{-0.019646in}{0.011410in}}{\pgfqpoint{-0.021960in}{0.005824in}}{\pgfqpoint{-0.021960in}{0.000000in}}%
\pgfpathcurveto{\pgfqpoint{-0.021960in}{-0.005824in}}{\pgfqpoint{-0.019646in}{-0.011410in}}{\pgfqpoint{-0.015528in}{-0.015528in}}%
\pgfpathcurveto{\pgfqpoint{-0.011410in}{-0.019646in}}{\pgfqpoint{-0.005824in}{-0.021960in}}{\pgfqpoint{0.000000in}{-0.021960in}}%
\pgfpathclose%
\pgfusepath{stroke,fill}%
}%
\begin{pgfscope}%
\pgfsys@transformshift{2.689007in}{1.005230in}%
\pgfsys@useobject{currentmarker}{}%
\end{pgfscope}%
\begin{pgfscope}%
\pgfsys@transformshift{2.807685in}{0.500893in}%
\pgfsys@useobject{currentmarker}{}%
\end{pgfscope}%
\begin{pgfscope}%
\pgfsys@transformshift{1.959442in}{1.020971in}%
\pgfsys@useobject{currentmarker}{}%
\end{pgfscope}%
\begin{pgfscope}%
\pgfsys@transformshift{2.551292in}{0.818105in}%
\pgfsys@useobject{currentmarker}{}%
\end{pgfscope}%
\begin{pgfscope}%
\pgfsys@transformshift{2.825776in}{1.347080in}%
\pgfsys@useobject{currentmarker}{}%
\end{pgfscope}%
\begin{pgfscope}%
\pgfsys@transformshift{2.607065in}{1.260810in}%
\pgfsys@useobject{currentmarker}{}%
\end{pgfscope}%
\begin{pgfscope}%
\pgfsys@transformshift{2.007090in}{1.258985in}%
\pgfsys@useobject{currentmarker}{}%
\end{pgfscope}%
\begin{pgfscope}%
\pgfsys@transformshift{2.553107in}{1.379242in}%
\pgfsys@useobject{currentmarker}{}%
\end{pgfscope}%
\begin{pgfscope}%
\pgfsys@transformshift{2.368017in}{1.093568in}%
\pgfsys@useobject{currentmarker}{}%
\end{pgfscope}%
\begin{pgfscope}%
\pgfsys@transformshift{2.339025in}{1.378425in}%
\pgfsys@useobject{currentmarker}{}%
\end{pgfscope}%
\begin{pgfscope}%
\pgfsys@transformshift{2.199834in}{1.256890in}%
\pgfsys@useobject{currentmarker}{}%
\end{pgfscope}%
\begin{pgfscope}%
\pgfsys@transformshift{2.718230in}{1.277017in}%
\pgfsys@useobject{currentmarker}{}%
\end{pgfscope}%
\begin{pgfscope}%
\pgfsys@transformshift{1.899977in}{0.859307in}%
\pgfsys@useobject{currentmarker}{}%
\end{pgfscope}%
\begin{pgfscope}%
\pgfsys@transformshift{1.896489in}{1.086535in}%
\pgfsys@useobject{currentmarker}{}%
\end{pgfscope}%
\begin{pgfscope}%
\pgfsys@transformshift{2.165807in}{0.999949in}%
\pgfsys@useobject{currentmarker}{}%
\end{pgfscope}%
\begin{pgfscope}%
\pgfsys@transformshift{2.667183in}{0.590330in}%
\pgfsys@useobject{currentmarker}{}%
\end{pgfscope}%
\begin{pgfscope}%
\pgfsys@transformshift{2.882063in}{1.320506in}%
\pgfsys@useobject{currentmarker}{}%
\end{pgfscope}%
\begin{pgfscope}%
\pgfsys@transformshift{2.917667in}{1.542219in}%
\pgfsys@useobject{currentmarker}{}%
\end{pgfscope}%
\begin{pgfscope}%
\pgfsys@transformshift{2.194854in}{1.463074in}%
\pgfsys@useobject{currentmarker}{}%
\end{pgfscope}%
\begin{pgfscope}%
\pgfsys@transformshift{1.999674in}{1.122642in}%
\pgfsys@useobject{currentmarker}{}%
\end{pgfscope}%
\begin{pgfscope}%
\pgfsys@transformshift{2.973596in}{0.816427in}%
\pgfsys@useobject{currentmarker}{}%
\end{pgfscope}%
\begin{pgfscope}%
\pgfsys@transformshift{2.961682in}{0.612454in}%
\pgfsys@useobject{currentmarker}{}%
\end{pgfscope}%
\begin{pgfscope}%
\pgfsys@transformshift{2.973596in}{1.109173in}%
\pgfsys@useobject{currentmarker}{}%
\end{pgfscope}%
\begin{pgfscope}%
\pgfsys@transformshift{2.871874in}{0.665700in}%
\pgfsys@useobject{currentmarker}{}%
\end{pgfscope}%
\begin{pgfscope}%
\pgfsys@transformshift{2.964318in}{0.534889in}%
\pgfsys@useobject{currentmarker}{}%
\end{pgfscope}%
\begin{pgfscope}%
\pgfsys@transformshift{2.058287in}{1.198628in}%
\pgfsys@useobject{currentmarker}{}%
\end{pgfscope}%
\begin{pgfscope}%
\pgfsys@transformshift{2.973596in}{0.707270in}%
\pgfsys@useobject{currentmarker}{}%
\end{pgfscope}%
\begin{pgfscope}%
\pgfsys@transformshift{2.973596in}{0.661205in}%
\pgfsys@useobject{currentmarker}{}%
\end{pgfscope}%
\begin{pgfscope}%
\pgfsys@transformshift{2.973596in}{0.691318in}%
\pgfsys@useobject{currentmarker}{}%
\end{pgfscope}%
\begin{pgfscope}%
\pgfsys@transformshift{2.159208in}{1.095709in}%
\pgfsys@useobject{currentmarker}{}%
\end{pgfscope}%
\begin{pgfscope}%
\pgfsys@transformshift{2.840021in}{0.828995in}%
\pgfsys@useobject{currentmarker}{}%
\end{pgfscope}%
\begin{pgfscope}%
\pgfsys@transformshift{2.973596in}{0.775462in}%
\pgfsys@useobject{currentmarker}{}%
\end{pgfscope}%
\begin{pgfscope}%
\pgfsys@transformshift{2.973596in}{0.605618in}%
\pgfsys@useobject{currentmarker}{}%
\end{pgfscope}%
\begin{pgfscope}%
\pgfsys@transformshift{2.973596in}{0.707183in}%
\pgfsys@useobject{currentmarker}{}%
\end{pgfscope}%
\begin{pgfscope}%
\pgfsys@transformshift{2.973596in}{0.683089in}%
\pgfsys@useobject{currentmarker}{}%
\end{pgfscope}%
\begin{pgfscope}%
\pgfsys@transformshift{1.994425in}{1.272660in}%
\pgfsys@useobject{currentmarker}{}%
\end{pgfscope}%
\begin{pgfscope}%
\pgfsys@transformshift{2.973596in}{0.646095in}%
\pgfsys@useobject{currentmarker}{}%
\end{pgfscope}%
\begin{pgfscope}%
\pgfsys@transformshift{2.033690in}{1.229196in}%
\pgfsys@useobject{currentmarker}{}%
\end{pgfscope}%
\begin{pgfscope}%
\pgfsys@transformshift{2.973596in}{0.618303in}%
\pgfsys@useobject{currentmarker}{}%
\end{pgfscope}%
\begin{pgfscope}%
\pgfsys@transformshift{2.950783in}{0.713338in}%
\pgfsys@useobject{currentmarker}{}%
\end{pgfscope}%
\begin{pgfscope}%
\pgfsys@transformshift{2.972112in}{0.729848in}%
\pgfsys@useobject{currentmarker}{}%
\end{pgfscope}%
\begin{pgfscope}%
\pgfsys@transformshift{2.973596in}{0.817388in}%
\pgfsys@useobject{currentmarker}{}%
\end{pgfscope}%
\begin{pgfscope}%
\pgfsys@transformshift{2.900634in}{0.702652in}%
\pgfsys@useobject{currentmarker}{}%
\end{pgfscope}%
\begin{pgfscope}%
\pgfsys@transformshift{2.973596in}{0.690686in}%
\pgfsys@useobject{currentmarker}{}%
\end{pgfscope}%
\begin{pgfscope}%
\pgfsys@transformshift{2.007821in}{1.249937in}%
\pgfsys@useobject{currentmarker}{}%
\end{pgfscope}%
\begin{pgfscope}%
\pgfsys@transformshift{2.049072in}{1.200753in}%
\pgfsys@useobject{currentmarker}{}%
\end{pgfscope}%
\begin{pgfscope}%
\pgfsys@transformshift{2.011531in}{1.249610in}%
\pgfsys@useobject{currentmarker}{}%
\end{pgfscope}%
\begin{pgfscope}%
\pgfsys@transformshift{2.933473in}{0.709190in}%
\pgfsys@useobject{currentmarker}{}%
\end{pgfscope}%
\begin{pgfscope}%
\pgfsys@transformshift{1.998005in}{1.246909in}%
\pgfsys@useobject{currentmarker}{}%
\end{pgfscope}%
\begin{pgfscope}%
\pgfsys@transformshift{2.921353in}{0.690448in}%
\pgfsys@useobject{currentmarker}{}%
\end{pgfscope}%
\begin{pgfscope}%
\pgfsys@transformshift{2.967188in}{0.703999in}%
\pgfsys@useobject{currentmarker}{}%
\end{pgfscope}%
\begin{pgfscope}%
\pgfsys@transformshift{2.929624in}{0.708758in}%
\pgfsys@useobject{currentmarker}{}%
\end{pgfscope}%
\begin{pgfscope}%
\pgfsys@transformshift{2.058411in}{1.221712in}%
\pgfsys@useobject{currentmarker}{}%
\end{pgfscope}%
\begin{pgfscope}%
\pgfsys@transformshift{2.914891in}{0.666164in}%
\pgfsys@useobject{currentmarker}{}%
\end{pgfscope}%
\begin{pgfscope}%
\pgfsys@transformshift{2.956912in}{0.646673in}%
\pgfsys@useobject{currentmarker}{}%
\end{pgfscope}%
\begin{pgfscope}%
\pgfsys@transformshift{2.945141in}{0.996907in}%
\pgfsys@useobject{currentmarker}{}%
\end{pgfscope}%
\begin{pgfscope}%
\pgfsys@transformshift{2.973596in}{0.675147in}%
\pgfsys@useobject{currentmarker}{}%
\end{pgfscope}%
\begin{pgfscope}%
\pgfsys@transformshift{2.920804in}{0.688220in}%
\pgfsys@useobject{currentmarker}{}%
\end{pgfscope}%
\begin{pgfscope}%
\pgfsys@transformshift{2.023356in}{1.249932in}%
\pgfsys@useobject{currentmarker}{}%
\end{pgfscope}%
\begin{pgfscope}%
\pgfsys@transformshift{2.933604in}{0.727976in}%
\pgfsys@useobject{currentmarker}{}%
\end{pgfscope}%
\begin{pgfscope}%
\pgfsys@transformshift{2.948030in}{0.705228in}%
\pgfsys@useobject{currentmarker}{}%
\end{pgfscope}%
\begin{pgfscope}%
\pgfsys@transformshift{1.992052in}{1.277088in}%
\pgfsys@useobject{currentmarker}{}%
\end{pgfscope}%
\begin{pgfscope}%
\pgfsys@transformshift{1.993692in}{1.263563in}%
\pgfsys@useobject{currentmarker}{}%
\end{pgfscope}%
\begin{pgfscope}%
\pgfsys@transformshift{2.962858in}{0.740609in}%
\pgfsys@useobject{currentmarker}{}%
\end{pgfscope}%
\begin{pgfscope}%
\pgfsys@transformshift{2.914274in}{0.708946in}%
\pgfsys@useobject{currentmarker}{}%
\end{pgfscope}%
\begin{pgfscope}%
\pgfsys@transformshift{2.932684in}{0.710535in}%
\pgfsys@useobject{currentmarker}{}%
\end{pgfscope}%
\begin{pgfscope}%
\pgfsys@transformshift{2.884987in}{0.699714in}%
\pgfsys@useobject{currentmarker}{}%
\end{pgfscope}%
\begin{pgfscope}%
\pgfsys@transformshift{2.944620in}{0.746519in}%
\pgfsys@useobject{currentmarker}{}%
\end{pgfscope}%
\begin{pgfscope}%
\pgfsys@transformshift{2.960650in}{0.689525in}%
\pgfsys@useobject{currentmarker}{}%
\end{pgfscope}%
\begin{pgfscope}%
\pgfsys@transformshift{2.966010in}{0.671440in}%
\pgfsys@useobject{currentmarker}{}%
\end{pgfscope}%
\begin{pgfscope}%
\pgfsys@transformshift{2.914174in}{0.677666in}%
\pgfsys@useobject{currentmarker}{}%
\end{pgfscope}%
\begin{pgfscope}%
\pgfsys@transformshift{2.973596in}{0.756905in}%
\pgfsys@useobject{currentmarker}{}%
\end{pgfscope}%
\begin{pgfscope}%
\pgfsys@transformshift{2.919815in}{0.658376in}%
\pgfsys@useobject{currentmarker}{}%
\end{pgfscope}%
\begin{pgfscope}%
\pgfsys@transformshift{2.973596in}{0.725244in}%
\pgfsys@useobject{currentmarker}{}%
\end{pgfscope}%
\begin{pgfscope}%
\pgfsys@transformshift{2.916282in}{0.626171in}%
\pgfsys@useobject{currentmarker}{}%
\end{pgfscope}%
\begin{pgfscope}%
\pgfsys@transformshift{2.937432in}{0.657082in}%
\pgfsys@useobject{currentmarker}{}%
\end{pgfscope}%
\begin{pgfscope}%
\pgfsys@transformshift{2.891847in}{0.674183in}%
\pgfsys@useobject{currentmarker}{}%
\end{pgfscope}%
\begin{pgfscope}%
\pgfsys@transformshift{2.938860in}{0.711471in}%
\pgfsys@useobject{currentmarker}{}%
\end{pgfscope}%
\begin{pgfscope}%
\pgfsys@transformshift{2.930242in}{0.660671in}%
\pgfsys@useobject{currentmarker}{}%
\end{pgfscope}%
\begin{pgfscope}%
\pgfsys@transformshift{2.958067in}{0.701330in}%
\pgfsys@useobject{currentmarker}{}%
\end{pgfscope}%
\begin{pgfscope}%
\pgfsys@transformshift{2.907753in}{0.644942in}%
\pgfsys@useobject{currentmarker}{}%
\end{pgfscope}%
\begin{pgfscope}%
\pgfsys@transformshift{2.918734in}{0.681516in}%
\pgfsys@useobject{currentmarker}{}%
\end{pgfscope}%
\begin{pgfscope}%
\pgfsys@transformshift{2.957753in}{0.703360in}%
\pgfsys@useobject{currentmarker}{}%
\end{pgfscope}%
\begin{pgfscope}%
\pgfsys@transformshift{2.944661in}{0.674517in}%
\pgfsys@useobject{currentmarker}{}%
\end{pgfscope}%
\begin{pgfscope}%
\pgfsys@transformshift{2.901721in}{0.686900in}%
\pgfsys@useobject{currentmarker}{}%
\end{pgfscope}%
\begin{pgfscope}%
\pgfsys@transformshift{2.929664in}{0.694538in}%
\pgfsys@useobject{currentmarker}{}%
\end{pgfscope}%
\begin{pgfscope}%
\pgfsys@transformshift{2.918504in}{0.697282in}%
\pgfsys@useobject{currentmarker}{}%
\end{pgfscope}%
\begin{pgfscope}%
\pgfsys@transformshift{2.920951in}{0.643102in}%
\pgfsys@useobject{currentmarker}{}%
\end{pgfscope}%
\begin{pgfscope}%
\pgfsys@transformshift{2.922308in}{0.654840in}%
\pgfsys@useobject{currentmarker}{}%
\end{pgfscope}%
\begin{pgfscope}%
\pgfsys@transformshift{2.909740in}{0.661572in}%
\pgfsys@useobject{currentmarker}{}%
\end{pgfscope}%
\begin{pgfscope}%
\pgfsys@transformshift{2.930501in}{0.640693in}%
\pgfsys@useobject{currentmarker}{}%
\end{pgfscope}%
\begin{pgfscope}%
\pgfsys@transformshift{2.973596in}{0.704898in}%
\pgfsys@useobject{currentmarker}{}%
\end{pgfscope}%
\begin{pgfscope}%
\pgfsys@transformshift{2.940289in}{0.674272in}%
\pgfsys@useobject{currentmarker}{}%
\end{pgfscope}%
\begin{pgfscope}%
\pgfsys@transformshift{2.951467in}{0.713106in}%
\pgfsys@useobject{currentmarker}{}%
\end{pgfscope}%
\begin{pgfscope}%
\pgfsys@transformshift{2.929429in}{0.681091in}%
\pgfsys@useobject{currentmarker}{}%
\end{pgfscope}%
\begin{pgfscope}%
\pgfsys@transformshift{2.919403in}{0.679928in}%
\pgfsys@useobject{currentmarker}{}%
\end{pgfscope}%
\begin{pgfscope}%
\pgfsys@transformshift{2.911129in}{0.622935in}%
\pgfsys@useobject{currentmarker}{}%
\end{pgfscope}%
\begin{pgfscope}%
\pgfsys@transformshift{2.939759in}{0.655349in}%
\pgfsys@useobject{currentmarker}{}%
\end{pgfscope}%
\begin{pgfscope}%
\pgfsys@transformshift{2.973596in}{0.768031in}%
\pgfsys@useobject{currentmarker}{}%
\end{pgfscope}%
\begin{pgfscope}%
\pgfsys@transformshift{2.936785in}{0.670962in}%
\pgfsys@useobject{currentmarker}{}%
\end{pgfscope}%
\begin{pgfscope}%
\pgfsys@transformshift{2.951168in}{0.672183in}%
\pgfsys@useobject{currentmarker}{}%
\end{pgfscope}%
\begin{pgfscope}%
\pgfsys@transformshift{2.927765in}{0.672476in}%
\pgfsys@useobject{currentmarker}{}%
\end{pgfscope}%
\begin{pgfscope}%
\pgfsys@transformshift{2.922757in}{0.614361in}%
\pgfsys@useobject{currentmarker}{}%
\end{pgfscope}%
\begin{pgfscope}%
\pgfsys@transformshift{2.939628in}{0.714481in}%
\pgfsys@useobject{currentmarker}{}%
\end{pgfscope}%
\begin{pgfscope}%
\pgfsys@transformshift{2.912785in}{0.589731in}%
\pgfsys@useobject{currentmarker}{}%
\end{pgfscope}%
\begin{pgfscope}%
\pgfsys@transformshift{2.000003in}{1.237900in}%
\pgfsys@useobject{currentmarker}{}%
\end{pgfscope}%
\begin{pgfscope}%
\pgfsys@transformshift{2.950156in}{0.667643in}%
\pgfsys@useobject{currentmarker}{}%
\end{pgfscope}%
\begin{pgfscope}%
\pgfsys@transformshift{2.926294in}{0.704910in}%
\pgfsys@useobject{currentmarker}{}%
\end{pgfscope}%
\begin{pgfscope}%
\pgfsys@transformshift{2.917140in}{0.591571in}%
\pgfsys@useobject{currentmarker}{}%
\end{pgfscope}%
\begin{pgfscope}%
\pgfsys@transformshift{2.914084in}{0.644066in}%
\pgfsys@useobject{currentmarker}{}%
\end{pgfscope}%
\begin{pgfscope}%
\pgfsys@transformshift{2.925691in}{0.675895in}%
\pgfsys@useobject{currentmarker}{}%
\end{pgfscope}%
\begin{pgfscope}%
\pgfsys@transformshift{2.900989in}{0.641865in}%
\pgfsys@useobject{currentmarker}{}%
\end{pgfscope}%
\begin{pgfscope}%
\pgfsys@transformshift{2.924443in}{0.649897in}%
\pgfsys@useobject{currentmarker}{}%
\end{pgfscope}%
\begin{pgfscope}%
\pgfsys@transformshift{2.905200in}{0.636193in}%
\pgfsys@useobject{currentmarker}{}%
\end{pgfscope}%
\begin{pgfscope}%
\pgfsys@transformshift{2.926619in}{0.680549in}%
\pgfsys@useobject{currentmarker}{}%
\end{pgfscope}%
\begin{pgfscope}%
\pgfsys@transformshift{2.917120in}{0.640512in}%
\pgfsys@useobject{currentmarker}{}%
\end{pgfscope}%
\begin{pgfscope}%
\pgfsys@transformshift{2.939831in}{0.660699in}%
\pgfsys@useobject{currentmarker}{}%
\end{pgfscope}%
\begin{pgfscope}%
\pgfsys@transformshift{2.943729in}{0.720294in}%
\pgfsys@useobject{currentmarker}{}%
\end{pgfscope}%
\begin{pgfscope}%
\pgfsys@transformshift{2.909277in}{0.625748in}%
\pgfsys@useobject{currentmarker}{}%
\end{pgfscope}%
\begin{pgfscope}%
\pgfsys@transformshift{1.884679in}{1.563313in}%
\pgfsys@useobject{currentmarker}{}%
\end{pgfscope}%
\begin{pgfscope}%
\pgfsys@transformshift{2.967820in}{0.659602in}%
\pgfsys@useobject{currentmarker}{}%
\end{pgfscope}%
\begin{pgfscope}%
\pgfsys@transformshift{2.939119in}{0.654812in}%
\pgfsys@useobject{currentmarker}{}%
\end{pgfscope}%
\begin{pgfscope}%
\pgfsys@transformshift{2.959523in}{0.690894in}%
\pgfsys@useobject{currentmarker}{}%
\end{pgfscope}%
\begin{pgfscope}%
\pgfsys@transformshift{1.930744in}{1.354732in}%
\pgfsys@useobject{currentmarker}{}%
\end{pgfscope}%
\begin{pgfscope}%
\pgfsys@transformshift{2.973596in}{0.778710in}%
\pgfsys@useobject{currentmarker}{}%
\end{pgfscope}%
\begin{pgfscope}%
\pgfsys@transformshift{2.933651in}{0.704075in}%
\pgfsys@useobject{currentmarker}{}%
\end{pgfscope}%
\begin{pgfscope}%
\pgfsys@transformshift{2.941699in}{0.644760in}%
\pgfsys@useobject{currentmarker}{}%
\end{pgfscope}%
\begin{pgfscope}%
\pgfsys@transformshift{2.892693in}{0.716125in}%
\pgfsys@useobject{currentmarker}{}%
\end{pgfscope}%
\begin{pgfscope}%
\pgfsys@transformshift{2.917473in}{0.735213in}%
\pgfsys@useobject{currentmarker}{}%
\end{pgfscope}%
\begin{pgfscope}%
\pgfsys@transformshift{2.935694in}{0.685953in}%
\pgfsys@useobject{currentmarker}{}%
\end{pgfscope}%
\begin{pgfscope}%
\pgfsys@transformshift{2.938685in}{0.744683in}%
\pgfsys@useobject{currentmarker}{}%
\end{pgfscope}%
\begin{pgfscope}%
\pgfsys@transformshift{2.940374in}{0.614731in}%
\pgfsys@useobject{currentmarker}{}%
\end{pgfscope}%
\begin{pgfscope}%
\pgfsys@transformshift{2.899569in}{0.653151in}%
\pgfsys@useobject{currentmarker}{}%
\end{pgfscope}%
\begin{pgfscope}%
\pgfsys@transformshift{2.945146in}{0.669256in}%
\pgfsys@useobject{currentmarker}{}%
\end{pgfscope}%
\begin{pgfscope}%
\pgfsys@transformshift{2.894984in}{0.633047in}%
\pgfsys@useobject{currentmarker}{}%
\end{pgfscope}%
\begin{pgfscope}%
\pgfsys@transformshift{1.884641in}{0.499835in}%
\pgfsys@useobject{currentmarker}{}%
\end{pgfscope}%
\begin{pgfscope}%
\pgfsys@transformshift{2.929554in}{0.711255in}%
\pgfsys@useobject{currentmarker}{}%
\end{pgfscope}%
\begin{pgfscope}%
\pgfsys@transformshift{2.954022in}{0.686915in}%
\pgfsys@useobject{currentmarker}{}%
\end{pgfscope}%
\begin{pgfscope}%
\pgfsys@transformshift{2.930746in}{0.668896in}%
\pgfsys@useobject{currentmarker}{}%
\end{pgfscope}%
\begin{pgfscope}%
\pgfsys@transformshift{2.916922in}{0.658808in}%
\pgfsys@useobject{currentmarker}{}%
\end{pgfscope}%
\begin{pgfscope}%
\pgfsys@transformshift{2.929444in}{0.650450in}%
\pgfsys@useobject{currentmarker}{}%
\end{pgfscope}%
\begin{pgfscope}%
\pgfsys@transformshift{2.919452in}{0.666489in}%
\pgfsys@useobject{currentmarker}{}%
\end{pgfscope}%
\begin{pgfscope}%
\pgfsys@transformshift{2.870322in}{0.660412in}%
\pgfsys@useobject{currentmarker}{}%
\end{pgfscope}%
\begin{pgfscope}%
\pgfsys@transformshift{2.930776in}{0.699376in}%
\pgfsys@useobject{currentmarker}{}%
\end{pgfscope}%
\begin{pgfscope}%
\pgfsys@transformshift{2.941495in}{0.655631in}%
\pgfsys@useobject{currentmarker}{}%
\end{pgfscope}%
\begin{pgfscope}%
\pgfsys@transformshift{2.919308in}{0.671878in}%
\pgfsys@useobject{currentmarker}{}%
\end{pgfscope}%
\begin{pgfscope}%
\pgfsys@transformshift{2.937390in}{0.702520in}%
\pgfsys@useobject{currentmarker}{}%
\end{pgfscope}%
\begin{pgfscope}%
\pgfsys@transformshift{2.931097in}{0.714894in}%
\pgfsys@useobject{currentmarker}{}%
\end{pgfscope}%
\begin{pgfscope}%
\pgfsys@transformshift{2.921015in}{0.698788in}%
\pgfsys@useobject{currentmarker}{}%
\end{pgfscope}%
\begin{pgfscope}%
\pgfsys@transformshift{2.894071in}{0.625702in}%
\pgfsys@useobject{currentmarker}{}%
\end{pgfscope}%
\begin{pgfscope}%
\pgfsys@transformshift{2.932881in}{0.685055in}%
\pgfsys@useobject{currentmarker}{}%
\end{pgfscope}%
\begin{pgfscope}%
\pgfsys@transformshift{2.932589in}{0.600069in}%
\pgfsys@useobject{currentmarker}{}%
\end{pgfscope}%
\begin{pgfscope}%
\pgfsys@transformshift{2.954002in}{0.720469in}%
\pgfsys@useobject{currentmarker}{}%
\end{pgfscope}%
\begin{pgfscope}%
\pgfsys@transformshift{2.946006in}{0.692495in}%
\pgfsys@useobject{currentmarker}{}%
\end{pgfscope}%
\begin{pgfscope}%
\pgfsys@transformshift{2.924736in}{0.714736in}%
\pgfsys@useobject{currentmarker}{}%
\end{pgfscope}%
\begin{pgfscope}%
\pgfsys@transformshift{2.928231in}{0.713733in}%
\pgfsys@useobject{currentmarker}{}%
\end{pgfscope}%
\begin{pgfscope}%
\pgfsys@transformshift{2.901505in}{0.676817in}%
\pgfsys@useobject{currentmarker}{}%
\end{pgfscope}%
\begin{pgfscope}%
\pgfsys@transformshift{2.889156in}{0.624583in}%
\pgfsys@useobject{currentmarker}{}%
\end{pgfscope}%
\begin{pgfscope}%
\pgfsys@transformshift{2.943084in}{0.717665in}%
\pgfsys@useobject{currentmarker}{}%
\end{pgfscope}%
\begin{pgfscope}%
\pgfsys@transformshift{2.916869in}{0.656445in}%
\pgfsys@useobject{currentmarker}{}%
\end{pgfscope}%
\begin{pgfscope}%
\pgfsys@transformshift{2.918095in}{0.632032in}%
\pgfsys@useobject{currentmarker}{}%
\end{pgfscope}%
\begin{pgfscope}%
\pgfsys@transformshift{2.902139in}{0.670840in}%
\pgfsys@useobject{currentmarker}{}%
\end{pgfscope}%
\begin{pgfscope}%
\pgfsys@transformshift{2.928450in}{0.606998in}%
\pgfsys@useobject{currentmarker}{}%
\end{pgfscope}%
\begin{pgfscope}%
\pgfsys@transformshift{2.929921in}{0.634540in}%
\pgfsys@useobject{currentmarker}{}%
\end{pgfscope}%
\begin{pgfscope}%
\pgfsys@transformshift{2.027130in}{1.221295in}%
\pgfsys@useobject{currentmarker}{}%
\end{pgfscope}%
\begin{pgfscope}%
\pgfsys@transformshift{2.927536in}{0.676162in}%
\pgfsys@useobject{currentmarker}{}%
\end{pgfscope}%
\begin{pgfscope}%
\pgfsys@transformshift{2.896214in}{0.664922in}%
\pgfsys@useobject{currentmarker}{}%
\end{pgfscope}%
\begin{pgfscope}%
\pgfsys@transformshift{2.912298in}{0.689522in}%
\pgfsys@useobject{currentmarker}{}%
\end{pgfscope}%
\begin{pgfscope}%
\pgfsys@transformshift{2.920624in}{0.638828in}%
\pgfsys@useobject{currentmarker}{}%
\end{pgfscope}%
\begin{pgfscope}%
\pgfsys@transformshift{2.912107in}{0.665875in}%
\pgfsys@useobject{currentmarker}{}%
\end{pgfscope}%
\end{pgfscope}%
\begin{pgfscope}%
\pgfsetrectcap%
\pgfsetmiterjoin%
\pgfsetlinewidth{0.000000pt}%
\definecolor{currentstroke}{rgb}{1.000000,1.000000,1.000000}%
\pgfsetstrokecolor{currentstroke}%
\pgfsetdash{}{0pt}%
\pgfpathmoveto{\pgfqpoint{1.856795in}{0.423750in}}%
\pgfpathlineto{\pgfqpoint{1.856795in}{1.587136in}}%
\pgfusepath{}%
\end{pgfscope}%
\begin{pgfscope}%
\pgfsetrectcap%
\pgfsetmiterjoin%
\pgfsetlinewidth{0.000000pt}%
\definecolor{currentstroke}{rgb}{1.000000,1.000000,1.000000}%
\pgfsetstrokecolor{currentstroke}%
\pgfsetdash{}{0pt}%
\pgfpathmoveto{\pgfqpoint{3.051000in}{0.423750in}}%
\pgfpathlineto{\pgfqpoint{3.051000in}{1.587136in}}%
\pgfusepath{}%
\end{pgfscope}%
\begin{pgfscope}%
\pgfsetrectcap%
\pgfsetmiterjoin%
\pgfsetlinewidth{0.000000pt}%
\definecolor{currentstroke}{rgb}{1.000000,1.000000,1.000000}%
\pgfsetstrokecolor{currentstroke}%
\pgfsetdash{}{0pt}%
\pgfpathmoveto{\pgfqpoint{1.856795in}{0.423750in}}%
\pgfpathlineto{\pgfqpoint{3.051000in}{0.423750in}}%
\pgfusepath{}%
\end{pgfscope}%
\begin{pgfscope}%
\pgfsetrectcap%
\pgfsetmiterjoin%
\pgfsetlinewidth{0.000000pt}%
\definecolor{currentstroke}{rgb}{1.000000,1.000000,1.000000}%
\pgfsetstrokecolor{currentstroke}%
\pgfsetdash{}{0pt}%
\pgfpathmoveto{\pgfqpoint{1.856795in}{1.587136in}}%
\pgfpathlineto{\pgfqpoint{3.051000in}{1.587136in}}%
\pgfusepath{}%
\end{pgfscope}%
\end{pgfpicture}%
\makeatother%
\endgroup%

            \caption{
            These plots shows 200 observations of the Branin function using three different models each plotting the associated acquisition function.
            Notice how the regularizing effect of the ensemble \emph{prevents} it from exploring the third optimum.
            \emph{Top left}: the Branin function. 
            \emph{Top right}: GP.
            \emph{Bottom left}: DNGO retrain-reset.
            \emph{Bottom right}: 5 $\times$ DNGO retrain-reset.}
            \label{fig:braningexploit}
        \end{figure}
        
        % SE is stationary
        % function whose length scale does not change are well modelen by them
        % misspecification, meanning it takes more observersation before convergence

    \subsection{DNGO Deviations}\label{sec:disc-diviations}

        The results obtained with DNGO on benchmarks from \parencite{eggensperger_towards_2013} are worse than in the original paper \parencite{snoek_scalable_2015}.
        We propose two possible reasons for this.

        Firstly, \parencite{snoek_scalable_2015} put a quadratic prior on their mean thus incorporating expert knowledge as explained in \cref{sec:priormean}. 
        We have purposely left this prior out to make methods more comparable on the benchmark suite since \parencite{eggensperger_towards_2013} do in fact suffer from design bias of placing the optimum in the midpoint of the domain \parencite{dewancker_stratified_2016}.

        Secondly, they optimized their architecture using Bayesian Optimization on an unknown training set where we instead hand-tuned a reimplementation of DNGO.
        Considering that we are interested in comparing DNGO with its ensembled equivalent, this is not so problematic.

    % TODO: ensemble aggregation method

    % \subsection{Ensemble running time}\label{sec:disc-time}

    %     - takes k more. but can be parallelised
    %     - if k is small enough training time > prediction time
    %     - voice concern with best case effectiveness of ensemble
    %     - expect faster convergence but not necessary.. 


    \subsection{Alternative Extensions}\label{sec:disc-extensions}

        The need for a scalable method was based on parallisability -- we can obtain many more samples if we sample in parallel.
        The original paper \parencite{snoek_scalable_2015} evaluates a maximum of 2500 points in their biggest experiment, however.
        This is within the bounderies of what is already computational feasible with GPs.

        This prompts the question of whether DNGO is useful because of it scalability or due to the possiblity of capturing non-stationarity.
        The latter would seem to be the case considering the relatively positive results for DNGO on the oscillatory benchmarks.
        However, these could also be because the GP generally have troubles in high dimensions. % TODO: really?

        Either way, relaxing the requirement of scalability could open up interesting approaches.
        One method is to jointly train the Bayesian linear regressor and neural network weights as referred to as Deep Kernel Learning by \parencite{wilson_deep_2016}.
        Instead of only considering Bayesian linear regressors (which can be seen as a GP with a linear kernel) it considers more expressive kernels like the SE kernel and spectral mixture kernel.
        This method is even made to scale linearly using KISS-GP \parencite{wilson_kernel_2015}.

        Another interesting way to combine GPs and neural networks is the recently proposed Neural Processes \parencite{garnelo_neural_2018}.
        They bear resemblance to variational autoencoders and provide an efficient way of learning a distribution over functions.
    
    \subsection{Performance Evaluation}\label{sec:disc-evaluation}
        
        We have provided a qualititive analysis on a small selection of benchmarks.
        For a more thorough testing, some aggregate of performance over several benchmarks should be considered.
        A simple approach used in \parencite{golovin_google_2017} is to normalize using Simple Regret for random search and average over all benchmarks.
        The ranking in \parencite{dewancker_stratified_2016} is more suphisticated and considers which method converges faster if there is no significant difference in performance.
        Most importantly, though, is to mitigate design bias, as was attempted in \parencite{dewancker_stratified_2016}.

        When considering performance in Bayesian Optimization the computational budget is another important metric.
        These are very practical methods for which even constants (which are usually disregarded in asymptotic analysis) relate directly to a very real expense.
        An experimental comparison between DNGO and GP on Hartmann6 can already been found in \parencite{snoek_scalable_2015}.
        DNGO is asymptotically much faster which shows even after only 300 rounds on Hartmann6.
        Even though a test has not been conducted for the introduced ensembled method, it clearly only scales the budget by a factor $k$ in the worst case where $k$ is the size of the ensemble.

\section{Conclusion}\label{sec:conclusion}
    
    This work has investigated the DNGO and ensembled DNGO and a collection of benchmark and three machine learning tasks.
    The machine learning tasks turned out to be a fruitless attempt.
    However the for embedded synthesised function and for sufficently high dimensional mostly boring benchmark the ensemble deamed a promising case.
    However, to confirm this, more thorough testing is needed across multiple benchmarks and with a statistically significant number of runs.
    This would also allow for comparison with existing methods such as BOHAMIANN \parencite{springenberg_bayesian_2016} and by applying deep kernel learning \parencite{wilson_deep_2016} to Bayesian Optimization.
    Lastly, the effect of the ensemble aggregator was apparent, and this suggests a deeper investigation of its effect by looking into the effect of various quantiles.
        
\printbibliography

%- NN could maybe get stuck in local maximum. Is it more probable than in 1D though?

% ## Linear in O(n)

% "The equations for linear regression can be performed in primal form (e.g. the regular normal equations), where there is one parameter per input variable, or dual form (e.g. kernel ridge regression with a linear kernel) where there is one parameter per training example. This means you can choose which form to use depending on which is more efficient, if N>>d then use the normal equation, which is O(N), if d>>N then use kernel ridge regression with a linear kernel, where d is the number of attributes. Bayesian linear regression is to GP with a linear covariance function what linear regression is to kernel ridge regression with a linear kernel."

\appendices
\onecolumn

\section{Results}
    Where otherwise noted, plots shows the average Simple Regret over 10 runs with a $1/4$ standard derivation confidence interval.
    The naming convention for models is described in \cref{sec:exp}.

    \subsection{Embedding}\label{sec:appembedding}

    \begin{minipage}{\linewidth}
        \centering
        \begin{minipage}{0.3\linewidth}
            \centering
            %% Creator: Matplotlib, PGF backend
%%
%% To include the figure in your LaTeX document, write
%%   \input{<filename>.pgf}
%%
%% Make sure the required packages are loaded in your preamble
%%   \usepackage{pgf}
%%
%% Figures using additional raster images can only be included by \input if
%% they are in the same directory as the main LaTeX file. For loading figures
%% from other directories you can use the `import` package
%%   \usepackage{import}
%% and then include the figures with
%%   \import{<path to file>}{<filename>.pgf}
%%
%% Matplotlib used the following preamble
%%   \usepackage{gensymb}
%%   \usepackage{fontspec}
%%   \setmainfont{DejaVu Serif}
%%   \setsansfont{Arial}
%%   \setmonofont{DejaVu Sans Mono}
%%
\begingroup%
\makeatletter%
\begin{pgfpicture}%
\pgfpathrectangle{\pgfpointorigin}{\pgfqpoint{2.300000in}{3.000000in}}%
\pgfusepath{use as bounding box, clip}%
\begin{pgfscope}%
\pgfsetbuttcap%
\pgfsetmiterjoin%
\definecolor{currentfill}{rgb}{1.000000,1.000000,1.000000}%
\pgfsetfillcolor{currentfill}%
\pgfsetlinewidth{0.000000pt}%
\definecolor{currentstroke}{rgb}{1.000000,1.000000,1.000000}%
\pgfsetstrokecolor{currentstroke}%
\pgfsetdash{}{0pt}%
\pgfpathmoveto{\pgfqpoint{0.000000in}{0.000000in}}%
\pgfpathlineto{\pgfqpoint{2.300000in}{0.000000in}}%
\pgfpathlineto{\pgfqpoint{2.300000in}{3.000000in}}%
\pgfpathlineto{\pgfqpoint{0.000000in}{3.000000in}}%
\pgfpathclose%
\pgfusepath{fill}%
\end{pgfscope}%
\begin{pgfscope}%
\pgfsetbuttcap%
\pgfsetmiterjoin%
\definecolor{currentfill}{rgb}{0.917647,0.917647,0.949020}%
\pgfsetfillcolor{currentfill}%
\pgfsetlinewidth{0.000000pt}%
\definecolor{currentstroke}{rgb}{0.000000,0.000000,0.000000}%
\pgfsetstrokecolor{currentstroke}%
\pgfsetstrokeopacity{0.000000}%
\pgfsetdash{}{0pt}%
\pgfpathmoveto{\pgfqpoint{0.287500in}{0.375000in}}%
\pgfpathlineto{\pgfqpoint{2.070000in}{0.375000in}}%
\pgfpathlineto{\pgfqpoint{2.070000in}{2.640000in}}%
\pgfpathlineto{\pgfqpoint{0.287500in}{2.640000in}}%
\pgfpathclose%
\pgfusepath{fill}%
\end{pgfscope}%
\begin{pgfscope}%
\pgfpathrectangle{\pgfqpoint{0.287500in}{0.375000in}}{\pgfqpoint{1.782500in}{2.265000in}}%
\pgfusepath{clip}%
\pgfsetroundcap%
\pgfsetroundjoin%
\pgfsetlinewidth{0.803000pt}%
\definecolor{currentstroke}{rgb}{1.000000,1.000000,1.000000}%
\pgfsetstrokecolor{currentstroke}%
\pgfsetdash{}{0pt}%
\pgfpathmoveto{\pgfqpoint{0.360420in}{0.375000in}}%
\pgfpathlineto{\pgfqpoint{0.360420in}{2.640000in}}%
\pgfusepath{stroke}%
\end{pgfscope}%
\begin{pgfscope}%
\definecolor{textcolor}{rgb}{0.150000,0.150000,0.150000}%
\pgfsetstrokecolor{textcolor}%
\pgfsetfillcolor{textcolor}%
\pgftext[x=0.360420in,y=0.326389in,,top]{\color{textcolor}\rmfamily\fontsize{8.000000}{9.600000}\selectfont \(\displaystyle 0\)}%
\end{pgfscope}%
\begin{pgfscope}%
\pgfpathrectangle{\pgfqpoint{0.287500in}{0.375000in}}{\pgfqpoint{1.782500in}{2.265000in}}%
\pgfusepath{clip}%
\pgfsetroundcap%
\pgfsetroundjoin%
\pgfsetlinewidth{0.803000pt}%
\definecolor{currentstroke}{rgb}{1.000000,1.000000,1.000000}%
\pgfsetstrokecolor{currentstroke}%
\pgfsetdash{}{0pt}%
\pgfpathmoveto{\pgfqpoint{0.765534in}{0.375000in}}%
\pgfpathlineto{\pgfqpoint{0.765534in}{2.640000in}}%
\pgfusepath{stroke}%
\end{pgfscope}%
\begin{pgfscope}%
\definecolor{textcolor}{rgb}{0.150000,0.150000,0.150000}%
\pgfsetstrokecolor{textcolor}%
\pgfsetfillcolor{textcolor}%
\pgftext[x=0.765534in,y=0.326389in,,top]{\color{textcolor}\rmfamily\fontsize{8.000000}{9.600000}\selectfont \(\displaystyle 50\)}%
\end{pgfscope}%
\begin{pgfscope}%
\pgfpathrectangle{\pgfqpoint{0.287500in}{0.375000in}}{\pgfqpoint{1.782500in}{2.265000in}}%
\pgfusepath{clip}%
\pgfsetroundcap%
\pgfsetroundjoin%
\pgfsetlinewidth{0.803000pt}%
\definecolor{currentstroke}{rgb}{1.000000,1.000000,1.000000}%
\pgfsetstrokecolor{currentstroke}%
\pgfsetdash{}{0pt}%
\pgfpathmoveto{\pgfqpoint{1.170648in}{0.375000in}}%
\pgfpathlineto{\pgfqpoint{1.170648in}{2.640000in}}%
\pgfusepath{stroke}%
\end{pgfscope}%
\begin{pgfscope}%
\definecolor{textcolor}{rgb}{0.150000,0.150000,0.150000}%
\pgfsetstrokecolor{textcolor}%
\pgfsetfillcolor{textcolor}%
\pgftext[x=1.170648in,y=0.326389in,,top]{\color{textcolor}\rmfamily\fontsize{8.000000}{9.600000}\selectfont \(\displaystyle 100\)}%
\end{pgfscope}%
\begin{pgfscope}%
\pgfpathrectangle{\pgfqpoint{0.287500in}{0.375000in}}{\pgfqpoint{1.782500in}{2.265000in}}%
\pgfusepath{clip}%
\pgfsetroundcap%
\pgfsetroundjoin%
\pgfsetlinewidth{0.803000pt}%
\definecolor{currentstroke}{rgb}{1.000000,1.000000,1.000000}%
\pgfsetstrokecolor{currentstroke}%
\pgfsetdash{}{0pt}%
\pgfpathmoveto{\pgfqpoint{1.575761in}{0.375000in}}%
\pgfpathlineto{\pgfqpoint{1.575761in}{2.640000in}}%
\pgfusepath{stroke}%
\end{pgfscope}%
\begin{pgfscope}%
\definecolor{textcolor}{rgb}{0.150000,0.150000,0.150000}%
\pgfsetstrokecolor{textcolor}%
\pgfsetfillcolor{textcolor}%
\pgftext[x=1.575761in,y=0.326389in,,top]{\color{textcolor}\rmfamily\fontsize{8.000000}{9.600000}\selectfont \(\displaystyle 150\)}%
\end{pgfscope}%
\begin{pgfscope}%
\pgfpathrectangle{\pgfqpoint{0.287500in}{0.375000in}}{\pgfqpoint{1.782500in}{2.265000in}}%
\pgfusepath{clip}%
\pgfsetroundcap%
\pgfsetroundjoin%
\pgfsetlinewidth{0.803000pt}%
\definecolor{currentstroke}{rgb}{1.000000,1.000000,1.000000}%
\pgfsetstrokecolor{currentstroke}%
\pgfsetdash{}{0pt}%
\pgfpathmoveto{\pgfqpoint{1.980875in}{0.375000in}}%
\pgfpathlineto{\pgfqpoint{1.980875in}{2.640000in}}%
\pgfusepath{stroke}%
\end{pgfscope}%
\begin{pgfscope}%
\definecolor{textcolor}{rgb}{0.150000,0.150000,0.150000}%
\pgfsetstrokecolor{textcolor}%
\pgfsetfillcolor{textcolor}%
\pgftext[x=1.980875in,y=0.326389in,,top]{\color{textcolor}\rmfamily\fontsize{8.000000}{9.600000}\selectfont \(\displaystyle 200\)}%
\end{pgfscope}%
\begin{pgfscope}%
\definecolor{textcolor}{rgb}{0.150000,0.150000,0.150000}%
\pgfsetstrokecolor{textcolor}%
\pgfsetfillcolor{textcolor}%
\pgftext[x=1.178750in,y=0.163303in,,top]{\color{textcolor}\rmfamily\fontsize{8.000000}{9.600000}\selectfont Step \(\displaystyle t\)}%
\end{pgfscope}%
\begin{pgfscope}%
\pgfpathrectangle{\pgfqpoint{0.287500in}{0.375000in}}{\pgfqpoint{1.782500in}{2.265000in}}%
\pgfusepath{clip}%
\pgfsetroundcap%
\pgfsetroundjoin%
\pgfsetlinewidth{0.803000pt}%
\definecolor{currentstroke}{rgb}{1.000000,1.000000,1.000000}%
\pgfsetstrokecolor{currentstroke}%
\pgfsetdash{}{0pt}%
\pgfpathmoveto{\pgfqpoint{0.287500in}{2.180761in}}%
\pgfpathlineto{\pgfqpoint{2.070000in}{2.180761in}}%
\pgfusepath{stroke}%
\end{pgfscope}%
\begin{pgfscope}%
\definecolor{textcolor}{rgb}{0.150000,0.150000,0.150000}%
\pgfsetstrokecolor{textcolor}%
\pgfsetfillcolor{textcolor}%
\pgftext[x=0.062962in,y=2.138552in,left,base]{\color{textcolor}\rmfamily\fontsize{8.000000}{9.600000}\selectfont \(\displaystyle 10^{0}\)}%
\end{pgfscope}%
\begin{pgfscope}%
\definecolor{textcolor}{rgb}{0.150000,0.150000,0.150000}%
\pgfsetstrokecolor{textcolor}%
\pgfsetfillcolor{textcolor}%
\pgftext[x=0.007407in,y=1.507500in,,bottom,rotate=90.000000]{\color{textcolor}\rmfamily\fontsize{8.000000}{9.600000}\selectfont \(\displaystyle R_t/t\)}%
\end{pgfscope}%
\begin{pgfscope}%
\pgfpathrectangle{\pgfqpoint{0.287500in}{0.375000in}}{\pgfqpoint{1.782500in}{2.265000in}}%
\pgfusepath{clip}%
\pgfsetbuttcap%
\pgfsetroundjoin%
\definecolor{currentfill}{rgb}{0.121569,0.466667,0.705882}%
\pgfsetfillcolor{currentfill}%
\pgfsetfillopacity{0.200000}%
\pgfsetlinewidth{0.000000pt}%
\definecolor{currentstroke}{rgb}{0.000000,0.000000,0.000000}%
\pgfsetstrokecolor{currentstroke}%
\pgfsetdash{}{0pt}%
\pgfpathmoveto{\pgfqpoint{0.368523in}{1.836121in}}%
\pgfpathlineto{\pgfqpoint{0.368523in}{2.294122in}}%
\pgfpathlineto{\pgfqpoint{0.376625in}{2.065827in}}%
\pgfpathlineto{\pgfqpoint{0.384727in}{1.966790in}}%
\pgfpathlineto{\pgfqpoint{0.392830in}{1.973446in}}%
\pgfpathlineto{\pgfqpoint{0.400932in}{1.972735in}}%
\pgfpathlineto{\pgfqpoint{0.409034in}{1.925138in}}%
\pgfpathlineto{\pgfqpoint{0.417136in}{1.889656in}}%
\pgfpathlineto{\pgfqpoint{0.425239in}{1.873650in}}%
\pgfpathlineto{\pgfqpoint{0.433341in}{1.859447in}}%
\pgfpathlineto{\pgfqpoint{0.441443in}{1.839632in}}%
\pgfpathlineto{\pgfqpoint{0.449545in}{1.799085in}}%
\pgfpathlineto{\pgfqpoint{0.457648in}{1.818601in}}%
\pgfpathlineto{\pgfqpoint{0.465750in}{1.809254in}}%
\pgfpathlineto{\pgfqpoint{0.473852in}{1.814859in}}%
\pgfpathlineto{\pgfqpoint{0.481955in}{1.813554in}}%
\pgfpathlineto{\pgfqpoint{0.490057in}{1.792903in}}%
\pgfpathlineto{\pgfqpoint{0.498159in}{1.771555in}}%
\pgfpathlineto{\pgfqpoint{0.506261in}{1.766098in}}%
\pgfpathlineto{\pgfqpoint{0.514364in}{1.762124in}}%
\pgfpathlineto{\pgfqpoint{0.522466in}{1.752330in}}%
\pgfpathlineto{\pgfqpoint{0.530568in}{1.737651in}}%
\pgfpathlineto{\pgfqpoint{0.538670in}{1.744451in}}%
\pgfpathlineto{\pgfqpoint{0.546773in}{1.733288in}}%
\pgfpathlineto{\pgfqpoint{0.554875in}{1.724752in}}%
\pgfpathlineto{\pgfqpoint{0.562977in}{1.695322in}}%
\pgfpathlineto{\pgfqpoint{0.571080in}{1.684768in}}%
\pgfpathlineto{\pgfqpoint{0.579182in}{1.669045in}}%
\pgfpathlineto{\pgfqpoint{0.587284in}{1.649546in}}%
\pgfpathlineto{\pgfqpoint{0.595386in}{1.641145in}}%
\pgfpathlineto{\pgfqpoint{0.603489in}{1.634888in}}%
\pgfpathlineto{\pgfqpoint{0.611591in}{1.645049in}}%
\pgfpathlineto{\pgfqpoint{0.619693in}{1.640165in}}%
\pgfpathlineto{\pgfqpoint{0.627795in}{1.627354in}}%
\pgfpathlineto{\pgfqpoint{0.635898in}{1.610352in}}%
\pgfpathlineto{\pgfqpoint{0.644000in}{1.609523in}}%
\pgfpathlineto{\pgfqpoint{0.652102in}{1.603210in}}%
\pgfpathlineto{\pgfqpoint{0.660205in}{1.604047in}}%
\pgfpathlineto{\pgfqpoint{0.668307in}{1.592112in}}%
\pgfpathlineto{\pgfqpoint{0.676409in}{1.581458in}}%
\pgfpathlineto{\pgfqpoint{0.684511in}{1.576875in}}%
\pgfpathlineto{\pgfqpoint{0.692614in}{1.565898in}}%
\pgfpathlineto{\pgfqpoint{0.700716in}{1.559448in}}%
\pgfpathlineto{\pgfqpoint{0.708818in}{1.563197in}}%
\pgfpathlineto{\pgfqpoint{0.716920in}{1.554460in}}%
\pgfpathlineto{\pgfqpoint{0.725023in}{1.539397in}}%
\pgfpathlineto{\pgfqpoint{0.733125in}{1.532298in}}%
\pgfpathlineto{\pgfqpoint{0.741227in}{1.524963in}}%
\pgfpathlineto{\pgfqpoint{0.749330in}{1.521585in}}%
\pgfpathlineto{\pgfqpoint{0.757432in}{1.511504in}}%
\pgfpathlineto{\pgfqpoint{0.765534in}{1.509219in}}%
\pgfpathlineto{\pgfqpoint{0.773636in}{1.494214in}}%
\pgfpathlineto{\pgfqpoint{0.781739in}{1.478836in}}%
\pgfpathlineto{\pgfqpoint{0.789841in}{1.462169in}}%
\pgfpathlineto{\pgfqpoint{0.797943in}{1.447491in}}%
\pgfpathlineto{\pgfqpoint{0.806045in}{1.433056in}}%
\pgfpathlineto{\pgfqpoint{0.814148in}{1.422550in}}%
\pgfpathlineto{\pgfqpoint{0.822250in}{1.417521in}}%
\pgfpathlineto{\pgfqpoint{0.830352in}{1.405067in}}%
\pgfpathlineto{\pgfqpoint{0.838455in}{1.404890in}}%
\pgfpathlineto{\pgfqpoint{0.846557in}{1.391668in}}%
\pgfpathlineto{\pgfqpoint{0.854659in}{1.389409in}}%
\pgfpathlineto{\pgfqpoint{0.862761in}{1.376564in}}%
\pgfpathlineto{\pgfqpoint{0.870864in}{1.364026in}}%
\pgfpathlineto{\pgfqpoint{0.878966in}{1.355461in}}%
\pgfpathlineto{\pgfqpoint{0.887068in}{1.354426in}}%
\pgfpathlineto{\pgfqpoint{0.895170in}{1.342409in}}%
\pgfpathlineto{\pgfqpoint{0.903273in}{1.330781in}}%
\pgfpathlineto{\pgfqpoint{0.911375in}{1.320696in}}%
\pgfpathlineto{\pgfqpoint{0.919477in}{1.305744in}}%
\pgfpathlineto{\pgfqpoint{0.927580in}{1.309198in}}%
\pgfpathlineto{\pgfqpoint{0.935682in}{1.298095in}}%
\pgfpathlineto{\pgfqpoint{0.943784in}{1.287130in}}%
\pgfpathlineto{\pgfqpoint{0.951886in}{1.276235in}}%
\pgfpathlineto{\pgfqpoint{0.959989in}{1.265512in}}%
\pgfpathlineto{\pgfqpoint{0.968091in}{1.254962in}}%
\pgfpathlineto{\pgfqpoint{0.976193in}{1.247112in}}%
\pgfpathlineto{\pgfqpoint{0.984295in}{1.244138in}}%
\pgfpathlineto{\pgfqpoint{0.992398in}{1.234179in}}%
\pgfpathlineto{\pgfqpoint{1.000500in}{1.225184in}}%
\pgfpathlineto{\pgfqpoint{1.008602in}{1.215256in}}%
\pgfpathlineto{\pgfqpoint{1.016705in}{1.205527in}}%
\pgfpathlineto{\pgfqpoint{1.024807in}{1.195840in}}%
\pgfpathlineto{\pgfqpoint{1.032909in}{1.186321in}}%
\pgfpathlineto{\pgfqpoint{1.041011in}{1.176859in}}%
\pgfpathlineto{\pgfqpoint{1.049114in}{1.169758in}}%
\pgfpathlineto{\pgfqpoint{1.057216in}{1.163789in}}%
\pgfpathlineto{\pgfqpoint{1.065318in}{1.159378in}}%
\pgfpathlineto{\pgfqpoint{1.073420in}{1.150340in}}%
\pgfpathlineto{\pgfqpoint{1.081523in}{1.141448in}}%
\pgfpathlineto{\pgfqpoint{1.089625in}{1.133382in}}%
\pgfpathlineto{\pgfqpoint{1.097727in}{1.137398in}}%
\pgfpathlineto{\pgfqpoint{1.105830in}{1.128799in}}%
\pgfpathlineto{\pgfqpoint{1.113932in}{1.120254in}}%
\pgfpathlineto{\pgfqpoint{1.122034in}{1.111815in}}%
\pgfpathlineto{\pgfqpoint{1.130136in}{1.103449in}}%
\pgfpathlineto{\pgfqpoint{1.138239in}{1.095178in}}%
\pgfpathlineto{\pgfqpoint{1.146341in}{1.090677in}}%
\pgfpathlineto{\pgfqpoint{1.154443in}{1.085310in}}%
\pgfpathlineto{\pgfqpoint{1.162545in}{1.077300in}}%
\pgfpathlineto{\pgfqpoint{1.170648in}{1.069352in}}%
\pgfpathlineto{\pgfqpoint{1.178750in}{1.067800in}}%
\pgfpathlineto{\pgfqpoint{1.186852in}{1.060029in}}%
\pgfpathlineto{\pgfqpoint{1.194955in}{1.056334in}}%
\pgfpathlineto{\pgfqpoint{1.203057in}{1.048708in}}%
\pgfpathlineto{\pgfqpoint{1.211159in}{1.041921in}}%
\pgfpathlineto{\pgfqpoint{1.219261in}{1.034856in}}%
\pgfpathlineto{\pgfqpoint{1.227364in}{1.027485in}}%
\pgfpathlineto{\pgfqpoint{1.235466in}{1.020158in}}%
\pgfpathlineto{\pgfqpoint{1.243568in}{1.012886in}}%
\pgfpathlineto{\pgfqpoint{1.251670in}{1.006643in}}%
\pgfpathlineto{\pgfqpoint{1.259773in}{0.999541in}}%
\pgfpathlineto{\pgfqpoint{1.267875in}{0.992459in}}%
\pgfpathlineto{\pgfqpoint{1.275977in}{0.985788in}}%
\pgfpathlineto{\pgfqpoint{1.284080in}{0.988924in}}%
\pgfpathlineto{\pgfqpoint{1.292182in}{0.982317in}}%
\pgfpathlineto{\pgfqpoint{1.300284in}{0.975225in}}%
\pgfpathlineto{\pgfqpoint{1.308386in}{0.968482in}}%
\pgfpathlineto{\pgfqpoint{1.316489in}{0.961778in}}%
\pgfpathlineto{\pgfqpoint{1.324591in}{0.955118in}}%
\pgfpathlineto{\pgfqpoint{1.332693in}{0.948525in}}%
\pgfpathlineto{\pgfqpoint{1.340795in}{0.941980in}}%
\pgfpathlineto{\pgfqpoint{1.348898in}{0.935506in}}%
\pgfpathlineto{\pgfqpoint{1.357000in}{0.929066in}}%
\pgfpathlineto{\pgfqpoint{1.365102in}{0.922778in}}%
\pgfpathlineto{\pgfqpoint{1.373205in}{0.916798in}}%
\pgfpathlineto{\pgfqpoint{1.381307in}{0.910898in}}%
\pgfpathlineto{\pgfqpoint{1.389409in}{0.904675in}}%
\pgfpathlineto{\pgfqpoint{1.397511in}{0.902979in}}%
\pgfpathlineto{\pgfqpoint{1.405614in}{0.897338in}}%
\pgfpathlineto{\pgfqpoint{1.413716in}{0.890515in}}%
\pgfpathlineto{\pgfqpoint{1.421818in}{0.884622in}}%
\pgfpathlineto{\pgfqpoint{1.429920in}{0.878818in}}%
\pgfpathlineto{\pgfqpoint{1.438023in}{0.872883in}}%
\pgfpathlineto{\pgfqpoint{1.446125in}{0.867900in}}%
\pgfpathlineto{\pgfqpoint{1.454227in}{0.867742in}}%
\pgfpathlineto{\pgfqpoint{1.462330in}{0.862577in}}%
\pgfpathlineto{\pgfqpoint{1.470432in}{0.856833in}}%
\pgfpathlineto{\pgfqpoint{1.478534in}{0.851113in}}%
\pgfpathlineto{\pgfqpoint{1.486636in}{0.845446in}}%
\pgfpathlineto{\pgfqpoint{1.494739in}{0.839804in}}%
\pgfpathlineto{\pgfqpoint{1.502841in}{0.834380in}}%
\pgfpathlineto{\pgfqpoint{1.510943in}{0.828837in}}%
\pgfpathlineto{\pgfqpoint{1.519045in}{0.823301in}}%
\pgfpathlineto{\pgfqpoint{1.527148in}{0.818425in}}%
\pgfpathlineto{\pgfqpoint{1.535250in}{0.812970in}}%
\pgfpathlineto{\pgfqpoint{1.543352in}{0.807587in}}%
\pgfpathlineto{\pgfqpoint{1.551455in}{0.802204in}}%
\pgfpathlineto{\pgfqpoint{1.559557in}{0.796848in}}%
\pgfpathlineto{\pgfqpoint{1.567659in}{0.791666in}}%
\pgfpathlineto{\pgfqpoint{1.575761in}{0.786435in}}%
\pgfpathlineto{\pgfqpoint{1.583864in}{0.781188in}}%
\pgfpathlineto{\pgfqpoint{1.591966in}{0.776007in}}%
\pgfpathlineto{\pgfqpoint{1.600068in}{0.771891in}}%
\pgfpathlineto{\pgfqpoint{1.608170in}{0.766589in}}%
\pgfpathlineto{\pgfqpoint{1.616273in}{0.761476in}}%
\pgfpathlineto{\pgfqpoint{1.624375in}{0.756395in}}%
\pgfpathlineto{\pgfqpoint{1.632477in}{0.754663in}}%
\pgfpathlineto{\pgfqpoint{1.640580in}{0.749657in}}%
\pgfpathlineto{\pgfqpoint{1.648682in}{0.747279in}}%
\pgfpathlineto{\pgfqpoint{1.656784in}{0.742551in}}%
\pgfpathlineto{\pgfqpoint{1.664886in}{0.737655in}}%
\pgfpathlineto{\pgfqpoint{1.672989in}{0.734957in}}%
\pgfpathlineto{\pgfqpoint{1.681091in}{0.732150in}}%
\pgfpathlineto{\pgfqpoint{1.689193in}{0.727398in}}%
\pgfpathlineto{\pgfqpoint{1.697295in}{0.722597in}}%
\pgfpathlineto{\pgfqpoint{1.705398in}{0.717936in}}%
\pgfpathlineto{\pgfqpoint{1.713500in}{0.713221in}}%
\pgfpathlineto{\pgfqpoint{1.721602in}{0.710322in}}%
\pgfpathlineto{\pgfqpoint{1.729705in}{0.705646in}}%
\pgfpathlineto{\pgfqpoint{1.737807in}{0.701044in}}%
\pgfpathlineto{\pgfqpoint{1.745909in}{0.698445in}}%
\pgfpathlineto{\pgfqpoint{1.754011in}{0.695055in}}%
\pgfpathlineto{\pgfqpoint{1.762114in}{0.698971in}}%
\pgfpathlineto{\pgfqpoint{1.770216in}{0.694438in}}%
\pgfpathlineto{\pgfqpoint{1.778318in}{0.689909in}}%
\pgfpathlineto{\pgfqpoint{1.786420in}{0.686409in}}%
\pgfpathlineto{\pgfqpoint{1.794523in}{0.682009in}}%
\pgfpathlineto{\pgfqpoint{1.802625in}{0.683722in}}%
\pgfpathlineto{\pgfqpoint{1.810727in}{0.679395in}}%
\pgfpathlineto{\pgfqpoint{1.818830in}{0.675035in}}%
\pgfpathlineto{\pgfqpoint{1.826932in}{0.670664in}}%
\pgfpathlineto{\pgfqpoint{1.835034in}{0.666437in}}%
\pgfpathlineto{\pgfqpoint{1.843136in}{0.662139in}}%
\pgfpathlineto{\pgfqpoint{1.851239in}{0.657848in}}%
\pgfpathlineto{\pgfqpoint{1.859341in}{0.653571in}}%
\pgfpathlineto{\pgfqpoint{1.867443in}{0.649329in}}%
\pgfpathlineto{\pgfqpoint{1.875545in}{0.645265in}}%
\pgfpathlineto{\pgfqpoint{1.883648in}{0.642110in}}%
\pgfpathlineto{\pgfqpoint{1.891750in}{0.637961in}}%
\pgfpathlineto{\pgfqpoint{1.899852in}{0.633794in}}%
\pgfpathlineto{\pgfqpoint{1.907955in}{0.629646in}}%
\pgfpathlineto{\pgfqpoint{1.916057in}{0.625559in}}%
\pgfpathlineto{\pgfqpoint{1.924159in}{0.622916in}}%
\pgfpathlineto{\pgfqpoint{1.932261in}{0.625669in}}%
\pgfpathlineto{\pgfqpoint{1.940364in}{0.621627in}}%
\pgfpathlineto{\pgfqpoint{1.948466in}{0.617388in}}%
\pgfpathlineto{\pgfqpoint{1.956568in}{0.613757in}}%
\pgfpathlineto{\pgfqpoint{1.964670in}{0.609765in}}%
\pgfpathlineto{\pgfqpoint{1.972773in}{0.605794in}}%
\pgfpathlineto{\pgfqpoint{1.980875in}{0.601849in}}%
\pgfpathlineto{\pgfqpoint{1.988977in}{0.597909in}}%
\pgfpathlineto{\pgfqpoint{1.988977in}{0.477955in}}%
\pgfpathlineto{\pgfqpoint{1.988977in}{0.477955in}}%
\pgfpathlineto{\pgfqpoint{1.980875in}{0.481898in}}%
\pgfpathlineto{\pgfqpoint{1.972773in}{0.485839in}}%
\pgfpathlineto{\pgfqpoint{1.964670in}{0.489791in}}%
\pgfpathlineto{\pgfqpoint{1.956568in}{0.493517in}}%
\pgfpathlineto{\pgfqpoint{1.948466in}{0.489661in}}%
\pgfpathlineto{\pgfqpoint{1.940364in}{0.486919in}}%
\pgfpathlineto{\pgfqpoint{1.932261in}{0.490889in}}%
\pgfpathlineto{\pgfqpoint{1.924159in}{0.496353in}}%
\pgfpathlineto{\pgfqpoint{1.916057in}{0.500041in}}%
\pgfpathlineto{\pgfqpoint{1.907955in}{0.503217in}}%
\pgfpathlineto{\pgfqpoint{1.899852in}{0.507323in}}%
\pgfpathlineto{\pgfqpoint{1.891750in}{0.511484in}}%
\pgfpathlineto{\pgfqpoint{1.883648in}{0.515601in}}%
\pgfpathlineto{\pgfqpoint{1.875545in}{0.518003in}}%
\pgfpathlineto{\pgfqpoint{1.867443in}{0.522271in}}%
\pgfpathlineto{\pgfqpoint{1.859341in}{0.526473in}}%
\pgfpathlineto{\pgfqpoint{1.851239in}{0.530757in}}%
\pgfpathlineto{\pgfqpoint{1.843136in}{0.535001in}}%
\pgfpathlineto{\pgfqpoint{1.835034in}{0.539321in}}%
\pgfpathlineto{\pgfqpoint{1.826932in}{0.543665in}}%
\pgfpathlineto{\pgfqpoint{1.818830in}{0.548043in}}%
\pgfpathlineto{\pgfqpoint{1.810727in}{0.552419in}}%
\pgfpathlineto{\pgfqpoint{1.802625in}{0.554898in}}%
\pgfpathlineto{\pgfqpoint{1.794523in}{0.556990in}}%
\pgfpathlineto{\pgfqpoint{1.786420in}{0.561483in}}%
\pgfpathlineto{\pgfqpoint{1.778318in}{0.565431in}}%
\pgfpathlineto{\pgfqpoint{1.770216in}{0.569955in}}%
\pgfpathlineto{\pgfqpoint{1.762114in}{0.574297in}}%
\pgfpathlineto{\pgfqpoint{1.754011in}{0.581081in}}%
\pgfpathlineto{\pgfqpoint{1.745909in}{0.582356in}}%
\pgfpathlineto{\pgfqpoint{1.737807in}{0.583085in}}%
\pgfpathlineto{\pgfqpoint{1.729705in}{0.587714in}}%
\pgfpathlineto{\pgfqpoint{1.721602in}{0.592367in}}%
\pgfpathlineto{\pgfqpoint{1.713500in}{0.592988in}}%
\pgfpathlineto{\pgfqpoint{1.705398in}{0.597705in}}%
\pgfpathlineto{\pgfqpoint{1.697295in}{0.602419in}}%
\pgfpathlineto{\pgfqpoint{1.689193in}{0.607195in}}%
\pgfpathlineto{\pgfqpoint{1.681091in}{0.612004in}}%
\pgfpathlineto{\pgfqpoint{1.672989in}{0.613376in}}%
\pgfpathlineto{\pgfqpoint{1.664886in}{0.617200in}}%
\pgfpathlineto{\pgfqpoint{1.656784in}{0.622108in}}%
\pgfpathlineto{\pgfqpoint{1.648682in}{0.626960in}}%
\pgfpathlineto{\pgfqpoint{1.640580in}{0.632655in}}%
\pgfpathlineto{\pgfqpoint{1.632477in}{0.637670in}}%
\pgfpathlineto{\pgfqpoint{1.624375in}{0.640586in}}%
\pgfpathlineto{\pgfqpoint{1.616273in}{0.643845in}}%
\pgfpathlineto{\pgfqpoint{1.608170in}{0.648942in}}%
\pgfpathlineto{\pgfqpoint{1.600068in}{0.649638in}}%
\pgfpathlineto{\pgfqpoint{1.591966in}{0.655031in}}%
\pgfpathlineto{\pgfqpoint{1.583864in}{0.659967in}}%
\pgfpathlineto{\pgfqpoint{1.575761in}{0.665177in}}%
\pgfpathlineto{\pgfqpoint{1.567659in}{0.670395in}}%
\pgfpathlineto{\pgfqpoint{1.559557in}{0.675516in}}%
\pgfpathlineto{\pgfqpoint{1.551455in}{0.680867in}}%
\pgfpathlineto{\pgfqpoint{1.543352in}{0.686258in}}%
\pgfpathlineto{\pgfqpoint{1.535250in}{0.691699in}}%
\pgfpathlineto{\pgfqpoint{1.527148in}{0.697168in}}%
\pgfpathlineto{\pgfqpoint{1.519045in}{0.701948in}}%
\pgfpathlineto{\pgfqpoint{1.510943in}{0.707475in}}%
\pgfpathlineto{\pgfqpoint{1.502841in}{0.713047in}}%
\pgfpathlineto{\pgfqpoint{1.494739in}{0.718658in}}%
\pgfpathlineto{\pgfqpoint{1.486636in}{0.724296in}}%
\pgfpathlineto{\pgfqpoint{1.478534in}{0.729905in}}%
\pgfpathlineto{\pgfqpoint{1.470432in}{0.735655in}}%
\pgfpathlineto{\pgfqpoint{1.462330in}{0.741407in}}%
\pgfpathlineto{\pgfqpoint{1.454227in}{0.746704in}}%
\pgfpathlineto{\pgfqpoint{1.446125in}{0.754067in}}%
\pgfpathlineto{\pgfqpoint{1.438023in}{0.753595in}}%
\pgfpathlineto{\pgfqpoint{1.429920in}{0.759521in}}%
\pgfpathlineto{\pgfqpoint{1.421818in}{0.765383in}}%
\pgfpathlineto{\pgfqpoint{1.413716in}{0.771370in}}%
\pgfpathlineto{\pgfqpoint{1.405614in}{0.765843in}}%
\pgfpathlineto{\pgfqpoint{1.397511in}{0.771482in}}%
\pgfpathlineto{\pgfqpoint{1.389409in}{0.770110in}}%
\pgfpathlineto{\pgfqpoint{1.381307in}{0.776324in}}%
\pgfpathlineto{\pgfqpoint{1.373205in}{0.782417in}}%
\pgfpathlineto{\pgfqpoint{1.365102in}{0.788743in}}%
\pgfpathlineto{\pgfqpoint{1.357000in}{0.795116in}}%
\pgfpathlineto{\pgfqpoint{1.348898in}{0.801566in}}%
\pgfpathlineto{\pgfqpoint{1.340795in}{0.808055in}}%
\pgfpathlineto{\pgfqpoint{1.332693in}{0.814611in}}%
\pgfpathlineto{\pgfqpoint{1.324591in}{0.821130in}}%
\pgfpathlineto{\pgfqpoint{1.316489in}{0.827776in}}%
\pgfpathlineto{\pgfqpoint{1.308386in}{0.834508in}}%
\pgfpathlineto{\pgfqpoint{1.300284in}{0.841290in}}%
\pgfpathlineto{\pgfqpoint{1.292182in}{0.844274in}}%
\pgfpathlineto{\pgfqpoint{1.284080in}{0.851243in}}%
\pgfpathlineto{\pgfqpoint{1.275977in}{0.860810in}}%
\pgfpathlineto{\pgfqpoint{1.267875in}{0.867896in}}%
\pgfpathlineto{\pgfqpoint{1.259773in}{0.874957in}}%
\pgfpathlineto{\pgfqpoint{1.251670in}{0.882111in}}%
\pgfpathlineto{\pgfqpoint{1.243568in}{0.881096in}}%
\pgfpathlineto{\pgfqpoint{1.235466in}{0.888376in}}%
\pgfpathlineto{\pgfqpoint{1.227364in}{0.895691in}}%
\pgfpathlineto{\pgfqpoint{1.219261in}{0.903094in}}%
\pgfpathlineto{\pgfqpoint{1.211159in}{0.910348in}}%
\pgfpathlineto{\pgfqpoint{1.203057in}{0.916633in}}%
\pgfpathlineto{\pgfqpoint{1.194955in}{0.924256in}}%
\pgfpathlineto{\pgfqpoint{1.186852in}{0.931436in}}%
\pgfpathlineto{\pgfqpoint{1.178750in}{0.939089in}}%
\pgfpathlineto{\pgfqpoint{1.170648in}{0.948281in}}%
\pgfpathlineto{\pgfqpoint{1.162545in}{0.956140in}}%
\pgfpathlineto{\pgfqpoint{1.154443in}{0.964130in}}%
\pgfpathlineto{\pgfqpoint{1.146341in}{0.970698in}}%
\pgfpathlineto{\pgfqpoint{1.138239in}{0.979793in}}%
\pgfpathlineto{\pgfqpoint{1.130136in}{0.988072in}}%
\pgfpathlineto{\pgfqpoint{1.122034in}{0.996420in}}%
\pgfpathlineto{\pgfqpoint{1.113932in}{1.004853in}}%
\pgfpathlineto{\pgfqpoint{1.105830in}{1.013383in}}%
\pgfpathlineto{\pgfqpoint{1.097727in}{1.021898in}}%
\pgfpathlineto{\pgfqpoint{1.089625in}{1.033363in}}%
\pgfpathlineto{\pgfqpoint{1.081523in}{1.040160in}}%
\pgfpathlineto{\pgfqpoint{1.073420in}{1.049035in}}%
\pgfpathlineto{\pgfqpoint{1.065318in}{1.058040in}}%
\pgfpathlineto{\pgfqpoint{1.057216in}{1.064866in}}%
\pgfpathlineto{\pgfqpoint{1.049114in}{1.069016in}}%
\pgfpathlineto{\pgfqpoint{1.041011in}{1.076899in}}%
\pgfpathlineto{\pgfqpoint{1.032909in}{1.086355in}}%
\pgfpathlineto{\pgfqpoint{1.024807in}{1.095789in}}%
\pgfpathlineto{\pgfqpoint{1.016705in}{1.105492in}}%
\pgfpathlineto{\pgfqpoint{1.008602in}{1.115126in}}%
\pgfpathlineto{\pgfqpoint{1.000500in}{1.124958in}}%
\pgfpathlineto{\pgfqpoint{0.992398in}{1.125828in}}%
\pgfpathlineto{\pgfqpoint{0.984295in}{1.135588in}}%
\pgfpathlineto{\pgfqpoint{0.976193in}{1.147349in}}%
\pgfpathlineto{\pgfqpoint{0.968091in}{1.153609in}}%
\pgfpathlineto{\pgfqpoint{0.959989in}{1.164092in}}%
\pgfpathlineto{\pgfqpoint{0.951886in}{1.174750in}}%
\pgfpathlineto{\pgfqpoint{0.943784in}{1.185606in}}%
\pgfpathlineto{\pgfqpoint{0.935682in}{1.196555in}}%
\pgfpathlineto{\pgfqpoint{0.927580in}{1.207669in}}%
\pgfpathlineto{\pgfqpoint{0.919477in}{1.221423in}}%
\pgfpathlineto{\pgfqpoint{0.911375in}{1.212699in}}%
\pgfpathlineto{\pgfqpoint{0.903273in}{1.223664in}}%
\pgfpathlineto{\pgfqpoint{0.895170in}{1.235194in}}%
\pgfpathlineto{\pgfqpoint{0.887068in}{1.247242in}}%
\pgfpathlineto{\pgfqpoint{0.878966in}{1.258501in}}%
\pgfpathlineto{\pgfqpoint{0.870864in}{1.268624in}}%
\pgfpathlineto{\pgfqpoint{0.862761in}{1.281211in}}%
\pgfpathlineto{\pgfqpoint{0.854659in}{1.294058in}}%
\pgfpathlineto{\pgfqpoint{0.846557in}{1.302814in}}%
\pgfpathlineto{\pgfqpoint{0.838455in}{1.316034in}}%
\pgfpathlineto{\pgfqpoint{0.830352in}{1.321735in}}%
\pgfpathlineto{\pgfqpoint{0.822250in}{1.329423in}}%
\pgfpathlineto{\pgfqpoint{0.814148in}{1.341884in}}%
\pgfpathlineto{\pgfqpoint{0.806045in}{1.345073in}}%
\pgfpathlineto{\pgfqpoint{0.797943in}{1.359495in}}%
\pgfpathlineto{\pgfqpoint{0.789841in}{1.374229in}}%
\pgfpathlineto{\pgfqpoint{0.781739in}{1.378474in}}%
\pgfpathlineto{\pgfqpoint{0.773636in}{1.385654in}}%
\pgfpathlineto{\pgfqpoint{0.765534in}{1.400929in}}%
\pgfpathlineto{\pgfqpoint{0.757432in}{1.412104in}}%
\pgfpathlineto{\pgfqpoint{0.749330in}{1.425941in}}%
\pgfpathlineto{\pgfqpoint{0.741227in}{1.436801in}}%
\pgfpathlineto{\pgfqpoint{0.733125in}{1.447811in}}%
\pgfpathlineto{\pgfqpoint{0.725023in}{1.443954in}}%
\pgfpathlineto{\pgfqpoint{0.716920in}{1.451906in}}%
\pgfpathlineto{\pgfqpoint{0.708818in}{1.467132in}}%
\pgfpathlineto{\pgfqpoint{0.700716in}{1.483596in}}%
\pgfpathlineto{\pgfqpoint{0.692614in}{1.491375in}}%
\pgfpathlineto{\pgfqpoint{0.684511in}{1.507123in}}%
\pgfpathlineto{\pgfqpoint{0.676409in}{1.523628in}}%
\pgfpathlineto{\pgfqpoint{0.668307in}{1.529224in}}%
\pgfpathlineto{\pgfqpoint{0.660205in}{1.545334in}}%
\pgfpathlineto{\pgfqpoint{0.652102in}{1.558027in}}%
\pgfpathlineto{\pgfqpoint{0.644000in}{1.563923in}}%
\pgfpathlineto{\pgfqpoint{0.635898in}{1.567375in}}%
\pgfpathlineto{\pgfqpoint{0.627795in}{1.578019in}}%
\pgfpathlineto{\pgfqpoint{0.619693in}{1.581231in}}%
\pgfpathlineto{\pgfqpoint{0.611591in}{1.591316in}}%
\pgfpathlineto{\pgfqpoint{0.603489in}{1.603063in}}%
\pgfpathlineto{\pgfqpoint{0.595386in}{1.615213in}}%
\pgfpathlineto{\pgfqpoint{0.587284in}{1.606476in}}%
\pgfpathlineto{\pgfqpoint{0.579182in}{1.600665in}}%
\pgfpathlineto{\pgfqpoint{0.571080in}{1.593085in}}%
\pgfpathlineto{\pgfqpoint{0.562977in}{1.607761in}}%
\pgfpathlineto{\pgfqpoint{0.554875in}{1.626892in}}%
\pgfpathlineto{\pgfqpoint{0.546773in}{1.622544in}}%
\pgfpathlineto{\pgfqpoint{0.538670in}{1.647044in}}%
\pgfpathlineto{\pgfqpoint{0.530568in}{1.629529in}}%
\pgfpathlineto{\pgfqpoint{0.522466in}{1.642987in}}%
\pgfpathlineto{\pgfqpoint{0.514364in}{1.655133in}}%
\pgfpathlineto{\pgfqpoint{0.506261in}{1.636951in}}%
\pgfpathlineto{\pgfqpoint{0.498159in}{1.659233in}}%
\pgfpathlineto{\pgfqpoint{0.490057in}{1.664430in}}%
\pgfpathlineto{\pgfqpoint{0.481955in}{1.676886in}}%
\pgfpathlineto{\pgfqpoint{0.473852in}{1.667872in}}%
\pgfpathlineto{\pgfqpoint{0.465750in}{1.684755in}}%
\pgfpathlineto{\pgfqpoint{0.457648in}{1.701662in}}%
\pgfpathlineto{\pgfqpoint{0.449545in}{1.714521in}}%
\pgfpathlineto{\pgfqpoint{0.441443in}{1.703060in}}%
\pgfpathlineto{\pgfqpoint{0.433341in}{1.710566in}}%
\pgfpathlineto{\pgfqpoint{0.425239in}{1.734004in}}%
\pgfpathlineto{\pgfqpoint{0.417136in}{1.732755in}}%
\pgfpathlineto{\pgfqpoint{0.409034in}{1.753621in}}%
\pgfpathlineto{\pgfqpoint{0.400932in}{1.757007in}}%
\pgfpathlineto{\pgfqpoint{0.392830in}{1.746208in}}%
\pgfpathlineto{\pgfqpoint{0.384727in}{1.733031in}}%
\pgfpathlineto{\pgfqpoint{0.376625in}{1.686063in}}%
\pgfpathlineto{\pgfqpoint{0.368523in}{1.836121in}}%
\pgfpathclose%
\pgfusepath{fill}%
\end{pgfscope}%
\begin{pgfscope}%
\pgfpathrectangle{\pgfqpoint{0.287500in}{0.375000in}}{\pgfqpoint{1.782500in}{2.265000in}}%
\pgfusepath{clip}%
\pgfsetbuttcap%
\pgfsetroundjoin%
\definecolor{currentfill}{rgb}{1.000000,0.498039,0.054902}%
\pgfsetfillcolor{currentfill}%
\pgfsetfillopacity{0.200000}%
\pgfsetlinewidth{0.000000pt}%
\definecolor{currentstroke}{rgb}{0.000000,0.000000,0.000000}%
\pgfsetstrokecolor{currentstroke}%
\pgfsetdash{}{0pt}%
\pgfpathmoveto{\pgfqpoint{0.368523in}{1.816135in}}%
\pgfpathlineto{\pgfqpoint{0.368523in}{2.345782in}}%
\pgfpathlineto{\pgfqpoint{0.376625in}{2.069868in}}%
\pgfpathlineto{\pgfqpoint{0.384727in}{1.976054in}}%
\pgfpathlineto{\pgfqpoint{0.392830in}{1.959879in}}%
\pgfpathlineto{\pgfqpoint{0.400932in}{1.921592in}}%
\pgfpathlineto{\pgfqpoint{0.409034in}{1.876423in}}%
\pgfpathlineto{\pgfqpoint{0.417136in}{1.878440in}}%
\pgfpathlineto{\pgfqpoint{0.425239in}{1.853156in}}%
\pgfpathlineto{\pgfqpoint{0.433341in}{1.811748in}}%
\pgfpathlineto{\pgfqpoint{0.441443in}{1.812012in}}%
\pgfpathlineto{\pgfqpoint{0.449545in}{1.813407in}}%
\pgfpathlineto{\pgfqpoint{0.457648in}{1.788548in}}%
\pgfpathlineto{\pgfqpoint{0.465750in}{1.779062in}}%
\pgfpathlineto{\pgfqpoint{0.473852in}{1.770614in}}%
\pgfpathlineto{\pgfqpoint{0.481955in}{1.758144in}}%
\pgfpathlineto{\pgfqpoint{0.490057in}{1.756633in}}%
\pgfpathlineto{\pgfqpoint{0.498159in}{1.747346in}}%
\pgfpathlineto{\pgfqpoint{0.506261in}{1.735228in}}%
\pgfpathlineto{\pgfqpoint{0.514364in}{1.735763in}}%
\pgfpathlineto{\pgfqpoint{0.522466in}{1.713377in}}%
\pgfpathlineto{\pgfqpoint{0.530568in}{1.705621in}}%
\pgfpathlineto{\pgfqpoint{0.538670in}{1.704754in}}%
\pgfpathlineto{\pgfqpoint{0.546773in}{1.686427in}}%
\pgfpathlineto{\pgfqpoint{0.554875in}{1.676176in}}%
\pgfpathlineto{\pgfqpoint{0.562977in}{1.663136in}}%
\pgfpathlineto{\pgfqpoint{0.571080in}{1.661064in}}%
\pgfpathlineto{\pgfqpoint{0.579182in}{1.662703in}}%
\pgfpathlineto{\pgfqpoint{0.587284in}{1.659137in}}%
\pgfpathlineto{\pgfqpoint{0.595386in}{1.652124in}}%
\pgfpathlineto{\pgfqpoint{0.603489in}{1.652639in}}%
\pgfpathlineto{\pgfqpoint{0.611591in}{1.657510in}}%
\pgfpathlineto{\pgfqpoint{0.619693in}{1.656309in}}%
\pgfpathlineto{\pgfqpoint{0.627795in}{1.654710in}}%
\pgfpathlineto{\pgfqpoint{0.635898in}{1.653870in}}%
\pgfpathlineto{\pgfqpoint{0.644000in}{1.646953in}}%
\pgfpathlineto{\pgfqpoint{0.652102in}{1.644592in}}%
\pgfpathlineto{\pgfqpoint{0.660205in}{1.651980in}}%
\pgfpathlineto{\pgfqpoint{0.668307in}{1.648255in}}%
\pgfpathlineto{\pgfqpoint{0.676409in}{1.654356in}}%
\pgfpathlineto{\pgfqpoint{0.684511in}{1.642319in}}%
\pgfpathlineto{\pgfqpoint{0.692614in}{1.637445in}}%
\pgfpathlineto{\pgfqpoint{0.700716in}{1.640118in}}%
\pgfpathlineto{\pgfqpoint{0.708818in}{1.638041in}}%
\pgfpathlineto{\pgfqpoint{0.716920in}{1.628099in}}%
\pgfpathlineto{\pgfqpoint{0.725023in}{1.615679in}}%
\pgfpathlineto{\pgfqpoint{0.733125in}{1.602405in}}%
\pgfpathlineto{\pgfqpoint{0.741227in}{1.598847in}}%
\pgfpathlineto{\pgfqpoint{0.749330in}{1.595588in}}%
\pgfpathlineto{\pgfqpoint{0.757432in}{1.579420in}}%
\pgfpathlineto{\pgfqpoint{0.765534in}{1.572483in}}%
\pgfpathlineto{\pgfqpoint{0.773636in}{1.563689in}}%
\pgfpathlineto{\pgfqpoint{0.781739in}{1.551073in}}%
\pgfpathlineto{\pgfqpoint{0.789841in}{1.543544in}}%
\pgfpathlineto{\pgfqpoint{0.797943in}{1.529175in}}%
\pgfpathlineto{\pgfqpoint{0.806045in}{1.522746in}}%
\pgfpathlineto{\pgfqpoint{0.814148in}{1.514796in}}%
\pgfpathlineto{\pgfqpoint{0.822250in}{1.504539in}}%
\pgfpathlineto{\pgfqpoint{0.830352in}{1.500110in}}%
\pgfpathlineto{\pgfqpoint{0.838455in}{1.500662in}}%
\pgfpathlineto{\pgfqpoint{0.846557in}{1.489404in}}%
\pgfpathlineto{\pgfqpoint{0.854659in}{1.475757in}}%
\pgfpathlineto{\pgfqpoint{0.862761in}{1.466048in}}%
\pgfpathlineto{\pgfqpoint{0.870864in}{1.471090in}}%
\pgfpathlineto{\pgfqpoint{0.878966in}{1.463881in}}%
\pgfpathlineto{\pgfqpoint{0.887068in}{1.455635in}}%
\pgfpathlineto{\pgfqpoint{0.895170in}{1.448263in}}%
\pgfpathlineto{\pgfqpoint{0.903273in}{1.451313in}}%
\pgfpathlineto{\pgfqpoint{0.911375in}{1.439133in}}%
\pgfpathlineto{\pgfqpoint{0.919477in}{1.431314in}}%
\pgfpathlineto{\pgfqpoint{0.927580in}{1.422128in}}%
\pgfpathlineto{\pgfqpoint{0.935682in}{1.423075in}}%
\pgfpathlineto{\pgfqpoint{0.943784in}{1.413970in}}%
\pgfpathlineto{\pgfqpoint{0.951886in}{1.404347in}}%
\pgfpathlineto{\pgfqpoint{0.959989in}{1.395430in}}%
\pgfpathlineto{\pgfqpoint{0.968091in}{1.386026in}}%
\pgfpathlineto{\pgfqpoint{0.976193in}{1.378043in}}%
\pgfpathlineto{\pgfqpoint{0.984295in}{1.368415in}}%
\pgfpathlineto{\pgfqpoint{0.992398in}{1.362515in}}%
\pgfpathlineto{\pgfqpoint{1.000500in}{1.354630in}}%
\pgfpathlineto{\pgfqpoint{1.008602in}{1.346991in}}%
\pgfpathlineto{\pgfqpoint{1.016705in}{1.338637in}}%
\pgfpathlineto{\pgfqpoint{1.024807in}{1.330332in}}%
\pgfpathlineto{\pgfqpoint{1.032909in}{1.325229in}}%
\pgfpathlineto{\pgfqpoint{1.041011in}{1.317289in}}%
\pgfpathlineto{\pgfqpoint{1.049114in}{1.314338in}}%
\pgfpathlineto{\pgfqpoint{1.057216in}{1.310808in}}%
\pgfpathlineto{\pgfqpoint{1.065318in}{1.314491in}}%
\pgfpathlineto{\pgfqpoint{1.073420in}{1.306222in}}%
\pgfpathlineto{\pgfqpoint{1.081523in}{1.296676in}}%
\pgfpathlineto{\pgfqpoint{1.089625in}{1.294017in}}%
\pgfpathlineto{\pgfqpoint{1.097727in}{1.286635in}}%
\pgfpathlineto{\pgfqpoint{1.105830in}{1.279660in}}%
\pgfpathlineto{\pgfqpoint{1.113932in}{1.276787in}}%
\pgfpathlineto{\pgfqpoint{1.122034in}{1.272047in}}%
\pgfpathlineto{\pgfqpoint{1.130136in}{1.263933in}}%
\pgfpathlineto{\pgfqpoint{1.138239in}{1.255813in}}%
\pgfpathlineto{\pgfqpoint{1.146341in}{1.253401in}}%
\pgfpathlineto{\pgfqpoint{1.154443in}{1.245854in}}%
\pgfpathlineto{\pgfqpoint{1.162545in}{1.240194in}}%
\pgfpathlineto{\pgfqpoint{1.170648in}{1.233558in}}%
\pgfpathlineto{\pgfqpoint{1.178750in}{1.229000in}}%
\pgfpathlineto{\pgfqpoint{1.186852in}{1.224964in}}%
\pgfpathlineto{\pgfqpoint{1.194955in}{1.215099in}}%
\pgfpathlineto{\pgfqpoint{1.203057in}{1.207928in}}%
\pgfpathlineto{\pgfqpoint{1.211159in}{1.205949in}}%
\pgfpathlineto{\pgfqpoint{1.219261in}{1.198751in}}%
\pgfpathlineto{\pgfqpoint{1.227364in}{1.199159in}}%
\pgfpathlineto{\pgfqpoint{1.235466in}{1.194170in}}%
\pgfpathlineto{\pgfqpoint{1.243568in}{1.186931in}}%
\pgfpathlineto{\pgfqpoint{1.251670in}{1.180252in}}%
\pgfpathlineto{\pgfqpoint{1.259773in}{1.173435in}}%
\pgfpathlineto{\pgfqpoint{1.267875in}{1.167922in}}%
\pgfpathlineto{\pgfqpoint{1.275977in}{1.161268in}}%
\pgfpathlineto{\pgfqpoint{1.284080in}{1.154290in}}%
\pgfpathlineto{\pgfqpoint{1.292182in}{1.148579in}}%
\pgfpathlineto{\pgfqpoint{1.300284in}{1.142070in}}%
\pgfpathlineto{\pgfqpoint{1.308386in}{1.135333in}}%
\pgfpathlineto{\pgfqpoint{1.316489in}{1.131806in}}%
\pgfpathlineto{\pgfqpoint{1.324591in}{1.125153in}}%
\pgfpathlineto{\pgfqpoint{1.332693in}{1.118963in}}%
\pgfpathlineto{\pgfqpoint{1.340795in}{1.113118in}}%
\pgfpathlineto{\pgfqpoint{1.348898in}{1.106501in}}%
\pgfpathlineto{\pgfqpoint{1.357000in}{1.102447in}}%
\pgfpathlineto{\pgfqpoint{1.365102in}{1.096299in}}%
\pgfpathlineto{\pgfqpoint{1.373205in}{1.089968in}}%
\pgfpathlineto{\pgfqpoint{1.381307in}{1.084482in}}%
\pgfpathlineto{\pgfqpoint{1.389409in}{1.078314in}}%
\pgfpathlineto{\pgfqpoint{1.397511in}{1.075500in}}%
\pgfpathlineto{\pgfqpoint{1.405614in}{1.069770in}}%
\pgfpathlineto{\pgfqpoint{1.413716in}{1.069901in}}%
\pgfpathlineto{\pgfqpoint{1.421818in}{1.063882in}}%
\pgfpathlineto{\pgfqpoint{1.429920in}{1.059739in}}%
\pgfpathlineto{\pgfqpoint{1.438023in}{1.053822in}}%
\pgfpathlineto{\pgfqpoint{1.446125in}{1.048491in}}%
\pgfpathlineto{\pgfqpoint{1.454227in}{1.044024in}}%
\pgfpathlineto{\pgfqpoint{1.462330in}{1.038227in}}%
\pgfpathlineto{\pgfqpoint{1.470432in}{1.032784in}}%
\pgfpathlineto{\pgfqpoint{1.478534in}{1.027201in}}%
\pgfpathlineto{\pgfqpoint{1.486636in}{1.021575in}}%
\pgfpathlineto{\pgfqpoint{1.494739in}{1.015460in}}%
\pgfpathlineto{\pgfqpoint{1.502841in}{1.009845in}}%
\pgfpathlineto{\pgfqpoint{1.510943in}{1.007391in}}%
\pgfpathlineto{\pgfqpoint{1.519045in}{1.001823in}}%
\pgfpathlineto{\pgfqpoint{1.527148in}{0.996371in}}%
\pgfpathlineto{\pgfqpoint{1.535250in}{0.990727in}}%
\pgfpathlineto{\pgfqpoint{1.543352in}{0.985376in}}%
\pgfpathlineto{\pgfqpoint{1.551455in}{0.987186in}}%
\pgfpathlineto{\pgfqpoint{1.559557in}{0.982134in}}%
\pgfpathlineto{\pgfqpoint{1.567659in}{0.977316in}}%
\pgfpathlineto{\pgfqpoint{1.575761in}{0.971748in}}%
\pgfpathlineto{\pgfqpoint{1.583864in}{0.967198in}}%
\pgfpathlineto{\pgfqpoint{1.591966in}{0.961988in}}%
\pgfpathlineto{\pgfqpoint{1.600068in}{0.957731in}}%
\pgfpathlineto{\pgfqpoint{1.608170in}{0.954534in}}%
\pgfpathlineto{\pgfqpoint{1.616273in}{0.949734in}}%
\pgfpathlineto{\pgfqpoint{1.624375in}{0.944661in}}%
\pgfpathlineto{\pgfqpoint{1.632477in}{0.939719in}}%
\pgfpathlineto{\pgfqpoint{1.640580in}{0.934759in}}%
\pgfpathlineto{\pgfqpoint{1.648682in}{0.929853in}}%
\pgfpathlineto{\pgfqpoint{1.656784in}{0.925976in}}%
\pgfpathlineto{\pgfqpoint{1.664886in}{0.920843in}}%
\pgfpathlineto{\pgfqpoint{1.672989in}{0.917162in}}%
\pgfpathlineto{\pgfqpoint{1.681091in}{0.912318in}}%
\pgfpathlineto{\pgfqpoint{1.689193in}{0.908171in}}%
\pgfpathlineto{\pgfqpoint{1.697295in}{0.906390in}}%
\pgfpathlineto{\pgfqpoint{1.705398in}{0.901614in}}%
\pgfpathlineto{\pgfqpoint{1.713500in}{0.898160in}}%
\pgfpathlineto{\pgfqpoint{1.721602in}{0.894476in}}%
\pgfpathlineto{\pgfqpoint{1.729705in}{0.889808in}}%
\pgfpathlineto{\pgfqpoint{1.737807in}{0.885275in}}%
\pgfpathlineto{\pgfqpoint{1.745909in}{0.881715in}}%
\pgfpathlineto{\pgfqpoint{1.754011in}{0.877673in}}%
\pgfpathlineto{\pgfqpoint{1.762114in}{0.878661in}}%
\pgfpathlineto{\pgfqpoint{1.770216in}{0.876410in}}%
\pgfpathlineto{\pgfqpoint{1.778318in}{0.871890in}}%
\pgfpathlineto{\pgfqpoint{1.786420in}{0.869017in}}%
\pgfpathlineto{\pgfqpoint{1.794523in}{0.865799in}}%
\pgfpathlineto{\pgfqpoint{1.802625in}{0.861376in}}%
\pgfpathlineto{\pgfqpoint{1.810727in}{0.856929in}}%
\pgfpathlineto{\pgfqpoint{1.818830in}{0.856154in}}%
\pgfpathlineto{\pgfqpoint{1.826932in}{0.851713in}}%
\pgfpathlineto{\pgfqpoint{1.835034in}{0.847928in}}%
\pgfpathlineto{\pgfqpoint{1.843136in}{0.847428in}}%
\pgfpathlineto{\pgfqpoint{1.851239in}{0.843023in}}%
\pgfpathlineto{\pgfqpoint{1.859341in}{0.839043in}}%
\pgfpathlineto{\pgfqpoint{1.867443in}{0.837801in}}%
\pgfpathlineto{\pgfqpoint{1.875545in}{0.836463in}}%
\pgfpathlineto{\pgfqpoint{1.883648in}{0.832230in}}%
\pgfpathlineto{\pgfqpoint{1.891750in}{0.828089in}}%
\pgfpathlineto{\pgfqpoint{1.899852in}{0.825387in}}%
\pgfpathlineto{\pgfqpoint{1.907955in}{0.821225in}}%
\pgfpathlineto{\pgfqpoint{1.916057in}{0.817145in}}%
\pgfpathlineto{\pgfqpoint{1.924159in}{0.813111in}}%
\pgfpathlineto{\pgfqpoint{1.932261in}{0.809323in}}%
\pgfpathlineto{\pgfqpoint{1.940364in}{0.805390in}}%
\pgfpathlineto{\pgfqpoint{1.948466in}{0.801382in}}%
\pgfpathlineto{\pgfqpoint{1.956568in}{0.799847in}}%
\pgfpathlineto{\pgfqpoint{1.964670in}{0.798615in}}%
\pgfpathlineto{\pgfqpoint{1.972773in}{0.797863in}}%
\pgfpathlineto{\pgfqpoint{1.980875in}{0.794436in}}%
\pgfpathlineto{\pgfqpoint{1.988977in}{0.790710in}}%
\pgfpathlineto{\pgfqpoint{1.988977in}{0.733242in}}%
\pgfpathlineto{\pgfqpoint{1.988977in}{0.733242in}}%
\pgfpathlineto{\pgfqpoint{1.980875in}{0.737169in}}%
\pgfpathlineto{\pgfqpoint{1.972773in}{0.739859in}}%
\pgfpathlineto{\pgfqpoint{1.964670in}{0.743875in}}%
\pgfpathlineto{\pgfqpoint{1.956568in}{0.747403in}}%
\pgfpathlineto{\pgfqpoint{1.948466in}{0.750872in}}%
\pgfpathlineto{\pgfqpoint{1.940364in}{0.754803in}}%
\pgfpathlineto{\pgfqpoint{1.932261in}{0.758828in}}%
\pgfpathlineto{\pgfqpoint{1.924159in}{0.762923in}}%
\pgfpathlineto{\pgfqpoint{1.916057in}{0.766987in}}%
\pgfpathlineto{\pgfqpoint{1.907955in}{0.770588in}}%
\pgfpathlineto{\pgfqpoint{1.899852in}{0.774041in}}%
\pgfpathlineto{\pgfqpoint{1.891750in}{0.778324in}}%
\pgfpathlineto{\pgfqpoint{1.883648in}{0.782496in}}%
\pgfpathlineto{\pgfqpoint{1.875545in}{0.786503in}}%
\pgfpathlineto{\pgfqpoint{1.867443in}{0.790576in}}%
\pgfpathlineto{\pgfqpoint{1.859341in}{0.794173in}}%
\pgfpathlineto{\pgfqpoint{1.851239in}{0.798357in}}%
\pgfpathlineto{\pgfqpoint{1.843136in}{0.801725in}}%
\pgfpathlineto{\pgfqpoint{1.835034in}{0.801779in}}%
\pgfpathlineto{\pgfqpoint{1.826932in}{0.804752in}}%
\pgfpathlineto{\pgfqpoint{1.818830in}{0.808345in}}%
\pgfpathlineto{\pgfqpoint{1.810727in}{0.805417in}}%
\pgfpathlineto{\pgfqpoint{1.802625in}{0.808486in}}%
\pgfpathlineto{\pgfqpoint{1.794523in}{0.812920in}}%
\pgfpathlineto{\pgfqpoint{1.786420in}{0.816806in}}%
\pgfpathlineto{\pgfqpoint{1.778318in}{0.820067in}}%
\pgfpathlineto{\pgfqpoint{1.770216in}{0.824587in}}%
\pgfpathlineto{\pgfqpoint{1.762114in}{0.828654in}}%
\pgfpathlineto{\pgfqpoint{1.754011in}{0.832985in}}%
\pgfpathlineto{\pgfqpoint{1.745909in}{0.835042in}}%
\pgfpathlineto{\pgfqpoint{1.737807in}{0.838869in}}%
\pgfpathlineto{\pgfqpoint{1.729705in}{0.842134in}}%
\pgfpathlineto{\pgfqpoint{1.721602in}{0.846815in}}%
\pgfpathlineto{\pgfqpoint{1.713500in}{0.851140in}}%
\pgfpathlineto{\pgfqpoint{1.705398in}{0.854550in}}%
\pgfpathlineto{\pgfqpoint{1.697295in}{0.859314in}}%
\pgfpathlineto{\pgfqpoint{1.689193in}{0.858265in}}%
\pgfpathlineto{\pgfqpoint{1.681091in}{0.862145in}}%
\pgfpathlineto{\pgfqpoint{1.672989in}{0.866976in}}%
\pgfpathlineto{\pgfqpoint{1.664886in}{0.863691in}}%
\pgfpathlineto{\pgfqpoint{1.656784in}{0.864380in}}%
\pgfpathlineto{\pgfqpoint{1.648682in}{0.861331in}}%
\pgfpathlineto{\pgfqpoint{1.640580in}{0.866179in}}%
\pgfpathlineto{\pgfqpoint{1.632477in}{0.870625in}}%
\pgfpathlineto{\pgfqpoint{1.624375in}{0.874519in}}%
\pgfpathlineto{\pgfqpoint{1.616273in}{0.879532in}}%
\pgfpathlineto{\pgfqpoint{1.608170in}{0.884162in}}%
\pgfpathlineto{\pgfqpoint{1.600068in}{0.887232in}}%
\pgfpathlineto{\pgfqpoint{1.591966in}{0.891797in}}%
\pgfpathlineto{\pgfqpoint{1.583864in}{0.896991in}}%
\pgfpathlineto{\pgfqpoint{1.575761in}{0.900580in}}%
\pgfpathlineto{\pgfqpoint{1.567659in}{0.901457in}}%
\pgfpathlineto{\pgfqpoint{1.559557in}{0.906128in}}%
\pgfpathlineto{\pgfqpoint{1.551455in}{0.911020in}}%
\pgfpathlineto{\pgfqpoint{1.543352in}{0.916934in}}%
\pgfpathlineto{\pgfqpoint{1.535250in}{0.922230in}}%
\pgfpathlineto{\pgfqpoint{1.527148in}{0.925919in}}%
\pgfpathlineto{\pgfqpoint{1.519045in}{0.931375in}}%
\pgfpathlineto{\pgfqpoint{1.510943in}{0.932773in}}%
\pgfpathlineto{\pgfqpoint{1.502841in}{0.937439in}}%
\pgfpathlineto{\pgfqpoint{1.494739in}{0.943054in}}%
\pgfpathlineto{\pgfqpoint{1.486636in}{0.944516in}}%
\pgfpathlineto{\pgfqpoint{1.478534in}{0.950156in}}%
\pgfpathlineto{\pgfqpoint{1.470432in}{0.955595in}}%
\pgfpathlineto{\pgfqpoint{1.462330in}{0.960705in}}%
\pgfpathlineto{\pgfqpoint{1.454227in}{0.966451in}}%
\pgfpathlineto{\pgfqpoint{1.446125in}{0.971953in}}%
\pgfpathlineto{\pgfqpoint{1.438023in}{0.976936in}}%
\pgfpathlineto{\pgfqpoint{1.429920in}{0.982723in}}%
\pgfpathlineto{\pgfqpoint{1.421818in}{0.988738in}}%
\pgfpathlineto{\pgfqpoint{1.413716in}{0.994757in}}%
\pgfpathlineto{\pgfqpoint{1.405614in}{0.994258in}}%
\pgfpathlineto{\pgfqpoint{1.397511in}{0.999256in}}%
\pgfpathlineto{\pgfqpoint{1.389409in}{1.005247in}}%
\pgfpathlineto{\pgfqpoint{1.381307in}{1.011399in}}%
\pgfpathlineto{\pgfqpoint{1.373205in}{1.014374in}}%
\pgfpathlineto{\pgfqpoint{1.365102in}{1.020688in}}%
\pgfpathlineto{\pgfqpoint{1.357000in}{1.026875in}}%
\pgfpathlineto{\pgfqpoint{1.348898in}{1.029687in}}%
\pgfpathlineto{\pgfqpoint{1.340795in}{1.031229in}}%
\pgfpathlineto{\pgfqpoint{1.332693in}{1.037321in}}%
\pgfpathlineto{\pgfqpoint{1.324591in}{1.043899in}}%
\pgfpathlineto{\pgfqpoint{1.316489in}{1.050500in}}%
\pgfpathlineto{\pgfqpoint{1.308386in}{1.057644in}}%
\pgfpathlineto{\pgfqpoint{1.300284in}{1.064401in}}%
\pgfpathlineto{\pgfqpoint{1.292182in}{1.070959in}}%
\pgfpathlineto{\pgfqpoint{1.284080in}{1.077737in}}%
\pgfpathlineto{\pgfqpoint{1.275977in}{1.084093in}}%
\pgfpathlineto{\pgfqpoint{1.267875in}{1.090711in}}%
\pgfpathlineto{\pgfqpoint{1.259773in}{1.097765in}}%
\pgfpathlineto{\pgfqpoint{1.251670in}{1.101129in}}%
\pgfpathlineto{\pgfqpoint{1.243568in}{1.099886in}}%
\pgfpathlineto{\pgfqpoint{1.235466in}{1.106990in}}%
\pgfpathlineto{\pgfqpoint{1.227364in}{1.106208in}}%
\pgfpathlineto{\pgfqpoint{1.219261in}{1.108754in}}%
\pgfpathlineto{\pgfqpoint{1.211159in}{1.113775in}}%
\pgfpathlineto{\pgfqpoint{1.203057in}{1.120785in}}%
\pgfpathlineto{\pgfqpoint{1.194955in}{1.127163in}}%
\pgfpathlineto{\pgfqpoint{1.186852in}{1.123117in}}%
\pgfpathlineto{\pgfqpoint{1.178750in}{1.129414in}}%
\pgfpathlineto{\pgfqpoint{1.170648in}{1.134357in}}%
\pgfpathlineto{\pgfqpoint{1.162545in}{1.141669in}}%
\pgfpathlineto{\pgfqpoint{1.154443in}{1.146664in}}%
\pgfpathlineto{\pgfqpoint{1.146341in}{1.152458in}}%
\pgfpathlineto{\pgfqpoint{1.138239in}{1.155676in}}%
\pgfpathlineto{\pgfqpoint{1.130136in}{1.160629in}}%
\pgfpathlineto{\pgfqpoint{1.122034in}{1.168948in}}%
\pgfpathlineto{\pgfqpoint{1.113932in}{1.174691in}}%
\pgfpathlineto{\pgfqpoint{1.105830in}{1.173231in}}%
\pgfpathlineto{\pgfqpoint{1.097727in}{1.180404in}}%
\pgfpathlineto{\pgfqpoint{1.089625in}{1.189325in}}%
\pgfpathlineto{\pgfqpoint{1.081523in}{1.192981in}}%
\pgfpathlineto{\pgfqpoint{1.073420in}{1.198167in}}%
\pgfpathlineto{\pgfqpoint{1.065318in}{1.205292in}}%
\pgfpathlineto{\pgfqpoint{1.057216in}{1.208878in}}%
\pgfpathlineto{\pgfqpoint{1.049114in}{1.217449in}}%
\pgfpathlineto{\pgfqpoint{1.041011in}{1.221239in}}%
\pgfpathlineto{\pgfqpoint{1.032909in}{1.230046in}}%
\pgfpathlineto{\pgfqpoint{1.024807in}{1.237287in}}%
\pgfpathlineto{\pgfqpoint{1.016705in}{1.247072in}}%
\pgfpathlineto{\pgfqpoint{1.008602in}{1.252454in}}%
\pgfpathlineto{\pgfqpoint{1.000500in}{1.261736in}}%
\pgfpathlineto{\pgfqpoint{0.992398in}{1.271714in}}%
\pgfpathlineto{\pgfqpoint{0.984295in}{1.271905in}}%
\pgfpathlineto{\pgfqpoint{0.976193in}{1.281780in}}%
\pgfpathlineto{\pgfqpoint{0.968091in}{1.290700in}}%
\pgfpathlineto{\pgfqpoint{0.959989in}{1.298719in}}%
\pgfpathlineto{\pgfqpoint{0.951886in}{1.309637in}}%
\pgfpathlineto{\pgfqpoint{0.943784in}{1.319689in}}%
\pgfpathlineto{\pgfqpoint{0.935682in}{1.329030in}}%
\pgfpathlineto{\pgfqpoint{0.927580in}{1.337866in}}%
\pgfpathlineto{\pgfqpoint{0.919477in}{1.342159in}}%
\pgfpathlineto{\pgfqpoint{0.911375in}{1.352556in}}%
\pgfpathlineto{\pgfqpoint{0.903273in}{1.359384in}}%
\pgfpathlineto{\pgfqpoint{0.895170in}{1.368849in}}%
\pgfpathlineto{\pgfqpoint{0.887068in}{1.380252in}}%
\pgfpathlineto{\pgfqpoint{0.878966in}{1.391810in}}%
\pgfpathlineto{\pgfqpoint{0.870864in}{1.401452in}}%
\pgfpathlineto{\pgfqpoint{0.862761in}{1.404983in}}%
\pgfpathlineto{\pgfqpoint{0.854659in}{1.403575in}}%
\pgfpathlineto{\pgfqpoint{0.846557in}{1.411191in}}%
\pgfpathlineto{\pgfqpoint{0.838455in}{1.411985in}}%
\pgfpathlineto{\pgfqpoint{0.830352in}{1.420013in}}%
\pgfpathlineto{\pgfqpoint{0.822250in}{1.429926in}}%
\pgfpathlineto{\pgfqpoint{0.814148in}{1.434510in}}%
\pgfpathlineto{\pgfqpoint{0.806045in}{1.441342in}}%
\pgfpathlineto{\pgfqpoint{0.797943in}{1.444907in}}%
\pgfpathlineto{\pgfqpoint{0.789841in}{1.459155in}}%
\pgfpathlineto{\pgfqpoint{0.781739in}{1.470539in}}%
\pgfpathlineto{\pgfqpoint{0.773636in}{1.482009in}}%
\pgfpathlineto{\pgfqpoint{0.765534in}{1.491331in}}%
\pgfpathlineto{\pgfqpoint{0.757432in}{1.500831in}}%
\pgfpathlineto{\pgfqpoint{0.749330in}{1.506900in}}%
\pgfpathlineto{\pgfqpoint{0.741227in}{1.507046in}}%
\pgfpathlineto{\pgfqpoint{0.733125in}{1.505525in}}%
\pgfpathlineto{\pgfqpoint{0.725023in}{1.502241in}}%
\pgfpathlineto{\pgfqpoint{0.716920in}{1.506798in}}%
\pgfpathlineto{\pgfqpoint{0.708818in}{1.508253in}}%
\pgfpathlineto{\pgfqpoint{0.700716in}{1.519485in}}%
\pgfpathlineto{\pgfqpoint{0.692614in}{1.527488in}}%
\pgfpathlineto{\pgfqpoint{0.684511in}{1.517544in}}%
\pgfpathlineto{\pgfqpoint{0.676409in}{1.530109in}}%
\pgfpathlineto{\pgfqpoint{0.668307in}{1.536084in}}%
\pgfpathlineto{\pgfqpoint{0.660205in}{1.530305in}}%
\pgfpathlineto{\pgfqpoint{0.652102in}{1.522953in}}%
\pgfpathlineto{\pgfqpoint{0.644000in}{1.530107in}}%
\pgfpathlineto{\pgfqpoint{0.635898in}{1.528670in}}%
\pgfpathlineto{\pgfqpoint{0.627795in}{1.536496in}}%
\pgfpathlineto{\pgfqpoint{0.619693in}{1.526972in}}%
\pgfpathlineto{\pgfqpoint{0.611591in}{1.532321in}}%
\pgfpathlineto{\pgfqpoint{0.603489in}{1.529907in}}%
\pgfpathlineto{\pgfqpoint{0.595386in}{1.543589in}}%
\pgfpathlineto{\pgfqpoint{0.587284in}{1.542318in}}%
\pgfpathlineto{\pgfqpoint{0.579182in}{1.541499in}}%
\pgfpathlineto{\pgfqpoint{0.571080in}{1.544845in}}%
\pgfpathlineto{\pgfqpoint{0.562977in}{1.555540in}}%
\pgfpathlineto{\pgfqpoint{0.554875in}{1.562556in}}%
\pgfpathlineto{\pgfqpoint{0.546773in}{1.572760in}}%
\pgfpathlineto{\pgfqpoint{0.538670in}{1.560593in}}%
\pgfpathlineto{\pgfqpoint{0.530568in}{1.575776in}}%
\pgfpathlineto{\pgfqpoint{0.522466in}{1.596928in}}%
\pgfpathlineto{\pgfqpoint{0.514364in}{1.599192in}}%
\pgfpathlineto{\pgfqpoint{0.506261in}{1.594455in}}%
\pgfpathlineto{\pgfqpoint{0.498159in}{1.600719in}}%
\pgfpathlineto{\pgfqpoint{0.490057in}{1.607047in}}%
\pgfpathlineto{\pgfqpoint{0.481955in}{1.620030in}}%
\pgfpathlineto{\pgfqpoint{0.473852in}{1.606576in}}%
\pgfpathlineto{\pgfqpoint{0.465750in}{1.580672in}}%
\pgfpathlineto{\pgfqpoint{0.457648in}{1.613628in}}%
\pgfpathlineto{\pgfqpoint{0.449545in}{1.622731in}}%
\pgfpathlineto{\pgfqpoint{0.441443in}{1.664792in}}%
\pgfpathlineto{\pgfqpoint{0.433341in}{1.685270in}}%
\pgfpathlineto{\pgfqpoint{0.425239in}{1.698420in}}%
\pgfpathlineto{\pgfqpoint{0.417136in}{1.716869in}}%
\pgfpathlineto{\pgfqpoint{0.409034in}{1.708597in}}%
\pgfpathlineto{\pgfqpoint{0.400932in}{1.728602in}}%
\pgfpathlineto{\pgfqpoint{0.392830in}{1.757763in}}%
\pgfpathlineto{\pgfqpoint{0.384727in}{1.782229in}}%
\pgfpathlineto{\pgfqpoint{0.376625in}{1.841440in}}%
\pgfpathlineto{\pgfqpoint{0.368523in}{1.816135in}}%
\pgfpathclose%
\pgfusepath{fill}%
\end{pgfscope}%
\begin{pgfscope}%
\pgfpathrectangle{\pgfqpoint{0.287500in}{0.375000in}}{\pgfqpoint{1.782500in}{2.265000in}}%
\pgfusepath{clip}%
\pgfsetbuttcap%
\pgfsetroundjoin%
\definecolor{currentfill}{rgb}{0.172549,0.627451,0.172549}%
\pgfsetfillcolor{currentfill}%
\pgfsetfillopacity{0.200000}%
\pgfsetlinewidth{0.000000pt}%
\definecolor{currentstroke}{rgb}{0.000000,0.000000,0.000000}%
\pgfsetstrokecolor{currentstroke}%
\pgfsetdash{}{0pt}%
\pgfpathmoveto{\pgfqpoint{0.368523in}{2.232213in}}%
\pgfpathlineto{\pgfqpoint{0.368523in}{2.537045in}}%
\pgfpathlineto{\pgfqpoint{0.376625in}{2.179455in}}%
\pgfpathlineto{\pgfqpoint{0.384727in}{2.041411in}}%
\pgfpathlineto{\pgfqpoint{0.392830in}{1.954001in}}%
\pgfpathlineto{\pgfqpoint{0.400932in}{1.902748in}}%
\pgfpathlineto{\pgfqpoint{0.409034in}{1.859594in}}%
\pgfpathlineto{\pgfqpoint{0.417136in}{1.802309in}}%
\pgfpathlineto{\pgfqpoint{0.425239in}{1.754260in}}%
\pgfpathlineto{\pgfqpoint{0.433341in}{1.733372in}}%
\pgfpathlineto{\pgfqpoint{0.441443in}{1.714659in}}%
\pgfpathlineto{\pgfqpoint{0.449545in}{1.693585in}}%
\pgfpathlineto{\pgfqpoint{0.457648in}{1.667187in}}%
\pgfpathlineto{\pgfqpoint{0.465750in}{1.642795in}}%
\pgfpathlineto{\pgfqpoint{0.473852in}{1.635587in}}%
\pgfpathlineto{\pgfqpoint{0.481955in}{1.632288in}}%
\pgfpathlineto{\pgfqpoint{0.490057in}{1.628510in}}%
\pgfpathlineto{\pgfqpoint{0.498159in}{1.615120in}}%
\pgfpathlineto{\pgfqpoint{0.506261in}{1.602395in}}%
\pgfpathlineto{\pgfqpoint{0.514364in}{1.589313in}}%
\pgfpathlineto{\pgfqpoint{0.522466in}{1.586450in}}%
\pgfpathlineto{\pgfqpoint{0.530568in}{1.585065in}}%
\pgfpathlineto{\pgfqpoint{0.538670in}{1.578944in}}%
\pgfpathlineto{\pgfqpoint{0.546773in}{1.578127in}}%
\pgfpathlineto{\pgfqpoint{0.554875in}{1.573057in}}%
\pgfpathlineto{\pgfqpoint{0.562977in}{1.564023in}}%
\pgfpathlineto{\pgfqpoint{0.571080in}{1.553216in}}%
\pgfpathlineto{\pgfqpoint{0.579182in}{1.549017in}}%
\pgfpathlineto{\pgfqpoint{0.587284in}{1.549300in}}%
\pgfpathlineto{\pgfqpoint{0.595386in}{1.540661in}}%
\pgfpathlineto{\pgfqpoint{0.603489in}{1.522514in}}%
\pgfpathlineto{\pgfqpoint{0.611591in}{1.507118in}}%
\pgfpathlineto{\pgfqpoint{0.619693in}{1.504961in}}%
\pgfpathlineto{\pgfqpoint{0.627795in}{1.503580in}}%
\pgfpathlineto{\pgfqpoint{0.635898in}{1.498597in}}%
\pgfpathlineto{\pgfqpoint{0.644000in}{1.495288in}}%
\pgfpathlineto{\pgfqpoint{0.652102in}{1.497708in}}%
\pgfpathlineto{\pgfqpoint{0.660205in}{1.507121in}}%
\pgfpathlineto{\pgfqpoint{0.668307in}{1.499992in}}%
\pgfpathlineto{\pgfqpoint{0.676409in}{1.489331in}}%
\pgfpathlineto{\pgfqpoint{0.684511in}{1.477653in}}%
\pgfpathlineto{\pgfqpoint{0.692614in}{1.478513in}}%
\pgfpathlineto{\pgfqpoint{0.700716in}{1.470232in}}%
\pgfpathlineto{\pgfqpoint{0.708818in}{1.458386in}}%
\pgfpathlineto{\pgfqpoint{0.716920in}{1.448994in}}%
\pgfpathlineto{\pgfqpoint{0.725023in}{1.443048in}}%
\pgfpathlineto{\pgfqpoint{0.733125in}{1.437242in}}%
\pgfpathlineto{\pgfqpoint{0.741227in}{1.428352in}}%
\pgfpathlineto{\pgfqpoint{0.749330in}{1.422501in}}%
\pgfpathlineto{\pgfqpoint{0.757432in}{1.422456in}}%
\pgfpathlineto{\pgfqpoint{0.765534in}{1.419569in}}%
\pgfpathlineto{\pgfqpoint{0.773636in}{1.415280in}}%
\pgfpathlineto{\pgfqpoint{0.781739in}{1.421569in}}%
\pgfpathlineto{\pgfqpoint{0.789841in}{1.422281in}}%
\pgfpathlineto{\pgfqpoint{0.797943in}{1.419410in}}%
\pgfpathlineto{\pgfqpoint{0.806045in}{1.424018in}}%
\pgfpathlineto{\pgfqpoint{0.814148in}{1.419433in}}%
\pgfpathlineto{\pgfqpoint{0.822250in}{1.425946in}}%
\pgfpathlineto{\pgfqpoint{0.830352in}{1.420318in}}%
\pgfpathlineto{\pgfqpoint{0.838455in}{1.423139in}}%
\pgfpathlineto{\pgfqpoint{0.846557in}{1.423365in}}%
\pgfpathlineto{\pgfqpoint{0.854659in}{1.422027in}}%
\pgfpathlineto{\pgfqpoint{0.862761in}{1.426772in}}%
\pgfpathlineto{\pgfqpoint{0.870864in}{1.425962in}}%
\pgfpathlineto{\pgfqpoint{0.878966in}{1.423793in}}%
\pgfpathlineto{\pgfqpoint{0.887068in}{1.423480in}}%
\pgfpathlineto{\pgfqpoint{0.895170in}{1.424264in}}%
\pgfpathlineto{\pgfqpoint{0.903273in}{1.429073in}}%
\pgfpathlineto{\pgfqpoint{0.911375in}{1.426344in}}%
\pgfpathlineto{\pgfqpoint{0.919477in}{1.431274in}}%
\pgfpathlineto{\pgfqpoint{0.927580in}{1.436570in}}%
\pgfpathlineto{\pgfqpoint{0.935682in}{1.438887in}}%
\pgfpathlineto{\pgfqpoint{0.943784in}{1.437358in}}%
\pgfpathlineto{\pgfqpoint{0.951886in}{1.435131in}}%
\pgfpathlineto{\pgfqpoint{0.959989in}{1.440288in}}%
\pgfpathlineto{\pgfqpoint{0.968091in}{1.437453in}}%
\pgfpathlineto{\pgfqpoint{0.976193in}{1.436553in}}%
\pgfpathlineto{\pgfqpoint{0.984295in}{1.434561in}}%
\pgfpathlineto{\pgfqpoint{0.992398in}{1.431822in}}%
\pgfpathlineto{\pgfqpoint{1.000500in}{1.430611in}}%
\pgfpathlineto{\pgfqpoint{1.008602in}{1.434854in}}%
\pgfpathlineto{\pgfqpoint{1.016705in}{1.433060in}}%
\pgfpathlineto{\pgfqpoint{1.024807in}{1.434236in}}%
\pgfpathlineto{\pgfqpoint{1.032909in}{1.429184in}}%
\pgfpathlineto{\pgfqpoint{1.041011in}{1.430787in}}%
\pgfpathlineto{\pgfqpoint{1.049114in}{1.428338in}}%
\pgfpathlineto{\pgfqpoint{1.057216in}{1.428144in}}%
\pgfpathlineto{\pgfqpoint{1.065318in}{1.430969in}}%
\pgfpathlineto{\pgfqpoint{1.073420in}{1.429639in}}%
\pgfpathlineto{\pgfqpoint{1.081523in}{1.427626in}}%
\pgfpathlineto{\pgfqpoint{1.089625in}{1.422876in}}%
\pgfpathlineto{\pgfqpoint{1.097727in}{1.422775in}}%
\pgfpathlineto{\pgfqpoint{1.105830in}{1.421411in}}%
\pgfpathlineto{\pgfqpoint{1.113932in}{1.420924in}}%
\pgfpathlineto{\pgfqpoint{1.122034in}{1.417869in}}%
\pgfpathlineto{\pgfqpoint{1.130136in}{1.416576in}}%
\pgfpathlineto{\pgfqpoint{1.138239in}{1.411045in}}%
\pgfpathlineto{\pgfqpoint{1.146341in}{1.407025in}}%
\pgfpathlineto{\pgfqpoint{1.154443in}{1.403836in}}%
\pgfpathlineto{\pgfqpoint{1.162545in}{1.403047in}}%
\pgfpathlineto{\pgfqpoint{1.170648in}{1.402393in}}%
\pgfpathlineto{\pgfqpoint{1.178750in}{1.403491in}}%
\pgfpathlineto{\pgfqpoint{1.186852in}{1.410615in}}%
\pgfpathlineto{\pgfqpoint{1.194955in}{1.412489in}}%
\pgfpathlineto{\pgfqpoint{1.203057in}{1.415425in}}%
\pgfpathlineto{\pgfqpoint{1.211159in}{1.415548in}}%
\pgfpathlineto{\pgfqpoint{1.219261in}{1.415117in}}%
\pgfpathlineto{\pgfqpoint{1.227364in}{1.416086in}}%
\pgfpathlineto{\pgfqpoint{1.235466in}{1.416758in}}%
\pgfpathlineto{\pgfqpoint{1.243568in}{1.414393in}}%
\pgfpathlineto{\pgfqpoint{1.251670in}{1.412911in}}%
\pgfpathlineto{\pgfqpoint{1.259773in}{1.417613in}}%
\pgfpathlineto{\pgfqpoint{1.267875in}{1.418873in}}%
\pgfpathlineto{\pgfqpoint{1.275977in}{1.425717in}}%
\pgfpathlineto{\pgfqpoint{1.284080in}{1.426032in}}%
\pgfpathlineto{\pgfqpoint{1.292182in}{1.423040in}}%
\pgfpathlineto{\pgfqpoint{1.300284in}{1.421158in}}%
\pgfpathlineto{\pgfqpoint{1.308386in}{1.420670in}}%
\pgfpathlineto{\pgfqpoint{1.316489in}{1.417955in}}%
\pgfpathlineto{\pgfqpoint{1.324591in}{1.418294in}}%
\pgfpathlineto{\pgfqpoint{1.332693in}{1.418957in}}%
\pgfpathlineto{\pgfqpoint{1.340795in}{1.416944in}}%
\pgfpathlineto{\pgfqpoint{1.348898in}{1.415207in}}%
\pgfpathlineto{\pgfqpoint{1.357000in}{1.411382in}}%
\pgfpathlineto{\pgfqpoint{1.365102in}{1.409967in}}%
\pgfpathlineto{\pgfqpoint{1.373205in}{1.413290in}}%
\pgfpathlineto{\pgfqpoint{1.381307in}{1.411724in}}%
\pgfpathlineto{\pgfqpoint{1.389409in}{1.407980in}}%
\pgfpathlineto{\pgfqpoint{1.397511in}{1.406757in}}%
\pgfpathlineto{\pgfqpoint{1.405614in}{1.409550in}}%
\pgfpathlineto{\pgfqpoint{1.413716in}{1.409874in}}%
\pgfpathlineto{\pgfqpoint{1.421818in}{1.414706in}}%
\pgfpathlineto{\pgfqpoint{1.429920in}{1.412103in}}%
\pgfpathlineto{\pgfqpoint{1.438023in}{1.412328in}}%
\pgfpathlineto{\pgfqpoint{1.446125in}{1.412032in}}%
\pgfpathlineto{\pgfqpoint{1.454227in}{1.411522in}}%
\pgfpathlineto{\pgfqpoint{1.462330in}{1.411066in}}%
\pgfpathlineto{\pgfqpoint{1.470432in}{1.408935in}}%
\pgfpathlineto{\pgfqpoint{1.478534in}{1.410567in}}%
\pgfpathlineto{\pgfqpoint{1.486636in}{1.411769in}}%
\pgfpathlineto{\pgfqpoint{1.494739in}{1.413405in}}%
\pgfpathlineto{\pgfqpoint{1.502841in}{1.413956in}}%
\pgfpathlineto{\pgfqpoint{1.510943in}{1.413238in}}%
\pgfpathlineto{\pgfqpoint{1.519045in}{1.414272in}}%
\pgfpathlineto{\pgfqpoint{1.527148in}{1.412436in}}%
\pgfpathlineto{\pgfqpoint{1.535250in}{1.408467in}}%
\pgfpathlineto{\pgfqpoint{1.543352in}{1.407607in}}%
\pgfpathlineto{\pgfqpoint{1.551455in}{1.404424in}}%
\pgfpathlineto{\pgfqpoint{1.559557in}{1.406367in}}%
\pgfpathlineto{\pgfqpoint{1.567659in}{1.406046in}}%
\pgfpathlineto{\pgfqpoint{1.575761in}{1.405499in}}%
\pgfpathlineto{\pgfqpoint{1.583864in}{1.405262in}}%
\pgfpathlineto{\pgfqpoint{1.591966in}{1.400278in}}%
\pgfpathlineto{\pgfqpoint{1.600068in}{1.398088in}}%
\pgfpathlineto{\pgfqpoint{1.608170in}{1.394425in}}%
\pgfpathlineto{\pgfqpoint{1.616273in}{1.397187in}}%
\pgfpathlineto{\pgfqpoint{1.624375in}{1.392563in}}%
\pgfpathlineto{\pgfqpoint{1.632477in}{1.390502in}}%
\pgfpathlineto{\pgfqpoint{1.640580in}{1.390659in}}%
\pgfpathlineto{\pgfqpoint{1.648682in}{1.389269in}}%
\pgfpathlineto{\pgfqpoint{1.656784in}{1.385752in}}%
\pgfpathlineto{\pgfqpoint{1.664886in}{1.382454in}}%
\pgfpathlineto{\pgfqpoint{1.672989in}{1.378433in}}%
\pgfpathlineto{\pgfqpoint{1.681091in}{1.379782in}}%
\pgfpathlineto{\pgfqpoint{1.689193in}{1.376327in}}%
\pgfpathlineto{\pgfqpoint{1.697295in}{1.372896in}}%
\pgfpathlineto{\pgfqpoint{1.705398in}{1.370782in}}%
\pgfpathlineto{\pgfqpoint{1.713500in}{1.369851in}}%
\pgfpathlineto{\pgfqpoint{1.721602in}{1.366224in}}%
\pgfpathlineto{\pgfqpoint{1.729705in}{1.368726in}}%
\pgfpathlineto{\pgfqpoint{1.737807in}{1.366093in}}%
\pgfpathlineto{\pgfqpoint{1.745909in}{1.363799in}}%
\pgfpathlineto{\pgfqpoint{1.754011in}{1.361905in}}%
\pgfpathlineto{\pgfqpoint{1.762114in}{1.361857in}}%
\pgfpathlineto{\pgfqpoint{1.770216in}{1.357693in}}%
\pgfpathlineto{\pgfqpoint{1.778318in}{1.353444in}}%
\pgfpathlineto{\pgfqpoint{1.786420in}{1.353740in}}%
\pgfpathlineto{\pgfqpoint{1.794523in}{1.351663in}}%
\pgfpathlineto{\pgfqpoint{1.802625in}{1.351973in}}%
\pgfpathlineto{\pgfqpoint{1.810727in}{1.352684in}}%
\pgfpathlineto{\pgfqpoint{1.818830in}{1.351871in}}%
\pgfpathlineto{\pgfqpoint{1.826932in}{1.347873in}}%
\pgfpathlineto{\pgfqpoint{1.835034in}{1.344372in}}%
\pgfpathlineto{\pgfqpoint{1.843136in}{1.344157in}}%
\pgfpathlineto{\pgfqpoint{1.851239in}{1.339937in}}%
\pgfpathlineto{\pgfqpoint{1.859341in}{1.340110in}}%
\pgfpathlineto{\pgfqpoint{1.867443in}{1.340963in}}%
\pgfpathlineto{\pgfqpoint{1.875545in}{1.337143in}}%
\pgfpathlineto{\pgfqpoint{1.883648in}{1.333617in}}%
\pgfpathlineto{\pgfqpoint{1.891750in}{1.330522in}}%
\pgfpathlineto{\pgfqpoint{1.899852in}{1.327697in}}%
\pgfpathlineto{\pgfqpoint{1.907955in}{1.323374in}}%
\pgfpathlineto{\pgfqpoint{1.916057in}{1.320248in}}%
\pgfpathlineto{\pgfqpoint{1.924159in}{1.321420in}}%
\pgfpathlineto{\pgfqpoint{1.932261in}{1.318734in}}%
\pgfpathlineto{\pgfqpoint{1.940364in}{1.318281in}}%
\pgfpathlineto{\pgfqpoint{1.948466in}{1.314256in}}%
\pgfpathlineto{\pgfqpoint{1.956568in}{1.310620in}}%
\pgfpathlineto{\pgfqpoint{1.964670in}{1.306651in}}%
\pgfpathlineto{\pgfqpoint{1.972773in}{1.306103in}}%
\pgfpathlineto{\pgfqpoint{1.980875in}{1.304400in}}%
\pgfpathlineto{\pgfqpoint{1.988977in}{1.302118in}}%
\pgfpathlineto{\pgfqpoint{1.988977in}{1.068158in}}%
\pgfpathlineto{\pgfqpoint{1.988977in}{1.068158in}}%
\pgfpathlineto{\pgfqpoint{1.980875in}{1.064367in}}%
\pgfpathlineto{\pgfqpoint{1.972773in}{1.068338in}}%
\pgfpathlineto{\pgfqpoint{1.964670in}{1.072275in}}%
\pgfpathlineto{\pgfqpoint{1.956568in}{1.076105in}}%
\pgfpathlineto{\pgfqpoint{1.948466in}{1.079878in}}%
\pgfpathlineto{\pgfqpoint{1.940364in}{1.083894in}}%
\pgfpathlineto{\pgfqpoint{1.932261in}{1.087857in}}%
\pgfpathlineto{\pgfqpoint{1.924159in}{1.089853in}}%
\pgfpathlineto{\pgfqpoint{1.916057in}{1.087559in}}%
\pgfpathlineto{\pgfqpoint{1.907955in}{1.091370in}}%
\pgfpathlineto{\pgfqpoint{1.899852in}{1.090858in}}%
\pgfpathlineto{\pgfqpoint{1.891750in}{1.094878in}}%
\pgfpathlineto{\pgfqpoint{1.883648in}{1.098644in}}%
\pgfpathlineto{\pgfqpoint{1.875545in}{1.100940in}}%
\pgfpathlineto{\pgfqpoint{1.867443in}{1.105149in}}%
\pgfpathlineto{\pgfqpoint{1.859341in}{1.107697in}}%
\pgfpathlineto{\pgfqpoint{1.851239in}{1.109290in}}%
\pgfpathlineto{\pgfqpoint{1.843136in}{1.112964in}}%
\pgfpathlineto{\pgfqpoint{1.835034in}{1.114592in}}%
\pgfpathlineto{\pgfqpoint{1.826932in}{1.118839in}}%
\pgfpathlineto{\pgfqpoint{1.818830in}{1.122779in}}%
\pgfpathlineto{\pgfqpoint{1.810727in}{1.126784in}}%
\pgfpathlineto{\pgfqpoint{1.802625in}{1.130925in}}%
\pgfpathlineto{\pgfqpoint{1.794523in}{1.130247in}}%
\pgfpathlineto{\pgfqpoint{1.786420in}{1.131623in}}%
\pgfpathlineto{\pgfqpoint{1.778318in}{1.135537in}}%
\pgfpathlineto{\pgfqpoint{1.770216in}{1.139504in}}%
\pgfpathlineto{\pgfqpoint{1.762114in}{1.143163in}}%
\pgfpathlineto{\pgfqpoint{1.754011in}{1.146770in}}%
\pgfpathlineto{\pgfqpoint{1.745909in}{1.143154in}}%
\pgfpathlineto{\pgfqpoint{1.737807in}{1.146672in}}%
\pgfpathlineto{\pgfqpoint{1.729705in}{1.147648in}}%
\pgfpathlineto{\pgfqpoint{1.721602in}{1.148973in}}%
\pgfpathlineto{\pgfqpoint{1.713500in}{1.152861in}}%
\pgfpathlineto{\pgfqpoint{1.705398in}{1.157089in}}%
\pgfpathlineto{\pgfqpoint{1.697295in}{1.158850in}}%
\pgfpathlineto{\pgfqpoint{1.689193in}{1.159863in}}%
\pgfpathlineto{\pgfqpoint{1.681091in}{1.163428in}}%
\pgfpathlineto{\pgfqpoint{1.672989in}{1.164911in}}%
\pgfpathlineto{\pgfqpoint{1.664886in}{1.168981in}}%
\pgfpathlineto{\pgfqpoint{1.656784in}{1.168669in}}%
\pgfpathlineto{\pgfqpoint{1.648682in}{1.172695in}}%
\pgfpathlineto{\pgfqpoint{1.640580in}{1.171578in}}%
\pgfpathlineto{\pgfqpoint{1.632477in}{1.172821in}}%
\pgfpathlineto{\pgfqpoint{1.624375in}{1.176847in}}%
\pgfpathlineto{\pgfqpoint{1.616273in}{1.181205in}}%
\pgfpathlineto{\pgfqpoint{1.608170in}{1.184120in}}%
\pgfpathlineto{\pgfqpoint{1.600068in}{1.181943in}}%
\pgfpathlineto{\pgfqpoint{1.591966in}{1.183240in}}%
\pgfpathlineto{\pgfqpoint{1.583864in}{1.184647in}}%
\pgfpathlineto{\pgfqpoint{1.575761in}{1.189203in}}%
\pgfpathlineto{\pgfqpoint{1.567659in}{1.189679in}}%
\pgfpathlineto{\pgfqpoint{1.559557in}{1.191672in}}%
\pgfpathlineto{\pgfqpoint{1.551455in}{1.196418in}}%
\pgfpathlineto{\pgfqpoint{1.543352in}{1.196865in}}%
\pgfpathlineto{\pgfqpoint{1.535250in}{1.201814in}}%
\pgfpathlineto{\pgfqpoint{1.527148in}{1.205983in}}%
\pgfpathlineto{\pgfqpoint{1.519045in}{1.209735in}}%
\pgfpathlineto{\pgfqpoint{1.510943in}{1.212520in}}%
\pgfpathlineto{\pgfqpoint{1.502841in}{1.216108in}}%
\pgfpathlineto{\pgfqpoint{1.494739in}{1.219932in}}%
\pgfpathlineto{\pgfqpoint{1.486636in}{1.220913in}}%
\pgfpathlineto{\pgfqpoint{1.478534in}{1.219457in}}%
\pgfpathlineto{\pgfqpoint{1.470432in}{1.222220in}}%
\pgfpathlineto{\pgfqpoint{1.462330in}{1.225249in}}%
\pgfpathlineto{\pgfqpoint{1.454227in}{1.229306in}}%
\pgfpathlineto{\pgfqpoint{1.446125in}{1.234085in}}%
\pgfpathlineto{\pgfqpoint{1.438023in}{1.236738in}}%
\pgfpathlineto{\pgfqpoint{1.429920in}{1.240964in}}%
\pgfpathlineto{\pgfqpoint{1.421818in}{1.244177in}}%
\pgfpathlineto{\pgfqpoint{1.413716in}{1.249738in}}%
\pgfpathlineto{\pgfqpoint{1.405614in}{1.250052in}}%
\pgfpathlineto{\pgfqpoint{1.397511in}{1.249329in}}%
\pgfpathlineto{\pgfqpoint{1.389409in}{1.254492in}}%
\pgfpathlineto{\pgfqpoint{1.381307in}{1.258463in}}%
\pgfpathlineto{\pgfqpoint{1.373205in}{1.258884in}}%
\pgfpathlineto{\pgfqpoint{1.365102in}{1.263029in}}%
\pgfpathlineto{\pgfqpoint{1.357000in}{1.264142in}}%
\pgfpathlineto{\pgfqpoint{1.348898in}{1.269831in}}%
\pgfpathlineto{\pgfqpoint{1.340795in}{1.271303in}}%
\pgfpathlineto{\pgfqpoint{1.332693in}{1.274771in}}%
\pgfpathlineto{\pgfqpoint{1.324591in}{1.273197in}}%
\pgfpathlineto{\pgfqpoint{1.316489in}{1.279160in}}%
\pgfpathlineto{\pgfqpoint{1.308386in}{1.275522in}}%
\pgfpathlineto{\pgfqpoint{1.300284in}{1.280459in}}%
\pgfpathlineto{\pgfqpoint{1.292182in}{1.285874in}}%
\pgfpathlineto{\pgfqpoint{1.284080in}{1.288129in}}%
\pgfpathlineto{\pgfqpoint{1.275977in}{1.285183in}}%
\pgfpathlineto{\pgfqpoint{1.267875in}{1.286046in}}%
\pgfpathlineto{\pgfqpoint{1.259773in}{1.279585in}}%
\pgfpathlineto{\pgfqpoint{1.251670in}{1.281234in}}%
\pgfpathlineto{\pgfqpoint{1.243568in}{1.284178in}}%
\pgfpathlineto{\pgfqpoint{1.235466in}{1.288116in}}%
\pgfpathlineto{\pgfqpoint{1.227364in}{1.287318in}}%
\pgfpathlineto{\pgfqpoint{1.219261in}{1.285832in}}%
\pgfpathlineto{\pgfqpoint{1.211159in}{1.285002in}}%
\pgfpathlineto{\pgfqpoint{1.203057in}{1.279968in}}%
\pgfpathlineto{\pgfqpoint{1.194955in}{1.277040in}}%
\pgfpathlineto{\pgfqpoint{1.186852in}{1.265981in}}%
\pgfpathlineto{\pgfqpoint{1.178750in}{1.268414in}}%
\pgfpathlineto{\pgfqpoint{1.170648in}{1.265873in}}%
\pgfpathlineto{\pgfqpoint{1.162545in}{1.258339in}}%
\pgfpathlineto{\pgfqpoint{1.154443in}{1.258826in}}%
\pgfpathlineto{\pgfqpoint{1.146341in}{1.261646in}}%
\pgfpathlineto{\pgfqpoint{1.138239in}{1.259008in}}%
\pgfpathlineto{\pgfqpoint{1.130136in}{1.249704in}}%
\pgfpathlineto{\pgfqpoint{1.122034in}{1.252943in}}%
\pgfpathlineto{\pgfqpoint{1.113932in}{1.257226in}}%
\pgfpathlineto{\pgfqpoint{1.105830in}{1.259248in}}%
\pgfpathlineto{\pgfqpoint{1.097727in}{1.250401in}}%
\pgfpathlineto{\pgfqpoint{1.089625in}{1.256257in}}%
\pgfpathlineto{\pgfqpoint{1.081523in}{1.262464in}}%
\pgfpathlineto{\pgfqpoint{1.073420in}{1.254930in}}%
\pgfpathlineto{\pgfqpoint{1.065318in}{1.251365in}}%
\pgfpathlineto{\pgfqpoint{1.057216in}{1.254360in}}%
\pgfpathlineto{\pgfqpoint{1.049114in}{1.260056in}}%
\pgfpathlineto{\pgfqpoint{1.041011in}{1.263964in}}%
\pgfpathlineto{\pgfqpoint{1.032909in}{1.268064in}}%
\pgfpathlineto{\pgfqpoint{1.024807in}{1.258504in}}%
\pgfpathlineto{\pgfqpoint{1.016705in}{1.263966in}}%
\pgfpathlineto{\pgfqpoint{1.008602in}{1.269897in}}%
\pgfpathlineto{\pgfqpoint{1.000500in}{1.267364in}}%
\pgfpathlineto{\pgfqpoint{0.992398in}{1.259757in}}%
\pgfpathlineto{\pgfqpoint{0.984295in}{1.260447in}}%
\pgfpathlineto{\pgfqpoint{0.976193in}{1.256459in}}%
\pgfpathlineto{\pgfqpoint{0.968091in}{1.261053in}}%
\pgfpathlineto{\pgfqpoint{0.959989in}{1.263057in}}%
\pgfpathlineto{\pgfqpoint{0.951886in}{1.257807in}}%
\pgfpathlineto{\pgfqpoint{0.943784in}{1.255767in}}%
\pgfpathlineto{\pgfqpoint{0.935682in}{1.256863in}}%
\pgfpathlineto{\pgfqpoint{0.927580in}{1.252402in}}%
\pgfpathlineto{\pgfqpoint{0.919477in}{1.254463in}}%
\pgfpathlineto{\pgfqpoint{0.911375in}{1.248079in}}%
\pgfpathlineto{\pgfqpoint{0.903273in}{1.240763in}}%
\pgfpathlineto{\pgfqpoint{0.895170in}{1.245334in}}%
\pgfpathlineto{\pgfqpoint{0.887068in}{1.251246in}}%
\pgfpathlineto{\pgfqpoint{0.878966in}{1.251896in}}%
\pgfpathlineto{\pgfqpoint{0.870864in}{1.247585in}}%
\pgfpathlineto{\pgfqpoint{0.862761in}{1.252901in}}%
\pgfpathlineto{\pgfqpoint{0.854659in}{1.256797in}}%
\pgfpathlineto{\pgfqpoint{0.846557in}{1.259728in}}%
\pgfpathlineto{\pgfqpoint{0.838455in}{1.264897in}}%
\pgfpathlineto{\pgfqpoint{0.830352in}{1.264334in}}%
\pgfpathlineto{\pgfqpoint{0.822250in}{1.254373in}}%
\pgfpathlineto{\pgfqpoint{0.814148in}{1.257892in}}%
\pgfpathlineto{\pgfqpoint{0.806045in}{1.263349in}}%
\pgfpathlineto{\pgfqpoint{0.797943in}{1.252497in}}%
\pgfpathlineto{\pgfqpoint{0.789841in}{1.253798in}}%
\pgfpathlineto{\pgfqpoint{0.781739in}{1.245794in}}%
\pgfpathlineto{\pgfqpoint{0.773636in}{1.246135in}}%
\pgfpathlineto{\pgfqpoint{0.765534in}{1.246871in}}%
\pgfpathlineto{\pgfqpoint{0.757432in}{1.242120in}}%
\pgfpathlineto{\pgfqpoint{0.749330in}{1.239669in}}%
\pgfpathlineto{\pgfqpoint{0.741227in}{1.237446in}}%
\pgfpathlineto{\pgfqpoint{0.733125in}{1.246441in}}%
\pgfpathlineto{\pgfqpoint{0.725023in}{1.240687in}}%
\pgfpathlineto{\pgfqpoint{0.716920in}{1.242052in}}%
\pgfpathlineto{\pgfqpoint{0.708818in}{1.237714in}}%
\pgfpathlineto{\pgfqpoint{0.700716in}{1.239195in}}%
\pgfpathlineto{\pgfqpoint{0.692614in}{1.230342in}}%
\pgfpathlineto{\pgfqpoint{0.684511in}{1.219162in}}%
\pgfpathlineto{\pgfqpoint{0.676409in}{1.225437in}}%
\pgfpathlineto{\pgfqpoint{0.668307in}{1.228280in}}%
\pgfpathlineto{\pgfqpoint{0.660205in}{1.218727in}}%
\pgfpathlineto{\pgfqpoint{0.652102in}{1.233048in}}%
\pgfpathlineto{\pgfqpoint{0.644000in}{1.241492in}}%
\pgfpathlineto{\pgfqpoint{0.635898in}{1.251830in}}%
\pgfpathlineto{\pgfqpoint{0.627795in}{1.236065in}}%
\pgfpathlineto{\pgfqpoint{0.619693in}{1.225724in}}%
\pgfpathlineto{\pgfqpoint{0.611591in}{1.228689in}}%
\pgfpathlineto{\pgfqpoint{0.603489in}{1.233931in}}%
\pgfpathlineto{\pgfqpoint{0.595386in}{1.249105in}}%
\pgfpathlineto{\pgfqpoint{0.587284in}{1.218959in}}%
\pgfpathlineto{\pgfqpoint{0.579182in}{1.200530in}}%
\pgfpathlineto{\pgfqpoint{0.571080in}{1.209341in}}%
\pgfpathlineto{\pgfqpoint{0.562977in}{1.241958in}}%
\pgfpathlineto{\pgfqpoint{0.554875in}{1.263211in}}%
\pgfpathlineto{\pgfqpoint{0.546773in}{1.258574in}}%
\pgfpathlineto{\pgfqpoint{0.538670in}{1.253689in}}%
\pgfpathlineto{\pgfqpoint{0.530568in}{1.264050in}}%
\pgfpathlineto{\pgfqpoint{0.522466in}{1.246098in}}%
\pgfpathlineto{\pgfqpoint{0.514364in}{1.234816in}}%
\pgfpathlineto{\pgfqpoint{0.506261in}{1.265182in}}%
\pgfpathlineto{\pgfqpoint{0.498159in}{1.285314in}}%
\pgfpathlineto{\pgfqpoint{0.490057in}{1.301906in}}%
\pgfpathlineto{\pgfqpoint{0.481955in}{1.281116in}}%
\pgfpathlineto{\pgfqpoint{0.473852in}{1.258542in}}%
\pgfpathlineto{\pgfqpoint{0.465750in}{1.240773in}}%
\pgfpathlineto{\pgfqpoint{0.457648in}{1.292031in}}%
\pgfpathlineto{\pgfqpoint{0.449545in}{1.326662in}}%
\pgfpathlineto{\pgfqpoint{0.441443in}{1.350971in}}%
\pgfpathlineto{\pgfqpoint{0.433341in}{1.336990in}}%
\pgfpathlineto{\pgfqpoint{0.425239in}{1.331948in}}%
\pgfpathlineto{\pgfqpoint{0.417136in}{1.409327in}}%
\pgfpathlineto{\pgfqpoint{0.409034in}{1.490776in}}%
\pgfpathlineto{\pgfqpoint{0.400932in}{1.571253in}}%
\pgfpathlineto{\pgfqpoint{0.392830in}{1.663089in}}%
\pgfpathlineto{\pgfqpoint{0.384727in}{1.769744in}}%
\pgfpathlineto{\pgfqpoint{0.376625in}{1.918295in}}%
\pgfpathlineto{\pgfqpoint{0.368523in}{2.232213in}}%
\pgfpathclose%
\pgfusepath{fill}%
\end{pgfscope}%
\begin{pgfscope}%
\pgfpathrectangle{\pgfqpoint{0.287500in}{0.375000in}}{\pgfqpoint{1.782500in}{2.265000in}}%
\pgfusepath{clip}%
\pgfsetroundcap%
\pgfsetroundjoin%
\pgfsetlinewidth{1.505625pt}%
\definecolor{currentstroke}{rgb}{0.121569,0.466667,0.705882}%
\pgfsetstrokecolor{currentstroke}%
\pgfsetdash{}{0pt}%
\pgfpathmoveto{\pgfqpoint{0.368523in}{2.092095in}}%
\pgfpathlineto{\pgfqpoint{0.376625in}{1.893736in}}%
\pgfpathlineto{\pgfqpoint{0.384727in}{1.855536in}}%
\pgfpathlineto{\pgfqpoint{0.392830in}{1.865063in}}%
\pgfpathlineto{\pgfqpoint{0.400932in}{1.869452in}}%
\pgfpathlineto{\pgfqpoint{0.417136in}{1.813083in}}%
\pgfpathlineto{\pgfqpoint{0.425239in}{1.805118in}}%
\pgfpathlineto{\pgfqpoint{0.433341in}{1.786600in}}%
\pgfpathlineto{\pgfqpoint{0.449545in}{1.756849in}}%
\pgfpathlineto{\pgfqpoint{0.457648in}{1.760794in}}%
\pgfpathlineto{\pgfqpoint{0.465750in}{1.747858in}}%
\pgfpathlineto{\pgfqpoint{0.473852in}{1.742895in}}%
\pgfpathlineto{\pgfqpoint{0.481955in}{1.746419in}}%
\pgfpathlineto{\pgfqpoint{0.498159in}{1.715948in}}%
\pgfpathlineto{\pgfqpoint{0.506261in}{1.702505in}}%
\pgfpathlineto{\pgfqpoint{0.514364in}{1.709066in}}%
\pgfpathlineto{\pgfqpoint{0.522466in}{1.698147in}}%
\pgfpathlineto{\pgfqpoint{0.530568in}{1.684052in}}%
\pgfpathlineto{\pgfqpoint{0.538670in}{1.695998in}}%
\pgfpathlineto{\pgfqpoint{0.546773in}{1.678434in}}%
\pgfpathlineto{\pgfqpoint{0.554875in}{1.676081in}}%
\pgfpathlineto{\pgfqpoint{0.562977in}{1.651630in}}%
\pgfpathlineto{\pgfqpoint{0.571080in}{1.639080in}}%
\pgfpathlineto{\pgfqpoint{0.579182in}{1.634717in}}%
\pgfpathlineto{\pgfqpoint{0.587284in}{1.627752in}}%
\pgfpathlineto{\pgfqpoint{0.595386in}{1.627953in}}%
\pgfpathlineto{\pgfqpoint{0.603489in}{1.618727in}}%
\pgfpathlineto{\pgfqpoint{0.611591in}{1.617950in}}%
\pgfpathlineto{\pgfqpoint{0.627795in}{1.602439in}}%
\pgfpathlineto{\pgfqpoint{0.635898in}{1.588605in}}%
\pgfpathlineto{\pgfqpoint{0.644000in}{1.586467in}}%
\pgfpathlineto{\pgfqpoint{0.660205in}{1.574483in}}%
\pgfpathlineto{\pgfqpoint{0.668307in}{1.560487in}}%
\pgfpathlineto{\pgfqpoint{0.676409in}{1.552330in}}%
\pgfpathlineto{\pgfqpoint{0.684511in}{1.541874in}}%
\pgfpathlineto{\pgfqpoint{0.692614in}{1.528559in}}%
\pgfpathlineto{\pgfqpoint{0.708818in}{1.515391in}}%
\pgfpathlineto{\pgfqpoint{0.725023in}{1.491892in}}%
\pgfpathlineto{\pgfqpoint{0.733125in}{1.490100in}}%
\pgfpathlineto{\pgfqpoint{0.765534in}{1.455539in}}%
\pgfpathlineto{\pgfqpoint{0.773636in}{1.440405in}}%
\pgfpathlineto{\pgfqpoint{0.797943in}{1.403588in}}%
\pgfpathlineto{\pgfqpoint{0.806045in}{1.389160in}}%
\pgfpathlineto{\pgfqpoint{0.822250in}{1.373569in}}%
\pgfpathlineto{\pgfqpoint{0.830352in}{1.363430in}}%
\pgfpathlineto{\pgfqpoint{0.838455in}{1.360571in}}%
\pgfpathlineto{\pgfqpoint{0.846557in}{1.347350in}}%
\pgfpathlineto{\pgfqpoint{0.854659in}{1.341948in}}%
\pgfpathlineto{\pgfqpoint{0.870864in}{1.316540in}}%
\pgfpathlineto{\pgfqpoint{0.878966in}{1.307223in}}%
\pgfpathlineto{\pgfqpoint{0.887068in}{1.301276in}}%
\pgfpathlineto{\pgfqpoint{0.911375in}{1.267156in}}%
\pgfpathlineto{\pgfqpoint{0.927580in}{1.258761in}}%
\pgfpathlineto{\pgfqpoint{0.968091in}{1.204610in}}%
\pgfpathlineto{\pgfqpoint{0.984295in}{1.190334in}}%
\pgfpathlineto{\pgfqpoint{0.992398in}{1.180469in}}%
\pgfpathlineto{\pgfqpoint{1.000500in}{1.175373in}}%
\pgfpathlineto{\pgfqpoint{1.041011in}{1.127177in}}%
\pgfpathlineto{\pgfqpoint{1.049114in}{1.119699in}}%
\pgfpathlineto{\pgfqpoint{1.065318in}{1.109033in}}%
\pgfpathlineto{\pgfqpoint{1.089625in}{1.083672in}}%
\pgfpathlineto{\pgfqpoint{1.097727in}{1.080275in}}%
\pgfpathlineto{\pgfqpoint{1.138239in}{1.038111in}}%
\pgfpathlineto{\pgfqpoint{1.227364in}{0.962643in}}%
\pgfpathlineto{\pgfqpoint{1.243568in}{0.948046in}}%
\pgfpathlineto{\pgfqpoint{1.251670in}{0.945231in}}%
\pgfpathlineto{\pgfqpoint{1.275977in}{0.924165in}}%
\pgfpathlineto{\pgfqpoint{1.284080in}{0.921314in}}%
\pgfpathlineto{\pgfqpoint{1.324591in}{0.889243in}}%
\pgfpathlineto{\pgfqpoint{1.389409in}{0.838528in}}%
\pgfpathlineto{\pgfqpoint{1.397511in}{0.838277in}}%
\pgfpathlineto{\pgfqpoint{1.405614in}{0.832637in}}%
\pgfpathlineto{\pgfqpoint{1.413716in}{0.831658in}}%
\pgfpathlineto{\pgfqpoint{1.438023in}{0.813959in}}%
\pgfpathlineto{\pgfqpoint{1.454227in}{0.807987in}}%
\pgfpathlineto{\pgfqpoint{1.486636in}{0.785637in}}%
\pgfpathlineto{\pgfqpoint{1.551455in}{0.742307in}}%
\pgfpathlineto{\pgfqpoint{1.600068in}{0.711559in}}%
\pgfpathlineto{\pgfqpoint{1.608170in}{0.708445in}}%
\pgfpathlineto{\pgfqpoint{1.624375in}{0.699126in}}%
\pgfpathlineto{\pgfqpoint{1.632477in}{0.696830in}}%
\pgfpathlineto{\pgfqpoint{1.664886in}{0.678176in}}%
\pgfpathlineto{\pgfqpoint{1.689193in}{0.668039in}}%
\pgfpathlineto{\pgfqpoint{1.713500in}{0.653848in}}%
\pgfpathlineto{\pgfqpoint{1.721602in}{0.652031in}}%
\pgfpathlineto{\pgfqpoint{1.737807in}{0.642751in}}%
\pgfpathlineto{\pgfqpoint{1.762114in}{0.637492in}}%
\pgfpathlineto{\pgfqpoint{1.794523in}{0.620367in}}%
\pgfpathlineto{\pgfqpoint{1.802625in}{0.620281in}}%
\pgfpathlineto{\pgfqpoint{1.818830in}{0.612459in}}%
\pgfpathlineto{\pgfqpoint{1.875545in}{0.582562in}}%
\pgfpathlineto{\pgfqpoint{1.891750in}{0.575629in}}%
\pgfpathlineto{\pgfqpoint{1.924159in}{0.560543in}}%
\pgfpathlineto{\pgfqpoint{1.932261in}{0.559422in}}%
\pgfpathlineto{\pgfqpoint{1.940364in}{0.555414in}}%
\pgfpathlineto{\pgfqpoint{1.956568in}{0.554380in}}%
\pgfpathlineto{\pgfqpoint{1.988977in}{0.538668in}}%
\pgfpathlineto{\pgfqpoint{1.988977in}{0.538668in}}%
\pgfusepath{stroke}%
\end{pgfscope}%
\begin{pgfscope}%
\pgfpathrectangle{\pgfqpoint{0.287500in}{0.375000in}}{\pgfqpoint{1.782500in}{2.265000in}}%
\pgfusepath{clip}%
\pgfsetroundcap%
\pgfsetroundjoin%
\pgfsetlinewidth{1.505625pt}%
\definecolor{currentstroke}{rgb}{1.000000,0.498039,0.054902}%
\pgfsetstrokecolor{currentstroke}%
\pgfsetdash{}{0pt}%
\pgfpathmoveto{\pgfqpoint{0.368523in}{2.117925in}}%
\pgfpathlineto{\pgfqpoint{0.376625in}{1.960960in}}%
\pgfpathlineto{\pgfqpoint{0.384727in}{1.882589in}}%
\pgfpathlineto{\pgfqpoint{0.392830in}{1.862680in}}%
\pgfpathlineto{\pgfqpoint{0.409034in}{1.794808in}}%
\pgfpathlineto{\pgfqpoint{0.417136in}{1.799707in}}%
\pgfpathlineto{\pgfqpoint{0.425239in}{1.777587in}}%
\pgfpathlineto{\pgfqpoint{0.433341in}{1.749415in}}%
\pgfpathlineto{\pgfqpoint{0.441443in}{1.739939in}}%
\pgfpathlineto{\pgfqpoint{0.465750in}{1.683538in}}%
\pgfpathlineto{\pgfqpoint{0.473852in}{1.690743in}}%
\pgfpathlineto{\pgfqpoint{0.481955in}{1.690330in}}%
\pgfpathlineto{\pgfqpoint{0.498159in}{1.675549in}}%
\pgfpathlineto{\pgfqpoint{0.506261in}{1.666168in}}%
\pgfpathlineto{\pgfqpoint{0.514364in}{1.668674in}}%
\pgfpathlineto{\pgfqpoint{0.530568in}{1.641698in}}%
\pgfpathlineto{\pgfqpoint{0.538670in}{1.634109in}}%
\pgfpathlineto{\pgfqpoint{0.546773in}{1.630178in}}%
\pgfpathlineto{\pgfqpoint{0.562977in}{1.609788in}}%
\pgfpathlineto{\pgfqpoint{0.571080in}{1.603599in}}%
\pgfpathlineto{\pgfqpoint{0.587284in}{1.601387in}}%
\pgfpathlineto{\pgfqpoint{0.595386in}{1.598327in}}%
\pgfpathlineto{\pgfqpoint{0.603489in}{1.592080in}}%
\pgfpathlineto{\pgfqpoint{0.611591in}{1.595787in}}%
\pgfpathlineto{\pgfqpoint{0.619693in}{1.592626in}}%
\pgfpathlineto{\pgfqpoint{0.627795in}{1.596296in}}%
\pgfpathlineto{\pgfqpoint{0.652102in}{1.584551in}}%
\pgfpathlineto{\pgfqpoint{0.660205in}{1.591922in}}%
\pgfpathlineto{\pgfqpoint{0.676409in}{1.593079in}}%
\pgfpathlineto{\pgfqpoint{0.684511in}{1.580792in}}%
\pgfpathlineto{\pgfqpoint{0.692614in}{1.582967in}}%
\pgfpathlineto{\pgfqpoint{0.700716in}{1.580555in}}%
\pgfpathlineto{\pgfqpoint{0.733125in}{1.554206in}}%
\pgfpathlineto{\pgfqpoint{0.749330in}{1.551350in}}%
\pgfpathlineto{\pgfqpoint{0.757432in}{1.540094in}}%
\pgfpathlineto{\pgfqpoint{0.773636in}{1.522856in}}%
\pgfpathlineto{\pgfqpoint{0.797943in}{1.487083in}}%
\pgfpathlineto{\pgfqpoint{0.806045in}{1.482048in}}%
\pgfpathlineto{\pgfqpoint{0.830352in}{1.460048in}}%
\pgfpathlineto{\pgfqpoint{0.838455in}{1.456430in}}%
\pgfpathlineto{\pgfqpoint{0.846557in}{1.450262in}}%
\pgfpathlineto{\pgfqpoint{0.854659in}{1.439564in}}%
\pgfpathlineto{\pgfqpoint{0.862761in}{1.435322in}}%
\pgfpathlineto{\pgfqpoint{0.870864in}{1.436145in}}%
\pgfpathlineto{\pgfqpoint{0.887068in}{1.417875in}}%
\pgfpathlineto{\pgfqpoint{0.895170in}{1.408535in}}%
\pgfpathlineto{\pgfqpoint{0.903273in}{1.405505in}}%
\pgfpathlineto{\pgfqpoint{0.919477in}{1.386850in}}%
\pgfpathlineto{\pgfqpoint{0.927580in}{1.380039in}}%
\pgfpathlineto{\pgfqpoint{0.935682in}{1.376244in}}%
\pgfpathlineto{\pgfqpoint{0.984295in}{1.320395in}}%
\pgfpathlineto{\pgfqpoint{0.992398in}{1.317254in}}%
\pgfpathlineto{\pgfqpoint{1.008602in}{1.299923in}}%
\pgfpathlineto{\pgfqpoint{1.016705in}{1.293006in}}%
\pgfpathlineto{\pgfqpoint{1.024807in}{1.283985in}}%
\pgfpathlineto{\pgfqpoint{1.032909in}{1.277849in}}%
\pgfpathlineto{\pgfqpoint{1.041011in}{1.269491in}}%
\pgfpathlineto{\pgfqpoint{1.049114in}{1.266135in}}%
\pgfpathlineto{\pgfqpoint{1.057216in}{1.260178in}}%
\pgfpathlineto{\pgfqpoint{1.065318in}{1.260376in}}%
\pgfpathlineto{\pgfqpoint{1.081523in}{1.245199in}}%
\pgfpathlineto{\pgfqpoint{1.089625in}{1.242061in}}%
\pgfpathlineto{\pgfqpoint{1.105830in}{1.226871in}}%
\pgfpathlineto{\pgfqpoint{1.113932in}{1.226077in}}%
\pgfpathlineto{\pgfqpoint{1.122034in}{1.220856in}}%
\pgfpathlineto{\pgfqpoint{1.138239in}{1.206046in}}%
\pgfpathlineto{\pgfqpoint{1.146341in}{1.203246in}}%
\pgfpathlineto{\pgfqpoint{1.170648in}{1.184241in}}%
\pgfpathlineto{\pgfqpoint{1.211159in}{1.160023in}}%
\pgfpathlineto{\pgfqpoint{1.219261in}{1.153879in}}%
\pgfpathlineto{\pgfqpoint{1.235466in}{1.150664in}}%
\pgfpathlineto{\pgfqpoint{1.243568in}{1.143490in}}%
\pgfpathlineto{\pgfqpoint{1.251670in}{1.140666in}}%
\pgfpathlineto{\pgfqpoint{1.267875in}{1.129269in}}%
\pgfpathlineto{\pgfqpoint{1.332693in}{1.078149in}}%
\pgfpathlineto{\pgfqpoint{1.348898in}{1.068042in}}%
\pgfpathlineto{\pgfqpoint{1.357000in}{1.064595in}}%
\pgfpathlineto{\pgfqpoint{1.389409in}{1.041687in}}%
\pgfpathlineto{\pgfqpoint{1.405614in}{1.031947in}}%
\pgfpathlineto{\pgfqpoint{1.413716in}{1.032258in}}%
\pgfpathlineto{\pgfqpoint{1.438023in}{1.015328in}}%
\pgfpathlineto{\pgfqpoint{1.510943in}{0.970005in}}%
\pgfpathlineto{\pgfqpoint{1.519045in}{0.966480in}}%
\pgfpathlineto{\pgfqpoint{1.543352in}{0.951018in}}%
\pgfpathlineto{\pgfqpoint{1.551455in}{0.949043in}}%
\pgfpathlineto{\pgfqpoint{1.567659in}{0.939323in}}%
\pgfpathlineto{\pgfqpoint{1.583864in}{0.931973in}}%
\pgfpathlineto{\pgfqpoint{1.600068in}{0.922363in}}%
\pgfpathlineto{\pgfqpoint{1.616273in}{0.914512in}}%
\pgfpathlineto{\pgfqpoint{1.648682in}{0.895455in}}%
\pgfpathlineto{\pgfqpoint{1.656784in}{0.894988in}}%
\pgfpathlineto{\pgfqpoint{1.664886in}{0.892050in}}%
\pgfpathlineto{\pgfqpoint{1.672989in}{0.891824in}}%
\pgfpathlineto{\pgfqpoint{1.689193in}{0.882972in}}%
\pgfpathlineto{\pgfqpoint{1.697295in}{0.882599in}}%
\pgfpathlineto{\pgfqpoint{1.713500in}{0.874396in}}%
\pgfpathlineto{\pgfqpoint{1.754011in}{0.855072in}}%
\pgfpathlineto{\pgfqpoint{1.770216in}{0.850259in}}%
\pgfpathlineto{\pgfqpoint{1.786420in}{0.842673in}}%
\pgfpathlineto{\pgfqpoint{1.810727in}{0.830932in}}%
\pgfpathlineto{\pgfqpoint{1.818830in}{0.831998in}}%
\pgfpathlineto{\pgfqpoint{1.835034in}{0.824598in}}%
\pgfpathlineto{\pgfqpoint{1.843136in}{0.824321in}}%
\pgfpathlineto{\pgfqpoint{1.867443in}{0.813935in}}%
\pgfpathlineto{\pgfqpoint{1.883648in}{0.807116in}}%
\pgfpathlineto{\pgfqpoint{1.932261in}{0.783831in}}%
\pgfpathlineto{\pgfqpoint{1.956568in}{0.773387in}}%
\pgfpathlineto{\pgfqpoint{1.980875in}{0.765586in}}%
\pgfpathlineto{\pgfqpoint{1.988977in}{0.761761in}}%
\pgfpathlineto{\pgfqpoint{1.988977in}{0.761761in}}%
\pgfusepath{stroke}%
\end{pgfscope}%
\begin{pgfscope}%
\pgfpathrectangle{\pgfqpoint{0.287500in}{0.375000in}}{\pgfqpoint{1.782500in}{2.265000in}}%
\pgfusepath{clip}%
\pgfsetroundcap%
\pgfsetroundjoin%
\pgfsetlinewidth{1.505625pt}%
\definecolor{currentstroke}{rgb}{0.172549,0.627451,0.172549}%
\pgfsetstrokecolor{currentstroke}%
\pgfsetdash{}{0pt}%
\pgfpathmoveto{\pgfqpoint{0.368523in}{2.395359in}}%
\pgfpathlineto{\pgfqpoint{0.376625in}{2.056283in}}%
\pgfpathlineto{\pgfqpoint{0.384727in}{1.913731in}}%
\pgfpathlineto{\pgfqpoint{0.392830in}{1.818152in}}%
\pgfpathlineto{\pgfqpoint{0.417136in}{1.625032in}}%
\pgfpathlineto{\pgfqpoint{0.425239in}{1.565662in}}%
\pgfpathlineto{\pgfqpoint{0.433341in}{1.554770in}}%
\pgfpathlineto{\pgfqpoint{0.441443in}{1.548948in}}%
\pgfpathlineto{\pgfqpoint{0.449545in}{1.526584in}}%
\pgfpathlineto{\pgfqpoint{0.465750in}{1.462001in}}%
\pgfpathlineto{\pgfqpoint{0.473852in}{1.464570in}}%
\pgfpathlineto{\pgfqpoint{0.490057in}{1.477813in}}%
\pgfpathlineto{\pgfqpoint{0.506261in}{1.447360in}}%
\pgfpathlineto{\pgfqpoint{0.514364in}{1.427284in}}%
\pgfpathlineto{\pgfqpoint{0.522466in}{1.430138in}}%
\pgfpathlineto{\pgfqpoint{0.530568in}{1.436668in}}%
\pgfpathlineto{\pgfqpoint{0.538670in}{1.428801in}}%
\pgfpathlineto{\pgfqpoint{0.546773in}{1.430332in}}%
\pgfpathlineto{\pgfqpoint{0.554875in}{1.429283in}}%
\pgfpathlineto{\pgfqpoint{0.562977in}{1.415193in}}%
\pgfpathlineto{\pgfqpoint{0.571080in}{1.395474in}}%
\pgfpathlineto{\pgfqpoint{0.579182in}{1.389409in}}%
\pgfpathlineto{\pgfqpoint{0.595386in}{1.404541in}}%
\pgfpathlineto{\pgfqpoint{0.603489in}{1.387648in}}%
\pgfpathlineto{\pgfqpoint{0.611591in}{1.376555in}}%
\pgfpathlineto{\pgfqpoint{0.619693in}{1.374055in}}%
\pgfpathlineto{\pgfqpoint{0.635898in}{1.381657in}}%
\pgfpathlineto{\pgfqpoint{0.644000in}{1.375296in}}%
\pgfpathlineto{\pgfqpoint{0.660205in}{1.372335in}}%
\pgfpathlineto{\pgfqpoint{0.668307in}{1.372293in}}%
\pgfpathlineto{\pgfqpoint{0.684511in}{1.355632in}}%
\pgfpathlineto{\pgfqpoint{0.692614in}{1.360962in}}%
\pgfpathlineto{\pgfqpoint{0.700716in}{1.360175in}}%
\pgfpathlineto{\pgfqpoint{0.708818in}{1.352907in}}%
\pgfpathlineto{\pgfqpoint{0.725023in}{1.345739in}}%
\pgfpathlineto{\pgfqpoint{0.733125in}{1.345145in}}%
\pgfpathlineto{\pgfqpoint{0.741227in}{1.336207in}}%
\pgfpathlineto{\pgfqpoint{0.749330in}{1.334021in}}%
\pgfpathlineto{\pgfqpoint{0.765534in}{1.335717in}}%
\pgfpathlineto{\pgfqpoint{0.773636in}{1.333059in}}%
\pgfpathlineto{\pgfqpoint{0.789841in}{1.340364in}}%
\pgfpathlineto{\pgfqpoint{0.797943in}{1.338214in}}%
\pgfpathlineto{\pgfqpoint{0.806045in}{1.345702in}}%
\pgfpathlineto{\pgfqpoint{0.814148in}{1.340714in}}%
\pgfpathlineto{\pgfqpoint{0.838455in}{1.345946in}}%
\pgfpathlineto{\pgfqpoint{0.854659in}{1.341607in}}%
\pgfpathlineto{\pgfqpoint{0.862761in}{1.342383in}}%
\pgfpathlineto{\pgfqpoint{0.870864in}{1.339512in}}%
\pgfpathlineto{\pgfqpoint{0.887068in}{1.339840in}}%
\pgfpathlineto{\pgfqpoint{0.895170in}{1.337562in}}%
\pgfpathlineto{\pgfqpoint{0.911375in}{1.339946in}}%
\pgfpathlineto{\pgfqpoint{0.919477in}{1.345540in}}%
\pgfpathlineto{\pgfqpoint{0.935682in}{1.350775in}}%
\pgfpathlineto{\pgfqpoint{0.951886in}{1.349162in}}%
\pgfpathlineto{\pgfqpoint{0.959989in}{1.354361in}}%
\pgfpathlineto{\pgfqpoint{0.976193in}{1.349320in}}%
\pgfpathlineto{\pgfqpoint{0.984295in}{1.350060in}}%
\pgfpathlineto{\pgfqpoint{0.992398in}{1.348260in}}%
\pgfpathlineto{\pgfqpoint{1.008602in}{1.354559in}}%
\pgfpathlineto{\pgfqpoint{1.016705in}{1.350862in}}%
\pgfpathlineto{\pgfqpoint{1.024807in}{1.348995in}}%
\pgfpathlineto{\pgfqpoint{1.032909in}{1.350660in}}%
\pgfpathlineto{\pgfqpoint{1.041011in}{1.349633in}}%
\pgfpathlineto{\pgfqpoint{1.057216in}{1.343794in}}%
\pgfpathlineto{\pgfqpoint{1.073420in}{1.344866in}}%
\pgfpathlineto{\pgfqpoint{1.081523in}{1.347237in}}%
\pgfpathlineto{\pgfqpoint{1.089625in}{1.341816in}}%
\pgfpathlineto{\pgfqpoint{1.097727in}{1.339071in}}%
\pgfpathlineto{\pgfqpoint{1.105830in}{1.342405in}}%
\pgfpathlineto{\pgfqpoint{1.113932in}{1.341209in}}%
\pgfpathlineto{\pgfqpoint{1.130136in}{1.335400in}}%
\pgfpathlineto{\pgfqpoint{1.138239in}{1.336730in}}%
\pgfpathlineto{\pgfqpoint{1.146341in}{1.335811in}}%
\pgfpathlineto{\pgfqpoint{1.154443in}{1.332794in}}%
\pgfpathlineto{\pgfqpoint{1.162545in}{1.332147in}}%
\pgfpathlineto{\pgfqpoint{1.178750in}{1.337104in}}%
\pgfpathlineto{\pgfqpoint{1.186852in}{1.339749in}}%
\pgfpathlineto{\pgfqpoint{1.194955in}{1.345927in}}%
\pgfpathlineto{\pgfqpoint{1.211159in}{1.351294in}}%
\pgfpathlineto{\pgfqpoint{1.227364in}{1.352671in}}%
\pgfpathlineto{\pgfqpoint{1.235466in}{1.353403in}}%
\pgfpathlineto{\pgfqpoint{1.251670in}{1.348124in}}%
\pgfpathlineto{\pgfqpoint{1.259773in}{1.349840in}}%
\pgfpathlineto{\pgfqpoint{1.275977in}{1.356769in}}%
\pgfpathlineto{\pgfqpoint{1.284080in}{1.358318in}}%
\pgfpathlineto{\pgfqpoint{1.308386in}{1.349564in}}%
\pgfpathlineto{\pgfqpoint{1.316489in}{1.349822in}}%
\pgfpathlineto{\pgfqpoint{1.324591in}{1.347211in}}%
\pgfpathlineto{\pgfqpoint{1.332693in}{1.348300in}}%
\pgfpathlineto{\pgfqpoint{1.365102in}{1.338026in}}%
\pgfpathlineto{\pgfqpoint{1.381307in}{1.336840in}}%
\pgfpathlineto{\pgfqpoint{1.397511in}{1.329940in}}%
\pgfpathlineto{\pgfqpoint{1.405614in}{1.331775in}}%
\pgfpathlineto{\pgfqpoint{1.421818in}{1.331849in}}%
\pgfpathlineto{\pgfqpoint{1.438023in}{1.327152in}}%
\pgfpathlineto{\pgfqpoint{1.454227in}{1.323322in}}%
\pgfpathlineto{\pgfqpoint{1.478534in}{1.318330in}}%
\pgfpathlineto{\pgfqpoint{1.494739in}{1.320099in}}%
\pgfpathlineto{\pgfqpoint{1.527148in}{1.313292in}}%
\pgfpathlineto{\pgfqpoint{1.543352in}{1.306546in}}%
\pgfpathlineto{\pgfqpoint{1.559557in}{1.303544in}}%
\pgfpathlineto{\pgfqpoint{1.583864in}{1.299808in}}%
\pgfpathlineto{\pgfqpoint{1.600068in}{1.294620in}}%
\pgfpathlineto{\pgfqpoint{1.616273in}{1.293791in}}%
\pgfpathlineto{\pgfqpoint{1.632477in}{1.286350in}}%
\pgfpathlineto{\pgfqpoint{1.648682in}{1.285610in}}%
\pgfpathlineto{\pgfqpoint{1.656784in}{1.281866in}}%
\pgfpathlineto{\pgfqpoint{1.664886in}{1.280175in}}%
\pgfpathlineto{\pgfqpoint{1.672989in}{1.276132in}}%
\pgfpathlineto{\pgfqpoint{1.681091in}{1.276220in}}%
\pgfpathlineto{\pgfqpoint{1.697295in}{1.270361in}}%
\pgfpathlineto{\pgfqpoint{1.713500in}{1.266006in}}%
\pgfpathlineto{\pgfqpoint{1.721602in}{1.262264in}}%
\pgfpathlineto{\pgfqpoint{1.729705in}{1.263067in}}%
\pgfpathlineto{\pgfqpoint{1.754011in}{1.258886in}}%
\pgfpathlineto{\pgfqpoint{1.762114in}{1.257255in}}%
\pgfpathlineto{\pgfqpoint{1.778318in}{1.249192in}}%
\pgfpathlineto{\pgfqpoint{1.794523in}{1.245855in}}%
\pgfpathlineto{\pgfqpoint{1.802625in}{1.246328in}}%
\pgfpathlineto{\pgfqpoint{1.818830in}{1.242670in}}%
\pgfpathlineto{\pgfqpoint{1.835034in}{1.234869in}}%
\pgfpathlineto{\pgfqpoint{1.843136in}{1.234032in}}%
\pgfpathlineto{\pgfqpoint{1.851239in}{1.230052in}}%
\pgfpathlineto{\pgfqpoint{1.867443in}{1.228808in}}%
\pgfpathlineto{\pgfqpoint{1.891750in}{1.218441in}}%
\pgfpathlineto{\pgfqpoint{1.916057in}{1.209465in}}%
\pgfpathlineto{\pgfqpoint{1.924159in}{1.211130in}}%
\pgfpathlineto{\pgfqpoint{1.948466in}{1.202731in}}%
\pgfpathlineto{\pgfqpoint{1.972773in}{1.193093in}}%
\pgfpathlineto{\pgfqpoint{1.980875in}{1.190397in}}%
\pgfpathlineto{\pgfqpoint{1.988977in}{1.190776in}}%
\pgfpathlineto{\pgfqpoint{1.988977in}{1.190776in}}%
\pgfusepath{stroke}%
\end{pgfscope}%
\begin{pgfscope}%
\pgfsetrectcap%
\pgfsetmiterjoin%
\pgfsetlinewidth{0.000000pt}%
\definecolor{currentstroke}{rgb}{1.000000,1.000000,1.000000}%
\pgfsetstrokecolor{currentstroke}%
\pgfsetdash{}{0pt}%
\pgfpathmoveto{\pgfqpoint{0.287500in}{0.375000in}}%
\pgfpathlineto{\pgfqpoint{0.287500in}{2.640000in}}%
\pgfusepath{}%
\end{pgfscope}%
\begin{pgfscope}%
\pgfsetrectcap%
\pgfsetmiterjoin%
\pgfsetlinewidth{0.000000pt}%
\definecolor{currentstroke}{rgb}{1.000000,1.000000,1.000000}%
\pgfsetstrokecolor{currentstroke}%
\pgfsetdash{}{0pt}%
\pgfpathmoveto{\pgfqpoint{2.070000in}{0.375000in}}%
\pgfpathlineto{\pgfqpoint{2.070000in}{2.640000in}}%
\pgfusepath{}%
\end{pgfscope}%
\begin{pgfscope}%
\pgfsetrectcap%
\pgfsetmiterjoin%
\pgfsetlinewidth{0.000000pt}%
\definecolor{currentstroke}{rgb}{1.000000,1.000000,1.000000}%
\pgfsetstrokecolor{currentstroke}%
\pgfsetdash{}{0pt}%
\pgfpathmoveto{\pgfqpoint{0.287500in}{0.375000in}}%
\pgfpathlineto{\pgfqpoint{2.070000in}{0.375000in}}%
\pgfusepath{}%
\end{pgfscope}%
\begin{pgfscope}%
\pgfsetrectcap%
\pgfsetmiterjoin%
\pgfsetlinewidth{0.000000pt}%
\definecolor{currentstroke}{rgb}{1.000000,1.000000,1.000000}%
\pgfsetstrokecolor{currentstroke}%
\pgfsetdash{}{0pt}%
\pgfpathmoveto{\pgfqpoint{0.287500in}{2.640000in}}%
\pgfpathlineto{\pgfqpoint{2.070000in}{2.640000in}}%
\pgfusepath{}%
\end{pgfscope}%
\begin{pgfscope}%
\definecolor{textcolor}{rgb}{0.150000,0.150000,0.150000}%
\pgfsetstrokecolor{textcolor}%
\pgfsetfillcolor{textcolor}%
\pgftext[x=1.178750in,y=2.723333in,,base]{\color{textcolor}\rmfamily\fontsize{8.000000}{9.600000}\selectfont Embedded SinOne in 1D}%
\end{pgfscope}%
\begin{pgfscope}%
\pgfsetroundcap%
\pgfsetroundjoin%
\pgfsetlinewidth{1.505625pt}%
\definecolor{currentstroke}{rgb}{0.121569,0.466667,0.705882}%
\pgfsetstrokecolor{currentstroke}%
\pgfsetdash{}{0pt}%
\pgfpathmoveto{\pgfqpoint{0.772205in}{2.530853in}}%
\pgfpathlineto{\pgfqpoint{0.938872in}{2.530853in}}%
\pgfusepath{stroke}%
\end{pgfscope}%
\begin{pgfscope}%
\definecolor{textcolor}{rgb}{0.150000,0.150000,0.150000}%
\pgfsetstrokecolor{textcolor}%
\pgfsetfillcolor{textcolor}%
\pgftext[x=1.005539in,y=2.501686in,left,base]{\color{textcolor}\rmfamily\fontsize{6.000000}{7.200000}\selectfont 5 x DNGO retrain-reset}%
\end{pgfscope}%
\begin{pgfscope}%
\pgfsetroundcap%
\pgfsetroundjoin%
\pgfsetlinewidth{1.505625pt}%
\definecolor{currentstroke}{rgb}{1.000000,0.498039,0.054902}%
\pgfsetstrokecolor{currentstroke}%
\pgfsetdash{}{0pt}%
\pgfpathmoveto{\pgfqpoint{0.772205in}{2.408538in}}%
\pgfpathlineto{\pgfqpoint{0.938872in}{2.408538in}}%
\pgfusepath{stroke}%
\end{pgfscope}%
\begin{pgfscope}%
\definecolor{textcolor}{rgb}{0.150000,0.150000,0.150000}%
\pgfsetstrokecolor{textcolor}%
\pgfsetfillcolor{textcolor}%
\pgftext[x=1.005539in,y=2.379372in,left,base]{\color{textcolor}\rmfamily\fontsize{6.000000}{7.200000}\selectfont DNGO retrain-reset}%
\end{pgfscope}%
\begin{pgfscope}%
\pgfsetroundcap%
\pgfsetroundjoin%
\pgfsetlinewidth{1.505625pt}%
\definecolor{currentstroke}{rgb}{0.172549,0.627451,0.172549}%
\pgfsetstrokecolor{currentstroke}%
\pgfsetdash{}{0pt}%
\pgfpathmoveto{\pgfqpoint{0.772205in}{2.286224in}}%
\pgfpathlineto{\pgfqpoint{0.938872in}{2.286224in}}%
\pgfusepath{stroke}%
\end{pgfscope}%
\begin{pgfscope}%
\definecolor{textcolor}{rgb}{0.150000,0.150000,0.150000}%
\pgfsetstrokecolor{textcolor}%
\pgfsetfillcolor{textcolor}%
\pgftext[x=1.005539in,y=2.257058in,left,base]{\color{textcolor}\rmfamily\fontsize{6.000000}{7.200000}\selectfont GP}%
\end{pgfscope}%
\end{pgfpicture}%
\makeatother%
\endgroup%

        \end{minipage}\qquad
        \begin{minipage}{0.3\linewidth}
            \centering
            %% Creator: Matplotlib, PGF backend
%%
%% To include the figure in your LaTeX document, write
%%   \input{<filename>.pgf}
%%
%% Make sure the required packages are loaded in your preamble
%%   \usepackage{pgf}
%%
%% Figures using additional raster images can only be included by \input if
%% they are in the same directory as the main LaTeX file. For loading figures
%% from other directories you can use the `import` package
%%   \usepackage{import}
%% and then include the figures with
%%   \import{<path to file>}{<filename>.pgf}
%%
%% Matplotlib used the following preamble
%%   \usepackage{gensymb}
%%   \usepackage{fontspec}
%%   \setmainfont{DejaVu Serif}
%%   \setsansfont{Arial}
%%   \setmonofont{DejaVu Sans Mono}
%%
\begingroup%
\makeatletter%
\begin{pgfpicture}%
\pgfpathrectangle{\pgfpointorigin}{\pgfqpoint{2.300000in}{3.000000in}}%
\pgfusepath{use as bounding box, clip}%
\begin{pgfscope}%
\pgfsetbuttcap%
\pgfsetmiterjoin%
\definecolor{currentfill}{rgb}{1.000000,1.000000,1.000000}%
\pgfsetfillcolor{currentfill}%
\pgfsetlinewidth{0.000000pt}%
\definecolor{currentstroke}{rgb}{1.000000,1.000000,1.000000}%
\pgfsetstrokecolor{currentstroke}%
\pgfsetdash{}{0pt}%
\pgfpathmoveto{\pgfqpoint{0.000000in}{0.000000in}}%
\pgfpathlineto{\pgfqpoint{2.300000in}{0.000000in}}%
\pgfpathlineto{\pgfqpoint{2.300000in}{3.000000in}}%
\pgfpathlineto{\pgfqpoint{0.000000in}{3.000000in}}%
\pgfpathclose%
\pgfusepath{fill}%
\end{pgfscope}%
\begin{pgfscope}%
\pgfsetbuttcap%
\pgfsetmiterjoin%
\definecolor{currentfill}{rgb}{0.917647,0.917647,0.949020}%
\pgfsetfillcolor{currentfill}%
\pgfsetlinewidth{0.000000pt}%
\definecolor{currentstroke}{rgb}{0.000000,0.000000,0.000000}%
\pgfsetstrokecolor{currentstroke}%
\pgfsetstrokeopacity{0.000000}%
\pgfsetdash{}{0pt}%
\pgfpathmoveto{\pgfqpoint{0.287500in}{0.375000in}}%
\pgfpathlineto{\pgfqpoint{2.070000in}{0.375000in}}%
\pgfpathlineto{\pgfqpoint{2.070000in}{2.640000in}}%
\pgfpathlineto{\pgfqpoint{0.287500in}{2.640000in}}%
\pgfpathclose%
\pgfusepath{fill}%
\end{pgfscope}%
\begin{pgfscope}%
\pgfpathrectangle{\pgfqpoint{0.287500in}{0.375000in}}{\pgfqpoint{1.782500in}{2.265000in}}%
\pgfusepath{clip}%
\pgfsetroundcap%
\pgfsetroundjoin%
\pgfsetlinewidth{0.803000pt}%
\definecolor{currentstroke}{rgb}{1.000000,1.000000,1.000000}%
\pgfsetstrokecolor{currentstroke}%
\pgfsetdash{}{0pt}%
\pgfpathmoveto{\pgfqpoint{0.360420in}{0.375000in}}%
\pgfpathlineto{\pgfqpoint{0.360420in}{2.640000in}}%
\pgfusepath{stroke}%
\end{pgfscope}%
\begin{pgfscope}%
\definecolor{textcolor}{rgb}{0.150000,0.150000,0.150000}%
\pgfsetstrokecolor{textcolor}%
\pgfsetfillcolor{textcolor}%
\pgftext[x=0.360420in,y=0.326389in,,top]{\color{textcolor}\rmfamily\fontsize{8.000000}{9.600000}\selectfont \(\displaystyle 0\)}%
\end{pgfscope}%
\begin{pgfscope}%
\pgfpathrectangle{\pgfqpoint{0.287500in}{0.375000in}}{\pgfqpoint{1.782500in}{2.265000in}}%
\pgfusepath{clip}%
\pgfsetroundcap%
\pgfsetroundjoin%
\pgfsetlinewidth{0.803000pt}%
\definecolor{currentstroke}{rgb}{1.000000,1.000000,1.000000}%
\pgfsetstrokecolor{currentstroke}%
\pgfsetdash{}{0pt}%
\pgfpathmoveto{\pgfqpoint{0.765534in}{0.375000in}}%
\pgfpathlineto{\pgfqpoint{0.765534in}{2.640000in}}%
\pgfusepath{stroke}%
\end{pgfscope}%
\begin{pgfscope}%
\definecolor{textcolor}{rgb}{0.150000,0.150000,0.150000}%
\pgfsetstrokecolor{textcolor}%
\pgfsetfillcolor{textcolor}%
\pgftext[x=0.765534in,y=0.326389in,,top]{\color{textcolor}\rmfamily\fontsize{8.000000}{9.600000}\selectfont \(\displaystyle 50\)}%
\end{pgfscope}%
\begin{pgfscope}%
\pgfpathrectangle{\pgfqpoint{0.287500in}{0.375000in}}{\pgfqpoint{1.782500in}{2.265000in}}%
\pgfusepath{clip}%
\pgfsetroundcap%
\pgfsetroundjoin%
\pgfsetlinewidth{0.803000pt}%
\definecolor{currentstroke}{rgb}{1.000000,1.000000,1.000000}%
\pgfsetstrokecolor{currentstroke}%
\pgfsetdash{}{0pt}%
\pgfpathmoveto{\pgfqpoint{1.170648in}{0.375000in}}%
\pgfpathlineto{\pgfqpoint{1.170648in}{2.640000in}}%
\pgfusepath{stroke}%
\end{pgfscope}%
\begin{pgfscope}%
\definecolor{textcolor}{rgb}{0.150000,0.150000,0.150000}%
\pgfsetstrokecolor{textcolor}%
\pgfsetfillcolor{textcolor}%
\pgftext[x=1.170648in,y=0.326389in,,top]{\color{textcolor}\rmfamily\fontsize{8.000000}{9.600000}\selectfont \(\displaystyle 100\)}%
\end{pgfscope}%
\begin{pgfscope}%
\pgfpathrectangle{\pgfqpoint{0.287500in}{0.375000in}}{\pgfqpoint{1.782500in}{2.265000in}}%
\pgfusepath{clip}%
\pgfsetroundcap%
\pgfsetroundjoin%
\pgfsetlinewidth{0.803000pt}%
\definecolor{currentstroke}{rgb}{1.000000,1.000000,1.000000}%
\pgfsetstrokecolor{currentstroke}%
\pgfsetdash{}{0pt}%
\pgfpathmoveto{\pgfqpoint{1.575761in}{0.375000in}}%
\pgfpathlineto{\pgfqpoint{1.575761in}{2.640000in}}%
\pgfusepath{stroke}%
\end{pgfscope}%
\begin{pgfscope}%
\definecolor{textcolor}{rgb}{0.150000,0.150000,0.150000}%
\pgfsetstrokecolor{textcolor}%
\pgfsetfillcolor{textcolor}%
\pgftext[x=1.575761in,y=0.326389in,,top]{\color{textcolor}\rmfamily\fontsize{8.000000}{9.600000}\selectfont \(\displaystyle 150\)}%
\end{pgfscope}%
\begin{pgfscope}%
\pgfpathrectangle{\pgfqpoint{0.287500in}{0.375000in}}{\pgfqpoint{1.782500in}{2.265000in}}%
\pgfusepath{clip}%
\pgfsetroundcap%
\pgfsetroundjoin%
\pgfsetlinewidth{0.803000pt}%
\definecolor{currentstroke}{rgb}{1.000000,1.000000,1.000000}%
\pgfsetstrokecolor{currentstroke}%
\pgfsetdash{}{0pt}%
\pgfpathmoveto{\pgfqpoint{1.980875in}{0.375000in}}%
\pgfpathlineto{\pgfqpoint{1.980875in}{2.640000in}}%
\pgfusepath{stroke}%
\end{pgfscope}%
\begin{pgfscope}%
\definecolor{textcolor}{rgb}{0.150000,0.150000,0.150000}%
\pgfsetstrokecolor{textcolor}%
\pgfsetfillcolor{textcolor}%
\pgftext[x=1.980875in,y=0.326389in,,top]{\color{textcolor}\rmfamily\fontsize{8.000000}{9.600000}\selectfont \(\displaystyle 200\)}%
\end{pgfscope}%
\begin{pgfscope}%
\definecolor{textcolor}{rgb}{0.150000,0.150000,0.150000}%
\pgfsetstrokecolor{textcolor}%
\pgfsetfillcolor{textcolor}%
\pgftext[x=1.178750in,y=0.163303in,,top]{\color{textcolor}\rmfamily\fontsize{8.000000}{9.600000}\selectfont Step \(\displaystyle t\)}%
\end{pgfscope}%
\begin{pgfscope}%
\pgfpathrectangle{\pgfqpoint{0.287500in}{0.375000in}}{\pgfqpoint{1.782500in}{2.265000in}}%
\pgfusepath{clip}%
\pgfsetroundcap%
\pgfsetroundjoin%
\pgfsetlinewidth{0.803000pt}%
\definecolor{currentstroke}{rgb}{1.000000,1.000000,1.000000}%
\pgfsetstrokecolor{currentstroke}%
\pgfsetdash{}{0pt}%
\pgfpathmoveto{\pgfqpoint{0.287500in}{2.325353in}}%
\pgfpathlineto{\pgfqpoint{2.070000in}{2.325353in}}%
\pgfusepath{stroke}%
\end{pgfscope}%
\begin{pgfscope}%
\definecolor{textcolor}{rgb}{0.150000,0.150000,0.150000}%
\pgfsetstrokecolor{textcolor}%
\pgfsetfillcolor{textcolor}%
\pgftext[x=0.062962in,y=2.283143in,left,base]{\color{textcolor}\rmfamily\fontsize{8.000000}{9.600000}\selectfont \(\displaystyle 10^{0}\)}%
\end{pgfscope}%
\begin{pgfscope}%
\definecolor{textcolor}{rgb}{0.150000,0.150000,0.150000}%
\pgfsetstrokecolor{textcolor}%
\pgfsetfillcolor{textcolor}%
\pgftext[x=0.007407in,y=1.507500in,,bottom,rotate=90.000000]{\color{textcolor}\rmfamily\fontsize{8.000000}{9.600000}\selectfont \(\displaystyle R_t/t\)}%
\end{pgfscope}%
\begin{pgfscope}%
\pgfpathrectangle{\pgfqpoint{0.287500in}{0.375000in}}{\pgfqpoint{1.782500in}{2.265000in}}%
\pgfusepath{clip}%
\pgfsetbuttcap%
\pgfsetroundjoin%
\definecolor{currentfill}{rgb}{0.121569,0.466667,0.705882}%
\pgfsetfillcolor{currentfill}%
\pgfsetfillopacity{0.200000}%
\pgfsetlinewidth{0.000000pt}%
\definecolor{currentstroke}{rgb}{0.000000,0.000000,0.000000}%
\pgfsetstrokecolor{currentstroke}%
\pgfsetdash{}{0pt}%
\pgfpathmoveto{\pgfqpoint{0.368523in}{1.927136in}}%
\pgfpathlineto{\pgfqpoint{0.368523in}{2.232400in}}%
\pgfpathlineto{\pgfqpoint{0.376625in}{2.004068in}}%
\pgfpathlineto{\pgfqpoint{0.384727in}{1.983097in}}%
\pgfpathlineto{\pgfqpoint{0.392830in}{1.925711in}}%
\pgfpathlineto{\pgfqpoint{0.400932in}{1.899141in}}%
\pgfpathlineto{\pgfqpoint{0.409034in}{1.855194in}}%
\pgfpathlineto{\pgfqpoint{0.417136in}{1.841899in}}%
\pgfpathlineto{\pgfqpoint{0.425239in}{1.851451in}}%
\pgfpathlineto{\pgfqpoint{0.433341in}{1.879753in}}%
\pgfpathlineto{\pgfqpoint{0.441443in}{1.860766in}}%
\pgfpathlineto{\pgfqpoint{0.449545in}{1.822532in}}%
\pgfpathlineto{\pgfqpoint{0.457648in}{1.786073in}}%
\pgfpathlineto{\pgfqpoint{0.465750in}{1.797389in}}%
\pgfpathlineto{\pgfqpoint{0.473852in}{1.797615in}}%
\pgfpathlineto{\pgfqpoint{0.481955in}{1.774321in}}%
\pgfpathlineto{\pgfqpoint{0.490057in}{1.766453in}}%
\pgfpathlineto{\pgfqpoint{0.498159in}{1.739081in}}%
\pgfpathlineto{\pgfqpoint{0.506261in}{1.743847in}}%
\pgfpathlineto{\pgfqpoint{0.514364in}{1.742794in}}%
\pgfpathlineto{\pgfqpoint{0.522466in}{1.740857in}}%
\pgfpathlineto{\pgfqpoint{0.530568in}{1.745617in}}%
\pgfpathlineto{\pgfqpoint{0.538670in}{1.735078in}}%
\pgfpathlineto{\pgfqpoint{0.546773in}{1.710332in}}%
\pgfpathlineto{\pgfqpoint{0.554875in}{1.706604in}}%
\pgfpathlineto{\pgfqpoint{0.562977in}{1.708461in}}%
\pgfpathlineto{\pgfqpoint{0.571080in}{1.700079in}}%
\pgfpathlineto{\pgfqpoint{0.579182in}{1.705538in}}%
\pgfpathlineto{\pgfqpoint{0.587284in}{1.695780in}}%
\pgfpathlineto{\pgfqpoint{0.595386in}{1.693846in}}%
\pgfpathlineto{\pgfqpoint{0.603489in}{1.676064in}}%
\pgfpathlineto{\pgfqpoint{0.611591in}{1.682674in}}%
\pgfpathlineto{\pgfqpoint{0.619693in}{1.692713in}}%
\pgfpathlineto{\pgfqpoint{0.627795in}{1.698241in}}%
\pgfpathlineto{\pgfqpoint{0.635898in}{1.692989in}}%
\pgfpathlineto{\pgfqpoint{0.644000in}{1.691782in}}%
\pgfpathlineto{\pgfqpoint{0.652102in}{1.675050in}}%
\pgfpathlineto{\pgfqpoint{0.660205in}{1.652302in}}%
\pgfpathlineto{\pgfqpoint{0.668307in}{1.655392in}}%
\pgfpathlineto{\pgfqpoint{0.676409in}{1.650144in}}%
\pgfpathlineto{\pgfqpoint{0.684511in}{1.643596in}}%
\pgfpathlineto{\pgfqpoint{0.692614in}{1.640182in}}%
\pgfpathlineto{\pgfqpoint{0.700716in}{1.639110in}}%
\pgfpathlineto{\pgfqpoint{0.708818in}{1.631486in}}%
\pgfpathlineto{\pgfqpoint{0.716920in}{1.633282in}}%
\pgfpathlineto{\pgfqpoint{0.725023in}{1.628619in}}%
\pgfpathlineto{\pgfqpoint{0.733125in}{1.635141in}}%
\pgfpathlineto{\pgfqpoint{0.741227in}{1.624550in}}%
\pgfpathlineto{\pgfqpoint{0.749330in}{1.616074in}}%
\pgfpathlineto{\pgfqpoint{0.757432in}{1.604345in}}%
\pgfpathlineto{\pgfqpoint{0.765534in}{1.608158in}}%
\pgfpathlineto{\pgfqpoint{0.773636in}{1.603340in}}%
\pgfpathlineto{\pgfqpoint{0.781739in}{1.592366in}}%
\pgfpathlineto{\pgfqpoint{0.789841in}{1.596013in}}%
\pgfpathlineto{\pgfqpoint{0.797943in}{1.580202in}}%
\pgfpathlineto{\pgfqpoint{0.806045in}{1.576001in}}%
\pgfpathlineto{\pgfqpoint{0.814148in}{1.561676in}}%
\pgfpathlineto{\pgfqpoint{0.822250in}{1.547212in}}%
\pgfpathlineto{\pgfqpoint{0.830352in}{1.531467in}}%
\pgfpathlineto{\pgfqpoint{0.838455in}{1.516825in}}%
\pgfpathlineto{\pgfqpoint{0.846557in}{1.503486in}}%
\pgfpathlineto{\pgfqpoint{0.854659in}{1.490423in}}%
\pgfpathlineto{\pgfqpoint{0.862761in}{1.478583in}}%
\pgfpathlineto{\pgfqpoint{0.870864in}{1.464266in}}%
\pgfpathlineto{\pgfqpoint{0.878966in}{1.449980in}}%
\pgfpathlineto{\pgfqpoint{0.887068in}{1.443053in}}%
\pgfpathlineto{\pgfqpoint{0.895170in}{1.429554in}}%
\pgfpathlineto{\pgfqpoint{0.903273in}{1.418944in}}%
\pgfpathlineto{\pgfqpoint{0.911375in}{1.406195in}}%
\pgfpathlineto{\pgfqpoint{0.919477in}{1.396194in}}%
\pgfpathlineto{\pgfqpoint{0.927580in}{1.382970in}}%
\pgfpathlineto{\pgfqpoint{0.935682in}{1.371106in}}%
\pgfpathlineto{\pgfqpoint{0.943784in}{1.364095in}}%
\pgfpathlineto{\pgfqpoint{0.951886in}{1.354880in}}%
\pgfpathlineto{\pgfqpoint{0.959989in}{1.353346in}}%
\pgfpathlineto{\pgfqpoint{0.968091in}{1.341087in}}%
\pgfpathlineto{\pgfqpoint{0.976193in}{1.328908in}}%
\pgfpathlineto{\pgfqpoint{0.984295in}{1.316872in}}%
\pgfpathlineto{\pgfqpoint{0.992398in}{1.305102in}}%
\pgfpathlineto{\pgfqpoint{1.000500in}{1.293572in}}%
\pgfpathlineto{\pgfqpoint{1.008602in}{1.282391in}}%
\pgfpathlineto{\pgfqpoint{1.016705in}{1.271082in}}%
\pgfpathlineto{\pgfqpoint{1.024807in}{1.265177in}}%
\pgfpathlineto{\pgfqpoint{1.032909in}{1.254029in}}%
\pgfpathlineto{\pgfqpoint{1.041011in}{1.243043in}}%
\pgfpathlineto{\pgfqpoint{1.049114in}{1.229451in}}%
\pgfpathlineto{\pgfqpoint{1.057216in}{1.219596in}}%
\pgfpathlineto{\pgfqpoint{1.065318in}{1.209490in}}%
\pgfpathlineto{\pgfqpoint{1.073420in}{1.200081in}}%
\pgfpathlineto{\pgfqpoint{1.081523in}{1.201668in}}%
\pgfpathlineto{\pgfqpoint{1.089625in}{1.193784in}}%
\pgfpathlineto{\pgfqpoint{1.097727in}{1.183728in}}%
\pgfpathlineto{\pgfqpoint{1.105830in}{1.174964in}}%
\pgfpathlineto{\pgfqpoint{1.113932in}{1.164879in}}%
\pgfpathlineto{\pgfqpoint{1.122034in}{1.156539in}}%
\pgfpathlineto{\pgfqpoint{1.130136in}{1.146946in}}%
\pgfpathlineto{\pgfqpoint{1.138239in}{1.136950in}}%
\pgfpathlineto{\pgfqpoint{1.146341in}{1.128720in}}%
\pgfpathlineto{\pgfqpoint{1.154443in}{1.119939in}}%
\pgfpathlineto{\pgfqpoint{1.162545in}{1.110954in}}%
\pgfpathlineto{\pgfqpoint{1.170648in}{1.101778in}}%
\pgfpathlineto{\pgfqpoint{1.178750in}{1.095032in}}%
\pgfpathlineto{\pgfqpoint{1.186852in}{1.088581in}}%
\pgfpathlineto{\pgfqpoint{1.194955in}{1.079814in}}%
\pgfpathlineto{\pgfqpoint{1.203057in}{1.069600in}}%
\pgfpathlineto{\pgfqpoint{1.211159in}{1.074482in}}%
\pgfpathlineto{\pgfqpoint{1.219261in}{1.066387in}}%
\pgfpathlineto{\pgfqpoint{1.227364in}{1.054939in}}%
\pgfpathlineto{\pgfqpoint{1.235466in}{1.046235in}}%
\pgfpathlineto{\pgfqpoint{1.243568in}{1.038589in}}%
\pgfpathlineto{\pgfqpoint{1.251670in}{1.030217in}}%
\pgfpathlineto{\pgfqpoint{1.259773in}{1.024405in}}%
\pgfpathlineto{\pgfqpoint{1.267875in}{1.016823in}}%
\pgfpathlineto{\pgfqpoint{1.275977in}{1.008650in}}%
\pgfpathlineto{\pgfqpoint{1.284080in}{0.998814in}}%
\pgfpathlineto{\pgfqpoint{1.292182in}{0.993667in}}%
\pgfpathlineto{\pgfqpoint{1.300284in}{0.985799in}}%
\pgfpathlineto{\pgfqpoint{1.308386in}{0.977965in}}%
\pgfpathlineto{\pgfqpoint{1.316489in}{0.972869in}}%
\pgfpathlineto{\pgfqpoint{1.324591in}{0.964878in}}%
\pgfpathlineto{\pgfqpoint{1.332693in}{0.962172in}}%
\pgfpathlineto{\pgfqpoint{1.340795in}{0.954597in}}%
\pgfpathlineto{\pgfqpoint{1.348898in}{0.945509in}}%
\pgfpathlineto{\pgfqpoint{1.357000in}{0.937988in}}%
\pgfpathlineto{\pgfqpoint{1.365102in}{0.930667in}}%
\pgfpathlineto{\pgfqpoint{1.373205in}{0.924668in}}%
\pgfpathlineto{\pgfqpoint{1.381307in}{0.917376in}}%
\pgfpathlineto{\pgfqpoint{1.389409in}{0.910203in}}%
\pgfpathlineto{\pgfqpoint{1.397511in}{0.911073in}}%
\pgfpathlineto{\pgfqpoint{1.405614in}{0.915872in}}%
\pgfpathlineto{\pgfqpoint{1.413716in}{0.908985in}}%
\pgfpathlineto{\pgfqpoint{1.421818in}{0.901989in}}%
\pgfpathlineto{\pgfqpoint{1.429920in}{0.895083in}}%
\pgfpathlineto{\pgfqpoint{1.438023in}{0.888316in}}%
\pgfpathlineto{\pgfqpoint{1.446125in}{0.881418in}}%
\pgfpathlineto{\pgfqpoint{1.454227in}{0.874625in}}%
\pgfpathlineto{\pgfqpoint{1.462330in}{0.867738in}}%
\pgfpathlineto{\pgfqpoint{1.470432in}{0.861499in}}%
\pgfpathlineto{\pgfqpoint{1.478534in}{0.857307in}}%
\pgfpathlineto{\pgfqpoint{1.486636in}{0.856603in}}%
\pgfpathlineto{\pgfqpoint{1.494739in}{0.850018in}}%
\pgfpathlineto{\pgfqpoint{1.502841in}{0.844182in}}%
\pgfpathlineto{\pgfqpoint{1.510943in}{0.837696in}}%
\pgfpathlineto{\pgfqpoint{1.519045in}{0.831241in}}%
\pgfpathlineto{\pgfqpoint{1.527148in}{0.824867in}}%
\pgfpathlineto{\pgfqpoint{1.535250in}{0.818506in}}%
\pgfpathlineto{\pgfqpoint{1.543352in}{0.818306in}}%
\pgfpathlineto{\pgfqpoint{1.551455in}{0.812261in}}%
\pgfpathlineto{\pgfqpoint{1.559557in}{0.806000in}}%
\pgfpathlineto{\pgfqpoint{1.567659in}{0.799937in}}%
\pgfpathlineto{\pgfqpoint{1.575761in}{0.794154in}}%
\pgfpathlineto{\pgfqpoint{1.583864in}{0.794265in}}%
\pgfpathlineto{\pgfqpoint{1.591966in}{0.788457in}}%
\pgfpathlineto{\pgfqpoint{1.600068in}{0.782532in}}%
\pgfpathlineto{\pgfqpoint{1.608170in}{0.776599in}}%
\pgfpathlineto{\pgfqpoint{1.616273in}{0.770799in}}%
\pgfpathlineto{\pgfqpoint{1.624375in}{0.765031in}}%
\pgfpathlineto{\pgfqpoint{1.632477in}{0.760324in}}%
\pgfpathlineto{\pgfqpoint{1.640580in}{0.754483in}}%
\pgfpathlineto{\pgfqpoint{1.648682in}{0.748674in}}%
\pgfpathlineto{\pgfqpoint{1.656784in}{0.743207in}}%
\pgfpathlineto{\pgfqpoint{1.664886in}{0.737512in}}%
\pgfpathlineto{\pgfqpoint{1.672989in}{0.732064in}}%
\pgfpathlineto{\pgfqpoint{1.681091in}{0.726543in}}%
\pgfpathlineto{\pgfqpoint{1.689193in}{0.720917in}}%
\pgfpathlineto{\pgfqpoint{1.697295in}{0.715460in}}%
\pgfpathlineto{\pgfqpoint{1.705398in}{0.710124in}}%
\pgfpathlineto{\pgfqpoint{1.713500in}{0.709013in}}%
\pgfpathlineto{\pgfqpoint{1.721602in}{0.703804in}}%
\pgfpathlineto{\pgfqpoint{1.729705in}{0.698382in}}%
\pgfpathlineto{\pgfqpoint{1.737807in}{0.692962in}}%
\pgfpathlineto{\pgfqpoint{1.745909in}{0.687576in}}%
\pgfpathlineto{\pgfqpoint{1.754011in}{0.682330in}}%
\pgfpathlineto{\pgfqpoint{1.762114in}{0.677000in}}%
\pgfpathlineto{\pgfqpoint{1.770216in}{0.671906in}}%
\pgfpathlineto{\pgfqpoint{1.778318in}{0.666638in}}%
\pgfpathlineto{\pgfqpoint{1.786420in}{0.661387in}}%
\pgfpathlineto{\pgfqpoint{1.794523in}{0.656254in}}%
\pgfpathlineto{\pgfqpoint{1.802625in}{0.651135in}}%
\pgfpathlineto{\pgfqpoint{1.810727in}{0.648517in}}%
\pgfpathlineto{\pgfqpoint{1.818830in}{0.644219in}}%
\pgfpathlineto{\pgfqpoint{1.826932in}{0.639263in}}%
\pgfpathlineto{\pgfqpoint{1.835034in}{0.634127in}}%
\pgfpathlineto{\pgfqpoint{1.843136in}{0.629118in}}%
\pgfpathlineto{\pgfqpoint{1.851239in}{0.624237in}}%
\pgfpathlineto{\pgfqpoint{1.859341in}{0.620020in}}%
\pgfpathlineto{\pgfqpoint{1.867443in}{0.617105in}}%
\pgfpathlineto{\pgfqpoint{1.875545in}{0.612179in}}%
\pgfpathlineto{\pgfqpoint{1.883648in}{0.607368in}}%
\pgfpathlineto{\pgfqpoint{1.891750in}{0.605692in}}%
\pgfpathlineto{\pgfqpoint{1.899852in}{0.600833in}}%
\pgfpathlineto{\pgfqpoint{1.907955in}{0.596092in}}%
\pgfpathlineto{\pgfqpoint{1.916057in}{0.590444in}}%
\pgfpathlineto{\pgfqpoint{1.924159in}{0.588468in}}%
\pgfpathlineto{\pgfqpoint{1.932261in}{0.583752in}}%
\pgfpathlineto{\pgfqpoint{1.940364in}{0.582005in}}%
\pgfpathlineto{\pgfqpoint{1.948466in}{0.577432in}}%
\pgfpathlineto{\pgfqpoint{1.956568in}{0.572773in}}%
\pgfpathlineto{\pgfqpoint{1.964670in}{0.568172in}}%
\pgfpathlineto{\pgfqpoint{1.972773in}{0.563574in}}%
\pgfpathlineto{\pgfqpoint{1.980875in}{0.560802in}}%
\pgfpathlineto{\pgfqpoint{1.988977in}{0.556225in}}%
\pgfpathlineto{\pgfqpoint{1.988977in}{0.477955in}}%
\pgfpathlineto{\pgfqpoint{1.988977in}{0.477955in}}%
\pgfpathlineto{\pgfqpoint{1.980875in}{0.482471in}}%
\pgfpathlineto{\pgfqpoint{1.972773in}{0.486286in}}%
\pgfpathlineto{\pgfqpoint{1.964670in}{0.490929in}}%
\pgfpathlineto{\pgfqpoint{1.956568in}{0.495516in}}%
\pgfpathlineto{\pgfqpoint{1.948466in}{0.499985in}}%
\pgfpathlineto{\pgfqpoint{1.940364in}{0.493648in}}%
\pgfpathlineto{\pgfqpoint{1.932261in}{0.497339in}}%
\pgfpathlineto{\pgfqpoint{1.924159in}{0.502056in}}%
\pgfpathlineto{\pgfqpoint{1.916057in}{0.505032in}}%
\pgfpathlineto{\pgfqpoint{1.907955in}{0.500621in}}%
\pgfpathlineto{\pgfqpoint{1.899852in}{0.505378in}}%
\pgfpathlineto{\pgfqpoint{1.891750in}{0.510221in}}%
\pgfpathlineto{\pgfqpoint{1.883648in}{0.510384in}}%
\pgfpathlineto{\pgfqpoint{1.875545in}{0.515258in}}%
\pgfpathlineto{\pgfqpoint{1.867443in}{0.520120in}}%
\pgfpathlineto{\pgfqpoint{1.859341in}{0.524780in}}%
\pgfpathlineto{\pgfqpoint{1.851239in}{0.529755in}}%
\pgfpathlineto{\pgfqpoint{1.843136in}{0.534590in}}%
\pgfpathlineto{\pgfqpoint{1.835034in}{0.539622in}}%
\pgfpathlineto{\pgfqpoint{1.826932in}{0.542796in}}%
\pgfpathlineto{\pgfqpoint{1.818830in}{0.547717in}}%
\pgfpathlineto{\pgfqpoint{1.810727in}{0.552306in}}%
\pgfpathlineto{\pgfqpoint{1.802625in}{0.555973in}}%
\pgfpathlineto{\pgfqpoint{1.794523in}{0.561079in}}%
\pgfpathlineto{\pgfqpoint{1.786420in}{0.566303in}}%
\pgfpathlineto{\pgfqpoint{1.778318in}{0.571425in}}%
\pgfpathlineto{\pgfqpoint{1.770216in}{0.576662in}}%
\pgfpathlineto{\pgfqpoint{1.762114in}{0.581962in}}%
\pgfpathlineto{\pgfqpoint{1.754011in}{0.586624in}}%
\pgfpathlineto{\pgfqpoint{1.745909in}{0.591936in}}%
\pgfpathlineto{\pgfqpoint{1.737807in}{0.597274in}}%
\pgfpathlineto{\pgfqpoint{1.729705in}{0.602663in}}%
\pgfpathlineto{\pgfqpoint{1.721602in}{0.608101in}}%
\pgfpathlineto{\pgfqpoint{1.713500in}{0.613324in}}%
\pgfpathlineto{\pgfqpoint{1.705398in}{0.617524in}}%
\pgfpathlineto{\pgfqpoint{1.697295in}{0.623091in}}%
\pgfpathlineto{\pgfqpoint{1.689193in}{0.628367in}}%
\pgfpathlineto{\pgfqpoint{1.681091in}{0.633941in}}%
\pgfpathlineto{\pgfqpoint{1.672989in}{0.639577in}}%
\pgfpathlineto{\pgfqpoint{1.664886in}{0.645209in}}%
\pgfpathlineto{\pgfqpoint{1.656784in}{0.650913in}}%
\pgfpathlineto{\pgfqpoint{1.648682in}{0.656663in}}%
\pgfpathlineto{\pgfqpoint{1.640580in}{0.661992in}}%
\pgfpathlineto{\pgfqpoint{1.632477in}{0.667761in}}%
\pgfpathlineto{\pgfqpoint{1.624375in}{0.672919in}}%
\pgfpathlineto{\pgfqpoint{1.616273in}{0.678740in}}%
\pgfpathlineto{\pgfqpoint{1.608170in}{0.684599in}}%
\pgfpathlineto{\pgfqpoint{1.600068in}{0.690508in}}%
\pgfpathlineto{\pgfqpoint{1.591966in}{0.696504in}}%
\pgfpathlineto{\pgfqpoint{1.583864in}{0.702440in}}%
\pgfpathlineto{\pgfqpoint{1.575761in}{0.709319in}}%
\pgfpathlineto{\pgfqpoint{1.567659in}{0.715267in}}%
\pgfpathlineto{\pgfqpoint{1.559557in}{0.721405in}}%
\pgfpathlineto{\pgfqpoint{1.551455in}{0.726866in}}%
\pgfpathlineto{\pgfqpoint{1.543352in}{0.733067in}}%
\pgfpathlineto{\pgfqpoint{1.535250in}{0.732970in}}%
\pgfpathlineto{\pgfqpoint{1.527148in}{0.739306in}}%
\pgfpathlineto{\pgfqpoint{1.519045in}{0.745612in}}%
\pgfpathlineto{\pgfqpoint{1.510943in}{0.752068in}}%
\pgfpathlineto{\pgfqpoint{1.502841in}{0.758544in}}%
\pgfpathlineto{\pgfqpoint{1.494739in}{0.764629in}}%
\pgfpathlineto{\pgfqpoint{1.486636in}{0.771178in}}%
\pgfpathlineto{\pgfqpoint{1.478534in}{0.776092in}}%
\pgfpathlineto{\pgfqpoint{1.470432in}{0.773121in}}%
\pgfpathlineto{\pgfqpoint{1.462330in}{0.774471in}}%
\pgfpathlineto{\pgfqpoint{1.454227in}{0.772261in}}%
\pgfpathlineto{\pgfqpoint{1.446125in}{0.779047in}}%
\pgfpathlineto{\pgfqpoint{1.438023in}{0.785937in}}%
\pgfpathlineto{\pgfqpoint{1.429920in}{0.792726in}}%
\pgfpathlineto{\pgfqpoint{1.421818in}{0.799593in}}%
\pgfpathlineto{\pgfqpoint{1.413716in}{0.806647in}}%
\pgfpathlineto{\pgfqpoint{1.405614in}{0.813759in}}%
\pgfpathlineto{\pgfqpoint{1.397511in}{0.823289in}}%
\pgfpathlineto{\pgfqpoint{1.389409in}{0.827823in}}%
\pgfpathlineto{\pgfqpoint{1.381307in}{0.834717in}}%
\pgfpathlineto{\pgfqpoint{1.373205in}{0.841876in}}%
\pgfpathlineto{\pgfqpoint{1.365102in}{0.848579in}}%
\pgfpathlineto{\pgfqpoint{1.357000in}{0.855260in}}%
\pgfpathlineto{\pgfqpoint{1.348898in}{0.862619in}}%
\pgfpathlineto{\pgfqpoint{1.340795in}{0.859281in}}%
\pgfpathlineto{\pgfqpoint{1.332693in}{0.866809in}}%
\pgfpathlineto{\pgfqpoint{1.324591in}{0.870580in}}%
\pgfpathlineto{\pgfqpoint{1.316489in}{0.876039in}}%
\pgfpathlineto{\pgfqpoint{1.308386in}{0.884524in}}%
\pgfpathlineto{\pgfqpoint{1.300284in}{0.892332in}}%
\pgfpathlineto{\pgfqpoint{1.292182in}{0.900160in}}%
\pgfpathlineto{\pgfqpoint{1.284080in}{0.906837in}}%
\pgfpathlineto{\pgfqpoint{1.275977in}{0.905775in}}%
\pgfpathlineto{\pgfqpoint{1.267875in}{0.913913in}}%
\pgfpathlineto{\pgfqpoint{1.259773in}{0.922287in}}%
\pgfpathlineto{\pgfqpoint{1.251670in}{0.928538in}}%
\pgfpathlineto{\pgfqpoint{1.243568in}{0.936840in}}%
\pgfpathlineto{\pgfqpoint{1.235466in}{0.944216in}}%
\pgfpathlineto{\pgfqpoint{1.227364in}{0.951644in}}%
\pgfpathlineto{\pgfqpoint{1.219261in}{0.943252in}}%
\pgfpathlineto{\pgfqpoint{1.211159in}{0.952074in}}%
\pgfpathlineto{\pgfqpoint{1.203057in}{0.961331in}}%
\pgfpathlineto{\pgfqpoint{1.194955in}{0.964652in}}%
\pgfpathlineto{\pgfqpoint{1.186852in}{0.973469in}}%
\pgfpathlineto{\pgfqpoint{1.178750in}{0.980788in}}%
\pgfpathlineto{\pgfqpoint{1.170648in}{0.988078in}}%
\pgfpathlineto{\pgfqpoint{1.162545in}{0.997249in}}%
\pgfpathlineto{\pgfqpoint{1.154443in}{1.006174in}}%
\pgfpathlineto{\pgfqpoint{1.146341in}{1.014891in}}%
\pgfpathlineto{\pgfqpoint{1.138239in}{1.023343in}}%
\pgfpathlineto{\pgfqpoint{1.130136in}{1.031041in}}%
\pgfpathlineto{\pgfqpoint{1.122034in}{1.040066in}}%
\pgfpathlineto{\pgfqpoint{1.113932in}{1.044143in}}%
\pgfpathlineto{\pgfqpoint{1.105830in}{1.053559in}}%
\pgfpathlineto{\pgfqpoint{1.097727in}{1.062447in}}%
\pgfpathlineto{\pgfqpoint{1.089625in}{1.071968in}}%
\pgfpathlineto{\pgfqpoint{1.081523in}{1.073044in}}%
\pgfpathlineto{\pgfqpoint{1.073420in}{1.083384in}}%
\pgfpathlineto{\pgfqpoint{1.065318in}{1.091841in}}%
\pgfpathlineto{\pgfqpoint{1.057216in}{1.102124in}}%
\pgfpathlineto{\pgfqpoint{1.049114in}{1.112320in}}%
\pgfpathlineto{\pgfqpoint{1.041011in}{1.112941in}}%
\pgfpathlineto{\pgfqpoint{1.032909in}{1.123944in}}%
\pgfpathlineto{\pgfqpoint{1.024807in}{1.134996in}}%
\pgfpathlineto{\pgfqpoint{1.016705in}{1.138528in}}%
\pgfpathlineto{\pgfqpoint{1.008602in}{1.149829in}}%
\pgfpathlineto{\pgfqpoint{1.000500in}{1.157677in}}%
\pgfpathlineto{\pgfqpoint{0.992398in}{1.169210in}}%
\pgfpathlineto{\pgfqpoint{0.984295in}{1.181021in}}%
\pgfpathlineto{\pgfqpoint{0.976193in}{1.193028in}}%
\pgfpathlineto{\pgfqpoint{0.968091in}{1.205098in}}%
\pgfpathlineto{\pgfqpoint{0.959989in}{1.217390in}}%
\pgfpathlineto{\pgfqpoint{0.951886in}{1.227675in}}%
\pgfpathlineto{\pgfqpoint{0.943784in}{1.236241in}}%
\pgfpathlineto{\pgfqpoint{0.935682in}{1.246896in}}%
\pgfpathlineto{\pgfqpoint{0.927580in}{1.259370in}}%
\pgfpathlineto{\pgfqpoint{0.919477in}{1.272595in}}%
\pgfpathlineto{\pgfqpoint{0.911375in}{1.284833in}}%
\pgfpathlineto{\pgfqpoint{0.903273in}{1.298102in}}%
\pgfpathlineto{\pgfqpoint{0.895170in}{1.310112in}}%
\pgfpathlineto{\pgfqpoint{0.887068in}{1.323261in}}%
\pgfpathlineto{\pgfqpoint{0.878966in}{1.329618in}}%
\pgfpathlineto{\pgfqpoint{0.870864in}{1.343875in}}%
\pgfpathlineto{\pgfqpoint{0.862761in}{1.358226in}}%
\pgfpathlineto{\pgfqpoint{0.854659in}{1.369541in}}%
\pgfpathlineto{\pgfqpoint{0.846557in}{1.382189in}}%
\pgfpathlineto{\pgfqpoint{0.838455in}{1.394974in}}%
\pgfpathlineto{\pgfqpoint{0.830352in}{1.410392in}}%
\pgfpathlineto{\pgfqpoint{0.822250in}{1.426303in}}%
\pgfpathlineto{\pgfqpoint{0.814148in}{1.439922in}}%
\pgfpathlineto{\pgfqpoint{0.806045in}{1.455621in}}%
\pgfpathlineto{\pgfqpoint{0.797943in}{1.472035in}}%
\pgfpathlineto{\pgfqpoint{0.789841in}{1.488015in}}%
\pgfpathlineto{\pgfqpoint{0.781739in}{1.498028in}}%
\pgfpathlineto{\pgfqpoint{0.773636in}{1.507634in}}%
\pgfpathlineto{\pgfqpoint{0.765534in}{1.523516in}}%
\pgfpathlineto{\pgfqpoint{0.757432in}{1.530376in}}%
\pgfpathlineto{\pgfqpoint{0.749330in}{1.533429in}}%
\pgfpathlineto{\pgfqpoint{0.741227in}{1.535297in}}%
\pgfpathlineto{\pgfqpoint{0.733125in}{1.550589in}}%
\pgfpathlineto{\pgfqpoint{0.725023in}{1.559434in}}%
\pgfpathlineto{\pgfqpoint{0.716920in}{1.566849in}}%
\pgfpathlineto{\pgfqpoint{0.708818in}{1.565855in}}%
\pgfpathlineto{\pgfqpoint{0.700716in}{1.580562in}}%
\pgfpathlineto{\pgfqpoint{0.692614in}{1.582647in}}%
\pgfpathlineto{\pgfqpoint{0.684511in}{1.584160in}}%
\pgfpathlineto{\pgfqpoint{0.676409in}{1.592535in}}%
\pgfpathlineto{\pgfqpoint{0.668307in}{1.605465in}}%
\pgfpathlineto{\pgfqpoint{0.660205in}{1.606722in}}%
\pgfpathlineto{\pgfqpoint{0.652102in}{1.597210in}}%
\pgfpathlineto{\pgfqpoint{0.644000in}{1.599360in}}%
\pgfpathlineto{\pgfqpoint{0.635898in}{1.611811in}}%
\pgfpathlineto{\pgfqpoint{0.627795in}{1.602219in}}%
\pgfpathlineto{\pgfqpoint{0.619693in}{1.608935in}}%
\pgfpathlineto{\pgfqpoint{0.611591in}{1.606490in}}%
\pgfpathlineto{\pgfqpoint{0.603489in}{1.604123in}}%
\pgfpathlineto{\pgfqpoint{0.595386in}{1.606647in}}%
\pgfpathlineto{\pgfqpoint{0.587284in}{1.597009in}}%
\pgfpathlineto{\pgfqpoint{0.579182in}{1.603256in}}%
\pgfpathlineto{\pgfqpoint{0.571080in}{1.592835in}}%
\pgfpathlineto{\pgfqpoint{0.562977in}{1.604220in}}%
\pgfpathlineto{\pgfqpoint{0.554875in}{1.601476in}}%
\pgfpathlineto{\pgfqpoint{0.546773in}{1.628143in}}%
\pgfpathlineto{\pgfqpoint{0.538670in}{1.614813in}}%
\pgfpathlineto{\pgfqpoint{0.530568in}{1.591523in}}%
\pgfpathlineto{\pgfqpoint{0.522466in}{1.582896in}}%
\pgfpathlineto{\pgfqpoint{0.514364in}{1.586046in}}%
\pgfpathlineto{\pgfqpoint{0.506261in}{1.618124in}}%
\pgfpathlineto{\pgfqpoint{0.498159in}{1.611172in}}%
\pgfpathlineto{\pgfqpoint{0.490057in}{1.637866in}}%
\pgfpathlineto{\pgfqpoint{0.481955in}{1.646642in}}%
\pgfpathlineto{\pgfqpoint{0.473852in}{1.686977in}}%
\pgfpathlineto{\pgfqpoint{0.465750in}{1.685120in}}%
\pgfpathlineto{\pgfqpoint{0.457648in}{1.744745in}}%
\pgfpathlineto{\pgfqpoint{0.449545in}{1.733263in}}%
\pgfpathlineto{\pgfqpoint{0.441443in}{1.734307in}}%
\pgfpathlineto{\pgfqpoint{0.433341in}{1.757824in}}%
\pgfpathlineto{\pgfqpoint{0.425239in}{1.776378in}}%
\pgfpathlineto{\pgfqpoint{0.417136in}{1.776437in}}%
\pgfpathlineto{\pgfqpoint{0.409034in}{1.788905in}}%
\pgfpathlineto{\pgfqpoint{0.400932in}{1.739932in}}%
\pgfpathlineto{\pgfqpoint{0.392830in}{1.812357in}}%
\pgfpathlineto{\pgfqpoint{0.384727in}{1.913249in}}%
\pgfpathlineto{\pgfqpoint{0.376625in}{1.899751in}}%
\pgfpathlineto{\pgfqpoint{0.368523in}{1.927136in}}%
\pgfpathclose%
\pgfusepath{fill}%
\end{pgfscope}%
\begin{pgfscope}%
\pgfpathrectangle{\pgfqpoint{0.287500in}{0.375000in}}{\pgfqpoint{1.782500in}{2.265000in}}%
\pgfusepath{clip}%
\pgfsetbuttcap%
\pgfsetroundjoin%
\definecolor{currentfill}{rgb}{1.000000,0.498039,0.054902}%
\pgfsetfillcolor{currentfill}%
\pgfsetfillopacity{0.200000}%
\pgfsetlinewidth{0.000000pt}%
\definecolor{currentstroke}{rgb}{0.000000,0.000000,0.000000}%
\pgfsetstrokecolor{currentstroke}%
\pgfsetdash{}{0pt}%
\pgfpathmoveto{\pgfqpoint{0.368523in}{0.810342in}}%
\pgfpathlineto{\pgfqpoint{0.368523in}{2.537045in}}%
\pgfpathlineto{\pgfqpoint{0.376625in}{2.243769in}}%
\pgfpathlineto{\pgfqpoint{0.384727in}{2.119745in}}%
\pgfpathlineto{\pgfqpoint{0.392830in}{2.038072in}}%
\pgfpathlineto{\pgfqpoint{0.400932in}{1.989565in}}%
\pgfpathlineto{\pgfqpoint{0.409034in}{1.961469in}}%
\pgfpathlineto{\pgfqpoint{0.417136in}{1.950731in}}%
\pgfpathlineto{\pgfqpoint{0.425239in}{1.957197in}}%
\pgfpathlineto{\pgfqpoint{0.433341in}{1.949500in}}%
\pgfpathlineto{\pgfqpoint{0.441443in}{1.932939in}}%
\pgfpathlineto{\pgfqpoint{0.449545in}{1.923687in}}%
\pgfpathlineto{\pgfqpoint{0.457648in}{1.906425in}}%
\pgfpathlineto{\pgfqpoint{0.465750in}{1.894625in}}%
\pgfpathlineto{\pgfqpoint{0.473852in}{1.908556in}}%
\pgfpathlineto{\pgfqpoint{0.481955in}{1.907851in}}%
\pgfpathlineto{\pgfqpoint{0.490057in}{1.892731in}}%
\pgfpathlineto{\pgfqpoint{0.498159in}{1.899899in}}%
\pgfpathlineto{\pgfqpoint{0.506261in}{1.906689in}}%
\pgfpathlineto{\pgfqpoint{0.514364in}{1.910592in}}%
\pgfpathlineto{\pgfqpoint{0.522466in}{1.900065in}}%
\pgfpathlineto{\pgfqpoint{0.530568in}{1.886960in}}%
\pgfpathlineto{\pgfqpoint{0.538670in}{1.879733in}}%
\pgfpathlineto{\pgfqpoint{0.546773in}{1.868473in}}%
\pgfpathlineto{\pgfqpoint{0.554875in}{1.850802in}}%
\pgfpathlineto{\pgfqpoint{0.562977in}{1.828748in}}%
\pgfpathlineto{\pgfqpoint{0.571080in}{1.827199in}}%
\pgfpathlineto{\pgfqpoint{0.579182in}{1.828835in}}%
\pgfpathlineto{\pgfqpoint{0.587284in}{1.810930in}}%
\pgfpathlineto{\pgfqpoint{0.595386in}{1.805901in}}%
\pgfpathlineto{\pgfqpoint{0.603489in}{1.795168in}}%
\pgfpathlineto{\pgfqpoint{0.611591in}{1.794258in}}%
\pgfpathlineto{\pgfqpoint{0.619693in}{1.797308in}}%
\pgfpathlineto{\pgfqpoint{0.627795in}{1.807964in}}%
\pgfpathlineto{\pgfqpoint{0.635898in}{1.806468in}}%
\pgfpathlineto{\pgfqpoint{0.644000in}{1.800243in}}%
\pgfpathlineto{\pgfqpoint{0.652102in}{1.792776in}}%
\pgfpathlineto{\pgfqpoint{0.660205in}{1.790404in}}%
\pgfpathlineto{\pgfqpoint{0.668307in}{1.783810in}}%
\pgfpathlineto{\pgfqpoint{0.676409in}{1.772148in}}%
\pgfpathlineto{\pgfqpoint{0.684511in}{1.771607in}}%
\pgfpathlineto{\pgfqpoint{0.692614in}{1.766439in}}%
\pgfpathlineto{\pgfqpoint{0.700716in}{1.755676in}}%
\pgfpathlineto{\pgfqpoint{0.708818in}{1.745078in}}%
\pgfpathlineto{\pgfqpoint{0.716920in}{1.735236in}}%
\pgfpathlineto{\pgfqpoint{0.725023in}{1.733297in}}%
\pgfpathlineto{\pgfqpoint{0.733125in}{1.723436in}}%
\pgfpathlineto{\pgfqpoint{0.741227in}{1.718273in}}%
\pgfpathlineto{\pgfqpoint{0.749330in}{1.710638in}}%
\pgfpathlineto{\pgfqpoint{0.757432in}{1.701659in}}%
\pgfpathlineto{\pgfqpoint{0.765534in}{1.692676in}}%
\pgfpathlineto{\pgfqpoint{0.773636in}{1.682616in}}%
\pgfpathlineto{\pgfqpoint{0.781739in}{1.683309in}}%
\pgfpathlineto{\pgfqpoint{0.789841in}{1.670014in}}%
\pgfpathlineto{\pgfqpoint{0.797943in}{1.657391in}}%
\pgfpathlineto{\pgfqpoint{0.806045in}{1.651758in}}%
\pgfpathlineto{\pgfqpoint{0.814148in}{1.646307in}}%
\pgfpathlineto{\pgfqpoint{0.822250in}{1.634029in}}%
\pgfpathlineto{\pgfqpoint{0.830352in}{1.628326in}}%
\pgfpathlineto{\pgfqpoint{0.838455in}{1.623338in}}%
\pgfpathlineto{\pgfqpoint{0.846557in}{1.616589in}}%
\pgfpathlineto{\pgfqpoint{0.854659in}{1.608082in}}%
\pgfpathlineto{\pgfqpoint{0.862761in}{1.599098in}}%
\pgfpathlineto{\pgfqpoint{0.870864in}{1.588509in}}%
\pgfpathlineto{\pgfqpoint{0.878966in}{1.576327in}}%
\pgfpathlineto{\pgfqpoint{0.887068in}{1.566880in}}%
\pgfpathlineto{\pgfqpoint{0.895170in}{1.565942in}}%
\pgfpathlineto{\pgfqpoint{0.903273in}{1.553753in}}%
\pgfpathlineto{\pgfqpoint{0.911375in}{1.542277in}}%
\pgfpathlineto{\pgfqpoint{0.919477in}{1.530686in}}%
\pgfpathlineto{\pgfqpoint{0.927580in}{1.525760in}}%
\pgfpathlineto{\pgfqpoint{0.935682in}{1.520474in}}%
\pgfpathlineto{\pgfqpoint{0.943784in}{1.513296in}}%
\pgfpathlineto{\pgfqpoint{0.951886in}{1.507776in}}%
\pgfpathlineto{\pgfqpoint{0.959989in}{1.500969in}}%
\pgfpathlineto{\pgfqpoint{0.968091in}{1.491046in}}%
\pgfpathlineto{\pgfqpoint{0.976193in}{1.485028in}}%
\pgfpathlineto{\pgfqpoint{0.984295in}{1.483069in}}%
\pgfpathlineto{\pgfqpoint{0.992398in}{1.477504in}}%
\pgfpathlineto{\pgfqpoint{1.000500in}{1.466141in}}%
\pgfpathlineto{\pgfqpoint{1.008602in}{1.456236in}}%
\pgfpathlineto{\pgfqpoint{1.016705in}{1.445971in}}%
\pgfpathlineto{\pgfqpoint{1.024807in}{1.437756in}}%
\pgfpathlineto{\pgfqpoint{1.032909in}{1.432846in}}%
\pgfpathlineto{\pgfqpoint{1.041011in}{1.431177in}}%
\pgfpathlineto{\pgfqpoint{1.049114in}{1.423176in}}%
\pgfpathlineto{\pgfqpoint{1.057216in}{1.413459in}}%
\pgfpathlineto{\pgfqpoint{1.065318in}{1.404996in}}%
\pgfpathlineto{\pgfqpoint{1.073420in}{1.399415in}}%
\pgfpathlineto{\pgfqpoint{1.081523in}{1.396178in}}%
\pgfpathlineto{\pgfqpoint{1.089625in}{1.392771in}}%
\pgfpathlineto{\pgfqpoint{1.097727in}{1.392872in}}%
\pgfpathlineto{\pgfqpoint{1.105830in}{1.388056in}}%
\pgfpathlineto{\pgfqpoint{1.113932in}{1.383089in}}%
\pgfpathlineto{\pgfqpoint{1.122034in}{1.379861in}}%
\pgfpathlineto{\pgfqpoint{1.130136in}{1.372616in}}%
\pgfpathlineto{\pgfqpoint{1.138239in}{1.377883in}}%
\pgfpathlineto{\pgfqpoint{1.146341in}{1.373193in}}%
\pgfpathlineto{\pgfqpoint{1.154443in}{1.365151in}}%
\pgfpathlineto{\pgfqpoint{1.162545in}{1.360705in}}%
\pgfpathlineto{\pgfqpoint{1.170648in}{1.351577in}}%
\pgfpathlineto{\pgfqpoint{1.178750in}{1.346525in}}%
\pgfpathlineto{\pgfqpoint{1.186852in}{1.346370in}}%
\pgfpathlineto{\pgfqpoint{1.194955in}{1.339657in}}%
\pgfpathlineto{\pgfqpoint{1.203057in}{1.332088in}}%
\pgfpathlineto{\pgfqpoint{1.211159in}{1.325825in}}%
\pgfpathlineto{\pgfqpoint{1.219261in}{1.320136in}}%
\pgfpathlineto{\pgfqpoint{1.227364in}{1.313143in}}%
\pgfpathlineto{\pgfqpoint{1.235466in}{1.307315in}}%
\pgfpathlineto{\pgfqpoint{1.243568in}{1.302759in}}%
\pgfpathlineto{\pgfqpoint{1.251670in}{1.297884in}}%
\pgfpathlineto{\pgfqpoint{1.259773in}{1.290348in}}%
\pgfpathlineto{\pgfqpoint{1.267875in}{1.283499in}}%
\pgfpathlineto{\pgfqpoint{1.275977in}{1.279293in}}%
\pgfpathlineto{\pgfqpoint{1.284080in}{1.271559in}}%
\pgfpathlineto{\pgfqpoint{1.292182in}{1.270530in}}%
\pgfpathlineto{\pgfqpoint{1.300284in}{1.265204in}}%
\pgfpathlineto{\pgfqpoint{1.308386in}{1.259194in}}%
\pgfpathlineto{\pgfqpoint{1.316489in}{1.256236in}}%
\pgfpathlineto{\pgfqpoint{1.324591in}{1.251535in}}%
\pgfpathlineto{\pgfqpoint{1.332693in}{1.249200in}}%
\pgfpathlineto{\pgfqpoint{1.340795in}{1.247440in}}%
\pgfpathlineto{\pgfqpoint{1.348898in}{1.245381in}}%
\pgfpathlineto{\pgfqpoint{1.357000in}{1.238465in}}%
\pgfpathlineto{\pgfqpoint{1.365102in}{1.236221in}}%
\pgfpathlineto{\pgfqpoint{1.373205in}{1.235107in}}%
\pgfpathlineto{\pgfqpoint{1.381307in}{1.237083in}}%
\pgfpathlineto{\pgfqpoint{1.389409in}{1.236272in}}%
\pgfpathlineto{\pgfqpoint{1.397511in}{1.237088in}}%
\pgfpathlineto{\pgfqpoint{1.405614in}{1.231410in}}%
\pgfpathlineto{\pgfqpoint{1.413716in}{1.224321in}}%
\pgfpathlineto{\pgfqpoint{1.421818in}{1.222648in}}%
\pgfpathlineto{\pgfqpoint{1.429920in}{1.219872in}}%
\pgfpathlineto{\pgfqpoint{1.438023in}{1.214351in}}%
\pgfpathlineto{\pgfqpoint{1.446125in}{1.210095in}}%
\pgfpathlineto{\pgfqpoint{1.454227in}{1.208018in}}%
\pgfpathlineto{\pgfqpoint{1.462330in}{1.204176in}}%
\pgfpathlineto{\pgfqpoint{1.470432in}{1.202916in}}%
\pgfpathlineto{\pgfqpoint{1.478534in}{1.198791in}}%
\pgfpathlineto{\pgfqpoint{1.486636in}{1.202668in}}%
\pgfpathlineto{\pgfqpoint{1.494739in}{1.197070in}}%
\pgfpathlineto{\pgfqpoint{1.502841in}{1.193553in}}%
\pgfpathlineto{\pgfqpoint{1.510943in}{1.167294in}}%
\pgfpathlineto{\pgfqpoint{1.519045in}{1.166892in}}%
\pgfpathlineto{\pgfqpoint{1.527148in}{1.163407in}}%
\pgfpathlineto{\pgfqpoint{1.535250in}{1.155472in}}%
\pgfpathlineto{\pgfqpoint{1.543352in}{1.151394in}}%
\pgfpathlineto{\pgfqpoint{1.551455in}{1.147475in}}%
\pgfpathlineto{\pgfqpoint{1.559557in}{1.141277in}}%
\pgfpathlineto{\pgfqpoint{1.567659in}{1.135284in}}%
\pgfpathlineto{\pgfqpoint{1.575761in}{1.129325in}}%
\pgfpathlineto{\pgfqpoint{1.583864in}{1.134144in}}%
\pgfpathlineto{\pgfqpoint{1.591966in}{1.130812in}}%
\pgfpathlineto{\pgfqpoint{1.600068in}{1.125037in}}%
\pgfpathlineto{\pgfqpoint{1.608170in}{1.119097in}}%
\pgfpathlineto{\pgfqpoint{1.616273in}{1.115282in}}%
\pgfpathlineto{\pgfqpoint{1.624375in}{1.110646in}}%
\pgfpathlineto{\pgfqpoint{1.632477in}{1.105805in}}%
\pgfpathlineto{\pgfqpoint{1.640580in}{1.101003in}}%
\pgfpathlineto{\pgfqpoint{1.648682in}{1.100721in}}%
\pgfpathlineto{\pgfqpoint{1.656784in}{1.097601in}}%
\pgfpathlineto{\pgfqpoint{1.664886in}{1.100859in}}%
\pgfpathlineto{\pgfqpoint{1.672989in}{1.097416in}}%
\pgfpathlineto{\pgfqpoint{1.681091in}{1.091540in}}%
\pgfpathlineto{\pgfqpoint{1.689193in}{1.088838in}}%
\pgfpathlineto{\pgfqpoint{1.697295in}{1.084987in}}%
\pgfpathlineto{\pgfqpoint{1.705398in}{1.079756in}}%
\pgfpathlineto{\pgfqpoint{1.713500in}{1.074425in}}%
\pgfpathlineto{\pgfqpoint{1.721602in}{1.074230in}}%
\pgfpathlineto{\pgfqpoint{1.729705in}{1.069441in}}%
\pgfpathlineto{\pgfqpoint{1.737807in}{1.065218in}}%
\pgfpathlineto{\pgfqpoint{1.745909in}{1.061445in}}%
\pgfpathlineto{\pgfqpoint{1.754011in}{1.055673in}}%
\pgfpathlineto{\pgfqpoint{1.762114in}{1.051177in}}%
\pgfpathlineto{\pgfqpoint{1.770216in}{1.048506in}}%
\pgfpathlineto{\pgfqpoint{1.778318in}{1.048182in}}%
\pgfpathlineto{\pgfqpoint{1.786420in}{1.046620in}}%
\pgfpathlineto{\pgfqpoint{1.794523in}{1.044932in}}%
\pgfpathlineto{\pgfqpoint{1.802625in}{1.044060in}}%
\pgfpathlineto{\pgfqpoint{1.810727in}{1.039286in}}%
\pgfpathlineto{\pgfqpoint{1.818830in}{1.034267in}}%
\pgfpathlineto{\pgfqpoint{1.826932in}{1.029605in}}%
\pgfpathlineto{\pgfqpoint{1.835034in}{1.025647in}}%
\pgfpathlineto{\pgfqpoint{1.843136in}{1.020782in}}%
\pgfpathlineto{\pgfqpoint{1.851239in}{1.017008in}}%
\pgfpathlineto{\pgfqpoint{1.859341in}{1.021200in}}%
\pgfpathlineto{\pgfqpoint{1.867443in}{1.021156in}}%
\pgfpathlineto{\pgfqpoint{1.875545in}{1.016255in}}%
\pgfpathlineto{\pgfqpoint{1.883648in}{1.012520in}}%
\pgfpathlineto{\pgfqpoint{1.891750in}{1.008037in}}%
\pgfpathlineto{\pgfqpoint{1.899852in}{1.003301in}}%
\pgfpathlineto{\pgfqpoint{1.907955in}{0.998507in}}%
\pgfpathlineto{\pgfqpoint{1.916057in}{0.997660in}}%
\pgfpathlineto{\pgfqpoint{1.924159in}{0.993908in}}%
\pgfpathlineto{\pgfqpoint{1.932261in}{0.990019in}}%
\pgfpathlineto{\pgfqpoint{1.940364in}{0.992710in}}%
\pgfpathlineto{\pgfqpoint{1.948466in}{0.988897in}}%
\pgfpathlineto{\pgfqpoint{1.956568in}{0.992325in}}%
\pgfpathlineto{\pgfqpoint{1.964670in}{0.991901in}}%
\pgfpathlineto{\pgfqpoint{1.972773in}{0.989020in}}%
\pgfpathlineto{\pgfqpoint{1.980875in}{0.984002in}}%
\pgfpathlineto{\pgfqpoint{1.988977in}{0.981772in}}%
\pgfpathlineto{\pgfqpoint{1.988977in}{0.773281in}}%
\pgfpathlineto{\pgfqpoint{1.988977in}{0.773281in}}%
\pgfpathlineto{\pgfqpoint{1.980875in}{0.777194in}}%
\pgfpathlineto{\pgfqpoint{1.972773in}{0.770102in}}%
\pgfpathlineto{\pgfqpoint{1.964670in}{0.774806in}}%
\pgfpathlineto{\pgfqpoint{1.956568in}{0.777659in}}%
\pgfpathlineto{\pgfqpoint{1.948466in}{0.780377in}}%
\pgfpathlineto{\pgfqpoint{1.940364in}{0.785025in}}%
\pgfpathlineto{\pgfqpoint{1.932261in}{0.789456in}}%
\pgfpathlineto{\pgfqpoint{1.924159in}{0.791209in}}%
\pgfpathlineto{\pgfqpoint{1.916057in}{0.789991in}}%
\pgfpathlineto{\pgfqpoint{1.907955in}{0.781904in}}%
\pgfpathlineto{\pgfqpoint{1.899852in}{0.786655in}}%
\pgfpathlineto{\pgfqpoint{1.891750in}{0.791342in}}%
\pgfpathlineto{\pgfqpoint{1.883648in}{0.794078in}}%
\pgfpathlineto{\pgfqpoint{1.875545in}{0.797457in}}%
\pgfpathlineto{\pgfqpoint{1.867443in}{0.802316in}}%
\pgfpathlineto{\pgfqpoint{1.859341in}{0.806773in}}%
\pgfpathlineto{\pgfqpoint{1.851239in}{0.811375in}}%
\pgfpathlineto{\pgfqpoint{1.843136in}{0.816433in}}%
\pgfpathlineto{\pgfqpoint{1.835034in}{0.821281in}}%
\pgfpathlineto{\pgfqpoint{1.826932in}{0.823200in}}%
\pgfpathlineto{\pgfqpoint{1.818830in}{0.828139in}}%
\pgfpathlineto{\pgfqpoint{1.810727in}{0.833035in}}%
\pgfpathlineto{\pgfqpoint{1.802625in}{0.837825in}}%
\pgfpathlineto{\pgfqpoint{1.794523in}{0.842617in}}%
\pgfpathlineto{\pgfqpoint{1.786420in}{0.846898in}}%
\pgfpathlineto{\pgfqpoint{1.778318in}{0.851347in}}%
\pgfpathlineto{\pgfqpoint{1.770216in}{0.856633in}}%
\pgfpathlineto{\pgfqpoint{1.762114in}{0.858430in}}%
\pgfpathlineto{\pgfqpoint{1.754011in}{0.862989in}}%
\pgfpathlineto{\pgfqpoint{1.745909in}{0.864373in}}%
\pgfpathlineto{\pgfqpoint{1.737807in}{0.869247in}}%
\pgfpathlineto{\pgfqpoint{1.729705in}{0.871143in}}%
\pgfpathlineto{\pgfqpoint{1.721602in}{0.874385in}}%
\pgfpathlineto{\pgfqpoint{1.713500in}{0.871831in}}%
\pgfpathlineto{\pgfqpoint{1.705398in}{0.877145in}}%
\pgfpathlineto{\pgfqpoint{1.697295in}{0.882387in}}%
\pgfpathlineto{\pgfqpoint{1.689193in}{0.887446in}}%
\pgfpathlineto{\pgfqpoint{1.681091in}{0.892273in}}%
\pgfpathlineto{\pgfqpoint{1.672989in}{0.894629in}}%
\pgfpathlineto{\pgfqpoint{1.664886in}{0.900181in}}%
\pgfpathlineto{\pgfqpoint{1.656784in}{0.900265in}}%
\pgfpathlineto{\pgfqpoint{1.648682in}{0.905132in}}%
\pgfpathlineto{\pgfqpoint{1.640580in}{0.909815in}}%
\pgfpathlineto{\pgfqpoint{1.632477in}{0.915359in}}%
\pgfpathlineto{\pgfqpoint{1.624375in}{0.919037in}}%
\pgfpathlineto{\pgfqpoint{1.616273in}{0.924620in}}%
\pgfpathlineto{\pgfqpoint{1.608170in}{0.929097in}}%
\pgfpathlineto{\pgfqpoint{1.600068in}{0.935052in}}%
\pgfpathlineto{\pgfqpoint{1.591966in}{0.940877in}}%
\pgfpathlineto{\pgfqpoint{1.583864in}{0.945430in}}%
\pgfpathlineto{\pgfqpoint{1.575761in}{0.943118in}}%
\pgfpathlineto{\pgfqpoint{1.567659in}{0.946925in}}%
\pgfpathlineto{\pgfqpoint{1.559557in}{0.952757in}}%
\pgfpathlineto{\pgfqpoint{1.551455in}{0.958961in}}%
\pgfpathlineto{\pgfqpoint{1.543352in}{0.963746in}}%
\pgfpathlineto{\pgfqpoint{1.535250in}{0.967839in}}%
\pgfpathlineto{\pgfqpoint{1.527148in}{0.963914in}}%
\pgfpathlineto{\pgfqpoint{1.519045in}{0.966287in}}%
\pgfpathlineto{\pgfqpoint{1.510943in}{0.967909in}}%
\pgfpathlineto{\pgfqpoint{1.502841in}{0.990397in}}%
\pgfpathlineto{\pgfqpoint{1.494739in}{0.996551in}}%
\pgfpathlineto{\pgfqpoint{1.486636in}{1.002995in}}%
\pgfpathlineto{\pgfqpoint{1.478534in}{1.003493in}}%
\pgfpathlineto{\pgfqpoint{1.470432in}{1.010216in}}%
\pgfpathlineto{\pgfqpoint{1.462330in}{1.016990in}}%
\pgfpathlineto{\pgfqpoint{1.454227in}{1.023041in}}%
\pgfpathlineto{\pgfqpoint{1.446125in}{1.027010in}}%
\pgfpathlineto{\pgfqpoint{1.438023in}{1.031927in}}%
\pgfpathlineto{\pgfqpoint{1.429920in}{1.037295in}}%
\pgfpathlineto{\pgfqpoint{1.421818in}{1.028328in}}%
\pgfpathlineto{\pgfqpoint{1.413716in}{1.033814in}}%
\pgfpathlineto{\pgfqpoint{1.405614in}{1.040092in}}%
\pgfpathlineto{\pgfqpoint{1.397511in}{1.043627in}}%
\pgfpathlineto{\pgfqpoint{1.389409in}{1.048962in}}%
\pgfpathlineto{\pgfqpoint{1.381307in}{1.051303in}}%
\pgfpathlineto{\pgfqpoint{1.373205in}{1.058810in}}%
\pgfpathlineto{\pgfqpoint{1.365102in}{1.063274in}}%
\pgfpathlineto{\pgfqpoint{1.357000in}{1.071128in}}%
\pgfpathlineto{\pgfqpoint{1.348898in}{1.078450in}}%
\pgfpathlineto{\pgfqpoint{1.340795in}{1.081596in}}%
\pgfpathlineto{\pgfqpoint{1.332693in}{1.082058in}}%
\pgfpathlineto{\pgfqpoint{1.324591in}{1.087807in}}%
\pgfpathlineto{\pgfqpoint{1.316489in}{1.092544in}}%
\pgfpathlineto{\pgfqpoint{1.308386in}{1.100327in}}%
\pgfpathlineto{\pgfqpoint{1.300284in}{1.107487in}}%
\pgfpathlineto{\pgfqpoint{1.292182in}{1.114272in}}%
\pgfpathlineto{\pgfqpoint{1.284080in}{1.117190in}}%
\pgfpathlineto{\pgfqpoint{1.275977in}{1.124453in}}%
\pgfpathlineto{\pgfqpoint{1.267875in}{1.118369in}}%
\pgfpathlineto{\pgfqpoint{1.259773in}{1.123723in}}%
\pgfpathlineto{\pgfqpoint{1.251670in}{1.131665in}}%
\pgfpathlineto{\pgfqpoint{1.243568in}{1.135366in}}%
\pgfpathlineto{\pgfqpoint{1.235466in}{1.143506in}}%
\pgfpathlineto{\pgfqpoint{1.227364in}{1.149521in}}%
\pgfpathlineto{\pgfqpoint{1.219261in}{1.158357in}}%
\pgfpathlineto{\pgfqpoint{1.211159in}{1.164411in}}%
\pgfpathlineto{\pgfqpoint{1.203057in}{1.172303in}}%
\pgfpathlineto{\pgfqpoint{1.194955in}{1.177170in}}%
\pgfpathlineto{\pgfqpoint{1.186852in}{1.181342in}}%
\pgfpathlineto{\pgfqpoint{1.178750in}{1.189296in}}%
\pgfpathlineto{\pgfqpoint{1.170648in}{1.195730in}}%
\pgfpathlineto{\pgfqpoint{1.162545in}{1.204311in}}%
\pgfpathlineto{\pgfqpoint{1.154443in}{1.213722in}}%
\pgfpathlineto{\pgfqpoint{1.146341in}{1.217023in}}%
\pgfpathlineto{\pgfqpoint{1.138239in}{1.224562in}}%
\pgfpathlineto{\pgfqpoint{1.130136in}{1.228599in}}%
\pgfpathlineto{\pgfqpoint{1.122034in}{1.235317in}}%
\pgfpathlineto{\pgfqpoint{1.113932in}{1.242661in}}%
\pgfpathlineto{\pgfqpoint{1.105830in}{1.249340in}}%
\pgfpathlineto{\pgfqpoint{1.097727in}{1.260188in}}%
\pgfpathlineto{\pgfqpoint{1.089625in}{1.264939in}}%
\pgfpathlineto{\pgfqpoint{1.081523in}{1.272983in}}%
\pgfpathlineto{\pgfqpoint{1.073420in}{1.279943in}}%
\pgfpathlineto{\pgfqpoint{1.065318in}{1.287062in}}%
\pgfpathlineto{\pgfqpoint{1.057216in}{1.297107in}}%
\pgfpathlineto{\pgfqpoint{1.049114in}{1.307608in}}%
\pgfpathlineto{\pgfqpoint{1.041011in}{1.315108in}}%
\pgfpathlineto{\pgfqpoint{1.032909in}{1.323310in}}%
\pgfpathlineto{\pgfqpoint{1.024807in}{1.335658in}}%
\pgfpathlineto{\pgfqpoint{1.016705in}{1.341654in}}%
\pgfpathlineto{\pgfqpoint{1.008602in}{1.352509in}}%
\pgfpathlineto{\pgfqpoint{1.000500in}{1.359803in}}%
\pgfpathlineto{\pgfqpoint{0.992398in}{1.370329in}}%
\pgfpathlineto{\pgfqpoint{0.984295in}{1.376118in}}%
\pgfpathlineto{\pgfqpoint{0.976193in}{1.387725in}}%
\pgfpathlineto{\pgfqpoint{0.968091in}{1.389876in}}%
\pgfpathlineto{\pgfqpoint{0.959989in}{1.400671in}}%
\pgfpathlineto{\pgfqpoint{0.951886in}{1.401919in}}%
\pgfpathlineto{\pgfqpoint{0.943784in}{1.405528in}}%
\pgfpathlineto{\pgfqpoint{0.935682in}{1.411018in}}%
\pgfpathlineto{\pgfqpoint{0.927580in}{1.417758in}}%
\pgfpathlineto{\pgfqpoint{0.919477in}{1.426013in}}%
\pgfpathlineto{\pgfqpoint{0.911375in}{1.433276in}}%
\pgfpathlineto{\pgfqpoint{0.903273in}{1.442904in}}%
\pgfpathlineto{\pgfqpoint{0.895170in}{1.456218in}}%
\pgfpathlineto{\pgfqpoint{0.887068in}{1.472182in}}%
\pgfpathlineto{\pgfqpoint{0.878966in}{1.483900in}}%
\pgfpathlineto{\pgfqpoint{0.870864in}{1.488818in}}%
\pgfpathlineto{\pgfqpoint{0.862761in}{1.498110in}}%
\pgfpathlineto{\pgfqpoint{0.854659in}{1.503881in}}%
\pgfpathlineto{\pgfqpoint{0.846557in}{1.513206in}}%
\pgfpathlineto{\pgfqpoint{0.838455in}{1.514370in}}%
\pgfpathlineto{\pgfqpoint{0.830352in}{1.523817in}}%
\pgfpathlineto{\pgfqpoint{0.822250in}{1.530428in}}%
\pgfpathlineto{\pgfqpoint{0.814148in}{1.528635in}}%
\pgfpathlineto{\pgfqpoint{0.806045in}{1.533584in}}%
\pgfpathlineto{\pgfqpoint{0.797943in}{1.542045in}}%
\pgfpathlineto{\pgfqpoint{0.789841in}{1.547364in}}%
\pgfpathlineto{\pgfqpoint{0.781739in}{1.555198in}}%
\pgfpathlineto{\pgfqpoint{0.773636in}{1.567571in}}%
\pgfpathlineto{\pgfqpoint{0.765534in}{1.575690in}}%
\pgfpathlineto{\pgfqpoint{0.757432in}{1.576976in}}%
\pgfpathlineto{\pgfqpoint{0.749330in}{1.587137in}}%
\pgfpathlineto{\pgfqpoint{0.741227in}{1.587253in}}%
\pgfpathlineto{\pgfqpoint{0.733125in}{1.594197in}}%
\pgfpathlineto{\pgfqpoint{0.725023in}{1.609300in}}%
\pgfpathlineto{\pgfqpoint{0.716920in}{1.605112in}}%
\pgfpathlineto{\pgfqpoint{0.708818in}{1.608482in}}%
\pgfpathlineto{\pgfqpoint{0.700716in}{1.605478in}}%
\pgfpathlineto{\pgfqpoint{0.692614in}{1.608475in}}%
\pgfpathlineto{\pgfqpoint{0.684511in}{1.601346in}}%
\pgfpathlineto{\pgfqpoint{0.676409in}{1.610758in}}%
\pgfpathlineto{\pgfqpoint{0.668307in}{1.612907in}}%
\pgfpathlineto{\pgfqpoint{0.660205in}{1.621274in}}%
\pgfpathlineto{\pgfqpoint{0.652102in}{1.619962in}}%
\pgfpathlineto{\pgfqpoint{0.644000in}{1.634188in}}%
\pgfpathlineto{\pgfqpoint{0.635898in}{1.629825in}}%
\pgfpathlineto{\pgfqpoint{0.627795in}{1.637721in}}%
\pgfpathlineto{\pgfqpoint{0.619693in}{1.628875in}}%
\pgfpathlineto{\pgfqpoint{0.611591in}{1.631585in}}%
\pgfpathlineto{\pgfqpoint{0.603489in}{1.633377in}}%
\pgfpathlineto{\pgfqpoint{0.595386in}{1.636583in}}%
\pgfpathlineto{\pgfqpoint{0.587284in}{1.637025in}}%
\pgfpathlineto{\pgfqpoint{0.579182in}{1.648197in}}%
\pgfpathlineto{\pgfqpoint{0.571080in}{1.648098in}}%
\pgfpathlineto{\pgfqpoint{0.562977in}{1.644163in}}%
\pgfpathlineto{\pgfqpoint{0.554875in}{1.639893in}}%
\pgfpathlineto{\pgfqpoint{0.546773in}{1.642835in}}%
\pgfpathlineto{\pgfqpoint{0.538670in}{1.636995in}}%
\pgfpathlineto{\pgfqpoint{0.530568in}{1.646432in}}%
\pgfpathlineto{\pgfqpoint{0.522466in}{1.645335in}}%
\pgfpathlineto{\pgfqpoint{0.514364in}{1.651616in}}%
\pgfpathlineto{\pgfqpoint{0.506261in}{1.650987in}}%
\pgfpathlineto{\pgfqpoint{0.498159in}{1.622753in}}%
\pgfpathlineto{\pgfqpoint{0.490057in}{1.635526in}}%
\pgfpathlineto{\pgfqpoint{0.481955in}{1.654387in}}%
\pgfpathlineto{\pgfqpoint{0.473852in}{1.661121in}}%
\pgfpathlineto{\pgfqpoint{0.465750in}{1.633360in}}%
\pgfpathlineto{\pgfqpoint{0.457648in}{1.651266in}}%
\pgfpathlineto{\pgfqpoint{0.449545in}{1.650274in}}%
\pgfpathlineto{\pgfqpoint{0.441443in}{1.660680in}}%
\pgfpathlineto{\pgfqpoint{0.433341in}{1.631969in}}%
\pgfpathlineto{\pgfqpoint{0.425239in}{1.622001in}}%
\pgfpathlineto{\pgfqpoint{0.417136in}{1.629782in}}%
\pgfpathlineto{\pgfqpoint{0.409034in}{1.621443in}}%
\pgfpathlineto{\pgfqpoint{0.400932in}{1.607802in}}%
\pgfpathlineto{\pgfqpoint{0.392830in}{1.558483in}}%
\pgfpathlineto{\pgfqpoint{0.384727in}{1.488807in}}%
\pgfpathlineto{\pgfqpoint{0.376625in}{1.483347in}}%
\pgfpathlineto{\pgfqpoint{0.368523in}{0.810342in}}%
\pgfpathclose%
\pgfusepath{fill}%
\end{pgfscope}%
\begin{pgfscope}%
\pgfpathrectangle{\pgfqpoint{0.287500in}{0.375000in}}{\pgfqpoint{1.782500in}{2.265000in}}%
\pgfusepath{clip}%
\pgfsetbuttcap%
\pgfsetroundjoin%
\definecolor{currentfill}{rgb}{0.172549,0.627451,0.172549}%
\pgfsetfillcolor{currentfill}%
\pgfsetfillopacity{0.200000}%
\pgfsetlinewidth{0.000000pt}%
\definecolor{currentstroke}{rgb}{0.000000,0.000000,0.000000}%
\pgfsetstrokecolor{currentstroke}%
\pgfsetdash{}{0pt}%
\pgfpathmoveto{\pgfqpoint{0.368523in}{1.908288in}}%
\pgfpathlineto{\pgfqpoint{0.368523in}{2.515739in}}%
\pgfpathlineto{\pgfqpoint{0.376625in}{2.266965in}}%
\pgfpathlineto{\pgfqpoint{0.384727in}{1.949859in}}%
\pgfpathlineto{\pgfqpoint{0.392830in}{1.910579in}}%
\pgfpathlineto{\pgfqpoint{0.400932in}{1.808147in}}%
\pgfpathlineto{\pgfqpoint{0.409034in}{1.708618in}}%
\pgfpathlineto{\pgfqpoint{0.417136in}{1.672173in}}%
\pgfpathlineto{\pgfqpoint{0.425239in}{1.580783in}}%
\pgfpathlineto{\pgfqpoint{0.433341in}{1.500082in}}%
\pgfpathlineto{\pgfqpoint{0.441443in}{1.495067in}}%
\pgfpathlineto{\pgfqpoint{0.449545in}{1.502387in}}%
\pgfpathlineto{\pgfqpoint{0.457648in}{1.438013in}}%
\pgfpathlineto{\pgfqpoint{0.465750in}{1.367839in}}%
\pgfpathlineto{\pgfqpoint{0.473852in}{1.299034in}}%
\pgfpathlineto{\pgfqpoint{0.481955in}{1.275440in}}%
\pgfpathlineto{\pgfqpoint{0.490057in}{1.304490in}}%
\pgfpathlineto{\pgfqpoint{0.498159in}{1.291451in}}%
\pgfpathlineto{\pgfqpoint{0.506261in}{1.277496in}}%
\pgfpathlineto{\pgfqpoint{0.514364in}{1.268223in}}%
\pgfpathlineto{\pgfqpoint{0.522466in}{1.222670in}}%
\pgfpathlineto{\pgfqpoint{0.530568in}{1.184541in}}%
\pgfpathlineto{\pgfqpoint{0.538670in}{1.181766in}}%
\pgfpathlineto{\pgfqpoint{0.546773in}{1.150880in}}%
\pgfpathlineto{\pgfqpoint{0.554875in}{1.129675in}}%
\pgfpathlineto{\pgfqpoint{0.562977in}{1.142801in}}%
\pgfpathlineto{\pgfqpoint{0.571080in}{1.135501in}}%
\pgfpathlineto{\pgfqpoint{0.579182in}{1.139068in}}%
\pgfpathlineto{\pgfqpoint{0.587284in}{1.194975in}}%
\pgfpathlineto{\pgfqpoint{0.595386in}{1.161875in}}%
\pgfpathlineto{\pgfqpoint{0.603489in}{1.150272in}}%
\pgfpathlineto{\pgfqpoint{0.611591in}{1.201941in}}%
\pgfpathlineto{\pgfqpoint{0.619693in}{1.187893in}}%
\pgfpathlineto{\pgfqpoint{0.627795in}{1.198569in}}%
\pgfpathlineto{\pgfqpoint{0.635898in}{1.171900in}}%
\pgfpathlineto{\pgfqpoint{0.644000in}{1.161790in}}%
\pgfpathlineto{\pgfqpoint{0.652102in}{1.169801in}}%
\pgfpathlineto{\pgfqpoint{0.660205in}{1.167348in}}%
\pgfpathlineto{\pgfqpoint{0.668307in}{1.193246in}}%
\pgfpathlineto{\pgfqpoint{0.676409in}{1.187198in}}%
\pgfpathlineto{\pgfqpoint{0.684511in}{1.184233in}}%
\pgfpathlineto{\pgfqpoint{0.692614in}{1.201089in}}%
\pgfpathlineto{\pgfqpoint{0.700716in}{1.218445in}}%
\pgfpathlineto{\pgfqpoint{0.708818in}{1.211388in}}%
\pgfpathlineto{\pgfqpoint{0.716920in}{1.204384in}}%
\pgfpathlineto{\pgfqpoint{0.725023in}{1.211683in}}%
\pgfpathlineto{\pgfqpoint{0.733125in}{1.211009in}}%
\pgfpathlineto{\pgfqpoint{0.741227in}{1.201455in}}%
\pgfpathlineto{\pgfqpoint{0.749330in}{1.219899in}}%
\pgfpathlineto{\pgfqpoint{0.757432in}{1.241870in}}%
\pgfpathlineto{\pgfqpoint{0.765534in}{1.246709in}}%
\pgfpathlineto{\pgfqpoint{0.773636in}{1.242494in}}%
\pgfpathlineto{\pgfqpoint{0.781739in}{1.225284in}}%
\pgfpathlineto{\pgfqpoint{0.789841in}{1.227622in}}%
\pgfpathlineto{\pgfqpoint{0.797943in}{1.227848in}}%
\pgfpathlineto{\pgfqpoint{0.806045in}{1.218721in}}%
\pgfpathlineto{\pgfqpoint{0.814148in}{1.215093in}}%
\pgfpathlineto{\pgfqpoint{0.822250in}{1.227832in}}%
\pgfpathlineto{\pgfqpoint{0.830352in}{1.236770in}}%
\pgfpathlineto{\pgfqpoint{0.838455in}{1.228905in}}%
\pgfpathlineto{\pgfqpoint{0.846557in}{1.220526in}}%
\pgfpathlineto{\pgfqpoint{0.854659in}{1.219013in}}%
\pgfpathlineto{\pgfqpoint{0.862761in}{1.237981in}}%
\pgfpathlineto{\pgfqpoint{0.870864in}{1.238023in}}%
\pgfpathlineto{\pgfqpoint{0.878966in}{1.230121in}}%
\pgfpathlineto{\pgfqpoint{0.887068in}{1.227015in}}%
\pgfpathlineto{\pgfqpoint{0.895170in}{1.231898in}}%
\pgfpathlineto{\pgfqpoint{0.903273in}{1.217395in}}%
\pgfpathlineto{\pgfqpoint{0.911375in}{1.215094in}}%
\pgfpathlineto{\pgfqpoint{0.919477in}{1.219810in}}%
\pgfpathlineto{\pgfqpoint{0.927580in}{1.225498in}}%
\pgfpathlineto{\pgfqpoint{0.935682in}{1.239692in}}%
\pgfpathlineto{\pgfqpoint{0.943784in}{1.230128in}}%
\pgfpathlineto{\pgfqpoint{0.951886in}{1.223254in}}%
\pgfpathlineto{\pgfqpoint{0.959989in}{1.225375in}}%
\pgfpathlineto{\pgfqpoint{0.968091in}{1.230183in}}%
\pgfpathlineto{\pgfqpoint{0.976193in}{1.219322in}}%
\pgfpathlineto{\pgfqpoint{0.984295in}{1.210355in}}%
\pgfpathlineto{\pgfqpoint{0.992398in}{1.215189in}}%
\pgfpathlineto{\pgfqpoint{1.000500in}{1.215341in}}%
\pgfpathlineto{\pgfqpoint{1.008602in}{1.217279in}}%
\pgfpathlineto{\pgfqpoint{1.016705in}{1.208313in}}%
\pgfpathlineto{\pgfqpoint{1.024807in}{1.219056in}}%
\pgfpathlineto{\pgfqpoint{1.032909in}{1.220736in}}%
\pgfpathlineto{\pgfqpoint{1.041011in}{1.215312in}}%
\pgfpathlineto{\pgfqpoint{1.049114in}{1.212895in}}%
\pgfpathlineto{\pgfqpoint{1.057216in}{1.206517in}}%
\pgfpathlineto{\pgfqpoint{1.065318in}{1.200980in}}%
\pgfpathlineto{\pgfqpoint{1.073420in}{1.201104in}}%
\pgfpathlineto{\pgfqpoint{1.081523in}{1.195495in}}%
\pgfpathlineto{\pgfqpoint{1.089625in}{1.198511in}}%
\pgfpathlineto{\pgfqpoint{1.097727in}{1.205721in}}%
\pgfpathlineto{\pgfqpoint{1.105830in}{1.205628in}}%
\pgfpathlineto{\pgfqpoint{1.113932in}{1.202766in}}%
\pgfpathlineto{\pgfqpoint{1.122034in}{1.198756in}}%
\pgfpathlineto{\pgfqpoint{1.130136in}{1.199122in}}%
\pgfpathlineto{\pgfqpoint{1.138239in}{1.202799in}}%
\pgfpathlineto{\pgfqpoint{1.146341in}{1.205867in}}%
\pgfpathlineto{\pgfqpoint{1.154443in}{1.200831in}}%
\pgfpathlineto{\pgfqpoint{1.162545in}{1.196406in}}%
\pgfpathlineto{\pgfqpoint{1.170648in}{1.191803in}}%
\pgfpathlineto{\pgfqpoint{1.178750in}{1.192299in}}%
\pgfpathlineto{\pgfqpoint{1.186852in}{1.198998in}}%
\pgfpathlineto{\pgfqpoint{1.194955in}{1.205484in}}%
\pgfpathlineto{\pgfqpoint{1.203057in}{1.201699in}}%
\pgfpathlineto{\pgfqpoint{1.211159in}{1.193636in}}%
\pgfpathlineto{\pgfqpoint{1.219261in}{1.193709in}}%
\pgfpathlineto{\pgfqpoint{1.227364in}{1.194224in}}%
\pgfpathlineto{\pgfqpoint{1.235466in}{1.188176in}}%
\pgfpathlineto{\pgfqpoint{1.243568in}{1.181361in}}%
\pgfpathlineto{\pgfqpoint{1.251670in}{1.176692in}}%
\pgfpathlineto{\pgfqpoint{1.259773in}{1.168283in}}%
\pgfpathlineto{\pgfqpoint{1.267875in}{1.166800in}}%
\pgfpathlineto{\pgfqpoint{1.275977in}{1.167132in}}%
\pgfpathlineto{\pgfqpoint{1.284080in}{1.162025in}}%
\pgfpathlineto{\pgfqpoint{1.292182in}{1.157654in}}%
\pgfpathlineto{\pgfqpoint{1.300284in}{1.156910in}}%
\pgfpathlineto{\pgfqpoint{1.308386in}{1.155203in}}%
\pgfpathlineto{\pgfqpoint{1.316489in}{1.151702in}}%
\pgfpathlineto{\pgfqpoint{1.324591in}{1.149452in}}%
\pgfpathlineto{\pgfqpoint{1.332693in}{1.144144in}}%
\pgfpathlineto{\pgfqpoint{1.340795in}{1.142684in}}%
\pgfpathlineto{\pgfqpoint{1.348898in}{1.138772in}}%
\pgfpathlineto{\pgfqpoint{1.357000in}{1.145394in}}%
\pgfpathlineto{\pgfqpoint{1.365102in}{1.150798in}}%
\pgfpathlineto{\pgfqpoint{1.373205in}{1.147041in}}%
\pgfpathlineto{\pgfqpoint{1.381307in}{1.149745in}}%
\pgfpathlineto{\pgfqpoint{1.389409in}{1.151679in}}%
\pgfpathlineto{\pgfqpoint{1.397511in}{1.156468in}}%
\pgfpathlineto{\pgfqpoint{1.405614in}{1.159678in}}%
\pgfpathlineto{\pgfqpoint{1.413716in}{1.165679in}}%
\pgfpathlineto{\pgfqpoint{1.421818in}{1.171058in}}%
\pgfpathlineto{\pgfqpoint{1.429920in}{1.170051in}}%
\pgfpathlineto{\pgfqpoint{1.438023in}{1.173851in}}%
\pgfpathlineto{\pgfqpoint{1.446125in}{1.167855in}}%
\pgfpathlineto{\pgfqpoint{1.454227in}{1.164365in}}%
\pgfpathlineto{\pgfqpoint{1.462330in}{1.160779in}}%
\pgfpathlineto{\pgfqpoint{1.470432in}{1.160853in}}%
\pgfpathlineto{\pgfqpoint{1.478534in}{1.157689in}}%
\pgfpathlineto{\pgfqpoint{1.486636in}{1.154323in}}%
\pgfpathlineto{\pgfqpoint{1.494739in}{1.155743in}}%
\pgfpathlineto{\pgfqpoint{1.502841in}{1.160678in}}%
\pgfpathlineto{\pgfqpoint{1.510943in}{1.158068in}}%
\pgfpathlineto{\pgfqpoint{1.519045in}{1.157567in}}%
\pgfpathlineto{\pgfqpoint{1.527148in}{1.164060in}}%
\pgfpathlineto{\pgfqpoint{1.535250in}{1.159713in}}%
\pgfpathlineto{\pgfqpoint{1.543352in}{1.156750in}}%
\pgfpathlineto{\pgfqpoint{1.551455in}{1.152639in}}%
\pgfpathlineto{\pgfqpoint{1.559557in}{1.148843in}}%
\pgfpathlineto{\pgfqpoint{1.567659in}{1.143583in}}%
\pgfpathlineto{\pgfqpoint{1.575761in}{1.147352in}}%
\pgfpathlineto{\pgfqpoint{1.583864in}{1.144094in}}%
\pgfpathlineto{\pgfqpoint{1.591966in}{1.143887in}}%
\pgfpathlineto{\pgfqpoint{1.600068in}{1.145706in}}%
\pgfpathlineto{\pgfqpoint{1.608170in}{1.144499in}}%
\pgfpathlineto{\pgfqpoint{1.616273in}{1.144214in}}%
\pgfpathlineto{\pgfqpoint{1.624375in}{1.143427in}}%
\pgfpathlineto{\pgfqpoint{1.632477in}{1.143631in}}%
\pgfpathlineto{\pgfqpoint{1.640580in}{1.146473in}}%
\pgfpathlineto{\pgfqpoint{1.648682in}{1.143669in}}%
\pgfpathlineto{\pgfqpoint{1.656784in}{1.150371in}}%
\pgfpathlineto{\pgfqpoint{1.664886in}{1.155258in}}%
\pgfpathlineto{\pgfqpoint{1.672989in}{1.162205in}}%
\pgfpathlineto{\pgfqpoint{1.681091in}{1.159480in}}%
\pgfpathlineto{\pgfqpoint{1.689193in}{1.161696in}}%
\pgfpathlineto{\pgfqpoint{1.697295in}{1.161072in}}%
\pgfpathlineto{\pgfqpoint{1.705398in}{1.157374in}}%
\pgfpathlineto{\pgfqpoint{1.713500in}{1.154371in}}%
\pgfpathlineto{\pgfqpoint{1.721602in}{1.157290in}}%
\pgfpathlineto{\pgfqpoint{1.729705in}{1.160705in}}%
\pgfpathlineto{\pgfqpoint{1.737807in}{1.157336in}}%
\pgfpathlineto{\pgfqpoint{1.745909in}{1.154956in}}%
\pgfpathlineto{\pgfqpoint{1.754011in}{1.154378in}}%
\pgfpathlineto{\pgfqpoint{1.762114in}{1.152234in}}%
\pgfpathlineto{\pgfqpoint{1.770216in}{1.156224in}}%
\pgfpathlineto{\pgfqpoint{1.778318in}{1.157539in}}%
\pgfpathlineto{\pgfqpoint{1.786420in}{1.159361in}}%
\pgfpathlineto{\pgfqpoint{1.794523in}{1.160274in}}%
\pgfpathlineto{\pgfqpoint{1.802625in}{1.167070in}}%
\pgfpathlineto{\pgfqpoint{1.810727in}{1.169067in}}%
\pgfpathlineto{\pgfqpoint{1.818830in}{1.176578in}}%
\pgfpathlineto{\pgfqpoint{1.826932in}{1.175467in}}%
\pgfpathlineto{\pgfqpoint{1.835034in}{1.178105in}}%
\pgfpathlineto{\pgfqpoint{1.843136in}{1.176329in}}%
\pgfpathlineto{\pgfqpoint{1.851239in}{1.181483in}}%
\pgfpathlineto{\pgfqpoint{1.859341in}{1.182814in}}%
\pgfpathlineto{\pgfqpoint{1.867443in}{1.179629in}}%
\pgfpathlineto{\pgfqpoint{1.875545in}{1.183772in}}%
\pgfpathlineto{\pgfqpoint{1.883648in}{1.190557in}}%
\pgfpathlineto{\pgfqpoint{1.891750in}{1.190065in}}%
\pgfpathlineto{\pgfqpoint{1.899852in}{1.190172in}}%
\pgfpathlineto{\pgfqpoint{1.907955in}{1.188415in}}%
\pgfpathlineto{\pgfqpoint{1.916057in}{1.187917in}}%
\pgfpathlineto{\pgfqpoint{1.924159in}{1.185211in}}%
\pgfpathlineto{\pgfqpoint{1.932261in}{1.184101in}}%
\pgfpathlineto{\pgfqpoint{1.940364in}{1.191003in}}%
\pgfpathlineto{\pgfqpoint{1.948466in}{1.193027in}}%
\pgfpathlineto{\pgfqpoint{1.956568in}{1.190774in}}%
\pgfpathlineto{\pgfqpoint{1.964670in}{1.189712in}}%
\pgfpathlineto{\pgfqpoint{1.972773in}{1.189139in}}%
\pgfpathlineto{\pgfqpoint{1.980875in}{1.193705in}}%
\pgfpathlineto{\pgfqpoint{1.988977in}{1.195935in}}%
\pgfpathlineto{\pgfqpoint{1.988977in}{1.037949in}}%
\pgfpathlineto{\pgfqpoint{1.988977in}{1.037949in}}%
\pgfpathlineto{\pgfqpoint{1.980875in}{1.037416in}}%
\pgfpathlineto{\pgfqpoint{1.972773in}{1.042139in}}%
\pgfpathlineto{\pgfqpoint{1.964670in}{1.042023in}}%
\pgfpathlineto{\pgfqpoint{1.956568in}{1.039951in}}%
\pgfpathlineto{\pgfqpoint{1.948466in}{1.037946in}}%
\pgfpathlineto{\pgfqpoint{1.940364in}{1.037054in}}%
\pgfpathlineto{\pgfqpoint{1.932261in}{1.032470in}}%
\pgfpathlineto{\pgfqpoint{1.924159in}{1.032643in}}%
\pgfpathlineto{\pgfqpoint{1.916057in}{1.033533in}}%
\pgfpathlineto{\pgfqpoint{1.907955in}{1.030904in}}%
\pgfpathlineto{\pgfqpoint{1.899852in}{1.026955in}}%
\pgfpathlineto{\pgfqpoint{1.891750in}{1.028164in}}%
\pgfpathlineto{\pgfqpoint{1.883648in}{1.028612in}}%
\pgfpathlineto{\pgfqpoint{1.875545in}{1.031036in}}%
\pgfpathlineto{\pgfqpoint{1.867443in}{1.032599in}}%
\pgfpathlineto{\pgfqpoint{1.859341in}{1.032269in}}%
\pgfpathlineto{\pgfqpoint{1.851239in}{1.030987in}}%
\pgfpathlineto{\pgfqpoint{1.843136in}{1.027795in}}%
\pgfpathlineto{\pgfqpoint{1.835034in}{1.030494in}}%
\pgfpathlineto{\pgfqpoint{1.826932in}{1.026258in}}%
\pgfpathlineto{\pgfqpoint{1.818830in}{1.019528in}}%
\pgfpathlineto{\pgfqpoint{1.810727in}{1.019837in}}%
\pgfpathlineto{\pgfqpoint{1.802625in}{1.014266in}}%
\pgfpathlineto{\pgfqpoint{1.794523in}{1.010085in}}%
\pgfpathlineto{\pgfqpoint{1.786420in}{1.005377in}}%
\pgfpathlineto{\pgfqpoint{1.778318in}{1.009493in}}%
\pgfpathlineto{\pgfqpoint{1.770216in}{1.014296in}}%
\pgfpathlineto{\pgfqpoint{1.762114in}{1.012924in}}%
\pgfpathlineto{\pgfqpoint{1.754011in}{1.011540in}}%
\pgfpathlineto{\pgfqpoint{1.745909in}{1.008564in}}%
\pgfpathlineto{\pgfqpoint{1.737807in}{1.010407in}}%
\pgfpathlineto{\pgfqpoint{1.729705in}{1.010562in}}%
\pgfpathlineto{\pgfqpoint{1.721602in}{1.016664in}}%
\pgfpathlineto{\pgfqpoint{1.713500in}{1.015747in}}%
\pgfpathlineto{\pgfqpoint{1.705398in}{1.016763in}}%
\pgfpathlineto{\pgfqpoint{1.697295in}{1.021194in}}%
\pgfpathlineto{\pgfqpoint{1.689193in}{1.019502in}}%
\pgfpathlineto{\pgfqpoint{1.681091in}{1.020178in}}%
\pgfpathlineto{\pgfqpoint{1.672989in}{1.024504in}}%
\pgfpathlineto{\pgfqpoint{1.664886in}{1.027034in}}%
\pgfpathlineto{\pgfqpoint{1.656784in}{1.027607in}}%
\pgfpathlineto{\pgfqpoint{1.648682in}{1.030226in}}%
\pgfpathlineto{\pgfqpoint{1.640580in}{1.025891in}}%
\pgfpathlineto{\pgfqpoint{1.632477in}{1.025820in}}%
\pgfpathlineto{\pgfqpoint{1.624375in}{1.026551in}}%
\pgfpathlineto{\pgfqpoint{1.616273in}{1.023173in}}%
\pgfpathlineto{\pgfqpoint{1.608170in}{1.023965in}}%
\pgfpathlineto{\pgfqpoint{1.600068in}{1.025816in}}%
\pgfpathlineto{\pgfqpoint{1.591966in}{1.021575in}}%
\pgfpathlineto{\pgfqpoint{1.583864in}{1.022320in}}%
\pgfpathlineto{\pgfqpoint{1.575761in}{1.018580in}}%
\pgfpathlineto{\pgfqpoint{1.567659in}{1.017305in}}%
\pgfpathlineto{\pgfqpoint{1.559557in}{1.021928in}}%
\pgfpathlineto{\pgfqpoint{1.551455in}{1.027644in}}%
\pgfpathlineto{\pgfqpoint{1.543352in}{1.032790in}}%
\pgfpathlineto{\pgfqpoint{1.535250in}{1.033113in}}%
\pgfpathlineto{\pgfqpoint{1.527148in}{1.036482in}}%
\pgfpathlineto{\pgfqpoint{1.519045in}{1.038235in}}%
\pgfpathlineto{\pgfqpoint{1.510943in}{1.034168in}}%
\pgfpathlineto{\pgfqpoint{1.502841in}{1.037947in}}%
\pgfpathlineto{\pgfqpoint{1.494739in}{1.045282in}}%
\pgfpathlineto{\pgfqpoint{1.486636in}{1.043891in}}%
\pgfpathlineto{\pgfqpoint{1.478534in}{1.044462in}}%
\pgfpathlineto{\pgfqpoint{1.470432in}{1.043912in}}%
\pgfpathlineto{\pgfqpoint{1.462330in}{1.043596in}}%
\pgfpathlineto{\pgfqpoint{1.454227in}{1.046660in}}%
\pgfpathlineto{\pgfqpoint{1.446125in}{1.042889in}}%
\pgfpathlineto{\pgfqpoint{1.438023in}{1.041491in}}%
\pgfpathlineto{\pgfqpoint{1.429920in}{1.048296in}}%
\pgfpathlineto{\pgfqpoint{1.421818in}{1.052951in}}%
\pgfpathlineto{\pgfqpoint{1.413716in}{1.051164in}}%
\pgfpathlineto{\pgfqpoint{1.405614in}{1.055024in}}%
\pgfpathlineto{\pgfqpoint{1.397511in}{1.056392in}}%
\pgfpathlineto{\pgfqpoint{1.389409in}{1.052416in}}%
\pgfpathlineto{\pgfqpoint{1.381307in}{1.054655in}}%
\pgfpathlineto{\pgfqpoint{1.373205in}{1.060606in}}%
\pgfpathlineto{\pgfqpoint{1.365102in}{1.065855in}}%
\pgfpathlineto{\pgfqpoint{1.357000in}{1.068576in}}%
\pgfpathlineto{\pgfqpoint{1.348898in}{1.064800in}}%
\pgfpathlineto{\pgfqpoint{1.340795in}{1.062471in}}%
\pgfpathlineto{\pgfqpoint{1.332693in}{1.063107in}}%
\pgfpathlineto{\pgfqpoint{1.324591in}{1.056823in}}%
\pgfpathlineto{\pgfqpoint{1.316489in}{1.056465in}}%
\pgfpathlineto{\pgfqpoint{1.308386in}{1.048848in}}%
\pgfpathlineto{\pgfqpoint{1.300284in}{1.050806in}}%
\pgfpathlineto{\pgfqpoint{1.292182in}{1.051012in}}%
\pgfpathlineto{\pgfqpoint{1.284080in}{1.056402in}}%
\pgfpathlineto{\pgfqpoint{1.275977in}{1.055735in}}%
\pgfpathlineto{\pgfqpoint{1.267875in}{1.062341in}}%
\pgfpathlineto{\pgfqpoint{1.259773in}{1.060974in}}%
\pgfpathlineto{\pgfqpoint{1.251670in}{1.050334in}}%
\pgfpathlineto{\pgfqpoint{1.243568in}{1.055984in}}%
\pgfpathlineto{\pgfqpoint{1.235466in}{1.052010in}}%
\pgfpathlineto{\pgfqpoint{1.227364in}{1.054964in}}%
\pgfpathlineto{\pgfqpoint{1.219261in}{1.049087in}}%
\pgfpathlineto{\pgfqpoint{1.211159in}{1.042620in}}%
\pgfpathlineto{\pgfqpoint{1.203057in}{1.034307in}}%
\pgfpathlineto{\pgfqpoint{1.194955in}{1.040025in}}%
\pgfpathlineto{\pgfqpoint{1.186852in}{1.040505in}}%
\pgfpathlineto{\pgfqpoint{1.178750in}{1.042490in}}%
\pgfpathlineto{\pgfqpoint{1.170648in}{1.046075in}}%
\pgfpathlineto{\pgfqpoint{1.162545in}{1.050872in}}%
\pgfpathlineto{\pgfqpoint{1.154443in}{1.047930in}}%
\pgfpathlineto{\pgfqpoint{1.146341in}{1.040287in}}%
\pgfpathlineto{\pgfqpoint{1.138239in}{1.041746in}}%
\pgfpathlineto{\pgfqpoint{1.130136in}{1.046084in}}%
\pgfpathlineto{\pgfqpoint{1.122034in}{1.048633in}}%
\pgfpathlineto{\pgfqpoint{1.113932in}{1.052579in}}%
\pgfpathlineto{\pgfqpoint{1.105830in}{1.055123in}}%
\pgfpathlineto{\pgfqpoint{1.097727in}{1.039149in}}%
\pgfpathlineto{\pgfqpoint{1.089625in}{1.040029in}}%
\pgfpathlineto{\pgfqpoint{1.081523in}{1.040509in}}%
\pgfpathlineto{\pgfqpoint{1.073420in}{1.041379in}}%
\pgfpathlineto{\pgfqpoint{1.065318in}{1.041186in}}%
\pgfpathlineto{\pgfqpoint{1.057216in}{1.037218in}}%
\pgfpathlineto{\pgfqpoint{1.049114in}{1.027708in}}%
\pgfpathlineto{\pgfqpoint{1.041011in}{1.033184in}}%
\pgfpathlineto{\pgfqpoint{1.032909in}{1.038581in}}%
\pgfpathlineto{\pgfqpoint{1.024807in}{1.038188in}}%
\pgfpathlineto{\pgfqpoint{1.016705in}{1.043605in}}%
\pgfpathlineto{\pgfqpoint{1.008602in}{1.045538in}}%
\pgfpathlineto{\pgfqpoint{1.000500in}{1.042184in}}%
\pgfpathlineto{\pgfqpoint{0.992398in}{1.050458in}}%
\pgfpathlineto{\pgfqpoint{0.984295in}{1.049719in}}%
\pgfpathlineto{\pgfqpoint{0.976193in}{1.057337in}}%
\pgfpathlineto{\pgfqpoint{0.968091in}{1.065145in}}%
\pgfpathlineto{\pgfqpoint{0.959989in}{1.070887in}}%
\pgfpathlineto{\pgfqpoint{0.951886in}{1.076081in}}%
\pgfpathlineto{\pgfqpoint{0.943784in}{1.068402in}}%
\pgfpathlineto{\pgfqpoint{0.935682in}{1.076861in}}%
\pgfpathlineto{\pgfqpoint{0.927580in}{1.070570in}}%
\pgfpathlineto{\pgfqpoint{0.919477in}{1.076735in}}%
\pgfpathlineto{\pgfqpoint{0.911375in}{1.080162in}}%
\pgfpathlineto{\pgfqpoint{0.903273in}{1.082953in}}%
\pgfpathlineto{\pgfqpoint{0.895170in}{1.080793in}}%
\pgfpathlineto{\pgfqpoint{0.887068in}{1.076806in}}%
\pgfpathlineto{\pgfqpoint{0.878966in}{1.075062in}}%
\pgfpathlineto{\pgfqpoint{0.870864in}{1.072504in}}%
\pgfpathlineto{\pgfqpoint{0.862761in}{1.071468in}}%
\pgfpathlineto{\pgfqpoint{0.854659in}{1.069628in}}%
\pgfpathlineto{\pgfqpoint{0.846557in}{1.056435in}}%
\pgfpathlineto{\pgfqpoint{0.838455in}{1.038765in}}%
\pgfpathlineto{\pgfqpoint{0.830352in}{1.032413in}}%
\pgfpathlineto{\pgfqpoint{0.822250in}{1.041653in}}%
\pgfpathlineto{\pgfqpoint{0.814148in}{1.052662in}}%
\pgfpathlineto{\pgfqpoint{0.806045in}{1.035167in}}%
\pgfpathlineto{\pgfqpoint{0.797943in}{1.028479in}}%
\pgfpathlineto{\pgfqpoint{0.789841in}{1.046707in}}%
\pgfpathlineto{\pgfqpoint{0.781739in}{1.042037in}}%
\pgfpathlineto{\pgfqpoint{0.773636in}{1.056330in}}%
\pgfpathlineto{\pgfqpoint{0.765534in}{1.052172in}}%
\pgfpathlineto{\pgfqpoint{0.757432in}{1.038881in}}%
\pgfpathlineto{\pgfqpoint{0.749330in}{1.028719in}}%
\pgfpathlineto{\pgfqpoint{0.741227in}{1.005135in}}%
\pgfpathlineto{\pgfqpoint{0.733125in}{1.001720in}}%
\pgfpathlineto{\pgfqpoint{0.725023in}{1.012288in}}%
\pgfpathlineto{\pgfqpoint{0.716920in}{0.983127in}}%
\pgfpathlineto{\pgfqpoint{0.708818in}{0.941826in}}%
\pgfpathlineto{\pgfqpoint{0.700716in}{0.942824in}}%
\pgfpathlineto{\pgfqpoint{0.692614in}{0.951127in}}%
\pgfpathlineto{\pgfqpoint{0.684511in}{0.946099in}}%
\pgfpathlineto{\pgfqpoint{0.676409in}{0.956563in}}%
\pgfpathlineto{\pgfqpoint{0.668307in}{0.968908in}}%
\pgfpathlineto{\pgfqpoint{0.660205in}{0.945973in}}%
\pgfpathlineto{\pgfqpoint{0.652102in}{0.939846in}}%
\pgfpathlineto{\pgfqpoint{0.644000in}{0.944333in}}%
\pgfpathlineto{\pgfqpoint{0.635898in}{0.958928in}}%
\pgfpathlineto{\pgfqpoint{0.627795in}{0.975428in}}%
\pgfpathlineto{\pgfqpoint{0.619693in}{0.983319in}}%
\pgfpathlineto{\pgfqpoint{0.611591in}{0.934887in}}%
\pgfpathlineto{\pgfqpoint{0.603489in}{0.927264in}}%
\pgfpathlineto{\pgfqpoint{0.595386in}{0.945381in}}%
\pgfpathlineto{\pgfqpoint{0.587284in}{0.963315in}}%
\pgfpathlineto{\pgfqpoint{0.579182in}{0.951519in}}%
\pgfpathlineto{\pgfqpoint{0.571080in}{0.953933in}}%
\pgfpathlineto{\pgfqpoint{0.562977in}{0.971824in}}%
\pgfpathlineto{\pgfqpoint{0.554875in}{1.000410in}}%
\pgfpathlineto{\pgfqpoint{0.546773in}{0.983490in}}%
\pgfpathlineto{\pgfqpoint{0.538670in}{0.996208in}}%
\pgfpathlineto{\pgfqpoint{0.530568in}{0.993791in}}%
\pgfpathlineto{\pgfqpoint{0.522466in}{1.021642in}}%
\pgfpathlineto{\pgfqpoint{0.514364in}{1.049667in}}%
\pgfpathlineto{\pgfqpoint{0.506261in}{1.075475in}}%
\pgfpathlineto{\pgfqpoint{0.498159in}{1.102647in}}%
\pgfpathlineto{\pgfqpoint{0.490057in}{1.072114in}}%
\pgfpathlineto{\pgfqpoint{0.481955in}{1.049738in}}%
\pgfpathlineto{\pgfqpoint{0.473852in}{1.021972in}}%
\pgfpathlineto{\pgfqpoint{0.465750in}{1.012353in}}%
\pgfpathlineto{\pgfqpoint{0.457648in}{1.028877in}}%
\pgfpathlineto{\pgfqpoint{0.449545in}{1.062023in}}%
\pgfpathlineto{\pgfqpoint{0.441443in}{1.069302in}}%
\pgfpathlineto{\pgfqpoint{0.433341in}{1.098418in}}%
\pgfpathlineto{\pgfqpoint{0.425239in}{1.176292in}}%
\pgfpathlineto{\pgfqpoint{0.417136in}{1.258436in}}%
\pgfpathlineto{\pgfqpoint{0.409034in}{1.306757in}}%
\pgfpathlineto{\pgfqpoint{0.400932in}{1.288369in}}%
\pgfpathlineto{\pgfqpoint{0.392830in}{1.360104in}}%
\pgfpathlineto{\pgfqpoint{0.384727in}{1.558862in}}%
\pgfpathlineto{\pgfqpoint{0.376625in}{1.654697in}}%
\pgfpathlineto{\pgfqpoint{0.368523in}{1.908288in}}%
\pgfpathclose%
\pgfusepath{fill}%
\end{pgfscope}%
\begin{pgfscope}%
\pgfpathrectangle{\pgfqpoint{0.287500in}{0.375000in}}{\pgfqpoint{1.782500in}{2.265000in}}%
\pgfusepath{clip}%
\pgfsetroundcap%
\pgfsetroundjoin%
\pgfsetlinewidth{1.505625pt}%
\definecolor{currentstroke}{rgb}{0.121569,0.466667,0.705882}%
\pgfsetstrokecolor{currentstroke}%
\pgfsetdash{}{0pt}%
\pgfpathmoveto{\pgfqpoint{0.368523in}{2.088466in}}%
\pgfpathlineto{\pgfqpoint{0.376625in}{1.952048in}}%
\pgfpathlineto{\pgfqpoint{0.384727in}{1.947940in}}%
\pgfpathlineto{\pgfqpoint{0.392830in}{1.869323in}}%
\pgfpathlineto{\pgfqpoint{0.400932in}{1.820932in}}%
\pgfpathlineto{\pgfqpoint{0.409034in}{1.821796in}}%
\pgfpathlineto{\pgfqpoint{0.417136in}{1.808910in}}%
\pgfpathlineto{\pgfqpoint{0.433341in}{1.819241in}}%
\pgfpathlineto{\pgfqpoint{0.449545in}{1.777833in}}%
\pgfpathlineto{\pgfqpoint{0.457648in}{1.765111in}}%
\pgfpathlineto{\pgfqpoint{0.465750in}{1.741524in}}%
\pgfpathlineto{\pgfqpoint{0.473852in}{1.742538in}}%
\pgfpathlineto{\pgfqpoint{0.481955in}{1.711055in}}%
\pgfpathlineto{\pgfqpoint{0.490057in}{1.702753in}}%
\pgfpathlineto{\pgfqpoint{0.498159in}{1.675705in}}%
\pgfpathlineto{\pgfqpoint{0.506261in}{1.681517in}}%
\pgfpathlineto{\pgfqpoint{0.514364in}{1.665742in}}%
\pgfpathlineto{\pgfqpoint{0.522466in}{1.663235in}}%
\pgfpathlineto{\pgfqpoint{0.538670in}{1.675365in}}%
\pgfpathlineto{\pgfqpoint{0.546773in}{1.669100in}}%
\pgfpathlineto{\pgfqpoint{0.554875in}{1.654191in}}%
\pgfpathlineto{\pgfqpoint{0.562977in}{1.656478in}}%
\pgfpathlineto{\pgfqpoint{0.571080in}{1.646642in}}%
\pgfpathlineto{\pgfqpoint{0.579182in}{1.654505in}}%
\pgfpathlineto{\pgfqpoint{0.587284in}{1.646451in}}%
\pgfpathlineto{\pgfqpoint{0.595386in}{1.650160in}}%
\pgfpathlineto{\pgfqpoint{0.603489in}{1.639873in}}%
\pgfpathlineto{\pgfqpoint{0.611591in}{1.644392in}}%
\pgfpathlineto{\pgfqpoint{0.619693in}{1.650702in}}%
\pgfpathlineto{\pgfqpoint{0.627795in}{1.650249in}}%
\pgfpathlineto{\pgfqpoint{0.635898in}{1.652253in}}%
\pgfpathlineto{\pgfqpoint{0.644000in}{1.645545in}}%
\pgfpathlineto{\pgfqpoint{0.652102in}{1.635954in}}%
\pgfpathlineto{\pgfqpoint{0.660205in}{1.629209in}}%
\pgfpathlineto{\pgfqpoint{0.668307in}{1.630126in}}%
\pgfpathlineto{\pgfqpoint{0.684511in}{1.613595in}}%
\pgfpathlineto{\pgfqpoint{0.700716in}{1.609550in}}%
\pgfpathlineto{\pgfqpoint{0.708818in}{1.598413in}}%
\pgfpathlineto{\pgfqpoint{0.716920in}{1.599812in}}%
\pgfpathlineto{\pgfqpoint{0.725023in}{1.593789in}}%
\pgfpathlineto{\pgfqpoint{0.733125in}{1.592750in}}%
\pgfpathlineto{\pgfqpoint{0.741227in}{1.579860in}}%
\pgfpathlineto{\pgfqpoint{0.749330in}{1.574618in}}%
\pgfpathlineto{\pgfqpoint{0.757432in}{1.567154in}}%
\pgfpathlineto{\pgfqpoint{0.765534in}{1.565724in}}%
\pgfpathlineto{\pgfqpoint{0.781739in}{1.545195in}}%
\pgfpathlineto{\pgfqpoint{0.789841in}{1.542212in}}%
\pgfpathlineto{\pgfqpoint{0.797943in}{1.526318in}}%
\pgfpathlineto{\pgfqpoint{0.806045in}{1.516233in}}%
\pgfpathlineto{\pgfqpoint{0.846557in}{1.443278in}}%
\pgfpathlineto{\pgfqpoint{0.878966in}{1.390220in}}%
\pgfpathlineto{\pgfqpoint{0.887068in}{1.383567in}}%
\pgfpathlineto{\pgfqpoint{0.911375in}{1.345956in}}%
\pgfpathlineto{\pgfqpoint{0.927580in}{1.321657in}}%
\pgfpathlineto{\pgfqpoint{0.935682in}{1.309501in}}%
\pgfpathlineto{\pgfqpoint{0.951886in}{1.291841in}}%
\pgfpathlineto{\pgfqpoint{0.959989in}{1.286132in}}%
\pgfpathlineto{\pgfqpoint{1.000500in}{1.226387in}}%
\pgfpathlineto{\pgfqpoint{1.016705in}{1.205489in}}%
\pgfpathlineto{\pgfqpoint{1.024807in}{1.200716in}}%
\pgfpathlineto{\pgfqpoint{1.041011in}{1.178620in}}%
\pgfpathlineto{\pgfqpoint{1.049114in}{1.171245in}}%
\pgfpathlineto{\pgfqpoint{1.073420in}{1.142083in}}%
\pgfpathlineto{\pgfqpoint{1.089625in}{1.133326in}}%
\pgfpathlineto{\pgfqpoint{1.113932in}{1.104940in}}%
\pgfpathlineto{\pgfqpoint{1.122034in}{1.098649in}}%
\pgfpathlineto{\pgfqpoint{1.162545in}{1.054397in}}%
\pgfpathlineto{\pgfqpoint{1.178750in}{1.038215in}}%
\pgfpathlineto{\pgfqpoint{1.203057in}{1.015667in}}%
\pgfpathlineto{\pgfqpoint{1.211159in}{1.013741in}}%
\pgfpathlineto{\pgfqpoint{1.219261in}{1.005297in}}%
\pgfpathlineto{\pgfqpoint{1.227364in}{1.003414in}}%
\pgfpathlineto{\pgfqpoint{1.251670in}{0.979476in}}%
\pgfpathlineto{\pgfqpoint{1.259773in}{0.973451in}}%
\pgfpathlineto{\pgfqpoint{1.275977in}{0.957329in}}%
\pgfpathlineto{\pgfqpoint{1.292182in}{0.946900in}}%
\pgfpathlineto{\pgfqpoint{1.316489in}{0.924484in}}%
\pgfpathlineto{\pgfqpoint{1.324591in}{0.917726in}}%
\pgfpathlineto{\pgfqpoint{1.332693in}{0.914501in}}%
\pgfpathlineto{\pgfqpoint{1.340795in}{0.906949in}}%
\pgfpathlineto{\pgfqpoint{1.348898in}{0.903933in}}%
\pgfpathlineto{\pgfqpoint{1.381307in}{0.875913in}}%
\pgfpathlineto{\pgfqpoint{1.389409in}{0.868877in}}%
\pgfpathlineto{\pgfqpoint{1.405614in}{0.864921in}}%
\pgfpathlineto{\pgfqpoint{1.454227in}{0.823552in}}%
\pgfpathlineto{\pgfqpoint{1.486636in}{0.813785in}}%
\pgfpathlineto{\pgfqpoint{1.535250in}{0.775634in}}%
\pgfpathlineto{\pgfqpoint{1.543352in}{0.775579in}}%
\pgfpathlineto{\pgfqpoint{1.575761in}{0.751625in}}%
\pgfpathlineto{\pgfqpoint{1.583864in}{0.748319in}}%
\pgfpathlineto{\pgfqpoint{1.664886in}{0.691333in}}%
\pgfpathlineto{\pgfqpoint{1.705398in}{0.663799in}}%
\pgfpathlineto{\pgfqpoint{1.713500in}{0.661183in}}%
\pgfpathlineto{\pgfqpoint{1.802625in}{0.603562in}}%
\pgfpathlineto{\pgfqpoint{1.818830in}{0.595994in}}%
\pgfpathlineto{\pgfqpoint{1.859341in}{0.572409in}}%
\pgfpathlineto{\pgfqpoint{1.875545in}{0.563750in}}%
\pgfpathlineto{\pgfqpoint{1.883648in}{0.558908in}}%
\pgfpathlineto{\pgfqpoint{1.891750in}{0.557968in}}%
\pgfpathlineto{\pgfqpoint{1.907955in}{0.548369in}}%
\pgfpathlineto{\pgfqpoint{1.916057in}{0.547632in}}%
\pgfpathlineto{\pgfqpoint{1.924159in}{0.545167in}}%
\pgfpathlineto{\pgfqpoint{1.932261in}{0.540451in}}%
\pgfpathlineto{\pgfqpoint{1.940364in}{0.537753in}}%
\pgfpathlineto{\pgfqpoint{1.948466in}{0.538529in}}%
\pgfpathlineto{\pgfqpoint{1.988977in}{0.516917in}}%
\pgfpathlineto{\pgfqpoint{1.988977in}{0.516917in}}%
\pgfusepath{stroke}%
\end{pgfscope}%
\begin{pgfscope}%
\pgfpathrectangle{\pgfqpoint{0.287500in}{0.375000in}}{\pgfqpoint{1.782500in}{2.265000in}}%
\pgfusepath{clip}%
\pgfsetroundcap%
\pgfsetroundjoin%
\pgfsetlinewidth{1.505625pt}%
\definecolor{currentstroke}{rgb}{1.000000,0.498039,0.054902}%
\pgfsetstrokecolor{currentstroke}%
\pgfsetdash{}{0pt}%
\pgfpathmoveto{\pgfqpoint{0.368523in}{2.012444in}}%
\pgfpathlineto{\pgfqpoint{0.384727in}{1.849435in}}%
\pgfpathlineto{\pgfqpoint{0.392830in}{1.823100in}}%
\pgfpathlineto{\pgfqpoint{0.409034in}{1.802731in}}%
\pgfpathlineto{\pgfqpoint{0.417136in}{1.800078in}}%
\pgfpathlineto{\pgfqpoint{0.433341in}{1.800305in}}%
\pgfpathlineto{\pgfqpoint{0.441443in}{1.803357in}}%
\pgfpathlineto{\pgfqpoint{0.457648in}{1.784393in}}%
\pgfpathlineto{\pgfqpoint{0.465750in}{1.769888in}}%
\pgfpathlineto{\pgfqpoint{0.473852in}{1.789960in}}%
\pgfpathlineto{\pgfqpoint{0.481955in}{1.786572in}}%
\pgfpathlineto{\pgfqpoint{0.490057in}{1.769791in}}%
\pgfpathlineto{\pgfqpoint{0.498159in}{1.768173in}}%
\pgfpathlineto{\pgfqpoint{0.506261in}{1.784416in}}%
\pgfpathlineto{\pgfqpoint{0.514364in}{1.786868in}}%
\pgfpathlineto{\pgfqpoint{0.530568in}{1.771450in}}%
\pgfpathlineto{\pgfqpoint{0.538670in}{1.763234in}}%
\pgfpathlineto{\pgfqpoint{0.546773in}{1.759661in}}%
\pgfpathlineto{\pgfqpoint{0.562977in}{1.738708in}}%
\pgfpathlineto{\pgfqpoint{0.579182in}{1.740624in}}%
\pgfpathlineto{\pgfqpoint{0.587284in}{1.725848in}}%
\pgfpathlineto{\pgfqpoint{0.595386in}{1.722958in}}%
\pgfpathlineto{\pgfqpoint{0.603489in}{1.715748in}}%
\pgfpathlineto{\pgfqpoint{0.611591in}{1.714424in}}%
\pgfpathlineto{\pgfqpoint{0.619693in}{1.714779in}}%
\pgfpathlineto{\pgfqpoint{0.627795in}{1.724589in}}%
\pgfpathlineto{\pgfqpoint{0.635898in}{1.720112in}}%
\pgfpathlineto{\pgfqpoint{0.644000in}{1.718825in}}%
\pgfpathlineto{\pgfqpoint{0.652102in}{1.708202in}}%
\pgfpathlineto{\pgfqpoint{0.660205in}{1.707549in}}%
\pgfpathlineto{\pgfqpoint{0.676409in}{1.692916in}}%
\pgfpathlineto{\pgfqpoint{0.684511in}{1.688224in}}%
\pgfpathlineto{\pgfqpoint{0.692614in}{1.688816in}}%
\pgfpathlineto{\pgfqpoint{0.700716in}{1.681710in}}%
\pgfpathlineto{\pgfqpoint{0.708818in}{1.677559in}}%
\pgfpathlineto{\pgfqpoint{0.716920in}{1.670802in}}%
\pgfpathlineto{\pgfqpoint{0.725023in}{1.671794in}}%
\pgfpathlineto{\pgfqpoint{0.733125in}{1.659425in}}%
\pgfpathlineto{\pgfqpoint{0.741227in}{1.653412in}}%
\pgfpathlineto{\pgfqpoint{0.749330in}{1.649372in}}%
\pgfpathlineto{\pgfqpoint{0.757432in}{1.639827in}}%
\pgfpathlineto{\pgfqpoint{0.765534in}{1.634539in}}%
\pgfpathlineto{\pgfqpoint{0.773636in}{1.625413in}}%
\pgfpathlineto{\pgfqpoint{0.781739in}{1.619837in}}%
\pgfpathlineto{\pgfqpoint{0.797943in}{1.600044in}}%
\pgfpathlineto{\pgfqpoint{0.814148in}{1.587840in}}%
\pgfpathlineto{\pgfqpoint{0.830352in}{1.576213in}}%
\pgfpathlineto{\pgfqpoint{0.838455in}{1.569068in}}%
\pgfpathlineto{\pgfqpoint{0.846557in}{1.565022in}}%
\pgfpathlineto{\pgfqpoint{0.895170in}{1.511306in}}%
\pgfpathlineto{\pgfqpoint{0.911375in}{1.487991in}}%
\pgfpathlineto{\pgfqpoint{0.919477in}{1.478493in}}%
\pgfpathlineto{\pgfqpoint{0.943784in}{1.459606in}}%
\pgfpathlineto{\pgfqpoint{0.959989in}{1.450898in}}%
\pgfpathlineto{\pgfqpoint{0.968091in}{1.440552in}}%
\pgfpathlineto{\pgfqpoint{0.976193in}{1.436413in}}%
\pgfpathlineto{\pgfqpoint{0.992398in}{1.424101in}}%
\pgfpathlineto{\pgfqpoint{1.008602in}{1.404502in}}%
\pgfpathlineto{\pgfqpoint{1.016705in}{1.393951in}}%
\pgfpathlineto{\pgfqpoint{1.049114in}{1.365722in}}%
\pgfpathlineto{\pgfqpoint{1.065318in}{1.346403in}}%
\pgfpathlineto{\pgfqpoint{1.081523in}{1.335059in}}%
\pgfpathlineto{\pgfqpoint{1.089625in}{1.329432in}}%
\pgfpathlineto{\pgfqpoint{1.097727in}{1.327216in}}%
\pgfpathlineto{\pgfqpoint{1.113932in}{1.313749in}}%
\pgfpathlineto{\pgfqpoint{1.130136in}{1.301573in}}%
\pgfpathlineto{\pgfqpoint{1.138239in}{1.302444in}}%
\pgfpathlineto{\pgfqpoint{1.162545in}{1.283820in}}%
\pgfpathlineto{\pgfqpoint{1.170648in}{1.274949in}}%
\pgfpathlineto{\pgfqpoint{1.178750in}{1.269247in}}%
\pgfpathlineto{\pgfqpoint{1.186852in}{1.265433in}}%
\pgfpathlineto{\pgfqpoint{1.203057in}{1.253609in}}%
\pgfpathlineto{\pgfqpoint{1.235466in}{1.226949in}}%
\pgfpathlineto{\pgfqpoint{1.267875in}{1.202514in}}%
\pgfpathlineto{\pgfqpoint{1.275977in}{1.203139in}}%
\pgfpathlineto{\pgfqpoint{1.284080in}{1.195627in}}%
\pgfpathlineto{\pgfqpoint{1.292182in}{1.193708in}}%
\pgfpathlineto{\pgfqpoint{1.324591in}{1.171207in}}%
\pgfpathlineto{\pgfqpoint{1.332693in}{1.167274in}}%
\pgfpathlineto{\pgfqpoint{1.348898in}{1.163554in}}%
\pgfpathlineto{\pgfqpoint{1.357000in}{1.156448in}}%
\pgfpathlineto{\pgfqpoint{1.365102in}{1.151585in}}%
\pgfpathlineto{\pgfqpoint{1.389409in}{1.144972in}}%
\pgfpathlineto{\pgfqpoint{1.397511in}{1.142951in}}%
\pgfpathlineto{\pgfqpoint{1.405614in}{1.138260in}}%
\pgfpathlineto{\pgfqpoint{1.413716in}{1.131545in}}%
\pgfpathlineto{\pgfqpoint{1.421818in}{1.128115in}}%
\pgfpathlineto{\pgfqpoint{1.429920in}{1.130762in}}%
\pgfpathlineto{\pgfqpoint{1.446125in}{1.120750in}}%
\pgfpathlineto{\pgfqpoint{1.454227in}{1.117797in}}%
\pgfpathlineto{\pgfqpoint{1.478534in}{1.103809in}}%
\pgfpathlineto{\pgfqpoint{1.486636in}{1.105676in}}%
\pgfpathlineto{\pgfqpoint{1.502841in}{1.094965in}}%
\pgfpathlineto{\pgfqpoint{1.510943in}{1.070434in}}%
\pgfpathlineto{\pgfqpoint{1.519045in}{1.069472in}}%
\pgfpathlineto{\pgfqpoint{1.543352in}{1.059938in}}%
\pgfpathlineto{\pgfqpoint{1.551455in}{1.055619in}}%
\pgfpathlineto{\pgfqpoint{1.575761in}{1.038535in}}%
\pgfpathlineto{\pgfqpoint{1.583864in}{1.042195in}}%
\pgfpathlineto{\pgfqpoint{1.591966in}{1.038300in}}%
\pgfpathlineto{\pgfqpoint{1.616273in}{1.022435in}}%
\pgfpathlineto{\pgfqpoint{1.640580in}{1.007913in}}%
\pgfpathlineto{\pgfqpoint{1.648682in}{1.005605in}}%
\pgfpathlineto{\pgfqpoint{1.656784in}{1.001682in}}%
\pgfpathlineto{\pgfqpoint{1.664886in}{1.003407in}}%
\pgfpathlineto{\pgfqpoint{1.713500in}{0.976094in}}%
\pgfpathlineto{\pgfqpoint{1.721602in}{0.977159in}}%
\pgfpathlineto{\pgfqpoint{1.762114in}{0.957368in}}%
\pgfpathlineto{\pgfqpoint{1.794523in}{0.946729in}}%
\pgfpathlineto{\pgfqpoint{1.802625in}{0.944063in}}%
\pgfpathlineto{\pgfqpoint{1.843136in}{0.921648in}}%
\pgfpathlineto{\pgfqpoint{1.851239in}{0.917286in}}%
\pgfpathlineto{\pgfqpoint{1.859341in}{0.917469in}}%
\pgfpathlineto{\pgfqpoint{1.867443in}{0.915421in}}%
\pgfpathlineto{\pgfqpoint{1.883648in}{0.906965in}}%
\pgfpathlineto{\pgfqpoint{1.899852in}{0.898561in}}%
\pgfpathlineto{\pgfqpoint{1.907955in}{0.893787in}}%
\pgfpathlineto{\pgfqpoint{1.916057in}{0.897008in}}%
\pgfpathlineto{\pgfqpoint{1.940364in}{0.892051in}}%
\pgfpathlineto{\pgfqpoint{1.948466in}{0.887857in}}%
\pgfpathlineto{\pgfqpoint{1.956568in}{0.888485in}}%
\pgfpathlineto{\pgfqpoint{1.964670in}{0.886957in}}%
\pgfpathlineto{\pgfqpoint{1.972773in}{0.883250in}}%
\pgfpathlineto{\pgfqpoint{1.980875in}{0.883744in}}%
\pgfpathlineto{\pgfqpoint{1.988977in}{0.880745in}}%
\pgfpathlineto{\pgfqpoint{1.988977in}{0.880745in}}%
\pgfusepath{stroke}%
\end{pgfscope}%
\begin{pgfscope}%
\pgfpathrectangle{\pgfqpoint{0.287500in}{0.375000in}}{\pgfqpoint{1.782500in}{2.265000in}}%
\pgfusepath{clip}%
\pgfsetroundcap%
\pgfsetroundjoin%
\pgfsetlinewidth{1.505625pt}%
\definecolor{currentstroke}{rgb}{0.172549,0.627451,0.172549}%
\pgfsetstrokecolor{currentstroke}%
\pgfsetdash{}{0pt}%
\pgfpathmoveto{\pgfqpoint{0.368523in}{2.253641in}}%
\pgfpathlineto{\pgfqpoint{0.384727in}{1.769994in}}%
\pgfpathlineto{\pgfqpoint{0.400932in}{1.577916in}}%
\pgfpathlineto{\pgfqpoint{0.409034in}{1.524337in}}%
\pgfpathlineto{\pgfqpoint{0.417136in}{1.483102in}}%
\pgfpathlineto{\pgfqpoint{0.433341in}{1.315881in}}%
\pgfpathlineto{\pgfqpoint{0.441443in}{1.301181in}}%
\pgfpathlineto{\pgfqpoint{0.449545in}{1.302707in}}%
\pgfpathlineto{\pgfqpoint{0.465750in}{1.202621in}}%
\pgfpathlineto{\pgfqpoint{0.473852in}{1.167345in}}%
\pgfpathlineto{\pgfqpoint{0.481955in}{1.166599in}}%
\pgfpathlineto{\pgfqpoint{0.490057in}{1.192639in}}%
\pgfpathlineto{\pgfqpoint{0.498159in}{1.199461in}}%
\pgfpathlineto{\pgfqpoint{0.514364in}{1.162617in}}%
\pgfpathlineto{\pgfqpoint{0.530568in}{1.091653in}}%
\pgfpathlineto{\pgfqpoint{0.538670in}{1.091276in}}%
\pgfpathlineto{\pgfqpoint{0.546773in}{1.068838in}}%
\pgfpathlineto{\pgfqpoint{0.554875in}{1.065652in}}%
\pgfpathlineto{\pgfqpoint{0.562977in}{1.059084in}}%
\pgfpathlineto{\pgfqpoint{0.571080in}{1.046859in}}%
\pgfpathlineto{\pgfqpoint{0.579182in}{1.047657in}}%
\pgfpathlineto{\pgfqpoint{0.587284in}{1.083447in}}%
\pgfpathlineto{\pgfqpoint{0.595386in}{1.057205in}}%
\pgfpathlineto{\pgfqpoint{0.603489in}{1.042649in}}%
\pgfpathlineto{\pgfqpoint{0.611591in}{1.074649in}}%
\pgfpathlineto{\pgfqpoint{0.619693in}{1.088656in}}%
\pgfpathlineto{\pgfqpoint{0.627795in}{1.090885in}}%
\pgfpathlineto{\pgfqpoint{0.635898in}{1.068831in}}%
\pgfpathlineto{\pgfqpoint{0.644000in}{1.056682in}}%
\pgfpathlineto{\pgfqpoint{0.660205in}{1.060464in}}%
\pgfpathlineto{\pgfqpoint{0.668307in}{1.085021in}}%
\pgfpathlineto{\pgfqpoint{0.676409in}{1.076131in}}%
\pgfpathlineto{\pgfqpoint{0.684511in}{1.069796in}}%
\pgfpathlineto{\pgfqpoint{0.692614in}{1.081367in}}%
\pgfpathlineto{\pgfqpoint{0.700716in}{1.087387in}}%
\pgfpathlineto{\pgfqpoint{0.708818in}{1.082991in}}%
\pgfpathlineto{\pgfqpoint{0.725023in}{1.114819in}}%
\pgfpathlineto{\pgfqpoint{0.741227in}{1.106003in}}%
\pgfpathlineto{\pgfqpoint{0.749330in}{1.126813in}}%
\pgfpathlineto{\pgfqpoint{0.757432in}{1.143358in}}%
\pgfpathlineto{\pgfqpoint{0.765534in}{1.152077in}}%
\pgfpathlineto{\pgfqpoint{0.773636in}{1.151724in}}%
\pgfpathlineto{\pgfqpoint{0.781739in}{1.135864in}}%
\pgfpathlineto{\pgfqpoint{0.789841in}{1.139282in}}%
\pgfpathlineto{\pgfqpoint{0.797943in}{1.130995in}}%
\pgfpathlineto{\pgfqpoint{0.806045in}{1.129159in}}%
\pgfpathlineto{\pgfqpoint{0.814148in}{1.135372in}}%
\pgfpathlineto{\pgfqpoint{0.830352in}{1.137632in}}%
\pgfpathlineto{\pgfqpoint{0.838455in}{1.136298in}}%
\pgfpathlineto{\pgfqpoint{0.854659in}{1.145432in}}%
\pgfpathlineto{\pgfqpoint{0.862761in}{1.156349in}}%
\pgfpathlineto{\pgfqpoint{0.870864in}{1.156856in}}%
\pgfpathlineto{\pgfqpoint{0.878966in}{1.153864in}}%
\pgfpathlineto{\pgfqpoint{0.887068in}{1.153044in}}%
\pgfpathlineto{\pgfqpoint{0.895170in}{1.157504in}}%
\pgfpathlineto{\pgfqpoint{0.903273in}{1.150902in}}%
\pgfpathlineto{\pgfqpoint{0.911375in}{1.148367in}}%
\pgfpathlineto{\pgfqpoint{0.927580in}{1.149303in}}%
\pgfpathlineto{\pgfqpoint{0.935682in}{1.159784in}}%
\pgfpathlineto{\pgfqpoint{0.943784in}{1.150738in}}%
\pgfpathlineto{\pgfqpoint{0.968091in}{1.149241in}}%
\pgfpathlineto{\pgfqpoint{0.984295in}{1.131477in}}%
\pgfpathlineto{\pgfqpoint{0.992398in}{1.134391in}}%
\pgfpathlineto{\pgfqpoint{1.000500in}{1.130608in}}%
\pgfpathlineto{\pgfqpoint{1.008602in}{1.133206in}}%
\pgfpathlineto{\pgfqpoint{1.016705in}{1.127526in}}%
\pgfpathlineto{\pgfqpoint{1.024807in}{1.130739in}}%
\pgfpathlineto{\pgfqpoint{1.032909in}{1.131822in}}%
\pgfpathlineto{\pgfqpoint{1.049114in}{1.122577in}}%
\pgfpathlineto{\pgfqpoint{1.057216in}{1.123583in}}%
\pgfpathlineto{\pgfqpoint{1.089625in}{1.120644in}}%
\pgfpathlineto{\pgfqpoint{1.097727in}{1.124061in}}%
\pgfpathlineto{\pgfqpoint{1.105830in}{1.131518in}}%
\pgfpathlineto{\pgfqpoint{1.130136in}{1.123817in}}%
\pgfpathlineto{\pgfqpoint{1.146341in}{1.124671in}}%
\pgfpathlineto{\pgfqpoint{1.154443in}{1.125591in}}%
\pgfpathlineto{\pgfqpoint{1.162545in}{1.124646in}}%
\pgfpathlineto{\pgfqpoint{1.170648in}{1.119951in}}%
\pgfpathlineto{\pgfqpoint{1.178750in}{1.118517in}}%
\pgfpathlineto{\pgfqpoint{1.194955in}{1.124345in}}%
\pgfpathlineto{\pgfqpoint{1.203057in}{1.119656in}}%
\pgfpathlineto{\pgfqpoint{1.211159in}{1.119285in}}%
\pgfpathlineto{\pgfqpoint{1.227364in}{1.125438in}}%
\pgfpathlineto{\pgfqpoint{1.235466in}{1.120862in}}%
\pgfpathlineto{\pgfqpoint{1.243568in}{1.119197in}}%
\pgfpathlineto{\pgfqpoint{1.251670in}{1.114058in}}%
\pgfpathlineto{\pgfqpoint{1.267875in}{1.114711in}}%
\pgfpathlineto{\pgfqpoint{1.300284in}{1.104025in}}%
\pgfpathlineto{\pgfqpoint{1.308386in}{1.102196in}}%
\pgfpathlineto{\pgfqpoint{1.316489in}{1.104093in}}%
\pgfpathlineto{\pgfqpoint{1.348898in}{1.101580in}}%
\pgfpathlineto{\pgfqpoint{1.357000in}{1.106800in}}%
\pgfpathlineto{\pgfqpoint{1.365102in}{1.108216in}}%
\pgfpathlineto{\pgfqpoint{1.373205in}{1.103728in}}%
\pgfpathlineto{\pgfqpoint{1.389409in}{1.102111in}}%
\pgfpathlineto{\pgfqpoint{1.397511in}{1.106505in}}%
\pgfpathlineto{\pgfqpoint{1.413716in}{1.108732in}}%
\pgfpathlineto{\pgfqpoint{1.421818in}{1.112382in}}%
\pgfpathlineto{\pgfqpoint{1.438023in}{1.108350in}}%
\pgfpathlineto{\pgfqpoint{1.446125in}{1.105888in}}%
\pgfpathlineto{\pgfqpoint{1.454227in}{1.105882in}}%
\pgfpathlineto{\pgfqpoint{1.462330in}{1.102548in}}%
\pgfpathlineto{\pgfqpoint{1.470432in}{1.102738in}}%
\pgfpathlineto{\pgfqpoint{1.486636in}{1.099345in}}%
\pgfpathlineto{\pgfqpoint{1.494739in}{1.100751in}}%
\pgfpathlineto{\pgfqpoint{1.502841in}{1.099782in}}%
\pgfpathlineto{\pgfqpoint{1.510943in}{1.096611in}}%
\pgfpathlineto{\pgfqpoint{1.527148in}{1.100843in}}%
\pgfpathlineto{\pgfqpoint{1.535250in}{1.096964in}}%
\pgfpathlineto{\pgfqpoint{1.543352in}{1.095265in}}%
\pgfpathlineto{\pgfqpoint{1.567659in}{1.080988in}}%
\pgfpathlineto{\pgfqpoint{1.575761in}{1.083564in}}%
\pgfpathlineto{\pgfqpoint{1.591966in}{1.083192in}}%
\pgfpathlineto{\pgfqpoint{1.600068in}{1.086173in}}%
\pgfpathlineto{\pgfqpoint{1.616273in}{1.084129in}}%
\pgfpathlineto{\pgfqpoint{1.640580in}{1.086608in}}%
\pgfpathlineto{\pgfqpoint{1.648682in}{1.087238in}}%
\pgfpathlineto{\pgfqpoint{1.672989in}{1.094161in}}%
\pgfpathlineto{\pgfqpoint{1.681091in}{1.090675in}}%
\pgfpathlineto{\pgfqpoint{1.697295in}{1.091993in}}%
\pgfpathlineto{\pgfqpoint{1.705398in}{1.087947in}}%
\pgfpathlineto{\pgfqpoint{1.713500in}{1.085887in}}%
\pgfpathlineto{\pgfqpoint{1.721602in}{1.087856in}}%
\pgfpathlineto{\pgfqpoint{1.737807in}{1.084916in}}%
\pgfpathlineto{\pgfqpoint{1.745909in}{1.082789in}}%
\pgfpathlineto{\pgfqpoint{1.754011in}{1.083895in}}%
\pgfpathlineto{\pgfqpoint{1.762114in}{1.083425in}}%
\pgfpathlineto{\pgfqpoint{1.770216in}{1.086172in}}%
\pgfpathlineto{\pgfqpoint{1.786420in}{1.083610in}}%
\pgfpathlineto{\pgfqpoint{1.794523in}{1.086313in}}%
\pgfpathlineto{\pgfqpoint{1.802625in}{1.091875in}}%
\pgfpathlineto{\pgfqpoint{1.835034in}{1.105362in}}%
\pgfpathlineto{\pgfqpoint{1.843136in}{1.103150in}}%
\pgfpathlineto{\pgfqpoint{1.851239in}{1.107377in}}%
\pgfpathlineto{\pgfqpoint{1.859341in}{1.108685in}}%
\pgfpathlineto{\pgfqpoint{1.867443in}{1.107161in}}%
\pgfpathlineto{\pgfqpoint{1.891750in}{1.110593in}}%
\pgfpathlineto{\pgfqpoint{1.907955in}{1.111005in}}%
\pgfpathlineto{\pgfqpoint{1.916057in}{1.111978in}}%
\pgfpathlineto{\pgfqpoint{1.932261in}{1.109459in}}%
\pgfpathlineto{\pgfqpoint{1.940364in}{1.115269in}}%
\pgfpathlineto{\pgfqpoint{1.948466in}{1.116760in}}%
\pgfpathlineto{\pgfqpoint{1.980875in}{1.116869in}}%
\pgfpathlineto{\pgfqpoint{1.988977in}{1.118301in}}%
\pgfpathlineto{\pgfqpoint{1.988977in}{1.118301in}}%
\pgfusepath{stroke}%
\end{pgfscope}%
\begin{pgfscope}%
\pgfsetrectcap%
\pgfsetmiterjoin%
\pgfsetlinewidth{0.000000pt}%
\definecolor{currentstroke}{rgb}{1.000000,1.000000,1.000000}%
\pgfsetstrokecolor{currentstroke}%
\pgfsetdash{}{0pt}%
\pgfpathmoveto{\pgfqpoint{0.287500in}{0.375000in}}%
\pgfpathlineto{\pgfqpoint{0.287500in}{2.640000in}}%
\pgfusepath{}%
\end{pgfscope}%
\begin{pgfscope}%
\pgfsetrectcap%
\pgfsetmiterjoin%
\pgfsetlinewidth{0.000000pt}%
\definecolor{currentstroke}{rgb}{1.000000,1.000000,1.000000}%
\pgfsetstrokecolor{currentstroke}%
\pgfsetdash{}{0pt}%
\pgfpathmoveto{\pgfqpoint{2.070000in}{0.375000in}}%
\pgfpathlineto{\pgfqpoint{2.070000in}{2.640000in}}%
\pgfusepath{}%
\end{pgfscope}%
\begin{pgfscope}%
\pgfsetrectcap%
\pgfsetmiterjoin%
\pgfsetlinewidth{0.000000pt}%
\definecolor{currentstroke}{rgb}{1.000000,1.000000,1.000000}%
\pgfsetstrokecolor{currentstroke}%
\pgfsetdash{}{0pt}%
\pgfpathmoveto{\pgfqpoint{0.287500in}{0.375000in}}%
\pgfpathlineto{\pgfqpoint{2.070000in}{0.375000in}}%
\pgfusepath{}%
\end{pgfscope}%
\begin{pgfscope}%
\pgfsetrectcap%
\pgfsetmiterjoin%
\pgfsetlinewidth{0.000000pt}%
\definecolor{currentstroke}{rgb}{1.000000,1.000000,1.000000}%
\pgfsetstrokecolor{currentstroke}%
\pgfsetdash{}{0pt}%
\pgfpathmoveto{\pgfqpoint{0.287500in}{2.640000in}}%
\pgfpathlineto{\pgfqpoint{2.070000in}{2.640000in}}%
\pgfusepath{}%
\end{pgfscope}%
\begin{pgfscope}%
\definecolor{textcolor}{rgb}{0.150000,0.150000,0.150000}%
\pgfsetstrokecolor{textcolor}%
\pgfsetfillcolor{textcolor}%
\pgftext[x=1.178750in,y=2.723333in,,base]{\color{textcolor}\rmfamily\fontsize{8.000000}{9.600000}\selectfont Embedded SinOne in 2D}%
\end{pgfscope}%
\begin{pgfscope}%
\pgfsetroundcap%
\pgfsetroundjoin%
\pgfsetlinewidth{1.505625pt}%
\definecolor{currentstroke}{rgb}{0.121569,0.466667,0.705882}%
\pgfsetstrokecolor{currentstroke}%
\pgfsetdash{}{0pt}%
\pgfpathmoveto{\pgfqpoint{0.339607in}{2.494470in}}%
\pgfpathlineto{\pgfqpoint{0.561829in}{2.494470in}}%
\pgfusepath{stroke}%
\end{pgfscope}%
\begin{pgfscope}%
\definecolor{textcolor}{rgb}{0.150000,0.150000,0.150000}%
\pgfsetstrokecolor{textcolor}%
\pgfsetfillcolor{textcolor}%
\pgftext[x=0.650718in,y=2.455582in,left,base]{\color{textcolor}\rmfamily\fontsize{8.000000}{9.600000}\selectfont 5 x DNGO retrain-reset}%
\end{pgfscope}%
\begin{pgfscope}%
\pgfsetroundcap%
\pgfsetroundjoin%
\pgfsetlinewidth{1.505625pt}%
\definecolor{currentstroke}{rgb}{1.000000,0.498039,0.054902}%
\pgfsetstrokecolor{currentstroke}%
\pgfsetdash{}{0pt}%
\pgfpathmoveto{\pgfqpoint{0.339607in}{2.331385in}}%
\pgfpathlineto{\pgfqpoint{0.561829in}{2.331385in}}%
\pgfusepath{stroke}%
\end{pgfscope}%
\begin{pgfscope}%
\definecolor{textcolor}{rgb}{0.150000,0.150000,0.150000}%
\pgfsetstrokecolor{textcolor}%
\pgfsetfillcolor{textcolor}%
\pgftext[x=0.650718in,y=2.292496in,left,base]{\color{textcolor}\rmfamily\fontsize{8.000000}{9.600000}\selectfont DNGO retrain-reset}%
\end{pgfscope}%
\begin{pgfscope}%
\pgfsetroundcap%
\pgfsetroundjoin%
\pgfsetlinewidth{1.505625pt}%
\definecolor{currentstroke}{rgb}{0.172549,0.627451,0.172549}%
\pgfsetstrokecolor{currentstroke}%
\pgfsetdash{}{0pt}%
\pgfpathmoveto{\pgfqpoint{0.339607in}{2.168299in}}%
\pgfpathlineto{\pgfqpoint{0.561829in}{2.168299in}}%
\pgfusepath{stroke}%
\end{pgfscope}%
\begin{pgfscope}%
\definecolor{textcolor}{rgb}{0.150000,0.150000,0.150000}%
\pgfsetstrokecolor{textcolor}%
\pgfsetfillcolor{textcolor}%
\pgftext[x=0.650718in,y=2.129410in,left,base]{\color{textcolor}\rmfamily\fontsize{8.000000}{9.600000}\selectfont GP}%
\end{pgfscope}%
\end{pgfpicture}%
\makeatother%
\endgroup%

        \end{minipage}\qquad
        \begin{minipage}{0.3\linewidth}
            \centering
            %% Creator: Matplotlib, PGF backend
%%
%% To include the figure in your LaTeX document, write
%%   \input{<filename>.pgf}
%%
%% Make sure the required packages are loaded in your preamble
%%   \usepackage{pgf}
%%
%% Figures using additional raster images can only be included by \input if
%% they are in the same directory as the main LaTeX file. For loading figures
%% from other directories you can use the `import` package
%%   \usepackage{import}
%% and then include the figures with
%%   \import{<path to file>}{<filename>.pgf}
%%
%% Matplotlib used the following preamble
%%   \usepackage{gensymb}
%%   \usepackage{fontspec}
%%   \setmainfont{DejaVu Serif}
%%   \setsansfont{Arial}
%%   \setmonofont{DejaVu Sans Mono}
%%
\begingroup%
\makeatletter%
\begin{pgfpicture}%
\pgfpathrectangle{\pgfpointorigin}{\pgfqpoint{2.300000in}{3.000000in}}%
\pgfusepath{use as bounding box, clip}%
\begin{pgfscope}%
\pgfsetbuttcap%
\pgfsetmiterjoin%
\definecolor{currentfill}{rgb}{1.000000,1.000000,1.000000}%
\pgfsetfillcolor{currentfill}%
\pgfsetlinewidth{0.000000pt}%
\definecolor{currentstroke}{rgb}{1.000000,1.000000,1.000000}%
\pgfsetstrokecolor{currentstroke}%
\pgfsetdash{}{0pt}%
\pgfpathmoveto{\pgfqpoint{0.000000in}{0.000000in}}%
\pgfpathlineto{\pgfqpoint{2.300000in}{0.000000in}}%
\pgfpathlineto{\pgfqpoint{2.300000in}{3.000000in}}%
\pgfpathlineto{\pgfqpoint{0.000000in}{3.000000in}}%
\pgfpathclose%
\pgfusepath{fill}%
\end{pgfscope}%
\begin{pgfscope}%
\pgfsetbuttcap%
\pgfsetmiterjoin%
\definecolor{currentfill}{rgb}{0.917647,0.917647,0.949020}%
\pgfsetfillcolor{currentfill}%
\pgfsetlinewidth{0.000000pt}%
\definecolor{currentstroke}{rgb}{0.000000,0.000000,0.000000}%
\pgfsetstrokecolor{currentstroke}%
\pgfsetstrokeopacity{0.000000}%
\pgfsetdash{}{0pt}%
\pgfpathmoveto{\pgfqpoint{0.287500in}{0.375000in}}%
\pgfpathlineto{\pgfqpoint{2.070000in}{0.375000in}}%
\pgfpathlineto{\pgfqpoint{2.070000in}{2.640000in}}%
\pgfpathlineto{\pgfqpoint{0.287500in}{2.640000in}}%
\pgfpathclose%
\pgfusepath{fill}%
\end{pgfscope}%
\begin{pgfscope}%
\pgfpathrectangle{\pgfqpoint{0.287500in}{0.375000in}}{\pgfqpoint{1.782500in}{2.265000in}}%
\pgfusepath{clip}%
\pgfsetroundcap%
\pgfsetroundjoin%
\pgfsetlinewidth{0.803000pt}%
\definecolor{currentstroke}{rgb}{1.000000,1.000000,1.000000}%
\pgfsetstrokecolor{currentstroke}%
\pgfsetdash{}{0pt}%
\pgfpathmoveto{\pgfqpoint{0.360420in}{0.375000in}}%
\pgfpathlineto{\pgfqpoint{0.360420in}{2.640000in}}%
\pgfusepath{stroke}%
\end{pgfscope}%
\begin{pgfscope}%
\definecolor{textcolor}{rgb}{0.150000,0.150000,0.150000}%
\pgfsetstrokecolor{textcolor}%
\pgfsetfillcolor{textcolor}%
\pgftext[x=0.360420in,y=0.326389in,,top]{\color{textcolor}\rmfamily\fontsize{8.000000}{9.600000}\selectfont \(\displaystyle 0\)}%
\end{pgfscope}%
\begin{pgfscope}%
\pgfpathrectangle{\pgfqpoint{0.287500in}{0.375000in}}{\pgfqpoint{1.782500in}{2.265000in}}%
\pgfusepath{clip}%
\pgfsetroundcap%
\pgfsetroundjoin%
\pgfsetlinewidth{0.803000pt}%
\definecolor{currentstroke}{rgb}{1.000000,1.000000,1.000000}%
\pgfsetstrokecolor{currentstroke}%
\pgfsetdash{}{0pt}%
\pgfpathmoveto{\pgfqpoint{0.765534in}{0.375000in}}%
\pgfpathlineto{\pgfqpoint{0.765534in}{2.640000in}}%
\pgfusepath{stroke}%
\end{pgfscope}%
\begin{pgfscope}%
\definecolor{textcolor}{rgb}{0.150000,0.150000,0.150000}%
\pgfsetstrokecolor{textcolor}%
\pgfsetfillcolor{textcolor}%
\pgftext[x=0.765534in,y=0.326389in,,top]{\color{textcolor}\rmfamily\fontsize{8.000000}{9.600000}\selectfont \(\displaystyle 50\)}%
\end{pgfscope}%
\begin{pgfscope}%
\pgfpathrectangle{\pgfqpoint{0.287500in}{0.375000in}}{\pgfqpoint{1.782500in}{2.265000in}}%
\pgfusepath{clip}%
\pgfsetroundcap%
\pgfsetroundjoin%
\pgfsetlinewidth{0.803000pt}%
\definecolor{currentstroke}{rgb}{1.000000,1.000000,1.000000}%
\pgfsetstrokecolor{currentstroke}%
\pgfsetdash{}{0pt}%
\pgfpathmoveto{\pgfqpoint{1.170648in}{0.375000in}}%
\pgfpathlineto{\pgfqpoint{1.170648in}{2.640000in}}%
\pgfusepath{stroke}%
\end{pgfscope}%
\begin{pgfscope}%
\definecolor{textcolor}{rgb}{0.150000,0.150000,0.150000}%
\pgfsetstrokecolor{textcolor}%
\pgfsetfillcolor{textcolor}%
\pgftext[x=1.170648in,y=0.326389in,,top]{\color{textcolor}\rmfamily\fontsize{8.000000}{9.600000}\selectfont \(\displaystyle 100\)}%
\end{pgfscope}%
\begin{pgfscope}%
\pgfpathrectangle{\pgfqpoint{0.287500in}{0.375000in}}{\pgfqpoint{1.782500in}{2.265000in}}%
\pgfusepath{clip}%
\pgfsetroundcap%
\pgfsetroundjoin%
\pgfsetlinewidth{0.803000pt}%
\definecolor{currentstroke}{rgb}{1.000000,1.000000,1.000000}%
\pgfsetstrokecolor{currentstroke}%
\pgfsetdash{}{0pt}%
\pgfpathmoveto{\pgfqpoint{1.575761in}{0.375000in}}%
\pgfpathlineto{\pgfqpoint{1.575761in}{2.640000in}}%
\pgfusepath{stroke}%
\end{pgfscope}%
\begin{pgfscope}%
\definecolor{textcolor}{rgb}{0.150000,0.150000,0.150000}%
\pgfsetstrokecolor{textcolor}%
\pgfsetfillcolor{textcolor}%
\pgftext[x=1.575761in,y=0.326389in,,top]{\color{textcolor}\rmfamily\fontsize{8.000000}{9.600000}\selectfont \(\displaystyle 150\)}%
\end{pgfscope}%
\begin{pgfscope}%
\pgfpathrectangle{\pgfqpoint{0.287500in}{0.375000in}}{\pgfqpoint{1.782500in}{2.265000in}}%
\pgfusepath{clip}%
\pgfsetroundcap%
\pgfsetroundjoin%
\pgfsetlinewidth{0.803000pt}%
\definecolor{currentstroke}{rgb}{1.000000,1.000000,1.000000}%
\pgfsetstrokecolor{currentstroke}%
\pgfsetdash{}{0pt}%
\pgfpathmoveto{\pgfqpoint{1.980875in}{0.375000in}}%
\pgfpathlineto{\pgfqpoint{1.980875in}{2.640000in}}%
\pgfusepath{stroke}%
\end{pgfscope}%
\begin{pgfscope}%
\definecolor{textcolor}{rgb}{0.150000,0.150000,0.150000}%
\pgfsetstrokecolor{textcolor}%
\pgfsetfillcolor{textcolor}%
\pgftext[x=1.980875in,y=0.326389in,,top]{\color{textcolor}\rmfamily\fontsize{8.000000}{9.600000}\selectfont \(\displaystyle 200\)}%
\end{pgfscope}%
\begin{pgfscope}%
\definecolor{textcolor}{rgb}{0.150000,0.150000,0.150000}%
\pgfsetstrokecolor{textcolor}%
\pgfsetfillcolor{textcolor}%
\pgftext[x=1.178750in,y=0.163303in,,top]{\color{textcolor}\rmfamily\fontsize{8.000000}{9.600000}\selectfont Step \(\displaystyle t\)}%
\end{pgfscope}%
\begin{pgfscope}%
\pgfpathrectangle{\pgfqpoint{0.287500in}{0.375000in}}{\pgfqpoint{1.782500in}{2.265000in}}%
\pgfusepath{clip}%
\pgfsetroundcap%
\pgfsetroundjoin%
\pgfsetlinewidth{0.803000pt}%
\definecolor{currentstroke}{rgb}{1.000000,1.000000,1.000000}%
\pgfsetstrokecolor{currentstroke}%
\pgfsetdash{}{0pt}%
\pgfpathmoveto{\pgfqpoint{0.287500in}{2.233700in}}%
\pgfpathlineto{\pgfqpoint{2.070000in}{2.233700in}}%
\pgfusepath{stroke}%
\end{pgfscope}%
\begin{pgfscope}%
\definecolor{textcolor}{rgb}{0.150000,0.150000,0.150000}%
\pgfsetstrokecolor{textcolor}%
\pgfsetfillcolor{textcolor}%
\pgftext[x=0.062962in,y=2.191490in,left,base]{\color{textcolor}\rmfamily\fontsize{8.000000}{9.600000}\selectfont \(\displaystyle 10^{0}\)}%
\end{pgfscope}%
\begin{pgfscope}%
\definecolor{textcolor}{rgb}{0.150000,0.150000,0.150000}%
\pgfsetstrokecolor{textcolor}%
\pgfsetfillcolor{textcolor}%
\pgftext[x=0.007407in,y=1.507500in,,bottom,rotate=90.000000]{\color{textcolor}\rmfamily\fontsize{8.000000}{9.600000}\selectfont \(\displaystyle R_t/t\)}%
\end{pgfscope}%
\begin{pgfscope}%
\pgfpathrectangle{\pgfqpoint{0.287500in}{0.375000in}}{\pgfqpoint{1.782500in}{2.265000in}}%
\pgfusepath{clip}%
\pgfsetbuttcap%
\pgfsetroundjoin%
\definecolor{currentfill}{rgb}{0.121569,0.466667,0.705882}%
\pgfsetfillcolor{currentfill}%
\pgfsetfillopacity{0.200000}%
\pgfsetlinewidth{0.000000pt}%
\definecolor{currentstroke}{rgb}{0.000000,0.000000,0.000000}%
\pgfsetstrokecolor{currentstroke}%
\pgfsetdash{}{0pt}%
\pgfpathmoveto{\pgfqpoint{0.368523in}{1.553724in}}%
\pgfpathlineto{\pgfqpoint{0.368523in}{2.077427in}}%
\pgfpathlineto{\pgfqpoint{0.376625in}{1.962347in}}%
\pgfpathlineto{\pgfqpoint{0.384727in}{1.866702in}}%
\pgfpathlineto{\pgfqpoint{0.392830in}{1.813427in}}%
\pgfpathlineto{\pgfqpoint{0.400932in}{1.769675in}}%
\pgfpathlineto{\pgfqpoint{0.409034in}{1.715150in}}%
\pgfpathlineto{\pgfqpoint{0.417136in}{1.707036in}}%
\pgfpathlineto{\pgfqpoint{0.425239in}{1.766010in}}%
\pgfpathlineto{\pgfqpoint{0.433341in}{1.765853in}}%
\pgfpathlineto{\pgfqpoint{0.441443in}{1.716707in}}%
\pgfpathlineto{\pgfqpoint{0.449545in}{1.751289in}}%
\pgfpathlineto{\pgfqpoint{0.457648in}{1.718776in}}%
\pgfpathlineto{\pgfqpoint{0.465750in}{1.746975in}}%
\pgfpathlineto{\pgfqpoint{0.473852in}{1.734597in}}%
\pgfpathlineto{\pgfqpoint{0.481955in}{1.741932in}}%
\pgfpathlineto{\pgfqpoint{0.490057in}{1.721524in}}%
\pgfpathlineto{\pgfqpoint{0.498159in}{1.724809in}}%
\pgfpathlineto{\pgfqpoint{0.506261in}{1.723869in}}%
\pgfpathlineto{\pgfqpoint{0.514364in}{1.693273in}}%
\pgfpathlineto{\pgfqpoint{0.522466in}{1.687791in}}%
\pgfpathlineto{\pgfqpoint{0.530568in}{1.702314in}}%
\pgfpathlineto{\pgfqpoint{0.538670in}{1.706424in}}%
\pgfpathlineto{\pgfqpoint{0.546773in}{1.711056in}}%
\pgfpathlineto{\pgfqpoint{0.554875in}{1.705990in}}%
\pgfpathlineto{\pgfqpoint{0.562977in}{1.711601in}}%
\pgfpathlineto{\pgfqpoint{0.571080in}{1.706653in}}%
\pgfpathlineto{\pgfqpoint{0.579182in}{1.701442in}}%
\pgfpathlineto{\pgfqpoint{0.587284in}{1.712243in}}%
\pgfpathlineto{\pgfqpoint{0.595386in}{1.707879in}}%
\pgfpathlineto{\pgfqpoint{0.603489in}{1.700564in}}%
\pgfpathlineto{\pgfqpoint{0.611591in}{1.694557in}}%
\pgfpathlineto{\pgfqpoint{0.619693in}{1.682079in}}%
\pgfpathlineto{\pgfqpoint{0.627795in}{1.673758in}}%
\pgfpathlineto{\pgfqpoint{0.635898in}{1.684767in}}%
\pgfpathlineto{\pgfqpoint{0.644000in}{1.681147in}}%
\pgfpathlineto{\pgfqpoint{0.652102in}{1.675903in}}%
\pgfpathlineto{\pgfqpoint{0.660205in}{1.670830in}}%
\pgfpathlineto{\pgfqpoint{0.668307in}{1.663187in}}%
\pgfpathlineto{\pgfqpoint{0.676409in}{1.658782in}}%
\pgfpathlineto{\pgfqpoint{0.684511in}{1.650916in}}%
\pgfpathlineto{\pgfqpoint{0.692614in}{1.663665in}}%
\pgfpathlineto{\pgfqpoint{0.700716in}{1.655340in}}%
\pgfpathlineto{\pgfqpoint{0.708818in}{1.662670in}}%
\pgfpathlineto{\pgfqpoint{0.716920in}{1.658979in}}%
\pgfpathlineto{\pgfqpoint{0.725023in}{1.656204in}}%
\pgfpathlineto{\pgfqpoint{0.733125in}{1.649050in}}%
\pgfpathlineto{\pgfqpoint{0.741227in}{1.640133in}}%
\pgfpathlineto{\pgfqpoint{0.749330in}{1.628952in}}%
\pgfpathlineto{\pgfqpoint{0.757432in}{1.618986in}}%
\pgfpathlineto{\pgfqpoint{0.765534in}{1.608604in}}%
\pgfpathlineto{\pgfqpoint{0.773636in}{1.600647in}}%
\pgfpathlineto{\pgfqpoint{0.781739in}{1.604622in}}%
\pgfpathlineto{\pgfqpoint{0.789841in}{1.598283in}}%
\pgfpathlineto{\pgfqpoint{0.797943in}{1.593422in}}%
\pgfpathlineto{\pgfqpoint{0.806045in}{1.587753in}}%
\pgfpathlineto{\pgfqpoint{0.814148in}{1.591224in}}%
\pgfpathlineto{\pgfqpoint{0.822250in}{1.593883in}}%
\pgfpathlineto{\pgfqpoint{0.830352in}{1.586611in}}%
\pgfpathlineto{\pgfqpoint{0.838455in}{1.589809in}}%
\pgfpathlineto{\pgfqpoint{0.846557in}{1.582054in}}%
\pgfpathlineto{\pgfqpoint{0.854659in}{1.577508in}}%
\pgfpathlineto{\pgfqpoint{0.862761in}{1.570805in}}%
\pgfpathlineto{\pgfqpoint{0.870864in}{1.570251in}}%
\pgfpathlineto{\pgfqpoint{0.878966in}{1.563832in}}%
\pgfpathlineto{\pgfqpoint{0.887068in}{1.554617in}}%
\pgfpathlineto{\pgfqpoint{0.895170in}{1.544495in}}%
\pgfpathlineto{\pgfqpoint{0.903273in}{1.536859in}}%
\pgfpathlineto{\pgfqpoint{0.911375in}{1.529352in}}%
\pgfpathlineto{\pgfqpoint{0.919477in}{1.526298in}}%
\pgfpathlineto{\pgfqpoint{0.927580in}{1.525935in}}%
\pgfpathlineto{\pgfqpoint{0.935682in}{1.521912in}}%
\pgfpathlineto{\pgfqpoint{0.943784in}{1.509266in}}%
\pgfpathlineto{\pgfqpoint{0.951886in}{1.499327in}}%
\pgfpathlineto{\pgfqpoint{0.959989in}{1.497032in}}%
\pgfpathlineto{\pgfqpoint{0.968091in}{1.493200in}}%
\pgfpathlineto{\pgfqpoint{0.976193in}{1.487667in}}%
\pgfpathlineto{\pgfqpoint{0.984295in}{1.478006in}}%
\pgfpathlineto{\pgfqpoint{0.992398in}{1.470180in}}%
\pgfpathlineto{\pgfqpoint{1.000500in}{1.468672in}}%
\pgfpathlineto{\pgfqpoint{1.008602in}{1.470219in}}%
\pgfpathlineto{\pgfqpoint{1.016705in}{1.463059in}}%
\pgfpathlineto{\pgfqpoint{1.024807in}{1.453160in}}%
\pgfpathlineto{\pgfqpoint{1.032909in}{1.444217in}}%
\pgfpathlineto{\pgfqpoint{1.041011in}{1.435435in}}%
\pgfpathlineto{\pgfqpoint{1.049114in}{1.427121in}}%
\pgfpathlineto{\pgfqpoint{1.057216in}{1.417667in}}%
\pgfpathlineto{\pgfqpoint{1.065318in}{1.407971in}}%
\pgfpathlineto{\pgfqpoint{1.073420in}{1.402004in}}%
\pgfpathlineto{\pgfqpoint{1.081523in}{1.394453in}}%
\pgfpathlineto{\pgfqpoint{1.089625in}{1.386611in}}%
\pgfpathlineto{\pgfqpoint{1.097727in}{1.378067in}}%
\pgfpathlineto{\pgfqpoint{1.105830in}{1.395433in}}%
\pgfpathlineto{\pgfqpoint{1.113932in}{1.385611in}}%
\pgfpathlineto{\pgfqpoint{1.122034in}{1.378571in}}%
\pgfpathlineto{\pgfqpoint{1.130136in}{1.371064in}}%
\pgfpathlineto{\pgfqpoint{1.138239in}{1.361403in}}%
\pgfpathlineto{\pgfqpoint{1.146341in}{1.354209in}}%
\pgfpathlineto{\pgfqpoint{1.154443in}{1.344770in}}%
\pgfpathlineto{\pgfqpoint{1.162545in}{1.338612in}}%
\pgfpathlineto{\pgfqpoint{1.170648in}{1.331517in}}%
\pgfpathlineto{\pgfqpoint{1.178750in}{1.322371in}}%
\pgfpathlineto{\pgfqpoint{1.186852in}{1.320090in}}%
\pgfpathlineto{\pgfqpoint{1.194955in}{1.312897in}}%
\pgfpathlineto{\pgfqpoint{1.203057in}{1.303937in}}%
\pgfpathlineto{\pgfqpoint{1.211159in}{1.302110in}}%
\pgfpathlineto{\pgfqpoint{1.219261in}{1.295128in}}%
\pgfpathlineto{\pgfqpoint{1.227364in}{1.292883in}}%
\pgfpathlineto{\pgfqpoint{1.235466in}{1.289770in}}%
\pgfpathlineto{\pgfqpoint{1.243568in}{1.282469in}}%
\pgfpathlineto{\pgfqpoint{1.251670in}{1.277876in}}%
\pgfpathlineto{\pgfqpoint{1.259773in}{1.269902in}}%
\pgfpathlineto{\pgfqpoint{1.267875in}{1.271510in}}%
\pgfpathlineto{\pgfqpoint{1.275977in}{1.265330in}}%
\pgfpathlineto{\pgfqpoint{1.284080in}{1.257090in}}%
\pgfpathlineto{\pgfqpoint{1.292182in}{1.251144in}}%
\pgfpathlineto{\pgfqpoint{1.300284in}{1.243509in}}%
\pgfpathlineto{\pgfqpoint{1.308386in}{1.243915in}}%
\pgfpathlineto{\pgfqpoint{1.316489in}{1.238281in}}%
\pgfpathlineto{\pgfqpoint{1.324591in}{1.234029in}}%
\pgfpathlineto{\pgfqpoint{1.332693in}{1.228373in}}%
\pgfpathlineto{\pgfqpoint{1.340795in}{1.220654in}}%
\pgfpathlineto{\pgfqpoint{1.348898in}{1.212963in}}%
\pgfpathlineto{\pgfqpoint{1.357000in}{1.207622in}}%
\pgfpathlineto{\pgfqpoint{1.365102in}{1.201289in}}%
\pgfpathlineto{\pgfqpoint{1.373205in}{1.196278in}}%
\pgfpathlineto{\pgfqpoint{1.381307in}{1.190949in}}%
\pgfpathlineto{\pgfqpoint{1.389409in}{1.187837in}}%
\pgfpathlineto{\pgfqpoint{1.397511in}{1.189455in}}%
\pgfpathlineto{\pgfqpoint{1.405614in}{1.184884in}}%
\pgfpathlineto{\pgfqpoint{1.413716in}{1.180361in}}%
\pgfpathlineto{\pgfqpoint{1.421818in}{1.175299in}}%
\pgfpathlineto{\pgfqpoint{1.429920in}{1.170753in}}%
\pgfpathlineto{\pgfqpoint{1.438023in}{1.165744in}}%
\pgfpathlineto{\pgfqpoint{1.446125in}{1.160814in}}%
\pgfpathlineto{\pgfqpoint{1.454227in}{1.156331in}}%
\pgfpathlineto{\pgfqpoint{1.462330in}{1.149763in}}%
\pgfpathlineto{\pgfqpoint{1.470432in}{1.144943in}}%
\pgfpathlineto{\pgfqpoint{1.478534in}{1.139768in}}%
\pgfpathlineto{\pgfqpoint{1.486636in}{1.135050in}}%
\pgfpathlineto{\pgfqpoint{1.494739in}{1.131019in}}%
\pgfpathlineto{\pgfqpoint{1.502841in}{1.126762in}}%
\pgfpathlineto{\pgfqpoint{1.510943in}{1.122163in}}%
\pgfpathlineto{\pgfqpoint{1.519045in}{1.117686in}}%
\pgfpathlineto{\pgfqpoint{1.527148in}{1.114641in}}%
\pgfpathlineto{\pgfqpoint{1.535250in}{1.110171in}}%
\pgfpathlineto{\pgfqpoint{1.543352in}{1.106007in}}%
\pgfpathlineto{\pgfqpoint{1.551455in}{1.099662in}}%
\pgfpathlineto{\pgfqpoint{1.559557in}{1.095094in}}%
\pgfpathlineto{\pgfqpoint{1.567659in}{1.088971in}}%
\pgfpathlineto{\pgfqpoint{1.575761in}{1.085753in}}%
\pgfpathlineto{\pgfqpoint{1.583864in}{1.081606in}}%
\pgfpathlineto{\pgfqpoint{1.591966in}{1.075489in}}%
\pgfpathlineto{\pgfqpoint{1.600068in}{1.069356in}}%
\pgfpathlineto{\pgfqpoint{1.608170in}{1.063672in}}%
\pgfpathlineto{\pgfqpoint{1.616273in}{1.065362in}}%
\pgfpathlineto{\pgfqpoint{1.624375in}{1.061314in}}%
\pgfpathlineto{\pgfqpoint{1.632477in}{1.055396in}}%
\pgfpathlineto{\pgfqpoint{1.640580in}{1.050176in}}%
\pgfpathlineto{\pgfqpoint{1.648682in}{1.044295in}}%
\pgfpathlineto{\pgfqpoint{1.656784in}{1.040543in}}%
\pgfpathlineto{\pgfqpoint{1.664886in}{1.036750in}}%
\pgfpathlineto{\pgfqpoint{1.672989in}{1.032756in}}%
\pgfpathlineto{\pgfqpoint{1.681091in}{1.028228in}}%
\pgfpathlineto{\pgfqpoint{1.689193in}{1.023498in}}%
\pgfpathlineto{\pgfqpoint{1.697295in}{1.020685in}}%
\pgfpathlineto{\pgfqpoint{1.705398in}{1.016948in}}%
\pgfpathlineto{\pgfqpoint{1.713500in}{1.011437in}}%
\pgfpathlineto{\pgfqpoint{1.721602in}{1.006205in}}%
\pgfpathlineto{\pgfqpoint{1.729705in}{1.002228in}}%
\pgfpathlineto{\pgfqpoint{1.737807in}{0.997031in}}%
\pgfpathlineto{\pgfqpoint{1.745909in}{0.993569in}}%
\pgfpathlineto{\pgfqpoint{1.754011in}{0.988146in}}%
\pgfpathlineto{\pgfqpoint{1.762114in}{0.982723in}}%
\pgfpathlineto{\pgfqpoint{1.770216in}{0.977404in}}%
\pgfpathlineto{\pgfqpoint{1.778318in}{0.973959in}}%
\pgfpathlineto{\pgfqpoint{1.786420in}{0.968723in}}%
\pgfpathlineto{\pgfqpoint{1.794523in}{0.963958in}}%
\pgfpathlineto{\pgfqpoint{1.802625in}{0.958947in}}%
\pgfpathlineto{\pgfqpoint{1.810727in}{0.953738in}}%
\pgfpathlineto{\pgfqpoint{1.818830in}{0.948982in}}%
\pgfpathlineto{\pgfqpoint{1.826932in}{0.945731in}}%
\pgfpathlineto{\pgfqpoint{1.835034in}{0.940918in}}%
\pgfpathlineto{\pgfqpoint{1.843136in}{0.935830in}}%
\pgfpathlineto{\pgfqpoint{1.851239in}{0.933899in}}%
\pgfpathlineto{\pgfqpoint{1.859341in}{0.933490in}}%
\pgfpathlineto{\pgfqpoint{1.867443in}{0.931935in}}%
\pgfpathlineto{\pgfqpoint{1.875545in}{0.926939in}}%
\pgfpathlineto{\pgfqpoint{1.883648in}{0.922267in}}%
\pgfpathlineto{\pgfqpoint{1.891750in}{0.918026in}}%
\pgfpathlineto{\pgfqpoint{1.899852in}{0.915000in}}%
\pgfpathlineto{\pgfqpoint{1.907955in}{0.910186in}}%
\pgfpathlineto{\pgfqpoint{1.916057in}{0.905627in}}%
\pgfpathlineto{\pgfqpoint{1.924159in}{0.902664in}}%
\pgfpathlineto{\pgfqpoint{1.932261in}{0.899857in}}%
\pgfpathlineto{\pgfqpoint{1.940364in}{0.895288in}}%
\pgfpathlineto{\pgfqpoint{1.948466in}{0.890707in}}%
\pgfpathlineto{\pgfqpoint{1.956568in}{0.886025in}}%
\pgfpathlineto{\pgfqpoint{1.964670in}{0.881373in}}%
\pgfpathlineto{\pgfqpoint{1.972773in}{0.877849in}}%
\pgfpathlineto{\pgfqpoint{1.980875in}{0.875068in}}%
\pgfpathlineto{\pgfqpoint{1.988977in}{0.871951in}}%
\pgfpathlineto{\pgfqpoint{1.988977in}{0.480289in}}%
\pgfpathlineto{\pgfqpoint{1.988977in}{0.480289in}}%
\pgfpathlineto{\pgfqpoint{1.980875in}{0.477955in}}%
\pgfpathlineto{\pgfqpoint{1.972773in}{0.483038in}}%
\pgfpathlineto{\pgfqpoint{1.964670in}{0.487958in}}%
\pgfpathlineto{\pgfqpoint{1.956568in}{0.492345in}}%
\pgfpathlineto{\pgfqpoint{1.948466in}{0.497121in}}%
\pgfpathlineto{\pgfqpoint{1.940364in}{0.500660in}}%
\pgfpathlineto{\pgfqpoint{1.932261in}{0.504480in}}%
\pgfpathlineto{\pgfqpoint{1.924159in}{0.509427in}}%
\pgfpathlineto{\pgfqpoint{1.916057in}{0.514664in}}%
\pgfpathlineto{\pgfqpoint{1.907955in}{0.511052in}}%
\pgfpathlineto{\pgfqpoint{1.899852in}{0.515615in}}%
\pgfpathlineto{\pgfqpoint{1.891750in}{0.520582in}}%
\pgfpathlineto{\pgfqpoint{1.883648in}{0.522447in}}%
\pgfpathlineto{\pgfqpoint{1.875545in}{0.527425in}}%
\pgfpathlineto{\pgfqpoint{1.867443in}{0.532353in}}%
\pgfpathlineto{\pgfqpoint{1.859341in}{0.538051in}}%
\pgfpathlineto{\pgfqpoint{1.851239in}{0.544012in}}%
\pgfpathlineto{\pgfqpoint{1.843136in}{0.549731in}}%
\pgfpathlineto{\pgfqpoint{1.835034in}{0.554734in}}%
\pgfpathlineto{\pgfqpoint{1.826932in}{0.552571in}}%
\pgfpathlineto{\pgfqpoint{1.818830in}{0.558116in}}%
\pgfpathlineto{\pgfqpoint{1.810727in}{0.551245in}}%
\pgfpathlineto{\pgfqpoint{1.802625in}{0.555592in}}%
\pgfpathlineto{\pgfqpoint{1.794523in}{0.560676in}}%
\pgfpathlineto{\pgfqpoint{1.786420in}{0.564731in}}%
\pgfpathlineto{\pgfqpoint{1.778318in}{0.570027in}}%
\pgfpathlineto{\pgfqpoint{1.770216in}{0.575763in}}%
\pgfpathlineto{\pgfqpoint{1.762114in}{0.581090in}}%
\pgfpathlineto{\pgfqpoint{1.754011in}{0.586509in}}%
\pgfpathlineto{\pgfqpoint{1.745909in}{0.591661in}}%
\pgfpathlineto{\pgfqpoint{1.737807in}{0.597474in}}%
\pgfpathlineto{\pgfqpoint{1.729705in}{0.603020in}}%
\pgfpathlineto{\pgfqpoint{1.721602in}{0.608866in}}%
\pgfpathlineto{\pgfqpoint{1.713500in}{0.614479in}}%
\pgfpathlineto{\pgfqpoint{1.705398in}{0.619954in}}%
\pgfpathlineto{\pgfqpoint{1.697295in}{0.625956in}}%
\pgfpathlineto{\pgfqpoint{1.689193in}{0.620115in}}%
\pgfpathlineto{\pgfqpoint{1.681091in}{0.617377in}}%
\pgfpathlineto{\pgfqpoint{1.672989in}{0.623335in}}%
\pgfpathlineto{\pgfqpoint{1.664886in}{0.627948in}}%
\pgfpathlineto{\pgfqpoint{1.656784in}{0.634126in}}%
\pgfpathlineto{\pgfqpoint{1.648682in}{0.639663in}}%
\pgfpathlineto{\pgfqpoint{1.640580in}{0.645482in}}%
\pgfpathlineto{\pgfqpoint{1.632477in}{0.651491in}}%
\pgfpathlineto{\pgfqpoint{1.624375in}{0.657425in}}%
\pgfpathlineto{\pgfqpoint{1.616273in}{0.663749in}}%
\pgfpathlineto{\pgfqpoint{1.608170in}{0.671362in}}%
\pgfpathlineto{\pgfqpoint{1.600068in}{0.677311in}}%
\pgfpathlineto{\pgfqpoint{1.591966in}{0.683445in}}%
\pgfpathlineto{\pgfqpoint{1.583864in}{0.689381in}}%
\pgfpathlineto{\pgfqpoint{1.575761in}{0.696003in}}%
\pgfpathlineto{\pgfqpoint{1.567659in}{0.698435in}}%
\pgfpathlineto{\pgfqpoint{1.559557in}{0.698291in}}%
\pgfpathlineto{\pgfqpoint{1.551455in}{0.704456in}}%
\pgfpathlineto{\pgfqpoint{1.543352in}{0.710737in}}%
\pgfpathlineto{\pgfqpoint{1.535250in}{0.717598in}}%
\pgfpathlineto{\pgfqpoint{1.527148in}{0.724403in}}%
\pgfpathlineto{\pgfqpoint{1.519045in}{0.731566in}}%
\pgfpathlineto{\pgfqpoint{1.510943in}{0.738535in}}%
\pgfpathlineto{\pgfqpoint{1.502841in}{0.745506in}}%
\pgfpathlineto{\pgfqpoint{1.494739in}{0.752532in}}%
\pgfpathlineto{\pgfqpoint{1.486636in}{0.751311in}}%
\pgfpathlineto{\pgfqpoint{1.478534in}{0.758441in}}%
\pgfpathlineto{\pgfqpoint{1.470432in}{0.765564in}}%
\pgfpathlineto{\pgfqpoint{1.462330in}{0.772785in}}%
\pgfpathlineto{\pgfqpoint{1.454227in}{0.779755in}}%
\pgfpathlineto{\pgfqpoint{1.446125in}{0.786925in}}%
\pgfpathlineto{\pgfqpoint{1.438023in}{0.794191in}}%
\pgfpathlineto{\pgfqpoint{1.429920in}{0.801590in}}%
\pgfpathlineto{\pgfqpoint{1.421818in}{0.809186in}}%
\pgfpathlineto{\pgfqpoint{1.413716in}{0.816342in}}%
\pgfpathlineto{\pgfqpoint{1.405614in}{0.823993in}}%
\pgfpathlineto{\pgfqpoint{1.397511in}{0.831790in}}%
\pgfpathlineto{\pgfqpoint{1.389409in}{0.840703in}}%
\pgfpathlineto{\pgfqpoint{1.381307in}{0.848918in}}%
\pgfpathlineto{\pgfqpoint{1.373205in}{0.856723in}}%
\pgfpathlineto{\pgfqpoint{1.365102in}{0.864668in}}%
\pgfpathlineto{\pgfqpoint{1.357000in}{0.872455in}}%
\pgfpathlineto{\pgfqpoint{1.348898in}{0.880196in}}%
\pgfpathlineto{\pgfqpoint{1.340795in}{0.887878in}}%
\pgfpathlineto{\pgfqpoint{1.332693in}{0.895555in}}%
\pgfpathlineto{\pgfqpoint{1.324591in}{0.903605in}}%
\pgfpathlineto{\pgfqpoint{1.316489in}{0.912082in}}%
\pgfpathlineto{\pgfqpoint{1.308386in}{0.911063in}}%
\pgfpathlineto{\pgfqpoint{1.300284in}{0.919915in}}%
\pgfpathlineto{\pgfqpoint{1.292182in}{0.927770in}}%
\pgfpathlineto{\pgfqpoint{1.284080in}{0.931707in}}%
\pgfpathlineto{\pgfqpoint{1.275977in}{0.938929in}}%
\pgfpathlineto{\pgfqpoint{1.267875in}{0.943367in}}%
\pgfpathlineto{\pgfqpoint{1.259773in}{0.953396in}}%
\pgfpathlineto{\pgfqpoint{1.251670in}{0.961682in}}%
\pgfpathlineto{\pgfqpoint{1.243568in}{0.970720in}}%
\pgfpathlineto{\pgfqpoint{1.235466in}{0.979538in}}%
\pgfpathlineto{\pgfqpoint{1.227364in}{0.989142in}}%
\pgfpathlineto{\pgfqpoint{1.219261in}{0.988012in}}%
\pgfpathlineto{\pgfqpoint{1.211159in}{0.997000in}}%
\pgfpathlineto{\pgfqpoint{1.203057in}{1.005026in}}%
\pgfpathlineto{\pgfqpoint{1.194955in}{1.013889in}}%
\pgfpathlineto{\pgfqpoint{1.186852in}{1.023144in}}%
\pgfpathlineto{\pgfqpoint{1.178750in}{1.033545in}}%
\pgfpathlineto{\pgfqpoint{1.170648in}{1.042767in}}%
\pgfpathlineto{\pgfqpoint{1.162545in}{1.052396in}}%
\pgfpathlineto{\pgfqpoint{1.154443in}{1.062451in}}%
\pgfpathlineto{\pgfqpoint{1.146341in}{1.071775in}}%
\pgfpathlineto{\pgfqpoint{1.138239in}{1.081770in}}%
\pgfpathlineto{\pgfqpoint{1.130136in}{1.091195in}}%
\pgfpathlineto{\pgfqpoint{1.122034in}{1.090986in}}%
\pgfpathlineto{\pgfqpoint{1.113932in}{1.101372in}}%
\pgfpathlineto{\pgfqpoint{1.105830in}{1.110891in}}%
\pgfpathlineto{\pgfqpoint{1.097727in}{1.133485in}}%
\pgfpathlineto{\pgfqpoint{1.089625in}{1.142273in}}%
\pgfpathlineto{\pgfqpoint{1.081523in}{1.146609in}}%
\pgfpathlineto{\pgfqpoint{1.073420in}{1.157610in}}%
\pgfpathlineto{\pgfqpoint{1.065318in}{1.169245in}}%
\pgfpathlineto{\pgfqpoint{1.057216in}{1.179955in}}%
\pgfpathlineto{\pgfqpoint{1.049114in}{1.188890in}}%
\pgfpathlineto{\pgfqpoint{1.041011in}{1.200054in}}%
\pgfpathlineto{\pgfqpoint{1.032909in}{1.211257in}}%
\pgfpathlineto{\pgfqpoint{1.024807in}{1.222377in}}%
\pgfpathlineto{\pgfqpoint{1.016705in}{1.234078in}}%
\pgfpathlineto{\pgfqpoint{1.008602in}{1.240844in}}%
\pgfpathlineto{\pgfqpoint{1.000500in}{1.253353in}}%
\pgfpathlineto{\pgfqpoint{0.992398in}{1.262006in}}%
\pgfpathlineto{\pgfqpoint{0.984295in}{1.266646in}}%
\pgfpathlineto{\pgfqpoint{0.976193in}{1.278229in}}%
\pgfpathlineto{\pgfqpoint{0.968091in}{1.292226in}}%
\pgfpathlineto{\pgfqpoint{0.959989in}{1.306651in}}%
\pgfpathlineto{\pgfqpoint{0.951886in}{1.321082in}}%
\pgfpathlineto{\pgfqpoint{0.943784in}{1.327358in}}%
\pgfpathlineto{\pgfqpoint{0.935682in}{1.340385in}}%
\pgfpathlineto{\pgfqpoint{0.927580in}{1.354553in}}%
\pgfpathlineto{\pgfqpoint{0.919477in}{1.368135in}}%
\pgfpathlineto{\pgfqpoint{0.911375in}{1.379239in}}%
\pgfpathlineto{\pgfqpoint{0.903273in}{1.391023in}}%
\pgfpathlineto{\pgfqpoint{0.895170in}{1.404927in}}%
\pgfpathlineto{\pgfqpoint{0.887068in}{1.416119in}}%
\pgfpathlineto{\pgfqpoint{0.878966in}{1.422566in}}%
\pgfpathlineto{\pgfqpoint{0.870864in}{1.436756in}}%
\pgfpathlineto{\pgfqpoint{0.862761in}{1.434422in}}%
\pgfpathlineto{\pgfqpoint{0.854659in}{1.444286in}}%
\pgfpathlineto{\pgfqpoint{0.846557in}{1.449120in}}%
\pgfpathlineto{\pgfqpoint{0.838455in}{1.458351in}}%
\pgfpathlineto{\pgfqpoint{0.830352in}{1.469969in}}%
\pgfpathlineto{\pgfqpoint{0.822250in}{1.475442in}}%
\pgfpathlineto{\pgfqpoint{0.814148in}{1.480253in}}%
\pgfpathlineto{\pgfqpoint{0.806045in}{1.486278in}}%
\pgfpathlineto{\pgfqpoint{0.797943in}{1.495671in}}%
\pgfpathlineto{\pgfqpoint{0.789841in}{1.501334in}}%
\pgfpathlineto{\pgfqpoint{0.781739in}{1.516024in}}%
\pgfpathlineto{\pgfqpoint{0.773636in}{1.525440in}}%
\pgfpathlineto{\pgfqpoint{0.765534in}{1.532318in}}%
\pgfpathlineto{\pgfqpoint{0.757432in}{1.546123in}}%
\pgfpathlineto{\pgfqpoint{0.749330in}{1.551023in}}%
\pgfpathlineto{\pgfqpoint{0.741227in}{1.559564in}}%
\pgfpathlineto{\pgfqpoint{0.733125in}{1.557368in}}%
\pgfpathlineto{\pgfqpoint{0.725023in}{1.564592in}}%
\pgfpathlineto{\pgfqpoint{0.716920in}{1.568999in}}%
\pgfpathlineto{\pgfqpoint{0.708818in}{1.568401in}}%
\pgfpathlineto{\pgfqpoint{0.700716in}{1.570094in}}%
\pgfpathlineto{\pgfqpoint{0.692614in}{1.563749in}}%
\pgfpathlineto{\pgfqpoint{0.684511in}{1.561500in}}%
\pgfpathlineto{\pgfqpoint{0.676409in}{1.567706in}}%
\pgfpathlineto{\pgfqpoint{0.668307in}{1.568830in}}%
\pgfpathlineto{\pgfqpoint{0.660205in}{1.569131in}}%
\pgfpathlineto{\pgfqpoint{0.652102in}{1.583188in}}%
\pgfpathlineto{\pgfqpoint{0.644000in}{1.574655in}}%
\pgfpathlineto{\pgfqpoint{0.635898in}{1.593655in}}%
\pgfpathlineto{\pgfqpoint{0.627795in}{1.598554in}}%
\pgfpathlineto{\pgfqpoint{0.619693in}{1.604768in}}%
\pgfpathlineto{\pgfqpoint{0.611591in}{1.600013in}}%
\pgfpathlineto{\pgfqpoint{0.603489in}{1.598496in}}%
\pgfpathlineto{\pgfqpoint{0.595386in}{1.605111in}}%
\pgfpathlineto{\pgfqpoint{0.587284in}{1.587512in}}%
\pgfpathlineto{\pgfqpoint{0.579182in}{1.571606in}}%
\pgfpathlineto{\pgfqpoint{0.571080in}{1.582506in}}%
\pgfpathlineto{\pgfqpoint{0.562977in}{1.565547in}}%
\pgfpathlineto{\pgfqpoint{0.554875in}{1.590614in}}%
\pgfpathlineto{\pgfqpoint{0.546773in}{1.593244in}}%
\pgfpathlineto{\pgfqpoint{0.538670in}{1.585198in}}%
\pgfpathlineto{\pgfqpoint{0.530568in}{1.609422in}}%
\pgfpathlineto{\pgfqpoint{0.522466in}{1.597396in}}%
\pgfpathlineto{\pgfqpoint{0.514364in}{1.622406in}}%
\pgfpathlineto{\pgfqpoint{0.506261in}{1.610254in}}%
\pgfpathlineto{\pgfqpoint{0.498159in}{1.610738in}}%
\pgfpathlineto{\pgfqpoint{0.490057in}{1.589673in}}%
\pgfpathlineto{\pgfqpoint{0.481955in}{1.558761in}}%
\pgfpathlineto{\pgfqpoint{0.473852in}{1.570066in}}%
\pgfpathlineto{\pgfqpoint{0.465750in}{1.548598in}}%
\pgfpathlineto{\pgfqpoint{0.457648in}{1.575610in}}%
\pgfpathlineto{\pgfqpoint{0.449545in}{1.581267in}}%
\pgfpathlineto{\pgfqpoint{0.441443in}{1.623998in}}%
\pgfpathlineto{\pgfqpoint{0.433341in}{1.613806in}}%
\pgfpathlineto{\pgfqpoint{0.425239in}{1.641768in}}%
\pgfpathlineto{\pgfqpoint{0.417136in}{1.622140in}}%
\pgfpathlineto{\pgfqpoint{0.409034in}{1.605732in}}%
\pgfpathlineto{\pgfqpoint{0.400932in}{1.643322in}}%
\pgfpathlineto{\pgfqpoint{0.392830in}{1.659012in}}%
\pgfpathlineto{\pgfqpoint{0.384727in}{1.702170in}}%
\pgfpathlineto{\pgfqpoint{0.376625in}{1.703307in}}%
\pgfpathlineto{\pgfqpoint{0.368523in}{1.553724in}}%
\pgfpathclose%
\pgfusepath{fill}%
\end{pgfscope}%
\begin{pgfscope}%
\pgfpathrectangle{\pgfqpoint{0.287500in}{0.375000in}}{\pgfqpoint{1.782500in}{2.265000in}}%
\pgfusepath{clip}%
\pgfsetbuttcap%
\pgfsetroundjoin%
\definecolor{currentfill}{rgb}{1.000000,0.498039,0.054902}%
\pgfsetfillcolor{currentfill}%
\pgfsetfillopacity{0.200000}%
\pgfsetlinewidth{0.000000pt}%
\definecolor{currentstroke}{rgb}{0.000000,0.000000,0.000000}%
\pgfsetstrokecolor{currentstroke}%
\pgfsetdash{}{0pt}%
\pgfpathmoveto{\pgfqpoint{0.368523in}{1.881063in}}%
\pgfpathlineto{\pgfqpoint{0.368523in}{2.477209in}}%
\pgfpathlineto{\pgfqpoint{0.376625in}{2.170677in}}%
\pgfpathlineto{\pgfqpoint{0.384727in}{2.028661in}}%
\pgfpathlineto{\pgfqpoint{0.392830in}{1.935522in}}%
\pgfpathlineto{\pgfqpoint{0.400932in}{1.953882in}}%
\pgfpathlineto{\pgfqpoint{0.409034in}{1.916067in}}%
\pgfpathlineto{\pgfqpoint{0.417136in}{1.867678in}}%
\pgfpathlineto{\pgfqpoint{0.425239in}{1.842438in}}%
\pgfpathlineto{\pgfqpoint{0.433341in}{1.832219in}}%
\pgfpathlineto{\pgfqpoint{0.441443in}{1.802327in}}%
\pgfpathlineto{\pgfqpoint{0.449545in}{1.787329in}}%
\pgfpathlineto{\pgfqpoint{0.457648in}{1.770168in}}%
\pgfpathlineto{\pgfqpoint{0.465750in}{1.781523in}}%
\pgfpathlineto{\pgfqpoint{0.473852in}{1.780997in}}%
\pgfpathlineto{\pgfqpoint{0.481955in}{1.788840in}}%
\pgfpathlineto{\pgfqpoint{0.490057in}{1.753430in}}%
\pgfpathlineto{\pgfqpoint{0.498159in}{1.760763in}}%
\pgfpathlineto{\pgfqpoint{0.506261in}{1.748864in}}%
\pgfpathlineto{\pgfqpoint{0.514364in}{1.736378in}}%
\pgfpathlineto{\pgfqpoint{0.522466in}{1.718635in}}%
\pgfpathlineto{\pgfqpoint{0.530568in}{1.713044in}}%
\pgfpathlineto{\pgfqpoint{0.538670in}{1.704338in}}%
\pgfpathlineto{\pgfqpoint{0.546773in}{1.695473in}}%
\pgfpathlineto{\pgfqpoint{0.554875in}{1.687521in}}%
\pgfpathlineto{\pgfqpoint{0.562977in}{1.679430in}}%
\pgfpathlineto{\pgfqpoint{0.571080in}{1.678832in}}%
\pgfpathlineto{\pgfqpoint{0.579182in}{1.661872in}}%
\pgfpathlineto{\pgfqpoint{0.587284in}{1.653526in}}%
\pgfpathlineto{\pgfqpoint{0.595386in}{1.660569in}}%
\pgfpathlineto{\pgfqpoint{0.603489in}{1.663359in}}%
\pgfpathlineto{\pgfqpoint{0.611591in}{1.663056in}}%
\pgfpathlineto{\pgfqpoint{0.619693in}{1.664216in}}%
\pgfpathlineto{\pgfqpoint{0.627795in}{1.659880in}}%
\pgfpathlineto{\pgfqpoint{0.635898in}{1.663968in}}%
\pgfpathlineto{\pgfqpoint{0.644000in}{1.654443in}}%
\pgfpathlineto{\pgfqpoint{0.652102in}{1.653776in}}%
\pgfpathlineto{\pgfqpoint{0.660205in}{1.648648in}}%
\pgfpathlineto{\pgfqpoint{0.668307in}{1.650469in}}%
\pgfpathlineto{\pgfqpoint{0.676409in}{1.646701in}}%
\pgfpathlineto{\pgfqpoint{0.684511in}{1.651802in}}%
\pgfpathlineto{\pgfqpoint{0.692614in}{1.655421in}}%
\pgfpathlineto{\pgfqpoint{0.700716in}{1.648763in}}%
\pgfpathlineto{\pgfqpoint{0.708818in}{1.640172in}}%
\pgfpathlineto{\pgfqpoint{0.716920in}{1.637163in}}%
\pgfpathlineto{\pgfqpoint{0.725023in}{1.638200in}}%
\pgfpathlineto{\pgfqpoint{0.733125in}{1.641415in}}%
\pgfpathlineto{\pgfqpoint{0.741227in}{1.641961in}}%
\pgfpathlineto{\pgfqpoint{0.749330in}{1.638567in}}%
\pgfpathlineto{\pgfqpoint{0.757432in}{1.633856in}}%
\pgfpathlineto{\pgfqpoint{0.765534in}{1.626340in}}%
\pgfpathlineto{\pgfqpoint{0.773636in}{1.626834in}}%
\pgfpathlineto{\pgfqpoint{0.781739in}{1.630016in}}%
\pgfpathlineto{\pgfqpoint{0.789841in}{1.622432in}}%
\pgfpathlineto{\pgfqpoint{0.797943in}{1.622659in}}%
\pgfpathlineto{\pgfqpoint{0.806045in}{1.632855in}}%
\pgfpathlineto{\pgfqpoint{0.814148in}{1.633711in}}%
\pgfpathlineto{\pgfqpoint{0.822250in}{1.635204in}}%
\pgfpathlineto{\pgfqpoint{0.830352in}{1.638117in}}%
\pgfpathlineto{\pgfqpoint{0.838455in}{1.628286in}}%
\pgfpathlineto{\pgfqpoint{0.846557in}{1.623689in}}%
\pgfpathlineto{\pgfqpoint{0.854659in}{1.618206in}}%
\pgfpathlineto{\pgfqpoint{0.862761in}{1.609580in}}%
\pgfpathlineto{\pgfqpoint{0.870864in}{1.614275in}}%
\pgfpathlineto{\pgfqpoint{0.878966in}{1.605504in}}%
\pgfpathlineto{\pgfqpoint{0.887068in}{1.598751in}}%
\pgfpathlineto{\pgfqpoint{0.895170in}{1.599713in}}%
\pgfpathlineto{\pgfqpoint{0.903273in}{1.593388in}}%
\pgfpathlineto{\pgfqpoint{0.911375in}{1.585282in}}%
\pgfpathlineto{\pgfqpoint{0.919477in}{1.579982in}}%
\pgfpathlineto{\pgfqpoint{0.927580in}{1.571989in}}%
\pgfpathlineto{\pgfqpoint{0.935682in}{1.568178in}}%
\pgfpathlineto{\pgfqpoint{0.943784in}{1.561507in}}%
\pgfpathlineto{\pgfqpoint{0.951886in}{1.565064in}}%
\pgfpathlineto{\pgfqpoint{0.959989in}{1.553810in}}%
\pgfpathlineto{\pgfqpoint{0.968091in}{1.560213in}}%
\pgfpathlineto{\pgfqpoint{0.976193in}{1.555474in}}%
\pgfpathlineto{\pgfqpoint{0.984295in}{1.555598in}}%
\pgfpathlineto{\pgfqpoint{0.992398in}{1.545950in}}%
\pgfpathlineto{\pgfqpoint{1.000500in}{1.537205in}}%
\pgfpathlineto{\pgfqpoint{1.008602in}{1.531451in}}%
\pgfpathlineto{\pgfqpoint{1.016705in}{1.526876in}}%
\pgfpathlineto{\pgfqpoint{1.024807in}{1.521010in}}%
\pgfpathlineto{\pgfqpoint{1.032909in}{1.517154in}}%
\pgfpathlineto{\pgfqpoint{1.041011in}{1.513733in}}%
\pgfpathlineto{\pgfqpoint{1.049114in}{1.515101in}}%
\pgfpathlineto{\pgfqpoint{1.057216in}{1.515023in}}%
\pgfpathlineto{\pgfqpoint{1.065318in}{1.518438in}}%
\pgfpathlineto{\pgfqpoint{1.073420in}{1.517062in}}%
\pgfpathlineto{\pgfqpoint{1.081523in}{1.523088in}}%
\pgfpathlineto{\pgfqpoint{1.089625in}{1.522651in}}%
\pgfpathlineto{\pgfqpoint{1.097727in}{1.526020in}}%
\pgfpathlineto{\pgfqpoint{1.105830in}{1.529259in}}%
\pgfpathlineto{\pgfqpoint{1.113932in}{1.529735in}}%
\pgfpathlineto{\pgfqpoint{1.122034in}{1.530983in}}%
\pgfpathlineto{\pgfqpoint{1.130136in}{1.523514in}}%
\pgfpathlineto{\pgfqpoint{1.138239in}{1.518442in}}%
\pgfpathlineto{\pgfqpoint{1.146341in}{1.517375in}}%
\pgfpathlineto{\pgfqpoint{1.154443in}{1.521944in}}%
\pgfpathlineto{\pgfqpoint{1.162545in}{1.521137in}}%
\pgfpathlineto{\pgfqpoint{1.170648in}{1.523413in}}%
\pgfpathlineto{\pgfqpoint{1.178750in}{1.521808in}}%
\pgfpathlineto{\pgfqpoint{1.186852in}{1.515905in}}%
\pgfpathlineto{\pgfqpoint{1.194955in}{1.512122in}}%
\pgfpathlineto{\pgfqpoint{1.203057in}{1.513227in}}%
\pgfpathlineto{\pgfqpoint{1.211159in}{1.509278in}}%
\pgfpathlineto{\pgfqpoint{1.219261in}{1.506575in}}%
\pgfpathlineto{\pgfqpoint{1.227364in}{1.506121in}}%
\pgfpathlineto{\pgfqpoint{1.235466in}{1.504028in}}%
\pgfpathlineto{\pgfqpoint{1.243568in}{1.505040in}}%
\pgfpathlineto{\pgfqpoint{1.251670in}{1.507721in}}%
\pgfpathlineto{\pgfqpoint{1.259773in}{1.506823in}}%
\pgfpathlineto{\pgfqpoint{1.267875in}{1.508591in}}%
\pgfpathlineto{\pgfqpoint{1.275977in}{1.505123in}}%
\pgfpathlineto{\pgfqpoint{1.284080in}{1.504611in}}%
\pgfpathlineto{\pgfqpoint{1.292182in}{1.507815in}}%
\pgfpathlineto{\pgfqpoint{1.300284in}{1.511232in}}%
\pgfpathlineto{\pgfqpoint{1.308386in}{1.511536in}}%
\pgfpathlineto{\pgfqpoint{1.316489in}{1.509597in}}%
\pgfpathlineto{\pgfqpoint{1.324591in}{1.510482in}}%
\pgfpathlineto{\pgfqpoint{1.332693in}{1.512784in}}%
\pgfpathlineto{\pgfqpoint{1.340795in}{1.512777in}}%
\pgfpathlineto{\pgfqpoint{1.348898in}{1.511308in}}%
\pgfpathlineto{\pgfqpoint{1.357000in}{1.510628in}}%
\pgfpathlineto{\pgfqpoint{1.365102in}{1.506373in}}%
\pgfpathlineto{\pgfqpoint{1.373205in}{1.500278in}}%
\pgfpathlineto{\pgfqpoint{1.381307in}{1.500532in}}%
\pgfpathlineto{\pgfqpoint{1.389409in}{1.499853in}}%
\pgfpathlineto{\pgfqpoint{1.397511in}{1.504730in}}%
\pgfpathlineto{\pgfqpoint{1.405614in}{1.503341in}}%
\pgfpathlineto{\pgfqpoint{1.413716in}{1.498571in}}%
\pgfpathlineto{\pgfqpoint{1.421818in}{1.494858in}}%
\pgfpathlineto{\pgfqpoint{1.429920in}{1.495270in}}%
\pgfpathlineto{\pgfqpoint{1.438023in}{1.494454in}}%
\pgfpathlineto{\pgfqpoint{1.446125in}{1.491341in}}%
\pgfpathlineto{\pgfqpoint{1.454227in}{1.488727in}}%
\pgfpathlineto{\pgfqpoint{1.462330in}{1.488401in}}%
\pgfpathlineto{\pgfqpoint{1.470432in}{1.486073in}}%
\pgfpathlineto{\pgfqpoint{1.478534in}{1.482493in}}%
\pgfpathlineto{\pgfqpoint{1.486636in}{1.479560in}}%
\pgfpathlineto{\pgfqpoint{1.494739in}{1.479944in}}%
\pgfpathlineto{\pgfqpoint{1.502841in}{1.476474in}}%
\pgfpathlineto{\pgfqpoint{1.510943in}{1.476022in}}%
\pgfpathlineto{\pgfqpoint{1.519045in}{1.470771in}}%
\pgfpathlineto{\pgfqpoint{1.527148in}{1.471363in}}%
\pgfpathlineto{\pgfqpoint{1.535250in}{1.470606in}}%
\pgfpathlineto{\pgfqpoint{1.543352in}{1.467850in}}%
\pgfpathlineto{\pgfqpoint{1.551455in}{1.466850in}}%
\pgfpathlineto{\pgfqpoint{1.559557in}{1.465484in}}%
\pgfpathlineto{\pgfqpoint{1.567659in}{1.463876in}}%
\pgfpathlineto{\pgfqpoint{1.575761in}{1.462758in}}%
\pgfpathlineto{\pgfqpoint{1.583864in}{1.461195in}}%
\pgfpathlineto{\pgfqpoint{1.591966in}{1.457372in}}%
\pgfpathlineto{\pgfqpoint{1.600068in}{1.454206in}}%
\pgfpathlineto{\pgfqpoint{1.608170in}{1.449458in}}%
\pgfpathlineto{\pgfqpoint{1.616273in}{1.449089in}}%
\pgfpathlineto{\pgfqpoint{1.624375in}{1.446512in}}%
\pgfpathlineto{\pgfqpoint{1.632477in}{1.443046in}}%
\pgfpathlineto{\pgfqpoint{1.640580in}{1.441409in}}%
\pgfpathlineto{\pgfqpoint{1.648682in}{1.442835in}}%
\pgfpathlineto{\pgfqpoint{1.656784in}{1.440103in}}%
\pgfpathlineto{\pgfqpoint{1.664886in}{1.437094in}}%
\pgfpathlineto{\pgfqpoint{1.672989in}{1.432797in}}%
\pgfpathlineto{\pgfqpoint{1.681091in}{1.427490in}}%
\pgfpathlineto{\pgfqpoint{1.689193in}{1.422026in}}%
\pgfpathlineto{\pgfqpoint{1.697295in}{1.422620in}}%
\pgfpathlineto{\pgfqpoint{1.705398in}{1.421160in}}%
\pgfpathlineto{\pgfqpoint{1.713500in}{1.417841in}}%
\pgfpathlineto{\pgfqpoint{1.721602in}{1.412829in}}%
\pgfpathlineto{\pgfqpoint{1.729705in}{1.414398in}}%
\pgfpathlineto{\pgfqpoint{1.737807in}{1.413691in}}%
\pgfpathlineto{\pgfqpoint{1.745909in}{1.413774in}}%
\pgfpathlineto{\pgfqpoint{1.754011in}{1.409050in}}%
\pgfpathlineto{\pgfqpoint{1.762114in}{1.409026in}}%
\pgfpathlineto{\pgfqpoint{1.770216in}{1.404528in}}%
\pgfpathlineto{\pgfqpoint{1.778318in}{1.404839in}}%
\pgfpathlineto{\pgfqpoint{1.786420in}{1.403953in}}%
\pgfpathlineto{\pgfqpoint{1.794523in}{1.402130in}}%
\pgfpathlineto{\pgfqpoint{1.802625in}{1.399846in}}%
\pgfpathlineto{\pgfqpoint{1.810727in}{1.401353in}}%
\pgfpathlineto{\pgfqpoint{1.818830in}{1.397763in}}%
\pgfpathlineto{\pgfqpoint{1.826932in}{1.393161in}}%
\pgfpathlineto{\pgfqpoint{1.835034in}{1.389383in}}%
\pgfpathlineto{\pgfqpoint{1.843136in}{1.386126in}}%
\pgfpathlineto{\pgfqpoint{1.851239in}{1.387003in}}%
\pgfpathlineto{\pgfqpoint{1.859341in}{1.385442in}}%
\pgfpathlineto{\pgfqpoint{1.867443in}{1.385641in}}%
\pgfpathlineto{\pgfqpoint{1.875545in}{1.382806in}}%
\pgfpathlineto{\pgfqpoint{1.883648in}{1.381346in}}%
\pgfpathlineto{\pgfqpoint{1.891750in}{1.378479in}}%
\pgfpathlineto{\pgfqpoint{1.899852in}{1.374647in}}%
\pgfpathlineto{\pgfqpoint{1.907955in}{1.373342in}}%
\pgfpathlineto{\pgfqpoint{1.916057in}{1.370948in}}%
\pgfpathlineto{\pgfqpoint{1.924159in}{1.370646in}}%
\pgfpathlineto{\pgfqpoint{1.932261in}{1.366982in}}%
\pgfpathlineto{\pgfqpoint{1.940364in}{1.363501in}}%
\pgfpathlineto{\pgfqpoint{1.948466in}{1.359266in}}%
\pgfpathlineto{\pgfqpoint{1.956568in}{1.357263in}}%
\pgfpathlineto{\pgfqpoint{1.964670in}{1.353076in}}%
\pgfpathlineto{\pgfqpoint{1.972773in}{1.350425in}}%
\pgfpathlineto{\pgfqpoint{1.980875in}{1.349293in}}%
\pgfpathlineto{\pgfqpoint{1.988977in}{1.344038in}}%
\pgfpathlineto{\pgfqpoint{1.988977in}{1.075073in}}%
\pgfpathlineto{\pgfqpoint{1.988977in}{1.075073in}}%
\pgfpathlineto{\pgfqpoint{1.980875in}{1.071464in}}%
\pgfpathlineto{\pgfqpoint{1.972773in}{1.076026in}}%
\pgfpathlineto{\pgfqpoint{1.964670in}{1.079424in}}%
\pgfpathlineto{\pgfqpoint{1.956568in}{1.079175in}}%
\pgfpathlineto{\pgfqpoint{1.948466in}{1.079288in}}%
\pgfpathlineto{\pgfqpoint{1.940364in}{1.083979in}}%
\pgfpathlineto{\pgfqpoint{1.932261in}{1.085065in}}%
\pgfpathlineto{\pgfqpoint{1.924159in}{1.088566in}}%
\pgfpathlineto{\pgfqpoint{1.916057in}{1.089400in}}%
\pgfpathlineto{\pgfqpoint{1.907955in}{1.094271in}}%
\pgfpathlineto{\pgfqpoint{1.899852in}{1.095884in}}%
\pgfpathlineto{\pgfqpoint{1.891750in}{1.099985in}}%
\pgfpathlineto{\pgfqpoint{1.883648in}{1.104672in}}%
\pgfpathlineto{\pgfqpoint{1.875545in}{1.107215in}}%
\pgfpathlineto{\pgfqpoint{1.867443in}{1.106741in}}%
\pgfpathlineto{\pgfqpoint{1.859341in}{1.110551in}}%
\pgfpathlineto{\pgfqpoint{1.851239in}{1.113816in}}%
\pgfpathlineto{\pgfqpoint{1.843136in}{1.114181in}}%
\pgfpathlineto{\pgfqpoint{1.835034in}{1.118517in}}%
\pgfpathlineto{\pgfqpoint{1.826932in}{1.122896in}}%
\pgfpathlineto{\pgfqpoint{1.818830in}{1.127515in}}%
\pgfpathlineto{\pgfqpoint{1.810727in}{1.130933in}}%
\pgfpathlineto{\pgfqpoint{1.802625in}{1.131365in}}%
\pgfpathlineto{\pgfqpoint{1.794523in}{1.136160in}}%
\pgfpathlineto{\pgfqpoint{1.786420in}{1.140899in}}%
\pgfpathlineto{\pgfqpoint{1.778318in}{1.139047in}}%
\pgfpathlineto{\pgfqpoint{1.770216in}{1.143483in}}%
\pgfpathlineto{\pgfqpoint{1.762114in}{1.144400in}}%
\pgfpathlineto{\pgfqpoint{1.754011in}{1.147396in}}%
\pgfpathlineto{\pgfqpoint{1.745909in}{1.150418in}}%
\pgfpathlineto{\pgfqpoint{1.737807in}{1.154167in}}%
\pgfpathlineto{\pgfqpoint{1.729705in}{1.157406in}}%
\pgfpathlineto{\pgfqpoint{1.721602in}{1.162076in}}%
\pgfpathlineto{\pgfqpoint{1.713500in}{1.167466in}}%
\pgfpathlineto{\pgfqpoint{1.705398in}{1.172684in}}%
\pgfpathlineto{\pgfqpoint{1.697295in}{1.177839in}}%
\pgfpathlineto{\pgfqpoint{1.689193in}{1.179611in}}%
\pgfpathlineto{\pgfqpoint{1.681091in}{1.184537in}}%
\pgfpathlineto{\pgfqpoint{1.672989in}{1.188858in}}%
\pgfpathlineto{\pgfqpoint{1.664886in}{1.191740in}}%
\pgfpathlineto{\pgfqpoint{1.656784in}{1.185379in}}%
\pgfpathlineto{\pgfqpoint{1.648682in}{1.190021in}}%
\pgfpathlineto{\pgfqpoint{1.640580in}{1.191995in}}%
\pgfpathlineto{\pgfqpoint{1.632477in}{1.196826in}}%
\pgfpathlineto{\pgfqpoint{1.624375in}{1.194232in}}%
\pgfpathlineto{\pgfqpoint{1.616273in}{1.199195in}}%
\pgfpathlineto{\pgfqpoint{1.608170in}{1.203204in}}%
\pgfpathlineto{\pgfqpoint{1.600068in}{1.206447in}}%
\pgfpathlineto{\pgfqpoint{1.591966in}{1.211369in}}%
\pgfpathlineto{\pgfqpoint{1.583864in}{1.216962in}}%
\pgfpathlineto{\pgfqpoint{1.575761in}{1.217914in}}%
\pgfpathlineto{\pgfqpoint{1.567659in}{1.222464in}}%
\pgfpathlineto{\pgfqpoint{1.559557in}{1.225670in}}%
\pgfpathlineto{\pgfqpoint{1.551455in}{1.231534in}}%
\pgfpathlineto{\pgfqpoint{1.543352in}{1.237294in}}%
\pgfpathlineto{\pgfqpoint{1.535250in}{1.242464in}}%
\pgfpathlineto{\pgfqpoint{1.527148in}{1.243054in}}%
\pgfpathlineto{\pgfqpoint{1.519045in}{1.247818in}}%
\pgfpathlineto{\pgfqpoint{1.510943in}{1.245292in}}%
\pgfpathlineto{\pgfqpoint{1.502841in}{1.247496in}}%
\pgfpathlineto{\pgfqpoint{1.494739in}{1.252479in}}%
\pgfpathlineto{\pgfqpoint{1.486636in}{1.253710in}}%
\pgfpathlineto{\pgfqpoint{1.478534in}{1.260012in}}%
\pgfpathlineto{\pgfqpoint{1.470432in}{1.263217in}}%
\pgfpathlineto{\pgfqpoint{1.462330in}{1.268808in}}%
\pgfpathlineto{\pgfqpoint{1.454227in}{1.270698in}}%
\pgfpathlineto{\pgfqpoint{1.446125in}{1.276870in}}%
\pgfpathlineto{\pgfqpoint{1.438023in}{1.280208in}}%
\pgfpathlineto{\pgfqpoint{1.429920in}{1.284624in}}%
\pgfpathlineto{\pgfqpoint{1.421818in}{1.286505in}}%
\pgfpathlineto{\pgfqpoint{1.413716in}{1.291045in}}%
\pgfpathlineto{\pgfqpoint{1.405614in}{1.297068in}}%
\pgfpathlineto{\pgfqpoint{1.397511in}{1.300023in}}%
\pgfpathlineto{\pgfqpoint{1.389409in}{1.302533in}}%
\pgfpathlineto{\pgfqpoint{1.381307in}{1.305055in}}%
\pgfpathlineto{\pgfqpoint{1.373205in}{1.307551in}}%
\pgfpathlineto{\pgfqpoint{1.365102in}{1.304114in}}%
\pgfpathlineto{\pgfqpoint{1.357000in}{1.308486in}}%
\pgfpathlineto{\pgfqpoint{1.348898in}{1.311971in}}%
\pgfpathlineto{\pgfqpoint{1.340795in}{1.316693in}}%
\pgfpathlineto{\pgfqpoint{1.332693in}{1.316343in}}%
\pgfpathlineto{\pgfqpoint{1.324591in}{1.322959in}}%
\pgfpathlineto{\pgfqpoint{1.316489in}{1.327419in}}%
\pgfpathlineto{\pgfqpoint{1.308386in}{1.332014in}}%
\pgfpathlineto{\pgfqpoint{1.300284in}{1.337460in}}%
\pgfpathlineto{\pgfqpoint{1.292182in}{1.342534in}}%
\pgfpathlineto{\pgfqpoint{1.284080in}{1.347797in}}%
\pgfpathlineto{\pgfqpoint{1.275977in}{1.354791in}}%
\pgfpathlineto{\pgfqpoint{1.267875in}{1.355274in}}%
\pgfpathlineto{\pgfqpoint{1.259773in}{1.360849in}}%
\pgfpathlineto{\pgfqpoint{1.251670in}{1.360716in}}%
\pgfpathlineto{\pgfqpoint{1.243568in}{1.361083in}}%
\pgfpathlineto{\pgfqpoint{1.235466in}{1.363144in}}%
\pgfpathlineto{\pgfqpoint{1.227364in}{1.369010in}}%
\pgfpathlineto{\pgfqpoint{1.219261in}{1.374433in}}%
\pgfpathlineto{\pgfqpoint{1.211159in}{1.379367in}}%
\pgfpathlineto{\pgfqpoint{1.203057in}{1.385849in}}%
\pgfpathlineto{\pgfqpoint{1.194955in}{1.392546in}}%
\pgfpathlineto{\pgfqpoint{1.186852in}{1.397295in}}%
\pgfpathlineto{\pgfqpoint{1.178750in}{1.404240in}}%
\pgfpathlineto{\pgfqpoint{1.170648in}{1.408883in}}%
\pgfpathlineto{\pgfqpoint{1.162545in}{1.412001in}}%
\pgfpathlineto{\pgfqpoint{1.154443in}{1.418396in}}%
\pgfpathlineto{\pgfqpoint{1.146341in}{1.421955in}}%
\pgfpathlineto{\pgfqpoint{1.138239in}{1.429037in}}%
\pgfpathlineto{\pgfqpoint{1.130136in}{1.434760in}}%
\pgfpathlineto{\pgfqpoint{1.122034in}{1.438550in}}%
\pgfpathlineto{\pgfqpoint{1.113932in}{1.445794in}}%
\pgfpathlineto{\pgfqpoint{1.105830in}{1.456117in}}%
\pgfpathlineto{\pgfqpoint{1.097727in}{1.453914in}}%
\pgfpathlineto{\pgfqpoint{1.089625in}{1.457334in}}%
\pgfpathlineto{\pgfqpoint{1.081523in}{1.465130in}}%
\pgfpathlineto{\pgfqpoint{1.073420in}{1.466016in}}%
\pgfpathlineto{\pgfqpoint{1.065318in}{1.470391in}}%
\pgfpathlineto{\pgfqpoint{1.057216in}{1.475997in}}%
\pgfpathlineto{\pgfqpoint{1.049114in}{1.479215in}}%
\pgfpathlineto{\pgfqpoint{1.041011in}{1.477877in}}%
\pgfpathlineto{\pgfqpoint{1.032909in}{1.487212in}}%
\pgfpathlineto{\pgfqpoint{1.024807in}{1.477778in}}%
\pgfpathlineto{\pgfqpoint{1.016705in}{1.481411in}}%
\pgfpathlineto{\pgfqpoint{1.008602in}{1.488282in}}%
\pgfpathlineto{\pgfqpoint{1.000500in}{1.492756in}}%
\pgfpathlineto{\pgfqpoint{0.992398in}{1.497061in}}%
\pgfpathlineto{\pgfqpoint{0.984295in}{1.497269in}}%
\pgfpathlineto{\pgfqpoint{0.976193in}{1.498092in}}%
\pgfpathlineto{\pgfqpoint{0.968091in}{1.496080in}}%
\pgfpathlineto{\pgfqpoint{0.959989in}{1.501716in}}%
\pgfpathlineto{\pgfqpoint{0.951886in}{1.509674in}}%
\pgfpathlineto{\pgfqpoint{0.943784in}{1.506022in}}%
\pgfpathlineto{\pgfqpoint{0.935682in}{1.511349in}}%
\pgfpathlineto{\pgfqpoint{0.927580in}{1.510409in}}%
\pgfpathlineto{\pgfqpoint{0.919477in}{1.519701in}}%
\pgfpathlineto{\pgfqpoint{0.911375in}{1.517751in}}%
\pgfpathlineto{\pgfqpoint{0.903273in}{1.519259in}}%
\pgfpathlineto{\pgfqpoint{0.895170in}{1.525828in}}%
\pgfpathlineto{\pgfqpoint{0.887068in}{1.526208in}}%
\pgfpathlineto{\pgfqpoint{0.878966in}{1.534600in}}%
\pgfpathlineto{\pgfqpoint{0.870864in}{1.538474in}}%
\pgfpathlineto{\pgfqpoint{0.862761in}{1.539373in}}%
\pgfpathlineto{\pgfqpoint{0.854659in}{1.535092in}}%
\pgfpathlineto{\pgfqpoint{0.846557in}{1.539673in}}%
\pgfpathlineto{\pgfqpoint{0.838455in}{1.550088in}}%
\pgfpathlineto{\pgfqpoint{0.830352in}{1.550739in}}%
\pgfpathlineto{\pgfqpoint{0.822250in}{1.557234in}}%
\pgfpathlineto{\pgfqpoint{0.814148in}{1.564045in}}%
\pgfpathlineto{\pgfqpoint{0.806045in}{1.562015in}}%
\pgfpathlineto{\pgfqpoint{0.797943in}{1.571751in}}%
\pgfpathlineto{\pgfqpoint{0.789841in}{1.568510in}}%
\pgfpathlineto{\pgfqpoint{0.781739in}{1.569995in}}%
\pgfpathlineto{\pgfqpoint{0.773636in}{1.562024in}}%
\pgfpathlineto{\pgfqpoint{0.765534in}{1.564368in}}%
\pgfpathlineto{\pgfqpoint{0.757432in}{1.566478in}}%
\pgfpathlineto{\pgfqpoint{0.749330in}{1.570767in}}%
\pgfpathlineto{\pgfqpoint{0.741227in}{1.569210in}}%
\pgfpathlineto{\pgfqpoint{0.733125in}{1.565877in}}%
\pgfpathlineto{\pgfqpoint{0.725023in}{1.574274in}}%
\pgfpathlineto{\pgfqpoint{0.716920in}{1.572367in}}%
\pgfpathlineto{\pgfqpoint{0.708818in}{1.579517in}}%
\pgfpathlineto{\pgfqpoint{0.700716in}{1.575182in}}%
\pgfpathlineto{\pgfqpoint{0.692614in}{1.577925in}}%
\pgfpathlineto{\pgfqpoint{0.684511in}{1.586790in}}%
\pgfpathlineto{\pgfqpoint{0.676409in}{1.582439in}}%
\pgfpathlineto{\pgfqpoint{0.668307in}{1.581707in}}%
\pgfpathlineto{\pgfqpoint{0.660205in}{1.583930in}}%
\pgfpathlineto{\pgfqpoint{0.652102in}{1.589563in}}%
\pgfpathlineto{\pgfqpoint{0.644000in}{1.597953in}}%
\pgfpathlineto{\pgfqpoint{0.635898in}{1.589782in}}%
\pgfpathlineto{\pgfqpoint{0.627795in}{1.582545in}}%
\pgfpathlineto{\pgfqpoint{0.619693in}{1.582863in}}%
\pgfpathlineto{\pgfqpoint{0.611591in}{1.575109in}}%
\pgfpathlineto{\pgfqpoint{0.603489in}{1.581254in}}%
\pgfpathlineto{\pgfqpoint{0.595386in}{1.590278in}}%
\pgfpathlineto{\pgfqpoint{0.587284in}{1.604115in}}%
\pgfpathlineto{\pgfqpoint{0.579182in}{1.606271in}}%
\pgfpathlineto{\pgfqpoint{0.571080in}{1.602220in}}%
\pgfpathlineto{\pgfqpoint{0.562977in}{1.610793in}}%
\pgfpathlineto{\pgfqpoint{0.554875in}{1.603049in}}%
\pgfpathlineto{\pgfqpoint{0.546773in}{1.600164in}}%
\pgfpathlineto{\pgfqpoint{0.538670in}{1.601849in}}%
\pgfpathlineto{\pgfqpoint{0.530568in}{1.606611in}}%
\pgfpathlineto{\pgfqpoint{0.522466in}{1.615280in}}%
\pgfpathlineto{\pgfqpoint{0.514364in}{1.628339in}}%
\pgfpathlineto{\pgfqpoint{0.506261in}{1.647943in}}%
\pgfpathlineto{\pgfqpoint{0.498159in}{1.655534in}}%
\pgfpathlineto{\pgfqpoint{0.490057in}{1.646131in}}%
\pgfpathlineto{\pgfqpoint{0.481955in}{1.639127in}}%
\pgfpathlineto{\pgfqpoint{0.473852in}{1.668804in}}%
\pgfpathlineto{\pgfqpoint{0.465750in}{1.661910in}}%
\pgfpathlineto{\pgfqpoint{0.457648in}{1.654183in}}%
\pgfpathlineto{\pgfqpoint{0.449545in}{1.685444in}}%
\pgfpathlineto{\pgfqpoint{0.441443in}{1.675218in}}%
\pgfpathlineto{\pgfqpoint{0.433341in}{1.664829in}}%
\pgfpathlineto{\pgfqpoint{0.425239in}{1.680573in}}%
\pgfpathlineto{\pgfqpoint{0.417136in}{1.704092in}}%
\pgfpathlineto{\pgfqpoint{0.409034in}{1.693634in}}%
\pgfpathlineto{\pgfqpoint{0.400932in}{1.674231in}}%
\pgfpathlineto{\pgfqpoint{0.392830in}{1.717714in}}%
\pgfpathlineto{\pgfqpoint{0.384727in}{1.685272in}}%
\pgfpathlineto{\pgfqpoint{0.376625in}{1.751664in}}%
\pgfpathlineto{\pgfqpoint{0.368523in}{1.881063in}}%
\pgfpathclose%
\pgfusepath{fill}%
\end{pgfscope}%
\begin{pgfscope}%
\pgfpathrectangle{\pgfqpoint{0.287500in}{0.375000in}}{\pgfqpoint{1.782500in}{2.265000in}}%
\pgfusepath{clip}%
\pgfsetbuttcap%
\pgfsetroundjoin%
\definecolor{currentfill}{rgb}{0.172549,0.627451,0.172549}%
\pgfsetfillcolor{currentfill}%
\pgfsetfillopacity{0.200000}%
\pgfsetlinewidth{0.000000pt}%
\definecolor{currentstroke}{rgb}{0.000000,0.000000,0.000000}%
\pgfsetstrokecolor{currentstroke}%
\pgfsetdash{}{0pt}%
\pgfpathmoveto{\pgfqpoint{0.368523in}{2.175692in}}%
\pgfpathlineto{\pgfqpoint{0.368523in}{2.537045in}}%
\pgfpathlineto{\pgfqpoint{0.376625in}{2.263368in}}%
\pgfpathlineto{\pgfqpoint{0.384727in}{2.064042in}}%
\pgfpathlineto{\pgfqpoint{0.392830in}{1.984889in}}%
\pgfpathlineto{\pgfqpoint{0.400932in}{1.942524in}}%
\pgfpathlineto{\pgfqpoint{0.409034in}{1.886055in}}%
\pgfpathlineto{\pgfqpoint{0.417136in}{1.838411in}}%
\pgfpathlineto{\pgfqpoint{0.425239in}{1.839497in}}%
\pgfpathlineto{\pgfqpoint{0.433341in}{1.799200in}}%
\pgfpathlineto{\pgfqpoint{0.441443in}{1.784223in}}%
\pgfpathlineto{\pgfqpoint{0.449545in}{1.752312in}}%
\pgfpathlineto{\pgfqpoint{0.457648in}{1.724244in}}%
\pgfpathlineto{\pgfqpoint{0.465750in}{1.715249in}}%
\pgfpathlineto{\pgfqpoint{0.473852in}{1.703462in}}%
\pgfpathlineto{\pgfqpoint{0.481955in}{1.687686in}}%
\pgfpathlineto{\pgfqpoint{0.490057in}{1.672591in}}%
\pgfpathlineto{\pgfqpoint{0.498159in}{1.663564in}}%
\pgfpathlineto{\pgfqpoint{0.506261in}{1.652421in}}%
\pgfpathlineto{\pgfqpoint{0.514364in}{1.661990in}}%
\pgfpathlineto{\pgfqpoint{0.522466in}{1.661522in}}%
\pgfpathlineto{\pgfqpoint{0.530568in}{1.650781in}}%
\pgfpathlineto{\pgfqpoint{0.538670in}{1.633173in}}%
\pgfpathlineto{\pgfqpoint{0.546773in}{1.622255in}}%
\pgfpathlineto{\pgfqpoint{0.554875in}{1.609842in}}%
\pgfpathlineto{\pgfqpoint{0.562977in}{1.601790in}}%
\pgfpathlineto{\pgfqpoint{0.571080in}{1.590241in}}%
\pgfpathlineto{\pgfqpoint{0.579182in}{1.576195in}}%
\pgfpathlineto{\pgfqpoint{0.587284in}{1.576464in}}%
\pgfpathlineto{\pgfqpoint{0.595386in}{1.566016in}}%
\pgfpathlineto{\pgfqpoint{0.603489in}{1.553377in}}%
\pgfpathlineto{\pgfqpoint{0.611591in}{1.552036in}}%
\pgfpathlineto{\pgfqpoint{0.619693in}{1.541339in}}%
\pgfpathlineto{\pgfqpoint{0.627795in}{1.534025in}}%
\pgfpathlineto{\pgfqpoint{0.635898in}{1.534034in}}%
\pgfpathlineto{\pgfqpoint{0.644000in}{1.530400in}}%
\pgfpathlineto{\pgfqpoint{0.652102in}{1.529409in}}%
\pgfpathlineto{\pgfqpoint{0.660205in}{1.523305in}}%
\pgfpathlineto{\pgfqpoint{0.668307in}{1.514475in}}%
\pgfpathlineto{\pgfqpoint{0.676409in}{1.506469in}}%
\pgfpathlineto{\pgfqpoint{0.684511in}{1.508879in}}%
\pgfpathlineto{\pgfqpoint{0.692614in}{1.505984in}}%
\pgfpathlineto{\pgfqpoint{0.700716in}{1.502542in}}%
\pgfpathlineto{\pgfqpoint{0.708818in}{1.495249in}}%
\pgfpathlineto{\pgfqpoint{0.716920in}{1.489570in}}%
\pgfpathlineto{\pgfqpoint{0.725023in}{1.488177in}}%
\pgfpathlineto{\pgfqpoint{0.733125in}{1.483556in}}%
\pgfpathlineto{\pgfqpoint{0.741227in}{1.481953in}}%
\pgfpathlineto{\pgfqpoint{0.749330in}{1.483935in}}%
\pgfpathlineto{\pgfqpoint{0.757432in}{1.484780in}}%
\pgfpathlineto{\pgfqpoint{0.765534in}{1.486905in}}%
\pgfpathlineto{\pgfqpoint{0.773636in}{1.485640in}}%
\pgfpathlineto{\pgfqpoint{0.781739in}{1.483077in}}%
\pgfpathlineto{\pgfqpoint{0.789841in}{1.480259in}}%
\pgfpathlineto{\pgfqpoint{0.797943in}{1.474716in}}%
\pgfpathlineto{\pgfqpoint{0.806045in}{1.474205in}}%
\pgfpathlineto{\pgfqpoint{0.814148in}{1.470109in}}%
\pgfpathlineto{\pgfqpoint{0.822250in}{1.468665in}}%
\pgfpathlineto{\pgfqpoint{0.830352in}{1.468684in}}%
\pgfpathlineto{\pgfqpoint{0.838455in}{1.469965in}}%
\pgfpathlineto{\pgfqpoint{0.846557in}{1.476676in}}%
\pgfpathlineto{\pgfqpoint{0.854659in}{1.476775in}}%
\pgfpathlineto{\pgfqpoint{0.862761in}{1.478013in}}%
\pgfpathlineto{\pgfqpoint{0.870864in}{1.478048in}}%
\pgfpathlineto{\pgfqpoint{0.878966in}{1.477560in}}%
\pgfpathlineto{\pgfqpoint{0.887068in}{1.478095in}}%
\pgfpathlineto{\pgfqpoint{0.895170in}{1.482009in}}%
\pgfpathlineto{\pgfqpoint{0.903273in}{1.480068in}}%
\pgfpathlineto{\pgfqpoint{0.911375in}{1.476259in}}%
\pgfpathlineto{\pgfqpoint{0.919477in}{1.473908in}}%
\pgfpathlineto{\pgfqpoint{0.927580in}{1.471429in}}%
\pgfpathlineto{\pgfqpoint{0.935682in}{1.468289in}}%
\pgfpathlineto{\pgfqpoint{0.943784in}{1.467967in}}%
\pgfpathlineto{\pgfqpoint{0.951886in}{1.464200in}}%
\pgfpathlineto{\pgfqpoint{0.959989in}{1.464188in}}%
\pgfpathlineto{\pgfqpoint{0.968091in}{1.457746in}}%
\pgfpathlineto{\pgfqpoint{0.976193in}{1.449833in}}%
\pgfpathlineto{\pgfqpoint{0.984295in}{1.441113in}}%
\pgfpathlineto{\pgfqpoint{0.992398in}{1.431715in}}%
\pgfpathlineto{\pgfqpoint{1.000500in}{1.425075in}}%
\pgfpathlineto{\pgfqpoint{1.008602in}{1.417536in}}%
\pgfpathlineto{\pgfqpoint{1.016705in}{1.410464in}}%
\pgfpathlineto{\pgfqpoint{1.024807in}{1.402393in}}%
\pgfpathlineto{\pgfqpoint{1.032909in}{1.395878in}}%
\pgfpathlineto{\pgfqpoint{1.041011in}{1.388011in}}%
\pgfpathlineto{\pgfqpoint{1.049114in}{1.381335in}}%
\pgfpathlineto{\pgfqpoint{1.057216in}{1.376734in}}%
\pgfpathlineto{\pgfqpoint{1.065318in}{1.370510in}}%
\pgfpathlineto{\pgfqpoint{1.073420in}{1.362029in}}%
\pgfpathlineto{\pgfqpoint{1.081523in}{1.354385in}}%
\pgfpathlineto{\pgfqpoint{1.089625in}{1.349317in}}%
\pgfpathlineto{\pgfqpoint{1.097727in}{1.343421in}}%
\pgfpathlineto{\pgfqpoint{1.105830in}{1.340545in}}%
\pgfpathlineto{\pgfqpoint{1.113932in}{1.337800in}}%
\pgfpathlineto{\pgfqpoint{1.122034in}{1.332530in}}%
\pgfpathlineto{\pgfqpoint{1.130136in}{1.331825in}}%
\pgfpathlineto{\pgfqpoint{1.138239in}{1.325398in}}%
\pgfpathlineto{\pgfqpoint{1.146341in}{1.318590in}}%
\pgfpathlineto{\pgfqpoint{1.154443in}{1.313442in}}%
\pgfpathlineto{\pgfqpoint{1.162545in}{1.306605in}}%
\pgfpathlineto{\pgfqpoint{1.170648in}{1.302937in}}%
\pgfpathlineto{\pgfqpoint{1.178750in}{1.296840in}}%
\pgfpathlineto{\pgfqpoint{1.186852in}{1.292948in}}%
\pgfpathlineto{\pgfqpoint{1.194955in}{1.288069in}}%
\pgfpathlineto{\pgfqpoint{1.203057in}{1.282784in}}%
\pgfpathlineto{\pgfqpoint{1.211159in}{1.278153in}}%
\pgfpathlineto{\pgfqpoint{1.219261in}{1.273976in}}%
\pgfpathlineto{\pgfqpoint{1.227364in}{1.273372in}}%
\pgfpathlineto{\pgfqpoint{1.235466in}{1.270551in}}%
\pgfpathlineto{\pgfqpoint{1.243568in}{1.267703in}}%
\pgfpathlineto{\pgfqpoint{1.251670in}{1.264098in}}%
\pgfpathlineto{\pgfqpoint{1.259773in}{1.258917in}}%
\pgfpathlineto{\pgfqpoint{1.267875in}{1.256766in}}%
\pgfpathlineto{\pgfqpoint{1.275977in}{1.254088in}}%
\pgfpathlineto{\pgfqpoint{1.284080in}{1.249121in}}%
\pgfpathlineto{\pgfqpoint{1.292182in}{1.243443in}}%
\pgfpathlineto{\pgfqpoint{1.300284in}{1.238229in}}%
\pgfpathlineto{\pgfqpoint{1.308386in}{1.234444in}}%
\pgfpathlineto{\pgfqpoint{1.316489in}{1.234350in}}%
\pgfpathlineto{\pgfqpoint{1.324591in}{1.237158in}}%
\pgfpathlineto{\pgfqpoint{1.332693in}{1.233378in}}%
\pgfpathlineto{\pgfqpoint{1.340795in}{1.228521in}}%
\pgfpathlineto{\pgfqpoint{1.348898in}{1.226057in}}%
\pgfpathlineto{\pgfqpoint{1.357000in}{1.221915in}}%
\pgfpathlineto{\pgfqpoint{1.365102in}{1.217094in}}%
\pgfpathlineto{\pgfqpoint{1.373205in}{1.214579in}}%
\pgfpathlineto{\pgfqpoint{1.381307in}{1.209839in}}%
\pgfpathlineto{\pgfqpoint{1.389409in}{1.207926in}}%
\pgfpathlineto{\pgfqpoint{1.397511in}{1.202609in}}%
\pgfpathlineto{\pgfqpoint{1.405614in}{1.198246in}}%
\pgfpathlineto{\pgfqpoint{1.413716in}{1.197241in}}%
\pgfpathlineto{\pgfqpoint{1.421818in}{1.191596in}}%
\pgfpathlineto{\pgfqpoint{1.429920in}{1.195054in}}%
\pgfpathlineto{\pgfqpoint{1.438023in}{1.191221in}}%
\pgfpathlineto{\pgfqpoint{1.446125in}{1.188401in}}%
\pgfpathlineto{\pgfqpoint{1.454227in}{1.189016in}}%
\pgfpathlineto{\pgfqpoint{1.462330in}{1.187549in}}%
\pgfpathlineto{\pgfqpoint{1.470432in}{1.185150in}}%
\pgfpathlineto{\pgfqpoint{1.478534in}{1.185046in}}%
\pgfpathlineto{\pgfqpoint{1.486636in}{1.180542in}}%
\pgfpathlineto{\pgfqpoint{1.494739in}{1.180055in}}%
\pgfpathlineto{\pgfqpoint{1.502841in}{1.176582in}}%
\pgfpathlineto{\pgfqpoint{1.510943in}{1.174576in}}%
\pgfpathlineto{\pgfqpoint{1.519045in}{1.171890in}}%
\pgfpathlineto{\pgfqpoint{1.527148in}{1.170195in}}%
\pgfpathlineto{\pgfqpoint{1.535250in}{1.167045in}}%
\pgfpathlineto{\pgfqpoint{1.543352in}{1.164968in}}%
\pgfpathlineto{\pgfqpoint{1.551455in}{1.164862in}}%
\pgfpathlineto{\pgfqpoint{1.559557in}{1.168724in}}%
\pgfpathlineto{\pgfqpoint{1.567659in}{1.167808in}}%
\pgfpathlineto{\pgfqpoint{1.575761in}{1.166334in}}%
\pgfpathlineto{\pgfqpoint{1.583864in}{1.167644in}}%
\pgfpathlineto{\pgfqpoint{1.591966in}{1.165992in}}%
\pgfpathlineto{\pgfqpoint{1.600068in}{1.166974in}}%
\pgfpathlineto{\pgfqpoint{1.608170in}{1.164022in}}%
\pgfpathlineto{\pgfqpoint{1.616273in}{1.160690in}}%
\pgfpathlineto{\pgfqpoint{1.624375in}{1.158985in}}%
\pgfpathlineto{\pgfqpoint{1.632477in}{1.154965in}}%
\pgfpathlineto{\pgfqpoint{1.640580in}{1.151276in}}%
\pgfpathlineto{\pgfqpoint{1.648682in}{1.152168in}}%
\pgfpathlineto{\pgfqpoint{1.656784in}{1.152795in}}%
\pgfpathlineto{\pgfqpoint{1.664886in}{1.151658in}}%
\pgfpathlineto{\pgfqpoint{1.672989in}{1.151534in}}%
\pgfpathlineto{\pgfqpoint{1.681091in}{1.149608in}}%
\pgfpathlineto{\pgfqpoint{1.689193in}{1.147037in}}%
\pgfpathlineto{\pgfqpoint{1.697295in}{1.150958in}}%
\pgfpathlineto{\pgfqpoint{1.705398in}{1.152292in}}%
\pgfpathlineto{\pgfqpoint{1.713500in}{1.149362in}}%
\pgfpathlineto{\pgfqpoint{1.721602in}{1.147038in}}%
\pgfpathlineto{\pgfqpoint{1.729705in}{1.150104in}}%
\pgfpathlineto{\pgfqpoint{1.737807in}{1.148782in}}%
\pgfpathlineto{\pgfqpoint{1.745909in}{1.148623in}}%
\pgfpathlineto{\pgfqpoint{1.754011in}{1.149538in}}%
\pgfpathlineto{\pgfqpoint{1.762114in}{1.150020in}}%
\pgfpathlineto{\pgfqpoint{1.770216in}{1.146034in}}%
\pgfpathlineto{\pgfqpoint{1.778318in}{1.142964in}}%
\pgfpathlineto{\pgfqpoint{1.786420in}{1.142886in}}%
\pgfpathlineto{\pgfqpoint{1.794523in}{1.144912in}}%
\pgfpathlineto{\pgfqpoint{1.802625in}{1.146622in}}%
\pgfpathlineto{\pgfqpoint{1.810727in}{1.148529in}}%
\pgfpathlineto{\pgfqpoint{1.818830in}{1.145914in}}%
\pgfpathlineto{\pgfqpoint{1.826932in}{1.145580in}}%
\pgfpathlineto{\pgfqpoint{1.835034in}{1.142474in}}%
\pgfpathlineto{\pgfqpoint{1.843136in}{1.140826in}}%
\pgfpathlineto{\pgfqpoint{1.851239in}{1.143071in}}%
\pgfpathlineto{\pgfqpoint{1.859341in}{1.141416in}}%
\pgfpathlineto{\pgfqpoint{1.867443in}{1.139017in}}%
\pgfpathlineto{\pgfqpoint{1.875545in}{1.134795in}}%
\pgfpathlineto{\pgfqpoint{1.883648in}{1.136422in}}%
\pgfpathlineto{\pgfqpoint{1.891750in}{1.136106in}}%
\pgfpathlineto{\pgfqpoint{1.899852in}{1.139333in}}%
\pgfpathlineto{\pgfqpoint{1.907955in}{1.137766in}}%
\pgfpathlineto{\pgfqpoint{1.916057in}{1.135842in}}%
\pgfpathlineto{\pgfqpoint{1.924159in}{1.134183in}}%
\pgfpathlineto{\pgfqpoint{1.932261in}{1.131209in}}%
\pgfpathlineto{\pgfqpoint{1.940364in}{1.133114in}}%
\pgfpathlineto{\pgfqpoint{1.948466in}{1.133356in}}%
\pgfpathlineto{\pgfqpoint{1.956568in}{1.135870in}}%
\pgfpathlineto{\pgfqpoint{1.964670in}{1.134508in}}%
\pgfpathlineto{\pgfqpoint{1.972773in}{1.133341in}}%
\pgfpathlineto{\pgfqpoint{1.980875in}{1.128717in}}%
\pgfpathlineto{\pgfqpoint{1.988977in}{1.126552in}}%
\pgfpathlineto{\pgfqpoint{1.988977in}{0.867717in}}%
\pgfpathlineto{\pgfqpoint{1.988977in}{0.867717in}}%
\pgfpathlineto{\pgfqpoint{1.980875in}{0.867625in}}%
\pgfpathlineto{\pgfqpoint{1.972773in}{0.865790in}}%
\pgfpathlineto{\pgfqpoint{1.964670in}{0.862445in}}%
\pgfpathlineto{\pgfqpoint{1.956568in}{0.864464in}}%
\pgfpathlineto{\pgfqpoint{1.948466in}{0.865584in}}%
\pgfpathlineto{\pgfqpoint{1.940364in}{0.867726in}}%
\pgfpathlineto{\pgfqpoint{1.932261in}{0.868547in}}%
\pgfpathlineto{\pgfqpoint{1.924159in}{0.871225in}}%
\pgfpathlineto{\pgfqpoint{1.916057in}{0.873680in}}%
\pgfpathlineto{\pgfqpoint{1.907955in}{0.876498in}}%
\pgfpathlineto{\pgfqpoint{1.899852in}{0.878923in}}%
\pgfpathlineto{\pgfqpoint{1.891750in}{0.880008in}}%
\pgfpathlineto{\pgfqpoint{1.883648in}{0.880229in}}%
\pgfpathlineto{\pgfqpoint{1.875545in}{0.882270in}}%
\pgfpathlineto{\pgfqpoint{1.867443in}{0.881905in}}%
\pgfpathlineto{\pgfqpoint{1.859341in}{0.883607in}}%
\pgfpathlineto{\pgfqpoint{1.851239in}{0.887049in}}%
\pgfpathlineto{\pgfqpoint{1.843136in}{0.887322in}}%
\pgfpathlineto{\pgfqpoint{1.835034in}{0.889442in}}%
\pgfpathlineto{\pgfqpoint{1.826932in}{0.889953in}}%
\pgfpathlineto{\pgfqpoint{1.818830in}{0.889396in}}%
\pgfpathlineto{\pgfqpoint{1.810727in}{0.892447in}}%
\pgfpathlineto{\pgfqpoint{1.802625in}{0.890887in}}%
\pgfpathlineto{\pgfqpoint{1.794523in}{0.891171in}}%
\pgfpathlineto{\pgfqpoint{1.786420in}{0.891825in}}%
\pgfpathlineto{\pgfqpoint{1.778318in}{0.895779in}}%
\pgfpathlineto{\pgfqpoint{1.770216in}{0.895724in}}%
\pgfpathlineto{\pgfqpoint{1.762114in}{0.893364in}}%
\pgfpathlineto{\pgfqpoint{1.754011in}{0.889519in}}%
\pgfpathlineto{\pgfqpoint{1.745909in}{0.886352in}}%
\pgfpathlineto{\pgfqpoint{1.737807in}{0.884908in}}%
\pgfpathlineto{\pgfqpoint{1.729705in}{0.888156in}}%
\pgfpathlineto{\pgfqpoint{1.721602in}{0.889510in}}%
\pgfpathlineto{\pgfqpoint{1.713500in}{0.888384in}}%
\pgfpathlineto{\pgfqpoint{1.705398in}{0.890022in}}%
\pgfpathlineto{\pgfqpoint{1.697295in}{0.889680in}}%
\pgfpathlineto{\pgfqpoint{1.689193in}{0.891886in}}%
\pgfpathlineto{\pgfqpoint{1.681091in}{0.892978in}}%
\pgfpathlineto{\pgfqpoint{1.672989in}{0.895024in}}%
\pgfpathlineto{\pgfqpoint{1.664886in}{0.895097in}}%
\pgfpathlineto{\pgfqpoint{1.656784in}{0.893644in}}%
\pgfpathlineto{\pgfqpoint{1.648682in}{0.885884in}}%
\pgfpathlineto{\pgfqpoint{1.640580in}{0.888858in}}%
\pgfpathlineto{\pgfqpoint{1.632477in}{0.893729in}}%
\pgfpathlineto{\pgfqpoint{1.624375in}{0.896729in}}%
\pgfpathlineto{\pgfqpoint{1.616273in}{0.898762in}}%
\pgfpathlineto{\pgfqpoint{1.608170in}{0.902403in}}%
\pgfpathlineto{\pgfqpoint{1.600068in}{0.899657in}}%
\pgfpathlineto{\pgfqpoint{1.591966in}{0.901912in}}%
\pgfpathlineto{\pgfqpoint{1.583864in}{0.895479in}}%
\pgfpathlineto{\pgfqpoint{1.575761in}{0.893519in}}%
\pgfpathlineto{\pgfqpoint{1.567659in}{0.891922in}}%
\pgfpathlineto{\pgfqpoint{1.559557in}{0.891895in}}%
\pgfpathlineto{\pgfqpoint{1.551455in}{0.895347in}}%
\pgfpathlineto{\pgfqpoint{1.543352in}{0.900246in}}%
\pgfpathlineto{\pgfqpoint{1.535250in}{0.899756in}}%
\pgfpathlineto{\pgfqpoint{1.527148in}{0.902172in}}%
\pgfpathlineto{\pgfqpoint{1.519045in}{0.906190in}}%
\pgfpathlineto{\pgfqpoint{1.510943in}{0.905789in}}%
\pgfpathlineto{\pgfqpoint{1.502841in}{0.907052in}}%
\pgfpathlineto{\pgfqpoint{1.494739in}{0.908112in}}%
\pgfpathlineto{\pgfqpoint{1.486636in}{0.907943in}}%
\pgfpathlineto{\pgfqpoint{1.478534in}{0.904164in}}%
\pgfpathlineto{\pgfqpoint{1.470432in}{0.906741in}}%
\pgfpathlineto{\pgfqpoint{1.462330in}{0.907643in}}%
\pgfpathlineto{\pgfqpoint{1.454227in}{0.904125in}}%
\pgfpathlineto{\pgfqpoint{1.446125in}{0.905200in}}%
\pgfpathlineto{\pgfqpoint{1.438023in}{0.905330in}}%
\pgfpathlineto{\pgfqpoint{1.429920in}{0.911277in}}%
\pgfpathlineto{\pgfqpoint{1.421818in}{0.912844in}}%
\pgfpathlineto{\pgfqpoint{1.413716in}{0.915547in}}%
\pgfpathlineto{\pgfqpoint{1.405614in}{0.919362in}}%
\pgfpathlineto{\pgfqpoint{1.397511in}{0.922497in}}%
\pgfpathlineto{\pgfqpoint{1.389409in}{0.923580in}}%
\pgfpathlineto{\pgfqpoint{1.381307in}{0.921080in}}%
\pgfpathlineto{\pgfqpoint{1.373205in}{0.923268in}}%
\pgfpathlineto{\pgfqpoint{1.365102in}{0.925159in}}%
\pgfpathlineto{\pgfqpoint{1.357000in}{0.927926in}}%
\pgfpathlineto{\pgfqpoint{1.348898in}{0.926112in}}%
\pgfpathlineto{\pgfqpoint{1.340795in}{0.929662in}}%
\pgfpathlineto{\pgfqpoint{1.332693in}{0.932063in}}%
\pgfpathlineto{\pgfqpoint{1.324591in}{0.930600in}}%
\pgfpathlineto{\pgfqpoint{1.316489in}{0.935141in}}%
\pgfpathlineto{\pgfqpoint{1.308386in}{0.939314in}}%
\pgfpathlineto{\pgfqpoint{1.300284in}{0.937136in}}%
\pgfpathlineto{\pgfqpoint{1.292182in}{0.942838in}}%
\pgfpathlineto{\pgfqpoint{1.284080in}{0.947647in}}%
\pgfpathlineto{\pgfqpoint{1.275977in}{0.947483in}}%
\pgfpathlineto{\pgfqpoint{1.267875in}{0.947015in}}%
\pgfpathlineto{\pgfqpoint{1.259773in}{0.947731in}}%
\pgfpathlineto{\pgfqpoint{1.251670in}{0.944233in}}%
\pgfpathlineto{\pgfqpoint{1.243568in}{0.941101in}}%
\pgfpathlineto{\pgfqpoint{1.235466in}{0.943451in}}%
\pgfpathlineto{\pgfqpoint{1.227364in}{0.945654in}}%
\pgfpathlineto{\pgfqpoint{1.219261in}{0.943341in}}%
\pgfpathlineto{\pgfqpoint{1.211159in}{0.942284in}}%
\pgfpathlineto{\pgfqpoint{1.203057in}{0.941773in}}%
\pgfpathlineto{\pgfqpoint{1.194955in}{0.943705in}}%
\pgfpathlineto{\pgfqpoint{1.186852in}{0.946576in}}%
\pgfpathlineto{\pgfqpoint{1.178750in}{0.948984in}}%
\pgfpathlineto{\pgfqpoint{1.170648in}{0.952910in}}%
\pgfpathlineto{\pgfqpoint{1.162545in}{0.944303in}}%
\pgfpathlineto{\pgfqpoint{1.154443in}{0.948865in}}%
\pgfpathlineto{\pgfqpoint{1.146341in}{0.950692in}}%
\pgfpathlineto{\pgfqpoint{1.138239in}{0.957796in}}%
\pgfpathlineto{\pgfqpoint{1.130136in}{0.955333in}}%
\pgfpathlineto{\pgfqpoint{1.122034in}{0.941815in}}%
\pgfpathlineto{\pgfqpoint{1.113932in}{0.930527in}}%
\pgfpathlineto{\pgfqpoint{1.105830in}{0.913229in}}%
\pgfpathlineto{\pgfqpoint{1.097727in}{0.906422in}}%
\pgfpathlineto{\pgfqpoint{1.089625in}{0.907671in}}%
\pgfpathlineto{\pgfqpoint{1.081523in}{0.907179in}}%
\pgfpathlineto{\pgfqpoint{1.073420in}{0.908164in}}%
\pgfpathlineto{\pgfqpoint{1.065318in}{0.913394in}}%
\pgfpathlineto{\pgfqpoint{1.057216in}{0.913035in}}%
\pgfpathlineto{\pgfqpoint{1.049114in}{0.900310in}}%
\pgfpathlineto{\pgfqpoint{1.041011in}{0.897821in}}%
\pgfpathlineto{\pgfqpoint{1.032909in}{0.894455in}}%
\pgfpathlineto{\pgfqpoint{1.024807in}{0.892857in}}%
\pgfpathlineto{\pgfqpoint{1.016705in}{0.890858in}}%
\pgfpathlineto{\pgfqpoint{1.008602in}{0.891270in}}%
\pgfpathlineto{\pgfqpoint{1.000500in}{0.876525in}}%
\pgfpathlineto{\pgfqpoint{0.992398in}{0.863988in}}%
\pgfpathlineto{\pgfqpoint{0.984295in}{0.868578in}}%
\pgfpathlineto{\pgfqpoint{0.976193in}{0.874020in}}%
\pgfpathlineto{\pgfqpoint{0.968091in}{0.877456in}}%
\pgfpathlineto{\pgfqpoint{0.959989in}{0.874353in}}%
\pgfpathlineto{\pgfqpoint{0.951886in}{0.862601in}}%
\pgfpathlineto{\pgfqpoint{0.943784in}{0.856644in}}%
\pgfpathlineto{\pgfqpoint{0.935682in}{0.854308in}}%
\pgfpathlineto{\pgfqpoint{0.927580in}{0.862120in}}%
\pgfpathlineto{\pgfqpoint{0.919477in}{0.864615in}}%
\pgfpathlineto{\pgfqpoint{0.911375in}{0.860170in}}%
\pgfpathlineto{\pgfqpoint{0.903273in}{0.857105in}}%
\pgfpathlineto{\pgfqpoint{0.895170in}{0.853610in}}%
\pgfpathlineto{\pgfqpoint{0.887068in}{0.849304in}}%
\pgfpathlineto{\pgfqpoint{0.878966in}{0.859656in}}%
\pgfpathlineto{\pgfqpoint{0.870864in}{0.849720in}}%
\pgfpathlineto{\pgfqpoint{0.862761in}{0.838665in}}%
\pgfpathlineto{\pgfqpoint{0.854659in}{0.838884in}}%
\pgfpathlineto{\pgfqpoint{0.846557in}{0.832232in}}%
\pgfpathlineto{\pgfqpoint{0.838455in}{0.827596in}}%
\pgfpathlineto{\pgfqpoint{0.830352in}{0.830921in}}%
\pgfpathlineto{\pgfqpoint{0.822250in}{0.827636in}}%
\pgfpathlineto{\pgfqpoint{0.814148in}{0.833154in}}%
\pgfpathlineto{\pgfqpoint{0.806045in}{0.837284in}}%
\pgfpathlineto{\pgfqpoint{0.797943in}{0.837676in}}%
\pgfpathlineto{\pgfqpoint{0.789841in}{0.847171in}}%
\pgfpathlineto{\pgfqpoint{0.781739in}{0.838731in}}%
\pgfpathlineto{\pgfqpoint{0.773636in}{0.840365in}}%
\pgfpathlineto{\pgfqpoint{0.765534in}{0.825944in}}%
\pgfpathlineto{\pgfqpoint{0.757432in}{0.817402in}}%
\pgfpathlineto{\pgfqpoint{0.749330in}{0.826332in}}%
\pgfpathlineto{\pgfqpoint{0.741227in}{0.827473in}}%
\pgfpathlineto{\pgfqpoint{0.733125in}{0.836351in}}%
\pgfpathlineto{\pgfqpoint{0.725023in}{0.819830in}}%
\pgfpathlineto{\pgfqpoint{0.716920in}{0.818263in}}%
\pgfpathlineto{\pgfqpoint{0.708818in}{0.808704in}}%
\pgfpathlineto{\pgfqpoint{0.700716in}{0.805635in}}%
\pgfpathlineto{\pgfqpoint{0.692614in}{0.787058in}}%
\pgfpathlineto{\pgfqpoint{0.684511in}{0.788472in}}%
\pgfpathlineto{\pgfqpoint{0.676409in}{0.781939in}}%
\pgfpathlineto{\pgfqpoint{0.668307in}{0.784749in}}%
\pgfpathlineto{\pgfqpoint{0.660205in}{0.793093in}}%
\pgfpathlineto{\pgfqpoint{0.652102in}{0.808533in}}%
\pgfpathlineto{\pgfqpoint{0.644000in}{0.824598in}}%
\pgfpathlineto{\pgfqpoint{0.635898in}{0.831394in}}%
\pgfpathlineto{\pgfqpoint{0.627795in}{0.853531in}}%
\pgfpathlineto{\pgfqpoint{0.619693in}{0.874148in}}%
\pgfpathlineto{\pgfqpoint{0.611591in}{0.880785in}}%
\pgfpathlineto{\pgfqpoint{0.603489in}{0.893391in}}%
\pgfpathlineto{\pgfqpoint{0.595386in}{0.913681in}}%
\pgfpathlineto{\pgfqpoint{0.587284in}{0.910968in}}%
\pgfpathlineto{\pgfqpoint{0.579182in}{0.913645in}}%
\pgfpathlineto{\pgfqpoint{0.571080in}{0.918387in}}%
\pgfpathlineto{\pgfqpoint{0.562977in}{0.914404in}}%
\pgfpathlineto{\pgfqpoint{0.554875in}{0.933891in}}%
\pgfpathlineto{\pgfqpoint{0.546773in}{0.951230in}}%
\pgfpathlineto{\pgfqpoint{0.538670in}{0.967584in}}%
\pgfpathlineto{\pgfqpoint{0.530568in}{0.994640in}}%
\pgfpathlineto{\pgfqpoint{0.522466in}{0.992374in}}%
\pgfpathlineto{\pgfqpoint{0.514364in}{1.016858in}}%
\pgfpathlineto{\pgfqpoint{0.506261in}{1.002989in}}%
\pgfpathlineto{\pgfqpoint{0.498159in}{1.004196in}}%
\pgfpathlineto{\pgfqpoint{0.490057in}{0.981798in}}%
\pgfpathlineto{\pgfqpoint{0.481955in}{1.006575in}}%
\pgfpathlineto{\pgfqpoint{0.473852in}{1.042183in}}%
\pgfpathlineto{\pgfqpoint{0.465750in}{1.064119in}}%
\pgfpathlineto{\pgfqpoint{0.457648in}{1.093948in}}%
\pgfpathlineto{\pgfqpoint{0.449545in}{1.141603in}}%
\pgfpathlineto{\pgfqpoint{0.441443in}{1.158354in}}%
\pgfpathlineto{\pgfqpoint{0.433341in}{1.199904in}}%
\pgfpathlineto{\pgfqpoint{0.425239in}{1.202106in}}%
\pgfpathlineto{\pgfqpoint{0.417136in}{1.208706in}}%
\pgfpathlineto{\pgfqpoint{0.409034in}{1.267601in}}%
\pgfpathlineto{\pgfqpoint{0.400932in}{1.338376in}}%
\pgfpathlineto{\pgfqpoint{0.392830in}{1.464202in}}%
\pgfpathlineto{\pgfqpoint{0.384727in}{1.607956in}}%
\pgfpathlineto{\pgfqpoint{0.376625in}{1.770750in}}%
\pgfpathlineto{\pgfqpoint{0.368523in}{2.175692in}}%
\pgfpathclose%
\pgfusepath{fill}%
\end{pgfscope}%
\begin{pgfscope}%
\pgfpathrectangle{\pgfqpoint{0.287500in}{0.375000in}}{\pgfqpoint{1.782500in}{2.265000in}}%
\pgfusepath{clip}%
\pgfsetroundcap%
\pgfsetroundjoin%
\pgfsetlinewidth{1.505625pt}%
\definecolor{currentstroke}{rgb}{0.121569,0.466667,0.705882}%
\pgfsetstrokecolor{currentstroke}%
\pgfsetdash{}{0pt}%
\pgfpathmoveto{\pgfqpoint{0.368523in}{1.845183in}}%
\pgfpathlineto{\pgfqpoint{0.376625in}{1.838458in}}%
\pgfpathlineto{\pgfqpoint{0.392830in}{1.737424in}}%
\pgfpathlineto{\pgfqpoint{0.400932in}{1.707011in}}%
\pgfpathlineto{\pgfqpoint{0.409034in}{1.660638in}}%
\pgfpathlineto{\pgfqpoint{0.417136in}{1.664462in}}%
\pgfpathlineto{\pgfqpoint{0.425239in}{1.704358in}}%
\pgfpathlineto{\pgfqpoint{0.433341in}{1.690967in}}%
\pgfpathlineto{\pgfqpoint{0.441443in}{1.670312in}}%
\pgfpathlineto{\pgfqpoint{0.449545in}{1.667959in}}%
\pgfpathlineto{\pgfqpoint{0.457648in}{1.648095in}}%
\pgfpathlineto{\pgfqpoint{0.473852in}{1.653837in}}%
\pgfpathlineto{\pgfqpoint{0.481955in}{1.652478in}}%
\pgfpathlineto{\pgfqpoint{0.490057in}{1.656230in}}%
\pgfpathlineto{\pgfqpoint{0.498159in}{1.668049in}}%
\pgfpathlineto{\pgfqpoint{0.506261in}{1.667329in}}%
\pgfpathlineto{\pgfqpoint{0.514364in}{1.657601in}}%
\pgfpathlineto{\pgfqpoint{0.522466in}{1.642526in}}%
\pgfpathlineto{\pgfqpoint{0.530568in}{1.655830in}}%
\pgfpathlineto{\pgfqpoint{0.538670in}{1.646220in}}%
\pgfpathlineto{\pgfqpoint{0.546773in}{1.652493in}}%
\pgfpathlineto{\pgfqpoint{0.554875in}{1.648601in}}%
\pgfpathlineto{\pgfqpoint{0.562977in}{1.639551in}}%
\pgfpathlineto{\pgfqpoint{0.571080in}{1.645047in}}%
\pgfpathlineto{\pgfqpoint{0.579182in}{1.637111in}}%
\pgfpathlineto{\pgfqpoint{0.587284in}{1.650357in}}%
\pgfpathlineto{\pgfqpoint{0.595386in}{1.656588in}}%
\pgfpathlineto{\pgfqpoint{0.603489in}{1.649613in}}%
\pgfpathlineto{\pgfqpoint{0.611591in}{1.647267in}}%
\pgfpathlineto{\pgfqpoint{0.619693in}{1.643230in}}%
\pgfpathlineto{\pgfqpoint{0.627795in}{1.635948in}}%
\pgfpathlineto{\pgfqpoint{0.635898in}{1.639152in}}%
\pgfpathlineto{\pgfqpoint{0.644000in}{1.628050in}}%
\pgfpathlineto{\pgfqpoint{0.652102in}{1.629505in}}%
\pgfpathlineto{\pgfqpoint{0.660205in}{1.620058in}}%
\pgfpathlineto{\pgfqpoint{0.684511in}{1.606130in}}%
\pgfpathlineto{\pgfqpoint{0.692614in}{1.613759in}}%
\pgfpathlineto{\pgfqpoint{0.700716in}{1.612595in}}%
\pgfpathlineto{\pgfqpoint{0.708818in}{1.615514in}}%
\pgfpathlineto{\pgfqpoint{0.716920in}{1.613917in}}%
\pgfpathlineto{\pgfqpoint{0.725023in}{1.610345in}}%
\pgfpathlineto{\pgfqpoint{0.733125in}{1.603157in}}%
\pgfpathlineto{\pgfqpoint{0.741227in}{1.599683in}}%
\pgfpathlineto{\pgfqpoint{0.765534in}{1.570260in}}%
\pgfpathlineto{\pgfqpoint{0.773636in}{1.562835in}}%
\pgfpathlineto{\pgfqpoint{0.781739in}{1.560236in}}%
\pgfpathlineto{\pgfqpoint{0.789841in}{1.549821in}}%
\pgfpathlineto{\pgfqpoint{0.797943in}{1.544570in}}%
\pgfpathlineto{\pgfqpoint{0.806045in}{1.537090in}}%
\pgfpathlineto{\pgfqpoint{0.822250in}{1.535018in}}%
\pgfpathlineto{\pgfqpoint{0.830352in}{1.528612in}}%
\pgfpathlineto{\pgfqpoint{0.838455in}{1.524703in}}%
\pgfpathlineto{\pgfqpoint{0.846557in}{1.516243in}}%
\pgfpathlineto{\pgfqpoint{0.854659in}{1.511560in}}%
\pgfpathlineto{\pgfqpoint{0.862761in}{1.503349in}}%
\pgfpathlineto{\pgfqpoint{0.870864in}{1.504172in}}%
\pgfpathlineto{\pgfqpoint{0.911375in}{1.455381in}}%
\pgfpathlineto{\pgfqpoint{0.935682in}{1.433221in}}%
\pgfpathlineto{\pgfqpoint{0.943784in}{1.420399in}}%
\pgfpathlineto{\pgfqpoint{0.976193in}{1.386119in}}%
\pgfpathlineto{\pgfqpoint{0.984295in}{1.375580in}}%
\pgfpathlineto{\pgfqpoint{1.000500in}{1.364441in}}%
\pgfpathlineto{\pgfqpoint{1.008602in}{1.359613in}}%
\pgfpathlineto{\pgfqpoint{1.016705in}{1.352630in}}%
\pgfpathlineto{\pgfqpoint{1.041011in}{1.322121in}}%
\pgfpathlineto{\pgfqpoint{1.081523in}{1.275550in}}%
\pgfpathlineto{\pgfqpoint{1.089625in}{1.269276in}}%
\pgfpathlineto{\pgfqpoint{1.097727in}{1.260622in}}%
\pgfpathlineto{\pgfqpoint{1.105830in}{1.260311in}}%
\pgfpathlineto{\pgfqpoint{1.122034in}{1.242119in}}%
\pgfpathlineto{\pgfqpoint{1.130136in}{1.237987in}}%
\pgfpathlineto{\pgfqpoint{1.154443in}{1.210620in}}%
\pgfpathlineto{\pgfqpoint{1.203057in}{1.162560in}}%
\pgfpathlineto{\pgfqpoint{1.211159in}{1.158052in}}%
\pgfpathlineto{\pgfqpoint{1.219261in}{1.150205in}}%
\pgfpathlineto{\pgfqpoint{1.227364in}{1.149416in}}%
\pgfpathlineto{\pgfqpoint{1.235466in}{1.143504in}}%
\pgfpathlineto{\pgfqpoint{1.259773in}{1.120942in}}%
\pgfpathlineto{\pgfqpoint{1.267875in}{1.117578in}}%
\pgfpathlineto{\pgfqpoint{1.275977in}{1.112140in}}%
\pgfpathlineto{\pgfqpoint{1.284080in}{1.104334in}}%
\pgfpathlineto{\pgfqpoint{1.292182in}{1.099245in}}%
\pgfpathlineto{\pgfqpoint{1.300284in}{1.091516in}}%
\pgfpathlineto{\pgfqpoint{1.316489in}{1.085177in}}%
\pgfpathlineto{\pgfqpoint{1.332693in}{1.072454in}}%
\pgfpathlineto{\pgfqpoint{1.365102in}{1.043757in}}%
\pgfpathlineto{\pgfqpoint{1.389409in}{1.025868in}}%
\pgfpathlineto{\pgfqpoint{1.397511in}{1.023069in}}%
\pgfpathlineto{\pgfqpoint{1.486636in}{0.957853in}}%
\pgfpathlineto{\pgfqpoint{1.494739in}{0.955986in}}%
\pgfpathlineto{\pgfqpoint{1.567659in}{0.908986in}}%
\pgfpathlineto{\pgfqpoint{1.575761in}{0.906089in}}%
\pgfpathlineto{\pgfqpoint{1.591966in}{0.894886in}}%
\pgfpathlineto{\pgfqpoint{1.608170in}{0.882960in}}%
\pgfpathlineto{\pgfqpoint{1.616273in}{0.880856in}}%
\pgfpathlineto{\pgfqpoint{1.640580in}{0.864418in}}%
\pgfpathlineto{\pgfqpoint{1.656784in}{0.854087in}}%
\pgfpathlineto{\pgfqpoint{1.681091in}{0.839977in}}%
\pgfpathlineto{\pgfqpoint{1.689193in}{0.838273in}}%
\pgfpathlineto{\pgfqpoint{1.697295in}{0.838985in}}%
\pgfpathlineto{\pgfqpoint{1.729705in}{0.818701in}}%
\pgfpathlineto{\pgfqpoint{1.778318in}{0.788511in}}%
\pgfpathlineto{\pgfqpoint{1.810727in}{0.768875in}}%
\pgfpathlineto{\pgfqpoint{1.818830in}{0.768861in}}%
\pgfpathlineto{\pgfqpoint{1.826932in}{0.764672in}}%
\pgfpathlineto{\pgfqpoint{1.835034in}{0.762717in}}%
\pgfpathlineto{\pgfqpoint{1.851239in}{0.754179in}}%
\pgfpathlineto{\pgfqpoint{1.867443in}{0.748256in}}%
\pgfpathlineto{\pgfqpoint{1.883648in}{0.738491in}}%
\pgfpathlineto{\pgfqpoint{1.899852in}{0.731401in}}%
\pgfpathlineto{\pgfqpoint{1.907955in}{0.726689in}}%
\pgfpathlineto{\pgfqpoint{1.916057in}{0.725466in}}%
\pgfpathlineto{\pgfqpoint{1.956568in}{0.704753in}}%
\pgfpathlineto{\pgfqpoint{1.980875in}{0.692394in}}%
\pgfpathlineto{\pgfqpoint{1.988977in}{0.691504in}}%
\pgfpathlineto{\pgfqpoint{1.988977in}{0.691504in}}%
\pgfusepath{stroke}%
\end{pgfscope}%
\begin{pgfscope}%
\pgfpathrectangle{\pgfqpoint{0.287500in}{0.375000in}}{\pgfqpoint{1.782500in}{2.265000in}}%
\pgfusepath{clip}%
\pgfsetroundcap%
\pgfsetroundjoin%
\pgfsetlinewidth{1.505625pt}%
\definecolor{currentstroke}{rgb}{1.000000,0.498039,0.054902}%
\pgfsetstrokecolor{currentstroke}%
\pgfsetdash{}{0pt}%
\pgfpathmoveto{\pgfqpoint{0.368523in}{2.218412in}}%
\pgfpathlineto{\pgfqpoint{0.376625in}{1.979135in}}%
\pgfpathlineto{\pgfqpoint{0.384727in}{1.868269in}}%
\pgfpathlineto{\pgfqpoint{0.392830in}{1.830158in}}%
\pgfpathlineto{\pgfqpoint{0.400932in}{1.820901in}}%
\pgfpathlineto{\pgfqpoint{0.409034in}{1.808602in}}%
\pgfpathlineto{\pgfqpoint{0.425239in}{1.762929in}}%
\pgfpathlineto{\pgfqpoint{0.441443in}{1.739301in}}%
\pgfpathlineto{\pgfqpoint{0.449545in}{1.736467in}}%
\pgfpathlineto{\pgfqpoint{0.457648in}{1.712485in}}%
\pgfpathlineto{\pgfqpoint{0.465750in}{1.722094in}}%
\pgfpathlineto{\pgfqpoint{0.473852in}{1.725143in}}%
\pgfpathlineto{\pgfqpoint{0.481955in}{1.715057in}}%
\pgfpathlineto{\pgfqpoint{0.490057in}{1.699943in}}%
\pgfpathlineto{\pgfqpoint{0.498159in}{1.708278in}}%
\pgfpathlineto{\pgfqpoint{0.506261in}{1.698470in}}%
\pgfpathlineto{\pgfqpoint{0.522466in}{1.667059in}}%
\pgfpathlineto{\pgfqpoint{0.546773in}{1.647810in}}%
\pgfpathlineto{\pgfqpoint{0.554875in}{1.645155in}}%
\pgfpathlineto{\pgfqpoint{0.562977in}{1.644861in}}%
\pgfpathlineto{\pgfqpoint{0.571080in}{1.640328in}}%
\pgfpathlineto{\pgfqpoint{0.587284in}{1.628513in}}%
\pgfpathlineto{\pgfqpoint{0.611591in}{1.618988in}}%
\pgfpathlineto{\pgfqpoint{0.619693in}{1.623381in}}%
\pgfpathlineto{\pgfqpoint{0.627795in}{1.621020in}}%
\pgfpathlineto{\pgfqpoint{0.635898in}{1.626659in}}%
\pgfpathlineto{\pgfqpoint{0.644000in}{1.625900in}}%
\pgfpathlineto{\pgfqpoint{0.660205in}{1.616019in}}%
\pgfpathlineto{\pgfqpoint{0.676409in}{1.614298in}}%
\pgfpathlineto{\pgfqpoint{0.684511in}{1.619027in}}%
\pgfpathlineto{\pgfqpoint{0.692614in}{1.616481in}}%
\pgfpathlineto{\pgfqpoint{0.700716in}{1.611753in}}%
\pgfpathlineto{\pgfqpoint{0.708818in}{1.609558in}}%
\pgfpathlineto{\pgfqpoint{0.716920in}{1.604495in}}%
\pgfpathlineto{\pgfqpoint{0.725023in}{1.605963in}}%
\pgfpathlineto{\pgfqpoint{0.733125in}{1.603440in}}%
\pgfpathlineto{\pgfqpoint{0.741227in}{1.605360in}}%
\pgfpathlineto{\pgfqpoint{0.749330in}{1.604412in}}%
\pgfpathlineto{\pgfqpoint{0.765534in}{1.595072in}}%
\pgfpathlineto{\pgfqpoint{0.773636in}{1.594159in}}%
\pgfpathlineto{\pgfqpoint{0.781739in}{1.599717in}}%
\pgfpathlineto{\pgfqpoint{0.789841in}{1.595168in}}%
\pgfpathlineto{\pgfqpoint{0.806045in}{1.597197in}}%
\pgfpathlineto{\pgfqpoint{0.814148in}{1.598633in}}%
\pgfpathlineto{\pgfqpoint{0.830352in}{1.594328in}}%
\pgfpathlineto{\pgfqpoint{0.838455in}{1.589001in}}%
\pgfpathlineto{\pgfqpoint{0.846557in}{1.581547in}}%
\pgfpathlineto{\pgfqpoint{0.854659in}{1.576506in}}%
\pgfpathlineto{\pgfqpoint{0.862761in}{1.574235in}}%
\pgfpathlineto{\pgfqpoint{0.870864in}{1.576170in}}%
\pgfpathlineto{\pgfqpoint{0.887068in}{1.562253in}}%
\pgfpathlineto{\pgfqpoint{0.895170in}{1.562553in}}%
\pgfpathlineto{\pgfqpoint{0.911375in}{1.551260in}}%
\pgfpathlineto{\pgfqpoint{0.919477in}{1.549554in}}%
\pgfpathlineto{\pgfqpoint{0.927580in}{1.540916in}}%
\pgfpathlineto{\pgfqpoint{0.935682in}{1.539467in}}%
\pgfpathlineto{\pgfqpoint{0.943784in}{1.533464in}}%
\pgfpathlineto{\pgfqpoint{0.951886in}{1.537068in}}%
\pgfpathlineto{\pgfqpoint{0.959989in}{1.527458in}}%
\pgfpathlineto{\pgfqpoint{0.968091in}{1.527874in}}%
\pgfpathlineto{\pgfqpoint{0.984295in}{1.526140in}}%
\pgfpathlineto{\pgfqpoint{1.008602in}{1.509562in}}%
\pgfpathlineto{\pgfqpoint{1.024807in}{1.499090in}}%
\pgfpathlineto{\pgfqpoint{1.032909in}{1.501919in}}%
\pgfpathlineto{\pgfqpoint{1.041011in}{1.495517in}}%
\pgfpathlineto{\pgfqpoint{1.049114in}{1.496870in}}%
\pgfpathlineto{\pgfqpoint{1.073420in}{1.491232in}}%
\pgfpathlineto{\pgfqpoint{1.081523in}{1.493815in}}%
\pgfpathlineto{\pgfqpoint{1.089625in}{1.489725in}}%
\pgfpathlineto{\pgfqpoint{1.097727in}{1.489737in}}%
\pgfpathlineto{\pgfqpoint{1.105830in}{1.492465in}}%
\pgfpathlineto{\pgfqpoint{1.113932in}{1.487630in}}%
\pgfpathlineto{\pgfqpoint{1.122034in}{1.484723in}}%
\pgfpathlineto{\pgfqpoint{1.138239in}{1.473661in}}%
\pgfpathlineto{\pgfqpoint{1.146341in}{1.469657in}}%
\pgfpathlineto{\pgfqpoint{1.154443in}{1.470275in}}%
\pgfpathlineto{\pgfqpoint{1.162545in}{1.466761in}}%
\pgfpathlineto{\pgfqpoint{1.170648in}{1.466432in}}%
\pgfpathlineto{\pgfqpoint{1.178750in}{1.463363in}}%
\pgfpathlineto{\pgfqpoint{1.186852in}{1.456958in}}%
\pgfpathlineto{\pgfqpoint{1.219261in}{1.441142in}}%
\pgfpathlineto{\pgfqpoint{1.243568in}{1.433983in}}%
\pgfpathlineto{\pgfqpoint{1.251670in}{1.435221in}}%
\pgfpathlineto{\pgfqpoint{1.267875in}{1.433107in}}%
\pgfpathlineto{\pgfqpoint{1.332693in}{1.417197in}}%
\pgfpathlineto{\pgfqpoint{1.340795in}{1.417354in}}%
\pgfpathlineto{\pgfqpoint{1.373205in}{1.406403in}}%
\pgfpathlineto{\pgfqpoint{1.389409in}{1.403861in}}%
\pgfpathlineto{\pgfqpoint{1.397511in}{1.405347in}}%
\pgfpathlineto{\pgfqpoint{1.405614in}{1.403240in}}%
\pgfpathlineto{\pgfqpoint{1.421818in}{1.393805in}}%
\pgfpathlineto{\pgfqpoint{1.429920in}{1.393170in}}%
\pgfpathlineto{\pgfqpoint{1.446125in}{1.387496in}}%
\pgfpathlineto{\pgfqpoint{1.454227in}{1.383262in}}%
\pgfpathlineto{\pgfqpoint{1.462330in}{1.382226in}}%
\pgfpathlineto{\pgfqpoint{1.486636in}{1.370547in}}%
\pgfpathlineto{\pgfqpoint{1.494739in}{1.370201in}}%
\pgfpathlineto{\pgfqpoint{1.502841in}{1.366047in}}%
\pgfpathlineto{\pgfqpoint{1.535250in}{1.360557in}}%
\pgfpathlineto{\pgfqpoint{1.559557in}{1.350177in}}%
\pgfpathlineto{\pgfqpoint{1.591966in}{1.339292in}}%
\pgfpathlineto{\pgfqpoint{1.608170in}{1.331265in}}%
\pgfpathlineto{\pgfqpoint{1.616273in}{1.329270in}}%
\pgfpathlineto{\pgfqpoint{1.624375in}{1.325629in}}%
\pgfpathlineto{\pgfqpoint{1.632477in}{1.324869in}}%
\pgfpathlineto{\pgfqpoint{1.640580in}{1.321804in}}%
\pgfpathlineto{\pgfqpoint{1.648682in}{1.321714in}}%
\pgfpathlineto{\pgfqpoint{1.656784in}{1.318131in}}%
\pgfpathlineto{\pgfqpoint{1.664886in}{1.319304in}}%
\pgfpathlineto{\pgfqpoint{1.681091in}{1.310775in}}%
\pgfpathlineto{\pgfqpoint{1.689193in}{1.305552in}}%
\pgfpathlineto{\pgfqpoint{1.697295in}{1.305086in}}%
\pgfpathlineto{\pgfqpoint{1.713500in}{1.297808in}}%
\pgfpathlineto{\pgfqpoint{1.721602in}{1.292627in}}%
\pgfpathlineto{\pgfqpoint{1.745909in}{1.287972in}}%
\pgfpathlineto{\pgfqpoint{1.754011in}{1.284001in}}%
\pgfpathlineto{\pgfqpoint{1.762114in}{1.282661in}}%
\pgfpathlineto{\pgfqpoint{1.778318in}{1.277959in}}%
\pgfpathlineto{\pgfqpoint{1.786420in}{1.278284in}}%
\pgfpathlineto{\pgfqpoint{1.802625in}{1.271778in}}%
\pgfpathlineto{\pgfqpoint{1.810727in}{1.272430in}}%
\pgfpathlineto{\pgfqpoint{1.835034in}{1.260263in}}%
\pgfpathlineto{\pgfqpoint{1.843136in}{1.256531in}}%
\pgfpathlineto{\pgfqpoint{1.851239in}{1.256861in}}%
\pgfpathlineto{\pgfqpoint{1.867443in}{1.252989in}}%
\pgfpathlineto{\pgfqpoint{1.883648in}{1.249671in}}%
\pgfpathlineto{\pgfqpoint{1.899852in}{1.242055in}}%
\pgfpathlineto{\pgfqpoint{1.907955in}{1.240615in}}%
\pgfpathlineto{\pgfqpoint{1.916057in}{1.237135in}}%
\pgfpathlineto{\pgfqpoint{1.924159in}{1.236601in}}%
\pgfpathlineto{\pgfqpoint{1.948466in}{1.226141in}}%
\pgfpathlineto{\pgfqpoint{1.964670in}{1.222729in}}%
\pgfpathlineto{\pgfqpoint{1.980875in}{1.217111in}}%
\pgfpathlineto{\pgfqpoint{1.988977in}{1.215757in}}%
\pgfpathlineto{\pgfqpoint{1.988977in}{1.215757in}}%
\pgfusepath{stroke}%
\end{pgfscope}%
\begin{pgfscope}%
\pgfpathrectangle{\pgfqpoint{0.287500in}{0.375000in}}{\pgfqpoint{1.782500in}{2.265000in}}%
\pgfusepath{clip}%
\pgfsetroundcap%
\pgfsetroundjoin%
\pgfsetlinewidth{1.505625pt}%
\definecolor{currentstroke}{rgb}{0.172549,0.627451,0.172549}%
\pgfsetstrokecolor{currentstroke}%
\pgfsetdash{}{0pt}%
\pgfpathmoveto{\pgfqpoint{0.368523in}{2.369120in}}%
\pgfpathlineto{\pgfqpoint{0.376625in}{2.042919in}}%
\pgfpathlineto{\pgfqpoint{0.384727in}{1.857768in}}%
\pgfpathlineto{\pgfqpoint{0.392830in}{1.753779in}}%
\pgfpathlineto{\pgfqpoint{0.409034in}{1.619342in}}%
\pgfpathlineto{\pgfqpoint{0.417136in}{1.567751in}}%
\pgfpathlineto{\pgfqpoint{0.425239in}{1.566159in}}%
\pgfpathlineto{\pgfqpoint{0.457648in}{1.453378in}}%
\pgfpathlineto{\pgfqpoint{0.473852in}{1.421891in}}%
\pgfpathlineto{\pgfqpoint{0.490057in}{1.381038in}}%
\pgfpathlineto{\pgfqpoint{0.498159in}{1.382647in}}%
\pgfpathlineto{\pgfqpoint{0.506261in}{1.374916in}}%
\pgfpathlineto{\pgfqpoint{0.514364in}{1.385969in}}%
\pgfpathlineto{\pgfqpoint{0.522466in}{1.377270in}}%
\pgfpathlineto{\pgfqpoint{0.530568in}{1.370969in}}%
\pgfpathlineto{\pgfqpoint{0.538670in}{1.350132in}}%
\pgfpathlineto{\pgfqpoint{0.562977in}{1.311379in}}%
\pgfpathlineto{\pgfqpoint{0.571080in}{1.305070in}}%
\pgfpathlineto{\pgfqpoint{0.579182in}{1.294190in}}%
\pgfpathlineto{\pgfqpoint{0.587284in}{1.293454in}}%
\pgfpathlineto{\pgfqpoint{0.595386in}{1.287512in}}%
\pgfpathlineto{\pgfqpoint{0.603489in}{1.272248in}}%
\pgfpathlineto{\pgfqpoint{0.611591in}{1.267070in}}%
\pgfpathlineto{\pgfqpoint{0.619693in}{1.257753in}}%
\pgfpathlineto{\pgfqpoint{0.627795in}{1.245932in}}%
\pgfpathlineto{\pgfqpoint{0.635898in}{1.238533in}}%
\pgfpathlineto{\pgfqpoint{0.652102in}{1.227896in}}%
\pgfpathlineto{\pgfqpoint{0.668307in}{1.210072in}}%
\pgfpathlineto{\pgfqpoint{0.676409in}{1.203760in}}%
\pgfpathlineto{\pgfqpoint{0.684511in}{1.207519in}}%
\pgfpathlineto{\pgfqpoint{0.692614in}{1.205109in}}%
\pgfpathlineto{\pgfqpoint{0.700716in}{1.208947in}}%
\pgfpathlineto{\pgfqpoint{0.708818in}{1.205120in}}%
\pgfpathlineto{\pgfqpoint{0.725023in}{1.204197in}}%
\pgfpathlineto{\pgfqpoint{0.733125in}{1.206819in}}%
\pgfpathlineto{\pgfqpoint{0.741227in}{1.202712in}}%
\pgfpathlineto{\pgfqpoint{0.749330in}{1.203622in}}%
\pgfpathlineto{\pgfqpoint{0.757432in}{1.201130in}}%
\pgfpathlineto{\pgfqpoint{0.773636in}{1.209569in}}%
\pgfpathlineto{\pgfqpoint{0.781739in}{1.207328in}}%
\pgfpathlineto{\pgfqpoint{0.789841in}{1.208419in}}%
\pgfpathlineto{\pgfqpoint{0.797943in}{1.201500in}}%
\pgfpathlineto{\pgfqpoint{0.806045in}{1.201030in}}%
\pgfpathlineto{\pgfqpoint{0.822250in}{1.194064in}}%
\pgfpathlineto{\pgfqpoint{0.838455in}{1.194900in}}%
\pgfpathlineto{\pgfqpoint{0.846557in}{1.200893in}}%
\pgfpathlineto{\pgfqpoint{0.854659in}{1.203264in}}%
\pgfpathlineto{\pgfqpoint{0.862761in}{1.203995in}}%
\pgfpathlineto{\pgfqpoint{0.878966in}{1.211040in}}%
\pgfpathlineto{\pgfqpoint{0.887068in}{1.207755in}}%
\pgfpathlineto{\pgfqpoint{0.895170in}{1.211806in}}%
\pgfpathlineto{\pgfqpoint{0.919477in}{1.210430in}}%
\pgfpathlineto{\pgfqpoint{0.927580in}{1.207946in}}%
\pgfpathlineto{\pgfqpoint{0.935682in}{1.203153in}}%
\pgfpathlineto{\pgfqpoint{0.951886in}{1.203456in}}%
\pgfpathlineto{\pgfqpoint{0.959989in}{1.207652in}}%
\pgfpathlineto{\pgfqpoint{0.968091in}{1.204649in}}%
\pgfpathlineto{\pgfqpoint{0.984295in}{1.190827in}}%
\pgfpathlineto{\pgfqpoint{0.992398in}{1.183179in}}%
\pgfpathlineto{\pgfqpoint{1.008602in}{1.184330in}}%
\pgfpathlineto{\pgfqpoint{1.032909in}{1.172063in}}%
\pgfpathlineto{\pgfqpoint{1.049114in}{1.165347in}}%
\pgfpathlineto{\pgfqpoint{1.057216in}{1.167478in}}%
\pgfpathlineto{\pgfqpoint{1.065318in}{1.163832in}}%
\pgfpathlineto{\pgfqpoint{1.073420in}{1.156627in}}%
\pgfpathlineto{\pgfqpoint{1.081523in}{1.151608in}}%
\pgfpathlineto{\pgfqpoint{1.097727in}{1.144688in}}%
\pgfpathlineto{\pgfqpoint{1.105830in}{1.145673in}}%
\pgfpathlineto{\pgfqpoint{1.113932in}{1.150997in}}%
\pgfpathlineto{\pgfqpoint{1.122034in}{1.152471in}}%
\pgfpathlineto{\pgfqpoint{1.130136in}{1.157616in}}%
\pgfpathlineto{\pgfqpoint{1.138239in}{1.154871in}}%
\pgfpathlineto{\pgfqpoint{1.146341in}{1.147941in}}%
\pgfpathlineto{\pgfqpoint{1.154443in}{1.144173in}}%
\pgfpathlineto{\pgfqpoint{1.162545in}{1.138284in}}%
\pgfpathlineto{\pgfqpoint{1.170648in}{1.139751in}}%
\pgfpathlineto{\pgfqpoint{1.178750in}{1.134567in}}%
\pgfpathlineto{\pgfqpoint{1.219261in}{1.118984in}}%
\pgfpathlineto{\pgfqpoint{1.227364in}{1.119621in}}%
\pgfpathlineto{\pgfqpoint{1.243568in}{1.114427in}}%
\pgfpathlineto{\pgfqpoint{1.267875in}{1.110707in}}%
\pgfpathlineto{\pgfqpoint{1.284080in}{1.106635in}}%
\pgfpathlineto{\pgfqpoint{1.300284in}{1.095908in}}%
\pgfpathlineto{\pgfqpoint{1.332693in}{1.090961in}}%
\pgfpathlineto{\pgfqpoint{1.348898in}{1.084233in}}%
\pgfpathlineto{\pgfqpoint{1.357000in}{1.082675in}}%
\pgfpathlineto{\pgfqpoint{1.373205in}{1.076504in}}%
\pgfpathlineto{\pgfqpoint{1.381307in}{1.072875in}}%
\pgfpathlineto{\pgfqpoint{1.389409in}{1.072889in}}%
\pgfpathlineto{\pgfqpoint{1.421818in}{1.059009in}}%
\pgfpathlineto{\pgfqpoint{1.429920in}{1.060266in}}%
\pgfpathlineto{\pgfqpoint{1.438023in}{1.055509in}}%
\pgfpathlineto{\pgfqpoint{1.454227in}{1.053741in}}%
\pgfpathlineto{\pgfqpoint{1.462330in}{1.054456in}}%
\pgfpathlineto{\pgfqpoint{1.486636in}{1.050659in}}%
\pgfpathlineto{\pgfqpoint{1.494739in}{1.050461in}}%
\pgfpathlineto{\pgfqpoint{1.510943in}{1.046373in}}%
\pgfpathlineto{\pgfqpoint{1.527148in}{1.042329in}}%
\pgfpathlineto{\pgfqpoint{1.535250in}{1.039503in}}%
\pgfpathlineto{\pgfqpoint{1.567659in}{1.036479in}}%
\pgfpathlineto{\pgfqpoint{1.575761in}{1.036356in}}%
\pgfpathlineto{\pgfqpoint{1.591966in}{1.039869in}}%
\pgfpathlineto{\pgfqpoint{1.608170in}{1.038989in}}%
\pgfpathlineto{\pgfqpoint{1.640580in}{1.025889in}}%
\pgfpathlineto{\pgfqpoint{1.648682in}{1.025070in}}%
\pgfpathlineto{\pgfqpoint{1.656784in}{1.028857in}}%
\pgfpathlineto{\pgfqpoint{1.672989in}{1.028769in}}%
\pgfpathlineto{\pgfqpoint{1.689193in}{1.024876in}}%
\pgfpathlineto{\pgfqpoint{1.705398in}{1.026971in}}%
\pgfpathlineto{\pgfqpoint{1.721602in}{1.023820in}}%
\pgfpathlineto{\pgfqpoint{1.729705in}{1.024925in}}%
\pgfpathlineto{\pgfqpoint{1.737807in}{1.022750in}}%
\pgfpathlineto{\pgfqpoint{1.754011in}{1.025214in}}%
\pgfpathlineto{\pgfqpoint{1.762114in}{1.027190in}}%
\pgfpathlineto{\pgfqpoint{1.794523in}{1.023378in}}%
\pgfpathlineto{\pgfqpoint{1.810727in}{1.025954in}}%
\pgfpathlineto{\pgfqpoint{1.818830in}{1.023145in}}%
\pgfpathlineto{\pgfqpoint{1.826932in}{1.023207in}}%
\pgfpathlineto{\pgfqpoint{1.843136in}{1.019398in}}%
\pgfpathlineto{\pgfqpoint{1.851239in}{1.020522in}}%
\pgfpathlineto{\pgfqpoint{1.875545in}{1.013803in}}%
\pgfpathlineto{\pgfqpoint{1.907955in}{1.012888in}}%
\pgfpathlineto{\pgfqpoint{1.932261in}{1.005714in}}%
\pgfpathlineto{\pgfqpoint{1.956568in}{1.006512in}}%
\pgfpathlineto{\pgfqpoint{1.964670in}{1.004861in}}%
\pgfpathlineto{\pgfqpoint{1.972773in}{1.005684in}}%
\pgfpathlineto{\pgfqpoint{1.988977in}{1.002754in}}%
\pgfpathlineto{\pgfqpoint{1.988977in}{1.002754in}}%
\pgfusepath{stroke}%
\end{pgfscope}%
\begin{pgfscope}%
\pgfsetrectcap%
\pgfsetmiterjoin%
\pgfsetlinewidth{0.000000pt}%
\definecolor{currentstroke}{rgb}{1.000000,1.000000,1.000000}%
\pgfsetstrokecolor{currentstroke}%
\pgfsetdash{}{0pt}%
\pgfpathmoveto{\pgfqpoint{0.287500in}{0.375000in}}%
\pgfpathlineto{\pgfqpoint{0.287500in}{2.640000in}}%
\pgfusepath{}%
\end{pgfscope}%
\begin{pgfscope}%
\pgfsetrectcap%
\pgfsetmiterjoin%
\pgfsetlinewidth{0.000000pt}%
\definecolor{currentstroke}{rgb}{1.000000,1.000000,1.000000}%
\pgfsetstrokecolor{currentstroke}%
\pgfsetdash{}{0pt}%
\pgfpathmoveto{\pgfqpoint{2.070000in}{0.375000in}}%
\pgfpathlineto{\pgfqpoint{2.070000in}{2.640000in}}%
\pgfusepath{}%
\end{pgfscope}%
\begin{pgfscope}%
\pgfsetrectcap%
\pgfsetmiterjoin%
\pgfsetlinewidth{0.000000pt}%
\definecolor{currentstroke}{rgb}{1.000000,1.000000,1.000000}%
\pgfsetstrokecolor{currentstroke}%
\pgfsetdash{}{0pt}%
\pgfpathmoveto{\pgfqpoint{0.287500in}{0.375000in}}%
\pgfpathlineto{\pgfqpoint{2.070000in}{0.375000in}}%
\pgfusepath{}%
\end{pgfscope}%
\begin{pgfscope}%
\pgfsetrectcap%
\pgfsetmiterjoin%
\pgfsetlinewidth{0.000000pt}%
\definecolor{currentstroke}{rgb}{1.000000,1.000000,1.000000}%
\pgfsetstrokecolor{currentstroke}%
\pgfsetdash{}{0pt}%
\pgfpathmoveto{\pgfqpoint{0.287500in}{2.640000in}}%
\pgfpathlineto{\pgfqpoint{2.070000in}{2.640000in}}%
\pgfusepath{}%
\end{pgfscope}%
\begin{pgfscope}%
\definecolor{textcolor}{rgb}{0.150000,0.150000,0.150000}%
\pgfsetstrokecolor{textcolor}%
\pgfsetfillcolor{textcolor}%
\pgftext[x=1.178750in,y=2.723333in,,base]{\color{textcolor}\rmfamily\fontsize{8.000000}{9.600000}\selectfont Embedded SinOne in 3D}%
\end{pgfscope}%
\begin{pgfscope}%
\pgfsetroundcap%
\pgfsetroundjoin%
\pgfsetlinewidth{1.505625pt}%
\definecolor{currentstroke}{rgb}{0.121569,0.466667,0.705882}%
\pgfsetstrokecolor{currentstroke}%
\pgfsetdash{}{0pt}%
\pgfpathmoveto{\pgfqpoint{0.339607in}{2.494470in}}%
\pgfpathlineto{\pgfqpoint{0.561829in}{2.494470in}}%
\pgfusepath{stroke}%
\end{pgfscope}%
\begin{pgfscope}%
\definecolor{textcolor}{rgb}{0.150000,0.150000,0.150000}%
\pgfsetstrokecolor{textcolor}%
\pgfsetfillcolor{textcolor}%
\pgftext[x=0.650718in,y=2.455582in,left,base]{\color{textcolor}\rmfamily\fontsize{8.000000}{9.600000}\selectfont 5 x DNGO retrain-reset}%
\end{pgfscope}%
\begin{pgfscope}%
\pgfsetroundcap%
\pgfsetroundjoin%
\pgfsetlinewidth{1.505625pt}%
\definecolor{currentstroke}{rgb}{1.000000,0.498039,0.054902}%
\pgfsetstrokecolor{currentstroke}%
\pgfsetdash{}{0pt}%
\pgfpathmoveto{\pgfqpoint{0.339607in}{2.331385in}}%
\pgfpathlineto{\pgfqpoint{0.561829in}{2.331385in}}%
\pgfusepath{stroke}%
\end{pgfscope}%
\begin{pgfscope}%
\definecolor{textcolor}{rgb}{0.150000,0.150000,0.150000}%
\pgfsetstrokecolor{textcolor}%
\pgfsetfillcolor{textcolor}%
\pgftext[x=0.650718in,y=2.292496in,left,base]{\color{textcolor}\rmfamily\fontsize{8.000000}{9.600000}\selectfont DNGO retrain-reset}%
\end{pgfscope}%
\begin{pgfscope}%
\pgfsetroundcap%
\pgfsetroundjoin%
\pgfsetlinewidth{1.505625pt}%
\definecolor{currentstroke}{rgb}{0.172549,0.627451,0.172549}%
\pgfsetstrokecolor{currentstroke}%
\pgfsetdash{}{0pt}%
\pgfpathmoveto{\pgfqpoint{0.339607in}{2.168299in}}%
\pgfpathlineto{\pgfqpoint{0.561829in}{2.168299in}}%
\pgfusepath{stroke}%
\end{pgfscope}%
\begin{pgfscope}%
\definecolor{textcolor}{rgb}{0.150000,0.150000,0.150000}%
\pgfsetstrokecolor{textcolor}%
\pgfsetfillcolor{textcolor}%
\pgftext[x=0.650718in,y=2.129410in,left,base]{\color{textcolor}\rmfamily\fontsize{8.000000}{9.600000}\selectfont GP}%
\end{pgfscope}%
\end{pgfpicture}%
\makeatother%
\endgroup%

        \end{minipage}
        \captionof{figure}{
            For every embedded SinOne the Cumulative Regret is normalized by the round $t$ and plotted similarly to \parencite{contal_gaussian_2014}.
            }
    \end{minipage}


    \subsection{Training}\label{sec:apptraining}

        \begin{minipage}{\linewidth}
            \centering
            %% Creator: Matplotlib, PGF backend
%%
%% To include the figure in your LaTeX document, write
%%   \input{<filename>.pgf}
%%
%% Make sure the required packages are loaded in your preamble
%%   \usepackage{pgf}
%%
%% Figures using additional raster images can only be included by \input if
%% they are in the same directory as the main LaTeX file. For loading figures
%% from other directories you can use the `import` package
%%   \usepackage{import}
%% and then include the figures with
%%   \import{<path to file>}{<filename>.pgf}
%%
%% Matplotlib used the following preamble
%%   \usepackage{gensymb}
%%   \usepackage{fontspec}
%%   \setmainfont{DejaVu Serif}
%%   \setsansfont{Arial}
%%   \setmonofont{DejaVu Sans Mono}
%%
\begingroup%
\makeatletter%
\begin{pgfpicture}%
\pgfpathrectangle{\pgfpointorigin}{\pgfqpoint{6.900000in}{3.000000in}}%
\pgfusepath{use as bounding box, clip}%
\begin{pgfscope}%
\pgfsetbuttcap%
\pgfsetmiterjoin%
\definecolor{currentfill}{rgb}{1.000000,1.000000,1.000000}%
\pgfsetfillcolor{currentfill}%
\pgfsetlinewidth{0.000000pt}%
\definecolor{currentstroke}{rgb}{1.000000,1.000000,1.000000}%
\pgfsetstrokecolor{currentstroke}%
\pgfsetdash{}{0pt}%
\pgfpathmoveto{\pgfqpoint{0.000000in}{0.000000in}}%
\pgfpathlineto{\pgfqpoint{6.900000in}{0.000000in}}%
\pgfpathlineto{\pgfqpoint{6.900000in}{3.000000in}}%
\pgfpathlineto{\pgfqpoint{0.000000in}{3.000000in}}%
\pgfpathclose%
\pgfusepath{fill}%
\end{pgfscope}%
\begin{pgfscope}%
\pgfsetbuttcap%
\pgfsetmiterjoin%
\definecolor{currentfill}{rgb}{0.917647,0.917647,0.949020}%
\pgfsetfillcolor{currentfill}%
\pgfsetlinewidth{0.000000pt}%
\definecolor{currentstroke}{rgb}{0.000000,0.000000,0.000000}%
\pgfsetstrokecolor{currentstroke}%
\pgfsetstrokeopacity{0.000000}%
\pgfsetdash{}{0pt}%
\pgfpathmoveto{\pgfqpoint{0.862500in}{0.375000in}}%
\pgfpathlineto{\pgfqpoint{6.210000in}{0.375000in}}%
\pgfpathlineto{\pgfqpoint{6.210000in}{2.640000in}}%
\pgfpathlineto{\pgfqpoint{0.862500in}{2.640000in}}%
\pgfpathclose%
\pgfusepath{fill}%
\end{pgfscope}%
\begin{pgfscope}%
\pgfpathrectangle{\pgfqpoint{0.862500in}{0.375000in}}{\pgfqpoint{5.347500in}{2.265000in}}%
\pgfusepath{clip}%
\pgfsetroundcap%
\pgfsetroundjoin%
\pgfsetlinewidth{0.803000pt}%
\definecolor{currentstroke}{rgb}{1.000000,1.000000,1.000000}%
\pgfsetstrokecolor{currentstroke}%
\pgfsetdash{}{0pt}%
\pgfpathmoveto{\pgfqpoint{0.862500in}{0.375000in}}%
\pgfpathlineto{\pgfqpoint{0.862500in}{2.640000in}}%
\pgfusepath{stroke}%
\end{pgfscope}%
\begin{pgfscope}%
\definecolor{textcolor}{rgb}{0.150000,0.150000,0.150000}%
\pgfsetstrokecolor{textcolor}%
\pgfsetfillcolor{textcolor}%
\pgftext[x=0.862500in,y=0.326389in,,top]{\color{textcolor}\rmfamily\fontsize{8.000000}{9.600000}\selectfont \(\displaystyle 0\)}%
\end{pgfscope}%
\begin{pgfscope}%
\pgfpathrectangle{\pgfqpoint{0.862500in}{0.375000in}}{\pgfqpoint{5.347500in}{2.265000in}}%
\pgfusepath{clip}%
\pgfsetroundcap%
\pgfsetroundjoin%
\pgfsetlinewidth{0.803000pt}%
\definecolor{currentstroke}{rgb}{1.000000,1.000000,1.000000}%
\pgfsetstrokecolor{currentstroke}%
\pgfsetdash{}{0pt}%
\pgfpathmoveto{\pgfqpoint{1.932000in}{0.375000in}}%
\pgfpathlineto{\pgfqpoint{1.932000in}{2.640000in}}%
\pgfusepath{stroke}%
\end{pgfscope}%
\begin{pgfscope}%
\definecolor{textcolor}{rgb}{0.150000,0.150000,0.150000}%
\pgfsetstrokecolor{textcolor}%
\pgfsetfillcolor{textcolor}%
\pgftext[x=1.932000in,y=0.326389in,,top]{\color{textcolor}\rmfamily\fontsize{8.000000}{9.600000}\selectfont \(\displaystyle 50\)}%
\end{pgfscope}%
\begin{pgfscope}%
\pgfpathrectangle{\pgfqpoint{0.862500in}{0.375000in}}{\pgfqpoint{5.347500in}{2.265000in}}%
\pgfusepath{clip}%
\pgfsetroundcap%
\pgfsetroundjoin%
\pgfsetlinewidth{0.803000pt}%
\definecolor{currentstroke}{rgb}{1.000000,1.000000,1.000000}%
\pgfsetstrokecolor{currentstroke}%
\pgfsetdash{}{0pt}%
\pgfpathmoveto{\pgfqpoint{3.001500in}{0.375000in}}%
\pgfpathlineto{\pgfqpoint{3.001500in}{2.640000in}}%
\pgfusepath{stroke}%
\end{pgfscope}%
\begin{pgfscope}%
\definecolor{textcolor}{rgb}{0.150000,0.150000,0.150000}%
\pgfsetstrokecolor{textcolor}%
\pgfsetfillcolor{textcolor}%
\pgftext[x=3.001500in,y=0.326389in,,top]{\color{textcolor}\rmfamily\fontsize{8.000000}{9.600000}\selectfont \(\displaystyle 100\)}%
\end{pgfscope}%
\begin{pgfscope}%
\pgfpathrectangle{\pgfqpoint{0.862500in}{0.375000in}}{\pgfqpoint{5.347500in}{2.265000in}}%
\pgfusepath{clip}%
\pgfsetroundcap%
\pgfsetroundjoin%
\pgfsetlinewidth{0.803000pt}%
\definecolor{currentstroke}{rgb}{1.000000,1.000000,1.000000}%
\pgfsetstrokecolor{currentstroke}%
\pgfsetdash{}{0pt}%
\pgfpathmoveto{\pgfqpoint{4.071000in}{0.375000in}}%
\pgfpathlineto{\pgfqpoint{4.071000in}{2.640000in}}%
\pgfusepath{stroke}%
\end{pgfscope}%
\begin{pgfscope}%
\definecolor{textcolor}{rgb}{0.150000,0.150000,0.150000}%
\pgfsetstrokecolor{textcolor}%
\pgfsetfillcolor{textcolor}%
\pgftext[x=4.071000in,y=0.326389in,,top]{\color{textcolor}\rmfamily\fontsize{8.000000}{9.600000}\selectfont \(\displaystyle 150\)}%
\end{pgfscope}%
\begin{pgfscope}%
\pgfpathrectangle{\pgfqpoint{0.862500in}{0.375000in}}{\pgfqpoint{5.347500in}{2.265000in}}%
\pgfusepath{clip}%
\pgfsetroundcap%
\pgfsetroundjoin%
\pgfsetlinewidth{0.803000pt}%
\definecolor{currentstroke}{rgb}{1.000000,1.000000,1.000000}%
\pgfsetstrokecolor{currentstroke}%
\pgfsetdash{}{0pt}%
\pgfpathmoveto{\pgfqpoint{5.140500in}{0.375000in}}%
\pgfpathlineto{\pgfqpoint{5.140500in}{2.640000in}}%
\pgfusepath{stroke}%
\end{pgfscope}%
\begin{pgfscope}%
\definecolor{textcolor}{rgb}{0.150000,0.150000,0.150000}%
\pgfsetstrokecolor{textcolor}%
\pgfsetfillcolor{textcolor}%
\pgftext[x=5.140500in,y=0.326389in,,top]{\color{textcolor}\rmfamily\fontsize{8.000000}{9.600000}\selectfont \(\displaystyle 200\)}%
\end{pgfscope}%
\begin{pgfscope}%
\pgfpathrectangle{\pgfqpoint{0.862500in}{0.375000in}}{\pgfqpoint{5.347500in}{2.265000in}}%
\pgfusepath{clip}%
\pgfsetroundcap%
\pgfsetroundjoin%
\pgfsetlinewidth{0.803000pt}%
\definecolor{currentstroke}{rgb}{1.000000,1.000000,1.000000}%
\pgfsetstrokecolor{currentstroke}%
\pgfsetdash{}{0pt}%
\pgfpathmoveto{\pgfqpoint{6.210000in}{0.375000in}}%
\pgfpathlineto{\pgfqpoint{6.210000in}{2.640000in}}%
\pgfusepath{stroke}%
\end{pgfscope}%
\begin{pgfscope}%
\definecolor{textcolor}{rgb}{0.150000,0.150000,0.150000}%
\pgfsetstrokecolor{textcolor}%
\pgfsetfillcolor{textcolor}%
\pgftext[x=6.210000in,y=0.326389in,,top]{\color{textcolor}\rmfamily\fontsize{8.000000}{9.600000}\selectfont \(\displaystyle 250\)}%
\end{pgfscope}%
\begin{pgfscope}%
\definecolor{textcolor}{rgb}{0.150000,0.150000,0.150000}%
\pgfsetstrokecolor{textcolor}%
\pgfsetfillcolor{textcolor}%
\pgftext[x=3.536250in,y=0.163303in,,top]{\color{textcolor}\rmfamily\fontsize{8.000000}{9.600000}\selectfont Step}%
\end{pgfscope}%
\begin{pgfscope}%
\pgfpathrectangle{\pgfqpoint{0.862500in}{0.375000in}}{\pgfqpoint{5.347500in}{2.265000in}}%
\pgfusepath{clip}%
\pgfsetroundcap%
\pgfsetroundjoin%
\pgfsetlinewidth{0.803000pt}%
\definecolor{currentstroke}{rgb}{1.000000,1.000000,1.000000}%
\pgfsetstrokecolor{currentstroke}%
\pgfsetdash{}{0pt}%
\pgfpathmoveto{\pgfqpoint{0.862500in}{0.800270in}}%
\pgfpathlineto{\pgfqpoint{6.210000in}{0.800270in}}%
\pgfusepath{stroke}%
\end{pgfscope}%
\begin{pgfscope}%
\definecolor{textcolor}{rgb}{0.150000,0.150000,0.150000}%
\pgfsetstrokecolor{textcolor}%
\pgfsetfillcolor{textcolor}%
\pgftext[x=0.557716in,y=0.758060in,left,base]{\color{textcolor}\rmfamily\fontsize{8.000000}{9.600000}\selectfont \(\displaystyle 10^{-2}\)}%
\end{pgfscope}%
\begin{pgfscope}%
\pgfpathrectangle{\pgfqpoint{0.862500in}{0.375000in}}{\pgfqpoint{5.347500in}{2.265000in}}%
\pgfusepath{clip}%
\pgfsetroundcap%
\pgfsetroundjoin%
\pgfsetlinewidth{0.803000pt}%
\definecolor{currentstroke}{rgb}{1.000000,1.000000,1.000000}%
\pgfsetstrokecolor{currentstroke}%
\pgfsetdash{}{0pt}%
\pgfpathmoveto{\pgfqpoint{0.862500in}{1.480727in}}%
\pgfpathlineto{\pgfqpoint{6.210000in}{1.480727in}}%
\pgfusepath{stroke}%
\end{pgfscope}%
\begin{pgfscope}%
\definecolor{textcolor}{rgb}{0.150000,0.150000,0.150000}%
\pgfsetstrokecolor{textcolor}%
\pgfsetfillcolor{textcolor}%
\pgftext[x=0.557716in,y=1.438518in,left,base]{\color{textcolor}\rmfamily\fontsize{8.000000}{9.600000}\selectfont \(\displaystyle 10^{-1}\)}%
\end{pgfscope}%
\begin{pgfscope}%
\pgfpathrectangle{\pgfqpoint{0.862500in}{0.375000in}}{\pgfqpoint{5.347500in}{2.265000in}}%
\pgfusepath{clip}%
\pgfsetroundcap%
\pgfsetroundjoin%
\pgfsetlinewidth{0.803000pt}%
\definecolor{currentstroke}{rgb}{1.000000,1.000000,1.000000}%
\pgfsetstrokecolor{currentstroke}%
\pgfsetdash{}{0pt}%
\pgfpathmoveto{\pgfqpoint{0.862500in}{2.161185in}}%
\pgfpathlineto{\pgfqpoint{6.210000in}{2.161185in}}%
\pgfusepath{stroke}%
\end{pgfscope}%
\begin{pgfscope}%
\definecolor{textcolor}{rgb}{0.150000,0.150000,0.150000}%
\pgfsetstrokecolor{textcolor}%
\pgfsetfillcolor{textcolor}%
\pgftext[x=0.637962in,y=2.118976in,left,base]{\color{textcolor}\rmfamily\fontsize{8.000000}{9.600000}\selectfont \(\displaystyle 10^{0}\)}%
\end{pgfscope}%
\begin{pgfscope}%
\definecolor{textcolor}{rgb}{0.150000,0.150000,0.150000}%
\pgfsetstrokecolor{textcolor}%
\pgfsetfillcolor{textcolor}%
\pgftext[x=0.502160in,y=1.507500in,,bottom,rotate=90.000000]{\color{textcolor}\rmfamily\fontsize{8.000000}{9.600000}\selectfont Simple Regret}%
\end{pgfscope}%
\begin{pgfscope}%
\pgfpathrectangle{\pgfqpoint{0.862500in}{0.375000in}}{\pgfqpoint{5.347500in}{2.265000in}}%
\pgfusepath{clip}%
\pgfsetbuttcap%
\pgfsetroundjoin%
\definecolor{currentfill}{rgb}{0.121569,0.466667,0.705882}%
\pgfsetfillcolor{currentfill}%
\pgfsetfillopacity{0.200000}%
\pgfsetlinewidth{0.000000pt}%
\definecolor{currentstroke}{rgb}{0.000000,0.000000,0.000000}%
\pgfsetstrokecolor{currentstroke}%
\pgfsetdash{}{0pt}%
\pgfpathmoveto{\pgfqpoint{0.862500in}{2.414113in}}%
\pgfpathlineto{\pgfqpoint{0.862500in}{2.525532in}}%
\pgfpathlineto{\pgfqpoint{0.883890in}{2.517852in}}%
\pgfpathlineto{\pgfqpoint{0.905280in}{2.457137in}}%
\pgfpathlineto{\pgfqpoint{0.926670in}{2.419527in}}%
\pgfpathlineto{\pgfqpoint{0.948060in}{2.399596in}}%
\pgfpathlineto{\pgfqpoint{0.969450in}{2.357206in}}%
\pgfpathlineto{\pgfqpoint{0.990840in}{2.334826in}}%
\pgfpathlineto{\pgfqpoint{1.012230in}{2.333870in}}%
\pgfpathlineto{\pgfqpoint{1.033620in}{2.333870in}}%
\pgfpathlineto{\pgfqpoint{1.055010in}{2.333870in}}%
\pgfpathlineto{\pgfqpoint{1.076400in}{2.288791in}}%
\pgfpathlineto{\pgfqpoint{1.097790in}{2.288791in}}%
\pgfpathlineto{\pgfqpoint{1.119180in}{2.288791in}}%
\pgfpathlineto{\pgfqpoint{1.140570in}{2.232923in}}%
\pgfpathlineto{\pgfqpoint{1.161960in}{2.232923in}}%
\pgfpathlineto{\pgfqpoint{1.183350in}{2.225336in}}%
\pgfpathlineto{\pgfqpoint{1.204740in}{2.225336in}}%
\pgfpathlineto{\pgfqpoint{1.226130in}{2.177677in}}%
\pgfpathlineto{\pgfqpoint{1.247520in}{2.177677in}}%
\pgfpathlineto{\pgfqpoint{1.268910in}{2.177677in}}%
\pgfpathlineto{\pgfqpoint{1.290300in}{2.177677in}}%
\pgfpathlineto{\pgfqpoint{1.311690in}{2.177677in}}%
\pgfpathlineto{\pgfqpoint{1.333080in}{2.141172in}}%
\pgfpathlineto{\pgfqpoint{1.354470in}{2.141172in}}%
\pgfpathlineto{\pgfqpoint{1.375860in}{2.128741in}}%
\pgfpathlineto{\pgfqpoint{1.397250in}{2.101450in}}%
\pgfpathlineto{\pgfqpoint{1.418640in}{2.071269in}}%
\pgfpathlineto{\pgfqpoint{1.440030in}{2.071269in}}%
\pgfpathlineto{\pgfqpoint{1.461420in}{2.071269in}}%
\pgfpathlineto{\pgfqpoint{1.482810in}{2.071269in}}%
\pgfpathlineto{\pgfqpoint{1.504200in}{2.061537in}}%
\pgfpathlineto{\pgfqpoint{1.525590in}{2.061537in}}%
\pgfpathlineto{\pgfqpoint{1.546980in}{2.061537in}}%
\pgfpathlineto{\pgfqpoint{1.568370in}{2.061537in}}%
\pgfpathlineto{\pgfqpoint{1.589760in}{2.061537in}}%
\pgfpathlineto{\pgfqpoint{1.611150in}{2.061537in}}%
\pgfpathlineto{\pgfqpoint{1.632540in}{2.060498in}}%
\pgfpathlineto{\pgfqpoint{1.653930in}{2.060498in}}%
\pgfpathlineto{\pgfqpoint{1.675320in}{2.060498in}}%
\pgfpathlineto{\pgfqpoint{1.696710in}{2.030074in}}%
\pgfpathlineto{\pgfqpoint{1.718100in}{2.010134in}}%
\pgfpathlineto{\pgfqpoint{1.739490in}{2.010134in}}%
\pgfpathlineto{\pgfqpoint{1.760880in}{2.010134in}}%
\pgfpathlineto{\pgfqpoint{1.782270in}{2.010134in}}%
\pgfpathlineto{\pgfqpoint{1.803660in}{2.010134in}}%
\pgfpathlineto{\pgfqpoint{1.825050in}{2.010134in}}%
\pgfpathlineto{\pgfqpoint{1.846440in}{2.010134in}}%
\pgfpathlineto{\pgfqpoint{1.867830in}{2.008629in}}%
\pgfpathlineto{\pgfqpoint{1.889220in}{2.008629in}}%
\pgfpathlineto{\pgfqpoint{1.910610in}{2.008629in}}%
\pgfpathlineto{\pgfqpoint{1.932000in}{1.978666in}}%
\pgfpathlineto{\pgfqpoint{1.953390in}{1.978666in}}%
\pgfpathlineto{\pgfqpoint{1.974780in}{1.978666in}}%
\pgfpathlineto{\pgfqpoint{1.996170in}{1.978666in}}%
\pgfpathlineto{\pgfqpoint{2.017560in}{1.974085in}}%
\pgfpathlineto{\pgfqpoint{2.038950in}{1.974085in}}%
\pgfpathlineto{\pgfqpoint{2.060340in}{1.974085in}}%
\pgfpathlineto{\pgfqpoint{2.081730in}{1.967434in}}%
\pgfpathlineto{\pgfqpoint{2.103120in}{1.965906in}}%
\pgfpathlineto{\pgfqpoint{2.124510in}{1.965906in}}%
\pgfpathlineto{\pgfqpoint{2.145900in}{1.965906in}}%
\pgfpathlineto{\pgfqpoint{2.167290in}{1.965906in}}%
\pgfpathlineto{\pgfqpoint{2.188680in}{1.965906in}}%
\pgfpathlineto{\pgfqpoint{2.210070in}{1.965906in}}%
\pgfpathlineto{\pgfqpoint{2.231460in}{1.965906in}}%
\pgfpathlineto{\pgfqpoint{2.252850in}{1.965906in}}%
\pgfpathlineto{\pgfqpoint{2.274240in}{1.965906in}}%
\pgfpathlineto{\pgfqpoint{2.295630in}{1.965906in}}%
\pgfpathlineto{\pgfqpoint{2.317020in}{1.902601in}}%
\pgfpathlineto{\pgfqpoint{2.338410in}{1.902601in}}%
\pgfpathlineto{\pgfqpoint{2.359800in}{1.902601in}}%
\pgfpathlineto{\pgfqpoint{2.381190in}{1.902601in}}%
\pgfpathlineto{\pgfqpoint{2.402580in}{1.902601in}}%
\pgfpathlineto{\pgfqpoint{2.423970in}{1.824540in}}%
\pgfpathlineto{\pgfqpoint{2.445360in}{1.824540in}}%
\pgfpathlineto{\pgfqpoint{2.466750in}{1.824540in}}%
\pgfpathlineto{\pgfqpoint{2.488140in}{1.824540in}}%
\pgfpathlineto{\pgfqpoint{2.509530in}{1.824540in}}%
\pgfpathlineto{\pgfqpoint{2.530920in}{1.824540in}}%
\pgfpathlineto{\pgfqpoint{2.552310in}{1.824540in}}%
\pgfpathlineto{\pgfqpoint{2.573700in}{1.824540in}}%
\pgfpathlineto{\pgfqpoint{2.595090in}{1.824540in}}%
\pgfpathlineto{\pgfqpoint{2.616480in}{1.824540in}}%
\pgfpathlineto{\pgfqpoint{2.637870in}{1.824540in}}%
\pgfpathlineto{\pgfqpoint{2.659260in}{1.824540in}}%
\pgfpathlineto{\pgfqpoint{2.680650in}{1.824540in}}%
\pgfpathlineto{\pgfqpoint{2.702040in}{1.824540in}}%
\pgfpathlineto{\pgfqpoint{2.723430in}{1.824540in}}%
\pgfpathlineto{\pgfqpoint{2.744820in}{1.824540in}}%
\pgfpathlineto{\pgfqpoint{2.766210in}{1.824540in}}%
\pgfpathlineto{\pgfqpoint{2.787600in}{1.824540in}}%
\pgfpathlineto{\pgfqpoint{2.808990in}{1.824540in}}%
\pgfpathlineto{\pgfqpoint{2.830380in}{1.824540in}}%
\pgfpathlineto{\pgfqpoint{2.851770in}{1.786933in}}%
\pgfpathlineto{\pgfqpoint{2.873160in}{1.786933in}}%
\pgfpathlineto{\pgfqpoint{2.894550in}{1.786933in}}%
\pgfpathlineto{\pgfqpoint{2.915940in}{1.786933in}}%
\pgfpathlineto{\pgfqpoint{2.937330in}{1.786933in}}%
\pgfpathlineto{\pgfqpoint{2.958720in}{1.786933in}}%
\pgfpathlineto{\pgfqpoint{2.980110in}{1.786933in}}%
\pgfpathlineto{\pgfqpoint{3.001500in}{1.786933in}}%
\pgfpathlineto{\pgfqpoint{3.022890in}{1.786933in}}%
\pgfpathlineto{\pgfqpoint{3.044280in}{1.786933in}}%
\pgfpathlineto{\pgfqpoint{3.065670in}{1.786933in}}%
\pgfpathlineto{\pgfqpoint{3.087060in}{1.786933in}}%
\pgfpathlineto{\pgfqpoint{3.108450in}{1.786933in}}%
\pgfpathlineto{\pgfqpoint{3.129840in}{1.786933in}}%
\pgfpathlineto{\pgfqpoint{3.151230in}{1.786933in}}%
\pgfpathlineto{\pgfqpoint{3.172620in}{1.786933in}}%
\pgfpathlineto{\pgfqpoint{3.194010in}{1.786933in}}%
\pgfpathlineto{\pgfqpoint{3.215400in}{1.786933in}}%
\pgfpathlineto{\pgfqpoint{3.236790in}{1.786933in}}%
\pgfpathlineto{\pgfqpoint{3.258180in}{1.786933in}}%
\pgfpathlineto{\pgfqpoint{3.279570in}{1.786933in}}%
\pgfpathlineto{\pgfqpoint{3.300960in}{1.786933in}}%
\pgfpathlineto{\pgfqpoint{3.322350in}{1.786933in}}%
\pgfpathlineto{\pgfqpoint{3.343740in}{1.786933in}}%
\pgfpathlineto{\pgfqpoint{3.365130in}{1.786933in}}%
\pgfpathlineto{\pgfqpoint{3.386520in}{1.786933in}}%
\pgfpathlineto{\pgfqpoint{3.407910in}{1.786933in}}%
\pgfpathlineto{\pgfqpoint{3.429300in}{1.786933in}}%
\pgfpathlineto{\pgfqpoint{3.450690in}{1.786933in}}%
\pgfpathlineto{\pgfqpoint{3.472080in}{1.786933in}}%
\pgfpathlineto{\pgfqpoint{3.493470in}{1.786933in}}%
\pgfpathlineto{\pgfqpoint{3.514860in}{1.786933in}}%
\pgfpathlineto{\pgfqpoint{3.536250in}{1.786933in}}%
\pgfpathlineto{\pgfqpoint{3.557640in}{1.786933in}}%
\pgfpathlineto{\pgfqpoint{3.579030in}{1.786933in}}%
\pgfpathlineto{\pgfqpoint{3.600420in}{1.786933in}}%
\pgfpathlineto{\pgfqpoint{3.621810in}{1.786933in}}%
\pgfpathlineto{\pgfqpoint{3.643200in}{1.786933in}}%
\pgfpathlineto{\pgfqpoint{3.664590in}{1.786933in}}%
\pgfpathlineto{\pgfqpoint{3.685980in}{1.786933in}}%
\pgfpathlineto{\pgfqpoint{3.707370in}{1.786933in}}%
\pgfpathlineto{\pgfqpoint{3.728760in}{1.786933in}}%
\pgfpathlineto{\pgfqpoint{3.750150in}{1.786933in}}%
\pgfpathlineto{\pgfqpoint{3.771540in}{1.786933in}}%
\pgfpathlineto{\pgfqpoint{3.792930in}{1.786933in}}%
\pgfpathlineto{\pgfqpoint{3.814320in}{1.786933in}}%
\pgfpathlineto{\pgfqpoint{3.835710in}{1.786933in}}%
\pgfpathlineto{\pgfqpoint{3.857100in}{1.786933in}}%
\pgfpathlineto{\pgfqpoint{3.878490in}{1.786933in}}%
\pgfpathlineto{\pgfqpoint{3.899880in}{1.786933in}}%
\pgfpathlineto{\pgfqpoint{3.921270in}{1.786933in}}%
\pgfpathlineto{\pgfqpoint{3.942660in}{1.786933in}}%
\pgfpathlineto{\pgfqpoint{3.964050in}{1.786933in}}%
\pgfpathlineto{\pgfqpoint{3.985440in}{1.786933in}}%
\pgfpathlineto{\pgfqpoint{4.006830in}{1.786933in}}%
\pgfpathlineto{\pgfqpoint{4.028220in}{1.786933in}}%
\pgfpathlineto{\pgfqpoint{4.049610in}{1.786933in}}%
\pgfpathlineto{\pgfqpoint{4.071000in}{1.786933in}}%
\pgfpathlineto{\pgfqpoint{4.092390in}{1.786933in}}%
\pgfpathlineto{\pgfqpoint{4.113780in}{1.786933in}}%
\pgfpathlineto{\pgfqpoint{4.135170in}{1.786933in}}%
\pgfpathlineto{\pgfqpoint{4.156560in}{1.786933in}}%
\pgfpathlineto{\pgfqpoint{4.177950in}{1.786933in}}%
\pgfpathlineto{\pgfqpoint{4.199340in}{1.786933in}}%
\pgfpathlineto{\pgfqpoint{4.220730in}{1.786933in}}%
\pgfpathlineto{\pgfqpoint{4.242120in}{1.786933in}}%
\pgfpathlineto{\pgfqpoint{4.263510in}{1.786933in}}%
\pgfpathlineto{\pgfqpoint{4.284900in}{1.786933in}}%
\pgfpathlineto{\pgfqpoint{4.306290in}{1.786933in}}%
\pgfpathlineto{\pgfqpoint{4.327680in}{1.786933in}}%
\pgfpathlineto{\pgfqpoint{4.349070in}{1.786933in}}%
\pgfpathlineto{\pgfqpoint{4.370460in}{1.786933in}}%
\pgfpathlineto{\pgfqpoint{4.391850in}{1.786933in}}%
\pgfpathlineto{\pgfqpoint{4.413240in}{1.786933in}}%
\pgfpathlineto{\pgfqpoint{4.434630in}{1.786933in}}%
\pgfpathlineto{\pgfqpoint{4.456020in}{1.786933in}}%
\pgfpathlineto{\pgfqpoint{4.477410in}{1.786933in}}%
\pgfpathlineto{\pgfqpoint{4.498800in}{1.786933in}}%
\pgfpathlineto{\pgfqpoint{4.520190in}{1.786933in}}%
\pgfpathlineto{\pgfqpoint{4.541580in}{1.786933in}}%
\pgfpathlineto{\pgfqpoint{4.562970in}{1.786933in}}%
\pgfpathlineto{\pgfqpoint{4.584360in}{1.786933in}}%
\pgfpathlineto{\pgfqpoint{4.605750in}{1.786933in}}%
\pgfpathlineto{\pgfqpoint{4.627140in}{1.786933in}}%
\pgfpathlineto{\pgfqpoint{4.648530in}{1.786933in}}%
\pgfpathlineto{\pgfqpoint{4.669920in}{1.786933in}}%
\pgfpathlineto{\pgfqpoint{4.691310in}{1.786933in}}%
\pgfpathlineto{\pgfqpoint{4.712700in}{1.786933in}}%
\pgfpathlineto{\pgfqpoint{4.734090in}{1.786933in}}%
\pgfpathlineto{\pgfqpoint{4.755480in}{1.786933in}}%
\pgfpathlineto{\pgfqpoint{4.776870in}{1.786933in}}%
\pgfpathlineto{\pgfqpoint{4.798260in}{1.786933in}}%
\pgfpathlineto{\pgfqpoint{4.819650in}{1.786933in}}%
\pgfpathlineto{\pgfqpoint{4.841040in}{1.786933in}}%
\pgfpathlineto{\pgfqpoint{4.862430in}{1.786933in}}%
\pgfpathlineto{\pgfqpoint{4.883820in}{1.786933in}}%
\pgfpathlineto{\pgfqpoint{4.905210in}{1.786933in}}%
\pgfpathlineto{\pgfqpoint{4.926600in}{1.786933in}}%
\pgfpathlineto{\pgfqpoint{4.947990in}{1.786933in}}%
\pgfpathlineto{\pgfqpoint{4.969380in}{1.786933in}}%
\pgfpathlineto{\pgfqpoint{4.990770in}{1.786933in}}%
\pgfpathlineto{\pgfqpoint{5.012160in}{1.786933in}}%
\pgfpathlineto{\pgfqpoint{5.033550in}{1.775056in}}%
\pgfpathlineto{\pgfqpoint{5.054940in}{1.775056in}}%
\pgfpathlineto{\pgfqpoint{5.076330in}{1.775056in}}%
\pgfpathlineto{\pgfqpoint{5.097720in}{1.775056in}}%
\pgfpathlineto{\pgfqpoint{5.119110in}{1.775056in}}%
\pgfpathlineto{\pgfqpoint{5.140500in}{1.775056in}}%
\pgfpathlineto{\pgfqpoint{5.161890in}{1.775056in}}%
\pgfpathlineto{\pgfqpoint{5.183280in}{1.745015in}}%
\pgfpathlineto{\pgfqpoint{5.204670in}{1.745015in}}%
\pgfpathlineto{\pgfqpoint{5.226060in}{1.745015in}}%
\pgfpathlineto{\pgfqpoint{5.247450in}{1.745015in}}%
\pgfpathlineto{\pgfqpoint{5.268840in}{1.745015in}}%
\pgfpathlineto{\pgfqpoint{5.290230in}{1.745015in}}%
\pgfpathlineto{\pgfqpoint{5.311620in}{1.745015in}}%
\pgfpathlineto{\pgfqpoint{5.333010in}{1.745015in}}%
\pgfpathlineto{\pgfqpoint{5.354400in}{1.731309in}}%
\pgfpathlineto{\pgfqpoint{5.375790in}{1.731309in}}%
\pgfpathlineto{\pgfqpoint{5.397180in}{1.731309in}}%
\pgfpathlineto{\pgfqpoint{5.418570in}{1.731309in}}%
\pgfpathlineto{\pgfqpoint{5.439960in}{1.731309in}}%
\pgfpathlineto{\pgfqpoint{5.461350in}{1.722640in}}%
\pgfpathlineto{\pgfqpoint{5.482740in}{1.722640in}}%
\pgfpathlineto{\pgfqpoint{5.504130in}{1.722640in}}%
\pgfpathlineto{\pgfqpoint{5.525520in}{1.722640in}}%
\pgfpathlineto{\pgfqpoint{5.546910in}{1.722640in}}%
\pgfpathlineto{\pgfqpoint{5.568300in}{1.722640in}}%
\pgfpathlineto{\pgfqpoint{5.589690in}{1.722640in}}%
\pgfpathlineto{\pgfqpoint{5.611080in}{1.710292in}}%
\pgfpathlineto{\pgfqpoint{5.632470in}{1.710292in}}%
\pgfpathlineto{\pgfqpoint{5.653860in}{1.710292in}}%
\pgfpathlineto{\pgfqpoint{5.675250in}{1.710292in}}%
\pgfpathlineto{\pgfqpoint{5.696640in}{1.710292in}}%
\pgfpathlineto{\pgfqpoint{5.718030in}{1.710292in}}%
\pgfpathlineto{\pgfqpoint{5.739420in}{1.710292in}}%
\pgfpathlineto{\pgfqpoint{5.760810in}{1.710292in}}%
\pgfpathlineto{\pgfqpoint{5.782200in}{1.710292in}}%
\pgfpathlineto{\pgfqpoint{5.803590in}{1.710292in}}%
\pgfpathlineto{\pgfqpoint{5.824980in}{1.710292in}}%
\pgfpathlineto{\pgfqpoint{5.846370in}{1.710292in}}%
\pgfpathlineto{\pgfqpoint{5.867760in}{1.700445in}}%
\pgfpathlineto{\pgfqpoint{5.889150in}{1.700445in}}%
\pgfpathlineto{\pgfqpoint{5.910540in}{1.700445in}}%
\pgfpathlineto{\pgfqpoint{5.931930in}{1.700445in}}%
\pgfpathlineto{\pgfqpoint{5.953320in}{1.700445in}}%
\pgfpathlineto{\pgfqpoint{5.974710in}{1.700445in}}%
\pgfpathlineto{\pgfqpoint{5.996100in}{1.700445in}}%
\pgfpathlineto{\pgfqpoint{6.017490in}{1.700445in}}%
\pgfpathlineto{\pgfqpoint{6.038880in}{1.700445in}}%
\pgfpathlineto{\pgfqpoint{6.060270in}{1.700445in}}%
\pgfpathlineto{\pgfqpoint{6.081660in}{1.700445in}}%
\pgfpathlineto{\pgfqpoint{6.103050in}{1.700445in}}%
\pgfpathlineto{\pgfqpoint{6.124440in}{1.700445in}}%
\pgfpathlineto{\pgfqpoint{6.145830in}{1.700445in}}%
\pgfpathlineto{\pgfqpoint{6.167220in}{1.700445in}}%
\pgfpathlineto{\pgfqpoint{6.188610in}{1.700445in}}%
\pgfpathlineto{\pgfqpoint{6.210000in}{1.700445in}}%
\pgfpathlineto{\pgfqpoint{6.231390in}{1.700445in}}%
\pgfpathlineto{\pgfqpoint{6.252780in}{1.700445in}}%
\pgfpathlineto{\pgfqpoint{6.274170in}{1.700445in}}%
\pgfpathlineto{\pgfqpoint{6.295560in}{1.700445in}}%
\pgfpathlineto{\pgfqpoint{6.316950in}{1.700445in}}%
\pgfpathlineto{\pgfqpoint{6.338340in}{1.688668in}}%
\pgfpathlineto{\pgfqpoint{6.359730in}{1.688668in}}%
\pgfpathlineto{\pgfqpoint{6.381120in}{1.688668in}}%
\pgfpathlineto{\pgfqpoint{6.402510in}{1.688668in}}%
\pgfpathlineto{\pgfqpoint{6.423900in}{1.688668in}}%
\pgfpathlineto{\pgfqpoint{6.445290in}{1.688668in}}%
\pgfpathlineto{\pgfqpoint{6.466680in}{1.688668in}}%
\pgfpathlineto{\pgfqpoint{6.488070in}{1.688668in}}%
\pgfpathlineto{\pgfqpoint{6.509460in}{1.688668in}}%
\pgfpathlineto{\pgfqpoint{6.530850in}{1.688668in}}%
\pgfpathlineto{\pgfqpoint{6.552240in}{1.688668in}}%
\pgfpathlineto{\pgfqpoint{6.573630in}{1.688668in}}%
\pgfpathlineto{\pgfqpoint{6.595020in}{1.688668in}}%
\pgfpathlineto{\pgfqpoint{6.616410in}{1.688668in}}%
\pgfpathlineto{\pgfqpoint{6.616410in}{1.461345in}}%
\pgfpathlineto{\pgfqpoint{6.616410in}{1.461345in}}%
\pgfpathlineto{\pgfqpoint{6.595020in}{1.461345in}}%
\pgfpathlineto{\pgfqpoint{6.573630in}{1.461345in}}%
\pgfpathlineto{\pgfqpoint{6.552240in}{1.461345in}}%
\pgfpathlineto{\pgfqpoint{6.530850in}{1.461345in}}%
\pgfpathlineto{\pgfqpoint{6.509460in}{1.461345in}}%
\pgfpathlineto{\pgfqpoint{6.488070in}{1.461345in}}%
\pgfpathlineto{\pgfqpoint{6.466680in}{1.461345in}}%
\pgfpathlineto{\pgfqpoint{6.445290in}{1.461345in}}%
\pgfpathlineto{\pgfqpoint{6.423900in}{1.461345in}}%
\pgfpathlineto{\pgfqpoint{6.402510in}{1.461345in}}%
\pgfpathlineto{\pgfqpoint{6.381120in}{1.461345in}}%
\pgfpathlineto{\pgfqpoint{6.359730in}{1.461345in}}%
\pgfpathlineto{\pgfqpoint{6.338340in}{1.461345in}}%
\pgfpathlineto{\pgfqpoint{6.316950in}{1.476881in}}%
\pgfpathlineto{\pgfqpoint{6.295560in}{1.476881in}}%
\pgfpathlineto{\pgfqpoint{6.274170in}{1.476881in}}%
\pgfpathlineto{\pgfqpoint{6.252780in}{1.476881in}}%
\pgfpathlineto{\pgfqpoint{6.231390in}{1.476881in}}%
\pgfpathlineto{\pgfqpoint{6.210000in}{1.476881in}}%
\pgfpathlineto{\pgfqpoint{6.188610in}{1.476881in}}%
\pgfpathlineto{\pgfqpoint{6.167220in}{1.476881in}}%
\pgfpathlineto{\pgfqpoint{6.145830in}{1.476881in}}%
\pgfpathlineto{\pgfqpoint{6.124440in}{1.476881in}}%
\pgfpathlineto{\pgfqpoint{6.103050in}{1.476881in}}%
\pgfpathlineto{\pgfqpoint{6.081660in}{1.476881in}}%
\pgfpathlineto{\pgfqpoint{6.060270in}{1.476881in}}%
\pgfpathlineto{\pgfqpoint{6.038880in}{1.476881in}}%
\pgfpathlineto{\pgfqpoint{6.017490in}{1.476881in}}%
\pgfpathlineto{\pgfqpoint{5.996100in}{1.476881in}}%
\pgfpathlineto{\pgfqpoint{5.974710in}{1.476881in}}%
\pgfpathlineto{\pgfqpoint{5.953320in}{1.476881in}}%
\pgfpathlineto{\pgfqpoint{5.931930in}{1.476881in}}%
\pgfpathlineto{\pgfqpoint{5.910540in}{1.476881in}}%
\pgfpathlineto{\pgfqpoint{5.889150in}{1.476881in}}%
\pgfpathlineto{\pgfqpoint{5.867760in}{1.476881in}}%
\pgfpathlineto{\pgfqpoint{5.846370in}{1.488428in}}%
\pgfpathlineto{\pgfqpoint{5.824980in}{1.488428in}}%
\pgfpathlineto{\pgfqpoint{5.803590in}{1.488428in}}%
\pgfpathlineto{\pgfqpoint{5.782200in}{1.488428in}}%
\pgfpathlineto{\pgfqpoint{5.760810in}{1.488428in}}%
\pgfpathlineto{\pgfqpoint{5.739420in}{1.488428in}}%
\pgfpathlineto{\pgfqpoint{5.718030in}{1.488428in}}%
\pgfpathlineto{\pgfqpoint{5.696640in}{1.488428in}}%
\pgfpathlineto{\pgfqpoint{5.675250in}{1.488428in}}%
\pgfpathlineto{\pgfqpoint{5.653860in}{1.488428in}}%
\pgfpathlineto{\pgfqpoint{5.632470in}{1.488428in}}%
\pgfpathlineto{\pgfqpoint{5.611080in}{1.488428in}}%
\pgfpathlineto{\pgfqpoint{5.589690in}{1.489338in}}%
\pgfpathlineto{\pgfqpoint{5.568300in}{1.489338in}}%
\pgfpathlineto{\pgfqpoint{5.546910in}{1.489338in}}%
\pgfpathlineto{\pgfqpoint{5.525520in}{1.489338in}}%
\pgfpathlineto{\pgfqpoint{5.504130in}{1.489338in}}%
\pgfpathlineto{\pgfqpoint{5.482740in}{1.489338in}}%
\pgfpathlineto{\pgfqpoint{5.461350in}{1.489338in}}%
\pgfpathlineto{\pgfqpoint{5.439960in}{1.525226in}}%
\pgfpathlineto{\pgfqpoint{5.418570in}{1.525226in}}%
\pgfpathlineto{\pgfqpoint{5.397180in}{1.525226in}}%
\pgfpathlineto{\pgfqpoint{5.375790in}{1.525226in}}%
\pgfpathlineto{\pgfqpoint{5.354400in}{1.525226in}}%
\pgfpathlineto{\pgfqpoint{5.333010in}{1.561437in}}%
\pgfpathlineto{\pgfqpoint{5.311620in}{1.561437in}}%
\pgfpathlineto{\pgfqpoint{5.290230in}{1.561437in}}%
\pgfpathlineto{\pgfqpoint{5.268840in}{1.561437in}}%
\pgfpathlineto{\pgfqpoint{5.247450in}{1.561437in}}%
\pgfpathlineto{\pgfqpoint{5.226060in}{1.561437in}}%
\pgfpathlineto{\pgfqpoint{5.204670in}{1.561437in}}%
\pgfpathlineto{\pgfqpoint{5.183280in}{1.561437in}}%
\pgfpathlineto{\pgfqpoint{5.161890in}{1.583744in}}%
\pgfpathlineto{\pgfqpoint{5.140500in}{1.583744in}}%
\pgfpathlineto{\pgfqpoint{5.119110in}{1.583744in}}%
\pgfpathlineto{\pgfqpoint{5.097720in}{1.583744in}}%
\pgfpathlineto{\pgfqpoint{5.076330in}{1.583744in}}%
\pgfpathlineto{\pgfqpoint{5.054940in}{1.583744in}}%
\pgfpathlineto{\pgfqpoint{5.033550in}{1.583744in}}%
\pgfpathlineto{\pgfqpoint{5.012160in}{1.605955in}}%
\pgfpathlineto{\pgfqpoint{4.990770in}{1.605955in}}%
\pgfpathlineto{\pgfqpoint{4.969380in}{1.605955in}}%
\pgfpathlineto{\pgfqpoint{4.947990in}{1.605955in}}%
\pgfpathlineto{\pgfqpoint{4.926600in}{1.605955in}}%
\pgfpathlineto{\pgfqpoint{4.905210in}{1.605955in}}%
\pgfpathlineto{\pgfqpoint{4.883820in}{1.605955in}}%
\pgfpathlineto{\pgfqpoint{4.862430in}{1.605955in}}%
\pgfpathlineto{\pgfqpoint{4.841040in}{1.605955in}}%
\pgfpathlineto{\pgfqpoint{4.819650in}{1.605955in}}%
\pgfpathlineto{\pgfqpoint{4.798260in}{1.605955in}}%
\pgfpathlineto{\pgfqpoint{4.776870in}{1.605955in}}%
\pgfpathlineto{\pgfqpoint{4.755480in}{1.605955in}}%
\pgfpathlineto{\pgfqpoint{4.734090in}{1.605955in}}%
\pgfpathlineto{\pgfqpoint{4.712700in}{1.605955in}}%
\pgfpathlineto{\pgfqpoint{4.691310in}{1.605955in}}%
\pgfpathlineto{\pgfqpoint{4.669920in}{1.605955in}}%
\pgfpathlineto{\pgfqpoint{4.648530in}{1.605955in}}%
\pgfpathlineto{\pgfqpoint{4.627140in}{1.605955in}}%
\pgfpathlineto{\pgfqpoint{4.605750in}{1.605955in}}%
\pgfpathlineto{\pgfqpoint{4.584360in}{1.605955in}}%
\pgfpathlineto{\pgfqpoint{4.562970in}{1.605955in}}%
\pgfpathlineto{\pgfqpoint{4.541580in}{1.605955in}}%
\pgfpathlineto{\pgfqpoint{4.520190in}{1.605955in}}%
\pgfpathlineto{\pgfqpoint{4.498800in}{1.605955in}}%
\pgfpathlineto{\pgfqpoint{4.477410in}{1.605955in}}%
\pgfpathlineto{\pgfqpoint{4.456020in}{1.605955in}}%
\pgfpathlineto{\pgfqpoint{4.434630in}{1.605955in}}%
\pgfpathlineto{\pgfqpoint{4.413240in}{1.605955in}}%
\pgfpathlineto{\pgfqpoint{4.391850in}{1.605955in}}%
\pgfpathlineto{\pgfqpoint{4.370460in}{1.605955in}}%
\pgfpathlineto{\pgfqpoint{4.349070in}{1.605955in}}%
\pgfpathlineto{\pgfqpoint{4.327680in}{1.605955in}}%
\pgfpathlineto{\pgfqpoint{4.306290in}{1.605955in}}%
\pgfpathlineto{\pgfqpoint{4.284900in}{1.605955in}}%
\pgfpathlineto{\pgfqpoint{4.263510in}{1.605955in}}%
\pgfpathlineto{\pgfqpoint{4.242120in}{1.605955in}}%
\pgfpathlineto{\pgfqpoint{4.220730in}{1.605955in}}%
\pgfpathlineto{\pgfqpoint{4.199340in}{1.605955in}}%
\pgfpathlineto{\pgfqpoint{4.177950in}{1.605955in}}%
\pgfpathlineto{\pgfqpoint{4.156560in}{1.605955in}}%
\pgfpathlineto{\pgfqpoint{4.135170in}{1.605955in}}%
\pgfpathlineto{\pgfqpoint{4.113780in}{1.605955in}}%
\pgfpathlineto{\pgfqpoint{4.092390in}{1.605955in}}%
\pgfpathlineto{\pgfqpoint{4.071000in}{1.605955in}}%
\pgfpathlineto{\pgfqpoint{4.049610in}{1.605955in}}%
\pgfpathlineto{\pgfqpoint{4.028220in}{1.605955in}}%
\pgfpathlineto{\pgfqpoint{4.006830in}{1.605955in}}%
\pgfpathlineto{\pgfqpoint{3.985440in}{1.605955in}}%
\pgfpathlineto{\pgfqpoint{3.964050in}{1.605955in}}%
\pgfpathlineto{\pgfqpoint{3.942660in}{1.605955in}}%
\pgfpathlineto{\pgfqpoint{3.921270in}{1.605955in}}%
\pgfpathlineto{\pgfqpoint{3.899880in}{1.605955in}}%
\pgfpathlineto{\pgfqpoint{3.878490in}{1.605955in}}%
\pgfpathlineto{\pgfqpoint{3.857100in}{1.605955in}}%
\pgfpathlineto{\pgfqpoint{3.835710in}{1.605955in}}%
\pgfpathlineto{\pgfqpoint{3.814320in}{1.605955in}}%
\pgfpathlineto{\pgfqpoint{3.792930in}{1.605955in}}%
\pgfpathlineto{\pgfqpoint{3.771540in}{1.605955in}}%
\pgfpathlineto{\pgfqpoint{3.750150in}{1.605955in}}%
\pgfpathlineto{\pgfqpoint{3.728760in}{1.605955in}}%
\pgfpathlineto{\pgfqpoint{3.707370in}{1.605955in}}%
\pgfpathlineto{\pgfqpoint{3.685980in}{1.605955in}}%
\pgfpathlineto{\pgfqpoint{3.664590in}{1.605955in}}%
\pgfpathlineto{\pgfqpoint{3.643200in}{1.605955in}}%
\pgfpathlineto{\pgfqpoint{3.621810in}{1.605955in}}%
\pgfpathlineto{\pgfqpoint{3.600420in}{1.605955in}}%
\pgfpathlineto{\pgfqpoint{3.579030in}{1.605955in}}%
\pgfpathlineto{\pgfqpoint{3.557640in}{1.605955in}}%
\pgfpathlineto{\pgfqpoint{3.536250in}{1.605955in}}%
\pgfpathlineto{\pgfqpoint{3.514860in}{1.605955in}}%
\pgfpathlineto{\pgfqpoint{3.493470in}{1.605955in}}%
\pgfpathlineto{\pgfqpoint{3.472080in}{1.605955in}}%
\pgfpathlineto{\pgfqpoint{3.450690in}{1.605955in}}%
\pgfpathlineto{\pgfqpoint{3.429300in}{1.605955in}}%
\pgfpathlineto{\pgfqpoint{3.407910in}{1.605955in}}%
\pgfpathlineto{\pgfqpoint{3.386520in}{1.605955in}}%
\pgfpathlineto{\pgfqpoint{3.365130in}{1.605955in}}%
\pgfpathlineto{\pgfqpoint{3.343740in}{1.605955in}}%
\pgfpathlineto{\pgfqpoint{3.322350in}{1.605955in}}%
\pgfpathlineto{\pgfqpoint{3.300960in}{1.605955in}}%
\pgfpathlineto{\pgfqpoint{3.279570in}{1.605955in}}%
\pgfpathlineto{\pgfqpoint{3.258180in}{1.605955in}}%
\pgfpathlineto{\pgfqpoint{3.236790in}{1.605955in}}%
\pgfpathlineto{\pgfqpoint{3.215400in}{1.605955in}}%
\pgfpathlineto{\pgfqpoint{3.194010in}{1.605955in}}%
\pgfpathlineto{\pgfqpoint{3.172620in}{1.605955in}}%
\pgfpathlineto{\pgfqpoint{3.151230in}{1.605955in}}%
\pgfpathlineto{\pgfqpoint{3.129840in}{1.605955in}}%
\pgfpathlineto{\pgfqpoint{3.108450in}{1.605955in}}%
\pgfpathlineto{\pgfqpoint{3.087060in}{1.605955in}}%
\pgfpathlineto{\pgfqpoint{3.065670in}{1.605955in}}%
\pgfpathlineto{\pgfqpoint{3.044280in}{1.605955in}}%
\pgfpathlineto{\pgfqpoint{3.022890in}{1.605955in}}%
\pgfpathlineto{\pgfqpoint{3.001500in}{1.605955in}}%
\pgfpathlineto{\pgfqpoint{2.980110in}{1.605955in}}%
\pgfpathlineto{\pgfqpoint{2.958720in}{1.605955in}}%
\pgfpathlineto{\pgfqpoint{2.937330in}{1.605955in}}%
\pgfpathlineto{\pgfqpoint{2.915940in}{1.605955in}}%
\pgfpathlineto{\pgfqpoint{2.894550in}{1.605955in}}%
\pgfpathlineto{\pgfqpoint{2.873160in}{1.605955in}}%
\pgfpathlineto{\pgfqpoint{2.851770in}{1.605955in}}%
\pgfpathlineto{\pgfqpoint{2.830380in}{1.665600in}}%
\pgfpathlineto{\pgfqpoint{2.808990in}{1.665600in}}%
\pgfpathlineto{\pgfqpoint{2.787600in}{1.665600in}}%
\pgfpathlineto{\pgfqpoint{2.766210in}{1.665600in}}%
\pgfpathlineto{\pgfqpoint{2.744820in}{1.665600in}}%
\pgfpathlineto{\pgfqpoint{2.723430in}{1.665600in}}%
\pgfpathlineto{\pgfqpoint{2.702040in}{1.665600in}}%
\pgfpathlineto{\pgfqpoint{2.680650in}{1.665600in}}%
\pgfpathlineto{\pgfqpoint{2.659260in}{1.665600in}}%
\pgfpathlineto{\pgfqpoint{2.637870in}{1.665600in}}%
\pgfpathlineto{\pgfqpoint{2.616480in}{1.665600in}}%
\pgfpathlineto{\pgfqpoint{2.595090in}{1.665600in}}%
\pgfpathlineto{\pgfqpoint{2.573700in}{1.665600in}}%
\pgfpathlineto{\pgfqpoint{2.552310in}{1.665600in}}%
\pgfpathlineto{\pgfqpoint{2.530920in}{1.665600in}}%
\pgfpathlineto{\pgfqpoint{2.509530in}{1.665600in}}%
\pgfpathlineto{\pgfqpoint{2.488140in}{1.665600in}}%
\pgfpathlineto{\pgfqpoint{2.466750in}{1.665600in}}%
\pgfpathlineto{\pgfqpoint{2.445360in}{1.665600in}}%
\pgfpathlineto{\pgfqpoint{2.423970in}{1.665600in}}%
\pgfpathlineto{\pgfqpoint{2.402580in}{1.721180in}}%
\pgfpathlineto{\pgfqpoint{2.381190in}{1.721180in}}%
\pgfpathlineto{\pgfqpoint{2.359800in}{1.721180in}}%
\pgfpathlineto{\pgfqpoint{2.338410in}{1.721180in}}%
\pgfpathlineto{\pgfqpoint{2.317020in}{1.721180in}}%
\pgfpathlineto{\pgfqpoint{2.295630in}{1.737430in}}%
\pgfpathlineto{\pgfqpoint{2.274240in}{1.737430in}}%
\pgfpathlineto{\pgfqpoint{2.252850in}{1.737430in}}%
\pgfpathlineto{\pgfqpoint{2.231460in}{1.737430in}}%
\pgfpathlineto{\pgfqpoint{2.210070in}{1.737430in}}%
\pgfpathlineto{\pgfqpoint{2.188680in}{1.737430in}}%
\pgfpathlineto{\pgfqpoint{2.167290in}{1.737430in}}%
\pgfpathlineto{\pgfqpoint{2.145900in}{1.737430in}}%
\pgfpathlineto{\pgfqpoint{2.124510in}{1.737430in}}%
\pgfpathlineto{\pgfqpoint{2.103120in}{1.737430in}}%
\pgfpathlineto{\pgfqpoint{2.081730in}{1.742384in}}%
\pgfpathlineto{\pgfqpoint{2.060340in}{1.782436in}}%
\pgfpathlineto{\pgfqpoint{2.038950in}{1.782436in}}%
\pgfpathlineto{\pgfqpoint{2.017560in}{1.782436in}}%
\pgfpathlineto{\pgfqpoint{1.996170in}{1.789891in}}%
\pgfpathlineto{\pgfqpoint{1.974780in}{1.789891in}}%
\pgfpathlineto{\pgfqpoint{1.953390in}{1.789891in}}%
\pgfpathlineto{\pgfqpoint{1.932000in}{1.789891in}}%
\pgfpathlineto{\pgfqpoint{1.910610in}{1.830799in}}%
\pgfpathlineto{\pgfqpoint{1.889220in}{1.830799in}}%
\pgfpathlineto{\pgfqpoint{1.867830in}{1.830799in}}%
\pgfpathlineto{\pgfqpoint{1.846440in}{1.831533in}}%
\pgfpathlineto{\pgfqpoint{1.825050in}{1.831533in}}%
\pgfpathlineto{\pgfqpoint{1.803660in}{1.831533in}}%
\pgfpathlineto{\pgfqpoint{1.782270in}{1.831533in}}%
\pgfpathlineto{\pgfqpoint{1.760880in}{1.831533in}}%
\pgfpathlineto{\pgfqpoint{1.739490in}{1.831533in}}%
\pgfpathlineto{\pgfqpoint{1.718100in}{1.831533in}}%
\pgfpathlineto{\pgfqpoint{1.696710in}{1.837848in}}%
\pgfpathlineto{\pgfqpoint{1.675320in}{1.886190in}}%
\pgfpathlineto{\pgfqpoint{1.653930in}{1.886190in}}%
\pgfpathlineto{\pgfqpoint{1.632540in}{1.886190in}}%
\pgfpathlineto{\pgfqpoint{1.611150in}{1.888411in}}%
\pgfpathlineto{\pgfqpoint{1.589760in}{1.888411in}}%
\pgfpathlineto{\pgfqpoint{1.568370in}{1.888411in}}%
\pgfpathlineto{\pgfqpoint{1.546980in}{1.888411in}}%
\pgfpathlineto{\pgfqpoint{1.525590in}{1.888411in}}%
\pgfpathlineto{\pgfqpoint{1.504200in}{1.888411in}}%
\pgfpathlineto{\pgfqpoint{1.482810in}{1.902913in}}%
\pgfpathlineto{\pgfqpoint{1.461420in}{1.902913in}}%
\pgfpathlineto{\pgfqpoint{1.440030in}{1.902913in}}%
\pgfpathlineto{\pgfqpoint{1.418640in}{1.902913in}}%
\pgfpathlineto{\pgfqpoint{1.397250in}{1.907365in}}%
\pgfpathlineto{\pgfqpoint{1.375860in}{1.916847in}}%
\pgfpathlineto{\pgfqpoint{1.354470in}{1.952667in}}%
\pgfpathlineto{\pgfqpoint{1.333080in}{1.952667in}}%
\pgfpathlineto{\pgfqpoint{1.311690in}{1.986750in}}%
\pgfpathlineto{\pgfqpoint{1.290300in}{1.986750in}}%
\pgfpathlineto{\pgfqpoint{1.268910in}{1.986750in}}%
\pgfpathlineto{\pgfqpoint{1.247520in}{1.986750in}}%
\pgfpathlineto{\pgfqpoint{1.226130in}{1.986750in}}%
\pgfpathlineto{\pgfqpoint{1.204740in}{2.072434in}}%
\pgfpathlineto{\pgfqpoint{1.183350in}{2.072434in}}%
\pgfpathlineto{\pgfqpoint{1.161960in}{2.099679in}}%
\pgfpathlineto{\pgfqpoint{1.140570in}{2.099679in}}%
\pgfpathlineto{\pgfqpoint{1.119180in}{2.178235in}}%
\pgfpathlineto{\pgfqpoint{1.097790in}{2.178235in}}%
\pgfpathlineto{\pgfqpoint{1.076400in}{2.178235in}}%
\pgfpathlineto{\pgfqpoint{1.055010in}{2.198269in}}%
\pgfpathlineto{\pgfqpoint{1.033620in}{2.198269in}}%
\pgfpathlineto{\pgfqpoint{1.012230in}{2.198269in}}%
\pgfpathlineto{\pgfqpoint{0.990840in}{2.199700in}}%
\pgfpathlineto{\pgfqpoint{0.969450in}{2.217016in}}%
\pgfpathlineto{\pgfqpoint{0.948060in}{2.285420in}}%
\pgfpathlineto{\pgfqpoint{0.926670in}{2.304427in}}%
\pgfpathlineto{\pgfqpoint{0.905280in}{2.345339in}}%
\pgfpathlineto{\pgfqpoint{0.883890in}{2.409145in}}%
\pgfpathlineto{\pgfqpoint{0.862500in}{2.414113in}}%
\pgfpathclose%
\pgfusepath{fill}%
\end{pgfscope}%
\begin{pgfscope}%
\pgfpathrectangle{\pgfqpoint{0.862500in}{0.375000in}}{\pgfqpoint{5.347500in}{2.265000in}}%
\pgfusepath{clip}%
\pgfsetbuttcap%
\pgfsetroundjoin%
\definecolor{currentfill}{rgb}{1.000000,0.498039,0.054902}%
\pgfsetfillcolor{currentfill}%
\pgfsetfillopacity{0.200000}%
\pgfsetlinewidth{0.000000pt}%
\definecolor{currentstroke}{rgb}{0.000000,0.000000,0.000000}%
\pgfsetstrokecolor{currentstroke}%
\pgfsetdash{}{0pt}%
\pgfpathmoveto{\pgfqpoint{0.862500in}{2.476284in}}%
\pgfpathlineto{\pgfqpoint{0.862500in}{2.510605in}}%
\pgfpathlineto{\pgfqpoint{0.883890in}{2.483516in}}%
\pgfpathlineto{\pgfqpoint{0.905280in}{2.438117in}}%
\pgfpathlineto{\pgfqpoint{0.926670in}{2.438117in}}%
\pgfpathlineto{\pgfqpoint{0.948060in}{2.438117in}}%
\pgfpathlineto{\pgfqpoint{0.969450in}{2.421333in}}%
\pgfpathlineto{\pgfqpoint{0.990840in}{2.379843in}}%
\pgfpathlineto{\pgfqpoint{1.012230in}{2.379843in}}%
\pgfpathlineto{\pgfqpoint{1.033620in}{2.317218in}}%
\pgfpathlineto{\pgfqpoint{1.055010in}{2.294596in}}%
\pgfpathlineto{\pgfqpoint{1.076400in}{2.294596in}}%
\pgfpathlineto{\pgfqpoint{1.097790in}{2.294596in}}%
\pgfpathlineto{\pgfqpoint{1.119180in}{2.294596in}}%
\pgfpathlineto{\pgfqpoint{1.140570in}{2.272485in}}%
\pgfpathlineto{\pgfqpoint{1.161960in}{2.272485in}}%
\pgfpathlineto{\pgfqpoint{1.183350in}{2.272485in}}%
\pgfpathlineto{\pgfqpoint{1.204740in}{2.270703in}}%
\pgfpathlineto{\pgfqpoint{1.226130in}{2.270703in}}%
\pgfpathlineto{\pgfqpoint{1.247520in}{2.270703in}}%
\pgfpathlineto{\pgfqpoint{1.268910in}{2.248678in}}%
\pgfpathlineto{\pgfqpoint{1.290300in}{2.248678in}}%
\pgfpathlineto{\pgfqpoint{1.311690in}{2.245794in}}%
\pgfpathlineto{\pgfqpoint{1.333080in}{2.209159in}}%
\pgfpathlineto{\pgfqpoint{1.354470in}{2.205912in}}%
\pgfpathlineto{\pgfqpoint{1.375860in}{2.189705in}}%
\pgfpathlineto{\pgfqpoint{1.397250in}{2.150345in}}%
\pgfpathlineto{\pgfqpoint{1.418640in}{2.147517in}}%
\pgfpathlineto{\pgfqpoint{1.440030in}{2.136159in}}%
\pgfpathlineto{\pgfqpoint{1.461420in}{2.115775in}}%
\pgfpathlineto{\pgfqpoint{1.482810in}{2.084392in}}%
\pgfpathlineto{\pgfqpoint{1.504200in}{2.070236in}}%
\pgfpathlineto{\pgfqpoint{1.525590in}{1.949207in}}%
\pgfpathlineto{\pgfqpoint{1.546980in}{1.914445in}}%
\pgfpathlineto{\pgfqpoint{1.568370in}{1.840215in}}%
\pgfpathlineto{\pgfqpoint{1.589760in}{1.814797in}}%
\pgfpathlineto{\pgfqpoint{1.611150in}{1.800350in}}%
\pgfpathlineto{\pgfqpoint{1.632540in}{1.798385in}}%
\pgfpathlineto{\pgfqpoint{1.653930in}{1.596572in}}%
\pgfpathlineto{\pgfqpoint{1.675320in}{1.596572in}}%
\pgfpathlineto{\pgfqpoint{1.696710in}{1.596572in}}%
\pgfpathlineto{\pgfqpoint{1.718100in}{1.596572in}}%
\pgfpathlineto{\pgfqpoint{1.739490in}{1.594831in}}%
\pgfpathlineto{\pgfqpoint{1.760880in}{1.594831in}}%
\pgfpathlineto{\pgfqpoint{1.782270in}{1.594831in}}%
\pgfpathlineto{\pgfqpoint{1.803660in}{1.594831in}}%
\pgfpathlineto{\pgfqpoint{1.825050in}{1.594831in}}%
\pgfpathlineto{\pgfqpoint{1.846440in}{1.594831in}}%
\pgfpathlineto{\pgfqpoint{1.867830in}{1.594831in}}%
\pgfpathlineto{\pgfqpoint{1.889220in}{1.594831in}}%
\pgfpathlineto{\pgfqpoint{1.910610in}{1.594831in}}%
\pgfpathlineto{\pgfqpoint{1.932000in}{1.591729in}}%
\pgfpathlineto{\pgfqpoint{1.953390in}{1.591729in}}%
\pgfpathlineto{\pgfqpoint{1.974780in}{1.586982in}}%
\pgfpathlineto{\pgfqpoint{1.996170in}{1.574019in}}%
\pgfpathlineto{\pgfqpoint{2.017560in}{1.574019in}}%
\pgfpathlineto{\pgfqpoint{2.038950in}{1.574019in}}%
\pgfpathlineto{\pgfqpoint{2.060340in}{1.574019in}}%
\pgfpathlineto{\pgfqpoint{2.081730in}{1.574019in}}%
\pgfpathlineto{\pgfqpoint{2.103120in}{1.574019in}}%
\pgfpathlineto{\pgfqpoint{2.124510in}{1.571809in}}%
\pgfpathlineto{\pgfqpoint{2.145900in}{1.571809in}}%
\pgfpathlineto{\pgfqpoint{2.167290in}{1.571809in}}%
\pgfpathlineto{\pgfqpoint{2.188680in}{1.571809in}}%
\pgfpathlineto{\pgfqpoint{2.210070in}{1.571809in}}%
\pgfpathlineto{\pgfqpoint{2.231460in}{1.571809in}}%
\pgfpathlineto{\pgfqpoint{2.252850in}{1.571809in}}%
\pgfpathlineto{\pgfqpoint{2.274240in}{1.571809in}}%
\pgfpathlineto{\pgfqpoint{2.295630in}{1.571809in}}%
\pgfpathlineto{\pgfqpoint{2.317020in}{1.568066in}}%
\pgfpathlineto{\pgfqpoint{2.338410in}{1.568066in}}%
\pgfpathlineto{\pgfqpoint{2.359800in}{1.568066in}}%
\pgfpathlineto{\pgfqpoint{2.381190in}{1.568066in}}%
\pgfpathlineto{\pgfqpoint{2.402580in}{1.568066in}}%
\pgfpathlineto{\pgfqpoint{2.423970in}{1.568066in}}%
\pgfpathlineto{\pgfqpoint{2.445360in}{1.567782in}}%
\pgfpathlineto{\pgfqpoint{2.466750in}{1.567647in}}%
\pgfpathlineto{\pgfqpoint{2.488140in}{1.567647in}}%
\pgfpathlineto{\pgfqpoint{2.509530in}{1.567647in}}%
\pgfpathlineto{\pgfqpoint{2.530920in}{1.567647in}}%
\pgfpathlineto{\pgfqpoint{2.552310in}{1.567647in}}%
\pgfpathlineto{\pgfqpoint{2.573700in}{1.561087in}}%
\pgfpathlineto{\pgfqpoint{2.595090in}{1.561087in}}%
\pgfpathlineto{\pgfqpoint{2.616480in}{1.561087in}}%
\pgfpathlineto{\pgfqpoint{2.637870in}{1.561087in}}%
\pgfpathlineto{\pgfqpoint{2.659260in}{1.561087in}}%
\pgfpathlineto{\pgfqpoint{2.680650in}{1.561087in}}%
\pgfpathlineto{\pgfqpoint{2.702040in}{1.561087in}}%
\pgfpathlineto{\pgfqpoint{2.723430in}{1.561087in}}%
\pgfpathlineto{\pgfqpoint{2.744820in}{1.561087in}}%
\pgfpathlineto{\pgfqpoint{2.766210in}{1.561087in}}%
\pgfpathlineto{\pgfqpoint{2.787600in}{1.561087in}}%
\pgfpathlineto{\pgfqpoint{2.808990in}{1.560648in}}%
\pgfpathlineto{\pgfqpoint{2.830380in}{1.560648in}}%
\pgfpathlineto{\pgfqpoint{2.851770in}{1.560648in}}%
\pgfpathlineto{\pgfqpoint{2.873160in}{1.560648in}}%
\pgfpathlineto{\pgfqpoint{2.894550in}{1.560648in}}%
\pgfpathlineto{\pgfqpoint{2.915940in}{1.560648in}}%
\pgfpathlineto{\pgfqpoint{2.937330in}{1.560453in}}%
\pgfpathlineto{\pgfqpoint{2.958720in}{1.560453in}}%
\pgfpathlineto{\pgfqpoint{2.980110in}{1.560453in}}%
\pgfpathlineto{\pgfqpoint{3.001500in}{1.513652in}}%
\pgfpathlineto{\pgfqpoint{3.022890in}{1.513652in}}%
\pgfpathlineto{\pgfqpoint{3.044280in}{1.513652in}}%
\pgfpathlineto{\pgfqpoint{3.065670in}{1.486614in}}%
\pgfpathlineto{\pgfqpoint{3.087060in}{1.486614in}}%
\pgfpathlineto{\pgfqpoint{3.108450in}{1.486614in}}%
\pgfpathlineto{\pgfqpoint{3.129840in}{1.486614in}}%
\pgfpathlineto{\pgfqpoint{3.151230in}{1.486614in}}%
\pgfpathlineto{\pgfqpoint{3.172620in}{1.486614in}}%
\pgfpathlineto{\pgfqpoint{3.194010in}{1.471079in}}%
\pgfpathlineto{\pgfqpoint{3.215400in}{1.471079in}}%
\pgfpathlineto{\pgfqpoint{3.236790in}{1.471079in}}%
\pgfpathlineto{\pgfqpoint{3.258180in}{1.471079in}}%
\pgfpathlineto{\pgfqpoint{3.279570in}{1.469685in}}%
\pgfpathlineto{\pgfqpoint{3.300960in}{1.469685in}}%
\pgfpathlineto{\pgfqpoint{3.322350in}{1.468157in}}%
\pgfpathlineto{\pgfqpoint{3.343740in}{1.458622in}}%
\pgfpathlineto{\pgfqpoint{3.365130in}{1.458622in}}%
\pgfpathlineto{\pgfqpoint{3.386520in}{1.458622in}}%
\pgfpathlineto{\pgfqpoint{3.407910in}{1.458622in}}%
\pgfpathlineto{\pgfqpoint{3.429300in}{1.458622in}}%
\pgfpathlineto{\pgfqpoint{3.450690in}{1.458622in}}%
\pgfpathlineto{\pgfqpoint{3.472080in}{1.458622in}}%
\pgfpathlineto{\pgfqpoint{3.493470in}{1.458622in}}%
\pgfpathlineto{\pgfqpoint{3.514860in}{1.458622in}}%
\pgfpathlineto{\pgfqpoint{3.536250in}{1.458622in}}%
\pgfpathlineto{\pgfqpoint{3.557640in}{1.458622in}}%
\pgfpathlineto{\pgfqpoint{3.579030in}{1.458622in}}%
\pgfpathlineto{\pgfqpoint{3.600420in}{1.458622in}}%
\pgfpathlineto{\pgfqpoint{3.621810in}{1.458622in}}%
\pgfpathlineto{\pgfqpoint{3.643200in}{1.458622in}}%
\pgfpathlineto{\pgfqpoint{3.664590in}{1.458622in}}%
\pgfpathlineto{\pgfqpoint{3.685980in}{1.458622in}}%
\pgfpathlineto{\pgfqpoint{3.707370in}{1.457152in}}%
\pgfpathlineto{\pgfqpoint{3.728760in}{1.457152in}}%
\pgfpathlineto{\pgfqpoint{3.750150in}{1.457152in}}%
\pgfpathlineto{\pgfqpoint{3.771540in}{1.432362in}}%
\pgfpathlineto{\pgfqpoint{3.792930in}{1.432362in}}%
\pgfpathlineto{\pgfqpoint{3.814320in}{1.432362in}}%
\pgfpathlineto{\pgfqpoint{3.835710in}{1.432359in}}%
\pgfpathlineto{\pgfqpoint{3.857100in}{1.432359in}}%
\pgfpathlineto{\pgfqpoint{3.878490in}{1.432359in}}%
\pgfpathlineto{\pgfqpoint{3.899880in}{1.382650in}}%
\pgfpathlineto{\pgfqpoint{3.921270in}{1.382650in}}%
\pgfpathlineto{\pgfqpoint{3.942660in}{1.382650in}}%
\pgfpathlineto{\pgfqpoint{3.964050in}{1.382650in}}%
\pgfpathlineto{\pgfqpoint{3.985440in}{1.382650in}}%
\pgfpathlineto{\pgfqpoint{4.006830in}{1.382650in}}%
\pgfpathlineto{\pgfqpoint{4.028220in}{1.382650in}}%
\pgfpathlineto{\pgfqpoint{4.049610in}{1.382650in}}%
\pgfpathlineto{\pgfqpoint{4.071000in}{1.382650in}}%
\pgfpathlineto{\pgfqpoint{4.092390in}{1.382650in}}%
\pgfpathlineto{\pgfqpoint{4.113780in}{1.382650in}}%
\pgfpathlineto{\pgfqpoint{4.135170in}{1.381791in}}%
\pgfpathlineto{\pgfqpoint{4.156560in}{1.381791in}}%
\pgfpathlineto{\pgfqpoint{4.177950in}{1.381791in}}%
\pgfpathlineto{\pgfqpoint{4.199340in}{1.381791in}}%
\pgfpathlineto{\pgfqpoint{4.220730in}{1.381791in}}%
\pgfpathlineto{\pgfqpoint{4.242120in}{1.381791in}}%
\pgfpathlineto{\pgfqpoint{4.263510in}{1.381791in}}%
\pgfpathlineto{\pgfqpoint{4.284900in}{1.381791in}}%
\pgfpathlineto{\pgfqpoint{4.306290in}{1.381791in}}%
\pgfpathlineto{\pgfqpoint{4.327680in}{1.381791in}}%
\pgfpathlineto{\pgfqpoint{4.349070in}{1.381791in}}%
\pgfpathlineto{\pgfqpoint{4.370460in}{1.381791in}}%
\pgfpathlineto{\pgfqpoint{4.391850in}{1.381791in}}%
\pgfpathlineto{\pgfqpoint{4.413240in}{1.381791in}}%
\pgfpathlineto{\pgfqpoint{4.434630in}{1.381791in}}%
\pgfpathlineto{\pgfqpoint{4.456020in}{1.381791in}}%
\pgfpathlineto{\pgfqpoint{4.477410in}{1.378867in}}%
\pgfpathlineto{\pgfqpoint{4.498800in}{1.378867in}}%
\pgfpathlineto{\pgfqpoint{4.520190in}{1.378867in}}%
\pgfpathlineto{\pgfqpoint{4.541580in}{1.378867in}}%
\pgfpathlineto{\pgfqpoint{4.562970in}{1.378867in}}%
\pgfpathlineto{\pgfqpoint{4.584360in}{1.378867in}}%
\pgfpathlineto{\pgfqpoint{4.605750in}{1.378867in}}%
\pgfpathlineto{\pgfqpoint{4.627140in}{1.378867in}}%
\pgfpathlineto{\pgfqpoint{4.648530in}{1.378867in}}%
\pgfpathlineto{\pgfqpoint{4.669920in}{1.378867in}}%
\pgfpathlineto{\pgfqpoint{4.691310in}{1.378867in}}%
\pgfpathlineto{\pgfqpoint{4.712700in}{1.378867in}}%
\pgfpathlineto{\pgfqpoint{4.734090in}{1.360991in}}%
\pgfpathlineto{\pgfqpoint{4.755480in}{1.360991in}}%
\pgfpathlineto{\pgfqpoint{4.776870in}{1.360991in}}%
\pgfpathlineto{\pgfqpoint{4.798260in}{1.360991in}}%
\pgfpathlineto{\pgfqpoint{4.819650in}{1.360991in}}%
\pgfpathlineto{\pgfqpoint{4.841040in}{1.360991in}}%
\pgfpathlineto{\pgfqpoint{4.862430in}{1.360991in}}%
\pgfpathlineto{\pgfqpoint{4.883820in}{1.360991in}}%
\pgfpathlineto{\pgfqpoint{4.905210in}{1.360991in}}%
\pgfpathlineto{\pgfqpoint{4.926600in}{1.360991in}}%
\pgfpathlineto{\pgfqpoint{4.947990in}{1.350737in}}%
\pgfpathlineto{\pgfqpoint{4.969380in}{1.350737in}}%
\pgfpathlineto{\pgfqpoint{4.990770in}{1.350737in}}%
\pgfpathlineto{\pgfqpoint{5.012160in}{1.350737in}}%
\pgfpathlineto{\pgfqpoint{5.033550in}{1.350737in}}%
\pgfpathlineto{\pgfqpoint{5.054940in}{1.350737in}}%
\pgfpathlineto{\pgfqpoint{5.076330in}{1.350737in}}%
\pgfpathlineto{\pgfqpoint{5.097720in}{1.350737in}}%
\pgfpathlineto{\pgfqpoint{5.119110in}{1.350737in}}%
\pgfpathlineto{\pgfqpoint{5.140500in}{1.350737in}}%
\pgfpathlineto{\pgfqpoint{5.161890in}{1.350737in}}%
\pgfpathlineto{\pgfqpoint{5.183280in}{1.350737in}}%
\pgfpathlineto{\pgfqpoint{5.204670in}{1.350737in}}%
\pgfpathlineto{\pgfqpoint{5.226060in}{1.350737in}}%
\pgfpathlineto{\pgfqpoint{5.247450in}{1.350737in}}%
\pgfpathlineto{\pgfqpoint{5.268840in}{1.350737in}}%
\pgfpathlineto{\pgfqpoint{5.290230in}{1.350093in}}%
\pgfpathlineto{\pgfqpoint{5.311620in}{1.350093in}}%
\pgfpathlineto{\pgfqpoint{5.333010in}{1.350093in}}%
\pgfpathlineto{\pgfqpoint{5.354400in}{1.350093in}}%
\pgfpathlineto{\pgfqpoint{5.375790in}{1.350093in}}%
\pgfpathlineto{\pgfqpoint{5.397180in}{1.350093in}}%
\pgfpathlineto{\pgfqpoint{5.418570in}{1.350093in}}%
\pgfpathlineto{\pgfqpoint{5.439960in}{1.350093in}}%
\pgfpathlineto{\pgfqpoint{5.461350in}{1.324969in}}%
\pgfpathlineto{\pgfqpoint{5.482740in}{1.324969in}}%
\pgfpathlineto{\pgfqpoint{5.504130in}{1.324969in}}%
\pgfpathlineto{\pgfqpoint{5.525520in}{1.324969in}}%
\pgfpathlineto{\pgfqpoint{5.546910in}{1.324969in}}%
\pgfpathlineto{\pgfqpoint{5.568300in}{1.324969in}}%
\pgfpathlineto{\pgfqpoint{5.589690in}{1.324969in}}%
\pgfpathlineto{\pgfqpoint{5.611080in}{1.324969in}}%
\pgfpathlineto{\pgfqpoint{5.632470in}{1.324969in}}%
\pgfpathlineto{\pgfqpoint{5.653860in}{1.300186in}}%
\pgfpathlineto{\pgfqpoint{5.675250in}{1.300186in}}%
\pgfpathlineto{\pgfqpoint{5.696640in}{1.300186in}}%
\pgfpathlineto{\pgfqpoint{5.718030in}{1.300186in}}%
\pgfpathlineto{\pgfqpoint{5.739420in}{1.300186in}}%
\pgfpathlineto{\pgfqpoint{5.760810in}{1.300186in}}%
\pgfpathlineto{\pgfqpoint{5.782200in}{1.300186in}}%
\pgfpathlineto{\pgfqpoint{5.803590in}{1.300186in}}%
\pgfpathlineto{\pgfqpoint{5.824980in}{1.300186in}}%
\pgfpathlineto{\pgfqpoint{5.846370in}{1.300186in}}%
\pgfpathlineto{\pgfqpoint{5.867760in}{1.300186in}}%
\pgfpathlineto{\pgfqpoint{5.889150in}{1.300186in}}%
\pgfpathlineto{\pgfqpoint{5.910540in}{1.300186in}}%
\pgfpathlineto{\pgfqpoint{5.931930in}{1.300186in}}%
\pgfpathlineto{\pgfqpoint{5.953320in}{1.300186in}}%
\pgfpathlineto{\pgfqpoint{5.974710in}{1.300186in}}%
\pgfpathlineto{\pgfqpoint{5.996100in}{1.300186in}}%
\pgfpathlineto{\pgfqpoint{6.017490in}{1.300186in}}%
\pgfpathlineto{\pgfqpoint{6.038880in}{1.300186in}}%
\pgfpathlineto{\pgfqpoint{6.060270in}{1.300186in}}%
\pgfpathlineto{\pgfqpoint{6.081660in}{1.300186in}}%
\pgfpathlineto{\pgfqpoint{6.103050in}{1.300186in}}%
\pgfpathlineto{\pgfqpoint{6.124440in}{1.300186in}}%
\pgfpathlineto{\pgfqpoint{6.145830in}{1.300186in}}%
\pgfpathlineto{\pgfqpoint{6.167220in}{1.300186in}}%
\pgfpathlineto{\pgfqpoint{6.188610in}{1.289685in}}%
\pgfpathlineto{\pgfqpoint{6.210000in}{1.289685in}}%
\pgfpathlineto{\pgfqpoint{6.231390in}{1.289685in}}%
\pgfpathlineto{\pgfqpoint{6.252780in}{1.289685in}}%
\pgfpathlineto{\pgfqpoint{6.274170in}{1.289685in}}%
\pgfpathlineto{\pgfqpoint{6.295560in}{1.289685in}}%
\pgfpathlineto{\pgfqpoint{6.316950in}{1.289685in}}%
\pgfpathlineto{\pgfqpoint{6.338340in}{1.289685in}}%
\pgfpathlineto{\pgfqpoint{6.359730in}{1.289685in}}%
\pgfpathlineto{\pgfqpoint{6.381120in}{1.289685in}}%
\pgfpathlineto{\pgfqpoint{6.402510in}{1.289685in}}%
\pgfpathlineto{\pgfqpoint{6.423900in}{1.289685in}}%
\pgfpathlineto{\pgfqpoint{6.445290in}{1.289685in}}%
\pgfpathlineto{\pgfqpoint{6.466680in}{1.289685in}}%
\pgfpathlineto{\pgfqpoint{6.488070in}{1.289685in}}%
\pgfpathlineto{\pgfqpoint{6.509460in}{1.289685in}}%
\pgfpathlineto{\pgfqpoint{6.530850in}{1.289685in}}%
\pgfpathlineto{\pgfqpoint{6.552240in}{1.289685in}}%
\pgfpathlineto{\pgfqpoint{6.573630in}{1.289685in}}%
\pgfpathlineto{\pgfqpoint{6.595020in}{1.289685in}}%
\pgfpathlineto{\pgfqpoint{6.616410in}{1.289685in}}%
\pgfpathlineto{\pgfqpoint{6.616410in}{0.926121in}}%
\pgfpathlineto{\pgfqpoint{6.616410in}{0.926121in}}%
\pgfpathlineto{\pgfqpoint{6.595020in}{0.926121in}}%
\pgfpathlineto{\pgfqpoint{6.573630in}{0.926121in}}%
\pgfpathlineto{\pgfqpoint{6.552240in}{0.926121in}}%
\pgfpathlineto{\pgfqpoint{6.530850in}{0.926121in}}%
\pgfpathlineto{\pgfqpoint{6.509460in}{0.926121in}}%
\pgfpathlineto{\pgfqpoint{6.488070in}{0.926121in}}%
\pgfpathlineto{\pgfqpoint{6.466680in}{0.926121in}}%
\pgfpathlineto{\pgfqpoint{6.445290in}{0.926121in}}%
\pgfpathlineto{\pgfqpoint{6.423900in}{0.926121in}}%
\pgfpathlineto{\pgfqpoint{6.402510in}{0.926121in}}%
\pgfpathlineto{\pgfqpoint{6.381120in}{0.926121in}}%
\pgfpathlineto{\pgfqpoint{6.359730in}{0.926121in}}%
\pgfpathlineto{\pgfqpoint{6.338340in}{0.926121in}}%
\pgfpathlineto{\pgfqpoint{6.316950in}{0.926121in}}%
\pgfpathlineto{\pgfqpoint{6.295560in}{0.926121in}}%
\pgfpathlineto{\pgfqpoint{6.274170in}{0.926121in}}%
\pgfpathlineto{\pgfqpoint{6.252780in}{0.926121in}}%
\pgfpathlineto{\pgfqpoint{6.231390in}{0.926121in}}%
\pgfpathlineto{\pgfqpoint{6.210000in}{0.926121in}}%
\pgfpathlineto{\pgfqpoint{6.188610in}{0.926121in}}%
\pgfpathlineto{\pgfqpoint{6.167220in}{0.951766in}}%
\pgfpathlineto{\pgfqpoint{6.145830in}{0.951766in}}%
\pgfpathlineto{\pgfqpoint{6.124440in}{0.951766in}}%
\pgfpathlineto{\pgfqpoint{6.103050in}{0.951766in}}%
\pgfpathlineto{\pgfqpoint{6.081660in}{0.951766in}}%
\pgfpathlineto{\pgfqpoint{6.060270in}{0.951766in}}%
\pgfpathlineto{\pgfqpoint{6.038880in}{0.951766in}}%
\pgfpathlineto{\pgfqpoint{6.017490in}{0.951766in}}%
\pgfpathlineto{\pgfqpoint{5.996100in}{0.951766in}}%
\pgfpathlineto{\pgfqpoint{5.974710in}{0.951766in}}%
\pgfpathlineto{\pgfqpoint{5.953320in}{0.951766in}}%
\pgfpathlineto{\pgfqpoint{5.931930in}{0.951766in}}%
\pgfpathlineto{\pgfqpoint{5.910540in}{0.951766in}}%
\pgfpathlineto{\pgfqpoint{5.889150in}{0.951766in}}%
\pgfpathlineto{\pgfqpoint{5.867760in}{0.951766in}}%
\pgfpathlineto{\pgfqpoint{5.846370in}{0.951766in}}%
\pgfpathlineto{\pgfqpoint{5.824980in}{0.951766in}}%
\pgfpathlineto{\pgfqpoint{5.803590in}{0.951766in}}%
\pgfpathlineto{\pgfqpoint{5.782200in}{0.951766in}}%
\pgfpathlineto{\pgfqpoint{5.760810in}{0.951766in}}%
\pgfpathlineto{\pgfqpoint{5.739420in}{0.951766in}}%
\pgfpathlineto{\pgfqpoint{5.718030in}{0.951766in}}%
\pgfpathlineto{\pgfqpoint{5.696640in}{0.951766in}}%
\pgfpathlineto{\pgfqpoint{5.675250in}{0.951766in}}%
\pgfpathlineto{\pgfqpoint{5.653860in}{0.951766in}}%
\pgfpathlineto{\pgfqpoint{5.632470in}{1.042487in}}%
\pgfpathlineto{\pgfqpoint{5.611080in}{1.042487in}}%
\pgfpathlineto{\pgfqpoint{5.589690in}{1.042487in}}%
\pgfpathlineto{\pgfqpoint{5.568300in}{1.042487in}}%
\pgfpathlineto{\pgfqpoint{5.546910in}{1.042487in}}%
\pgfpathlineto{\pgfqpoint{5.525520in}{1.042487in}}%
\pgfpathlineto{\pgfqpoint{5.504130in}{1.042487in}}%
\pgfpathlineto{\pgfqpoint{5.482740in}{1.042487in}}%
\pgfpathlineto{\pgfqpoint{5.461350in}{1.042487in}}%
\pgfpathlineto{\pgfqpoint{5.439960in}{1.082453in}}%
\pgfpathlineto{\pgfqpoint{5.418570in}{1.082453in}}%
\pgfpathlineto{\pgfqpoint{5.397180in}{1.082453in}}%
\pgfpathlineto{\pgfqpoint{5.375790in}{1.082453in}}%
\pgfpathlineto{\pgfqpoint{5.354400in}{1.082453in}}%
\pgfpathlineto{\pgfqpoint{5.333010in}{1.082453in}}%
\pgfpathlineto{\pgfqpoint{5.311620in}{1.082453in}}%
\pgfpathlineto{\pgfqpoint{5.290230in}{1.082453in}}%
\pgfpathlineto{\pgfqpoint{5.268840in}{1.083271in}}%
\pgfpathlineto{\pgfqpoint{5.247450in}{1.083271in}}%
\pgfpathlineto{\pgfqpoint{5.226060in}{1.083271in}}%
\pgfpathlineto{\pgfqpoint{5.204670in}{1.083271in}}%
\pgfpathlineto{\pgfqpoint{5.183280in}{1.083271in}}%
\pgfpathlineto{\pgfqpoint{5.161890in}{1.083271in}}%
\pgfpathlineto{\pgfqpoint{5.140500in}{1.083271in}}%
\pgfpathlineto{\pgfqpoint{5.119110in}{1.083271in}}%
\pgfpathlineto{\pgfqpoint{5.097720in}{1.083271in}}%
\pgfpathlineto{\pgfqpoint{5.076330in}{1.083271in}}%
\pgfpathlineto{\pgfqpoint{5.054940in}{1.083271in}}%
\pgfpathlineto{\pgfqpoint{5.033550in}{1.083271in}}%
\pgfpathlineto{\pgfqpoint{5.012160in}{1.083271in}}%
\pgfpathlineto{\pgfqpoint{4.990770in}{1.083271in}}%
\pgfpathlineto{\pgfqpoint{4.969380in}{1.083271in}}%
\pgfpathlineto{\pgfqpoint{4.947990in}{1.083271in}}%
\pgfpathlineto{\pgfqpoint{4.926600in}{1.107183in}}%
\pgfpathlineto{\pgfqpoint{4.905210in}{1.107183in}}%
\pgfpathlineto{\pgfqpoint{4.883820in}{1.107183in}}%
\pgfpathlineto{\pgfqpoint{4.862430in}{1.107183in}}%
\pgfpathlineto{\pgfqpoint{4.841040in}{1.107183in}}%
\pgfpathlineto{\pgfqpoint{4.819650in}{1.107183in}}%
\pgfpathlineto{\pgfqpoint{4.798260in}{1.107183in}}%
\pgfpathlineto{\pgfqpoint{4.776870in}{1.107183in}}%
\pgfpathlineto{\pgfqpoint{4.755480in}{1.107183in}}%
\pgfpathlineto{\pgfqpoint{4.734090in}{1.107183in}}%
\pgfpathlineto{\pgfqpoint{4.712700in}{1.137658in}}%
\pgfpathlineto{\pgfqpoint{4.691310in}{1.137658in}}%
\pgfpathlineto{\pgfqpoint{4.669920in}{1.137658in}}%
\pgfpathlineto{\pgfqpoint{4.648530in}{1.137658in}}%
\pgfpathlineto{\pgfqpoint{4.627140in}{1.137658in}}%
\pgfpathlineto{\pgfqpoint{4.605750in}{1.137658in}}%
\pgfpathlineto{\pgfqpoint{4.584360in}{1.137658in}}%
\pgfpathlineto{\pgfqpoint{4.562970in}{1.137658in}}%
\pgfpathlineto{\pgfqpoint{4.541580in}{1.137658in}}%
\pgfpathlineto{\pgfqpoint{4.520190in}{1.137658in}}%
\pgfpathlineto{\pgfqpoint{4.498800in}{1.137658in}}%
\pgfpathlineto{\pgfqpoint{4.477410in}{1.137658in}}%
\pgfpathlineto{\pgfqpoint{4.456020in}{1.141526in}}%
\pgfpathlineto{\pgfqpoint{4.434630in}{1.141526in}}%
\pgfpathlineto{\pgfqpoint{4.413240in}{1.141526in}}%
\pgfpathlineto{\pgfqpoint{4.391850in}{1.141526in}}%
\pgfpathlineto{\pgfqpoint{4.370460in}{1.141526in}}%
\pgfpathlineto{\pgfqpoint{4.349070in}{1.141526in}}%
\pgfpathlineto{\pgfqpoint{4.327680in}{1.141526in}}%
\pgfpathlineto{\pgfqpoint{4.306290in}{1.141526in}}%
\pgfpathlineto{\pgfqpoint{4.284900in}{1.141526in}}%
\pgfpathlineto{\pgfqpoint{4.263510in}{1.141526in}}%
\pgfpathlineto{\pgfqpoint{4.242120in}{1.141526in}}%
\pgfpathlineto{\pgfqpoint{4.220730in}{1.141526in}}%
\pgfpathlineto{\pgfqpoint{4.199340in}{1.141526in}}%
\pgfpathlineto{\pgfqpoint{4.177950in}{1.141526in}}%
\pgfpathlineto{\pgfqpoint{4.156560in}{1.141526in}}%
\pgfpathlineto{\pgfqpoint{4.135170in}{1.141526in}}%
\pgfpathlineto{\pgfqpoint{4.113780in}{1.142626in}}%
\pgfpathlineto{\pgfqpoint{4.092390in}{1.142626in}}%
\pgfpathlineto{\pgfqpoint{4.071000in}{1.142626in}}%
\pgfpathlineto{\pgfqpoint{4.049610in}{1.142626in}}%
\pgfpathlineto{\pgfqpoint{4.028220in}{1.142626in}}%
\pgfpathlineto{\pgfqpoint{4.006830in}{1.142626in}}%
\pgfpathlineto{\pgfqpoint{3.985440in}{1.142626in}}%
\pgfpathlineto{\pgfqpoint{3.964050in}{1.142626in}}%
\pgfpathlineto{\pgfqpoint{3.942660in}{1.142626in}}%
\pgfpathlineto{\pgfqpoint{3.921270in}{1.142626in}}%
\pgfpathlineto{\pgfqpoint{3.899880in}{1.142626in}}%
\pgfpathlineto{\pgfqpoint{3.878490in}{1.187434in}}%
\pgfpathlineto{\pgfqpoint{3.857100in}{1.187434in}}%
\pgfpathlineto{\pgfqpoint{3.835710in}{1.187434in}}%
\pgfpathlineto{\pgfqpoint{3.814320in}{1.187459in}}%
\pgfpathlineto{\pgfqpoint{3.792930in}{1.187459in}}%
\pgfpathlineto{\pgfqpoint{3.771540in}{1.187459in}}%
\pgfpathlineto{\pgfqpoint{3.750150in}{1.201770in}}%
\pgfpathlineto{\pgfqpoint{3.728760in}{1.201770in}}%
\pgfpathlineto{\pgfqpoint{3.707370in}{1.201770in}}%
\pgfpathlineto{\pgfqpoint{3.685980in}{1.202522in}}%
\pgfpathlineto{\pgfqpoint{3.664590in}{1.202522in}}%
\pgfpathlineto{\pgfqpoint{3.643200in}{1.202522in}}%
\pgfpathlineto{\pgfqpoint{3.621810in}{1.202522in}}%
\pgfpathlineto{\pgfqpoint{3.600420in}{1.202522in}}%
\pgfpathlineto{\pgfqpoint{3.579030in}{1.202522in}}%
\pgfpathlineto{\pgfqpoint{3.557640in}{1.202522in}}%
\pgfpathlineto{\pgfqpoint{3.536250in}{1.202522in}}%
\pgfpathlineto{\pgfqpoint{3.514860in}{1.202522in}}%
\pgfpathlineto{\pgfqpoint{3.493470in}{1.202522in}}%
\pgfpathlineto{\pgfqpoint{3.472080in}{1.202522in}}%
\pgfpathlineto{\pgfqpoint{3.450690in}{1.202522in}}%
\pgfpathlineto{\pgfqpoint{3.429300in}{1.202522in}}%
\pgfpathlineto{\pgfqpoint{3.407910in}{1.202522in}}%
\pgfpathlineto{\pgfqpoint{3.386520in}{1.202522in}}%
\pgfpathlineto{\pgfqpoint{3.365130in}{1.202522in}}%
\pgfpathlineto{\pgfqpoint{3.343740in}{1.202522in}}%
\pgfpathlineto{\pgfqpoint{3.322350in}{1.207189in}}%
\pgfpathlineto{\pgfqpoint{3.300960in}{1.218752in}}%
\pgfpathlineto{\pgfqpoint{3.279570in}{1.218752in}}%
\pgfpathlineto{\pgfqpoint{3.258180in}{1.229614in}}%
\pgfpathlineto{\pgfqpoint{3.236790in}{1.229614in}}%
\pgfpathlineto{\pgfqpoint{3.215400in}{1.229614in}}%
\pgfpathlineto{\pgfqpoint{3.194010in}{1.229614in}}%
\pgfpathlineto{\pgfqpoint{3.172620in}{1.235859in}}%
\pgfpathlineto{\pgfqpoint{3.151230in}{1.235859in}}%
\pgfpathlineto{\pgfqpoint{3.129840in}{1.235859in}}%
\pgfpathlineto{\pgfqpoint{3.108450in}{1.235859in}}%
\pgfpathlineto{\pgfqpoint{3.087060in}{1.235859in}}%
\pgfpathlineto{\pgfqpoint{3.065670in}{1.235859in}}%
\pgfpathlineto{\pgfqpoint{3.044280in}{1.283450in}}%
\pgfpathlineto{\pgfqpoint{3.022890in}{1.283450in}}%
\pgfpathlineto{\pgfqpoint{3.001500in}{1.283450in}}%
\pgfpathlineto{\pgfqpoint{2.980110in}{1.302457in}}%
\pgfpathlineto{\pgfqpoint{2.958720in}{1.302457in}}%
\pgfpathlineto{\pgfqpoint{2.937330in}{1.302457in}}%
\pgfpathlineto{\pgfqpoint{2.915940in}{1.302524in}}%
\pgfpathlineto{\pgfqpoint{2.894550in}{1.302524in}}%
\pgfpathlineto{\pgfqpoint{2.873160in}{1.302524in}}%
\pgfpathlineto{\pgfqpoint{2.851770in}{1.302524in}}%
\pgfpathlineto{\pgfqpoint{2.830380in}{1.302524in}}%
\pgfpathlineto{\pgfqpoint{2.808990in}{1.302524in}}%
\pgfpathlineto{\pgfqpoint{2.787600in}{1.305722in}}%
\pgfpathlineto{\pgfqpoint{2.766210in}{1.305722in}}%
\pgfpathlineto{\pgfqpoint{2.744820in}{1.305722in}}%
\pgfpathlineto{\pgfqpoint{2.723430in}{1.305722in}}%
\pgfpathlineto{\pgfqpoint{2.702040in}{1.305722in}}%
\pgfpathlineto{\pgfqpoint{2.680650in}{1.305722in}}%
\pgfpathlineto{\pgfqpoint{2.659260in}{1.305722in}}%
\pgfpathlineto{\pgfqpoint{2.637870in}{1.305722in}}%
\pgfpathlineto{\pgfqpoint{2.616480in}{1.305722in}}%
\pgfpathlineto{\pgfqpoint{2.595090in}{1.305722in}}%
\pgfpathlineto{\pgfqpoint{2.573700in}{1.305722in}}%
\pgfpathlineto{\pgfqpoint{2.552310in}{1.316904in}}%
\pgfpathlineto{\pgfqpoint{2.530920in}{1.316904in}}%
\pgfpathlineto{\pgfqpoint{2.509530in}{1.316904in}}%
\pgfpathlineto{\pgfqpoint{2.488140in}{1.316904in}}%
\pgfpathlineto{\pgfqpoint{2.466750in}{1.316904in}}%
\pgfpathlineto{\pgfqpoint{2.445360in}{1.317917in}}%
\pgfpathlineto{\pgfqpoint{2.423970in}{1.320023in}}%
\pgfpathlineto{\pgfqpoint{2.402580in}{1.320023in}}%
\pgfpathlineto{\pgfqpoint{2.381190in}{1.320023in}}%
\pgfpathlineto{\pgfqpoint{2.359800in}{1.320023in}}%
\pgfpathlineto{\pgfqpoint{2.338410in}{1.320023in}}%
\pgfpathlineto{\pgfqpoint{2.317020in}{1.320023in}}%
\pgfpathlineto{\pgfqpoint{2.295630in}{1.344381in}}%
\pgfpathlineto{\pgfqpoint{2.274240in}{1.344381in}}%
\pgfpathlineto{\pgfqpoint{2.252850in}{1.344381in}}%
\pgfpathlineto{\pgfqpoint{2.231460in}{1.344381in}}%
\pgfpathlineto{\pgfqpoint{2.210070in}{1.344381in}}%
\pgfpathlineto{\pgfqpoint{2.188680in}{1.344381in}}%
\pgfpathlineto{\pgfqpoint{2.167290in}{1.344381in}}%
\pgfpathlineto{\pgfqpoint{2.145900in}{1.344381in}}%
\pgfpathlineto{\pgfqpoint{2.124510in}{1.344381in}}%
\pgfpathlineto{\pgfqpoint{2.103120in}{1.356333in}}%
\pgfpathlineto{\pgfqpoint{2.081730in}{1.356333in}}%
\pgfpathlineto{\pgfqpoint{2.060340in}{1.356333in}}%
\pgfpathlineto{\pgfqpoint{2.038950in}{1.356333in}}%
\pgfpathlineto{\pgfqpoint{2.017560in}{1.356333in}}%
\pgfpathlineto{\pgfqpoint{1.996170in}{1.356333in}}%
\pgfpathlineto{\pgfqpoint{1.974780in}{1.360100in}}%
\pgfpathlineto{\pgfqpoint{1.953390in}{1.381649in}}%
\pgfpathlineto{\pgfqpoint{1.932000in}{1.381649in}}%
\pgfpathlineto{\pgfqpoint{1.910610in}{1.393274in}}%
\pgfpathlineto{\pgfqpoint{1.889220in}{1.393274in}}%
\pgfpathlineto{\pgfqpoint{1.867830in}{1.393274in}}%
\pgfpathlineto{\pgfqpoint{1.846440in}{1.393274in}}%
\pgfpathlineto{\pgfqpoint{1.825050in}{1.393274in}}%
\pgfpathlineto{\pgfqpoint{1.803660in}{1.393274in}}%
\pgfpathlineto{\pgfqpoint{1.782270in}{1.393274in}}%
\pgfpathlineto{\pgfqpoint{1.760880in}{1.393274in}}%
\pgfpathlineto{\pgfqpoint{1.739490in}{1.393274in}}%
\pgfpathlineto{\pgfqpoint{1.718100in}{1.406516in}}%
\pgfpathlineto{\pgfqpoint{1.696710in}{1.406516in}}%
\pgfpathlineto{\pgfqpoint{1.675320in}{1.406516in}}%
\pgfpathlineto{\pgfqpoint{1.653930in}{1.406516in}}%
\pgfpathlineto{\pgfqpoint{1.632540in}{1.527737in}}%
\pgfpathlineto{\pgfqpoint{1.611150in}{1.540855in}}%
\pgfpathlineto{\pgfqpoint{1.589760in}{1.605317in}}%
\pgfpathlineto{\pgfqpoint{1.568370in}{1.680214in}}%
\pgfpathlineto{\pgfqpoint{1.546980in}{1.799327in}}%
\pgfpathlineto{\pgfqpoint{1.525590in}{1.860716in}}%
\pgfpathlineto{\pgfqpoint{1.504200in}{1.967086in}}%
\pgfpathlineto{\pgfqpoint{1.482810in}{1.984518in}}%
\pgfpathlineto{\pgfqpoint{1.461420in}{2.011667in}}%
\pgfpathlineto{\pgfqpoint{1.440030in}{2.041224in}}%
\pgfpathlineto{\pgfqpoint{1.418640in}{2.048980in}}%
\pgfpathlineto{\pgfqpoint{1.397250in}{2.050450in}}%
\pgfpathlineto{\pgfqpoint{1.375860in}{2.081846in}}%
\pgfpathlineto{\pgfqpoint{1.354470in}{2.094352in}}%
\pgfpathlineto{\pgfqpoint{1.333080in}{2.117370in}}%
\pgfpathlineto{\pgfqpoint{1.311690in}{2.167284in}}%
\pgfpathlineto{\pgfqpoint{1.290300in}{2.176343in}}%
\pgfpathlineto{\pgfqpoint{1.268910in}{2.176343in}}%
\pgfpathlineto{\pgfqpoint{1.247520in}{2.184727in}}%
\pgfpathlineto{\pgfqpoint{1.226130in}{2.184727in}}%
\pgfpathlineto{\pgfqpoint{1.204740in}{2.184727in}}%
\pgfpathlineto{\pgfqpoint{1.183350in}{2.189513in}}%
\pgfpathlineto{\pgfqpoint{1.161960in}{2.189513in}}%
\pgfpathlineto{\pgfqpoint{1.140570in}{2.189513in}}%
\pgfpathlineto{\pgfqpoint{1.119180in}{2.218698in}}%
\pgfpathlineto{\pgfqpoint{1.097790in}{2.218698in}}%
\pgfpathlineto{\pgfqpoint{1.076400in}{2.218698in}}%
\pgfpathlineto{\pgfqpoint{1.055010in}{2.218698in}}%
\pgfpathlineto{\pgfqpoint{1.033620in}{2.268262in}}%
\pgfpathlineto{\pgfqpoint{1.012230in}{2.309182in}}%
\pgfpathlineto{\pgfqpoint{0.990840in}{2.309182in}}%
\pgfpathlineto{\pgfqpoint{0.969450in}{2.338515in}}%
\pgfpathlineto{\pgfqpoint{0.948060in}{2.400905in}}%
\pgfpathlineto{\pgfqpoint{0.926670in}{2.400905in}}%
\pgfpathlineto{\pgfqpoint{0.905280in}{2.400905in}}%
\pgfpathlineto{\pgfqpoint{0.883890in}{2.415034in}}%
\pgfpathlineto{\pgfqpoint{0.862500in}{2.476284in}}%
\pgfpathclose%
\pgfusepath{fill}%
\end{pgfscope}%
\begin{pgfscope}%
\pgfpathrectangle{\pgfqpoint{0.862500in}{0.375000in}}{\pgfqpoint{5.347500in}{2.265000in}}%
\pgfusepath{clip}%
\pgfsetbuttcap%
\pgfsetroundjoin%
\definecolor{currentfill}{rgb}{0.172549,0.627451,0.172549}%
\pgfsetfillcolor{currentfill}%
\pgfsetfillopacity{0.200000}%
\pgfsetlinewidth{0.000000pt}%
\definecolor{currentstroke}{rgb}{0.000000,0.000000,0.000000}%
\pgfsetstrokecolor{currentstroke}%
\pgfsetdash{}{0pt}%
\pgfpathmoveto{\pgfqpoint{0.862500in}{2.351316in}}%
\pgfpathlineto{\pgfqpoint{0.862500in}{2.473323in}}%
\pgfpathlineto{\pgfqpoint{0.883890in}{2.459160in}}%
\pgfpathlineto{\pgfqpoint{0.905280in}{2.445479in}}%
\pgfpathlineto{\pgfqpoint{0.926670in}{2.445479in}}%
\pgfpathlineto{\pgfqpoint{0.948060in}{2.333637in}}%
\pgfpathlineto{\pgfqpoint{0.969450in}{2.321554in}}%
\pgfpathlineto{\pgfqpoint{0.990840in}{2.238928in}}%
\pgfpathlineto{\pgfqpoint{1.012230in}{2.224194in}}%
\pgfpathlineto{\pgfqpoint{1.033620in}{2.224194in}}%
\pgfpathlineto{\pgfqpoint{1.055010in}{2.224194in}}%
\pgfpathlineto{\pgfqpoint{1.076400in}{2.224194in}}%
\pgfpathlineto{\pgfqpoint{1.097790in}{2.224194in}}%
\pgfpathlineto{\pgfqpoint{1.119180in}{2.224194in}}%
\pgfpathlineto{\pgfqpoint{1.140570in}{2.215755in}}%
\pgfpathlineto{\pgfqpoint{1.161960in}{2.215755in}}%
\pgfpathlineto{\pgfqpoint{1.183350in}{2.215755in}}%
\pgfpathlineto{\pgfqpoint{1.204740in}{2.215755in}}%
\pgfpathlineto{\pgfqpoint{1.226130in}{2.209066in}}%
\pgfpathlineto{\pgfqpoint{1.247520in}{2.209066in}}%
\pgfpathlineto{\pgfqpoint{1.268910in}{2.209066in}}%
\pgfpathlineto{\pgfqpoint{1.290300in}{2.209066in}}%
\pgfpathlineto{\pgfqpoint{1.311690in}{2.204177in}}%
\pgfpathlineto{\pgfqpoint{1.333080in}{2.204177in}}%
\pgfpathlineto{\pgfqpoint{1.354470in}{2.204177in}}%
\pgfpathlineto{\pgfqpoint{1.375860in}{2.182097in}}%
\pgfpathlineto{\pgfqpoint{1.397250in}{2.182097in}}%
\pgfpathlineto{\pgfqpoint{1.418640in}{2.139825in}}%
\pgfpathlineto{\pgfqpoint{1.440030in}{2.139825in}}%
\pgfpathlineto{\pgfqpoint{1.461420in}{2.139825in}}%
\pgfpathlineto{\pgfqpoint{1.482810in}{2.139825in}}%
\pgfpathlineto{\pgfqpoint{1.504200in}{2.139825in}}%
\pgfpathlineto{\pgfqpoint{1.525590in}{2.139825in}}%
\pgfpathlineto{\pgfqpoint{1.546980in}{2.139825in}}%
\pgfpathlineto{\pgfqpoint{1.568370in}{2.139825in}}%
\pgfpathlineto{\pgfqpoint{1.589760in}{2.093223in}}%
\pgfpathlineto{\pgfqpoint{1.611150in}{2.093223in}}%
\pgfpathlineto{\pgfqpoint{1.632540in}{2.093223in}}%
\pgfpathlineto{\pgfqpoint{1.653930in}{1.896077in}}%
\pgfpathlineto{\pgfqpoint{1.675320in}{1.896077in}}%
\pgfpathlineto{\pgfqpoint{1.696710in}{1.896077in}}%
\pgfpathlineto{\pgfqpoint{1.718100in}{1.896077in}}%
\pgfpathlineto{\pgfqpoint{1.739490in}{1.896077in}}%
\pgfpathlineto{\pgfqpoint{1.760880in}{1.896077in}}%
\pgfpathlineto{\pgfqpoint{1.782270in}{1.896077in}}%
\pgfpathlineto{\pgfqpoint{1.803660in}{1.818792in}}%
\pgfpathlineto{\pgfqpoint{1.825050in}{1.818792in}}%
\pgfpathlineto{\pgfqpoint{1.846440in}{1.818792in}}%
\pgfpathlineto{\pgfqpoint{1.867830in}{1.818792in}}%
\pgfpathlineto{\pgfqpoint{1.889220in}{1.818792in}}%
\pgfpathlineto{\pgfqpoint{1.910610in}{1.818792in}}%
\pgfpathlineto{\pgfqpoint{1.932000in}{1.766500in}}%
\pgfpathlineto{\pgfqpoint{1.953390in}{1.750294in}}%
\pgfpathlineto{\pgfqpoint{1.974780in}{1.703171in}}%
\pgfpathlineto{\pgfqpoint{1.996170in}{1.703171in}}%
\pgfpathlineto{\pgfqpoint{2.017560in}{1.703171in}}%
\pgfpathlineto{\pgfqpoint{2.038950in}{1.625024in}}%
\pgfpathlineto{\pgfqpoint{2.060340in}{1.625024in}}%
\pgfpathlineto{\pgfqpoint{2.081730in}{1.625024in}}%
\pgfpathlineto{\pgfqpoint{2.103120in}{1.615680in}}%
\pgfpathlineto{\pgfqpoint{2.124510in}{1.615680in}}%
\pgfpathlineto{\pgfqpoint{2.145900in}{1.615680in}}%
\pgfpathlineto{\pgfqpoint{2.167290in}{1.614884in}}%
\pgfpathlineto{\pgfqpoint{2.188680in}{1.614884in}}%
\pgfpathlineto{\pgfqpoint{2.210070in}{1.609253in}}%
\pgfpathlineto{\pgfqpoint{2.231460in}{1.584646in}}%
\pgfpathlineto{\pgfqpoint{2.252850in}{1.568101in}}%
\pgfpathlineto{\pgfqpoint{2.274240in}{1.557566in}}%
\pgfpathlineto{\pgfqpoint{2.295630in}{1.557566in}}%
\pgfpathlineto{\pgfqpoint{2.317020in}{1.557566in}}%
\pgfpathlineto{\pgfqpoint{2.338410in}{1.557566in}}%
\pgfpathlineto{\pgfqpoint{2.359800in}{1.557566in}}%
\pgfpathlineto{\pgfqpoint{2.381190in}{1.480567in}}%
\pgfpathlineto{\pgfqpoint{2.402580in}{1.480567in}}%
\pgfpathlineto{\pgfqpoint{2.423970in}{1.480567in}}%
\pgfpathlineto{\pgfqpoint{2.445360in}{1.480567in}}%
\pgfpathlineto{\pgfqpoint{2.466750in}{1.480567in}}%
\pgfpathlineto{\pgfqpoint{2.488140in}{1.457941in}}%
\pgfpathlineto{\pgfqpoint{2.509530in}{1.457941in}}%
\pgfpathlineto{\pgfqpoint{2.530920in}{1.457941in}}%
\pgfpathlineto{\pgfqpoint{2.552310in}{1.457941in}}%
\pgfpathlineto{\pgfqpoint{2.573700in}{1.457941in}}%
\pgfpathlineto{\pgfqpoint{2.595090in}{1.457941in}}%
\pgfpathlineto{\pgfqpoint{2.616480in}{1.457941in}}%
\pgfpathlineto{\pgfqpoint{2.637870in}{1.457941in}}%
\pgfpathlineto{\pgfqpoint{2.659260in}{1.457941in}}%
\pgfpathlineto{\pgfqpoint{2.680650in}{1.457941in}}%
\pgfpathlineto{\pgfqpoint{2.702040in}{1.457941in}}%
\pgfpathlineto{\pgfqpoint{2.723430in}{1.457941in}}%
\pgfpathlineto{\pgfqpoint{2.744820in}{1.446271in}}%
\pgfpathlineto{\pgfqpoint{2.766210in}{1.446271in}}%
\pgfpathlineto{\pgfqpoint{2.787600in}{1.446271in}}%
\pgfpathlineto{\pgfqpoint{2.808990in}{1.446271in}}%
\pgfpathlineto{\pgfqpoint{2.830380in}{1.446271in}}%
\pgfpathlineto{\pgfqpoint{2.851770in}{1.444949in}}%
\pgfpathlineto{\pgfqpoint{2.873160in}{1.444949in}}%
\pgfpathlineto{\pgfqpoint{2.894550in}{1.444949in}}%
\pgfpathlineto{\pgfqpoint{2.915940in}{1.444949in}}%
\pgfpathlineto{\pgfqpoint{2.937330in}{1.444949in}}%
\pgfpathlineto{\pgfqpoint{2.958720in}{1.444949in}}%
\pgfpathlineto{\pgfqpoint{2.980110in}{1.444949in}}%
\pgfpathlineto{\pgfqpoint{3.001500in}{1.431089in}}%
\pgfpathlineto{\pgfqpoint{3.022890in}{1.431089in}}%
\pgfpathlineto{\pgfqpoint{3.044280in}{1.431089in}}%
\pgfpathlineto{\pgfqpoint{3.065670in}{1.431089in}}%
\pgfpathlineto{\pgfqpoint{3.087060in}{1.431089in}}%
\pgfpathlineto{\pgfqpoint{3.108450in}{1.431089in}}%
\pgfpathlineto{\pgfqpoint{3.129840in}{1.431089in}}%
\pgfpathlineto{\pgfqpoint{3.151230in}{1.431089in}}%
\pgfpathlineto{\pgfqpoint{3.172620in}{1.431089in}}%
\pgfpathlineto{\pgfqpoint{3.194010in}{1.431089in}}%
\pgfpathlineto{\pgfqpoint{3.215400in}{1.431089in}}%
\pgfpathlineto{\pgfqpoint{3.236790in}{1.431089in}}%
\pgfpathlineto{\pgfqpoint{3.258180in}{1.431089in}}%
\pgfpathlineto{\pgfqpoint{3.279570in}{1.431089in}}%
\pgfpathlineto{\pgfqpoint{3.300960in}{1.431089in}}%
\pgfpathlineto{\pgfqpoint{3.322350in}{1.431089in}}%
\pgfpathlineto{\pgfqpoint{3.343740in}{1.431089in}}%
\pgfpathlineto{\pgfqpoint{3.365130in}{1.431089in}}%
\pgfpathlineto{\pgfqpoint{3.386520in}{1.431089in}}%
\pgfpathlineto{\pgfqpoint{3.407910in}{1.431089in}}%
\pgfpathlineto{\pgfqpoint{3.429300in}{1.431089in}}%
\pgfpathlineto{\pgfqpoint{3.450690in}{1.431089in}}%
\pgfpathlineto{\pgfqpoint{3.472080in}{1.427131in}}%
\pgfpathlineto{\pgfqpoint{3.493470in}{1.427131in}}%
\pgfpathlineto{\pgfqpoint{3.514860in}{1.427131in}}%
\pgfpathlineto{\pgfqpoint{3.536250in}{1.427131in}}%
\pgfpathlineto{\pgfqpoint{3.557640in}{1.427131in}}%
\pgfpathlineto{\pgfqpoint{3.579030in}{1.427131in}}%
\pgfpathlineto{\pgfqpoint{3.600420in}{1.427131in}}%
\pgfpathlineto{\pgfqpoint{3.621810in}{1.427131in}}%
\pgfpathlineto{\pgfqpoint{3.643200in}{1.371850in}}%
\pgfpathlineto{\pgfqpoint{3.664590in}{1.371850in}}%
\pgfpathlineto{\pgfqpoint{3.685980in}{1.371850in}}%
\pgfpathlineto{\pgfqpoint{3.707370in}{1.371850in}}%
\pgfpathlineto{\pgfqpoint{3.728760in}{1.364208in}}%
\pgfpathlineto{\pgfqpoint{3.750150in}{1.364208in}}%
\pgfpathlineto{\pgfqpoint{3.771540in}{1.364208in}}%
\pgfpathlineto{\pgfqpoint{3.792930in}{1.364208in}}%
\pgfpathlineto{\pgfqpoint{3.814320in}{1.364208in}}%
\pgfpathlineto{\pgfqpoint{3.835710in}{1.364208in}}%
\pgfpathlineto{\pgfqpoint{3.857100in}{1.361942in}}%
\pgfpathlineto{\pgfqpoint{3.878490in}{1.355885in}}%
\pgfpathlineto{\pgfqpoint{3.899880in}{1.355885in}}%
\pgfpathlineto{\pgfqpoint{3.921270in}{1.355885in}}%
\pgfpathlineto{\pgfqpoint{3.942660in}{1.355885in}}%
\pgfpathlineto{\pgfqpoint{3.964050in}{1.355885in}}%
\pgfpathlineto{\pgfqpoint{3.985440in}{1.355885in}}%
\pgfpathlineto{\pgfqpoint{4.006830in}{1.355885in}}%
\pgfpathlineto{\pgfqpoint{4.028220in}{1.355885in}}%
\pgfpathlineto{\pgfqpoint{4.049610in}{1.355885in}}%
\pgfpathlineto{\pgfqpoint{4.071000in}{1.353428in}}%
\pgfpathlineto{\pgfqpoint{4.092390in}{1.353428in}}%
\pgfpathlineto{\pgfqpoint{4.113780in}{1.353428in}}%
\pgfpathlineto{\pgfqpoint{4.135170in}{1.353428in}}%
\pgfpathlineto{\pgfqpoint{4.156560in}{1.353428in}}%
\pgfpathlineto{\pgfqpoint{4.177950in}{1.294822in}}%
\pgfpathlineto{\pgfqpoint{4.199340in}{1.289948in}}%
\pgfpathlineto{\pgfqpoint{4.220730in}{1.289948in}}%
\pgfpathlineto{\pgfqpoint{4.242120in}{1.289948in}}%
\pgfpathlineto{\pgfqpoint{4.263510in}{1.289948in}}%
\pgfpathlineto{\pgfqpoint{4.284900in}{1.289948in}}%
\pgfpathlineto{\pgfqpoint{4.306290in}{1.289948in}}%
\pgfpathlineto{\pgfqpoint{4.327680in}{1.289948in}}%
\pgfpathlineto{\pgfqpoint{4.349070in}{1.289948in}}%
\pgfpathlineto{\pgfqpoint{4.370460in}{1.289948in}}%
\pgfpathlineto{\pgfqpoint{4.391850in}{1.289948in}}%
\pgfpathlineto{\pgfqpoint{4.413240in}{1.289948in}}%
\pgfpathlineto{\pgfqpoint{4.434630in}{1.289948in}}%
\pgfpathlineto{\pgfqpoint{4.456020in}{1.289948in}}%
\pgfpathlineto{\pgfqpoint{4.477410in}{1.289948in}}%
\pgfpathlineto{\pgfqpoint{4.498800in}{1.289948in}}%
\pgfpathlineto{\pgfqpoint{4.520190in}{1.289948in}}%
\pgfpathlineto{\pgfqpoint{4.541580in}{1.289948in}}%
\pgfpathlineto{\pgfqpoint{4.562970in}{1.289948in}}%
\pgfpathlineto{\pgfqpoint{4.584360in}{1.289948in}}%
\pgfpathlineto{\pgfqpoint{4.605750in}{1.289948in}}%
\pgfpathlineto{\pgfqpoint{4.627140in}{1.289948in}}%
\pgfpathlineto{\pgfqpoint{4.648530in}{1.289948in}}%
\pgfpathlineto{\pgfqpoint{4.669920in}{1.289948in}}%
\pgfpathlineto{\pgfqpoint{4.691310in}{1.289948in}}%
\pgfpathlineto{\pgfqpoint{4.712700in}{1.289948in}}%
\pgfpathlineto{\pgfqpoint{4.734090in}{1.289948in}}%
\pgfpathlineto{\pgfqpoint{4.755480in}{1.289948in}}%
\pgfpathlineto{\pgfqpoint{4.776870in}{1.289948in}}%
\pgfpathlineto{\pgfqpoint{4.798260in}{1.289948in}}%
\pgfpathlineto{\pgfqpoint{4.819650in}{1.289948in}}%
\pgfpathlineto{\pgfqpoint{4.841040in}{1.289948in}}%
\pgfpathlineto{\pgfqpoint{4.862430in}{1.289948in}}%
\pgfpathlineto{\pgfqpoint{4.883820in}{1.264985in}}%
\pgfpathlineto{\pgfqpoint{4.905210in}{1.264985in}}%
\pgfpathlineto{\pgfqpoint{4.926600in}{1.264985in}}%
\pgfpathlineto{\pgfqpoint{4.947990in}{1.264985in}}%
\pgfpathlineto{\pgfqpoint{4.969380in}{1.264985in}}%
\pgfpathlineto{\pgfqpoint{4.990770in}{1.264985in}}%
\pgfpathlineto{\pgfqpoint{5.012160in}{1.264985in}}%
\pgfpathlineto{\pgfqpoint{5.033550in}{1.211731in}}%
\pgfpathlineto{\pgfqpoint{5.054940in}{1.211731in}}%
\pgfpathlineto{\pgfqpoint{5.076330in}{1.211731in}}%
\pgfpathlineto{\pgfqpoint{5.097720in}{1.211731in}}%
\pgfpathlineto{\pgfqpoint{5.119110in}{1.211731in}}%
\pgfpathlineto{\pgfqpoint{5.140500in}{1.211731in}}%
\pgfpathlineto{\pgfqpoint{5.161890in}{1.211731in}}%
\pgfpathlineto{\pgfqpoint{5.183280in}{1.211731in}}%
\pgfpathlineto{\pgfqpoint{5.204670in}{1.211731in}}%
\pgfpathlineto{\pgfqpoint{5.226060in}{1.211731in}}%
\pgfpathlineto{\pgfqpoint{5.247450in}{1.190129in}}%
\pgfpathlineto{\pgfqpoint{5.268840in}{1.190129in}}%
\pgfpathlineto{\pgfqpoint{5.290230in}{1.190129in}}%
\pgfpathlineto{\pgfqpoint{5.311620in}{1.155779in}}%
\pgfpathlineto{\pgfqpoint{5.333010in}{1.155779in}}%
\pgfpathlineto{\pgfqpoint{5.354400in}{1.155779in}}%
\pgfpathlineto{\pgfqpoint{5.375790in}{1.155779in}}%
\pgfpathlineto{\pgfqpoint{5.397180in}{1.155779in}}%
\pgfpathlineto{\pgfqpoint{5.418570in}{1.155779in}}%
\pgfpathlineto{\pgfqpoint{5.439960in}{1.155779in}}%
\pgfpathlineto{\pgfqpoint{5.461350in}{1.155779in}}%
\pgfpathlineto{\pgfqpoint{5.482740in}{1.155779in}}%
\pgfpathlineto{\pgfqpoint{5.504130in}{1.155779in}}%
\pgfpathlineto{\pgfqpoint{5.525520in}{1.155779in}}%
\pgfpathlineto{\pgfqpoint{5.546910in}{1.155779in}}%
\pgfpathlineto{\pgfqpoint{5.568300in}{1.155779in}}%
\pgfpathlineto{\pgfqpoint{5.589690in}{1.155779in}}%
\pgfpathlineto{\pgfqpoint{5.611080in}{1.155779in}}%
\pgfpathlineto{\pgfqpoint{5.632470in}{1.155779in}}%
\pgfpathlineto{\pgfqpoint{5.653860in}{1.155779in}}%
\pgfpathlineto{\pgfqpoint{5.675250in}{1.155779in}}%
\pgfpathlineto{\pgfqpoint{5.696640in}{1.155779in}}%
\pgfpathlineto{\pgfqpoint{5.718030in}{1.155779in}}%
\pgfpathlineto{\pgfqpoint{5.739420in}{1.155779in}}%
\pgfpathlineto{\pgfqpoint{5.760810in}{1.155779in}}%
\pgfpathlineto{\pgfqpoint{5.782200in}{1.155779in}}%
\pgfpathlineto{\pgfqpoint{5.803590in}{1.155779in}}%
\pgfpathlineto{\pgfqpoint{5.824980in}{1.155779in}}%
\pgfpathlineto{\pgfqpoint{5.846370in}{1.152558in}}%
\pgfpathlineto{\pgfqpoint{5.867760in}{1.152558in}}%
\pgfpathlineto{\pgfqpoint{5.889150in}{1.152558in}}%
\pgfpathlineto{\pgfqpoint{5.910540in}{1.152558in}}%
\pgfpathlineto{\pgfqpoint{5.931930in}{1.150343in}}%
\pgfpathlineto{\pgfqpoint{5.953320in}{1.150343in}}%
\pgfpathlineto{\pgfqpoint{5.974710in}{1.150343in}}%
\pgfpathlineto{\pgfqpoint{5.996100in}{1.150343in}}%
\pgfpathlineto{\pgfqpoint{6.017490in}{1.150343in}}%
\pgfpathlineto{\pgfqpoint{6.038880in}{1.150343in}}%
\pgfpathlineto{\pgfqpoint{6.060270in}{1.150343in}}%
\pgfpathlineto{\pgfqpoint{6.081660in}{1.150343in}}%
\pgfpathlineto{\pgfqpoint{6.103050in}{1.150343in}}%
\pgfpathlineto{\pgfqpoint{6.124440in}{1.150343in}}%
\pgfpathlineto{\pgfqpoint{6.145830in}{1.150343in}}%
\pgfpathlineto{\pgfqpoint{6.167220in}{1.150343in}}%
\pgfpathlineto{\pgfqpoint{6.188610in}{1.150343in}}%
\pgfpathlineto{\pgfqpoint{6.188610in}{0.710224in}}%
\pgfpathlineto{\pgfqpoint{6.188610in}{0.710224in}}%
\pgfpathlineto{\pgfqpoint{6.167220in}{0.710224in}}%
\pgfpathlineto{\pgfqpoint{6.145830in}{0.710224in}}%
\pgfpathlineto{\pgfqpoint{6.124440in}{0.710224in}}%
\pgfpathlineto{\pgfqpoint{6.103050in}{0.710224in}}%
\pgfpathlineto{\pgfqpoint{6.081660in}{0.710224in}}%
\pgfpathlineto{\pgfqpoint{6.060270in}{0.710224in}}%
\pgfpathlineto{\pgfqpoint{6.038880in}{0.710224in}}%
\pgfpathlineto{\pgfqpoint{6.017490in}{0.710224in}}%
\pgfpathlineto{\pgfqpoint{5.996100in}{0.710224in}}%
\pgfpathlineto{\pgfqpoint{5.974710in}{0.710224in}}%
\pgfpathlineto{\pgfqpoint{5.953320in}{0.710224in}}%
\pgfpathlineto{\pgfqpoint{5.931930in}{0.710224in}}%
\pgfpathlineto{\pgfqpoint{5.910540in}{0.724982in}}%
\pgfpathlineto{\pgfqpoint{5.889150in}{0.724982in}}%
\pgfpathlineto{\pgfqpoint{5.867760in}{0.724982in}}%
\pgfpathlineto{\pgfqpoint{5.846370in}{0.724982in}}%
\pgfpathlineto{\pgfqpoint{5.824980in}{0.749660in}}%
\pgfpathlineto{\pgfqpoint{5.803590in}{0.749660in}}%
\pgfpathlineto{\pgfqpoint{5.782200in}{0.749660in}}%
\pgfpathlineto{\pgfqpoint{5.760810in}{0.749660in}}%
\pgfpathlineto{\pgfqpoint{5.739420in}{0.749660in}}%
\pgfpathlineto{\pgfqpoint{5.718030in}{0.749660in}}%
\pgfpathlineto{\pgfqpoint{5.696640in}{0.749660in}}%
\pgfpathlineto{\pgfqpoint{5.675250in}{0.749660in}}%
\pgfpathlineto{\pgfqpoint{5.653860in}{0.749660in}}%
\pgfpathlineto{\pgfqpoint{5.632470in}{0.749660in}}%
\pgfpathlineto{\pgfqpoint{5.611080in}{0.749660in}}%
\pgfpathlineto{\pgfqpoint{5.589690in}{0.749660in}}%
\pgfpathlineto{\pgfqpoint{5.568300in}{0.749660in}}%
\pgfpathlineto{\pgfqpoint{5.546910in}{0.749660in}}%
\pgfpathlineto{\pgfqpoint{5.525520in}{0.749660in}}%
\pgfpathlineto{\pgfqpoint{5.504130in}{0.749660in}}%
\pgfpathlineto{\pgfqpoint{5.482740in}{0.749660in}}%
\pgfpathlineto{\pgfqpoint{5.461350in}{0.749660in}}%
\pgfpathlineto{\pgfqpoint{5.439960in}{0.749660in}}%
\pgfpathlineto{\pgfqpoint{5.418570in}{0.749660in}}%
\pgfpathlineto{\pgfqpoint{5.397180in}{0.749660in}}%
\pgfpathlineto{\pgfqpoint{5.375790in}{0.749660in}}%
\pgfpathlineto{\pgfqpoint{5.354400in}{0.749660in}}%
\pgfpathlineto{\pgfqpoint{5.333010in}{0.749660in}}%
\pgfpathlineto{\pgfqpoint{5.311620in}{0.749660in}}%
\pgfpathlineto{\pgfqpoint{5.290230in}{0.883537in}}%
\pgfpathlineto{\pgfqpoint{5.268840in}{0.883537in}}%
\pgfpathlineto{\pgfqpoint{5.247450in}{0.883537in}}%
\pgfpathlineto{\pgfqpoint{5.226060in}{0.965571in}}%
\pgfpathlineto{\pgfqpoint{5.204670in}{0.965571in}}%
\pgfpathlineto{\pgfqpoint{5.183280in}{0.965571in}}%
\pgfpathlineto{\pgfqpoint{5.161890in}{0.965571in}}%
\pgfpathlineto{\pgfqpoint{5.140500in}{0.965571in}}%
\pgfpathlineto{\pgfqpoint{5.119110in}{0.965571in}}%
\pgfpathlineto{\pgfqpoint{5.097720in}{0.965571in}}%
\pgfpathlineto{\pgfqpoint{5.076330in}{0.965571in}}%
\pgfpathlineto{\pgfqpoint{5.054940in}{0.965571in}}%
\pgfpathlineto{\pgfqpoint{5.033550in}{0.965571in}}%
\pgfpathlineto{\pgfqpoint{5.012160in}{1.041252in}}%
\pgfpathlineto{\pgfqpoint{4.990770in}{1.041252in}}%
\pgfpathlineto{\pgfqpoint{4.969380in}{1.041252in}}%
\pgfpathlineto{\pgfqpoint{4.947990in}{1.041252in}}%
\pgfpathlineto{\pgfqpoint{4.926600in}{1.041252in}}%
\pgfpathlineto{\pgfqpoint{4.905210in}{1.041252in}}%
\pgfpathlineto{\pgfqpoint{4.883820in}{1.041252in}}%
\pgfpathlineto{\pgfqpoint{4.862430in}{1.063120in}}%
\pgfpathlineto{\pgfqpoint{4.841040in}{1.063120in}}%
\pgfpathlineto{\pgfqpoint{4.819650in}{1.063120in}}%
\pgfpathlineto{\pgfqpoint{4.798260in}{1.063120in}}%
\pgfpathlineto{\pgfqpoint{4.776870in}{1.063120in}}%
\pgfpathlineto{\pgfqpoint{4.755480in}{1.063120in}}%
\pgfpathlineto{\pgfqpoint{4.734090in}{1.063120in}}%
\pgfpathlineto{\pgfqpoint{4.712700in}{1.063120in}}%
\pgfpathlineto{\pgfqpoint{4.691310in}{1.063120in}}%
\pgfpathlineto{\pgfqpoint{4.669920in}{1.063120in}}%
\pgfpathlineto{\pgfqpoint{4.648530in}{1.063120in}}%
\pgfpathlineto{\pgfqpoint{4.627140in}{1.063120in}}%
\pgfpathlineto{\pgfqpoint{4.605750in}{1.063120in}}%
\pgfpathlineto{\pgfqpoint{4.584360in}{1.063120in}}%
\pgfpathlineto{\pgfqpoint{4.562970in}{1.063120in}}%
\pgfpathlineto{\pgfqpoint{4.541580in}{1.063120in}}%
\pgfpathlineto{\pgfqpoint{4.520190in}{1.063120in}}%
\pgfpathlineto{\pgfqpoint{4.498800in}{1.063120in}}%
\pgfpathlineto{\pgfqpoint{4.477410in}{1.063120in}}%
\pgfpathlineto{\pgfqpoint{4.456020in}{1.063120in}}%
\pgfpathlineto{\pgfqpoint{4.434630in}{1.063120in}}%
\pgfpathlineto{\pgfqpoint{4.413240in}{1.063120in}}%
\pgfpathlineto{\pgfqpoint{4.391850in}{1.063120in}}%
\pgfpathlineto{\pgfqpoint{4.370460in}{1.063120in}}%
\pgfpathlineto{\pgfqpoint{4.349070in}{1.063120in}}%
\pgfpathlineto{\pgfqpoint{4.327680in}{1.063120in}}%
\pgfpathlineto{\pgfqpoint{4.306290in}{1.063120in}}%
\pgfpathlineto{\pgfqpoint{4.284900in}{1.063120in}}%
\pgfpathlineto{\pgfqpoint{4.263510in}{1.063120in}}%
\pgfpathlineto{\pgfqpoint{4.242120in}{1.063120in}}%
\pgfpathlineto{\pgfqpoint{4.220730in}{1.063120in}}%
\pgfpathlineto{\pgfqpoint{4.199340in}{1.063120in}}%
\pgfpathlineto{\pgfqpoint{4.177950in}{1.088862in}}%
\pgfpathlineto{\pgfqpoint{4.156560in}{1.118503in}}%
\pgfpathlineto{\pgfqpoint{4.135170in}{1.118503in}}%
\pgfpathlineto{\pgfqpoint{4.113780in}{1.118503in}}%
\pgfpathlineto{\pgfqpoint{4.092390in}{1.118503in}}%
\pgfpathlineto{\pgfqpoint{4.071000in}{1.118503in}}%
\pgfpathlineto{\pgfqpoint{4.049610in}{1.120983in}}%
\pgfpathlineto{\pgfqpoint{4.028220in}{1.120983in}}%
\pgfpathlineto{\pgfqpoint{4.006830in}{1.120983in}}%
\pgfpathlineto{\pgfqpoint{3.985440in}{1.120983in}}%
\pgfpathlineto{\pgfqpoint{3.964050in}{1.120983in}}%
\pgfpathlineto{\pgfqpoint{3.942660in}{1.120983in}}%
\pgfpathlineto{\pgfqpoint{3.921270in}{1.120983in}}%
\pgfpathlineto{\pgfqpoint{3.899880in}{1.120983in}}%
\pgfpathlineto{\pgfqpoint{3.878490in}{1.120983in}}%
\pgfpathlineto{\pgfqpoint{3.857100in}{1.147156in}}%
\pgfpathlineto{\pgfqpoint{3.835710in}{1.163298in}}%
\pgfpathlineto{\pgfqpoint{3.814320in}{1.163298in}}%
\pgfpathlineto{\pgfqpoint{3.792930in}{1.163298in}}%
\pgfpathlineto{\pgfqpoint{3.771540in}{1.163298in}}%
\pgfpathlineto{\pgfqpoint{3.750150in}{1.163298in}}%
\pgfpathlineto{\pgfqpoint{3.728760in}{1.163298in}}%
\pgfpathlineto{\pgfqpoint{3.707370in}{1.169844in}}%
\pgfpathlineto{\pgfqpoint{3.685980in}{1.169844in}}%
\pgfpathlineto{\pgfqpoint{3.664590in}{1.169844in}}%
\pgfpathlineto{\pgfqpoint{3.643200in}{1.169844in}}%
\pgfpathlineto{\pgfqpoint{3.621810in}{1.201944in}}%
\pgfpathlineto{\pgfqpoint{3.600420in}{1.201944in}}%
\pgfpathlineto{\pgfqpoint{3.579030in}{1.201944in}}%
\pgfpathlineto{\pgfqpoint{3.557640in}{1.201944in}}%
\pgfpathlineto{\pgfqpoint{3.536250in}{1.201944in}}%
\pgfpathlineto{\pgfqpoint{3.514860in}{1.201944in}}%
\pgfpathlineto{\pgfqpoint{3.493470in}{1.201944in}}%
\pgfpathlineto{\pgfqpoint{3.472080in}{1.201944in}}%
\pgfpathlineto{\pgfqpoint{3.450690in}{1.224652in}}%
\pgfpathlineto{\pgfqpoint{3.429300in}{1.224652in}}%
\pgfpathlineto{\pgfqpoint{3.407910in}{1.224652in}}%
\pgfpathlineto{\pgfqpoint{3.386520in}{1.224652in}}%
\pgfpathlineto{\pgfqpoint{3.365130in}{1.224652in}}%
\pgfpathlineto{\pgfqpoint{3.343740in}{1.224652in}}%
\pgfpathlineto{\pgfqpoint{3.322350in}{1.224652in}}%
\pgfpathlineto{\pgfqpoint{3.300960in}{1.224652in}}%
\pgfpathlineto{\pgfqpoint{3.279570in}{1.224652in}}%
\pgfpathlineto{\pgfqpoint{3.258180in}{1.224652in}}%
\pgfpathlineto{\pgfqpoint{3.236790in}{1.224652in}}%
\pgfpathlineto{\pgfqpoint{3.215400in}{1.224652in}}%
\pgfpathlineto{\pgfqpoint{3.194010in}{1.224652in}}%
\pgfpathlineto{\pgfqpoint{3.172620in}{1.224652in}}%
\pgfpathlineto{\pgfqpoint{3.151230in}{1.224652in}}%
\pgfpathlineto{\pgfqpoint{3.129840in}{1.224652in}}%
\pgfpathlineto{\pgfqpoint{3.108450in}{1.224652in}}%
\pgfpathlineto{\pgfqpoint{3.087060in}{1.224652in}}%
\pgfpathlineto{\pgfqpoint{3.065670in}{1.224652in}}%
\pgfpathlineto{\pgfqpoint{3.044280in}{1.224652in}}%
\pgfpathlineto{\pgfqpoint{3.022890in}{1.224652in}}%
\pgfpathlineto{\pgfqpoint{3.001500in}{1.224652in}}%
\pgfpathlineto{\pgfqpoint{2.980110in}{1.234761in}}%
\pgfpathlineto{\pgfqpoint{2.958720in}{1.234761in}}%
\pgfpathlineto{\pgfqpoint{2.937330in}{1.234761in}}%
\pgfpathlineto{\pgfqpoint{2.915940in}{1.234761in}}%
\pgfpathlineto{\pgfqpoint{2.894550in}{1.234761in}}%
\pgfpathlineto{\pgfqpoint{2.873160in}{1.234761in}}%
\pgfpathlineto{\pgfqpoint{2.851770in}{1.234761in}}%
\pgfpathlineto{\pgfqpoint{2.830380in}{1.240893in}}%
\pgfpathlineto{\pgfqpoint{2.808990in}{1.240893in}}%
\pgfpathlineto{\pgfqpoint{2.787600in}{1.240893in}}%
\pgfpathlineto{\pgfqpoint{2.766210in}{1.240893in}}%
\pgfpathlineto{\pgfqpoint{2.744820in}{1.240893in}}%
\pgfpathlineto{\pgfqpoint{2.723430in}{1.244642in}}%
\pgfpathlineto{\pgfqpoint{2.702040in}{1.244642in}}%
\pgfpathlineto{\pgfqpoint{2.680650in}{1.244642in}}%
\pgfpathlineto{\pgfqpoint{2.659260in}{1.244642in}}%
\pgfpathlineto{\pgfqpoint{2.637870in}{1.244642in}}%
\pgfpathlineto{\pgfqpoint{2.616480in}{1.244642in}}%
\pgfpathlineto{\pgfqpoint{2.595090in}{1.244642in}}%
\pgfpathlineto{\pgfqpoint{2.573700in}{1.244642in}}%
\pgfpathlineto{\pgfqpoint{2.552310in}{1.244642in}}%
\pgfpathlineto{\pgfqpoint{2.530920in}{1.244642in}}%
\pgfpathlineto{\pgfqpoint{2.509530in}{1.244642in}}%
\pgfpathlineto{\pgfqpoint{2.488140in}{1.244642in}}%
\pgfpathlineto{\pgfqpoint{2.466750in}{1.251139in}}%
\pgfpathlineto{\pgfqpoint{2.445360in}{1.251139in}}%
\pgfpathlineto{\pgfqpoint{2.423970in}{1.251139in}}%
\pgfpathlineto{\pgfqpoint{2.402580in}{1.251139in}}%
\pgfpathlineto{\pgfqpoint{2.381190in}{1.251139in}}%
\pgfpathlineto{\pgfqpoint{2.359800in}{1.377902in}}%
\pgfpathlineto{\pgfqpoint{2.338410in}{1.377902in}}%
\pgfpathlineto{\pgfqpoint{2.317020in}{1.377902in}}%
\pgfpathlineto{\pgfqpoint{2.295630in}{1.377902in}}%
\pgfpathlineto{\pgfqpoint{2.274240in}{1.377902in}}%
\pgfpathlineto{\pgfqpoint{2.252850in}{1.426005in}}%
\pgfpathlineto{\pgfqpoint{2.231460in}{1.434050in}}%
\pgfpathlineto{\pgfqpoint{2.210070in}{1.443124in}}%
\pgfpathlineto{\pgfqpoint{2.188680in}{1.447856in}}%
\pgfpathlineto{\pgfqpoint{2.167290in}{1.447856in}}%
\pgfpathlineto{\pgfqpoint{2.145900in}{1.448485in}}%
\pgfpathlineto{\pgfqpoint{2.124510in}{1.448485in}}%
\pgfpathlineto{\pgfqpoint{2.103120in}{1.448485in}}%
\pgfpathlineto{\pgfqpoint{2.081730in}{1.498250in}}%
\pgfpathlineto{\pgfqpoint{2.060340in}{1.498250in}}%
\pgfpathlineto{\pgfqpoint{2.038950in}{1.498250in}}%
\pgfpathlineto{\pgfqpoint{2.017560in}{1.564451in}}%
\pgfpathlineto{\pgfqpoint{1.996170in}{1.564451in}}%
\pgfpathlineto{\pgfqpoint{1.974780in}{1.564451in}}%
\pgfpathlineto{\pgfqpoint{1.953390in}{1.643514in}}%
\pgfpathlineto{\pgfqpoint{1.932000in}{1.658392in}}%
\pgfpathlineto{\pgfqpoint{1.910610in}{1.707332in}}%
\pgfpathlineto{\pgfqpoint{1.889220in}{1.707332in}}%
\pgfpathlineto{\pgfqpoint{1.867830in}{1.707332in}}%
\pgfpathlineto{\pgfqpoint{1.846440in}{1.707332in}}%
\pgfpathlineto{\pgfqpoint{1.825050in}{1.707332in}}%
\pgfpathlineto{\pgfqpoint{1.803660in}{1.707332in}}%
\pgfpathlineto{\pgfqpoint{1.782270in}{1.748809in}}%
\pgfpathlineto{\pgfqpoint{1.760880in}{1.748809in}}%
\pgfpathlineto{\pgfqpoint{1.739490in}{1.748809in}}%
\pgfpathlineto{\pgfqpoint{1.718100in}{1.748809in}}%
\pgfpathlineto{\pgfqpoint{1.696710in}{1.748809in}}%
\pgfpathlineto{\pgfqpoint{1.675320in}{1.748809in}}%
\pgfpathlineto{\pgfqpoint{1.653930in}{1.748809in}}%
\pgfpathlineto{\pgfqpoint{1.632540in}{1.845410in}}%
\pgfpathlineto{\pgfqpoint{1.611150in}{1.845410in}}%
\pgfpathlineto{\pgfqpoint{1.589760in}{1.845410in}}%
\pgfpathlineto{\pgfqpoint{1.568370in}{1.975933in}}%
\pgfpathlineto{\pgfqpoint{1.546980in}{1.975933in}}%
\pgfpathlineto{\pgfqpoint{1.525590in}{1.975933in}}%
\pgfpathlineto{\pgfqpoint{1.504200in}{1.975933in}}%
\pgfpathlineto{\pgfqpoint{1.482810in}{1.975933in}}%
\pgfpathlineto{\pgfqpoint{1.461420in}{1.975933in}}%
\pgfpathlineto{\pgfqpoint{1.440030in}{1.975933in}}%
\pgfpathlineto{\pgfqpoint{1.418640in}{1.975933in}}%
\pgfpathlineto{\pgfqpoint{1.397250in}{2.009408in}}%
\pgfpathlineto{\pgfqpoint{1.375860in}{2.009408in}}%
\pgfpathlineto{\pgfqpoint{1.354470in}{2.052077in}}%
\pgfpathlineto{\pgfqpoint{1.333080in}{2.052077in}}%
\pgfpathlineto{\pgfqpoint{1.311690in}{2.052077in}}%
\pgfpathlineto{\pgfqpoint{1.290300in}{2.073764in}}%
\pgfpathlineto{\pgfqpoint{1.268910in}{2.073764in}}%
\pgfpathlineto{\pgfqpoint{1.247520in}{2.073764in}}%
\pgfpathlineto{\pgfqpoint{1.226130in}{2.073764in}}%
\pgfpathlineto{\pgfqpoint{1.204740in}{2.093973in}}%
\pgfpathlineto{\pgfqpoint{1.183350in}{2.093973in}}%
\pgfpathlineto{\pgfqpoint{1.161960in}{2.093973in}}%
\pgfpathlineto{\pgfqpoint{1.140570in}{2.093973in}}%
\pgfpathlineto{\pgfqpoint{1.119180in}{2.098445in}}%
\pgfpathlineto{\pgfqpoint{1.097790in}{2.098445in}}%
\pgfpathlineto{\pgfqpoint{1.076400in}{2.098445in}}%
\pgfpathlineto{\pgfqpoint{1.055010in}{2.098445in}}%
\pgfpathlineto{\pgfqpoint{1.033620in}{2.098445in}}%
\pgfpathlineto{\pgfqpoint{1.012230in}{2.098445in}}%
\pgfpathlineto{\pgfqpoint{0.990840in}{2.115601in}}%
\pgfpathlineto{\pgfqpoint{0.969450in}{2.146673in}}%
\pgfpathlineto{\pgfqpoint{0.948060in}{2.205183in}}%
\pgfpathlineto{\pgfqpoint{0.926670in}{2.324532in}}%
\pgfpathlineto{\pgfqpoint{0.905280in}{2.324532in}}%
\pgfpathlineto{\pgfqpoint{0.883890in}{2.332211in}}%
\pgfpathlineto{\pgfqpoint{0.862500in}{2.351316in}}%
\pgfpathclose%
\pgfusepath{fill}%
\end{pgfscope}%
\begin{pgfscope}%
\pgfpathrectangle{\pgfqpoint{0.862500in}{0.375000in}}{\pgfqpoint{5.347500in}{2.265000in}}%
\pgfusepath{clip}%
\pgfsetbuttcap%
\pgfsetroundjoin%
\definecolor{currentfill}{rgb}{0.839216,0.152941,0.156863}%
\pgfsetfillcolor{currentfill}%
\pgfsetfillopacity{0.200000}%
\pgfsetlinewidth{0.000000pt}%
\definecolor{currentstroke}{rgb}{0.000000,0.000000,0.000000}%
\pgfsetstrokecolor{currentstroke}%
\pgfsetdash{}{0pt}%
\pgfpathmoveto{\pgfqpoint{0.862500in}{2.442793in}}%
\pgfpathlineto{\pgfqpoint{0.862500in}{2.530266in}}%
\pgfpathlineto{\pgfqpoint{0.883890in}{2.469540in}}%
\pgfpathlineto{\pgfqpoint{0.905280in}{2.468663in}}%
\pgfpathlineto{\pgfqpoint{0.926670in}{2.415743in}}%
\pgfpathlineto{\pgfqpoint{0.948060in}{2.415743in}}%
\pgfpathlineto{\pgfqpoint{0.969450in}{2.356711in}}%
\pgfpathlineto{\pgfqpoint{0.990840in}{2.356711in}}%
\pgfpathlineto{\pgfqpoint{1.012230in}{2.356711in}}%
\pgfpathlineto{\pgfqpoint{1.033620in}{2.335931in}}%
\pgfpathlineto{\pgfqpoint{1.055010in}{2.335931in}}%
\pgfpathlineto{\pgfqpoint{1.076400in}{2.332395in}}%
\pgfpathlineto{\pgfqpoint{1.097790in}{2.299643in}}%
\pgfpathlineto{\pgfqpoint{1.119180in}{2.130998in}}%
\pgfpathlineto{\pgfqpoint{1.140570in}{2.130998in}}%
\pgfpathlineto{\pgfqpoint{1.161960in}{2.130998in}}%
\pgfpathlineto{\pgfqpoint{1.183350in}{2.130998in}}%
\pgfpathlineto{\pgfqpoint{1.204740in}{2.130998in}}%
\pgfpathlineto{\pgfqpoint{1.226130in}{2.130998in}}%
\pgfpathlineto{\pgfqpoint{1.247520in}{2.130998in}}%
\pgfpathlineto{\pgfqpoint{1.268910in}{2.097294in}}%
\pgfpathlineto{\pgfqpoint{1.290300in}{2.097294in}}%
\pgfpathlineto{\pgfqpoint{1.311690in}{2.097294in}}%
\pgfpathlineto{\pgfqpoint{1.333080in}{2.094940in}}%
\pgfpathlineto{\pgfqpoint{1.354470in}{2.094940in}}%
\pgfpathlineto{\pgfqpoint{1.375860in}{2.094940in}}%
\pgfpathlineto{\pgfqpoint{1.397250in}{2.020622in}}%
\pgfpathlineto{\pgfqpoint{1.418640in}{1.988336in}}%
\pgfpathlineto{\pgfqpoint{1.440030in}{1.988336in}}%
\pgfpathlineto{\pgfqpoint{1.461420in}{1.931654in}}%
\pgfpathlineto{\pgfqpoint{1.482810in}{1.931654in}}%
\pgfpathlineto{\pgfqpoint{1.504200in}{1.931654in}}%
\pgfpathlineto{\pgfqpoint{1.525590in}{1.931654in}}%
\pgfpathlineto{\pgfqpoint{1.546980in}{1.931654in}}%
\pgfpathlineto{\pgfqpoint{1.568370in}{1.931654in}}%
\pgfpathlineto{\pgfqpoint{1.589760in}{1.931654in}}%
\pgfpathlineto{\pgfqpoint{1.611150in}{1.859713in}}%
\pgfpathlineto{\pgfqpoint{1.632540in}{1.859713in}}%
\pgfpathlineto{\pgfqpoint{1.653930in}{1.859713in}}%
\pgfpathlineto{\pgfqpoint{1.675320in}{1.859713in}}%
\pgfpathlineto{\pgfqpoint{1.696710in}{1.859713in}}%
\pgfpathlineto{\pgfqpoint{1.718100in}{1.859713in}}%
\pgfpathlineto{\pgfqpoint{1.739490in}{1.859713in}}%
\pgfpathlineto{\pgfqpoint{1.760880in}{1.859713in}}%
\pgfpathlineto{\pgfqpoint{1.782270in}{1.858314in}}%
\pgfpathlineto{\pgfqpoint{1.803660in}{1.858314in}}%
\pgfpathlineto{\pgfqpoint{1.825050in}{1.858314in}}%
\pgfpathlineto{\pgfqpoint{1.846440in}{1.858314in}}%
\pgfpathlineto{\pgfqpoint{1.867830in}{1.858314in}}%
\pgfpathlineto{\pgfqpoint{1.889220in}{1.858314in}}%
\pgfpathlineto{\pgfqpoint{1.910610in}{1.858314in}}%
\pgfpathlineto{\pgfqpoint{1.932000in}{1.858314in}}%
\pgfpathlineto{\pgfqpoint{1.953390in}{1.858314in}}%
\pgfpathlineto{\pgfqpoint{1.974780in}{1.858314in}}%
\pgfpathlineto{\pgfqpoint{1.996170in}{1.858314in}}%
\pgfpathlineto{\pgfqpoint{2.017560in}{1.858314in}}%
\pgfpathlineto{\pgfqpoint{2.038950in}{1.858314in}}%
\pgfpathlineto{\pgfqpoint{2.060340in}{1.858314in}}%
\pgfpathlineto{\pgfqpoint{2.081730in}{1.858314in}}%
\pgfpathlineto{\pgfqpoint{2.103120in}{1.858314in}}%
\pgfpathlineto{\pgfqpoint{2.124510in}{1.858314in}}%
\pgfpathlineto{\pgfqpoint{2.145900in}{1.858314in}}%
\pgfpathlineto{\pgfqpoint{2.167290in}{1.858314in}}%
\pgfpathlineto{\pgfqpoint{2.188680in}{1.858314in}}%
\pgfpathlineto{\pgfqpoint{2.210070in}{1.858314in}}%
\pgfpathlineto{\pgfqpoint{2.231460in}{1.858314in}}%
\pgfpathlineto{\pgfqpoint{2.252850in}{1.858314in}}%
\pgfpathlineto{\pgfqpoint{2.274240in}{1.858314in}}%
\pgfpathlineto{\pgfqpoint{2.295630in}{1.858314in}}%
\pgfpathlineto{\pgfqpoint{2.317020in}{1.858314in}}%
\pgfpathlineto{\pgfqpoint{2.338410in}{1.858314in}}%
\pgfpathlineto{\pgfqpoint{2.359800in}{1.858314in}}%
\pgfpathlineto{\pgfqpoint{2.381190in}{1.858314in}}%
\pgfpathlineto{\pgfqpoint{2.402580in}{1.858314in}}%
\pgfpathlineto{\pgfqpoint{2.423970in}{1.858314in}}%
\pgfpathlineto{\pgfqpoint{2.445360in}{1.858314in}}%
\pgfpathlineto{\pgfqpoint{2.466750in}{1.858314in}}%
\pgfpathlineto{\pgfqpoint{2.488140in}{1.858314in}}%
\pgfpathlineto{\pgfqpoint{2.509530in}{1.858314in}}%
\pgfpathlineto{\pgfqpoint{2.530920in}{1.858314in}}%
\pgfpathlineto{\pgfqpoint{2.552310in}{1.858314in}}%
\pgfpathlineto{\pgfqpoint{2.573700in}{1.858314in}}%
\pgfpathlineto{\pgfqpoint{2.595090in}{1.858314in}}%
\pgfpathlineto{\pgfqpoint{2.616480in}{1.858314in}}%
\pgfpathlineto{\pgfqpoint{2.637870in}{1.858314in}}%
\pgfpathlineto{\pgfqpoint{2.659260in}{1.858314in}}%
\pgfpathlineto{\pgfqpoint{2.680650in}{1.858314in}}%
\pgfpathlineto{\pgfqpoint{2.702040in}{1.858314in}}%
\pgfpathlineto{\pgfqpoint{2.723430in}{1.858314in}}%
\pgfpathlineto{\pgfqpoint{2.744820in}{1.858314in}}%
\pgfpathlineto{\pgfqpoint{2.766210in}{1.858314in}}%
\pgfpathlineto{\pgfqpoint{2.787600in}{1.858314in}}%
\pgfpathlineto{\pgfqpoint{2.808990in}{1.858314in}}%
\pgfpathlineto{\pgfqpoint{2.830380in}{1.858314in}}%
\pgfpathlineto{\pgfqpoint{2.851770in}{1.858314in}}%
\pgfpathlineto{\pgfqpoint{2.873160in}{1.858314in}}%
\pgfpathlineto{\pgfqpoint{2.894550in}{1.858314in}}%
\pgfpathlineto{\pgfqpoint{2.915940in}{1.858314in}}%
\pgfpathlineto{\pgfqpoint{2.937330in}{1.858314in}}%
\pgfpathlineto{\pgfqpoint{2.958720in}{1.858314in}}%
\pgfpathlineto{\pgfqpoint{2.980110in}{1.858314in}}%
\pgfpathlineto{\pgfqpoint{3.001500in}{1.689817in}}%
\pgfpathlineto{\pgfqpoint{3.022890in}{1.588201in}}%
\pgfpathlineto{\pgfqpoint{3.044280in}{1.588201in}}%
\pgfpathlineto{\pgfqpoint{3.065670in}{1.588201in}}%
\pgfpathlineto{\pgfqpoint{3.087060in}{1.588201in}}%
\pgfpathlineto{\pgfqpoint{3.108450in}{1.555834in}}%
\pgfpathlineto{\pgfqpoint{3.129840in}{1.555834in}}%
\pgfpathlineto{\pgfqpoint{3.151230in}{1.555834in}}%
\pgfpathlineto{\pgfqpoint{3.172620in}{1.540573in}}%
\pgfpathlineto{\pgfqpoint{3.194010in}{1.502989in}}%
\pgfpathlineto{\pgfqpoint{3.215400in}{1.502989in}}%
\pgfpathlineto{\pgfqpoint{3.236790in}{1.502989in}}%
\pgfpathlineto{\pgfqpoint{3.258180in}{1.502989in}}%
\pgfpathlineto{\pgfqpoint{3.279570in}{1.502989in}}%
\pgfpathlineto{\pgfqpoint{3.300960in}{1.501361in}}%
\pgfpathlineto{\pgfqpoint{3.322350in}{1.501361in}}%
\pgfpathlineto{\pgfqpoint{3.343740in}{1.499104in}}%
\pgfpathlineto{\pgfqpoint{3.365130in}{1.499104in}}%
\pgfpathlineto{\pgfqpoint{3.386520in}{1.499104in}}%
\pgfpathlineto{\pgfqpoint{3.407910in}{1.499104in}}%
\pgfpathlineto{\pgfqpoint{3.429300in}{1.497210in}}%
\pgfpathlineto{\pgfqpoint{3.450690in}{1.497210in}}%
\pgfpathlineto{\pgfqpoint{3.472080in}{1.497210in}}%
\pgfpathlineto{\pgfqpoint{3.493470in}{1.497210in}}%
\pgfpathlineto{\pgfqpoint{3.514860in}{1.484649in}}%
\pgfpathlineto{\pgfqpoint{3.536250in}{1.484649in}}%
\pgfpathlineto{\pgfqpoint{3.557640in}{1.484649in}}%
\pgfpathlineto{\pgfqpoint{3.579030in}{1.484649in}}%
\pgfpathlineto{\pgfqpoint{3.600420in}{1.484649in}}%
\pgfpathlineto{\pgfqpoint{3.621810in}{1.484649in}}%
\pgfpathlineto{\pgfqpoint{3.643200in}{1.483273in}}%
\pgfpathlineto{\pgfqpoint{3.664590in}{1.483273in}}%
\pgfpathlineto{\pgfqpoint{3.685980in}{1.478328in}}%
\pgfpathlineto{\pgfqpoint{3.707370in}{1.476053in}}%
\pgfpathlineto{\pgfqpoint{3.728760in}{1.476053in}}%
\pgfpathlineto{\pgfqpoint{3.750150in}{1.476053in}}%
\pgfpathlineto{\pgfqpoint{3.771540in}{1.476053in}}%
\pgfpathlineto{\pgfqpoint{3.792930in}{1.476053in}}%
\pgfpathlineto{\pgfqpoint{3.814320in}{1.451079in}}%
\pgfpathlineto{\pgfqpoint{3.835710in}{1.451079in}}%
\pgfpathlineto{\pgfqpoint{3.857100in}{1.429650in}}%
\pgfpathlineto{\pgfqpoint{3.878490in}{1.429650in}}%
\pgfpathlineto{\pgfqpoint{3.899880in}{1.429650in}}%
\pgfpathlineto{\pgfqpoint{3.921270in}{1.429650in}}%
\pgfpathlineto{\pgfqpoint{3.942660in}{1.429650in}}%
\pgfpathlineto{\pgfqpoint{3.964050in}{1.429650in}}%
\pgfpathlineto{\pgfqpoint{3.985440in}{1.429650in}}%
\pgfpathlineto{\pgfqpoint{4.006830in}{1.429650in}}%
\pgfpathlineto{\pgfqpoint{4.028220in}{1.429650in}}%
\pgfpathlineto{\pgfqpoint{4.049610in}{1.429650in}}%
\pgfpathlineto{\pgfqpoint{4.071000in}{1.429650in}}%
\pgfpathlineto{\pgfqpoint{4.092390in}{1.429650in}}%
\pgfpathlineto{\pgfqpoint{4.113780in}{1.400616in}}%
\pgfpathlineto{\pgfqpoint{4.135170in}{1.392927in}}%
\pgfpathlineto{\pgfqpoint{4.156560in}{1.392927in}}%
\pgfpathlineto{\pgfqpoint{4.177950in}{1.392927in}}%
\pgfpathlineto{\pgfqpoint{4.199340in}{1.392927in}}%
\pgfpathlineto{\pgfqpoint{4.220730in}{1.392927in}}%
\pgfpathlineto{\pgfqpoint{4.242120in}{1.392927in}}%
\pgfpathlineto{\pgfqpoint{4.263510in}{1.392927in}}%
\pgfpathlineto{\pgfqpoint{4.284900in}{1.392927in}}%
\pgfpathlineto{\pgfqpoint{4.306290in}{1.392927in}}%
\pgfpathlineto{\pgfqpoint{4.327680in}{1.392927in}}%
\pgfpathlineto{\pgfqpoint{4.349070in}{1.392927in}}%
\pgfpathlineto{\pgfqpoint{4.370460in}{1.392927in}}%
\pgfpathlineto{\pgfqpoint{4.391850in}{1.392927in}}%
\pgfpathlineto{\pgfqpoint{4.413240in}{1.382842in}}%
\pgfpathlineto{\pgfqpoint{4.434630in}{1.382842in}}%
\pgfpathlineto{\pgfqpoint{4.456020in}{1.382842in}}%
\pgfpathlineto{\pgfqpoint{4.477410in}{1.382842in}}%
\pgfpathlineto{\pgfqpoint{4.498800in}{1.382842in}}%
\pgfpathlineto{\pgfqpoint{4.520190in}{1.382842in}}%
\pgfpathlineto{\pgfqpoint{4.541580in}{1.382842in}}%
\pgfpathlineto{\pgfqpoint{4.562970in}{1.382842in}}%
\pgfpathlineto{\pgfqpoint{4.584360in}{1.382842in}}%
\pgfpathlineto{\pgfqpoint{4.605750in}{1.382842in}}%
\pgfpathlineto{\pgfqpoint{4.627140in}{1.382842in}}%
\pgfpathlineto{\pgfqpoint{4.648530in}{1.382842in}}%
\pgfpathlineto{\pgfqpoint{4.669920in}{1.382842in}}%
\pgfpathlineto{\pgfqpoint{4.691310in}{1.382842in}}%
\pgfpathlineto{\pgfqpoint{4.712700in}{1.382842in}}%
\pgfpathlineto{\pgfqpoint{4.734090in}{1.382842in}}%
\pgfpathlineto{\pgfqpoint{4.755480in}{1.382842in}}%
\pgfpathlineto{\pgfqpoint{4.776870in}{1.382842in}}%
\pgfpathlineto{\pgfqpoint{4.798260in}{1.382842in}}%
\pgfpathlineto{\pgfqpoint{4.819650in}{1.369057in}}%
\pgfpathlineto{\pgfqpoint{4.841040in}{1.369057in}}%
\pgfpathlineto{\pgfqpoint{4.862430in}{1.369057in}}%
\pgfpathlineto{\pgfqpoint{4.883820in}{1.369057in}}%
\pgfpathlineto{\pgfqpoint{4.905210in}{1.369057in}}%
\pgfpathlineto{\pgfqpoint{4.926600in}{1.369057in}}%
\pgfpathlineto{\pgfqpoint{4.947990in}{1.369057in}}%
\pgfpathlineto{\pgfqpoint{4.969380in}{1.369057in}}%
\pgfpathlineto{\pgfqpoint{4.990770in}{1.369057in}}%
\pgfpathlineto{\pgfqpoint{5.012160in}{1.369057in}}%
\pgfpathlineto{\pgfqpoint{5.033550in}{1.369057in}}%
\pgfpathlineto{\pgfqpoint{5.054940in}{1.369057in}}%
\pgfpathlineto{\pgfqpoint{5.076330in}{1.369057in}}%
\pgfpathlineto{\pgfqpoint{5.097720in}{1.369057in}}%
\pgfpathlineto{\pgfqpoint{5.119110in}{1.369057in}}%
\pgfpathlineto{\pgfqpoint{5.140500in}{1.369057in}}%
\pgfpathlineto{\pgfqpoint{5.161890in}{1.369057in}}%
\pgfpathlineto{\pgfqpoint{5.183280in}{1.369057in}}%
\pgfpathlineto{\pgfqpoint{5.204670in}{1.369057in}}%
\pgfpathlineto{\pgfqpoint{5.226060in}{1.369057in}}%
\pgfpathlineto{\pgfqpoint{5.247450in}{1.369057in}}%
\pgfpathlineto{\pgfqpoint{5.268840in}{1.367617in}}%
\pgfpathlineto{\pgfqpoint{5.290230in}{1.367617in}}%
\pgfpathlineto{\pgfqpoint{5.311620in}{1.367617in}}%
\pgfpathlineto{\pgfqpoint{5.333010in}{1.367617in}}%
\pgfpathlineto{\pgfqpoint{5.354400in}{1.367617in}}%
\pgfpathlineto{\pgfqpoint{5.375790in}{1.367617in}}%
\pgfpathlineto{\pgfqpoint{5.397180in}{1.367617in}}%
\pgfpathlineto{\pgfqpoint{5.418570in}{1.367617in}}%
\pgfpathlineto{\pgfqpoint{5.439960in}{1.367617in}}%
\pgfpathlineto{\pgfqpoint{5.461350in}{1.367617in}}%
\pgfpathlineto{\pgfqpoint{5.482740in}{1.367617in}}%
\pgfpathlineto{\pgfqpoint{5.504130in}{1.367617in}}%
\pgfpathlineto{\pgfqpoint{5.525520in}{1.367617in}}%
\pgfpathlineto{\pgfqpoint{5.546910in}{1.367617in}}%
\pgfpathlineto{\pgfqpoint{5.568300in}{1.367617in}}%
\pgfpathlineto{\pgfqpoint{5.589690in}{1.367617in}}%
\pgfpathlineto{\pgfqpoint{5.611080in}{1.367617in}}%
\pgfpathlineto{\pgfqpoint{5.632470in}{1.367617in}}%
\pgfpathlineto{\pgfqpoint{5.653860in}{1.367617in}}%
\pgfpathlineto{\pgfqpoint{5.675250in}{1.367617in}}%
\pgfpathlineto{\pgfqpoint{5.696640in}{1.367617in}}%
\pgfpathlineto{\pgfqpoint{5.718030in}{1.367617in}}%
\pgfpathlineto{\pgfqpoint{5.739420in}{1.367617in}}%
\pgfpathlineto{\pgfqpoint{5.760810in}{1.367617in}}%
\pgfpathlineto{\pgfqpoint{5.782200in}{1.367617in}}%
\pgfpathlineto{\pgfqpoint{5.803590in}{1.367617in}}%
\pgfpathlineto{\pgfqpoint{5.824980in}{1.367617in}}%
\pgfpathlineto{\pgfqpoint{5.846370in}{1.367617in}}%
\pgfpathlineto{\pgfqpoint{5.867760in}{1.367617in}}%
\pgfpathlineto{\pgfqpoint{5.889150in}{1.367617in}}%
\pgfpathlineto{\pgfqpoint{5.910540in}{1.367617in}}%
\pgfpathlineto{\pgfqpoint{5.931930in}{1.367617in}}%
\pgfpathlineto{\pgfqpoint{5.953320in}{1.367617in}}%
\pgfpathlineto{\pgfqpoint{5.974710in}{1.367617in}}%
\pgfpathlineto{\pgfqpoint{5.996100in}{1.367617in}}%
\pgfpathlineto{\pgfqpoint{6.017490in}{1.367617in}}%
\pgfpathlineto{\pgfqpoint{6.038880in}{1.367617in}}%
\pgfpathlineto{\pgfqpoint{6.060270in}{1.367617in}}%
\pgfpathlineto{\pgfqpoint{6.081660in}{1.367617in}}%
\pgfpathlineto{\pgfqpoint{6.103050in}{1.367617in}}%
\pgfpathlineto{\pgfqpoint{6.124440in}{1.367617in}}%
\pgfpathlineto{\pgfqpoint{6.145830in}{1.367617in}}%
\pgfpathlineto{\pgfqpoint{6.167220in}{1.367617in}}%
\pgfpathlineto{\pgfqpoint{6.188610in}{1.367617in}}%
\pgfpathlineto{\pgfqpoint{6.210000in}{1.367617in}}%
\pgfpathlineto{\pgfqpoint{6.231390in}{1.367617in}}%
\pgfpathlineto{\pgfqpoint{6.252780in}{1.367617in}}%
\pgfpathlineto{\pgfqpoint{6.274170in}{1.367617in}}%
\pgfpathlineto{\pgfqpoint{6.295560in}{1.367617in}}%
\pgfpathlineto{\pgfqpoint{6.316950in}{1.367617in}}%
\pgfpathlineto{\pgfqpoint{6.338340in}{1.367617in}}%
\pgfpathlineto{\pgfqpoint{6.359730in}{1.367617in}}%
\pgfpathlineto{\pgfqpoint{6.381120in}{1.367617in}}%
\pgfpathlineto{\pgfqpoint{6.402510in}{1.367617in}}%
\pgfpathlineto{\pgfqpoint{6.423900in}{1.367617in}}%
\pgfpathlineto{\pgfqpoint{6.445290in}{1.367617in}}%
\pgfpathlineto{\pgfqpoint{6.466680in}{1.367617in}}%
\pgfpathlineto{\pgfqpoint{6.488070in}{1.367617in}}%
\pgfpathlineto{\pgfqpoint{6.509460in}{1.367617in}}%
\pgfpathlineto{\pgfqpoint{6.530850in}{1.367617in}}%
\pgfpathlineto{\pgfqpoint{6.552240in}{1.367617in}}%
\pgfpathlineto{\pgfqpoint{6.573630in}{1.367617in}}%
\pgfpathlineto{\pgfqpoint{6.595020in}{1.367617in}}%
\pgfpathlineto{\pgfqpoint{6.616410in}{1.367617in}}%
\pgfpathlineto{\pgfqpoint{6.637800in}{1.367617in}}%
\pgfpathlineto{\pgfqpoint{6.659190in}{1.367617in}}%
\pgfpathlineto{\pgfqpoint{6.680580in}{1.367148in}}%
\pgfpathlineto{\pgfqpoint{6.701970in}{1.359459in}}%
\pgfpathlineto{\pgfqpoint{6.723360in}{1.359459in}}%
\pgfpathlineto{\pgfqpoint{6.744750in}{1.359459in}}%
\pgfpathlineto{\pgfqpoint{6.766140in}{1.359459in}}%
\pgfpathlineto{\pgfqpoint{6.787530in}{1.359459in}}%
\pgfpathlineto{\pgfqpoint{6.808920in}{1.359459in}}%
\pgfpathlineto{\pgfqpoint{6.830310in}{1.359459in}}%
\pgfpathlineto{\pgfqpoint{6.851700in}{1.359459in}}%
\pgfpathlineto{\pgfqpoint{6.873090in}{1.359459in}}%
\pgfpathlineto{\pgfqpoint{6.894480in}{1.359459in}}%
\pgfpathlineto{\pgfqpoint{6.915870in}{1.359459in}}%
\pgfpathlineto{\pgfqpoint{6.937260in}{1.359459in}}%
\pgfpathlineto{\pgfqpoint{6.958650in}{1.359459in}}%
\pgfpathlineto{\pgfqpoint{6.980040in}{1.359459in}}%
\pgfpathlineto{\pgfqpoint{7.001430in}{1.359459in}}%
\pgfpathlineto{\pgfqpoint{7.022820in}{1.350467in}}%
\pgfpathlineto{\pgfqpoint{7.044210in}{1.350467in}}%
\pgfpathlineto{\pgfqpoint{7.065600in}{1.350467in}}%
\pgfpathlineto{\pgfqpoint{7.086990in}{1.350467in}}%
\pgfpathlineto{\pgfqpoint{7.108380in}{1.350467in}}%
\pgfpathlineto{\pgfqpoint{7.129770in}{1.350467in}}%
\pgfpathlineto{\pgfqpoint{7.151160in}{1.350467in}}%
\pgfpathlineto{\pgfqpoint{7.172550in}{1.350467in}}%
\pgfpathlineto{\pgfqpoint{7.193940in}{1.350467in}}%
\pgfpathlineto{\pgfqpoint{7.215330in}{1.350467in}}%
\pgfpathlineto{\pgfqpoint{7.236720in}{1.350467in}}%
\pgfpathlineto{\pgfqpoint{7.258110in}{1.350467in}}%
\pgfpathlineto{\pgfqpoint{7.258110in}{0.859325in}}%
\pgfpathlineto{\pgfqpoint{7.258110in}{0.859325in}}%
\pgfpathlineto{\pgfqpoint{7.236720in}{0.859325in}}%
\pgfpathlineto{\pgfqpoint{7.215330in}{0.859325in}}%
\pgfpathlineto{\pgfqpoint{7.193940in}{0.859325in}}%
\pgfpathlineto{\pgfqpoint{7.172550in}{0.859325in}}%
\pgfpathlineto{\pgfqpoint{7.151160in}{0.859325in}}%
\pgfpathlineto{\pgfqpoint{7.129770in}{0.859325in}}%
\pgfpathlineto{\pgfqpoint{7.108380in}{0.859325in}}%
\pgfpathlineto{\pgfqpoint{7.086990in}{0.859325in}}%
\pgfpathlineto{\pgfqpoint{7.065600in}{0.859325in}}%
\pgfpathlineto{\pgfqpoint{7.044210in}{0.859325in}}%
\pgfpathlineto{\pgfqpoint{7.022820in}{0.859325in}}%
\pgfpathlineto{\pgfqpoint{7.001430in}{0.909336in}}%
\pgfpathlineto{\pgfqpoint{6.980040in}{0.909336in}}%
\pgfpathlineto{\pgfqpoint{6.958650in}{0.909336in}}%
\pgfpathlineto{\pgfqpoint{6.937260in}{0.909336in}}%
\pgfpathlineto{\pgfqpoint{6.915870in}{0.909336in}}%
\pgfpathlineto{\pgfqpoint{6.894480in}{0.909336in}}%
\pgfpathlineto{\pgfqpoint{6.873090in}{0.909336in}}%
\pgfpathlineto{\pgfqpoint{6.851700in}{0.909336in}}%
\pgfpathlineto{\pgfqpoint{6.830310in}{0.909336in}}%
\pgfpathlineto{\pgfqpoint{6.808920in}{0.909336in}}%
\pgfpathlineto{\pgfqpoint{6.787530in}{0.909336in}}%
\pgfpathlineto{\pgfqpoint{6.766140in}{0.909336in}}%
\pgfpathlineto{\pgfqpoint{6.744750in}{0.909336in}}%
\pgfpathlineto{\pgfqpoint{6.723360in}{0.909336in}}%
\pgfpathlineto{\pgfqpoint{6.701970in}{0.909336in}}%
\pgfpathlineto{\pgfqpoint{6.680580in}{0.911547in}}%
\pgfpathlineto{\pgfqpoint{6.659190in}{0.915816in}}%
\pgfpathlineto{\pgfqpoint{6.637800in}{0.915816in}}%
\pgfpathlineto{\pgfqpoint{6.616410in}{0.915816in}}%
\pgfpathlineto{\pgfqpoint{6.595020in}{0.915816in}}%
\pgfpathlineto{\pgfqpoint{6.573630in}{0.915816in}}%
\pgfpathlineto{\pgfqpoint{6.552240in}{0.915816in}}%
\pgfpathlineto{\pgfqpoint{6.530850in}{0.915816in}}%
\pgfpathlineto{\pgfqpoint{6.509460in}{0.915816in}}%
\pgfpathlineto{\pgfqpoint{6.488070in}{0.915816in}}%
\pgfpathlineto{\pgfqpoint{6.466680in}{0.915816in}}%
\pgfpathlineto{\pgfqpoint{6.445290in}{0.915816in}}%
\pgfpathlineto{\pgfqpoint{6.423900in}{0.915816in}}%
\pgfpathlineto{\pgfqpoint{6.402510in}{0.915816in}}%
\pgfpathlineto{\pgfqpoint{6.381120in}{0.915816in}}%
\pgfpathlineto{\pgfqpoint{6.359730in}{0.915816in}}%
\pgfpathlineto{\pgfqpoint{6.338340in}{0.915816in}}%
\pgfpathlineto{\pgfqpoint{6.316950in}{0.915816in}}%
\pgfpathlineto{\pgfqpoint{6.295560in}{0.915816in}}%
\pgfpathlineto{\pgfqpoint{6.274170in}{0.915816in}}%
\pgfpathlineto{\pgfqpoint{6.252780in}{0.915816in}}%
\pgfpathlineto{\pgfqpoint{6.231390in}{0.915816in}}%
\pgfpathlineto{\pgfqpoint{6.210000in}{0.915816in}}%
\pgfpathlineto{\pgfqpoint{6.188610in}{0.915816in}}%
\pgfpathlineto{\pgfqpoint{6.167220in}{0.915816in}}%
\pgfpathlineto{\pgfqpoint{6.145830in}{0.915816in}}%
\pgfpathlineto{\pgfqpoint{6.124440in}{0.915816in}}%
\pgfpathlineto{\pgfqpoint{6.103050in}{0.915816in}}%
\pgfpathlineto{\pgfqpoint{6.081660in}{0.915816in}}%
\pgfpathlineto{\pgfqpoint{6.060270in}{0.915816in}}%
\pgfpathlineto{\pgfqpoint{6.038880in}{0.915816in}}%
\pgfpathlineto{\pgfqpoint{6.017490in}{0.915816in}}%
\pgfpathlineto{\pgfqpoint{5.996100in}{0.915816in}}%
\pgfpathlineto{\pgfqpoint{5.974710in}{0.915816in}}%
\pgfpathlineto{\pgfqpoint{5.953320in}{0.915816in}}%
\pgfpathlineto{\pgfqpoint{5.931930in}{0.915816in}}%
\pgfpathlineto{\pgfqpoint{5.910540in}{0.915816in}}%
\pgfpathlineto{\pgfqpoint{5.889150in}{0.915816in}}%
\pgfpathlineto{\pgfqpoint{5.867760in}{0.915816in}}%
\pgfpathlineto{\pgfqpoint{5.846370in}{0.915816in}}%
\pgfpathlineto{\pgfqpoint{5.824980in}{0.915816in}}%
\pgfpathlineto{\pgfqpoint{5.803590in}{0.915816in}}%
\pgfpathlineto{\pgfqpoint{5.782200in}{0.915816in}}%
\pgfpathlineto{\pgfqpoint{5.760810in}{0.915816in}}%
\pgfpathlineto{\pgfqpoint{5.739420in}{0.915816in}}%
\pgfpathlineto{\pgfqpoint{5.718030in}{0.915816in}}%
\pgfpathlineto{\pgfqpoint{5.696640in}{0.915816in}}%
\pgfpathlineto{\pgfqpoint{5.675250in}{0.915816in}}%
\pgfpathlineto{\pgfqpoint{5.653860in}{0.915816in}}%
\pgfpathlineto{\pgfqpoint{5.632470in}{0.915816in}}%
\pgfpathlineto{\pgfqpoint{5.611080in}{0.915816in}}%
\pgfpathlineto{\pgfqpoint{5.589690in}{0.915816in}}%
\pgfpathlineto{\pgfqpoint{5.568300in}{0.915816in}}%
\pgfpathlineto{\pgfqpoint{5.546910in}{0.915816in}}%
\pgfpathlineto{\pgfqpoint{5.525520in}{0.915816in}}%
\pgfpathlineto{\pgfqpoint{5.504130in}{0.915816in}}%
\pgfpathlineto{\pgfqpoint{5.482740in}{0.915816in}}%
\pgfpathlineto{\pgfqpoint{5.461350in}{0.915816in}}%
\pgfpathlineto{\pgfqpoint{5.439960in}{0.915816in}}%
\pgfpathlineto{\pgfqpoint{5.418570in}{0.915816in}}%
\pgfpathlineto{\pgfqpoint{5.397180in}{0.915816in}}%
\pgfpathlineto{\pgfqpoint{5.375790in}{0.915816in}}%
\pgfpathlineto{\pgfqpoint{5.354400in}{0.915816in}}%
\pgfpathlineto{\pgfqpoint{5.333010in}{0.915816in}}%
\pgfpathlineto{\pgfqpoint{5.311620in}{0.915816in}}%
\pgfpathlineto{\pgfqpoint{5.290230in}{0.915816in}}%
\pgfpathlineto{\pgfqpoint{5.268840in}{0.915816in}}%
\pgfpathlineto{\pgfqpoint{5.247450in}{0.928323in}}%
\pgfpathlineto{\pgfqpoint{5.226060in}{0.928323in}}%
\pgfpathlineto{\pgfqpoint{5.204670in}{0.928323in}}%
\pgfpathlineto{\pgfqpoint{5.183280in}{0.928323in}}%
\pgfpathlineto{\pgfqpoint{5.161890in}{0.928323in}}%
\pgfpathlineto{\pgfqpoint{5.140500in}{0.928323in}}%
\pgfpathlineto{\pgfqpoint{5.119110in}{0.928323in}}%
\pgfpathlineto{\pgfqpoint{5.097720in}{0.928323in}}%
\pgfpathlineto{\pgfqpoint{5.076330in}{0.928323in}}%
\pgfpathlineto{\pgfqpoint{5.054940in}{0.928323in}}%
\pgfpathlineto{\pgfqpoint{5.033550in}{0.928323in}}%
\pgfpathlineto{\pgfqpoint{5.012160in}{0.928323in}}%
\pgfpathlineto{\pgfqpoint{4.990770in}{0.928323in}}%
\pgfpathlineto{\pgfqpoint{4.969380in}{0.928323in}}%
\pgfpathlineto{\pgfqpoint{4.947990in}{0.928323in}}%
\pgfpathlineto{\pgfqpoint{4.926600in}{0.928323in}}%
\pgfpathlineto{\pgfqpoint{4.905210in}{0.928323in}}%
\pgfpathlineto{\pgfqpoint{4.883820in}{0.928323in}}%
\pgfpathlineto{\pgfqpoint{4.862430in}{0.928323in}}%
\pgfpathlineto{\pgfqpoint{4.841040in}{0.928323in}}%
\pgfpathlineto{\pgfqpoint{4.819650in}{0.928323in}}%
\pgfpathlineto{\pgfqpoint{4.798260in}{0.983451in}}%
\pgfpathlineto{\pgfqpoint{4.776870in}{0.983451in}}%
\pgfpathlineto{\pgfqpoint{4.755480in}{0.983451in}}%
\pgfpathlineto{\pgfqpoint{4.734090in}{0.983451in}}%
\pgfpathlineto{\pgfqpoint{4.712700in}{0.983451in}}%
\pgfpathlineto{\pgfqpoint{4.691310in}{0.983451in}}%
\pgfpathlineto{\pgfqpoint{4.669920in}{0.983451in}}%
\pgfpathlineto{\pgfqpoint{4.648530in}{0.983451in}}%
\pgfpathlineto{\pgfqpoint{4.627140in}{0.983451in}}%
\pgfpathlineto{\pgfqpoint{4.605750in}{0.983451in}}%
\pgfpathlineto{\pgfqpoint{4.584360in}{0.983451in}}%
\pgfpathlineto{\pgfqpoint{4.562970in}{0.983451in}}%
\pgfpathlineto{\pgfqpoint{4.541580in}{0.983451in}}%
\pgfpathlineto{\pgfqpoint{4.520190in}{0.983451in}}%
\pgfpathlineto{\pgfqpoint{4.498800in}{0.983451in}}%
\pgfpathlineto{\pgfqpoint{4.477410in}{0.983451in}}%
\pgfpathlineto{\pgfqpoint{4.456020in}{0.983451in}}%
\pgfpathlineto{\pgfqpoint{4.434630in}{0.983451in}}%
\pgfpathlineto{\pgfqpoint{4.413240in}{0.983451in}}%
\pgfpathlineto{\pgfqpoint{4.391850in}{0.986406in}}%
\pgfpathlineto{\pgfqpoint{4.370460in}{0.986406in}}%
\pgfpathlineto{\pgfqpoint{4.349070in}{0.986406in}}%
\pgfpathlineto{\pgfqpoint{4.327680in}{0.986406in}}%
\pgfpathlineto{\pgfqpoint{4.306290in}{0.986406in}}%
\pgfpathlineto{\pgfqpoint{4.284900in}{0.986406in}}%
\pgfpathlineto{\pgfqpoint{4.263510in}{0.986406in}}%
\pgfpathlineto{\pgfqpoint{4.242120in}{0.986406in}}%
\pgfpathlineto{\pgfqpoint{4.220730in}{0.986406in}}%
\pgfpathlineto{\pgfqpoint{4.199340in}{0.986406in}}%
\pgfpathlineto{\pgfqpoint{4.177950in}{0.986406in}}%
\pgfpathlineto{\pgfqpoint{4.156560in}{0.986406in}}%
\pgfpathlineto{\pgfqpoint{4.135170in}{0.986406in}}%
\pgfpathlineto{\pgfqpoint{4.113780in}{1.035730in}}%
\pgfpathlineto{\pgfqpoint{4.092390in}{1.043000in}}%
\pgfpathlineto{\pgfqpoint{4.071000in}{1.043000in}}%
\pgfpathlineto{\pgfqpoint{4.049610in}{1.043000in}}%
\pgfpathlineto{\pgfqpoint{4.028220in}{1.043000in}}%
\pgfpathlineto{\pgfqpoint{4.006830in}{1.043000in}}%
\pgfpathlineto{\pgfqpoint{3.985440in}{1.043000in}}%
\pgfpathlineto{\pgfqpoint{3.964050in}{1.043000in}}%
\pgfpathlineto{\pgfqpoint{3.942660in}{1.043000in}}%
\pgfpathlineto{\pgfqpoint{3.921270in}{1.043000in}}%
\pgfpathlineto{\pgfqpoint{3.899880in}{1.043000in}}%
\pgfpathlineto{\pgfqpoint{3.878490in}{1.043000in}}%
\pgfpathlineto{\pgfqpoint{3.857100in}{1.043000in}}%
\pgfpathlineto{\pgfqpoint{3.835710in}{1.048385in}}%
\pgfpathlineto{\pgfqpoint{3.814320in}{1.048385in}}%
\pgfpathlineto{\pgfqpoint{3.792930in}{1.054741in}}%
\pgfpathlineto{\pgfqpoint{3.771540in}{1.054741in}}%
\pgfpathlineto{\pgfqpoint{3.750150in}{1.054741in}}%
\pgfpathlineto{\pgfqpoint{3.728760in}{1.054741in}}%
\pgfpathlineto{\pgfqpoint{3.707370in}{1.054741in}}%
\pgfpathlineto{\pgfqpoint{3.685980in}{1.072601in}}%
\pgfpathlineto{\pgfqpoint{3.664590in}{1.105620in}}%
\pgfpathlineto{\pgfqpoint{3.643200in}{1.105620in}}%
\pgfpathlineto{\pgfqpoint{3.621810in}{1.115917in}}%
\pgfpathlineto{\pgfqpoint{3.600420in}{1.115917in}}%
\pgfpathlineto{\pgfqpoint{3.579030in}{1.115917in}}%
\pgfpathlineto{\pgfqpoint{3.557640in}{1.115917in}}%
\pgfpathlineto{\pgfqpoint{3.536250in}{1.115917in}}%
\pgfpathlineto{\pgfqpoint{3.514860in}{1.115917in}}%
\pgfpathlineto{\pgfqpoint{3.493470in}{1.185818in}}%
\pgfpathlineto{\pgfqpoint{3.472080in}{1.185818in}}%
\pgfpathlineto{\pgfqpoint{3.450690in}{1.185818in}}%
\pgfpathlineto{\pgfqpoint{3.429300in}{1.185818in}}%
\pgfpathlineto{\pgfqpoint{3.407910in}{1.193715in}}%
\pgfpathlineto{\pgfqpoint{3.386520in}{1.193715in}}%
\pgfpathlineto{\pgfqpoint{3.365130in}{1.193715in}}%
\pgfpathlineto{\pgfqpoint{3.343740in}{1.193715in}}%
\pgfpathlineto{\pgfqpoint{3.322350in}{1.205447in}}%
\pgfpathlineto{\pgfqpoint{3.300960in}{1.205447in}}%
\pgfpathlineto{\pgfqpoint{3.279570in}{1.214181in}}%
\pgfpathlineto{\pgfqpoint{3.258180in}{1.214181in}}%
\pgfpathlineto{\pgfqpoint{3.236790in}{1.214181in}}%
\pgfpathlineto{\pgfqpoint{3.215400in}{1.214181in}}%
\pgfpathlineto{\pgfqpoint{3.194010in}{1.214181in}}%
\pgfpathlineto{\pgfqpoint{3.172620in}{1.294905in}}%
\pgfpathlineto{\pgfqpoint{3.151230in}{1.325464in}}%
\pgfpathlineto{\pgfqpoint{3.129840in}{1.325464in}}%
\pgfpathlineto{\pgfqpoint{3.108450in}{1.325464in}}%
\pgfpathlineto{\pgfqpoint{3.087060in}{1.417644in}}%
\pgfpathlineto{\pgfqpoint{3.065670in}{1.417644in}}%
\pgfpathlineto{\pgfqpoint{3.044280in}{1.417644in}}%
\pgfpathlineto{\pgfqpoint{3.022890in}{1.417644in}}%
\pgfpathlineto{\pgfqpoint{3.001500in}{1.496231in}}%
\pgfpathlineto{\pgfqpoint{2.980110in}{1.627514in}}%
\pgfpathlineto{\pgfqpoint{2.958720in}{1.627514in}}%
\pgfpathlineto{\pgfqpoint{2.937330in}{1.627514in}}%
\pgfpathlineto{\pgfqpoint{2.915940in}{1.627514in}}%
\pgfpathlineto{\pgfqpoint{2.894550in}{1.627514in}}%
\pgfpathlineto{\pgfqpoint{2.873160in}{1.627514in}}%
\pgfpathlineto{\pgfqpoint{2.851770in}{1.627514in}}%
\pgfpathlineto{\pgfqpoint{2.830380in}{1.627514in}}%
\pgfpathlineto{\pgfqpoint{2.808990in}{1.627514in}}%
\pgfpathlineto{\pgfqpoint{2.787600in}{1.627514in}}%
\pgfpathlineto{\pgfqpoint{2.766210in}{1.627514in}}%
\pgfpathlineto{\pgfqpoint{2.744820in}{1.627514in}}%
\pgfpathlineto{\pgfqpoint{2.723430in}{1.627514in}}%
\pgfpathlineto{\pgfqpoint{2.702040in}{1.627514in}}%
\pgfpathlineto{\pgfqpoint{2.680650in}{1.627514in}}%
\pgfpathlineto{\pgfqpoint{2.659260in}{1.627514in}}%
\pgfpathlineto{\pgfqpoint{2.637870in}{1.627514in}}%
\pgfpathlineto{\pgfqpoint{2.616480in}{1.627514in}}%
\pgfpathlineto{\pgfqpoint{2.595090in}{1.627514in}}%
\pgfpathlineto{\pgfqpoint{2.573700in}{1.627514in}}%
\pgfpathlineto{\pgfqpoint{2.552310in}{1.627514in}}%
\pgfpathlineto{\pgfqpoint{2.530920in}{1.627514in}}%
\pgfpathlineto{\pgfqpoint{2.509530in}{1.627514in}}%
\pgfpathlineto{\pgfqpoint{2.488140in}{1.627514in}}%
\pgfpathlineto{\pgfqpoint{2.466750in}{1.627514in}}%
\pgfpathlineto{\pgfqpoint{2.445360in}{1.627514in}}%
\pgfpathlineto{\pgfqpoint{2.423970in}{1.627514in}}%
\pgfpathlineto{\pgfqpoint{2.402580in}{1.627514in}}%
\pgfpathlineto{\pgfqpoint{2.381190in}{1.627514in}}%
\pgfpathlineto{\pgfqpoint{2.359800in}{1.627514in}}%
\pgfpathlineto{\pgfqpoint{2.338410in}{1.627514in}}%
\pgfpathlineto{\pgfqpoint{2.317020in}{1.627514in}}%
\pgfpathlineto{\pgfqpoint{2.295630in}{1.627514in}}%
\pgfpathlineto{\pgfqpoint{2.274240in}{1.627514in}}%
\pgfpathlineto{\pgfqpoint{2.252850in}{1.627514in}}%
\pgfpathlineto{\pgfqpoint{2.231460in}{1.627514in}}%
\pgfpathlineto{\pgfqpoint{2.210070in}{1.627514in}}%
\pgfpathlineto{\pgfqpoint{2.188680in}{1.627514in}}%
\pgfpathlineto{\pgfqpoint{2.167290in}{1.627514in}}%
\pgfpathlineto{\pgfqpoint{2.145900in}{1.627514in}}%
\pgfpathlineto{\pgfqpoint{2.124510in}{1.627514in}}%
\pgfpathlineto{\pgfqpoint{2.103120in}{1.627514in}}%
\pgfpathlineto{\pgfqpoint{2.081730in}{1.627514in}}%
\pgfpathlineto{\pgfqpoint{2.060340in}{1.627514in}}%
\pgfpathlineto{\pgfqpoint{2.038950in}{1.627514in}}%
\pgfpathlineto{\pgfqpoint{2.017560in}{1.627514in}}%
\pgfpathlineto{\pgfqpoint{1.996170in}{1.627514in}}%
\pgfpathlineto{\pgfqpoint{1.974780in}{1.627514in}}%
\pgfpathlineto{\pgfqpoint{1.953390in}{1.627514in}}%
\pgfpathlineto{\pgfqpoint{1.932000in}{1.627514in}}%
\pgfpathlineto{\pgfqpoint{1.910610in}{1.627514in}}%
\pgfpathlineto{\pgfqpoint{1.889220in}{1.627514in}}%
\pgfpathlineto{\pgfqpoint{1.867830in}{1.627514in}}%
\pgfpathlineto{\pgfqpoint{1.846440in}{1.627514in}}%
\pgfpathlineto{\pgfqpoint{1.825050in}{1.627514in}}%
\pgfpathlineto{\pgfqpoint{1.803660in}{1.627514in}}%
\pgfpathlineto{\pgfqpoint{1.782270in}{1.627514in}}%
\pgfpathlineto{\pgfqpoint{1.760880in}{1.627892in}}%
\pgfpathlineto{\pgfqpoint{1.739490in}{1.627892in}}%
\pgfpathlineto{\pgfqpoint{1.718100in}{1.627892in}}%
\pgfpathlineto{\pgfqpoint{1.696710in}{1.627892in}}%
\pgfpathlineto{\pgfqpoint{1.675320in}{1.627892in}}%
\pgfpathlineto{\pgfqpoint{1.653930in}{1.627892in}}%
\pgfpathlineto{\pgfqpoint{1.632540in}{1.627892in}}%
\pgfpathlineto{\pgfqpoint{1.611150in}{1.627892in}}%
\pgfpathlineto{\pgfqpoint{1.589760in}{1.704995in}}%
\pgfpathlineto{\pgfqpoint{1.568370in}{1.704995in}}%
\pgfpathlineto{\pgfqpoint{1.546980in}{1.704995in}}%
\pgfpathlineto{\pgfqpoint{1.525590in}{1.704995in}}%
\pgfpathlineto{\pgfqpoint{1.504200in}{1.704995in}}%
\pgfpathlineto{\pgfqpoint{1.482810in}{1.704995in}}%
\pgfpathlineto{\pgfqpoint{1.461420in}{1.704995in}}%
\pgfpathlineto{\pgfqpoint{1.440030in}{1.731760in}}%
\pgfpathlineto{\pgfqpoint{1.418640in}{1.731760in}}%
\pgfpathlineto{\pgfqpoint{1.397250in}{1.848855in}}%
\pgfpathlineto{\pgfqpoint{1.375860in}{1.896993in}}%
\pgfpathlineto{\pgfqpoint{1.354470in}{1.896993in}}%
\pgfpathlineto{\pgfqpoint{1.333080in}{1.896993in}}%
\pgfpathlineto{\pgfqpoint{1.311690in}{1.899023in}}%
\pgfpathlineto{\pgfqpoint{1.290300in}{1.899023in}}%
\pgfpathlineto{\pgfqpoint{1.268910in}{1.899023in}}%
\pgfpathlineto{\pgfqpoint{1.247520in}{1.960883in}}%
\pgfpathlineto{\pgfqpoint{1.226130in}{1.960883in}}%
\pgfpathlineto{\pgfqpoint{1.204740in}{1.960883in}}%
\pgfpathlineto{\pgfqpoint{1.183350in}{1.960883in}}%
\pgfpathlineto{\pgfqpoint{1.161960in}{1.960883in}}%
\pgfpathlineto{\pgfqpoint{1.140570in}{1.960883in}}%
\pgfpathlineto{\pgfqpoint{1.119180in}{1.960883in}}%
\pgfpathlineto{\pgfqpoint{1.097790in}{2.034787in}}%
\pgfpathlineto{\pgfqpoint{1.076400in}{2.098263in}}%
\pgfpathlineto{\pgfqpoint{1.055010in}{2.123969in}}%
\pgfpathlineto{\pgfqpoint{1.033620in}{2.123969in}}%
\pgfpathlineto{\pgfqpoint{1.012230in}{2.192113in}}%
\pgfpathlineto{\pgfqpoint{0.990840in}{2.192113in}}%
\pgfpathlineto{\pgfqpoint{0.969450in}{2.192113in}}%
\pgfpathlineto{\pgfqpoint{0.948060in}{2.235902in}}%
\pgfpathlineto{\pgfqpoint{0.926670in}{2.235902in}}%
\pgfpathlineto{\pgfqpoint{0.905280in}{2.310054in}}%
\pgfpathlineto{\pgfqpoint{0.883890in}{2.317684in}}%
\pgfpathlineto{\pgfqpoint{0.862500in}{2.442793in}}%
\pgfpathclose%
\pgfusepath{fill}%
\end{pgfscope}%
\begin{pgfscope}%
\pgfpathrectangle{\pgfqpoint{0.862500in}{0.375000in}}{\pgfqpoint{5.347500in}{2.265000in}}%
\pgfusepath{clip}%
\pgfsetbuttcap%
\pgfsetroundjoin%
\definecolor{currentfill}{rgb}{0.580392,0.403922,0.741176}%
\pgfsetfillcolor{currentfill}%
\pgfsetfillopacity{0.200000}%
\pgfsetlinewidth{0.000000pt}%
\definecolor{currentstroke}{rgb}{0.000000,0.000000,0.000000}%
\pgfsetstrokecolor{currentstroke}%
\pgfsetdash{}{0pt}%
\pgfpathmoveto{\pgfqpoint{0.862500in}{2.493381in}}%
\pgfpathlineto{\pgfqpoint{0.862500in}{2.537045in}}%
\pgfpathlineto{\pgfqpoint{0.883890in}{2.512512in}}%
\pgfpathlineto{\pgfqpoint{0.905280in}{2.426127in}}%
\pgfpathlineto{\pgfqpoint{0.926670in}{2.422390in}}%
\pgfpathlineto{\pgfqpoint{0.948060in}{2.373221in}}%
\pgfpathlineto{\pgfqpoint{0.969450in}{2.260781in}}%
\pgfpathlineto{\pgfqpoint{0.990840in}{2.260781in}}%
\pgfpathlineto{\pgfqpoint{1.012230in}{2.260781in}}%
\pgfpathlineto{\pgfqpoint{1.033620in}{2.260781in}}%
\pgfpathlineto{\pgfqpoint{1.055010in}{2.260781in}}%
\pgfpathlineto{\pgfqpoint{1.076400in}{2.260781in}}%
\pgfpathlineto{\pgfqpoint{1.097790in}{2.260781in}}%
\pgfpathlineto{\pgfqpoint{1.119180in}{2.248433in}}%
\pgfpathlineto{\pgfqpoint{1.140570in}{2.248433in}}%
\pgfpathlineto{\pgfqpoint{1.161960in}{2.248433in}}%
\pgfpathlineto{\pgfqpoint{1.183350in}{2.248433in}}%
\pgfpathlineto{\pgfqpoint{1.204740in}{2.224511in}}%
\pgfpathlineto{\pgfqpoint{1.226130in}{2.224511in}}%
\pgfpathlineto{\pgfqpoint{1.247520in}{2.224511in}}%
\pgfpathlineto{\pgfqpoint{1.268910in}{2.224511in}}%
\pgfpathlineto{\pgfqpoint{1.290300in}{2.163877in}}%
\pgfpathlineto{\pgfqpoint{1.311690in}{2.163877in}}%
\pgfpathlineto{\pgfqpoint{1.333080in}{2.163063in}}%
\pgfpathlineto{\pgfqpoint{1.354470in}{2.163063in}}%
\pgfpathlineto{\pgfqpoint{1.375860in}{2.163063in}}%
\pgfpathlineto{\pgfqpoint{1.397250in}{2.163063in}}%
\pgfpathlineto{\pgfqpoint{1.418640in}{2.163063in}}%
\pgfpathlineto{\pgfqpoint{1.440030in}{2.163063in}}%
\pgfpathlineto{\pgfqpoint{1.461420in}{2.163063in}}%
\pgfpathlineto{\pgfqpoint{1.482810in}{2.163063in}}%
\pgfpathlineto{\pgfqpoint{1.504200in}{2.155468in}}%
\pgfpathlineto{\pgfqpoint{1.525590in}{2.155468in}}%
\pgfpathlineto{\pgfqpoint{1.546980in}{2.155468in}}%
\pgfpathlineto{\pgfqpoint{1.568370in}{2.155468in}}%
\pgfpathlineto{\pgfqpoint{1.589760in}{2.155468in}}%
\pgfpathlineto{\pgfqpoint{1.611150in}{2.155468in}}%
\pgfpathlineto{\pgfqpoint{1.632540in}{2.101567in}}%
\pgfpathlineto{\pgfqpoint{1.653930in}{2.101567in}}%
\pgfpathlineto{\pgfqpoint{1.675320in}{1.959066in}}%
\pgfpathlineto{\pgfqpoint{1.696710in}{1.951576in}}%
\pgfpathlineto{\pgfqpoint{1.718100in}{1.951576in}}%
\pgfpathlineto{\pgfqpoint{1.739490in}{1.951576in}}%
\pgfpathlineto{\pgfqpoint{1.760880in}{1.951576in}}%
\pgfpathlineto{\pgfqpoint{1.782270in}{1.951576in}}%
\pgfpathlineto{\pgfqpoint{1.803660in}{1.941312in}}%
\pgfpathlineto{\pgfqpoint{1.825050in}{1.941312in}}%
\pgfpathlineto{\pgfqpoint{1.846440in}{1.941312in}}%
\pgfpathlineto{\pgfqpoint{1.867830in}{1.941312in}}%
\pgfpathlineto{\pgfqpoint{1.889220in}{1.941312in}}%
\pgfpathlineto{\pgfqpoint{1.910610in}{1.941312in}}%
\pgfpathlineto{\pgfqpoint{1.932000in}{1.941312in}}%
\pgfpathlineto{\pgfqpoint{1.953390in}{1.941312in}}%
\pgfpathlineto{\pgfqpoint{1.974780in}{1.941312in}}%
\pgfpathlineto{\pgfqpoint{1.996170in}{1.941312in}}%
\pgfpathlineto{\pgfqpoint{2.017560in}{1.941312in}}%
\pgfpathlineto{\pgfqpoint{2.038950in}{1.941312in}}%
\pgfpathlineto{\pgfqpoint{2.060340in}{1.941312in}}%
\pgfpathlineto{\pgfqpoint{2.081730in}{1.941312in}}%
\pgfpathlineto{\pgfqpoint{2.103120in}{1.941312in}}%
\pgfpathlineto{\pgfqpoint{2.124510in}{1.941312in}}%
\pgfpathlineto{\pgfqpoint{2.145900in}{1.931982in}}%
\pgfpathlineto{\pgfqpoint{2.167290in}{1.931982in}}%
\pgfpathlineto{\pgfqpoint{2.188680in}{1.931982in}}%
\pgfpathlineto{\pgfqpoint{2.210070in}{1.931982in}}%
\pgfpathlineto{\pgfqpoint{2.231460in}{1.931982in}}%
\pgfpathlineto{\pgfqpoint{2.252850in}{1.931982in}}%
\pgfpathlineto{\pgfqpoint{2.274240in}{1.931982in}}%
\pgfpathlineto{\pgfqpoint{2.295630in}{1.931982in}}%
\pgfpathlineto{\pgfqpoint{2.317020in}{1.931982in}}%
\pgfpathlineto{\pgfqpoint{2.338410in}{1.931982in}}%
\pgfpathlineto{\pgfqpoint{2.359800in}{1.931982in}}%
\pgfpathlineto{\pgfqpoint{2.381190in}{1.931982in}}%
\pgfpathlineto{\pgfqpoint{2.402580in}{1.931982in}}%
\pgfpathlineto{\pgfqpoint{2.423970in}{1.931982in}}%
\pgfpathlineto{\pgfqpoint{2.445360in}{1.931982in}}%
\pgfpathlineto{\pgfqpoint{2.466750in}{1.931982in}}%
\pgfpathlineto{\pgfqpoint{2.488140in}{1.886361in}}%
\pgfpathlineto{\pgfqpoint{2.509530in}{1.886361in}}%
\pgfpathlineto{\pgfqpoint{2.530920in}{1.886361in}}%
\pgfpathlineto{\pgfqpoint{2.552310in}{1.886361in}}%
\pgfpathlineto{\pgfqpoint{2.573700in}{1.886361in}}%
\pgfpathlineto{\pgfqpoint{2.595090in}{1.886361in}}%
\pgfpathlineto{\pgfqpoint{2.616480in}{1.886361in}}%
\pgfpathlineto{\pgfqpoint{2.637870in}{1.886361in}}%
\pgfpathlineto{\pgfqpoint{2.659260in}{1.886361in}}%
\pgfpathlineto{\pgfqpoint{2.680650in}{1.886361in}}%
\pgfpathlineto{\pgfqpoint{2.702040in}{1.886361in}}%
\pgfpathlineto{\pgfqpoint{2.723430in}{1.886361in}}%
\pgfpathlineto{\pgfqpoint{2.744820in}{1.886361in}}%
\pgfpathlineto{\pgfqpoint{2.766210in}{1.886361in}}%
\pgfpathlineto{\pgfqpoint{2.787600in}{1.886361in}}%
\pgfpathlineto{\pgfqpoint{2.808990in}{1.886361in}}%
\pgfpathlineto{\pgfqpoint{2.830380in}{1.886361in}}%
\pgfpathlineto{\pgfqpoint{2.851770in}{1.886361in}}%
\pgfpathlineto{\pgfqpoint{2.873160in}{1.886361in}}%
\pgfpathlineto{\pgfqpoint{2.894550in}{1.886361in}}%
\pgfpathlineto{\pgfqpoint{2.915940in}{1.886361in}}%
\pgfpathlineto{\pgfqpoint{2.937330in}{1.886361in}}%
\pgfpathlineto{\pgfqpoint{2.958720in}{1.886361in}}%
\pgfpathlineto{\pgfqpoint{2.980110in}{1.886361in}}%
\pgfpathlineto{\pgfqpoint{3.001500in}{1.886361in}}%
\pgfpathlineto{\pgfqpoint{3.022890in}{1.886361in}}%
\pgfpathlineto{\pgfqpoint{3.044280in}{1.886361in}}%
\pgfpathlineto{\pgfqpoint{3.065670in}{1.886361in}}%
\pgfpathlineto{\pgfqpoint{3.087060in}{1.886361in}}%
\pgfpathlineto{\pgfqpoint{3.108450in}{1.886361in}}%
\pgfpathlineto{\pgfqpoint{3.129840in}{1.886361in}}%
\pgfpathlineto{\pgfqpoint{3.151230in}{1.886361in}}%
\pgfpathlineto{\pgfqpoint{3.172620in}{1.886361in}}%
\pgfpathlineto{\pgfqpoint{3.194010in}{1.886361in}}%
\pgfpathlineto{\pgfqpoint{3.215400in}{1.886361in}}%
\pgfpathlineto{\pgfqpoint{3.236790in}{1.886361in}}%
\pgfpathlineto{\pgfqpoint{3.258180in}{1.886361in}}%
\pgfpathlineto{\pgfqpoint{3.279570in}{1.886361in}}%
\pgfpathlineto{\pgfqpoint{3.300960in}{1.886361in}}%
\pgfpathlineto{\pgfqpoint{3.322350in}{1.886361in}}%
\pgfpathlineto{\pgfqpoint{3.343740in}{1.886361in}}%
\pgfpathlineto{\pgfqpoint{3.365130in}{1.886361in}}%
\pgfpathlineto{\pgfqpoint{3.386520in}{1.886361in}}%
\pgfpathlineto{\pgfqpoint{3.407910in}{1.886361in}}%
\pgfpathlineto{\pgfqpoint{3.429300in}{1.821607in}}%
\pgfpathlineto{\pgfqpoint{3.450690in}{1.821607in}}%
\pgfpathlineto{\pgfqpoint{3.472080in}{1.821607in}}%
\pgfpathlineto{\pgfqpoint{3.493470in}{1.821607in}}%
\pgfpathlineto{\pgfqpoint{3.514860in}{1.821607in}}%
\pgfpathlineto{\pgfqpoint{3.536250in}{1.821607in}}%
\pgfpathlineto{\pgfqpoint{3.557640in}{1.821607in}}%
\pgfpathlineto{\pgfqpoint{3.579030in}{1.821607in}}%
\pgfpathlineto{\pgfqpoint{3.600420in}{1.821607in}}%
\pgfpathlineto{\pgfqpoint{3.621810in}{1.821607in}}%
\pgfpathlineto{\pgfqpoint{3.643200in}{1.821607in}}%
\pgfpathlineto{\pgfqpoint{3.664590in}{1.821607in}}%
\pgfpathlineto{\pgfqpoint{3.685980in}{1.821607in}}%
\pgfpathlineto{\pgfqpoint{3.707370in}{1.821607in}}%
\pgfpathlineto{\pgfqpoint{3.728760in}{1.821607in}}%
\pgfpathlineto{\pgfqpoint{3.750150in}{1.821607in}}%
\pgfpathlineto{\pgfqpoint{3.771540in}{1.821607in}}%
\pgfpathlineto{\pgfqpoint{3.792930in}{1.821607in}}%
\pgfpathlineto{\pgfqpoint{3.814320in}{1.821607in}}%
\pgfpathlineto{\pgfqpoint{3.835710in}{1.821607in}}%
\pgfpathlineto{\pgfqpoint{3.857100in}{1.821607in}}%
\pgfpathlineto{\pgfqpoint{3.878490in}{1.821607in}}%
\pgfpathlineto{\pgfqpoint{3.899880in}{1.818720in}}%
\pgfpathlineto{\pgfqpoint{3.921270in}{1.818720in}}%
\pgfpathlineto{\pgfqpoint{3.942660in}{1.818720in}}%
\pgfpathlineto{\pgfqpoint{3.964050in}{1.818720in}}%
\pgfpathlineto{\pgfqpoint{3.985440in}{1.818720in}}%
\pgfpathlineto{\pgfqpoint{4.006830in}{1.818720in}}%
\pgfpathlineto{\pgfqpoint{4.028220in}{1.818720in}}%
\pgfpathlineto{\pgfqpoint{4.049610in}{1.818720in}}%
\pgfpathlineto{\pgfqpoint{4.071000in}{1.818720in}}%
\pgfpathlineto{\pgfqpoint{4.092390in}{1.818720in}}%
\pgfpathlineto{\pgfqpoint{4.113780in}{1.818720in}}%
\pgfpathlineto{\pgfqpoint{4.135170in}{1.818720in}}%
\pgfpathlineto{\pgfqpoint{4.156560in}{1.818720in}}%
\pgfpathlineto{\pgfqpoint{4.177950in}{1.818720in}}%
\pgfpathlineto{\pgfqpoint{4.199340in}{1.818720in}}%
\pgfpathlineto{\pgfqpoint{4.220730in}{1.818720in}}%
\pgfpathlineto{\pgfqpoint{4.242120in}{1.818720in}}%
\pgfpathlineto{\pgfqpoint{4.263510in}{1.818720in}}%
\pgfpathlineto{\pgfqpoint{4.284900in}{1.818720in}}%
\pgfpathlineto{\pgfqpoint{4.306290in}{1.818720in}}%
\pgfpathlineto{\pgfqpoint{4.327680in}{1.818720in}}%
\pgfpathlineto{\pgfqpoint{4.349070in}{1.818720in}}%
\pgfpathlineto{\pgfqpoint{4.370460in}{1.818720in}}%
\pgfpathlineto{\pgfqpoint{4.391850in}{1.818720in}}%
\pgfpathlineto{\pgfqpoint{4.413240in}{1.818720in}}%
\pgfpathlineto{\pgfqpoint{4.434630in}{1.818720in}}%
\pgfpathlineto{\pgfqpoint{4.456020in}{1.818720in}}%
\pgfpathlineto{\pgfqpoint{4.477410in}{1.818720in}}%
\pgfpathlineto{\pgfqpoint{4.498800in}{1.818720in}}%
\pgfpathlineto{\pgfqpoint{4.520190in}{1.818720in}}%
\pgfpathlineto{\pgfqpoint{4.541580in}{1.818720in}}%
\pgfpathlineto{\pgfqpoint{4.562970in}{1.818720in}}%
\pgfpathlineto{\pgfqpoint{4.584360in}{1.818720in}}%
\pgfpathlineto{\pgfqpoint{4.605750in}{1.818720in}}%
\pgfpathlineto{\pgfqpoint{4.627140in}{1.818720in}}%
\pgfpathlineto{\pgfqpoint{4.648530in}{1.818720in}}%
\pgfpathlineto{\pgfqpoint{4.669920in}{1.818720in}}%
\pgfpathlineto{\pgfqpoint{4.691310in}{1.818720in}}%
\pgfpathlineto{\pgfqpoint{4.712700in}{1.818720in}}%
\pgfpathlineto{\pgfqpoint{4.734090in}{1.818720in}}%
\pgfpathlineto{\pgfqpoint{4.755480in}{1.818720in}}%
\pgfpathlineto{\pgfqpoint{4.776870in}{1.818720in}}%
\pgfpathlineto{\pgfqpoint{4.798260in}{1.818720in}}%
\pgfpathlineto{\pgfqpoint{4.819650in}{1.818720in}}%
\pgfpathlineto{\pgfqpoint{4.841040in}{1.818720in}}%
\pgfpathlineto{\pgfqpoint{4.862430in}{1.818720in}}%
\pgfpathlineto{\pgfqpoint{4.883820in}{1.818720in}}%
\pgfpathlineto{\pgfqpoint{4.905210in}{1.818720in}}%
\pgfpathlineto{\pgfqpoint{4.926600in}{1.818720in}}%
\pgfpathlineto{\pgfqpoint{4.947990in}{1.818720in}}%
\pgfpathlineto{\pgfqpoint{4.969380in}{1.818720in}}%
\pgfpathlineto{\pgfqpoint{4.990770in}{1.818720in}}%
\pgfpathlineto{\pgfqpoint{5.012160in}{1.818720in}}%
\pgfpathlineto{\pgfqpoint{5.033550in}{1.818720in}}%
\pgfpathlineto{\pgfqpoint{5.054940in}{1.818720in}}%
\pgfpathlineto{\pgfqpoint{5.076330in}{1.818720in}}%
\pgfpathlineto{\pgfqpoint{5.097720in}{1.818720in}}%
\pgfpathlineto{\pgfqpoint{5.119110in}{1.818720in}}%
\pgfpathlineto{\pgfqpoint{5.140500in}{1.635766in}}%
\pgfpathlineto{\pgfqpoint{5.161890in}{1.582941in}}%
\pgfpathlineto{\pgfqpoint{5.183280in}{1.544978in}}%
\pgfpathlineto{\pgfqpoint{5.204670in}{1.519393in}}%
\pgfpathlineto{\pgfqpoint{5.226060in}{1.469424in}}%
\pgfpathlineto{\pgfqpoint{5.247450in}{1.469424in}}%
\pgfpathlineto{\pgfqpoint{5.268840in}{1.469424in}}%
\pgfpathlineto{\pgfqpoint{5.290230in}{1.469424in}}%
\pgfpathlineto{\pgfqpoint{5.311620in}{1.253005in}}%
\pgfpathlineto{\pgfqpoint{5.333010in}{1.253005in}}%
\pgfpathlineto{\pgfqpoint{5.354400in}{1.253005in}}%
\pgfpathlineto{\pgfqpoint{5.375790in}{1.253005in}}%
\pgfpathlineto{\pgfqpoint{5.397180in}{1.253005in}}%
\pgfpathlineto{\pgfqpoint{5.418570in}{1.224721in}}%
\pgfpathlineto{\pgfqpoint{5.439960in}{1.216913in}}%
\pgfpathlineto{\pgfqpoint{5.461350in}{1.110416in}}%
\pgfpathlineto{\pgfqpoint{5.482740in}{1.063249in}}%
\pgfpathlineto{\pgfqpoint{5.504130in}{1.063249in}}%
\pgfpathlineto{\pgfqpoint{5.525520in}{1.025889in}}%
\pgfpathlineto{\pgfqpoint{5.546910in}{1.025889in}}%
\pgfpathlineto{\pgfqpoint{5.568300in}{1.025889in}}%
\pgfpathlineto{\pgfqpoint{5.589690in}{1.025889in}}%
\pgfpathlineto{\pgfqpoint{5.611080in}{1.025889in}}%
\pgfpathlineto{\pgfqpoint{5.632470in}{1.025889in}}%
\pgfpathlineto{\pgfqpoint{5.653860in}{1.025889in}}%
\pgfpathlineto{\pgfqpoint{5.675250in}{1.025889in}}%
\pgfpathlineto{\pgfqpoint{5.696640in}{1.025889in}}%
\pgfpathlineto{\pgfqpoint{5.718030in}{1.025889in}}%
\pgfpathlineto{\pgfqpoint{5.739420in}{1.025889in}}%
\pgfpathlineto{\pgfqpoint{5.760810in}{0.954824in}}%
\pgfpathlineto{\pgfqpoint{5.782200in}{0.954824in}}%
\pgfpathlineto{\pgfqpoint{5.803590in}{0.954824in}}%
\pgfpathlineto{\pgfqpoint{5.824980in}{0.954824in}}%
\pgfpathlineto{\pgfqpoint{5.846370in}{0.954824in}}%
\pgfpathlineto{\pgfqpoint{5.867760in}{0.954824in}}%
\pgfpathlineto{\pgfqpoint{5.889150in}{0.954824in}}%
\pgfpathlineto{\pgfqpoint{5.910540in}{0.954824in}}%
\pgfpathlineto{\pgfqpoint{5.931930in}{0.954824in}}%
\pgfpathlineto{\pgfqpoint{5.953320in}{0.951159in}}%
\pgfpathlineto{\pgfqpoint{5.974710in}{0.951159in}}%
\pgfpathlineto{\pgfqpoint{5.996100in}{0.951159in}}%
\pgfpathlineto{\pgfqpoint{6.017490in}{0.951159in}}%
\pgfpathlineto{\pgfqpoint{6.038880in}{0.951159in}}%
\pgfpathlineto{\pgfqpoint{6.060270in}{0.951159in}}%
\pgfpathlineto{\pgfqpoint{6.081660in}{0.947982in}}%
\pgfpathlineto{\pgfqpoint{6.103050in}{0.947982in}}%
\pgfpathlineto{\pgfqpoint{6.124440in}{0.947982in}}%
\pgfpathlineto{\pgfqpoint{6.145830in}{0.877351in}}%
\pgfpathlineto{\pgfqpoint{6.167220in}{0.877351in}}%
\pgfpathlineto{\pgfqpoint{6.188610in}{0.877351in}}%
\pgfpathlineto{\pgfqpoint{6.210000in}{0.877351in}}%
\pgfpathlineto{\pgfqpoint{6.231390in}{0.877351in}}%
\pgfpathlineto{\pgfqpoint{6.252780in}{0.877351in}}%
\pgfpathlineto{\pgfqpoint{6.274170in}{0.877351in}}%
\pgfpathlineto{\pgfqpoint{6.295560in}{0.877351in}}%
\pgfpathlineto{\pgfqpoint{6.316950in}{0.877351in}}%
\pgfpathlineto{\pgfqpoint{6.338340in}{0.877351in}}%
\pgfpathlineto{\pgfqpoint{6.359730in}{0.877351in}}%
\pgfpathlineto{\pgfqpoint{6.381120in}{0.877351in}}%
\pgfpathlineto{\pgfqpoint{6.402510in}{0.877351in}}%
\pgfpathlineto{\pgfqpoint{6.423900in}{0.877351in}}%
\pgfpathlineto{\pgfqpoint{6.445290in}{0.877351in}}%
\pgfpathlineto{\pgfqpoint{6.466680in}{0.877351in}}%
\pgfpathlineto{\pgfqpoint{6.488070in}{0.877351in}}%
\pgfpathlineto{\pgfqpoint{6.509460in}{0.877351in}}%
\pgfpathlineto{\pgfqpoint{6.530850in}{0.877351in}}%
\pgfpathlineto{\pgfqpoint{6.552240in}{0.877351in}}%
\pgfpathlineto{\pgfqpoint{6.573630in}{0.872965in}}%
\pgfpathlineto{\pgfqpoint{6.595020in}{0.872965in}}%
\pgfpathlineto{\pgfqpoint{6.616410in}{0.872965in}}%
\pgfpathlineto{\pgfqpoint{6.637800in}{0.872965in}}%
\pgfpathlineto{\pgfqpoint{6.659190in}{0.872965in}}%
\pgfpathlineto{\pgfqpoint{6.680580in}{0.872965in}}%
\pgfpathlineto{\pgfqpoint{6.701970in}{0.872965in}}%
\pgfpathlineto{\pgfqpoint{6.723360in}{0.872965in}}%
\pgfpathlineto{\pgfqpoint{6.744750in}{0.872965in}}%
\pgfpathlineto{\pgfqpoint{6.766140in}{0.872965in}}%
\pgfpathlineto{\pgfqpoint{6.787530in}{0.872965in}}%
\pgfpathlineto{\pgfqpoint{6.808920in}{0.872965in}}%
\pgfpathlineto{\pgfqpoint{6.830310in}{0.872965in}}%
\pgfpathlineto{\pgfqpoint{6.851700in}{0.872965in}}%
\pgfpathlineto{\pgfqpoint{6.873090in}{0.872965in}}%
\pgfpathlineto{\pgfqpoint{6.894480in}{0.872965in}}%
\pgfpathlineto{\pgfqpoint{6.915870in}{0.863205in}}%
\pgfpathlineto{\pgfqpoint{6.937260in}{0.863205in}}%
\pgfpathlineto{\pgfqpoint{6.958650in}{0.863205in}}%
\pgfpathlineto{\pgfqpoint{6.980040in}{0.863205in}}%
\pgfpathlineto{\pgfqpoint{7.001430in}{0.863205in}}%
\pgfpathlineto{\pgfqpoint{7.022820in}{0.863205in}}%
\pgfpathlineto{\pgfqpoint{7.044210in}{0.863205in}}%
\pgfpathlineto{\pgfqpoint{7.065600in}{0.863205in}}%
\pgfpathlineto{\pgfqpoint{7.086990in}{0.863205in}}%
\pgfpathlineto{\pgfqpoint{7.108380in}{0.863205in}}%
\pgfpathlineto{\pgfqpoint{7.129770in}{0.863205in}}%
\pgfpathlineto{\pgfqpoint{7.151160in}{0.863205in}}%
\pgfpathlineto{\pgfqpoint{7.172550in}{0.863205in}}%
\pgfpathlineto{\pgfqpoint{7.193940in}{0.863205in}}%
\pgfpathlineto{\pgfqpoint{7.215330in}{0.863205in}}%
\pgfpathlineto{\pgfqpoint{7.236720in}{0.863205in}}%
\pgfpathlineto{\pgfqpoint{7.258110in}{0.861518in}}%
\pgfpathlineto{\pgfqpoint{7.279500in}{0.861518in}}%
\pgfpathlineto{\pgfqpoint{7.300890in}{0.861518in}}%
\pgfpathlineto{\pgfqpoint{7.322280in}{0.861518in}}%
\pgfpathlineto{\pgfqpoint{7.343670in}{0.861518in}}%
\pgfpathlineto{\pgfqpoint{7.365060in}{0.859187in}}%
\pgfpathlineto{\pgfqpoint{7.386450in}{0.859187in}}%
\pgfpathlineto{\pgfqpoint{7.407840in}{0.859187in}}%
\pgfpathlineto{\pgfqpoint{7.429230in}{0.859187in}}%
\pgfpathlineto{\pgfqpoint{7.450620in}{0.859187in}}%
\pgfpathlineto{\pgfqpoint{7.472010in}{0.859187in}}%
\pgfpathlineto{\pgfqpoint{7.493400in}{0.859187in}}%
\pgfpathlineto{\pgfqpoint{7.514790in}{0.859187in}}%
\pgfpathlineto{\pgfqpoint{7.536180in}{0.859187in}}%
\pgfpathlineto{\pgfqpoint{7.557570in}{0.859187in}}%
\pgfpathlineto{\pgfqpoint{7.578960in}{0.859187in}}%
\pgfpathlineto{\pgfqpoint{7.600350in}{0.859187in}}%
\pgfpathlineto{\pgfqpoint{7.621740in}{0.859187in}}%
\pgfpathlineto{\pgfqpoint{7.643130in}{0.859187in}}%
\pgfpathlineto{\pgfqpoint{7.664520in}{0.859187in}}%
\pgfpathlineto{\pgfqpoint{7.685910in}{0.859187in}}%
\pgfpathlineto{\pgfqpoint{7.707300in}{0.859187in}}%
\pgfpathlineto{\pgfqpoint{7.728690in}{0.859187in}}%
\pgfpathlineto{\pgfqpoint{7.750080in}{0.859187in}}%
\pgfpathlineto{\pgfqpoint{7.771470in}{0.859187in}}%
\pgfpathlineto{\pgfqpoint{7.792860in}{0.859187in}}%
\pgfpathlineto{\pgfqpoint{7.814250in}{0.859187in}}%
\pgfpathlineto{\pgfqpoint{7.835640in}{0.859187in}}%
\pgfpathlineto{\pgfqpoint{7.857030in}{0.859187in}}%
\pgfpathlineto{\pgfqpoint{7.878420in}{0.794237in}}%
\pgfpathlineto{\pgfqpoint{7.899810in}{0.794237in}}%
\pgfpathlineto{\pgfqpoint{7.921200in}{0.794237in}}%
\pgfpathlineto{\pgfqpoint{7.942590in}{0.794237in}}%
\pgfpathlineto{\pgfqpoint{7.963980in}{0.794237in}}%
\pgfpathlineto{\pgfqpoint{7.985370in}{0.794237in}}%
\pgfpathlineto{\pgfqpoint{8.006760in}{0.794237in}}%
\pgfpathlineto{\pgfqpoint{8.028150in}{0.794237in}}%
\pgfpathlineto{\pgfqpoint{8.049540in}{0.794237in}}%
\pgfpathlineto{\pgfqpoint{8.070930in}{0.794237in}}%
\pgfpathlineto{\pgfqpoint{8.092320in}{0.794237in}}%
\pgfpathlineto{\pgfqpoint{8.113710in}{0.707752in}}%
\pgfpathlineto{\pgfqpoint{8.135100in}{0.707752in}}%
\pgfpathlineto{\pgfqpoint{8.156490in}{0.707752in}}%
\pgfpathlineto{\pgfqpoint{8.177880in}{0.707752in}}%
\pgfpathlineto{\pgfqpoint{8.199270in}{0.707752in}}%
\pgfpathlineto{\pgfqpoint{8.220660in}{0.707752in}}%
\pgfpathlineto{\pgfqpoint{8.242050in}{0.707752in}}%
\pgfpathlineto{\pgfqpoint{8.263440in}{0.707752in}}%
\pgfpathlineto{\pgfqpoint{8.284830in}{0.707752in}}%
\pgfpathlineto{\pgfqpoint{8.306220in}{0.707752in}}%
\pgfpathlineto{\pgfqpoint{8.327610in}{0.707752in}}%
\pgfpathlineto{\pgfqpoint{8.349000in}{0.707752in}}%
\pgfpathlineto{\pgfqpoint{8.370390in}{0.707752in}}%
\pgfpathlineto{\pgfqpoint{8.391780in}{0.707752in}}%
\pgfpathlineto{\pgfqpoint{8.413170in}{0.707752in}}%
\pgfpathlineto{\pgfqpoint{8.434560in}{0.707752in}}%
\pgfpathlineto{\pgfqpoint{8.455950in}{0.707752in}}%
\pgfpathlineto{\pgfqpoint{8.477340in}{0.707752in}}%
\pgfpathlineto{\pgfqpoint{8.498730in}{0.707752in}}%
\pgfpathlineto{\pgfqpoint{8.520120in}{0.707752in}}%
\pgfpathlineto{\pgfqpoint{8.541510in}{0.707752in}}%
\pgfpathlineto{\pgfqpoint{8.562900in}{0.707752in}}%
\pgfpathlineto{\pgfqpoint{8.584290in}{0.707752in}}%
\pgfpathlineto{\pgfqpoint{8.605680in}{0.707752in}}%
\pgfpathlineto{\pgfqpoint{8.627070in}{0.707752in}}%
\pgfpathlineto{\pgfqpoint{8.648460in}{0.707752in}}%
\pgfpathlineto{\pgfqpoint{8.669850in}{0.707752in}}%
\pgfpathlineto{\pgfqpoint{8.691240in}{0.707752in}}%
\pgfpathlineto{\pgfqpoint{8.712630in}{0.707752in}}%
\pgfpathlineto{\pgfqpoint{8.734020in}{0.707752in}}%
\pgfpathlineto{\pgfqpoint{8.755410in}{0.707752in}}%
\pgfpathlineto{\pgfqpoint{8.776800in}{0.707752in}}%
\pgfpathlineto{\pgfqpoint{8.798190in}{0.707752in}}%
\pgfpathlineto{\pgfqpoint{8.819580in}{0.707752in}}%
\pgfpathlineto{\pgfqpoint{8.840970in}{0.707752in}}%
\pgfpathlineto{\pgfqpoint{8.862360in}{0.698912in}}%
\pgfpathlineto{\pgfqpoint{8.883750in}{0.698912in}}%
\pgfpathlineto{\pgfqpoint{8.905140in}{0.698912in}}%
\pgfpathlineto{\pgfqpoint{8.926530in}{0.698912in}}%
\pgfpathlineto{\pgfqpoint{8.947920in}{0.698912in}}%
\pgfpathlineto{\pgfqpoint{8.969310in}{0.698912in}}%
\pgfpathlineto{\pgfqpoint{8.990700in}{0.698912in}}%
\pgfpathlineto{\pgfqpoint{9.012090in}{0.698912in}}%
\pgfpathlineto{\pgfqpoint{9.033480in}{0.698912in}}%
\pgfpathlineto{\pgfqpoint{9.054870in}{0.698912in}}%
\pgfpathlineto{\pgfqpoint{9.076260in}{0.698912in}}%
\pgfpathlineto{\pgfqpoint{9.097650in}{0.698912in}}%
\pgfpathlineto{\pgfqpoint{9.119040in}{0.698912in}}%
\pgfpathlineto{\pgfqpoint{9.140430in}{0.696721in}}%
\pgfpathlineto{\pgfqpoint{9.161820in}{0.696721in}}%
\pgfpathlineto{\pgfqpoint{9.183210in}{0.696721in}}%
\pgfpathlineto{\pgfqpoint{9.204600in}{0.644468in}}%
\pgfpathlineto{\pgfqpoint{9.225990in}{0.644468in}}%
\pgfpathlineto{\pgfqpoint{9.247380in}{0.644468in}}%
\pgfpathlineto{\pgfqpoint{9.268770in}{0.644468in}}%
\pgfpathlineto{\pgfqpoint{9.290160in}{0.644468in}}%
\pgfpathlineto{\pgfqpoint{9.311550in}{0.644468in}}%
\pgfpathlineto{\pgfqpoint{9.332940in}{0.644468in}}%
\pgfpathlineto{\pgfqpoint{9.354330in}{0.644468in}}%
\pgfpathlineto{\pgfqpoint{9.375720in}{0.644468in}}%
\pgfpathlineto{\pgfqpoint{9.397110in}{0.644468in}}%
\pgfpathlineto{\pgfqpoint{9.397110in}{0.477955in}}%
\pgfpathlineto{\pgfqpoint{9.397110in}{0.477955in}}%
\pgfpathlineto{\pgfqpoint{9.375720in}{0.477955in}}%
\pgfpathlineto{\pgfqpoint{9.354330in}{0.477955in}}%
\pgfpathlineto{\pgfqpoint{9.332940in}{0.477955in}}%
\pgfpathlineto{\pgfqpoint{9.311550in}{0.477955in}}%
\pgfpathlineto{\pgfqpoint{9.290160in}{0.477955in}}%
\pgfpathlineto{\pgfqpoint{9.268770in}{0.477955in}}%
\pgfpathlineto{\pgfqpoint{9.247380in}{0.477955in}}%
\pgfpathlineto{\pgfqpoint{9.225990in}{0.477955in}}%
\pgfpathlineto{\pgfqpoint{9.204600in}{0.477955in}}%
\pgfpathlineto{\pgfqpoint{9.183210in}{0.580497in}}%
\pgfpathlineto{\pgfqpoint{9.161820in}{0.580497in}}%
\pgfpathlineto{\pgfqpoint{9.140430in}{0.580497in}}%
\pgfpathlineto{\pgfqpoint{9.119040in}{0.582748in}}%
\pgfpathlineto{\pgfqpoint{9.097650in}{0.582748in}}%
\pgfpathlineto{\pgfqpoint{9.076260in}{0.582748in}}%
\pgfpathlineto{\pgfqpoint{9.054870in}{0.582748in}}%
\pgfpathlineto{\pgfqpoint{9.033480in}{0.582748in}}%
\pgfpathlineto{\pgfqpoint{9.012090in}{0.582748in}}%
\pgfpathlineto{\pgfqpoint{8.990700in}{0.582748in}}%
\pgfpathlineto{\pgfqpoint{8.969310in}{0.582748in}}%
\pgfpathlineto{\pgfqpoint{8.947920in}{0.582748in}}%
\pgfpathlineto{\pgfqpoint{8.926530in}{0.582748in}}%
\pgfpathlineto{\pgfqpoint{8.905140in}{0.582748in}}%
\pgfpathlineto{\pgfqpoint{8.883750in}{0.582748in}}%
\pgfpathlineto{\pgfqpoint{8.862360in}{0.582748in}}%
\pgfpathlineto{\pgfqpoint{8.840970in}{0.590502in}}%
\pgfpathlineto{\pgfqpoint{8.819580in}{0.590502in}}%
\pgfpathlineto{\pgfqpoint{8.798190in}{0.590502in}}%
\pgfpathlineto{\pgfqpoint{8.776800in}{0.590502in}}%
\pgfpathlineto{\pgfqpoint{8.755410in}{0.590502in}}%
\pgfpathlineto{\pgfqpoint{8.734020in}{0.590502in}}%
\pgfpathlineto{\pgfqpoint{8.712630in}{0.590502in}}%
\pgfpathlineto{\pgfqpoint{8.691240in}{0.590502in}}%
\pgfpathlineto{\pgfqpoint{8.669850in}{0.590502in}}%
\pgfpathlineto{\pgfqpoint{8.648460in}{0.590502in}}%
\pgfpathlineto{\pgfqpoint{8.627070in}{0.590502in}}%
\pgfpathlineto{\pgfqpoint{8.605680in}{0.590502in}}%
\pgfpathlineto{\pgfqpoint{8.584290in}{0.590502in}}%
\pgfpathlineto{\pgfqpoint{8.562900in}{0.590502in}}%
\pgfpathlineto{\pgfqpoint{8.541510in}{0.590502in}}%
\pgfpathlineto{\pgfqpoint{8.520120in}{0.590502in}}%
\pgfpathlineto{\pgfqpoint{8.498730in}{0.590502in}}%
\pgfpathlineto{\pgfqpoint{8.477340in}{0.590502in}}%
\pgfpathlineto{\pgfqpoint{8.455950in}{0.590502in}}%
\pgfpathlineto{\pgfqpoint{8.434560in}{0.590502in}}%
\pgfpathlineto{\pgfqpoint{8.413170in}{0.590502in}}%
\pgfpathlineto{\pgfqpoint{8.391780in}{0.590502in}}%
\pgfpathlineto{\pgfqpoint{8.370390in}{0.590502in}}%
\pgfpathlineto{\pgfqpoint{8.349000in}{0.590502in}}%
\pgfpathlineto{\pgfqpoint{8.327610in}{0.590502in}}%
\pgfpathlineto{\pgfqpoint{8.306220in}{0.590502in}}%
\pgfpathlineto{\pgfqpoint{8.284830in}{0.590502in}}%
\pgfpathlineto{\pgfqpoint{8.263440in}{0.590502in}}%
\pgfpathlineto{\pgfqpoint{8.242050in}{0.590502in}}%
\pgfpathlineto{\pgfqpoint{8.220660in}{0.590502in}}%
\pgfpathlineto{\pgfqpoint{8.199270in}{0.590502in}}%
\pgfpathlineto{\pgfqpoint{8.177880in}{0.590502in}}%
\pgfpathlineto{\pgfqpoint{8.156490in}{0.590502in}}%
\pgfpathlineto{\pgfqpoint{8.135100in}{0.590502in}}%
\pgfpathlineto{\pgfqpoint{8.113710in}{0.590502in}}%
\pgfpathlineto{\pgfqpoint{8.092320in}{0.614589in}}%
\pgfpathlineto{\pgfqpoint{8.070930in}{0.614589in}}%
\pgfpathlineto{\pgfqpoint{8.049540in}{0.614589in}}%
\pgfpathlineto{\pgfqpoint{8.028150in}{0.614589in}}%
\pgfpathlineto{\pgfqpoint{8.006760in}{0.614589in}}%
\pgfpathlineto{\pgfqpoint{7.985370in}{0.614589in}}%
\pgfpathlineto{\pgfqpoint{7.963980in}{0.614589in}}%
\pgfpathlineto{\pgfqpoint{7.942590in}{0.614589in}}%
\pgfpathlineto{\pgfqpoint{7.921200in}{0.614589in}}%
\pgfpathlineto{\pgfqpoint{7.899810in}{0.614589in}}%
\pgfpathlineto{\pgfqpoint{7.878420in}{0.614589in}}%
\pgfpathlineto{\pgfqpoint{7.857030in}{0.674427in}}%
\pgfpathlineto{\pgfqpoint{7.835640in}{0.674427in}}%
\pgfpathlineto{\pgfqpoint{7.814250in}{0.674427in}}%
\pgfpathlineto{\pgfqpoint{7.792860in}{0.674427in}}%
\pgfpathlineto{\pgfqpoint{7.771470in}{0.674427in}}%
\pgfpathlineto{\pgfqpoint{7.750080in}{0.674427in}}%
\pgfpathlineto{\pgfqpoint{7.728690in}{0.674427in}}%
\pgfpathlineto{\pgfqpoint{7.707300in}{0.674427in}}%
\pgfpathlineto{\pgfqpoint{7.685910in}{0.674427in}}%
\pgfpathlineto{\pgfqpoint{7.664520in}{0.674427in}}%
\pgfpathlineto{\pgfqpoint{7.643130in}{0.674427in}}%
\pgfpathlineto{\pgfqpoint{7.621740in}{0.674427in}}%
\pgfpathlineto{\pgfqpoint{7.600350in}{0.674427in}}%
\pgfpathlineto{\pgfqpoint{7.578960in}{0.674427in}}%
\pgfpathlineto{\pgfqpoint{7.557570in}{0.674427in}}%
\pgfpathlineto{\pgfqpoint{7.536180in}{0.674427in}}%
\pgfpathlineto{\pgfqpoint{7.514790in}{0.674427in}}%
\pgfpathlineto{\pgfqpoint{7.493400in}{0.674427in}}%
\pgfpathlineto{\pgfqpoint{7.472010in}{0.674427in}}%
\pgfpathlineto{\pgfqpoint{7.450620in}{0.674427in}}%
\pgfpathlineto{\pgfqpoint{7.429230in}{0.674427in}}%
\pgfpathlineto{\pgfqpoint{7.407840in}{0.674427in}}%
\pgfpathlineto{\pgfqpoint{7.386450in}{0.674427in}}%
\pgfpathlineto{\pgfqpoint{7.365060in}{0.674427in}}%
\pgfpathlineto{\pgfqpoint{7.343670in}{0.688120in}}%
\pgfpathlineto{\pgfqpoint{7.322280in}{0.688120in}}%
\pgfpathlineto{\pgfqpoint{7.300890in}{0.688120in}}%
\pgfpathlineto{\pgfqpoint{7.279500in}{0.688120in}}%
\pgfpathlineto{\pgfqpoint{7.258110in}{0.688120in}}%
\pgfpathlineto{\pgfqpoint{7.236720in}{0.696659in}}%
\pgfpathlineto{\pgfqpoint{7.215330in}{0.696659in}}%
\pgfpathlineto{\pgfqpoint{7.193940in}{0.696659in}}%
\pgfpathlineto{\pgfqpoint{7.172550in}{0.696659in}}%
\pgfpathlineto{\pgfqpoint{7.151160in}{0.696659in}}%
\pgfpathlineto{\pgfqpoint{7.129770in}{0.696659in}}%
\pgfpathlineto{\pgfqpoint{7.108380in}{0.696659in}}%
\pgfpathlineto{\pgfqpoint{7.086990in}{0.696659in}}%
\pgfpathlineto{\pgfqpoint{7.065600in}{0.696659in}}%
\pgfpathlineto{\pgfqpoint{7.044210in}{0.696659in}}%
\pgfpathlineto{\pgfqpoint{7.022820in}{0.696659in}}%
\pgfpathlineto{\pgfqpoint{7.001430in}{0.696659in}}%
\pgfpathlineto{\pgfqpoint{6.980040in}{0.696659in}}%
\pgfpathlineto{\pgfqpoint{6.958650in}{0.696659in}}%
\pgfpathlineto{\pgfqpoint{6.937260in}{0.696659in}}%
\pgfpathlineto{\pgfqpoint{6.915870in}{0.696659in}}%
\pgfpathlineto{\pgfqpoint{6.894480in}{0.717019in}}%
\pgfpathlineto{\pgfqpoint{6.873090in}{0.717019in}}%
\pgfpathlineto{\pgfqpoint{6.851700in}{0.717019in}}%
\pgfpathlineto{\pgfqpoint{6.830310in}{0.717019in}}%
\pgfpathlineto{\pgfqpoint{6.808920in}{0.717019in}}%
\pgfpathlineto{\pgfqpoint{6.787530in}{0.717019in}}%
\pgfpathlineto{\pgfqpoint{6.766140in}{0.717019in}}%
\pgfpathlineto{\pgfqpoint{6.744750in}{0.717019in}}%
\pgfpathlineto{\pgfqpoint{6.723360in}{0.717019in}}%
\pgfpathlineto{\pgfqpoint{6.701970in}{0.717019in}}%
\pgfpathlineto{\pgfqpoint{6.680580in}{0.717019in}}%
\pgfpathlineto{\pgfqpoint{6.659190in}{0.717019in}}%
\pgfpathlineto{\pgfqpoint{6.637800in}{0.717019in}}%
\pgfpathlineto{\pgfqpoint{6.616410in}{0.717019in}}%
\pgfpathlineto{\pgfqpoint{6.595020in}{0.717019in}}%
\pgfpathlineto{\pgfqpoint{6.573630in}{0.717019in}}%
\pgfpathlineto{\pgfqpoint{6.552240in}{0.736615in}}%
\pgfpathlineto{\pgfqpoint{6.530850in}{0.736615in}}%
\pgfpathlineto{\pgfqpoint{6.509460in}{0.736615in}}%
\pgfpathlineto{\pgfqpoint{6.488070in}{0.736615in}}%
\pgfpathlineto{\pgfqpoint{6.466680in}{0.736615in}}%
\pgfpathlineto{\pgfqpoint{6.445290in}{0.736615in}}%
\pgfpathlineto{\pgfqpoint{6.423900in}{0.736615in}}%
\pgfpathlineto{\pgfqpoint{6.402510in}{0.736615in}}%
\pgfpathlineto{\pgfqpoint{6.381120in}{0.736615in}}%
\pgfpathlineto{\pgfqpoint{6.359730in}{0.736615in}}%
\pgfpathlineto{\pgfqpoint{6.338340in}{0.736615in}}%
\pgfpathlineto{\pgfqpoint{6.316950in}{0.736615in}}%
\pgfpathlineto{\pgfqpoint{6.295560in}{0.736615in}}%
\pgfpathlineto{\pgfqpoint{6.274170in}{0.736615in}}%
\pgfpathlineto{\pgfqpoint{6.252780in}{0.736615in}}%
\pgfpathlineto{\pgfqpoint{6.231390in}{0.736615in}}%
\pgfpathlineto{\pgfqpoint{6.210000in}{0.736615in}}%
\pgfpathlineto{\pgfqpoint{6.188610in}{0.736615in}}%
\pgfpathlineto{\pgfqpoint{6.167220in}{0.736615in}}%
\pgfpathlineto{\pgfqpoint{6.145830in}{0.736615in}}%
\pgfpathlineto{\pgfqpoint{6.124440in}{0.758721in}}%
\pgfpathlineto{\pgfqpoint{6.103050in}{0.758721in}}%
\pgfpathlineto{\pgfqpoint{6.081660in}{0.758721in}}%
\pgfpathlineto{\pgfqpoint{6.060270in}{0.775948in}}%
\pgfpathlineto{\pgfqpoint{6.038880in}{0.775948in}}%
\pgfpathlineto{\pgfqpoint{6.017490in}{0.775948in}}%
\pgfpathlineto{\pgfqpoint{5.996100in}{0.775948in}}%
\pgfpathlineto{\pgfqpoint{5.974710in}{0.775948in}}%
\pgfpathlineto{\pgfqpoint{5.953320in}{0.775948in}}%
\pgfpathlineto{\pgfqpoint{5.931930in}{0.790334in}}%
\pgfpathlineto{\pgfqpoint{5.910540in}{0.790334in}}%
\pgfpathlineto{\pgfqpoint{5.889150in}{0.790334in}}%
\pgfpathlineto{\pgfqpoint{5.867760in}{0.790334in}}%
\pgfpathlineto{\pgfqpoint{5.846370in}{0.790334in}}%
\pgfpathlineto{\pgfqpoint{5.824980in}{0.790334in}}%
\pgfpathlineto{\pgfqpoint{5.803590in}{0.790334in}}%
\pgfpathlineto{\pgfqpoint{5.782200in}{0.790334in}}%
\pgfpathlineto{\pgfqpoint{5.760810in}{0.790334in}}%
\pgfpathlineto{\pgfqpoint{5.739420in}{0.867297in}}%
\pgfpathlineto{\pgfqpoint{5.718030in}{0.867297in}}%
\pgfpathlineto{\pgfqpoint{5.696640in}{0.867297in}}%
\pgfpathlineto{\pgfqpoint{5.675250in}{0.867297in}}%
\pgfpathlineto{\pgfqpoint{5.653860in}{0.867297in}}%
\pgfpathlineto{\pgfqpoint{5.632470in}{0.867297in}}%
\pgfpathlineto{\pgfqpoint{5.611080in}{0.867297in}}%
\pgfpathlineto{\pgfqpoint{5.589690in}{0.867297in}}%
\pgfpathlineto{\pgfqpoint{5.568300in}{0.867297in}}%
\pgfpathlineto{\pgfqpoint{5.546910in}{0.867297in}}%
\pgfpathlineto{\pgfqpoint{5.525520in}{0.867297in}}%
\pgfpathlineto{\pgfqpoint{5.504130in}{0.910598in}}%
\pgfpathlineto{\pgfqpoint{5.482740in}{0.910598in}}%
\pgfpathlineto{\pgfqpoint{5.461350in}{0.934857in}}%
\pgfpathlineto{\pgfqpoint{5.439960in}{0.994501in}}%
\pgfpathlineto{\pgfqpoint{5.418570in}{1.003985in}}%
\pgfpathlineto{\pgfqpoint{5.397180in}{1.012978in}}%
\pgfpathlineto{\pgfqpoint{5.375790in}{1.012978in}}%
\pgfpathlineto{\pgfqpoint{5.354400in}{1.012978in}}%
\pgfpathlineto{\pgfqpoint{5.333010in}{1.012978in}}%
\pgfpathlineto{\pgfqpoint{5.311620in}{1.012978in}}%
\pgfpathlineto{\pgfqpoint{5.290230in}{1.110799in}}%
\pgfpathlineto{\pgfqpoint{5.268840in}{1.110799in}}%
\pgfpathlineto{\pgfqpoint{5.247450in}{1.110799in}}%
\pgfpathlineto{\pgfqpoint{5.226060in}{1.110799in}}%
\pgfpathlineto{\pgfqpoint{5.204670in}{1.132910in}}%
\pgfpathlineto{\pgfqpoint{5.183280in}{1.152193in}}%
\pgfpathlineto{\pgfqpoint{5.161890in}{1.320104in}}%
\pgfpathlineto{\pgfqpoint{5.140500in}{1.391721in}}%
\pgfpathlineto{\pgfqpoint{5.119110in}{1.686627in}}%
\pgfpathlineto{\pgfqpoint{5.097720in}{1.686627in}}%
\pgfpathlineto{\pgfqpoint{5.076330in}{1.686627in}}%
\pgfpathlineto{\pgfqpoint{5.054940in}{1.686627in}}%
\pgfpathlineto{\pgfqpoint{5.033550in}{1.686627in}}%
\pgfpathlineto{\pgfqpoint{5.012160in}{1.686627in}}%
\pgfpathlineto{\pgfqpoint{4.990770in}{1.686627in}}%
\pgfpathlineto{\pgfqpoint{4.969380in}{1.686627in}}%
\pgfpathlineto{\pgfqpoint{4.947990in}{1.686627in}}%
\pgfpathlineto{\pgfqpoint{4.926600in}{1.686627in}}%
\pgfpathlineto{\pgfqpoint{4.905210in}{1.686627in}}%
\pgfpathlineto{\pgfqpoint{4.883820in}{1.686627in}}%
\pgfpathlineto{\pgfqpoint{4.862430in}{1.686627in}}%
\pgfpathlineto{\pgfqpoint{4.841040in}{1.686627in}}%
\pgfpathlineto{\pgfqpoint{4.819650in}{1.686627in}}%
\pgfpathlineto{\pgfqpoint{4.798260in}{1.686627in}}%
\pgfpathlineto{\pgfqpoint{4.776870in}{1.686627in}}%
\pgfpathlineto{\pgfqpoint{4.755480in}{1.686627in}}%
\pgfpathlineto{\pgfqpoint{4.734090in}{1.686627in}}%
\pgfpathlineto{\pgfqpoint{4.712700in}{1.686627in}}%
\pgfpathlineto{\pgfqpoint{4.691310in}{1.686627in}}%
\pgfpathlineto{\pgfqpoint{4.669920in}{1.686627in}}%
\pgfpathlineto{\pgfqpoint{4.648530in}{1.686627in}}%
\pgfpathlineto{\pgfqpoint{4.627140in}{1.686627in}}%
\pgfpathlineto{\pgfqpoint{4.605750in}{1.686627in}}%
\pgfpathlineto{\pgfqpoint{4.584360in}{1.686627in}}%
\pgfpathlineto{\pgfqpoint{4.562970in}{1.686627in}}%
\pgfpathlineto{\pgfqpoint{4.541580in}{1.686627in}}%
\pgfpathlineto{\pgfqpoint{4.520190in}{1.686627in}}%
\pgfpathlineto{\pgfqpoint{4.498800in}{1.686627in}}%
\pgfpathlineto{\pgfqpoint{4.477410in}{1.686627in}}%
\pgfpathlineto{\pgfqpoint{4.456020in}{1.686627in}}%
\pgfpathlineto{\pgfqpoint{4.434630in}{1.686627in}}%
\pgfpathlineto{\pgfqpoint{4.413240in}{1.686627in}}%
\pgfpathlineto{\pgfqpoint{4.391850in}{1.686627in}}%
\pgfpathlineto{\pgfqpoint{4.370460in}{1.686627in}}%
\pgfpathlineto{\pgfqpoint{4.349070in}{1.686627in}}%
\pgfpathlineto{\pgfqpoint{4.327680in}{1.686627in}}%
\pgfpathlineto{\pgfqpoint{4.306290in}{1.686627in}}%
\pgfpathlineto{\pgfqpoint{4.284900in}{1.686627in}}%
\pgfpathlineto{\pgfqpoint{4.263510in}{1.686627in}}%
\pgfpathlineto{\pgfqpoint{4.242120in}{1.686627in}}%
\pgfpathlineto{\pgfqpoint{4.220730in}{1.686627in}}%
\pgfpathlineto{\pgfqpoint{4.199340in}{1.686627in}}%
\pgfpathlineto{\pgfqpoint{4.177950in}{1.686627in}}%
\pgfpathlineto{\pgfqpoint{4.156560in}{1.686627in}}%
\pgfpathlineto{\pgfqpoint{4.135170in}{1.686627in}}%
\pgfpathlineto{\pgfqpoint{4.113780in}{1.686627in}}%
\pgfpathlineto{\pgfqpoint{4.092390in}{1.686627in}}%
\pgfpathlineto{\pgfqpoint{4.071000in}{1.686627in}}%
\pgfpathlineto{\pgfqpoint{4.049610in}{1.686627in}}%
\pgfpathlineto{\pgfqpoint{4.028220in}{1.686627in}}%
\pgfpathlineto{\pgfqpoint{4.006830in}{1.686627in}}%
\pgfpathlineto{\pgfqpoint{3.985440in}{1.686627in}}%
\pgfpathlineto{\pgfqpoint{3.964050in}{1.686627in}}%
\pgfpathlineto{\pgfqpoint{3.942660in}{1.686627in}}%
\pgfpathlineto{\pgfqpoint{3.921270in}{1.686627in}}%
\pgfpathlineto{\pgfqpoint{3.899880in}{1.686627in}}%
\pgfpathlineto{\pgfqpoint{3.878490in}{1.691491in}}%
\pgfpathlineto{\pgfqpoint{3.857100in}{1.691491in}}%
\pgfpathlineto{\pgfqpoint{3.835710in}{1.691491in}}%
\pgfpathlineto{\pgfqpoint{3.814320in}{1.691491in}}%
\pgfpathlineto{\pgfqpoint{3.792930in}{1.691491in}}%
\pgfpathlineto{\pgfqpoint{3.771540in}{1.691491in}}%
\pgfpathlineto{\pgfqpoint{3.750150in}{1.691491in}}%
\pgfpathlineto{\pgfqpoint{3.728760in}{1.691491in}}%
\pgfpathlineto{\pgfqpoint{3.707370in}{1.691491in}}%
\pgfpathlineto{\pgfqpoint{3.685980in}{1.691491in}}%
\pgfpathlineto{\pgfqpoint{3.664590in}{1.691491in}}%
\pgfpathlineto{\pgfqpoint{3.643200in}{1.691491in}}%
\pgfpathlineto{\pgfqpoint{3.621810in}{1.691491in}}%
\pgfpathlineto{\pgfqpoint{3.600420in}{1.691491in}}%
\pgfpathlineto{\pgfqpoint{3.579030in}{1.691491in}}%
\pgfpathlineto{\pgfqpoint{3.557640in}{1.691491in}}%
\pgfpathlineto{\pgfqpoint{3.536250in}{1.691491in}}%
\pgfpathlineto{\pgfqpoint{3.514860in}{1.691491in}}%
\pgfpathlineto{\pgfqpoint{3.493470in}{1.691491in}}%
\pgfpathlineto{\pgfqpoint{3.472080in}{1.691491in}}%
\pgfpathlineto{\pgfqpoint{3.450690in}{1.691491in}}%
\pgfpathlineto{\pgfqpoint{3.429300in}{1.691491in}}%
\pgfpathlineto{\pgfqpoint{3.407910in}{1.719381in}}%
\pgfpathlineto{\pgfqpoint{3.386520in}{1.719381in}}%
\pgfpathlineto{\pgfqpoint{3.365130in}{1.719381in}}%
\pgfpathlineto{\pgfqpoint{3.343740in}{1.719381in}}%
\pgfpathlineto{\pgfqpoint{3.322350in}{1.719381in}}%
\pgfpathlineto{\pgfqpoint{3.300960in}{1.719381in}}%
\pgfpathlineto{\pgfqpoint{3.279570in}{1.719381in}}%
\pgfpathlineto{\pgfqpoint{3.258180in}{1.719381in}}%
\pgfpathlineto{\pgfqpoint{3.236790in}{1.719381in}}%
\pgfpathlineto{\pgfqpoint{3.215400in}{1.719381in}}%
\pgfpathlineto{\pgfqpoint{3.194010in}{1.719381in}}%
\pgfpathlineto{\pgfqpoint{3.172620in}{1.719381in}}%
\pgfpathlineto{\pgfqpoint{3.151230in}{1.719381in}}%
\pgfpathlineto{\pgfqpoint{3.129840in}{1.719381in}}%
\pgfpathlineto{\pgfqpoint{3.108450in}{1.719381in}}%
\pgfpathlineto{\pgfqpoint{3.087060in}{1.719381in}}%
\pgfpathlineto{\pgfqpoint{3.065670in}{1.719381in}}%
\pgfpathlineto{\pgfqpoint{3.044280in}{1.719381in}}%
\pgfpathlineto{\pgfqpoint{3.022890in}{1.719381in}}%
\pgfpathlineto{\pgfqpoint{3.001500in}{1.719381in}}%
\pgfpathlineto{\pgfqpoint{2.980110in}{1.719381in}}%
\pgfpathlineto{\pgfqpoint{2.958720in}{1.719381in}}%
\pgfpathlineto{\pgfqpoint{2.937330in}{1.719381in}}%
\pgfpathlineto{\pgfqpoint{2.915940in}{1.719381in}}%
\pgfpathlineto{\pgfqpoint{2.894550in}{1.719381in}}%
\pgfpathlineto{\pgfqpoint{2.873160in}{1.719381in}}%
\pgfpathlineto{\pgfqpoint{2.851770in}{1.719381in}}%
\pgfpathlineto{\pgfqpoint{2.830380in}{1.719381in}}%
\pgfpathlineto{\pgfqpoint{2.808990in}{1.719381in}}%
\pgfpathlineto{\pgfqpoint{2.787600in}{1.719381in}}%
\pgfpathlineto{\pgfqpoint{2.766210in}{1.719381in}}%
\pgfpathlineto{\pgfqpoint{2.744820in}{1.719381in}}%
\pgfpathlineto{\pgfqpoint{2.723430in}{1.719381in}}%
\pgfpathlineto{\pgfqpoint{2.702040in}{1.719381in}}%
\pgfpathlineto{\pgfqpoint{2.680650in}{1.719381in}}%
\pgfpathlineto{\pgfqpoint{2.659260in}{1.719381in}}%
\pgfpathlineto{\pgfqpoint{2.637870in}{1.719381in}}%
\pgfpathlineto{\pgfqpoint{2.616480in}{1.719381in}}%
\pgfpathlineto{\pgfqpoint{2.595090in}{1.719381in}}%
\pgfpathlineto{\pgfqpoint{2.573700in}{1.719381in}}%
\pgfpathlineto{\pgfqpoint{2.552310in}{1.719381in}}%
\pgfpathlineto{\pgfqpoint{2.530920in}{1.719381in}}%
\pgfpathlineto{\pgfqpoint{2.509530in}{1.719381in}}%
\pgfpathlineto{\pgfqpoint{2.488140in}{1.719381in}}%
\pgfpathlineto{\pgfqpoint{2.466750in}{1.797970in}}%
\pgfpathlineto{\pgfqpoint{2.445360in}{1.797970in}}%
\pgfpathlineto{\pgfqpoint{2.423970in}{1.797970in}}%
\pgfpathlineto{\pgfqpoint{2.402580in}{1.797970in}}%
\pgfpathlineto{\pgfqpoint{2.381190in}{1.797970in}}%
\pgfpathlineto{\pgfqpoint{2.359800in}{1.797970in}}%
\pgfpathlineto{\pgfqpoint{2.338410in}{1.797970in}}%
\pgfpathlineto{\pgfqpoint{2.317020in}{1.797970in}}%
\pgfpathlineto{\pgfqpoint{2.295630in}{1.797970in}}%
\pgfpathlineto{\pgfqpoint{2.274240in}{1.797970in}}%
\pgfpathlineto{\pgfqpoint{2.252850in}{1.797970in}}%
\pgfpathlineto{\pgfqpoint{2.231460in}{1.797970in}}%
\pgfpathlineto{\pgfqpoint{2.210070in}{1.797970in}}%
\pgfpathlineto{\pgfqpoint{2.188680in}{1.797970in}}%
\pgfpathlineto{\pgfqpoint{2.167290in}{1.797970in}}%
\pgfpathlineto{\pgfqpoint{2.145900in}{1.797970in}}%
\pgfpathlineto{\pgfqpoint{2.124510in}{1.857640in}}%
\pgfpathlineto{\pgfqpoint{2.103120in}{1.857640in}}%
\pgfpathlineto{\pgfqpoint{2.081730in}{1.857640in}}%
\pgfpathlineto{\pgfqpoint{2.060340in}{1.857640in}}%
\pgfpathlineto{\pgfqpoint{2.038950in}{1.857640in}}%
\pgfpathlineto{\pgfqpoint{2.017560in}{1.857640in}}%
\pgfpathlineto{\pgfqpoint{1.996170in}{1.857640in}}%
\pgfpathlineto{\pgfqpoint{1.974780in}{1.857640in}}%
\pgfpathlineto{\pgfqpoint{1.953390in}{1.857640in}}%
\pgfpathlineto{\pgfqpoint{1.932000in}{1.857640in}}%
\pgfpathlineto{\pgfqpoint{1.910610in}{1.857640in}}%
\pgfpathlineto{\pgfqpoint{1.889220in}{1.857640in}}%
\pgfpathlineto{\pgfqpoint{1.867830in}{1.857640in}}%
\pgfpathlineto{\pgfqpoint{1.846440in}{1.857640in}}%
\pgfpathlineto{\pgfqpoint{1.825050in}{1.857640in}}%
\pgfpathlineto{\pgfqpoint{1.803660in}{1.857640in}}%
\pgfpathlineto{\pgfqpoint{1.782270in}{1.880509in}}%
\pgfpathlineto{\pgfqpoint{1.760880in}{1.880509in}}%
\pgfpathlineto{\pgfqpoint{1.739490in}{1.880509in}}%
\pgfpathlineto{\pgfqpoint{1.718100in}{1.880509in}}%
\pgfpathlineto{\pgfqpoint{1.696710in}{1.880509in}}%
\pgfpathlineto{\pgfqpoint{1.675320in}{1.889355in}}%
\pgfpathlineto{\pgfqpoint{1.653930in}{1.911320in}}%
\pgfpathlineto{\pgfqpoint{1.632540in}{1.911320in}}%
\pgfpathlineto{\pgfqpoint{1.611150in}{1.917751in}}%
\pgfpathlineto{\pgfqpoint{1.589760in}{1.917751in}}%
\pgfpathlineto{\pgfqpoint{1.568370in}{1.917751in}}%
\pgfpathlineto{\pgfqpoint{1.546980in}{1.917751in}}%
\pgfpathlineto{\pgfqpoint{1.525590in}{1.917751in}}%
\pgfpathlineto{\pgfqpoint{1.504200in}{1.917751in}}%
\pgfpathlineto{\pgfqpoint{1.482810in}{1.938708in}}%
\pgfpathlineto{\pgfqpoint{1.461420in}{1.938708in}}%
\pgfpathlineto{\pgfqpoint{1.440030in}{1.938708in}}%
\pgfpathlineto{\pgfqpoint{1.418640in}{1.938708in}}%
\pgfpathlineto{\pgfqpoint{1.397250in}{1.938708in}}%
\pgfpathlineto{\pgfqpoint{1.375860in}{1.938708in}}%
\pgfpathlineto{\pgfqpoint{1.354470in}{1.938708in}}%
\pgfpathlineto{\pgfqpoint{1.333080in}{1.938708in}}%
\pgfpathlineto{\pgfqpoint{1.311690in}{1.942727in}}%
\pgfpathlineto{\pgfqpoint{1.290300in}{1.942727in}}%
\pgfpathlineto{\pgfqpoint{1.268910in}{1.950811in}}%
\pgfpathlineto{\pgfqpoint{1.247520in}{1.950811in}}%
\pgfpathlineto{\pgfqpoint{1.226130in}{1.950811in}}%
\pgfpathlineto{\pgfqpoint{1.204740in}{1.950811in}}%
\pgfpathlineto{\pgfqpoint{1.183350in}{2.021211in}}%
\pgfpathlineto{\pgfqpoint{1.161960in}{2.021211in}}%
\pgfpathlineto{\pgfqpoint{1.140570in}{2.021211in}}%
\pgfpathlineto{\pgfqpoint{1.119180in}{2.021211in}}%
\pgfpathlineto{\pgfqpoint{1.097790in}{2.042876in}}%
\pgfpathlineto{\pgfqpoint{1.076400in}{2.042876in}}%
\pgfpathlineto{\pgfqpoint{1.055010in}{2.042876in}}%
\pgfpathlineto{\pgfqpoint{1.033620in}{2.042876in}}%
\pgfpathlineto{\pgfqpoint{1.012230in}{2.042876in}}%
\pgfpathlineto{\pgfqpoint{0.990840in}{2.042876in}}%
\pgfpathlineto{\pgfqpoint{0.969450in}{2.042876in}}%
\pgfpathlineto{\pgfqpoint{0.948060in}{2.124146in}}%
\pgfpathlineto{\pgfqpoint{0.926670in}{2.239418in}}%
\pgfpathlineto{\pgfqpoint{0.905280in}{2.257444in}}%
\pgfpathlineto{\pgfqpoint{0.883890in}{2.467246in}}%
\pgfpathlineto{\pgfqpoint{0.862500in}{2.493381in}}%
\pgfpathclose%
\pgfusepath{fill}%
\end{pgfscope}%
\begin{pgfscope}%
\pgfpathrectangle{\pgfqpoint{0.862500in}{0.375000in}}{\pgfqpoint{5.347500in}{2.265000in}}%
\pgfusepath{clip}%
\pgfsetroundcap%
\pgfsetroundjoin%
\pgfsetlinewidth{1.505625pt}%
\definecolor{currentstroke}{rgb}{0.121569,0.466667,0.705882}%
\pgfsetstrokecolor{currentstroke}%
\pgfsetdash{}{0pt}%
\pgfpathmoveto{\pgfqpoint{0.862500in}{2.475043in}}%
\pgfpathlineto{\pgfqpoint{0.883890in}{2.468469in}}%
\pgfpathlineto{\pgfqpoint{0.905280in}{2.406493in}}%
\pgfpathlineto{\pgfqpoint{0.926670in}{2.367545in}}%
\pgfpathlineto{\pgfqpoint{0.948060in}{2.347988in}}%
\pgfpathlineto{\pgfqpoint{0.969450in}{2.295347in}}%
\pgfpathlineto{\pgfqpoint{0.990840in}{2.274920in}}%
\pgfpathlineto{\pgfqpoint{1.033620in}{2.273780in}}%
\pgfpathlineto{\pgfqpoint{1.055010in}{2.273780in}}%
\pgfpathlineto{\pgfqpoint{1.076400in}{2.238653in}}%
\pgfpathlineto{\pgfqpoint{1.119180in}{2.238653in}}%
\pgfpathlineto{\pgfqpoint{1.140570in}{2.173748in}}%
\pgfpathlineto{\pgfqpoint{1.161960in}{2.173748in}}%
\pgfpathlineto{\pgfqpoint{1.183350in}{2.158666in}}%
\pgfpathlineto{\pgfqpoint{1.204740in}{2.158666in}}%
\pgfpathlineto{\pgfqpoint{1.226130in}{2.097371in}}%
\pgfpathlineto{\pgfqpoint{1.311690in}{2.097371in}}%
\pgfpathlineto{\pgfqpoint{1.333080in}{2.061702in}}%
\pgfpathlineto{\pgfqpoint{1.354470in}{2.061702in}}%
\pgfpathlineto{\pgfqpoint{1.418640in}{1.998921in}}%
\pgfpathlineto{\pgfqpoint{1.482810in}{1.998921in}}%
\pgfpathlineto{\pgfqpoint{1.504200in}{1.987475in}}%
\pgfpathlineto{\pgfqpoint{1.675320in}{1.986013in}}%
\pgfpathlineto{\pgfqpoint{1.696710in}{1.949323in}}%
\pgfpathlineto{\pgfqpoint{1.718100in}{1.934125in}}%
\pgfpathlineto{\pgfqpoint{1.910610in}{1.932893in}}%
\pgfpathlineto{\pgfqpoint{1.932000in}{1.899103in}}%
\pgfpathlineto{\pgfqpoint{1.996170in}{1.899103in}}%
\pgfpathlineto{\pgfqpoint{2.017560in}{1.893532in}}%
\pgfpathlineto{\pgfqpoint{2.060340in}{1.893532in}}%
\pgfpathlineto{\pgfqpoint{2.081730in}{1.875833in}}%
\pgfpathlineto{\pgfqpoint{2.103120in}{1.873219in}}%
\pgfpathlineto{\pgfqpoint{2.295630in}{1.873219in}}%
\pgfpathlineto{\pgfqpoint{2.317020in}{1.825599in}}%
\pgfpathlineto{\pgfqpoint{2.402580in}{1.825599in}}%
\pgfpathlineto{\pgfqpoint{2.423970in}{1.755629in}}%
\pgfpathlineto{\pgfqpoint{2.830380in}{1.755629in}}%
\pgfpathlineto{\pgfqpoint{2.851770in}{1.710086in}}%
\pgfpathlineto{\pgfqpoint{5.012160in}{1.710086in}}%
\pgfpathlineto{\pgfqpoint{5.033550in}{1.694618in}}%
\pgfpathlineto{\pgfqpoint{5.161890in}{1.694618in}}%
\pgfpathlineto{\pgfqpoint{5.183280in}{1.667257in}}%
\pgfpathlineto{\pgfqpoint{5.333010in}{1.667257in}}%
\pgfpathlineto{\pgfqpoint{5.354400in}{1.645879in}}%
\pgfpathlineto{\pgfqpoint{5.439960in}{1.645879in}}%
\pgfpathlineto{\pgfqpoint{5.461350in}{1.628438in}}%
\pgfpathlineto{\pgfqpoint{5.589690in}{1.628438in}}%
\pgfpathlineto{\pgfqpoint{5.611080in}{1.619710in}}%
\pgfpathlineto{\pgfqpoint{5.846370in}{1.619710in}}%
\pgfpathlineto{\pgfqpoint{5.867760in}{1.609318in}}%
\pgfpathlineto{\pgfqpoint{6.223889in}{1.609318in}}%
\pgfpathlineto{\pgfqpoint{6.223889in}{1.609318in}}%
\pgfusepath{stroke}%
\end{pgfscope}%
\begin{pgfscope}%
\pgfpathrectangle{\pgfqpoint{0.862500in}{0.375000in}}{\pgfqpoint{5.347500in}{2.265000in}}%
\pgfusepath{clip}%
\pgfsetroundcap%
\pgfsetroundjoin%
\pgfsetlinewidth{1.505625pt}%
\definecolor{currentstroke}{rgb}{1.000000,0.498039,0.054902}%
\pgfsetstrokecolor{currentstroke}%
\pgfsetdash{}{0pt}%
\pgfpathmoveto{\pgfqpoint{0.862500in}{2.493039in}}%
\pgfpathlineto{\pgfqpoint{0.883890in}{2.449455in}}%
\pgfpathlineto{\pgfqpoint{0.905280in}{2.419117in}}%
\pgfpathlineto{\pgfqpoint{0.948060in}{2.419117in}}%
\pgfpathlineto{\pgfqpoint{0.990840in}{2.344763in}}%
\pgfpathlineto{\pgfqpoint{1.012230in}{2.344763in}}%
\pgfpathlineto{\pgfqpoint{1.033620in}{2.292464in}}%
\pgfpathlineto{\pgfqpoint{1.055010in}{2.257084in}}%
\pgfpathlineto{\pgfqpoint{1.119180in}{2.257084in}}%
\pgfpathlineto{\pgfqpoint{1.140570in}{2.231724in}}%
\pgfpathlineto{\pgfqpoint{1.183350in}{2.231724in}}%
\pgfpathlineto{\pgfqpoint{1.204740in}{2.228576in}}%
\pgfpathlineto{\pgfqpoint{1.247520in}{2.228576in}}%
\pgfpathlineto{\pgfqpoint{1.268910in}{2.212818in}}%
\pgfpathlineto{\pgfqpoint{1.290300in}{2.212818in}}%
\pgfpathlineto{\pgfqpoint{1.311690in}{2.207077in}}%
\pgfpathlineto{\pgfqpoint{1.333080in}{2.164408in}}%
\pgfpathlineto{\pgfqpoint{1.354470in}{2.152449in}}%
\pgfpathlineto{\pgfqpoint{1.375860in}{2.137849in}}%
\pgfpathlineto{\pgfqpoint{1.397250in}{2.101983in}}%
\pgfpathlineto{\pgfqpoint{1.418640in}{2.099756in}}%
\pgfpathlineto{\pgfqpoint{1.440030in}{2.090000in}}%
\pgfpathlineto{\pgfqpoint{1.461420in}{2.065558in}}%
\pgfpathlineto{\pgfqpoint{1.482810in}{2.036039in}}%
\pgfpathlineto{\pgfqpoint{1.504200in}{2.020439in}}%
\pgfpathlineto{\pgfqpoint{1.525590in}{1.905941in}}%
\pgfpathlineto{\pgfqpoint{1.546980in}{1.859449in}}%
\pgfpathlineto{\pgfqpoint{1.568370in}{1.766774in}}%
\pgfpathlineto{\pgfqpoint{1.589760in}{1.722902in}}%
\pgfpathlineto{\pgfqpoint{1.611150in}{1.691722in}}%
\pgfpathlineto{\pgfqpoint{1.632540in}{1.686279in}}%
\pgfpathlineto{\pgfqpoint{1.653930in}{1.511691in}}%
\pgfpathlineto{\pgfqpoint{1.718100in}{1.511691in}}%
\pgfpathlineto{\pgfqpoint{1.739490in}{1.505761in}}%
\pgfpathlineto{\pgfqpoint{1.910610in}{1.505761in}}%
\pgfpathlineto{\pgfqpoint{1.932000in}{1.499622in}}%
\pgfpathlineto{\pgfqpoint{1.953390in}{1.499622in}}%
\pgfpathlineto{\pgfqpoint{1.996170in}{1.479248in}}%
\pgfpathlineto{\pgfqpoint{2.103120in}{1.479248in}}%
\pgfpathlineto{\pgfqpoint{2.124510in}{1.473692in}}%
\pgfpathlineto{\pgfqpoint{2.295630in}{1.473692in}}%
\pgfpathlineto{\pgfqpoint{2.317020in}{1.463104in}}%
\pgfpathlineto{\pgfqpoint{2.552310in}{1.461812in}}%
\pgfpathlineto{\pgfqpoint{2.573700in}{1.453771in}}%
\pgfpathlineto{\pgfqpoint{2.980110in}{1.452300in}}%
\pgfpathlineto{\pgfqpoint{3.001500in}{1.414596in}}%
\pgfpathlineto{\pgfqpoint{3.044280in}{1.414596in}}%
\pgfpathlineto{\pgfqpoint{3.065670in}{1.380776in}}%
\pgfpathlineto{\pgfqpoint{3.172620in}{1.380776in}}%
\pgfpathlineto{\pgfqpoint{3.194010in}{1.368267in}}%
\pgfpathlineto{\pgfqpoint{3.258180in}{1.368267in}}%
\pgfpathlineto{\pgfqpoint{3.279570in}{1.363789in}}%
\pgfpathlineto{\pgfqpoint{3.300960in}{1.363789in}}%
\pgfpathlineto{\pgfqpoint{3.322350in}{1.359065in}}%
\pgfpathlineto{\pgfqpoint{3.343740in}{1.351071in}}%
\pgfpathlineto{\pgfqpoint{3.750150in}{1.349830in}}%
\pgfpathlineto{\pgfqpoint{3.771540in}{1.328422in}}%
\pgfpathlineto{\pgfqpoint{3.878490in}{1.328412in}}%
\pgfpathlineto{\pgfqpoint{3.899880in}{1.280313in}}%
\pgfpathlineto{\pgfqpoint{4.456020in}{1.279374in}}%
\pgfpathlineto{\pgfqpoint{4.477410in}{1.276139in}}%
\pgfpathlineto{\pgfqpoint{4.712700in}{1.276139in}}%
\pgfpathlineto{\pgfqpoint{4.734090in}{1.254172in}}%
\pgfpathlineto{\pgfqpoint{4.926600in}{1.254172in}}%
\pgfpathlineto{\pgfqpoint{4.947990in}{1.239614in}}%
\pgfpathlineto{\pgfqpoint{5.439960in}{1.238917in}}%
\pgfpathlineto{\pgfqpoint{5.461350in}{1.209272in}}%
\pgfpathlineto{\pgfqpoint{5.632470in}{1.209272in}}%
\pgfpathlineto{\pgfqpoint{5.653860in}{1.166247in}}%
\pgfpathlineto{\pgfqpoint{6.167220in}{1.166247in}}%
\pgfpathlineto{\pgfqpoint{6.188610in}{1.151962in}}%
\pgfpathlineto{\pgfqpoint{6.223889in}{1.151962in}}%
\pgfpathlineto{\pgfqpoint{6.223889in}{1.151962in}}%
\pgfusepath{stroke}%
\end{pgfscope}%
\begin{pgfscope}%
\pgfpathrectangle{\pgfqpoint{0.862500in}{0.375000in}}{\pgfqpoint{5.347500in}{2.265000in}}%
\pgfusepath{clip}%
\pgfsetroundcap%
\pgfsetroundjoin%
\pgfsetlinewidth{1.505625pt}%
\definecolor{currentstroke}{rgb}{0.172549,0.627451,0.172549}%
\pgfsetstrokecolor{currentstroke}%
\pgfsetdash{}{0pt}%
\pgfpathmoveto{\pgfqpoint{0.862500in}{2.415388in}}%
\pgfpathlineto{\pgfqpoint{0.883890in}{2.399142in}}%
\pgfpathlineto{\pgfqpoint{0.905280in}{2.387994in}}%
\pgfpathlineto{\pgfqpoint{0.926670in}{2.387994in}}%
\pgfpathlineto{\pgfqpoint{0.948060in}{2.272988in}}%
\pgfpathlineto{\pgfqpoint{0.969450in}{2.242359in}}%
\pgfpathlineto{\pgfqpoint{0.990840in}{2.180435in}}%
\pgfpathlineto{\pgfqpoint{1.012230in}{2.164680in}}%
\pgfpathlineto{\pgfqpoint{1.119180in}{2.164680in}}%
\pgfpathlineto{\pgfqpoint{1.140570in}{2.157916in}}%
\pgfpathlineto{\pgfqpoint{1.204740in}{2.157916in}}%
\pgfpathlineto{\pgfqpoint{1.226130in}{2.145571in}}%
\pgfpathlineto{\pgfqpoint{1.290300in}{2.145571in}}%
\pgfpathlineto{\pgfqpoint{1.311690in}{2.133864in}}%
\pgfpathlineto{\pgfqpoint{1.354470in}{2.133864in}}%
\pgfpathlineto{\pgfqpoint{1.375860in}{2.103738in}}%
\pgfpathlineto{\pgfqpoint{1.397250in}{2.103738in}}%
\pgfpathlineto{\pgfqpoint{1.418640in}{2.064862in}}%
\pgfpathlineto{\pgfqpoint{1.568370in}{2.064862in}}%
\pgfpathlineto{\pgfqpoint{1.589760in}{1.988336in}}%
\pgfpathlineto{\pgfqpoint{1.632540in}{1.988336in}}%
\pgfpathlineto{\pgfqpoint{1.653930in}{1.827701in}}%
\pgfpathlineto{\pgfqpoint{1.782270in}{1.827701in}}%
\pgfpathlineto{\pgfqpoint{1.803660in}{1.765372in}}%
\pgfpathlineto{\pgfqpoint{1.910610in}{1.765372in}}%
\pgfpathlineto{\pgfqpoint{1.932000in}{1.714535in}}%
\pgfpathlineto{\pgfqpoint{1.953390in}{1.698908in}}%
\pgfpathlineto{\pgfqpoint{1.974780in}{1.638270in}}%
\pgfpathlineto{\pgfqpoint{2.017560in}{1.638270in}}%
\pgfpathlineto{\pgfqpoint{2.038950in}{1.565079in}}%
\pgfpathlineto{\pgfqpoint{2.081730in}{1.565079in}}%
\pgfpathlineto{\pgfqpoint{2.103120in}{1.539435in}}%
\pgfpathlineto{\pgfqpoint{2.188680in}{1.538703in}}%
\pgfpathlineto{\pgfqpoint{2.210070in}{1.533420in}}%
\pgfpathlineto{\pgfqpoint{2.231460in}{1.514934in}}%
\pgfpathlineto{\pgfqpoint{2.252850in}{1.501821in}}%
\pgfpathlineto{\pgfqpoint{2.274240in}{1.476559in}}%
\pgfpathlineto{\pgfqpoint{2.359800in}{1.476559in}}%
\pgfpathlineto{\pgfqpoint{2.381190in}{1.381772in}}%
\pgfpathlineto{\pgfqpoint{2.466750in}{1.381772in}}%
\pgfpathlineto{\pgfqpoint{2.488140in}{1.364701in}}%
\pgfpathlineto{\pgfqpoint{2.723430in}{1.364701in}}%
\pgfpathlineto{\pgfqpoint{2.744820in}{1.355832in}}%
\pgfpathlineto{\pgfqpoint{2.830380in}{1.355832in}}%
\pgfpathlineto{\pgfqpoint{2.851770in}{1.352803in}}%
\pgfpathlineto{\pgfqpoint{2.980110in}{1.352803in}}%
\pgfpathlineto{\pgfqpoint{3.001500in}{1.340273in}}%
\pgfpathlineto{\pgfqpoint{3.450690in}{1.340273in}}%
\pgfpathlineto{\pgfqpoint{3.472080in}{1.329778in}}%
\pgfpathlineto{\pgfqpoint{3.621810in}{1.329778in}}%
\pgfpathlineto{\pgfqpoint{3.643200in}{1.282618in}}%
\pgfpathlineto{\pgfqpoint{3.707370in}{1.282618in}}%
\pgfpathlineto{\pgfqpoint{3.728760in}{1.275371in}}%
\pgfpathlineto{\pgfqpoint{3.835710in}{1.275371in}}%
\pgfpathlineto{\pgfqpoint{3.857100in}{1.268181in}}%
\pgfpathlineto{\pgfqpoint{3.878490in}{1.255250in}}%
\pgfpathlineto{\pgfqpoint{4.049610in}{1.255250in}}%
\pgfpathlineto{\pgfqpoint{4.071000in}{1.252785in}}%
\pgfpathlineto{\pgfqpoint{4.156560in}{1.252785in}}%
\pgfpathlineto{\pgfqpoint{4.177950in}{1.204176in}}%
\pgfpathlineto{\pgfqpoint{4.199340in}{1.192035in}}%
\pgfpathlineto{\pgfqpoint{4.862430in}{1.192035in}}%
\pgfpathlineto{\pgfqpoint{4.883820in}{1.168129in}}%
\pgfpathlineto{\pgfqpoint{5.012160in}{1.168129in}}%
\pgfpathlineto{\pgfqpoint{5.033550in}{1.107381in}}%
\pgfpathlineto{\pgfqpoint{5.226060in}{1.107381in}}%
\pgfpathlineto{\pgfqpoint{5.247450in}{1.067418in}}%
\pgfpathlineto{\pgfqpoint{5.290230in}{1.067418in}}%
\pgfpathlineto{\pgfqpoint{5.311620in}{1.008179in}}%
\pgfpathlineto{\pgfqpoint{5.824980in}{1.008179in}}%
\pgfpathlineto{\pgfqpoint{5.846370in}{1.000381in}}%
\pgfpathlineto{\pgfqpoint{5.910540in}{1.000381in}}%
\pgfpathlineto{\pgfqpoint{5.931930in}{0.995609in}}%
\pgfpathlineto{\pgfqpoint{6.188610in}{0.995609in}}%
\pgfpathlineto{\pgfqpoint{6.188610in}{0.995609in}}%
\pgfusepath{stroke}%
\end{pgfscope}%
\begin{pgfscope}%
\pgfpathrectangle{\pgfqpoint{0.862500in}{0.375000in}}{\pgfqpoint{5.347500in}{2.265000in}}%
\pgfusepath{clip}%
\pgfsetroundcap%
\pgfsetroundjoin%
\pgfsetlinewidth{1.505625pt}%
\definecolor{currentstroke}{rgb}{0.839216,0.152941,0.156863}%
\pgfsetstrokecolor{currentstroke}%
\pgfsetdash{}{0pt}%
\pgfpathmoveto{\pgfqpoint{0.862500in}{2.487460in}}%
\pgfpathlineto{\pgfqpoint{0.883890in}{2.399324in}}%
\pgfpathlineto{\pgfqpoint{0.905280in}{2.395769in}}%
\pgfpathlineto{\pgfqpoint{0.926670in}{2.334669in}}%
\pgfpathlineto{\pgfqpoint{0.948060in}{2.334669in}}%
\pgfpathlineto{\pgfqpoint{0.969450in}{2.281473in}}%
\pgfpathlineto{\pgfqpoint{1.012230in}{2.281473in}}%
\pgfpathlineto{\pgfqpoint{1.033620in}{2.243160in}}%
\pgfpathlineto{\pgfqpoint{1.055010in}{2.243160in}}%
\pgfpathlineto{\pgfqpoint{1.076400in}{2.232017in}}%
\pgfpathlineto{\pgfqpoint{1.097790in}{2.189332in}}%
\pgfpathlineto{\pgfqpoint{1.119180in}{2.053626in}}%
\pgfpathlineto{\pgfqpoint{1.247520in}{2.053626in}}%
\pgfpathlineto{\pgfqpoint{1.268910in}{2.009410in}}%
\pgfpathlineto{\pgfqpoint{1.311690in}{2.009410in}}%
\pgfpathlineto{\pgfqpoint{1.333080in}{2.007173in}}%
\pgfpathlineto{\pgfqpoint{1.375860in}{2.007173in}}%
\pgfpathlineto{\pgfqpoint{1.418640in}{1.880634in}}%
\pgfpathlineto{\pgfqpoint{1.440030in}{1.880634in}}%
\pgfpathlineto{\pgfqpoint{1.461420in}{1.833798in}}%
\pgfpathlineto{\pgfqpoint{1.589760in}{1.833798in}}%
\pgfpathlineto{\pgfqpoint{1.611150in}{1.760111in}}%
\pgfpathlineto{\pgfqpoint{2.980110in}{1.759055in}}%
\pgfpathlineto{\pgfqpoint{3.001500in}{1.603639in}}%
\pgfpathlineto{\pgfqpoint{3.022890in}{1.510659in}}%
\pgfpathlineto{\pgfqpoint{3.087060in}{1.510659in}}%
\pgfpathlineto{\pgfqpoint{3.108450in}{1.456720in}}%
\pgfpathlineto{\pgfqpoint{3.151230in}{1.456720in}}%
\pgfpathlineto{\pgfqpoint{3.172620in}{1.436383in}}%
\pgfpathlineto{\pgfqpoint{3.194010in}{1.385412in}}%
\pgfpathlineto{\pgfqpoint{3.279570in}{1.385412in}}%
\pgfpathlineto{\pgfqpoint{3.300960in}{1.381707in}}%
\pgfpathlineto{\pgfqpoint{3.322350in}{1.381707in}}%
\pgfpathlineto{\pgfqpoint{3.343740in}{1.376734in}}%
\pgfpathlineto{\pgfqpoint{3.407910in}{1.376734in}}%
\pgfpathlineto{\pgfqpoint{3.429300in}{1.373151in}}%
\pgfpathlineto{\pgfqpoint{3.493470in}{1.373151in}}%
\pgfpathlineto{\pgfqpoint{3.514860in}{1.345667in}}%
\pgfpathlineto{\pgfqpoint{3.621810in}{1.345667in}}%
\pgfpathlineto{\pgfqpoint{3.643200in}{1.342158in}}%
\pgfpathlineto{\pgfqpoint{3.664590in}{1.342158in}}%
\pgfpathlineto{\pgfqpoint{3.685980in}{1.330813in}}%
\pgfpathlineto{\pgfqpoint{3.707370in}{1.325184in}}%
\pgfpathlineto{\pgfqpoint{3.792930in}{1.325184in}}%
\pgfpathlineto{\pgfqpoint{3.814320in}{1.304233in}}%
\pgfpathlineto{\pgfqpoint{3.835710in}{1.304233in}}%
\pgfpathlineto{\pgfqpoint{3.857100in}{1.286433in}}%
\pgfpathlineto{\pgfqpoint{4.092390in}{1.286433in}}%
\pgfpathlineto{\pgfqpoint{4.113780in}{1.262569in}}%
\pgfpathlineto{\pgfqpoint{4.135170in}{1.245238in}}%
\pgfpathlineto{\pgfqpoint{4.391850in}{1.245238in}}%
\pgfpathlineto{\pgfqpoint{4.413240in}{1.236731in}}%
\pgfpathlineto{\pgfqpoint{4.798260in}{1.236731in}}%
\pgfpathlineto{\pgfqpoint{4.819650in}{1.214200in}}%
\pgfpathlineto{\pgfqpoint{5.247450in}{1.214200in}}%
\pgfpathlineto{\pgfqpoint{5.268840in}{1.210579in}}%
\pgfpathlineto{\pgfqpoint{6.223889in}{1.210579in}}%
\pgfpathlineto{\pgfqpoint{6.223889in}{1.210579in}}%
\pgfusepath{stroke}%
\end{pgfscope}%
\begin{pgfscope}%
\pgfpathrectangle{\pgfqpoint{0.862500in}{0.375000in}}{\pgfqpoint{5.347500in}{2.265000in}}%
\pgfusepath{clip}%
\pgfsetroundcap%
\pgfsetroundjoin%
\pgfsetlinewidth{1.505625pt}%
\definecolor{currentstroke}{rgb}{0.580392,0.403922,0.741176}%
\pgfsetstrokecolor{currentstroke}%
\pgfsetdash{}{0pt}%
\pgfpathmoveto{\pgfqpoint{0.862500in}{2.514870in}}%
\pgfpathlineto{\pgfqpoint{0.883890in}{2.489554in}}%
\pgfpathlineto{\pgfqpoint{0.905280in}{2.349307in}}%
\pgfpathlineto{\pgfqpoint{0.926670in}{2.340140in}}%
\pgfpathlineto{\pgfqpoint{0.948060in}{2.267925in}}%
\pgfpathlineto{\pgfqpoint{0.969450in}{2.165934in}}%
\pgfpathlineto{\pgfqpoint{1.097790in}{2.165934in}}%
\pgfpathlineto{\pgfqpoint{1.119180in}{2.150386in}}%
\pgfpathlineto{\pgfqpoint{1.183350in}{2.150386in}}%
\pgfpathlineto{\pgfqpoint{1.204740in}{2.111470in}}%
\pgfpathlineto{\pgfqpoint{1.268910in}{2.111470in}}%
\pgfpathlineto{\pgfqpoint{1.290300in}{2.067908in}}%
\pgfpathlineto{\pgfqpoint{1.311690in}{2.067908in}}%
\pgfpathlineto{\pgfqpoint{1.333080in}{2.065994in}}%
\pgfpathlineto{\pgfqpoint{1.482810in}{2.065994in}}%
\pgfpathlineto{\pgfqpoint{1.504200in}{2.053895in}}%
\pgfpathlineto{\pgfqpoint{1.611150in}{2.053895in}}%
\pgfpathlineto{\pgfqpoint{1.632540in}{2.016615in}}%
\pgfpathlineto{\pgfqpoint{1.653930in}{2.016615in}}%
\pgfpathlineto{\pgfqpoint{1.675320in}{1.924430in}}%
\pgfpathlineto{\pgfqpoint{1.696710in}{1.916307in}}%
\pgfpathlineto{\pgfqpoint{1.782270in}{1.916307in}}%
\pgfpathlineto{\pgfqpoint{1.803660in}{1.900232in}}%
\pgfpathlineto{\pgfqpoint{2.124510in}{1.900232in}}%
\pgfpathlineto{\pgfqpoint{2.145900in}{1.869020in}}%
\pgfpathlineto{\pgfqpoint{2.466750in}{1.869020in}}%
\pgfpathlineto{\pgfqpoint{2.488140in}{1.810199in}}%
\pgfpathlineto{\pgfqpoint{3.407910in}{1.810199in}}%
\pgfpathlineto{\pgfqpoint{3.429300in}{1.760264in}}%
\pgfpathlineto{\pgfqpoint{3.878490in}{1.760264in}}%
\pgfpathlineto{\pgfqpoint{3.899880in}{1.756554in}}%
\pgfpathlineto{\pgfqpoint{5.119110in}{1.756554in}}%
\pgfpathlineto{\pgfqpoint{5.140500in}{1.532107in}}%
\pgfpathlineto{\pgfqpoint{5.161890in}{1.473262in}}%
\pgfpathlineto{\pgfqpoint{5.183280in}{1.400355in}}%
\pgfpathlineto{\pgfqpoint{5.204670in}{1.376215in}}%
\pgfpathlineto{\pgfqpoint{5.226060in}{1.332919in}}%
\pgfpathlineto{\pgfqpoint{5.290230in}{1.332919in}}%
\pgfpathlineto{\pgfqpoint{5.311620in}{1.150667in}}%
\pgfpathlineto{\pgfqpoint{5.397180in}{1.150667in}}%
\pgfpathlineto{\pgfqpoint{5.418570in}{1.128895in}}%
\pgfpathlineto{\pgfqpoint{5.439960in}{1.120510in}}%
\pgfpathlineto{\pgfqpoint{5.461350in}{1.030963in}}%
\pgfpathlineto{\pgfqpoint{5.482740in}{0.992716in}}%
\pgfpathlineto{\pgfqpoint{5.504130in}{0.992716in}}%
\pgfpathlineto{\pgfqpoint{5.525520in}{0.953002in}}%
\pgfpathlineto{\pgfqpoint{5.739420in}{0.953002in}}%
\pgfpathlineto{\pgfqpoint{5.760810in}{0.879628in}}%
\pgfpathlineto{\pgfqpoint{5.931930in}{0.879628in}}%
\pgfpathlineto{\pgfqpoint{5.953320in}{0.871838in}}%
\pgfpathlineto{\pgfqpoint{6.060270in}{0.871838in}}%
\pgfpathlineto{\pgfqpoint{6.081660in}{0.863394in}}%
\pgfpathlineto{\pgfqpoint{6.124440in}{0.863394in}}%
\pgfpathlineto{\pgfqpoint{6.145830in}{0.811625in}}%
\pgfpathlineto{\pgfqpoint{6.223889in}{0.811625in}}%
\pgfpathlineto{\pgfqpoint{6.223889in}{0.811625in}}%
\pgfusepath{stroke}%
\end{pgfscope}%
\begin{pgfscope}%
\pgfsetrectcap%
\pgfsetmiterjoin%
\pgfsetlinewidth{0.000000pt}%
\definecolor{currentstroke}{rgb}{1.000000,1.000000,1.000000}%
\pgfsetstrokecolor{currentstroke}%
\pgfsetdash{}{0pt}%
\pgfpathmoveto{\pgfqpoint{0.862500in}{0.375000in}}%
\pgfpathlineto{\pgfqpoint{0.862500in}{2.640000in}}%
\pgfusepath{}%
\end{pgfscope}%
\begin{pgfscope}%
\pgfsetrectcap%
\pgfsetmiterjoin%
\pgfsetlinewidth{0.000000pt}%
\definecolor{currentstroke}{rgb}{1.000000,1.000000,1.000000}%
\pgfsetstrokecolor{currentstroke}%
\pgfsetdash{}{0pt}%
\pgfpathmoveto{\pgfqpoint{6.210000in}{0.375000in}}%
\pgfpathlineto{\pgfqpoint{6.210000in}{2.640000in}}%
\pgfusepath{}%
\end{pgfscope}%
\begin{pgfscope}%
\pgfsetrectcap%
\pgfsetmiterjoin%
\pgfsetlinewidth{0.000000pt}%
\definecolor{currentstroke}{rgb}{1.000000,1.000000,1.000000}%
\pgfsetstrokecolor{currentstroke}%
\pgfsetdash{}{0pt}%
\pgfpathmoveto{\pgfqpoint{0.862500in}{0.375000in}}%
\pgfpathlineto{\pgfqpoint{6.210000in}{0.375000in}}%
\pgfusepath{}%
\end{pgfscope}%
\begin{pgfscope}%
\pgfsetrectcap%
\pgfsetmiterjoin%
\pgfsetlinewidth{0.000000pt}%
\definecolor{currentstroke}{rgb}{1.000000,1.000000,1.000000}%
\pgfsetstrokecolor{currentstroke}%
\pgfsetdash{}{0pt}%
\pgfpathmoveto{\pgfqpoint{0.862500in}{2.640000in}}%
\pgfpathlineto{\pgfqpoint{6.210000in}{2.640000in}}%
\pgfusepath{}%
\end{pgfscope}%
\begin{pgfscope}%
\definecolor{textcolor}{rgb}{0.150000,0.150000,0.150000}%
\pgfsetstrokecolor{textcolor}%
\pgfsetfillcolor{textcolor}%
\pgftext[x=3.536250in,y=2.723333in,,base]{\color{textcolor}\rmfamily\fontsize{8.000000}{9.600000}\selectfont Hartmann3 Fixed Training Set}%
\end{pgfscope}%
\begin{pgfscope}%
\pgfsetroundcap%
\pgfsetroundjoin%
\pgfsetlinewidth{1.505625pt}%
\definecolor{currentstroke}{rgb}{0.121569,0.466667,0.705882}%
\pgfsetstrokecolor{currentstroke}%
\pgfsetdash{}{0pt}%
\pgfpathmoveto{\pgfqpoint{0.962500in}{1.189344in}}%
\pgfpathlineto{\pgfqpoint{1.184722in}{1.189344in}}%
\pgfusepath{stroke}%
\end{pgfscope}%
\begin{pgfscope}%
\definecolor{textcolor}{rgb}{0.150000,0.150000,0.150000}%
\pgfsetstrokecolor{textcolor}%
\pgfsetfillcolor{textcolor}%
\pgftext[x=1.273611in,y=1.150455in,left,base]{\color{textcolor}\rmfamily\fontsize{8.000000}{9.600000}\selectfont random}%
\end{pgfscope}%
\begin{pgfscope}%
\pgfsetroundcap%
\pgfsetroundjoin%
\pgfsetlinewidth{1.505625pt}%
\definecolor{currentstroke}{rgb}{1.000000,0.498039,0.054902}%
\pgfsetstrokecolor{currentstroke}%
\pgfsetdash{}{0pt}%
\pgfpathmoveto{\pgfqpoint{0.962500in}{1.026258in}}%
\pgfpathlineto{\pgfqpoint{1.184722in}{1.026258in}}%
\pgfusepath{stroke}%
\end{pgfscope}%
\begin{pgfscope}%
\definecolor{textcolor}{rgb}{0.150000,0.150000,0.150000}%
\pgfsetstrokecolor{textcolor}%
\pgfsetfillcolor{textcolor}%
\pgftext[x=1.273611in,y=0.987369in,left,base]{\color{textcolor}\rmfamily\fontsize{8.000000}{9.600000}\selectfont DNGO fixed with 20 init samples}%
\end{pgfscope}%
\begin{pgfscope}%
\pgfsetroundcap%
\pgfsetroundjoin%
\pgfsetlinewidth{1.505625pt}%
\definecolor{currentstroke}{rgb}{0.172549,0.627451,0.172549}%
\pgfsetstrokecolor{currentstroke}%
\pgfsetdash{}{0pt}%
\pgfpathmoveto{\pgfqpoint{0.962500in}{0.863172in}}%
\pgfpathlineto{\pgfqpoint{1.184722in}{0.863172in}}%
\pgfusepath{stroke}%
\end{pgfscope}%
\begin{pgfscope}%
\definecolor{textcolor}{rgb}{0.150000,0.150000,0.150000}%
\pgfsetstrokecolor{textcolor}%
\pgfsetfillcolor{textcolor}%
\pgftext[x=1.273611in,y=0.824283in,left,base]{\color{textcolor}\rmfamily\fontsize{8.000000}{9.600000}\selectfont DNGO fixed with 50 init samples}%
\end{pgfscope}%
\begin{pgfscope}%
\pgfsetroundcap%
\pgfsetroundjoin%
\pgfsetlinewidth{1.505625pt}%
\definecolor{currentstroke}{rgb}{0.839216,0.152941,0.156863}%
\pgfsetstrokecolor{currentstroke}%
\pgfsetdash{}{0pt}%
\pgfpathmoveto{\pgfqpoint{0.962500in}{0.700087in}}%
\pgfpathlineto{\pgfqpoint{1.184722in}{0.700087in}}%
\pgfusepath{stroke}%
\end{pgfscope}%
\begin{pgfscope}%
\definecolor{textcolor}{rgb}{0.150000,0.150000,0.150000}%
\pgfsetstrokecolor{textcolor}%
\pgfsetfillcolor{textcolor}%
\pgftext[x=1.273611in,y=0.661198in,left,base]{\color{textcolor}\rmfamily\fontsize{8.000000}{9.600000}\selectfont DNGO fixed with 100 init samples}%
\end{pgfscope}%
\begin{pgfscope}%
\pgfsetroundcap%
\pgfsetroundjoin%
\pgfsetlinewidth{1.505625pt}%
\definecolor{currentstroke}{rgb}{0.580392,0.403922,0.741176}%
\pgfsetstrokecolor{currentstroke}%
\pgfsetdash{}{0pt}%
\pgfpathmoveto{\pgfqpoint{0.962500in}{0.537001in}}%
\pgfpathlineto{\pgfqpoint{1.184722in}{0.537001in}}%
\pgfusepath{stroke}%
\end{pgfscope}%
\begin{pgfscope}%
\definecolor{textcolor}{rgb}{0.150000,0.150000,0.150000}%
\pgfsetstrokecolor{textcolor}%
\pgfsetfillcolor{textcolor}%
\pgftext[x=1.273611in,y=0.498112in,left,base]{\color{textcolor}\rmfamily\fontsize{8.000000}{9.600000}\selectfont DNGO fixed with 200 init samples}%
\end{pgfscope}%
\end{pgfpicture}%
\makeatother%
\endgroup%

            \captionof{figure}{This plots shows Simple Regret for DNGO fixed with different initial samples sizes as described in \cref{sec:methoddngo}.}
        \end{minipage}
    
        \begin{minipage}{0.47\linewidth}
            \centering
            %% Creator: Matplotlib, PGF backend
%%
%% To include the figure in your LaTeX document, write
%%   \input{<filename>.pgf}
%%
%% Make sure the required packages are loaded in your preamble
%%   \usepackage{pgf}
%%
%% Figures using additional raster images can only be included by \input if
%% they are in the same directory as the main LaTeX file. For loading figures
%% from other directories you can use the `import` package
%%   \usepackage{import}
%% and then include the figures with
%%   \import{<path to file>}{<filename>.pgf}
%%
%% Matplotlib used the following preamble
%%   \usepackage{gensymb}
%%   \usepackage{fontspec}
%%   \setmainfont{DejaVu Serif}
%%   \setsansfont{Arial}
%%   \setmonofont{DejaVu Sans Mono}
%%
\begingroup%
\makeatletter%
\begin{pgfpicture}%
\pgfpathrectangle{\pgfpointorigin}{\pgfqpoint{3.390000in}{3.000000in}}%
\pgfusepath{use as bounding box, clip}%
\begin{pgfscope}%
\pgfsetbuttcap%
\pgfsetmiterjoin%
\definecolor{currentfill}{rgb}{1.000000,1.000000,1.000000}%
\pgfsetfillcolor{currentfill}%
\pgfsetlinewidth{0.000000pt}%
\definecolor{currentstroke}{rgb}{1.000000,1.000000,1.000000}%
\pgfsetstrokecolor{currentstroke}%
\pgfsetdash{}{0pt}%
\pgfpathmoveto{\pgfqpoint{0.000000in}{0.000000in}}%
\pgfpathlineto{\pgfqpoint{3.390000in}{0.000000in}}%
\pgfpathlineto{\pgfqpoint{3.390000in}{3.000000in}}%
\pgfpathlineto{\pgfqpoint{0.000000in}{3.000000in}}%
\pgfpathclose%
\pgfusepath{fill}%
\end{pgfscope}%
\begin{pgfscope}%
\pgfsetbuttcap%
\pgfsetmiterjoin%
\definecolor{currentfill}{rgb}{0.917647,0.917647,0.949020}%
\pgfsetfillcolor{currentfill}%
\pgfsetlinewidth{0.000000pt}%
\definecolor{currentstroke}{rgb}{0.000000,0.000000,0.000000}%
\pgfsetstrokecolor{currentstroke}%
\pgfsetstrokeopacity{0.000000}%
\pgfsetdash{}{0pt}%
\pgfpathmoveto{\pgfqpoint{0.423750in}{0.375000in}}%
\pgfpathlineto{\pgfqpoint{3.051000in}{0.375000in}}%
\pgfpathlineto{\pgfqpoint{3.051000in}{2.640000in}}%
\pgfpathlineto{\pgfqpoint{0.423750in}{2.640000in}}%
\pgfpathclose%
\pgfusepath{fill}%
\end{pgfscope}%
\begin{pgfscope}%
\pgfpathrectangle{\pgfqpoint{0.423750in}{0.375000in}}{\pgfqpoint{2.627250in}{2.265000in}}%
\pgfusepath{clip}%
\pgfsetroundcap%
\pgfsetroundjoin%
\pgfsetlinewidth{0.803000pt}%
\definecolor{currentstroke}{rgb}{1.000000,1.000000,1.000000}%
\pgfsetstrokecolor{currentstroke}%
\pgfsetdash{}{0pt}%
\pgfpathmoveto{\pgfqpoint{0.543170in}{0.375000in}}%
\pgfpathlineto{\pgfqpoint{0.543170in}{2.640000in}}%
\pgfusepath{stroke}%
\end{pgfscope}%
\begin{pgfscope}%
\definecolor{textcolor}{rgb}{0.150000,0.150000,0.150000}%
\pgfsetstrokecolor{textcolor}%
\pgfsetfillcolor{textcolor}%
\pgftext[x=0.543170in,y=0.326389in,,top]{\color{textcolor}\rmfamily\fontsize{8.000000}{9.600000}\selectfont \(\displaystyle 0\)}%
\end{pgfscope}%
\begin{pgfscope}%
\pgfpathrectangle{\pgfqpoint{0.423750in}{0.375000in}}{\pgfqpoint{2.627250in}{2.265000in}}%
\pgfusepath{clip}%
\pgfsetroundcap%
\pgfsetroundjoin%
\pgfsetlinewidth{0.803000pt}%
\definecolor{currentstroke}{rgb}{1.000000,1.000000,1.000000}%
\pgfsetstrokecolor{currentstroke}%
\pgfsetdash{}{0pt}%
\pgfpathmoveto{\pgfqpoint{1.249800in}{0.375000in}}%
\pgfpathlineto{\pgfqpoint{1.249800in}{2.640000in}}%
\pgfusepath{stroke}%
\end{pgfscope}%
\begin{pgfscope}%
\definecolor{textcolor}{rgb}{0.150000,0.150000,0.150000}%
\pgfsetstrokecolor{textcolor}%
\pgfsetfillcolor{textcolor}%
\pgftext[x=1.249800in,y=0.326389in,,top]{\color{textcolor}\rmfamily\fontsize{8.000000}{9.600000}\selectfont \(\displaystyle 50\)}%
\end{pgfscope}%
\begin{pgfscope}%
\pgfpathrectangle{\pgfqpoint{0.423750in}{0.375000in}}{\pgfqpoint{2.627250in}{2.265000in}}%
\pgfusepath{clip}%
\pgfsetroundcap%
\pgfsetroundjoin%
\pgfsetlinewidth{0.803000pt}%
\definecolor{currentstroke}{rgb}{1.000000,1.000000,1.000000}%
\pgfsetstrokecolor{currentstroke}%
\pgfsetdash{}{0pt}%
\pgfpathmoveto{\pgfqpoint{1.956430in}{0.375000in}}%
\pgfpathlineto{\pgfqpoint{1.956430in}{2.640000in}}%
\pgfusepath{stroke}%
\end{pgfscope}%
\begin{pgfscope}%
\definecolor{textcolor}{rgb}{0.150000,0.150000,0.150000}%
\pgfsetstrokecolor{textcolor}%
\pgfsetfillcolor{textcolor}%
\pgftext[x=1.956430in,y=0.326389in,,top]{\color{textcolor}\rmfamily\fontsize{8.000000}{9.600000}\selectfont \(\displaystyle 100\)}%
\end{pgfscope}%
\begin{pgfscope}%
\pgfpathrectangle{\pgfqpoint{0.423750in}{0.375000in}}{\pgfqpoint{2.627250in}{2.265000in}}%
\pgfusepath{clip}%
\pgfsetroundcap%
\pgfsetroundjoin%
\pgfsetlinewidth{0.803000pt}%
\definecolor{currentstroke}{rgb}{1.000000,1.000000,1.000000}%
\pgfsetstrokecolor{currentstroke}%
\pgfsetdash{}{0pt}%
\pgfpathmoveto{\pgfqpoint{2.663060in}{0.375000in}}%
\pgfpathlineto{\pgfqpoint{2.663060in}{2.640000in}}%
\pgfusepath{stroke}%
\end{pgfscope}%
\begin{pgfscope}%
\definecolor{textcolor}{rgb}{0.150000,0.150000,0.150000}%
\pgfsetstrokecolor{textcolor}%
\pgfsetfillcolor{textcolor}%
\pgftext[x=2.663060in,y=0.326389in,,top]{\color{textcolor}\rmfamily\fontsize{8.000000}{9.600000}\selectfont \(\displaystyle 150\)}%
\end{pgfscope}%
\begin{pgfscope}%
\definecolor{textcolor}{rgb}{0.150000,0.150000,0.150000}%
\pgfsetstrokecolor{textcolor}%
\pgfsetfillcolor{textcolor}%
\pgftext[x=1.737375in,y=0.163303in,,top]{\color{textcolor}\rmfamily\fontsize{8.000000}{9.600000}\selectfont Step}%
\end{pgfscope}%
\begin{pgfscope}%
\pgfpathrectangle{\pgfqpoint{0.423750in}{0.375000in}}{\pgfqpoint{2.627250in}{2.265000in}}%
\pgfusepath{clip}%
\pgfsetroundcap%
\pgfsetroundjoin%
\pgfsetlinewidth{0.803000pt}%
\definecolor{currentstroke}{rgb}{1.000000,1.000000,1.000000}%
\pgfsetstrokecolor{currentstroke}%
\pgfsetdash{}{0pt}%
\pgfpathmoveto{\pgfqpoint{0.423750in}{0.588795in}}%
\pgfpathlineto{\pgfqpoint{3.051000in}{0.588795in}}%
\pgfusepath{stroke}%
\end{pgfscope}%
\begin{pgfscope}%
\definecolor{textcolor}{rgb}{0.150000,0.150000,0.150000}%
\pgfsetstrokecolor{textcolor}%
\pgfsetfillcolor{textcolor}%
\pgftext[x=0.118966in,y=0.546585in,left,base]{\color{textcolor}\rmfamily\fontsize{8.000000}{9.600000}\selectfont \(\displaystyle 10^{-6}\)}%
\end{pgfscope}%
\begin{pgfscope}%
\pgfpathrectangle{\pgfqpoint{0.423750in}{0.375000in}}{\pgfqpoint{2.627250in}{2.265000in}}%
\pgfusepath{clip}%
\pgfsetroundcap%
\pgfsetroundjoin%
\pgfsetlinewidth{0.803000pt}%
\definecolor{currentstroke}{rgb}{1.000000,1.000000,1.000000}%
\pgfsetstrokecolor{currentstroke}%
\pgfsetdash{}{0pt}%
\pgfpathmoveto{\pgfqpoint{0.423750in}{0.832204in}}%
\pgfpathlineto{\pgfqpoint{3.051000in}{0.832204in}}%
\pgfusepath{stroke}%
\end{pgfscope}%
\begin{pgfscope}%
\definecolor{textcolor}{rgb}{0.150000,0.150000,0.150000}%
\pgfsetstrokecolor{textcolor}%
\pgfsetfillcolor{textcolor}%
\pgftext[x=0.118966in,y=0.789995in,left,base]{\color{textcolor}\rmfamily\fontsize{8.000000}{9.600000}\selectfont \(\displaystyle 10^{-5}\)}%
\end{pgfscope}%
\begin{pgfscope}%
\pgfpathrectangle{\pgfqpoint{0.423750in}{0.375000in}}{\pgfqpoint{2.627250in}{2.265000in}}%
\pgfusepath{clip}%
\pgfsetroundcap%
\pgfsetroundjoin%
\pgfsetlinewidth{0.803000pt}%
\definecolor{currentstroke}{rgb}{1.000000,1.000000,1.000000}%
\pgfsetstrokecolor{currentstroke}%
\pgfsetdash{}{0pt}%
\pgfpathmoveto{\pgfqpoint{0.423750in}{1.075614in}}%
\pgfpathlineto{\pgfqpoint{3.051000in}{1.075614in}}%
\pgfusepath{stroke}%
\end{pgfscope}%
\begin{pgfscope}%
\definecolor{textcolor}{rgb}{0.150000,0.150000,0.150000}%
\pgfsetstrokecolor{textcolor}%
\pgfsetfillcolor{textcolor}%
\pgftext[x=0.118966in,y=1.033405in,left,base]{\color{textcolor}\rmfamily\fontsize{8.000000}{9.600000}\selectfont \(\displaystyle 10^{-4}\)}%
\end{pgfscope}%
\begin{pgfscope}%
\pgfpathrectangle{\pgfqpoint{0.423750in}{0.375000in}}{\pgfqpoint{2.627250in}{2.265000in}}%
\pgfusepath{clip}%
\pgfsetroundcap%
\pgfsetroundjoin%
\pgfsetlinewidth{0.803000pt}%
\definecolor{currentstroke}{rgb}{1.000000,1.000000,1.000000}%
\pgfsetstrokecolor{currentstroke}%
\pgfsetdash{}{0pt}%
\pgfpathmoveto{\pgfqpoint{0.423750in}{1.319024in}}%
\pgfpathlineto{\pgfqpoint{3.051000in}{1.319024in}}%
\pgfusepath{stroke}%
\end{pgfscope}%
\begin{pgfscope}%
\definecolor{textcolor}{rgb}{0.150000,0.150000,0.150000}%
\pgfsetstrokecolor{textcolor}%
\pgfsetfillcolor{textcolor}%
\pgftext[x=0.118966in,y=1.276814in,left,base]{\color{textcolor}\rmfamily\fontsize{8.000000}{9.600000}\selectfont \(\displaystyle 10^{-3}\)}%
\end{pgfscope}%
\begin{pgfscope}%
\pgfpathrectangle{\pgfqpoint{0.423750in}{0.375000in}}{\pgfqpoint{2.627250in}{2.265000in}}%
\pgfusepath{clip}%
\pgfsetroundcap%
\pgfsetroundjoin%
\pgfsetlinewidth{0.803000pt}%
\definecolor{currentstroke}{rgb}{1.000000,1.000000,1.000000}%
\pgfsetstrokecolor{currentstroke}%
\pgfsetdash{}{0pt}%
\pgfpathmoveto{\pgfqpoint{0.423750in}{1.562433in}}%
\pgfpathlineto{\pgfqpoint{3.051000in}{1.562433in}}%
\pgfusepath{stroke}%
\end{pgfscope}%
\begin{pgfscope}%
\definecolor{textcolor}{rgb}{0.150000,0.150000,0.150000}%
\pgfsetstrokecolor{textcolor}%
\pgfsetfillcolor{textcolor}%
\pgftext[x=0.118966in,y=1.520224in,left,base]{\color{textcolor}\rmfamily\fontsize{8.000000}{9.600000}\selectfont \(\displaystyle 10^{-2}\)}%
\end{pgfscope}%
\begin{pgfscope}%
\pgfpathrectangle{\pgfqpoint{0.423750in}{0.375000in}}{\pgfqpoint{2.627250in}{2.265000in}}%
\pgfusepath{clip}%
\pgfsetroundcap%
\pgfsetroundjoin%
\pgfsetlinewidth{0.803000pt}%
\definecolor{currentstroke}{rgb}{1.000000,1.000000,1.000000}%
\pgfsetstrokecolor{currentstroke}%
\pgfsetdash{}{0pt}%
\pgfpathmoveto{\pgfqpoint{0.423750in}{1.805843in}}%
\pgfpathlineto{\pgfqpoint{3.051000in}{1.805843in}}%
\pgfusepath{stroke}%
\end{pgfscope}%
\begin{pgfscope}%
\definecolor{textcolor}{rgb}{0.150000,0.150000,0.150000}%
\pgfsetstrokecolor{textcolor}%
\pgfsetfillcolor{textcolor}%
\pgftext[x=0.118966in,y=1.763634in,left,base]{\color{textcolor}\rmfamily\fontsize{8.000000}{9.600000}\selectfont \(\displaystyle 10^{-1}\)}%
\end{pgfscope}%
\begin{pgfscope}%
\pgfpathrectangle{\pgfqpoint{0.423750in}{0.375000in}}{\pgfqpoint{2.627250in}{2.265000in}}%
\pgfusepath{clip}%
\pgfsetroundcap%
\pgfsetroundjoin%
\pgfsetlinewidth{0.803000pt}%
\definecolor{currentstroke}{rgb}{1.000000,1.000000,1.000000}%
\pgfsetstrokecolor{currentstroke}%
\pgfsetdash{}{0pt}%
\pgfpathmoveto{\pgfqpoint{0.423750in}{2.049253in}}%
\pgfpathlineto{\pgfqpoint{3.051000in}{2.049253in}}%
\pgfusepath{stroke}%
\end{pgfscope}%
\begin{pgfscope}%
\definecolor{textcolor}{rgb}{0.150000,0.150000,0.150000}%
\pgfsetstrokecolor{textcolor}%
\pgfsetfillcolor{textcolor}%
\pgftext[x=0.199212in,y=2.007043in,left,base]{\color{textcolor}\rmfamily\fontsize{8.000000}{9.600000}\selectfont \(\displaystyle 10^{0}\)}%
\end{pgfscope}%
\begin{pgfscope}%
\pgfpathrectangle{\pgfqpoint{0.423750in}{0.375000in}}{\pgfqpoint{2.627250in}{2.265000in}}%
\pgfusepath{clip}%
\pgfsetroundcap%
\pgfsetroundjoin%
\pgfsetlinewidth{0.803000pt}%
\definecolor{currentstroke}{rgb}{1.000000,1.000000,1.000000}%
\pgfsetstrokecolor{currentstroke}%
\pgfsetdash{}{0pt}%
\pgfpathmoveto{\pgfqpoint{0.423750in}{2.292662in}}%
\pgfpathlineto{\pgfqpoint{3.051000in}{2.292662in}}%
\pgfusepath{stroke}%
\end{pgfscope}%
\begin{pgfscope}%
\definecolor{textcolor}{rgb}{0.150000,0.150000,0.150000}%
\pgfsetstrokecolor{textcolor}%
\pgfsetfillcolor{textcolor}%
\pgftext[x=0.199212in,y=2.250453in,left,base]{\color{textcolor}\rmfamily\fontsize{8.000000}{9.600000}\selectfont \(\displaystyle 10^{1}\)}%
\end{pgfscope}%
\begin{pgfscope}%
\pgfpathrectangle{\pgfqpoint{0.423750in}{0.375000in}}{\pgfqpoint{2.627250in}{2.265000in}}%
\pgfusepath{clip}%
\pgfsetroundcap%
\pgfsetroundjoin%
\pgfsetlinewidth{0.803000pt}%
\definecolor{currentstroke}{rgb}{1.000000,1.000000,1.000000}%
\pgfsetstrokecolor{currentstroke}%
\pgfsetdash{}{0pt}%
\pgfpathmoveto{\pgfqpoint{0.423750in}{2.536072in}}%
\pgfpathlineto{\pgfqpoint{3.051000in}{2.536072in}}%
\pgfusepath{stroke}%
\end{pgfscope}%
\begin{pgfscope}%
\definecolor{textcolor}{rgb}{0.150000,0.150000,0.150000}%
\pgfsetstrokecolor{textcolor}%
\pgfsetfillcolor{textcolor}%
\pgftext[x=0.199212in,y=2.493863in,left,base]{\color{textcolor}\rmfamily\fontsize{8.000000}{9.600000}\selectfont \(\displaystyle 10^{2}\)}%
\end{pgfscope}%
\begin{pgfscope}%
\definecolor{textcolor}{rgb}{0.150000,0.150000,0.150000}%
\pgfsetstrokecolor{textcolor}%
\pgfsetfillcolor{textcolor}%
\pgftext[x=0.063410in,y=1.507500in,,bottom,rotate=90.000000]{\color{textcolor}\rmfamily\fontsize{8.000000}{9.600000}\selectfont Simple Regret}%
\end{pgfscope}%
\begin{pgfscope}%
\pgfpathrectangle{\pgfqpoint{0.423750in}{0.375000in}}{\pgfqpoint{2.627250in}{2.265000in}}%
\pgfusepath{clip}%
\pgfsetbuttcap%
\pgfsetroundjoin%
\definecolor{currentfill}{rgb}{0.121569,0.466667,0.705882}%
\pgfsetfillcolor{currentfill}%
\pgfsetfillopacity{0.200000}%
\pgfsetlinewidth{0.000000pt}%
\definecolor{currentstroke}{rgb}{0.000000,0.000000,0.000000}%
\pgfsetstrokecolor{currentstroke}%
\pgfsetdash{}{0pt}%
\pgfpathmoveto{\pgfqpoint{0.543170in}{2.409843in}}%
\pgfpathlineto{\pgfqpoint{0.543170in}{2.488030in}}%
\pgfpathlineto{\pgfqpoint{0.557303in}{2.436828in}}%
\pgfpathlineto{\pgfqpoint{0.571436in}{2.384565in}}%
\pgfpathlineto{\pgfqpoint{0.585568in}{2.335485in}}%
\pgfpathlineto{\pgfqpoint{0.599701in}{2.330961in}}%
\pgfpathlineto{\pgfqpoint{0.613833in}{2.330961in}}%
\pgfpathlineto{\pgfqpoint{0.627966in}{2.297928in}}%
\pgfpathlineto{\pgfqpoint{0.642099in}{2.289680in}}%
\pgfpathlineto{\pgfqpoint{0.656231in}{2.273261in}}%
\pgfpathlineto{\pgfqpoint{0.670364in}{2.264940in}}%
\pgfpathlineto{\pgfqpoint{0.684496in}{2.254941in}}%
\pgfpathlineto{\pgfqpoint{0.698629in}{2.254941in}}%
\pgfpathlineto{\pgfqpoint{0.712762in}{2.254941in}}%
\pgfpathlineto{\pgfqpoint{0.726894in}{2.229227in}}%
\pgfpathlineto{\pgfqpoint{0.741027in}{2.212262in}}%
\pgfpathlineto{\pgfqpoint{0.755159in}{2.197303in}}%
\pgfpathlineto{\pgfqpoint{0.769292in}{2.184468in}}%
\pgfpathlineto{\pgfqpoint{0.783425in}{2.184468in}}%
\pgfpathlineto{\pgfqpoint{0.797557in}{2.184468in}}%
\pgfpathlineto{\pgfqpoint{0.811690in}{2.184262in}}%
\pgfpathlineto{\pgfqpoint{0.825822in}{2.184262in}}%
\pgfpathlineto{\pgfqpoint{0.839955in}{2.184262in}}%
\pgfpathlineto{\pgfqpoint{0.854088in}{2.183358in}}%
\pgfpathlineto{\pgfqpoint{0.868220in}{2.183358in}}%
\pgfpathlineto{\pgfqpoint{0.882353in}{2.183358in}}%
\pgfpathlineto{\pgfqpoint{0.896485in}{2.181618in}}%
\pgfpathlineto{\pgfqpoint{0.910618in}{2.181618in}}%
\pgfpathlineto{\pgfqpoint{0.924751in}{2.181589in}}%
\pgfpathlineto{\pgfqpoint{0.938883in}{2.181589in}}%
\pgfpathlineto{\pgfqpoint{0.953016in}{2.181589in}}%
\pgfpathlineto{\pgfqpoint{0.967148in}{2.169408in}}%
\pgfpathlineto{\pgfqpoint{0.981281in}{2.169408in}}%
\pgfpathlineto{\pgfqpoint{0.995414in}{2.169408in}}%
\pgfpathlineto{\pgfqpoint{1.009546in}{2.169408in}}%
\pgfpathlineto{\pgfqpoint{1.023679in}{2.169408in}}%
\pgfpathlineto{\pgfqpoint{1.037811in}{2.169408in}}%
\pgfpathlineto{\pgfqpoint{1.051944in}{2.169408in}}%
\pgfpathlineto{\pgfqpoint{1.066077in}{2.169408in}}%
\pgfpathlineto{\pgfqpoint{1.080209in}{2.169408in}}%
\pgfpathlineto{\pgfqpoint{1.094342in}{2.169408in}}%
\pgfpathlineto{\pgfqpoint{1.108474in}{2.165269in}}%
\pgfpathlineto{\pgfqpoint{1.122607in}{2.165269in}}%
\pgfpathlineto{\pgfqpoint{1.136740in}{2.165269in}}%
\pgfpathlineto{\pgfqpoint{1.150872in}{2.165269in}}%
\pgfpathlineto{\pgfqpoint{1.165005in}{2.165269in}}%
\pgfpathlineto{\pgfqpoint{1.179137in}{2.165269in}}%
\pgfpathlineto{\pgfqpoint{1.193270in}{2.165269in}}%
\pgfpathlineto{\pgfqpoint{1.207403in}{2.157135in}}%
\pgfpathlineto{\pgfqpoint{1.221535in}{2.156679in}}%
\pgfpathlineto{\pgfqpoint{1.235668in}{2.108732in}}%
\pgfpathlineto{\pgfqpoint{1.249800in}{2.108732in}}%
\pgfpathlineto{\pgfqpoint{1.263933in}{2.108732in}}%
\pgfpathlineto{\pgfqpoint{1.278066in}{2.108732in}}%
\pgfpathlineto{\pgfqpoint{1.292198in}{2.108732in}}%
\pgfpathlineto{\pgfqpoint{1.306331in}{2.108732in}}%
\pgfpathlineto{\pgfqpoint{1.320463in}{2.108732in}}%
\pgfpathlineto{\pgfqpoint{1.334596in}{2.108732in}}%
\pgfpathlineto{\pgfqpoint{1.348729in}{2.104233in}}%
\pgfpathlineto{\pgfqpoint{1.362861in}{2.104233in}}%
\pgfpathlineto{\pgfqpoint{1.376994in}{2.104233in}}%
\pgfpathlineto{\pgfqpoint{1.391126in}{2.104233in}}%
\pgfpathlineto{\pgfqpoint{1.405259in}{2.090919in}}%
\pgfpathlineto{\pgfqpoint{1.419392in}{2.090919in}}%
\pgfpathlineto{\pgfqpoint{1.433524in}{2.090919in}}%
\pgfpathlineto{\pgfqpoint{1.447657in}{2.090919in}}%
\pgfpathlineto{\pgfqpoint{1.461789in}{2.069885in}}%
\pgfpathlineto{\pgfqpoint{1.475922in}{2.069885in}}%
\pgfpathlineto{\pgfqpoint{1.490055in}{2.069885in}}%
\pgfpathlineto{\pgfqpoint{1.504187in}{2.069885in}}%
\pgfpathlineto{\pgfqpoint{1.518320in}{2.069885in}}%
\pgfpathlineto{\pgfqpoint{1.532452in}{2.069885in}}%
\pgfpathlineto{\pgfqpoint{1.546585in}{2.069885in}}%
\pgfpathlineto{\pgfqpoint{1.560718in}{2.069885in}}%
\pgfpathlineto{\pgfqpoint{1.574850in}{2.069885in}}%
\pgfpathlineto{\pgfqpoint{1.588983in}{2.069885in}}%
\pgfpathlineto{\pgfqpoint{1.603115in}{2.069885in}}%
\pgfpathlineto{\pgfqpoint{1.617248in}{2.069885in}}%
\pgfpathlineto{\pgfqpoint{1.631381in}{2.069885in}}%
\pgfpathlineto{\pgfqpoint{1.645513in}{2.058989in}}%
\pgfpathlineto{\pgfqpoint{1.659646in}{2.058989in}}%
\pgfpathlineto{\pgfqpoint{1.673778in}{2.058989in}}%
\pgfpathlineto{\pgfqpoint{1.687911in}{2.055535in}}%
\pgfpathlineto{\pgfqpoint{1.702044in}{2.055535in}}%
\pgfpathlineto{\pgfqpoint{1.716176in}{2.055535in}}%
\pgfpathlineto{\pgfqpoint{1.730309in}{2.052522in}}%
\pgfpathlineto{\pgfqpoint{1.744441in}{2.038708in}}%
\pgfpathlineto{\pgfqpoint{1.758574in}{2.038708in}}%
\pgfpathlineto{\pgfqpoint{1.772706in}{2.038708in}}%
\pgfpathlineto{\pgfqpoint{1.786839in}{2.038708in}}%
\pgfpathlineto{\pgfqpoint{1.800972in}{2.038708in}}%
\pgfpathlineto{\pgfqpoint{1.815104in}{2.038708in}}%
\pgfpathlineto{\pgfqpoint{1.829237in}{2.038708in}}%
\pgfpathlineto{\pgfqpoint{1.843369in}{2.038708in}}%
\pgfpathlineto{\pgfqpoint{1.857502in}{2.038708in}}%
\pgfpathlineto{\pgfqpoint{1.871635in}{2.038708in}}%
\pgfpathlineto{\pgfqpoint{1.885767in}{2.038708in}}%
\pgfpathlineto{\pgfqpoint{1.899900in}{2.038708in}}%
\pgfpathlineto{\pgfqpoint{1.914032in}{2.038708in}}%
\pgfpathlineto{\pgfqpoint{1.928165in}{2.038708in}}%
\pgfpathlineto{\pgfqpoint{1.942298in}{2.038708in}}%
\pgfpathlineto{\pgfqpoint{1.956430in}{2.038708in}}%
\pgfpathlineto{\pgfqpoint{1.970563in}{2.038708in}}%
\pgfpathlineto{\pgfqpoint{1.984695in}{2.033252in}}%
\pgfpathlineto{\pgfqpoint{1.998828in}{2.033252in}}%
\pgfpathlineto{\pgfqpoint{2.012961in}{2.033252in}}%
\pgfpathlineto{\pgfqpoint{2.027093in}{2.033252in}}%
\pgfpathlineto{\pgfqpoint{2.041226in}{2.033252in}}%
\pgfpathlineto{\pgfqpoint{2.055358in}{2.033252in}}%
\pgfpathlineto{\pgfqpoint{2.069491in}{2.033252in}}%
\pgfpathlineto{\pgfqpoint{2.083624in}{2.033252in}}%
\pgfpathlineto{\pgfqpoint{2.097756in}{2.033252in}}%
\pgfpathlineto{\pgfqpoint{2.111889in}{2.033252in}}%
\pgfpathlineto{\pgfqpoint{2.126021in}{2.033252in}}%
\pgfpathlineto{\pgfqpoint{2.140154in}{2.033252in}}%
\pgfpathlineto{\pgfqpoint{2.154287in}{2.033252in}}%
\pgfpathlineto{\pgfqpoint{2.168419in}{2.033252in}}%
\pgfpathlineto{\pgfqpoint{2.182552in}{2.033252in}}%
\pgfpathlineto{\pgfqpoint{2.196684in}{2.033252in}}%
\pgfpathlineto{\pgfqpoint{2.210817in}{2.033252in}}%
\pgfpathlineto{\pgfqpoint{2.224950in}{2.033252in}}%
\pgfpathlineto{\pgfqpoint{2.239082in}{2.033252in}}%
\pgfpathlineto{\pgfqpoint{2.253215in}{2.033252in}}%
\pgfpathlineto{\pgfqpoint{2.267347in}{2.033252in}}%
\pgfpathlineto{\pgfqpoint{2.281480in}{2.013199in}}%
\pgfpathlineto{\pgfqpoint{2.295613in}{2.013199in}}%
\pgfpathlineto{\pgfqpoint{2.309745in}{2.013199in}}%
\pgfpathlineto{\pgfqpoint{2.323878in}{2.013199in}}%
\pgfpathlineto{\pgfqpoint{2.338010in}{2.013199in}}%
\pgfpathlineto{\pgfqpoint{2.352143in}{2.013199in}}%
\pgfpathlineto{\pgfqpoint{2.366276in}{2.013199in}}%
\pgfpathlineto{\pgfqpoint{2.380408in}{2.013199in}}%
\pgfpathlineto{\pgfqpoint{2.394541in}{2.009565in}}%
\pgfpathlineto{\pgfqpoint{2.408673in}{2.009565in}}%
\pgfpathlineto{\pgfqpoint{2.422806in}{2.009565in}}%
\pgfpathlineto{\pgfqpoint{2.436939in}{2.009565in}}%
\pgfpathlineto{\pgfqpoint{2.451071in}{2.009565in}}%
\pgfpathlineto{\pgfqpoint{2.465204in}{2.009565in}}%
\pgfpathlineto{\pgfqpoint{2.479336in}{2.009565in}}%
\pgfpathlineto{\pgfqpoint{2.493469in}{2.009565in}}%
\pgfpathlineto{\pgfqpoint{2.507602in}{2.009565in}}%
\pgfpathlineto{\pgfqpoint{2.521734in}{2.009565in}}%
\pgfpathlineto{\pgfqpoint{2.535867in}{2.009565in}}%
\pgfpathlineto{\pgfqpoint{2.549999in}{2.009565in}}%
\pgfpathlineto{\pgfqpoint{2.564132in}{2.009565in}}%
\pgfpathlineto{\pgfqpoint{2.578265in}{2.009565in}}%
\pgfpathlineto{\pgfqpoint{2.592397in}{2.009565in}}%
\pgfpathlineto{\pgfqpoint{2.606530in}{2.008313in}}%
\pgfpathlineto{\pgfqpoint{2.620662in}{2.008313in}}%
\pgfpathlineto{\pgfqpoint{2.634795in}{2.008313in}}%
\pgfpathlineto{\pgfqpoint{2.648928in}{1.983074in}}%
\pgfpathlineto{\pgfqpoint{2.663060in}{1.983074in}}%
\pgfpathlineto{\pgfqpoint{2.677193in}{1.983074in}}%
\pgfpathlineto{\pgfqpoint{2.691325in}{1.983074in}}%
\pgfpathlineto{\pgfqpoint{2.705458in}{1.983074in}}%
\pgfpathlineto{\pgfqpoint{2.719591in}{1.983074in}}%
\pgfpathlineto{\pgfqpoint{2.733723in}{1.983074in}}%
\pgfpathlineto{\pgfqpoint{2.747856in}{1.983074in}}%
\pgfpathlineto{\pgfqpoint{2.761988in}{1.980413in}}%
\pgfpathlineto{\pgfqpoint{2.776121in}{1.980413in}}%
\pgfpathlineto{\pgfqpoint{2.790254in}{1.980413in}}%
\pgfpathlineto{\pgfqpoint{2.804386in}{1.980413in}}%
\pgfpathlineto{\pgfqpoint{2.818519in}{1.980413in}}%
\pgfpathlineto{\pgfqpoint{2.832651in}{1.980413in}}%
\pgfpathlineto{\pgfqpoint{2.846784in}{1.980413in}}%
\pgfpathlineto{\pgfqpoint{2.860917in}{1.980413in}}%
\pgfpathlineto{\pgfqpoint{2.875049in}{1.980413in}}%
\pgfpathlineto{\pgfqpoint{2.889182in}{1.980413in}}%
\pgfpathlineto{\pgfqpoint{2.903314in}{1.980413in}}%
\pgfpathlineto{\pgfqpoint{2.917447in}{1.980413in}}%
\pgfpathlineto{\pgfqpoint{2.931580in}{1.980413in}}%
\pgfpathlineto{\pgfqpoint{2.931580in}{1.920417in}}%
\pgfpathlineto{\pgfqpoint{2.931580in}{1.920417in}}%
\pgfpathlineto{\pgfqpoint{2.917447in}{1.920417in}}%
\pgfpathlineto{\pgfqpoint{2.903314in}{1.920417in}}%
\pgfpathlineto{\pgfqpoint{2.889182in}{1.920417in}}%
\pgfpathlineto{\pgfqpoint{2.875049in}{1.920417in}}%
\pgfpathlineto{\pgfqpoint{2.860917in}{1.920417in}}%
\pgfpathlineto{\pgfqpoint{2.846784in}{1.920417in}}%
\pgfpathlineto{\pgfqpoint{2.832651in}{1.920417in}}%
\pgfpathlineto{\pgfqpoint{2.818519in}{1.920417in}}%
\pgfpathlineto{\pgfqpoint{2.804386in}{1.920417in}}%
\pgfpathlineto{\pgfqpoint{2.790254in}{1.920417in}}%
\pgfpathlineto{\pgfqpoint{2.776121in}{1.920417in}}%
\pgfpathlineto{\pgfqpoint{2.761988in}{1.920417in}}%
\pgfpathlineto{\pgfqpoint{2.747856in}{1.920708in}}%
\pgfpathlineto{\pgfqpoint{2.733723in}{1.920708in}}%
\pgfpathlineto{\pgfqpoint{2.719591in}{1.920708in}}%
\pgfpathlineto{\pgfqpoint{2.705458in}{1.920708in}}%
\pgfpathlineto{\pgfqpoint{2.691325in}{1.920708in}}%
\pgfpathlineto{\pgfqpoint{2.677193in}{1.920708in}}%
\pgfpathlineto{\pgfqpoint{2.663060in}{1.920708in}}%
\pgfpathlineto{\pgfqpoint{2.648928in}{1.920708in}}%
\pgfpathlineto{\pgfqpoint{2.634795in}{1.930021in}}%
\pgfpathlineto{\pgfqpoint{2.620662in}{1.930021in}}%
\pgfpathlineto{\pgfqpoint{2.606530in}{1.930021in}}%
\pgfpathlineto{\pgfqpoint{2.592397in}{1.930826in}}%
\pgfpathlineto{\pgfqpoint{2.578265in}{1.930826in}}%
\pgfpathlineto{\pgfqpoint{2.564132in}{1.930826in}}%
\pgfpathlineto{\pgfqpoint{2.549999in}{1.930826in}}%
\pgfpathlineto{\pgfqpoint{2.535867in}{1.930826in}}%
\pgfpathlineto{\pgfqpoint{2.521734in}{1.930826in}}%
\pgfpathlineto{\pgfqpoint{2.507602in}{1.930826in}}%
\pgfpathlineto{\pgfqpoint{2.493469in}{1.930826in}}%
\pgfpathlineto{\pgfqpoint{2.479336in}{1.930826in}}%
\pgfpathlineto{\pgfqpoint{2.465204in}{1.930826in}}%
\pgfpathlineto{\pgfqpoint{2.451071in}{1.930826in}}%
\pgfpathlineto{\pgfqpoint{2.436939in}{1.930826in}}%
\pgfpathlineto{\pgfqpoint{2.422806in}{1.930826in}}%
\pgfpathlineto{\pgfqpoint{2.408673in}{1.930826in}}%
\pgfpathlineto{\pgfqpoint{2.394541in}{1.930826in}}%
\pgfpathlineto{\pgfqpoint{2.380408in}{1.940985in}}%
\pgfpathlineto{\pgfqpoint{2.366276in}{1.940985in}}%
\pgfpathlineto{\pgfqpoint{2.352143in}{1.940985in}}%
\pgfpathlineto{\pgfqpoint{2.338010in}{1.940985in}}%
\pgfpathlineto{\pgfqpoint{2.323878in}{1.940985in}}%
\pgfpathlineto{\pgfqpoint{2.309745in}{1.940985in}}%
\pgfpathlineto{\pgfqpoint{2.295613in}{1.940985in}}%
\pgfpathlineto{\pgfqpoint{2.281480in}{1.940985in}}%
\pgfpathlineto{\pgfqpoint{2.267347in}{1.936564in}}%
\pgfpathlineto{\pgfqpoint{2.253215in}{1.936564in}}%
\pgfpathlineto{\pgfqpoint{2.239082in}{1.936564in}}%
\pgfpathlineto{\pgfqpoint{2.224950in}{1.936564in}}%
\pgfpathlineto{\pgfqpoint{2.210817in}{1.936564in}}%
\pgfpathlineto{\pgfqpoint{2.196684in}{1.936564in}}%
\pgfpathlineto{\pgfqpoint{2.182552in}{1.936564in}}%
\pgfpathlineto{\pgfqpoint{2.168419in}{1.936564in}}%
\pgfpathlineto{\pgfqpoint{2.154287in}{1.936564in}}%
\pgfpathlineto{\pgfqpoint{2.140154in}{1.936564in}}%
\pgfpathlineto{\pgfqpoint{2.126021in}{1.936564in}}%
\pgfpathlineto{\pgfqpoint{2.111889in}{1.936564in}}%
\pgfpathlineto{\pgfqpoint{2.097756in}{1.936564in}}%
\pgfpathlineto{\pgfqpoint{2.083624in}{1.936564in}}%
\pgfpathlineto{\pgfqpoint{2.069491in}{1.936564in}}%
\pgfpathlineto{\pgfqpoint{2.055358in}{1.936564in}}%
\pgfpathlineto{\pgfqpoint{2.041226in}{1.936564in}}%
\pgfpathlineto{\pgfqpoint{2.027093in}{1.936564in}}%
\pgfpathlineto{\pgfqpoint{2.012961in}{1.936564in}}%
\pgfpathlineto{\pgfqpoint{1.998828in}{1.936564in}}%
\pgfpathlineto{\pgfqpoint{1.984695in}{1.936564in}}%
\pgfpathlineto{\pgfqpoint{1.970563in}{1.942023in}}%
\pgfpathlineto{\pgfqpoint{1.956430in}{1.942023in}}%
\pgfpathlineto{\pgfqpoint{1.942298in}{1.942023in}}%
\pgfpathlineto{\pgfqpoint{1.928165in}{1.942023in}}%
\pgfpathlineto{\pgfqpoint{1.914032in}{1.942023in}}%
\pgfpathlineto{\pgfqpoint{1.899900in}{1.942023in}}%
\pgfpathlineto{\pgfqpoint{1.885767in}{1.942023in}}%
\pgfpathlineto{\pgfqpoint{1.871635in}{1.942023in}}%
\pgfpathlineto{\pgfqpoint{1.857502in}{1.942023in}}%
\pgfpathlineto{\pgfqpoint{1.843369in}{1.942023in}}%
\pgfpathlineto{\pgfqpoint{1.829237in}{1.942023in}}%
\pgfpathlineto{\pgfqpoint{1.815104in}{1.942023in}}%
\pgfpathlineto{\pgfqpoint{1.800972in}{1.942023in}}%
\pgfpathlineto{\pgfqpoint{1.786839in}{1.942023in}}%
\pgfpathlineto{\pgfqpoint{1.772706in}{1.942023in}}%
\pgfpathlineto{\pgfqpoint{1.758574in}{1.942023in}}%
\pgfpathlineto{\pgfqpoint{1.744441in}{1.942023in}}%
\pgfpathlineto{\pgfqpoint{1.730309in}{1.958307in}}%
\pgfpathlineto{\pgfqpoint{1.716176in}{1.964285in}}%
\pgfpathlineto{\pgfqpoint{1.702044in}{1.964285in}}%
\pgfpathlineto{\pgfqpoint{1.687911in}{1.964285in}}%
\pgfpathlineto{\pgfqpoint{1.673778in}{1.968757in}}%
\pgfpathlineto{\pgfqpoint{1.659646in}{1.968757in}}%
\pgfpathlineto{\pgfqpoint{1.645513in}{1.968757in}}%
\pgfpathlineto{\pgfqpoint{1.631381in}{1.998611in}}%
\pgfpathlineto{\pgfqpoint{1.617248in}{1.998611in}}%
\pgfpathlineto{\pgfqpoint{1.603115in}{1.998611in}}%
\pgfpathlineto{\pgfqpoint{1.588983in}{1.998611in}}%
\pgfpathlineto{\pgfqpoint{1.574850in}{1.998611in}}%
\pgfpathlineto{\pgfqpoint{1.560718in}{1.998611in}}%
\pgfpathlineto{\pgfqpoint{1.546585in}{1.998611in}}%
\pgfpathlineto{\pgfqpoint{1.532452in}{1.998611in}}%
\pgfpathlineto{\pgfqpoint{1.518320in}{1.998611in}}%
\pgfpathlineto{\pgfqpoint{1.504187in}{1.998611in}}%
\pgfpathlineto{\pgfqpoint{1.490055in}{1.998611in}}%
\pgfpathlineto{\pgfqpoint{1.475922in}{1.998611in}}%
\pgfpathlineto{\pgfqpoint{1.461789in}{1.998611in}}%
\pgfpathlineto{\pgfqpoint{1.447657in}{2.001392in}}%
\pgfpathlineto{\pgfqpoint{1.433524in}{2.001392in}}%
\pgfpathlineto{\pgfqpoint{1.419392in}{2.001392in}}%
\pgfpathlineto{\pgfqpoint{1.405259in}{2.001392in}}%
\pgfpathlineto{\pgfqpoint{1.391126in}{2.028997in}}%
\pgfpathlineto{\pgfqpoint{1.376994in}{2.028997in}}%
\pgfpathlineto{\pgfqpoint{1.362861in}{2.028997in}}%
\pgfpathlineto{\pgfqpoint{1.348729in}{2.028997in}}%
\pgfpathlineto{\pgfqpoint{1.334596in}{2.043784in}}%
\pgfpathlineto{\pgfqpoint{1.320463in}{2.043784in}}%
\pgfpathlineto{\pgfqpoint{1.306331in}{2.043784in}}%
\pgfpathlineto{\pgfqpoint{1.292198in}{2.043784in}}%
\pgfpathlineto{\pgfqpoint{1.278066in}{2.043784in}}%
\pgfpathlineto{\pgfqpoint{1.263933in}{2.043784in}}%
\pgfpathlineto{\pgfqpoint{1.249800in}{2.043784in}}%
\pgfpathlineto{\pgfqpoint{1.235668in}{2.043784in}}%
\pgfpathlineto{\pgfqpoint{1.221535in}{2.063124in}}%
\pgfpathlineto{\pgfqpoint{1.207403in}{2.064740in}}%
\pgfpathlineto{\pgfqpoint{1.193270in}{2.079941in}}%
\pgfpathlineto{\pgfqpoint{1.179137in}{2.079941in}}%
\pgfpathlineto{\pgfqpoint{1.165005in}{2.079941in}}%
\pgfpathlineto{\pgfqpoint{1.150872in}{2.079941in}}%
\pgfpathlineto{\pgfqpoint{1.136740in}{2.079941in}}%
\pgfpathlineto{\pgfqpoint{1.122607in}{2.079941in}}%
\pgfpathlineto{\pgfqpoint{1.108474in}{2.079941in}}%
\pgfpathlineto{\pgfqpoint{1.094342in}{2.091898in}}%
\pgfpathlineto{\pgfqpoint{1.080209in}{2.091898in}}%
\pgfpathlineto{\pgfqpoint{1.066077in}{2.091898in}}%
\pgfpathlineto{\pgfqpoint{1.051944in}{2.091898in}}%
\pgfpathlineto{\pgfqpoint{1.037811in}{2.091898in}}%
\pgfpathlineto{\pgfqpoint{1.023679in}{2.091898in}}%
\pgfpathlineto{\pgfqpoint{1.009546in}{2.091898in}}%
\pgfpathlineto{\pgfqpoint{0.995414in}{2.091898in}}%
\pgfpathlineto{\pgfqpoint{0.981281in}{2.091898in}}%
\pgfpathlineto{\pgfqpoint{0.967148in}{2.091898in}}%
\pgfpathlineto{\pgfqpoint{0.953016in}{2.088406in}}%
\pgfpathlineto{\pgfqpoint{0.938883in}{2.088406in}}%
\pgfpathlineto{\pgfqpoint{0.924751in}{2.088406in}}%
\pgfpathlineto{\pgfqpoint{0.910618in}{2.088496in}}%
\pgfpathlineto{\pgfqpoint{0.896485in}{2.088496in}}%
\pgfpathlineto{\pgfqpoint{0.882353in}{2.092729in}}%
\pgfpathlineto{\pgfqpoint{0.868220in}{2.092729in}}%
\pgfpathlineto{\pgfqpoint{0.854088in}{2.092729in}}%
\pgfpathlineto{\pgfqpoint{0.839955in}{2.096416in}}%
\pgfpathlineto{\pgfqpoint{0.825822in}{2.096416in}}%
\pgfpathlineto{\pgfqpoint{0.811690in}{2.096416in}}%
\pgfpathlineto{\pgfqpoint{0.797557in}{2.097040in}}%
\pgfpathlineto{\pgfqpoint{0.783425in}{2.097040in}}%
\pgfpathlineto{\pgfqpoint{0.769292in}{2.097040in}}%
\pgfpathlineto{\pgfqpoint{0.755159in}{2.123936in}}%
\pgfpathlineto{\pgfqpoint{0.741027in}{2.137226in}}%
\pgfpathlineto{\pgfqpoint{0.726894in}{2.152082in}}%
\pgfpathlineto{\pgfqpoint{0.712762in}{2.159079in}}%
\pgfpathlineto{\pgfqpoint{0.698629in}{2.159079in}}%
\pgfpathlineto{\pgfqpoint{0.684496in}{2.159079in}}%
\pgfpathlineto{\pgfqpoint{0.670364in}{2.189929in}}%
\pgfpathlineto{\pgfqpoint{0.656231in}{2.195647in}}%
\pgfpathlineto{\pgfqpoint{0.642099in}{2.219936in}}%
\pgfpathlineto{\pgfqpoint{0.627966in}{2.234343in}}%
\pgfpathlineto{\pgfqpoint{0.613833in}{2.246036in}}%
\pgfpathlineto{\pgfqpoint{0.599701in}{2.246036in}}%
\pgfpathlineto{\pgfqpoint{0.585568in}{2.256707in}}%
\pgfpathlineto{\pgfqpoint{0.571436in}{2.268697in}}%
\pgfpathlineto{\pgfqpoint{0.557303in}{2.335062in}}%
\pgfpathlineto{\pgfqpoint{0.543170in}{2.409843in}}%
\pgfpathclose%
\pgfusepath{fill}%
\end{pgfscope}%
\begin{pgfscope}%
\pgfpathrectangle{\pgfqpoint{0.423750in}{0.375000in}}{\pgfqpoint{2.627250in}{2.265000in}}%
\pgfusepath{clip}%
\pgfsetbuttcap%
\pgfsetroundjoin%
\definecolor{currentfill}{rgb}{1.000000,0.498039,0.054902}%
\pgfsetfillcolor{currentfill}%
\pgfsetfillopacity{0.200000}%
\pgfsetlinewidth{0.000000pt}%
\definecolor{currentstroke}{rgb}{0.000000,0.000000,0.000000}%
\pgfsetstrokecolor{currentstroke}%
\pgfsetdash{}{0pt}%
\pgfpathmoveto{\pgfqpoint{0.543170in}{2.403862in}}%
\pgfpathlineto{\pgfqpoint{0.543170in}{2.484452in}}%
\pgfpathlineto{\pgfqpoint{0.557303in}{2.411854in}}%
\pgfpathlineto{\pgfqpoint{0.571436in}{2.411854in}}%
\pgfpathlineto{\pgfqpoint{0.585568in}{2.377812in}}%
\pgfpathlineto{\pgfqpoint{0.599701in}{2.371916in}}%
\pgfpathlineto{\pgfqpoint{0.613833in}{2.371916in}}%
\pgfpathlineto{\pgfqpoint{0.627966in}{2.310148in}}%
\pgfpathlineto{\pgfqpoint{0.642099in}{2.278202in}}%
\pgfpathlineto{\pgfqpoint{0.656231in}{2.278202in}}%
\pgfpathlineto{\pgfqpoint{0.670364in}{2.276133in}}%
\pgfpathlineto{\pgfqpoint{0.684496in}{2.223576in}}%
\pgfpathlineto{\pgfqpoint{0.698629in}{2.223576in}}%
\pgfpathlineto{\pgfqpoint{0.712762in}{2.223576in}}%
\pgfpathlineto{\pgfqpoint{0.726894in}{2.223576in}}%
\pgfpathlineto{\pgfqpoint{0.741027in}{2.223576in}}%
\pgfpathlineto{\pgfqpoint{0.755159in}{2.223576in}}%
\pgfpathlineto{\pgfqpoint{0.769292in}{2.223576in}}%
\pgfpathlineto{\pgfqpoint{0.783425in}{2.223576in}}%
\pgfpathlineto{\pgfqpoint{0.797557in}{2.216325in}}%
\pgfpathlineto{\pgfqpoint{0.811690in}{2.197242in}}%
\pgfpathlineto{\pgfqpoint{0.825822in}{2.156452in}}%
\pgfpathlineto{\pgfqpoint{0.839955in}{2.156452in}}%
\pgfpathlineto{\pgfqpoint{0.854088in}{2.156017in}}%
\pgfpathlineto{\pgfqpoint{0.868220in}{2.137729in}}%
\pgfpathlineto{\pgfqpoint{0.882353in}{2.137729in}}%
\pgfpathlineto{\pgfqpoint{0.896485in}{2.100561in}}%
\pgfpathlineto{\pgfqpoint{0.910618in}{2.100402in}}%
\pgfpathlineto{\pgfqpoint{0.924751in}{2.098372in}}%
\pgfpathlineto{\pgfqpoint{0.938883in}{2.077169in}}%
\pgfpathlineto{\pgfqpoint{0.953016in}{2.076266in}}%
\pgfpathlineto{\pgfqpoint{0.967148in}{2.076266in}}%
\pgfpathlineto{\pgfqpoint{0.981281in}{2.060354in}}%
\pgfpathlineto{\pgfqpoint{0.995414in}{2.060354in}}%
\pgfpathlineto{\pgfqpoint{1.009546in}{2.045683in}}%
\pgfpathlineto{\pgfqpoint{1.023679in}{2.045683in}}%
\pgfpathlineto{\pgfqpoint{1.037811in}{2.045683in}}%
\pgfpathlineto{\pgfqpoint{1.051944in}{2.045683in}}%
\pgfpathlineto{\pgfqpoint{1.066077in}{2.025609in}}%
\pgfpathlineto{\pgfqpoint{1.080209in}{1.998535in}}%
\pgfpathlineto{\pgfqpoint{1.094342in}{1.998535in}}%
\pgfpathlineto{\pgfqpoint{1.108474in}{1.998535in}}%
\pgfpathlineto{\pgfqpoint{1.122607in}{1.975626in}}%
\pgfpathlineto{\pgfqpoint{1.136740in}{1.975626in}}%
\pgfpathlineto{\pgfqpoint{1.150872in}{1.975626in}}%
\pgfpathlineto{\pgfqpoint{1.165005in}{1.975626in}}%
\pgfpathlineto{\pgfqpoint{1.179137in}{1.975626in}}%
\pgfpathlineto{\pgfqpoint{1.193270in}{1.975626in}}%
\pgfpathlineto{\pgfqpoint{1.207403in}{1.975626in}}%
\pgfpathlineto{\pgfqpoint{1.221535in}{1.975626in}}%
\pgfpathlineto{\pgfqpoint{1.235668in}{1.975626in}}%
\pgfpathlineto{\pgfqpoint{1.249800in}{1.975626in}}%
\pgfpathlineto{\pgfqpoint{1.263933in}{1.975626in}}%
\pgfpathlineto{\pgfqpoint{1.278066in}{1.975626in}}%
\pgfpathlineto{\pgfqpoint{1.292198in}{1.949990in}}%
\pgfpathlineto{\pgfqpoint{1.306331in}{1.949990in}}%
\pgfpathlineto{\pgfqpoint{1.320463in}{1.905840in}}%
\pgfpathlineto{\pgfqpoint{1.334596in}{1.905840in}}%
\pgfpathlineto{\pgfqpoint{1.348729in}{1.905840in}}%
\pgfpathlineto{\pgfqpoint{1.362861in}{1.905840in}}%
\pgfpathlineto{\pgfqpoint{1.376994in}{1.905840in}}%
\pgfpathlineto{\pgfqpoint{1.391126in}{1.905840in}}%
\pgfpathlineto{\pgfqpoint{1.405259in}{1.905840in}}%
\pgfpathlineto{\pgfqpoint{1.419392in}{1.905840in}}%
\pgfpathlineto{\pgfqpoint{1.433524in}{1.905840in}}%
\pgfpathlineto{\pgfqpoint{1.447657in}{1.905840in}}%
\pgfpathlineto{\pgfqpoint{1.461789in}{1.905840in}}%
\pgfpathlineto{\pgfqpoint{1.475922in}{1.905840in}}%
\pgfpathlineto{\pgfqpoint{1.490055in}{1.903431in}}%
\pgfpathlineto{\pgfqpoint{1.504187in}{1.903431in}}%
\pgfpathlineto{\pgfqpoint{1.518320in}{1.903431in}}%
\pgfpathlineto{\pgfqpoint{1.532452in}{1.903431in}}%
\pgfpathlineto{\pgfqpoint{1.546585in}{1.895356in}}%
\pgfpathlineto{\pgfqpoint{1.560718in}{1.895356in}}%
\pgfpathlineto{\pgfqpoint{1.574850in}{1.834197in}}%
\pgfpathlineto{\pgfqpoint{1.588983in}{1.830184in}}%
\pgfpathlineto{\pgfqpoint{1.603115in}{1.830184in}}%
\pgfpathlineto{\pgfqpoint{1.617248in}{1.830184in}}%
\pgfpathlineto{\pgfqpoint{1.631381in}{1.830184in}}%
\pgfpathlineto{\pgfqpoint{1.645513in}{1.825554in}}%
\pgfpathlineto{\pgfqpoint{1.659646in}{1.825554in}}%
\pgfpathlineto{\pgfqpoint{1.673778in}{1.825554in}}%
\pgfpathlineto{\pgfqpoint{1.687911in}{1.825554in}}%
\pgfpathlineto{\pgfqpoint{1.702044in}{1.825554in}}%
\pgfpathlineto{\pgfqpoint{1.716176in}{1.825554in}}%
\pgfpathlineto{\pgfqpoint{1.730309in}{1.825554in}}%
\pgfpathlineto{\pgfqpoint{1.744441in}{1.807041in}}%
\pgfpathlineto{\pgfqpoint{1.758574in}{1.807041in}}%
\pgfpathlineto{\pgfqpoint{1.772706in}{1.807041in}}%
\pgfpathlineto{\pgfqpoint{1.786839in}{1.807041in}}%
\pgfpathlineto{\pgfqpoint{1.800972in}{1.807041in}}%
\pgfpathlineto{\pgfqpoint{1.815104in}{1.807041in}}%
\pgfpathlineto{\pgfqpoint{1.829237in}{1.807041in}}%
\pgfpathlineto{\pgfqpoint{1.843369in}{1.807041in}}%
\pgfpathlineto{\pgfqpoint{1.857502in}{1.754110in}}%
\pgfpathlineto{\pgfqpoint{1.871635in}{1.754110in}}%
\pgfpathlineto{\pgfqpoint{1.885767in}{1.754110in}}%
\pgfpathlineto{\pgfqpoint{1.899900in}{1.754110in}}%
\pgfpathlineto{\pgfqpoint{1.914032in}{1.754110in}}%
\pgfpathlineto{\pgfqpoint{1.928165in}{1.754110in}}%
\pgfpathlineto{\pgfqpoint{1.942298in}{1.754110in}}%
\pgfpathlineto{\pgfqpoint{1.956430in}{1.754110in}}%
\pgfpathlineto{\pgfqpoint{1.970563in}{1.754110in}}%
\pgfpathlineto{\pgfqpoint{1.984695in}{1.754110in}}%
\pgfpathlineto{\pgfqpoint{1.998828in}{1.751205in}}%
\pgfpathlineto{\pgfqpoint{2.012961in}{1.751205in}}%
\pgfpathlineto{\pgfqpoint{2.027093in}{1.751205in}}%
\pgfpathlineto{\pgfqpoint{2.041226in}{1.751205in}}%
\pgfpathlineto{\pgfqpoint{2.055358in}{1.751205in}}%
\pgfpathlineto{\pgfqpoint{2.069491in}{1.751205in}}%
\pgfpathlineto{\pgfqpoint{2.083624in}{1.751205in}}%
\pgfpathlineto{\pgfqpoint{2.097756in}{1.751205in}}%
\pgfpathlineto{\pgfqpoint{2.111889in}{1.751205in}}%
\pgfpathlineto{\pgfqpoint{2.126021in}{1.751205in}}%
\pgfpathlineto{\pgfqpoint{2.140154in}{1.751205in}}%
\pgfpathlineto{\pgfqpoint{2.154287in}{1.751205in}}%
\pgfpathlineto{\pgfqpoint{2.168419in}{1.751205in}}%
\pgfpathlineto{\pgfqpoint{2.182552in}{1.751205in}}%
\pgfpathlineto{\pgfqpoint{2.196684in}{1.751205in}}%
\pgfpathlineto{\pgfqpoint{2.210817in}{1.751205in}}%
\pgfpathlineto{\pgfqpoint{2.224950in}{1.712671in}}%
\pgfpathlineto{\pgfqpoint{2.239082in}{1.712671in}}%
\pgfpathlineto{\pgfqpoint{2.253215in}{1.712671in}}%
\pgfpathlineto{\pgfqpoint{2.267347in}{1.712671in}}%
\pgfpathlineto{\pgfqpoint{2.281480in}{1.712671in}}%
\pgfpathlineto{\pgfqpoint{2.295613in}{1.712671in}}%
\pgfpathlineto{\pgfqpoint{2.309745in}{1.712671in}}%
\pgfpathlineto{\pgfqpoint{2.323878in}{1.712671in}}%
\pgfpathlineto{\pgfqpoint{2.338010in}{1.712671in}}%
\pgfpathlineto{\pgfqpoint{2.352143in}{1.712671in}}%
\pgfpathlineto{\pgfqpoint{2.366276in}{1.712671in}}%
\pgfpathlineto{\pgfqpoint{2.380408in}{1.712671in}}%
\pgfpathlineto{\pgfqpoint{2.394541in}{1.712671in}}%
\pgfpathlineto{\pgfqpoint{2.408673in}{1.712671in}}%
\pgfpathlineto{\pgfqpoint{2.422806in}{1.712671in}}%
\pgfpathlineto{\pgfqpoint{2.436939in}{1.712671in}}%
\pgfpathlineto{\pgfqpoint{2.451071in}{1.712671in}}%
\pgfpathlineto{\pgfqpoint{2.465204in}{1.712671in}}%
\pgfpathlineto{\pgfqpoint{2.479336in}{1.712671in}}%
\pgfpathlineto{\pgfqpoint{2.493469in}{1.712671in}}%
\pgfpathlineto{\pgfqpoint{2.507602in}{1.712671in}}%
\pgfpathlineto{\pgfqpoint{2.521734in}{1.712671in}}%
\pgfpathlineto{\pgfqpoint{2.535867in}{1.712671in}}%
\pgfpathlineto{\pgfqpoint{2.549999in}{1.712671in}}%
\pgfpathlineto{\pgfqpoint{2.564132in}{1.711177in}}%
\pgfpathlineto{\pgfqpoint{2.578265in}{1.711177in}}%
\pgfpathlineto{\pgfqpoint{2.592397in}{1.711177in}}%
\pgfpathlineto{\pgfqpoint{2.606530in}{1.711177in}}%
\pgfpathlineto{\pgfqpoint{2.620662in}{1.711177in}}%
\pgfpathlineto{\pgfqpoint{2.634795in}{1.711177in}}%
\pgfpathlineto{\pgfqpoint{2.648928in}{1.711177in}}%
\pgfpathlineto{\pgfqpoint{2.663060in}{1.711177in}}%
\pgfpathlineto{\pgfqpoint{2.677193in}{1.711177in}}%
\pgfpathlineto{\pgfqpoint{2.691325in}{1.711177in}}%
\pgfpathlineto{\pgfqpoint{2.705458in}{1.711177in}}%
\pgfpathlineto{\pgfqpoint{2.719591in}{1.711177in}}%
\pgfpathlineto{\pgfqpoint{2.733723in}{1.711177in}}%
\pgfpathlineto{\pgfqpoint{2.747856in}{1.711177in}}%
\pgfpathlineto{\pgfqpoint{2.761988in}{1.711177in}}%
\pgfpathlineto{\pgfqpoint{2.776121in}{1.711177in}}%
\pgfpathlineto{\pgfqpoint{2.790254in}{1.711177in}}%
\pgfpathlineto{\pgfqpoint{2.804386in}{1.711177in}}%
\pgfpathlineto{\pgfqpoint{2.818519in}{1.695898in}}%
\pgfpathlineto{\pgfqpoint{2.832651in}{1.695898in}}%
\pgfpathlineto{\pgfqpoint{2.846784in}{1.695898in}}%
\pgfpathlineto{\pgfqpoint{2.860917in}{1.695898in}}%
\pgfpathlineto{\pgfqpoint{2.875049in}{1.695898in}}%
\pgfpathlineto{\pgfqpoint{2.889182in}{1.695139in}}%
\pgfpathlineto{\pgfqpoint{2.903314in}{1.695139in}}%
\pgfpathlineto{\pgfqpoint{2.917447in}{1.695139in}}%
\pgfpathlineto{\pgfqpoint{2.931580in}{1.695139in}}%
\pgfpathlineto{\pgfqpoint{2.931580in}{1.612526in}}%
\pgfpathlineto{\pgfqpoint{2.931580in}{1.612526in}}%
\pgfpathlineto{\pgfqpoint{2.917447in}{1.612526in}}%
\pgfpathlineto{\pgfqpoint{2.903314in}{1.612526in}}%
\pgfpathlineto{\pgfqpoint{2.889182in}{1.612526in}}%
\pgfpathlineto{\pgfqpoint{2.875049in}{1.617467in}}%
\pgfpathlineto{\pgfqpoint{2.860917in}{1.617467in}}%
\pgfpathlineto{\pgfqpoint{2.846784in}{1.617467in}}%
\pgfpathlineto{\pgfqpoint{2.832651in}{1.617467in}}%
\pgfpathlineto{\pgfqpoint{2.818519in}{1.617467in}}%
\pgfpathlineto{\pgfqpoint{2.804386in}{1.625137in}}%
\pgfpathlineto{\pgfqpoint{2.790254in}{1.625137in}}%
\pgfpathlineto{\pgfqpoint{2.776121in}{1.625137in}}%
\pgfpathlineto{\pgfqpoint{2.761988in}{1.625137in}}%
\pgfpathlineto{\pgfqpoint{2.747856in}{1.625137in}}%
\pgfpathlineto{\pgfqpoint{2.733723in}{1.625137in}}%
\pgfpathlineto{\pgfqpoint{2.719591in}{1.625137in}}%
\pgfpathlineto{\pgfqpoint{2.705458in}{1.625137in}}%
\pgfpathlineto{\pgfqpoint{2.691325in}{1.625137in}}%
\pgfpathlineto{\pgfqpoint{2.677193in}{1.625137in}}%
\pgfpathlineto{\pgfqpoint{2.663060in}{1.625137in}}%
\pgfpathlineto{\pgfqpoint{2.648928in}{1.625137in}}%
\pgfpathlineto{\pgfqpoint{2.634795in}{1.625137in}}%
\pgfpathlineto{\pgfqpoint{2.620662in}{1.625137in}}%
\pgfpathlineto{\pgfqpoint{2.606530in}{1.625137in}}%
\pgfpathlineto{\pgfqpoint{2.592397in}{1.625137in}}%
\pgfpathlineto{\pgfqpoint{2.578265in}{1.625137in}}%
\pgfpathlineto{\pgfqpoint{2.564132in}{1.625137in}}%
\pgfpathlineto{\pgfqpoint{2.549999in}{1.633370in}}%
\pgfpathlineto{\pgfqpoint{2.535867in}{1.633370in}}%
\pgfpathlineto{\pgfqpoint{2.521734in}{1.633370in}}%
\pgfpathlineto{\pgfqpoint{2.507602in}{1.633370in}}%
\pgfpathlineto{\pgfqpoint{2.493469in}{1.633370in}}%
\pgfpathlineto{\pgfqpoint{2.479336in}{1.633370in}}%
\pgfpathlineto{\pgfqpoint{2.465204in}{1.633370in}}%
\pgfpathlineto{\pgfqpoint{2.451071in}{1.633370in}}%
\pgfpathlineto{\pgfqpoint{2.436939in}{1.633370in}}%
\pgfpathlineto{\pgfqpoint{2.422806in}{1.633370in}}%
\pgfpathlineto{\pgfqpoint{2.408673in}{1.633370in}}%
\pgfpathlineto{\pgfqpoint{2.394541in}{1.633370in}}%
\pgfpathlineto{\pgfqpoint{2.380408in}{1.633370in}}%
\pgfpathlineto{\pgfqpoint{2.366276in}{1.633370in}}%
\pgfpathlineto{\pgfqpoint{2.352143in}{1.633370in}}%
\pgfpathlineto{\pgfqpoint{2.338010in}{1.633370in}}%
\pgfpathlineto{\pgfqpoint{2.323878in}{1.633370in}}%
\pgfpathlineto{\pgfqpoint{2.309745in}{1.633370in}}%
\pgfpathlineto{\pgfqpoint{2.295613in}{1.633370in}}%
\pgfpathlineto{\pgfqpoint{2.281480in}{1.633370in}}%
\pgfpathlineto{\pgfqpoint{2.267347in}{1.633370in}}%
\pgfpathlineto{\pgfqpoint{2.253215in}{1.633370in}}%
\pgfpathlineto{\pgfqpoint{2.239082in}{1.633370in}}%
\pgfpathlineto{\pgfqpoint{2.224950in}{1.633370in}}%
\pgfpathlineto{\pgfqpoint{2.210817in}{1.677820in}}%
\pgfpathlineto{\pgfqpoint{2.196684in}{1.677820in}}%
\pgfpathlineto{\pgfqpoint{2.182552in}{1.677820in}}%
\pgfpathlineto{\pgfqpoint{2.168419in}{1.677820in}}%
\pgfpathlineto{\pgfqpoint{2.154287in}{1.677820in}}%
\pgfpathlineto{\pgfqpoint{2.140154in}{1.677820in}}%
\pgfpathlineto{\pgfqpoint{2.126021in}{1.677820in}}%
\pgfpathlineto{\pgfqpoint{2.111889in}{1.677820in}}%
\pgfpathlineto{\pgfqpoint{2.097756in}{1.677820in}}%
\pgfpathlineto{\pgfqpoint{2.083624in}{1.677820in}}%
\pgfpathlineto{\pgfqpoint{2.069491in}{1.677820in}}%
\pgfpathlineto{\pgfqpoint{2.055358in}{1.677820in}}%
\pgfpathlineto{\pgfqpoint{2.041226in}{1.677820in}}%
\pgfpathlineto{\pgfqpoint{2.027093in}{1.677820in}}%
\pgfpathlineto{\pgfqpoint{2.012961in}{1.677820in}}%
\pgfpathlineto{\pgfqpoint{1.998828in}{1.677820in}}%
\pgfpathlineto{\pgfqpoint{1.984695in}{1.691754in}}%
\pgfpathlineto{\pgfqpoint{1.970563in}{1.691754in}}%
\pgfpathlineto{\pgfqpoint{1.956430in}{1.691754in}}%
\pgfpathlineto{\pgfqpoint{1.942298in}{1.691754in}}%
\pgfpathlineto{\pgfqpoint{1.928165in}{1.691754in}}%
\pgfpathlineto{\pgfqpoint{1.914032in}{1.691754in}}%
\pgfpathlineto{\pgfqpoint{1.899900in}{1.691754in}}%
\pgfpathlineto{\pgfqpoint{1.885767in}{1.691754in}}%
\pgfpathlineto{\pgfqpoint{1.871635in}{1.691754in}}%
\pgfpathlineto{\pgfqpoint{1.857502in}{1.691754in}}%
\pgfpathlineto{\pgfqpoint{1.843369in}{1.722713in}}%
\pgfpathlineto{\pgfqpoint{1.829237in}{1.722713in}}%
\pgfpathlineto{\pgfqpoint{1.815104in}{1.722713in}}%
\pgfpathlineto{\pgfqpoint{1.800972in}{1.722713in}}%
\pgfpathlineto{\pgfqpoint{1.786839in}{1.722713in}}%
\pgfpathlineto{\pgfqpoint{1.772706in}{1.722713in}}%
\pgfpathlineto{\pgfqpoint{1.758574in}{1.722713in}}%
\pgfpathlineto{\pgfqpoint{1.744441in}{1.722713in}}%
\pgfpathlineto{\pgfqpoint{1.730309in}{1.762955in}}%
\pgfpathlineto{\pgfqpoint{1.716176in}{1.762955in}}%
\pgfpathlineto{\pgfqpoint{1.702044in}{1.762955in}}%
\pgfpathlineto{\pgfqpoint{1.687911in}{1.762955in}}%
\pgfpathlineto{\pgfqpoint{1.673778in}{1.762955in}}%
\pgfpathlineto{\pgfqpoint{1.659646in}{1.762955in}}%
\pgfpathlineto{\pgfqpoint{1.645513in}{1.762955in}}%
\pgfpathlineto{\pgfqpoint{1.631381in}{1.767027in}}%
\pgfpathlineto{\pgfqpoint{1.617248in}{1.767027in}}%
\pgfpathlineto{\pgfqpoint{1.603115in}{1.767027in}}%
\pgfpathlineto{\pgfqpoint{1.588983in}{1.767027in}}%
\pgfpathlineto{\pgfqpoint{1.574850in}{1.769896in}}%
\pgfpathlineto{\pgfqpoint{1.560718in}{1.813356in}}%
\pgfpathlineto{\pgfqpoint{1.546585in}{1.813356in}}%
\pgfpathlineto{\pgfqpoint{1.532452in}{1.827159in}}%
\pgfpathlineto{\pgfqpoint{1.518320in}{1.827159in}}%
\pgfpathlineto{\pgfqpoint{1.504187in}{1.827159in}}%
\pgfpathlineto{\pgfqpoint{1.490055in}{1.827159in}}%
\pgfpathlineto{\pgfqpoint{1.475922in}{1.830141in}}%
\pgfpathlineto{\pgfqpoint{1.461789in}{1.830141in}}%
\pgfpathlineto{\pgfqpoint{1.447657in}{1.830141in}}%
\pgfpathlineto{\pgfqpoint{1.433524in}{1.830141in}}%
\pgfpathlineto{\pgfqpoint{1.419392in}{1.830141in}}%
\pgfpathlineto{\pgfqpoint{1.405259in}{1.830141in}}%
\pgfpathlineto{\pgfqpoint{1.391126in}{1.830141in}}%
\pgfpathlineto{\pgfqpoint{1.376994in}{1.830141in}}%
\pgfpathlineto{\pgfqpoint{1.362861in}{1.830141in}}%
\pgfpathlineto{\pgfqpoint{1.348729in}{1.830141in}}%
\pgfpathlineto{\pgfqpoint{1.334596in}{1.830141in}}%
\pgfpathlineto{\pgfqpoint{1.320463in}{1.830141in}}%
\pgfpathlineto{\pgfqpoint{1.306331in}{1.878990in}}%
\pgfpathlineto{\pgfqpoint{1.292198in}{1.878990in}}%
\pgfpathlineto{\pgfqpoint{1.278066in}{1.886037in}}%
\pgfpathlineto{\pgfqpoint{1.263933in}{1.886037in}}%
\pgfpathlineto{\pgfqpoint{1.249800in}{1.886037in}}%
\pgfpathlineto{\pgfqpoint{1.235668in}{1.886037in}}%
\pgfpathlineto{\pgfqpoint{1.221535in}{1.886037in}}%
\pgfpathlineto{\pgfqpoint{1.207403in}{1.886037in}}%
\pgfpathlineto{\pgfqpoint{1.193270in}{1.886037in}}%
\pgfpathlineto{\pgfqpoint{1.179137in}{1.886037in}}%
\pgfpathlineto{\pgfqpoint{1.165005in}{1.886037in}}%
\pgfpathlineto{\pgfqpoint{1.150872in}{1.886037in}}%
\pgfpathlineto{\pgfqpoint{1.136740in}{1.886037in}}%
\pgfpathlineto{\pgfqpoint{1.122607in}{1.886037in}}%
\pgfpathlineto{\pgfqpoint{1.108474in}{1.890954in}}%
\pgfpathlineto{\pgfqpoint{1.094342in}{1.890954in}}%
\pgfpathlineto{\pgfqpoint{1.080209in}{1.890954in}}%
\pgfpathlineto{\pgfqpoint{1.066077in}{1.938534in}}%
\pgfpathlineto{\pgfqpoint{1.051944in}{1.951195in}}%
\pgfpathlineto{\pgfqpoint{1.037811in}{1.951195in}}%
\pgfpathlineto{\pgfqpoint{1.023679in}{1.951195in}}%
\pgfpathlineto{\pgfqpoint{1.009546in}{1.951195in}}%
\pgfpathlineto{\pgfqpoint{0.995414in}{1.962973in}}%
\pgfpathlineto{\pgfqpoint{0.981281in}{1.962973in}}%
\pgfpathlineto{\pgfqpoint{0.967148in}{1.998992in}}%
\pgfpathlineto{\pgfqpoint{0.953016in}{1.998992in}}%
\pgfpathlineto{\pgfqpoint{0.938883in}{1.999779in}}%
\pgfpathlineto{\pgfqpoint{0.924751in}{2.011453in}}%
\pgfpathlineto{\pgfqpoint{0.910618in}{2.025663in}}%
\pgfpathlineto{\pgfqpoint{0.896485in}{2.026553in}}%
\pgfpathlineto{\pgfqpoint{0.882353in}{2.056476in}}%
\pgfpathlineto{\pgfqpoint{0.868220in}{2.056476in}}%
\pgfpathlineto{\pgfqpoint{0.854088in}{2.105889in}}%
\pgfpathlineto{\pgfqpoint{0.839955in}{2.106900in}}%
\pgfpathlineto{\pgfqpoint{0.825822in}{2.106900in}}%
\pgfpathlineto{\pgfqpoint{0.811690in}{2.132968in}}%
\pgfpathlineto{\pgfqpoint{0.797557in}{2.137038in}}%
\pgfpathlineto{\pgfqpoint{0.783425in}{2.148747in}}%
\pgfpathlineto{\pgfqpoint{0.769292in}{2.148747in}}%
\pgfpathlineto{\pgfqpoint{0.755159in}{2.148747in}}%
\pgfpathlineto{\pgfqpoint{0.741027in}{2.148747in}}%
\pgfpathlineto{\pgfqpoint{0.726894in}{2.148747in}}%
\pgfpathlineto{\pgfqpoint{0.712762in}{2.148747in}}%
\pgfpathlineto{\pgfqpoint{0.698629in}{2.148747in}}%
\pgfpathlineto{\pgfqpoint{0.684496in}{2.148747in}}%
\pgfpathlineto{\pgfqpoint{0.670364in}{2.234131in}}%
\pgfpathlineto{\pgfqpoint{0.656231in}{2.249789in}}%
\pgfpathlineto{\pgfqpoint{0.642099in}{2.249789in}}%
\pgfpathlineto{\pgfqpoint{0.627966in}{2.269972in}}%
\pgfpathlineto{\pgfqpoint{0.613833in}{2.290381in}}%
\pgfpathlineto{\pgfqpoint{0.599701in}{2.290381in}}%
\pgfpathlineto{\pgfqpoint{0.585568in}{2.306599in}}%
\pgfpathlineto{\pgfqpoint{0.571436in}{2.378233in}}%
\pgfpathlineto{\pgfqpoint{0.557303in}{2.378233in}}%
\pgfpathlineto{\pgfqpoint{0.543170in}{2.403862in}}%
\pgfpathclose%
\pgfusepath{fill}%
\end{pgfscope}%
\begin{pgfscope}%
\pgfpathrectangle{\pgfqpoint{0.423750in}{0.375000in}}{\pgfqpoint{2.627250in}{2.265000in}}%
\pgfusepath{clip}%
\pgfsetbuttcap%
\pgfsetroundjoin%
\definecolor{currentfill}{rgb}{0.172549,0.627451,0.172549}%
\pgfsetfillcolor{currentfill}%
\pgfsetfillopacity{0.200000}%
\pgfsetlinewidth{0.000000pt}%
\definecolor{currentstroke}{rgb}{0.000000,0.000000,0.000000}%
\pgfsetstrokecolor{currentstroke}%
\pgfsetdash{}{0pt}%
\pgfpathmoveto{\pgfqpoint{0.543170in}{2.405828in}}%
\pgfpathlineto{\pgfqpoint{0.543170in}{2.463435in}}%
\pgfpathlineto{\pgfqpoint{0.557303in}{2.406995in}}%
\pgfpathlineto{\pgfqpoint{0.571436in}{2.375951in}}%
\pgfpathlineto{\pgfqpoint{0.585568in}{2.344480in}}%
\pgfpathlineto{\pgfqpoint{0.599701in}{2.344480in}}%
\pgfpathlineto{\pgfqpoint{0.613833in}{2.307847in}}%
\pgfpathlineto{\pgfqpoint{0.627966in}{2.293758in}}%
\pgfpathlineto{\pgfqpoint{0.642099in}{2.293758in}}%
\pgfpathlineto{\pgfqpoint{0.656231in}{2.293758in}}%
\pgfpathlineto{\pgfqpoint{0.670364in}{2.284088in}}%
\pgfpathlineto{\pgfqpoint{0.684496in}{2.284088in}}%
\pgfpathlineto{\pgfqpoint{0.698629in}{2.284088in}}%
\pgfpathlineto{\pgfqpoint{0.712762in}{2.284088in}}%
\pgfpathlineto{\pgfqpoint{0.726894in}{2.284088in}}%
\pgfpathlineto{\pgfqpoint{0.741027in}{2.279203in}}%
\pgfpathlineto{\pgfqpoint{0.755159in}{2.279203in}}%
\pgfpathlineto{\pgfqpoint{0.769292in}{2.279203in}}%
\pgfpathlineto{\pgfqpoint{0.783425in}{2.279203in}}%
\pgfpathlineto{\pgfqpoint{0.797557in}{2.279203in}}%
\pgfpathlineto{\pgfqpoint{0.811690in}{2.279203in}}%
\pgfpathlineto{\pgfqpoint{0.825822in}{2.184074in}}%
\pgfpathlineto{\pgfqpoint{0.839955in}{2.178199in}}%
\pgfpathlineto{\pgfqpoint{0.854088in}{2.178199in}}%
\pgfpathlineto{\pgfqpoint{0.868220in}{2.114298in}}%
\pgfpathlineto{\pgfqpoint{0.882353in}{2.112954in}}%
\pgfpathlineto{\pgfqpoint{0.896485in}{2.109602in}}%
\pgfpathlineto{\pgfqpoint{0.910618in}{2.089656in}}%
\pgfpathlineto{\pgfqpoint{0.924751in}{2.061203in}}%
\pgfpathlineto{\pgfqpoint{0.938883in}{2.001520in}}%
\pgfpathlineto{\pgfqpoint{0.953016in}{2.001520in}}%
\pgfpathlineto{\pgfqpoint{0.967148in}{1.991400in}}%
\pgfpathlineto{\pgfqpoint{0.981281in}{1.987398in}}%
\pgfpathlineto{\pgfqpoint{0.995414in}{1.987398in}}%
\pgfpathlineto{\pgfqpoint{1.009546in}{1.985415in}}%
\pgfpathlineto{\pgfqpoint{1.023679in}{1.985415in}}%
\pgfpathlineto{\pgfqpoint{1.037811in}{1.929494in}}%
\pgfpathlineto{\pgfqpoint{1.051944in}{1.929494in}}%
\pgfpathlineto{\pgfqpoint{1.066077in}{1.929494in}}%
\pgfpathlineto{\pgfqpoint{1.080209in}{1.868764in}}%
\pgfpathlineto{\pgfqpoint{1.094342in}{1.868764in}}%
\pgfpathlineto{\pgfqpoint{1.108474in}{1.847333in}}%
\pgfpathlineto{\pgfqpoint{1.122607in}{1.811511in}}%
\pgfpathlineto{\pgfqpoint{1.136740in}{1.811511in}}%
\pgfpathlineto{\pgfqpoint{1.150872in}{1.797575in}}%
\pgfpathlineto{\pgfqpoint{1.165005in}{1.780077in}}%
\pgfpathlineto{\pgfqpoint{1.179137in}{1.780077in}}%
\pgfpathlineto{\pgfqpoint{1.193270in}{1.635863in}}%
\pgfpathlineto{\pgfqpoint{1.207403in}{1.635863in}}%
\pgfpathlineto{\pgfqpoint{1.221535in}{1.635863in}}%
\pgfpathlineto{\pgfqpoint{1.235668in}{1.635863in}}%
\pgfpathlineto{\pgfqpoint{1.249800in}{1.635863in}}%
\pgfpathlineto{\pgfqpoint{1.263933in}{1.635863in}}%
\pgfpathlineto{\pgfqpoint{1.278066in}{1.635863in}}%
\pgfpathlineto{\pgfqpoint{1.292198in}{1.635863in}}%
\pgfpathlineto{\pgfqpoint{1.306331in}{1.635863in}}%
\pgfpathlineto{\pgfqpoint{1.320463in}{1.635863in}}%
\pgfpathlineto{\pgfqpoint{1.334596in}{1.635863in}}%
\pgfpathlineto{\pgfqpoint{1.348729in}{1.635863in}}%
\pgfpathlineto{\pgfqpoint{1.362861in}{1.635863in}}%
\pgfpathlineto{\pgfqpoint{1.376994in}{1.622805in}}%
\pgfpathlineto{\pgfqpoint{1.391126in}{1.622805in}}%
\pgfpathlineto{\pgfqpoint{1.405259in}{1.622805in}}%
\pgfpathlineto{\pgfqpoint{1.419392in}{1.622805in}}%
\pgfpathlineto{\pgfqpoint{1.433524in}{1.622805in}}%
\pgfpathlineto{\pgfqpoint{1.447657in}{1.622805in}}%
\pgfpathlineto{\pgfqpoint{1.461789in}{1.622805in}}%
\pgfpathlineto{\pgfqpoint{1.475922in}{1.622805in}}%
\pgfpathlineto{\pgfqpoint{1.490055in}{1.622805in}}%
\pgfpathlineto{\pgfqpoint{1.504187in}{1.622805in}}%
\pgfpathlineto{\pgfqpoint{1.518320in}{1.581694in}}%
\pgfpathlineto{\pgfqpoint{1.532452in}{1.581694in}}%
\pgfpathlineto{\pgfqpoint{1.546585in}{1.581694in}}%
\pgfpathlineto{\pgfqpoint{1.560718in}{1.581694in}}%
\pgfpathlineto{\pgfqpoint{1.574850in}{1.581694in}}%
\pgfpathlineto{\pgfqpoint{1.588983in}{1.581694in}}%
\pgfpathlineto{\pgfqpoint{1.603115in}{1.581694in}}%
\pgfpathlineto{\pgfqpoint{1.617248in}{1.581694in}}%
\pgfpathlineto{\pgfqpoint{1.631381in}{1.581694in}}%
\pgfpathlineto{\pgfqpoint{1.645513in}{1.581694in}}%
\pgfpathlineto{\pgfqpoint{1.659646in}{1.581694in}}%
\pgfpathlineto{\pgfqpoint{1.673778in}{1.581694in}}%
\pgfpathlineto{\pgfqpoint{1.687911in}{1.581694in}}%
\pgfpathlineto{\pgfqpoint{1.702044in}{1.581694in}}%
\pgfpathlineto{\pgfqpoint{1.716176in}{1.581694in}}%
\pgfpathlineto{\pgfqpoint{1.730309in}{1.581694in}}%
\pgfpathlineto{\pgfqpoint{1.744441in}{1.581694in}}%
\pgfpathlineto{\pgfqpoint{1.758574in}{1.581694in}}%
\pgfpathlineto{\pgfqpoint{1.772706in}{1.566586in}}%
\pgfpathlineto{\pgfqpoint{1.786839in}{1.566586in}}%
\pgfpathlineto{\pgfqpoint{1.800972in}{1.566586in}}%
\pgfpathlineto{\pgfqpoint{1.815104in}{1.566586in}}%
\pgfpathlineto{\pgfqpoint{1.829237in}{1.566586in}}%
\pgfpathlineto{\pgfqpoint{1.843369in}{1.566586in}}%
\pgfpathlineto{\pgfqpoint{1.857502in}{1.566586in}}%
\pgfpathlineto{\pgfqpoint{1.871635in}{1.566586in}}%
\pgfpathlineto{\pgfqpoint{1.885767in}{1.566586in}}%
\pgfpathlineto{\pgfqpoint{1.899900in}{1.566586in}}%
\pgfpathlineto{\pgfqpoint{1.914032in}{1.566586in}}%
\pgfpathlineto{\pgfqpoint{1.928165in}{1.566586in}}%
\pgfpathlineto{\pgfqpoint{1.942298in}{1.566586in}}%
\pgfpathlineto{\pgfqpoint{1.956430in}{1.566586in}}%
\pgfpathlineto{\pgfqpoint{1.970563in}{1.566586in}}%
\pgfpathlineto{\pgfqpoint{1.984695in}{1.566586in}}%
\pgfpathlineto{\pgfqpoint{1.998828in}{1.566586in}}%
\pgfpathlineto{\pgfqpoint{2.012961in}{1.566586in}}%
\pgfpathlineto{\pgfqpoint{2.027093in}{1.530328in}}%
\pgfpathlineto{\pgfqpoint{2.041226in}{1.530328in}}%
\pgfpathlineto{\pgfqpoint{2.055358in}{1.530328in}}%
\pgfpathlineto{\pgfqpoint{2.069491in}{1.530328in}}%
\pgfpathlineto{\pgfqpoint{2.083624in}{1.530328in}}%
\pgfpathlineto{\pgfqpoint{2.097756in}{1.530328in}}%
\pgfpathlineto{\pgfqpoint{2.111889in}{1.530328in}}%
\pgfpathlineto{\pgfqpoint{2.126021in}{1.530328in}}%
\pgfpathlineto{\pgfqpoint{2.140154in}{1.530328in}}%
\pgfpathlineto{\pgfqpoint{2.154287in}{1.530328in}}%
\pgfpathlineto{\pgfqpoint{2.168419in}{1.530328in}}%
\pgfpathlineto{\pgfqpoint{2.182552in}{1.530328in}}%
\pgfpathlineto{\pgfqpoint{2.196684in}{1.530328in}}%
\pgfpathlineto{\pgfqpoint{2.210817in}{1.530328in}}%
\pgfpathlineto{\pgfqpoint{2.224950in}{1.530328in}}%
\pgfpathlineto{\pgfqpoint{2.239082in}{1.530328in}}%
\pgfpathlineto{\pgfqpoint{2.253215in}{1.530328in}}%
\pgfpathlineto{\pgfqpoint{2.267347in}{1.530328in}}%
\pgfpathlineto{\pgfqpoint{2.281480in}{1.530328in}}%
\pgfpathlineto{\pgfqpoint{2.295613in}{1.530328in}}%
\pgfpathlineto{\pgfqpoint{2.309745in}{1.530328in}}%
\pgfpathlineto{\pgfqpoint{2.323878in}{1.530328in}}%
\pgfpathlineto{\pgfqpoint{2.338010in}{1.530328in}}%
\pgfpathlineto{\pgfqpoint{2.352143in}{1.530328in}}%
\pgfpathlineto{\pgfqpoint{2.366276in}{1.530328in}}%
\pgfpathlineto{\pgfqpoint{2.380408in}{1.530328in}}%
\pgfpathlineto{\pgfqpoint{2.394541in}{1.530328in}}%
\pgfpathlineto{\pgfqpoint{2.408673in}{1.530328in}}%
\pgfpathlineto{\pgfqpoint{2.422806in}{1.530328in}}%
\pgfpathlineto{\pgfqpoint{2.436939in}{1.530328in}}%
\pgfpathlineto{\pgfqpoint{2.451071in}{1.530328in}}%
\pgfpathlineto{\pgfqpoint{2.465204in}{1.530328in}}%
\pgfpathlineto{\pgfqpoint{2.479336in}{1.530328in}}%
\pgfpathlineto{\pgfqpoint{2.493469in}{1.530328in}}%
\pgfpathlineto{\pgfqpoint{2.507602in}{1.530328in}}%
\pgfpathlineto{\pgfqpoint{2.521734in}{1.530328in}}%
\pgfpathlineto{\pgfqpoint{2.535867in}{1.530328in}}%
\pgfpathlineto{\pgfqpoint{2.549999in}{1.530328in}}%
\pgfpathlineto{\pgfqpoint{2.564132in}{1.530328in}}%
\pgfpathlineto{\pgfqpoint{2.578265in}{1.530328in}}%
\pgfpathlineto{\pgfqpoint{2.592397in}{1.530328in}}%
\pgfpathlineto{\pgfqpoint{2.606530in}{1.530328in}}%
\pgfpathlineto{\pgfqpoint{2.620662in}{1.530328in}}%
\pgfpathlineto{\pgfqpoint{2.634795in}{1.530328in}}%
\pgfpathlineto{\pgfqpoint{2.648928in}{1.530328in}}%
\pgfpathlineto{\pgfqpoint{2.663060in}{1.511018in}}%
\pgfpathlineto{\pgfqpoint{2.677193in}{1.511018in}}%
\pgfpathlineto{\pgfqpoint{2.691325in}{1.511018in}}%
\pgfpathlineto{\pgfqpoint{2.705458in}{1.511018in}}%
\pgfpathlineto{\pgfqpoint{2.719591in}{1.511018in}}%
\pgfpathlineto{\pgfqpoint{2.733723in}{1.511018in}}%
\pgfpathlineto{\pgfqpoint{2.747856in}{1.511018in}}%
\pgfpathlineto{\pgfqpoint{2.761988in}{1.511018in}}%
\pgfpathlineto{\pgfqpoint{2.776121in}{1.511018in}}%
\pgfpathlineto{\pgfqpoint{2.790254in}{1.511018in}}%
\pgfpathlineto{\pgfqpoint{2.804386in}{1.511018in}}%
\pgfpathlineto{\pgfqpoint{2.818519in}{1.511018in}}%
\pgfpathlineto{\pgfqpoint{2.832651in}{1.511018in}}%
\pgfpathlineto{\pgfqpoint{2.846784in}{1.511018in}}%
\pgfpathlineto{\pgfqpoint{2.860917in}{1.511018in}}%
\pgfpathlineto{\pgfqpoint{2.875049in}{1.511018in}}%
\pgfpathlineto{\pgfqpoint{2.889182in}{1.511018in}}%
\pgfpathlineto{\pgfqpoint{2.903314in}{1.511018in}}%
\pgfpathlineto{\pgfqpoint{2.917447in}{1.511018in}}%
\pgfpathlineto{\pgfqpoint{2.931580in}{1.511018in}}%
\pgfpathlineto{\pgfqpoint{2.931580in}{1.433823in}}%
\pgfpathlineto{\pgfqpoint{2.931580in}{1.433823in}}%
\pgfpathlineto{\pgfqpoint{2.917447in}{1.433823in}}%
\pgfpathlineto{\pgfqpoint{2.903314in}{1.433823in}}%
\pgfpathlineto{\pgfqpoint{2.889182in}{1.433823in}}%
\pgfpathlineto{\pgfqpoint{2.875049in}{1.433823in}}%
\pgfpathlineto{\pgfqpoint{2.860917in}{1.433823in}}%
\pgfpathlineto{\pgfqpoint{2.846784in}{1.433823in}}%
\pgfpathlineto{\pgfqpoint{2.832651in}{1.433823in}}%
\pgfpathlineto{\pgfqpoint{2.818519in}{1.433823in}}%
\pgfpathlineto{\pgfqpoint{2.804386in}{1.433823in}}%
\pgfpathlineto{\pgfqpoint{2.790254in}{1.433823in}}%
\pgfpathlineto{\pgfqpoint{2.776121in}{1.433823in}}%
\pgfpathlineto{\pgfqpoint{2.761988in}{1.433823in}}%
\pgfpathlineto{\pgfqpoint{2.747856in}{1.433823in}}%
\pgfpathlineto{\pgfqpoint{2.733723in}{1.433823in}}%
\pgfpathlineto{\pgfqpoint{2.719591in}{1.433823in}}%
\pgfpathlineto{\pgfqpoint{2.705458in}{1.433823in}}%
\pgfpathlineto{\pgfqpoint{2.691325in}{1.433823in}}%
\pgfpathlineto{\pgfqpoint{2.677193in}{1.433823in}}%
\pgfpathlineto{\pgfqpoint{2.663060in}{1.433823in}}%
\pgfpathlineto{\pgfqpoint{2.648928in}{1.466376in}}%
\pgfpathlineto{\pgfqpoint{2.634795in}{1.466376in}}%
\pgfpathlineto{\pgfqpoint{2.620662in}{1.466376in}}%
\pgfpathlineto{\pgfqpoint{2.606530in}{1.466376in}}%
\pgfpathlineto{\pgfqpoint{2.592397in}{1.466376in}}%
\pgfpathlineto{\pgfqpoint{2.578265in}{1.466376in}}%
\pgfpathlineto{\pgfqpoint{2.564132in}{1.466376in}}%
\pgfpathlineto{\pgfqpoint{2.549999in}{1.466376in}}%
\pgfpathlineto{\pgfqpoint{2.535867in}{1.466376in}}%
\pgfpathlineto{\pgfqpoint{2.521734in}{1.466376in}}%
\pgfpathlineto{\pgfqpoint{2.507602in}{1.466376in}}%
\pgfpathlineto{\pgfqpoint{2.493469in}{1.466376in}}%
\pgfpathlineto{\pgfqpoint{2.479336in}{1.466376in}}%
\pgfpathlineto{\pgfqpoint{2.465204in}{1.466376in}}%
\pgfpathlineto{\pgfqpoint{2.451071in}{1.466376in}}%
\pgfpathlineto{\pgfqpoint{2.436939in}{1.466376in}}%
\pgfpathlineto{\pgfqpoint{2.422806in}{1.466376in}}%
\pgfpathlineto{\pgfqpoint{2.408673in}{1.466376in}}%
\pgfpathlineto{\pgfqpoint{2.394541in}{1.466376in}}%
\pgfpathlineto{\pgfqpoint{2.380408in}{1.466376in}}%
\pgfpathlineto{\pgfqpoint{2.366276in}{1.466376in}}%
\pgfpathlineto{\pgfqpoint{2.352143in}{1.466376in}}%
\pgfpathlineto{\pgfqpoint{2.338010in}{1.466376in}}%
\pgfpathlineto{\pgfqpoint{2.323878in}{1.466376in}}%
\pgfpathlineto{\pgfqpoint{2.309745in}{1.466376in}}%
\pgfpathlineto{\pgfqpoint{2.295613in}{1.466376in}}%
\pgfpathlineto{\pgfqpoint{2.281480in}{1.466376in}}%
\pgfpathlineto{\pgfqpoint{2.267347in}{1.466376in}}%
\pgfpathlineto{\pgfqpoint{2.253215in}{1.466376in}}%
\pgfpathlineto{\pgfqpoint{2.239082in}{1.466376in}}%
\pgfpathlineto{\pgfqpoint{2.224950in}{1.466376in}}%
\pgfpathlineto{\pgfqpoint{2.210817in}{1.466376in}}%
\pgfpathlineto{\pgfqpoint{2.196684in}{1.466376in}}%
\pgfpathlineto{\pgfqpoint{2.182552in}{1.466376in}}%
\pgfpathlineto{\pgfqpoint{2.168419in}{1.466376in}}%
\pgfpathlineto{\pgfqpoint{2.154287in}{1.466376in}}%
\pgfpathlineto{\pgfqpoint{2.140154in}{1.466376in}}%
\pgfpathlineto{\pgfqpoint{2.126021in}{1.466376in}}%
\pgfpathlineto{\pgfqpoint{2.111889in}{1.466376in}}%
\pgfpathlineto{\pgfqpoint{2.097756in}{1.466376in}}%
\pgfpathlineto{\pgfqpoint{2.083624in}{1.466376in}}%
\pgfpathlineto{\pgfqpoint{2.069491in}{1.466376in}}%
\pgfpathlineto{\pgfqpoint{2.055358in}{1.466376in}}%
\pgfpathlineto{\pgfqpoint{2.041226in}{1.466376in}}%
\pgfpathlineto{\pgfqpoint{2.027093in}{1.466376in}}%
\pgfpathlineto{\pgfqpoint{2.012961in}{1.507878in}}%
\pgfpathlineto{\pgfqpoint{1.998828in}{1.507878in}}%
\pgfpathlineto{\pgfqpoint{1.984695in}{1.507878in}}%
\pgfpathlineto{\pgfqpoint{1.970563in}{1.507878in}}%
\pgfpathlineto{\pgfqpoint{1.956430in}{1.507878in}}%
\pgfpathlineto{\pgfqpoint{1.942298in}{1.507878in}}%
\pgfpathlineto{\pgfqpoint{1.928165in}{1.507878in}}%
\pgfpathlineto{\pgfqpoint{1.914032in}{1.507878in}}%
\pgfpathlineto{\pgfqpoint{1.899900in}{1.507878in}}%
\pgfpathlineto{\pgfqpoint{1.885767in}{1.507878in}}%
\pgfpathlineto{\pgfqpoint{1.871635in}{1.507878in}}%
\pgfpathlineto{\pgfqpoint{1.857502in}{1.507878in}}%
\pgfpathlineto{\pgfqpoint{1.843369in}{1.507878in}}%
\pgfpathlineto{\pgfqpoint{1.829237in}{1.507878in}}%
\pgfpathlineto{\pgfqpoint{1.815104in}{1.507878in}}%
\pgfpathlineto{\pgfqpoint{1.800972in}{1.507878in}}%
\pgfpathlineto{\pgfqpoint{1.786839in}{1.507878in}}%
\pgfpathlineto{\pgfqpoint{1.772706in}{1.507878in}}%
\pgfpathlineto{\pgfqpoint{1.758574in}{1.523135in}}%
\pgfpathlineto{\pgfqpoint{1.744441in}{1.523135in}}%
\pgfpathlineto{\pgfqpoint{1.730309in}{1.523135in}}%
\pgfpathlineto{\pgfqpoint{1.716176in}{1.523135in}}%
\pgfpathlineto{\pgfqpoint{1.702044in}{1.523135in}}%
\pgfpathlineto{\pgfqpoint{1.687911in}{1.523135in}}%
\pgfpathlineto{\pgfqpoint{1.673778in}{1.523135in}}%
\pgfpathlineto{\pgfqpoint{1.659646in}{1.523135in}}%
\pgfpathlineto{\pgfqpoint{1.645513in}{1.523135in}}%
\pgfpathlineto{\pgfqpoint{1.631381in}{1.523135in}}%
\pgfpathlineto{\pgfqpoint{1.617248in}{1.523135in}}%
\pgfpathlineto{\pgfqpoint{1.603115in}{1.523135in}}%
\pgfpathlineto{\pgfqpoint{1.588983in}{1.523135in}}%
\pgfpathlineto{\pgfqpoint{1.574850in}{1.523135in}}%
\pgfpathlineto{\pgfqpoint{1.560718in}{1.523135in}}%
\pgfpathlineto{\pgfqpoint{1.546585in}{1.523135in}}%
\pgfpathlineto{\pgfqpoint{1.532452in}{1.523135in}}%
\pgfpathlineto{\pgfqpoint{1.518320in}{1.523135in}}%
\pgfpathlineto{\pgfqpoint{1.504187in}{1.546257in}}%
\pgfpathlineto{\pgfqpoint{1.490055in}{1.546257in}}%
\pgfpathlineto{\pgfqpoint{1.475922in}{1.546257in}}%
\pgfpathlineto{\pgfqpoint{1.461789in}{1.546257in}}%
\pgfpathlineto{\pgfqpoint{1.447657in}{1.546257in}}%
\pgfpathlineto{\pgfqpoint{1.433524in}{1.546257in}}%
\pgfpathlineto{\pgfqpoint{1.419392in}{1.546257in}}%
\pgfpathlineto{\pgfqpoint{1.405259in}{1.546257in}}%
\pgfpathlineto{\pgfqpoint{1.391126in}{1.546257in}}%
\pgfpathlineto{\pgfqpoint{1.376994in}{1.546257in}}%
\pgfpathlineto{\pgfqpoint{1.362861in}{1.579631in}}%
\pgfpathlineto{\pgfqpoint{1.348729in}{1.579631in}}%
\pgfpathlineto{\pgfqpoint{1.334596in}{1.579631in}}%
\pgfpathlineto{\pgfqpoint{1.320463in}{1.579631in}}%
\pgfpathlineto{\pgfqpoint{1.306331in}{1.579631in}}%
\pgfpathlineto{\pgfqpoint{1.292198in}{1.579631in}}%
\pgfpathlineto{\pgfqpoint{1.278066in}{1.579631in}}%
\pgfpathlineto{\pgfqpoint{1.263933in}{1.579631in}}%
\pgfpathlineto{\pgfqpoint{1.249800in}{1.579631in}}%
\pgfpathlineto{\pgfqpoint{1.235668in}{1.579631in}}%
\pgfpathlineto{\pgfqpoint{1.221535in}{1.579631in}}%
\pgfpathlineto{\pgfqpoint{1.207403in}{1.579631in}}%
\pgfpathlineto{\pgfqpoint{1.193270in}{1.579631in}}%
\pgfpathlineto{\pgfqpoint{1.179137in}{1.670647in}}%
\pgfpathlineto{\pgfqpoint{1.165005in}{1.670647in}}%
\pgfpathlineto{\pgfqpoint{1.150872in}{1.689122in}}%
\pgfpathlineto{\pgfqpoint{1.136740in}{1.732635in}}%
\pgfpathlineto{\pgfqpoint{1.122607in}{1.732635in}}%
\pgfpathlineto{\pgfqpoint{1.108474in}{1.749164in}}%
\pgfpathlineto{\pgfqpoint{1.094342in}{1.806954in}}%
\pgfpathlineto{\pgfqpoint{1.080209in}{1.806954in}}%
\pgfpathlineto{\pgfqpoint{1.066077in}{1.863588in}}%
\pgfpathlineto{\pgfqpoint{1.051944in}{1.863588in}}%
\pgfpathlineto{\pgfqpoint{1.037811in}{1.863588in}}%
\pgfpathlineto{\pgfqpoint{1.023679in}{1.879972in}}%
\pgfpathlineto{\pgfqpoint{1.009546in}{1.879972in}}%
\pgfpathlineto{\pgfqpoint{0.995414in}{1.887738in}}%
\pgfpathlineto{\pgfqpoint{0.981281in}{1.887738in}}%
\pgfpathlineto{\pgfqpoint{0.967148in}{1.888399in}}%
\pgfpathlineto{\pgfqpoint{0.953016in}{1.924567in}}%
\pgfpathlineto{\pgfqpoint{0.938883in}{1.924567in}}%
\pgfpathlineto{\pgfqpoint{0.924751in}{1.978606in}}%
\pgfpathlineto{\pgfqpoint{0.910618in}{2.019017in}}%
\pgfpathlineto{\pgfqpoint{0.896485in}{2.027531in}}%
\pgfpathlineto{\pgfqpoint{0.882353in}{2.042634in}}%
\pgfpathlineto{\pgfqpoint{0.868220in}{2.045616in}}%
\pgfpathlineto{\pgfqpoint{0.854088in}{2.133672in}}%
\pgfpathlineto{\pgfqpoint{0.839955in}{2.133672in}}%
\pgfpathlineto{\pgfqpoint{0.825822in}{2.147250in}}%
\pgfpathlineto{\pgfqpoint{0.811690in}{2.189969in}}%
\pgfpathlineto{\pgfqpoint{0.797557in}{2.189969in}}%
\pgfpathlineto{\pgfqpoint{0.783425in}{2.189969in}}%
\pgfpathlineto{\pgfqpoint{0.769292in}{2.189969in}}%
\pgfpathlineto{\pgfqpoint{0.755159in}{2.189969in}}%
\pgfpathlineto{\pgfqpoint{0.741027in}{2.189969in}}%
\pgfpathlineto{\pgfqpoint{0.726894in}{2.202059in}}%
\pgfpathlineto{\pgfqpoint{0.712762in}{2.202059in}}%
\pgfpathlineto{\pgfqpoint{0.698629in}{2.202059in}}%
\pgfpathlineto{\pgfqpoint{0.684496in}{2.202059in}}%
\pgfpathlineto{\pgfqpoint{0.670364in}{2.202059in}}%
\pgfpathlineto{\pgfqpoint{0.656231in}{2.229526in}}%
\pgfpathlineto{\pgfqpoint{0.642099in}{2.229526in}}%
\pgfpathlineto{\pgfqpoint{0.627966in}{2.229526in}}%
\pgfpathlineto{\pgfqpoint{0.613833in}{2.255091in}}%
\pgfpathlineto{\pgfqpoint{0.599701in}{2.278181in}}%
\pgfpathlineto{\pgfqpoint{0.585568in}{2.278181in}}%
\pgfpathlineto{\pgfqpoint{0.571436in}{2.291032in}}%
\pgfpathlineto{\pgfqpoint{0.557303in}{2.352998in}}%
\pgfpathlineto{\pgfqpoint{0.543170in}{2.405828in}}%
\pgfpathclose%
\pgfusepath{fill}%
\end{pgfscope}%
\begin{pgfscope}%
\pgfpathrectangle{\pgfqpoint{0.423750in}{0.375000in}}{\pgfqpoint{2.627250in}{2.265000in}}%
\pgfusepath{clip}%
\pgfsetbuttcap%
\pgfsetroundjoin%
\definecolor{currentfill}{rgb}{0.839216,0.152941,0.156863}%
\pgfsetfillcolor{currentfill}%
\pgfsetfillopacity{0.200000}%
\pgfsetlinewidth{0.000000pt}%
\definecolor{currentstroke}{rgb}{0.000000,0.000000,0.000000}%
\pgfsetstrokecolor{currentstroke}%
\pgfsetdash{}{0pt}%
\pgfpathmoveto{\pgfqpoint{0.543170in}{2.466371in}}%
\pgfpathlineto{\pgfqpoint{0.543170in}{2.515587in}}%
\pgfpathlineto{\pgfqpoint{0.557303in}{2.463797in}}%
\pgfpathlineto{\pgfqpoint{0.571436in}{2.388973in}}%
\pgfpathlineto{\pgfqpoint{0.585568in}{2.378665in}}%
\pgfpathlineto{\pgfqpoint{0.599701in}{2.377157in}}%
\pgfpathlineto{\pgfqpoint{0.613833in}{2.309127in}}%
\pgfpathlineto{\pgfqpoint{0.627966in}{2.299719in}}%
\pgfpathlineto{\pgfqpoint{0.642099in}{2.295599in}}%
\pgfpathlineto{\pgfqpoint{0.656231in}{2.295599in}}%
\pgfpathlineto{\pgfqpoint{0.670364in}{2.289762in}}%
\pgfpathlineto{\pgfqpoint{0.684496in}{2.289762in}}%
\pgfpathlineto{\pgfqpoint{0.698629in}{2.289762in}}%
\pgfpathlineto{\pgfqpoint{0.712762in}{2.289762in}}%
\pgfpathlineto{\pgfqpoint{0.726894in}{2.286262in}}%
\pgfpathlineto{\pgfqpoint{0.741027in}{2.286262in}}%
\pgfpathlineto{\pgfqpoint{0.755159in}{2.286262in}}%
\pgfpathlineto{\pgfqpoint{0.769292in}{2.274734in}}%
\pgfpathlineto{\pgfqpoint{0.783425in}{2.257592in}}%
\pgfpathlineto{\pgfqpoint{0.797557in}{2.200633in}}%
\pgfpathlineto{\pgfqpoint{0.811690in}{2.200633in}}%
\pgfpathlineto{\pgfqpoint{0.825822in}{2.031652in}}%
\pgfpathlineto{\pgfqpoint{0.839955in}{2.031652in}}%
\pgfpathlineto{\pgfqpoint{0.854088in}{2.031652in}}%
\pgfpathlineto{\pgfqpoint{0.868220in}{2.031652in}}%
\pgfpathlineto{\pgfqpoint{0.882353in}{2.031652in}}%
\pgfpathlineto{\pgfqpoint{0.896485in}{2.031652in}}%
\pgfpathlineto{\pgfqpoint{0.910618in}{2.031652in}}%
\pgfpathlineto{\pgfqpoint{0.924751in}{2.031652in}}%
\pgfpathlineto{\pgfqpoint{0.938883in}{2.031652in}}%
\pgfpathlineto{\pgfqpoint{0.953016in}{2.031652in}}%
\pgfpathlineto{\pgfqpoint{0.967148in}{2.023525in}}%
\pgfpathlineto{\pgfqpoint{0.981281in}{1.991673in}}%
\pgfpathlineto{\pgfqpoint{0.995414in}{1.991673in}}%
\pgfpathlineto{\pgfqpoint{1.009546in}{1.989673in}}%
\pgfpathlineto{\pgfqpoint{1.023679in}{1.989673in}}%
\pgfpathlineto{\pgfqpoint{1.037811in}{1.978210in}}%
\pgfpathlineto{\pgfqpoint{1.051944in}{1.978210in}}%
\pgfpathlineto{\pgfqpoint{1.066077in}{1.973552in}}%
\pgfpathlineto{\pgfqpoint{1.080209in}{1.973552in}}%
\pgfpathlineto{\pgfqpoint{1.094342in}{1.965325in}}%
\pgfpathlineto{\pgfqpoint{1.108474in}{1.883619in}}%
\pgfpathlineto{\pgfqpoint{1.122607in}{1.883619in}}%
\pgfpathlineto{\pgfqpoint{1.136740in}{1.810710in}}%
\pgfpathlineto{\pgfqpoint{1.150872in}{1.781665in}}%
\pgfpathlineto{\pgfqpoint{1.165005in}{1.781665in}}%
\pgfpathlineto{\pgfqpoint{1.179137in}{1.781665in}}%
\pgfpathlineto{\pgfqpoint{1.193270in}{1.781223in}}%
\pgfpathlineto{\pgfqpoint{1.207403in}{1.781223in}}%
\pgfpathlineto{\pgfqpoint{1.221535in}{1.781223in}}%
\pgfpathlineto{\pgfqpoint{1.235668in}{1.781223in}}%
\pgfpathlineto{\pgfqpoint{1.249800in}{1.781223in}}%
\pgfpathlineto{\pgfqpoint{1.263933in}{1.781223in}}%
\pgfpathlineto{\pgfqpoint{1.278066in}{1.770418in}}%
\pgfpathlineto{\pgfqpoint{1.292198in}{1.770418in}}%
\pgfpathlineto{\pgfqpoint{1.306331in}{1.770418in}}%
\pgfpathlineto{\pgfqpoint{1.320463in}{1.770418in}}%
\pgfpathlineto{\pgfqpoint{1.334596in}{1.770418in}}%
\pgfpathlineto{\pgfqpoint{1.348729in}{1.770418in}}%
\pgfpathlineto{\pgfqpoint{1.362861in}{1.770418in}}%
\pgfpathlineto{\pgfqpoint{1.376994in}{1.770418in}}%
\pgfpathlineto{\pgfqpoint{1.391126in}{1.770418in}}%
\pgfpathlineto{\pgfqpoint{1.405259in}{1.770418in}}%
\pgfpathlineto{\pgfqpoint{1.419392in}{1.770418in}}%
\pgfpathlineto{\pgfqpoint{1.433524in}{1.758863in}}%
\pgfpathlineto{\pgfqpoint{1.447657in}{1.730024in}}%
\pgfpathlineto{\pgfqpoint{1.461789in}{1.730024in}}%
\pgfpathlineto{\pgfqpoint{1.475922in}{1.730024in}}%
\pgfpathlineto{\pgfqpoint{1.490055in}{1.730024in}}%
\pgfpathlineto{\pgfqpoint{1.504187in}{1.730024in}}%
\pgfpathlineto{\pgfqpoint{1.518320in}{1.730024in}}%
\pgfpathlineto{\pgfqpoint{1.532452in}{1.730024in}}%
\pgfpathlineto{\pgfqpoint{1.546585in}{1.730024in}}%
\pgfpathlineto{\pgfqpoint{1.560718in}{1.730024in}}%
\pgfpathlineto{\pgfqpoint{1.574850in}{1.730024in}}%
\pgfpathlineto{\pgfqpoint{1.588983in}{1.730024in}}%
\pgfpathlineto{\pgfqpoint{1.603115in}{1.730024in}}%
\pgfpathlineto{\pgfqpoint{1.617248in}{1.693540in}}%
\pgfpathlineto{\pgfqpoint{1.631381in}{1.693540in}}%
\pgfpathlineto{\pgfqpoint{1.645513in}{1.693540in}}%
\pgfpathlineto{\pgfqpoint{1.659646in}{1.693540in}}%
\pgfpathlineto{\pgfqpoint{1.673778in}{1.693540in}}%
\pgfpathlineto{\pgfqpoint{1.687911in}{1.693540in}}%
\pgfpathlineto{\pgfqpoint{1.702044in}{1.693540in}}%
\pgfpathlineto{\pgfqpoint{1.716176in}{1.693540in}}%
\pgfpathlineto{\pgfqpoint{1.730309in}{1.693540in}}%
\pgfpathlineto{\pgfqpoint{1.744441in}{1.693540in}}%
\pgfpathlineto{\pgfqpoint{1.758574in}{1.693540in}}%
\pgfpathlineto{\pgfqpoint{1.772706in}{1.693540in}}%
\pgfpathlineto{\pgfqpoint{1.786839in}{1.693540in}}%
\pgfpathlineto{\pgfqpoint{1.800972in}{1.693540in}}%
\pgfpathlineto{\pgfqpoint{1.815104in}{1.693540in}}%
\pgfpathlineto{\pgfqpoint{1.829237in}{1.693540in}}%
\pgfpathlineto{\pgfqpoint{1.843369in}{1.693540in}}%
\pgfpathlineto{\pgfqpoint{1.857502in}{1.693540in}}%
\pgfpathlineto{\pgfqpoint{1.871635in}{1.693540in}}%
\pgfpathlineto{\pgfqpoint{1.885767in}{1.693540in}}%
\pgfpathlineto{\pgfqpoint{1.899900in}{1.693540in}}%
\pgfpathlineto{\pgfqpoint{1.914032in}{1.667092in}}%
\pgfpathlineto{\pgfqpoint{1.928165in}{1.667092in}}%
\pgfpathlineto{\pgfqpoint{1.942298in}{1.667092in}}%
\pgfpathlineto{\pgfqpoint{1.956430in}{1.667092in}}%
\pgfpathlineto{\pgfqpoint{1.970563in}{1.667092in}}%
\pgfpathlineto{\pgfqpoint{1.984695in}{1.667092in}}%
\pgfpathlineto{\pgfqpoint{1.998828in}{1.667092in}}%
\pgfpathlineto{\pgfqpoint{2.012961in}{1.667092in}}%
\pgfpathlineto{\pgfqpoint{2.027093in}{1.667092in}}%
\pgfpathlineto{\pgfqpoint{2.041226in}{1.648882in}}%
\pgfpathlineto{\pgfqpoint{2.055358in}{1.594556in}}%
\pgfpathlineto{\pgfqpoint{2.069491in}{1.594556in}}%
\pgfpathlineto{\pgfqpoint{2.083624in}{1.594556in}}%
\pgfpathlineto{\pgfqpoint{2.097756in}{1.594556in}}%
\pgfpathlineto{\pgfqpoint{2.111889in}{1.594556in}}%
\pgfpathlineto{\pgfqpoint{2.126021in}{1.594556in}}%
\pgfpathlineto{\pgfqpoint{2.140154in}{1.594556in}}%
\pgfpathlineto{\pgfqpoint{2.154287in}{1.594556in}}%
\pgfpathlineto{\pgfqpoint{2.168419in}{1.594556in}}%
\pgfpathlineto{\pgfqpoint{2.182552in}{1.594556in}}%
\pgfpathlineto{\pgfqpoint{2.196684in}{1.594556in}}%
\pgfpathlineto{\pgfqpoint{2.210817in}{1.594556in}}%
\pgfpathlineto{\pgfqpoint{2.224950in}{1.594556in}}%
\pgfpathlineto{\pgfqpoint{2.239082in}{1.594556in}}%
\pgfpathlineto{\pgfqpoint{2.253215in}{1.594556in}}%
\pgfpathlineto{\pgfqpoint{2.267347in}{1.594556in}}%
\pgfpathlineto{\pgfqpoint{2.281480in}{1.594556in}}%
\pgfpathlineto{\pgfqpoint{2.295613in}{1.594556in}}%
\pgfpathlineto{\pgfqpoint{2.309745in}{1.594556in}}%
\pgfpathlineto{\pgfqpoint{2.323878in}{1.594556in}}%
\pgfpathlineto{\pgfqpoint{2.338010in}{1.552752in}}%
\pgfpathlineto{\pgfqpoint{2.352143in}{1.552752in}}%
\pgfpathlineto{\pgfqpoint{2.366276in}{1.552752in}}%
\pgfpathlineto{\pgfqpoint{2.380408in}{1.552752in}}%
\pgfpathlineto{\pgfqpoint{2.394541in}{1.552752in}}%
\pgfpathlineto{\pgfqpoint{2.408673in}{1.552752in}}%
\pgfpathlineto{\pgfqpoint{2.422806in}{1.552752in}}%
\pgfpathlineto{\pgfqpoint{2.436939in}{1.552752in}}%
\pgfpathlineto{\pgfqpoint{2.451071in}{1.552752in}}%
\pgfpathlineto{\pgfqpoint{2.465204in}{1.552752in}}%
\pgfpathlineto{\pgfqpoint{2.479336in}{1.552752in}}%
\pgfpathlineto{\pgfqpoint{2.493469in}{1.552752in}}%
\pgfpathlineto{\pgfqpoint{2.507602in}{1.552752in}}%
\pgfpathlineto{\pgfqpoint{2.521734in}{1.552752in}}%
\pgfpathlineto{\pgfqpoint{2.535867in}{1.552752in}}%
\pgfpathlineto{\pgfqpoint{2.549999in}{1.552752in}}%
\pgfpathlineto{\pgfqpoint{2.564132in}{1.552752in}}%
\pgfpathlineto{\pgfqpoint{2.578265in}{1.552752in}}%
\pgfpathlineto{\pgfqpoint{2.592397in}{1.552752in}}%
\pgfpathlineto{\pgfqpoint{2.606530in}{1.552752in}}%
\pgfpathlineto{\pgfqpoint{2.620662in}{1.552752in}}%
\pgfpathlineto{\pgfqpoint{2.634795in}{1.535510in}}%
\pgfpathlineto{\pgfqpoint{2.648928in}{1.535510in}}%
\pgfpathlineto{\pgfqpoint{2.663060in}{1.535510in}}%
\pgfpathlineto{\pgfqpoint{2.677193in}{1.535510in}}%
\pgfpathlineto{\pgfqpoint{2.691325in}{1.535510in}}%
\pgfpathlineto{\pgfqpoint{2.705458in}{1.535510in}}%
\pgfpathlineto{\pgfqpoint{2.719591in}{1.535510in}}%
\pgfpathlineto{\pgfqpoint{2.733723in}{1.535510in}}%
\pgfpathlineto{\pgfqpoint{2.747856in}{1.535510in}}%
\pgfpathlineto{\pgfqpoint{2.761988in}{1.535510in}}%
\pgfpathlineto{\pgfqpoint{2.776121in}{1.535510in}}%
\pgfpathlineto{\pgfqpoint{2.790254in}{1.535510in}}%
\pgfpathlineto{\pgfqpoint{2.804386in}{1.535510in}}%
\pgfpathlineto{\pgfqpoint{2.818519in}{1.535510in}}%
\pgfpathlineto{\pgfqpoint{2.832651in}{1.522856in}}%
\pgfpathlineto{\pgfqpoint{2.846784in}{1.522856in}}%
\pgfpathlineto{\pgfqpoint{2.860917in}{1.522856in}}%
\pgfpathlineto{\pgfqpoint{2.875049in}{1.522856in}}%
\pgfpathlineto{\pgfqpoint{2.889182in}{1.522856in}}%
\pgfpathlineto{\pgfqpoint{2.903314in}{1.522856in}}%
\pgfpathlineto{\pgfqpoint{2.917447in}{1.522856in}}%
\pgfpathlineto{\pgfqpoint{2.931580in}{1.522856in}}%
\pgfpathlineto{\pgfqpoint{2.931580in}{1.454642in}}%
\pgfpathlineto{\pgfqpoint{2.931580in}{1.454642in}}%
\pgfpathlineto{\pgfqpoint{2.917447in}{1.454642in}}%
\pgfpathlineto{\pgfqpoint{2.903314in}{1.454642in}}%
\pgfpathlineto{\pgfqpoint{2.889182in}{1.454642in}}%
\pgfpathlineto{\pgfqpoint{2.875049in}{1.454642in}}%
\pgfpathlineto{\pgfqpoint{2.860917in}{1.454642in}}%
\pgfpathlineto{\pgfqpoint{2.846784in}{1.454642in}}%
\pgfpathlineto{\pgfqpoint{2.832651in}{1.454642in}}%
\pgfpathlineto{\pgfqpoint{2.818519in}{1.456646in}}%
\pgfpathlineto{\pgfqpoint{2.804386in}{1.456646in}}%
\pgfpathlineto{\pgfqpoint{2.790254in}{1.456646in}}%
\pgfpathlineto{\pgfqpoint{2.776121in}{1.456646in}}%
\pgfpathlineto{\pgfqpoint{2.761988in}{1.456646in}}%
\pgfpathlineto{\pgfqpoint{2.747856in}{1.456646in}}%
\pgfpathlineto{\pgfqpoint{2.733723in}{1.456646in}}%
\pgfpathlineto{\pgfqpoint{2.719591in}{1.456646in}}%
\pgfpathlineto{\pgfqpoint{2.705458in}{1.456646in}}%
\pgfpathlineto{\pgfqpoint{2.691325in}{1.456646in}}%
\pgfpathlineto{\pgfqpoint{2.677193in}{1.456646in}}%
\pgfpathlineto{\pgfqpoint{2.663060in}{1.456646in}}%
\pgfpathlineto{\pgfqpoint{2.648928in}{1.456646in}}%
\pgfpathlineto{\pgfqpoint{2.634795in}{1.456646in}}%
\pgfpathlineto{\pgfqpoint{2.620662in}{1.459268in}}%
\pgfpathlineto{\pgfqpoint{2.606530in}{1.459268in}}%
\pgfpathlineto{\pgfqpoint{2.592397in}{1.459268in}}%
\pgfpathlineto{\pgfqpoint{2.578265in}{1.459268in}}%
\pgfpathlineto{\pgfqpoint{2.564132in}{1.459268in}}%
\pgfpathlineto{\pgfqpoint{2.549999in}{1.459268in}}%
\pgfpathlineto{\pgfqpoint{2.535867in}{1.459268in}}%
\pgfpathlineto{\pgfqpoint{2.521734in}{1.459268in}}%
\pgfpathlineto{\pgfqpoint{2.507602in}{1.459268in}}%
\pgfpathlineto{\pgfqpoint{2.493469in}{1.459268in}}%
\pgfpathlineto{\pgfqpoint{2.479336in}{1.459268in}}%
\pgfpathlineto{\pgfqpoint{2.465204in}{1.459268in}}%
\pgfpathlineto{\pgfqpoint{2.451071in}{1.459268in}}%
\pgfpathlineto{\pgfqpoint{2.436939in}{1.459268in}}%
\pgfpathlineto{\pgfqpoint{2.422806in}{1.459268in}}%
\pgfpathlineto{\pgfqpoint{2.408673in}{1.459268in}}%
\pgfpathlineto{\pgfqpoint{2.394541in}{1.459268in}}%
\pgfpathlineto{\pgfqpoint{2.380408in}{1.459268in}}%
\pgfpathlineto{\pgfqpoint{2.366276in}{1.459268in}}%
\pgfpathlineto{\pgfqpoint{2.352143in}{1.459268in}}%
\pgfpathlineto{\pgfqpoint{2.338010in}{1.459268in}}%
\pgfpathlineto{\pgfqpoint{2.323878in}{1.508177in}}%
\pgfpathlineto{\pgfqpoint{2.309745in}{1.508177in}}%
\pgfpathlineto{\pgfqpoint{2.295613in}{1.508177in}}%
\pgfpathlineto{\pgfqpoint{2.281480in}{1.508177in}}%
\pgfpathlineto{\pgfqpoint{2.267347in}{1.508177in}}%
\pgfpathlineto{\pgfqpoint{2.253215in}{1.508177in}}%
\pgfpathlineto{\pgfqpoint{2.239082in}{1.508177in}}%
\pgfpathlineto{\pgfqpoint{2.224950in}{1.508177in}}%
\pgfpathlineto{\pgfqpoint{2.210817in}{1.508177in}}%
\pgfpathlineto{\pgfqpoint{2.196684in}{1.508177in}}%
\pgfpathlineto{\pgfqpoint{2.182552in}{1.508177in}}%
\pgfpathlineto{\pgfqpoint{2.168419in}{1.508177in}}%
\pgfpathlineto{\pgfqpoint{2.154287in}{1.508177in}}%
\pgfpathlineto{\pgfqpoint{2.140154in}{1.508177in}}%
\pgfpathlineto{\pgfqpoint{2.126021in}{1.508177in}}%
\pgfpathlineto{\pgfqpoint{2.111889in}{1.508177in}}%
\pgfpathlineto{\pgfqpoint{2.097756in}{1.508177in}}%
\pgfpathlineto{\pgfqpoint{2.083624in}{1.508177in}}%
\pgfpathlineto{\pgfqpoint{2.069491in}{1.508177in}}%
\pgfpathlineto{\pgfqpoint{2.055358in}{1.508177in}}%
\pgfpathlineto{\pgfqpoint{2.041226in}{1.530880in}}%
\pgfpathlineto{\pgfqpoint{2.027093in}{1.558458in}}%
\pgfpathlineto{\pgfqpoint{2.012961in}{1.558458in}}%
\pgfpathlineto{\pgfqpoint{1.998828in}{1.558458in}}%
\pgfpathlineto{\pgfqpoint{1.984695in}{1.558458in}}%
\pgfpathlineto{\pgfqpoint{1.970563in}{1.558458in}}%
\pgfpathlineto{\pgfqpoint{1.956430in}{1.558458in}}%
\pgfpathlineto{\pgfqpoint{1.942298in}{1.558458in}}%
\pgfpathlineto{\pgfqpoint{1.928165in}{1.558458in}}%
\pgfpathlineto{\pgfqpoint{1.914032in}{1.558458in}}%
\pgfpathlineto{\pgfqpoint{1.899900in}{1.618536in}}%
\pgfpathlineto{\pgfqpoint{1.885767in}{1.618536in}}%
\pgfpathlineto{\pgfqpoint{1.871635in}{1.618536in}}%
\pgfpathlineto{\pgfqpoint{1.857502in}{1.618536in}}%
\pgfpathlineto{\pgfqpoint{1.843369in}{1.618536in}}%
\pgfpathlineto{\pgfqpoint{1.829237in}{1.618536in}}%
\pgfpathlineto{\pgfqpoint{1.815104in}{1.618536in}}%
\pgfpathlineto{\pgfqpoint{1.800972in}{1.618536in}}%
\pgfpathlineto{\pgfqpoint{1.786839in}{1.618536in}}%
\pgfpathlineto{\pgfqpoint{1.772706in}{1.618536in}}%
\pgfpathlineto{\pgfqpoint{1.758574in}{1.618536in}}%
\pgfpathlineto{\pgfqpoint{1.744441in}{1.618536in}}%
\pgfpathlineto{\pgfqpoint{1.730309in}{1.618536in}}%
\pgfpathlineto{\pgfqpoint{1.716176in}{1.618536in}}%
\pgfpathlineto{\pgfqpoint{1.702044in}{1.618536in}}%
\pgfpathlineto{\pgfqpoint{1.687911in}{1.618536in}}%
\pgfpathlineto{\pgfqpoint{1.673778in}{1.618536in}}%
\pgfpathlineto{\pgfqpoint{1.659646in}{1.618536in}}%
\pgfpathlineto{\pgfqpoint{1.645513in}{1.618536in}}%
\pgfpathlineto{\pgfqpoint{1.631381in}{1.618536in}}%
\pgfpathlineto{\pgfqpoint{1.617248in}{1.618536in}}%
\pgfpathlineto{\pgfqpoint{1.603115in}{1.635322in}}%
\pgfpathlineto{\pgfqpoint{1.588983in}{1.635322in}}%
\pgfpathlineto{\pgfqpoint{1.574850in}{1.635322in}}%
\pgfpathlineto{\pgfqpoint{1.560718in}{1.635322in}}%
\pgfpathlineto{\pgfqpoint{1.546585in}{1.635322in}}%
\pgfpathlineto{\pgfqpoint{1.532452in}{1.635322in}}%
\pgfpathlineto{\pgfqpoint{1.518320in}{1.635322in}}%
\pgfpathlineto{\pgfqpoint{1.504187in}{1.635322in}}%
\pgfpathlineto{\pgfqpoint{1.490055in}{1.635322in}}%
\pgfpathlineto{\pgfqpoint{1.475922in}{1.635322in}}%
\pgfpathlineto{\pgfqpoint{1.461789in}{1.635322in}}%
\pgfpathlineto{\pgfqpoint{1.447657in}{1.635322in}}%
\pgfpathlineto{\pgfqpoint{1.433524in}{1.666199in}}%
\pgfpathlineto{\pgfqpoint{1.419392in}{1.674410in}}%
\pgfpathlineto{\pgfqpoint{1.405259in}{1.674410in}}%
\pgfpathlineto{\pgfqpoint{1.391126in}{1.674410in}}%
\pgfpathlineto{\pgfqpoint{1.376994in}{1.674410in}}%
\pgfpathlineto{\pgfqpoint{1.362861in}{1.674410in}}%
\pgfpathlineto{\pgfqpoint{1.348729in}{1.674410in}}%
\pgfpathlineto{\pgfqpoint{1.334596in}{1.674410in}}%
\pgfpathlineto{\pgfqpoint{1.320463in}{1.674410in}}%
\pgfpathlineto{\pgfqpoint{1.306331in}{1.674410in}}%
\pgfpathlineto{\pgfqpoint{1.292198in}{1.674410in}}%
\pgfpathlineto{\pgfqpoint{1.278066in}{1.674410in}}%
\pgfpathlineto{\pgfqpoint{1.263933in}{1.698583in}}%
\pgfpathlineto{\pgfqpoint{1.249800in}{1.698583in}}%
\pgfpathlineto{\pgfqpoint{1.235668in}{1.698583in}}%
\pgfpathlineto{\pgfqpoint{1.221535in}{1.698583in}}%
\pgfpathlineto{\pgfqpoint{1.207403in}{1.698583in}}%
\pgfpathlineto{\pgfqpoint{1.193270in}{1.698583in}}%
\pgfpathlineto{\pgfqpoint{1.179137in}{1.698997in}}%
\pgfpathlineto{\pgfqpoint{1.165005in}{1.698997in}}%
\pgfpathlineto{\pgfqpoint{1.150872in}{1.698997in}}%
\pgfpathlineto{\pgfqpoint{1.136740in}{1.755618in}}%
\pgfpathlineto{\pgfqpoint{1.122607in}{1.784556in}}%
\pgfpathlineto{\pgfqpoint{1.108474in}{1.784556in}}%
\pgfpathlineto{\pgfqpoint{1.094342in}{1.801816in}}%
\pgfpathlineto{\pgfqpoint{1.080209in}{1.803925in}}%
\pgfpathlineto{\pgfqpoint{1.066077in}{1.803925in}}%
\pgfpathlineto{\pgfqpoint{1.051944in}{1.833667in}}%
\pgfpathlineto{\pgfqpoint{1.037811in}{1.833667in}}%
\pgfpathlineto{\pgfqpoint{1.023679in}{1.877658in}}%
\pgfpathlineto{\pgfqpoint{1.009546in}{1.877658in}}%
\pgfpathlineto{\pgfqpoint{0.995414in}{1.891127in}}%
\pgfpathlineto{\pgfqpoint{0.981281in}{1.891127in}}%
\pgfpathlineto{\pgfqpoint{0.967148in}{1.951113in}}%
\pgfpathlineto{\pgfqpoint{0.953016in}{1.982163in}}%
\pgfpathlineto{\pgfqpoint{0.938883in}{1.982163in}}%
\pgfpathlineto{\pgfqpoint{0.924751in}{1.982163in}}%
\pgfpathlineto{\pgfqpoint{0.910618in}{1.982163in}}%
\pgfpathlineto{\pgfqpoint{0.896485in}{1.982163in}}%
\pgfpathlineto{\pgfqpoint{0.882353in}{1.982163in}}%
\pgfpathlineto{\pgfqpoint{0.868220in}{1.982163in}}%
\pgfpathlineto{\pgfqpoint{0.854088in}{1.982163in}}%
\pgfpathlineto{\pgfqpoint{0.839955in}{1.982163in}}%
\pgfpathlineto{\pgfqpoint{0.825822in}{1.982163in}}%
\pgfpathlineto{\pgfqpoint{0.811690in}{2.019007in}}%
\pgfpathlineto{\pgfqpoint{0.797557in}{2.019007in}}%
\pgfpathlineto{\pgfqpoint{0.783425in}{2.141958in}}%
\pgfpathlineto{\pgfqpoint{0.769292in}{2.157090in}}%
\pgfpathlineto{\pgfqpoint{0.755159in}{2.199733in}}%
\pgfpathlineto{\pgfqpoint{0.741027in}{2.199733in}}%
\pgfpathlineto{\pgfqpoint{0.726894in}{2.199733in}}%
\pgfpathlineto{\pgfqpoint{0.712762in}{2.220608in}}%
\pgfpathlineto{\pgfqpoint{0.698629in}{2.220608in}}%
\pgfpathlineto{\pgfqpoint{0.684496in}{2.220608in}}%
\pgfpathlineto{\pgfqpoint{0.670364in}{2.220608in}}%
\pgfpathlineto{\pgfqpoint{0.656231in}{2.224538in}}%
\pgfpathlineto{\pgfqpoint{0.642099in}{2.224538in}}%
\pgfpathlineto{\pgfqpoint{0.627966in}{2.248526in}}%
\pgfpathlineto{\pgfqpoint{0.613833in}{2.267492in}}%
\pgfpathlineto{\pgfqpoint{0.599701in}{2.299598in}}%
\pgfpathlineto{\pgfqpoint{0.585568in}{2.308113in}}%
\pgfpathlineto{\pgfqpoint{0.571436in}{2.335644in}}%
\pgfpathlineto{\pgfqpoint{0.557303in}{2.381043in}}%
\pgfpathlineto{\pgfqpoint{0.543170in}{2.466371in}}%
\pgfpathclose%
\pgfusepath{fill}%
\end{pgfscope}%
\begin{pgfscope}%
\pgfpathrectangle{\pgfqpoint{0.423750in}{0.375000in}}{\pgfqpoint{2.627250in}{2.265000in}}%
\pgfusepath{clip}%
\pgfsetbuttcap%
\pgfsetroundjoin%
\definecolor{currentfill}{rgb}{0.580392,0.403922,0.741176}%
\pgfsetfillcolor{currentfill}%
\pgfsetfillopacity{0.200000}%
\pgfsetlinewidth{0.000000pt}%
\definecolor{currentstroke}{rgb}{0.000000,0.000000,0.000000}%
\pgfsetstrokecolor{currentstroke}%
\pgfsetdash{}{0pt}%
\pgfpathmoveto{\pgfqpoint{0.543170in}{2.461632in}}%
\pgfpathlineto{\pgfqpoint{0.543170in}{2.537045in}}%
\pgfpathlineto{\pgfqpoint{0.557303in}{2.495566in}}%
\pgfpathlineto{\pgfqpoint{0.571436in}{2.444689in}}%
\pgfpathlineto{\pgfqpoint{0.585568in}{2.396421in}}%
\pgfpathlineto{\pgfqpoint{0.599701in}{2.396421in}}%
\pgfpathlineto{\pgfqpoint{0.613833in}{2.383805in}}%
\pgfpathlineto{\pgfqpoint{0.627966in}{2.376013in}}%
\pgfpathlineto{\pgfqpoint{0.642099in}{2.376013in}}%
\pgfpathlineto{\pgfqpoint{0.656231in}{2.225575in}}%
\pgfpathlineto{\pgfqpoint{0.670364in}{2.225575in}}%
\pgfpathlineto{\pgfqpoint{0.684496in}{2.225575in}}%
\pgfpathlineto{\pgfqpoint{0.698629in}{2.225575in}}%
\pgfpathlineto{\pgfqpoint{0.712762in}{2.196488in}}%
\pgfpathlineto{\pgfqpoint{0.726894in}{2.188993in}}%
\pgfpathlineto{\pgfqpoint{0.741027in}{2.186873in}}%
\pgfpathlineto{\pgfqpoint{0.755159in}{2.186873in}}%
\pgfpathlineto{\pgfqpoint{0.769292in}{2.186873in}}%
\pgfpathlineto{\pgfqpoint{0.783425in}{2.186873in}}%
\pgfpathlineto{\pgfqpoint{0.797557in}{2.186873in}}%
\pgfpathlineto{\pgfqpoint{0.811690in}{2.154398in}}%
\pgfpathlineto{\pgfqpoint{0.825822in}{2.154398in}}%
\pgfpathlineto{\pgfqpoint{0.839955in}{2.128108in}}%
\pgfpathlineto{\pgfqpoint{0.854088in}{2.128108in}}%
\pgfpathlineto{\pgfqpoint{0.868220in}{2.128108in}}%
\pgfpathlineto{\pgfqpoint{0.882353in}{2.110347in}}%
\pgfpathlineto{\pgfqpoint{0.896485in}{2.068987in}}%
\pgfpathlineto{\pgfqpoint{0.910618in}{2.066092in}}%
\pgfpathlineto{\pgfqpoint{0.924751in}{2.066089in}}%
\pgfpathlineto{\pgfqpoint{0.938883in}{2.066089in}}%
\pgfpathlineto{\pgfqpoint{0.953016in}{2.066089in}}%
\pgfpathlineto{\pgfqpoint{0.967148in}{2.066089in}}%
\pgfpathlineto{\pgfqpoint{0.981281in}{2.066089in}}%
\pgfpathlineto{\pgfqpoint{0.995414in}{2.066089in}}%
\pgfpathlineto{\pgfqpoint{1.009546in}{2.066078in}}%
\pgfpathlineto{\pgfqpoint{1.023679in}{2.066078in}}%
\pgfpathlineto{\pgfqpoint{1.037811in}{2.065830in}}%
\pgfpathlineto{\pgfqpoint{1.051944in}{2.065830in}}%
\pgfpathlineto{\pgfqpoint{1.066077in}{2.039377in}}%
\pgfpathlineto{\pgfqpoint{1.080209in}{2.039377in}}%
\pgfpathlineto{\pgfqpoint{1.094342in}{2.039377in}}%
\pgfpathlineto{\pgfqpoint{1.108474in}{2.039377in}}%
\pgfpathlineto{\pgfqpoint{1.122607in}{2.039377in}}%
\pgfpathlineto{\pgfqpoint{1.136740in}{1.978744in}}%
\pgfpathlineto{\pgfqpoint{1.150872in}{1.978744in}}%
\pgfpathlineto{\pgfqpoint{1.165005in}{1.978744in}}%
\pgfpathlineto{\pgfqpoint{1.179137in}{1.978744in}}%
\pgfpathlineto{\pgfqpoint{1.193270in}{1.978190in}}%
\pgfpathlineto{\pgfqpoint{1.207403in}{1.959012in}}%
\pgfpathlineto{\pgfqpoint{1.221535in}{1.956988in}}%
\pgfpathlineto{\pgfqpoint{1.235668in}{1.914676in}}%
\pgfpathlineto{\pgfqpoint{1.249800in}{1.911442in}}%
\pgfpathlineto{\pgfqpoint{1.263933in}{1.910339in}}%
\pgfpathlineto{\pgfqpoint{1.278066in}{1.909714in}}%
\pgfpathlineto{\pgfqpoint{1.292198in}{1.909235in}}%
\pgfpathlineto{\pgfqpoint{1.306331in}{1.908820in}}%
\pgfpathlineto{\pgfqpoint{1.320463in}{1.887649in}}%
\pgfpathlineto{\pgfqpoint{1.334596in}{1.887118in}}%
\pgfpathlineto{\pgfqpoint{1.348729in}{1.884801in}}%
\pgfpathlineto{\pgfqpoint{1.362861in}{1.776324in}}%
\pgfpathlineto{\pgfqpoint{1.376994in}{1.771612in}}%
\pgfpathlineto{\pgfqpoint{1.391126in}{1.770215in}}%
\pgfpathlineto{\pgfqpoint{1.405259in}{1.768025in}}%
\pgfpathlineto{\pgfqpoint{1.419392in}{1.767173in}}%
\pgfpathlineto{\pgfqpoint{1.433524in}{1.561954in}}%
\pgfpathlineto{\pgfqpoint{1.447657in}{1.529772in}}%
\pgfpathlineto{\pgfqpoint{1.461789in}{1.521824in}}%
\pgfpathlineto{\pgfqpoint{1.475922in}{1.515177in}}%
\pgfpathlineto{\pgfqpoint{1.490055in}{1.510005in}}%
\pgfpathlineto{\pgfqpoint{1.504187in}{1.502926in}}%
\pgfpathlineto{\pgfqpoint{1.518320in}{1.497472in}}%
\pgfpathlineto{\pgfqpoint{1.532452in}{1.491177in}}%
\pgfpathlineto{\pgfqpoint{1.546585in}{1.483448in}}%
\pgfpathlineto{\pgfqpoint{1.560718in}{1.473897in}}%
\pgfpathlineto{\pgfqpoint{1.574850in}{1.461423in}}%
\pgfpathlineto{\pgfqpoint{1.588983in}{1.445224in}}%
\pgfpathlineto{\pgfqpoint{1.603115in}{1.421393in}}%
\pgfpathlineto{\pgfqpoint{1.617248in}{1.387113in}}%
\pgfpathlineto{\pgfqpoint{1.631381in}{1.387113in}}%
\pgfpathlineto{\pgfqpoint{1.645513in}{1.387019in}}%
\pgfpathlineto{\pgfqpoint{1.659646in}{1.385635in}}%
\pgfpathlineto{\pgfqpoint{1.673778in}{1.382225in}}%
\pgfpathlineto{\pgfqpoint{1.687911in}{1.380707in}}%
\pgfpathlineto{\pgfqpoint{1.702044in}{1.380119in}}%
\pgfpathlineto{\pgfqpoint{1.716176in}{1.380119in}}%
\pgfpathlineto{\pgfqpoint{1.730309in}{1.378974in}}%
\pgfpathlineto{\pgfqpoint{1.744441in}{1.377686in}}%
\pgfpathlineto{\pgfqpoint{1.758574in}{1.307244in}}%
\pgfpathlineto{\pgfqpoint{1.772706in}{0.977323in}}%
\pgfpathlineto{\pgfqpoint{1.786839in}{0.953768in}}%
\pgfpathlineto{\pgfqpoint{1.800972in}{0.902802in}}%
\pgfpathlineto{\pgfqpoint{1.815104in}{0.899777in}}%
\pgfpathlineto{\pgfqpoint{1.829237in}{0.880755in}}%
\pgfpathlineto{\pgfqpoint{1.843369in}{0.869998in}}%
\pgfpathlineto{\pgfqpoint{1.857502in}{0.864412in}}%
\pgfpathlineto{\pgfqpoint{1.871635in}{0.864412in}}%
\pgfpathlineto{\pgfqpoint{1.885767in}{0.853203in}}%
\pgfpathlineto{\pgfqpoint{1.899900in}{0.849593in}}%
\pgfpathlineto{\pgfqpoint{1.914032in}{0.849593in}}%
\pgfpathlineto{\pgfqpoint{1.928165in}{0.849593in}}%
\pgfpathlineto{\pgfqpoint{1.942298in}{0.840257in}}%
\pgfpathlineto{\pgfqpoint{1.956430in}{0.833933in}}%
\pgfpathlineto{\pgfqpoint{1.970563in}{0.833933in}}%
\pgfpathlineto{\pgfqpoint{1.984695in}{0.831401in}}%
\pgfpathlineto{\pgfqpoint{1.998828in}{0.831401in}}%
\pgfpathlineto{\pgfqpoint{2.012961in}{0.829638in}}%
\pgfpathlineto{\pgfqpoint{2.027093in}{0.829354in}}%
\pgfpathlineto{\pgfqpoint{2.041226in}{0.808082in}}%
\pgfpathlineto{\pgfqpoint{2.055358in}{0.808082in}}%
\pgfpathlineto{\pgfqpoint{2.069491in}{0.801983in}}%
\pgfpathlineto{\pgfqpoint{2.083624in}{0.801983in}}%
\pgfpathlineto{\pgfqpoint{2.097756in}{0.800704in}}%
\pgfpathlineto{\pgfqpoint{2.111889in}{0.800704in}}%
\pgfpathlineto{\pgfqpoint{2.126021in}{0.794722in}}%
\pgfpathlineto{\pgfqpoint{2.140154in}{0.773477in}}%
\pgfpathlineto{\pgfqpoint{2.154287in}{0.764538in}}%
\pgfpathlineto{\pgfqpoint{2.168419in}{0.764336in}}%
\pgfpathlineto{\pgfqpoint{2.182552in}{0.764150in}}%
\pgfpathlineto{\pgfqpoint{2.196684in}{0.764150in}}%
\pgfpathlineto{\pgfqpoint{2.210817in}{0.758805in}}%
\pgfpathlineto{\pgfqpoint{2.224950in}{0.758805in}}%
\pgfpathlineto{\pgfqpoint{2.239082in}{0.758805in}}%
\pgfpathlineto{\pgfqpoint{2.253215in}{0.758782in}}%
\pgfpathlineto{\pgfqpoint{2.267347in}{0.758782in}}%
\pgfpathlineto{\pgfqpoint{2.281480in}{0.758782in}}%
\pgfpathlineto{\pgfqpoint{2.295613in}{0.747988in}}%
\pgfpathlineto{\pgfqpoint{2.309745in}{0.747988in}}%
\pgfpathlineto{\pgfqpoint{2.323878in}{0.747988in}}%
\pgfpathlineto{\pgfqpoint{2.338010in}{0.747988in}}%
\pgfpathlineto{\pgfqpoint{2.352143in}{0.747988in}}%
\pgfpathlineto{\pgfqpoint{2.366276in}{0.747988in}}%
\pgfpathlineto{\pgfqpoint{2.380408in}{0.747975in}}%
\pgfpathlineto{\pgfqpoint{2.394541in}{0.747975in}}%
\pgfpathlineto{\pgfqpoint{2.408673in}{0.747975in}}%
\pgfpathlineto{\pgfqpoint{2.422806in}{0.745946in}}%
\pgfpathlineto{\pgfqpoint{2.436939in}{0.745946in}}%
\pgfpathlineto{\pgfqpoint{2.451071in}{0.745946in}}%
\pgfpathlineto{\pgfqpoint{2.465204in}{0.745946in}}%
\pgfpathlineto{\pgfqpoint{2.479336in}{0.745946in}}%
\pgfpathlineto{\pgfqpoint{2.493469in}{0.745946in}}%
\pgfpathlineto{\pgfqpoint{2.507602in}{0.745946in}}%
\pgfpathlineto{\pgfqpoint{2.521734in}{0.745946in}}%
\pgfpathlineto{\pgfqpoint{2.535867in}{0.742646in}}%
\pgfpathlineto{\pgfqpoint{2.549999in}{0.742646in}}%
\pgfpathlineto{\pgfqpoint{2.564132in}{0.742646in}}%
\pgfpathlineto{\pgfqpoint{2.578265in}{0.742646in}}%
\pgfpathlineto{\pgfqpoint{2.592397in}{0.742646in}}%
\pgfpathlineto{\pgfqpoint{2.606530in}{0.742646in}}%
\pgfpathlineto{\pgfqpoint{2.620662in}{0.742646in}}%
\pgfpathlineto{\pgfqpoint{2.634795in}{0.742646in}}%
\pgfpathlineto{\pgfqpoint{2.648928in}{0.742646in}}%
\pgfpathlineto{\pgfqpoint{2.663060in}{0.742646in}}%
\pgfpathlineto{\pgfqpoint{2.677193in}{0.742646in}}%
\pgfpathlineto{\pgfqpoint{2.691325in}{0.742646in}}%
\pgfpathlineto{\pgfqpoint{2.705458in}{0.742646in}}%
\pgfpathlineto{\pgfqpoint{2.719591in}{0.742646in}}%
\pgfpathlineto{\pgfqpoint{2.733723in}{0.742646in}}%
\pgfpathlineto{\pgfqpoint{2.747856in}{0.742646in}}%
\pgfpathlineto{\pgfqpoint{2.761988in}{0.742646in}}%
\pgfpathlineto{\pgfqpoint{2.776121in}{0.742646in}}%
\pgfpathlineto{\pgfqpoint{2.790254in}{0.742646in}}%
\pgfpathlineto{\pgfqpoint{2.804386in}{0.742646in}}%
\pgfpathlineto{\pgfqpoint{2.818519in}{0.742646in}}%
\pgfpathlineto{\pgfqpoint{2.832651in}{0.742646in}}%
\pgfpathlineto{\pgfqpoint{2.846784in}{0.742646in}}%
\pgfpathlineto{\pgfqpoint{2.860917in}{0.742646in}}%
\pgfpathlineto{\pgfqpoint{2.875049in}{0.742646in}}%
\pgfpathlineto{\pgfqpoint{2.889182in}{0.736455in}}%
\pgfpathlineto{\pgfqpoint{2.903314in}{0.736455in}}%
\pgfpathlineto{\pgfqpoint{2.917447in}{0.736455in}}%
\pgfpathlineto{\pgfqpoint{2.931580in}{0.733516in}}%
\pgfpathlineto{\pgfqpoint{2.931580in}{0.477955in}}%
\pgfpathlineto{\pgfqpoint{2.931580in}{0.477955in}}%
\pgfpathlineto{\pgfqpoint{2.917447in}{0.519407in}}%
\pgfpathlineto{\pgfqpoint{2.903314in}{0.519407in}}%
\pgfpathlineto{\pgfqpoint{2.889182in}{0.519407in}}%
\pgfpathlineto{\pgfqpoint{2.875049in}{0.572062in}}%
\pgfpathlineto{\pgfqpoint{2.860917in}{0.572062in}}%
\pgfpathlineto{\pgfqpoint{2.846784in}{0.572062in}}%
\pgfpathlineto{\pgfqpoint{2.832651in}{0.572062in}}%
\pgfpathlineto{\pgfqpoint{2.818519in}{0.572062in}}%
\pgfpathlineto{\pgfqpoint{2.804386in}{0.572062in}}%
\pgfpathlineto{\pgfqpoint{2.790254in}{0.572062in}}%
\pgfpathlineto{\pgfqpoint{2.776121in}{0.572062in}}%
\pgfpathlineto{\pgfqpoint{2.761988in}{0.572062in}}%
\pgfpathlineto{\pgfqpoint{2.747856in}{0.572062in}}%
\pgfpathlineto{\pgfqpoint{2.733723in}{0.572062in}}%
\pgfpathlineto{\pgfqpoint{2.719591in}{0.572062in}}%
\pgfpathlineto{\pgfqpoint{2.705458in}{0.572062in}}%
\pgfpathlineto{\pgfqpoint{2.691325in}{0.572062in}}%
\pgfpathlineto{\pgfqpoint{2.677193in}{0.572062in}}%
\pgfpathlineto{\pgfqpoint{2.663060in}{0.572062in}}%
\pgfpathlineto{\pgfqpoint{2.648928in}{0.572062in}}%
\pgfpathlineto{\pgfqpoint{2.634795in}{0.572062in}}%
\pgfpathlineto{\pgfqpoint{2.620662in}{0.572062in}}%
\pgfpathlineto{\pgfqpoint{2.606530in}{0.572062in}}%
\pgfpathlineto{\pgfqpoint{2.592397in}{0.572062in}}%
\pgfpathlineto{\pgfqpoint{2.578265in}{0.572062in}}%
\pgfpathlineto{\pgfqpoint{2.564132in}{0.572062in}}%
\pgfpathlineto{\pgfqpoint{2.549999in}{0.572062in}}%
\pgfpathlineto{\pgfqpoint{2.535867in}{0.572062in}}%
\pgfpathlineto{\pgfqpoint{2.521734in}{0.593525in}}%
\pgfpathlineto{\pgfqpoint{2.507602in}{0.593525in}}%
\pgfpathlineto{\pgfqpoint{2.493469in}{0.593525in}}%
\pgfpathlineto{\pgfqpoint{2.479336in}{0.593525in}}%
\pgfpathlineto{\pgfqpoint{2.465204in}{0.593525in}}%
\pgfpathlineto{\pgfqpoint{2.451071in}{0.593525in}}%
\pgfpathlineto{\pgfqpoint{2.436939in}{0.593525in}}%
\pgfpathlineto{\pgfqpoint{2.422806in}{0.593525in}}%
\pgfpathlineto{\pgfqpoint{2.408673in}{0.609155in}}%
\pgfpathlineto{\pgfqpoint{2.394541in}{0.609155in}}%
\pgfpathlineto{\pgfqpoint{2.380408in}{0.609155in}}%
\pgfpathlineto{\pgfqpoint{2.366276in}{0.609264in}}%
\pgfpathlineto{\pgfqpoint{2.352143in}{0.609264in}}%
\pgfpathlineto{\pgfqpoint{2.338010in}{0.609264in}}%
\pgfpathlineto{\pgfqpoint{2.323878in}{0.609264in}}%
\pgfpathlineto{\pgfqpoint{2.309745in}{0.609264in}}%
\pgfpathlineto{\pgfqpoint{2.295613in}{0.609264in}}%
\pgfpathlineto{\pgfqpoint{2.281480in}{0.644387in}}%
\pgfpathlineto{\pgfqpoint{2.267347in}{0.644387in}}%
\pgfpathlineto{\pgfqpoint{2.253215in}{0.644387in}}%
\pgfpathlineto{\pgfqpoint{2.239082in}{0.644566in}}%
\pgfpathlineto{\pgfqpoint{2.224950in}{0.644566in}}%
\pgfpathlineto{\pgfqpoint{2.210817in}{0.644566in}}%
\pgfpathlineto{\pgfqpoint{2.196684in}{0.663847in}}%
\pgfpathlineto{\pgfqpoint{2.182552in}{0.663847in}}%
\pgfpathlineto{\pgfqpoint{2.168419in}{0.665284in}}%
\pgfpathlineto{\pgfqpoint{2.154287in}{0.666510in}}%
\pgfpathlineto{\pgfqpoint{2.140154in}{0.681247in}}%
\pgfpathlineto{\pgfqpoint{2.126021in}{0.688142in}}%
\pgfpathlineto{\pgfqpoint{2.111889in}{0.705665in}}%
\pgfpathlineto{\pgfqpoint{2.097756in}{0.705665in}}%
\pgfpathlineto{\pgfqpoint{2.083624in}{0.709155in}}%
\pgfpathlineto{\pgfqpoint{2.069491in}{0.709155in}}%
\pgfpathlineto{\pgfqpoint{2.055358in}{0.735890in}}%
\pgfpathlineto{\pgfqpoint{2.041226in}{0.735890in}}%
\pgfpathlineto{\pgfqpoint{2.027093in}{0.740382in}}%
\pgfpathlineto{\pgfqpoint{2.012961in}{0.741389in}}%
\pgfpathlineto{\pgfqpoint{1.998828in}{0.745908in}}%
\pgfpathlineto{\pgfqpoint{1.984695in}{0.745908in}}%
\pgfpathlineto{\pgfqpoint{1.970563in}{0.758409in}}%
\pgfpathlineto{\pgfqpoint{1.956430in}{0.758409in}}%
\pgfpathlineto{\pgfqpoint{1.942298in}{0.772673in}}%
\pgfpathlineto{\pgfqpoint{1.928165in}{0.786384in}}%
\pgfpathlineto{\pgfqpoint{1.914032in}{0.786384in}}%
\pgfpathlineto{\pgfqpoint{1.899900in}{0.786384in}}%
\pgfpathlineto{\pgfqpoint{1.885767in}{0.792377in}}%
\pgfpathlineto{\pgfqpoint{1.871635in}{0.811572in}}%
\pgfpathlineto{\pgfqpoint{1.857502in}{0.811572in}}%
\pgfpathlineto{\pgfqpoint{1.843369in}{0.826552in}}%
\pgfpathlineto{\pgfqpoint{1.829237in}{0.829734in}}%
\pgfpathlineto{\pgfqpoint{1.815104in}{0.847920in}}%
\pgfpathlineto{\pgfqpoint{1.800972in}{0.852187in}}%
\pgfpathlineto{\pgfqpoint{1.786839in}{0.907441in}}%
\pgfpathlineto{\pgfqpoint{1.772706in}{0.927113in}}%
\pgfpathlineto{\pgfqpoint{1.758574in}{1.047233in}}%
\pgfpathlineto{\pgfqpoint{1.744441in}{1.098968in}}%
\pgfpathlineto{\pgfqpoint{1.730309in}{1.124616in}}%
\pgfpathlineto{\pgfqpoint{1.716176in}{1.143566in}}%
\pgfpathlineto{\pgfqpoint{1.702044in}{1.143566in}}%
\pgfpathlineto{\pgfqpoint{1.687911in}{1.151894in}}%
\pgfpathlineto{\pgfqpoint{1.673778in}{1.170846in}}%
\pgfpathlineto{\pgfqpoint{1.659646in}{1.201082in}}%
\pgfpathlineto{\pgfqpoint{1.645513in}{1.211108in}}%
\pgfpathlineto{\pgfqpoint{1.631381in}{1.212015in}}%
\pgfpathlineto{\pgfqpoint{1.617248in}{1.212015in}}%
\pgfpathlineto{\pgfqpoint{1.603115in}{1.222948in}}%
\pgfpathlineto{\pgfqpoint{1.588983in}{1.231710in}}%
\pgfpathlineto{\pgfqpoint{1.574850in}{1.238297in}}%
\pgfpathlineto{\pgfqpoint{1.560718in}{1.243741in}}%
\pgfpathlineto{\pgfqpoint{1.546585in}{1.248137in}}%
\pgfpathlineto{\pgfqpoint{1.532452in}{1.251841in}}%
\pgfpathlineto{\pgfqpoint{1.518320in}{1.254956in}}%
\pgfpathlineto{\pgfqpoint{1.504187in}{1.257727in}}%
\pgfpathlineto{\pgfqpoint{1.490055in}{1.288213in}}%
\pgfpathlineto{\pgfqpoint{1.475922in}{1.290442in}}%
\pgfpathlineto{\pgfqpoint{1.461789in}{1.293384in}}%
\pgfpathlineto{\pgfqpoint{1.447657in}{1.297021in}}%
\pgfpathlineto{\pgfqpoint{1.433524in}{1.437046in}}%
\pgfpathlineto{\pgfqpoint{1.419392in}{1.545774in}}%
\pgfpathlineto{\pgfqpoint{1.405259in}{1.554304in}}%
\pgfpathlineto{\pgfqpoint{1.391126in}{1.576441in}}%
\pgfpathlineto{\pgfqpoint{1.376994in}{1.586292in}}%
\pgfpathlineto{\pgfqpoint{1.362861in}{1.620794in}}%
\pgfpathlineto{\pgfqpoint{1.348729in}{1.680860in}}%
\pgfpathlineto{\pgfqpoint{1.334596in}{1.704962in}}%
\pgfpathlineto{\pgfqpoint{1.320463in}{1.709144in}}%
\pgfpathlineto{\pgfqpoint{1.306331in}{1.800985in}}%
\pgfpathlineto{\pgfqpoint{1.292198in}{1.803076in}}%
\pgfpathlineto{\pgfqpoint{1.278066in}{1.805368in}}%
\pgfpathlineto{\pgfqpoint{1.263933in}{1.808185in}}%
\pgfpathlineto{\pgfqpoint{1.249800in}{1.812713in}}%
\pgfpathlineto{\pgfqpoint{1.235668in}{1.823522in}}%
\pgfpathlineto{\pgfqpoint{1.221535in}{1.869413in}}%
\pgfpathlineto{\pgfqpoint{1.207403in}{1.882882in}}%
\pgfpathlineto{\pgfqpoint{1.193270in}{1.927177in}}%
\pgfpathlineto{\pgfqpoint{1.179137in}{1.932703in}}%
\pgfpathlineto{\pgfqpoint{1.165005in}{1.932703in}}%
\pgfpathlineto{\pgfqpoint{1.150872in}{1.932703in}}%
\pgfpathlineto{\pgfqpoint{1.136740in}{1.932703in}}%
\pgfpathlineto{\pgfqpoint{1.122607in}{1.968952in}}%
\pgfpathlineto{\pgfqpoint{1.108474in}{1.968952in}}%
\pgfpathlineto{\pgfqpoint{1.094342in}{1.968952in}}%
\pgfpathlineto{\pgfqpoint{1.080209in}{1.968952in}}%
\pgfpathlineto{\pgfqpoint{1.066077in}{1.968952in}}%
\pgfpathlineto{\pgfqpoint{1.051944in}{2.001177in}}%
\pgfpathlineto{\pgfqpoint{1.037811in}{2.001177in}}%
\pgfpathlineto{\pgfqpoint{1.023679in}{2.001313in}}%
\pgfpathlineto{\pgfqpoint{1.009546in}{2.001313in}}%
\pgfpathlineto{\pgfqpoint{0.995414in}{2.001319in}}%
\pgfpathlineto{\pgfqpoint{0.981281in}{2.001319in}}%
\pgfpathlineto{\pgfqpoint{0.967148in}{2.001319in}}%
\pgfpathlineto{\pgfqpoint{0.953016in}{2.001319in}}%
\pgfpathlineto{\pgfqpoint{0.938883in}{2.001319in}}%
\pgfpathlineto{\pgfqpoint{0.924751in}{2.001319in}}%
\pgfpathlineto{\pgfqpoint{0.910618in}{2.001341in}}%
\pgfpathlineto{\pgfqpoint{0.896485in}{2.017763in}}%
\pgfpathlineto{\pgfqpoint{0.882353in}{2.028948in}}%
\pgfpathlineto{\pgfqpoint{0.868220in}{2.067329in}}%
\pgfpathlineto{\pgfqpoint{0.854088in}{2.067329in}}%
\pgfpathlineto{\pgfqpoint{0.839955in}{2.067329in}}%
\pgfpathlineto{\pgfqpoint{0.825822in}{2.090437in}}%
\pgfpathlineto{\pgfqpoint{0.811690in}{2.090437in}}%
\pgfpathlineto{\pgfqpoint{0.797557in}{2.135031in}}%
\pgfpathlineto{\pgfqpoint{0.783425in}{2.135031in}}%
\pgfpathlineto{\pgfqpoint{0.769292in}{2.135031in}}%
\pgfpathlineto{\pgfqpoint{0.755159in}{2.135031in}}%
\pgfpathlineto{\pgfqpoint{0.741027in}{2.135031in}}%
\pgfpathlineto{\pgfqpoint{0.726894in}{2.139955in}}%
\pgfpathlineto{\pgfqpoint{0.712762in}{2.150622in}}%
\pgfpathlineto{\pgfqpoint{0.698629in}{2.177126in}}%
\pgfpathlineto{\pgfqpoint{0.684496in}{2.177126in}}%
\pgfpathlineto{\pgfqpoint{0.670364in}{2.177126in}}%
\pgfpathlineto{\pgfqpoint{0.656231in}{2.177126in}}%
\pgfpathlineto{\pgfqpoint{0.642099in}{2.262479in}}%
\pgfpathlineto{\pgfqpoint{0.627966in}{2.262479in}}%
\pgfpathlineto{\pgfqpoint{0.613833in}{2.290396in}}%
\pgfpathlineto{\pgfqpoint{0.599701in}{2.293731in}}%
\pgfpathlineto{\pgfqpoint{0.585568in}{2.293731in}}%
\pgfpathlineto{\pgfqpoint{0.571436in}{2.340695in}}%
\pgfpathlineto{\pgfqpoint{0.557303in}{2.380788in}}%
\pgfpathlineto{\pgfqpoint{0.543170in}{2.461632in}}%
\pgfpathclose%
\pgfusepath{fill}%
\end{pgfscope}%
\begin{pgfscope}%
\pgfpathrectangle{\pgfqpoint{0.423750in}{0.375000in}}{\pgfqpoint{2.627250in}{2.265000in}}%
\pgfusepath{clip}%
\pgfsetroundcap%
\pgfsetroundjoin%
\pgfsetlinewidth{1.505625pt}%
\definecolor{currentstroke}{rgb}{0.121569,0.466667,0.705882}%
\pgfsetstrokecolor{currentstroke}%
\pgfsetdash{}{0pt}%
\pgfpathmoveto{\pgfqpoint{0.543170in}{2.456006in}}%
\pgfpathlineto{\pgfqpoint{0.571436in}{2.341770in}}%
\pgfpathlineto{\pgfqpoint{0.585568in}{2.303270in}}%
\pgfpathlineto{\pgfqpoint{0.599701in}{2.296807in}}%
\pgfpathlineto{\pgfqpoint{0.613833in}{2.296807in}}%
\pgfpathlineto{\pgfqpoint{0.627966in}{2.270846in}}%
\pgfpathlineto{\pgfqpoint{0.642099in}{2.260458in}}%
\pgfpathlineto{\pgfqpoint{0.656231in}{2.241423in}}%
\pgfpathlineto{\pgfqpoint{0.670364in}{2.233953in}}%
\pgfpathlineto{\pgfqpoint{0.684496in}{2.217523in}}%
\pgfpathlineto{\pgfqpoint{0.712762in}{2.217523in}}%
\pgfpathlineto{\pgfqpoint{0.726894in}{2.197541in}}%
\pgfpathlineto{\pgfqpoint{0.769292in}{2.149546in}}%
\pgfpathlineto{\pgfqpoint{0.839955in}{2.149213in}}%
\pgfpathlineto{\pgfqpoint{0.854088in}{2.147472in}}%
\pgfpathlineto{\pgfqpoint{0.882353in}{2.147472in}}%
\pgfpathlineto{\pgfqpoint{0.896485in}{2.144995in}}%
\pgfpathlineto{\pgfqpoint{0.953016in}{2.144948in}}%
\pgfpathlineto{\pgfqpoint{0.967148in}{2.137603in}}%
\pgfpathlineto{\pgfqpoint{1.094342in}{2.137603in}}%
\pgfpathlineto{\pgfqpoint{1.108474in}{2.130990in}}%
\pgfpathlineto{\pgfqpoint{1.193270in}{2.130990in}}%
\pgfpathlineto{\pgfqpoint{1.207403in}{2.120726in}}%
\pgfpathlineto{\pgfqpoint{1.221535in}{2.119930in}}%
\pgfpathlineto{\pgfqpoint{1.235668in}{2.081169in}}%
\pgfpathlineto{\pgfqpoint{1.334596in}{2.081169in}}%
\pgfpathlineto{\pgfqpoint{1.348729in}{2.073171in}}%
\pgfpathlineto{\pgfqpoint{1.391126in}{2.073171in}}%
\pgfpathlineto{\pgfqpoint{1.405259in}{2.055363in}}%
\pgfpathlineto{\pgfqpoint{1.447657in}{2.055363in}}%
\pgfpathlineto{\pgfqpoint{1.461789in}{2.040144in}}%
\pgfpathlineto{\pgfqpoint{1.631381in}{2.040144in}}%
\pgfpathlineto{\pgfqpoint{1.645513in}{2.023222in}}%
\pgfpathlineto{\pgfqpoint{1.673778in}{2.023222in}}%
\pgfpathlineto{\pgfqpoint{1.687911in}{2.019464in}}%
\pgfpathlineto{\pgfqpoint{1.716176in}{2.019464in}}%
\pgfpathlineto{\pgfqpoint{1.730309in}{2.015580in}}%
\pgfpathlineto{\pgfqpoint{1.744441in}{2.001054in}}%
\pgfpathlineto{\pgfqpoint{1.970563in}{2.001054in}}%
\pgfpathlineto{\pgfqpoint{1.984695in}{1.995597in}}%
\pgfpathlineto{\pgfqpoint{2.267347in}{1.995597in}}%
\pgfpathlineto{\pgfqpoint{2.281480in}{1.983142in}}%
\pgfpathlineto{\pgfqpoint{2.380408in}{1.983142in}}%
\pgfpathlineto{\pgfqpoint{2.394541in}{1.977363in}}%
\pgfpathlineto{\pgfqpoint{2.634795in}{1.976255in}}%
\pgfpathlineto{\pgfqpoint{2.648928in}{1.956425in}}%
\pgfpathlineto{\pgfqpoint{2.747856in}{1.956425in}}%
\pgfpathlineto{\pgfqpoint{2.761988in}{1.954616in}}%
\pgfpathlineto{\pgfqpoint{2.931580in}{1.954616in}}%
\pgfpathlineto{\pgfqpoint{2.931580in}{1.954616in}}%
\pgfusepath{stroke}%
\end{pgfscope}%
\begin{pgfscope}%
\pgfpathrectangle{\pgfqpoint{0.423750in}{0.375000in}}{\pgfqpoint{2.627250in}{2.265000in}}%
\pgfusepath{clip}%
\pgfsetroundcap%
\pgfsetroundjoin%
\pgfsetlinewidth{1.505625pt}%
\definecolor{currentstroke}{rgb}{1.000000,0.498039,0.054902}%
\pgfsetstrokecolor{currentstroke}%
\pgfsetdash{}{0pt}%
\pgfpathmoveto{\pgfqpoint{0.543170in}{2.449615in}}%
\pgfpathlineto{\pgfqpoint{0.557303in}{2.395493in}}%
\pgfpathlineto{\pgfqpoint{0.571436in}{2.395493in}}%
\pgfpathlineto{\pgfqpoint{0.585568in}{2.346270in}}%
\pgfpathlineto{\pgfqpoint{0.599701in}{2.336757in}}%
\pgfpathlineto{\pgfqpoint{0.613833in}{2.336757in}}%
\pgfpathlineto{\pgfqpoint{0.627966in}{2.290908in}}%
\pgfpathlineto{\pgfqpoint{0.642099in}{2.264201in}}%
\pgfpathlineto{\pgfqpoint{0.656231in}{2.264201in}}%
\pgfpathlineto{\pgfqpoint{0.670364in}{2.256108in}}%
\pgfpathlineto{\pgfqpoint{0.684496in}{2.190741in}}%
\pgfpathlineto{\pgfqpoint{0.783425in}{2.190741in}}%
\pgfpathlineto{\pgfqpoint{0.797557in}{2.181934in}}%
\pgfpathlineto{\pgfqpoint{0.811690in}{2.168263in}}%
\pgfpathlineto{\pgfqpoint{0.825822in}{2.133264in}}%
\pgfpathlineto{\pgfqpoint{0.854088in}{2.132594in}}%
\pgfpathlineto{\pgfqpoint{0.868220in}{2.102666in}}%
\pgfpathlineto{\pgfqpoint{0.882353in}{2.102666in}}%
\pgfpathlineto{\pgfqpoint{0.896485in}{2.068017in}}%
\pgfpathlineto{\pgfqpoint{0.910618in}{2.067599in}}%
\pgfpathlineto{\pgfqpoint{0.924751in}{2.061416in}}%
\pgfpathlineto{\pgfqpoint{0.938883in}{2.043435in}}%
\pgfpathlineto{\pgfqpoint{0.967148in}{2.042572in}}%
\pgfpathlineto{\pgfqpoint{0.981281in}{2.020079in}}%
\pgfpathlineto{\pgfqpoint{0.995414in}{2.020079in}}%
\pgfpathlineto{\pgfqpoint{1.009546in}{2.006303in}}%
\pgfpathlineto{\pgfqpoint{1.051944in}{2.006303in}}%
\pgfpathlineto{\pgfqpoint{1.066077in}{1.988602in}}%
\pgfpathlineto{\pgfqpoint{1.080209in}{1.955235in}}%
\pgfpathlineto{\pgfqpoint{1.108474in}{1.955235in}}%
\pgfpathlineto{\pgfqpoint{1.122607in}{1.937801in}}%
\pgfpathlineto{\pgfqpoint{1.278066in}{1.937801in}}%
\pgfpathlineto{\pgfqpoint{1.292198in}{1.918525in}}%
\pgfpathlineto{\pgfqpoint{1.306331in}{1.918525in}}%
\pgfpathlineto{\pgfqpoint{1.320463in}{1.872697in}}%
\pgfpathlineto{\pgfqpoint{1.475922in}{1.872697in}}%
\pgfpathlineto{\pgfqpoint{1.490055in}{1.870087in}}%
\pgfpathlineto{\pgfqpoint{1.532452in}{1.870087in}}%
\pgfpathlineto{\pgfqpoint{1.546585in}{1.860039in}}%
\pgfpathlineto{\pgfqpoint{1.560718in}{1.860039in}}%
\pgfpathlineto{\pgfqpoint{1.574850in}{1.805207in}}%
\pgfpathlineto{\pgfqpoint{1.588983in}{1.801627in}}%
\pgfpathlineto{\pgfqpoint{1.631381in}{1.801627in}}%
\pgfpathlineto{\pgfqpoint{1.645513in}{1.797209in}}%
\pgfpathlineto{\pgfqpoint{1.730309in}{1.797209in}}%
\pgfpathlineto{\pgfqpoint{1.744441in}{1.770942in}}%
\pgfpathlineto{\pgfqpoint{1.843369in}{1.770942in}}%
\pgfpathlineto{\pgfqpoint{1.857502in}{1.725858in}}%
\pgfpathlineto{\pgfqpoint{1.984695in}{1.725858in}}%
\pgfpathlineto{\pgfqpoint{1.998828in}{1.718883in}}%
\pgfpathlineto{\pgfqpoint{2.210817in}{1.718883in}}%
\pgfpathlineto{\pgfqpoint{2.224950in}{1.678276in}}%
\pgfpathlineto{\pgfqpoint{2.549999in}{1.678276in}}%
\pgfpathlineto{\pgfqpoint{2.564132in}{1.674510in}}%
\pgfpathlineto{\pgfqpoint{2.804386in}{1.674510in}}%
\pgfpathlineto{\pgfqpoint{2.818519in}{1.661803in}}%
\pgfpathlineto{\pgfqpoint{2.875049in}{1.661803in}}%
\pgfpathlineto{\pgfqpoint{2.889182in}{1.659615in}}%
\pgfpathlineto{\pgfqpoint{2.931580in}{1.659615in}}%
\pgfpathlineto{\pgfqpoint{2.931580in}{1.659615in}}%
\pgfusepath{stroke}%
\end{pgfscope}%
\begin{pgfscope}%
\pgfpathrectangle{\pgfqpoint{0.423750in}{0.375000in}}{\pgfqpoint{2.627250in}{2.265000in}}%
\pgfusepath{clip}%
\pgfsetroundcap%
\pgfsetroundjoin%
\pgfsetlinewidth{1.505625pt}%
\definecolor{currentstroke}{rgb}{0.172549,0.627451,0.172549}%
\pgfsetstrokecolor{currentstroke}%
\pgfsetdash{}{0pt}%
\pgfpathmoveto{\pgfqpoint{0.543170in}{2.437018in}}%
\pgfpathlineto{\pgfqpoint{0.557303in}{2.382008in}}%
\pgfpathlineto{\pgfqpoint{0.571436in}{2.339656in}}%
\pgfpathlineto{\pgfqpoint{0.585568in}{2.314742in}}%
\pgfpathlineto{\pgfqpoint{0.599701in}{2.314742in}}%
\pgfpathlineto{\pgfqpoint{0.613833in}{2.283358in}}%
\pgfpathlineto{\pgfqpoint{0.627966in}{2.264794in}}%
\pgfpathlineto{\pgfqpoint{0.656231in}{2.264794in}}%
\pgfpathlineto{\pgfqpoint{0.670364in}{2.248762in}}%
\pgfpathlineto{\pgfqpoint{0.726894in}{2.248762in}}%
\pgfpathlineto{\pgfqpoint{0.741027in}{2.241493in}}%
\pgfpathlineto{\pgfqpoint{0.811690in}{2.241493in}}%
\pgfpathlineto{\pgfqpoint{0.825822in}{2.166294in}}%
\pgfpathlineto{\pgfqpoint{0.839955in}{2.157101in}}%
\pgfpathlineto{\pgfqpoint{0.854088in}{2.157101in}}%
\pgfpathlineto{\pgfqpoint{0.868220in}{2.083679in}}%
\pgfpathlineto{\pgfqpoint{0.882353in}{2.081736in}}%
\pgfpathlineto{\pgfqpoint{0.896485in}{2.074261in}}%
\pgfpathlineto{\pgfqpoint{0.910618in}{2.058322in}}%
\pgfpathlineto{\pgfqpoint{0.924751in}{2.025684in}}%
\pgfpathlineto{\pgfqpoint{0.938883in}{1.967938in}}%
\pgfpathlineto{\pgfqpoint{0.953016in}{1.967938in}}%
\pgfpathlineto{\pgfqpoint{0.967148in}{1.949432in}}%
\pgfpathlineto{\pgfqpoint{0.981281in}{1.946429in}}%
\pgfpathlineto{\pgfqpoint{0.995414in}{1.946429in}}%
\pgfpathlineto{\pgfqpoint{1.009546in}{1.942732in}}%
\pgfpathlineto{\pgfqpoint{1.023679in}{1.942732in}}%
\pgfpathlineto{\pgfqpoint{1.037811in}{1.899902in}}%
\pgfpathlineto{\pgfqpoint{1.066077in}{1.899902in}}%
\pgfpathlineto{\pgfqpoint{1.080209in}{1.840720in}}%
\pgfpathlineto{\pgfqpoint{1.094342in}{1.840720in}}%
\pgfpathlineto{\pgfqpoint{1.108474in}{1.806816in}}%
\pgfpathlineto{\pgfqpoint{1.122607in}{1.777261in}}%
\pgfpathlineto{\pgfqpoint{1.136740in}{1.777261in}}%
\pgfpathlineto{\pgfqpoint{1.150872in}{1.754026in}}%
\pgfpathlineto{\pgfqpoint{1.165005in}{1.736250in}}%
\pgfpathlineto{\pgfqpoint{1.179137in}{1.736250in}}%
\pgfpathlineto{\pgfqpoint{1.193270in}{1.609987in}}%
\pgfpathlineto{\pgfqpoint{1.362861in}{1.609987in}}%
\pgfpathlineto{\pgfqpoint{1.376994in}{1.589365in}}%
\pgfpathlineto{\pgfqpoint{1.504187in}{1.589365in}}%
\pgfpathlineto{\pgfqpoint{1.518320in}{1.554905in}}%
\pgfpathlineto{\pgfqpoint{1.758574in}{1.554905in}}%
\pgfpathlineto{\pgfqpoint{1.772706in}{1.539739in}}%
\pgfpathlineto{\pgfqpoint{2.012961in}{1.539739in}}%
\pgfpathlineto{\pgfqpoint{2.027093in}{1.501470in}}%
\pgfpathlineto{\pgfqpoint{2.648928in}{1.501470in}}%
\pgfpathlineto{\pgfqpoint{2.663060in}{1.477351in}}%
\pgfpathlineto{\pgfqpoint{2.931580in}{1.477351in}}%
\pgfpathlineto{\pgfqpoint{2.931580in}{1.477351in}}%
\pgfusepath{stroke}%
\end{pgfscope}%
\begin{pgfscope}%
\pgfpathrectangle{\pgfqpoint{0.423750in}{0.375000in}}{\pgfqpoint{2.627250in}{2.265000in}}%
\pgfusepath{clip}%
\pgfsetroundcap%
\pgfsetroundjoin%
\pgfsetlinewidth{1.505625pt}%
\definecolor{currentstroke}{rgb}{0.839216,0.152941,0.156863}%
\pgfsetstrokecolor{currentstroke}%
\pgfsetdash{}{0pt}%
\pgfpathmoveto{\pgfqpoint{0.543170in}{2.492538in}}%
\pgfpathlineto{\pgfqpoint{0.571436in}{2.364253in}}%
\pgfpathlineto{\pgfqpoint{0.585568in}{2.347362in}}%
\pgfpathlineto{\pgfqpoint{0.599701in}{2.343364in}}%
\pgfpathlineto{\pgfqpoint{0.613833in}{2.289259in}}%
\pgfpathlineto{\pgfqpoint{0.642099in}{2.264112in}}%
\pgfpathlineto{\pgfqpoint{0.656231in}{2.264112in}}%
\pgfpathlineto{\pgfqpoint{0.670364in}{2.258969in}}%
\pgfpathlineto{\pgfqpoint{0.712762in}{2.258969in}}%
\pgfpathlineto{\pgfqpoint{0.726894in}{2.249434in}}%
\pgfpathlineto{\pgfqpoint{0.755159in}{2.249434in}}%
\pgfpathlineto{\pgfqpoint{0.769292in}{2.228647in}}%
\pgfpathlineto{\pgfqpoint{0.783425in}{2.212046in}}%
\pgfpathlineto{\pgfqpoint{0.797557in}{2.140859in}}%
\pgfpathlineto{\pgfqpoint{0.811690in}{2.140859in}}%
\pgfpathlineto{\pgfqpoint{0.825822in}{2.008490in}}%
\pgfpathlineto{\pgfqpoint{0.953016in}{2.008490in}}%
\pgfpathlineto{\pgfqpoint{0.967148in}{1.991552in}}%
\pgfpathlineto{\pgfqpoint{0.981281in}{1.950437in}}%
\pgfpathlineto{\pgfqpoint{0.995414in}{1.950437in}}%
\pgfpathlineto{\pgfqpoint{1.009546in}{1.945121in}}%
\pgfpathlineto{\pgfqpoint{1.023679in}{1.945121in}}%
\pgfpathlineto{\pgfqpoint{1.037811in}{1.925572in}}%
\pgfpathlineto{\pgfqpoint{1.051944in}{1.925572in}}%
\pgfpathlineto{\pgfqpoint{1.066077in}{1.915869in}}%
\pgfpathlineto{\pgfqpoint{1.080209in}{1.915869in}}%
\pgfpathlineto{\pgfqpoint{1.094342in}{1.908785in}}%
\pgfpathlineto{\pgfqpoint{1.108474in}{1.842830in}}%
\pgfpathlineto{\pgfqpoint{1.122607in}{1.842830in}}%
\pgfpathlineto{\pgfqpoint{1.136740in}{1.785286in}}%
\pgfpathlineto{\pgfqpoint{1.150872in}{1.746123in}}%
\pgfpathlineto{\pgfqpoint{1.263933in}{1.745690in}}%
\pgfpathlineto{\pgfqpoint{1.278066in}{1.730565in}}%
\pgfpathlineto{\pgfqpoint{1.419392in}{1.730565in}}%
\pgfpathlineto{\pgfqpoint{1.433524in}{1.720056in}}%
\pgfpathlineto{\pgfqpoint{1.447657in}{1.690577in}}%
\pgfpathlineto{\pgfqpoint{1.603115in}{1.690577in}}%
\pgfpathlineto{\pgfqpoint{1.617248in}{1.660643in}}%
\pgfpathlineto{\pgfqpoint{1.899900in}{1.660643in}}%
\pgfpathlineto{\pgfqpoint{1.914032in}{1.623491in}}%
\pgfpathlineto{\pgfqpoint{2.027093in}{1.623491in}}%
\pgfpathlineto{\pgfqpoint{2.041226in}{1.602699in}}%
\pgfpathlineto{\pgfqpoint{2.055358in}{1.557777in}}%
\pgfpathlineto{\pgfqpoint{2.323878in}{1.557777in}}%
\pgfpathlineto{\pgfqpoint{2.338010in}{1.513687in}}%
\pgfpathlineto{\pgfqpoint{2.620662in}{1.513687in}}%
\pgfpathlineto{\pgfqpoint{2.634795in}{1.501265in}}%
\pgfpathlineto{\pgfqpoint{2.818519in}{1.501265in}}%
\pgfpathlineto{\pgfqpoint{2.832651in}{1.492409in}}%
\pgfpathlineto{\pgfqpoint{2.931580in}{1.492409in}}%
\pgfpathlineto{\pgfqpoint{2.931580in}{1.492409in}}%
\pgfusepath{stroke}%
\end{pgfscope}%
\begin{pgfscope}%
\pgfpathrectangle{\pgfqpoint{0.423750in}{0.375000in}}{\pgfqpoint{2.627250in}{2.265000in}}%
\pgfusepath{clip}%
\pgfsetroundcap%
\pgfsetroundjoin%
\pgfsetlinewidth{1.505625pt}%
\definecolor{currentstroke}{rgb}{0.580392,0.403922,0.741176}%
\pgfsetstrokecolor{currentstroke}%
\pgfsetdash{}{0pt}%
\pgfpathmoveto{\pgfqpoint{0.543170in}{2.504004in}}%
\pgfpathlineto{\pgfqpoint{0.571436in}{2.402429in}}%
\pgfpathlineto{\pgfqpoint{0.585568in}{2.354546in}}%
\pgfpathlineto{\pgfqpoint{0.599701in}{2.354546in}}%
\pgfpathlineto{\pgfqpoint{0.613833in}{2.344763in}}%
\pgfpathlineto{\pgfqpoint{0.627966in}{2.331041in}}%
\pgfpathlineto{\pgfqpoint{0.642099in}{2.331041in}}%
\pgfpathlineto{\pgfqpoint{0.656231in}{2.202842in}}%
\pgfpathlineto{\pgfqpoint{0.698629in}{2.202842in}}%
\pgfpathlineto{\pgfqpoint{0.712762in}{2.174828in}}%
\pgfpathlineto{\pgfqpoint{0.726894in}{2.166017in}}%
\pgfpathlineto{\pgfqpoint{0.741027in}{2.162752in}}%
\pgfpathlineto{\pgfqpoint{0.797557in}{2.162752in}}%
\pgfpathlineto{\pgfqpoint{0.811690in}{2.125537in}}%
\pgfpathlineto{\pgfqpoint{0.825822in}{2.125537in}}%
\pgfpathlineto{\pgfqpoint{0.839955in}{2.100459in}}%
\pgfpathlineto{\pgfqpoint{0.868220in}{2.100459in}}%
\pgfpathlineto{\pgfqpoint{0.882353in}{2.075234in}}%
\pgfpathlineto{\pgfqpoint{0.896485in}{2.045117in}}%
\pgfpathlineto{\pgfqpoint{0.910618in}{2.036933in}}%
\pgfpathlineto{\pgfqpoint{1.051944in}{2.036707in}}%
\pgfpathlineto{\pgfqpoint{1.066077in}{2.008121in}}%
\pgfpathlineto{\pgfqpoint{1.122607in}{2.008121in}}%
\pgfpathlineto{\pgfqpoint{1.136740in}{1.957011in}}%
\pgfpathlineto{\pgfqpoint{1.179137in}{1.957011in}}%
\pgfpathlineto{\pgfqpoint{1.193270in}{1.954406in}}%
\pgfpathlineto{\pgfqpoint{1.207403in}{1.925718in}}%
\pgfpathlineto{\pgfqpoint{1.221535in}{1.919817in}}%
\pgfpathlineto{\pgfqpoint{1.235668in}{1.876349in}}%
\pgfpathlineto{\pgfqpoint{1.249800in}{1.870755in}}%
\pgfpathlineto{\pgfqpoint{1.278066in}{1.867351in}}%
\pgfpathlineto{\pgfqpoint{1.306331in}{1.865447in}}%
\pgfpathlineto{\pgfqpoint{1.320463in}{1.828400in}}%
\pgfpathlineto{\pgfqpoint{1.334596in}{1.827256in}}%
\pgfpathlineto{\pgfqpoint{1.348729in}{1.821629in}}%
\pgfpathlineto{\pgfqpoint{1.362861in}{1.721358in}}%
\pgfpathlineto{\pgfqpoint{1.376994in}{1.711232in}}%
\pgfpathlineto{\pgfqpoint{1.391126in}{1.708516in}}%
\pgfpathlineto{\pgfqpoint{1.405259in}{1.703547in}}%
\pgfpathlineto{\pgfqpoint{1.419392in}{1.701740in}}%
\pgfpathlineto{\pgfqpoint{1.433524in}{1.513973in}}%
\pgfpathlineto{\pgfqpoint{1.447657in}{1.463036in}}%
\pgfpathlineto{\pgfqpoint{1.475922in}{1.449349in}}%
\pgfpathlineto{\pgfqpoint{1.490055in}{1.444526in}}%
\pgfpathlineto{\pgfqpoint{1.504187in}{1.434897in}}%
\pgfpathlineto{\pgfqpoint{1.532452in}{1.423740in}}%
\pgfpathlineto{\pgfqpoint{1.560718in}{1.407448in}}%
\pgfpathlineto{\pgfqpoint{1.574850in}{1.395784in}}%
\pgfpathlineto{\pgfqpoint{1.588983in}{1.380772in}}%
\pgfpathlineto{\pgfqpoint{1.603115in}{1.359003in}}%
\pgfpathlineto{\pgfqpoint{1.617248in}{1.328452in}}%
\pgfpathlineto{\pgfqpoint{1.645513in}{1.328216in}}%
\pgfpathlineto{\pgfqpoint{1.659646in}{1.325380in}}%
\pgfpathlineto{\pgfqpoint{1.673778in}{1.318050in}}%
\pgfpathlineto{\pgfqpoint{1.687911in}{1.314409in}}%
\pgfpathlineto{\pgfqpoint{1.730309in}{1.310078in}}%
\pgfpathlineto{\pgfqpoint{1.744441in}{1.306789in}}%
\pgfpathlineto{\pgfqpoint{1.758574in}{1.237845in}}%
\pgfpathlineto{\pgfqpoint{1.772706in}{0.953866in}}%
\pgfpathlineto{\pgfqpoint{1.786839in}{0.931915in}}%
\pgfpathlineto{\pgfqpoint{1.800972in}{0.879180in}}%
\pgfpathlineto{\pgfqpoint{1.815104in}{0.875650in}}%
\pgfpathlineto{\pgfqpoint{1.829237in}{0.856968in}}%
\pgfpathlineto{\pgfqpoint{1.843369in}{0.849358in}}%
\pgfpathlineto{\pgfqpoint{1.857502in}{0.839889in}}%
\pgfpathlineto{\pgfqpoint{1.871635in}{0.839889in}}%
\pgfpathlineto{\pgfqpoint{1.885767in}{0.825536in}}%
\pgfpathlineto{\pgfqpoint{1.899900in}{0.821016in}}%
\pgfpathlineto{\pgfqpoint{1.928165in}{0.821016in}}%
\pgfpathlineto{\pgfqpoint{1.956430in}{0.800853in}}%
\pgfpathlineto{\pgfqpoint{1.970563in}{0.800853in}}%
\pgfpathlineto{\pgfqpoint{1.984695in}{0.794915in}}%
\pgfpathlineto{\pgfqpoint{1.998828in}{0.794915in}}%
\pgfpathlineto{\pgfqpoint{2.012961in}{0.792247in}}%
\pgfpathlineto{\pgfqpoint{2.027093in}{0.791729in}}%
\pgfpathlineto{\pgfqpoint{2.041226in}{0.776187in}}%
\pgfpathlineto{\pgfqpoint{2.055358in}{0.776187in}}%
\pgfpathlineto{\pgfqpoint{2.069491in}{0.763124in}}%
\pgfpathlineto{\pgfqpoint{2.083624in}{0.763124in}}%
\pgfpathlineto{\pgfqpoint{2.097756in}{0.761152in}}%
\pgfpathlineto{\pgfqpoint{2.111889in}{0.761152in}}%
\pgfpathlineto{\pgfqpoint{2.126021in}{0.751710in}}%
\pgfpathlineto{\pgfqpoint{2.140154in}{0.734807in}}%
\pgfpathlineto{\pgfqpoint{2.154287in}{0.724064in}}%
\pgfpathlineto{\pgfqpoint{2.196684in}{0.722987in}}%
\pgfpathlineto{\pgfqpoint{2.210817in}{0.713639in}}%
\pgfpathlineto{\pgfqpoint{2.281480in}{0.713574in}}%
\pgfpathlineto{\pgfqpoint{2.295613in}{0.696662in}}%
\pgfpathlineto{\pgfqpoint{2.408673in}{0.696626in}}%
\pgfpathlineto{\pgfqpoint{2.422806in}{0.691620in}}%
\pgfpathlineto{\pgfqpoint{2.521734in}{0.691620in}}%
\pgfpathlineto{\pgfqpoint{2.535867in}{0.684789in}}%
\pgfpathlineto{\pgfqpoint{2.875049in}{0.684789in}}%
\pgfpathlineto{\pgfqpoint{2.889182in}{0.671556in}}%
\pgfpathlineto{\pgfqpoint{2.917447in}{0.671556in}}%
\pgfpathlineto{\pgfqpoint{2.931580in}{0.664511in}}%
\pgfpathlineto{\pgfqpoint{2.931580in}{0.664511in}}%
\pgfusepath{stroke}%
\end{pgfscope}%
\begin{pgfscope}%
\pgfsetrectcap%
\pgfsetmiterjoin%
\pgfsetlinewidth{0.000000pt}%
\definecolor{currentstroke}{rgb}{1.000000,1.000000,1.000000}%
\pgfsetstrokecolor{currentstroke}%
\pgfsetdash{}{0pt}%
\pgfpathmoveto{\pgfqpoint{0.423750in}{0.375000in}}%
\pgfpathlineto{\pgfqpoint{0.423750in}{2.640000in}}%
\pgfusepath{}%
\end{pgfscope}%
\begin{pgfscope}%
\pgfsetrectcap%
\pgfsetmiterjoin%
\pgfsetlinewidth{0.000000pt}%
\definecolor{currentstroke}{rgb}{1.000000,1.000000,1.000000}%
\pgfsetstrokecolor{currentstroke}%
\pgfsetdash{}{0pt}%
\pgfpathmoveto{\pgfqpoint{3.051000in}{0.375000in}}%
\pgfpathlineto{\pgfqpoint{3.051000in}{2.640000in}}%
\pgfusepath{}%
\end{pgfscope}%
\begin{pgfscope}%
\pgfsetrectcap%
\pgfsetmiterjoin%
\pgfsetlinewidth{0.000000pt}%
\definecolor{currentstroke}{rgb}{1.000000,1.000000,1.000000}%
\pgfsetstrokecolor{currentstroke}%
\pgfsetdash{}{0pt}%
\pgfpathmoveto{\pgfqpoint{0.423750in}{0.375000in}}%
\pgfpathlineto{\pgfqpoint{3.051000in}{0.375000in}}%
\pgfusepath{}%
\end{pgfscope}%
\begin{pgfscope}%
\pgfsetrectcap%
\pgfsetmiterjoin%
\pgfsetlinewidth{0.000000pt}%
\definecolor{currentstroke}{rgb}{1.000000,1.000000,1.000000}%
\pgfsetstrokecolor{currentstroke}%
\pgfsetdash{}{0pt}%
\pgfpathmoveto{\pgfqpoint{0.423750in}{2.640000in}}%
\pgfpathlineto{\pgfqpoint{3.051000in}{2.640000in}}%
\pgfusepath{}%
\end{pgfscope}%
\begin{pgfscope}%
\definecolor{textcolor}{rgb}{0.150000,0.150000,0.150000}%
\pgfsetstrokecolor{textcolor}%
\pgfsetfillcolor{textcolor}%
\pgftext[x=1.737375in,y=2.723333in,,base]{\color{textcolor}\rmfamily\fontsize{8.000000}{9.600000}\selectfont Branin}%
\end{pgfscope}%
\begin{pgfscope}%
\pgfsetroundcap%
\pgfsetroundjoin%
\pgfsetlinewidth{1.505625pt}%
\definecolor{currentstroke}{rgb}{0.121569,0.466667,0.705882}%
\pgfsetstrokecolor{currentstroke}%
\pgfsetdash{}{0pt}%
\pgfpathmoveto{\pgfqpoint{0.523750in}{1.189344in}}%
\pgfpathlineto{\pgfqpoint{0.745972in}{1.189344in}}%
\pgfusepath{stroke}%
\end{pgfscope}%
\begin{pgfscope}%
\definecolor{textcolor}{rgb}{0.150000,0.150000,0.150000}%
\pgfsetstrokecolor{textcolor}%
\pgfsetfillcolor{textcolor}%
\pgftext[x=0.834861in,y=1.150455in,left,base]{\color{textcolor}\rmfamily\fontsize{8.000000}{9.600000}\selectfont random}%
\end{pgfscope}%
\begin{pgfscope}%
\pgfsetroundcap%
\pgfsetroundjoin%
\pgfsetlinewidth{1.505625pt}%
\definecolor{currentstroke}{rgb}{1.000000,0.498039,0.054902}%
\pgfsetstrokecolor{currentstroke}%
\pgfsetdash{}{0pt}%
\pgfpathmoveto{\pgfqpoint{0.523750in}{1.026258in}}%
\pgfpathlineto{\pgfqpoint{0.745972in}{1.026258in}}%
\pgfusepath{stroke}%
\end{pgfscope}%
\begin{pgfscope}%
\definecolor{textcolor}{rgb}{0.150000,0.150000,0.150000}%
\pgfsetstrokecolor{textcolor}%
\pgfsetfillcolor{textcolor}%
\pgftext[x=0.834861in,y=0.987369in,left,base]{\color{textcolor}\rmfamily\fontsize{8.000000}{9.600000}\selectfont DNGO fixed}%
\end{pgfscope}%
\begin{pgfscope}%
\pgfsetroundcap%
\pgfsetroundjoin%
\pgfsetlinewidth{1.505625pt}%
\definecolor{currentstroke}{rgb}{0.172549,0.627451,0.172549}%
\pgfsetstrokecolor{currentstroke}%
\pgfsetdash{}{0pt}%
\pgfpathmoveto{\pgfqpoint{0.523750in}{0.863172in}}%
\pgfpathlineto{\pgfqpoint{0.745972in}{0.863172in}}%
\pgfusepath{stroke}%
\end{pgfscope}%
\begin{pgfscope}%
\definecolor{textcolor}{rgb}{0.150000,0.150000,0.150000}%
\pgfsetstrokecolor{textcolor}%
\pgfsetfillcolor{textcolor}%
\pgftext[x=0.834861in,y=0.824283in,left,base]{\color{textcolor}\rmfamily\fontsize{8.000000}{9.600000}\selectfont DNGO retrain}%
\end{pgfscope}%
\begin{pgfscope}%
\pgfsetroundcap%
\pgfsetroundjoin%
\pgfsetlinewidth{1.505625pt}%
\definecolor{currentstroke}{rgb}{0.839216,0.152941,0.156863}%
\pgfsetstrokecolor{currentstroke}%
\pgfsetdash{}{0pt}%
\pgfpathmoveto{\pgfqpoint{0.523750in}{0.700087in}}%
\pgfpathlineto{\pgfqpoint{0.745972in}{0.700087in}}%
\pgfusepath{stroke}%
\end{pgfscope}%
\begin{pgfscope}%
\definecolor{textcolor}{rgb}{0.150000,0.150000,0.150000}%
\pgfsetstrokecolor{textcolor}%
\pgfsetfillcolor{textcolor}%
\pgftext[x=0.834861in,y=0.661198in,left,base]{\color{textcolor}\rmfamily\fontsize{8.000000}{9.600000}\selectfont DNGO retrain-reset}%
\end{pgfscope}%
\begin{pgfscope}%
\pgfsetroundcap%
\pgfsetroundjoin%
\pgfsetlinewidth{1.505625pt}%
\definecolor{currentstroke}{rgb}{0.580392,0.403922,0.741176}%
\pgfsetstrokecolor{currentstroke}%
\pgfsetdash{}{0pt}%
\pgfpathmoveto{\pgfqpoint{0.523750in}{0.537001in}}%
\pgfpathlineto{\pgfqpoint{0.745972in}{0.537001in}}%
\pgfusepath{stroke}%
\end{pgfscope}%
\begin{pgfscope}%
\definecolor{textcolor}{rgb}{0.150000,0.150000,0.150000}%
\pgfsetstrokecolor{textcolor}%
\pgfsetfillcolor{textcolor}%
\pgftext[x=0.834861in,y=0.498112in,left,base]{\color{textcolor}\rmfamily\fontsize{8.000000}{9.600000}\selectfont GP}%
\end{pgfscope}%
\end{pgfpicture}%
\makeatother%
\endgroup%

        \end{minipage}\qquad
        \begin{minipage}{0.47\linewidth}
            \centering
        %% Creator: Matplotlib, PGF backend
%%
%% To include the figure in your LaTeX document, write
%%   \input{<filename>.pgf}
%%
%% Make sure the required packages are loaded in your preamble
%%   \usepackage{pgf}
%%
%% Figures using additional raster images can only be included by \input if
%% they are in the same directory as the main LaTeX file. For loading figures
%% from other directories you can use the `import` package
%%   \usepackage{import}
%% and then include the figures with
%%   \import{<path to file>}{<filename>.pgf}
%%
%% Matplotlib used the following preamble
%%   \usepackage{gensymb}
%%   \usepackage{fontspec}
%%   \setmainfont{DejaVu Serif}
%%   \setsansfont{Arial}
%%   \setmonofont{DejaVu Sans Mono}
%%
\begingroup%
\makeatletter%
\begin{pgfpicture}%
\pgfpathrectangle{\pgfpointorigin}{\pgfqpoint{3.390000in}{3.000000in}}%
\pgfusepath{use as bounding box, clip}%
\begin{pgfscope}%
\pgfsetbuttcap%
\pgfsetmiterjoin%
\definecolor{currentfill}{rgb}{1.000000,1.000000,1.000000}%
\pgfsetfillcolor{currentfill}%
\pgfsetlinewidth{0.000000pt}%
\definecolor{currentstroke}{rgb}{1.000000,1.000000,1.000000}%
\pgfsetstrokecolor{currentstroke}%
\pgfsetdash{}{0pt}%
\pgfpathmoveto{\pgfqpoint{0.000000in}{0.000000in}}%
\pgfpathlineto{\pgfqpoint{3.390000in}{0.000000in}}%
\pgfpathlineto{\pgfqpoint{3.390000in}{3.000000in}}%
\pgfpathlineto{\pgfqpoint{0.000000in}{3.000000in}}%
\pgfpathclose%
\pgfusepath{fill}%
\end{pgfscope}%
\begin{pgfscope}%
\pgfsetbuttcap%
\pgfsetmiterjoin%
\definecolor{currentfill}{rgb}{0.917647,0.917647,0.949020}%
\pgfsetfillcolor{currentfill}%
\pgfsetlinewidth{0.000000pt}%
\definecolor{currentstroke}{rgb}{0.000000,0.000000,0.000000}%
\pgfsetstrokecolor{currentstroke}%
\pgfsetstrokeopacity{0.000000}%
\pgfsetdash{}{0pt}%
\pgfpathmoveto{\pgfqpoint{0.423750in}{0.375000in}}%
\pgfpathlineto{\pgfqpoint{3.051000in}{0.375000in}}%
\pgfpathlineto{\pgfqpoint{3.051000in}{2.640000in}}%
\pgfpathlineto{\pgfqpoint{0.423750in}{2.640000in}}%
\pgfpathclose%
\pgfusepath{fill}%
\end{pgfscope}%
\begin{pgfscope}%
\pgfpathrectangle{\pgfqpoint{0.423750in}{0.375000in}}{\pgfqpoint{2.627250in}{2.265000in}}%
\pgfusepath{clip}%
\pgfsetroundcap%
\pgfsetroundjoin%
\pgfsetlinewidth{0.803000pt}%
\definecolor{currentstroke}{rgb}{1.000000,1.000000,1.000000}%
\pgfsetstrokecolor{currentstroke}%
\pgfsetdash{}{0pt}%
\pgfpathmoveto{\pgfqpoint{0.543170in}{0.375000in}}%
\pgfpathlineto{\pgfqpoint{0.543170in}{2.640000in}}%
\pgfusepath{stroke}%
\end{pgfscope}%
\begin{pgfscope}%
\definecolor{textcolor}{rgb}{0.150000,0.150000,0.150000}%
\pgfsetstrokecolor{textcolor}%
\pgfsetfillcolor{textcolor}%
\pgftext[x=0.543170in,y=0.326389in,,top]{\color{textcolor}\rmfamily\fontsize{8.000000}{9.600000}\selectfont \(\displaystyle 0\)}%
\end{pgfscope}%
\begin{pgfscope}%
\pgfpathrectangle{\pgfqpoint{0.423750in}{0.375000in}}{\pgfqpoint{2.627250in}{2.265000in}}%
\pgfusepath{clip}%
\pgfsetroundcap%
\pgfsetroundjoin%
\pgfsetlinewidth{0.803000pt}%
\definecolor{currentstroke}{rgb}{1.000000,1.000000,1.000000}%
\pgfsetstrokecolor{currentstroke}%
\pgfsetdash{}{0pt}%
\pgfpathmoveto{\pgfqpoint{1.249800in}{0.375000in}}%
\pgfpathlineto{\pgfqpoint{1.249800in}{2.640000in}}%
\pgfusepath{stroke}%
\end{pgfscope}%
\begin{pgfscope}%
\definecolor{textcolor}{rgb}{0.150000,0.150000,0.150000}%
\pgfsetstrokecolor{textcolor}%
\pgfsetfillcolor{textcolor}%
\pgftext[x=1.249800in,y=0.326389in,,top]{\color{textcolor}\rmfamily\fontsize{8.000000}{9.600000}\selectfont \(\displaystyle 50\)}%
\end{pgfscope}%
\begin{pgfscope}%
\pgfpathrectangle{\pgfqpoint{0.423750in}{0.375000in}}{\pgfqpoint{2.627250in}{2.265000in}}%
\pgfusepath{clip}%
\pgfsetroundcap%
\pgfsetroundjoin%
\pgfsetlinewidth{0.803000pt}%
\definecolor{currentstroke}{rgb}{1.000000,1.000000,1.000000}%
\pgfsetstrokecolor{currentstroke}%
\pgfsetdash{}{0pt}%
\pgfpathmoveto{\pgfqpoint{1.956430in}{0.375000in}}%
\pgfpathlineto{\pgfqpoint{1.956430in}{2.640000in}}%
\pgfusepath{stroke}%
\end{pgfscope}%
\begin{pgfscope}%
\definecolor{textcolor}{rgb}{0.150000,0.150000,0.150000}%
\pgfsetstrokecolor{textcolor}%
\pgfsetfillcolor{textcolor}%
\pgftext[x=1.956430in,y=0.326389in,,top]{\color{textcolor}\rmfamily\fontsize{8.000000}{9.600000}\selectfont \(\displaystyle 100\)}%
\end{pgfscope}%
\begin{pgfscope}%
\pgfpathrectangle{\pgfqpoint{0.423750in}{0.375000in}}{\pgfqpoint{2.627250in}{2.265000in}}%
\pgfusepath{clip}%
\pgfsetroundcap%
\pgfsetroundjoin%
\pgfsetlinewidth{0.803000pt}%
\definecolor{currentstroke}{rgb}{1.000000,1.000000,1.000000}%
\pgfsetstrokecolor{currentstroke}%
\pgfsetdash{}{0pt}%
\pgfpathmoveto{\pgfqpoint{2.663060in}{0.375000in}}%
\pgfpathlineto{\pgfqpoint{2.663060in}{2.640000in}}%
\pgfusepath{stroke}%
\end{pgfscope}%
\begin{pgfscope}%
\definecolor{textcolor}{rgb}{0.150000,0.150000,0.150000}%
\pgfsetstrokecolor{textcolor}%
\pgfsetfillcolor{textcolor}%
\pgftext[x=2.663060in,y=0.326389in,,top]{\color{textcolor}\rmfamily\fontsize{8.000000}{9.600000}\selectfont \(\displaystyle 150\)}%
\end{pgfscope}%
\begin{pgfscope}%
\definecolor{textcolor}{rgb}{0.150000,0.150000,0.150000}%
\pgfsetstrokecolor{textcolor}%
\pgfsetfillcolor{textcolor}%
\pgftext[x=1.737375in,y=0.163303in,,top]{\color{textcolor}\rmfamily\fontsize{8.000000}{9.600000}\selectfont Step}%
\end{pgfscope}%
\begin{pgfscope}%
\pgfpathrectangle{\pgfqpoint{0.423750in}{0.375000in}}{\pgfqpoint{2.627250in}{2.265000in}}%
\pgfusepath{clip}%
\pgfsetroundcap%
\pgfsetroundjoin%
\pgfsetlinewidth{0.803000pt}%
\definecolor{currentstroke}{rgb}{1.000000,1.000000,1.000000}%
\pgfsetstrokecolor{currentstroke}%
\pgfsetdash{}{0pt}%
\pgfpathmoveto{\pgfqpoint{0.423750in}{0.665080in}}%
\pgfpathlineto{\pgfqpoint{3.051000in}{0.665080in}}%
\pgfusepath{stroke}%
\end{pgfscope}%
\begin{pgfscope}%
\definecolor{textcolor}{rgb}{0.150000,0.150000,0.150000}%
\pgfsetstrokecolor{textcolor}%
\pgfsetfillcolor{textcolor}%
\pgftext[x=0.118966in,y=0.622871in,left,base]{\color{textcolor}\rmfamily\fontsize{8.000000}{9.600000}\selectfont \(\displaystyle 10^{-5}\)}%
\end{pgfscope}%
\begin{pgfscope}%
\pgfpathrectangle{\pgfqpoint{0.423750in}{0.375000in}}{\pgfqpoint{2.627250in}{2.265000in}}%
\pgfusepath{clip}%
\pgfsetroundcap%
\pgfsetroundjoin%
\pgfsetlinewidth{0.803000pt}%
\definecolor{currentstroke}{rgb}{1.000000,1.000000,1.000000}%
\pgfsetstrokecolor{currentstroke}%
\pgfsetdash{}{0pt}%
\pgfpathmoveto{\pgfqpoint{0.423750in}{1.001249in}}%
\pgfpathlineto{\pgfqpoint{3.051000in}{1.001249in}}%
\pgfusepath{stroke}%
\end{pgfscope}%
\begin{pgfscope}%
\definecolor{textcolor}{rgb}{0.150000,0.150000,0.150000}%
\pgfsetstrokecolor{textcolor}%
\pgfsetfillcolor{textcolor}%
\pgftext[x=0.118966in,y=0.959039in,left,base]{\color{textcolor}\rmfamily\fontsize{8.000000}{9.600000}\selectfont \(\displaystyle 10^{-4}\)}%
\end{pgfscope}%
\begin{pgfscope}%
\pgfpathrectangle{\pgfqpoint{0.423750in}{0.375000in}}{\pgfqpoint{2.627250in}{2.265000in}}%
\pgfusepath{clip}%
\pgfsetroundcap%
\pgfsetroundjoin%
\pgfsetlinewidth{0.803000pt}%
\definecolor{currentstroke}{rgb}{1.000000,1.000000,1.000000}%
\pgfsetstrokecolor{currentstroke}%
\pgfsetdash{}{0pt}%
\pgfpathmoveto{\pgfqpoint{0.423750in}{1.337417in}}%
\pgfpathlineto{\pgfqpoint{3.051000in}{1.337417in}}%
\pgfusepath{stroke}%
\end{pgfscope}%
\begin{pgfscope}%
\definecolor{textcolor}{rgb}{0.150000,0.150000,0.150000}%
\pgfsetstrokecolor{textcolor}%
\pgfsetfillcolor{textcolor}%
\pgftext[x=0.118966in,y=1.295208in,left,base]{\color{textcolor}\rmfamily\fontsize{8.000000}{9.600000}\selectfont \(\displaystyle 10^{-3}\)}%
\end{pgfscope}%
\begin{pgfscope}%
\pgfpathrectangle{\pgfqpoint{0.423750in}{0.375000in}}{\pgfqpoint{2.627250in}{2.265000in}}%
\pgfusepath{clip}%
\pgfsetroundcap%
\pgfsetroundjoin%
\pgfsetlinewidth{0.803000pt}%
\definecolor{currentstroke}{rgb}{1.000000,1.000000,1.000000}%
\pgfsetstrokecolor{currentstroke}%
\pgfsetdash{}{0pt}%
\pgfpathmoveto{\pgfqpoint{0.423750in}{1.673586in}}%
\pgfpathlineto{\pgfqpoint{3.051000in}{1.673586in}}%
\pgfusepath{stroke}%
\end{pgfscope}%
\begin{pgfscope}%
\definecolor{textcolor}{rgb}{0.150000,0.150000,0.150000}%
\pgfsetstrokecolor{textcolor}%
\pgfsetfillcolor{textcolor}%
\pgftext[x=0.118966in,y=1.631377in,left,base]{\color{textcolor}\rmfamily\fontsize{8.000000}{9.600000}\selectfont \(\displaystyle 10^{-2}\)}%
\end{pgfscope}%
\begin{pgfscope}%
\pgfpathrectangle{\pgfqpoint{0.423750in}{0.375000in}}{\pgfqpoint{2.627250in}{2.265000in}}%
\pgfusepath{clip}%
\pgfsetroundcap%
\pgfsetroundjoin%
\pgfsetlinewidth{0.803000pt}%
\definecolor{currentstroke}{rgb}{1.000000,1.000000,1.000000}%
\pgfsetstrokecolor{currentstroke}%
\pgfsetdash{}{0pt}%
\pgfpathmoveto{\pgfqpoint{0.423750in}{2.009755in}}%
\pgfpathlineto{\pgfqpoint{3.051000in}{2.009755in}}%
\pgfusepath{stroke}%
\end{pgfscope}%
\begin{pgfscope}%
\definecolor{textcolor}{rgb}{0.150000,0.150000,0.150000}%
\pgfsetstrokecolor{textcolor}%
\pgfsetfillcolor{textcolor}%
\pgftext[x=0.118966in,y=1.967545in,left,base]{\color{textcolor}\rmfamily\fontsize{8.000000}{9.600000}\selectfont \(\displaystyle 10^{-1}\)}%
\end{pgfscope}%
\begin{pgfscope}%
\pgfpathrectangle{\pgfqpoint{0.423750in}{0.375000in}}{\pgfqpoint{2.627250in}{2.265000in}}%
\pgfusepath{clip}%
\pgfsetroundcap%
\pgfsetroundjoin%
\pgfsetlinewidth{0.803000pt}%
\definecolor{currentstroke}{rgb}{1.000000,1.000000,1.000000}%
\pgfsetstrokecolor{currentstroke}%
\pgfsetdash{}{0pt}%
\pgfpathmoveto{\pgfqpoint{0.423750in}{2.345923in}}%
\pgfpathlineto{\pgfqpoint{3.051000in}{2.345923in}}%
\pgfusepath{stroke}%
\end{pgfscope}%
\begin{pgfscope}%
\definecolor{textcolor}{rgb}{0.150000,0.150000,0.150000}%
\pgfsetstrokecolor{textcolor}%
\pgfsetfillcolor{textcolor}%
\pgftext[x=0.199212in,y=2.303714in,left,base]{\color{textcolor}\rmfamily\fontsize{8.000000}{9.600000}\selectfont \(\displaystyle 10^{0}\)}%
\end{pgfscope}%
\begin{pgfscope}%
\definecolor{textcolor}{rgb}{0.150000,0.150000,0.150000}%
\pgfsetstrokecolor{textcolor}%
\pgfsetfillcolor{textcolor}%
\pgftext[x=0.063410in,y=1.507500in,,bottom,rotate=90.000000]{\color{textcolor}\rmfamily\fontsize{8.000000}{9.600000}\selectfont Simple Regret}%
\end{pgfscope}%
\begin{pgfscope}%
\pgfpathrectangle{\pgfqpoint{0.423750in}{0.375000in}}{\pgfqpoint{2.627250in}{2.265000in}}%
\pgfusepath{clip}%
\pgfsetbuttcap%
\pgfsetroundjoin%
\definecolor{currentfill}{rgb}{0.121569,0.466667,0.705882}%
\pgfsetfillcolor{currentfill}%
\pgfsetfillopacity{0.200000}%
\pgfsetlinewidth{0.000000pt}%
\definecolor{currentstroke}{rgb}{0.000000,0.000000,0.000000}%
\pgfsetstrokecolor{currentstroke}%
\pgfsetdash{}{0pt}%
\pgfpathmoveto{\pgfqpoint{0.543170in}{2.480261in}}%
\pgfpathlineto{\pgfqpoint{0.543170in}{2.533735in}}%
\pgfpathlineto{\pgfqpoint{0.557303in}{2.488137in}}%
\pgfpathlineto{\pgfqpoint{0.571436in}{2.474291in}}%
\pgfpathlineto{\pgfqpoint{0.585568in}{2.441598in}}%
\pgfpathlineto{\pgfqpoint{0.599701in}{2.416197in}}%
\pgfpathlineto{\pgfqpoint{0.613833in}{2.380002in}}%
\pgfpathlineto{\pgfqpoint{0.627966in}{2.355426in}}%
\pgfpathlineto{\pgfqpoint{0.642099in}{2.355426in}}%
\pgfpathlineto{\pgfqpoint{0.656231in}{2.346633in}}%
\pgfpathlineto{\pgfqpoint{0.670364in}{2.346633in}}%
\pgfpathlineto{\pgfqpoint{0.684496in}{2.302577in}}%
\pgfpathlineto{\pgfqpoint{0.698629in}{2.302577in}}%
\pgfpathlineto{\pgfqpoint{0.712762in}{2.302577in}}%
\pgfpathlineto{\pgfqpoint{0.726894in}{2.302577in}}%
\pgfpathlineto{\pgfqpoint{0.741027in}{2.299157in}}%
\pgfpathlineto{\pgfqpoint{0.755159in}{2.299157in}}%
\pgfpathlineto{\pgfqpoint{0.769292in}{2.299157in}}%
\pgfpathlineto{\pgfqpoint{0.783425in}{2.298975in}}%
\pgfpathlineto{\pgfqpoint{0.797557in}{2.298975in}}%
\pgfpathlineto{\pgfqpoint{0.811690in}{2.289168in}}%
\pgfpathlineto{\pgfqpoint{0.825822in}{2.289168in}}%
\pgfpathlineto{\pgfqpoint{0.839955in}{2.284703in}}%
\pgfpathlineto{\pgfqpoint{0.854088in}{2.280813in}}%
\pgfpathlineto{\pgfqpoint{0.868220in}{2.280813in}}%
\pgfpathlineto{\pgfqpoint{0.882353in}{2.280813in}}%
\pgfpathlineto{\pgfqpoint{0.896485in}{2.280813in}}%
\pgfpathlineto{\pgfqpoint{0.910618in}{2.280813in}}%
\pgfpathlineto{\pgfqpoint{0.924751in}{2.280813in}}%
\pgfpathlineto{\pgfqpoint{0.938883in}{2.280813in}}%
\pgfpathlineto{\pgfqpoint{0.953016in}{2.273443in}}%
\pgfpathlineto{\pgfqpoint{0.967148in}{2.273443in}}%
\pgfpathlineto{\pgfqpoint{0.981281in}{2.273443in}}%
\pgfpathlineto{\pgfqpoint{0.995414in}{2.273443in}}%
\pgfpathlineto{\pgfqpoint{1.009546in}{2.273443in}}%
\pgfpathlineto{\pgfqpoint{1.023679in}{2.273443in}}%
\pgfpathlineto{\pgfqpoint{1.037811in}{2.273443in}}%
\pgfpathlineto{\pgfqpoint{1.051944in}{2.273443in}}%
\pgfpathlineto{\pgfqpoint{1.066077in}{2.251590in}}%
\pgfpathlineto{\pgfqpoint{1.080209in}{2.251590in}}%
\pgfpathlineto{\pgfqpoint{1.094342in}{2.251590in}}%
\pgfpathlineto{\pgfqpoint{1.108474in}{2.251590in}}%
\pgfpathlineto{\pgfqpoint{1.122607in}{2.251590in}}%
\pgfpathlineto{\pgfqpoint{1.136740in}{2.251590in}}%
\pgfpathlineto{\pgfqpoint{1.150872in}{2.251590in}}%
\pgfpathlineto{\pgfqpoint{1.165005in}{2.251590in}}%
\pgfpathlineto{\pgfqpoint{1.179137in}{2.251590in}}%
\pgfpathlineto{\pgfqpoint{1.193270in}{2.251590in}}%
\pgfpathlineto{\pgfqpoint{1.207403in}{2.251590in}}%
\pgfpathlineto{\pgfqpoint{1.221535in}{2.251590in}}%
\pgfpathlineto{\pgfqpoint{1.235668in}{2.251590in}}%
\pgfpathlineto{\pgfqpoint{1.249800in}{2.251590in}}%
\pgfpathlineto{\pgfqpoint{1.263933in}{2.251590in}}%
\pgfpathlineto{\pgfqpoint{1.278066in}{2.249092in}}%
\pgfpathlineto{\pgfqpoint{1.292198in}{2.248276in}}%
\pgfpathlineto{\pgfqpoint{1.306331in}{2.248276in}}%
\pgfpathlineto{\pgfqpoint{1.320463in}{2.232402in}}%
\pgfpathlineto{\pgfqpoint{1.334596in}{2.227217in}}%
\pgfpathlineto{\pgfqpoint{1.348729in}{2.227217in}}%
\pgfpathlineto{\pgfqpoint{1.362861in}{2.227217in}}%
\pgfpathlineto{\pgfqpoint{1.376994in}{2.227217in}}%
\pgfpathlineto{\pgfqpoint{1.391126in}{2.227217in}}%
\pgfpathlineto{\pgfqpoint{1.405259in}{2.227217in}}%
\pgfpathlineto{\pgfqpoint{1.419392in}{2.227217in}}%
\pgfpathlineto{\pgfqpoint{1.433524in}{2.227217in}}%
\pgfpathlineto{\pgfqpoint{1.447657in}{2.227217in}}%
\pgfpathlineto{\pgfqpoint{1.461789in}{2.227217in}}%
\pgfpathlineto{\pgfqpoint{1.475922in}{2.227217in}}%
\pgfpathlineto{\pgfqpoint{1.490055in}{2.227217in}}%
\pgfpathlineto{\pgfqpoint{1.504187in}{2.227217in}}%
\pgfpathlineto{\pgfqpoint{1.518320in}{2.227217in}}%
\pgfpathlineto{\pgfqpoint{1.532452in}{2.227217in}}%
\pgfpathlineto{\pgfqpoint{1.546585in}{2.227217in}}%
\pgfpathlineto{\pgfqpoint{1.560718in}{2.227217in}}%
\pgfpathlineto{\pgfqpoint{1.574850in}{2.227217in}}%
\pgfpathlineto{\pgfqpoint{1.588983in}{2.227217in}}%
\pgfpathlineto{\pgfqpoint{1.603115in}{2.227217in}}%
\pgfpathlineto{\pgfqpoint{1.617248in}{2.227217in}}%
\pgfpathlineto{\pgfqpoint{1.631381in}{2.227217in}}%
\pgfpathlineto{\pgfqpoint{1.645513in}{2.227217in}}%
\pgfpathlineto{\pgfqpoint{1.659646in}{2.227217in}}%
\pgfpathlineto{\pgfqpoint{1.673778in}{2.227217in}}%
\pgfpathlineto{\pgfqpoint{1.687911in}{2.227217in}}%
\pgfpathlineto{\pgfqpoint{1.702044in}{2.227217in}}%
\pgfpathlineto{\pgfqpoint{1.716176in}{2.227217in}}%
\pgfpathlineto{\pgfqpoint{1.730309in}{2.227217in}}%
\pgfpathlineto{\pgfqpoint{1.744441in}{2.227217in}}%
\pgfpathlineto{\pgfqpoint{1.758574in}{2.227217in}}%
\pgfpathlineto{\pgfqpoint{1.772706in}{2.227217in}}%
\pgfpathlineto{\pgfqpoint{1.786839in}{2.227217in}}%
\pgfpathlineto{\pgfqpoint{1.800972in}{2.227217in}}%
\pgfpathlineto{\pgfqpoint{1.815104in}{2.227217in}}%
\pgfpathlineto{\pgfqpoint{1.829237in}{2.227217in}}%
\pgfpathlineto{\pgfqpoint{1.843369in}{2.227217in}}%
\pgfpathlineto{\pgfqpoint{1.857502in}{2.219342in}}%
\pgfpathlineto{\pgfqpoint{1.871635in}{2.217215in}}%
\pgfpathlineto{\pgfqpoint{1.885767in}{2.217215in}}%
\pgfpathlineto{\pgfqpoint{1.899900in}{2.217215in}}%
\pgfpathlineto{\pgfqpoint{1.914032in}{2.217215in}}%
\pgfpathlineto{\pgfqpoint{1.928165in}{2.217215in}}%
\pgfpathlineto{\pgfqpoint{1.942298in}{2.217215in}}%
\pgfpathlineto{\pgfqpoint{1.956430in}{2.217215in}}%
\pgfpathlineto{\pgfqpoint{1.970563in}{2.217215in}}%
\pgfpathlineto{\pgfqpoint{1.984695in}{2.217215in}}%
\pgfpathlineto{\pgfqpoint{1.998828in}{2.217215in}}%
\pgfpathlineto{\pgfqpoint{2.012961in}{2.217215in}}%
\pgfpathlineto{\pgfqpoint{2.027093in}{2.217215in}}%
\pgfpathlineto{\pgfqpoint{2.041226in}{2.210617in}}%
\pgfpathlineto{\pgfqpoint{2.055358in}{2.210617in}}%
\pgfpathlineto{\pgfqpoint{2.069491in}{2.210617in}}%
\pgfpathlineto{\pgfqpoint{2.083624in}{2.210617in}}%
\pgfpathlineto{\pgfqpoint{2.097756in}{2.210617in}}%
\pgfpathlineto{\pgfqpoint{2.111889in}{2.210617in}}%
\pgfpathlineto{\pgfqpoint{2.126021in}{2.210617in}}%
\pgfpathlineto{\pgfqpoint{2.140154in}{2.210617in}}%
\pgfpathlineto{\pgfqpoint{2.154287in}{2.210617in}}%
\pgfpathlineto{\pgfqpoint{2.168419in}{2.210617in}}%
\pgfpathlineto{\pgfqpoint{2.182552in}{2.210617in}}%
\pgfpathlineto{\pgfqpoint{2.196684in}{2.210617in}}%
\pgfpathlineto{\pgfqpoint{2.210817in}{2.210617in}}%
\pgfpathlineto{\pgfqpoint{2.224950in}{2.210617in}}%
\pgfpathlineto{\pgfqpoint{2.239082in}{2.204078in}}%
\pgfpathlineto{\pgfqpoint{2.253215in}{2.204078in}}%
\pgfpathlineto{\pgfqpoint{2.267347in}{2.204078in}}%
\pgfpathlineto{\pgfqpoint{2.281480in}{2.196950in}}%
\pgfpathlineto{\pgfqpoint{2.295613in}{2.196950in}}%
\pgfpathlineto{\pgfqpoint{2.309745in}{2.196950in}}%
\pgfpathlineto{\pgfqpoint{2.323878in}{2.196950in}}%
\pgfpathlineto{\pgfqpoint{2.338010in}{2.196950in}}%
\pgfpathlineto{\pgfqpoint{2.352143in}{2.196950in}}%
\pgfpathlineto{\pgfqpoint{2.366276in}{2.196950in}}%
\pgfpathlineto{\pgfqpoint{2.380408in}{2.196950in}}%
\pgfpathlineto{\pgfqpoint{2.394541in}{2.186083in}}%
\pgfpathlineto{\pgfqpoint{2.408673in}{2.186083in}}%
\pgfpathlineto{\pgfqpoint{2.422806in}{2.186083in}}%
\pgfpathlineto{\pgfqpoint{2.436939in}{2.186083in}}%
\pgfpathlineto{\pgfqpoint{2.451071in}{2.186083in}}%
\pgfpathlineto{\pgfqpoint{2.465204in}{2.186083in}}%
\pgfpathlineto{\pgfqpoint{2.479336in}{2.186083in}}%
\pgfpathlineto{\pgfqpoint{2.493469in}{2.186083in}}%
\pgfpathlineto{\pgfqpoint{2.507602in}{2.172643in}}%
\pgfpathlineto{\pgfqpoint{2.521734in}{2.172643in}}%
\pgfpathlineto{\pgfqpoint{2.535867in}{2.172643in}}%
\pgfpathlineto{\pgfqpoint{2.549999in}{2.172643in}}%
\pgfpathlineto{\pgfqpoint{2.564132in}{2.172643in}}%
\pgfpathlineto{\pgfqpoint{2.578265in}{2.172643in}}%
\pgfpathlineto{\pgfqpoint{2.592397in}{2.172643in}}%
\pgfpathlineto{\pgfqpoint{2.606530in}{2.172643in}}%
\pgfpathlineto{\pgfqpoint{2.620662in}{2.172643in}}%
\pgfpathlineto{\pgfqpoint{2.634795in}{2.171341in}}%
\pgfpathlineto{\pgfqpoint{2.648928in}{2.171341in}}%
\pgfpathlineto{\pgfqpoint{2.663060in}{2.171341in}}%
\pgfpathlineto{\pgfqpoint{2.677193in}{2.171341in}}%
\pgfpathlineto{\pgfqpoint{2.691325in}{2.171341in}}%
\pgfpathlineto{\pgfqpoint{2.705458in}{2.171341in}}%
\pgfpathlineto{\pgfqpoint{2.719591in}{2.164741in}}%
\pgfpathlineto{\pgfqpoint{2.733723in}{2.164741in}}%
\pgfpathlineto{\pgfqpoint{2.747856in}{2.164741in}}%
\pgfpathlineto{\pgfqpoint{2.761988in}{2.164741in}}%
\pgfpathlineto{\pgfqpoint{2.776121in}{2.164741in}}%
\pgfpathlineto{\pgfqpoint{2.790254in}{2.158825in}}%
\pgfpathlineto{\pgfqpoint{2.804386in}{2.158825in}}%
\pgfpathlineto{\pgfqpoint{2.818519in}{2.158825in}}%
\pgfpathlineto{\pgfqpoint{2.832651in}{2.158825in}}%
\pgfpathlineto{\pgfqpoint{2.846784in}{2.158825in}}%
\pgfpathlineto{\pgfqpoint{2.860917in}{2.158825in}}%
\pgfpathlineto{\pgfqpoint{2.875049in}{2.158825in}}%
\pgfpathlineto{\pgfqpoint{2.889182in}{2.158825in}}%
\pgfpathlineto{\pgfqpoint{2.903314in}{2.158825in}}%
\pgfpathlineto{\pgfqpoint{2.917447in}{2.158825in}}%
\pgfpathlineto{\pgfqpoint{2.931580in}{2.139051in}}%
\pgfpathlineto{\pgfqpoint{2.931580in}{2.009514in}}%
\pgfpathlineto{\pgfqpoint{2.931580in}{2.009514in}}%
\pgfpathlineto{\pgfqpoint{2.917447in}{2.003069in}}%
\pgfpathlineto{\pgfqpoint{2.903314in}{2.003069in}}%
\pgfpathlineto{\pgfqpoint{2.889182in}{2.003069in}}%
\pgfpathlineto{\pgfqpoint{2.875049in}{2.003069in}}%
\pgfpathlineto{\pgfqpoint{2.860917in}{2.003069in}}%
\pgfpathlineto{\pgfqpoint{2.846784in}{2.003069in}}%
\pgfpathlineto{\pgfqpoint{2.832651in}{2.003069in}}%
\pgfpathlineto{\pgfqpoint{2.818519in}{2.003069in}}%
\pgfpathlineto{\pgfqpoint{2.804386in}{2.003069in}}%
\pgfpathlineto{\pgfqpoint{2.790254in}{2.003069in}}%
\pgfpathlineto{\pgfqpoint{2.776121in}{2.016765in}}%
\pgfpathlineto{\pgfqpoint{2.761988in}{2.016765in}}%
\pgfpathlineto{\pgfqpoint{2.747856in}{2.016765in}}%
\pgfpathlineto{\pgfqpoint{2.733723in}{2.016765in}}%
\pgfpathlineto{\pgfqpoint{2.719591in}{2.016765in}}%
\pgfpathlineto{\pgfqpoint{2.705458in}{2.033354in}}%
\pgfpathlineto{\pgfqpoint{2.691325in}{2.033354in}}%
\pgfpathlineto{\pgfqpoint{2.677193in}{2.033354in}}%
\pgfpathlineto{\pgfqpoint{2.663060in}{2.033354in}}%
\pgfpathlineto{\pgfqpoint{2.648928in}{2.033354in}}%
\pgfpathlineto{\pgfqpoint{2.634795in}{2.033354in}}%
\pgfpathlineto{\pgfqpoint{2.620662in}{2.037692in}}%
\pgfpathlineto{\pgfqpoint{2.606530in}{2.037692in}}%
\pgfpathlineto{\pgfqpoint{2.592397in}{2.037692in}}%
\pgfpathlineto{\pgfqpoint{2.578265in}{2.037692in}}%
\pgfpathlineto{\pgfqpoint{2.564132in}{2.037692in}}%
\pgfpathlineto{\pgfqpoint{2.549999in}{2.037692in}}%
\pgfpathlineto{\pgfqpoint{2.535867in}{2.037692in}}%
\pgfpathlineto{\pgfqpoint{2.521734in}{2.037692in}}%
\pgfpathlineto{\pgfqpoint{2.507602in}{2.037692in}}%
\pgfpathlineto{\pgfqpoint{2.493469in}{2.076498in}}%
\pgfpathlineto{\pgfqpoint{2.479336in}{2.076498in}}%
\pgfpathlineto{\pgfqpoint{2.465204in}{2.076498in}}%
\pgfpathlineto{\pgfqpoint{2.451071in}{2.076498in}}%
\pgfpathlineto{\pgfqpoint{2.436939in}{2.076498in}}%
\pgfpathlineto{\pgfqpoint{2.422806in}{2.076498in}}%
\pgfpathlineto{\pgfqpoint{2.408673in}{2.076498in}}%
\pgfpathlineto{\pgfqpoint{2.394541in}{2.076498in}}%
\pgfpathlineto{\pgfqpoint{2.380408in}{2.096165in}}%
\pgfpathlineto{\pgfqpoint{2.366276in}{2.096165in}}%
\pgfpathlineto{\pgfqpoint{2.352143in}{2.096165in}}%
\pgfpathlineto{\pgfqpoint{2.338010in}{2.096165in}}%
\pgfpathlineto{\pgfqpoint{2.323878in}{2.096165in}}%
\pgfpathlineto{\pgfqpoint{2.309745in}{2.096165in}}%
\pgfpathlineto{\pgfqpoint{2.295613in}{2.096165in}}%
\pgfpathlineto{\pgfqpoint{2.281480in}{2.096165in}}%
\pgfpathlineto{\pgfqpoint{2.267347in}{2.100931in}}%
\pgfpathlineto{\pgfqpoint{2.253215in}{2.100931in}}%
\pgfpathlineto{\pgfqpoint{2.239082in}{2.100931in}}%
\pgfpathlineto{\pgfqpoint{2.224950in}{2.114554in}}%
\pgfpathlineto{\pgfqpoint{2.210817in}{2.114554in}}%
\pgfpathlineto{\pgfqpoint{2.196684in}{2.114554in}}%
\pgfpathlineto{\pgfqpoint{2.182552in}{2.114554in}}%
\pgfpathlineto{\pgfqpoint{2.168419in}{2.114554in}}%
\pgfpathlineto{\pgfqpoint{2.154287in}{2.114554in}}%
\pgfpathlineto{\pgfqpoint{2.140154in}{2.114554in}}%
\pgfpathlineto{\pgfqpoint{2.126021in}{2.114554in}}%
\pgfpathlineto{\pgfqpoint{2.111889in}{2.114554in}}%
\pgfpathlineto{\pgfqpoint{2.097756in}{2.114554in}}%
\pgfpathlineto{\pgfqpoint{2.083624in}{2.114554in}}%
\pgfpathlineto{\pgfqpoint{2.069491in}{2.114554in}}%
\pgfpathlineto{\pgfqpoint{2.055358in}{2.114554in}}%
\pgfpathlineto{\pgfqpoint{2.041226in}{2.114554in}}%
\pgfpathlineto{\pgfqpoint{2.027093in}{2.144074in}}%
\pgfpathlineto{\pgfqpoint{2.012961in}{2.144074in}}%
\pgfpathlineto{\pgfqpoint{1.998828in}{2.144074in}}%
\pgfpathlineto{\pgfqpoint{1.984695in}{2.144074in}}%
\pgfpathlineto{\pgfqpoint{1.970563in}{2.144074in}}%
\pgfpathlineto{\pgfqpoint{1.956430in}{2.144074in}}%
\pgfpathlineto{\pgfqpoint{1.942298in}{2.144074in}}%
\pgfpathlineto{\pgfqpoint{1.928165in}{2.144074in}}%
\pgfpathlineto{\pgfqpoint{1.914032in}{2.144074in}}%
\pgfpathlineto{\pgfqpoint{1.899900in}{2.144074in}}%
\pgfpathlineto{\pgfqpoint{1.885767in}{2.144074in}}%
\pgfpathlineto{\pgfqpoint{1.871635in}{2.144074in}}%
\pgfpathlineto{\pgfqpoint{1.857502in}{2.147706in}}%
\pgfpathlineto{\pgfqpoint{1.843369in}{2.147117in}}%
\pgfpathlineto{\pgfqpoint{1.829237in}{2.147117in}}%
\pgfpathlineto{\pgfqpoint{1.815104in}{2.147117in}}%
\pgfpathlineto{\pgfqpoint{1.800972in}{2.147117in}}%
\pgfpathlineto{\pgfqpoint{1.786839in}{2.147117in}}%
\pgfpathlineto{\pgfqpoint{1.772706in}{2.147117in}}%
\pgfpathlineto{\pgfqpoint{1.758574in}{2.147117in}}%
\pgfpathlineto{\pgfqpoint{1.744441in}{2.147117in}}%
\pgfpathlineto{\pgfqpoint{1.730309in}{2.147117in}}%
\pgfpathlineto{\pgfqpoint{1.716176in}{2.147117in}}%
\pgfpathlineto{\pgfqpoint{1.702044in}{2.147117in}}%
\pgfpathlineto{\pgfqpoint{1.687911in}{2.147117in}}%
\pgfpathlineto{\pgfqpoint{1.673778in}{2.147117in}}%
\pgfpathlineto{\pgfqpoint{1.659646in}{2.147117in}}%
\pgfpathlineto{\pgfqpoint{1.645513in}{2.147117in}}%
\pgfpathlineto{\pgfqpoint{1.631381in}{2.147117in}}%
\pgfpathlineto{\pgfqpoint{1.617248in}{2.147117in}}%
\pgfpathlineto{\pgfqpoint{1.603115in}{2.147117in}}%
\pgfpathlineto{\pgfqpoint{1.588983in}{2.147117in}}%
\pgfpathlineto{\pgfqpoint{1.574850in}{2.147117in}}%
\pgfpathlineto{\pgfqpoint{1.560718in}{2.147117in}}%
\pgfpathlineto{\pgfqpoint{1.546585in}{2.147117in}}%
\pgfpathlineto{\pgfqpoint{1.532452in}{2.147117in}}%
\pgfpathlineto{\pgfqpoint{1.518320in}{2.147117in}}%
\pgfpathlineto{\pgfqpoint{1.504187in}{2.147117in}}%
\pgfpathlineto{\pgfqpoint{1.490055in}{2.147117in}}%
\pgfpathlineto{\pgfqpoint{1.475922in}{2.147117in}}%
\pgfpathlineto{\pgfqpoint{1.461789in}{2.147117in}}%
\pgfpathlineto{\pgfqpoint{1.447657in}{2.147117in}}%
\pgfpathlineto{\pgfqpoint{1.433524in}{2.147117in}}%
\pgfpathlineto{\pgfqpoint{1.419392in}{2.147117in}}%
\pgfpathlineto{\pgfqpoint{1.405259in}{2.147117in}}%
\pgfpathlineto{\pgfqpoint{1.391126in}{2.147117in}}%
\pgfpathlineto{\pgfqpoint{1.376994in}{2.147117in}}%
\pgfpathlineto{\pgfqpoint{1.362861in}{2.147117in}}%
\pgfpathlineto{\pgfqpoint{1.348729in}{2.147117in}}%
\pgfpathlineto{\pgfqpoint{1.334596in}{2.147117in}}%
\pgfpathlineto{\pgfqpoint{1.320463in}{2.152949in}}%
\pgfpathlineto{\pgfqpoint{1.306331in}{2.158001in}}%
\pgfpathlineto{\pgfqpoint{1.292198in}{2.158001in}}%
\pgfpathlineto{\pgfqpoint{1.278066in}{2.160292in}}%
\pgfpathlineto{\pgfqpoint{1.263933in}{2.164805in}}%
\pgfpathlineto{\pgfqpoint{1.249800in}{2.164805in}}%
\pgfpathlineto{\pgfqpoint{1.235668in}{2.164805in}}%
\pgfpathlineto{\pgfqpoint{1.221535in}{2.164805in}}%
\pgfpathlineto{\pgfqpoint{1.207403in}{2.164805in}}%
\pgfpathlineto{\pgfqpoint{1.193270in}{2.164805in}}%
\pgfpathlineto{\pgfqpoint{1.179137in}{2.164805in}}%
\pgfpathlineto{\pgfqpoint{1.165005in}{2.164805in}}%
\pgfpathlineto{\pgfqpoint{1.150872in}{2.164805in}}%
\pgfpathlineto{\pgfqpoint{1.136740in}{2.164805in}}%
\pgfpathlineto{\pgfqpoint{1.122607in}{2.164805in}}%
\pgfpathlineto{\pgfqpoint{1.108474in}{2.164805in}}%
\pgfpathlineto{\pgfqpoint{1.094342in}{2.164805in}}%
\pgfpathlineto{\pgfqpoint{1.080209in}{2.164805in}}%
\pgfpathlineto{\pgfqpoint{1.066077in}{2.164805in}}%
\pgfpathlineto{\pgfqpoint{1.051944in}{2.191526in}}%
\pgfpathlineto{\pgfqpoint{1.037811in}{2.191526in}}%
\pgfpathlineto{\pgfqpoint{1.023679in}{2.191526in}}%
\pgfpathlineto{\pgfqpoint{1.009546in}{2.191526in}}%
\pgfpathlineto{\pgfqpoint{0.995414in}{2.191526in}}%
\pgfpathlineto{\pgfqpoint{0.981281in}{2.191526in}}%
\pgfpathlineto{\pgfqpoint{0.967148in}{2.191526in}}%
\pgfpathlineto{\pgfqpoint{0.953016in}{2.191526in}}%
\pgfpathlineto{\pgfqpoint{0.938883in}{2.204632in}}%
\pgfpathlineto{\pgfqpoint{0.924751in}{2.204632in}}%
\pgfpathlineto{\pgfqpoint{0.910618in}{2.204632in}}%
\pgfpathlineto{\pgfqpoint{0.896485in}{2.204632in}}%
\pgfpathlineto{\pgfqpoint{0.882353in}{2.204632in}}%
\pgfpathlineto{\pgfqpoint{0.868220in}{2.204632in}}%
\pgfpathlineto{\pgfqpoint{0.854088in}{2.204632in}}%
\pgfpathlineto{\pgfqpoint{0.839955in}{2.206273in}}%
\pgfpathlineto{\pgfqpoint{0.825822in}{2.209994in}}%
\pgfpathlineto{\pgfqpoint{0.811690in}{2.209994in}}%
\pgfpathlineto{\pgfqpoint{0.797557in}{2.212153in}}%
\pgfpathlineto{\pgfqpoint{0.783425in}{2.212153in}}%
\pgfpathlineto{\pgfqpoint{0.769292in}{2.214744in}}%
\pgfpathlineto{\pgfqpoint{0.755159in}{2.214744in}}%
\pgfpathlineto{\pgfqpoint{0.741027in}{2.214744in}}%
\pgfpathlineto{\pgfqpoint{0.726894in}{2.216832in}}%
\pgfpathlineto{\pgfqpoint{0.712762in}{2.216832in}}%
\pgfpathlineto{\pgfqpoint{0.698629in}{2.216832in}}%
\pgfpathlineto{\pgfqpoint{0.684496in}{2.216832in}}%
\pgfpathlineto{\pgfqpoint{0.670364in}{2.286978in}}%
\pgfpathlineto{\pgfqpoint{0.656231in}{2.286978in}}%
\pgfpathlineto{\pgfqpoint{0.642099in}{2.307607in}}%
\pgfpathlineto{\pgfqpoint{0.627966in}{2.307607in}}%
\pgfpathlineto{\pgfqpoint{0.613833in}{2.313607in}}%
\pgfpathlineto{\pgfqpoint{0.599701in}{2.335469in}}%
\pgfpathlineto{\pgfqpoint{0.585568in}{2.350710in}}%
\pgfpathlineto{\pgfqpoint{0.571436in}{2.389633in}}%
\pgfpathlineto{\pgfqpoint{0.557303in}{2.411537in}}%
\pgfpathlineto{\pgfqpoint{0.543170in}{2.480261in}}%
\pgfpathclose%
\pgfusepath{fill}%
\end{pgfscope}%
\begin{pgfscope}%
\pgfpathrectangle{\pgfqpoint{0.423750in}{0.375000in}}{\pgfqpoint{2.627250in}{2.265000in}}%
\pgfusepath{clip}%
\pgfsetbuttcap%
\pgfsetroundjoin%
\definecolor{currentfill}{rgb}{1.000000,0.498039,0.054902}%
\pgfsetfillcolor{currentfill}%
\pgfsetfillopacity{0.200000}%
\pgfsetlinewidth{0.000000pt}%
\definecolor{currentstroke}{rgb}{0.000000,0.000000,0.000000}%
\pgfsetstrokecolor{currentstroke}%
\pgfsetdash{}{0pt}%
\pgfpathmoveto{\pgfqpoint{0.543170in}{2.501123in}}%
\pgfpathlineto{\pgfqpoint{0.543170in}{2.537045in}}%
\pgfpathlineto{\pgfqpoint{0.557303in}{2.521601in}}%
\pgfpathlineto{\pgfqpoint{0.571436in}{2.467858in}}%
\pgfpathlineto{\pgfqpoint{0.585568in}{2.449478in}}%
\pgfpathlineto{\pgfqpoint{0.599701in}{2.449478in}}%
\pgfpathlineto{\pgfqpoint{0.613833in}{2.449411in}}%
\pgfpathlineto{\pgfqpoint{0.627966in}{2.449411in}}%
\pgfpathlineto{\pgfqpoint{0.642099in}{2.449411in}}%
\pgfpathlineto{\pgfqpoint{0.656231in}{2.449411in}}%
\pgfpathlineto{\pgfqpoint{0.670364in}{2.444866in}}%
\pgfpathlineto{\pgfqpoint{0.684496in}{2.444866in}}%
\pgfpathlineto{\pgfqpoint{0.698629in}{2.433284in}}%
\pgfpathlineto{\pgfqpoint{0.712762in}{2.404068in}}%
\pgfpathlineto{\pgfqpoint{0.726894in}{2.374850in}}%
\pgfpathlineto{\pgfqpoint{0.741027in}{2.331575in}}%
\pgfpathlineto{\pgfqpoint{0.755159in}{2.331575in}}%
\pgfpathlineto{\pgfqpoint{0.769292in}{2.316160in}}%
\pgfpathlineto{\pgfqpoint{0.783425in}{2.316160in}}%
\pgfpathlineto{\pgfqpoint{0.797557in}{2.316160in}}%
\pgfpathlineto{\pgfqpoint{0.811690in}{2.316160in}}%
\pgfpathlineto{\pgfqpoint{0.825822in}{2.316160in}}%
\pgfpathlineto{\pgfqpoint{0.839955in}{2.316160in}}%
\pgfpathlineto{\pgfqpoint{0.854088in}{2.316160in}}%
\pgfpathlineto{\pgfqpoint{0.868220in}{2.316160in}}%
\pgfpathlineto{\pgfqpoint{0.882353in}{2.316160in}}%
\pgfpathlineto{\pgfqpoint{0.896485in}{2.286250in}}%
\pgfpathlineto{\pgfqpoint{0.910618in}{2.246115in}}%
\pgfpathlineto{\pgfqpoint{0.924751in}{2.246115in}}%
\pgfpathlineto{\pgfqpoint{0.938883in}{2.246115in}}%
\pgfpathlineto{\pgfqpoint{0.953016in}{2.246115in}}%
\pgfpathlineto{\pgfqpoint{0.967148in}{2.245892in}}%
\pgfpathlineto{\pgfqpoint{0.981281in}{2.205958in}}%
\pgfpathlineto{\pgfqpoint{0.995414in}{2.191793in}}%
\pgfpathlineto{\pgfqpoint{1.009546in}{2.148816in}}%
\pgfpathlineto{\pgfqpoint{1.023679in}{2.148816in}}%
\pgfpathlineto{\pgfqpoint{1.037811in}{2.125478in}}%
\pgfpathlineto{\pgfqpoint{1.051944in}{2.111678in}}%
\pgfpathlineto{\pgfqpoint{1.066077in}{2.105672in}}%
\pgfpathlineto{\pgfqpoint{1.080209in}{2.100356in}}%
\pgfpathlineto{\pgfqpoint{1.094342in}{2.100356in}}%
\pgfpathlineto{\pgfqpoint{1.108474in}{2.080394in}}%
\pgfpathlineto{\pgfqpoint{1.122607in}{2.080394in}}%
\pgfpathlineto{\pgfqpoint{1.136740in}{2.080394in}}%
\pgfpathlineto{\pgfqpoint{1.150872in}{2.058652in}}%
\pgfpathlineto{\pgfqpoint{1.165005in}{1.895742in}}%
\pgfpathlineto{\pgfqpoint{1.179137in}{1.895742in}}%
\pgfpathlineto{\pgfqpoint{1.193270in}{1.895742in}}%
\pgfpathlineto{\pgfqpoint{1.207403in}{1.895742in}}%
\pgfpathlineto{\pgfqpoint{1.221535in}{1.894601in}}%
\pgfpathlineto{\pgfqpoint{1.235668in}{1.894601in}}%
\pgfpathlineto{\pgfqpoint{1.249800in}{1.894601in}}%
\pgfpathlineto{\pgfqpoint{1.263933in}{1.894601in}}%
\pgfpathlineto{\pgfqpoint{1.278066in}{1.894601in}}%
\pgfpathlineto{\pgfqpoint{1.292198in}{1.894601in}}%
\pgfpathlineto{\pgfqpoint{1.306331in}{1.894601in}}%
\pgfpathlineto{\pgfqpoint{1.320463in}{1.874417in}}%
\pgfpathlineto{\pgfqpoint{1.334596in}{1.849033in}}%
\pgfpathlineto{\pgfqpoint{1.348729in}{1.839398in}}%
\pgfpathlineto{\pgfqpoint{1.362861in}{1.832289in}}%
\pgfpathlineto{\pgfqpoint{1.376994in}{1.831204in}}%
\pgfpathlineto{\pgfqpoint{1.391126in}{1.831204in}}%
\pgfpathlineto{\pgfqpoint{1.405259in}{1.831204in}}%
\pgfpathlineto{\pgfqpoint{1.419392in}{1.831204in}}%
\pgfpathlineto{\pgfqpoint{1.433524in}{1.831204in}}%
\pgfpathlineto{\pgfqpoint{1.447657in}{1.831204in}}%
\pgfpathlineto{\pgfqpoint{1.461789in}{1.831204in}}%
\pgfpathlineto{\pgfqpoint{1.475922in}{1.831204in}}%
\pgfpathlineto{\pgfqpoint{1.490055in}{1.831204in}}%
\pgfpathlineto{\pgfqpoint{1.504187in}{1.831204in}}%
\pgfpathlineto{\pgfqpoint{1.518320in}{1.810775in}}%
\pgfpathlineto{\pgfqpoint{1.532452in}{1.810775in}}%
\pgfpathlineto{\pgfqpoint{1.546585in}{1.810775in}}%
\pgfpathlineto{\pgfqpoint{1.560718in}{1.810732in}}%
\pgfpathlineto{\pgfqpoint{1.574850in}{1.810732in}}%
\pgfpathlineto{\pgfqpoint{1.588983in}{1.810732in}}%
\pgfpathlineto{\pgfqpoint{1.603115in}{1.810732in}}%
\pgfpathlineto{\pgfqpoint{1.617248in}{1.794201in}}%
\pgfpathlineto{\pgfqpoint{1.631381in}{1.792567in}}%
\pgfpathlineto{\pgfqpoint{1.645513in}{1.785866in}}%
\pgfpathlineto{\pgfqpoint{1.659646in}{1.785866in}}%
\pgfpathlineto{\pgfqpoint{1.673778in}{1.785866in}}%
\pgfpathlineto{\pgfqpoint{1.687911in}{1.754118in}}%
\pgfpathlineto{\pgfqpoint{1.702044in}{1.754118in}}%
\pgfpathlineto{\pgfqpoint{1.716176in}{1.754118in}}%
\pgfpathlineto{\pgfqpoint{1.730309in}{1.754118in}}%
\pgfpathlineto{\pgfqpoint{1.744441in}{1.754118in}}%
\pgfpathlineto{\pgfqpoint{1.758574in}{1.754118in}}%
\pgfpathlineto{\pgfqpoint{1.772706in}{1.754118in}}%
\pgfpathlineto{\pgfqpoint{1.786839in}{1.754118in}}%
\pgfpathlineto{\pgfqpoint{1.800972in}{1.754118in}}%
\pgfpathlineto{\pgfqpoint{1.815104in}{1.754118in}}%
\pgfpathlineto{\pgfqpoint{1.829237in}{1.754118in}}%
\pgfpathlineto{\pgfqpoint{1.843369in}{1.754118in}}%
\pgfpathlineto{\pgfqpoint{1.857502in}{1.742007in}}%
\pgfpathlineto{\pgfqpoint{1.871635in}{1.742007in}}%
\pgfpathlineto{\pgfqpoint{1.885767in}{1.742007in}}%
\pgfpathlineto{\pgfqpoint{1.899900in}{1.742007in}}%
\pgfpathlineto{\pgfqpoint{1.914032in}{1.742007in}}%
\pgfpathlineto{\pgfqpoint{1.928165in}{1.742007in}}%
\pgfpathlineto{\pgfqpoint{1.942298in}{1.742007in}}%
\pgfpathlineto{\pgfqpoint{1.956430in}{1.742007in}}%
\pgfpathlineto{\pgfqpoint{1.970563in}{1.742007in}}%
\pgfpathlineto{\pgfqpoint{1.984695in}{1.742007in}}%
\pgfpathlineto{\pgfqpoint{1.998828in}{1.742007in}}%
\pgfpathlineto{\pgfqpoint{2.012961in}{1.742007in}}%
\pgfpathlineto{\pgfqpoint{2.027093in}{1.742007in}}%
\pgfpathlineto{\pgfqpoint{2.041226in}{1.742007in}}%
\pgfpathlineto{\pgfqpoint{2.055358in}{1.742007in}}%
\pgfpathlineto{\pgfqpoint{2.069491in}{1.742007in}}%
\pgfpathlineto{\pgfqpoint{2.083624in}{1.742007in}}%
\pgfpathlineto{\pgfqpoint{2.097756in}{1.742007in}}%
\pgfpathlineto{\pgfqpoint{2.111889in}{1.742007in}}%
\pgfpathlineto{\pgfqpoint{2.126021in}{1.742007in}}%
\pgfpathlineto{\pgfqpoint{2.140154in}{1.742007in}}%
\pgfpathlineto{\pgfqpoint{2.154287in}{1.742007in}}%
\pgfpathlineto{\pgfqpoint{2.168419in}{1.742007in}}%
\pgfpathlineto{\pgfqpoint{2.182552in}{1.742007in}}%
\pgfpathlineto{\pgfqpoint{2.196684in}{1.742007in}}%
\pgfpathlineto{\pgfqpoint{2.210817in}{1.742007in}}%
\pgfpathlineto{\pgfqpoint{2.224950in}{1.742007in}}%
\pgfpathlineto{\pgfqpoint{2.239082in}{1.742007in}}%
\pgfpathlineto{\pgfqpoint{2.253215in}{1.742007in}}%
\pgfpathlineto{\pgfqpoint{2.267347in}{1.742007in}}%
\pgfpathlineto{\pgfqpoint{2.281480in}{1.742007in}}%
\pgfpathlineto{\pgfqpoint{2.295613in}{1.742007in}}%
\pgfpathlineto{\pgfqpoint{2.309745in}{1.742007in}}%
\pgfpathlineto{\pgfqpoint{2.323878in}{1.742007in}}%
\pgfpathlineto{\pgfqpoint{2.338010in}{1.742007in}}%
\pgfpathlineto{\pgfqpoint{2.352143in}{1.742007in}}%
\pgfpathlineto{\pgfqpoint{2.366276in}{1.727732in}}%
\pgfpathlineto{\pgfqpoint{2.380408in}{1.727732in}}%
\pgfpathlineto{\pgfqpoint{2.394541in}{1.727732in}}%
\pgfpathlineto{\pgfqpoint{2.408673in}{1.727732in}}%
\pgfpathlineto{\pgfqpoint{2.422806in}{1.727433in}}%
\pgfpathlineto{\pgfqpoint{2.436939in}{1.726542in}}%
\pgfpathlineto{\pgfqpoint{2.451071in}{1.726542in}}%
\pgfpathlineto{\pgfqpoint{2.465204in}{1.726542in}}%
\pgfpathlineto{\pgfqpoint{2.479336in}{1.726542in}}%
\pgfpathlineto{\pgfqpoint{2.493469in}{1.717558in}}%
\pgfpathlineto{\pgfqpoint{2.507602in}{1.713361in}}%
\pgfpathlineto{\pgfqpoint{2.521734in}{1.713361in}}%
\pgfpathlineto{\pgfqpoint{2.535867in}{1.713361in}}%
\pgfpathlineto{\pgfqpoint{2.549999in}{1.713361in}}%
\pgfpathlineto{\pgfqpoint{2.564132in}{1.713361in}}%
\pgfpathlineto{\pgfqpoint{2.578265in}{1.713361in}}%
\pgfpathlineto{\pgfqpoint{2.592397in}{1.713361in}}%
\pgfpathlineto{\pgfqpoint{2.606530in}{1.713361in}}%
\pgfpathlineto{\pgfqpoint{2.620662in}{1.710983in}}%
\pgfpathlineto{\pgfqpoint{2.634795in}{1.710983in}}%
\pgfpathlineto{\pgfqpoint{2.648928in}{1.710983in}}%
\pgfpathlineto{\pgfqpoint{2.663060in}{1.703443in}}%
\pgfpathlineto{\pgfqpoint{2.677193in}{1.703443in}}%
\pgfpathlineto{\pgfqpoint{2.691325in}{1.703443in}}%
\pgfpathlineto{\pgfqpoint{2.705458in}{1.703443in}}%
\pgfpathlineto{\pgfqpoint{2.719591in}{1.703443in}}%
\pgfpathlineto{\pgfqpoint{2.733723in}{1.703443in}}%
\pgfpathlineto{\pgfqpoint{2.747856in}{1.703443in}}%
\pgfpathlineto{\pgfqpoint{2.761988in}{1.703443in}}%
\pgfpathlineto{\pgfqpoint{2.776121in}{1.703443in}}%
\pgfpathlineto{\pgfqpoint{2.790254in}{1.703443in}}%
\pgfpathlineto{\pgfqpoint{2.804386in}{1.703443in}}%
\pgfpathlineto{\pgfqpoint{2.818519in}{1.703443in}}%
\pgfpathlineto{\pgfqpoint{2.832651in}{1.703443in}}%
\pgfpathlineto{\pgfqpoint{2.846784in}{1.703443in}}%
\pgfpathlineto{\pgfqpoint{2.860917in}{1.703443in}}%
\pgfpathlineto{\pgfqpoint{2.875049in}{1.703443in}}%
\pgfpathlineto{\pgfqpoint{2.889182in}{1.703443in}}%
\pgfpathlineto{\pgfqpoint{2.903314in}{1.703443in}}%
\pgfpathlineto{\pgfqpoint{2.917447in}{1.703443in}}%
\pgfpathlineto{\pgfqpoint{2.931580in}{1.703233in}}%
\pgfpathlineto{\pgfqpoint{2.931580in}{1.618246in}}%
\pgfpathlineto{\pgfqpoint{2.931580in}{1.618246in}}%
\pgfpathlineto{\pgfqpoint{2.917447in}{1.620261in}}%
\pgfpathlineto{\pgfqpoint{2.903314in}{1.620261in}}%
\pgfpathlineto{\pgfqpoint{2.889182in}{1.620261in}}%
\pgfpathlineto{\pgfqpoint{2.875049in}{1.620261in}}%
\pgfpathlineto{\pgfqpoint{2.860917in}{1.620261in}}%
\pgfpathlineto{\pgfqpoint{2.846784in}{1.620261in}}%
\pgfpathlineto{\pgfqpoint{2.832651in}{1.620261in}}%
\pgfpathlineto{\pgfqpoint{2.818519in}{1.620261in}}%
\pgfpathlineto{\pgfqpoint{2.804386in}{1.620261in}}%
\pgfpathlineto{\pgfqpoint{2.790254in}{1.620261in}}%
\pgfpathlineto{\pgfqpoint{2.776121in}{1.620261in}}%
\pgfpathlineto{\pgfqpoint{2.761988in}{1.620261in}}%
\pgfpathlineto{\pgfqpoint{2.747856in}{1.620261in}}%
\pgfpathlineto{\pgfqpoint{2.733723in}{1.620261in}}%
\pgfpathlineto{\pgfqpoint{2.719591in}{1.620261in}}%
\pgfpathlineto{\pgfqpoint{2.705458in}{1.620261in}}%
\pgfpathlineto{\pgfqpoint{2.691325in}{1.620261in}}%
\pgfpathlineto{\pgfqpoint{2.677193in}{1.620261in}}%
\pgfpathlineto{\pgfqpoint{2.663060in}{1.620261in}}%
\pgfpathlineto{\pgfqpoint{2.648928in}{1.631799in}}%
\pgfpathlineto{\pgfqpoint{2.634795in}{1.631799in}}%
\pgfpathlineto{\pgfqpoint{2.620662in}{1.631799in}}%
\pgfpathlineto{\pgfqpoint{2.606530in}{1.632415in}}%
\pgfpathlineto{\pgfqpoint{2.592397in}{1.632415in}}%
\pgfpathlineto{\pgfqpoint{2.578265in}{1.632415in}}%
\pgfpathlineto{\pgfqpoint{2.564132in}{1.632415in}}%
\pgfpathlineto{\pgfqpoint{2.549999in}{1.632415in}}%
\pgfpathlineto{\pgfqpoint{2.535867in}{1.632415in}}%
\pgfpathlineto{\pgfqpoint{2.521734in}{1.632415in}}%
\pgfpathlineto{\pgfqpoint{2.507602in}{1.632415in}}%
\pgfpathlineto{\pgfqpoint{2.493469in}{1.637292in}}%
\pgfpathlineto{\pgfqpoint{2.479336in}{1.653675in}}%
\pgfpathlineto{\pgfqpoint{2.465204in}{1.653675in}}%
\pgfpathlineto{\pgfqpoint{2.451071in}{1.653675in}}%
\pgfpathlineto{\pgfqpoint{2.436939in}{1.653675in}}%
\pgfpathlineto{\pgfqpoint{2.422806in}{1.665205in}}%
\pgfpathlineto{\pgfqpoint{2.408673in}{1.668259in}}%
\pgfpathlineto{\pgfqpoint{2.394541in}{1.668259in}}%
\pgfpathlineto{\pgfqpoint{2.380408in}{1.668259in}}%
\pgfpathlineto{\pgfqpoint{2.366276in}{1.668259in}}%
\pgfpathlineto{\pgfqpoint{2.352143in}{1.680772in}}%
\pgfpathlineto{\pgfqpoint{2.338010in}{1.680772in}}%
\pgfpathlineto{\pgfqpoint{2.323878in}{1.680772in}}%
\pgfpathlineto{\pgfqpoint{2.309745in}{1.680772in}}%
\pgfpathlineto{\pgfqpoint{2.295613in}{1.680772in}}%
\pgfpathlineto{\pgfqpoint{2.281480in}{1.680772in}}%
\pgfpathlineto{\pgfqpoint{2.267347in}{1.680772in}}%
\pgfpathlineto{\pgfqpoint{2.253215in}{1.680772in}}%
\pgfpathlineto{\pgfqpoint{2.239082in}{1.680772in}}%
\pgfpathlineto{\pgfqpoint{2.224950in}{1.680772in}}%
\pgfpathlineto{\pgfqpoint{2.210817in}{1.680772in}}%
\pgfpathlineto{\pgfqpoint{2.196684in}{1.680772in}}%
\pgfpathlineto{\pgfqpoint{2.182552in}{1.680772in}}%
\pgfpathlineto{\pgfqpoint{2.168419in}{1.680772in}}%
\pgfpathlineto{\pgfqpoint{2.154287in}{1.680772in}}%
\pgfpathlineto{\pgfqpoint{2.140154in}{1.680772in}}%
\pgfpathlineto{\pgfqpoint{2.126021in}{1.680772in}}%
\pgfpathlineto{\pgfqpoint{2.111889in}{1.680772in}}%
\pgfpathlineto{\pgfqpoint{2.097756in}{1.680772in}}%
\pgfpathlineto{\pgfqpoint{2.083624in}{1.680772in}}%
\pgfpathlineto{\pgfqpoint{2.069491in}{1.680772in}}%
\pgfpathlineto{\pgfqpoint{2.055358in}{1.680772in}}%
\pgfpathlineto{\pgfqpoint{2.041226in}{1.680772in}}%
\pgfpathlineto{\pgfqpoint{2.027093in}{1.680772in}}%
\pgfpathlineto{\pgfqpoint{2.012961in}{1.680772in}}%
\pgfpathlineto{\pgfqpoint{1.998828in}{1.680772in}}%
\pgfpathlineto{\pgfqpoint{1.984695in}{1.680772in}}%
\pgfpathlineto{\pgfqpoint{1.970563in}{1.680772in}}%
\pgfpathlineto{\pgfqpoint{1.956430in}{1.680772in}}%
\pgfpathlineto{\pgfqpoint{1.942298in}{1.680772in}}%
\pgfpathlineto{\pgfqpoint{1.928165in}{1.680772in}}%
\pgfpathlineto{\pgfqpoint{1.914032in}{1.680772in}}%
\pgfpathlineto{\pgfqpoint{1.899900in}{1.680772in}}%
\pgfpathlineto{\pgfqpoint{1.885767in}{1.680772in}}%
\pgfpathlineto{\pgfqpoint{1.871635in}{1.680772in}}%
\pgfpathlineto{\pgfqpoint{1.857502in}{1.680772in}}%
\pgfpathlineto{\pgfqpoint{1.843369in}{1.686588in}}%
\pgfpathlineto{\pgfqpoint{1.829237in}{1.686588in}}%
\pgfpathlineto{\pgfqpoint{1.815104in}{1.686588in}}%
\pgfpathlineto{\pgfqpoint{1.800972in}{1.686588in}}%
\pgfpathlineto{\pgfqpoint{1.786839in}{1.686588in}}%
\pgfpathlineto{\pgfqpoint{1.772706in}{1.686588in}}%
\pgfpathlineto{\pgfqpoint{1.758574in}{1.686588in}}%
\pgfpathlineto{\pgfqpoint{1.744441in}{1.686588in}}%
\pgfpathlineto{\pgfqpoint{1.730309in}{1.686588in}}%
\pgfpathlineto{\pgfqpoint{1.716176in}{1.686588in}}%
\pgfpathlineto{\pgfqpoint{1.702044in}{1.686588in}}%
\pgfpathlineto{\pgfqpoint{1.687911in}{1.686588in}}%
\pgfpathlineto{\pgfqpoint{1.673778in}{1.711765in}}%
\pgfpathlineto{\pgfqpoint{1.659646in}{1.711765in}}%
\pgfpathlineto{\pgfqpoint{1.645513in}{1.711765in}}%
\pgfpathlineto{\pgfqpoint{1.631381in}{1.724243in}}%
\pgfpathlineto{\pgfqpoint{1.617248in}{1.733923in}}%
\pgfpathlineto{\pgfqpoint{1.603115in}{1.741629in}}%
\pgfpathlineto{\pgfqpoint{1.588983in}{1.741629in}}%
\pgfpathlineto{\pgfqpoint{1.574850in}{1.741629in}}%
\pgfpathlineto{\pgfqpoint{1.560718in}{1.741629in}}%
\pgfpathlineto{\pgfqpoint{1.546585in}{1.741841in}}%
\pgfpathlineto{\pgfqpoint{1.532452in}{1.741841in}}%
\pgfpathlineto{\pgfqpoint{1.518320in}{1.741841in}}%
\pgfpathlineto{\pgfqpoint{1.504187in}{1.782697in}}%
\pgfpathlineto{\pgfqpoint{1.490055in}{1.782697in}}%
\pgfpathlineto{\pgfqpoint{1.475922in}{1.782697in}}%
\pgfpathlineto{\pgfqpoint{1.461789in}{1.782697in}}%
\pgfpathlineto{\pgfqpoint{1.447657in}{1.782697in}}%
\pgfpathlineto{\pgfqpoint{1.433524in}{1.782697in}}%
\pgfpathlineto{\pgfqpoint{1.419392in}{1.782697in}}%
\pgfpathlineto{\pgfqpoint{1.405259in}{1.782697in}}%
\pgfpathlineto{\pgfqpoint{1.391126in}{1.782697in}}%
\pgfpathlineto{\pgfqpoint{1.376994in}{1.782697in}}%
\pgfpathlineto{\pgfqpoint{1.362861in}{1.794146in}}%
\pgfpathlineto{\pgfqpoint{1.348729in}{1.796397in}}%
\pgfpathlineto{\pgfqpoint{1.334596in}{1.798608in}}%
\pgfpathlineto{\pgfqpoint{1.320463in}{1.802615in}}%
\pgfpathlineto{\pgfqpoint{1.306331in}{1.805188in}}%
\pgfpathlineto{\pgfqpoint{1.292198in}{1.805188in}}%
\pgfpathlineto{\pgfqpoint{1.278066in}{1.805188in}}%
\pgfpathlineto{\pgfqpoint{1.263933in}{1.805188in}}%
\pgfpathlineto{\pgfqpoint{1.249800in}{1.805188in}}%
\pgfpathlineto{\pgfqpoint{1.235668in}{1.805188in}}%
\pgfpathlineto{\pgfqpoint{1.221535in}{1.805188in}}%
\pgfpathlineto{\pgfqpoint{1.207403in}{1.807527in}}%
\pgfpathlineto{\pgfqpoint{1.193270in}{1.807527in}}%
\pgfpathlineto{\pgfqpoint{1.179137in}{1.807527in}}%
\pgfpathlineto{\pgfqpoint{1.165005in}{1.807527in}}%
\pgfpathlineto{\pgfqpoint{1.150872in}{1.873948in}}%
\pgfpathlineto{\pgfqpoint{1.136740in}{1.917891in}}%
\pgfpathlineto{\pgfqpoint{1.122607in}{1.917891in}}%
\pgfpathlineto{\pgfqpoint{1.108474in}{1.917891in}}%
\pgfpathlineto{\pgfqpoint{1.094342in}{1.921675in}}%
\pgfpathlineto{\pgfqpoint{1.080209in}{1.921675in}}%
\pgfpathlineto{\pgfqpoint{1.066077in}{1.949283in}}%
\pgfpathlineto{\pgfqpoint{1.051944in}{1.970507in}}%
\pgfpathlineto{\pgfqpoint{1.037811in}{1.972913in}}%
\pgfpathlineto{\pgfqpoint{1.023679in}{2.029881in}}%
\pgfpathlineto{\pgfqpoint{1.009546in}{2.029881in}}%
\pgfpathlineto{\pgfqpoint{0.995414in}{2.040695in}}%
\pgfpathlineto{\pgfqpoint{0.981281in}{2.088137in}}%
\pgfpathlineto{\pgfqpoint{0.967148in}{2.096818in}}%
\pgfpathlineto{\pgfqpoint{0.953016in}{2.098456in}}%
\pgfpathlineto{\pgfqpoint{0.938883in}{2.098456in}}%
\pgfpathlineto{\pgfqpoint{0.924751in}{2.098456in}}%
\pgfpathlineto{\pgfqpoint{0.910618in}{2.098456in}}%
\pgfpathlineto{\pgfqpoint{0.896485in}{2.129040in}}%
\pgfpathlineto{\pgfqpoint{0.882353in}{2.188365in}}%
\pgfpathlineto{\pgfqpoint{0.868220in}{2.188365in}}%
\pgfpathlineto{\pgfqpoint{0.854088in}{2.188365in}}%
\pgfpathlineto{\pgfqpoint{0.839955in}{2.188365in}}%
\pgfpathlineto{\pgfqpoint{0.825822in}{2.188365in}}%
\pgfpathlineto{\pgfqpoint{0.811690in}{2.188365in}}%
\pgfpathlineto{\pgfqpoint{0.797557in}{2.188365in}}%
\pgfpathlineto{\pgfqpoint{0.783425in}{2.188365in}}%
\pgfpathlineto{\pgfqpoint{0.769292in}{2.188365in}}%
\pgfpathlineto{\pgfqpoint{0.755159in}{2.228128in}}%
\pgfpathlineto{\pgfqpoint{0.741027in}{2.228128in}}%
\pgfpathlineto{\pgfqpoint{0.726894in}{2.296337in}}%
\pgfpathlineto{\pgfqpoint{0.712762in}{2.328318in}}%
\pgfpathlineto{\pgfqpoint{0.698629in}{2.386847in}}%
\pgfpathlineto{\pgfqpoint{0.684496in}{2.395875in}}%
\pgfpathlineto{\pgfqpoint{0.670364in}{2.395875in}}%
\pgfpathlineto{\pgfqpoint{0.656231in}{2.398810in}}%
\pgfpathlineto{\pgfqpoint{0.642099in}{2.398810in}}%
\pgfpathlineto{\pgfqpoint{0.627966in}{2.398810in}}%
\pgfpathlineto{\pgfqpoint{0.613833in}{2.398810in}}%
\pgfpathlineto{\pgfqpoint{0.599701in}{2.400840in}}%
\pgfpathlineto{\pgfqpoint{0.585568in}{2.400840in}}%
\pgfpathlineto{\pgfqpoint{0.571436in}{2.407920in}}%
\pgfpathlineto{\pgfqpoint{0.557303in}{2.488679in}}%
\pgfpathlineto{\pgfqpoint{0.543170in}{2.501123in}}%
\pgfpathclose%
\pgfusepath{fill}%
\end{pgfscope}%
\begin{pgfscope}%
\pgfpathrectangle{\pgfqpoint{0.423750in}{0.375000in}}{\pgfqpoint{2.627250in}{2.265000in}}%
\pgfusepath{clip}%
\pgfsetbuttcap%
\pgfsetroundjoin%
\definecolor{currentfill}{rgb}{0.172549,0.627451,0.172549}%
\pgfsetfillcolor{currentfill}%
\pgfsetfillopacity{0.200000}%
\pgfsetlinewidth{0.000000pt}%
\definecolor{currentstroke}{rgb}{0.000000,0.000000,0.000000}%
\pgfsetstrokecolor{currentstroke}%
\pgfsetdash{}{0pt}%
\pgfpathmoveto{\pgfqpoint{0.543170in}{2.478383in}}%
\pgfpathlineto{\pgfqpoint{0.543170in}{2.528509in}}%
\pgfpathlineto{\pgfqpoint{0.557303in}{2.498454in}}%
\pgfpathlineto{\pgfqpoint{0.571436in}{2.481929in}}%
\pgfpathlineto{\pgfqpoint{0.585568in}{2.447357in}}%
\pgfpathlineto{\pgfqpoint{0.599701in}{2.429474in}}%
\pgfpathlineto{\pgfqpoint{0.613833in}{2.428700in}}%
\pgfpathlineto{\pgfqpoint{0.627966in}{2.428700in}}%
\pgfpathlineto{\pgfqpoint{0.642099in}{2.407854in}}%
\pgfpathlineto{\pgfqpoint{0.656231in}{2.399903in}}%
\pgfpathlineto{\pgfqpoint{0.670364in}{2.399903in}}%
\pgfpathlineto{\pgfqpoint{0.684496in}{2.399903in}}%
\pgfpathlineto{\pgfqpoint{0.698629in}{2.399903in}}%
\pgfpathlineto{\pgfqpoint{0.712762in}{2.399903in}}%
\pgfpathlineto{\pgfqpoint{0.726894in}{2.379825in}}%
\pgfpathlineto{\pgfqpoint{0.741027in}{2.379825in}}%
\pgfpathlineto{\pgfqpoint{0.755159in}{2.376186in}}%
\pgfpathlineto{\pgfqpoint{0.769292in}{2.376186in}}%
\pgfpathlineto{\pgfqpoint{0.783425in}{2.374668in}}%
\pgfpathlineto{\pgfqpoint{0.797557in}{2.365021in}}%
\pgfpathlineto{\pgfqpoint{0.811690in}{2.365021in}}%
\pgfpathlineto{\pgfqpoint{0.825822in}{2.365021in}}%
\pgfpathlineto{\pgfqpoint{0.839955in}{2.360695in}}%
\pgfpathlineto{\pgfqpoint{0.854088in}{2.337379in}}%
\pgfpathlineto{\pgfqpoint{0.868220in}{2.319488in}}%
\pgfpathlineto{\pgfqpoint{0.882353in}{2.317958in}}%
\pgfpathlineto{\pgfqpoint{0.896485in}{2.303714in}}%
\pgfpathlineto{\pgfqpoint{0.910618in}{2.122470in}}%
\pgfpathlineto{\pgfqpoint{0.924751in}{2.078486in}}%
\pgfpathlineto{\pgfqpoint{0.938883in}{2.050520in}}%
\pgfpathlineto{\pgfqpoint{0.953016in}{2.040877in}}%
\pgfpathlineto{\pgfqpoint{0.967148in}{2.028608in}}%
\pgfpathlineto{\pgfqpoint{0.981281in}{2.025440in}}%
\pgfpathlineto{\pgfqpoint{0.995414in}{2.025440in}}%
\pgfpathlineto{\pgfqpoint{1.009546in}{2.010881in}}%
\pgfpathlineto{\pgfqpoint{1.023679in}{1.999026in}}%
\pgfpathlineto{\pgfqpoint{1.037811in}{1.997554in}}%
\pgfpathlineto{\pgfqpoint{1.051944in}{1.993486in}}%
\pgfpathlineto{\pgfqpoint{1.066077in}{1.992983in}}%
\pgfpathlineto{\pgfqpoint{1.080209in}{1.990777in}}%
\pgfpathlineto{\pgfqpoint{1.094342in}{1.990777in}}%
\pgfpathlineto{\pgfqpoint{1.108474in}{1.979723in}}%
\pgfpathlineto{\pgfqpoint{1.122607in}{1.979723in}}%
\pgfpathlineto{\pgfqpoint{1.136740in}{1.977791in}}%
\pgfpathlineto{\pgfqpoint{1.150872in}{1.977791in}}%
\pgfpathlineto{\pgfqpoint{1.165005in}{1.977791in}}%
\pgfpathlineto{\pgfqpoint{1.179137in}{1.977791in}}%
\pgfpathlineto{\pgfqpoint{1.193270in}{1.976066in}}%
\pgfpathlineto{\pgfqpoint{1.207403in}{1.974395in}}%
\pgfpathlineto{\pgfqpoint{1.221535in}{1.974395in}}%
\pgfpathlineto{\pgfqpoint{1.235668in}{1.971244in}}%
\pgfpathlineto{\pgfqpoint{1.249800in}{1.965051in}}%
\pgfpathlineto{\pgfqpoint{1.263933in}{1.953433in}}%
\pgfpathlineto{\pgfqpoint{1.278066in}{1.953433in}}%
\pgfpathlineto{\pgfqpoint{1.292198in}{1.953433in}}%
\pgfpathlineto{\pgfqpoint{1.306331in}{1.953433in}}%
\pgfpathlineto{\pgfqpoint{1.320463in}{1.953433in}}%
\pgfpathlineto{\pgfqpoint{1.334596in}{1.953433in}}%
\pgfpathlineto{\pgfqpoint{1.348729in}{1.945429in}}%
\pgfpathlineto{\pgfqpoint{1.362861in}{1.942005in}}%
\pgfpathlineto{\pgfqpoint{1.376994in}{1.931725in}}%
\pgfpathlineto{\pgfqpoint{1.391126in}{1.931725in}}%
\pgfpathlineto{\pgfqpoint{1.405259in}{1.931725in}}%
\pgfpathlineto{\pgfqpoint{1.419392in}{1.931725in}}%
\pgfpathlineto{\pgfqpoint{1.433524in}{1.931725in}}%
\pgfpathlineto{\pgfqpoint{1.447657in}{1.931725in}}%
\pgfpathlineto{\pgfqpoint{1.461789in}{1.931725in}}%
\pgfpathlineto{\pgfqpoint{1.475922in}{1.931725in}}%
\pgfpathlineto{\pgfqpoint{1.490055in}{1.931671in}}%
\pgfpathlineto{\pgfqpoint{1.504187in}{1.931671in}}%
\pgfpathlineto{\pgfqpoint{1.518320in}{1.931671in}}%
\pgfpathlineto{\pgfqpoint{1.532452in}{1.931671in}}%
\pgfpathlineto{\pgfqpoint{1.546585in}{1.931671in}}%
\pgfpathlineto{\pgfqpoint{1.560718in}{1.931671in}}%
\pgfpathlineto{\pgfqpoint{1.574850in}{1.931671in}}%
\pgfpathlineto{\pgfqpoint{1.588983in}{1.931671in}}%
\pgfpathlineto{\pgfqpoint{1.603115in}{1.931671in}}%
\pgfpathlineto{\pgfqpoint{1.617248in}{1.922335in}}%
\pgfpathlineto{\pgfqpoint{1.631381in}{1.922335in}}%
\pgfpathlineto{\pgfqpoint{1.645513in}{1.922335in}}%
\pgfpathlineto{\pgfqpoint{1.659646in}{1.920588in}}%
\pgfpathlineto{\pgfqpoint{1.673778in}{1.913805in}}%
\pgfpathlineto{\pgfqpoint{1.687911in}{1.894352in}}%
\pgfpathlineto{\pgfqpoint{1.702044in}{1.894352in}}%
\pgfpathlineto{\pgfqpoint{1.716176in}{1.894352in}}%
\pgfpathlineto{\pgfqpoint{1.730309in}{1.894352in}}%
\pgfpathlineto{\pgfqpoint{1.744441in}{1.894352in}}%
\pgfpathlineto{\pgfqpoint{1.758574in}{1.894352in}}%
\pgfpathlineto{\pgfqpoint{1.772706in}{1.894352in}}%
\pgfpathlineto{\pgfqpoint{1.786839in}{1.894352in}}%
\pgfpathlineto{\pgfqpoint{1.800972in}{1.894352in}}%
\pgfpathlineto{\pgfqpoint{1.815104in}{1.887465in}}%
\pgfpathlineto{\pgfqpoint{1.829237in}{1.887465in}}%
\pgfpathlineto{\pgfqpoint{1.843369in}{1.887465in}}%
\pgfpathlineto{\pgfqpoint{1.857502in}{1.887465in}}%
\pgfpathlineto{\pgfqpoint{1.871635in}{1.887465in}}%
\pgfpathlineto{\pgfqpoint{1.885767in}{1.887465in}}%
\pgfpathlineto{\pgfqpoint{1.899900in}{1.887465in}}%
\pgfpathlineto{\pgfqpoint{1.914032in}{1.887465in}}%
\pgfpathlineto{\pgfqpoint{1.928165in}{1.887465in}}%
\pgfpathlineto{\pgfqpoint{1.942298in}{1.887465in}}%
\pgfpathlineto{\pgfqpoint{1.956430in}{1.887465in}}%
\pgfpathlineto{\pgfqpoint{1.970563in}{1.887465in}}%
\pgfpathlineto{\pgfqpoint{1.984695in}{1.887465in}}%
\pgfpathlineto{\pgfqpoint{1.998828in}{1.887465in}}%
\pgfpathlineto{\pgfqpoint{2.012961in}{1.887465in}}%
\pgfpathlineto{\pgfqpoint{2.027093in}{1.887193in}}%
\pgfpathlineto{\pgfqpoint{2.041226in}{1.887193in}}%
\pgfpathlineto{\pgfqpoint{2.055358in}{1.887193in}}%
\pgfpathlineto{\pgfqpoint{2.069491in}{1.887193in}}%
\pgfpathlineto{\pgfqpoint{2.083624in}{1.887193in}}%
\pgfpathlineto{\pgfqpoint{2.097756in}{1.887193in}}%
\pgfpathlineto{\pgfqpoint{2.111889in}{1.887193in}}%
\pgfpathlineto{\pgfqpoint{2.126021in}{1.887193in}}%
\pgfpathlineto{\pgfqpoint{2.140154in}{1.887193in}}%
\pgfpathlineto{\pgfqpoint{2.154287in}{1.887193in}}%
\pgfpathlineto{\pgfqpoint{2.168419in}{1.885999in}}%
\pgfpathlineto{\pgfqpoint{2.182552in}{1.885869in}}%
\pgfpathlineto{\pgfqpoint{2.196684in}{1.885869in}}%
\pgfpathlineto{\pgfqpoint{2.210817in}{1.885869in}}%
\pgfpathlineto{\pgfqpoint{2.224950in}{1.885869in}}%
\pgfpathlineto{\pgfqpoint{2.239082in}{1.879964in}}%
\pgfpathlineto{\pgfqpoint{2.253215in}{1.879964in}}%
\pgfpathlineto{\pgfqpoint{2.267347in}{1.879964in}}%
\pgfpathlineto{\pgfqpoint{2.281480in}{1.879964in}}%
\pgfpathlineto{\pgfqpoint{2.295613in}{1.879964in}}%
\pgfpathlineto{\pgfqpoint{2.309745in}{1.879964in}}%
\pgfpathlineto{\pgfqpoint{2.323878in}{1.878526in}}%
\pgfpathlineto{\pgfqpoint{2.338010in}{1.878526in}}%
\pgfpathlineto{\pgfqpoint{2.352143in}{1.878526in}}%
\pgfpathlineto{\pgfqpoint{2.366276in}{1.878526in}}%
\pgfpathlineto{\pgfqpoint{2.380408in}{1.878526in}}%
\pgfpathlineto{\pgfqpoint{2.394541in}{1.878526in}}%
\pgfpathlineto{\pgfqpoint{2.408673in}{1.878526in}}%
\pgfpathlineto{\pgfqpoint{2.422806in}{1.878526in}}%
\pgfpathlineto{\pgfqpoint{2.436939in}{1.878526in}}%
\pgfpathlineto{\pgfqpoint{2.451071in}{1.878287in}}%
\pgfpathlineto{\pgfqpoint{2.465204in}{1.878287in}}%
\pgfpathlineto{\pgfqpoint{2.479336in}{1.878287in}}%
\pgfpathlineto{\pgfqpoint{2.493469in}{1.878287in}}%
\pgfpathlineto{\pgfqpoint{2.507602in}{1.875795in}}%
\pgfpathlineto{\pgfqpoint{2.521734in}{1.875795in}}%
\pgfpathlineto{\pgfqpoint{2.535867in}{1.875795in}}%
\pgfpathlineto{\pgfqpoint{2.549999in}{1.875625in}}%
\pgfpathlineto{\pgfqpoint{2.564132in}{1.875625in}}%
\pgfpathlineto{\pgfqpoint{2.578265in}{1.875625in}}%
\pgfpathlineto{\pgfqpoint{2.592397in}{1.875625in}}%
\pgfpathlineto{\pgfqpoint{2.606530in}{1.875625in}}%
\pgfpathlineto{\pgfqpoint{2.620662in}{1.875625in}}%
\pgfpathlineto{\pgfqpoint{2.634795in}{1.875625in}}%
\pgfpathlineto{\pgfqpoint{2.648928in}{1.875625in}}%
\pgfpathlineto{\pgfqpoint{2.663060in}{1.875625in}}%
\pgfpathlineto{\pgfqpoint{2.677193in}{1.875625in}}%
\pgfpathlineto{\pgfqpoint{2.691325in}{1.875625in}}%
\pgfpathlineto{\pgfqpoint{2.705458in}{1.875625in}}%
\pgfpathlineto{\pgfqpoint{2.719591in}{1.875625in}}%
\pgfpathlineto{\pgfqpoint{2.733723in}{1.875625in}}%
\pgfpathlineto{\pgfqpoint{2.747856in}{1.875625in}}%
\pgfpathlineto{\pgfqpoint{2.761988in}{1.875625in}}%
\pgfpathlineto{\pgfqpoint{2.776121in}{1.875625in}}%
\pgfpathlineto{\pgfqpoint{2.790254in}{1.875625in}}%
\pgfpathlineto{\pgfqpoint{2.804386in}{1.875625in}}%
\pgfpathlineto{\pgfqpoint{2.818519in}{1.875625in}}%
\pgfpathlineto{\pgfqpoint{2.832651in}{1.875625in}}%
\pgfpathlineto{\pgfqpoint{2.846784in}{1.875625in}}%
\pgfpathlineto{\pgfqpoint{2.860917in}{1.875625in}}%
\pgfpathlineto{\pgfqpoint{2.875049in}{1.875625in}}%
\pgfpathlineto{\pgfqpoint{2.889182in}{1.875625in}}%
\pgfpathlineto{\pgfqpoint{2.903314in}{1.875625in}}%
\pgfpathlineto{\pgfqpoint{2.917447in}{1.875625in}}%
\pgfpathlineto{\pgfqpoint{2.931580in}{1.875625in}}%
\pgfpathlineto{\pgfqpoint{2.931580in}{1.512239in}}%
\pgfpathlineto{\pgfqpoint{2.931580in}{1.512239in}}%
\pgfpathlineto{\pgfqpoint{2.917447in}{1.512239in}}%
\pgfpathlineto{\pgfqpoint{2.903314in}{1.512239in}}%
\pgfpathlineto{\pgfqpoint{2.889182in}{1.512239in}}%
\pgfpathlineto{\pgfqpoint{2.875049in}{1.512239in}}%
\pgfpathlineto{\pgfqpoint{2.860917in}{1.512239in}}%
\pgfpathlineto{\pgfqpoint{2.846784in}{1.512239in}}%
\pgfpathlineto{\pgfqpoint{2.832651in}{1.512239in}}%
\pgfpathlineto{\pgfqpoint{2.818519in}{1.512239in}}%
\pgfpathlineto{\pgfqpoint{2.804386in}{1.512239in}}%
\pgfpathlineto{\pgfqpoint{2.790254in}{1.512239in}}%
\pgfpathlineto{\pgfqpoint{2.776121in}{1.512239in}}%
\pgfpathlineto{\pgfqpoint{2.761988in}{1.512239in}}%
\pgfpathlineto{\pgfqpoint{2.747856in}{1.512239in}}%
\pgfpathlineto{\pgfqpoint{2.733723in}{1.512239in}}%
\pgfpathlineto{\pgfqpoint{2.719591in}{1.512239in}}%
\pgfpathlineto{\pgfqpoint{2.705458in}{1.512239in}}%
\pgfpathlineto{\pgfqpoint{2.691325in}{1.512239in}}%
\pgfpathlineto{\pgfqpoint{2.677193in}{1.512239in}}%
\pgfpathlineto{\pgfqpoint{2.663060in}{1.512239in}}%
\pgfpathlineto{\pgfqpoint{2.648928in}{1.512239in}}%
\pgfpathlineto{\pgfqpoint{2.634795in}{1.512239in}}%
\pgfpathlineto{\pgfqpoint{2.620662in}{1.512239in}}%
\pgfpathlineto{\pgfqpoint{2.606530in}{1.512239in}}%
\pgfpathlineto{\pgfqpoint{2.592397in}{1.512239in}}%
\pgfpathlineto{\pgfqpoint{2.578265in}{1.512239in}}%
\pgfpathlineto{\pgfqpoint{2.564132in}{1.512239in}}%
\pgfpathlineto{\pgfqpoint{2.549999in}{1.512239in}}%
\pgfpathlineto{\pgfqpoint{2.535867in}{1.515400in}}%
\pgfpathlineto{\pgfqpoint{2.521734in}{1.515400in}}%
\pgfpathlineto{\pgfqpoint{2.507602in}{1.515400in}}%
\pgfpathlineto{\pgfqpoint{2.493469in}{1.516881in}}%
\pgfpathlineto{\pgfqpoint{2.479336in}{1.516881in}}%
\pgfpathlineto{\pgfqpoint{2.465204in}{1.516881in}}%
\pgfpathlineto{\pgfqpoint{2.451071in}{1.516881in}}%
\pgfpathlineto{\pgfqpoint{2.436939in}{1.517023in}}%
\pgfpathlineto{\pgfqpoint{2.422806in}{1.517023in}}%
\pgfpathlineto{\pgfqpoint{2.408673in}{1.517023in}}%
\pgfpathlineto{\pgfqpoint{2.394541in}{1.517023in}}%
\pgfpathlineto{\pgfqpoint{2.380408in}{1.517023in}}%
\pgfpathlineto{\pgfqpoint{2.366276in}{1.517023in}}%
\pgfpathlineto{\pgfqpoint{2.352143in}{1.517023in}}%
\pgfpathlineto{\pgfqpoint{2.338010in}{1.517023in}}%
\pgfpathlineto{\pgfqpoint{2.323878in}{1.517023in}}%
\pgfpathlineto{\pgfqpoint{2.309745in}{1.517883in}}%
\pgfpathlineto{\pgfqpoint{2.295613in}{1.517883in}}%
\pgfpathlineto{\pgfqpoint{2.281480in}{1.517883in}}%
\pgfpathlineto{\pgfqpoint{2.267347in}{1.517883in}}%
\pgfpathlineto{\pgfqpoint{2.253215in}{1.517883in}}%
\pgfpathlineto{\pgfqpoint{2.239082in}{1.517883in}}%
\pgfpathlineto{\pgfqpoint{2.224950in}{1.521449in}}%
\pgfpathlineto{\pgfqpoint{2.210817in}{1.521449in}}%
\pgfpathlineto{\pgfqpoint{2.196684in}{1.521449in}}%
\pgfpathlineto{\pgfqpoint{2.182552in}{1.521449in}}%
\pgfpathlineto{\pgfqpoint{2.168419in}{1.524051in}}%
\pgfpathlineto{\pgfqpoint{2.154287in}{1.524766in}}%
\pgfpathlineto{\pgfqpoint{2.140154in}{1.524766in}}%
\pgfpathlineto{\pgfqpoint{2.126021in}{1.524766in}}%
\pgfpathlineto{\pgfqpoint{2.111889in}{1.524766in}}%
\pgfpathlineto{\pgfqpoint{2.097756in}{1.524766in}}%
\pgfpathlineto{\pgfqpoint{2.083624in}{1.524766in}}%
\pgfpathlineto{\pgfqpoint{2.069491in}{1.524766in}}%
\pgfpathlineto{\pgfqpoint{2.055358in}{1.524766in}}%
\pgfpathlineto{\pgfqpoint{2.041226in}{1.524766in}}%
\pgfpathlineto{\pgfqpoint{2.027093in}{1.524766in}}%
\pgfpathlineto{\pgfqpoint{2.012961in}{1.530047in}}%
\pgfpathlineto{\pgfqpoint{1.998828in}{1.530047in}}%
\pgfpathlineto{\pgfqpoint{1.984695in}{1.530047in}}%
\pgfpathlineto{\pgfqpoint{1.970563in}{1.530047in}}%
\pgfpathlineto{\pgfqpoint{1.956430in}{1.530047in}}%
\pgfpathlineto{\pgfqpoint{1.942298in}{1.530047in}}%
\pgfpathlineto{\pgfqpoint{1.928165in}{1.530047in}}%
\pgfpathlineto{\pgfqpoint{1.914032in}{1.530047in}}%
\pgfpathlineto{\pgfqpoint{1.899900in}{1.530047in}}%
\pgfpathlineto{\pgfqpoint{1.885767in}{1.530047in}}%
\pgfpathlineto{\pgfqpoint{1.871635in}{1.530047in}}%
\pgfpathlineto{\pgfqpoint{1.857502in}{1.530047in}}%
\pgfpathlineto{\pgfqpoint{1.843369in}{1.530047in}}%
\pgfpathlineto{\pgfqpoint{1.829237in}{1.530047in}}%
\pgfpathlineto{\pgfqpoint{1.815104in}{1.530047in}}%
\pgfpathlineto{\pgfqpoint{1.800972in}{1.618472in}}%
\pgfpathlineto{\pgfqpoint{1.786839in}{1.618472in}}%
\pgfpathlineto{\pgfqpoint{1.772706in}{1.618472in}}%
\pgfpathlineto{\pgfqpoint{1.758574in}{1.618472in}}%
\pgfpathlineto{\pgfqpoint{1.744441in}{1.618472in}}%
\pgfpathlineto{\pgfqpoint{1.730309in}{1.618472in}}%
\pgfpathlineto{\pgfqpoint{1.716176in}{1.618472in}}%
\pgfpathlineto{\pgfqpoint{1.702044in}{1.618472in}}%
\pgfpathlineto{\pgfqpoint{1.687911in}{1.618472in}}%
\pgfpathlineto{\pgfqpoint{1.673778in}{1.713540in}}%
\pgfpathlineto{\pgfqpoint{1.659646in}{1.730564in}}%
\pgfpathlineto{\pgfqpoint{1.645513in}{1.734321in}}%
\pgfpathlineto{\pgfqpoint{1.631381in}{1.734321in}}%
\pgfpathlineto{\pgfqpoint{1.617248in}{1.734321in}}%
\pgfpathlineto{\pgfqpoint{1.603115in}{1.751251in}}%
\pgfpathlineto{\pgfqpoint{1.588983in}{1.751251in}}%
\pgfpathlineto{\pgfqpoint{1.574850in}{1.751251in}}%
\pgfpathlineto{\pgfqpoint{1.560718in}{1.751251in}}%
\pgfpathlineto{\pgfqpoint{1.546585in}{1.751251in}}%
\pgfpathlineto{\pgfqpoint{1.532452in}{1.751251in}}%
\pgfpathlineto{\pgfqpoint{1.518320in}{1.751251in}}%
\pgfpathlineto{\pgfqpoint{1.504187in}{1.751251in}}%
\pgfpathlineto{\pgfqpoint{1.490055in}{1.751251in}}%
\pgfpathlineto{\pgfqpoint{1.475922in}{1.751679in}}%
\pgfpathlineto{\pgfqpoint{1.461789in}{1.751679in}}%
\pgfpathlineto{\pgfqpoint{1.447657in}{1.751679in}}%
\pgfpathlineto{\pgfqpoint{1.433524in}{1.751679in}}%
\pgfpathlineto{\pgfqpoint{1.419392in}{1.751679in}}%
\pgfpathlineto{\pgfqpoint{1.405259in}{1.751679in}}%
\pgfpathlineto{\pgfqpoint{1.391126in}{1.751679in}}%
\pgfpathlineto{\pgfqpoint{1.376994in}{1.751679in}}%
\pgfpathlineto{\pgfqpoint{1.362861in}{1.763274in}}%
\pgfpathlineto{\pgfqpoint{1.348729in}{1.765319in}}%
\pgfpathlineto{\pgfqpoint{1.334596in}{1.786687in}}%
\pgfpathlineto{\pgfqpoint{1.320463in}{1.786687in}}%
\pgfpathlineto{\pgfqpoint{1.306331in}{1.786687in}}%
\pgfpathlineto{\pgfqpoint{1.292198in}{1.786687in}}%
\pgfpathlineto{\pgfqpoint{1.278066in}{1.786687in}}%
\pgfpathlineto{\pgfqpoint{1.263933in}{1.786687in}}%
\pgfpathlineto{\pgfqpoint{1.249800in}{1.793379in}}%
\pgfpathlineto{\pgfqpoint{1.235668in}{1.819769in}}%
\pgfpathlineto{\pgfqpoint{1.221535in}{1.821296in}}%
\pgfpathlineto{\pgfqpoint{1.207403in}{1.821296in}}%
\pgfpathlineto{\pgfqpoint{1.193270in}{1.823462in}}%
\pgfpathlineto{\pgfqpoint{1.179137in}{1.824310in}}%
\pgfpathlineto{\pgfqpoint{1.165005in}{1.824310in}}%
\pgfpathlineto{\pgfqpoint{1.150872in}{1.824310in}}%
\pgfpathlineto{\pgfqpoint{1.136740in}{1.824310in}}%
\pgfpathlineto{\pgfqpoint{1.122607in}{1.835690in}}%
\pgfpathlineto{\pgfqpoint{1.108474in}{1.835690in}}%
\pgfpathlineto{\pgfqpoint{1.094342in}{1.847575in}}%
\pgfpathlineto{\pgfqpoint{1.080209in}{1.847575in}}%
\pgfpathlineto{\pgfqpoint{1.066077in}{1.860464in}}%
\pgfpathlineto{\pgfqpoint{1.051944in}{1.860903in}}%
\pgfpathlineto{\pgfqpoint{1.037811in}{1.864264in}}%
\pgfpathlineto{\pgfqpoint{1.023679in}{1.870652in}}%
\pgfpathlineto{\pgfqpoint{1.009546in}{1.910652in}}%
\pgfpathlineto{\pgfqpoint{0.995414in}{1.923293in}}%
\pgfpathlineto{\pgfqpoint{0.981281in}{1.923293in}}%
\pgfpathlineto{\pgfqpoint{0.967148in}{1.925647in}}%
\pgfpathlineto{\pgfqpoint{0.953016in}{1.967057in}}%
\pgfpathlineto{\pgfqpoint{0.938883in}{1.971088in}}%
\pgfpathlineto{\pgfqpoint{0.924751in}{2.010528in}}%
\pgfpathlineto{\pgfqpoint{0.910618in}{2.053951in}}%
\pgfpathlineto{\pgfqpoint{0.896485in}{2.176279in}}%
\pgfpathlineto{\pgfqpoint{0.882353in}{2.201346in}}%
\pgfpathlineto{\pgfqpoint{0.868220in}{2.209204in}}%
\pgfpathlineto{\pgfqpoint{0.854088in}{2.255055in}}%
\pgfpathlineto{\pgfqpoint{0.839955in}{2.274831in}}%
\pgfpathlineto{\pgfqpoint{0.825822in}{2.289799in}}%
\pgfpathlineto{\pgfqpoint{0.811690in}{2.289799in}}%
\pgfpathlineto{\pgfqpoint{0.797557in}{2.289799in}}%
\pgfpathlineto{\pgfqpoint{0.783425in}{2.292354in}}%
\pgfpathlineto{\pgfqpoint{0.769292in}{2.296497in}}%
\pgfpathlineto{\pgfqpoint{0.755159in}{2.296497in}}%
\pgfpathlineto{\pgfqpoint{0.741027in}{2.299901in}}%
\pgfpathlineto{\pgfqpoint{0.726894in}{2.299901in}}%
\pgfpathlineto{\pgfqpoint{0.712762in}{2.318344in}}%
\pgfpathlineto{\pgfqpoint{0.698629in}{2.318344in}}%
\pgfpathlineto{\pgfqpoint{0.684496in}{2.318344in}}%
\pgfpathlineto{\pgfqpoint{0.670364in}{2.318344in}}%
\pgfpathlineto{\pgfqpoint{0.656231in}{2.318344in}}%
\pgfpathlineto{\pgfqpoint{0.642099in}{2.320208in}}%
\pgfpathlineto{\pgfqpoint{0.627966in}{2.324361in}}%
\pgfpathlineto{\pgfqpoint{0.613833in}{2.324361in}}%
\pgfpathlineto{\pgfqpoint{0.599701in}{2.324502in}}%
\pgfpathlineto{\pgfqpoint{0.585568in}{2.368361in}}%
\pgfpathlineto{\pgfqpoint{0.571436in}{2.418679in}}%
\pgfpathlineto{\pgfqpoint{0.557303in}{2.430350in}}%
\pgfpathlineto{\pgfqpoint{0.543170in}{2.478383in}}%
\pgfpathclose%
\pgfusepath{fill}%
\end{pgfscope}%
\begin{pgfscope}%
\pgfpathrectangle{\pgfqpoint{0.423750in}{0.375000in}}{\pgfqpoint{2.627250in}{2.265000in}}%
\pgfusepath{clip}%
\pgfsetbuttcap%
\pgfsetroundjoin%
\definecolor{currentfill}{rgb}{0.839216,0.152941,0.156863}%
\pgfsetfillcolor{currentfill}%
\pgfsetfillopacity{0.200000}%
\pgfsetlinewidth{0.000000pt}%
\definecolor{currentstroke}{rgb}{0.000000,0.000000,0.000000}%
\pgfsetstrokecolor{currentstroke}%
\pgfsetdash{}{0pt}%
\pgfpathmoveto{\pgfqpoint{0.543170in}{2.422879in}}%
\pgfpathlineto{\pgfqpoint{0.543170in}{2.455597in}}%
\pgfpathlineto{\pgfqpoint{0.557303in}{2.429311in}}%
\pgfpathlineto{\pgfqpoint{0.571436in}{2.421186in}}%
\pgfpathlineto{\pgfqpoint{0.585568in}{2.417247in}}%
\pgfpathlineto{\pgfqpoint{0.599701in}{2.386490in}}%
\pgfpathlineto{\pgfqpoint{0.613833in}{2.386490in}}%
\pgfpathlineto{\pgfqpoint{0.627966in}{2.349434in}}%
\pgfpathlineto{\pgfqpoint{0.642099in}{2.349434in}}%
\pgfpathlineto{\pgfqpoint{0.656231in}{2.349434in}}%
\pgfpathlineto{\pgfqpoint{0.670364in}{2.349434in}}%
\pgfpathlineto{\pgfqpoint{0.684496in}{2.346926in}}%
\pgfpathlineto{\pgfqpoint{0.698629in}{2.346926in}}%
\pgfpathlineto{\pgfqpoint{0.712762in}{2.346926in}}%
\pgfpathlineto{\pgfqpoint{0.726894in}{2.346926in}}%
\pgfpathlineto{\pgfqpoint{0.741027in}{2.346926in}}%
\pgfpathlineto{\pgfqpoint{0.755159in}{2.346926in}}%
\pgfpathlineto{\pgfqpoint{0.769292in}{2.346926in}}%
\pgfpathlineto{\pgfqpoint{0.783425in}{2.346926in}}%
\pgfpathlineto{\pgfqpoint{0.797557in}{2.346926in}}%
\pgfpathlineto{\pgfqpoint{0.811690in}{2.346926in}}%
\pgfpathlineto{\pgfqpoint{0.825822in}{2.346926in}}%
\pgfpathlineto{\pgfqpoint{0.839955in}{2.328850in}}%
\pgfpathlineto{\pgfqpoint{0.854088in}{2.328850in}}%
\pgfpathlineto{\pgfqpoint{0.868220in}{2.309907in}}%
\pgfpathlineto{\pgfqpoint{0.882353in}{2.309907in}}%
\pgfpathlineto{\pgfqpoint{0.896485in}{2.298442in}}%
\pgfpathlineto{\pgfqpoint{0.910618in}{2.298442in}}%
\pgfpathlineto{\pgfqpoint{0.924751in}{2.287735in}}%
\pgfpathlineto{\pgfqpoint{0.938883in}{2.245065in}}%
\pgfpathlineto{\pgfqpoint{0.953016in}{2.243818in}}%
\pgfpathlineto{\pgfqpoint{0.967148in}{2.154318in}}%
\pgfpathlineto{\pgfqpoint{0.981281in}{2.140888in}}%
\pgfpathlineto{\pgfqpoint{0.995414in}{2.140888in}}%
\pgfpathlineto{\pgfqpoint{1.009546in}{2.114954in}}%
\pgfpathlineto{\pgfqpoint{1.023679in}{2.105100in}}%
\pgfpathlineto{\pgfqpoint{1.037811in}{2.105100in}}%
\pgfpathlineto{\pgfqpoint{1.051944in}{2.101744in}}%
\pgfpathlineto{\pgfqpoint{1.066077in}{2.098466in}}%
\pgfpathlineto{\pgfqpoint{1.080209in}{2.091461in}}%
\pgfpathlineto{\pgfqpoint{1.094342in}{2.091461in}}%
\pgfpathlineto{\pgfqpoint{1.108474in}{2.079325in}}%
\pgfpathlineto{\pgfqpoint{1.122607in}{2.074136in}}%
\pgfpathlineto{\pgfqpoint{1.136740in}{2.057333in}}%
\pgfpathlineto{\pgfqpoint{1.150872in}{2.050255in}}%
\pgfpathlineto{\pgfqpoint{1.165005in}{2.050255in}}%
\pgfpathlineto{\pgfqpoint{1.179137in}{2.050196in}}%
\pgfpathlineto{\pgfqpoint{1.193270in}{2.040883in}}%
\pgfpathlineto{\pgfqpoint{1.207403in}{2.040883in}}%
\pgfpathlineto{\pgfqpoint{1.221535in}{2.040883in}}%
\pgfpathlineto{\pgfqpoint{1.235668in}{2.040883in}}%
\pgfpathlineto{\pgfqpoint{1.249800in}{2.022099in}}%
\pgfpathlineto{\pgfqpoint{1.263933in}{2.020372in}}%
\pgfpathlineto{\pgfqpoint{1.278066in}{2.020372in}}%
\pgfpathlineto{\pgfqpoint{1.292198in}{2.020372in}}%
\pgfpathlineto{\pgfqpoint{1.306331in}{2.020372in}}%
\pgfpathlineto{\pgfqpoint{1.320463in}{2.010216in}}%
\pgfpathlineto{\pgfqpoint{1.334596in}{2.010216in}}%
\pgfpathlineto{\pgfqpoint{1.348729in}{2.009346in}}%
\pgfpathlineto{\pgfqpoint{1.362861in}{2.009346in}}%
\pgfpathlineto{\pgfqpoint{1.376994in}{2.007771in}}%
\pgfpathlineto{\pgfqpoint{1.391126in}{2.004193in}}%
\pgfpathlineto{\pgfqpoint{1.405259in}{2.004193in}}%
\pgfpathlineto{\pgfqpoint{1.419392in}{2.004193in}}%
\pgfpathlineto{\pgfqpoint{1.433524in}{1.993300in}}%
\pgfpathlineto{\pgfqpoint{1.447657in}{1.992144in}}%
\pgfpathlineto{\pgfqpoint{1.461789in}{1.983070in}}%
\pgfpathlineto{\pgfqpoint{1.475922in}{1.905541in}}%
\pgfpathlineto{\pgfqpoint{1.490055in}{1.905541in}}%
\pgfpathlineto{\pgfqpoint{1.504187in}{1.873111in}}%
\pgfpathlineto{\pgfqpoint{1.518320in}{1.873111in}}%
\pgfpathlineto{\pgfqpoint{1.532452in}{1.868724in}}%
\pgfpathlineto{\pgfqpoint{1.546585in}{1.868724in}}%
\pgfpathlineto{\pgfqpoint{1.560718in}{1.868724in}}%
\pgfpathlineto{\pgfqpoint{1.574850in}{1.868724in}}%
\pgfpathlineto{\pgfqpoint{1.588983in}{1.868724in}}%
\pgfpathlineto{\pgfqpoint{1.603115in}{1.865284in}}%
\pgfpathlineto{\pgfqpoint{1.617248in}{1.854191in}}%
\pgfpathlineto{\pgfqpoint{1.631381in}{1.854191in}}%
\pgfpathlineto{\pgfqpoint{1.645513in}{1.854191in}}%
\pgfpathlineto{\pgfqpoint{1.659646in}{1.854191in}}%
\pgfpathlineto{\pgfqpoint{1.673778in}{1.780829in}}%
\pgfpathlineto{\pgfqpoint{1.687911in}{1.780829in}}%
\pgfpathlineto{\pgfqpoint{1.702044in}{1.717204in}}%
\pgfpathlineto{\pgfqpoint{1.716176in}{1.717204in}}%
\pgfpathlineto{\pgfqpoint{1.730309in}{1.717204in}}%
\pgfpathlineto{\pgfqpoint{1.744441in}{1.717204in}}%
\pgfpathlineto{\pgfqpoint{1.758574in}{1.717204in}}%
\pgfpathlineto{\pgfqpoint{1.772706in}{1.717204in}}%
\pgfpathlineto{\pgfqpoint{1.786839in}{1.717204in}}%
\pgfpathlineto{\pgfqpoint{1.800972in}{1.717204in}}%
\pgfpathlineto{\pgfqpoint{1.815104in}{1.717204in}}%
\pgfpathlineto{\pgfqpoint{1.829237in}{1.717204in}}%
\pgfpathlineto{\pgfqpoint{1.843369in}{1.717204in}}%
\pgfpathlineto{\pgfqpoint{1.857502in}{1.717204in}}%
\pgfpathlineto{\pgfqpoint{1.871635in}{1.717204in}}%
\pgfpathlineto{\pgfqpoint{1.885767in}{1.717204in}}%
\pgfpathlineto{\pgfqpoint{1.899900in}{1.714180in}}%
\pgfpathlineto{\pgfqpoint{1.914032in}{1.714180in}}%
\pgfpathlineto{\pgfqpoint{1.928165in}{1.689932in}}%
\pgfpathlineto{\pgfqpoint{1.942298in}{1.689932in}}%
\pgfpathlineto{\pgfqpoint{1.956430in}{1.689932in}}%
\pgfpathlineto{\pgfqpoint{1.970563in}{1.689932in}}%
\pgfpathlineto{\pgfqpoint{1.984695in}{1.689932in}}%
\pgfpathlineto{\pgfqpoint{1.998828in}{1.532591in}}%
\pgfpathlineto{\pgfqpoint{2.012961in}{1.532591in}}%
\pgfpathlineto{\pgfqpoint{2.027093in}{1.532591in}}%
\pgfpathlineto{\pgfqpoint{2.041226in}{1.532591in}}%
\pgfpathlineto{\pgfqpoint{2.055358in}{1.532591in}}%
\pgfpathlineto{\pgfqpoint{2.069491in}{1.532591in}}%
\pgfpathlineto{\pgfqpoint{2.083624in}{1.532591in}}%
\pgfpathlineto{\pgfqpoint{2.097756in}{1.485490in}}%
\pgfpathlineto{\pgfqpoint{2.111889in}{1.485490in}}%
\pgfpathlineto{\pgfqpoint{2.126021in}{1.485490in}}%
\pgfpathlineto{\pgfqpoint{2.140154in}{1.485490in}}%
\pgfpathlineto{\pgfqpoint{2.154287in}{1.485490in}}%
\pgfpathlineto{\pgfqpoint{2.168419in}{1.485490in}}%
\pgfpathlineto{\pgfqpoint{2.182552in}{1.485490in}}%
\pgfpathlineto{\pgfqpoint{2.196684in}{1.485490in}}%
\pgfpathlineto{\pgfqpoint{2.210817in}{1.485490in}}%
\pgfpathlineto{\pgfqpoint{2.224950in}{1.485490in}}%
\pgfpathlineto{\pgfqpoint{2.239082in}{1.485490in}}%
\pgfpathlineto{\pgfqpoint{2.253215in}{1.485490in}}%
\pgfpathlineto{\pgfqpoint{2.267347in}{1.485490in}}%
\pgfpathlineto{\pgfqpoint{2.281480in}{1.478764in}}%
\pgfpathlineto{\pgfqpoint{2.295613in}{1.478764in}}%
\pgfpathlineto{\pgfqpoint{2.309745in}{1.478764in}}%
\pgfpathlineto{\pgfqpoint{2.323878in}{1.478764in}}%
\pgfpathlineto{\pgfqpoint{2.338010in}{1.478764in}}%
\pgfpathlineto{\pgfqpoint{2.352143in}{1.478764in}}%
\pgfpathlineto{\pgfqpoint{2.366276in}{1.416911in}}%
\pgfpathlineto{\pgfqpoint{2.380408in}{1.285932in}}%
\pgfpathlineto{\pgfqpoint{2.394541in}{1.285932in}}%
\pgfpathlineto{\pgfqpoint{2.408673in}{1.285932in}}%
\pgfpathlineto{\pgfqpoint{2.422806in}{1.285932in}}%
\pgfpathlineto{\pgfqpoint{2.436939in}{1.285932in}}%
\pgfpathlineto{\pgfqpoint{2.451071in}{1.285932in}}%
\pgfpathlineto{\pgfqpoint{2.465204in}{1.265263in}}%
\pgfpathlineto{\pgfqpoint{2.479336in}{1.252393in}}%
\pgfpathlineto{\pgfqpoint{2.493469in}{1.252393in}}%
\pgfpathlineto{\pgfqpoint{2.507602in}{1.252393in}}%
\pgfpathlineto{\pgfqpoint{2.521734in}{1.252393in}}%
\pgfpathlineto{\pgfqpoint{2.535867in}{1.252393in}}%
\pgfpathlineto{\pgfqpoint{2.549999in}{1.252393in}}%
\pgfpathlineto{\pgfqpoint{2.564132in}{1.252393in}}%
\pgfpathlineto{\pgfqpoint{2.578265in}{1.252393in}}%
\pgfpathlineto{\pgfqpoint{2.592397in}{1.252393in}}%
\pgfpathlineto{\pgfqpoint{2.606530in}{1.252393in}}%
\pgfpathlineto{\pgfqpoint{2.620662in}{1.252393in}}%
\pgfpathlineto{\pgfqpoint{2.634795in}{1.248233in}}%
\pgfpathlineto{\pgfqpoint{2.648928in}{1.248233in}}%
\pgfpathlineto{\pgfqpoint{2.663060in}{1.241050in}}%
\pgfpathlineto{\pgfqpoint{2.677193in}{1.241050in}}%
\pgfpathlineto{\pgfqpoint{2.691325in}{1.241050in}}%
\pgfpathlineto{\pgfqpoint{2.705458in}{1.241050in}}%
\pgfpathlineto{\pgfqpoint{2.719591in}{1.230450in}}%
\pgfpathlineto{\pgfqpoint{2.733723in}{1.230450in}}%
\pgfpathlineto{\pgfqpoint{2.747856in}{1.230450in}}%
\pgfpathlineto{\pgfqpoint{2.761988in}{1.230450in}}%
\pgfpathlineto{\pgfqpoint{2.776121in}{1.230450in}}%
\pgfpathlineto{\pgfqpoint{2.790254in}{1.230450in}}%
\pgfpathlineto{\pgfqpoint{2.804386in}{1.230450in}}%
\pgfpathlineto{\pgfqpoint{2.818519in}{1.230450in}}%
\pgfpathlineto{\pgfqpoint{2.832651in}{1.230450in}}%
\pgfpathlineto{\pgfqpoint{2.846784in}{1.230450in}}%
\pgfpathlineto{\pgfqpoint{2.860917in}{1.230450in}}%
\pgfpathlineto{\pgfqpoint{2.875049in}{1.230450in}}%
\pgfpathlineto{\pgfqpoint{2.889182in}{1.230450in}}%
\pgfpathlineto{\pgfqpoint{2.903314in}{1.230450in}}%
\pgfpathlineto{\pgfqpoint{2.917447in}{1.181947in}}%
\pgfpathlineto{\pgfqpoint{2.931580in}{1.181947in}}%
\pgfpathlineto{\pgfqpoint{2.931580in}{1.107579in}}%
\pgfpathlineto{\pgfqpoint{2.931580in}{1.107579in}}%
\pgfpathlineto{\pgfqpoint{2.917447in}{1.107579in}}%
\pgfpathlineto{\pgfqpoint{2.903314in}{1.149309in}}%
\pgfpathlineto{\pgfqpoint{2.889182in}{1.149309in}}%
\pgfpathlineto{\pgfqpoint{2.875049in}{1.149309in}}%
\pgfpathlineto{\pgfqpoint{2.860917in}{1.149309in}}%
\pgfpathlineto{\pgfqpoint{2.846784in}{1.149309in}}%
\pgfpathlineto{\pgfqpoint{2.832651in}{1.149309in}}%
\pgfpathlineto{\pgfqpoint{2.818519in}{1.149309in}}%
\pgfpathlineto{\pgfqpoint{2.804386in}{1.149309in}}%
\pgfpathlineto{\pgfqpoint{2.790254in}{1.149309in}}%
\pgfpathlineto{\pgfqpoint{2.776121in}{1.149309in}}%
\pgfpathlineto{\pgfqpoint{2.761988in}{1.149309in}}%
\pgfpathlineto{\pgfqpoint{2.747856in}{1.149309in}}%
\pgfpathlineto{\pgfqpoint{2.733723in}{1.149309in}}%
\pgfpathlineto{\pgfqpoint{2.719591in}{1.149309in}}%
\pgfpathlineto{\pgfqpoint{2.705458in}{1.152317in}}%
\pgfpathlineto{\pgfqpoint{2.691325in}{1.152317in}}%
\pgfpathlineto{\pgfqpoint{2.677193in}{1.152317in}}%
\pgfpathlineto{\pgfqpoint{2.663060in}{1.152317in}}%
\pgfpathlineto{\pgfqpoint{2.648928in}{1.181696in}}%
\pgfpathlineto{\pgfqpoint{2.634795in}{1.181696in}}%
\pgfpathlineto{\pgfqpoint{2.620662in}{1.182439in}}%
\pgfpathlineto{\pgfqpoint{2.606530in}{1.182439in}}%
\pgfpathlineto{\pgfqpoint{2.592397in}{1.182439in}}%
\pgfpathlineto{\pgfqpoint{2.578265in}{1.182439in}}%
\pgfpathlineto{\pgfqpoint{2.564132in}{1.182439in}}%
\pgfpathlineto{\pgfqpoint{2.549999in}{1.182439in}}%
\pgfpathlineto{\pgfqpoint{2.535867in}{1.182439in}}%
\pgfpathlineto{\pgfqpoint{2.521734in}{1.182439in}}%
\pgfpathlineto{\pgfqpoint{2.507602in}{1.182439in}}%
\pgfpathlineto{\pgfqpoint{2.493469in}{1.182439in}}%
\pgfpathlineto{\pgfqpoint{2.479336in}{1.182439in}}%
\pgfpathlineto{\pgfqpoint{2.465204in}{1.204479in}}%
\pgfpathlineto{\pgfqpoint{2.451071in}{1.245191in}}%
\pgfpathlineto{\pgfqpoint{2.436939in}{1.245191in}}%
\pgfpathlineto{\pgfqpoint{2.422806in}{1.245191in}}%
\pgfpathlineto{\pgfqpoint{2.408673in}{1.245191in}}%
\pgfpathlineto{\pgfqpoint{2.394541in}{1.245191in}}%
\pgfpathlineto{\pgfqpoint{2.380408in}{1.245191in}}%
\pgfpathlineto{\pgfqpoint{2.366276in}{1.281397in}}%
\pgfpathlineto{\pgfqpoint{2.352143in}{1.291439in}}%
\pgfpathlineto{\pgfqpoint{2.338010in}{1.291439in}}%
\pgfpathlineto{\pgfqpoint{2.323878in}{1.291439in}}%
\pgfpathlineto{\pgfqpoint{2.309745in}{1.291439in}}%
\pgfpathlineto{\pgfqpoint{2.295613in}{1.291439in}}%
\pgfpathlineto{\pgfqpoint{2.281480in}{1.291439in}}%
\pgfpathlineto{\pgfqpoint{2.267347in}{1.292717in}}%
\pgfpathlineto{\pgfqpoint{2.253215in}{1.292717in}}%
\pgfpathlineto{\pgfqpoint{2.239082in}{1.292717in}}%
\pgfpathlineto{\pgfqpoint{2.224950in}{1.292717in}}%
\pgfpathlineto{\pgfqpoint{2.210817in}{1.292717in}}%
\pgfpathlineto{\pgfqpoint{2.196684in}{1.292717in}}%
\pgfpathlineto{\pgfqpoint{2.182552in}{1.292717in}}%
\pgfpathlineto{\pgfqpoint{2.168419in}{1.292717in}}%
\pgfpathlineto{\pgfqpoint{2.154287in}{1.292717in}}%
\pgfpathlineto{\pgfqpoint{2.140154in}{1.292717in}}%
\pgfpathlineto{\pgfqpoint{2.126021in}{1.292717in}}%
\pgfpathlineto{\pgfqpoint{2.111889in}{1.292717in}}%
\pgfpathlineto{\pgfqpoint{2.097756in}{1.292717in}}%
\pgfpathlineto{\pgfqpoint{2.083624in}{1.302982in}}%
\pgfpathlineto{\pgfqpoint{2.069491in}{1.302982in}}%
\pgfpathlineto{\pgfqpoint{2.055358in}{1.302982in}}%
\pgfpathlineto{\pgfqpoint{2.041226in}{1.302982in}}%
\pgfpathlineto{\pgfqpoint{2.027093in}{1.302982in}}%
\pgfpathlineto{\pgfqpoint{2.012961in}{1.302982in}}%
\pgfpathlineto{\pgfqpoint{1.998828in}{1.302982in}}%
\pgfpathlineto{\pgfqpoint{1.984695in}{1.359134in}}%
\pgfpathlineto{\pgfqpoint{1.970563in}{1.359134in}}%
\pgfpathlineto{\pgfqpoint{1.956430in}{1.359134in}}%
\pgfpathlineto{\pgfqpoint{1.942298in}{1.359134in}}%
\pgfpathlineto{\pgfqpoint{1.928165in}{1.359134in}}%
\pgfpathlineto{\pgfqpoint{1.914032in}{1.371336in}}%
\pgfpathlineto{\pgfqpoint{1.899900in}{1.371336in}}%
\pgfpathlineto{\pgfqpoint{1.885767in}{1.372928in}}%
\pgfpathlineto{\pgfqpoint{1.871635in}{1.372928in}}%
\pgfpathlineto{\pgfqpoint{1.857502in}{1.372928in}}%
\pgfpathlineto{\pgfqpoint{1.843369in}{1.372928in}}%
\pgfpathlineto{\pgfqpoint{1.829237in}{1.372928in}}%
\pgfpathlineto{\pgfqpoint{1.815104in}{1.372928in}}%
\pgfpathlineto{\pgfqpoint{1.800972in}{1.372928in}}%
\pgfpathlineto{\pgfqpoint{1.786839in}{1.372928in}}%
\pgfpathlineto{\pgfqpoint{1.772706in}{1.372928in}}%
\pgfpathlineto{\pgfqpoint{1.758574in}{1.372928in}}%
\pgfpathlineto{\pgfqpoint{1.744441in}{1.372928in}}%
\pgfpathlineto{\pgfqpoint{1.730309in}{1.372928in}}%
\pgfpathlineto{\pgfqpoint{1.716176in}{1.372928in}}%
\pgfpathlineto{\pgfqpoint{1.702044in}{1.372928in}}%
\pgfpathlineto{\pgfqpoint{1.687911in}{1.409975in}}%
\pgfpathlineto{\pgfqpoint{1.673778in}{1.409975in}}%
\pgfpathlineto{\pgfqpoint{1.659646in}{1.460556in}}%
\pgfpathlineto{\pgfqpoint{1.645513in}{1.460556in}}%
\pgfpathlineto{\pgfqpoint{1.631381in}{1.460556in}}%
\pgfpathlineto{\pgfqpoint{1.617248in}{1.460556in}}%
\pgfpathlineto{\pgfqpoint{1.603115in}{1.468859in}}%
\pgfpathlineto{\pgfqpoint{1.588983in}{1.533004in}}%
\pgfpathlineto{\pgfqpoint{1.574850in}{1.533004in}}%
\pgfpathlineto{\pgfqpoint{1.560718in}{1.533004in}}%
\pgfpathlineto{\pgfqpoint{1.546585in}{1.533004in}}%
\pgfpathlineto{\pgfqpoint{1.532452in}{1.533004in}}%
\pgfpathlineto{\pgfqpoint{1.518320in}{1.590256in}}%
\pgfpathlineto{\pgfqpoint{1.504187in}{1.590256in}}%
\pgfpathlineto{\pgfqpoint{1.490055in}{1.602640in}}%
\pgfpathlineto{\pgfqpoint{1.475922in}{1.602640in}}%
\pgfpathlineto{\pgfqpoint{1.461789in}{1.638924in}}%
\pgfpathlineto{\pgfqpoint{1.447657in}{1.643815in}}%
\pgfpathlineto{\pgfqpoint{1.433524in}{1.663428in}}%
\pgfpathlineto{\pgfqpoint{1.419392in}{1.668764in}}%
\pgfpathlineto{\pgfqpoint{1.405259in}{1.668764in}}%
\pgfpathlineto{\pgfqpoint{1.391126in}{1.668764in}}%
\pgfpathlineto{\pgfqpoint{1.376994in}{1.714348in}}%
\pgfpathlineto{\pgfqpoint{1.362861in}{1.732393in}}%
\pgfpathlineto{\pgfqpoint{1.348729in}{1.732393in}}%
\pgfpathlineto{\pgfqpoint{1.334596in}{1.741146in}}%
\pgfpathlineto{\pgfqpoint{1.320463in}{1.741146in}}%
\pgfpathlineto{\pgfqpoint{1.306331in}{1.744545in}}%
\pgfpathlineto{\pgfqpoint{1.292198in}{1.744545in}}%
\pgfpathlineto{\pgfqpoint{1.278066in}{1.744545in}}%
\pgfpathlineto{\pgfqpoint{1.263933in}{1.744545in}}%
\pgfpathlineto{\pgfqpoint{1.249800in}{1.760671in}}%
\pgfpathlineto{\pgfqpoint{1.235668in}{1.768047in}}%
\pgfpathlineto{\pgfqpoint{1.221535in}{1.768047in}}%
\pgfpathlineto{\pgfqpoint{1.207403in}{1.768047in}}%
\pgfpathlineto{\pgfqpoint{1.193270in}{1.768047in}}%
\pgfpathlineto{\pgfqpoint{1.179137in}{1.819954in}}%
\pgfpathlineto{\pgfqpoint{1.165005in}{1.820523in}}%
\pgfpathlineto{\pgfqpoint{1.150872in}{1.820523in}}%
\pgfpathlineto{\pgfqpoint{1.136740in}{1.845883in}}%
\pgfpathlineto{\pgfqpoint{1.122607in}{1.850312in}}%
\pgfpathlineto{\pgfqpoint{1.108474in}{1.880729in}}%
\pgfpathlineto{\pgfqpoint{1.094342in}{1.918486in}}%
\pgfpathlineto{\pgfqpoint{1.080209in}{1.918486in}}%
\pgfpathlineto{\pgfqpoint{1.066077in}{1.933020in}}%
\pgfpathlineto{\pgfqpoint{1.051944in}{1.954051in}}%
\pgfpathlineto{\pgfqpoint{1.037811in}{1.971911in}}%
\pgfpathlineto{\pgfqpoint{1.023679in}{1.971911in}}%
\pgfpathlineto{\pgfqpoint{1.009546in}{1.974588in}}%
\pgfpathlineto{\pgfqpoint{0.995414in}{2.036027in}}%
\pgfpathlineto{\pgfqpoint{0.981281in}{2.036027in}}%
\pgfpathlineto{\pgfqpoint{0.967148in}{2.049319in}}%
\pgfpathlineto{\pgfqpoint{0.953016in}{2.151024in}}%
\pgfpathlineto{\pgfqpoint{0.938883in}{2.161202in}}%
\pgfpathlineto{\pgfqpoint{0.924751in}{2.182020in}}%
\pgfpathlineto{\pgfqpoint{0.910618in}{2.191922in}}%
\pgfpathlineto{\pgfqpoint{0.896485in}{2.191922in}}%
\pgfpathlineto{\pgfqpoint{0.882353in}{2.218232in}}%
\pgfpathlineto{\pgfqpoint{0.868220in}{2.218232in}}%
\pgfpathlineto{\pgfqpoint{0.854088in}{2.230229in}}%
\pgfpathlineto{\pgfqpoint{0.839955in}{2.230229in}}%
\pgfpathlineto{\pgfqpoint{0.825822in}{2.237708in}}%
\pgfpathlineto{\pgfqpoint{0.811690in}{2.237708in}}%
\pgfpathlineto{\pgfqpoint{0.797557in}{2.237708in}}%
\pgfpathlineto{\pgfqpoint{0.783425in}{2.237708in}}%
\pgfpathlineto{\pgfqpoint{0.769292in}{2.237708in}}%
\pgfpathlineto{\pgfqpoint{0.755159in}{2.237708in}}%
\pgfpathlineto{\pgfqpoint{0.741027in}{2.237708in}}%
\pgfpathlineto{\pgfqpoint{0.726894in}{2.237708in}}%
\pgfpathlineto{\pgfqpoint{0.712762in}{2.237708in}}%
\pgfpathlineto{\pgfqpoint{0.698629in}{2.237708in}}%
\pgfpathlineto{\pgfqpoint{0.684496in}{2.237708in}}%
\pgfpathlineto{\pgfqpoint{0.670364in}{2.238576in}}%
\pgfpathlineto{\pgfqpoint{0.656231in}{2.238576in}}%
\pgfpathlineto{\pgfqpoint{0.642099in}{2.238576in}}%
\pgfpathlineto{\pgfqpoint{0.627966in}{2.238576in}}%
\pgfpathlineto{\pgfqpoint{0.613833in}{2.282539in}}%
\pgfpathlineto{\pgfqpoint{0.599701in}{2.282539in}}%
\pgfpathlineto{\pgfqpoint{0.585568in}{2.346355in}}%
\pgfpathlineto{\pgfqpoint{0.571436in}{2.390801in}}%
\pgfpathlineto{\pgfqpoint{0.557303in}{2.406711in}}%
\pgfpathlineto{\pgfqpoint{0.543170in}{2.422879in}}%
\pgfpathclose%
\pgfusepath{fill}%
\end{pgfscope}%
\begin{pgfscope}%
\pgfpathrectangle{\pgfqpoint{0.423750in}{0.375000in}}{\pgfqpoint{2.627250in}{2.265000in}}%
\pgfusepath{clip}%
\pgfsetbuttcap%
\pgfsetroundjoin%
\definecolor{currentfill}{rgb}{0.580392,0.403922,0.741176}%
\pgfsetfillcolor{currentfill}%
\pgfsetfillopacity{0.200000}%
\pgfsetlinewidth{0.000000pt}%
\definecolor{currentstroke}{rgb}{0.000000,0.000000,0.000000}%
\pgfsetstrokecolor{currentstroke}%
\pgfsetdash{}{0pt}%
\pgfpathmoveto{\pgfqpoint{0.543170in}{2.492825in}}%
\pgfpathlineto{\pgfqpoint{0.543170in}{2.524822in}}%
\pgfpathlineto{\pgfqpoint{0.557303in}{2.503074in}}%
\pgfpathlineto{\pgfqpoint{0.571436in}{2.480032in}}%
\pgfpathlineto{\pgfqpoint{0.585568in}{2.473542in}}%
\pgfpathlineto{\pgfqpoint{0.599701in}{2.473542in}}%
\pgfpathlineto{\pgfqpoint{0.613833in}{2.455212in}}%
\pgfpathlineto{\pgfqpoint{0.627966in}{2.455212in}}%
\pgfpathlineto{\pgfqpoint{0.642099in}{2.446374in}}%
\pgfpathlineto{\pgfqpoint{0.656231in}{2.446374in}}%
\pgfpathlineto{\pgfqpoint{0.670364in}{2.435330in}}%
\pgfpathlineto{\pgfqpoint{0.684496in}{2.435330in}}%
\pgfpathlineto{\pgfqpoint{0.698629in}{2.413537in}}%
\pgfpathlineto{\pgfqpoint{0.712762in}{2.409250in}}%
\pgfpathlineto{\pgfqpoint{0.726894in}{2.409250in}}%
\pgfpathlineto{\pgfqpoint{0.741027in}{2.409250in}}%
\pgfpathlineto{\pgfqpoint{0.755159in}{2.401127in}}%
\pgfpathlineto{\pgfqpoint{0.769292in}{2.351728in}}%
\pgfpathlineto{\pgfqpoint{0.783425in}{2.346576in}}%
\pgfpathlineto{\pgfqpoint{0.797557in}{2.326028in}}%
\pgfpathlineto{\pgfqpoint{0.811690in}{2.326028in}}%
\pgfpathlineto{\pgfqpoint{0.825822in}{2.326028in}}%
\pgfpathlineto{\pgfqpoint{0.839955in}{2.290991in}}%
\pgfpathlineto{\pgfqpoint{0.854088in}{2.283190in}}%
\pgfpathlineto{\pgfqpoint{0.868220in}{2.252766in}}%
\pgfpathlineto{\pgfqpoint{0.882353in}{2.191509in}}%
\pgfpathlineto{\pgfqpoint{0.896485in}{2.188966in}}%
\pgfpathlineto{\pgfqpoint{0.910618in}{2.180993in}}%
\pgfpathlineto{\pgfqpoint{0.924751in}{2.172221in}}%
\pgfpathlineto{\pgfqpoint{0.938883in}{2.169511in}}%
\pgfpathlineto{\pgfqpoint{0.953016in}{2.169101in}}%
\pgfpathlineto{\pgfqpoint{0.967148in}{2.169101in}}%
\pgfpathlineto{\pgfqpoint{0.981281in}{2.168060in}}%
\pgfpathlineto{\pgfqpoint{0.995414in}{2.168060in}}%
\pgfpathlineto{\pgfqpoint{1.009546in}{2.114348in}}%
\pgfpathlineto{\pgfqpoint{1.023679in}{2.108836in}}%
\pgfpathlineto{\pgfqpoint{1.037811in}{2.105628in}}%
\pgfpathlineto{\pgfqpoint{1.051944in}{2.103939in}}%
\pgfpathlineto{\pgfqpoint{1.066077in}{2.097403in}}%
\pgfpathlineto{\pgfqpoint{1.080209in}{2.097403in}}%
\pgfpathlineto{\pgfqpoint{1.094342in}{2.097403in}}%
\pgfpathlineto{\pgfqpoint{1.108474in}{2.097403in}}%
\pgfpathlineto{\pgfqpoint{1.122607in}{2.097403in}}%
\pgfpathlineto{\pgfqpoint{1.136740in}{2.023066in}}%
\pgfpathlineto{\pgfqpoint{1.150872in}{2.022879in}}%
\pgfpathlineto{\pgfqpoint{1.165005in}{2.014950in}}%
\pgfpathlineto{\pgfqpoint{1.179137in}{2.014950in}}%
\pgfpathlineto{\pgfqpoint{1.193270in}{2.014950in}}%
\pgfpathlineto{\pgfqpoint{1.207403in}{2.014950in}}%
\pgfpathlineto{\pgfqpoint{1.221535in}{2.014950in}}%
\pgfpathlineto{\pgfqpoint{1.235668in}{2.014950in}}%
\pgfpathlineto{\pgfqpoint{1.249800in}{2.014950in}}%
\pgfpathlineto{\pgfqpoint{1.263933in}{2.014235in}}%
\pgfpathlineto{\pgfqpoint{1.278066in}{2.014235in}}%
\pgfpathlineto{\pgfqpoint{1.292198in}{2.014222in}}%
\pgfpathlineto{\pgfqpoint{1.306331in}{2.014187in}}%
\pgfpathlineto{\pgfqpoint{1.320463in}{2.014187in}}%
\pgfpathlineto{\pgfqpoint{1.334596in}{2.012211in}}%
\pgfpathlineto{\pgfqpoint{1.348729in}{2.011918in}}%
\pgfpathlineto{\pgfqpoint{1.362861in}{2.010302in}}%
\pgfpathlineto{\pgfqpoint{1.376994in}{2.009692in}}%
\pgfpathlineto{\pgfqpoint{1.391126in}{2.009586in}}%
\pgfpathlineto{\pgfqpoint{1.405259in}{2.009586in}}%
\pgfpathlineto{\pgfqpoint{1.419392in}{2.009457in}}%
\pgfpathlineto{\pgfqpoint{1.433524in}{2.009413in}}%
\pgfpathlineto{\pgfqpoint{1.447657in}{1.977571in}}%
\pgfpathlineto{\pgfqpoint{1.461789in}{1.977571in}}%
\pgfpathlineto{\pgfqpoint{1.475922in}{1.964228in}}%
\pgfpathlineto{\pgfqpoint{1.490055in}{1.963607in}}%
\pgfpathlineto{\pgfqpoint{1.504187in}{1.830743in}}%
\pgfpathlineto{\pgfqpoint{1.518320in}{1.829563in}}%
\pgfpathlineto{\pgfqpoint{1.532452in}{1.829563in}}%
\pgfpathlineto{\pgfqpoint{1.546585in}{1.829480in}}%
\pgfpathlineto{\pgfqpoint{1.560718in}{1.829480in}}%
\pgfpathlineto{\pgfqpoint{1.574850in}{1.829255in}}%
\pgfpathlineto{\pgfqpoint{1.588983in}{1.800083in}}%
\pgfpathlineto{\pgfqpoint{1.603115in}{1.798597in}}%
\pgfpathlineto{\pgfqpoint{1.617248in}{1.775258in}}%
\pgfpathlineto{\pgfqpoint{1.631381in}{1.774363in}}%
\pgfpathlineto{\pgfqpoint{1.645513in}{1.771951in}}%
\pgfpathlineto{\pgfqpoint{1.659646in}{1.768591in}}%
\pgfpathlineto{\pgfqpoint{1.673778in}{1.767419in}}%
\pgfpathlineto{\pgfqpoint{1.687911in}{1.765767in}}%
\pgfpathlineto{\pgfqpoint{1.702044in}{1.719712in}}%
\pgfpathlineto{\pgfqpoint{1.716176in}{1.692346in}}%
\pgfpathlineto{\pgfqpoint{1.730309in}{1.687601in}}%
\pgfpathlineto{\pgfqpoint{1.744441in}{1.682996in}}%
\pgfpathlineto{\pgfqpoint{1.758574in}{1.671847in}}%
\pgfpathlineto{\pgfqpoint{1.772706in}{1.620940in}}%
\pgfpathlineto{\pgfqpoint{1.786839in}{1.477440in}}%
\pgfpathlineto{\pgfqpoint{1.800972in}{1.455574in}}%
\pgfpathlineto{\pgfqpoint{1.815104in}{1.420512in}}%
\pgfpathlineto{\pgfqpoint{1.829237in}{1.419224in}}%
\pgfpathlineto{\pgfqpoint{1.843369in}{1.415760in}}%
\pgfpathlineto{\pgfqpoint{1.857502in}{1.413080in}}%
\pgfpathlineto{\pgfqpoint{1.871635in}{1.411646in}}%
\pgfpathlineto{\pgfqpoint{1.885767in}{1.348789in}}%
\pgfpathlineto{\pgfqpoint{1.899900in}{1.216920in}}%
\pgfpathlineto{\pgfqpoint{1.914032in}{1.004642in}}%
\pgfpathlineto{\pgfqpoint{1.928165in}{0.989669in}}%
\pgfpathlineto{\pgfqpoint{1.942298in}{0.972642in}}%
\pgfpathlineto{\pgfqpoint{1.956430in}{0.967321in}}%
\pgfpathlineto{\pgfqpoint{1.970563in}{0.907405in}}%
\pgfpathlineto{\pgfqpoint{1.984695in}{0.906024in}}%
\pgfpathlineto{\pgfqpoint{1.998828in}{0.906024in}}%
\pgfpathlineto{\pgfqpoint{2.012961in}{0.901609in}}%
\pgfpathlineto{\pgfqpoint{2.027093in}{0.900866in}}%
\pgfpathlineto{\pgfqpoint{2.041226in}{0.900643in}}%
\pgfpathlineto{\pgfqpoint{2.055358in}{0.897553in}}%
\pgfpathlineto{\pgfqpoint{2.069491in}{0.881462in}}%
\pgfpathlineto{\pgfqpoint{2.083624in}{0.864095in}}%
\pgfpathlineto{\pgfqpoint{2.097756in}{0.857229in}}%
\pgfpathlineto{\pgfqpoint{2.111889in}{0.845464in}}%
\pgfpathlineto{\pgfqpoint{2.126021in}{0.826952in}}%
\pgfpathlineto{\pgfqpoint{2.140154in}{0.819890in}}%
\pgfpathlineto{\pgfqpoint{2.154287in}{0.812826in}}%
\pgfpathlineto{\pgfqpoint{2.168419in}{0.805248in}}%
\pgfpathlineto{\pgfqpoint{2.182552in}{0.789737in}}%
\pgfpathlineto{\pgfqpoint{2.196684in}{0.783288in}}%
\pgfpathlineto{\pgfqpoint{2.210817in}{0.769872in}}%
\pgfpathlineto{\pgfqpoint{2.224950in}{0.758752in}}%
\pgfpathlineto{\pgfqpoint{2.239082in}{0.750508in}}%
\pgfpathlineto{\pgfqpoint{2.253215in}{0.741991in}}%
\pgfpathlineto{\pgfqpoint{2.267347in}{0.736687in}}%
\pgfpathlineto{\pgfqpoint{2.281480in}{0.732361in}}%
\pgfpathlineto{\pgfqpoint{2.295613in}{0.727370in}}%
\pgfpathlineto{\pgfqpoint{2.309745in}{0.721634in}}%
\pgfpathlineto{\pgfqpoint{2.323878in}{0.717743in}}%
\pgfpathlineto{\pgfqpoint{2.338010in}{0.712346in}}%
\pgfpathlineto{\pgfqpoint{2.352143in}{0.707574in}}%
\pgfpathlineto{\pgfqpoint{2.366276in}{0.703340in}}%
\pgfpathlineto{\pgfqpoint{2.380408in}{0.700772in}}%
\pgfpathlineto{\pgfqpoint{2.394541in}{0.696045in}}%
\pgfpathlineto{\pgfqpoint{2.408673in}{0.693035in}}%
\pgfpathlineto{\pgfqpoint{2.422806in}{0.689694in}}%
\pgfpathlineto{\pgfqpoint{2.436939in}{0.688926in}}%
\pgfpathlineto{\pgfqpoint{2.451071in}{0.688460in}}%
\pgfpathlineto{\pgfqpoint{2.465204in}{0.687638in}}%
\pgfpathlineto{\pgfqpoint{2.479336in}{0.683611in}}%
\pgfpathlineto{\pgfqpoint{2.493469in}{0.682945in}}%
\pgfpathlineto{\pgfqpoint{2.507602in}{0.682726in}}%
\pgfpathlineto{\pgfqpoint{2.521734in}{0.681357in}}%
\pgfpathlineto{\pgfqpoint{2.535867in}{0.680005in}}%
\pgfpathlineto{\pgfqpoint{2.549999in}{0.677296in}}%
\pgfpathlineto{\pgfqpoint{2.564132in}{0.676547in}}%
\pgfpathlineto{\pgfqpoint{2.578265in}{0.676287in}}%
\pgfpathlineto{\pgfqpoint{2.592397in}{0.675852in}}%
\pgfpathlineto{\pgfqpoint{2.606530in}{0.674505in}}%
\pgfpathlineto{\pgfqpoint{2.620662in}{0.674369in}}%
\pgfpathlineto{\pgfqpoint{2.634795in}{0.674149in}}%
\pgfpathlineto{\pgfqpoint{2.648928in}{0.673378in}}%
\pgfpathlineto{\pgfqpoint{2.663060in}{0.673110in}}%
\pgfpathlineto{\pgfqpoint{2.677193in}{0.673043in}}%
\pgfpathlineto{\pgfqpoint{2.691325in}{0.673043in}}%
\pgfpathlineto{\pgfqpoint{2.705458in}{0.671777in}}%
\pgfpathlineto{\pgfqpoint{2.719591in}{0.671777in}}%
\pgfpathlineto{\pgfqpoint{2.733723in}{0.671774in}}%
\pgfpathlineto{\pgfqpoint{2.747856in}{0.670846in}}%
\pgfpathlineto{\pgfqpoint{2.761988in}{0.670103in}}%
\pgfpathlineto{\pgfqpoint{2.776121in}{0.670026in}}%
\pgfpathlineto{\pgfqpoint{2.790254in}{0.669940in}}%
\pgfpathlineto{\pgfqpoint{2.804386in}{0.669940in}}%
\pgfpathlineto{\pgfqpoint{2.818519in}{0.669940in}}%
\pgfpathlineto{\pgfqpoint{2.832651in}{0.669654in}}%
\pgfpathlineto{\pgfqpoint{2.846784in}{0.667991in}}%
\pgfpathlineto{\pgfqpoint{2.860917in}{0.666416in}}%
\pgfpathlineto{\pgfqpoint{2.875049in}{0.660826in}}%
\pgfpathlineto{\pgfqpoint{2.889182in}{0.660826in}}%
\pgfpathlineto{\pgfqpoint{2.903314in}{0.660812in}}%
\pgfpathlineto{\pgfqpoint{2.917447in}{0.660794in}}%
\pgfpathlineto{\pgfqpoint{2.931580in}{0.660772in}}%
\pgfpathlineto{\pgfqpoint{2.931580in}{0.477955in}}%
\pgfpathlineto{\pgfqpoint{2.931580in}{0.477955in}}%
\pgfpathlineto{\pgfqpoint{2.917447in}{0.477959in}}%
\pgfpathlineto{\pgfqpoint{2.903314in}{0.477964in}}%
\pgfpathlineto{\pgfqpoint{2.889182in}{0.478015in}}%
\pgfpathlineto{\pgfqpoint{2.875049in}{0.478015in}}%
\pgfpathlineto{\pgfqpoint{2.860917in}{0.503668in}}%
\pgfpathlineto{\pgfqpoint{2.846784in}{0.509398in}}%
\pgfpathlineto{\pgfqpoint{2.832651in}{0.510339in}}%
\pgfpathlineto{\pgfqpoint{2.818519in}{0.510986in}}%
\pgfpathlineto{\pgfqpoint{2.804386in}{0.510986in}}%
\pgfpathlineto{\pgfqpoint{2.790254in}{0.510986in}}%
\pgfpathlineto{\pgfqpoint{2.776121in}{0.511005in}}%
\pgfpathlineto{\pgfqpoint{2.761988in}{0.511257in}}%
\pgfpathlineto{\pgfqpoint{2.747856in}{0.512390in}}%
\pgfpathlineto{\pgfqpoint{2.733723in}{0.513665in}}%
\pgfpathlineto{\pgfqpoint{2.719591in}{0.513674in}}%
\pgfpathlineto{\pgfqpoint{2.705458in}{0.513674in}}%
\pgfpathlineto{\pgfqpoint{2.691325in}{0.514727in}}%
\pgfpathlineto{\pgfqpoint{2.677193in}{0.514727in}}%
\pgfpathlineto{\pgfqpoint{2.663060in}{0.514939in}}%
\pgfpathlineto{\pgfqpoint{2.648928in}{0.515350in}}%
\pgfpathlineto{\pgfqpoint{2.634795in}{0.516737in}}%
\pgfpathlineto{\pgfqpoint{2.620662in}{0.517411in}}%
\pgfpathlineto{\pgfqpoint{2.606530in}{0.517826in}}%
\pgfpathlineto{\pgfqpoint{2.592397in}{0.518213in}}%
\pgfpathlineto{\pgfqpoint{2.578265in}{0.519537in}}%
\pgfpathlineto{\pgfqpoint{2.564132in}{0.520311in}}%
\pgfpathlineto{\pgfqpoint{2.549999in}{0.522877in}}%
\pgfpathlineto{\pgfqpoint{2.535867in}{0.527248in}}%
\pgfpathlineto{\pgfqpoint{2.521734in}{0.530531in}}%
\pgfpathlineto{\pgfqpoint{2.507602in}{0.533726in}}%
\pgfpathlineto{\pgfqpoint{2.493469in}{0.534110in}}%
\pgfpathlineto{\pgfqpoint{2.479336in}{0.535338in}}%
\pgfpathlineto{\pgfqpoint{2.465204in}{0.544016in}}%
\pgfpathlineto{\pgfqpoint{2.451071in}{0.545497in}}%
\pgfpathlineto{\pgfqpoint{2.436939in}{0.546683in}}%
\pgfpathlineto{\pgfqpoint{2.422806in}{0.548372in}}%
\pgfpathlineto{\pgfqpoint{2.408673in}{0.554473in}}%
\pgfpathlineto{\pgfqpoint{2.394541in}{0.561940in}}%
\pgfpathlineto{\pgfqpoint{2.380408in}{0.569030in}}%
\pgfpathlineto{\pgfqpoint{2.366276in}{0.574887in}}%
\pgfpathlineto{\pgfqpoint{2.352143in}{0.582396in}}%
\pgfpathlineto{\pgfqpoint{2.338010in}{0.590644in}}%
\pgfpathlineto{\pgfqpoint{2.323878in}{0.599469in}}%
\pgfpathlineto{\pgfqpoint{2.309745in}{0.606162in}}%
\pgfpathlineto{\pgfqpoint{2.295613in}{0.615176in}}%
\pgfpathlineto{\pgfqpoint{2.281480in}{0.621910in}}%
\pgfpathlineto{\pgfqpoint{2.267347in}{0.627164in}}%
\pgfpathlineto{\pgfqpoint{2.253215in}{0.632301in}}%
\pgfpathlineto{\pgfqpoint{2.239082in}{0.639425in}}%
\pgfpathlineto{\pgfqpoint{2.224950in}{0.645878in}}%
\pgfpathlineto{\pgfqpoint{2.210817in}{0.654034in}}%
\pgfpathlineto{\pgfqpoint{2.196684in}{0.664070in}}%
\pgfpathlineto{\pgfqpoint{2.182552in}{0.668991in}}%
\pgfpathlineto{\pgfqpoint{2.168419in}{0.680489in}}%
\pgfpathlineto{\pgfqpoint{2.154287in}{0.686170in}}%
\pgfpathlineto{\pgfqpoint{2.140154in}{0.691112in}}%
\pgfpathlineto{\pgfqpoint{2.126021in}{0.695542in}}%
\pgfpathlineto{\pgfqpoint{2.111889in}{0.710979in}}%
\pgfpathlineto{\pgfqpoint{2.097756in}{0.719758in}}%
\pgfpathlineto{\pgfqpoint{2.083624in}{0.724011in}}%
\pgfpathlineto{\pgfqpoint{2.069491in}{0.738297in}}%
\pgfpathlineto{\pgfqpoint{2.055358in}{0.751880in}}%
\pgfpathlineto{\pgfqpoint{2.041226in}{0.754782in}}%
\pgfpathlineto{\pgfqpoint{2.027093in}{0.756219in}}%
\pgfpathlineto{\pgfqpoint{2.012961in}{0.760055in}}%
\pgfpathlineto{\pgfqpoint{1.998828in}{0.778277in}}%
\pgfpathlineto{\pgfqpoint{1.984695in}{0.778277in}}%
\pgfpathlineto{\pgfqpoint{1.970563in}{0.782689in}}%
\pgfpathlineto{\pgfqpoint{1.956430in}{0.817241in}}%
\pgfpathlineto{\pgfqpoint{1.942298in}{0.834303in}}%
\pgfpathlineto{\pgfqpoint{1.928165in}{0.868369in}}%
\pgfpathlineto{\pgfqpoint{1.914032in}{0.894142in}}%
\pgfpathlineto{\pgfqpoint{1.899900in}{0.982821in}}%
\pgfpathlineto{\pgfqpoint{1.885767in}{1.050015in}}%
\pgfpathlineto{\pgfqpoint{1.871635in}{1.096770in}}%
\pgfpathlineto{\pgfqpoint{1.857502in}{1.113892in}}%
\pgfpathlineto{\pgfqpoint{1.843369in}{1.141902in}}%
\pgfpathlineto{\pgfqpoint{1.829237in}{1.168572in}}%
\pgfpathlineto{\pgfqpoint{1.815104in}{1.180940in}}%
\pgfpathlineto{\pgfqpoint{1.800972in}{1.284247in}}%
\pgfpathlineto{\pgfqpoint{1.786839in}{1.352820in}}%
\pgfpathlineto{\pgfqpoint{1.772706in}{1.429808in}}%
\pgfpathlineto{\pgfqpoint{1.758574in}{1.444367in}}%
\pgfpathlineto{\pgfqpoint{1.744441in}{1.468897in}}%
\pgfpathlineto{\pgfqpoint{1.730309in}{1.470113in}}%
\pgfpathlineto{\pgfqpoint{1.716176in}{1.471379in}}%
\pgfpathlineto{\pgfqpoint{1.702044in}{1.547457in}}%
\pgfpathlineto{\pgfqpoint{1.687911in}{1.634207in}}%
\pgfpathlineto{\pgfqpoint{1.673778in}{1.635381in}}%
\pgfpathlineto{\pgfqpoint{1.659646in}{1.636280in}}%
\pgfpathlineto{\pgfqpoint{1.645513in}{1.638792in}}%
\pgfpathlineto{\pgfqpoint{1.631381in}{1.645757in}}%
\pgfpathlineto{\pgfqpoint{1.617248in}{1.646543in}}%
\pgfpathlineto{\pgfqpoint{1.603115in}{1.683606in}}%
\pgfpathlineto{\pgfqpoint{1.588983in}{1.684796in}}%
\pgfpathlineto{\pgfqpoint{1.574850in}{1.702637in}}%
\pgfpathlineto{\pgfqpoint{1.560718in}{1.704386in}}%
\pgfpathlineto{\pgfqpoint{1.546585in}{1.704386in}}%
\pgfpathlineto{\pgfqpoint{1.532452in}{1.705021in}}%
\pgfpathlineto{\pgfqpoint{1.518320in}{1.705021in}}%
\pgfpathlineto{\pgfqpoint{1.504187in}{1.714110in}}%
\pgfpathlineto{\pgfqpoint{1.490055in}{1.752037in}}%
\pgfpathlineto{\pgfqpoint{1.475922in}{1.757255in}}%
\pgfpathlineto{\pgfqpoint{1.461789in}{1.760476in}}%
\pgfpathlineto{\pgfqpoint{1.447657in}{1.760476in}}%
\pgfpathlineto{\pgfqpoint{1.433524in}{1.768802in}}%
\pgfpathlineto{\pgfqpoint{1.419392in}{1.769230in}}%
\pgfpathlineto{\pgfqpoint{1.405259in}{1.770465in}}%
\pgfpathlineto{\pgfqpoint{1.391126in}{1.770465in}}%
\pgfpathlineto{\pgfqpoint{1.376994in}{1.771469in}}%
\pgfpathlineto{\pgfqpoint{1.362861in}{1.776409in}}%
\pgfpathlineto{\pgfqpoint{1.348729in}{1.789599in}}%
\pgfpathlineto{\pgfqpoint{1.334596in}{1.791505in}}%
\pgfpathlineto{\pgfqpoint{1.320463in}{1.805762in}}%
\pgfpathlineto{\pgfqpoint{1.306331in}{1.805762in}}%
\pgfpathlineto{\pgfqpoint{1.292198in}{1.806053in}}%
\pgfpathlineto{\pgfqpoint{1.278066in}{1.806170in}}%
\pgfpathlineto{\pgfqpoint{1.263933in}{1.806170in}}%
\pgfpathlineto{\pgfqpoint{1.249800in}{1.811995in}}%
\pgfpathlineto{\pgfqpoint{1.235668in}{1.811995in}}%
\pgfpathlineto{\pgfqpoint{1.221535in}{1.811995in}}%
\pgfpathlineto{\pgfqpoint{1.207403in}{1.811995in}}%
\pgfpathlineto{\pgfqpoint{1.193270in}{1.811995in}}%
\pgfpathlineto{\pgfqpoint{1.179137in}{1.811995in}}%
\pgfpathlineto{\pgfqpoint{1.165005in}{1.811995in}}%
\pgfpathlineto{\pgfqpoint{1.150872in}{1.857300in}}%
\pgfpathlineto{\pgfqpoint{1.136740in}{1.858257in}}%
\pgfpathlineto{\pgfqpoint{1.122607in}{1.873940in}}%
\pgfpathlineto{\pgfqpoint{1.108474in}{1.873940in}}%
\pgfpathlineto{\pgfqpoint{1.094342in}{1.873940in}}%
\pgfpathlineto{\pgfqpoint{1.080209in}{1.873940in}}%
\pgfpathlineto{\pgfqpoint{1.066077in}{1.873940in}}%
\pgfpathlineto{\pgfqpoint{1.051944in}{1.913536in}}%
\pgfpathlineto{\pgfqpoint{1.037811in}{1.921121in}}%
\pgfpathlineto{\pgfqpoint{1.023679in}{1.939248in}}%
\pgfpathlineto{\pgfqpoint{1.009546in}{1.962018in}}%
\pgfpathlineto{\pgfqpoint{0.995414in}{2.036209in}}%
\pgfpathlineto{\pgfqpoint{0.981281in}{2.036209in}}%
\pgfpathlineto{\pgfqpoint{0.967148in}{2.036828in}}%
\pgfpathlineto{\pgfqpoint{0.953016in}{2.036828in}}%
\pgfpathlineto{\pgfqpoint{0.938883in}{2.038644in}}%
\pgfpathlineto{\pgfqpoint{0.924751in}{2.044632in}}%
\pgfpathlineto{\pgfqpoint{0.910618in}{2.055916in}}%
\pgfpathlineto{\pgfqpoint{0.896485in}{2.064646in}}%
\pgfpathlineto{\pgfqpoint{0.882353in}{2.066468in}}%
\pgfpathlineto{\pgfqpoint{0.868220in}{2.172507in}}%
\pgfpathlineto{\pgfqpoint{0.854088in}{2.220035in}}%
\pgfpathlineto{\pgfqpoint{0.839955in}{2.229500in}}%
\pgfpathlineto{\pgfqpoint{0.825822in}{2.256654in}}%
\pgfpathlineto{\pgfqpoint{0.811690in}{2.256654in}}%
\pgfpathlineto{\pgfqpoint{0.797557in}{2.256654in}}%
\pgfpathlineto{\pgfqpoint{0.783425in}{2.270660in}}%
\pgfpathlineto{\pgfqpoint{0.769292in}{2.301141in}}%
\pgfpathlineto{\pgfqpoint{0.755159in}{2.341079in}}%
\pgfpathlineto{\pgfqpoint{0.741027in}{2.350635in}}%
\pgfpathlineto{\pgfqpoint{0.726894in}{2.350635in}}%
\pgfpathlineto{\pgfqpoint{0.712762in}{2.350635in}}%
\pgfpathlineto{\pgfqpoint{0.698629in}{2.354085in}}%
\pgfpathlineto{\pgfqpoint{0.684496in}{2.386127in}}%
\pgfpathlineto{\pgfqpoint{0.670364in}{2.386127in}}%
\pgfpathlineto{\pgfqpoint{0.656231in}{2.391042in}}%
\pgfpathlineto{\pgfqpoint{0.642099in}{2.391042in}}%
\pgfpathlineto{\pgfqpoint{0.627966in}{2.426071in}}%
\pgfpathlineto{\pgfqpoint{0.613833in}{2.426071in}}%
\pgfpathlineto{\pgfqpoint{0.599701in}{2.428846in}}%
\pgfpathlineto{\pgfqpoint{0.585568in}{2.428846in}}%
\pgfpathlineto{\pgfqpoint{0.571436in}{2.438870in}}%
\pgfpathlineto{\pgfqpoint{0.557303in}{2.479005in}}%
\pgfpathlineto{\pgfqpoint{0.543170in}{2.492825in}}%
\pgfpathclose%
\pgfusepath{fill}%
\end{pgfscope}%
\begin{pgfscope}%
\pgfpathrectangle{\pgfqpoint{0.423750in}{0.375000in}}{\pgfqpoint{2.627250in}{2.265000in}}%
\pgfusepath{clip}%
\pgfsetroundcap%
\pgfsetroundjoin%
\pgfsetlinewidth{1.505625pt}%
\definecolor{currentstroke}{rgb}{0.121569,0.466667,0.705882}%
\pgfsetstrokecolor{currentstroke}%
\pgfsetdash{}{0pt}%
\pgfpathmoveto{\pgfqpoint{0.543170in}{2.509432in}}%
\pgfpathlineto{\pgfqpoint{0.557303in}{2.454804in}}%
\pgfpathlineto{\pgfqpoint{0.571436in}{2.438015in}}%
\pgfpathlineto{\pgfqpoint{0.585568in}{2.403116in}}%
\pgfpathlineto{\pgfqpoint{0.599701in}{2.381343in}}%
\pgfpathlineto{\pgfqpoint{0.613833in}{2.350547in}}%
\pgfpathlineto{\pgfqpoint{0.627966in}{2.333465in}}%
\pgfpathlineto{\pgfqpoint{0.642099in}{2.333465in}}%
\pgfpathlineto{\pgfqpoint{0.656231in}{2.319831in}}%
\pgfpathlineto{\pgfqpoint{0.670364in}{2.319831in}}%
\pgfpathlineto{\pgfqpoint{0.684496in}{2.265911in}}%
\pgfpathlineto{\pgfqpoint{0.726894in}{2.265911in}}%
\pgfpathlineto{\pgfqpoint{0.741027in}{2.262969in}}%
\pgfpathlineto{\pgfqpoint{0.797557in}{2.261925in}}%
\pgfpathlineto{\pgfqpoint{0.811690in}{2.254883in}}%
\pgfpathlineto{\pgfqpoint{0.825822in}{2.254883in}}%
\pgfpathlineto{\pgfqpoint{0.854088in}{2.247636in}}%
\pgfpathlineto{\pgfqpoint{0.938883in}{2.247636in}}%
\pgfpathlineto{\pgfqpoint{0.953016in}{2.238156in}}%
\pgfpathlineto{\pgfqpoint{1.051944in}{2.238156in}}%
\pgfpathlineto{\pgfqpoint{1.066077in}{2.214553in}}%
\pgfpathlineto{\pgfqpoint{1.263933in}{2.214553in}}%
\pgfpathlineto{\pgfqpoint{1.278066in}{2.211342in}}%
\pgfpathlineto{\pgfqpoint{1.306331in}{2.210008in}}%
\pgfpathlineto{\pgfqpoint{1.320463in}{2.198015in}}%
\pgfpathlineto{\pgfqpoint{1.334596in}{2.192593in}}%
\pgfpathlineto{\pgfqpoint{1.843369in}{2.192593in}}%
\pgfpathlineto{\pgfqpoint{1.857502in}{2.187874in}}%
\pgfpathlineto{\pgfqpoint{1.871635in}{2.185178in}}%
\pgfpathlineto{\pgfqpoint{2.027093in}{2.185178in}}%
\pgfpathlineto{\pgfqpoint{2.041226in}{2.170348in}}%
\pgfpathlineto{\pgfqpoint{2.224950in}{2.170348in}}%
\pgfpathlineto{\pgfqpoint{2.239082in}{2.161430in}}%
\pgfpathlineto{\pgfqpoint{2.267347in}{2.161430in}}%
\pgfpathlineto{\pgfqpoint{2.281480in}{2.155087in}}%
\pgfpathlineto{\pgfqpoint{2.380408in}{2.155087in}}%
\pgfpathlineto{\pgfqpoint{2.394541in}{2.141340in}}%
\pgfpathlineto{\pgfqpoint{2.493469in}{2.141340in}}%
\pgfpathlineto{\pgfqpoint{2.507602in}{2.120234in}}%
\pgfpathlineto{\pgfqpoint{2.620662in}{2.120234in}}%
\pgfpathlineto{\pgfqpoint{2.634795in}{2.118077in}}%
\pgfpathlineto{\pgfqpoint{2.705458in}{2.118077in}}%
\pgfpathlineto{\pgfqpoint{2.719591in}{2.108749in}}%
\pgfpathlineto{\pgfqpoint{2.776121in}{2.108749in}}%
\pgfpathlineto{\pgfqpoint{2.790254in}{2.100802in}}%
\pgfpathlineto{\pgfqpoint{2.917447in}{2.100802in}}%
\pgfpathlineto{\pgfqpoint{2.931580in}{2.088202in}}%
\pgfpathlineto{\pgfqpoint{2.931580in}{2.088202in}}%
\pgfusepath{stroke}%
\end{pgfscope}%
\begin{pgfscope}%
\pgfpathrectangle{\pgfqpoint{0.423750in}{0.375000in}}{\pgfqpoint{2.627250in}{2.265000in}}%
\pgfusepath{clip}%
\pgfsetroundcap%
\pgfsetroundjoin%
\pgfsetlinewidth{1.505625pt}%
\definecolor{currentstroke}{rgb}{1.000000,0.498039,0.054902}%
\pgfsetstrokecolor{currentstroke}%
\pgfsetdash{}{0pt}%
\pgfpathmoveto{\pgfqpoint{0.543170in}{2.519243in}}%
\pgfpathlineto{\pgfqpoint{0.557303in}{2.505201in}}%
\pgfpathlineto{\pgfqpoint{0.571436in}{2.439380in}}%
\pgfpathlineto{\pgfqpoint{0.585568in}{2.425901in}}%
\pgfpathlineto{\pgfqpoint{0.656231in}{2.424968in}}%
\pgfpathlineto{\pgfqpoint{0.670364in}{2.421133in}}%
\pgfpathlineto{\pgfqpoint{0.684496in}{2.421133in}}%
\pgfpathlineto{\pgfqpoint{0.698629in}{2.410687in}}%
\pgfpathlineto{\pgfqpoint{0.712762in}{2.369088in}}%
\pgfpathlineto{\pgfqpoint{0.726894in}{2.338777in}}%
\pgfpathlineto{\pgfqpoint{0.741027in}{2.286191in}}%
\pgfpathlineto{\pgfqpoint{0.755159in}{2.286191in}}%
\pgfpathlineto{\pgfqpoint{0.769292in}{2.262622in}}%
\pgfpathlineto{\pgfqpoint{0.882353in}{2.262622in}}%
\pgfpathlineto{\pgfqpoint{0.910618in}{2.186578in}}%
\pgfpathlineto{\pgfqpoint{0.967148in}{2.185949in}}%
\pgfpathlineto{\pgfqpoint{0.981281in}{2.155651in}}%
\pgfpathlineto{\pgfqpoint{0.995414in}{2.131275in}}%
\pgfpathlineto{\pgfqpoint{1.009546in}{2.098140in}}%
\pgfpathlineto{\pgfqpoint{1.023679in}{2.098140in}}%
\pgfpathlineto{\pgfqpoint{1.037811in}{2.064546in}}%
\pgfpathlineto{\pgfqpoint{1.080209in}{2.032545in}}%
\pgfpathlineto{\pgfqpoint{1.094342in}{2.032545in}}%
\pgfpathlineto{\pgfqpoint{1.108474in}{2.016737in}}%
\pgfpathlineto{\pgfqpoint{1.136740in}{2.016737in}}%
\pgfpathlineto{\pgfqpoint{1.150872in}{1.989380in}}%
\pgfpathlineto{\pgfqpoint{1.165005in}{1.855927in}}%
\pgfpathlineto{\pgfqpoint{1.207403in}{1.855927in}}%
\pgfpathlineto{\pgfqpoint{1.221535in}{1.854335in}}%
\pgfpathlineto{\pgfqpoint{1.306331in}{1.854335in}}%
\pgfpathlineto{\pgfqpoint{1.320463in}{1.841022in}}%
\pgfpathlineto{\pgfqpoint{1.334596in}{1.824667in}}%
\pgfpathlineto{\pgfqpoint{1.362861in}{1.813458in}}%
\pgfpathlineto{\pgfqpoint{1.376994in}{1.807686in}}%
\pgfpathlineto{\pgfqpoint{1.504187in}{1.807686in}}%
\pgfpathlineto{\pgfqpoint{1.518320in}{1.778547in}}%
\pgfpathlineto{\pgfqpoint{1.603115in}{1.778435in}}%
\pgfpathlineto{\pgfqpoint{1.617248in}{1.765578in}}%
\pgfpathlineto{\pgfqpoint{1.631381in}{1.760589in}}%
\pgfpathlineto{\pgfqpoint{1.645513in}{1.751546in}}%
\pgfpathlineto{\pgfqpoint{1.673778in}{1.751546in}}%
\pgfpathlineto{\pgfqpoint{1.687911in}{1.722467in}}%
\pgfpathlineto{\pgfqpoint{1.843369in}{1.722467in}}%
\pgfpathlineto{\pgfqpoint{1.857502in}{1.712980in}}%
\pgfpathlineto{\pgfqpoint{2.352143in}{1.712980in}}%
\pgfpathlineto{\pgfqpoint{2.366276in}{1.699451in}}%
\pgfpathlineto{\pgfqpoint{2.422806in}{1.697987in}}%
\pgfpathlineto{\pgfqpoint{2.436939in}{1.692717in}}%
\pgfpathlineto{\pgfqpoint{2.479336in}{1.692717in}}%
\pgfpathlineto{\pgfqpoint{2.493469in}{1.680797in}}%
\pgfpathlineto{\pgfqpoint{2.507602in}{1.676335in}}%
\pgfpathlineto{\pgfqpoint{2.606530in}{1.676335in}}%
\pgfpathlineto{\pgfqpoint{2.620662in}{1.674646in}}%
\pgfpathlineto{\pgfqpoint{2.648928in}{1.674646in}}%
\pgfpathlineto{\pgfqpoint{2.663060in}{1.665550in}}%
\pgfpathlineto{\pgfqpoint{2.931580in}{1.664646in}}%
\pgfpathlineto{\pgfqpoint{2.931580in}{1.664646in}}%
\pgfusepath{stroke}%
\end{pgfscope}%
\begin{pgfscope}%
\pgfpathrectangle{\pgfqpoint{0.423750in}{0.375000in}}{\pgfqpoint{2.627250in}{2.265000in}}%
\pgfusepath{clip}%
\pgfsetroundcap%
\pgfsetroundjoin%
\pgfsetlinewidth{1.505625pt}%
\definecolor{currentstroke}{rgb}{0.172549,0.627451,0.172549}%
\pgfsetstrokecolor{currentstroke}%
\pgfsetdash{}{0pt}%
\pgfpathmoveto{\pgfqpoint{0.543170in}{2.504275in}}%
\pgfpathlineto{\pgfqpoint{0.557303in}{2.466566in}}%
\pgfpathlineto{\pgfqpoint{0.571436in}{2.452055in}}%
\pgfpathlineto{\pgfqpoint{0.585568in}{2.411094in}}%
\pgfpathlineto{\pgfqpoint{0.599701in}{2.383552in}}%
\pgfpathlineto{\pgfqpoint{0.627966in}{2.383001in}}%
\pgfpathlineto{\pgfqpoint{0.642099in}{2.368255in}}%
\pgfpathlineto{\pgfqpoint{0.656231in}{2.362639in}}%
\pgfpathlineto{\pgfqpoint{0.712762in}{2.362639in}}%
\pgfpathlineto{\pgfqpoint{0.726894in}{2.343199in}}%
\pgfpathlineto{\pgfqpoint{0.741027in}{2.343199in}}%
\pgfpathlineto{\pgfqpoint{0.755159in}{2.339651in}}%
\pgfpathlineto{\pgfqpoint{0.769292in}{2.339651in}}%
\pgfpathlineto{\pgfqpoint{0.783425in}{2.337111in}}%
\pgfpathlineto{\pgfqpoint{0.797557in}{2.330253in}}%
\pgfpathlineto{\pgfqpoint{0.825822in}{2.330253in}}%
\pgfpathlineto{\pgfqpoint{0.839955in}{2.321773in}}%
\pgfpathlineto{\pgfqpoint{0.854088in}{2.299818in}}%
\pgfpathlineto{\pgfqpoint{0.868220in}{2.271722in}}%
\pgfpathlineto{\pgfqpoint{0.882353in}{2.268053in}}%
\pgfpathlineto{\pgfqpoint{0.896485in}{2.250290in}}%
\pgfpathlineto{\pgfqpoint{0.910618in}{2.090412in}}%
\pgfpathlineto{\pgfqpoint{0.924751in}{2.046659in}}%
\pgfpathlineto{\pgfqpoint{0.938883in}{2.014086in}}%
\pgfpathlineto{\pgfqpoint{0.953016in}{2.006669in}}%
\pgfpathlineto{\pgfqpoint{0.967148in}{1.983396in}}%
\pgfpathlineto{\pgfqpoint{0.981281in}{1.980517in}}%
\pgfpathlineto{\pgfqpoint{0.995414in}{1.980517in}}%
\pgfpathlineto{\pgfqpoint{1.009546in}{1.966643in}}%
\pgfpathlineto{\pgfqpoint{1.023679in}{1.945305in}}%
\pgfpathlineto{\pgfqpoint{1.051944in}{1.938450in}}%
\pgfpathlineto{\pgfqpoint{1.066077in}{1.937966in}}%
\pgfpathlineto{\pgfqpoint{1.080209in}{1.932537in}}%
\pgfpathlineto{\pgfqpoint{1.094342in}{1.932537in}}%
\pgfpathlineto{\pgfqpoint{1.108474in}{1.921239in}}%
\pgfpathlineto{\pgfqpoint{1.122607in}{1.921239in}}%
\pgfpathlineto{\pgfqpoint{1.136740in}{1.916602in}}%
\pgfpathlineto{\pgfqpoint{1.193270in}{1.915122in}}%
\pgfpathlineto{\pgfqpoint{1.207403in}{1.913313in}}%
\pgfpathlineto{\pgfqpoint{1.221535in}{1.913313in}}%
\pgfpathlineto{\pgfqpoint{1.235668in}{1.910619in}}%
\pgfpathlineto{\pgfqpoint{1.263933in}{1.888653in}}%
\pgfpathlineto{\pgfqpoint{1.334596in}{1.888653in}}%
\pgfpathlineto{\pgfqpoint{1.348729in}{1.877267in}}%
\pgfpathlineto{\pgfqpoint{1.362861in}{1.874182in}}%
\pgfpathlineto{\pgfqpoint{1.376994in}{1.863579in}}%
\pgfpathlineto{\pgfqpoint{1.603115in}{1.863433in}}%
\pgfpathlineto{\pgfqpoint{1.617248in}{1.852280in}}%
\pgfpathlineto{\pgfqpoint{1.645513in}{1.852280in}}%
\pgfpathlineto{\pgfqpoint{1.659646in}{1.850064in}}%
\pgfpathlineto{\pgfqpoint{1.673778in}{1.840967in}}%
\pgfpathlineto{\pgfqpoint{1.687911in}{1.807923in}}%
\pgfpathlineto{\pgfqpoint{1.800972in}{1.807923in}}%
\pgfpathlineto{\pgfqpoint{1.815104in}{1.791766in}}%
\pgfpathlineto{\pgfqpoint{2.154287in}{1.791062in}}%
\pgfpathlineto{\pgfqpoint{2.182552in}{1.789570in}}%
\pgfpathlineto{\pgfqpoint{2.224950in}{1.789570in}}%
\pgfpathlineto{\pgfqpoint{2.239082in}{1.783862in}}%
\pgfpathlineto{\pgfqpoint{2.436939in}{1.782473in}}%
\pgfpathlineto{\pgfqpoint{2.705458in}{1.779412in}}%
\pgfpathlineto{\pgfqpoint{2.931580in}{1.779412in}}%
\pgfpathlineto{\pgfqpoint{2.931580in}{1.779412in}}%
\pgfusepath{stroke}%
\end{pgfscope}%
\begin{pgfscope}%
\pgfpathrectangle{\pgfqpoint{0.423750in}{0.375000in}}{\pgfqpoint{2.627250in}{2.265000in}}%
\pgfusepath{clip}%
\pgfsetroundcap%
\pgfsetroundjoin%
\pgfsetlinewidth{1.505625pt}%
\definecolor{currentstroke}{rgb}{0.839216,0.152941,0.156863}%
\pgfsetstrokecolor{currentstroke}%
\pgfsetdash{}{0pt}%
\pgfpathmoveto{\pgfqpoint{0.543170in}{2.439292in}}%
\pgfpathlineto{\pgfqpoint{0.557303in}{2.417853in}}%
\pgfpathlineto{\pgfqpoint{0.571436in}{2.405984in}}%
\pgfpathlineto{\pgfqpoint{0.585568in}{2.384221in}}%
\pgfpathlineto{\pgfqpoint{0.599701in}{2.340928in}}%
\pgfpathlineto{\pgfqpoint{0.613833in}{2.340928in}}%
\pgfpathlineto{\pgfqpoint{0.627966in}{2.301472in}}%
\pgfpathlineto{\pgfqpoint{0.670364in}{2.301472in}}%
\pgfpathlineto{\pgfqpoint{0.684496in}{2.299527in}}%
\pgfpathlineto{\pgfqpoint{0.825822in}{2.299527in}}%
\pgfpathlineto{\pgfqpoint{0.839955in}{2.285191in}}%
\pgfpathlineto{\pgfqpoint{0.854088in}{2.285191in}}%
\pgfpathlineto{\pgfqpoint{0.868220in}{2.268795in}}%
\pgfpathlineto{\pgfqpoint{0.882353in}{2.268795in}}%
\pgfpathlineto{\pgfqpoint{0.896485in}{2.251979in}}%
\pgfpathlineto{\pgfqpoint{0.910618in}{2.251979in}}%
\pgfpathlineto{\pgfqpoint{0.924751in}{2.241553in}}%
\pgfpathlineto{\pgfqpoint{0.938883in}{2.206910in}}%
\pgfpathlineto{\pgfqpoint{0.953016in}{2.202291in}}%
\pgfpathlineto{\pgfqpoint{0.967148in}{2.108387in}}%
\pgfpathlineto{\pgfqpoint{0.981281in}{2.095005in}}%
\pgfpathlineto{\pgfqpoint{0.995414in}{2.095005in}}%
\pgfpathlineto{\pgfqpoint{1.009546in}{2.057554in}}%
\pgfpathlineto{\pgfqpoint{1.023679in}{2.049877in}}%
\pgfpathlineto{\pgfqpoint{1.037811in}{2.049877in}}%
\pgfpathlineto{\pgfqpoint{1.066077in}{2.034028in}}%
\pgfpathlineto{\pgfqpoint{1.080209in}{2.025076in}}%
\pgfpathlineto{\pgfqpoint{1.094342in}{2.025076in}}%
\pgfpathlineto{\pgfqpoint{1.108474in}{2.006855in}}%
\pgfpathlineto{\pgfqpoint{1.122607in}{1.996434in}}%
\pgfpathlineto{\pgfqpoint{1.136740in}{1.982108in}}%
\pgfpathlineto{\pgfqpoint{1.150872in}{1.971429in}}%
\pgfpathlineto{\pgfqpoint{1.179137in}{1.971274in}}%
\pgfpathlineto{\pgfqpoint{1.193270in}{1.954896in}}%
\pgfpathlineto{\pgfqpoint{1.235668in}{1.954896in}}%
\pgfpathlineto{\pgfqpoint{1.249800in}{1.937838in}}%
\pgfpathlineto{\pgfqpoint{1.263933in}{1.933951in}}%
\pgfpathlineto{\pgfqpoint{1.306331in}{1.933951in}}%
\pgfpathlineto{\pgfqpoint{1.320463in}{1.924786in}}%
\pgfpathlineto{\pgfqpoint{1.334596in}{1.924786in}}%
\pgfpathlineto{\pgfqpoint{1.348729in}{1.922763in}}%
\pgfpathlineto{\pgfqpoint{1.362861in}{1.922763in}}%
\pgfpathlineto{\pgfqpoint{1.376994in}{1.918946in}}%
\pgfpathlineto{\pgfqpoint{1.391126in}{1.910559in}}%
\pgfpathlineto{\pgfqpoint{1.419392in}{1.910559in}}%
\pgfpathlineto{\pgfqpoint{1.433524in}{1.900234in}}%
\pgfpathlineto{\pgfqpoint{1.447657in}{1.897265in}}%
\pgfpathlineto{\pgfqpoint{1.461789in}{1.888583in}}%
\pgfpathlineto{\pgfqpoint{1.475922in}{1.815522in}}%
\pgfpathlineto{\pgfqpoint{1.490055in}{1.815522in}}%
\pgfpathlineto{\pgfqpoint{1.504187in}{1.785700in}}%
\pgfpathlineto{\pgfqpoint{1.518320in}{1.785700in}}%
\pgfpathlineto{\pgfqpoint{1.532452in}{1.775061in}}%
\pgfpathlineto{\pgfqpoint{1.588983in}{1.775061in}}%
\pgfpathlineto{\pgfqpoint{1.603115in}{1.766554in}}%
\pgfpathlineto{\pgfqpoint{1.617248in}{1.755653in}}%
\pgfpathlineto{\pgfqpoint{1.659646in}{1.755653in}}%
\pgfpathlineto{\pgfqpoint{1.673778in}{1.684000in}}%
\pgfpathlineto{\pgfqpoint{1.687911in}{1.684000in}}%
\pgfpathlineto{\pgfqpoint{1.702044in}{1.622705in}}%
\pgfpathlineto{\pgfqpoint{1.885767in}{1.622705in}}%
\pgfpathlineto{\pgfqpoint{1.899900in}{1.619818in}}%
\pgfpathlineto{\pgfqpoint{1.914032in}{1.619818in}}%
\pgfpathlineto{\pgfqpoint{1.928165in}{1.596770in}}%
\pgfpathlineto{\pgfqpoint{1.984695in}{1.596770in}}%
\pgfpathlineto{\pgfqpoint{1.998828in}{1.453787in}}%
\pgfpathlineto{\pgfqpoint{2.083624in}{1.453787in}}%
\pgfpathlineto{\pgfqpoint{2.097756in}{1.414333in}}%
\pgfpathlineto{\pgfqpoint{2.267347in}{1.414333in}}%
\pgfpathlineto{\pgfqpoint{2.281480in}{1.408871in}}%
\pgfpathlineto{\pgfqpoint{2.352143in}{1.408871in}}%
\pgfpathlineto{\pgfqpoint{2.366276in}{1.360974in}}%
\pgfpathlineto{\pgfqpoint{2.380408in}{1.265909in}}%
\pgfpathlineto{\pgfqpoint{2.451071in}{1.265909in}}%
\pgfpathlineto{\pgfqpoint{2.465204in}{1.236426in}}%
\pgfpathlineto{\pgfqpoint{2.479336in}{1.219748in}}%
\pgfpathlineto{\pgfqpoint{2.620662in}{1.219748in}}%
\pgfpathlineto{\pgfqpoint{2.634795in}{1.216991in}}%
\pgfpathlineto{\pgfqpoint{2.648928in}{1.216991in}}%
\pgfpathlineto{\pgfqpoint{2.663060in}{1.201040in}}%
\pgfpathlineto{\pgfqpoint{2.705458in}{1.201040in}}%
\pgfpathlineto{\pgfqpoint{2.719591in}{1.193349in}}%
\pgfpathlineto{\pgfqpoint{2.903314in}{1.193349in}}%
\pgfpathlineto{\pgfqpoint{2.917447in}{1.147520in}}%
\pgfpathlineto{\pgfqpoint{2.931580in}{1.147520in}}%
\pgfpathlineto{\pgfqpoint{2.931580in}{1.147520in}}%
\pgfusepath{stroke}%
\end{pgfscope}%
\begin{pgfscope}%
\pgfpathrectangle{\pgfqpoint{0.423750in}{0.375000in}}{\pgfqpoint{2.627250in}{2.265000in}}%
\pgfusepath{clip}%
\pgfsetroundcap%
\pgfsetroundjoin%
\pgfsetlinewidth{1.505625pt}%
\definecolor{currentstroke}{rgb}{0.580392,0.403922,0.741176}%
\pgfsetstrokecolor{currentstroke}%
\pgfsetdash{}{0pt}%
\pgfpathmoveto{\pgfqpoint{0.543170in}{2.508857in}}%
\pgfpathlineto{\pgfqpoint{0.557303in}{2.490902in}}%
\pgfpathlineto{\pgfqpoint{0.571436in}{2.459817in}}%
\pgfpathlineto{\pgfqpoint{0.585568in}{2.451726in}}%
\pgfpathlineto{\pgfqpoint{0.599701in}{2.451726in}}%
\pgfpathlineto{\pgfqpoint{0.613833in}{2.440601in}}%
\pgfpathlineto{\pgfqpoint{0.627966in}{2.440601in}}%
\pgfpathlineto{\pgfqpoint{0.642099in}{2.419868in}}%
\pgfpathlineto{\pgfqpoint{0.656231in}{2.419868in}}%
\pgfpathlineto{\pgfqpoint{0.670364in}{2.411503in}}%
\pgfpathlineto{\pgfqpoint{0.684496in}{2.411503in}}%
\pgfpathlineto{\pgfqpoint{0.698629in}{2.385265in}}%
\pgfpathlineto{\pgfqpoint{0.712762in}{2.381335in}}%
\pgfpathlineto{\pgfqpoint{0.741027in}{2.381335in}}%
\pgfpathlineto{\pgfqpoint{0.755159in}{2.372602in}}%
\pgfpathlineto{\pgfqpoint{0.769292in}{2.327291in}}%
\pgfpathlineto{\pgfqpoint{0.797557in}{2.293621in}}%
\pgfpathlineto{\pgfqpoint{0.825822in}{2.293621in}}%
\pgfpathlineto{\pgfqpoint{0.839955in}{2.261856in}}%
\pgfpathlineto{\pgfqpoint{0.854088in}{2.253355in}}%
\pgfpathlineto{\pgfqpoint{0.868220in}{2.216008in}}%
\pgfpathlineto{\pgfqpoint{0.882353in}{2.138848in}}%
\pgfpathlineto{\pgfqpoint{0.896485in}{2.136537in}}%
\pgfpathlineto{\pgfqpoint{0.924751in}{2.118749in}}%
\pgfpathlineto{\pgfqpoint{0.938883in}{2.115008in}}%
\pgfpathlineto{\pgfqpoint{0.981281in}{2.113251in}}%
\pgfpathlineto{\pgfqpoint{0.995414in}{2.113251in}}%
\pgfpathlineto{\pgfqpoint{1.009546in}{2.053481in}}%
\pgfpathlineto{\pgfqpoint{1.023679in}{2.043317in}}%
\pgfpathlineto{\pgfqpoint{1.037811in}{2.036403in}}%
\pgfpathlineto{\pgfqpoint{1.051944in}{2.033328in}}%
\pgfpathlineto{\pgfqpoint{1.066077in}{2.019771in}}%
\pgfpathlineto{\pgfqpoint{1.122607in}{2.019771in}}%
\pgfpathlineto{\pgfqpoint{1.136740in}{1.958796in}}%
\pgfpathlineto{\pgfqpoint{1.150872in}{1.958406in}}%
\pgfpathlineto{\pgfqpoint{1.165005in}{1.941525in}}%
\pgfpathlineto{\pgfqpoint{1.249800in}{1.941525in}}%
\pgfpathlineto{\pgfqpoint{1.263933in}{1.939717in}}%
\pgfpathlineto{\pgfqpoint{1.320463in}{1.939593in}}%
\pgfpathlineto{\pgfqpoint{1.334596in}{1.935118in}}%
\pgfpathlineto{\pgfqpoint{1.348729in}{1.934508in}}%
\pgfpathlineto{\pgfqpoint{1.362861in}{1.930705in}}%
\pgfpathlineto{\pgfqpoint{1.391126in}{1.929047in}}%
\pgfpathlineto{\pgfqpoint{1.433524in}{1.928610in}}%
\pgfpathlineto{\pgfqpoint{1.447657in}{1.901195in}}%
\pgfpathlineto{\pgfqpoint{1.461789in}{1.901195in}}%
\pgfpathlineto{\pgfqpoint{1.475922in}{1.889941in}}%
\pgfpathlineto{\pgfqpoint{1.490055in}{1.888357in}}%
\pgfpathlineto{\pgfqpoint{1.504187in}{1.780831in}}%
\pgfpathlineto{\pgfqpoint{1.518320in}{1.777063in}}%
\pgfpathlineto{\pgfqpoint{1.574850in}{1.776091in}}%
\pgfpathlineto{\pgfqpoint{1.588983in}{1.750621in}}%
\pgfpathlineto{\pgfqpoint{1.603115in}{1.749234in}}%
\pgfpathlineto{\pgfqpoint{1.617248in}{1.721430in}}%
\pgfpathlineto{\pgfqpoint{1.631381in}{1.720569in}}%
\pgfpathlineto{\pgfqpoint{1.659646in}{1.713639in}}%
\pgfpathlineto{\pgfqpoint{1.687911in}{1.711048in}}%
\pgfpathlineto{\pgfqpoint{1.702044in}{1.653510in}}%
\pgfpathlineto{\pgfqpoint{1.716176in}{1.615201in}}%
\pgfpathlineto{\pgfqpoint{1.744441in}{1.607226in}}%
\pgfpathlineto{\pgfqpoint{1.758574in}{1.593445in}}%
\pgfpathlineto{\pgfqpoint{1.772706in}{1.550160in}}%
\pgfpathlineto{\pgfqpoint{1.786839in}{1.424915in}}%
\pgfpathlineto{\pgfqpoint{1.800972in}{1.389609in}}%
\pgfpathlineto{\pgfqpoint{1.815104in}{1.339893in}}%
\pgfpathlineto{\pgfqpoint{1.829237in}{1.336702in}}%
\pgfpathlineto{\pgfqpoint{1.857502in}{1.323521in}}%
\pgfpathlineto{\pgfqpoint{1.871635in}{1.320212in}}%
\pgfpathlineto{\pgfqpoint{1.885767in}{1.259282in}}%
\pgfpathlineto{\pgfqpoint{1.899900in}{1.137286in}}%
\pgfpathlineto{\pgfqpoint{1.914032in}{0.956802in}}%
\pgfpathlineto{\pgfqpoint{1.928165in}{0.938218in}}%
\pgfpathlineto{\pgfqpoint{1.942298in}{0.915848in}}%
\pgfpathlineto{\pgfqpoint{1.956430in}{0.907091in}}%
\pgfpathlineto{\pgfqpoint{1.970563in}{0.854849in}}%
\pgfpathlineto{\pgfqpoint{1.984695in}{0.852501in}}%
\pgfpathlineto{\pgfqpoint{1.998828in}{0.852501in}}%
\pgfpathlineto{\pgfqpoint{2.012961in}{0.843855in}}%
\pgfpathlineto{\pgfqpoint{2.055358in}{0.838590in}}%
\pgfpathlineto{\pgfqpoint{2.083624in}{0.806779in}}%
\pgfpathlineto{\pgfqpoint{2.097756in}{0.800698in}}%
\pgfpathlineto{\pgfqpoint{2.111889in}{0.789842in}}%
\pgfpathlineto{\pgfqpoint{2.126021in}{0.772280in}}%
\pgfpathlineto{\pgfqpoint{2.168419in}{0.752678in}}%
\pgfpathlineto{\pgfqpoint{2.182552in}{0.738467in}}%
\pgfpathlineto{\pgfqpoint{2.196684in}{0.732519in}}%
\pgfpathlineto{\pgfqpoint{2.210817in}{0.720225in}}%
\pgfpathlineto{\pgfqpoint{2.224950in}{0.710104in}}%
\pgfpathlineto{\pgfqpoint{2.253215in}{0.694430in}}%
\pgfpathlineto{\pgfqpoint{2.323878in}{0.667286in}}%
\pgfpathlineto{\pgfqpoint{2.352143in}{0.654870in}}%
\pgfpathlineto{\pgfqpoint{2.366276in}{0.649594in}}%
\pgfpathlineto{\pgfqpoint{2.380408in}{0.645997in}}%
\pgfpathlineto{\pgfqpoint{2.408673in}{0.636174in}}%
\pgfpathlineto{\pgfqpoint{2.422806in}{0.632009in}}%
\pgfpathlineto{\pgfqpoint{2.465204in}{0.629274in}}%
\pgfpathlineto{\pgfqpoint{2.479336in}{0.623898in}}%
\pgfpathlineto{\pgfqpoint{2.521734in}{0.620915in}}%
\pgfpathlineto{\pgfqpoint{2.578265in}{0.614191in}}%
\pgfpathlineto{\pgfqpoint{2.677193in}{0.610518in}}%
\pgfpathlineto{\pgfqpoint{2.733723in}{0.609305in}}%
\pgfpathlineto{\pgfqpoint{2.776121in}{0.607309in}}%
\pgfpathlineto{\pgfqpoint{2.832651in}{0.606857in}}%
\pgfpathlineto{\pgfqpoint{2.860917in}{0.602693in}}%
\pgfpathlineto{\pgfqpoint{2.875049in}{0.592009in}}%
\pgfpathlineto{\pgfqpoint{2.931580in}{0.591953in}}%
\pgfpathlineto{\pgfqpoint{2.931580in}{0.591953in}}%
\pgfusepath{stroke}%
\end{pgfscope}%
\begin{pgfscope}%
\pgfsetrectcap%
\pgfsetmiterjoin%
\pgfsetlinewidth{0.000000pt}%
\definecolor{currentstroke}{rgb}{1.000000,1.000000,1.000000}%
\pgfsetstrokecolor{currentstroke}%
\pgfsetdash{}{0pt}%
\pgfpathmoveto{\pgfqpoint{0.423750in}{0.375000in}}%
\pgfpathlineto{\pgfqpoint{0.423750in}{2.640000in}}%
\pgfusepath{}%
\end{pgfscope}%
\begin{pgfscope}%
\pgfsetrectcap%
\pgfsetmiterjoin%
\pgfsetlinewidth{0.000000pt}%
\definecolor{currentstroke}{rgb}{1.000000,1.000000,1.000000}%
\pgfsetstrokecolor{currentstroke}%
\pgfsetdash{}{0pt}%
\pgfpathmoveto{\pgfqpoint{3.051000in}{0.375000in}}%
\pgfpathlineto{\pgfqpoint{3.051000in}{2.640000in}}%
\pgfusepath{}%
\end{pgfscope}%
\begin{pgfscope}%
\pgfsetrectcap%
\pgfsetmiterjoin%
\pgfsetlinewidth{0.000000pt}%
\definecolor{currentstroke}{rgb}{1.000000,1.000000,1.000000}%
\pgfsetstrokecolor{currentstroke}%
\pgfsetdash{}{0pt}%
\pgfpathmoveto{\pgfqpoint{0.423750in}{0.375000in}}%
\pgfpathlineto{\pgfqpoint{3.051000in}{0.375000in}}%
\pgfusepath{}%
\end{pgfscope}%
\begin{pgfscope}%
\pgfsetrectcap%
\pgfsetmiterjoin%
\pgfsetlinewidth{0.000000pt}%
\definecolor{currentstroke}{rgb}{1.000000,1.000000,1.000000}%
\pgfsetstrokecolor{currentstroke}%
\pgfsetdash{}{0pt}%
\pgfpathmoveto{\pgfqpoint{0.423750in}{2.640000in}}%
\pgfpathlineto{\pgfqpoint{3.051000in}{2.640000in}}%
\pgfusepath{}%
\end{pgfscope}%
\begin{pgfscope}%
\definecolor{textcolor}{rgb}{0.150000,0.150000,0.150000}%
\pgfsetstrokecolor{textcolor}%
\pgfsetfillcolor{textcolor}%
\pgftext[x=1.737375in,y=2.723333in,,base]{\color{textcolor}\rmfamily\fontsize{8.000000}{9.600000}\selectfont Hartmann3}%
\end{pgfscope}%
\begin{pgfscope}%
\pgfsetroundcap%
\pgfsetroundjoin%
\pgfsetlinewidth{1.505625pt}%
\definecolor{currentstroke}{rgb}{0.121569,0.466667,0.705882}%
\pgfsetstrokecolor{currentstroke}%
\pgfsetdash{}{0pt}%
\pgfpathmoveto{\pgfqpoint{0.523750in}{1.189344in}}%
\pgfpathlineto{\pgfqpoint{0.745972in}{1.189344in}}%
\pgfusepath{stroke}%
\end{pgfscope}%
\begin{pgfscope}%
\definecolor{textcolor}{rgb}{0.150000,0.150000,0.150000}%
\pgfsetstrokecolor{textcolor}%
\pgfsetfillcolor{textcolor}%
\pgftext[x=0.834861in,y=1.150455in,left,base]{\color{textcolor}\rmfamily\fontsize{8.000000}{9.600000}\selectfont random}%
\end{pgfscope}%
\begin{pgfscope}%
\pgfsetroundcap%
\pgfsetroundjoin%
\pgfsetlinewidth{1.505625pt}%
\definecolor{currentstroke}{rgb}{1.000000,0.498039,0.054902}%
\pgfsetstrokecolor{currentstroke}%
\pgfsetdash{}{0pt}%
\pgfpathmoveto{\pgfqpoint{0.523750in}{1.026258in}}%
\pgfpathlineto{\pgfqpoint{0.745972in}{1.026258in}}%
\pgfusepath{stroke}%
\end{pgfscope}%
\begin{pgfscope}%
\definecolor{textcolor}{rgb}{0.150000,0.150000,0.150000}%
\pgfsetstrokecolor{textcolor}%
\pgfsetfillcolor{textcolor}%
\pgftext[x=0.834861in,y=0.987369in,left,base]{\color{textcolor}\rmfamily\fontsize{8.000000}{9.600000}\selectfont DNGO fixed}%
\end{pgfscope}%
\begin{pgfscope}%
\pgfsetroundcap%
\pgfsetroundjoin%
\pgfsetlinewidth{1.505625pt}%
\definecolor{currentstroke}{rgb}{0.172549,0.627451,0.172549}%
\pgfsetstrokecolor{currentstroke}%
\pgfsetdash{}{0pt}%
\pgfpathmoveto{\pgfqpoint{0.523750in}{0.863172in}}%
\pgfpathlineto{\pgfqpoint{0.745972in}{0.863172in}}%
\pgfusepath{stroke}%
\end{pgfscope}%
\begin{pgfscope}%
\definecolor{textcolor}{rgb}{0.150000,0.150000,0.150000}%
\pgfsetstrokecolor{textcolor}%
\pgfsetfillcolor{textcolor}%
\pgftext[x=0.834861in,y=0.824283in,left,base]{\color{textcolor}\rmfamily\fontsize{8.000000}{9.600000}\selectfont DNGO retrain}%
\end{pgfscope}%
\begin{pgfscope}%
\pgfsetroundcap%
\pgfsetroundjoin%
\pgfsetlinewidth{1.505625pt}%
\definecolor{currentstroke}{rgb}{0.839216,0.152941,0.156863}%
\pgfsetstrokecolor{currentstroke}%
\pgfsetdash{}{0pt}%
\pgfpathmoveto{\pgfqpoint{0.523750in}{0.700087in}}%
\pgfpathlineto{\pgfqpoint{0.745972in}{0.700087in}}%
\pgfusepath{stroke}%
\end{pgfscope}%
\begin{pgfscope}%
\definecolor{textcolor}{rgb}{0.150000,0.150000,0.150000}%
\pgfsetstrokecolor{textcolor}%
\pgfsetfillcolor{textcolor}%
\pgftext[x=0.834861in,y=0.661198in,left,base]{\color{textcolor}\rmfamily\fontsize{8.000000}{9.600000}\selectfont DNGO retrain-reset}%
\end{pgfscope}%
\begin{pgfscope}%
\pgfsetroundcap%
\pgfsetroundjoin%
\pgfsetlinewidth{1.505625pt}%
\definecolor{currentstroke}{rgb}{0.580392,0.403922,0.741176}%
\pgfsetstrokecolor{currentstroke}%
\pgfsetdash{}{0pt}%
\pgfpathmoveto{\pgfqpoint{0.523750in}{0.537001in}}%
\pgfpathlineto{\pgfqpoint{0.745972in}{0.537001in}}%
\pgfusepath{stroke}%
\end{pgfscope}%
\begin{pgfscope}%
\definecolor{textcolor}{rgb}{0.150000,0.150000,0.150000}%
\pgfsetstrokecolor{textcolor}%
\pgfsetfillcolor{textcolor}%
\pgftext[x=0.834861in,y=0.498112in,left,base]{\color{textcolor}\rmfamily\fontsize{8.000000}{9.600000}\selectfont GP}%
\end{pgfscope}%
\end{pgfpicture}%
\makeatother%
\endgroup%

        \end{minipage}

    \subsection{Benchmarks}\label{sec:appbenchmark}
        \begin{minipage}{0.45\linewidth}
            \centering
            %% Creator: Matplotlib, PGF backend
%%
%% To include the figure in your LaTeX document, write
%%   \input{<filename>.pgf}
%%
%% Make sure the required packages are loaded in your preamble
%%   \usepackage{pgf}
%%
%% Figures using additional raster images can only be included by \input if
%% they are in the same directory as the main LaTeX file. For loading figures
%% from other directories you can use the `import` package
%%   \usepackage{import}
%% and then include the figures with
%%   \import{<path to file>}{<filename>.pgf}
%%
%% Matplotlib used the following preamble
%%   \usepackage{gensymb}
%%   \usepackage{fontspec}
%%   \setmainfont{DejaVu Serif}
%%   \setsansfont{Arial}
%%   \setmonofont{DejaVu Sans Mono}
%%
\begingroup%
\makeatletter%
\begin{pgfpicture}%
\pgfpathrectangle{\pgfpointorigin}{\pgfqpoint{3.390000in}{3.000000in}}%
\pgfusepath{use as bounding box, clip}%
\begin{pgfscope}%
\pgfsetbuttcap%
\pgfsetmiterjoin%
\definecolor{currentfill}{rgb}{1.000000,1.000000,1.000000}%
\pgfsetfillcolor{currentfill}%
\pgfsetlinewidth{0.000000pt}%
\definecolor{currentstroke}{rgb}{1.000000,1.000000,1.000000}%
\pgfsetstrokecolor{currentstroke}%
\pgfsetdash{}{0pt}%
\pgfpathmoveto{\pgfqpoint{0.000000in}{0.000000in}}%
\pgfpathlineto{\pgfqpoint{3.390000in}{0.000000in}}%
\pgfpathlineto{\pgfqpoint{3.390000in}{3.000000in}}%
\pgfpathlineto{\pgfqpoint{0.000000in}{3.000000in}}%
\pgfpathclose%
\pgfusepath{fill}%
\end{pgfscope}%
\begin{pgfscope}%
\pgfsetbuttcap%
\pgfsetmiterjoin%
\definecolor{currentfill}{rgb}{0.917647,0.917647,0.949020}%
\pgfsetfillcolor{currentfill}%
\pgfsetlinewidth{0.000000pt}%
\definecolor{currentstroke}{rgb}{0.000000,0.000000,0.000000}%
\pgfsetstrokecolor{currentstroke}%
\pgfsetstrokeopacity{0.000000}%
\pgfsetdash{}{0pt}%
\pgfpathmoveto{\pgfqpoint{0.423750in}{0.375000in}}%
\pgfpathlineto{\pgfqpoint{3.051000in}{0.375000in}}%
\pgfpathlineto{\pgfqpoint{3.051000in}{2.640000in}}%
\pgfpathlineto{\pgfqpoint{0.423750in}{2.640000in}}%
\pgfpathclose%
\pgfusepath{fill}%
\end{pgfscope}%
\begin{pgfscope}%
\pgfpathrectangle{\pgfqpoint{0.423750in}{0.375000in}}{\pgfqpoint{2.627250in}{2.265000in}}%
\pgfusepath{clip}%
\pgfsetroundcap%
\pgfsetroundjoin%
\pgfsetlinewidth{0.803000pt}%
\definecolor{currentstroke}{rgb}{1.000000,1.000000,1.000000}%
\pgfsetstrokecolor{currentstroke}%
\pgfsetdash{}{0pt}%
\pgfpathmoveto{\pgfqpoint{0.543170in}{0.375000in}}%
\pgfpathlineto{\pgfqpoint{0.543170in}{2.640000in}}%
\pgfusepath{stroke}%
\end{pgfscope}%
\begin{pgfscope}%
\definecolor{textcolor}{rgb}{0.150000,0.150000,0.150000}%
\pgfsetstrokecolor{textcolor}%
\pgfsetfillcolor{textcolor}%
\pgftext[x=0.543170in,y=0.326389in,,top]{\color{textcolor}\rmfamily\fontsize{8.000000}{9.600000}\selectfont \(\displaystyle 0\)}%
\end{pgfscope}%
\begin{pgfscope}%
\pgfpathrectangle{\pgfqpoint{0.423750in}{0.375000in}}{\pgfqpoint{2.627250in}{2.265000in}}%
\pgfusepath{clip}%
\pgfsetroundcap%
\pgfsetroundjoin%
\pgfsetlinewidth{0.803000pt}%
\definecolor{currentstroke}{rgb}{1.000000,1.000000,1.000000}%
\pgfsetstrokecolor{currentstroke}%
\pgfsetdash{}{0pt}%
\pgfpathmoveto{\pgfqpoint{1.088469in}{0.375000in}}%
\pgfpathlineto{\pgfqpoint{1.088469in}{2.640000in}}%
\pgfusepath{stroke}%
\end{pgfscope}%
\begin{pgfscope}%
\definecolor{textcolor}{rgb}{0.150000,0.150000,0.150000}%
\pgfsetstrokecolor{textcolor}%
\pgfsetfillcolor{textcolor}%
\pgftext[x=1.088469in,y=0.326389in,,top]{\color{textcolor}\rmfamily\fontsize{8.000000}{9.600000}\selectfont \(\displaystyle 50\)}%
\end{pgfscope}%
\begin{pgfscope}%
\pgfpathrectangle{\pgfqpoint{0.423750in}{0.375000in}}{\pgfqpoint{2.627250in}{2.265000in}}%
\pgfusepath{clip}%
\pgfsetroundcap%
\pgfsetroundjoin%
\pgfsetlinewidth{0.803000pt}%
\definecolor{currentstroke}{rgb}{1.000000,1.000000,1.000000}%
\pgfsetstrokecolor{currentstroke}%
\pgfsetdash{}{0pt}%
\pgfpathmoveto{\pgfqpoint{1.633768in}{0.375000in}}%
\pgfpathlineto{\pgfqpoint{1.633768in}{2.640000in}}%
\pgfusepath{stroke}%
\end{pgfscope}%
\begin{pgfscope}%
\definecolor{textcolor}{rgb}{0.150000,0.150000,0.150000}%
\pgfsetstrokecolor{textcolor}%
\pgfsetfillcolor{textcolor}%
\pgftext[x=1.633768in,y=0.326389in,,top]{\color{textcolor}\rmfamily\fontsize{8.000000}{9.600000}\selectfont \(\displaystyle 100\)}%
\end{pgfscope}%
\begin{pgfscope}%
\pgfpathrectangle{\pgfqpoint{0.423750in}{0.375000in}}{\pgfqpoint{2.627250in}{2.265000in}}%
\pgfusepath{clip}%
\pgfsetroundcap%
\pgfsetroundjoin%
\pgfsetlinewidth{0.803000pt}%
\definecolor{currentstroke}{rgb}{1.000000,1.000000,1.000000}%
\pgfsetstrokecolor{currentstroke}%
\pgfsetdash{}{0pt}%
\pgfpathmoveto{\pgfqpoint{2.179067in}{0.375000in}}%
\pgfpathlineto{\pgfqpoint{2.179067in}{2.640000in}}%
\pgfusepath{stroke}%
\end{pgfscope}%
\begin{pgfscope}%
\definecolor{textcolor}{rgb}{0.150000,0.150000,0.150000}%
\pgfsetstrokecolor{textcolor}%
\pgfsetfillcolor{textcolor}%
\pgftext[x=2.179067in,y=0.326389in,,top]{\color{textcolor}\rmfamily\fontsize{8.000000}{9.600000}\selectfont \(\displaystyle 150\)}%
\end{pgfscope}%
\begin{pgfscope}%
\pgfpathrectangle{\pgfqpoint{0.423750in}{0.375000in}}{\pgfqpoint{2.627250in}{2.265000in}}%
\pgfusepath{clip}%
\pgfsetroundcap%
\pgfsetroundjoin%
\pgfsetlinewidth{0.803000pt}%
\definecolor{currentstroke}{rgb}{1.000000,1.000000,1.000000}%
\pgfsetstrokecolor{currentstroke}%
\pgfsetdash{}{0pt}%
\pgfpathmoveto{\pgfqpoint{2.724366in}{0.375000in}}%
\pgfpathlineto{\pgfqpoint{2.724366in}{2.640000in}}%
\pgfusepath{stroke}%
\end{pgfscope}%
\begin{pgfscope}%
\definecolor{textcolor}{rgb}{0.150000,0.150000,0.150000}%
\pgfsetstrokecolor{textcolor}%
\pgfsetfillcolor{textcolor}%
\pgftext[x=2.724366in,y=0.326389in,,top]{\color{textcolor}\rmfamily\fontsize{8.000000}{9.600000}\selectfont \(\displaystyle 200\)}%
\end{pgfscope}%
\begin{pgfscope}%
\definecolor{textcolor}{rgb}{0.150000,0.150000,0.150000}%
\pgfsetstrokecolor{textcolor}%
\pgfsetfillcolor{textcolor}%
\pgftext[x=1.737375in,y=0.163303in,,top]{\color{textcolor}\rmfamily\fontsize{8.000000}{9.600000}\selectfont Step}%
\end{pgfscope}%
\begin{pgfscope}%
\pgfpathrectangle{\pgfqpoint{0.423750in}{0.375000in}}{\pgfqpoint{2.627250in}{2.265000in}}%
\pgfusepath{clip}%
\pgfsetroundcap%
\pgfsetroundjoin%
\pgfsetlinewidth{0.803000pt}%
\definecolor{currentstroke}{rgb}{1.000000,1.000000,1.000000}%
\pgfsetstrokecolor{currentstroke}%
\pgfsetdash{}{0pt}%
\pgfpathmoveto{\pgfqpoint{0.423750in}{1.296164in}}%
\pgfpathlineto{\pgfqpoint{3.051000in}{1.296164in}}%
\pgfusepath{stroke}%
\end{pgfscope}%
\begin{pgfscope}%
\definecolor{textcolor}{rgb}{0.150000,0.150000,0.150000}%
\pgfsetstrokecolor{textcolor}%
\pgfsetfillcolor{textcolor}%
\pgftext[x=0.118966in,y=1.253954in,left,base]{\color{textcolor}\rmfamily\fontsize{8.000000}{9.600000}\selectfont \(\displaystyle 10^{-1}\)}%
\end{pgfscope}%
\begin{pgfscope}%
\pgfpathrectangle{\pgfqpoint{0.423750in}{0.375000in}}{\pgfqpoint{2.627250in}{2.265000in}}%
\pgfusepath{clip}%
\pgfsetroundcap%
\pgfsetroundjoin%
\pgfsetlinewidth{0.803000pt}%
\definecolor{currentstroke}{rgb}{1.000000,1.000000,1.000000}%
\pgfsetstrokecolor{currentstroke}%
\pgfsetdash{}{0pt}%
\pgfpathmoveto{\pgfqpoint{0.423750in}{2.285522in}}%
\pgfpathlineto{\pgfqpoint{3.051000in}{2.285522in}}%
\pgfusepath{stroke}%
\end{pgfscope}%
\begin{pgfscope}%
\definecolor{textcolor}{rgb}{0.150000,0.150000,0.150000}%
\pgfsetstrokecolor{textcolor}%
\pgfsetfillcolor{textcolor}%
\pgftext[x=0.199212in,y=2.243313in,left,base]{\color{textcolor}\rmfamily\fontsize{8.000000}{9.600000}\selectfont \(\displaystyle 10^{0}\)}%
\end{pgfscope}%
\begin{pgfscope}%
\definecolor{textcolor}{rgb}{0.150000,0.150000,0.150000}%
\pgfsetstrokecolor{textcolor}%
\pgfsetfillcolor{textcolor}%
\pgftext[x=0.063410in,y=1.507500in,,bottom,rotate=90.000000]{\color{textcolor}\rmfamily\fontsize{8.000000}{9.600000}\selectfont Simple Regret}%
\end{pgfscope}%
\begin{pgfscope}%
\pgfpathrectangle{\pgfqpoint{0.423750in}{0.375000in}}{\pgfqpoint{2.627250in}{2.265000in}}%
\pgfusepath{clip}%
\pgfsetbuttcap%
\pgfsetroundjoin%
\definecolor{currentfill}{rgb}{0.121569,0.466667,0.705882}%
\pgfsetfillcolor{currentfill}%
\pgfsetfillopacity{0.200000}%
\pgfsetlinewidth{0.000000pt}%
\definecolor{currentstroke}{rgb}{0.000000,0.000000,0.000000}%
\pgfsetstrokecolor{currentstroke}%
\pgfsetdash{}{0pt}%
\pgfpathmoveto{\pgfqpoint{0.543170in}{2.337578in}}%
\pgfpathlineto{\pgfqpoint{0.543170in}{2.515644in}}%
\pgfpathlineto{\pgfqpoint{0.554076in}{2.411059in}}%
\pgfpathlineto{\pgfqpoint{0.564982in}{2.386957in}}%
\pgfpathlineto{\pgfqpoint{0.575888in}{2.378941in}}%
\pgfpathlineto{\pgfqpoint{0.586794in}{2.360111in}}%
\pgfpathlineto{\pgfqpoint{0.597700in}{2.274248in}}%
\pgfpathlineto{\pgfqpoint{0.608606in}{2.245857in}}%
\pgfpathlineto{\pgfqpoint{0.619512in}{2.245857in}}%
\pgfpathlineto{\pgfqpoint{0.630418in}{2.199714in}}%
\pgfpathlineto{\pgfqpoint{0.641324in}{2.187949in}}%
\pgfpathlineto{\pgfqpoint{0.652230in}{2.137246in}}%
\pgfpathlineto{\pgfqpoint{0.663136in}{2.068647in}}%
\pgfpathlineto{\pgfqpoint{0.674042in}{2.068647in}}%
\pgfpathlineto{\pgfqpoint{0.684948in}{2.068647in}}%
\pgfpathlineto{\pgfqpoint{0.695854in}{1.983511in}}%
\pgfpathlineto{\pgfqpoint{0.706760in}{1.927438in}}%
\pgfpathlineto{\pgfqpoint{0.717666in}{1.908266in}}%
\pgfpathlineto{\pgfqpoint{0.728572in}{1.865665in}}%
\pgfpathlineto{\pgfqpoint{0.739478in}{1.865665in}}%
\pgfpathlineto{\pgfqpoint{0.750384in}{1.865665in}}%
\pgfpathlineto{\pgfqpoint{0.761290in}{1.865665in}}%
\pgfpathlineto{\pgfqpoint{0.772196in}{1.865665in}}%
\pgfpathlineto{\pgfqpoint{0.783102in}{1.865665in}}%
\pgfpathlineto{\pgfqpoint{0.794008in}{1.865665in}}%
\pgfpathlineto{\pgfqpoint{0.804914in}{1.732883in}}%
\pgfpathlineto{\pgfqpoint{0.815820in}{1.732883in}}%
\pgfpathlineto{\pgfqpoint{0.826726in}{1.732883in}}%
\pgfpathlineto{\pgfqpoint{0.837632in}{1.732883in}}%
\pgfpathlineto{\pgfqpoint{0.848538in}{1.732883in}}%
\pgfpathlineto{\pgfqpoint{0.859444in}{1.732883in}}%
\pgfpathlineto{\pgfqpoint{0.870350in}{1.732883in}}%
\pgfpathlineto{\pgfqpoint{0.881256in}{1.732883in}}%
\pgfpathlineto{\pgfqpoint{0.892162in}{1.732883in}}%
\pgfpathlineto{\pgfqpoint{0.903068in}{1.732883in}}%
\pgfpathlineto{\pgfqpoint{0.913974in}{1.730283in}}%
\pgfpathlineto{\pgfqpoint{0.924880in}{1.730283in}}%
\pgfpathlineto{\pgfqpoint{0.935786in}{1.730283in}}%
\pgfpathlineto{\pgfqpoint{0.946692in}{1.730283in}}%
\pgfpathlineto{\pgfqpoint{0.957598in}{1.730283in}}%
\pgfpathlineto{\pgfqpoint{0.968504in}{1.730283in}}%
\pgfpathlineto{\pgfqpoint{0.979410in}{1.707946in}}%
\pgfpathlineto{\pgfqpoint{0.990316in}{1.707946in}}%
\pgfpathlineto{\pgfqpoint{1.001222in}{1.707946in}}%
\pgfpathlineto{\pgfqpoint{1.012127in}{1.707946in}}%
\pgfpathlineto{\pgfqpoint{1.023033in}{1.707946in}}%
\pgfpathlineto{\pgfqpoint{1.033939in}{1.707946in}}%
\pgfpathlineto{\pgfqpoint{1.044845in}{1.685673in}}%
\pgfpathlineto{\pgfqpoint{1.055751in}{1.685673in}}%
\pgfpathlineto{\pgfqpoint{1.066657in}{1.685673in}}%
\pgfpathlineto{\pgfqpoint{1.077563in}{1.685673in}}%
\pgfpathlineto{\pgfqpoint{1.088469in}{1.685673in}}%
\pgfpathlineto{\pgfqpoint{1.099375in}{1.685673in}}%
\pgfpathlineto{\pgfqpoint{1.110281in}{1.685673in}}%
\pgfpathlineto{\pgfqpoint{1.121187in}{1.685673in}}%
\pgfpathlineto{\pgfqpoint{1.132093in}{1.685673in}}%
\pgfpathlineto{\pgfqpoint{1.142999in}{1.685673in}}%
\pgfpathlineto{\pgfqpoint{1.153905in}{1.685673in}}%
\pgfpathlineto{\pgfqpoint{1.164811in}{1.685673in}}%
\pgfpathlineto{\pgfqpoint{1.175717in}{1.685673in}}%
\pgfpathlineto{\pgfqpoint{1.186623in}{1.638935in}}%
\pgfpathlineto{\pgfqpoint{1.197529in}{1.620196in}}%
\pgfpathlineto{\pgfqpoint{1.208435in}{1.606683in}}%
\pgfpathlineto{\pgfqpoint{1.219341in}{1.606683in}}%
\pgfpathlineto{\pgfqpoint{1.230247in}{1.606683in}}%
\pgfpathlineto{\pgfqpoint{1.241153in}{1.550234in}}%
\pgfpathlineto{\pgfqpoint{1.252059in}{1.550234in}}%
\pgfpathlineto{\pgfqpoint{1.262965in}{1.550234in}}%
\pgfpathlineto{\pgfqpoint{1.273871in}{1.550234in}}%
\pgfpathlineto{\pgfqpoint{1.284777in}{1.550234in}}%
\pgfpathlineto{\pgfqpoint{1.295683in}{1.550234in}}%
\pgfpathlineto{\pgfqpoint{1.306589in}{1.550234in}}%
\pgfpathlineto{\pgfqpoint{1.317495in}{1.550234in}}%
\pgfpathlineto{\pgfqpoint{1.328401in}{1.524995in}}%
\pgfpathlineto{\pgfqpoint{1.339307in}{1.524995in}}%
\pgfpathlineto{\pgfqpoint{1.350213in}{1.524995in}}%
\pgfpathlineto{\pgfqpoint{1.361119in}{1.524995in}}%
\pgfpathlineto{\pgfqpoint{1.372025in}{1.524995in}}%
\pgfpathlineto{\pgfqpoint{1.382931in}{1.524995in}}%
\pgfpathlineto{\pgfqpoint{1.393837in}{1.524995in}}%
\pgfpathlineto{\pgfqpoint{1.404743in}{1.524995in}}%
\pgfpathlineto{\pgfqpoint{1.415649in}{1.524995in}}%
\pgfpathlineto{\pgfqpoint{1.426555in}{1.524995in}}%
\pgfpathlineto{\pgfqpoint{1.437461in}{1.524995in}}%
\pgfpathlineto{\pgfqpoint{1.448367in}{1.524995in}}%
\pgfpathlineto{\pgfqpoint{1.459273in}{1.524995in}}%
\pgfpathlineto{\pgfqpoint{1.470179in}{1.524657in}}%
\pgfpathlineto{\pgfqpoint{1.481085in}{1.524657in}}%
\pgfpathlineto{\pgfqpoint{1.491991in}{1.524657in}}%
\pgfpathlineto{\pgfqpoint{1.502896in}{1.524657in}}%
\pgfpathlineto{\pgfqpoint{1.513802in}{1.524657in}}%
\pgfpathlineto{\pgfqpoint{1.524708in}{1.524657in}}%
\pgfpathlineto{\pgfqpoint{1.535614in}{1.524657in}}%
\pgfpathlineto{\pgfqpoint{1.546520in}{1.524657in}}%
\pgfpathlineto{\pgfqpoint{1.557426in}{1.524657in}}%
\pgfpathlineto{\pgfqpoint{1.568332in}{1.524657in}}%
\pgfpathlineto{\pgfqpoint{1.579238in}{1.524657in}}%
\pgfpathlineto{\pgfqpoint{1.590144in}{1.524657in}}%
\pgfpathlineto{\pgfqpoint{1.601050in}{1.524657in}}%
\pgfpathlineto{\pgfqpoint{1.611956in}{1.524657in}}%
\pgfpathlineto{\pgfqpoint{1.622862in}{1.524657in}}%
\pgfpathlineto{\pgfqpoint{1.633768in}{1.524657in}}%
\pgfpathlineto{\pgfqpoint{1.644674in}{1.524657in}}%
\pgfpathlineto{\pgfqpoint{1.655580in}{1.524657in}}%
\pgfpathlineto{\pgfqpoint{1.666486in}{1.524657in}}%
\pgfpathlineto{\pgfqpoint{1.677392in}{1.503372in}}%
\pgfpathlineto{\pgfqpoint{1.688298in}{1.503372in}}%
\pgfpathlineto{\pgfqpoint{1.699204in}{1.503372in}}%
\pgfpathlineto{\pgfqpoint{1.710110in}{1.503372in}}%
\pgfpathlineto{\pgfqpoint{1.721016in}{1.503372in}}%
\pgfpathlineto{\pgfqpoint{1.731922in}{1.503372in}}%
\pgfpathlineto{\pgfqpoint{1.742828in}{1.503372in}}%
\pgfpathlineto{\pgfqpoint{1.753734in}{1.503372in}}%
\pgfpathlineto{\pgfqpoint{1.764640in}{1.503372in}}%
\pgfpathlineto{\pgfqpoint{1.775546in}{1.503372in}}%
\pgfpathlineto{\pgfqpoint{1.786452in}{1.503372in}}%
\pgfpathlineto{\pgfqpoint{1.797358in}{1.503372in}}%
\pgfpathlineto{\pgfqpoint{1.808264in}{1.503372in}}%
\pgfpathlineto{\pgfqpoint{1.819170in}{1.503372in}}%
\pgfpathlineto{\pgfqpoint{1.830076in}{1.503372in}}%
\pgfpathlineto{\pgfqpoint{1.840982in}{1.456858in}}%
\pgfpathlineto{\pgfqpoint{1.851888in}{1.456858in}}%
\pgfpathlineto{\pgfqpoint{1.862794in}{1.456858in}}%
\pgfpathlineto{\pgfqpoint{1.873700in}{1.456858in}}%
\pgfpathlineto{\pgfqpoint{1.884606in}{1.455275in}}%
\pgfpathlineto{\pgfqpoint{1.895512in}{1.455275in}}%
\pgfpathlineto{\pgfqpoint{1.906418in}{1.455275in}}%
\pgfpathlineto{\pgfqpoint{1.917324in}{1.455275in}}%
\pgfpathlineto{\pgfqpoint{1.928230in}{1.455275in}}%
\pgfpathlineto{\pgfqpoint{1.939136in}{1.455275in}}%
\pgfpathlineto{\pgfqpoint{1.950042in}{1.455275in}}%
\pgfpathlineto{\pgfqpoint{1.960948in}{1.455275in}}%
\pgfpathlineto{\pgfqpoint{1.971854in}{1.455275in}}%
\pgfpathlineto{\pgfqpoint{1.982759in}{1.455275in}}%
\pgfpathlineto{\pgfqpoint{1.993665in}{1.455275in}}%
\pgfpathlineto{\pgfqpoint{2.004571in}{1.455275in}}%
\pgfpathlineto{\pgfqpoint{2.015477in}{1.455275in}}%
\pgfpathlineto{\pgfqpoint{2.026383in}{1.455275in}}%
\pgfpathlineto{\pgfqpoint{2.037289in}{1.455275in}}%
\pgfpathlineto{\pgfqpoint{2.048195in}{1.455275in}}%
\pgfpathlineto{\pgfqpoint{2.059101in}{1.455275in}}%
\pgfpathlineto{\pgfqpoint{2.070007in}{1.455275in}}%
\pgfpathlineto{\pgfqpoint{2.080913in}{1.455275in}}%
\pgfpathlineto{\pgfqpoint{2.091819in}{1.455275in}}%
\pgfpathlineto{\pgfqpoint{2.102725in}{1.455275in}}%
\pgfpathlineto{\pgfqpoint{2.113631in}{1.455275in}}%
\pgfpathlineto{\pgfqpoint{2.124537in}{1.455275in}}%
\pgfpathlineto{\pgfqpoint{2.135443in}{1.455275in}}%
\pgfpathlineto{\pgfqpoint{2.146349in}{1.455275in}}%
\pgfpathlineto{\pgfqpoint{2.157255in}{1.455275in}}%
\pgfpathlineto{\pgfqpoint{2.168161in}{1.455275in}}%
\pgfpathlineto{\pgfqpoint{2.179067in}{1.455275in}}%
\pgfpathlineto{\pgfqpoint{2.189973in}{1.455275in}}%
\pgfpathlineto{\pgfqpoint{2.200879in}{1.455275in}}%
\pgfpathlineto{\pgfqpoint{2.211785in}{1.455275in}}%
\pgfpathlineto{\pgfqpoint{2.222691in}{1.455275in}}%
\pgfpathlineto{\pgfqpoint{2.233597in}{1.455275in}}%
\pgfpathlineto{\pgfqpoint{2.244503in}{1.444355in}}%
\pgfpathlineto{\pgfqpoint{2.255409in}{1.444355in}}%
\pgfpathlineto{\pgfqpoint{2.266315in}{1.444355in}}%
\pgfpathlineto{\pgfqpoint{2.277221in}{1.444355in}}%
\pgfpathlineto{\pgfqpoint{2.288127in}{1.444355in}}%
\pgfpathlineto{\pgfqpoint{2.299033in}{1.444355in}}%
\pgfpathlineto{\pgfqpoint{2.309939in}{1.440321in}}%
\pgfpathlineto{\pgfqpoint{2.320845in}{1.440321in}}%
\pgfpathlineto{\pgfqpoint{2.331751in}{1.440321in}}%
\pgfpathlineto{\pgfqpoint{2.342657in}{1.440321in}}%
\pgfpathlineto{\pgfqpoint{2.353563in}{1.440321in}}%
\pgfpathlineto{\pgfqpoint{2.364469in}{1.440321in}}%
\pgfpathlineto{\pgfqpoint{2.375375in}{1.440321in}}%
\pgfpathlineto{\pgfqpoint{2.386281in}{1.440321in}}%
\pgfpathlineto{\pgfqpoint{2.397187in}{1.440321in}}%
\pgfpathlineto{\pgfqpoint{2.408093in}{1.440321in}}%
\pgfpathlineto{\pgfqpoint{2.418999in}{1.440321in}}%
\pgfpathlineto{\pgfqpoint{2.429905in}{1.440321in}}%
\pgfpathlineto{\pgfqpoint{2.440811in}{1.440321in}}%
\pgfpathlineto{\pgfqpoint{2.451717in}{1.440321in}}%
\pgfpathlineto{\pgfqpoint{2.462623in}{1.440321in}}%
\pgfpathlineto{\pgfqpoint{2.473528in}{1.440321in}}%
\pgfpathlineto{\pgfqpoint{2.484434in}{1.440321in}}%
\pgfpathlineto{\pgfqpoint{2.495340in}{1.440321in}}%
\pgfpathlineto{\pgfqpoint{2.506246in}{1.440321in}}%
\pgfpathlineto{\pgfqpoint{2.517152in}{1.440321in}}%
\pgfpathlineto{\pgfqpoint{2.528058in}{1.440321in}}%
\pgfpathlineto{\pgfqpoint{2.538964in}{1.440321in}}%
\pgfpathlineto{\pgfqpoint{2.549870in}{1.440321in}}%
\pgfpathlineto{\pgfqpoint{2.560776in}{1.440321in}}%
\pgfpathlineto{\pgfqpoint{2.571682in}{1.440321in}}%
\pgfpathlineto{\pgfqpoint{2.582588in}{1.440321in}}%
\pgfpathlineto{\pgfqpoint{2.593494in}{1.440321in}}%
\pgfpathlineto{\pgfqpoint{2.604400in}{1.440321in}}%
\pgfpathlineto{\pgfqpoint{2.615306in}{1.440321in}}%
\pgfpathlineto{\pgfqpoint{2.626212in}{1.440321in}}%
\pgfpathlineto{\pgfqpoint{2.637118in}{1.440321in}}%
\pgfpathlineto{\pgfqpoint{2.648024in}{1.440321in}}%
\pgfpathlineto{\pgfqpoint{2.658930in}{1.440321in}}%
\pgfpathlineto{\pgfqpoint{2.669836in}{1.440321in}}%
\pgfpathlineto{\pgfqpoint{2.680742in}{1.440321in}}%
\pgfpathlineto{\pgfqpoint{2.691648in}{1.440321in}}%
\pgfpathlineto{\pgfqpoint{2.702554in}{1.440321in}}%
\pgfpathlineto{\pgfqpoint{2.713460in}{1.440321in}}%
\pgfpathlineto{\pgfqpoint{2.724366in}{1.440321in}}%
\pgfpathlineto{\pgfqpoint{2.735272in}{1.440321in}}%
\pgfpathlineto{\pgfqpoint{2.746178in}{1.440321in}}%
\pgfpathlineto{\pgfqpoint{2.757084in}{1.440321in}}%
\pgfpathlineto{\pgfqpoint{2.767990in}{1.440321in}}%
\pgfpathlineto{\pgfqpoint{2.778896in}{1.440321in}}%
\pgfpathlineto{\pgfqpoint{2.789802in}{1.410833in}}%
\pgfpathlineto{\pgfqpoint{2.800708in}{1.410833in}}%
\pgfpathlineto{\pgfqpoint{2.811614in}{1.410833in}}%
\pgfpathlineto{\pgfqpoint{2.822520in}{1.410833in}}%
\pgfpathlineto{\pgfqpoint{2.833426in}{1.410833in}}%
\pgfpathlineto{\pgfqpoint{2.844332in}{1.410833in}}%
\pgfpathlineto{\pgfqpoint{2.855238in}{1.410833in}}%
\pgfpathlineto{\pgfqpoint{2.866144in}{1.410833in}}%
\pgfpathlineto{\pgfqpoint{2.877050in}{1.410833in}}%
\pgfpathlineto{\pgfqpoint{2.887956in}{1.410833in}}%
\pgfpathlineto{\pgfqpoint{2.898862in}{1.410833in}}%
\pgfpathlineto{\pgfqpoint{2.909768in}{1.410833in}}%
\pgfpathlineto{\pgfqpoint{2.920674in}{1.410833in}}%
\pgfpathlineto{\pgfqpoint{2.931580in}{1.410833in}}%
\pgfpathlineto{\pgfqpoint{2.931580in}{1.222960in}}%
\pgfpathlineto{\pgfqpoint{2.931580in}{1.222960in}}%
\pgfpathlineto{\pgfqpoint{2.920674in}{1.222960in}}%
\pgfpathlineto{\pgfqpoint{2.909768in}{1.222960in}}%
\pgfpathlineto{\pgfqpoint{2.898862in}{1.222960in}}%
\pgfpathlineto{\pgfqpoint{2.887956in}{1.222960in}}%
\pgfpathlineto{\pgfqpoint{2.877050in}{1.222960in}}%
\pgfpathlineto{\pgfqpoint{2.866144in}{1.222960in}}%
\pgfpathlineto{\pgfqpoint{2.855238in}{1.222960in}}%
\pgfpathlineto{\pgfqpoint{2.844332in}{1.222960in}}%
\pgfpathlineto{\pgfqpoint{2.833426in}{1.222960in}}%
\pgfpathlineto{\pgfqpoint{2.822520in}{1.222960in}}%
\pgfpathlineto{\pgfqpoint{2.811614in}{1.222960in}}%
\pgfpathlineto{\pgfqpoint{2.800708in}{1.222960in}}%
\pgfpathlineto{\pgfqpoint{2.789802in}{1.222960in}}%
\pgfpathlineto{\pgfqpoint{2.778896in}{1.244637in}}%
\pgfpathlineto{\pgfqpoint{2.767990in}{1.244637in}}%
\pgfpathlineto{\pgfqpoint{2.757084in}{1.244637in}}%
\pgfpathlineto{\pgfqpoint{2.746178in}{1.244637in}}%
\pgfpathlineto{\pgfqpoint{2.735272in}{1.244637in}}%
\pgfpathlineto{\pgfqpoint{2.724366in}{1.244637in}}%
\pgfpathlineto{\pgfqpoint{2.713460in}{1.244637in}}%
\pgfpathlineto{\pgfqpoint{2.702554in}{1.244637in}}%
\pgfpathlineto{\pgfqpoint{2.691648in}{1.244637in}}%
\pgfpathlineto{\pgfqpoint{2.680742in}{1.244637in}}%
\pgfpathlineto{\pgfqpoint{2.669836in}{1.244637in}}%
\pgfpathlineto{\pgfqpoint{2.658930in}{1.244637in}}%
\pgfpathlineto{\pgfqpoint{2.648024in}{1.244637in}}%
\pgfpathlineto{\pgfqpoint{2.637118in}{1.244637in}}%
\pgfpathlineto{\pgfqpoint{2.626212in}{1.244637in}}%
\pgfpathlineto{\pgfqpoint{2.615306in}{1.244637in}}%
\pgfpathlineto{\pgfqpoint{2.604400in}{1.244637in}}%
\pgfpathlineto{\pgfqpoint{2.593494in}{1.244637in}}%
\pgfpathlineto{\pgfqpoint{2.582588in}{1.244637in}}%
\pgfpathlineto{\pgfqpoint{2.571682in}{1.244637in}}%
\pgfpathlineto{\pgfqpoint{2.560776in}{1.244637in}}%
\pgfpathlineto{\pgfqpoint{2.549870in}{1.244637in}}%
\pgfpathlineto{\pgfqpoint{2.538964in}{1.244637in}}%
\pgfpathlineto{\pgfqpoint{2.528058in}{1.244637in}}%
\pgfpathlineto{\pgfqpoint{2.517152in}{1.244637in}}%
\pgfpathlineto{\pgfqpoint{2.506246in}{1.244637in}}%
\pgfpathlineto{\pgfqpoint{2.495340in}{1.244637in}}%
\pgfpathlineto{\pgfqpoint{2.484434in}{1.244637in}}%
\pgfpathlineto{\pgfqpoint{2.473528in}{1.244637in}}%
\pgfpathlineto{\pgfqpoint{2.462623in}{1.244637in}}%
\pgfpathlineto{\pgfqpoint{2.451717in}{1.244637in}}%
\pgfpathlineto{\pgfqpoint{2.440811in}{1.244637in}}%
\pgfpathlineto{\pgfqpoint{2.429905in}{1.244637in}}%
\pgfpathlineto{\pgfqpoint{2.418999in}{1.244637in}}%
\pgfpathlineto{\pgfqpoint{2.408093in}{1.244637in}}%
\pgfpathlineto{\pgfqpoint{2.397187in}{1.244637in}}%
\pgfpathlineto{\pgfqpoint{2.386281in}{1.244637in}}%
\pgfpathlineto{\pgfqpoint{2.375375in}{1.244637in}}%
\pgfpathlineto{\pgfqpoint{2.364469in}{1.244637in}}%
\pgfpathlineto{\pgfqpoint{2.353563in}{1.244637in}}%
\pgfpathlineto{\pgfqpoint{2.342657in}{1.244637in}}%
\pgfpathlineto{\pgfqpoint{2.331751in}{1.244637in}}%
\pgfpathlineto{\pgfqpoint{2.320845in}{1.244637in}}%
\pgfpathlineto{\pgfqpoint{2.309939in}{1.244637in}}%
\pgfpathlineto{\pgfqpoint{2.299033in}{1.267226in}}%
\pgfpathlineto{\pgfqpoint{2.288127in}{1.267226in}}%
\pgfpathlineto{\pgfqpoint{2.277221in}{1.267226in}}%
\pgfpathlineto{\pgfqpoint{2.266315in}{1.267226in}}%
\pgfpathlineto{\pgfqpoint{2.255409in}{1.267226in}}%
\pgfpathlineto{\pgfqpoint{2.244503in}{1.267226in}}%
\pgfpathlineto{\pgfqpoint{2.233597in}{1.273841in}}%
\pgfpathlineto{\pgfqpoint{2.222691in}{1.273841in}}%
\pgfpathlineto{\pgfqpoint{2.211785in}{1.273841in}}%
\pgfpathlineto{\pgfqpoint{2.200879in}{1.273841in}}%
\pgfpathlineto{\pgfqpoint{2.189973in}{1.273841in}}%
\pgfpathlineto{\pgfqpoint{2.179067in}{1.273841in}}%
\pgfpathlineto{\pgfqpoint{2.168161in}{1.273841in}}%
\pgfpathlineto{\pgfqpoint{2.157255in}{1.273841in}}%
\pgfpathlineto{\pgfqpoint{2.146349in}{1.273841in}}%
\pgfpathlineto{\pgfqpoint{2.135443in}{1.273841in}}%
\pgfpathlineto{\pgfqpoint{2.124537in}{1.273841in}}%
\pgfpathlineto{\pgfqpoint{2.113631in}{1.273841in}}%
\pgfpathlineto{\pgfqpoint{2.102725in}{1.273841in}}%
\pgfpathlineto{\pgfqpoint{2.091819in}{1.273841in}}%
\pgfpathlineto{\pgfqpoint{2.080913in}{1.273841in}}%
\pgfpathlineto{\pgfqpoint{2.070007in}{1.273841in}}%
\pgfpathlineto{\pgfqpoint{2.059101in}{1.273841in}}%
\pgfpathlineto{\pgfqpoint{2.048195in}{1.273841in}}%
\pgfpathlineto{\pgfqpoint{2.037289in}{1.273841in}}%
\pgfpathlineto{\pgfqpoint{2.026383in}{1.273841in}}%
\pgfpathlineto{\pgfqpoint{2.015477in}{1.273841in}}%
\pgfpathlineto{\pgfqpoint{2.004571in}{1.273841in}}%
\pgfpathlineto{\pgfqpoint{1.993665in}{1.273841in}}%
\pgfpathlineto{\pgfqpoint{1.982759in}{1.273841in}}%
\pgfpathlineto{\pgfqpoint{1.971854in}{1.273841in}}%
\pgfpathlineto{\pgfqpoint{1.960948in}{1.273841in}}%
\pgfpathlineto{\pgfqpoint{1.950042in}{1.273841in}}%
\pgfpathlineto{\pgfqpoint{1.939136in}{1.273841in}}%
\pgfpathlineto{\pgfqpoint{1.928230in}{1.273841in}}%
\pgfpathlineto{\pgfqpoint{1.917324in}{1.273841in}}%
\pgfpathlineto{\pgfqpoint{1.906418in}{1.273841in}}%
\pgfpathlineto{\pgfqpoint{1.895512in}{1.273841in}}%
\pgfpathlineto{\pgfqpoint{1.884606in}{1.273841in}}%
\pgfpathlineto{\pgfqpoint{1.873700in}{1.279619in}}%
\pgfpathlineto{\pgfqpoint{1.862794in}{1.279619in}}%
\pgfpathlineto{\pgfqpoint{1.851888in}{1.279619in}}%
\pgfpathlineto{\pgfqpoint{1.840982in}{1.279619in}}%
\pgfpathlineto{\pgfqpoint{1.830076in}{1.277481in}}%
\pgfpathlineto{\pgfqpoint{1.819170in}{1.277481in}}%
\pgfpathlineto{\pgfqpoint{1.808264in}{1.277481in}}%
\pgfpathlineto{\pgfqpoint{1.797358in}{1.277481in}}%
\pgfpathlineto{\pgfqpoint{1.786452in}{1.277481in}}%
\pgfpathlineto{\pgfqpoint{1.775546in}{1.277481in}}%
\pgfpathlineto{\pgfqpoint{1.764640in}{1.277481in}}%
\pgfpathlineto{\pgfqpoint{1.753734in}{1.277481in}}%
\pgfpathlineto{\pgfqpoint{1.742828in}{1.277481in}}%
\pgfpathlineto{\pgfqpoint{1.731922in}{1.277481in}}%
\pgfpathlineto{\pgfqpoint{1.721016in}{1.277481in}}%
\pgfpathlineto{\pgfqpoint{1.710110in}{1.277481in}}%
\pgfpathlineto{\pgfqpoint{1.699204in}{1.277481in}}%
\pgfpathlineto{\pgfqpoint{1.688298in}{1.277481in}}%
\pgfpathlineto{\pgfqpoint{1.677392in}{1.277481in}}%
\pgfpathlineto{\pgfqpoint{1.666486in}{1.273556in}}%
\pgfpathlineto{\pgfqpoint{1.655580in}{1.273556in}}%
\pgfpathlineto{\pgfqpoint{1.644674in}{1.273556in}}%
\pgfpathlineto{\pgfqpoint{1.633768in}{1.273556in}}%
\pgfpathlineto{\pgfqpoint{1.622862in}{1.273556in}}%
\pgfpathlineto{\pgfqpoint{1.611956in}{1.273556in}}%
\pgfpathlineto{\pgfqpoint{1.601050in}{1.273556in}}%
\pgfpathlineto{\pgfqpoint{1.590144in}{1.273556in}}%
\pgfpathlineto{\pgfqpoint{1.579238in}{1.273556in}}%
\pgfpathlineto{\pgfqpoint{1.568332in}{1.273556in}}%
\pgfpathlineto{\pgfqpoint{1.557426in}{1.273556in}}%
\pgfpathlineto{\pgfqpoint{1.546520in}{1.273556in}}%
\pgfpathlineto{\pgfqpoint{1.535614in}{1.273556in}}%
\pgfpathlineto{\pgfqpoint{1.524708in}{1.273556in}}%
\pgfpathlineto{\pgfqpoint{1.513802in}{1.273556in}}%
\pgfpathlineto{\pgfqpoint{1.502896in}{1.273556in}}%
\pgfpathlineto{\pgfqpoint{1.491991in}{1.273556in}}%
\pgfpathlineto{\pgfqpoint{1.481085in}{1.273556in}}%
\pgfpathlineto{\pgfqpoint{1.470179in}{1.273556in}}%
\pgfpathlineto{\pgfqpoint{1.459273in}{1.274588in}}%
\pgfpathlineto{\pgfqpoint{1.448367in}{1.274588in}}%
\pgfpathlineto{\pgfqpoint{1.437461in}{1.274588in}}%
\pgfpathlineto{\pgfqpoint{1.426555in}{1.274588in}}%
\pgfpathlineto{\pgfqpoint{1.415649in}{1.274588in}}%
\pgfpathlineto{\pgfqpoint{1.404743in}{1.274588in}}%
\pgfpathlineto{\pgfqpoint{1.393837in}{1.274588in}}%
\pgfpathlineto{\pgfqpoint{1.382931in}{1.274588in}}%
\pgfpathlineto{\pgfqpoint{1.372025in}{1.274588in}}%
\pgfpathlineto{\pgfqpoint{1.361119in}{1.274588in}}%
\pgfpathlineto{\pgfqpoint{1.350213in}{1.274588in}}%
\pgfpathlineto{\pgfqpoint{1.339307in}{1.274588in}}%
\pgfpathlineto{\pgfqpoint{1.328401in}{1.274588in}}%
\pgfpathlineto{\pgfqpoint{1.317495in}{1.315463in}}%
\pgfpathlineto{\pgfqpoint{1.306589in}{1.315463in}}%
\pgfpathlineto{\pgfqpoint{1.295683in}{1.315463in}}%
\pgfpathlineto{\pgfqpoint{1.284777in}{1.315463in}}%
\pgfpathlineto{\pgfqpoint{1.273871in}{1.315463in}}%
\pgfpathlineto{\pgfqpoint{1.262965in}{1.315463in}}%
\pgfpathlineto{\pgfqpoint{1.252059in}{1.315463in}}%
\pgfpathlineto{\pgfqpoint{1.241153in}{1.315463in}}%
\pgfpathlineto{\pgfqpoint{1.230247in}{1.349200in}}%
\pgfpathlineto{\pgfqpoint{1.219341in}{1.349200in}}%
\pgfpathlineto{\pgfqpoint{1.208435in}{1.349200in}}%
\pgfpathlineto{\pgfqpoint{1.197529in}{1.350021in}}%
\pgfpathlineto{\pgfqpoint{1.186623in}{1.404847in}}%
\pgfpathlineto{\pgfqpoint{1.175717in}{1.474503in}}%
\pgfpathlineto{\pgfqpoint{1.164811in}{1.474503in}}%
\pgfpathlineto{\pgfqpoint{1.153905in}{1.474503in}}%
\pgfpathlineto{\pgfqpoint{1.142999in}{1.474503in}}%
\pgfpathlineto{\pgfqpoint{1.132093in}{1.474503in}}%
\pgfpathlineto{\pgfqpoint{1.121187in}{1.474503in}}%
\pgfpathlineto{\pgfqpoint{1.110281in}{1.474503in}}%
\pgfpathlineto{\pgfqpoint{1.099375in}{1.474503in}}%
\pgfpathlineto{\pgfqpoint{1.088469in}{1.474503in}}%
\pgfpathlineto{\pgfqpoint{1.077563in}{1.474503in}}%
\pgfpathlineto{\pgfqpoint{1.066657in}{1.474503in}}%
\pgfpathlineto{\pgfqpoint{1.055751in}{1.474503in}}%
\pgfpathlineto{\pgfqpoint{1.044845in}{1.474503in}}%
\pgfpathlineto{\pgfqpoint{1.033939in}{1.499542in}}%
\pgfpathlineto{\pgfqpoint{1.023033in}{1.499542in}}%
\pgfpathlineto{\pgfqpoint{1.012127in}{1.499542in}}%
\pgfpathlineto{\pgfqpoint{1.001222in}{1.499542in}}%
\pgfpathlineto{\pgfqpoint{0.990316in}{1.499542in}}%
\pgfpathlineto{\pgfqpoint{0.979410in}{1.499542in}}%
\pgfpathlineto{\pgfqpoint{0.968504in}{1.506328in}}%
\pgfpathlineto{\pgfqpoint{0.957598in}{1.506328in}}%
\pgfpathlineto{\pgfqpoint{0.946692in}{1.506328in}}%
\pgfpathlineto{\pgfqpoint{0.935786in}{1.506328in}}%
\pgfpathlineto{\pgfqpoint{0.924880in}{1.506328in}}%
\pgfpathlineto{\pgfqpoint{0.913974in}{1.506328in}}%
\pgfpathlineto{\pgfqpoint{0.903068in}{1.506728in}}%
\pgfpathlineto{\pgfqpoint{0.892162in}{1.506728in}}%
\pgfpathlineto{\pgfqpoint{0.881256in}{1.506728in}}%
\pgfpathlineto{\pgfqpoint{0.870350in}{1.506728in}}%
\pgfpathlineto{\pgfqpoint{0.859444in}{1.506728in}}%
\pgfpathlineto{\pgfqpoint{0.848538in}{1.506728in}}%
\pgfpathlineto{\pgfqpoint{0.837632in}{1.506728in}}%
\pgfpathlineto{\pgfqpoint{0.826726in}{1.506728in}}%
\pgfpathlineto{\pgfqpoint{0.815820in}{1.506728in}}%
\pgfpathlineto{\pgfqpoint{0.804914in}{1.506728in}}%
\pgfpathlineto{\pgfqpoint{0.794008in}{1.668840in}}%
\pgfpathlineto{\pgfqpoint{0.783102in}{1.668840in}}%
\pgfpathlineto{\pgfqpoint{0.772196in}{1.668840in}}%
\pgfpathlineto{\pgfqpoint{0.761290in}{1.668840in}}%
\pgfpathlineto{\pgfqpoint{0.750384in}{1.668840in}}%
\pgfpathlineto{\pgfqpoint{0.739478in}{1.668840in}}%
\pgfpathlineto{\pgfqpoint{0.728572in}{1.668840in}}%
\pgfpathlineto{\pgfqpoint{0.717666in}{1.669593in}}%
\pgfpathlineto{\pgfqpoint{0.706760in}{1.710109in}}%
\pgfpathlineto{\pgfqpoint{0.695854in}{1.759280in}}%
\pgfpathlineto{\pgfqpoint{0.684948in}{1.863963in}}%
\pgfpathlineto{\pgfqpoint{0.674042in}{1.863963in}}%
\pgfpathlineto{\pgfqpoint{0.663136in}{1.863963in}}%
\pgfpathlineto{\pgfqpoint{0.652230in}{1.918531in}}%
\pgfpathlineto{\pgfqpoint{0.641324in}{2.036989in}}%
\pgfpathlineto{\pgfqpoint{0.630418in}{2.065522in}}%
\pgfpathlineto{\pgfqpoint{0.619512in}{2.101231in}}%
\pgfpathlineto{\pgfqpoint{0.608606in}{2.101231in}}%
\pgfpathlineto{\pgfqpoint{0.597700in}{2.143979in}}%
\pgfpathlineto{\pgfqpoint{0.586794in}{2.183762in}}%
\pgfpathlineto{\pgfqpoint{0.575888in}{2.234212in}}%
\pgfpathlineto{\pgfqpoint{0.564982in}{2.241850in}}%
\pgfpathlineto{\pgfqpoint{0.554076in}{2.276083in}}%
\pgfpathlineto{\pgfqpoint{0.543170in}{2.337578in}}%
\pgfpathclose%
\pgfusepath{fill}%
\end{pgfscope}%
\begin{pgfscope}%
\pgfpathrectangle{\pgfqpoint{0.423750in}{0.375000in}}{\pgfqpoint{2.627250in}{2.265000in}}%
\pgfusepath{clip}%
\pgfsetbuttcap%
\pgfsetroundjoin%
\definecolor{currentfill}{rgb}{1.000000,0.498039,0.054902}%
\pgfsetfillcolor{currentfill}%
\pgfsetfillopacity{0.200000}%
\pgfsetlinewidth{0.000000pt}%
\definecolor{currentstroke}{rgb}{0.000000,0.000000,0.000000}%
\pgfsetstrokecolor{currentstroke}%
\pgfsetdash{}{0pt}%
\pgfpathmoveto{\pgfqpoint{0.543170in}{2.161353in}}%
\pgfpathlineto{\pgfqpoint{0.543170in}{2.375602in}}%
\pgfpathlineto{\pgfqpoint{0.554076in}{2.218141in}}%
\pgfpathlineto{\pgfqpoint{0.564982in}{2.163662in}}%
\pgfpathlineto{\pgfqpoint{0.575888in}{2.071187in}}%
\pgfpathlineto{\pgfqpoint{0.586794in}{2.071187in}}%
\pgfpathlineto{\pgfqpoint{0.597700in}{2.071187in}}%
\pgfpathlineto{\pgfqpoint{0.608606in}{2.008970in}}%
\pgfpathlineto{\pgfqpoint{0.619512in}{2.008970in}}%
\pgfpathlineto{\pgfqpoint{0.630418in}{2.008970in}}%
\pgfpathlineto{\pgfqpoint{0.641324in}{2.005390in}}%
\pgfpathlineto{\pgfqpoint{0.652230in}{2.005390in}}%
\pgfpathlineto{\pgfqpoint{0.663136in}{2.005390in}}%
\pgfpathlineto{\pgfqpoint{0.674042in}{2.005390in}}%
\pgfpathlineto{\pgfqpoint{0.684948in}{1.956448in}}%
\pgfpathlineto{\pgfqpoint{0.695854in}{1.956448in}}%
\pgfpathlineto{\pgfqpoint{0.706760in}{1.956448in}}%
\pgfpathlineto{\pgfqpoint{0.717666in}{1.956448in}}%
\pgfpathlineto{\pgfqpoint{0.728572in}{1.956448in}}%
\pgfpathlineto{\pgfqpoint{0.739478in}{1.956448in}}%
\pgfpathlineto{\pgfqpoint{0.750384in}{1.942056in}}%
\pgfpathlineto{\pgfqpoint{0.761290in}{1.942056in}}%
\pgfpathlineto{\pgfqpoint{0.772196in}{1.902764in}}%
\pgfpathlineto{\pgfqpoint{0.783102in}{1.902764in}}%
\pgfpathlineto{\pgfqpoint{0.794008in}{1.902764in}}%
\pgfpathlineto{\pgfqpoint{0.804914in}{1.902764in}}%
\pgfpathlineto{\pgfqpoint{0.815820in}{1.902764in}}%
\pgfpathlineto{\pgfqpoint{0.826726in}{1.829198in}}%
\pgfpathlineto{\pgfqpoint{0.837632in}{1.829198in}}%
\pgfpathlineto{\pgfqpoint{0.848538in}{1.827552in}}%
\pgfpathlineto{\pgfqpoint{0.859444in}{1.827552in}}%
\pgfpathlineto{\pgfqpoint{0.870350in}{1.827552in}}%
\pgfpathlineto{\pgfqpoint{0.881256in}{1.827552in}}%
\pgfpathlineto{\pgfqpoint{0.892162in}{1.827552in}}%
\pgfpathlineto{\pgfqpoint{0.903068in}{1.827552in}}%
\pgfpathlineto{\pgfqpoint{0.913974in}{1.827113in}}%
\pgfpathlineto{\pgfqpoint{0.924880in}{1.827113in}}%
\pgfpathlineto{\pgfqpoint{0.935786in}{1.827113in}}%
\pgfpathlineto{\pgfqpoint{0.946692in}{1.827113in}}%
\pgfpathlineto{\pgfqpoint{0.957598in}{1.827113in}}%
\pgfpathlineto{\pgfqpoint{0.968504in}{1.827113in}}%
\pgfpathlineto{\pgfqpoint{0.979410in}{1.785033in}}%
\pgfpathlineto{\pgfqpoint{0.990316in}{1.718371in}}%
\pgfpathlineto{\pgfqpoint{1.001222in}{1.718371in}}%
\pgfpathlineto{\pgfqpoint{1.012127in}{1.718371in}}%
\pgfpathlineto{\pgfqpoint{1.023033in}{1.718371in}}%
\pgfpathlineto{\pgfqpoint{1.033939in}{1.718371in}}%
\pgfpathlineto{\pgfqpoint{1.044845in}{1.718371in}}%
\pgfpathlineto{\pgfqpoint{1.055751in}{1.718371in}}%
\pgfpathlineto{\pgfqpoint{1.066657in}{1.718371in}}%
\pgfpathlineto{\pgfqpoint{1.077563in}{1.718371in}}%
\pgfpathlineto{\pgfqpoint{1.088469in}{1.718371in}}%
\pgfpathlineto{\pgfqpoint{1.099375in}{1.718371in}}%
\pgfpathlineto{\pgfqpoint{1.110281in}{1.625095in}}%
\pgfpathlineto{\pgfqpoint{1.121187in}{1.625095in}}%
\pgfpathlineto{\pgfqpoint{1.132093in}{1.625095in}}%
\pgfpathlineto{\pgfqpoint{1.142999in}{1.625095in}}%
\pgfpathlineto{\pgfqpoint{1.153905in}{1.625095in}}%
\pgfpathlineto{\pgfqpoint{1.164811in}{1.625095in}}%
\pgfpathlineto{\pgfqpoint{1.175717in}{1.625095in}}%
\pgfpathlineto{\pgfqpoint{1.186623in}{1.590584in}}%
\pgfpathlineto{\pgfqpoint{1.197529in}{1.590584in}}%
\pgfpathlineto{\pgfqpoint{1.208435in}{1.590584in}}%
\pgfpathlineto{\pgfqpoint{1.219341in}{1.579442in}}%
\pgfpathlineto{\pgfqpoint{1.230247in}{1.579442in}}%
\pgfpathlineto{\pgfqpoint{1.241153in}{1.579442in}}%
\pgfpathlineto{\pgfqpoint{1.252059in}{1.579442in}}%
\pgfpathlineto{\pgfqpoint{1.262965in}{1.579442in}}%
\pgfpathlineto{\pgfqpoint{1.273871in}{1.579442in}}%
\pgfpathlineto{\pgfqpoint{1.284777in}{1.579442in}}%
\pgfpathlineto{\pgfqpoint{1.295683in}{1.579442in}}%
\pgfpathlineto{\pgfqpoint{1.306589in}{1.577858in}}%
\pgfpathlineto{\pgfqpoint{1.317495in}{1.577858in}}%
\pgfpathlineto{\pgfqpoint{1.328401in}{1.577858in}}%
\pgfpathlineto{\pgfqpoint{1.339307in}{1.535136in}}%
\pgfpathlineto{\pgfqpoint{1.350213in}{1.535136in}}%
\pgfpathlineto{\pgfqpoint{1.361119in}{1.534291in}}%
\pgfpathlineto{\pgfqpoint{1.372025in}{1.534291in}}%
\pgfpathlineto{\pgfqpoint{1.382931in}{1.534291in}}%
\pgfpathlineto{\pgfqpoint{1.393837in}{1.534291in}}%
\pgfpathlineto{\pgfqpoint{1.404743in}{1.534291in}}%
\pgfpathlineto{\pgfqpoint{1.415649in}{1.534291in}}%
\pgfpathlineto{\pgfqpoint{1.426555in}{1.534291in}}%
\pgfpathlineto{\pgfqpoint{1.437461in}{1.534291in}}%
\pgfpathlineto{\pgfqpoint{1.448367in}{1.519478in}}%
\pgfpathlineto{\pgfqpoint{1.459273in}{1.519478in}}%
\pgfpathlineto{\pgfqpoint{1.470179in}{1.519478in}}%
\pgfpathlineto{\pgfqpoint{1.481085in}{1.447556in}}%
\pgfpathlineto{\pgfqpoint{1.491991in}{1.417613in}}%
\pgfpathlineto{\pgfqpoint{1.502896in}{1.417613in}}%
\pgfpathlineto{\pgfqpoint{1.513802in}{1.417613in}}%
\pgfpathlineto{\pgfqpoint{1.524708in}{1.417613in}}%
\pgfpathlineto{\pgfqpoint{1.535614in}{1.417613in}}%
\pgfpathlineto{\pgfqpoint{1.546520in}{1.417613in}}%
\pgfpathlineto{\pgfqpoint{1.557426in}{1.405539in}}%
\pgfpathlineto{\pgfqpoint{1.568332in}{1.405539in}}%
\pgfpathlineto{\pgfqpoint{1.579238in}{1.405539in}}%
\pgfpathlineto{\pgfqpoint{1.590144in}{1.405539in}}%
\pgfpathlineto{\pgfqpoint{1.601050in}{1.405539in}}%
\pgfpathlineto{\pgfqpoint{1.611956in}{1.384924in}}%
\pgfpathlineto{\pgfqpoint{1.622862in}{1.384924in}}%
\pgfpathlineto{\pgfqpoint{1.633768in}{1.384924in}}%
\pgfpathlineto{\pgfqpoint{1.644674in}{1.384924in}}%
\pgfpathlineto{\pgfqpoint{1.655580in}{1.384924in}}%
\pgfpathlineto{\pgfqpoint{1.666486in}{1.384924in}}%
\pgfpathlineto{\pgfqpoint{1.677392in}{1.384924in}}%
\pgfpathlineto{\pgfqpoint{1.688298in}{1.299957in}}%
\pgfpathlineto{\pgfqpoint{1.699204in}{1.299957in}}%
\pgfpathlineto{\pgfqpoint{1.710110in}{1.299957in}}%
\pgfpathlineto{\pgfqpoint{1.721016in}{1.299957in}}%
\pgfpathlineto{\pgfqpoint{1.731922in}{1.299957in}}%
\pgfpathlineto{\pgfqpoint{1.742828in}{1.299957in}}%
\pgfpathlineto{\pgfqpoint{1.753734in}{1.299957in}}%
\pgfpathlineto{\pgfqpoint{1.764640in}{1.299957in}}%
\pgfpathlineto{\pgfqpoint{1.775546in}{1.299957in}}%
\pgfpathlineto{\pgfqpoint{1.786452in}{1.299957in}}%
\pgfpathlineto{\pgfqpoint{1.797358in}{1.299957in}}%
\pgfpathlineto{\pgfqpoint{1.808264in}{1.299957in}}%
\pgfpathlineto{\pgfqpoint{1.819170in}{1.280509in}}%
\pgfpathlineto{\pgfqpoint{1.830076in}{1.280509in}}%
\pgfpathlineto{\pgfqpoint{1.840982in}{1.280509in}}%
\pgfpathlineto{\pgfqpoint{1.851888in}{1.280509in}}%
\pgfpathlineto{\pgfqpoint{1.862794in}{1.280509in}}%
\pgfpathlineto{\pgfqpoint{1.873700in}{1.280509in}}%
\pgfpathlineto{\pgfqpoint{1.884606in}{1.280509in}}%
\pgfpathlineto{\pgfqpoint{1.895512in}{1.280509in}}%
\pgfpathlineto{\pgfqpoint{1.906418in}{1.280509in}}%
\pgfpathlineto{\pgfqpoint{1.917324in}{1.276682in}}%
\pgfpathlineto{\pgfqpoint{1.928230in}{1.276682in}}%
\pgfpathlineto{\pgfqpoint{1.939136in}{1.276682in}}%
\pgfpathlineto{\pgfqpoint{1.950042in}{1.276682in}}%
\pgfpathlineto{\pgfqpoint{1.960948in}{1.276682in}}%
\pgfpathlineto{\pgfqpoint{1.971854in}{1.276682in}}%
\pgfpathlineto{\pgfqpoint{1.982759in}{1.276682in}}%
\pgfpathlineto{\pgfqpoint{1.993665in}{1.276682in}}%
\pgfpathlineto{\pgfqpoint{2.004571in}{1.270184in}}%
\pgfpathlineto{\pgfqpoint{2.015477in}{1.270184in}}%
\pgfpathlineto{\pgfqpoint{2.026383in}{1.270184in}}%
\pgfpathlineto{\pgfqpoint{2.037289in}{1.270184in}}%
\pgfpathlineto{\pgfqpoint{2.048195in}{1.270184in}}%
\pgfpathlineto{\pgfqpoint{2.059101in}{1.270184in}}%
\pgfpathlineto{\pgfqpoint{2.070007in}{1.270184in}}%
\pgfpathlineto{\pgfqpoint{2.080913in}{1.270184in}}%
\pgfpathlineto{\pgfqpoint{2.091819in}{1.270184in}}%
\pgfpathlineto{\pgfqpoint{2.102725in}{1.270184in}}%
\pgfpathlineto{\pgfqpoint{2.113631in}{1.270184in}}%
\pgfpathlineto{\pgfqpoint{2.124537in}{1.270184in}}%
\pgfpathlineto{\pgfqpoint{2.135443in}{1.244456in}}%
\pgfpathlineto{\pgfqpoint{2.146349in}{1.244456in}}%
\pgfpathlineto{\pgfqpoint{2.157255in}{1.244456in}}%
\pgfpathlineto{\pgfqpoint{2.168161in}{1.244456in}}%
\pgfpathlineto{\pgfqpoint{2.179067in}{1.244456in}}%
\pgfpathlineto{\pgfqpoint{2.189973in}{1.244456in}}%
\pgfpathlineto{\pgfqpoint{2.200879in}{1.244456in}}%
\pgfpathlineto{\pgfqpoint{2.211785in}{1.244456in}}%
\pgfpathlineto{\pgfqpoint{2.222691in}{1.244456in}}%
\pgfpathlineto{\pgfqpoint{2.233597in}{1.244456in}}%
\pgfpathlineto{\pgfqpoint{2.244503in}{1.244456in}}%
\pgfpathlineto{\pgfqpoint{2.255409in}{1.242649in}}%
\pgfpathlineto{\pgfqpoint{2.266315in}{1.242649in}}%
\pgfpathlineto{\pgfqpoint{2.277221in}{1.242649in}}%
\pgfpathlineto{\pgfqpoint{2.288127in}{1.242649in}}%
\pgfpathlineto{\pgfqpoint{2.299033in}{1.242649in}}%
\pgfpathlineto{\pgfqpoint{2.309939in}{1.242649in}}%
\pgfpathlineto{\pgfqpoint{2.320845in}{1.242649in}}%
\pgfpathlineto{\pgfqpoint{2.331751in}{1.166295in}}%
\pgfpathlineto{\pgfqpoint{2.342657in}{1.166295in}}%
\pgfpathlineto{\pgfqpoint{2.353563in}{1.166295in}}%
\pgfpathlineto{\pgfqpoint{2.364469in}{1.166295in}}%
\pgfpathlineto{\pgfqpoint{2.375375in}{1.166295in}}%
\pgfpathlineto{\pgfqpoint{2.386281in}{1.166295in}}%
\pgfpathlineto{\pgfqpoint{2.397187in}{1.166295in}}%
\pgfpathlineto{\pgfqpoint{2.408093in}{1.166295in}}%
\pgfpathlineto{\pgfqpoint{2.418999in}{1.166295in}}%
\pgfpathlineto{\pgfqpoint{2.429905in}{1.166295in}}%
\pgfpathlineto{\pgfqpoint{2.440811in}{1.166295in}}%
\pgfpathlineto{\pgfqpoint{2.451717in}{1.166295in}}%
\pgfpathlineto{\pgfqpoint{2.462623in}{1.166295in}}%
\pgfpathlineto{\pgfqpoint{2.473528in}{1.166295in}}%
\pgfpathlineto{\pgfqpoint{2.484434in}{1.166295in}}%
\pgfpathlineto{\pgfqpoint{2.495340in}{1.166295in}}%
\pgfpathlineto{\pgfqpoint{2.506246in}{1.166295in}}%
\pgfpathlineto{\pgfqpoint{2.517152in}{1.166295in}}%
\pgfpathlineto{\pgfqpoint{2.528058in}{1.166295in}}%
\pgfpathlineto{\pgfqpoint{2.538964in}{1.166295in}}%
\pgfpathlineto{\pgfqpoint{2.549870in}{1.166295in}}%
\pgfpathlineto{\pgfqpoint{2.560776in}{1.166295in}}%
\pgfpathlineto{\pgfqpoint{2.571682in}{1.166295in}}%
\pgfpathlineto{\pgfqpoint{2.582588in}{1.166295in}}%
\pgfpathlineto{\pgfqpoint{2.593494in}{1.166295in}}%
\pgfpathlineto{\pgfqpoint{2.604400in}{1.166295in}}%
\pgfpathlineto{\pgfqpoint{2.615306in}{1.166295in}}%
\pgfpathlineto{\pgfqpoint{2.626212in}{1.166295in}}%
\pgfpathlineto{\pgfqpoint{2.637118in}{1.166295in}}%
\pgfpathlineto{\pgfqpoint{2.648024in}{1.166295in}}%
\pgfpathlineto{\pgfqpoint{2.658930in}{1.166295in}}%
\pgfpathlineto{\pgfqpoint{2.669836in}{1.166295in}}%
\pgfpathlineto{\pgfqpoint{2.680742in}{1.166295in}}%
\pgfpathlineto{\pgfqpoint{2.691648in}{1.166295in}}%
\pgfpathlineto{\pgfqpoint{2.702554in}{1.166295in}}%
\pgfpathlineto{\pgfqpoint{2.713460in}{1.166295in}}%
\pgfpathlineto{\pgfqpoint{2.724366in}{1.166295in}}%
\pgfpathlineto{\pgfqpoint{2.735272in}{1.166295in}}%
\pgfpathlineto{\pgfqpoint{2.746178in}{1.166295in}}%
\pgfpathlineto{\pgfqpoint{2.757084in}{1.089725in}}%
\pgfpathlineto{\pgfqpoint{2.767990in}{1.089467in}}%
\pgfpathlineto{\pgfqpoint{2.778896in}{1.089467in}}%
\pgfpathlineto{\pgfqpoint{2.789802in}{1.089467in}}%
\pgfpathlineto{\pgfqpoint{2.800708in}{1.089467in}}%
\pgfpathlineto{\pgfqpoint{2.811614in}{1.089150in}}%
\pgfpathlineto{\pgfqpoint{2.822520in}{1.089150in}}%
\pgfpathlineto{\pgfqpoint{2.833426in}{1.089150in}}%
\pgfpathlineto{\pgfqpoint{2.844332in}{1.089150in}}%
\pgfpathlineto{\pgfqpoint{2.855238in}{1.089150in}}%
\pgfpathlineto{\pgfqpoint{2.866144in}{1.089150in}}%
\pgfpathlineto{\pgfqpoint{2.877050in}{1.089150in}}%
\pgfpathlineto{\pgfqpoint{2.887956in}{1.089150in}}%
\pgfpathlineto{\pgfqpoint{2.898862in}{1.089150in}}%
\pgfpathlineto{\pgfqpoint{2.909768in}{1.089150in}}%
\pgfpathlineto{\pgfqpoint{2.920674in}{1.089150in}}%
\pgfpathlineto{\pgfqpoint{2.931580in}{1.089150in}}%
\pgfpathlineto{\pgfqpoint{2.931580in}{0.888370in}}%
\pgfpathlineto{\pgfqpoint{2.931580in}{0.888370in}}%
\pgfpathlineto{\pgfqpoint{2.920674in}{0.888370in}}%
\pgfpathlineto{\pgfqpoint{2.909768in}{0.888370in}}%
\pgfpathlineto{\pgfqpoint{2.898862in}{0.888370in}}%
\pgfpathlineto{\pgfqpoint{2.887956in}{0.888370in}}%
\pgfpathlineto{\pgfqpoint{2.877050in}{0.888370in}}%
\pgfpathlineto{\pgfqpoint{2.866144in}{0.888370in}}%
\pgfpathlineto{\pgfqpoint{2.855238in}{0.888370in}}%
\pgfpathlineto{\pgfqpoint{2.844332in}{0.888370in}}%
\pgfpathlineto{\pgfqpoint{2.833426in}{0.888370in}}%
\pgfpathlineto{\pgfqpoint{2.822520in}{0.888370in}}%
\pgfpathlineto{\pgfqpoint{2.811614in}{0.888370in}}%
\pgfpathlineto{\pgfqpoint{2.800708in}{0.888983in}}%
\pgfpathlineto{\pgfqpoint{2.789802in}{0.888983in}}%
\pgfpathlineto{\pgfqpoint{2.778896in}{0.888983in}}%
\pgfpathlineto{\pgfqpoint{2.767990in}{0.888983in}}%
\pgfpathlineto{\pgfqpoint{2.757084in}{0.889617in}}%
\pgfpathlineto{\pgfqpoint{2.746178in}{0.974432in}}%
\pgfpathlineto{\pgfqpoint{2.735272in}{0.974432in}}%
\pgfpathlineto{\pgfqpoint{2.724366in}{0.974432in}}%
\pgfpathlineto{\pgfqpoint{2.713460in}{0.974432in}}%
\pgfpathlineto{\pgfqpoint{2.702554in}{0.974432in}}%
\pgfpathlineto{\pgfqpoint{2.691648in}{0.974432in}}%
\pgfpathlineto{\pgfqpoint{2.680742in}{0.974432in}}%
\pgfpathlineto{\pgfqpoint{2.669836in}{0.974432in}}%
\pgfpathlineto{\pgfqpoint{2.658930in}{0.974432in}}%
\pgfpathlineto{\pgfqpoint{2.648024in}{0.974432in}}%
\pgfpathlineto{\pgfqpoint{2.637118in}{0.974432in}}%
\pgfpathlineto{\pgfqpoint{2.626212in}{0.974432in}}%
\pgfpathlineto{\pgfqpoint{2.615306in}{0.974432in}}%
\pgfpathlineto{\pgfqpoint{2.604400in}{0.974432in}}%
\pgfpathlineto{\pgfqpoint{2.593494in}{0.974432in}}%
\pgfpathlineto{\pgfqpoint{2.582588in}{0.974432in}}%
\pgfpathlineto{\pgfqpoint{2.571682in}{0.974432in}}%
\pgfpathlineto{\pgfqpoint{2.560776in}{0.974432in}}%
\pgfpathlineto{\pgfqpoint{2.549870in}{0.974432in}}%
\pgfpathlineto{\pgfqpoint{2.538964in}{0.974432in}}%
\pgfpathlineto{\pgfqpoint{2.528058in}{0.974432in}}%
\pgfpathlineto{\pgfqpoint{2.517152in}{0.974432in}}%
\pgfpathlineto{\pgfqpoint{2.506246in}{0.974432in}}%
\pgfpathlineto{\pgfqpoint{2.495340in}{0.974432in}}%
\pgfpathlineto{\pgfqpoint{2.484434in}{0.974432in}}%
\pgfpathlineto{\pgfqpoint{2.473528in}{0.974432in}}%
\pgfpathlineto{\pgfqpoint{2.462623in}{0.974432in}}%
\pgfpathlineto{\pgfqpoint{2.451717in}{0.974432in}}%
\pgfpathlineto{\pgfqpoint{2.440811in}{0.974432in}}%
\pgfpathlineto{\pgfqpoint{2.429905in}{0.974432in}}%
\pgfpathlineto{\pgfqpoint{2.418999in}{0.974432in}}%
\pgfpathlineto{\pgfqpoint{2.408093in}{0.974432in}}%
\pgfpathlineto{\pgfqpoint{2.397187in}{0.974432in}}%
\pgfpathlineto{\pgfqpoint{2.386281in}{0.974432in}}%
\pgfpathlineto{\pgfqpoint{2.375375in}{0.974432in}}%
\pgfpathlineto{\pgfqpoint{2.364469in}{0.974432in}}%
\pgfpathlineto{\pgfqpoint{2.353563in}{0.974432in}}%
\pgfpathlineto{\pgfqpoint{2.342657in}{0.974432in}}%
\pgfpathlineto{\pgfqpoint{2.331751in}{0.974432in}}%
\pgfpathlineto{\pgfqpoint{2.320845in}{1.002473in}}%
\pgfpathlineto{\pgfqpoint{2.309939in}{1.002473in}}%
\pgfpathlineto{\pgfqpoint{2.299033in}{1.002473in}}%
\pgfpathlineto{\pgfqpoint{2.288127in}{1.002473in}}%
\pgfpathlineto{\pgfqpoint{2.277221in}{1.002473in}}%
\pgfpathlineto{\pgfqpoint{2.266315in}{1.002473in}}%
\pgfpathlineto{\pgfqpoint{2.255409in}{1.002473in}}%
\pgfpathlineto{\pgfqpoint{2.244503in}{1.005522in}}%
\pgfpathlineto{\pgfqpoint{2.233597in}{1.005522in}}%
\pgfpathlineto{\pgfqpoint{2.222691in}{1.005522in}}%
\pgfpathlineto{\pgfqpoint{2.211785in}{1.005522in}}%
\pgfpathlineto{\pgfqpoint{2.200879in}{1.005522in}}%
\pgfpathlineto{\pgfqpoint{2.189973in}{1.005522in}}%
\pgfpathlineto{\pgfqpoint{2.179067in}{1.005522in}}%
\pgfpathlineto{\pgfqpoint{2.168161in}{1.005522in}}%
\pgfpathlineto{\pgfqpoint{2.157255in}{1.005522in}}%
\pgfpathlineto{\pgfqpoint{2.146349in}{1.005522in}}%
\pgfpathlineto{\pgfqpoint{2.135443in}{1.005522in}}%
\pgfpathlineto{\pgfqpoint{2.124537in}{1.039288in}}%
\pgfpathlineto{\pgfqpoint{2.113631in}{1.039288in}}%
\pgfpathlineto{\pgfqpoint{2.102725in}{1.039288in}}%
\pgfpathlineto{\pgfqpoint{2.091819in}{1.039288in}}%
\pgfpathlineto{\pgfqpoint{2.080913in}{1.039288in}}%
\pgfpathlineto{\pgfqpoint{2.070007in}{1.039288in}}%
\pgfpathlineto{\pgfqpoint{2.059101in}{1.039288in}}%
\pgfpathlineto{\pgfqpoint{2.048195in}{1.039288in}}%
\pgfpathlineto{\pgfqpoint{2.037289in}{1.039288in}}%
\pgfpathlineto{\pgfqpoint{2.026383in}{1.039288in}}%
\pgfpathlineto{\pgfqpoint{2.015477in}{1.039288in}}%
\pgfpathlineto{\pgfqpoint{2.004571in}{1.039288in}}%
\pgfpathlineto{\pgfqpoint{1.993665in}{1.091623in}}%
\pgfpathlineto{\pgfqpoint{1.982759in}{1.091623in}}%
\pgfpathlineto{\pgfqpoint{1.971854in}{1.091623in}}%
\pgfpathlineto{\pgfqpoint{1.960948in}{1.091623in}}%
\pgfpathlineto{\pgfqpoint{1.950042in}{1.091623in}}%
\pgfpathlineto{\pgfqpoint{1.939136in}{1.091623in}}%
\pgfpathlineto{\pgfqpoint{1.928230in}{1.091623in}}%
\pgfpathlineto{\pgfqpoint{1.917324in}{1.091623in}}%
\pgfpathlineto{\pgfqpoint{1.906418in}{1.096604in}}%
\pgfpathlineto{\pgfqpoint{1.895512in}{1.096604in}}%
\pgfpathlineto{\pgfqpoint{1.884606in}{1.096604in}}%
\pgfpathlineto{\pgfqpoint{1.873700in}{1.096604in}}%
\pgfpathlineto{\pgfqpoint{1.862794in}{1.096604in}}%
\pgfpathlineto{\pgfqpoint{1.851888in}{1.096604in}}%
\pgfpathlineto{\pgfqpoint{1.840982in}{1.096604in}}%
\pgfpathlineto{\pgfqpoint{1.830076in}{1.096604in}}%
\pgfpathlineto{\pgfqpoint{1.819170in}{1.096604in}}%
\pgfpathlineto{\pgfqpoint{1.808264in}{1.116606in}}%
\pgfpathlineto{\pgfqpoint{1.797358in}{1.116606in}}%
\pgfpathlineto{\pgfqpoint{1.786452in}{1.116606in}}%
\pgfpathlineto{\pgfqpoint{1.775546in}{1.116606in}}%
\pgfpathlineto{\pgfqpoint{1.764640in}{1.116606in}}%
\pgfpathlineto{\pgfqpoint{1.753734in}{1.116606in}}%
\pgfpathlineto{\pgfqpoint{1.742828in}{1.116606in}}%
\pgfpathlineto{\pgfqpoint{1.731922in}{1.116606in}}%
\pgfpathlineto{\pgfqpoint{1.721016in}{1.116606in}}%
\pgfpathlineto{\pgfqpoint{1.710110in}{1.116606in}}%
\pgfpathlineto{\pgfqpoint{1.699204in}{1.116606in}}%
\pgfpathlineto{\pgfqpoint{1.688298in}{1.116606in}}%
\pgfpathlineto{\pgfqpoint{1.677392in}{1.229861in}}%
\pgfpathlineto{\pgfqpoint{1.666486in}{1.229861in}}%
\pgfpathlineto{\pgfqpoint{1.655580in}{1.229861in}}%
\pgfpathlineto{\pgfqpoint{1.644674in}{1.229861in}}%
\pgfpathlineto{\pgfqpoint{1.633768in}{1.229861in}}%
\pgfpathlineto{\pgfqpoint{1.622862in}{1.229861in}}%
\pgfpathlineto{\pgfqpoint{1.611956in}{1.229861in}}%
\pgfpathlineto{\pgfqpoint{1.601050in}{1.338905in}}%
\pgfpathlineto{\pgfqpoint{1.590144in}{1.338905in}}%
\pgfpathlineto{\pgfqpoint{1.579238in}{1.338905in}}%
\pgfpathlineto{\pgfqpoint{1.568332in}{1.338905in}}%
\pgfpathlineto{\pgfqpoint{1.557426in}{1.338905in}}%
\pgfpathlineto{\pgfqpoint{1.546520in}{1.341054in}}%
\pgfpathlineto{\pgfqpoint{1.535614in}{1.341054in}}%
\pgfpathlineto{\pgfqpoint{1.524708in}{1.341054in}}%
\pgfpathlineto{\pgfqpoint{1.513802in}{1.341054in}}%
\pgfpathlineto{\pgfqpoint{1.502896in}{1.341054in}}%
\pgfpathlineto{\pgfqpoint{1.491991in}{1.341054in}}%
\pgfpathlineto{\pgfqpoint{1.481085in}{1.373868in}}%
\pgfpathlineto{\pgfqpoint{1.470179in}{1.403593in}}%
\pgfpathlineto{\pgfqpoint{1.459273in}{1.403593in}}%
\pgfpathlineto{\pgfqpoint{1.448367in}{1.403593in}}%
\pgfpathlineto{\pgfqpoint{1.437461in}{1.415606in}}%
\pgfpathlineto{\pgfqpoint{1.426555in}{1.415606in}}%
\pgfpathlineto{\pgfqpoint{1.415649in}{1.415606in}}%
\pgfpathlineto{\pgfqpoint{1.404743in}{1.415606in}}%
\pgfpathlineto{\pgfqpoint{1.393837in}{1.415606in}}%
\pgfpathlineto{\pgfqpoint{1.382931in}{1.415606in}}%
\pgfpathlineto{\pgfqpoint{1.372025in}{1.415606in}}%
\pgfpathlineto{\pgfqpoint{1.361119in}{1.415606in}}%
\pgfpathlineto{\pgfqpoint{1.350213in}{1.416163in}}%
\pgfpathlineto{\pgfqpoint{1.339307in}{1.416163in}}%
\pgfpathlineto{\pgfqpoint{1.328401in}{1.491051in}}%
\pgfpathlineto{\pgfqpoint{1.317495in}{1.491051in}}%
\pgfpathlineto{\pgfqpoint{1.306589in}{1.491051in}}%
\pgfpathlineto{\pgfqpoint{1.295683in}{1.504945in}}%
\pgfpathlineto{\pgfqpoint{1.284777in}{1.504945in}}%
\pgfpathlineto{\pgfqpoint{1.273871in}{1.504945in}}%
\pgfpathlineto{\pgfqpoint{1.262965in}{1.504945in}}%
\pgfpathlineto{\pgfqpoint{1.252059in}{1.504945in}}%
\pgfpathlineto{\pgfqpoint{1.241153in}{1.504945in}}%
\pgfpathlineto{\pgfqpoint{1.230247in}{1.504945in}}%
\pgfpathlineto{\pgfqpoint{1.219341in}{1.504945in}}%
\pgfpathlineto{\pgfqpoint{1.208435in}{1.508226in}}%
\pgfpathlineto{\pgfqpoint{1.197529in}{1.508226in}}%
\pgfpathlineto{\pgfqpoint{1.186623in}{1.508226in}}%
\pgfpathlineto{\pgfqpoint{1.175717in}{1.522525in}}%
\pgfpathlineto{\pgfqpoint{1.164811in}{1.522525in}}%
\pgfpathlineto{\pgfqpoint{1.153905in}{1.522525in}}%
\pgfpathlineto{\pgfqpoint{1.142999in}{1.522525in}}%
\pgfpathlineto{\pgfqpoint{1.132093in}{1.522525in}}%
\pgfpathlineto{\pgfqpoint{1.121187in}{1.522525in}}%
\pgfpathlineto{\pgfqpoint{1.110281in}{1.522525in}}%
\pgfpathlineto{\pgfqpoint{1.099375in}{1.588542in}}%
\pgfpathlineto{\pgfqpoint{1.088469in}{1.588542in}}%
\pgfpathlineto{\pgfqpoint{1.077563in}{1.588542in}}%
\pgfpathlineto{\pgfqpoint{1.066657in}{1.588542in}}%
\pgfpathlineto{\pgfqpoint{1.055751in}{1.588542in}}%
\pgfpathlineto{\pgfqpoint{1.044845in}{1.588542in}}%
\pgfpathlineto{\pgfqpoint{1.033939in}{1.588542in}}%
\pgfpathlineto{\pgfqpoint{1.023033in}{1.588542in}}%
\pgfpathlineto{\pgfqpoint{1.012127in}{1.588542in}}%
\pgfpathlineto{\pgfqpoint{1.001222in}{1.588542in}}%
\pgfpathlineto{\pgfqpoint{0.990316in}{1.588542in}}%
\pgfpathlineto{\pgfqpoint{0.979410in}{1.652292in}}%
\pgfpathlineto{\pgfqpoint{0.968504in}{1.698602in}}%
\pgfpathlineto{\pgfqpoint{0.957598in}{1.698602in}}%
\pgfpathlineto{\pgfqpoint{0.946692in}{1.698602in}}%
\pgfpathlineto{\pgfqpoint{0.935786in}{1.698602in}}%
\pgfpathlineto{\pgfqpoint{0.924880in}{1.698602in}}%
\pgfpathlineto{\pgfqpoint{0.913974in}{1.698602in}}%
\pgfpathlineto{\pgfqpoint{0.903068in}{1.707891in}}%
\pgfpathlineto{\pgfqpoint{0.892162in}{1.707891in}}%
\pgfpathlineto{\pgfqpoint{0.881256in}{1.707891in}}%
\pgfpathlineto{\pgfqpoint{0.870350in}{1.707891in}}%
\pgfpathlineto{\pgfqpoint{0.859444in}{1.707891in}}%
\pgfpathlineto{\pgfqpoint{0.848538in}{1.707891in}}%
\pgfpathlineto{\pgfqpoint{0.837632in}{1.711233in}}%
\pgfpathlineto{\pgfqpoint{0.826726in}{1.711233in}}%
\pgfpathlineto{\pgfqpoint{0.815820in}{1.772759in}}%
\pgfpathlineto{\pgfqpoint{0.804914in}{1.772759in}}%
\pgfpathlineto{\pgfqpoint{0.794008in}{1.772759in}}%
\pgfpathlineto{\pgfqpoint{0.783102in}{1.772759in}}%
\pgfpathlineto{\pgfqpoint{0.772196in}{1.772759in}}%
\pgfpathlineto{\pgfqpoint{0.761290in}{1.793551in}}%
\pgfpathlineto{\pgfqpoint{0.750384in}{1.793551in}}%
\pgfpathlineto{\pgfqpoint{0.739478in}{1.812459in}}%
\pgfpathlineto{\pgfqpoint{0.728572in}{1.812459in}}%
\pgfpathlineto{\pgfqpoint{0.717666in}{1.812459in}}%
\pgfpathlineto{\pgfqpoint{0.706760in}{1.812459in}}%
\pgfpathlineto{\pgfqpoint{0.695854in}{1.812459in}}%
\pgfpathlineto{\pgfqpoint{0.684948in}{1.812459in}}%
\pgfpathlineto{\pgfqpoint{0.674042in}{1.828225in}}%
\pgfpathlineto{\pgfqpoint{0.663136in}{1.828225in}}%
\pgfpathlineto{\pgfqpoint{0.652230in}{1.828225in}}%
\pgfpathlineto{\pgfqpoint{0.641324in}{1.828225in}}%
\pgfpathlineto{\pgfqpoint{0.630418in}{1.859401in}}%
\pgfpathlineto{\pgfqpoint{0.619512in}{1.859401in}}%
\pgfpathlineto{\pgfqpoint{0.608606in}{1.859401in}}%
\pgfpathlineto{\pgfqpoint{0.597700in}{1.913081in}}%
\pgfpathlineto{\pgfqpoint{0.586794in}{1.913081in}}%
\pgfpathlineto{\pgfqpoint{0.575888in}{1.913081in}}%
\pgfpathlineto{\pgfqpoint{0.564982in}{2.002285in}}%
\pgfpathlineto{\pgfqpoint{0.554076in}{2.016702in}}%
\pgfpathlineto{\pgfqpoint{0.543170in}{2.161353in}}%
\pgfpathclose%
\pgfusepath{fill}%
\end{pgfscope}%
\begin{pgfscope}%
\pgfpathrectangle{\pgfqpoint{0.423750in}{0.375000in}}{\pgfqpoint{2.627250in}{2.265000in}}%
\pgfusepath{clip}%
\pgfsetbuttcap%
\pgfsetroundjoin%
\definecolor{currentfill}{rgb}{0.172549,0.627451,0.172549}%
\pgfsetfillcolor{currentfill}%
\pgfsetfillopacity{0.200000}%
\pgfsetlinewidth{0.000000pt}%
\definecolor{currentstroke}{rgb}{0.000000,0.000000,0.000000}%
\pgfsetstrokecolor{currentstroke}%
\pgfsetdash{}{0pt}%
\pgfpathmoveto{\pgfqpoint{0.543170in}{2.325930in}}%
\pgfpathlineto{\pgfqpoint{0.543170in}{2.478414in}}%
\pgfpathlineto{\pgfqpoint{0.554076in}{2.411857in}}%
\pgfpathlineto{\pgfqpoint{0.564982in}{2.324754in}}%
\pgfpathlineto{\pgfqpoint{0.575888in}{2.324754in}}%
\pgfpathlineto{\pgfqpoint{0.586794in}{2.264803in}}%
\pgfpathlineto{\pgfqpoint{0.597700in}{2.264803in}}%
\pgfpathlineto{\pgfqpoint{0.608606in}{2.234427in}}%
\pgfpathlineto{\pgfqpoint{0.619512in}{2.234427in}}%
\pgfpathlineto{\pgfqpoint{0.630418in}{2.192997in}}%
\pgfpathlineto{\pgfqpoint{0.641324in}{2.164706in}}%
\pgfpathlineto{\pgfqpoint{0.652230in}{2.164636in}}%
\pgfpathlineto{\pgfqpoint{0.663136in}{2.164636in}}%
\pgfpathlineto{\pgfqpoint{0.674042in}{2.164636in}}%
\pgfpathlineto{\pgfqpoint{0.684948in}{2.115681in}}%
\pgfpathlineto{\pgfqpoint{0.695854in}{2.115681in}}%
\pgfpathlineto{\pgfqpoint{0.706760in}{2.115681in}}%
\pgfpathlineto{\pgfqpoint{0.717666in}{2.076911in}}%
\pgfpathlineto{\pgfqpoint{0.728572in}{2.076911in}}%
\pgfpathlineto{\pgfqpoint{0.739478in}{2.062667in}}%
\pgfpathlineto{\pgfqpoint{0.750384in}{2.013088in}}%
\pgfpathlineto{\pgfqpoint{0.761290in}{2.013088in}}%
\pgfpathlineto{\pgfqpoint{0.772196in}{2.013088in}}%
\pgfpathlineto{\pgfqpoint{0.783102in}{2.012450in}}%
\pgfpathlineto{\pgfqpoint{0.794008in}{2.012450in}}%
\pgfpathlineto{\pgfqpoint{0.804914in}{2.012450in}}%
\pgfpathlineto{\pgfqpoint{0.815820in}{2.012450in}}%
\pgfpathlineto{\pgfqpoint{0.826726in}{1.929229in}}%
\pgfpathlineto{\pgfqpoint{0.837632in}{1.929229in}}%
\pgfpathlineto{\pgfqpoint{0.848538in}{1.929229in}}%
\pgfpathlineto{\pgfqpoint{0.859444in}{1.929229in}}%
\pgfpathlineto{\pgfqpoint{0.870350in}{1.929229in}}%
\pgfpathlineto{\pgfqpoint{0.881256in}{1.929229in}}%
\pgfpathlineto{\pgfqpoint{0.892162in}{1.929229in}}%
\pgfpathlineto{\pgfqpoint{0.903068in}{1.929229in}}%
\pgfpathlineto{\pgfqpoint{0.913974in}{1.924906in}}%
\pgfpathlineto{\pgfqpoint{0.924880in}{1.913319in}}%
\pgfpathlineto{\pgfqpoint{0.935786in}{1.850232in}}%
\pgfpathlineto{\pgfqpoint{0.946692in}{1.850232in}}%
\pgfpathlineto{\pgfqpoint{0.957598in}{1.678602in}}%
\pgfpathlineto{\pgfqpoint{0.968504in}{1.678602in}}%
\pgfpathlineto{\pgfqpoint{0.979410in}{1.678602in}}%
\pgfpathlineto{\pgfqpoint{0.990316in}{1.678602in}}%
\pgfpathlineto{\pgfqpoint{1.001222in}{1.678602in}}%
\pgfpathlineto{\pgfqpoint{1.012127in}{1.658522in}}%
\pgfpathlineto{\pgfqpoint{1.023033in}{1.658522in}}%
\pgfpathlineto{\pgfqpoint{1.033939in}{1.658522in}}%
\pgfpathlineto{\pgfqpoint{1.044845in}{1.658522in}}%
\pgfpathlineto{\pgfqpoint{1.055751in}{1.658522in}}%
\pgfpathlineto{\pgfqpoint{1.066657in}{1.658522in}}%
\pgfpathlineto{\pgfqpoint{1.077563in}{1.658522in}}%
\pgfpathlineto{\pgfqpoint{1.088469in}{1.658522in}}%
\pgfpathlineto{\pgfqpoint{1.099375in}{1.658522in}}%
\pgfpathlineto{\pgfqpoint{1.110281in}{1.656098in}}%
\pgfpathlineto{\pgfqpoint{1.121187in}{1.656098in}}%
\pgfpathlineto{\pgfqpoint{1.132093in}{1.656098in}}%
\pgfpathlineto{\pgfqpoint{1.142999in}{1.656098in}}%
\pgfpathlineto{\pgfqpoint{1.153905in}{1.656098in}}%
\pgfpathlineto{\pgfqpoint{1.164811in}{1.620284in}}%
\pgfpathlineto{\pgfqpoint{1.175717in}{1.620284in}}%
\pgfpathlineto{\pgfqpoint{1.186623in}{1.620284in}}%
\pgfpathlineto{\pgfqpoint{1.197529in}{1.620284in}}%
\pgfpathlineto{\pgfqpoint{1.208435in}{1.562120in}}%
\pgfpathlineto{\pgfqpoint{1.219341in}{1.562120in}}%
\pgfpathlineto{\pgfqpoint{1.230247in}{1.562120in}}%
\pgfpathlineto{\pgfqpoint{1.241153in}{1.562120in}}%
\pgfpathlineto{\pgfqpoint{1.252059in}{1.562120in}}%
\pgfpathlineto{\pgfqpoint{1.262965in}{1.562120in}}%
\pgfpathlineto{\pgfqpoint{1.273871in}{1.509856in}}%
\pgfpathlineto{\pgfqpoint{1.284777in}{1.509856in}}%
\pgfpathlineto{\pgfqpoint{1.295683in}{1.509856in}}%
\pgfpathlineto{\pgfqpoint{1.306589in}{1.500983in}}%
\pgfpathlineto{\pgfqpoint{1.317495in}{1.482628in}}%
\pgfpathlineto{\pgfqpoint{1.328401in}{1.457957in}}%
\pgfpathlineto{\pgfqpoint{1.339307in}{1.457957in}}%
\pgfpathlineto{\pgfqpoint{1.350213in}{1.457957in}}%
\pgfpathlineto{\pgfqpoint{1.361119in}{1.457957in}}%
\pgfpathlineto{\pgfqpoint{1.372025in}{1.457957in}}%
\pgfpathlineto{\pgfqpoint{1.382931in}{1.457184in}}%
\pgfpathlineto{\pgfqpoint{1.393837in}{1.457184in}}%
\pgfpathlineto{\pgfqpoint{1.404743in}{1.457184in}}%
\pgfpathlineto{\pgfqpoint{1.415649in}{1.457184in}}%
\pgfpathlineto{\pgfqpoint{1.426555in}{1.457184in}}%
\pgfpathlineto{\pgfqpoint{1.437461in}{1.457184in}}%
\pgfpathlineto{\pgfqpoint{1.448367in}{1.457184in}}%
\pgfpathlineto{\pgfqpoint{1.459273in}{1.457184in}}%
\pgfpathlineto{\pgfqpoint{1.470179in}{1.457184in}}%
\pgfpathlineto{\pgfqpoint{1.481085in}{1.457184in}}%
\pgfpathlineto{\pgfqpoint{1.491991in}{1.457184in}}%
\pgfpathlineto{\pgfqpoint{1.502896in}{1.435106in}}%
\pgfpathlineto{\pgfqpoint{1.513802in}{1.435106in}}%
\pgfpathlineto{\pgfqpoint{1.524708in}{1.435106in}}%
\pgfpathlineto{\pgfqpoint{1.535614in}{1.435106in}}%
\pgfpathlineto{\pgfqpoint{1.546520in}{1.435106in}}%
\pgfpathlineto{\pgfqpoint{1.557426in}{1.435106in}}%
\pgfpathlineto{\pgfqpoint{1.568332in}{1.435106in}}%
\pgfpathlineto{\pgfqpoint{1.579238in}{1.435106in}}%
\pgfpathlineto{\pgfqpoint{1.590144in}{1.435106in}}%
\pgfpathlineto{\pgfqpoint{1.601050in}{1.435106in}}%
\pgfpathlineto{\pgfqpoint{1.611956in}{1.435106in}}%
\pgfpathlineto{\pgfqpoint{1.622862in}{1.380897in}}%
\pgfpathlineto{\pgfqpoint{1.633768in}{1.380897in}}%
\pgfpathlineto{\pgfqpoint{1.644674in}{1.380897in}}%
\pgfpathlineto{\pgfqpoint{1.655580in}{1.380897in}}%
\pgfpathlineto{\pgfqpoint{1.666486in}{1.380897in}}%
\pgfpathlineto{\pgfqpoint{1.677392in}{1.380897in}}%
\pgfpathlineto{\pgfqpoint{1.688298in}{1.380897in}}%
\pgfpathlineto{\pgfqpoint{1.699204in}{1.380897in}}%
\pgfpathlineto{\pgfqpoint{1.710110in}{1.374200in}}%
\pgfpathlineto{\pgfqpoint{1.721016in}{1.374200in}}%
\pgfpathlineto{\pgfqpoint{1.731922in}{1.374200in}}%
\pgfpathlineto{\pgfqpoint{1.742828in}{1.374200in}}%
\pgfpathlineto{\pgfqpoint{1.753734in}{1.374200in}}%
\pgfpathlineto{\pgfqpoint{1.764640in}{1.374200in}}%
\pgfpathlineto{\pgfqpoint{1.775546in}{1.374200in}}%
\pgfpathlineto{\pgfqpoint{1.786452in}{1.374200in}}%
\pgfpathlineto{\pgfqpoint{1.797358in}{1.374200in}}%
\pgfpathlineto{\pgfqpoint{1.808264in}{1.374200in}}%
\pgfpathlineto{\pgfqpoint{1.819170in}{1.374200in}}%
\pgfpathlineto{\pgfqpoint{1.830076in}{1.374200in}}%
\pgfpathlineto{\pgfqpoint{1.840982in}{1.257592in}}%
\pgfpathlineto{\pgfqpoint{1.851888in}{1.257592in}}%
\pgfpathlineto{\pgfqpoint{1.862794in}{1.257592in}}%
\pgfpathlineto{\pgfqpoint{1.873700in}{1.257592in}}%
\pgfpathlineto{\pgfqpoint{1.884606in}{1.257592in}}%
\pgfpathlineto{\pgfqpoint{1.895512in}{1.257592in}}%
\pgfpathlineto{\pgfqpoint{1.906418in}{1.257592in}}%
\pgfpathlineto{\pgfqpoint{1.917324in}{1.257592in}}%
\pgfpathlineto{\pgfqpoint{1.928230in}{1.257592in}}%
\pgfpathlineto{\pgfqpoint{1.939136in}{1.257592in}}%
\pgfpathlineto{\pgfqpoint{1.950042in}{1.257592in}}%
\pgfpathlineto{\pgfqpoint{1.960948in}{1.257592in}}%
\pgfpathlineto{\pgfqpoint{1.971854in}{1.235575in}}%
\pgfpathlineto{\pgfqpoint{1.982759in}{1.122337in}}%
\pgfpathlineto{\pgfqpoint{1.993665in}{1.122337in}}%
\pgfpathlineto{\pgfqpoint{2.004571in}{1.122337in}}%
\pgfpathlineto{\pgfqpoint{2.015477in}{1.104010in}}%
\pgfpathlineto{\pgfqpoint{2.026383in}{1.104010in}}%
\pgfpathlineto{\pgfqpoint{2.037289in}{1.104010in}}%
\pgfpathlineto{\pgfqpoint{2.048195in}{1.104010in}}%
\pgfpathlineto{\pgfqpoint{2.059101in}{1.104010in}}%
\pgfpathlineto{\pgfqpoint{2.070007in}{1.104010in}}%
\pgfpathlineto{\pgfqpoint{2.080913in}{1.104010in}}%
\pgfpathlineto{\pgfqpoint{2.091819in}{1.104010in}}%
\pgfpathlineto{\pgfqpoint{2.102725in}{1.104010in}}%
\pgfpathlineto{\pgfqpoint{2.113631in}{1.104010in}}%
\pgfpathlineto{\pgfqpoint{2.124537in}{1.104010in}}%
\pgfpathlineto{\pgfqpoint{2.135443in}{1.094690in}}%
\pgfpathlineto{\pgfqpoint{2.146349in}{1.094690in}}%
\pgfpathlineto{\pgfqpoint{2.157255in}{1.094690in}}%
\pgfpathlineto{\pgfqpoint{2.168161in}{1.094690in}}%
\pgfpathlineto{\pgfqpoint{2.179067in}{1.094690in}}%
\pgfpathlineto{\pgfqpoint{2.189973in}{1.094690in}}%
\pgfpathlineto{\pgfqpoint{2.200879in}{1.094690in}}%
\pgfpathlineto{\pgfqpoint{2.211785in}{1.094690in}}%
\pgfpathlineto{\pgfqpoint{2.222691in}{1.094690in}}%
\pgfpathlineto{\pgfqpoint{2.233597in}{1.094690in}}%
\pgfpathlineto{\pgfqpoint{2.244503in}{1.094690in}}%
\pgfpathlineto{\pgfqpoint{2.255409in}{1.094690in}}%
\pgfpathlineto{\pgfqpoint{2.266315in}{1.094690in}}%
\pgfpathlineto{\pgfqpoint{2.277221in}{1.081783in}}%
\pgfpathlineto{\pgfqpoint{2.288127in}{1.056544in}}%
\pgfpathlineto{\pgfqpoint{2.299033in}{1.049411in}}%
\pgfpathlineto{\pgfqpoint{2.309939in}{1.049411in}}%
\pgfpathlineto{\pgfqpoint{2.320845in}{1.049411in}}%
\pgfpathlineto{\pgfqpoint{2.331751in}{0.978388in}}%
\pgfpathlineto{\pgfqpoint{2.342657in}{0.899955in}}%
\pgfpathlineto{\pgfqpoint{2.353563in}{0.871966in}}%
\pgfpathlineto{\pgfqpoint{2.364469in}{0.871966in}}%
\pgfpathlineto{\pgfqpoint{2.375375in}{0.871966in}}%
\pgfpathlineto{\pgfqpoint{2.386281in}{0.871966in}}%
\pgfpathlineto{\pgfqpoint{2.397187in}{0.871966in}}%
\pgfpathlineto{\pgfqpoint{2.408093in}{0.871966in}}%
\pgfpathlineto{\pgfqpoint{2.418999in}{0.871966in}}%
\pgfpathlineto{\pgfqpoint{2.429905in}{0.871966in}}%
\pgfpathlineto{\pgfqpoint{2.440811in}{0.871966in}}%
\pgfpathlineto{\pgfqpoint{2.451717in}{0.871966in}}%
\pgfpathlineto{\pgfqpoint{2.462623in}{0.871966in}}%
\pgfpathlineto{\pgfqpoint{2.473528in}{0.871966in}}%
\pgfpathlineto{\pgfqpoint{2.484434in}{0.871966in}}%
\pgfpathlineto{\pgfqpoint{2.495340in}{0.871966in}}%
\pgfpathlineto{\pgfqpoint{2.506246in}{0.871966in}}%
\pgfpathlineto{\pgfqpoint{2.517152in}{0.871966in}}%
\pgfpathlineto{\pgfqpoint{2.528058in}{0.854815in}}%
\pgfpathlineto{\pgfqpoint{2.538964in}{0.854815in}}%
\pgfpathlineto{\pgfqpoint{2.549870in}{0.854815in}}%
\pgfpathlineto{\pgfqpoint{2.560776in}{0.854815in}}%
\pgfpathlineto{\pgfqpoint{2.571682in}{0.854815in}}%
\pgfpathlineto{\pgfqpoint{2.582588in}{0.854815in}}%
\pgfpathlineto{\pgfqpoint{2.593494in}{0.854815in}}%
\pgfpathlineto{\pgfqpoint{2.604400in}{0.849898in}}%
\pgfpathlineto{\pgfqpoint{2.615306in}{0.849898in}}%
\pgfpathlineto{\pgfqpoint{2.626212in}{0.849898in}}%
\pgfpathlineto{\pgfqpoint{2.637118in}{0.849898in}}%
\pgfpathlineto{\pgfqpoint{2.648024in}{0.849898in}}%
\pgfpathlineto{\pgfqpoint{2.658930in}{0.849898in}}%
\pgfpathlineto{\pgfqpoint{2.669836in}{0.849898in}}%
\pgfpathlineto{\pgfqpoint{2.680742in}{0.849898in}}%
\pgfpathlineto{\pgfqpoint{2.691648in}{0.849898in}}%
\pgfpathlineto{\pgfqpoint{2.702554in}{0.849898in}}%
\pgfpathlineto{\pgfqpoint{2.713460in}{0.849898in}}%
\pgfpathlineto{\pgfqpoint{2.724366in}{0.849898in}}%
\pgfpathlineto{\pgfqpoint{2.735272in}{0.849898in}}%
\pgfpathlineto{\pgfqpoint{2.746178in}{0.849898in}}%
\pgfpathlineto{\pgfqpoint{2.757084in}{0.849898in}}%
\pgfpathlineto{\pgfqpoint{2.767990in}{0.849898in}}%
\pgfpathlineto{\pgfqpoint{2.778896in}{0.849898in}}%
\pgfpathlineto{\pgfqpoint{2.789802in}{0.849898in}}%
\pgfpathlineto{\pgfqpoint{2.800708in}{0.849898in}}%
\pgfpathlineto{\pgfqpoint{2.811614in}{0.849898in}}%
\pgfpathlineto{\pgfqpoint{2.822520in}{0.849898in}}%
\pgfpathlineto{\pgfqpoint{2.833426in}{0.849898in}}%
\pgfpathlineto{\pgfqpoint{2.844332in}{0.849898in}}%
\pgfpathlineto{\pgfqpoint{2.855238in}{0.849898in}}%
\pgfpathlineto{\pgfqpoint{2.866144in}{0.849898in}}%
\pgfpathlineto{\pgfqpoint{2.877050in}{0.849898in}}%
\pgfpathlineto{\pgfqpoint{2.887956in}{0.849898in}}%
\pgfpathlineto{\pgfqpoint{2.898862in}{0.849898in}}%
\pgfpathlineto{\pgfqpoint{2.909768in}{0.849898in}}%
\pgfpathlineto{\pgfqpoint{2.920674in}{0.849898in}}%
\pgfpathlineto{\pgfqpoint{2.931580in}{0.849898in}}%
\pgfpathlineto{\pgfqpoint{2.931580in}{0.477955in}}%
\pgfpathlineto{\pgfqpoint{2.931580in}{0.477955in}}%
\pgfpathlineto{\pgfqpoint{2.920674in}{0.477955in}}%
\pgfpathlineto{\pgfqpoint{2.909768in}{0.477955in}}%
\pgfpathlineto{\pgfqpoint{2.898862in}{0.477955in}}%
\pgfpathlineto{\pgfqpoint{2.887956in}{0.477955in}}%
\pgfpathlineto{\pgfqpoint{2.877050in}{0.477955in}}%
\pgfpathlineto{\pgfqpoint{2.866144in}{0.477955in}}%
\pgfpathlineto{\pgfqpoint{2.855238in}{0.477955in}}%
\pgfpathlineto{\pgfqpoint{2.844332in}{0.477955in}}%
\pgfpathlineto{\pgfqpoint{2.833426in}{0.477955in}}%
\pgfpathlineto{\pgfqpoint{2.822520in}{0.477955in}}%
\pgfpathlineto{\pgfqpoint{2.811614in}{0.477955in}}%
\pgfpathlineto{\pgfqpoint{2.800708in}{0.477955in}}%
\pgfpathlineto{\pgfqpoint{2.789802in}{0.477955in}}%
\pgfpathlineto{\pgfqpoint{2.778896in}{0.477955in}}%
\pgfpathlineto{\pgfqpoint{2.767990in}{0.477955in}}%
\pgfpathlineto{\pgfqpoint{2.757084in}{0.477955in}}%
\pgfpathlineto{\pgfqpoint{2.746178in}{0.477955in}}%
\pgfpathlineto{\pgfqpoint{2.735272in}{0.477955in}}%
\pgfpathlineto{\pgfqpoint{2.724366in}{0.477955in}}%
\pgfpathlineto{\pgfqpoint{2.713460in}{0.477955in}}%
\pgfpathlineto{\pgfqpoint{2.702554in}{0.477955in}}%
\pgfpathlineto{\pgfqpoint{2.691648in}{0.477955in}}%
\pgfpathlineto{\pgfqpoint{2.680742in}{0.477955in}}%
\pgfpathlineto{\pgfqpoint{2.669836in}{0.477955in}}%
\pgfpathlineto{\pgfqpoint{2.658930in}{0.477955in}}%
\pgfpathlineto{\pgfqpoint{2.648024in}{0.477955in}}%
\pgfpathlineto{\pgfqpoint{2.637118in}{0.477955in}}%
\pgfpathlineto{\pgfqpoint{2.626212in}{0.477955in}}%
\pgfpathlineto{\pgfqpoint{2.615306in}{0.477955in}}%
\pgfpathlineto{\pgfqpoint{2.604400in}{0.477955in}}%
\pgfpathlineto{\pgfqpoint{2.593494in}{0.495247in}}%
\pgfpathlineto{\pgfqpoint{2.582588in}{0.495247in}}%
\pgfpathlineto{\pgfqpoint{2.571682in}{0.495247in}}%
\pgfpathlineto{\pgfqpoint{2.560776in}{0.495247in}}%
\pgfpathlineto{\pgfqpoint{2.549870in}{0.495247in}}%
\pgfpathlineto{\pgfqpoint{2.538964in}{0.495247in}}%
\pgfpathlineto{\pgfqpoint{2.528058in}{0.495247in}}%
\pgfpathlineto{\pgfqpoint{2.517152in}{0.497520in}}%
\pgfpathlineto{\pgfqpoint{2.506246in}{0.497520in}}%
\pgfpathlineto{\pgfqpoint{2.495340in}{0.497520in}}%
\pgfpathlineto{\pgfqpoint{2.484434in}{0.497520in}}%
\pgfpathlineto{\pgfqpoint{2.473528in}{0.497520in}}%
\pgfpathlineto{\pgfqpoint{2.462623in}{0.497520in}}%
\pgfpathlineto{\pgfqpoint{2.451717in}{0.497520in}}%
\pgfpathlineto{\pgfqpoint{2.440811in}{0.497520in}}%
\pgfpathlineto{\pgfqpoint{2.429905in}{0.497520in}}%
\pgfpathlineto{\pgfqpoint{2.418999in}{0.497520in}}%
\pgfpathlineto{\pgfqpoint{2.408093in}{0.497520in}}%
\pgfpathlineto{\pgfqpoint{2.397187in}{0.497520in}}%
\pgfpathlineto{\pgfqpoint{2.386281in}{0.497520in}}%
\pgfpathlineto{\pgfqpoint{2.375375in}{0.497520in}}%
\pgfpathlineto{\pgfqpoint{2.364469in}{0.497520in}}%
\pgfpathlineto{\pgfqpoint{2.353563in}{0.497520in}}%
\pgfpathlineto{\pgfqpoint{2.342657in}{0.501334in}}%
\pgfpathlineto{\pgfqpoint{2.331751in}{0.650366in}}%
\pgfpathlineto{\pgfqpoint{2.320845in}{0.870565in}}%
\pgfpathlineto{\pgfqpoint{2.309939in}{0.870565in}}%
\pgfpathlineto{\pgfqpoint{2.299033in}{0.870565in}}%
\pgfpathlineto{\pgfqpoint{2.288127in}{0.872399in}}%
\pgfpathlineto{\pgfqpoint{2.277221in}{0.878147in}}%
\pgfpathlineto{\pgfqpoint{2.266315in}{0.892881in}}%
\pgfpathlineto{\pgfqpoint{2.255409in}{0.892881in}}%
\pgfpathlineto{\pgfqpoint{2.244503in}{0.892881in}}%
\pgfpathlineto{\pgfqpoint{2.233597in}{0.892881in}}%
\pgfpathlineto{\pgfqpoint{2.222691in}{0.892881in}}%
\pgfpathlineto{\pgfqpoint{2.211785in}{0.892881in}}%
\pgfpathlineto{\pgfqpoint{2.200879in}{0.892881in}}%
\pgfpathlineto{\pgfqpoint{2.189973in}{0.892881in}}%
\pgfpathlineto{\pgfqpoint{2.179067in}{0.892881in}}%
\pgfpathlineto{\pgfqpoint{2.168161in}{0.892881in}}%
\pgfpathlineto{\pgfqpoint{2.157255in}{0.892881in}}%
\pgfpathlineto{\pgfqpoint{2.146349in}{0.892881in}}%
\pgfpathlineto{\pgfqpoint{2.135443in}{0.892881in}}%
\pgfpathlineto{\pgfqpoint{2.124537in}{0.901597in}}%
\pgfpathlineto{\pgfqpoint{2.113631in}{0.901597in}}%
\pgfpathlineto{\pgfqpoint{2.102725in}{0.901597in}}%
\pgfpathlineto{\pgfqpoint{2.091819in}{0.901597in}}%
\pgfpathlineto{\pgfqpoint{2.080913in}{0.901597in}}%
\pgfpathlineto{\pgfqpoint{2.070007in}{0.901597in}}%
\pgfpathlineto{\pgfqpoint{2.059101in}{0.901597in}}%
\pgfpathlineto{\pgfqpoint{2.048195in}{0.901597in}}%
\pgfpathlineto{\pgfqpoint{2.037289in}{0.901597in}}%
\pgfpathlineto{\pgfqpoint{2.026383in}{0.901597in}}%
\pgfpathlineto{\pgfqpoint{2.015477in}{0.901597in}}%
\pgfpathlineto{\pgfqpoint{2.004571in}{0.906629in}}%
\pgfpathlineto{\pgfqpoint{1.993665in}{0.906629in}}%
\pgfpathlineto{\pgfqpoint{1.982759in}{0.906629in}}%
\pgfpathlineto{\pgfqpoint{1.971854in}{0.929232in}}%
\pgfpathlineto{\pgfqpoint{1.960948in}{0.932874in}}%
\pgfpathlineto{\pgfqpoint{1.950042in}{0.932874in}}%
\pgfpathlineto{\pgfqpoint{1.939136in}{0.932874in}}%
\pgfpathlineto{\pgfqpoint{1.928230in}{0.932874in}}%
\pgfpathlineto{\pgfqpoint{1.917324in}{0.932874in}}%
\pgfpathlineto{\pgfqpoint{1.906418in}{0.932874in}}%
\pgfpathlineto{\pgfqpoint{1.895512in}{0.932874in}}%
\pgfpathlineto{\pgfqpoint{1.884606in}{0.932874in}}%
\pgfpathlineto{\pgfqpoint{1.873700in}{0.932874in}}%
\pgfpathlineto{\pgfqpoint{1.862794in}{0.932874in}}%
\pgfpathlineto{\pgfqpoint{1.851888in}{0.932874in}}%
\pgfpathlineto{\pgfqpoint{1.840982in}{0.932874in}}%
\pgfpathlineto{\pgfqpoint{1.830076in}{1.094973in}}%
\pgfpathlineto{\pgfqpoint{1.819170in}{1.094973in}}%
\pgfpathlineto{\pgfqpoint{1.808264in}{1.094973in}}%
\pgfpathlineto{\pgfqpoint{1.797358in}{1.094973in}}%
\pgfpathlineto{\pgfqpoint{1.786452in}{1.094973in}}%
\pgfpathlineto{\pgfqpoint{1.775546in}{1.094973in}}%
\pgfpathlineto{\pgfqpoint{1.764640in}{1.094973in}}%
\pgfpathlineto{\pgfqpoint{1.753734in}{1.094973in}}%
\pgfpathlineto{\pgfqpoint{1.742828in}{1.094973in}}%
\pgfpathlineto{\pgfqpoint{1.731922in}{1.094973in}}%
\pgfpathlineto{\pgfqpoint{1.721016in}{1.094973in}}%
\pgfpathlineto{\pgfqpoint{1.710110in}{1.094973in}}%
\pgfpathlineto{\pgfqpoint{1.699204in}{1.098788in}}%
\pgfpathlineto{\pgfqpoint{1.688298in}{1.098788in}}%
\pgfpathlineto{\pgfqpoint{1.677392in}{1.098788in}}%
\pgfpathlineto{\pgfqpoint{1.666486in}{1.098788in}}%
\pgfpathlineto{\pgfqpoint{1.655580in}{1.098788in}}%
\pgfpathlineto{\pgfqpoint{1.644674in}{1.098788in}}%
\pgfpathlineto{\pgfqpoint{1.633768in}{1.098788in}}%
\pgfpathlineto{\pgfqpoint{1.622862in}{1.098788in}}%
\pgfpathlineto{\pgfqpoint{1.611956in}{1.259040in}}%
\pgfpathlineto{\pgfqpoint{1.601050in}{1.259040in}}%
\pgfpathlineto{\pgfqpoint{1.590144in}{1.259040in}}%
\pgfpathlineto{\pgfqpoint{1.579238in}{1.259040in}}%
\pgfpathlineto{\pgfqpoint{1.568332in}{1.259040in}}%
\pgfpathlineto{\pgfqpoint{1.557426in}{1.259040in}}%
\pgfpathlineto{\pgfqpoint{1.546520in}{1.259040in}}%
\pgfpathlineto{\pgfqpoint{1.535614in}{1.259040in}}%
\pgfpathlineto{\pgfqpoint{1.524708in}{1.259040in}}%
\pgfpathlineto{\pgfqpoint{1.513802in}{1.259040in}}%
\pgfpathlineto{\pgfqpoint{1.502896in}{1.259040in}}%
\pgfpathlineto{\pgfqpoint{1.491991in}{1.269075in}}%
\pgfpathlineto{\pgfqpoint{1.481085in}{1.269075in}}%
\pgfpathlineto{\pgfqpoint{1.470179in}{1.269075in}}%
\pgfpathlineto{\pgfqpoint{1.459273in}{1.269075in}}%
\pgfpathlineto{\pgfqpoint{1.448367in}{1.269075in}}%
\pgfpathlineto{\pgfqpoint{1.437461in}{1.269075in}}%
\pgfpathlineto{\pgfqpoint{1.426555in}{1.269075in}}%
\pgfpathlineto{\pgfqpoint{1.415649in}{1.269075in}}%
\pgfpathlineto{\pgfqpoint{1.404743in}{1.269075in}}%
\pgfpathlineto{\pgfqpoint{1.393837in}{1.269075in}}%
\pgfpathlineto{\pgfqpoint{1.382931in}{1.269075in}}%
\pgfpathlineto{\pgfqpoint{1.372025in}{1.274643in}}%
\pgfpathlineto{\pgfqpoint{1.361119in}{1.274643in}}%
\pgfpathlineto{\pgfqpoint{1.350213in}{1.274643in}}%
\pgfpathlineto{\pgfqpoint{1.339307in}{1.274643in}}%
\pgfpathlineto{\pgfqpoint{1.328401in}{1.274643in}}%
\pgfpathlineto{\pgfqpoint{1.317495in}{1.283122in}}%
\pgfpathlineto{\pgfqpoint{1.306589in}{1.288342in}}%
\pgfpathlineto{\pgfqpoint{1.295683in}{1.330525in}}%
\pgfpathlineto{\pgfqpoint{1.284777in}{1.330525in}}%
\pgfpathlineto{\pgfqpoint{1.273871in}{1.330525in}}%
\pgfpathlineto{\pgfqpoint{1.262965in}{1.340104in}}%
\pgfpathlineto{\pgfqpoint{1.252059in}{1.340104in}}%
\pgfpathlineto{\pgfqpoint{1.241153in}{1.340104in}}%
\pgfpathlineto{\pgfqpoint{1.230247in}{1.340104in}}%
\pgfpathlineto{\pgfqpoint{1.219341in}{1.340104in}}%
\pgfpathlineto{\pgfqpoint{1.208435in}{1.340104in}}%
\pgfpathlineto{\pgfqpoint{1.197529in}{1.452740in}}%
\pgfpathlineto{\pgfqpoint{1.186623in}{1.452740in}}%
\pgfpathlineto{\pgfqpoint{1.175717in}{1.452740in}}%
\pgfpathlineto{\pgfqpoint{1.164811in}{1.452740in}}%
\pgfpathlineto{\pgfqpoint{1.153905in}{1.478815in}}%
\pgfpathlineto{\pgfqpoint{1.142999in}{1.478815in}}%
\pgfpathlineto{\pgfqpoint{1.132093in}{1.478815in}}%
\pgfpathlineto{\pgfqpoint{1.121187in}{1.478815in}}%
\pgfpathlineto{\pgfqpoint{1.110281in}{1.478815in}}%
\pgfpathlineto{\pgfqpoint{1.099375in}{1.484022in}}%
\pgfpathlineto{\pgfqpoint{1.088469in}{1.484022in}}%
\pgfpathlineto{\pgfqpoint{1.077563in}{1.484022in}}%
\pgfpathlineto{\pgfqpoint{1.066657in}{1.484022in}}%
\pgfpathlineto{\pgfqpoint{1.055751in}{1.484022in}}%
\pgfpathlineto{\pgfqpoint{1.044845in}{1.484022in}}%
\pgfpathlineto{\pgfqpoint{1.033939in}{1.484022in}}%
\pgfpathlineto{\pgfqpoint{1.023033in}{1.484022in}}%
\pgfpathlineto{\pgfqpoint{1.012127in}{1.484022in}}%
\pgfpathlineto{\pgfqpoint{1.001222in}{1.575769in}}%
\pgfpathlineto{\pgfqpoint{0.990316in}{1.575769in}}%
\pgfpathlineto{\pgfqpoint{0.979410in}{1.575769in}}%
\pgfpathlineto{\pgfqpoint{0.968504in}{1.575769in}}%
\pgfpathlineto{\pgfqpoint{0.957598in}{1.575769in}}%
\pgfpathlineto{\pgfqpoint{0.946692in}{1.657260in}}%
\pgfpathlineto{\pgfqpoint{0.935786in}{1.657260in}}%
\pgfpathlineto{\pgfqpoint{0.924880in}{1.759559in}}%
\pgfpathlineto{\pgfqpoint{0.913974in}{1.762479in}}%
\pgfpathlineto{\pgfqpoint{0.903068in}{1.765524in}}%
\pgfpathlineto{\pgfqpoint{0.892162in}{1.765524in}}%
\pgfpathlineto{\pgfqpoint{0.881256in}{1.765524in}}%
\pgfpathlineto{\pgfqpoint{0.870350in}{1.765524in}}%
\pgfpathlineto{\pgfqpoint{0.859444in}{1.765524in}}%
\pgfpathlineto{\pgfqpoint{0.848538in}{1.765524in}}%
\pgfpathlineto{\pgfqpoint{0.837632in}{1.765524in}}%
\pgfpathlineto{\pgfqpoint{0.826726in}{1.765524in}}%
\pgfpathlineto{\pgfqpoint{0.815820in}{1.848319in}}%
\pgfpathlineto{\pgfqpoint{0.804914in}{1.848319in}}%
\pgfpathlineto{\pgfqpoint{0.794008in}{1.848319in}}%
\pgfpathlineto{\pgfqpoint{0.783102in}{1.848319in}}%
\pgfpathlineto{\pgfqpoint{0.772196in}{1.849078in}}%
\pgfpathlineto{\pgfqpoint{0.761290in}{1.849078in}}%
\pgfpathlineto{\pgfqpoint{0.750384in}{1.849078in}}%
\pgfpathlineto{\pgfqpoint{0.739478in}{1.888929in}}%
\pgfpathlineto{\pgfqpoint{0.728572in}{1.897530in}}%
\pgfpathlineto{\pgfqpoint{0.717666in}{1.897530in}}%
\pgfpathlineto{\pgfqpoint{0.706760in}{1.923633in}}%
\pgfpathlineto{\pgfqpoint{0.695854in}{1.923633in}}%
\pgfpathlineto{\pgfqpoint{0.684948in}{1.923633in}}%
\pgfpathlineto{\pgfqpoint{0.674042in}{2.018392in}}%
\pgfpathlineto{\pgfqpoint{0.663136in}{2.018392in}}%
\pgfpathlineto{\pgfqpoint{0.652230in}{2.018392in}}%
\pgfpathlineto{\pgfqpoint{0.641324in}{2.027139in}}%
\pgfpathlineto{\pgfqpoint{0.630418in}{2.050383in}}%
\pgfpathlineto{\pgfqpoint{0.619512in}{2.076723in}}%
\pgfpathlineto{\pgfqpoint{0.608606in}{2.076723in}}%
\pgfpathlineto{\pgfqpoint{0.597700in}{2.116074in}}%
\pgfpathlineto{\pgfqpoint{0.586794in}{2.116074in}}%
\pgfpathlineto{\pgfqpoint{0.575888in}{2.190489in}}%
\pgfpathlineto{\pgfqpoint{0.564982in}{2.190489in}}%
\pgfpathlineto{\pgfqpoint{0.554076in}{2.204512in}}%
\pgfpathlineto{\pgfqpoint{0.543170in}{2.325930in}}%
\pgfpathclose%
\pgfusepath{fill}%
\end{pgfscope}%
\begin{pgfscope}%
\pgfpathrectangle{\pgfqpoint{0.423750in}{0.375000in}}{\pgfqpoint{2.627250in}{2.265000in}}%
\pgfusepath{clip}%
\pgfsetbuttcap%
\pgfsetroundjoin%
\definecolor{currentfill}{rgb}{0.839216,0.152941,0.156863}%
\pgfsetfillcolor{currentfill}%
\pgfsetfillopacity{0.200000}%
\pgfsetlinewidth{0.000000pt}%
\definecolor{currentstroke}{rgb}{0.000000,0.000000,0.000000}%
\pgfsetstrokecolor{currentstroke}%
\pgfsetdash{}{0pt}%
\pgfpathmoveto{\pgfqpoint{0.543170in}{2.390136in}}%
\pgfpathlineto{\pgfqpoint{0.543170in}{2.537045in}}%
\pgfpathlineto{\pgfqpoint{0.554076in}{2.432178in}}%
\pgfpathlineto{\pgfqpoint{0.564982in}{2.375573in}}%
\pgfpathlineto{\pgfqpoint{0.575888in}{2.243183in}}%
\pgfpathlineto{\pgfqpoint{0.586794in}{2.243183in}}%
\pgfpathlineto{\pgfqpoint{0.597700in}{2.243183in}}%
\pgfpathlineto{\pgfqpoint{0.608606in}{2.243183in}}%
\pgfpathlineto{\pgfqpoint{0.619512in}{2.208063in}}%
\pgfpathlineto{\pgfqpoint{0.630418in}{2.130752in}}%
\pgfpathlineto{\pgfqpoint{0.641324in}{2.030897in}}%
\pgfpathlineto{\pgfqpoint{0.652230in}{2.014094in}}%
\pgfpathlineto{\pgfqpoint{0.663136in}{2.014094in}}%
\pgfpathlineto{\pgfqpoint{0.674042in}{2.014094in}}%
\pgfpathlineto{\pgfqpoint{0.684948in}{1.988714in}}%
\pgfpathlineto{\pgfqpoint{0.695854in}{1.988714in}}%
\pgfpathlineto{\pgfqpoint{0.706760in}{1.988714in}}%
\pgfpathlineto{\pgfqpoint{0.717666in}{1.988714in}}%
\pgfpathlineto{\pgfqpoint{0.728572in}{1.988714in}}%
\pgfpathlineto{\pgfqpoint{0.739478in}{1.988714in}}%
\pgfpathlineto{\pgfqpoint{0.750384in}{1.988714in}}%
\pgfpathlineto{\pgfqpoint{0.761290in}{1.988714in}}%
\pgfpathlineto{\pgfqpoint{0.772196in}{1.966990in}}%
\pgfpathlineto{\pgfqpoint{0.783102in}{1.924211in}}%
\pgfpathlineto{\pgfqpoint{0.794008in}{1.924211in}}%
\pgfpathlineto{\pgfqpoint{0.804914in}{1.790393in}}%
\pgfpathlineto{\pgfqpoint{0.815820in}{1.790393in}}%
\pgfpathlineto{\pgfqpoint{0.826726in}{1.790393in}}%
\pgfpathlineto{\pgfqpoint{0.837632in}{1.790393in}}%
\pgfpathlineto{\pgfqpoint{0.848538in}{1.790393in}}%
\pgfpathlineto{\pgfqpoint{0.859444in}{1.790393in}}%
\pgfpathlineto{\pgfqpoint{0.870350in}{1.787277in}}%
\pgfpathlineto{\pgfqpoint{0.881256in}{1.787277in}}%
\pgfpathlineto{\pgfqpoint{0.892162in}{1.787277in}}%
\pgfpathlineto{\pgfqpoint{0.903068in}{1.787277in}}%
\pgfpathlineto{\pgfqpoint{0.913974in}{1.754277in}}%
\pgfpathlineto{\pgfqpoint{0.924880in}{1.754277in}}%
\pgfpathlineto{\pgfqpoint{0.935786in}{1.754277in}}%
\pgfpathlineto{\pgfqpoint{0.946692in}{1.745563in}}%
\pgfpathlineto{\pgfqpoint{0.957598in}{1.742848in}}%
\pgfpathlineto{\pgfqpoint{0.968504in}{1.740413in}}%
\pgfpathlineto{\pgfqpoint{0.979410in}{1.740413in}}%
\pgfpathlineto{\pgfqpoint{0.990316in}{1.740413in}}%
\pgfpathlineto{\pgfqpoint{1.001222in}{1.738709in}}%
\pgfpathlineto{\pgfqpoint{1.012127in}{1.738570in}}%
\pgfpathlineto{\pgfqpoint{1.023033in}{1.738570in}}%
\pgfpathlineto{\pgfqpoint{1.033939in}{1.738570in}}%
\pgfpathlineto{\pgfqpoint{1.044845in}{1.689903in}}%
\pgfpathlineto{\pgfqpoint{1.055751in}{1.656445in}}%
\pgfpathlineto{\pgfqpoint{1.066657in}{1.656445in}}%
\pgfpathlineto{\pgfqpoint{1.077563in}{1.616620in}}%
\pgfpathlineto{\pgfqpoint{1.088469in}{1.613415in}}%
\pgfpathlineto{\pgfqpoint{1.099375in}{1.613410in}}%
\pgfpathlineto{\pgfqpoint{1.110281in}{1.613410in}}%
\pgfpathlineto{\pgfqpoint{1.121187in}{1.613410in}}%
\pgfpathlineto{\pgfqpoint{1.132093in}{1.613410in}}%
\pgfpathlineto{\pgfqpoint{1.142999in}{1.613410in}}%
\pgfpathlineto{\pgfqpoint{1.153905in}{1.613410in}}%
\pgfpathlineto{\pgfqpoint{1.164811in}{1.613410in}}%
\pgfpathlineto{\pgfqpoint{1.175717in}{1.613410in}}%
\pgfpathlineto{\pgfqpoint{1.186623in}{1.613410in}}%
\pgfpathlineto{\pgfqpoint{1.197529in}{1.613410in}}%
\pgfpathlineto{\pgfqpoint{1.208435in}{1.613410in}}%
\pgfpathlineto{\pgfqpoint{1.219341in}{1.613410in}}%
\pgfpathlineto{\pgfqpoint{1.230247in}{1.613410in}}%
\pgfpathlineto{\pgfqpoint{1.241153in}{1.613410in}}%
\pgfpathlineto{\pgfqpoint{1.252059in}{1.613410in}}%
\pgfpathlineto{\pgfqpoint{1.262965in}{1.613410in}}%
\pgfpathlineto{\pgfqpoint{1.273871in}{1.613410in}}%
\pgfpathlineto{\pgfqpoint{1.284777in}{1.608416in}}%
\pgfpathlineto{\pgfqpoint{1.295683in}{1.608416in}}%
\pgfpathlineto{\pgfqpoint{1.306589in}{1.608416in}}%
\pgfpathlineto{\pgfqpoint{1.317495in}{1.608416in}}%
\pgfpathlineto{\pgfqpoint{1.328401in}{1.608416in}}%
\pgfpathlineto{\pgfqpoint{1.339307in}{1.606762in}}%
\pgfpathlineto{\pgfqpoint{1.350213in}{1.546042in}}%
\pgfpathlineto{\pgfqpoint{1.361119in}{1.546042in}}%
\pgfpathlineto{\pgfqpoint{1.372025in}{1.546042in}}%
\pgfpathlineto{\pgfqpoint{1.382931in}{1.546042in}}%
\pgfpathlineto{\pgfqpoint{1.393837in}{1.546042in}}%
\pgfpathlineto{\pgfqpoint{1.404743in}{1.546042in}}%
\pgfpathlineto{\pgfqpoint{1.415649in}{1.546042in}}%
\pgfpathlineto{\pgfqpoint{1.426555in}{1.546042in}}%
\pgfpathlineto{\pgfqpoint{1.437461in}{1.546042in}}%
\pgfpathlineto{\pgfqpoint{1.448367in}{1.546042in}}%
\pgfpathlineto{\pgfqpoint{1.459273in}{1.546042in}}%
\pgfpathlineto{\pgfqpoint{1.470179in}{1.546042in}}%
\pgfpathlineto{\pgfqpoint{1.481085in}{1.546042in}}%
\pgfpathlineto{\pgfqpoint{1.491991in}{1.546042in}}%
\pgfpathlineto{\pgfqpoint{1.502896in}{1.546042in}}%
\pgfpathlineto{\pgfqpoint{1.513802in}{1.546042in}}%
\pgfpathlineto{\pgfqpoint{1.524708in}{1.546042in}}%
\pgfpathlineto{\pgfqpoint{1.535614in}{1.546042in}}%
\pgfpathlineto{\pgfqpoint{1.546520in}{1.546042in}}%
\pgfpathlineto{\pgfqpoint{1.557426in}{1.546042in}}%
\pgfpathlineto{\pgfqpoint{1.568332in}{1.546042in}}%
\pgfpathlineto{\pgfqpoint{1.579238in}{1.546042in}}%
\pgfpathlineto{\pgfqpoint{1.590144in}{1.546042in}}%
\pgfpathlineto{\pgfqpoint{1.601050in}{1.546042in}}%
\pgfpathlineto{\pgfqpoint{1.611956in}{1.546042in}}%
\pgfpathlineto{\pgfqpoint{1.622862in}{1.546042in}}%
\pgfpathlineto{\pgfqpoint{1.633768in}{1.546042in}}%
\pgfpathlineto{\pgfqpoint{1.644674in}{1.505540in}}%
\pgfpathlineto{\pgfqpoint{1.655580in}{1.505540in}}%
\pgfpathlineto{\pgfqpoint{1.666486in}{1.471307in}}%
\pgfpathlineto{\pgfqpoint{1.677392in}{1.393096in}}%
\pgfpathlineto{\pgfqpoint{1.688298in}{1.393096in}}%
\pgfpathlineto{\pgfqpoint{1.699204in}{1.393096in}}%
\pgfpathlineto{\pgfqpoint{1.710110in}{1.393096in}}%
\pgfpathlineto{\pgfqpoint{1.721016in}{1.393096in}}%
\pgfpathlineto{\pgfqpoint{1.731922in}{1.393096in}}%
\pgfpathlineto{\pgfqpoint{1.742828in}{1.393096in}}%
\pgfpathlineto{\pgfqpoint{1.753734in}{1.393096in}}%
\pgfpathlineto{\pgfqpoint{1.764640in}{1.393096in}}%
\pgfpathlineto{\pgfqpoint{1.775546in}{1.393096in}}%
\pgfpathlineto{\pgfqpoint{1.786452in}{1.393096in}}%
\pgfpathlineto{\pgfqpoint{1.797358in}{1.393096in}}%
\pgfpathlineto{\pgfqpoint{1.808264in}{1.393096in}}%
\pgfpathlineto{\pgfqpoint{1.819170in}{1.393096in}}%
\pgfpathlineto{\pgfqpoint{1.830076in}{1.393096in}}%
\pgfpathlineto{\pgfqpoint{1.840982in}{1.393096in}}%
\pgfpathlineto{\pgfqpoint{1.851888in}{1.393096in}}%
\pgfpathlineto{\pgfqpoint{1.862794in}{1.390602in}}%
\pgfpathlineto{\pgfqpoint{1.873700in}{1.390602in}}%
\pgfpathlineto{\pgfqpoint{1.884606in}{1.390602in}}%
\pgfpathlineto{\pgfqpoint{1.895512in}{1.390602in}}%
\pgfpathlineto{\pgfqpoint{1.906418in}{1.390602in}}%
\pgfpathlineto{\pgfqpoint{1.917324in}{1.390602in}}%
\pgfpathlineto{\pgfqpoint{1.928230in}{1.390602in}}%
\pgfpathlineto{\pgfqpoint{1.939136in}{1.390602in}}%
\pgfpathlineto{\pgfqpoint{1.950042in}{1.390602in}}%
\pgfpathlineto{\pgfqpoint{1.960948in}{1.390602in}}%
\pgfpathlineto{\pgfqpoint{1.971854in}{1.390602in}}%
\pgfpathlineto{\pgfqpoint{1.982759in}{1.390602in}}%
\pgfpathlineto{\pgfqpoint{1.993665in}{1.390602in}}%
\pgfpathlineto{\pgfqpoint{2.004571in}{1.390602in}}%
\pgfpathlineto{\pgfqpoint{2.015477in}{1.390602in}}%
\pgfpathlineto{\pgfqpoint{2.026383in}{1.390602in}}%
\pgfpathlineto{\pgfqpoint{2.037289in}{1.390602in}}%
\pgfpathlineto{\pgfqpoint{2.048195in}{1.390602in}}%
\pgfpathlineto{\pgfqpoint{2.059101in}{1.390602in}}%
\pgfpathlineto{\pgfqpoint{2.070007in}{1.390602in}}%
\pgfpathlineto{\pgfqpoint{2.080913in}{1.390602in}}%
\pgfpathlineto{\pgfqpoint{2.091819in}{1.390602in}}%
\pgfpathlineto{\pgfqpoint{2.102725in}{1.390602in}}%
\pgfpathlineto{\pgfqpoint{2.113631in}{1.390602in}}%
\pgfpathlineto{\pgfqpoint{2.124537in}{1.390602in}}%
\pgfpathlineto{\pgfqpoint{2.135443in}{1.390602in}}%
\pgfpathlineto{\pgfqpoint{2.146349in}{1.390602in}}%
\pgfpathlineto{\pgfqpoint{2.157255in}{1.390602in}}%
\pgfpathlineto{\pgfqpoint{2.168161in}{1.390602in}}%
\pgfpathlineto{\pgfqpoint{2.179067in}{1.390602in}}%
\pgfpathlineto{\pgfqpoint{2.189973in}{1.390602in}}%
\pgfpathlineto{\pgfqpoint{2.200879in}{1.390602in}}%
\pgfpathlineto{\pgfqpoint{2.211785in}{1.390602in}}%
\pgfpathlineto{\pgfqpoint{2.222691in}{1.379756in}}%
\pgfpathlineto{\pgfqpoint{2.233597in}{1.379756in}}%
\pgfpathlineto{\pgfqpoint{2.244503in}{1.379756in}}%
\pgfpathlineto{\pgfqpoint{2.255409in}{1.379756in}}%
\pgfpathlineto{\pgfqpoint{2.266315in}{1.379756in}}%
\pgfpathlineto{\pgfqpoint{2.277221in}{1.379756in}}%
\pgfpathlineto{\pgfqpoint{2.288127in}{1.379756in}}%
\pgfpathlineto{\pgfqpoint{2.299033in}{1.379756in}}%
\pgfpathlineto{\pgfqpoint{2.309939in}{1.379756in}}%
\pgfpathlineto{\pgfqpoint{2.320845in}{1.379756in}}%
\pgfpathlineto{\pgfqpoint{2.331751in}{1.379756in}}%
\pgfpathlineto{\pgfqpoint{2.342657in}{1.379756in}}%
\pgfpathlineto{\pgfqpoint{2.353563in}{1.379756in}}%
\pgfpathlineto{\pgfqpoint{2.364469in}{1.379756in}}%
\pgfpathlineto{\pgfqpoint{2.375375in}{1.379756in}}%
\pgfpathlineto{\pgfqpoint{2.386281in}{1.379756in}}%
\pgfpathlineto{\pgfqpoint{2.397187in}{1.379756in}}%
\pgfpathlineto{\pgfqpoint{2.408093in}{1.379756in}}%
\pgfpathlineto{\pgfqpoint{2.418999in}{1.379756in}}%
\pgfpathlineto{\pgfqpoint{2.429905in}{1.379756in}}%
\pgfpathlineto{\pgfqpoint{2.440811in}{1.379756in}}%
\pgfpathlineto{\pgfqpoint{2.451717in}{1.379756in}}%
\pgfpathlineto{\pgfqpoint{2.462623in}{1.379756in}}%
\pgfpathlineto{\pgfqpoint{2.473528in}{1.379756in}}%
\pgfpathlineto{\pgfqpoint{2.484434in}{1.379756in}}%
\pgfpathlineto{\pgfqpoint{2.495340in}{1.379756in}}%
\pgfpathlineto{\pgfqpoint{2.506246in}{1.379756in}}%
\pgfpathlineto{\pgfqpoint{2.517152in}{1.379756in}}%
\pgfpathlineto{\pgfqpoint{2.528058in}{1.379756in}}%
\pgfpathlineto{\pgfqpoint{2.538964in}{1.379756in}}%
\pgfpathlineto{\pgfqpoint{2.549870in}{1.379756in}}%
\pgfpathlineto{\pgfqpoint{2.560776in}{1.379756in}}%
\pgfpathlineto{\pgfqpoint{2.571682in}{1.379756in}}%
\pgfpathlineto{\pgfqpoint{2.582588in}{1.379756in}}%
\pgfpathlineto{\pgfqpoint{2.593494in}{1.379756in}}%
\pgfpathlineto{\pgfqpoint{2.604400in}{1.379756in}}%
\pgfpathlineto{\pgfqpoint{2.615306in}{1.379756in}}%
\pgfpathlineto{\pgfqpoint{2.626212in}{1.379756in}}%
\pgfpathlineto{\pgfqpoint{2.637118in}{1.379756in}}%
\pgfpathlineto{\pgfqpoint{2.648024in}{1.379756in}}%
\pgfpathlineto{\pgfqpoint{2.658930in}{1.379756in}}%
\pgfpathlineto{\pgfqpoint{2.669836in}{1.379756in}}%
\pgfpathlineto{\pgfqpoint{2.680742in}{1.379756in}}%
\pgfpathlineto{\pgfqpoint{2.691648in}{1.379756in}}%
\pgfpathlineto{\pgfqpoint{2.702554in}{1.379756in}}%
\pgfpathlineto{\pgfqpoint{2.713460in}{1.379756in}}%
\pgfpathlineto{\pgfqpoint{2.724366in}{1.379756in}}%
\pgfpathlineto{\pgfqpoint{2.735272in}{1.379756in}}%
\pgfpathlineto{\pgfqpoint{2.746178in}{1.379756in}}%
\pgfpathlineto{\pgfqpoint{2.757084in}{1.379756in}}%
\pgfpathlineto{\pgfqpoint{2.767990in}{1.379756in}}%
\pgfpathlineto{\pgfqpoint{2.778896in}{1.379756in}}%
\pgfpathlineto{\pgfqpoint{2.789802in}{1.379756in}}%
\pgfpathlineto{\pgfqpoint{2.800708in}{1.379756in}}%
\pgfpathlineto{\pgfqpoint{2.811614in}{1.379756in}}%
\pgfpathlineto{\pgfqpoint{2.822520in}{1.379756in}}%
\pgfpathlineto{\pgfqpoint{2.833426in}{1.379756in}}%
\pgfpathlineto{\pgfqpoint{2.844332in}{1.379756in}}%
\pgfpathlineto{\pgfqpoint{2.855238in}{1.379756in}}%
\pgfpathlineto{\pgfqpoint{2.866144in}{1.379756in}}%
\pgfpathlineto{\pgfqpoint{2.877050in}{1.379756in}}%
\pgfpathlineto{\pgfqpoint{2.887956in}{1.379756in}}%
\pgfpathlineto{\pgfqpoint{2.898862in}{1.379756in}}%
\pgfpathlineto{\pgfqpoint{2.909768in}{1.379756in}}%
\pgfpathlineto{\pgfqpoint{2.920674in}{1.379756in}}%
\pgfpathlineto{\pgfqpoint{2.931580in}{1.379756in}}%
\pgfpathlineto{\pgfqpoint{2.931580in}{0.955033in}}%
\pgfpathlineto{\pgfqpoint{2.931580in}{0.955033in}}%
\pgfpathlineto{\pgfqpoint{2.920674in}{0.955033in}}%
\pgfpathlineto{\pgfqpoint{2.909768in}{0.955033in}}%
\pgfpathlineto{\pgfqpoint{2.898862in}{0.955033in}}%
\pgfpathlineto{\pgfqpoint{2.887956in}{0.955033in}}%
\pgfpathlineto{\pgfqpoint{2.877050in}{0.955033in}}%
\pgfpathlineto{\pgfqpoint{2.866144in}{0.955033in}}%
\pgfpathlineto{\pgfqpoint{2.855238in}{0.955033in}}%
\pgfpathlineto{\pgfqpoint{2.844332in}{0.955033in}}%
\pgfpathlineto{\pgfqpoint{2.833426in}{0.955033in}}%
\pgfpathlineto{\pgfqpoint{2.822520in}{0.955033in}}%
\pgfpathlineto{\pgfqpoint{2.811614in}{0.955033in}}%
\pgfpathlineto{\pgfqpoint{2.800708in}{0.955033in}}%
\pgfpathlineto{\pgfqpoint{2.789802in}{0.955033in}}%
\pgfpathlineto{\pgfqpoint{2.778896in}{0.955033in}}%
\pgfpathlineto{\pgfqpoint{2.767990in}{0.955033in}}%
\pgfpathlineto{\pgfqpoint{2.757084in}{0.955033in}}%
\pgfpathlineto{\pgfqpoint{2.746178in}{0.955033in}}%
\pgfpathlineto{\pgfqpoint{2.735272in}{0.955033in}}%
\pgfpathlineto{\pgfqpoint{2.724366in}{0.955033in}}%
\pgfpathlineto{\pgfqpoint{2.713460in}{0.955033in}}%
\pgfpathlineto{\pgfqpoint{2.702554in}{0.955033in}}%
\pgfpathlineto{\pgfqpoint{2.691648in}{0.955033in}}%
\pgfpathlineto{\pgfqpoint{2.680742in}{0.955033in}}%
\pgfpathlineto{\pgfqpoint{2.669836in}{0.955033in}}%
\pgfpathlineto{\pgfqpoint{2.658930in}{0.955033in}}%
\pgfpathlineto{\pgfqpoint{2.648024in}{0.955033in}}%
\pgfpathlineto{\pgfqpoint{2.637118in}{0.955033in}}%
\pgfpathlineto{\pgfqpoint{2.626212in}{0.955033in}}%
\pgfpathlineto{\pgfqpoint{2.615306in}{0.955033in}}%
\pgfpathlineto{\pgfqpoint{2.604400in}{0.955033in}}%
\pgfpathlineto{\pgfqpoint{2.593494in}{0.955033in}}%
\pgfpathlineto{\pgfqpoint{2.582588in}{0.955033in}}%
\pgfpathlineto{\pgfqpoint{2.571682in}{0.955033in}}%
\pgfpathlineto{\pgfqpoint{2.560776in}{0.955033in}}%
\pgfpathlineto{\pgfqpoint{2.549870in}{0.955033in}}%
\pgfpathlineto{\pgfqpoint{2.538964in}{0.955033in}}%
\pgfpathlineto{\pgfqpoint{2.528058in}{0.955033in}}%
\pgfpathlineto{\pgfqpoint{2.517152in}{0.955033in}}%
\pgfpathlineto{\pgfqpoint{2.506246in}{0.955033in}}%
\pgfpathlineto{\pgfqpoint{2.495340in}{0.955033in}}%
\pgfpathlineto{\pgfqpoint{2.484434in}{0.955033in}}%
\pgfpathlineto{\pgfqpoint{2.473528in}{0.955033in}}%
\pgfpathlineto{\pgfqpoint{2.462623in}{0.955033in}}%
\pgfpathlineto{\pgfqpoint{2.451717in}{0.955033in}}%
\pgfpathlineto{\pgfqpoint{2.440811in}{0.955033in}}%
\pgfpathlineto{\pgfqpoint{2.429905in}{0.955033in}}%
\pgfpathlineto{\pgfqpoint{2.418999in}{0.955033in}}%
\pgfpathlineto{\pgfqpoint{2.408093in}{0.955033in}}%
\pgfpathlineto{\pgfqpoint{2.397187in}{0.955033in}}%
\pgfpathlineto{\pgfqpoint{2.386281in}{0.955033in}}%
\pgfpathlineto{\pgfqpoint{2.375375in}{0.955033in}}%
\pgfpathlineto{\pgfqpoint{2.364469in}{0.955033in}}%
\pgfpathlineto{\pgfqpoint{2.353563in}{0.955033in}}%
\pgfpathlineto{\pgfqpoint{2.342657in}{0.955033in}}%
\pgfpathlineto{\pgfqpoint{2.331751in}{0.955033in}}%
\pgfpathlineto{\pgfqpoint{2.320845in}{0.955033in}}%
\pgfpathlineto{\pgfqpoint{2.309939in}{0.955033in}}%
\pgfpathlineto{\pgfqpoint{2.299033in}{0.955033in}}%
\pgfpathlineto{\pgfqpoint{2.288127in}{0.955033in}}%
\pgfpathlineto{\pgfqpoint{2.277221in}{0.955033in}}%
\pgfpathlineto{\pgfqpoint{2.266315in}{0.955033in}}%
\pgfpathlineto{\pgfqpoint{2.255409in}{0.955033in}}%
\pgfpathlineto{\pgfqpoint{2.244503in}{0.955033in}}%
\pgfpathlineto{\pgfqpoint{2.233597in}{0.955033in}}%
\pgfpathlineto{\pgfqpoint{2.222691in}{0.955033in}}%
\pgfpathlineto{\pgfqpoint{2.211785in}{0.984484in}}%
\pgfpathlineto{\pgfqpoint{2.200879in}{0.984484in}}%
\pgfpathlineto{\pgfqpoint{2.189973in}{0.984484in}}%
\pgfpathlineto{\pgfqpoint{2.179067in}{0.984484in}}%
\pgfpathlineto{\pgfqpoint{2.168161in}{0.984484in}}%
\pgfpathlineto{\pgfqpoint{2.157255in}{0.984484in}}%
\pgfpathlineto{\pgfqpoint{2.146349in}{0.984484in}}%
\pgfpathlineto{\pgfqpoint{2.135443in}{0.984484in}}%
\pgfpathlineto{\pgfqpoint{2.124537in}{0.984484in}}%
\pgfpathlineto{\pgfqpoint{2.113631in}{0.984484in}}%
\pgfpathlineto{\pgfqpoint{2.102725in}{0.984484in}}%
\pgfpathlineto{\pgfqpoint{2.091819in}{0.984484in}}%
\pgfpathlineto{\pgfqpoint{2.080913in}{0.984484in}}%
\pgfpathlineto{\pgfqpoint{2.070007in}{0.984484in}}%
\pgfpathlineto{\pgfqpoint{2.059101in}{0.984484in}}%
\pgfpathlineto{\pgfqpoint{2.048195in}{0.984484in}}%
\pgfpathlineto{\pgfqpoint{2.037289in}{0.984484in}}%
\pgfpathlineto{\pgfqpoint{2.026383in}{0.984484in}}%
\pgfpathlineto{\pgfqpoint{2.015477in}{0.984484in}}%
\pgfpathlineto{\pgfqpoint{2.004571in}{0.984484in}}%
\pgfpathlineto{\pgfqpoint{1.993665in}{0.984484in}}%
\pgfpathlineto{\pgfqpoint{1.982759in}{0.984484in}}%
\pgfpathlineto{\pgfqpoint{1.971854in}{0.984484in}}%
\pgfpathlineto{\pgfqpoint{1.960948in}{0.984484in}}%
\pgfpathlineto{\pgfqpoint{1.950042in}{0.984484in}}%
\pgfpathlineto{\pgfqpoint{1.939136in}{0.984484in}}%
\pgfpathlineto{\pgfqpoint{1.928230in}{0.984484in}}%
\pgfpathlineto{\pgfqpoint{1.917324in}{0.984484in}}%
\pgfpathlineto{\pgfqpoint{1.906418in}{0.984484in}}%
\pgfpathlineto{\pgfqpoint{1.895512in}{0.984484in}}%
\pgfpathlineto{\pgfqpoint{1.884606in}{0.984484in}}%
\pgfpathlineto{\pgfqpoint{1.873700in}{0.984484in}}%
\pgfpathlineto{\pgfqpoint{1.862794in}{0.984484in}}%
\pgfpathlineto{\pgfqpoint{1.851888in}{0.990509in}}%
\pgfpathlineto{\pgfqpoint{1.840982in}{0.990509in}}%
\pgfpathlineto{\pgfqpoint{1.830076in}{0.990509in}}%
\pgfpathlineto{\pgfqpoint{1.819170in}{0.990509in}}%
\pgfpathlineto{\pgfqpoint{1.808264in}{0.990509in}}%
\pgfpathlineto{\pgfqpoint{1.797358in}{0.990509in}}%
\pgfpathlineto{\pgfqpoint{1.786452in}{0.990509in}}%
\pgfpathlineto{\pgfqpoint{1.775546in}{0.990509in}}%
\pgfpathlineto{\pgfqpoint{1.764640in}{0.990509in}}%
\pgfpathlineto{\pgfqpoint{1.753734in}{0.990509in}}%
\pgfpathlineto{\pgfqpoint{1.742828in}{0.990509in}}%
\pgfpathlineto{\pgfqpoint{1.731922in}{0.990509in}}%
\pgfpathlineto{\pgfqpoint{1.721016in}{0.990509in}}%
\pgfpathlineto{\pgfqpoint{1.710110in}{0.990509in}}%
\pgfpathlineto{\pgfqpoint{1.699204in}{0.990509in}}%
\pgfpathlineto{\pgfqpoint{1.688298in}{0.990509in}}%
\pgfpathlineto{\pgfqpoint{1.677392in}{0.990509in}}%
\pgfpathlineto{\pgfqpoint{1.666486in}{1.006432in}}%
\pgfpathlineto{\pgfqpoint{1.655580in}{1.013341in}}%
\pgfpathlineto{\pgfqpoint{1.644674in}{1.013341in}}%
\pgfpathlineto{\pgfqpoint{1.633768in}{1.021635in}}%
\pgfpathlineto{\pgfqpoint{1.622862in}{1.021635in}}%
\pgfpathlineto{\pgfqpoint{1.611956in}{1.021635in}}%
\pgfpathlineto{\pgfqpoint{1.601050in}{1.021635in}}%
\pgfpathlineto{\pgfqpoint{1.590144in}{1.021635in}}%
\pgfpathlineto{\pgfqpoint{1.579238in}{1.021635in}}%
\pgfpathlineto{\pgfqpoint{1.568332in}{1.021635in}}%
\pgfpathlineto{\pgfqpoint{1.557426in}{1.021635in}}%
\pgfpathlineto{\pgfqpoint{1.546520in}{1.021635in}}%
\pgfpathlineto{\pgfqpoint{1.535614in}{1.021635in}}%
\pgfpathlineto{\pgfqpoint{1.524708in}{1.021635in}}%
\pgfpathlineto{\pgfqpoint{1.513802in}{1.021635in}}%
\pgfpathlineto{\pgfqpoint{1.502896in}{1.021635in}}%
\pgfpathlineto{\pgfqpoint{1.491991in}{1.021635in}}%
\pgfpathlineto{\pgfqpoint{1.481085in}{1.021635in}}%
\pgfpathlineto{\pgfqpoint{1.470179in}{1.021635in}}%
\pgfpathlineto{\pgfqpoint{1.459273in}{1.021635in}}%
\pgfpathlineto{\pgfqpoint{1.448367in}{1.021635in}}%
\pgfpathlineto{\pgfqpoint{1.437461in}{1.021635in}}%
\pgfpathlineto{\pgfqpoint{1.426555in}{1.021635in}}%
\pgfpathlineto{\pgfqpoint{1.415649in}{1.021635in}}%
\pgfpathlineto{\pgfqpoint{1.404743in}{1.021635in}}%
\pgfpathlineto{\pgfqpoint{1.393837in}{1.021635in}}%
\pgfpathlineto{\pgfqpoint{1.382931in}{1.021635in}}%
\pgfpathlineto{\pgfqpoint{1.372025in}{1.021635in}}%
\pgfpathlineto{\pgfqpoint{1.361119in}{1.021635in}}%
\pgfpathlineto{\pgfqpoint{1.350213in}{1.021635in}}%
\pgfpathlineto{\pgfqpoint{1.339307in}{1.034532in}}%
\pgfpathlineto{\pgfqpoint{1.328401in}{1.042110in}}%
\pgfpathlineto{\pgfqpoint{1.317495in}{1.042110in}}%
\pgfpathlineto{\pgfqpoint{1.306589in}{1.042110in}}%
\pgfpathlineto{\pgfqpoint{1.295683in}{1.042110in}}%
\pgfpathlineto{\pgfqpoint{1.284777in}{1.042110in}}%
\pgfpathlineto{\pgfqpoint{1.273871in}{1.063482in}}%
\pgfpathlineto{\pgfqpoint{1.262965in}{1.063482in}}%
\pgfpathlineto{\pgfqpoint{1.252059in}{1.063482in}}%
\pgfpathlineto{\pgfqpoint{1.241153in}{1.063482in}}%
\pgfpathlineto{\pgfqpoint{1.230247in}{1.063482in}}%
\pgfpathlineto{\pgfqpoint{1.219341in}{1.063482in}}%
\pgfpathlineto{\pgfqpoint{1.208435in}{1.063482in}}%
\pgfpathlineto{\pgfqpoint{1.197529in}{1.063482in}}%
\pgfpathlineto{\pgfqpoint{1.186623in}{1.063482in}}%
\pgfpathlineto{\pgfqpoint{1.175717in}{1.063482in}}%
\pgfpathlineto{\pgfqpoint{1.164811in}{1.063482in}}%
\pgfpathlineto{\pgfqpoint{1.153905in}{1.063482in}}%
\pgfpathlineto{\pgfqpoint{1.142999in}{1.063482in}}%
\pgfpathlineto{\pgfqpoint{1.132093in}{1.063482in}}%
\pgfpathlineto{\pgfqpoint{1.121187in}{1.063482in}}%
\pgfpathlineto{\pgfqpoint{1.110281in}{1.063482in}}%
\pgfpathlineto{\pgfqpoint{1.099375in}{1.063482in}}%
\pgfpathlineto{\pgfqpoint{1.088469in}{1.063502in}}%
\pgfpathlineto{\pgfqpoint{1.077563in}{1.076140in}}%
\pgfpathlineto{\pgfqpoint{1.066657in}{1.284628in}}%
\pgfpathlineto{\pgfqpoint{1.055751in}{1.284628in}}%
\pgfpathlineto{\pgfqpoint{1.044845in}{1.378578in}}%
\pgfpathlineto{\pgfqpoint{1.033939in}{1.475945in}}%
\pgfpathlineto{\pgfqpoint{1.023033in}{1.475945in}}%
\pgfpathlineto{\pgfqpoint{1.012127in}{1.475945in}}%
\pgfpathlineto{\pgfqpoint{1.001222in}{1.476812in}}%
\pgfpathlineto{\pgfqpoint{0.990316in}{1.486949in}}%
\pgfpathlineto{\pgfqpoint{0.979410in}{1.486949in}}%
\pgfpathlineto{\pgfqpoint{0.968504in}{1.486949in}}%
\pgfpathlineto{\pgfqpoint{0.957598in}{1.499958in}}%
\pgfpathlineto{\pgfqpoint{0.946692in}{1.512741in}}%
\pgfpathlineto{\pgfqpoint{0.935786in}{1.544846in}}%
\pgfpathlineto{\pgfqpoint{0.924880in}{1.544846in}}%
\pgfpathlineto{\pgfqpoint{0.913974in}{1.544846in}}%
\pgfpathlineto{\pgfqpoint{0.903068in}{1.607920in}}%
\pgfpathlineto{\pgfqpoint{0.892162in}{1.607920in}}%
\pgfpathlineto{\pgfqpoint{0.881256in}{1.607920in}}%
\pgfpathlineto{\pgfqpoint{0.870350in}{1.607920in}}%
\pgfpathlineto{\pgfqpoint{0.859444in}{1.629742in}}%
\pgfpathlineto{\pgfqpoint{0.848538in}{1.629742in}}%
\pgfpathlineto{\pgfqpoint{0.837632in}{1.629742in}}%
\pgfpathlineto{\pgfqpoint{0.826726in}{1.629742in}}%
\pgfpathlineto{\pgfqpoint{0.815820in}{1.629742in}}%
\pgfpathlineto{\pgfqpoint{0.804914in}{1.629742in}}%
\pgfpathlineto{\pgfqpoint{0.794008in}{1.699957in}}%
\pgfpathlineto{\pgfqpoint{0.783102in}{1.699957in}}%
\pgfpathlineto{\pgfqpoint{0.772196in}{1.732137in}}%
\pgfpathlineto{\pgfqpoint{0.761290in}{1.743351in}}%
\pgfpathlineto{\pgfqpoint{0.750384in}{1.743351in}}%
\pgfpathlineto{\pgfqpoint{0.739478in}{1.743351in}}%
\pgfpathlineto{\pgfqpoint{0.728572in}{1.743351in}}%
\pgfpathlineto{\pgfqpoint{0.717666in}{1.743351in}}%
\pgfpathlineto{\pgfqpoint{0.706760in}{1.743351in}}%
\pgfpathlineto{\pgfqpoint{0.695854in}{1.743351in}}%
\pgfpathlineto{\pgfqpoint{0.684948in}{1.743351in}}%
\pgfpathlineto{\pgfqpoint{0.674042in}{1.756007in}}%
\pgfpathlineto{\pgfqpoint{0.663136in}{1.756007in}}%
\pgfpathlineto{\pgfqpoint{0.652230in}{1.756007in}}%
\pgfpathlineto{\pgfqpoint{0.641324in}{1.795201in}}%
\pgfpathlineto{\pgfqpoint{0.630418in}{1.850510in}}%
\pgfpathlineto{\pgfqpoint{0.619512in}{1.954830in}}%
\pgfpathlineto{\pgfqpoint{0.608606in}{1.980977in}}%
\pgfpathlineto{\pgfqpoint{0.597700in}{1.980977in}}%
\pgfpathlineto{\pgfqpoint{0.586794in}{1.980977in}}%
\pgfpathlineto{\pgfqpoint{0.575888in}{1.980977in}}%
\pgfpathlineto{\pgfqpoint{0.564982in}{2.273909in}}%
\pgfpathlineto{\pgfqpoint{0.554076in}{2.316923in}}%
\pgfpathlineto{\pgfqpoint{0.543170in}{2.390136in}}%
\pgfpathclose%
\pgfusepath{fill}%
\end{pgfscope}%
\begin{pgfscope}%
\pgfpathrectangle{\pgfqpoint{0.423750in}{0.375000in}}{\pgfqpoint{2.627250in}{2.265000in}}%
\pgfusepath{clip}%
\pgfsetroundcap%
\pgfsetroundjoin%
\pgfsetlinewidth{1.505625pt}%
\definecolor{currentstroke}{rgb}{0.121569,0.466667,0.705882}%
\pgfsetstrokecolor{currentstroke}%
\pgfsetdash{}{0pt}%
\pgfpathmoveto{\pgfqpoint{0.543170in}{2.435770in}}%
\pgfpathlineto{\pgfqpoint{0.554076in}{2.348850in}}%
\pgfpathlineto{\pgfqpoint{0.564982in}{2.320500in}}%
\pgfpathlineto{\pgfqpoint{0.575888in}{2.312642in}}%
\pgfpathlineto{\pgfqpoint{0.586794in}{2.280921in}}%
\pgfpathlineto{\pgfqpoint{0.597700in}{2.214032in}}%
\pgfpathlineto{\pgfqpoint{0.608606in}{2.179600in}}%
\pgfpathlineto{\pgfqpoint{0.619512in}{2.179600in}}%
\pgfpathlineto{\pgfqpoint{0.630418in}{2.137836in}}%
\pgfpathlineto{\pgfqpoint{0.641324in}{2.119065in}}%
\pgfpathlineto{\pgfqpoint{0.652230in}{2.041657in}}%
\pgfpathlineto{\pgfqpoint{0.663136in}{1.978380in}}%
\pgfpathlineto{\pgfqpoint{0.684948in}{1.978380in}}%
\pgfpathlineto{\pgfqpoint{0.695854in}{1.885860in}}%
\pgfpathlineto{\pgfqpoint{0.706760in}{1.832370in}}%
\pgfpathlineto{\pgfqpoint{0.728572in}{1.778426in}}%
\pgfpathlineto{\pgfqpoint{0.794008in}{1.778426in}}%
\pgfpathlineto{\pgfqpoint{0.804914in}{1.634516in}}%
\pgfpathlineto{\pgfqpoint{0.903068in}{1.634516in}}%
\pgfpathlineto{\pgfqpoint{0.913974in}{1.632735in}}%
\pgfpathlineto{\pgfqpoint{0.968504in}{1.632735in}}%
\pgfpathlineto{\pgfqpoint{0.979410in}{1.616257in}}%
\pgfpathlineto{\pgfqpoint{1.033939in}{1.616257in}}%
\pgfpathlineto{\pgfqpoint{1.044845in}{1.592932in}}%
\pgfpathlineto{\pgfqpoint{1.175717in}{1.592932in}}%
\pgfpathlineto{\pgfqpoint{1.186623in}{1.537639in}}%
\pgfpathlineto{\pgfqpoint{1.197529in}{1.506003in}}%
\pgfpathlineto{\pgfqpoint{1.208435in}{1.496947in}}%
\pgfpathlineto{\pgfqpoint{1.230247in}{1.496947in}}%
\pgfpathlineto{\pgfqpoint{1.241153in}{1.448688in}}%
\pgfpathlineto{\pgfqpoint{1.317495in}{1.448688in}}%
\pgfpathlineto{\pgfqpoint{1.328401in}{1.417781in}}%
\pgfpathlineto{\pgfqpoint{1.666486in}{1.417194in}}%
\pgfpathlineto{\pgfqpoint{1.677392in}{1.405103in}}%
\pgfpathlineto{\pgfqpoint{1.830076in}{1.405103in}}%
\pgfpathlineto{\pgfqpoint{1.840982in}{1.377314in}}%
\pgfpathlineto{\pgfqpoint{1.873700in}{1.377314in}}%
\pgfpathlineto{\pgfqpoint{1.884606in}{1.374064in}}%
\pgfpathlineto{\pgfqpoint{2.233597in}{1.374064in}}%
\pgfpathlineto{\pgfqpoint{2.244503in}{1.364854in}}%
\pgfpathlineto{\pgfqpoint{2.299033in}{1.364854in}}%
\pgfpathlineto{\pgfqpoint{2.309939in}{1.353524in}}%
\pgfpathlineto{\pgfqpoint{2.778896in}{1.353524in}}%
\pgfpathlineto{\pgfqpoint{2.789802in}{1.327084in}}%
\pgfpathlineto{\pgfqpoint{2.931580in}{1.327084in}}%
\pgfpathlineto{\pgfqpoint{2.931580in}{1.327084in}}%
\pgfusepath{stroke}%
\end{pgfscope}%
\begin{pgfscope}%
\pgfpathrectangle{\pgfqpoint{0.423750in}{0.375000in}}{\pgfqpoint{2.627250in}{2.265000in}}%
\pgfusepath{clip}%
\pgfsetroundcap%
\pgfsetroundjoin%
\pgfsetlinewidth{1.505625pt}%
\definecolor{currentstroke}{rgb}{1.000000,0.498039,0.054902}%
\pgfsetstrokecolor{currentstroke}%
\pgfsetdash{}{0pt}%
\pgfpathmoveto{\pgfqpoint{0.543170in}{2.276135in}}%
\pgfpathlineto{\pgfqpoint{0.554076in}{2.123882in}}%
\pgfpathlineto{\pgfqpoint{0.564982in}{2.086288in}}%
\pgfpathlineto{\pgfqpoint{0.575888in}{1.995231in}}%
\pgfpathlineto{\pgfqpoint{0.597700in}{1.995231in}}%
\pgfpathlineto{\pgfqpoint{0.608606in}{1.936746in}}%
\pgfpathlineto{\pgfqpoint{0.630418in}{1.936746in}}%
\pgfpathlineto{\pgfqpoint{0.641324in}{1.921252in}}%
\pgfpathlineto{\pgfqpoint{0.674042in}{1.921252in}}%
\pgfpathlineto{\pgfqpoint{0.684948in}{1.886686in}}%
\pgfpathlineto{\pgfqpoint{0.739478in}{1.886686in}}%
\pgfpathlineto{\pgfqpoint{0.750384in}{1.870300in}}%
\pgfpathlineto{\pgfqpoint{0.761290in}{1.870300in}}%
\pgfpathlineto{\pgfqpoint{0.772196in}{1.839251in}}%
\pgfpathlineto{\pgfqpoint{0.815820in}{1.839251in}}%
\pgfpathlineto{\pgfqpoint{0.826726in}{1.771155in}}%
\pgfpathlineto{\pgfqpoint{0.837632in}{1.771155in}}%
\pgfpathlineto{\pgfqpoint{0.848538in}{1.768733in}}%
\pgfpathlineto{\pgfqpoint{0.903068in}{1.768733in}}%
\pgfpathlineto{\pgfqpoint{0.913974in}{1.764274in}}%
\pgfpathlineto{\pgfqpoint{0.968504in}{1.764274in}}%
\pgfpathlineto{\pgfqpoint{0.979410in}{1.720288in}}%
\pgfpathlineto{\pgfqpoint{0.990316in}{1.654938in}}%
\pgfpathlineto{\pgfqpoint{1.099375in}{1.654938in}}%
\pgfpathlineto{\pgfqpoint{1.110281in}{1.574168in}}%
\pgfpathlineto{\pgfqpoint{1.175717in}{1.574168in}}%
\pgfpathlineto{\pgfqpoint{1.186623in}{1.549209in}}%
\pgfpathlineto{\pgfqpoint{1.208435in}{1.549209in}}%
\pgfpathlineto{\pgfqpoint{1.219341in}{1.541846in}}%
\pgfpathlineto{\pgfqpoint{1.295683in}{1.541846in}}%
\pgfpathlineto{\pgfqpoint{1.306589in}{1.534360in}}%
\pgfpathlineto{\pgfqpoint{1.328401in}{1.534360in}}%
\pgfpathlineto{\pgfqpoint{1.339307in}{1.476632in}}%
\pgfpathlineto{\pgfqpoint{1.437461in}{1.475918in}}%
\pgfpathlineto{\pgfqpoint{1.448367in}{1.462389in}}%
\pgfpathlineto{\pgfqpoint{1.470179in}{1.462389in}}%
\pgfpathlineto{\pgfqpoint{1.481085in}{1.410351in}}%
\pgfpathlineto{\pgfqpoint{1.491991in}{1.379022in}}%
\pgfpathlineto{\pgfqpoint{1.546520in}{1.379022in}}%
\pgfpathlineto{\pgfqpoint{1.557426in}{1.371759in}}%
\pgfpathlineto{\pgfqpoint{1.601050in}{1.371759in}}%
\pgfpathlineto{\pgfqpoint{1.611956in}{1.310294in}}%
\pgfpathlineto{\pgfqpoint{1.677392in}{1.310294in}}%
\pgfpathlineto{\pgfqpoint{1.688298in}{1.213209in}}%
\pgfpathlineto{\pgfqpoint{1.808264in}{1.213209in}}%
\pgfpathlineto{\pgfqpoint{1.819170in}{1.193528in}}%
\pgfpathlineto{\pgfqpoint{1.906418in}{1.193528in}}%
\pgfpathlineto{\pgfqpoint{1.917324in}{1.189216in}}%
\pgfpathlineto{\pgfqpoint{1.993665in}{1.189216in}}%
\pgfpathlineto{\pgfqpoint{2.004571in}{1.164087in}}%
\pgfpathlineto{\pgfqpoint{2.124537in}{1.164087in}}%
\pgfpathlineto{\pgfqpoint{2.135443in}{1.135213in}}%
\pgfpathlineto{\pgfqpoint{2.244503in}{1.135213in}}%
\pgfpathlineto{\pgfqpoint{2.255409in}{1.132923in}}%
\pgfpathlineto{\pgfqpoint{2.320845in}{1.132923in}}%
\pgfpathlineto{\pgfqpoint{2.331751in}{1.075989in}}%
\pgfpathlineto{\pgfqpoint{2.746178in}{1.075989in}}%
\pgfpathlineto{\pgfqpoint{2.757084in}{0.996012in}}%
\pgfpathlineto{\pgfqpoint{2.811614in}{0.995161in}}%
\pgfpathlineto{\pgfqpoint{2.931580in}{0.995161in}}%
\pgfpathlineto{\pgfqpoint{2.931580in}{0.995161in}}%
\pgfusepath{stroke}%
\end{pgfscope}%
\begin{pgfscope}%
\pgfpathrectangle{\pgfqpoint{0.423750in}{0.375000in}}{\pgfqpoint{2.627250in}{2.265000in}}%
\pgfusepath{clip}%
\pgfsetroundcap%
\pgfsetroundjoin%
\pgfsetlinewidth{1.505625pt}%
\definecolor{currentstroke}{rgb}{0.172549,0.627451,0.172549}%
\pgfsetstrokecolor{currentstroke}%
\pgfsetdash{}{0pt}%
\pgfpathmoveto{\pgfqpoint{0.543170in}{2.404912in}}%
\pgfpathlineto{\pgfqpoint{0.554076in}{2.315186in}}%
\pgfpathlineto{\pgfqpoint{0.564982in}{2.259326in}}%
\pgfpathlineto{\pgfqpoint{0.575888in}{2.259326in}}%
\pgfpathlineto{\pgfqpoint{0.586794in}{2.192949in}}%
\pgfpathlineto{\pgfqpoint{0.597700in}{2.192949in}}%
\pgfpathlineto{\pgfqpoint{0.608606in}{2.158646in}}%
\pgfpathlineto{\pgfqpoint{0.619512in}{2.158646in}}%
\pgfpathlineto{\pgfqpoint{0.630418in}{2.123845in}}%
\pgfpathlineto{\pgfqpoint{0.641324in}{2.097800in}}%
\pgfpathlineto{\pgfqpoint{0.652230in}{2.093878in}}%
\pgfpathlineto{\pgfqpoint{0.674042in}{2.093878in}}%
\pgfpathlineto{\pgfqpoint{0.684948in}{2.025298in}}%
\pgfpathlineto{\pgfqpoint{0.706760in}{2.025298in}}%
\pgfpathlineto{\pgfqpoint{0.717666in}{1.991836in}}%
\pgfpathlineto{\pgfqpoint{0.728572in}{1.991836in}}%
\pgfpathlineto{\pgfqpoint{0.739478in}{1.979986in}}%
\pgfpathlineto{\pgfqpoint{0.750384in}{1.934576in}}%
\pgfpathlineto{\pgfqpoint{0.815820in}{1.933886in}}%
\pgfpathlineto{\pgfqpoint{0.826726in}{1.850848in}}%
\pgfpathlineto{\pgfqpoint{0.903068in}{1.850848in}}%
\pgfpathlineto{\pgfqpoint{0.913974in}{1.847078in}}%
\pgfpathlineto{\pgfqpoint{0.924880in}{1.839258in}}%
\pgfpathlineto{\pgfqpoint{0.935786in}{1.759465in}}%
\pgfpathlineto{\pgfqpoint{0.946692in}{1.759465in}}%
\pgfpathlineto{\pgfqpoint{0.957598in}{1.627553in}}%
\pgfpathlineto{\pgfqpoint{1.001222in}{1.627553in}}%
\pgfpathlineto{\pgfqpoint{1.012127in}{1.575516in}}%
\pgfpathlineto{\pgfqpoint{1.099375in}{1.575516in}}%
\pgfpathlineto{\pgfqpoint{1.110281in}{1.571911in}}%
\pgfpathlineto{\pgfqpoint{1.153905in}{1.571911in}}%
\pgfpathlineto{\pgfqpoint{1.164811in}{1.540251in}}%
\pgfpathlineto{\pgfqpoint{1.197529in}{1.540251in}}%
\pgfpathlineto{\pgfqpoint{1.208435in}{1.459541in}}%
\pgfpathlineto{\pgfqpoint{1.262965in}{1.459541in}}%
\pgfpathlineto{\pgfqpoint{1.273871in}{1.424802in}}%
\pgfpathlineto{\pgfqpoint{1.295683in}{1.424802in}}%
\pgfpathlineto{\pgfqpoint{1.306589in}{1.402165in}}%
\pgfpathlineto{\pgfqpoint{1.317495in}{1.389163in}}%
\pgfpathlineto{\pgfqpoint{1.328401in}{1.371224in}}%
\pgfpathlineto{\pgfqpoint{1.372025in}{1.371224in}}%
\pgfpathlineto{\pgfqpoint{1.382931in}{1.368442in}}%
\pgfpathlineto{\pgfqpoint{1.491991in}{1.368442in}}%
\pgfpathlineto{\pgfqpoint{1.502896in}{1.351435in}}%
\pgfpathlineto{\pgfqpoint{1.611956in}{1.351435in}}%
\pgfpathlineto{\pgfqpoint{1.622862in}{1.255363in}}%
\pgfpathlineto{\pgfqpoint{1.699204in}{1.255363in}}%
\pgfpathlineto{\pgfqpoint{1.710110in}{1.249722in}}%
\pgfpathlineto{\pgfqpoint{1.830076in}{1.249722in}}%
\pgfpathlineto{\pgfqpoint{1.840982in}{1.116966in}}%
\pgfpathlineto{\pgfqpoint{1.960948in}{1.116966in}}%
\pgfpathlineto{\pgfqpoint{1.971854in}{1.101338in}}%
\pgfpathlineto{\pgfqpoint{1.982759in}{1.022283in}}%
\pgfpathlineto{\pgfqpoint{2.004571in}{1.022283in}}%
\pgfpathlineto{\pgfqpoint{2.015477in}{1.009352in}}%
\pgfpathlineto{\pgfqpoint{2.124537in}{1.009352in}}%
\pgfpathlineto{\pgfqpoint{2.135443in}{1.000279in}}%
\pgfpathlineto{\pgfqpoint{2.266315in}{1.000279in}}%
\pgfpathlineto{\pgfqpoint{2.277221in}{0.986624in}}%
\pgfpathlineto{\pgfqpoint{2.288127in}{0.969462in}}%
\pgfpathlineto{\pgfqpoint{2.299033in}{0.964562in}}%
\pgfpathlineto{\pgfqpoint{2.320845in}{0.964562in}}%
\pgfpathlineto{\pgfqpoint{2.331751in}{0.836631in}}%
\pgfpathlineto{\pgfqpoint{2.342657in}{0.735392in}}%
\pgfpathlineto{\pgfqpoint{2.353563in}{0.714928in}}%
\pgfpathlineto{\pgfqpoint{2.517152in}{0.714928in}}%
\pgfpathlineto{\pgfqpoint{2.528058in}{0.702554in}}%
\pgfpathlineto{\pgfqpoint{2.593494in}{0.702554in}}%
\pgfpathlineto{\pgfqpoint{2.604400in}{0.693656in}}%
\pgfpathlineto{\pgfqpoint{2.931580in}{0.693656in}}%
\pgfpathlineto{\pgfqpoint{2.931580in}{0.693656in}}%
\pgfusepath{stroke}%
\end{pgfscope}%
\begin{pgfscope}%
\pgfpathrectangle{\pgfqpoint{0.423750in}{0.375000in}}{\pgfqpoint{2.627250in}{2.265000in}}%
\pgfusepath{clip}%
\pgfsetroundcap%
\pgfsetroundjoin%
\pgfsetlinewidth{1.505625pt}%
\definecolor{currentstroke}{rgb}{0.839216,0.152941,0.156863}%
\pgfsetstrokecolor{currentstroke}%
\pgfsetdash{}{0pt}%
\pgfpathmoveto{\pgfqpoint{0.543170in}{2.465993in}}%
\pgfpathlineto{\pgfqpoint{0.554076in}{2.375378in}}%
\pgfpathlineto{\pgfqpoint{0.564982in}{2.325069in}}%
\pgfpathlineto{\pgfqpoint{0.575888in}{2.125032in}}%
\pgfpathlineto{\pgfqpoint{0.608606in}{2.125032in}}%
\pgfpathlineto{\pgfqpoint{0.619512in}{2.093311in}}%
\pgfpathlineto{\pgfqpoint{0.641324in}{1.922917in}}%
\pgfpathlineto{\pgfqpoint{0.652230in}{1.897498in}}%
\pgfpathlineto{\pgfqpoint{0.674042in}{1.897498in}}%
\pgfpathlineto{\pgfqpoint{0.684948in}{1.876980in}}%
\pgfpathlineto{\pgfqpoint{0.761290in}{1.876980in}}%
\pgfpathlineto{\pgfqpoint{0.772196in}{1.859340in}}%
\pgfpathlineto{\pgfqpoint{0.783102in}{1.820741in}}%
\pgfpathlineto{\pgfqpoint{0.794008in}{1.820741in}}%
\pgfpathlineto{\pgfqpoint{0.804914in}{1.713333in}}%
\pgfpathlineto{\pgfqpoint{0.859444in}{1.713333in}}%
\pgfpathlineto{\pgfqpoint{0.870350in}{1.702212in}}%
\pgfpathlineto{\pgfqpoint{0.903068in}{1.702212in}}%
\pgfpathlineto{\pgfqpoint{0.913974in}{1.656758in}}%
\pgfpathlineto{\pgfqpoint{0.935786in}{1.656758in}}%
\pgfpathlineto{\pgfqpoint{0.946692in}{1.638709in}}%
\pgfpathlineto{\pgfqpoint{0.968504in}{1.625573in}}%
\pgfpathlineto{\pgfqpoint{0.990316in}{1.625573in}}%
\pgfpathlineto{\pgfqpoint{1.001222in}{1.620674in}}%
\pgfpathlineto{\pgfqpoint{1.033939in}{1.620261in}}%
\pgfpathlineto{\pgfqpoint{1.044845in}{1.553917in}}%
\pgfpathlineto{\pgfqpoint{1.055751in}{1.500243in}}%
\pgfpathlineto{\pgfqpoint{1.066657in}{1.500243in}}%
\pgfpathlineto{\pgfqpoint{1.077563in}{1.413496in}}%
\pgfpathlineto{\pgfqpoint{1.088469in}{1.408046in}}%
\pgfpathlineto{\pgfqpoint{1.273871in}{1.408037in}}%
\pgfpathlineto{\pgfqpoint{1.284777in}{1.399233in}}%
\pgfpathlineto{\pgfqpoint{1.328401in}{1.399233in}}%
\pgfpathlineto{\pgfqpoint{1.339307in}{1.396229in}}%
\pgfpathlineto{\pgfqpoint{1.350213in}{1.346831in}}%
\pgfpathlineto{\pgfqpoint{1.633768in}{1.346831in}}%
\pgfpathlineto{\pgfqpoint{1.644674in}{1.314509in}}%
\pgfpathlineto{\pgfqpoint{1.655580in}{1.314509in}}%
\pgfpathlineto{\pgfqpoint{1.666486in}{1.287581in}}%
\pgfpathlineto{\pgfqpoint{1.677392in}{1.227327in}}%
\pgfpathlineto{\pgfqpoint{1.851888in}{1.227327in}}%
\pgfpathlineto{\pgfqpoint{1.862794in}{1.223764in}}%
\pgfpathlineto{\pgfqpoint{2.211785in}{1.223764in}}%
\pgfpathlineto{\pgfqpoint{2.222691in}{1.207392in}}%
\pgfpathlineto{\pgfqpoint{2.931580in}{1.207392in}}%
\pgfpathlineto{\pgfqpoint{2.931580in}{1.207392in}}%
\pgfusepath{stroke}%
\end{pgfscope}%
\begin{pgfscope}%
\pgfsetrectcap%
\pgfsetmiterjoin%
\pgfsetlinewidth{0.000000pt}%
\definecolor{currentstroke}{rgb}{1.000000,1.000000,1.000000}%
\pgfsetstrokecolor{currentstroke}%
\pgfsetdash{}{0pt}%
\pgfpathmoveto{\pgfqpoint{0.423750in}{0.375000in}}%
\pgfpathlineto{\pgfqpoint{0.423750in}{2.640000in}}%
\pgfusepath{}%
\end{pgfscope}%
\begin{pgfscope}%
\pgfsetrectcap%
\pgfsetmiterjoin%
\pgfsetlinewidth{0.000000pt}%
\definecolor{currentstroke}{rgb}{1.000000,1.000000,1.000000}%
\pgfsetstrokecolor{currentstroke}%
\pgfsetdash{}{0pt}%
\pgfpathmoveto{\pgfqpoint{3.051000in}{0.375000in}}%
\pgfpathlineto{\pgfqpoint{3.051000in}{2.640000in}}%
\pgfusepath{}%
\end{pgfscope}%
\begin{pgfscope}%
\pgfsetrectcap%
\pgfsetmiterjoin%
\pgfsetlinewidth{0.000000pt}%
\definecolor{currentstroke}{rgb}{1.000000,1.000000,1.000000}%
\pgfsetstrokecolor{currentstroke}%
\pgfsetdash{}{0pt}%
\pgfpathmoveto{\pgfqpoint{0.423750in}{0.375000in}}%
\pgfpathlineto{\pgfqpoint{3.051000in}{0.375000in}}%
\pgfusepath{}%
\end{pgfscope}%
\begin{pgfscope}%
\pgfsetrectcap%
\pgfsetmiterjoin%
\pgfsetlinewidth{0.000000pt}%
\definecolor{currentstroke}{rgb}{1.000000,1.000000,1.000000}%
\pgfsetstrokecolor{currentstroke}%
\pgfsetdash{}{0pt}%
\pgfpathmoveto{\pgfqpoint{0.423750in}{2.640000in}}%
\pgfpathlineto{\pgfqpoint{3.051000in}{2.640000in}}%
\pgfusepath{}%
\end{pgfscope}%
\begin{pgfscope}%
\definecolor{textcolor}{rgb}{0.150000,0.150000,0.150000}%
\pgfsetstrokecolor{textcolor}%
\pgfsetfillcolor{textcolor}%
\pgftext[x=1.737375in,y=2.723333in,,base]{\color{textcolor}\rmfamily\fontsize{8.000000}{9.600000}\selectfont Griewank}%
\end{pgfscope}%
\begin{pgfscope}%
\pgfsetroundcap%
\pgfsetroundjoin%
\pgfsetlinewidth{1.505625pt}%
\definecolor{currentstroke}{rgb}{0.121569,0.466667,0.705882}%
\pgfsetstrokecolor{currentstroke}%
\pgfsetdash{}{0pt}%
\pgfpathmoveto{\pgfqpoint{1.320607in}{2.494470in}}%
\pgfpathlineto{\pgfqpoint{1.542829in}{2.494470in}}%
\pgfusepath{stroke}%
\end{pgfscope}%
\begin{pgfscope}%
\definecolor{textcolor}{rgb}{0.150000,0.150000,0.150000}%
\pgfsetstrokecolor{textcolor}%
\pgfsetfillcolor{textcolor}%
\pgftext[x=1.631718in,y=2.455582in,left,base]{\color{textcolor}\rmfamily\fontsize{8.000000}{9.600000}\selectfont random}%
\end{pgfscope}%
\begin{pgfscope}%
\pgfsetroundcap%
\pgfsetroundjoin%
\pgfsetlinewidth{1.505625pt}%
\definecolor{currentstroke}{rgb}{1.000000,0.498039,0.054902}%
\pgfsetstrokecolor{currentstroke}%
\pgfsetdash{}{0pt}%
\pgfpathmoveto{\pgfqpoint{1.320607in}{2.331385in}}%
\pgfpathlineto{\pgfqpoint{1.542829in}{2.331385in}}%
\pgfusepath{stroke}%
\end{pgfscope}%
\begin{pgfscope}%
\definecolor{textcolor}{rgb}{0.150000,0.150000,0.150000}%
\pgfsetstrokecolor{textcolor}%
\pgfsetfillcolor{textcolor}%
\pgftext[x=1.631718in,y=2.292496in,left,base]{\color{textcolor}\rmfamily\fontsize{8.000000}{9.600000}\selectfont 5 x DNGO retrain-reset}%
\end{pgfscope}%
\begin{pgfscope}%
\pgfsetroundcap%
\pgfsetroundjoin%
\pgfsetlinewidth{1.505625pt}%
\definecolor{currentstroke}{rgb}{0.172549,0.627451,0.172549}%
\pgfsetstrokecolor{currentstroke}%
\pgfsetdash{}{0pt}%
\pgfpathmoveto{\pgfqpoint{1.320607in}{2.168299in}}%
\pgfpathlineto{\pgfqpoint{1.542829in}{2.168299in}}%
\pgfusepath{stroke}%
\end{pgfscope}%
\begin{pgfscope}%
\definecolor{textcolor}{rgb}{0.150000,0.150000,0.150000}%
\pgfsetstrokecolor{textcolor}%
\pgfsetfillcolor{textcolor}%
\pgftext[x=1.631718in,y=2.129410in,left,base]{\color{textcolor}\rmfamily\fontsize{8.000000}{9.600000}\selectfont DNGO retrain-reset}%
\end{pgfscope}%
\begin{pgfscope}%
\pgfsetroundcap%
\pgfsetroundjoin%
\pgfsetlinewidth{1.505625pt}%
\definecolor{currentstroke}{rgb}{0.839216,0.152941,0.156863}%
\pgfsetstrokecolor{currentstroke}%
\pgfsetdash{}{0pt}%
\pgfpathmoveto{\pgfqpoint{1.320607in}{2.005213in}}%
\pgfpathlineto{\pgfqpoint{1.542829in}{2.005213in}}%
\pgfusepath{stroke}%
\end{pgfscope}%
\begin{pgfscope}%
\definecolor{textcolor}{rgb}{0.150000,0.150000,0.150000}%
\pgfsetstrokecolor{textcolor}%
\pgfsetfillcolor{textcolor}%
\pgftext[x=1.631718in,y=1.966324in,left,base]{\color{textcolor}\rmfamily\fontsize{8.000000}{9.600000}\selectfont GP}%
\end{pgfscope}%
\end{pgfpicture}%
\makeatother%
\endgroup%

        \end{minipage}\qquad
        \begin{minipage}{0.45\linewidth}
            \centering
            \begin{minipage}{0.45\linewidth}
                %% Creator: Matplotlib, PGF backend
%%
%% To include the figure in your LaTeX document, write
%%   \input{<filename>.pgf}
%%
%% Make sure the required packages are loaded in your preamble
%%   \usepackage{pgf}
%%
%% Figures using additional raster images can only be included by \input if
%% they are in the same directory as the main LaTeX file. For loading figures
%% from other directories you can use the `import` package
%%   \usepackage{import}
%% and then include the figures with
%%   \import{<path to file>}{<filename>.pgf}
%%
%% Matplotlib used the following preamble
%%   \usepackage{gensymb}
%%   \usepackage{fontspec}
%%   \setmainfont{DejaVu Serif}
%%   \setsansfont{Arial}
%%   \setmonofont{DejaVu Sans Mono}
%%
\begingroup%
\makeatletter%
\begin{pgfpicture}%
\pgfpathrectangle{\pgfpointorigin}{\pgfqpoint{1.695000in}{1.695000in}}%
\pgfusepath{use as bounding box, clip}%
\begin{pgfscope}%
\pgfsetbuttcap%
\pgfsetmiterjoin%
\definecolor{currentfill}{rgb}{1.000000,1.000000,1.000000}%
\pgfsetfillcolor{currentfill}%
\pgfsetlinewidth{0.000000pt}%
\definecolor{currentstroke}{rgb}{1.000000,1.000000,1.000000}%
\pgfsetstrokecolor{currentstroke}%
\pgfsetdash{}{0pt}%
\pgfpathmoveto{\pgfqpoint{0.000000in}{0.000000in}}%
\pgfpathlineto{\pgfqpoint{1.695000in}{0.000000in}}%
\pgfpathlineto{\pgfqpoint{1.695000in}{1.695000in}}%
\pgfpathlineto{\pgfqpoint{0.000000in}{1.695000in}}%
\pgfpathclose%
\pgfusepath{fill}%
\end{pgfscope}%
\begin{pgfscope}%
\pgfsetbuttcap%
\pgfsetmiterjoin%
\definecolor{currentfill}{rgb}{0.917647,0.917647,0.949020}%
\pgfsetfillcolor{currentfill}%
\pgfsetlinewidth{0.000000pt}%
\definecolor{currentstroke}{rgb}{0.000000,0.000000,0.000000}%
\pgfsetstrokecolor{currentstroke}%
\pgfsetstrokeopacity{0.000000}%
\pgfsetdash{}{0pt}%
\pgfpathmoveto{\pgfqpoint{0.211875in}{0.211875in}}%
\pgfpathlineto{\pgfqpoint{1.525500in}{0.211875in}}%
\pgfpathlineto{\pgfqpoint{1.525500in}{1.491600in}}%
\pgfpathlineto{\pgfqpoint{0.211875in}{1.491600in}}%
\pgfpathclose%
\pgfusepath{fill}%
\end{pgfscope}%
\begin{pgfscope}%
\pgfpathrectangle{\pgfqpoint{0.211875in}{0.211875in}}{\pgfqpoint{1.313625in}{1.279725in}}%
\pgfusepath{clip}%
\pgfsetroundcap%
\pgfsetroundjoin%
\pgfsetlinewidth{0.803000pt}%
\definecolor{currentstroke}{rgb}{1.000000,1.000000,1.000000}%
\pgfsetstrokecolor{currentstroke}%
\pgfsetdash{}{0pt}%
\pgfpathmoveto{\pgfqpoint{0.211875in}{0.211875in}}%
\pgfpathlineto{\pgfqpoint{0.211875in}{1.491600in}}%
\pgfusepath{stroke}%
\end{pgfscope}%
\begin{pgfscope}%
\definecolor{textcolor}{rgb}{0.150000,0.150000,0.150000}%
\pgfsetstrokecolor{textcolor}%
\pgfsetfillcolor{textcolor}%
\pgftext[x=0.211875in,y=0.163264in,,top]{\color{textcolor}\rmfamily\fontsize{8.000000}{9.600000}\selectfont \(\displaystyle -50\)}%
\end{pgfscope}%
\begin{pgfscope}%
\pgfpathrectangle{\pgfqpoint{0.211875in}{0.211875in}}{\pgfqpoint{1.313625in}{1.279725in}}%
\pgfusepath{clip}%
\pgfsetroundcap%
\pgfsetroundjoin%
\pgfsetlinewidth{0.803000pt}%
\definecolor{currentstroke}{rgb}{1.000000,1.000000,1.000000}%
\pgfsetstrokecolor{currentstroke}%
\pgfsetdash{}{0pt}%
\pgfpathmoveto{\pgfqpoint{0.681027in}{0.211875in}}%
\pgfpathlineto{\pgfqpoint{0.681027in}{1.491600in}}%
\pgfusepath{stroke}%
\end{pgfscope}%
\begin{pgfscope}%
\definecolor{textcolor}{rgb}{0.150000,0.150000,0.150000}%
\pgfsetstrokecolor{textcolor}%
\pgfsetfillcolor{textcolor}%
\pgftext[x=0.681027in,y=0.163264in,,top]{\color{textcolor}\rmfamily\fontsize{8.000000}{9.600000}\selectfont \(\displaystyle -25\)}%
\end{pgfscope}%
\begin{pgfscope}%
\pgfpathrectangle{\pgfqpoint{0.211875in}{0.211875in}}{\pgfqpoint{1.313625in}{1.279725in}}%
\pgfusepath{clip}%
\pgfsetroundcap%
\pgfsetroundjoin%
\pgfsetlinewidth{0.803000pt}%
\definecolor{currentstroke}{rgb}{1.000000,1.000000,1.000000}%
\pgfsetstrokecolor{currentstroke}%
\pgfsetdash{}{0pt}%
\pgfpathmoveto{\pgfqpoint{1.150179in}{0.211875in}}%
\pgfpathlineto{\pgfqpoint{1.150179in}{1.491600in}}%
\pgfusepath{stroke}%
\end{pgfscope}%
\begin{pgfscope}%
\definecolor{textcolor}{rgb}{0.150000,0.150000,0.150000}%
\pgfsetstrokecolor{textcolor}%
\pgfsetfillcolor{textcolor}%
\pgftext[x=1.150179in,y=0.163264in,,top]{\color{textcolor}\rmfamily\fontsize{8.000000}{9.600000}\selectfont \(\displaystyle 0\)}%
\end{pgfscope}%
\begin{pgfscope}%
\pgfpathrectangle{\pgfqpoint{0.211875in}{0.211875in}}{\pgfqpoint{1.313625in}{1.279725in}}%
\pgfusepath{clip}%
\pgfsetroundcap%
\pgfsetroundjoin%
\pgfsetlinewidth{0.803000pt}%
\definecolor{currentstroke}{rgb}{1.000000,1.000000,1.000000}%
\pgfsetstrokecolor{currentstroke}%
\pgfsetdash{}{0pt}%
\pgfpathmoveto{\pgfqpoint{0.211875in}{0.394693in}}%
\pgfpathlineto{\pgfqpoint{1.525500in}{0.394693in}}%
\pgfusepath{stroke}%
\end{pgfscope}%
\begin{pgfscope}%
\definecolor{textcolor}{rgb}{0.150000,0.150000,0.150000}%
\pgfsetstrokecolor{textcolor}%
\pgfsetfillcolor{textcolor}%
\pgftext[x=-0.046616in,y=0.352484in,left,base]{\color{textcolor}\rmfamily\fontsize{8.000000}{9.600000}\selectfont \(\displaystyle -40\)}%
\end{pgfscope}%
\begin{pgfscope}%
\pgfpathrectangle{\pgfqpoint{0.211875in}{0.211875in}}{\pgfqpoint{1.313625in}{1.279725in}}%
\pgfusepath{clip}%
\pgfsetroundcap%
\pgfsetroundjoin%
\pgfsetlinewidth{0.803000pt}%
\definecolor{currentstroke}{rgb}{1.000000,1.000000,1.000000}%
\pgfsetstrokecolor{currentstroke}%
\pgfsetdash{}{0pt}%
\pgfpathmoveto{\pgfqpoint{0.211875in}{0.760329in}}%
\pgfpathlineto{\pgfqpoint{1.525500in}{0.760329in}}%
\pgfusepath{stroke}%
\end{pgfscope}%
\begin{pgfscope}%
\definecolor{textcolor}{rgb}{0.150000,0.150000,0.150000}%
\pgfsetstrokecolor{textcolor}%
\pgfsetfillcolor{textcolor}%
\pgftext[x=-0.046616in,y=0.718119in,left,base]{\color{textcolor}\rmfamily\fontsize{8.000000}{9.600000}\selectfont \(\displaystyle -20\)}%
\end{pgfscope}%
\begin{pgfscope}%
\pgfpathrectangle{\pgfqpoint{0.211875in}{0.211875in}}{\pgfqpoint{1.313625in}{1.279725in}}%
\pgfusepath{clip}%
\pgfsetroundcap%
\pgfsetroundjoin%
\pgfsetlinewidth{0.803000pt}%
\definecolor{currentstroke}{rgb}{1.000000,1.000000,1.000000}%
\pgfsetstrokecolor{currentstroke}%
\pgfsetdash{}{0pt}%
\pgfpathmoveto{\pgfqpoint{0.211875in}{1.125964in}}%
\pgfpathlineto{\pgfqpoint{1.525500in}{1.125964in}}%
\pgfusepath{stroke}%
\end{pgfscope}%
\begin{pgfscope}%
\definecolor{textcolor}{rgb}{0.150000,0.150000,0.150000}%
\pgfsetstrokecolor{textcolor}%
\pgfsetfillcolor{textcolor}%
\pgftext[x=0.104235in,y=1.083755in,left,base]{\color{textcolor}\rmfamily\fontsize{8.000000}{9.600000}\selectfont \(\displaystyle 0\)}%
\end{pgfscope}%
\begin{pgfscope}%
\pgfpathrectangle{\pgfqpoint{0.211875in}{0.211875in}}{\pgfqpoint{1.313625in}{1.279725in}}%
\pgfusepath{clip}%
\pgfsetroundcap%
\pgfsetroundjoin%
\pgfsetlinewidth{0.803000pt}%
\definecolor{currentstroke}{rgb}{1.000000,1.000000,1.000000}%
\pgfsetstrokecolor{currentstroke}%
\pgfsetdash{}{0pt}%
\pgfpathmoveto{\pgfqpoint{0.211875in}{1.491600in}}%
\pgfpathlineto{\pgfqpoint{1.525500in}{1.491600in}}%
\pgfusepath{stroke}%
\end{pgfscope}%
\begin{pgfscope}%
\definecolor{textcolor}{rgb}{0.150000,0.150000,0.150000}%
\pgfsetstrokecolor{textcolor}%
\pgfsetfillcolor{textcolor}%
\pgftext[x=0.045207in,y=1.449391in,left,base]{\color{textcolor}\rmfamily\fontsize{8.000000}{9.600000}\selectfont \(\displaystyle 20\)}%
\end{pgfscope}%
\begin{pgfscope}%
\pgfpathrectangle{\pgfqpoint{0.211875in}{0.211875in}}{\pgfqpoint{1.313625in}{1.279725in}}%
\pgfusepath{clip}%
\pgfsetbuttcap%
\pgfsetroundjoin%
\definecolor{currentfill}{rgb}{0.067555,0.047782,0.142002}%
\pgfsetfillcolor{currentfill}%
\pgfsetlinewidth{0.000000pt}%
\definecolor{currentstroke}{rgb}{0.000000,0.000000,0.000000}%
\pgfsetstrokecolor{currentstroke}%
\pgfsetdash{}{0pt}%
\pgfpathmoveto{\pgfqpoint{0.216993in}{0.224802in}}%
\pgfpathlineto{\pgfqpoint{0.217054in}{0.237728in}}%
\pgfpathlineto{\pgfqpoint{0.211875in}{0.246921in}}%
\pgfpathlineto{\pgfqpoint{0.211875in}{0.237728in}}%
\pgfpathlineto{\pgfqpoint{0.211875in}{0.224802in}}%
\pgfpathlineto{\pgfqpoint{0.211875in}{0.216136in}}%
\pgfpathclose%
\pgfusepath{fill}%
\end{pgfscope}%
\begin{pgfscope}%
\pgfpathrectangle{\pgfqpoint{0.211875in}{0.211875in}}{\pgfqpoint{1.313625in}{1.279725in}}%
\pgfusepath{clip}%
\pgfsetbuttcap%
\pgfsetroundjoin%
\definecolor{currentfill}{rgb}{0.067555,0.047782,0.142002}%
\pgfsetfillcolor{currentfill}%
\pgfsetlinewidth{0.000000pt}%
\definecolor{currentstroke}{rgb}{0.000000,0.000000,0.000000}%
\pgfsetstrokecolor{currentstroke}%
\pgfsetdash{}{0pt}%
\pgfpathmoveto{\pgfqpoint{0.264951in}{0.307781in}}%
\pgfpathlineto{\pgfqpoint{0.267605in}{0.315287in}}%
\pgfpathlineto{\pgfqpoint{0.264951in}{0.318774in}}%
\pgfpathlineto{\pgfqpoint{0.262597in}{0.315287in}}%
\pgfpathclose%
\pgfusepath{fill}%
\end{pgfscope}%
\begin{pgfscope}%
\pgfpathrectangle{\pgfqpoint{0.211875in}{0.211875in}}{\pgfqpoint{1.313625in}{1.279725in}}%
\pgfusepath{clip}%
\pgfsetbuttcap%
\pgfsetroundjoin%
\definecolor{currentfill}{rgb}{0.198046,0.094652,0.234785}%
\pgfsetfillcolor{currentfill}%
\pgfsetlinewidth{0.000000pt}%
\definecolor{currentstroke}{rgb}{0.000000,0.000000,0.000000}%
\pgfsetstrokecolor{currentstroke}%
\pgfsetdash{}{0pt}%
\pgfpathmoveto{\pgfqpoint{0.225144in}{0.219402in}}%
\pgfpathlineto{\pgfqpoint{0.226216in}{0.224802in}}%
\pgfpathlineto{\pgfqpoint{0.226113in}{0.237728in}}%
\pgfpathlineto{\pgfqpoint{0.225144in}{0.242322in}}%
\pgfpathlineto{\pgfqpoint{0.221483in}{0.250655in}}%
\pgfpathlineto{\pgfqpoint{0.211875in}{0.258136in}}%
\pgfpathlineto{\pgfqpoint{0.211875in}{0.250655in}}%
\pgfpathlineto{\pgfqpoint{0.211875in}{0.246921in}}%
\pgfpathlineto{\pgfqpoint{0.217054in}{0.237728in}}%
\pgfpathlineto{\pgfqpoint{0.216993in}{0.224802in}}%
\pgfpathlineto{\pgfqpoint{0.211875in}{0.216136in}}%
\pgfpathlineto{\pgfqpoint{0.211875in}{0.211875in}}%
\pgfpathlineto{\pgfqpoint{0.221837in}{0.211875in}}%
\pgfpathclose%
\pgfusepath{fill}%
\end{pgfscope}%
\begin{pgfscope}%
\pgfpathrectangle{\pgfqpoint{0.211875in}{0.211875in}}{\pgfqpoint{1.313625in}{1.279725in}}%
\pgfusepath{clip}%
\pgfsetbuttcap%
\pgfsetroundjoin%
\definecolor{currentfill}{rgb}{0.198046,0.094652,0.234785}%
\pgfsetfillcolor{currentfill}%
\pgfsetlinewidth{0.000000pt}%
\definecolor{currentstroke}{rgb}{0.000000,0.000000,0.000000}%
\pgfsetstrokecolor{currentstroke}%
\pgfsetdash{}{0pt}%
\pgfpathmoveto{\pgfqpoint{0.318027in}{0.211875in}}%
\pgfpathlineto{\pgfqpoint{0.331295in}{0.211875in}}%
\pgfpathlineto{\pgfqpoint{0.333649in}{0.211875in}}%
\pgfpathlineto{\pgfqpoint{0.339796in}{0.224802in}}%
\pgfpathlineto{\pgfqpoint{0.339743in}{0.237728in}}%
\pgfpathlineto{\pgfqpoint{0.333809in}{0.250655in}}%
\pgfpathlineto{\pgfqpoint{0.331295in}{0.252854in}}%
\pgfpathlineto{\pgfqpoint{0.318027in}{0.253245in}}%
\pgfpathlineto{\pgfqpoint{0.314860in}{0.250655in}}%
\pgfpathlineto{\pgfqpoint{0.308799in}{0.237728in}}%
\pgfpathlineto{\pgfqpoint{0.308715in}{0.224802in}}%
\pgfpathlineto{\pgfqpoint{0.314950in}{0.211875in}}%
\pgfpathclose%
\pgfusepath{fill}%
\end{pgfscope}%
\begin{pgfscope}%
\pgfpathrectangle{\pgfqpoint{0.211875in}{0.211875in}}{\pgfqpoint{1.313625in}{1.279725in}}%
\pgfusepath{clip}%
\pgfsetbuttcap%
\pgfsetroundjoin%
\definecolor{currentfill}{rgb}{0.198046,0.094652,0.234785}%
\pgfsetfillcolor{currentfill}%
\pgfsetlinewidth{0.000000pt}%
\definecolor{currentstroke}{rgb}{0.000000,0.000000,0.000000}%
\pgfsetstrokecolor{currentstroke}%
\pgfsetdash{}{0pt}%
\pgfpathmoveto{\pgfqpoint{0.437447in}{0.214316in}}%
\pgfpathlineto{\pgfqpoint{0.450716in}{0.217350in}}%
\pgfpathlineto{\pgfqpoint{0.454186in}{0.224802in}}%
\pgfpathlineto{\pgfqpoint{0.454219in}{0.237728in}}%
\pgfpathlineto{\pgfqpoint{0.450716in}{0.245549in}}%
\pgfpathlineto{\pgfqpoint{0.437447in}{0.248775in}}%
\pgfpathlineto{\pgfqpoint{0.430885in}{0.237728in}}%
\pgfpathlineto{\pgfqpoint{0.430908in}{0.224802in}}%
\pgfpathclose%
\pgfusepath{fill}%
\end{pgfscope}%
\begin{pgfscope}%
\pgfpathrectangle{\pgfqpoint{0.211875in}{0.211875in}}{\pgfqpoint{1.313625in}{1.279725in}}%
\pgfusepath{clip}%
\pgfsetbuttcap%
\pgfsetroundjoin%
\definecolor{currentfill}{rgb}{0.198046,0.094652,0.234785}%
\pgfsetfillcolor{currentfill}%
\pgfsetlinewidth{0.000000pt}%
\definecolor{currentstroke}{rgb}{0.000000,0.000000,0.000000}%
\pgfsetstrokecolor{currentstroke}%
\pgfsetdash{}{0pt}%
\pgfpathmoveto{\pgfqpoint{0.556867in}{0.219619in}}%
\pgfpathlineto{\pgfqpoint{0.568016in}{0.224802in}}%
\pgfpathlineto{\pgfqpoint{0.568455in}{0.237728in}}%
\pgfpathlineto{\pgfqpoint{0.556867in}{0.243382in}}%
\pgfpathlineto{\pgfqpoint{0.553022in}{0.237728in}}%
\pgfpathlineto{\pgfqpoint{0.553156in}{0.224802in}}%
\pgfpathclose%
\pgfusepath{fill}%
\end{pgfscope}%
\begin{pgfscope}%
\pgfpathrectangle{\pgfqpoint{0.211875in}{0.211875in}}{\pgfqpoint{1.313625in}{1.279725in}}%
\pgfusepath{clip}%
\pgfsetbuttcap%
\pgfsetroundjoin%
\definecolor{currentfill}{rgb}{0.198046,0.094652,0.234785}%
\pgfsetfillcolor{currentfill}%
\pgfsetlinewidth{0.000000pt}%
\definecolor{currentstroke}{rgb}{0.000000,0.000000,0.000000}%
\pgfsetstrokecolor{currentstroke}%
\pgfsetdash{}{0pt}%
\pgfpathmoveto{\pgfqpoint{0.676288in}{0.223934in}}%
\pgfpathlineto{\pgfqpoint{0.677597in}{0.224802in}}%
\pgfpathlineto{\pgfqpoint{0.678062in}{0.237728in}}%
\pgfpathlineto{\pgfqpoint{0.676288in}{0.238966in}}%
\pgfpathlineto{\pgfqpoint{0.675307in}{0.237728in}}%
\pgfpathlineto{\pgfqpoint{0.675563in}{0.224802in}}%
\pgfpathclose%
\pgfusepath{fill}%
\end{pgfscope}%
\begin{pgfscope}%
\pgfpathrectangle{\pgfqpoint{0.211875in}{0.211875in}}{\pgfqpoint{1.313625in}{1.279725in}}%
\pgfusepath{clip}%
\pgfsetbuttcap%
\pgfsetroundjoin%
\definecolor{currentfill}{rgb}{0.198046,0.094652,0.234785}%
\pgfsetfillcolor{currentfill}%
\pgfsetlinewidth{0.000000pt}%
\definecolor{currentstroke}{rgb}{0.000000,0.000000,0.000000}%
\pgfsetstrokecolor{currentstroke}%
\pgfsetdash{}{0pt}%
\pgfpathmoveto{\pgfqpoint{0.251682in}{0.299961in}}%
\pgfpathlineto{\pgfqpoint{0.264951in}{0.290769in}}%
\pgfpathlineto{\pgfqpoint{0.278220in}{0.298894in}}%
\pgfpathlineto{\pgfqpoint{0.279732in}{0.302361in}}%
\pgfpathlineto{\pgfqpoint{0.280869in}{0.315287in}}%
\pgfpathlineto{\pgfqpoint{0.278220in}{0.326810in}}%
\pgfpathlineto{\pgfqpoint{0.277212in}{0.328214in}}%
\pgfpathlineto{\pgfqpoint{0.264951in}{0.334391in}}%
\pgfpathlineto{\pgfqpoint{0.253801in}{0.328214in}}%
\pgfpathlineto{\pgfqpoint{0.251682in}{0.324723in}}%
\pgfpathlineto{\pgfqpoint{0.249659in}{0.315287in}}%
\pgfpathlineto{\pgfqpoint{0.250716in}{0.302361in}}%
\pgfpathclose%
\pgfpathmoveto{\pgfqpoint{0.262597in}{0.315287in}}%
\pgfpathlineto{\pgfqpoint{0.264951in}{0.318774in}}%
\pgfpathlineto{\pgfqpoint{0.267605in}{0.315287in}}%
\pgfpathlineto{\pgfqpoint{0.264951in}{0.307781in}}%
\pgfpathclose%
\pgfusepath{fill}%
\end{pgfscope}%
\begin{pgfscope}%
\pgfpathrectangle{\pgfqpoint{0.211875in}{0.211875in}}{\pgfqpoint{1.313625in}{1.279725in}}%
\pgfusepath{clip}%
\pgfsetbuttcap%
\pgfsetroundjoin%
\definecolor{currentfill}{rgb}{0.198046,0.094652,0.234785}%
\pgfsetfillcolor{currentfill}%
\pgfsetlinewidth{0.000000pt}%
\definecolor{currentstroke}{rgb}{0.000000,0.000000,0.000000}%
\pgfsetstrokecolor{currentstroke}%
\pgfsetdash{}{0pt}%
\pgfpathmoveto{\pgfqpoint{0.384371in}{0.296812in}}%
\pgfpathlineto{\pgfqpoint{0.391165in}{0.302361in}}%
\pgfpathlineto{\pgfqpoint{0.394303in}{0.315287in}}%
\pgfpathlineto{\pgfqpoint{0.386570in}{0.328214in}}%
\pgfpathlineto{\pgfqpoint{0.384371in}{0.329562in}}%
\pgfpathlineto{\pgfqpoint{0.381338in}{0.328214in}}%
\pgfpathlineto{\pgfqpoint{0.371102in}{0.315489in}}%
\pgfpathlineto{\pgfqpoint{0.371054in}{0.315287in}}%
\pgfpathlineto{\pgfqpoint{0.371102in}{0.314813in}}%
\pgfpathlineto{\pgfqpoint{0.375084in}{0.302361in}}%
\pgfpathclose%
\pgfusepath{fill}%
\end{pgfscope}%
\begin{pgfscope}%
\pgfpathrectangle{\pgfqpoint{0.211875in}{0.211875in}}{\pgfqpoint{1.313625in}{1.279725in}}%
\pgfusepath{clip}%
\pgfsetbuttcap%
\pgfsetroundjoin%
\definecolor{currentfill}{rgb}{0.198046,0.094652,0.234785}%
\pgfsetfillcolor{currentfill}%
\pgfsetlinewidth{0.000000pt}%
\definecolor{currentstroke}{rgb}{0.000000,0.000000,0.000000}%
\pgfsetstrokecolor{currentstroke}%
\pgfsetdash{}{0pt}%
\pgfpathmoveto{\pgfqpoint{0.503792in}{0.302224in}}%
\pgfpathlineto{\pgfqpoint{0.503933in}{0.302361in}}%
\pgfpathlineto{\pgfqpoint{0.507078in}{0.315287in}}%
\pgfpathlineto{\pgfqpoint{0.503792in}{0.321528in}}%
\pgfpathlineto{\pgfqpoint{0.496822in}{0.315287in}}%
\pgfpathlineto{\pgfqpoint{0.503487in}{0.302361in}}%
\pgfpathclose%
\pgfusepath{fill}%
\end{pgfscope}%
\begin{pgfscope}%
\pgfpathrectangle{\pgfqpoint{0.211875in}{0.211875in}}{\pgfqpoint{1.313625in}{1.279725in}}%
\pgfusepath{clip}%
\pgfsetbuttcap%
\pgfsetroundjoin%
\definecolor{currentfill}{rgb}{0.198046,0.094652,0.234785}%
\pgfsetfillcolor{currentfill}%
\pgfsetlinewidth{0.000000pt}%
\definecolor{currentstroke}{rgb}{0.000000,0.000000,0.000000}%
\pgfsetstrokecolor{currentstroke}%
\pgfsetdash{}{0pt}%
\pgfpathmoveto{\pgfqpoint{0.217425in}{0.379920in}}%
\pgfpathlineto{\pgfqpoint{0.221043in}{0.392846in}}%
\pgfpathlineto{\pgfqpoint{0.218589in}{0.405773in}}%
\pgfpathlineto{\pgfqpoint{0.211875in}{0.413894in}}%
\pgfpathlineto{\pgfqpoint{0.211875in}{0.405773in}}%
\pgfpathlineto{\pgfqpoint{0.211875in}{0.392846in}}%
\pgfpathlineto{\pgfqpoint{0.211875in}{0.379920in}}%
\pgfpathlineto{\pgfqpoint{0.211875in}{0.374271in}}%
\pgfpathclose%
\pgfusepath{fill}%
\end{pgfscope}%
\begin{pgfscope}%
\pgfpathrectangle{\pgfqpoint{0.211875in}{0.211875in}}{\pgfqpoint{1.313625in}{1.279725in}}%
\pgfusepath{clip}%
\pgfsetbuttcap%
\pgfsetroundjoin%
\definecolor{currentfill}{rgb}{0.198046,0.094652,0.234785}%
\pgfsetfillcolor{currentfill}%
\pgfsetlinewidth{0.000000pt}%
\definecolor{currentstroke}{rgb}{0.000000,0.000000,0.000000}%
\pgfsetstrokecolor{currentstroke}%
\pgfsetdash{}{0pt}%
\pgfpathmoveto{\pgfqpoint{0.318027in}{0.380894in}}%
\pgfpathlineto{\pgfqpoint{0.331295in}{0.381939in}}%
\pgfpathlineto{\pgfqpoint{0.334636in}{0.392846in}}%
\pgfpathlineto{\pgfqpoint{0.332034in}{0.405773in}}%
\pgfpathlineto{\pgfqpoint{0.331295in}{0.406779in}}%
\pgfpathlineto{\pgfqpoint{0.318027in}{0.407246in}}%
\pgfpathlineto{\pgfqpoint{0.316886in}{0.405773in}}%
\pgfpathlineto{\pgfqpoint{0.314196in}{0.392846in}}%
\pgfpathclose%
\pgfusepath{fill}%
\end{pgfscope}%
\begin{pgfscope}%
\pgfpathrectangle{\pgfqpoint{0.211875in}{0.211875in}}{\pgfqpoint{1.313625in}{1.279725in}}%
\pgfusepath{clip}%
\pgfsetbuttcap%
\pgfsetroundjoin%
\definecolor{currentfill}{rgb}{0.198046,0.094652,0.234785}%
\pgfsetfillcolor{currentfill}%
\pgfsetlinewidth{0.000000pt}%
\definecolor{currentstroke}{rgb}{0.000000,0.000000,0.000000}%
\pgfsetstrokecolor{currentstroke}%
\pgfsetdash{}{0pt}%
\pgfpathmoveto{\pgfqpoint{0.437447in}{0.390974in}}%
\pgfpathlineto{\pgfqpoint{0.441692in}{0.392846in}}%
\pgfpathlineto{\pgfqpoint{0.437447in}{0.395646in}}%
\pgfpathlineto{\pgfqpoint{0.436765in}{0.392846in}}%
\pgfpathclose%
\pgfusepath{fill}%
\end{pgfscope}%
\begin{pgfscope}%
\pgfpathrectangle{\pgfqpoint{0.211875in}{0.211875in}}{\pgfqpoint{1.313625in}{1.279725in}}%
\pgfusepath{clip}%
\pgfsetbuttcap%
\pgfsetroundjoin%
\definecolor{currentfill}{rgb}{0.198046,0.094652,0.234785}%
\pgfsetfillcolor{currentfill}%
\pgfsetlinewidth{0.000000pt}%
\definecolor{currentstroke}{rgb}{0.000000,0.000000,0.000000}%
\pgfsetstrokecolor{currentstroke}%
\pgfsetdash{}{0pt}%
\pgfpathmoveto{\pgfqpoint{0.264951in}{0.461163in}}%
\pgfpathlineto{\pgfqpoint{0.274118in}{0.470405in}}%
\pgfpathlineto{\pgfqpoint{0.272656in}{0.483332in}}%
\pgfpathlineto{\pgfqpoint{0.264951in}{0.489578in}}%
\pgfpathlineto{\pgfqpoint{0.258081in}{0.483332in}}%
\pgfpathlineto{\pgfqpoint{0.256794in}{0.470405in}}%
\pgfpathclose%
\pgfusepath{fill}%
\end{pgfscope}%
\begin{pgfscope}%
\pgfpathrectangle{\pgfqpoint{0.211875in}{0.211875in}}{\pgfqpoint{1.313625in}{1.279725in}}%
\pgfusepath{clip}%
\pgfsetbuttcap%
\pgfsetroundjoin%
\definecolor{currentfill}{rgb}{0.198046,0.094652,0.234785}%
\pgfsetfillcolor{currentfill}%
\pgfsetlinewidth{0.000000pt}%
\definecolor{currentstroke}{rgb}{0.000000,0.000000,0.000000}%
\pgfsetstrokecolor{currentstroke}%
\pgfsetdash{}{0pt}%
\pgfpathmoveto{\pgfqpoint{0.384371in}{0.469646in}}%
\pgfpathlineto{\pgfqpoint{0.384988in}{0.470405in}}%
\pgfpathlineto{\pgfqpoint{0.384371in}{0.476648in}}%
\pgfpathlineto{\pgfqpoint{0.383537in}{0.470405in}}%
\pgfpathclose%
\pgfusepath{fill}%
\end{pgfscope}%
\begin{pgfscope}%
\pgfpathrectangle{\pgfqpoint{0.211875in}{0.211875in}}{\pgfqpoint{1.313625in}{1.279725in}}%
\pgfusepath{clip}%
\pgfsetbuttcap%
\pgfsetroundjoin%
\definecolor{currentfill}{rgb}{0.198046,0.094652,0.234785}%
\pgfsetfillcolor{currentfill}%
\pgfsetlinewidth{0.000000pt}%
\definecolor{currentstroke}{rgb}{0.000000,0.000000,0.000000}%
\pgfsetstrokecolor{currentstroke}%
\pgfsetdash{}{0pt}%
\pgfpathmoveto{\pgfqpoint{0.214453in}{0.547964in}}%
\pgfpathlineto{\pgfqpoint{0.215706in}{0.560891in}}%
\pgfpathlineto{\pgfqpoint{0.211875in}{0.569049in}}%
\pgfpathlineto{\pgfqpoint{0.211875in}{0.560891in}}%
\pgfpathlineto{\pgfqpoint{0.211875in}{0.547964in}}%
\pgfpathlineto{\pgfqpoint{0.211875in}{0.544228in}}%
\pgfpathclose%
\pgfusepath{fill}%
\end{pgfscope}%
\begin{pgfscope}%
\pgfpathrectangle{\pgfqpoint{0.211875in}{0.211875in}}{\pgfqpoint{1.313625in}{1.279725in}}%
\pgfusepath{clip}%
\pgfsetbuttcap%
\pgfsetroundjoin%
\definecolor{currentfill}{rgb}{0.198046,0.094652,0.234785}%
\pgfsetfillcolor{currentfill}%
\pgfsetlinewidth{0.000000pt}%
\definecolor{currentstroke}{rgb}{0.000000,0.000000,0.000000}%
\pgfsetstrokecolor{currentstroke}%
\pgfsetdash{}{0pt}%
\pgfpathmoveto{\pgfqpoint{0.264951in}{0.634779in}}%
\pgfpathlineto{\pgfqpoint{0.266892in}{0.638450in}}%
\pgfpathlineto{\pgfqpoint{0.264951in}{0.641763in}}%
\pgfpathlineto{\pgfqpoint{0.263230in}{0.638450in}}%
\pgfpathclose%
\pgfusepath{fill}%
\end{pgfscope}%
\begin{pgfscope}%
\pgfpathrectangle{\pgfqpoint{0.211875in}{0.211875in}}{\pgfqpoint{1.313625in}{1.279725in}}%
\pgfusepath{clip}%
\pgfsetbuttcap%
\pgfsetroundjoin%
\definecolor{currentfill}{rgb}{0.198046,0.094652,0.234785}%
\pgfsetfillcolor{currentfill}%
\pgfsetlinewidth{0.000000pt}%
\definecolor{currentstroke}{rgb}{0.000000,0.000000,0.000000}%
\pgfsetstrokecolor{currentstroke}%
\pgfsetdash{}{0pt}%
\pgfpathmoveto{\pgfqpoint{0.212037in}{0.716009in}}%
\pgfpathlineto{\pgfqpoint{0.211875in}{0.717176in}}%
\pgfpathlineto{\pgfqpoint{0.211875in}{0.716009in}}%
\pgfpathlineto{\pgfqpoint{0.211875in}{0.715639in}}%
\pgfpathclose%
\pgfusepath{fill}%
\end{pgfscope}%
\begin{pgfscope}%
\pgfpathrectangle{\pgfqpoint{0.211875in}{0.211875in}}{\pgfqpoint{1.313625in}{1.279725in}}%
\pgfusepath{clip}%
\pgfsetbuttcap%
\pgfsetroundjoin%
\definecolor{currentfill}{rgb}{0.343142,0.118134,0.311397}%
\pgfsetfillcolor{currentfill}%
\pgfsetlinewidth{0.000000pt}%
\definecolor{currentstroke}{rgb}{0.000000,0.000000,0.000000}%
\pgfsetstrokecolor{currentstroke}%
\pgfsetdash{}{0pt}%
\pgfpathmoveto{\pgfqpoint{0.225144in}{0.211875in}}%
\pgfpathlineto{\pgfqpoint{0.231301in}{0.211875in}}%
\pgfpathlineto{\pgfqpoint{0.232252in}{0.224802in}}%
\pgfpathlineto{\pgfqpoint{0.232006in}{0.237728in}}%
\pgfpathlineto{\pgfqpoint{0.230413in}{0.250655in}}%
\pgfpathlineto{\pgfqpoint{0.225144in}{0.261817in}}%
\pgfpathlineto{\pgfqpoint{0.222425in}{0.263581in}}%
\pgfpathlineto{\pgfqpoint{0.211875in}{0.266982in}}%
\pgfpathlineto{\pgfqpoint{0.211875in}{0.263581in}}%
\pgfpathlineto{\pgfqpoint{0.211875in}{0.258136in}}%
\pgfpathlineto{\pgfqpoint{0.221483in}{0.250655in}}%
\pgfpathlineto{\pgfqpoint{0.225144in}{0.242322in}}%
\pgfpathlineto{\pgfqpoint{0.226113in}{0.237728in}}%
\pgfpathlineto{\pgfqpoint{0.226216in}{0.224802in}}%
\pgfpathlineto{\pgfqpoint{0.225144in}{0.219402in}}%
\pgfpathlineto{\pgfqpoint{0.221837in}{0.211875in}}%
\pgfpathclose%
\pgfusepath{fill}%
\end{pgfscope}%
\begin{pgfscope}%
\pgfpathrectangle{\pgfqpoint{0.211875in}{0.211875in}}{\pgfqpoint{1.313625in}{1.279725in}}%
\pgfusepath{clip}%
\pgfsetbuttcap%
\pgfsetroundjoin%
\definecolor{currentfill}{rgb}{0.343142,0.118134,0.311397}%
\pgfsetfillcolor{currentfill}%
\pgfsetlinewidth{0.000000pt}%
\definecolor{currentstroke}{rgb}{0.000000,0.000000,0.000000}%
\pgfsetstrokecolor{currentstroke}%
\pgfsetdash{}{0pt}%
\pgfpathmoveto{\pgfqpoint{0.304758in}{0.211875in}}%
\pgfpathlineto{\pgfqpoint{0.314950in}{0.211875in}}%
\pgfpathlineto{\pgfqpoint{0.308715in}{0.224802in}}%
\pgfpathlineto{\pgfqpoint{0.308799in}{0.237728in}}%
\pgfpathlineto{\pgfqpoint{0.314860in}{0.250655in}}%
\pgfpathlineto{\pgfqpoint{0.318027in}{0.253245in}}%
\pgfpathlineto{\pgfqpoint{0.331295in}{0.252854in}}%
\pgfpathlineto{\pgfqpoint{0.333809in}{0.250655in}}%
\pgfpathlineto{\pgfqpoint{0.339743in}{0.237728in}}%
\pgfpathlineto{\pgfqpoint{0.339796in}{0.224802in}}%
\pgfpathlineto{\pgfqpoint{0.333649in}{0.211875in}}%
\pgfpathlineto{\pgfqpoint{0.344564in}{0.211875in}}%
\pgfpathlineto{\pgfqpoint{0.345410in}{0.211875in}}%
\pgfpathlineto{\pgfqpoint{0.347372in}{0.224802in}}%
\pgfpathlineto{\pgfqpoint{0.347197in}{0.237728in}}%
\pgfpathlineto{\pgfqpoint{0.344854in}{0.250655in}}%
\pgfpathlineto{\pgfqpoint{0.344564in}{0.251355in}}%
\pgfpathlineto{\pgfqpoint{0.331295in}{0.262647in}}%
\pgfpathlineto{\pgfqpoint{0.318027in}{0.263090in}}%
\pgfpathlineto{\pgfqpoint{0.304758in}{0.253611in}}%
\pgfpathlineto{\pgfqpoint{0.303494in}{0.250655in}}%
\pgfpathlineto{\pgfqpoint{0.301324in}{0.237728in}}%
\pgfpathlineto{\pgfqpoint{0.301114in}{0.224802in}}%
\pgfpathlineto{\pgfqpoint{0.302784in}{0.211875in}}%
\pgfpathclose%
\pgfusepath{fill}%
\end{pgfscope}%
\begin{pgfscope}%
\pgfpathrectangle{\pgfqpoint{0.211875in}{0.211875in}}{\pgfqpoint{1.313625in}{1.279725in}}%
\pgfusepath{clip}%
\pgfsetbuttcap%
\pgfsetroundjoin%
\definecolor{currentfill}{rgb}{0.343142,0.118134,0.311397}%
\pgfsetfillcolor{currentfill}%
\pgfsetlinewidth{0.000000pt}%
\definecolor{currentstroke}{rgb}{0.000000,0.000000,0.000000}%
\pgfsetstrokecolor{currentstroke}%
\pgfsetdash{}{0pt}%
\pgfpathmoveto{\pgfqpoint{0.424178in}{0.213579in}}%
\pgfpathlineto{\pgfqpoint{0.424975in}{0.211875in}}%
\pgfpathlineto{\pgfqpoint{0.437447in}{0.211875in}}%
\pgfpathlineto{\pgfqpoint{0.450716in}{0.211875in}}%
\pgfpathlineto{\pgfqpoint{0.458535in}{0.211875in}}%
\pgfpathlineto{\pgfqpoint{0.462504in}{0.224802in}}%
\pgfpathlineto{\pgfqpoint{0.462342in}{0.237728in}}%
\pgfpathlineto{\pgfqpoint{0.458148in}{0.250655in}}%
\pgfpathlineto{\pgfqpoint{0.450716in}{0.257914in}}%
\pgfpathlineto{\pgfqpoint{0.437447in}{0.259206in}}%
\pgfpathlineto{\pgfqpoint{0.425612in}{0.250655in}}%
\pgfpathlineto{\pgfqpoint{0.424178in}{0.247577in}}%
\pgfpathlineto{\pgfqpoint{0.422051in}{0.237728in}}%
\pgfpathlineto{\pgfqpoint{0.421907in}{0.224802in}}%
\pgfpathclose%
\pgfpathmoveto{\pgfqpoint{0.430908in}{0.224802in}}%
\pgfpathlineto{\pgfqpoint{0.430885in}{0.237728in}}%
\pgfpathlineto{\pgfqpoint{0.437447in}{0.248775in}}%
\pgfpathlineto{\pgfqpoint{0.450716in}{0.245549in}}%
\pgfpathlineto{\pgfqpoint{0.454219in}{0.237728in}}%
\pgfpathlineto{\pgfqpoint{0.454186in}{0.224802in}}%
\pgfpathlineto{\pgfqpoint{0.450716in}{0.217350in}}%
\pgfpathlineto{\pgfqpoint{0.437447in}{0.214316in}}%
\pgfpathclose%
\pgfusepath{fill}%
\end{pgfscope}%
\begin{pgfscope}%
\pgfpathrectangle{\pgfqpoint{0.211875in}{0.211875in}}{\pgfqpoint{1.313625in}{1.279725in}}%
\pgfusepath{clip}%
\pgfsetbuttcap%
\pgfsetroundjoin%
\definecolor{currentfill}{rgb}{0.343142,0.118134,0.311397}%
\pgfsetfillcolor{currentfill}%
\pgfsetlinewidth{0.000000pt}%
\definecolor{currentstroke}{rgb}{0.000000,0.000000,0.000000}%
\pgfsetstrokecolor{currentstroke}%
\pgfsetdash{}{0pt}%
\pgfpathmoveto{\pgfqpoint{0.543598in}{0.219545in}}%
\pgfpathlineto{\pgfqpoint{0.548173in}{0.211875in}}%
\pgfpathlineto{\pgfqpoint{0.556867in}{0.211875in}}%
\pgfpathlineto{\pgfqpoint{0.570136in}{0.211875in}}%
\pgfpathlineto{\pgfqpoint{0.572750in}{0.211875in}}%
\pgfpathlineto{\pgfqpoint{0.577438in}{0.224802in}}%
\pgfpathlineto{\pgfqpoint{0.577351in}{0.237728in}}%
\pgfpathlineto{\pgfqpoint{0.572698in}{0.250655in}}%
\pgfpathlineto{\pgfqpoint{0.570136in}{0.253436in}}%
\pgfpathlineto{\pgfqpoint{0.556867in}{0.256029in}}%
\pgfpathlineto{\pgfqpoint{0.548349in}{0.250655in}}%
\pgfpathlineto{\pgfqpoint{0.543598in}{0.242434in}}%
\pgfpathlineto{\pgfqpoint{0.542455in}{0.237728in}}%
\pgfpathlineto{\pgfqpoint{0.542368in}{0.224802in}}%
\pgfpathclose%
\pgfpathmoveto{\pgfqpoint{0.553156in}{0.224802in}}%
\pgfpathlineto{\pgfqpoint{0.553022in}{0.237728in}}%
\pgfpathlineto{\pgfqpoint{0.556867in}{0.243382in}}%
\pgfpathlineto{\pgfqpoint{0.568455in}{0.237728in}}%
\pgfpathlineto{\pgfqpoint{0.568016in}{0.224802in}}%
\pgfpathlineto{\pgfqpoint{0.556867in}{0.219619in}}%
\pgfpathclose%
\pgfusepath{fill}%
\end{pgfscope}%
\begin{pgfscope}%
\pgfpathrectangle{\pgfqpoint{0.211875in}{0.211875in}}{\pgfqpoint{1.313625in}{1.279725in}}%
\pgfusepath{clip}%
\pgfsetbuttcap%
\pgfsetroundjoin%
\definecolor{currentfill}{rgb}{0.343142,0.118134,0.311397}%
\pgfsetfillcolor{currentfill}%
\pgfsetlinewidth{0.000000pt}%
\definecolor{currentstroke}{rgb}{0.000000,0.000000,0.000000}%
\pgfsetstrokecolor{currentstroke}%
\pgfsetdash{}{0pt}%
\pgfpathmoveto{\pgfqpoint{0.663019in}{0.222883in}}%
\pgfpathlineto{\pgfqpoint{0.671410in}{0.211875in}}%
\pgfpathlineto{\pgfqpoint{0.676288in}{0.211875in}}%
\pgfpathlineto{\pgfqpoint{0.684822in}{0.211875in}}%
\pgfpathlineto{\pgfqpoint{0.689557in}{0.214704in}}%
\pgfpathlineto{\pgfqpoint{0.693314in}{0.224802in}}%
\pgfpathlineto{\pgfqpoint{0.693282in}{0.237728in}}%
\pgfpathlineto{\pgfqpoint{0.689557in}{0.247982in}}%
\pgfpathlineto{\pgfqpoint{0.685440in}{0.250655in}}%
\pgfpathlineto{\pgfqpoint{0.676288in}{0.253461in}}%
\pgfpathlineto{\pgfqpoint{0.671109in}{0.250655in}}%
\pgfpathlineto{\pgfqpoint{0.663019in}{0.239470in}}%
\pgfpathlineto{\pgfqpoint{0.662546in}{0.237728in}}%
\pgfpathlineto{\pgfqpoint{0.662506in}{0.224802in}}%
\pgfpathclose%
\pgfpathmoveto{\pgfqpoint{0.675563in}{0.224802in}}%
\pgfpathlineto{\pgfqpoint{0.675307in}{0.237728in}}%
\pgfpathlineto{\pgfqpoint{0.676288in}{0.238966in}}%
\pgfpathlineto{\pgfqpoint{0.678062in}{0.237728in}}%
\pgfpathlineto{\pgfqpoint{0.677597in}{0.224802in}}%
\pgfpathlineto{\pgfqpoint{0.676288in}{0.223934in}}%
\pgfpathclose%
\pgfusepath{fill}%
\end{pgfscope}%
\begin{pgfscope}%
\pgfpathrectangle{\pgfqpoint{0.211875in}{0.211875in}}{\pgfqpoint{1.313625in}{1.279725in}}%
\pgfusepath{clip}%
\pgfsetbuttcap%
\pgfsetroundjoin%
\definecolor{currentfill}{rgb}{0.343142,0.118134,0.311397}%
\pgfsetfillcolor{currentfill}%
\pgfsetlinewidth{0.000000pt}%
\definecolor{currentstroke}{rgb}{0.000000,0.000000,0.000000}%
\pgfsetstrokecolor{currentstroke}%
\pgfsetdash{}{0pt}%
\pgfpathmoveto{\pgfqpoint{0.782439in}{0.224405in}}%
\pgfpathlineto{\pgfqpoint{0.794821in}{0.211875in}}%
\pgfpathlineto{\pgfqpoint{0.795708in}{0.211875in}}%
\pgfpathlineto{\pgfqpoint{0.796723in}{0.211875in}}%
\pgfpathlineto{\pgfqpoint{0.808977in}{0.221952in}}%
\pgfpathlineto{\pgfqpoint{0.809924in}{0.224802in}}%
\pgfpathlineto{\pgfqpoint{0.809931in}{0.237728in}}%
\pgfpathlineto{\pgfqpoint{0.808977in}{0.240637in}}%
\pgfpathlineto{\pgfqpoint{0.797675in}{0.250655in}}%
\pgfpathlineto{\pgfqpoint{0.795708in}{0.251431in}}%
\pgfpathlineto{\pgfqpoint{0.794000in}{0.250655in}}%
\pgfpathlineto{\pgfqpoint{0.782439in}{0.238121in}}%
\pgfpathlineto{\pgfqpoint{0.782321in}{0.237728in}}%
\pgfpathlineto{\pgfqpoint{0.782320in}{0.224802in}}%
\pgfpathclose%
\pgfusepath{fill}%
\end{pgfscope}%
\begin{pgfscope}%
\pgfpathrectangle{\pgfqpoint{0.211875in}{0.211875in}}{\pgfqpoint{1.313625in}{1.279725in}}%
\pgfusepath{clip}%
\pgfsetbuttcap%
\pgfsetroundjoin%
\definecolor{currentfill}{rgb}{0.343142,0.118134,0.311397}%
\pgfsetfillcolor{currentfill}%
\pgfsetlinewidth{0.000000pt}%
\definecolor{currentstroke}{rgb}{0.000000,0.000000,0.000000}%
\pgfsetstrokecolor{currentstroke}%
\pgfsetdash{}{0pt}%
\pgfpathmoveto{\pgfqpoint{0.901860in}{0.224597in}}%
\pgfpathlineto{\pgfqpoint{0.915129in}{0.213928in}}%
\pgfpathlineto{\pgfqpoint{0.925476in}{0.224802in}}%
\pgfpathlineto{\pgfqpoint{0.925548in}{0.237728in}}%
\pgfpathlineto{\pgfqpoint{0.915129in}{0.249276in}}%
\pgfpathlineto{\pgfqpoint{0.901860in}{0.238021in}}%
\pgfpathlineto{\pgfqpoint{0.901762in}{0.237728in}}%
\pgfpathlineto{\pgfqpoint{0.901790in}{0.224802in}}%
\pgfpathclose%
\pgfusepath{fill}%
\end{pgfscope}%
\begin{pgfscope}%
\pgfpathrectangle{\pgfqpoint{0.211875in}{0.211875in}}{\pgfqpoint{1.313625in}{1.279725in}}%
\pgfusepath{clip}%
\pgfsetbuttcap%
\pgfsetroundjoin%
\definecolor{currentfill}{rgb}{0.343142,0.118134,0.311397}%
\pgfsetfillcolor{currentfill}%
\pgfsetlinewidth{0.000000pt}%
\definecolor{currentstroke}{rgb}{0.000000,0.000000,0.000000}%
\pgfsetstrokecolor{currentstroke}%
\pgfsetdash{}{0pt}%
\pgfpathmoveto{\pgfqpoint{1.021280in}{0.223754in}}%
\pgfpathlineto{\pgfqpoint{1.034549in}{0.215803in}}%
\pgfpathlineto{\pgfqpoint{1.041768in}{0.224802in}}%
\pgfpathlineto{\pgfqpoint{1.041858in}{0.237728in}}%
\pgfpathlineto{\pgfqpoint{1.034549in}{0.247330in}}%
\pgfpathlineto{\pgfqpoint{1.021280in}{0.238935in}}%
\pgfpathlineto{\pgfqpoint{1.020832in}{0.237728in}}%
\pgfpathlineto{\pgfqpoint{1.020881in}{0.224802in}}%
\pgfpathclose%
\pgfusepath{fill}%
\end{pgfscope}%
\begin{pgfscope}%
\pgfpathrectangle{\pgfqpoint{0.211875in}{0.211875in}}{\pgfqpoint{1.313625in}{1.279725in}}%
\pgfusepath{clip}%
\pgfsetbuttcap%
\pgfsetroundjoin%
\definecolor{currentfill}{rgb}{0.343142,0.118134,0.311397}%
\pgfsetfillcolor{currentfill}%
\pgfsetlinewidth{0.000000pt}%
\definecolor{currentstroke}{rgb}{0.000000,0.000000,0.000000}%
\pgfsetstrokecolor{currentstroke}%
\pgfsetdash{}{0pt}%
\pgfpathmoveto{\pgfqpoint{1.140701in}{0.222059in}}%
\pgfpathlineto{\pgfqpoint{1.153970in}{0.216878in}}%
\pgfpathlineto{\pgfqpoint{1.159445in}{0.224802in}}%
\pgfpathlineto{\pgfqpoint{1.159531in}{0.237728in}}%
\pgfpathlineto{\pgfqpoint{1.153970in}{0.246195in}}%
\pgfpathlineto{\pgfqpoint{1.140701in}{0.240708in}}%
\pgfpathlineto{\pgfqpoint{1.139471in}{0.237728in}}%
\pgfpathlineto{\pgfqpoint{1.139531in}{0.224802in}}%
\pgfpathclose%
\pgfusepath{fill}%
\end{pgfscope}%
\begin{pgfscope}%
\pgfpathrectangle{\pgfqpoint{0.211875in}{0.211875in}}{\pgfqpoint{1.313625in}{1.279725in}}%
\pgfusepath{clip}%
\pgfsetbuttcap%
\pgfsetroundjoin%
\definecolor{currentfill}{rgb}{0.343142,0.118134,0.311397}%
\pgfsetfillcolor{currentfill}%
\pgfsetlinewidth{0.000000pt}%
\definecolor{currentstroke}{rgb}{0.000000,0.000000,0.000000}%
\pgfsetstrokecolor{currentstroke}%
\pgfsetdash{}{0pt}%
\pgfpathmoveto{\pgfqpoint{1.260121in}{0.219619in}}%
\pgfpathlineto{\pgfqpoint{1.273390in}{0.217138in}}%
\pgfpathlineto{\pgfqpoint{1.278019in}{0.224802in}}%
\pgfpathlineto{\pgfqpoint{1.278086in}{0.237728in}}%
\pgfpathlineto{\pgfqpoint{1.273390in}{0.245884in}}%
\pgfpathlineto{\pgfqpoint{1.260121in}{0.243247in}}%
\pgfpathlineto{\pgfqpoint{1.257585in}{0.237728in}}%
\pgfpathlineto{\pgfqpoint{1.257644in}{0.224802in}}%
\pgfpathclose%
\pgfusepath{fill}%
\end{pgfscope}%
\begin{pgfscope}%
\pgfpathrectangle{\pgfqpoint{0.211875in}{0.211875in}}{\pgfqpoint{1.313625in}{1.279725in}}%
\pgfusepath{clip}%
\pgfsetbuttcap%
\pgfsetroundjoin%
\definecolor{currentfill}{rgb}{0.343142,0.118134,0.311397}%
\pgfsetfillcolor{currentfill}%
\pgfsetlinewidth{0.000000pt}%
\definecolor{currentstroke}{rgb}{0.000000,0.000000,0.000000}%
\pgfsetstrokecolor{currentstroke}%
\pgfsetdash{}{0pt}%
\pgfpathmoveto{\pgfqpoint{1.379542in}{0.216489in}}%
\pgfpathlineto{\pgfqpoint{1.392811in}{0.216529in}}%
\pgfpathlineto{\pgfqpoint{1.397221in}{0.224802in}}%
\pgfpathlineto{\pgfqpoint{1.397259in}{0.237728in}}%
\pgfpathlineto{\pgfqpoint{1.392811in}{0.246445in}}%
\pgfpathlineto{\pgfqpoint{1.379542in}{0.246502in}}%
\pgfpathlineto{\pgfqpoint{1.375027in}{0.237728in}}%
\pgfpathlineto{\pgfqpoint{1.375071in}{0.224802in}}%
\pgfpathclose%
\pgfusepath{fill}%
\end{pgfscope}%
\begin{pgfscope}%
\pgfpathrectangle{\pgfqpoint{0.211875in}{0.211875in}}{\pgfqpoint{1.313625in}{1.279725in}}%
\pgfusepath{clip}%
\pgfsetbuttcap%
\pgfsetroundjoin%
\definecolor{currentfill}{rgb}{0.343142,0.118134,0.311397}%
\pgfsetfillcolor{currentfill}%
\pgfsetlinewidth{0.000000pt}%
\definecolor{currentstroke}{rgb}{0.000000,0.000000,0.000000}%
\pgfsetstrokecolor{currentstroke}%
\pgfsetdash{}{0pt}%
\pgfpathmoveto{\pgfqpoint{1.498962in}{0.212686in}}%
\pgfpathlineto{\pgfqpoint{1.512231in}{0.214950in}}%
\pgfpathlineto{\pgfqpoint{1.516897in}{0.224802in}}%
\pgfpathlineto{\pgfqpoint{1.516896in}{0.237728in}}%
\pgfpathlineto{\pgfqpoint{1.512231in}{0.247964in}}%
\pgfpathlineto{\pgfqpoint{1.498962in}{0.250457in}}%
\pgfpathlineto{\pgfqpoint{1.491571in}{0.237728in}}%
\pgfpathlineto{\pgfqpoint{1.491580in}{0.224802in}}%
\pgfpathclose%
\pgfusepath{fill}%
\end{pgfscope}%
\begin{pgfscope}%
\pgfpathrectangle{\pgfqpoint{0.211875in}{0.211875in}}{\pgfqpoint{1.313625in}{1.279725in}}%
\pgfusepath{clip}%
\pgfsetbuttcap%
\pgfsetroundjoin%
\definecolor{currentfill}{rgb}{0.343142,0.118134,0.311397}%
\pgfsetfillcolor{currentfill}%
\pgfsetlinewidth{0.000000pt}%
\definecolor{currentstroke}{rgb}{0.000000,0.000000,0.000000}%
\pgfsetstrokecolor{currentstroke}%
\pgfsetdash{}{0pt}%
\pgfpathmoveto{\pgfqpoint{0.251682in}{0.284301in}}%
\pgfpathlineto{\pgfqpoint{0.264951in}{0.281569in}}%
\pgfpathlineto{\pgfqpoint{0.278220in}{0.284600in}}%
\pgfpathlineto{\pgfqpoint{0.282960in}{0.289434in}}%
\pgfpathlineto{\pgfqpoint{0.286998in}{0.302361in}}%
\pgfpathlineto{\pgfqpoint{0.287475in}{0.315287in}}%
\pgfpathlineto{\pgfqpoint{0.285588in}{0.328214in}}%
\pgfpathlineto{\pgfqpoint{0.278220in}{0.340563in}}%
\pgfpathlineto{\pgfqpoint{0.276726in}{0.341140in}}%
\pgfpathlineto{\pgfqpoint{0.264951in}{0.343897in}}%
\pgfpathlineto{\pgfqpoint{0.252869in}{0.341140in}}%
\pgfpathlineto{\pgfqpoint{0.251682in}{0.340658in}}%
\pgfpathlineto{\pgfqpoint{0.244730in}{0.328214in}}%
\pgfpathlineto{\pgfqpoint{0.243042in}{0.315287in}}%
\pgfpathlineto{\pgfqpoint{0.243395in}{0.302361in}}%
\pgfpathlineto{\pgfqpoint{0.246873in}{0.289434in}}%
\pgfpathclose%
\pgfpathmoveto{\pgfqpoint{0.250716in}{0.302361in}}%
\pgfpathlineto{\pgfqpoint{0.249659in}{0.315287in}}%
\pgfpathlineto{\pgfqpoint{0.251682in}{0.324723in}}%
\pgfpathlineto{\pgfqpoint{0.253801in}{0.328214in}}%
\pgfpathlineto{\pgfqpoint{0.264951in}{0.334391in}}%
\pgfpathlineto{\pgfqpoint{0.277212in}{0.328214in}}%
\pgfpathlineto{\pgfqpoint{0.278220in}{0.326810in}}%
\pgfpathlineto{\pgfqpoint{0.280869in}{0.315287in}}%
\pgfpathlineto{\pgfqpoint{0.279732in}{0.302361in}}%
\pgfpathlineto{\pgfqpoint{0.278220in}{0.298894in}}%
\pgfpathlineto{\pgfqpoint{0.264951in}{0.290769in}}%
\pgfpathlineto{\pgfqpoint{0.251682in}{0.299961in}}%
\pgfpathclose%
\pgfusepath{fill}%
\end{pgfscope}%
\begin{pgfscope}%
\pgfpathrectangle{\pgfqpoint{0.211875in}{0.211875in}}{\pgfqpoint{1.313625in}{1.279725in}}%
\pgfusepath{clip}%
\pgfsetbuttcap%
\pgfsetroundjoin%
\definecolor{currentfill}{rgb}{0.343142,0.118134,0.311397}%
\pgfsetfillcolor{currentfill}%
\pgfsetlinewidth{0.000000pt}%
\definecolor{currentstroke}{rgb}{0.000000,0.000000,0.000000}%
\pgfsetstrokecolor{currentstroke}%
\pgfsetdash{}{0pt}%
\pgfpathmoveto{\pgfqpoint{0.371102in}{0.288859in}}%
\pgfpathlineto{\pgfqpoint{0.384371in}{0.285650in}}%
\pgfpathlineto{\pgfqpoint{0.394594in}{0.289434in}}%
\pgfpathlineto{\pgfqpoint{0.397640in}{0.291740in}}%
\pgfpathlineto{\pgfqpoint{0.401776in}{0.302361in}}%
\pgfpathlineto{\pgfqpoint{0.402520in}{0.315287in}}%
\pgfpathlineto{\pgfqpoint{0.400214in}{0.328214in}}%
\pgfpathlineto{\pgfqpoint{0.397640in}{0.333002in}}%
\pgfpathlineto{\pgfqpoint{0.384371in}{0.339942in}}%
\pgfpathlineto{\pgfqpoint{0.371102in}{0.335590in}}%
\pgfpathlineto{\pgfqpoint{0.366515in}{0.328214in}}%
\pgfpathlineto{\pgfqpoint{0.364155in}{0.315287in}}%
\pgfpathlineto{\pgfqpoint{0.364870in}{0.302361in}}%
\pgfpathlineto{\pgfqpoint{0.370499in}{0.289434in}}%
\pgfpathclose%
\pgfpathmoveto{\pgfqpoint{0.375084in}{0.302361in}}%
\pgfpathlineto{\pgfqpoint{0.371102in}{0.314813in}}%
\pgfpathlineto{\pgfqpoint{0.371054in}{0.315287in}}%
\pgfpathlineto{\pgfqpoint{0.371102in}{0.315489in}}%
\pgfpathlineto{\pgfqpoint{0.381338in}{0.328214in}}%
\pgfpathlineto{\pgfqpoint{0.384371in}{0.329562in}}%
\pgfpathlineto{\pgfqpoint{0.386570in}{0.328214in}}%
\pgfpathlineto{\pgfqpoint{0.394303in}{0.315287in}}%
\pgfpathlineto{\pgfqpoint{0.391165in}{0.302361in}}%
\pgfpathlineto{\pgfqpoint{0.384371in}{0.296812in}}%
\pgfpathclose%
\pgfusepath{fill}%
\end{pgfscope}%
\begin{pgfscope}%
\pgfpathrectangle{\pgfqpoint{0.211875in}{0.211875in}}{\pgfqpoint{1.313625in}{1.279725in}}%
\pgfusepath{clip}%
\pgfsetbuttcap%
\pgfsetroundjoin%
\definecolor{currentfill}{rgb}{0.343142,0.118134,0.311397}%
\pgfsetfillcolor{currentfill}%
\pgfsetlinewidth{0.000000pt}%
\definecolor{currentstroke}{rgb}{0.000000,0.000000,0.000000}%
\pgfsetstrokecolor{currentstroke}%
\pgfsetdash{}{0pt}%
\pgfpathmoveto{\pgfqpoint{0.503792in}{0.289246in}}%
\pgfpathlineto{\pgfqpoint{0.504226in}{0.289434in}}%
\pgfpathlineto{\pgfqpoint{0.517061in}{0.301762in}}%
\pgfpathlineto{\pgfqpoint{0.517268in}{0.302361in}}%
\pgfpathlineto{\pgfqpoint{0.518226in}{0.315287in}}%
\pgfpathlineto{\pgfqpoint{0.517061in}{0.321499in}}%
\pgfpathlineto{\pgfqpoint{0.514097in}{0.328214in}}%
\pgfpathlineto{\pgfqpoint{0.503792in}{0.335670in}}%
\pgfpathlineto{\pgfqpoint{0.490523in}{0.331810in}}%
\pgfpathlineto{\pgfqpoint{0.488036in}{0.328214in}}%
\pgfpathlineto{\pgfqpoint{0.485016in}{0.315287in}}%
\pgfpathlineto{\pgfqpoint{0.486078in}{0.302361in}}%
\pgfpathlineto{\pgfqpoint{0.490523in}{0.293584in}}%
\pgfpathlineto{\pgfqpoint{0.502791in}{0.289434in}}%
\pgfpathclose%
\pgfpathmoveto{\pgfqpoint{0.503487in}{0.302361in}}%
\pgfpathlineto{\pgfqpoint{0.496822in}{0.315287in}}%
\pgfpathlineto{\pgfqpoint{0.503792in}{0.321528in}}%
\pgfpathlineto{\pgfqpoint{0.507078in}{0.315287in}}%
\pgfpathlineto{\pgfqpoint{0.503933in}{0.302361in}}%
\pgfpathlineto{\pgfqpoint{0.503792in}{0.302224in}}%
\pgfpathclose%
\pgfusepath{fill}%
\end{pgfscope}%
\begin{pgfscope}%
\pgfpathrectangle{\pgfqpoint{0.211875in}{0.211875in}}{\pgfqpoint{1.313625in}{1.279725in}}%
\pgfusepath{clip}%
\pgfsetbuttcap%
\pgfsetroundjoin%
\definecolor{currentfill}{rgb}{0.343142,0.118134,0.311397}%
\pgfsetfillcolor{currentfill}%
\pgfsetlinewidth{0.000000pt}%
\definecolor{currentstroke}{rgb}{0.000000,0.000000,0.000000}%
\pgfsetstrokecolor{currentstroke}%
\pgfsetdash{}{0pt}%
\pgfpathmoveto{\pgfqpoint{0.609943in}{0.297214in}}%
\pgfpathlineto{\pgfqpoint{0.623212in}{0.293862in}}%
\pgfpathlineto{\pgfqpoint{0.630822in}{0.302361in}}%
\pgfpathlineto{\pgfqpoint{0.632850in}{0.315287in}}%
\pgfpathlineto{\pgfqpoint{0.627612in}{0.328214in}}%
\pgfpathlineto{\pgfqpoint{0.623212in}{0.331891in}}%
\pgfpathlineto{\pgfqpoint{0.609943in}{0.329043in}}%
\pgfpathlineto{\pgfqpoint{0.609305in}{0.328214in}}%
\pgfpathlineto{\pgfqpoint{0.605624in}{0.315287in}}%
\pgfpathlineto{\pgfqpoint{0.607026in}{0.302361in}}%
\pgfpathclose%
\pgfusepath{fill}%
\end{pgfscope}%
\begin{pgfscope}%
\pgfpathrectangle{\pgfqpoint{0.211875in}{0.211875in}}{\pgfqpoint{1.313625in}{1.279725in}}%
\pgfusepath{clip}%
\pgfsetbuttcap%
\pgfsetroundjoin%
\definecolor{currentfill}{rgb}{0.343142,0.118134,0.311397}%
\pgfsetfillcolor{currentfill}%
\pgfsetlinewidth{0.000000pt}%
\definecolor{currentstroke}{rgb}{0.000000,0.000000,0.000000}%
\pgfsetstrokecolor{currentstroke}%
\pgfsetdash{}{0pt}%
\pgfpathmoveto{\pgfqpoint{0.729364in}{0.299749in}}%
\pgfpathlineto{\pgfqpoint{0.742633in}{0.298015in}}%
\pgfpathlineto{\pgfqpoint{0.746042in}{0.302361in}}%
\pgfpathlineto{\pgfqpoint{0.748104in}{0.315287in}}%
\pgfpathlineto{\pgfqpoint{0.742987in}{0.328214in}}%
\pgfpathlineto{\pgfqpoint{0.742633in}{0.328551in}}%
\pgfpathlineto{\pgfqpoint{0.739414in}{0.328214in}}%
\pgfpathlineto{\pgfqpoint{0.729364in}{0.325725in}}%
\pgfpathlineto{\pgfqpoint{0.725964in}{0.315287in}}%
\pgfpathlineto{\pgfqpoint{0.727704in}{0.302361in}}%
\pgfpathclose%
\pgfusepath{fill}%
\end{pgfscope}%
\begin{pgfscope}%
\pgfpathrectangle{\pgfqpoint{0.211875in}{0.211875in}}{\pgfqpoint{1.313625in}{1.279725in}}%
\pgfusepath{clip}%
\pgfsetbuttcap%
\pgfsetroundjoin%
\definecolor{currentfill}{rgb}{0.343142,0.118134,0.311397}%
\pgfsetfillcolor{currentfill}%
\pgfsetlinewidth{0.000000pt}%
\definecolor{currentstroke}{rgb}{0.000000,0.000000,0.000000}%
\pgfsetstrokecolor{currentstroke}%
\pgfsetdash{}{0pt}%
\pgfpathmoveto{\pgfqpoint{0.848784in}{0.301384in}}%
\pgfpathlineto{\pgfqpoint{0.862053in}{0.301666in}}%
\pgfpathlineto{\pgfqpoint{0.862535in}{0.302361in}}%
\pgfpathlineto{\pgfqpoint{0.864581in}{0.315287in}}%
\pgfpathlineto{\pgfqpoint{0.862053in}{0.322409in}}%
\pgfpathlineto{\pgfqpoint{0.848784in}{0.322912in}}%
\pgfpathlineto{\pgfqpoint{0.846002in}{0.315287in}}%
\pgfpathlineto{\pgfqpoint{0.848085in}{0.302361in}}%
\pgfpathclose%
\pgfusepath{fill}%
\end{pgfscope}%
\begin{pgfscope}%
\pgfpathrectangle{\pgfqpoint{0.211875in}{0.211875in}}{\pgfqpoint{1.313625in}{1.279725in}}%
\pgfusepath{clip}%
\pgfsetbuttcap%
\pgfsetroundjoin%
\definecolor{currentfill}{rgb}{0.343142,0.118134,0.311397}%
\pgfsetfillcolor{currentfill}%
\pgfsetlinewidth{0.000000pt}%
\definecolor{currentstroke}{rgb}{0.000000,0.000000,0.000000}%
\pgfsetstrokecolor{currentstroke}%
\pgfsetdash{}{0pt}%
\pgfpathmoveto{\pgfqpoint{0.968205in}{0.302248in}}%
\pgfpathlineto{\pgfqpoint{0.968818in}{0.302361in}}%
\pgfpathlineto{\pgfqpoint{0.981473in}{0.312081in}}%
\pgfpathlineto{\pgfqpoint{0.981932in}{0.315287in}}%
\pgfpathlineto{\pgfqpoint{0.981473in}{0.316732in}}%
\pgfpathlineto{\pgfqpoint{0.968205in}{0.321427in}}%
\pgfpathlineto{\pgfqpoint{0.965674in}{0.315287in}}%
\pgfpathlineto{\pgfqpoint{0.968113in}{0.302361in}}%
\pgfpathclose%
\pgfusepath{fill}%
\end{pgfscope}%
\begin{pgfscope}%
\pgfpathrectangle{\pgfqpoint{0.211875in}{0.211875in}}{\pgfqpoint{1.313625in}{1.279725in}}%
\pgfusepath{clip}%
\pgfsetbuttcap%
\pgfsetroundjoin%
\definecolor{currentfill}{rgb}{0.343142,0.118134,0.311397}%
\pgfsetfillcolor{currentfill}%
\pgfsetlinewidth{0.000000pt}%
\definecolor{currentstroke}{rgb}{0.000000,0.000000,0.000000}%
\pgfsetstrokecolor{currentstroke}%
\pgfsetdash{}{0pt}%
\pgfpathmoveto{\pgfqpoint{1.207045in}{0.301968in}}%
\pgfpathlineto{\pgfqpoint{1.207847in}{0.302361in}}%
\pgfpathlineto{\pgfqpoint{1.213969in}{0.315287in}}%
\pgfpathlineto{\pgfqpoint{1.207045in}{0.321986in}}%
\pgfpathlineto{\pgfqpoint{1.203379in}{0.315287in}}%
\pgfpathlineto{\pgfqpoint{1.206621in}{0.302361in}}%
\pgfpathclose%
\pgfusepath{fill}%
\end{pgfscope}%
\begin{pgfscope}%
\pgfpathrectangle{\pgfqpoint{0.211875in}{0.211875in}}{\pgfqpoint{1.313625in}{1.279725in}}%
\pgfusepath{clip}%
\pgfsetbuttcap%
\pgfsetroundjoin%
\definecolor{currentfill}{rgb}{0.343142,0.118134,0.311397}%
\pgfsetfillcolor{currentfill}%
\pgfsetlinewidth{0.000000pt}%
\definecolor{currentstroke}{rgb}{0.000000,0.000000,0.000000}%
\pgfsetstrokecolor{currentstroke}%
\pgfsetdash{}{0pt}%
\pgfpathmoveto{\pgfqpoint{1.326466in}{0.300891in}}%
\pgfpathlineto{\pgfqpoint{1.328773in}{0.302361in}}%
\pgfpathlineto{\pgfqpoint{1.333314in}{0.315287in}}%
\pgfpathlineto{\pgfqpoint{1.326466in}{0.323927in}}%
\pgfpathlineto{\pgfqpoint{1.320835in}{0.315287in}}%
\pgfpathlineto{\pgfqpoint{1.324575in}{0.302361in}}%
\pgfpathclose%
\pgfusepath{fill}%
\end{pgfscope}%
\begin{pgfscope}%
\pgfpathrectangle{\pgfqpoint{0.211875in}{0.211875in}}{\pgfqpoint{1.313625in}{1.279725in}}%
\pgfusepath{clip}%
\pgfsetbuttcap%
\pgfsetroundjoin%
\definecolor{currentfill}{rgb}{0.343142,0.118134,0.311397}%
\pgfsetfillcolor{currentfill}%
\pgfsetlinewidth{0.000000pt}%
\definecolor{currentstroke}{rgb}{0.000000,0.000000,0.000000}%
\pgfsetstrokecolor{currentstroke}%
\pgfsetdash{}{0pt}%
\pgfpathmoveto{\pgfqpoint{1.445886in}{0.299186in}}%
\pgfpathlineto{\pgfqpoint{1.449935in}{0.302361in}}%
\pgfpathlineto{\pgfqpoint{1.453414in}{0.315287in}}%
\pgfpathlineto{\pgfqpoint{1.445886in}{0.326981in}}%
\pgfpathlineto{\pgfqpoint{1.436473in}{0.315287in}}%
\pgfpathlineto{\pgfqpoint{1.440848in}{0.302361in}}%
\pgfpathclose%
\pgfusepath{fill}%
\end{pgfscope}%
\begin{pgfscope}%
\pgfpathrectangle{\pgfqpoint{0.211875in}{0.211875in}}{\pgfqpoint{1.313625in}{1.279725in}}%
\pgfusepath{clip}%
\pgfsetbuttcap%
\pgfsetroundjoin%
\definecolor{currentfill}{rgb}{0.343142,0.118134,0.311397}%
\pgfsetfillcolor{currentfill}%
\pgfsetlinewidth{0.000000pt}%
\definecolor{currentstroke}{rgb}{0.000000,0.000000,0.000000}%
\pgfsetstrokecolor{currentstroke}%
\pgfsetdash{}{0pt}%
\pgfpathmoveto{\pgfqpoint{1.087625in}{0.302615in}}%
\pgfpathlineto{\pgfqpoint{1.096390in}{0.315287in}}%
\pgfpathlineto{\pgfqpoint{1.087625in}{0.321143in}}%
\pgfpathlineto{\pgfqpoint{1.084867in}{0.315287in}}%
\pgfpathclose%
\pgfusepath{fill}%
\end{pgfscope}%
\begin{pgfscope}%
\pgfpathrectangle{\pgfqpoint{0.211875in}{0.211875in}}{\pgfqpoint{1.313625in}{1.279725in}}%
\pgfusepath{clip}%
\pgfsetbuttcap%
\pgfsetroundjoin%
\definecolor{currentfill}{rgb}{0.343142,0.118134,0.311397}%
\pgfsetfillcolor{currentfill}%
\pgfsetlinewidth{0.000000pt}%
\definecolor{currentstroke}{rgb}{0.000000,0.000000,0.000000}%
\pgfsetstrokecolor{currentstroke}%
\pgfsetdash{}{0pt}%
\pgfpathmoveto{\pgfqpoint{0.218918in}{0.366993in}}%
\pgfpathlineto{\pgfqpoint{0.225144in}{0.373539in}}%
\pgfpathlineto{\pgfqpoint{0.227309in}{0.379920in}}%
\pgfpathlineto{\pgfqpoint{0.228443in}{0.392846in}}%
\pgfpathlineto{\pgfqpoint{0.227456in}{0.405773in}}%
\pgfpathlineto{\pgfqpoint{0.225144in}{0.413394in}}%
\pgfpathlineto{\pgfqpoint{0.221258in}{0.418699in}}%
\pgfpathlineto{\pgfqpoint{0.211875in}{0.423975in}}%
\pgfpathlineto{\pgfqpoint{0.211875in}{0.418699in}}%
\pgfpathlineto{\pgfqpoint{0.211875in}{0.413894in}}%
\pgfpathlineto{\pgfqpoint{0.218589in}{0.405773in}}%
\pgfpathlineto{\pgfqpoint{0.221043in}{0.392846in}}%
\pgfpathlineto{\pgfqpoint{0.217425in}{0.379920in}}%
\pgfpathlineto{\pgfqpoint{0.211875in}{0.374271in}}%
\pgfpathlineto{\pgfqpoint{0.211875in}{0.366993in}}%
\pgfpathlineto{\pgfqpoint{0.211875in}{0.363839in}}%
\pgfpathclose%
\pgfusepath{fill}%
\end{pgfscope}%
\begin{pgfscope}%
\pgfpathrectangle{\pgfqpoint{0.211875in}{0.211875in}}{\pgfqpoint{1.313625in}{1.279725in}}%
\pgfusepath{clip}%
\pgfsetbuttcap%
\pgfsetroundjoin%
\definecolor{currentfill}{rgb}{0.343142,0.118134,0.311397}%
\pgfsetfillcolor{currentfill}%
\pgfsetlinewidth{0.000000pt}%
\definecolor{currentstroke}{rgb}{0.000000,0.000000,0.000000}%
\pgfsetstrokecolor{currentstroke}%
\pgfsetdash{}{0pt}%
\pgfpathmoveto{\pgfqpoint{0.318027in}{0.368481in}}%
\pgfpathlineto{\pgfqpoint{0.331295in}{0.368986in}}%
\pgfpathlineto{\pgfqpoint{0.340826in}{0.379920in}}%
\pgfpathlineto{\pgfqpoint{0.343296in}{0.392846in}}%
\pgfpathlineto{\pgfqpoint{0.341454in}{0.405773in}}%
\pgfpathlineto{\pgfqpoint{0.332279in}{0.418699in}}%
\pgfpathlineto{\pgfqpoint{0.331295in}{0.419319in}}%
\pgfpathlineto{\pgfqpoint{0.318027in}{0.419686in}}%
\pgfpathlineto{\pgfqpoint{0.316312in}{0.418699in}}%
\pgfpathlineto{\pgfqpoint{0.306905in}{0.405773in}}%
\pgfpathlineto{\pgfqpoint{0.305076in}{0.392846in}}%
\pgfpathlineto{\pgfqpoint{0.307449in}{0.379920in}}%
\pgfpathclose%
\pgfpathmoveto{\pgfqpoint{0.314196in}{0.392846in}}%
\pgfpathlineto{\pgfqpoint{0.316886in}{0.405773in}}%
\pgfpathlineto{\pgfqpoint{0.318027in}{0.407246in}}%
\pgfpathlineto{\pgfqpoint{0.331295in}{0.406779in}}%
\pgfpathlineto{\pgfqpoint{0.332034in}{0.405773in}}%
\pgfpathlineto{\pgfqpoint{0.334636in}{0.392846in}}%
\pgfpathlineto{\pgfqpoint{0.331295in}{0.381939in}}%
\pgfpathlineto{\pgfqpoint{0.318027in}{0.380894in}}%
\pgfpathclose%
\pgfusepath{fill}%
\end{pgfscope}%
\begin{pgfscope}%
\pgfpathrectangle{\pgfqpoint{0.211875in}{0.211875in}}{\pgfqpoint{1.313625in}{1.279725in}}%
\pgfusepath{clip}%
\pgfsetbuttcap%
\pgfsetroundjoin%
\definecolor{currentfill}{rgb}{0.343142,0.118134,0.311397}%
\pgfsetfillcolor{currentfill}%
\pgfsetlinewidth{0.000000pt}%
\definecolor{currentstroke}{rgb}{0.000000,0.000000,0.000000}%
\pgfsetstrokecolor{currentstroke}%
\pgfsetdash{}{0pt}%
\pgfpathmoveto{\pgfqpoint{0.437447in}{0.373011in}}%
\pgfpathlineto{\pgfqpoint{0.450716in}{0.374861in}}%
\pgfpathlineto{\pgfqpoint{0.454651in}{0.379920in}}%
\pgfpathlineto{\pgfqpoint{0.457553in}{0.392846in}}%
\pgfpathlineto{\pgfqpoint{0.455559in}{0.405773in}}%
\pgfpathlineto{\pgfqpoint{0.450716in}{0.413145in}}%
\pgfpathlineto{\pgfqpoint{0.437447in}{0.415262in}}%
\pgfpathlineto{\pgfqpoint{0.429130in}{0.405773in}}%
\pgfpathlineto{\pgfqpoint{0.426697in}{0.392846in}}%
\pgfpathlineto{\pgfqpoint{0.430196in}{0.379920in}}%
\pgfpathclose%
\pgfpathmoveto{\pgfqpoint{0.436765in}{0.392846in}}%
\pgfpathlineto{\pgfqpoint{0.437447in}{0.395646in}}%
\pgfpathlineto{\pgfqpoint{0.441692in}{0.392846in}}%
\pgfpathlineto{\pgfqpoint{0.437447in}{0.390974in}}%
\pgfpathclose%
\pgfusepath{fill}%
\end{pgfscope}%
\begin{pgfscope}%
\pgfpathrectangle{\pgfqpoint{0.211875in}{0.211875in}}{\pgfqpoint{1.313625in}{1.279725in}}%
\pgfusepath{clip}%
\pgfsetbuttcap%
\pgfsetroundjoin%
\definecolor{currentfill}{rgb}{0.343142,0.118134,0.311397}%
\pgfsetfillcolor{currentfill}%
\pgfsetlinewidth{0.000000pt}%
\definecolor{currentstroke}{rgb}{0.000000,0.000000,0.000000}%
\pgfsetstrokecolor{currentstroke}%
\pgfsetdash{}{0pt}%
\pgfpathmoveto{\pgfqpoint{0.556867in}{0.376713in}}%
\pgfpathlineto{\pgfqpoint{0.568246in}{0.379920in}}%
\pgfpathlineto{\pgfqpoint{0.570136in}{0.381368in}}%
\pgfpathlineto{\pgfqpoint{0.572926in}{0.392846in}}%
\pgfpathlineto{\pgfqpoint{0.570829in}{0.405773in}}%
\pgfpathlineto{\pgfqpoint{0.570136in}{0.406947in}}%
\pgfpathlineto{\pgfqpoint{0.556867in}{0.411252in}}%
\pgfpathlineto{\pgfqpoint{0.551366in}{0.405773in}}%
\pgfpathlineto{\pgfqpoint{0.548238in}{0.392846in}}%
\pgfpathlineto{\pgfqpoint{0.553007in}{0.379920in}}%
\pgfpathclose%
\pgfusepath{fill}%
\end{pgfscope}%
\begin{pgfscope}%
\pgfpathrectangle{\pgfqpoint{0.211875in}{0.211875in}}{\pgfqpoint{1.313625in}{1.279725in}}%
\pgfusepath{clip}%
\pgfsetbuttcap%
\pgfsetroundjoin%
\definecolor{currentfill}{rgb}{0.343142,0.118134,0.311397}%
\pgfsetfillcolor{currentfill}%
\pgfsetlinewidth{0.000000pt}%
\definecolor{currentstroke}{rgb}{0.000000,0.000000,0.000000}%
\pgfsetstrokecolor{currentstroke}%
\pgfsetdash{}{0pt}%
\pgfpathmoveto{\pgfqpoint{0.676288in}{0.379716in}}%
\pgfpathlineto{\pgfqpoint{0.676798in}{0.379920in}}%
\pgfpathlineto{\pgfqpoint{0.688167in}{0.392846in}}%
\pgfpathlineto{\pgfqpoint{0.680943in}{0.405773in}}%
\pgfpathlineto{\pgfqpoint{0.676288in}{0.407986in}}%
\pgfpathlineto{\pgfqpoint{0.673699in}{0.405773in}}%
\pgfpathlineto{\pgfqpoint{0.669731in}{0.392846in}}%
\pgfpathlineto{\pgfqpoint{0.676002in}{0.379920in}}%
\pgfpathclose%
\pgfusepath{fill}%
\end{pgfscope}%
\begin{pgfscope}%
\pgfpathrectangle{\pgfqpoint{0.211875in}{0.211875in}}{\pgfqpoint{1.313625in}{1.279725in}}%
\pgfusepath{clip}%
\pgfsetbuttcap%
\pgfsetroundjoin%
\definecolor{currentfill}{rgb}{0.343142,0.118134,0.311397}%
\pgfsetfillcolor{currentfill}%
\pgfsetlinewidth{0.000000pt}%
\definecolor{currentstroke}{rgb}{0.000000,0.000000,0.000000}%
\pgfsetstrokecolor{currentstroke}%
\pgfsetdash{}{0pt}%
\pgfpathmoveto{\pgfqpoint{0.795708in}{0.385159in}}%
\pgfpathlineto{\pgfqpoint{0.800968in}{0.392846in}}%
\pgfpathlineto{\pgfqpoint{0.795708in}{0.404456in}}%
\pgfpathlineto{\pgfqpoint{0.791216in}{0.392846in}}%
\pgfpathclose%
\pgfusepath{fill}%
\end{pgfscope}%
\begin{pgfscope}%
\pgfpathrectangle{\pgfqpoint{0.211875in}{0.211875in}}{\pgfqpoint{1.313625in}{1.279725in}}%
\pgfusepath{clip}%
\pgfsetbuttcap%
\pgfsetroundjoin%
\definecolor{currentfill}{rgb}{0.343142,0.118134,0.311397}%
\pgfsetfillcolor{currentfill}%
\pgfsetlinewidth{0.000000pt}%
\definecolor{currentstroke}{rgb}{0.000000,0.000000,0.000000}%
\pgfsetstrokecolor{currentstroke}%
\pgfsetdash{}{0pt}%
\pgfpathmoveto{\pgfqpoint{0.915129in}{0.389574in}}%
\pgfpathlineto{\pgfqpoint{0.916954in}{0.392846in}}%
\pgfpathlineto{\pgfqpoint{0.915129in}{0.397788in}}%
\pgfpathlineto{\pgfqpoint{0.912758in}{0.392846in}}%
\pgfpathclose%
\pgfusepath{fill}%
\end{pgfscope}%
\begin{pgfscope}%
\pgfpathrectangle{\pgfqpoint{0.211875in}{0.211875in}}{\pgfqpoint{1.313625in}{1.279725in}}%
\pgfusepath{clip}%
\pgfsetbuttcap%
\pgfsetroundjoin%
\definecolor{currentfill}{rgb}{0.343142,0.118134,0.311397}%
\pgfsetfillcolor{currentfill}%
\pgfsetlinewidth{0.000000pt}%
\definecolor{currentstroke}{rgb}{0.000000,0.000000,0.000000}%
\pgfsetstrokecolor{currentstroke}%
\pgfsetdash{}{0pt}%
\pgfpathmoveto{\pgfqpoint{1.034549in}{0.392771in}}%
\pgfpathlineto{\pgfqpoint{1.034585in}{0.392846in}}%
\pgfpathlineto{\pgfqpoint{1.034549in}{0.392960in}}%
\pgfpathlineto{\pgfqpoint{1.034477in}{0.392846in}}%
\pgfpathclose%
\pgfusepath{fill}%
\end{pgfscope}%
\begin{pgfscope}%
\pgfpathrectangle{\pgfqpoint{0.211875in}{0.211875in}}{\pgfqpoint{1.313625in}{1.279725in}}%
\pgfusepath{clip}%
\pgfsetbuttcap%
\pgfsetroundjoin%
\definecolor{currentfill}{rgb}{0.343142,0.118134,0.311397}%
\pgfsetfillcolor{currentfill}%
\pgfsetlinewidth{0.000000pt}%
\definecolor{currentstroke}{rgb}{0.000000,0.000000,0.000000}%
\pgfsetstrokecolor{currentstroke}%
\pgfsetdash{}{0pt}%
\pgfpathmoveto{\pgfqpoint{1.498962in}{0.388209in}}%
\pgfpathlineto{\pgfqpoint{1.511125in}{0.392846in}}%
\pgfpathlineto{\pgfqpoint{1.498962in}{0.399791in}}%
\pgfpathlineto{\pgfqpoint{1.497307in}{0.392846in}}%
\pgfpathclose%
\pgfusepath{fill}%
\end{pgfscope}%
\begin{pgfscope}%
\pgfpathrectangle{\pgfqpoint{0.211875in}{0.211875in}}{\pgfqpoint{1.313625in}{1.279725in}}%
\pgfusepath{clip}%
\pgfsetbuttcap%
\pgfsetroundjoin%
\definecolor{currentfill}{rgb}{0.343142,0.118134,0.311397}%
\pgfsetfillcolor{currentfill}%
\pgfsetlinewidth{0.000000pt}%
\definecolor{currentstroke}{rgb}{0.000000,0.000000,0.000000}%
\pgfsetstrokecolor{currentstroke}%
\pgfsetdash{}{0pt}%
\pgfpathmoveto{\pgfqpoint{0.251682in}{0.455053in}}%
\pgfpathlineto{\pgfqpoint{0.264951in}{0.449652in}}%
\pgfpathlineto{\pgfqpoint{0.278220in}{0.454701in}}%
\pgfpathlineto{\pgfqpoint{0.280156in}{0.457479in}}%
\pgfpathlineto{\pgfqpoint{0.283446in}{0.470405in}}%
\pgfpathlineto{\pgfqpoint{0.282990in}{0.483332in}}%
\pgfpathlineto{\pgfqpoint{0.278221in}{0.496258in}}%
\pgfpathlineto{\pgfqpoint{0.278220in}{0.496260in}}%
\pgfpathlineto{\pgfqpoint{0.264951in}{0.501169in}}%
\pgfpathlineto{\pgfqpoint{0.252503in}{0.496258in}}%
\pgfpathlineto{\pgfqpoint{0.251682in}{0.495530in}}%
\pgfpathlineto{\pgfqpoint{0.247498in}{0.483332in}}%
\pgfpathlineto{\pgfqpoint{0.247054in}{0.470405in}}%
\pgfpathlineto{\pgfqpoint{0.250097in}{0.457479in}}%
\pgfpathclose%
\pgfpathmoveto{\pgfqpoint{0.256794in}{0.470405in}}%
\pgfpathlineto{\pgfqpoint{0.258081in}{0.483332in}}%
\pgfpathlineto{\pgfqpoint{0.264951in}{0.489578in}}%
\pgfpathlineto{\pgfqpoint{0.272656in}{0.483332in}}%
\pgfpathlineto{\pgfqpoint{0.274118in}{0.470405in}}%
\pgfpathlineto{\pgfqpoint{0.264951in}{0.461163in}}%
\pgfpathclose%
\pgfusepath{fill}%
\end{pgfscope}%
\begin{pgfscope}%
\pgfpathrectangle{\pgfqpoint{0.211875in}{0.211875in}}{\pgfqpoint{1.313625in}{1.279725in}}%
\pgfusepath{clip}%
\pgfsetbuttcap%
\pgfsetroundjoin%
\definecolor{currentfill}{rgb}{0.343142,0.118134,0.311397}%
\pgfsetfillcolor{currentfill}%
\pgfsetlinewidth{0.000000pt}%
\definecolor{currentstroke}{rgb}{0.000000,0.000000,0.000000}%
\pgfsetstrokecolor{currentstroke}%
\pgfsetdash{}{0pt}%
\pgfpathmoveto{\pgfqpoint{0.384371in}{0.454214in}}%
\pgfpathlineto{\pgfqpoint{0.390692in}{0.457479in}}%
\pgfpathlineto{\pgfqpoint{0.397640in}{0.466718in}}%
\pgfpathlineto{\pgfqpoint{0.398588in}{0.470405in}}%
\pgfpathlineto{\pgfqpoint{0.398111in}{0.483332in}}%
\pgfpathlineto{\pgfqpoint{0.397640in}{0.484747in}}%
\pgfpathlineto{\pgfqpoint{0.386073in}{0.496258in}}%
\pgfpathlineto{\pgfqpoint{0.384371in}{0.497021in}}%
\pgfpathlineto{\pgfqpoint{0.381958in}{0.496258in}}%
\pgfpathlineto{\pgfqpoint{0.371102in}{0.489031in}}%
\pgfpathlineto{\pgfqpoint{0.368926in}{0.483332in}}%
\pgfpathlineto{\pgfqpoint{0.368417in}{0.470405in}}%
\pgfpathlineto{\pgfqpoint{0.371102in}{0.461378in}}%
\pgfpathlineto{\pgfqpoint{0.375482in}{0.457479in}}%
\pgfpathclose%
\pgfpathmoveto{\pgfqpoint{0.383537in}{0.470405in}}%
\pgfpathlineto{\pgfqpoint{0.384371in}{0.476648in}}%
\pgfpathlineto{\pgfqpoint{0.384988in}{0.470405in}}%
\pgfpathlineto{\pgfqpoint{0.384371in}{0.469646in}}%
\pgfpathclose%
\pgfusepath{fill}%
\end{pgfscope}%
\begin{pgfscope}%
\pgfpathrectangle{\pgfqpoint{0.211875in}{0.211875in}}{\pgfqpoint{1.313625in}{1.279725in}}%
\pgfusepath{clip}%
\pgfsetbuttcap%
\pgfsetroundjoin%
\definecolor{currentfill}{rgb}{0.343142,0.118134,0.311397}%
\pgfsetfillcolor{currentfill}%
\pgfsetlinewidth{0.000000pt}%
\definecolor{currentstroke}{rgb}{0.000000,0.000000,0.000000}%
\pgfsetstrokecolor{currentstroke}%
\pgfsetdash{}{0pt}%
\pgfpathmoveto{\pgfqpoint{0.490523in}{0.467543in}}%
\pgfpathlineto{\pgfqpoint{0.503792in}{0.458957in}}%
\pgfpathlineto{\pgfqpoint{0.511650in}{0.470405in}}%
\pgfpathlineto{\pgfqpoint{0.510658in}{0.483332in}}%
\pgfpathlineto{\pgfqpoint{0.503792in}{0.491366in}}%
\pgfpathlineto{\pgfqpoint{0.490523in}{0.484217in}}%
\pgfpathlineto{\pgfqpoint{0.490147in}{0.483332in}}%
\pgfpathlineto{\pgfqpoint{0.489568in}{0.470405in}}%
\pgfpathclose%
\pgfusepath{fill}%
\end{pgfscope}%
\begin{pgfscope}%
\pgfpathrectangle{\pgfqpoint{0.211875in}{0.211875in}}{\pgfqpoint{1.313625in}{1.279725in}}%
\pgfusepath{clip}%
\pgfsetbuttcap%
\pgfsetroundjoin%
\definecolor{currentfill}{rgb}{0.343142,0.118134,0.311397}%
\pgfsetfillcolor{currentfill}%
\pgfsetlinewidth{0.000000pt}%
\definecolor{currentstroke}{rgb}{0.000000,0.000000,0.000000}%
\pgfsetstrokecolor{currentstroke}%
\pgfsetdash{}{0pt}%
\pgfpathmoveto{\pgfqpoint{0.623212in}{0.465717in}}%
\pgfpathlineto{\pgfqpoint{0.625990in}{0.470405in}}%
\pgfpathlineto{\pgfqpoint{0.625074in}{0.483332in}}%
\pgfpathlineto{\pgfqpoint{0.623212in}{0.485852in}}%
\pgfpathlineto{\pgfqpoint{0.616306in}{0.483332in}}%
\pgfpathlineto{\pgfqpoint{0.612927in}{0.470405in}}%
\pgfpathclose%
\pgfusepath{fill}%
\end{pgfscope}%
\begin{pgfscope}%
\pgfpathrectangle{\pgfqpoint{0.211875in}{0.211875in}}{\pgfqpoint{1.313625in}{1.279725in}}%
\pgfusepath{clip}%
\pgfsetbuttcap%
\pgfsetroundjoin%
\definecolor{currentfill}{rgb}{0.343142,0.118134,0.311397}%
\pgfsetfillcolor{currentfill}%
\pgfsetlinewidth{0.000000pt}%
\definecolor{currentstroke}{rgb}{0.000000,0.000000,0.000000}%
\pgfsetstrokecolor{currentstroke}%
\pgfsetdash{}{0pt}%
\pgfpathmoveto{\pgfqpoint{0.217228in}{0.535038in}}%
\pgfpathlineto{\pgfqpoint{0.224585in}{0.547964in}}%
\pgfpathlineto{\pgfqpoint{0.225144in}{0.558199in}}%
\pgfpathlineto{\pgfqpoint{0.225228in}{0.560891in}}%
\pgfpathlineto{\pgfqpoint{0.225144in}{0.561378in}}%
\pgfpathlineto{\pgfqpoint{0.220819in}{0.573817in}}%
\pgfpathlineto{\pgfqpoint{0.211875in}{0.581635in}}%
\pgfpathlineto{\pgfqpoint{0.211875in}{0.573817in}}%
\pgfpathlineto{\pgfqpoint{0.211875in}{0.569049in}}%
\pgfpathlineto{\pgfqpoint{0.215706in}{0.560891in}}%
\pgfpathlineto{\pgfqpoint{0.214453in}{0.547964in}}%
\pgfpathlineto{\pgfqpoint{0.211875in}{0.544228in}}%
\pgfpathlineto{\pgfqpoint{0.211875in}{0.535038in}}%
\pgfpathlineto{\pgfqpoint{0.211875in}{0.531560in}}%
\pgfpathclose%
\pgfusepath{fill}%
\end{pgfscope}%
\begin{pgfscope}%
\pgfpathrectangle{\pgfqpoint{0.211875in}{0.211875in}}{\pgfqpoint{1.313625in}{1.279725in}}%
\pgfusepath{clip}%
\pgfsetbuttcap%
\pgfsetroundjoin%
\definecolor{currentfill}{rgb}{0.343142,0.118134,0.311397}%
\pgfsetfillcolor{currentfill}%
\pgfsetlinewidth{0.000000pt}%
\definecolor{currentstroke}{rgb}{0.000000,0.000000,0.000000}%
\pgfsetstrokecolor{currentstroke}%
\pgfsetdash{}{0pt}%
\pgfpathmoveto{\pgfqpoint{0.318027in}{0.537075in}}%
\pgfpathlineto{\pgfqpoint{0.331295in}{0.537673in}}%
\pgfpathlineto{\pgfqpoint{0.337588in}{0.547964in}}%
\pgfpathlineto{\pgfqpoint{0.338468in}{0.560891in}}%
\pgfpathlineto{\pgfqpoint{0.333473in}{0.573817in}}%
\pgfpathlineto{\pgfqpoint{0.331295in}{0.575958in}}%
\pgfpathlineto{\pgfqpoint{0.318027in}{0.576366in}}%
\pgfpathlineto{\pgfqpoint{0.315266in}{0.573817in}}%
\pgfpathlineto{\pgfqpoint{0.310154in}{0.560891in}}%
\pgfpathlineto{\pgfqpoint{0.311024in}{0.547964in}}%
\pgfpathclose%
\pgfusepath{fill}%
\end{pgfscope}%
\begin{pgfscope}%
\pgfpathrectangle{\pgfqpoint{0.211875in}{0.211875in}}{\pgfqpoint{1.313625in}{1.279725in}}%
\pgfusepath{clip}%
\pgfsetbuttcap%
\pgfsetroundjoin%
\definecolor{currentfill}{rgb}{0.343142,0.118134,0.311397}%
\pgfsetfillcolor{currentfill}%
\pgfsetlinewidth{0.000000pt}%
\definecolor{currentstroke}{rgb}{0.000000,0.000000,0.000000}%
\pgfsetstrokecolor{currentstroke}%
\pgfsetdash{}{0pt}%
\pgfpathmoveto{\pgfqpoint{0.437447in}{0.542648in}}%
\pgfpathlineto{\pgfqpoint{0.450716in}{0.545481in}}%
\pgfpathlineto{\pgfqpoint{0.452070in}{0.547964in}}%
\pgfpathlineto{\pgfqpoint{0.453081in}{0.560891in}}%
\pgfpathlineto{\pgfqpoint{0.450716in}{0.567233in}}%
\pgfpathlineto{\pgfqpoint{0.437447in}{0.571235in}}%
\pgfpathlineto{\pgfqpoint{0.432330in}{0.560891in}}%
\pgfpathlineto{\pgfqpoint{0.433567in}{0.547964in}}%
\pgfpathclose%
\pgfusepath{fill}%
\end{pgfscope}%
\begin{pgfscope}%
\pgfpathrectangle{\pgfqpoint{0.211875in}{0.211875in}}{\pgfqpoint{1.313625in}{1.279725in}}%
\pgfusepath{clip}%
\pgfsetbuttcap%
\pgfsetroundjoin%
\definecolor{currentfill}{rgb}{0.343142,0.118134,0.311397}%
\pgfsetfillcolor{currentfill}%
\pgfsetlinewidth{0.000000pt}%
\definecolor{currentstroke}{rgb}{0.000000,0.000000,0.000000}%
\pgfsetstrokecolor{currentstroke}%
\pgfsetdash{}{0pt}%
\pgfpathmoveto{\pgfqpoint{0.556867in}{0.547227in}}%
\pgfpathlineto{\pgfqpoint{0.558716in}{0.547964in}}%
\pgfpathlineto{\pgfqpoint{0.563721in}{0.560891in}}%
\pgfpathlineto{\pgfqpoint{0.556867in}{0.564899in}}%
\pgfpathlineto{\pgfqpoint{0.554596in}{0.560891in}}%
\pgfpathlineto{\pgfqpoint{0.556250in}{0.547964in}}%
\pgfpathclose%
\pgfusepath{fill}%
\end{pgfscope}%
\begin{pgfscope}%
\pgfpathrectangle{\pgfqpoint{0.211875in}{0.211875in}}{\pgfqpoint{1.313625in}{1.279725in}}%
\pgfusepath{clip}%
\pgfsetbuttcap%
\pgfsetroundjoin%
\definecolor{currentfill}{rgb}{0.343142,0.118134,0.311397}%
\pgfsetfillcolor{currentfill}%
\pgfsetlinewidth{0.000000pt}%
\definecolor{currentstroke}{rgb}{0.000000,0.000000,0.000000}%
\pgfsetstrokecolor{currentstroke}%
\pgfsetdash{}{0pt}%
\pgfpathmoveto{\pgfqpoint{0.264951in}{0.617008in}}%
\pgfpathlineto{\pgfqpoint{0.278220in}{0.624612in}}%
\pgfpathlineto{\pgfqpoint{0.278677in}{0.625523in}}%
\pgfpathlineto{\pgfqpoint{0.280593in}{0.638450in}}%
\pgfpathlineto{\pgfqpoint{0.278329in}{0.651377in}}%
\pgfpathlineto{\pgfqpoint{0.278220in}{0.651585in}}%
\pgfpathlineto{\pgfqpoint{0.264951in}{0.659072in}}%
\pgfpathlineto{\pgfqpoint{0.252697in}{0.651377in}}%
\pgfpathlineto{\pgfqpoint{0.251682in}{0.649099in}}%
\pgfpathlineto{\pgfqpoint{0.249938in}{0.638450in}}%
\pgfpathlineto{\pgfqpoint{0.251682in}{0.625952in}}%
\pgfpathlineto{\pgfqpoint{0.251846in}{0.625523in}}%
\pgfpathclose%
\pgfpathmoveto{\pgfqpoint{0.263230in}{0.638450in}}%
\pgfpathlineto{\pgfqpoint{0.264951in}{0.641763in}}%
\pgfpathlineto{\pgfqpoint{0.266892in}{0.638450in}}%
\pgfpathlineto{\pgfqpoint{0.264951in}{0.634779in}}%
\pgfpathclose%
\pgfusepath{fill}%
\end{pgfscope}%
\begin{pgfscope}%
\pgfpathrectangle{\pgfqpoint{0.211875in}{0.211875in}}{\pgfqpoint{1.313625in}{1.279725in}}%
\pgfusepath{clip}%
\pgfsetbuttcap%
\pgfsetroundjoin%
\definecolor{currentfill}{rgb}{0.343142,0.118134,0.311397}%
\pgfsetfillcolor{currentfill}%
\pgfsetlinewidth{0.000000pt}%
\definecolor{currentstroke}{rgb}{0.000000,0.000000,0.000000}%
\pgfsetstrokecolor{currentstroke}%
\pgfsetdash{}{0pt}%
\pgfpathmoveto{\pgfqpoint{0.384371in}{0.622481in}}%
\pgfpathlineto{\pgfqpoint{0.388620in}{0.625523in}}%
\pgfpathlineto{\pgfqpoint{0.393707in}{0.638450in}}%
\pgfpathlineto{\pgfqpoint{0.387911in}{0.651377in}}%
\pgfpathlineto{\pgfqpoint{0.384371in}{0.653824in}}%
\pgfpathlineto{\pgfqpoint{0.379515in}{0.651377in}}%
\pgfpathlineto{\pgfqpoint{0.371773in}{0.638450in}}%
\pgfpathlineto{\pgfqpoint{0.378536in}{0.625523in}}%
\pgfpathclose%
\pgfusepath{fill}%
\end{pgfscope}%
\begin{pgfscope}%
\pgfpathrectangle{\pgfqpoint{0.211875in}{0.211875in}}{\pgfqpoint{1.313625in}{1.279725in}}%
\pgfusepath{clip}%
\pgfsetbuttcap%
\pgfsetroundjoin%
\definecolor{currentfill}{rgb}{0.343142,0.118134,0.311397}%
\pgfsetfillcolor{currentfill}%
\pgfsetlinewidth{0.000000pt}%
\definecolor{currentstroke}{rgb}{0.000000,0.000000,0.000000}%
\pgfsetstrokecolor{currentstroke}%
\pgfsetdash{}{0pt}%
\pgfpathmoveto{\pgfqpoint{0.503792in}{0.630824in}}%
\pgfpathlineto{\pgfqpoint{0.506579in}{0.638450in}}%
\pgfpathlineto{\pgfqpoint{0.503792in}{0.645327in}}%
\pgfpathlineto{\pgfqpoint{0.497881in}{0.638450in}}%
\pgfpathclose%
\pgfusepath{fill}%
\end{pgfscope}%
\begin{pgfscope}%
\pgfpathrectangle{\pgfqpoint{0.211875in}{0.211875in}}{\pgfqpoint{1.313625in}{1.279725in}}%
\pgfusepath{clip}%
\pgfsetbuttcap%
\pgfsetroundjoin%
\definecolor{currentfill}{rgb}{0.343142,0.118134,0.311397}%
\pgfsetfillcolor{currentfill}%
\pgfsetlinewidth{0.000000pt}%
\definecolor{currentstroke}{rgb}{0.000000,0.000000,0.000000}%
\pgfsetstrokecolor{currentstroke}%
\pgfsetdash{}{0pt}%
\pgfpathmoveto{\pgfqpoint{0.216900in}{0.703083in}}%
\pgfpathlineto{\pgfqpoint{0.221631in}{0.716009in}}%
\pgfpathlineto{\pgfqpoint{0.220239in}{0.728936in}}%
\pgfpathlineto{\pgfqpoint{0.211875in}{0.740516in}}%
\pgfpathlineto{\pgfqpoint{0.211875in}{0.728936in}}%
\pgfpathlineto{\pgfqpoint{0.211875in}{0.717176in}}%
\pgfpathlineto{\pgfqpoint{0.212037in}{0.716009in}}%
\pgfpathlineto{\pgfqpoint{0.211875in}{0.715639in}}%
\pgfpathlineto{\pgfqpoint{0.211875in}{0.703083in}}%
\pgfpathlineto{\pgfqpoint{0.211875in}{0.698555in}}%
\pgfpathclose%
\pgfusepath{fill}%
\end{pgfscope}%
\begin{pgfscope}%
\pgfpathrectangle{\pgfqpoint{0.211875in}{0.211875in}}{\pgfqpoint{1.313625in}{1.279725in}}%
\pgfusepath{clip}%
\pgfsetbuttcap%
\pgfsetroundjoin%
\definecolor{currentfill}{rgb}{0.343142,0.118134,0.311397}%
\pgfsetfillcolor{currentfill}%
\pgfsetlinewidth{0.000000pt}%
\definecolor{currentstroke}{rgb}{0.000000,0.000000,0.000000}%
\pgfsetstrokecolor{currentstroke}%
\pgfsetdash{}{0pt}%
\pgfpathmoveto{\pgfqpoint{0.318027in}{0.705267in}}%
\pgfpathlineto{\pgfqpoint{0.331295in}{0.706105in}}%
\pgfpathlineto{\pgfqpoint{0.335140in}{0.716009in}}%
\pgfpathlineto{\pgfqpoint{0.333655in}{0.728936in}}%
\pgfpathlineto{\pgfqpoint{0.331295in}{0.732616in}}%
\pgfpathlineto{\pgfqpoint{0.318027in}{0.733145in}}%
\pgfpathlineto{\pgfqpoint{0.315188in}{0.728936in}}%
\pgfpathlineto{\pgfqpoint{0.313659in}{0.716009in}}%
\pgfpathclose%
\pgfusepath{fill}%
\end{pgfscope}%
\begin{pgfscope}%
\pgfpathrectangle{\pgfqpoint{0.211875in}{0.211875in}}{\pgfqpoint{1.313625in}{1.279725in}}%
\pgfusepath{clip}%
\pgfsetbuttcap%
\pgfsetroundjoin%
\definecolor{currentfill}{rgb}{0.343142,0.118134,0.311397}%
\pgfsetfillcolor{currentfill}%
\pgfsetlinewidth{0.000000pt}%
\definecolor{currentstroke}{rgb}{0.000000,0.000000,0.000000}%
\pgfsetstrokecolor{currentstroke}%
\pgfsetdash{}{0pt}%
\pgfpathmoveto{\pgfqpoint{0.437447in}{0.713324in}}%
\pgfpathlineto{\pgfqpoint{0.445145in}{0.716009in}}%
\pgfpathlineto{\pgfqpoint{0.437447in}{0.724461in}}%
\pgfpathlineto{\pgfqpoint{0.436208in}{0.716009in}}%
\pgfpathclose%
\pgfusepath{fill}%
\end{pgfscope}%
\begin{pgfscope}%
\pgfpathrectangle{\pgfqpoint{0.211875in}{0.211875in}}{\pgfqpoint{1.313625in}{1.279725in}}%
\pgfusepath{clip}%
\pgfsetbuttcap%
\pgfsetroundjoin%
\definecolor{currentfill}{rgb}{0.343142,0.118134,0.311397}%
\pgfsetfillcolor{currentfill}%
\pgfsetlinewidth{0.000000pt}%
\definecolor{currentstroke}{rgb}{0.000000,0.000000,0.000000}%
\pgfsetstrokecolor{currentstroke}%
\pgfsetdash{}{0pt}%
\pgfpathmoveto{\pgfqpoint{0.264951in}{0.783332in}}%
\pgfpathlineto{\pgfqpoint{0.276981in}{0.793568in}}%
\pgfpathlineto{\pgfqpoint{0.277569in}{0.806495in}}%
\pgfpathlineto{\pgfqpoint{0.264951in}{0.818514in}}%
\pgfpathlineto{\pgfqpoint{0.253728in}{0.806495in}}%
\pgfpathlineto{\pgfqpoint{0.254219in}{0.793568in}}%
\pgfpathclose%
\pgfusepath{fill}%
\end{pgfscope}%
\begin{pgfscope}%
\pgfpathrectangle{\pgfqpoint{0.211875in}{0.211875in}}{\pgfqpoint{1.313625in}{1.279725in}}%
\pgfusepath{clip}%
\pgfsetbuttcap%
\pgfsetroundjoin%
\definecolor{currentfill}{rgb}{0.343142,0.118134,0.311397}%
\pgfsetfillcolor{currentfill}%
\pgfsetlinewidth{0.000000pt}%
\definecolor{currentstroke}{rgb}{0.000000,0.000000,0.000000}%
\pgfsetstrokecolor{currentstroke}%
\pgfsetdash{}{0pt}%
\pgfpathmoveto{\pgfqpoint{0.384371in}{0.790611in}}%
\pgfpathlineto{\pgfqpoint{0.387217in}{0.793568in}}%
\pgfpathlineto{\pgfqpoint{0.387829in}{0.806495in}}%
\pgfpathlineto{\pgfqpoint{0.384371in}{0.810519in}}%
\pgfpathlineto{\pgfqpoint{0.379692in}{0.806495in}}%
\pgfpathlineto{\pgfqpoint{0.380510in}{0.793568in}}%
\pgfpathclose%
\pgfusepath{fill}%
\end{pgfscope}%
\begin{pgfscope}%
\pgfpathrectangle{\pgfqpoint{0.211875in}{0.211875in}}{\pgfqpoint{1.313625in}{1.279725in}}%
\pgfusepath{clip}%
\pgfsetbuttcap%
\pgfsetroundjoin%
\definecolor{currentfill}{rgb}{0.343142,0.118134,0.311397}%
\pgfsetfillcolor{currentfill}%
\pgfsetlinewidth{0.000000pt}%
\definecolor{currentstroke}{rgb}{0.000000,0.000000,0.000000}%
\pgfsetstrokecolor{currentstroke}%
\pgfsetdash{}{0pt}%
\pgfpathmoveto{\pgfqpoint{0.217078in}{0.871127in}}%
\pgfpathlineto{\pgfqpoint{0.219217in}{0.884054in}}%
\pgfpathlineto{\pgfqpoint{0.214766in}{0.896980in}}%
\pgfpathlineto{\pgfqpoint{0.211875in}{0.899825in}}%
\pgfpathlineto{\pgfqpoint{0.211875in}{0.896980in}}%
\pgfpathlineto{\pgfqpoint{0.211875in}{0.884054in}}%
\pgfpathlineto{\pgfqpoint{0.211875in}{0.871127in}}%
\pgfpathlineto{\pgfqpoint{0.211875in}{0.864576in}}%
\pgfpathclose%
\pgfusepath{fill}%
\end{pgfscope}%
\begin{pgfscope}%
\pgfpathrectangle{\pgfqpoint{0.211875in}{0.211875in}}{\pgfqpoint{1.313625in}{1.279725in}}%
\pgfusepath{clip}%
\pgfsetbuttcap%
\pgfsetroundjoin%
\definecolor{currentfill}{rgb}{0.343142,0.118134,0.311397}%
\pgfsetfillcolor{currentfill}%
\pgfsetlinewidth{0.000000pt}%
\definecolor{currentstroke}{rgb}{0.000000,0.000000,0.000000}%
\pgfsetstrokecolor{currentstroke}%
\pgfsetdash{}{0pt}%
\pgfpathmoveto{\pgfqpoint{0.318027in}{0.872803in}}%
\pgfpathlineto{\pgfqpoint{0.331295in}{0.874547in}}%
\pgfpathlineto{\pgfqpoint{0.332979in}{0.884054in}}%
\pgfpathlineto{\pgfqpoint{0.331295in}{0.889081in}}%
\pgfpathlineto{\pgfqpoint{0.318027in}{0.890000in}}%
\pgfpathlineto{\pgfqpoint{0.315941in}{0.884054in}}%
\pgfpathclose%
\pgfusepath{fill}%
\end{pgfscope}%
\begin{pgfscope}%
\pgfpathrectangle{\pgfqpoint{0.211875in}{0.211875in}}{\pgfqpoint{1.313625in}{1.279725in}}%
\pgfusepath{clip}%
\pgfsetbuttcap%
\pgfsetroundjoin%
\definecolor{currentfill}{rgb}{0.343142,0.118134,0.311397}%
\pgfsetfillcolor{currentfill}%
\pgfsetlinewidth{0.000000pt}%
\definecolor{currentstroke}{rgb}{0.000000,0.000000,0.000000}%
\pgfsetstrokecolor{currentstroke}%
\pgfsetdash{}{0pt}%
\pgfpathmoveto{\pgfqpoint{0.264951in}{0.948010in}}%
\pgfpathlineto{\pgfqpoint{0.266252in}{0.948686in}}%
\pgfpathlineto{\pgfqpoint{0.275160in}{0.961613in}}%
\pgfpathlineto{\pgfqpoint{0.270282in}{0.974539in}}%
\pgfpathlineto{\pgfqpoint{0.264951in}{0.977972in}}%
\pgfpathlineto{\pgfqpoint{0.260158in}{0.974539in}}%
\pgfpathlineto{\pgfqpoint{0.255895in}{0.961613in}}%
\pgfpathlineto{\pgfqpoint{0.263766in}{0.948686in}}%
\pgfpathclose%
\pgfusepath{fill}%
\end{pgfscope}%
\begin{pgfscope}%
\pgfpathrectangle{\pgfqpoint{0.211875in}{0.211875in}}{\pgfqpoint{1.313625in}{1.279725in}}%
\pgfusepath{clip}%
\pgfsetbuttcap%
\pgfsetroundjoin%
\definecolor{currentfill}{rgb}{0.343142,0.118134,0.311397}%
\pgfsetfillcolor{currentfill}%
\pgfsetlinewidth{0.000000pt}%
\definecolor{currentstroke}{rgb}{0.000000,0.000000,0.000000}%
\pgfsetstrokecolor{currentstroke}%
\pgfsetdash{}{0pt}%
\pgfpathmoveto{\pgfqpoint{0.384371in}{0.958818in}}%
\pgfpathlineto{\pgfqpoint{0.385981in}{0.961613in}}%
\pgfpathlineto{\pgfqpoint{0.384371in}{0.966404in}}%
\pgfpathlineto{\pgfqpoint{0.382198in}{0.961613in}}%
\pgfpathclose%
\pgfusepath{fill}%
\end{pgfscope}%
\begin{pgfscope}%
\pgfpathrectangle{\pgfqpoint{0.211875in}{0.211875in}}{\pgfqpoint{1.313625in}{1.279725in}}%
\pgfusepath{clip}%
\pgfsetbuttcap%
\pgfsetroundjoin%
\definecolor{currentfill}{rgb}{0.343142,0.118134,0.311397}%
\pgfsetfillcolor{currentfill}%
\pgfsetlinewidth{0.000000pt}%
\definecolor{currentstroke}{rgb}{0.000000,0.000000,0.000000}%
\pgfsetstrokecolor{currentstroke}%
\pgfsetdash{}{0pt}%
\pgfpathmoveto{\pgfqpoint{0.217483in}{1.039172in}}%
\pgfpathlineto{\pgfqpoint{0.216984in}{1.052098in}}%
\pgfpathlineto{\pgfqpoint{0.211875in}{1.060304in}}%
\pgfpathlineto{\pgfqpoint{0.211875in}{1.052098in}}%
\pgfpathlineto{\pgfqpoint{0.211875in}{1.039172in}}%
\pgfpathlineto{\pgfqpoint{0.211875in}{1.028627in}}%
\pgfpathclose%
\pgfusepath{fill}%
\end{pgfscope}%
\begin{pgfscope}%
\pgfpathrectangle{\pgfqpoint{0.211875in}{0.211875in}}{\pgfqpoint{1.313625in}{1.279725in}}%
\pgfusepath{clip}%
\pgfsetbuttcap%
\pgfsetroundjoin%
\definecolor{currentfill}{rgb}{0.343142,0.118134,0.311397}%
\pgfsetfillcolor{currentfill}%
\pgfsetlinewidth{0.000000pt}%
\definecolor{currentstroke}{rgb}{0.000000,0.000000,0.000000}%
\pgfsetstrokecolor{currentstroke}%
\pgfsetdash{}{0pt}%
\pgfpathmoveto{\pgfqpoint{0.318027in}{1.038475in}}%
\pgfpathlineto{\pgfqpoint{0.331295in}{1.039129in}}%
\pgfpathlineto{\pgfqpoint{0.331316in}{1.039172in}}%
\pgfpathlineto{\pgfqpoint{0.331295in}{1.039678in}}%
\pgfpathlineto{\pgfqpoint{0.318027in}{1.047340in}}%
\pgfpathlineto{\pgfqpoint{0.317682in}{1.039172in}}%
\pgfpathclose%
\pgfusepath{fill}%
\end{pgfscope}%
\begin{pgfscope}%
\pgfpathrectangle{\pgfqpoint{0.211875in}{0.211875in}}{\pgfqpoint{1.313625in}{1.279725in}}%
\pgfusepath{clip}%
\pgfsetbuttcap%
\pgfsetroundjoin%
\definecolor{currentfill}{rgb}{0.343142,0.118134,0.311397}%
\pgfsetfillcolor{currentfill}%
\pgfsetlinewidth{0.000000pt}%
\definecolor{currentstroke}{rgb}{0.000000,0.000000,0.000000}%
\pgfsetstrokecolor{currentstroke}%
\pgfsetdash{}{0pt}%
\pgfpathmoveto{\pgfqpoint{0.264951in}{1.112500in}}%
\pgfpathlineto{\pgfqpoint{0.270726in}{1.116731in}}%
\pgfpathlineto{\pgfqpoint{0.273584in}{1.129658in}}%
\pgfpathlineto{\pgfqpoint{0.264951in}{1.139574in}}%
\pgfpathlineto{\pgfqpoint{0.257285in}{1.129658in}}%
\pgfpathlineto{\pgfqpoint{0.259782in}{1.116731in}}%
\pgfpathclose%
\pgfusepath{fill}%
\end{pgfscope}%
\begin{pgfscope}%
\pgfpathrectangle{\pgfqpoint{0.211875in}{0.211875in}}{\pgfqpoint{1.313625in}{1.279725in}}%
\pgfusepath{clip}%
\pgfsetbuttcap%
\pgfsetroundjoin%
\definecolor{currentfill}{rgb}{0.343142,0.118134,0.311397}%
\pgfsetfillcolor{currentfill}%
\pgfsetlinewidth{0.000000pt}%
\definecolor{currentstroke}{rgb}{0.000000,0.000000,0.000000}%
\pgfsetstrokecolor{currentstroke}%
\pgfsetdash{}{0pt}%
\pgfpathmoveto{\pgfqpoint{0.384371in}{1.128282in}}%
\pgfpathlineto{\pgfqpoint{0.384645in}{1.129658in}}%
\pgfpathlineto{\pgfqpoint{0.384371in}{1.130042in}}%
\pgfpathlineto{\pgfqpoint{0.384001in}{1.129658in}}%
\pgfpathclose%
\pgfusepath{fill}%
\end{pgfscope}%
\begin{pgfscope}%
\pgfpathrectangle{\pgfqpoint{0.211875in}{0.211875in}}{\pgfqpoint{1.313625in}{1.279725in}}%
\pgfusepath{clip}%
\pgfsetbuttcap%
\pgfsetroundjoin%
\definecolor{currentfill}{rgb}{0.343142,0.118134,0.311397}%
\pgfsetfillcolor{currentfill}%
\pgfsetlinewidth{0.000000pt}%
\definecolor{currentstroke}{rgb}{0.000000,0.000000,0.000000}%
\pgfsetstrokecolor{currentstroke}%
\pgfsetdash{}{0pt}%
\pgfpathmoveto{\pgfqpoint{0.214593in}{1.194290in}}%
\pgfpathlineto{\pgfqpoint{0.218042in}{1.207217in}}%
\pgfpathlineto{\pgfqpoint{0.214672in}{1.220143in}}%
\pgfpathlineto{\pgfqpoint{0.211875in}{1.223255in}}%
\pgfpathlineto{\pgfqpoint{0.211875in}{1.220143in}}%
\pgfpathlineto{\pgfqpoint{0.211875in}{1.207217in}}%
\pgfpathlineto{\pgfqpoint{0.211875in}{1.194290in}}%
\pgfpathlineto{\pgfqpoint{0.211875in}{1.191285in}}%
\pgfpathclose%
\pgfusepath{fill}%
\end{pgfscope}%
\begin{pgfscope}%
\pgfpathrectangle{\pgfqpoint{0.211875in}{0.211875in}}{\pgfqpoint{1.313625in}{1.279725in}}%
\pgfusepath{clip}%
\pgfsetbuttcap%
\pgfsetroundjoin%
\definecolor{currentfill}{rgb}{0.343142,0.118134,0.311397}%
\pgfsetfillcolor{currentfill}%
\pgfsetlinewidth{0.000000pt}%
\definecolor{currentstroke}{rgb}{0.000000,0.000000,0.000000}%
\pgfsetstrokecolor{currentstroke}%
\pgfsetdash{}{0pt}%
\pgfpathmoveto{\pgfqpoint{0.318027in}{1.203632in}}%
\pgfpathlineto{\pgfqpoint{0.331295in}{1.204797in}}%
\pgfpathlineto{\pgfqpoint{0.331921in}{1.207217in}}%
\pgfpathlineto{\pgfqpoint{0.331295in}{1.209682in}}%
\pgfpathlineto{\pgfqpoint{0.318027in}{1.210869in}}%
\pgfpathlineto{\pgfqpoint{0.317057in}{1.207217in}}%
\pgfpathclose%
\pgfusepath{fill}%
\end{pgfscope}%
\begin{pgfscope}%
\pgfpathrectangle{\pgfqpoint{0.211875in}{0.211875in}}{\pgfqpoint{1.313625in}{1.279725in}}%
\pgfusepath{clip}%
\pgfsetbuttcap%
\pgfsetroundjoin%
\definecolor{currentfill}{rgb}{0.343142,0.118134,0.311397}%
\pgfsetfillcolor{currentfill}%
\pgfsetlinewidth{0.000000pt}%
\definecolor{currentstroke}{rgb}{0.000000,0.000000,0.000000}%
\pgfsetstrokecolor{currentstroke}%
\pgfsetdash{}{0pt}%
\pgfpathmoveto{\pgfqpoint{0.264951in}{1.273674in}}%
\pgfpathlineto{\pgfqpoint{0.274723in}{1.284776in}}%
\pgfpathlineto{\pgfqpoint{0.272061in}{1.297702in}}%
\pgfpathlineto{\pgfqpoint{0.264951in}{1.302949in}}%
\pgfpathlineto{\pgfqpoint{0.258586in}{1.297702in}}%
\pgfpathlineto{\pgfqpoint{0.256274in}{1.284776in}}%
\pgfpathclose%
\pgfusepath{fill}%
\end{pgfscope}%
\begin{pgfscope}%
\pgfpathrectangle{\pgfqpoint{0.211875in}{0.211875in}}{\pgfqpoint{1.313625in}{1.279725in}}%
\pgfusepath{clip}%
\pgfsetbuttcap%
\pgfsetroundjoin%
\definecolor{currentfill}{rgb}{0.343142,0.118134,0.311397}%
\pgfsetfillcolor{currentfill}%
\pgfsetlinewidth{0.000000pt}%
\definecolor{currentstroke}{rgb}{0.000000,0.000000,0.000000}%
\pgfsetstrokecolor{currentstroke}%
\pgfsetdash{}{0pt}%
\pgfpathmoveto{\pgfqpoint{0.384371in}{1.283100in}}%
\pgfpathlineto{\pgfqpoint{0.385578in}{1.284776in}}%
\pgfpathlineto{\pgfqpoint{0.384371in}{1.291087in}}%
\pgfpathlineto{\pgfqpoint{0.382740in}{1.284776in}}%
\pgfpathclose%
\pgfusepath{fill}%
\end{pgfscope}%
\begin{pgfscope}%
\pgfpathrectangle{\pgfqpoint{0.211875in}{0.211875in}}{\pgfqpoint{1.313625in}{1.279725in}}%
\pgfusepath{clip}%
\pgfsetbuttcap%
\pgfsetroundjoin%
\definecolor{currentfill}{rgb}{0.343142,0.118134,0.311397}%
\pgfsetfillcolor{currentfill}%
\pgfsetlinewidth{0.000000pt}%
\definecolor{currentstroke}{rgb}{0.000000,0.000000,0.000000}%
\pgfsetstrokecolor{currentstroke}%
\pgfsetdash{}{0pt}%
\pgfpathmoveto{\pgfqpoint{0.218241in}{1.362335in}}%
\pgfpathlineto{\pgfqpoint{0.218794in}{1.375261in}}%
\pgfpathlineto{\pgfqpoint{0.212018in}{1.388188in}}%
\pgfpathlineto{\pgfqpoint{0.211875in}{1.388303in}}%
\pgfpathlineto{\pgfqpoint{0.211875in}{1.388188in}}%
\pgfpathlineto{\pgfqpoint{0.211875in}{1.375261in}}%
\pgfpathlineto{\pgfqpoint{0.211875in}{1.362335in}}%
\pgfpathlineto{\pgfqpoint{0.211875in}{1.352198in}}%
\pgfpathclose%
\pgfusepath{fill}%
\end{pgfscope}%
\begin{pgfscope}%
\pgfpathrectangle{\pgfqpoint{0.211875in}{0.211875in}}{\pgfqpoint{1.313625in}{1.279725in}}%
\pgfusepath{clip}%
\pgfsetbuttcap%
\pgfsetroundjoin%
\definecolor{currentfill}{rgb}{0.343142,0.118134,0.311397}%
\pgfsetfillcolor{currentfill}%
\pgfsetlinewidth{0.000000pt}%
\definecolor{currentstroke}{rgb}{0.000000,0.000000,0.000000}%
\pgfsetstrokecolor{currentstroke}%
\pgfsetdash{}{0pt}%
\pgfpathmoveto{\pgfqpoint{0.318027in}{1.360617in}}%
\pgfpathlineto{\pgfqpoint{0.331295in}{1.361190in}}%
\pgfpathlineto{\pgfqpoint{0.331934in}{1.362335in}}%
\pgfpathlineto{\pgfqpoint{0.332506in}{1.375261in}}%
\pgfpathlineto{\pgfqpoint{0.331295in}{1.377855in}}%
\pgfpathlineto{\pgfqpoint{0.318027in}{1.378530in}}%
\pgfpathlineto{\pgfqpoint{0.316427in}{1.375261in}}%
\pgfpathlineto{\pgfqpoint{0.317021in}{1.362335in}}%
\pgfpathclose%
\pgfusepath{fill}%
\end{pgfscope}%
\begin{pgfscope}%
\pgfpathrectangle{\pgfqpoint{0.211875in}{0.211875in}}{\pgfqpoint{1.313625in}{1.279725in}}%
\pgfusepath{clip}%
\pgfsetbuttcap%
\pgfsetroundjoin%
\definecolor{currentfill}{rgb}{0.343142,0.118134,0.311397}%
\pgfsetfillcolor{currentfill}%
\pgfsetlinewidth{0.000000pt}%
\definecolor{currentstroke}{rgb}{0.000000,0.000000,0.000000}%
\pgfsetstrokecolor{currentstroke}%
\pgfsetdash{}{0pt}%
\pgfpathmoveto{\pgfqpoint{0.264951in}{1.434124in}}%
\pgfpathlineto{\pgfqpoint{0.273971in}{1.439894in}}%
\pgfpathlineto{\pgfqpoint{0.278220in}{1.451451in}}%
\pgfpathlineto{\pgfqpoint{0.278392in}{1.452820in}}%
\pgfpathlineto{\pgfqpoint{0.278220in}{1.453674in}}%
\pgfpathlineto{\pgfqpoint{0.270529in}{1.465747in}}%
\pgfpathlineto{\pgfqpoint{0.264951in}{1.468663in}}%
\pgfpathlineto{\pgfqpoint{0.259870in}{1.465747in}}%
\pgfpathlineto{\pgfqpoint{0.252770in}{1.452820in}}%
\pgfpathlineto{\pgfqpoint{0.256844in}{1.439894in}}%
\pgfpathclose%
\pgfusepath{fill}%
\end{pgfscope}%
\begin{pgfscope}%
\pgfpathrectangle{\pgfqpoint{0.211875in}{0.211875in}}{\pgfqpoint{1.313625in}{1.279725in}}%
\pgfusepath{clip}%
\pgfsetbuttcap%
\pgfsetroundjoin%
\definecolor{currentfill}{rgb}{0.343142,0.118134,0.311397}%
\pgfsetfillcolor{currentfill}%
\pgfsetlinewidth{0.000000pt}%
\definecolor{currentstroke}{rgb}{0.000000,0.000000,0.000000}%
\pgfsetstrokecolor{currentstroke}%
\pgfsetdash{}{0pt}%
\pgfpathmoveto{\pgfqpoint{0.384371in}{1.439855in}}%
\pgfpathlineto{\pgfqpoint{0.384421in}{1.439894in}}%
\pgfpathlineto{\pgfqpoint{0.388861in}{1.452820in}}%
\pgfpathlineto{\pgfqpoint{0.384371in}{1.460723in}}%
\pgfpathlineto{\pgfqpoint{0.378310in}{1.452820in}}%
\pgfpathlineto{\pgfqpoint{0.384303in}{1.439894in}}%
\pgfpathclose%
\pgfusepath{fill}%
\end{pgfscope}%
\begin{pgfscope}%
\pgfpathrectangle{\pgfqpoint{0.211875in}{0.211875in}}{\pgfqpoint{1.313625in}{1.279725in}}%
\pgfusepath{clip}%
\pgfsetbuttcap%
\pgfsetroundjoin%
\definecolor{currentfill}{rgb}{0.490838,0.119982,0.351115}%
\pgfsetfillcolor{currentfill}%
\pgfsetlinewidth{0.000000pt}%
\definecolor{currentstroke}{rgb}{0.000000,0.000000,0.000000}%
\pgfsetstrokecolor{currentstroke}%
\pgfsetdash{}{0pt}%
\pgfpathmoveto{\pgfqpoint{0.238413in}{0.211875in}}%
\pgfpathlineto{\pgfqpoint{0.239572in}{0.211875in}}%
\pgfpathlineto{\pgfqpoint{0.238413in}{0.223031in}}%
\pgfpathlineto{\pgfqpoint{0.238287in}{0.224802in}}%
\pgfpathlineto{\pgfqpoint{0.237900in}{0.237728in}}%
\pgfpathlineto{\pgfqpoint{0.237920in}{0.250655in}}%
\pgfpathlineto{\pgfqpoint{0.238413in}{0.260086in}}%
\pgfpathlineto{\pgfqpoint{0.238815in}{0.263581in}}%
\pgfpathlineto{\pgfqpoint{0.251682in}{0.272411in}}%
\pgfpathlineto{\pgfqpoint{0.264951in}{0.273052in}}%
\pgfpathlineto{\pgfqpoint{0.278220in}{0.273624in}}%
\pgfpathlineto{\pgfqpoint{0.290030in}{0.276508in}}%
\pgfpathlineto{\pgfqpoint{0.291489in}{0.278228in}}%
\pgfpathlineto{\pgfqpoint{0.293595in}{0.289434in}}%
\pgfpathlineto{\pgfqpoint{0.293880in}{0.302361in}}%
\pgfpathlineto{\pgfqpoint{0.293721in}{0.315287in}}%
\pgfpathlineto{\pgfqpoint{0.293080in}{0.328214in}}%
\pgfpathlineto{\pgfqpoint{0.291489in}{0.338702in}}%
\pgfpathlineto{\pgfqpoint{0.290762in}{0.341140in}}%
\pgfpathlineto{\pgfqpoint{0.278220in}{0.350570in}}%
\pgfpathlineto{\pgfqpoint{0.264951in}{0.351769in}}%
\pgfpathlineto{\pgfqpoint{0.251682in}{0.351448in}}%
\pgfpathlineto{\pgfqpoint{0.238413in}{0.341901in}}%
\pgfpathlineto{\pgfqpoint{0.238245in}{0.341140in}}%
\pgfpathlineto{\pgfqpoint{0.237014in}{0.328214in}}%
\pgfpathlineto{\pgfqpoint{0.236582in}{0.315287in}}%
\pgfpathlineto{\pgfqpoint{0.236260in}{0.302361in}}%
\pgfpathlineto{\pgfqpoint{0.235694in}{0.289434in}}%
\pgfpathlineto{\pgfqpoint{0.230127in}{0.276508in}}%
\pgfpathlineto{\pgfqpoint{0.225144in}{0.275339in}}%
\pgfpathlineto{\pgfqpoint{0.211875in}{0.274886in}}%
\pgfpathlineto{\pgfqpoint{0.211875in}{0.266982in}}%
\pgfpathlineto{\pgfqpoint{0.222425in}{0.263581in}}%
\pgfpathlineto{\pgfqpoint{0.225144in}{0.261817in}}%
\pgfpathlineto{\pgfqpoint{0.230413in}{0.250655in}}%
\pgfpathlineto{\pgfqpoint{0.232006in}{0.237728in}}%
\pgfpathlineto{\pgfqpoint{0.232252in}{0.224802in}}%
\pgfpathlineto{\pgfqpoint{0.231301in}{0.211875in}}%
\pgfpathclose%
\pgfpathmoveto{\pgfqpoint{0.246873in}{0.289434in}}%
\pgfpathlineto{\pgfqpoint{0.243395in}{0.302361in}}%
\pgfpathlineto{\pgfqpoint{0.243042in}{0.315287in}}%
\pgfpathlineto{\pgfqpoint{0.244730in}{0.328214in}}%
\pgfpathlineto{\pgfqpoint{0.251682in}{0.340658in}}%
\pgfpathlineto{\pgfqpoint{0.252869in}{0.341140in}}%
\pgfpathlineto{\pgfqpoint{0.264951in}{0.343897in}}%
\pgfpathlineto{\pgfqpoint{0.276726in}{0.341140in}}%
\pgfpathlineto{\pgfqpoint{0.278220in}{0.340563in}}%
\pgfpathlineto{\pgfqpoint{0.285588in}{0.328214in}}%
\pgfpathlineto{\pgfqpoint{0.287475in}{0.315287in}}%
\pgfpathlineto{\pgfqpoint{0.286998in}{0.302361in}}%
\pgfpathlineto{\pgfqpoint{0.282960in}{0.289434in}}%
\pgfpathlineto{\pgfqpoint{0.278220in}{0.284600in}}%
\pgfpathlineto{\pgfqpoint{0.264951in}{0.281569in}}%
\pgfpathlineto{\pgfqpoint{0.251682in}{0.284301in}}%
\pgfpathclose%
\pgfusepath{fill}%
\end{pgfscope}%
\begin{pgfscope}%
\pgfpathrectangle{\pgfqpoint{0.211875in}{0.211875in}}{\pgfqpoint{1.313625in}{1.279725in}}%
\pgfusepath{clip}%
\pgfsetbuttcap%
\pgfsetroundjoin%
\definecolor{currentfill}{rgb}{0.490838,0.119982,0.351115}%
\pgfsetfillcolor{currentfill}%
\pgfsetlinewidth{0.000000pt}%
\definecolor{currentstroke}{rgb}{0.000000,0.000000,0.000000}%
\pgfsetstrokecolor{currentstroke}%
\pgfsetdash{}{0pt}%
\pgfpathmoveto{\pgfqpoint{0.301114in}{0.224802in}}%
\pgfpathlineto{\pgfqpoint{0.301324in}{0.237728in}}%
\pgfpathlineto{\pgfqpoint{0.303494in}{0.250655in}}%
\pgfpathlineto{\pgfqpoint{0.304758in}{0.253611in}}%
\pgfpathlineto{\pgfqpoint{0.318027in}{0.263090in}}%
\pgfpathlineto{\pgfqpoint{0.331295in}{0.262647in}}%
\pgfpathlineto{\pgfqpoint{0.344564in}{0.251355in}}%
\pgfpathlineto{\pgfqpoint{0.344854in}{0.250655in}}%
\pgfpathlineto{\pgfqpoint{0.347197in}{0.237728in}}%
\pgfpathlineto{\pgfqpoint{0.347372in}{0.224802in}}%
\pgfpathlineto{\pgfqpoint{0.345410in}{0.211875in}}%
\pgfpathlineto{\pgfqpoint{0.353500in}{0.211875in}}%
\pgfpathlineto{\pgfqpoint{0.353330in}{0.224802in}}%
\pgfpathlineto{\pgfqpoint{0.353015in}{0.237728in}}%
\pgfpathlineto{\pgfqpoint{0.352273in}{0.250655in}}%
\pgfpathlineto{\pgfqpoint{0.349411in}{0.263581in}}%
\pgfpathlineto{\pgfqpoint{0.344564in}{0.268330in}}%
\pgfpathlineto{\pgfqpoint{0.331295in}{0.270909in}}%
\pgfpathlineto{\pgfqpoint{0.318027in}{0.271319in}}%
\pgfpathlineto{\pgfqpoint{0.304758in}{0.270590in}}%
\pgfpathlineto{\pgfqpoint{0.296863in}{0.263581in}}%
\pgfpathlineto{\pgfqpoint{0.295639in}{0.250655in}}%
\pgfpathlineto{\pgfqpoint{0.295245in}{0.237728in}}%
\pgfpathlineto{\pgfqpoint{0.294881in}{0.224802in}}%
\pgfpathlineto{\pgfqpoint{0.294169in}{0.211875in}}%
\pgfpathlineto{\pgfqpoint{0.302784in}{0.211875in}}%
\pgfpathclose%
\pgfusepath{fill}%
\end{pgfscope}%
\begin{pgfscope}%
\pgfpathrectangle{\pgfqpoint{0.211875in}{0.211875in}}{\pgfqpoint{1.313625in}{1.279725in}}%
\pgfusepath{clip}%
\pgfsetbuttcap%
\pgfsetroundjoin%
\definecolor{currentfill}{rgb}{0.490838,0.119982,0.351115}%
\pgfsetfillcolor{currentfill}%
\pgfsetlinewidth{0.000000pt}%
\definecolor{currentstroke}{rgb}{0.000000,0.000000,0.000000}%
\pgfsetstrokecolor{currentstroke}%
\pgfsetdash{}{0pt}%
\pgfpathmoveto{\pgfqpoint{0.424178in}{0.211875in}}%
\pgfpathlineto{\pgfqpoint{0.424975in}{0.211875in}}%
\pgfpathlineto{\pgfqpoint{0.424178in}{0.213579in}}%
\pgfpathlineto{\pgfqpoint{0.421907in}{0.224802in}}%
\pgfpathlineto{\pgfqpoint{0.422051in}{0.237728in}}%
\pgfpathlineto{\pgfqpoint{0.424178in}{0.247577in}}%
\pgfpathlineto{\pgfqpoint{0.425612in}{0.250655in}}%
\pgfpathlineto{\pgfqpoint{0.437447in}{0.259206in}}%
\pgfpathlineto{\pgfqpoint{0.450716in}{0.257914in}}%
\pgfpathlineto{\pgfqpoint{0.458148in}{0.250655in}}%
\pgfpathlineto{\pgfqpoint{0.462342in}{0.237728in}}%
\pgfpathlineto{\pgfqpoint{0.462504in}{0.224802in}}%
\pgfpathlineto{\pgfqpoint{0.458535in}{0.211875in}}%
\pgfpathlineto{\pgfqpoint{0.463985in}{0.211875in}}%
\pgfpathlineto{\pgfqpoint{0.468143in}{0.211875in}}%
\pgfpathlineto{\pgfqpoint{0.468851in}{0.224802in}}%
\pgfpathlineto{\pgfqpoint{0.468595in}{0.237728in}}%
\pgfpathlineto{\pgfqpoint{0.467199in}{0.250655in}}%
\pgfpathlineto{\pgfqpoint{0.463985in}{0.259666in}}%
\pgfpathlineto{\pgfqpoint{0.459988in}{0.263581in}}%
\pgfpathlineto{\pgfqpoint{0.450716in}{0.267253in}}%
\pgfpathlineto{\pgfqpoint{0.437447in}{0.267899in}}%
\pgfpathlineto{\pgfqpoint{0.424178in}{0.264882in}}%
\pgfpathlineto{\pgfqpoint{0.422505in}{0.263581in}}%
\pgfpathlineto{\pgfqpoint{0.417090in}{0.250655in}}%
\pgfpathlineto{\pgfqpoint{0.415896in}{0.237728in}}%
\pgfpathlineto{\pgfqpoint{0.415597in}{0.224802in}}%
\pgfpathlineto{\pgfqpoint{0.415944in}{0.211875in}}%
\pgfpathclose%
\pgfusepath{fill}%
\end{pgfscope}%
\begin{pgfscope}%
\pgfpathrectangle{\pgfqpoint{0.211875in}{0.211875in}}{\pgfqpoint{1.313625in}{1.279725in}}%
\pgfusepath{clip}%
\pgfsetbuttcap%
\pgfsetroundjoin%
\definecolor{currentfill}{rgb}{0.490838,0.119982,0.351115}%
\pgfsetfillcolor{currentfill}%
\pgfsetlinewidth{0.000000pt}%
\definecolor{currentstroke}{rgb}{0.000000,0.000000,0.000000}%
\pgfsetstrokecolor{currentstroke}%
\pgfsetdash{}{0pt}%
\pgfpathmoveto{\pgfqpoint{0.543598in}{0.211875in}}%
\pgfpathlineto{\pgfqpoint{0.548173in}{0.211875in}}%
\pgfpathlineto{\pgfqpoint{0.543598in}{0.219545in}}%
\pgfpathlineto{\pgfqpoint{0.542368in}{0.224802in}}%
\pgfpathlineto{\pgfqpoint{0.542455in}{0.237728in}}%
\pgfpathlineto{\pgfqpoint{0.543598in}{0.242434in}}%
\pgfpathlineto{\pgfqpoint{0.548349in}{0.250655in}}%
\pgfpathlineto{\pgfqpoint{0.556867in}{0.256029in}}%
\pgfpathlineto{\pgfqpoint{0.570136in}{0.253436in}}%
\pgfpathlineto{\pgfqpoint{0.572698in}{0.250655in}}%
\pgfpathlineto{\pgfqpoint{0.577351in}{0.237728in}}%
\pgfpathlineto{\pgfqpoint{0.577438in}{0.224802in}}%
\pgfpathlineto{\pgfqpoint{0.572750in}{0.211875in}}%
\pgfpathlineto{\pgfqpoint{0.583299in}{0.211875in}}%
\pgfpathlineto{\pgfqpoint{0.583405in}{0.212412in}}%
\pgfpathlineto{\pgfqpoint{0.584776in}{0.224802in}}%
\pgfpathlineto{\pgfqpoint{0.584569in}{0.237728in}}%
\pgfpathlineto{\pgfqpoint{0.583405in}{0.246168in}}%
\pgfpathlineto{\pgfqpoint{0.582369in}{0.250655in}}%
\pgfpathlineto{\pgfqpoint{0.570816in}{0.263581in}}%
\pgfpathlineto{\pgfqpoint{0.570136in}{0.263878in}}%
\pgfpathlineto{\pgfqpoint{0.556867in}{0.265123in}}%
\pgfpathlineto{\pgfqpoint{0.550295in}{0.263581in}}%
\pgfpathlineto{\pgfqpoint{0.543598in}{0.260629in}}%
\pgfpathlineto{\pgfqpoint{0.538082in}{0.250655in}}%
\pgfpathlineto{\pgfqpoint{0.536181in}{0.237728in}}%
\pgfpathlineto{\pgfqpoint{0.535935in}{0.224802in}}%
\pgfpathlineto{\pgfqpoint{0.537207in}{0.211875in}}%
\pgfpathclose%
\pgfusepath{fill}%
\end{pgfscope}%
\begin{pgfscope}%
\pgfpathrectangle{\pgfqpoint{0.211875in}{0.211875in}}{\pgfqpoint{1.313625in}{1.279725in}}%
\pgfusepath{clip}%
\pgfsetbuttcap%
\pgfsetroundjoin%
\definecolor{currentfill}{rgb}{0.490838,0.119982,0.351115}%
\pgfsetfillcolor{currentfill}%
\pgfsetlinewidth{0.000000pt}%
\definecolor{currentstroke}{rgb}{0.000000,0.000000,0.000000}%
\pgfsetstrokecolor{currentstroke}%
\pgfsetdash{}{0pt}%
\pgfpathmoveto{\pgfqpoint{0.663019in}{0.211875in}}%
\pgfpathlineto{\pgfqpoint{0.671410in}{0.211875in}}%
\pgfpathlineto{\pgfqpoint{0.663019in}{0.222883in}}%
\pgfpathlineto{\pgfqpoint{0.662506in}{0.224802in}}%
\pgfpathlineto{\pgfqpoint{0.662546in}{0.237728in}}%
\pgfpathlineto{\pgfqpoint{0.663019in}{0.239470in}}%
\pgfpathlineto{\pgfqpoint{0.671109in}{0.250655in}}%
\pgfpathlineto{\pgfqpoint{0.676288in}{0.253461in}}%
\pgfpathlineto{\pgfqpoint{0.685440in}{0.250655in}}%
\pgfpathlineto{\pgfqpoint{0.689557in}{0.247982in}}%
\pgfpathlineto{\pgfqpoint{0.693282in}{0.237728in}}%
\pgfpathlineto{\pgfqpoint{0.693314in}{0.224802in}}%
\pgfpathlineto{\pgfqpoint{0.689557in}{0.214704in}}%
\pgfpathlineto{\pgfqpoint{0.684822in}{0.211875in}}%
\pgfpathlineto{\pgfqpoint{0.689557in}{0.211875in}}%
\pgfpathlineto{\pgfqpoint{0.698095in}{0.211875in}}%
\pgfpathlineto{\pgfqpoint{0.700639in}{0.224802in}}%
\pgfpathlineto{\pgfqpoint{0.700432in}{0.237728in}}%
\pgfpathlineto{\pgfqpoint{0.697452in}{0.250655in}}%
\pgfpathlineto{\pgfqpoint{0.689557in}{0.260167in}}%
\pgfpathlineto{\pgfqpoint{0.676288in}{0.262761in}}%
\pgfpathlineto{\pgfqpoint{0.663019in}{0.257730in}}%
\pgfpathlineto{\pgfqpoint{0.658631in}{0.250655in}}%
\pgfpathlineto{\pgfqpoint{0.656105in}{0.237728in}}%
\pgfpathlineto{\pgfqpoint{0.655903in}{0.224802in}}%
\pgfpathlineto{\pgfqpoint{0.657978in}{0.211875in}}%
\pgfpathclose%
\pgfusepath{fill}%
\end{pgfscope}%
\begin{pgfscope}%
\pgfpathrectangle{\pgfqpoint{0.211875in}{0.211875in}}{\pgfqpoint{1.313625in}{1.279725in}}%
\pgfusepath{clip}%
\pgfsetbuttcap%
\pgfsetroundjoin%
\definecolor{currentfill}{rgb}{0.490838,0.119982,0.351115}%
\pgfsetfillcolor{currentfill}%
\pgfsetlinewidth{0.000000pt}%
\definecolor{currentstroke}{rgb}{0.000000,0.000000,0.000000}%
\pgfsetstrokecolor{currentstroke}%
\pgfsetdash{}{0pt}%
\pgfpathmoveto{\pgfqpoint{0.782439in}{0.211875in}}%
\pgfpathlineto{\pgfqpoint{0.794821in}{0.211875in}}%
\pgfpathlineto{\pgfqpoint{0.782439in}{0.224405in}}%
\pgfpathlineto{\pgfqpoint{0.782320in}{0.224802in}}%
\pgfpathlineto{\pgfqpoint{0.782321in}{0.237728in}}%
\pgfpathlineto{\pgfqpoint{0.782439in}{0.238121in}}%
\pgfpathlineto{\pgfqpoint{0.794000in}{0.250655in}}%
\pgfpathlineto{\pgfqpoint{0.795708in}{0.251431in}}%
\pgfpathlineto{\pgfqpoint{0.797675in}{0.250655in}}%
\pgfpathlineto{\pgfqpoint{0.808977in}{0.240637in}}%
\pgfpathlineto{\pgfqpoint{0.809931in}{0.237728in}}%
\pgfpathlineto{\pgfqpoint{0.809924in}{0.224802in}}%
\pgfpathlineto{\pgfqpoint{0.808977in}{0.221952in}}%
\pgfpathlineto{\pgfqpoint{0.796723in}{0.211875in}}%
\pgfpathlineto{\pgfqpoint{0.808977in}{0.211875in}}%
\pgfpathlineto{\pgfqpoint{0.813920in}{0.211875in}}%
\pgfpathlineto{\pgfqpoint{0.816898in}{0.224802in}}%
\pgfpathlineto{\pgfqpoint{0.816738in}{0.237728in}}%
\pgfpathlineto{\pgfqpoint{0.813484in}{0.250655in}}%
\pgfpathlineto{\pgfqpoint{0.808977in}{0.256676in}}%
\pgfpathlineto{\pgfqpoint{0.795708in}{0.260674in}}%
\pgfpathlineto{\pgfqpoint{0.782439in}{0.256012in}}%
\pgfpathlineto{\pgfqpoint{0.778734in}{0.250655in}}%
\pgfpathlineto{\pgfqpoint{0.775658in}{0.237728in}}%
\pgfpathlineto{\pgfqpoint{0.775490in}{0.224802in}}%
\pgfpathlineto{\pgfqpoint{0.778256in}{0.211875in}}%
\pgfpathclose%
\pgfusepath{fill}%
\end{pgfscope}%
\begin{pgfscope}%
\pgfpathrectangle{\pgfqpoint{0.211875in}{0.211875in}}{\pgfqpoint{1.313625in}{1.279725in}}%
\pgfusepath{clip}%
\pgfsetbuttcap%
\pgfsetroundjoin%
\definecolor{currentfill}{rgb}{0.490838,0.119982,0.351115}%
\pgfsetfillcolor{currentfill}%
\pgfsetlinewidth{0.000000pt}%
\definecolor{currentstroke}{rgb}{0.000000,0.000000,0.000000}%
\pgfsetstrokecolor{currentstroke}%
\pgfsetdash{}{0pt}%
\pgfpathmoveto{\pgfqpoint{0.901860in}{0.211875in}}%
\pgfpathlineto{\pgfqpoint{0.915129in}{0.211875in}}%
\pgfpathlineto{\pgfqpoint{0.928398in}{0.211875in}}%
\pgfpathlineto{\pgfqpoint{0.930583in}{0.211875in}}%
\pgfpathlineto{\pgfqpoint{0.933822in}{0.224802in}}%
\pgfpathlineto{\pgfqpoint{0.933693in}{0.237728in}}%
\pgfpathlineto{\pgfqpoint{0.930283in}{0.250655in}}%
\pgfpathlineto{\pgfqpoint{0.928398in}{0.253453in}}%
\pgfpathlineto{\pgfqpoint{0.915129in}{0.259133in}}%
\pgfpathlineto{\pgfqpoint{0.901860in}{0.255195in}}%
\pgfpathlineto{\pgfqpoint{0.898369in}{0.250655in}}%
\pgfpathlineto{\pgfqpoint{0.894815in}{0.237728in}}%
\pgfpathlineto{\pgfqpoint{0.894670in}{0.224802in}}%
\pgfpathlineto{\pgfqpoint{0.898019in}{0.211875in}}%
\pgfpathclose%
\pgfpathmoveto{\pgfqpoint{0.901790in}{0.224802in}}%
\pgfpathlineto{\pgfqpoint{0.901762in}{0.237728in}}%
\pgfpathlineto{\pgfqpoint{0.901860in}{0.238021in}}%
\pgfpathlineto{\pgfqpoint{0.915129in}{0.249276in}}%
\pgfpathlineto{\pgfqpoint{0.925548in}{0.237728in}}%
\pgfpathlineto{\pgfqpoint{0.925476in}{0.224802in}}%
\pgfpathlineto{\pgfqpoint{0.915129in}{0.213928in}}%
\pgfpathlineto{\pgfqpoint{0.901860in}{0.224597in}}%
\pgfpathclose%
\pgfusepath{fill}%
\end{pgfscope}%
\begin{pgfscope}%
\pgfpathrectangle{\pgfqpoint{0.211875in}{0.211875in}}{\pgfqpoint{1.313625in}{1.279725in}}%
\pgfusepath{clip}%
\pgfsetbuttcap%
\pgfsetroundjoin%
\definecolor{currentfill}{rgb}{0.490838,0.119982,0.351115}%
\pgfsetfillcolor{currentfill}%
\pgfsetlinewidth{0.000000pt}%
\definecolor{currentstroke}{rgb}{0.000000,0.000000,0.000000}%
\pgfsetstrokecolor{currentstroke}%
\pgfsetdash{}{0pt}%
\pgfpathmoveto{\pgfqpoint{1.021280in}{0.211875in}}%
\pgfpathlineto{\pgfqpoint{1.034549in}{0.211875in}}%
\pgfpathlineto{\pgfqpoint{1.047818in}{0.211875in}}%
\pgfpathlineto{\pgfqpoint{1.047948in}{0.211875in}}%
\pgfpathlineto{\pgfqpoint{1.051295in}{0.224802in}}%
\pgfpathlineto{\pgfqpoint{1.051184in}{0.237728in}}%
\pgfpathlineto{\pgfqpoint{1.047818in}{0.250382in}}%
\pgfpathlineto{\pgfqpoint{1.047631in}{0.250655in}}%
\pgfpathlineto{\pgfqpoint{1.034549in}{0.258110in}}%
\pgfpathlineto{\pgfqpoint{1.021280in}{0.255096in}}%
\pgfpathlineto{\pgfqpoint{1.017488in}{0.250655in}}%
\pgfpathlineto{\pgfqpoint{1.013526in}{0.237728in}}%
\pgfpathlineto{\pgfqpoint{1.013394in}{0.224802in}}%
\pgfpathlineto{\pgfqpoint{1.017215in}{0.211875in}}%
\pgfpathclose%
\pgfpathmoveto{\pgfqpoint{1.020881in}{0.224802in}}%
\pgfpathlineto{\pgfqpoint{1.020832in}{0.237728in}}%
\pgfpathlineto{\pgfqpoint{1.021280in}{0.238935in}}%
\pgfpathlineto{\pgfqpoint{1.034549in}{0.247330in}}%
\pgfpathlineto{\pgfqpoint{1.041858in}{0.237728in}}%
\pgfpathlineto{\pgfqpoint{1.041768in}{0.224802in}}%
\pgfpathlineto{\pgfqpoint{1.034549in}{0.215803in}}%
\pgfpathlineto{\pgfqpoint{1.021280in}{0.223754in}}%
\pgfpathclose%
\pgfusepath{fill}%
\end{pgfscope}%
\begin{pgfscope}%
\pgfpathrectangle{\pgfqpoint{0.211875in}{0.211875in}}{\pgfqpoint{1.313625in}{1.279725in}}%
\pgfusepath{clip}%
\pgfsetbuttcap%
\pgfsetroundjoin%
\definecolor{currentfill}{rgb}{0.490838,0.119982,0.351115}%
\pgfsetfillcolor{currentfill}%
\pgfsetlinewidth{0.000000pt}%
\definecolor{currentstroke}{rgb}{0.000000,0.000000,0.000000}%
\pgfsetstrokecolor{currentstroke}%
\pgfsetdash{}{0pt}%
\pgfpathmoveto{\pgfqpoint{1.140701in}{0.211875in}}%
\pgfpathlineto{\pgfqpoint{1.153970in}{0.211875in}}%
\pgfpathlineto{\pgfqpoint{1.164857in}{0.211875in}}%
\pgfpathlineto{\pgfqpoint{1.167239in}{0.216074in}}%
\pgfpathlineto{\pgfqpoint{1.169240in}{0.224802in}}%
\pgfpathlineto{\pgfqpoint{1.169135in}{0.237728in}}%
\pgfpathlineto{\pgfqpoint{1.167239in}{0.245690in}}%
\pgfpathlineto{\pgfqpoint{1.164497in}{0.250655in}}%
\pgfpathlineto{\pgfqpoint{1.153970in}{0.257596in}}%
\pgfpathlineto{\pgfqpoint{1.140701in}{0.255598in}}%
\pgfpathlineto{\pgfqpoint{1.136011in}{0.250655in}}%
\pgfpathlineto{\pgfqpoint{1.131714in}{0.237728in}}%
\pgfpathlineto{\pgfqpoint{1.131582in}{0.224802in}}%
\pgfpathlineto{\pgfqpoint{1.135760in}{0.211875in}}%
\pgfpathclose%
\pgfpathmoveto{\pgfqpoint{1.139531in}{0.224802in}}%
\pgfpathlineto{\pgfqpoint{1.139471in}{0.237728in}}%
\pgfpathlineto{\pgfqpoint{1.140701in}{0.240708in}}%
\pgfpathlineto{\pgfqpoint{1.153970in}{0.246195in}}%
\pgfpathlineto{\pgfqpoint{1.159531in}{0.237728in}}%
\pgfpathlineto{\pgfqpoint{1.159445in}{0.224802in}}%
\pgfpathlineto{\pgfqpoint{1.153970in}{0.216878in}}%
\pgfpathlineto{\pgfqpoint{1.140701in}{0.222059in}}%
\pgfpathclose%
\pgfusepath{fill}%
\end{pgfscope}%
\begin{pgfscope}%
\pgfpathrectangle{\pgfqpoint{0.211875in}{0.211875in}}{\pgfqpoint{1.313625in}{1.279725in}}%
\pgfusepath{clip}%
\pgfsetbuttcap%
\pgfsetroundjoin%
\definecolor{currentfill}{rgb}{0.490838,0.119982,0.351115}%
\pgfsetfillcolor{currentfill}%
\pgfsetlinewidth{0.000000pt}%
\definecolor{currentstroke}{rgb}{0.000000,0.000000,0.000000}%
\pgfsetstrokecolor{currentstroke}%
\pgfsetdash{}{0pt}%
\pgfpathmoveto{\pgfqpoint{1.260121in}{0.211875in}}%
\pgfpathlineto{\pgfqpoint{1.273390in}{0.211875in}}%
\pgfpathlineto{\pgfqpoint{1.283007in}{0.211875in}}%
\pgfpathlineto{\pgfqpoint{1.286659in}{0.220049in}}%
\pgfpathlineto{\pgfqpoint{1.287601in}{0.224802in}}%
\pgfpathlineto{\pgfqpoint{1.287491in}{0.237728in}}%
\pgfpathlineto{\pgfqpoint{1.286659in}{0.241658in}}%
\pgfpathlineto{\pgfqpoint{1.282636in}{0.250655in}}%
\pgfpathlineto{\pgfqpoint{1.273390in}{0.257603in}}%
\pgfpathlineto{\pgfqpoint{1.260121in}{0.256624in}}%
\pgfpathlineto{\pgfqpoint{1.253812in}{0.250655in}}%
\pgfpathlineto{\pgfqpoint{1.249259in}{0.237728in}}%
\pgfpathlineto{\pgfqpoint{1.249113in}{0.224802in}}%
\pgfpathlineto{\pgfqpoint{1.253519in}{0.211875in}}%
\pgfpathclose%
\pgfpathmoveto{\pgfqpoint{1.257644in}{0.224802in}}%
\pgfpathlineto{\pgfqpoint{1.257585in}{0.237728in}}%
\pgfpathlineto{\pgfqpoint{1.260121in}{0.243247in}}%
\pgfpathlineto{\pgfqpoint{1.273390in}{0.245884in}}%
\pgfpathlineto{\pgfqpoint{1.278086in}{0.237728in}}%
\pgfpathlineto{\pgfqpoint{1.278019in}{0.224802in}}%
\pgfpathlineto{\pgfqpoint{1.273390in}{0.217138in}}%
\pgfpathlineto{\pgfqpoint{1.260121in}{0.219619in}}%
\pgfpathclose%
\pgfusepath{fill}%
\end{pgfscope}%
\begin{pgfscope}%
\pgfpathrectangle{\pgfqpoint{0.211875in}{0.211875in}}{\pgfqpoint{1.313625in}{1.279725in}}%
\pgfusepath{clip}%
\pgfsetbuttcap%
\pgfsetroundjoin%
\definecolor{currentfill}{rgb}{0.490838,0.119982,0.351115}%
\pgfsetfillcolor{currentfill}%
\pgfsetlinewidth{0.000000pt}%
\definecolor{currentstroke}{rgb}{0.000000,0.000000,0.000000}%
\pgfsetstrokecolor{currentstroke}%
\pgfsetdash{}{0pt}%
\pgfpathmoveto{\pgfqpoint{1.366273in}{0.223016in}}%
\pgfpathlineto{\pgfqpoint{1.370283in}{0.211875in}}%
\pgfpathlineto{\pgfqpoint{1.379542in}{0.211875in}}%
\pgfpathlineto{\pgfqpoint{1.392811in}{0.211875in}}%
\pgfpathlineto{\pgfqpoint{1.402124in}{0.211875in}}%
\pgfpathlineto{\pgfqpoint{1.406080in}{0.223228in}}%
\pgfpathlineto{\pgfqpoint{1.406343in}{0.224802in}}%
\pgfpathlineto{\pgfqpoint{1.406219in}{0.237728in}}%
\pgfpathlineto{\pgfqpoint{1.406080in}{0.238476in}}%
\pgfpathlineto{\pgfqpoint{1.401669in}{0.250655in}}%
\pgfpathlineto{\pgfqpoint{1.392811in}{0.258163in}}%
\pgfpathlineto{\pgfqpoint{1.379542in}{0.258130in}}%
\pgfpathlineto{\pgfqpoint{1.370698in}{0.250655in}}%
\pgfpathlineto{\pgfqpoint{1.366273in}{0.238736in}}%
\pgfpathlineto{\pgfqpoint{1.366081in}{0.237728in}}%
\pgfpathlineto{\pgfqpoint{1.365965in}{0.224802in}}%
\pgfpathclose%
\pgfpathmoveto{\pgfqpoint{1.375071in}{0.224802in}}%
\pgfpathlineto{\pgfqpoint{1.375027in}{0.237728in}}%
\pgfpathlineto{\pgfqpoint{1.379542in}{0.246502in}}%
\pgfpathlineto{\pgfqpoint{1.392811in}{0.246445in}}%
\pgfpathlineto{\pgfqpoint{1.397259in}{0.237728in}}%
\pgfpathlineto{\pgfqpoint{1.397221in}{0.224802in}}%
\pgfpathlineto{\pgfqpoint{1.392811in}{0.216529in}}%
\pgfpathlineto{\pgfqpoint{1.379542in}{0.216489in}}%
\pgfpathclose%
\pgfusepath{fill}%
\end{pgfscope}%
\begin{pgfscope}%
\pgfpathrectangle{\pgfqpoint{0.211875in}{0.211875in}}{\pgfqpoint{1.313625in}{1.279725in}}%
\pgfusepath{clip}%
\pgfsetbuttcap%
\pgfsetroundjoin%
\definecolor{currentfill}{rgb}{0.490838,0.119982,0.351115}%
\pgfsetfillcolor{currentfill}%
\pgfsetlinewidth{0.000000pt}%
\definecolor{currentstroke}{rgb}{0.000000,0.000000,0.000000}%
\pgfsetstrokecolor{currentstroke}%
\pgfsetdash{}{0pt}%
\pgfpathmoveto{\pgfqpoint{1.485693in}{0.211956in}}%
\pgfpathlineto{\pgfqpoint{1.485730in}{0.211875in}}%
\pgfpathlineto{\pgfqpoint{1.498962in}{0.211875in}}%
\pgfpathlineto{\pgfqpoint{1.512231in}{0.211875in}}%
\pgfpathlineto{\pgfqpoint{1.521974in}{0.211875in}}%
\pgfpathlineto{\pgfqpoint{1.525431in}{0.224802in}}%
\pgfpathlineto{\pgfqpoint{1.525220in}{0.237728in}}%
\pgfpathlineto{\pgfqpoint{1.521377in}{0.250655in}}%
\pgfpathlineto{\pgfqpoint{1.512231in}{0.259336in}}%
\pgfpathlineto{\pgfqpoint{1.498962in}{0.260098in}}%
\pgfpathlineto{\pgfqpoint{1.486368in}{0.250655in}}%
\pgfpathlineto{\pgfqpoint{1.485693in}{0.249179in}}%
\pgfpathlineto{\pgfqpoint{1.483231in}{0.237728in}}%
\pgfpathlineto{\pgfqpoint{1.483090in}{0.224802in}}%
\pgfpathclose%
\pgfpathmoveto{\pgfqpoint{1.491580in}{0.224802in}}%
\pgfpathlineto{\pgfqpoint{1.491571in}{0.237728in}}%
\pgfpathlineto{\pgfqpoint{1.498962in}{0.250457in}}%
\pgfpathlineto{\pgfqpoint{1.512231in}{0.247964in}}%
\pgfpathlineto{\pgfqpoint{1.516896in}{0.237728in}}%
\pgfpathlineto{\pgfqpoint{1.516897in}{0.224802in}}%
\pgfpathlineto{\pgfqpoint{1.512231in}{0.214950in}}%
\pgfpathlineto{\pgfqpoint{1.498962in}{0.212686in}}%
\pgfpathclose%
\pgfusepath{fill}%
\end{pgfscope}%
\begin{pgfscope}%
\pgfpathrectangle{\pgfqpoint{0.211875in}{0.211875in}}{\pgfqpoint{1.313625in}{1.279725in}}%
\pgfusepath{clip}%
\pgfsetbuttcap%
\pgfsetroundjoin%
\definecolor{currentfill}{rgb}{0.490838,0.119982,0.351115}%
\pgfsetfillcolor{currentfill}%
\pgfsetlinewidth{0.000000pt}%
\definecolor{currentstroke}{rgb}{0.000000,0.000000,0.000000}%
\pgfsetstrokecolor{currentstroke}%
\pgfsetdash{}{0pt}%
\pgfpathmoveto{\pgfqpoint{0.371102in}{0.277533in}}%
\pgfpathlineto{\pgfqpoint{0.384371in}{0.276891in}}%
\pgfpathlineto{\pgfqpoint{0.397640in}{0.278972in}}%
\pgfpathlineto{\pgfqpoint{0.406841in}{0.289434in}}%
\pgfpathlineto{\pgfqpoint{0.408753in}{0.302361in}}%
\pgfpathlineto{\pgfqpoint{0.408860in}{0.315287in}}%
\pgfpathlineto{\pgfqpoint{0.407655in}{0.328214in}}%
\pgfpathlineto{\pgfqpoint{0.402857in}{0.341140in}}%
\pgfpathlineto{\pgfqpoint{0.397640in}{0.345481in}}%
\pgfpathlineto{\pgfqpoint{0.384371in}{0.348100in}}%
\pgfpathlineto{\pgfqpoint{0.371102in}{0.346904in}}%
\pgfpathlineto{\pgfqpoint{0.362741in}{0.341140in}}%
\pgfpathlineto{\pgfqpoint{0.358348in}{0.328214in}}%
\pgfpathlineto{\pgfqpoint{0.357833in}{0.322655in}}%
\pgfpathlineto{\pgfqpoint{0.357334in}{0.315287in}}%
\pgfpathlineto{\pgfqpoint{0.357320in}{0.302361in}}%
\pgfpathlineto{\pgfqpoint{0.357833in}{0.295376in}}%
\pgfpathlineto{\pgfqpoint{0.358619in}{0.289434in}}%
\pgfpathclose%
\pgfpathmoveto{\pgfqpoint{0.370499in}{0.289434in}}%
\pgfpathlineto{\pgfqpoint{0.364870in}{0.302361in}}%
\pgfpathlineto{\pgfqpoint{0.364155in}{0.315287in}}%
\pgfpathlineto{\pgfqpoint{0.366515in}{0.328214in}}%
\pgfpathlineto{\pgfqpoint{0.371102in}{0.335590in}}%
\pgfpathlineto{\pgfqpoint{0.384371in}{0.339942in}}%
\pgfpathlineto{\pgfqpoint{0.397640in}{0.333002in}}%
\pgfpathlineto{\pgfqpoint{0.400214in}{0.328214in}}%
\pgfpathlineto{\pgfqpoint{0.402520in}{0.315287in}}%
\pgfpathlineto{\pgfqpoint{0.401776in}{0.302361in}}%
\pgfpathlineto{\pgfqpoint{0.397640in}{0.291740in}}%
\pgfpathlineto{\pgfqpoint{0.394594in}{0.289434in}}%
\pgfpathlineto{\pgfqpoint{0.384371in}{0.285650in}}%
\pgfpathlineto{\pgfqpoint{0.371102in}{0.288859in}}%
\pgfpathclose%
\pgfusepath{fill}%
\end{pgfscope}%
\begin{pgfscope}%
\pgfpathrectangle{\pgfqpoint{0.211875in}{0.211875in}}{\pgfqpoint{1.313625in}{1.279725in}}%
\pgfusepath{clip}%
\pgfsetbuttcap%
\pgfsetroundjoin%
\definecolor{currentfill}{rgb}{0.490838,0.119982,0.351115}%
\pgfsetfillcolor{currentfill}%
\pgfsetlinewidth{0.000000pt}%
\definecolor{currentstroke}{rgb}{0.000000,0.000000,0.000000}%
\pgfsetstrokecolor{currentstroke}%
\pgfsetdash{}{0pt}%
\pgfpathmoveto{\pgfqpoint{0.490523in}{0.281578in}}%
\pgfpathlineto{\pgfqpoint{0.503792in}{0.280433in}}%
\pgfpathlineto{\pgfqpoint{0.517061in}{0.284623in}}%
\pgfpathlineto{\pgfqpoint{0.520835in}{0.289434in}}%
\pgfpathlineto{\pgfqpoint{0.524019in}{0.302361in}}%
\pgfpathlineto{\pgfqpoint{0.524358in}{0.315287in}}%
\pgfpathlineto{\pgfqpoint{0.522784in}{0.328214in}}%
\pgfpathlineto{\pgfqpoint{0.517061in}{0.340073in}}%
\pgfpathlineto{\pgfqpoint{0.515328in}{0.341140in}}%
\pgfpathlineto{\pgfqpoint{0.503792in}{0.344907in}}%
\pgfpathlineto{\pgfqpoint{0.490523in}{0.343483in}}%
\pgfpathlineto{\pgfqpoint{0.486750in}{0.341140in}}%
\pgfpathlineto{\pgfqpoint{0.479451in}{0.328214in}}%
\pgfpathlineto{\pgfqpoint{0.477761in}{0.315287in}}%
\pgfpathlineto{\pgfqpoint{0.478056in}{0.302361in}}%
\pgfpathlineto{\pgfqpoint{0.481321in}{0.289434in}}%
\pgfpathclose%
\pgfpathmoveto{\pgfqpoint{0.502791in}{0.289434in}}%
\pgfpathlineto{\pgfqpoint{0.490523in}{0.293584in}}%
\pgfpathlineto{\pgfqpoint{0.486078in}{0.302361in}}%
\pgfpathlineto{\pgfqpoint{0.485016in}{0.315287in}}%
\pgfpathlineto{\pgfqpoint{0.488036in}{0.328214in}}%
\pgfpathlineto{\pgfqpoint{0.490523in}{0.331810in}}%
\pgfpathlineto{\pgfqpoint{0.503792in}{0.335670in}}%
\pgfpathlineto{\pgfqpoint{0.514097in}{0.328214in}}%
\pgfpathlineto{\pgfqpoint{0.517061in}{0.321499in}}%
\pgfpathlineto{\pgfqpoint{0.518226in}{0.315287in}}%
\pgfpathlineto{\pgfqpoint{0.517268in}{0.302361in}}%
\pgfpathlineto{\pgfqpoint{0.517061in}{0.301762in}}%
\pgfpathlineto{\pgfqpoint{0.504226in}{0.289434in}}%
\pgfpathlineto{\pgfqpoint{0.503792in}{0.289246in}}%
\pgfpathclose%
\pgfusepath{fill}%
\end{pgfscope}%
\begin{pgfscope}%
\pgfpathrectangle{\pgfqpoint{0.211875in}{0.211875in}}{\pgfqpoint{1.313625in}{1.279725in}}%
\pgfusepath{clip}%
\pgfsetbuttcap%
\pgfsetroundjoin%
\definecolor{currentfill}{rgb}{0.490838,0.119982,0.351115}%
\pgfsetfillcolor{currentfill}%
\pgfsetlinewidth{0.000000pt}%
\definecolor{currentstroke}{rgb}{0.000000,0.000000,0.000000}%
\pgfsetstrokecolor{currentstroke}%
\pgfsetdash{}{0pt}%
\pgfpathmoveto{\pgfqpoint{0.609943in}{0.284561in}}%
\pgfpathlineto{\pgfqpoint{0.623212in}{0.283492in}}%
\pgfpathlineto{\pgfqpoint{0.635128in}{0.289434in}}%
\pgfpathlineto{\pgfqpoint{0.636481in}{0.291067in}}%
\pgfpathlineto{\pgfqpoint{0.639920in}{0.302361in}}%
\pgfpathlineto{\pgfqpoint{0.640442in}{0.315287in}}%
\pgfpathlineto{\pgfqpoint{0.638576in}{0.328214in}}%
\pgfpathlineto{\pgfqpoint{0.636481in}{0.333076in}}%
\pgfpathlineto{\pgfqpoint{0.625878in}{0.341140in}}%
\pgfpathlineto{\pgfqpoint{0.623212in}{0.342137in}}%
\pgfpathlineto{\pgfqpoint{0.611917in}{0.341140in}}%
\pgfpathlineto{\pgfqpoint{0.609943in}{0.340883in}}%
\pgfpathlineto{\pgfqpoint{0.600194in}{0.328214in}}%
\pgfpathlineto{\pgfqpoint{0.597919in}{0.315287in}}%
\pgfpathlineto{\pgfqpoint{0.598510in}{0.302361in}}%
\pgfpathlineto{\pgfqpoint{0.603572in}{0.289434in}}%
\pgfpathclose%
\pgfpathmoveto{\pgfqpoint{0.607026in}{0.302361in}}%
\pgfpathlineto{\pgfqpoint{0.605624in}{0.315287in}}%
\pgfpathlineto{\pgfqpoint{0.609305in}{0.328214in}}%
\pgfpathlineto{\pgfqpoint{0.609943in}{0.329043in}}%
\pgfpathlineto{\pgfqpoint{0.623212in}{0.331891in}}%
\pgfpathlineto{\pgfqpoint{0.627612in}{0.328214in}}%
\pgfpathlineto{\pgfqpoint{0.632850in}{0.315287in}}%
\pgfpathlineto{\pgfqpoint{0.630822in}{0.302361in}}%
\pgfpathlineto{\pgfqpoint{0.623212in}{0.293862in}}%
\pgfpathlineto{\pgfqpoint{0.609943in}{0.297214in}}%
\pgfpathclose%
\pgfusepath{fill}%
\end{pgfscope}%
\begin{pgfscope}%
\pgfpathrectangle{\pgfqpoint{0.211875in}{0.211875in}}{\pgfqpoint{1.313625in}{1.279725in}}%
\pgfusepath{clip}%
\pgfsetbuttcap%
\pgfsetroundjoin%
\definecolor{currentfill}{rgb}{0.490838,0.119982,0.351115}%
\pgfsetfillcolor{currentfill}%
\pgfsetlinewidth{0.000000pt}%
\definecolor{currentstroke}{rgb}{0.000000,0.000000,0.000000}%
\pgfsetstrokecolor{currentstroke}%
\pgfsetdash{}{0pt}%
\pgfpathmoveto{\pgfqpoint{0.729364in}{0.286697in}}%
\pgfpathlineto{\pgfqpoint{0.742633in}{0.286111in}}%
\pgfpathlineto{\pgfqpoint{0.748502in}{0.289434in}}%
\pgfpathlineto{\pgfqpoint{0.755902in}{0.300652in}}%
\pgfpathlineto{\pgfqpoint{0.756354in}{0.302361in}}%
\pgfpathlineto{\pgfqpoint{0.757016in}{0.315287in}}%
\pgfpathlineto{\pgfqpoint{0.755902in}{0.322696in}}%
\pgfpathlineto{\pgfqpoint{0.754299in}{0.328214in}}%
\pgfpathlineto{\pgfqpoint{0.742633in}{0.339311in}}%
\pgfpathlineto{\pgfqpoint{0.729364in}{0.338481in}}%
\pgfpathlineto{\pgfqpoint{0.720546in}{0.328214in}}%
\pgfpathlineto{\pgfqpoint{0.717693in}{0.315287in}}%
\pgfpathlineto{\pgfqpoint{0.718565in}{0.302361in}}%
\pgfpathlineto{\pgfqpoint{0.725363in}{0.289434in}}%
\pgfpathclose%
\pgfpathmoveto{\pgfqpoint{0.727704in}{0.302361in}}%
\pgfpathlineto{\pgfqpoint{0.725964in}{0.315287in}}%
\pgfpathlineto{\pgfqpoint{0.729364in}{0.325725in}}%
\pgfpathlineto{\pgfqpoint{0.739414in}{0.328214in}}%
\pgfpathlineto{\pgfqpoint{0.742633in}{0.328551in}}%
\pgfpathlineto{\pgfqpoint{0.742987in}{0.328214in}}%
\pgfpathlineto{\pgfqpoint{0.748104in}{0.315287in}}%
\pgfpathlineto{\pgfqpoint{0.746042in}{0.302361in}}%
\pgfpathlineto{\pgfqpoint{0.742633in}{0.298015in}}%
\pgfpathlineto{\pgfqpoint{0.729364in}{0.299749in}}%
\pgfpathclose%
\pgfusepath{fill}%
\end{pgfscope}%
\begin{pgfscope}%
\pgfpathrectangle{\pgfqpoint{0.211875in}{0.211875in}}{\pgfqpoint{1.313625in}{1.279725in}}%
\pgfusepath{clip}%
\pgfsetbuttcap%
\pgfsetroundjoin%
\definecolor{currentfill}{rgb}{0.490838,0.119982,0.351115}%
\pgfsetfillcolor{currentfill}%
\pgfsetlinewidth{0.000000pt}%
\definecolor{currentstroke}{rgb}{0.000000,0.000000,0.000000}%
\pgfsetstrokecolor{currentstroke}%
\pgfsetdash{}{0pt}%
\pgfpathmoveto{\pgfqpoint{0.848784in}{0.288133in}}%
\pgfpathlineto{\pgfqpoint{0.862053in}{0.288316in}}%
\pgfpathlineto{\pgfqpoint{0.863807in}{0.289434in}}%
\pgfpathlineto{\pgfqpoint{0.872179in}{0.302361in}}%
\pgfpathlineto{\pgfqpoint{0.873335in}{0.315287in}}%
\pgfpathlineto{\pgfqpoint{0.869923in}{0.328214in}}%
\pgfpathlineto{\pgfqpoint{0.862053in}{0.336649in}}%
\pgfpathlineto{\pgfqpoint{0.848784in}{0.336867in}}%
\pgfpathlineto{\pgfqpoint{0.840449in}{0.328214in}}%
\pgfpathlineto{\pgfqpoint{0.837011in}{0.315287in}}%
\pgfpathlineto{\pgfqpoint{0.838154in}{0.302361in}}%
\pgfpathlineto{\pgfqpoint{0.846648in}{0.289434in}}%
\pgfpathclose%
\pgfpathmoveto{\pgfqpoint{0.848085in}{0.302361in}}%
\pgfpathlineto{\pgfqpoint{0.846002in}{0.315287in}}%
\pgfpathlineto{\pgfqpoint{0.848784in}{0.322912in}}%
\pgfpathlineto{\pgfqpoint{0.862053in}{0.322409in}}%
\pgfpathlineto{\pgfqpoint{0.864581in}{0.315287in}}%
\pgfpathlineto{\pgfqpoint{0.862535in}{0.302361in}}%
\pgfpathlineto{\pgfqpoint{0.862053in}{0.301666in}}%
\pgfpathlineto{\pgfqpoint{0.848784in}{0.301384in}}%
\pgfpathclose%
\pgfusepath{fill}%
\end{pgfscope}%
\begin{pgfscope}%
\pgfpathrectangle{\pgfqpoint{0.211875in}{0.211875in}}{\pgfqpoint{1.313625in}{1.279725in}}%
\pgfusepath{clip}%
\pgfsetbuttcap%
\pgfsetroundjoin%
\definecolor{currentfill}{rgb}{0.490838,0.119982,0.351115}%
\pgfsetfillcolor{currentfill}%
\pgfsetlinewidth{0.000000pt}%
\definecolor{currentstroke}{rgb}{0.000000,0.000000,0.000000}%
\pgfsetstrokecolor{currentstroke}%
\pgfsetdash{}{0pt}%
\pgfpathmoveto{\pgfqpoint{0.968205in}{0.288968in}}%
\pgfpathlineto{\pgfqpoint{0.973787in}{0.289434in}}%
\pgfpathlineto{\pgfqpoint{0.981473in}{0.290449in}}%
\pgfpathlineto{\pgfqpoint{0.988836in}{0.302361in}}%
\pgfpathlineto{\pgfqpoint{0.990003in}{0.315287in}}%
\pgfpathlineto{\pgfqpoint{0.986681in}{0.328214in}}%
\pgfpathlineto{\pgfqpoint{0.981473in}{0.334459in}}%
\pgfpathlineto{\pgfqpoint{0.968205in}{0.335936in}}%
\pgfpathlineto{\pgfqpoint{0.959797in}{0.328214in}}%
\pgfpathlineto{\pgfqpoint{0.955754in}{0.315287in}}%
\pgfpathlineto{\pgfqpoint{0.957160in}{0.302361in}}%
\pgfpathlineto{\pgfqpoint{0.967340in}{0.289434in}}%
\pgfpathclose%
\pgfpathmoveto{\pgfqpoint{0.968113in}{0.302361in}}%
\pgfpathlineto{\pgfqpoint{0.965674in}{0.315287in}}%
\pgfpathlineto{\pgfqpoint{0.968205in}{0.321427in}}%
\pgfpathlineto{\pgfqpoint{0.981473in}{0.316732in}}%
\pgfpathlineto{\pgfqpoint{0.981932in}{0.315287in}}%
\pgfpathlineto{\pgfqpoint{0.981473in}{0.312081in}}%
\pgfpathlineto{\pgfqpoint{0.968818in}{0.302361in}}%
\pgfpathlineto{\pgfqpoint{0.968205in}{0.302248in}}%
\pgfpathclose%
\pgfusepath{fill}%
\end{pgfscope}%
\begin{pgfscope}%
\pgfpathrectangle{\pgfqpoint{0.211875in}{0.211875in}}{\pgfqpoint{1.313625in}{1.279725in}}%
\pgfusepath{clip}%
\pgfsetbuttcap%
\pgfsetroundjoin%
\definecolor{currentfill}{rgb}{0.490838,0.119982,0.351115}%
\pgfsetfillcolor{currentfill}%
\pgfsetlinewidth{0.000000pt}%
\definecolor{currentstroke}{rgb}{0.000000,0.000000,0.000000}%
\pgfsetstrokecolor{currentstroke}%
\pgfsetdash{}{0pt}%
\pgfpathmoveto{\pgfqpoint{1.087625in}{0.289269in}}%
\pgfpathlineto{\pgfqpoint{1.088724in}{0.289434in}}%
\pgfpathlineto{\pgfqpoint{1.100894in}{0.292509in}}%
\pgfpathlineto{\pgfqpoint{1.106335in}{0.302361in}}%
\pgfpathlineto{\pgfqpoint{1.107467in}{0.315287in}}%
\pgfpathlineto{\pgfqpoint{1.104289in}{0.328214in}}%
\pgfpathlineto{\pgfqpoint{1.100894in}{0.332749in}}%
\pgfpathlineto{\pgfqpoint{1.087625in}{0.335618in}}%
\pgfpathlineto{\pgfqpoint{1.078410in}{0.328214in}}%
\pgfpathlineto{\pgfqpoint{1.074356in}{0.317300in}}%
\pgfpathlineto{\pgfqpoint{1.074010in}{0.315287in}}%
\pgfpathlineto{\pgfqpoint{1.074356in}{0.310064in}}%
\pgfpathlineto{\pgfqpoint{1.075388in}{0.302361in}}%
\pgfpathlineto{\pgfqpoint{1.087276in}{0.289434in}}%
\pgfpathclose%
\pgfpathmoveto{\pgfqpoint{1.084867in}{0.315287in}}%
\pgfpathlineto{\pgfqpoint{1.087625in}{0.321143in}}%
\pgfpathlineto{\pgfqpoint{1.096390in}{0.315287in}}%
\pgfpathlineto{\pgfqpoint{1.087625in}{0.302615in}}%
\pgfpathclose%
\pgfusepath{fill}%
\end{pgfscope}%
\begin{pgfscope}%
\pgfpathrectangle{\pgfqpoint{0.211875in}{0.211875in}}{\pgfqpoint{1.313625in}{1.279725in}}%
\pgfusepath{clip}%
\pgfsetbuttcap%
\pgfsetroundjoin%
\definecolor{currentfill}{rgb}{0.490838,0.119982,0.351115}%
\pgfsetfillcolor{currentfill}%
\pgfsetlinewidth{0.000000pt}%
\definecolor{currentstroke}{rgb}{0.000000,0.000000,0.000000}%
\pgfsetstrokecolor{currentstroke}%
\pgfsetdash{}{0pt}%
\pgfpathmoveto{\pgfqpoint{1.207045in}{0.289075in}}%
\pgfpathlineto{\pgfqpoint{1.208720in}{0.289434in}}%
\pgfpathlineto{\pgfqpoint{1.220314in}{0.293907in}}%
\pgfpathlineto{\pgfqpoint{1.224491in}{0.302361in}}%
\pgfpathlineto{\pgfqpoint{1.225553in}{0.315287in}}%
\pgfpathlineto{\pgfqpoint{1.222565in}{0.328214in}}%
\pgfpathlineto{\pgfqpoint{1.220314in}{0.331555in}}%
\pgfpathlineto{\pgfqpoint{1.207045in}{0.335873in}}%
\pgfpathlineto{\pgfqpoint{1.195975in}{0.328214in}}%
\pgfpathlineto{\pgfqpoint{1.193777in}{0.323458in}}%
\pgfpathlineto{\pgfqpoint{1.192213in}{0.315287in}}%
\pgfpathlineto{\pgfqpoint{1.193153in}{0.302361in}}%
\pgfpathlineto{\pgfqpoint{1.193777in}{0.300596in}}%
\pgfpathlineto{\pgfqpoint{1.206165in}{0.289434in}}%
\pgfpathclose%
\pgfpathmoveto{\pgfqpoint{1.206621in}{0.302361in}}%
\pgfpathlineto{\pgfqpoint{1.203379in}{0.315287in}}%
\pgfpathlineto{\pgfqpoint{1.207045in}{0.321986in}}%
\pgfpathlineto{\pgfqpoint{1.213969in}{0.315287in}}%
\pgfpathlineto{\pgfqpoint{1.207847in}{0.302361in}}%
\pgfpathlineto{\pgfqpoint{1.207045in}{0.301968in}}%
\pgfpathclose%
\pgfusepath{fill}%
\end{pgfscope}%
\begin{pgfscope}%
\pgfpathrectangle{\pgfqpoint{0.211875in}{0.211875in}}{\pgfqpoint{1.313625in}{1.279725in}}%
\pgfusepath{clip}%
\pgfsetbuttcap%
\pgfsetroundjoin%
\definecolor{currentfill}{rgb}{0.490838,0.119982,0.351115}%
\pgfsetfillcolor{currentfill}%
\pgfsetlinewidth{0.000000pt}%
\definecolor{currentstroke}{rgb}{0.000000,0.000000,0.000000}%
\pgfsetstrokecolor{currentstroke}%
\pgfsetdash{}{0pt}%
\pgfpathmoveto{\pgfqpoint{1.326466in}{0.288402in}}%
\pgfpathlineto{\pgfqpoint{1.330185in}{0.289434in}}%
\pgfpathlineto{\pgfqpoint{1.339735in}{0.294548in}}%
\pgfpathlineto{\pgfqpoint{1.343185in}{0.302361in}}%
\pgfpathlineto{\pgfqpoint{1.344143in}{0.315287in}}%
\pgfpathlineto{\pgfqpoint{1.341389in}{0.328214in}}%
\pgfpathlineto{\pgfqpoint{1.339735in}{0.330940in}}%
\pgfpathlineto{\pgfqpoint{1.326466in}{0.336682in}}%
\pgfpathlineto{\pgfqpoint{1.313197in}{0.329210in}}%
\pgfpathlineto{\pgfqpoint{1.312649in}{0.328214in}}%
\pgfpathlineto{\pgfqpoint{1.309989in}{0.315287in}}%
\pgfpathlineto{\pgfqpoint{1.310923in}{0.302361in}}%
\pgfpathlineto{\pgfqpoint{1.313197in}{0.296651in}}%
\pgfpathlineto{\pgfqpoint{1.323467in}{0.289434in}}%
\pgfpathclose%
\pgfpathmoveto{\pgfqpoint{1.324575in}{0.302361in}}%
\pgfpathlineto{\pgfqpoint{1.320835in}{0.315287in}}%
\pgfpathlineto{\pgfqpoint{1.326466in}{0.323927in}}%
\pgfpathlineto{\pgfqpoint{1.333314in}{0.315287in}}%
\pgfpathlineto{\pgfqpoint{1.328773in}{0.302361in}}%
\pgfpathlineto{\pgfqpoint{1.326466in}{0.300891in}}%
\pgfpathclose%
\pgfusepath{fill}%
\end{pgfscope}%
\begin{pgfscope}%
\pgfpathrectangle{\pgfqpoint{0.211875in}{0.211875in}}{\pgfqpoint{1.313625in}{1.279725in}}%
\pgfusepath{clip}%
\pgfsetbuttcap%
\pgfsetroundjoin%
\definecolor{currentfill}{rgb}{0.490838,0.119982,0.351115}%
\pgfsetfillcolor{currentfill}%
\pgfsetlinewidth{0.000000pt}%
\definecolor{currentstroke}{rgb}{0.000000,0.000000,0.000000}%
\pgfsetstrokecolor{currentstroke}%
\pgfsetdash{}{0pt}%
\pgfpathmoveto{\pgfqpoint{1.445886in}{0.287245in}}%
\pgfpathlineto{\pgfqpoint{1.452316in}{0.289434in}}%
\pgfpathlineto{\pgfqpoint{1.459155in}{0.294274in}}%
\pgfpathlineto{\pgfqpoint{1.462336in}{0.302361in}}%
\pgfpathlineto{\pgfqpoint{1.463162in}{0.315287in}}%
\pgfpathlineto{\pgfqpoint{1.460681in}{0.328214in}}%
\pgfpathlineto{\pgfqpoint{1.459155in}{0.331011in}}%
\pgfpathlineto{\pgfqpoint{1.445886in}{0.338051in}}%
\pgfpathlineto{\pgfqpoint{1.432617in}{0.332672in}}%
\pgfpathlineto{\pgfqpoint{1.429892in}{0.328214in}}%
\pgfpathlineto{\pgfqpoint{1.427266in}{0.315287in}}%
\pgfpathlineto{\pgfqpoint{1.428155in}{0.302361in}}%
\pgfpathlineto{\pgfqpoint{1.432617in}{0.292362in}}%
\pgfpathlineto{\pgfqpoint{1.438103in}{0.289434in}}%
\pgfpathclose%
\pgfpathmoveto{\pgfqpoint{1.440848in}{0.302361in}}%
\pgfpathlineto{\pgfqpoint{1.436473in}{0.315287in}}%
\pgfpathlineto{\pgfqpoint{1.445886in}{0.326981in}}%
\pgfpathlineto{\pgfqpoint{1.453414in}{0.315287in}}%
\pgfpathlineto{\pgfqpoint{1.449935in}{0.302361in}}%
\pgfpathlineto{\pgfqpoint{1.445886in}{0.299186in}}%
\pgfpathclose%
\pgfusepath{fill}%
\end{pgfscope}%
\begin{pgfscope}%
\pgfpathrectangle{\pgfqpoint{0.211875in}{0.211875in}}{\pgfqpoint{1.313625in}{1.279725in}}%
\pgfusepath{clip}%
\pgfsetbuttcap%
\pgfsetroundjoin%
\definecolor{currentfill}{rgb}{0.490838,0.119982,0.351115}%
\pgfsetfillcolor{currentfill}%
\pgfsetlinewidth{0.000000pt}%
\definecolor{currentstroke}{rgb}{0.000000,0.000000,0.000000}%
\pgfsetstrokecolor{currentstroke}%
\pgfsetdash{}{0pt}%
\pgfpathmoveto{\pgfqpoint{0.225144in}{0.356733in}}%
\pgfpathlineto{\pgfqpoint{0.233234in}{0.366993in}}%
\pgfpathlineto{\pgfqpoint{0.234187in}{0.379920in}}%
\pgfpathlineto{\pgfqpoint{0.234239in}{0.392846in}}%
\pgfpathlineto{\pgfqpoint{0.233764in}{0.405773in}}%
\pgfpathlineto{\pgfqpoint{0.232119in}{0.418699in}}%
\pgfpathlineto{\pgfqpoint{0.225144in}{0.429398in}}%
\pgfpathlineto{\pgfqpoint{0.217277in}{0.431626in}}%
\pgfpathlineto{\pgfqpoint{0.211875in}{0.432471in}}%
\pgfpathlineto{\pgfqpoint{0.211875in}{0.431626in}}%
\pgfpathlineto{\pgfqpoint{0.211875in}{0.423975in}}%
\pgfpathlineto{\pgfqpoint{0.221258in}{0.418699in}}%
\pgfpathlineto{\pgfqpoint{0.225144in}{0.413394in}}%
\pgfpathlineto{\pgfqpoint{0.227456in}{0.405773in}}%
\pgfpathlineto{\pgfqpoint{0.228443in}{0.392846in}}%
\pgfpathlineto{\pgfqpoint{0.227309in}{0.379920in}}%
\pgfpathlineto{\pgfqpoint{0.225144in}{0.373539in}}%
\pgfpathlineto{\pgfqpoint{0.218918in}{0.366993in}}%
\pgfpathlineto{\pgfqpoint{0.211875in}{0.363839in}}%
\pgfpathlineto{\pgfqpoint{0.211875in}{0.355202in}}%
\pgfpathclose%
\pgfusepath{fill}%
\end{pgfscope}%
\begin{pgfscope}%
\pgfpathrectangle{\pgfqpoint{0.211875in}{0.211875in}}{\pgfqpoint{1.313625in}{1.279725in}}%
\pgfusepath{clip}%
\pgfsetbuttcap%
\pgfsetroundjoin%
\definecolor{currentfill}{rgb}{0.490838,0.119982,0.351115}%
\pgfsetfillcolor{currentfill}%
\pgfsetlinewidth{0.000000pt}%
\definecolor{currentstroke}{rgb}{0.000000,0.000000,0.000000}%
\pgfsetstrokecolor{currentstroke}%
\pgfsetdash{}{0pt}%
\pgfpathmoveto{\pgfqpoint{0.304758in}{0.363132in}}%
\pgfpathlineto{\pgfqpoint{0.318027in}{0.359196in}}%
\pgfpathlineto{\pgfqpoint{0.331295in}{0.359627in}}%
\pgfpathlineto{\pgfqpoint{0.344564in}{0.365584in}}%
\pgfpathlineto{\pgfqpoint{0.345524in}{0.366993in}}%
\pgfpathlineto{\pgfqpoint{0.348883in}{0.379920in}}%
\pgfpathlineto{\pgfqpoint{0.349447in}{0.392846in}}%
\pgfpathlineto{\pgfqpoint{0.348735in}{0.405773in}}%
\pgfpathlineto{\pgfqpoint{0.345655in}{0.418699in}}%
\pgfpathlineto{\pgfqpoint{0.344564in}{0.420610in}}%
\pgfpathlineto{\pgfqpoint{0.331295in}{0.428095in}}%
\pgfpathlineto{\pgfqpoint{0.318027in}{0.428510in}}%
\pgfpathlineto{\pgfqpoint{0.304758in}{0.422802in}}%
\pgfpathlineto{\pgfqpoint{0.302303in}{0.418699in}}%
\pgfpathlineto{\pgfqpoint{0.299632in}{0.405773in}}%
\pgfpathlineto{\pgfqpoint{0.298993in}{0.392846in}}%
\pgfpathlineto{\pgfqpoint{0.299360in}{0.379920in}}%
\pgfpathlineto{\pgfqpoint{0.301949in}{0.366993in}}%
\pgfpathclose%
\pgfpathmoveto{\pgfqpoint{0.307449in}{0.379920in}}%
\pgfpathlineto{\pgfqpoint{0.305076in}{0.392846in}}%
\pgfpathlineto{\pgfqpoint{0.306905in}{0.405773in}}%
\pgfpathlineto{\pgfqpoint{0.316312in}{0.418699in}}%
\pgfpathlineto{\pgfqpoint{0.318027in}{0.419686in}}%
\pgfpathlineto{\pgfqpoint{0.331295in}{0.419319in}}%
\pgfpathlineto{\pgfqpoint{0.332279in}{0.418699in}}%
\pgfpathlineto{\pgfqpoint{0.341454in}{0.405773in}}%
\pgfpathlineto{\pgfqpoint{0.343296in}{0.392846in}}%
\pgfpathlineto{\pgfqpoint{0.340826in}{0.379920in}}%
\pgfpathlineto{\pgfqpoint{0.331295in}{0.368986in}}%
\pgfpathlineto{\pgfqpoint{0.318027in}{0.368481in}}%
\pgfpathclose%
\pgfusepath{fill}%
\end{pgfscope}%
\begin{pgfscope}%
\pgfpathrectangle{\pgfqpoint{0.211875in}{0.211875in}}{\pgfqpoint{1.313625in}{1.279725in}}%
\pgfusepath{clip}%
\pgfsetbuttcap%
\pgfsetroundjoin%
\definecolor{currentfill}{rgb}{0.490838,0.119982,0.351115}%
\pgfsetfillcolor{currentfill}%
\pgfsetlinewidth{0.000000pt}%
\definecolor{currentstroke}{rgb}{0.000000,0.000000,0.000000}%
\pgfsetstrokecolor{currentstroke}%
\pgfsetdash{}{0pt}%
\pgfpathmoveto{\pgfqpoint{0.437447in}{0.362851in}}%
\pgfpathlineto{\pgfqpoint{0.450716in}{0.363744in}}%
\pgfpathlineto{\pgfqpoint{0.456543in}{0.366993in}}%
\pgfpathlineto{\pgfqpoint{0.463985in}{0.379501in}}%
\pgfpathlineto{\pgfqpoint{0.464087in}{0.379920in}}%
\pgfpathlineto{\pgfqpoint{0.465091in}{0.392846in}}%
\pgfpathlineto{\pgfqpoint{0.464175in}{0.405773in}}%
\pgfpathlineto{\pgfqpoint{0.463985in}{0.406600in}}%
\pgfpathlineto{\pgfqpoint{0.458201in}{0.418699in}}%
\pgfpathlineto{\pgfqpoint{0.450716in}{0.423956in}}%
\pgfpathlineto{\pgfqpoint{0.437447in}{0.424946in}}%
\pgfpathlineto{\pgfqpoint{0.425164in}{0.418699in}}%
\pgfpathlineto{\pgfqpoint{0.424178in}{0.417461in}}%
\pgfpathlineto{\pgfqpoint{0.420539in}{0.405773in}}%
\pgfpathlineto{\pgfqpoint{0.419644in}{0.392846in}}%
\pgfpathlineto{\pgfqpoint{0.420542in}{0.379920in}}%
\pgfpathlineto{\pgfqpoint{0.424178in}{0.369492in}}%
\pgfpathlineto{\pgfqpoint{0.426848in}{0.366993in}}%
\pgfpathclose%
\pgfpathmoveto{\pgfqpoint{0.430196in}{0.379920in}}%
\pgfpathlineto{\pgfqpoint{0.426697in}{0.392846in}}%
\pgfpathlineto{\pgfqpoint{0.429130in}{0.405773in}}%
\pgfpathlineto{\pgfqpoint{0.437447in}{0.415262in}}%
\pgfpathlineto{\pgfqpoint{0.450716in}{0.413145in}}%
\pgfpathlineto{\pgfqpoint{0.455559in}{0.405773in}}%
\pgfpathlineto{\pgfqpoint{0.457553in}{0.392846in}}%
\pgfpathlineto{\pgfqpoint{0.454651in}{0.379920in}}%
\pgfpathlineto{\pgfqpoint{0.450716in}{0.374861in}}%
\pgfpathlineto{\pgfqpoint{0.437447in}{0.373011in}}%
\pgfpathclose%
\pgfusepath{fill}%
\end{pgfscope}%
\begin{pgfscope}%
\pgfpathrectangle{\pgfqpoint{0.211875in}{0.211875in}}{\pgfqpoint{1.313625in}{1.279725in}}%
\pgfusepath{clip}%
\pgfsetbuttcap%
\pgfsetroundjoin%
\definecolor{currentfill}{rgb}{0.490838,0.119982,0.351115}%
\pgfsetfillcolor{currentfill}%
\pgfsetlinewidth{0.000000pt}%
\definecolor{currentstroke}{rgb}{0.000000,0.000000,0.000000}%
\pgfsetstrokecolor{currentstroke}%
\pgfsetdash{}{0pt}%
\pgfpathmoveto{\pgfqpoint{0.556867in}{0.365816in}}%
\pgfpathlineto{\pgfqpoint{0.565940in}{0.366993in}}%
\pgfpathlineto{\pgfqpoint{0.570136in}{0.367807in}}%
\pgfpathlineto{\pgfqpoint{0.578574in}{0.379920in}}%
\pgfpathlineto{\pgfqpoint{0.580380in}{0.392846in}}%
\pgfpathlineto{\pgfqpoint{0.578946in}{0.405773in}}%
\pgfpathlineto{\pgfqpoint{0.571895in}{0.418699in}}%
\pgfpathlineto{\pgfqpoint{0.570136in}{0.420070in}}%
\pgfpathlineto{\pgfqpoint{0.556867in}{0.422042in}}%
\pgfpathlineto{\pgfqpoint{0.549346in}{0.418699in}}%
\pgfpathlineto{\pgfqpoint{0.543598in}{0.412906in}}%
\pgfpathlineto{\pgfqpoint{0.541087in}{0.405773in}}%
\pgfpathlineto{\pgfqpoint{0.539960in}{0.392846in}}%
\pgfpathlineto{\pgfqpoint{0.541330in}{0.379920in}}%
\pgfpathlineto{\pgfqpoint{0.543598in}{0.374259in}}%
\pgfpathlineto{\pgfqpoint{0.553420in}{0.366993in}}%
\pgfpathclose%
\pgfpathmoveto{\pgfqpoint{0.553007in}{0.379920in}}%
\pgfpathlineto{\pgfqpoint{0.548238in}{0.392846in}}%
\pgfpathlineto{\pgfqpoint{0.551366in}{0.405773in}}%
\pgfpathlineto{\pgfqpoint{0.556867in}{0.411252in}}%
\pgfpathlineto{\pgfqpoint{0.570136in}{0.406947in}}%
\pgfpathlineto{\pgfqpoint{0.570829in}{0.405773in}}%
\pgfpathlineto{\pgfqpoint{0.572926in}{0.392846in}}%
\pgfpathlineto{\pgfqpoint{0.570136in}{0.381368in}}%
\pgfpathlineto{\pgfqpoint{0.568246in}{0.379920in}}%
\pgfpathlineto{\pgfqpoint{0.556867in}{0.376713in}}%
\pgfpathclose%
\pgfusepath{fill}%
\end{pgfscope}%
\begin{pgfscope}%
\pgfpathrectangle{\pgfqpoint{0.211875in}{0.211875in}}{\pgfqpoint{1.313625in}{1.279725in}}%
\pgfusepath{clip}%
\pgfsetbuttcap%
\pgfsetroundjoin%
\definecolor{currentfill}{rgb}{0.490838,0.119982,0.351115}%
\pgfsetfillcolor{currentfill}%
\pgfsetlinewidth{0.000000pt}%
\definecolor{currentstroke}{rgb}{0.000000,0.000000,0.000000}%
\pgfsetstrokecolor{currentstroke}%
\pgfsetdash{}{0pt}%
\pgfpathmoveto{\pgfqpoint{0.663019in}{0.377104in}}%
\pgfpathlineto{\pgfqpoint{0.676288in}{0.368585in}}%
\pgfpathlineto{\pgfqpoint{0.689557in}{0.372664in}}%
\pgfpathlineto{\pgfqpoint{0.694089in}{0.379920in}}%
\pgfpathlineto{\pgfqpoint{0.696173in}{0.392846in}}%
\pgfpathlineto{\pgfqpoint{0.694641in}{0.405773in}}%
\pgfpathlineto{\pgfqpoint{0.689557in}{0.415339in}}%
\pgfpathlineto{\pgfqpoint{0.680811in}{0.418699in}}%
\pgfpathlineto{\pgfqpoint{0.676288in}{0.419704in}}%
\pgfpathlineto{\pgfqpoint{0.673657in}{0.418699in}}%
\pgfpathlineto{\pgfqpoint{0.663019in}{0.410159in}}%
\pgfpathlineto{\pgfqpoint{0.661287in}{0.405773in}}%
\pgfpathlineto{\pgfqpoint{0.659950in}{0.392846in}}%
\pgfpathlineto{\pgfqpoint{0.661738in}{0.379920in}}%
\pgfpathclose%
\pgfpathmoveto{\pgfqpoint{0.676002in}{0.379920in}}%
\pgfpathlineto{\pgfqpoint{0.669731in}{0.392846in}}%
\pgfpathlineto{\pgfqpoint{0.673699in}{0.405773in}}%
\pgfpathlineto{\pgfqpoint{0.676288in}{0.407986in}}%
\pgfpathlineto{\pgfqpoint{0.680943in}{0.405773in}}%
\pgfpathlineto{\pgfqpoint{0.688167in}{0.392846in}}%
\pgfpathlineto{\pgfqpoint{0.676798in}{0.379920in}}%
\pgfpathlineto{\pgfqpoint{0.676288in}{0.379716in}}%
\pgfpathclose%
\pgfusepath{fill}%
\end{pgfscope}%
\begin{pgfscope}%
\pgfpathrectangle{\pgfqpoint{0.211875in}{0.211875in}}{\pgfqpoint{1.313625in}{1.279725in}}%
\pgfusepath{clip}%
\pgfsetbuttcap%
\pgfsetroundjoin%
\definecolor{currentfill}{rgb}{0.490838,0.119982,0.351115}%
\pgfsetfillcolor{currentfill}%
\pgfsetlinewidth{0.000000pt}%
\definecolor{currentstroke}{rgb}{0.000000,0.000000,0.000000}%
\pgfsetstrokecolor{currentstroke}%
\pgfsetdash{}{0pt}%
\pgfpathmoveto{\pgfqpoint{0.782439in}{0.378602in}}%
\pgfpathlineto{\pgfqpoint{0.795708in}{0.371052in}}%
\pgfpathlineto{\pgfqpoint{0.808977in}{0.377301in}}%
\pgfpathlineto{\pgfqpoint{0.810443in}{0.379920in}}%
\pgfpathlineto{\pgfqpoint{0.812709in}{0.392846in}}%
\pgfpathlineto{\pgfqpoint{0.811118in}{0.405773in}}%
\pgfpathlineto{\pgfqpoint{0.808977in}{0.410242in}}%
\pgfpathlineto{\pgfqpoint{0.795708in}{0.417482in}}%
\pgfpathlineto{\pgfqpoint{0.782439in}{0.408733in}}%
\pgfpathlineto{\pgfqpoint{0.781135in}{0.405773in}}%
\pgfpathlineto{\pgfqpoint{0.779608in}{0.392846in}}%
\pgfpathlineto{\pgfqpoint{0.781765in}{0.379920in}}%
\pgfpathclose%
\pgfpathmoveto{\pgfqpoint{0.791216in}{0.392846in}}%
\pgfpathlineto{\pgfqpoint{0.795708in}{0.404456in}}%
\pgfpathlineto{\pgfqpoint{0.800968in}{0.392846in}}%
\pgfpathlineto{\pgfqpoint{0.795708in}{0.385159in}}%
\pgfpathclose%
\pgfusepath{fill}%
\end{pgfscope}%
\begin{pgfscope}%
\pgfpathrectangle{\pgfqpoint{0.211875in}{0.211875in}}{\pgfqpoint{1.313625in}{1.279725in}}%
\pgfusepath{clip}%
\pgfsetbuttcap%
\pgfsetroundjoin%
\definecolor{currentfill}{rgb}{0.490838,0.119982,0.351115}%
\pgfsetfillcolor{currentfill}%
\pgfsetlinewidth{0.000000pt}%
\definecolor{currentstroke}{rgb}{0.000000,0.000000,0.000000}%
\pgfsetstrokecolor{currentstroke}%
\pgfsetdash{}{0pt}%
\pgfpathmoveto{\pgfqpoint{0.901860in}{0.379102in}}%
\pgfpathlineto{\pgfqpoint{0.915129in}{0.372895in}}%
\pgfpathlineto{\pgfqpoint{0.926278in}{0.379920in}}%
\pgfpathlineto{\pgfqpoint{0.928398in}{0.384438in}}%
\pgfpathlineto{\pgfqpoint{0.929842in}{0.392846in}}%
\pgfpathlineto{\pgfqpoint{0.928398in}{0.404483in}}%
\pgfpathlineto{\pgfqpoint{0.927997in}{0.405773in}}%
\pgfpathlineto{\pgfqpoint{0.915129in}{0.415463in}}%
\pgfpathlineto{\pgfqpoint{0.901860in}{0.408317in}}%
\pgfpathlineto{\pgfqpoint{0.900613in}{0.405773in}}%
\pgfpathlineto{\pgfqpoint{0.898912in}{0.392846in}}%
\pgfpathlineto{\pgfqpoint{0.901391in}{0.379920in}}%
\pgfpathclose%
\pgfpathmoveto{\pgfqpoint{0.912758in}{0.392846in}}%
\pgfpathlineto{\pgfqpoint{0.915129in}{0.397788in}}%
\pgfpathlineto{\pgfqpoint{0.916954in}{0.392846in}}%
\pgfpathlineto{\pgfqpoint{0.915129in}{0.389574in}}%
\pgfpathclose%
\pgfusepath{fill}%
\end{pgfscope}%
\begin{pgfscope}%
\pgfpathrectangle{\pgfqpoint{0.211875in}{0.211875in}}{\pgfqpoint{1.313625in}{1.279725in}}%
\pgfusepath{clip}%
\pgfsetbuttcap%
\pgfsetroundjoin%
\definecolor{currentfill}{rgb}{0.490838,0.119982,0.351115}%
\pgfsetfillcolor{currentfill}%
\pgfsetlinewidth{0.000000pt}%
\definecolor{currentstroke}{rgb}{0.000000,0.000000,0.000000}%
\pgfsetstrokecolor{currentstroke}%
\pgfsetdash{}{0pt}%
\pgfpathmoveto{\pgfqpoint{1.021280in}{0.378822in}}%
\pgfpathlineto{\pgfqpoint{1.034549in}{0.374153in}}%
\pgfpathlineto{\pgfqpoint{1.042278in}{0.379920in}}%
\pgfpathlineto{\pgfqpoint{1.047113in}{0.392846in}}%
\pgfpathlineto{\pgfqpoint{1.043854in}{0.405773in}}%
\pgfpathlineto{\pgfqpoint{1.034549in}{0.414074in}}%
\pgfpathlineto{\pgfqpoint{1.021280in}{0.408711in}}%
\pgfpathlineto{\pgfqpoint{1.019681in}{0.405773in}}%
\pgfpathlineto{\pgfqpoint{1.017822in}{0.392846in}}%
\pgfpathlineto{\pgfqpoint{1.020576in}{0.379920in}}%
\pgfpathclose%
\pgfpathmoveto{\pgfqpoint{1.034477in}{0.392846in}}%
\pgfpathlineto{\pgfqpoint{1.034549in}{0.392960in}}%
\pgfpathlineto{\pgfqpoint{1.034585in}{0.392846in}}%
\pgfpathlineto{\pgfqpoint{1.034549in}{0.392771in}}%
\pgfpathclose%
\pgfusepath{fill}%
\end{pgfscope}%
\begin{pgfscope}%
\pgfpathrectangle{\pgfqpoint{0.211875in}{0.211875in}}{\pgfqpoint{1.313625in}{1.279725in}}%
\pgfusepath{clip}%
\pgfsetbuttcap%
\pgfsetroundjoin%
\definecolor{currentfill}{rgb}{0.490838,0.119982,0.351115}%
\pgfsetfillcolor{currentfill}%
\pgfsetlinewidth{0.000000pt}%
\definecolor{currentstroke}{rgb}{0.000000,0.000000,0.000000}%
\pgfsetstrokecolor{currentstroke}%
\pgfsetdash{}{0pt}%
\pgfpathmoveto{\pgfqpoint{1.140701in}{0.377897in}}%
\pgfpathlineto{\pgfqpoint{1.153970in}{0.374838in}}%
\pgfpathlineto{\pgfqpoint{1.159846in}{0.379920in}}%
\pgfpathlineto{\pgfqpoint{1.164126in}{0.392846in}}%
\pgfpathlineto{\pgfqpoint{1.161254in}{0.405773in}}%
\pgfpathlineto{\pgfqpoint{1.153970in}{0.413302in}}%
\pgfpathlineto{\pgfqpoint{1.140701in}{0.409786in}}%
\pgfpathlineto{\pgfqpoint{1.138271in}{0.405773in}}%
\pgfpathlineto{\pgfqpoint{1.136270in}{0.392846in}}%
\pgfpathlineto{\pgfqpoint{1.139252in}{0.379920in}}%
\pgfpathclose%
\pgfusepath{fill}%
\end{pgfscope}%
\begin{pgfscope}%
\pgfpathrectangle{\pgfqpoint{0.211875in}{0.211875in}}{\pgfqpoint{1.313625in}{1.279725in}}%
\pgfusepath{clip}%
\pgfsetbuttcap%
\pgfsetroundjoin%
\definecolor{currentfill}{rgb}{0.490838,0.119982,0.351115}%
\pgfsetfillcolor{currentfill}%
\pgfsetlinewidth{0.000000pt}%
\definecolor{currentstroke}{rgb}{0.000000,0.000000,0.000000}%
\pgfsetstrokecolor{currentstroke}%
\pgfsetdash{}{0pt}%
\pgfpathmoveto{\pgfqpoint{1.260121in}{0.376413in}}%
\pgfpathlineto{\pgfqpoint{1.273390in}{0.374937in}}%
\pgfpathlineto{\pgfqpoint{1.278433in}{0.379920in}}%
\pgfpathlineto{\pgfqpoint{1.282182in}{0.392846in}}%
\pgfpathlineto{\pgfqpoint{1.279652in}{0.405773in}}%
\pgfpathlineto{\pgfqpoint{1.273390in}{0.413159in}}%
\pgfpathlineto{\pgfqpoint{1.260121in}{0.411462in}}%
\pgfpathlineto{\pgfqpoint{1.256281in}{0.405773in}}%
\pgfpathlineto{\pgfqpoint{1.254152in}{0.392846in}}%
\pgfpathlineto{\pgfqpoint{1.257312in}{0.379920in}}%
\pgfpathclose%
\pgfusepath{fill}%
\end{pgfscope}%
\begin{pgfscope}%
\pgfpathrectangle{\pgfqpoint{0.211875in}{0.211875in}}{\pgfqpoint{1.313625in}{1.279725in}}%
\pgfusepath{clip}%
\pgfsetbuttcap%
\pgfsetroundjoin%
\definecolor{currentfill}{rgb}{0.490838,0.119982,0.351115}%
\pgfsetfillcolor{currentfill}%
\pgfsetlinewidth{0.000000pt}%
\definecolor{currentstroke}{rgb}{0.000000,0.000000,0.000000}%
\pgfsetstrokecolor{currentstroke}%
\pgfsetdash{}{0pt}%
\pgfpathmoveto{\pgfqpoint{1.379542in}{0.374413in}}%
\pgfpathlineto{\pgfqpoint{1.392811in}{0.374412in}}%
\pgfpathlineto{\pgfqpoint{1.397739in}{0.379920in}}%
\pgfpathlineto{\pgfqpoint{1.400963in}{0.392846in}}%
\pgfpathlineto{\pgfqpoint{1.398751in}{0.405773in}}%
\pgfpathlineto{\pgfqpoint{1.392811in}{0.413683in}}%
\pgfpathlineto{\pgfqpoint{1.379542in}{0.413694in}}%
\pgfpathlineto{\pgfqpoint{1.373551in}{0.405773in}}%
\pgfpathlineto{\pgfqpoint{1.371311in}{0.392846in}}%
\pgfpathlineto{\pgfqpoint{1.374587in}{0.379920in}}%
\pgfpathclose%
\pgfusepath{fill}%
\end{pgfscope}%
\begin{pgfscope}%
\pgfpathrectangle{\pgfqpoint{0.211875in}{0.211875in}}{\pgfqpoint{1.313625in}{1.279725in}}%
\pgfusepath{clip}%
\pgfsetbuttcap%
\pgfsetroundjoin%
\definecolor{currentfill}{rgb}{0.490838,0.119982,0.351115}%
\pgfsetfillcolor{currentfill}%
\pgfsetlinewidth{0.000000pt}%
\definecolor{currentstroke}{rgb}{0.000000,0.000000,0.000000}%
\pgfsetstrokecolor{currentstroke}%
\pgfsetdash{}{0pt}%
\pgfpathmoveto{\pgfqpoint{1.498962in}{0.371915in}}%
\pgfpathlineto{\pgfqpoint{1.512231in}{0.373191in}}%
\pgfpathlineto{\pgfqpoint{1.517589in}{0.379920in}}%
\pgfpathlineto{\pgfqpoint{1.520285in}{0.392846in}}%
\pgfpathlineto{\pgfqpoint{1.518376in}{0.405773in}}%
\pgfpathlineto{\pgfqpoint{1.512231in}{0.414943in}}%
\pgfpathlineto{\pgfqpoint{1.498962in}{0.416467in}}%
\pgfpathlineto{\pgfqpoint{1.489831in}{0.405773in}}%
\pgfpathlineto{\pgfqpoint{1.487500in}{0.392846in}}%
\pgfpathlineto{\pgfqpoint{1.490818in}{0.379920in}}%
\pgfpathclose%
\pgfpathmoveto{\pgfqpoint{1.497307in}{0.392846in}}%
\pgfpathlineto{\pgfqpoint{1.498962in}{0.399791in}}%
\pgfpathlineto{\pgfqpoint{1.511125in}{0.392846in}}%
\pgfpathlineto{\pgfqpoint{1.498962in}{0.388209in}}%
\pgfpathclose%
\pgfusepath{fill}%
\end{pgfscope}%
\begin{pgfscope}%
\pgfpathrectangle{\pgfqpoint{0.211875in}{0.211875in}}{\pgfqpoint{1.313625in}{1.279725in}}%
\pgfusepath{clip}%
\pgfsetbuttcap%
\pgfsetroundjoin%
\definecolor{currentfill}{rgb}{0.490838,0.119982,0.351115}%
\pgfsetfillcolor{currentfill}%
\pgfsetlinewidth{0.000000pt}%
\definecolor{currentstroke}{rgb}{0.000000,0.000000,0.000000}%
\pgfsetstrokecolor{currentstroke}%
\pgfsetdash{}{0pt}%
\pgfpathmoveto{\pgfqpoint{0.251682in}{0.441984in}}%
\pgfpathlineto{\pgfqpoint{0.264951in}{0.440663in}}%
\pgfpathlineto{\pgfqpoint{0.278220in}{0.442612in}}%
\pgfpathlineto{\pgfqpoint{0.281678in}{0.444552in}}%
\pgfpathlineto{\pgfqpoint{0.288872in}{0.457479in}}%
\pgfpathlineto{\pgfqpoint{0.290203in}{0.470405in}}%
\pgfpathlineto{\pgfqpoint{0.289838in}{0.483332in}}%
\pgfpathlineto{\pgfqpoint{0.287368in}{0.496258in}}%
\pgfpathlineto{\pgfqpoint{0.278220in}{0.507445in}}%
\pgfpathlineto{\pgfqpoint{0.270173in}{0.509185in}}%
\pgfpathlineto{\pgfqpoint{0.264951in}{0.509959in}}%
\pgfpathlineto{\pgfqpoint{0.258427in}{0.509185in}}%
\pgfpathlineto{\pgfqpoint{0.251682in}{0.507922in}}%
\pgfpathlineto{\pgfqpoint{0.242654in}{0.496258in}}%
\pgfpathlineto{\pgfqpoint{0.240623in}{0.483332in}}%
\pgfpathlineto{\pgfqpoint{0.240276in}{0.470405in}}%
\pgfpathlineto{\pgfqpoint{0.241204in}{0.457479in}}%
\pgfpathlineto{\pgfqpoint{0.246993in}{0.444552in}}%
\pgfpathclose%
\pgfpathmoveto{\pgfqpoint{0.250097in}{0.457479in}}%
\pgfpathlineto{\pgfqpoint{0.247054in}{0.470405in}}%
\pgfpathlineto{\pgfqpoint{0.247498in}{0.483332in}}%
\pgfpathlineto{\pgfqpoint{0.251682in}{0.495530in}}%
\pgfpathlineto{\pgfqpoint{0.252503in}{0.496258in}}%
\pgfpathlineto{\pgfqpoint{0.264951in}{0.501169in}}%
\pgfpathlineto{\pgfqpoint{0.278220in}{0.496260in}}%
\pgfpathlineto{\pgfqpoint{0.278221in}{0.496258in}}%
\pgfpathlineto{\pgfqpoint{0.282990in}{0.483332in}}%
\pgfpathlineto{\pgfqpoint{0.283446in}{0.470405in}}%
\pgfpathlineto{\pgfqpoint{0.280156in}{0.457479in}}%
\pgfpathlineto{\pgfqpoint{0.278220in}{0.454701in}}%
\pgfpathlineto{\pgfqpoint{0.264951in}{0.449652in}}%
\pgfpathlineto{\pgfqpoint{0.251682in}{0.455053in}}%
\pgfpathclose%
\pgfusepath{fill}%
\end{pgfscope}%
\begin{pgfscope}%
\pgfpathrectangle{\pgfqpoint{0.211875in}{0.211875in}}{\pgfqpoint{1.313625in}{1.279725in}}%
\pgfusepath{clip}%
\pgfsetbuttcap%
\pgfsetroundjoin%
\definecolor{currentfill}{rgb}{0.490838,0.119982,0.351115}%
\pgfsetfillcolor{currentfill}%
\pgfsetlinewidth{0.000000pt}%
\definecolor{currentstroke}{rgb}{0.000000,0.000000,0.000000}%
\pgfsetstrokecolor{currentstroke}%
\pgfsetdash{}{0pt}%
\pgfpathmoveto{\pgfqpoint{0.384371in}{0.444438in}}%
\pgfpathlineto{\pgfqpoint{0.384915in}{0.444552in}}%
\pgfpathlineto{\pgfqpoint{0.397640in}{0.448977in}}%
\pgfpathlineto{\pgfqpoint{0.402950in}{0.457479in}}%
\pgfpathlineto{\pgfqpoint{0.405074in}{0.470405in}}%
\pgfpathlineto{\pgfqpoint{0.404684in}{0.483332in}}%
\pgfpathlineto{\pgfqpoint{0.401317in}{0.496258in}}%
\pgfpathlineto{\pgfqpoint{0.397640in}{0.501249in}}%
\pgfpathlineto{\pgfqpoint{0.384371in}{0.505932in}}%
\pgfpathlineto{\pgfqpoint{0.371102in}{0.503216in}}%
\pgfpathlineto{\pgfqpoint{0.365103in}{0.496258in}}%
\pgfpathlineto{\pgfqpoint{0.361761in}{0.483332in}}%
\pgfpathlineto{\pgfqpoint{0.361351in}{0.470405in}}%
\pgfpathlineto{\pgfqpoint{0.363360in}{0.457479in}}%
\pgfpathlineto{\pgfqpoint{0.371102in}{0.446900in}}%
\pgfpathlineto{\pgfqpoint{0.383434in}{0.444552in}}%
\pgfpathclose%
\pgfpathmoveto{\pgfqpoint{0.375482in}{0.457479in}}%
\pgfpathlineto{\pgfqpoint{0.371102in}{0.461378in}}%
\pgfpathlineto{\pgfqpoint{0.368417in}{0.470405in}}%
\pgfpathlineto{\pgfqpoint{0.368926in}{0.483332in}}%
\pgfpathlineto{\pgfqpoint{0.371102in}{0.489031in}}%
\pgfpathlineto{\pgfqpoint{0.381958in}{0.496258in}}%
\pgfpathlineto{\pgfqpoint{0.384371in}{0.497021in}}%
\pgfpathlineto{\pgfqpoint{0.386073in}{0.496258in}}%
\pgfpathlineto{\pgfqpoint{0.397640in}{0.484747in}}%
\pgfpathlineto{\pgfqpoint{0.398111in}{0.483332in}}%
\pgfpathlineto{\pgfqpoint{0.398588in}{0.470405in}}%
\pgfpathlineto{\pgfqpoint{0.397640in}{0.466718in}}%
\pgfpathlineto{\pgfqpoint{0.390692in}{0.457479in}}%
\pgfpathlineto{\pgfqpoint{0.384371in}{0.454214in}}%
\pgfpathclose%
\pgfusepath{fill}%
\end{pgfscope}%
\begin{pgfscope}%
\pgfpathrectangle{\pgfqpoint{0.211875in}{0.211875in}}{\pgfqpoint{1.313625in}{1.279725in}}%
\pgfusepath{clip}%
\pgfsetbuttcap%
\pgfsetroundjoin%
\definecolor{currentfill}{rgb}{0.490838,0.119982,0.351115}%
\pgfsetfillcolor{currentfill}%
\pgfsetlinewidth{0.000000pt}%
\definecolor{currentstroke}{rgb}{0.000000,0.000000,0.000000}%
\pgfsetstrokecolor{currentstroke}%
\pgfsetdash{}{0pt}%
\pgfpathmoveto{\pgfqpoint{0.490523in}{0.450953in}}%
\pgfpathlineto{\pgfqpoint{0.503792in}{0.448413in}}%
\pgfpathlineto{\pgfqpoint{0.517061in}{0.456009in}}%
\pgfpathlineto{\pgfqpoint{0.517879in}{0.457479in}}%
\pgfpathlineto{\pgfqpoint{0.520641in}{0.470405in}}%
\pgfpathlineto{\pgfqpoint{0.520231in}{0.483332in}}%
\pgfpathlineto{\pgfqpoint{0.517061in}{0.493957in}}%
\pgfpathlineto{\pgfqpoint{0.515225in}{0.496258in}}%
\pgfpathlineto{\pgfqpoint{0.503792in}{0.502286in}}%
\pgfpathlineto{\pgfqpoint{0.490523in}{0.499695in}}%
\pgfpathlineto{\pgfqpoint{0.487226in}{0.496258in}}%
\pgfpathlineto{\pgfqpoint{0.482612in}{0.483332in}}%
\pgfpathlineto{\pgfqpoint{0.482138in}{0.470405in}}%
\pgfpathlineto{\pgfqpoint{0.485186in}{0.457479in}}%
\pgfpathclose%
\pgfpathmoveto{\pgfqpoint{0.489568in}{0.470405in}}%
\pgfpathlineto{\pgfqpoint{0.490147in}{0.483332in}}%
\pgfpathlineto{\pgfqpoint{0.490523in}{0.484217in}}%
\pgfpathlineto{\pgfqpoint{0.503792in}{0.491366in}}%
\pgfpathlineto{\pgfqpoint{0.510658in}{0.483332in}}%
\pgfpathlineto{\pgfqpoint{0.511650in}{0.470405in}}%
\pgfpathlineto{\pgfqpoint{0.503792in}{0.458957in}}%
\pgfpathlineto{\pgfqpoint{0.490523in}{0.467543in}}%
\pgfpathclose%
\pgfusepath{fill}%
\end{pgfscope}%
\begin{pgfscope}%
\pgfpathrectangle{\pgfqpoint{0.211875in}{0.211875in}}{\pgfqpoint{1.313625in}{1.279725in}}%
\pgfusepath{clip}%
\pgfsetbuttcap%
\pgfsetroundjoin%
\definecolor{currentfill}{rgb}{0.490838,0.119982,0.351115}%
\pgfsetfillcolor{currentfill}%
\pgfsetlinewidth{0.000000pt}%
\definecolor{currentstroke}{rgb}{0.000000,0.000000,0.000000}%
\pgfsetstrokecolor{currentstroke}%
\pgfsetdash{}{0pt}%
\pgfpathmoveto{\pgfqpoint{0.609943in}{0.453919in}}%
\pgfpathlineto{\pgfqpoint{0.623212in}{0.451914in}}%
\pgfpathlineto{\pgfqpoint{0.631138in}{0.457479in}}%
\pgfpathlineto{\pgfqpoint{0.636481in}{0.468928in}}%
\pgfpathlineto{\pgfqpoint{0.636777in}{0.470405in}}%
\pgfpathlineto{\pgfqpoint{0.636481in}{0.479434in}}%
\pgfpathlineto{\pgfqpoint{0.636247in}{0.483332in}}%
\pgfpathlineto{\pgfqpoint{0.627867in}{0.496258in}}%
\pgfpathlineto{\pgfqpoint{0.623212in}{0.499087in}}%
\pgfpathlineto{\pgfqpoint{0.609943in}{0.497105in}}%
\pgfpathlineto{\pgfqpoint{0.609040in}{0.496258in}}%
\pgfpathlineto{\pgfqpoint{0.603171in}{0.483332in}}%
\pgfpathlineto{\pgfqpoint{0.602631in}{0.470405in}}%
\pgfpathlineto{\pgfqpoint{0.606691in}{0.457479in}}%
\pgfpathclose%
\pgfpathmoveto{\pgfqpoint{0.612927in}{0.470405in}}%
\pgfpathlineto{\pgfqpoint{0.616306in}{0.483332in}}%
\pgfpathlineto{\pgfqpoint{0.623212in}{0.485852in}}%
\pgfpathlineto{\pgfqpoint{0.625074in}{0.483332in}}%
\pgfpathlineto{\pgfqpoint{0.625990in}{0.470405in}}%
\pgfpathlineto{\pgfqpoint{0.623212in}{0.465717in}}%
\pgfpathclose%
\pgfusepath{fill}%
\end{pgfscope}%
\begin{pgfscope}%
\pgfpathrectangle{\pgfqpoint{0.211875in}{0.211875in}}{\pgfqpoint{1.313625in}{1.279725in}}%
\pgfusepath{clip}%
\pgfsetbuttcap%
\pgfsetroundjoin%
\definecolor{currentfill}{rgb}{0.490838,0.119982,0.351115}%
\pgfsetfillcolor{currentfill}%
\pgfsetlinewidth{0.000000pt}%
\definecolor{currentstroke}{rgb}{0.000000,0.000000,0.000000}%
\pgfsetstrokecolor{currentstroke}%
\pgfsetdash{}{0pt}%
\pgfpathmoveto{\pgfqpoint{0.729364in}{0.456017in}}%
\pgfpathlineto{\pgfqpoint{0.742633in}{0.454967in}}%
\pgfpathlineto{\pgfqpoint{0.745775in}{0.457479in}}%
\pgfpathlineto{\pgfqpoint{0.751769in}{0.470405in}}%
\pgfpathlineto{\pgfqpoint{0.751049in}{0.483332in}}%
\pgfpathlineto{\pgfqpoint{0.742679in}{0.496258in}}%
\pgfpathlineto{\pgfqpoint{0.742633in}{0.496291in}}%
\pgfpathlineto{\pgfqpoint{0.742191in}{0.496258in}}%
\pgfpathlineto{\pgfqpoint{0.729364in}{0.494608in}}%
\pgfpathlineto{\pgfqpoint{0.723417in}{0.483332in}}%
\pgfpathlineto{\pgfqpoint{0.722804in}{0.470405in}}%
\pgfpathlineto{\pgfqpoint{0.727868in}{0.457479in}}%
\pgfpathclose%
\pgfusepath{fill}%
\end{pgfscope}%
\begin{pgfscope}%
\pgfpathrectangle{\pgfqpoint{0.211875in}{0.211875in}}{\pgfqpoint{1.313625in}{1.279725in}}%
\pgfusepath{clip}%
\pgfsetbuttcap%
\pgfsetroundjoin%
\definecolor{currentfill}{rgb}{0.490838,0.119982,0.351115}%
\pgfsetfillcolor{currentfill}%
\pgfsetlinewidth{0.000000pt}%
\definecolor{currentstroke}{rgb}{0.000000,0.000000,0.000000}%
\pgfsetstrokecolor{currentstroke}%
\pgfsetdash{}{0pt}%
\pgfpathmoveto{\pgfqpoint{0.848784in}{0.457396in}}%
\pgfpathlineto{\pgfqpoint{0.854030in}{0.457479in}}%
\pgfpathlineto{\pgfqpoint{0.862053in}{0.457716in}}%
\pgfpathlineto{\pgfqpoint{0.867875in}{0.470405in}}%
\pgfpathlineto{\pgfqpoint{0.867197in}{0.483332in}}%
\pgfpathlineto{\pgfqpoint{0.862053in}{0.492277in}}%
\pgfpathlineto{\pgfqpoint{0.848784in}{0.492595in}}%
\pgfpathlineto{\pgfqpoint{0.843304in}{0.483332in}}%
\pgfpathlineto{\pgfqpoint{0.842612in}{0.470405in}}%
\pgfpathlineto{\pgfqpoint{0.848689in}{0.457479in}}%
\pgfpathclose%
\pgfusepath{fill}%
\end{pgfscope}%
\begin{pgfscope}%
\pgfpathrectangle{\pgfqpoint{0.211875in}{0.211875in}}{\pgfqpoint{1.313625in}{1.279725in}}%
\pgfusepath{clip}%
\pgfsetbuttcap%
\pgfsetroundjoin%
\definecolor{currentfill}{rgb}{0.490838,0.119982,0.351115}%
\pgfsetfillcolor{currentfill}%
\pgfsetlinewidth{0.000000pt}%
\definecolor{currentstroke}{rgb}{0.000000,0.000000,0.000000}%
\pgfsetstrokecolor{currentstroke}%
\pgfsetdash{}{0pt}%
\pgfpathmoveto{\pgfqpoint{1.326466in}{0.457291in}}%
\pgfpathlineto{\pgfqpoint{1.326938in}{0.457479in}}%
\pgfpathlineto{\pgfqpoint{1.339735in}{0.469805in}}%
\pgfpathlineto{\pgfqpoint{1.339910in}{0.470405in}}%
\pgfpathlineto{\pgfqpoint{1.339735in}{0.474675in}}%
\pgfpathlineto{\pgfqpoint{1.338755in}{0.483332in}}%
\pgfpathlineto{\pgfqpoint{1.326466in}{0.492863in}}%
\pgfpathlineto{\pgfqpoint{1.316372in}{0.483332in}}%
\pgfpathlineto{\pgfqpoint{1.315171in}{0.470405in}}%
\pgfpathlineto{\pgfqpoint{1.326081in}{0.457479in}}%
\pgfpathclose%
\pgfusepath{fill}%
\end{pgfscope}%
\begin{pgfscope}%
\pgfpathrectangle{\pgfqpoint{0.211875in}{0.211875in}}{\pgfqpoint{1.313625in}{1.279725in}}%
\pgfusepath{clip}%
\pgfsetbuttcap%
\pgfsetroundjoin%
\definecolor{currentfill}{rgb}{0.490838,0.119982,0.351115}%
\pgfsetfillcolor{currentfill}%
\pgfsetlinewidth{0.000000pt}%
\definecolor{currentstroke}{rgb}{0.000000,0.000000,0.000000}%
\pgfsetstrokecolor{currentstroke}%
\pgfsetdash{}{0pt}%
\pgfpathmoveto{\pgfqpoint{1.445886in}{0.456002in}}%
\pgfpathlineto{\pgfqpoint{1.448915in}{0.457479in}}%
\pgfpathlineto{\pgfqpoint{1.459070in}{0.470405in}}%
\pgfpathlineto{\pgfqpoint{1.457915in}{0.483332in}}%
\pgfpathlineto{\pgfqpoint{1.445886in}{0.494818in}}%
\pgfpathlineto{\pgfqpoint{1.432617in}{0.484996in}}%
\pgfpathlineto{\pgfqpoint{1.431990in}{0.483332in}}%
\pgfpathlineto{\pgfqpoint{1.431468in}{0.470405in}}%
\pgfpathlineto{\pgfqpoint{1.432617in}{0.466502in}}%
\pgfpathlineto{\pgfqpoint{1.442158in}{0.457479in}}%
\pgfpathclose%
\pgfusepath{fill}%
\end{pgfscope}%
\begin{pgfscope}%
\pgfpathrectangle{\pgfqpoint{0.211875in}{0.211875in}}{\pgfqpoint{1.313625in}{1.279725in}}%
\pgfusepath{clip}%
\pgfsetbuttcap%
\pgfsetroundjoin%
\definecolor{currentfill}{rgb}{0.490838,0.119982,0.351115}%
\pgfsetfillcolor{currentfill}%
\pgfsetlinewidth{0.000000pt}%
\definecolor{currentstroke}{rgb}{0.000000,0.000000,0.000000}%
\pgfsetstrokecolor{currentstroke}%
\pgfsetdash{}{0pt}%
\pgfpathmoveto{\pgfqpoint{0.968205in}{0.458748in}}%
\pgfpathlineto{\pgfqpoint{0.981473in}{0.461900in}}%
\pgfpathlineto{\pgfqpoint{0.984945in}{0.470405in}}%
\pgfpathlineto{\pgfqpoint{0.984306in}{0.483332in}}%
\pgfpathlineto{\pgfqpoint{0.981473in}{0.488846in}}%
\pgfpathlineto{\pgfqpoint{0.968205in}{0.491486in}}%
\pgfpathlineto{\pgfqpoint{0.962749in}{0.483332in}}%
\pgfpathlineto{\pgfqpoint{0.961966in}{0.470405in}}%
\pgfpathclose%
\pgfusepath{fill}%
\end{pgfscope}%
\begin{pgfscope}%
\pgfpathrectangle{\pgfqpoint{0.211875in}{0.211875in}}{\pgfqpoint{1.313625in}{1.279725in}}%
\pgfusepath{clip}%
\pgfsetbuttcap%
\pgfsetroundjoin%
\definecolor{currentfill}{rgb}{0.490838,0.119982,0.351115}%
\pgfsetfillcolor{currentfill}%
\pgfsetlinewidth{0.000000pt}%
\definecolor{currentstroke}{rgb}{0.000000,0.000000,0.000000}%
\pgfsetstrokecolor{currentstroke}%
\pgfsetdash{}{0pt}%
\pgfpathmoveto{\pgfqpoint{1.087625in}{0.459148in}}%
\pgfpathlineto{\pgfqpoint{1.100894in}{0.465347in}}%
\pgfpathlineto{\pgfqpoint{1.102737in}{0.470405in}}%
\pgfpathlineto{\pgfqpoint{1.102134in}{0.483332in}}%
\pgfpathlineto{\pgfqpoint{1.100894in}{0.486023in}}%
\pgfpathlineto{\pgfqpoint{1.087625in}{0.491191in}}%
\pgfpathlineto{\pgfqpoint{1.081611in}{0.483332in}}%
\pgfpathlineto{\pgfqpoint{1.080720in}{0.470405in}}%
\pgfpathclose%
\pgfusepath{fill}%
\end{pgfscope}%
\begin{pgfscope}%
\pgfpathrectangle{\pgfqpoint{0.211875in}{0.211875in}}{\pgfqpoint{1.313625in}{1.279725in}}%
\pgfusepath{clip}%
\pgfsetbuttcap%
\pgfsetroundjoin%
\definecolor{currentfill}{rgb}{0.490838,0.119982,0.351115}%
\pgfsetfillcolor{currentfill}%
\pgfsetlinewidth{0.000000pt}%
\definecolor{currentstroke}{rgb}{0.000000,0.000000,0.000000}%
\pgfsetstrokecolor{currentstroke}%
\pgfsetdash{}{0pt}%
\pgfpathmoveto{\pgfqpoint{1.207045in}{0.458599in}}%
\pgfpathlineto{\pgfqpoint{1.220314in}{0.468014in}}%
\pgfpathlineto{\pgfqpoint{1.221093in}{0.470405in}}%
\pgfpathlineto{\pgfqpoint{1.220525in}{0.483332in}}%
\pgfpathlineto{\pgfqpoint{1.220314in}{0.483841in}}%
\pgfpathlineto{\pgfqpoint{1.207045in}{0.491657in}}%
\pgfpathlineto{\pgfqpoint{1.199640in}{0.483332in}}%
\pgfpathlineto{\pgfqpoint{1.198614in}{0.470405in}}%
\pgfpathclose%
\pgfusepath{fill}%
\end{pgfscope}%
\begin{pgfscope}%
\pgfpathrectangle{\pgfqpoint{0.211875in}{0.211875in}}{\pgfqpoint{1.313625in}{1.279725in}}%
\pgfusepath{clip}%
\pgfsetbuttcap%
\pgfsetroundjoin%
\definecolor{currentfill}{rgb}{0.490838,0.119982,0.351115}%
\pgfsetfillcolor{currentfill}%
\pgfsetlinewidth{0.000000pt}%
\definecolor{currentstroke}{rgb}{0.000000,0.000000,0.000000}%
\pgfsetstrokecolor{currentstroke}%
\pgfsetdash{}{0pt}%
\pgfpathmoveto{\pgfqpoint{0.225144in}{0.527778in}}%
\pgfpathlineto{\pgfqpoint{0.229069in}{0.535038in}}%
\pgfpathlineto{\pgfqpoint{0.230976in}{0.547964in}}%
\pgfpathlineto{\pgfqpoint{0.231057in}{0.560891in}}%
\pgfpathlineto{\pgfqpoint{0.229634in}{0.573817in}}%
\pgfpathlineto{\pgfqpoint{0.225144in}{0.584609in}}%
\pgfpathlineto{\pgfqpoint{0.222495in}{0.586744in}}%
\pgfpathlineto{\pgfqpoint{0.211875in}{0.590781in}}%
\pgfpathlineto{\pgfqpoint{0.211875in}{0.586744in}}%
\pgfpathlineto{\pgfqpoint{0.211875in}{0.581635in}}%
\pgfpathlineto{\pgfqpoint{0.220819in}{0.573817in}}%
\pgfpathlineto{\pgfqpoint{0.225144in}{0.561378in}}%
\pgfpathlineto{\pgfqpoint{0.225228in}{0.560891in}}%
\pgfpathlineto{\pgfqpoint{0.225144in}{0.558199in}}%
\pgfpathlineto{\pgfqpoint{0.224585in}{0.547964in}}%
\pgfpathlineto{\pgfqpoint{0.217228in}{0.535038in}}%
\pgfpathlineto{\pgfqpoint{0.211875in}{0.531560in}}%
\pgfpathlineto{\pgfqpoint{0.211875in}{0.522262in}}%
\pgfpathclose%
\pgfusepath{fill}%
\end{pgfscope}%
\begin{pgfscope}%
\pgfpathrectangle{\pgfqpoint{0.211875in}{0.211875in}}{\pgfqpoint{1.313625in}{1.279725in}}%
\pgfusepath{clip}%
\pgfsetbuttcap%
\pgfsetroundjoin%
\definecolor{currentfill}{rgb}{0.490838,0.119982,0.351115}%
\pgfsetfillcolor{currentfill}%
\pgfsetlinewidth{0.000000pt}%
\definecolor{currentstroke}{rgb}{0.000000,0.000000,0.000000}%
\pgfsetstrokecolor{currentstroke}%
\pgfsetdash{}{0pt}%
\pgfpathmoveto{\pgfqpoint{0.318027in}{0.526729in}}%
\pgfpathlineto{\pgfqpoint{0.331295in}{0.527163in}}%
\pgfpathlineto{\pgfqpoint{0.342083in}{0.535038in}}%
\pgfpathlineto{\pgfqpoint{0.344564in}{0.540703in}}%
\pgfpathlineto{\pgfqpoint{0.346047in}{0.547964in}}%
\pgfpathlineto{\pgfqpoint{0.346291in}{0.560891in}}%
\pgfpathlineto{\pgfqpoint{0.344564in}{0.572211in}}%
\pgfpathlineto{\pgfqpoint{0.344111in}{0.573817in}}%
\pgfpathlineto{\pgfqpoint{0.331295in}{0.586416in}}%
\pgfpathlineto{\pgfqpoint{0.321925in}{0.586744in}}%
\pgfpathlineto{\pgfqpoint{0.318027in}{0.586853in}}%
\pgfpathlineto{\pgfqpoint{0.317731in}{0.586744in}}%
\pgfpathlineto{\pgfqpoint{0.304758in}{0.575355in}}%
\pgfpathlineto{\pgfqpoint{0.304182in}{0.573817in}}%
\pgfpathlineto{\pgfqpoint{0.302283in}{0.560891in}}%
\pgfpathlineto{\pgfqpoint{0.302475in}{0.547964in}}%
\pgfpathlineto{\pgfqpoint{0.304758in}{0.536975in}}%
\pgfpathlineto{\pgfqpoint{0.305658in}{0.535038in}}%
\pgfpathclose%
\pgfpathmoveto{\pgfqpoint{0.311024in}{0.547964in}}%
\pgfpathlineto{\pgfqpoint{0.310154in}{0.560891in}}%
\pgfpathlineto{\pgfqpoint{0.315266in}{0.573817in}}%
\pgfpathlineto{\pgfqpoint{0.318027in}{0.576366in}}%
\pgfpathlineto{\pgfqpoint{0.331295in}{0.575958in}}%
\pgfpathlineto{\pgfqpoint{0.333473in}{0.573817in}}%
\pgfpathlineto{\pgfqpoint{0.338468in}{0.560891in}}%
\pgfpathlineto{\pgfqpoint{0.337588in}{0.547964in}}%
\pgfpathlineto{\pgfqpoint{0.331295in}{0.537673in}}%
\pgfpathlineto{\pgfqpoint{0.318027in}{0.537075in}}%
\pgfpathclose%
\pgfusepath{fill}%
\end{pgfscope}%
\begin{pgfscope}%
\pgfpathrectangle{\pgfqpoint{0.211875in}{0.211875in}}{\pgfqpoint{1.313625in}{1.279725in}}%
\pgfusepath{clip}%
\pgfsetbuttcap%
\pgfsetroundjoin%
\definecolor{currentfill}{rgb}{0.490838,0.119982,0.351115}%
\pgfsetfillcolor{currentfill}%
\pgfsetlinewidth{0.000000pt}%
\definecolor{currentstroke}{rgb}{0.000000,0.000000,0.000000}%
\pgfsetstrokecolor{currentstroke}%
\pgfsetdash{}{0pt}%
\pgfpathmoveto{\pgfqpoint{0.437447in}{0.530518in}}%
\pgfpathlineto{\pgfqpoint{0.450716in}{0.531774in}}%
\pgfpathlineto{\pgfqpoint{0.454717in}{0.535038in}}%
\pgfpathlineto{\pgfqpoint{0.460577in}{0.547964in}}%
\pgfpathlineto{\pgfqpoint{0.461114in}{0.560891in}}%
\pgfpathlineto{\pgfqpoint{0.457540in}{0.573817in}}%
\pgfpathlineto{\pgfqpoint{0.450716in}{0.581313in}}%
\pgfpathlineto{\pgfqpoint{0.437447in}{0.582771in}}%
\pgfpathlineto{\pgfqpoint{0.426464in}{0.573817in}}%
\pgfpathlineto{\pgfqpoint{0.424178in}{0.567581in}}%
\pgfpathlineto{\pgfqpoint{0.422991in}{0.560891in}}%
\pgfpathlineto{\pgfqpoint{0.423351in}{0.547964in}}%
\pgfpathlineto{\pgfqpoint{0.424178in}{0.544562in}}%
\pgfpathlineto{\pgfqpoint{0.429821in}{0.535038in}}%
\pgfpathclose%
\pgfpathmoveto{\pgfqpoint{0.433567in}{0.547964in}}%
\pgfpathlineto{\pgfqpoint{0.432330in}{0.560891in}}%
\pgfpathlineto{\pgfqpoint{0.437447in}{0.571235in}}%
\pgfpathlineto{\pgfqpoint{0.450716in}{0.567233in}}%
\pgfpathlineto{\pgfqpoint{0.453081in}{0.560891in}}%
\pgfpathlineto{\pgfqpoint{0.452070in}{0.547964in}}%
\pgfpathlineto{\pgfqpoint{0.450716in}{0.545481in}}%
\pgfpathlineto{\pgfqpoint{0.437447in}{0.542648in}}%
\pgfpathclose%
\pgfusepath{fill}%
\end{pgfscope}%
\begin{pgfscope}%
\pgfpathrectangle{\pgfqpoint{0.211875in}{0.211875in}}{\pgfqpoint{1.313625in}{1.279725in}}%
\pgfusepath{clip}%
\pgfsetbuttcap%
\pgfsetroundjoin%
\definecolor{currentfill}{rgb}{0.490838,0.119982,0.351115}%
\pgfsetfillcolor{currentfill}%
\pgfsetlinewidth{0.000000pt}%
\definecolor{currentstroke}{rgb}{0.000000,0.000000,0.000000}%
\pgfsetstrokecolor{currentstroke}%
\pgfsetdash{}{0pt}%
\pgfpathmoveto{\pgfqpoint{0.556867in}{0.533607in}}%
\pgfpathlineto{\pgfqpoint{0.564681in}{0.535038in}}%
\pgfpathlineto{\pgfqpoint{0.570136in}{0.536815in}}%
\pgfpathlineto{\pgfqpoint{0.575576in}{0.547964in}}%
\pgfpathlineto{\pgfqpoint{0.576234in}{0.560891in}}%
\pgfpathlineto{\pgfqpoint{0.572301in}{0.573817in}}%
\pgfpathlineto{\pgfqpoint{0.570136in}{0.576464in}}%
\pgfpathlineto{\pgfqpoint{0.556867in}{0.579408in}}%
\pgfpathlineto{\pgfqpoint{0.549011in}{0.573817in}}%
\pgfpathlineto{\pgfqpoint{0.543598in}{0.561953in}}%
\pgfpathlineto{\pgfqpoint{0.543386in}{0.560891in}}%
\pgfpathlineto{\pgfqpoint{0.543598in}{0.555315in}}%
\pgfpathlineto{\pgfqpoint{0.544146in}{0.547964in}}%
\pgfpathlineto{\pgfqpoint{0.554101in}{0.535038in}}%
\pgfpathclose%
\pgfpathmoveto{\pgfqpoint{0.556250in}{0.547964in}}%
\pgfpathlineto{\pgfqpoint{0.554596in}{0.560891in}}%
\pgfpathlineto{\pgfqpoint{0.556867in}{0.564899in}}%
\pgfpathlineto{\pgfqpoint{0.563721in}{0.560891in}}%
\pgfpathlineto{\pgfqpoint{0.558716in}{0.547964in}}%
\pgfpathlineto{\pgfqpoint{0.556867in}{0.547227in}}%
\pgfpathclose%
\pgfusepath{fill}%
\end{pgfscope}%
\begin{pgfscope}%
\pgfpathrectangle{\pgfqpoint{0.211875in}{0.211875in}}{\pgfqpoint{1.313625in}{1.279725in}}%
\pgfusepath{clip}%
\pgfsetbuttcap%
\pgfsetroundjoin%
\definecolor{currentfill}{rgb}{0.490838,0.119982,0.351115}%
\pgfsetfillcolor{currentfill}%
\pgfsetlinewidth{0.000000pt}%
\definecolor{currentstroke}{rgb}{0.000000,0.000000,0.000000}%
\pgfsetstrokecolor{currentstroke}%
\pgfsetdash{}{0pt}%
\pgfpathmoveto{\pgfqpoint{0.676288in}{0.536713in}}%
\pgfpathlineto{\pgfqpoint{0.689557in}{0.543499in}}%
\pgfpathlineto{\pgfqpoint{0.691509in}{0.547964in}}%
\pgfpathlineto{\pgfqpoint{0.692251in}{0.560891in}}%
\pgfpathlineto{\pgfqpoint{0.689557in}{0.569819in}}%
\pgfpathlineto{\pgfqpoint{0.684630in}{0.573817in}}%
\pgfpathlineto{\pgfqpoint{0.676288in}{0.576686in}}%
\pgfpathlineto{\pgfqpoint{0.671593in}{0.573817in}}%
\pgfpathlineto{\pgfqpoint{0.663965in}{0.560891in}}%
\pgfpathlineto{\pgfqpoint{0.665294in}{0.547964in}}%
\pgfpathclose%
\pgfusepath{fill}%
\end{pgfscope}%
\begin{pgfscope}%
\pgfpathrectangle{\pgfqpoint{0.211875in}{0.211875in}}{\pgfqpoint{1.313625in}{1.279725in}}%
\pgfusepath{clip}%
\pgfsetbuttcap%
\pgfsetroundjoin%
\definecolor{currentfill}{rgb}{0.490838,0.119982,0.351115}%
\pgfsetfillcolor{currentfill}%
\pgfsetlinewidth{0.000000pt}%
\definecolor{currentstroke}{rgb}{0.000000,0.000000,0.000000}%
\pgfsetstrokecolor{currentstroke}%
\pgfsetdash{}{0pt}%
\pgfpathmoveto{\pgfqpoint{0.795708in}{0.539821in}}%
\pgfpathlineto{\pgfqpoint{0.806797in}{0.547964in}}%
\pgfpathlineto{\pgfqpoint{0.808950in}{0.560891in}}%
\pgfpathlineto{\pgfqpoint{0.797313in}{0.573817in}}%
\pgfpathlineto{\pgfqpoint{0.795708in}{0.574529in}}%
\pgfpathlineto{\pgfqpoint{0.794319in}{0.573817in}}%
\pgfpathlineto{\pgfqpoint{0.784395in}{0.560891in}}%
\pgfpathlineto{\pgfqpoint{0.786202in}{0.547964in}}%
\pgfpathclose%
\pgfusepath{fill}%
\end{pgfscope}%
\begin{pgfscope}%
\pgfpathrectangle{\pgfqpoint{0.211875in}{0.211875in}}{\pgfqpoint{1.313625in}{1.279725in}}%
\pgfusepath{clip}%
\pgfsetbuttcap%
\pgfsetroundjoin%
\definecolor{currentfill}{rgb}{0.490838,0.119982,0.351115}%
\pgfsetfillcolor{currentfill}%
\pgfsetlinewidth{0.000000pt}%
\definecolor{currentstroke}{rgb}{0.000000,0.000000,0.000000}%
\pgfsetstrokecolor{currentstroke}%
\pgfsetdash{}{0pt}%
\pgfpathmoveto{\pgfqpoint{0.915129in}{0.542182in}}%
\pgfpathlineto{\pgfqpoint{0.921553in}{0.547964in}}%
\pgfpathlineto{\pgfqpoint{0.923449in}{0.560891in}}%
\pgfpathlineto{\pgfqpoint{0.915129in}{0.571949in}}%
\pgfpathlineto{\pgfqpoint{0.904318in}{0.560891in}}%
\pgfpathlineto{\pgfqpoint{0.906761in}{0.547964in}}%
\pgfpathclose%
\pgfusepath{fill}%
\end{pgfscope}%
\begin{pgfscope}%
\pgfpathrectangle{\pgfqpoint{0.211875in}{0.211875in}}{\pgfqpoint{1.313625in}{1.279725in}}%
\pgfusepath{clip}%
\pgfsetbuttcap%
\pgfsetroundjoin%
\definecolor{currentfill}{rgb}{0.490838,0.119982,0.351115}%
\pgfsetfillcolor{currentfill}%
\pgfsetlinewidth{0.000000pt}%
\definecolor{currentstroke}{rgb}{0.000000,0.000000,0.000000}%
\pgfsetstrokecolor{currentstroke}%
\pgfsetdash{}{0pt}%
\pgfpathmoveto{\pgfqpoint{1.034549in}{0.543846in}}%
\pgfpathlineto{\pgfqpoint{1.038409in}{0.547964in}}%
\pgfpathlineto{\pgfqpoint{1.040090in}{0.560891in}}%
\pgfpathlineto{\pgfqpoint{1.034549in}{0.569621in}}%
\pgfpathlineto{\pgfqpoint{1.023289in}{0.560891in}}%
\pgfpathlineto{\pgfqpoint{1.026694in}{0.547964in}}%
\pgfpathclose%
\pgfusepath{fill}%
\end{pgfscope}%
\begin{pgfscope}%
\pgfpathrectangle{\pgfqpoint{0.211875in}{0.211875in}}{\pgfqpoint{1.313625in}{1.279725in}}%
\pgfusepath{clip}%
\pgfsetbuttcap%
\pgfsetroundjoin%
\definecolor{currentfill}{rgb}{0.490838,0.119982,0.351115}%
\pgfsetfillcolor{currentfill}%
\pgfsetlinewidth{0.000000pt}%
\definecolor{currentstroke}{rgb}{0.000000,0.000000,0.000000}%
\pgfsetstrokecolor{currentstroke}%
\pgfsetdash{}{0pt}%
\pgfpathmoveto{\pgfqpoint{1.153970in}{0.544829in}}%
\pgfpathlineto{\pgfqpoint{1.156502in}{0.547964in}}%
\pgfpathlineto{\pgfqpoint{1.157990in}{0.560891in}}%
\pgfpathlineto{\pgfqpoint{1.153970in}{0.568235in}}%
\pgfpathlineto{\pgfqpoint{1.140701in}{0.561344in}}%
\pgfpathlineto{\pgfqpoint{1.140545in}{0.560891in}}%
\pgfpathlineto{\pgfqpoint{1.140701in}{0.558854in}}%
\pgfpathlineto{\pgfqpoint{1.145120in}{0.547964in}}%
\pgfpathclose%
\pgfusepath{fill}%
\end{pgfscope}%
\begin{pgfscope}%
\pgfpathrectangle{\pgfqpoint{0.211875in}{0.211875in}}{\pgfqpoint{1.313625in}{1.279725in}}%
\pgfusepath{clip}%
\pgfsetbuttcap%
\pgfsetroundjoin%
\definecolor{currentfill}{rgb}{0.490838,0.119982,0.351115}%
\pgfsetfillcolor{currentfill}%
\pgfsetlinewidth{0.000000pt}%
\definecolor{currentstroke}{rgb}{0.000000,0.000000,0.000000}%
\pgfsetstrokecolor{currentstroke}%
\pgfsetdash{}{0pt}%
\pgfpathmoveto{\pgfqpoint{1.260121in}{0.547460in}}%
\pgfpathlineto{\pgfqpoint{1.273390in}{0.545120in}}%
\pgfpathlineto{\pgfqpoint{1.275400in}{0.547964in}}%
\pgfpathlineto{\pgfqpoint{1.276706in}{0.560891in}}%
\pgfpathlineto{\pgfqpoint{1.273390in}{0.567804in}}%
\pgfpathlineto{\pgfqpoint{1.260121in}{0.564501in}}%
\pgfpathlineto{\pgfqpoint{1.258740in}{0.560891in}}%
\pgfpathlineto{\pgfqpoint{1.259839in}{0.547964in}}%
\pgfpathclose%
\pgfusepath{fill}%
\end{pgfscope}%
\begin{pgfscope}%
\pgfpathrectangle{\pgfqpoint{0.211875in}{0.211875in}}{\pgfqpoint{1.313625in}{1.279725in}}%
\pgfusepath{clip}%
\pgfsetbuttcap%
\pgfsetroundjoin%
\definecolor{currentfill}{rgb}{0.490838,0.119982,0.351115}%
\pgfsetfillcolor{currentfill}%
\pgfsetlinewidth{0.000000pt}%
\definecolor{currentstroke}{rgb}{0.000000,0.000000,0.000000}%
\pgfsetstrokecolor{currentstroke}%
\pgfsetdash{}{0pt}%
\pgfpathmoveto{\pgfqpoint{1.379542in}{0.544617in}}%
\pgfpathlineto{\pgfqpoint{1.392811in}{0.544675in}}%
\pgfpathlineto{\pgfqpoint{1.394864in}{0.547964in}}%
\pgfpathlineto{\pgfqpoint{1.395994in}{0.560891in}}%
\pgfpathlineto{\pgfqpoint{1.392811in}{0.568383in}}%
\pgfpathlineto{\pgfqpoint{1.379542in}{0.568473in}}%
\pgfpathlineto{\pgfqpoint{1.376292in}{0.560891in}}%
\pgfpathlineto{\pgfqpoint{1.377435in}{0.547964in}}%
\pgfpathclose%
\pgfusepath{fill}%
\end{pgfscope}%
\begin{pgfscope}%
\pgfpathrectangle{\pgfqpoint{0.211875in}{0.211875in}}{\pgfqpoint{1.313625in}{1.279725in}}%
\pgfusepath{clip}%
\pgfsetbuttcap%
\pgfsetroundjoin%
\definecolor{currentfill}{rgb}{0.490838,0.119982,0.351115}%
\pgfsetfillcolor{currentfill}%
\pgfsetlinewidth{0.000000pt}%
\definecolor{currentstroke}{rgb}{0.000000,0.000000,0.000000}%
\pgfsetstrokecolor{currentstroke}%
\pgfsetdash{}{0pt}%
\pgfpathmoveto{\pgfqpoint{1.498962in}{0.541207in}}%
\pgfpathlineto{\pgfqpoint{1.512231in}{0.543413in}}%
\pgfpathlineto{\pgfqpoint{1.514759in}{0.547964in}}%
\pgfpathlineto{\pgfqpoint{1.515714in}{0.560891in}}%
\pgfpathlineto{\pgfqpoint{1.512231in}{0.570076in}}%
\pgfpathlineto{\pgfqpoint{1.498962in}{0.573246in}}%
\pgfpathlineto{\pgfqpoint{1.492983in}{0.560891in}}%
\pgfpathlineto{\pgfqpoint{1.494149in}{0.547964in}}%
\pgfpathclose%
\pgfusepath{fill}%
\end{pgfscope}%
\begin{pgfscope}%
\pgfpathrectangle{\pgfqpoint{0.211875in}{0.211875in}}{\pgfqpoint{1.313625in}{1.279725in}}%
\pgfusepath{clip}%
\pgfsetbuttcap%
\pgfsetroundjoin%
\definecolor{currentfill}{rgb}{0.490838,0.119982,0.351115}%
\pgfsetfillcolor{currentfill}%
\pgfsetlinewidth{0.000000pt}%
\definecolor{currentstroke}{rgb}{0.000000,0.000000,0.000000}%
\pgfsetstrokecolor{currentstroke}%
\pgfsetdash{}{0pt}%
\pgfpathmoveto{\pgfqpoint{0.251682in}{0.610277in}}%
\pgfpathlineto{\pgfqpoint{0.264951in}{0.607326in}}%
\pgfpathlineto{\pgfqpoint{0.278220in}{0.610469in}}%
\pgfpathlineto{\pgfqpoint{0.280612in}{0.612597in}}%
\pgfpathlineto{\pgfqpoint{0.286188in}{0.625523in}}%
\pgfpathlineto{\pgfqpoint{0.287187in}{0.638450in}}%
\pgfpathlineto{\pgfqpoint{0.285789in}{0.651377in}}%
\pgfpathlineto{\pgfqpoint{0.279428in}{0.664303in}}%
\pgfpathlineto{\pgfqpoint{0.278220in}{0.665349in}}%
\pgfpathlineto{\pgfqpoint{0.264951in}{0.668668in}}%
\pgfpathlineto{\pgfqpoint{0.251682in}{0.665416in}}%
\pgfpathlineto{\pgfqpoint{0.250438in}{0.664303in}}%
\pgfpathlineto{\pgfqpoint{0.244579in}{0.651377in}}%
\pgfpathlineto{\pgfqpoint{0.243333in}{0.638450in}}%
\pgfpathlineto{\pgfqpoint{0.244166in}{0.625523in}}%
\pgfpathlineto{\pgfqpoint{0.249161in}{0.612597in}}%
\pgfpathclose%
\pgfpathmoveto{\pgfqpoint{0.251846in}{0.625523in}}%
\pgfpathlineto{\pgfqpoint{0.251682in}{0.625952in}}%
\pgfpathlineto{\pgfqpoint{0.249938in}{0.638450in}}%
\pgfpathlineto{\pgfqpoint{0.251682in}{0.649099in}}%
\pgfpathlineto{\pgfqpoint{0.252697in}{0.651377in}}%
\pgfpathlineto{\pgfqpoint{0.264951in}{0.659072in}}%
\pgfpathlineto{\pgfqpoint{0.278220in}{0.651585in}}%
\pgfpathlineto{\pgfqpoint{0.278329in}{0.651377in}}%
\pgfpathlineto{\pgfqpoint{0.280593in}{0.638450in}}%
\pgfpathlineto{\pgfqpoint{0.278677in}{0.625523in}}%
\pgfpathlineto{\pgfqpoint{0.278220in}{0.624612in}}%
\pgfpathlineto{\pgfqpoint{0.264951in}{0.617008in}}%
\pgfpathclose%
\pgfusepath{fill}%
\end{pgfscope}%
\begin{pgfscope}%
\pgfpathrectangle{\pgfqpoint{0.211875in}{0.211875in}}{\pgfqpoint{1.313625in}{1.279725in}}%
\pgfusepath{clip}%
\pgfsetbuttcap%
\pgfsetroundjoin%
\definecolor{currentfill}{rgb}{0.490838,0.119982,0.351115}%
\pgfsetfillcolor{currentfill}%
\pgfsetlinewidth{0.000000pt}%
\definecolor{currentstroke}{rgb}{0.000000,0.000000,0.000000}%
\pgfsetstrokecolor{currentstroke}%
\pgfsetdash{}{0pt}%
\pgfpathmoveto{\pgfqpoint{0.384371in}{0.611251in}}%
\pgfpathlineto{\pgfqpoint{0.388507in}{0.612597in}}%
\pgfpathlineto{\pgfqpoint{0.397640in}{0.618272in}}%
\pgfpathlineto{\pgfqpoint{0.400893in}{0.625523in}}%
\pgfpathlineto{\pgfqpoint{0.402249in}{0.638450in}}%
\pgfpathlineto{\pgfqpoint{0.400532in}{0.651377in}}%
\pgfpathlineto{\pgfqpoint{0.397640in}{0.657481in}}%
\pgfpathlineto{\pgfqpoint{0.386181in}{0.664303in}}%
\pgfpathlineto{\pgfqpoint{0.384371in}{0.664882in}}%
\pgfpathlineto{\pgfqpoint{0.381663in}{0.664303in}}%
\pgfpathlineto{\pgfqpoint{0.371102in}{0.660323in}}%
\pgfpathlineto{\pgfqpoint{0.366205in}{0.651377in}}%
\pgfpathlineto{\pgfqpoint{0.364453in}{0.638450in}}%
\pgfpathlineto{\pgfqpoint{0.365802in}{0.625523in}}%
\pgfpathlineto{\pgfqpoint{0.371102in}{0.615345in}}%
\pgfpathlineto{\pgfqpoint{0.378151in}{0.612597in}}%
\pgfpathclose%
\pgfpathmoveto{\pgfqpoint{0.378536in}{0.625523in}}%
\pgfpathlineto{\pgfqpoint{0.371773in}{0.638450in}}%
\pgfpathlineto{\pgfqpoint{0.379515in}{0.651377in}}%
\pgfpathlineto{\pgfqpoint{0.384371in}{0.653824in}}%
\pgfpathlineto{\pgfqpoint{0.387911in}{0.651377in}}%
\pgfpathlineto{\pgfqpoint{0.393707in}{0.638450in}}%
\pgfpathlineto{\pgfqpoint{0.388620in}{0.625523in}}%
\pgfpathlineto{\pgfqpoint{0.384371in}{0.622481in}}%
\pgfpathclose%
\pgfusepath{fill}%
\end{pgfscope}%
\begin{pgfscope}%
\pgfpathrectangle{\pgfqpoint{0.211875in}{0.211875in}}{\pgfqpoint{1.313625in}{1.279725in}}%
\pgfusepath{clip}%
\pgfsetbuttcap%
\pgfsetroundjoin%
\definecolor{currentfill}{rgb}{0.490838,0.119982,0.351115}%
\pgfsetfillcolor{currentfill}%
\pgfsetlinewidth{0.000000pt}%
\definecolor{currentstroke}{rgb}{0.000000,0.000000,0.000000}%
\pgfsetstrokecolor{currentstroke}%
\pgfsetdash{}{0pt}%
\pgfpathmoveto{\pgfqpoint{0.490523in}{0.619772in}}%
\pgfpathlineto{\pgfqpoint{0.503792in}{0.615552in}}%
\pgfpathlineto{\pgfqpoint{0.515583in}{0.625523in}}%
\pgfpathlineto{\pgfqpoint{0.517061in}{0.630904in}}%
\pgfpathlineto{\pgfqpoint{0.517967in}{0.638450in}}%
\pgfpathlineto{\pgfqpoint{0.517061in}{0.644787in}}%
\pgfpathlineto{\pgfqpoint{0.514920in}{0.651377in}}%
\pgfpathlineto{\pgfqpoint{0.503792in}{0.660462in}}%
\pgfpathlineto{\pgfqpoint{0.490523in}{0.656216in}}%
\pgfpathlineto{\pgfqpoint{0.487574in}{0.651377in}}%
\pgfpathlineto{\pgfqpoint{0.485324in}{0.638450in}}%
\pgfpathlineto{\pgfqpoint{0.487175in}{0.625523in}}%
\pgfpathclose%
\pgfpathmoveto{\pgfqpoint{0.497881in}{0.638450in}}%
\pgfpathlineto{\pgfqpoint{0.503792in}{0.645327in}}%
\pgfpathlineto{\pgfqpoint{0.506579in}{0.638450in}}%
\pgfpathlineto{\pgfqpoint{0.503792in}{0.630824in}}%
\pgfpathclose%
\pgfusepath{fill}%
\end{pgfscope}%
\begin{pgfscope}%
\pgfpathrectangle{\pgfqpoint{0.211875in}{0.211875in}}{\pgfqpoint{1.313625in}{1.279725in}}%
\pgfusepath{clip}%
\pgfsetbuttcap%
\pgfsetroundjoin%
\definecolor{currentfill}{rgb}{0.490838,0.119982,0.351115}%
\pgfsetfillcolor{currentfill}%
\pgfsetlinewidth{0.000000pt}%
\definecolor{currentstroke}{rgb}{0.000000,0.000000,0.000000}%
\pgfsetstrokecolor{currentstroke}%
\pgfsetdash{}{0pt}%
\pgfpathmoveto{\pgfqpoint{0.609943in}{0.622987in}}%
\pgfpathlineto{\pgfqpoint{0.623212in}{0.619828in}}%
\pgfpathlineto{\pgfqpoint{0.629036in}{0.625523in}}%
\pgfpathlineto{\pgfqpoint{0.632401in}{0.638450in}}%
\pgfpathlineto{\pgfqpoint{0.628483in}{0.651377in}}%
\pgfpathlineto{\pgfqpoint{0.623212in}{0.656350in}}%
\pgfpathlineto{\pgfqpoint{0.609943in}{0.653216in}}%
\pgfpathlineto{\pgfqpoint{0.608694in}{0.651377in}}%
\pgfpathlineto{\pgfqpoint{0.605945in}{0.638450in}}%
\pgfpathlineto{\pgfqpoint{0.608293in}{0.625523in}}%
\pgfpathclose%
\pgfusepath{fill}%
\end{pgfscope}%
\begin{pgfscope}%
\pgfpathrectangle{\pgfqpoint{0.211875in}{0.211875in}}{\pgfqpoint{1.313625in}{1.279725in}}%
\pgfusepath{clip}%
\pgfsetbuttcap%
\pgfsetroundjoin%
\definecolor{currentfill}{rgb}{0.490838,0.119982,0.351115}%
\pgfsetfillcolor{currentfill}%
\pgfsetlinewidth{0.000000pt}%
\definecolor{currentstroke}{rgb}{0.000000,0.000000,0.000000}%
\pgfsetstrokecolor{currentstroke}%
\pgfsetdash{}{0pt}%
\pgfpathmoveto{\pgfqpoint{0.729364in}{0.625233in}}%
\pgfpathlineto{\pgfqpoint{0.742633in}{0.623607in}}%
\pgfpathlineto{\pgfqpoint{0.744351in}{0.625523in}}%
\pgfpathlineto{\pgfqpoint{0.747707in}{0.638450in}}%
\pgfpathlineto{\pgfqpoint{0.743877in}{0.651377in}}%
\pgfpathlineto{\pgfqpoint{0.742633in}{0.652714in}}%
\pgfpathlineto{\pgfqpoint{0.731445in}{0.651377in}}%
\pgfpathlineto{\pgfqpoint{0.729364in}{0.650675in}}%
\pgfpathlineto{\pgfqpoint{0.726305in}{0.638450in}}%
\pgfpathlineto{\pgfqpoint{0.729152in}{0.625523in}}%
\pgfpathclose%
\pgfusepath{fill}%
\end{pgfscope}%
\begin{pgfscope}%
\pgfpathrectangle{\pgfqpoint{0.211875in}{0.211875in}}{\pgfqpoint{1.313625in}{1.279725in}}%
\pgfusepath{clip}%
\pgfsetbuttcap%
\pgfsetroundjoin%
\definecolor{currentfill}{rgb}{0.490838,0.119982,0.351115}%
\pgfsetfillcolor{currentfill}%
\pgfsetlinewidth{0.000000pt}%
\definecolor{currentstroke}{rgb}{0.000000,0.000000,0.000000}%
\pgfsetstrokecolor{currentstroke}%
\pgfsetdash{}{0pt}%
\pgfpathmoveto{\pgfqpoint{1.445886in}{0.624632in}}%
\pgfpathlineto{\pgfqpoint{1.447187in}{0.625523in}}%
\pgfpathlineto{\pgfqpoint{1.452801in}{0.638450in}}%
\pgfpathlineto{\pgfqpoint{1.446467in}{0.651377in}}%
\pgfpathlineto{\pgfqpoint{1.445886in}{0.651761in}}%
\pgfpathlineto{\pgfqpoint{1.445165in}{0.651377in}}%
\pgfpathlineto{\pgfqpoint{1.437238in}{0.638450in}}%
\pgfpathlineto{\pgfqpoint{1.444270in}{0.625523in}}%
\pgfpathclose%
\pgfusepath{fill}%
\end{pgfscope}%
\begin{pgfscope}%
\pgfpathrectangle{\pgfqpoint{0.211875in}{0.211875in}}{\pgfqpoint{1.313625in}{1.279725in}}%
\pgfusepath{clip}%
\pgfsetbuttcap%
\pgfsetroundjoin%
\definecolor{currentfill}{rgb}{0.490838,0.119982,0.351115}%
\pgfsetfillcolor{currentfill}%
\pgfsetlinewidth{0.000000pt}%
\definecolor{currentstroke}{rgb}{0.000000,0.000000,0.000000}%
\pgfsetstrokecolor{currentstroke}%
\pgfsetdash{}{0pt}%
\pgfpathmoveto{\pgfqpoint{0.848784in}{0.628817in}}%
\pgfpathlineto{\pgfqpoint{0.862053in}{0.629547in}}%
\pgfpathlineto{\pgfqpoint{0.864224in}{0.638450in}}%
\pgfpathlineto{\pgfqpoint{0.862053in}{0.646405in}}%
\pgfpathlineto{\pgfqpoint{0.848784in}{0.647058in}}%
\pgfpathlineto{\pgfqpoint{0.846370in}{0.638450in}}%
\pgfpathclose%
\pgfusepath{fill}%
\end{pgfscope}%
\begin{pgfscope}%
\pgfpathrectangle{\pgfqpoint{0.211875in}{0.211875in}}{\pgfqpoint{1.313625in}{1.279725in}}%
\pgfusepath{clip}%
\pgfsetbuttcap%
\pgfsetroundjoin%
\definecolor{currentfill}{rgb}{0.490838,0.119982,0.351115}%
\pgfsetfillcolor{currentfill}%
\pgfsetlinewidth{0.000000pt}%
\definecolor{currentstroke}{rgb}{0.000000,0.000000,0.000000}%
\pgfsetstrokecolor{currentstroke}%
\pgfsetdash{}{0pt}%
\pgfpathmoveto{\pgfqpoint{0.968205in}{0.630972in}}%
\pgfpathlineto{\pgfqpoint{0.981473in}{0.637845in}}%
\pgfpathlineto{\pgfqpoint{0.981605in}{0.638450in}}%
\pgfpathlineto{\pgfqpoint{0.981473in}{0.638988in}}%
\pgfpathlineto{\pgfqpoint{0.968205in}{0.645161in}}%
\pgfpathlineto{\pgfqpoint{0.966077in}{0.638450in}}%
\pgfpathclose%
\pgfusepath{fill}%
\end{pgfscope}%
\begin{pgfscope}%
\pgfpathrectangle{\pgfqpoint{0.211875in}{0.211875in}}{\pgfqpoint{1.313625in}{1.279725in}}%
\pgfusepath{clip}%
\pgfsetbuttcap%
\pgfsetroundjoin%
\definecolor{currentfill}{rgb}{0.490838,0.119982,0.351115}%
\pgfsetfillcolor{currentfill}%
\pgfsetlinewidth{0.000000pt}%
\definecolor{currentstroke}{rgb}{0.000000,0.000000,0.000000}%
\pgfsetstrokecolor{currentstroke}%
\pgfsetdash{}{0pt}%
\pgfpathmoveto{\pgfqpoint{1.087625in}{0.631379in}}%
\pgfpathlineto{\pgfqpoint{1.094956in}{0.638450in}}%
\pgfpathlineto{\pgfqpoint{1.087625in}{0.644815in}}%
\pgfpathlineto{\pgfqpoint{1.085318in}{0.638450in}}%
\pgfpathclose%
\pgfusepath{fill}%
\end{pgfscope}%
\begin{pgfscope}%
\pgfpathrectangle{\pgfqpoint{0.211875in}{0.211875in}}{\pgfqpoint{1.313625in}{1.279725in}}%
\pgfusepath{clip}%
\pgfsetbuttcap%
\pgfsetroundjoin%
\definecolor{currentfill}{rgb}{0.490838,0.119982,0.351115}%
\pgfsetfillcolor{currentfill}%
\pgfsetlinewidth{0.000000pt}%
\definecolor{currentstroke}{rgb}{0.000000,0.000000,0.000000}%
\pgfsetstrokecolor{currentstroke}%
\pgfsetdash{}{0pt}%
\pgfpathmoveto{\pgfqpoint{1.207045in}{0.630165in}}%
\pgfpathlineto{\pgfqpoint{1.212989in}{0.638450in}}%
\pgfpathlineto{\pgfqpoint{1.207045in}{0.645922in}}%
\pgfpathlineto{\pgfqpoint{1.203897in}{0.638450in}}%
\pgfpathclose%
\pgfusepath{fill}%
\end{pgfscope}%
\begin{pgfscope}%
\pgfpathrectangle{\pgfqpoint{0.211875in}{0.211875in}}{\pgfqpoint{1.313625in}{1.279725in}}%
\pgfusepath{clip}%
\pgfsetbuttcap%
\pgfsetroundjoin%
\definecolor{currentfill}{rgb}{0.490838,0.119982,0.351115}%
\pgfsetfillcolor{currentfill}%
\pgfsetlinewidth{0.000000pt}%
\definecolor{currentstroke}{rgb}{0.000000,0.000000,0.000000}%
\pgfsetstrokecolor{currentstroke}%
\pgfsetdash{}{0pt}%
\pgfpathmoveto{\pgfqpoint{1.326466in}{0.627377in}}%
\pgfpathlineto{\pgfqpoint{1.332565in}{0.638450in}}%
\pgfpathlineto{\pgfqpoint{1.326466in}{0.648445in}}%
\pgfpathlineto{\pgfqpoint{1.321451in}{0.638450in}}%
\pgfpathclose%
\pgfusepath{fill}%
\end{pgfscope}%
\begin{pgfscope}%
\pgfpathrectangle{\pgfqpoint{0.211875in}{0.211875in}}{\pgfqpoint{1.313625in}{1.279725in}}%
\pgfusepath{clip}%
\pgfsetbuttcap%
\pgfsetroundjoin%
\definecolor{currentfill}{rgb}{0.490838,0.119982,0.351115}%
\pgfsetfillcolor{currentfill}%
\pgfsetlinewidth{0.000000pt}%
\definecolor{currentstroke}{rgb}{0.000000,0.000000,0.000000}%
\pgfsetstrokecolor{currentstroke}%
\pgfsetdash{}{0pt}%
\pgfpathmoveto{\pgfqpoint{0.216441in}{0.690156in}}%
\pgfpathlineto{\pgfqpoint{0.225144in}{0.697527in}}%
\pgfpathlineto{\pgfqpoint{0.227307in}{0.703083in}}%
\pgfpathlineto{\pgfqpoint{0.228845in}{0.716009in}}%
\pgfpathlineto{\pgfqpoint{0.228299in}{0.728936in}}%
\pgfpathlineto{\pgfqpoint{0.225144in}{0.740989in}}%
\pgfpathlineto{\pgfqpoint{0.224619in}{0.741862in}}%
\pgfpathlineto{\pgfqpoint{0.211875in}{0.749952in}}%
\pgfpathlineto{\pgfqpoint{0.211875in}{0.741862in}}%
\pgfpathlineto{\pgfqpoint{0.211875in}{0.740516in}}%
\pgfpathlineto{\pgfqpoint{0.220239in}{0.728936in}}%
\pgfpathlineto{\pgfqpoint{0.221631in}{0.716009in}}%
\pgfpathlineto{\pgfqpoint{0.216900in}{0.703083in}}%
\pgfpathlineto{\pgfqpoint{0.211875in}{0.698555in}}%
\pgfpathlineto{\pgfqpoint{0.211875in}{0.690156in}}%
\pgfpathlineto{\pgfqpoint{0.211875in}{0.688400in}}%
\pgfpathclose%
\pgfusepath{fill}%
\end{pgfscope}%
\begin{pgfscope}%
\pgfpathrectangle{\pgfqpoint{0.211875in}{0.211875in}}{\pgfqpoint{1.313625in}{1.279725in}}%
\pgfusepath{clip}%
\pgfsetbuttcap%
\pgfsetroundjoin%
\definecolor{currentfill}{rgb}{0.490838,0.119982,0.351115}%
\pgfsetfillcolor{currentfill}%
\pgfsetlinewidth{0.000000pt}%
\definecolor{currentstroke}{rgb}{0.000000,0.000000,0.000000}%
\pgfsetstrokecolor{currentstroke}%
\pgfsetdash{}{0pt}%
\pgfpathmoveto{\pgfqpoint{0.318027in}{0.693180in}}%
\pgfpathlineto{\pgfqpoint{0.331295in}{0.693650in}}%
\pgfpathlineto{\pgfqpoint{0.340591in}{0.703083in}}%
\pgfpathlineto{\pgfqpoint{0.343865in}{0.716009in}}%
\pgfpathlineto{\pgfqpoint{0.342831in}{0.728936in}}%
\pgfpathlineto{\pgfqpoint{0.335854in}{0.741862in}}%
\pgfpathlineto{\pgfqpoint{0.331295in}{0.745110in}}%
\pgfpathlineto{\pgfqpoint{0.318027in}{0.745500in}}%
\pgfpathlineto{\pgfqpoint{0.312486in}{0.741862in}}%
\pgfpathlineto{\pgfqpoint{0.305485in}{0.728936in}}%
\pgfpathlineto{\pgfqpoint{0.304758in}{0.719850in}}%
\pgfpathlineto{\pgfqpoint{0.304567in}{0.716009in}}%
\pgfpathlineto{\pgfqpoint{0.304758in}{0.714554in}}%
\pgfpathlineto{\pgfqpoint{0.307631in}{0.703083in}}%
\pgfpathclose%
\pgfpathmoveto{\pgfqpoint{0.313659in}{0.716009in}}%
\pgfpathlineto{\pgfqpoint{0.315188in}{0.728936in}}%
\pgfpathlineto{\pgfqpoint{0.318027in}{0.733145in}}%
\pgfpathlineto{\pgfqpoint{0.331295in}{0.732616in}}%
\pgfpathlineto{\pgfqpoint{0.333655in}{0.728936in}}%
\pgfpathlineto{\pgfqpoint{0.335140in}{0.716009in}}%
\pgfpathlineto{\pgfqpoint{0.331295in}{0.706105in}}%
\pgfpathlineto{\pgfqpoint{0.318027in}{0.705267in}}%
\pgfpathclose%
\pgfusepath{fill}%
\end{pgfscope}%
\begin{pgfscope}%
\pgfpathrectangle{\pgfqpoint{0.211875in}{0.211875in}}{\pgfqpoint{1.313625in}{1.279725in}}%
\pgfusepath{clip}%
\pgfsetbuttcap%
\pgfsetroundjoin%
\definecolor{currentfill}{rgb}{0.490838,0.119982,0.351115}%
\pgfsetfillcolor{currentfill}%
\pgfsetlinewidth{0.000000pt}%
\definecolor{currentstroke}{rgb}{0.000000,0.000000,0.000000}%
\pgfsetstrokecolor{currentstroke}%
\pgfsetdash{}{0pt}%
\pgfpathmoveto{\pgfqpoint{0.437447in}{0.697385in}}%
\pgfpathlineto{\pgfqpoint{0.450716in}{0.699071in}}%
\pgfpathlineto{\pgfqpoint{0.454248in}{0.703083in}}%
\pgfpathlineto{\pgfqpoint{0.458052in}{0.716009in}}%
\pgfpathlineto{\pgfqpoint{0.456924in}{0.728936in}}%
\pgfpathlineto{\pgfqpoint{0.450716in}{0.739768in}}%
\pgfpathlineto{\pgfqpoint{0.438585in}{0.741862in}}%
\pgfpathlineto{\pgfqpoint{0.437447in}{0.741982in}}%
\pgfpathlineto{\pgfqpoint{0.437241in}{0.741862in}}%
\pgfpathlineto{\pgfqpoint{0.427428in}{0.728936in}}%
\pgfpathlineto{\pgfqpoint{0.426059in}{0.716009in}}%
\pgfpathlineto{\pgfqpoint{0.430663in}{0.703083in}}%
\pgfpathclose%
\pgfpathmoveto{\pgfqpoint{0.436208in}{0.716009in}}%
\pgfpathlineto{\pgfqpoint{0.437447in}{0.724461in}}%
\pgfpathlineto{\pgfqpoint{0.445145in}{0.716009in}}%
\pgfpathlineto{\pgfqpoint{0.437447in}{0.713324in}}%
\pgfpathclose%
\pgfusepath{fill}%
\end{pgfscope}%
\begin{pgfscope}%
\pgfpathrectangle{\pgfqpoint{0.211875in}{0.211875in}}{\pgfqpoint{1.313625in}{1.279725in}}%
\pgfusepath{clip}%
\pgfsetbuttcap%
\pgfsetroundjoin%
\definecolor{currentfill}{rgb}{0.490838,0.119982,0.351115}%
\pgfsetfillcolor{currentfill}%
\pgfsetlinewidth{0.000000pt}%
\definecolor{currentstroke}{rgb}{0.000000,0.000000,0.000000}%
\pgfsetstrokecolor{currentstroke}%
\pgfsetdash{}{0pt}%
\pgfpathmoveto{\pgfqpoint{0.556867in}{0.700824in}}%
\pgfpathlineto{\pgfqpoint{0.565889in}{0.703083in}}%
\pgfpathlineto{\pgfqpoint{0.570136in}{0.705552in}}%
\pgfpathlineto{\pgfqpoint{0.573372in}{0.716009in}}%
\pgfpathlineto{\pgfqpoint{0.572180in}{0.728936in}}%
\pgfpathlineto{\pgfqpoint{0.570136in}{0.732908in}}%
\pgfpathlineto{\pgfqpoint{0.556867in}{0.737548in}}%
\pgfpathlineto{\pgfqpoint{0.549326in}{0.728936in}}%
\pgfpathlineto{\pgfqpoint{0.547554in}{0.716009in}}%
\pgfpathlineto{\pgfqpoint{0.553784in}{0.703083in}}%
\pgfpathclose%
\pgfusepath{fill}%
\end{pgfscope}%
\begin{pgfscope}%
\pgfpathrectangle{\pgfqpoint{0.211875in}{0.211875in}}{\pgfqpoint{1.313625in}{1.279725in}}%
\pgfusepath{clip}%
\pgfsetbuttcap%
\pgfsetroundjoin%
\definecolor{currentfill}{rgb}{0.490838,0.119982,0.351115}%
\pgfsetfillcolor{currentfill}%
\pgfsetlinewidth{0.000000pt}%
\definecolor{currentstroke}{rgb}{0.000000,0.000000,0.000000}%
\pgfsetstrokecolor{currentstroke}%
\pgfsetdash{}{0pt}%
\pgfpathmoveto{\pgfqpoint{0.676288in}{0.704176in}}%
\pgfpathlineto{\pgfqpoint{0.689528in}{0.716009in}}%
\pgfpathlineto{\pgfqpoint{0.685398in}{0.728936in}}%
\pgfpathlineto{\pgfqpoint{0.676288in}{0.733889in}}%
\pgfpathlineto{\pgfqpoint{0.671234in}{0.728936in}}%
\pgfpathlineto{\pgfqpoint{0.668975in}{0.716009in}}%
\pgfpathclose%
\pgfusepath{fill}%
\end{pgfscope}%
\begin{pgfscope}%
\pgfpathrectangle{\pgfqpoint{0.211875in}{0.211875in}}{\pgfqpoint{1.313625in}{1.279725in}}%
\pgfusepath{clip}%
\pgfsetbuttcap%
\pgfsetroundjoin%
\definecolor{currentfill}{rgb}{0.490838,0.119982,0.351115}%
\pgfsetfillcolor{currentfill}%
\pgfsetlinewidth{0.000000pt}%
\definecolor{currentstroke}{rgb}{0.000000,0.000000,0.000000}%
\pgfsetstrokecolor{currentstroke}%
\pgfsetdash{}{0pt}%
\pgfpathmoveto{\pgfqpoint{0.795708in}{0.708751in}}%
\pgfpathlineto{\pgfqpoint{0.801981in}{0.716009in}}%
\pgfpathlineto{\pgfqpoint{0.798599in}{0.728936in}}%
\pgfpathlineto{\pgfqpoint{0.795708in}{0.730969in}}%
\pgfpathlineto{\pgfqpoint{0.793231in}{0.728936in}}%
\pgfpathlineto{\pgfqpoint{0.790348in}{0.716009in}}%
\pgfpathclose%
\pgfusepath{fill}%
\end{pgfscope}%
\begin{pgfscope}%
\pgfpathrectangle{\pgfqpoint{0.211875in}{0.211875in}}{\pgfqpoint{1.313625in}{1.279725in}}%
\pgfusepath{clip}%
\pgfsetbuttcap%
\pgfsetroundjoin%
\definecolor{currentfill}{rgb}{0.490838,0.119982,0.351115}%
\pgfsetfillcolor{currentfill}%
\pgfsetlinewidth{0.000000pt}%
\definecolor{currentstroke}{rgb}{0.000000,0.000000,0.000000}%
\pgfsetstrokecolor{currentstroke}%
\pgfsetdash{}{0pt}%
\pgfpathmoveto{\pgfqpoint{0.915129in}{0.712272in}}%
\pgfpathlineto{\pgfqpoint{0.917762in}{0.716009in}}%
\pgfpathlineto{\pgfqpoint{0.915129in}{0.727855in}}%
\pgfpathlineto{\pgfqpoint{0.911707in}{0.716009in}}%
\pgfpathclose%
\pgfusepath{fill}%
\end{pgfscope}%
\begin{pgfscope}%
\pgfpathrectangle{\pgfqpoint{0.211875in}{0.211875in}}{\pgfqpoint{1.313625in}{1.279725in}}%
\pgfusepath{clip}%
\pgfsetbuttcap%
\pgfsetroundjoin%
\definecolor{currentfill}{rgb}{0.490838,0.119982,0.351115}%
\pgfsetfillcolor{currentfill}%
\pgfsetlinewidth{0.000000pt}%
\definecolor{currentstroke}{rgb}{0.000000,0.000000,0.000000}%
\pgfsetstrokecolor{currentstroke}%
\pgfsetdash{}{0pt}%
\pgfpathmoveto{\pgfqpoint{1.034549in}{0.714811in}}%
\pgfpathlineto{\pgfqpoint{1.035261in}{0.716009in}}%
\pgfpathlineto{\pgfqpoint{1.034549in}{0.719804in}}%
\pgfpathlineto{\pgfqpoint{1.033102in}{0.716009in}}%
\pgfpathclose%
\pgfusepath{fill}%
\end{pgfscope}%
\begin{pgfscope}%
\pgfpathrectangle{\pgfqpoint{0.211875in}{0.211875in}}{\pgfqpoint{1.313625in}{1.279725in}}%
\pgfusepath{clip}%
\pgfsetbuttcap%
\pgfsetroundjoin%
\definecolor{currentfill}{rgb}{0.490838,0.119982,0.351115}%
\pgfsetfillcolor{currentfill}%
\pgfsetlinewidth{0.000000pt}%
\definecolor{currentstroke}{rgb}{0.000000,0.000000,0.000000}%
\pgfsetstrokecolor{currentstroke}%
\pgfsetdash{}{0pt}%
\pgfpathmoveto{\pgfqpoint{1.498962in}{0.711130in}}%
\pgfpathlineto{\pgfqpoint{1.512231in}{0.715059in}}%
\pgfpathlineto{\pgfqpoint{1.512565in}{0.716009in}}%
\pgfpathlineto{\pgfqpoint{1.512231in}{0.718965in}}%
\pgfpathlineto{\pgfqpoint{1.501696in}{0.728936in}}%
\pgfpathlineto{\pgfqpoint{1.498962in}{0.729430in}}%
\pgfpathlineto{\pgfqpoint{1.498593in}{0.728936in}}%
\pgfpathlineto{\pgfqpoint{1.496758in}{0.716009in}}%
\pgfpathclose%
\pgfusepath{fill}%
\end{pgfscope}%
\begin{pgfscope}%
\pgfpathrectangle{\pgfqpoint{0.211875in}{0.211875in}}{\pgfqpoint{1.313625in}{1.279725in}}%
\pgfusepath{clip}%
\pgfsetbuttcap%
\pgfsetroundjoin%
\definecolor{currentfill}{rgb}{0.490838,0.119982,0.351115}%
\pgfsetfillcolor{currentfill}%
\pgfsetlinewidth{0.000000pt}%
\definecolor{currentstroke}{rgb}{0.000000,0.000000,0.000000}%
\pgfsetstrokecolor{currentstroke}%
\pgfsetdash{}{0pt}%
\pgfpathmoveto{\pgfqpoint{0.251682in}{0.777457in}}%
\pgfpathlineto{\pgfqpoint{0.264951in}{0.772984in}}%
\pgfpathlineto{\pgfqpoint{0.278220in}{0.777308in}}%
\pgfpathlineto{\pgfqpoint{0.280852in}{0.780642in}}%
\pgfpathlineto{\pgfqpoint{0.284630in}{0.793568in}}%
\pgfpathlineto{\pgfqpoint{0.284715in}{0.806495in}}%
\pgfpathlineto{\pgfqpoint{0.281368in}{0.819421in}}%
\pgfpathlineto{\pgfqpoint{0.278220in}{0.823764in}}%
\pgfpathlineto{\pgfqpoint{0.264951in}{0.828368in}}%
\pgfpathlineto{\pgfqpoint{0.251682in}{0.823536in}}%
\pgfpathlineto{\pgfqpoint{0.248875in}{0.819421in}}%
\pgfpathlineto{\pgfqpoint{0.245785in}{0.806495in}}%
\pgfpathlineto{\pgfqpoint{0.245847in}{0.793568in}}%
\pgfpathlineto{\pgfqpoint{0.249311in}{0.780642in}}%
\pgfpathclose%
\pgfpathmoveto{\pgfqpoint{0.254219in}{0.793568in}}%
\pgfpathlineto{\pgfqpoint{0.253728in}{0.806495in}}%
\pgfpathlineto{\pgfqpoint{0.264951in}{0.818514in}}%
\pgfpathlineto{\pgfqpoint{0.277569in}{0.806495in}}%
\pgfpathlineto{\pgfqpoint{0.276981in}{0.793568in}}%
\pgfpathlineto{\pgfqpoint{0.264951in}{0.783332in}}%
\pgfpathclose%
\pgfusepath{fill}%
\end{pgfscope}%
\begin{pgfscope}%
\pgfpathrectangle{\pgfqpoint{0.211875in}{0.211875in}}{\pgfqpoint{1.313625in}{1.279725in}}%
\pgfusepath{clip}%
\pgfsetbuttcap%
\pgfsetroundjoin%
\definecolor{currentfill}{rgb}{0.490838,0.119982,0.351115}%
\pgfsetfillcolor{currentfill}%
\pgfsetlinewidth{0.000000pt}%
\definecolor{currentstroke}{rgb}{0.000000,0.000000,0.000000}%
\pgfsetstrokecolor{currentstroke}%
\pgfsetdash{}{0pt}%
\pgfpathmoveto{\pgfqpoint{0.384371in}{0.777287in}}%
\pgfpathlineto{\pgfqpoint{0.391677in}{0.780642in}}%
\pgfpathlineto{\pgfqpoint{0.397640in}{0.786978in}}%
\pgfpathlineto{\pgfqpoint{0.399674in}{0.793568in}}%
\pgfpathlineto{\pgfqpoint{0.399814in}{0.806495in}}%
\pgfpathlineto{\pgfqpoint{0.397640in}{0.814241in}}%
\pgfpathlineto{\pgfqpoint{0.393422in}{0.819421in}}%
\pgfpathlineto{\pgfqpoint{0.384371in}{0.823973in}}%
\pgfpathlineto{\pgfqpoint{0.371667in}{0.819421in}}%
\pgfpathlineto{\pgfqpoint{0.371102in}{0.818953in}}%
\pgfpathlineto{\pgfqpoint{0.367084in}{0.806495in}}%
\pgfpathlineto{\pgfqpoint{0.367219in}{0.793568in}}%
\pgfpathlineto{\pgfqpoint{0.371102in}{0.782676in}}%
\pgfpathlineto{\pgfqpoint{0.373985in}{0.780642in}}%
\pgfpathclose%
\pgfpathmoveto{\pgfqpoint{0.380510in}{0.793568in}}%
\pgfpathlineto{\pgfqpoint{0.379692in}{0.806495in}}%
\pgfpathlineto{\pgfqpoint{0.384371in}{0.810519in}}%
\pgfpathlineto{\pgfqpoint{0.387829in}{0.806495in}}%
\pgfpathlineto{\pgfqpoint{0.387217in}{0.793568in}}%
\pgfpathlineto{\pgfqpoint{0.384371in}{0.790611in}}%
\pgfpathclose%
\pgfusepath{fill}%
\end{pgfscope}%
\begin{pgfscope}%
\pgfpathrectangle{\pgfqpoint{0.211875in}{0.211875in}}{\pgfqpoint{1.313625in}{1.279725in}}%
\pgfusepath{clip}%
\pgfsetbuttcap%
\pgfsetroundjoin%
\definecolor{currentfill}{rgb}{0.490838,0.119982,0.351115}%
\pgfsetfillcolor{currentfill}%
\pgfsetlinewidth{0.000000pt}%
\definecolor{currentstroke}{rgb}{0.000000,0.000000,0.000000}%
\pgfsetstrokecolor{currentstroke}%
\pgfsetdash{}{0pt}%
\pgfpathmoveto{\pgfqpoint{0.490523in}{0.788150in}}%
\pgfpathlineto{\pgfqpoint{0.503792in}{0.781424in}}%
\pgfpathlineto{\pgfqpoint{0.513677in}{0.793568in}}%
\pgfpathlineto{\pgfqpoint{0.514046in}{0.806495in}}%
\pgfpathlineto{\pgfqpoint{0.504888in}{0.819421in}}%
\pgfpathlineto{\pgfqpoint{0.503792in}{0.820071in}}%
\pgfpathlineto{\pgfqpoint{0.501377in}{0.819421in}}%
\pgfpathlineto{\pgfqpoint{0.490523in}{0.813079in}}%
\pgfpathlineto{\pgfqpoint{0.488158in}{0.806495in}}%
\pgfpathlineto{\pgfqpoint{0.488363in}{0.793568in}}%
\pgfpathclose%
\pgfusepath{fill}%
\end{pgfscope}%
\begin{pgfscope}%
\pgfpathrectangle{\pgfqpoint{0.211875in}{0.211875in}}{\pgfqpoint{1.313625in}{1.279725in}}%
\pgfusepath{clip}%
\pgfsetbuttcap%
\pgfsetroundjoin%
\definecolor{currentfill}{rgb}{0.490838,0.119982,0.351115}%
\pgfsetfillcolor{currentfill}%
\pgfsetlinewidth{0.000000pt}%
\definecolor{currentstroke}{rgb}{0.000000,0.000000,0.000000}%
\pgfsetstrokecolor{currentstroke}%
\pgfsetdash{}{0pt}%
\pgfpathmoveto{\pgfqpoint{0.609943in}{0.792099in}}%
\pgfpathlineto{\pgfqpoint{0.623212in}{0.787183in}}%
\pgfpathlineto{\pgfqpoint{0.627702in}{0.793568in}}%
\pgfpathlineto{\pgfqpoint{0.628096in}{0.806495in}}%
\pgfpathlineto{\pgfqpoint{0.623212in}{0.814271in}}%
\pgfpathlineto{\pgfqpoint{0.609943in}{0.808816in}}%
\pgfpathlineto{\pgfqpoint{0.609014in}{0.806495in}}%
\pgfpathlineto{\pgfqpoint{0.609289in}{0.793568in}}%
\pgfpathclose%
\pgfusepath{fill}%
\end{pgfscope}%
\begin{pgfscope}%
\pgfpathrectangle{\pgfqpoint{0.211875in}{0.211875in}}{\pgfqpoint{1.313625in}{1.279725in}}%
\pgfusepath{clip}%
\pgfsetbuttcap%
\pgfsetroundjoin%
\definecolor{currentfill}{rgb}{0.490838,0.119982,0.351115}%
\pgfsetfillcolor{currentfill}%
\pgfsetlinewidth{0.000000pt}%
\definecolor{currentstroke}{rgb}{0.000000,0.000000,0.000000}%
\pgfsetstrokecolor{currentstroke}%
\pgfsetdash{}{0pt}%
\pgfpathmoveto{\pgfqpoint{0.742633in}{0.792320in}}%
\pgfpathlineto{\pgfqpoint{0.743402in}{0.793568in}}%
\pgfpathlineto{\pgfqpoint{0.743806in}{0.806495in}}%
\pgfpathlineto{\pgfqpoint{0.742633in}{0.808625in}}%
\pgfpathlineto{\pgfqpoint{0.732301in}{0.806495in}}%
\pgfpathlineto{\pgfqpoint{0.735832in}{0.793568in}}%
\pgfpathclose%
\pgfusepath{fill}%
\end{pgfscope}%
\begin{pgfscope}%
\pgfpathrectangle{\pgfqpoint{0.211875in}{0.211875in}}{\pgfqpoint{1.313625in}{1.279725in}}%
\pgfusepath{clip}%
\pgfsetbuttcap%
\pgfsetroundjoin%
\definecolor{currentfill}{rgb}{0.490838,0.119982,0.351115}%
\pgfsetfillcolor{currentfill}%
\pgfsetlinewidth{0.000000pt}%
\definecolor{currentstroke}{rgb}{0.000000,0.000000,0.000000}%
\pgfsetstrokecolor{currentstroke}%
\pgfsetdash{}{0pt}%
\pgfpathmoveto{\pgfqpoint{1.445886in}{0.793477in}}%
\pgfpathlineto{\pgfqpoint{1.445978in}{0.793568in}}%
\pgfpathlineto{\pgfqpoint{1.446666in}{0.806495in}}%
\pgfpathlineto{\pgfqpoint{1.445886in}{0.807371in}}%
\pgfpathlineto{\pgfqpoint{1.444913in}{0.806495in}}%
\pgfpathlineto{\pgfqpoint{1.445772in}{0.793568in}}%
\pgfpathclose%
\pgfusepath{fill}%
\end{pgfscope}%
\begin{pgfscope}%
\pgfpathrectangle{\pgfqpoint{0.211875in}{0.211875in}}{\pgfqpoint{1.313625in}{1.279725in}}%
\pgfusepath{clip}%
\pgfsetbuttcap%
\pgfsetroundjoin%
\definecolor{currentfill}{rgb}{0.490838,0.119982,0.351115}%
\pgfsetfillcolor{currentfill}%
\pgfsetlinewidth{0.000000pt}%
\definecolor{currentstroke}{rgb}{0.000000,0.000000,0.000000}%
\pgfsetstrokecolor{currentstroke}%
\pgfsetdash{}{0pt}%
\pgfpathmoveto{\pgfqpoint{0.219826in}{0.858201in}}%
\pgfpathlineto{\pgfqpoint{0.225144in}{0.866011in}}%
\pgfpathlineto{\pgfqpoint{0.226572in}{0.871127in}}%
\pgfpathlineto{\pgfqpoint{0.227327in}{0.884054in}}%
\pgfpathlineto{\pgfqpoint{0.225631in}{0.896980in}}%
\pgfpathlineto{\pgfqpoint{0.225144in}{0.898312in}}%
\pgfpathlineto{\pgfqpoint{0.213437in}{0.909907in}}%
\pgfpathlineto{\pgfqpoint{0.211875in}{0.910596in}}%
\pgfpathlineto{\pgfqpoint{0.211875in}{0.909907in}}%
\pgfpathlineto{\pgfqpoint{0.211875in}{0.899825in}}%
\pgfpathlineto{\pgfqpoint{0.214766in}{0.896980in}}%
\pgfpathlineto{\pgfqpoint{0.219217in}{0.884054in}}%
\pgfpathlineto{\pgfqpoint{0.217078in}{0.871127in}}%
\pgfpathlineto{\pgfqpoint{0.211875in}{0.864576in}}%
\pgfpathlineto{\pgfqpoint{0.211875in}{0.858201in}}%
\pgfpathlineto{\pgfqpoint{0.211875in}{0.853637in}}%
\pgfpathclose%
\pgfusepath{fill}%
\end{pgfscope}%
\begin{pgfscope}%
\pgfpathrectangle{\pgfqpoint{0.211875in}{0.211875in}}{\pgfqpoint{1.313625in}{1.279725in}}%
\pgfusepath{clip}%
\pgfsetbuttcap%
\pgfsetroundjoin%
\definecolor{currentfill}{rgb}{0.490838,0.119982,0.351115}%
\pgfsetfillcolor{currentfill}%
\pgfsetlinewidth{0.000000pt}%
\definecolor{currentstroke}{rgb}{0.000000,0.000000,0.000000}%
\pgfsetstrokecolor{currentstroke}%
\pgfsetdash{}{0pt}%
\pgfpathmoveto{\pgfqpoint{0.318027in}{0.858106in}}%
\pgfpathlineto{\pgfqpoint{0.321349in}{0.858201in}}%
\pgfpathlineto{\pgfqpoint{0.331295in}{0.858618in}}%
\pgfpathlineto{\pgfqpoint{0.340106in}{0.871127in}}%
\pgfpathlineto{\pgfqpoint{0.341632in}{0.884054in}}%
\pgfpathlineto{\pgfqpoint{0.338361in}{0.896980in}}%
\pgfpathlineto{\pgfqpoint{0.331295in}{0.904805in}}%
\pgfpathlineto{\pgfqpoint{0.318027in}{0.905269in}}%
\pgfpathlineto{\pgfqpoint{0.310089in}{0.896980in}}%
\pgfpathlineto{\pgfqpoint{0.306830in}{0.884054in}}%
\pgfpathlineto{\pgfqpoint{0.308325in}{0.871127in}}%
\pgfpathlineto{\pgfqpoint{0.317865in}{0.858201in}}%
\pgfpathclose%
\pgfpathmoveto{\pgfqpoint{0.315941in}{0.884054in}}%
\pgfpathlineto{\pgfqpoint{0.318027in}{0.890000in}}%
\pgfpathlineto{\pgfqpoint{0.331295in}{0.889081in}}%
\pgfpathlineto{\pgfqpoint{0.332979in}{0.884054in}}%
\pgfpathlineto{\pgfqpoint{0.331295in}{0.874547in}}%
\pgfpathlineto{\pgfqpoint{0.318027in}{0.872803in}}%
\pgfpathclose%
\pgfusepath{fill}%
\end{pgfscope}%
\begin{pgfscope}%
\pgfpathrectangle{\pgfqpoint{0.211875in}{0.211875in}}{\pgfqpoint{1.313625in}{1.279725in}}%
\pgfusepath{clip}%
\pgfsetbuttcap%
\pgfsetroundjoin%
\definecolor{currentfill}{rgb}{0.490838,0.119982,0.351115}%
\pgfsetfillcolor{currentfill}%
\pgfsetlinewidth{0.000000pt}%
\definecolor{currentstroke}{rgb}{0.000000,0.000000,0.000000}%
\pgfsetstrokecolor{currentstroke}%
\pgfsetdash{}{0pt}%
\pgfpathmoveto{\pgfqpoint{0.437447in}{0.863148in}}%
\pgfpathlineto{\pgfqpoint{0.450716in}{0.865436in}}%
\pgfpathlineto{\pgfqpoint{0.454294in}{0.871127in}}%
\pgfpathlineto{\pgfqpoint{0.456021in}{0.884054in}}%
\pgfpathlineto{\pgfqpoint{0.452408in}{0.896980in}}%
\pgfpathlineto{\pgfqpoint{0.450716in}{0.899076in}}%
\pgfpathlineto{\pgfqpoint{0.437447in}{0.901020in}}%
\pgfpathlineto{\pgfqpoint{0.433063in}{0.896980in}}%
\pgfpathlineto{\pgfqpoint{0.428629in}{0.884054in}}%
\pgfpathlineto{\pgfqpoint{0.430728in}{0.871127in}}%
\pgfpathclose%
\pgfusepath{fill}%
\end{pgfscope}%
\begin{pgfscope}%
\pgfpathrectangle{\pgfqpoint{0.211875in}{0.211875in}}{\pgfqpoint{1.313625in}{1.279725in}}%
\pgfusepath{clip}%
\pgfsetbuttcap%
\pgfsetroundjoin%
\definecolor{currentfill}{rgb}{0.490838,0.119982,0.351115}%
\pgfsetfillcolor{currentfill}%
\pgfsetlinewidth{0.000000pt}%
\definecolor{currentstroke}{rgb}{0.000000,0.000000,0.000000}%
\pgfsetstrokecolor{currentstroke}%
\pgfsetdash{}{0pt}%
\pgfpathmoveto{\pgfqpoint{0.556867in}{0.867320in}}%
\pgfpathlineto{\pgfqpoint{0.567836in}{0.871127in}}%
\pgfpathlineto{\pgfqpoint{0.570136in}{0.874395in}}%
\pgfpathlineto{\pgfqpoint{0.571499in}{0.884054in}}%
\pgfpathlineto{\pgfqpoint{0.570136in}{0.889126in}}%
\pgfpathlineto{\pgfqpoint{0.558881in}{0.896980in}}%
\pgfpathlineto{\pgfqpoint{0.556867in}{0.897529in}}%
\pgfpathlineto{\pgfqpoint{0.556185in}{0.896980in}}%
\pgfpathlineto{\pgfqpoint{0.550406in}{0.884054in}}%
\pgfpathlineto{\pgfqpoint{0.553190in}{0.871127in}}%
\pgfpathclose%
\pgfusepath{fill}%
\end{pgfscope}%
\begin{pgfscope}%
\pgfpathrectangle{\pgfqpoint{0.211875in}{0.211875in}}{\pgfqpoint{1.313625in}{1.279725in}}%
\pgfusepath{clip}%
\pgfsetbuttcap%
\pgfsetroundjoin%
\definecolor{currentfill}{rgb}{0.490838,0.119982,0.351115}%
\pgfsetfillcolor{currentfill}%
\pgfsetlinewidth{0.000000pt}%
\definecolor{currentstroke}{rgb}{0.000000,0.000000,0.000000}%
\pgfsetstrokecolor{currentstroke}%
\pgfsetdash{}{0pt}%
\pgfpathmoveto{\pgfqpoint{0.676288in}{0.870716in}}%
\pgfpathlineto{\pgfqpoint{0.677119in}{0.871127in}}%
\pgfpathlineto{\pgfqpoint{0.683655in}{0.884054in}}%
\pgfpathlineto{\pgfqpoint{0.676288in}{0.891713in}}%
\pgfpathlineto{\pgfqpoint{0.672222in}{0.884054in}}%
\pgfpathlineto{\pgfqpoint{0.675825in}{0.871127in}}%
\pgfpathclose%
\pgfusepath{fill}%
\end{pgfscope}%
\begin{pgfscope}%
\pgfpathrectangle{\pgfqpoint{0.211875in}{0.211875in}}{\pgfqpoint{1.313625in}{1.279725in}}%
\pgfusepath{clip}%
\pgfsetbuttcap%
\pgfsetroundjoin%
\definecolor{currentfill}{rgb}{0.490838,0.119982,0.351115}%
\pgfsetfillcolor{currentfill}%
\pgfsetlinewidth{0.000000pt}%
\definecolor{currentstroke}{rgb}{0.000000,0.000000,0.000000}%
\pgfsetstrokecolor{currentstroke}%
\pgfsetdash{}{0pt}%
\pgfpathmoveto{\pgfqpoint{0.795708in}{0.879489in}}%
\pgfpathlineto{\pgfqpoint{0.797510in}{0.884054in}}%
\pgfpathlineto{\pgfqpoint{0.795708in}{0.886480in}}%
\pgfpathlineto{\pgfqpoint{0.794170in}{0.884054in}}%
\pgfpathclose%
\pgfusepath{fill}%
\end{pgfscope}%
\begin{pgfscope}%
\pgfpathrectangle{\pgfqpoint{0.211875in}{0.211875in}}{\pgfqpoint{1.313625in}{1.279725in}}%
\pgfusepath{clip}%
\pgfsetbuttcap%
\pgfsetroundjoin%
\definecolor{currentfill}{rgb}{0.490838,0.119982,0.351115}%
\pgfsetfillcolor{currentfill}%
\pgfsetlinewidth{0.000000pt}%
\definecolor{currentstroke}{rgb}{0.000000,0.000000,0.000000}%
\pgfsetstrokecolor{currentstroke}%
\pgfsetdash{}{0pt}%
\pgfpathmoveto{\pgfqpoint{0.251682in}{0.943315in}}%
\pgfpathlineto{\pgfqpoint{0.264951in}{0.937208in}}%
\pgfpathlineto{\pgfqpoint{0.278220in}{0.942884in}}%
\pgfpathlineto{\pgfqpoint{0.281531in}{0.948686in}}%
\pgfpathlineto{\pgfqpoint{0.283691in}{0.961613in}}%
\pgfpathlineto{\pgfqpoint{0.282471in}{0.974539in}}%
\pgfpathlineto{\pgfqpoint{0.278220in}{0.983758in}}%
\pgfpathlineto{\pgfqpoint{0.271572in}{0.987466in}}%
\pgfpathlineto{\pgfqpoint{0.264951in}{0.989459in}}%
\pgfpathlineto{\pgfqpoint{0.258510in}{0.987466in}}%
\pgfpathlineto{\pgfqpoint{0.251682in}{0.983222in}}%
\pgfpathlineto{\pgfqpoint{0.247962in}{0.974539in}}%
\pgfpathlineto{\pgfqpoint{0.246832in}{0.961613in}}%
\pgfpathlineto{\pgfqpoint{0.248826in}{0.948686in}}%
\pgfpathclose%
\pgfpathmoveto{\pgfqpoint{0.263766in}{0.948686in}}%
\pgfpathlineto{\pgfqpoint{0.255895in}{0.961613in}}%
\pgfpathlineto{\pgfqpoint{0.260158in}{0.974539in}}%
\pgfpathlineto{\pgfqpoint{0.264951in}{0.977972in}}%
\pgfpathlineto{\pgfqpoint{0.270282in}{0.974539in}}%
\pgfpathlineto{\pgfqpoint{0.275160in}{0.961613in}}%
\pgfpathlineto{\pgfqpoint{0.266252in}{0.948686in}}%
\pgfpathlineto{\pgfqpoint{0.264951in}{0.948010in}}%
\pgfpathclose%
\pgfusepath{fill}%
\end{pgfscope}%
\begin{pgfscope}%
\pgfpathrectangle{\pgfqpoint{0.211875in}{0.211875in}}{\pgfqpoint{1.313625in}{1.279725in}}%
\pgfusepath{clip}%
\pgfsetbuttcap%
\pgfsetroundjoin%
\definecolor{currentfill}{rgb}{0.490838,0.119982,0.351115}%
\pgfsetfillcolor{currentfill}%
\pgfsetlinewidth{0.000000pt}%
\definecolor{currentstroke}{rgb}{0.000000,0.000000,0.000000}%
\pgfsetstrokecolor{currentstroke}%
\pgfsetdash{}{0pt}%
\pgfpathmoveto{\pgfqpoint{0.371102in}{0.948205in}}%
\pgfpathlineto{\pgfqpoint{0.384371in}{0.942233in}}%
\pgfpathlineto{\pgfqpoint{0.394568in}{0.948686in}}%
\pgfpathlineto{\pgfqpoint{0.397640in}{0.954870in}}%
\pgfpathlineto{\pgfqpoint{0.398886in}{0.961613in}}%
\pgfpathlineto{\pgfqpoint{0.397640in}{0.972947in}}%
\pgfpathlineto{\pgfqpoint{0.397204in}{0.974539in}}%
\pgfpathlineto{\pgfqpoint{0.384371in}{0.984616in}}%
\pgfpathlineto{\pgfqpoint{0.371102in}{0.977665in}}%
\pgfpathlineto{\pgfqpoint{0.369607in}{0.974539in}}%
\pgfpathlineto{\pgfqpoint{0.368110in}{0.961613in}}%
\pgfpathlineto{\pgfqpoint{0.370816in}{0.948686in}}%
\pgfpathclose%
\pgfpathmoveto{\pgfqpoint{0.382198in}{0.961613in}}%
\pgfpathlineto{\pgfqpoint{0.384371in}{0.966404in}}%
\pgfpathlineto{\pgfqpoint{0.385981in}{0.961613in}}%
\pgfpathlineto{\pgfqpoint{0.384371in}{0.958818in}}%
\pgfpathclose%
\pgfusepath{fill}%
\end{pgfscope}%
\begin{pgfscope}%
\pgfpathrectangle{\pgfqpoint{0.211875in}{0.211875in}}{\pgfqpoint{1.313625in}{1.279725in}}%
\pgfusepath{clip}%
\pgfsetbuttcap%
\pgfsetroundjoin%
\definecolor{currentfill}{rgb}{0.490838,0.119982,0.351115}%
\pgfsetfillcolor{currentfill}%
\pgfsetlinewidth{0.000000pt}%
\definecolor{currentstroke}{rgb}{0.000000,0.000000,0.000000}%
\pgfsetstrokecolor{currentstroke}%
\pgfsetdash{}{0pt}%
\pgfpathmoveto{\pgfqpoint{0.503792in}{0.946706in}}%
\pgfpathlineto{\pgfqpoint{0.506443in}{0.948686in}}%
\pgfpathlineto{\pgfqpoint{0.512333in}{0.961613in}}%
\pgfpathlineto{\pgfqpoint{0.509106in}{0.974539in}}%
\pgfpathlineto{\pgfqpoint{0.503792in}{0.979469in}}%
\pgfpathlineto{\pgfqpoint{0.492391in}{0.974539in}}%
\pgfpathlineto{\pgfqpoint{0.490523in}{0.971193in}}%
\pgfpathlineto{\pgfqpoint{0.489177in}{0.961613in}}%
\pgfpathlineto{\pgfqpoint{0.490523in}{0.955960in}}%
\pgfpathlineto{\pgfqpoint{0.498041in}{0.948686in}}%
\pgfpathclose%
\pgfusepath{fill}%
\end{pgfscope}%
\begin{pgfscope}%
\pgfpathrectangle{\pgfqpoint{0.211875in}{0.211875in}}{\pgfqpoint{1.313625in}{1.279725in}}%
\pgfusepath{clip}%
\pgfsetbuttcap%
\pgfsetroundjoin%
\definecolor{currentfill}{rgb}{0.490838,0.119982,0.351115}%
\pgfsetfillcolor{currentfill}%
\pgfsetlinewidth{0.000000pt}%
\definecolor{currentstroke}{rgb}{0.000000,0.000000,0.000000}%
\pgfsetstrokecolor{currentstroke}%
\pgfsetdash{}{0pt}%
\pgfpathmoveto{\pgfqpoint{0.623212in}{0.953399in}}%
\pgfpathlineto{\pgfqpoint{0.626664in}{0.961613in}}%
\pgfpathlineto{\pgfqpoint{0.623517in}{0.974539in}}%
\pgfpathlineto{\pgfqpoint{0.623212in}{0.974867in}}%
\pgfpathlineto{\pgfqpoint{0.622071in}{0.974539in}}%
\pgfpathlineto{\pgfqpoint{0.610469in}{0.961613in}}%
\pgfpathclose%
\pgfusepath{fill}%
\end{pgfscope}%
\begin{pgfscope}%
\pgfpathrectangle{\pgfqpoint{0.211875in}{0.211875in}}{\pgfqpoint{1.313625in}{1.279725in}}%
\pgfusepath{clip}%
\pgfsetbuttcap%
\pgfsetroundjoin%
\definecolor{currentfill}{rgb}{0.490838,0.119982,0.351115}%
\pgfsetfillcolor{currentfill}%
\pgfsetlinewidth{0.000000pt}%
\definecolor{currentstroke}{rgb}{0.000000,0.000000,0.000000}%
\pgfsetstrokecolor{currentstroke}%
\pgfsetdash{}{0pt}%
\pgfpathmoveto{\pgfqpoint{0.222663in}{1.026245in}}%
\pgfpathlineto{\pgfqpoint{0.225144in}{1.032537in}}%
\pgfpathlineto{\pgfqpoint{0.226390in}{1.039172in}}%
\pgfpathlineto{\pgfqpoint{0.226187in}{1.052098in}}%
\pgfpathlineto{\pgfqpoint{0.225144in}{1.056796in}}%
\pgfpathlineto{\pgfqpoint{0.221135in}{1.065025in}}%
\pgfpathlineto{\pgfqpoint{0.211875in}{1.071643in}}%
\pgfpathlineto{\pgfqpoint{0.211875in}{1.065025in}}%
\pgfpathlineto{\pgfqpoint{0.211875in}{1.060304in}}%
\pgfpathlineto{\pgfqpoint{0.216984in}{1.052098in}}%
\pgfpathlineto{\pgfqpoint{0.217483in}{1.039172in}}%
\pgfpathlineto{\pgfqpoint{0.211875in}{1.028627in}}%
\pgfpathlineto{\pgfqpoint{0.211875in}{1.026245in}}%
\pgfpathlineto{\pgfqpoint{0.211875in}{1.017611in}}%
\pgfpathclose%
\pgfusepath{fill}%
\end{pgfscope}%
\begin{pgfscope}%
\pgfpathrectangle{\pgfqpoint{0.211875in}{0.211875in}}{\pgfqpoint{1.313625in}{1.279725in}}%
\pgfusepath{clip}%
\pgfsetbuttcap%
\pgfsetroundjoin%
\definecolor{currentfill}{rgb}{0.490838,0.119982,0.351115}%
\pgfsetfillcolor{currentfill}%
\pgfsetlinewidth{0.000000pt}%
\definecolor{currentstroke}{rgb}{0.000000,0.000000,0.000000}%
\pgfsetstrokecolor{currentstroke}%
\pgfsetdash{}{0pt}%
\pgfpathmoveto{\pgfqpoint{0.318027in}{1.022653in}}%
\pgfpathlineto{\pgfqpoint{0.331295in}{1.023067in}}%
\pgfpathlineto{\pgfqpoint{0.334826in}{1.026245in}}%
\pgfpathlineto{\pgfqpoint{0.340146in}{1.039172in}}%
\pgfpathlineto{\pgfqpoint{0.339770in}{1.052098in}}%
\pgfpathlineto{\pgfqpoint{0.333084in}{1.065025in}}%
\pgfpathlineto{\pgfqpoint{0.331295in}{1.066461in}}%
\pgfpathlineto{\pgfqpoint{0.318027in}{1.066853in}}%
\pgfpathlineto{\pgfqpoint{0.315575in}{1.065025in}}%
\pgfpathlineto{\pgfqpoint{0.308747in}{1.052098in}}%
\pgfpathlineto{\pgfqpoint{0.308371in}{1.039172in}}%
\pgfpathlineto{\pgfqpoint{0.313755in}{1.026245in}}%
\pgfpathclose%
\pgfpathmoveto{\pgfqpoint{0.317682in}{1.039172in}}%
\pgfpathlineto{\pgfqpoint{0.318027in}{1.047340in}}%
\pgfpathlineto{\pgfqpoint{0.331295in}{1.039678in}}%
\pgfpathlineto{\pgfqpoint{0.331316in}{1.039172in}}%
\pgfpathlineto{\pgfqpoint{0.331295in}{1.039129in}}%
\pgfpathlineto{\pgfqpoint{0.318027in}{1.038475in}}%
\pgfpathclose%
\pgfusepath{fill}%
\end{pgfscope}%
\begin{pgfscope}%
\pgfpathrectangle{\pgfqpoint{0.211875in}{0.211875in}}{\pgfqpoint{1.313625in}{1.279725in}}%
\pgfusepath{clip}%
\pgfsetbuttcap%
\pgfsetroundjoin%
\definecolor{currentfill}{rgb}{0.490838,0.119982,0.351115}%
\pgfsetfillcolor{currentfill}%
\pgfsetlinewidth{0.000000pt}%
\definecolor{currentstroke}{rgb}{0.000000,0.000000,0.000000}%
\pgfsetstrokecolor{currentstroke}%
\pgfsetdash{}{0pt}%
\pgfpathmoveto{\pgfqpoint{0.437447in}{1.026650in}}%
\pgfpathlineto{\pgfqpoint{0.450716in}{1.029986in}}%
\pgfpathlineto{\pgfqpoint{0.454581in}{1.039172in}}%
\pgfpathlineto{\pgfqpoint{0.454175in}{1.052098in}}%
\pgfpathlineto{\pgfqpoint{0.450716in}{1.059112in}}%
\pgfpathlineto{\pgfqpoint{0.437447in}{1.062020in}}%
\pgfpathlineto{\pgfqpoint{0.430925in}{1.052098in}}%
\pgfpathlineto{\pgfqpoint{0.430426in}{1.039172in}}%
\pgfpathclose%
\pgfusepath{fill}%
\end{pgfscope}%
\begin{pgfscope}%
\pgfpathrectangle{\pgfqpoint{0.211875in}{0.211875in}}{\pgfqpoint{1.313625in}{1.279725in}}%
\pgfusepath{clip}%
\pgfsetbuttcap%
\pgfsetroundjoin%
\definecolor{currentfill}{rgb}{0.490838,0.119982,0.351115}%
\pgfsetfillcolor{currentfill}%
\pgfsetlinewidth{0.000000pt}%
\definecolor{currentstroke}{rgb}{0.000000,0.000000,0.000000}%
\pgfsetstrokecolor{currentstroke}%
\pgfsetdash{}{0pt}%
\pgfpathmoveto{\pgfqpoint{0.556867in}{1.032409in}}%
\pgfpathlineto{\pgfqpoint{0.569970in}{1.039172in}}%
\pgfpathlineto{\pgfqpoint{0.568019in}{1.052098in}}%
\pgfpathlineto{\pgfqpoint{0.556867in}{1.057021in}}%
\pgfpathlineto{\pgfqpoint{0.553157in}{1.052098in}}%
\pgfpathlineto{\pgfqpoint{0.552518in}{1.039172in}}%
\pgfpathclose%
\pgfusepath{fill}%
\end{pgfscope}%
\begin{pgfscope}%
\pgfpathrectangle{\pgfqpoint{0.211875in}{0.211875in}}{\pgfqpoint{1.313625in}{1.279725in}}%
\pgfusepath{clip}%
\pgfsetbuttcap%
\pgfsetroundjoin%
\definecolor{currentfill}{rgb}{0.490838,0.119982,0.351115}%
\pgfsetfillcolor{currentfill}%
\pgfsetlinewidth{0.000000pt}%
\definecolor{currentstroke}{rgb}{0.000000,0.000000,0.000000}%
\pgfsetstrokecolor{currentstroke}%
\pgfsetdash{}{0pt}%
\pgfpathmoveto{\pgfqpoint{0.676288in}{1.037105in}}%
\pgfpathlineto{\pgfqpoint{0.679090in}{1.039172in}}%
\pgfpathlineto{\pgfqpoint{0.677625in}{1.052098in}}%
\pgfpathlineto{\pgfqpoint{0.676288in}{1.052941in}}%
\pgfpathlineto{\pgfqpoint{0.675547in}{1.052098in}}%
\pgfpathlineto{\pgfqpoint{0.674738in}{1.039172in}}%
\pgfpathclose%
\pgfusepath{fill}%
\end{pgfscope}%
\begin{pgfscope}%
\pgfpathrectangle{\pgfqpoint{0.211875in}{0.211875in}}{\pgfqpoint{1.313625in}{1.279725in}}%
\pgfusepath{clip}%
\pgfsetbuttcap%
\pgfsetroundjoin%
\definecolor{currentfill}{rgb}{0.490838,0.119982,0.351115}%
\pgfsetfillcolor{currentfill}%
\pgfsetlinewidth{0.000000pt}%
\definecolor{currentstroke}{rgb}{0.000000,0.000000,0.000000}%
\pgfsetstrokecolor{currentstroke}%
\pgfsetdash{}{0pt}%
\pgfpathmoveto{\pgfqpoint{0.264951in}{1.100568in}}%
\pgfpathlineto{\pgfqpoint{0.274539in}{1.103805in}}%
\pgfpathlineto{\pgfqpoint{0.278220in}{1.106266in}}%
\pgfpathlineto{\pgfqpoint{0.282444in}{1.116731in}}%
\pgfpathlineto{\pgfqpoint{0.283151in}{1.129658in}}%
\pgfpathlineto{\pgfqpoint{0.280201in}{1.142584in}}%
\pgfpathlineto{\pgfqpoint{0.278220in}{1.145671in}}%
\pgfpathlineto{\pgfqpoint{0.264951in}{1.151191in}}%
\pgfpathlineto{\pgfqpoint{0.251682in}{1.145180in}}%
\pgfpathlineto{\pgfqpoint{0.250123in}{1.142584in}}%
\pgfpathlineto{\pgfqpoint{0.247365in}{1.129658in}}%
\pgfpathlineto{\pgfqpoint{0.248020in}{1.116731in}}%
\pgfpathlineto{\pgfqpoint{0.251682in}{1.106937in}}%
\pgfpathlineto{\pgfqpoint{0.255800in}{1.103805in}}%
\pgfpathclose%
\pgfpathmoveto{\pgfqpoint{0.259782in}{1.116731in}}%
\pgfpathlineto{\pgfqpoint{0.257285in}{1.129658in}}%
\pgfpathlineto{\pgfqpoint{0.264951in}{1.139574in}}%
\pgfpathlineto{\pgfqpoint{0.273584in}{1.129658in}}%
\pgfpathlineto{\pgfqpoint{0.270726in}{1.116731in}}%
\pgfpathlineto{\pgfqpoint{0.264951in}{1.112500in}}%
\pgfpathclose%
\pgfusepath{fill}%
\end{pgfscope}%
\begin{pgfscope}%
\pgfpathrectangle{\pgfqpoint{0.211875in}{0.211875in}}{\pgfqpoint{1.313625in}{1.279725in}}%
\pgfusepath{clip}%
\pgfsetbuttcap%
\pgfsetroundjoin%
\definecolor{currentfill}{rgb}{0.490838,0.119982,0.351115}%
\pgfsetfillcolor{currentfill}%
\pgfsetlinewidth{0.000000pt}%
\definecolor{currentstroke}{rgb}{0.000000,0.000000,0.000000}%
\pgfsetstrokecolor{currentstroke}%
\pgfsetdash{}{0pt}%
\pgfpathmoveto{\pgfqpoint{0.371102in}{1.113049in}}%
\pgfpathlineto{\pgfqpoint{0.384371in}{1.105136in}}%
\pgfpathlineto{\pgfqpoint{0.397340in}{1.116731in}}%
\pgfpathlineto{\pgfqpoint{0.397640in}{1.118667in}}%
\pgfpathlineto{\pgfqpoint{0.398347in}{1.129658in}}%
\pgfpathlineto{\pgfqpoint{0.397640in}{1.132728in}}%
\pgfpathlineto{\pgfqpoint{0.391303in}{1.142584in}}%
\pgfpathlineto{\pgfqpoint{0.384371in}{1.146499in}}%
\pgfpathlineto{\pgfqpoint{0.374725in}{1.142584in}}%
\pgfpathlineto{\pgfqpoint{0.371102in}{1.138756in}}%
\pgfpathlineto{\pgfqpoint{0.368690in}{1.129658in}}%
\pgfpathlineto{\pgfqpoint{0.369563in}{1.116731in}}%
\pgfpathclose%
\pgfpathmoveto{\pgfqpoint{0.384001in}{1.129658in}}%
\pgfpathlineto{\pgfqpoint{0.384371in}{1.130042in}}%
\pgfpathlineto{\pgfqpoint{0.384645in}{1.129658in}}%
\pgfpathlineto{\pgfqpoint{0.384371in}{1.128282in}}%
\pgfpathclose%
\pgfusepath{fill}%
\end{pgfscope}%
\begin{pgfscope}%
\pgfpathrectangle{\pgfqpoint{0.211875in}{0.211875in}}{\pgfqpoint{1.313625in}{1.279725in}}%
\pgfusepath{clip}%
\pgfsetbuttcap%
\pgfsetroundjoin%
\definecolor{currentfill}{rgb}{0.490838,0.119982,0.351115}%
\pgfsetfillcolor{currentfill}%
\pgfsetlinewidth{0.000000pt}%
\definecolor{currentstroke}{rgb}{0.000000,0.000000,0.000000}%
\pgfsetstrokecolor{currentstroke}%
\pgfsetdash{}{0pt}%
\pgfpathmoveto{\pgfqpoint{0.503792in}{1.110844in}}%
\pgfpathlineto{\pgfqpoint{0.509363in}{1.116731in}}%
\pgfpathlineto{\pgfqpoint{0.511251in}{1.129658in}}%
\pgfpathlineto{\pgfqpoint{0.503792in}{1.142041in}}%
\pgfpathlineto{\pgfqpoint{0.490523in}{1.132064in}}%
\pgfpathlineto{\pgfqpoint{0.489811in}{1.129658in}}%
\pgfpathlineto{\pgfqpoint{0.490523in}{1.121042in}}%
\pgfpathlineto{\pgfqpoint{0.491886in}{1.116731in}}%
\pgfpathclose%
\pgfusepath{fill}%
\end{pgfscope}%
\begin{pgfscope}%
\pgfpathrectangle{\pgfqpoint{0.211875in}{0.211875in}}{\pgfqpoint{1.313625in}{1.279725in}}%
\pgfusepath{clip}%
\pgfsetbuttcap%
\pgfsetroundjoin%
\definecolor{currentfill}{rgb}{0.490838,0.119982,0.351115}%
\pgfsetfillcolor{currentfill}%
\pgfsetlinewidth{0.000000pt}%
\definecolor{currentstroke}{rgb}{0.000000,0.000000,0.000000}%
\pgfsetstrokecolor{currentstroke}%
\pgfsetdash{}{0pt}%
\pgfpathmoveto{\pgfqpoint{0.623212in}{1.115955in}}%
\pgfpathlineto{\pgfqpoint{0.623847in}{1.116731in}}%
\pgfpathlineto{\pgfqpoint{0.625693in}{1.129658in}}%
\pgfpathlineto{\pgfqpoint{0.623212in}{1.134425in}}%
\pgfpathlineto{\pgfqpoint{0.614047in}{1.129658in}}%
\pgfpathlineto{\pgfqpoint{0.620850in}{1.116731in}}%
\pgfpathclose%
\pgfusepath{fill}%
\end{pgfscope}%
\begin{pgfscope}%
\pgfpathrectangle{\pgfqpoint{0.211875in}{0.211875in}}{\pgfqpoint{1.313625in}{1.279725in}}%
\pgfusepath{clip}%
\pgfsetbuttcap%
\pgfsetroundjoin%
\definecolor{currentfill}{rgb}{0.490838,0.119982,0.351115}%
\pgfsetfillcolor{currentfill}%
\pgfsetlinewidth{0.000000pt}%
\definecolor{currentstroke}{rgb}{0.000000,0.000000,0.000000}%
\pgfsetstrokecolor{currentstroke}%
\pgfsetdash{}{0pt}%
\pgfpathmoveto{\pgfqpoint{0.214676in}{1.181364in}}%
\pgfpathlineto{\pgfqpoint{0.225144in}{1.193918in}}%
\pgfpathlineto{\pgfqpoint{0.225264in}{1.194290in}}%
\pgfpathlineto{\pgfqpoint{0.226599in}{1.207217in}}%
\pgfpathlineto{\pgfqpoint{0.225320in}{1.220143in}}%
\pgfpathlineto{\pgfqpoint{0.225144in}{1.220695in}}%
\pgfpathlineto{\pgfqpoint{0.214939in}{1.233070in}}%
\pgfpathlineto{\pgfqpoint{0.211875in}{1.234620in}}%
\pgfpathlineto{\pgfqpoint{0.211875in}{1.233070in}}%
\pgfpathlineto{\pgfqpoint{0.211875in}{1.223255in}}%
\pgfpathlineto{\pgfqpoint{0.214672in}{1.220143in}}%
\pgfpathlineto{\pgfqpoint{0.218042in}{1.207217in}}%
\pgfpathlineto{\pgfqpoint{0.214593in}{1.194290in}}%
\pgfpathlineto{\pgfqpoint{0.211875in}{1.191285in}}%
\pgfpathlineto{\pgfqpoint{0.211875in}{1.181364in}}%
\pgfpathlineto{\pgfqpoint{0.211875in}{1.179951in}}%
\pgfpathclose%
\pgfusepath{fill}%
\end{pgfscope}%
\begin{pgfscope}%
\pgfpathrectangle{\pgfqpoint{0.211875in}{0.211875in}}{\pgfqpoint{1.313625in}{1.279725in}}%
\pgfusepath{clip}%
\pgfsetbuttcap%
\pgfsetroundjoin%
\definecolor{currentfill}{rgb}{0.490838,0.119982,0.351115}%
\pgfsetfillcolor{currentfill}%
\pgfsetlinewidth{0.000000pt}%
\definecolor{currentstroke}{rgb}{0.000000,0.000000,0.000000}%
\pgfsetstrokecolor{currentstroke}%
\pgfsetdash{}{0pt}%
\pgfpathmoveto{\pgfqpoint{0.318027in}{1.185440in}}%
\pgfpathlineto{\pgfqpoint{0.331295in}{1.185934in}}%
\pgfpathlineto{\pgfqpoint{0.338003in}{1.194290in}}%
\pgfpathlineto{\pgfqpoint{0.340554in}{1.207217in}}%
\pgfpathlineto{\pgfqpoint{0.338081in}{1.220143in}}%
\pgfpathlineto{\pgfqpoint{0.331295in}{1.228651in}}%
\pgfpathlineto{\pgfqpoint{0.318027in}{1.229149in}}%
\pgfpathlineto{\pgfqpoint{0.310434in}{1.220143in}}%
\pgfpathlineto{\pgfqpoint{0.307968in}{1.207217in}}%
\pgfpathlineto{\pgfqpoint{0.310518in}{1.194290in}}%
\pgfpathclose%
\pgfpathmoveto{\pgfqpoint{0.317057in}{1.207217in}}%
\pgfpathlineto{\pgfqpoint{0.318027in}{1.210869in}}%
\pgfpathlineto{\pgfqpoint{0.331295in}{1.209682in}}%
\pgfpathlineto{\pgfqpoint{0.331921in}{1.207217in}}%
\pgfpathlineto{\pgfqpoint{0.331295in}{1.204797in}}%
\pgfpathlineto{\pgfqpoint{0.318027in}{1.203632in}}%
\pgfpathclose%
\pgfusepath{fill}%
\end{pgfscope}%
\begin{pgfscope}%
\pgfpathrectangle{\pgfqpoint{0.211875in}{0.211875in}}{\pgfqpoint{1.313625in}{1.279725in}}%
\pgfusepath{clip}%
\pgfsetbuttcap%
\pgfsetroundjoin%
\definecolor{currentfill}{rgb}{0.490838,0.119982,0.351115}%
\pgfsetfillcolor{currentfill}%
\pgfsetlinewidth{0.000000pt}%
\definecolor{currentstroke}{rgb}{0.000000,0.000000,0.000000}%
\pgfsetstrokecolor{currentstroke}%
\pgfsetdash{}{0pt}%
\pgfpathmoveto{\pgfqpoint{0.437447in}{1.189998in}}%
\pgfpathlineto{\pgfqpoint{0.450716in}{1.192178in}}%
\pgfpathlineto{\pgfqpoint{0.452231in}{1.194290in}}%
\pgfpathlineto{\pgfqpoint{0.455034in}{1.207217in}}%
\pgfpathlineto{\pgfqpoint{0.452298in}{1.220143in}}%
\pgfpathlineto{\pgfqpoint{0.450716in}{1.222364in}}%
\pgfpathlineto{\pgfqpoint{0.437447in}{1.224552in}}%
\pgfpathlineto{\pgfqpoint{0.433233in}{1.220143in}}%
\pgfpathlineto{\pgfqpoint{0.429875in}{1.207217in}}%
\pgfpathlineto{\pgfqpoint{0.433319in}{1.194290in}}%
\pgfpathclose%
\pgfusepath{fill}%
\end{pgfscope}%
\begin{pgfscope}%
\pgfpathrectangle{\pgfqpoint{0.211875in}{0.211875in}}{\pgfqpoint{1.313625in}{1.279725in}}%
\pgfusepath{clip}%
\pgfsetbuttcap%
\pgfsetroundjoin%
\definecolor{currentfill}{rgb}{0.490838,0.119982,0.351115}%
\pgfsetfillcolor{currentfill}%
\pgfsetlinewidth{0.000000pt}%
\definecolor{currentstroke}{rgb}{0.000000,0.000000,0.000000}%
\pgfsetstrokecolor{currentstroke}%
\pgfsetdash{}{0pt}%
\pgfpathmoveto{\pgfqpoint{0.556867in}{1.193745in}}%
\pgfpathlineto{\pgfqpoint{0.558651in}{1.194290in}}%
\pgfpathlineto{\pgfqpoint{0.570136in}{1.205054in}}%
\pgfpathlineto{\pgfqpoint{0.570584in}{1.207217in}}%
\pgfpathlineto{\pgfqpoint{0.570136in}{1.209424in}}%
\pgfpathlineto{\pgfqpoint{0.558922in}{1.220143in}}%
\pgfpathlineto{\pgfqpoint{0.556867in}{1.220775in}}%
\pgfpathlineto{\pgfqpoint{0.556175in}{1.220143in}}%
\pgfpathlineto{\pgfqpoint{0.551796in}{1.207217in}}%
\pgfpathlineto{\pgfqpoint{0.556267in}{1.194290in}}%
\pgfpathclose%
\pgfusepath{fill}%
\end{pgfscope}%
\begin{pgfscope}%
\pgfpathrectangle{\pgfqpoint{0.211875in}{0.211875in}}{\pgfqpoint{1.313625in}{1.279725in}}%
\pgfusepath{clip}%
\pgfsetbuttcap%
\pgfsetroundjoin%
\definecolor{currentfill}{rgb}{0.490838,0.119982,0.351115}%
\pgfsetfillcolor{currentfill}%
\pgfsetlinewidth{0.000000pt}%
\definecolor{currentstroke}{rgb}{0.000000,0.000000,0.000000}%
\pgfsetstrokecolor{currentstroke}%
\pgfsetdash{}{0pt}%
\pgfpathmoveto{\pgfqpoint{0.676288in}{1.201179in}}%
\pgfpathlineto{\pgfqpoint{0.680773in}{1.207217in}}%
\pgfpathlineto{\pgfqpoint{0.676288in}{1.213359in}}%
\pgfpathlineto{\pgfqpoint{0.673813in}{1.207217in}}%
\pgfpathclose%
\pgfusepath{fill}%
\end{pgfscope}%
\begin{pgfscope}%
\pgfpathrectangle{\pgfqpoint{0.211875in}{0.211875in}}{\pgfqpoint{1.313625in}{1.279725in}}%
\pgfusepath{clip}%
\pgfsetbuttcap%
\pgfsetroundjoin%
\definecolor{currentfill}{rgb}{0.490838,0.119982,0.351115}%
\pgfsetfillcolor{currentfill}%
\pgfsetlinewidth{0.000000pt}%
\definecolor{currentstroke}{rgb}{0.000000,0.000000,0.000000}%
\pgfsetstrokecolor{currentstroke}%
\pgfsetdash{}{0pt}%
\pgfpathmoveto{\pgfqpoint{0.251682in}{1.268516in}}%
\pgfpathlineto{\pgfqpoint{0.264951in}{1.262728in}}%
\pgfpathlineto{\pgfqpoint{0.278220in}{1.268082in}}%
\pgfpathlineto{\pgfqpoint{0.280654in}{1.271849in}}%
\pgfpathlineto{\pgfqpoint{0.283574in}{1.284776in}}%
\pgfpathlineto{\pgfqpoint{0.282948in}{1.297702in}}%
\pgfpathlineto{\pgfqpoint{0.278220in}{1.309537in}}%
\pgfpathlineto{\pgfqpoint{0.276602in}{1.310629in}}%
\pgfpathlineto{\pgfqpoint{0.264951in}{1.314577in}}%
\pgfpathlineto{\pgfqpoint{0.253825in}{1.310629in}}%
\pgfpathlineto{\pgfqpoint{0.251682in}{1.308983in}}%
\pgfpathlineto{\pgfqpoint{0.247511in}{1.297702in}}%
\pgfpathlineto{\pgfqpoint{0.246942in}{1.284776in}}%
\pgfpathlineto{\pgfqpoint{0.249665in}{1.271849in}}%
\pgfpathclose%
\pgfpathmoveto{\pgfqpoint{0.256274in}{1.284776in}}%
\pgfpathlineto{\pgfqpoint{0.258586in}{1.297702in}}%
\pgfpathlineto{\pgfqpoint{0.264951in}{1.302949in}}%
\pgfpathlineto{\pgfqpoint{0.272061in}{1.297702in}}%
\pgfpathlineto{\pgfqpoint{0.274723in}{1.284776in}}%
\pgfpathlineto{\pgfqpoint{0.264951in}{1.273674in}}%
\pgfpathclose%
\pgfusepath{fill}%
\end{pgfscope}%
\begin{pgfscope}%
\pgfpathrectangle{\pgfqpoint{0.211875in}{0.211875in}}{\pgfqpoint{1.313625in}{1.279725in}}%
\pgfusepath{clip}%
\pgfsetbuttcap%
\pgfsetroundjoin%
\definecolor{currentfill}{rgb}{0.490838,0.119982,0.351115}%
\pgfsetfillcolor{currentfill}%
\pgfsetlinewidth{0.000000pt}%
\definecolor{currentstroke}{rgb}{0.000000,0.000000,0.000000}%
\pgfsetstrokecolor{currentstroke}%
\pgfsetdash{}{0pt}%
\pgfpathmoveto{\pgfqpoint{0.384371in}{1.267390in}}%
\pgfpathlineto{\pgfqpoint{0.392304in}{1.271849in}}%
\pgfpathlineto{\pgfqpoint{0.397640in}{1.280021in}}%
\pgfpathlineto{\pgfqpoint{0.398754in}{1.284776in}}%
\pgfpathlineto{\pgfqpoint{0.397991in}{1.297702in}}%
\pgfpathlineto{\pgfqpoint{0.397640in}{1.298681in}}%
\pgfpathlineto{\pgfqpoint{0.384371in}{1.310372in}}%
\pgfpathlineto{\pgfqpoint{0.371102in}{1.302696in}}%
\pgfpathlineto{\pgfqpoint{0.369036in}{1.297702in}}%
\pgfpathlineto{\pgfqpoint{0.368248in}{1.284776in}}%
\pgfpathlineto{\pgfqpoint{0.371102in}{1.274176in}}%
\pgfpathlineto{\pgfqpoint{0.373335in}{1.271849in}}%
\pgfpathclose%
\pgfpathmoveto{\pgfqpoint{0.382740in}{1.284776in}}%
\pgfpathlineto{\pgfqpoint{0.384371in}{1.291087in}}%
\pgfpathlineto{\pgfqpoint{0.385578in}{1.284776in}}%
\pgfpathlineto{\pgfqpoint{0.384371in}{1.283100in}}%
\pgfpathclose%
\pgfusepath{fill}%
\end{pgfscope}%
\begin{pgfscope}%
\pgfpathrectangle{\pgfqpoint{0.211875in}{0.211875in}}{\pgfqpoint{1.313625in}{1.279725in}}%
\pgfusepath{clip}%
\pgfsetbuttcap%
\pgfsetroundjoin%
\definecolor{currentfill}{rgb}{0.490838,0.119982,0.351115}%
\pgfsetfillcolor{currentfill}%
\pgfsetlinewidth{0.000000pt}%
\definecolor{currentstroke}{rgb}{0.000000,0.000000,0.000000}%
\pgfsetstrokecolor{currentstroke}%
\pgfsetdash{}{0pt}%
\pgfpathmoveto{\pgfqpoint{0.503792in}{1.271543in}}%
\pgfpathlineto{\pgfqpoint{0.504255in}{1.271849in}}%
\pgfpathlineto{\pgfqpoint{0.512042in}{1.284776in}}%
\pgfpathlineto{\pgfqpoint{0.510295in}{1.297702in}}%
\pgfpathlineto{\pgfqpoint{0.503792in}{1.304624in}}%
\pgfpathlineto{\pgfqpoint{0.490523in}{1.298080in}}%
\pgfpathlineto{\pgfqpoint{0.490348in}{1.297702in}}%
\pgfpathlineto{\pgfqpoint{0.489345in}{1.284776in}}%
\pgfpathlineto{\pgfqpoint{0.490523in}{1.280851in}}%
\pgfpathlineto{\pgfqpoint{0.502781in}{1.271849in}}%
\pgfpathclose%
\pgfusepath{fill}%
\end{pgfscope}%
\begin{pgfscope}%
\pgfpathrectangle{\pgfqpoint{0.211875in}{0.211875in}}{\pgfqpoint{1.313625in}{1.279725in}}%
\pgfusepath{clip}%
\pgfsetbuttcap%
\pgfsetroundjoin%
\definecolor{currentfill}{rgb}{0.490838,0.119982,0.351115}%
\pgfsetfillcolor{currentfill}%
\pgfsetlinewidth{0.000000pt}%
\definecolor{currentstroke}{rgb}{0.000000,0.000000,0.000000}%
\pgfsetstrokecolor{currentstroke}%
\pgfsetdash{}{0pt}%
\pgfpathmoveto{\pgfqpoint{0.623212in}{1.278744in}}%
\pgfpathlineto{\pgfqpoint{0.626386in}{1.284776in}}%
\pgfpathlineto{\pgfqpoint{0.624660in}{1.297702in}}%
\pgfpathlineto{\pgfqpoint{0.623212in}{1.299485in}}%
\pgfpathlineto{\pgfqpoint{0.617825in}{1.297702in}}%
\pgfpathlineto{\pgfqpoint{0.611487in}{1.284776in}}%
\pgfpathclose%
\pgfusepath{fill}%
\end{pgfscope}%
\begin{pgfscope}%
\pgfpathrectangle{\pgfqpoint{0.211875in}{0.211875in}}{\pgfqpoint{1.313625in}{1.279725in}}%
\pgfusepath{clip}%
\pgfsetbuttcap%
\pgfsetroundjoin%
\definecolor{currentfill}{rgb}{0.490838,0.119982,0.351115}%
\pgfsetfillcolor{currentfill}%
\pgfsetlinewidth{0.000000pt}%
\definecolor{currentstroke}{rgb}{0.000000,0.000000,0.000000}%
\pgfsetstrokecolor{currentstroke}%
\pgfsetdash{}{0pt}%
\pgfpathmoveto{\pgfqpoint{0.222698in}{1.349408in}}%
\pgfpathlineto{\pgfqpoint{0.225144in}{1.354349in}}%
\pgfpathlineto{\pgfqpoint{0.226949in}{1.362335in}}%
\pgfpathlineto{\pgfqpoint{0.227191in}{1.375261in}}%
\pgfpathlineto{\pgfqpoint{0.225144in}{1.386388in}}%
\pgfpathlineto{\pgfqpoint{0.224448in}{1.388188in}}%
\pgfpathlineto{\pgfqpoint{0.211875in}{1.398298in}}%
\pgfpathlineto{\pgfqpoint{0.211875in}{1.388303in}}%
\pgfpathlineto{\pgfqpoint{0.212018in}{1.388188in}}%
\pgfpathlineto{\pgfqpoint{0.218794in}{1.375261in}}%
\pgfpathlineto{\pgfqpoint{0.218241in}{1.362335in}}%
\pgfpathlineto{\pgfqpoint{0.211875in}{1.352198in}}%
\pgfpathlineto{\pgfqpoint{0.211875in}{1.349408in}}%
\pgfpathlineto{\pgfqpoint{0.211875in}{1.341706in}}%
\pgfpathclose%
\pgfusepath{fill}%
\end{pgfscope}%
\begin{pgfscope}%
\pgfpathrectangle{\pgfqpoint{0.211875in}{0.211875in}}{\pgfqpoint{1.313625in}{1.279725in}}%
\pgfusepath{clip}%
\pgfsetbuttcap%
\pgfsetroundjoin%
\definecolor{currentfill}{rgb}{0.490838,0.119982,0.351115}%
\pgfsetfillcolor{currentfill}%
\pgfsetlinewidth{0.000000pt}%
\definecolor{currentstroke}{rgb}{0.000000,0.000000,0.000000}%
\pgfsetstrokecolor{currentstroke}%
\pgfsetdash{}{0pt}%
\pgfpathmoveto{\pgfqpoint{0.318027in}{1.346433in}}%
\pgfpathlineto{\pgfqpoint{0.331295in}{1.346828in}}%
\pgfpathlineto{\pgfqpoint{0.334522in}{1.349408in}}%
\pgfpathlineto{\pgfqpoint{0.340910in}{1.362335in}}%
\pgfpathlineto{\pgfqpoint{0.341340in}{1.375261in}}%
\pgfpathlineto{\pgfqpoint{0.336443in}{1.388188in}}%
\pgfpathlineto{\pgfqpoint{0.331295in}{1.392846in}}%
\pgfpathlineto{\pgfqpoint{0.318027in}{1.393271in}}%
\pgfpathlineto{\pgfqpoint{0.312011in}{1.388188in}}%
\pgfpathlineto{\pgfqpoint{0.307111in}{1.375261in}}%
\pgfpathlineto{\pgfqpoint{0.307544in}{1.362335in}}%
\pgfpathlineto{\pgfqpoint{0.314019in}{1.349408in}}%
\pgfpathclose%
\pgfpathmoveto{\pgfqpoint{0.317021in}{1.362335in}}%
\pgfpathlineto{\pgfqpoint{0.316427in}{1.375261in}}%
\pgfpathlineto{\pgfqpoint{0.318027in}{1.378530in}}%
\pgfpathlineto{\pgfqpoint{0.331295in}{1.377855in}}%
\pgfpathlineto{\pgfqpoint{0.332506in}{1.375261in}}%
\pgfpathlineto{\pgfqpoint{0.331934in}{1.362335in}}%
\pgfpathlineto{\pgfqpoint{0.331295in}{1.361190in}}%
\pgfpathlineto{\pgfqpoint{0.318027in}{1.360617in}}%
\pgfpathclose%
\pgfusepath{fill}%
\end{pgfscope}%
\begin{pgfscope}%
\pgfpathrectangle{\pgfqpoint{0.211875in}{0.211875in}}{\pgfqpoint{1.313625in}{1.279725in}}%
\pgfusepath{clip}%
\pgfsetbuttcap%
\pgfsetroundjoin%
\definecolor{currentfill}{rgb}{0.490838,0.119982,0.351115}%
\pgfsetfillcolor{currentfill}%
\pgfsetlinewidth{0.000000pt}%
\definecolor{currentstroke}{rgb}{0.000000,0.000000,0.000000}%
\pgfsetstrokecolor{currentstroke}%
\pgfsetdash{}{0pt}%
\pgfpathmoveto{\pgfqpoint{0.437447in}{1.350489in}}%
\pgfpathlineto{\pgfqpoint{0.450716in}{1.353267in}}%
\pgfpathlineto{\pgfqpoint{0.455229in}{1.362335in}}%
\pgfpathlineto{\pgfqpoint{0.455681in}{1.375261in}}%
\pgfpathlineto{\pgfqpoint{0.450716in}{1.387190in}}%
\pgfpathlineto{\pgfqpoint{0.446757in}{1.388188in}}%
\pgfpathlineto{\pgfqpoint{0.437447in}{1.389392in}}%
\pgfpathlineto{\pgfqpoint{0.435832in}{1.388188in}}%
\pgfpathlineto{\pgfqpoint{0.429038in}{1.375261in}}%
\pgfpathlineto{\pgfqpoint{0.429593in}{1.362335in}}%
\pgfpathclose%
\pgfusepath{fill}%
\end{pgfscope}%
\begin{pgfscope}%
\pgfpathrectangle{\pgfqpoint{0.211875in}{0.211875in}}{\pgfqpoint{1.313625in}{1.279725in}}%
\pgfusepath{clip}%
\pgfsetbuttcap%
\pgfsetroundjoin%
\definecolor{currentfill}{rgb}{0.490838,0.119982,0.351115}%
\pgfsetfillcolor{currentfill}%
\pgfsetlinewidth{0.000000pt}%
\definecolor{currentstroke}{rgb}{0.000000,0.000000,0.000000}%
\pgfsetstrokecolor{currentstroke}%
\pgfsetdash{}{0pt}%
\pgfpathmoveto{\pgfqpoint{0.556867in}{1.355479in}}%
\pgfpathlineto{\pgfqpoint{0.570136in}{1.361164in}}%
\pgfpathlineto{\pgfqpoint{0.570659in}{1.362335in}}%
\pgfpathlineto{\pgfqpoint{0.571125in}{1.375261in}}%
\pgfpathlineto{\pgfqpoint{0.570136in}{1.377914in}}%
\pgfpathlineto{\pgfqpoint{0.556867in}{1.384538in}}%
\pgfpathlineto{\pgfqpoint{0.550959in}{1.375261in}}%
\pgfpathlineto{\pgfqpoint{0.551657in}{1.362335in}}%
\pgfpathclose%
\pgfusepath{fill}%
\end{pgfscope}%
\begin{pgfscope}%
\pgfpathrectangle{\pgfqpoint{0.211875in}{0.211875in}}{\pgfqpoint{1.313625in}{1.279725in}}%
\pgfusepath{clip}%
\pgfsetbuttcap%
\pgfsetroundjoin%
\definecolor{currentfill}{rgb}{0.490838,0.119982,0.351115}%
\pgfsetfillcolor{currentfill}%
\pgfsetlinewidth{0.000000pt}%
\definecolor{currentstroke}{rgb}{0.000000,0.000000,0.000000}%
\pgfsetstrokecolor{currentstroke}%
\pgfsetdash{}{0pt}%
\pgfpathmoveto{\pgfqpoint{0.676288in}{1.359547in}}%
\pgfpathlineto{\pgfqpoint{0.680747in}{1.362335in}}%
\pgfpathlineto{\pgfqpoint{0.682331in}{1.375261in}}%
\pgfpathlineto{\pgfqpoint{0.676288in}{1.379760in}}%
\pgfpathlineto{\pgfqpoint{0.672946in}{1.375261in}}%
\pgfpathlineto{\pgfqpoint{0.673819in}{1.362335in}}%
\pgfpathclose%
\pgfusepath{fill}%
\end{pgfscope}%
\begin{pgfscope}%
\pgfpathrectangle{\pgfqpoint{0.211875in}{0.211875in}}{\pgfqpoint{1.313625in}{1.279725in}}%
\pgfusepath{clip}%
\pgfsetbuttcap%
\pgfsetroundjoin%
\definecolor{currentfill}{rgb}{0.490838,0.119982,0.351115}%
\pgfsetfillcolor{currentfill}%
\pgfsetlinewidth{0.000000pt}%
\definecolor{currentstroke}{rgb}{0.000000,0.000000,0.000000}%
\pgfsetstrokecolor{currentstroke}%
\pgfsetdash{}{0pt}%
\pgfpathmoveto{\pgfqpoint{0.795708in}{1.368177in}}%
\pgfpathlineto{\pgfqpoint{0.796407in}{1.375261in}}%
\pgfpathlineto{\pgfqpoint{0.795708in}{1.375935in}}%
\pgfpathlineto{\pgfqpoint{0.795111in}{1.375261in}}%
\pgfpathclose%
\pgfusepath{fill}%
\end{pgfscope}%
\begin{pgfscope}%
\pgfpathrectangle{\pgfqpoint{0.211875in}{0.211875in}}{\pgfqpoint{1.313625in}{1.279725in}}%
\pgfusepath{clip}%
\pgfsetbuttcap%
\pgfsetroundjoin%
\definecolor{currentfill}{rgb}{0.490838,0.119982,0.351115}%
\pgfsetfillcolor{currentfill}%
\pgfsetlinewidth{0.000000pt}%
\definecolor{currentstroke}{rgb}{0.000000,0.000000,0.000000}%
\pgfsetstrokecolor{currentstroke}%
\pgfsetdash{}{0pt}%
\pgfpathmoveto{\pgfqpoint{0.264951in}{1.423441in}}%
\pgfpathlineto{\pgfqpoint{0.276707in}{1.426967in}}%
\pgfpathlineto{\pgfqpoint{0.278220in}{1.427805in}}%
\pgfpathlineto{\pgfqpoint{0.283845in}{1.439894in}}%
\pgfpathlineto{\pgfqpoint{0.285005in}{1.452820in}}%
\pgfpathlineto{\pgfqpoint{0.283149in}{1.465747in}}%
\pgfpathlineto{\pgfqpoint{0.278220in}{1.474452in}}%
\pgfpathlineto{\pgfqpoint{0.267688in}{1.478673in}}%
\pgfpathlineto{\pgfqpoint{0.264951in}{1.479326in}}%
\pgfpathlineto{\pgfqpoint{0.262119in}{1.478673in}}%
\pgfpathlineto{\pgfqpoint{0.251682in}{1.474275in}}%
\pgfpathlineto{\pgfqpoint{0.247186in}{1.465747in}}%
\pgfpathlineto{\pgfqpoint{0.245515in}{1.452820in}}%
\pgfpathlineto{\pgfqpoint{0.246580in}{1.439894in}}%
\pgfpathlineto{\pgfqpoint{0.251682in}{1.428102in}}%
\pgfpathlineto{\pgfqpoint{0.253522in}{1.426967in}}%
\pgfpathclose%
\pgfpathmoveto{\pgfqpoint{0.256844in}{1.439894in}}%
\pgfpathlineto{\pgfqpoint{0.252770in}{1.452820in}}%
\pgfpathlineto{\pgfqpoint{0.259870in}{1.465747in}}%
\pgfpathlineto{\pgfqpoint{0.264951in}{1.468663in}}%
\pgfpathlineto{\pgfqpoint{0.270529in}{1.465747in}}%
\pgfpathlineto{\pgfqpoint{0.278220in}{1.453674in}}%
\pgfpathlineto{\pgfqpoint{0.278392in}{1.452820in}}%
\pgfpathlineto{\pgfqpoint{0.278220in}{1.451451in}}%
\pgfpathlineto{\pgfqpoint{0.273971in}{1.439894in}}%
\pgfpathlineto{\pgfqpoint{0.264951in}{1.434124in}}%
\pgfpathclose%
\pgfusepath{fill}%
\end{pgfscope}%
\begin{pgfscope}%
\pgfpathrectangle{\pgfqpoint{0.211875in}{0.211875in}}{\pgfqpoint{1.313625in}{1.279725in}}%
\pgfusepath{clip}%
\pgfsetbuttcap%
\pgfsetroundjoin%
\definecolor{currentfill}{rgb}{0.490838,0.119982,0.351115}%
\pgfsetfillcolor{currentfill}%
\pgfsetlinewidth{0.000000pt}%
\definecolor{currentstroke}{rgb}{0.000000,0.000000,0.000000}%
\pgfsetstrokecolor{currentstroke}%
\pgfsetdash{}{0pt}%
\pgfpathmoveto{\pgfqpoint{0.371102in}{1.433804in}}%
\pgfpathlineto{\pgfqpoint{0.384371in}{1.427529in}}%
\pgfpathlineto{\pgfqpoint{0.397640in}{1.437190in}}%
\pgfpathlineto{\pgfqpoint{0.398771in}{1.439894in}}%
\pgfpathlineto{\pgfqpoint{0.400147in}{1.452820in}}%
\pgfpathlineto{\pgfqpoint{0.397839in}{1.465747in}}%
\pgfpathlineto{\pgfqpoint{0.397640in}{1.466139in}}%
\pgfpathlineto{\pgfqpoint{0.384371in}{1.474477in}}%
\pgfpathlineto{\pgfqpoint{0.371102in}{1.469118in}}%
\pgfpathlineto{\pgfqpoint{0.369113in}{1.465747in}}%
\pgfpathlineto{\pgfqpoint{0.366738in}{1.452820in}}%
\pgfpathlineto{\pgfqpoint{0.368162in}{1.439894in}}%
\pgfpathclose%
\pgfpathmoveto{\pgfqpoint{0.384303in}{1.439894in}}%
\pgfpathlineto{\pgfqpoint{0.378310in}{1.452820in}}%
\pgfpathlineto{\pgfqpoint{0.384371in}{1.460723in}}%
\pgfpathlineto{\pgfqpoint{0.388861in}{1.452820in}}%
\pgfpathlineto{\pgfqpoint{0.384421in}{1.439894in}}%
\pgfpathlineto{\pgfqpoint{0.384371in}{1.439855in}}%
\pgfpathclose%
\pgfusepath{fill}%
\end{pgfscope}%
\begin{pgfscope}%
\pgfpathrectangle{\pgfqpoint{0.211875in}{0.211875in}}{\pgfqpoint{1.313625in}{1.279725in}}%
\pgfusepath{clip}%
\pgfsetbuttcap%
\pgfsetroundjoin%
\definecolor{currentfill}{rgb}{0.490838,0.119982,0.351115}%
\pgfsetfillcolor{currentfill}%
\pgfsetlinewidth{0.000000pt}%
\definecolor{currentstroke}{rgb}{0.000000,0.000000,0.000000}%
\pgfsetstrokecolor{currentstroke}%
\pgfsetdash{}{0pt}%
\pgfpathmoveto{\pgfqpoint{0.490523in}{1.438023in}}%
\pgfpathlineto{\pgfqpoint{0.503792in}{1.432625in}}%
\pgfpathlineto{\pgfqpoint{0.511678in}{1.439894in}}%
\pgfpathlineto{\pgfqpoint{0.514783in}{1.452820in}}%
\pgfpathlineto{\pgfqpoint{0.509440in}{1.465747in}}%
\pgfpathlineto{\pgfqpoint{0.503792in}{1.469987in}}%
\pgfpathlineto{\pgfqpoint{0.491539in}{1.465747in}}%
\pgfpathlineto{\pgfqpoint{0.490523in}{1.464747in}}%
\pgfpathlineto{\pgfqpoint{0.487735in}{1.452820in}}%
\pgfpathlineto{\pgfqpoint{0.489516in}{1.439894in}}%
\pgfpathclose%
\pgfusepath{fill}%
\end{pgfscope}%
\begin{pgfscope}%
\pgfpathrectangle{\pgfqpoint{0.211875in}{0.211875in}}{\pgfqpoint{1.313625in}{1.279725in}}%
\pgfusepath{clip}%
\pgfsetbuttcap%
\pgfsetroundjoin%
\definecolor{currentfill}{rgb}{0.490838,0.119982,0.351115}%
\pgfsetfillcolor{currentfill}%
\pgfsetlinewidth{0.000000pt}%
\definecolor{currentstroke}{rgb}{0.000000,0.000000,0.000000}%
\pgfsetstrokecolor{currentstroke}%
\pgfsetdash{}{0pt}%
\pgfpathmoveto{\pgfqpoint{0.623212in}{1.437165in}}%
\pgfpathlineto{\pgfqpoint{0.625772in}{1.439894in}}%
\pgfpathlineto{\pgfqpoint{0.628806in}{1.452820in}}%
\pgfpathlineto{\pgfqpoint{0.623492in}{1.465747in}}%
\pgfpathlineto{\pgfqpoint{0.623212in}{1.465990in}}%
\pgfpathlineto{\pgfqpoint{0.622153in}{1.465747in}}%
\pgfpathlineto{\pgfqpoint{0.609943in}{1.458286in}}%
\pgfpathlineto{\pgfqpoint{0.608514in}{1.452820in}}%
\pgfpathlineto{\pgfqpoint{0.609943in}{1.443932in}}%
\pgfpathlineto{\pgfqpoint{0.613648in}{1.439894in}}%
\pgfpathclose%
\pgfusepath{fill}%
\end{pgfscope}%
\begin{pgfscope}%
\pgfpathrectangle{\pgfqpoint{0.211875in}{0.211875in}}{\pgfqpoint{1.313625in}{1.279725in}}%
\pgfusepath{clip}%
\pgfsetbuttcap%
\pgfsetroundjoin%
\definecolor{currentfill}{rgb}{0.490838,0.119982,0.351115}%
\pgfsetfillcolor{currentfill}%
\pgfsetlinewidth{0.000000pt}%
\definecolor{currentstroke}{rgb}{0.000000,0.000000,0.000000}%
\pgfsetstrokecolor{currentstroke}%
\pgfsetdash{}{0pt}%
\pgfpathmoveto{\pgfqpoint{0.729364in}{1.451176in}}%
\pgfpathlineto{\pgfqpoint{0.742633in}{1.444406in}}%
\pgfpathlineto{\pgfqpoint{0.744488in}{1.452820in}}%
\pgfpathlineto{\pgfqpoint{0.742633in}{1.457933in}}%
\pgfpathlineto{\pgfqpoint{0.729364in}{1.453827in}}%
\pgfpathlineto{\pgfqpoint{0.729069in}{1.452820in}}%
\pgfpathclose%
\pgfusepath{fill}%
\end{pgfscope}%
\begin{pgfscope}%
\pgfpathrectangle{\pgfqpoint{0.211875in}{0.211875in}}{\pgfqpoint{1.313625in}{1.279725in}}%
\pgfusepath{clip}%
\pgfsetbuttcap%
\pgfsetroundjoin%
\definecolor{currentfill}{rgb}{0.490838,0.119982,0.351115}%
\pgfsetfillcolor{currentfill}%
\pgfsetlinewidth{0.000000pt}%
\definecolor{currentstroke}{rgb}{0.000000,0.000000,0.000000}%
\pgfsetstrokecolor{currentstroke}%
\pgfsetdash{}{0pt}%
\pgfpathmoveto{\pgfqpoint{1.445886in}{1.447487in}}%
\pgfpathlineto{\pgfqpoint{1.447782in}{1.452820in}}%
\pgfpathlineto{\pgfqpoint{1.445886in}{1.456048in}}%
\pgfpathlineto{\pgfqpoint{1.443516in}{1.452820in}}%
\pgfpathclose%
\pgfusepath{fill}%
\end{pgfscope}%
\begin{pgfscope}%
\pgfpathrectangle{\pgfqpoint{0.211875in}{0.211875in}}{\pgfqpoint{1.313625in}{1.279725in}}%
\pgfusepath{clip}%
\pgfsetbuttcap%
\pgfsetroundjoin%
\definecolor{currentfill}{rgb}{0.644838,0.098089,0.355336}%
\pgfsetfillcolor{currentfill}%
\pgfsetlinewidth{0.000000pt}%
\definecolor{currentstroke}{rgb}{0.000000,0.000000,0.000000}%
\pgfsetstrokecolor{currentstroke}%
\pgfsetdash{}{0pt}%
\pgfpathmoveto{\pgfqpoint{0.238413in}{0.223031in}}%
\pgfpathlineto{\pgfqpoint{0.239572in}{0.211875in}}%
\pgfpathlineto{\pgfqpoint{0.248423in}{0.211875in}}%
\pgfpathlineto{\pgfqpoint{0.244808in}{0.224802in}}%
\pgfpathlineto{\pgfqpoint{0.244235in}{0.237728in}}%
\pgfpathlineto{\pgfqpoint{0.246001in}{0.250655in}}%
\pgfpathlineto{\pgfqpoint{0.251682in}{0.260457in}}%
\pgfpathlineto{\pgfqpoint{0.259446in}{0.263581in}}%
\pgfpathlineto{\pgfqpoint{0.264951in}{0.264874in}}%
\pgfpathlineto{\pgfqpoint{0.275696in}{0.263581in}}%
\pgfpathlineto{\pgfqpoint{0.278220in}{0.263104in}}%
\pgfpathlineto{\pgfqpoint{0.287144in}{0.250655in}}%
\pgfpathlineto{\pgfqpoint{0.288768in}{0.237728in}}%
\pgfpathlineto{\pgfqpoint{0.288160in}{0.224802in}}%
\pgfpathlineto{\pgfqpoint{0.284527in}{0.211875in}}%
\pgfpathlineto{\pgfqpoint{0.291489in}{0.211875in}}%
\pgfpathlineto{\pgfqpoint{0.294169in}{0.211875in}}%
\pgfpathlineto{\pgfqpoint{0.294881in}{0.224802in}}%
\pgfpathlineto{\pgfqpoint{0.295245in}{0.237728in}}%
\pgfpathlineto{\pgfqpoint{0.295639in}{0.250655in}}%
\pgfpathlineto{\pgfqpoint{0.296863in}{0.263581in}}%
\pgfpathlineto{\pgfqpoint{0.304758in}{0.270590in}}%
\pgfpathlineto{\pgfqpoint{0.318027in}{0.271319in}}%
\pgfpathlineto{\pgfqpoint{0.331295in}{0.270909in}}%
\pgfpathlineto{\pgfqpoint{0.344564in}{0.268330in}}%
\pgfpathlineto{\pgfqpoint{0.349411in}{0.263581in}}%
\pgfpathlineto{\pgfqpoint{0.352273in}{0.250655in}}%
\pgfpathlineto{\pgfqpoint{0.353015in}{0.237728in}}%
\pgfpathlineto{\pgfqpoint{0.353330in}{0.224802in}}%
\pgfpathlineto{\pgfqpoint{0.353500in}{0.211875in}}%
\pgfpathlineto{\pgfqpoint{0.357833in}{0.211875in}}%
\pgfpathlineto{\pgfqpoint{0.362144in}{0.211875in}}%
\pgfpathlineto{\pgfqpoint{0.359505in}{0.224802in}}%
\pgfpathlineto{\pgfqpoint{0.358981in}{0.237728in}}%
\pgfpathlineto{\pgfqpoint{0.359966in}{0.250655in}}%
\pgfpathlineto{\pgfqpoint{0.365624in}{0.263581in}}%
\pgfpathlineto{\pgfqpoint{0.371102in}{0.266902in}}%
\pgfpathlineto{\pgfqpoint{0.384371in}{0.268692in}}%
\pgfpathlineto{\pgfqpoint{0.397640in}{0.267609in}}%
\pgfpathlineto{\pgfqpoint{0.404538in}{0.263581in}}%
\pgfpathlineto{\pgfqpoint{0.408968in}{0.250655in}}%
\pgfpathlineto{\pgfqpoint{0.409625in}{0.237728in}}%
\pgfpathlineto{\pgfqpoint{0.409125in}{0.224802in}}%
\pgfpathlineto{\pgfqpoint{0.406865in}{0.211875in}}%
\pgfpathlineto{\pgfqpoint{0.410909in}{0.211875in}}%
\pgfpathlineto{\pgfqpoint{0.415944in}{0.211875in}}%
\pgfpathlineto{\pgfqpoint{0.415597in}{0.224802in}}%
\pgfpathlineto{\pgfqpoint{0.415896in}{0.237728in}}%
\pgfpathlineto{\pgfqpoint{0.417090in}{0.250655in}}%
\pgfpathlineto{\pgfqpoint{0.422505in}{0.263581in}}%
\pgfpathlineto{\pgfqpoint{0.424178in}{0.264882in}}%
\pgfpathlineto{\pgfqpoint{0.437447in}{0.267899in}}%
\pgfpathlineto{\pgfqpoint{0.450716in}{0.267253in}}%
\pgfpathlineto{\pgfqpoint{0.459988in}{0.263581in}}%
\pgfpathlineto{\pgfqpoint{0.463985in}{0.259666in}}%
\pgfpathlineto{\pgfqpoint{0.467199in}{0.250655in}}%
\pgfpathlineto{\pgfqpoint{0.468595in}{0.237728in}}%
\pgfpathlineto{\pgfqpoint{0.468851in}{0.224802in}}%
\pgfpathlineto{\pgfqpoint{0.468143in}{0.211875in}}%
\pgfpathlineto{\pgfqpoint{0.476191in}{0.211875in}}%
\pgfpathlineto{\pgfqpoint{0.474771in}{0.224802in}}%
\pgfpathlineto{\pgfqpoint{0.474375in}{0.237728in}}%
\pgfpathlineto{\pgfqpoint{0.474576in}{0.250655in}}%
\pgfpathlineto{\pgfqpoint{0.476326in}{0.263581in}}%
\pgfpathlineto{\pgfqpoint{0.477254in}{0.265648in}}%
\pgfpathlineto{\pgfqpoint{0.490523in}{0.271308in}}%
\pgfpathlineto{\pgfqpoint{0.503792in}{0.271947in}}%
\pgfpathlineto{\pgfqpoint{0.517061in}{0.271899in}}%
\pgfpathlineto{\pgfqpoint{0.529519in}{0.263581in}}%
\pgfpathlineto{\pgfqpoint{0.529979in}{0.250655in}}%
\pgfpathlineto{\pgfqpoint{0.529892in}{0.237728in}}%
\pgfpathlineto{\pgfqpoint{0.529475in}{0.224802in}}%
\pgfpathlineto{\pgfqpoint{0.528271in}{0.211875in}}%
\pgfpathlineto{\pgfqpoint{0.530330in}{0.211875in}}%
\pgfpathlineto{\pgfqpoint{0.537207in}{0.211875in}}%
\pgfpathlineto{\pgfqpoint{0.535935in}{0.224802in}}%
\pgfpathlineto{\pgfqpoint{0.536181in}{0.237728in}}%
\pgfpathlineto{\pgfqpoint{0.538082in}{0.250655in}}%
\pgfpathlineto{\pgfqpoint{0.543598in}{0.260629in}}%
\pgfpathlineto{\pgfqpoint{0.550295in}{0.263581in}}%
\pgfpathlineto{\pgfqpoint{0.556867in}{0.265123in}}%
\pgfpathlineto{\pgfqpoint{0.570136in}{0.263878in}}%
\pgfpathlineto{\pgfqpoint{0.570816in}{0.263581in}}%
\pgfpathlineto{\pgfqpoint{0.582369in}{0.250655in}}%
\pgfpathlineto{\pgfqpoint{0.583405in}{0.246168in}}%
\pgfpathlineto{\pgfqpoint{0.584569in}{0.237728in}}%
\pgfpathlineto{\pgfqpoint{0.584776in}{0.224802in}}%
\pgfpathlineto{\pgfqpoint{0.583405in}{0.212412in}}%
\pgfpathlineto{\pgfqpoint{0.583299in}{0.211875in}}%
\pgfpathlineto{\pgfqpoint{0.583405in}{0.211875in}}%
\pgfpathlineto{\pgfqpoint{0.591382in}{0.211875in}}%
\pgfpathlineto{\pgfqpoint{0.590695in}{0.224802in}}%
\pgfpathlineto{\pgfqpoint{0.590348in}{0.237728in}}%
\pgfpathlineto{\pgfqpoint{0.589997in}{0.250655in}}%
\pgfpathlineto{\pgfqpoint{0.589029in}{0.263581in}}%
\pgfpathlineto{\pgfqpoint{0.583405in}{0.271154in}}%
\pgfpathlineto{\pgfqpoint{0.570136in}{0.272571in}}%
\pgfpathlineto{\pgfqpoint{0.556867in}{0.272909in}}%
\pgfpathlineto{\pgfqpoint{0.543598in}{0.273281in}}%
\pgfpathlineto{\pgfqpoint{0.532796in}{0.276508in}}%
\pgfpathlineto{\pgfqpoint{0.531055in}{0.289434in}}%
\pgfpathlineto{\pgfqpoint{0.530756in}{0.302361in}}%
\pgfpathlineto{\pgfqpoint{0.530485in}{0.315287in}}%
\pgfpathlineto{\pgfqpoint{0.530330in}{0.319855in}}%
\pgfpathlineto{\pgfqpoint{0.529986in}{0.328214in}}%
\pgfpathlineto{\pgfqpoint{0.528476in}{0.341140in}}%
\pgfpathlineto{\pgfqpoint{0.517061in}{0.351694in}}%
\pgfpathlineto{\pgfqpoint{0.503792in}{0.352819in}}%
\pgfpathlineto{\pgfqpoint{0.490523in}{0.352912in}}%
\pgfpathlineto{\pgfqpoint{0.477254in}{0.351338in}}%
\pgfpathlineto{\pgfqpoint{0.472656in}{0.341140in}}%
\pgfpathlineto{\pgfqpoint{0.472073in}{0.328214in}}%
\pgfpathlineto{\pgfqpoint{0.471776in}{0.315287in}}%
\pgfpathlineto{\pgfqpoint{0.471396in}{0.302361in}}%
\pgfpathlineto{\pgfqpoint{0.470453in}{0.289434in}}%
\pgfpathlineto{\pgfqpoint{0.463985in}{0.277646in}}%
\pgfpathlineto{\pgfqpoint{0.459122in}{0.276508in}}%
\pgfpathlineto{\pgfqpoint{0.450716in}{0.275613in}}%
\pgfpathlineto{\pgfqpoint{0.437447in}{0.275836in}}%
\pgfpathlineto{\pgfqpoint{0.432435in}{0.276508in}}%
\pgfpathlineto{\pgfqpoint{0.424178in}{0.278518in}}%
\pgfpathlineto{\pgfqpoint{0.416853in}{0.289434in}}%
\pgfpathlineto{\pgfqpoint{0.415318in}{0.302361in}}%
\pgfpathlineto{\pgfqpoint{0.414832in}{0.315287in}}%
\pgfpathlineto{\pgfqpoint{0.414738in}{0.328214in}}%
\pgfpathlineto{\pgfqpoint{0.415128in}{0.341140in}}%
\pgfpathlineto{\pgfqpoint{0.424178in}{0.353334in}}%
\pgfpathlineto{\pgfqpoint{0.434635in}{0.354067in}}%
\pgfpathlineto{\pgfqpoint{0.437447in}{0.354176in}}%
\pgfpathlineto{\pgfqpoint{0.450716in}{0.354573in}}%
\pgfpathlineto{\pgfqpoint{0.463985in}{0.355744in}}%
\pgfpathlineto{\pgfqpoint{0.470449in}{0.366993in}}%
\pgfpathlineto{\pgfqpoint{0.470840in}{0.379920in}}%
\pgfpathlineto{\pgfqpoint{0.470774in}{0.392846in}}%
\pgfpathlineto{\pgfqpoint{0.470363in}{0.405773in}}%
\pgfpathlineto{\pgfqpoint{0.469073in}{0.418699in}}%
\pgfpathlineto{\pgfqpoint{0.463985in}{0.429064in}}%
\pgfpathlineto{\pgfqpoint{0.457740in}{0.431626in}}%
\pgfpathlineto{\pgfqpoint{0.450716in}{0.432914in}}%
\pgfpathlineto{\pgfqpoint{0.437447in}{0.433405in}}%
\pgfpathlineto{\pgfqpoint{0.424178in}{0.432456in}}%
\pgfpathlineto{\pgfqpoint{0.421436in}{0.431626in}}%
\pgfpathlineto{\pgfqpoint{0.414699in}{0.418699in}}%
\pgfpathlineto{\pgfqpoint{0.413929in}{0.405773in}}%
\pgfpathlineto{\pgfqpoint{0.413596in}{0.392846in}}%
\pgfpathlineto{\pgfqpoint{0.413301in}{0.379920in}}%
\pgfpathlineto{\pgfqpoint{0.412630in}{0.366993in}}%
\pgfpathlineto{\pgfqpoint{0.410909in}{0.360149in}}%
\pgfpathlineto{\pgfqpoint{0.397640in}{0.356005in}}%
\pgfpathlineto{\pgfqpoint{0.384371in}{0.355959in}}%
\pgfpathlineto{\pgfqpoint{0.371102in}{0.356874in}}%
\pgfpathlineto{\pgfqpoint{0.357833in}{0.365531in}}%
\pgfpathlineto{\pgfqpoint{0.357363in}{0.366993in}}%
\pgfpathlineto{\pgfqpoint{0.355677in}{0.379920in}}%
\pgfpathlineto{\pgfqpoint{0.355167in}{0.392846in}}%
\pgfpathlineto{\pgfqpoint{0.354963in}{0.405773in}}%
\pgfpathlineto{\pgfqpoint{0.354920in}{0.418699in}}%
\pgfpathlineto{\pgfqpoint{0.355346in}{0.431626in}}%
\pgfpathlineto{\pgfqpoint{0.344564in}{0.436474in}}%
\pgfpathlineto{\pgfqpoint{0.331295in}{0.436490in}}%
\pgfpathlineto{\pgfqpoint{0.318027in}{0.436919in}}%
\pgfpathlineto{\pgfqpoint{0.304758in}{0.438904in}}%
\pgfpathlineto{\pgfqpoint{0.298935in}{0.444552in}}%
\pgfpathlineto{\pgfqpoint{0.296750in}{0.457479in}}%
\pgfpathlineto{\pgfqpoint{0.296202in}{0.470405in}}%
\pgfpathlineto{\pgfqpoint{0.295966in}{0.483332in}}%
\pgfpathlineto{\pgfqpoint{0.295826in}{0.496258in}}%
\pgfpathlineto{\pgfqpoint{0.295623in}{0.509185in}}%
\pgfpathlineto{\pgfqpoint{0.291489in}{0.516886in}}%
\pgfpathlineto{\pgfqpoint{0.278220in}{0.517320in}}%
\pgfpathlineto{\pgfqpoint{0.264951in}{0.517630in}}%
\pgfpathlineto{\pgfqpoint{0.251682in}{0.518526in}}%
\pgfpathlineto{\pgfqpoint{0.242166in}{0.522111in}}%
\pgfpathlineto{\pgfqpoint{0.238413in}{0.530905in}}%
\pgfpathlineto{\pgfqpoint{0.237853in}{0.535038in}}%
\pgfpathlineto{\pgfqpoint{0.237149in}{0.547964in}}%
\pgfpathlineto{\pgfqpoint{0.236885in}{0.560891in}}%
\pgfpathlineto{\pgfqpoint{0.236767in}{0.573817in}}%
\pgfpathlineto{\pgfqpoint{0.236728in}{0.586744in}}%
\pgfpathlineto{\pgfqpoint{0.238413in}{0.598361in}}%
\pgfpathlineto{\pgfqpoint{0.251682in}{0.598654in}}%
\pgfpathlineto{\pgfqpoint{0.264951in}{0.598920in}}%
\pgfpathlineto{\pgfqpoint{0.277190in}{0.599670in}}%
\pgfpathlineto{\pgfqpoint{0.278220in}{0.599753in}}%
\pgfpathlineto{\pgfqpoint{0.291489in}{0.608188in}}%
\pgfpathlineto{\pgfqpoint{0.292501in}{0.612597in}}%
\pgfpathlineto{\pgfqpoint{0.293395in}{0.625523in}}%
\pgfpathlineto{\pgfqpoint{0.293463in}{0.638450in}}%
\pgfpathlineto{\pgfqpoint{0.293005in}{0.651377in}}%
\pgfpathlineto{\pgfqpoint{0.291489in}{0.663248in}}%
\pgfpathlineto{\pgfqpoint{0.291241in}{0.664303in}}%
\pgfpathlineto{\pgfqpoint{0.278220in}{0.675578in}}%
\pgfpathlineto{\pgfqpoint{0.264951in}{0.676802in}}%
\pgfpathlineto{\pgfqpoint{0.251682in}{0.676495in}}%
\pgfpathlineto{\pgfqpoint{0.238413in}{0.666222in}}%
\pgfpathlineto{\pgfqpoint{0.238076in}{0.664303in}}%
\pgfpathlineto{\pgfqpoint{0.237159in}{0.651377in}}%
\pgfpathlineto{\pgfqpoint{0.236861in}{0.638450in}}%
\pgfpathlineto{\pgfqpoint{0.236727in}{0.625523in}}%
\pgfpathlineto{\pgfqpoint{0.236653in}{0.612597in}}%
\pgfpathlineto{\pgfqpoint{0.236201in}{0.599670in}}%
\pgfpathlineto{\pgfqpoint{0.225144in}{0.598691in}}%
\pgfpathlineto{\pgfqpoint{0.211875in}{0.598868in}}%
\pgfpathlineto{\pgfqpoint{0.211875in}{0.590781in}}%
\pgfpathlineto{\pgfqpoint{0.222495in}{0.586744in}}%
\pgfpathlineto{\pgfqpoint{0.225144in}{0.584609in}}%
\pgfpathlineto{\pgfqpoint{0.229634in}{0.573817in}}%
\pgfpathlineto{\pgfqpoint{0.231057in}{0.560891in}}%
\pgfpathlineto{\pgfqpoint{0.230976in}{0.547964in}}%
\pgfpathlineto{\pgfqpoint{0.229069in}{0.535038in}}%
\pgfpathlineto{\pgfqpoint{0.225144in}{0.527778in}}%
\pgfpathlineto{\pgfqpoint{0.211875in}{0.522262in}}%
\pgfpathlineto{\pgfqpoint{0.211875in}{0.522111in}}%
\pgfpathlineto{\pgfqpoint{0.211875in}{0.513910in}}%
\pgfpathlineto{\pgfqpoint{0.225144in}{0.512637in}}%
\pgfpathlineto{\pgfqpoint{0.230470in}{0.509185in}}%
\pgfpathlineto{\pgfqpoint{0.233695in}{0.496258in}}%
\pgfpathlineto{\pgfqpoint{0.234119in}{0.483332in}}%
\pgfpathlineto{\pgfqpoint{0.233888in}{0.470405in}}%
\pgfpathlineto{\pgfqpoint{0.232797in}{0.457479in}}%
\pgfpathlineto{\pgfqpoint{0.227474in}{0.444552in}}%
\pgfpathlineto{\pgfqpoint{0.225144in}{0.442900in}}%
\pgfpathlineto{\pgfqpoint{0.211875in}{0.440410in}}%
\pgfpathlineto{\pgfqpoint{0.211875in}{0.432471in}}%
\pgfpathlineto{\pgfqpoint{0.217277in}{0.431626in}}%
\pgfpathlineto{\pgfqpoint{0.225144in}{0.429398in}}%
\pgfpathlineto{\pgfqpoint{0.232119in}{0.418699in}}%
\pgfpathlineto{\pgfqpoint{0.233764in}{0.405773in}}%
\pgfpathlineto{\pgfqpoint{0.234239in}{0.392846in}}%
\pgfpathlineto{\pgfqpoint{0.234187in}{0.379920in}}%
\pgfpathlineto{\pgfqpoint{0.233234in}{0.366993in}}%
\pgfpathlineto{\pgfqpoint{0.225144in}{0.356733in}}%
\pgfpathlineto{\pgfqpoint{0.211875in}{0.355202in}}%
\pgfpathlineto{\pgfqpoint{0.211875in}{0.354067in}}%
\pgfpathlineto{\pgfqpoint{0.211875in}{0.346520in}}%
\pgfpathlineto{\pgfqpoint{0.225144in}{0.341495in}}%
\pgfpathlineto{\pgfqpoint{0.225442in}{0.341140in}}%
\pgfpathlineto{\pgfqpoint{0.229799in}{0.328214in}}%
\pgfpathlineto{\pgfqpoint{0.230488in}{0.315287in}}%
\pgfpathlineto{\pgfqpoint{0.229521in}{0.302361in}}%
\pgfpathlineto{\pgfqpoint{0.225196in}{0.289434in}}%
\pgfpathlineto{\pgfqpoint{0.225144in}{0.289365in}}%
\pgfpathlineto{\pgfqpoint{0.211875in}{0.283216in}}%
\pgfpathlineto{\pgfqpoint{0.211875in}{0.276508in}}%
\pgfpathlineto{\pgfqpoint{0.211875in}{0.274886in}}%
\pgfpathlineto{\pgfqpoint{0.225144in}{0.275339in}}%
\pgfpathlineto{\pgfqpoint{0.230127in}{0.276508in}}%
\pgfpathlineto{\pgfqpoint{0.235694in}{0.289434in}}%
\pgfpathlineto{\pgfqpoint{0.236260in}{0.302361in}}%
\pgfpathlineto{\pgfqpoint{0.236582in}{0.315287in}}%
\pgfpathlineto{\pgfqpoint{0.237014in}{0.328214in}}%
\pgfpathlineto{\pgfqpoint{0.238245in}{0.341140in}}%
\pgfpathlineto{\pgfqpoint{0.238413in}{0.341901in}}%
\pgfpathlineto{\pgfqpoint{0.251682in}{0.351448in}}%
\pgfpathlineto{\pgfqpoint{0.264951in}{0.351769in}}%
\pgfpathlineto{\pgfqpoint{0.278220in}{0.350570in}}%
\pgfpathlineto{\pgfqpoint{0.290762in}{0.341140in}}%
\pgfpathlineto{\pgfqpoint{0.291489in}{0.338702in}}%
\pgfpathlineto{\pgfqpoint{0.293080in}{0.328214in}}%
\pgfpathlineto{\pgfqpoint{0.293721in}{0.315287in}}%
\pgfpathlineto{\pgfqpoint{0.293880in}{0.302361in}}%
\pgfpathlineto{\pgfqpoint{0.293595in}{0.289434in}}%
\pgfpathlineto{\pgfqpoint{0.291489in}{0.278228in}}%
\pgfpathlineto{\pgfqpoint{0.290030in}{0.276508in}}%
\pgfpathlineto{\pgfqpoint{0.278220in}{0.273624in}}%
\pgfpathlineto{\pgfqpoint{0.264951in}{0.273052in}}%
\pgfpathlineto{\pgfqpoint{0.251682in}{0.272411in}}%
\pgfpathlineto{\pgfqpoint{0.238815in}{0.263581in}}%
\pgfpathlineto{\pgfqpoint{0.238413in}{0.260086in}}%
\pgfpathlineto{\pgfqpoint{0.237920in}{0.250655in}}%
\pgfpathlineto{\pgfqpoint{0.237900in}{0.237728in}}%
\pgfpathlineto{\pgfqpoint{0.238287in}{0.224802in}}%
\pgfpathclose%
\pgfpathmoveto{\pgfqpoint{0.303070in}{0.289434in}}%
\pgfpathlineto{\pgfqpoint{0.300142in}{0.302361in}}%
\pgfpathlineto{\pgfqpoint{0.299412in}{0.315287in}}%
\pgfpathlineto{\pgfqpoint{0.299758in}{0.328214in}}%
\pgfpathlineto{\pgfqpoint{0.302213in}{0.341140in}}%
\pgfpathlineto{\pgfqpoint{0.304758in}{0.345178in}}%
\pgfpathlineto{\pgfqpoint{0.318027in}{0.350255in}}%
\pgfpathlineto{\pgfqpoint{0.331295in}{0.350741in}}%
\pgfpathlineto{\pgfqpoint{0.344564in}{0.348162in}}%
\pgfpathlineto{\pgfqpoint{0.349576in}{0.341140in}}%
\pgfpathlineto{\pgfqpoint{0.351216in}{0.328214in}}%
\pgfpathlineto{\pgfqpoint{0.351365in}{0.315287in}}%
\pgfpathlineto{\pgfqpoint{0.350720in}{0.302361in}}%
\pgfpathlineto{\pgfqpoint{0.348255in}{0.289434in}}%
\pgfpathlineto{\pgfqpoint{0.344564in}{0.283707in}}%
\pgfpathlineto{\pgfqpoint{0.331295in}{0.279179in}}%
\pgfpathlineto{\pgfqpoint{0.318027in}{0.279664in}}%
\pgfpathlineto{\pgfqpoint{0.304758in}{0.286526in}}%
\pgfpathclose%
\pgfpathmoveto{\pgfqpoint{0.358619in}{0.289434in}}%
\pgfpathlineto{\pgfqpoint{0.357833in}{0.295376in}}%
\pgfpathlineto{\pgfqpoint{0.357320in}{0.302361in}}%
\pgfpathlineto{\pgfqpoint{0.357334in}{0.315287in}}%
\pgfpathlineto{\pgfqpoint{0.357833in}{0.322655in}}%
\pgfpathlineto{\pgfqpoint{0.358348in}{0.328214in}}%
\pgfpathlineto{\pgfqpoint{0.362741in}{0.341140in}}%
\pgfpathlineto{\pgfqpoint{0.371102in}{0.346904in}}%
\pgfpathlineto{\pgfqpoint{0.384371in}{0.348100in}}%
\pgfpathlineto{\pgfqpoint{0.397640in}{0.345481in}}%
\pgfpathlineto{\pgfqpoint{0.402857in}{0.341140in}}%
\pgfpathlineto{\pgfqpoint{0.407655in}{0.328214in}}%
\pgfpathlineto{\pgfqpoint{0.408860in}{0.315287in}}%
\pgfpathlineto{\pgfqpoint{0.408753in}{0.302361in}}%
\pgfpathlineto{\pgfqpoint{0.406841in}{0.289434in}}%
\pgfpathlineto{\pgfqpoint{0.397640in}{0.278972in}}%
\pgfpathlineto{\pgfqpoint{0.384371in}{0.276891in}}%
\pgfpathlineto{\pgfqpoint{0.371102in}{0.277533in}}%
\pgfpathclose%
\pgfpathmoveto{\pgfqpoint{0.481321in}{0.289434in}}%
\pgfpathlineto{\pgfqpoint{0.478056in}{0.302361in}}%
\pgfpathlineto{\pgfqpoint{0.477761in}{0.315287in}}%
\pgfpathlineto{\pgfqpoint{0.479451in}{0.328214in}}%
\pgfpathlineto{\pgfqpoint{0.486750in}{0.341140in}}%
\pgfpathlineto{\pgfqpoint{0.490523in}{0.343483in}}%
\pgfpathlineto{\pgfqpoint{0.503792in}{0.344907in}}%
\pgfpathlineto{\pgfqpoint{0.515328in}{0.341140in}}%
\pgfpathlineto{\pgfqpoint{0.517061in}{0.340073in}}%
\pgfpathlineto{\pgfqpoint{0.522784in}{0.328214in}}%
\pgfpathlineto{\pgfqpoint{0.524358in}{0.315287in}}%
\pgfpathlineto{\pgfqpoint{0.524019in}{0.302361in}}%
\pgfpathlineto{\pgfqpoint{0.520835in}{0.289434in}}%
\pgfpathlineto{\pgfqpoint{0.517061in}{0.284623in}}%
\pgfpathlineto{\pgfqpoint{0.503792in}{0.280433in}}%
\pgfpathlineto{\pgfqpoint{0.490523in}{0.281578in}}%
\pgfpathclose%
\pgfpathmoveto{\pgfqpoint{0.245725in}{0.366993in}}%
\pgfpathlineto{\pgfqpoint{0.241283in}{0.379920in}}%
\pgfpathlineto{\pgfqpoint{0.240167in}{0.392846in}}%
\pgfpathlineto{\pgfqpoint{0.240207in}{0.405773in}}%
\pgfpathlineto{\pgfqpoint{0.241733in}{0.418699in}}%
\pgfpathlineto{\pgfqpoint{0.251682in}{0.430680in}}%
\pgfpathlineto{\pgfqpoint{0.257019in}{0.431626in}}%
\pgfpathlineto{\pgfqpoint{0.264951in}{0.432559in}}%
\pgfpathlineto{\pgfqpoint{0.278220in}{0.432297in}}%
\pgfpathlineto{\pgfqpoint{0.281778in}{0.431626in}}%
\pgfpathlineto{\pgfqpoint{0.291489in}{0.422900in}}%
\pgfpathlineto{\pgfqpoint{0.292340in}{0.418699in}}%
\pgfpathlineto{\pgfqpoint{0.293102in}{0.405773in}}%
\pgfpathlineto{\pgfqpoint{0.293020in}{0.392846in}}%
\pgfpathlineto{\pgfqpoint{0.292203in}{0.379920in}}%
\pgfpathlineto{\pgfqpoint{0.291489in}{0.375561in}}%
\pgfpathlineto{\pgfqpoint{0.288484in}{0.366993in}}%
\pgfpathlineto{\pgfqpoint{0.278220in}{0.360410in}}%
\pgfpathlineto{\pgfqpoint{0.264951in}{0.359616in}}%
\pgfpathlineto{\pgfqpoint{0.251682in}{0.362088in}}%
\pgfpathclose%
\pgfpathmoveto{\pgfqpoint{0.301949in}{0.366993in}}%
\pgfpathlineto{\pgfqpoint{0.299360in}{0.379920in}}%
\pgfpathlineto{\pgfqpoint{0.298993in}{0.392846in}}%
\pgfpathlineto{\pgfqpoint{0.299632in}{0.405773in}}%
\pgfpathlineto{\pgfqpoint{0.302303in}{0.418699in}}%
\pgfpathlineto{\pgfqpoint{0.304758in}{0.422802in}}%
\pgfpathlineto{\pgfqpoint{0.318027in}{0.428510in}}%
\pgfpathlineto{\pgfqpoint{0.331295in}{0.428095in}}%
\pgfpathlineto{\pgfqpoint{0.344564in}{0.420610in}}%
\pgfpathlineto{\pgfqpoint{0.345655in}{0.418699in}}%
\pgfpathlineto{\pgfqpoint{0.348735in}{0.405773in}}%
\pgfpathlineto{\pgfqpoint{0.349447in}{0.392846in}}%
\pgfpathlineto{\pgfqpoint{0.348883in}{0.379920in}}%
\pgfpathlineto{\pgfqpoint{0.345524in}{0.366993in}}%
\pgfpathlineto{\pgfqpoint{0.344564in}{0.365584in}}%
\pgfpathlineto{\pgfqpoint{0.331295in}{0.359627in}}%
\pgfpathlineto{\pgfqpoint{0.318027in}{0.359196in}}%
\pgfpathlineto{\pgfqpoint{0.304758in}{0.363132in}}%
\pgfpathclose%
\pgfpathmoveto{\pgfqpoint{0.426848in}{0.366993in}}%
\pgfpathlineto{\pgfqpoint{0.424178in}{0.369492in}}%
\pgfpathlineto{\pgfqpoint{0.420542in}{0.379920in}}%
\pgfpathlineto{\pgfqpoint{0.419644in}{0.392846in}}%
\pgfpathlineto{\pgfqpoint{0.420539in}{0.405773in}}%
\pgfpathlineto{\pgfqpoint{0.424178in}{0.417461in}}%
\pgfpathlineto{\pgfqpoint{0.425164in}{0.418699in}}%
\pgfpathlineto{\pgfqpoint{0.437447in}{0.424946in}}%
\pgfpathlineto{\pgfqpoint{0.450716in}{0.423956in}}%
\pgfpathlineto{\pgfqpoint{0.458201in}{0.418699in}}%
\pgfpathlineto{\pgfqpoint{0.463985in}{0.406600in}}%
\pgfpathlineto{\pgfqpoint{0.464175in}{0.405773in}}%
\pgfpathlineto{\pgfqpoint{0.465091in}{0.392846in}}%
\pgfpathlineto{\pgfqpoint{0.464087in}{0.379920in}}%
\pgfpathlineto{\pgfqpoint{0.463985in}{0.379501in}}%
\pgfpathlineto{\pgfqpoint{0.456543in}{0.366993in}}%
\pgfpathlineto{\pgfqpoint{0.450716in}{0.363744in}}%
\pgfpathlineto{\pgfqpoint{0.437447in}{0.362851in}}%
\pgfpathclose%
\pgfpathmoveto{\pgfqpoint{0.246993in}{0.444552in}}%
\pgfpathlineto{\pgfqpoint{0.241204in}{0.457479in}}%
\pgfpathlineto{\pgfqpoint{0.240276in}{0.470405in}}%
\pgfpathlineto{\pgfqpoint{0.240623in}{0.483332in}}%
\pgfpathlineto{\pgfqpoint{0.242654in}{0.496258in}}%
\pgfpathlineto{\pgfqpoint{0.251682in}{0.507922in}}%
\pgfpathlineto{\pgfqpoint{0.258427in}{0.509185in}}%
\pgfpathlineto{\pgfqpoint{0.264951in}{0.509959in}}%
\pgfpathlineto{\pgfqpoint{0.270173in}{0.509185in}}%
\pgfpathlineto{\pgfqpoint{0.278220in}{0.507445in}}%
\pgfpathlineto{\pgfqpoint{0.287368in}{0.496258in}}%
\pgfpathlineto{\pgfqpoint{0.289838in}{0.483332in}}%
\pgfpathlineto{\pgfqpoint{0.290203in}{0.470405in}}%
\pgfpathlineto{\pgfqpoint{0.288872in}{0.457479in}}%
\pgfpathlineto{\pgfqpoint{0.281678in}{0.444552in}}%
\pgfpathlineto{\pgfqpoint{0.278220in}{0.442612in}}%
\pgfpathlineto{\pgfqpoint{0.264951in}{0.440663in}}%
\pgfpathlineto{\pgfqpoint{0.251682in}{0.441984in}}%
\pgfpathclose%
\pgfpathmoveto{\pgfqpoint{0.249161in}{0.612597in}}%
\pgfpathlineto{\pgfqpoint{0.244166in}{0.625523in}}%
\pgfpathlineto{\pgfqpoint{0.243333in}{0.638450in}}%
\pgfpathlineto{\pgfqpoint{0.244579in}{0.651377in}}%
\pgfpathlineto{\pgfqpoint{0.250438in}{0.664303in}}%
\pgfpathlineto{\pgfqpoint{0.251682in}{0.665416in}}%
\pgfpathlineto{\pgfqpoint{0.264951in}{0.668668in}}%
\pgfpathlineto{\pgfqpoint{0.278220in}{0.665349in}}%
\pgfpathlineto{\pgfqpoint{0.279428in}{0.664303in}}%
\pgfpathlineto{\pgfqpoint{0.285789in}{0.651377in}}%
\pgfpathlineto{\pgfqpoint{0.287187in}{0.638450in}}%
\pgfpathlineto{\pgfqpoint{0.286188in}{0.625523in}}%
\pgfpathlineto{\pgfqpoint{0.280612in}{0.612597in}}%
\pgfpathlineto{\pgfqpoint{0.278220in}{0.610469in}}%
\pgfpathlineto{\pgfqpoint{0.264951in}{0.607326in}}%
\pgfpathlineto{\pgfqpoint{0.251682in}{0.610277in}}%
\pgfpathclose%
\pgfusepath{fill}%
\end{pgfscope}%
\begin{pgfscope}%
\pgfpathrectangle{\pgfqpoint{0.211875in}{0.211875in}}{\pgfqpoint{1.313625in}{1.279725in}}%
\pgfusepath{clip}%
\pgfsetbuttcap%
\pgfsetroundjoin%
\definecolor{currentfill}{rgb}{0.644838,0.098089,0.355336}%
\pgfsetfillcolor{currentfill}%
\pgfsetlinewidth{0.000000pt}%
\definecolor{currentstroke}{rgb}{0.000000,0.000000,0.000000}%
\pgfsetstrokecolor{currentstroke}%
\pgfsetdash{}{0pt}%
\pgfpathmoveto{\pgfqpoint{0.649750in}{0.211875in}}%
\pgfpathlineto{\pgfqpoint{0.657978in}{0.211875in}}%
\pgfpathlineto{\pgfqpoint{0.655903in}{0.224802in}}%
\pgfpathlineto{\pgfqpoint{0.656105in}{0.237728in}}%
\pgfpathlineto{\pgfqpoint{0.658631in}{0.250655in}}%
\pgfpathlineto{\pgfqpoint{0.663019in}{0.257730in}}%
\pgfpathlineto{\pgfqpoint{0.676288in}{0.262761in}}%
\pgfpathlineto{\pgfqpoint{0.689557in}{0.260167in}}%
\pgfpathlineto{\pgfqpoint{0.697452in}{0.250655in}}%
\pgfpathlineto{\pgfqpoint{0.700432in}{0.237728in}}%
\pgfpathlineto{\pgfqpoint{0.700639in}{0.224802in}}%
\pgfpathlineto{\pgfqpoint{0.698095in}{0.211875in}}%
\pgfpathlineto{\pgfqpoint{0.702826in}{0.211875in}}%
\pgfpathlineto{\pgfqpoint{0.707092in}{0.211875in}}%
\pgfpathlineto{\pgfqpoint{0.707004in}{0.224802in}}%
\pgfpathlineto{\pgfqpoint{0.706694in}{0.237728in}}%
\pgfpathlineto{\pgfqpoint{0.705888in}{0.250655in}}%
\pgfpathlineto{\pgfqpoint{0.702826in}{0.263248in}}%
\pgfpathlineto{\pgfqpoint{0.702619in}{0.263581in}}%
\pgfpathlineto{\pgfqpoint{0.689557in}{0.269867in}}%
\pgfpathlineto{\pgfqpoint{0.676288in}{0.270592in}}%
\pgfpathlineto{\pgfqpoint{0.663019in}{0.269830in}}%
\pgfpathlineto{\pgfqpoint{0.652904in}{0.263581in}}%
\pgfpathlineto{\pgfqpoint{0.650331in}{0.250655in}}%
\pgfpathlineto{\pgfqpoint{0.649750in}{0.239813in}}%
\pgfpathlineto{\pgfqpoint{0.649666in}{0.237728in}}%
\pgfpathlineto{\pgfqpoint{0.649312in}{0.224802in}}%
\pgfpathlineto{\pgfqpoint{0.648909in}{0.211875in}}%
\pgfpathclose%
\pgfusepath{fill}%
\end{pgfscope}%
\begin{pgfscope}%
\pgfpathrectangle{\pgfqpoint{0.211875in}{0.211875in}}{\pgfqpoint{1.313625in}{1.279725in}}%
\pgfusepath{clip}%
\pgfsetbuttcap%
\pgfsetroundjoin%
\definecolor{currentfill}{rgb}{0.644838,0.098089,0.355336}%
\pgfsetfillcolor{currentfill}%
\pgfsetlinewidth{0.000000pt}%
\definecolor{currentstroke}{rgb}{0.000000,0.000000,0.000000}%
\pgfsetstrokecolor{currentstroke}%
\pgfsetdash{}{0pt}%
\pgfpathmoveto{\pgfqpoint{0.769170in}{0.211875in}}%
\pgfpathlineto{\pgfqpoint{0.778256in}{0.211875in}}%
\pgfpathlineto{\pgfqpoint{0.775490in}{0.224802in}}%
\pgfpathlineto{\pgfqpoint{0.775658in}{0.237728in}}%
\pgfpathlineto{\pgfqpoint{0.778734in}{0.250655in}}%
\pgfpathlineto{\pgfqpoint{0.782439in}{0.256012in}}%
\pgfpathlineto{\pgfqpoint{0.795708in}{0.260674in}}%
\pgfpathlineto{\pgfqpoint{0.808977in}{0.256676in}}%
\pgfpathlineto{\pgfqpoint{0.813484in}{0.250655in}}%
\pgfpathlineto{\pgfqpoint{0.816738in}{0.237728in}}%
\pgfpathlineto{\pgfqpoint{0.816898in}{0.224802in}}%
\pgfpathlineto{\pgfqpoint{0.813920in}{0.211875in}}%
\pgfpathlineto{\pgfqpoint{0.822246in}{0.211875in}}%
\pgfpathlineto{\pgfqpoint{0.823270in}{0.211875in}}%
\pgfpathlineto{\pgfqpoint{0.823653in}{0.224802in}}%
\pgfpathlineto{\pgfqpoint{0.823370in}{0.237728in}}%
\pgfpathlineto{\pgfqpoint{0.822246in}{0.250242in}}%
\pgfpathlineto{\pgfqpoint{0.822191in}{0.250655in}}%
\pgfpathlineto{\pgfqpoint{0.816362in}{0.263581in}}%
\pgfpathlineto{\pgfqpoint{0.808977in}{0.267499in}}%
\pgfpathlineto{\pgfqpoint{0.795708in}{0.268819in}}%
\pgfpathlineto{\pgfqpoint{0.782439in}{0.267587in}}%
\pgfpathlineto{\pgfqpoint{0.775237in}{0.263581in}}%
\pgfpathlineto{\pgfqpoint{0.770156in}{0.250655in}}%
\pgfpathlineto{\pgfqpoint{0.769170in}{0.240130in}}%
\pgfpathlineto{\pgfqpoint{0.769010in}{0.237728in}}%
\pgfpathlineto{\pgfqpoint{0.768703in}{0.224802in}}%
\pgfpathlineto{\pgfqpoint{0.768890in}{0.211875in}}%
\pgfpathclose%
\pgfusepath{fill}%
\end{pgfscope}%
\begin{pgfscope}%
\pgfpathrectangle{\pgfqpoint{0.211875in}{0.211875in}}{\pgfqpoint{1.313625in}{1.279725in}}%
\pgfusepath{clip}%
\pgfsetbuttcap%
\pgfsetroundjoin%
\definecolor{currentfill}{rgb}{0.644838,0.098089,0.355336}%
\pgfsetfillcolor{currentfill}%
\pgfsetlinewidth{0.000000pt}%
\definecolor{currentstroke}{rgb}{0.000000,0.000000,0.000000}%
\pgfsetstrokecolor{currentstroke}%
\pgfsetdash{}{0pt}%
\pgfpathmoveto{\pgfqpoint{0.888591in}{0.211875in}}%
\pgfpathlineto{\pgfqpoint{0.898019in}{0.211875in}}%
\pgfpathlineto{\pgfqpoint{0.894670in}{0.224802in}}%
\pgfpathlineto{\pgfqpoint{0.894815in}{0.237728in}}%
\pgfpathlineto{\pgfqpoint{0.898369in}{0.250655in}}%
\pgfpathlineto{\pgfqpoint{0.901860in}{0.255195in}}%
\pgfpathlineto{\pgfqpoint{0.915129in}{0.259133in}}%
\pgfpathlineto{\pgfqpoint{0.928398in}{0.253453in}}%
\pgfpathlineto{\pgfqpoint{0.930283in}{0.250655in}}%
\pgfpathlineto{\pgfqpoint{0.933693in}{0.237728in}}%
\pgfpathlineto{\pgfqpoint{0.933822in}{0.224802in}}%
\pgfpathlineto{\pgfqpoint{0.930583in}{0.211875in}}%
\pgfpathlineto{\pgfqpoint{0.939727in}{0.211875in}}%
\pgfpathlineto{\pgfqpoint{0.940518in}{0.224802in}}%
\pgfpathlineto{\pgfqpoint{0.940228in}{0.237728in}}%
\pgfpathlineto{\pgfqpoint{0.938653in}{0.250655in}}%
\pgfpathlineto{\pgfqpoint{0.931639in}{0.263581in}}%
\pgfpathlineto{\pgfqpoint{0.928398in}{0.265481in}}%
\pgfpathlineto{\pgfqpoint{0.915129in}{0.267546in}}%
\pgfpathlineto{\pgfqpoint{0.901860in}{0.266259in}}%
\pgfpathlineto{\pgfqpoint{0.896529in}{0.263581in}}%
\pgfpathlineto{\pgfqpoint{0.889434in}{0.250655in}}%
\pgfpathlineto{\pgfqpoint{0.888591in}{0.244506in}}%
\pgfpathlineto{\pgfqpoint{0.887968in}{0.237728in}}%
\pgfpathlineto{\pgfqpoint{0.887694in}{0.224802in}}%
\pgfpathlineto{\pgfqpoint{0.888288in}{0.211875in}}%
\pgfpathclose%
\pgfusepath{fill}%
\end{pgfscope}%
\begin{pgfscope}%
\pgfpathrectangle{\pgfqpoint{0.211875in}{0.211875in}}{\pgfqpoint{1.313625in}{1.279725in}}%
\pgfusepath{clip}%
\pgfsetbuttcap%
\pgfsetroundjoin%
\definecolor{currentfill}{rgb}{0.644838,0.098089,0.355336}%
\pgfsetfillcolor{currentfill}%
\pgfsetlinewidth{0.000000pt}%
\definecolor{currentstroke}{rgb}{0.000000,0.000000,0.000000}%
\pgfsetstrokecolor{currentstroke}%
\pgfsetdash{}{0pt}%
\pgfpathmoveto{\pgfqpoint{1.008011in}{0.211875in}}%
\pgfpathlineto{\pgfqpoint{1.017215in}{0.211875in}}%
\pgfpathlineto{\pgfqpoint{1.013394in}{0.224802in}}%
\pgfpathlineto{\pgfqpoint{1.013526in}{0.237728in}}%
\pgfpathlineto{\pgfqpoint{1.017488in}{0.250655in}}%
\pgfpathlineto{\pgfqpoint{1.021280in}{0.255096in}}%
\pgfpathlineto{\pgfqpoint{1.034549in}{0.258110in}}%
\pgfpathlineto{\pgfqpoint{1.047631in}{0.250655in}}%
\pgfpathlineto{\pgfqpoint{1.047818in}{0.250382in}}%
\pgfpathlineto{\pgfqpoint{1.051184in}{0.237728in}}%
\pgfpathlineto{\pgfqpoint{1.051295in}{0.224802in}}%
\pgfpathlineto{\pgfqpoint{1.047948in}{0.211875in}}%
\pgfpathlineto{\pgfqpoint{1.056808in}{0.211875in}}%
\pgfpathlineto{\pgfqpoint{1.057775in}{0.224802in}}%
\pgfpathlineto{\pgfqpoint{1.057507in}{0.237728in}}%
\pgfpathlineto{\pgfqpoint{1.055832in}{0.250655in}}%
\pgfpathlineto{\pgfqpoint{1.048236in}{0.263581in}}%
\pgfpathlineto{\pgfqpoint{1.047818in}{0.263853in}}%
\pgfpathlineto{\pgfqpoint{1.034549in}{0.266748in}}%
\pgfpathlineto{\pgfqpoint{1.021280in}{0.265659in}}%
\pgfpathlineto{\pgfqpoint{1.016711in}{0.263581in}}%
\pgfpathlineto{\pgfqpoint{1.008101in}{0.250655in}}%
\pgfpathlineto{\pgfqpoint{1.008011in}{0.250170in}}%
\pgfpathlineto{\pgfqpoint{1.006563in}{0.237728in}}%
\pgfpathlineto{\pgfqpoint{1.006308in}{0.224802in}}%
\pgfpathlineto{\pgfqpoint{1.007150in}{0.211875in}}%
\pgfpathclose%
\pgfusepath{fill}%
\end{pgfscope}%
\begin{pgfscope}%
\pgfpathrectangle{\pgfqpoint{0.211875in}{0.211875in}}{\pgfqpoint{1.313625in}{1.279725in}}%
\pgfusepath{clip}%
\pgfsetbuttcap%
\pgfsetroundjoin%
\definecolor{currentfill}{rgb}{0.644838,0.098089,0.355336}%
\pgfsetfillcolor{currentfill}%
\pgfsetlinewidth{0.000000pt}%
\definecolor{currentstroke}{rgb}{0.000000,0.000000,0.000000}%
\pgfsetstrokecolor{currentstroke}%
\pgfsetdash{}{0pt}%
\pgfpathmoveto{\pgfqpoint{1.127432in}{0.211875in}}%
\pgfpathlineto{\pgfqpoint{1.135760in}{0.211875in}}%
\pgfpathlineto{\pgfqpoint{1.131582in}{0.224802in}}%
\pgfpathlineto{\pgfqpoint{1.131714in}{0.237728in}}%
\pgfpathlineto{\pgfqpoint{1.136011in}{0.250655in}}%
\pgfpathlineto{\pgfqpoint{1.140701in}{0.255598in}}%
\pgfpathlineto{\pgfqpoint{1.153970in}{0.257596in}}%
\pgfpathlineto{\pgfqpoint{1.164497in}{0.250655in}}%
\pgfpathlineto{\pgfqpoint{1.167239in}{0.245690in}}%
\pgfpathlineto{\pgfqpoint{1.169135in}{0.237728in}}%
\pgfpathlineto{\pgfqpoint{1.169240in}{0.224802in}}%
\pgfpathlineto{\pgfqpoint{1.167239in}{0.216074in}}%
\pgfpathlineto{\pgfqpoint{1.164857in}{0.211875in}}%
\pgfpathlineto{\pgfqpoint{1.167239in}{0.211875in}}%
\pgfpathlineto{\pgfqpoint{1.174569in}{0.211875in}}%
\pgfpathlineto{\pgfqpoint{1.175555in}{0.224802in}}%
\pgfpathlineto{\pgfqpoint{1.175298in}{0.237728in}}%
\pgfpathlineto{\pgfqpoint{1.173630in}{0.250655in}}%
\pgfpathlineto{\pgfqpoint{1.167239in}{0.262504in}}%
\pgfpathlineto{\pgfqpoint{1.165014in}{0.263581in}}%
\pgfpathlineto{\pgfqpoint{1.153970in}{0.266421in}}%
\pgfpathlineto{\pgfqpoint{1.140701in}{0.265659in}}%
\pgfpathlineto{\pgfqpoint{1.135651in}{0.263581in}}%
\pgfpathlineto{\pgfqpoint{1.127432in}{0.254041in}}%
\pgfpathlineto{\pgfqpoint{1.126390in}{0.250655in}}%
\pgfpathlineto{\pgfqpoint{1.124801in}{0.237728in}}%
\pgfpathlineto{\pgfqpoint{1.124555in}{0.224802in}}%
\pgfpathlineto{\pgfqpoint{1.125495in}{0.211875in}}%
\pgfpathclose%
\pgfusepath{fill}%
\end{pgfscope}%
\begin{pgfscope}%
\pgfpathrectangle{\pgfqpoint{0.211875in}{0.211875in}}{\pgfqpoint{1.313625in}{1.279725in}}%
\pgfusepath{clip}%
\pgfsetbuttcap%
\pgfsetroundjoin%
\definecolor{currentfill}{rgb}{0.644838,0.098089,0.355336}%
\pgfsetfillcolor{currentfill}%
\pgfsetlinewidth{0.000000pt}%
\definecolor{currentstroke}{rgb}{0.000000,0.000000,0.000000}%
\pgfsetstrokecolor{currentstroke}%
\pgfsetdash{}{0pt}%
\pgfpathmoveto{\pgfqpoint{1.246852in}{0.211875in}}%
\pgfpathlineto{\pgfqpoint{1.253519in}{0.211875in}}%
\pgfpathlineto{\pgfqpoint{1.249113in}{0.224802in}}%
\pgfpathlineto{\pgfqpoint{1.249259in}{0.237728in}}%
\pgfpathlineto{\pgfqpoint{1.253812in}{0.250655in}}%
\pgfpathlineto{\pgfqpoint{1.260121in}{0.256624in}}%
\pgfpathlineto{\pgfqpoint{1.273390in}{0.257603in}}%
\pgfpathlineto{\pgfqpoint{1.282636in}{0.250655in}}%
\pgfpathlineto{\pgfqpoint{1.286659in}{0.241658in}}%
\pgfpathlineto{\pgfqpoint{1.287491in}{0.237728in}}%
\pgfpathlineto{\pgfqpoint{1.287601in}{0.224802in}}%
\pgfpathlineto{\pgfqpoint{1.286659in}{0.220049in}}%
\pgfpathlineto{\pgfqpoint{1.283007in}{0.211875in}}%
\pgfpathlineto{\pgfqpoint{1.286659in}{0.211875in}}%
\pgfpathlineto{\pgfqpoint{1.292938in}{0.211875in}}%
\pgfpathlineto{\pgfqpoint{1.293800in}{0.224802in}}%
\pgfpathlineto{\pgfqpoint{1.293540in}{0.237728in}}%
\pgfpathlineto{\pgfqpoint{1.291982in}{0.250655in}}%
\pgfpathlineto{\pgfqpoint{1.286659in}{0.261812in}}%
\pgfpathlineto{\pgfqpoint{1.283693in}{0.263581in}}%
\pgfpathlineto{\pgfqpoint{1.273390in}{0.266577in}}%
\pgfpathlineto{\pgfqpoint{1.260121in}{0.266179in}}%
\pgfpathlineto{\pgfqpoint{1.253133in}{0.263581in}}%
\pgfpathlineto{\pgfqpoint{1.246852in}{0.257821in}}%
\pgfpathlineto{\pgfqpoint{1.244233in}{0.250655in}}%
\pgfpathlineto{\pgfqpoint{1.242676in}{0.237728in}}%
\pgfpathlineto{\pgfqpoint{1.242427in}{0.224802in}}%
\pgfpathlineto{\pgfqpoint{1.243325in}{0.211875in}}%
\pgfpathclose%
\pgfusepath{fill}%
\end{pgfscope}%
\begin{pgfscope}%
\pgfpathrectangle{\pgfqpoint{0.211875in}{0.211875in}}{\pgfqpoint{1.313625in}{1.279725in}}%
\pgfusepath{clip}%
\pgfsetbuttcap%
\pgfsetroundjoin%
\definecolor{currentfill}{rgb}{0.644838,0.098089,0.355336}%
\pgfsetfillcolor{currentfill}%
\pgfsetlinewidth{0.000000pt}%
\definecolor{currentstroke}{rgb}{0.000000,0.000000,0.000000}%
\pgfsetstrokecolor{currentstroke}%
\pgfsetdash{}{0pt}%
\pgfpathmoveto{\pgfqpoint{1.366273in}{0.211875in}}%
\pgfpathlineto{\pgfqpoint{1.370283in}{0.211875in}}%
\pgfpathlineto{\pgfqpoint{1.366273in}{0.223016in}}%
\pgfpathlineto{\pgfqpoint{1.365965in}{0.224802in}}%
\pgfpathlineto{\pgfqpoint{1.366081in}{0.237728in}}%
\pgfpathlineto{\pgfqpoint{1.366273in}{0.238736in}}%
\pgfpathlineto{\pgfqpoint{1.370698in}{0.250655in}}%
\pgfpathlineto{\pgfqpoint{1.379542in}{0.258130in}}%
\pgfpathlineto{\pgfqpoint{1.392811in}{0.258163in}}%
\pgfpathlineto{\pgfqpoint{1.401669in}{0.250655in}}%
\pgfpathlineto{\pgfqpoint{1.406080in}{0.238476in}}%
\pgfpathlineto{\pgfqpoint{1.406219in}{0.237728in}}%
\pgfpathlineto{\pgfqpoint{1.406343in}{0.224802in}}%
\pgfpathlineto{\pgfqpoint{1.406080in}{0.223228in}}%
\pgfpathlineto{\pgfqpoint{1.402124in}{0.211875in}}%
\pgfpathlineto{\pgfqpoint{1.406080in}{0.211875in}}%
\pgfpathlineto{\pgfqpoint{1.411875in}{0.211875in}}%
\pgfpathlineto{\pgfqpoint{1.412469in}{0.224802in}}%
\pgfpathlineto{\pgfqpoint{1.412195in}{0.237728in}}%
\pgfpathlineto{\pgfqpoint{1.410848in}{0.250655in}}%
\pgfpathlineto{\pgfqpoint{1.406080in}{0.262096in}}%
\pgfpathlineto{\pgfqpoint{1.404065in}{0.263581in}}%
\pgfpathlineto{\pgfqpoint{1.392811in}{0.267249in}}%
\pgfpathlineto{\pgfqpoint{1.379542in}{0.267168in}}%
\pgfpathlineto{\pgfqpoint{1.368822in}{0.263581in}}%
\pgfpathlineto{\pgfqpoint{1.366273in}{0.261698in}}%
\pgfpathlineto{\pgfqpoint{1.361592in}{0.250655in}}%
\pgfpathlineto{\pgfqpoint{1.360166in}{0.237728in}}%
\pgfpathlineto{\pgfqpoint{1.359904in}{0.224802in}}%
\pgfpathlineto{\pgfqpoint{1.360619in}{0.211875in}}%
\pgfpathclose%
\pgfusepath{fill}%
\end{pgfscope}%
\begin{pgfscope}%
\pgfpathrectangle{\pgfqpoint{0.211875in}{0.211875in}}{\pgfqpoint{1.313625in}{1.279725in}}%
\pgfusepath{clip}%
\pgfsetbuttcap%
\pgfsetroundjoin%
\definecolor{currentfill}{rgb}{0.644838,0.098089,0.355336}%
\pgfsetfillcolor{currentfill}%
\pgfsetlinewidth{0.000000pt}%
\definecolor{currentstroke}{rgb}{0.000000,0.000000,0.000000}%
\pgfsetstrokecolor{currentstroke}%
\pgfsetdash{}{0pt}%
\pgfpathmoveto{\pgfqpoint{1.485693in}{0.211875in}}%
\pgfpathlineto{\pgfqpoint{1.485730in}{0.211875in}}%
\pgfpathlineto{\pgfqpoint{1.485693in}{0.211956in}}%
\pgfpathlineto{\pgfqpoint{1.483090in}{0.224802in}}%
\pgfpathlineto{\pgfqpoint{1.483231in}{0.237728in}}%
\pgfpathlineto{\pgfqpoint{1.485693in}{0.249179in}}%
\pgfpathlineto{\pgfqpoint{1.486368in}{0.250655in}}%
\pgfpathlineto{\pgfqpoint{1.498962in}{0.260098in}}%
\pgfpathlineto{\pgfqpoint{1.512231in}{0.259336in}}%
\pgfpathlineto{\pgfqpoint{1.521377in}{0.250655in}}%
\pgfpathlineto{\pgfqpoint{1.525220in}{0.237728in}}%
\pgfpathlineto{\pgfqpoint{1.525431in}{0.224802in}}%
\pgfpathlineto{\pgfqpoint{1.521974in}{0.211875in}}%
\pgfpathlineto{\pgfqpoint{1.525500in}{0.211875in}}%
\pgfpathlineto{\pgfqpoint{1.525500in}{0.224802in}}%
\pgfpathlineto{\pgfqpoint{1.525500in}{0.237728in}}%
\pgfpathlineto{\pgfqpoint{1.525500in}{0.250655in}}%
\pgfpathlineto{\pgfqpoint{1.525500in}{0.263581in}}%
\pgfpathlineto{\pgfqpoint{1.525500in}{0.263782in}}%
\pgfpathlineto{\pgfqpoint{1.512231in}{0.268493in}}%
\pgfpathlineto{\pgfqpoint{1.498962in}{0.268601in}}%
\pgfpathlineto{\pgfqpoint{1.485693in}{0.265328in}}%
\pgfpathlineto{\pgfqpoint{1.483570in}{0.263581in}}%
\pgfpathlineto{\pgfqpoint{1.478428in}{0.250655in}}%
\pgfpathlineto{\pgfqpoint{1.477236in}{0.237728in}}%
\pgfpathlineto{\pgfqpoint{1.476948in}{0.224802in}}%
\pgfpathlineto{\pgfqpoint{1.477337in}{0.211875in}}%
\pgfpathclose%
\pgfusepath{fill}%
\end{pgfscope}%
\begin{pgfscope}%
\pgfpathrectangle{\pgfqpoint{0.211875in}{0.211875in}}{\pgfqpoint{1.313625in}{1.279725in}}%
\pgfusepath{clip}%
\pgfsetbuttcap%
\pgfsetroundjoin%
\definecolor{currentfill}{rgb}{0.644838,0.098089,0.355336}%
\pgfsetfillcolor{currentfill}%
\pgfsetlinewidth{0.000000pt}%
\definecolor{currentstroke}{rgb}{0.000000,0.000000,0.000000}%
\pgfsetstrokecolor{currentstroke}%
\pgfsetdash{}{0pt}%
\pgfpathmoveto{\pgfqpoint{0.596674in}{0.276326in}}%
\pgfpathlineto{\pgfqpoint{0.609943in}{0.274593in}}%
\pgfpathlineto{\pgfqpoint{0.623212in}{0.274694in}}%
\pgfpathlineto{\pgfqpoint{0.636481in}{0.276133in}}%
\pgfpathlineto{\pgfqpoint{0.637641in}{0.276508in}}%
\pgfpathlineto{\pgfqpoint{0.645730in}{0.289434in}}%
\pgfpathlineto{\pgfqpoint{0.646500in}{0.302361in}}%
\pgfpathlineto{\pgfqpoint{0.646416in}{0.315287in}}%
\pgfpathlineto{\pgfqpoint{0.645598in}{0.328214in}}%
\pgfpathlineto{\pgfqpoint{0.642555in}{0.341140in}}%
\pgfpathlineto{\pgfqpoint{0.636481in}{0.347418in}}%
\pgfpathlineto{\pgfqpoint{0.623212in}{0.350145in}}%
\pgfpathlineto{\pgfqpoint{0.609943in}{0.349932in}}%
\pgfpathlineto{\pgfqpoint{0.596674in}{0.344449in}}%
\pgfpathlineto{\pgfqpoint{0.594819in}{0.341140in}}%
\pgfpathlineto{\pgfqpoint{0.592433in}{0.328214in}}%
\pgfpathlineto{\pgfqpoint{0.591771in}{0.315287in}}%
\pgfpathlineto{\pgfqpoint{0.591606in}{0.302361in}}%
\pgfpathlineto{\pgfqpoint{0.591895in}{0.289434in}}%
\pgfpathlineto{\pgfqpoint{0.596321in}{0.276508in}}%
\pgfpathclose%
\pgfpathmoveto{\pgfqpoint{0.603572in}{0.289434in}}%
\pgfpathlineto{\pgfqpoint{0.598510in}{0.302361in}}%
\pgfpathlineto{\pgfqpoint{0.597919in}{0.315287in}}%
\pgfpathlineto{\pgfqpoint{0.600194in}{0.328214in}}%
\pgfpathlineto{\pgfqpoint{0.609943in}{0.340883in}}%
\pgfpathlineto{\pgfqpoint{0.611917in}{0.341140in}}%
\pgfpathlineto{\pgfqpoint{0.623212in}{0.342137in}}%
\pgfpathlineto{\pgfqpoint{0.625878in}{0.341140in}}%
\pgfpathlineto{\pgfqpoint{0.636481in}{0.333076in}}%
\pgfpathlineto{\pgfqpoint{0.638576in}{0.328214in}}%
\pgfpathlineto{\pgfqpoint{0.640442in}{0.315287in}}%
\pgfpathlineto{\pgfqpoint{0.639920in}{0.302361in}}%
\pgfpathlineto{\pgfqpoint{0.636481in}{0.291067in}}%
\pgfpathlineto{\pgfqpoint{0.635128in}{0.289434in}}%
\pgfpathlineto{\pgfqpoint{0.623212in}{0.283492in}}%
\pgfpathlineto{\pgfqpoint{0.609943in}{0.284561in}}%
\pgfpathclose%
\pgfusepath{fill}%
\end{pgfscope}%
\begin{pgfscope}%
\pgfpathrectangle{\pgfqpoint{0.211875in}{0.211875in}}{\pgfqpoint{1.313625in}{1.279725in}}%
\pgfusepath{clip}%
\pgfsetbuttcap%
\pgfsetroundjoin%
\definecolor{currentfill}{rgb}{0.644838,0.098089,0.355336}%
\pgfsetfillcolor{currentfill}%
\pgfsetlinewidth{0.000000pt}%
\definecolor{currentstroke}{rgb}{0.000000,0.000000,0.000000}%
\pgfsetstrokecolor{currentstroke}%
\pgfsetdash{}{0pt}%
\pgfpathmoveto{\pgfqpoint{0.716095in}{0.281803in}}%
\pgfpathlineto{\pgfqpoint{0.729364in}{0.277016in}}%
\pgfpathlineto{\pgfqpoint{0.742633in}{0.277004in}}%
\pgfpathlineto{\pgfqpoint{0.755902in}{0.280676in}}%
\pgfpathlineto{\pgfqpoint{0.761215in}{0.289434in}}%
\pgfpathlineto{\pgfqpoint{0.762811in}{0.302361in}}%
\pgfpathlineto{\pgfqpoint{0.762877in}{0.315287in}}%
\pgfpathlineto{\pgfqpoint{0.761817in}{0.328214in}}%
\pgfpathlineto{\pgfqpoint{0.757591in}{0.341140in}}%
\pgfpathlineto{\pgfqpoint{0.755902in}{0.343110in}}%
\pgfpathlineto{\pgfqpoint{0.742633in}{0.347912in}}%
\pgfpathlineto{\pgfqpoint{0.729364in}{0.347759in}}%
\pgfpathlineto{\pgfqpoint{0.716183in}{0.341140in}}%
\pgfpathlineto{\pgfqpoint{0.716095in}{0.341019in}}%
\pgfpathlineto{\pgfqpoint{0.712334in}{0.328214in}}%
\pgfpathlineto{\pgfqpoint{0.711378in}{0.315287in}}%
\pgfpathlineto{\pgfqpoint{0.711383in}{0.302361in}}%
\pgfpathlineto{\pgfqpoint{0.712652in}{0.289434in}}%
\pgfpathclose%
\pgfpathmoveto{\pgfqpoint{0.725363in}{0.289434in}}%
\pgfpathlineto{\pgfqpoint{0.718565in}{0.302361in}}%
\pgfpathlineto{\pgfqpoint{0.717693in}{0.315287in}}%
\pgfpathlineto{\pgfqpoint{0.720546in}{0.328214in}}%
\pgfpathlineto{\pgfqpoint{0.729364in}{0.338481in}}%
\pgfpathlineto{\pgfqpoint{0.742633in}{0.339311in}}%
\pgfpathlineto{\pgfqpoint{0.754299in}{0.328214in}}%
\pgfpathlineto{\pgfqpoint{0.755902in}{0.322696in}}%
\pgfpathlineto{\pgfqpoint{0.757016in}{0.315287in}}%
\pgfpathlineto{\pgfqpoint{0.756354in}{0.302361in}}%
\pgfpathlineto{\pgfqpoint{0.755902in}{0.300652in}}%
\pgfpathlineto{\pgfqpoint{0.748502in}{0.289434in}}%
\pgfpathlineto{\pgfqpoint{0.742633in}{0.286111in}}%
\pgfpathlineto{\pgfqpoint{0.729364in}{0.286697in}}%
\pgfpathclose%
\pgfusepath{fill}%
\end{pgfscope}%
\begin{pgfscope}%
\pgfpathrectangle{\pgfqpoint{0.211875in}{0.211875in}}{\pgfqpoint{1.313625in}{1.279725in}}%
\pgfusepath{clip}%
\pgfsetbuttcap%
\pgfsetroundjoin%
\definecolor{currentfill}{rgb}{0.644838,0.098089,0.355336}%
\pgfsetfillcolor{currentfill}%
\pgfsetlinewidth{0.000000pt}%
\definecolor{currentstroke}{rgb}{0.000000,0.000000,0.000000}%
\pgfsetstrokecolor{currentstroke}%
\pgfsetdash{}{0pt}%
\pgfpathmoveto{\pgfqpoint{0.835515in}{0.284323in}}%
\pgfpathlineto{\pgfqpoint{0.848784in}{0.278781in}}%
\pgfpathlineto{\pgfqpoint{0.862053in}{0.278959in}}%
\pgfpathlineto{\pgfqpoint{0.875322in}{0.285400in}}%
\pgfpathlineto{\pgfqpoint{0.877426in}{0.289434in}}%
\pgfpathlineto{\pgfqpoint{0.879622in}{0.302361in}}%
\pgfpathlineto{\pgfqpoint{0.879795in}{0.315287in}}%
\pgfpathlineto{\pgfqpoint{0.878561in}{0.328214in}}%
\pgfpathlineto{\pgfqpoint{0.875322in}{0.337915in}}%
\pgfpathlineto{\pgfqpoint{0.872529in}{0.341140in}}%
\pgfpathlineto{\pgfqpoint{0.862053in}{0.346102in}}%
\pgfpathlineto{\pgfqpoint{0.848784in}{0.346262in}}%
\pgfpathlineto{\pgfqpoint{0.837378in}{0.341140in}}%
\pgfpathlineto{\pgfqpoint{0.835515in}{0.339106in}}%
\pgfpathlineto{\pgfqpoint{0.831797in}{0.328214in}}%
\pgfpathlineto{\pgfqpoint{0.830610in}{0.315287in}}%
\pgfpathlineto{\pgfqpoint{0.830745in}{0.302361in}}%
\pgfpathlineto{\pgfqpoint{0.832765in}{0.289434in}}%
\pgfpathclose%
\pgfpathmoveto{\pgfqpoint{0.846648in}{0.289434in}}%
\pgfpathlineto{\pgfqpoint{0.838154in}{0.302361in}}%
\pgfpathlineto{\pgfqpoint{0.837011in}{0.315287in}}%
\pgfpathlineto{\pgfqpoint{0.840449in}{0.328214in}}%
\pgfpathlineto{\pgfqpoint{0.848784in}{0.336867in}}%
\pgfpathlineto{\pgfqpoint{0.862053in}{0.336649in}}%
\pgfpathlineto{\pgfqpoint{0.869923in}{0.328214in}}%
\pgfpathlineto{\pgfqpoint{0.873335in}{0.315287in}}%
\pgfpathlineto{\pgfqpoint{0.872179in}{0.302361in}}%
\pgfpathlineto{\pgfqpoint{0.863807in}{0.289434in}}%
\pgfpathlineto{\pgfqpoint{0.862053in}{0.288316in}}%
\pgfpathlineto{\pgfqpoint{0.848784in}{0.288133in}}%
\pgfpathclose%
\pgfusepath{fill}%
\end{pgfscope}%
\begin{pgfscope}%
\pgfpathrectangle{\pgfqpoint{0.211875in}{0.211875in}}{\pgfqpoint{1.313625in}{1.279725in}}%
\pgfusepath{clip}%
\pgfsetbuttcap%
\pgfsetroundjoin%
\definecolor{currentfill}{rgb}{0.644838,0.098089,0.355336}%
\pgfsetfillcolor{currentfill}%
\pgfsetlinewidth{0.000000pt}%
\definecolor{currentstroke}{rgb}{0.000000,0.000000,0.000000}%
\pgfsetstrokecolor{currentstroke}%
\pgfsetdash{}{0pt}%
\pgfpathmoveto{\pgfqpoint{0.954936in}{0.285144in}}%
\pgfpathlineto{\pgfqpoint{0.968205in}{0.279866in}}%
\pgfpathlineto{\pgfqpoint{0.981473in}{0.280427in}}%
\pgfpathlineto{\pgfqpoint{0.994107in}{0.289434in}}%
\pgfpathlineto{\pgfqpoint{0.994742in}{0.291012in}}%
\pgfpathlineto{\pgfqpoint{0.996874in}{0.302361in}}%
\pgfpathlineto{\pgfqpoint{0.997116in}{0.315287in}}%
\pgfpathlineto{\pgfqpoint{0.995768in}{0.328214in}}%
\pgfpathlineto{\pgfqpoint{0.994742in}{0.331785in}}%
\pgfpathlineto{\pgfqpoint{0.988247in}{0.341140in}}%
\pgfpathlineto{\pgfqpoint{0.981473in}{0.344714in}}%
\pgfpathlineto{\pgfqpoint{0.968205in}{0.345349in}}%
\pgfpathlineto{\pgfqpoint{0.957647in}{0.341140in}}%
\pgfpathlineto{\pgfqpoint{0.954936in}{0.338752in}}%
\pgfpathlineto{\pgfqpoint{0.950824in}{0.328214in}}%
\pgfpathlineto{\pgfqpoint{0.949467in}{0.315287in}}%
\pgfpathlineto{\pgfqpoint{0.949694in}{0.302361in}}%
\pgfpathlineto{\pgfqpoint{0.952251in}{0.289434in}}%
\pgfpathclose%
\pgfpathmoveto{\pgfqpoint{0.967340in}{0.289434in}}%
\pgfpathlineto{\pgfqpoint{0.957160in}{0.302361in}}%
\pgfpathlineto{\pgfqpoint{0.955754in}{0.315287in}}%
\pgfpathlineto{\pgfqpoint{0.959797in}{0.328214in}}%
\pgfpathlineto{\pgfqpoint{0.968205in}{0.335936in}}%
\pgfpathlineto{\pgfqpoint{0.981473in}{0.334459in}}%
\pgfpathlineto{\pgfqpoint{0.986681in}{0.328214in}}%
\pgfpathlineto{\pgfqpoint{0.990003in}{0.315287in}}%
\pgfpathlineto{\pgfqpoint{0.988836in}{0.302361in}}%
\pgfpathlineto{\pgfqpoint{0.981473in}{0.290449in}}%
\pgfpathlineto{\pgfqpoint{0.973787in}{0.289434in}}%
\pgfpathlineto{\pgfqpoint{0.968205in}{0.288968in}}%
\pgfpathclose%
\pgfusepath{fill}%
\end{pgfscope}%
\begin{pgfscope}%
\pgfpathrectangle{\pgfqpoint{0.211875in}{0.211875in}}{\pgfqpoint{1.313625in}{1.279725in}}%
\pgfusepath{clip}%
\pgfsetbuttcap%
\pgfsetroundjoin%
\definecolor{currentfill}{rgb}{0.644838,0.098089,0.355336}%
\pgfsetfillcolor{currentfill}%
\pgfsetlinewidth{0.000000pt}%
\definecolor{currentstroke}{rgb}{0.000000,0.000000,0.000000}%
\pgfsetstrokecolor{currentstroke}%
\pgfsetdash{}{0pt}%
\pgfpathmoveto{\pgfqpoint{1.074356in}{0.284884in}}%
\pgfpathlineto{\pgfqpoint{1.087625in}{0.280345in}}%
\pgfpathlineto{\pgfqpoint{1.100894in}{0.281386in}}%
\pgfpathlineto{\pgfqpoint{1.111017in}{0.289434in}}%
\pgfpathlineto{\pgfqpoint{1.114163in}{0.299923in}}%
\pgfpathlineto{\pgfqpoint{1.114528in}{0.302361in}}%
\pgfpathlineto{\pgfqpoint{1.114801in}{0.315287in}}%
\pgfpathlineto{\pgfqpoint{1.114163in}{0.321693in}}%
\pgfpathlineto{\pgfqpoint{1.113161in}{0.328214in}}%
\pgfpathlineto{\pgfqpoint{1.105379in}{0.341140in}}%
\pgfpathlineto{\pgfqpoint{1.100894in}{0.343767in}}%
\pgfpathlineto{\pgfqpoint{1.087625in}{0.344960in}}%
\pgfpathlineto{\pgfqpoint{1.076720in}{0.341140in}}%
\pgfpathlineto{\pgfqpoint{1.074356in}{0.339458in}}%
\pgfpathlineto{\pgfqpoint{1.069401in}{0.328214in}}%
\pgfpathlineto{\pgfqpoint{1.067932in}{0.315287in}}%
\pgfpathlineto{\pgfqpoint{1.068212in}{0.302361in}}%
\pgfpathlineto{\pgfqpoint{1.071098in}{0.289434in}}%
\pgfpathclose%
\pgfpathmoveto{\pgfqpoint{1.087276in}{0.289434in}}%
\pgfpathlineto{\pgfqpoint{1.075388in}{0.302361in}}%
\pgfpathlineto{\pgfqpoint{1.074356in}{0.310064in}}%
\pgfpathlineto{\pgfqpoint{1.074010in}{0.315287in}}%
\pgfpathlineto{\pgfqpoint{1.074356in}{0.317300in}}%
\pgfpathlineto{\pgfqpoint{1.078410in}{0.328214in}}%
\pgfpathlineto{\pgfqpoint{1.087625in}{0.335618in}}%
\pgfpathlineto{\pgfqpoint{1.100894in}{0.332749in}}%
\pgfpathlineto{\pgfqpoint{1.104289in}{0.328214in}}%
\pgfpathlineto{\pgfqpoint{1.107467in}{0.315287in}}%
\pgfpathlineto{\pgfqpoint{1.106335in}{0.302361in}}%
\pgfpathlineto{\pgfqpoint{1.100894in}{0.292509in}}%
\pgfpathlineto{\pgfqpoint{1.088724in}{0.289434in}}%
\pgfpathlineto{\pgfqpoint{1.087625in}{0.289269in}}%
\pgfpathclose%
\pgfusepath{fill}%
\end{pgfscope}%
\begin{pgfscope}%
\pgfpathrectangle{\pgfqpoint{0.211875in}{0.211875in}}{\pgfqpoint{1.313625in}{1.279725in}}%
\pgfusepath{clip}%
\pgfsetbuttcap%
\pgfsetroundjoin%
\definecolor{currentfill}{rgb}{0.644838,0.098089,0.355336}%
\pgfsetfillcolor{currentfill}%
\pgfsetlinewidth{0.000000pt}%
\definecolor{currentstroke}{rgb}{0.000000,0.000000,0.000000}%
\pgfsetstrokecolor{currentstroke}%
\pgfsetdash{}{0pt}%
\pgfpathmoveto{\pgfqpoint{1.193777in}{0.283869in}}%
\pgfpathlineto{\pgfqpoint{1.207045in}{0.280264in}}%
\pgfpathlineto{\pgfqpoint{1.220314in}{0.281781in}}%
\pgfpathlineto{\pgfqpoint{1.228955in}{0.289434in}}%
\pgfpathlineto{\pgfqpoint{1.232327in}{0.302361in}}%
\pgfpathlineto{\pgfqpoint{1.232655in}{0.315287in}}%
\pgfpathlineto{\pgfqpoint{1.230938in}{0.328214in}}%
\pgfpathlineto{\pgfqpoint{1.223643in}{0.341140in}}%
\pgfpathlineto{\pgfqpoint{1.220314in}{0.343300in}}%
\pgfpathlineto{\pgfqpoint{1.207045in}{0.345058in}}%
\pgfpathlineto{\pgfqpoint{1.194125in}{0.341140in}}%
\pgfpathlineto{\pgfqpoint{1.193777in}{0.340941in}}%
\pgfpathlineto{\pgfqpoint{1.187493in}{0.328214in}}%
\pgfpathlineto{\pgfqpoint{1.185970in}{0.315287in}}%
\pgfpathlineto{\pgfqpoint{1.186267in}{0.302361in}}%
\pgfpathlineto{\pgfqpoint{1.189271in}{0.289434in}}%
\pgfpathclose%
\pgfpathmoveto{\pgfqpoint{1.206165in}{0.289434in}}%
\pgfpathlineto{\pgfqpoint{1.193777in}{0.300596in}}%
\pgfpathlineto{\pgfqpoint{1.193153in}{0.302361in}}%
\pgfpathlineto{\pgfqpoint{1.192213in}{0.315287in}}%
\pgfpathlineto{\pgfqpoint{1.193777in}{0.323458in}}%
\pgfpathlineto{\pgfqpoint{1.195975in}{0.328214in}}%
\pgfpathlineto{\pgfqpoint{1.207045in}{0.335873in}}%
\pgfpathlineto{\pgfqpoint{1.220314in}{0.331555in}}%
\pgfpathlineto{\pgfqpoint{1.222565in}{0.328214in}}%
\pgfpathlineto{\pgfqpoint{1.225553in}{0.315287in}}%
\pgfpathlineto{\pgfqpoint{1.224491in}{0.302361in}}%
\pgfpathlineto{\pgfqpoint{1.220314in}{0.293907in}}%
\pgfpathlineto{\pgfqpoint{1.208720in}{0.289434in}}%
\pgfpathlineto{\pgfqpoint{1.207045in}{0.289075in}}%
\pgfpathclose%
\pgfusepath{fill}%
\end{pgfscope}%
\begin{pgfscope}%
\pgfpathrectangle{\pgfqpoint{0.211875in}{0.211875in}}{\pgfqpoint{1.313625in}{1.279725in}}%
\pgfusepath{clip}%
\pgfsetbuttcap%
\pgfsetroundjoin%
\definecolor{currentfill}{rgb}{0.644838,0.098089,0.355336}%
\pgfsetfillcolor{currentfill}%
\pgfsetlinewidth{0.000000pt}%
\definecolor{currentstroke}{rgb}{0.000000,0.000000,0.000000}%
\pgfsetstrokecolor{currentstroke}%
\pgfsetdash{}{0pt}%
\pgfpathmoveto{\pgfqpoint{1.313197in}{0.282284in}}%
\pgfpathlineto{\pgfqpoint{1.326466in}{0.279643in}}%
\pgfpathlineto{\pgfqpoint{1.339735in}{0.281524in}}%
\pgfpathlineto{\pgfqpoint{1.347745in}{0.289434in}}%
\pgfpathlineto{\pgfqpoint{1.350647in}{0.302361in}}%
\pgfpathlineto{\pgfqpoint{1.350904in}{0.315287in}}%
\pgfpathlineto{\pgfqpoint{1.349365in}{0.328214in}}%
\pgfpathlineto{\pgfqpoint{1.342854in}{0.341140in}}%
\pgfpathlineto{\pgfqpoint{1.339735in}{0.343383in}}%
\pgfpathlineto{\pgfqpoint{1.326466in}{0.345628in}}%
\pgfpathlineto{\pgfqpoint{1.313197in}{0.342580in}}%
\pgfpathlineto{\pgfqpoint{1.311398in}{0.341140in}}%
\pgfpathlineto{\pgfqpoint{1.305043in}{0.328214in}}%
\pgfpathlineto{\pgfqpoint{1.303529in}{0.315287in}}%
\pgfpathlineto{\pgfqpoint{1.303801in}{0.302361in}}%
\pgfpathlineto{\pgfqpoint{1.306701in}{0.289434in}}%
\pgfpathclose%
\pgfpathmoveto{\pgfqpoint{1.323467in}{0.289434in}}%
\pgfpathlineto{\pgfqpoint{1.313197in}{0.296651in}}%
\pgfpathlineto{\pgfqpoint{1.310923in}{0.302361in}}%
\pgfpathlineto{\pgfqpoint{1.309989in}{0.315287in}}%
\pgfpathlineto{\pgfqpoint{1.312649in}{0.328214in}}%
\pgfpathlineto{\pgfqpoint{1.313197in}{0.329210in}}%
\pgfpathlineto{\pgfqpoint{1.326466in}{0.336682in}}%
\pgfpathlineto{\pgfqpoint{1.339735in}{0.330940in}}%
\pgfpathlineto{\pgfqpoint{1.341389in}{0.328214in}}%
\pgfpathlineto{\pgfqpoint{1.344143in}{0.315287in}}%
\pgfpathlineto{\pgfqpoint{1.343185in}{0.302361in}}%
\pgfpathlineto{\pgfqpoint{1.339735in}{0.294548in}}%
\pgfpathlineto{\pgfqpoint{1.330185in}{0.289434in}}%
\pgfpathlineto{\pgfqpoint{1.326466in}{0.288402in}}%
\pgfpathclose%
\pgfusepath{fill}%
\end{pgfscope}%
\begin{pgfscope}%
\pgfpathrectangle{\pgfqpoint{0.211875in}{0.211875in}}{\pgfqpoint{1.313625in}{1.279725in}}%
\pgfusepath{clip}%
\pgfsetbuttcap%
\pgfsetroundjoin%
\definecolor{currentfill}{rgb}{0.644838,0.098089,0.355336}%
\pgfsetfillcolor{currentfill}%
\pgfsetlinewidth{0.000000pt}%
\definecolor{currentstroke}{rgb}{0.000000,0.000000,0.000000}%
\pgfsetstrokecolor{currentstroke}%
\pgfsetdash{}{0pt}%
\pgfpathmoveto{\pgfqpoint{1.432617in}{0.280232in}}%
\pgfpathlineto{\pgfqpoint{1.445886in}{0.278479in}}%
\pgfpathlineto{\pgfqpoint{1.459155in}{0.280468in}}%
\pgfpathlineto{\pgfqpoint{1.467273in}{0.289434in}}%
\pgfpathlineto{\pgfqpoint{1.469504in}{0.302361in}}%
\pgfpathlineto{\pgfqpoint{1.469653in}{0.315287in}}%
\pgfpathlineto{\pgfqpoint{1.468344in}{0.328214in}}%
\pgfpathlineto{\pgfqpoint{1.462896in}{0.341140in}}%
\pgfpathlineto{\pgfqpoint{1.459155in}{0.344125in}}%
\pgfpathlineto{\pgfqpoint{1.445886in}{0.346672in}}%
\pgfpathlineto{\pgfqpoint{1.432617in}{0.344560in}}%
\pgfpathlineto{\pgfqpoint{1.427885in}{0.341140in}}%
\pgfpathlineto{\pgfqpoint{1.421965in}{0.328214in}}%
\pgfpathlineto{\pgfqpoint{1.420529in}{0.315287in}}%
\pgfpathlineto{\pgfqpoint{1.420731in}{0.302361in}}%
\pgfpathlineto{\pgfqpoint{1.423285in}{0.289434in}}%
\pgfpathclose%
\pgfpathmoveto{\pgfqpoint{1.438103in}{0.289434in}}%
\pgfpathlineto{\pgfqpoint{1.432617in}{0.292362in}}%
\pgfpathlineto{\pgfqpoint{1.428155in}{0.302361in}}%
\pgfpathlineto{\pgfqpoint{1.427266in}{0.315287in}}%
\pgfpathlineto{\pgfqpoint{1.429892in}{0.328214in}}%
\pgfpathlineto{\pgfqpoint{1.432617in}{0.332672in}}%
\pgfpathlineto{\pgfqpoint{1.445886in}{0.338051in}}%
\pgfpathlineto{\pgfqpoint{1.459155in}{0.331011in}}%
\pgfpathlineto{\pgfqpoint{1.460681in}{0.328214in}}%
\pgfpathlineto{\pgfqpoint{1.463162in}{0.315287in}}%
\pgfpathlineto{\pgfqpoint{1.462336in}{0.302361in}}%
\pgfpathlineto{\pgfqpoint{1.459155in}{0.294274in}}%
\pgfpathlineto{\pgfqpoint{1.452316in}{0.289434in}}%
\pgfpathlineto{\pgfqpoint{1.445886in}{0.287245in}}%
\pgfpathclose%
\pgfusepath{fill}%
\end{pgfscope}%
\begin{pgfscope}%
\pgfpathrectangle{\pgfqpoint{0.211875in}{0.211875in}}{\pgfqpoint{1.313625in}{1.279725in}}%
\pgfusepath{clip}%
\pgfsetbuttcap%
\pgfsetroundjoin%
\definecolor{currentfill}{rgb}{0.644838,0.098089,0.355336}%
\pgfsetfillcolor{currentfill}%
\pgfsetlinewidth{0.000000pt}%
\definecolor{currentstroke}{rgb}{0.000000,0.000000,0.000000}%
\pgfsetstrokecolor{currentstroke}%
\pgfsetdash{}{0pt}%
\pgfpathmoveto{\pgfqpoint{0.543598in}{0.358644in}}%
\pgfpathlineto{\pgfqpoint{0.556867in}{0.357316in}}%
\pgfpathlineto{\pgfqpoint{0.570136in}{0.358041in}}%
\pgfpathlineto{\pgfqpoint{0.583405in}{0.365037in}}%
\pgfpathlineto{\pgfqpoint{0.584320in}{0.366993in}}%
\pgfpathlineto{\pgfqpoint{0.586475in}{0.379920in}}%
\pgfpathlineto{\pgfqpoint{0.586781in}{0.392846in}}%
\pgfpathlineto{\pgfqpoint{0.586195in}{0.405773in}}%
\pgfpathlineto{\pgfqpoint{0.583844in}{0.418699in}}%
\pgfpathlineto{\pgfqpoint{0.583405in}{0.419767in}}%
\pgfpathlineto{\pgfqpoint{0.570136in}{0.429492in}}%
\pgfpathlineto{\pgfqpoint{0.556867in}{0.430477in}}%
\pgfpathlineto{\pgfqpoint{0.543598in}{0.427897in}}%
\pgfpathlineto{\pgfqpoint{0.536458in}{0.418699in}}%
\pgfpathlineto{\pgfqpoint{0.534351in}{0.405773in}}%
\pgfpathlineto{\pgfqpoint{0.533795in}{0.392846in}}%
\pgfpathlineto{\pgfqpoint{0.533953in}{0.379920in}}%
\pgfpathlineto{\pgfqpoint{0.535524in}{0.366993in}}%
\pgfpathclose%
\pgfpathmoveto{\pgfqpoint{0.553420in}{0.366993in}}%
\pgfpathlineto{\pgfqpoint{0.543598in}{0.374259in}}%
\pgfpathlineto{\pgfqpoint{0.541330in}{0.379920in}}%
\pgfpathlineto{\pgfqpoint{0.539960in}{0.392846in}}%
\pgfpathlineto{\pgfqpoint{0.541087in}{0.405773in}}%
\pgfpathlineto{\pgfqpoint{0.543598in}{0.412906in}}%
\pgfpathlineto{\pgfqpoint{0.549346in}{0.418699in}}%
\pgfpathlineto{\pgfqpoint{0.556867in}{0.422042in}}%
\pgfpathlineto{\pgfqpoint{0.570136in}{0.420070in}}%
\pgfpathlineto{\pgfqpoint{0.571895in}{0.418699in}}%
\pgfpathlineto{\pgfqpoint{0.578946in}{0.405773in}}%
\pgfpathlineto{\pgfqpoint{0.580380in}{0.392846in}}%
\pgfpathlineto{\pgfqpoint{0.578574in}{0.379920in}}%
\pgfpathlineto{\pgfqpoint{0.570136in}{0.367807in}}%
\pgfpathlineto{\pgfqpoint{0.565940in}{0.366993in}}%
\pgfpathlineto{\pgfqpoint{0.556867in}{0.365816in}}%
\pgfpathclose%
\pgfusepath{fill}%
\end{pgfscope}%
\begin{pgfscope}%
\pgfpathrectangle{\pgfqpoint{0.211875in}{0.211875in}}{\pgfqpoint{1.313625in}{1.279725in}}%
\pgfusepath{clip}%
\pgfsetbuttcap%
\pgfsetroundjoin%
\definecolor{currentfill}{rgb}{0.644838,0.098089,0.355336}%
\pgfsetfillcolor{currentfill}%
\pgfsetlinewidth{0.000000pt}%
\definecolor{currentstroke}{rgb}{0.000000,0.000000,0.000000}%
\pgfsetstrokecolor{currentstroke}%
\pgfsetdash{}{0pt}%
\pgfpathmoveto{\pgfqpoint{0.663019in}{0.362094in}}%
\pgfpathlineto{\pgfqpoint{0.676288in}{0.359806in}}%
\pgfpathlineto{\pgfqpoint{0.689557in}{0.361193in}}%
\pgfpathlineto{\pgfqpoint{0.698003in}{0.366993in}}%
\pgfpathlineto{\pgfqpoint{0.702449in}{0.379920in}}%
\pgfpathlineto{\pgfqpoint{0.702826in}{0.385814in}}%
\pgfpathlineto{\pgfqpoint{0.703133in}{0.392846in}}%
\pgfpathlineto{\pgfqpoint{0.702826in}{0.398521in}}%
\pgfpathlineto{\pgfqpoint{0.702300in}{0.405773in}}%
\pgfpathlineto{\pgfqpoint{0.698324in}{0.418699in}}%
\pgfpathlineto{\pgfqpoint{0.689557in}{0.426274in}}%
\pgfpathlineto{\pgfqpoint{0.676288in}{0.428033in}}%
\pgfpathlineto{\pgfqpoint{0.663019in}{0.424870in}}%
\pgfpathlineto{\pgfqpoint{0.657644in}{0.418699in}}%
\pgfpathlineto{\pgfqpoint{0.654373in}{0.405773in}}%
\pgfpathlineto{\pgfqpoint{0.653621in}{0.392846in}}%
\pgfpathlineto{\pgfqpoint{0.654168in}{0.379920in}}%
\pgfpathlineto{\pgfqpoint{0.657664in}{0.366993in}}%
\pgfpathclose%
\pgfpathmoveto{\pgfqpoint{0.661738in}{0.379920in}}%
\pgfpathlineto{\pgfqpoint{0.659950in}{0.392846in}}%
\pgfpathlineto{\pgfqpoint{0.661287in}{0.405773in}}%
\pgfpathlineto{\pgfqpoint{0.663019in}{0.410159in}}%
\pgfpathlineto{\pgfqpoint{0.673657in}{0.418699in}}%
\pgfpathlineto{\pgfqpoint{0.676288in}{0.419704in}}%
\pgfpathlineto{\pgfqpoint{0.680811in}{0.418699in}}%
\pgfpathlineto{\pgfqpoint{0.689557in}{0.415339in}}%
\pgfpathlineto{\pgfqpoint{0.694641in}{0.405773in}}%
\pgfpathlineto{\pgfqpoint{0.696173in}{0.392846in}}%
\pgfpathlineto{\pgfqpoint{0.694089in}{0.379920in}}%
\pgfpathlineto{\pgfqpoint{0.689557in}{0.372664in}}%
\pgfpathlineto{\pgfqpoint{0.676288in}{0.368585in}}%
\pgfpathlineto{\pgfqpoint{0.663019in}{0.377104in}}%
\pgfpathclose%
\pgfusepath{fill}%
\end{pgfscope}%
\begin{pgfscope}%
\pgfpathrectangle{\pgfqpoint{0.211875in}{0.211875in}}{\pgfqpoint{1.313625in}{1.279725in}}%
\pgfusepath{clip}%
\pgfsetbuttcap%
\pgfsetroundjoin%
\definecolor{currentfill}{rgb}{0.644838,0.098089,0.355336}%
\pgfsetfillcolor{currentfill}%
\pgfsetlinewidth{0.000000pt}%
\definecolor{currentstroke}{rgb}{0.000000,0.000000,0.000000}%
\pgfsetstrokecolor{currentstroke}%
\pgfsetdash{}{0pt}%
\pgfpathmoveto{\pgfqpoint{0.782439in}{0.364238in}}%
\pgfpathlineto{\pgfqpoint{0.795708in}{0.361720in}}%
\pgfpathlineto{\pgfqpoint{0.808977in}{0.364045in}}%
\pgfpathlineto{\pgfqpoint{0.812843in}{0.366993in}}%
\pgfpathlineto{\pgfqpoint{0.818408in}{0.379920in}}%
\pgfpathlineto{\pgfqpoint{0.819401in}{0.392846in}}%
\pgfpathlineto{\pgfqpoint{0.818411in}{0.405773in}}%
\pgfpathlineto{\pgfqpoint{0.813831in}{0.418699in}}%
\pgfpathlineto{\pgfqpoint{0.808977in}{0.423346in}}%
\pgfpathlineto{\pgfqpoint{0.795708in}{0.426145in}}%
\pgfpathlineto{\pgfqpoint{0.782439in}{0.423001in}}%
\pgfpathlineto{\pgfqpoint{0.778260in}{0.418699in}}%
\pgfpathlineto{\pgfqpoint{0.773985in}{0.405773in}}%
\pgfpathlineto{\pgfqpoint{0.773061in}{0.392846in}}%
\pgfpathlineto{\pgfqpoint{0.773939in}{0.379920in}}%
\pgfpathlineto{\pgfqpoint{0.779062in}{0.366993in}}%
\pgfpathclose%
\pgfpathmoveto{\pgfqpoint{0.781765in}{0.379920in}}%
\pgfpathlineto{\pgfqpoint{0.779608in}{0.392846in}}%
\pgfpathlineto{\pgfqpoint{0.781135in}{0.405773in}}%
\pgfpathlineto{\pgfqpoint{0.782439in}{0.408733in}}%
\pgfpathlineto{\pgfqpoint{0.795708in}{0.417482in}}%
\pgfpathlineto{\pgfqpoint{0.808977in}{0.410242in}}%
\pgfpathlineto{\pgfqpoint{0.811118in}{0.405773in}}%
\pgfpathlineto{\pgfqpoint{0.812709in}{0.392846in}}%
\pgfpathlineto{\pgfqpoint{0.810443in}{0.379920in}}%
\pgfpathlineto{\pgfqpoint{0.808977in}{0.377301in}}%
\pgfpathlineto{\pgfqpoint{0.795708in}{0.371052in}}%
\pgfpathlineto{\pgfqpoint{0.782439in}{0.378602in}}%
\pgfpathclose%
\pgfusepath{fill}%
\end{pgfscope}%
\begin{pgfscope}%
\pgfpathrectangle{\pgfqpoint{0.211875in}{0.211875in}}{\pgfqpoint{1.313625in}{1.279725in}}%
\pgfusepath{clip}%
\pgfsetbuttcap%
\pgfsetroundjoin%
\definecolor{currentfill}{rgb}{0.644838,0.098089,0.355336}%
\pgfsetfillcolor{currentfill}%
\pgfsetlinewidth{0.000000pt}%
\definecolor{currentstroke}{rgb}{0.000000,0.000000,0.000000}%
\pgfsetstrokecolor{currentstroke}%
\pgfsetdash{}{0pt}%
\pgfpathmoveto{\pgfqpoint{0.901860in}{0.365414in}}%
\pgfpathlineto{\pgfqpoint{0.915129in}{0.363109in}}%
\pgfpathlineto{\pgfqpoint{0.928398in}{0.366595in}}%
\pgfpathlineto{\pgfqpoint{0.928866in}{0.366993in}}%
\pgfpathlineto{\pgfqpoint{0.935125in}{0.379920in}}%
\pgfpathlineto{\pgfqpoint{0.936266in}{0.392846in}}%
\pgfpathlineto{\pgfqpoint{0.935227in}{0.405773in}}%
\pgfpathlineto{\pgfqpoint{0.930286in}{0.418699in}}%
\pgfpathlineto{\pgfqpoint{0.928398in}{0.420707in}}%
\pgfpathlineto{\pgfqpoint{0.915129in}{0.424765in}}%
\pgfpathlineto{\pgfqpoint{0.901860in}{0.422011in}}%
\pgfpathlineto{\pgfqpoint{0.898283in}{0.418699in}}%
\pgfpathlineto{\pgfqpoint{0.893160in}{0.405773in}}%
\pgfpathlineto{\pgfqpoint{0.892085in}{0.392846in}}%
\pgfpathlineto{\pgfqpoint{0.893236in}{0.379920in}}%
\pgfpathlineto{\pgfqpoint{0.899700in}{0.366993in}}%
\pgfpathclose%
\pgfpathmoveto{\pgfqpoint{0.901391in}{0.379920in}}%
\pgfpathlineto{\pgfqpoint{0.898912in}{0.392846in}}%
\pgfpathlineto{\pgfqpoint{0.900613in}{0.405773in}}%
\pgfpathlineto{\pgfqpoint{0.901860in}{0.408317in}}%
\pgfpathlineto{\pgfqpoint{0.915129in}{0.415463in}}%
\pgfpathlineto{\pgfqpoint{0.927997in}{0.405773in}}%
\pgfpathlineto{\pgfqpoint{0.928398in}{0.404483in}}%
\pgfpathlineto{\pgfqpoint{0.929842in}{0.392846in}}%
\pgfpathlineto{\pgfqpoint{0.928398in}{0.384438in}}%
\pgfpathlineto{\pgfqpoint{0.926278in}{0.379920in}}%
\pgfpathlineto{\pgfqpoint{0.915129in}{0.372895in}}%
\pgfpathlineto{\pgfqpoint{0.901860in}{0.379102in}}%
\pgfpathclose%
\pgfusepath{fill}%
\end{pgfscope}%
\begin{pgfscope}%
\pgfpathrectangle{\pgfqpoint{0.211875in}{0.211875in}}{\pgfqpoint{1.313625in}{1.279725in}}%
\pgfusepath{clip}%
\pgfsetbuttcap%
\pgfsetroundjoin%
\definecolor{currentfill}{rgb}{0.644838,0.098089,0.355336}%
\pgfsetfillcolor{currentfill}%
\pgfsetlinewidth{0.000000pt}%
\definecolor{currentstroke}{rgb}{0.000000,0.000000,0.000000}%
\pgfsetstrokecolor{currentstroke}%
\pgfsetdash{}{0pt}%
\pgfpathmoveto{\pgfqpoint{1.021280in}{0.365836in}}%
\pgfpathlineto{\pgfqpoint{1.034549in}{0.363999in}}%
\pgfpathlineto{\pgfqpoint{1.043959in}{0.366993in}}%
\pgfpathlineto{\pgfqpoint{1.047818in}{0.369434in}}%
\pgfpathlineto{\pgfqpoint{1.052476in}{0.379920in}}%
\pgfpathlineto{\pgfqpoint{1.053683in}{0.392846in}}%
\pgfpathlineto{\pgfqpoint{1.052629in}{0.405773in}}%
\pgfpathlineto{\pgfqpoint{1.047818in}{0.418221in}}%
\pgfpathlineto{\pgfqpoint{1.047260in}{0.418699in}}%
\pgfpathlineto{\pgfqpoint{1.034549in}{0.423868in}}%
\pgfpathlineto{\pgfqpoint{1.021280in}{0.421719in}}%
\pgfpathlineto{\pgfqpoint{1.017661in}{0.418699in}}%
\pgfpathlineto{\pgfqpoint{1.011845in}{0.405773in}}%
\pgfpathlineto{\pgfqpoint{1.010642in}{0.392846in}}%
\pgfpathlineto{\pgfqpoint{1.012005in}{0.379920in}}%
\pgfpathlineto{\pgfqpoint{1.019519in}{0.366993in}}%
\pgfpathclose%
\pgfpathmoveto{\pgfqpoint{1.020576in}{0.379920in}}%
\pgfpathlineto{\pgfqpoint{1.017822in}{0.392846in}}%
\pgfpathlineto{\pgfqpoint{1.019681in}{0.405773in}}%
\pgfpathlineto{\pgfqpoint{1.021280in}{0.408711in}}%
\pgfpathlineto{\pgfqpoint{1.034549in}{0.414074in}}%
\pgfpathlineto{\pgfqpoint{1.043854in}{0.405773in}}%
\pgfpathlineto{\pgfqpoint{1.047113in}{0.392846in}}%
\pgfpathlineto{\pgfqpoint{1.042278in}{0.379920in}}%
\pgfpathlineto{\pgfqpoint{1.034549in}{0.374153in}}%
\pgfpathlineto{\pgfqpoint{1.021280in}{0.378822in}}%
\pgfpathclose%
\pgfusepath{fill}%
\end{pgfscope}%
\begin{pgfscope}%
\pgfpathrectangle{\pgfqpoint{0.211875in}{0.211875in}}{\pgfqpoint{1.313625in}{1.279725in}}%
\pgfusepath{clip}%
\pgfsetbuttcap%
\pgfsetroundjoin%
\definecolor{currentfill}{rgb}{0.644838,0.098089,0.355336}%
\pgfsetfillcolor{currentfill}%
\pgfsetlinewidth{0.000000pt}%
\definecolor{currentstroke}{rgb}{0.000000,0.000000,0.000000}%
\pgfsetstrokecolor{currentstroke}%
\pgfsetdash{}{0pt}%
\pgfpathmoveto{\pgfqpoint{1.140701in}{0.365642in}}%
\pgfpathlineto{\pgfqpoint{1.153970in}{0.364400in}}%
\pgfpathlineto{\pgfqpoint{1.161048in}{0.366993in}}%
\pgfpathlineto{\pgfqpoint{1.167239in}{0.371917in}}%
\pgfpathlineto{\pgfqpoint{1.170371in}{0.379920in}}%
\pgfpathlineto{\pgfqpoint{1.171573in}{0.392846in}}%
\pgfpathlineto{\pgfqpoint{1.170534in}{0.405773in}}%
\pgfpathlineto{\pgfqpoint{1.167239in}{0.415333in}}%
\pgfpathlineto{\pgfqpoint{1.164082in}{0.418699in}}%
\pgfpathlineto{\pgfqpoint{1.153970in}{0.423446in}}%
\pgfpathlineto{\pgfqpoint{1.140701in}{0.422006in}}%
\pgfpathlineto{\pgfqpoint{1.136300in}{0.418699in}}%
\pgfpathlineto{\pgfqpoint{1.129953in}{0.405773in}}%
\pgfpathlineto{\pgfqpoint{1.128645in}{0.392846in}}%
\pgfpathlineto{\pgfqpoint{1.130158in}{0.379920in}}%
\pgfpathlineto{\pgfqpoint{1.138414in}{0.366993in}}%
\pgfpathclose%
\pgfpathmoveto{\pgfqpoint{1.139252in}{0.379920in}}%
\pgfpathlineto{\pgfqpoint{1.136270in}{0.392846in}}%
\pgfpathlineto{\pgfqpoint{1.138271in}{0.405773in}}%
\pgfpathlineto{\pgfqpoint{1.140701in}{0.409786in}}%
\pgfpathlineto{\pgfqpoint{1.153970in}{0.413302in}}%
\pgfpathlineto{\pgfqpoint{1.161254in}{0.405773in}}%
\pgfpathlineto{\pgfqpoint{1.164126in}{0.392846in}}%
\pgfpathlineto{\pgfqpoint{1.159846in}{0.379920in}}%
\pgfpathlineto{\pgfqpoint{1.153970in}{0.374838in}}%
\pgfpathlineto{\pgfqpoint{1.140701in}{0.377897in}}%
\pgfpathclose%
\pgfusepath{fill}%
\end{pgfscope}%
\begin{pgfscope}%
\pgfpathrectangle{\pgfqpoint{0.211875in}{0.211875in}}{\pgfqpoint{1.313625in}{1.279725in}}%
\pgfusepath{clip}%
\pgfsetbuttcap%
\pgfsetroundjoin%
\definecolor{currentfill}{rgb}{0.644838,0.098089,0.355336}%
\pgfsetfillcolor{currentfill}%
\pgfsetlinewidth{0.000000pt}%
\definecolor{currentstroke}{rgb}{0.000000,0.000000,0.000000}%
\pgfsetstrokecolor{currentstroke}%
\pgfsetdash{}{0pt}%
\pgfpathmoveto{\pgfqpoint{1.260121in}{0.364920in}}%
\pgfpathlineto{\pgfqpoint{1.273390in}{0.364297in}}%
\pgfpathlineto{\pgfqpoint{1.279867in}{0.366993in}}%
\pgfpathlineto{\pgfqpoint{1.286659in}{0.373780in}}%
\pgfpathlineto{\pgfqpoint{1.288750in}{0.379920in}}%
\pgfpathlineto{\pgfqpoint{1.289879in}{0.392846in}}%
\pgfpathlineto{\pgfqpoint{1.288883in}{0.405773in}}%
\pgfpathlineto{\pgfqpoint{1.286659in}{0.413070in}}%
\pgfpathlineto{\pgfqpoint{1.282400in}{0.418699in}}%
\pgfpathlineto{\pgfqpoint{1.273390in}{0.423512in}}%
\pgfpathlineto{\pgfqpoint{1.260121in}{0.422797in}}%
\pgfpathlineto{\pgfqpoint{1.254053in}{0.418699in}}%
\pgfpathlineto{\pgfqpoint{1.247357in}{0.405773in}}%
\pgfpathlineto{\pgfqpoint{1.246852in}{0.401332in}}%
\pgfpathlineto{\pgfqpoint{1.246229in}{0.392846in}}%
\pgfpathlineto{\pgfqpoint{1.246852in}{0.385109in}}%
\pgfpathlineto{\pgfqpoint{1.247559in}{0.379920in}}%
\pgfpathlineto{\pgfqpoint{1.256213in}{0.366993in}}%
\pgfpathclose%
\pgfpathmoveto{\pgfqpoint{1.257312in}{0.379920in}}%
\pgfpathlineto{\pgfqpoint{1.254152in}{0.392846in}}%
\pgfpathlineto{\pgfqpoint{1.256281in}{0.405773in}}%
\pgfpathlineto{\pgfqpoint{1.260121in}{0.411462in}}%
\pgfpathlineto{\pgfqpoint{1.273390in}{0.413159in}}%
\pgfpathlineto{\pgfqpoint{1.279652in}{0.405773in}}%
\pgfpathlineto{\pgfqpoint{1.282182in}{0.392846in}}%
\pgfpathlineto{\pgfqpoint{1.278433in}{0.379920in}}%
\pgfpathlineto{\pgfqpoint{1.273390in}{0.374937in}}%
\pgfpathlineto{\pgfqpoint{1.260121in}{0.376413in}}%
\pgfpathclose%
\pgfusepath{fill}%
\end{pgfscope}%
\begin{pgfscope}%
\pgfpathrectangle{\pgfqpoint{0.211875in}{0.211875in}}{\pgfqpoint{1.313625in}{1.279725in}}%
\pgfusepath{clip}%
\pgfsetbuttcap%
\pgfsetroundjoin%
\definecolor{currentfill}{rgb}{0.644838,0.098089,0.355336}%
\pgfsetfillcolor{currentfill}%
\pgfsetlinewidth{0.000000pt}%
\definecolor{currentstroke}{rgb}{0.000000,0.000000,0.000000}%
\pgfsetstrokecolor{currentstroke}%
\pgfsetdash{}{0pt}%
\pgfpathmoveto{\pgfqpoint{1.379542in}{0.363720in}}%
\pgfpathlineto{\pgfqpoint{1.392811in}{0.363655in}}%
\pgfpathlineto{\pgfqpoint{1.399931in}{0.366993in}}%
\pgfpathlineto{\pgfqpoint{1.406080in}{0.374774in}}%
\pgfpathlineto{\pgfqpoint{1.407578in}{0.379920in}}%
\pgfpathlineto{\pgfqpoint{1.408567in}{0.392846in}}%
\pgfpathlineto{\pgfqpoint{1.407643in}{0.405773in}}%
\pgfpathlineto{\pgfqpoint{1.406080in}{0.411642in}}%
\pgfpathlineto{\pgfqpoint{1.401779in}{0.418699in}}%
\pgfpathlineto{\pgfqpoint{1.392811in}{0.424096in}}%
\pgfpathlineto{\pgfqpoint{1.379542in}{0.424046in}}%
\pgfpathlineto{\pgfqpoint{1.370692in}{0.418699in}}%
\pgfpathlineto{\pgfqpoint{1.366273in}{0.411591in}}%
\pgfpathlineto{\pgfqpoint{1.364693in}{0.405773in}}%
\pgfpathlineto{\pgfqpoint{1.363752in}{0.392846in}}%
\pgfpathlineto{\pgfqpoint{1.364785in}{0.379920in}}%
\pgfpathlineto{\pgfqpoint{1.366273in}{0.374931in}}%
\pgfpathlineto{\pgfqpoint{1.372649in}{0.366993in}}%
\pgfpathclose%
\pgfpathmoveto{\pgfqpoint{1.374587in}{0.379920in}}%
\pgfpathlineto{\pgfqpoint{1.371311in}{0.392846in}}%
\pgfpathlineto{\pgfqpoint{1.373551in}{0.405773in}}%
\pgfpathlineto{\pgfqpoint{1.379542in}{0.413694in}}%
\pgfpathlineto{\pgfqpoint{1.392811in}{0.413683in}}%
\pgfpathlineto{\pgfqpoint{1.398751in}{0.405773in}}%
\pgfpathlineto{\pgfqpoint{1.400963in}{0.392846in}}%
\pgfpathlineto{\pgfqpoint{1.397739in}{0.379920in}}%
\pgfpathlineto{\pgfqpoint{1.392811in}{0.374412in}}%
\pgfpathlineto{\pgfqpoint{1.379542in}{0.374413in}}%
\pgfpathclose%
\pgfusepath{fill}%
\end{pgfscope}%
\begin{pgfscope}%
\pgfpathrectangle{\pgfqpoint{0.211875in}{0.211875in}}{\pgfqpoint{1.313625in}{1.279725in}}%
\pgfusepath{clip}%
\pgfsetbuttcap%
\pgfsetroundjoin%
\definecolor{currentfill}{rgb}{0.644838,0.098089,0.355336}%
\pgfsetfillcolor{currentfill}%
\pgfsetlinewidth{0.000000pt}%
\definecolor{currentstroke}{rgb}{0.000000,0.000000,0.000000}%
\pgfsetstrokecolor{currentstroke}%
\pgfsetdash{}{0pt}%
\pgfpathmoveto{\pgfqpoint{1.498962in}{0.362068in}}%
\pgfpathlineto{\pgfqpoint{1.512231in}{0.362412in}}%
\pgfpathlineto{\pgfqpoint{1.520962in}{0.366993in}}%
\pgfpathlineto{\pgfqpoint{1.525500in}{0.374411in}}%
\pgfpathlineto{\pgfqpoint{1.525500in}{0.379920in}}%
\pgfpathlineto{\pgfqpoint{1.525500in}{0.392846in}}%
\pgfpathlineto{\pgfqpoint{1.525500in}{0.405773in}}%
\pgfpathlineto{\pgfqpoint{1.525500in}{0.411430in}}%
\pgfpathlineto{\pgfqpoint{1.521969in}{0.418699in}}%
\pgfpathlineto{\pgfqpoint{1.512231in}{0.425256in}}%
\pgfpathlineto{\pgfqpoint{1.498962in}{0.425731in}}%
\pgfpathlineto{\pgfqpoint{1.485863in}{0.418699in}}%
\pgfpathlineto{\pgfqpoint{1.485693in}{0.418478in}}%
\pgfpathlineto{\pgfqpoint{1.481758in}{0.405773in}}%
\pgfpathlineto{\pgfqpoint{1.480886in}{0.392846in}}%
\pgfpathlineto{\pgfqpoint{1.481761in}{0.379920in}}%
\pgfpathlineto{\pgfqpoint{1.485693in}{0.368591in}}%
\pgfpathlineto{\pgfqpoint{1.487308in}{0.366993in}}%
\pgfpathclose%
\pgfpathmoveto{\pgfqpoint{1.490818in}{0.379920in}}%
\pgfpathlineto{\pgfqpoint{1.487500in}{0.392846in}}%
\pgfpathlineto{\pgfqpoint{1.489831in}{0.405773in}}%
\pgfpathlineto{\pgfqpoint{1.498962in}{0.416467in}}%
\pgfpathlineto{\pgfqpoint{1.512231in}{0.414943in}}%
\pgfpathlineto{\pgfqpoint{1.518376in}{0.405773in}}%
\pgfpathlineto{\pgfqpoint{1.520285in}{0.392846in}}%
\pgfpathlineto{\pgfqpoint{1.517589in}{0.379920in}}%
\pgfpathlineto{\pgfqpoint{1.512231in}{0.373191in}}%
\pgfpathlineto{\pgfqpoint{1.498962in}{0.371915in}}%
\pgfpathclose%
\pgfusepath{fill}%
\end{pgfscope}%
\begin{pgfscope}%
\pgfpathrectangle{\pgfqpoint{0.211875in}{0.211875in}}{\pgfqpoint{1.313625in}{1.279725in}}%
\pgfusepath{clip}%
\pgfsetbuttcap%
\pgfsetroundjoin%
\definecolor{currentfill}{rgb}{0.644838,0.098089,0.355336}%
\pgfsetfillcolor{currentfill}%
\pgfsetlinewidth{0.000000pt}%
\definecolor{currentstroke}{rgb}{0.000000,0.000000,0.000000}%
\pgfsetstrokecolor{currentstroke}%
\pgfsetdash{}{0pt}%
\pgfpathmoveto{\pgfqpoint{0.357833in}{0.435248in}}%
\pgfpathlineto{\pgfqpoint{0.371102in}{0.436045in}}%
\pgfpathlineto{\pgfqpoint{0.384371in}{0.436340in}}%
\pgfpathlineto{\pgfqpoint{0.397640in}{0.437136in}}%
\pgfpathlineto{\pgfqpoint{0.409571in}{0.444552in}}%
\pgfpathlineto{\pgfqpoint{0.410909in}{0.452318in}}%
\pgfpathlineto{\pgfqpoint{0.411292in}{0.457479in}}%
\pgfpathlineto{\pgfqpoint{0.411504in}{0.470405in}}%
\pgfpathlineto{\pgfqpoint{0.411227in}{0.483332in}}%
\pgfpathlineto{\pgfqpoint{0.410909in}{0.488097in}}%
\pgfpathlineto{\pgfqpoint{0.410112in}{0.496258in}}%
\pgfpathlineto{\pgfqpoint{0.404585in}{0.509185in}}%
\pgfpathlineto{\pgfqpoint{0.397640in}{0.512635in}}%
\pgfpathlineto{\pgfqpoint{0.384371in}{0.514053in}}%
\pgfpathlineto{\pgfqpoint{0.371102in}{0.513790in}}%
\pgfpathlineto{\pgfqpoint{0.359122in}{0.509185in}}%
\pgfpathlineto{\pgfqpoint{0.357833in}{0.507035in}}%
\pgfpathlineto{\pgfqpoint{0.355684in}{0.496258in}}%
\pgfpathlineto{\pgfqpoint{0.355032in}{0.483332in}}%
\pgfpathlineto{\pgfqpoint{0.354764in}{0.470405in}}%
\pgfpathlineto{\pgfqpoint{0.354603in}{0.457479in}}%
\pgfpathlineto{\pgfqpoint{0.354353in}{0.444552in}}%
\pgfpathclose%
\pgfpathmoveto{\pgfqpoint{0.383434in}{0.444552in}}%
\pgfpathlineto{\pgfqpoint{0.371102in}{0.446900in}}%
\pgfpathlineto{\pgfqpoint{0.363360in}{0.457479in}}%
\pgfpathlineto{\pgfqpoint{0.361351in}{0.470405in}}%
\pgfpathlineto{\pgfqpoint{0.361761in}{0.483332in}}%
\pgfpathlineto{\pgfqpoint{0.365103in}{0.496258in}}%
\pgfpathlineto{\pgfqpoint{0.371102in}{0.503216in}}%
\pgfpathlineto{\pgfqpoint{0.384371in}{0.505932in}}%
\pgfpathlineto{\pgfqpoint{0.397640in}{0.501249in}}%
\pgfpathlineto{\pgfqpoint{0.401317in}{0.496258in}}%
\pgfpathlineto{\pgfqpoint{0.404684in}{0.483332in}}%
\pgfpathlineto{\pgfqpoint{0.405074in}{0.470405in}}%
\pgfpathlineto{\pgfqpoint{0.402950in}{0.457479in}}%
\pgfpathlineto{\pgfqpoint{0.397640in}{0.448977in}}%
\pgfpathlineto{\pgfqpoint{0.384915in}{0.444552in}}%
\pgfpathlineto{\pgfqpoint{0.384371in}{0.444438in}}%
\pgfpathclose%
\pgfusepath{fill}%
\end{pgfscope}%
\begin{pgfscope}%
\pgfpathrectangle{\pgfqpoint{0.211875in}{0.211875in}}{\pgfqpoint{1.313625in}{1.279725in}}%
\pgfusepath{clip}%
\pgfsetbuttcap%
\pgfsetroundjoin%
\definecolor{currentfill}{rgb}{0.644838,0.098089,0.355336}%
\pgfsetfillcolor{currentfill}%
\pgfsetlinewidth{0.000000pt}%
\definecolor{currentstroke}{rgb}{0.000000,0.000000,0.000000}%
\pgfsetstrokecolor{currentstroke}%
\pgfsetdash{}{0pt}%
\pgfpathmoveto{\pgfqpoint{0.490523in}{0.440013in}}%
\pgfpathlineto{\pgfqpoint{0.503792in}{0.439595in}}%
\pgfpathlineto{\pgfqpoint{0.517061in}{0.441975in}}%
\pgfpathlineto{\pgfqpoint{0.520781in}{0.444552in}}%
\pgfpathlineto{\pgfqpoint{0.525993in}{0.457479in}}%
\pgfpathlineto{\pgfqpoint{0.526915in}{0.470405in}}%
\pgfpathlineto{\pgfqpoint{0.526590in}{0.483332in}}%
\pgfpathlineto{\pgfqpoint{0.524661in}{0.496258in}}%
\pgfpathlineto{\pgfqpoint{0.517061in}{0.507756in}}%
\pgfpathlineto{\pgfqpoint{0.512749in}{0.509185in}}%
\pgfpathlineto{\pgfqpoint{0.503792in}{0.510960in}}%
\pgfpathlineto{\pgfqpoint{0.490523in}{0.510215in}}%
\pgfpathlineto{\pgfqpoint{0.487559in}{0.509185in}}%
\pgfpathlineto{\pgfqpoint{0.477254in}{0.497096in}}%
\pgfpathlineto{\pgfqpoint{0.477030in}{0.496258in}}%
\pgfpathlineto{\pgfqpoint{0.475487in}{0.483332in}}%
\pgfpathlineto{\pgfqpoint{0.475188in}{0.470405in}}%
\pgfpathlineto{\pgfqpoint{0.475794in}{0.457479in}}%
\pgfpathlineto{\pgfqpoint{0.477254in}{0.449999in}}%
\pgfpathlineto{\pgfqpoint{0.480249in}{0.444552in}}%
\pgfpathclose%
\pgfpathmoveto{\pgfqpoint{0.485186in}{0.457479in}}%
\pgfpathlineto{\pgfqpoint{0.482138in}{0.470405in}}%
\pgfpathlineto{\pgfqpoint{0.482612in}{0.483332in}}%
\pgfpathlineto{\pgfqpoint{0.487226in}{0.496258in}}%
\pgfpathlineto{\pgfqpoint{0.490523in}{0.499695in}}%
\pgfpathlineto{\pgfqpoint{0.503792in}{0.502286in}}%
\pgfpathlineto{\pgfqpoint{0.515225in}{0.496258in}}%
\pgfpathlineto{\pgfqpoint{0.517061in}{0.493957in}}%
\pgfpathlineto{\pgfqpoint{0.520231in}{0.483332in}}%
\pgfpathlineto{\pgfqpoint{0.520641in}{0.470405in}}%
\pgfpathlineto{\pgfqpoint{0.517879in}{0.457479in}}%
\pgfpathlineto{\pgfqpoint{0.517061in}{0.456009in}}%
\pgfpathlineto{\pgfqpoint{0.503792in}{0.448413in}}%
\pgfpathlineto{\pgfqpoint{0.490523in}{0.450953in}}%
\pgfpathclose%
\pgfusepath{fill}%
\end{pgfscope}%
\begin{pgfscope}%
\pgfpathrectangle{\pgfqpoint{0.211875in}{0.211875in}}{\pgfqpoint{1.313625in}{1.279725in}}%
\pgfusepath{clip}%
\pgfsetbuttcap%
\pgfsetroundjoin%
\definecolor{currentfill}{rgb}{0.644838,0.098089,0.355336}%
\pgfsetfillcolor{currentfill}%
\pgfsetlinewidth{0.000000pt}%
\definecolor{currentstroke}{rgb}{0.000000,0.000000,0.000000}%
\pgfsetstrokecolor{currentstroke}%
\pgfsetdash{}{0pt}%
\pgfpathmoveto{\pgfqpoint{0.609943in}{0.442956in}}%
\pgfpathlineto{\pgfqpoint{0.623212in}{0.442383in}}%
\pgfpathlineto{\pgfqpoint{0.631107in}{0.444552in}}%
\pgfpathlineto{\pgfqpoint{0.636481in}{0.447432in}}%
\pgfpathlineto{\pgfqpoint{0.641423in}{0.457479in}}%
\pgfpathlineto{\pgfqpoint{0.642890in}{0.470405in}}%
\pgfpathlineto{\pgfqpoint{0.642547in}{0.483332in}}%
\pgfpathlineto{\pgfqpoint{0.639998in}{0.496258in}}%
\pgfpathlineto{\pgfqpoint{0.636481in}{0.502222in}}%
\pgfpathlineto{\pgfqpoint{0.623212in}{0.508158in}}%
\pgfpathlineto{\pgfqpoint{0.609943in}{0.507288in}}%
\pgfpathlineto{\pgfqpoint{0.598169in}{0.496258in}}%
\pgfpathlineto{\pgfqpoint{0.596674in}{0.490797in}}%
\pgfpathlineto{\pgfqpoint{0.595533in}{0.483332in}}%
\pgfpathlineto{\pgfqpoint{0.595207in}{0.470405in}}%
\pgfpathlineto{\pgfqpoint{0.596440in}{0.457479in}}%
\pgfpathlineto{\pgfqpoint{0.596674in}{0.456568in}}%
\pgfpathlineto{\pgfqpoint{0.605929in}{0.444552in}}%
\pgfpathclose%
\pgfpathmoveto{\pgfqpoint{0.606691in}{0.457479in}}%
\pgfpathlineto{\pgfqpoint{0.602631in}{0.470405in}}%
\pgfpathlineto{\pgfqpoint{0.603171in}{0.483332in}}%
\pgfpathlineto{\pgfqpoint{0.609040in}{0.496258in}}%
\pgfpathlineto{\pgfqpoint{0.609943in}{0.497105in}}%
\pgfpathlineto{\pgfqpoint{0.623212in}{0.499087in}}%
\pgfpathlineto{\pgfqpoint{0.627867in}{0.496258in}}%
\pgfpathlineto{\pgfqpoint{0.636247in}{0.483332in}}%
\pgfpathlineto{\pgfqpoint{0.636481in}{0.479434in}}%
\pgfpathlineto{\pgfqpoint{0.636777in}{0.470405in}}%
\pgfpathlineto{\pgfqpoint{0.636481in}{0.468928in}}%
\pgfpathlineto{\pgfqpoint{0.631138in}{0.457479in}}%
\pgfpathlineto{\pgfqpoint{0.623212in}{0.451914in}}%
\pgfpathlineto{\pgfqpoint{0.609943in}{0.453919in}}%
\pgfpathclose%
\pgfusepath{fill}%
\end{pgfscope}%
\begin{pgfscope}%
\pgfpathrectangle{\pgfqpoint{0.211875in}{0.211875in}}{\pgfqpoint{1.313625in}{1.279725in}}%
\pgfusepath{clip}%
\pgfsetbuttcap%
\pgfsetroundjoin%
\definecolor{currentfill}{rgb}{0.644838,0.098089,0.355336}%
\pgfsetfillcolor{currentfill}%
\pgfsetlinewidth{0.000000pt}%
\definecolor{currentstroke}{rgb}{0.000000,0.000000,0.000000}%
\pgfsetstrokecolor{currentstroke}%
\pgfsetdash{}{0pt}%
\pgfpathmoveto{\pgfqpoint{0.729364in}{0.445190in}}%
\pgfpathlineto{\pgfqpoint{0.742633in}{0.444780in}}%
\pgfpathlineto{\pgfqpoint{0.755902in}{0.453765in}}%
\pgfpathlineto{\pgfqpoint{0.757498in}{0.457479in}}%
\pgfpathlineto{\pgfqpoint{0.759383in}{0.470405in}}%
\pgfpathlineto{\pgfqpoint{0.759027in}{0.483332in}}%
\pgfpathlineto{\pgfqpoint{0.755999in}{0.496258in}}%
\pgfpathlineto{\pgfqpoint{0.755902in}{0.496445in}}%
\pgfpathlineto{\pgfqpoint{0.742633in}{0.505532in}}%
\pgfpathlineto{\pgfqpoint{0.729364in}{0.505052in}}%
\pgfpathlineto{\pgfqpoint{0.718889in}{0.496258in}}%
\pgfpathlineto{\pgfqpoint{0.716095in}{0.488253in}}%
\pgfpathlineto{\pgfqpoint{0.715200in}{0.483332in}}%
\pgfpathlineto{\pgfqpoint{0.714851in}{0.470405in}}%
\pgfpathlineto{\pgfqpoint{0.716095in}{0.460497in}}%
\pgfpathlineto{\pgfqpoint{0.716795in}{0.457479in}}%
\pgfpathclose%
\pgfpathmoveto{\pgfqpoint{0.727868in}{0.457479in}}%
\pgfpathlineto{\pgfqpoint{0.722804in}{0.470405in}}%
\pgfpathlineto{\pgfqpoint{0.723417in}{0.483332in}}%
\pgfpathlineto{\pgfqpoint{0.729364in}{0.494608in}}%
\pgfpathlineto{\pgfqpoint{0.742191in}{0.496258in}}%
\pgfpathlineto{\pgfqpoint{0.742633in}{0.496291in}}%
\pgfpathlineto{\pgfqpoint{0.742679in}{0.496258in}}%
\pgfpathlineto{\pgfqpoint{0.751049in}{0.483332in}}%
\pgfpathlineto{\pgfqpoint{0.751769in}{0.470405in}}%
\pgfpathlineto{\pgfqpoint{0.745775in}{0.457479in}}%
\pgfpathlineto{\pgfqpoint{0.742633in}{0.454967in}}%
\pgfpathlineto{\pgfqpoint{0.729364in}{0.456017in}}%
\pgfpathclose%
\pgfusepath{fill}%
\end{pgfscope}%
\begin{pgfscope}%
\pgfpathrectangle{\pgfqpoint{0.211875in}{0.211875in}}{\pgfqpoint{1.313625in}{1.279725in}}%
\pgfusepath{clip}%
\pgfsetbuttcap%
\pgfsetroundjoin%
\definecolor{currentfill}{rgb}{0.644838,0.098089,0.355336}%
\pgfsetfillcolor{currentfill}%
\pgfsetlinewidth{0.000000pt}%
\definecolor{currentstroke}{rgb}{0.000000,0.000000,0.000000}%
\pgfsetstrokecolor{currentstroke}%
\pgfsetdash{}{0pt}%
\pgfpathmoveto{\pgfqpoint{0.848784in}{0.446937in}}%
\pgfpathlineto{\pgfqpoint{0.862053in}{0.447140in}}%
\pgfpathlineto{\pgfqpoint{0.873521in}{0.457479in}}%
\pgfpathlineto{\pgfqpoint{0.875322in}{0.463674in}}%
\pgfpathlineto{\pgfqpoint{0.876320in}{0.470405in}}%
\pgfpathlineto{\pgfqpoint{0.875953in}{0.483332in}}%
\pgfpathlineto{\pgfqpoint{0.875322in}{0.486383in}}%
\pgfpathlineto{\pgfqpoint{0.871173in}{0.496258in}}%
\pgfpathlineto{\pgfqpoint{0.862053in}{0.503347in}}%
\pgfpathlineto{\pgfqpoint{0.848784in}{0.503532in}}%
\pgfpathlineto{\pgfqpoint{0.839070in}{0.496258in}}%
\pgfpathlineto{\pgfqpoint{0.835515in}{0.488126in}}%
\pgfpathlineto{\pgfqpoint{0.834504in}{0.483332in}}%
\pgfpathlineto{\pgfqpoint{0.834135in}{0.470405in}}%
\pgfpathlineto{\pgfqpoint{0.835515in}{0.461288in}}%
\pgfpathlineto{\pgfqpoint{0.836666in}{0.457479in}}%
\pgfpathclose%
\pgfpathmoveto{\pgfqpoint{0.848689in}{0.457479in}}%
\pgfpathlineto{\pgfqpoint{0.842612in}{0.470405in}}%
\pgfpathlineto{\pgfqpoint{0.843304in}{0.483332in}}%
\pgfpathlineto{\pgfqpoint{0.848784in}{0.492595in}}%
\pgfpathlineto{\pgfqpoint{0.862053in}{0.492277in}}%
\pgfpathlineto{\pgfqpoint{0.867197in}{0.483332in}}%
\pgfpathlineto{\pgfqpoint{0.867875in}{0.470405in}}%
\pgfpathlineto{\pgfqpoint{0.862053in}{0.457716in}}%
\pgfpathlineto{\pgfqpoint{0.854030in}{0.457479in}}%
\pgfpathlineto{\pgfqpoint{0.848784in}{0.457396in}}%
\pgfpathclose%
\pgfusepath{fill}%
\end{pgfscope}%
\begin{pgfscope}%
\pgfpathrectangle{\pgfqpoint{0.211875in}{0.211875in}}{\pgfqpoint{1.313625in}{1.279725in}}%
\pgfusepath{clip}%
\pgfsetbuttcap%
\pgfsetroundjoin%
\definecolor{currentfill}{rgb}{0.644838,0.098089,0.355336}%
\pgfsetfillcolor{currentfill}%
\pgfsetlinewidth{0.000000pt}%
\definecolor{currentstroke}{rgb}{0.000000,0.000000,0.000000}%
\pgfsetstrokecolor{currentstroke}%
\pgfsetdash{}{0pt}%
\pgfpathmoveto{\pgfqpoint{0.968205in}{0.447978in}}%
\pgfpathlineto{\pgfqpoint{0.981473in}{0.449010in}}%
\pgfpathlineto{\pgfqpoint{0.989853in}{0.457479in}}%
\pgfpathlineto{\pgfqpoint{0.993204in}{0.470405in}}%
\pgfpathlineto{\pgfqpoint{0.992678in}{0.483332in}}%
\pgfpathlineto{\pgfqpoint{0.987617in}{0.496258in}}%
\pgfpathlineto{\pgfqpoint{0.981473in}{0.501597in}}%
\pgfpathlineto{\pgfqpoint{0.968205in}{0.502631in}}%
\pgfpathlineto{\pgfqpoint{0.958590in}{0.496258in}}%
\pgfpathlineto{\pgfqpoint{0.954936in}{0.489525in}}%
\pgfpathlineto{\pgfqpoint{0.953443in}{0.483332in}}%
\pgfpathlineto{\pgfqpoint{0.953058in}{0.470405in}}%
\pgfpathlineto{\pgfqpoint{0.954936in}{0.459835in}}%
\pgfpathlineto{\pgfqpoint{0.955844in}{0.457479in}}%
\pgfpathclose%
\pgfpathmoveto{\pgfqpoint{0.961966in}{0.470405in}}%
\pgfpathlineto{\pgfqpoint{0.962749in}{0.483332in}}%
\pgfpathlineto{\pgfqpoint{0.968205in}{0.491486in}}%
\pgfpathlineto{\pgfqpoint{0.981473in}{0.488846in}}%
\pgfpathlineto{\pgfqpoint{0.984306in}{0.483332in}}%
\pgfpathlineto{\pgfqpoint{0.984945in}{0.470405in}}%
\pgfpathlineto{\pgfqpoint{0.981473in}{0.461900in}}%
\pgfpathlineto{\pgfqpoint{0.968205in}{0.458748in}}%
\pgfpathclose%
\pgfusepath{fill}%
\end{pgfscope}%
\begin{pgfscope}%
\pgfpathrectangle{\pgfqpoint{0.211875in}{0.211875in}}{\pgfqpoint{1.313625in}{1.279725in}}%
\pgfusepath{clip}%
\pgfsetbuttcap%
\pgfsetroundjoin%
\definecolor{currentfill}{rgb}{0.644838,0.098089,0.355336}%
\pgfsetfillcolor{currentfill}%
\pgfsetlinewidth{0.000000pt}%
\definecolor{currentstroke}{rgb}{0.000000,0.000000,0.000000}%
\pgfsetstrokecolor{currentstroke}%
\pgfsetdash{}{0pt}%
\pgfpathmoveto{\pgfqpoint{1.074356in}{0.457206in}}%
\pgfpathlineto{\pgfqpoint{1.087625in}{0.448392in}}%
\pgfpathlineto{\pgfqpoint{1.100894in}{0.450376in}}%
\pgfpathlineto{\pgfqpoint{1.107187in}{0.457479in}}%
\pgfpathlineto{\pgfqpoint{1.110445in}{0.470405in}}%
\pgfpathlineto{\pgfqpoint{1.109949in}{0.483332in}}%
\pgfpathlineto{\pgfqpoint{1.105067in}{0.496258in}}%
\pgfpathlineto{\pgfqpoint{1.100894in}{0.500297in}}%
\pgfpathlineto{\pgfqpoint{1.087625in}{0.502287in}}%
\pgfpathlineto{\pgfqpoint{1.077239in}{0.496258in}}%
\pgfpathlineto{\pgfqpoint{1.074356in}{0.491978in}}%
\pgfpathlineto{\pgfqpoint{1.072005in}{0.483332in}}%
\pgfpathlineto{\pgfqpoint{1.071605in}{0.470405in}}%
\pgfpathlineto{\pgfqpoint{1.074218in}{0.457479in}}%
\pgfpathclose%
\pgfpathmoveto{\pgfqpoint{1.080720in}{0.470405in}}%
\pgfpathlineto{\pgfqpoint{1.081611in}{0.483332in}}%
\pgfpathlineto{\pgfqpoint{1.087625in}{0.491191in}}%
\pgfpathlineto{\pgfqpoint{1.100894in}{0.486023in}}%
\pgfpathlineto{\pgfqpoint{1.102134in}{0.483332in}}%
\pgfpathlineto{\pgfqpoint{1.102737in}{0.470405in}}%
\pgfpathlineto{\pgfqpoint{1.100894in}{0.465347in}}%
\pgfpathlineto{\pgfqpoint{1.087625in}{0.459148in}}%
\pgfpathclose%
\pgfusepath{fill}%
\end{pgfscope}%
\begin{pgfscope}%
\pgfpathrectangle{\pgfqpoint{0.211875in}{0.211875in}}{\pgfqpoint{1.313625in}{1.279725in}}%
\pgfusepath{clip}%
\pgfsetbuttcap%
\pgfsetroundjoin%
\definecolor{currentfill}{rgb}{0.644838,0.098089,0.355336}%
\pgfsetfillcolor{currentfill}%
\pgfsetlinewidth{0.000000pt}%
\definecolor{currentstroke}{rgb}{0.000000,0.000000,0.000000}%
\pgfsetstrokecolor{currentstroke}%
\pgfsetdash{}{0pt}%
\pgfpathmoveto{\pgfqpoint{1.193777in}{0.455149in}}%
\pgfpathlineto{\pgfqpoint{1.207045in}{0.448223in}}%
\pgfpathlineto{\pgfqpoint{1.220314in}{0.451190in}}%
\pgfpathlineto{\pgfqpoint{1.225309in}{0.457479in}}%
\pgfpathlineto{\pgfqpoint{1.228363in}{0.470405in}}%
\pgfpathlineto{\pgfqpoint{1.227896in}{0.483332in}}%
\pgfpathlineto{\pgfqpoint{1.223312in}{0.496258in}}%
\pgfpathlineto{\pgfqpoint{1.220314in}{0.499481in}}%
\pgfpathlineto{\pgfqpoint{1.207045in}{0.502459in}}%
\pgfpathlineto{\pgfqpoint{1.194653in}{0.496258in}}%
\pgfpathlineto{\pgfqpoint{1.193777in}{0.495218in}}%
\pgfpathlineto{\pgfqpoint{1.190156in}{0.483332in}}%
\pgfpathlineto{\pgfqpoint{1.189745in}{0.470405in}}%
\pgfpathlineto{\pgfqpoint{1.192447in}{0.457479in}}%
\pgfpathclose%
\pgfpathmoveto{\pgfqpoint{1.198614in}{0.470405in}}%
\pgfpathlineto{\pgfqpoint{1.199640in}{0.483332in}}%
\pgfpathlineto{\pgfqpoint{1.207045in}{0.491657in}}%
\pgfpathlineto{\pgfqpoint{1.220314in}{0.483841in}}%
\pgfpathlineto{\pgfqpoint{1.220525in}{0.483332in}}%
\pgfpathlineto{\pgfqpoint{1.221093in}{0.470405in}}%
\pgfpathlineto{\pgfqpoint{1.220314in}{0.468014in}}%
\pgfpathlineto{\pgfqpoint{1.207045in}{0.458599in}}%
\pgfpathclose%
\pgfusepath{fill}%
\end{pgfscope}%
\begin{pgfscope}%
\pgfpathrectangle{\pgfqpoint{0.211875in}{0.211875in}}{\pgfqpoint{1.313625in}{1.279725in}}%
\pgfusepath{clip}%
\pgfsetbuttcap%
\pgfsetroundjoin%
\definecolor{currentfill}{rgb}{0.644838,0.098089,0.355336}%
\pgfsetfillcolor{currentfill}%
\pgfsetlinewidth{0.000000pt}%
\definecolor{currentstroke}{rgb}{0.000000,0.000000,0.000000}%
\pgfsetstrokecolor{currentstroke}%
\pgfsetdash{}{0pt}%
\pgfpathmoveto{\pgfqpoint{1.313197in}{0.452663in}}%
\pgfpathlineto{\pgfqpoint{1.326466in}{0.447492in}}%
\pgfpathlineto{\pgfqpoint{1.339735in}{0.451372in}}%
\pgfpathlineto{\pgfqpoint{1.344081in}{0.457479in}}%
\pgfpathlineto{\pgfqpoint{1.346831in}{0.470405in}}%
\pgfpathlineto{\pgfqpoint{1.346393in}{0.483332in}}%
\pgfpathlineto{\pgfqpoint{1.342211in}{0.496258in}}%
\pgfpathlineto{\pgfqpoint{1.339735in}{0.499216in}}%
\pgfpathlineto{\pgfqpoint{1.326466in}{0.503133in}}%
\pgfpathlineto{\pgfqpoint{1.313197in}{0.497971in}}%
\pgfpathlineto{\pgfqpoint{1.311900in}{0.496258in}}%
\pgfpathlineto{\pgfqpoint{1.307847in}{0.483332in}}%
\pgfpathlineto{\pgfqpoint{1.307426in}{0.470405in}}%
\pgfpathlineto{\pgfqpoint{1.310110in}{0.457479in}}%
\pgfpathclose%
\pgfpathmoveto{\pgfqpoint{1.326081in}{0.457479in}}%
\pgfpathlineto{\pgfqpoint{1.315171in}{0.470405in}}%
\pgfpathlineto{\pgfqpoint{1.316372in}{0.483332in}}%
\pgfpathlineto{\pgfqpoint{1.326466in}{0.492863in}}%
\pgfpathlineto{\pgfqpoint{1.338755in}{0.483332in}}%
\pgfpathlineto{\pgfqpoint{1.339735in}{0.474675in}}%
\pgfpathlineto{\pgfqpoint{1.339910in}{0.470405in}}%
\pgfpathlineto{\pgfqpoint{1.339735in}{0.469805in}}%
\pgfpathlineto{\pgfqpoint{1.326938in}{0.457479in}}%
\pgfpathlineto{\pgfqpoint{1.326466in}{0.457291in}}%
\pgfpathclose%
\pgfusepath{fill}%
\end{pgfscope}%
\begin{pgfscope}%
\pgfpathrectangle{\pgfqpoint{0.211875in}{0.211875in}}{\pgfqpoint{1.313625in}{1.279725in}}%
\pgfusepath{clip}%
\pgfsetbuttcap%
\pgfsetroundjoin%
\definecolor{currentfill}{rgb}{0.644838,0.098089,0.355336}%
\pgfsetfillcolor{currentfill}%
\pgfsetlinewidth{0.000000pt}%
\definecolor{currentstroke}{rgb}{0.000000,0.000000,0.000000}%
\pgfsetstrokecolor{currentstroke}%
\pgfsetdash{}{0pt}%
\pgfpathmoveto{\pgfqpoint{1.432617in}{0.449808in}}%
\pgfpathlineto{\pgfqpoint{1.445886in}{0.446195in}}%
\pgfpathlineto{\pgfqpoint{1.459155in}{0.450787in}}%
\pgfpathlineto{\pgfqpoint{1.463410in}{0.457479in}}%
\pgfpathlineto{\pgfqpoint{1.465764in}{0.470405in}}%
\pgfpathlineto{\pgfqpoint{1.465357in}{0.483332in}}%
\pgfpathlineto{\pgfqpoint{1.461676in}{0.496258in}}%
\pgfpathlineto{\pgfqpoint{1.459155in}{0.499609in}}%
\pgfpathlineto{\pgfqpoint{1.445886in}{0.504311in}}%
\pgfpathlineto{\pgfqpoint{1.432617in}{0.500641in}}%
\pgfpathlineto{\pgfqpoint{1.428930in}{0.496258in}}%
\pgfpathlineto{\pgfqpoint{1.425001in}{0.483332in}}%
\pgfpathlineto{\pgfqpoint{1.424574in}{0.470405in}}%
\pgfpathlineto{\pgfqpoint{1.427121in}{0.457479in}}%
\pgfpathclose%
\pgfpathmoveto{\pgfqpoint{1.442158in}{0.457479in}}%
\pgfpathlineto{\pgfqpoint{1.432617in}{0.466502in}}%
\pgfpathlineto{\pgfqpoint{1.431468in}{0.470405in}}%
\pgfpathlineto{\pgfqpoint{1.431990in}{0.483332in}}%
\pgfpathlineto{\pgfqpoint{1.432617in}{0.484996in}}%
\pgfpathlineto{\pgfqpoint{1.445886in}{0.494818in}}%
\pgfpathlineto{\pgfqpoint{1.457915in}{0.483332in}}%
\pgfpathlineto{\pgfqpoint{1.459070in}{0.470405in}}%
\pgfpathlineto{\pgfqpoint{1.448915in}{0.457479in}}%
\pgfpathlineto{\pgfqpoint{1.445886in}{0.456002in}}%
\pgfpathclose%
\pgfusepath{fill}%
\end{pgfscope}%
\begin{pgfscope}%
\pgfpathrectangle{\pgfqpoint{0.211875in}{0.211875in}}{\pgfqpoint{1.313625in}{1.279725in}}%
\pgfusepath{clip}%
\pgfsetbuttcap%
\pgfsetroundjoin%
\definecolor{currentfill}{rgb}{0.644838,0.098089,0.355336}%
\pgfsetfillcolor{currentfill}%
\pgfsetlinewidth{0.000000pt}%
\definecolor{currentstroke}{rgb}{0.000000,0.000000,0.000000}%
\pgfsetstrokecolor{currentstroke}%
\pgfsetdash{}{0pt}%
\pgfpathmoveto{\pgfqpoint{0.304758in}{0.517618in}}%
\pgfpathlineto{\pgfqpoint{0.318027in}{0.517645in}}%
\pgfpathlineto{\pgfqpoint{0.331295in}{0.518084in}}%
\pgfpathlineto{\pgfqpoint{0.344564in}{0.520184in}}%
\pgfpathlineto{\pgfqpoint{0.347541in}{0.522111in}}%
\pgfpathlineto{\pgfqpoint{0.351602in}{0.535038in}}%
\pgfpathlineto{\pgfqpoint{0.352142in}{0.547964in}}%
\pgfpathlineto{\pgfqpoint{0.352044in}{0.560891in}}%
\pgfpathlineto{\pgfqpoint{0.351310in}{0.573817in}}%
\pgfpathlineto{\pgfqpoint{0.348375in}{0.586744in}}%
\pgfpathlineto{\pgfqpoint{0.344564in}{0.591231in}}%
\pgfpathlineto{\pgfqpoint{0.331295in}{0.594774in}}%
\pgfpathlineto{\pgfqpoint{0.318027in}{0.595189in}}%
\pgfpathlineto{\pgfqpoint{0.304758in}{0.593520in}}%
\pgfpathlineto{\pgfqpoint{0.298526in}{0.586744in}}%
\pgfpathlineto{\pgfqpoint{0.296744in}{0.573817in}}%
\pgfpathlineto{\pgfqpoint{0.296274in}{0.560891in}}%
\pgfpathlineto{\pgfqpoint{0.296092in}{0.547964in}}%
\pgfpathlineto{\pgfqpoint{0.296047in}{0.535038in}}%
\pgfpathlineto{\pgfqpoint{0.296341in}{0.522111in}}%
\pgfpathclose%
\pgfpathmoveto{\pgfqpoint{0.305658in}{0.535038in}}%
\pgfpathlineto{\pgfqpoint{0.304758in}{0.536975in}}%
\pgfpathlineto{\pgfqpoint{0.302475in}{0.547964in}}%
\pgfpathlineto{\pgfqpoint{0.302283in}{0.560891in}}%
\pgfpathlineto{\pgfqpoint{0.304182in}{0.573817in}}%
\pgfpathlineto{\pgfqpoint{0.304758in}{0.575355in}}%
\pgfpathlineto{\pgfqpoint{0.317731in}{0.586744in}}%
\pgfpathlineto{\pgfqpoint{0.318027in}{0.586853in}}%
\pgfpathlineto{\pgfqpoint{0.321925in}{0.586744in}}%
\pgfpathlineto{\pgfqpoint{0.331295in}{0.586416in}}%
\pgfpathlineto{\pgfqpoint{0.344111in}{0.573817in}}%
\pgfpathlineto{\pgfqpoint{0.344564in}{0.572211in}}%
\pgfpathlineto{\pgfqpoint{0.346291in}{0.560891in}}%
\pgfpathlineto{\pgfqpoint{0.346047in}{0.547964in}}%
\pgfpathlineto{\pgfqpoint{0.344564in}{0.540703in}}%
\pgfpathlineto{\pgfqpoint{0.342083in}{0.535038in}}%
\pgfpathlineto{\pgfqpoint{0.331295in}{0.527163in}}%
\pgfpathlineto{\pgfqpoint{0.318027in}{0.526729in}}%
\pgfpathclose%
\pgfusepath{fill}%
\end{pgfscope}%
\begin{pgfscope}%
\pgfpathrectangle{\pgfqpoint{0.211875in}{0.211875in}}{\pgfqpoint{1.313625in}{1.279725in}}%
\pgfusepath{clip}%
\pgfsetbuttcap%
\pgfsetroundjoin%
\definecolor{currentfill}{rgb}{0.644838,0.098089,0.355336}%
\pgfsetfillcolor{currentfill}%
\pgfsetlinewidth{0.000000pt}%
\definecolor{currentstroke}{rgb}{0.000000,0.000000,0.000000}%
\pgfsetstrokecolor{currentstroke}%
\pgfsetdash{}{0pt}%
\pgfpathmoveto{\pgfqpoint{0.437447in}{0.521277in}}%
\pgfpathlineto{\pgfqpoint{0.450716in}{0.521926in}}%
\pgfpathlineto{\pgfqpoint{0.451457in}{0.522111in}}%
\pgfpathlineto{\pgfqpoint{0.463985in}{0.529897in}}%
\pgfpathlineto{\pgfqpoint{0.466009in}{0.535038in}}%
\pgfpathlineto{\pgfqpoint{0.467614in}{0.547964in}}%
\pgfpathlineto{\pgfqpoint{0.467657in}{0.560891in}}%
\pgfpathlineto{\pgfqpoint{0.466396in}{0.573817in}}%
\pgfpathlineto{\pgfqpoint{0.463985in}{0.581544in}}%
\pgfpathlineto{\pgfqpoint{0.459766in}{0.586744in}}%
\pgfpathlineto{\pgfqpoint{0.450716in}{0.590998in}}%
\pgfpathlineto{\pgfqpoint{0.437447in}{0.591715in}}%
\pgfpathlineto{\pgfqpoint{0.424178in}{0.587861in}}%
\pgfpathlineto{\pgfqpoint{0.423000in}{0.586744in}}%
\pgfpathlineto{\pgfqpoint{0.418022in}{0.573817in}}%
\pgfpathlineto{\pgfqpoint{0.416907in}{0.560891in}}%
\pgfpathlineto{\pgfqpoint{0.416890in}{0.547964in}}%
\pgfpathlineto{\pgfqpoint{0.418147in}{0.535038in}}%
\pgfpathlineto{\pgfqpoint{0.424178in}{0.524356in}}%
\pgfpathlineto{\pgfqpoint{0.431809in}{0.522111in}}%
\pgfpathclose%
\pgfpathmoveto{\pgfqpoint{0.429821in}{0.535038in}}%
\pgfpathlineto{\pgfqpoint{0.424178in}{0.544562in}}%
\pgfpathlineto{\pgfqpoint{0.423351in}{0.547964in}}%
\pgfpathlineto{\pgfqpoint{0.422991in}{0.560891in}}%
\pgfpathlineto{\pgfqpoint{0.424178in}{0.567581in}}%
\pgfpathlineto{\pgfqpoint{0.426464in}{0.573817in}}%
\pgfpathlineto{\pgfqpoint{0.437447in}{0.582771in}}%
\pgfpathlineto{\pgfqpoint{0.450716in}{0.581313in}}%
\pgfpathlineto{\pgfqpoint{0.457540in}{0.573817in}}%
\pgfpathlineto{\pgfqpoint{0.461114in}{0.560891in}}%
\pgfpathlineto{\pgfqpoint{0.460577in}{0.547964in}}%
\pgfpathlineto{\pgfqpoint{0.454717in}{0.535038in}}%
\pgfpathlineto{\pgfqpoint{0.450716in}{0.531774in}}%
\pgfpathlineto{\pgfqpoint{0.437447in}{0.530518in}}%
\pgfpathclose%
\pgfusepath{fill}%
\end{pgfscope}%
\begin{pgfscope}%
\pgfpathrectangle{\pgfqpoint{0.211875in}{0.211875in}}{\pgfqpoint{1.313625in}{1.279725in}}%
\pgfusepath{clip}%
\pgfsetbuttcap%
\pgfsetroundjoin%
\definecolor{currentfill}{rgb}{0.644838,0.098089,0.355336}%
\pgfsetfillcolor{currentfill}%
\pgfsetlinewidth{0.000000pt}%
\definecolor{currentstroke}{rgb}{0.000000,0.000000,0.000000}%
\pgfsetstrokecolor{currentstroke}%
\pgfsetdash{}{0pt}%
\pgfpathmoveto{\pgfqpoint{0.543598in}{0.529022in}}%
\pgfpathlineto{\pgfqpoint{0.556867in}{0.524457in}}%
\pgfpathlineto{\pgfqpoint{0.570136in}{0.525870in}}%
\pgfpathlineto{\pgfqpoint{0.580229in}{0.535038in}}%
\pgfpathlineto{\pgfqpoint{0.583405in}{0.547335in}}%
\pgfpathlineto{\pgfqpoint{0.583492in}{0.547964in}}%
\pgfpathlineto{\pgfqpoint{0.583654in}{0.560891in}}%
\pgfpathlineto{\pgfqpoint{0.583405in}{0.563120in}}%
\pgfpathlineto{\pgfqpoint{0.581486in}{0.573817in}}%
\pgfpathlineto{\pgfqpoint{0.571583in}{0.586744in}}%
\pgfpathlineto{\pgfqpoint{0.570136in}{0.587496in}}%
\pgfpathlineto{\pgfqpoint{0.556867in}{0.588895in}}%
\pgfpathlineto{\pgfqpoint{0.549318in}{0.586744in}}%
\pgfpathlineto{\pgfqpoint{0.543598in}{0.583524in}}%
\pgfpathlineto{\pgfqpoint{0.538871in}{0.573817in}}%
\pgfpathlineto{\pgfqpoint{0.537184in}{0.560891in}}%
\pgfpathlineto{\pgfqpoint{0.537311in}{0.547964in}}%
\pgfpathlineto{\pgfqpoint{0.539711in}{0.535038in}}%
\pgfpathclose%
\pgfpathmoveto{\pgfqpoint{0.554101in}{0.535038in}}%
\pgfpathlineto{\pgfqpoint{0.544146in}{0.547964in}}%
\pgfpathlineto{\pgfqpoint{0.543598in}{0.555315in}}%
\pgfpathlineto{\pgfqpoint{0.543386in}{0.560891in}}%
\pgfpathlineto{\pgfqpoint{0.543598in}{0.561953in}}%
\pgfpathlineto{\pgfqpoint{0.549011in}{0.573817in}}%
\pgfpathlineto{\pgfqpoint{0.556867in}{0.579408in}}%
\pgfpathlineto{\pgfqpoint{0.570136in}{0.576464in}}%
\pgfpathlineto{\pgfqpoint{0.572301in}{0.573817in}}%
\pgfpathlineto{\pgfqpoint{0.576234in}{0.560891in}}%
\pgfpathlineto{\pgfqpoint{0.575576in}{0.547964in}}%
\pgfpathlineto{\pgfqpoint{0.570136in}{0.536815in}}%
\pgfpathlineto{\pgfqpoint{0.564681in}{0.535038in}}%
\pgfpathlineto{\pgfqpoint{0.556867in}{0.533607in}}%
\pgfpathclose%
\pgfusepath{fill}%
\end{pgfscope}%
\begin{pgfscope}%
\pgfpathrectangle{\pgfqpoint{0.211875in}{0.211875in}}{\pgfqpoint{1.313625in}{1.279725in}}%
\pgfusepath{clip}%
\pgfsetbuttcap%
\pgfsetroundjoin%
\definecolor{currentfill}{rgb}{0.644838,0.098089,0.355336}%
\pgfsetfillcolor{currentfill}%
\pgfsetlinewidth{0.000000pt}%
\definecolor{currentstroke}{rgb}{0.000000,0.000000,0.000000}%
\pgfsetstrokecolor{currentstroke}%
\pgfsetdash{}{0pt}%
\pgfpathmoveto{\pgfqpoint{0.663019in}{0.531954in}}%
\pgfpathlineto{\pgfqpoint{0.676288in}{0.527070in}}%
\pgfpathlineto{\pgfqpoint{0.689557in}{0.529557in}}%
\pgfpathlineto{\pgfqpoint{0.694981in}{0.535038in}}%
\pgfpathlineto{\pgfqpoint{0.699002in}{0.547964in}}%
\pgfpathlineto{\pgfqpoint{0.699322in}{0.560891in}}%
\pgfpathlineto{\pgfqpoint{0.696749in}{0.573817in}}%
\pgfpathlineto{\pgfqpoint{0.689557in}{0.583583in}}%
\pgfpathlineto{\pgfqpoint{0.676288in}{0.586610in}}%
\pgfpathlineto{\pgfqpoint{0.663019in}{0.580603in}}%
\pgfpathlineto{\pgfqpoint{0.659305in}{0.573817in}}%
\pgfpathlineto{\pgfqpoint{0.657112in}{0.560891in}}%
\pgfpathlineto{\pgfqpoint{0.657364in}{0.547964in}}%
\pgfpathlineto{\pgfqpoint{0.660764in}{0.535038in}}%
\pgfpathclose%
\pgfpathmoveto{\pgfqpoint{0.665294in}{0.547964in}}%
\pgfpathlineto{\pgfqpoint{0.663965in}{0.560891in}}%
\pgfpathlineto{\pgfqpoint{0.671593in}{0.573817in}}%
\pgfpathlineto{\pgfqpoint{0.676288in}{0.576686in}}%
\pgfpathlineto{\pgfqpoint{0.684630in}{0.573817in}}%
\pgfpathlineto{\pgfqpoint{0.689557in}{0.569819in}}%
\pgfpathlineto{\pgfqpoint{0.692251in}{0.560891in}}%
\pgfpathlineto{\pgfqpoint{0.691509in}{0.547964in}}%
\pgfpathlineto{\pgfqpoint{0.689557in}{0.543499in}}%
\pgfpathlineto{\pgfqpoint{0.676288in}{0.536713in}}%
\pgfpathclose%
\pgfusepath{fill}%
\end{pgfscope}%
\begin{pgfscope}%
\pgfpathrectangle{\pgfqpoint{0.211875in}{0.211875in}}{\pgfqpoint{1.313625in}{1.279725in}}%
\pgfusepath{clip}%
\pgfsetbuttcap%
\pgfsetroundjoin%
\definecolor{currentfill}{rgb}{0.644838,0.098089,0.355336}%
\pgfsetfillcolor{currentfill}%
\pgfsetlinewidth{0.000000pt}%
\definecolor{currentstroke}{rgb}{0.000000,0.000000,0.000000}%
\pgfsetstrokecolor{currentstroke}%
\pgfsetdash{}{0pt}%
\pgfpathmoveto{\pgfqpoint{0.782439in}{0.533662in}}%
\pgfpathlineto{\pgfqpoint{0.795708in}{0.529098in}}%
\pgfpathlineto{\pgfqpoint{0.808977in}{0.532996in}}%
\pgfpathlineto{\pgfqpoint{0.810793in}{0.535038in}}%
\pgfpathlineto{\pgfqpoint{0.815305in}{0.547964in}}%
\pgfpathlineto{\pgfqpoint{0.815698in}{0.560891in}}%
\pgfpathlineto{\pgfqpoint{0.812911in}{0.573817in}}%
\pgfpathlineto{\pgfqpoint{0.808977in}{0.579748in}}%
\pgfpathlineto{\pgfqpoint{0.795708in}{0.584391in}}%
\pgfpathlineto{\pgfqpoint{0.782439in}{0.578915in}}%
\pgfpathlineto{\pgfqpoint{0.779323in}{0.573817in}}%
\pgfpathlineto{\pgfqpoint{0.776681in}{0.560891in}}%
\pgfpathlineto{\pgfqpoint{0.777040in}{0.547964in}}%
\pgfpathlineto{\pgfqpoint{0.781310in}{0.535038in}}%
\pgfpathclose%
\pgfpathmoveto{\pgfqpoint{0.786202in}{0.547964in}}%
\pgfpathlineto{\pgfqpoint{0.784395in}{0.560891in}}%
\pgfpathlineto{\pgfqpoint{0.794319in}{0.573817in}}%
\pgfpathlineto{\pgfqpoint{0.795708in}{0.574529in}}%
\pgfpathlineto{\pgfqpoint{0.797313in}{0.573817in}}%
\pgfpathlineto{\pgfqpoint{0.808950in}{0.560891in}}%
\pgfpathlineto{\pgfqpoint{0.806797in}{0.547964in}}%
\pgfpathlineto{\pgfqpoint{0.795708in}{0.539821in}}%
\pgfpathclose%
\pgfusepath{fill}%
\end{pgfscope}%
\begin{pgfscope}%
\pgfpathrectangle{\pgfqpoint{0.211875in}{0.211875in}}{\pgfqpoint{1.313625in}{1.279725in}}%
\pgfusepath{clip}%
\pgfsetbuttcap%
\pgfsetroundjoin%
\definecolor{currentfill}{rgb}{0.644838,0.098089,0.355336}%
\pgfsetfillcolor{currentfill}%
\pgfsetlinewidth{0.000000pt}%
\definecolor{currentstroke}{rgb}{0.000000,0.000000,0.000000}%
\pgfsetstrokecolor{currentstroke}%
\pgfsetdash{}{0pt}%
\pgfpathmoveto{\pgfqpoint{0.901860in}{0.534459in}}%
\pgfpathlineto{\pgfqpoint{0.915129in}{0.530593in}}%
\pgfpathlineto{\pgfqpoint{0.926270in}{0.535038in}}%
\pgfpathlineto{\pgfqpoint{0.928398in}{0.536892in}}%
\pgfpathlineto{\pgfqpoint{0.932268in}{0.547964in}}%
\pgfpathlineto{\pgfqpoint{0.932705in}{0.560891in}}%
\pgfpathlineto{\pgfqpoint{0.929800in}{0.573817in}}%
\pgfpathlineto{\pgfqpoint{0.928398in}{0.576169in}}%
\pgfpathlineto{\pgfqpoint{0.915129in}{0.582747in}}%
\pgfpathlineto{\pgfqpoint{0.901860in}{0.578165in}}%
\pgfpathlineto{\pgfqpoint{0.898901in}{0.573817in}}%
\pgfpathlineto{\pgfqpoint{0.895868in}{0.560891in}}%
\pgfpathlineto{\pgfqpoint{0.896315in}{0.547964in}}%
\pgfpathlineto{\pgfqpoint{0.901329in}{0.535038in}}%
\pgfpathclose%
\pgfpathmoveto{\pgfqpoint{0.906761in}{0.547964in}}%
\pgfpathlineto{\pgfqpoint{0.904318in}{0.560891in}}%
\pgfpathlineto{\pgfqpoint{0.915129in}{0.571949in}}%
\pgfpathlineto{\pgfqpoint{0.923449in}{0.560891in}}%
\pgfpathlineto{\pgfqpoint{0.921553in}{0.547964in}}%
\pgfpathlineto{\pgfqpoint{0.915129in}{0.542182in}}%
\pgfpathclose%
\pgfusepath{fill}%
\end{pgfscope}%
\begin{pgfscope}%
\pgfpathrectangle{\pgfqpoint{0.211875in}{0.211875in}}{\pgfqpoint{1.313625in}{1.279725in}}%
\pgfusepath{clip}%
\pgfsetbuttcap%
\pgfsetroundjoin%
\definecolor{currentfill}{rgb}{0.644838,0.098089,0.355336}%
\pgfsetfillcolor{currentfill}%
\pgfsetlinewidth{0.000000pt}%
\definecolor{currentstroke}{rgb}{0.000000,0.000000,0.000000}%
\pgfsetstrokecolor{currentstroke}%
\pgfsetdash{}{0pt}%
\pgfpathmoveto{\pgfqpoint{1.021280in}{0.534544in}}%
\pgfpathlineto{\pgfqpoint{1.034549in}{0.531586in}}%
\pgfpathlineto{\pgfqpoint{1.041881in}{0.535038in}}%
\pgfpathlineto{\pgfqpoint{1.047818in}{0.541649in}}%
\pgfpathlineto{\pgfqpoint{1.049779in}{0.547964in}}%
\pgfpathlineto{\pgfqpoint{1.050235in}{0.560891in}}%
\pgfpathlineto{\pgfqpoint{1.047818in}{0.571882in}}%
\pgfpathlineto{\pgfqpoint{1.046764in}{0.573817in}}%
\pgfpathlineto{\pgfqpoint{1.034549in}{0.581646in}}%
\pgfpathlineto{\pgfqpoint{1.021280in}{0.578160in}}%
\pgfpathlineto{\pgfqpoint{1.017994in}{0.573817in}}%
\pgfpathlineto{\pgfqpoint{1.014624in}{0.560891in}}%
\pgfpathlineto{\pgfqpoint{1.015143in}{0.547964in}}%
\pgfpathlineto{\pgfqpoint{1.020775in}{0.535038in}}%
\pgfpathclose%
\pgfpathmoveto{\pgfqpoint{1.026694in}{0.547964in}}%
\pgfpathlineto{\pgfqpoint{1.023289in}{0.560891in}}%
\pgfpathlineto{\pgfqpoint{1.034549in}{0.569621in}}%
\pgfpathlineto{\pgfqpoint{1.040090in}{0.560891in}}%
\pgfpathlineto{\pgfqpoint{1.038409in}{0.547964in}}%
\pgfpathlineto{\pgfqpoint{1.034549in}{0.543846in}}%
\pgfpathclose%
\pgfusepath{fill}%
\end{pgfscope}%
\begin{pgfscope}%
\pgfpathrectangle{\pgfqpoint{0.211875in}{0.211875in}}{\pgfqpoint{1.313625in}{1.279725in}}%
\pgfusepath{clip}%
\pgfsetbuttcap%
\pgfsetroundjoin%
\definecolor{currentfill}{rgb}{0.644838,0.098089,0.355336}%
\pgfsetfillcolor{currentfill}%
\pgfsetlinewidth{0.000000pt}%
\definecolor{currentstroke}{rgb}{0.000000,0.000000,0.000000}%
\pgfsetstrokecolor{currentstroke}%
\pgfsetdash{}{0pt}%
\pgfpathmoveto{\pgfqpoint{1.140701in}{0.534043in}}%
\pgfpathlineto{\pgfqpoint{1.153970in}{0.532085in}}%
\pgfpathlineto{\pgfqpoint{1.159394in}{0.535038in}}%
\pgfpathlineto{\pgfqpoint{1.167239in}{0.546064in}}%
\pgfpathlineto{\pgfqpoint{1.167758in}{0.547964in}}%
\pgfpathlineto{\pgfqpoint{1.168212in}{0.560891in}}%
\pgfpathlineto{\pgfqpoint{1.167239in}{0.565845in}}%
\pgfpathlineto{\pgfqpoint{1.163756in}{0.573817in}}%
\pgfpathlineto{\pgfqpoint{1.153970in}{0.581080in}}%
\pgfpathlineto{\pgfqpoint{1.140701in}{0.578776in}}%
\pgfpathlineto{\pgfqpoint{1.136527in}{0.573817in}}%
\pgfpathlineto{\pgfqpoint{1.132876in}{0.560891in}}%
\pgfpathlineto{\pgfqpoint{1.133447in}{0.547964in}}%
\pgfpathlineto{\pgfqpoint{1.139567in}{0.535038in}}%
\pgfpathclose%
\pgfpathmoveto{\pgfqpoint{1.145120in}{0.547964in}}%
\pgfpathlineto{\pgfqpoint{1.140701in}{0.558854in}}%
\pgfpathlineto{\pgfqpoint{1.140545in}{0.560891in}}%
\pgfpathlineto{\pgfqpoint{1.140701in}{0.561344in}}%
\pgfpathlineto{\pgfqpoint{1.153970in}{0.568235in}}%
\pgfpathlineto{\pgfqpoint{1.157990in}{0.560891in}}%
\pgfpathlineto{\pgfqpoint{1.156502in}{0.547964in}}%
\pgfpathlineto{\pgfqpoint{1.153970in}{0.544829in}}%
\pgfpathclose%
\pgfusepath{fill}%
\end{pgfscope}%
\begin{pgfscope}%
\pgfpathrectangle{\pgfqpoint{0.211875in}{0.211875in}}{\pgfqpoint{1.313625in}{1.279725in}}%
\pgfusepath{clip}%
\pgfsetbuttcap%
\pgfsetroundjoin%
\definecolor{currentfill}{rgb}{0.644838,0.098089,0.355336}%
\pgfsetfillcolor{currentfill}%
\pgfsetlinewidth{0.000000pt}%
\definecolor{currentstroke}{rgb}{0.000000,0.000000,0.000000}%
\pgfsetstrokecolor{currentstroke}%
\pgfsetdash{}{0pt}%
\pgfpathmoveto{\pgfqpoint{1.260121in}{0.533035in}}%
\pgfpathlineto{\pgfqpoint{1.273390in}{0.532078in}}%
\pgfpathlineto{\pgfqpoint{1.278160in}{0.535038in}}%
\pgfpathlineto{\pgfqpoint{1.285827in}{0.547964in}}%
\pgfpathlineto{\pgfqpoint{1.286532in}{0.560891in}}%
\pgfpathlineto{\pgfqpoint{1.281947in}{0.573817in}}%
\pgfpathlineto{\pgfqpoint{1.273390in}{0.581059in}}%
\pgfpathlineto{\pgfqpoint{1.260121in}{0.579936in}}%
\pgfpathlineto{\pgfqpoint{1.254380in}{0.573817in}}%
\pgfpathlineto{\pgfqpoint{1.250509in}{0.560891in}}%
\pgfpathlineto{\pgfqpoint{1.251110in}{0.547964in}}%
\pgfpathlineto{\pgfqpoint{1.257573in}{0.535038in}}%
\pgfpathclose%
\pgfpathmoveto{\pgfqpoint{1.259839in}{0.547964in}}%
\pgfpathlineto{\pgfqpoint{1.258740in}{0.560891in}}%
\pgfpathlineto{\pgfqpoint{1.260121in}{0.564501in}}%
\pgfpathlineto{\pgfqpoint{1.273390in}{0.567804in}}%
\pgfpathlineto{\pgfqpoint{1.276706in}{0.560891in}}%
\pgfpathlineto{\pgfqpoint{1.275400in}{0.547964in}}%
\pgfpathlineto{\pgfqpoint{1.273390in}{0.545120in}}%
\pgfpathlineto{\pgfqpoint{1.260121in}{0.547460in}}%
\pgfpathclose%
\pgfusepath{fill}%
\end{pgfscope}%
\begin{pgfscope}%
\pgfpathrectangle{\pgfqpoint{0.211875in}{0.211875in}}{\pgfqpoint{1.313625in}{1.279725in}}%
\pgfusepath{clip}%
\pgfsetbuttcap%
\pgfsetroundjoin%
\definecolor{currentfill}{rgb}{0.644838,0.098089,0.355336}%
\pgfsetfillcolor{currentfill}%
\pgfsetlinewidth{0.000000pt}%
\definecolor{currentstroke}{rgb}{0.000000,0.000000,0.000000}%
\pgfsetstrokecolor{currentstroke}%
\pgfsetdash{}{0pt}%
\pgfpathmoveto{\pgfqpoint{1.379542in}{0.531564in}}%
\pgfpathlineto{\pgfqpoint{1.392811in}{0.531532in}}%
\pgfpathlineto{\pgfqpoint{1.397815in}{0.535038in}}%
\pgfpathlineto{\pgfqpoint{1.404340in}{0.547964in}}%
\pgfpathlineto{\pgfqpoint{1.404921in}{0.560891in}}%
\pgfpathlineto{\pgfqpoint{1.400977in}{0.573817in}}%
\pgfpathlineto{\pgfqpoint{1.392811in}{0.581618in}}%
\pgfpathlineto{\pgfqpoint{1.379542in}{0.581593in}}%
\pgfpathlineto{\pgfqpoint{1.371367in}{0.573817in}}%
\pgfpathlineto{\pgfqpoint{1.367347in}{0.560891in}}%
\pgfpathlineto{\pgfqpoint{1.367952in}{0.547964in}}%
\pgfpathlineto{\pgfqpoint{1.374590in}{0.535038in}}%
\pgfpathclose%
\pgfpathmoveto{\pgfqpoint{1.377435in}{0.547964in}}%
\pgfpathlineto{\pgfqpoint{1.376292in}{0.560891in}}%
\pgfpathlineto{\pgfqpoint{1.379542in}{0.568473in}}%
\pgfpathlineto{\pgfqpoint{1.392811in}{0.568383in}}%
\pgfpathlineto{\pgfqpoint{1.395994in}{0.560891in}}%
\pgfpathlineto{\pgfqpoint{1.394864in}{0.547964in}}%
\pgfpathlineto{\pgfqpoint{1.392811in}{0.544675in}}%
\pgfpathlineto{\pgfqpoint{1.379542in}{0.544617in}}%
\pgfpathclose%
\pgfusepath{fill}%
\end{pgfscope}%
\begin{pgfscope}%
\pgfpathrectangle{\pgfqpoint{0.211875in}{0.211875in}}{\pgfqpoint{1.313625in}{1.279725in}}%
\pgfusepath{clip}%
\pgfsetbuttcap%
\pgfsetroundjoin%
\definecolor{currentfill}{rgb}{0.644838,0.098089,0.355336}%
\pgfsetfillcolor{currentfill}%
\pgfsetlinewidth{0.000000pt}%
\definecolor{currentstroke}{rgb}{0.000000,0.000000,0.000000}%
\pgfsetstrokecolor{currentstroke}%
\pgfsetdash{}{0pt}%
\pgfpathmoveto{\pgfqpoint{1.498962in}{0.529651in}}%
\pgfpathlineto{\pgfqpoint{1.512231in}{0.530385in}}%
\pgfpathlineto{\pgfqpoint{1.518153in}{0.535038in}}%
\pgfpathlineto{\pgfqpoint{1.523494in}{0.547964in}}%
\pgfpathlineto{\pgfqpoint{1.523942in}{0.560891in}}%
\pgfpathlineto{\pgfqpoint{1.520641in}{0.573817in}}%
\pgfpathlineto{\pgfqpoint{1.512231in}{0.582820in}}%
\pgfpathlineto{\pgfqpoint{1.498962in}{0.583730in}}%
\pgfpathlineto{\pgfqpoint{1.487205in}{0.573817in}}%
\pgfpathlineto{\pgfqpoint{1.485693in}{0.569632in}}%
\pgfpathlineto{\pgfqpoint{1.484147in}{0.560891in}}%
\pgfpathlineto{\pgfqpoint{1.484493in}{0.547964in}}%
\pgfpathlineto{\pgfqpoint{1.485693in}{0.543021in}}%
\pgfpathlineto{\pgfqpoint{1.490302in}{0.535038in}}%
\pgfpathclose%
\pgfpathmoveto{\pgfqpoint{1.494149in}{0.547964in}}%
\pgfpathlineto{\pgfqpoint{1.492983in}{0.560891in}}%
\pgfpathlineto{\pgfqpoint{1.498962in}{0.573246in}}%
\pgfpathlineto{\pgfqpoint{1.512231in}{0.570076in}}%
\pgfpathlineto{\pgfqpoint{1.515714in}{0.560891in}}%
\pgfpathlineto{\pgfqpoint{1.514759in}{0.547964in}}%
\pgfpathlineto{\pgfqpoint{1.512231in}{0.543413in}}%
\pgfpathlineto{\pgfqpoint{1.498962in}{0.541207in}}%
\pgfpathclose%
\pgfusepath{fill}%
\end{pgfscope}%
\begin{pgfscope}%
\pgfpathrectangle{\pgfqpoint{0.211875in}{0.211875in}}{\pgfqpoint{1.313625in}{1.279725in}}%
\pgfusepath{clip}%
\pgfsetbuttcap%
\pgfsetroundjoin%
\definecolor{currentfill}{rgb}{0.644838,0.098089,0.355336}%
\pgfsetfillcolor{currentfill}%
\pgfsetlinewidth{0.000000pt}%
\definecolor{currentstroke}{rgb}{0.000000,0.000000,0.000000}%
\pgfsetstrokecolor{currentstroke}%
\pgfsetdash{}{0pt}%
\pgfpathmoveto{\pgfqpoint{0.371102in}{0.603713in}}%
\pgfpathlineto{\pgfqpoint{0.384371in}{0.602825in}}%
\pgfpathlineto{\pgfqpoint{0.397640in}{0.605161in}}%
\pgfpathlineto{\pgfqpoint{0.405154in}{0.612597in}}%
\pgfpathlineto{\pgfqpoint{0.408107in}{0.625523in}}%
\pgfpathlineto{\pgfqpoint{0.408577in}{0.638450in}}%
\pgfpathlineto{\pgfqpoint{0.407695in}{0.651377in}}%
\pgfpathlineto{\pgfqpoint{0.403888in}{0.664303in}}%
\pgfpathlineto{\pgfqpoint{0.397640in}{0.670302in}}%
\pgfpathlineto{\pgfqpoint{0.384371in}{0.673011in}}%
\pgfpathlineto{\pgfqpoint{0.371102in}{0.671778in}}%
\pgfpathlineto{\pgfqpoint{0.361794in}{0.664303in}}%
\pgfpathlineto{\pgfqpoint{0.358360in}{0.651377in}}%
\pgfpathlineto{\pgfqpoint{0.357833in}{0.643176in}}%
\pgfpathlineto{\pgfqpoint{0.357603in}{0.638450in}}%
\pgfpathlineto{\pgfqpoint{0.357833in}{0.627794in}}%
\pgfpathlineto{\pgfqpoint{0.357898in}{0.625523in}}%
\pgfpathlineto{\pgfqpoint{0.360317in}{0.612597in}}%
\pgfpathclose%
\pgfpathmoveto{\pgfqpoint{0.378151in}{0.612597in}}%
\pgfpathlineto{\pgfqpoint{0.371102in}{0.615345in}}%
\pgfpathlineto{\pgfqpoint{0.365802in}{0.625523in}}%
\pgfpathlineto{\pgfqpoint{0.364453in}{0.638450in}}%
\pgfpathlineto{\pgfqpoint{0.366205in}{0.651377in}}%
\pgfpathlineto{\pgfqpoint{0.371102in}{0.660323in}}%
\pgfpathlineto{\pgfqpoint{0.381663in}{0.664303in}}%
\pgfpathlineto{\pgfqpoint{0.384371in}{0.664882in}}%
\pgfpathlineto{\pgfqpoint{0.386181in}{0.664303in}}%
\pgfpathlineto{\pgfqpoint{0.397640in}{0.657481in}}%
\pgfpathlineto{\pgfqpoint{0.400532in}{0.651377in}}%
\pgfpathlineto{\pgfqpoint{0.402249in}{0.638450in}}%
\pgfpathlineto{\pgfqpoint{0.400893in}{0.625523in}}%
\pgfpathlineto{\pgfqpoint{0.397640in}{0.618272in}}%
\pgfpathlineto{\pgfqpoint{0.388507in}{0.612597in}}%
\pgfpathlineto{\pgfqpoint{0.384371in}{0.611251in}}%
\pgfpathclose%
\pgfusepath{fill}%
\end{pgfscope}%
\begin{pgfscope}%
\pgfpathrectangle{\pgfqpoint{0.211875in}{0.211875in}}{\pgfqpoint{1.313625in}{1.279725in}}%
\pgfusepath{clip}%
\pgfsetbuttcap%
\pgfsetroundjoin%
\definecolor{currentfill}{rgb}{0.644838,0.098089,0.355336}%
\pgfsetfillcolor{currentfill}%
\pgfsetlinewidth{0.000000pt}%
\definecolor{currentstroke}{rgb}{0.000000,0.000000,0.000000}%
\pgfsetstrokecolor{currentstroke}%
\pgfsetdash{}{0pt}%
\pgfpathmoveto{\pgfqpoint{0.490523in}{0.607520in}}%
\pgfpathlineto{\pgfqpoint{0.503792in}{0.606237in}}%
\pgfpathlineto{\pgfqpoint{0.517061in}{0.610666in}}%
\pgfpathlineto{\pgfqpoint{0.518805in}{0.612597in}}%
\pgfpathlineto{\pgfqpoint{0.523305in}{0.625523in}}%
\pgfpathlineto{\pgfqpoint{0.524088in}{0.638450in}}%
\pgfpathlineto{\pgfqpoint{0.522926in}{0.651377in}}%
\pgfpathlineto{\pgfqpoint{0.517662in}{0.664303in}}%
\pgfpathlineto{\pgfqpoint{0.517061in}{0.664947in}}%
\pgfpathlineto{\pgfqpoint{0.503792in}{0.669712in}}%
\pgfpathlineto{\pgfqpoint{0.490523in}{0.668234in}}%
\pgfpathlineto{\pgfqpoint{0.485081in}{0.664303in}}%
\pgfpathlineto{\pgfqpoint{0.479326in}{0.651377in}}%
\pgfpathlineto{\pgfqpoint{0.478082in}{0.638450in}}%
\pgfpathlineto{\pgfqpoint{0.478866in}{0.625523in}}%
\pgfpathlineto{\pgfqpoint{0.483651in}{0.612597in}}%
\pgfpathclose%
\pgfpathmoveto{\pgfqpoint{0.487175in}{0.625523in}}%
\pgfpathlineto{\pgfqpoint{0.485324in}{0.638450in}}%
\pgfpathlineto{\pgfqpoint{0.487574in}{0.651377in}}%
\pgfpathlineto{\pgfqpoint{0.490523in}{0.656216in}}%
\pgfpathlineto{\pgfqpoint{0.503792in}{0.660462in}}%
\pgfpathlineto{\pgfqpoint{0.514920in}{0.651377in}}%
\pgfpathlineto{\pgfqpoint{0.517061in}{0.644787in}}%
\pgfpathlineto{\pgfqpoint{0.517967in}{0.638450in}}%
\pgfpathlineto{\pgfqpoint{0.517061in}{0.630904in}}%
\pgfpathlineto{\pgfqpoint{0.515583in}{0.625523in}}%
\pgfpathlineto{\pgfqpoint{0.503792in}{0.615552in}}%
\pgfpathlineto{\pgfqpoint{0.490523in}{0.619772in}}%
\pgfpathclose%
\pgfusepath{fill}%
\end{pgfscope}%
\begin{pgfscope}%
\pgfpathrectangle{\pgfqpoint{0.211875in}{0.211875in}}{\pgfqpoint{1.313625in}{1.279725in}}%
\pgfusepath{clip}%
\pgfsetbuttcap%
\pgfsetroundjoin%
\definecolor{currentfill}{rgb}{0.644838,0.098089,0.355336}%
\pgfsetfillcolor{currentfill}%
\pgfsetlinewidth{0.000000pt}%
\definecolor{currentstroke}{rgb}{0.000000,0.000000,0.000000}%
\pgfsetstrokecolor{currentstroke}%
\pgfsetdash{}{0pt}%
\pgfpathmoveto{\pgfqpoint{0.609943in}{0.610331in}}%
\pgfpathlineto{\pgfqpoint{0.623212in}{0.609190in}}%
\pgfpathlineto{\pgfqpoint{0.631016in}{0.612597in}}%
\pgfpathlineto{\pgfqpoint{0.636481in}{0.617955in}}%
\pgfpathlineto{\pgfqpoint{0.639153in}{0.625523in}}%
\pgfpathlineto{\pgfqpoint{0.640183in}{0.638450in}}%
\pgfpathlineto{\pgfqpoint{0.638798in}{0.651377in}}%
\pgfpathlineto{\pgfqpoint{0.636481in}{0.657504in}}%
\pgfpathlineto{\pgfqpoint{0.629143in}{0.664303in}}%
\pgfpathlineto{\pgfqpoint{0.623212in}{0.666849in}}%
\pgfpathlineto{\pgfqpoint{0.609943in}{0.665608in}}%
\pgfpathlineto{\pgfqpoint{0.607933in}{0.664303in}}%
\pgfpathlineto{\pgfqpoint{0.599939in}{0.651377in}}%
\pgfpathlineto{\pgfqpoint{0.598255in}{0.638450in}}%
\pgfpathlineto{\pgfqpoint{0.599473in}{0.625523in}}%
\pgfpathlineto{\pgfqpoint{0.606524in}{0.612597in}}%
\pgfpathclose%
\pgfpathmoveto{\pgfqpoint{0.608293in}{0.625523in}}%
\pgfpathlineto{\pgfqpoint{0.605945in}{0.638450in}}%
\pgfpathlineto{\pgfqpoint{0.608694in}{0.651377in}}%
\pgfpathlineto{\pgfqpoint{0.609943in}{0.653216in}}%
\pgfpathlineto{\pgfqpoint{0.623212in}{0.656350in}}%
\pgfpathlineto{\pgfqpoint{0.628483in}{0.651377in}}%
\pgfpathlineto{\pgfqpoint{0.632401in}{0.638450in}}%
\pgfpathlineto{\pgfqpoint{0.629036in}{0.625523in}}%
\pgfpathlineto{\pgfqpoint{0.623212in}{0.619828in}}%
\pgfpathlineto{\pgfqpoint{0.609943in}{0.622987in}}%
\pgfpathclose%
\pgfusepath{fill}%
\end{pgfscope}%
\begin{pgfscope}%
\pgfpathrectangle{\pgfqpoint{0.211875in}{0.211875in}}{\pgfqpoint{1.313625in}{1.279725in}}%
\pgfusepath{clip}%
\pgfsetbuttcap%
\pgfsetroundjoin%
\definecolor{currentfill}{rgb}{0.644838,0.098089,0.355336}%
\pgfsetfillcolor{currentfill}%
\pgfsetlinewidth{0.000000pt}%
\definecolor{currentstroke}{rgb}{0.000000,0.000000,0.000000}%
\pgfsetstrokecolor{currentstroke}%
\pgfsetdash{}{0pt}%
\pgfpathmoveto{\pgfqpoint{0.729364in}{0.612342in}}%
\pgfpathlineto{\pgfqpoint{0.742633in}{0.611726in}}%
\pgfpathlineto{\pgfqpoint{0.744394in}{0.612597in}}%
\pgfpathlineto{\pgfqpoint{0.755316in}{0.625523in}}%
\pgfpathlineto{\pgfqpoint{0.755902in}{0.628957in}}%
\pgfpathlineto{\pgfqpoint{0.756765in}{0.638450in}}%
\pgfpathlineto{\pgfqpoint{0.755902in}{0.646019in}}%
\pgfpathlineto{\pgfqpoint{0.754764in}{0.651377in}}%
\pgfpathlineto{\pgfqpoint{0.742799in}{0.664303in}}%
\pgfpathlineto{\pgfqpoint{0.742633in}{0.664384in}}%
\pgfpathlineto{\pgfqpoint{0.740915in}{0.664303in}}%
\pgfpathlineto{\pgfqpoint{0.729364in}{0.663503in}}%
\pgfpathlineto{\pgfqpoint{0.720169in}{0.651377in}}%
\pgfpathlineto{\pgfqpoint{0.718049in}{0.638450in}}%
\pgfpathlineto{\pgfqpoint{0.719688in}{0.625523in}}%
\pgfpathlineto{\pgfqpoint{0.728934in}{0.612597in}}%
\pgfpathclose%
\pgfpathmoveto{\pgfqpoint{0.729152in}{0.625523in}}%
\pgfpathlineto{\pgfqpoint{0.726305in}{0.638450in}}%
\pgfpathlineto{\pgfqpoint{0.729364in}{0.650675in}}%
\pgfpathlineto{\pgfqpoint{0.731445in}{0.651377in}}%
\pgfpathlineto{\pgfqpoint{0.742633in}{0.652714in}}%
\pgfpathlineto{\pgfqpoint{0.743877in}{0.651377in}}%
\pgfpathlineto{\pgfqpoint{0.747707in}{0.638450in}}%
\pgfpathlineto{\pgfqpoint{0.744351in}{0.625523in}}%
\pgfpathlineto{\pgfqpoint{0.742633in}{0.623607in}}%
\pgfpathlineto{\pgfqpoint{0.729364in}{0.625233in}}%
\pgfpathclose%
\pgfusepath{fill}%
\end{pgfscope}%
\begin{pgfscope}%
\pgfpathrectangle{\pgfqpoint{0.211875in}{0.211875in}}{\pgfqpoint{1.313625in}{1.279725in}}%
\pgfusepath{clip}%
\pgfsetbuttcap%
\pgfsetroundjoin%
\definecolor{currentfill}{rgb}{0.644838,0.098089,0.355336}%
\pgfsetfillcolor{currentfill}%
\pgfsetlinewidth{0.000000pt}%
\definecolor{currentstroke}{rgb}{0.000000,0.000000,0.000000}%
\pgfsetstrokecolor{currentstroke}%
\pgfsetdash{}{0pt}%
\pgfpathmoveto{\pgfqpoint{0.848784in}{0.614125in}}%
\pgfpathlineto{\pgfqpoint{0.862053in}{0.614373in}}%
\pgfpathlineto{\pgfqpoint{0.870909in}{0.625523in}}%
\pgfpathlineto{\pgfqpoint{0.872962in}{0.638450in}}%
\pgfpathlineto{\pgfqpoint{0.870420in}{0.651377in}}%
\pgfpathlineto{\pgfqpoint{0.862053in}{0.661516in}}%
\pgfpathlineto{\pgfqpoint{0.848784in}{0.661753in}}%
\pgfpathlineto{\pgfqpoint{0.839954in}{0.651377in}}%
\pgfpathlineto{\pgfqpoint{0.837395in}{0.638450in}}%
\pgfpathlineto{\pgfqpoint{0.839447in}{0.625523in}}%
\pgfpathclose%
\pgfpathmoveto{\pgfqpoint{0.846370in}{0.638450in}}%
\pgfpathlineto{\pgfqpoint{0.848784in}{0.647058in}}%
\pgfpathlineto{\pgfqpoint{0.862053in}{0.646405in}}%
\pgfpathlineto{\pgfqpoint{0.864224in}{0.638450in}}%
\pgfpathlineto{\pgfqpoint{0.862053in}{0.629547in}}%
\pgfpathlineto{\pgfqpoint{0.848784in}{0.628817in}}%
\pgfpathclose%
\pgfusepath{fill}%
\end{pgfscope}%
\begin{pgfscope}%
\pgfpathrectangle{\pgfqpoint{0.211875in}{0.211875in}}{\pgfqpoint{1.313625in}{1.279725in}}%
\pgfusepath{clip}%
\pgfsetbuttcap%
\pgfsetroundjoin%
\definecolor{currentfill}{rgb}{0.644838,0.098089,0.355336}%
\pgfsetfillcolor{currentfill}%
\pgfsetlinewidth{0.000000pt}%
\definecolor{currentstroke}{rgb}{0.000000,0.000000,0.000000}%
\pgfsetstrokecolor{currentstroke}%
\pgfsetdash{}{0pt}%
\pgfpathmoveto{\pgfqpoint{0.968205in}{0.615216in}}%
\pgfpathlineto{\pgfqpoint{0.981473in}{0.616831in}}%
\pgfpathlineto{\pgfqpoint{0.987627in}{0.625523in}}%
\pgfpathlineto{\pgfqpoint{0.989660in}{0.638450in}}%
\pgfpathlineto{\pgfqpoint{0.987184in}{0.651377in}}%
\pgfpathlineto{\pgfqpoint{0.981473in}{0.659125in}}%
\pgfpathlineto{\pgfqpoint{0.968205in}{0.660745in}}%
\pgfpathlineto{\pgfqpoint{0.959188in}{0.651377in}}%
\pgfpathlineto{\pgfqpoint{0.956175in}{0.638450in}}%
\pgfpathlineto{\pgfqpoint{0.958640in}{0.625523in}}%
\pgfpathclose%
\pgfpathmoveto{\pgfqpoint{0.966077in}{0.638450in}}%
\pgfpathlineto{\pgfqpoint{0.968205in}{0.645161in}}%
\pgfpathlineto{\pgfqpoint{0.981473in}{0.638988in}}%
\pgfpathlineto{\pgfqpoint{0.981605in}{0.638450in}}%
\pgfpathlineto{\pgfqpoint{0.981473in}{0.637845in}}%
\pgfpathlineto{\pgfqpoint{0.968205in}{0.630972in}}%
\pgfpathclose%
\pgfusepath{fill}%
\end{pgfscope}%
\begin{pgfscope}%
\pgfpathrectangle{\pgfqpoint{0.211875in}{0.211875in}}{\pgfqpoint{1.313625in}{1.279725in}}%
\pgfusepath{clip}%
\pgfsetbuttcap%
\pgfsetroundjoin%
\definecolor{currentfill}{rgb}{0.644838,0.098089,0.355336}%
\pgfsetfillcolor{currentfill}%
\pgfsetlinewidth{0.000000pt}%
\definecolor{currentstroke}{rgb}{0.000000,0.000000,0.000000}%
\pgfsetstrokecolor{currentstroke}%
\pgfsetdash{}{0pt}%
\pgfpathmoveto{\pgfqpoint{1.087625in}{0.615598in}}%
\pgfpathlineto{\pgfqpoint{1.100894in}{0.618742in}}%
\pgfpathlineto{\pgfqpoint{1.105189in}{0.625523in}}%
\pgfpathlineto{\pgfqpoint{1.107149in}{0.638450in}}%
\pgfpathlineto{\pgfqpoint{1.104778in}{0.651377in}}%
\pgfpathlineto{\pgfqpoint{1.100894in}{0.657253in}}%
\pgfpathlineto{\pgfqpoint{1.087625in}{0.660403in}}%
\pgfpathlineto{\pgfqpoint{1.077690in}{0.651377in}}%
\pgfpathlineto{\pgfqpoint{1.074356in}{0.639129in}}%
\pgfpathlineto{\pgfqpoint{1.074267in}{0.638450in}}%
\pgfpathlineto{\pgfqpoint{1.074356in}{0.637624in}}%
\pgfpathlineto{\pgfqpoint{1.077082in}{0.625523in}}%
\pgfpathclose%
\pgfpathmoveto{\pgfqpoint{1.085318in}{0.638450in}}%
\pgfpathlineto{\pgfqpoint{1.087625in}{0.644815in}}%
\pgfpathlineto{\pgfqpoint{1.094956in}{0.638450in}}%
\pgfpathlineto{\pgfqpoint{1.087625in}{0.631379in}}%
\pgfpathclose%
\pgfusepath{fill}%
\end{pgfscope}%
\begin{pgfscope}%
\pgfpathrectangle{\pgfqpoint{0.211875in}{0.211875in}}{\pgfqpoint{1.313625in}{1.279725in}}%
\pgfusepath{clip}%
\pgfsetbuttcap%
\pgfsetroundjoin%
\definecolor{currentfill}{rgb}{0.644838,0.098089,0.355336}%
\pgfsetfillcolor{currentfill}%
\pgfsetlinewidth{0.000000pt}%
\definecolor{currentstroke}{rgb}{0.000000,0.000000,0.000000}%
\pgfsetstrokecolor{currentstroke}%
\pgfsetdash{}{0pt}%
\pgfpathmoveto{\pgfqpoint{1.207045in}{0.615323in}}%
\pgfpathlineto{\pgfqpoint{1.220314in}{0.620065in}}%
\pgfpathlineto{\pgfqpoint{1.223412in}{0.625523in}}%
\pgfpathlineto{\pgfqpoint{1.225252in}{0.638450in}}%
\pgfpathlineto{\pgfqpoint{1.223024in}{0.651377in}}%
\pgfpathlineto{\pgfqpoint{1.220314in}{0.655937in}}%
\pgfpathlineto{\pgfqpoint{1.207045in}{0.660682in}}%
\pgfpathlineto{\pgfqpoint{1.195143in}{0.651377in}}%
\pgfpathlineto{\pgfqpoint{1.193777in}{0.647354in}}%
\pgfpathlineto{\pgfqpoint{1.192478in}{0.638450in}}%
\pgfpathlineto{\pgfqpoint{1.193777in}{0.627852in}}%
\pgfpathlineto{\pgfqpoint{1.194446in}{0.625523in}}%
\pgfpathclose%
\pgfpathmoveto{\pgfqpoint{1.203897in}{0.638450in}}%
\pgfpathlineto{\pgfqpoint{1.207045in}{0.645922in}}%
\pgfpathlineto{\pgfqpoint{1.212989in}{0.638450in}}%
\pgfpathlineto{\pgfqpoint{1.207045in}{0.630165in}}%
\pgfpathclose%
\pgfusepath{fill}%
\end{pgfscope}%
\begin{pgfscope}%
\pgfpathrectangle{\pgfqpoint{0.211875in}{0.211875in}}{\pgfqpoint{1.313625in}{1.279725in}}%
\pgfusepath{clip}%
\pgfsetbuttcap%
\pgfsetroundjoin%
\definecolor{currentfill}{rgb}{0.644838,0.098089,0.355336}%
\pgfsetfillcolor{currentfill}%
\pgfsetlinewidth{0.000000pt}%
\definecolor{currentstroke}{rgb}{0.000000,0.000000,0.000000}%
\pgfsetstrokecolor{currentstroke}%
\pgfsetdash{}{0pt}%
\pgfpathmoveto{\pgfqpoint{1.313197in}{0.622679in}}%
\pgfpathlineto{\pgfqpoint{1.326466in}{0.614412in}}%
\pgfpathlineto{\pgfqpoint{1.339735in}{0.620719in}}%
\pgfpathlineto{\pgfqpoint{1.342176in}{0.625523in}}%
\pgfpathlineto{\pgfqpoint{1.343856in}{0.638450in}}%
\pgfpathlineto{\pgfqpoint{1.341803in}{0.651377in}}%
\pgfpathlineto{\pgfqpoint{1.339735in}{0.655247in}}%
\pgfpathlineto{\pgfqpoint{1.326466in}{0.661563in}}%
\pgfpathlineto{\pgfqpoint{1.313197in}{0.653339in}}%
\pgfpathlineto{\pgfqpoint{1.312246in}{0.651377in}}%
\pgfpathlineto{\pgfqpoint{1.310263in}{0.638450in}}%
\pgfpathlineto{\pgfqpoint{1.311892in}{0.625523in}}%
\pgfpathclose%
\pgfpathmoveto{\pgfqpoint{1.321451in}{0.638450in}}%
\pgfpathlineto{\pgfqpoint{1.326466in}{0.648445in}}%
\pgfpathlineto{\pgfqpoint{1.332565in}{0.638450in}}%
\pgfpathlineto{\pgfqpoint{1.326466in}{0.627377in}}%
\pgfpathclose%
\pgfusepath{fill}%
\end{pgfscope}%
\begin{pgfscope}%
\pgfpathrectangle{\pgfqpoint{0.211875in}{0.211875in}}{\pgfqpoint{1.313625in}{1.279725in}}%
\pgfusepath{clip}%
\pgfsetbuttcap%
\pgfsetroundjoin%
\definecolor{currentfill}{rgb}{0.644838,0.098089,0.355336}%
\pgfsetfillcolor{currentfill}%
\pgfsetlinewidth{0.000000pt}%
\definecolor{currentstroke}{rgb}{0.000000,0.000000,0.000000}%
\pgfsetstrokecolor{currentstroke}%
\pgfsetdash{}{0pt}%
\pgfpathmoveto{\pgfqpoint{1.432617in}{0.618732in}}%
\pgfpathlineto{\pgfqpoint{1.445886in}{0.612860in}}%
\pgfpathlineto{\pgfqpoint{1.459155in}{0.620569in}}%
\pgfpathlineto{\pgfqpoint{1.461404in}{0.625523in}}%
\pgfpathlineto{\pgfqpoint{1.462884in}{0.638450in}}%
\pgfpathlineto{\pgfqpoint{1.461038in}{0.651377in}}%
\pgfpathlineto{\pgfqpoint{1.459155in}{0.655300in}}%
\pgfpathlineto{\pgfqpoint{1.445886in}{0.663051in}}%
\pgfpathlineto{\pgfqpoint{1.432617in}{0.657142in}}%
\pgfpathlineto{\pgfqpoint{1.429509in}{0.651377in}}%
\pgfpathlineto{\pgfqpoint{1.427553in}{0.638450in}}%
\pgfpathlineto{\pgfqpoint{1.429133in}{0.625523in}}%
\pgfpathclose%
\pgfpathmoveto{\pgfqpoint{1.444270in}{0.625523in}}%
\pgfpathlineto{\pgfqpoint{1.437238in}{0.638450in}}%
\pgfpathlineto{\pgfqpoint{1.445165in}{0.651377in}}%
\pgfpathlineto{\pgfqpoint{1.445886in}{0.651761in}}%
\pgfpathlineto{\pgfqpoint{1.446467in}{0.651377in}}%
\pgfpathlineto{\pgfqpoint{1.452801in}{0.638450in}}%
\pgfpathlineto{\pgfqpoint{1.447187in}{0.625523in}}%
\pgfpathlineto{\pgfqpoint{1.445886in}{0.624632in}}%
\pgfpathclose%
\pgfusepath{fill}%
\end{pgfscope}%
\begin{pgfscope}%
\pgfpathrectangle{\pgfqpoint{0.211875in}{0.211875in}}{\pgfqpoint{1.313625in}{1.279725in}}%
\pgfusepath{clip}%
\pgfsetbuttcap%
\pgfsetroundjoin%
\definecolor{currentfill}{rgb}{0.644838,0.098089,0.355336}%
\pgfsetfillcolor{currentfill}%
\pgfsetlinewidth{0.000000pt}%
\definecolor{currentstroke}{rgb}{0.000000,0.000000,0.000000}%
\pgfsetstrokecolor{currentstroke}%
\pgfsetdash{}{0pt}%
\pgfpathmoveto{\pgfqpoint{0.225144in}{0.681387in}}%
\pgfpathlineto{\pgfqpoint{0.233322in}{0.690156in}}%
\pgfpathlineto{\pgfqpoint{0.234508in}{0.703083in}}%
\pgfpathlineto{\pgfqpoint{0.234684in}{0.716009in}}%
\pgfpathlineto{\pgfqpoint{0.234442in}{0.728936in}}%
\pgfpathlineto{\pgfqpoint{0.233478in}{0.741862in}}%
\pgfpathlineto{\pgfqpoint{0.227823in}{0.754789in}}%
\pgfpathlineto{\pgfqpoint{0.225144in}{0.756282in}}%
\pgfpathlineto{\pgfqpoint{0.211875in}{0.758470in}}%
\pgfpathlineto{\pgfqpoint{0.211875in}{0.754789in}}%
\pgfpathlineto{\pgfqpoint{0.211875in}{0.749952in}}%
\pgfpathlineto{\pgfqpoint{0.224619in}{0.741862in}}%
\pgfpathlineto{\pgfqpoint{0.225144in}{0.740989in}}%
\pgfpathlineto{\pgfqpoint{0.228299in}{0.728936in}}%
\pgfpathlineto{\pgfqpoint{0.228845in}{0.716009in}}%
\pgfpathlineto{\pgfqpoint{0.227307in}{0.703083in}}%
\pgfpathlineto{\pgfqpoint{0.225144in}{0.697527in}}%
\pgfpathlineto{\pgfqpoint{0.216441in}{0.690156in}}%
\pgfpathlineto{\pgfqpoint{0.211875in}{0.688400in}}%
\pgfpathlineto{\pgfqpoint{0.211875in}{0.680003in}}%
\pgfpathclose%
\pgfusepath{fill}%
\end{pgfscope}%
\begin{pgfscope}%
\pgfpathrectangle{\pgfqpoint{0.211875in}{0.211875in}}{\pgfqpoint{1.313625in}{1.279725in}}%
\pgfusepath{clip}%
\pgfsetbuttcap%
\pgfsetroundjoin%
\definecolor{currentfill}{rgb}{0.644838,0.098089,0.355336}%
\pgfsetfillcolor{currentfill}%
\pgfsetlinewidth{0.000000pt}%
\definecolor{currentstroke}{rgb}{0.000000,0.000000,0.000000}%
\pgfsetstrokecolor{currentstroke}%
\pgfsetdash{}{0pt}%
\pgfpathmoveto{\pgfqpoint{0.304758in}{0.687473in}}%
\pgfpathlineto{\pgfqpoint{0.318027in}{0.683881in}}%
\pgfpathlineto{\pgfqpoint{0.331295in}{0.684301in}}%
\pgfpathlineto{\pgfqpoint{0.344564in}{0.689837in}}%
\pgfpathlineto{\pgfqpoint{0.344823in}{0.690156in}}%
\pgfpathlineto{\pgfqpoint{0.349047in}{0.703083in}}%
\pgfpathlineto{\pgfqpoint{0.349866in}{0.716009in}}%
\pgfpathlineto{\pgfqpoint{0.349483in}{0.728936in}}%
\pgfpathlineto{\pgfqpoint{0.347337in}{0.741862in}}%
\pgfpathlineto{\pgfqpoint{0.344564in}{0.747426in}}%
\pgfpathlineto{\pgfqpoint{0.331295in}{0.754290in}}%
\pgfpathlineto{\pgfqpoint{0.318027in}{0.754730in}}%
\pgfpathlineto{\pgfqpoint{0.304758in}{0.749804in}}%
\pgfpathlineto{\pgfqpoint{0.300638in}{0.741862in}}%
\pgfpathlineto{\pgfqpoint{0.298883in}{0.728936in}}%
\pgfpathlineto{\pgfqpoint{0.298546in}{0.716009in}}%
\pgfpathlineto{\pgfqpoint{0.299119in}{0.703083in}}%
\pgfpathlineto{\pgfqpoint{0.302404in}{0.690156in}}%
\pgfpathclose%
\pgfpathmoveto{\pgfqpoint{0.307631in}{0.703083in}}%
\pgfpathlineto{\pgfqpoint{0.304758in}{0.714554in}}%
\pgfpathlineto{\pgfqpoint{0.304567in}{0.716009in}}%
\pgfpathlineto{\pgfqpoint{0.304758in}{0.719850in}}%
\pgfpathlineto{\pgfqpoint{0.305485in}{0.728936in}}%
\pgfpathlineto{\pgfqpoint{0.312486in}{0.741862in}}%
\pgfpathlineto{\pgfqpoint{0.318027in}{0.745500in}}%
\pgfpathlineto{\pgfqpoint{0.331295in}{0.745110in}}%
\pgfpathlineto{\pgfqpoint{0.335854in}{0.741862in}}%
\pgfpathlineto{\pgfqpoint{0.342831in}{0.728936in}}%
\pgfpathlineto{\pgfqpoint{0.343865in}{0.716009in}}%
\pgfpathlineto{\pgfqpoint{0.340591in}{0.703083in}}%
\pgfpathlineto{\pgfqpoint{0.331295in}{0.693650in}}%
\pgfpathlineto{\pgfqpoint{0.318027in}{0.693180in}}%
\pgfpathclose%
\pgfusepath{fill}%
\end{pgfscope}%
\begin{pgfscope}%
\pgfpathrectangle{\pgfqpoint{0.211875in}{0.211875in}}{\pgfqpoint{1.313625in}{1.279725in}}%
\pgfusepath{clip}%
\pgfsetbuttcap%
\pgfsetroundjoin%
\definecolor{currentfill}{rgb}{0.644838,0.098089,0.355336}%
\pgfsetfillcolor{currentfill}%
\pgfsetlinewidth{0.000000pt}%
\definecolor{currentstroke}{rgb}{0.000000,0.000000,0.000000}%
\pgfsetstrokecolor{currentstroke}%
\pgfsetdash{}{0pt}%
\pgfpathmoveto{\pgfqpoint{0.437447in}{0.687439in}}%
\pgfpathlineto{\pgfqpoint{0.450716in}{0.688294in}}%
\pgfpathlineto{\pgfqpoint{0.454627in}{0.690156in}}%
\pgfpathlineto{\pgfqpoint{0.463985in}{0.702616in}}%
\pgfpathlineto{\pgfqpoint{0.464117in}{0.703083in}}%
\pgfpathlineto{\pgfqpoint{0.465490in}{0.716009in}}%
\pgfpathlineto{\pgfqpoint{0.464985in}{0.728936in}}%
\pgfpathlineto{\pgfqpoint{0.463985in}{0.734038in}}%
\pgfpathlineto{\pgfqpoint{0.460950in}{0.741862in}}%
\pgfpathlineto{\pgfqpoint{0.450716in}{0.749998in}}%
\pgfpathlineto{\pgfqpoint{0.437447in}{0.750972in}}%
\pgfpathlineto{\pgfqpoint{0.424178in}{0.744302in}}%
\pgfpathlineto{\pgfqpoint{0.422722in}{0.741862in}}%
\pgfpathlineto{\pgfqpoint{0.419700in}{0.728936in}}%
\pgfpathlineto{\pgfqpoint{0.419212in}{0.716009in}}%
\pgfpathlineto{\pgfqpoint{0.420459in}{0.703083in}}%
\pgfpathlineto{\pgfqpoint{0.424178in}{0.693818in}}%
\pgfpathlineto{\pgfqpoint{0.429155in}{0.690156in}}%
\pgfpathclose%
\pgfpathmoveto{\pgfqpoint{0.430663in}{0.703083in}}%
\pgfpathlineto{\pgfqpoint{0.426059in}{0.716009in}}%
\pgfpathlineto{\pgfqpoint{0.427428in}{0.728936in}}%
\pgfpathlineto{\pgfqpoint{0.437241in}{0.741862in}}%
\pgfpathlineto{\pgfqpoint{0.437447in}{0.741982in}}%
\pgfpathlineto{\pgfqpoint{0.438585in}{0.741862in}}%
\pgfpathlineto{\pgfqpoint{0.450716in}{0.739768in}}%
\pgfpathlineto{\pgfqpoint{0.456924in}{0.728936in}}%
\pgfpathlineto{\pgfqpoint{0.458052in}{0.716009in}}%
\pgfpathlineto{\pgfqpoint{0.454248in}{0.703083in}}%
\pgfpathlineto{\pgfqpoint{0.450716in}{0.699071in}}%
\pgfpathlineto{\pgfqpoint{0.437447in}{0.697385in}}%
\pgfpathclose%
\pgfusepath{fill}%
\end{pgfscope}%
\begin{pgfscope}%
\pgfpathrectangle{\pgfqpoint{0.211875in}{0.211875in}}{\pgfqpoint{1.313625in}{1.279725in}}%
\pgfusepath{clip}%
\pgfsetbuttcap%
\pgfsetroundjoin%
\definecolor{currentfill}{rgb}{0.644838,0.098089,0.355336}%
\pgfsetfillcolor{currentfill}%
\pgfsetlinewidth{0.000000pt}%
\definecolor{currentstroke}{rgb}{0.000000,0.000000,0.000000}%
\pgfsetstrokecolor{currentstroke}%
\pgfsetdash{}{0pt}%
\pgfpathmoveto{\pgfqpoint{0.543598in}{0.698259in}}%
\pgfpathlineto{\pgfqpoint{0.556867in}{0.690372in}}%
\pgfpathlineto{\pgfqpoint{0.570136in}{0.692530in}}%
\pgfpathlineto{\pgfqpoint{0.578467in}{0.703083in}}%
\pgfpathlineto{\pgfqpoint{0.580883in}{0.716009in}}%
\pgfpathlineto{\pgfqpoint{0.580084in}{0.728936in}}%
\pgfpathlineto{\pgfqpoint{0.574794in}{0.741862in}}%
\pgfpathlineto{\pgfqpoint{0.570136in}{0.745978in}}%
\pgfpathlineto{\pgfqpoint{0.556867in}{0.747914in}}%
\pgfpathlineto{\pgfqpoint{0.544917in}{0.741862in}}%
\pgfpathlineto{\pgfqpoint{0.543598in}{0.740231in}}%
\pgfpathlineto{\pgfqpoint{0.540163in}{0.728936in}}%
\pgfpathlineto{\pgfqpoint{0.539539in}{0.716009in}}%
\pgfpathlineto{\pgfqpoint{0.541382in}{0.703083in}}%
\pgfpathclose%
\pgfpathmoveto{\pgfqpoint{0.553784in}{0.703083in}}%
\pgfpathlineto{\pgfqpoint{0.547554in}{0.716009in}}%
\pgfpathlineto{\pgfqpoint{0.549326in}{0.728936in}}%
\pgfpathlineto{\pgfqpoint{0.556867in}{0.737548in}}%
\pgfpathlineto{\pgfqpoint{0.570136in}{0.732908in}}%
\pgfpathlineto{\pgfqpoint{0.572180in}{0.728936in}}%
\pgfpathlineto{\pgfqpoint{0.573372in}{0.716009in}}%
\pgfpathlineto{\pgfqpoint{0.570136in}{0.705552in}}%
\pgfpathlineto{\pgfqpoint{0.565889in}{0.703083in}}%
\pgfpathlineto{\pgfqpoint{0.556867in}{0.700824in}}%
\pgfpathclose%
\pgfusepath{fill}%
\end{pgfscope}%
\begin{pgfscope}%
\pgfpathrectangle{\pgfqpoint{0.211875in}{0.211875in}}{\pgfqpoint{1.313625in}{1.279725in}}%
\pgfusepath{clip}%
\pgfsetbuttcap%
\pgfsetroundjoin%
\definecolor{currentfill}{rgb}{0.644838,0.098089,0.355336}%
\pgfsetfillcolor{currentfill}%
\pgfsetlinewidth{0.000000pt}%
\definecolor{currentstroke}{rgb}{0.000000,0.000000,0.000000}%
\pgfsetstrokecolor{currentstroke}%
\pgfsetdash{}{0pt}%
\pgfpathmoveto{\pgfqpoint{0.663019in}{0.700943in}}%
\pgfpathlineto{\pgfqpoint{0.676288in}{0.693299in}}%
\pgfpathlineto{\pgfqpoint{0.689557in}{0.696992in}}%
\pgfpathlineto{\pgfqpoint{0.693875in}{0.703083in}}%
\pgfpathlineto{\pgfqpoint{0.696632in}{0.716009in}}%
\pgfpathlineto{\pgfqpoint{0.695772in}{0.728936in}}%
\pgfpathlineto{\pgfqpoint{0.689887in}{0.741862in}}%
\pgfpathlineto{\pgfqpoint{0.689557in}{0.742186in}}%
\pgfpathlineto{\pgfqpoint{0.676288in}{0.745456in}}%
\pgfpathlineto{\pgfqpoint{0.668025in}{0.741862in}}%
\pgfpathlineto{\pgfqpoint{0.663019in}{0.736943in}}%
\pgfpathlineto{\pgfqpoint{0.660280in}{0.728936in}}%
\pgfpathlineto{\pgfqpoint{0.659533in}{0.716009in}}%
\pgfpathlineto{\pgfqpoint{0.661906in}{0.703083in}}%
\pgfpathclose%
\pgfpathmoveto{\pgfqpoint{0.668975in}{0.716009in}}%
\pgfpathlineto{\pgfqpoint{0.671234in}{0.728936in}}%
\pgfpathlineto{\pgfqpoint{0.676288in}{0.733889in}}%
\pgfpathlineto{\pgfqpoint{0.685398in}{0.728936in}}%
\pgfpathlineto{\pgfqpoint{0.689528in}{0.716009in}}%
\pgfpathlineto{\pgfqpoint{0.676288in}{0.704176in}}%
\pgfpathclose%
\pgfusepath{fill}%
\end{pgfscope}%
\begin{pgfscope}%
\pgfpathrectangle{\pgfqpoint{0.211875in}{0.211875in}}{\pgfqpoint{1.313625in}{1.279725in}}%
\pgfusepath{clip}%
\pgfsetbuttcap%
\pgfsetroundjoin%
\definecolor{currentfill}{rgb}{0.644838,0.098089,0.355336}%
\pgfsetfillcolor{currentfill}%
\pgfsetlinewidth{0.000000pt}%
\definecolor{currentstroke}{rgb}{0.000000,0.000000,0.000000}%
\pgfsetstrokecolor{currentstroke}%
\pgfsetdash{}{0pt}%
\pgfpathmoveto{\pgfqpoint{0.782439in}{0.702383in}}%
\pgfpathlineto{\pgfqpoint{0.795708in}{0.695588in}}%
\pgfpathlineto{\pgfqpoint{0.808977in}{0.701233in}}%
\pgfpathlineto{\pgfqpoint{0.810154in}{0.703083in}}%
\pgfpathlineto{\pgfqpoint{0.813135in}{0.716009in}}%
\pgfpathlineto{\pgfqpoint{0.812237in}{0.728936in}}%
\pgfpathlineto{\pgfqpoint{0.808977in}{0.736764in}}%
\pgfpathlineto{\pgfqpoint{0.800910in}{0.741862in}}%
\pgfpathlineto{\pgfqpoint{0.795708in}{0.743530in}}%
\pgfpathlineto{\pgfqpoint{0.791136in}{0.741862in}}%
\pgfpathlineto{\pgfqpoint{0.782439in}{0.735191in}}%
\pgfpathlineto{\pgfqpoint{0.780048in}{0.728936in}}%
\pgfpathlineto{\pgfqpoint{0.779189in}{0.716009in}}%
\pgfpathlineto{\pgfqpoint{0.782031in}{0.703083in}}%
\pgfpathclose%
\pgfpathmoveto{\pgfqpoint{0.790348in}{0.716009in}}%
\pgfpathlineto{\pgfqpoint{0.793231in}{0.728936in}}%
\pgfpathlineto{\pgfqpoint{0.795708in}{0.730969in}}%
\pgfpathlineto{\pgfqpoint{0.798599in}{0.728936in}}%
\pgfpathlineto{\pgfqpoint{0.801981in}{0.716009in}}%
\pgfpathlineto{\pgfqpoint{0.795708in}{0.708751in}}%
\pgfpathclose%
\pgfusepath{fill}%
\end{pgfscope}%
\begin{pgfscope}%
\pgfpathrectangle{\pgfqpoint{0.211875in}{0.211875in}}{\pgfqpoint{1.313625in}{1.279725in}}%
\pgfusepath{clip}%
\pgfsetbuttcap%
\pgfsetroundjoin%
\definecolor{currentfill}{rgb}{0.644838,0.098089,0.355336}%
\pgfsetfillcolor{currentfill}%
\pgfsetlinewidth{0.000000pt}%
\definecolor{currentstroke}{rgb}{0.000000,0.000000,0.000000}%
\pgfsetstrokecolor{currentstroke}%
\pgfsetdash{}{0pt}%
\pgfpathmoveto{\pgfqpoint{0.901860in}{0.702894in}}%
\pgfpathlineto{\pgfqpoint{0.915129in}{0.697296in}}%
\pgfpathlineto{\pgfqpoint{0.925510in}{0.703083in}}%
\pgfpathlineto{\pgfqpoint{0.928398in}{0.707675in}}%
\pgfpathlineto{\pgfqpoint{0.930243in}{0.716009in}}%
\pgfpathlineto{\pgfqpoint{0.929327in}{0.728936in}}%
\pgfpathlineto{\pgfqpoint{0.928398in}{0.731421in}}%
\pgfpathlineto{\pgfqpoint{0.915706in}{0.741862in}}%
\pgfpathlineto{\pgfqpoint{0.915129in}{0.742088in}}%
\pgfpathlineto{\pgfqpoint{0.914364in}{0.741862in}}%
\pgfpathlineto{\pgfqpoint{0.901860in}{0.734606in}}%
\pgfpathlineto{\pgfqpoint{0.899445in}{0.728936in}}%
\pgfpathlineto{\pgfqpoint{0.898484in}{0.716009in}}%
\pgfpathlineto{\pgfqpoint{0.901737in}{0.703083in}}%
\pgfpathclose%
\pgfpathmoveto{\pgfqpoint{0.911707in}{0.716009in}}%
\pgfpathlineto{\pgfqpoint{0.915129in}{0.727855in}}%
\pgfpathlineto{\pgfqpoint{0.917762in}{0.716009in}}%
\pgfpathlineto{\pgfqpoint{0.915129in}{0.712272in}}%
\pgfpathclose%
\pgfusepath{fill}%
\end{pgfscope}%
\begin{pgfscope}%
\pgfpathrectangle{\pgfqpoint{0.211875in}{0.211875in}}{\pgfqpoint{1.313625in}{1.279725in}}%
\pgfusepath{clip}%
\pgfsetbuttcap%
\pgfsetroundjoin%
\definecolor{currentfill}{rgb}{0.644838,0.098089,0.355336}%
\pgfsetfillcolor{currentfill}%
\pgfsetlinewidth{0.000000pt}%
\definecolor{currentstroke}{rgb}{0.000000,0.000000,0.000000}%
\pgfsetstrokecolor{currentstroke}%
\pgfsetdash{}{0pt}%
\pgfpathmoveto{\pgfqpoint{1.021280in}{0.702676in}}%
\pgfpathlineto{\pgfqpoint{1.034549in}{0.698459in}}%
\pgfpathlineto{\pgfqpoint{1.041559in}{0.703083in}}%
\pgfpathlineto{\pgfqpoint{1.047818in}{0.715839in}}%
\pgfpathlineto{\pgfqpoint{1.047852in}{0.716009in}}%
\pgfpathlineto{\pgfqpoint{1.047818in}{0.716506in}}%
\pgfpathlineto{\pgfqpoint{1.046040in}{0.728936in}}%
\pgfpathlineto{\pgfqpoint{1.034549in}{0.740675in}}%
\pgfpathlineto{\pgfqpoint{1.021280in}{0.734953in}}%
\pgfpathlineto{\pgfqpoint{1.018430in}{0.728936in}}%
\pgfpathlineto{\pgfqpoint{1.017378in}{0.716009in}}%
\pgfpathlineto{\pgfqpoint{1.020984in}{0.703083in}}%
\pgfpathclose%
\pgfpathmoveto{\pgfqpoint{1.033102in}{0.716009in}}%
\pgfpathlineto{\pgfqpoint{1.034549in}{0.719804in}}%
\pgfpathlineto{\pgfqpoint{1.035261in}{0.716009in}}%
\pgfpathlineto{\pgfqpoint{1.034549in}{0.714811in}}%
\pgfpathclose%
\pgfusepath{fill}%
\end{pgfscope}%
\begin{pgfscope}%
\pgfpathrectangle{\pgfqpoint{0.211875in}{0.211875in}}{\pgfqpoint{1.313625in}{1.279725in}}%
\pgfusepath{clip}%
\pgfsetbuttcap%
\pgfsetroundjoin%
\definecolor{currentfill}{rgb}{0.644838,0.098089,0.355336}%
\pgfsetfillcolor{currentfill}%
\pgfsetlinewidth{0.000000pt}%
\definecolor{currentstroke}{rgb}{0.000000,0.000000,0.000000}%
\pgfsetstrokecolor{currentstroke}%
\pgfsetdash{}{0pt}%
\pgfpathmoveto{\pgfqpoint{1.140701in}{0.701854in}}%
\pgfpathlineto{\pgfqpoint{1.153970in}{0.699087in}}%
\pgfpathlineto{\pgfqpoint{1.159199in}{0.703083in}}%
\pgfpathlineto{\pgfqpoint{1.164799in}{0.716009in}}%
\pgfpathlineto{\pgfqpoint{1.163172in}{0.728936in}}%
\pgfpathlineto{\pgfqpoint{1.153970in}{0.739834in}}%
\pgfpathlineto{\pgfqpoint{1.140701in}{0.736078in}}%
\pgfpathlineto{\pgfqpoint{1.136934in}{0.728936in}}%
\pgfpathlineto{\pgfqpoint{1.135801in}{0.716009in}}%
\pgfpathlineto{\pgfqpoint{1.139703in}{0.703083in}}%
\pgfpathclose%
\pgfusepath{fill}%
\end{pgfscope}%
\begin{pgfscope}%
\pgfpathrectangle{\pgfqpoint{0.211875in}{0.211875in}}{\pgfqpoint{1.313625in}{1.279725in}}%
\pgfusepath{clip}%
\pgfsetbuttcap%
\pgfsetroundjoin%
\definecolor{currentfill}{rgb}{0.644838,0.098089,0.355336}%
\pgfsetfillcolor{currentfill}%
\pgfsetlinewidth{0.000000pt}%
\definecolor{currentstroke}{rgb}{0.000000,0.000000,0.000000}%
\pgfsetstrokecolor{currentstroke}%
\pgfsetdash{}{0pt}%
\pgfpathmoveto{\pgfqpoint{1.260121in}{0.700506in}}%
\pgfpathlineto{\pgfqpoint{1.273390in}{0.699169in}}%
\pgfpathlineto{\pgfqpoint{1.277877in}{0.703083in}}%
\pgfpathlineto{\pgfqpoint{1.282786in}{0.716009in}}%
\pgfpathlineto{\pgfqpoint{1.281355in}{0.728936in}}%
\pgfpathlineto{\pgfqpoint{1.273390in}{0.739703in}}%
\pgfpathlineto{\pgfqpoint{1.260121in}{0.737888in}}%
\pgfpathlineto{\pgfqpoint{1.254853in}{0.728936in}}%
\pgfpathlineto{\pgfqpoint{1.253647in}{0.716009in}}%
\pgfpathlineto{\pgfqpoint{1.257782in}{0.703083in}}%
\pgfpathclose%
\pgfusepath{fill}%
\end{pgfscope}%
\begin{pgfscope}%
\pgfpathrectangle{\pgfqpoint{0.211875in}{0.211875in}}{\pgfqpoint{1.313625in}{1.279725in}}%
\pgfusepath{clip}%
\pgfsetbuttcap%
\pgfsetroundjoin%
\definecolor{currentfill}{rgb}{0.644838,0.098089,0.355336}%
\pgfsetfillcolor{currentfill}%
\pgfsetlinewidth{0.000000pt}%
\definecolor{currentstroke}{rgb}{0.000000,0.000000,0.000000}%
\pgfsetstrokecolor{currentstroke}%
\pgfsetdash{}{0pt}%
\pgfpathmoveto{\pgfqpoint{1.379542in}{0.698673in}}%
\pgfpathlineto{\pgfqpoint{1.392811in}{0.698669in}}%
\pgfpathlineto{\pgfqpoint{1.397288in}{0.703083in}}%
\pgfpathlineto{\pgfqpoint{1.401519in}{0.716009in}}%
\pgfpathlineto{\pgfqpoint{1.400270in}{0.728936in}}%
\pgfpathlineto{\pgfqpoint{1.392811in}{0.740328in}}%
\pgfpathlineto{\pgfqpoint{1.379542in}{0.740330in}}%
\pgfpathlineto{\pgfqpoint{1.372023in}{0.728936in}}%
\pgfpathlineto{\pgfqpoint{1.370756in}{0.716009in}}%
\pgfpathlineto{\pgfqpoint{1.375050in}{0.703083in}}%
\pgfpathclose%
\pgfusepath{fill}%
\end{pgfscope}%
\begin{pgfscope}%
\pgfpathrectangle{\pgfqpoint{0.211875in}{0.211875in}}{\pgfqpoint{1.313625in}{1.279725in}}%
\pgfusepath{clip}%
\pgfsetbuttcap%
\pgfsetroundjoin%
\definecolor{currentfill}{rgb}{0.644838,0.098089,0.355336}%
\pgfsetfillcolor{currentfill}%
\pgfsetlinewidth{0.000000pt}%
\definecolor{currentstroke}{rgb}{0.000000,0.000000,0.000000}%
\pgfsetstrokecolor{currentstroke}%
\pgfsetdash{}{0pt}%
\pgfpathmoveto{\pgfqpoint{1.498962in}{0.696373in}}%
\pgfpathlineto{\pgfqpoint{1.512231in}{0.697522in}}%
\pgfpathlineto{\pgfqpoint{1.517255in}{0.703083in}}%
\pgfpathlineto{\pgfqpoint{1.520809in}{0.716009in}}%
\pgfpathlineto{\pgfqpoint{1.519735in}{0.728936in}}%
\pgfpathlineto{\pgfqpoint{1.512231in}{0.741788in}}%
\pgfpathlineto{\pgfqpoint{1.511653in}{0.741862in}}%
\pgfpathlineto{\pgfqpoint{1.498962in}{0.742839in}}%
\pgfpathlineto{\pgfqpoint{1.497356in}{0.741862in}}%
\pgfpathlineto{\pgfqpoint{1.488192in}{0.728936in}}%
\pgfpathlineto{\pgfqpoint{1.486876in}{0.716009in}}%
\pgfpathlineto{\pgfqpoint{1.491240in}{0.703083in}}%
\pgfpathclose%
\pgfpathmoveto{\pgfqpoint{1.496758in}{0.716009in}}%
\pgfpathlineto{\pgfqpoint{1.498593in}{0.728936in}}%
\pgfpathlineto{\pgfqpoint{1.498962in}{0.729430in}}%
\pgfpathlineto{\pgfqpoint{1.501696in}{0.728936in}}%
\pgfpathlineto{\pgfqpoint{1.512231in}{0.718965in}}%
\pgfpathlineto{\pgfqpoint{1.512565in}{0.716009in}}%
\pgfpathlineto{\pgfqpoint{1.512231in}{0.715059in}}%
\pgfpathlineto{\pgfqpoint{1.498962in}{0.711130in}}%
\pgfpathclose%
\pgfusepath{fill}%
\end{pgfscope}%
\begin{pgfscope}%
\pgfpathrectangle{\pgfqpoint{0.211875in}{0.211875in}}{\pgfqpoint{1.313625in}{1.279725in}}%
\pgfusepath{clip}%
\pgfsetbuttcap%
\pgfsetroundjoin%
\definecolor{currentfill}{rgb}{0.644838,0.098089,0.355336}%
\pgfsetfillcolor{currentfill}%
\pgfsetlinewidth{0.000000pt}%
\definecolor{currentstroke}{rgb}{0.000000,0.000000,0.000000}%
\pgfsetstrokecolor{currentstroke}%
\pgfsetdash{}{0pt}%
\pgfpathmoveto{\pgfqpoint{0.251682in}{0.765120in}}%
\pgfpathlineto{\pgfqpoint{0.264951in}{0.764290in}}%
\pgfpathlineto{\pgfqpoint{0.278220in}{0.765872in}}%
\pgfpathlineto{\pgfqpoint{0.282277in}{0.767715in}}%
\pgfpathlineto{\pgfqpoint{0.290116in}{0.780642in}}%
\pgfpathlineto{\pgfqpoint{0.291489in}{0.793380in}}%
\pgfpathlineto{\pgfqpoint{0.291502in}{0.793568in}}%
\pgfpathlineto{\pgfqpoint{0.291489in}{0.797312in}}%
\pgfpathlineto{\pgfqpoint{0.291452in}{0.806495in}}%
\pgfpathlineto{\pgfqpoint{0.289987in}{0.819421in}}%
\pgfpathlineto{\pgfqpoint{0.283166in}{0.832348in}}%
\pgfpathlineto{\pgfqpoint{0.278220in}{0.835083in}}%
\pgfpathlineto{\pgfqpoint{0.264951in}{0.836865in}}%
\pgfpathlineto{\pgfqpoint{0.251682in}{0.835718in}}%
\pgfpathlineto{\pgfqpoint{0.245448in}{0.832348in}}%
\pgfpathlineto{\pgfqpoint{0.240088in}{0.819421in}}%
\pgfpathlineto{\pgfqpoint{0.239027in}{0.806495in}}%
\pgfpathlineto{\pgfqpoint{0.238945in}{0.793568in}}%
\pgfpathlineto{\pgfqpoint{0.239813in}{0.780642in}}%
\pgfpathlineto{\pgfqpoint{0.245600in}{0.767715in}}%
\pgfpathclose%
\pgfpathmoveto{\pgfqpoint{0.249311in}{0.780642in}}%
\pgfpathlineto{\pgfqpoint{0.245847in}{0.793568in}}%
\pgfpathlineto{\pgfqpoint{0.245785in}{0.806495in}}%
\pgfpathlineto{\pgfqpoint{0.248875in}{0.819421in}}%
\pgfpathlineto{\pgfqpoint{0.251682in}{0.823536in}}%
\pgfpathlineto{\pgfqpoint{0.264951in}{0.828368in}}%
\pgfpathlineto{\pgfqpoint{0.278220in}{0.823764in}}%
\pgfpathlineto{\pgfqpoint{0.281368in}{0.819421in}}%
\pgfpathlineto{\pgfqpoint{0.284715in}{0.806495in}}%
\pgfpathlineto{\pgfqpoint{0.284630in}{0.793568in}}%
\pgfpathlineto{\pgfqpoint{0.280852in}{0.780642in}}%
\pgfpathlineto{\pgfqpoint{0.278220in}{0.777308in}}%
\pgfpathlineto{\pgfqpoint{0.264951in}{0.772984in}}%
\pgfpathlineto{\pgfqpoint{0.251682in}{0.777457in}}%
\pgfpathclose%
\pgfusepath{fill}%
\end{pgfscope}%
\begin{pgfscope}%
\pgfpathrectangle{\pgfqpoint{0.211875in}{0.211875in}}{\pgfqpoint{1.313625in}{1.279725in}}%
\pgfusepath{clip}%
\pgfsetbuttcap%
\pgfsetroundjoin%
\definecolor{currentfill}{rgb}{0.644838,0.098089,0.355336}%
\pgfsetfillcolor{currentfill}%
\pgfsetlinewidth{0.000000pt}%
\definecolor{currentstroke}{rgb}{0.000000,0.000000,0.000000}%
\pgfsetstrokecolor{currentstroke}%
\pgfsetdash{}{0pt}%
\pgfpathmoveto{\pgfqpoint{0.371102in}{0.769961in}}%
\pgfpathlineto{\pgfqpoint{0.384371in}{0.768047in}}%
\pgfpathlineto{\pgfqpoint{0.397640in}{0.771794in}}%
\pgfpathlineto{\pgfqpoint{0.403908in}{0.780642in}}%
\pgfpathlineto{\pgfqpoint{0.406272in}{0.793568in}}%
\pgfpathlineto{\pgfqpoint{0.406281in}{0.806495in}}%
\pgfpathlineto{\pgfqpoint{0.404063in}{0.819421in}}%
\pgfpathlineto{\pgfqpoint{0.397640in}{0.829273in}}%
\pgfpathlineto{\pgfqpoint{0.388589in}{0.832348in}}%
\pgfpathlineto{\pgfqpoint{0.384371in}{0.833227in}}%
\pgfpathlineto{\pgfqpoint{0.377201in}{0.832348in}}%
\pgfpathlineto{\pgfqpoint{0.371102in}{0.831194in}}%
\pgfpathlineto{\pgfqpoint{0.362145in}{0.819421in}}%
\pgfpathlineto{\pgfqpoint{0.360040in}{0.806495in}}%
\pgfpathlineto{\pgfqpoint{0.360025in}{0.793568in}}%
\pgfpathlineto{\pgfqpoint{0.362209in}{0.780642in}}%
\pgfpathclose%
\pgfpathmoveto{\pgfqpoint{0.373985in}{0.780642in}}%
\pgfpathlineto{\pgfqpoint{0.371102in}{0.782676in}}%
\pgfpathlineto{\pgfqpoint{0.367219in}{0.793568in}}%
\pgfpathlineto{\pgfqpoint{0.367084in}{0.806495in}}%
\pgfpathlineto{\pgfqpoint{0.371102in}{0.818953in}}%
\pgfpathlineto{\pgfqpoint{0.371667in}{0.819421in}}%
\pgfpathlineto{\pgfqpoint{0.384371in}{0.823973in}}%
\pgfpathlineto{\pgfqpoint{0.393422in}{0.819421in}}%
\pgfpathlineto{\pgfqpoint{0.397640in}{0.814241in}}%
\pgfpathlineto{\pgfqpoint{0.399814in}{0.806495in}}%
\pgfpathlineto{\pgfqpoint{0.399674in}{0.793568in}}%
\pgfpathlineto{\pgfqpoint{0.397640in}{0.786978in}}%
\pgfpathlineto{\pgfqpoint{0.391677in}{0.780642in}}%
\pgfpathlineto{\pgfqpoint{0.384371in}{0.777287in}}%
\pgfpathclose%
\pgfusepath{fill}%
\end{pgfscope}%
\begin{pgfscope}%
\pgfpathrectangle{\pgfqpoint{0.211875in}{0.211875in}}{\pgfqpoint{1.313625in}{1.279725in}}%
\pgfusepath{clip}%
\pgfsetbuttcap%
\pgfsetroundjoin%
\definecolor{currentfill}{rgb}{0.644838,0.098089,0.355336}%
\pgfsetfillcolor{currentfill}%
\pgfsetlinewidth{0.000000pt}%
\definecolor{currentstroke}{rgb}{0.000000,0.000000,0.000000}%
\pgfsetstrokecolor{currentstroke}%
\pgfsetdash{}{0pt}%
\pgfpathmoveto{\pgfqpoint{0.490523in}{0.773883in}}%
\pgfpathlineto{\pgfqpoint{0.503792in}{0.771808in}}%
\pgfpathlineto{\pgfqpoint{0.517061in}{0.778195in}}%
\pgfpathlineto{\pgfqpoint{0.518609in}{0.780642in}}%
\pgfpathlineto{\pgfqpoint{0.521758in}{0.793568in}}%
\pgfpathlineto{\pgfqpoint{0.521816in}{0.806495in}}%
\pgfpathlineto{\pgfqpoint{0.518992in}{0.819421in}}%
\pgfpathlineto{\pgfqpoint{0.517061in}{0.822728in}}%
\pgfpathlineto{\pgfqpoint{0.503792in}{0.829564in}}%
\pgfpathlineto{\pgfqpoint{0.490523in}{0.827281in}}%
\pgfpathlineto{\pgfqpoint{0.483862in}{0.819421in}}%
\pgfpathlineto{\pgfqpoint{0.480750in}{0.806495in}}%
\pgfpathlineto{\pgfqpoint{0.480798in}{0.793568in}}%
\pgfpathlineto{\pgfqpoint{0.484243in}{0.780642in}}%
\pgfpathclose%
\pgfpathmoveto{\pgfqpoint{0.488363in}{0.793568in}}%
\pgfpathlineto{\pgfqpoint{0.488158in}{0.806495in}}%
\pgfpathlineto{\pgfqpoint{0.490523in}{0.813079in}}%
\pgfpathlineto{\pgfqpoint{0.501377in}{0.819421in}}%
\pgfpathlineto{\pgfqpoint{0.503792in}{0.820071in}}%
\pgfpathlineto{\pgfqpoint{0.504888in}{0.819421in}}%
\pgfpathlineto{\pgfqpoint{0.514046in}{0.806495in}}%
\pgfpathlineto{\pgfqpoint{0.513677in}{0.793568in}}%
\pgfpathlineto{\pgfqpoint{0.503792in}{0.781424in}}%
\pgfpathlineto{\pgfqpoint{0.490523in}{0.788150in}}%
\pgfpathclose%
\pgfusepath{fill}%
\end{pgfscope}%
\begin{pgfscope}%
\pgfpathrectangle{\pgfqpoint{0.211875in}{0.211875in}}{\pgfqpoint{1.313625in}{1.279725in}}%
\pgfusepath{clip}%
\pgfsetbuttcap%
\pgfsetroundjoin%
\definecolor{currentfill}{rgb}{0.644838,0.098089,0.355336}%
\pgfsetfillcolor{currentfill}%
\pgfsetlinewidth{0.000000pt}%
\definecolor{currentstroke}{rgb}{0.000000,0.000000,0.000000}%
\pgfsetstrokecolor{currentstroke}%
\pgfsetdash{}{0pt}%
\pgfpathmoveto{\pgfqpoint{0.609943in}{0.776767in}}%
\pgfpathlineto{\pgfqpoint{0.623212in}{0.775089in}}%
\pgfpathlineto{\pgfqpoint{0.632131in}{0.780642in}}%
\pgfpathlineto{\pgfqpoint{0.636481in}{0.788008in}}%
\pgfpathlineto{\pgfqpoint{0.637832in}{0.793568in}}%
\pgfpathlineto{\pgfqpoint{0.637927in}{0.806495in}}%
\pgfpathlineto{\pgfqpoint{0.636481in}{0.812955in}}%
\pgfpathlineto{\pgfqpoint{0.633140in}{0.819421in}}%
\pgfpathlineto{\pgfqpoint{0.623212in}{0.826204in}}%
\pgfpathlineto{\pgfqpoint{0.609943in}{0.824397in}}%
\pgfpathlineto{\pgfqpoint{0.605246in}{0.819421in}}%
\pgfpathlineto{\pgfqpoint{0.601148in}{0.806495in}}%
\pgfpathlineto{\pgfqpoint{0.601256in}{0.793568in}}%
\pgfpathlineto{\pgfqpoint{0.605926in}{0.780642in}}%
\pgfpathclose%
\pgfpathmoveto{\pgfqpoint{0.609289in}{0.793568in}}%
\pgfpathlineto{\pgfqpoint{0.609014in}{0.806495in}}%
\pgfpathlineto{\pgfqpoint{0.609943in}{0.808816in}}%
\pgfpathlineto{\pgfqpoint{0.623212in}{0.814271in}}%
\pgfpathlineto{\pgfqpoint{0.628096in}{0.806495in}}%
\pgfpathlineto{\pgfqpoint{0.627702in}{0.793568in}}%
\pgfpathlineto{\pgfqpoint{0.623212in}{0.787183in}}%
\pgfpathlineto{\pgfqpoint{0.609943in}{0.792099in}}%
\pgfpathclose%
\pgfusepath{fill}%
\end{pgfscope}%
\begin{pgfscope}%
\pgfpathrectangle{\pgfqpoint{0.211875in}{0.211875in}}{\pgfqpoint{1.313625in}{1.279725in}}%
\pgfusepath{clip}%
\pgfsetbuttcap%
\pgfsetroundjoin%
\definecolor{currentfill}{rgb}{0.644838,0.098089,0.355336}%
\pgfsetfillcolor{currentfill}%
\pgfsetlinewidth{0.000000pt}%
\definecolor{currentstroke}{rgb}{0.000000,0.000000,0.000000}%
\pgfsetstrokecolor{currentstroke}%
\pgfsetdash{}{0pt}%
\pgfpathmoveto{\pgfqpoint{0.729364in}{0.778817in}}%
\pgfpathlineto{\pgfqpoint{0.742633in}{0.777934in}}%
\pgfpathlineto{\pgfqpoint{0.746457in}{0.780642in}}%
\pgfpathlineto{\pgfqpoint{0.753426in}{0.793568in}}%
\pgfpathlineto{\pgfqpoint{0.753630in}{0.806495in}}%
\pgfpathlineto{\pgfqpoint{0.747605in}{0.819421in}}%
\pgfpathlineto{\pgfqpoint{0.742633in}{0.823286in}}%
\pgfpathlineto{\pgfqpoint{0.729364in}{0.822342in}}%
\pgfpathlineto{\pgfqpoint{0.726283in}{0.819421in}}%
\pgfpathlineto{\pgfqpoint{0.721206in}{0.806495in}}%
\pgfpathlineto{\pgfqpoint{0.721371in}{0.793568in}}%
\pgfpathlineto{\pgfqpoint{0.727249in}{0.780642in}}%
\pgfpathclose%
\pgfpathmoveto{\pgfqpoint{0.735832in}{0.793568in}}%
\pgfpathlineto{\pgfqpoint{0.732301in}{0.806495in}}%
\pgfpathlineto{\pgfqpoint{0.742633in}{0.808625in}}%
\pgfpathlineto{\pgfqpoint{0.743806in}{0.806495in}}%
\pgfpathlineto{\pgfqpoint{0.743402in}{0.793568in}}%
\pgfpathlineto{\pgfqpoint{0.742633in}{0.792320in}}%
\pgfpathclose%
\pgfusepath{fill}%
\end{pgfscope}%
\begin{pgfscope}%
\pgfpathrectangle{\pgfqpoint{0.211875in}{0.211875in}}{\pgfqpoint{1.313625in}{1.279725in}}%
\pgfusepath{clip}%
\pgfsetbuttcap%
\pgfsetroundjoin%
\definecolor{currentfill}{rgb}{0.644838,0.098089,0.355336}%
\pgfsetfillcolor{currentfill}%
\pgfsetlinewidth{0.000000pt}%
\definecolor{currentstroke}{rgb}{0.000000,0.000000,0.000000}%
\pgfsetstrokecolor{currentstroke}%
\pgfsetdash{}{0pt}%
\pgfpathmoveto{\pgfqpoint{0.848784in}{0.780177in}}%
\pgfpathlineto{\pgfqpoint{0.862053in}{0.780373in}}%
\pgfpathlineto{\pgfqpoint{0.862389in}{0.780642in}}%
\pgfpathlineto{\pgfqpoint{0.869351in}{0.793568in}}%
\pgfpathlineto{\pgfqpoint{0.869571in}{0.806495in}}%
\pgfpathlineto{\pgfqpoint{0.863604in}{0.819421in}}%
\pgfpathlineto{\pgfqpoint{0.862053in}{0.820780in}}%
\pgfpathlineto{\pgfqpoint{0.848784in}{0.820980in}}%
\pgfpathlineto{\pgfqpoint{0.846939in}{0.819421in}}%
\pgfpathlineto{\pgfqpoint{0.840869in}{0.806495in}}%
\pgfpathlineto{\pgfqpoint{0.841088in}{0.793568in}}%
\pgfpathlineto{\pgfqpoint{0.848179in}{0.780642in}}%
\pgfpathclose%
\pgfusepath{fill}%
\end{pgfscope}%
\begin{pgfscope}%
\pgfpathrectangle{\pgfqpoint{0.211875in}{0.211875in}}{\pgfqpoint{1.313625in}{1.279725in}}%
\pgfusepath{clip}%
\pgfsetbuttcap%
\pgfsetroundjoin%
\definecolor{currentfill}{rgb}{0.644838,0.098089,0.355336}%
\pgfsetfillcolor{currentfill}%
\pgfsetlinewidth{0.000000pt}%
\definecolor{currentstroke}{rgb}{0.000000,0.000000,0.000000}%
\pgfsetstrokecolor{currentstroke}%
\pgfsetdash{}{0pt}%
\pgfpathmoveto{\pgfqpoint{1.326466in}{0.780189in}}%
\pgfpathlineto{\pgfqpoint{1.327757in}{0.780642in}}%
\pgfpathlineto{\pgfqpoint{1.339735in}{0.789830in}}%
\pgfpathlineto{\pgfqpoint{1.341038in}{0.793568in}}%
\pgfpathlineto{\pgfqpoint{1.341223in}{0.806495in}}%
\pgfpathlineto{\pgfqpoint{1.339735in}{0.811209in}}%
\pgfpathlineto{\pgfqpoint{1.330571in}{0.819421in}}%
\pgfpathlineto{\pgfqpoint{1.326466in}{0.821009in}}%
\pgfpathlineto{\pgfqpoint{1.323120in}{0.819421in}}%
\pgfpathlineto{\pgfqpoint{1.313197in}{0.807941in}}%
\pgfpathlineto{\pgfqpoint{1.312783in}{0.806495in}}%
\pgfpathlineto{\pgfqpoint{1.312962in}{0.793568in}}%
\pgfpathlineto{\pgfqpoint{1.313197in}{0.792823in}}%
\pgfpathlineto{\pgfqpoint{1.325416in}{0.780642in}}%
\pgfpathclose%
\pgfusepath{fill}%
\end{pgfscope}%
\begin{pgfscope}%
\pgfpathrectangle{\pgfqpoint{0.211875in}{0.211875in}}{\pgfqpoint{1.313625in}{1.279725in}}%
\pgfusepath{clip}%
\pgfsetbuttcap%
\pgfsetroundjoin%
\definecolor{currentfill}{rgb}{0.644838,0.098089,0.355336}%
\pgfsetfillcolor{currentfill}%
\pgfsetlinewidth{0.000000pt}%
\definecolor{currentstroke}{rgb}{0.000000,0.000000,0.000000}%
\pgfsetstrokecolor{currentstroke}%
\pgfsetdash{}{0pt}%
\pgfpathmoveto{\pgfqpoint{1.445886in}{0.778972in}}%
\pgfpathlineto{\pgfqpoint{1.449763in}{0.780642in}}%
\pgfpathlineto{\pgfqpoint{1.459155in}{0.790153in}}%
\pgfpathlineto{\pgfqpoint{1.460218in}{0.793568in}}%
\pgfpathlineto{\pgfqpoint{1.460376in}{0.806495in}}%
\pgfpathlineto{\pgfqpoint{1.459155in}{0.810806in}}%
\pgfpathlineto{\pgfqpoint{1.451841in}{0.819421in}}%
\pgfpathlineto{\pgfqpoint{1.445886in}{0.822252in}}%
\pgfpathlineto{\pgfqpoint{1.438551in}{0.819421in}}%
\pgfpathlineto{\pgfqpoint{1.432617in}{0.814195in}}%
\pgfpathlineto{\pgfqpoint{1.430162in}{0.806495in}}%
\pgfpathlineto{\pgfqpoint{1.430332in}{0.793568in}}%
\pgfpathlineto{\pgfqpoint{1.432617in}{0.787083in}}%
\pgfpathlineto{\pgfqpoint{1.441134in}{0.780642in}}%
\pgfpathclose%
\pgfpathmoveto{\pgfqpoint{1.445772in}{0.793568in}}%
\pgfpathlineto{\pgfqpoint{1.444913in}{0.806495in}}%
\pgfpathlineto{\pgfqpoint{1.445886in}{0.807371in}}%
\pgfpathlineto{\pgfqpoint{1.446666in}{0.806495in}}%
\pgfpathlineto{\pgfqpoint{1.445978in}{0.793568in}}%
\pgfpathlineto{\pgfqpoint{1.445886in}{0.793477in}}%
\pgfpathclose%
\pgfusepath{fill}%
\end{pgfscope}%
\begin{pgfscope}%
\pgfpathrectangle{\pgfqpoint{0.211875in}{0.211875in}}{\pgfqpoint{1.313625in}{1.279725in}}%
\pgfusepath{clip}%
\pgfsetbuttcap%
\pgfsetroundjoin%
\definecolor{currentfill}{rgb}{0.644838,0.098089,0.355336}%
\pgfsetfillcolor{currentfill}%
\pgfsetlinewidth{0.000000pt}%
\definecolor{currentstroke}{rgb}{0.000000,0.000000,0.000000}%
\pgfsetstrokecolor{currentstroke}%
\pgfsetdash{}{0pt}%
\pgfpathmoveto{\pgfqpoint{0.968205in}{0.781152in}}%
\pgfpathlineto{\pgfqpoint{0.981473in}{0.783659in}}%
\pgfpathlineto{\pgfqpoint{0.986289in}{0.793568in}}%
\pgfpathlineto{\pgfqpoint{0.986513in}{0.806495in}}%
\pgfpathlineto{\pgfqpoint{0.981473in}{0.818058in}}%
\pgfpathlineto{\pgfqpoint{0.975230in}{0.819421in}}%
\pgfpathlineto{\pgfqpoint{0.968205in}{0.820216in}}%
\pgfpathlineto{\pgfqpoint{0.967142in}{0.819421in}}%
\pgfpathlineto{\pgfqpoint{0.960039in}{0.806495in}}%
\pgfpathlineto{\pgfqpoint{0.960310in}{0.793568in}}%
\pgfpathclose%
\pgfusepath{fill}%
\end{pgfscope}%
\begin{pgfscope}%
\pgfpathrectangle{\pgfqpoint{0.211875in}{0.211875in}}{\pgfqpoint{1.313625in}{1.279725in}}%
\pgfusepath{clip}%
\pgfsetbuttcap%
\pgfsetroundjoin%
\definecolor{currentfill}{rgb}{0.644838,0.098089,0.355336}%
\pgfsetfillcolor{currentfill}%
\pgfsetlinewidth{0.000000pt}%
\definecolor{currentstroke}{rgb}{0.000000,0.000000,0.000000}%
\pgfsetstrokecolor{currentstroke}%
\pgfsetdash{}{0pt}%
\pgfpathmoveto{\pgfqpoint{1.087625in}{0.781553in}}%
\pgfpathlineto{\pgfqpoint{1.100894in}{0.786460in}}%
\pgfpathlineto{\pgfqpoint{1.103983in}{0.793568in}}%
\pgfpathlineto{\pgfqpoint{1.104201in}{0.806495in}}%
\pgfpathlineto{\pgfqpoint{1.100894in}{0.814961in}}%
\pgfpathlineto{\pgfqpoint{1.090357in}{0.819421in}}%
\pgfpathlineto{\pgfqpoint{1.087625in}{0.819986in}}%
\pgfpathlineto{\pgfqpoint{1.086762in}{0.819421in}}%
\pgfpathlineto{\pgfqpoint{1.078553in}{0.806495in}}%
\pgfpathlineto{\pgfqpoint{1.078875in}{0.793568in}}%
\pgfpathclose%
\pgfusepath{fill}%
\end{pgfscope}%
\begin{pgfscope}%
\pgfpathrectangle{\pgfqpoint{0.211875in}{0.211875in}}{\pgfqpoint{1.313625in}{1.279725in}}%
\pgfusepath{clip}%
\pgfsetbuttcap%
\pgfsetroundjoin%
\definecolor{currentfill}{rgb}{0.644838,0.098089,0.355336}%
\pgfsetfillcolor{currentfill}%
\pgfsetlinewidth{0.000000pt}%
\definecolor{currentstroke}{rgb}{0.000000,0.000000,0.000000}%
\pgfsetstrokecolor{currentstroke}%
\pgfsetdash{}{0pt}%
\pgfpathmoveto{\pgfqpoint{1.207045in}{0.781118in}}%
\pgfpathlineto{\pgfqpoint{1.220314in}{0.788547in}}%
\pgfpathlineto{\pgfqpoint{1.222269in}{0.793568in}}%
\pgfpathlineto{\pgfqpoint{1.222474in}{0.806495in}}%
\pgfpathlineto{\pgfqpoint{1.220314in}{0.812647in}}%
\pgfpathlineto{\pgfqpoint{1.209849in}{0.819421in}}%
\pgfpathlineto{\pgfqpoint{1.207045in}{0.820255in}}%
\pgfpathlineto{\pgfqpoint{1.205566in}{0.819421in}}%
\pgfpathlineto{\pgfqpoint{1.196124in}{0.806495in}}%
\pgfpathlineto{\pgfqpoint{1.196496in}{0.793568in}}%
\pgfpathclose%
\pgfusepath{fill}%
\end{pgfscope}%
\begin{pgfscope}%
\pgfpathrectangle{\pgfqpoint{0.211875in}{0.211875in}}{\pgfqpoint{1.313625in}{1.279725in}}%
\pgfusepath{clip}%
\pgfsetbuttcap%
\pgfsetroundjoin%
\definecolor{currentfill}{rgb}{0.644838,0.098089,0.355336}%
\pgfsetfillcolor{currentfill}%
\pgfsetlinewidth{0.000000pt}%
\definecolor{currentstroke}{rgb}{0.000000,0.000000,0.000000}%
\pgfsetstrokecolor{currentstroke}%
\pgfsetdash{}{0pt}%
\pgfpathmoveto{\pgfqpoint{0.215068in}{0.845274in}}%
\pgfpathlineto{\pgfqpoint{0.225144in}{0.848120in}}%
\pgfpathlineto{\pgfqpoint{0.231397in}{0.858201in}}%
\pgfpathlineto{\pgfqpoint{0.232916in}{0.871127in}}%
\pgfpathlineto{\pgfqpoint{0.233116in}{0.884054in}}%
\pgfpathlineto{\pgfqpoint{0.232452in}{0.896980in}}%
\pgfpathlineto{\pgfqpoint{0.229569in}{0.909907in}}%
\pgfpathlineto{\pgfqpoint{0.225144in}{0.915275in}}%
\pgfpathlineto{\pgfqpoint{0.211875in}{0.918912in}}%
\pgfpathlineto{\pgfqpoint{0.211875in}{0.910596in}}%
\pgfpathlineto{\pgfqpoint{0.213437in}{0.909907in}}%
\pgfpathlineto{\pgfqpoint{0.225144in}{0.898312in}}%
\pgfpathlineto{\pgfqpoint{0.225631in}{0.896980in}}%
\pgfpathlineto{\pgfqpoint{0.227327in}{0.884054in}}%
\pgfpathlineto{\pgfqpoint{0.226572in}{0.871127in}}%
\pgfpathlineto{\pgfqpoint{0.225144in}{0.866011in}}%
\pgfpathlineto{\pgfqpoint{0.219826in}{0.858201in}}%
\pgfpathlineto{\pgfqpoint{0.211875in}{0.853637in}}%
\pgfpathlineto{\pgfqpoint{0.211875in}{0.845274in}}%
\pgfpathlineto{\pgfqpoint{0.211875in}{0.844789in}}%
\pgfpathclose%
\pgfusepath{fill}%
\end{pgfscope}%
\begin{pgfscope}%
\pgfpathrectangle{\pgfqpoint{0.211875in}{0.211875in}}{\pgfqpoint{1.313625in}{1.279725in}}%
\pgfusepath{clip}%
\pgfsetbuttcap%
\pgfsetroundjoin%
\definecolor{currentfill}{rgb}{0.644838,0.098089,0.355336}%
\pgfsetfillcolor{currentfill}%
\pgfsetlinewidth{0.000000pt}%
\definecolor{currentstroke}{rgb}{0.000000,0.000000,0.000000}%
\pgfsetstrokecolor{currentstroke}%
\pgfsetdash{}{0pt}%
\pgfpathmoveto{\pgfqpoint{0.304758in}{0.855301in}}%
\pgfpathlineto{\pgfqpoint{0.318027in}{0.848938in}}%
\pgfpathlineto{\pgfqpoint{0.331295in}{0.849367in}}%
\pgfpathlineto{\pgfqpoint{0.344564in}{0.857644in}}%
\pgfpathlineto{\pgfqpoint{0.344864in}{0.858201in}}%
\pgfpathlineto{\pgfqpoint{0.347881in}{0.871127in}}%
\pgfpathlineto{\pgfqpoint{0.348342in}{0.884054in}}%
\pgfpathlineto{\pgfqpoint{0.347196in}{0.896980in}}%
\pgfpathlineto{\pgfqpoint{0.344564in}{0.905255in}}%
\pgfpathlineto{\pgfqpoint{0.340766in}{0.909907in}}%
\pgfpathlineto{\pgfqpoint{0.331295in}{0.914575in}}%
\pgfpathlineto{\pgfqpoint{0.318027in}{0.914973in}}%
\pgfpathlineto{\pgfqpoint{0.306432in}{0.909907in}}%
\pgfpathlineto{\pgfqpoint{0.304758in}{0.908050in}}%
\pgfpathlineto{\pgfqpoint{0.301148in}{0.896980in}}%
\pgfpathlineto{\pgfqpoint{0.300148in}{0.884054in}}%
\pgfpathlineto{\pgfqpoint{0.300521in}{0.871127in}}%
\pgfpathlineto{\pgfqpoint{0.303125in}{0.858201in}}%
\pgfpathclose%
\pgfpathmoveto{\pgfqpoint{0.317865in}{0.858201in}}%
\pgfpathlineto{\pgfqpoint{0.308325in}{0.871127in}}%
\pgfpathlineto{\pgfqpoint{0.306830in}{0.884054in}}%
\pgfpathlineto{\pgfqpoint{0.310089in}{0.896980in}}%
\pgfpathlineto{\pgfqpoint{0.318027in}{0.905269in}}%
\pgfpathlineto{\pgfqpoint{0.331295in}{0.904805in}}%
\pgfpathlineto{\pgfqpoint{0.338361in}{0.896980in}}%
\pgfpathlineto{\pgfqpoint{0.341632in}{0.884054in}}%
\pgfpathlineto{\pgfqpoint{0.340106in}{0.871127in}}%
\pgfpathlineto{\pgfqpoint{0.331295in}{0.858618in}}%
\pgfpathlineto{\pgfqpoint{0.321349in}{0.858201in}}%
\pgfpathlineto{\pgfqpoint{0.318027in}{0.858106in}}%
\pgfpathclose%
\pgfusepath{fill}%
\end{pgfscope}%
\begin{pgfscope}%
\pgfpathrectangle{\pgfqpoint{0.211875in}{0.211875in}}{\pgfqpoint{1.313625in}{1.279725in}}%
\pgfusepath{clip}%
\pgfsetbuttcap%
\pgfsetroundjoin%
\definecolor{currentfill}{rgb}{0.644838,0.098089,0.355336}%
\pgfsetfillcolor{currentfill}%
\pgfsetlinewidth{0.000000pt}%
\definecolor{currentstroke}{rgb}{0.000000,0.000000,0.000000}%
\pgfsetstrokecolor{currentstroke}%
\pgfsetdash{}{0pt}%
\pgfpathmoveto{\pgfqpoint{0.437447in}{0.852631in}}%
\pgfpathlineto{\pgfqpoint{0.450716in}{0.853680in}}%
\pgfpathlineto{\pgfqpoint{0.457009in}{0.858201in}}%
\pgfpathlineto{\pgfqpoint{0.463037in}{0.871127in}}%
\pgfpathlineto{\pgfqpoint{0.463985in}{0.883825in}}%
\pgfpathlineto{\pgfqpoint{0.463996in}{0.884054in}}%
\pgfpathlineto{\pgfqpoint{0.463985in}{0.884162in}}%
\pgfpathlineto{\pgfqpoint{0.461807in}{0.896980in}}%
\pgfpathlineto{\pgfqpoint{0.451836in}{0.909907in}}%
\pgfpathlineto{\pgfqpoint{0.450716in}{0.910521in}}%
\pgfpathlineto{\pgfqpoint{0.437447in}{0.911535in}}%
\pgfpathlineto{\pgfqpoint{0.433231in}{0.909907in}}%
\pgfpathlineto{\pgfqpoint{0.424178in}{0.901911in}}%
\pgfpathlineto{\pgfqpoint{0.422324in}{0.896980in}}%
\pgfpathlineto{\pgfqpoint{0.420810in}{0.884054in}}%
\pgfpathlineto{\pgfqpoint{0.421457in}{0.871127in}}%
\pgfpathlineto{\pgfqpoint{0.424178in}{0.861604in}}%
\pgfpathlineto{\pgfqpoint{0.426696in}{0.858201in}}%
\pgfpathclose%
\pgfpathmoveto{\pgfqpoint{0.430728in}{0.871127in}}%
\pgfpathlineto{\pgfqpoint{0.428629in}{0.884054in}}%
\pgfpathlineto{\pgfqpoint{0.433063in}{0.896980in}}%
\pgfpathlineto{\pgfqpoint{0.437447in}{0.901020in}}%
\pgfpathlineto{\pgfqpoint{0.450716in}{0.899076in}}%
\pgfpathlineto{\pgfqpoint{0.452408in}{0.896980in}}%
\pgfpathlineto{\pgfqpoint{0.456021in}{0.884054in}}%
\pgfpathlineto{\pgfqpoint{0.454294in}{0.871127in}}%
\pgfpathlineto{\pgfqpoint{0.450716in}{0.865436in}}%
\pgfpathlineto{\pgfqpoint{0.437447in}{0.863148in}}%
\pgfpathclose%
\pgfusepath{fill}%
\end{pgfscope}%
\begin{pgfscope}%
\pgfpathrectangle{\pgfqpoint{0.211875in}{0.211875in}}{\pgfqpoint{1.313625in}{1.279725in}}%
\pgfusepath{clip}%
\pgfsetbuttcap%
\pgfsetroundjoin%
\definecolor{currentfill}{rgb}{0.644838,0.098089,0.355336}%
\pgfsetfillcolor{currentfill}%
\pgfsetlinewidth{0.000000pt}%
\definecolor{currentstroke}{rgb}{0.000000,0.000000,0.000000}%
\pgfsetstrokecolor{currentstroke}%
\pgfsetdash{}{0pt}%
\pgfpathmoveto{\pgfqpoint{0.556867in}{0.855636in}}%
\pgfpathlineto{\pgfqpoint{0.570136in}{0.857747in}}%
\pgfpathlineto{\pgfqpoint{0.570704in}{0.858201in}}%
\pgfpathlineto{\pgfqpoint{0.577796in}{0.871127in}}%
\pgfpathlineto{\pgfqpoint{0.578946in}{0.884054in}}%
\pgfpathlineto{\pgfqpoint{0.576431in}{0.896980in}}%
\pgfpathlineto{\pgfqpoint{0.570136in}{0.905666in}}%
\pgfpathlineto{\pgfqpoint{0.556867in}{0.908355in}}%
\pgfpathlineto{\pgfqpoint{0.543598in}{0.898095in}}%
\pgfpathlineto{\pgfqpoint{0.543122in}{0.896980in}}%
\pgfpathlineto{\pgfqpoint{0.541147in}{0.884054in}}%
\pgfpathlineto{\pgfqpoint{0.542037in}{0.871127in}}%
\pgfpathlineto{\pgfqpoint{0.543598in}{0.866350in}}%
\pgfpathlineto{\pgfqpoint{0.551197in}{0.858201in}}%
\pgfpathclose%
\pgfpathmoveto{\pgfqpoint{0.553190in}{0.871127in}}%
\pgfpathlineto{\pgfqpoint{0.550406in}{0.884054in}}%
\pgfpathlineto{\pgfqpoint{0.556185in}{0.896980in}}%
\pgfpathlineto{\pgfqpoint{0.556867in}{0.897529in}}%
\pgfpathlineto{\pgfqpoint{0.558881in}{0.896980in}}%
\pgfpathlineto{\pgfqpoint{0.570136in}{0.889126in}}%
\pgfpathlineto{\pgfqpoint{0.571499in}{0.884054in}}%
\pgfpathlineto{\pgfqpoint{0.570136in}{0.874395in}}%
\pgfpathlineto{\pgfqpoint{0.567836in}{0.871127in}}%
\pgfpathlineto{\pgfqpoint{0.556867in}{0.867320in}}%
\pgfpathclose%
\pgfusepath{fill}%
\end{pgfscope}%
\begin{pgfscope}%
\pgfpathrectangle{\pgfqpoint{0.211875in}{0.211875in}}{\pgfqpoint{1.313625in}{1.279725in}}%
\pgfusepath{clip}%
\pgfsetbuttcap%
\pgfsetroundjoin%
\definecolor{currentfill}{rgb}{0.644838,0.098089,0.355336}%
\pgfsetfillcolor{currentfill}%
\pgfsetlinewidth{0.000000pt}%
\definecolor{currentstroke}{rgb}{0.000000,0.000000,0.000000}%
\pgfsetstrokecolor{currentstroke}%
\pgfsetdash{}{0pt}%
\pgfpathmoveto{\pgfqpoint{0.676288in}{0.858053in}}%
\pgfpathlineto{\pgfqpoint{0.676941in}{0.858201in}}%
\pgfpathlineto{\pgfqpoint{0.689557in}{0.863244in}}%
\pgfpathlineto{\pgfqpoint{0.693543in}{0.871127in}}%
\pgfpathlineto{\pgfqpoint{0.694822in}{0.884054in}}%
\pgfpathlineto{\pgfqpoint{0.692086in}{0.896980in}}%
\pgfpathlineto{\pgfqpoint{0.689557in}{0.900861in}}%
\pgfpathlineto{\pgfqpoint{0.676288in}{0.905372in}}%
\pgfpathlineto{\pgfqpoint{0.664128in}{0.896980in}}%
\pgfpathlineto{\pgfqpoint{0.663019in}{0.894441in}}%
\pgfpathlineto{\pgfqpoint{0.661167in}{0.884054in}}%
\pgfpathlineto{\pgfqpoint{0.662275in}{0.871127in}}%
\pgfpathlineto{\pgfqpoint{0.663019in}{0.869116in}}%
\pgfpathlineto{\pgfqpoint{0.675907in}{0.858201in}}%
\pgfpathclose%
\pgfpathmoveto{\pgfqpoint{0.675825in}{0.871127in}}%
\pgfpathlineto{\pgfqpoint{0.672222in}{0.884054in}}%
\pgfpathlineto{\pgfqpoint{0.676288in}{0.891713in}}%
\pgfpathlineto{\pgfqpoint{0.683655in}{0.884054in}}%
\pgfpathlineto{\pgfqpoint{0.677119in}{0.871127in}}%
\pgfpathlineto{\pgfqpoint{0.676288in}{0.870716in}}%
\pgfpathclose%
\pgfusepath{fill}%
\end{pgfscope}%
\begin{pgfscope}%
\pgfpathrectangle{\pgfqpoint{0.211875in}{0.211875in}}{\pgfqpoint{1.313625in}{1.279725in}}%
\pgfusepath{clip}%
\pgfsetbuttcap%
\pgfsetroundjoin%
\definecolor{currentfill}{rgb}{0.644838,0.098089,0.355336}%
\pgfsetfillcolor{currentfill}%
\pgfsetlinewidth{0.000000pt}%
\definecolor{currentstroke}{rgb}{0.000000,0.000000,0.000000}%
\pgfsetstrokecolor{currentstroke}%
\pgfsetdash{}{0pt}%
\pgfpathmoveto{\pgfqpoint{0.782439in}{0.870474in}}%
\pgfpathlineto{\pgfqpoint{0.795708in}{0.860785in}}%
\pgfpathlineto{\pgfqpoint{0.808977in}{0.868736in}}%
\pgfpathlineto{\pgfqpoint{0.810062in}{0.871127in}}%
\pgfpathlineto{\pgfqpoint{0.811425in}{0.884054in}}%
\pgfpathlineto{\pgfqpoint{0.808977in}{0.895315in}}%
\pgfpathlineto{\pgfqpoint{0.807824in}{0.896980in}}%
\pgfpathlineto{\pgfqpoint{0.795708in}{0.903023in}}%
\pgfpathlineto{\pgfqpoint{0.785254in}{0.896980in}}%
\pgfpathlineto{\pgfqpoint{0.782439in}{0.891953in}}%
\pgfpathlineto{\pgfqpoint{0.780865in}{0.884054in}}%
\pgfpathlineto{\pgfqpoint{0.782168in}{0.871127in}}%
\pgfpathclose%
\pgfpathmoveto{\pgfqpoint{0.794170in}{0.884054in}}%
\pgfpathlineto{\pgfqpoint{0.795708in}{0.886480in}}%
\pgfpathlineto{\pgfqpoint{0.797510in}{0.884054in}}%
\pgfpathlineto{\pgfqpoint{0.795708in}{0.879489in}}%
\pgfpathclose%
\pgfusepath{fill}%
\end{pgfscope}%
\begin{pgfscope}%
\pgfpathrectangle{\pgfqpoint{0.211875in}{0.211875in}}{\pgfqpoint{1.313625in}{1.279725in}}%
\pgfusepath{clip}%
\pgfsetbuttcap%
\pgfsetroundjoin%
\definecolor{currentfill}{rgb}{0.644838,0.098089,0.355336}%
\pgfsetfillcolor{currentfill}%
\pgfsetlinewidth{0.000000pt}%
\definecolor{currentstroke}{rgb}{0.000000,0.000000,0.000000}%
\pgfsetstrokecolor{currentstroke}%
\pgfsetdash{}{0pt}%
\pgfpathmoveto{\pgfqpoint{0.901860in}{0.870778in}}%
\pgfpathlineto{\pgfqpoint{0.915129in}{0.862894in}}%
\pgfpathlineto{\pgfqpoint{0.925655in}{0.871127in}}%
\pgfpathlineto{\pgfqpoint{0.928398in}{0.881946in}}%
\pgfpathlineto{\pgfqpoint{0.928610in}{0.884054in}}%
\pgfpathlineto{\pgfqpoint{0.928398in}{0.885140in}}%
\pgfpathlineto{\pgfqpoint{0.922129in}{0.896980in}}%
\pgfpathlineto{\pgfqpoint{0.915129in}{0.901252in}}%
\pgfpathlineto{\pgfqpoint{0.905972in}{0.896980in}}%
\pgfpathlineto{\pgfqpoint{0.901860in}{0.891432in}}%
\pgfpathlineto{\pgfqpoint{0.900222in}{0.884054in}}%
\pgfpathlineto{\pgfqpoint{0.901698in}{0.871127in}}%
\pgfpathclose%
\pgfusepath{fill}%
\end{pgfscope}%
\begin{pgfscope}%
\pgfpathrectangle{\pgfqpoint{0.211875in}{0.211875in}}{\pgfqpoint{1.313625in}{1.279725in}}%
\pgfusepath{clip}%
\pgfsetbuttcap%
\pgfsetroundjoin%
\definecolor{currentfill}{rgb}{0.644838,0.098089,0.355336}%
\pgfsetfillcolor{currentfill}%
\pgfsetlinewidth{0.000000pt}%
\definecolor{currentstroke}{rgb}{0.000000,0.000000,0.000000}%
\pgfsetstrokecolor{currentstroke}%
\pgfsetdash{}{0pt}%
\pgfpathmoveto{\pgfqpoint{1.021280in}{0.870250in}}%
\pgfpathlineto{\pgfqpoint{1.034549in}{0.864353in}}%
\pgfpathlineto{\pgfqpoint{1.041857in}{0.871127in}}%
\pgfpathlineto{\pgfqpoint{1.044711in}{0.884054in}}%
\pgfpathlineto{\pgfqpoint{1.038762in}{0.896980in}}%
\pgfpathlineto{\pgfqpoint{1.034549in}{0.900024in}}%
\pgfpathlineto{\pgfqpoint{1.025954in}{0.896980in}}%
\pgfpathlineto{\pgfqpoint{1.021280in}{0.892483in}}%
\pgfpathlineto{\pgfqpoint{1.019198in}{0.884054in}}%
\pgfpathlineto{\pgfqpoint{1.020826in}{0.871127in}}%
\pgfpathclose%
\pgfusepath{fill}%
\end{pgfscope}%
\begin{pgfscope}%
\pgfpathrectangle{\pgfqpoint{0.211875in}{0.211875in}}{\pgfqpoint{1.313625in}{1.279725in}}%
\pgfusepath{clip}%
\pgfsetbuttcap%
\pgfsetroundjoin%
\definecolor{currentfill}{rgb}{0.644838,0.098089,0.355336}%
\pgfsetfillcolor{currentfill}%
\pgfsetlinewidth{0.000000pt}%
\definecolor{currentstroke}{rgb}{0.000000,0.000000,0.000000}%
\pgfsetstrokecolor{currentstroke}%
\pgfsetdash{}{0pt}%
\pgfpathmoveto{\pgfqpoint{1.140701in}{0.869031in}}%
\pgfpathlineto{\pgfqpoint{1.153970in}{0.865177in}}%
\pgfpathlineto{\pgfqpoint{1.159506in}{0.871127in}}%
\pgfpathlineto{\pgfqpoint{1.162029in}{0.884054in}}%
\pgfpathlineto{\pgfqpoint{1.156774in}{0.896980in}}%
\pgfpathlineto{\pgfqpoint{1.153970in}{0.899328in}}%
\pgfpathlineto{\pgfqpoint{1.144171in}{0.896980in}}%
\pgfpathlineto{\pgfqpoint{1.140701in}{0.894854in}}%
\pgfpathlineto{\pgfqpoint{1.137731in}{0.884054in}}%
\pgfpathlineto{\pgfqpoint{1.139489in}{0.871127in}}%
\pgfpathclose%
\pgfusepath{fill}%
\end{pgfscope}%
\begin{pgfscope}%
\pgfpathrectangle{\pgfqpoint{0.211875in}{0.211875in}}{\pgfqpoint{1.313625in}{1.279725in}}%
\pgfusepath{clip}%
\pgfsetbuttcap%
\pgfsetroundjoin%
\definecolor{currentfill}{rgb}{0.644838,0.098089,0.355336}%
\pgfsetfillcolor{currentfill}%
\pgfsetlinewidth{0.000000pt}%
\definecolor{currentstroke}{rgb}{0.000000,0.000000,0.000000}%
\pgfsetstrokecolor{currentstroke}%
\pgfsetdash{}{0pt}%
\pgfpathmoveto{\pgfqpoint{1.260121in}{0.867204in}}%
\pgfpathlineto{\pgfqpoint{1.273390in}{0.865352in}}%
\pgfpathlineto{\pgfqpoint{1.278094in}{0.871127in}}%
\pgfpathlineto{\pgfqpoint{1.280309in}{0.884054in}}%
\pgfpathlineto{\pgfqpoint{1.275684in}{0.896980in}}%
\pgfpathlineto{\pgfqpoint{1.273390in}{0.899172in}}%
\pgfpathlineto{\pgfqpoint{1.260121in}{0.897596in}}%
\pgfpathlineto{\pgfqpoint{1.259609in}{0.896980in}}%
\pgfpathlineto{\pgfqpoint{1.255721in}{0.884054in}}%
\pgfpathlineto{\pgfqpoint{1.257586in}{0.871127in}}%
\pgfpathclose%
\pgfusepath{fill}%
\end{pgfscope}%
\begin{pgfscope}%
\pgfpathrectangle{\pgfqpoint{0.211875in}{0.211875in}}{\pgfqpoint{1.313625in}{1.279725in}}%
\pgfusepath{clip}%
\pgfsetbuttcap%
\pgfsetroundjoin%
\definecolor{currentfill}{rgb}{0.644838,0.098089,0.355336}%
\pgfsetfillcolor{currentfill}%
\pgfsetlinewidth{0.000000pt}%
\definecolor{currentstroke}{rgb}{0.000000,0.000000,0.000000}%
\pgfsetstrokecolor{currentstroke}%
\pgfsetdash{}{0pt}%
\pgfpathmoveto{\pgfqpoint{1.379542in}{0.864816in}}%
\pgfpathlineto{\pgfqpoint{1.392811in}{0.864837in}}%
\pgfpathlineto{\pgfqpoint{1.397341in}{0.871127in}}%
\pgfpathlineto{\pgfqpoint{1.399261in}{0.884054in}}%
\pgfpathlineto{\pgfqpoint{1.395230in}{0.896980in}}%
\pgfpathlineto{\pgfqpoint{1.392811in}{0.899591in}}%
\pgfpathlineto{\pgfqpoint{1.379542in}{0.899612in}}%
\pgfpathlineto{\pgfqpoint{1.377090in}{0.896980in}}%
\pgfpathlineto{\pgfqpoint{1.373017in}{0.884054in}}%
\pgfpathlineto{\pgfqpoint{1.374963in}{0.871127in}}%
\pgfpathclose%
\pgfusepath{fill}%
\end{pgfscope}%
\begin{pgfscope}%
\pgfpathrectangle{\pgfqpoint{0.211875in}{0.211875in}}{\pgfqpoint{1.313625in}{1.279725in}}%
\pgfusepath{clip}%
\pgfsetbuttcap%
\pgfsetroundjoin%
\definecolor{currentfill}{rgb}{0.644838,0.098089,0.355336}%
\pgfsetfillcolor{currentfill}%
\pgfsetlinewidth{0.000000pt}%
\definecolor{currentstroke}{rgb}{0.000000,0.000000,0.000000}%
\pgfsetstrokecolor{currentstroke}%
\pgfsetdash{}{0pt}%
\pgfpathmoveto{\pgfqpoint{1.498962in}{0.861881in}}%
\pgfpathlineto{\pgfqpoint{1.512231in}{0.863557in}}%
\pgfpathlineto{\pgfqpoint{1.517086in}{0.871127in}}%
\pgfpathlineto{\pgfqpoint{1.518715in}{0.884054in}}%
\pgfpathlineto{\pgfqpoint{1.515260in}{0.896980in}}%
\pgfpathlineto{\pgfqpoint{1.512231in}{0.900645in}}%
\pgfpathlineto{\pgfqpoint{1.498962in}{0.902085in}}%
\pgfpathlineto{\pgfqpoint{1.493590in}{0.896980in}}%
\pgfpathlineto{\pgfqpoint{1.489382in}{0.884054in}}%
\pgfpathlineto{\pgfqpoint{1.491378in}{0.871127in}}%
\pgfpathclose%
\pgfusepath{fill}%
\end{pgfscope}%
\begin{pgfscope}%
\pgfpathrectangle{\pgfqpoint{0.211875in}{0.211875in}}{\pgfqpoint{1.313625in}{1.279725in}}%
\pgfusepath{clip}%
\pgfsetbuttcap%
\pgfsetroundjoin%
\definecolor{currentfill}{rgb}{0.644838,0.098089,0.355336}%
\pgfsetfillcolor{currentfill}%
\pgfsetlinewidth{0.000000pt}%
\definecolor{currentstroke}{rgb}{0.000000,0.000000,0.000000}%
\pgfsetstrokecolor{currentstroke}%
\pgfsetdash{}{0pt}%
\pgfpathmoveto{\pgfqpoint{0.251682in}{0.930241in}}%
\pgfpathlineto{\pgfqpoint{0.264951in}{0.928696in}}%
\pgfpathlineto{\pgfqpoint{0.278220in}{0.930807in}}%
\pgfpathlineto{\pgfqpoint{0.284657in}{0.935760in}}%
\pgfpathlineto{\pgfqpoint{0.289341in}{0.948686in}}%
\pgfpathlineto{\pgfqpoint{0.290302in}{0.961613in}}%
\pgfpathlineto{\pgfqpoint{0.289694in}{0.974539in}}%
\pgfpathlineto{\pgfqpoint{0.286449in}{0.987466in}}%
\pgfpathlineto{\pgfqpoint{0.278220in}{0.995606in}}%
\pgfpathlineto{\pgfqpoint{0.264951in}{0.997911in}}%
\pgfpathlineto{\pgfqpoint{0.251682in}{0.996142in}}%
\pgfpathlineto{\pgfqpoint{0.243278in}{0.987466in}}%
\pgfpathlineto{\pgfqpoint{0.240688in}{0.974539in}}%
\pgfpathlineto{\pgfqpoint{0.240209in}{0.961613in}}%
\pgfpathlineto{\pgfqpoint{0.240920in}{0.948686in}}%
\pgfpathlineto{\pgfqpoint{0.244653in}{0.935760in}}%
\pgfpathclose%
\pgfpathmoveto{\pgfqpoint{0.248826in}{0.948686in}}%
\pgfpathlineto{\pgfqpoint{0.246832in}{0.961613in}}%
\pgfpathlineto{\pgfqpoint{0.247962in}{0.974539in}}%
\pgfpathlineto{\pgfqpoint{0.251682in}{0.983222in}}%
\pgfpathlineto{\pgfqpoint{0.258510in}{0.987466in}}%
\pgfpathlineto{\pgfqpoint{0.264951in}{0.989459in}}%
\pgfpathlineto{\pgfqpoint{0.271572in}{0.987466in}}%
\pgfpathlineto{\pgfqpoint{0.278220in}{0.983758in}}%
\pgfpathlineto{\pgfqpoint{0.282471in}{0.974539in}}%
\pgfpathlineto{\pgfqpoint{0.283691in}{0.961613in}}%
\pgfpathlineto{\pgfqpoint{0.281531in}{0.948686in}}%
\pgfpathlineto{\pgfqpoint{0.278220in}{0.942884in}}%
\pgfpathlineto{\pgfqpoint{0.264951in}{0.937208in}}%
\pgfpathlineto{\pgfqpoint{0.251682in}{0.943315in}}%
\pgfpathclose%
\pgfusepath{fill}%
\end{pgfscope}%
\begin{pgfscope}%
\pgfpathrectangle{\pgfqpoint{0.211875in}{0.211875in}}{\pgfqpoint{1.313625in}{1.279725in}}%
\pgfusepath{clip}%
\pgfsetbuttcap%
\pgfsetroundjoin%
\definecolor{currentfill}{rgb}{0.644838,0.098089,0.355336}%
\pgfsetfillcolor{currentfill}%
\pgfsetlinewidth{0.000000pt}%
\definecolor{currentstroke}{rgb}{0.000000,0.000000,0.000000}%
\pgfsetstrokecolor{currentstroke}%
\pgfsetdash{}{0pt}%
\pgfpathmoveto{\pgfqpoint{0.371102in}{0.934697in}}%
\pgfpathlineto{\pgfqpoint{0.384371in}{0.932496in}}%
\pgfpathlineto{\pgfqpoint{0.395862in}{0.935760in}}%
\pgfpathlineto{\pgfqpoint{0.397640in}{0.936668in}}%
\pgfpathlineto{\pgfqpoint{0.403792in}{0.948686in}}%
\pgfpathlineto{\pgfqpoint{0.405231in}{0.961613in}}%
\pgfpathlineto{\pgfqpoint{0.404384in}{0.974539in}}%
\pgfpathlineto{\pgfqpoint{0.399721in}{0.987466in}}%
\pgfpathlineto{\pgfqpoint{0.397640in}{0.989753in}}%
\pgfpathlineto{\pgfqpoint{0.384371in}{0.993974in}}%
\pgfpathlineto{\pgfqpoint{0.371102in}{0.991584in}}%
\pgfpathlineto{\pgfqpoint{0.366653in}{0.987466in}}%
\pgfpathlineto{\pgfqpoint{0.362025in}{0.974539in}}%
\pgfpathlineto{\pgfqpoint{0.361205in}{0.961613in}}%
\pgfpathlineto{\pgfqpoint{0.362578in}{0.948686in}}%
\pgfpathlineto{\pgfqpoint{0.369593in}{0.935760in}}%
\pgfpathclose%
\pgfpathmoveto{\pgfqpoint{0.370816in}{0.948686in}}%
\pgfpathlineto{\pgfqpoint{0.368110in}{0.961613in}}%
\pgfpathlineto{\pgfqpoint{0.369607in}{0.974539in}}%
\pgfpathlineto{\pgfqpoint{0.371102in}{0.977665in}}%
\pgfpathlineto{\pgfqpoint{0.384371in}{0.984616in}}%
\pgfpathlineto{\pgfqpoint{0.397204in}{0.974539in}}%
\pgfpathlineto{\pgfqpoint{0.397640in}{0.972947in}}%
\pgfpathlineto{\pgfqpoint{0.398886in}{0.961613in}}%
\pgfpathlineto{\pgfqpoint{0.397640in}{0.954870in}}%
\pgfpathlineto{\pgfqpoint{0.394568in}{0.948686in}}%
\pgfpathlineto{\pgfqpoint{0.384371in}{0.942233in}}%
\pgfpathlineto{\pgfqpoint{0.371102in}{0.948205in}}%
\pgfpathclose%
\pgfusepath{fill}%
\end{pgfscope}%
\begin{pgfscope}%
\pgfpathrectangle{\pgfqpoint{0.211875in}{0.211875in}}{\pgfqpoint{1.313625in}{1.279725in}}%
\pgfusepath{clip}%
\pgfsetbuttcap%
\pgfsetroundjoin%
\definecolor{currentfill}{rgb}{0.644838,0.098089,0.355336}%
\pgfsetfillcolor{currentfill}%
\pgfsetlinewidth{0.000000pt}%
\definecolor{currentstroke}{rgb}{0.000000,0.000000,0.000000}%
\pgfsetstrokecolor{currentstroke}%
\pgfsetdash{}{0pt}%
\pgfpathmoveto{\pgfqpoint{0.490523in}{0.938750in}}%
\pgfpathlineto{\pgfqpoint{0.503792in}{0.935847in}}%
\pgfpathlineto{\pgfqpoint{0.517061in}{0.944396in}}%
\pgfpathlineto{\pgfqpoint{0.519021in}{0.948686in}}%
\pgfpathlineto{\pgfqpoint{0.520846in}{0.961613in}}%
\pgfpathlineto{\pgfqpoint{0.519807in}{0.974539in}}%
\pgfpathlineto{\pgfqpoint{0.517061in}{0.981953in}}%
\pgfpathlineto{\pgfqpoint{0.510979in}{0.987466in}}%
\pgfpathlineto{\pgfqpoint{0.503792in}{0.990522in}}%
\pgfpathlineto{\pgfqpoint{0.490523in}{0.988183in}}%
\pgfpathlineto{\pgfqpoint{0.489661in}{0.987466in}}%
\pgfpathlineto{\pgfqpoint{0.483067in}{0.974539in}}%
\pgfpathlineto{\pgfqpoint{0.481916in}{0.961613in}}%
\pgfpathlineto{\pgfqpoint{0.483928in}{0.948686in}}%
\pgfpathclose%
\pgfpathmoveto{\pgfqpoint{0.498041in}{0.948686in}}%
\pgfpathlineto{\pgfqpoint{0.490523in}{0.955960in}}%
\pgfpathlineto{\pgfqpoint{0.489177in}{0.961613in}}%
\pgfpathlineto{\pgfqpoint{0.490523in}{0.971193in}}%
\pgfpathlineto{\pgfqpoint{0.492391in}{0.974539in}}%
\pgfpathlineto{\pgfqpoint{0.503792in}{0.979469in}}%
\pgfpathlineto{\pgfqpoint{0.509106in}{0.974539in}}%
\pgfpathlineto{\pgfqpoint{0.512333in}{0.961613in}}%
\pgfpathlineto{\pgfqpoint{0.506443in}{0.948686in}}%
\pgfpathlineto{\pgfqpoint{0.503792in}{0.946706in}}%
\pgfpathclose%
\pgfusepath{fill}%
\end{pgfscope}%
\begin{pgfscope}%
\pgfpathrectangle{\pgfqpoint{0.211875in}{0.211875in}}{\pgfqpoint{1.313625in}{1.279725in}}%
\pgfusepath{clip}%
\pgfsetbuttcap%
\pgfsetroundjoin%
\definecolor{currentfill}{rgb}{0.644838,0.098089,0.355336}%
\pgfsetfillcolor{currentfill}%
\pgfsetlinewidth{0.000000pt}%
\definecolor{currentstroke}{rgb}{0.000000,0.000000,0.000000}%
\pgfsetstrokecolor{currentstroke}%
\pgfsetdash{}{0pt}%
\pgfpathmoveto{\pgfqpoint{0.609943in}{0.941975in}}%
\pgfpathlineto{\pgfqpoint{0.623212in}{0.939709in}}%
\pgfpathlineto{\pgfqpoint{0.633616in}{0.948686in}}%
\pgfpathlineto{\pgfqpoint{0.636481in}{0.957918in}}%
\pgfpathlineto{\pgfqpoint{0.637019in}{0.961613in}}%
\pgfpathlineto{\pgfqpoint{0.636481in}{0.967738in}}%
\pgfpathlineto{\pgfqpoint{0.635303in}{0.974539in}}%
\pgfpathlineto{\pgfqpoint{0.623274in}{0.987466in}}%
\pgfpathlineto{\pgfqpoint{0.623212in}{0.987496in}}%
\pgfpathlineto{\pgfqpoint{0.622965in}{0.987466in}}%
\pgfpathlineto{\pgfqpoint{0.609943in}{0.984849in}}%
\pgfpathlineto{\pgfqpoint{0.603811in}{0.974539in}}%
\pgfpathlineto{\pgfqpoint{0.602334in}{0.961613in}}%
\pgfpathlineto{\pgfqpoint{0.604971in}{0.948686in}}%
\pgfpathclose%
\pgfpathmoveto{\pgfqpoint{0.610469in}{0.961613in}}%
\pgfpathlineto{\pgfqpoint{0.622071in}{0.974539in}}%
\pgfpathlineto{\pgfqpoint{0.623212in}{0.974867in}}%
\pgfpathlineto{\pgfqpoint{0.623517in}{0.974539in}}%
\pgfpathlineto{\pgfqpoint{0.626664in}{0.961613in}}%
\pgfpathlineto{\pgfqpoint{0.623212in}{0.953399in}}%
\pgfpathclose%
\pgfusepath{fill}%
\end{pgfscope}%
\begin{pgfscope}%
\pgfpathrectangle{\pgfqpoint{0.211875in}{0.211875in}}{\pgfqpoint{1.313625in}{1.279725in}}%
\pgfusepath{clip}%
\pgfsetbuttcap%
\pgfsetroundjoin%
\definecolor{currentfill}{rgb}{0.644838,0.098089,0.355336}%
\pgfsetfillcolor{currentfill}%
\pgfsetlinewidth{0.000000pt}%
\definecolor{currentstroke}{rgb}{0.000000,0.000000,0.000000}%
\pgfsetstrokecolor{currentstroke}%
\pgfsetdash{}{0pt}%
\pgfpathmoveto{\pgfqpoint{0.729364in}{0.944259in}}%
\pgfpathlineto{\pgfqpoint{0.742633in}{0.943079in}}%
\pgfpathlineto{\pgfqpoint{0.748336in}{0.948686in}}%
\pgfpathlineto{\pgfqpoint{0.752213in}{0.961613in}}%
\pgfpathlineto{\pgfqpoint{0.750066in}{0.974539in}}%
\pgfpathlineto{\pgfqpoint{0.742633in}{0.983627in}}%
\pgfpathlineto{\pgfqpoint{0.729364in}{0.982247in}}%
\pgfpathlineto{\pgfqpoint{0.724239in}{0.974539in}}%
\pgfpathlineto{\pgfqpoint{0.722434in}{0.961613in}}%
\pgfpathlineto{\pgfqpoint{0.725695in}{0.948686in}}%
\pgfpathclose%
\pgfusepath{fill}%
\end{pgfscope}%
\begin{pgfscope}%
\pgfpathrectangle{\pgfqpoint{0.211875in}{0.211875in}}{\pgfqpoint{1.313625in}{1.279725in}}%
\pgfusepath{clip}%
\pgfsetbuttcap%
\pgfsetroundjoin%
\definecolor{currentfill}{rgb}{0.644838,0.098089,0.355336}%
\pgfsetfillcolor{currentfill}%
\pgfsetlinewidth{0.000000pt}%
\definecolor{currentstroke}{rgb}{0.000000,0.000000,0.000000}%
\pgfsetstrokecolor{currentstroke}%
\pgfsetdash{}{0pt}%
\pgfpathmoveto{\pgfqpoint{0.848784in}{0.945762in}}%
\pgfpathlineto{\pgfqpoint{0.862053in}{0.945993in}}%
\pgfpathlineto{\pgfqpoint{0.864481in}{0.948686in}}%
\pgfpathlineto{\pgfqpoint{0.868315in}{0.961613in}}%
\pgfpathlineto{\pgfqpoint{0.866206in}{0.974539in}}%
\pgfpathlineto{\pgfqpoint{0.862053in}{0.980269in}}%
\pgfpathlineto{\pgfqpoint{0.848784in}{0.980534in}}%
\pgfpathlineto{\pgfqpoint{0.844307in}{0.974539in}}%
\pgfpathlineto{\pgfqpoint{0.842166in}{0.961613in}}%
\pgfpathlineto{\pgfqpoint{0.846061in}{0.948686in}}%
\pgfpathclose%
\pgfusepath{fill}%
\end{pgfscope}%
\begin{pgfscope}%
\pgfpathrectangle{\pgfqpoint{0.211875in}{0.211875in}}{\pgfqpoint{1.313625in}{1.279725in}}%
\pgfusepath{clip}%
\pgfsetbuttcap%
\pgfsetroundjoin%
\definecolor{currentfill}{rgb}{0.644838,0.098089,0.355336}%
\pgfsetfillcolor{currentfill}%
\pgfsetlinewidth{0.000000pt}%
\definecolor{currentstroke}{rgb}{0.000000,0.000000,0.000000}%
\pgfsetstrokecolor{currentstroke}%
\pgfsetdash{}{0pt}%
\pgfpathmoveto{\pgfqpoint{0.968205in}{0.946593in}}%
\pgfpathlineto{\pgfqpoint{0.981473in}{0.948469in}}%
\pgfpathlineto{\pgfqpoint{0.981648in}{0.948686in}}%
\pgfpathlineto{\pgfqpoint{0.985374in}{0.961613in}}%
\pgfpathlineto{\pgfqpoint{0.983333in}{0.974539in}}%
\pgfpathlineto{\pgfqpoint{0.981473in}{0.977414in}}%
\pgfpathlineto{\pgfqpoint{0.968205in}{0.979589in}}%
\pgfpathlineto{\pgfqpoint{0.963937in}{0.974539in}}%
\pgfpathlineto{\pgfqpoint{0.961443in}{0.961613in}}%
\pgfpathlineto{\pgfqpoint{0.965999in}{0.948686in}}%
\pgfpathclose%
\pgfusepath{fill}%
\end{pgfscope}%
\begin{pgfscope}%
\pgfpathrectangle{\pgfqpoint{0.211875in}{0.211875in}}{\pgfqpoint{1.313625in}{1.279725in}}%
\pgfusepath{clip}%
\pgfsetbuttcap%
\pgfsetroundjoin%
\definecolor{currentfill}{rgb}{0.644838,0.098089,0.355336}%
\pgfsetfillcolor{currentfill}%
\pgfsetlinewidth{0.000000pt}%
\definecolor{currentstroke}{rgb}{0.000000,0.000000,0.000000}%
\pgfsetstrokecolor{currentstroke}%
\pgfsetdash{}{0pt}%
\pgfpathmoveto{\pgfqpoint{1.087625in}{0.946823in}}%
\pgfpathlineto{\pgfqpoint{1.094745in}{0.948686in}}%
\pgfpathlineto{\pgfqpoint{1.100894in}{0.952934in}}%
\pgfpathlineto{\pgfqpoint{1.103148in}{0.961613in}}%
\pgfpathlineto{\pgfqpoint{1.101200in}{0.974539in}}%
\pgfpathlineto{\pgfqpoint{1.100894in}{0.975067in}}%
\pgfpathlineto{\pgfqpoint{1.087625in}{0.979331in}}%
\pgfpathlineto{\pgfqpoint{1.082991in}{0.974539in}}%
\pgfpathlineto{\pgfqpoint{1.080114in}{0.961613in}}%
\pgfpathlineto{\pgfqpoint{1.085379in}{0.948686in}}%
\pgfpathclose%
\pgfusepath{fill}%
\end{pgfscope}%
\begin{pgfscope}%
\pgfpathrectangle{\pgfqpoint{0.211875in}{0.211875in}}{\pgfqpoint{1.313625in}{1.279725in}}%
\pgfusepath{clip}%
\pgfsetbuttcap%
\pgfsetroundjoin%
\definecolor{currentfill}{rgb}{0.644838,0.098089,0.355336}%
\pgfsetfillcolor{currentfill}%
\pgfsetlinewidth{0.000000pt}%
\definecolor{currentstroke}{rgb}{0.000000,0.000000,0.000000}%
\pgfsetstrokecolor{currentstroke}%
\pgfsetdash{}{0pt}%
\pgfpathmoveto{\pgfqpoint{1.207045in}{0.946495in}}%
\pgfpathlineto{\pgfqpoint{1.212854in}{0.948686in}}%
\pgfpathlineto{\pgfqpoint{1.220314in}{0.956606in}}%
\pgfpathlineto{\pgfqpoint{1.221479in}{0.961613in}}%
\pgfpathlineto{\pgfqpoint{1.220314in}{0.970097in}}%
\pgfpathlineto{\pgfqpoint{1.218038in}{0.974539in}}%
\pgfpathlineto{\pgfqpoint{1.207045in}{0.979713in}}%
\pgfpathlineto{\pgfqpoint{1.201229in}{0.974539in}}%
\pgfpathlineto{\pgfqpoint{1.197917in}{0.961613in}}%
\pgfpathlineto{\pgfqpoint{1.203975in}{0.948686in}}%
\pgfpathclose%
\pgfusepath{fill}%
\end{pgfscope}%
\begin{pgfscope}%
\pgfpathrectangle{\pgfqpoint{0.211875in}{0.211875in}}{\pgfqpoint{1.313625in}{1.279725in}}%
\pgfusepath{clip}%
\pgfsetbuttcap%
\pgfsetroundjoin%
\definecolor{currentfill}{rgb}{0.644838,0.098089,0.355336}%
\pgfsetfillcolor{currentfill}%
\pgfsetlinewidth{0.000000pt}%
\definecolor{currentstroke}{rgb}{0.000000,0.000000,0.000000}%
\pgfsetstrokecolor{currentstroke}%
\pgfsetdash{}{0pt}%
\pgfpathmoveto{\pgfqpoint{1.326466in}{0.945624in}}%
\pgfpathlineto{\pgfqpoint{1.332706in}{0.948686in}}%
\pgfpathlineto{\pgfqpoint{1.339735in}{0.959065in}}%
\pgfpathlineto{\pgfqpoint{1.340266in}{0.961613in}}%
\pgfpathlineto{\pgfqpoint{1.339735in}{0.965913in}}%
\pgfpathlineto{\pgfqpoint{1.336546in}{0.974539in}}%
\pgfpathlineto{\pgfqpoint{1.326466in}{0.980717in}}%
\pgfpathlineto{\pgfqpoint{1.318200in}{0.974539in}}%
\pgfpathlineto{\pgfqpoint{1.314364in}{0.961613in}}%
\pgfpathlineto{\pgfqpoint{1.321362in}{0.948686in}}%
\pgfpathclose%
\pgfusepath{fill}%
\end{pgfscope}%
\begin{pgfscope}%
\pgfpathrectangle{\pgfqpoint{0.211875in}{0.211875in}}{\pgfqpoint{1.313625in}{1.279725in}}%
\pgfusepath{clip}%
\pgfsetbuttcap%
\pgfsetroundjoin%
\definecolor{currentfill}{rgb}{0.644838,0.098089,0.355336}%
\pgfsetfillcolor{currentfill}%
\pgfsetlinewidth{0.000000pt}%
\definecolor{currentstroke}{rgb}{0.000000,0.000000,0.000000}%
\pgfsetstrokecolor{currentstroke}%
\pgfsetdash{}{0pt}%
\pgfpathmoveto{\pgfqpoint{1.445886in}{0.944203in}}%
\pgfpathlineto{\pgfqpoint{1.453312in}{0.948686in}}%
\pgfpathlineto{\pgfqpoint{1.459155in}{0.960086in}}%
\pgfpathlineto{\pgfqpoint{1.459440in}{0.961613in}}%
\pgfpathlineto{\pgfqpoint{1.459155in}{0.964176in}}%
\pgfpathlineto{\pgfqpoint{1.456242in}{0.974539in}}%
\pgfpathlineto{\pgfqpoint{1.445886in}{0.982350in}}%
\pgfpathlineto{\pgfqpoint{1.432995in}{0.974539in}}%
\pgfpathlineto{\pgfqpoint{1.432617in}{0.973544in}}%
\pgfpathlineto{\pgfqpoint{1.431129in}{0.961613in}}%
\pgfpathlineto{\pgfqpoint{1.432617in}{0.954551in}}%
\pgfpathlineto{\pgfqpoint{1.436683in}{0.948686in}}%
\pgfpathclose%
\pgfusepath{fill}%
\end{pgfscope}%
\begin{pgfscope}%
\pgfpathrectangle{\pgfqpoint{0.211875in}{0.211875in}}{\pgfqpoint{1.313625in}{1.279725in}}%
\pgfusepath{clip}%
\pgfsetbuttcap%
\pgfsetroundjoin%
\definecolor{currentfill}{rgb}{0.644838,0.098089,0.355336}%
\pgfsetfillcolor{currentfill}%
\pgfsetlinewidth{0.000000pt}%
\definecolor{currentstroke}{rgb}{0.000000,0.000000,0.000000}%
\pgfsetstrokecolor{currentstroke}%
\pgfsetdash{}{0pt}%
\pgfpathmoveto{\pgfqpoint{0.225144in}{1.012710in}}%
\pgfpathlineto{\pgfqpoint{0.225828in}{1.013319in}}%
\pgfpathlineto{\pgfqpoint{0.231217in}{1.026245in}}%
\pgfpathlineto{\pgfqpoint{0.232300in}{1.039172in}}%
\pgfpathlineto{\pgfqpoint{0.232200in}{1.052098in}}%
\pgfpathlineto{\pgfqpoint{0.230767in}{1.065025in}}%
\pgfpathlineto{\pgfqpoint{0.225144in}{1.076243in}}%
\pgfpathlineto{\pgfqpoint{0.221922in}{1.077952in}}%
\pgfpathlineto{\pgfqpoint{0.211875in}{1.080634in}}%
\pgfpathlineto{\pgfqpoint{0.211875in}{1.077952in}}%
\pgfpathlineto{\pgfqpoint{0.211875in}{1.071643in}}%
\pgfpathlineto{\pgfqpoint{0.221135in}{1.065025in}}%
\pgfpathlineto{\pgfqpoint{0.225144in}{1.056796in}}%
\pgfpathlineto{\pgfqpoint{0.226187in}{1.052098in}}%
\pgfpathlineto{\pgfqpoint{0.226390in}{1.039172in}}%
\pgfpathlineto{\pgfqpoint{0.225144in}{1.032537in}}%
\pgfpathlineto{\pgfqpoint{0.222663in}{1.026245in}}%
\pgfpathlineto{\pgfqpoint{0.211875in}{1.017611in}}%
\pgfpathlineto{\pgfqpoint{0.211875in}{1.013319in}}%
\pgfpathlineto{\pgfqpoint{0.211875in}{1.008659in}}%
\pgfpathclose%
\pgfusepath{fill}%
\end{pgfscope}%
\begin{pgfscope}%
\pgfpathrectangle{\pgfqpoint{0.211875in}{0.211875in}}{\pgfqpoint{1.313625in}{1.279725in}}%
\pgfusepath{clip}%
\pgfsetbuttcap%
\pgfsetroundjoin%
\definecolor{currentfill}{rgb}{0.644838,0.098089,0.355336}%
\pgfsetfillcolor{currentfill}%
\pgfsetlinewidth{0.000000pt}%
\definecolor{currentstroke}{rgb}{0.000000,0.000000,0.000000}%
\pgfsetstrokecolor{currentstroke}%
\pgfsetdash{}{0pt}%
\pgfpathmoveto{\pgfqpoint{0.318027in}{1.012567in}}%
\pgfpathlineto{\pgfqpoint{0.331295in}{1.012956in}}%
\pgfpathlineto{\pgfqpoint{0.332302in}{1.013319in}}%
\pgfpathlineto{\pgfqpoint{0.344564in}{1.023535in}}%
\pgfpathlineto{\pgfqpoint{0.345609in}{1.026245in}}%
\pgfpathlineto{\pgfqpoint{0.347479in}{1.039172in}}%
\pgfpathlineto{\pgfqpoint{0.347332in}{1.052098in}}%
\pgfpathlineto{\pgfqpoint{0.344931in}{1.065025in}}%
\pgfpathlineto{\pgfqpoint{0.344564in}{1.065868in}}%
\pgfpathlineto{\pgfqpoint{0.331295in}{1.076127in}}%
\pgfpathlineto{\pgfqpoint{0.318027in}{1.076573in}}%
\pgfpathlineto{\pgfqpoint{0.304758in}{1.068230in}}%
\pgfpathlineto{\pgfqpoint{0.303312in}{1.065025in}}%
\pgfpathlineto{\pgfqpoint{0.301161in}{1.052098in}}%
\pgfpathlineto{\pgfqpoint{0.301026in}{1.039172in}}%
\pgfpathlineto{\pgfqpoint{0.302679in}{1.026245in}}%
\pgfpathlineto{\pgfqpoint{0.304758in}{1.021017in}}%
\pgfpathlineto{\pgfqpoint{0.315538in}{1.013319in}}%
\pgfpathclose%
\pgfpathmoveto{\pgfqpoint{0.313755in}{1.026245in}}%
\pgfpathlineto{\pgfqpoint{0.308371in}{1.039172in}}%
\pgfpathlineto{\pgfqpoint{0.308747in}{1.052098in}}%
\pgfpathlineto{\pgfqpoint{0.315575in}{1.065025in}}%
\pgfpathlineto{\pgfqpoint{0.318027in}{1.066853in}}%
\pgfpathlineto{\pgfqpoint{0.331295in}{1.066461in}}%
\pgfpathlineto{\pgfqpoint{0.333084in}{1.065025in}}%
\pgfpathlineto{\pgfqpoint{0.339770in}{1.052098in}}%
\pgfpathlineto{\pgfqpoint{0.340146in}{1.039172in}}%
\pgfpathlineto{\pgfqpoint{0.334826in}{1.026245in}}%
\pgfpathlineto{\pgfqpoint{0.331295in}{1.023067in}}%
\pgfpathlineto{\pgfqpoint{0.318027in}{1.022653in}}%
\pgfpathclose%
\pgfusepath{fill}%
\end{pgfscope}%
\begin{pgfscope}%
\pgfpathrectangle{\pgfqpoint{0.211875in}{0.211875in}}{\pgfqpoint{1.313625in}{1.279725in}}%
\pgfusepath{clip}%
\pgfsetbuttcap%
\pgfsetroundjoin%
\definecolor{currentfill}{rgb}{0.644838,0.098089,0.355336}%
\pgfsetfillcolor{currentfill}%
\pgfsetlinewidth{0.000000pt}%
\definecolor{currentstroke}{rgb}{0.000000,0.000000,0.000000}%
\pgfsetstrokecolor{currentstroke}%
\pgfsetdash{}{0pt}%
\pgfpathmoveto{\pgfqpoint{0.437447in}{1.016493in}}%
\pgfpathlineto{\pgfqpoint{0.450716in}{1.017752in}}%
\pgfpathlineto{\pgfqpoint{0.459153in}{1.026245in}}%
\pgfpathlineto{\pgfqpoint{0.462726in}{1.039172in}}%
\pgfpathlineto{\pgfqpoint{0.462461in}{1.052098in}}%
\pgfpathlineto{\pgfqpoint{0.457929in}{1.065025in}}%
\pgfpathlineto{\pgfqpoint{0.450716in}{1.071497in}}%
\pgfpathlineto{\pgfqpoint{0.437447in}{1.072703in}}%
\pgfpathlineto{\pgfqpoint{0.425780in}{1.065025in}}%
\pgfpathlineto{\pgfqpoint{0.424178in}{1.061941in}}%
\pgfpathlineto{\pgfqpoint{0.421943in}{1.052098in}}%
\pgfpathlineto{\pgfqpoint{0.421757in}{1.039172in}}%
\pgfpathlineto{\pgfqpoint{0.424178in}{1.026545in}}%
\pgfpathlineto{\pgfqpoint{0.424303in}{1.026245in}}%
\pgfpathclose%
\pgfpathmoveto{\pgfqpoint{0.430426in}{1.039172in}}%
\pgfpathlineto{\pgfqpoint{0.430925in}{1.052098in}}%
\pgfpathlineto{\pgfqpoint{0.437447in}{1.062020in}}%
\pgfpathlineto{\pgfqpoint{0.450716in}{1.059112in}}%
\pgfpathlineto{\pgfqpoint{0.454175in}{1.052098in}}%
\pgfpathlineto{\pgfqpoint{0.454581in}{1.039172in}}%
\pgfpathlineto{\pgfqpoint{0.450716in}{1.029986in}}%
\pgfpathlineto{\pgfqpoint{0.437447in}{1.026650in}}%
\pgfpathclose%
\pgfusepath{fill}%
\end{pgfscope}%
\begin{pgfscope}%
\pgfpathrectangle{\pgfqpoint{0.211875in}{0.211875in}}{\pgfqpoint{1.313625in}{1.279725in}}%
\pgfusepath{clip}%
\pgfsetbuttcap%
\pgfsetroundjoin%
\definecolor{currentfill}{rgb}{0.644838,0.098089,0.355336}%
\pgfsetfillcolor{currentfill}%
\pgfsetlinewidth{0.000000pt}%
\definecolor{currentstroke}{rgb}{0.000000,0.000000,0.000000}%
\pgfsetstrokecolor{currentstroke}%
\pgfsetdash{}{0pt}%
\pgfpathmoveto{\pgfqpoint{0.556867in}{1.019822in}}%
\pgfpathlineto{\pgfqpoint{0.570136in}{1.022362in}}%
\pgfpathlineto{\pgfqpoint{0.573600in}{1.026245in}}%
\pgfpathlineto{\pgfqpoint{0.577702in}{1.039172in}}%
\pgfpathlineto{\pgfqpoint{0.577409in}{1.052098in}}%
\pgfpathlineto{\pgfqpoint{0.572231in}{1.065025in}}%
\pgfpathlineto{\pgfqpoint{0.570136in}{1.067117in}}%
\pgfpathlineto{\pgfqpoint{0.556867in}{1.069548in}}%
\pgfpathlineto{\pgfqpoint{0.548993in}{1.065025in}}%
\pgfpathlineto{\pgfqpoint{0.543598in}{1.056753in}}%
\pgfpathlineto{\pgfqpoint{0.542395in}{1.052098in}}%
\pgfpathlineto{\pgfqpoint{0.542162in}{1.039172in}}%
\pgfpathlineto{\pgfqpoint{0.543598in}{1.032613in}}%
\pgfpathlineto{\pgfqpoint{0.546945in}{1.026245in}}%
\pgfpathclose%
\pgfpathmoveto{\pgfqpoint{0.552518in}{1.039172in}}%
\pgfpathlineto{\pgfqpoint{0.553157in}{1.052098in}}%
\pgfpathlineto{\pgfqpoint{0.556867in}{1.057021in}}%
\pgfpathlineto{\pgfqpoint{0.568019in}{1.052098in}}%
\pgfpathlineto{\pgfqpoint{0.569970in}{1.039172in}}%
\pgfpathlineto{\pgfqpoint{0.556867in}{1.032409in}}%
\pgfpathclose%
\pgfusepath{fill}%
\end{pgfscope}%
\begin{pgfscope}%
\pgfpathrectangle{\pgfqpoint{0.211875in}{0.211875in}}{\pgfqpoint{1.313625in}{1.279725in}}%
\pgfusepath{clip}%
\pgfsetbuttcap%
\pgfsetroundjoin%
\definecolor{currentfill}{rgb}{0.644838,0.098089,0.355336}%
\pgfsetfillcolor{currentfill}%
\pgfsetlinewidth{0.000000pt}%
\definecolor{currentstroke}{rgb}{0.000000,0.000000,0.000000}%
\pgfsetstrokecolor{currentstroke}%
\pgfsetdash{}{0pt}%
\pgfpathmoveto{\pgfqpoint{0.676288in}{1.022504in}}%
\pgfpathlineto{\pgfqpoint{0.688171in}{1.026245in}}%
\pgfpathlineto{\pgfqpoint{0.689557in}{1.027203in}}%
\pgfpathlineto{\pgfqpoint{0.693607in}{1.039172in}}%
\pgfpathlineto{\pgfqpoint{0.693294in}{1.052098in}}%
\pgfpathlineto{\pgfqpoint{0.689557in}{1.061506in}}%
\pgfpathlineto{\pgfqpoint{0.683320in}{1.065025in}}%
\pgfpathlineto{\pgfqpoint{0.676288in}{1.067004in}}%
\pgfpathlineto{\pgfqpoint{0.672275in}{1.065025in}}%
\pgfpathlineto{\pgfqpoint{0.663019in}{1.053790in}}%
\pgfpathlineto{\pgfqpoint{0.662525in}{1.052098in}}%
\pgfpathlineto{\pgfqpoint{0.662251in}{1.039172in}}%
\pgfpathlineto{\pgfqpoint{0.663019in}{1.036069in}}%
\pgfpathlineto{\pgfqpoint{0.669555in}{1.026245in}}%
\pgfpathclose%
\pgfpathmoveto{\pgfqpoint{0.674738in}{1.039172in}}%
\pgfpathlineto{\pgfqpoint{0.675547in}{1.052098in}}%
\pgfpathlineto{\pgfqpoint{0.676288in}{1.052941in}}%
\pgfpathlineto{\pgfqpoint{0.677625in}{1.052098in}}%
\pgfpathlineto{\pgfqpoint{0.679090in}{1.039172in}}%
\pgfpathlineto{\pgfqpoint{0.676288in}{1.037105in}}%
\pgfpathclose%
\pgfusepath{fill}%
\end{pgfscope}%
\begin{pgfscope}%
\pgfpathrectangle{\pgfqpoint{0.211875in}{0.211875in}}{\pgfqpoint{1.313625in}{1.279725in}}%
\pgfusepath{clip}%
\pgfsetbuttcap%
\pgfsetroundjoin%
\definecolor{currentfill}{rgb}{0.644838,0.098089,0.355336}%
\pgfsetfillcolor{currentfill}%
\pgfsetlinewidth{0.000000pt}%
\definecolor{currentstroke}{rgb}{0.000000,0.000000,0.000000}%
\pgfsetstrokecolor{currentstroke}%
\pgfsetdash{}{0pt}%
\pgfpathmoveto{\pgfqpoint{0.795708in}{1.024618in}}%
\pgfpathlineto{\pgfqpoint{0.799727in}{1.026245in}}%
\pgfpathlineto{\pgfqpoint{0.808977in}{1.035024in}}%
\pgfpathlineto{\pgfqpoint{0.810237in}{1.039172in}}%
\pgfpathlineto{\pgfqpoint{0.809912in}{1.052098in}}%
\pgfpathlineto{\pgfqpoint{0.808977in}{1.054717in}}%
\pgfpathlineto{\pgfqpoint{0.795708in}{1.064980in}}%
\pgfpathlineto{\pgfqpoint{0.782439in}{1.052424in}}%
\pgfpathlineto{\pgfqpoint{0.782333in}{1.052098in}}%
\pgfpathlineto{\pgfqpoint{0.782020in}{1.039172in}}%
\pgfpathlineto{\pgfqpoint{0.782439in}{1.037664in}}%
\pgfpathlineto{\pgfqpoint{0.792214in}{1.026245in}}%
\pgfpathclose%
\pgfusepath{fill}%
\end{pgfscope}%
\begin{pgfscope}%
\pgfpathrectangle{\pgfqpoint{0.211875in}{0.211875in}}{\pgfqpoint{1.313625in}{1.279725in}}%
\pgfusepath{clip}%
\pgfsetbuttcap%
\pgfsetroundjoin%
\definecolor{currentfill}{rgb}{0.644838,0.098089,0.355336}%
\pgfsetfillcolor{currentfill}%
\pgfsetlinewidth{0.000000pt}%
\definecolor{currentstroke}{rgb}{0.000000,0.000000,0.000000}%
\pgfsetstrokecolor{currentstroke}%
\pgfsetdash{}{0pt}%
\pgfpathmoveto{\pgfqpoint{0.915129in}{1.026217in}}%
\pgfpathlineto{\pgfqpoint{0.915186in}{1.026245in}}%
\pgfpathlineto{\pgfqpoint{0.926217in}{1.039172in}}%
\pgfpathlineto{\pgfqpoint{0.925458in}{1.052098in}}%
\pgfpathlineto{\pgfqpoint{0.915129in}{1.062426in}}%
\pgfpathlineto{\pgfqpoint{0.901860in}{1.052264in}}%
\pgfpathlineto{\pgfqpoint{0.901799in}{1.052098in}}%
\pgfpathlineto{\pgfqpoint{0.901452in}{1.039172in}}%
\pgfpathlineto{\pgfqpoint{0.901860in}{1.037858in}}%
\pgfpathlineto{\pgfqpoint{0.915054in}{1.026245in}}%
\pgfpathclose%
\pgfusepath{fill}%
\end{pgfscope}%
\begin{pgfscope}%
\pgfpathrectangle{\pgfqpoint{0.211875in}{0.211875in}}{\pgfqpoint{1.313625in}{1.279725in}}%
\pgfusepath{clip}%
\pgfsetbuttcap%
\pgfsetroundjoin%
\definecolor{currentfill}{rgb}{0.644838,0.098089,0.355336}%
\pgfsetfillcolor{currentfill}%
\pgfsetlinewidth{0.000000pt}%
\definecolor{currentstroke}{rgb}{0.000000,0.000000,0.000000}%
\pgfsetstrokecolor{currentstroke}%
\pgfsetdash{}{0pt}%
\pgfpathmoveto{\pgfqpoint{1.498962in}{1.025499in}}%
\pgfpathlineto{\pgfqpoint{1.506410in}{1.026245in}}%
\pgfpathlineto{\pgfqpoint{1.512231in}{1.027392in}}%
\pgfpathlineto{\pgfqpoint{1.517271in}{1.039172in}}%
\pgfpathlineto{\pgfqpoint{1.516880in}{1.052098in}}%
\pgfpathlineto{\pgfqpoint{1.512231in}{1.061362in}}%
\pgfpathlineto{\pgfqpoint{1.498962in}{1.063565in}}%
\pgfpathlineto{\pgfqpoint{1.491598in}{1.052098in}}%
\pgfpathlineto{\pgfqpoint{1.491123in}{1.039172in}}%
\pgfpathlineto{\pgfqpoint{1.497993in}{1.026245in}}%
\pgfpathclose%
\pgfusepath{fill}%
\end{pgfscope}%
\begin{pgfscope}%
\pgfpathrectangle{\pgfqpoint{0.211875in}{0.211875in}}{\pgfqpoint{1.313625in}{1.279725in}}%
\pgfusepath{clip}%
\pgfsetbuttcap%
\pgfsetroundjoin%
\definecolor{currentfill}{rgb}{0.644838,0.098089,0.355336}%
\pgfsetfillcolor{currentfill}%
\pgfsetlinewidth{0.000000pt}%
\definecolor{currentstroke}{rgb}{0.000000,0.000000,0.000000}%
\pgfsetstrokecolor{currentstroke}%
\pgfsetdash{}{0pt}%
\pgfpathmoveto{\pgfqpoint{1.021280in}{1.036942in}}%
\pgfpathlineto{\pgfqpoint{1.034549in}{1.028243in}}%
\pgfpathlineto{\pgfqpoint{1.042422in}{1.039172in}}%
\pgfpathlineto{\pgfqpoint{1.041758in}{1.052098in}}%
\pgfpathlineto{\pgfqpoint{1.034549in}{1.060643in}}%
\pgfpathlineto{\pgfqpoint{1.021280in}{1.053064in}}%
\pgfpathlineto{\pgfqpoint{1.020888in}{1.052098in}}%
\pgfpathlineto{\pgfqpoint{1.020508in}{1.039172in}}%
\pgfpathclose%
\pgfusepath{fill}%
\end{pgfscope}%
\begin{pgfscope}%
\pgfpathrectangle{\pgfqpoint{0.211875in}{0.211875in}}{\pgfqpoint{1.313625in}{1.279725in}}%
\pgfusepath{clip}%
\pgfsetbuttcap%
\pgfsetroundjoin%
\definecolor{currentfill}{rgb}{0.644838,0.098089,0.355336}%
\pgfsetfillcolor{currentfill}%
\pgfsetlinewidth{0.000000pt}%
\definecolor{currentstroke}{rgb}{0.000000,0.000000,0.000000}%
\pgfsetstrokecolor{currentstroke}%
\pgfsetdash{}{0pt}%
\pgfpathmoveto{\pgfqpoint{1.140701in}{1.035101in}}%
\pgfpathlineto{\pgfqpoint{1.153970in}{1.029424in}}%
\pgfpathlineto{\pgfqpoint{1.160023in}{1.039172in}}%
\pgfpathlineto{\pgfqpoint{1.159437in}{1.052098in}}%
\pgfpathlineto{\pgfqpoint{1.153970in}{1.059614in}}%
\pgfpathlineto{\pgfqpoint{1.140701in}{1.054667in}}%
\pgfpathlineto{\pgfqpoint{1.139537in}{1.052098in}}%
\pgfpathlineto{\pgfqpoint{1.139129in}{1.039172in}}%
\pgfpathclose%
\pgfusepath{fill}%
\end{pgfscope}%
\begin{pgfscope}%
\pgfpathrectangle{\pgfqpoint{0.211875in}{0.211875in}}{\pgfqpoint{1.313625in}{1.279725in}}%
\pgfusepath{clip}%
\pgfsetbuttcap%
\pgfsetroundjoin%
\definecolor{currentfill}{rgb}{0.644838,0.098089,0.355336}%
\pgfsetfillcolor{currentfill}%
\pgfsetlinewidth{0.000000pt}%
\definecolor{currentstroke}{rgb}{0.000000,0.000000,0.000000}%
\pgfsetstrokecolor{currentstroke}%
\pgfsetdash{}{0pt}%
\pgfpathmoveto{\pgfqpoint{1.260121in}{1.032444in}}%
\pgfpathlineto{\pgfqpoint{1.273390in}{1.029720in}}%
\pgfpathlineto{\pgfqpoint{1.278527in}{1.039172in}}%
\pgfpathlineto{\pgfqpoint{1.278010in}{1.052098in}}%
\pgfpathlineto{\pgfqpoint{1.273390in}{1.059353in}}%
\pgfpathlineto{\pgfqpoint{1.260121in}{1.056979in}}%
\pgfpathlineto{\pgfqpoint{1.257651in}{1.052098in}}%
\pgfpathlineto{\pgfqpoint{1.257217in}{1.039172in}}%
\pgfpathclose%
\pgfusepath{fill}%
\end{pgfscope}%
\begin{pgfscope}%
\pgfpathrectangle{\pgfqpoint{0.211875in}{0.211875in}}{\pgfqpoint{1.313625in}{1.279725in}}%
\pgfusepath{clip}%
\pgfsetbuttcap%
\pgfsetroundjoin%
\definecolor{currentfill}{rgb}{0.644838,0.098089,0.355336}%
\pgfsetfillcolor{currentfill}%
\pgfsetlinewidth{0.000000pt}%
\definecolor{currentstroke}{rgb}{0.000000,0.000000,0.000000}%
\pgfsetstrokecolor{currentstroke}%
\pgfsetdash{}{0pt}%
\pgfpathmoveto{\pgfqpoint{1.379542in}{1.029029in}}%
\pgfpathlineto{\pgfqpoint{1.392811in}{1.029077in}}%
\pgfpathlineto{\pgfqpoint{1.397662in}{1.039172in}}%
\pgfpathlineto{\pgfqpoint{1.397210in}{1.052098in}}%
\pgfpathlineto{\pgfqpoint{1.392811in}{1.059906in}}%
\pgfpathlineto{\pgfqpoint{1.379542in}{1.059950in}}%
\pgfpathlineto{\pgfqpoint{1.375082in}{1.052098in}}%
\pgfpathlineto{\pgfqpoint{1.374625in}{1.039172in}}%
\pgfpathclose%
\pgfusepath{fill}%
\end{pgfscope}%
\begin{pgfscope}%
\pgfpathrectangle{\pgfqpoint{0.211875in}{0.211875in}}{\pgfqpoint{1.313625in}{1.279725in}}%
\pgfusepath{clip}%
\pgfsetbuttcap%
\pgfsetroundjoin%
\definecolor{currentfill}{rgb}{0.644838,0.098089,0.355336}%
\pgfsetfillcolor{currentfill}%
\pgfsetlinewidth{0.000000pt}%
\definecolor{currentstroke}{rgb}{0.000000,0.000000,0.000000}%
\pgfsetstrokecolor{currentstroke}%
\pgfsetdash{}{0pt}%
\pgfpathmoveto{\pgfqpoint{0.251682in}{1.093751in}}%
\pgfpathlineto{\pgfqpoint{0.264951in}{1.091784in}}%
\pgfpathlineto{\pgfqpoint{0.278220in}{1.094271in}}%
\pgfpathlineto{\pgfqpoint{0.286746in}{1.103805in}}%
\pgfpathlineto{\pgfqpoint{0.289469in}{1.116731in}}%
\pgfpathlineto{\pgfqpoint{0.289811in}{1.129658in}}%
\pgfpathlineto{\pgfqpoint{0.288384in}{1.142584in}}%
\pgfpathlineto{\pgfqpoint{0.281713in}{1.155511in}}%
\pgfpathlineto{\pgfqpoint{0.278220in}{1.157814in}}%
\pgfpathlineto{\pgfqpoint{0.264951in}{1.160075in}}%
\pgfpathlineto{\pgfqpoint{0.251682in}{1.158288in}}%
\pgfpathlineto{\pgfqpoint{0.247480in}{1.155511in}}%
\pgfpathlineto{\pgfqpoint{0.241812in}{1.142584in}}%
\pgfpathlineto{\pgfqpoint{0.240690in}{1.129658in}}%
\pgfpathlineto{\pgfqpoint{0.240956in}{1.116731in}}%
\pgfpathlineto{\pgfqpoint{0.243138in}{1.103805in}}%
\pgfpathclose%
\pgfpathmoveto{\pgfqpoint{0.255800in}{1.103805in}}%
\pgfpathlineto{\pgfqpoint{0.251682in}{1.106937in}}%
\pgfpathlineto{\pgfqpoint{0.248020in}{1.116731in}}%
\pgfpathlineto{\pgfqpoint{0.247365in}{1.129658in}}%
\pgfpathlineto{\pgfqpoint{0.250123in}{1.142584in}}%
\pgfpathlineto{\pgfqpoint{0.251682in}{1.145180in}}%
\pgfpathlineto{\pgfqpoint{0.264951in}{1.151191in}}%
\pgfpathlineto{\pgfqpoint{0.278220in}{1.145671in}}%
\pgfpathlineto{\pgfqpoint{0.280201in}{1.142584in}}%
\pgfpathlineto{\pgfqpoint{0.283151in}{1.129658in}}%
\pgfpathlineto{\pgfqpoint{0.282444in}{1.116731in}}%
\pgfpathlineto{\pgfqpoint{0.278220in}{1.106266in}}%
\pgfpathlineto{\pgfqpoint{0.274539in}{1.103805in}}%
\pgfpathlineto{\pgfqpoint{0.264951in}{1.100568in}}%
\pgfpathclose%
\pgfusepath{fill}%
\end{pgfscope}%
\begin{pgfscope}%
\pgfpathrectangle{\pgfqpoint{0.211875in}{0.211875in}}{\pgfqpoint{1.313625in}{1.279725in}}%
\pgfusepath{clip}%
\pgfsetbuttcap%
\pgfsetroundjoin%
\definecolor{currentfill}{rgb}{0.644838,0.098089,0.355336}%
\pgfsetfillcolor{currentfill}%
\pgfsetlinewidth{0.000000pt}%
\definecolor{currentstroke}{rgb}{0.000000,0.000000,0.000000}%
\pgfsetstrokecolor{currentstroke}%
\pgfsetdash{}{0pt}%
\pgfpathmoveto{\pgfqpoint{0.371102in}{1.098465in}}%
\pgfpathlineto{\pgfqpoint{0.384371in}{1.095875in}}%
\pgfpathlineto{\pgfqpoint{0.397640in}{1.100400in}}%
\pgfpathlineto{\pgfqpoint{0.400378in}{1.103805in}}%
\pgfpathlineto{\pgfqpoint{0.404253in}{1.116731in}}%
\pgfpathlineto{\pgfqpoint{0.404739in}{1.129658in}}%
\pgfpathlineto{\pgfqpoint{0.402711in}{1.142584in}}%
\pgfpathlineto{\pgfqpoint{0.397640in}{1.151384in}}%
\pgfpathlineto{\pgfqpoint{0.387815in}{1.155511in}}%
\pgfpathlineto{\pgfqpoint{0.384371in}{1.156357in}}%
\pgfpathlineto{\pgfqpoint{0.378825in}{1.155511in}}%
\pgfpathlineto{\pgfqpoint{0.371102in}{1.153605in}}%
\pgfpathlineto{\pgfqpoint{0.363707in}{1.142584in}}%
\pgfpathlineto{\pgfqpoint{0.361732in}{1.129658in}}%
\pgfpathlineto{\pgfqpoint{0.362201in}{1.116731in}}%
\pgfpathlineto{\pgfqpoint{0.366036in}{1.103805in}}%
\pgfpathclose%
\pgfpathmoveto{\pgfqpoint{0.369563in}{1.116731in}}%
\pgfpathlineto{\pgfqpoint{0.368690in}{1.129658in}}%
\pgfpathlineto{\pgfqpoint{0.371102in}{1.138756in}}%
\pgfpathlineto{\pgfqpoint{0.374725in}{1.142584in}}%
\pgfpathlineto{\pgfqpoint{0.384371in}{1.146499in}}%
\pgfpathlineto{\pgfqpoint{0.391303in}{1.142584in}}%
\pgfpathlineto{\pgfqpoint{0.397640in}{1.132728in}}%
\pgfpathlineto{\pgfqpoint{0.398347in}{1.129658in}}%
\pgfpathlineto{\pgfqpoint{0.397640in}{1.118667in}}%
\pgfpathlineto{\pgfqpoint{0.397340in}{1.116731in}}%
\pgfpathlineto{\pgfqpoint{0.384371in}{1.105136in}}%
\pgfpathlineto{\pgfqpoint{0.371102in}{1.113049in}}%
\pgfpathclose%
\pgfusepath{fill}%
\end{pgfscope}%
\begin{pgfscope}%
\pgfpathrectangle{\pgfqpoint{0.211875in}{0.211875in}}{\pgfqpoint{1.313625in}{1.279725in}}%
\pgfusepath{clip}%
\pgfsetbuttcap%
\pgfsetroundjoin%
\definecolor{currentfill}{rgb}{0.644838,0.098089,0.355336}%
\pgfsetfillcolor{currentfill}%
\pgfsetlinewidth{0.000000pt}%
\definecolor{currentstroke}{rgb}{0.000000,0.000000,0.000000}%
\pgfsetstrokecolor{currentstroke}%
\pgfsetdash{}{0pt}%
\pgfpathmoveto{\pgfqpoint{0.490523in}{1.101977in}}%
\pgfpathlineto{\pgfqpoint{0.503792in}{1.099465in}}%
\pgfpathlineto{\pgfqpoint{0.512849in}{1.103805in}}%
\pgfpathlineto{\pgfqpoint{0.517061in}{1.108414in}}%
\pgfpathlineto{\pgfqpoint{0.519751in}{1.116731in}}%
\pgfpathlineto{\pgfqpoint{0.520352in}{1.129658in}}%
\pgfpathlineto{\pgfqpoint{0.517840in}{1.142584in}}%
\pgfpathlineto{\pgfqpoint{0.517061in}{1.144098in}}%
\pgfpathlineto{\pgfqpoint{0.503792in}{1.152457in}}%
\pgfpathlineto{\pgfqpoint{0.490523in}{1.149573in}}%
\pgfpathlineto{\pgfqpoint{0.485293in}{1.142584in}}%
\pgfpathlineto{\pgfqpoint{0.482493in}{1.129658in}}%
\pgfpathlineto{\pgfqpoint{0.483158in}{1.116731in}}%
\pgfpathlineto{\pgfqpoint{0.488590in}{1.103805in}}%
\pgfpathclose%
\pgfpathmoveto{\pgfqpoint{0.491886in}{1.116731in}}%
\pgfpathlineto{\pgfqpoint{0.490523in}{1.121042in}}%
\pgfpathlineto{\pgfqpoint{0.489811in}{1.129658in}}%
\pgfpathlineto{\pgfqpoint{0.490523in}{1.132064in}}%
\pgfpathlineto{\pgfqpoint{0.503792in}{1.142041in}}%
\pgfpathlineto{\pgfqpoint{0.511251in}{1.129658in}}%
\pgfpathlineto{\pgfqpoint{0.509363in}{1.116731in}}%
\pgfpathlineto{\pgfqpoint{0.503792in}{1.110844in}}%
\pgfpathclose%
\pgfusepath{fill}%
\end{pgfscope}%
\begin{pgfscope}%
\pgfpathrectangle{\pgfqpoint{0.211875in}{0.211875in}}{\pgfqpoint{1.313625in}{1.279725in}}%
\pgfusepath{clip}%
\pgfsetbuttcap%
\pgfsetroundjoin%
\definecolor{currentfill}{rgb}{0.644838,0.098089,0.355336}%
\pgfsetfillcolor{currentfill}%
\pgfsetlinewidth{0.000000pt}%
\definecolor{currentstroke}{rgb}{0.000000,0.000000,0.000000}%
\pgfsetstrokecolor{currentstroke}%
\pgfsetdash{}{0pt}%
\pgfpathmoveto{\pgfqpoint{0.623212in}{1.102615in}}%
\pgfpathlineto{\pgfqpoint{0.625369in}{1.103805in}}%
\pgfpathlineto{\pgfqpoint{0.635310in}{1.116731in}}%
\pgfpathlineto{\pgfqpoint{0.636481in}{1.128808in}}%
\pgfpathlineto{\pgfqpoint{0.636524in}{1.129658in}}%
\pgfpathlineto{\pgfqpoint{0.636481in}{1.129895in}}%
\pgfpathlineto{\pgfqpoint{0.631349in}{1.142584in}}%
\pgfpathlineto{\pgfqpoint{0.623212in}{1.148840in}}%
\pgfpathlineto{\pgfqpoint{0.609943in}{1.146616in}}%
\pgfpathlineto{\pgfqpoint{0.606578in}{1.142584in}}%
\pgfpathlineto{\pgfqpoint{0.602967in}{1.129658in}}%
\pgfpathlineto{\pgfqpoint{0.603825in}{1.116731in}}%
\pgfpathlineto{\pgfqpoint{0.609943in}{1.104976in}}%
\pgfpathlineto{\pgfqpoint{0.614693in}{1.103805in}}%
\pgfpathclose%
\pgfpathmoveto{\pgfqpoint{0.620850in}{1.116731in}}%
\pgfpathlineto{\pgfqpoint{0.614047in}{1.129658in}}%
\pgfpathlineto{\pgfqpoint{0.623212in}{1.134425in}}%
\pgfpathlineto{\pgfqpoint{0.625693in}{1.129658in}}%
\pgfpathlineto{\pgfqpoint{0.623847in}{1.116731in}}%
\pgfpathlineto{\pgfqpoint{0.623212in}{1.115955in}}%
\pgfpathclose%
\pgfusepath{fill}%
\end{pgfscope}%
\begin{pgfscope}%
\pgfpathrectangle{\pgfqpoint{0.211875in}{0.211875in}}{\pgfqpoint{1.313625in}{1.279725in}}%
\pgfusepath{clip}%
\pgfsetbuttcap%
\pgfsetroundjoin%
\definecolor{currentfill}{rgb}{0.644838,0.098089,0.355336}%
\pgfsetfillcolor{currentfill}%
\pgfsetlinewidth{0.000000pt}%
\definecolor{currentstroke}{rgb}{0.000000,0.000000,0.000000}%
\pgfsetstrokecolor{currentstroke}%
\pgfsetdash{}{0pt}%
\pgfpathmoveto{\pgfqpoint{0.729364in}{1.107832in}}%
\pgfpathlineto{\pgfqpoint{0.742633in}{1.106257in}}%
\pgfpathlineto{\pgfqpoint{0.750145in}{1.116731in}}%
\pgfpathlineto{\pgfqpoint{0.751398in}{1.129658in}}%
\pgfpathlineto{\pgfqpoint{0.746166in}{1.142584in}}%
\pgfpathlineto{\pgfqpoint{0.742633in}{1.145677in}}%
\pgfpathlineto{\pgfqpoint{0.729364in}{1.144524in}}%
\pgfpathlineto{\pgfqpoint{0.727554in}{1.142584in}}%
\pgfpathlineto{\pgfqpoint{0.723131in}{1.129658in}}%
\pgfpathlineto{\pgfqpoint{0.724183in}{1.116731in}}%
\pgfpathclose%
\pgfusepath{fill}%
\end{pgfscope}%
\begin{pgfscope}%
\pgfpathrectangle{\pgfqpoint{0.211875in}{0.211875in}}{\pgfqpoint{1.313625in}{1.279725in}}%
\pgfusepath{clip}%
\pgfsetbuttcap%
\pgfsetroundjoin%
\definecolor{currentfill}{rgb}{0.644838,0.098089,0.355336}%
\pgfsetfillcolor{currentfill}%
\pgfsetlinewidth{0.000000pt}%
\definecolor{currentstroke}{rgb}{0.000000,0.000000,0.000000}%
\pgfsetstrokecolor{currentstroke}%
\pgfsetdash{}{0pt}%
\pgfpathmoveto{\pgfqpoint{0.848784in}{1.109706in}}%
\pgfpathlineto{\pgfqpoint{0.862053in}{1.110001in}}%
\pgfpathlineto{\pgfqpoint{0.866329in}{1.116731in}}%
\pgfpathlineto{\pgfqpoint{0.867562in}{1.129658in}}%
\pgfpathlineto{\pgfqpoint{0.862408in}{1.142584in}}%
\pgfpathlineto{\pgfqpoint{0.862053in}{1.142935in}}%
\pgfpathlineto{\pgfqpoint{0.848784in}{1.143151in}}%
\pgfpathlineto{\pgfqpoint{0.848190in}{1.142584in}}%
\pgfpathlineto{\pgfqpoint{0.842938in}{1.129658in}}%
\pgfpathlineto{\pgfqpoint{0.844188in}{1.116731in}}%
\pgfpathclose%
\pgfusepath{fill}%
\end{pgfscope}%
\begin{pgfscope}%
\pgfpathrectangle{\pgfqpoint{0.211875in}{0.211875in}}{\pgfqpoint{1.313625in}{1.279725in}}%
\pgfusepath{clip}%
\pgfsetbuttcap%
\pgfsetroundjoin%
\definecolor{currentfill}{rgb}{0.644838,0.098089,0.355336}%
\pgfsetfillcolor{currentfill}%
\pgfsetlinewidth{0.000000pt}%
\definecolor{currentstroke}{rgb}{0.000000,0.000000,0.000000}%
\pgfsetstrokecolor{currentstroke}%
\pgfsetdash{}{0pt}%
\pgfpathmoveto{\pgfqpoint{0.968205in}{1.110734in}}%
\pgfpathlineto{\pgfqpoint{0.981473in}{1.113194in}}%
\pgfpathlineto{\pgfqpoint{0.983478in}{1.116731in}}%
\pgfpathlineto{\pgfqpoint{0.984672in}{1.129658in}}%
\pgfpathlineto{\pgfqpoint{0.981473in}{1.138540in}}%
\pgfpathlineto{\pgfqpoint{0.968205in}{1.142205in}}%
\pgfpathlineto{\pgfqpoint{0.962305in}{1.129658in}}%
\pgfpathlineto{\pgfqpoint{0.963763in}{1.116731in}}%
\pgfpathclose%
\pgfusepath{fill}%
\end{pgfscope}%
\begin{pgfscope}%
\pgfpathrectangle{\pgfqpoint{0.211875in}{0.211875in}}{\pgfqpoint{1.313625in}{1.279725in}}%
\pgfusepath{clip}%
\pgfsetbuttcap%
\pgfsetroundjoin%
\definecolor{currentfill}{rgb}{0.644838,0.098089,0.355336}%
\pgfsetfillcolor{currentfill}%
\pgfsetlinewidth{0.000000pt}%
\definecolor{currentstroke}{rgb}{0.000000,0.000000,0.000000}%
\pgfsetstrokecolor{currentstroke}%
\pgfsetdash{}{0pt}%
\pgfpathmoveto{\pgfqpoint{1.087625in}{1.111007in}}%
\pgfpathlineto{\pgfqpoint{1.100894in}{1.115834in}}%
\pgfpathlineto{\pgfqpoint{1.101349in}{1.116731in}}%
\pgfpathlineto{\pgfqpoint{1.102489in}{1.129658in}}%
\pgfpathlineto{\pgfqpoint{1.100894in}{1.134605in}}%
\pgfpathlineto{\pgfqpoint{1.087625in}{1.141799in}}%
\pgfpathlineto{\pgfqpoint{1.081089in}{1.129658in}}%
\pgfpathlineto{\pgfqpoint{1.082772in}{1.116731in}}%
\pgfpathclose%
\pgfusepath{fill}%
\end{pgfscope}%
\begin{pgfscope}%
\pgfpathrectangle{\pgfqpoint{0.211875in}{0.211875in}}{\pgfqpoint{1.313625in}{1.279725in}}%
\pgfusepath{clip}%
\pgfsetbuttcap%
\pgfsetroundjoin%
\definecolor{currentfill}{rgb}{0.644838,0.098089,0.355336}%
\pgfsetfillcolor{currentfill}%
\pgfsetlinewidth{0.000000pt}%
\definecolor{currentstroke}{rgb}{0.000000,0.000000,0.000000}%
\pgfsetstrokecolor{currentstroke}%
\pgfsetdash{}{0pt}%
\pgfpathmoveto{\pgfqpoint{1.207045in}{1.110574in}}%
\pgfpathlineto{\pgfqpoint{1.218514in}{1.116731in}}%
\pgfpathlineto{\pgfqpoint{1.220314in}{1.122925in}}%
\pgfpathlineto{\pgfqpoint{1.220858in}{1.129658in}}%
\pgfpathlineto{\pgfqpoint{1.220314in}{1.131538in}}%
\pgfpathlineto{\pgfqpoint{1.207045in}{1.142444in}}%
\pgfpathlineto{\pgfqpoint{1.199037in}{1.129658in}}%
\pgfpathlineto{\pgfqpoint{1.200975in}{1.116731in}}%
\pgfpathclose%
\pgfusepath{fill}%
\end{pgfscope}%
\begin{pgfscope}%
\pgfpathrectangle{\pgfqpoint{0.211875in}{0.211875in}}{\pgfqpoint{1.313625in}{1.279725in}}%
\pgfusepath{clip}%
\pgfsetbuttcap%
\pgfsetroundjoin%
\definecolor{currentfill}{rgb}{0.644838,0.098089,0.355336}%
\pgfsetfillcolor{currentfill}%
\pgfsetlinewidth{0.000000pt}%
\definecolor{currentstroke}{rgb}{0.000000,0.000000,0.000000}%
\pgfsetstrokecolor{currentstroke}%
\pgfsetdash{}{0pt}%
\pgfpathmoveto{\pgfqpoint{1.326466in}{1.109457in}}%
\pgfpathlineto{\pgfqpoint{1.336868in}{1.116731in}}%
\pgfpathlineto{\pgfqpoint{1.339580in}{1.129658in}}%
\pgfpathlineto{\pgfqpoint{1.328183in}{1.142584in}}%
\pgfpathlineto{\pgfqpoint{1.326466in}{1.143333in}}%
\pgfpathlineto{\pgfqpoint{1.325064in}{1.142584in}}%
\pgfpathlineto{\pgfqpoint{1.315685in}{1.129658in}}%
\pgfpathlineto{\pgfqpoint{1.317928in}{1.116731in}}%
\pgfpathclose%
\pgfusepath{fill}%
\end{pgfscope}%
\begin{pgfscope}%
\pgfpathrectangle{\pgfqpoint{0.211875in}{0.211875in}}{\pgfqpoint{1.313625in}{1.279725in}}%
\pgfusepath{clip}%
\pgfsetbuttcap%
\pgfsetroundjoin%
\definecolor{currentfill}{rgb}{0.644838,0.098089,0.355336}%
\pgfsetfillcolor{currentfill}%
\pgfsetlinewidth{0.000000pt}%
\definecolor{currentstroke}{rgb}{0.000000,0.000000,0.000000}%
\pgfsetstrokecolor{currentstroke}%
\pgfsetdash{}{0pt}%
\pgfpathmoveto{\pgfqpoint{1.445886in}{1.107648in}}%
\pgfpathlineto{\pgfqpoint{1.456440in}{1.116731in}}%
\pgfpathlineto{\pgfqpoint{1.458525in}{1.129658in}}%
\pgfpathlineto{\pgfqpoint{1.449753in}{1.142584in}}%
\pgfpathlineto{\pgfqpoint{1.445886in}{1.144659in}}%
\pgfpathlineto{\pgfqpoint{1.441107in}{1.142584in}}%
\pgfpathlineto{\pgfqpoint{1.432617in}{1.133122in}}%
\pgfpathlineto{\pgfqpoint{1.431709in}{1.129658in}}%
\pgfpathlineto{\pgfqpoint{1.432617in}{1.117257in}}%
\pgfpathlineto{\pgfqpoint{1.432730in}{1.116731in}}%
\pgfpathclose%
\pgfusepath{fill}%
\end{pgfscope}%
\begin{pgfscope}%
\pgfpathrectangle{\pgfqpoint{0.211875in}{0.211875in}}{\pgfqpoint{1.313625in}{1.279725in}}%
\pgfusepath{clip}%
\pgfsetbuttcap%
\pgfsetroundjoin%
\definecolor{currentfill}{rgb}{0.644838,0.098089,0.355336}%
\pgfsetfillcolor{currentfill}%
\pgfsetlinewidth{0.000000pt}%
\definecolor{currentstroke}{rgb}{0.000000,0.000000,0.000000}%
\pgfsetstrokecolor{currentstroke}%
\pgfsetdash{}{0pt}%
\pgfpathmoveto{\pgfqpoint{0.225144in}{1.175780in}}%
\pgfpathlineto{\pgfqpoint{0.229135in}{1.181364in}}%
\pgfpathlineto{\pgfqpoint{0.231819in}{1.194290in}}%
\pgfpathlineto{\pgfqpoint{0.232376in}{1.207217in}}%
\pgfpathlineto{\pgfqpoint{0.231883in}{1.220143in}}%
\pgfpathlineto{\pgfqpoint{0.229331in}{1.233070in}}%
\pgfpathlineto{\pgfqpoint{0.225144in}{1.238974in}}%
\pgfpathlineto{\pgfqpoint{0.211875in}{1.243218in}}%
\pgfpathlineto{\pgfqpoint{0.211875in}{1.234620in}}%
\pgfpathlineto{\pgfqpoint{0.214939in}{1.233070in}}%
\pgfpathlineto{\pgfqpoint{0.225144in}{1.220695in}}%
\pgfpathlineto{\pgfqpoint{0.225320in}{1.220143in}}%
\pgfpathlineto{\pgfqpoint{0.226599in}{1.207217in}}%
\pgfpathlineto{\pgfqpoint{0.225264in}{1.194290in}}%
\pgfpathlineto{\pgfqpoint{0.225144in}{1.193918in}}%
\pgfpathlineto{\pgfqpoint{0.214676in}{1.181364in}}%
\pgfpathlineto{\pgfqpoint{0.211875in}{1.179951in}}%
\pgfpathlineto{\pgfqpoint{0.211875in}{1.171406in}}%
\pgfpathclose%
\pgfusepath{fill}%
\end{pgfscope}%
\begin{pgfscope}%
\pgfpathrectangle{\pgfqpoint{0.211875in}{0.211875in}}{\pgfqpoint{1.313625in}{1.279725in}}%
\pgfusepath{clip}%
\pgfsetbuttcap%
\pgfsetroundjoin%
\definecolor{currentfill}{rgb}{0.644838,0.098089,0.355336}%
\pgfsetfillcolor{currentfill}%
\pgfsetlinewidth{0.000000pt}%
\definecolor{currentstroke}{rgb}{0.000000,0.000000,0.000000}%
\pgfsetstrokecolor{currentstroke}%
\pgfsetdash{}{0pt}%
\pgfpathmoveto{\pgfqpoint{0.318027in}{1.175481in}}%
\pgfpathlineto{\pgfqpoint{0.331295in}{1.175885in}}%
\pgfpathlineto{\pgfqpoint{0.340993in}{1.181364in}}%
\pgfpathlineto{\pgfqpoint{0.344564in}{1.186680in}}%
\pgfpathlineto{\pgfqpoint{0.346698in}{1.194290in}}%
\pgfpathlineto{\pgfqpoint{0.347617in}{1.207217in}}%
\pgfpathlineto{\pgfqpoint{0.346757in}{1.220143in}}%
\pgfpathlineto{\pgfqpoint{0.344564in}{1.228085in}}%
\pgfpathlineto{\pgfqpoint{0.341260in}{1.233070in}}%
\pgfpathlineto{\pgfqpoint{0.331295in}{1.238717in}}%
\pgfpathlineto{\pgfqpoint{0.318027in}{1.239124in}}%
\pgfpathlineto{\pgfqpoint{0.306160in}{1.233070in}}%
\pgfpathlineto{\pgfqpoint{0.304758in}{1.231123in}}%
\pgfpathlineto{\pgfqpoint{0.301659in}{1.220143in}}%
\pgfpathlineto{\pgfqpoint{0.300908in}{1.207217in}}%
\pgfpathlineto{\pgfqpoint{0.301722in}{1.194290in}}%
\pgfpathlineto{\pgfqpoint{0.304758in}{1.183704in}}%
\pgfpathlineto{\pgfqpoint{0.306466in}{1.181364in}}%
\pgfpathclose%
\pgfpathmoveto{\pgfqpoint{0.310518in}{1.194290in}}%
\pgfpathlineto{\pgfqpoint{0.307968in}{1.207217in}}%
\pgfpathlineto{\pgfqpoint{0.310434in}{1.220143in}}%
\pgfpathlineto{\pgfqpoint{0.318027in}{1.229149in}}%
\pgfpathlineto{\pgfqpoint{0.331295in}{1.228651in}}%
\pgfpathlineto{\pgfqpoint{0.338081in}{1.220143in}}%
\pgfpathlineto{\pgfqpoint{0.340554in}{1.207217in}}%
\pgfpathlineto{\pgfqpoint{0.338003in}{1.194290in}}%
\pgfpathlineto{\pgfqpoint{0.331295in}{1.185934in}}%
\pgfpathlineto{\pgfqpoint{0.318027in}{1.185440in}}%
\pgfpathclose%
\pgfusepath{fill}%
\end{pgfscope}%
\begin{pgfscope}%
\pgfpathrectangle{\pgfqpoint{0.211875in}{0.211875in}}{\pgfqpoint{1.313625in}{1.279725in}}%
\pgfusepath{clip}%
\pgfsetbuttcap%
\pgfsetroundjoin%
\definecolor{currentfill}{rgb}{0.644838,0.098089,0.355336}%
\pgfsetfillcolor{currentfill}%
\pgfsetlinewidth{0.000000pt}%
\definecolor{currentstroke}{rgb}{0.000000,0.000000,0.000000}%
\pgfsetstrokecolor{currentstroke}%
\pgfsetdash{}{0pt}%
\pgfpathmoveto{\pgfqpoint{0.437447in}{1.178990in}}%
\pgfpathlineto{\pgfqpoint{0.450716in}{1.180083in}}%
\pgfpathlineto{\pgfqpoint{0.452751in}{1.181364in}}%
\pgfpathlineto{\pgfqpoint{0.461265in}{1.194290in}}%
\pgfpathlineto{\pgfqpoint{0.462996in}{1.207217in}}%
\pgfpathlineto{\pgfqpoint{0.461343in}{1.220143in}}%
\pgfpathlineto{\pgfqpoint{0.452979in}{1.233070in}}%
\pgfpathlineto{\pgfqpoint{0.450716in}{1.234498in}}%
\pgfpathlineto{\pgfqpoint{0.437447in}{1.235588in}}%
\pgfpathlineto{\pgfqpoint{0.431859in}{1.233070in}}%
\pgfpathlineto{\pgfqpoint{0.424178in}{1.224618in}}%
\pgfpathlineto{\pgfqpoint{0.422715in}{1.220143in}}%
\pgfpathlineto{\pgfqpoint{0.421573in}{1.207217in}}%
\pgfpathlineto{\pgfqpoint{0.422776in}{1.194290in}}%
\pgfpathlineto{\pgfqpoint{0.424178in}{1.190057in}}%
\pgfpathlineto{\pgfqpoint{0.432165in}{1.181364in}}%
\pgfpathclose%
\pgfpathmoveto{\pgfqpoint{0.433319in}{1.194290in}}%
\pgfpathlineto{\pgfqpoint{0.429875in}{1.207217in}}%
\pgfpathlineto{\pgfqpoint{0.433233in}{1.220143in}}%
\pgfpathlineto{\pgfqpoint{0.437447in}{1.224552in}}%
\pgfpathlineto{\pgfqpoint{0.450716in}{1.222364in}}%
\pgfpathlineto{\pgfqpoint{0.452298in}{1.220143in}}%
\pgfpathlineto{\pgfqpoint{0.455034in}{1.207217in}}%
\pgfpathlineto{\pgfqpoint{0.452231in}{1.194290in}}%
\pgfpathlineto{\pgfqpoint{0.450716in}{1.192178in}}%
\pgfpathlineto{\pgfqpoint{0.437447in}{1.189998in}}%
\pgfpathclose%
\pgfusepath{fill}%
\end{pgfscope}%
\begin{pgfscope}%
\pgfpathrectangle{\pgfqpoint{0.211875in}{0.211875in}}{\pgfqpoint{1.313625in}{1.279725in}}%
\pgfusepath{clip}%
\pgfsetbuttcap%
\pgfsetroundjoin%
\definecolor{currentfill}{rgb}{0.644838,0.098089,0.355336}%
\pgfsetfillcolor{currentfill}%
\pgfsetlinewidth{0.000000pt}%
\definecolor{currentstroke}{rgb}{0.000000,0.000000,0.000000}%
\pgfsetstrokecolor{currentstroke}%
\pgfsetdash{}{0pt}%
\pgfpathmoveto{\pgfqpoint{0.543598in}{1.193949in}}%
\pgfpathlineto{\pgfqpoint{0.556867in}{1.182041in}}%
\pgfpathlineto{\pgfqpoint{0.570136in}{1.185107in}}%
\pgfpathlineto{\pgfqpoint{0.576045in}{1.194290in}}%
\pgfpathlineto{\pgfqpoint{0.578014in}{1.207217in}}%
\pgfpathlineto{\pgfqpoint{0.576114in}{1.220143in}}%
\pgfpathlineto{\pgfqpoint{0.570136in}{1.229500in}}%
\pgfpathlineto{\pgfqpoint{0.556867in}{1.232563in}}%
\pgfpathlineto{\pgfqpoint{0.543598in}{1.220645in}}%
\pgfpathlineto{\pgfqpoint{0.543411in}{1.220143in}}%
\pgfpathlineto{\pgfqpoint{0.541919in}{1.207217in}}%
\pgfpathlineto{\pgfqpoint{0.543470in}{1.194290in}}%
\pgfpathclose%
\pgfpathmoveto{\pgfqpoint{0.556267in}{1.194290in}}%
\pgfpathlineto{\pgfqpoint{0.551796in}{1.207217in}}%
\pgfpathlineto{\pgfqpoint{0.556175in}{1.220143in}}%
\pgfpathlineto{\pgfqpoint{0.556867in}{1.220775in}}%
\pgfpathlineto{\pgfqpoint{0.558922in}{1.220143in}}%
\pgfpathlineto{\pgfqpoint{0.570136in}{1.209424in}}%
\pgfpathlineto{\pgfqpoint{0.570584in}{1.207217in}}%
\pgfpathlineto{\pgfqpoint{0.570136in}{1.205054in}}%
\pgfpathlineto{\pgfqpoint{0.558651in}{1.194290in}}%
\pgfpathlineto{\pgfqpoint{0.556867in}{1.193745in}}%
\pgfpathclose%
\pgfusepath{fill}%
\end{pgfscope}%
\begin{pgfscope}%
\pgfpathrectangle{\pgfqpoint{0.211875in}{0.211875in}}{\pgfqpoint{1.313625in}{1.279725in}}%
\pgfusepath{clip}%
\pgfsetbuttcap%
\pgfsetroundjoin%
\definecolor{currentfill}{rgb}{0.644838,0.098089,0.355336}%
\pgfsetfillcolor{currentfill}%
\pgfsetlinewidth{0.000000pt}%
\definecolor{currentstroke}{rgb}{0.000000,0.000000,0.000000}%
\pgfsetstrokecolor{currentstroke}%
\pgfsetdash{}{0pt}%
\pgfpathmoveto{\pgfqpoint{0.676288in}{1.185249in}}%
\pgfpathlineto{\pgfqpoint{0.689557in}{1.190384in}}%
\pgfpathlineto{\pgfqpoint{0.691816in}{1.194290in}}%
\pgfpathlineto{\pgfqpoint{0.693948in}{1.207217in}}%
\pgfpathlineto{\pgfqpoint{0.691879in}{1.220143in}}%
\pgfpathlineto{\pgfqpoint{0.689557in}{1.224189in}}%
\pgfpathlineto{\pgfqpoint{0.676288in}{1.229329in}}%
\pgfpathlineto{\pgfqpoint{0.664552in}{1.220143in}}%
\pgfpathlineto{\pgfqpoint{0.663019in}{1.215238in}}%
\pgfpathlineto{\pgfqpoint{0.661955in}{1.207217in}}%
\pgfpathlineto{\pgfqpoint{0.663019in}{1.199427in}}%
\pgfpathlineto{\pgfqpoint{0.664671in}{1.194290in}}%
\pgfpathclose%
\pgfpathmoveto{\pgfqpoint{0.673813in}{1.207217in}}%
\pgfpathlineto{\pgfqpoint{0.676288in}{1.213359in}}%
\pgfpathlineto{\pgfqpoint{0.680773in}{1.207217in}}%
\pgfpathlineto{\pgfqpoint{0.676288in}{1.201179in}}%
\pgfpathclose%
\pgfusepath{fill}%
\end{pgfscope}%
\begin{pgfscope}%
\pgfpathrectangle{\pgfqpoint{0.211875in}{0.211875in}}{\pgfqpoint{1.313625in}{1.279725in}}%
\pgfusepath{clip}%
\pgfsetbuttcap%
\pgfsetroundjoin%
\definecolor{currentfill}{rgb}{0.644838,0.098089,0.355336}%
\pgfsetfillcolor{currentfill}%
\pgfsetlinewidth{0.000000pt}%
\definecolor{currentstroke}{rgb}{0.000000,0.000000,0.000000}%
\pgfsetstrokecolor{currentstroke}%
\pgfsetdash{}{0pt}%
\pgfpathmoveto{\pgfqpoint{0.795708in}{1.187779in}}%
\pgfpathlineto{\pgfqpoint{0.807317in}{1.194290in}}%
\pgfpathlineto{\pgfqpoint{0.808977in}{1.197529in}}%
\pgfpathlineto{\pgfqpoint{0.810596in}{1.207217in}}%
\pgfpathlineto{\pgfqpoint{0.808977in}{1.217141in}}%
\pgfpathlineto{\pgfqpoint{0.807473in}{1.220143in}}%
\pgfpathlineto{\pgfqpoint{0.795708in}{1.226780in}}%
\pgfpathlineto{\pgfqpoint{0.785584in}{1.220143in}}%
\pgfpathlineto{\pgfqpoint{0.782439in}{1.212336in}}%
\pgfpathlineto{\pgfqpoint{0.781676in}{1.207217in}}%
\pgfpathlineto{\pgfqpoint{0.782439in}{1.202230in}}%
\pgfpathlineto{\pgfqpoint{0.785719in}{1.194290in}}%
\pgfpathclose%
\pgfusepath{fill}%
\end{pgfscope}%
\begin{pgfscope}%
\pgfpathrectangle{\pgfqpoint{0.211875in}{0.211875in}}{\pgfqpoint{1.313625in}{1.279725in}}%
\pgfusepath{clip}%
\pgfsetbuttcap%
\pgfsetroundjoin%
\definecolor{currentfill}{rgb}{0.644838,0.098089,0.355336}%
\pgfsetfillcolor{currentfill}%
\pgfsetlinewidth{0.000000pt}%
\definecolor{currentstroke}{rgb}{0.000000,0.000000,0.000000}%
\pgfsetstrokecolor{currentstroke}%
\pgfsetdash{}{0pt}%
\pgfpathmoveto{\pgfqpoint{0.915129in}{1.189694in}}%
\pgfpathlineto{\pgfqpoint{0.921822in}{1.194290in}}%
\pgfpathlineto{\pgfqpoint{0.927063in}{1.207217in}}%
\pgfpathlineto{\pgfqpoint{0.921946in}{1.220143in}}%
\pgfpathlineto{\pgfqpoint{0.915129in}{1.224852in}}%
\pgfpathlineto{\pgfqpoint{0.906227in}{1.220143in}}%
\pgfpathlineto{\pgfqpoint{0.901860in}{1.211986in}}%
\pgfpathlineto{\pgfqpoint{0.901065in}{1.207217in}}%
\pgfpathlineto{\pgfqpoint{0.901860in}{1.202561in}}%
\pgfpathlineto{\pgfqpoint{0.906389in}{1.194290in}}%
\pgfpathclose%
\pgfusepath{fill}%
\end{pgfscope}%
\begin{pgfscope}%
\pgfpathrectangle{\pgfqpoint{0.211875in}{0.211875in}}{\pgfqpoint{1.313625in}{1.279725in}}%
\pgfusepath{clip}%
\pgfsetbuttcap%
\pgfsetroundjoin%
\definecolor{currentfill}{rgb}{0.644838,0.098089,0.355336}%
\pgfsetfillcolor{currentfill}%
\pgfsetlinewidth{0.000000pt}%
\definecolor{currentstroke}{rgb}{0.000000,0.000000,0.000000}%
\pgfsetstrokecolor{currentstroke}%
\pgfsetdash{}{0pt}%
\pgfpathmoveto{\pgfqpoint{1.034549in}{1.191030in}}%
\pgfpathlineto{\pgfqpoint{1.038557in}{1.194290in}}%
\pgfpathlineto{\pgfqpoint{1.043167in}{1.207217in}}%
\pgfpathlineto{\pgfqpoint{1.038661in}{1.220143in}}%
\pgfpathlineto{\pgfqpoint{1.034549in}{1.223507in}}%
\pgfpathlineto{\pgfqpoint{1.026169in}{1.220143in}}%
\pgfpathlineto{\pgfqpoint{1.021280in}{1.213653in}}%
\pgfpathlineto{\pgfqpoint{1.020084in}{1.207217in}}%
\pgfpathlineto{\pgfqpoint{1.021280in}{1.200922in}}%
\pgfpathlineto{\pgfqpoint{1.026382in}{1.194290in}}%
\pgfpathclose%
\pgfusepath{fill}%
\end{pgfscope}%
\begin{pgfscope}%
\pgfpathrectangle{\pgfqpoint{0.211875in}{0.211875in}}{\pgfqpoint{1.313625in}{1.279725in}}%
\pgfusepath{clip}%
\pgfsetbuttcap%
\pgfsetroundjoin%
\definecolor{currentfill}{rgb}{0.644838,0.098089,0.355336}%
\pgfsetfillcolor{currentfill}%
\pgfsetlinewidth{0.000000pt}%
\definecolor{currentstroke}{rgb}{0.000000,0.000000,0.000000}%
\pgfsetstrokecolor{currentstroke}%
\pgfsetdash{}{0pt}%
\pgfpathmoveto{\pgfqpoint{1.153970in}{1.191801in}}%
\pgfpathlineto{\pgfqpoint{1.156610in}{1.194290in}}%
\pgfpathlineto{\pgfqpoint{1.160681in}{1.207217in}}%
\pgfpathlineto{\pgfqpoint{1.156700in}{1.220143in}}%
\pgfpathlineto{\pgfqpoint{1.153970in}{1.222733in}}%
\pgfpathlineto{\pgfqpoint{1.144429in}{1.220143in}}%
\pgfpathlineto{\pgfqpoint{1.140701in}{1.217001in}}%
\pgfpathlineto{\pgfqpoint{1.138670in}{1.207217in}}%
\pgfpathlineto{\pgfqpoint{1.140701in}{1.197633in}}%
\pgfpathlineto{\pgfqpoint{1.144745in}{1.194290in}}%
\pgfpathclose%
\pgfusepath{fill}%
\end{pgfscope}%
\begin{pgfscope}%
\pgfpathrectangle{\pgfqpoint{0.211875in}{0.211875in}}{\pgfqpoint{1.313625in}{1.279725in}}%
\pgfusepath{clip}%
\pgfsetbuttcap%
\pgfsetroundjoin%
\definecolor{currentfill}{rgb}{0.644838,0.098089,0.355336}%
\pgfsetfillcolor{currentfill}%
\pgfsetlinewidth{0.000000pt}%
\definecolor{currentstroke}{rgb}{0.000000,0.000000,0.000000}%
\pgfsetstrokecolor{currentstroke}%
\pgfsetdash{}{0pt}%
\pgfpathmoveto{\pgfqpoint{1.260121in}{1.193777in}}%
\pgfpathlineto{\pgfqpoint{1.273390in}{1.191997in}}%
\pgfpathlineto{\pgfqpoint{1.275520in}{1.194290in}}%
\pgfpathlineto{\pgfqpoint{1.279105in}{1.207217in}}%
\pgfpathlineto{\pgfqpoint{1.275602in}{1.220143in}}%
\pgfpathlineto{\pgfqpoint{1.273390in}{1.222538in}}%
\pgfpathlineto{\pgfqpoint{1.260121in}{1.220752in}}%
\pgfpathlineto{\pgfqpoint{1.259675in}{1.220143in}}%
\pgfpathlineto{\pgfqpoint{1.256730in}{1.207217in}}%
\pgfpathlineto{\pgfqpoint{1.259743in}{1.194290in}}%
\pgfpathclose%
\pgfusepath{fill}%
\end{pgfscope}%
\begin{pgfscope}%
\pgfpathrectangle{\pgfqpoint{0.211875in}{0.211875in}}{\pgfqpoint{1.313625in}{1.279725in}}%
\pgfusepath{clip}%
\pgfsetbuttcap%
\pgfsetroundjoin%
\definecolor{currentfill}{rgb}{0.644838,0.098089,0.355336}%
\pgfsetfillcolor{currentfill}%
\pgfsetlinewidth{0.000000pt}%
\definecolor{currentstroke}{rgb}{0.000000,0.000000,0.000000}%
\pgfsetstrokecolor{currentstroke}%
\pgfsetdash{}{0pt}%
\pgfpathmoveto{\pgfqpoint{1.379542in}{1.191550in}}%
\pgfpathlineto{\pgfqpoint{1.392811in}{1.191583in}}%
\pgfpathlineto{\pgfqpoint{1.395037in}{1.194290in}}%
\pgfpathlineto{\pgfqpoint{1.398165in}{1.207217in}}%
\pgfpathlineto{\pgfqpoint{1.395112in}{1.220143in}}%
\pgfpathlineto{\pgfqpoint{1.392811in}{1.222960in}}%
\pgfpathlineto{\pgfqpoint{1.379542in}{1.222992in}}%
\pgfpathlineto{\pgfqpoint{1.377199in}{1.220143in}}%
\pgfpathlineto{\pgfqpoint{1.374115in}{1.207217in}}%
\pgfpathlineto{\pgfqpoint{1.377274in}{1.194290in}}%
\pgfpathclose%
\pgfusepath{fill}%
\end{pgfscope}%
\begin{pgfscope}%
\pgfpathrectangle{\pgfqpoint{0.211875in}{0.211875in}}{\pgfqpoint{1.313625in}{1.279725in}}%
\pgfusepath{clip}%
\pgfsetbuttcap%
\pgfsetroundjoin%
\definecolor{currentfill}{rgb}{0.644838,0.098089,0.355336}%
\pgfsetfillcolor{currentfill}%
\pgfsetlinewidth{0.000000pt}%
\definecolor{currentstroke}{rgb}{0.000000,0.000000,0.000000}%
\pgfsetstrokecolor{currentstroke}%
\pgfsetdash{}{0pt}%
\pgfpathmoveto{\pgfqpoint{1.498962in}{1.188840in}}%
\pgfpathlineto{\pgfqpoint{1.512231in}{1.190492in}}%
\pgfpathlineto{\pgfqpoint{1.515014in}{1.194290in}}%
\pgfpathlineto{\pgfqpoint{1.517701in}{1.207217in}}%
\pgfpathlineto{\pgfqpoint{1.515086in}{1.220143in}}%
\pgfpathlineto{\pgfqpoint{1.512231in}{1.224065in}}%
\pgfpathlineto{\pgfqpoint{1.498962in}{1.225718in}}%
\pgfpathlineto{\pgfqpoint{1.493782in}{1.220143in}}%
\pgfpathlineto{\pgfqpoint{1.490597in}{1.207217in}}%
\pgfpathlineto{\pgfqpoint{1.493867in}{1.194290in}}%
\pgfpathclose%
\pgfusepath{fill}%
\end{pgfscope}%
\begin{pgfscope}%
\pgfpathrectangle{\pgfqpoint{0.211875in}{0.211875in}}{\pgfqpoint{1.313625in}{1.279725in}}%
\pgfusepath{clip}%
\pgfsetbuttcap%
\pgfsetroundjoin%
\definecolor{currentfill}{rgb}{0.644838,0.098089,0.355336}%
\pgfsetfillcolor{currentfill}%
\pgfsetlinewidth{0.000000pt}%
\definecolor{currentstroke}{rgb}{0.000000,0.000000,0.000000}%
\pgfsetstrokecolor{currentstroke}%
\pgfsetdash{}{0pt}%
\pgfpathmoveto{\pgfqpoint{0.251682in}{1.255661in}}%
\pgfpathlineto{\pgfqpoint{0.264951in}{1.254000in}}%
\pgfpathlineto{\pgfqpoint{0.278220in}{1.256164in}}%
\pgfpathlineto{\pgfqpoint{0.282416in}{1.258923in}}%
\pgfpathlineto{\pgfqpoint{0.288824in}{1.271849in}}%
\pgfpathlineto{\pgfqpoint{0.290232in}{1.284776in}}%
\pgfpathlineto{\pgfqpoint{0.289979in}{1.297702in}}%
\pgfpathlineto{\pgfqpoint{0.287557in}{1.310629in}}%
\pgfpathlineto{\pgfqpoint{0.278220in}{1.321126in}}%
\pgfpathlineto{\pgfqpoint{0.264951in}{1.323414in}}%
\pgfpathlineto{\pgfqpoint{0.251682in}{1.321740in}}%
\pgfpathlineto{\pgfqpoint{0.242296in}{1.310629in}}%
\pgfpathlineto{\pgfqpoint{0.240442in}{1.297702in}}%
\pgfpathlineto{\pgfqpoint{0.240269in}{1.284776in}}%
\pgfpathlineto{\pgfqpoint{0.241368in}{1.271849in}}%
\pgfpathlineto{\pgfqpoint{0.246731in}{1.258923in}}%
\pgfpathclose%
\pgfpathmoveto{\pgfqpoint{0.249665in}{1.271849in}}%
\pgfpathlineto{\pgfqpoint{0.246942in}{1.284776in}}%
\pgfpathlineto{\pgfqpoint{0.247511in}{1.297702in}}%
\pgfpathlineto{\pgfqpoint{0.251682in}{1.308983in}}%
\pgfpathlineto{\pgfqpoint{0.253825in}{1.310629in}}%
\pgfpathlineto{\pgfqpoint{0.264951in}{1.314577in}}%
\pgfpathlineto{\pgfqpoint{0.276602in}{1.310629in}}%
\pgfpathlineto{\pgfqpoint{0.278220in}{1.309537in}}%
\pgfpathlineto{\pgfqpoint{0.282948in}{1.297702in}}%
\pgfpathlineto{\pgfqpoint{0.283574in}{1.284776in}}%
\pgfpathlineto{\pgfqpoint{0.280654in}{1.271849in}}%
\pgfpathlineto{\pgfqpoint{0.278220in}{1.268082in}}%
\pgfpathlineto{\pgfqpoint{0.264951in}{1.262728in}}%
\pgfpathlineto{\pgfqpoint{0.251682in}{1.268516in}}%
\pgfpathclose%
\pgfusepath{fill}%
\end{pgfscope}%
\begin{pgfscope}%
\pgfpathrectangle{\pgfqpoint{0.211875in}{0.211875in}}{\pgfqpoint{1.313625in}{1.279725in}}%
\pgfusepath{clip}%
\pgfsetbuttcap%
\pgfsetroundjoin%
\definecolor{currentfill}{rgb}{0.644838,0.098089,0.355336}%
\pgfsetfillcolor{currentfill}%
\pgfsetlinewidth{0.000000pt}%
\definecolor{currentstroke}{rgb}{0.000000,0.000000,0.000000}%
\pgfsetstrokecolor{currentstroke}%
\pgfsetdash{}{0pt}%
\pgfpathmoveto{\pgfqpoint{0.384371in}{1.257696in}}%
\pgfpathlineto{\pgfqpoint{0.389372in}{1.258923in}}%
\pgfpathlineto{\pgfqpoint{0.397640in}{1.262378in}}%
\pgfpathlineto{\pgfqpoint{0.403139in}{1.271849in}}%
\pgfpathlineto{\pgfqpoint{0.405144in}{1.284776in}}%
\pgfpathlineto{\pgfqpoint{0.404741in}{1.297702in}}%
\pgfpathlineto{\pgfqpoint{0.401148in}{1.310629in}}%
\pgfpathlineto{\pgfqpoint{0.397640in}{1.315018in}}%
\pgfpathlineto{\pgfqpoint{0.384371in}{1.319297in}}%
\pgfpathlineto{\pgfqpoint{0.371102in}{1.316912in}}%
\pgfpathlineto{\pgfqpoint{0.365173in}{1.310629in}}%
\pgfpathlineto{\pgfqpoint{0.361668in}{1.297702in}}%
\pgfpathlineto{\pgfqpoint{0.361292in}{1.284776in}}%
\pgfpathlineto{\pgfqpoint{0.363237in}{1.271849in}}%
\pgfpathlineto{\pgfqpoint{0.371102in}{1.260209in}}%
\pgfpathlineto{\pgfqpoint{0.376330in}{1.258923in}}%
\pgfpathclose%
\pgfpathmoveto{\pgfqpoint{0.373335in}{1.271849in}}%
\pgfpathlineto{\pgfqpoint{0.371102in}{1.274176in}}%
\pgfpathlineto{\pgfqpoint{0.368248in}{1.284776in}}%
\pgfpathlineto{\pgfqpoint{0.369036in}{1.297702in}}%
\pgfpathlineto{\pgfqpoint{0.371102in}{1.302696in}}%
\pgfpathlineto{\pgfqpoint{0.384371in}{1.310372in}}%
\pgfpathlineto{\pgfqpoint{0.397640in}{1.298681in}}%
\pgfpathlineto{\pgfqpoint{0.397991in}{1.297702in}}%
\pgfpathlineto{\pgfqpoint{0.398754in}{1.284776in}}%
\pgfpathlineto{\pgfqpoint{0.397640in}{1.280021in}}%
\pgfpathlineto{\pgfqpoint{0.392304in}{1.271849in}}%
\pgfpathlineto{\pgfqpoint{0.384371in}{1.267390in}}%
\pgfpathclose%
\pgfusepath{fill}%
\end{pgfscope}%
\begin{pgfscope}%
\pgfpathrectangle{\pgfqpoint{0.211875in}{0.211875in}}{\pgfqpoint{1.313625in}{1.279725in}}%
\pgfusepath{clip}%
\pgfsetbuttcap%
\pgfsetroundjoin%
\definecolor{currentfill}{rgb}{0.644838,0.098089,0.355336}%
\pgfsetfillcolor{currentfill}%
\pgfsetlinewidth{0.000000pt}%
\definecolor{currentstroke}{rgb}{0.000000,0.000000,0.000000}%
\pgfsetstrokecolor{currentstroke}%
\pgfsetdash{}{0pt}%
\pgfpathmoveto{\pgfqpoint{0.490523in}{1.264240in}}%
\pgfpathlineto{\pgfqpoint{0.503792in}{1.261468in}}%
\pgfpathlineto{\pgfqpoint{0.517061in}{1.269541in}}%
\pgfpathlineto{\pgfqpoint{0.518260in}{1.271849in}}%
\pgfpathlineto{\pgfqpoint{0.520744in}{1.284776in}}%
\pgfpathlineto{\pgfqpoint{0.520220in}{1.297702in}}%
\pgfpathlineto{\pgfqpoint{0.517061in}{1.307594in}}%
\pgfpathlineto{\pgfqpoint{0.514317in}{1.310629in}}%
\pgfpathlineto{\pgfqpoint{0.503792in}{1.315690in}}%
\pgfpathlineto{\pgfqpoint{0.490523in}{1.313315in}}%
\pgfpathlineto{\pgfqpoint{0.487696in}{1.310629in}}%
\pgfpathlineto{\pgfqpoint{0.482601in}{1.297702in}}%
\pgfpathlineto{\pgfqpoint{0.482029in}{1.284776in}}%
\pgfpathlineto{\pgfqpoint{0.484793in}{1.271849in}}%
\pgfpathclose%
\pgfpathmoveto{\pgfqpoint{0.502781in}{1.271849in}}%
\pgfpathlineto{\pgfqpoint{0.490523in}{1.280851in}}%
\pgfpathlineto{\pgfqpoint{0.489345in}{1.284776in}}%
\pgfpathlineto{\pgfqpoint{0.490348in}{1.297702in}}%
\pgfpathlineto{\pgfqpoint{0.490523in}{1.298080in}}%
\pgfpathlineto{\pgfqpoint{0.503792in}{1.304624in}}%
\pgfpathlineto{\pgfqpoint{0.510295in}{1.297702in}}%
\pgfpathlineto{\pgfqpoint{0.512042in}{1.284776in}}%
\pgfpathlineto{\pgfqpoint{0.504255in}{1.271849in}}%
\pgfpathlineto{\pgfqpoint{0.503792in}{1.271543in}}%
\pgfpathclose%
\pgfusepath{fill}%
\end{pgfscope}%
\begin{pgfscope}%
\pgfpathrectangle{\pgfqpoint{0.211875in}{0.211875in}}{\pgfqpoint{1.313625in}{1.279725in}}%
\pgfusepath{clip}%
\pgfsetbuttcap%
\pgfsetroundjoin%
\definecolor{currentfill}{rgb}{0.644838,0.098089,0.355336}%
\pgfsetfillcolor{currentfill}%
\pgfsetlinewidth{0.000000pt}%
\definecolor{currentstroke}{rgb}{0.000000,0.000000,0.000000}%
\pgfsetstrokecolor{currentstroke}%
\pgfsetdash{}{0pt}%
\pgfpathmoveto{\pgfqpoint{0.609943in}{1.267202in}}%
\pgfpathlineto{\pgfqpoint{0.623212in}{1.265055in}}%
\pgfpathlineto{\pgfqpoint{0.632092in}{1.271849in}}%
\pgfpathlineto{\pgfqpoint{0.636481in}{1.282491in}}%
\pgfpathlineto{\pgfqpoint{0.636907in}{1.284776in}}%
\pgfpathlineto{\pgfqpoint{0.636481in}{1.293810in}}%
\pgfpathlineto{\pgfqpoint{0.636132in}{1.297702in}}%
\pgfpathlineto{\pgfqpoint{0.626647in}{1.310629in}}%
\pgfpathlineto{\pgfqpoint{0.623212in}{1.312530in}}%
\pgfpathlineto{\pgfqpoint{0.609943in}{1.310677in}}%
\pgfpathlineto{\pgfqpoint{0.609886in}{1.310629in}}%
\pgfpathlineto{\pgfqpoint{0.603236in}{1.297702in}}%
\pgfpathlineto{\pgfqpoint{0.602474in}{1.284776in}}%
\pgfpathlineto{\pgfqpoint{0.606042in}{1.271849in}}%
\pgfpathclose%
\pgfpathmoveto{\pgfqpoint{0.611487in}{1.284776in}}%
\pgfpathlineto{\pgfqpoint{0.617825in}{1.297702in}}%
\pgfpathlineto{\pgfqpoint{0.623212in}{1.299485in}}%
\pgfpathlineto{\pgfqpoint{0.624660in}{1.297702in}}%
\pgfpathlineto{\pgfqpoint{0.626386in}{1.284776in}}%
\pgfpathlineto{\pgfqpoint{0.623212in}{1.278744in}}%
\pgfpathclose%
\pgfusepath{fill}%
\end{pgfscope}%
\begin{pgfscope}%
\pgfpathrectangle{\pgfqpoint{0.211875in}{0.211875in}}{\pgfqpoint{1.313625in}{1.279725in}}%
\pgfusepath{clip}%
\pgfsetbuttcap%
\pgfsetroundjoin%
\definecolor{currentfill}{rgb}{0.644838,0.098089,0.355336}%
\pgfsetfillcolor{currentfill}%
\pgfsetlinewidth{0.000000pt}%
\definecolor{currentstroke}{rgb}{0.000000,0.000000,0.000000}%
\pgfsetstrokecolor{currentstroke}%
\pgfsetdash{}{0pt}%
\pgfpathmoveto{\pgfqpoint{0.729364in}{1.269302in}}%
\pgfpathlineto{\pgfqpoint{0.742633in}{1.268188in}}%
\pgfpathlineto{\pgfqpoint{0.746836in}{1.271849in}}%
\pgfpathlineto{\pgfqpoint{0.752015in}{1.284776in}}%
\pgfpathlineto{\pgfqpoint{0.750878in}{1.297702in}}%
\pgfpathlineto{\pgfqpoint{0.742633in}{1.309285in}}%
\pgfpathlineto{\pgfqpoint{0.729364in}{1.307769in}}%
\pgfpathlineto{\pgfqpoint{0.723552in}{1.297702in}}%
\pgfpathlineto{\pgfqpoint{0.722601in}{1.284776in}}%
\pgfpathlineto{\pgfqpoint{0.726974in}{1.271849in}}%
\pgfpathclose%
\pgfusepath{fill}%
\end{pgfscope}%
\begin{pgfscope}%
\pgfpathrectangle{\pgfqpoint{0.211875in}{0.211875in}}{\pgfqpoint{1.313625in}{1.279725in}}%
\pgfusepath{clip}%
\pgfsetbuttcap%
\pgfsetroundjoin%
\definecolor{currentfill}{rgb}{0.644838,0.098089,0.355336}%
\pgfsetfillcolor{currentfill}%
\pgfsetlinewidth{0.000000pt}%
\definecolor{currentstroke}{rgb}{0.000000,0.000000,0.000000}%
\pgfsetstrokecolor{currentstroke}%
\pgfsetdash{}{0pt}%
\pgfpathmoveto{\pgfqpoint{0.848784in}{1.270684in}}%
\pgfpathlineto{\pgfqpoint{0.862053in}{1.270898in}}%
\pgfpathlineto{\pgfqpoint{0.863022in}{1.271849in}}%
\pgfpathlineto{\pgfqpoint{0.868123in}{1.284776in}}%
\pgfpathlineto{\pgfqpoint{0.866994in}{1.297702in}}%
\pgfpathlineto{\pgfqpoint{0.862053in}{1.305540in}}%
\pgfpathlineto{\pgfqpoint{0.848784in}{1.305837in}}%
\pgfpathlineto{\pgfqpoint{0.843504in}{1.297702in}}%
\pgfpathlineto{\pgfqpoint{0.842361in}{1.284776in}}%
\pgfpathlineto{\pgfqpoint{0.847556in}{1.271849in}}%
\pgfpathclose%
\pgfusepath{fill}%
\end{pgfscope}%
\begin{pgfscope}%
\pgfpathrectangle{\pgfqpoint{0.211875in}{0.211875in}}{\pgfqpoint{1.313625in}{1.279725in}}%
\pgfusepath{clip}%
\pgfsetbuttcap%
\pgfsetroundjoin%
\definecolor{currentfill}{rgb}{0.644838,0.098089,0.355336}%
\pgfsetfillcolor{currentfill}%
\pgfsetlinewidth{0.000000pt}%
\definecolor{currentstroke}{rgb}{0.000000,0.000000,0.000000}%
\pgfsetstrokecolor{currentstroke}%
\pgfsetdash{}{0pt}%
\pgfpathmoveto{\pgfqpoint{0.968205in}{1.271446in}}%
\pgfpathlineto{\pgfqpoint{0.971395in}{1.271849in}}%
\pgfpathlineto{\pgfqpoint{0.981473in}{1.274584in}}%
\pgfpathlineto{\pgfqpoint{0.985190in}{1.284776in}}%
\pgfpathlineto{\pgfqpoint{0.984090in}{1.297702in}}%
\pgfpathlineto{\pgfqpoint{0.981473in}{1.302358in}}%
\pgfpathlineto{\pgfqpoint{0.968205in}{1.304770in}}%
\pgfpathlineto{\pgfqpoint{0.963010in}{1.297702in}}%
\pgfpathlineto{\pgfqpoint{0.961668in}{1.284776in}}%
\pgfpathlineto{\pgfqpoint{0.967724in}{1.271849in}}%
\pgfpathclose%
\pgfusepath{fill}%
\end{pgfscope}%
\begin{pgfscope}%
\pgfpathrectangle{\pgfqpoint{0.211875in}{0.211875in}}{\pgfqpoint{1.313625in}{1.279725in}}%
\pgfusepath{clip}%
\pgfsetbuttcap%
\pgfsetroundjoin%
\definecolor{currentfill}{rgb}{0.644838,0.098089,0.355336}%
\pgfsetfillcolor{currentfill}%
\pgfsetlinewidth{0.000000pt}%
\definecolor{currentstroke}{rgb}{0.000000,0.000000,0.000000}%
\pgfsetstrokecolor{currentstroke}%
\pgfsetdash{}{0pt}%
\pgfpathmoveto{\pgfqpoint{1.087625in}{1.271654in}}%
\pgfpathlineto{\pgfqpoint{1.088466in}{1.271849in}}%
\pgfpathlineto{\pgfqpoint{1.100894in}{1.278414in}}%
\pgfpathlineto{\pgfqpoint{1.102972in}{1.284776in}}%
\pgfpathlineto{\pgfqpoint{1.101920in}{1.297702in}}%
\pgfpathlineto{\pgfqpoint{1.100894in}{1.299742in}}%
\pgfpathlineto{\pgfqpoint{1.087625in}{1.304474in}}%
\pgfpathlineto{\pgfqpoint{1.081927in}{1.297702in}}%
\pgfpathlineto{\pgfqpoint{1.080373in}{1.284776in}}%
\pgfpathlineto{\pgfqpoint{1.087360in}{1.271849in}}%
\pgfpathclose%
\pgfusepath{fill}%
\end{pgfscope}%
\begin{pgfscope}%
\pgfpathrectangle{\pgfqpoint{0.211875in}{0.211875in}}{\pgfqpoint{1.313625in}{1.279725in}}%
\pgfusepath{clip}%
\pgfsetbuttcap%
\pgfsetroundjoin%
\definecolor{currentfill}{rgb}{0.644838,0.098089,0.355336}%
\pgfsetfillcolor{currentfill}%
\pgfsetlinewidth{0.000000pt}%
\definecolor{currentstroke}{rgb}{0.000000,0.000000,0.000000}%
\pgfsetstrokecolor{currentstroke}%
\pgfsetdash{}{0pt}%
\pgfpathmoveto{\pgfqpoint{1.207045in}{1.271346in}}%
\pgfpathlineto{\pgfqpoint{1.208552in}{1.271849in}}%
\pgfpathlineto{\pgfqpoint{1.220314in}{1.281367in}}%
\pgfpathlineto{\pgfqpoint{1.221314in}{1.284776in}}%
\pgfpathlineto{\pgfqpoint{1.220324in}{1.297702in}}%
\pgfpathlineto{\pgfqpoint{1.220314in}{1.297725in}}%
\pgfpathlineto{\pgfqpoint{1.207045in}{1.304896in}}%
\pgfpathlineto{\pgfqpoint{1.200004in}{1.297702in}}%
\pgfpathlineto{\pgfqpoint{1.198215in}{1.284776in}}%
\pgfpathlineto{\pgfqpoint{1.206250in}{1.271849in}}%
\pgfpathclose%
\pgfusepath{fill}%
\end{pgfscope}%
\begin{pgfscope}%
\pgfpathrectangle{\pgfqpoint{0.211875in}{0.211875in}}{\pgfqpoint{1.313625in}{1.279725in}}%
\pgfusepath{clip}%
\pgfsetbuttcap%
\pgfsetroundjoin%
\definecolor{currentfill}{rgb}{0.644838,0.098089,0.355336}%
\pgfsetfillcolor{currentfill}%
\pgfsetlinewidth{0.000000pt}%
\definecolor{currentstroke}{rgb}{0.000000,0.000000,0.000000}%
\pgfsetstrokecolor{currentstroke}%
\pgfsetdash{}{0pt}%
\pgfpathmoveto{\pgfqpoint{1.326466in}{1.270536in}}%
\pgfpathlineto{\pgfqpoint{1.329489in}{1.271849in}}%
\pgfpathlineto{\pgfqpoint{1.339735in}{1.283343in}}%
\pgfpathlineto{\pgfqpoint{1.340112in}{1.284776in}}%
\pgfpathlineto{\pgfqpoint{1.339735in}{1.290292in}}%
\pgfpathlineto{\pgfqpoint{1.338270in}{1.297702in}}%
\pgfpathlineto{\pgfqpoint{1.326466in}{1.306016in}}%
\pgfpathlineto{\pgfqpoint{1.316777in}{1.297702in}}%
\pgfpathlineto{\pgfqpoint{1.314711in}{1.284776in}}%
\pgfpathlineto{\pgfqpoint{1.323997in}{1.271849in}}%
\pgfpathclose%
\pgfusepath{fill}%
\end{pgfscope}%
\begin{pgfscope}%
\pgfpathrectangle{\pgfqpoint{0.211875in}{0.211875in}}{\pgfqpoint{1.313625in}{1.279725in}}%
\pgfusepath{clip}%
\pgfsetbuttcap%
\pgfsetroundjoin%
\definecolor{currentfill}{rgb}{0.644838,0.098089,0.355336}%
\pgfsetfillcolor{currentfill}%
\pgfsetlinewidth{0.000000pt}%
\definecolor{currentstroke}{rgb}{0.000000,0.000000,0.000000}%
\pgfsetstrokecolor{currentstroke}%
\pgfsetdash{}{0pt}%
\pgfpathmoveto{\pgfqpoint{1.445886in}{1.269219in}}%
\pgfpathlineto{\pgfqpoint{1.450812in}{1.271849in}}%
\pgfpathlineto{\pgfqpoint{1.459155in}{1.284168in}}%
\pgfpathlineto{\pgfqpoint{1.459299in}{1.284776in}}%
\pgfpathlineto{\pgfqpoint{1.459155in}{1.287132in}}%
\pgfpathlineto{\pgfqpoint{1.457581in}{1.297702in}}%
\pgfpathlineto{\pgfqpoint{1.445886in}{1.307841in}}%
\pgfpathlineto{\pgfqpoint{1.432617in}{1.298855in}}%
\pgfpathlineto{\pgfqpoint{1.432146in}{1.297702in}}%
\pgfpathlineto{\pgfqpoint{1.431277in}{1.284776in}}%
\pgfpathlineto{\pgfqpoint{1.432617in}{1.279741in}}%
\pgfpathlineto{\pgfqpoint{1.439797in}{1.271849in}}%
\pgfpathclose%
\pgfusepath{fill}%
\end{pgfscope}%
\begin{pgfscope}%
\pgfpathrectangle{\pgfqpoint{0.211875in}{0.211875in}}{\pgfqpoint{1.313625in}{1.279725in}}%
\pgfusepath{clip}%
\pgfsetbuttcap%
\pgfsetroundjoin%
\definecolor{currentfill}{rgb}{0.644838,0.098089,0.355336}%
\pgfsetfillcolor{currentfill}%
\pgfsetlinewidth{0.000000pt}%
\definecolor{currentstroke}{rgb}{0.000000,0.000000,0.000000}%
\pgfsetstrokecolor{currentstroke}%
\pgfsetdash{}{0pt}%
\pgfpathmoveto{\pgfqpoint{0.225144in}{1.336445in}}%
\pgfpathlineto{\pgfqpoint{0.225196in}{1.336482in}}%
\pgfpathlineto{\pgfqpoint{0.231707in}{1.349408in}}%
\pgfpathlineto{\pgfqpoint{0.232958in}{1.362335in}}%
\pgfpathlineto{\pgfqpoint{0.233103in}{1.375261in}}%
\pgfpathlineto{\pgfqpoint{0.232323in}{1.388188in}}%
\pgfpathlineto{\pgfqpoint{0.228275in}{1.401114in}}%
\pgfpathlineto{\pgfqpoint{0.225144in}{1.403917in}}%
\pgfpathlineto{\pgfqpoint{0.211875in}{1.407047in}}%
\pgfpathlineto{\pgfqpoint{0.211875in}{1.401114in}}%
\pgfpathlineto{\pgfqpoint{0.211875in}{1.398298in}}%
\pgfpathlineto{\pgfqpoint{0.224448in}{1.388188in}}%
\pgfpathlineto{\pgfqpoint{0.225144in}{1.386388in}}%
\pgfpathlineto{\pgfqpoint{0.227191in}{1.375261in}}%
\pgfpathlineto{\pgfqpoint{0.226949in}{1.362335in}}%
\pgfpathlineto{\pgfqpoint{0.225144in}{1.354349in}}%
\pgfpathlineto{\pgfqpoint{0.222698in}{1.349408in}}%
\pgfpathlineto{\pgfqpoint{0.211875in}{1.341706in}}%
\pgfpathlineto{\pgfqpoint{0.211875in}{1.336482in}}%
\pgfpathlineto{\pgfqpoint{0.211875in}{1.332917in}}%
\pgfpathclose%
\pgfusepath{fill}%
\end{pgfscope}%
\begin{pgfscope}%
\pgfpathrectangle{\pgfqpoint{0.211875in}{0.211875in}}{\pgfqpoint{1.313625in}{1.279725in}}%
\pgfusepath{clip}%
\pgfsetbuttcap%
\pgfsetroundjoin%
\definecolor{currentfill}{rgb}{0.644838,0.098089,0.355336}%
\pgfsetfillcolor{currentfill}%
\pgfsetlinewidth{0.000000pt}%
\definecolor{currentstroke}{rgb}{0.000000,0.000000,0.000000}%
\pgfsetstrokecolor{currentstroke}%
\pgfsetdash{}{0pt}%
\pgfpathmoveto{\pgfqpoint{0.304758in}{1.344061in}}%
\pgfpathlineto{\pgfqpoint{0.318027in}{1.336776in}}%
\pgfpathlineto{\pgfqpoint{0.331295in}{1.337224in}}%
\pgfpathlineto{\pgfqpoint{0.344564in}{1.346447in}}%
\pgfpathlineto{\pgfqpoint{0.345869in}{1.349408in}}%
\pgfpathlineto{\pgfqpoint{0.348081in}{1.362335in}}%
\pgfpathlineto{\pgfqpoint{0.348271in}{1.375261in}}%
\pgfpathlineto{\pgfqpoint{0.346695in}{1.388188in}}%
\pgfpathlineto{\pgfqpoint{0.344564in}{1.393785in}}%
\pgfpathlineto{\pgfqpoint{0.335845in}{1.401114in}}%
\pgfpathlineto{\pgfqpoint{0.331295in}{1.402757in}}%
\pgfpathlineto{\pgfqpoint{0.318027in}{1.403156in}}%
\pgfpathlineto{\pgfqpoint{0.311274in}{1.401114in}}%
\pgfpathlineto{\pgfqpoint{0.304758in}{1.396424in}}%
\pgfpathlineto{\pgfqpoint{0.301526in}{1.388188in}}%
\pgfpathlineto{\pgfqpoint{0.300199in}{1.375261in}}%
\pgfpathlineto{\pgfqpoint{0.300377in}{1.362335in}}%
\pgfpathlineto{\pgfqpoint{0.302317in}{1.349408in}}%
\pgfpathclose%
\pgfpathmoveto{\pgfqpoint{0.314019in}{1.349408in}}%
\pgfpathlineto{\pgfqpoint{0.307544in}{1.362335in}}%
\pgfpathlineto{\pgfqpoint{0.307111in}{1.375261in}}%
\pgfpathlineto{\pgfqpoint{0.312011in}{1.388188in}}%
\pgfpathlineto{\pgfqpoint{0.318027in}{1.393271in}}%
\pgfpathlineto{\pgfqpoint{0.331295in}{1.392846in}}%
\pgfpathlineto{\pgfqpoint{0.336443in}{1.388188in}}%
\pgfpathlineto{\pgfqpoint{0.341340in}{1.375261in}}%
\pgfpathlineto{\pgfqpoint{0.340910in}{1.362335in}}%
\pgfpathlineto{\pgfqpoint{0.334522in}{1.349408in}}%
\pgfpathlineto{\pgfqpoint{0.331295in}{1.346828in}}%
\pgfpathlineto{\pgfqpoint{0.318027in}{1.346433in}}%
\pgfpathclose%
\pgfusepath{fill}%
\end{pgfscope}%
\begin{pgfscope}%
\pgfpathrectangle{\pgfqpoint{0.211875in}{0.211875in}}{\pgfqpoint{1.313625in}{1.279725in}}%
\pgfusepath{clip}%
\pgfsetbuttcap%
\pgfsetroundjoin%
\definecolor{currentfill}{rgb}{0.644838,0.098089,0.355336}%
\pgfsetfillcolor{currentfill}%
\pgfsetlinewidth{0.000000pt}%
\definecolor{currentstroke}{rgb}{0.000000,0.000000,0.000000}%
\pgfsetstrokecolor{currentstroke}%
\pgfsetdash{}{0pt}%
\pgfpathmoveto{\pgfqpoint{0.424178in}{1.349300in}}%
\pgfpathlineto{\pgfqpoint{0.437447in}{1.340648in}}%
\pgfpathlineto{\pgfqpoint{0.450716in}{1.341790in}}%
\pgfpathlineto{\pgfqpoint{0.459246in}{1.349408in}}%
\pgfpathlineto{\pgfqpoint{0.463510in}{1.362335in}}%
\pgfpathlineto{\pgfqpoint{0.463830in}{1.375261in}}%
\pgfpathlineto{\pgfqpoint{0.460660in}{1.388188in}}%
\pgfpathlineto{\pgfqpoint{0.450716in}{1.398249in}}%
\pgfpathlineto{\pgfqpoint{0.437447in}{1.399430in}}%
\pgfpathlineto{\pgfqpoint{0.424178in}{1.390571in}}%
\pgfpathlineto{\pgfqpoint{0.423093in}{1.388188in}}%
\pgfpathlineto{\pgfqpoint{0.420921in}{1.375261in}}%
\pgfpathlineto{\pgfqpoint{0.421148in}{1.362335in}}%
\pgfpathlineto{\pgfqpoint{0.424121in}{1.349408in}}%
\pgfpathclose%
\pgfpathmoveto{\pgfqpoint{0.429593in}{1.362335in}}%
\pgfpathlineto{\pgfqpoint{0.429038in}{1.375261in}}%
\pgfpathlineto{\pgfqpoint{0.435832in}{1.388188in}}%
\pgfpathlineto{\pgfqpoint{0.437447in}{1.389392in}}%
\pgfpathlineto{\pgfqpoint{0.446757in}{1.388188in}}%
\pgfpathlineto{\pgfqpoint{0.450716in}{1.387190in}}%
\pgfpathlineto{\pgfqpoint{0.455681in}{1.375261in}}%
\pgfpathlineto{\pgfqpoint{0.455229in}{1.362335in}}%
\pgfpathlineto{\pgfqpoint{0.450716in}{1.353267in}}%
\pgfpathlineto{\pgfqpoint{0.437447in}{1.350489in}}%
\pgfpathclose%
\pgfusepath{fill}%
\end{pgfscope}%
\begin{pgfscope}%
\pgfpathrectangle{\pgfqpoint{0.211875in}{0.211875in}}{\pgfqpoint{1.313625in}{1.279725in}}%
\pgfusepath{clip}%
\pgfsetbuttcap%
\pgfsetroundjoin%
\definecolor{currentfill}{rgb}{0.644838,0.098089,0.355336}%
\pgfsetfillcolor{currentfill}%
\pgfsetlinewidth{0.000000pt}%
\definecolor{currentstroke}{rgb}{0.000000,0.000000,0.000000}%
\pgfsetstrokecolor{currentstroke}%
\pgfsetdash{}{0pt}%
\pgfpathmoveto{\pgfqpoint{0.556867in}{1.343803in}}%
\pgfpathlineto{\pgfqpoint{0.570136in}{1.346095in}}%
\pgfpathlineto{\pgfqpoint{0.573471in}{1.349408in}}%
\pgfpathlineto{\pgfqpoint{0.578390in}{1.362335in}}%
\pgfpathlineto{\pgfqpoint{0.578731in}{1.375261in}}%
\pgfpathlineto{\pgfqpoint{0.575002in}{1.388188in}}%
\pgfpathlineto{\pgfqpoint{0.570136in}{1.393674in}}%
\pgfpathlineto{\pgfqpoint{0.556867in}{1.396049in}}%
\pgfpathlineto{\pgfqpoint{0.544777in}{1.388188in}}%
\pgfpathlineto{\pgfqpoint{0.543598in}{1.385907in}}%
\pgfpathlineto{\pgfqpoint{0.541311in}{1.375261in}}%
\pgfpathlineto{\pgfqpoint{0.541582in}{1.362335in}}%
\pgfpathlineto{\pgfqpoint{0.543598in}{1.354660in}}%
\pgfpathlineto{\pgfqpoint{0.547072in}{1.349408in}}%
\pgfpathclose%
\pgfpathmoveto{\pgfqpoint{0.551657in}{1.362335in}}%
\pgfpathlineto{\pgfqpoint{0.550959in}{1.375261in}}%
\pgfpathlineto{\pgfqpoint{0.556867in}{1.384538in}}%
\pgfpathlineto{\pgfqpoint{0.570136in}{1.377914in}}%
\pgfpathlineto{\pgfqpoint{0.571125in}{1.375261in}}%
\pgfpathlineto{\pgfqpoint{0.570659in}{1.362335in}}%
\pgfpathlineto{\pgfqpoint{0.570136in}{1.361164in}}%
\pgfpathlineto{\pgfqpoint{0.556867in}{1.355479in}}%
\pgfpathclose%
\pgfusepath{fill}%
\end{pgfscope}%
\begin{pgfscope}%
\pgfpathrectangle{\pgfqpoint{0.211875in}{0.211875in}}{\pgfqpoint{1.313625in}{1.279725in}}%
\pgfusepath{clip}%
\pgfsetbuttcap%
\pgfsetroundjoin%
\definecolor{currentfill}{rgb}{0.644838,0.098089,0.355336}%
\pgfsetfillcolor{currentfill}%
\pgfsetlinewidth{0.000000pt}%
\definecolor{currentstroke}{rgb}{0.000000,0.000000,0.000000}%
\pgfsetstrokecolor{currentstroke}%
\pgfsetdash{}{0pt}%
\pgfpathmoveto{\pgfqpoint{0.676288in}{1.346345in}}%
\pgfpathlineto{\pgfqpoint{0.687216in}{1.349408in}}%
\pgfpathlineto{\pgfqpoint{0.689557in}{1.350717in}}%
\pgfpathlineto{\pgfqpoint{0.694221in}{1.362335in}}%
\pgfpathlineto{\pgfqpoint{0.694577in}{1.375261in}}%
\pgfpathlineto{\pgfqpoint{0.690459in}{1.388188in}}%
\pgfpathlineto{\pgfqpoint{0.689557in}{1.389319in}}%
\pgfpathlineto{\pgfqpoint{0.676288in}{1.393330in}}%
\pgfpathlineto{\pgfqpoint{0.667073in}{1.388188in}}%
\pgfpathlineto{\pgfqpoint{0.663019in}{1.382003in}}%
\pgfpathlineto{\pgfqpoint{0.661378in}{1.375261in}}%
\pgfpathlineto{\pgfqpoint{0.661690in}{1.362335in}}%
\pgfpathlineto{\pgfqpoint{0.663019in}{1.357843in}}%
\pgfpathlineto{\pgfqpoint{0.670053in}{1.349408in}}%
\pgfpathclose%
\pgfpathmoveto{\pgfqpoint{0.673819in}{1.362335in}}%
\pgfpathlineto{\pgfqpoint{0.672946in}{1.375261in}}%
\pgfpathlineto{\pgfqpoint{0.676288in}{1.379760in}}%
\pgfpathlineto{\pgfqpoint{0.682331in}{1.375261in}}%
\pgfpathlineto{\pgfqpoint{0.680747in}{1.362335in}}%
\pgfpathlineto{\pgfqpoint{0.676288in}{1.359547in}}%
\pgfpathclose%
\pgfusepath{fill}%
\end{pgfscope}%
\begin{pgfscope}%
\pgfpathrectangle{\pgfqpoint{0.211875in}{0.211875in}}{\pgfqpoint{1.313625in}{1.279725in}}%
\pgfusepath{clip}%
\pgfsetbuttcap%
\pgfsetroundjoin%
\definecolor{currentfill}{rgb}{0.644838,0.098089,0.355336}%
\pgfsetfillcolor{currentfill}%
\pgfsetlinewidth{0.000000pt}%
\definecolor{currentstroke}{rgb}{0.000000,0.000000,0.000000}%
\pgfsetstrokecolor{currentstroke}%
\pgfsetdash{}{0pt}%
\pgfpathmoveto{\pgfqpoint{0.795708in}{1.348345in}}%
\pgfpathlineto{\pgfqpoint{0.798662in}{1.349408in}}%
\pgfpathlineto{\pgfqpoint{0.808977in}{1.357299in}}%
\pgfpathlineto{\pgfqpoint{0.810795in}{1.362335in}}%
\pgfpathlineto{\pgfqpoint{0.811159in}{1.375261in}}%
\pgfpathlineto{\pgfqpoint{0.808977in}{1.382542in}}%
\pgfpathlineto{\pgfqpoint{0.803099in}{1.388188in}}%
\pgfpathlineto{\pgfqpoint{0.795708in}{1.391194in}}%
\pgfpathlineto{\pgfqpoint{0.789280in}{1.388188in}}%
\pgfpathlineto{\pgfqpoint{0.782439in}{1.380090in}}%
\pgfpathlineto{\pgfqpoint{0.781118in}{1.375261in}}%
\pgfpathlineto{\pgfqpoint{0.781468in}{1.362335in}}%
\pgfpathlineto{\pgfqpoint{0.782439in}{1.359401in}}%
\pgfpathlineto{\pgfqpoint{0.793128in}{1.349408in}}%
\pgfpathclose%
\pgfpathmoveto{\pgfqpoint{0.795111in}{1.375261in}}%
\pgfpathlineto{\pgfqpoint{0.795708in}{1.375935in}}%
\pgfpathlineto{\pgfqpoint{0.796407in}{1.375261in}}%
\pgfpathlineto{\pgfqpoint{0.795708in}{1.368177in}}%
\pgfpathclose%
\pgfusepath{fill}%
\end{pgfscope}%
\begin{pgfscope}%
\pgfpathrectangle{\pgfqpoint{0.211875in}{0.211875in}}{\pgfqpoint{1.313625in}{1.279725in}}%
\pgfusepath{clip}%
\pgfsetbuttcap%
\pgfsetroundjoin%
\definecolor{currentfill}{rgb}{0.644838,0.098089,0.355336}%
\pgfsetfillcolor{currentfill}%
\pgfsetlinewidth{0.000000pt}%
\definecolor{currentstroke}{rgb}{0.000000,0.000000,0.000000}%
\pgfsetstrokecolor{currentstroke}%
\pgfsetdash{}{0pt}%
\pgfpathmoveto{\pgfqpoint{1.498962in}{1.349145in}}%
\pgfpathlineto{\pgfqpoint{1.501959in}{1.349408in}}%
\pgfpathlineto{\pgfqpoint{1.512231in}{1.351025in}}%
\pgfpathlineto{\pgfqpoint{1.517961in}{1.362335in}}%
\pgfpathlineto{\pgfqpoint{1.518399in}{1.375261in}}%
\pgfpathlineto{\pgfqpoint{1.513139in}{1.388188in}}%
\pgfpathlineto{\pgfqpoint{1.512231in}{1.389080in}}%
\pgfpathlineto{\pgfqpoint{1.498962in}{1.390358in}}%
\pgfpathlineto{\pgfqpoint{1.496158in}{1.388188in}}%
\pgfpathlineto{\pgfqpoint{1.489771in}{1.375261in}}%
\pgfpathlineto{\pgfqpoint{1.490304in}{1.362335in}}%
\pgfpathlineto{\pgfqpoint{1.498577in}{1.349408in}}%
\pgfpathclose%
\pgfusepath{fill}%
\end{pgfscope}%
\begin{pgfscope}%
\pgfpathrectangle{\pgfqpoint{0.211875in}{0.211875in}}{\pgfqpoint{1.313625in}{1.279725in}}%
\pgfusepath{clip}%
\pgfsetbuttcap%
\pgfsetroundjoin%
\definecolor{currentfill}{rgb}{0.644838,0.098089,0.355336}%
\pgfsetfillcolor{currentfill}%
\pgfsetlinewidth{0.000000pt}%
\definecolor{currentstroke}{rgb}{0.000000,0.000000,0.000000}%
\pgfsetstrokecolor{currentstroke}%
\pgfsetdash{}{0pt}%
\pgfpathmoveto{\pgfqpoint{0.901860in}{1.359726in}}%
\pgfpathlineto{\pgfqpoint{0.915129in}{1.350157in}}%
\pgfpathlineto{\pgfqpoint{0.927408in}{1.362335in}}%
\pgfpathlineto{\pgfqpoint{0.928249in}{1.375261in}}%
\pgfpathlineto{\pgfqpoint{0.917941in}{1.388188in}}%
\pgfpathlineto{\pgfqpoint{0.915129in}{1.389585in}}%
\pgfpathlineto{\pgfqpoint{0.911430in}{1.388188in}}%
\pgfpathlineto{\pgfqpoint{0.901860in}{1.379665in}}%
\pgfpathlineto{\pgfqpoint{0.900511in}{1.375261in}}%
\pgfpathlineto{\pgfqpoint{0.900897in}{1.362335in}}%
\pgfpathclose%
\pgfusepath{fill}%
\end{pgfscope}%
\begin{pgfscope}%
\pgfpathrectangle{\pgfqpoint{0.211875in}{0.211875in}}{\pgfqpoint{1.313625in}{1.279725in}}%
\pgfusepath{clip}%
\pgfsetbuttcap%
\pgfsetroundjoin%
\definecolor{currentfill}{rgb}{0.644838,0.098089,0.355336}%
\pgfsetfillcolor{currentfill}%
\pgfsetlinewidth{0.000000pt}%
\definecolor{currentstroke}{rgb}{0.000000,0.000000,0.000000}%
\pgfsetstrokecolor{currentstroke}%
\pgfsetdash{}{0pt}%
\pgfpathmoveto{\pgfqpoint{1.021280in}{1.359069in}}%
\pgfpathlineto{\pgfqpoint{1.034549in}{1.351912in}}%
\pgfpathlineto{\pgfqpoint{1.043413in}{1.362335in}}%
\pgfpathlineto{\pgfqpoint{1.044147in}{1.375261in}}%
\pgfpathlineto{\pgfqpoint{1.035033in}{1.388188in}}%
\pgfpathlineto{\pgfqpoint{1.034549in}{1.388472in}}%
\pgfpathlineto{\pgfqpoint{1.033559in}{1.388188in}}%
\pgfpathlineto{\pgfqpoint{1.021280in}{1.380410in}}%
\pgfpathlineto{\pgfqpoint{1.019519in}{1.375261in}}%
\pgfpathlineto{\pgfqpoint{1.019938in}{1.362335in}}%
\pgfpathclose%
\pgfusepath{fill}%
\end{pgfscope}%
\begin{pgfscope}%
\pgfpathrectangle{\pgfqpoint{0.211875in}{0.211875in}}{\pgfqpoint{1.313625in}{1.279725in}}%
\pgfusepath{clip}%
\pgfsetbuttcap%
\pgfsetroundjoin%
\definecolor{currentfill}{rgb}{0.644838,0.098089,0.355336}%
\pgfsetfillcolor{currentfill}%
\pgfsetlinewidth{0.000000pt}%
\definecolor{currentstroke}{rgb}{0.000000,0.000000,0.000000}%
\pgfsetstrokecolor{currentstroke}%
\pgfsetdash{}{0pt}%
\pgfpathmoveto{\pgfqpoint{1.140701in}{1.357588in}}%
\pgfpathlineto{\pgfqpoint{1.153970in}{1.352908in}}%
\pgfpathlineto{\pgfqpoint{1.160884in}{1.362335in}}%
\pgfpathlineto{\pgfqpoint{1.161530in}{1.375261in}}%
\pgfpathlineto{\pgfqpoint{1.153970in}{1.387554in}}%
\pgfpathlineto{\pgfqpoint{1.140701in}{1.382129in}}%
\pgfpathlineto{\pgfqpoint{1.138079in}{1.375261in}}%
\pgfpathlineto{\pgfqpoint{1.138529in}{1.362335in}}%
\pgfpathclose%
\pgfusepath{fill}%
\end{pgfscope}%
\begin{pgfscope}%
\pgfpathrectangle{\pgfqpoint{0.211875in}{0.211875in}}{\pgfqpoint{1.313625in}{1.279725in}}%
\pgfusepath{clip}%
\pgfsetbuttcap%
\pgfsetroundjoin%
\definecolor{currentfill}{rgb}{0.644838,0.098089,0.355336}%
\pgfsetfillcolor{currentfill}%
\pgfsetlinewidth{0.000000pt}%
\definecolor{currentstroke}{rgb}{0.000000,0.000000,0.000000}%
\pgfsetstrokecolor{currentstroke}%
\pgfsetdash{}{0pt}%
\pgfpathmoveto{\pgfqpoint{1.260121in}{1.355382in}}%
\pgfpathlineto{\pgfqpoint{1.273390in}{1.353131in}}%
\pgfpathlineto{\pgfqpoint{1.279302in}{1.362335in}}%
\pgfpathlineto{\pgfqpoint{1.279872in}{1.375261in}}%
\pgfpathlineto{\pgfqpoint{1.273390in}{1.387308in}}%
\pgfpathlineto{\pgfqpoint{1.260121in}{1.384701in}}%
\pgfpathlineto{\pgfqpoint{1.256090in}{1.375261in}}%
\pgfpathlineto{\pgfqpoint{1.256569in}{1.362335in}}%
\pgfpathclose%
\pgfusepath{fill}%
\end{pgfscope}%
\begin{pgfscope}%
\pgfpathrectangle{\pgfqpoint{0.211875in}{0.211875in}}{\pgfqpoint{1.313625in}{1.279725in}}%
\pgfusepath{clip}%
\pgfsetbuttcap%
\pgfsetroundjoin%
\definecolor{currentfill}{rgb}{0.644838,0.098089,0.355336}%
\pgfsetfillcolor{currentfill}%
\pgfsetlinewidth{0.000000pt}%
\definecolor{currentstroke}{rgb}{0.000000,0.000000,0.000000}%
\pgfsetstrokecolor{currentstroke}%
\pgfsetdash{}{0pt}%
\pgfpathmoveto{\pgfqpoint{1.379542in}{1.352502in}}%
\pgfpathlineto{\pgfqpoint{1.392811in}{1.352532in}}%
\pgfpathlineto{\pgfqpoint{1.398383in}{1.362335in}}%
\pgfpathlineto{\pgfqpoint{1.398885in}{1.375261in}}%
\pgfpathlineto{\pgfqpoint{1.392811in}{1.388033in}}%
\pgfpathlineto{\pgfqpoint{1.379542in}{1.388063in}}%
\pgfpathlineto{\pgfqpoint{1.373400in}{1.375261in}}%
\pgfpathlineto{\pgfqpoint{1.373907in}{1.362335in}}%
\pgfpathclose%
\pgfusepath{fill}%
\end{pgfscope}%
\begin{pgfscope}%
\pgfpathrectangle{\pgfqpoint{0.211875in}{0.211875in}}{\pgfqpoint{1.313625in}{1.279725in}}%
\pgfusepath{clip}%
\pgfsetbuttcap%
\pgfsetroundjoin%
\definecolor{currentfill}{rgb}{0.644838,0.098089,0.355336}%
\pgfsetfillcolor{currentfill}%
\pgfsetlinewidth{0.000000pt}%
\definecolor{currentstroke}{rgb}{0.000000,0.000000,0.000000}%
\pgfsetstrokecolor{currentstroke}%
\pgfsetdash{}{0pt}%
\pgfpathmoveto{\pgfqpoint{0.251682in}{1.416272in}}%
\pgfpathlineto{\pgfqpoint{0.264951in}{1.415038in}}%
\pgfpathlineto{\pgfqpoint{0.278220in}{1.416956in}}%
\pgfpathlineto{\pgfqpoint{0.288391in}{1.426967in}}%
\pgfpathlineto{\pgfqpoint{0.291061in}{1.439894in}}%
\pgfpathlineto{\pgfqpoint{0.291489in}{1.449629in}}%
\pgfpathlineto{\pgfqpoint{0.291599in}{1.452820in}}%
\pgfpathlineto{\pgfqpoint{0.291489in}{1.455750in}}%
\pgfpathlineto{\pgfqpoint{0.290970in}{1.465747in}}%
\pgfpathlineto{\pgfqpoint{0.287578in}{1.478673in}}%
\pgfpathlineto{\pgfqpoint{0.278220in}{1.485902in}}%
\pgfpathlineto{\pgfqpoint{0.264951in}{1.487533in}}%
\pgfpathlineto{\pgfqpoint{0.251682in}{1.486676in}}%
\pgfpathlineto{\pgfqpoint{0.241532in}{1.478673in}}%
\pgfpathlineto{\pgfqpoint{0.239268in}{1.465747in}}%
\pgfpathlineto{\pgfqpoint{0.238891in}{1.452820in}}%
\pgfpathlineto{\pgfqpoint{0.239313in}{1.439894in}}%
\pgfpathlineto{\pgfqpoint{0.241266in}{1.426967in}}%
\pgfpathclose%
\pgfpathmoveto{\pgfqpoint{0.253522in}{1.426967in}}%
\pgfpathlineto{\pgfqpoint{0.251682in}{1.428102in}}%
\pgfpathlineto{\pgfqpoint{0.246580in}{1.439894in}}%
\pgfpathlineto{\pgfqpoint{0.245515in}{1.452820in}}%
\pgfpathlineto{\pgfqpoint{0.247186in}{1.465747in}}%
\pgfpathlineto{\pgfqpoint{0.251682in}{1.474275in}}%
\pgfpathlineto{\pgfqpoint{0.262119in}{1.478673in}}%
\pgfpathlineto{\pgfqpoint{0.264951in}{1.479326in}}%
\pgfpathlineto{\pgfqpoint{0.267688in}{1.478673in}}%
\pgfpathlineto{\pgfqpoint{0.278220in}{1.474452in}}%
\pgfpathlineto{\pgfqpoint{0.283149in}{1.465747in}}%
\pgfpathlineto{\pgfqpoint{0.285005in}{1.452820in}}%
\pgfpathlineto{\pgfqpoint{0.283845in}{1.439894in}}%
\pgfpathlineto{\pgfqpoint{0.278220in}{1.427805in}}%
\pgfpathlineto{\pgfqpoint{0.276707in}{1.426967in}}%
\pgfpathlineto{\pgfqpoint{0.264951in}{1.423441in}}%
\pgfpathclose%
\pgfusepath{fill}%
\end{pgfscope}%
\begin{pgfscope}%
\pgfpathrectangle{\pgfqpoint{0.211875in}{0.211875in}}{\pgfqpoint{1.313625in}{1.279725in}}%
\pgfusepath{clip}%
\pgfsetbuttcap%
\pgfsetroundjoin%
\definecolor{currentfill}{rgb}{0.644838,0.098089,0.355336}%
\pgfsetfillcolor{currentfill}%
\pgfsetlinewidth{0.000000pt}%
\definecolor{currentstroke}{rgb}{0.000000,0.000000,0.000000}%
\pgfsetstrokecolor{currentstroke}%
\pgfsetdash{}{0pt}%
\pgfpathmoveto{\pgfqpoint{0.371102in}{1.420925in}}%
\pgfpathlineto{\pgfqpoint{0.384371in}{1.418953in}}%
\pgfpathlineto{\pgfqpoint{0.397640in}{1.422635in}}%
\pgfpathlineto{\pgfqpoint{0.401604in}{1.426967in}}%
\pgfpathlineto{\pgfqpoint{0.405700in}{1.439894in}}%
\pgfpathlineto{\pgfqpoint{0.406493in}{1.452820in}}%
\pgfpathlineto{\pgfqpoint{0.405352in}{1.465747in}}%
\pgfpathlineto{\pgfqpoint{0.399633in}{1.478673in}}%
\pgfpathlineto{\pgfqpoint{0.397640in}{1.480383in}}%
\pgfpathlineto{\pgfqpoint{0.384371in}{1.483709in}}%
\pgfpathlineto{\pgfqpoint{0.371102in}{1.482009in}}%
\pgfpathlineto{\pgfqpoint{0.366380in}{1.478673in}}%
\pgfpathlineto{\pgfqpoint{0.360863in}{1.465747in}}%
\pgfpathlineto{\pgfqpoint{0.359831in}{1.452820in}}%
\pgfpathlineto{\pgfqpoint{0.360588in}{1.439894in}}%
\pgfpathlineto{\pgfqpoint{0.364544in}{1.426967in}}%
\pgfpathclose%
\pgfpathmoveto{\pgfqpoint{0.368162in}{1.439894in}}%
\pgfpathlineto{\pgfqpoint{0.366738in}{1.452820in}}%
\pgfpathlineto{\pgfqpoint{0.369113in}{1.465747in}}%
\pgfpathlineto{\pgfqpoint{0.371102in}{1.469118in}}%
\pgfpathlineto{\pgfqpoint{0.384371in}{1.474477in}}%
\pgfpathlineto{\pgfqpoint{0.397640in}{1.466139in}}%
\pgfpathlineto{\pgfqpoint{0.397839in}{1.465747in}}%
\pgfpathlineto{\pgfqpoint{0.400147in}{1.452820in}}%
\pgfpathlineto{\pgfqpoint{0.398771in}{1.439894in}}%
\pgfpathlineto{\pgfqpoint{0.397640in}{1.437190in}}%
\pgfpathlineto{\pgfqpoint{0.384371in}{1.427529in}}%
\pgfpathlineto{\pgfqpoint{0.371102in}{1.433804in}}%
\pgfpathclose%
\pgfusepath{fill}%
\end{pgfscope}%
\begin{pgfscope}%
\pgfpathrectangle{\pgfqpoint{0.211875in}{0.211875in}}{\pgfqpoint{1.313625in}{1.279725in}}%
\pgfusepath{clip}%
\pgfsetbuttcap%
\pgfsetroundjoin%
\definecolor{currentfill}{rgb}{0.644838,0.098089,0.355336}%
\pgfsetfillcolor{currentfill}%
\pgfsetlinewidth{0.000000pt}%
\definecolor{currentstroke}{rgb}{0.000000,0.000000,0.000000}%
\pgfsetstrokecolor{currentstroke}%
\pgfsetdash{}{0pt}%
\pgfpathmoveto{\pgfqpoint{0.490523in}{1.424408in}}%
\pgfpathlineto{\pgfqpoint{0.503792in}{1.422376in}}%
\pgfpathlineto{\pgfqpoint{0.514625in}{1.426967in}}%
\pgfpathlineto{\pgfqpoint{0.517061in}{1.429156in}}%
\pgfpathlineto{\pgfqpoint{0.521082in}{1.439894in}}%
\pgfpathlineto{\pgfqpoint{0.522066in}{1.452820in}}%
\pgfpathlineto{\pgfqpoint{0.520528in}{1.465747in}}%
\pgfpathlineto{\pgfqpoint{0.517061in}{1.473406in}}%
\pgfpathlineto{\pgfqpoint{0.508905in}{1.478673in}}%
\pgfpathlineto{\pgfqpoint{0.503792in}{1.480369in}}%
\pgfpathlineto{\pgfqpoint{0.491401in}{1.478673in}}%
\pgfpathlineto{\pgfqpoint{0.490523in}{1.478482in}}%
\pgfpathlineto{\pgfqpoint{0.482130in}{1.465747in}}%
\pgfpathlineto{\pgfqpoint{0.480472in}{1.452820in}}%
\pgfpathlineto{\pgfqpoint{0.481553in}{1.439894in}}%
\pgfpathlineto{\pgfqpoint{0.487431in}{1.426967in}}%
\pgfpathclose%
\pgfpathmoveto{\pgfqpoint{0.489516in}{1.439894in}}%
\pgfpathlineto{\pgfqpoint{0.487735in}{1.452820in}}%
\pgfpathlineto{\pgfqpoint{0.490523in}{1.464747in}}%
\pgfpathlineto{\pgfqpoint{0.491539in}{1.465747in}}%
\pgfpathlineto{\pgfqpoint{0.503792in}{1.469987in}}%
\pgfpathlineto{\pgfqpoint{0.509440in}{1.465747in}}%
\pgfpathlineto{\pgfqpoint{0.514783in}{1.452820in}}%
\pgfpathlineto{\pgfqpoint{0.511678in}{1.439894in}}%
\pgfpathlineto{\pgfqpoint{0.503792in}{1.432625in}}%
\pgfpathlineto{\pgfqpoint{0.490523in}{1.438023in}}%
\pgfpathclose%
\pgfusepath{fill}%
\end{pgfscope}%
\begin{pgfscope}%
\pgfpathrectangle{\pgfqpoint{0.211875in}{0.211875in}}{\pgfqpoint{1.313625in}{1.279725in}}%
\pgfusepath{clip}%
\pgfsetbuttcap%
\pgfsetroundjoin%
\definecolor{currentfill}{rgb}{0.644838,0.098089,0.355336}%
\pgfsetfillcolor{currentfill}%
\pgfsetlinewidth{0.000000pt}%
\definecolor{currentstroke}{rgb}{0.000000,0.000000,0.000000}%
\pgfsetstrokecolor{currentstroke}%
\pgfsetdash{}{0pt}%
\pgfpathmoveto{\pgfqpoint{0.623212in}{1.425366in}}%
\pgfpathlineto{\pgfqpoint{0.626500in}{1.426967in}}%
\pgfpathlineto{\pgfqpoint{0.636481in}{1.438119in}}%
\pgfpathlineto{\pgfqpoint{0.637072in}{1.439894in}}%
\pgfpathlineto{\pgfqpoint{0.638207in}{1.452820in}}%
\pgfpathlineto{\pgfqpoint{0.636481in}{1.464998in}}%
\pgfpathlineto{\pgfqpoint{0.636254in}{1.465747in}}%
\pgfpathlineto{\pgfqpoint{0.623212in}{1.477063in}}%
\pgfpathlineto{\pgfqpoint{0.609943in}{1.475089in}}%
\pgfpathlineto{\pgfqpoint{0.603068in}{1.465747in}}%
\pgfpathlineto{\pgfqpoint{0.600801in}{1.452820in}}%
\pgfpathlineto{\pgfqpoint{0.602201in}{1.439894in}}%
\pgfpathlineto{\pgfqpoint{0.609943in}{1.426978in}}%
\pgfpathlineto{\pgfqpoint{0.609995in}{1.426967in}}%
\pgfpathclose%
\pgfpathmoveto{\pgfqpoint{0.613648in}{1.439894in}}%
\pgfpathlineto{\pgfqpoint{0.609943in}{1.443932in}}%
\pgfpathlineto{\pgfqpoint{0.608514in}{1.452820in}}%
\pgfpathlineto{\pgfqpoint{0.609943in}{1.458286in}}%
\pgfpathlineto{\pgfqpoint{0.622153in}{1.465747in}}%
\pgfpathlineto{\pgfqpoint{0.623212in}{1.465990in}}%
\pgfpathlineto{\pgfqpoint{0.623492in}{1.465747in}}%
\pgfpathlineto{\pgfqpoint{0.628806in}{1.452820in}}%
\pgfpathlineto{\pgfqpoint{0.625772in}{1.439894in}}%
\pgfpathlineto{\pgfqpoint{0.623212in}{1.437165in}}%
\pgfpathclose%
\pgfusepath{fill}%
\end{pgfscope}%
\begin{pgfscope}%
\pgfpathrectangle{\pgfqpoint{0.211875in}{0.211875in}}{\pgfqpoint{1.313625in}{1.279725in}}%
\pgfusepath{clip}%
\pgfsetbuttcap%
\pgfsetroundjoin%
\definecolor{currentfill}{rgb}{0.644838,0.098089,0.355336}%
\pgfsetfillcolor{currentfill}%
\pgfsetlinewidth{0.000000pt}%
\definecolor{currentstroke}{rgb}{0.000000,0.000000,0.000000}%
\pgfsetstrokecolor{currentstroke}%
\pgfsetdash{}{0pt}%
\pgfpathmoveto{\pgfqpoint{0.729364in}{1.429659in}}%
\pgfpathlineto{\pgfqpoint{0.742633in}{1.428427in}}%
\pgfpathlineto{\pgfqpoint{0.752074in}{1.439894in}}%
\pgfpathlineto{\pgfqpoint{0.754128in}{1.452820in}}%
\pgfpathlineto{\pgfqpoint{0.750693in}{1.465747in}}%
\pgfpathlineto{\pgfqpoint{0.742633in}{1.473716in}}%
\pgfpathlineto{\pgfqpoint{0.729364in}{1.472676in}}%
\pgfpathlineto{\pgfqpoint{0.723657in}{1.465747in}}%
\pgfpathlineto{\pgfqpoint{0.720789in}{1.452820in}}%
\pgfpathlineto{\pgfqpoint{0.722508in}{1.439894in}}%
\pgfpathclose%
\pgfpathmoveto{\pgfqpoint{0.729069in}{1.452820in}}%
\pgfpathlineto{\pgfqpoint{0.729364in}{1.453827in}}%
\pgfpathlineto{\pgfqpoint{0.742633in}{1.457933in}}%
\pgfpathlineto{\pgfqpoint{0.744488in}{1.452820in}}%
\pgfpathlineto{\pgfqpoint{0.742633in}{1.444406in}}%
\pgfpathlineto{\pgfqpoint{0.729364in}{1.451176in}}%
\pgfpathclose%
\pgfusepath{fill}%
\end{pgfscope}%
\begin{pgfscope}%
\pgfpathrectangle{\pgfqpoint{0.211875in}{0.211875in}}{\pgfqpoint{1.313625in}{1.279725in}}%
\pgfusepath{clip}%
\pgfsetbuttcap%
\pgfsetroundjoin%
\definecolor{currentfill}{rgb}{0.644838,0.098089,0.355336}%
\pgfsetfillcolor{currentfill}%
\pgfsetlinewidth{0.000000pt}%
\definecolor{currentstroke}{rgb}{0.000000,0.000000,0.000000}%
\pgfsetstrokecolor{currentstroke}%
\pgfsetdash{}{0pt}%
\pgfpathmoveto{\pgfqpoint{0.848784in}{1.431437in}}%
\pgfpathlineto{\pgfqpoint{0.862053in}{1.431699in}}%
\pgfpathlineto{\pgfqpoint{0.868035in}{1.439894in}}%
\pgfpathlineto{\pgfqpoint{0.870056in}{1.452820in}}%
\pgfpathlineto{\pgfqpoint{0.866623in}{1.465747in}}%
\pgfpathlineto{\pgfqpoint{0.862053in}{1.470846in}}%
\pgfpathlineto{\pgfqpoint{0.848784in}{1.471077in}}%
\pgfpathlineto{\pgfqpoint{0.843850in}{1.465747in}}%
\pgfpathlineto{\pgfqpoint{0.840378in}{1.452820in}}%
\pgfpathlineto{\pgfqpoint{0.842424in}{1.439894in}}%
\pgfpathclose%
\pgfusepath{fill}%
\end{pgfscope}%
\begin{pgfscope}%
\pgfpathrectangle{\pgfqpoint{0.211875in}{0.211875in}}{\pgfqpoint{1.313625in}{1.279725in}}%
\pgfusepath{clip}%
\pgfsetbuttcap%
\pgfsetroundjoin%
\definecolor{currentfill}{rgb}{0.644838,0.098089,0.355336}%
\pgfsetfillcolor{currentfill}%
\pgfsetlinewidth{0.000000pt}%
\definecolor{currentstroke}{rgb}{0.000000,0.000000,0.000000}%
\pgfsetstrokecolor{currentstroke}%
\pgfsetdash{}{0pt}%
\pgfpathmoveto{\pgfqpoint{0.968205in}{1.432436in}}%
\pgfpathlineto{\pgfqpoint{0.981473in}{1.434449in}}%
\pgfpathlineto{\pgfqpoint{0.985021in}{1.439894in}}%
\pgfpathlineto{\pgfqpoint{0.986979in}{1.452820in}}%
\pgfpathlineto{\pgfqpoint{0.983624in}{1.465747in}}%
\pgfpathlineto{\pgfqpoint{0.981473in}{1.468437in}}%
\pgfpathlineto{\pgfqpoint{0.968205in}{1.470176in}}%
\pgfpathlineto{\pgfqpoint{0.963563in}{1.465747in}}%
\pgfpathlineto{\pgfqpoint{0.959471in}{1.452820in}}%
\pgfpathlineto{\pgfqpoint{0.961859in}{1.439894in}}%
\pgfpathclose%
\pgfusepath{fill}%
\end{pgfscope}%
\begin{pgfscope}%
\pgfpathrectangle{\pgfqpoint{0.211875in}{0.211875in}}{\pgfqpoint{1.313625in}{1.279725in}}%
\pgfusepath{clip}%
\pgfsetbuttcap%
\pgfsetroundjoin%
\definecolor{currentfill}{rgb}{0.644838,0.098089,0.355336}%
\pgfsetfillcolor{currentfill}%
\pgfsetlinewidth{0.000000pt}%
\definecolor{currentstroke}{rgb}{0.000000,0.000000,0.000000}%
\pgfsetstrokecolor{currentstroke}%
\pgfsetdash{}{0pt}%
\pgfpathmoveto{\pgfqpoint{1.087625in}{1.432736in}}%
\pgfpathlineto{\pgfqpoint{1.100894in}{1.436672in}}%
\pgfpathlineto{\pgfqpoint{1.102776in}{1.439894in}}%
\pgfpathlineto{\pgfqpoint{1.104645in}{1.452820in}}%
\pgfpathlineto{\pgfqpoint{1.101429in}{1.465747in}}%
\pgfpathlineto{\pgfqpoint{1.100894in}{1.466494in}}%
\pgfpathlineto{\pgfqpoint{1.087625in}{1.469898in}}%
\pgfpathlineto{\pgfqpoint{1.082647in}{1.465747in}}%
\pgfpathlineto{\pgfqpoint{1.077898in}{1.452820in}}%
\pgfpathlineto{\pgfqpoint{1.080657in}{1.439894in}}%
\pgfpathclose%
\pgfusepath{fill}%
\end{pgfscope}%
\begin{pgfscope}%
\pgfpathrectangle{\pgfqpoint{0.211875in}{0.211875in}}{\pgfqpoint{1.313625in}{1.279725in}}%
\pgfusepath{clip}%
\pgfsetbuttcap%
\pgfsetroundjoin%
\definecolor{currentfill}{rgb}{0.644838,0.098089,0.355336}%
\pgfsetfillcolor{currentfill}%
\pgfsetlinewidth{0.000000pt}%
\definecolor{currentstroke}{rgb}{0.000000,0.000000,0.000000}%
\pgfsetstrokecolor{currentstroke}%
\pgfsetdash{}{0pt}%
\pgfpathmoveto{\pgfqpoint{1.207045in}{1.432384in}}%
\pgfpathlineto{\pgfqpoint{1.220314in}{1.438333in}}%
\pgfpathlineto{\pgfqpoint{1.221133in}{1.439894in}}%
\pgfpathlineto{\pgfqpoint{1.222891in}{1.452820in}}%
\pgfpathlineto{\pgfqpoint{1.220314in}{1.464097in}}%
\pgfpathlineto{\pgfqpoint{1.218786in}{1.465747in}}%
\pgfpathlineto{\pgfqpoint{1.207045in}{1.470200in}}%
\pgfpathlineto{\pgfqpoint{1.200839in}{1.465747in}}%
\pgfpathlineto{\pgfqpoint{1.195370in}{1.452820in}}%
\pgfpathlineto{\pgfqpoint{1.198548in}{1.439894in}}%
\pgfpathclose%
\pgfusepath{fill}%
\end{pgfscope}%
\begin{pgfscope}%
\pgfpathrectangle{\pgfqpoint{0.211875in}{0.211875in}}{\pgfqpoint{1.313625in}{1.279725in}}%
\pgfusepath{clip}%
\pgfsetbuttcap%
\pgfsetroundjoin%
\definecolor{currentfill}{rgb}{0.644838,0.098089,0.355336}%
\pgfsetfillcolor{currentfill}%
\pgfsetlinewidth{0.000000pt}%
\definecolor{currentstroke}{rgb}{0.000000,0.000000,0.000000}%
\pgfsetstrokecolor{currentstroke}%
\pgfsetdash{}{0pt}%
\pgfpathmoveto{\pgfqpoint{1.326466in}{1.431400in}}%
\pgfpathlineto{\pgfqpoint{1.339735in}{1.439367in}}%
\pgfpathlineto{\pgfqpoint{1.339983in}{1.439894in}}%
\pgfpathlineto{\pgfqpoint{1.341610in}{1.452820in}}%
\pgfpathlineto{\pgfqpoint{1.339735in}{1.461997in}}%
\pgfpathlineto{\pgfqpoint{1.337241in}{1.465747in}}%
\pgfpathlineto{\pgfqpoint{1.326466in}{1.471064in}}%
\pgfpathlineto{\pgfqpoint{1.317653in}{1.465747in}}%
\pgfpathlineto{\pgfqpoint{1.313197in}{1.457085in}}%
\pgfpathlineto{\pgfqpoint{1.312410in}{1.452820in}}%
\pgfpathlineto{\pgfqpoint{1.313197in}{1.446039in}}%
\pgfpathlineto{\pgfqpoint{1.315028in}{1.439894in}}%
\pgfpathclose%
\pgfusepath{fill}%
\end{pgfscope}%
\begin{pgfscope}%
\pgfpathrectangle{\pgfqpoint{0.211875in}{0.211875in}}{\pgfqpoint{1.313625in}{1.279725in}}%
\pgfusepath{clip}%
\pgfsetbuttcap%
\pgfsetroundjoin%
\definecolor{currentfill}{rgb}{0.644838,0.098089,0.355336}%
\pgfsetfillcolor{currentfill}%
\pgfsetlinewidth{0.000000pt}%
\definecolor{currentstroke}{rgb}{0.000000,0.000000,0.000000}%
\pgfsetstrokecolor{currentstroke}%
\pgfsetdash{}{0pt}%
\pgfpathmoveto{\pgfqpoint{1.432617in}{1.437222in}}%
\pgfpathlineto{\pgfqpoint{1.445886in}{1.429777in}}%
\pgfpathlineto{\pgfqpoint{1.459155in}{1.439660in}}%
\pgfpathlineto{\pgfqpoint{1.459254in}{1.439894in}}%
\pgfpathlineto{\pgfqpoint{1.460730in}{1.452820in}}%
\pgfpathlineto{\pgfqpoint{1.459155in}{1.461466in}}%
\pgfpathlineto{\pgfqpoint{1.457005in}{1.465747in}}%
\pgfpathlineto{\pgfqpoint{1.445886in}{1.472495in}}%
\pgfpathlineto{\pgfqpoint{1.432617in}{1.466064in}}%
\pgfpathlineto{\pgfqpoint{1.432433in}{1.465747in}}%
\pgfpathlineto{\pgfqpoint{1.429789in}{1.452820in}}%
\pgfpathlineto{\pgfqpoint{1.431345in}{1.439894in}}%
\pgfpathclose%
\pgfpathmoveto{\pgfqpoint{1.443516in}{1.452820in}}%
\pgfpathlineto{\pgfqpoint{1.445886in}{1.456048in}}%
\pgfpathlineto{\pgfqpoint{1.447782in}{1.452820in}}%
\pgfpathlineto{\pgfqpoint{1.445886in}{1.447487in}}%
\pgfpathclose%
\pgfusepath{fill}%
\end{pgfscope}%
\begin{pgfscope}%
\pgfpathrectangle{\pgfqpoint{0.211875in}{0.211875in}}{\pgfqpoint{1.313625in}{1.279725in}}%
\pgfusepath{clip}%
\pgfsetbuttcap%
\pgfsetroundjoin%
\definecolor{currentfill}{rgb}{0.796501,0.105066,0.310630}%
\pgfsetfillcolor{currentfill}%
\pgfsetlinewidth{0.000000pt}%
\definecolor{currentstroke}{rgb}{0.000000,0.000000,0.000000}%
\pgfsetstrokecolor{currentstroke}%
\pgfsetdash{}{0pt}%
\pgfpathmoveto{\pgfqpoint{0.251682in}{0.211875in}}%
\pgfpathlineto{\pgfqpoint{0.263907in}{0.211875in}}%
\pgfpathlineto{\pgfqpoint{0.251682in}{0.223811in}}%
\pgfpathlineto{\pgfqpoint{0.251339in}{0.224802in}}%
\pgfpathlineto{\pgfqpoint{0.250613in}{0.237728in}}%
\pgfpathlineto{\pgfqpoint{0.251682in}{0.242357in}}%
\pgfpathlineto{\pgfqpoint{0.257036in}{0.250655in}}%
\pgfpathlineto{\pgfqpoint{0.264951in}{0.255118in}}%
\pgfpathlineto{\pgfqpoint{0.277351in}{0.250655in}}%
\pgfpathlineto{\pgfqpoint{0.278220in}{0.249870in}}%
\pgfpathlineto{\pgfqpoint{0.281648in}{0.237728in}}%
\pgfpathlineto{\pgfqpoint{0.280859in}{0.224802in}}%
\pgfpathlineto{\pgfqpoint{0.278220in}{0.218465in}}%
\pgfpathlineto{\pgfqpoint{0.266615in}{0.211875in}}%
\pgfpathlineto{\pgfqpoint{0.278220in}{0.211875in}}%
\pgfpathlineto{\pgfqpoint{0.284527in}{0.211875in}}%
\pgfpathlineto{\pgfqpoint{0.288160in}{0.224802in}}%
\pgfpathlineto{\pgfqpoint{0.288768in}{0.237728in}}%
\pgfpathlineto{\pgfqpoint{0.287144in}{0.250655in}}%
\pgfpathlineto{\pgfqpoint{0.278220in}{0.263104in}}%
\pgfpathlineto{\pgfqpoint{0.275696in}{0.263581in}}%
\pgfpathlineto{\pgfqpoint{0.264951in}{0.264874in}}%
\pgfpathlineto{\pgfqpoint{0.259446in}{0.263581in}}%
\pgfpathlineto{\pgfqpoint{0.251682in}{0.260457in}}%
\pgfpathlineto{\pgfqpoint{0.246001in}{0.250655in}}%
\pgfpathlineto{\pgfqpoint{0.244235in}{0.237728in}}%
\pgfpathlineto{\pgfqpoint{0.244808in}{0.224802in}}%
\pgfpathlineto{\pgfqpoint{0.248423in}{0.211875in}}%
\pgfpathclose%
\pgfusepath{fill}%
\end{pgfscope}%
\begin{pgfscope}%
\pgfpathrectangle{\pgfqpoint{0.211875in}{0.211875in}}{\pgfqpoint{1.313625in}{1.279725in}}%
\pgfusepath{clip}%
\pgfsetbuttcap%
\pgfsetroundjoin%
\definecolor{currentfill}{rgb}{0.796501,0.105066,0.310630}%
\pgfsetfillcolor{currentfill}%
\pgfsetlinewidth{0.000000pt}%
\definecolor{currentstroke}{rgb}{0.000000,0.000000,0.000000}%
\pgfsetstrokecolor{currentstroke}%
\pgfsetdash{}{0pt}%
\pgfpathmoveto{\pgfqpoint{0.371102in}{0.211875in}}%
\pgfpathlineto{\pgfqpoint{0.371936in}{0.211875in}}%
\pgfpathlineto{\pgfqpoint{0.371102in}{0.212495in}}%
\pgfpathlineto{\pgfqpoint{0.366349in}{0.224802in}}%
\pgfpathlineto{\pgfqpoint{0.365664in}{0.237728in}}%
\pgfpathlineto{\pgfqpoint{0.368481in}{0.250655in}}%
\pgfpathlineto{\pgfqpoint{0.371102in}{0.254660in}}%
\pgfpathlineto{\pgfqpoint{0.384371in}{0.259818in}}%
\pgfpathlineto{\pgfqpoint{0.397640in}{0.254743in}}%
\pgfpathlineto{\pgfqpoint{0.400229in}{0.250655in}}%
\pgfpathlineto{\pgfqpoint{0.402860in}{0.237728in}}%
\pgfpathlineto{\pgfqpoint{0.402188in}{0.224802in}}%
\pgfpathlineto{\pgfqpoint{0.397640in}{0.212573in}}%
\pgfpathlineto{\pgfqpoint{0.396752in}{0.211875in}}%
\pgfpathlineto{\pgfqpoint{0.397640in}{0.211875in}}%
\pgfpathlineto{\pgfqpoint{0.406865in}{0.211875in}}%
\pgfpathlineto{\pgfqpoint{0.409125in}{0.224802in}}%
\pgfpathlineto{\pgfqpoint{0.409625in}{0.237728in}}%
\pgfpathlineto{\pgfqpoint{0.408968in}{0.250655in}}%
\pgfpathlineto{\pgfqpoint{0.404538in}{0.263581in}}%
\pgfpathlineto{\pgfqpoint{0.397640in}{0.267609in}}%
\pgfpathlineto{\pgfqpoint{0.384371in}{0.268692in}}%
\pgfpathlineto{\pgfqpoint{0.371102in}{0.266902in}}%
\pgfpathlineto{\pgfqpoint{0.365624in}{0.263581in}}%
\pgfpathlineto{\pgfqpoint{0.359966in}{0.250655in}}%
\pgfpathlineto{\pgfqpoint{0.358981in}{0.237728in}}%
\pgfpathlineto{\pgfqpoint{0.359505in}{0.224802in}}%
\pgfpathlineto{\pgfqpoint{0.362144in}{0.211875in}}%
\pgfpathclose%
\pgfusepath{fill}%
\end{pgfscope}%
\begin{pgfscope}%
\pgfpathrectangle{\pgfqpoint{0.211875in}{0.211875in}}{\pgfqpoint{1.313625in}{1.279725in}}%
\pgfusepath{clip}%
\pgfsetbuttcap%
\pgfsetroundjoin%
\definecolor{currentfill}{rgb}{0.796501,0.105066,0.310630}%
\pgfsetfillcolor{currentfill}%
\pgfsetlinewidth{0.000000pt}%
\definecolor{currentstroke}{rgb}{0.000000,0.000000,0.000000}%
\pgfsetstrokecolor{currentstroke}%
\pgfsetdash{}{0pt}%
\pgfpathmoveto{\pgfqpoint{0.477254in}{0.211875in}}%
\pgfpathlineto{\pgfqpoint{0.485782in}{0.211875in}}%
\pgfpathlineto{\pgfqpoint{0.481455in}{0.224802in}}%
\pgfpathlineto{\pgfqpoint{0.480800in}{0.237728in}}%
\pgfpathlineto{\pgfqpoint{0.482993in}{0.250655in}}%
\pgfpathlineto{\pgfqpoint{0.490523in}{0.260871in}}%
\pgfpathlineto{\pgfqpoint{0.502826in}{0.263581in}}%
\pgfpathlineto{\pgfqpoint{0.503792in}{0.263724in}}%
\pgfpathlineto{\pgfqpoint{0.504483in}{0.263581in}}%
\pgfpathlineto{\pgfqpoint{0.517061in}{0.258824in}}%
\pgfpathlineto{\pgfqpoint{0.521611in}{0.250655in}}%
\pgfpathlineto{\pgfqpoint{0.523407in}{0.237728in}}%
\pgfpathlineto{\pgfqpoint{0.522825in}{0.224802in}}%
\pgfpathlineto{\pgfqpoint{0.519101in}{0.211875in}}%
\pgfpathlineto{\pgfqpoint{0.528271in}{0.211875in}}%
\pgfpathlineto{\pgfqpoint{0.529475in}{0.224802in}}%
\pgfpathlineto{\pgfqpoint{0.529892in}{0.237728in}}%
\pgfpathlineto{\pgfqpoint{0.529979in}{0.250655in}}%
\pgfpathlineto{\pgfqpoint{0.529519in}{0.263581in}}%
\pgfpathlineto{\pgfqpoint{0.517061in}{0.271899in}}%
\pgfpathlineto{\pgfqpoint{0.503792in}{0.271947in}}%
\pgfpathlineto{\pgfqpoint{0.490523in}{0.271308in}}%
\pgfpathlineto{\pgfqpoint{0.477254in}{0.265648in}}%
\pgfpathlineto{\pgfqpoint{0.476326in}{0.263581in}}%
\pgfpathlineto{\pgfqpoint{0.474576in}{0.250655in}}%
\pgfpathlineto{\pgfqpoint{0.474375in}{0.237728in}}%
\pgfpathlineto{\pgfqpoint{0.474771in}{0.224802in}}%
\pgfpathlineto{\pgfqpoint{0.476191in}{0.211875in}}%
\pgfpathclose%
\pgfusepath{fill}%
\end{pgfscope}%
\begin{pgfscope}%
\pgfpathrectangle{\pgfqpoint{0.211875in}{0.211875in}}{\pgfqpoint{1.313625in}{1.279725in}}%
\pgfusepath{clip}%
\pgfsetbuttcap%
\pgfsetroundjoin%
\definecolor{currentfill}{rgb}{0.796501,0.105066,0.310630}%
\pgfsetfillcolor{currentfill}%
\pgfsetlinewidth{0.000000pt}%
\definecolor{currentstroke}{rgb}{0.000000,0.000000,0.000000}%
\pgfsetstrokecolor{currentstroke}%
\pgfsetdash{}{0pt}%
\pgfpathmoveto{\pgfqpoint{0.596674in}{0.211875in}}%
\pgfpathlineto{\pgfqpoint{0.600279in}{0.211875in}}%
\pgfpathlineto{\pgfqpoint{0.596674in}{0.224432in}}%
\pgfpathlineto{\pgfqpoint{0.596615in}{0.224802in}}%
\pgfpathlineto{\pgfqpoint{0.596126in}{0.237728in}}%
\pgfpathlineto{\pgfqpoint{0.596674in}{0.244168in}}%
\pgfpathlineto{\pgfqpoint{0.597594in}{0.250655in}}%
\pgfpathlineto{\pgfqpoint{0.606606in}{0.263581in}}%
\pgfpathlineto{\pgfqpoint{0.609943in}{0.265166in}}%
\pgfpathlineto{\pgfqpoint{0.623212in}{0.266365in}}%
\pgfpathlineto{\pgfqpoint{0.634297in}{0.263581in}}%
\pgfpathlineto{\pgfqpoint{0.636481in}{0.262488in}}%
\pgfpathlineto{\pgfqpoint{0.642238in}{0.250655in}}%
\pgfpathlineto{\pgfqpoint{0.643397in}{0.237728in}}%
\pgfpathlineto{\pgfqpoint{0.642886in}{0.224802in}}%
\pgfpathlineto{\pgfqpoint{0.640061in}{0.211875in}}%
\pgfpathlineto{\pgfqpoint{0.648909in}{0.211875in}}%
\pgfpathlineto{\pgfqpoint{0.649312in}{0.224802in}}%
\pgfpathlineto{\pgfqpoint{0.649666in}{0.237728in}}%
\pgfpathlineto{\pgfqpoint{0.649750in}{0.239813in}}%
\pgfpathlineto{\pgfqpoint{0.650331in}{0.250655in}}%
\pgfpathlineto{\pgfqpoint{0.652904in}{0.263581in}}%
\pgfpathlineto{\pgfqpoint{0.663019in}{0.269830in}}%
\pgfpathlineto{\pgfqpoint{0.676288in}{0.270592in}}%
\pgfpathlineto{\pgfqpoint{0.689557in}{0.269867in}}%
\pgfpathlineto{\pgfqpoint{0.702619in}{0.263581in}}%
\pgfpathlineto{\pgfqpoint{0.702826in}{0.263248in}}%
\pgfpathlineto{\pgfqpoint{0.705888in}{0.250655in}}%
\pgfpathlineto{\pgfqpoint{0.706694in}{0.237728in}}%
\pgfpathlineto{\pgfqpoint{0.707004in}{0.224802in}}%
\pgfpathlineto{\pgfqpoint{0.707092in}{0.211875in}}%
\pgfpathlineto{\pgfqpoint{0.715209in}{0.211875in}}%
\pgfpathlineto{\pgfqpoint{0.712960in}{0.224802in}}%
\pgfpathlineto{\pgfqpoint{0.712508in}{0.237728in}}%
\pgfpathlineto{\pgfqpoint{0.713323in}{0.250655in}}%
\pgfpathlineto{\pgfqpoint{0.716095in}{0.260201in}}%
\pgfpathlineto{\pgfqpoint{0.718939in}{0.263581in}}%
\pgfpathlineto{\pgfqpoint{0.729364in}{0.267950in}}%
\pgfpathlineto{\pgfqpoint{0.742633in}{0.268474in}}%
\pgfpathlineto{\pgfqpoint{0.755902in}{0.265283in}}%
\pgfpathlineto{\pgfqpoint{0.757807in}{0.263581in}}%
\pgfpathlineto{\pgfqpoint{0.762217in}{0.250655in}}%
\pgfpathlineto{\pgfqpoint{0.762907in}{0.237728in}}%
\pgfpathlineto{\pgfqpoint{0.762447in}{0.224802in}}%
\pgfpathlineto{\pgfqpoint{0.760288in}{0.211875in}}%
\pgfpathlineto{\pgfqpoint{0.768890in}{0.211875in}}%
\pgfpathlineto{\pgfqpoint{0.768703in}{0.224802in}}%
\pgfpathlineto{\pgfqpoint{0.769010in}{0.237728in}}%
\pgfpathlineto{\pgfqpoint{0.769170in}{0.240130in}}%
\pgfpathlineto{\pgfqpoint{0.770156in}{0.250655in}}%
\pgfpathlineto{\pgfqpoint{0.775237in}{0.263581in}}%
\pgfpathlineto{\pgfqpoint{0.782439in}{0.267587in}}%
\pgfpathlineto{\pgfqpoint{0.795708in}{0.268819in}}%
\pgfpathlineto{\pgfqpoint{0.808977in}{0.267499in}}%
\pgfpathlineto{\pgfqpoint{0.816362in}{0.263581in}}%
\pgfpathlineto{\pgfqpoint{0.822191in}{0.250655in}}%
\pgfpathlineto{\pgfqpoint{0.822246in}{0.250242in}}%
\pgfpathlineto{\pgfqpoint{0.823370in}{0.237728in}}%
\pgfpathlineto{\pgfqpoint{0.823653in}{0.224802in}}%
\pgfpathlineto{\pgfqpoint{0.823270in}{0.211875in}}%
\pgfpathlineto{\pgfqpoint{0.831502in}{0.211875in}}%
\pgfpathlineto{\pgfqpoint{0.829686in}{0.224802in}}%
\pgfpathlineto{\pgfqpoint{0.829259in}{0.237728in}}%
\pgfpathlineto{\pgfqpoint{0.829736in}{0.250655in}}%
\pgfpathlineto{\pgfqpoint{0.832903in}{0.263581in}}%
\pgfpathlineto{\pgfqpoint{0.835515in}{0.266453in}}%
\pgfpathlineto{\pgfqpoint{0.848784in}{0.269905in}}%
\pgfpathlineto{\pgfqpoint{0.862053in}{0.270064in}}%
\pgfpathlineto{\pgfqpoint{0.875322in}{0.267396in}}%
\pgfpathlineto{\pgfqpoint{0.878938in}{0.263581in}}%
\pgfpathlineto{\pgfqpoint{0.881621in}{0.250655in}}%
\pgfpathlineto{\pgfqpoint{0.881983in}{0.237728in}}%
\pgfpathlineto{\pgfqpoint{0.881559in}{0.224802in}}%
\pgfpathlineto{\pgfqpoint{0.879865in}{0.211875in}}%
\pgfpathlineto{\pgfqpoint{0.888288in}{0.211875in}}%
\pgfpathlineto{\pgfqpoint{0.887694in}{0.224802in}}%
\pgfpathlineto{\pgfqpoint{0.887968in}{0.237728in}}%
\pgfpathlineto{\pgfqpoint{0.888591in}{0.244506in}}%
\pgfpathlineto{\pgfqpoint{0.889434in}{0.250655in}}%
\pgfpathlineto{\pgfqpoint{0.896529in}{0.263581in}}%
\pgfpathlineto{\pgfqpoint{0.901860in}{0.266259in}}%
\pgfpathlineto{\pgfqpoint{0.915129in}{0.267546in}}%
\pgfpathlineto{\pgfqpoint{0.928398in}{0.265481in}}%
\pgfpathlineto{\pgfqpoint{0.931639in}{0.263581in}}%
\pgfpathlineto{\pgfqpoint{0.938653in}{0.250655in}}%
\pgfpathlineto{\pgfqpoint{0.940228in}{0.237728in}}%
\pgfpathlineto{\pgfqpoint{0.940518in}{0.224802in}}%
\pgfpathlineto{\pgfqpoint{0.939727in}{0.211875in}}%
\pgfpathlineto{\pgfqpoint{0.941667in}{0.211875in}}%
\pgfpathlineto{\pgfqpoint{0.948288in}{0.211875in}}%
\pgfpathlineto{\pgfqpoint{0.946763in}{0.224802in}}%
\pgfpathlineto{\pgfqpoint{0.946349in}{0.237728in}}%
\pgfpathlineto{\pgfqpoint{0.946588in}{0.250655in}}%
\pgfpathlineto{\pgfqpoint{0.948599in}{0.263581in}}%
\pgfpathlineto{\pgfqpoint{0.954936in}{0.269488in}}%
\pgfpathlineto{\pgfqpoint{0.968205in}{0.271149in}}%
\pgfpathlineto{\pgfqpoint{0.981473in}{0.271125in}}%
\pgfpathlineto{\pgfqpoint{0.994742in}{0.268843in}}%
\pgfpathlineto{\pgfqpoint{0.998864in}{0.263581in}}%
\pgfpathlineto{\pgfqpoint{1.000492in}{0.250655in}}%
\pgfpathlineto{\pgfqpoint{1.000653in}{0.237728in}}%
\pgfpathlineto{\pgfqpoint{1.000252in}{0.224802in}}%
\pgfpathlineto{\pgfqpoint{0.998845in}{0.211875in}}%
\pgfpathlineto{\pgfqpoint{1.007150in}{0.211875in}}%
\pgfpathlineto{\pgfqpoint{1.006308in}{0.224802in}}%
\pgfpathlineto{\pgfqpoint{1.006563in}{0.237728in}}%
\pgfpathlineto{\pgfqpoint{1.008011in}{0.250170in}}%
\pgfpathlineto{\pgfqpoint{1.008101in}{0.250655in}}%
\pgfpathlineto{\pgfqpoint{1.016711in}{0.263581in}}%
\pgfpathlineto{\pgfqpoint{1.021280in}{0.265659in}}%
\pgfpathlineto{\pgfqpoint{1.034549in}{0.266748in}}%
\pgfpathlineto{\pgfqpoint{1.047818in}{0.263853in}}%
\pgfpathlineto{\pgfqpoint{1.048236in}{0.263581in}}%
\pgfpathlineto{\pgfqpoint{1.055832in}{0.250655in}}%
\pgfpathlineto{\pgfqpoint{1.057507in}{0.237728in}}%
\pgfpathlineto{\pgfqpoint{1.057775in}{0.224802in}}%
\pgfpathlineto{\pgfqpoint{1.056808in}{0.211875in}}%
\pgfpathlineto{\pgfqpoint{1.061087in}{0.211875in}}%
\pgfpathlineto{\pgfqpoint{1.065555in}{0.211875in}}%
\pgfpathlineto{\pgfqpoint{1.064174in}{0.224802in}}%
\pgfpathlineto{\pgfqpoint{1.063761in}{0.237728in}}%
\pgfpathlineto{\pgfqpoint{1.063867in}{0.250655in}}%
\pgfpathlineto{\pgfqpoint{1.065235in}{0.263581in}}%
\pgfpathlineto{\pgfqpoint{1.074356in}{0.270909in}}%
\pgfpathlineto{\pgfqpoint{1.087625in}{0.271762in}}%
\pgfpathlineto{\pgfqpoint{1.100894in}{0.271621in}}%
\pgfpathlineto{\pgfqpoint{1.114163in}{0.269210in}}%
\pgfpathlineto{\pgfqpoint{1.117671in}{0.263581in}}%
\pgfpathlineto{\pgfqpoint{1.118852in}{0.250655in}}%
\pgfpathlineto{\pgfqpoint{1.118927in}{0.237728in}}%
\pgfpathlineto{\pgfqpoint{1.118536in}{0.224802in}}%
\pgfpathlineto{\pgfqpoint{1.117253in}{0.211875in}}%
\pgfpathlineto{\pgfqpoint{1.125495in}{0.211875in}}%
\pgfpathlineto{\pgfqpoint{1.124555in}{0.224802in}}%
\pgfpathlineto{\pgfqpoint{1.124801in}{0.237728in}}%
\pgfpathlineto{\pgfqpoint{1.126390in}{0.250655in}}%
\pgfpathlineto{\pgfqpoint{1.127432in}{0.254041in}}%
\pgfpathlineto{\pgfqpoint{1.135651in}{0.263581in}}%
\pgfpathlineto{\pgfqpoint{1.140701in}{0.265659in}}%
\pgfpathlineto{\pgfqpoint{1.153970in}{0.266421in}}%
\pgfpathlineto{\pgfqpoint{1.165014in}{0.263581in}}%
\pgfpathlineto{\pgfqpoint{1.167239in}{0.262504in}}%
\pgfpathlineto{\pgfqpoint{1.173630in}{0.250655in}}%
\pgfpathlineto{\pgfqpoint{1.175298in}{0.237728in}}%
\pgfpathlineto{\pgfqpoint{1.175555in}{0.224802in}}%
\pgfpathlineto{\pgfqpoint{1.174569in}{0.211875in}}%
\pgfpathlineto{\pgfqpoint{1.180508in}{0.211875in}}%
\pgfpathlineto{\pgfqpoint{1.183311in}{0.211875in}}%
\pgfpathlineto{\pgfqpoint{1.181918in}{0.224802in}}%
\pgfpathlineto{\pgfqpoint{1.181493in}{0.237728in}}%
\pgfpathlineto{\pgfqpoint{1.181575in}{0.250655in}}%
\pgfpathlineto{\pgfqpoint{1.182857in}{0.263581in}}%
\pgfpathlineto{\pgfqpoint{1.193777in}{0.271222in}}%
\pgfpathlineto{\pgfqpoint{1.207045in}{0.271791in}}%
\pgfpathlineto{\pgfqpoint{1.220314in}{0.271485in}}%
\pgfpathlineto{\pgfqpoint{1.233583in}{0.267474in}}%
\pgfpathlineto{\pgfqpoint{1.235395in}{0.263581in}}%
\pgfpathlineto{\pgfqpoint{1.236700in}{0.250655in}}%
\pgfpathlineto{\pgfqpoint{1.236800in}{0.237728in}}%
\pgfpathlineto{\pgfqpoint{1.236407in}{0.224802in}}%
\pgfpathlineto{\pgfqpoint{1.235091in}{0.211875in}}%
\pgfpathlineto{\pgfqpoint{1.243325in}{0.211875in}}%
\pgfpathlineto{\pgfqpoint{1.242427in}{0.224802in}}%
\pgfpathlineto{\pgfqpoint{1.242676in}{0.237728in}}%
\pgfpathlineto{\pgfqpoint{1.244233in}{0.250655in}}%
\pgfpathlineto{\pgfqpoint{1.246852in}{0.257821in}}%
\pgfpathlineto{\pgfqpoint{1.253133in}{0.263581in}}%
\pgfpathlineto{\pgfqpoint{1.260121in}{0.266179in}}%
\pgfpathlineto{\pgfqpoint{1.273390in}{0.266577in}}%
\pgfpathlineto{\pgfqpoint{1.283693in}{0.263581in}}%
\pgfpathlineto{\pgfqpoint{1.286659in}{0.261812in}}%
\pgfpathlineto{\pgfqpoint{1.291982in}{0.250655in}}%
\pgfpathlineto{\pgfqpoint{1.293540in}{0.237728in}}%
\pgfpathlineto{\pgfqpoint{1.293800in}{0.224802in}}%
\pgfpathlineto{\pgfqpoint{1.292938in}{0.211875in}}%
\pgfpathlineto{\pgfqpoint{1.299928in}{0.211875in}}%
\pgfpathlineto{\pgfqpoint{1.301588in}{0.211875in}}%
\pgfpathlineto{\pgfqpoint{1.300005in}{0.224802in}}%
\pgfpathlineto{\pgfqpoint{1.299928in}{0.226975in}}%
\pgfpathlineto{\pgfqpoint{1.299588in}{0.237728in}}%
\pgfpathlineto{\pgfqpoint{1.299752in}{0.250655in}}%
\pgfpathlineto{\pgfqpoint{1.299928in}{0.253254in}}%
\pgfpathlineto{\pgfqpoint{1.301567in}{0.263581in}}%
\pgfpathlineto{\pgfqpoint{1.313197in}{0.270716in}}%
\pgfpathlineto{\pgfqpoint{1.326466in}{0.271260in}}%
\pgfpathlineto{\pgfqpoint{1.339735in}{0.270608in}}%
\pgfpathlineto{\pgfqpoint{1.351887in}{0.263581in}}%
\pgfpathlineto{\pgfqpoint{1.353004in}{0.259522in}}%
\pgfpathlineto{\pgfqpoint{1.354014in}{0.250655in}}%
\pgfpathlineto{\pgfqpoint{1.354251in}{0.237728in}}%
\pgfpathlineto{\pgfqpoint{1.353843in}{0.224802in}}%
\pgfpathlineto{\pgfqpoint{1.353004in}{0.216615in}}%
\pgfpathlineto{\pgfqpoint{1.352241in}{0.211875in}}%
\pgfpathlineto{\pgfqpoint{1.353004in}{0.211875in}}%
\pgfpathlineto{\pgfqpoint{1.360619in}{0.211875in}}%
\pgfpathlineto{\pgfqpoint{1.359904in}{0.224802in}}%
\pgfpathlineto{\pgfqpoint{1.360166in}{0.237728in}}%
\pgfpathlineto{\pgfqpoint{1.361592in}{0.250655in}}%
\pgfpathlineto{\pgfqpoint{1.366273in}{0.261698in}}%
\pgfpathlineto{\pgfqpoint{1.368822in}{0.263581in}}%
\pgfpathlineto{\pgfqpoint{1.379542in}{0.267168in}}%
\pgfpathlineto{\pgfqpoint{1.392811in}{0.267249in}}%
\pgfpathlineto{\pgfqpoint{1.404065in}{0.263581in}}%
\pgfpathlineto{\pgfqpoint{1.406080in}{0.262096in}}%
\pgfpathlineto{\pgfqpoint{1.410848in}{0.250655in}}%
\pgfpathlineto{\pgfqpoint{1.412195in}{0.237728in}}%
\pgfpathlineto{\pgfqpoint{1.412469in}{0.224802in}}%
\pgfpathlineto{\pgfqpoint{1.411875in}{0.211875in}}%
\pgfpathlineto{\pgfqpoint{1.419348in}{0.211875in}}%
\pgfpathlineto{\pgfqpoint{1.420445in}{0.211875in}}%
\pgfpathlineto{\pgfqpoint{1.419348in}{0.218018in}}%
\pgfpathlineto{\pgfqpoint{1.418595in}{0.224802in}}%
\pgfpathlineto{\pgfqpoint{1.418172in}{0.237728in}}%
\pgfpathlineto{\pgfqpoint{1.418533in}{0.250655in}}%
\pgfpathlineto{\pgfqpoint{1.419348in}{0.256914in}}%
\pgfpathlineto{\pgfqpoint{1.421530in}{0.263581in}}%
\pgfpathlineto{\pgfqpoint{1.432617in}{0.269565in}}%
\pgfpathlineto{\pgfqpoint{1.445886in}{0.270168in}}%
\pgfpathlineto{\pgfqpoint{1.459155in}{0.268817in}}%
\pgfpathlineto{\pgfqpoint{1.467134in}{0.263581in}}%
\pgfpathlineto{\pgfqpoint{1.470619in}{0.250655in}}%
\pgfpathlineto{\pgfqpoint{1.471144in}{0.237728in}}%
\pgfpathlineto{\pgfqpoint{1.470673in}{0.224802in}}%
\pgfpathlineto{\pgfqpoint{1.468673in}{0.211875in}}%
\pgfpathlineto{\pgfqpoint{1.472424in}{0.211875in}}%
\pgfpathlineto{\pgfqpoint{1.477337in}{0.211875in}}%
\pgfpathlineto{\pgfqpoint{1.476948in}{0.224802in}}%
\pgfpathlineto{\pgfqpoint{1.477236in}{0.237728in}}%
\pgfpathlineto{\pgfqpoint{1.478428in}{0.250655in}}%
\pgfpathlineto{\pgfqpoint{1.483570in}{0.263581in}}%
\pgfpathlineto{\pgfqpoint{1.485693in}{0.265328in}}%
\pgfpathlineto{\pgfqpoint{1.498962in}{0.268601in}}%
\pgfpathlineto{\pgfqpoint{1.512231in}{0.268493in}}%
\pgfpathlineto{\pgfqpoint{1.525500in}{0.263782in}}%
\pgfpathlineto{\pgfqpoint{1.525500in}{0.276508in}}%
\pgfpathlineto{\pgfqpoint{1.525500in}{0.281986in}}%
\pgfpathlineto{\pgfqpoint{1.512231in}{0.276944in}}%
\pgfpathlineto{\pgfqpoint{1.499191in}{0.276508in}}%
\pgfpathlineto{\pgfqpoint{1.498962in}{0.276501in}}%
\pgfpathlineto{\pgfqpoint{1.498888in}{0.276508in}}%
\pgfpathlineto{\pgfqpoint{1.485693in}{0.278653in}}%
\pgfpathlineto{\pgfqpoint{1.477874in}{0.289434in}}%
\pgfpathlineto{\pgfqpoint{1.476346in}{0.302361in}}%
\pgfpathlineto{\pgfqpoint{1.475860in}{0.315287in}}%
\pgfpathlineto{\pgfqpoint{1.475732in}{0.328214in}}%
\pgfpathlineto{\pgfqpoint{1.475996in}{0.341140in}}%
\pgfpathlineto{\pgfqpoint{1.485693in}{0.353069in}}%
\pgfpathlineto{\pgfqpoint{1.498962in}{0.353432in}}%
\pgfpathlineto{\pgfqpoint{1.512231in}{0.353159in}}%
\pgfpathlineto{\pgfqpoint{1.525500in}{0.351094in}}%
\pgfpathlineto{\pgfqpoint{1.525500in}{0.354067in}}%
\pgfpathlineto{\pgfqpoint{1.525500in}{0.366993in}}%
\pgfpathlineto{\pgfqpoint{1.525500in}{0.374411in}}%
\pgfpathlineto{\pgfqpoint{1.520962in}{0.366993in}}%
\pgfpathlineto{\pgfqpoint{1.512231in}{0.362412in}}%
\pgfpathlineto{\pgfqpoint{1.498962in}{0.362068in}}%
\pgfpathlineto{\pgfqpoint{1.487308in}{0.366993in}}%
\pgfpathlineto{\pgfqpoint{1.485693in}{0.368591in}}%
\pgfpathlineto{\pgfqpoint{1.481761in}{0.379920in}}%
\pgfpathlineto{\pgfqpoint{1.480886in}{0.392846in}}%
\pgfpathlineto{\pgfqpoint{1.481758in}{0.405773in}}%
\pgfpathlineto{\pgfqpoint{1.485693in}{0.418478in}}%
\pgfpathlineto{\pgfqpoint{1.485863in}{0.418699in}}%
\pgfpathlineto{\pgfqpoint{1.498962in}{0.425731in}}%
\pgfpathlineto{\pgfqpoint{1.512231in}{0.425256in}}%
\pgfpathlineto{\pgfqpoint{1.521969in}{0.418699in}}%
\pgfpathlineto{\pgfqpoint{1.525500in}{0.411430in}}%
\pgfpathlineto{\pgfqpoint{1.525500in}{0.418699in}}%
\pgfpathlineto{\pgfqpoint{1.525500in}{0.431626in}}%
\pgfpathlineto{\pgfqpoint{1.525500in}{0.432983in}}%
\pgfpathlineto{\pgfqpoint{1.512231in}{0.434184in}}%
\pgfpathlineto{\pgfqpoint{1.498962in}{0.434096in}}%
\pgfpathlineto{\pgfqpoint{1.485693in}{0.432786in}}%
\pgfpathlineto{\pgfqpoint{1.482480in}{0.431626in}}%
\pgfpathlineto{\pgfqpoint{1.476175in}{0.418699in}}%
\pgfpathlineto{\pgfqpoint{1.475335in}{0.405773in}}%
\pgfpathlineto{\pgfqpoint{1.474993in}{0.392846in}}%
\pgfpathlineto{\pgfqpoint{1.474745in}{0.379920in}}%
\pgfpathlineto{\pgfqpoint{1.474305in}{0.366993in}}%
\pgfpathlineto{\pgfqpoint{1.472424in}{0.358076in}}%
\pgfpathlineto{\pgfqpoint{1.459155in}{0.354815in}}%
\pgfpathlineto{\pgfqpoint{1.445886in}{0.354543in}}%
\pgfpathlineto{\pgfqpoint{1.432617in}{0.354414in}}%
\pgfpathlineto{\pgfqpoint{1.420213in}{0.354067in}}%
\pgfpathlineto{\pgfqpoint{1.419348in}{0.353982in}}%
\pgfpathlineto{\pgfqpoint{1.414963in}{0.341140in}}%
\pgfpathlineto{\pgfqpoint{1.414792in}{0.328214in}}%
\pgfpathlineto{\pgfqpoint{1.414580in}{0.315287in}}%
\pgfpathlineto{\pgfqpoint{1.414163in}{0.302361in}}%
\pgfpathlineto{\pgfqpoint{1.413001in}{0.289434in}}%
\pgfpathlineto{\pgfqpoint{1.406080in}{0.277784in}}%
\pgfpathlineto{\pgfqpoint{1.401246in}{0.276508in}}%
\pgfpathlineto{\pgfqpoint{1.392811in}{0.275399in}}%
\pgfpathlineto{\pgfqpoint{1.379542in}{0.275263in}}%
\pgfpathlineto{\pgfqpoint{1.368215in}{0.276508in}}%
\pgfpathlineto{\pgfqpoint{1.366273in}{0.276926in}}%
\pgfpathlineto{\pgfqpoint{1.358412in}{0.289434in}}%
\pgfpathlineto{\pgfqpoint{1.357439in}{0.302361in}}%
\pgfpathlineto{\pgfqpoint{1.357054in}{0.315287in}}%
\pgfpathlineto{\pgfqpoint{1.356771in}{0.328214in}}%
\pgfpathlineto{\pgfqpoint{1.356264in}{0.341140in}}%
\pgfpathlineto{\pgfqpoint{1.353004in}{0.351192in}}%
\pgfpathlineto{\pgfqpoint{1.339735in}{0.353379in}}%
\pgfpathlineto{\pgfqpoint{1.326466in}{0.353496in}}%
\pgfpathlineto{\pgfqpoint{1.313197in}{0.353096in}}%
\pgfpathlineto{\pgfqpoint{1.299928in}{0.347679in}}%
\pgfpathlineto{\pgfqpoint{1.298412in}{0.341140in}}%
\pgfpathlineto{\pgfqpoint{1.297657in}{0.328214in}}%
\pgfpathlineto{\pgfqpoint{1.297322in}{0.315287in}}%
\pgfpathlineto{\pgfqpoint{1.296966in}{0.302361in}}%
\pgfpathlineto{\pgfqpoint{1.296173in}{0.289434in}}%
\pgfpathlineto{\pgfqpoint{1.289415in}{0.276508in}}%
\pgfpathlineto{\pgfqpoint{1.286659in}{0.275654in}}%
\pgfpathlineto{\pgfqpoint{1.273390in}{0.274518in}}%
\pgfpathlineto{\pgfqpoint{1.260121in}{0.274532in}}%
\pgfpathlineto{\pgfqpoint{1.246852in}{0.275976in}}%
\pgfpathlineto{\pgfqpoint{1.245464in}{0.276508in}}%
\pgfpathlineto{\pgfqpoint{1.239625in}{0.289434in}}%
\pgfpathlineto{\pgfqpoint{1.238952in}{0.302361in}}%
\pgfpathlineto{\pgfqpoint{1.238621in}{0.315287in}}%
\pgfpathlineto{\pgfqpoint{1.238257in}{0.328214in}}%
\pgfpathlineto{\pgfqpoint{1.237343in}{0.341140in}}%
\pgfpathlineto{\pgfqpoint{1.233583in}{0.349813in}}%
\pgfpathlineto{\pgfqpoint{1.220314in}{0.352740in}}%
\pgfpathlineto{\pgfqpoint{1.207045in}{0.352970in}}%
\pgfpathlineto{\pgfqpoint{1.193777in}{0.352334in}}%
\pgfpathlineto{\pgfqpoint{1.181207in}{0.341140in}}%
\pgfpathlineto{\pgfqpoint{1.180508in}{0.334399in}}%
\pgfpathlineto{\pgfqpoint{1.180149in}{0.328214in}}%
\pgfpathlineto{\pgfqpoint{1.179752in}{0.315287in}}%
\pgfpathlineto{\pgfqpoint{1.179418in}{0.302361in}}%
\pgfpathlineto{\pgfqpoint{1.178777in}{0.289434in}}%
\pgfpathlineto{\pgfqpoint{1.173426in}{0.276508in}}%
\pgfpathlineto{\pgfqpoint{1.167239in}{0.274865in}}%
\pgfpathlineto{\pgfqpoint{1.153970in}{0.274211in}}%
\pgfpathlineto{\pgfqpoint{1.140701in}{0.274344in}}%
\pgfpathlineto{\pgfqpoint{1.127432in}{0.276109in}}%
\pgfpathlineto{\pgfqpoint{1.126583in}{0.276508in}}%
\pgfpathlineto{\pgfqpoint{1.121486in}{0.289434in}}%
\pgfpathlineto{\pgfqpoint{1.120875in}{0.302361in}}%
\pgfpathlineto{\pgfqpoint{1.120556in}{0.315287in}}%
\pgfpathlineto{\pgfqpoint{1.120178in}{0.328214in}}%
\pgfpathlineto{\pgfqpoint{1.119193in}{0.341140in}}%
\pgfpathlineto{\pgfqpoint{1.114163in}{0.350454in}}%
\pgfpathlineto{\pgfqpoint{1.100894in}{0.352763in}}%
\pgfpathlineto{\pgfqpoint{1.087625in}{0.352966in}}%
\pgfpathlineto{\pgfqpoint{1.074356in}{0.352260in}}%
\pgfpathlineto{\pgfqpoint{1.063194in}{0.341140in}}%
\pgfpathlineto{\pgfqpoint{1.062235in}{0.328214in}}%
\pgfpathlineto{\pgfqpoint{1.061853in}{0.315287in}}%
\pgfpathlineto{\pgfqpoint{1.061506in}{0.302361in}}%
\pgfpathlineto{\pgfqpoint{1.061087in}{0.293428in}}%
\pgfpathlineto{\pgfqpoint{1.060792in}{0.289434in}}%
\pgfpathlineto{\pgfqpoint{1.054493in}{0.276508in}}%
\pgfpathlineto{\pgfqpoint{1.047818in}{0.274977in}}%
\pgfpathlineto{\pgfqpoint{1.034549in}{0.274439in}}%
\pgfpathlineto{\pgfqpoint{1.021280in}{0.274762in}}%
\pgfpathlineto{\pgfqpoint{1.011192in}{0.276508in}}%
\pgfpathlineto{\pgfqpoint{1.008011in}{0.278154in}}%
\pgfpathlineto{\pgfqpoint{1.003992in}{0.289434in}}%
\pgfpathlineto{\pgfqpoint{1.003216in}{0.302361in}}%
\pgfpathlineto{\pgfqpoint{1.002869in}{0.315287in}}%
\pgfpathlineto{\pgfqpoint{1.002541in}{0.328214in}}%
\pgfpathlineto{\pgfqpoint{1.001805in}{0.341140in}}%
\pgfpathlineto{\pgfqpoint{0.994742in}{0.352084in}}%
\pgfpathlineto{\pgfqpoint{0.981473in}{0.353358in}}%
\pgfpathlineto{\pgfqpoint{0.968205in}{0.353504in}}%
\pgfpathlineto{\pgfqpoint{0.954936in}{0.353084in}}%
\pgfpathlineto{\pgfqpoint{0.944310in}{0.341140in}}%
\pgfpathlineto{\pgfqpoint{0.943793in}{0.328214in}}%
\pgfpathlineto{\pgfqpoint{0.943507in}{0.315287in}}%
\pgfpathlineto{\pgfqpoint{0.943115in}{0.302361in}}%
\pgfpathlineto{\pgfqpoint{0.942130in}{0.289434in}}%
\pgfpathlineto{\pgfqpoint{0.941667in}{0.286871in}}%
\pgfpathlineto{\pgfqpoint{0.932073in}{0.276508in}}%
\pgfpathlineto{\pgfqpoint{0.928398in}{0.275778in}}%
\pgfpathlineto{\pgfqpoint{0.915129in}{0.275188in}}%
\pgfpathlineto{\pgfqpoint{0.901860in}{0.275885in}}%
\pgfpathlineto{\pgfqpoint{0.898721in}{0.276508in}}%
\pgfpathlineto{\pgfqpoint{0.888591in}{0.284103in}}%
\pgfpathlineto{\pgfqpoint{0.887165in}{0.289434in}}%
\pgfpathlineto{\pgfqpoint{0.886000in}{0.302361in}}%
\pgfpathlineto{\pgfqpoint{0.885582in}{0.315287in}}%
\pgfpathlineto{\pgfqpoint{0.885371in}{0.328214in}}%
\pgfpathlineto{\pgfqpoint{0.885200in}{0.341140in}}%
\pgfpathlineto{\pgfqpoint{0.880121in}{0.354067in}}%
\pgfpathlineto{\pgfqpoint{0.875322in}{0.354296in}}%
\pgfpathlineto{\pgfqpoint{0.862053in}{0.354468in}}%
\pgfpathlineto{\pgfqpoint{0.848784in}{0.354622in}}%
\pgfpathlineto{\pgfqpoint{0.835515in}{0.355158in}}%
\pgfpathlineto{\pgfqpoint{0.826247in}{0.366993in}}%
\pgfpathlineto{\pgfqpoint{0.825817in}{0.379920in}}%
\pgfpathlineto{\pgfqpoint{0.825574in}{0.392846in}}%
\pgfpathlineto{\pgfqpoint{0.825238in}{0.405773in}}%
\pgfpathlineto{\pgfqpoint{0.824411in}{0.418699in}}%
\pgfpathlineto{\pgfqpoint{0.822246in}{0.427388in}}%
\pgfpathlineto{\pgfqpoint{0.817607in}{0.431626in}}%
\pgfpathlineto{\pgfqpoint{0.808977in}{0.433640in}}%
\pgfpathlineto{\pgfqpoint{0.795708in}{0.434218in}}%
\pgfpathlineto{\pgfqpoint{0.782439in}{0.433969in}}%
\pgfpathlineto{\pgfqpoint{0.772020in}{0.431626in}}%
\pgfpathlineto{\pgfqpoint{0.769170in}{0.428033in}}%
\pgfpathlineto{\pgfqpoint{0.767575in}{0.418699in}}%
\pgfpathlineto{\pgfqpoint{0.767032in}{0.405773in}}%
\pgfpathlineto{\pgfqpoint{0.766736in}{0.392846in}}%
\pgfpathlineto{\pgfqpoint{0.766370in}{0.379920in}}%
\pgfpathlineto{\pgfqpoint{0.765353in}{0.366993in}}%
\pgfpathlineto{\pgfqpoint{0.755902in}{0.356901in}}%
\pgfpathlineto{\pgfqpoint{0.742633in}{0.356061in}}%
\pgfpathlineto{\pgfqpoint{0.729364in}{0.356385in}}%
\pgfpathlineto{\pgfqpoint{0.716095in}{0.359096in}}%
\pgfpathlineto{\pgfqpoint{0.710848in}{0.366993in}}%
\pgfpathlineto{\pgfqpoint{0.709321in}{0.379920in}}%
\pgfpathlineto{\pgfqpoint{0.708849in}{0.392846in}}%
\pgfpathlineto{\pgfqpoint{0.708627in}{0.405773in}}%
\pgfpathlineto{\pgfqpoint{0.708474in}{0.418699in}}%
\pgfpathlineto{\pgfqpoint{0.708001in}{0.431626in}}%
\pgfpathlineto{\pgfqpoint{0.702826in}{0.435483in}}%
\pgfpathlineto{\pgfqpoint{0.689557in}{0.435825in}}%
\pgfpathlineto{\pgfqpoint{0.676288in}{0.436017in}}%
\pgfpathlineto{\pgfqpoint{0.663019in}{0.436538in}}%
\pgfpathlineto{\pgfqpoint{0.650481in}{0.444552in}}%
\pgfpathlineto{\pgfqpoint{0.649750in}{0.450310in}}%
\pgfpathlineto{\pgfqpoint{0.649338in}{0.457479in}}%
\pgfpathlineto{\pgfqpoint{0.649002in}{0.470405in}}%
\pgfpathlineto{\pgfqpoint{0.648743in}{0.483332in}}%
\pgfpathlineto{\pgfqpoint{0.648312in}{0.496258in}}%
\pgfpathlineto{\pgfqpoint{0.646546in}{0.509185in}}%
\pgfpathlineto{\pgfqpoint{0.636481in}{0.515342in}}%
\pgfpathlineto{\pgfqpoint{0.623212in}{0.516106in}}%
\pgfpathlineto{\pgfqpoint{0.609943in}{0.516319in}}%
\pgfpathlineto{\pgfqpoint{0.596674in}{0.516415in}}%
\pgfpathlineto{\pgfqpoint{0.589705in}{0.522111in}}%
\pgfpathlineto{\pgfqpoint{0.589636in}{0.535038in}}%
\pgfpathlineto{\pgfqpoint{0.589547in}{0.547964in}}%
\pgfpathlineto{\pgfqpoint{0.589368in}{0.560891in}}%
\pgfpathlineto{\pgfqpoint{0.588950in}{0.573817in}}%
\pgfpathlineto{\pgfqpoint{0.587480in}{0.586744in}}%
\pgfpathlineto{\pgfqpoint{0.583405in}{0.593414in}}%
\pgfpathlineto{\pgfqpoint{0.570136in}{0.596402in}}%
\pgfpathlineto{\pgfqpoint{0.556867in}{0.596860in}}%
\pgfpathlineto{\pgfqpoint{0.543598in}{0.596701in}}%
\pgfpathlineto{\pgfqpoint{0.531730in}{0.586744in}}%
\pgfpathlineto{\pgfqpoint{0.531205in}{0.573817in}}%
\pgfpathlineto{\pgfqpoint{0.530982in}{0.560891in}}%
\pgfpathlineto{\pgfqpoint{0.530726in}{0.547964in}}%
\pgfpathlineto{\pgfqpoint{0.530330in}{0.538133in}}%
\pgfpathlineto{\pgfqpoint{0.530142in}{0.535038in}}%
\pgfpathlineto{\pgfqpoint{0.526243in}{0.522111in}}%
\pgfpathlineto{\pgfqpoint{0.517061in}{0.519066in}}%
\pgfpathlineto{\pgfqpoint{0.503792in}{0.518671in}}%
\pgfpathlineto{\pgfqpoint{0.490523in}{0.519417in}}%
\pgfpathlineto{\pgfqpoint{0.481384in}{0.522111in}}%
\pgfpathlineto{\pgfqpoint{0.477254in}{0.526062in}}%
\pgfpathlineto{\pgfqpoint{0.474654in}{0.535038in}}%
\pgfpathlineto{\pgfqpoint{0.473670in}{0.547964in}}%
\pgfpathlineto{\pgfqpoint{0.473372in}{0.560891in}}%
\pgfpathlineto{\pgfqpoint{0.473401in}{0.573817in}}%
\pgfpathlineto{\pgfqpoint{0.474045in}{0.586744in}}%
\pgfpathlineto{\pgfqpoint{0.477254in}{0.594712in}}%
\pgfpathlineto{\pgfqpoint{0.490523in}{0.597389in}}%
\pgfpathlineto{\pgfqpoint{0.503792in}{0.597823in}}%
\pgfpathlineto{\pgfqpoint{0.517061in}{0.598241in}}%
\pgfpathlineto{\pgfqpoint{0.525990in}{0.599670in}}%
\pgfpathlineto{\pgfqpoint{0.530073in}{0.612597in}}%
\pgfpathlineto{\pgfqpoint{0.530287in}{0.625523in}}%
\pgfpathlineto{\pgfqpoint{0.530209in}{0.638450in}}%
\pgfpathlineto{\pgfqpoint{0.529858in}{0.651377in}}%
\pgfpathlineto{\pgfqpoint{0.528708in}{0.664303in}}%
\pgfpathlineto{\pgfqpoint{0.517061in}{0.676756in}}%
\pgfpathlineto{\pgfqpoint{0.512652in}{0.677230in}}%
\pgfpathlineto{\pgfqpoint{0.503792in}{0.677863in}}%
\pgfpathlineto{\pgfqpoint{0.490523in}{0.677972in}}%
\pgfpathlineto{\pgfqpoint{0.480377in}{0.677230in}}%
\pgfpathlineto{\pgfqpoint{0.477254in}{0.676524in}}%
\pgfpathlineto{\pgfqpoint{0.472646in}{0.664303in}}%
\pgfpathlineto{\pgfqpoint{0.472244in}{0.651377in}}%
\pgfpathlineto{\pgfqpoint{0.472048in}{0.638450in}}%
\pgfpathlineto{\pgfqpoint{0.471821in}{0.625523in}}%
\pgfpathlineto{\pgfqpoint{0.471253in}{0.612597in}}%
\pgfpathlineto{\pgfqpoint{0.463985in}{0.600551in}}%
\pgfpathlineto{\pgfqpoint{0.453598in}{0.599670in}}%
\pgfpathlineto{\pgfqpoint{0.450716in}{0.599558in}}%
\pgfpathlineto{\pgfqpoint{0.445502in}{0.599670in}}%
\pgfpathlineto{\pgfqpoint{0.437447in}{0.599841in}}%
\pgfpathlineto{\pgfqpoint{0.424178in}{0.601880in}}%
\pgfpathlineto{\pgfqpoint{0.416270in}{0.612597in}}%
\pgfpathlineto{\pgfqpoint{0.414944in}{0.625523in}}%
\pgfpathlineto{\pgfqpoint{0.414563in}{0.638450in}}%
\pgfpathlineto{\pgfqpoint{0.414521in}{0.651377in}}%
\pgfpathlineto{\pgfqpoint{0.414902in}{0.664303in}}%
\pgfpathlineto{\pgfqpoint{0.420592in}{0.677230in}}%
\pgfpathlineto{\pgfqpoint{0.424178in}{0.678117in}}%
\pgfpathlineto{\pgfqpoint{0.437447in}{0.679005in}}%
\pgfpathlineto{\pgfqpoint{0.450716in}{0.679384in}}%
\pgfpathlineto{\pgfqpoint{0.463985in}{0.680359in}}%
\pgfpathlineto{\pgfqpoint{0.470715in}{0.690156in}}%
\pgfpathlineto{\pgfqpoint{0.471190in}{0.703083in}}%
\pgfpathlineto{\pgfqpoint{0.471216in}{0.716009in}}%
\pgfpathlineto{\pgfqpoint{0.471011in}{0.728936in}}%
\pgfpathlineto{\pgfqpoint{0.470334in}{0.741862in}}%
\pgfpathlineto{\pgfqpoint{0.466382in}{0.754789in}}%
\pgfpathlineto{\pgfqpoint{0.463985in}{0.756546in}}%
\pgfpathlineto{\pgfqpoint{0.450716in}{0.758983in}}%
\pgfpathlineto{\pgfqpoint{0.437447in}{0.759416in}}%
\pgfpathlineto{\pgfqpoint{0.424178in}{0.759276in}}%
\pgfpathlineto{\pgfqpoint{0.414493in}{0.754789in}}%
\pgfpathlineto{\pgfqpoint{0.413473in}{0.741862in}}%
\pgfpathlineto{\pgfqpoint{0.413272in}{0.728936in}}%
\pgfpathlineto{\pgfqpoint{0.413117in}{0.716009in}}%
\pgfpathlineto{\pgfqpoint{0.412857in}{0.703083in}}%
\pgfpathlineto{\pgfqpoint{0.411954in}{0.690156in}}%
\pgfpathlineto{\pgfqpoint{0.410909in}{0.686272in}}%
\pgfpathlineto{\pgfqpoint{0.397640in}{0.681081in}}%
\pgfpathlineto{\pgfqpoint{0.384371in}{0.680993in}}%
\pgfpathlineto{\pgfqpoint{0.371102in}{0.681944in}}%
\pgfpathlineto{\pgfqpoint{0.358379in}{0.690156in}}%
\pgfpathlineto{\pgfqpoint{0.357833in}{0.691845in}}%
\pgfpathlineto{\pgfqpoint{0.356161in}{0.703083in}}%
\pgfpathlineto{\pgfqpoint{0.355630in}{0.716009in}}%
\pgfpathlineto{\pgfqpoint{0.355548in}{0.728936in}}%
\pgfpathlineto{\pgfqpoint{0.355896in}{0.741862in}}%
\pgfpathlineto{\pgfqpoint{0.357833in}{0.753455in}}%
\pgfpathlineto{\pgfqpoint{0.358554in}{0.754789in}}%
\pgfpathlineto{\pgfqpoint{0.371102in}{0.759423in}}%
\pgfpathlineto{\pgfqpoint{0.384371in}{0.760042in}}%
\pgfpathlineto{\pgfqpoint{0.397640in}{0.760402in}}%
\pgfpathlineto{\pgfqpoint{0.410909in}{0.763358in}}%
\pgfpathlineto{\pgfqpoint{0.412116in}{0.767715in}}%
\pgfpathlineto{\pgfqpoint{0.412654in}{0.780642in}}%
\pgfpathlineto{\pgfqpoint{0.412702in}{0.793568in}}%
\pgfpathlineto{\pgfqpoint{0.412590in}{0.806495in}}%
\pgfpathlineto{\pgfqpoint{0.412225in}{0.819421in}}%
\pgfpathlineto{\pgfqpoint{0.410909in}{0.831459in}}%
\pgfpathlineto{\pgfqpoint{0.410688in}{0.832348in}}%
\pgfpathlineto{\pgfqpoint{0.397640in}{0.840319in}}%
\pgfpathlineto{\pgfqpoint{0.384371in}{0.841030in}}%
\pgfpathlineto{\pgfqpoint{0.371102in}{0.841346in}}%
\pgfpathlineto{\pgfqpoint{0.357833in}{0.842446in}}%
\pgfpathlineto{\pgfqpoint{0.353339in}{0.832348in}}%
\pgfpathlineto{\pgfqpoint{0.353648in}{0.819421in}}%
\pgfpathlineto{\pgfqpoint{0.353647in}{0.806495in}}%
\pgfpathlineto{\pgfqpoint{0.353505in}{0.793568in}}%
\pgfpathlineto{\pgfqpoint{0.353067in}{0.780642in}}%
\pgfpathlineto{\pgfqpoint{0.350688in}{0.767715in}}%
\pgfpathlineto{\pgfqpoint{0.344564in}{0.763588in}}%
\pgfpathlineto{\pgfqpoint{0.331295in}{0.762553in}}%
\pgfpathlineto{\pgfqpoint{0.318027in}{0.762993in}}%
\pgfpathlineto{\pgfqpoint{0.304758in}{0.766130in}}%
\pgfpathlineto{\pgfqpoint{0.302819in}{0.767715in}}%
\pgfpathlineto{\pgfqpoint{0.298298in}{0.780642in}}%
\pgfpathlineto{\pgfqpoint{0.297424in}{0.793568in}}%
\pgfpathlineto{\pgfqpoint{0.297262in}{0.806495in}}%
\pgfpathlineto{\pgfqpoint{0.297628in}{0.819421in}}%
\pgfpathlineto{\pgfqpoint{0.299702in}{0.832348in}}%
\pgfpathlineto{\pgfqpoint{0.304758in}{0.837839in}}%
\pgfpathlineto{\pgfqpoint{0.318027in}{0.840151in}}%
\pgfpathlineto{\pgfqpoint{0.331295in}{0.840599in}}%
\pgfpathlineto{\pgfqpoint{0.344564in}{0.840479in}}%
\pgfpathlineto{\pgfqpoint{0.355238in}{0.845274in}}%
\pgfpathlineto{\pgfqpoint{0.354287in}{0.858201in}}%
\pgfpathlineto{\pgfqpoint{0.354145in}{0.871127in}}%
\pgfpathlineto{\pgfqpoint{0.354057in}{0.884054in}}%
\pgfpathlineto{\pgfqpoint{0.353933in}{0.896980in}}%
\pgfpathlineto{\pgfqpoint{0.353590in}{0.909907in}}%
\pgfpathlineto{\pgfqpoint{0.344564in}{0.922456in}}%
\pgfpathlineto{\pgfqpoint{0.338915in}{0.922833in}}%
\pgfpathlineto{\pgfqpoint{0.331295in}{0.923103in}}%
\pgfpathlineto{\pgfqpoint{0.318027in}{0.923548in}}%
\pgfpathlineto{\pgfqpoint{0.304758in}{0.925012in}}%
\pgfpathlineto{\pgfqpoint{0.297195in}{0.935760in}}%
\pgfpathlineto{\pgfqpoint{0.296370in}{0.948686in}}%
\pgfpathlineto{\pgfqpoint{0.296162in}{0.961613in}}%
\pgfpathlineto{\pgfqpoint{0.296167in}{0.974539in}}%
\pgfpathlineto{\pgfqpoint{0.296449in}{0.987466in}}%
\pgfpathlineto{\pgfqpoint{0.298950in}{1.000392in}}%
\pgfpathlineto{\pgfqpoint{0.304758in}{1.003156in}}%
\pgfpathlineto{\pgfqpoint{0.318027in}{1.004099in}}%
\pgfpathlineto{\pgfqpoint{0.331295in}{1.004536in}}%
\pgfpathlineto{\pgfqpoint{0.344564in}{1.005684in}}%
\pgfpathlineto{\pgfqpoint{0.352043in}{1.013319in}}%
\pgfpathlineto{\pgfqpoint{0.353110in}{1.026245in}}%
\pgfpathlineto{\pgfqpoint{0.353313in}{1.039172in}}%
\pgfpathlineto{\pgfqpoint{0.353267in}{1.052098in}}%
\pgfpathlineto{\pgfqpoint{0.352917in}{1.065025in}}%
\pgfpathlineto{\pgfqpoint{0.351087in}{1.077952in}}%
\pgfpathlineto{\pgfqpoint{0.344564in}{1.083332in}}%
\pgfpathlineto{\pgfqpoint{0.331295in}{1.084703in}}%
\pgfpathlineto{\pgfqpoint{0.318027in}{1.085134in}}%
\pgfpathlineto{\pgfqpoint{0.304758in}{1.085801in}}%
\pgfpathlineto{\pgfqpoint{0.297036in}{1.090878in}}%
\pgfpathlineto{\pgfqpoint{0.295968in}{1.103805in}}%
\pgfpathlineto{\pgfqpoint{0.295802in}{1.116731in}}%
\pgfpathlineto{\pgfqpoint{0.295781in}{1.129658in}}%
\pgfpathlineto{\pgfqpoint{0.295868in}{1.142584in}}%
\pgfpathlineto{\pgfqpoint{0.296278in}{1.155511in}}%
\pgfpathlineto{\pgfqpoint{0.304758in}{1.166054in}}%
\pgfpathlineto{\pgfqpoint{0.318027in}{1.166731in}}%
\pgfpathlineto{\pgfqpoint{0.331295in}{1.167168in}}%
\pgfpathlineto{\pgfqpoint{0.343913in}{1.168437in}}%
\pgfpathlineto{\pgfqpoint{0.344564in}{1.168566in}}%
\pgfpathlineto{\pgfqpoint{0.352528in}{1.181364in}}%
\pgfpathlineto{\pgfqpoint{0.353171in}{1.194290in}}%
\pgfpathlineto{\pgfqpoint{0.353319in}{1.207217in}}%
\pgfpathlineto{\pgfqpoint{0.353238in}{1.220143in}}%
\pgfpathlineto{\pgfqpoint{0.352740in}{1.233070in}}%
\pgfpathlineto{\pgfqpoint{0.345837in}{1.245996in}}%
\pgfpathlineto{\pgfqpoint{0.344564in}{1.246322in}}%
\pgfpathlineto{\pgfqpoint{0.331295in}{1.247483in}}%
\pgfpathlineto{\pgfqpoint{0.318027in}{1.247925in}}%
\pgfpathlineto{\pgfqpoint{0.304758in}{1.248879in}}%
\pgfpathlineto{\pgfqpoint{0.296827in}{1.258923in}}%
\pgfpathlineto{\pgfqpoint{0.296236in}{1.271849in}}%
\pgfpathlineto{\pgfqpoint{0.296141in}{1.284776in}}%
\pgfpathlineto{\pgfqpoint{0.296246in}{1.297702in}}%
\pgfpathlineto{\pgfqpoint{0.296689in}{1.310629in}}%
\pgfpathlineto{\pgfqpoint{0.299399in}{1.323555in}}%
\pgfpathlineto{\pgfqpoint{0.304758in}{1.327001in}}%
\pgfpathlineto{\pgfqpoint{0.318027in}{1.328422in}}%
\pgfpathlineto{\pgfqpoint{0.331295in}{1.328854in}}%
\pgfpathlineto{\pgfqpoint{0.344564in}{1.329480in}}%
\pgfpathlineto{\pgfqpoint{0.353100in}{1.336482in}}%
\pgfpathlineto{\pgfqpoint{0.353838in}{1.349408in}}%
\pgfpathlineto{\pgfqpoint{0.354013in}{1.362335in}}%
\pgfpathlineto{\pgfqpoint{0.354107in}{1.375261in}}%
\pgfpathlineto{\pgfqpoint{0.354210in}{1.388188in}}%
\pgfpathlineto{\pgfqpoint{0.354519in}{1.401114in}}%
\pgfpathlineto{\pgfqpoint{0.357833in}{1.409392in}}%
\pgfpathlineto{\pgfqpoint{0.371102in}{1.410489in}}%
\pgfpathlineto{\pgfqpoint{0.384371in}{1.410805in}}%
\pgfpathlineto{\pgfqpoint{0.397640in}{1.411516in}}%
\pgfpathlineto{\pgfqpoint{0.406841in}{1.414041in}}%
\pgfpathlineto{\pgfqpoint{0.410909in}{1.420669in}}%
\pgfpathlineto{\pgfqpoint{0.411866in}{1.426967in}}%
\pgfpathlineto{\pgfqpoint{0.412481in}{1.439894in}}%
\pgfpathlineto{\pgfqpoint{0.412674in}{1.452820in}}%
\pgfpathlineto{\pgfqpoint{0.412698in}{1.465747in}}%
\pgfpathlineto{\pgfqpoint{0.412489in}{1.478673in}}%
\pgfpathlineto{\pgfqpoint{0.410909in}{1.488483in}}%
\pgfpathlineto{\pgfqpoint{0.397640in}{1.491539in}}%
\pgfpathlineto{\pgfqpoint{0.395959in}{1.491600in}}%
\pgfpathlineto{\pgfqpoint{0.384371in}{1.491600in}}%
\pgfpathlineto{\pgfqpoint{0.371102in}{1.491600in}}%
\pgfpathlineto{\pgfqpoint{0.357833in}{1.491600in}}%
\pgfpathlineto{\pgfqpoint{0.344564in}{1.491600in}}%
\pgfpathlineto{\pgfqpoint{0.331295in}{1.491600in}}%
\pgfpathlineto{\pgfqpoint{0.318027in}{1.491600in}}%
\pgfpathlineto{\pgfqpoint{0.304758in}{1.491600in}}%
\pgfpathlineto{\pgfqpoint{0.291489in}{1.491600in}}%
\pgfpathlineto{\pgfqpoint{0.278220in}{1.491600in}}%
\pgfpathlineto{\pgfqpoint{0.264951in}{1.491600in}}%
\pgfpathlineto{\pgfqpoint{0.251682in}{1.491600in}}%
\pgfpathlineto{\pgfqpoint{0.238413in}{1.491600in}}%
\pgfpathlineto{\pgfqpoint{0.225144in}{1.491600in}}%
\pgfpathlineto{\pgfqpoint{0.211875in}{1.491600in}}%
\pgfpathlineto{\pgfqpoint{0.211875in}{1.485282in}}%
\pgfpathlineto{\pgfqpoint{0.225144in}{1.481799in}}%
\pgfpathlineto{\pgfqpoint{0.228091in}{1.478673in}}%
\pgfpathlineto{\pgfqpoint{0.231912in}{1.465747in}}%
\pgfpathlineto{\pgfqpoint{0.232753in}{1.452820in}}%
\pgfpathlineto{\pgfqpoint{0.232551in}{1.439894in}}%
\pgfpathlineto{\pgfqpoint{0.230764in}{1.426967in}}%
\pgfpathlineto{\pgfqpoint{0.225144in}{1.418526in}}%
\pgfpathlineto{\pgfqpoint{0.211875in}{1.415409in}}%
\pgfpathlineto{\pgfqpoint{0.211875in}{1.414041in}}%
\pgfpathlineto{\pgfqpoint{0.211875in}{1.407047in}}%
\pgfpathlineto{\pgfqpoint{0.225144in}{1.403917in}}%
\pgfpathlineto{\pgfqpoint{0.228275in}{1.401114in}}%
\pgfpathlineto{\pgfqpoint{0.232323in}{1.388188in}}%
\pgfpathlineto{\pgfqpoint{0.233103in}{1.375261in}}%
\pgfpathlineto{\pgfqpoint{0.232958in}{1.362335in}}%
\pgfpathlineto{\pgfqpoint{0.231707in}{1.349408in}}%
\pgfpathlineto{\pgfqpoint{0.225196in}{1.336482in}}%
\pgfpathlineto{\pgfqpoint{0.225144in}{1.336445in}}%
\pgfpathlineto{\pgfqpoint{0.211875in}{1.332917in}}%
\pgfpathlineto{\pgfqpoint{0.211875in}{1.324855in}}%
\pgfpathlineto{\pgfqpoint{0.221486in}{1.323555in}}%
\pgfpathlineto{\pgfqpoint{0.225144in}{1.322615in}}%
\pgfpathlineto{\pgfqpoint{0.232538in}{1.310629in}}%
\pgfpathlineto{\pgfqpoint{0.233773in}{1.297702in}}%
\pgfpathlineto{\pgfqpoint{0.233978in}{1.284776in}}%
\pgfpathlineto{\pgfqpoint{0.233497in}{1.271849in}}%
\pgfpathlineto{\pgfqpoint{0.230909in}{1.258923in}}%
\pgfpathlineto{\pgfqpoint{0.225144in}{1.253489in}}%
\pgfpathlineto{\pgfqpoint{0.211875in}{1.251613in}}%
\pgfpathlineto{\pgfqpoint{0.211875in}{1.245996in}}%
\pgfpathlineto{\pgfqpoint{0.211875in}{1.243218in}}%
\pgfpathlineto{\pgfqpoint{0.225144in}{1.238974in}}%
\pgfpathlineto{\pgfqpoint{0.229331in}{1.233070in}}%
\pgfpathlineto{\pgfqpoint{0.231883in}{1.220143in}}%
\pgfpathlineto{\pgfqpoint{0.232376in}{1.207217in}}%
\pgfpathlineto{\pgfqpoint{0.231819in}{1.194290in}}%
\pgfpathlineto{\pgfqpoint{0.229135in}{1.181364in}}%
\pgfpathlineto{\pgfqpoint{0.225144in}{1.175780in}}%
\pgfpathlineto{\pgfqpoint{0.211875in}{1.171406in}}%
\pgfpathlineto{\pgfqpoint{0.211875in}{1.168437in}}%
\pgfpathlineto{\pgfqpoint{0.211875in}{1.163060in}}%
\pgfpathlineto{\pgfqpoint{0.225144in}{1.161418in}}%
\pgfpathlineto{\pgfqpoint{0.231519in}{1.155511in}}%
\pgfpathlineto{\pgfqpoint{0.233893in}{1.142584in}}%
\pgfpathlineto{\pgfqpoint{0.234363in}{1.129658in}}%
\pgfpathlineto{\pgfqpoint{0.234252in}{1.116731in}}%
\pgfpathlineto{\pgfqpoint{0.233337in}{1.103805in}}%
\pgfpathlineto{\pgfqpoint{0.226319in}{1.090878in}}%
\pgfpathlineto{\pgfqpoint{0.225144in}{1.090365in}}%
\pgfpathlineto{\pgfqpoint{0.211875in}{1.088748in}}%
\pgfpathlineto{\pgfqpoint{0.211875in}{1.080634in}}%
\pgfpathlineto{\pgfqpoint{0.221922in}{1.077952in}}%
\pgfpathlineto{\pgfqpoint{0.225144in}{1.076243in}}%
\pgfpathlineto{\pgfqpoint{0.230767in}{1.065025in}}%
\pgfpathlineto{\pgfqpoint{0.232200in}{1.052098in}}%
\pgfpathlineto{\pgfqpoint{0.232300in}{1.039172in}}%
\pgfpathlineto{\pgfqpoint{0.231217in}{1.026245in}}%
\pgfpathlineto{\pgfqpoint{0.225828in}{1.013319in}}%
\pgfpathlineto{\pgfqpoint{0.225144in}{1.012710in}}%
\pgfpathlineto{\pgfqpoint{0.211875in}{1.008659in}}%
\pgfpathlineto{\pgfqpoint{0.211875in}{1.000452in}}%
\pgfpathlineto{\pgfqpoint{0.212587in}{1.000392in}}%
\pgfpathlineto{\pgfqpoint{0.225144in}{0.998457in}}%
\pgfpathlineto{\pgfqpoint{0.232582in}{0.987466in}}%
\pgfpathlineto{\pgfqpoint{0.233810in}{0.974539in}}%
\pgfpathlineto{\pgfqpoint{0.233967in}{0.961613in}}%
\pgfpathlineto{\pgfqpoint{0.233443in}{0.948686in}}%
\pgfpathlineto{\pgfqpoint{0.231015in}{0.935760in}}%
\pgfpathlineto{\pgfqpoint{0.225144in}{0.929432in}}%
\pgfpathlineto{\pgfqpoint{0.211875in}{0.927224in}}%
\pgfpathlineto{\pgfqpoint{0.211875in}{0.922833in}}%
\pgfpathlineto{\pgfqpoint{0.211875in}{0.918912in}}%
\pgfpathlineto{\pgfqpoint{0.225144in}{0.915275in}}%
\pgfpathlineto{\pgfqpoint{0.229569in}{0.909907in}}%
\pgfpathlineto{\pgfqpoint{0.232452in}{0.896980in}}%
\pgfpathlineto{\pgfqpoint{0.233116in}{0.884054in}}%
\pgfpathlineto{\pgfqpoint{0.232916in}{0.871127in}}%
\pgfpathlineto{\pgfqpoint{0.231397in}{0.858201in}}%
\pgfpathlineto{\pgfqpoint{0.225144in}{0.848120in}}%
\pgfpathlineto{\pgfqpoint{0.215068in}{0.845274in}}%
\pgfpathlineto{\pgfqpoint{0.211875in}{0.844789in}}%
\pgfpathlineto{\pgfqpoint{0.211875in}{0.836519in}}%
\pgfpathlineto{\pgfqpoint{0.225144in}{0.833633in}}%
\pgfpathlineto{\pgfqpoint{0.226772in}{0.832348in}}%
\pgfpathlineto{\pgfqpoint{0.231868in}{0.819421in}}%
\pgfpathlineto{\pgfqpoint{0.232757in}{0.806495in}}%
\pgfpathlineto{\pgfqpoint{0.232548in}{0.793568in}}%
\pgfpathlineto{\pgfqpoint{0.230962in}{0.780642in}}%
\pgfpathlineto{\pgfqpoint{0.225144in}{0.770195in}}%
\pgfpathlineto{\pgfqpoint{0.218434in}{0.767715in}}%
\pgfpathlineto{\pgfqpoint{0.211875in}{0.766478in}}%
\pgfpathlineto{\pgfqpoint{0.211875in}{0.758470in}}%
\pgfpathlineto{\pgfqpoint{0.225144in}{0.756282in}}%
\pgfpathlineto{\pgfqpoint{0.227823in}{0.754789in}}%
\pgfpathlineto{\pgfqpoint{0.233478in}{0.741862in}}%
\pgfpathlineto{\pgfqpoint{0.234442in}{0.728936in}}%
\pgfpathlineto{\pgfqpoint{0.234684in}{0.716009in}}%
\pgfpathlineto{\pgfqpoint{0.234508in}{0.703083in}}%
\pgfpathlineto{\pgfqpoint{0.233322in}{0.690156in}}%
\pgfpathlineto{\pgfqpoint{0.225144in}{0.681387in}}%
\pgfpathlineto{\pgfqpoint{0.211875in}{0.680003in}}%
\pgfpathlineto{\pgfqpoint{0.211875in}{0.677230in}}%
\pgfpathlineto{\pgfqpoint{0.211875in}{0.671371in}}%
\pgfpathlineto{\pgfqpoint{0.225144in}{0.666360in}}%
\pgfpathlineto{\pgfqpoint{0.226685in}{0.664303in}}%
\pgfpathlineto{\pgfqpoint{0.230229in}{0.651377in}}%
\pgfpathlineto{\pgfqpoint{0.230779in}{0.638450in}}%
\pgfpathlineto{\pgfqpoint{0.229745in}{0.625523in}}%
\pgfpathlineto{\pgfqpoint{0.225144in}{0.612816in}}%
\pgfpathlineto{\pgfqpoint{0.224915in}{0.612597in}}%
\pgfpathlineto{\pgfqpoint{0.211875in}{0.607197in}}%
\pgfpathlineto{\pgfqpoint{0.211875in}{0.599670in}}%
\pgfpathlineto{\pgfqpoint{0.211875in}{0.598868in}}%
\pgfpathlineto{\pgfqpoint{0.225144in}{0.598691in}}%
\pgfpathlineto{\pgfqpoint{0.236201in}{0.599670in}}%
\pgfpathlineto{\pgfqpoint{0.236653in}{0.612597in}}%
\pgfpathlineto{\pgfqpoint{0.236727in}{0.625523in}}%
\pgfpathlineto{\pgfqpoint{0.236861in}{0.638450in}}%
\pgfpathlineto{\pgfqpoint{0.237159in}{0.651377in}}%
\pgfpathlineto{\pgfqpoint{0.238076in}{0.664303in}}%
\pgfpathlineto{\pgfqpoint{0.238413in}{0.666222in}}%
\pgfpathlineto{\pgfqpoint{0.251682in}{0.676495in}}%
\pgfpathlineto{\pgfqpoint{0.264951in}{0.676802in}}%
\pgfpathlineto{\pgfqpoint{0.278220in}{0.675578in}}%
\pgfpathlineto{\pgfqpoint{0.291241in}{0.664303in}}%
\pgfpathlineto{\pgfqpoint{0.291489in}{0.663248in}}%
\pgfpathlineto{\pgfqpoint{0.293005in}{0.651377in}}%
\pgfpathlineto{\pgfqpoint{0.293463in}{0.638450in}}%
\pgfpathlineto{\pgfqpoint{0.293395in}{0.625523in}}%
\pgfpathlineto{\pgfqpoint{0.292501in}{0.612597in}}%
\pgfpathlineto{\pgfqpoint{0.291489in}{0.608188in}}%
\pgfpathlineto{\pgfqpoint{0.278220in}{0.599753in}}%
\pgfpathlineto{\pgfqpoint{0.277190in}{0.599670in}}%
\pgfpathlineto{\pgfqpoint{0.264951in}{0.598920in}}%
\pgfpathlineto{\pgfqpoint{0.251682in}{0.598654in}}%
\pgfpathlineto{\pgfqpoint{0.238413in}{0.598361in}}%
\pgfpathlineto{\pgfqpoint{0.236728in}{0.586744in}}%
\pgfpathlineto{\pgfqpoint{0.236767in}{0.573817in}}%
\pgfpathlineto{\pgfqpoint{0.236885in}{0.560891in}}%
\pgfpathlineto{\pgfqpoint{0.237149in}{0.547964in}}%
\pgfpathlineto{\pgfqpoint{0.237853in}{0.535038in}}%
\pgfpathlineto{\pgfqpoint{0.238413in}{0.530905in}}%
\pgfpathlineto{\pgfqpoint{0.242166in}{0.522111in}}%
\pgfpathlineto{\pgfqpoint{0.251682in}{0.518526in}}%
\pgfpathlineto{\pgfqpoint{0.264951in}{0.517630in}}%
\pgfpathlineto{\pgfqpoint{0.278220in}{0.517320in}}%
\pgfpathlineto{\pgfqpoint{0.291489in}{0.516886in}}%
\pgfpathlineto{\pgfqpoint{0.295623in}{0.509185in}}%
\pgfpathlineto{\pgfqpoint{0.295826in}{0.496258in}}%
\pgfpathlineto{\pgfqpoint{0.295966in}{0.483332in}}%
\pgfpathlineto{\pgfqpoint{0.296202in}{0.470405in}}%
\pgfpathlineto{\pgfqpoint{0.296750in}{0.457479in}}%
\pgfpathlineto{\pgfqpoint{0.298935in}{0.444552in}}%
\pgfpathlineto{\pgfqpoint{0.304758in}{0.438904in}}%
\pgfpathlineto{\pgfqpoint{0.318027in}{0.436919in}}%
\pgfpathlineto{\pgfqpoint{0.331295in}{0.436490in}}%
\pgfpathlineto{\pgfqpoint{0.344564in}{0.436474in}}%
\pgfpathlineto{\pgfqpoint{0.355346in}{0.431626in}}%
\pgfpathlineto{\pgfqpoint{0.354920in}{0.418699in}}%
\pgfpathlineto{\pgfqpoint{0.354963in}{0.405773in}}%
\pgfpathlineto{\pgfqpoint{0.355167in}{0.392846in}}%
\pgfpathlineto{\pgfqpoint{0.355677in}{0.379920in}}%
\pgfpathlineto{\pgfqpoint{0.357363in}{0.366993in}}%
\pgfpathlineto{\pgfqpoint{0.357833in}{0.365531in}}%
\pgfpathlineto{\pgfqpoint{0.371102in}{0.356874in}}%
\pgfpathlineto{\pgfqpoint{0.384371in}{0.355959in}}%
\pgfpathlineto{\pgfqpoint{0.397640in}{0.356005in}}%
\pgfpathlineto{\pgfqpoint{0.410909in}{0.360149in}}%
\pgfpathlineto{\pgfqpoint{0.412630in}{0.366993in}}%
\pgfpathlineto{\pgfqpoint{0.413301in}{0.379920in}}%
\pgfpathlineto{\pgfqpoint{0.413596in}{0.392846in}}%
\pgfpathlineto{\pgfqpoint{0.413929in}{0.405773in}}%
\pgfpathlineto{\pgfqpoint{0.414699in}{0.418699in}}%
\pgfpathlineto{\pgfqpoint{0.421436in}{0.431626in}}%
\pgfpathlineto{\pgfqpoint{0.424178in}{0.432456in}}%
\pgfpathlineto{\pgfqpoint{0.437447in}{0.433405in}}%
\pgfpathlineto{\pgfqpoint{0.450716in}{0.432914in}}%
\pgfpathlineto{\pgfqpoint{0.457740in}{0.431626in}}%
\pgfpathlineto{\pgfqpoint{0.463985in}{0.429064in}}%
\pgfpathlineto{\pgfqpoint{0.469073in}{0.418699in}}%
\pgfpathlineto{\pgfqpoint{0.470363in}{0.405773in}}%
\pgfpathlineto{\pgfqpoint{0.470774in}{0.392846in}}%
\pgfpathlineto{\pgfqpoint{0.470840in}{0.379920in}}%
\pgfpathlineto{\pgfqpoint{0.470449in}{0.366993in}}%
\pgfpathlineto{\pgfqpoint{0.463985in}{0.355744in}}%
\pgfpathlineto{\pgfqpoint{0.450716in}{0.354573in}}%
\pgfpathlineto{\pgfqpoint{0.437447in}{0.354176in}}%
\pgfpathlineto{\pgfqpoint{0.434635in}{0.354067in}}%
\pgfpathlineto{\pgfqpoint{0.424178in}{0.353334in}}%
\pgfpathlineto{\pgfqpoint{0.415128in}{0.341140in}}%
\pgfpathlineto{\pgfqpoint{0.414738in}{0.328214in}}%
\pgfpathlineto{\pgfqpoint{0.414832in}{0.315287in}}%
\pgfpathlineto{\pgfqpoint{0.415318in}{0.302361in}}%
\pgfpathlineto{\pgfqpoint{0.416853in}{0.289434in}}%
\pgfpathlineto{\pgfqpoint{0.424178in}{0.278518in}}%
\pgfpathlineto{\pgfqpoint{0.432435in}{0.276508in}}%
\pgfpathlineto{\pgfqpoint{0.437447in}{0.275836in}}%
\pgfpathlineto{\pgfqpoint{0.450716in}{0.275613in}}%
\pgfpathlineto{\pgfqpoint{0.459122in}{0.276508in}}%
\pgfpathlineto{\pgfqpoint{0.463985in}{0.277646in}}%
\pgfpathlineto{\pgfqpoint{0.470453in}{0.289434in}}%
\pgfpathlineto{\pgfqpoint{0.471396in}{0.302361in}}%
\pgfpathlineto{\pgfqpoint{0.471776in}{0.315287in}}%
\pgfpathlineto{\pgfqpoint{0.472073in}{0.328214in}}%
\pgfpathlineto{\pgfqpoint{0.472656in}{0.341140in}}%
\pgfpathlineto{\pgfqpoint{0.477254in}{0.351338in}}%
\pgfpathlineto{\pgfqpoint{0.490523in}{0.352912in}}%
\pgfpathlineto{\pgfqpoint{0.503792in}{0.352819in}}%
\pgfpathlineto{\pgfqpoint{0.517061in}{0.351694in}}%
\pgfpathlineto{\pgfqpoint{0.528476in}{0.341140in}}%
\pgfpathlineto{\pgfqpoint{0.529986in}{0.328214in}}%
\pgfpathlineto{\pgfqpoint{0.530330in}{0.319855in}}%
\pgfpathlineto{\pgfqpoint{0.530485in}{0.315287in}}%
\pgfpathlineto{\pgfqpoint{0.530756in}{0.302361in}}%
\pgfpathlineto{\pgfqpoint{0.531055in}{0.289434in}}%
\pgfpathlineto{\pgfqpoint{0.532796in}{0.276508in}}%
\pgfpathlineto{\pgfqpoint{0.543598in}{0.273281in}}%
\pgfpathlineto{\pgfqpoint{0.556867in}{0.272909in}}%
\pgfpathlineto{\pgfqpoint{0.570136in}{0.272571in}}%
\pgfpathlineto{\pgfqpoint{0.583405in}{0.271154in}}%
\pgfpathlineto{\pgfqpoint{0.589029in}{0.263581in}}%
\pgfpathlineto{\pgfqpoint{0.589997in}{0.250655in}}%
\pgfpathlineto{\pgfqpoint{0.590348in}{0.237728in}}%
\pgfpathlineto{\pgfqpoint{0.590695in}{0.224802in}}%
\pgfpathlineto{\pgfqpoint{0.591382in}{0.211875in}}%
\pgfpathclose%
\pgfpathmoveto{\pgfqpoint{0.596321in}{0.276508in}}%
\pgfpathlineto{\pgfqpoint{0.591895in}{0.289434in}}%
\pgfpathlineto{\pgfqpoint{0.591606in}{0.302361in}}%
\pgfpathlineto{\pgfqpoint{0.591771in}{0.315287in}}%
\pgfpathlineto{\pgfqpoint{0.592433in}{0.328214in}}%
\pgfpathlineto{\pgfqpoint{0.594819in}{0.341140in}}%
\pgfpathlineto{\pgfqpoint{0.596674in}{0.344449in}}%
\pgfpathlineto{\pgfqpoint{0.609943in}{0.349932in}}%
\pgfpathlineto{\pgfqpoint{0.623212in}{0.350145in}}%
\pgfpathlineto{\pgfqpoint{0.636481in}{0.347418in}}%
\pgfpathlineto{\pgfqpoint{0.642555in}{0.341140in}}%
\pgfpathlineto{\pgfqpoint{0.645598in}{0.328214in}}%
\pgfpathlineto{\pgfqpoint{0.646416in}{0.315287in}}%
\pgfpathlineto{\pgfqpoint{0.646500in}{0.302361in}}%
\pgfpathlineto{\pgfqpoint{0.645730in}{0.289434in}}%
\pgfpathlineto{\pgfqpoint{0.637641in}{0.276508in}}%
\pgfpathlineto{\pgfqpoint{0.636481in}{0.276133in}}%
\pgfpathlineto{\pgfqpoint{0.623212in}{0.274694in}}%
\pgfpathlineto{\pgfqpoint{0.609943in}{0.274593in}}%
\pgfpathlineto{\pgfqpoint{0.596674in}{0.276326in}}%
\pgfpathclose%
\pgfpathmoveto{\pgfqpoint{0.795120in}{0.276508in}}%
\pgfpathlineto{\pgfqpoint{0.782439in}{0.277958in}}%
\pgfpathlineto{\pgfqpoint{0.771223in}{0.289434in}}%
\pgfpathlineto{\pgfqpoint{0.769276in}{0.302361in}}%
\pgfpathlineto{\pgfqpoint{0.769170in}{0.304546in}}%
\pgfpathlineto{\pgfqpoint{0.768737in}{0.315287in}}%
\pgfpathlineto{\pgfqpoint{0.768709in}{0.328214in}}%
\pgfpathlineto{\pgfqpoint{0.769170in}{0.337943in}}%
\pgfpathlineto{\pgfqpoint{0.769454in}{0.341140in}}%
\pgfpathlineto{\pgfqpoint{0.782439in}{0.352759in}}%
\pgfpathlineto{\pgfqpoint{0.795708in}{0.353385in}}%
\pgfpathlineto{\pgfqpoint{0.808977in}{0.353346in}}%
\pgfpathlineto{\pgfqpoint{0.822246in}{0.350758in}}%
\pgfpathlineto{\pgfqpoint{0.824595in}{0.341140in}}%
\pgfpathlineto{\pgfqpoint{0.824852in}{0.328214in}}%
\pgfpathlineto{\pgfqpoint{0.824726in}{0.315287in}}%
\pgfpathlineto{\pgfqpoint{0.824249in}{0.302361in}}%
\pgfpathlineto{\pgfqpoint{0.822752in}{0.289434in}}%
\pgfpathlineto{\pgfqpoint{0.822246in}{0.287583in}}%
\pgfpathlineto{\pgfqpoint{0.808977in}{0.277180in}}%
\pgfpathlineto{\pgfqpoint{0.796846in}{0.276508in}}%
\pgfpathlineto{\pgfqpoint{0.795708in}{0.276461in}}%
\pgfpathclose%
\pgfpathmoveto{\pgfqpoint{0.427951in}{0.289434in}}%
\pgfpathlineto{\pgfqpoint{0.424178in}{0.294397in}}%
\pgfpathlineto{\pgfqpoint{0.421697in}{0.302361in}}%
\pgfpathlineto{\pgfqpoint{0.420627in}{0.315287in}}%
\pgfpathlineto{\pgfqpoint{0.421543in}{0.328214in}}%
\pgfpathlineto{\pgfqpoint{0.424178in}{0.336461in}}%
\pgfpathlineto{\pgfqpoint{0.428264in}{0.341140in}}%
\pgfpathlineto{\pgfqpoint{0.437447in}{0.345449in}}%
\pgfpathlineto{\pgfqpoint{0.450716in}{0.345348in}}%
\pgfpathlineto{\pgfqpoint{0.458788in}{0.341140in}}%
\pgfpathlineto{\pgfqpoint{0.463985in}{0.332802in}}%
\pgfpathlineto{\pgfqpoint{0.465110in}{0.328214in}}%
\pgfpathlineto{\pgfqpoint{0.465887in}{0.315287in}}%
\pgfpathlineto{\pgfqpoint{0.464888in}{0.302361in}}%
\pgfpathlineto{\pgfqpoint{0.463985in}{0.298759in}}%
\pgfpathlineto{\pgfqpoint{0.458772in}{0.289434in}}%
\pgfpathlineto{\pgfqpoint{0.450716in}{0.284479in}}%
\pgfpathlineto{\pgfqpoint{0.437447in}{0.284265in}}%
\pgfpathclose%
\pgfpathmoveto{\pgfqpoint{0.540989in}{0.289434in}}%
\pgfpathlineto{\pgfqpoint{0.537300in}{0.302361in}}%
\pgfpathlineto{\pgfqpoint{0.536428in}{0.315287in}}%
\pgfpathlineto{\pgfqpoint{0.536978in}{0.328214in}}%
\pgfpathlineto{\pgfqpoint{0.540459in}{0.341140in}}%
\pgfpathlineto{\pgfqpoint{0.543598in}{0.344794in}}%
\pgfpathlineto{\pgfqpoint{0.556867in}{0.348785in}}%
\pgfpathlineto{\pgfqpoint{0.570136in}{0.348441in}}%
\pgfpathlineto{\pgfqpoint{0.582429in}{0.341140in}}%
\pgfpathlineto{\pgfqpoint{0.583405in}{0.339014in}}%
\pgfpathlineto{\pgfqpoint{0.585521in}{0.328214in}}%
\pgfpathlineto{\pgfqpoint{0.585922in}{0.315287in}}%
\pgfpathlineto{\pgfqpoint{0.585144in}{0.302361in}}%
\pgfpathlineto{\pgfqpoint{0.583405in}{0.294081in}}%
\pgfpathlineto{\pgfqpoint{0.581394in}{0.289434in}}%
\pgfpathlineto{\pgfqpoint{0.570136in}{0.281586in}}%
\pgfpathlineto{\pgfqpoint{0.556867in}{0.280980in}}%
\pgfpathlineto{\pgfqpoint{0.543598in}{0.286028in}}%
\pgfpathclose%
\pgfpathmoveto{\pgfqpoint{0.655900in}{0.289434in}}%
\pgfpathlineto{\pgfqpoint{0.653171in}{0.302361in}}%
\pgfpathlineto{\pgfqpoint{0.652463in}{0.315287in}}%
\pgfpathlineto{\pgfqpoint{0.652698in}{0.328214in}}%
\pgfpathlineto{\pgfqpoint{0.654713in}{0.341140in}}%
\pgfpathlineto{\pgfqpoint{0.663019in}{0.349584in}}%
\pgfpathlineto{\pgfqpoint{0.676288in}{0.351404in}}%
\pgfpathlineto{\pgfqpoint{0.689557in}{0.351111in}}%
\pgfpathlineto{\pgfqpoint{0.702826in}{0.344561in}}%
\pgfpathlineto{\pgfqpoint{0.704058in}{0.341140in}}%
\pgfpathlineto{\pgfqpoint{0.705429in}{0.328214in}}%
\pgfpathlineto{\pgfqpoint{0.705531in}{0.315287in}}%
\pgfpathlineto{\pgfqpoint{0.704925in}{0.302361in}}%
\pgfpathlineto{\pgfqpoint{0.702826in}{0.289997in}}%
\pgfpathlineto{\pgfqpoint{0.702643in}{0.289434in}}%
\pgfpathlineto{\pgfqpoint{0.689557in}{0.279148in}}%
\pgfpathlineto{\pgfqpoint{0.676288in}{0.278402in}}%
\pgfpathlineto{\pgfqpoint{0.663019in}{0.281230in}}%
\pgfpathclose%
\pgfpathmoveto{\pgfqpoint{0.712652in}{0.289434in}}%
\pgfpathlineto{\pgfqpoint{0.711383in}{0.302361in}}%
\pgfpathlineto{\pgfqpoint{0.711378in}{0.315287in}}%
\pgfpathlineto{\pgfqpoint{0.712334in}{0.328214in}}%
\pgfpathlineto{\pgfqpoint{0.716095in}{0.341019in}}%
\pgfpathlineto{\pgfqpoint{0.716183in}{0.341140in}}%
\pgfpathlineto{\pgfqpoint{0.729364in}{0.347759in}}%
\pgfpathlineto{\pgfqpoint{0.742633in}{0.347912in}}%
\pgfpathlineto{\pgfqpoint{0.755902in}{0.343110in}}%
\pgfpathlineto{\pgfqpoint{0.757591in}{0.341140in}}%
\pgfpathlineto{\pgfqpoint{0.761817in}{0.328214in}}%
\pgfpathlineto{\pgfqpoint{0.762877in}{0.315287in}}%
\pgfpathlineto{\pgfqpoint{0.762811in}{0.302361in}}%
\pgfpathlineto{\pgfqpoint{0.761215in}{0.289434in}}%
\pgfpathlineto{\pgfqpoint{0.755902in}{0.280676in}}%
\pgfpathlineto{\pgfqpoint{0.742633in}{0.277004in}}%
\pgfpathlineto{\pgfqpoint{0.729364in}{0.277016in}}%
\pgfpathlineto{\pgfqpoint{0.716095in}{0.281803in}}%
\pgfpathclose%
\pgfpathmoveto{\pgfqpoint{0.832765in}{0.289434in}}%
\pgfpathlineto{\pgfqpoint{0.830745in}{0.302361in}}%
\pgfpathlineto{\pgfqpoint{0.830610in}{0.315287in}}%
\pgfpathlineto{\pgfqpoint{0.831797in}{0.328214in}}%
\pgfpathlineto{\pgfqpoint{0.835515in}{0.339106in}}%
\pgfpathlineto{\pgfqpoint{0.837378in}{0.341140in}}%
\pgfpathlineto{\pgfqpoint{0.848784in}{0.346262in}}%
\pgfpathlineto{\pgfqpoint{0.862053in}{0.346102in}}%
\pgfpathlineto{\pgfqpoint{0.872529in}{0.341140in}}%
\pgfpathlineto{\pgfqpoint{0.875322in}{0.337915in}}%
\pgfpathlineto{\pgfqpoint{0.878561in}{0.328214in}}%
\pgfpathlineto{\pgfqpoint{0.879795in}{0.315287in}}%
\pgfpathlineto{\pgfqpoint{0.879622in}{0.302361in}}%
\pgfpathlineto{\pgfqpoint{0.877426in}{0.289434in}}%
\pgfpathlineto{\pgfqpoint{0.875322in}{0.285400in}}%
\pgfpathlineto{\pgfqpoint{0.862053in}{0.278959in}}%
\pgfpathlineto{\pgfqpoint{0.848784in}{0.278781in}}%
\pgfpathlineto{\pgfqpoint{0.835515in}{0.284323in}}%
\pgfpathclose%
\pgfpathmoveto{\pgfqpoint{0.952251in}{0.289434in}}%
\pgfpathlineto{\pgfqpoint{0.949694in}{0.302361in}}%
\pgfpathlineto{\pgfqpoint{0.949467in}{0.315287in}}%
\pgfpathlineto{\pgfqpoint{0.950824in}{0.328214in}}%
\pgfpathlineto{\pgfqpoint{0.954936in}{0.338752in}}%
\pgfpathlineto{\pgfqpoint{0.957647in}{0.341140in}}%
\pgfpathlineto{\pgfqpoint{0.968205in}{0.345349in}}%
\pgfpathlineto{\pgfqpoint{0.981473in}{0.344714in}}%
\pgfpathlineto{\pgfqpoint{0.988247in}{0.341140in}}%
\pgfpathlineto{\pgfqpoint{0.994742in}{0.331785in}}%
\pgfpathlineto{\pgfqpoint{0.995768in}{0.328214in}}%
\pgfpathlineto{\pgfqpoint{0.997116in}{0.315287in}}%
\pgfpathlineto{\pgfqpoint{0.996874in}{0.302361in}}%
\pgfpathlineto{\pgfqpoint{0.994742in}{0.291012in}}%
\pgfpathlineto{\pgfqpoint{0.994107in}{0.289434in}}%
\pgfpathlineto{\pgfqpoint{0.981473in}{0.280427in}}%
\pgfpathlineto{\pgfqpoint{0.968205in}{0.279866in}}%
\pgfpathlineto{\pgfqpoint{0.954936in}{0.285144in}}%
\pgfpathclose%
\pgfpathmoveto{\pgfqpoint{1.071098in}{0.289434in}}%
\pgfpathlineto{\pgfqpoint{1.068212in}{0.302361in}}%
\pgfpathlineto{\pgfqpoint{1.067932in}{0.315287in}}%
\pgfpathlineto{\pgfqpoint{1.069401in}{0.328214in}}%
\pgfpathlineto{\pgfqpoint{1.074356in}{0.339458in}}%
\pgfpathlineto{\pgfqpoint{1.076720in}{0.341140in}}%
\pgfpathlineto{\pgfqpoint{1.087625in}{0.344960in}}%
\pgfpathlineto{\pgfqpoint{1.100894in}{0.343767in}}%
\pgfpathlineto{\pgfqpoint{1.105379in}{0.341140in}}%
\pgfpathlineto{\pgfqpoint{1.113161in}{0.328214in}}%
\pgfpathlineto{\pgfqpoint{1.114163in}{0.321693in}}%
\pgfpathlineto{\pgfqpoint{1.114801in}{0.315287in}}%
\pgfpathlineto{\pgfqpoint{1.114528in}{0.302361in}}%
\pgfpathlineto{\pgfqpoint{1.114163in}{0.299923in}}%
\pgfpathlineto{\pgfqpoint{1.111017in}{0.289434in}}%
\pgfpathlineto{\pgfqpoint{1.100894in}{0.281386in}}%
\pgfpathlineto{\pgfqpoint{1.087625in}{0.280345in}}%
\pgfpathlineto{\pgfqpoint{1.074356in}{0.284884in}}%
\pgfpathclose%
\pgfpathmoveto{\pgfqpoint{1.189271in}{0.289434in}}%
\pgfpathlineto{\pgfqpoint{1.186267in}{0.302361in}}%
\pgfpathlineto{\pgfqpoint{1.185970in}{0.315287in}}%
\pgfpathlineto{\pgfqpoint{1.187493in}{0.328214in}}%
\pgfpathlineto{\pgfqpoint{1.193777in}{0.340941in}}%
\pgfpathlineto{\pgfqpoint{1.194125in}{0.341140in}}%
\pgfpathlineto{\pgfqpoint{1.207045in}{0.345058in}}%
\pgfpathlineto{\pgfqpoint{1.220314in}{0.343300in}}%
\pgfpathlineto{\pgfqpoint{1.223643in}{0.341140in}}%
\pgfpathlineto{\pgfqpoint{1.230938in}{0.328214in}}%
\pgfpathlineto{\pgfqpoint{1.232655in}{0.315287in}}%
\pgfpathlineto{\pgfqpoint{1.232327in}{0.302361in}}%
\pgfpathlineto{\pgfqpoint{1.228955in}{0.289434in}}%
\pgfpathlineto{\pgfqpoint{1.220314in}{0.281781in}}%
\pgfpathlineto{\pgfqpoint{1.207045in}{0.280264in}}%
\pgfpathlineto{\pgfqpoint{1.193777in}{0.283869in}}%
\pgfpathclose%
\pgfpathmoveto{\pgfqpoint{1.306701in}{0.289434in}}%
\pgfpathlineto{\pgfqpoint{1.303801in}{0.302361in}}%
\pgfpathlineto{\pgfqpoint{1.303529in}{0.315287in}}%
\pgfpathlineto{\pgfqpoint{1.305043in}{0.328214in}}%
\pgfpathlineto{\pgfqpoint{1.311398in}{0.341140in}}%
\pgfpathlineto{\pgfqpoint{1.313197in}{0.342580in}}%
\pgfpathlineto{\pgfqpoint{1.326466in}{0.345628in}}%
\pgfpathlineto{\pgfqpoint{1.339735in}{0.343383in}}%
\pgfpathlineto{\pgfqpoint{1.342854in}{0.341140in}}%
\pgfpathlineto{\pgfqpoint{1.349365in}{0.328214in}}%
\pgfpathlineto{\pgfqpoint{1.350904in}{0.315287in}}%
\pgfpathlineto{\pgfqpoint{1.350647in}{0.302361in}}%
\pgfpathlineto{\pgfqpoint{1.347745in}{0.289434in}}%
\pgfpathlineto{\pgfqpoint{1.339735in}{0.281524in}}%
\pgfpathlineto{\pgfqpoint{1.326466in}{0.279643in}}%
\pgfpathlineto{\pgfqpoint{1.313197in}{0.282284in}}%
\pgfpathclose%
\pgfpathmoveto{\pgfqpoint{1.423285in}{0.289434in}}%
\pgfpathlineto{\pgfqpoint{1.420731in}{0.302361in}}%
\pgfpathlineto{\pgfqpoint{1.420529in}{0.315287in}}%
\pgfpathlineto{\pgfqpoint{1.421965in}{0.328214in}}%
\pgfpathlineto{\pgfqpoint{1.427885in}{0.341140in}}%
\pgfpathlineto{\pgfqpoint{1.432617in}{0.344560in}}%
\pgfpathlineto{\pgfqpoint{1.445886in}{0.346672in}}%
\pgfpathlineto{\pgfqpoint{1.459155in}{0.344125in}}%
\pgfpathlineto{\pgfqpoint{1.462896in}{0.341140in}}%
\pgfpathlineto{\pgfqpoint{1.468344in}{0.328214in}}%
\pgfpathlineto{\pgfqpoint{1.469653in}{0.315287in}}%
\pgfpathlineto{\pgfqpoint{1.469504in}{0.302361in}}%
\pgfpathlineto{\pgfqpoint{1.467273in}{0.289434in}}%
\pgfpathlineto{\pgfqpoint{1.459155in}{0.280468in}}%
\pgfpathlineto{\pgfqpoint{1.445886in}{0.278479in}}%
\pgfpathlineto{\pgfqpoint{1.432617in}{0.280232in}}%
\pgfpathclose%
\pgfpathmoveto{\pgfqpoint{0.370855in}{0.366993in}}%
\pgfpathlineto{\pgfqpoint{0.363157in}{0.379920in}}%
\pgfpathlineto{\pgfqpoint{0.361342in}{0.392846in}}%
\pgfpathlineto{\pgfqpoint{0.361690in}{0.405773in}}%
\pgfpathlineto{\pgfqpoint{0.365117in}{0.418699in}}%
\pgfpathlineto{\pgfqpoint{0.371102in}{0.425086in}}%
\pgfpathlineto{\pgfqpoint{0.384371in}{0.427955in}}%
\pgfpathlineto{\pgfqpoint{0.397640in}{0.425593in}}%
\pgfpathlineto{\pgfqpoint{0.403999in}{0.418699in}}%
\pgfpathlineto{\pgfqpoint{0.406963in}{0.405773in}}%
\pgfpathlineto{\pgfqpoint{0.407216in}{0.392846in}}%
\pgfpathlineto{\pgfqpoint{0.405576in}{0.379920in}}%
\pgfpathlineto{\pgfqpoint{0.398335in}{0.366993in}}%
\pgfpathlineto{\pgfqpoint{0.397640in}{0.366490in}}%
\pgfpathlineto{\pgfqpoint{0.384371in}{0.363790in}}%
\pgfpathlineto{\pgfqpoint{0.371102in}{0.366812in}}%
\pgfpathclose%
\pgfpathmoveto{\pgfqpoint{0.483342in}{0.366993in}}%
\pgfpathlineto{\pgfqpoint{0.477669in}{0.379920in}}%
\pgfpathlineto{\pgfqpoint{0.477254in}{0.383330in}}%
\pgfpathlineto{\pgfqpoint{0.476456in}{0.392846in}}%
\pgfpathlineto{\pgfqpoint{0.476552in}{0.405773in}}%
\pgfpathlineto{\pgfqpoint{0.477254in}{0.412249in}}%
\pgfpathlineto{\pgfqpoint{0.478525in}{0.418699in}}%
\pgfpathlineto{\pgfqpoint{0.490523in}{0.430067in}}%
\pgfpathlineto{\pgfqpoint{0.503792in}{0.431433in}}%
\pgfpathlineto{\pgfqpoint{0.517061in}{0.429810in}}%
\pgfpathlineto{\pgfqpoint{0.526075in}{0.418699in}}%
\pgfpathlineto{\pgfqpoint{0.527523in}{0.405773in}}%
\pgfpathlineto{\pgfqpoint{0.527539in}{0.392846in}}%
\pgfpathlineto{\pgfqpoint{0.526448in}{0.379920in}}%
\pgfpathlineto{\pgfqpoint{0.521904in}{0.366993in}}%
\pgfpathlineto{\pgfqpoint{0.517061in}{0.363022in}}%
\pgfpathlineto{\pgfqpoint{0.503792in}{0.360701in}}%
\pgfpathlineto{\pgfqpoint{0.490523in}{0.362304in}}%
\pgfpathclose%
\pgfpathmoveto{\pgfqpoint{0.535524in}{0.366993in}}%
\pgfpathlineto{\pgfqpoint{0.533953in}{0.379920in}}%
\pgfpathlineto{\pgfqpoint{0.533795in}{0.392846in}}%
\pgfpathlineto{\pgfqpoint{0.534351in}{0.405773in}}%
\pgfpathlineto{\pgfqpoint{0.536458in}{0.418699in}}%
\pgfpathlineto{\pgfqpoint{0.543598in}{0.427897in}}%
\pgfpathlineto{\pgfqpoint{0.556867in}{0.430477in}}%
\pgfpathlineto{\pgfqpoint{0.570136in}{0.429492in}}%
\pgfpathlineto{\pgfqpoint{0.583405in}{0.419767in}}%
\pgfpathlineto{\pgfqpoint{0.583844in}{0.418699in}}%
\pgfpathlineto{\pgfqpoint{0.586195in}{0.405773in}}%
\pgfpathlineto{\pgfqpoint{0.586781in}{0.392846in}}%
\pgfpathlineto{\pgfqpoint{0.586475in}{0.379920in}}%
\pgfpathlineto{\pgfqpoint{0.584320in}{0.366993in}}%
\pgfpathlineto{\pgfqpoint{0.583405in}{0.365037in}}%
\pgfpathlineto{\pgfqpoint{0.570136in}{0.358041in}}%
\pgfpathlineto{\pgfqpoint{0.556867in}{0.357316in}}%
\pgfpathlineto{\pgfqpoint{0.543598in}{0.358644in}}%
\pgfpathclose%
\pgfpathmoveto{\pgfqpoint{0.596164in}{0.366993in}}%
\pgfpathlineto{\pgfqpoint{0.593231in}{0.379920in}}%
\pgfpathlineto{\pgfqpoint{0.592462in}{0.392846in}}%
\pgfpathlineto{\pgfqpoint{0.592384in}{0.405773in}}%
\pgfpathlineto{\pgfqpoint{0.593084in}{0.418699in}}%
\pgfpathlineto{\pgfqpoint{0.596674in}{0.429262in}}%
\pgfpathlineto{\pgfqpoint{0.600504in}{0.431626in}}%
\pgfpathlineto{\pgfqpoint{0.609943in}{0.433625in}}%
\pgfpathlineto{\pgfqpoint{0.623212in}{0.434131in}}%
\pgfpathlineto{\pgfqpoint{0.636481in}{0.433669in}}%
\pgfpathlineto{\pgfqpoint{0.643282in}{0.431626in}}%
\pgfpathlineto{\pgfqpoint{0.647220in}{0.418699in}}%
\pgfpathlineto{\pgfqpoint{0.647520in}{0.405773in}}%
\pgfpathlineto{\pgfqpoint{0.647357in}{0.392846in}}%
\pgfpathlineto{\pgfqpoint{0.646683in}{0.379920in}}%
\pgfpathlineto{\pgfqpoint{0.644174in}{0.366993in}}%
\pgfpathlineto{\pgfqpoint{0.636481in}{0.359822in}}%
\pgfpathlineto{\pgfqpoint{0.623212in}{0.358135in}}%
\pgfpathlineto{\pgfqpoint{0.609943in}{0.358898in}}%
\pgfpathlineto{\pgfqpoint{0.596674in}{0.366063in}}%
\pgfpathclose%
\pgfpathmoveto{\pgfqpoint{0.657664in}{0.366993in}}%
\pgfpathlineto{\pgfqpoint{0.654168in}{0.379920in}}%
\pgfpathlineto{\pgfqpoint{0.653621in}{0.392846in}}%
\pgfpathlineto{\pgfqpoint{0.654373in}{0.405773in}}%
\pgfpathlineto{\pgfqpoint{0.657644in}{0.418699in}}%
\pgfpathlineto{\pgfqpoint{0.663019in}{0.424870in}}%
\pgfpathlineto{\pgfqpoint{0.676288in}{0.428033in}}%
\pgfpathlineto{\pgfqpoint{0.689557in}{0.426274in}}%
\pgfpathlineto{\pgfqpoint{0.698324in}{0.418699in}}%
\pgfpathlineto{\pgfqpoint{0.702300in}{0.405773in}}%
\pgfpathlineto{\pgfqpoint{0.702826in}{0.398521in}}%
\pgfpathlineto{\pgfqpoint{0.703133in}{0.392846in}}%
\pgfpathlineto{\pgfqpoint{0.702826in}{0.385814in}}%
\pgfpathlineto{\pgfqpoint{0.702449in}{0.379920in}}%
\pgfpathlineto{\pgfqpoint{0.698003in}{0.366993in}}%
\pgfpathlineto{\pgfqpoint{0.689557in}{0.361193in}}%
\pgfpathlineto{\pgfqpoint{0.676288in}{0.359806in}}%
\pgfpathlineto{\pgfqpoint{0.663019in}{0.362094in}}%
\pgfpathclose%
\pgfpathmoveto{\pgfqpoint{0.779062in}{0.366993in}}%
\pgfpathlineto{\pgfqpoint{0.773939in}{0.379920in}}%
\pgfpathlineto{\pgfqpoint{0.773061in}{0.392846in}}%
\pgfpathlineto{\pgfqpoint{0.773985in}{0.405773in}}%
\pgfpathlineto{\pgfqpoint{0.778260in}{0.418699in}}%
\pgfpathlineto{\pgfqpoint{0.782439in}{0.423001in}}%
\pgfpathlineto{\pgfqpoint{0.795708in}{0.426145in}}%
\pgfpathlineto{\pgfqpoint{0.808977in}{0.423346in}}%
\pgfpathlineto{\pgfqpoint{0.813831in}{0.418699in}}%
\pgfpathlineto{\pgfqpoint{0.818411in}{0.405773in}}%
\pgfpathlineto{\pgfqpoint{0.819401in}{0.392846in}}%
\pgfpathlineto{\pgfqpoint{0.818408in}{0.379920in}}%
\pgfpathlineto{\pgfqpoint{0.812843in}{0.366993in}}%
\pgfpathlineto{\pgfqpoint{0.808977in}{0.364045in}}%
\pgfpathlineto{\pgfqpoint{0.795708in}{0.361720in}}%
\pgfpathlineto{\pgfqpoint{0.782439in}{0.364238in}}%
\pgfpathclose%
\pgfpathmoveto{\pgfqpoint{0.354353in}{0.444552in}}%
\pgfpathlineto{\pgfqpoint{0.354603in}{0.457479in}}%
\pgfpathlineto{\pgfqpoint{0.354764in}{0.470405in}}%
\pgfpathlineto{\pgfqpoint{0.355032in}{0.483332in}}%
\pgfpathlineto{\pgfqpoint{0.355684in}{0.496258in}}%
\pgfpathlineto{\pgfqpoint{0.357833in}{0.507035in}}%
\pgfpathlineto{\pgfqpoint{0.359122in}{0.509185in}}%
\pgfpathlineto{\pgfqpoint{0.371102in}{0.513790in}}%
\pgfpathlineto{\pgfqpoint{0.384371in}{0.514053in}}%
\pgfpathlineto{\pgfqpoint{0.397640in}{0.512635in}}%
\pgfpathlineto{\pgfqpoint{0.404585in}{0.509185in}}%
\pgfpathlineto{\pgfqpoint{0.410112in}{0.496258in}}%
\pgfpathlineto{\pgfqpoint{0.410909in}{0.488097in}}%
\pgfpathlineto{\pgfqpoint{0.411227in}{0.483332in}}%
\pgfpathlineto{\pgfqpoint{0.411504in}{0.470405in}}%
\pgfpathlineto{\pgfqpoint{0.411292in}{0.457479in}}%
\pgfpathlineto{\pgfqpoint{0.410909in}{0.452318in}}%
\pgfpathlineto{\pgfqpoint{0.409571in}{0.444552in}}%
\pgfpathlineto{\pgfqpoint{0.397640in}{0.437136in}}%
\pgfpathlineto{\pgfqpoint{0.384371in}{0.436340in}}%
\pgfpathlineto{\pgfqpoint{0.371102in}{0.436045in}}%
\pgfpathlineto{\pgfqpoint{0.357833in}{0.435248in}}%
\pgfpathclose%
\pgfpathmoveto{\pgfqpoint{0.427108in}{0.444552in}}%
\pgfpathlineto{\pgfqpoint{0.424178in}{0.446390in}}%
\pgfpathlineto{\pgfqpoint{0.418957in}{0.457479in}}%
\pgfpathlineto{\pgfqpoint{0.417433in}{0.470405in}}%
\pgfpathlineto{\pgfqpoint{0.417237in}{0.483332in}}%
\pgfpathlineto{\pgfqpoint{0.418227in}{0.496258in}}%
\pgfpathlineto{\pgfqpoint{0.424126in}{0.509185in}}%
\pgfpathlineto{\pgfqpoint{0.424178in}{0.509225in}}%
\pgfpathlineto{\pgfqpoint{0.437447in}{0.512904in}}%
\pgfpathlineto{\pgfqpoint{0.450716in}{0.513077in}}%
\pgfpathlineto{\pgfqpoint{0.463985in}{0.509441in}}%
\pgfpathlineto{\pgfqpoint{0.464262in}{0.509185in}}%
\pgfpathlineto{\pgfqpoint{0.468724in}{0.496258in}}%
\pgfpathlineto{\pgfqpoint{0.469371in}{0.483332in}}%
\pgfpathlineto{\pgfqpoint{0.469157in}{0.470405in}}%
\pgfpathlineto{\pgfqpoint{0.467904in}{0.457479in}}%
\pgfpathlineto{\pgfqpoint{0.463985in}{0.447137in}}%
\pgfpathlineto{\pgfqpoint{0.460752in}{0.444552in}}%
\pgfpathlineto{\pgfqpoint{0.450716in}{0.441314in}}%
\pgfpathlineto{\pgfqpoint{0.437447in}{0.441377in}}%
\pgfpathclose%
\pgfpathmoveto{\pgfqpoint{0.480249in}{0.444552in}}%
\pgfpathlineto{\pgfqpoint{0.477254in}{0.449999in}}%
\pgfpathlineto{\pgfqpoint{0.475794in}{0.457479in}}%
\pgfpathlineto{\pgfqpoint{0.475188in}{0.470405in}}%
\pgfpathlineto{\pgfqpoint{0.475487in}{0.483332in}}%
\pgfpathlineto{\pgfqpoint{0.477030in}{0.496258in}}%
\pgfpathlineto{\pgfqpoint{0.477254in}{0.497096in}}%
\pgfpathlineto{\pgfqpoint{0.487559in}{0.509185in}}%
\pgfpathlineto{\pgfqpoint{0.490523in}{0.510215in}}%
\pgfpathlineto{\pgfqpoint{0.503792in}{0.510960in}}%
\pgfpathlineto{\pgfqpoint{0.512749in}{0.509185in}}%
\pgfpathlineto{\pgfqpoint{0.517061in}{0.507756in}}%
\pgfpathlineto{\pgfqpoint{0.524661in}{0.496258in}}%
\pgfpathlineto{\pgfqpoint{0.526590in}{0.483332in}}%
\pgfpathlineto{\pgfqpoint{0.526915in}{0.470405in}}%
\pgfpathlineto{\pgfqpoint{0.525993in}{0.457479in}}%
\pgfpathlineto{\pgfqpoint{0.520781in}{0.444552in}}%
\pgfpathlineto{\pgfqpoint{0.517061in}{0.441975in}}%
\pgfpathlineto{\pgfqpoint{0.503792in}{0.439595in}}%
\pgfpathlineto{\pgfqpoint{0.490523in}{0.440013in}}%
\pgfpathclose%
\pgfpathmoveto{\pgfqpoint{0.537906in}{0.444552in}}%
\pgfpathlineto{\pgfqpoint{0.533992in}{0.457479in}}%
\pgfpathlineto{\pgfqpoint{0.533100in}{0.470405in}}%
\pgfpathlineto{\pgfqpoint{0.532868in}{0.483332in}}%
\pgfpathlineto{\pgfqpoint{0.533095in}{0.496258in}}%
\pgfpathlineto{\pgfqpoint{0.534909in}{0.509185in}}%
\pgfpathlineto{\pgfqpoint{0.543598in}{0.515047in}}%
\pgfpathlineto{\pgfqpoint{0.556867in}{0.516010in}}%
\pgfpathlineto{\pgfqpoint{0.570136in}{0.516256in}}%
\pgfpathlineto{\pgfqpoint{0.583405in}{0.516307in}}%
\pgfpathlineto{\pgfqpoint{0.589593in}{0.509185in}}%
\pgfpathlineto{\pgfqpoint{0.589570in}{0.496258in}}%
\pgfpathlineto{\pgfqpoint{0.589460in}{0.483332in}}%
\pgfpathlineto{\pgfqpoint{0.589218in}{0.470405in}}%
\pgfpathlineto{\pgfqpoint{0.588614in}{0.457479in}}%
\pgfpathlineto{\pgfqpoint{0.585978in}{0.444552in}}%
\pgfpathlineto{\pgfqpoint{0.583405in}{0.441504in}}%
\pgfpathlineto{\pgfqpoint{0.570136in}{0.438384in}}%
\pgfpathlineto{\pgfqpoint{0.556867in}{0.438381in}}%
\pgfpathlineto{\pgfqpoint{0.543598in}{0.440413in}}%
\pgfpathclose%
\pgfpathmoveto{\pgfqpoint{0.605929in}{0.444552in}}%
\pgfpathlineto{\pgfqpoint{0.596674in}{0.456568in}}%
\pgfpathlineto{\pgfqpoint{0.596440in}{0.457479in}}%
\pgfpathlineto{\pgfqpoint{0.595207in}{0.470405in}}%
\pgfpathlineto{\pgfqpoint{0.595533in}{0.483332in}}%
\pgfpathlineto{\pgfqpoint{0.596674in}{0.490797in}}%
\pgfpathlineto{\pgfqpoint{0.598169in}{0.496258in}}%
\pgfpathlineto{\pgfqpoint{0.609943in}{0.507288in}}%
\pgfpathlineto{\pgfqpoint{0.623212in}{0.508158in}}%
\pgfpathlineto{\pgfqpoint{0.636481in}{0.502222in}}%
\pgfpathlineto{\pgfqpoint{0.639998in}{0.496258in}}%
\pgfpathlineto{\pgfqpoint{0.642547in}{0.483332in}}%
\pgfpathlineto{\pgfqpoint{0.642890in}{0.470405in}}%
\pgfpathlineto{\pgfqpoint{0.641423in}{0.457479in}}%
\pgfpathlineto{\pgfqpoint{0.636481in}{0.447432in}}%
\pgfpathlineto{\pgfqpoint{0.631107in}{0.444552in}}%
\pgfpathlineto{\pgfqpoint{0.623212in}{0.442383in}}%
\pgfpathlineto{\pgfqpoint{0.609943in}{0.442956in}}%
\pgfpathclose%
\pgfpathmoveto{\pgfqpoint{0.304269in}{0.457479in}}%
\pgfpathlineto{\pgfqpoint{0.302023in}{0.470405in}}%
\pgfpathlineto{\pgfqpoint{0.301866in}{0.483332in}}%
\pgfpathlineto{\pgfqpoint{0.303718in}{0.496258in}}%
\pgfpathlineto{\pgfqpoint{0.304758in}{0.498921in}}%
\pgfpathlineto{\pgfqpoint{0.318027in}{0.509014in}}%
\pgfpathlineto{\pgfqpoint{0.321961in}{0.509185in}}%
\pgfpathlineto{\pgfqpoint{0.331295in}{0.509527in}}%
\pgfpathlineto{\pgfqpoint{0.332695in}{0.509185in}}%
\pgfpathlineto{\pgfqpoint{0.344564in}{0.502518in}}%
\pgfpathlineto{\pgfqpoint{0.347252in}{0.496258in}}%
\pgfpathlineto{\pgfqpoint{0.348832in}{0.483332in}}%
\pgfpathlineto{\pgfqpoint{0.348650in}{0.470405in}}%
\pgfpathlineto{\pgfqpoint{0.346596in}{0.457479in}}%
\pgfpathlineto{\pgfqpoint{0.344564in}{0.452913in}}%
\pgfpathlineto{\pgfqpoint{0.331295in}{0.444642in}}%
\pgfpathlineto{\pgfqpoint{0.318027in}{0.445209in}}%
\pgfpathlineto{\pgfqpoint{0.304758in}{0.456278in}}%
\pgfpathclose%
\pgfpathmoveto{\pgfqpoint{0.296341in}{0.522111in}}%
\pgfpathlineto{\pgfqpoint{0.296047in}{0.535038in}}%
\pgfpathlineto{\pgfqpoint{0.296092in}{0.547964in}}%
\pgfpathlineto{\pgfqpoint{0.296274in}{0.560891in}}%
\pgfpathlineto{\pgfqpoint{0.296744in}{0.573817in}}%
\pgfpathlineto{\pgfqpoint{0.298526in}{0.586744in}}%
\pgfpathlineto{\pgfqpoint{0.304758in}{0.593520in}}%
\pgfpathlineto{\pgfqpoint{0.318027in}{0.595189in}}%
\pgfpathlineto{\pgfqpoint{0.331295in}{0.594774in}}%
\pgfpathlineto{\pgfqpoint{0.344564in}{0.591231in}}%
\pgfpathlineto{\pgfqpoint{0.348375in}{0.586744in}}%
\pgfpathlineto{\pgfqpoint{0.351310in}{0.573817in}}%
\pgfpathlineto{\pgfqpoint{0.352044in}{0.560891in}}%
\pgfpathlineto{\pgfqpoint{0.352142in}{0.547964in}}%
\pgfpathlineto{\pgfqpoint{0.351602in}{0.535038in}}%
\pgfpathlineto{\pgfqpoint{0.347541in}{0.522111in}}%
\pgfpathlineto{\pgfqpoint{0.344564in}{0.520184in}}%
\pgfpathlineto{\pgfqpoint{0.331295in}{0.518084in}}%
\pgfpathlineto{\pgfqpoint{0.318027in}{0.517645in}}%
\pgfpathlineto{\pgfqpoint{0.304758in}{0.517618in}}%
\pgfpathclose%
\pgfpathmoveto{\pgfqpoint{0.380931in}{0.522111in}}%
\pgfpathlineto{\pgfqpoint{0.371102in}{0.523693in}}%
\pgfpathlineto{\pgfqpoint{0.360650in}{0.535038in}}%
\pgfpathlineto{\pgfqpoint{0.358296in}{0.547964in}}%
\pgfpathlineto{\pgfqpoint{0.357833in}{0.559752in}}%
\pgfpathlineto{\pgfqpoint{0.357797in}{0.560891in}}%
\pgfpathlineto{\pgfqpoint{0.357833in}{0.561914in}}%
\pgfpathlineto{\pgfqpoint{0.358433in}{0.573817in}}%
\pgfpathlineto{\pgfqpoint{0.362026in}{0.586744in}}%
\pgfpathlineto{\pgfqpoint{0.371102in}{0.593086in}}%
\pgfpathlineto{\pgfqpoint{0.384371in}{0.594571in}}%
\pgfpathlineto{\pgfqpoint{0.397640in}{0.593867in}}%
\pgfpathlineto{\pgfqpoint{0.408031in}{0.586744in}}%
\pgfpathlineto{\pgfqpoint{0.410456in}{0.573817in}}%
\pgfpathlineto{\pgfqpoint{0.410815in}{0.560891in}}%
\pgfpathlineto{\pgfqpoint{0.410381in}{0.547964in}}%
\pgfpathlineto{\pgfqpoint{0.408534in}{0.535038in}}%
\pgfpathlineto{\pgfqpoint{0.397640in}{0.523013in}}%
\pgfpathlineto{\pgfqpoint{0.389590in}{0.522111in}}%
\pgfpathlineto{\pgfqpoint{0.384371in}{0.521719in}}%
\pgfpathclose%
\pgfpathmoveto{\pgfqpoint{0.431809in}{0.522111in}}%
\pgfpathlineto{\pgfqpoint{0.424178in}{0.524356in}}%
\pgfpathlineto{\pgfqpoint{0.418147in}{0.535038in}}%
\pgfpathlineto{\pgfqpoint{0.416890in}{0.547964in}}%
\pgfpathlineto{\pgfqpoint{0.416907in}{0.560891in}}%
\pgfpathlineto{\pgfqpoint{0.418022in}{0.573817in}}%
\pgfpathlineto{\pgfqpoint{0.423000in}{0.586744in}}%
\pgfpathlineto{\pgfqpoint{0.424178in}{0.587861in}}%
\pgfpathlineto{\pgfqpoint{0.437447in}{0.591715in}}%
\pgfpathlineto{\pgfqpoint{0.450716in}{0.590998in}}%
\pgfpathlineto{\pgfqpoint{0.459766in}{0.586744in}}%
\pgfpathlineto{\pgfqpoint{0.463985in}{0.581544in}}%
\pgfpathlineto{\pgfqpoint{0.466396in}{0.573817in}}%
\pgfpathlineto{\pgfqpoint{0.467657in}{0.560891in}}%
\pgfpathlineto{\pgfqpoint{0.467614in}{0.547964in}}%
\pgfpathlineto{\pgfqpoint{0.466009in}{0.535038in}}%
\pgfpathlineto{\pgfqpoint{0.463985in}{0.529897in}}%
\pgfpathlineto{\pgfqpoint{0.451457in}{0.522111in}}%
\pgfpathlineto{\pgfqpoint{0.450716in}{0.521926in}}%
\pgfpathlineto{\pgfqpoint{0.437447in}{0.521277in}}%
\pgfpathclose%
\pgfpathmoveto{\pgfqpoint{0.247305in}{0.535038in}}%
\pgfpathlineto{\pgfqpoint{0.243725in}{0.547964in}}%
\pgfpathlineto{\pgfqpoint{0.243067in}{0.560891in}}%
\pgfpathlineto{\pgfqpoint{0.244347in}{0.573817in}}%
\pgfpathlineto{\pgfqpoint{0.250880in}{0.586744in}}%
\pgfpathlineto{\pgfqpoint{0.251682in}{0.587377in}}%
\pgfpathlineto{\pgfqpoint{0.264951in}{0.590763in}}%
\pgfpathlineto{\pgfqpoint{0.278220in}{0.589384in}}%
\pgfpathlineto{\pgfqpoint{0.282596in}{0.586744in}}%
\pgfpathlineto{\pgfqpoint{0.288931in}{0.573817in}}%
\pgfpathlineto{\pgfqpoint{0.290056in}{0.560891in}}%
\pgfpathlineto{\pgfqpoint{0.289404in}{0.547964in}}%
\pgfpathlineto{\pgfqpoint{0.285927in}{0.535038in}}%
\pgfpathlineto{\pgfqpoint{0.278220in}{0.527486in}}%
\pgfpathlineto{\pgfqpoint{0.264951in}{0.525653in}}%
\pgfpathlineto{\pgfqpoint{0.251682in}{0.529706in}}%
\pgfpathclose%
\pgfpathmoveto{\pgfqpoint{0.539711in}{0.535038in}}%
\pgfpathlineto{\pgfqpoint{0.537311in}{0.547964in}}%
\pgfpathlineto{\pgfqpoint{0.537184in}{0.560891in}}%
\pgfpathlineto{\pgfqpoint{0.538871in}{0.573817in}}%
\pgfpathlineto{\pgfqpoint{0.543598in}{0.583524in}}%
\pgfpathlineto{\pgfqpoint{0.549318in}{0.586744in}}%
\pgfpathlineto{\pgfqpoint{0.556867in}{0.588895in}}%
\pgfpathlineto{\pgfqpoint{0.570136in}{0.587496in}}%
\pgfpathlineto{\pgfqpoint{0.571583in}{0.586744in}}%
\pgfpathlineto{\pgfqpoint{0.581486in}{0.573817in}}%
\pgfpathlineto{\pgfqpoint{0.583405in}{0.563120in}}%
\pgfpathlineto{\pgfqpoint{0.583654in}{0.560891in}}%
\pgfpathlineto{\pgfqpoint{0.583492in}{0.547964in}}%
\pgfpathlineto{\pgfqpoint{0.583405in}{0.547335in}}%
\pgfpathlineto{\pgfqpoint{0.580229in}{0.535038in}}%
\pgfpathlineto{\pgfqpoint{0.570136in}{0.525870in}}%
\pgfpathlineto{\pgfqpoint{0.556867in}{0.524457in}}%
\pgfpathlineto{\pgfqpoint{0.543598in}{0.529022in}}%
\pgfpathclose%
\pgfpathmoveto{\pgfqpoint{0.302915in}{0.612597in}}%
\pgfpathlineto{\pgfqpoint{0.299870in}{0.625523in}}%
\pgfpathlineto{\pgfqpoint{0.299143in}{0.638450in}}%
\pgfpathlineto{\pgfqpoint{0.299434in}{0.651377in}}%
\pgfpathlineto{\pgfqpoint{0.301485in}{0.664303in}}%
\pgfpathlineto{\pgfqpoint{0.304758in}{0.670034in}}%
\pgfpathlineto{\pgfqpoint{0.318027in}{0.675129in}}%
\pgfpathlineto{\pgfqpoint{0.331295in}{0.675619in}}%
\pgfpathlineto{\pgfqpoint{0.344564in}{0.672998in}}%
\pgfpathlineto{\pgfqpoint{0.350128in}{0.664303in}}%
\pgfpathlineto{\pgfqpoint{0.351503in}{0.651377in}}%
\pgfpathlineto{\pgfqpoint{0.351644in}{0.638450in}}%
\pgfpathlineto{\pgfqpoint{0.351053in}{0.625523in}}%
\pgfpathlineto{\pgfqpoint{0.348611in}{0.612597in}}%
\pgfpathlineto{\pgfqpoint{0.344564in}{0.606914in}}%
\pgfpathlineto{\pgfqpoint{0.331295in}{0.603177in}}%
\pgfpathlineto{\pgfqpoint{0.318027in}{0.603653in}}%
\pgfpathlineto{\pgfqpoint{0.304758in}{0.609699in}}%
\pgfpathclose%
\pgfpathmoveto{\pgfqpoint{0.360317in}{0.612597in}}%
\pgfpathlineto{\pgfqpoint{0.357898in}{0.625523in}}%
\pgfpathlineto{\pgfqpoint{0.357833in}{0.627794in}}%
\pgfpathlineto{\pgfqpoint{0.357603in}{0.638450in}}%
\pgfpathlineto{\pgfqpoint{0.357833in}{0.643176in}}%
\pgfpathlineto{\pgfqpoint{0.358360in}{0.651377in}}%
\pgfpathlineto{\pgfqpoint{0.361794in}{0.664303in}}%
\pgfpathlineto{\pgfqpoint{0.371102in}{0.671778in}}%
\pgfpathlineto{\pgfqpoint{0.384371in}{0.673011in}}%
\pgfpathlineto{\pgfqpoint{0.397640in}{0.670302in}}%
\pgfpathlineto{\pgfqpoint{0.403888in}{0.664303in}}%
\pgfpathlineto{\pgfqpoint{0.407695in}{0.651377in}}%
\pgfpathlineto{\pgfqpoint{0.408577in}{0.638450in}}%
\pgfpathlineto{\pgfqpoint{0.408107in}{0.625523in}}%
\pgfpathlineto{\pgfqpoint{0.405154in}{0.612597in}}%
\pgfpathlineto{\pgfqpoint{0.397640in}{0.605161in}}%
\pgfpathlineto{\pgfqpoint{0.384371in}{0.602825in}}%
\pgfpathlineto{\pgfqpoint{0.371102in}{0.603713in}}%
\pgfpathclose%
\pgfpathmoveto{\pgfqpoint{0.483651in}{0.612597in}}%
\pgfpathlineto{\pgfqpoint{0.478866in}{0.625523in}}%
\pgfpathlineto{\pgfqpoint{0.478082in}{0.638450in}}%
\pgfpathlineto{\pgfqpoint{0.479326in}{0.651377in}}%
\pgfpathlineto{\pgfqpoint{0.485081in}{0.664303in}}%
\pgfpathlineto{\pgfqpoint{0.490523in}{0.668234in}}%
\pgfpathlineto{\pgfqpoint{0.503792in}{0.669712in}}%
\pgfpathlineto{\pgfqpoint{0.517061in}{0.664947in}}%
\pgfpathlineto{\pgfqpoint{0.517662in}{0.664303in}}%
\pgfpathlineto{\pgfqpoint{0.522926in}{0.651377in}}%
\pgfpathlineto{\pgfqpoint{0.524088in}{0.638450in}}%
\pgfpathlineto{\pgfqpoint{0.523305in}{0.625523in}}%
\pgfpathlineto{\pgfqpoint{0.518805in}{0.612597in}}%
\pgfpathlineto{\pgfqpoint{0.517061in}{0.610666in}}%
\pgfpathlineto{\pgfqpoint{0.503792in}{0.606237in}}%
\pgfpathlineto{\pgfqpoint{0.490523in}{0.607520in}}%
\pgfpathclose%
\pgfpathmoveto{\pgfqpoint{0.247597in}{0.690156in}}%
\pgfpathlineto{\pgfqpoint{0.241979in}{0.703083in}}%
\pgfpathlineto{\pgfqpoint{0.240697in}{0.716009in}}%
\pgfpathlineto{\pgfqpoint{0.240763in}{0.728936in}}%
\pgfpathlineto{\pgfqpoint{0.242436in}{0.741862in}}%
\pgfpathlineto{\pgfqpoint{0.251682in}{0.754059in}}%
\pgfpathlineto{\pgfqpoint{0.254794in}{0.754789in}}%
\pgfpathlineto{\pgfqpoint{0.264951in}{0.756326in}}%
\pgfpathlineto{\pgfqpoint{0.278220in}{0.755769in}}%
\pgfpathlineto{\pgfqpoint{0.281687in}{0.754789in}}%
\pgfpathlineto{\pgfqpoint{0.291488in}{0.741862in}}%
\pgfpathlineto{\pgfqpoint{0.291489in}{0.741857in}}%
\pgfpathlineto{\pgfqpoint{0.292532in}{0.728936in}}%
\pgfpathlineto{\pgfqpoint{0.292526in}{0.716009in}}%
\pgfpathlineto{\pgfqpoint{0.291605in}{0.703083in}}%
\pgfpathlineto{\pgfqpoint{0.291489in}{0.702390in}}%
\pgfpathlineto{\pgfqpoint{0.286628in}{0.690156in}}%
\pgfpathlineto{\pgfqpoint{0.278220in}{0.685486in}}%
\pgfpathlineto{\pgfqpoint{0.264951in}{0.684646in}}%
\pgfpathlineto{\pgfqpoint{0.251682in}{0.687188in}}%
\pgfpathclose%
\pgfpathmoveto{\pgfqpoint{0.302404in}{0.690156in}}%
\pgfpathlineto{\pgfqpoint{0.299119in}{0.703083in}}%
\pgfpathlineto{\pgfqpoint{0.298546in}{0.716009in}}%
\pgfpathlineto{\pgfqpoint{0.298883in}{0.728936in}}%
\pgfpathlineto{\pgfqpoint{0.300638in}{0.741862in}}%
\pgfpathlineto{\pgfqpoint{0.304758in}{0.749804in}}%
\pgfpathlineto{\pgfqpoint{0.318027in}{0.754730in}}%
\pgfpathlineto{\pgfqpoint{0.331295in}{0.754290in}}%
\pgfpathlineto{\pgfqpoint{0.344564in}{0.747426in}}%
\pgfpathlineto{\pgfqpoint{0.347337in}{0.741862in}}%
\pgfpathlineto{\pgfqpoint{0.349483in}{0.728936in}}%
\pgfpathlineto{\pgfqpoint{0.349866in}{0.716009in}}%
\pgfpathlineto{\pgfqpoint{0.349047in}{0.703083in}}%
\pgfpathlineto{\pgfqpoint{0.344823in}{0.690156in}}%
\pgfpathlineto{\pgfqpoint{0.344564in}{0.689837in}}%
\pgfpathlineto{\pgfqpoint{0.331295in}{0.684301in}}%
\pgfpathlineto{\pgfqpoint{0.318027in}{0.683881in}}%
\pgfpathlineto{\pgfqpoint{0.304758in}{0.687473in}}%
\pgfpathclose%
\pgfpathmoveto{\pgfqpoint{0.429155in}{0.690156in}}%
\pgfpathlineto{\pgfqpoint{0.424178in}{0.693818in}}%
\pgfpathlineto{\pgfqpoint{0.420459in}{0.703083in}}%
\pgfpathlineto{\pgfqpoint{0.419212in}{0.716009in}}%
\pgfpathlineto{\pgfqpoint{0.419700in}{0.728936in}}%
\pgfpathlineto{\pgfqpoint{0.422722in}{0.741862in}}%
\pgfpathlineto{\pgfqpoint{0.424178in}{0.744302in}}%
\pgfpathlineto{\pgfqpoint{0.437447in}{0.750972in}}%
\pgfpathlineto{\pgfqpoint{0.450716in}{0.749998in}}%
\pgfpathlineto{\pgfqpoint{0.460950in}{0.741862in}}%
\pgfpathlineto{\pgfqpoint{0.463985in}{0.734038in}}%
\pgfpathlineto{\pgfqpoint{0.464985in}{0.728936in}}%
\pgfpathlineto{\pgfqpoint{0.465490in}{0.716009in}}%
\pgfpathlineto{\pgfqpoint{0.464117in}{0.703083in}}%
\pgfpathlineto{\pgfqpoint{0.463985in}{0.702616in}}%
\pgfpathlineto{\pgfqpoint{0.454627in}{0.690156in}}%
\pgfpathlineto{\pgfqpoint{0.450716in}{0.688294in}}%
\pgfpathlineto{\pgfqpoint{0.437447in}{0.687439in}}%
\pgfpathclose%
\pgfpathmoveto{\pgfqpoint{0.245600in}{0.767715in}}%
\pgfpathlineto{\pgfqpoint{0.239813in}{0.780642in}}%
\pgfpathlineto{\pgfqpoint{0.238945in}{0.793568in}}%
\pgfpathlineto{\pgfqpoint{0.239027in}{0.806495in}}%
\pgfpathlineto{\pgfqpoint{0.240088in}{0.819421in}}%
\pgfpathlineto{\pgfqpoint{0.245448in}{0.832348in}}%
\pgfpathlineto{\pgfqpoint{0.251682in}{0.835718in}}%
\pgfpathlineto{\pgfqpoint{0.264951in}{0.836865in}}%
\pgfpathlineto{\pgfqpoint{0.278220in}{0.835083in}}%
\pgfpathlineto{\pgfqpoint{0.283166in}{0.832348in}}%
\pgfpathlineto{\pgfqpoint{0.289987in}{0.819421in}}%
\pgfpathlineto{\pgfqpoint{0.291452in}{0.806495in}}%
\pgfpathlineto{\pgfqpoint{0.291489in}{0.797312in}}%
\pgfpathlineto{\pgfqpoint{0.291502in}{0.793568in}}%
\pgfpathlineto{\pgfqpoint{0.291489in}{0.793380in}}%
\pgfpathlineto{\pgfqpoint{0.290116in}{0.780642in}}%
\pgfpathlineto{\pgfqpoint{0.282277in}{0.767715in}}%
\pgfpathlineto{\pgfqpoint{0.278220in}{0.765872in}}%
\pgfpathlineto{\pgfqpoint{0.264951in}{0.764290in}}%
\pgfpathlineto{\pgfqpoint{0.251682in}{0.765120in}}%
\pgfpathclose%
\pgfpathmoveto{\pgfqpoint{0.362209in}{0.780642in}}%
\pgfpathlineto{\pgfqpoint{0.360025in}{0.793568in}}%
\pgfpathlineto{\pgfqpoint{0.360040in}{0.806495in}}%
\pgfpathlineto{\pgfqpoint{0.362145in}{0.819421in}}%
\pgfpathlineto{\pgfqpoint{0.371102in}{0.831194in}}%
\pgfpathlineto{\pgfqpoint{0.377201in}{0.832348in}}%
\pgfpathlineto{\pgfqpoint{0.384371in}{0.833227in}}%
\pgfpathlineto{\pgfqpoint{0.388589in}{0.832348in}}%
\pgfpathlineto{\pgfqpoint{0.397640in}{0.829273in}}%
\pgfpathlineto{\pgfqpoint{0.404063in}{0.819421in}}%
\pgfpathlineto{\pgfqpoint{0.406281in}{0.806495in}}%
\pgfpathlineto{\pgfqpoint{0.406272in}{0.793568in}}%
\pgfpathlineto{\pgfqpoint{0.403908in}{0.780642in}}%
\pgfpathlineto{\pgfqpoint{0.397640in}{0.771794in}}%
\pgfpathlineto{\pgfqpoint{0.384371in}{0.768047in}}%
\pgfpathlineto{\pgfqpoint{0.371102in}{0.769961in}}%
\pgfpathclose%
\pgfpathmoveto{\pgfqpoint{0.259290in}{0.845274in}}%
\pgfpathlineto{\pgfqpoint{0.251682in}{0.846462in}}%
\pgfpathlineto{\pgfqpoint{0.241124in}{0.858201in}}%
\pgfpathlineto{\pgfqpoint{0.239330in}{0.871127in}}%
\pgfpathlineto{\pgfqpoint{0.238947in}{0.884054in}}%
\pgfpathlineto{\pgfqpoint{0.239343in}{0.896980in}}%
\pgfpathlineto{\pgfqpoint{0.241470in}{0.909907in}}%
\pgfpathlineto{\pgfqpoint{0.251682in}{0.919012in}}%
\pgfpathlineto{\pgfqpoint{0.264951in}{0.920534in}}%
\pgfpathlineto{\pgfqpoint{0.278220in}{0.920472in}}%
\pgfpathlineto{\pgfqpoint{0.291489in}{0.915381in}}%
\pgfpathlineto{\pgfqpoint{0.293158in}{0.909907in}}%
\pgfpathlineto{\pgfqpoint{0.294054in}{0.896980in}}%
\pgfpathlineto{\pgfqpoint{0.294181in}{0.884054in}}%
\pgfpathlineto{\pgfqpoint{0.293950in}{0.871127in}}%
\pgfpathlineto{\pgfqpoint{0.292978in}{0.858201in}}%
\pgfpathlineto{\pgfqpoint{0.291489in}{0.852409in}}%
\pgfpathlineto{\pgfqpoint{0.280429in}{0.845274in}}%
\pgfpathlineto{\pgfqpoint{0.278220in}{0.844919in}}%
\pgfpathlineto{\pgfqpoint{0.264951in}{0.844672in}}%
\pgfpathclose%
\pgfpathmoveto{\pgfqpoint{0.303125in}{0.858201in}}%
\pgfpathlineto{\pgfqpoint{0.300521in}{0.871127in}}%
\pgfpathlineto{\pgfqpoint{0.300148in}{0.884054in}}%
\pgfpathlineto{\pgfqpoint{0.301148in}{0.896980in}}%
\pgfpathlineto{\pgfqpoint{0.304758in}{0.908050in}}%
\pgfpathlineto{\pgfqpoint{0.306432in}{0.909907in}}%
\pgfpathlineto{\pgfqpoint{0.318027in}{0.914973in}}%
\pgfpathlineto{\pgfqpoint{0.331295in}{0.914575in}}%
\pgfpathlineto{\pgfqpoint{0.340766in}{0.909907in}}%
\pgfpathlineto{\pgfqpoint{0.344564in}{0.905255in}}%
\pgfpathlineto{\pgfqpoint{0.347196in}{0.896980in}}%
\pgfpathlineto{\pgfqpoint{0.348342in}{0.884054in}}%
\pgfpathlineto{\pgfqpoint{0.347881in}{0.871127in}}%
\pgfpathlineto{\pgfqpoint{0.344864in}{0.858201in}}%
\pgfpathlineto{\pgfqpoint{0.344564in}{0.857644in}}%
\pgfpathlineto{\pgfqpoint{0.331295in}{0.849367in}}%
\pgfpathlineto{\pgfqpoint{0.318027in}{0.848938in}}%
\pgfpathlineto{\pgfqpoint{0.304758in}{0.855301in}}%
\pgfpathclose%
\pgfpathmoveto{\pgfqpoint{0.244653in}{0.935760in}}%
\pgfpathlineto{\pgfqpoint{0.240920in}{0.948686in}}%
\pgfpathlineto{\pgfqpoint{0.240209in}{0.961613in}}%
\pgfpathlineto{\pgfqpoint{0.240688in}{0.974539in}}%
\pgfpathlineto{\pgfqpoint{0.243278in}{0.987466in}}%
\pgfpathlineto{\pgfqpoint{0.251682in}{0.996142in}}%
\pgfpathlineto{\pgfqpoint{0.264951in}{0.997911in}}%
\pgfpathlineto{\pgfqpoint{0.278220in}{0.995606in}}%
\pgfpathlineto{\pgfqpoint{0.286449in}{0.987466in}}%
\pgfpathlineto{\pgfqpoint{0.289694in}{0.974539in}}%
\pgfpathlineto{\pgfqpoint{0.290302in}{0.961613in}}%
\pgfpathlineto{\pgfqpoint{0.289341in}{0.948686in}}%
\pgfpathlineto{\pgfqpoint{0.284657in}{0.935760in}}%
\pgfpathlineto{\pgfqpoint{0.278220in}{0.930807in}}%
\pgfpathlineto{\pgfqpoint{0.264951in}{0.928696in}}%
\pgfpathlineto{\pgfqpoint{0.251682in}{0.930241in}}%
\pgfpathclose%
\pgfpathmoveto{\pgfqpoint{0.241938in}{1.013319in}}%
\pgfpathlineto{\pgfqpoint{0.238840in}{1.026245in}}%
\pgfpathlineto{\pgfqpoint{0.238413in}{1.034024in}}%
\pgfpathlineto{\pgfqpoint{0.238210in}{1.039172in}}%
\pgfpathlineto{\pgfqpoint{0.238212in}{1.052098in}}%
\pgfpathlineto{\pgfqpoint{0.238413in}{1.057064in}}%
\pgfpathlineto{\pgfqpoint{0.238881in}{1.065025in}}%
\pgfpathlineto{\pgfqpoint{0.242642in}{1.077952in}}%
\pgfpathlineto{\pgfqpoint{0.251682in}{1.082658in}}%
\pgfpathlineto{\pgfqpoint{0.264951in}{1.083824in}}%
\pgfpathlineto{\pgfqpoint{0.278220in}{1.083974in}}%
\pgfpathlineto{\pgfqpoint{0.291489in}{1.082127in}}%
\pgfpathlineto{\pgfqpoint{0.293896in}{1.077952in}}%
\pgfpathlineto{\pgfqpoint{0.294816in}{1.065025in}}%
\pgfpathlineto{\pgfqpoint{0.294952in}{1.052098in}}%
\pgfpathlineto{\pgfqpoint{0.294929in}{1.039172in}}%
\pgfpathlineto{\pgfqpoint{0.294732in}{1.026245in}}%
\pgfpathlineto{\pgfqpoint{0.293742in}{1.013319in}}%
\pgfpathlineto{\pgfqpoint{0.291489in}{1.008546in}}%
\pgfpathlineto{\pgfqpoint{0.278220in}{1.005856in}}%
\pgfpathlineto{\pgfqpoint{0.264951in}{1.005940in}}%
\pgfpathlineto{\pgfqpoint{0.251682in}{1.007194in}}%
\pgfpathclose%
\pgfpathmoveto{\pgfqpoint{0.315538in}{1.013319in}}%
\pgfpathlineto{\pgfqpoint{0.304758in}{1.021017in}}%
\pgfpathlineto{\pgfqpoint{0.302679in}{1.026245in}}%
\pgfpathlineto{\pgfqpoint{0.301026in}{1.039172in}}%
\pgfpathlineto{\pgfqpoint{0.301161in}{1.052098in}}%
\pgfpathlineto{\pgfqpoint{0.303312in}{1.065025in}}%
\pgfpathlineto{\pgfqpoint{0.304758in}{1.068230in}}%
\pgfpathlineto{\pgfqpoint{0.318027in}{1.076573in}}%
\pgfpathlineto{\pgfqpoint{0.331295in}{1.076127in}}%
\pgfpathlineto{\pgfqpoint{0.344564in}{1.065868in}}%
\pgfpathlineto{\pgfqpoint{0.344931in}{1.065025in}}%
\pgfpathlineto{\pgfqpoint{0.347332in}{1.052098in}}%
\pgfpathlineto{\pgfqpoint{0.347479in}{1.039172in}}%
\pgfpathlineto{\pgfqpoint{0.345609in}{1.026245in}}%
\pgfpathlineto{\pgfqpoint{0.344564in}{1.023535in}}%
\pgfpathlineto{\pgfqpoint{0.332302in}{1.013319in}}%
\pgfpathlineto{\pgfqpoint{0.331295in}{1.012956in}}%
\pgfpathlineto{\pgfqpoint{0.318027in}{1.012567in}}%
\pgfpathclose%
\pgfpathmoveto{\pgfqpoint{0.243138in}{1.103805in}}%
\pgfpathlineto{\pgfqpoint{0.240956in}{1.116731in}}%
\pgfpathlineto{\pgfqpoint{0.240690in}{1.129658in}}%
\pgfpathlineto{\pgfqpoint{0.241812in}{1.142584in}}%
\pgfpathlineto{\pgfqpoint{0.247480in}{1.155511in}}%
\pgfpathlineto{\pgfqpoint{0.251682in}{1.158288in}}%
\pgfpathlineto{\pgfqpoint{0.264951in}{1.160075in}}%
\pgfpathlineto{\pgfqpoint{0.278220in}{1.157814in}}%
\pgfpathlineto{\pgfqpoint{0.281713in}{1.155511in}}%
\pgfpathlineto{\pgfqpoint{0.288384in}{1.142584in}}%
\pgfpathlineto{\pgfqpoint{0.289811in}{1.129658in}}%
\pgfpathlineto{\pgfqpoint{0.289469in}{1.116731in}}%
\pgfpathlineto{\pgfqpoint{0.286746in}{1.103805in}}%
\pgfpathlineto{\pgfqpoint{0.278220in}{1.094271in}}%
\pgfpathlineto{\pgfqpoint{0.264951in}{1.091784in}}%
\pgfpathlineto{\pgfqpoint{0.251682in}{1.093751in}}%
\pgfpathclose%
\pgfpathmoveto{\pgfqpoint{0.259757in}{1.168437in}}%
\pgfpathlineto{\pgfqpoint{0.251682in}{1.169274in}}%
\pgfpathlineto{\pgfqpoint{0.239680in}{1.181364in}}%
\pgfpathlineto{\pgfqpoint{0.238413in}{1.193656in}}%
\pgfpathlineto{\pgfqpoint{0.238375in}{1.194290in}}%
\pgfpathlineto{\pgfqpoint{0.238152in}{1.207217in}}%
\pgfpathlineto{\pgfqpoint{0.238413in}{1.218837in}}%
\pgfpathlineto{\pgfqpoint{0.238449in}{1.220143in}}%
\pgfpathlineto{\pgfqpoint{0.239929in}{1.233070in}}%
\pgfpathlineto{\pgfqpoint{0.251682in}{1.244754in}}%
\pgfpathlineto{\pgfqpoint{0.263997in}{1.245996in}}%
\pgfpathlineto{\pgfqpoint{0.264951in}{1.246064in}}%
\pgfpathlineto{\pgfqpoint{0.278220in}{1.246149in}}%
\pgfpathlineto{\pgfqpoint{0.280689in}{1.245996in}}%
\pgfpathlineto{\pgfqpoint{0.291489in}{1.243339in}}%
\pgfpathlineto{\pgfqpoint{0.294400in}{1.233070in}}%
\pgfpathlineto{\pgfqpoint{0.294849in}{1.220143in}}%
\pgfpathlineto{\pgfqpoint{0.294955in}{1.207217in}}%
\pgfpathlineto{\pgfqpoint{0.294921in}{1.194290in}}%
\pgfpathlineto{\pgfqpoint{0.294639in}{1.181364in}}%
\pgfpathlineto{\pgfqpoint{0.291489in}{1.169832in}}%
\pgfpathlineto{\pgfqpoint{0.286279in}{1.168437in}}%
\pgfpathlineto{\pgfqpoint{0.278220in}{1.167907in}}%
\pgfpathlineto{\pgfqpoint{0.264951in}{1.168059in}}%
\pgfpathclose%
\pgfpathmoveto{\pgfqpoint{0.306466in}{1.181364in}}%
\pgfpathlineto{\pgfqpoint{0.304758in}{1.183704in}}%
\pgfpathlineto{\pgfqpoint{0.301722in}{1.194290in}}%
\pgfpathlineto{\pgfqpoint{0.300908in}{1.207217in}}%
\pgfpathlineto{\pgfqpoint{0.301659in}{1.220143in}}%
\pgfpathlineto{\pgfqpoint{0.304758in}{1.231123in}}%
\pgfpathlineto{\pgfqpoint{0.306160in}{1.233070in}}%
\pgfpathlineto{\pgfqpoint{0.318027in}{1.239124in}}%
\pgfpathlineto{\pgfqpoint{0.331295in}{1.238717in}}%
\pgfpathlineto{\pgfqpoint{0.341260in}{1.233070in}}%
\pgfpathlineto{\pgfqpoint{0.344564in}{1.228085in}}%
\pgfpathlineto{\pgfqpoint{0.346757in}{1.220143in}}%
\pgfpathlineto{\pgfqpoint{0.347617in}{1.207217in}}%
\pgfpathlineto{\pgfqpoint{0.346698in}{1.194290in}}%
\pgfpathlineto{\pgfqpoint{0.344564in}{1.186680in}}%
\pgfpathlineto{\pgfqpoint{0.340993in}{1.181364in}}%
\pgfpathlineto{\pgfqpoint{0.331295in}{1.175885in}}%
\pgfpathlineto{\pgfqpoint{0.318027in}{1.175481in}}%
\pgfpathclose%
\pgfpathmoveto{\pgfqpoint{0.246731in}{1.258923in}}%
\pgfpathlineto{\pgfqpoint{0.241368in}{1.271849in}}%
\pgfpathlineto{\pgfqpoint{0.240269in}{1.284776in}}%
\pgfpathlineto{\pgfqpoint{0.240442in}{1.297702in}}%
\pgfpathlineto{\pgfqpoint{0.242296in}{1.310629in}}%
\pgfpathlineto{\pgfqpoint{0.251682in}{1.321740in}}%
\pgfpathlineto{\pgfqpoint{0.264951in}{1.323414in}}%
\pgfpathlineto{\pgfqpoint{0.278220in}{1.321126in}}%
\pgfpathlineto{\pgfqpoint{0.287557in}{1.310629in}}%
\pgfpathlineto{\pgfqpoint{0.289979in}{1.297702in}}%
\pgfpathlineto{\pgfqpoint{0.290232in}{1.284776in}}%
\pgfpathlineto{\pgfqpoint{0.288824in}{1.271849in}}%
\pgfpathlineto{\pgfqpoint{0.282416in}{1.258923in}}%
\pgfpathlineto{\pgfqpoint{0.278220in}{1.256164in}}%
\pgfpathlineto{\pgfqpoint{0.264951in}{1.254000in}}%
\pgfpathlineto{\pgfqpoint{0.251682in}{1.255661in}}%
\pgfpathclose%
\pgfpathmoveto{\pgfqpoint{0.244754in}{1.336482in}}%
\pgfpathlineto{\pgfqpoint{0.239879in}{1.349408in}}%
\pgfpathlineto{\pgfqpoint{0.239012in}{1.362335in}}%
\pgfpathlineto{\pgfqpoint{0.239065in}{1.375261in}}%
\pgfpathlineto{\pgfqpoint{0.240051in}{1.388188in}}%
\pgfpathlineto{\pgfqpoint{0.244673in}{1.401114in}}%
\pgfpathlineto{\pgfqpoint{0.251682in}{1.405457in}}%
\pgfpathlineto{\pgfqpoint{0.264951in}{1.407165in}}%
\pgfpathlineto{\pgfqpoint{0.278220in}{1.406918in}}%
\pgfpathlineto{\pgfqpoint{0.290792in}{1.401114in}}%
\pgfpathlineto{\pgfqpoint{0.291489in}{1.399671in}}%
\pgfpathlineto{\pgfqpoint{0.293562in}{1.388188in}}%
\pgfpathlineto{\pgfqpoint{0.294098in}{1.375261in}}%
\pgfpathlineto{\pgfqpoint{0.294173in}{1.362335in}}%
\pgfpathlineto{\pgfqpoint{0.293840in}{1.349408in}}%
\pgfpathlineto{\pgfqpoint{0.291576in}{1.336482in}}%
\pgfpathlineto{\pgfqpoint{0.291489in}{1.336324in}}%
\pgfpathlineto{\pgfqpoint{0.278220in}{1.331404in}}%
\pgfpathlineto{\pgfqpoint{0.264951in}{1.331344in}}%
\pgfpathlineto{\pgfqpoint{0.251682in}{1.332818in}}%
\pgfpathclose%
\pgfpathmoveto{\pgfqpoint{0.302317in}{1.349408in}}%
\pgfpathlineto{\pgfqpoint{0.300377in}{1.362335in}}%
\pgfpathlineto{\pgfqpoint{0.300199in}{1.375261in}}%
\pgfpathlineto{\pgfqpoint{0.301526in}{1.388188in}}%
\pgfpathlineto{\pgfqpoint{0.304758in}{1.396424in}}%
\pgfpathlineto{\pgfqpoint{0.311274in}{1.401114in}}%
\pgfpathlineto{\pgfqpoint{0.318027in}{1.403156in}}%
\pgfpathlineto{\pgfqpoint{0.331295in}{1.402757in}}%
\pgfpathlineto{\pgfqpoint{0.335845in}{1.401114in}}%
\pgfpathlineto{\pgfqpoint{0.344564in}{1.393785in}}%
\pgfpathlineto{\pgfqpoint{0.346695in}{1.388188in}}%
\pgfpathlineto{\pgfqpoint{0.348271in}{1.375261in}}%
\pgfpathlineto{\pgfqpoint{0.348081in}{1.362335in}}%
\pgfpathlineto{\pgfqpoint{0.345869in}{1.349408in}}%
\pgfpathlineto{\pgfqpoint{0.344564in}{1.346447in}}%
\pgfpathlineto{\pgfqpoint{0.331295in}{1.337224in}}%
\pgfpathlineto{\pgfqpoint{0.318027in}{1.336776in}}%
\pgfpathlineto{\pgfqpoint{0.304758in}{1.344061in}}%
\pgfpathclose%
\pgfpathmoveto{\pgfqpoint{0.304648in}{1.414041in}}%
\pgfpathlineto{\pgfqpoint{0.298100in}{1.426967in}}%
\pgfpathlineto{\pgfqpoint{0.297339in}{1.439894in}}%
\pgfpathlineto{\pgfqpoint{0.297295in}{1.452820in}}%
\pgfpathlineto{\pgfqpoint{0.297784in}{1.465747in}}%
\pgfpathlineto{\pgfqpoint{0.299767in}{1.478673in}}%
\pgfpathlineto{\pgfqpoint{0.304758in}{1.485642in}}%
\pgfpathlineto{\pgfqpoint{0.318027in}{1.488872in}}%
\pgfpathlineto{\pgfqpoint{0.331295in}{1.489325in}}%
\pgfpathlineto{\pgfqpoint{0.344564in}{1.488260in}}%
\pgfpathlineto{\pgfqpoint{0.352340in}{1.478673in}}%
\pgfpathlineto{\pgfqpoint{0.353318in}{1.465747in}}%
\pgfpathlineto{\pgfqpoint{0.353586in}{1.452820in}}%
\pgfpathlineto{\pgfqpoint{0.353664in}{1.439894in}}%
\pgfpathlineto{\pgfqpoint{0.353590in}{1.426967in}}%
\pgfpathlineto{\pgfqpoint{0.352317in}{1.414041in}}%
\pgfpathlineto{\pgfqpoint{0.344564in}{1.411351in}}%
\pgfpathlineto{\pgfqpoint{0.331295in}{1.411232in}}%
\pgfpathlineto{\pgfqpoint{0.318027in}{1.411680in}}%
\pgfpathlineto{\pgfqpoint{0.304758in}{1.413988in}}%
\pgfpathclose%
\pgfpathmoveto{\pgfqpoint{0.241266in}{1.426967in}}%
\pgfpathlineto{\pgfqpoint{0.239313in}{1.439894in}}%
\pgfpathlineto{\pgfqpoint{0.238891in}{1.452820in}}%
\pgfpathlineto{\pgfqpoint{0.239268in}{1.465747in}}%
\pgfpathlineto{\pgfqpoint{0.241532in}{1.478673in}}%
\pgfpathlineto{\pgfqpoint{0.251682in}{1.486676in}}%
\pgfpathlineto{\pgfqpoint{0.264951in}{1.487533in}}%
\pgfpathlineto{\pgfqpoint{0.278220in}{1.485902in}}%
\pgfpathlineto{\pgfqpoint{0.287578in}{1.478673in}}%
\pgfpathlineto{\pgfqpoint{0.290970in}{1.465747in}}%
\pgfpathlineto{\pgfqpoint{0.291489in}{1.455750in}}%
\pgfpathlineto{\pgfqpoint{0.291599in}{1.452820in}}%
\pgfpathlineto{\pgfqpoint{0.291489in}{1.449629in}}%
\pgfpathlineto{\pgfqpoint{0.291061in}{1.439894in}}%
\pgfpathlineto{\pgfqpoint{0.288391in}{1.426967in}}%
\pgfpathlineto{\pgfqpoint{0.278220in}{1.416956in}}%
\pgfpathlineto{\pgfqpoint{0.264951in}{1.415038in}}%
\pgfpathlineto{\pgfqpoint{0.251682in}{1.416272in}}%
\pgfpathclose%
\pgfpathmoveto{\pgfqpoint{0.364544in}{1.426967in}}%
\pgfpathlineto{\pgfqpoint{0.360588in}{1.439894in}}%
\pgfpathlineto{\pgfqpoint{0.359831in}{1.452820in}}%
\pgfpathlineto{\pgfqpoint{0.360863in}{1.465747in}}%
\pgfpathlineto{\pgfqpoint{0.366380in}{1.478673in}}%
\pgfpathlineto{\pgfqpoint{0.371102in}{1.482009in}}%
\pgfpathlineto{\pgfqpoint{0.384371in}{1.483709in}}%
\pgfpathlineto{\pgfqpoint{0.397640in}{1.480383in}}%
\pgfpathlineto{\pgfqpoint{0.399633in}{1.478673in}}%
\pgfpathlineto{\pgfqpoint{0.405352in}{1.465747in}}%
\pgfpathlineto{\pgfqpoint{0.406493in}{1.452820in}}%
\pgfpathlineto{\pgfqpoint{0.405700in}{1.439894in}}%
\pgfpathlineto{\pgfqpoint{0.401604in}{1.426967in}}%
\pgfpathlineto{\pgfqpoint{0.397640in}{1.422635in}}%
\pgfpathlineto{\pgfqpoint{0.384371in}{1.418953in}}%
\pgfpathlineto{\pgfqpoint{0.371102in}{1.420925in}}%
\pgfpathclose%
\pgfusepath{fill}%
\end{pgfscope}%
\begin{pgfscope}%
\pgfpathrectangle{\pgfqpoint{0.211875in}{0.211875in}}{\pgfqpoint{1.313625in}{1.279725in}}%
\pgfusepath{clip}%
\pgfsetbuttcap%
\pgfsetroundjoin%
\definecolor{currentfill}{rgb}{0.796501,0.105066,0.310630}%
\pgfsetfillcolor{currentfill}%
\pgfsetlinewidth{0.000000pt}%
\definecolor{currentstroke}{rgb}{0.000000,0.000000,0.000000}%
\pgfsetstrokecolor{currentstroke}%
\pgfsetdash{}{0pt}%
\pgfpathmoveto{\pgfqpoint{0.225144in}{0.289365in}}%
\pgfpathlineto{\pgfqpoint{0.225196in}{0.289434in}}%
\pgfpathlineto{\pgfqpoint{0.229521in}{0.302361in}}%
\pgfpathlineto{\pgfqpoint{0.230488in}{0.315287in}}%
\pgfpathlineto{\pgfqpoint{0.229799in}{0.328214in}}%
\pgfpathlineto{\pgfqpoint{0.225442in}{0.341140in}}%
\pgfpathlineto{\pgfqpoint{0.225144in}{0.341495in}}%
\pgfpathlineto{\pgfqpoint{0.211875in}{0.346520in}}%
\pgfpathlineto{\pgfqpoint{0.211875in}{0.341140in}}%
\pgfpathlineto{\pgfqpoint{0.211875in}{0.336673in}}%
\pgfpathlineto{\pgfqpoint{0.220769in}{0.328214in}}%
\pgfpathlineto{\pgfqpoint{0.223868in}{0.315287in}}%
\pgfpathlineto{\pgfqpoint{0.221110in}{0.302361in}}%
\pgfpathlineto{\pgfqpoint{0.211875in}{0.292626in}}%
\pgfpathlineto{\pgfqpoint{0.211875in}{0.289434in}}%
\pgfpathlineto{\pgfqpoint{0.211875in}{0.283216in}}%
\pgfpathclose%
\pgfusepath{fill}%
\end{pgfscope}%
\begin{pgfscope}%
\pgfpathrectangle{\pgfqpoint{0.211875in}{0.211875in}}{\pgfqpoint{1.313625in}{1.279725in}}%
\pgfusepath{clip}%
\pgfsetbuttcap%
\pgfsetroundjoin%
\definecolor{currentfill}{rgb}{0.796501,0.105066,0.310630}%
\pgfsetfillcolor{currentfill}%
\pgfsetlinewidth{0.000000pt}%
\definecolor{currentstroke}{rgb}{0.000000,0.000000,0.000000}%
\pgfsetstrokecolor{currentstroke}%
\pgfsetdash{}{0pt}%
\pgfpathmoveto{\pgfqpoint{0.304758in}{0.286526in}}%
\pgfpathlineto{\pgfqpoint{0.318027in}{0.279664in}}%
\pgfpathlineto{\pgfqpoint{0.331295in}{0.279179in}}%
\pgfpathlineto{\pgfqpoint{0.344564in}{0.283707in}}%
\pgfpathlineto{\pgfqpoint{0.348255in}{0.289434in}}%
\pgfpathlineto{\pgfqpoint{0.350720in}{0.302361in}}%
\pgfpathlineto{\pgfqpoint{0.351365in}{0.315287in}}%
\pgfpathlineto{\pgfqpoint{0.351216in}{0.328214in}}%
\pgfpathlineto{\pgfqpoint{0.349576in}{0.341140in}}%
\pgfpathlineto{\pgfqpoint{0.344564in}{0.348162in}}%
\pgfpathlineto{\pgfqpoint{0.331295in}{0.350741in}}%
\pgfpathlineto{\pgfqpoint{0.318027in}{0.350255in}}%
\pgfpathlineto{\pgfqpoint{0.304758in}{0.345178in}}%
\pgfpathlineto{\pgfqpoint{0.302213in}{0.341140in}}%
\pgfpathlineto{\pgfqpoint{0.299758in}{0.328214in}}%
\pgfpathlineto{\pgfqpoint{0.299412in}{0.315287in}}%
\pgfpathlineto{\pgfqpoint{0.300142in}{0.302361in}}%
\pgfpathlineto{\pgfqpoint{0.303070in}{0.289434in}}%
\pgfpathclose%
\pgfpathmoveto{\pgfqpoint{0.316293in}{0.289434in}}%
\pgfpathlineto{\pgfqpoint{0.307212in}{0.302361in}}%
\pgfpathlineto{\pgfqpoint{0.305271in}{0.315287in}}%
\pgfpathlineto{\pgfqpoint{0.307257in}{0.328214in}}%
\pgfpathlineto{\pgfqpoint{0.317761in}{0.341140in}}%
\pgfpathlineto{\pgfqpoint{0.318027in}{0.341282in}}%
\pgfpathlineto{\pgfqpoint{0.331295in}{0.341820in}}%
\pgfpathlineto{\pgfqpoint{0.332794in}{0.341140in}}%
\pgfpathlineto{\pgfqpoint{0.343924in}{0.328214in}}%
\pgfpathlineto{\pgfqpoint{0.344564in}{0.324396in}}%
\pgfpathlineto{\pgfqpoint{0.345395in}{0.315287in}}%
\pgfpathlineto{\pgfqpoint{0.344564in}{0.306578in}}%
\pgfpathlineto{\pgfqpoint{0.343873in}{0.302361in}}%
\pgfpathlineto{\pgfqpoint{0.334279in}{0.289434in}}%
\pgfpathlineto{\pgfqpoint{0.331295in}{0.287828in}}%
\pgfpathlineto{\pgfqpoint{0.318027in}{0.288360in}}%
\pgfpathclose%
\pgfusepath{fill}%
\end{pgfscope}%
\begin{pgfscope}%
\pgfpathrectangle{\pgfqpoint{0.211875in}{0.211875in}}{\pgfqpoint{1.313625in}{1.279725in}}%
\pgfusepath{clip}%
\pgfsetbuttcap%
\pgfsetroundjoin%
\definecolor{currentfill}{rgb}{0.796501,0.105066,0.310630}%
\pgfsetfillcolor{currentfill}%
\pgfsetlinewidth{0.000000pt}%
\definecolor{currentstroke}{rgb}{0.000000,0.000000,0.000000}%
\pgfsetstrokecolor{currentstroke}%
\pgfsetdash{}{0pt}%
\pgfpathmoveto{\pgfqpoint{0.251682in}{0.362088in}}%
\pgfpathlineto{\pgfqpoint{0.264951in}{0.359616in}}%
\pgfpathlineto{\pgfqpoint{0.278220in}{0.360410in}}%
\pgfpathlineto{\pgfqpoint{0.288484in}{0.366993in}}%
\pgfpathlineto{\pgfqpoint{0.291489in}{0.375561in}}%
\pgfpathlineto{\pgfqpoint{0.292203in}{0.379920in}}%
\pgfpathlineto{\pgfqpoint{0.293020in}{0.392846in}}%
\pgfpathlineto{\pgfqpoint{0.293102in}{0.405773in}}%
\pgfpathlineto{\pgfqpoint{0.292340in}{0.418699in}}%
\pgfpathlineto{\pgfqpoint{0.291489in}{0.422900in}}%
\pgfpathlineto{\pgfqpoint{0.281778in}{0.431626in}}%
\pgfpathlineto{\pgfqpoint{0.278220in}{0.432297in}}%
\pgfpathlineto{\pgfqpoint{0.264951in}{0.432559in}}%
\pgfpathlineto{\pgfqpoint{0.257019in}{0.431626in}}%
\pgfpathlineto{\pgfqpoint{0.251682in}{0.430680in}}%
\pgfpathlineto{\pgfqpoint{0.241733in}{0.418699in}}%
\pgfpathlineto{\pgfqpoint{0.240207in}{0.405773in}}%
\pgfpathlineto{\pgfqpoint{0.240167in}{0.392846in}}%
\pgfpathlineto{\pgfqpoint{0.241283in}{0.379920in}}%
\pgfpathlineto{\pgfqpoint{0.245725in}{0.366993in}}%
\pgfpathclose%
\pgfpathmoveto{\pgfqpoint{0.248723in}{0.379920in}}%
\pgfpathlineto{\pgfqpoint{0.246439in}{0.392846in}}%
\pgfpathlineto{\pgfqpoint{0.247032in}{0.405773in}}%
\pgfpathlineto{\pgfqpoint{0.251682in}{0.418378in}}%
\pgfpathlineto{\pgfqpoint{0.252063in}{0.418699in}}%
\pgfpathlineto{\pgfqpoint{0.264951in}{0.423849in}}%
\pgfpathlineto{\pgfqpoint{0.278220in}{0.421157in}}%
\pgfpathlineto{\pgfqpoint{0.280788in}{0.418699in}}%
\pgfpathlineto{\pgfqpoint{0.285729in}{0.405773in}}%
\pgfpathlineto{\pgfqpoint{0.286288in}{0.392846in}}%
\pgfpathlineto{\pgfqpoint{0.283938in}{0.379920in}}%
\pgfpathlineto{\pgfqpoint{0.278220in}{0.371208in}}%
\pgfpathlineto{\pgfqpoint{0.264951in}{0.367590in}}%
\pgfpathlineto{\pgfqpoint{0.251682in}{0.374429in}}%
\pgfpathclose%
\pgfusepath{fill}%
\end{pgfscope}%
\begin{pgfscope}%
\pgfpathrectangle{\pgfqpoint{0.211875in}{0.211875in}}{\pgfqpoint{1.313625in}{1.279725in}}%
\pgfusepath{clip}%
\pgfsetbuttcap%
\pgfsetroundjoin%
\definecolor{currentfill}{rgb}{0.796501,0.105066,0.310630}%
\pgfsetfillcolor{currentfill}%
\pgfsetlinewidth{0.000000pt}%
\definecolor{currentstroke}{rgb}{0.000000,0.000000,0.000000}%
\pgfsetstrokecolor{currentstroke}%
\pgfsetdash{}{0pt}%
\pgfpathmoveto{\pgfqpoint{0.888591in}{0.355334in}}%
\pgfpathlineto{\pgfqpoint{0.901860in}{0.354736in}}%
\pgfpathlineto{\pgfqpoint{0.915129in}{0.354777in}}%
\pgfpathlineto{\pgfqpoint{0.928398in}{0.355104in}}%
\pgfpathlineto{\pgfqpoint{0.941667in}{0.361918in}}%
\pgfpathlineto{\pgfqpoint{0.942327in}{0.366993in}}%
\pgfpathlineto{\pgfqpoint{0.942687in}{0.379920in}}%
\pgfpathlineto{\pgfqpoint{0.942607in}{0.392846in}}%
\pgfpathlineto{\pgfqpoint{0.942185in}{0.405773in}}%
\pgfpathlineto{\pgfqpoint{0.941667in}{0.412126in}}%
\pgfpathlineto{\pgfqpoint{0.940791in}{0.418699in}}%
\pgfpathlineto{\pgfqpoint{0.929305in}{0.431626in}}%
\pgfpathlineto{\pgfqpoint{0.928398in}{0.431856in}}%
\pgfpathlineto{\pgfqpoint{0.915129in}{0.432940in}}%
\pgfpathlineto{\pgfqpoint{0.901860in}{0.432389in}}%
\pgfpathlineto{\pgfqpoint{0.898155in}{0.431626in}}%
\pgfpathlineto{\pgfqpoint{0.888591in}{0.423970in}}%
\pgfpathlineto{\pgfqpoint{0.887232in}{0.418699in}}%
\pgfpathlineto{\pgfqpoint{0.886107in}{0.405773in}}%
\pgfpathlineto{\pgfqpoint{0.885720in}{0.392846in}}%
\pgfpathlineto{\pgfqpoint{0.885567in}{0.379920in}}%
\pgfpathlineto{\pgfqpoint{0.885580in}{0.366993in}}%
\pgfpathclose%
\pgfpathmoveto{\pgfqpoint{0.899700in}{0.366993in}}%
\pgfpathlineto{\pgfqpoint{0.893236in}{0.379920in}}%
\pgfpathlineto{\pgfqpoint{0.892085in}{0.392846in}}%
\pgfpathlineto{\pgfqpoint{0.893160in}{0.405773in}}%
\pgfpathlineto{\pgfqpoint{0.898283in}{0.418699in}}%
\pgfpathlineto{\pgfqpoint{0.901860in}{0.422011in}}%
\pgfpathlineto{\pgfqpoint{0.915129in}{0.424765in}}%
\pgfpathlineto{\pgfqpoint{0.928398in}{0.420707in}}%
\pgfpathlineto{\pgfqpoint{0.930286in}{0.418699in}}%
\pgfpathlineto{\pgfqpoint{0.935227in}{0.405773in}}%
\pgfpathlineto{\pgfqpoint{0.936266in}{0.392846in}}%
\pgfpathlineto{\pgfqpoint{0.935125in}{0.379920in}}%
\pgfpathlineto{\pgfqpoint{0.928866in}{0.366993in}}%
\pgfpathlineto{\pgfqpoint{0.928398in}{0.366595in}}%
\pgfpathlineto{\pgfqpoint{0.915129in}{0.363109in}}%
\pgfpathlineto{\pgfqpoint{0.901860in}{0.365414in}}%
\pgfpathclose%
\pgfusepath{fill}%
\end{pgfscope}%
\begin{pgfscope}%
\pgfpathrectangle{\pgfqpoint{0.211875in}{0.211875in}}{\pgfqpoint{1.313625in}{1.279725in}}%
\pgfusepath{clip}%
\pgfsetbuttcap%
\pgfsetroundjoin%
\definecolor{currentfill}{rgb}{0.796501,0.105066,0.310630}%
\pgfsetfillcolor{currentfill}%
\pgfsetlinewidth{0.000000pt}%
\definecolor{currentstroke}{rgb}{0.000000,0.000000,0.000000}%
\pgfsetstrokecolor{currentstroke}%
\pgfsetdash{}{0pt}%
\pgfpathmoveto{\pgfqpoint{1.008011in}{0.358793in}}%
\pgfpathlineto{\pgfqpoint{1.021280in}{0.355784in}}%
\pgfpathlineto{\pgfqpoint{1.034549in}{0.355611in}}%
\pgfpathlineto{\pgfqpoint{1.047818in}{0.356317in}}%
\pgfpathlineto{\pgfqpoint{1.059030in}{0.366993in}}%
\pgfpathlineto{\pgfqpoint{1.059884in}{0.379920in}}%
\pgfpathlineto{\pgfqpoint{1.059898in}{0.392846in}}%
\pgfpathlineto{\pgfqpoint{1.059407in}{0.405773in}}%
\pgfpathlineto{\pgfqpoint{1.057728in}{0.418699in}}%
\pgfpathlineto{\pgfqpoint{1.047818in}{0.430441in}}%
\pgfpathlineto{\pgfqpoint{1.040016in}{0.431626in}}%
\pgfpathlineto{\pgfqpoint{1.034549in}{0.432157in}}%
\pgfpathlineto{\pgfqpoint{1.022010in}{0.431626in}}%
\pgfpathlineto{\pgfqpoint{1.021280in}{0.431586in}}%
\pgfpathlineto{\pgfqpoint{1.008011in}{0.423849in}}%
\pgfpathlineto{\pgfqpoint{1.006251in}{0.418699in}}%
\pgfpathlineto{\pgfqpoint{1.004774in}{0.405773in}}%
\pgfpathlineto{\pgfqpoint{1.004331in}{0.392846in}}%
\pgfpathlineto{\pgfqpoint{1.004310in}{0.379920in}}%
\pgfpathlineto{\pgfqpoint{1.004946in}{0.366993in}}%
\pgfpathclose%
\pgfpathmoveto{\pgfqpoint{1.019519in}{0.366993in}}%
\pgfpathlineto{\pgfqpoint{1.012005in}{0.379920in}}%
\pgfpathlineto{\pgfqpoint{1.010642in}{0.392846in}}%
\pgfpathlineto{\pgfqpoint{1.011845in}{0.405773in}}%
\pgfpathlineto{\pgfqpoint{1.017661in}{0.418699in}}%
\pgfpathlineto{\pgfqpoint{1.021280in}{0.421719in}}%
\pgfpathlineto{\pgfqpoint{1.034549in}{0.423868in}}%
\pgfpathlineto{\pgfqpoint{1.047260in}{0.418699in}}%
\pgfpathlineto{\pgfqpoint{1.047818in}{0.418221in}}%
\pgfpathlineto{\pgfqpoint{1.052629in}{0.405773in}}%
\pgfpathlineto{\pgfqpoint{1.053683in}{0.392846in}}%
\pgfpathlineto{\pgfqpoint{1.052476in}{0.379920in}}%
\pgfpathlineto{\pgfqpoint{1.047818in}{0.369434in}}%
\pgfpathlineto{\pgfqpoint{1.043959in}{0.366993in}}%
\pgfpathlineto{\pgfqpoint{1.034549in}{0.363999in}}%
\pgfpathlineto{\pgfqpoint{1.021280in}{0.365836in}}%
\pgfpathclose%
\pgfusepath{fill}%
\end{pgfscope}%
\begin{pgfscope}%
\pgfpathrectangle{\pgfqpoint{0.211875in}{0.211875in}}{\pgfqpoint{1.313625in}{1.279725in}}%
\pgfusepath{clip}%
\pgfsetbuttcap%
\pgfsetroundjoin%
\definecolor{currentfill}{rgb}{0.796501,0.105066,0.310630}%
\pgfsetfillcolor{currentfill}%
\pgfsetlinewidth{0.000000pt}%
\definecolor{currentstroke}{rgb}{0.000000,0.000000,0.000000}%
\pgfsetstrokecolor{currentstroke}%
\pgfsetdash{}{0pt}%
\pgfpathmoveto{\pgfqpoint{1.127432in}{0.358907in}}%
\pgfpathlineto{\pgfqpoint{1.140701in}{0.356087in}}%
\pgfpathlineto{\pgfqpoint{1.153970in}{0.355897in}}%
\pgfpathlineto{\pgfqpoint{1.167239in}{0.356857in}}%
\pgfpathlineto{\pgfqpoint{1.176661in}{0.366993in}}%
\pgfpathlineto{\pgfqpoint{1.177595in}{0.379920in}}%
\pgfpathlineto{\pgfqpoint{1.177630in}{0.392846in}}%
\pgfpathlineto{\pgfqpoint{1.177142in}{0.405773in}}%
\pgfpathlineto{\pgfqpoint{1.175442in}{0.418699in}}%
\pgfpathlineto{\pgfqpoint{1.167239in}{0.429599in}}%
\pgfpathlineto{\pgfqpoint{1.156179in}{0.431626in}}%
\pgfpathlineto{\pgfqpoint{1.153970in}{0.431865in}}%
\pgfpathlineto{\pgfqpoint{1.146250in}{0.431626in}}%
\pgfpathlineto{\pgfqpoint{1.140701in}{0.431419in}}%
\pgfpathlineto{\pgfqpoint{1.127432in}{0.425230in}}%
\pgfpathlineto{\pgfqpoint{1.124663in}{0.418699in}}%
\pgfpathlineto{\pgfqpoint{1.123043in}{0.405773in}}%
\pgfpathlineto{\pgfqpoint{1.122578in}{0.392846in}}%
\pgfpathlineto{\pgfqpoint{1.122611in}{0.379920in}}%
\pgfpathlineto{\pgfqpoint{1.123501in}{0.366993in}}%
\pgfpathclose%
\pgfpathmoveto{\pgfqpoint{1.138414in}{0.366993in}}%
\pgfpathlineto{\pgfqpoint{1.130158in}{0.379920in}}%
\pgfpathlineto{\pgfqpoint{1.128645in}{0.392846in}}%
\pgfpathlineto{\pgfqpoint{1.129953in}{0.405773in}}%
\pgfpathlineto{\pgfqpoint{1.136300in}{0.418699in}}%
\pgfpathlineto{\pgfqpoint{1.140701in}{0.422006in}}%
\pgfpathlineto{\pgfqpoint{1.153970in}{0.423446in}}%
\pgfpathlineto{\pgfqpoint{1.164082in}{0.418699in}}%
\pgfpathlineto{\pgfqpoint{1.167239in}{0.415333in}}%
\pgfpathlineto{\pgfqpoint{1.170534in}{0.405773in}}%
\pgfpathlineto{\pgfqpoint{1.171573in}{0.392846in}}%
\pgfpathlineto{\pgfqpoint{1.170371in}{0.379920in}}%
\pgfpathlineto{\pgfqpoint{1.167239in}{0.371917in}}%
\pgfpathlineto{\pgfqpoint{1.161048in}{0.366993in}}%
\pgfpathlineto{\pgfqpoint{1.153970in}{0.364400in}}%
\pgfpathlineto{\pgfqpoint{1.140701in}{0.365642in}}%
\pgfpathclose%
\pgfusepath{fill}%
\end{pgfscope}%
\begin{pgfscope}%
\pgfpathrectangle{\pgfqpoint{0.211875in}{0.211875in}}{\pgfqpoint{1.313625in}{1.279725in}}%
\pgfusepath{clip}%
\pgfsetbuttcap%
\pgfsetroundjoin%
\definecolor{currentfill}{rgb}{0.796501,0.105066,0.310630}%
\pgfsetfillcolor{currentfill}%
\pgfsetlinewidth{0.000000pt}%
\definecolor{currentstroke}{rgb}{0.000000,0.000000,0.000000}%
\pgfsetstrokecolor{currentstroke}%
\pgfsetdash{}{0pt}%
\pgfpathmoveto{\pgfqpoint{1.246852in}{0.357623in}}%
\pgfpathlineto{\pgfqpoint{1.260121in}{0.355756in}}%
\pgfpathlineto{\pgfqpoint{1.273390in}{0.355618in}}%
\pgfpathlineto{\pgfqpoint{1.286659in}{0.356498in}}%
\pgfpathlineto{\pgfqpoint{1.295194in}{0.366993in}}%
\pgfpathlineto{\pgfqpoint{1.295845in}{0.379920in}}%
\pgfpathlineto{\pgfqpoint{1.295824in}{0.392846in}}%
\pgfpathlineto{\pgfqpoint{1.295371in}{0.405773in}}%
\pgfpathlineto{\pgfqpoint{1.293858in}{0.418699in}}%
\pgfpathlineto{\pgfqpoint{1.286659in}{0.429522in}}%
\pgfpathlineto{\pgfqpoint{1.277161in}{0.431626in}}%
\pgfpathlineto{\pgfqpoint{1.273390in}{0.432079in}}%
\pgfpathlineto{\pgfqpoint{1.260121in}{0.431832in}}%
\pgfpathlineto{\pgfqpoint{1.258827in}{0.431626in}}%
\pgfpathlineto{\pgfqpoint{1.246852in}{0.427365in}}%
\pgfpathlineto{\pgfqpoint{1.242470in}{0.418699in}}%
\pgfpathlineto{\pgfqpoint{1.240910in}{0.405773in}}%
\pgfpathlineto{\pgfqpoint{1.240454in}{0.392846in}}%
\pgfpathlineto{\pgfqpoint{1.240467in}{0.379920in}}%
\pgfpathlineto{\pgfqpoint{1.241261in}{0.366993in}}%
\pgfpathclose%
\pgfpathmoveto{\pgfqpoint{1.256213in}{0.366993in}}%
\pgfpathlineto{\pgfqpoint{1.247559in}{0.379920in}}%
\pgfpathlineto{\pgfqpoint{1.246852in}{0.385109in}}%
\pgfpathlineto{\pgfqpoint{1.246229in}{0.392846in}}%
\pgfpathlineto{\pgfqpoint{1.246852in}{0.401332in}}%
\pgfpathlineto{\pgfqpoint{1.247357in}{0.405773in}}%
\pgfpathlineto{\pgfqpoint{1.254053in}{0.418699in}}%
\pgfpathlineto{\pgfqpoint{1.260121in}{0.422797in}}%
\pgfpathlineto{\pgfqpoint{1.273390in}{0.423512in}}%
\pgfpathlineto{\pgfqpoint{1.282400in}{0.418699in}}%
\pgfpathlineto{\pgfqpoint{1.286659in}{0.413070in}}%
\pgfpathlineto{\pgfqpoint{1.288883in}{0.405773in}}%
\pgfpathlineto{\pgfqpoint{1.289879in}{0.392846in}}%
\pgfpathlineto{\pgfqpoint{1.288750in}{0.379920in}}%
\pgfpathlineto{\pgfqpoint{1.286659in}{0.373780in}}%
\pgfpathlineto{\pgfqpoint{1.279867in}{0.366993in}}%
\pgfpathlineto{\pgfqpoint{1.273390in}{0.364297in}}%
\pgfpathlineto{\pgfqpoint{1.260121in}{0.364920in}}%
\pgfpathclose%
\pgfusepath{fill}%
\end{pgfscope}%
\begin{pgfscope}%
\pgfpathrectangle{\pgfqpoint{0.211875in}{0.211875in}}{\pgfqpoint{1.313625in}{1.279725in}}%
\pgfusepath{clip}%
\pgfsetbuttcap%
\pgfsetroundjoin%
\definecolor{currentfill}{rgb}{0.796501,0.105066,0.310630}%
\pgfsetfillcolor{currentfill}%
\pgfsetlinewidth{0.000000pt}%
\definecolor{currentstroke}{rgb}{0.000000,0.000000,0.000000}%
\pgfsetstrokecolor{currentstroke}%
\pgfsetdash{}{0pt}%
\pgfpathmoveto{\pgfqpoint{1.366273in}{0.355590in}}%
\pgfpathlineto{\pgfqpoint{1.379542in}{0.354860in}}%
\pgfpathlineto{\pgfqpoint{1.392811in}{0.354731in}}%
\pgfpathlineto{\pgfqpoint{1.406080in}{0.354837in}}%
\pgfpathlineto{\pgfqpoint{1.414579in}{0.366993in}}%
\pgfpathlineto{\pgfqpoint{1.414592in}{0.379920in}}%
\pgfpathlineto{\pgfqpoint{1.414441in}{0.392846in}}%
\pgfpathlineto{\pgfqpoint{1.414054in}{0.405773in}}%
\pgfpathlineto{\pgfqpoint{1.412930in}{0.418699in}}%
\pgfpathlineto{\pgfqpoint{1.406080in}{0.430507in}}%
\pgfpathlineto{\pgfqpoint{1.401932in}{0.431626in}}%
\pgfpathlineto{\pgfqpoint{1.392811in}{0.432831in}}%
\pgfpathlineto{\pgfqpoint{1.379542in}{0.432730in}}%
\pgfpathlineto{\pgfqpoint{1.372002in}{0.431626in}}%
\pgfpathlineto{\pgfqpoint{1.366273in}{0.429957in}}%
\pgfpathlineto{\pgfqpoint{1.359655in}{0.418699in}}%
\pgfpathlineto{\pgfqpoint{1.358353in}{0.405773in}}%
\pgfpathlineto{\pgfqpoint{1.357938in}{0.392846in}}%
\pgfpathlineto{\pgfqpoint{1.357858in}{0.379920in}}%
\pgfpathlineto{\pgfqpoint{1.358211in}{0.366993in}}%
\pgfpathclose%
\pgfpathmoveto{\pgfqpoint{1.372649in}{0.366993in}}%
\pgfpathlineto{\pgfqpoint{1.366273in}{0.374931in}}%
\pgfpathlineto{\pgfqpoint{1.364785in}{0.379920in}}%
\pgfpathlineto{\pgfqpoint{1.363752in}{0.392846in}}%
\pgfpathlineto{\pgfqpoint{1.364693in}{0.405773in}}%
\pgfpathlineto{\pgfqpoint{1.366273in}{0.411591in}}%
\pgfpathlineto{\pgfqpoint{1.370692in}{0.418699in}}%
\pgfpathlineto{\pgfqpoint{1.379542in}{0.424046in}}%
\pgfpathlineto{\pgfqpoint{1.392811in}{0.424096in}}%
\pgfpathlineto{\pgfqpoint{1.401779in}{0.418699in}}%
\pgfpathlineto{\pgfqpoint{1.406080in}{0.411642in}}%
\pgfpathlineto{\pgfqpoint{1.407643in}{0.405773in}}%
\pgfpathlineto{\pgfqpoint{1.408567in}{0.392846in}}%
\pgfpathlineto{\pgfqpoint{1.407578in}{0.379920in}}%
\pgfpathlineto{\pgfqpoint{1.406080in}{0.374774in}}%
\pgfpathlineto{\pgfqpoint{1.399931in}{0.366993in}}%
\pgfpathlineto{\pgfqpoint{1.392811in}{0.363655in}}%
\pgfpathlineto{\pgfqpoint{1.379542in}{0.363720in}}%
\pgfpathclose%
\pgfusepath{fill}%
\end{pgfscope}%
\begin{pgfscope}%
\pgfpathrectangle{\pgfqpoint{0.211875in}{0.211875in}}{\pgfqpoint{1.313625in}{1.279725in}}%
\pgfusepath{clip}%
\pgfsetbuttcap%
\pgfsetroundjoin%
\definecolor{currentfill}{rgb}{0.796501,0.105066,0.310630}%
\pgfsetfillcolor{currentfill}%
\pgfsetlinewidth{0.000000pt}%
\definecolor{currentstroke}{rgb}{0.000000,0.000000,0.000000}%
\pgfsetstrokecolor{currentstroke}%
\pgfsetdash{}{0pt}%
\pgfpathmoveto{\pgfqpoint{0.225144in}{0.442900in}}%
\pgfpathlineto{\pgfqpoint{0.227474in}{0.444552in}}%
\pgfpathlineto{\pgfqpoint{0.232797in}{0.457479in}}%
\pgfpathlineto{\pgfqpoint{0.233888in}{0.470405in}}%
\pgfpathlineto{\pgfqpoint{0.234119in}{0.483332in}}%
\pgfpathlineto{\pgfqpoint{0.233695in}{0.496258in}}%
\pgfpathlineto{\pgfqpoint{0.230470in}{0.509185in}}%
\pgfpathlineto{\pgfqpoint{0.225144in}{0.512637in}}%
\pgfpathlineto{\pgfqpoint{0.211875in}{0.513910in}}%
\pgfpathlineto{\pgfqpoint{0.211875in}{0.509185in}}%
\pgfpathlineto{\pgfqpoint{0.211875in}{0.504948in}}%
\pgfpathlineto{\pgfqpoint{0.225024in}{0.496258in}}%
\pgfpathlineto{\pgfqpoint{0.225144in}{0.496013in}}%
\pgfpathlineto{\pgfqpoint{0.227789in}{0.483332in}}%
\pgfpathlineto{\pgfqpoint{0.227647in}{0.470405in}}%
\pgfpathlineto{\pgfqpoint{0.225144in}{0.459277in}}%
\pgfpathlineto{\pgfqpoint{0.224234in}{0.457479in}}%
\pgfpathlineto{\pgfqpoint{0.211875in}{0.449101in}}%
\pgfpathlineto{\pgfqpoint{0.211875in}{0.444552in}}%
\pgfpathlineto{\pgfqpoint{0.211875in}{0.440410in}}%
\pgfpathclose%
\pgfusepath{fill}%
\end{pgfscope}%
\begin{pgfscope}%
\pgfpathrectangle{\pgfqpoint{0.211875in}{0.211875in}}{\pgfqpoint{1.313625in}{1.279725in}}%
\pgfusepath{clip}%
\pgfsetbuttcap%
\pgfsetroundjoin%
\definecolor{currentfill}{rgb}{0.796501,0.105066,0.310630}%
\pgfsetfillcolor{currentfill}%
\pgfsetlinewidth{0.000000pt}%
\definecolor{currentstroke}{rgb}{0.000000,0.000000,0.000000}%
\pgfsetstrokecolor{currentstroke}%
\pgfsetdash{}{0pt}%
\pgfpathmoveto{\pgfqpoint{0.716095in}{0.436260in}}%
\pgfpathlineto{\pgfqpoint{0.729364in}{0.436136in}}%
\pgfpathlineto{\pgfqpoint{0.742633in}{0.436324in}}%
\pgfpathlineto{\pgfqpoint{0.755902in}{0.437250in}}%
\pgfpathlineto{\pgfqpoint{0.764152in}{0.444552in}}%
\pgfpathlineto{\pgfqpoint{0.765272in}{0.457479in}}%
\pgfpathlineto{\pgfqpoint{0.765380in}{0.470405in}}%
\pgfpathlineto{\pgfqpoint{0.765105in}{0.483332in}}%
\pgfpathlineto{\pgfqpoint{0.764166in}{0.496258in}}%
\pgfpathlineto{\pgfqpoint{0.759577in}{0.509185in}}%
\pgfpathlineto{\pgfqpoint{0.755902in}{0.511712in}}%
\pgfpathlineto{\pgfqpoint{0.742633in}{0.513994in}}%
\pgfpathlineto{\pgfqpoint{0.729364in}{0.514049in}}%
\pgfpathlineto{\pgfqpoint{0.716095in}{0.511512in}}%
\pgfpathlineto{\pgfqpoint{0.713377in}{0.509185in}}%
\pgfpathlineto{\pgfqpoint{0.709843in}{0.496258in}}%
\pgfpathlineto{\pgfqpoint{0.709130in}{0.483332in}}%
\pgfpathlineto{\pgfqpoint{0.708865in}{0.470405in}}%
\pgfpathlineto{\pgfqpoint{0.708774in}{0.457479in}}%
\pgfpathlineto{\pgfqpoint{0.708884in}{0.444552in}}%
\pgfpathclose%
\pgfpathmoveto{\pgfqpoint{0.716795in}{0.457479in}}%
\pgfpathlineto{\pgfqpoint{0.716095in}{0.460497in}}%
\pgfpathlineto{\pgfqpoint{0.714851in}{0.470405in}}%
\pgfpathlineto{\pgfqpoint{0.715200in}{0.483332in}}%
\pgfpathlineto{\pgfqpoint{0.716095in}{0.488253in}}%
\pgfpathlineto{\pgfqpoint{0.718889in}{0.496258in}}%
\pgfpathlineto{\pgfqpoint{0.729364in}{0.505052in}}%
\pgfpathlineto{\pgfqpoint{0.742633in}{0.505532in}}%
\pgfpathlineto{\pgfqpoint{0.755902in}{0.496445in}}%
\pgfpathlineto{\pgfqpoint{0.755999in}{0.496258in}}%
\pgfpathlineto{\pgfqpoint{0.759027in}{0.483332in}}%
\pgfpathlineto{\pgfqpoint{0.759383in}{0.470405in}}%
\pgfpathlineto{\pgfqpoint{0.757498in}{0.457479in}}%
\pgfpathlineto{\pgfqpoint{0.755902in}{0.453765in}}%
\pgfpathlineto{\pgfqpoint{0.742633in}{0.444780in}}%
\pgfpathlineto{\pgfqpoint{0.729364in}{0.445190in}}%
\pgfpathclose%
\pgfusepath{fill}%
\end{pgfscope}%
\begin{pgfscope}%
\pgfpathrectangle{\pgfqpoint{0.211875in}{0.211875in}}{\pgfqpoint{1.313625in}{1.279725in}}%
\pgfusepath{clip}%
\pgfsetbuttcap%
\pgfsetroundjoin%
\definecolor{currentfill}{rgb}{0.796501,0.105066,0.310630}%
\pgfsetfillcolor{currentfill}%
\pgfsetlinewidth{0.000000pt}%
\definecolor{currentstroke}{rgb}{0.000000,0.000000,0.000000}%
\pgfsetstrokecolor{currentstroke}%
\pgfsetdash{}{0pt}%
\pgfpathmoveto{\pgfqpoint{0.835515in}{0.439764in}}%
\pgfpathlineto{\pgfqpoint{0.848784in}{0.437882in}}%
\pgfpathlineto{\pgfqpoint{0.862053in}{0.438043in}}%
\pgfpathlineto{\pgfqpoint{0.875322in}{0.440728in}}%
\pgfpathlineto{\pgfqpoint{0.879060in}{0.444552in}}%
\pgfpathlineto{\pgfqpoint{0.881815in}{0.457479in}}%
\pgfpathlineto{\pgfqpoint{0.882243in}{0.470405in}}%
\pgfpathlineto{\pgfqpoint{0.881957in}{0.483332in}}%
\pgfpathlineto{\pgfqpoint{0.880651in}{0.496258in}}%
\pgfpathlineto{\pgfqpoint{0.875322in}{0.507970in}}%
\pgfpathlineto{\pgfqpoint{0.873315in}{0.509185in}}%
\pgfpathlineto{\pgfqpoint{0.862053in}{0.512318in}}%
\pgfpathlineto{\pgfqpoint{0.848784in}{0.512472in}}%
\pgfpathlineto{\pgfqpoint{0.835881in}{0.509185in}}%
\pgfpathlineto{\pgfqpoint{0.835515in}{0.508983in}}%
\pgfpathlineto{\pgfqpoint{0.829573in}{0.496258in}}%
\pgfpathlineto{\pgfqpoint{0.828396in}{0.483332in}}%
\pgfpathlineto{\pgfqpoint{0.828111in}{0.470405in}}%
\pgfpathlineto{\pgfqpoint{0.828411in}{0.457479in}}%
\pgfpathlineto{\pgfqpoint{0.830588in}{0.444552in}}%
\pgfpathclose%
\pgfpathmoveto{\pgfqpoint{0.836666in}{0.457479in}}%
\pgfpathlineto{\pgfqpoint{0.835515in}{0.461288in}}%
\pgfpathlineto{\pgfqpoint{0.834135in}{0.470405in}}%
\pgfpathlineto{\pgfqpoint{0.834504in}{0.483332in}}%
\pgfpathlineto{\pgfqpoint{0.835515in}{0.488126in}}%
\pgfpathlineto{\pgfqpoint{0.839070in}{0.496258in}}%
\pgfpathlineto{\pgfqpoint{0.848784in}{0.503532in}}%
\pgfpathlineto{\pgfqpoint{0.862053in}{0.503347in}}%
\pgfpathlineto{\pgfqpoint{0.871173in}{0.496258in}}%
\pgfpathlineto{\pgfqpoint{0.875322in}{0.486383in}}%
\pgfpathlineto{\pgfqpoint{0.875953in}{0.483332in}}%
\pgfpathlineto{\pgfqpoint{0.876320in}{0.470405in}}%
\pgfpathlineto{\pgfqpoint{0.875322in}{0.463674in}}%
\pgfpathlineto{\pgfqpoint{0.873521in}{0.457479in}}%
\pgfpathlineto{\pgfqpoint{0.862053in}{0.447140in}}%
\pgfpathlineto{\pgfqpoint{0.848784in}{0.446937in}}%
\pgfpathclose%
\pgfusepath{fill}%
\end{pgfscope}%
\begin{pgfscope}%
\pgfpathrectangle{\pgfqpoint{0.211875in}{0.211875in}}{\pgfqpoint{1.313625in}{1.279725in}}%
\pgfusepath{clip}%
\pgfsetbuttcap%
\pgfsetroundjoin%
\definecolor{currentfill}{rgb}{0.796501,0.105066,0.310630}%
\pgfsetfillcolor{currentfill}%
\pgfsetlinewidth{0.000000pt}%
\definecolor{currentstroke}{rgb}{0.000000,0.000000,0.000000}%
\pgfsetstrokecolor{currentstroke}%
\pgfsetdash{}{0pt}%
\pgfpathmoveto{\pgfqpoint{0.954936in}{0.441366in}}%
\pgfpathlineto{\pgfqpoint{0.968205in}{0.438971in}}%
\pgfpathlineto{\pgfqpoint{0.981473in}{0.439284in}}%
\pgfpathlineto{\pgfqpoint{0.994742in}{0.444136in}}%
\pgfpathlineto{\pgfqpoint{0.995086in}{0.444552in}}%
\pgfpathlineto{\pgfqpoint{0.998902in}{0.457479in}}%
\pgfpathlineto{\pgfqpoint{0.999533in}{0.470405in}}%
\pgfpathlineto{\pgfqpoint{0.999239in}{0.483332in}}%
\pgfpathlineto{\pgfqpoint{0.997698in}{0.496258in}}%
\pgfpathlineto{\pgfqpoint{0.994742in}{0.503838in}}%
\pgfpathlineto{\pgfqpoint{0.987640in}{0.509185in}}%
\pgfpathlineto{\pgfqpoint{0.981473in}{0.511083in}}%
\pgfpathlineto{\pgfqpoint{0.968205in}{0.511494in}}%
\pgfpathlineto{\pgfqpoint{0.958056in}{0.509185in}}%
\pgfpathlineto{\pgfqpoint{0.954936in}{0.507789in}}%
\pgfpathlineto{\pgfqpoint{0.948767in}{0.496258in}}%
\pgfpathlineto{\pgfqpoint{0.947257in}{0.483332in}}%
\pgfpathlineto{\pgfqpoint{0.946956in}{0.470405in}}%
\pgfpathlineto{\pgfqpoint{0.947531in}{0.457479in}}%
\pgfpathlineto{\pgfqpoint{0.951146in}{0.444552in}}%
\pgfpathclose%
\pgfpathmoveto{\pgfqpoint{0.955844in}{0.457479in}}%
\pgfpathlineto{\pgfqpoint{0.954936in}{0.459835in}}%
\pgfpathlineto{\pgfqpoint{0.953058in}{0.470405in}}%
\pgfpathlineto{\pgfqpoint{0.953443in}{0.483332in}}%
\pgfpathlineto{\pgfqpoint{0.954936in}{0.489525in}}%
\pgfpathlineto{\pgfqpoint{0.958590in}{0.496258in}}%
\pgfpathlineto{\pgfqpoint{0.968205in}{0.502631in}}%
\pgfpathlineto{\pgfqpoint{0.981473in}{0.501597in}}%
\pgfpathlineto{\pgfqpoint{0.987617in}{0.496258in}}%
\pgfpathlineto{\pgfqpoint{0.992678in}{0.483332in}}%
\pgfpathlineto{\pgfqpoint{0.993204in}{0.470405in}}%
\pgfpathlineto{\pgfqpoint{0.989853in}{0.457479in}}%
\pgfpathlineto{\pgfqpoint{0.981473in}{0.449010in}}%
\pgfpathlineto{\pgfqpoint{0.968205in}{0.447978in}}%
\pgfpathclose%
\pgfusepath{fill}%
\end{pgfscope}%
\begin{pgfscope}%
\pgfpathrectangle{\pgfqpoint{0.211875in}{0.211875in}}{\pgfqpoint{1.313625in}{1.279725in}}%
\pgfusepath{clip}%
\pgfsetbuttcap%
\pgfsetroundjoin%
\definecolor{currentfill}{rgb}{0.796501,0.105066,0.310630}%
\pgfsetfillcolor{currentfill}%
\pgfsetlinewidth{0.000000pt}%
\definecolor{currentstroke}{rgb}{0.000000,0.000000,0.000000}%
\pgfsetstrokecolor{currentstroke}%
\pgfsetdash{}{0pt}%
\pgfpathmoveto{\pgfqpoint{1.074356in}{0.441751in}}%
\pgfpathlineto{\pgfqpoint{1.087625in}{0.439476in}}%
\pgfpathlineto{\pgfqpoint{1.100894in}{0.440020in}}%
\pgfpathlineto{\pgfqpoint{1.111565in}{0.444552in}}%
\pgfpathlineto{\pgfqpoint{1.114163in}{0.448170in}}%
\pgfpathlineto{\pgfqpoint{1.116490in}{0.457479in}}%
\pgfpathlineto{\pgfqpoint{1.117212in}{0.470405in}}%
\pgfpathlineto{\pgfqpoint{1.116914in}{0.483332in}}%
\pgfpathlineto{\pgfqpoint{1.115263in}{0.496258in}}%
\pgfpathlineto{\pgfqpoint{1.114163in}{0.499641in}}%
\pgfpathlineto{\pgfqpoint{1.104203in}{0.509185in}}%
\pgfpathlineto{\pgfqpoint{1.100894in}{0.510308in}}%
\pgfpathlineto{\pgfqpoint{1.087625in}{0.511052in}}%
\pgfpathlineto{\pgfqpoint{1.078356in}{0.509185in}}%
\pgfpathlineto{\pgfqpoint{1.074356in}{0.507732in}}%
\pgfpathlineto{\pgfqpoint{1.067411in}{0.496258in}}%
\pgfpathlineto{\pgfqpoint{1.065696in}{0.483332in}}%
\pgfpathlineto{\pgfqpoint{1.065383in}{0.470405in}}%
\pgfpathlineto{\pgfqpoint{1.066119in}{0.457479in}}%
\pgfpathlineto{\pgfqpoint{1.070568in}{0.444552in}}%
\pgfpathclose%
\pgfpathmoveto{\pgfqpoint{1.074218in}{0.457479in}}%
\pgfpathlineto{\pgfqpoint{1.071605in}{0.470405in}}%
\pgfpathlineto{\pgfqpoint{1.072005in}{0.483332in}}%
\pgfpathlineto{\pgfqpoint{1.074356in}{0.491978in}}%
\pgfpathlineto{\pgfqpoint{1.077239in}{0.496258in}}%
\pgfpathlineto{\pgfqpoint{1.087625in}{0.502287in}}%
\pgfpathlineto{\pgfqpoint{1.100894in}{0.500297in}}%
\pgfpathlineto{\pgfqpoint{1.105067in}{0.496258in}}%
\pgfpathlineto{\pgfqpoint{1.109949in}{0.483332in}}%
\pgfpathlineto{\pgfqpoint{1.110445in}{0.470405in}}%
\pgfpathlineto{\pgfqpoint{1.107187in}{0.457479in}}%
\pgfpathlineto{\pgfqpoint{1.100894in}{0.450376in}}%
\pgfpathlineto{\pgfqpoint{1.087625in}{0.448392in}}%
\pgfpathlineto{\pgfqpoint{1.074356in}{0.457206in}}%
\pgfpathclose%
\pgfusepath{fill}%
\end{pgfscope}%
\begin{pgfscope}%
\pgfpathrectangle{\pgfqpoint{0.211875in}{0.211875in}}{\pgfqpoint{1.313625in}{1.279725in}}%
\pgfusepath{clip}%
\pgfsetbuttcap%
\pgfsetroundjoin%
\definecolor{currentfill}{rgb}{0.796501,0.105066,0.310630}%
\pgfsetfillcolor{currentfill}%
\pgfsetlinewidth{0.000000pt}%
\definecolor{currentstroke}{rgb}{0.000000,0.000000,0.000000}%
\pgfsetstrokecolor{currentstroke}%
\pgfsetdash{}{0pt}%
\pgfpathmoveto{\pgfqpoint{1.193777in}{0.441291in}}%
\pgfpathlineto{\pgfqpoint{1.207045in}{0.439440in}}%
\pgfpathlineto{\pgfqpoint{1.220314in}{0.440196in}}%
\pgfpathlineto{\pgfqpoint{1.229574in}{0.444552in}}%
\pgfpathlineto{\pgfqpoint{1.233583in}{0.452361in}}%
\pgfpathlineto{\pgfqpoint{1.234554in}{0.457479in}}%
\pgfpathlineto{\pgfqpoint{1.235256in}{0.470405in}}%
\pgfpathlineto{\pgfqpoint{1.234957in}{0.483332in}}%
\pgfpathlineto{\pgfqpoint{1.233583in}{0.494662in}}%
\pgfpathlineto{\pgfqpoint{1.233263in}{0.496258in}}%
\pgfpathlineto{\pgfqpoint{1.222603in}{0.509185in}}%
\pgfpathlineto{\pgfqpoint{1.220314in}{0.510041in}}%
\pgfpathlineto{\pgfqpoint{1.207045in}{0.511107in}}%
\pgfpathlineto{\pgfqpoint{1.196122in}{0.509185in}}%
\pgfpathlineto{\pgfqpoint{1.193777in}{0.508496in}}%
\pgfpathlineto{\pgfqpoint{1.185469in}{0.496258in}}%
\pgfpathlineto{\pgfqpoint{1.183678in}{0.483332in}}%
\pgfpathlineto{\pgfqpoint{1.183355in}{0.470405in}}%
\pgfpathlineto{\pgfqpoint{1.184138in}{0.457479in}}%
\pgfpathlineto{\pgfqpoint{1.188816in}{0.444552in}}%
\pgfpathclose%
\pgfpathmoveto{\pgfqpoint{1.192447in}{0.457479in}}%
\pgfpathlineto{\pgfqpoint{1.189745in}{0.470405in}}%
\pgfpathlineto{\pgfqpoint{1.190156in}{0.483332in}}%
\pgfpathlineto{\pgfqpoint{1.193777in}{0.495218in}}%
\pgfpathlineto{\pgfqpoint{1.194653in}{0.496258in}}%
\pgfpathlineto{\pgfqpoint{1.207045in}{0.502459in}}%
\pgfpathlineto{\pgfqpoint{1.220314in}{0.499481in}}%
\pgfpathlineto{\pgfqpoint{1.223312in}{0.496258in}}%
\pgfpathlineto{\pgfqpoint{1.227896in}{0.483332in}}%
\pgfpathlineto{\pgfqpoint{1.228363in}{0.470405in}}%
\pgfpathlineto{\pgfqpoint{1.225309in}{0.457479in}}%
\pgfpathlineto{\pgfqpoint{1.220314in}{0.451190in}}%
\pgfpathlineto{\pgfqpoint{1.207045in}{0.448223in}}%
\pgfpathlineto{\pgfqpoint{1.193777in}{0.455149in}}%
\pgfpathclose%
\pgfusepath{fill}%
\end{pgfscope}%
\begin{pgfscope}%
\pgfpathrectangle{\pgfqpoint{0.211875in}{0.211875in}}{\pgfqpoint{1.313625in}{1.279725in}}%
\pgfusepath{clip}%
\pgfsetbuttcap%
\pgfsetroundjoin%
\definecolor{currentfill}{rgb}{0.796501,0.105066,0.310630}%
\pgfsetfillcolor{currentfill}%
\pgfsetlinewidth{0.000000pt}%
\definecolor{currentstroke}{rgb}{0.000000,0.000000,0.000000}%
\pgfsetstrokecolor{currentstroke}%
\pgfsetdash{}{0pt}%
\pgfpathmoveto{\pgfqpoint{1.313197in}{0.440196in}}%
\pgfpathlineto{\pgfqpoint{1.326466in}{0.438884in}}%
\pgfpathlineto{\pgfqpoint{1.339735in}{0.439720in}}%
\pgfpathlineto{\pgfqpoint{1.348997in}{0.444552in}}%
\pgfpathlineto{\pgfqpoint{1.353004in}{0.456820in}}%
\pgfpathlineto{\pgfqpoint{1.353088in}{0.457479in}}%
\pgfpathlineto{\pgfqpoint{1.353654in}{0.470405in}}%
\pgfpathlineto{\pgfqpoint{1.353358in}{0.483332in}}%
\pgfpathlineto{\pgfqpoint{1.353004in}{0.487167in}}%
\pgfpathlineto{\pgfqpoint{1.351699in}{0.496258in}}%
\pgfpathlineto{\pgfqpoint{1.342578in}{0.509185in}}%
\pgfpathlineto{\pgfqpoint{1.339735in}{0.510358in}}%
\pgfpathlineto{\pgfqpoint{1.326466in}{0.511644in}}%
\pgfpathlineto{\pgfqpoint{1.313197in}{0.509795in}}%
\pgfpathlineto{\pgfqpoint{1.311891in}{0.509185in}}%
\pgfpathlineto{\pgfqpoint{1.302876in}{0.496258in}}%
\pgfpathlineto{\pgfqpoint{1.301145in}{0.483332in}}%
\pgfpathlineto{\pgfqpoint{1.300815in}{0.470405in}}%
\pgfpathlineto{\pgfqpoint{1.301523in}{0.457479in}}%
\pgfpathlineto{\pgfqpoint{1.305803in}{0.444552in}}%
\pgfpathclose%
\pgfpathmoveto{\pgfqpoint{1.310110in}{0.457479in}}%
\pgfpathlineto{\pgfqpoint{1.307426in}{0.470405in}}%
\pgfpathlineto{\pgfqpoint{1.307847in}{0.483332in}}%
\pgfpathlineto{\pgfqpoint{1.311900in}{0.496258in}}%
\pgfpathlineto{\pgfqpoint{1.313197in}{0.497971in}}%
\pgfpathlineto{\pgfqpoint{1.326466in}{0.503133in}}%
\pgfpathlineto{\pgfqpoint{1.339735in}{0.499216in}}%
\pgfpathlineto{\pgfqpoint{1.342211in}{0.496258in}}%
\pgfpathlineto{\pgfqpoint{1.346393in}{0.483332in}}%
\pgfpathlineto{\pgfqpoint{1.346831in}{0.470405in}}%
\pgfpathlineto{\pgfqpoint{1.344081in}{0.457479in}}%
\pgfpathlineto{\pgfqpoint{1.339735in}{0.451372in}}%
\pgfpathlineto{\pgfqpoint{1.326466in}{0.447492in}}%
\pgfpathlineto{\pgfqpoint{1.313197in}{0.452663in}}%
\pgfpathclose%
\pgfusepath{fill}%
\end{pgfscope}%
\begin{pgfscope}%
\pgfpathrectangle{\pgfqpoint{0.211875in}{0.211875in}}{\pgfqpoint{1.313625in}{1.279725in}}%
\pgfusepath{clip}%
\pgfsetbuttcap%
\pgfsetroundjoin%
\definecolor{currentfill}{rgb}{0.796501,0.105066,0.310630}%
\pgfsetfillcolor{currentfill}%
\pgfsetlinewidth{0.000000pt}%
\definecolor{currentstroke}{rgb}{0.000000,0.000000,0.000000}%
\pgfsetstrokecolor{currentstroke}%
\pgfsetdash{}{0pt}%
\pgfpathmoveto{\pgfqpoint{1.432617in}{0.438589in}}%
\pgfpathlineto{\pgfqpoint{1.445886in}{0.437806in}}%
\pgfpathlineto{\pgfqpoint{1.459155in}{0.438445in}}%
\pgfpathlineto{\pgfqpoint{1.469675in}{0.444552in}}%
\pgfpathlineto{\pgfqpoint{1.472080in}{0.457479in}}%
\pgfpathlineto{\pgfqpoint{1.472410in}{0.470405in}}%
\pgfpathlineto{\pgfqpoint{1.472096in}{0.483332in}}%
\pgfpathlineto{\pgfqpoint{1.470798in}{0.496258in}}%
\pgfpathlineto{\pgfqpoint{1.463967in}{0.509185in}}%
\pgfpathlineto{\pgfqpoint{1.459155in}{0.511379in}}%
\pgfpathlineto{\pgfqpoint{1.445886in}{0.512664in}}%
\pgfpathlineto{\pgfqpoint{1.432617in}{0.511447in}}%
\pgfpathlineto{\pgfqpoint{1.427275in}{0.509185in}}%
\pgfpathlineto{\pgfqpoint{1.419533in}{0.496258in}}%
\pgfpathlineto{\pgfqpoint{1.419348in}{0.495051in}}%
\pgfpathlineto{\pgfqpoint{1.418202in}{0.483332in}}%
\pgfpathlineto{\pgfqpoint{1.417916in}{0.470405in}}%
\pgfpathlineto{\pgfqpoint{1.418344in}{0.457479in}}%
\pgfpathlineto{\pgfqpoint{1.419348in}{0.450303in}}%
\pgfpathlineto{\pgfqpoint{1.421383in}{0.444552in}}%
\pgfpathclose%
\pgfpathmoveto{\pgfqpoint{1.427121in}{0.457479in}}%
\pgfpathlineto{\pgfqpoint{1.424574in}{0.470405in}}%
\pgfpathlineto{\pgfqpoint{1.425001in}{0.483332in}}%
\pgfpathlineto{\pgfqpoint{1.428930in}{0.496258in}}%
\pgfpathlineto{\pgfqpoint{1.432617in}{0.500641in}}%
\pgfpathlineto{\pgfqpoint{1.445886in}{0.504311in}}%
\pgfpathlineto{\pgfqpoint{1.459155in}{0.499609in}}%
\pgfpathlineto{\pgfqpoint{1.461676in}{0.496258in}}%
\pgfpathlineto{\pgfqpoint{1.465357in}{0.483332in}}%
\pgfpathlineto{\pgfqpoint{1.465764in}{0.470405in}}%
\pgfpathlineto{\pgfqpoint{1.463410in}{0.457479in}}%
\pgfpathlineto{\pgfqpoint{1.459155in}{0.450787in}}%
\pgfpathlineto{\pgfqpoint{1.445886in}{0.446195in}}%
\pgfpathlineto{\pgfqpoint{1.432617in}{0.449808in}}%
\pgfpathclose%
\pgfusepath{fill}%
\end{pgfscope}%
\begin{pgfscope}%
\pgfpathrectangle{\pgfqpoint{0.211875in}{0.211875in}}{\pgfqpoint{1.313625in}{1.279725in}}%
\pgfusepath{clip}%
\pgfsetbuttcap%
\pgfsetroundjoin%
\definecolor{currentfill}{rgb}{0.796501,0.105066,0.310630}%
\pgfsetfillcolor{currentfill}%
\pgfsetlinewidth{0.000000pt}%
\definecolor{currentstroke}{rgb}{0.000000,0.000000,0.000000}%
\pgfsetstrokecolor{currentstroke}%
\pgfsetdash{}{0pt}%
\pgfpathmoveto{\pgfqpoint{0.663019in}{0.518920in}}%
\pgfpathlineto{\pgfqpoint{0.676288in}{0.518461in}}%
\pgfpathlineto{\pgfqpoint{0.689557in}{0.519083in}}%
\pgfpathlineto{\pgfqpoint{0.699659in}{0.522111in}}%
\pgfpathlineto{\pgfqpoint{0.702826in}{0.525571in}}%
\pgfpathlineto{\pgfqpoint{0.705162in}{0.535038in}}%
\pgfpathlineto{\pgfqpoint{0.705810in}{0.547964in}}%
\pgfpathlineto{\pgfqpoint{0.705726in}{0.560891in}}%
\pgfpathlineto{\pgfqpoint{0.704939in}{0.573817in}}%
\pgfpathlineto{\pgfqpoint{0.702826in}{0.583908in}}%
\pgfpathlineto{\pgfqpoint{0.701457in}{0.586744in}}%
\pgfpathlineto{\pgfqpoint{0.689557in}{0.593578in}}%
\pgfpathlineto{\pgfqpoint{0.676288in}{0.594502in}}%
\pgfpathlineto{\pgfqpoint{0.663019in}{0.593283in}}%
\pgfpathlineto{\pgfqpoint{0.654284in}{0.586744in}}%
\pgfpathlineto{\pgfqpoint{0.651441in}{0.573817in}}%
\pgfpathlineto{\pgfqpoint{0.650744in}{0.560891in}}%
\pgfpathlineto{\pgfqpoint{0.650604in}{0.547964in}}%
\pgfpathlineto{\pgfqpoint{0.650962in}{0.535038in}}%
\pgfpathlineto{\pgfqpoint{0.654323in}{0.522111in}}%
\pgfpathclose%
\pgfpathmoveto{\pgfqpoint{0.660764in}{0.535038in}}%
\pgfpathlineto{\pgfqpoint{0.657364in}{0.547964in}}%
\pgfpathlineto{\pgfqpoint{0.657112in}{0.560891in}}%
\pgfpathlineto{\pgfqpoint{0.659305in}{0.573817in}}%
\pgfpathlineto{\pgfqpoint{0.663019in}{0.580603in}}%
\pgfpathlineto{\pgfqpoint{0.676288in}{0.586610in}}%
\pgfpathlineto{\pgfqpoint{0.689557in}{0.583583in}}%
\pgfpathlineto{\pgfqpoint{0.696749in}{0.573817in}}%
\pgfpathlineto{\pgfqpoint{0.699322in}{0.560891in}}%
\pgfpathlineto{\pgfqpoint{0.699002in}{0.547964in}}%
\pgfpathlineto{\pgfqpoint{0.694981in}{0.535038in}}%
\pgfpathlineto{\pgfqpoint{0.689557in}{0.529557in}}%
\pgfpathlineto{\pgfqpoint{0.676288in}{0.527070in}}%
\pgfpathlineto{\pgfqpoint{0.663019in}{0.531954in}}%
\pgfpathclose%
\pgfusepath{fill}%
\end{pgfscope}%
\begin{pgfscope}%
\pgfpathrectangle{\pgfqpoint{0.211875in}{0.211875in}}{\pgfqpoint{1.313625in}{1.279725in}}%
\pgfusepath{clip}%
\pgfsetbuttcap%
\pgfsetroundjoin%
\definecolor{currentfill}{rgb}{0.796501,0.105066,0.310630}%
\pgfsetfillcolor{currentfill}%
\pgfsetlinewidth{0.000000pt}%
\definecolor{currentstroke}{rgb}{0.000000,0.000000,0.000000}%
\pgfsetstrokecolor{currentstroke}%
\pgfsetdash{}{0pt}%
\pgfpathmoveto{\pgfqpoint{0.782439in}{0.521415in}}%
\pgfpathlineto{\pgfqpoint{0.795708in}{0.520333in}}%
\pgfpathlineto{\pgfqpoint{0.808977in}{0.521562in}}%
\pgfpathlineto{\pgfqpoint{0.810648in}{0.522111in}}%
\pgfpathlineto{\pgfqpoint{0.821015in}{0.535038in}}%
\pgfpathlineto{\pgfqpoint{0.822246in}{0.545584in}}%
\pgfpathlineto{\pgfqpoint{0.822414in}{0.547964in}}%
\pgfpathlineto{\pgfqpoint{0.822404in}{0.560891in}}%
\pgfpathlineto{\pgfqpoint{0.822246in}{0.563131in}}%
\pgfpathlineto{\pgfqpoint{0.821176in}{0.573817in}}%
\pgfpathlineto{\pgfqpoint{0.815808in}{0.586744in}}%
\pgfpathlineto{\pgfqpoint{0.808977in}{0.591080in}}%
\pgfpathlineto{\pgfqpoint{0.795708in}{0.592695in}}%
\pgfpathlineto{\pgfqpoint{0.782439in}{0.591085in}}%
\pgfpathlineto{\pgfqpoint{0.775981in}{0.586744in}}%
\pgfpathlineto{\pgfqpoint{0.771194in}{0.573817in}}%
\pgfpathlineto{\pgfqpoint{0.770095in}{0.560891in}}%
\pgfpathlineto{\pgfqpoint{0.770049in}{0.547964in}}%
\pgfpathlineto{\pgfqpoint{0.771188in}{0.535038in}}%
\pgfpathlineto{\pgfqpoint{0.780334in}{0.522111in}}%
\pgfpathclose%
\pgfpathmoveto{\pgfqpoint{0.781310in}{0.535038in}}%
\pgfpathlineto{\pgfqpoint{0.777040in}{0.547964in}}%
\pgfpathlineto{\pgfqpoint{0.776681in}{0.560891in}}%
\pgfpathlineto{\pgfqpoint{0.779323in}{0.573817in}}%
\pgfpathlineto{\pgfqpoint{0.782439in}{0.578915in}}%
\pgfpathlineto{\pgfqpoint{0.795708in}{0.584391in}}%
\pgfpathlineto{\pgfqpoint{0.808977in}{0.579748in}}%
\pgfpathlineto{\pgfqpoint{0.812911in}{0.573817in}}%
\pgfpathlineto{\pgfqpoint{0.815698in}{0.560891in}}%
\pgfpathlineto{\pgfqpoint{0.815305in}{0.547964in}}%
\pgfpathlineto{\pgfqpoint{0.810793in}{0.535038in}}%
\pgfpathlineto{\pgfqpoint{0.808977in}{0.532996in}}%
\pgfpathlineto{\pgfqpoint{0.795708in}{0.529098in}}%
\pgfpathlineto{\pgfqpoint{0.782439in}{0.533662in}}%
\pgfpathclose%
\pgfusepath{fill}%
\end{pgfscope}%
\begin{pgfscope}%
\pgfpathrectangle{\pgfqpoint{0.211875in}{0.211875in}}{\pgfqpoint{1.313625in}{1.279725in}}%
\pgfusepath{clip}%
\pgfsetbuttcap%
\pgfsetroundjoin%
\definecolor{currentfill}{rgb}{0.796501,0.105066,0.310630}%
\pgfsetfillcolor{currentfill}%
\pgfsetlinewidth{0.000000pt}%
\definecolor{currentstroke}{rgb}{0.000000,0.000000,0.000000}%
\pgfsetstrokecolor{currentstroke}%
\pgfsetdash{}{0pt}%
\pgfpathmoveto{\pgfqpoint{0.915129in}{0.521673in}}%
\pgfpathlineto{\pgfqpoint{0.918818in}{0.522111in}}%
\pgfpathlineto{\pgfqpoint{0.928398in}{0.523858in}}%
\pgfpathlineto{\pgfqpoint{0.937302in}{0.535038in}}%
\pgfpathlineto{\pgfqpoint{0.939119in}{0.547964in}}%
\pgfpathlineto{\pgfqpoint{0.939167in}{0.560891in}}%
\pgfpathlineto{\pgfqpoint{0.937741in}{0.573817in}}%
\pgfpathlineto{\pgfqpoint{0.931493in}{0.586744in}}%
\pgfpathlineto{\pgfqpoint{0.928398in}{0.588921in}}%
\pgfpathlineto{\pgfqpoint{0.915129in}{0.591392in}}%
\pgfpathlineto{\pgfqpoint{0.901860in}{0.589810in}}%
\pgfpathlineto{\pgfqpoint{0.896798in}{0.586744in}}%
\pgfpathlineto{\pgfqpoint{0.890432in}{0.573817in}}%
\pgfpathlineto{\pgfqpoint{0.889000in}{0.560891in}}%
\pgfpathlineto{\pgfqpoint{0.889028in}{0.547964in}}%
\pgfpathlineto{\pgfqpoint{0.890794in}{0.535038in}}%
\pgfpathlineto{\pgfqpoint{0.901860in}{0.522978in}}%
\pgfpathlineto{\pgfqpoint{0.909567in}{0.522111in}}%
\pgfpathclose%
\pgfpathmoveto{\pgfqpoint{0.901329in}{0.535038in}}%
\pgfpathlineto{\pgfqpoint{0.896315in}{0.547964in}}%
\pgfpathlineto{\pgfqpoint{0.895868in}{0.560891in}}%
\pgfpathlineto{\pgfqpoint{0.898901in}{0.573817in}}%
\pgfpathlineto{\pgfqpoint{0.901860in}{0.578165in}}%
\pgfpathlineto{\pgfqpoint{0.915129in}{0.582747in}}%
\pgfpathlineto{\pgfqpoint{0.928398in}{0.576169in}}%
\pgfpathlineto{\pgfqpoint{0.929800in}{0.573817in}}%
\pgfpathlineto{\pgfqpoint{0.932705in}{0.560891in}}%
\pgfpathlineto{\pgfqpoint{0.932268in}{0.547964in}}%
\pgfpathlineto{\pgfqpoint{0.928398in}{0.536892in}}%
\pgfpathlineto{\pgfqpoint{0.926270in}{0.535038in}}%
\pgfpathlineto{\pgfqpoint{0.915129in}{0.530593in}}%
\pgfpathlineto{\pgfqpoint{0.901860in}{0.534459in}}%
\pgfpathclose%
\pgfusepath{fill}%
\end{pgfscope}%
\begin{pgfscope}%
\pgfpathrectangle{\pgfqpoint{0.211875in}{0.211875in}}{\pgfqpoint{1.313625in}{1.279725in}}%
\pgfusepath{clip}%
\pgfsetbuttcap%
\pgfsetroundjoin%
\definecolor{currentfill}{rgb}{0.796501,0.105066,0.310630}%
\pgfsetfillcolor{currentfill}%
\pgfsetlinewidth{0.000000pt}%
\definecolor{currentstroke}{rgb}{0.000000,0.000000,0.000000}%
\pgfsetstrokecolor{currentstroke}%
\pgfsetdash{}{0pt}%
\pgfpathmoveto{\pgfqpoint{1.379542in}{0.522036in}}%
\pgfpathlineto{\pgfqpoint{1.392811in}{0.521946in}}%
\pgfpathlineto{\pgfqpoint{1.393676in}{0.522111in}}%
\pgfpathlineto{\pgfqpoint{1.406080in}{0.527401in}}%
\pgfpathlineto{\pgfqpoint{1.409678in}{0.535038in}}%
\pgfpathlineto{\pgfqpoint{1.411198in}{0.547964in}}%
\pgfpathlineto{\pgfqpoint{1.411221in}{0.560891in}}%
\pgfpathlineto{\pgfqpoint{1.409990in}{0.573817in}}%
\pgfpathlineto{\pgfqpoint{1.406080in}{0.584525in}}%
\pgfpathlineto{\pgfqpoint{1.403714in}{0.586744in}}%
\pgfpathlineto{\pgfqpoint{1.392811in}{0.591022in}}%
\pgfpathlineto{\pgfqpoint{1.379542in}{0.590946in}}%
\pgfpathlineto{\pgfqpoint{1.369024in}{0.586744in}}%
\pgfpathlineto{\pgfqpoint{1.366273in}{0.584180in}}%
\pgfpathlineto{\pgfqpoint{1.362417in}{0.573817in}}%
\pgfpathlineto{\pgfqpoint{1.361126in}{0.560891in}}%
\pgfpathlineto{\pgfqpoint{1.361170in}{0.547964in}}%
\pgfpathlineto{\pgfqpoint{1.362815in}{0.535038in}}%
\pgfpathlineto{\pgfqpoint{1.366273in}{0.527835in}}%
\pgfpathlineto{\pgfqpoint{1.379173in}{0.522111in}}%
\pgfpathclose%
\pgfpathmoveto{\pgfqpoint{1.374590in}{0.535038in}}%
\pgfpathlineto{\pgfqpoint{1.367952in}{0.547964in}}%
\pgfpathlineto{\pgfqpoint{1.367347in}{0.560891in}}%
\pgfpathlineto{\pgfqpoint{1.371367in}{0.573817in}}%
\pgfpathlineto{\pgfqpoint{1.379542in}{0.581593in}}%
\pgfpathlineto{\pgfqpoint{1.392811in}{0.581618in}}%
\pgfpathlineto{\pgfqpoint{1.400977in}{0.573817in}}%
\pgfpathlineto{\pgfqpoint{1.404921in}{0.560891in}}%
\pgfpathlineto{\pgfqpoint{1.404340in}{0.547964in}}%
\pgfpathlineto{\pgfqpoint{1.397815in}{0.535038in}}%
\pgfpathlineto{\pgfqpoint{1.392811in}{0.531532in}}%
\pgfpathlineto{\pgfqpoint{1.379542in}{0.531564in}}%
\pgfpathclose%
\pgfusepath{fill}%
\end{pgfscope}%
\begin{pgfscope}%
\pgfpathrectangle{\pgfqpoint{0.211875in}{0.211875in}}{\pgfqpoint{1.313625in}{1.279725in}}%
\pgfusepath{clip}%
\pgfsetbuttcap%
\pgfsetroundjoin%
\definecolor{currentfill}{rgb}{0.796501,0.105066,0.310630}%
\pgfsetfillcolor{currentfill}%
\pgfsetlinewidth{0.000000pt}%
\definecolor{currentstroke}{rgb}{0.000000,0.000000,0.000000}%
\pgfsetstrokecolor{currentstroke}%
\pgfsetdash{}{0pt}%
\pgfpathmoveto{\pgfqpoint{1.498962in}{0.520540in}}%
\pgfpathlineto{\pgfqpoint{1.512231in}{0.520608in}}%
\pgfpathlineto{\pgfqpoint{1.519343in}{0.522111in}}%
\pgfpathlineto{\pgfqpoint{1.525500in}{0.525446in}}%
\pgfpathlineto{\pgfqpoint{1.525500in}{0.535038in}}%
\pgfpathlineto{\pgfqpoint{1.525500in}{0.547964in}}%
\pgfpathlineto{\pgfqpoint{1.525500in}{0.560891in}}%
\pgfpathlineto{\pgfqpoint{1.525500in}{0.573817in}}%
\pgfpathlineto{\pgfqpoint{1.525500in}{0.585948in}}%
\pgfpathlineto{\pgfqpoint{1.524825in}{0.586744in}}%
\pgfpathlineto{\pgfqpoint{1.512231in}{0.592259in}}%
\pgfpathlineto{\pgfqpoint{1.498962in}{0.592438in}}%
\pgfpathlineto{\pgfqpoint{1.485693in}{0.588365in}}%
\pgfpathlineto{\pgfqpoint{1.484056in}{0.586744in}}%
\pgfpathlineto{\pgfqpoint{1.479322in}{0.573817in}}%
\pgfpathlineto{\pgfqpoint{1.478219in}{0.560891in}}%
\pgfpathlineto{\pgfqpoint{1.478209in}{0.547964in}}%
\pgfpathlineto{\pgfqpoint{1.479463in}{0.535038in}}%
\pgfpathlineto{\pgfqpoint{1.485693in}{0.523831in}}%
\pgfpathlineto{\pgfqpoint{1.490471in}{0.522111in}}%
\pgfpathclose%
\pgfpathmoveto{\pgfqpoint{1.490302in}{0.535038in}}%
\pgfpathlineto{\pgfqpoint{1.485693in}{0.543021in}}%
\pgfpathlineto{\pgfqpoint{1.484493in}{0.547964in}}%
\pgfpathlineto{\pgfqpoint{1.484147in}{0.560891in}}%
\pgfpathlineto{\pgfqpoint{1.485693in}{0.569632in}}%
\pgfpathlineto{\pgfqpoint{1.487205in}{0.573817in}}%
\pgfpathlineto{\pgfqpoint{1.498962in}{0.583730in}}%
\pgfpathlineto{\pgfqpoint{1.512231in}{0.582820in}}%
\pgfpathlineto{\pgfqpoint{1.520641in}{0.573817in}}%
\pgfpathlineto{\pgfqpoint{1.523942in}{0.560891in}}%
\pgfpathlineto{\pgfqpoint{1.523494in}{0.547964in}}%
\pgfpathlineto{\pgfqpoint{1.518153in}{0.535038in}}%
\pgfpathlineto{\pgfqpoint{1.512231in}{0.530385in}}%
\pgfpathlineto{\pgfqpoint{1.498962in}{0.529651in}}%
\pgfpathclose%
\pgfusepath{fill}%
\end{pgfscope}%
\begin{pgfscope}%
\pgfpathrectangle{\pgfqpoint{0.211875in}{0.211875in}}{\pgfqpoint{1.313625in}{1.279725in}}%
\pgfusepath{clip}%
\pgfsetbuttcap%
\pgfsetroundjoin%
\definecolor{currentfill}{rgb}{0.796501,0.105066,0.310630}%
\pgfsetfillcolor{currentfill}%
\pgfsetlinewidth{0.000000pt}%
\definecolor{currentstroke}{rgb}{0.000000,0.000000,0.000000}%
\pgfsetstrokecolor{currentstroke}%
\pgfsetdash{}{0pt}%
\pgfpathmoveto{\pgfqpoint{1.021280in}{0.523733in}}%
\pgfpathlineto{\pgfqpoint{1.034549in}{0.522555in}}%
\pgfpathlineto{\pgfqpoint{1.047818in}{0.525764in}}%
\pgfpathlineto{\pgfqpoint{1.054400in}{0.535038in}}%
\pgfpathlineto{\pgfqpoint{1.056409in}{0.547964in}}%
\pgfpathlineto{\pgfqpoint{1.056487in}{0.560891in}}%
\pgfpathlineto{\pgfqpoint{1.054984in}{0.573817in}}%
\pgfpathlineto{\pgfqpoint{1.048315in}{0.586744in}}%
\pgfpathlineto{\pgfqpoint{1.047818in}{0.587133in}}%
\pgfpathlineto{\pgfqpoint{1.034549in}{0.590570in}}%
\pgfpathlineto{\pgfqpoint{1.021280in}{0.589265in}}%
\pgfpathlineto{\pgfqpoint{1.016670in}{0.586744in}}%
\pgfpathlineto{\pgfqpoint{1.009095in}{0.573817in}}%
\pgfpathlineto{\pgfqpoint{1.008011in}{0.566186in}}%
\pgfpathlineto{\pgfqpoint{1.007518in}{0.560891in}}%
\pgfpathlineto{\pgfqpoint{1.007581in}{0.547964in}}%
\pgfpathlineto{\pgfqpoint{1.008011in}{0.543954in}}%
\pgfpathlineto{\pgfqpoint{1.009717in}{0.535038in}}%
\pgfpathclose%
\pgfpathmoveto{\pgfqpoint{1.020775in}{0.535038in}}%
\pgfpathlineto{\pgfqpoint{1.015143in}{0.547964in}}%
\pgfpathlineto{\pgfqpoint{1.014624in}{0.560891in}}%
\pgfpathlineto{\pgfqpoint{1.017994in}{0.573817in}}%
\pgfpathlineto{\pgfqpoint{1.021280in}{0.578160in}}%
\pgfpathlineto{\pgfqpoint{1.034549in}{0.581646in}}%
\pgfpathlineto{\pgfqpoint{1.046764in}{0.573817in}}%
\pgfpathlineto{\pgfqpoint{1.047818in}{0.571882in}}%
\pgfpathlineto{\pgfqpoint{1.050235in}{0.560891in}}%
\pgfpathlineto{\pgfqpoint{1.049779in}{0.547964in}}%
\pgfpathlineto{\pgfqpoint{1.047818in}{0.541649in}}%
\pgfpathlineto{\pgfqpoint{1.041881in}{0.535038in}}%
\pgfpathlineto{\pgfqpoint{1.034549in}{0.531586in}}%
\pgfpathlineto{\pgfqpoint{1.021280in}{0.534544in}}%
\pgfpathclose%
\pgfusepath{fill}%
\end{pgfscope}%
\begin{pgfscope}%
\pgfpathrectangle{\pgfqpoint{0.211875in}{0.211875in}}{\pgfqpoint{1.313625in}{1.279725in}}%
\pgfusepath{clip}%
\pgfsetbuttcap%
\pgfsetroundjoin%
\definecolor{currentfill}{rgb}{0.796501,0.105066,0.310630}%
\pgfsetfillcolor{currentfill}%
\pgfsetlinewidth{0.000000pt}%
\definecolor{currentstroke}{rgb}{0.000000,0.000000,0.000000}%
\pgfsetstrokecolor{currentstroke}%
\pgfsetdash{}{0pt}%
\pgfpathmoveto{\pgfqpoint{1.140701in}{0.523763in}}%
\pgfpathlineto{\pgfqpoint{1.153970in}{0.522931in}}%
\pgfpathlineto{\pgfqpoint{1.167239in}{0.527124in}}%
\pgfpathlineto{\pgfqpoint{1.172205in}{0.535038in}}%
\pgfpathlineto{\pgfqpoint{1.174221in}{0.547964in}}%
\pgfpathlineto{\pgfqpoint{1.174304in}{0.560891in}}%
\pgfpathlineto{\pgfqpoint{1.172812in}{0.573817in}}%
\pgfpathlineto{\pgfqpoint{1.167239in}{0.585537in}}%
\pgfpathlineto{\pgfqpoint{1.165260in}{0.586744in}}%
\pgfpathlineto{\pgfqpoint{1.153970in}{0.590223in}}%
\pgfpathlineto{\pgfqpoint{1.140701in}{0.589322in}}%
\pgfpathlineto{\pgfqpoint{1.135480in}{0.586744in}}%
\pgfpathlineto{\pgfqpoint{1.127432in}{0.574798in}}%
\pgfpathlineto{\pgfqpoint{1.127170in}{0.573817in}}%
\pgfpathlineto{\pgfqpoint{1.125747in}{0.560891in}}%
\pgfpathlineto{\pgfqpoint{1.125827in}{0.547964in}}%
\pgfpathlineto{\pgfqpoint{1.127432in}{0.536593in}}%
\pgfpathlineto{\pgfqpoint{1.127850in}{0.535038in}}%
\pgfpathclose%
\pgfpathmoveto{\pgfqpoint{1.139567in}{0.535038in}}%
\pgfpathlineto{\pgfqpoint{1.133447in}{0.547964in}}%
\pgfpathlineto{\pgfqpoint{1.132876in}{0.560891in}}%
\pgfpathlineto{\pgfqpoint{1.136527in}{0.573817in}}%
\pgfpathlineto{\pgfqpoint{1.140701in}{0.578776in}}%
\pgfpathlineto{\pgfqpoint{1.153970in}{0.581080in}}%
\pgfpathlineto{\pgfqpoint{1.163756in}{0.573817in}}%
\pgfpathlineto{\pgfqpoint{1.167239in}{0.565845in}}%
\pgfpathlineto{\pgfqpoint{1.168212in}{0.560891in}}%
\pgfpathlineto{\pgfqpoint{1.167758in}{0.547964in}}%
\pgfpathlineto{\pgfqpoint{1.167239in}{0.546064in}}%
\pgfpathlineto{\pgfqpoint{1.159394in}{0.535038in}}%
\pgfpathlineto{\pgfqpoint{1.153970in}{0.532085in}}%
\pgfpathlineto{\pgfqpoint{1.140701in}{0.534043in}}%
\pgfpathclose%
\pgfusepath{fill}%
\end{pgfscope}%
\begin{pgfscope}%
\pgfpathrectangle{\pgfqpoint{0.211875in}{0.211875in}}{\pgfqpoint{1.313625in}{1.279725in}}%
\pgfusepath{clip}%
\pgfsetbuttcap%
\pgfsetroundjoin%
\definecolor{currentfill}{rgb}{0.796501,0.105066,0.310630}%
\pgfsetfillcolor{currentfill}%
\pgfsetlinewidth{0.000000pt}%
\definecolor{currentstroke}{rgb}{0.000000,0.000000,0.000000}%
\pgfsetstrokecolor{currentstroke}%
\pgfsetdash{}{0pt}%
\pgfpathmoveto{\pgfqpoint{1.246852in}{0.531855in}}%
\pgfpathlineto{\pgfqpoint{1.260121in}{0.523173in}}%
\pgfpathlineto{\pgfqpoint{1.273390in}{0.522735in}}%
\pgfpathlineto{\pgfqpoint{1.286659in}{0.527770in}}%
\pgfpathlineto{\pgfqpoint{1.290645in}{0.535038in}}%
\pgfpathlineto{\pgfqpoint{1.292497in}{0.547964in}}%
\pgfpathlineto{\pgfqpoint{1.292562in}{0.560891in}}%
\pgfpathlineto{\pgfqpoint{1.291160in}{0.573817in}}%
\pgfpathlineto{\pgfqpoint{1.286659in}{0.584546in}}%
\pgfpathlineto{\pgfqpoint{1.283749in}{0.586744in}}%
\pgfpathlineto{\pgfqpoint{1.273390in}{0.590362in}}%
\pgfpathlineto{\pgfqpoint{1.260121in}{0.589899in}}%
\pgfpathlineto{\pgfqpoint{1.253031in}{0.586744in}}%
\pgfpathlineto{\pgfqpoint{1.246852in}{0.579553in}}%
\pgfpathlineto{\pgfqpoint{1.245020in}{0.573817in}}%
\pgfpathlineto{\pgfqpoint{1.243624in}{0.560891in}}%
\pgfpathlineto{\pgfqpoint{1.243696in}{0.547964in}}%
\pgfpathlineto{\pgfqpoint{1.245563in}{0.535038in}}%
\pgfpathclose%
\pgfpathmoveto{\pgfqpoint{1.257573in}{0.535038in}}%
\pgfpathlineto{\pgfqpoint{1.251110in}{0.547964in}}%
\pgfpathlineto{\pgfqpoint{1.250509in}{0.560891in}}%
\pgfpathlineto{\pgfqpoint{1.254380in}{0.573817in}}%
\pgfpathlineto{\pgfqpoint{1.260121in}{0.579936in}}%
\pgfpathlineto{\pgfqpoint{1.273390in}{0.581059in}}%
\pgfpathlineto{\pgfqpoint{1.281947in}{0.573817in}}%
\pgfpathlineto{\pgfqpoint{1.286532in}{0.560891in}}%
\pgfpathlineto{\pgfqpoint{1.285827in}{0.547964in}}%
\pgfpathlineto{\pgfqpoint{1.278160in}{0.535038in}}%
\pgfpathlineto{\pgfqpoint{1.273390in}{0.532078in}}%
\pgfpathlineto{\pgfqpoint{1.260121in}{0.533035in}}%
\pgfpathclose%
\pgfusepath{fill}%
\end{pgfscope}%
\begin{pgfscope}%
\pgfpathrectangle{\pgfqpoint{0.211875in}{0.211875in}}{\pgfqpoint{1.313625in}{1.279725in}}%
\pgfusepath{clip}%
\pgfsetbuttcap%
\pgfsetroundjoin%
\definecolor{currentfill}{rgb}{0.796501,0.105066,0.310630}%
\pgfsetfillcolor{currentfill}%
\pgfsetlinewidth{0.000000pt}%
\definecolor{currentstroke}{rgb}{0.000000,0.000000,0.000000}%
\pgfsetstrokecolor{currentstroke}%
\pgfsetdash{}{0pt}%
\pgfpathmoveto{\pgfqpoint{0.596674in}{0.604222in}}%
\pgfpathlineto{\pgfqpoint{0.609943in}{0.600631in}}%
\pgfpathlineto{\pgfqpoint{0.623212in}{0.600606in}}%
\pgfpathlineto{\pgfqpoint{0.636481in}{0.602675in}}%
\pgfpathlineto{\pgfqpoint{0.644434in}{0.612597in}}%
\pgfpathlineto{\pgfqpoint{0.645959in}{0.625523in}}%
\pgfpathlineto{\pgfqpoint{0.646146in}{0.638450in}}%
\pgfpathlineto{\pgfqpoint{0.645556in}{0.651377in}}%
\pgfpathlineto{\pgfqpoint{0.643168in}{0.664303in}}%
\pgfpathlineto{\pgfqpoint{0.636481in}{0.672316in}}%
\pgfpathlineto{\pgfqpoint{0.623212in}{0.675125in}}%
\pgfpathlineto{\pgfqpoint{0.609943in}{0.674911in}}%
\pgfpathlineto{\pgfqpoint{0.596674in}{0.669198in}}%
\pgfpathlineto{\pgfqpoint{0.594356in}{0.664303in}}%
\pgfpathlineto{\pgfqpoint{0.592509in}{0.651377in}}%
\pgfpathlineto{\pgfqpoint{0.592036in}{0.638450in}}%
\pgfpathlineto{\pgfqpoint{0.592106in}{0.625523in}}%
\pgfpathlineto{\pgfqpoint{0.593057in}{0.612597in}}%
\pgfpathclose%
\pgfpathmoveto{\pgfqpoint{0.606524in}{0.612597in}}%
\pgfpathlineto{\pgfqpoint{0.599473in}{0.625523in}}%
\pgfpathlineto{\pgfqpoint{0.598255in}{0.638450in}}%
\pgfpathlineto{\pgfqpoint{0.599939in}{0.651377in}}%
\pgfpathlineto{\pgfqpoint{0.607933in}{0.664303in}}%
\pgfpathlineto{\pgfqpoint{0.609943in}{0.665608in}}%
\pgfpathlineto{\pgfqpoint{0.623212in}{0.666849in}}%
\pgfpathlineto{\pgfqpoint{0.629143in}{0.664303in}}%
\pgfpathlineto{\pgfqpoint{0.636481in}{0.657504in}}%
\pgfpathlineto{\pgfqpoint{0.638798in}{0.651377in}}%
\pgfpathlineto{\pgfqpoint{0.640183in}{0.638450in}}%
\pgfpathlineto{\pgfqpoint{0.639153in}{0.625523in}}%
\pgfpathlineto{\pgfqpoint{0.636481in}{0.617955in}}%
\pgfpathlineto{\pgfqpoint{0.631016in}{0.612597in}}%
\pgfpathlineto{\pgfqpoint{0.623212in}{0.609190in}}%
\pgfpathlineto{\pgfqpoint{0.609943in}{0.610331in}}%
\pgfpathclose%
\pgfusepath{fill}%
\end{pgfscope}%
\begin{pgfscope}%
\pgfpathrectangle{\pgfqpoint{0.211875in}{0.211875in}}{\pgfqpoint{1.313625in}{1.279725in}}%
\pgfusepath{clip}%
\pgfsetbuttcap%
\pgfsetroundjoin%
\definecolor{currentfill}{rgb}{0.796501,0.105066,0.310630}%
\pgfsetfillcolor{currentfill}%
\pgfsetlinewidth{0.000000pt}%
\definecolor{currentstroke}{rgb}{0.000000,0.000000,0.000000}%
\pgfsetstrokecolor{currentstroke}%
\pgfsetdash{}{0pt}%
\pgfpathmoveto{\pgfqpoint{0.716095in}{0.608843in}}%
\pgfpathlineto{\pgfqpoint{0.729364in}{0.603044in}}%
\pgfpathlineto{\pgfqpoint{0.742633in}{0.602971in}}%
\pgfpathlineto{\pgfqpoint{0.755902in}{0.607209in}}%
\pgfpathlineto{\pgfqpoint{0.759691in}{0.612597in}}%
\pgfpathlineto{\pgfqpoint{0.762224in}{0.625523in}}%
\pgfpathlineto{\pgfqpoint{0.762615in}{0.638450in}}%
\pgfpathlineto{\pgfqpoint{0.761841in}{0.651377in}}%
\pgfpathlineto{\pgfqpoint{0.758500in}{0.664303in}}%
\pgfpathlineto{\pgfqpoint{0.755902in}{0.667825in}}%
\pgfpathlineto{\pgfqpoint{0.742633in}{0.672818in}}%
\pgfpathlineto{\pgfqpoint{0.729364in}{0.672661in}}%
\pgfpathlineto{\pgfqpoint{0.716095in}{0.665636in}}%
\pgfpathlineto{\pgfqpoint{0.715336in}{0.664303in}}%
\pgfpathlineto{\pgfqpoint{0.712336in}{0.651377in}}%
\pgfpathlineto{\pgfqpoint{0.711640in}{0.638450in}}%
\pgfpathlineto{\pgfqpoint{0.711947in}{0.625523in}}%
\pgfpathlineto{\pgfqpoint{0.714104in}{0.612597in}}%
\pgfpathclose%
\pgfpathmoveto{\pgfqpoint{0.728934in}{0.612597in}}%
\pgfpathlineto{\pgfqpoint{0.719688in}{0.625523in}}%
\pgfpathlineto{\pgfqpoint{0.718049in}{0.638450in}}%
\pgfpathlineto{\pgfqpoint{0.720169in}{0.651377in}}%
\pgfpathlineto{\pgfqpoint{0.729364in}{0.663503in}}%
\pgfpathlineto{\pgfqpoint{0.740915in}{0.664303in}}%
\pgfpathlineto{\pgfqpoint{0.742633in}{0.664384in}}%
\pgfpathlineto{\pgfqpoint{0.742799in}{0.664303in}}%
\pgfpathlineto{\pgfqpoint{0.754764in}{0.651377in}}%
\pgfpathlineto{\pgfqpoint{0.755902in}{0.646019in}}%
\pgfpathlineto{\pgfqpoint{0.756765in}{0.638450in}}%
\pgfpathlineto{\pgfqpoint{0.755902in}{0.628957in}}%
\pgfpathlineto{\pgfqpoint{0.755316in}{0.625523in}}%
\pgfpathlineto{\pgfqpoint{0.744394in}{0.612597in}}%
\pgfpathlineto{\pgfqpoint{0.742633in}{0.611726in}}%
\pgfpathlineto{\pgfqpoint{0.729364in}{0.612342in}}%
\pgfpathclose%
\pgfusepath{fill}%
\end{pgfscope}%
\begin{pgfscope}%
\pgfpathrectangle{\pgfqpoint{0.211875in}{0.211875in}}{\pgfqpoint{1.313625in}{1.279725in}}%
\pgfusepath{clip}%
\pgfsetbuttcap%
\pgfsetroundjoin%
\definecolor{currentfill}{rgb}{0.796501,0.105066,0.310630}%
\pgfsetfillcolor{currentfill}%
\pgfsetlinewidth{0.000000pt}%
\definecolor{currentstroke}{rgb}{0.000000,0.000000,0.000000}%
\pgfsetstrokecolor{currentstroke}%
\pgfsetdash{}{0pt}%
\pgfpathmoveto{\pgfqpoint{0.835515in}{0.610895in}}%
\pgfpathlineto{\pgfqpoint{0.848784in}{0.604705in}}%
\pgfpathlineto{\pgfqpoint{0.862053in}{0.604876in}}%
\pgfpathlineto{\pgfqpoint{0.875322in}{0.611920in}}%
\pgfpathlineto{\pgfqpoint{0.875733in}{0.612597in}}%
\pgfpathlineto{\pgfqpoint{0.879002in}{0.625523in}}%
\pgfpathlineto{\pgfqpoint{0.879538in}{0.638450in}}%
\pgfpathlineto{\pgfqpoint{0.878631in}{0.651377in}}%
\pgfpathlineto{\pgfqpoint{0.875322in}{0.662744in}}%
\pgfpathlineto{\pgfqpoint{0.874224in}{0.664303in}}%
\pgfpathlineto{\pgfqpoint{0.862053in}{0.670946in}}%
\pgfpathlineto{\pgfqpoint{0.848784in}{0.671112in}}%
\pgfpathlineto{\pgfqpoint{0.835696in}{0.664303in}}%
\pgfpathlineto{\pgfqpoint{0.835515in}{0.664058in}}%
\pgfpathlineto{\pgfqpoint{0.831742in}{0.651377in}}%
\pgfpathlineto{\pgfqpoint{0.830872in}{0.638450in}}%
\pgfpathlineto{\pgfqpoint{0.831361in}{0.625523in}}%
\pgfpathlineto{\pgfqpoint{0.834446in}{0.612597in}}%
\pgfpathclose%
\pgfpathmoveto{\pgfqpoint{0.839447in}{0.625523in}}%
\pgfpathlineto{\pgfqpoint{0.837395in}{0.638450in}}%
\pgfpathlineto{\pgfqpoint{0.839954in}{0.651377in}}%
\pgfpathlineto{\pgfqpoint{0.848784in}{0.661753in}}%
\pgfpathlineto{\pgfqpoint{0.862053in}{0.661516in}}%
\pgfpathlineto{\pgfqpoint{0.870420in}{0.651377in}}%
\pgfpathlineto{\pgfqpoint{0.872962in}{0.638450in}}%
\pgfpathlineto{\pgfqpoint{0.870909in}{0.625523in}}%
\pgfpathlineto{\pgfqpoint{0.862053in}{0.614373in}}%
\pgfpathlineto{\pgfqpoint{0.848784in}{0.614125in}}%
\pgfpathclose%
\pgfusepath{fill}%
\end{pgfscope}%
\begin{pgfscope}%
\pgfpathrectangle{\pgfqpoint{0.211875in}{0.211875in}}{\pgfqpoint{1.313625in}{1.279725in}}%
\pgfusepath{clip}%
\pgfsetbuttcap%
\pgfsetroundjoin%
\definecolor{currentfill}{rgb}{0.796501,0.105066,0.310630}%
\pgfsetfillcolor{currentfill}%
\pgfsetlinewidth{0.000000pt}%
\definecolor{currentstroke}{rgb}{0.000000,0.000000,0.000000}%
\pgfsetstrokecolor{currentstroke}%
\pgfsetdash{}{0pt}%
\pgfpathmoveto{\pgfqpoint{0.954936in}{0.611450in}}%
\pgfpathlineto{\pgfqpoint{0.968205in}{0.605722in}}%
\pgfpathlineto{\pgfqpoint{0.981473in}{0.606321in}}%
\pgfpathlineto{\pgfqpoint{0.991578in}{0.612597in}}%
\pgfpathlineto{\pgfqpoint{0.994742in}{0.618828in}}%
\pgfpathlineto{\pgfqpoint{0.996232in}{0.625523in}}%
\pgfpathlineto{\pgfqpoint{0.996862in}{0.638450in}}%
\pgfpathlineto{\pgfqpoint{0.995869in}{0.651377in}}%
\pgfpathlineto{\pgfqpoint{0.994742in}{0.655897in}}%
\pgfpathlineto{\pgfqpoint{0.990024in}{0.664303in}}%
\pgfpathlineto{\pgfqpoint{0.981473in}{0.669511in}}%
\pgfpathlineto{\pgfqpoint{0.968205in}{0.670168in}}%
\pgfpathlineto{\pgfqpoint{0.955490in}{0.664303in}}%
\pgfpathlineto{\pgfqpoint{0.954936in}{0.663702in}}%
\pgfpathlineto{\pgfqpoint{0.950729in}{0.651377in}}%
\pgfpathlineto{\pgfqpoint{0.949731in}{0.638450in}}%
\pgfpathlineto{\pgfqpoint{0.950351in}{0.625523in}}%
\pgfpathlineto{\pgfqpoint{0.954101in}{0.612597in}}%
\pgfpathclose%
\pgfpathmoveto{\pgfqpoint{0.958640in}{0.625523in}}%
\pgfpathlineto{\pgfqpoint{0.956175in}{0.638450in}}%
\pgfpathlineto{\pgfqpoint{0.959188in}{0.651377in}}%
\pgfpathlineto{\pgfqpoint{0.968205in}{0.660745in}}%
\pgfpathlineto{\pgfqpoint{0.981473in}{0.659125in}}%
\pgfpathlineto{\pgfqpoint{0.987184in}{0.651377in}}%
\pgfpathlineto{\pgfqpoint{0.989660in}{0.638450in}}%
\pgfpathlineto{\pgfqpoint{0.987627in}{0.625523in}}%
\pgfpathlineto{\pgfqpoint{0.981473in}{0.616831in}}%
\pgfpathlineto{\pgfqpoint{0.968205in}{0.615216in}}%
\pgfpathclose%
\pgfusepath{fill}%
\end{pgfscope}%
\begin{pgfscope}%
\pgfpathrectangle{\pgfqpoint{0.211875in}{0.211875in}}{\pgfqpoint{1.313625in}{1.279725in}}%
\pgfusepath{clip}%
\pgfsetbuttcap%
\pgfsetroundjoin%
\definecolor{currentfill}{rgb}{0.796501,0.105066,0.310630}%
\pgfsetfillcolor{currentfill}%
\pgfsetlinewidth{0.000000pt}%
\definecolor{currentstroke}{rgb}{0.000000,0.000000,0.000000}%
\pgfsetstrokecolor{currentstroke}%
\pgfsetdash{}{0pt}%
\pgfpathmoveto{\pgfqpoint{1.074356in}{0.611038in}}%
\pgfpathlineto{\pgfqpoint{1.087625in}{0.606164in}}%
\pgfpathlineto{\pgfqpoint{1.100894in}{0.607284in}}%
\pgfpathlineto{\pgfqpoint{1.108576in}{0.612597in}}%
\pgfpathlineto{\pgfqpoint{1.113784in}{0.625523in}}%
\pgfpathlineto{\pgfqpoint{1.114163in}{0.630659in}}%
\pgfpathlineto{\pgfqpoint{1.114547in}{0.638450in}}%
\pgfpathlineto{\pgfqpoint{1.114163in}{0.643641in}}%
\pgfpathlineto{\pgfqpoint{1.113312in}{0.651377in}}%
\pgfpathlineto{\pgfqpoint{1.107136in}{0.664303in}}%
\pgfpathlineto{\pgfqpoint{1.100894in}{0.668529in}}%
\pgfpathlineto{\pgfqpoint{1.087625in}{0.669767in}}%
\pgfpathlineto{\pgfqpoint{1.074356in}{0.664447in}}%
\pgfpathlineto{\pgfqpoint{1.074232in}{0.664303in}}%
\pgfpathlineto{\pgfqpoint{1.069283in}{0.651377in}}%
\pgfpathlineto{\pgfqpoint{1.068200in}{0.638450in}}%
\pgfpathlineto{\pgfqpoint{1.068901in}{0.625523in}}%
\pgfpathlineto{\pgfqpoint{1.073063in}{0.612597in}}%
\pgfpathclose%
\pgfpathmoveto{\pgfqpoint{1.077082in}{0.625523in}}%
\pgfpathlineto{\pgfqpoint{1.074356in}{0.637624in}}%
\pgfpathlineto{\pgfqpoint{1.074267in}{0.638450in}}%
\pgfpathlineto{\pgfqpoint{1.074356in}{0.639129in}}%
\pgfpathlineto{\pgfqpoint{1.077690in}{0.651377in}}%
\pgfpathlineto{\pgfqpoint{1.087625in}{0.660403in}}%
\pgfpathlineto{\pgfqpoint{1.100894in}{0.657253in}}%
\pgfpathlineto{\pgfqpoint{1.104778in}{0.651377in}}%
\pgfpathlineto{\pgfqpoint{1.107149in}{0.638450in}}%
\pgfpathlineto{\pgfqpoint{1.105189in}{0.625523in}}%
\pgfpathlineto{\pgfqpoint{1.100894in}{0.618742in}}%
\pgfpathlineto{\pgfqpoint{1.087625in}{0.615598in}}%
\pgfpathclose%
\pgfusepath{fill}%
\end{pgfscope}%
\begin{pgfscope}%
\pgfpathrectangle{\pgfqpoint{0.211875in}{0.211875in}}{\pgfqpoint{1.313625in}{1.279725in}}%
\pgfusepath{clip}%
\pgfsetbuttcap%
\pgfsetroundjoin%
\definecolor{currentfill}{rgb}{0.796501,0.105066,0.310630}%
\pgfsetfillcolor{currentfill}%
\pgfsetlinewidth{0.000000pt}%
\definecolor{currentstroke}{rgb}{0.000000,0.000000,0.000000}%
\pgfsetstrokecolor{currentstroke}%
\pgfsetdash{}{0pt}%
\pgfpathmoveto{\pgfqpoint{1.193777in}{0.609944in}}%
\pgfpathlineto{\pgfqpoint{1.207045in}{0.606075in}}%
\pgfpathlineto{\pgfqpoint{1.220314in}{0.607717in}}%
\pgfpathlineto{\pgfqpoint{1.226659in}{0.612597in}}%
\pgfpathlineto{\pgfqpoint{1.231522in}{0.625523in}}%
\pgfpathlineto{\pgfqpoint{1.232342in}{0.638450in}}%
\pgfpathlineto{\pgfqpoint{1.231076in}{0.651377in}}%
\pgfpathlineto{\pgfqpoint{1.225293in}{0.664303in}}%
\pgfpathlineto{\pgfqpoint{1.220314in}{0.668044in}}%
\pgfpathlineto{\pgfqpoint{1.207045in}{0.669869in}}%
\pgfpathlineto{\pgfqpoint{1.193777in}{0.665633in}}%
\pgfpathlineto{\pgfqpoint{1.192489in}{0.664303in}}%
\pgfpathlineto{\pgfqpoint{1.187368in}{0.651377in}}%
\pgfpathlineto{\pgfqpoint{1.186246in}{0.638450in}}%
\pgfpathlineto{\pgfqpoint{1.186977in}{0.625523in}}%
\pgfpathlineto{\pgfqpoint{1.191295in}{0.612597in}}%
\pgfpathclose%
\pgfpathmoveto{\pgfqpoint{1.194446in}{0.625523in}}%
\pgfpathlineto{\pgfqpoint{1.193777in}{0.627852in}}%
\pgfpathlineto{\pgfqpoint{1.192478in}{0.638450in}}%
\pgfpathlineto{\pgfqpoint{1.193777in}{0.647354in}}%
\pgfpathlineto{\pgfqpoint{1.195143in}{0.651377in}}%
\pgfpathlineto{\pgfqpoint{1.207045in}{0.660682in}}%
\pgfpathlineto{\pgfqpoint{1.220314in}{0.655937in}}%
\pgfpathlineto{\pgfqpoint{1.223024in}{0.651377in}}%
\pgfpathlineto{\pgfqpoint{1.225252in}{0.638450in}}%
\pgfpathlineto{\pgfqpoint{1.223412in}{0.625523in}}%
\pgfpathlineto{\pgfqpoint{1.220314in}{0.620065in}}%
\pgfpathlineto{\pgfqpoint{1.207045in}{0.615323in}}%
\pgfpathclose%
\pgfusepath{fill}%
\end{pgfscope}%
\begin{pgfscope}%
\pgfpathrectangle{\pgfqpoint{0.211875in}{0.211875in}}{\pgfqpoint{1.313625in}{1.279725in}}%
\pgfusepath{clip}%
\pgfsetbuttcap%
\pgfsetroundjoin%
\definecolor{currentfill}{rgb}{0.796501,0.105066,0.310630}%
\pgfsetfillcolor{currentfill}%
\pgfsetlinewidth{0.000000pt}%
\definecolor{currentstroke}{rgb}{0.000000,0.000000,0.000000}%
\pgfsetstrokecolor{currentstroke}%
\pgfsetdash{}{0pt}%
\pgfpathmoveto{\pgfqpoint{1.313197in}{0.608328in}}%
\pgfpathlineto{\pgfqpoint{1.326466in}{0.605472in}}%
\pgfpathlineto{\pgfqpoint{1.339735in}{0.607537in}}%
\pgfpathlineto{\pgfqpoint{1.345645in}{0.612597in}}%
\pgfpathlineto{\pgfqpoint{1.349901in}{0.625523in}}%
\pgfpathlineto{\pgfqpoint{1.350605in}{0.638450in}}%
\pgfpathlineto{\pgfqpoint{1.349472in}{0.651377in}}%
\pgfpathlineto{\pgfqpoint{1.344321in}{0.664303in}}%
\pgfpathlineto{\pgfqpoint{1.339735in}{0.668127in}}%
\pgfpathlineto{\pgfqpoint{1.326466in}{0.670457in}}%
\pgfpathlineto{\pgfqpoint{1.313197in}{0.667292in}}%
\pgfpathlineto{\pgfqpoint{1.309974in}{0.664303in}}%
\pgfpathlineto{\pgfqpoint{1.304930in}{0.651377in}}%
\pgfpathlineto{\pgfqpoint{1.303815in}{0.638450in}}%
\pgfpathlineto{\pgfqpoint{1.304522in}{0.625523in}}%
\pgfpathlineto{\pgfqpoint{1.308728in}{0.612597in}}%
\pgfpathclose%
\pgfpathmoveto{\pgfqpoint{1.311892in}{0.625523in}}%
\pgfpathlineto{\pgfqpoint{1.310263in}{0.638450in}}%
\pgfpathlineto{\pgfqpoint{1.312246in}{0.651377in}}%
\pgfpathlineto{\pgfqpoint{1.313197in}{0.653339in}}%
\pgfpathlineto{\pgfqpoint{1.326466in}{0.661563in}}%
\pgfpathlineto{\pgfqpoint{1.339735in}{0.655247in}}%
\pgfpathlineto{\pgfqpoint{1.341803in}{0.651377in}}%
\pgfpathlineto{\pgfqpoint{1.343856in}{0.638450in}}%
\pgfpathlineto{\pgfqpoint{1.342176in}{0.625523in}}%
\pgfpathlineto{\pgfqpoint{1.339735in}{0.620719in}}%
\pgfpathlineto{\pgfqpoint{1.326466in}{0.614412in}}%
\pgfpathlineto{\pgfqpoint{1.313197in}{0.622679in}}%
\pgfpathclose%
\pgfusepath{fill}%
\end{pgfscope}%
\begin{pgfscope}%
\pgfpathrectangle{\pgfqpoint{0.211875in}{0.211875in}}{\pgfqpoint{1.313625in}{1.279725in}}%
\pgfusepath{clip}%
\pgfsetbuttcap%
\pgfsetroundjoin%
\definecolor{currentfill}{rgb}{0.796501,0.105066,0.310630}%
\pgfsetfillcolor{currentfill}%
\pgfsetlinewidth{0.000000pt}%
\definecolor{currentstroke}{rgb}{0.000000,0.000000,0.000000}%
\pgfsetstrokecolor{currentstroke}%
\pgfsetdash{}{0pt}%
\pgfpathmoveto{\pgfqpoint{1.432617in}{0.606282in}}%
\pgfpathlineto{\pgfqpoint{1.445886in}{0.604353in}}%
\pgfpathlineto{\pgfqpoint{1.459155in}{0.606613in}}%
\pgfpathlineto{\pgfqpoint{1.465417in}{0.612597in}}%
\pgfpathlineto{\pgfqpoint{1.468824in}{0.625523in}}%
\pgfpathlineto{\pgfqpoint{1.469364in}{0.638450in}}%
\pgfpathlineto{\pgfqpoint{1.468404in}{0.651377in}}%
\pgfpathlineto{\pgfqpoint{1.464106in}{0.664303in}}%
\pgfpathlineto{\pgfqpoint{1.459155in}{0.668895in}}%
\pgfpathlineto{\pgfqpoint{1.445886in}{0.671536in}}%
\pgfpathlineto{\pgfqpoint{1.432617in}{0.669348in}}%
\pgfpathlineto{\pgfqpoint{1.426578in}{0.664303in}}%
\pgfpathlineto{\pgfqpoint{1.421883in}{0.651377in}}%
\pgfpathlineto{\pgfqpoint{1.420828in}{0.638450in}}%
\pgfpathlineto{\pgfqpoint{1.421452in}{0.625523in}}%
\pgfpathlineto{\pgfqpoint{1.425252in}{0.612597in}}%
\pgfpathclose%
\pgfpathmoveto{\pgfqpoint{1.429133in}{0.625523in}}%
\pgfpathlineto{\pgfqpoint{1.427553in}{0.638450in}}%
\pgfpathlineto{\pgfqpoint{1.429509in}{0.651377in}}%
\pgfpathlineto{\pgfqpoint{1.432617in}{0.657142in}}%
\pgfpathlineto{\pgfqpoint{1.445886in}{0.663051in}}%
\pgfpathlineto{\pgfqpoint{1.459155in}{0.655300in}}%
\pgfpathlineto{\pgfqpoint{1.461038in}{0.651377in}}%
\pgfpathlineto{\pgfqpoint{1.462884in}{0.638450in}}%
\pgfpathlineto{\pgfqpoint{1.461404in}{0.625523in}}%
\pgfpathlineto{\pgfqpoint{1.459155in}{0.620569in}}%
\pgfpathlineto{\pgfqpoint{1.445886in}{0.612860in}}%
\pgfpathlineto{\pgfqpoint{1.432617in}{0.618732in}}%
\pgfpathclose%
\pgfusepath{fill}%
\end{pgfscope}%
\begin{pgfscope}%
\pgfpathrectangle{\pgfqpoint{0.211875in}{0.211875in}}{\pgfqpoint{1.313625in}{1.279725in}}%
\pgfusepath{clip}%
\pgfsetbuttcap%
\pgfsetroundjoin%
\definecolor{currentfill}{rgb}{0.796501,0.105066,0.310630}%
\pgfsetfillcolor{currentfill}%
\pgfsetlinewidth{0.000000pt}%
\definecolor{currentstroke}{rgb}{0.000000,0.000000,0.000000}%
\pgfsetstrokecolor{currentstroke}%
\pgfsetdash{}{0pt}%
\pgfpathmoveto{\pgfqpoint{0.543598in}{0.683251in}}%
\pgfpathlineto{\pgfqpoint{0.556867in}{0.682061in}}%
\pgfpathlineto{\pgfqpoint{0.570136in}{0.682746in}}%
\pgfpathlineto{\pgfqpoint{0.583405in}{0.689089in}}%
\pgfpathlineto{\pgfqpoint{0.584007in}{0.690156in}}%
\pgfpathlineto{\pgfqpoint{0.586714in}{0.703083in}}%
\pgfpathlineto{\pgfqpoint{0.587208in}{0.716009in}}%
\pgfpathlineto{\pgfqpoint{0.586899in}{0.728936in}}%
\pgfpathlineto{\pgfqpoint{0.585354in}{0.741862in}}%
\pgfpathlineto{\pgfqpoint{0.583405in}{0.747350in}}%
\pgfpathlineto{\pgfqpoint{0.573539in}{0.754789in}}%
\pgfpathlineto{\pgfqpoint{0.570136in}{0.755731in}}%
\pgfpathlineto{\pgfqpoint{0.556867in}{0.756526in}}%
\pgfpathlineto{\pgfqpoint{0.543951in}{0.754789in}}%
\pgfpathlineto{\pgfqpoint{0.543598in}{0.754705in}}%
\pgfpathlineto{\pgfqpoint{0.534898in}{0.741862in}}%
\pgfpathlineto{\pgfqpoint{0.533611in}{0.728936in}}%
\pgfpathlineto{\pgfqpoint{0.533325in}{0.716009in}}%
\pgfpathlineto{\pgfqpoint{0.533639in}{0.703083in}}%
\pgfpathlineto{\pgfqpoint{0.535612in}{0.690156in}}%
\pgfpathclose%
\pgfpathmoveto{\pgfqpoint{0.541382in}{0.703083in}}%
\pgfpathlineto{\pgfqpoint{0.539539in}{0.716009in}}%
\pgfpathlineto{\pgfqpoint{0.540163in}{0.728936in}}%
\pgfpathlineto{\pgfqpoint{0.543598in}{0.740231in}}%
\pgfpathlineto{\pgfqpoint{0.544917in}{0.741862in}}%
\pgfpathlineto{\pgfqpoint{0.556867in}{0.747914in}}%
\pgfpathlineto{\pgfqpoint{0.570136in}{0.745978in}}%
\pgfpathlineto{\pgfqpoint{0.574794in}{0.741862in}}%
\pgfpathlineto{\pgfqpoint{0.580084in}{0.728936in}}%
\pgfpathlineto{\pgfqpoint{0.580883in}{0.716009in}}%
\pgfpathlineto{\pgfqpoint{0.578467in}{0.703083in}}%
\pgfpathlineto{\pgfqpoint{0.570136in}{0.692530in}}%
\pgfpathlineto{\pgfqpoint{0.556867in}{0.690372in}}%
\pgfpathlineto{\pgfqpoint{0.543598in}{0.698259in}}%
\pgfpathclose%
\pgfusepath{fill}%
\end{pgfscope}%
\begin{pgfscope}%
\pgfpathrectangle{\pgfqpoint{0.211875in}{0.211875in}}{\pgfqpoint{1.313625in}{1.279725in}}%
\pgfusepath{clip}%
\pgfsetbuttcap%
\pgfsetroundjoin%
\definecolor{currentfill}{rgb}{0.796501,0.105066,0.310630}%
\pgfsetfillcolor{currentfill}%
\pgfsetlinewidth{0.000000pt}%
\definecolor{currentstroke}{rgb}{0.000000,0.000000,0.000000}%
\pgfsetstrokecolor{currentstroke}%
\pgfsetdash{}{0pt}%
\pgfpathmoveto{\pgfqpoint{0.663019in}{0.686610in}}%
\pgfpathlineto{\pgfqpoint{0.676288in}{0.684485in}}%
\pgfpathlineto{\pgfqpoint{0.689557in}{0.685795in}}%
\pgfpathlineto{\pgfqpoint{0.697029in}{0.690156in}}%
\pgfpathlineto{\pgfqpoint{0.702633in}{0.703083in}}%
\pgfpathlineto{\pgfqpoint{0.702826in}{0.705013in}}%
\pgfpathlineto{\pgfqpoint{0.703550in}{0.716009in}}%
\pgfpathlineto{\pgfqpoint{0.703153in}{0.728936in}}%
\pgfpathlineto{\pgfqpoint{0.702826in}{0.731431in}}%
\pgfpathlineto{\pgfqpoint{0.700442in}{0.741862in}}%
\pgfpathlineto{\pgfqpoint{0.689557in}{0.752536in}}%
\pgfpathlineto{\pgfqpoint{0.676288in}{0.754162in}}%
\pgfpathlineto{\pgfqpoint{0.663019in}{0.751361in}}%
\pgfpathlineto{\pgfqpoint{0.655783in}{0.741862in}}%
\pgfpathlineto{\pgfqpoint{0.653555in}{0.728936in}}%
\pgfpathlineto{\pgfqpoint{0.653153in}{0.716009in}}%
\pgfpathlineto{\pgfqpoint{0.653962in}{0.703083in}}%
\pgfpathlineto{\pgfqpoint{0.658397in}{0.690156in}}%
\pgfpathclose%
\pgfpathmoveto{\pgfqpoint{0.661906in}{0.703083in}}%
\pgfpathlineto{\pgfqpoint{0.659533in}{0.716009in}}%
\pgfpathlineto{\pgfqpoint{0.660280in}{0.728936in}}%
\pgfpathlineto{\pgfqpoint{0.663019in}{0.736943in}}%
\pgfpathlineto{\pgfqpoint{0.668025in}{0.741862in}}%
\pgfpathlineto{\pgfqpoint{0.676288in}{0.745456in}}%
\pgfpathlineto{\pgfqpoint{0.689557in}{0.742186in}}%
\pgfpathlineto{\pgfqpoint{0.689887in}{0.741862in}}%
\pgfpathlineto{\pgfqpoint{0.695772in}{0.728936in}}%
\pgfpathlineto{\pgfqpoint{0.696632in}{0.716009in}}%
\pgfpathlineto{\pgfqpoint{0.693875in}{0.703083in}}%
\pgfpathlineto{\pgfqpoint{0.689557in}{0.696992in}}%
\pgfpathlineto{\pgfqpoint{0.676288in}{0.693299in}}%
\pgfpathlineto{\pgfqpoint{0.663019in}{0.700943in}}%
\pgfpathclose%
\pgfusepath{fill}%
\end{pgfscope}%
\begin{pgfscope}%
\pgfpathrectangle{\pgfqpoint{0.211875in}{0.211875in}}{\pgfqpoint{1.313625in}{1.279725in}}%
\pgfusepath{clip}%
\pgfsetbuttcap%
\pgfsetroundjoin%
\definecolor{currentfill}{rgb}{0.796501,0.105066,0.310630}%
\pgfsetfillcolor{currentfill}%
\pgfsetlinewidth{0.000000pt}%
\definecolor{currentstroke}{rgb}{0.000000,0.000000,0.000000}%
\pgfsetstrokecolor{currentstroke}%
\pgfsetdash{}{0pt}%
\pgfpathmoveto{\pgfqpoint{0.782439in}{0.688710in}}%
\pgfpathlineto{\pgfqpoint{0.795708in}{0.686348in}}%
\pgfpathlineto{\pgfqpoint{0.808977in}{0.688544in}}%
\pgfpathlineto{\pgfqpoint{0.811470in}{0.690156in}}%
\pgfpathlineto{\pgfqpoint{0.818500in}{0.703083in}}%
\pgfpathlineto{\pgfqpoint{0.819878in}{0.716009in}}%
\pgfpathlineto{\pgfqpoint{0.819337in}{0.728936in}}%
\pgfpathlineto{\pgfqpoint{0.816036in}{0.741862in}}%
\pgfpathlineto{\pgfqpoint{0.808977in}{0.749545in}}%
\pgfpathlineto{\pgfqpoint{0.795708in}{0.752181in}}%
\pgfpathlineto{\pgfqpoint{0.782439in}{0.749276in}}%
\pgfpathlineto{\pgfqpoint{0.776128in}{0.741862in}}%
\pgfpathlineto{\pgfqpoint{0.773092in}{0.728936in}}%
\pgfpathlineto{\pgfqpoint{0.772590in}{0.716009in}}%
\pgfpathlineto{\pgfqpoint{0.773820in}{0.703083in}}%
\pgfpathlineto{\pgfqpoint{0.780331in}{0.690156in}}%
\pgfpathclose%
\pgfpathmoveto{\pgfqpoint{0.782031in}{0.703083in}}%
\pgfpathlineto{\pgfqpoint{0.779189in}{0.716009in}}%
\pgfpathlineto{\pgfqpoint{0.780048in}{0.728936in}}%
\pgfpathlineto{\pgfqpoint{0.782439in}{0.735191in}}%
\pgfpathlineto{\pgfqpoint{0.791136in}{0.741862in}}%
\pgfpathlineto{\pgfqpoint{0.795708in}{0.743530in}}%
\pgfpathlineto{\pgfqpoint{0.800910in}{0.741862in}}%
\pgfpathlineto{\pgfqpoint{0.808977in}{0.736764in}}%
\pgfpathlineto{\pgfqpoint{0.812237in}{0.728936in}}%
\pgfpathlineto{\pgfqpoint{0.813135in}{0.716009in}}%
\pgfpathlineto{\pgfqpoint{0.810154in}{0.703083in}}%
\pgfpathlineto{\pgfqpoint{0.808977in}{0.701233in}}%
\pgfpathlineto{\pgfqpoint{0.795708in}{0.695588in}}%
\pgfpathlineto{\pgfqpoint{0.782439in}{0.702383in}}%
\pgfpathclose%
\pgfusepath{fill}%
\end{pgfscope}%
\begin{pgfscope}%
\pgfpathrectangle{\pgfqpoint{0.211875in}{0.211875in}}{\pgfqpoint{1.313625in}{1.279725in}}%
\pgfusepath{clip}%
\pgfsetbuttcap%
\pgfsetroundjoin%
\definecolor{currentfill}{rgb}{0.796501,0.105066,0.310630}%
\pgfsetfillcolor{currentfill}%
\pgfsetlinewidth{0.000000pt}%
\definecolor{currentstroke}{rgb}{0.000000,0.000000,0.000000}%
\pgfsetstrokecolor{currentstroke}%
\pgfsetdash{}{0pt}%
\pgfpathmoveto{\pgfqpoint{0.901860in}{0.689871in}}%
\pgfpathlineto{\pgfqpoint{0.915129in}{0.687699in}}%
\pgfpathlineto{\pgfqpoint{0.925766in}{0.690156in}}%
\pgfpathlineto{\pgfqpoint{0.928398in}{0.691212in}}%
\pgfpathlineto{\pgfqpoint{0.935158in}{0.703083in}}%
\pgfpathlineto{\pgfqpoint{0.936716in}{0.716009in}}%
\pgfpathlineto{\pgfqpoint{0.936145in}{0.728936in}}%
\pgfpathlineto{\pgfqpoint{0.932534in}{0.741862in}}%
\pgfpathlineto{\pgfqpoint{0.928398in}{0.746869in}}%
\pgfpathlineto{\pgfqpoint{0.915129in}{0.750739in}}%
\pgfpathlineto{\pgfqpoint{0.901860in}{0.748143in}}%
\pgfpathlineto{\pgfqpoint{0.895908in}{0.741862in}}%
\pgfpathlineto{\pgfqpoint{0.892194in}{0.728936in}}%
\pgfpathlineto{\pgfqpoint{0.891604in}{0.716009in}}%
\pgfpathlineto{\pgfqpoint{0.893183in}{0.703083in}}%
\pgfpathlineto{\pgfqpoint{0.901398in}{0.690156in}}%
\pgfpathclose%
\pgfpathmoveto{\pgfqpoint{0.901737in}{0.703083in}}%
\pgfpathlineto{\pgfqpoint{0.898484in}{0.716009in}}%
\pgfpathlineto{\pgfqpoint{0.899445in}{0.728936in}}%
\pgfpathlineto{\pgfqpoint{0.901860in}{0.734606in}}%
\pgfpathlineto{\pgfqpoint{0.914364in}{0.741862in}}%
\pgfpathlineto{\pgfqpoint{0.915129in}{0.742088in}}%
\pgfpathlineto{\pgfqpoint{0.915706in}{0.741862in}}%
\pgfpathlineto{\pgfqpoint{0.928398in}{0.731421in}}%
\pgfpathlineto{\pgfqpoint{0.929327in}{0.728936in}}%
\pgfpathlineto{\pgfqpoint{0.930243in}{0.716009in}}%
\pgfpathlineto{\pgfqpoint{0.928398in}{0.707675in}}%
\pgfpathlineto{\pgfqpoint{0.925510in}{0.703083in}}%
\pgfpathlineto{\pgfqpoint{0.915129in}{0.697296in}}%
\pgfpathlineto{\pgfqpoint{0.901860in}{0.702894in}}%
\pgfpathclose%
\pgfusepath{fill}%
\end{pgfscope}%
\begin{pgfscope}%
\pgfpathrectangle{\pgfqpoint{0.211875in}{0.211875in}}{\pgfqpoint{1.313625in}{1.279725in}}%
\pgfusepath{clip}%
\pgfsetbuttcap%
\pgfsetroundjoin%
\definecolor{currentfill}{rgb}{0.796501,0.105066,0.310630}%
\pgfsetfillcolor{currentfill}%
\pgfsetlinewidth{0.000000pt}%
\definecolor{currentstroke}{rgb}{0.000000,0.000000,0.000000}%
\pgfsetstrokecolor{currentstroke}%
\pgfsetdash{}{0pt}%
\pgfpathmoveto{\pgfqpoint{1.034549in}{0.688564in}}%
\pgfpathlineto{\pgfqpoint{1.040439in}{0.690156in}}%
\pgfpathlineto{\pgfqpoint{1.047818in}{0.693900in}}%
\pgfpathlineto{\pgfqpoint{1.052476in}{0.703083in}}%
\pgfpathlineto{\pgfqpoint{1.054115in}{0.716009in}}%
\pgfpathlineto{\pgfqpoint{1.053533in}{0.728936in}}%
\pgfpathlineto{\pgfqpoint{1.049792in}{0.741862in}}%
\pgfpathlineto{\pgfqpoint{1.047818in}{0.744530in}}%
\pgfpathlineto{\pgfqpoint{1.034549in}{0.749809in}}%
\pgfpathlineto{\pgfqpoint{1.021280in}{0.747762in}}%
\pgfpathlineto{\pgfqpoint{1.015066in}{0.741862in}}%
\pgfpathlineto{\pgfqpoint{1.010805in}{0.728936in}}%
\pgfpathlineto{\pgfqpoint{1.010141in}{0.716009in}}%
\pgfpathlineto{\pgfqpoint{1.011996in}{0.703083in}}%
\pgfpathlineto{\pgfqpoint{1.021280in}{0.690338in}}%
\pgfpathlineto{\pgfqpoint{1.022210in}{0.690156in}}%
\pgfpathclose%
\pgfpathmoveto{\pgfqpoint{1.020984in}{0.703083in}}%
\pgfpathlineto{\pgfqpoint{1.017378in}{0.716009in}}%
\pgfpathlineto{\pgfqpoint{1.018430in}{0.728936in}}%
\pgfpathlineto{\pgfqpoint{1.021280in}{0.734953in}}%
\pgfpathlineto{\pgfqpoint{1.034549in}{0.740675in}}%
\pgfpathlineto{\pgfqpoint{1.046040in}{0.728936in}}%
\pgfpathlineto{\pgfqpoint{1.047818in}{0.716506in}}%
\pgfpathlineto{\pgfqpoint{1.047852in}{0.716009in}}%
\pgfpathlineto{\pgfqpoint{1.047818in}{0.715839in}}%
\pgfpathlineto{\pgfqpoint{1.041559in}{0.703083in}}%
\pgfpathlineto{\pgfqpoint{1.034549in}{0.698459in}}%
\pgfpathlineto{\pgfqpoint{1.021280in}{0.702676in}}%
\pgfpathclose%
\pgfusepath{fill}%
\end{pgfscope}%
\begin{pgfscope}%
\pgfpathrectangle{\pgfqpoint{0.211875in}{0.211875in}}{\pgfqpoint{1.313625in}{1.279725in}}%
\pgfusepath{clip}%
\pgfsetbuttcap%
\pgfsetroundjoin%
\definecolor{currentfill}{rgb}{0.796501,0.105066,0.310630}%
\pgfsetfillcolor{currentfill}%
\pgfsetlinewidth{0.000000pt}%
\definecolor{currentstroke}{rgb}{0.000000,0.000000,0.000000}%
\pgfsetstrokecolor{currentstroke}%
\pgfsetdash{}{0pt}%
\pgfpathmoveto{\pgfqpoint{1.140701in}{0.690126in}}%
\pgfpathlineto{\pgfqpoint{1.153970in}{0.688950in}}%
\pgfpathlineto{\pgfqpoint{1.157852in}{0.690156in}}%
\pgfpathlineto{\pgfqpoint{1.167239in}{0.696113in}}%
\pgfpathlineto{\pgfqpoint{1.170363in}{0.703083in}}%
\pgfpathlineto{\pgfqpoint{1.171992in}{0.716009in}}%
\pgfpathlineto{\pgfqpoint{1.171418in}{0.728936in}}%
\pgfpathlineto{\pgfqpoint{1.167708in}{0.741862in}}%
\pgfpathlineto{\pgfqpoint{1.167239in}{0.742576in}}%
\pgfpathlineto{\pgfqpoint{1.153970in}{0.749382in}}%
\pgfpathlineto{\pgfqpoint{1.140701in}{0.748004in}}%
\pgfpathlineto{\pgfqpoint{1.133510in}{0.741862in}}%
\pgfpathlineto{\pgfqpoint{1.128839in}{0.728936in}}%
\pgfpathlineto{\pgfqpoint{1.128117in}{0.716009in}}%
\pgfpathlineto{\pgfqpoint{1.130167in}{0.703083in}}%
\pgfpathlineto{\pgfqpoint{1.140641in}{0.690156in}}%
\pgfpathclose%
\pgfpathmoveto{\pgfqpoint{1.139703in}{0.703083in}}%
\pgfpathlineto{\pgfqpoint{1.135801in}{0.716009in}}%
\pgfpathlineto{\pgfqpoint{1.136934in}{0.728936in}}%
\pgfpathlineto{\pgfqpoint{1.140701in}{0.736078in}}%
\pgfpathlineto{\pgfqpoint{1.153970in}{0.739834in}}%
\pgfpathlineto{\pgfqpoint{1.163172in}{0.728936in}}%
\pgfpathlineto{\pgfqpoint{1.164799in}{0.716009in}}%
\pgfpathlineto{\pgfqpoint{1.159199in}{0.703083in}}%
\pgfpathlineto{\pgfqpoint{1.153970in}{0.699087in}}%
\pgfpathlineto{\pgfqpoint{1.140701in}{0.701854in}}%
\pgfpathclose%
\pgfusepath{fill}%
\end{pgfscope}%
\begin{pgfscope}%
\pgfpathrectangle{\pgfqpoint{0.211875in}{0.211875in}}{\pgfqpoint{1.313625in}{1.279725in}}%
\pgfusepath{clip}%
\pgfsetbuttcap%
\pgfsetroundjoin%
\definecolor{currentfill}{rgb}{0.796501,0.105066,0.310630}%
\pgfsetfillcolor{currentfill}%
\pgfsetlinewidth{0.000000pt}%
\definecolor{currentstroke}{rgb}{0.000000,0.000000,0.000000}%
\pgfsetstrokecolor{currentstroke}%
\pgfsetdash{}{0pt}%
\pgfpathmoveto{\pgfqpoint{1.260121in}{0.689436in}}%
\pgfpathlineto{\pgfqpoint{1.273390in}{0.688845in}}%
\pgfpathlineto{\pgfqpoint{1.277112in}{0.690156in}}%
\pgfpathlineto{\pgfqpoint{1.286659in}{0.697725in}}%
\pgfpathlineto{\pgfqpoint{1.288758in}{0.703083in}}%
\pgfpathlineto{\pgfqpoint{1.290293in}{0.716009in}}%
\pgfpathlineto{\pgfqpoint{1.289744in}{0.728936in}}%
\pgfpathlineto{\pgfqpoint{1.286659in}{0.740685in}}%
\pgfpathlineto{\pgfqpoint{1.285933in}{0.741862in}}%
\pgfpathlineto{\pgfqpoint{1.273390in}{0.749473in}}%
\pgfpathlineto{\pgfqpoint{1.260121in}{0.748784in}}%
\pgfpathlineto{\pgfqpoint{1.251091in}{0.741862in}}%
\pgfpathlineto{\pgfqpoint{1.246852in}{0.731354in}}%
\pgfpathlineto{\pgfqpoint{1.246368in}{0.728936in}}%
\pgfpathlineto{\pgfqpoint{1.245828in}{0.716009in}}%
\pgfpathlineto{\pgfqpoint{1.246852in}{0.706725in}}%
\pgfpathlineto{\pgfqpoint{1.247558in}{0.703083in}}%
\pgfpathlineto{\pgfqpoint{1.258522in}{0.690156in}}%
\pgfpathclose%
\pgfpathmoveto{\pgfqpoint{1.257782in}{0.703083in}}%
\pgfpathlineto{\pgfqpoint{1.253647in}{0.716009in}}%
\pgfpathlineto{\pgfqpoint{1.254853in}{0.728936in}}%
\pgfpathlineto{\pgfqpoint{1.260121in}{0.737888in}}%
\pgfpathlineto{\pgfqpoint{1.273390in}{0.739703in}}%
\pgfpathlineto{\pgfqpoint{1.281355in}{0.728936in}}%
\pgfpathlineto{\pgfqpoint{1.282786in}{0.716009in}}%
\pgfpathlineto{\pgfqpoint{1.277877in}{0.703083in}}%
\pgfpathlineto{\pgfqpoint{1.273390in}{0.699169in}}%
\pgfpathlineto{\pgfqpoint{1.260121in}{0.700506in}}%
\pgfpathclose%
\pgfusepath{fill}%
\end{pgfscope}%
\begin{pgfscope}%
\pgfpathrectangle{\pgfqpoint{0.211875in}{0.211875in}}{\pgfqpoint{1.313625in}{1.279725in}}%
\pgfusepath{clip}%
\pgfsetbuttcap%
\pgfsetroundjoin%
\definecolor{currentfill}{rgb}{0.796501,0.105066,0.310630}%
\pgfsetfillcolor{currentfill}%
\pgfsetlinewidth{0.000000pt}%
\definecolor{currentstroke}{rgb}{0.000000,0.000000,0.000000}%
\pgfsetstrokecolor{currentstroke}%
\pgfsetdash{}{0pt}%
\pgfpathmoveto{\pgfqpoint{1.379542in}{0.688279in}}%
\pgfpathlineto{\pgfqpoint{1.392811in}{0.688214in}}%
\pgfpathlineto{\pgfqpoint{1.397715in}{0.690156in}}%
\pgfpathlineto{\pgfqpoint{1.406080in}{0.698499in}}%
\pgfpathlineto{\pgfqpoint{1.407624in}{0.703083in}}%
\pgfpathlineto{\pgfqpoint{1.408981in}{0.716009in}}%
\pgfpathlineto{\pgfqpoint{1.408474in}{0.728936in}}%
\pgfpathlineto{\pgfqpoint{1.406080in}{0.739418in}}%
\pgfpathlineto{\pgfqpoint{1.404871in}{0.741862in}}%
\pgfpathlineto{\pgfqpoint{1.392811in}{0.750114in}}%
\pgfpathlineto{\pgfqpoint{1.379542in}{0.750053in}}%
\pgfpathlineto{\pgfqpoint{1.367585in}{0.741862in}}%
\pgfpathlineto{\pgfqpoint{1.366273in}{0.739270in}}%
\pgfpathlineto{\pgfqpoint{1.363862in}{0.728936in}}%
\pgfpathlineto{\pgfqpoint{1.363345in}{0.716009in}}%
\pgfpathlineto{\pgfqpoint{1.364756in}{0.703083in}}%
\pgfpathlineto{\pgfqpoint{1.366273in}{0.698676in}}%
\pgfpathlineto{\pgfqpoint{1.374892in}{0.690156in}}%
\pgfpathclose%
\pgfpathmoveto{\pgfqpoint{1.375050in}{0.703083in}}%
\pgfpathlineto{\pgfqpoint{1.370756in}{0.716009in}}%
\pgfpathlineto{\pgfqpoint{1.372023in}{0.728936in}}%
\pgfpathlineto{\pgfqpoint{1.379542in}{0.740330in}}%
\pgfpathlineto{\pgfqpoint{1.392811in}{0.740328in}}%
\pgfpathlineto{\pgfqpoint{1.400270in}{0.728936in}}%
\pgfpathlineto{\pgfqpoint{1.401519in}{0.716009in}}%
\pgfpathlineto{\pgfqpoint{1.397288in}{0.703083in}}%
\pgfpathlineto{\pgfqpoint{1.392811in}{0.698669in}}%
\pgfpathlineto{\pgfqpoint{1.379542in}{0.698673in}}%
\pgfpathclose%
\pgfusepath{fill}%
\end{pgfscope}%
\begin{pgfscope}%
\pgfpathrectangle{\pgfqpoint{0.211875in}{0.211875in}}{\pgfqpoint{1.313625in}{1.279725in}}%
\pgfusepath{clip}%
\pgfsetbuttcap%
\pgfsetroundjoin%
\definecolor{currentfill}{rgb}{0.796501,0.105066,0.310630}%
\pgfsetfillcolor{currentfill}%
\pgfsetlinewidth{0.000000pt}%
\definecolor{currentstroke}{rgb}{0.000000,0.000000,0.000000}%
\pgfsetstrokecolor{currentstroke}%
\pgfsetdash{}{0pt}%
\pgfpathmoveto{\pgfqpoint{1.498962in}{0.686679in}}%
\pgfpathlineto{\pgfqpoint{1.512231in}{0.686998in}}%
\pgfpathlineto{\pgfqpoint{1.519369in}{0.690156in}}%
\pgfpathlineto{\pgfqpoint{1.525500in}{0.697983in}}%
\pgfpathlineto{\pgfqpoint{1.525500in}{0.703083in}}%
\pgfpathlineto{\pgfqpoint{1.525500in}{0.716009in}}%
\pgfpathlineto{\pgfqpoint{1.525500in}{0.728936in}}%
\pgfpathlineto{\pgfqpoint{1.525500in}{0.739655in}}%
\pgfpathlineto{\pgfqpoint{1.524636in}{0.741862in}}%
\pgfpathlineto{\pgfqpoint{1.512231in}{0.751368in}}%
\pgfpathlineto{\pgfqpoint{1.498962in}{0.751787in}}%
\pgfpathlineto{\pgfqpoint{1.485693in}{0.745000in}}%
\pgfpathlineto{\pgfqpoint{1.483851in}{0.741862in}}%
\pgfpathlineto{\pgfqpoint{1.480942in}{0.728936in}}%
\pgfpathlineto{\pgfqpoint{1.480466in}{0.716009in}}%
\pgfpathlineto{\pgfqpoint{1.481680in}{0.703083in}}%
\pgfpathlineto{\pgfqpoint{1.485693in}{0.693015in}}%
\pgfpathlineto{\pgfqpoint{1.489306in}{0.690156in}}%
\pgfpathclose%
\pgfpathmoveto{\pgfqpoint{1.491240in}{0.703083in}}%
\pgfpathlineto{\pgfqpoint{1.486876in}{0.716009in}}%
\pgfpathlineto{\pgfqpoint{1.488192in}{0.728936in}}%
\pgfpathlineto{\pgfqpoint{1.497356in}{0.741862in}}%
\pgfpathlineto{\pgfqpoint{1.498962in}{0.742839in}}%
\pgfpathlineto{\pgfqpoint{1.511653in}{0.741862in}}%
\pgfpathlineto{\pgfqpoint{1.512231in}{0.741788in}}%
\pgfpathlineto{\pgfqpoint{1.519735in}{0.728936in}}%
\pgfpathlineto{\pgfqpoint{1.520809in}{0.716009in}}%
\pgfpathlineto{\pgfqpoint{1.517255in}{0.703083in}}%
\pgfpathlineto{\pgfqpoint{1.512231in}{0.697522in}}%
\pgfpathlineto{\pgfqpoint{1.498962in}{0.696373in}}%
\pgfpathclose%
\pgfusepath{fill}%
\end{pgfscope}%
\begin{pgfscope}%
\pgfpathrectangle{\pgfqpoint{0.211875in}{0.211875in}}{\pgfqpoint{1.313625in}{1.279725in}}%
\pgfusepath{clip}%
\pgfsetbuttcap%
\pgfsetroundjoin%
\definecolor{currentfill}{rgb}{0.796501,0.105066,0.310630}%
\pgfsetfillcolor{currentfill}%
\pgfsetlinewidth{0.000000pt}%
\definecolor{currentstroke}{rgb}{0.000000,0.000000,0.000000}%
\pgfsetstrokecolor{currentstroke}%
\pgfsetdash{}{0pt}%
\pgfpathmoveto{\pgfqpoint{0.490523in}{0.763398in}}%
\pgfpathlineto{\pgfqpoint{0.503792in}{0.763234in}}%
\pgfpathlineto{\pgfqpoint{0.517061in}{0.764994in}}%
\pgfpathlineto{\pgfqpoint{0.521950in}{0.767715in}}%
\pgfpathlineto{\pgfqpoint{0.527240in}{0.780642in}}%
\pgfpathlineto{\pgfqpoint{0.528141in}{0.793568in}}%
\pgfpathlineto{\pgfqpoint{0.528071in}{0.806495in}}%
\pgfpathlineto{\pgfqpoint{0.527015in}{0.819421in}}%
\pgfpathlineto{\pgfqpoint{0.522090in}{0.832348in}}%
\pgfpathlineto{\pgfqpoint{0.517061in}{0.835758in}}%
\pgfpathlineto{\pgfqpoint{0.503792in}{0.837894in}}%
\pgfpathlineto{\pgfqpoint{0.490523in}{0.837546in}}%
\pgfpathlineto{\pgfqpoint{0.478676in}{0.832348in}}%
\pgfpathlineto{\pgfqpoint{0.477254in}{0.830083in}}%
\pgfpathlineto{\pgfqpoint{0.474814in}{0.819421in}}%
\pgfpathlineto{\pgfqpoint{0.474079in}{0.806495in}}%
\pgfpathlineto{\pgfqpoint{0.473991in}{0.793568in}}%
\pgfpathlineto{\pgfqpoint{0.474498in}{0.780642in}}%
\pgfpathlineto{\pgfqpoint{0.477254in}{0.768769in}}%
\pgfpathlineto{\pgfqpoint{0.478061in}{0.767715in}}%
\pgfpathclose%
\pgfpathmoveto{\pgfqpoint{0.484243in}{0.780642in}}%
\pgfpathlineto{\pgfqpoint{0.480798in}{0.793568in}}%
\pgfpathlineto{\pgfqpoint{0.480750in}{0.806495in}}%
\pgfpathlineto{\pgfqpoint{0.483862in}{0.819421in}}%
\pgfpathlineto{\pgfqpoint{0.490523in}{0.827281in}}%
\pgfpathlineto{\pgfqpoint{0.503792in}{0.829564in}}%
\pgfpathlineto{\pgfqpoint{0.517061in}{0.822728in}}%
\pgfpathlineto{\pgfqpoint{0.518992in}{0.819421in}}%
\pgfpathlineto{\pgfqpoint{0.521816in}{0.806495in}}%
\pgfpathlineto{\pgfqpoint{0.521758in}{0.793568in}}%
\pgfpathlineto{\pgfqpoint{0.518609in}{0.780642in}}%
\pgfpathlineto{\pgfqpoint{0.517061in}{0.778195in}}%
\pgfpathlineto{\pgfqpoint{0.503792in}{0.771808in}}%
\pgfpathlineto{\pgfqpoint{0.490523in}{0.773883in}}%
\pgfpathclose%
\pgfusepath{fill}%
\end{pgfscope}%
\begin{pgfscope}%
\pgfpathrectangle{\pgfqpoint{0.211875in}{0.211875in}}{\pgfqpoint{1.313625in}{1.279725in}}%
\pgfusepath{clip}%
\pgfsetbuttcap%
\pgfsetroundjoin%
\definecolor{currentfill}{rgb}{0.796501,0.105066,0.310630}%
\pgfsetfillcolor{currentfill}%
\pgfsetlinewidth{0.000000pt}%
\definecolor{currentstroke}{rgb}{0.000000,0.000000,0.000000}%
\pgfsetstrokecolor{currentstroke}%
\pgfsetdash{}{0pt}%
\pgfpathmoveto{\pgfqpoint{0.609943in}{0.766357in}}%
\pgfpathlineto{\pgfqpoint{0.623212in}{0.765958in}}%
\pgfpathlineto{\pgfqpoint{0.631084in}{0.767715in}}%
\pgfpathlineto{\pgfqpoint{0.636481in}{0.769946in}}%
\pgfpathlineto{\pgfqpoint{0.642480in}{0.780642in}}%
\pgfpathlineto{\pgfqpoint{0.644051in}{0.793568in}}%
\pgfpathlineto{\pgfqpoint{0.644022in}{0.806495in}}%
\pgfpathlineto{\pgfqpoint{0.642448in}{0.819421in}}%
\pgfpathlineto{\pgfqpoint{0.636481in}{0.830902in}}%
\pgfpathlineto{\pgfqpoint{0.633692in}{0.832348in}}%
\pgfpathlineto{\pgfqpoint{0.623212in}{0.835214in}}%
\pgfpathlineto{\pgfqpoint{0.609943in}{0.834711in}}%
\pgfpathlineto{\pgfqpoint{0.603969in}{0.832348in}}%
\pgfpathlineto{\pgfqpoint{0.596674in}{0.823704in}}%
\pgfpathlineto{\pgfqpoint{0.595437in}{0.819421in}}%
\pgfpathlineto{\pgfqpoint{0.594099in}{0.806495in}}%
\pgfpathlineto{\pgfqpoint{0.594056in}{0.793568in}}%
\pgfpathlineto{\pgfqpoint{0.595335in}{0.780642in}}%
\pgfpathlineto{\pgfqpoint{0.596674in}{0.776185in}}%
\pgfpathlineto{\pgfqpoint{0.605600in}{0.767715in}}%
\pgfpathclose%
\pgfpathmoveto{\pgfqpoint{0.605926in}{0.780642in}}%
\pgfpathlineto{\pgfqpoint{0.601256in}{0.793568in}}%
\pgfpathlineto{\pgfqpoint{0.601148in}{0.806495in}}%
\pgfpathlineto{\pgfqpoint{0.605246in}{0.819421in}}%
\pgfpathlineto{\pgfqpoint{0.609943in}{0.824397in}}%
\pgfpathlineto{\pgfqpoint{0.623212in}{0.826204in}}%
\pgfpathlineto{\pgfqpoint{0.633140in}{0.819421in}}%
\pgfpathlineto{\pgfqpoint{0.636481in}{0.812955in}}%
\pgfpathlineto{\pgfqpoint{0.637927in}{0.806495in}}%
\pgfpathlineto{\pgfqpoint{0.637832in}{0.793568in}}%
\pgfpathlineto{\pgfqpoint{0.636481in}{0.788008in}}%
\pgfpathlineto{\pgfqpoint{0.632131in}{0.780642in}}%
\pgfpathlineto{\pgfqpoint{0.623212in}{0.775089in}}%
\pgfpathlineto{\pgfqpoint{0.609943in}{0.776767in}}%
\pgfpathclose%
\pgfusepath{fill}%
\end{pgfscope}%
\begin{pgfscope}%
\pgfpathrectangle{\pgfqpoint{0.211875in}{0.211875in}}{\pgfqpoint{1.313625in}{1.279725in}}%
\pgfusepath{clip}%
\pgfsetbuttcap%
\pgfsetroundjoin%
\definecolor{currentfill}{rgb}{0.796501,0.105066,0.310630}%
\pgfsetfillcolor{currentfill}%
\pgfsetlinewidth{0.000000pt}%
\definecolor{currentstroke}{rgb}{0.000000,0.000000,0.000000}%
\pgfsetstrokecolor{currentstroke}%
\pgfsetdash{}{0pt}%
\pgfpathmoveto{\pgfqpoint{0.716095in}{0.779378in}}%
\pgfpathlineto{\pgfqpoint{0.729364in}{0.768626in}}%
\pgfpathlineto{\pgfqpoint{0.742633in}{0.768335in}}%
\pgfpathlineto{\pgfqpoint{0.755902in}{0.775541in}}%
\pgfpathlineto{\pgfqpoint{0.758410in}{0.780642in}}%
\pgfpathlineto{\pgfqpoint{0.760496in}{0.793568in}}%
\pgfpathlineto{\pgfqpoint{0.760499in}{0.806495in}}%
\pgfpathlineto{\pgfqpoint{0.758527in}{0.819421in}}%
\pgfpathlineto{\pgfqpoint{0.755902in}{0.825150in}}%
\pgfpathlineto{\pgfqpoint{0.744601in}{0.832348in}}%
\pgfpathlineto{\pgfqpoint{0.742633in}{0.832953in}}%
\pgfpathlineto{\pgfqpoint{0.729364in}{0.832656in}}%
\pgfpathlineto{\pgfqpoint{0.728497in}{0.832348in}}%
\pgfpathlineto{\pgfqpoint{0.716095in}{0.820948in}}%
\pgfpathlineto{\pgfqpoint{0.715562in}{0.819421in}}%
\pgfpathlineto{\pgfqpoint{0.713739in}{0.806495in}}%
\pgfpathlineto{\pgfqpoint{0.713730in}{0.793568in}}%
\pgfpathlineto{\pgfqpoint{0.715629in}{0.780642in}}%
\pgfpathclose%
\pgfpathmoveto{\pgfqpoint{0.727249in}{0.780642in}}%
\pgfpathlineto{\pgfqpoint{0.721371in}{0.793568in}}%
\pgfpathlineto{\pgfqpoint{0.721206in}{0.806495in}}%
\pgfpathlineto{\pgfqpoint{0.726283in}{0.819421in}}%
\pgfpathlineto{\pgfqpoint{0.729364in}{0.822342in}}%
\pgfpathlineto{\pgfqpoint{0.742633in}{0.823286in}}%
\pgfpathlineto{\pgfqpoint{0.747605in}{0.819421in}}%
\pgfpathlineto{\pgfqpoint{0.753630in}{0.806495in}}%
\pgfpathlineto{\pgfqpoint{0.753426in}{0.793568in}}%
\pgfpathlineto{\pgfqpoint{0.746457in}{0.780642in}}%
\pgfpathlineto{\pgfqpoint{0.742633in}{0.777934in}}%
\pgfpathlineto{\pgfqpoint{0.729364in}{0.778817in}}%
\pgfpathclose%
\pgfusepath{fill}%
\end{pgfscope}%
\begin{pgfscope}%
\pgfpathrectangle{\pgfqpoint{0.211875in}{0.211875in}}{\pgfqpoint{1.313625in}{1.279725in}}%
\pgfusepath{clip}%
\pgfsetbuttcap%
\pgfsetroundjoin%
\definecolor{currentfill}{rgb}{0.796501,0.105066,0.310630}%
\pgfsetfillcolor{currentfill}%
\pgfsetlinewidth{0.000000pt}%
\definecolor{currentstroke}{rgb}{0.000000,0.000000,0.000000}%
\pgfsetstrokecolor{currentstroke}%
\pgfsetdash{}{0pt}%
\pgfpathmoveto{\pgfqpoint{0.835515in}{0.780403in}}%
\pgfpathlineto{\pgfqpoint{0.848784in}{0.770326in}}%
\pgfpathlineto{\pgfqpoint{0.862053in}{0.770516in}}%
\pgfpathlineto{\pgfqpoint{0.874744in}{0.780642in}}%
\pgfpathlineto{\pgfqpoint{0.875322in}{0.782186in}}%
\pgfpathlineto{\pgfqpoint{0.877400in}{0.793568in}}%
\pgfpathlineto{\pgfqpoint{0.877425in}{0.806495in}}%
\pgfpathlineto{\pgfqpoint{0.875322in}{0.818704in}}%
\pgfpathlineto{\pgfqpoint{0.875082in}{0.819421in}}%
\pgfpathlineto{\pgfqpoint{0.862053in}{0.830829in}}%
\pgfpathlineto{\pgfqpoint{0.848784in}{0.831023in}}%
\pgfpathlineto{\pgfqpoint{0.835515in}{0.820154in}}%
\pgfpathlineto{\pgfqpoint{0.835216in}{0.819421in}}%
\pgfpathlineto{\pgfqpoint{0.833012in}{0.806495in}}%
\pgfpathlineto{\pgfqpoint{0.833030in}{0.793568in}}%
\pgfpathlineto{\pgfqpoint{0.835411in}{0.780642in}}%
\pgfpathclose%
\pgfpathmoveto{\pgfqpoint{0.848179in}{0.780642in}}%
\pgfpathlineto{\pgfqpoint{0.841088in}{0.793568in}}%
\pgfpathlineto{\pgfqpoint{0.840869in}{0.806495in}}%
\pgfpathlineto{\pgfqpoint{0.846939in}{0.819421in}}%
\pgfpathlineto{\pgfqpoint{0.848784in}{0.820980in}}%
\pgfpathlineto{\pgfqpoint{0.862053in}{0.820780in}}%
\pgfpathlineto{\pgfqpoint{0.863604in}{0.819421in}}%
\pgfpathlineto{\pgfqpoint{0.869571in}{0.806495in}}%
\pgfpathlineto{\pgfqpoint{0.869351in}{0.793568in}}%
\pgfpathlineto{\pgfqpoint{0.862389in}{0.780642in}}%
\pgfpathlineto{\pgfqpoint{0.862053in}{0.780373in}}%
\pgfpathlineto{\pgfqpoint{0.848784in}{0.780177in}}%
\pgfpathclose%
\pgfusepath{fill}%
\end{pgfscope}%
\begin{pgfscope}%
\pgfpathrectangle{\pgfqpoint{0.211875in}{0.211875in}}{\pgfqpoint{1.313625in}{1.279725in}}%
\pgfusepath{clip}%
\pgfsetbuttcap%
\pgfsetroundjoin%
\definecolor{currentfill}{rgb}{0.796501,0.105066,0.310630}%
\pgfsetfillcolor{currentfill}%
\pgfsetlinewidth{0.000000pt}%
\definecolor{currentstroke}{rgb}{0.000000,0.000000,0.000000}%
\pgfsetstrokecolor{currentstroke}%
\pgfsetdash{}{0pt}%
\pgfpathmoveto{\pgfqpoint{0.954936in}{0.780155in}}%
\pgfpathlineto{\pgfqpoint{0.968205in}{0.771349in}}%
\pgfpathlineto{\pgfqpoint{0.981473in}{0.772221in}}%
\pgfpathlineto{\pgfqpoint{0.990899in}{0.780642in}}%
\pgfpathlineto{\pgfqpoint{0.994693in}{0.793568in}}%
\pgfpathlineto{\pgfqpoint{0.994742in}{0.805221in}}%
\pgfpathlineto{\pgfqpoint{0.994746in}{0.806495in}}%
\pgfpathlineto{\pgfqpoint{0.994742in}{0.806520in}}%
\pgfpathlineto{\pgfqpoint{0.991309in}{0.819421in}}%
\pgfpathlineto{\pgfqpoint{0.981473in}{0.829059in}}%
\pgfpathlineto{\pgfqpoint{0.968205in}{0.830005in}}%
\pgfpathlineto{\pgfqpoint{0.954936in}{0.820553in}}%
\pgfpathlineto{\pgfqpoint{0.954404in}{0.819421in}}%
\pgfpathlineto{\pgfqpoint{0.951918in}{0.806495in}}%
\pgfpathlineto{\pgfqpoint{0.951954in}{0.793568in}}%
\pgfpathlineto{\pgfqpoint{0.954690in}{0.780642in}}%
\pgfpathclose%
\pgfpathmoveto{\pgfqpoint{0.960310in}{0.793568in}}%
\pgfpathlineto{\pgfqpoint{0.960039in}{0.806495in}}%
\pgfpathlineto{\pgfqpoint{0.967142in}{0.819421in}}%
\pgfpathlineto{\pgfqpoint{0.968205in}{0.820216in}}%
\pgfpathlineto{\pgfqpoint{0.975230in}{0.819421in}}%
\pgfpathlineto{\pgfqpoint{0.981473in}{0.818058in}}%
\pgfpathlineto{\pgfqpoint{0.986513in}{0.806495in}}%
\pgfpathlineto{\pgfqpoint{0.986289in}{0.793568in}}%
\pgfpathlineto{\pgfqpoint{0.981473in}{0.783659in}}%
\pgfpathlineto{\pgfqpoint{0.968205in}{0.781152in}}%
\pgfpathclose%
\pgfusepath{fill}%
\end{pgfscope}%
\begin{pgfscope}%
\pgfpathrectangle{\pgfqpoint{0.211875in}{0.211875in}}{\pgfqpoint{1.313625in}{1.279725in}}%
\pgfusepath{clip}%
\pgfsetbuttcap%
\pgfsetroundjoin%
\definecolor{currentfill}{rgb}{0.796501,0.105066,0.310630}%
\pgfsetfillcolor{currentfill}%
\pgfsetlinewidth{0.000000pt}%
\definecolor{currentstroke}{rgb}{0.000000,0.000000,0.000000}%
\pgfsetstrokecolor{currentstroke}%
\pgfsetdash{}{0pt}%
\pgfpathmoveto{\pgfqpoint{1.074356in}{0.779073in}}%
\pgfpathlineto{\pgfqpoint{1.087625in}{0.771769in}}%
\pgfpathlineto{\pgfqpoint{1.100894in}{0.773432in}}%
\pgfpathlineto{\pgfqpoint{1.108128in}{0.780642in}}%
\pgfpathlineto{\pgfqpoint{1.111828in}{0.793568in}}%
\pgfpathlineto{\pgfqpoint{1.111887in}{0.806495in}}%
\pgfpathlineto{\pgfqpoint{1.108554in}{0.819421in}}%
\pgfpathlineto{\pgfqpoint{1.100894in}{0.827787in}}%
\pgfpathlineto{\pgfqpoint{1.087625in}{0.829593in}}%
\pgfpathlineto{\pgfqpoint{1.074356in}{0.821753in}}%
\pgfpathlineto{\pgfqpoint{1.073114in}{0.819421in}}%
\pgfpathlineto{\pgfqpoint{1.070441in}{0.806495in}}%
\pgfpathlineto{\pgfqpoint{1.070488in}{0.793568in}}%
\pgfpathlineto{\pgfqpoint{1.073452in}{0.780642in}}%
\pgfpathclose%
\pgfpathmoveto{\pgfqpoint{1.078875in}{0.793568in}}%
\pgfpathlineto{\pgfqpoint{1.078553in}{0.806495in}}%
\pgfpathlineto{\pgfqpoint{1.086762in}{0.819421in}}%
\pgfpathlineto{\pgfqpoint{1.087625in}{0.819986in}}%
\pgfpathlineto{\pgfqpoint{1.090357in}{0.819421in}}%
\pgfpathlineto{\pgfqpoint{1.100894in}{0.814961in}}%
\pgfpathlineto{\pgfqpoint{1.104201in}{0.806495in}}%
\pgfpathlineto{\pgfqpoint{1.103983in}{0.793568in}}%
\pgfpathlineto{\pgfqpoint{1.100894in}{0.786460in}}%
\pgfpathlineto{\pgfqpoint{1.087625in}{0.781553in}}%
\pgfpathclose%
\pgfusepath{fill}%
\end{pgfscope}%
\begin{pgfscope}%
\pgfpathrectangle{\pgfqpoint{0.211875in}{0.211875in}}{\pgfqpoint{1.313625in}{1.279725in}}%
\pgfusepath{clip}%
\pgfsetbuttcap%
\pgfsetroundjoin%
\definecolor{currentfill}{rgb}{0.796501,0.105066,0.310630}%
\pgfsetfillcolor{currentfill}%
\pgfsetlinewidth{0.000000pt}%
\definecolor{currentstroke}{rgb}{0.000000,0.000000,0.000000}%
\pgfsetstrokecolor{currentstroke}%
\pgfsetdash{}{0pt}%
\pgfpathmoveto{\pgfqpoint{1.193777in}{0.777396in}}%
\pgfpathlineto{\pgfqpoint{1.207045in}{0.771630in}}%
\pgfpathlineto{\pgfqpoint{1.220314in}{0.774103in}}%
\pgfpathlineto{\pgfqpoint{1.226204in}{0.780642in}}%
\pgfpathlineto{\pgfqpoint{1.229668in}{0.793568in}}%
\pgfpathlineto{\pgfqpoint{1.229723in}{0.806495in}}%
\pgfpathlineto{\pgfqpoint{1.226600in}{0.819421in}}%
\pgfpathlineto{\pgfqpoint{1.220314in}{0.827057in}}%
\pgfpathlineto{\pgfqpoint{1.207045in}{0.829747in}}%
\pgfpathlineto{\pgfqpoint{1.193777in}{0.823535in}}%
\pgfpathlineto{\pgfqpoint{1.191313in}{0.819421in}}%
\pgfpathlineto{\pgfqpoint{1.188549in}{0.806495in}}%
\pgfpathlineto{\pgfqpoint{1.188599in}{0.793568in}}%
\pgfpathlineto{\pgfqpoint{1.191666in}{0.780642in}}%
\pgfpathclose%
\pgfpathmoveto{\pgfqpoint{1.196496in}{0.793568in}}%
\pgfpathlineto{\pgfqpoint{1.196124in}{0.806495in}}%
\pgfpathlineto{\pgfqpoint{1.205566in}{0.819421in}}%
\pgfpathlineto{\pgfqpoint{1.207045in}{0.820255in}}%
\pgfpathlineto{\pgfqpoint{1.209849in}{0.819421in}}%
\pgfpathlineto{\pgfqpoint{1.220314in}{0.812647in}}%
\pgfpathlineto{\pgfqpoint{1.222474in}{0.806495in}}%
\pgfpathlineto{\pgfqpoint{1.222269in}{0.793568in}}%
\pgfpathlineto{\pgfqpoint{1.220314in}{0.788547in}}%
\pgfpathlineto{\pgfqpoint{1.207045in}{0.781118in}}%
\pgfpathclose%
\pgfusepath{fill}%
\end{pgfscope}%
\begin{pgfscope}%
\pgfpathrectangle{\pgfqpoint{0.211875in}{0.211875in}}{\pgfqpoint{1.313625in}{1.279725in}}%
\pgfusepath{clip}%
\pgfsetbuttcap%
\pgfsetroundjoin%
\definecolor{currentfill}{rgb}{0.796501,0.105066,0.310630}%
\pgfsetfillcolor{currentfill}%
\pgfsetlinewidth{0.000000pt}%
\definecolor{currentstroke}{rgb}{0.000000,0.000000,0.000000}%
\pgfsetstrokecolor{currentstroke}%
\pgfsetdash{}{0pt}%
\pgfpathmoveto{\pgfqpoint{1.313197in}{0.775253in}}%
\pgfpathlineto{\pgfqpoint{1.326466in}{0.770949in}}%
\pgfpathlineto{\pgfqpoint{1.339735in}{0.774155in}}%
\pgfpathlineto{\pgfqpoint{1.344978in}{0.780642in}}%
\pgfpathlineto{\pgfqpoint{1.348083in}{0.793568in}}%
\pgfpathlineto{\pgfqpoint{1.348124in}{0.806495in}}%
\pgfpathlineto{\pgfqpoint{1.345303in}{0.819421in}}%
\pgfpathlineto{\pgfqpoint{1.339735in}{0.826945in}}%
\pgfpathlineto{\pgfqpoint{1.326466in}{0.830447in}}%
\pgfpathlineto{\pgfqpoint{1.313197in}{0.825779in}}%
\pgfpathlineto{\pgfqpoint{1.308943in}{0.819421in}}%
\pgfpathlineto{\pgfqpoint{1.306191in}{0.806495in}}%
\pgfpathlineto{\pgfqpoint{1.306234in}{0.793568in}}%
\pgfpathlineto{\pgfqpoint{1.309272in}{0.780642in}}%
\pgfpathclose%
\pgfpathmoveto{\pgfqpoint{1.325416in}{0.780642in}}%
\pgfpathlineto{\pgfqpoint{1.313197in}{0.792823in}}%
\pgfpathlineto{\pgfqpoint{1.312962in}{0.793568in}}%
\pgfpathlineto{\pgfqpoint{1.312783in}{0.806495in}}%
\pgfpathlineto{\pgfqpoint{1.313197in}{0.807941in}}%
\pgfpathlineto{\pgfqpoint{1.323120in}{0.819421in}}%
\pgfpathlineto{\pgfqpoint{1.326466in}{0.821009in}}%
\pgfpathlineto{\pgfqpoint{1.330571in}{0.819421in}}%
\pgfpathlineto{\pgfqpoint{1.339735in}{0.811209in}}%
\pgfpathlineto{\pgfqpoint{1.341223in}{0.806495in}}%
\pgfpathlineto{\pgfqpoint{1.341038in}{0.793568in}}%
\pgfpathlineto{\pgfqpoint{1.339735in}{0.789830in}}%
\pgfpathlineto{\pgfqpoint{1.327757in}{0.780642in}}%
\pgfpathlineto{\pgfqpoint{1.326466in}{0.780189in}}%
\pgfpathclose%
\pgfusepath{fill}%
\end{pgfscope}%
\begin{pgfscope}%
\pgfpathrectangle{\pgfqpoint{0.211875in}{0.211875in}}{\pgfqpoint{1.313625in}{1.279725in}}%
\pgfusepath{clip}%
\pgfsetbuttcap%
\pgfsetroundjoin%
\definecolor{currentfill}{rgb}{0.796501,0.105066,0.310630}%
\pgfsetfillcolor{currentfill}%
\pgfsetlinewidth{0.000000pt}%
\definecolor{currentstroke}{rgb}{0.000000,0.000000,0.000000}%
\pgfsetstrokecolor{currentstroke}%
\pgfsetdash{}{0pt}%
\pgfpathmoveto{\pgfqpoint{1.432617in}{0.772717in}}%
\pgfpathlineto{\pgfqpoint{1.445886in}{0.769725in}}%
\pgfpathlineto{\pgfqpoint{1.459155in}{0.773456in}}%
\pgfpathlineto{\pgfqpoint{1.464354in}{0.780642in}}%
\pgfpathlineto{\pgfqpoint{1.466984in}{0.793568in}}%
\pgfpathlineto{\pgfqpoint{1.467003in}{0.806495in}}%
\pgfpathlineto{\pgfqpoint{1.464570in}{0.819421in}}%
\pgfpathlineto{\pgfqpoint{1.459155in}{0.827576in}}%
\pgfpathlineto{\pgfqpoint{1.445886in}{0.831697in}}%
\pgfpathlineto{\pgfqpoint{1.432617in}{0.828417in}}%
\pgfpathlineto{\pgfqpoint{1.425919in}{0.819421in}}%
\pgfpathlineto{\pgfqpoint{1.423288in}{0.806495in}}%
\pgfpathlineto{\pgfqpoint{1.423316in}{0.793568in}}%
\pgfpathlineto{\pgfqpoint{1.426179in}{0.780642in}}%
\pgfpathclose%
\pgfpathmoveto{\pgfqpoint{1.441134in}{0.780642in}}%
\pgfpathlineto{\pgfqpoint{1.432617in}{0.787083in}}%
\pgfpathlineto{\pgfqpoint{1.430332in}{0.793568in}}%
\pgfpathlineto{\pgfqpoint{1.430162in}{0.806495in}}%
\pgfpathlineto{\pgfqpoint{1.432617in}{0.814195in}}%
\pgfpathlineto{\pgfqpoint{1.438551in}{0.819421in}}%
\pgfpathlineto{\pgfqpoint{1.445886in}{0.822252in}}%
\pgfpathlineto{\pgfqpoint{1.451841in}{0.819421in}}%
\pgfpathlineto{\pgfqpoint{1.459155in}{0.810806in}}%
\pgfpathlineto{\pgfqpoint{1.460376in}{0.806495in}}%
\pgfpathlineto{\pgfqpoint{1.460218in}{0.793568in}}%
\pgfpathlineto{\pgfqpoint{1.459155in}{0.790153in}}%
\pgfpathlineto{\pgfqpoint{1.449763in}{0.780642in}}%
\pgfpathlineto{\pgfqpoint{1.445886in}{0.778972in}}%
\pgfpathclose%
\pgfusepath{fill}%
\end{pgfscope}%
\begin{pgfscope}%
\pgfpathrectangle{\pgfqpoint{0.211875in}{0.211875in}}{\pgfqpoint{1.313625in}{1.279725in}}%
\pgfusepath{clip}%
\pgfsetbuttcap%
\pgfsetroundjoin%
\definecolor{currentfill}{rgb}{0.796501,0.105066,0.310630}%
\pgfsetfillcolor{currentfill}%
\pgfsetlinewidth{0.000000pt}%
\definecolor{currentstroke}{rgb}{0.000000,0.000000,0.000000}%
\pgfsetstrokecolor{currentstroke}%
\pgfsetdash{}{0pt}%
\pgfpathmoveto{\pgfqpoint{0.424178in}{0.844817in}}%
\pgfpathlineto{\pgfqpoint{0.437447in}{0.843816in}}%
\pgfpathlineto{\pgfqpoint{0.450716in}{0.844328in}}%
\pgfpathlineto{\pgfqpoint{0.456053in}{0.845274in}}%
\pgfpathlineto{\pgfqpoint{0.463985in}{0.848573in}}%
\pgfpathlineto{\pgfqpoint{0.468378in}{0.858201in}}%
\pgfpathlineto{\pgfqpoint{0.469535in}{0.871127in}}%
\pgfpathlineto{\pgfqpoint{0.469673in}{0.884054in}}%
\pgfpathlineto{\pgfqpoint{0.469130in}{0.896980in}}%
\pgfpathlineto{\pgfqpoint{0.466800in}{0.909907in}}%
\pgfpathlineto{\pgfqpoint{0.463985in}{0.914482in}}%
\pgfpathlineto{\pgfqpoint{0.450716in}{0.919330in}}%
\pgfpathlineto{\pgfqpoint{0.437447in}{0.919886in}}%
\pgfpathlineto{\pgfqpoint{0.424178in}{0.918323in}}%
\pgfpathlineto{\pgfqpoint{0.416741in}{0.909907in}}%
\pgfpathlineto{\pgfqpoint{0.415146in}{0.896980in}}%
\pgfpathlineto{\pgfqpoint{0.414769in}{0.884054in}}%
\pgfpathlineto{\pgfqpoint{0.414806in}{0.871127in}}%
\pgfpathlineto{\pgfqpoint{0.415426in}{0.858201in}}%
\pgfpathlineto{\pgfqpoint{0.422615in}{0.845274in}}%
\pgfpathclose%
\pgfpathmoveto{\pgfqpoint{0.426696in}{0.858201in}}%
\pgfpathlineto{\pgfqpoint{0.424178in}{0.861604in}}%
\pgfpathlineto{\pgfqpoint{0.421457in}{0.871127in}}%
\pgfpathlineto{\pgfqpoint{0.420810in}{0.884054in}}%
\pgfpathlineto{\pgfqpoint{0.422324in}{0.896980in}}%
\pgfpathlineto{\pgfqpoint{0.424178in}{0.901911in}}%
\pgfpathlineto{\pgfqpoint{0.433231in}{0.909907in}}%
\pgfpathlineto{\pgfqpoint{0.437447in}{0.911535in}}%
\pgfpathlineto{\pgfqpoint{0.450716in}{0.910521in}}%
\pgfpathlineto{\pgfqpoint{0.451836in}{0.909907in}}%
\pgfpathlineto{\pgfqpoint{0.461807in}{0.896980in}}%
\pgfpathlineto{\pgfqpoint{0.463985in}{0.884162in}}%
\pgfpathlineto{\pgfqpoint{0.463996in}{0.884054in}}%
\pgfpathlineto{\pgfqpoint{0.463985in}{0.883825in}}%
\pgfpathlineto{\pgfqpoint{0.463037in}{0.871127in}}%
\pgfpathlineto{\pgfqpoint{0.457009in}{0.858201in}}%
\pgfpathlineto{\pgfqpoint{0.450716in}{0.853680in}}%
\pgfpathlineto{\pgfqpoint{0.437447in}{0.852631in}}%
\pgfpathclose%
\pgfusepath{fill}%
\end{pgfscope}%
\begin{pgfscope}%
\pgfpathrectangle{\pgfqpoint{0.211875in}{0.211875in}}{\pgfqpoint{1.313625in}{1.279725in}}%
\pgfusepath{clip}%
\pgfsetbuttcap%
\pgfsetroundjoin%
\definecolor{currentfill}{rgb}{0.796501,0.105066,0.310630}%
\pgfsetfillcolor{currentfill}%
\pgfsetlinewidth{0.000000pt}%
\definecolor{currentstroke}{rgb}{0.000000,0.000000,0.000000}%
\pgfsetstrokecolor{currentstroke}%
\pgfsetdash{}{0pt}%
\pgfpathmoveto{\pgfqpoint{0.543598in}{0.849692in}}%
\pgfpathlineto{\pgfqpoint{0.556867in}{0.846889in}}%
\pgfpathlineto{\pgfqpoint{0.570136in}{0.847929in}}%
\pgfpathlineto{\pgfqpoint{0.582996in}{0.858201in}}%
\pgfpathlineto{\pgfqpoint{0.583405in}{0.859443in}}%
\pgfpathlineto{\pgfqpoint{0.585354in}{0.871127in}}%
\pgfpathlineto{\pgfqpoint{0.585682in}{0.884054in}}%
\pgfpathlineto{\pgfqpoint{0.584784in}{0.896980in}}%
\pgfpathlineto{\pgfqpoint{0.583405in}{0.903099in}}%
\pgfpathlineto{\pgfqpoint{0.579948in}{0.909907in}}%
\pgfpathlineto{\pgfqpoint{0.570136in}{0.915868in}}%
\pgfpathlineto{\pgfqpoint{0.556867in}{0.916922in}}%
\pgfpathlineto{\pgfqpoint{0.543598in}{0.913920in}}%
\pgfpathlineto{\pgfqpoint{0.539574in}{0.909907in}}%
\pgfpathlineto{\pgfqpoint{0.535808in}{0.896980in}}%
\pgfpathlineto{\pgfqpoint{0.534988in}{0.884054in}}%
\pgfpathlineto{\pgfqpoint{0.535260in}{0.871127in}}%
\pgfpathlineto{\pgfqpoint{0.537268in}{0.858201in}}%
\pgfpathclose%
\pgfpathmoveto{\pgfqpoint{0.551197in}{0.858201in}}%
\pgfpathlineto{\pgfqpoint{0.543598in}{0.866350in}}%
\pgfpathlineto{\pgfqpoint{0.542037in}{0.871127in}}%
\pgfpathlineto{\pgfqpoint{0.541147in}{0.884054in}}%
\pgfpathlineto{\pgfqpoint{0.543122in}{0.896980in}}%
\pgfpathlineto{\pgfqpoint{0.543598in}{0.898095in}}%
\pgfpathlineto{\pgfqpoint{0.556867in}{0.908355in}}%
\pgfpathlineto{\pgfqpoint{0.570136in}{0.905666in}}%
\pgfpathlineto{\pgfqpoint{0.576431in}{0.896980in}}%
\pgfpathlineto{\pgfqpoint{0.578946in}{0.884054in}}%
\pgfpathlineto{\pgfqpoint{0.577796in}{0.871127in}}%
\pgfpathlineto{\pgfqpoint{0.570704in}{0.858201in}}%
\pgfpathlineto{\pgfqpoint{0.570136in}{0.857747in}}%
\pgfpathlineto{\pgfqpoint{0.556867in}{0.855636in}}%
\pgfpathclose%
\pgfusepath{fill}%
\end{pgfscope}%
\begin{pgfscope}%
\pgfpathrectangle{\pgfqpoint{0.211875in}{0.211875in}}{\pgfqpoint{1.313625in}{1.279725in}}%
\pgfusepath{clip}%
\pgfsetbuttcap%
\pgfsetroundjoin%
\definecolor{currentfill}{rgb}{0.796501,0.105066,0.310630}%
\pgfsetfillcolor{currentfill}%
\pgfsetlinewidth{0.000000pt}%
\definecolor{currentstroke}{rgb}{0.000000,0.000000,0.000000}%
\pgfsetstrokecolor{currentstroke}%
\pgfsetdash{}{0pt}%
\pgfpathmoveto{\pgfqpoint{0.663019in}{0.852858in}}%
\pgfpathlineto{\pgfqpoint{0.676288in}{0.849421in}}%
\pgfpathlineto{\pgfqpoint{0.689557in}{0.851298in}}%
\pgfpathlineto{\pgfqpoint{0.697338in}{0.858201in}}%
\pgfpathlineto{\pgfqpoint{0.701247in}{0.871127in}}%
\pgfpathlineto{\pgfqpoint{0.701846in}{0.884054in}}%
\pgfpathlineto{\pgfqpoint{0.700376in}{0.896980in}}%
\pgfpathlineto{\pgfqpoint{0.693690in}{0.909907in}}%
\pgfpathlineto{\pgfqpoint{0.689557in}{0.912687in}}%
\pgfpathlineto{\pgfqpoint{0.676288in}{0.914561in}}%
\pgfpathlineto{\pgfqpoint{0.663019in}{0.911075in}}%
\pgfpathlineto{\pgfqpoint{0.661703in}{0.909907in}}%
\pgfpathlineto{\pgfqpoint{0.656052in}{0.896980in}}%
\pgfpathlineto{\pgfqpoint{0.654844in}{0.884054in}}%
\pgfpathlineto{\pgfqpoint{0.655319in}{0.871127in}}%
\pgfpathlineto{\pgfqpoint{0.658533in}{0.858201in}}%
\pgfpathclose%
\pgfpathmoveto{\pgfqpoint{0.675907in}{0.858201in}}%
\pgfpathlineto{\pgfqpoint{0.663019in}{0.869116in}}%
\pgfpathlineto{\pgfqpoint{0.662275in}{0.871127in}}%
\pgfpathlineto{\pgfqpoint{0.661167in}{0.884054in}}%
\pgfpathlineto{\pgfqpoint{0.663019in}{0.894441in}}%
\pgfpathlineto{\pgfqpoint{0.664128in}{0.896980in}}%
\pgfpathlineto{\pgfqpoint{0.676288in}{0.905372in}}%
\pgfpathlineto{\pgfqpoint{0.689557in}{0.900861in}}%
\pgfpathlineto{\pgfqpoint{0.692086in}{0.896980in}}%
\pgfpathlineto{\pgfqpoint{0.694822in}{0.884054in}}%
\pgfpathlineto{\pgfqpoint{0.693543in}{0.871127in}}%
\pgfpathlineto{\pgfqpoint{0.689557in}{0.863244in}}%
\pgfpathlineto{\pgfqpoint{0.676941in}{0.858201in}}%
\pgfpathlineto{\pgfqpoint{0.676288in}{0.858053in}}%
\pgfpathclose%
\pgfusepath{fill}%
\end{pgfscope}%
\begin{pgfscope}%
\pgfpathrectangle{\pgfqpoint{0.211875in}{0.211875in}}{\pgfqpoint{1.313625in}{1.279725in}}%
\pgfusepath{clip}%
\pgfsetbuttcap%
\pgfsetroundjoin%
\definecolor{currentfill}{rgb}{0.796501,0.105066,0.310630}%
\pgfsetfillcolor{currentfill}%
\pgfsetlinewidth{0.000000pt}%
\definecolor{currentstroke}{rgb}{0.000000,0.000000,0.000000}%
\pgfsetstrokecolor{currentstroke}%
\pgfsetdash{}{0pt}%
\pgfpathmoveto{\pgfqpoint{0.782439in}{0.854784in}}%
\pgfpathlineto{\pgfqpoint{0.795708in}{0.851376in}}%
\pgfpathlineto{\pgfqpoint{0.808977in}{0.854385in}}%
\pgfpathlineto{\pgfqpoint{0.812848in}{0.858201in}}%
\pgfpathlineto{\pgfqpoint{0.817399in}{0.871127in}}%
\pgfpathlineto{\pgfqpoint{0.818110in}{0.884054in}}%
\pgfpathlineto{\pgfqpoint{0.816447in}{0.896980in}}%
\pgfpathlineto{\pgfqpoint{0.808977in}{0.909723in}}%
\pgfpathlineto{\pgfqpoint{0.808481in}{0.909907in}}%
\pgfpathlineto{\pgfqpoint{0.795708in}{0.912734in}}%
\pgfpathlineto{\pgfqpoint{0.784125in}{0.909907in}}%
\pgfpathlineto{\pgfqpoint{0.782439in}{0.909170in}}%
\pgfpathlineto{\pgfqpoint{0.775870in}{0.896980in}}%
\pgfpathlineto{\pgfqpoint{0.774324in}{0.884054in}}%
\pgfpathlineto{\pgfqpoint{0.774975in}{0.871127in}}%
\pgfpathlineto{\pgfqpoint{0.779225in}{0.858201in}}%
\pgfpathclose%
\pgfpathmoveto{\pgfqpoint{0.782168in}{0.871127in}}%
\pgfpathlineto{\pgfqpoint{0.780865in}{0.884054in}}%
\pgfpathlineto{\pgfqpoint{0.782439in}{0.891953in}}%
\pgfpathlineto{\pgfqpoint{0.785254in}{0.896980in}}%
\pgfpathlineto{\pgfqpoint{0.795708in}{0.903023in}}%
\pgfpathlineto{\pgfqpoint{0.807824in}{0.896980in}}%
\pgfpathlineto{\pgfqpoint{0.808977in}{0.895315in}}%
\pgfpathlineto{\pgfqpoint{0.811425in}{0.884054in}}%
\pgfpathlineto{\pgfqpoint{0.810062in}{0.871127in}}%
\pgfpathlineto{\pgfqpoint{0.808977in}{0.868736in}}%
\pgfpathlineto{\pgfqpoint{0.795708in}{0.860785in}}%
\pgfpathlineto{\pgfqpoint{0.782439in}{0.870474in}}%
\pgfpathclose%
\pgfusepath{fill}%
\end{pgfscope}%
\begin{pgfscope}%
\pgfpathrectangle{\pgfqpoint{0.211875in}{0.211875in}}{\pgfqpoint{1.313625in}{1.279725in}}%
\pgfusepath{clip}%
\pgfsetbuttcap%
\pgfsetroundjoin%
\definecolor{currentfill}{rgb}{0.796501,0.105066,0.310630}%
\pgfsetfillcolor{currentfill}%
\pgfsetlinewidth{0.000000pt}%
\definecolor{currentstroke}{rgb}{0.000000,0.000000,0.000000}%
\pgfsetstrokecolor{currentstroke}%
\pgfsetdash{}{0pt}%
\pgfpathmoveto{\pgfqpoint{0.901860in}{0.855783in}}%
\pgfpathlineto{\pgfqpoint{0.915129in}{0.852806in}}%
\pgfpathlineto{\pgfqpoint{0.928398in}{0.857197in}}%
\pgfpathlineto{\pgfqpoint{0.929311in}{0.858201in}}%
\pgfpathlineto{\pgfqpoint{0.934249in}{0.871127in}}%
\pgfpathlineto{\pgfqpoint{0.935028in}{0.884054in}}%
\pgfpathlineto{\pgfqpoint{0.933252in}{0.896980in}}%
\pgfpathlineto{\pgfqpoint{0.928398in}{0.906205in}}%
\pgfpathlineto{\pgfqpoint{0.920689in}{0.909907in}}%
\pgfpathlineto{\pgfqpoint{0.915129in}{0.911395in}}%
\pgfpathlineto{\pgfqpoint{0.907606in}{0.909907in}}%
\pgfpathlineto{\pgfqpoint{0.901860in}{0.908004in}}%
\pgfpathlineto{\pgfqpoint{0.895236in}{0.896980in}}%
\pgfpathlineto{\pgfqpoint{0.893402in}{0.884054in}}%
\pgfpathlineto{\pgfqpoint{0.894200in}{0.871127in}}%
\pgfpathlineto{\pgfqpoint{0.899323in}{0.858201in}}%
\pgfpathclose%
\pgfpathmoveto{\pgfqpoint{0.901698in}{0.871127in}}%
\pgfpathlineto{\pgfqpoint{0.900222in}{0.884054in}}%
\pgfpathlineto{\pgfqpoint{0.901860in}{0.891432in}}%
\pgfpathlineto{\pgfqpoint{0.905972in}{0.896980in}}%
\pgfpathlineto{\pgfqpoint{0.915129in}{0.901252in}}%
\pgfpathlineto{\pgfqpoint{0.922129in}{0.896980in}}%
\pgfpathlineto{\pgfqpoint{0.928398in}{0.885140in}}%
\pgfpathlineto{\pgfqpoint{0.928610in}{0.884054in}}%
\pgfpathlineto{\pgfqpoint{0.928398in}{0.881946in}}%
\pgfpathlineto{\pgfqpoint{0.925655in}{0.871127in}}%
\pgfpathlineto{\pgfqpoint{0.915129in}{0.862894in}}%
\pgfpathlineto{\pgfqpoint{0.901860in}{0.870778in}}%
\pgfpathclose%
\pgfusepath{fill}%
\end{pgfscope}%
\begin{pgfscope}%
\pgfpathrectangle{\pgfqpoint{0.211875in}{0.211875in}}{\pgfqpoint{1.313625in}{1.279725in}}%
\pgfusepath{clip}%
\pgfsetbuttcap%
\pgfsetroundjoin%
\definecolor{currentfill}{rgb}{0.796501,0.105066,0.310630}%
\pgfsetfillcolor{currentfill}%
\pgfsetlinewidth{0.000000pt}%
\definecolor{currentstroke}{rgb}{0.000000,0.000000,0.000000}%
\pgfsetstrokecolor{currentstroke}%
\pgfsetdash{}{0pt}%
\pgfpathmoveto{\pgfqpoint{1.021280in}{0.856055in}}%
\pgfpathlineto{\pgfqpoint{1.034549in}{0.853737in}}%
\pgfpathlineto{\pgfqpoint{1.045326in}{0.858201in}}%
\pgfpathlineto{\pgfqpoint{1.047818in}{0.860454in}}%
\pgfpathlineto{\pgfqpoint{1.051679in}{0.871127in}}%
\pgfpathlineto{\pgfqpoint{1.052486in}{0.884054in}}%
\pgfpathlineto{\pgfqpoint{1.050668in}{0.896980in}}%
\pgfpathlineto{\pgfqpoint{1.047818in}{0.903037in}}%
\pgfpathlineto{\pgfqpoint{1.036495in}{0.909907in}}%
\pgfpathlineto{\pgfqpoint{1.034549in}{0.910518in}}%
\pgfpathlineto{\pgfqpoint{1.030505in}{0.909907in}}%
\pgfpathlineto{\pgfqpoint{1.021280in}{0.907717in}}%
\pgfpathlineto{\pgfqpoint{1.014099in}{0.896980in}}%
\pgfpathlineto{\pgfqpoint{1.012025in}{0.884054in}}%
\pgfpathlineto{\pgfqpoint{1.012942in}{0.871127in}}%
\pgfpathlineto{\pgfqpoint{1.018774in}{0.858201in}}%
\pgfpathclose%
\pgfpathmoveto{\pgfqpoint{1.020826in}{0.871127in}}%
\pgfpathlineto{\pgfqpoint{1.019198in}{0.884054in}}%
\pgfpathlineto{\pgfqpoint{1.021280in}{0.892483in}}%
\pgfpathlineto{\pgfqpoint{1.025954in}{0.896980in}}%
\pgfpathlineto{\pgfqpoint{1.034549in}{0.900024in}}%
\pgfpathlineto{\pgfqpoint{1.038762in}{0.896980in}}%
\pgfpathlineto{\pgfqpoint{1.044711in}{0.884054in}}%
\pgfpathlineto{\pgfqpoint{1.041857in}{0.871127in}}%
\pgfpathlineto{\pgfqpoint{1.034549in}{0.864353in}}%
\pgfpathlineto{\pgfqpoint{1.021280in}{0.870250in}}%
\pgfpathclose%
\pgfusepath{fill}%
\end{pgfscope}%
\begin{pgfscope}%
\pgfpathrectangle{\pgfqpoint{0.211875in}{0.211875in}}{\pgfqpoint{1.313625in}{1.279725in}}%
\pgfusepath{clip}%
\pgfsetbuttcap%
\pgfsetroundjoin%
\definecolor{currentfill}{rgb}{0.796501,0.105066,0.310630}%
\pgfsetfillcolor{currentfill}%
\pgfsetlinewidth{0.000000pt}%
\definecolor{currentstroke}{rgb}{0.000000,0.000000,0.000000}%
\pgfsetstrokecolor{currentstroke}%
\pgfsetdash{}{0pt}%
\pgfpathmoveto{\pgfqpoint{1.140701in}{0.855729in}}%
\pgfpathlineto{\pgfqpoint{1.153970in}{0.854180in}}%
\pgfpathlineto{\pgfqpoint{1.162377in}{0.858201in}}%
\pgfpathlineto{\pgfqpoint{1.167239in}{0.863725in}}%
\pgfpathlineto{\pgfqpoint{1.169606in}{0.871127in}}%
\pgfpathlineto{\pgfqpoint{1.170406in}{0.884054in}}%
\pgfpathlineto{\pgfqpoint{1.168609in}{0.896980in}}%
\pgfpathlineto{\pgfqpoint{1.167239in}{0.900259in}}%
\pgfpathlineto{\pgfqpoint{1.154495in}{0.909907in}}%
\pgfpathlineto{\pgfqpoint{1.153970in}{0.910097in}}%
\pgfpathlineto{\pgfqpoint{1.152136in}{0.909907in}}%
\pgfpathlineto{\pgfqpoint{1.140701in}{0.908157in}}%
\pgfpathlineto{\pgfqpoint{1.132376in}{0.896980in}}%
\pgfpathlineto{\pgfqpoint{1.130114in}{0.884054in}}%
\pgfpathlineto{\pgfqpoint{1.131121in}{0.871127in}}%
\pgfpathlineto{\pgfqpoint{1.137488in}{0.858201in}}%
\pgfpathclose%
\pgfpathmoveto{\pgfqpoint{1.139489in}{0.871127in}}%
\pgfpathlineto{\pgfqpoint{1.137731in}{0.884054in}}%
\pgfpathlineto{\pgfqpoint{1.140701in}{0.894854in}}%
\pgfpathlineto{\pgfqpoint{1.144171in}{0.896980in}}%
\pgfpathlineto{\pgfqpoint{1.153970in}{0.899328in}}%
\pgfpathlineto{\pgfqpoint{1.156774in}{0.896980in}}%
\pgfpathlineto{\pgfqpoint{1.162029in}{0.884054in}}%
\pgfpathlineto{\pgfqpoint{1.159506in}{0.871127in}}%
\pgfpathlineto{\pgfqpoint{1.153970in}{0.865177in}}%
\pgfpathlineto{\pgfqpoint{1.140701in}{0.869031in}}%
\pgfpathclose%
\pgfusepath{fill}%
\end{pgfscope}%
\begin{pgfscope}%
\pgfpathrectangle{\pgfqpoint{0.211875in}{0.211875in}}{\pgfqpoint{1.313625in}{1.279725in}}%
\pgfusepath{clip}%
\pgfsetbuttcap%
\pgfsetroundjoin%
\definecolor{currentfill}{rgb}{0.796501,0.105066,0.310630}%
\pgfsetfillcolor{currentfill}%
\pgfsetlinewidth{0.000000pt}%
\definecolor{currentstroke}{rgb}{0.000000,0.000000,0.000000}%
\pgfsetstrokecolor{currentstroke}%
\pgfsetdash{}{0pt}%
\pgfpathmoveto{\pgfqpoint{1.260121in}{0.854886in}}%
\pgfpathlineto{\pgfqpoint{1.273390in}{0.854120in}}%
\pgfpathlineto{\pgfqpoint{1.280883in}{0.858201in}}%
\pgfpathlineto{\pgfqpoint{1.286659in}{0.866424in}}%
\pgfpathlineto{\pgfqpoint{1.287974in}{0.871127in}}%
\pgfpathlineto{\pgfqpoint{1.288734in}{0.884054in}}%
\pgfpathlineto{\pgfqpoint{1.287017in}{0.896980in}}%
\pgfpathlineto{\pgfqpoint{1.286659in}{0.897955in}}%
\pgfpathlineto{\pgfqpoint{1.273961in}{0.909907in}}%
\pgfpathlineto{\pgfqpoint{1.273390in}{0.910141in}}%
\pgfpathlineto{\pgfqpoint{1.269072in}{0.909907in}}%
\pgfpathlineto{\pgfqpoint{1.260121in}{0.909230in}}%
\pgfpathlineto{\pgfqpoint{1.249939in}{0.896980in}}%
\pgfpathlineto{\pgfqpoint{1.247545in}{0.884054in}}%
\pgfpathlineto{\pgfqpoint{1.248608in}{0.871127in}}%
\pgfpathlineto{\pgfqpoint{1.255319in}{0.858201in}}%
\pgfpathclose%
\pgfpathmoveto{\pgfqpoint{1.257586in}{0.871127in}}%
\pgfpathlineto{\pgfqpoint{1.255721in}{0.884054in}}%
\pgfpathlineto{\pgfqpoint{1.259609in}{0.896980in}}%
\pgfpathlineto{\pgfqpoint{1.260121in}{0.897596in}}%
\pgfpathlineto{\pgfqpoint{1.273390in}{0.899172in}}%
\pgfpathlineto{\pgfqpoint{1.275684in}{0.896980in}}%
\pgfpathlineto{\pgfqpoint{1.280309in}{0.884054in}}%
\pgfpathlineto{\pgfqpoint{1.278094in}{0.871127in}}%
\pgfpathlineto{\pgfqpoint{1.273390in}{0.865352in}}%
\pgfpathlineto{\pgfqpoint{1.260121in}{0.867204in}}%
\pgfpathclose%
\pgfusepath{fill}%
\end{pgfscope}%
\begin{pgfscope}%
\pgfpathrectangle{\pgfqpoint{0.211875in}{0.211875in}}{\pgfqpoint{1.313625in}{1.279725in}}%
\pgfusepath{clip}%
\pgfsetbuttcap%
\pgfsetroundjoin%
\definecolor{currentfill}{rgb}{0.796501,0.105066,0.310630}%
\pgfsetfillcolor{currentfill}%
\pgfsetlinewidth{0.000000pt}%
\definecolor{currentstroke}{rgb}{0.000000,0.000000,0.000000}%
\pgfsetstrokecolor{currentstroke}%
\pgfsetdash{}{0pt}%
\pgfpathmoveto{\pgfqpoint{1.379542in}{0.853574in}}%
\pgfpathlineto{\pgfqpoint{1.392811in}{0.853524in}}%
\pgfpathlineto{\pgfqpoint{1.400426in}{0.858201in}}%
\pgfpathlineto{\pgfqpoint{1.406080in}{0.868347in}}%
\pgfpathlineto{\pgfqpoint{1.406749in}{0.871127in}}%
\pgfpathlineto{\pgfqpoint{1.407435in}{0.884054in}}%
\pgfpathlineto{\pgfqpoint{1.406080in}{0.895411in}}%
\pgfpathlineto{\pgfqpoint{1.405744in}{0.896980in}}%
\pgfpathlineto{\pgfqpoint{1.394491in}{0.909907in}}%
\pgfpathlineto{\pgfqpoint{1.392811in}{0.910683in}}%
\pgfpathlineto{\pgfqpoint{1.379542in}{0.910641in}}%
\pgfpathlineto{\pgfqpoint{1.377973in}{0.909907in}}%
\pgfpathlineto{\pgfqpoint{1.366591in}{0.896980in}}%
\pgfpathlineto{\pgfqpoint{1.366273in}{0.895535in}}%
\pgfpathlineto{\pgfqpoint{1.364873in}{0.884054in}}%
\pgfpathlineto{\pgfqpoint{1.365579in}{0.871127in}}%
\pgfpathlineto{\pgfqpoint{1.366273in}{0.868309in}}%
\pgfpathlineto{\pgfqpoint{1.372039in}{0.858201in}}%
\pgfpathclose%
\pgfpathmoveto{\pgfqpoint{1.374963in}{0.871127in}}%
\pgfpathlineto{\pgfqpoint{1.373017in}{0.884054in}}%
\pgfpathlineto{\pgfqpoint{1.377090in}{0.896980in}}%
\pgfpathlineto{\pgfqpoint{1.379542in}{0.899612in}}%
\pgfpathlineto{\pgfqpoint{1.392811in}{0.899591in}}%
\pgfpathlineto{\pgfqpoint{1.395230in}{0.896980in}}%
\pgfpathlineto{\pgfqpoint{1.399261in}{0.884054in}}%
\pgfpathlineto{\pgfqpoint{1.397341in}{0.871127in}}%
\pgfpathlineto{\pgfqpoint{1.392811in}{0.864837in}}%
\pgfpathlineto{\pgfqpoint{1.379542in}{0.864816in}}%
\pgfpathclose%
\pgfusepath{fill}%
\end{pgfscope}%
\begin{pgfscope}%
\pgfpathrectangle{\pgfqpoint{0.211875in}{0.211875in}}{\pgfqpoint{1.313625in}{1.279725in}}%
\pgfusepath{clip}%
\pgfsetbuttcap%
\pgfsetroundjoin%
\definecolor{currentfill}{rgb}{0.796501,0.105066,0.310630}%
\pgfsetfillcolor{currentfill}%
\pgfsetlinewidth{0.000000pt}%
\definecolor{currentstroke}{rgb}{0.000000,0.000000,0.000000}%
\pgfsetstrokecolor{currentstroke}%
\pgfsetdash{}{0pt}%
\pgfpathmoveto{\pgfqpoint{1.498962in}{0.851815in}}%
\pgfpathlineto{\pgfqpoint{1.512231in}{0.852331in}}%
\pgfpathlineto{\pgfqpoint{1.520761in}{0.858201in}}%
\pgfpathlineto{\pgfqpoint{1.525500in}{0.869097in}}%
\pgfpathlineto{\pgfqpoint{1.525500in}{0.871127in}}%
\pgfpathlineto{\pgfqpoint{1.525500in}{0.884054in}}%
\pgfpathlineto{\pgfqpoint{1.525500in}{0.893774in}}%
\pgfpathlineto{\pgfqpoint{1.524957in}{0.896980in}}%
\pgfpathlineto{\pgfqpoint{1.515851in}{0.909907in}}%
\pgfpathlineto{\pgfqpoint{1.512231in}{0.911777in}}%
\pgfpathlineto{\pgfqpoint{1.498962in}{0.912302in}}%
\pgfpathlineto{\pgfqpoint{1.493214in}{0.909907in}}%
\pgfpathlineto{\pgfqpoint{1.485693in}{0.902885in}}%
\pgfpathlineto{\pgfqpoint{1.483488in}{0.896980in}}%
\pgfpathlineto{\pgfqpoint{1.482023in}{0.884054in}}%
\pgfpathlineto{\pgfqpoint{1.482649in}{0.871127in}}%
\pgfpathlineto{\pgfqpoint{1.485693in}{0.860447in}}%
\pgfpathlineto{\pgfqpoint{1.487299in}{0.858201in}}%
\pgfpathclose%
\pgfpathmoveto{\pgfqpoint{1.491378in}{0.871127in}}%
\pgfpathlineto{\pgfqpoint{1.489382in}{0.884054in}}%
\pgfpathlineto{\pgfqpoint{1.493590in}{0.896980in}}%
\pgfpathlineto{\pgfqpoint{1.498962in}{0.902085in}}%
\pgfpathlineto{\pgfqpoint{1.512231in}{0.900645in}}%
\pgfpathlineto{\pgfqpoint{1.515260in}{0.896980in}}%
\pgfpathlineto{\pgfqpoint{1.518715in}{0.884054in}}%
\pgfpathlineto{\pgfqpoint{1.517086in}{0.871127in}}%
\pgfpathlineto{\pgfqpoint{1.512231in}{0.863557in}}%
\pgfpathlineto{\pgfqpoint{1.498962in}{0.861881in}}%
\pgfpathclose%
\pgfusepath{fill}%
\end{pgfscope}%
\begin{pgfscope}%
\pgfpathrectangle{\pgfqpoint{0.211875in}{0.211875in}}{\pgfqpoint{1.313625in}{1.279725in}}%
\pgfusepath{clip}%
\pgfsetbuttcap%
\pgfsetroundjoin%
\definecolor{currentfill}{rgb}{0.796501,0.105066,0.310630}%
\pgfsetfillcolor{currentfill}%
\pgfsetlinewidth{0.000000pt}%
\definecolor{currentstroke}{rgb}{0.000000,0.000000,0.000000}%
\pgfsetstrokecolor{currentstroke}%
\pgfsetdash{}{0pt}%
\pgfpathmoveto{\pgfqpoint{0.357833in}{0.926418in}}%
\pgfpathlineto{\pgfqpoint{0.371102in}{0.924242in}}%
\pgfpathlineto{\pgfqpoint{0.384371in}{0.924343in}}%
\pgfpathlineto{\pgfqpoint{0.397640in}{0.925383in}}%
\pgfpathlineto{\pgfqpoint{0.409788in}{0.935760in}}%
\pgfpathlineto{\pgfqpoint{0.410909in}{0.943875in}}%
\pgfpathlineto{\pgfqpoint{0.411261in}{0.948686in}}%
\pgfpathlineto{\pgfqpoint{0.411519in}{0.961613in}}%
\pgfpathlineto{\pgfqpoint{0.411285in}{0.974539in}}%
\pgfpathlineto{\pgfqpoint{0.410909in}{0.980123in}}%
\pgfpathlineto{\pgfqpoint{0.410101in}{0.987466in}}%
\pgfpathlineto{\pgfqpoint{0.400251in}{1.000392in}}%
\pgfpathlineto{\pgfqpoint{0.397640in}{1.001106in}}%
\pgfpathlineto{\pgfqpoint{0.384371in}{1.002277in}}%
\pgfpathlineto{\pgfqpoint{0.371102in}{1.002236in}}%
\pgfpathlineto{\pgfqpoint{0.360735in}{1.000392in}}%
\pgfpathlineto{\pgfqpoint{0.357833in}{0.998138in}}%
\pgfpathlineto{\pgfqpoint{0.355381in}{0.987466in}}%
\pgfpathlineto{\pgfqpoint{0.354901in}{0.974539in}}%
\pgfpathlineto{\pgfqpoint{0.354776in}{0.961613in}}%
\pgfpathlineto{\pgfqpoint{0.354812in}{0.948686in}}%
\pgfpathlineto{\pgfqpoint{0.355166in}{0.935760in}}%
\pgfpathclose%
\pgfpathmoveto{\pgfqpoint{0.369593in}{0.935760in}}%
\pgfpathlineto{\pgfqpoint{0.362578in}{0.948686in}}%
\pgfpathlineto{\pgfqpoint{0.361205in}{0.961613in}}%
\pgfpathlineto{\pgfqpoint{0.362025in}{0.974539in}}%
\pgfpathlineto{\pgfqpoint{0.366653in}{0.987466in}}%
\pgfpathlineto{\pgfqpoint{0.371102in}{0.991584in}}%
\pgfpathlineto{\pgfqpoint{0.384371in}{0.993974in}}%
\pgfpathlineto{\pgfqpoint{0.397640in}{0.989753in}}%
\pgfpathlineto{\pgfqpoint{0.399721in}{0.987466in}}%
\pgfpathlineto{\pgfqpoint{0.404384in}{0.974539in}}%
\pgfpathlineto{\pgfqpoint{0.405231in}{0.961613in}}%
\pgfpathlineto{\pgfqpoint{0.403792in}{0.948686in}}%
\pgfpathlineto{\pgfqpoint{0.397640in}{0.936668in}}%
\pgfpathlineto{\pgfqpoint{0.395862in}{0.935760in}}%
\pgfpathlineto{\pgfqpoint{0.384371in}{0.932496in}}%
\pgfpathlineto{\pgfqpoint{0.371102in}{0.934697in}}%
\pgfpathclose%
\pgfusepath{fill}%
\end{pgfscope}%
\begin{pgfscope}%
\pgfpathrectangle{\pgfqpoint{0.211875in}{0.211875in}}{\pgfqpoint{1.313625in}{1.279725in}}%
\pgfusepath{clip}%
\pgfsetbuttcap%
\pgfsetroundjoin%
\definecolor{currentfill}{rgb}{0.796501,0.105066,0.310630}%
\pgfsetfillcolor{currentfill}%
\pgfsetlinewidth{0.000000pt}%
\definecolor{currentstroke}{rgb}{0.000000,0.000000,0.000000}%
\pgfsetstrokecolor{currentstroke}%
\pgfsetdash{}{0pt}%
\pgfpathmoveto{\pgfqpoint{0.490523in}{0.928168in}}%
\pgfpathlineto{\pgfqpoint{0.503792in}{0.927625in}}%
\pgfpathlineto{\pgfqpoint{0.517061in}{0.930286in}}%
\pgfpathlineto{\pgfqpoint{0.522805in}{0.935760in}}%
\pgfpathlineto{\pgfqpoint{0.526283in}{0.948686in}}%
\pgfpathlineto{\pgfqpoint{0.526983in}{0.961613in}}%
\pgfpathlineto{\pgfqpoint{0.526517in}{0.974539in}}%
\pgfpathlineto{\pgfqpoint{0.524064in}{0.987466in}}%
\pgfpathlineto{\pgfqpoint{0.517061in}{0.996054in}}%
\pgfpathlineto{\pgfqpoint{0.503792in}{0.999018in}}%
\pgfpathlineto{\pgfqpoint{0.490523in}{0.998359in}}%
\pgfpathlineto{\pgfqpoint{0.477421in}{0.987466in}}%
\pgfpathlineto{\pgfqpoint{0.477254in}{0.986845in}}%
\pgfpathlineto{\pgfqpoint{0.475502in}{0.974539in}}%
\pgfpathlineto{\pgfqpoint{0.475143in}{0.961613in}}%
\pgfpathlineto{\pgfqpoint{0.475643in}{0.948686in}}%
\pgfpathlineto{\pgfqpoint{0.477254in}{0.939142in}}%
\pgfpathlineto{\pgfqpoint{0.478519in}{0.935760in}}%
\pgfpathclose%
\pgfpathmoveto{\pgfqpoint{0.483928in}{0.948686in}}%
\pgfpathlineto{\pgfqpoint{0.481916in}{0.961613in}}%
\pgfpathlineto{\pgfqpoint{0.483067in}{0.974539in}}%
\pgfpathlineto{\pgfqpoint{0.489661in}{0.987466in}}%
\pgfpathlineto{\pgfqpoint{0.490523in}{0.988183in}}%
\pgfpathlineto{\pgfqpoint{0.503792in}{0.990522in}}%
\pgfpathlineto{\pgfqpoint{0.510979in}{0.987466in}}%
\pgfpathlineto{\pgfqpoint{0.517061in}{0.981953in}}%
\pgfpathlineto{\pgfqpoint{0.519807in}{0.974539in}}%
\pgfpathlineto{\pgfqpoint{0.520846in}{0.961613in}}%
\pgfpathlineto{\pgfqpoint{0.519021in}{0.948686in}}%
\pgfpathlineto{\pgfqpoint{0.517061in}{0.944396in}}%
\pgfpathlineto{\pgfqpoint{0.503792in}{0.935847in}}%
\pgfpathlineto{\pgfqpoint{0.490523in}{0.938750in}}%
\pgfpathclose%
\pgfusepath{fill}%
\end{pgfscope}%
\begin{pgfscope}%
\pgfpathrectangle{\pgfqpoint{0.211875in}{0.211875in}}{\pgfqpoint{1.313625in}{1.279725in}}%
\pgfusepath{clip}%
\pgfsetbuttcap%
\pgfsetroundjoin%
\definecolor{currentfill}{rgb}{0.796501,0.105066,0.310630}%
\pgfsetfillcolor{currentfill}%
\pgfsetlinewidth{0.000000pt}%
\definecolor{currentstroke}{rgb}{0.000000,0.000000,0.000000}%
\pgfsetstrokecolor{currentstroke}%
\pgfsetdash{}{0pt}%
\pgfpathmoveto{\pgfqpoint{0.609943in}{0.931086in}}%
\pgfpathlineto{\pgfqpoint{0.623212in}{0.930438in}}%
\pgfpathlineto{\pgfqpoint{0.636481in}{0.935291in}}%
\pgfpathlineto{\pgfqpoint{0.636919in}{0.935760in}}%
\pgfpathlineto{\pgfqpoint{0.641970in}{0.948686in}}%
\pgfpathlineto{\pgfqpoint{0.642998in}{0.961613in}}%
\pgfpathlineto{\pgfqpoint{0.642368in}{0.974539in}}%
\pgfpathlineto{\pgfqpoint{0.638934in}{0.987466in}}%
\pgfpathlineto{\pgfqpoint{0.636481in}{0.990846in}}%
\pgfpathlineto{\pgfqpoint{0.623212in}{0.996098in}}%
\pgfpathlineto{\pgfqpoint{0.609943in}{0.995369in}}%
\pgfpathlineto{\pgfqpoint{0.599357in}{0.987466in}}%
\pgfpathlineto{\pgfqpoint{0.596674in}{0.980110in}}%
\pgfpathlineto{\pgfqpoint{0.595668in}{0.974539in}}%
\pgfpathlineto{\pgfqpoint{0.595116in}{0.961613in}}%
\pgfpathlineto{\pgfqpoint{0.595997in}{0.948686in}}%
\pgfpathlineto{\pgfqpoint{0.596674in}{0.945571in}}%
\pgfpathlineto{\pgfqpoint{0.601719in}{0.935760in}}%
\pgfpathclose%
\pgfpathmoveto{\pgfqpoint{0.604971in}{0.948686in}}%
\pgfpathlineto{\pgfqpoint{0.602334in}{0.961613in}}%
\pgfpathlineto{\pgfqpoint{0.603811in}{0.974539in}}%
\pgfpathlineto{\pgfqpoint{0.609943in}{0.984849in}}%
\pgfpathlineto{\pgfqpoint{0.622965in}{0.987466in}}%
\pgfpathlineto{\pgfqpoint{0.623212in}{0.987496in}}%
\pgfpathlineto{\pgfqpoint{0.623274in}{0.987466in}}%
\pgfpathlineto{\pgfqpoint{0.635303in}{0.974539in}}%
\pgfpathlineto{\pgfqpoint{0.636481in}{0.967738in}}%
\pgfpathlineto{\pgfqpoint{0.637019in}{0.961613in}}%
\pgfpathlineto{\pgfqpoint{0.636481in}{0.957918in}}%
\pgfpathlineto{\pgfqpoint{0.633616in}{0.948686in}}%
\pgfpathlineto{\pgfqpoint{0.623212in}{0.939709in}}%
\pgfpathlineto{\pgfqpoint{0.609943in}{0.941975in}}%
\pgfpathclose%
\pgfusepath{fill}%
\end{pgfscope}%
\begin{pgfscope}%
\pgfpathrectangle{\pgfqpoint{0.211875in}{0.211875in}}{\pgfqpoint{1.313625in}{1.279725in}}%
\pgfusepath{clip}%
\pgfsetbuttcap%
\pgfsetroundjoin%
\definecolor{currentfill}{rgb}{0.796501,0.105066,0.310630}%
\pgfsetfillcolor{currentfill}%
\pgfsetlinewidth{0.000000pt}%
\definecolor{currentstroke}{rgb}{0.000000,0.000000,0.000000}%
\pgfsetstrokecolor{currentstroke}%
\pgfsetdash{}{0pt}%
\pgfpathmoveto{\pgfqpoint{0.729364in}{0.933193in}}%
\pgfpathlineto{\pgfqpoint{0.742633in}{0.932820in}}%
\pgfpathlineto{\pgfqpoint{0.749493in}{0.935760in}}%
\pgfpathlineto{\pgfqpoint{0.755902in}{0.942103in}}%
\pgfpathlineto{\pgfqpoint{0.758241in}{0.948686in}}%
\pgfpathlineto{\pgfqpoint{0.759522in}{0.961613in}}%
\pgfpathlineto{\pgfqpoint{0.758766in}{0.974539in}}%
\pgfpathlineto{\pgfqpoint{0.755902in}{0.984426in}}%
\pgfpathlineto{\pgfqpoint{0.753732in}{0.987466in}}%
\pgfpathlineto{\pgfqpoint{0.742633in}{0.993622in}}%
\pgfpathlineto{\pgfqpoint{0.729364in}{0.993210in}}%
\pgfpathlineto{\pgfqpoint{0.720777in}{0.987466in}}%
\pgfpathlineto{\pgfqpoint{0.716095in}{0.977572in}}%
\pgfpathlineto{\pgfqpoint{0.715432in}{0.974539in}}%
\pgfpathlineto{\pgfqpoint{0.714724in}{0.961613in}}%
\pgfpathlineto{\pgfqpoint{0.715911in}{0.948686in}}%
\pgfpathlineto{\pgfqpoint{0.716095in}{0.947996in}}%
\pgfpathlineto{\pgfqpoint{0.724326in}{0.935760in}}%
\pgfpathclose%
\pgfpathmoveto{\pgfqpoint{0.725695in}{0.948686in}}%
\pgfpathlineto{\pgfqpoint{0.722434in}{0.961613in}}%
\pgfpathlineto{\pgfqpoint{0.724239in}{0.974539in}}%
\pgfpathlineto{\pgfqpoint{0.729364in}{0.982247in}}%
\pgfpathlineto{\pgfqpoint{0.742633in}{0.983627in}}%
\pgfpathlineto{\pgfqpoint{0.750066in}{0.974539in}}%
\pgfpathlineto{\pgfqpoint{0.752213in}{0.961613in}}%
\pgfpathlineto{\pgfqpoint{0.748336in}{0.948686in}}%
\pgfpathlineto{\pgfqpoint{0.742633in}{0.943079in}}%
\pgfpathlineto{\pgfqpoint{0.729364in}{0.944259in}}%
\pgfpathclose%
\pgfusepath{fill}%
\end{pgfscope}%
\begin{pgfscope}%
\pgfpathrectangle{\pgfqpoint{0.211875in}{0.211875in}}{\pgfqpoint{1.313625in}{1.279725in}}%
\pgfusepath{clip}%
\pgfsetbuttcap%
\pgfsetroundjoin%
\definecolor{currentfill}{rgb}{0.796501,0.105066,0.310630}%
\pgfsetfillcolor{currentfill}%
\pgfsetlinewidth{0.000000pt}%
\definecolor{currentstroke}{rgb}{0.000000,0.000000,0.000000}%
\pgfsetstrokecolor{currentstroke}%
\pgfsetdash{}{0pt}%
\pgfpathmoveto{\pgfqpoint{0.848784in}{0.934626in}}%
\pgfpathlineto{\pgfqpoint{0.862053in}{0.934794in}}%
\pgfpathlineto{\pgfqpoint{0.864062in}{0.935760in}}%
\pgfpathlineto{\pgfqpoint{0.874861in}{0.948686in}}%
\pgfpathlineto{\pgfqpoint{0.875322in}{0.951034in}}%
\pgfpathlineto{\pgfqpoint{0.876482in}{0.961613in}}%
\pgfpathlineto{\pgfqpoint{0.875633in}{0.974539in}}%
\pgfpathlineto{\pgfqpoint{0.875322in}{0.975775in}}%
\pgfpathlineto{\pgfqpoint{0.868630in}{0.987466in}}%
\pgfpathlineto{\pgfqpoint{0.862053in}{0.991567in}}%
\pgfpathlineto{\pgfqpoint{0.848784in}{0.991742in}}%
\pgfpathlineto{\pgfqpoint{0.841619in}{0.987466in}}%
\pgfpathlineto{\pgfqpoint{0.835515in}{0.977286in}}%
\pgfpathlineto{\pgfqpoint{0.834810in}{0.974539in}}%
\pgfpathlineto{\pgfqpoint{0.833978in}{0.961613in}}%
\pgfpathlineto{\pgfqpoint{0.835407in}{0.948686in}}%
\pgfpathlineto{\pgfqpoint{0.835515in}{0.948342in}}%
\pgfpathlineto{\pgfqpoint{0.846291in}{0.935760in}}%
\pgfpathclose%
\pgfpathmoveto{\pgfqpoint{0.846061in}{0.948686in}}%
\pgfpathlineto{\pgfqpoint{0.842166in}{0.961613in}}%
\pgfpathlineto{\pgfqpoint{0.844307in}{0.974539in}}%
\pgfpathlineto{\pgfqpoint{0.848784in}{0.980534in}}%
\pgfpathlineto{\pgfqpoint{0.862053in}{0.980269in}}%
\pgfpathlineto{\pgfqpoint{0.866206in}{0.974539in}}%
\pgfpathlineto{\pgfqpoint{0.868315in}{0.961613in}}%
\pgfpathlineto{\pgfqpoint{0.864481in}{0.948686in}}%
\pgfpathlineto{\pgfqpoint{0.862053in}{0.945993in}}%
\pgfpathlineto{\pgfqpoint{0.848784in}{0.945762in}}%
\pgfpathclose%
\pgfusepath{fill}%
\end{pgfscope}%
\begin{pgfscope}%
\pgfpathrectangle{\pgfqpoint{0.211875in}{0.211875in}}{\pgfqpoint{1.313625in}{1.279725in}}%
\pgfusepath{clip}%
\pgfsetbuttcap%
\pgfsetroundjoin%
\definecolor{currentfill}{rgb}{0.796501,0.105066,0.310630}%
\pgfsetfillcolor{currentfill}%
\pgfsetlinewidth{0.000000pt}%
\definecolor{currentstroke}{rgb}{0.000000,0.000000,0.000000}%
\pgfsetstrokecolor{currentstroke}%
\pgfsetdash{}{0pt}%
\pgfpathmoveto{\pgfqpoint{0.968205in}{0.935478in}}%
\pgfpathlineto{\pgfqpoint{0.972601in}{0.935760in}}%
\pgfpathlineto{\pgfqpoint{0.981473in}{0.936560in}}%
\pgfpathlineto{\pgfqpoint{0.991228in}{0.948686in}}%
\pgfpathlineto{\pgfqpoint{0.993451in}{0.961613in}}%
\pgfpathlineto{\pgfqpoint{0.992175in}{0.974539in}}%
\pgfpathlineto{\pgfqpoint{0.985008in}{0.987466in}}%
\pgfpathlineto{\pgfqpoint{0.981473in}{0.989929in}}%
\pgfpathlineto{\pgfqpoint{0.968205in}{0.990870in}}%
\pgfpathlineto{\pgfqpoint{0.961765in}{0.987466in}}%
\pgfpathlineto{\pgfqpoint{0.954936in}{0.978364in}}%
\pgfpathlineto{\pgfqpoint{0.953804in}{0.974539in}}%
\pgfpathlineto{\pgfqpoint{0.952880in}{0.961613in}}%
\pgfpathlineto{\pgfqpoint{0.954488in}{0.948686in}}%
\pgfpathlineto{\pgfqpoint{0.954936in}{0.947454in}}%
\pgfpathlineto{\pgfqpoint{0.967507in}{0.935760in}}%
\pgfpathclose%
\pgfpathmoveto{\pgfqpoint{0.965999in}{0.948686in}}%
\pgfpathlineto{\pgfqpoint{0.961443in}{0.961613in}}%
\pgfpathlineto{\pgfqpoint{0.963937in}{0.974539in}}%
\pgfpathlineto{\pgfqpoint{0.968205in}{0.979589in}}%
\pgfpathlineto{\pgfqpoint{0.981473in}{0.977414in}}%
\pgfpathlineto{\pgfqpoint{0.983333in}{0.974539in}}%
\pgfpathlineto{\pgfqpoint{0.985374in}{0.961613in}}%
\pgfpathlineto{\pgfqpoint{0.981648in}{0.948686in}}%
\pgfpathlineto{\pgfqpoint{0.981473in}{0.948469in}}%
\pgfpathlineto{\pgfqpoint{0.968205in}{0.946593in}}%
\pgfpathclose%
\pgfusepath{fill}%
\end{pgfscope}%
\begin{pgfscope}%
\pgfpathrectangle{\pgfqpoint{0.211875in}{0.211875in}}{\pgfqpoint{1.313625in}{1.279725in}}%
\pgfusepath{clip}%
\pgfsetbuttcap%
\pgfsetroundjoin%
\definecolor{currentfill}{rgb}{0.796501,0.105066,0.310630}%
\pgfsetfillcolor{currentfill}%
\pgfsetlinewidth{0.000000pt}%
\definecolor{currentstroke}{rgb}{0.000000,0.000000,0.000000}%
\pgfsetstrokecolor{currentstroke}%
\pgfsetdash{}{0pt}%
\pgfpathmoveto{\pgfqpoint{1.207045in}{0.935667in}}%
\pgfpathlineto{\pgfqpoint{1.207621in}{0.935760in}}%
\pgfpathlineto{\pgfqpoint{1.220314in}{0.939010in}}%
\pgfpathlineto{\pgfqpoint{1.226570in}{0.948686in}}%
\pgfpathlineto{\pgfqpoint{1.228588in}{0.961613in}}%
\pgfpathlineto{\pgfqpoint{1.227434in}{0.974539in}}%
\pgfpathlineto{\pgfqpoint{1.220910in}{0.987466in}}%
\pgfpathlineto{\pgfqpoint{1.220314in}{0.987979in}}%
\pgfpathlineto{\pgfqpoint{1.207045in}{0.990686in}}%
\pgfpathlineto{\pgfqpoint{1.198995in}{0.987466in}}%
\pgfpathlineto{\pgfqpoint{1.193777in}{0.983023in}}%
\pgfpathlineto{\pgfqpoint{1.190566in}{0.974539in}}%
\pgfpathlineto{\pgfqpoint{1.189546in}{0.961613in}}%
\pgfpathlineto{\pgfqpoint{1.191331in}{0.948686in}}%
\pgfpathlineto{\pgfqpoint{1.193777in}{0.943458in}}%
\pgfpathlineto{\pgfqpoint{1.206744in}{0.935760in}}%
\pgfpathclose%
\pgfpathmoveto{\pgfqpoint{1.203975in}{0.948686in}}%
\pgfpathlineto{\pgfqpoint{1.197917in}{0.961613in}}%
\pgfpathlineto{\pgfqpoint{1.201229in}{0.974539in}}%
\pgfpathlineto{\pgfqpoint{1.207045in}{0.979713in}}%
\pgfpathlineto{\pgfqpoint{1.218038in}{0.974539in}}%
\pgfpathlineto{\pgfqpoint{1.220314in}{0.970097in}}%
\pgfpathlineto{\pgfqpoint{1.221479in}{0.961613in}}%
\pgfpathlineto{\pgfqpoint{1.220314in}{0.956606in}}%
\pgfpathlineto{\pgfqpoint{1.212854in}{0.948686in}}%
\pgfpathlineto{\pgfqpoint{1.207045in}{0.946495in}}%
\pgfpathclose%
\pgfusepath{fill}%
\end{pgfscope}%
\begin{pgfscope}%
\pgfpathrectangle{\pgfqpoint{0.211875in}{0.211875in}}{\pgfqpoint{1.313625in}{1.279725in}}%
\pgfusepath{clip}%
\pgfsetbuttcap%
\pgfsetroundjoin%
\definecolor{currentfill}{rgb}{0.796501,0.105066,0.310630}%
\pgfsetfillcolor{currentfill}%
\pgfsetlinewidth{0.000000pt}%
\definecolor{currentstroke}{rgb}{0.000000,0.000000,0.000000}%
\pgfsetstrokecolor{currentstroke}%
\pgfsetdash{}{0pt}%
\pgfpathmoveto{\pgfqpoint{1.326466in}{0.935057in}}%
\pgfpathlineto{\pgfqpoint{1.329855in}{0.935760in}}%
\pgfpathlineto{\pgfqpoint{1.339735in}{0.939247in}}%
\pgfpathlineto{\pgfqpoint{1.345208in}{0.948686in}}%
\pgfpathlineto{\pgfqpoint{1.347033in}{0.961613in}}%
\pgfpathlineto{\pgfqpoint{1.345984in}{0.974539in}}%
\pgfpathlineto{\pgfqpoint{1.340052in}{0.987466in}}%
\pgfpathlineto{\pgfqpoint{1.339735in}{0.987770in}}%
\pgfpathlineto{\pgfqpoint{1.326466in}{0.991321in}}%
\pgfpathlineto{\pgfqpoint{1.315054in}{0.987466in}}%
\pgfpathlineto{\pgfqpoint{1.313197in}{0.986224in}}%
\pgfpathlineto{\pgfqpoint{1.308250in}{0.974539in}}%
\pgfpathlineto{\pgfqpoint{1.307228in}{0.961613in}}%
\pgfpathlineto{\pgfqpoint{1.309008in}{0.948686in}}%
\pgfpathlineto{\pgfqpoint{1.313197in}{0.940695in}}%
\pgfpathlineto{\pgfqpoint{1.323763in}{0.935760in}}%
\pgfpathclose%
\pgfpathmoveto{\pgfqpoint{1.321362in}{0.948686in}}%
\pgfpathlineto{\pgfqpoint{1.314364in}{0.961613in}}%
\pgfpathlineto{\pgfqpoint{1.318200in}{0.974539in}}%
\pgfpathlineto{\pgfqpoint{1.326466in}{0.980717in}}%
\pgfpathlineto{\pgfqpoint{1.336546in}{0.974539in}}%
\pgfpathlineto{\pgfqpoint{1.339735in}{0.965913in}}%
\pgfpathlineto{\pgfqpoint{1.340266in}{0.961613in}}%
\pgfpathlineto{\pgfqpoint{1.339735in}{0.959065in}}%
\pgfpathlineto{\pgfqpoint{1.332706in}{0.948686in}}%
\pgfpathlineto{\pgfqpoint{1.326466in}{0.945624in}}%
\pgfpathclose%
\pgfusepath{fill}%
\end{pgfscope}%
\begin{pgfscope}%
\pgfpathrectangle{\pgfqpoint{0.211875in}{0.211875in}}{\pgfqpoint{1.313625in}{1.279725in}}%
\pgfusepath{clip}%
\pgfsetbuttcap%
\pgfsetroundjoin%
\definecolor{currentfill}{rgb}{0.796501,0.105066,0.310630}%
\pgfsetfillcolor{currentfill}%
\pgfsetlinewidth{0.000000pt}%
\definecolor{currentstroke}{rgb}{0.000000,0.000000,0.000000}%
\pgfsetstrokecolor{currentstroke}%
\pgfsetdash{}{0pt}%
\pgfpathmoveto{\pgfqpoint{1.445886in}{0.933978in}}%
\pgfpathlineto{\pgfqpoint{1.452905in}{0.935760in}}%
\pgfpathlineto{\pgfqpoint{1.459155in}{0.938657in}}%
\pgfpathlineto{\pgfqpoint{1.464360in}{0.948686in}}%
\pgfpathlineto{\pgfqpoint{1.465937in}{0.961613in}}%
\pgfpathlineto{\pgfqpoint{1.465019in}{0.974539in}}%
\pgfpathlineto{\pgfqpoint{1.459841in}{0.987466in}}%
\pgfpathlineto{\pgfqpoint{1.459155in}{0.988198in}}%
\pgfpathlineto{\pgfqpoint{1.445886in}{0.992438in}}%
\pgfpathlineto{\pgfqpoint{1.432617in}{0.989118in}}%
\pgfpathlineto{\pgfqpoint{1.430892in}{0.987466in}}%
\pgfpathlineto{\pgfqpoint{1.425374in}{0.974539in}}%
\pgfpathlineto{\pgfqpoint{1.424386in}{0.961613in}}%
\pgfpathlineto{\pgfqpoint{1.426090in}{0.948686in}}%
\pgfpathlineto{\pgfqpoint{1.432617in}{0.937525in}}%
\pgfpathlineto{\pgfqpoint{1.437549in}{0.935760in}}%
\pgfpathclose%
\pgfpathmoveto{\pgfqpoint{1.436683in}{0.948686in}}%
\pgfpathlineto{\pgfqpoint{1.432617in}{0.954551in}}%
\pgfpathlineto{\pgfqpoint{1.431129in}{0.961613in}}%
\pgfpathlineto{\pgfqpoint{1.432617in}{0.973544in}}%
\pgfpathlineto{\pgfqpoint{1.432995in}{0.974539in}}%
\pgfpathlineto{\pgfqpoint{1.445886in}{0.982350in}}%
\pgfpathlineto{\pgfqpoint{1.456242in}{0.974539in}}%
\pgfpathlineto{\pgfqpoint{1.459155in}{0.964176in}}%
\pgfpathlineto{\pgfqpoint{1.459440in}{0.961613in}}%
\pgfpathlineto{\pgfqpoint{1.459155in}{0.960086in}}%
\pgfpathlineto{\pgfqpoint{1.453312in}{0.948686in}}%
\pgfpathlineto{\pgfqpoint{1.445886in}{0.944203in}}%
\pgfpathclose%
\pgfusepath{fill}%
\end{pgfscope}%
\begin{pgfscope}%
\pgfpathrectangle{\pgfqpoint{0.211875in}{0.211875in}}{\pgfqpoint{1.313625in}{1.279725in}}%
\pgfusepath{clip}%
\pgfsetbuttcap%
\pgfsetroundjoin%
\definecolor{currentfill}{rgb}{0.796501,0.105066,0.310630}%
\pgfsetfillcolor{currentfill}%
\pgfsetlinewidth{0.000000pt}%
\definecolor{currentstroke}{rgb}{0.000000,0.000000,0.000000}%
\pgfsetstrokecolor{currentstroke}%
\pgfsetdash{}{0pt}%
\pgfpathmoveto{\pgfqpoint{1.074356in}{0.945752in}}%
\pgfpathlineto{\pgfqpoint{1.087625in}{0.935830in}}%
\pgfpathlineto{\pgfqpoint{1.100894in}{0.938085in}}%
\pgfpathlineto{\pgfqpoint{1.108532in}{0.948686in}}%
\pgfpathlineto{\pgfqpoint{1.110685in}{0.961613in}}%
\pgfpathlineto{\pgfqpoint{1.109455in}{0.974539in}}%
\pgfpathlineto{\pgfqpoint{1.102512in}{0.987466in}}%
\pgfpathlineto{\pgfqpoint{1.100894in}{0.988721in}}%
\pgfpathlineto{\pgfqpoint{1.087625in}{0.990530in}}%
\pgfpathlineto{\pgfqpoint{1.081011in}{0.987466in}}%
\pgfpathlineto{\pgfqpoint{1.074356in}{0.980358in}}%
\pgfpathlineto{\pgfqpoint{1.072400in}{0.974539in}}%
\pgfpathlineto{\pgfqpoint{1.071413in}{0.961613in}}%
\pgfpathlineto{\pgfqpoint{1.073140in}{0.948686in}}%
\pgfpathclose%
\pgfpathmoveto{\pgfqpoint{1.085379in}{0.948686in}}%
\pgfpathlineto{\pgfqpoint{1.080114in}{0.961613in}}%
\pgfpathlineto{\pgfqpoint{1.082991in}{0.974539in}}%
\pgfpathlineto{\pgfqpoint{1.087625in}{0.979331in}}%
\pgfpathlineto{\pgfqpoint{1.100894in}{0.975067in}}%
\pgfpathlineto{\pgfqpoint{1.101200in}{0.974539in}}%
\pgfpathlineto{\pgfqpoint{1.103148in}{0.961613in}}%
\pgfpathlineto{\pgfqpoint{1.100894in}{0.952934in}}%
\pgfpathlineto{\pgfqpoint{1.094745in}{0.948686in}}%
\pgfpathlineto{\pgfqpoint{1.087625in}{0.946823in}}%
\pgfpathclose%
\pgfusepath{fill}%
\end{pgfscope}%
\begin{pgfscope}%
\pgfpathrectangle{\pgfqpoint{0.211875in}{0.211875in}}{\pgfqpoint{1.313625in}{1.279725in}}%
\pgfusepath{clip}%
\pgfsetbuttcap%
\pgfsetroundjoin%
\definecolor{currentfill}{rgb}{0.796501,0.105066,0.310630}%
\pgfsetfillcolor{currentfill}%
\pgfsetlinewidth{0.000000pt}%
\definecolor{currentstroke}{rgb}{0.000000,0.000000,0.000000}%
\pgfsetstrokecolor{currentstroke}%
\pgfsetdash{}{0pt}%
\pgfpathmoveto{\pgfqpoint{0.424178in}{1.009696in}}%
\pgfpathlineto{\pgfqpoint{0.437447in}{1.007700in}}%
\pgfpathlineto{\pgfqpoint{0.450716in}{1.008282in}}%
\pgfpathlineto{\pgfqpoint{0.463342in}{1.013319in}}%
\pgfpathlineto{\pgfqpoint{0.463985in}{1.013991in}}%
\pgfpathlineto{\pgfqpoint{0.467998in}{1.026245in}}%
\pgfpathlineto{\pgfqpoint{0.468885in}{1.039172in}}%
\pgfpathlineto{\pgfqpoint{0.468797in}{1.052098in}}%
\pgfpathlineto{\pgfqpoint{0.467606in}{1.065025in}}%
\pgfpathlineto{\pgfqpoint{0.463985in}{1.074782in}}%
\pgfpathlineto{\pgfqpoint{0.460069in}{1.077952in}}%
\pgfpathlineto{\pgfqpoint{0.450716in}{1.080993in}}%
\pgfpathlineto{\pgfqpoint{0.437447in}{1.081582in}}%
\pgfpathlineto{\pgfqpoint{0.424178in}{1.079436in}}%
\pgfpathlineto{\pgfqpoint{0.421822in}{1.077952in}}%
\pgfpathlineto{\pgfqpoint{0.416545in}{1.065025in}}%
\pgfpathlineto{\pgfqpoint{0.415658in}{1.052098in}}%
\pgfpathlineto{\pgfqpoint{0.415585in}{1.039172in}}%
\pgfpathlineto{\pgfqpoint{0.416214in}{1.026245in}}%
\pgfpathlineto{\pgfqpoint{0.419656in}{1.013319in}}%
\pgfpathclose%
\pgfpathmoveto{\pgfqpoint{0.424303in}{1.026245in}}%
\pgfpathlineto{\pgfqpoint{0.424178in}{1.026545in}}%
\pgfpathlineto{\pgfqpoint{0.421757in}{1.039172in}}%
\pgfpathlineto{\pgfqpoint{0.421943in}{1.052098in}}%
\pgfpathlineto{\pgfqpoint{0.424178in}{1.061941in}}%
\pgfpathlineto{\pgfqpoint{0.425780in}{1.065025in}}%
\pgfpathlineto{\pgfqpoint{0.437447in}{1.072703in}}%
\pgfpathlineto{\pgfqpoint{0.450716in}{1.071497in}}%
\pgfpathlineto{\pgfqpoint{0.457929in}{1.065025in}}%
\pgfpathlineto{\pgfqpoint{0.462461in}{1.052098in}}%
\pgfpathlineto{\pgfqpoint{0.462726in}{1.039172in}}%
\pgfpathlineto{\pgfqpoint{0.459153in}{1.026245in}}%
\pgfpathlineto{\pgfqpoint{0.450716in}{1.017752in}}%
\pgfpathlineto{\pgfqpoint{0.437447in}{1.016493in}}%
\pgfpathclose%
\pgfusepath{fill}%
\end{pgfscope}%
\begin{pgfscope}%
\pgfpathrectangle{\pgfqpoint{0.211875in}{0.211875in}}{\pgfqpoint{1.313625in}{1.279725in}}%
\pgfusepath{clip}%
\pgfsetbuttcap%
\pgfsetroundjoin%
\definecolor{currentfill}{rgb}{0.796501,0.105066,0.310630}%
\pgfsetfillcolor{currentfill}%
\pgfsetlinewidth{0.000000pt}%
\definecolor{currentstroke}{rgb}{0.000000,0.000000,0.000000}%
\pgfsetstrokecolor{currentstroke}%
\pgfsetdash{}{0pt}%
\pgfpathmoveto{\pgfqpoint{0.556867in}{1.010613in}}%
\pgfpathlineto{\pgfqpoint{0.570136in}{1.011729in}}%
\pgfpathlineto{\pgfqpoint{0.573744in}{1.013319in}}%
\pgfpathlineto{\pgfqpoint{0.583379in}{1.026245in}}%
\pgfpathlineto{\pgfqpoint{0.583405in}{1.026382in}}%
\pgfpathlineto{\pgfqpoint{0.584853in}{1.039172in}}%
\pgfpathlineto{\pgfqpoint{0.584731in}{1.052098in}}%
\pgfpathlineto{\pgfqpoint{0.583405in}{1.061851in}}%
\pgfpathlineto{\pgfqpoint{0.582644in}{1.065025in}}%
\pgfpathlineto{\pgfqpoint{0.570136in}{1.077516in}}%
\pgfpathlineto{\pgfqpoint{0.566059in}{1.077952in}}%
\pgfpathlineto{\pgfqpoint{0.556867in}{1.078706in}}%
\pgfpathlineto{\pgfqpoint{0.552759in}{1.077952in}}%
\pgfpathlineto{\pgfqpoint{0.543598in}{1.074846in}}%
\pgfpathlineto{\pgfqpoint{0.537773in}{1.065025in}}%
\pgfpathlineto{\pgfqpoint{0.535988in}{1.052098in}}%
\pgfpathlineto{\pgfqpoint{0.535870in}{1.039172in}}%
\pgfpathlineto{\pgfqpoint{0.537226in}{1.026245in}}%
\pgfpathlineto{\pgfqpoint{0.543598in}{1.014098in}}%
\pgfpathlineto{\pgfqpoint{0.545319in}{1.013319in}}%
\pgfpathclose%
\pgfpathmoveto{\pgfqpoint{0.546945in}{1.026245in}}%
\pgfpathlineto{\pgfqpoint{0.543598in}{1.032613in}}%
\pgfpathlineto{\pgfqpoint{0.542162in}{1.039172in}}%
\pgfpathlineto{\pgfqpoint{0.542395in}{1.052098in}}%
\pgfpathlineto{\pgfqpoint{0.543598in}{1.056753in}}%
\pgfpathlineto{\pgfqpoint{0.548993in}{1.065025in}}%
\pgfpathlineto{\pgfqpoint{0.556867in}{1.069548in}}%
\pgfpathlineto{\pgfqpoint{0.570136in}{1.067117in}}%
\pgfpathlineto{\pgfqpoint{0.572231in}{1.065025in}}%
\pgfpathlineto{\pgfqpoint{0.577409in}{1.052098in}}%
\pgfpathlineto{\pgfqpoint{0.577702in}{1.039172in}}%
\pgfpathlineto{\pgfqpoint{0.573600in}{1.026245in}}%
\pgfpathlineto{\pgfqpoint{0.570136in}{1.022362in}}%
\pgfpathlineto{\pgfqpoint{0.556867in}{1.019822in}}%
\pgfpathclose%
\pgfusepath{fill}%
\end{pgfscope}%
\begin{pgfscope}%
\pgfpathrectangle{\pgfqpoint{0.211875in}{0.211875in}}{\pgfqpoint{1.313625in}{1.279725in}}%
\pgfusepath{clip}%
\pgfsetbuttcap%
\pgfsetroundjoin%
\definecolor{currentfill}{rgb}{0.796501,0.105066,0.310630}%
\pgfsetfillcolor{currentfill}%
\pgfsetlinewidth{0.000000pt}%
\definecolor{currentstroke}{rgb}{0.000000,0.000000,0.000000}%
\pgfsetstrokecolor{currentstroke}%
\pgfsetdash{}{0pt}%
\pgfpathmoveto{\pgfqpoint{0.676288in}{1.012935in}}%
\pgfpathlineto{\pgfqpoint{0.679255in}{1.013319in}}%
\pgfpathlineto{\pgfqpoint{0.689557in}{1.015246in}}%
\pgfpathlineto{\pgfqpoint{0.698377in}{1.026245in}}%
\pgfpathlineto{\pgfqpoint{0.700778in}{1.039172in}}%
\pgfpathlineto{\pgfqpoint{0.700591in}{1.052098in}}%
\pgfpathlineto{\pgfqpoint{0.697512in}{1.065025in}}%
\pgfpathlineto{\pgfqpoint{0.689557in}{1.073853in}}%
\pgfpathlineto{\pgfqpoint{0.676288in}{1.076165in}}%
\pgfpathlineto{\pgfqpoint{0.663019in}{1.071754in}}%
\pgfpathlineto{\pgfqpoint{0.658520in}{1.065025in}}%
\pgfpathlineto{\pgfqpoint{0.655949in}{1.052098in}}%
\pgfpathlineto{\pgfqpoint{0.655791in}{1.039172in}}%
\pgfpathlineto{\pgfqpoint{0.657782in}{1.026245in}}%
\pgfpathlineto{\pgfqpoint{0.663019in}{1.017401in}}%
\pgfpathlineto{\pgfqpoint{0.674386in}{1.013319in}}%
\pgfpathclose%
\pgfpathmoveto{\pgfqpoint{0.669555in}{1.026245in}}%
\pgfpathlineto{\pgfqpoint{0.663019in}{1.036069in}}%
\pgfpathlineto{\pgfqpoint{0.662251in}{1.039172in}}%
\pgfpathlineto{\pgfqpoint{0.662525in}{1.052098in}}%
\pgfpathlineto{\pgfqpoint{0.663019in}{1.053790in}}%
\pgfpathlineto{\pgfqpoint{0.672275in}{1.065025in}}%
\pgfpathlineto{\pgfqpoint{0.676288in}{1.067004in}}%
\pgfpathlineto{\pgfqpoint{0.683320in}{1.065025in}}%
\pgfpathlineto{\pgfqpoint{0.689557in}{1.061506in}}%
\pgfpathlineto{\pgfqpoint{0.693294in}{1.052098in}}%
\pgfpathlineto{\pgfqpoint{0.693607in}{1.039172in}}%
\pgfpathlineto{\pgfqpoint{0.689557in}{1.027203in}}%
\pgfpathlineto{\pgfqpoint{0.688171in}{1.026245in}}%
\pgfpathlineto{\pgfqpoint{0.676288in}{1.022504in}}%
\pgfpathclose%
\pgfusepath{fill}%
\end{pgfscope}%
\begin{pgfscope}%
\pgfpathrectangle{\pgfqpoint{0.211875in}{0.211875in}}{\pgfqpoint{1.313625in}{1.279725in}}%
\pgfusepath{clip}%
\pgfsetbuttcap%
\pgfsetroundjoin%
\definecolor{currentfill}{rgb}{0.796501,0.105066,0.310630}%
\pgfsetfillcolor{currentfill}%
\pgfsetlinewidth{0.000000pt}%
\definecolor{currentstroke}{rgb}{0.000000,0.000000,0.000000}%
\pgfsetstrokecolor{currentstroke}%
\pgfsetdash{}{0pt}%
\pgfpathmoveto{\pgfqpoint{0.782439in}{1.019377in}}%
\pgfpathlineto{\pgfqpoint{0.795708in}{1.015030in}}%
\pgfpathlineto{\pgfqpoint{0.808977in}{1.018800in}}%
\pgfpathlineto{\pgfqpoint{0.814344in}{1.026245in}}%
\pgfpathlineto{\pgfqpoint{0.817063in}{1.039172in}}%
\pgfpathlineto{\pgfqpoint{0.816859in}{1.052098in}}%
\pgfpathlineto{\pgfqpoint{0.813392in}{1.065025in}}%
\pgfpathlineto{\pgfqpoint{0.808977in}{1.070470in}}%
\pgfpathlineto{\pgfqpoint{0.795708in}{1.074100in}}%
\pgfpathlineto{\pgfqpoint{0.782439in}{1.069907in}}%
\pgfpathlineto{\pgfqpoint{0.778785in}{1.065025in}}%
\pgfpathlineto{\pgfqpoint{0.775531in}{1.052098in}}%
\pgfpathlineto{\pgfqpoint{0.775339in}{1.039172in}}%
\pgfpathlineto{\pgfqpoint{0.777881in}{1.026245in}}%
\pgfpathclose%
\pgfpathmoveto{\pgfqpoint{0.792214in}{1.026245in}}%
\pgfpathlineto{\pgfqpoint{0.782439in}{1.037664in}}%
\pgfpathlineto{\pgfqpoint{0.782020in}{1.039172in}}%
\pgfpathlineto{\pgfqpoint{0.782333in}{1.052098in}}%
\pgfpathlineto{\pgfqpoint{0.782439in}{1.052424in}}%
\pgfpathlineto{\pgfqpoint{0.795708in}{1.064980in}}%
\pgfpathlineto{\pgfqpoint{0.808977in}{1.054717in}}%
\pgfpathlineto{\pgfqpoint{0.809912in}{1.052098in}}%
\pgfpathlineto{\pgfqpoint{0.810237in}{1.039172in}}%
\pgfpathlineto{\pgfqpoint{0.808977in}{1.035024in}}%
\pgfpathlineto{\pgfqpoint{0.799727in}{1.026245in}}%
\pgfpathlineto{\pgfqpoint{0.795708in}{1.024618in}}%
\pgfpathclose%
\pgfusepath{fill}%
\end{pgfscope}%
\begin{pgfscope}%
\pgfpathrectangle{\pgfqpoint{0.211875in}{0.211875in}}{\pgfqpoint{1.313625in}{1.279725in}}%
\pgfusepath{clip}%
\pgfsetbuttcap%
\pgfsetroundjoin%
\definecolor{currentfill}{rgb}{0.796501,0.105066,0.310630}%
\pgfsetfillcolor{currentfill}%
\pgfsetlinewidth{0.000000pt}%
\definecolor{currentstroke}{rgb}{0.000000,0.000000,0.000000}%
\pgfsetstrokecolor{currentstroke}%
\pgfsetdash{}{0pt}%
\pgfpathmoveto{\pgfqpoint{0.901860in}{1.020357in}}%
\pgfpathlineto{\pgfqpoint{0.915129in}{1.016629in}}%
\pgfpathlineto{\pgfqpoint{0.928398in}{1.022072in}}%
\pgfpathlineto{\pgfqpoint{0.931095in}{1.026245in}}%
\pgfpathlineto{\pgfqpoint{0.934002in}{1.039172in}}%
\pgfpathlineto{\pgfqpoint{0.933788in}{1.052098in}}%
\pgfpathlineto{\pgfqpoint{0.930093in}{1.065025in}}%
\pgfpathlineto{\pgfqpoint{0.928398in}{1.067355in}}%
\pgfpathlineto{\pgfqpoint{0.915129in}{1.072583in}}%
\pgfpathlineto{\pgfqpoint{0.901860in}{1.068995in}}%
\pgfpathlineto{\pgfqpoint{0.898547in}{1.065025in}}%
\pgfpathlineto{\pgfqpoint{0.894708in}{1.052098in}}%
\pgfpathlineto{\pgfqpoint{0.894485in}{1.039172in}}%
\pgfpathlineto{\pgfqpoint{0.897498in}{1.026245in}}%
\pgfpathclose%
\pgfpathmoveto{\pgfqpoint{0.915054in}{1.026245in}}%
\pgfpathlineto{\pgfqpoint{0.901860in}{1.037858in}}%
\pgfpathlineto{\pgfqpoint{0.901452in}{1.039172in}}%
\pgfpathlineto{\pgfqpoint{0.901799in}{1.052098in}}%
\pgfpathlineto{\pgfqpoint{0.901860in}{1.052264in}}%
\pgfpathlineto{\pgfqpoint{0.915129in}{1.062426in}}%
\pgfpathlineto{\pgfqpoint{0.925458in}{1.052098in}}%
\pgfpathlineto{\pgfqpoint{0.926217in}{1.039172in}}%
\pgfpathlineto{\pgfqpoint{0.915186in}{1.026245in}}%
\pgfpathlineto{\pgfqpoint{0.915129in}{1.026217in}}%
\pgfpathclose%
\pgfusepath{fill}%
\end{pgfscope}%
\begin{pgfscope}%
\pgfpathrectangle{\pgfqpoint{0.211875in}{0.211875in}}{\pgfqpoint{1.313625in}{1.279725in}}%
\pgfusepath{clip}%
\pgfsetbuttcap%
\pgfsetroundjoin%
\definecolor{currentfill}{rgb}{0.796501,0.105066,0.310630}%
\pgfsetfillcolor{currentfill}%
\pgfsetlinewidth{0.000000pt}%
\definecolor{currentstroke}{rgb}{0.000000,0.000000,0.000000}%
\pgfsetstrokecolor{currentstroke}%
\pgfsetdash{}{0pt}%
\pgfpathmoveto{\pgfqpoint{1.021280in}{1.020555in}}%
\pgfpathlineto{\pgfqpoint{1.034549in}{1.017680in}}%
\pgfpathlineto{\pgfqpoint{1.047818in}{1.025058in}}%
\pgfpathlineto{\pgfqpoint{1.048503in}{1.026245in}}%
\pgfpathlineto{\pgfqpoint{1.051482in}{1.039172in}}%
\pgfpathlineto{\pgfqpoint{1.051265in}{1.052098in}}%
\pgfpathlineto{\pgfqpoint{1.047818in}{1.064161in}}%
\pgfpathlineto{\pgfqpoint{1.047145in}{1.065025in}}%
\pgfpathlineto{\pgfqpoint{1.034549in}{1.071583in}}%
\pgfpathlineto{\pgfqpoint{1.021280in}{1.068820in}}%
\pgfpathlineto{\pgfqpoint{1.017755in}{1.065025in}}%
\pgfpathlineto{\pgfqpoint{1.013430in}{1.052098in}}%
\pgfpathlineto{\pgfqpoint{1.013182in}{1.039172in}}%
\pgfpathlineto{\pgfqpoint{1.016585in}{1.026245in}}%
\pgfpathclose%
\pgfpathmoveto{\pgfqpoint{1.020508in}{1.039172in}}%
\pgfpathlineto{\pgfqpoint{1.020888in}{1.052098in}}%
\pgfpathlineto{\pgfqpoint{1.021280in}{1.053064in}}%
\pgfpathlineto{\pgfqpoint{1.034549in}{1.060643in}}%
\pgfpathlineto{\pgfqpoint{1.041758in}{1.052098in}}%
\pgfpathlineto{\pgfqpoint{1.042422in}{1.039172in}}%
\pgfpathlineto{\pgfqpoint{1.034549in}{1.028243in}}%
\pgfpathlineto{\pgfqpoint{1.021280in}{1.036942in}}%
\pgfpathclose%
\pgfusepath{fill}%
\end{pgfscope}%
\begin{pgfscope}%
\pgfpathrectangle{\pgfqpoint{0.211875in}{0.211875in}}{\pgfqpoint{1.313625in}{1.279725in}}%
\pgfusepath{clip}%
\pgfsetbuttcap%
\pgfsetroundjoin%
\definecolor{currentfill}{rgb}{0.796501,0.105066,0.310630}%
\pgfsetfillcolor{currentfill}%
\pgfsetlinewidth{0.000000pt}%
\definecolor{currentstroke}{rgb}{0.000000,0.000000,0.000000}%
\pgfsetstrokecolor{currentstroke}%
\pgfsetdash{}{0pt}%
\pgfpathmoveto{\pgfqpoint{1.140701in}{1.020106in}}%
\pgfpathlineto{\pgfqpoint{1.153970in}{1.018194in}}%
\pgfpathlineto{\pgfqpoint{1.165865in}{1.026245in}}%
\pgfpathlineto{\pgfqpoint{1.167239in}{1.028984in}}%
\pgfpathlineto{\pgfqpoint{1.169424in}{1.039172in}}%
\pgfpathlineto{\pgfqpoint{1.169210in}{1.052098in}}%
\pgfpathlineto{\pgfqpoint{1.167239in}{1.059880in}}%
\pgfpathlineto{\pgfqpoint{1.164045in}{1.065025in}}%
\pgfpathlineto{\pgfqpoint{1.153970in}{1.071093in}}%
\pgfpathlineto{\pgfqpoint{1.140701in}{1.069256in}}%
\pgfpathlineto{\pgfqpoint{1.136326in}{1.065025in}}%
\pgfpathlineto{\pgfqpoint{1.131619in}{1.052098in}}%
\pgfpathlineto{\pgfqpoint{1.131349in}{1.039172in}}%
\pgfpathlineto{\pgfqpoint{1.135058in}{1.026245in}}%
\pgfpathclose%
\pgfpathmoveto{\pgfqpoint{1.139129in}{1.039172in}}%
\pgfpathlineto{\pgfqpoint{1.139537in}{1.052098in}}%
\pgfpathlineto{\pgfqpoint{1.140701in}{1.054667in}}%
\pgfpathlineto{\pgfqpoint{1.153970in}{1.059614in}}%
\pgfpathlineto{\pgfqpoint{1.159437in}{1.052098in}}%
\pgfpathlineto{\pgfqpoint{1.160023in}{1.039172in}}%
\pgfpathlineto{\pgfqpoint{1.153970in}{1.029424in}}%
\pgfpathlineto{\pgfqpoint{1.140701in}{1.035101in}}%
\pgfpathclose%
\pgfusepath{fill}%
\end{pgfscope}%
\begin{pgfscope}%
\pgfpathrectangle{\pgfqpoint{0.211875in}{0.211875in}}{\pgfqpoint{1.313625in}{1.279725in}}%
\pgfusepath{clip}%
\pgfsetbuttcap%
\pgfsetroundjoin%
\definecolor{currentfill}{rgb}{0.796501,0.105066,0.310630}%
\pgfsetfillcolor{currentfill}%
\pgfsetlinewidth{0.000000pt}%
\definecolor{currentstroke}{rgb}{0.000000,0.000000,0.000000}%
\pgfsetstrokecolor{currentstroke}%
\pgfsetdash{}{0pt}%
\pgfpathmoveto{\pgfqpoint{1.260121in}{1.019098in}}%
\pgfpathlineto{\pgfqpoint{1.273390in}{1.018158in}}%
\pgfpathlineto{\pgfqpoint{1.283865in}{1.026245in}}%
\pgfpathlineto{\pgfqpoint{1.286659in}{1.033226in}}%
\pgfpathlineto{\pgfqpoint{1.287776in}{1.039172in}}%
\pgfpathlineto{\pgfqpoint{1.287571in}{1.052098in}}%
\pgfpathlineto{\pgfqpoint{1.286659in}{1.056196in}}%
\pgfpathlineto{\pgfqpoint{1.282272in}{1.065025in}}%
\pgfpathlineto{\pgfqpoint{1.273390in}{1.071123in}}%
\pgfpathlineto{\pgfqpoint{1.260121in}{1.070221in}}%
\pgfpathlineto{\pgfqpoint{1.254129in}{1.065025in}}%
\pgfpathlineto{\pgfqpoint{1.249153in}{1.052098in}}%
\pgfpathlineto{\pgfqpoint{1.248867in}{1.039172in}}%
\pgfpathlineto{\pgfqpoint{1.252788in}{1.026245in}}%
\pgfpathclose%
\pgfpathmoveto{\pgfqpoint{1.257217in}{1.039172in}}%
\pgfpathlineto{\pgfqpoint{1.257651in}{1.052098in}}%
\pgfpathlineto{\pgfqpoint{1.260121in}{1.056979in}}%
\pgfpathlineto{\pgfqpoint{1.273390in}{1.059353in}}%
\pgfpathlineto{\pgfqpoint{1.278010in}{1.052098in}}%
\pgfpathlineto{\pgfqpoint{1.278527in}{1.039172in}}%
\pgfpathlineto{\pgfqpoint{1.273390in}{1.029720in}}%
\pgfpathlineto{\pgfqpoint{1.260121in}{1.032444in}}%
\pgfpathclose%
\pgfusepath{fill}%
\end{pgfscope}%
\begin{pgfscope}%
\pgfpathrectangle{\pgfqpoint{0.211875in}{0.211875in}}{\pgfqpoint{1.313625in}{1.279725in}}%
\pgfusepath{clip}%
\pgfsetbuttcap%
\pgfsetroundjoin%
\definecolor{currentfill}{rgb}{0.796501,0.105066,0.310630}%
\pgfsetfillcolor{currentfill}%
\pgfsetlinewidth{0.000000pt}%
\definecolor{currentstroke}{rgb}{0.000000,0.000000,0.000000}%
\pgfsetstrokecolor{currentstroke}%
\pgfsetdash{}{0pt}%
\pgfpathmoveto{\pgfqpoint{1.379542in}{1.017579in}}%
\pgfpathlineto{\pgfqpoint{1.392811in}{1.017534in}}%
\pgfpathlineto{\pgfqpoint{1.402801in}{1.026245in}}%
\pgfpathlineto{\pgfqpoint{1.406080in}{1.036562in}}%
\pgfpathlineto{\pgfqpoint{1.406502in}{1.039172in}}%
\pgfpathlineto{\pgfqpoint{1.406311in}{1.052098in}}%
\pgfpathlineto{\pgfqpoint{1.406080in}{1.053302in}}%
\pgfpathlineto{\pgfqpoint{1.401437in}{1.065025in}}%
\pgfpathlineto{\pgfqpoint{1.392811in}{1.071710in}}%
\pgfpathlineto{\pgfqpoint{1.379542in}{1.071668in}}%
\pgfpathlineto{\pgfqpoint{1.370960in}{1.065025in}}%
\pgfpathlineto{\pgfqpoint{1.366273in}{1.053506in}}%
\pgfpathlineto{\pgfqpoint{1.365996in}{1.052098in}}%
\pgfpathlineto{\pgfqpoint{1.365802in}{1.039172in}}%
\pgfpathlineto{\pgfqpoint{1.366273in}{1.036332in}}%
\pgfpathlineto{\pgfqpoint{1.369575in}{1.026245in}}%
\pgfpathclose%
\pgfpathmoveto{\pgfqpoint{1.374625in}{1.039172in}}%
\pgfpathlineto{\pgfqpoint{1.375082in}{1.052098in}}%
\pgfpathlineto{\pgfqpoint{1.379542in}{1.059950in}}%
\pgfpathlineto{\pgfqpoint{1.392811in}{1.059906in}}%
\pgfpathlineto{\pgfqpoint{1.397210in}{1.052098in}}%
\pgfpathlineto{\pgfqpoint{1.397662in}{1.039172in}}%
\pgfpathlineto{\pgfqpoint{1.392811in}{1.029077in}}%
\pgfpathlineto{\pgfqpoint{1.379542in}{1.029029in}}%
\pgfpathclose%
\pgfusepath{fill}%
\end{pgfscope}%
\begin{pgfscope}%
\pgfpathrectangle{\pgfqpoint{0.211875in}{0.211875in}}{\pgfqpoint{1.313625in}{1.279725in}}%
\pgfusepath{clip}%
\pgfsetbuttcap%
\pgfsetroundjoin%
\definecolor{currentfill}{rgb}{0.796501,0.105066,0.310630}%
\pgfsetfillcolor{currentfill}%
\pgfsetlinewidth{0.000000pt}%
\definecolor{currentstroke}{rgb}{0.000000,0.000000,0.000000}%
\pgfsetstrokecolor{currentstroke}%
\pgfsetdash{}{0pt}%
\pgfpathmoveto{\pgfqpoint{1.485693in}{1.025474in}}%
\pgfpathlineto{\pgfqpoint{1.498962in}{1.015574in}}%
\pgfpathlineto{\pgfqpoint{1.512231in}{1.016258in}}%
\pgfpathlineto{\pgfqpoint{1.522443in}{1.026245in}}%
\pgfpathlineto{\pgfqpoint{1.525500in}{1.038548in}}%
\pgfpathlineto{\pgfqpoint{1.525500in}{1.039172in}}%
\pgfpathlineto{\pgfqpoint{1.525500in}{1.045663in}}%
\pgfpathlineto{\pgfqpoint{1.525380in}{1.052098in}}%
\pgfpathlineto{\pgfqpoint{1.521312in}{1.065025in}}%
\pgfpathlineto{\pgfqpoint{1.512231in}{1.072913in}}%
\pgfpathlineto{\pgfqpoint{1.498962in}{1.073577in}}%
\pgfpathlineto{\pgfqpoint{1.486503in}{1.065025in}}%
\pgfpathlineto{\pgfqpoint{1.485693in}{1.063432in}}%
\pgfpathlineto{\pgfqpoint{1.483125in}{1.052098in}}%
\pgfpathlineto{\pgfqpoint{1.482945in}{1.039172in}}%
\pgfpathlineto{\pgfqpoint{1.485342in}{1.026245in}}%
\pgfpathclose%
\pgfpathmoveto{\pgfqpoint{1.497993in}{1.026245in}}%
\pgfpathlineto{\pgfqpoint{1.491123in}{1.039172in}}%
\pgfpathlineto{\pgfqpoint{1.491598in}{1.052098in}}%
\pgfpathlineto{\pgfqpoint{1.498962in}{1.063565in}}%
\pgfpathlineto{\pgfqpoint{1.512231in}{1.061362in}}%
\pgfpathlineto{\pgfqpoint{1.516880in}{1.052098in}}%
\pgfpathlineto{\pgfqpoint{1.517271in}{1.039172in}}%
\pgfpathlineto{\pgfqpoint{1.512231in}{1.027392in}}%
\pgfpathlineto{\pgfqpoint{1.506410in}{1.026245in}}%
\pgfpathlineto{\pgfqpoint{1.498962in}{1.025499in}}%
\pgfpathclose%
\pgfusepath{fill}%
\end{pgfscope}%
\begin{pgfscope}%
\pgfpathrectangle{\pgfqpoint{0.211875in}{0.211875in}}{\pgfqpoint{1.313625in}{1.279725in}}%
\pgfusepath{clip}%
\pgfsetbuttcap%
\pgfsetroundjoin%
\definecolor{currentfill}{rgb}{0.796501,0.105066,0.310630}%
\pgfsetfillcolor{currentfill}%
\pgfsetlinewidth{0.000000pt}%
\definecolor{currentstroke}{rgb}{0.000000,0.000000,0.000000}%
\pgfsetstrokecolor{currentstroke}%
\pgfsetdash{}{0pt}%
\pgfpathmoveto{\pgfqpoint{0.371102in}{1.087609in}}%
\pgfpathlineto{\pgfqpoint{0.384371in}{1.087492in}}%
\pgfpathlineto{\pgfqpoint{0.397640in}{1.088763in}}%
\pgfpathlineto{\pgfqpoint{0.403115in}{1.090878in}}%
\pgfpathlineto{\pgfqpoint{0.409932in}{1.103805in}}%
\pgfpathlineto{\pgfqpoint{0.410909in}{1.115259in}}%
\pgfpathlineto{\pgfqpoint{0.410990in}{1.116731in}}%
\pgfpathlineto{\pgfqpoint{0.411112in}{1.129658in}}%
\pgfpathlineto{\pgfqpoint{0.410909in}{1.135462in}}%
\pgfpathlineto{\pgfqpoint{0.410574in}{1.142584in}}%
\pgfpathlineto{\pgfqpoint{0.407951in}{1.155511in}}%
\pgfpathlineto{\pgfqpoint{0.397640in}{1.163045in}}%
\pgfpathlineto{\pgfqpoint{0.384371in}{1.164336in}}%
\pgfpathlineto{\pgfqpoint{0.371102in}{1.164217in}}%
\pgfpathlineto{\pgfqpoint{0.357833in}{1.158983in}}%
\pgfpathlineto{\pgfqpoint{0.356626in}{1.155511in}}%
\pgfpathlineto{\pgfqpoint{0.355426in}{1.142584in}}%
\pgfpathlineto{\pgfqpoint{0.355186in}{1.129658in}}%
\pgfpathlineto{\pgfqpoint{0.355243in}{1.116731in}}%
\pgfpathlineto{\pgfqpoint{0.355708in}{1.103805in}}%
\pgfpathlineto{\pgfqpoint{0.357833in}{1.092987in}}%
\pgfpathlineto{\pgfqpoint{0.359447in}{1.090878in}}%
\pgfpathclose%
\pgfpathmoveto{\pgfqpoint{0.366036in}{1.103805in}}%
\pgfpathlineto{\pgfqpoint{0.362201in}{1.116731in}}%
\pgfpathlineto{\pgfqpoint{0.361732in}{1.129658in}}%
\pgfpathlineto{\pgfqpoint{0.363707in}{1.142584in}}%
\pgfpathlineto{\pgfqpoint{0.371102in}{1.153605in}}%
\pgfpathlineto{\pgfqpoint{0.378825in}{1.155511in}}%
\pgfpathlineto{\pgfqpoint{0.384371in}{1.156357in}}%
\pgfpathlineto{\pgfqpoint{0.387815in}{1.155511in}}%
\pgfpathlineto{\pgfqpoint{0.397640in}{1.151384in}}%
\pgfpathlineto{\pgfqpoint{0.402711in}{1.142584in}}%
\pgfpathlineto{\pgfqpoint{0.404739in}{1.129658in}}%
\pgfpathlineto{\pgfqpoint{0.404253in}{1.116731in}}%
\pgfpathlineto{\pgfqpoint{0.400378in}{1.103805in}}%
\pgfpathlineto{\pgfqpoint{0.397640in}{1.100400in}}%
\pgfpathlineto{\pgfqpoint{0.384371in}{1.095875in}}%
\pgfpathlineto{\pgfqpoint{0.371102in}{1.098465in}}%
\pgfpathclose%
\pgfusepath{fill}%
\end{pgfscope}%
\begin{pgfscope}%
\pgfpathrectangle{\pgfqpoint{0.211875in}{0.211875in}}{\pgfqpoint{1.313625in}{1.279725in}}%
\pgfusepath{clip}%
\pgfsetbuttcap%
\pgfsetroundjoin%
\definecolor{currentfill}{rgb}{0.796501,0.105066,0.310630}%
\pgfsetfillcolor{currentfill}%
\pgfsetlinewidth{0.000000pt}%
\definecolor{currentstroke}{rgb}{0.000000,0.000000,0.000000}%
\pgfsetstrokecolor{currentstroke}%
\pgfsetdash{}{0pt}%
\pgfpathmoveto{\pgfqpoint{0.503792in}{1.090660in}}%
\pgfpathlineto{\pgfqpoint{0.505196in}{1.090878in}}%
\pgfpathlineto{\pgfqpoint{0.517061in}{1.093875in}}%
\pgfpathlineto{\pgfqpoint{0.524206in}{1.103805in}}%
\pgfpathlineto{\pgfqpoint{0.526276in}{1.116731in}}%
\pgfpathlineto{\pgfqpoint{0.526535in}{1.129658in}}%
\pgfpathlineto{\pgfqpoint{0.525453in}{1.142584in}}%
\pgfpathlineto{\pgfqpoint{0.520335in}{1.155511in}}%
\pgfpathlineto{\pgfqpoint{0.517061in}{1.158175in}}%
\pgfpathlineto{\pgfqpoint{0.503792in}{1.161120in}}%
\pgfpathlineto{\pgfqpoint{0.490523in}{1.160437in}}%
\pgfpathlineto{\pgfqpoint{0.481297in}{1.155511in}}%
\pgfpathlineto{\pgfqpoint{0.477254in}{1.147003in}}%
\pgfpathlineto{\pgfqpoint{0.476394in}{1.142584in}}%
\pgfpathlineto{\pgfqpoint{0.475566in}{1.129658in}}%
\pgfpathlineto{\pgfqpoint{0.475763in}{1.116731in}}%
\pgfpathlineto{\pgfqpoint{0.477254in}{1.104449in}}%
\pgfpathlineto{\pgfqpoint{0.477393in}{1.103805in}}%
\pgfpathlineto{\pgfqpoint{0.490523in}{1.091386in}}%
\pgfpathlineto{\pgfqpoint{0.498952in}{1.090878in}}%
\pgfpathclose%
\pgfpathmoveto{\pgfqpoint{0.488590in}{1.103805in}}%
\pgfpathlineto{\pgfqpoint{0.483158in}{1.116731in}}%
\pgfpathlineto{\pgfqpoint{0.482493in}{1.129658in}}%
\pgfpathlineto{\pgfqpoint{0.485293in}{1.142584in}}%
\pgfpathlineto{\pgfqpoint{0.490523in}{1.149573in}}%
\pgfpathlineto{\pgfqpoint{0.503792in}{1.152457in}}%
\pgfpathlineto{\pgfqpoint{0.517061in}{1.144098in}}%
\pgfpathlineto{\pgfqpoint{0.517840in}{1.142584in}}%
\pgfpathlineto{\pgfqpoint{0.520352in}{1.129658in}}%
\pgfpathlineto{\pgfqpoint{0.519751in}{1.116731in}}%
\pgfpathlineto{\pgfqpoint{0.517061in}{1.108414in}}%
\pgfpathlineto{\pgfqpoint{0.512849in}{1.103805in}}%
\pgfpathlineto{\pgfqpoint{0.503792in}{1.099465in}}%
\pgfpathlineto{\pgfqpoint{0.490523in}{1.101977in}}%
\pgfpathclose%
\pgfusepath{fill}%
\end{pgfscope}%
\begin{pgfscope}%
\pgfpathrectangle{\pgfqpoint{0.211875in}{0.211875in}}{\pgfqpoint{1.313625in}{1.279725in}}%
\pgfusepath{clip}%
\pgfsetbuttcap%
\pgfsetroundjoin%
\definecolor{currentfill}{rgb}{0.796501,0.105066,0.310630}%
\pgfsetfillcolor{currentfill}%
\pgfsetlinewidth{0.000000pt}%
\definecolor{currentstroke}{rgb}{0.000000,0.000000,0.000000}%
\pgfsetstrokecolor{currentstroke}%
\pgfsetdash{}{0pt}%
\pgfpathmoveto{\pgfqpoint{0.609943in}{1.094477in}}%
\pgfpathlineto{\pgfqpoint{0.623212in}{1.093674in}}%
\pgfpathlineto{\pgfqpoint{0.636481in}{1.099360in}}%
\pgfpathlineto{\pgfqpoint{0.639323in}{1.103805in}}%
\pgfpathlineto{\pgfqpoint{0.642190in}{1.116731in}}%
\pgfpathlineto{\pgfqpoint{0.642548in}{1.129658in}}%
\pgfpathlineto{\pgfqpoint{0.641052in}{1.142584in}}%
\pgfpathlineto{\pgfqpoint{0.636481in}{1.152577in}}%
\pgfpathlineto{\pgfqpoint{0.631965in}{1.155511in}}%
\pgfpathlineto{\pgfqpoint{0.623212in}{1.158357in}}%
\pgfpathlineto{\pgfqpoint{0.609943in}{1.157628in}}%
\pgfpathlineto{\pgfqpoint{0.605536in}{1.155511in}}%
\pgfpathlineto{\pgfqpoint{0.596922in}{1.142584in}}%
\pgfpathlineto{\pgfqpoint{0.596674in}{1.141045in}}%
\pgfpathlineto{\pgfqpoint{0.595552in}{1.129658in}}%
\pgfpathlineto{\pgfqpoint{0.595864in}{1.116731in}}%
\pgfpathlineto{\pgfqpoint{0.596674in}{1.111512in}}%
\pgfpathlineto{\pgfqpoint{0.598952in}{1.103805in}}%
\pgfpathclose%
\pgfpathmoveto{\pgfqpoint{0.614693in}{1.103805in}}%
\pgfpathlineto{\pgfqpoint{0.609943in}{1.104976in}}%
\pgfpathlineto{\pgfqpoint{0.603825in}{1.116731in}}%
\pgfpathlineto{\pgfqpoint{0.602967in}{1.129658in}}%
\pgfpathlineto{\pgfqpoint{0.606578in}{1.142584in}}%
\pgfpathlineto{\pgfqpoint{0.609943in}{1.146616in}}%
\pgfpathlineto{\pgfqpoint{0.623212in}{1.148840in}}%
\pgfpathlineto{\pgfqpoint{0.631349in}{1.142584in}}%
\pgfpathlineto{\pgfqpoint{0.636481in}{1.129895in}}%
\pgfpathlineto{\pgfqpoint{0.636524in}{1.129658in}}%
\pgfpathlineto{\pgfqpoint{0.636481in}{1.128808in}}%
\pgfpathlineto{\pgfqpoint{0.635310in}{1.116731in}}%
\pgfpathlineto{\pgfqpoint{0.625369in}{1.103805in}}%
\pgfpathlineto{\pgfqpoint{0.623212in}{1.102615in}}%
\pgfpathclose%
\pgfusepath{fill}%
\end{pgfscope}%
\begin{pgfscope}%
\pgfpathrectangle{\pgfqpoint{0.211875in}{0.211875in}}{\pgfqpoint{1.313625in}{1.279725in}}%
\pgfusepath{clip}%
\pgfsetbuttcap%
\pgfsetroundjoin%
\definecolor{currentfill}{rgb}{0.796501,0.105066,0.310630}%
\pgfsetfillcolor{currentfill}%
\pgfsetlinewidth{0.000000pt}%
\definecolor{currentstroke}{rgb}{0.000000,0.000000,0.000000}%
\pgfsetstrokecolor{currentstroke}%
\pgfsetdash{}{0pt}%
\pgfpathmoveto{\pgfqpoint{0.729364in}{1.096706in}}%
\pgfpathlineto{\pgfqpoint{0.742633in}{1.096255in}}%
\pgfpathlineto{\pgfqpoint{0.754680in}{1.103805in}}%
\pgfpathlineto{\pgfqpoint{0.755902in}{1.105884in}}%
\pgfpathlineto{\pgfqpoint{0.758636in}{1.116731in}}%
\pgfpathlineto{\pgfqpoint{0.759070in}{1.129658in}}%
\pgfpathlineto{\pgfqpoint{0.757255in}{1.142584in}}%
\pgfpathlineto{\pgfqpoint{0.755902in}{1.145951in}}%
\pgfpathlineto{\pgfqpoint{0.743997in}{1.155511in}}%
\pgfpathlineto{\pgfqpoint{0.742633in}{1.156011in}}%
\pgfpathlineto{\pgfqpoint{0.729364in}{1.155601in}}%
\pgfpathlineto{\pgfqpoint{0.729154in}{1.155511in}}%
\pgfpathlineto{\pgfqpoint{0.717196in}{1.142584in}}%
\pgfpathlineto{\pgfqpoint{0.716095in}{1.137355in}}%
\pgfpathlineto{\pgfqpoint{0.715171in}{1.129658in}}%
\pgfpathlineto{\pgfqpoint{0.715577in}{1.116731in}}%
\pgfpathlineto{\pgfqpoint{0.716095in}{1.113989in}}%
\pgfpathlineto{\pgfqpoint{0.720022in}{1.103805in}}%
\pgfpathclose%
\pgfpathmoveto{\pgfqpoint{0.724183in}{1.116731in}}%
\pgfpathlineto{\pgfqpoint{0.723131in}{1.129658in}}%
\pgfpathlineto{\pgfqpoint{0.727554in}{1.142584in}}%
\pgfpathlineto{\pgfqpoint{0.729364in}{1.144524in}}%
\pgfpathlineto{\pgfqpoint{0.742633in}{1.145677in}}%
\pgfpathlineto{\pgfqpoint{0.746166in}{1.142584in}}%
\pgfpathlineto{\pgfqpoint{0.751398in}{1.129658in}}%
\pgfpathlineto{\pgfqpoint{0.750145in}{1.116731in}}%
\pgfpathlineto{\pgfqpoint{0.742633in}{1.106257in}}%
\pgfpathlineto{\pgfqpoint{0.729364in}{1.107832in}}%
\pgfpathclose%
\pgfusepath{fill}%
\end{pgfscope}%
\begin{pgfscope}%
\pgfpathrectangle{\pgfqpoint{0.211875in}{0.211875in}}{\pgfqpoint{1.313625in}{1.279725in}}%
\pgfusepath{clip}%
\pgfsetbuttcap%
\pgfsetroundjoin%
\definecolor{currentfill}{rgb}{0.796501,0.105066,0.310630}%
\pgfsetfillcolor{currentfill}%
\pgfsetlinewidth{0.000000pt}%
\definecolor{currentstroke}{rgb}{0.000000,0.000000,0.000000}%
\pgfsetstrokecolor{currentstroke}%
\pgfsetdash{}{0pt}%
\pgfpathmoveto{\pgfqpoint{0.848784in}{1.098219in}}%
\pgfpathlineto{\pgfqpoint{0.862053in}{1.098401in}}%
\pgfpathlineto{\pgfqpoint{0.869714in}{1.103805in}}%
\pgfpathlineto{\pgfqpoint{0.875322in}{1.115743in}}%
\pgfpathlineto{\pgfqpoint{0.875538in}{1.116731in}}%
\pgfpathlineto{\pgfqpoint{0.876027in}{1.129658in}}%
\pgfpathlineto{\pgfqpoint{0.875322in}{1.134741in}}%
\pgfpathlineto{\pgfqpoint{0.873293in}{1.142584in}}%
\pgfpathlineto{\pgfqpoint{0.862053in}{1.153678in}}%
\pgfpathlineto{\pgfqpoint{0.848784in}{1.153888in}}%
\pgfpathlineto{\pgfqpoint{0.836940in}{1.142584in}}%
\pgfpathlineto{\pgfqpoint{0.835515in}{1.137270in}}%
\pgfpathlineto{\pgfqpoint{0.834438in}{1.129658in}}%
\pgfpathlineto{\pgfqpoint{0.834917in}{1.116731in}}%
\pgfpathlineto{\pgfqpoint{0.835515in}{1.114046in}}%
\pgfpathlineto{\pgfqpoint{0.840538in}{1.103805in}}%
\pgfpathclose%
\pgfpathmoveto{\pgfqpoint{0.844188in}{1.116731in}}%
\pgfpathlineto{\pgfqpoint{0.842938in}{1.129658in}}%
\pgfpathlineto{\pgfqpoint{0.848190in}{1.142584in}}%
\pgfpathlineto{\pgfqpoint{0.848784in}{1.143151in}}%
\pgfpathlineto{\pgfqpoint{0.862053in}{1.142935in}}%
\pgfpathlineto{\pgfqpoint{0.862408in}{1.142584in}}%
\pgfpathlineto{\pgfqpoint{0.867562in}{1.129658in}}%
\pgfpathlineto{\pgfqpoint{0.866329in}{1.116731in}}%
\pgfpathlineto{\pgfqpoint{0.862053in}{1.110001in}}%
\pgfpathlineto{\pgfqpoint{0.848784in}{1.109706in}}%
\pgfpathclose%
\pgfusepath{fill}%
\end{pgfscope}%
\begin{pgfscope}%
\pgfpathrectangle{\pgfqpoint{0.211875in}{0.211875in}}{\pgfqpoint{1.313625in}{1.279725in}}%
\pgfusepath{clip}%
\pgfsetbuttcap%
\pgfsetroundjoin%
\definecolor{currentfill}{rgb}{0.796501,0.105066,0.310630}%
\pgfsetfillcolor{currentfill}%
\pgfsetlinewidth{0.000000pt}%
\definecolor{currentstroke}{rgb}{0.000000,0.000000,0.000000}%
\pgfsetstrokecolor{currentstroke}%
\pgfsetdash{}{0pt}%
\pgfpathmoveto{\pgfqpoint{0.968205in}{1.099116in}}%
\pgfpathlineto{\pgfqpoint{0.981473in}{1.100119in}}%
\pgfpathlineto{\pgfqpoint{0.986146in}{1.103805in}}%
\pgfpathlineto{\pgfqpoint{0.992072in}{1.116731in}}%
\pgfpathlineto{\pgfqpoint{0.992810in}{1.129658in}}%
\pgfpathlineto{\pgfqpoint{0.989724in}{1.142584in}}%
\pgfpathlineto{\pgfqpoint{0.981473in}{1.151706in}}%
\pgfpathlineto{\pgfqpoint{0.968205in}{1.152858in}}%
\pgfpathlineto{\pgfqpoint{0.956027in}{1.142584in}}%
\pgfpathlineto{\pgfqpoint{0.954936in}{1.139347in}}%
\pgfpathlineto{\pgfqpoint{0.953352in}{1.129658in}}%
\pgfpathlineto{\pgfqpoint{0.953886in}{1.116731in}}%
\pgfpathlineto{\pgfqpoint{0.954936in}{1.112652in}}%
\pgfpathlineto{\pgfqpoint{0.960383in}{1.103805in}}%
\pgfpathclose%
\pgfpathmoveto{\pgfqpoint{0.963763in}{1.116731in}}%
\pgfpathlineto{\pgfqpoint{0.962305in}{1.129658in}}%
\pgfpathlineto{\pgfqpoint{0.968205in}{1.142205in}}%
\pgfpathlineto{\pgfqpoint{0.981473in}{1.138540in}}%
\pgfpathlineto{\pgfqpoint{0.984672in}{1.129658in}}%
\pgfpathlineto{\pgfqpoint{0.983478in}{1.116731in}}%
\pgfpathlineto{\pgfqpoint{0.981473in}{1.113194in}}%
\pgfpathlineto{\pgfqpoint{0.968205in}{1.110734in}}%
\pgfpathclose%
\pgfusepath{fill}%
\end{pgfscope}%
\begin{pgfscope}%
\pgfpathrectangle{\pgfqpoint{0.211875in}{0.211875in}}{\pgfqpoint{1.313625in}{1.279725in}}%
\pgfusepath{clip}%
\pgfsetbuttcap%
\pgfsetroundjoin%
\definecolor{currentfill}{rgb}{0.796501,0.105066,0.310630}%
\pgfsetfillcolor{currentfill}%
\pgfsetlinewidth{0.000000pt}%
\definecolor{currentstroke}{rgb}{0.000000,0.000000,0.000000}%
\pgfsetstrokecolor{currentstroke}%
\pgfsetdash{}{0pt}%
\pgfpathmoveto{\pgfqpoint{1.087625in}{1.099461in}}%
\pgfpathlineto{\pgfqpoint{1.100894in}{1.101394in}}%
\pgfpathlineto{\pgfqpoint{1.103636in}{1.103805in}}%
\pgfpathlineto{\pgfqpoint{1.109372in}{1.116731in}}%
\pgfpathlineto{\pgfqpoint{1.110083in}{1.129658in}}%
\pgfpathlineto{\pgfqpoint{1.107102in}{1.142584in}}%
\pgfpathlineto{\pgfqpoint{1.100894in}{1.150242in}}%
\pgfpathlineto{\pgfqpoint{1.087625in}{1.152462in}}%
\pgfpathlineto{\pgfqpoint{1.074356in}{1.142723in}}%
\pgfpathlineto{\pgfqpoint{1.074291in}{1.142584in}}%
\pgfpathlineto{\pgfqpoint{1.071898in}{1.129658in}}%
\pgfpathlineto{\pgfqpoint{1.072469in}{1.116731in}}%
\pgfpathlineto{\pgfqpoint{1.074356in}{1.110290in}}%
\pgfpathlineto{\pgfqpoint{1.079351in}{1.103805in}}%
\pgfpathclose%
\pgfpathmoveto{\pgfqpoint{1.082772in}{1.116731in}}%
\pgfpathlineto{\pgfqpoint{1.081089in}{1.129658in}}%
\pgfpathlineto{\pgfqpoint{1.087625in}{1.141799in}}%
\pgfpathlineto{\pgfqpoint{1.100894in}{1.134605in}}%
\pgfpathlineto{\pgfqpoint{1.102489in}{1.129658in}}%
\pgfpathlineto{\pgfqpoint{1.101349in}{1.116731in}}%
\pgfpathlineto{\pgfqpoint{1.100894in}{1.115834in}}%
\pgfpathlineto{\pgfqpoint{1.087625in}{1.111007in}}%
\pgfpathclose%
\pgfusepath{fill}%
\end{pgfscope}%
\begin{pgfscope}%
\pgfpathrectangle{\pgfqpoint{0.211875in}{0.211875in}}{\pgfqpoint{1.313625in}{1.279725in}}%
\pgfusepath{clip}%
\pgfsetbuttcap%
\pgfsetroundjoin%
\definecolor{currentfill}{rgb}{0.796501,0.105066,0.310630}%
\pgfsetfillcolor{currentfill}%
\pgfsetlinewidth{0.000000pt}%
\definecolor{currentstroke}{rgb}{0.000000,0.000000,0.000000}%
\pgfsetstrokecolor{currentstroke}%
\pgfsetdash{}{0pt}%
\pgfpathmoveto{\pgfqpoint{1.207045in}{1.099295in}}%
\pgfpathlineto{\pgfqpoint{1.220314in}{1.102189in}}%
\pgfpathlineto{\pgfqpoint{1.221967in}{1.103805in}}%
\pgfpathlineto{\pgfqpoint{1.227354in}{1.116731in}}%
\pgfpathlineto{\pgfqpoint{1.228021in}{1.129658in}}%
\pgfpathlineto{\pgfqpoint{1.225224in}{1.142584in}}%
\pgfpathlineto{\pgfqpoint{1.220314in}{1.149329in}}%
\pgfpathlineto{\pgfqpoint{1.207045in}{1.152653in}}%
\pgfpathlineto{\pgfqpoint{1.193777in}{1.144972in}}%
\pgfpathlineto{\pgfqpoint{1.192516in}{1.142584in}}%
\pgfpathlineto{\pgfqpoint{1.190045in}{1.129658in}}%
\pgfpathlineto{\pgfqpoint{1.190635in}{1.116731in}}%
\pgfpathlineto{\pgfqpoint{1.193777in}{1.107220in}}%
\pgfpathlineto{\pgfqpoint{1.197085in}{1.103805in}}%
\pgfpathclose%
\pgfpathmoveto{\pgfqpoint{1.200975in}{1.116731in}}%
\pgfpathlineto{\pgfqpoint{1.199037in}{1.129658in}}%
\pgfpathlineto{\pgfqpoint{1.207045in}{1.142444in}}%
\pgfpathlineto{\pgfqpoint{1.220314in}{1.131538in}}%
\pgfpathlineto{\pgfqpoint{1.220858in}{1.129658in}}%
\pgfpathlineto{\pgfqpoint{1.220314in}{1.122925in}}%
\pgfpathlineto{\pgfqpoint{1.218514in}{1.116731in}}%
\pgfpathlineto{\pgfqpoint{1.207045in}{1.110574in}}%
\pgfpathclose%
\pgfusepath{fill}%
\end{pgfscope}%
\begin{pgfscope}%
\pgfpathrectangle{\pgfqpoint{0.211875in}{0.211875in}}{\pgfqpoint{1.313625in}{1.279725in}}%
\pgfusepath{clip}%
\pgfsetbuttcap%
\pgfsetroundjoin%
\definecolor{currentfill}{rgb}{0.796501,0.105066,0.310630}%
\pgfsetfillcolor{currentfill}%
\pgfsetlinewidth{0.000000pt}%
\definecolor{currentstroke}{rgb}{0.000000,0.000000,0.000000}%
\pgfsetstrokecolor{currentstroke}%
\pgfsetdash{}{0pt}%
\pgfpathmoveto{\pgfqpoint{1.313197in}{1.103663in}}%
\pgfpathlineto{\pgfqpoint{1.326466in}{1.098633in}}%
\pgfpathlineto{\pgfqpoint{1.339735in}{1.102437in}}%
\pgfpathlineto{\pgfqpoint{1.340991in}{1.103805in}}%
\pgfpathlineto{\pgfqpoint{1.345891in}{1.116731in}}%
\pgfpathlineto{\pgfqpoint{1.346497in}{1.129658in}}%
\pgfpathlineto{\pgfqpoint{1.343957in}{1.142584in}}%
\pgfpathlineto{\pgfqpoint{1.339735in}{1.149044in}}%
\pgfpathlineto{\pgfqpoint{1.326466in}{1.153413in}}%
\pgfpathlineto{\pgfqpoint{1.313197in}{1.147637in}}%
\pgfpathlineto{\pgfqpoint{1.310212in}{1.142584in}}%
\pgfpathlineto{\pgfqpoint{1.307742in}{1.129658in}}%
\pgfpathlineto{\pgfqpoint{1.308332in}{1.116731in}}%
\pgfpathlineto{\pgfqpoint{1.313080in}{1.103805in}}%
\pgfpathclose%
\pgfpathmoveto{\pgfqpoint{1.317928in}{1.116731in}}%
\pgfpathlineto{\pgfqpoint{1.315685in}{1.129658in}}%
\pgfpathlineto{\pgfqpoint{1.325064in}{1.142584in}}%
\pgfpathlineto{\pgfqpoint{1.326466in}{1.143333in}}%
\pgfpathlineto{\pgfqpoint{1.328183in}{1.142584in}}%
\pgfpathlineto{\pgfqpoint{1.339580in}{1.129658in}}%
\pgfpathlineto{\pgfqpoint{1.336868in}{1.116731in}}%
\pgfpathlineto{\pgfqpoint{1.326466in}{1.109457in}}%
\pgfpathclose%
\pgfusepath{fill}%
\end{pgfscope}%
\begin{pgfscope}%
\pgfpathrectangle{\pgfqpoint{0.211875in}{0.211875in}}{\pgfqpoint{1.313625in}{1.279725in}}%
\pgfusepath{clip}%
\pgfsetbuttcap%
\pgfsetroundjoin%
\definecolor{currentfill}{rgb}{0.796501,0.105066,0.310630}%
\pgfsetfillcolor{currentfill}%
\pgfsetlinewidth{0.000000pt}%
\definecolor{currentstroke}{rgb}{0.000000,0.000000,0.000000}%
\pgfsetstrokecolor{currentstroke}%
\pgfsetdash{}{0pt}%
\pgfpathmoveto{\pgfqpoint{1.432617in}{1.101026in}}%
\pgfpathlineto{\pgfqpoint{1.445886in}{1.097472in}}%
\pgfpathlineto{\pgfqpoint{1.459155in}{1.102029in}}%
\pgfpathlineto{\pgfqpoint{1.460619in}{1.103805in}}%
\pgfpathlineto{\pgfqpoint{1.464901in}{1.116731in}}%
\pgfpathlineto{\pgfqpoint{1.465429in}{1.129658in}}%
\pgfpathlineto{\pgfqpoint{1.463212in}{1.142584in}}%
\pgfpathlineto{\pgfqpoint{1.459155in}{1.149514in}}%
\pgfpathlineto{\pgfqpoint{1.445886in}{1.154746in}}%
\pgfpathlineto{\pgfqpoint{1.432617in}{1.150665in}}%
\pgfpathlineto{\pgfqpoint{1.427297in}{1.142584in}}%
\pgfpathlineto{\pgfqpoint{1.424915in}{1.129658in}}%
\pgfpathlineto{\pgfqpoint{1.425485in}{1.116731in}}%
\pgfpathlineto{\pgfqpoint{1.430055in}{1.103805in}}%
\pgfpathclose%
\pgfpathmoveto{\pgfqpoint{1.432730in}{1.116731in}}%
\pgfpathlineto{\pgfqpoint{1.432617in}{1.117257in}}%
\pgfpathlineto{\pgfqpoint{1.431709in}{1.129658in}}%
\pgfpathlineto{\pgfqpoint{1.432617in}{1.133122in}}%
\pgfpathlineto{\pgfqpoint{1.441107in}{1.142584in}}%
\pgfpathlineto{\pgfqpoint{1.445886in}{1.144659in}}%
\pgfpathlineto{\pgfqpoint{1.449753in}{1.142584in}}%
\pgfpathlineto{\pgfqpoint{1.458525in}{1.129658in}}%
\pgfpathlineto{\pgfqpoint{1.456440in}{1.116731in}}%
\pgfpathlineto{\pgfqpoint{1.445886in}{1.107648in}}%
\pgfpathclose%
\pgfusepath{fill}%
\end{pgfscope}%
\begin{pgfscope}%
\pgfpathrectangle{\pgfqpoint{0.211875in}{0.211875in}}{\pgfqpoint{1.313625in}{1.279725in}}%
\pgfusepath{clip}%
\pgfsetbuttcap%
\pgfsetroundjoin%
\definecolor{currentfill}{rgb}{0.796501,0.105066,0.310630}%
\pgfsetfillcolor{currentfill}%
\pgfsetlinewidth{0.000000pt}%
\definecolor{currentstroke}{rgb}{0.000000,0.000000,0.000000}%
\pgfsetstrokecolor{currentstroke}%
\pgfsetdash{}{0pt}%
\pgfpathmoveto{\pgfqpoint{0.424178in}{1.172668in}}%
\pgfpathlineto{\pgfqpoint{0.437447in}{1.170409in}}%
\pgfpathlineto{\pgfqpoint{0.450716in}{1.171029in}}%
\pgfpathlineto{\pgfqpoint{0.463985in}{1.177105in}}%
\pgfpathlineto{\pgfqpoint{0.466245in}{1.181364in}}%
\pgfpathlineto{\pgfqpoint{0.468481in}{1.194290in}}%
\pgfpathlineto{\pgfqpoint{0.468945in}{1.207217in}}%
\pgfpathlineto{\pgfqpoint{0.468545in}{1.220143in}}%
\pgfpathlineto{\pgfqpoint{0.466442in}{1.233070in}}%
\pgfpathlineto{\pgfqpoint{0.463985in}{1.237755in}}%
\pgfpathlineto{\pgfqpoint{0.450716in}{1.243613in}}%
\pgfpathlineto{\pgfqpoint{0.437447in}{1.244223in}}%
\pgfpathlineto{\pgfqpoint{0.424178in}{1.242131in}}%
\pgfpathlineto{\pgfqpoint{0.417362in}{1.233070in}}%
\pgfpathlineto{\pgfqpoint{0.415823in}{1.220143in}}%
\pgfpathlineto{\pgfqpoint{0.415546in}{1.207217in}}%
\pgfpathlineto{\pgfqpoint{0.415892in}{1.194290in}}%
\pgfpathlineto{\pgfqpoint{0.417586in}{1.181364in}}%
\pgfpathclose%
\pgfpathmoveto{\pgfqpoint{0.432165in}{1.181364in}}%
\pgfpathlineto{\pgfqpoint{0.424178in}{1.190057in}}%
\pgfpathlineto{\pgfqpoint{0.422776in}{1.194290in}}%
\pgfpathlineto{\pgfqpoint{0.421573in}{1.207217in}}%
\pgfpathlineto{\pgfqpoint{0.422715in}{1.220143in}}%
\pgfpathlineto{\pgfqpoint{0.424178in}{1.224618in}}%
\pgfpathlineto{\pgfqpoint{0.431859in}{1.233070in}}%
\pgfpathlineto{\pgfqpoint{0.437447in}{1.235588in}}%
\pgfpathlineto{\pgfqpoint{0.450716in}{1.234498in}}%
\pgfpathlineto{\pgfqpoint{0.452979in}{1.233070in}}%
\pgfpathlineto{\pgfqpoint{0.461343in}{1.220143in}}%
\pgfpathlineto{\pgfqpoint{0.462996in}{1.207217in}}%
\pgfpathlineto{\pgfqpoint{0.461265in}{1.194290in}}%
\pgfpathlineto{\pgfqpoint{0.452751in}{1.181364in}}%
\pgfpathlineto{\pgfqpoint{0.450716in}{1.180083in}}%
\pgfpathlineto{\pgfqpoint{0.437447in}{1.178990in}}%
\pgfpathclose%
\pgfusepath{fill}%
\end{pgfscope}%
\begin{pgfscope}%
\pgfpathrectangle{\pgfqpoint{0.211875in}{0.211875in}}{\pgfqpoint{1.313625in}{1.279725in}}%
\pgfusepath{clip}%
\pgfsetbuttcap%
\pgfsetroundjoin%
\definecolor{currentfill}{rgb}{0.796501,0.105066,0.310630}%
\pgfsetfillcolor{currentfill}%
\pgfsetlinewidth{0.000000pt}%
\definecolor{currentstroke}{rgb}{0.000000,0.000000,0.000000}%
\pgfsetstrokecolor{currentstroke}%
\pgfsetdash{}{0pt}%
\pgfpathmoveto{\pgfqpoint{0.543598in}{1.177047in}}%
\pgfpathlineto{\pgfqpoint{0.556867in}{1.173437in}}%
\pgfpathlineto{\pgfqpoint{0.570136in}{1.174626in}}%
\pgfpathlineto{\pgfqpoint{0.579781in}{1.181364in}}%
\pgfpathlineto{\pgfqpoint{0.583405in}{1.190124in}}%
\pgfpathlineto{\pgfqpoint{0.584227in}{1.194290in}}%
\pgfpathlineto{\pgfqpoint{0.584959in}{1.207217in}}%
\pgfpathlineto{\pgfqpoint{0.584288in}{1.220143in}}%
\pgfpathlineto{\pgfqpoint{0.583405in}{1.224721in}}%
\pgfpathlineto{\pgfqpoint{0.580023in}{1.233070in}}%
\pgfpathlineto{\pgfqpoint{0.570136in}{1.240001in}}%
\pgfpathlineto{\pgfqpoint{0.556867in}{1.241171in}}%
\pgfpathlineto{\pgfqpoint{0.543598in}{1.237662in}}%
\pgfpathlineto{\pgfqpoint{0.539666in}{1.233070in}}%
\pgfpathlineto{\pgfqpoint{0.536389in}{1.220143in}}%
\pgfpathlineto{\pgfqpoint{0.535775in}{1.207217in}}%
\pgfpathlineto{\pgfqpoint{0.536456in}{1.194290in}}%
\pgfpathlineto{\pgfqpoint{0.539878in}{1.181364in}}%
\pgfpathclose%
\pgfpathmoveto{\pgfqpoint{0.543470in}{1.194290in}}%
\pgfpathlineto{\pgfqpoint{0.541919in}{1.207217in}}%
\pgfpathlineto{\pgfqpoint{0.543411in}{1.220143in}}%
\pgfpathlineto{\pgfqpoint{0.543598in}{1.220645in}}%
\pgfpathlineto{\pgfqpoint{0.556867in}{1.232563in}}%
\pgfpathlineto{\pgfqpoint{0.570136in}{1.229500in}}%
\pgfpathlineto{\pgfqpoint{0.576114in}{1.220143in}}%
\pgfpathlineto{\pgfqpoint{0.578014in}{1.207217in}}%
\pgfpathlineto{\pgfqpoint{0.576045in}{1.194290in}}%
\pgfpathlineto{\pgfqpoint{0.570136in}{1.185107in}}%
\pgfpathlineto{\pgfqpoint{0.556867in}{1.182041in}}%
\pgfpathlineto{\pgfqpoint{0.543598in}{1.193949in}}%
\pgfpathclose%
\pgfusepath{fill}%
\end{pgfscope}%
\begin{pgfscope}%
\pgfpathrectangle{\pgfqpoint{0.211875in}{0.211875in}}{\pgfqpoint{1.313625in}{1.279725in}}%
\pgfusepath{clip}%
\pgfsetbuttcap%
\pgfsetroundjoin%
\definecolor{currentfill}{rgb}{0.796501,0.105066,0.310630}%
\pgfsetfillcolor{currentfill}%
\pgfsetlinewidth{0.000000pt}%
\definecolor{currentstroke}{rgb}{0.000000,0.000000,0.000000}%
\pgfsetstrokecolor{currentstroke}%
\pgfsetdash{}{0pt}%
\pgfpathmoveto{\pgfqpoint{0.663019in}{1.179850in}}%
\pgfpathlineto{\pgfqpoint{0.676288in}{1.175852in}}%
\pgfpathlineto{\pgfqpoint{0.689557in}{1.177948in}}%
\pgfpathlineto{\pgfqpoint{0.693967in}{1.181364in}}%
\pgfpathlineto{\pgfqpoint{0.699779in}{1.194290in}}%
\pgfpathlineto{\pgfqpoint{0.700956in}{1.207217in}}%
\pgfpathlineto{\pgfqpoint{0.699851in}{1.220143in}}%
\pgfpathlineto{\pgfqpoint{0.694184in}{1.233070in}}%
\pgfpathlineto{\pgfqpoint{0.689557in}{1.236668in}}%
\pgfpathlineto{\pgfqpoint{0.676288in}{1.238739in}}%
\pgfpathlineto{\pgfqpoint{0.663019in}{1.234803in}}%
\pgfpathlineto{\pgfqpoint{0.661347in}{1.233070in}}%
\pgfpathlineto{\pgfqpoint{0.656555in}{1.220143in}}%
\pgfpathlineto{\pgfqpoint{0.655647in}{1.207217in}}%
\pgfpathlineto{\pgfqpoint{0.656621in}{1.194290in}}%
\pgfpathlineto{\pgfqpoint{0.661551in}{1.181364in}}%
\pgfpathclose%
\pgfpathmoveto{\pgfqpoint{0.664671in}{1.194290in}}%
\pgfpathlineto{\pgfqpoint{0.663019in}{1.199427in}}%
\pgfpathlineto{\pgfqpoint{0.661955in}{1.207217in}}%
\pgfpathlineto{\pgfqpoint{0.663019in}{1.215238in}}%
\pgfpathlineto{\pgfqpoint{0.664552in}{1.220143in}}%
\pgfpathlineto{\pgfqpoint{0.676288in}{1.229329in}}%
\pgfpathlineto{\pgfqpoint{0.689557in}{1.224189in}}%
\pgfpathlineto{\pgfqpoint{0.691879in}{1.220143in}}%
\pgfpathlineto{\pgfqpoint{0.693948in}{1.207217in}}%
\pgfpathlineto{\pgfqpoint{0.691816in}{1.194290in}}%
\pgfpathlineto{\pgfqpoint{0.689557in}{1.190384in}}%
\pgfpathlineto{\pgfqpoint{0.676288in}{1.185249in}}%
\pgfpathclose%
\pgfusepath{fill}%
\end{pgfscope}%
\begin{pgfscope}%
\pgfpathrectangle{\pgfqpoint{0.211875in}{0.211875in}}{\pgfqpoint{1.313625in}{1.279725in}}%
\pgfusepath{clip}%
\pgfsetbuttcap%
\pgfsetroundjoin%
\definecolor{currentfill}{rgb}{0.796501,0.105066,0.310630}%
\pgfsetfillcolor{currentfill}%
\pgfsetlinewidth{0.000000pt}%
\definecolor{currentstroke}{rgb}{0.000000,0.000000,0.000000}%
\pgfsetstrokecolor{currentstroke}%
\pgfsetdash{}{0pt}%
\pgfpathmoveto{\pgfqpoint{0.795708in}{1.177723in}}%
\pgfpathlineto{\pgfqpoint{0.808977in}{1.181015in}}%
\pgfpathlineto{\pgfqpoint{0.809383in}{1.181364in}}%
\pgfpathlineto{\pgfqpoint{0.815947in}{1.194290in}}%
\pgfpathlineto{\pgfqpoint{0.817266in}{1.207217in}}%
\pgfpathlineto{\pgfqpoint{0.816014in}{1.220143in}}%
\pgfpathlineto{\pgfqpoint{0.809581in}{1.233070in}}%
\pgfpathlineto{\pgfqpoint{0.808977in}{1.233591in}}%
\pgfpathlineto{\pgfqpoint{0.795708in}{1.236854in}}%
\pgfpathlineto{\pgfqpoint{0.782439in}{1.233092in}}%
\pgfpathlineto{\pgfqpoint{0.782416in}{1.233070in}}%
\pgfpathlineto{\pgfqpoint{0.776316in}{1.220143in}}%
\pgfpathlineto{\pgfqpoint{0.775151in}{1.207217in}}%
\pgfpathlineto{\pgfqpoint{0.776382in}{1.194290in}}%
\pgfpathlineto{\pgfqpoint{0.782439in}{1.181589in}}%
\pgfpathlineto{\pgfqpoint{0.782860in}{1.181364in}}%
\pgfpathclose%
\pgfpathmoveto{\pgfqpoint{0.785719in}{1.194290in}}%
\pgfpathlineto{\pgfqpoint{0.782439in}{1.202230in}}%
\pgfpathlineto{\pgfqpoint{0.781676in}{1.207217in}}%
\pgfpathlineto{\pgfqpoint{0.782439in}{1.212336in}}%
\pgfpathlineto{\pgfqpoint{0.785584in}{1.220143in}}%
\pgfpathlineto{\pgfqpoint{0.795708in}{1.226780in}}%
\pgfpathlineto{\pgfqpoint{0.807473in}{1.220143in}}%
\pgfpathlineto{\pgfqpoint{0.808977in}{1.217141in}}%
\pgfpathlineto{\pgfqpoint{0.810596in}{1.207217in}}%
\pgfpathlineto{\pgfqpoint{0.808977in}{1.197529in}}%
\pgfpathlineto{\pgfqpoint{0.807317in}{1.194290in}}%
\pgfpathlineto{\pgfqpoint{0.795708in}{1.187779in}}%
\pgfpathclose%
\pgfusepath{fill}%
\end{pgfscope}%
\begin{pgfscope}%
\pgfpathrectangle{\pgfqpoint{0.211875in}{0.211875in}}{\pgfqpoint{1.313625in}{1.279725in}}%
\pgfusepath{clip}%
\pgfsetbuttcap%
\pgfsetroundjoin%
\definecolor{currentfill}{rgb}{0.796501,0.105066,0.310630}%
\pgfsetfillcolor{currentfill}%
\pgfsetlinewidth{0.000000pt}%
\definecolor{currentstroke}{rgb}{0.000000,0.000000,0.000000}%
\pgfsetstrokecolor{currentstroke}%
\pgfsetdash{}{0pt}%
\pgfpathmoveto{\pgfqpoint{0.915129in}{1.179099in}}%
\pgfpathlineto{\pgfqpoint{0.922490in}{1.181364in}}%
\pgfpathlineto{\pgfqpoint{0.928398in}{1.184807in}}%
\pgfpathlineto{\pgfqpoint{0.932817in}{1.194290in}}%
\pgfpathlineto{\pgfqpoint{0.934220in}{1.207217in}}%
\pgfpathlineto{\pgfqpoint{0.932880in}{1.220143in}}%
\pgfpathlineto{\pgfqpoint{0.928398in}{1.229850in}}%
\pgfpathlineto{\pgfqpoint{0.922912in}{1.233070in}}%
\pgfpathlineto{\pgfqpoint{0.915129in}{1.235471in}}%
\pgfpathlineto{\pgfqpoint{0.904687in}{1.233070in}}%
\pgfpathlineto{\pgfqpoint{0.901860in}{1.231918in}}%
\pgfpathlineto{\pgfqpoint{0.895645in}{1.220143in}}%
\pgfpathlineto{\pgfqpoint{0.894261in}{1.207217in}}%
\pgfpathlineto{\pgfqpoint{0.895712in}{1.194290in}}%
\pgfpathlineto{\pgfqpoint{0.901860in}{1.182739in}}%
\pgfpathlineto{\pgfqpoint{0.905255in}{1.181364in}}%
\pgfpathclose%
\pgfpathmoveto{\pgfqpoint{0.906389in}{1.194290in}}%
\pgfpathlineto{\pgfqpoint{0.901860in}{1.202561in}}%
\pgfpathlineto{\pgfqpoint{0.901065in}{1.207217in}}%
\pgfpathlineto{\pgfqpoint{0.901860in}{1.211986in}}%
\pgfpathlineto{\pgfqpoint{0.906227in}{1.220143in}}%
\pgfpathlineto{\pgfqpoint{0.915129in}{1.224852in}}%
\pgfpathlineto{\pgfqpoint{0.921946in}{1.220143in}}%
\pgfpathlineto{\pgfqpoint{0.927063in}{1.207217in}}%
\pgfpathlineto{\pgfqpoint{0.921822in}{1.194290in}}%
\pgfpathlineto{\pgfqpoint{0.915129in}{1.189694in}}%
\pgfpathclose%
\pgfusepath{fill}%
\end{pgfscope}%
\begin{pgfscope}%
\pgfpathrectangle{\pgfqpoint{0.211875in}{0.211875in}}{\pgfqpoint{1.313625in}{1.279725in}}%
\pgfusepath{clip}%
\pgfsetbuttcap%
\pgfsetroundjoin%
\definecolor{currentfill}{rgb}{0.796501,0.105066,0.310630}%
\pgfsetfillcolor{currentfill}%
\pgfsetlinewidth{0.000000pt}%
\definecolor{currentstroke}{rgb}{0.000000,0.000000,0.000000}%
\pgfsetstrokecolor{currentstroke}%
\pgfsetdash{}{0pt}%
\pgfpathmoveto{\pgfqpoint{1.034549in}{1.180005in}}%
\pgfpathlineto{\pgfqpoint{1.038302in}{1.181364in}}%
\pgfpathlineto{\pgfqpoint{1.047818in}{1.188392in}}%
\pgfpathlineto{\pgfqpoint{1.050273in}{1.194290in}}%
\pgfpathlineto{\pgfqpoint{1.051705in}{1.207217in}}%
\pgfpathlineto{\pgfqpoint{1.050333in}{1.220143in}}%
\pgfpathlineto{\pgfqpoint{1.047818in}{1.226248in}}%
\pgfpathlineto{\pgfqpoint{1.038655in}{1.233070in}}%
\pgfpathlineto{\pgfqpoint{1.034549in}{1.234560in}}%
\pgfpathlineto{\pgfqpoint{1.026053in}{1.233070in}}%
\pgfpathlineto{\pgfqpoint{1.021280in}{1.231679in}}%
\pgfpathlineto{\pgfqpoint{1.014492in}{1.220143in}}%
\pgfpathlineto{\pgfqpoint{1.012927in}{1.207217in}}%
\pgfpathlineto{\pgfqpoint{1.014562in}{1.194290in}}%
\pgfpathlineto{\pgfqpoint{1.021280in}{1.182959in}}%
\pgfpathlineto{\pgfqpoint{1.026786in}{1.181364in}}%
\pgfpathclose%
\pgfpathmoveto{\pgfqpoint{1.026382in}{1.194290in}}%
\pgfpathlineto{\pgfqpoint{1.021280in}{1.200922in}}%
\pgfpathlineto{\pgfqpoint{1.020084in}{1.207217in}}%
\pgfpathlineto{\pgfqpoint{1.021280in}{1.213653in}}%
\pgfpathlineto{\pgfqpoint{1.026169in}{1.220143in}}%
\pgfpathlineto{\pgfqpoint{1.034549in}{1.223507in}}%
\pgfpathlineto{\pgfqpoint{1.038661in}{1.220143in}}%
\pgfpathlineto{\pgfqpoint{1.043167in}{1.207217in}}%
\pgfpathlineto{\pgfqpoint{1.038557in}{1.194290in}}%
\pgfpathlineto{\pgfqpoint{1.034549in}{1.191030in}}%
\pgfpathclose%
\pgfusepath{fill}%
\end{pgfscope}%
\begin{pgfscope}%
\pgfpathrectangle{\pgfqpoint{0.211875in}{0.211875in}}{\pgfqpoint{1.313625in}{1.279725in}}%
\pgfusepath{clip}%
\pgfsetbuttcap%
\pgfsetroundjoin%
\definecolor{currentfill}{rgb}{0.796501,0.105066,0.310630}%
\pgfsetfillcolor{currentfill}%
\pgfsetlinewidth{0.000000pt}%
\definecolor{currentstroke}{rgb}{0.000000,0.000000,0.000000}%
\pgfsetstrokecolor{currentstroke}%
\pgfsetdash{}{0pt}%
\pgfpathmoveto{\pgfqpoint{1.153970in}{1.180450in}}%
\pgfpathlineto{\pgfqpoint{1.156158in}{1.181364in}}%
\pgfpathlineto{\pgfqpoint{1.167239in}{1.191603in}}%
\pgfpathlineto{\pgfqpoint{1.168230in}{1.194290in}}%
\pgfpathlineto{\pgfqpoint{1.169646in}{1.207217in}}%
\pgfpathlineto{\pgfqpoint{1.168289in}{1.220143in}}%
\pgfpathlineto{\pgfqpoint{1.167239in}{1.223022in}}%
\pgfpathlineto{\pgfqpoint{1.156467in}{1.233070in}}%
\pgfpathlineto{\pgfqpoint{1.153970in}{1.234115in}}%
\pgfpathlineto{\pgfqpoint{1.145244in}{1.233070in}}%
\pgfpathlineto{\pgfqpoint{1.140701in}{1.232220in}}%
\pgfpathlineto{\pgfqpoint{1.132779in}{1.220143in}}%
\pgfpathlineto{\pgfqpoint{1.131071in}{1.207217in}}%
\pgfpathlineto{\pgfqpoint{1.132852in}{1.194290in}}%
\pgfpathlineto{\pgfqpoint{1.140701in}{1.182409in}}%
\pgfpathlineto{\pgfqpoint{1.146322in}{1.181364in}}%
\pgfpathclose%
\pgfpathmoveto{\pgfqpoint{1.144745in}{1.194290in}}%
\pgfpathlineto{\pgfqpoint{1.140701in}{1.197633in}}%
\pgfpathlineto{\pgfqpoint{1.138670in}{1.207217in}}%
\pgfpathlineto{\pgfqpoint{1.140701in}{1.217001in}}%
\pgfpathlineto{\pgfqpoint{1.144429in}{1.220143in}}%
\pgfpathlineto{\pgfqpoint{1.153970in}{1.222733in}}%
\pgfpathlineto{\pgfqpoint{1.156700in}{1.220143in}}%
\pgfpathlineto{\pgfqpoint{1.160681in}{1.207217in}}%
\pgfpathlineto{\pgfqpoint{1.156610in}{1.194290in}}%
\pgfpathlineto{\pgfqpoint{1.153970in}{1.191801in}}%
\pgfpathclose%
\pgfusepath{fill}%
\end{pgfscope}%
\begin{pgfscope}%
\pgfpathrectangle{\pgfqpoint{0.211875in}{0.211875in}}{\pgfqpoint{1.313625in}{1.279725in}}%
\pgfusepath{clip}%
\pgfsetbuttcap%
\pgfsetroundjoin%
\definecolor{currentfill}{rgb}{0.796501,0.105066,0.310630}%
\pgfsetfillcolor{currentfill}%
\pgfsetlinewidth{0.000000pt}%
\definecolor{currentstroke}{rgb}{0.000000,0.000000,0.000000}%
\pgfsetstrokecolor{currentstroke}%
\pgfsetdash{}{0pt}%
\pgfpathmoveto{\pgfqpoint{1.260121in}{1.181240in}}%
\pgfpathlineto{\pgfqpoint{1.273390in}{1.180422in}}%
\pgfpathlineto{\pgfqpoint{1.275373in}{1.181364in}}%
\pgfpathlineto{\pgfqpoint{1.286617in}{1.194290in}}%
\pgfpathlineto{\pgfqpoint{1.286659in}{1.194504in}}%
\pgfpathlineto{\pgfqpoint{1.287987in}{1.207217in}}%
\pgfpathlineto{\pgfqpoint{1.286691in}{1.220143in}}%
\pgfpathlineto{\pgfqpoint{1.286659in}{1.220244in}}%
\pgfpathlineto{\pgfqpoint{1.275652in}{1.233070in}}%
\pgfpathlineto{\pgfqpoint{1.273390in}{1.234147in}}%
\pgfpathlineto{\pgfqpoint{1.260121in}{1.233333in}}%
\pgfpathlineto{\pgfqpoint{1.259685in}{1.233070in}}%
\pgfpathlineto{\pgfqpoint{1.250380in}{1.220143in}}%
\pgfpathlineto{\pgfqpoint{1.248573in}{1.207217in}}%
\pgfpathlineto{\pgfqpoint{1.250459in}{1.194290in}}%
\pgfpathlineto{\pgfqpoint{1.259917in}{1.181364in}}%
\pgfpathclose%
\pgfpathmoveto{\pgfqpoint{1.259743in}{1.194290in}}%
\pgfpathlineto{\pgfqpoint{1.256730in}{1.207217in}}%
\pgfpathlineto{\pgfqpoint{1.259675in}{1.220143in}}%
\pgfpathlineto{\pgfqpoint{1.260121in}{1.220752in}}%
\pgfpathlineto{\pgfqpoint{1.273390in}{1.222538in}}%
\pgfpathlineto{\pgfqpoint{1.275602in}{1.220143in}}%
\pgfpathlineto{\pgfqpoint{1.279105in}{1.207217in}}%
\pgfpathlineto{\pgfqpoint{1.275520in}{1.194290in}}%
\pgfpathlineto{\pgfqpoint{1.273390in}{1.191997in}}%
\pgfpathlineto{\pgfqpoint{1.260121in}{1.193777in}}%
\pgfpathclose%
\pgfusepath{fill}%
\end{pgfscope}%
\begin{pgfscope}%
\pgfpathrectangle{\pgfqpoint{0.211875in}{0.211875in}}{\pgfqpoint{1.313625in}{1.279725in}}%
\pgfusepath{clip}%
\pgfsetbuttcap%
\pgfsetroundjoin%
\definecolor{currentfill}{rgb}{0.796501,0.105066,0.310630}%
\pgfsetfillcolor{currentfill}%
\pgfsetlinewidth{0.000000pt}%
\definecolor{currentstroke}{rgb}{0.000000,0.000000,0.000000}%
\pgfsetstrokecolor{currentstroke}%
\pgfsetdash{}{0pt}%
\pgfpathmoveto{\pgfqpoint{1.379542in}{1.179928in}}%
\pgfpathlineto{\pgfqpoint{1.392811in}{1.179891in}}%
\pgfpathlineto{\pgfqpoint{1.395564in}{1.181364in}}%
\pgfpathlineto{\pgfqpoint{1.405124in}{1.194290in}}%
\pgfpathlineto{\pgfqpoint{1.406080in}{1.200414in}}%
\pgfpathlineto{\pgfqpoint{1.406695in}{1.207217in}}%
\pgfpathlineto{\pgfqpoint{1.406080in}{1.214332in}}%
\pgfpathlineto{\pgfqpoint{1.405212in}{1.220143in}}%
\pgfpathlineto{\pgfqpoint{1.395825in}{1.233070in}}%
\pgfpathlineto{\pgfqpoint{1.392811in}{1.234687in}}%
\pgfpathlineto{\pgfqpoint{1.379542in}{1.234648in}}%
\pgfpathlineto{\pgfqpoint{1.376623in}{1.233070in}}%
\pgfpathlineto{\pgfqpoint{1.367105in}{1.220143in}}%
\pgfpathlineto{\pgfqpoint{1.366273in}{1.214754in}}%
\pgfpathlineto{\pgfqpoint{1.365604in}{1.207217in}}%
\pgfpathlineto{\pgfqpoint{1.366273in}{1.199998in}}%
\pgfpathlineto{\pgfqpoint{1.367192in}{1.194290in}}%
\pgfpathlineto{\pgfqpoint{1.376878in}{1.181364in}}%
\pgfpathclose%
\pgfpathmoveto{\pgfqpoint{1.377274in}{1.194290in}}%
\pgfpathlineto{\pgfqpoint{1.374115in}{1.207217in}}%
\pgfpathlineto{\pgfqpoint{1.377199in}{1.220143in}}%
\pgfpathlineto{\pgfqpoint{1.379542in}{1.222992in}}%
\pgfpathlineto{\pgfqpoint{1.392811in}{1.222960in}}%
\pgfpathlineto{\pgfqpoint{1.395112in}{1.220143in}}%
\pgfpathlineto{\pgfqpoint{1.398165in}{1.207217in}}%
\pgfpathlineto{\pgfqpoint{1.395037in}{1.194290in}}%
\pgfpathlineto{\pgfqpoint{1.392811in}{1.191583in}}%
\pgfpathlineto{\pgfqpoint{1.379542in}{1.191550in}}%
\pgfpathclose%
\pgfusepath{fill}%
\end{pgfscope}%
\begin{pgfscope}%
\pgfpathrectangle{\pgfqpoint{0.211875in}{0.211875in}}{\pgfqpoint{1.313625in}{1.279725in}}%
\pgfusepath{clip}%
\pgfsetbuttcap%
\pgfsetroundjoin%
\definecolor{currentfill}{rgb}{0.796501,0.105066,0.310630}%
\pgfsetfillcolor{currentfill}%
\pgfsetlinewidth{0.000000pt}%
\definecolor{currentstroke}{rgb}{0.000000,0.000000,0.000000}%
\pgfsetstrokecolor{currentstroke}%
\pgfsetdash{}{0pt}%
\pgfpathmoveto{\pgfqpoint{1.498962in}{1.178198in}}%
\pgfpathlineto{\pgfqpoint{1.512231in}{1.178800in}}%
\pgfpathlineto{\pgfqpoint{1.516514in}{1.181364in}}%
\pgfpathlineto{\pgfqpoint{1.524317in}{1.194290in}}%
\pgfpathlineto{\pgfqpoint{1.525500in}{1.203928in}}%
\pgfpathlineto{\pgfqpoint{1.525500in}{1.207217in}}%
\pgfpathlineto{\pgfqpoint{1.525500in}{1.210689in}}%
\pgfpathlineto{\pgfqpoint{1.524399in}{1.220143in}}%
\pgfpathlineto{\pgfqpoint{1.516764in}{1.233070in}}%
\pgfpathlineto{\pgfqpoint{1.512231in}{1.235792in}}%
\pgfpathlineto{\pgfqpoint{1.498962in}{1.236384in}}%
\pgfpathlineto{\pgfqpoint{1.492066in}{1.233070in}}%
\pgfpathlineto{\pgfqpoint{1.485693in}{1.225746in}}%
\pgfpathlineto{\pgfqpoint{1.483870in}{1.220143in}}%
\pgfpathlineto{\pgfqpoint{1.482766in}{1.207217in}}%
\pgfpathlineto{\pgfqpoint{1.483928in}{1.194290in}}%
\pgfpathlineto{\pgfqpoint{1.485693in}{1.188938in}}%
\pgfpathlineto{\pgfqpoint{1.492355in}{1.181364in}}%
\pgfpathclose%
\pgfpathmoveto{\pgfqpoint{1.493867in}{1.194290in}}%
\pgfpathlineto{\pgfqpoint{1.490597in}{1.207217in}}%
\pgfpathlineto{\pgfqpoint{1.493782in}{1.220143in}}%
\pgfpathlineto{\pgfqpoint{1.498962in}{1.225718in}}%
\pgfpathlineto{\pgfqpoint{1.512231in}{1.224065in}}%
\pgfpathlineto{\pgfqpoint{1.515086in}{1.220143in}}%
\pgfpathlineto{\pgfqpoint{1.517701in}{1.207217in}}%
\pgfpathlineto{\pgfqpoint{1.515014in}{1.194290in}}%
\pgfpathlineto{\pgfqpoint{1.512231in}{1.190492in}}%
\pgfpathlineto{\pgfqpoint{1.498962in}{1.188840in}}%
\pgfpathclose%
\pgfusepath{fill}%
\end{pgfscope}%
\begin{pgfscope}%
\pgfpathrectangle{\pgfqpoint{0.211875in}{0.211875in}}{\pgfqpoint{1.313625in}{1.279725in}}%
\pgfusepath{clip}%
\pgfsetbuttcap%
\pgfsetroundjoin%
\definecolor{currentfill}{rgb}{0.796501,0.105066,0.310630}%
\pgfsetfillcolor{currentfill}%
\pgfsetlinewidth{0.000000pt}%
\definecolor{currentstroke}{rgb}{0.000000,0.000000,0.000000}%
\pgfsetstrokecolor{currentstroke}%
\pgfsetdash{}{0pt}%
\pgfpathmoveto{\pgfqpoint{0.357833in}{1.253785in}}%
\pgfpathlineto{\pgfqpoint{0.371102in}{1.249807in}}%
\pgfpathlineto{\pgfqpoint{0.384371in}{1.249765in}}%
\pgfpathlineto{\pgfqpoint{0.397640in}{1.250949in}}%
\pgfpathlineto{\pgfqpoint{0.408593in}{1.258923in}}%
\pgfpathlineto{\pgfqpoint{0.410909in}{1.271059in}}%
\pgfpathlineto{\pgfqpoint{0.410983in}{1.271849in}}%
\pgfpathlineto{\pgfqpoint{0.411480in}{1.284776in}}%
\pgfpathlineto{\pgfqpoint{0.411441in}{1.297702in}}%
\pgfpathlineto{\pgfqpoint{0.410909in}{1.308279in}}%
\pgfpathlineto{\pgfqpoint{0.410726in}{1.310629in}}%
\pgfpathlineto{\pgfqpoint{0.405639in}{1.323555in}}%
\pgfpathlineto{\pgfqpoint{0.397640in}{1.326643in}}%
\pgfpathlineto{\pgfqpoint{0.384371in}{1.327652in}}%
\pgfpathlineto{\pgfqpoint{0.371102in}{1.327751in}}%
\pgfpathlineto{\pgfqpoint{0.357833in}{1.325642in}}%
\pgfpathlineto{\pgfqpoint{0.356347in}{1.323555in}}%
\pgfpathlineto{\pgfqpoint{0.354938in}{1.310629in}}%
\pgfpathlineto{\pgfqpoint{0.354776in}{1.297702in}}%
\pgfpathlineto{\pgfqpoint{0.354807in}{1.284776in}}%
\pgfpathlineto{\pgfqpoint{0.355032in}{1.271849in}}%
\pgfpathlineto{\pgfqpoint{0.355996in}{1.258923in}}%
\pgfpathclose%
\pgfpathmoveto{\pgfqpoint{0.376330in}{1.258923in}}%
\pgfpathlineto{\pgfqpoint{0.371102in}{1.260209in}}%
\pgfpathlineto{\pgfqpoint{0.363237in}{1.271849in}}%
\pgfpathlineto{\pgfqpoint{0.361292in}{1.284776in}}%
\pgfpathlineto{\pgfqpoint{0.361668in}{1.297702in}}%
\pgfpathlineto{\pgfqpoint{0.365173in}{1.310629in}}%
\pgfpathlineto{\pgfqpoint{0.371102in}{1.316912in}}%
\pgfpathlineto{\pgfqpoint{0.384371in}{1.319297in}}%
\pgfpathlineto{\pgfqpoint{0.397640in}{1.315018in}}%
\pgfpathlineto{\pgfqpoint{0.401148in}{1.310629in}}%
\pgfpathlineto{\pgfqpoint{0.404741in}{1.297702in}}%
\pgfpathlineto{\pgfqpoint{0.405144in}{1.284776in}}%
\pgfpathlineto{\pgfqpoint{0.403139in}{1.271849in}}%
\pgfpathlineto{\pgfqpoint{0.397640in}{1.262378in}}%
\pgfpathlineto{\pgfqpoint{0.389372in}{1.258923in}}%
\pgfpathlineto{\pgfqpoint{0.384371in}{1.257696in}}%
\pgfpathclose%
\pgfusepath{fill}%
\end{pgfscope}%
\begin{pgfscope}%
\pgfpathrectangle{\pgfqpoint{0.211875in}{0.211875in}}{\pgfqpoint{1.313625in}{1.279725in}}%
\pgfusepath{clip}%
\pgfsetbuttcap%
\pgfsetroundjoin%
\definecolor{currentfill}{rgb}{0.796501,0.105066,0.310630}%
\pgfsetfillcolor{currentfill}%
\pgfsetlinewidth{0.000000pt}%
\definecolor{currentstroke}{rgb}{0.000000,0.000000,0.000000}%
\pgfsetstrokecolor{currentstroke}%
\pgfsetdash{}{0pt}%
\pgfpathmoveto{\pgfqpoint{0.490523in}{1.253580in}}%
\pgfpathlineto{\pgfqpoint{0.503792in}{1.252960in}}%
\pgfpathlineto{\pgfqpoint{0.517061in}{1.255744in}}%
\pgfpathlineto{\pgfqpoint{0.520985in}{1.258923in}}%
\pgfpathlineto{\pgfqpoint{0.525860in}{1.271849in}}%
\pgfpathlineto{\pgfqpoint{0.526925in}{1.284776in}}%
\pgfpathlineto{\pgfqpoint{0.526751in}{1.297702in}}%
\pgfpathlineto{\pgfqpoint{0.524966in}{1.310629in}}%
\pgfpathlineto{\pgfqpoint{0.517061in}{1.321691in}}%
\pgfpathlineto{\pgfqpoint{0.509687in}{1.323555in}}%
\pgfpathlineto{\pgfqpoint{0.503792in}{1.324468in}}%
\pgfpathlineto{\pgfqpoint{0.490523in}{1.323940in}}%
\pgfpathlineto{\pgfqpoint{0.489002in}{1.323555in}}%
\pgfpathlineto{\pgfqpoint{0.477254in}{1.312998in}}%
\pgfpathlineto{\pgfqpoint{0.476618in}{1.310629in}}%
\pgfpathlineto{\pgfqpoint{0.475305in}{1.297702in}}%
\pgfpathlineto{\pgfqpoint{0.475192in}{1.284776in}}%
\pgfpathlineto{\pgfqpoint{0.476001in}{1.271849in}}%
\pgfpathlineto{\pgfqpoint{0.477254in}{1.265581in}}%
\pgfpathlineto{\pgfqpoint{0.480493in}{1.258923in}}%
\pgfpathclose%
\pgfpathmoveto{\pgfqpoint{0.484793in}{1.271849in}}%
\pgfpathlineto{\pgfqpoint{0.482029in}{1.284776in}}%
\pgfpathlineto{\pgfqpoint{0.482601in}{1.297702in}}%
\pgfpathlineto{\pgfqpoint{0.487696in}{1.310629in}}%
\pgfpathlineto{\pgfqpoint{0.490523in}{1.313315in}}%
\pgfpathlineto{\pgfqpoint{0.503792in}{1.315690in}}%
\pgfpathlineto{\pgfqpoint{0.514317in}{1.310629in}}%
\pgfpathlineto{\pgfqpoint{0.517061in}{1.307594in}}%
\pgfpathlineto{\pgfqpoint{0.520220in}{1.297702in}}%
\pgfpathlineto{\pgfqpoint{0.520744in}{1.284776in}}%
\pgfpathlineto{\pgfqpoint{0.518260in}{1.271849in}}%
\pgfpathlineto{\pgfqpoint{0.517061in}{1.269541in}}%
\pgfpathlineto{\pgfqpoint{0.503792in}{1.261468in}}%
\pgfpathlineto{\pgfqpoint{0.490523in}{1.264240in}}%
\pgfpathclose%
\pgfusepath{fill}%
\end{pgfscope}%
\begin{pgfscope}%
\pgfpathrectangle{\pgfqpoint{0.211875in}{0.211875in}}{\pgfqpoint{1.313625in}{1.279725in}}%
\pgfusepath{clip}%
\pgfsetbuttcap%
\pgfsetroundjoin%
\definecolor{currentfill}{rgb}{0.796501,0.105066,0.310630}%
\pgfsetfillcolor{currentfill}%
\pgfsetlinewidth{0.000000pt}%
\definecolor{currentstroke}{rgb}{0.000000,0.000000,0.000000}%
\pgfsetstrokecolor{currentstroke}%
\pgfsetdash{}{0pt}%
\pgfpathmoveto{\pgfqpoint{0.609943in}{1.256386in}}%
\pgfpathlineto{\pgfqpoint{0.623212in}{1.255702in}}%
\pgfpathlineto{\pgfqpoint{0.633132in}{1.258923in}}%
\pgfpathlineto{\pgfqpoint{0.636481in}{1.261084in}}%
\pgfpathlineto{\pgfqpoint{0.641453in}{1.271849in}}%
\pgfpathlineto{\pgfqpoint{0.642928in}{1.284776in}}%
\pgfpathlineto{\pgfqpoint{0.642651in}{1.297702in}}%
\pgfpathlineto{\pgfqpoint{0.640059in}{1.310629in}}%
\pgfpathlineto{\pgfqpoint{0.636481in}{1.316267in}}%
\pgfpathlineto{\pgfqpoint{0.623212in}{1.321527in}}%
\pgfpathlineto{\pgfqpoint{0.609943in}{1.320824in}}%
\pgfpathlineto{\pgfqpoint{0.597983in}{1.310629in}}%
\pgfpathlineto{\pgfqpoint{0.596674in}{1.306077in}}%
\pgfpathlineto{\pgfqpoint{0.595411in}{1.297702in}}%
\pgfpathlineto{\pgfqpoint{0.595180in}{1.284776in}}%
\pgfpathlineto{\pgfqpoint{0.596468in}{1.271849in}}%
\pgfpathlineto{\pgfqpoint{0.596674in}{1.271038in}}%
\pgfpathlineto{\pgfqpoint{0.604653in}{1.258923in}}%
\pgfpathclose%
\pgfpathmoveto{\pgfqpoint{0.606042in}{1.271849in}}%
\pgfpathlineto{\pgfqpoint{0.602474in}{1.284776in}}%
\pgfpathlineto{\pgfqpoint{0.603236in}{1.297702in}}%
\pgfpathlineto{\pgfqpoint{0.609886in}{1.310629in}}%
\pgfpathlineto{\pgfqpoint{0.609943in}{1.310677in}}%
\pgfpathlineto{\pgfqpoint{0.623212in}{1.312530in}}%
\pgfpathlineto{\pgfqpoint{0.626647in}{1.310629in}}%
\pgfpathlineto{\pgfqpoint{0.636132in}{1.297702in}}%
\pgfpathlineto{\pgfqpoint{0.636481in}{1.293810in}}%
\pgfpathlineto{\pgfqpoint{0.636907in}{1.284776in}}%
\pgfpathlineto{\pgfqpoint{0.636481in}{1.282491in}}%
\pgfpathlineto{\pgfqpoint{0.632092in}{1.271849in}}%
\pgfpathlineto{\pgfqpoint{0.623212in}{1.265055in}}%
\pgfpathlineto{\pgfqpoint{0.609943in}{1.267202in}}%
\pgfpathclose%
\pgfusepath{fill}%
\end{pgfscope}%
\begin{pgfscope}%
\pgfpathrectangle{\pgfqpoint{0.211875in}{0.211875in}}{\pgfqpoint{1.313625in}{1.279725in}}%
\pgfusepath{clip}%
\pgfsetbuttcap%
\pgfsetroundjoin%
\definecolor{currentfill}{rgb}{0.796501,0.105066,0.310630}%
\pgfsetfillcolor{currentfill}%
\pgfsetlinewidth{0.000000pt}%
\definecolor{currentstroke}{rgb}{0.000000,0.000000,0.000000}%
\pgfsetstrokecolor{currentstroke}%
\pgfsetdash{}{0pt}%
\pgfpathmoveto{\pgfqpoint{0.729364in}{1.258414in}}%
\pgfpathlineto{\pgfqpoint{0.742633in}{1.258027in}}%
\pgfpathlineto{\pgfqpoint{0.745078in}{1.258923in}}%
\pgfpathlineto{\pgfqpoint{0.755902in}{1.267544in}}%
\pgfpathlineto{\pgfqpoint{0.757652in}{1.271849in}}%
\pgfpathlineto{\pgfqpoint{0.759444in}{1.284776in}}%
\pgfpathlineto{\pgfqpoint{0.759087in}{1.297702in}}%
\pgfpathlineto{\pgfqpoint{0.755902in}{1.310546in}}%
\pgfpathlineto{\pgfqpoint{0.755853in}{1.310629in}}%
\pgfpathlineto{\pgfqpoint{0.742633in}{1.318946in}}%
\pgfpathlineto{\pgfqpoint{0.729364in}{1.318542in}}%
\pgfpathlineto{\pgfqpoint{0.718993in}{1.310629in}}%
\pgfpathlineto{\pgfqpoint{0.716095in}{1.302952in}}%
\pgfpathlineto{\pgfqpoint{0.715125in}{1.297702in}}%
\pgfpathlineto{\pgfqpoint{0.714799in}{1.284776in}}%
\pgfpathlineto{\pgfqpoint{0.716095in}{1.274327in}}%
\pgfpathlineto{\pgfqpoint{0.716634in}{1.271849in}}%
\pgfpathlineto{\pgfqpoint{0.728182in}{1.258923in}}%
\pgfpathclose%
\pgfpathmoveto{\pgfqpoint{0.726974in}{1.271849in}}%
\pgfpathlineto{\pgfqpoint{0.722601in}{1.284776in}}%
\pgfpathlineto{\pgfqpoint{0.723552in}{1.297702in}}%
\pgfpathlineto{\pgfqpoint{0.729364in}{1.307769in}}%
\pgfpathlineto{\pgfqpoint{0.742633in}{1.309285in}}%
\pgfpathlineto{\pgfqpoint{0.750878in}{1.297702in}}%
\pgfpathlineto{\pgfqpoint{0.752015in}{1.284776in}}%
\pgfpathlineto{\pgfqpoint{0.746836in}{1.271849in}}%
\pgfpathlineto{\pgfqpoint{0.742633in}{1.268188in}}%
\pgfpathlineto{\pgfqpoint{0.729364in}{1.269302in}}%
\pgfpathclose%
\pgfusepath{fill}%
\end{pgfscope}%
\begin{pgfscope}%
\pgfpathrectangle{\pgfqpoint{0.211875in}{0.211875in}}{\pgfqpoint{1.313625in}{1.279725in}}%
\pgfusepath{clip}%
\pgfsetbuttcap%
\pgfsetroundjoin%
\definecolor{currentfill}{rgb}{0.796501,0.105066,0.310630}%
\pgfsetfillcolor{currentfill}%
\pgfsetlinewidth{0.000000pt}%
\definecolor{currentstroke}{rgb}{0.000000,0.000000,0.000000}%
\pgfsetstrokecolor{currentstroke}%
\pgfsetdash{}{0pt}%
\pgfpathmoveto{\pgfqpoint{0.848784in}{1.260021in}}%
\pgfpathlineto{\pgfqpoint{0.862053in}{1.260228in}}%
\pgfpathlineto{\pgfqpoint{0.873889in}{1.271849in}}%
\pgfpathlineto{\pgfqpoint{0.875322in}{1.277243in}}%
\pgfpathlineto{\pgfqpoint{0.876396in}{1.284776in}}%
\pgfpathlineto{\pgfqpoint{0.875982in}{1.297702in}}%
\pgfpathlineto{\pgfqpoint{0.875322in}{1.300786in}}%
\pgfpathlineto{\pgfqpoint{0.870778in}{1.310629in}}%
\pgfpathlineto{\pgfqpoint{0.862053in}{1.316807in}}%
\pgfpathlineto{\pgfqpoint{0.848784in}{1.316990in}}%
\pgfpathlineto{\pgfqpoint{0.839429in}{1.310629in}}%
\pgfpathlineto{\pgfqpoint{0.835515in}{1.302510in}}%
\pgfpathlineto{\pgfqpoint{0.834464in}{1.297702in}}%
\pgfpathlineto{\pgfqpoint{0.834063in}{1.284776in}}%
\pgfpathlineto{\pgfqpoint{0.835515in}{1.274792in}}%
\pgfpathlineto{\pgfqpoint{0.836326in}{1.271849in}}%
\pgfpathclose%
\pgfpathmoveto{\pgfqpoint{0.847556in}{1.271849in}}%
\pgfpathlineto{\pgfqpoint{0.842361in}{1.284776in}}%
\pgfpathlineto{\pgfqpoint{0.843504in}{1.297702in}}%
\pgfpathlineto{\pgfqpoint{0.848784in}{1.305837in}}%
\pgfpathlineto{\pgfqpoint{0.862053in}{1.305540in}}%
\pgfpathlineto{\pgfqpoint{0.866994in}{1.297702in}}%
\pgfpathlineto{\pgfqpoint{0.868123in}{1.284776in}}%
\pgfpathlineto{\pgfqpoint{0.863022in}{1.271849in}}%
\pgfpathlineto{\pgfqpoint{0.862053in}{1.270898in}}%
\pgfpathlineto{\pgfqpoint{0.848784in}{1.270684in}}%
\pgfpathclose%
\pgfusepath{fill}%
\end{pgfscope}%
\begin{pgfscope}%
\pgfpathrectangle{\pgfqpoint{0.211875in}{0.211875in}}{\pgfqpoint{1.313625in}{1.279725in}}%
\pgfusepath{clip}%
\pgfsetbuttcap%
\pgfsetroundjoin%
\definecolor{currentfill}{rgb}{0.796501,0.105066,0.310630}%
\pgfsetfillcolor{currentfill}%
\pgfsetlinewidth{0.000000pt}%
\definecolor{currentstroke}{rgb}{0.000000,0.000000,0.000000}%
\pgfsetstrokecolor{currentstroke}%
\pgfsetdash{}{0pt}%
\pgfpathmoveto{\pgfqpoint{0.968205in}{1.261055in}}%
\pgfpathlineto{\pgfqpoint{0.981473in}{1.262171in}}%
\pgfpathlineto{\pgfqpoint{0.990276in}{1.271849in}}%
\pgfpathlineto{\pgfqpoint{0.993325in}{1.284776in}}%
\pgfpathlineto{\pgfqpoint{0.992692in}{1.297702in}}%
\pgfpathlineto{\pgfqpoint{0.987126in}{1.310629in}}%
\pgfpathlineto{\pgfqpoint{0.981473in}{1.315107in}}%
\pgfpathlineto{\pgfqpoint{0.968205in}{1.316066in}}%
\pgfpathlineto{\pgfqpoint{0.959167in}{1.310629in}}%
\pgfpathlineto{\pgfqpoint{0.954936in}{1.303656in}}%
\pgfpathlineto{\pgfqpoint{0.953428in}{1.297702in}}%
\pgfpathlineto{\pgfqpoint{0.952972in}{1.284776in}}%
\pgfpathlineto{\pgfqpoint{0.954936in}{1.273040in}}%
\pgfpathlineto{\pgfqpoint{0.955346in}{1.271849in}}%
\pgfpathclose%
\pgfpathmoveto{\pgfqpoint{0.967724in}{1.271849in}}%
\pgfpathlineto{\pgfqpoint{0.961668in}{1.284776in}}%
\pgfpathlineto{\pgfqpoint{0.963010in}{1.297702in}}%
\pgfpathlineto{\pgfqpoint{0.968205in}{1.304770in}}%
\pgfpathlineto{\pgfqpoint{0.981473in}{1.302358in}}%
\pgfpathlineto{\pgfqpoint{0.984090in}{1.297702in}}%
\pgfpathlineto{\pgfqpoint{0.985190in}{1.284776in}}%
\pgfpathlineto{\pgfqpoint{0.981473in}{1.274584in}}%
\pgfpathlineto{\pgfqpoint{0.971395in}{1.271849in}}%
\pgfpathlineto{\pgfqpoint{0.968205in}{1.271446in}}%
\pgfpathclose%
\pgfusepath{fill}%
\end{pgfscope}%
\begin{pgfscope}%
\pgfpathrectangle{\pgfqpoint{0.211875in}{0.211875in}}{\pgfqpoint{1.313625in}{1.279725in}}%
\pgfusepath{clip}%
\pgfsetbuttcap%
\pgfsetroundjoin%
\definecolor{currentfill}{rgb}{0.796501,0.105066,0.310630}%
\pgfsetfillcolor{currentfill}%
\pgfsetlinewidth{0.000000pt}%
\definecolor{currentstroke}{rgb}{0.000000,0.000000,0.000000}%
\pgfsetstrokecolor{currentstroke}%
\pgfsetdash{}{0pt}%
\pgfpathmoveto{\pgfqpoint{1.074356in}{1.270829in}}%
\pgfpathlineto{\pgfqpoint{1.087625in}{1.261458in}}%
\pgfpathlineto{\pgfqpoint{1.100894in}{1.263603in}}%
\pgfpathlineto{\pgfqpoint{1.107619in}{1.271849in}}%
\pgfpathlineto{\pgfqpoint{1.110564in}{1.284776in}}%
\pgfpathlineto{\pgfqpoint{1.109949in}{1.297702in}}%
\pgfpathlineto{\pgfqpoint{1.104552in}{1.310629in}}%
\pgfpathlineto{\pgfqpoint{1.100894in}{1.313859in}}%
\pgfpathlineto{\pgfqpoint{1.087625in}{1.315703in}}%
\pgfpathlineto{\pgfqpoint{1.077993in}{1.310629in}}%
\pgfpathlineto{\pgfqpoint{1.074356in}{1.305849in}}%
\pgfpathlineto{\pgfqpoint{1.072002in}{1.297702in}}%
\pgfpathlineto{\pgfqpoint{1.071510in}{1.284776in}}%
\pgfpathlineto{\pgfqpoint{1.073874in}{1.271849in}}%
\pgfpathclose%
\pgfpathmoveto{\pgfqpoint{1.087360in}{1.271849in}}%
\pgfpathlineto{\pgfqpoint{1.080373in}{1.284776in}}%
\pgfpathlineto{\pgfqpoint{1.081927in}{1.297702in}}%
\pgfpathlineto{\pgfqpoint{1.087625in}{1.304474in}}%
\pgfpathlineto{\pgfqpoint{1.100894in}{1.299742in}}%
\pgfpathlineto{\pgfqpoint{1.101920in}{1.297702in}}%
\pgfpathlineto{\pgfqpoint{1.102972in}{1.284776in}}%
\pgfpathlineto{\pgfqpoint{1.100894in}{1.278414in}}%
\pgfpathlineto{\pgfqpoint{1.088466in}{1.271849in}}%
\pgfpathlineto{\pgfqpoint{1.087625in}{1.271654in}}%
\pgfpathclose%
\pgfusepath{fill}%
\end{pgfscope}%
\begin{pgfscope}%
\pgfpathrectangle{\pgfqpoint{0.211875in}{0.211875in}}{\pgfqpoint{1.313625in}{1.279725in}}%
\pgfusepath{clip}%
\pgfsetbuttcap%
\pgfsetroundjoin%
\definecolor{currentfill}{rgb}{0.796501,0.105066,0.310630}%
\pgfsetfillcolor{currentfill}%
\pgfsetlinewidth{0.000000pt}%
\definecolor{currentstroke}{rgb}{0.000000,0.000000,0.000000}%
\pgfsetstrokecolor{currentstroke}%
\pgfsetdash{}{0pt}%
\pgfpathmoveto{\pgfqpoint{1.193777in}{1.268677in}}%
\pgfpathlineto{\pgfqpoint{1.207045in}{1.261273in}}%
\pgfpathlineto{\pgfqpoint{1.220314in}{1.264482in}}%
\pgfpathlineto{\pgfqpoint{1.225712in}{1.271849in}}%
\pgfpathlineto{\pgfqpoint{1.228474in}{1.284776in}}%
\pgfpathlineto{\pgfqpoint{1.227899in}{1.297702in}}%
\pgfpathlineto{\pgfqpoint{1.222832in}{1.310629in}}%
\pgfpathlineto{\pgfqpoint{1.220314in}{1.313102in}}%
\pgfpathlineto{\pgfqpoint{1.207045in}{1.315861in}}%
\pgfpathlineto{\pgfqpoint{1.195529in}{1.310629in}}%
\pgfpathlineto{\pgfqpoint{1.193777in}{1.308800in}}%
\pgfpathlineto{\pgfqpoint{1.190156in}{1.297702in}}%
\pgfpathlineto{\pgfqpoint{1.189646in}{1.284776in}}%
\pgfpathlineto{\pgfqpoint{1.192088in}{1.271849in}}%
\pgfpathclose%
\pgfpathmoveto{\pgfqpoint{1.206250in}{1.271849in}}%
\pgfpathlineto{\pgfqpoint{1.198215in}{1.284776in}}%
\pgfpathlineto{\pgfqpoint{1.200004in}{1.297702in}}%
\pgfpathlineto{\pgfqpoint{1.207045in}{1.304896in}}%
\pgfpathlineto{\pgfqpoint{1.220314in}{1.297725in}}%
\pgfpathlineto{\pgfqpoint{1.220324in}{1.297702in}}%
\pgfpathlineto{\pgfqpoint{1.221314in}{1.284776in}}%
\pgfpathlineto{\pgfqpoint{1.220314in}{1.281367in}}%
\pgfpathlineto{\pgfqpoint{1.208552in}{1.271849in}}%
\pgfpathlineto{\pgfqpoint{1.207045in}{1.271346in}}%
\pgfpathclose%
\pgfusepath{fill}%
\end{pgfscope}%
\begin{pgfscope}%
\pgfpathrectangle{\pgfqpoint{0.211875in}{0.211875in}}{\pgfqpoint{1.313625in}{1.279725in}}%
\pgfusepath{clip}%
\pgfsetbuttcap%
\pgfsetroundjoin%
\definecolor{currentfill}{rgb}{0.796501,0.105066,0.310630}%
\pgfsetfillcolor{currentfill}%
\pgfsetlinewidth{0.000000pt}%
\definecolor{currentstroke}{rgb}{0.000000,0.000000,0.000000}%
\pgfsetstrokecolor{currentstroke}%
\pgfsetdash{}{0pt}%
\pgfpathmoveto{\pgfqpoint{1.313197in}{1.266092in}}%
\pgfpathlineto{\pgfqpoint{1.326466in}{1.260520in}}%
\pgfpathlineto{\pgfqpoint{1.339735in}{1.264730in}}%
\pgfpathlineto{\pgfqpoint{1.344420in}{1.271849in}}%
\pgfpathlineto{\pgfqpoint{1.346929in}{1.284776in}}%
\pgfpathlineto{\pgfqpoint{1.346411in}{1.297702in}}%
\pgfpathlineto{\pgfqpoint{1.341819in}{1.310629in}}%
\pgfpathlineto{\pgfqpoint{1.339735in}{1.312908in}}%
\pgfpathlineto{\pgfqpoint{1.326466in}{1.316523in}}%
\pgfpathlineto{\pgfqpoint{1.313197in}{1.311725in}}%
\pgfpathlineto{\pgfqpoint{1.312295in}{1.310629in}}%
\pgfpathlineto{\pgfqpoint{1.307836in}{1.297702in}}%
\pgfpathlineto{\pgfqpoint{1.307329in}{1.284776in}}%
\pgfpathlineto{\pgfqpoint{1.309770in}{1.271849in}}%
\pgfpathclose%
\pgfpathmoveto{\pgfqpoint{1.323997in}{1.271849in}}%
\pgfpathlineto{\pgfqpoint{1.314711in}{1.284776in}}%
\pgfpathlineto{\pgfqpoint{1.316777in}{1.297702in}}%
\pgfpathlineto{\pgfqpoint{1.326466in}{1.306016in}}%
\pgfpathlineto{\pgfqpoint{1.338270in}{1.297702in}}%
\pgfpathlineto{\pgfqpoint{1.339735in}{1.290292in}}%
\pgfpathlineto{\pgfqpoint{1.340112in}{1.284776in}}%
\pgfpathlineto{\pgfqpoint{1.339735in}{1.283343in}}%
\pgfpathlineto{\pgfqpoint{1.329489in}{1.271849in}}%
\pgfpathlineto{\pgfqpoint{1.326466in}{1.270536in}}%
\pgfpathclose%
\pgfusepath{fill}%
\end{pgfscope}%
\begin{pgfscope}%
\pgfpathrectangle{\pgfqpoint{0.211875in}{0.211875in}}{\pgfqpoint{1.313625in}{1.279725in}}%
\pgfusepath{clip}%
\pgfsetbuttcap%
\pgfsetroundjoin%
\definecolor{currentfill}{rgb}{0.796501,0.105066,0.310630}%
\pgfsetfillcolor{currentfill}%
\pgfsetlinewidth{0.000000pt}%
\definecolor{currentstroke}{rgb}{0.000000,0.000000,0.000000}%
\pgfsetstrokecolor{currentstroke}%
\pgfsetdash{}{0pt}%
\pgfpathmoveto{\pgfqpoint{1.432617in}{1.263132in}}%
\pgfpathlineto{\pgfqpoint{1.445886in}{1.259195in}}%
\pgfpathlineto{\pgfqpoint{1.459155in}{1.264221in}}%
\pgfpathlineto{\pgfqpoint{1.463655in}{1.271849in}}%
\pgfpathlineto{\pgfqpoint{1.465844in}{1.284776in}}%
\pgfpathlineto{\pgfqpoint{1.465401in}{1.297702in}}%
\pgfpathlineto{\pgfqpoint{1.461418in}{1.310629in}}%
\pgfpathlineto{\pgfqpoint{1.459155in}{1.313392in}}%
\pgfpathlineto{\pgfqpoint{1.445886in}{1.317692in}}%
\pgfpathlineto{\pgfqpoint{1.432617in}{1.314317in}}%
\pgfpathlineto{\pgfqpoint{1.429235in}{1.310629in}}%
\pgfpathlineto{\pgfqpoint{1.424967in}{1.297702in}}%
\pgfpathlineto{\pgfqpoint{1.424485in}{1.284776in}}%
\pgfpathlineto{\pgfqpoint{1.426838in}{1.271849in}}%
\pgfpathclose%
\pgfpathmoveto{\pgfqpoint{1.439797in}{1.271849in}}%
\pgfpathlineto{\pgfqpoint{1.432617in}{1.279741in}}%
\pgfpathlineto{\pgfqpoint{1.431277in}{1.284776in}}%
\pgfpathlineto{\pgfqpoint{1.432146in}{1.297702in}}%
\pgfpathlineto{\pgfqpoint{1.432617in}{1.298855in}}%
\pgfpathlineto{\pgfqpoint{1.445886in}{1.307841in}}%
\pgfpathlineto{\pgfqpoint{1.457581in}{1.297702in}}%
\pgfpathlineto{\pgfqpoint{1.459155in}{1.287132in}}%
\pgfpathlineto{\pgfqpoint{1.459299in}{1.284776in}}%
\pgfpathlineto{\pgfqpoint{1.459155in}{1.284168in}}%
\pgfpathlineto{\pgfqpoint{1.450812in}{1.271849in}}%
\pgfpathlineto{\pgfqpoint{1.445886in}{1.269219in}}%
\pgfpathclose%
\pgfusepath{fill}%
\end{pgfscope}%
\begin{pgfscope}%
\pgfpathrectangle{\pgfqpoint{0.211875in}{0.211875in}}{\pgfqpoint{1.313625in}{1.279725in}}%
\pgfusepath{clip}%
\pgfsetbuttcap%
\pgfsetroundjoin%
\definecolor{currentfill}{rgb}{0.796501,0.105066,0.310630}%
\pgfsetfillcolor{currentfill}%
\pgfsetlinewidth{0.000000pt}%
\definecolor{currentstroke}{rgb}{0.000000,0.000000,0.000000}%
\pgfsetstrokecolor{currentstroke}%
\pgfsetdash{}{0pt}%
\pgfpathmoveto{\pgfqpoint{0.424178in}{1.333489in}}%
\pgfpathlineto{\pgfqpoint{0.437447in}{1.331972in}}%
\pgfpathlineto{\pgfqpoint{0.450716in}{1.332512in}}%
\pgfpathlineto{\pgfqpoint{0.462925in}{1.336482in}}%
\pgfpathlineto{\pgfqpoint{0.463985in}{1.337332in}}%
\pgfpathlineto{\pgfqpoint{0.468529in}{1.349408in}}%
\pgfpathlineto{\pgfqpoint{0.469540in}{1.362335in}}%
\pgfpathlineto{\pgfqpoint{0.469673in}{1.375261in}}%
\pgfpathlineto{\pgfqpoint{0.469086in}{1.388188in}}%
\pgfpathlineto{\pgfqpoint{0.465959in}{1.401114in}}%
\pgfpathlineto{\pgfqpoint{0.463985in}{1.403497in}}%
\pgfpathlineto{\pgfqpoint{0.450716in}{1.407508in}}%
\pgfpathlineto{\pgfqpoint{0.437447in}{1.408019in}}%
\pgfpathlineto{\pgfqpoint{0.424178in}{1.407018in}}%
\pgfpathlineto{\pgfqpoint{0.416846in}{1.401114in}}%
\pgfpathlineto{\pgfqpoint{0.415039in}{1.388188in}}%
\pgfpathlineto{\pgfqpoint{0.414745in}{1.375261in}}%
\pgfpathlineto{\pgfqpoint{0.414867in}{1.362335in}}%
\pgfpathlineto{\pgfqpoint{0.415550in}{1.349408in}}%
\pgfpathlineto{\pgfqpoint{0.419412in}{1.336482in}}%
\pgfpathclose%
\pgfpathmoveto{\pgfqpoint{0.424121in}{1.349408in}}%
\pgfpathlineto{\pgfqpoint{0.421148in}{1.362335in}}%
\pgfpathlineto{\pgfqpoint{0.420921in}{1.375261in}}%
\pgfpathlineto{\pgfqpoint{0.423093in}{1.388188in}}%
\pgfpathlineto{\pgfqpoint{0.424178in}{1.390571in}}%
\pgfpathlineto{\pgfqpoint{0.437447in}{1.399430in}}%
\pgfpathlineto{\pgfqpoint{0.450716in}{1.398249in}}%
\pgfpathlineto{\pgfqpoint{0.460660in}{1.388188in}}%
\pgfpathlineto{\pgfqpoint{0.463830in}{1.375261in}}%
\pgfpathlineto{\pgfqpoint{0.463510in}{1.362335in}}%
\pgfpathlineto{\pgfqpoint{0.459246in}{1.349408in}}%
\pgfpathlineto{\pgfqpoint{0.450716in}{1.341790in}}%
\pgfpathlineto{\pgfqpoint{0.437447in}{1.340648in}}%
\pgfpathlineto{\pgfqpoint{0.424178in}{1.349300in}}%
\pgfpathclose%
\pgfusepath{fill}%
\end{pgfscope}%
\begin{pgfscope}%
\pgfpathrectangle{\pgfqpoint{0.211875in}{0.211875in}}{\pgfqpoint{1.313625in}{1.279725in}}%
\pgfusepath{clip}%
\pgfsetbuttcap%
\pgfsetroundjoin%
\definecolor{currentfill}{rgb}{0.796501,0.105066,0.310630}%
\pgfsetfillcolor{currentfill}%
\pgfsetlinewidth{0.000000pt}%
\definecolor{currentstroke}{rgb}{0.000000,0.000000,0.000000}%
\pgfsetstrokecolor{currentstroke}%
\pgfsetdash{}{0pt}%
\pgfpathmoveto{\pgfqpoint{0.556867in}{1.334845in}}%
\pgfpathlineto{\pgfqpoint{0.570136in}{1.335868in}}%
\pgfpathlineto{\pgfqpoint{0.571852in}{1.336482in}}%
\pgfpathlineto{\pgfqpoint{0.583405in}{1.348293in}}%
\pgfpathlineto{\pgfqpoint{0.583754in}{1.349408in}}%
\pgfpathlineto{\pgfqpoint{0.585475in}{1.362335in}}%
\pgfpathlineto{\pgfqpoint{0.585640in}{1.375261in}}%
\pgfpathlineto{\pgfqpoint{0.584470in}{1.388188in}}%
\pgfpathlineto{\pgfqpoint{0.583405in}{1.392211in}}%
\pgfpathlineto{\pgfqpoint{0.576891in}{1.401114in}}%
\pgfpathlineto{\pgfqpoint{0.570136in}{1.404094in}}%
\pgfpathlineto{\pgfqpoint{0.556867in}{1.405060in}}%
\pgfpathlineto{\pgfqpoint{0.543598in}{1.402456in}}%
\pgfpathlineto{\pgfqpoint{0.541706in}{1.401114in}}%
\pgfpathlineto{\pgfqpoint{0.536034in}{1.388188in}}%
\pgfpathlineto{\pgfqpoint{0.535015in}{1.375261in}}%
\pgfpathlineto{\pgfqpoint{0.535180in}{1.362335in}}%
\pgfpathlineto{\pgfqpoint{0.536752in}{1.349408in}}%
\pgfpathlineto{\pgfqpoint{0.543598in}{1.337963in}}%
\pgfpathlineto{\pgfqpoint{0.547977in}{1.336482in}}%
\pgfpathclose%
\pgfpathmoveto{\pgfqpoint{0.547072in}{1.349408in}}%
\pgfpathlineto{\pgfqpoint{0.543598in}{1.354660in}}%
\pgfpathlineto{\pgfqpoint{0.541582in}{1.362335in}}%
\pgfpathlineto{\pgfqpoint{0.541311in}{1.375261in}}%
\pgfpathlineto{\pgfqpoint{0.543598in}{1.385907in}}%
\pgfpathlineto{\pgfqpoint{0.544777in}{1.388188in}}%
\pgfpathlineto{\pgfqpoint{0.556867in}{1.396049in}}%
\pgfpathlineto{\pgfqpoint{0.570136in}{1.393674in}}%
\pgfpathlineto{\pgfqpoint{0.575002in}{1.388188in}}%
\pgfpathlineto{\pgfqpoint{0.578731in}{1.375261in}}%
\pgfpathlineto{\pgfqpoint{0.578390in}{1.362335in}}%
\pgfpathlineto{\pgfqpoint{0.573471in}{1.349408in}}%
\pgfpathlineto{\pgfqpoint{0.570136in}{1.346095in}}%
\pgfpathlineto{\pgfqpoint{0.556867in}{1.343803in}}%
\pgfpathclose%
\pgfusepath{fill}%
\end{pgfscope}%
\begin{pgfscope}%
\pgfpathrectangle{\pgfqpoint{0.211875in}{0.211875in}}{\pgfqpoint{1.313625in}{1.279725in}}%
\pgfusepath{clip}%
\pgfsetbuttcap%
\pgfsetroundjoin%
\definecolor{currentfill}{rgb}{0.796501,0.105066,0.310630}%
\pgfsetfillcolor{currentfill}%
\pgfsetlinewidth{0.000000pt}%
\definecolor{currentstroke}{rgb}{0.000000,0.000000,0.000000}%
\pgfsetstrokecolor{currentstroke}%
\pgfsetdash{}{0pt}%
\pgfpathmoveto{\pgfqpoint{0.663019in}{1.341165in}}%
\pgfpathlineto{\pgfqpoint{0.676288in}{1.337240in}}%
\pgfpathlineto{\pgfqpoint{0.689557in}{1.339351in}}%
\pgfpathlineto{\pgfqpoint{0.698669in}{1.349408in}}%
\pgfpathlineto{\pgfqpoint{0.701513in}{1.362335in}}%
\pgfpathlineto{\pgfqpoint{0.701751in}{1.375261in}}%
\pgfpathlineto{\pgfqpoint{0.699714in}{1.388188in}}%
\pgfpathlineto{\pgfqpoint{0.689557in}{1.400930in}}%
\pgfpathlineto{\pgfqpoint{0.688575in}{1.401114in}}%
\pgfpathlineto{\pgfqpoint{0.676288in}{1.402707in}}%
\pgfpathlineto{\pgfqpoint{0.668400in}{1.401114in}}%
\pgfpathlineto{\pgfqpoint{0.663019in}{1.399175in}}%
\pgfpathlineto{\pgfqpoint{0.656565in}{1.388188in}}%
\pgfpathlineto{\pgfqpoint{0.654915in}{1.375261in}}%
\pgfpathlineto{\pgfqpoint{0.655118in}{1.362335in}}%
\pgfpathlineto{\pgfqpoint{0.657467in}{1.349408in}}%
\pgfpathclose%
\pgfpathmoveto{\pgfqpoint{0.670053in}{1.349408in}}%
\pgfpathlineto{\pgfqpoint{0.663019in}{1.357843in}}%
\pgfpathlineto{\pgfqpoint{0.661690in}{1.362335in}}%
\pgfpathlineto{\pgfqpoint{0.661378in}{1.375261in}}%
\pgfpathlineto{\pgfqpoint{0.663019in}{1.382003in}}%
\pgfpathlineto{\pgfqpoint{0.667073in}{1.388188in}}%
\pgfpathlineto{\pgfqpoint{0.676288in}{1.393330in}}%
\pgfpathlineto{\pgfqpoint{0.689557in}{1.389319in}}%
\pgfpathlineto{\pgfqpoint{0.690459in}{1.388188in}}%
\pgfpathlineto{\pgfqpoint{0.694577in}{1.375261in}}%
\pgfpathlineto{\pgfqpoint{0.694221in}{1.362335in}}%
\pgfpathlineto{\pgfqpoint{0.689557in}{1.350717in}}%
\pgfpathlineto{\pgfqpoint{0.687216in}{1.349408in}}%
\pgfpathlineto{\pgfqpoint{0.676288in}{1.346345in}}%
\pgfpathclose%
\pgfusepath{fill}%
\end{pgfscope}%
\begin{pgfscope}%
\pgfpathrectangle{\pgfqpoint{0.211875in}{0.211875in}}{\pgfqpoint{1.313625in}{1.279725in}}%
\pgfusepath{clip}%
\pgfsetbuttcap%
\pgfsetroundjoin%
\definecolor{currentfill}{rgb}{0.796501,0.105066,0.310630}%
\pgfsetfillcolor{currentfill}%
\pgfsetlinewidth{0.000000pt}%
\definecolor{currentstroke}{rgb}{0.000000,0.000000,0.000000}%
\pgfsetstrokecolor{currentstroke}%
\pgfsetdash{}{0pt}%
\pgfpathmoveto{\pgfqpoint{0.782439in}{1.343108in}}%
\pgfpathlineto{\pgfqpoint{0.795708in}{1.339298in}}%
\pgfpathlineto{\pgfqpoint{0.808977in}{1.342637in}}%
\pgfpathlineto{\pgfqpoint{0.814499in}{1.349408in}}%
\pgfpathlineto{\pgfqpoint{0.817737in}{1.362335in}}%
\pgfpathlineto{\pgfqpoint{0.817989in}{1.375261in}}%
\pgfpathlineto{\pgfqpoint{0.815614in}{1.388188in}}%
\pgfpathlineto{\pgfqpoint{0.808977in}{1.397456in}}%
\pgfpathlineto{\pgfqpoint{0.795708in}{1.400842in}}%
\pgfpathlineto{\pgfqpoint{0.782439in}{1.397007in}}%
\pgfpathlineto{\pgfqpoint{0.776628in}{1.388188in}}%
\pgfpathlineto{\pgfqpoint{0.774433in}{1.375261in}}%
\pgfpathlineto{\pgfqpoint{0.774671in}{1.362335in}}%
\pgfpathlineto{\pgfqpoint{0.777693in}{1.349408in}}%
\pgfpathclose%
\pgfpathmoveto{\pgfqpoint{0.793128in}{1.349408in}}%
\pgfpathlineto{\pgfqpoint{0.782439in}{1.359401in}}%
\pgfpathlineto{\pgfqpoint{0.781468in}{1.362335in}}%
\pgfpathlineto{\pgfqpoint{0.781118in}{1.375261in}}%
\pgfpathlineto{\pgfqpoint{0.782439in}{1.380090in}}%
\pgfpathlineto{\pgfqpoint{0.789280in}{1.388188in}}%
\pgfpathlineto{\pgfqpoint{0.795708in}{1.391194in}}%
\pgfpathlineto{\pgfqpoint{0.803099in}{1.388188in}}%
\pgfpathlineto{\pgfqpoint{0.808977in}{1.382542in}}%
\pgfpathlineto{\pgfqpoint{0.811159in}{1.375261in}}%
\pgfpathlineto{\pgfqpoint{0.810795in}{1.362335in}}%
\pgfpathlineto{\pgfqpoint{0.808977in}{1.357299in}}%
\pgfpathlineto{\pgfqpoint{0.798662in}{1.349408in}}%
\pgfpathlineto{\pgfqpoint{0.795708in}{1.348345in}}%
\pgfpathclose%
\pgfusepath{fill}%
\end{pgfscope}%
\begin{pgfscope}%
\pgfpathrectangle{\pgfqpoint{0.211875in}{0.211875in}}{\pgfqpoint{1.313625in}{1.279725in}}%
\pgfusepath{clip}%
\pgfsetbuttcap%
\pgfsetroundjoin%
\definecolor{currentfill}{rgb}{0.796501,0.105066,0.310630}%
\pgfsetfillcolor{currentfill}%
\pgfsetlinewidth{0.000000pt}%
\definecolor{currentstroke}{rgb}{0.000000,0.000000,0.000000}%
\pgfsetstrokecolor{currentstroke}%
\pgfsetdash{}{0pt}%
\pgfpathmoveto{\pgfqpoint{0.901860in}{1.344102in}}%
\pgfpathlineto{\pgfqpoint{0.915129in}{1.340806in}}%
\pgfpathlineto{\pgfqpoint{0.928398in}{1.345636in}}%
\pgfpathlineto{\pgfqpoint{0.931162in}{1.349408in}}%
\pgfpathlineto{\pgfqpoint{0.934632in}{1.362335in}}%
\pgfpathlineto{\pgfqpoint{0.934890in}{1.375261in}}%
\pgfpathlineto{\pgfqpoint{0.932314in}{1.388188in}}%
\pgfpathlineto{\pgfqpoint{0.928398in}{1.394291in}}%
\pgfpathlineto{\pgfqpoint{0.915129in}{1.399233in}}%
\pgfpathlineto{\pgfqpoint{0.901860in}{1.395883in}}%
\pgfpathlineto{\pgfqpoint{0.896196in}{1.388188in}}%
\pgfpathlineto{\pgfqpoint{0.893541in}{1.375261in}}%
\pgfpathlineto{\pgfqpoint{0.893810in}{1.362335in}}%
\pgfpathlineto{\pgfqpoint{0.897406in}{1.349408in}}%
\pgfpathclose%
\pgfpathmoveto{\pgfqpoint{0.900897in}{1.362335in}}%
\pgfpathlineto{\pgfqpoint{0.900511in}{1.375261in}}%
\pgfpathlineto{\pgfqpoint{0.901860in}{1.379665in}}%
\pgfpathlineto{\pgfqpoint{0.911430in}{1.388188in}}%
\pgfpathlineto{\pgfqpoint{0.915129in}{1.389585in}}%
\pgfpathlineto{\pgfqpoint{0.917941in}{1.388188in}}%
\pgfpathlineto{\pgfqpoint{0.928249in}{1.375261in}}%
\pgfpathlineto{\pgfqpoint{0.927408in}{1.362335in}}%
\pgfpathlineto{\pgfqpoint{0.915129in}{1.350157in}}%
\pgfpathlineto{\pgfqpoint{0.901860in}{1.359726in}}%
\pgfpathclose%
\pgfusepath{fill}%
\end{pgfscope}%
\begin{pgfscope}%
\pgfpathrectangle{\pgfqpoint{0.211875in}{0.211875in}}{\pgfqpoint{1.313625in}{1.279725in}}%
\pgfusepath{clip}%
\pgfsetbuttcap%
\pgfsetroundjoin%
\definecolor{currentfill}{rgb}{0.796501,0.105066,0.310630}%
\pgfsetfillcolor{currentfill}%
\pgfsetlinewidth{0.000000pt}%
\definecolor{currentstroke}{rgb}{0.000000,0.000000,0.000000}%
\pgfsetstrokecolor{currentstroke}%
\pgfsetdash{}{0pt}%
\pgfpathmoveto{\pgfqpoint{1.021280in}{1.344347in}}%
\pgfpathlineto{\pgfqpoint{1.034549in}{1.341794in}}%
\pgfpathlineto{\pgfqpoint{1.047818in}{1.348337in}}%
\pgfpathlineto{\pgfqpoint{1.048520in}{1.349408in}}%
\pgfpathlineto{\pgfqpoint{1.052082in}{1.362335in}}%
\pgfpathlineto{\pgfqpoint{1.052341in}{1.375261in}}%
\pgfpathlineto{\pgfqpoint{1.049683in}{1.388188in}}%
\pgfpathlineto{\pgfqpoint{1.047818in}{1.391445in}}%
\pgfpathlineto{\pgfqpoint{1.034549in}{1.398185in}}%
\pgfpathlineto{\pgfqpoint{1.021280in}{1.395577in}}%
\pgfpathlineto{\pgfqpoint{1.015219in}{1.388188in}}%
\pgfpathlineto{\pgfqpoint{1.012189in}{1.375261in}}%
\pgfpathlineto{\pgfqpoint{1.012485in}{1.362335in}}%
\pgfpathlineto{\pgfqpoint{1.016554in}{1.349408in}}%
\pgfpathclose%
\pgfpathmoveto{\pgfqpoint{1.019938in}{1.362335in}}%
\pgfpathlineto{\pgfqpoint{1.019519in}{1.375261in}}%
\pgfpathlineto{\pgfqpoint{1.021280in}{1.380410in}}%
\pgfpathlineto{\pgfqpoint{1.033559in}{1.388188in}}%
\pgfpathlineto{\pgfqpoint{1.034549in}{1.388472in}}%
\pgfpathlineto{\pgfqpoint{1.035033in}{1.388188in}}%
\pgfpathlineto{\pgfqpoint{1.044147in}{1.375261in}}%
\pgfpathlineto{\pgfqpoint{1.043413in}{1.362335in}}%
\pgfpathlineto{\pgfqpoint{1.034549in}{1.351912in}}%
\pgfpathlineto{\pgfqpoint{1.021280in}{1.359069in}}%
\pgfpathclose%
\pgfusepath{fill}%
\end{pgfscope}%
\begin{pgfscope}%
\pgfpathrectangle{\pgfqpoint{0.211875in}{0.211875in}}{\pgfqpoint{1.313625in}{1.279725in}}%
\pgfusepath{clip}%
\pgfsetbuttcap%
\pgfsetroundjoin%
\definecolor{currentfill}{rgb}{0.796501,0.105066,0.310630}%
\pgfsetfillcolor{currentfill}%
\pgfsetlinewidth{0.000000pt}%
\definecolor{currentstroke}{rgb}{0.000000,0.000000,0.000000}%
\pgfsetstrokecolor{currentstroke}%
\pgfsetdash{}{0pt}%
\pgfpathmoveto{\pgfqpoint{1.140701in}{1.343972in}}%
\pgfpathlineto{\pgfqpoint{1.153970in}{1.342268in}}%
\pgfpathlineto{\pgfqpoint{1.165873in}{1.349408in}}%
\pgfpathlineto{\pgfqpoint{1.167239in}{1.351577in}}%
\pgfpathlineto{\pgfqpoint{1.170007in}{1.362335in}}%
\pgfpathlineto{\pgfqpoint{1.170261in}{1.375261in}}%
\pgfpathlineto{\pgfqpoint{1.167628in}{1.388188in}}%
\pgfpathlineto{\pgfqpoint{1.167239in}{1.388957in}}%
\pgfpathlineto{\pgfqpoint{1.153970in}{1.397688in}}%
\pgfpathlineto{\pgfqpoint{1.140701in}{1.395944in}}%
\pgfpathlineto{\pgfqpoint{1.133610in}{1.388188in}}%
\pgfpathlineto{\pgfqpoint{1.130296in}{1.375261in}}%
\pgfpathlineto{\pgfqpoint{1.130616in}{1.362335in}}%
\pgfpathlineto{\pgfqpoint{1.135052in}{1.349408in}}%
\pgfpathclose%
\pgfpathmoveto{\pgfqpoint{1.138529in}{1.362335in}}%
\pgfpathlineto{\pgfqpoint{1.138079in}{1.375261in}}%
\pgfpathlineto{\pgfqpoint{1.140701in}{1.382129in}}%
\pgfpathlineto{\pgfqpoint{1.153970in}{1.387554in}}%
\pgfpathlineto{\pgfqpoint{1.161530in}{1.375261in}}%
\pgfpathlineto{\pgfqpoint{1.160884in}{1.362335in}}%
\pgfpathlineto{\pgfqpoint{1.153970in}{1.352908in}}%
\pgfpathlineto{\pgfqpoint{1.140701in}{1.357588in}}%
\pgfpathclose%
\pgfusepath{fill}%
\end{pgfscope}%
\begin{pgfscope}%
\pgfpathrectangle{\pgfqpoint{0.211875in}{0.211875in}}{\pgfqpoint{1.313625in}{1.279725in}}%
\pgfusepath{clip}%
\pgfsetbuttcap%
\pgfsetroundjoin%
\definecolor{currentfill}{rgb}{0.796501,0.105066,0.310630}%
\pgfsetfillcolor{currentfill}%
\pgfsetlinewidth{0.000000pt}%
\definecolor{currentstroke}{rgb}{0.000000,0.000000,0.000000}%
\pgfsetstrokecolor{currentstroke}%
\pgfsetdash{}{0pt}%
\pgfpathmoveto{\pgfqpoint{1.260121in}{1.343058in}}%
\pgfpathlineto{\pgfqpoint{1.273390in}{1.342218in}}%
\pgfpathlineto{\pgfqpoint{1.283907in}{1.349408in}}%
\pgfpathlineto{\pgfqpoint{1.286659in}{1.354857in}}%
\pgfpathlineto{\pgfqpoint{1.288353in}{1.362335in}}%
\pgfpathlineto{\pgfqpoint{1.288598in}{1.375261in}}%
\pgfpathlineto{\pgfqpoint{1.286659in}{1.385804in}}%
\pgfpathlineto{\pgfqpoint{1.285724in}{1.388188in}}%
\pgfpathlineto{\pgfqpoint{1.273390in}{1.397755in}}%
\pgfpathlineto{\pgfqpoint{1.260121in}{1.396893in}}%
\pgfpathlineto{\pgfqpoint{1.251236in}{1.388188in}}%
\pgfpathlineto{\pgfqpoint{1.247736in}{1.375261in}}%
\pgfpathlineto{\pgfqpoint{1.248077in}{1.362335in}}%
\pgfpathlineto{\pgfqpoint{1.252765in}{1.349408in}}%
\pgfpathclose%
\pgfpathmoveto{\pgfqpoint{1.256569in}{1.362335in}}%
\pgfpathlineto{\pgfqpoint{1.256090in}{1.375261in}}%
\pgfpathlineto{\pgfqpoint{1.260121in}{1.384701in}}%
\pgfpathlineto{\pgfqpoint{1.273390in}{1.387308in}}%
\pgfpathlineto{\pgfqpoint{1.279872in}{1.375261in}}%
\pgfpathlineto{\pgfqpoint{1.279302in}{1.362335in}}%
\pgfpathlineto{\pgfqpoint{1.273390in}{1.353131in}}%
\pgfpathlineto{\pgfqpoint{1.260121in}{1.355382in}}%
\pgfpathclose%
\pgfusepath{fill}%
\end{pgfscope}%
\begin{pgfscope}%
\pgfpathrectangle{\pgfqpoint{0.211875in}{0.211875in}}{\pgfqpoint{1.313625in}{1.279725in}}%
\pgfusepath{clip}%
\pgfsetbuttcap%
\pgfsetroundjoin%
\definecolor{currentfill}{rgb}{0.796501,0.105066,0.310630}%
\pgfsetfillcolor{currentfill}%
\pgfsetlinewidth{0.000000pt}%
\definecolor{currentstroke}{rgb}{0.000000,0.000000,0.000000}%
\pgfsetstrokecolor{currentstroke}%
\pgfsetdash{}{0pt}%
\pgfpathmoveto{\pgfqpoint{1.379542in}{1.341655in}}%
\pgfpathlineto{\pgfqpoint{1.392811in}{1.341607in}}%
\pgfpathlineto{\pgfqpoint{1.402922in}{1.349408in}}%
\pgfpathlineto{\pgfqpoint{1.406080in}{1.357228in}}%
\pgfpathlineto{\pgfqpoint{1.407083in}{1.362335in}}%
\pgfpathlineto{\pgfqpoint{1.407315in}{1.375261in}}%
\pgfpathlineto{\pgfqpoint{1.406080in}{1.383082in}}%
\pgfpathlineto{\pgfqpoint{1.404495in}{1.388188in}}%
\pgfpathlineto{\pgfqpoint{1.392811in}{1.398425in}}%
\pgfpathlineto{\pgfqpoint{1.379542in}{1.398369in}}%
\pgfpathlineto{\pgfqpoint{1.367889in}{1.388188in}}%
\pgfpathlineto{\pgfqpoint{1.366273in}{1.383136in}}%
\pgfpathlineto{\pgfqpoint{1.364998in}{1.375261in}}%
\pgfpathlineto{\pgfqpoint{1.365231in}{1.362335in}}%
\pgfpathlineto{\pgfqpoint{1.366273in}{1.357151in}}%
\pgfpathlineto{\pgfqpoint{1.369483in}{1.349408in}}%
\pgfpathclose%
\pgfpathmoveto{\pgfqpoint{1.373907in}{1.362335in}}%
\pgfpathlineto{\pgfqpoint{1.373400in}{1.375261in}}%
\pgfpathlineto{\pgfqpoint{1.379542in}{1.388063in}}%
\pgfpathlineto{\pgfqpoint{1.392811in}{1.388033in}}%
\pgfpathlineto{\pgfqpoint{1.398885in}{1.375261in}}%
\pgfpathlineto{\pgfqpoint{1.398383in}{1.362335in}}%
\pgfpathlineto{\pgfqpoint{1.392811in}{1.352532in}}%
\pgfpathlineto{\pgfqpoint{1.379542in}{1.352502in}}%
\pgfpathclose%
\pgfusepath{fill}%
\end{pgfscope}%
\begin{pgfscope}%
\pgfpathrectangle{\pgfqpoint{0.211875in}{0.211875in}}{\pgfqpoint{1.313625in}{1.279725in}}%
\pgfusepath{clip}%
\pgfsetbuttcap%
\pgfsetroundjoin%
\definecolor{currentfill}{rgb}{0.796501,0.105066,0.310630}%
\pgfsetfillcolor{currentfill}%
\pgfsetlinewidth{0.000000pt}%
\definecolor{currentstroke}{rgb}{0.000000,0.000000,0.000000}%
\pgfsetstrokecolor{currentstroke}%
\pgfsetdash{}{0pt}%
\pgfpathmoveto{\pgfqpoint{1.485693in}{1.348468in}}%
\pgfpathlineto{\pgfqpoint{1.498962in}{1.339785in}}%
\pgfpathlineto{\pgfqpoint{1.512231in}{1.340375in}}%
\pgfpathlineto{\pgfqpoint{1.522678in}{1.349408in}}%
\pgfpathlineto{\pgfqpoint{1.525500in}{1.358260in}}%
\pgfpathlineto{\pgfqpoint{1.525500in}{1.362335in}}%
\pgfpathlineto{\pgfqpoint{1.525500in}{1.375261in}}%
\pgfpathlineto{\pgfqpoint{1.525500in}{1.382013in}}%
\pgfpathlineto{\pgfqpoint{1.524010in}{1.388188in}}%
\pgfpathlineto{\pgfqpoint{1.512231in}{1.399767in}}%
\pgfpathlineto{\pgfqpoint{1.498962in}{1.400348in}}%
\pgfpathlineto{\pgfqpoint{1.485693in}{1.391451in}}%
\pgfpathlineto{\pgfqpoint{1.484222in}{1.388188in}}%
\pgfpathlineto{\pgfqpoint{1.482130in}{1.375261in}}%
\pgfpathlineto{\pgfqpoint{1.482351in}{1.362335in}}%
\pgfpathlineto{\pgfqpoint{1.485205in}{1.349408in}}%
\pgfpathclose%
\pgfpathmoveto{\pgfqpoint{1.498577in}{1.349408in}}%
\pgfpathlineto{\pgfqpoint{1.490304in}{1.362335in}}%
\pgfpathlineto{\pgfqpoint{1.489771in}{1.375261in}}%
\pgfpathlineto{\pgfqpoint{1.496158in}{1.388188in}}%
\pgfpathlineto{\pgfqpoint{1.498962in}{1.390358in}}%
\pgfpathlineto{\pgfqpoint{1.512231in}{1.389080in}}%
\pgfpathlineto{\pgfqpoint{1.513139in}{1.388188in}}%
\pgfpathlineto{\pgfqpoint{1.518399in}{1.375261in}}%
\pgfpathlineto{\pgfqpoint{1.517961in}{1.362335in}}%
\pgfpathlineto{\pgfqpoint{1.512231in}{1.351025in}}%
\pgfpathlineto{\pgfqpoint{1.501959in}{1.349408in}}%
\pgfpathlineto{\pgfqpoint{1.498962in}{1.349145in}}%
\pgfpathclose%
\pgfusepath{fill}%
\end{pgfscope}%
\begin{pgfscope}%
\pgfpathrectangle{\pgfqpoint{0.211875in}{0.211875in}}{\pgfqpoint{1.313625in}{1.279725in}}%
\pgfusepath{clip}%
\pgfsetbuttcap%
\pgfsetroundjoin%
\definecolor{currentfill}{rgb}{0.796501,0.105066,0.310630}%
\pgfsetfillcolor{currentfill}%
\pgfsetlinewidth{0.000000pt}%
\definecolor{currentstroke}{rgb}{0.000000,0.000000,0.000000}%
\pgfsetstrokecolor{currentstroke}%
\pgfsetdash{}{0pt}%
\pgfpathmoveto{\pgfqpoint{0.503792in}{1.413938in}}%
\pgfpathlineto{\pgfqpoint{0.504698in}{1.414041in}}%
\pgfpathlineto{\pgfqpoint{0.517061in}{1.416229in}}%
\pgfpathlineto{\pgfqpoint{0.525872in}{1.426967in}}%
\pgfpathlineto{\pgfqpoint{0.527786in}{1.439894in}}%
\pgfpathlineto{\pgfqpoint{0.528204in}{1.452820in}}%
\pgfpathlineto{\pgfqpoint{0.527800in}{1.465747in}}%
\pgfpathlineto{\pgfqpoint{0.525551in}{1.478673in}}%
\pgfpathlineto{\pgfqpoint{0.517061in}{1.486807in}}%
\pgfpathlineto{\pgfqpoint{0.503792in}{1.488620in}}%
\pgfpathlineto{\pgfqpoint{0.490523in}{1.488451in}}%
\pgfpathlineto{\pgfqpoint{0.477254in}{1.483065in}}%
\pgfpathlineto{\pgfqpoint{0.475522in}{1.478673in}}%
\pgfpathlineto{\pgfqpoint{0.474175in}{1.465747in}}%
\pgfpathlineto{\pgfqpoint{0.473970in}{1.452820in}}%
\pgfpathlineto{\pgfqpoint{0.474281in}{1.439894in}}%
\pgfpathlineto{\pgfqpoint{0.475614in}{1.426967in}}%
\pgfpathlineto{\pgfqpoint{0.477254in}{1.421911in}}%
\pgfpathlineto{\pgfqpoint{0.490523in}{1.414306in}}%
\pgfpathlineto{\pgfqpoint{0.499368in}{1.414041in}}%
\pgfpathclose%
\pgfpathmoveto{\pgfqpoint{0.487431in}{1.426967in}}%
\pgfpathlineto{\pgfqpoint{0.481553in}{1.439894in}}%
\pgfpathlineto{\pgfqpoint{0.480472in}{1.452820in}}%
\pgfpathlineto{\pgfqpoint{0.482130in}{1.465747in}}%
\pgfpathlineto{\pgfqpoint{0.490523in}{1.478482in}}%
\pgfpathlineto{\pgfqpoint{0.491401in}{1.478673in}}%
\pgfpathlineto{\pgfqpoint{0.503792in}{1.480369in}}%
\pgfpathlineto{\pgfqpoint{0.508905in}{1.478673in}}%
\pgfpathlineto{\pgfqpoint{0.517061in}{1.473406in}}%
\pgfpathlineto{\pgfqpoint{0.520528in}{1.465747in}}%
\pgfpathlineto{\pgfqpoint{0.522066in}{1.452820in}}%
\pgfpathlineto{\pgfqpoint{0.521082in}{1.439894in}}%
\pgfpathlineto{\pgfqpoint{0.517061in}{1.429156in}}%
\pgfpathlineto{\pgfqpoint{0.514625in}{1.426967in}}%
\pgfpathlineto{\pgfqpoint{0.503792in}{1.422376in}}%
\pgfpathlineto{\pgfqpoint{0.490523in}{1.424408in}}%
\pgfpathclose%
\pgfusepath{fill}%
\end{pgfscope}%
\begin{pgfscope}%
\pgfpathrectangle{\pgfqpoint{0.211875in}{0.211875in}}{\pgfqpoint{1.313625in}{1.279725in}}%
\pgfusepath{clip}%
\pgfsetbuttcap%
\pgfsetroundjoin%
\definecolor{currentfill}{rgb}{0.796501,0.105066,0.310630}%
\pgfsetfillcolor{currentfill}%
\pgfsetlinewidth{0.000000pt}%
\definecolor{currentstroke}{rgb}{0.000000,0.000000,0.000000}%
\pgfsetstrokecolor{currentstroke}%
\pgfsetdash{}{0pt}%
\pgfpathmoveto{\pgfqpoint{0.609943in}{1.417356in}}%
\pgfpathlineto{\pgfqpoint{0.623212in}{1.416815in}}%
\pgfpathlineto{\pgfqpoint{0.636481in}{1.421184in}}%
\pgfpathlineto{\pgfqpoint{0.640709in}{1.426967in}}%
\pgfpathlineto{\pgfqpoint{0.643606in}{1.439894in}}%
\pgfpathlineto{\pgfqpoint{0.644187in}{1.452820in}}%
\pgfpathlineto{\pgfqpoint{0.643445in}{1.465747in}}%
\pgfpathlineto{\pgfqpoint{0.639589in}{1.478673in}}%
\pgfpathlineto{\pgfqpoint{0.636481in}{1.482023in}}%
\pgfpathlineto{\pgfqpoint{0.623212in}{1.485814in}}%
\pgfpathlineto{\pgfqpoint{0.609943in}{1.485402in}}%
\pgfpathlineto{\pgfqpoint{0.598136in}{1.478673in}}%
\pgfpathlineto{\pgfqpoint{0.596674in}{1.475774in}}%
\pgfpathlineto{\pgfqpoint{0.594543in}{1.465747in}}%
\pgfpathlineto{\pgfqpoint{0.593952in}{1.452820in}}%
\pgfpathlineto{\pgfqpoint{0.594453in}{1.439894in}}%
\pgfpathlineto{\pgfqpoint{0.596674in}{1.427895in}}%
\pgfpathlineto{\pgfqpoint{0.597020in}{1.426967in}}%
\pgfpathclose%
\pgfpathmoveto{\pgfqpoint{0.609995in}{1.426967in}}%
\pgfpathlineto{\pgfqpoint{0.609943in}{1.426978in}}%
\pgfpathlineto{\pgfqpoint{0.602201in}{1.439894in}}%
\pgfpathlineto{\pgfqpoint{0.600801in}{1.452820in}}%
\pgfpathlineto{\pgfqpoint{0.603068in}{1.465747in}}%
\pgfpathlineto{\pgfqpoint{0.609943in}{1.475089in}}%
\pgfpathlineto{\pgfqpoint{0.623212in}{1.477063in}}%
\pgfpathlineto{\pgfqpoint{0.636254in}{1.465747in}}%
\pgfpathlineto{\pgfqpoint{0.636481in}{1.464998in}}%
\pgfpathlineto{\pgfqpoint{0.638207in}{1.452820in}}%
\pgfpathlineto{\pgfqpoint{0.637072in}{1.439894in}}%
\pgfpathlineto{\pgfqpoint{0.636481in}{1.438119in}}%
\pgfpathlineto{\pgfqpoint{0.626500in}{1.426967in}}%
\pgfpathlineto{\pgfqpoint{0.623212in}{1.425366in}}%
\pgfpathclose%
\pgfusepath{fill}%
\end{pgfscope}%
\begin{pgfscope}%
\pgfpathrectangle{\pgfqpoint{0.211875in}{0.211875in}}{\pgfqpoint{1.313625in}{1.279725in}}%
\pgfusepath{clip}%
\pgfsetbuttcap%
\pgfsetroundjoin%
\definecolor{currentfill}{rgb}{0.796501,0.105066,0.310630}%
\pgfsetfillcolor{currentfill}%
\pgfsetlinewidth{0.000000pt}%
\definecolor{currentstroke}{rgb}{0.000000,0.000000,0.000000}%
\pgfsetstrokecolor{currentstroke}%
\pgfsetdash{}{0pt}%
\pgfpathmoveto{\pgfqpoint{0.729364in}{1.419567in}}%
\pgfpathlineto{\pgfqpoint{0.742633in}{1.419248in}}%
\pgfpathlineto{\pgfqpoint{0.755902in}{1.426308in}}%
\pgfpathlineto{\pgfqpoint{0.756327in}{1.426967in}}%
\pgfpathlineto{\pgfqpoint{0.759982in}{1.439894in}}%
\pgfpathlineto{\pgfqpoint{0.760689in}{1.452820in}}%
\pgfpathlineto{\pgfqpoint{0.759687in}{1.465747in}}%
\pgfpathlineto{\pgfqpoint{0.755902in}{1.476532in}}%
\pgfpathlineto{\pgfqpoint{0.753759in}{1.478673in}}%
\pgfpathlineto{\pgfqpoint{0.742633in}{1.483453in}}%
\pgfpathlineto{\pgfqpoint{0.729364in}{1.483194in}}%
\pgfpathlineto{\pgfqpoint{0.720514in}{1.478673in}}%
\pgfpathlineto{\pgfqpoint{0.716095in}{1.472004in}}%
\pgfpathlineto{\pgfqpoint{0.714461in}{1.465747in}}%
\pgfpathlineto{\pgfqpoint{0.713560in}{1.452820in}}%
\pgfpathlineto{\pgfqpoint{0.714215in}{1.439894in}}%
\pgfpathlineto{\pgfqpoint{0.716095in}{1.431466in}}%
\pgfpathlineto{\pgfqpoint{0.718262in}{1.426967in}}%
\pgfpathclose%
\pgfpathmoveto{\pgfqpoint{0.722508in}{1.439894in}}%
\pgfpathlineto{\pgfqpoint{0.720789in}{1.452820in}}%
\pgfpathlineto{\pgfqpoint{0.723657in}{1.465747in}}%
\pgfpathlineto{\pgfqpoint{0.729364in}{1.472676in}}%
\pgfpathlineto{\pgfqpoint{0.742633in}{1.473716in}}%
\pgfpathlineto{\pgfqpoint{0.750693in}{1.465747in}}%
\pgfpathlineto{\pgfqpoint{0.754128in}{1.452820in}}%
\pgfpathlineto{\pgfqpoint{0.752074in}{1.439894in}}%
\pgfpathlineto{\pgfqpoint{0.742633in}{1.428427in}}%
\pgfpathlineto{\pgfqpoint{0.729364in}{1.429659in}}%
\pgfpathclose%
\pgfusepath{fill}%
\end{pgfscope}%
\begin{pgfscope}%
\pgfpathrectangle{\pgfqpoint{0.211875in}{0.211875in}}{\pgfqpoint{1.313625in}{1.279725in}}%
\pgfusepath{clip}%
\pgfsetbuttcap%
\pgfsetroundjoin%
\definecolor{currentfill}{rgb}{0.796501,0.105066,0.310630}%
\pgfsetfillcolor{currentfill}%
\pgfsetlinewidth{0.000000pt}%
\definecolor{currentstroke}{rgb}{0.000000,0.000000,0.000000}%
\pgfsetstrokecolor{currentstroke}%
\pgfsetdash{}{0pt}%
\pgfpathmoveto{\pgfqpoint{0.848784in}{1.421078in}}%
\pgfpathlineto{\pgfqpoint{0.862053in}{1.421251in}}%
\pgfpathlineto{\pgfqpoint{0.871253in}{1.426967in}}%
\pgfpathlineto{\pgfqpoint{0.875322in}{1.433972in}}%
\pgfpathlineto{\pgfqpoint{0.876836in}{1.439894in}}%
\pgfpathlineto{\pgfqpoint{0.877634in}{1.452820in}}%
\pgfpathlineto{\pgfqpoint{0.876444in}{1.465747in}}%
\pgfpathlineto{\pgfqpoint{0.875322in}{1.469454in}}%
\pgfpathlineto{\pgfqpoint{0.867955in}{1.478673in}}%
\pgfpathlineto{\pgfqpoint{0.862053in}{1.481517in}}%
\pgfpathlineto{\pgfqpoint{0.848784in}{1.481685in}}%
\pgfpathlineto{\pgfqpoint{0.842180in}{1.478673in}}%
\pgfpathlineto{\pgfqpoint{0.835515in}{1.470798in}}%
\pgfpathlineto{\pgfqpoint{0.833951in}{1.465747in}}%
\pgfpathlineto{\pgfqpoint{0.832807in}{1.452820in}}%
\pgfpathlineto{\pgfqpoint{0.833585in}{1.439894in}}%
\pgfpathlineto{\pgfqpoint{0.835515in}{1.432502in}}%
\pgfpathlineto{\pgfqpoint{0.838882in}{1.426967in}}%
\pgfpathclose%
\pgfpathmoveto{\pgfqpoint{0.842424in}{1.439894in}}%
\pgfpathlineto{\pgfqpoint{0.840378in}{1.452820in}}%
\pgfpathlineto{\pgfqpoint{0.843850in}{1.465747in}}%
\pgfpathlineto{\pgfqpoint{0.848784in}{1.471077in}}%
\pgfpathlineto{\pgfqpoint{0.862053in}{1.470846in}}%
\pgfpathlineto{\pgfqpoint{0.866623in}{1.465747in}}%
\pgfpathlineto{\pgfqpoint{0.870056in}{1.452820in}}%
\pgfpathlineto{\pgfqpoint{0.868035in}{1.439894in}}%
\pgfpathlineto{\pgfqpoint{0.862053in}{1.431699in}}%
\pgfpathlineto{\pgfqpoint{0.848784in}{1.431437in}}%
\pgfpathclose%
\pgfusepath{fill}%
\end{pgfscope}%
\begin{pgfscope}%
\pgfpathrectangle{\pgfqpoint{0.211875in}{0.211875in}}{\pgfqpoint{1.313625in}{1.279725in}}%
\pgfusepath{clip}%
\pgfsetbuttcap%
\pgfsetroundjoin%
\definecolor{currentfill}{rgb}{0.796501,0.105066,0.310630}%
\pgfsetfillcolor{currentfill}%
\pgfsetlinewidth{0.000000pt}%
\definecolor{currentstroke}{rgb}{0.000000,0.000000,0.000000}%
\pgfsetstrokecolor{currentstroke}%
\pgfsetdash{}{0pt}%
\pgfpathmoveto{\pgfqpoint{0.968205in}{1.421984in}}%
\pgfpathlineto{\pgfqpoint{0.981473in}{1.422826in}}%
\pgfpathlineto{\pgfqpoint{0.987440in}{1.426967in}}%
\pgfpathlineto{\pgfqpoint{0.993855in}{1.439894in}}%
\pgfpathlineto{\pgfqpoint{0.994742in}{1.449209in}}%
\pgfpathlineto{\pgfqpoint{0.994967in}{1.452820in}}%
\pgfpathlineto{\pgfqpoint{0.994742in}{1.455364in}}%
\pgfpathlineto{\pgfqpoint{0.993217in}{1.465747in}}%
\pgfpathlineto{\pgfqpoint{0.983948in}{1.478673in}}%
\pgfpathlineto{\pgfqpoint{0.981473in}{1.480004in}}%
\pgfpathlineto{\pgfqpoint{0.968205in}{1.480777in}}%
\pgfpathlineto{\pgfqpoint{0.963004in}{1.478673in}}%
\pgfpathlineto{\pgfqpoint{0.954936in}{1.471095in}}%
\pgfpathlineto{\pgfqpoint{0.953015in}{1.465747in}}%
\pgfpathlineto{\pgfqpoint{0.951694in}{1.452820in}}%
\pgfpathlineto{\pgfqpoint{0.952563in}{1.439894in}}%
\pgfpathlineto{\pgfqpoint{0.954936in}{1.431986in}}%
\pgfpathlineto{\pgfqpoint{0.958747in}{1.426967in}}%
\pgfpathclose%
\pgfpathmoveto{\pgfqpoint{0.961859in}{1.439894in}}%
\pgfpathlineto{\pgfqpoint{0.959471in}{1.452820in}}%
\pgfpathlineto{\pgfqpoint{0.963563in}{1.465747in}}%
\pgfpathlineto{\pgfqpoint{0.968205in}{1.470176in}}%
\pgfpathlineto{\pgfqpoint{0.981473in}{1.468437in}}%
\pgfpathlineto{\pgfqpoint{0.983624in}{1.465747in}}%
\pgfpathlineto{\pgfqpoint{0.986979in}{1.452820in}}%
\pgfpathlineto{\pgfqpoint{0.985021in}{1.439894in}}%
\pgfpathlineto{\pgfqpoint{0.981473in}{1.434449in}}%
\pgfpathlineto{\pgfqpoint{0.968205in}{1.432436in}}%
\pgfpathclose%
\pgfusepath{fill}%
\end{pgfscope}%
\begin{pgfscope}%
\pgfpathrectangle{\pgfqpoint{0.211875in}{0.211875in}}{\pgfqpoint{1.313625in}{1.279725in}}%
\pgfusepath{clip}%
\pgfsetbuttcap%
\pgfsetroundjoin%
\definecolor{currentfill}{rgb}{0.796501,0.105066,0.310630}%
\pgfsetfillcolor{currentfill}%
\pgfsetlinewidth{0.000000pt}%
\definecolor{currentstroke}{rgb}{0.000000,0.000000,0.000000}%
\pgfsetstrokecolor{currentstroke}%
\pgfsetdash{}{0pt}%
\pgfpathmoveto{\pgfqpoint{1.087625in}{1.422350in}}%
\pgfpathlineto{\pgfqpoint{1.100894in}{1.423957in}}%
\pgfpathlineto{\pgfqpoint{1.104790in}{1.426967in}}%
\pgfpathlineto{\pgfqpoint{1.111023in}{1.439894in}}%
\pgfpathlineto{\pgfqpoint{1.112184in}{1.452820in}}%
\pgfpathlineto{\pgfqpoint{1.110388in}{1.465747in}}%
\pgfpathlineto{\pgfqpoint{1.101322in}{1.478673in}}%
\pgfpathlineto{\pgfqpoint{1.100894in}{1.478929in}}%
\pgfpathlineto{\pgfqpoint{1.087625in}{1.480404in}}%
\pgfpathlineto{\pgfqpoint{1.082753in}{1.478673in}}%
\pgfpathlineto{\pgfqpoint{1.074356in}{1.472371in}}%
\pgfpathlineto{\pgfqpoint{1.071640in}{1.465747in}}%
\pgfpathlineto{\pgfqpoint{1.070203in}{1.452820in}}%
\pgfpathlineto{\pgfqpoint{1.071134in}{1.439894in}}%
\pgfpathlineto{\pgfqpoint{1.074356in}{1.430426in}}%
\pgfpathlineto{\pgfqpoint{1.077628in}{1.426967in}}%
\pgfpathclose%
\pgfpathmoveto{\pgfqpoint{1.080657in}{1.439894in}}%
\pgfpathlineto{\pgfqpoint{1.077898in}{1.452820in}}%
\pgfpathlineto{\pgfqpoint{1.082647in}{1.465747in}}%
\pgfpathlineto{\pgfqpoint{1.087625in}{1.469898in}}%
\pgfpathlineto{\pgfqpoint{1.100894in}{1.466494in}}%
\pgfpathlineto{\pgfqpoint{1.101429in}{1.465747in}}%
\pgfpathlineto{\pgfqpoint{1.104645in}{1.452820in}}%
\pgfpathlineto{\pgfqpoint{1.102776in}{1.439894in}}%
\pgfpathlineto{\pgfqpoint{1.100894in}{1.436672in}}%
\pgfpathlineto{\pgfqpoint{1.087625in}{1.432736in}}%
\pgfpathclose%
\pgfusepath{fill}%
\end{pgfscope}%
\begin{pgfscope}%
\pgfpathrectangle{\pgfqpoint{0.211875in}{0.211875in}}{\pgfqpoint{1.313625in}{1.279725in}}%
\pgfusepath{clip}%
\pgfsetbuttcap%
\pgfsetroundjoin%
\definecolor{currentfill}{rgb}{0.796501,0.105066,0.310630}%
\pgfsetfillcolor{currentfill}%
\pgfsetlinewidth{0.000000pt}%
\definecolor{currentstroke}{rgb}{0.000000,0.000000,0.000000}%
\pgfsetstrokecolor{currentstroke}%
\pgfsetdash{}{0pt}%
\pgfpathmoveto{\pgfqpoint{1.207045in}{1.422214in}}%
\pgfpathlineto{\pgfqpoint{1.220314in}{1.424608in}}%
\pgfpathlineto{\pgfqpoint{1.223064in}{1.426967in}}%
\pgfpathlineto{\pgfqpoint{1.228914in}{1.439894in}}%
\pgfpathlineto{\pgfqpoint{1.230001in}{1.452820in}}%
\pgfpathlineto{\pgfqpoint{1.228322in}{1.465747in}}%
\pgfpathlineto{\pgfqpoint{1.220314in}{1.478223in}}%
\pgfpathlineto{\pgfqpoint{1.218562in}{1.478673in}}%
\pgfpathlineto{\pgfqpoint{1.207045in}{1.480528in}}%
\pgfpathlineto{\pgfqpoint{1.201015in}{1.478673in}}%
\pgfpathlineto{\pgfqpoint{1.193777in}{1.474347in}}%
\pgfpathlineto{\pgfqpoint{1.189792in}{1.465747in}}%
\pgfpathlineto{\pgfqpoint{1.188303in}{1.452820in}}%
\pgfpathlineto{\pgfqpoint{1.189266in}{1.439894in}}%
\pgfpathlineto{\pgfqpoint{1.193777in}{1.428104in}}%
\pgfpathlineto{\pgfqpoint{1.195124in}{1.426967in}}%
\pgfpathclose%
\pgfpathmoveto{\pgfqpoint{1.198548in}{1.439894in}}%
\pgfpathlineto{\pgfqpoint{1.195370in}{1.452820in}}%
\pgfpathlineto{\pgfqpoint{1.200839in}{1.465747in}}%
\pgfpathlineto{\pgfqpoint{1.207045in}{1.470200in}}%
\pgfpathlineto{\pgfqpoint{1.218786in}{1.465747in}}%
\pgfpathlineto{\pgfqpoint{1.220314in}{1.464097in}}%
\pgfpathlineto{\pgfqpoint{1.222891in}{1.452820in}}%
\pgfpathlineto{\pgfqpoint{1.221133in}{1.439894in}}%
\pgfpathlineto{\pgfqpoint{1.220314in}{1.438333in}}%
\pgfpathlineto{\pgfqpoint{1.207045in}{1.432384in}}%
\pgfpathclose%
\pgfusepath{fill}%
\end{pgfscope}%
\begin{pgfscope}%
\pgfpathrectangle{\pgfqpoint{0.211875in}{0.211875in}}{\pgfqpoint{1.313625in}{1.279725in}}%
\pgfusepath{clip}%
\pgfsetbuttcap%
\pgfsetroundjoin%
\definecolor{currentfill}{rgb}{0.796501,0.105066,0.310630}%
\pgfsetfillcolor{currentfill}%
\pgfsetlinewidth{0.000000pt}%
\definecolor{currentstroke}{rgb}{0.000000,0.000000,0.000000}%
\pgfsetstrokecolor{currentstroke}%
\pgfsetdash{}{0pt}%
\pgfpathmoveto{\pgfqpoint{1.313197in}{1.425746in}}%
\pgfpathlineto{\pgfqpoint{1.326466in}{1.421591in}}%
\pgfpathlineto{\pgfqpoint{1.339735in}{1.424707in}}%
\pgfpathlineto{\pgfqpoint{1.342105in}{1.426967in}}%
\pgfpathlineto{\pgfqpoint{1.347393in}{1.439894in}}%
\pgfpathlineto{\pgfqpoint{1.348378in}{1.452820in}}%
\pgfpathlineto{\pgfqpoint{1.346879in}{1.465747in}}%
\pgfpathlineto{\pgfqpoint{1.339735in}{1.478163in}}%
\pgfpathlineto{\pgfqpoint{1.338296in}{1.478673in}}%
\pgfpathlineto{\pgfqpoint{1.326466in}{1.481132in}}%
\pgfpathlineto{\pgfqpoint{1.317036in}{1.478673in}}%
\pgfpathlineto{\pgfqpoint{1.313197in}{1.476870in}}%
\pgfpathlineto{\pgfqpoint{1.307416in}{1.465747in}}%
\pgfpathlineto{\pgfqpoint{1.305943in}{1.452820in}}%
\pgfpathlineto{\pgfqpoint{1.306905in}{1.439894in}}%
\pgfpathlineto{\pgfqpoint{1.312043in}{1.426967in}}%
\pgfpathclose%
\pgfpathmoveto{\pgfqpoint{1.315028in}{1.439894in}}%
\pgfpathlineto{\pgfqpoint{1.313197in}{1.446039in}}%
\pgfpathlineto{\pgfqpoint{1.312410in}{1.452820in}}%
\pgfpathlineto{\pgfqpoint{1.313197in}{1.457085in}}%
\pgfpathlineto{\pgfqpoint{1.317653in}{1.465747in}}%
\pgfpathlineto{\pgfqpoint{1.326466in}{1.471064in}}%
\pgfpathlineto{\pgfqpoint{1.337241in}{1.465747in}}%
\pgfpathlineto{\pgfqpoint{1.339735in}{1.461997in}}%
\pgfpathlineto{\pgfqpoint{1.341610in}{1.452820in}}%
\pgfpathlineto{\pgfqpoint{1.339983in}{1.439894in}}%
\pgfpathlineto{\pgfqpoint{1.339735in}{1.439367in}}%
\pgfpathlineto{\pgfqpoint{1.326466in}{1.431400in}}%
\pgfpathclose%
\pgfusepath{fill}%
\end{pgfscope}%
\begin{pgfscope}%
\pgfpathrectangle{\pgfqpoint{0.211875in}{0.211875in}}{\pgfqpoint{1.313625in}{1.279725in}}%
\pgfusepath{clip}%
\pgfsetbuttcap%
\pgfsetroundjoin%
\definecolor{currentfill}{rgb}{0.796501,0.105066,0.310630}%
\pgfsetfillcolor{currentfill}%
\pgfsetlinewidth{0.000000pt}%
\definecolor{currentstroke}{rgb}{0.000000,0.000000,0.000000}%
\pgfsetstrokecolor{currentstroke}%
\pgfsetdash{}{0pt}%
\pgfpathmoveto{\pgfqpoint{1.432617in}{1.423397in}}%
\pgfpathlineto{\pgfqpoint{1.445886in}{1.420479in}}%
\pgfpathlineto{\pgfqpoint{1.459155in}{1.424146in}}%
\pgfpathlineto{\pgfqpoint{1.461813in}{1.426967in}}%
\pgfpathlineto{\pgfqpoint{1.466371in}{1.439894in}}%
\pgfpathlineto{\pgfqpoint{1.467229in}{1.452820in}}%
\pgfpathlineto{\pgfqpoint{1.465966in}{1.465747in}}%
\pgfpathlineto{\pgfqpoint{1.459442in}{1.478673in}}%
\pgfpathlineto{\pgfqpoint{1.459155in}{1.478909in}}%
\pgfpathlineto{\pgfqpoint{1.445886in}{1.482219in}}%
\pgfpathlineto{\pgfqpoint{1.432617in}{1.479564in}}%
\pgfpathlineto{\pgfqpoint{1.431416in}{1.478673in}}%
\pgfpathlineto{\pgfqpoint{1.424429in}{1.465747in}}%
\pgfpathlineto{\pgfqpoint{1.423045in}{1.452820in}}%
\pgfpathlineto{\pgfqpoint{1.423973in}{1.439894in}}%
\pgfpathlineto{\pgfqpoint{1.428869in}{1.426967in}}%
\pgfpathclose%
\pgfpathmoveto{\pgfqpoint{1.431345in}{1.439894in}}%
\pgfpathlineto{\pgfqpoint{1.429789in}{1.452820in}}%
\pgfpathlineto{\pgfqpoint{1.432433in}{1.465747in}}%
\pgfpathlineto{\pgfqpoint{1.432617in}{1.466064in}}%
\pgfpathlineto{\pgfqpoint{1.445886in}{1.472495in}}%
\pgfpathlineto{\pgfqpoint{1.457005in}{1.465747in}}%
\pgfpathlineto{\pgfqpoint{1.459155in}{1.461466in}}%
\pgfpathlineto{\pgfqpoint{1.460730in}{1.452820in}}%
\pgfpathlineto{\pgfqpoint{1.459254in}{1.439894in}}%
\pgfpathlineto{\pgfqpoint{1.459155in}{1.439660in}}%
\pgfpathlineto{\pgfqpoint{1.445886in}{1.429777in}}%
\pgfpathlineto{\pgfqpoint{1.432617in}{1.437222in}}%
\pgfpathclose%
\pgfusepath{fill}%
\end{pgfscope}%
\begin{pgfscope}%
\pgfpathrectangle{\pgfqpoint{0.211875in}{0.211875in}}{\pgfqpoint{1.313625in}{1.279725in}}%
\pgfusepath{clip}%
\pgfsetbuttcap%
\pgfsetroundjoin%
\definecolor{currentfill}{rgb}{0.901975,0.231521,0.249182}%
\pgfsetfillcolor{currentfill}%
\pgfsetlinewidth{0.000000pt}%
\definecolor{currentstroke}{rgb}{0.000000,0.000000,0.000000}%
\pgfsetstrokecolor{currentstroke}%
\pgfsetdash{}{0pt}%
\pgfpathmoveto{\pgfqpoint{0.251682in}{0.223811in}}%
\pgfpathlineto{\pgfqpoint{0.263907in}{0.211875in}}%
\pgfpathlineto{\pgfqpoint{0.264951in}{0.211875in}}%
\pgfpathlineto{\pgfqpoint{0.266615in}{0.211875in}}%
\pgfpathlineto{\pgfqpoint{0.278220in}{0.218465in}}%
\pgfpathlineto{\pgfqpoint{0.280859in}{0.224802in}}%
\pgfpathlineto{\pgfqpoint{0.281648in}{0.237728in}}%
\pgfpathlineto{\pgfqpoint{0.278220in}{0.249870in}}%
\pgfpathlineto{\pgfqpoint{0.277351in}{0.250655in}}%
\pgfpathlineto{\pgfqpoint{0.264951in}{0.255118in}}%
\pgfpathlineto{\pgfqpoint{0.257036in}{0.250655in}}%
\pgfpathlineto{\pgfqpoint{0.251682in}{0.242357in}}%
\pgfpathlineto{\pgfqpoint{0.250613in}{0.237728in}}%
\pgfpathlineto{\pgfqpoint{0.251339in}{0.224802in}}%
\pgfpathclose%
\pgfpathmoveto{\pgfqpoint{0.263440in}{0.237728in}}%
\pgfpathlineto{\pgfqpoint{0.264951in}{0.239813in}}%
\pgfpathlineto{\pgfqpoint{0.267220in}{0.237728in}}%
\pgfpathlineto{\pgfqpoint{0.264951in}{0.227600in}}%
\pgfpathclose%
\pgfusepath{fill}%
\end{pgfscope}%
\begin{pgfscope}%
\pgfpathrectangle{\pgfqpoint{0.211875in}{0.211875in}}{\pgfqpoint{1.313625in}{1.279725in}}%
\pgfusepath{clip}%
\pgfsetbuttcap%
\pgfsetroundjoin%
\definecolor{currentfill}{rgb}{0.901975,0.231521,0.249182}%
\pgfsetfillcolor{currentfill}%
\pgfsetlinewidth{0.000000pt}%
\definecolor{currentstroke}{rgb}{0.000000,0.000000,0.000000}%
\pgfsetstrokecolor{currentstroke}%
\pgfsetdash{}{0pt}%
\pgfpathmoveto{\pgfqpoint{0.371102in}{0.212495in}}%
\pgfpathlineto{\pgfqpoint{0.371936in}{0.211875in}}%
\pgfpathlineto{\pgfqpoint{0.384371in}{0.211875in}}%
\pgfpathlineto{\pgfqpoint{0.396752in}{0.211875in}}%
\pgfpathlineto{\pgfqpoint{0.397640in}{0.212573in}}%
\pgfpathlineto{\pgfqpoint{0.402188in}{0.224802in}}%
\pgfpathlineto{\pgfqpoint{0.402860in}{0.237728in}}%
\pgfpathlineto{\pgfqpoint{0.400229in}{0.250655in}}%
\pgfpathlineto{\pgfqpoint{0.397640in}{0.254743in}}%
\pgfpathlineto{\pgfqpoint{0.384371in}{0.259818in}}%
\pgfpathlineto{\pgfqpoint{0.371102in}{0.254660in}}%
\pgfpathlineto{\pgfqpoint{0.368481in}{0.250655in}}%
\pgfpathlineto{\pgfqpoint{0.365664in}{0.237728in}}%
\pgfpathlineto{\pgfqpoint{0.366349in}{0.224802in}}%
\pgfpathclose%
\pgfpathmoveto{\pgfqpoint{0.376548in}{0.224802in}}%
\pgfpathlineto{\pgfqpoint{0.374346in}{0.237728in}}%
\pgfpathlineto{\pgfqpoint{0.384371in}{0.248943in}}%
\pgfpathlineto{\pgfqpoint{0.393870in}{0.237728in}}%
\pgfpathlineto{\pgfqpoint{0.391807in}{0.224802in}}%
\pgfpathlineto{\pgfqpoint{0.384371in}{0.218614in}}%
\pgfpathclose%
\pgfusepath{fill}%
\end{pgfscope}%
\begin{pgfscope}%
\pgfpathrectangle{\pgfqpoint{0.211875in}{0.211875in}}{\pgfqpoint{1.313625in}{1.279725in}}%
\pgfusepath{clip}%
\pgfsetbuttcap%
\pgfsetroundjoin%
\definecolor{currentfill}{rgb}{0.901975,0.231521,0.249182}%
\pgfsetfillcolor{currentfill}%
\pgfsetlinewidth{0.000000pt}%
\definecolor{currentstroke}{rgb}{0.000000,0.000000,0.000000}%
\pgfsetstrokecolor{currentstroke}%
\pgfsetdash{}{0pt}%
\pgfpathmoveto{\pgfqpoint{0.490523in}{0.211875in}}%
\pgfpathlineto{\pgfqpoint{0.503792in}{0.211875in}}%
\pgfpathlineto{\pgfqpoint{0.517061in}{0.211875in}}%
\pgfpathlineto{\pgfqpoint{0.519101in}{0.211875in}}%
\pgfpathlineto{\pgfqpoint{0.522825in}{0.224802in}}%
\pgfpathlineto{\pgfqpoint{0.523407in}{0.237728in}}%
\pgfpathlineto{\pgfqpoint{0.521611in}{0.250655in}}%
\pgfpathlineto{\pgfqpoint{0.517061in}{0.258824in}}%
\pgfpathlineto{\pgfqpoint{0.504483in}{0.263581in}}%
\pgfpathlineto{\pgfqpoint{0.503792in}{0.263724in}}%
\pgfpathlineto{\pgfqpoint{0.502826in}{0.263581in}}%
\pgfpathlineto{\pgfqpoint{0.490523in}{0.260871in}}%
\pgfpathlineto{\pgfqpoint{0.482993in}{0.250655in}}%
\pgfpathlineto{\pgfqpoint{0.480800in}{0.237728in}}%
\pgfpathlineto{\pgfqpoint{0.481455in}{0.224802in}}%
\pgfpathlineto{\pgfqpoint{0.485782in}{0.211875in}}%
\pgfpathclose%
\pgfpathmoveto{\pgfqpoint{0.488693in}{0.224802in}}%
\pgfpathlineto{\pgfqpoint{0.487866in}{0.237728in}}%
\pgfpathlineto{\pgfqpoint{0.490523in}{0.246723in}}%
\pgfpathlineto{\pgfqpoint{0.495255in}{0.250655in}}%
\pgfpathlineto{\pgfqpoint{0.503792in}{0.253650in}}%
\pgfpathlineto{\pgfqpoint{0.509017in}{0.250655in}}%
\pgfpathlineto{\pgfqpoint{0.516768in}{0.237728in}}%
\pgfpathlineto{\pgfqpoint{0.515207in}{0.224802in}}%
\pgfpathlineto{\pgfqpoint{0.503792in}{0.213188in}}%
\pgfpathlineto{\pgfqpoint{0.490523in}{0.220558in}}%
\pgfpathclose%
\pgfusepath{fill}%
\end{pgfscope}%
\begin{pgfscope}%
\pgfpathrectangle{\pgfqpoint{0.211875in}{0.211875in}}{\pgfqpoint{1.313625in}{1.279725in}}%
\pgfusepath{clip}%
\pgfsetbuttcap%
\pgfsetroundjoin%
\definecolor{currentfill}{rgb}{0.901975,0.231521,0.249182}%
\pgfsetfillcolor{currentfill}%
\pgfsetlinewidth{0.000000pt}%
\definecolor{currentstroke}{rgb}{0.000000,0.000000,0.000000}%
\pgfsetstrokecolor{currentstroke}%
\pgfsetdash{}{0pt}%
\pgfpathmoveto{\pgfqpoint{0.596674in}{0.224432in}}%
\pgfpathlineto{\pgfqpoint{0.600279in}{0.211875in}}%
\pgfpathlineto{\pgfqpoint{0.609943in}{0.211875in}}%
\pgfpathlineto{\pgfqpoint{0.613598in}{0.211875in}}%
\pgfpathlineto{\pgfqpoint{0.609943in}{0.213156in}}%
\pgfpathlineto{\pgfqpoint{0.604326in}{0.224802in}}%
\pgfpathlineto{\pgfqpoint{0.603505in}{0.237728in}}%
\pgfpathlineto{\pgfqpoint{0.607224in}{0.250655in}}%
\pgfpathlineto{\pgfqpoint{0.609943in}{0.253931in}}%
\pgfpathlineto{\pgfqpoint{0.623212in}{0.256764in}}%
\pgfpathlineto{\pgfqpoint{0.632169in}{0.250655in}}%
\pgfpathlineto{\pgfqpoint{0.636481in}{0.241149in}}%
\pgfpathlineto{\pgfqpoint{0.637129in}{0.237728in}}%
\pgfpathlineto{\pgfqpoint{0.636481in}{0.225199in}}%
\pgfpathlineto{\pgfqpoint{0.636442in}{0.224802in}}%
\pgfpathlineto{\pgfqpoint{0.626676in}{0.211875in}}%
\pgfpathlineto{\pgfqpoint{0.636481in}{0.211875in}}%
\pgfpathlineto{\pgfqpoint{0.640061in}{0.211875in}}%
\pgfpathlineto{\pgfqpoint{0.642886in}{0.224802in}}%
\pgfpathlineto{\pgfqpoint{0.643397in}{0.237728in}}%
\pgfpathlineto{\pgfqpoint{0.642238in}{0.250655in}}%
\pgfpathlineto{\pgfqpoint{0.636481in}{0.262488in}}%
\pgfpathlineto{\pgfqpoint{0.634297in}{0.263581in}}%
\pgfpathlineto{\pgfqpoint{0.623212in}{0.266365in}}%
\pgfpathlineto{\pgfqpoint{0.609943in}{0.265166in}}%
\pgfpathlineto{\pgfqpoint{0.606606in}{0.263581in}}%
\pgfpathlineto{\pgfqpoint{0.597594in}{0.250655in}}%
\pgfpathlineto{\pgfqpoint{0.596674in}{0.244168in}}%
\pgfpathlineto{\pgfqpoint{0.596126in}{0.237728in}}%
\pgfpathlineto{\pgfqpoint{0.596615in}{0.224802in}}%
\pgfpathclose%
\pgfusepath{fill}%
\end{pgfscope}%
\begin{pgfscope}%
\pgfpathrectangle{\pgfqpoint{0.211875in}{0.211875in}}{\pgfqpoint{1.313625in}{1.279725in}}%
\pgfusepath{clip}%
\pgfsetbuttcap%
\pgfsetroundjoin%
\definecolor{currentfill}{rgb}{0.901975,0.231521,0.249182}%
\pgfsetfillcolor{currentfill}%
\pgfsetlinewidth{0.000000pt}%
\definecolor{currentstroke}{rgb}{0.000000,0.000000,0.000000}%
\pgfsetstrokecolor{currentstroke}%
\pgfsetdash{}{0pt}%
\pgfpathmoveto{\pgfqpoint{0.716095in}{0.211875in}}%
\pgfpathlineto{\pgfqpoint{0.726217in}{0.211875in}}%
\pgfpathlineto{\pgfqpoint{0.720051in}{0.224802in}}%
\pgfpathlineto{\pgfqpoint{0.719219in}{0.237728in}}%
\pgfpathlineto{\pgfqpoint{0.722624in}{0.250655in}}%
\pgfpathlineto{\pgfqpoint{0.729364in}{0.257846in}}%
\pgfpathlineto{\pgfqpoint{0.742633in}{0.259151in}}%
\pgfpathlineto{\pgfqpoint{0.753324in}{0.250655in}}%
\pgfpathlineto{\pgfqpoint{0.755902in}{0.243282in}}%
\pgfpathlineto{\pgfqpoint{0.756803in}{0.237728in}}%
\pgfpathlineto{\pgfqpoint{0.756191in}{0.224802in}}%
\pgfpathlineto{\pgfqpoint{0.755902in}{0.223688in}}%
\pgfpathlineto{\pgfqpoint{0.748842in}{0.211875in}}%
\pgfpathlineto{\pgfqpoint{0.755902in}{0.211875in}}%
\pgfpathlineto{\pgfqpoint{0.760288in}{0.211875in}}%
\pgfpathlineto{\pgfqpoint{0.762447in}{0.224802in}}%
\pgfpathlineto{\pgfqpoint{0.762907in}{0.237728in}}%
\pgfpathlineto{\pgfqpoint{0.762217in}{0.250655in}}%
\pgfpathlineto{\pgfqpoint{0.757807in}{0.263581in}}%
\pgfpathlineto{\pgfqpoint{0.755902in}{0.265283in}}%
\pgfpathlineto{\pgfqpoint{0.742633in}{0.268474in}}%
\pgfpathlineto{\pgfqpoint{0.729364in}{0.267950in}}%
\pgfpathlineto{\pgfqpoint{0.718939in}{0.263581in}}%
\pgfpathlineto{\pgfqpoint{0.716095in}{0.260201in}}%
\pgfpathlineto{\pgfqpoint{0.713323in}{0.250655in}}%
\pgfpathlineto{\pgfqpoint{0.712508in}{0.237728in}}%
\pgfpathlineto{\pgfqpoint{0.712960in}{0.224802in}}%
\pgfpathlineto{\pgfqpoint{0.715209in}{0.211875in}}%
\pgfpathclose%
\pgfusepath{fill}%
\end{pgfscope}%
\begin{pgfscope}%
\pgfpathrectangle{\pgfqpoint{0.211875in}{0.211875in}}{\pgfqpoint{1.313625in}{1.279725in}}%
\pgfusepath{clip}%
\pgfsetbuttcap%
\pgfsetroundjoin%
\definecolor{currentfill}{rgb}{0.901975,0.231521,0.249182}%
\pgfsetfillcolor{currentfill}%
\pgfsetlinewidth{0.000000pt}%
\definecolor{currentstroke}{rgb}{0.000000,0.000000,0.000000}%
\pgfsetstrokecolor{currentstroke}%
\pgfsetdash{}{0pt}%
\pgfpathmoveto{\pgfqpoint{0.835515in}{0.211875in}}%
\pgfpathlineto{\pgfqpoint{0.841899in}{0.211875in}}%
\pgfpathlineto{\pgfqpoint{0.835825in}{0.224802in}}%
\pgfpathlineto{\pgfqpoint{0.835515in}{0.229345in}}%
\pgfpathlineto{\pgfqpoint{0.835148in}{0.237728in}}%
\pgfpathlineto{\pgfqpoint{0.835515in}{0.240458in}}%
\pgfpathlineto{\pgfqpoint{0.838175in}{0.250655in}}%
\pgfpathlineto{\pgfqpoint{0.848784in}{0.260629in}}%
\pgfpathlineto{\pgfqpoint{0.862053in}{0.260818in}}%
\pgfpathlineto{\pgfqpoint{0.873186in}{0.250655in}}%
\pgfpathlineto{\pgfqpoint{0.875322in}{0.242683in}}%
\pgfpathlineto{\pgfqpoint{0.875998in}{0.237728in}}%
\pgfpathlineto{\pgfqpoint{0.875425in}{0.224802in}}%
\pgfpathlineto{\pgfqpoint{0.875322in}{0.224348in}}%
\pgfpathlineto{\pgfqpoint{0.869391in}{0.211875in}}%
\pgfpathlineto{\pgfqpoint{0.875322in}{0.211875in}}%
\pgfpathlineto{\pgfqpoint{0.879865in}{0.211875in}}%
\pgfpathlineto{\pgfqpoint{0.881559in}{0.224802in}}%
\pgfpathlineto{\pgfqpoint{0.881983in}{0.237728in}}%
\pgfpathlineto{\pgfqpoint{0.881621in}{0.250655in}}%
\pgfpathlineto{\pgfqpoint{0.878938in}{0.263581in}}%
\pgfpathlineto{\pgfqpoint{0.875322in}{0.267396in}}%
\pgfpathlineto{\pgfqpoint{0.862053in}{0.270064in}}%
\pgfpathlineto{\pgfqpoint{0.848784in}{0.269905in}}%
\pgfpathlineto{\pgfqpoint{0.835515in}{0.266453in}}%
\pgfpathlineto{\pgfqpoint{0.832903in}{0.263581in}}%
\pgfpathlineto{\pgfqpoint{0.829736in}{0.250655in}}%
\pgfpathlineto{\pgfqpoint{0.829259in}{0.237728in}}%
\pgfpathlineto{\pgfqpoint{0.829686in}{0.224802in}}%
\pgfpathlineto{\pgfqpoint{0.831502in}{0.211875in}}%
\pgfpathclose%
\pgfusepath{fill}%
\end{pgfscope}%
\begin{pgfscope}%
\pgfpathrectangle{\pgfqpoint{0.211875in}{0.211875in}}{\pgfqpoint{1.313625in}{1.279725in}}%
\pgfusepath{clip}%
\pgfsetbuttcap%
\pgfsetroundjoin%
\definecolor{currentfill}{rgb}{0.901975,0.231521,0.249182}%
\pgfsetfillcolor{currentfill}%
\pgfsetlinewidth{0.000000pt}%
\definecolor{currentstroke}{rgb}{0.000000,0.000000,0.000000}%
\pgfsetstrokecolor{currentstroke}%
\pgfsetdash{}{0pt}%
\pgfpathmoveto{\pgfqpoint{0.954936in}{0.211875in}}%
\pgfpathlineto{\pgfqpoint{0.957836in}{0.211875in}}%
\pgfpathlineto{\pgfqpoint{0.954936in}{0.216900in}}%
\pgfpathlineto{\pgfqpoint{0.952914in}{0.224802in}}%
\pgfpathlineto{\pgfqpoint{0.952352in}{0.237728in}}%
\pgfpathlineto{\pgfqpoint{0.954280in}{0.250655in}}%
\pgfpathlineto{\pgfqpoint{0.954936in}{0.252245in}}%
\pgfpathlineto{\pgfqpoint{0.968205in}{0.262445in}}%
\pgfpathlineto{\pgfqpoint{0.981473in}{0.261742in}}%
\pgfpathlineto{\pgfqpoint{0.992146in}{0.250655in}}%
\pgfpathlineto{\pgfqpoint{0.994742in}{0.237738in}}%
\pgfpathlineto{\pgfqpoint{0.994744in}{0.237728in}}%
\pgfpathlineto{\pgfqpoint{0.994742in}{0.237700in}}%
\pgfpathlineto{\pgfqpoint{0.993972in}{0.224802in}}%
\pgfpathlineto{\pgfqpoint{0.988816in}{0.211875in}}%
\pgfpathlineto{\pgfqpoint{0.994742in}{0.211875in}}%
\pgfpathlineto{\pgfqpoint{0.998845in}{0.211875in}}%
\pgfpathlineto{\pgfqpoint{1.000252in}{0.224802in}}%
\pgfpathlineto{\pgfqpoint{1.000653in}{0.237728in}}%
\pgfpathlineto{\pgfqpoint{1.000492in}{0.250655in}}%
\pgfpathlineto{\pgfqpoint{0.998864in}{0.263581in}}%
\pgfpathlineto{\pgfqpoint{0.994742in}{0.268843in}}%
\pgfpathlineto{\pgfqpoint{0.981473in}{0.271125in}}%
\pgfpathlineto{\pgfqpoint{0.968205in}{0.271149in}}%
\pgfpathlineto{\pgfqpoint{0.954936in}{0.269488in}}%
\pgfpathlineto{\pgfqpoint{0.948599in}{0.263581in}}%
\pgfpathlineto{\pgfqpoint{0.946588in}{0.250655in}}%
\pgfpathlineto{\pgfqpoint{0.946349in}{0.237728in}}%
\pgfpathlineto{\pgfqpoint{0.946763in}{0.224802in}}%
\pgfpathlineto{\pgfqpoint{0.948288in}{0.211875in}}%
\pgfpathclose%
\pgfusepath{fill}%
\end{pgfscope}%
\begin{pgfscope}%
\pgfpathrectangle{\pgfqpoint{0.211875in}{0.211875in}}{\pgfqpoint{1.313625in}{1.279725in}}%
\pgfusepath{clip}%
\pgfsetbuttcap%
\pgfsetroundjoin%
\definecolor{currentfill}{rgb}{0.901975,0.231521,0.249182}%
\pgfsetfillcolor{currentfill}%
\pgfsetlinewidth{0.000000pt}%
\definecolor{currentstroke}{rgb}{0.000000,0.000000,0.000000}%
\pgfsetstrokecolor{currentstroke}%
\pgfsetdash{}{0pt}%
\pgfpathmoveto{\pgfqpoint{1.070490in}{0.224802in}}%
\pgfpathlineto{\pgfqpoint{1.069925in}{0.237728in}}%
\pgfpathlineto{\pgfqpoint{1.071771in}{0.250655in}}%
\pgfpathlineto{\pgfqpoint{1.074356in}{0.256072in}}%
\pgfpathlineto{\pgfqpoint{1.087625in}{0.263402in}}%
\pgfpathlineto{\pgfqpoint{1.100894in}{0.261862in}}%
\pgfpathlineto{\pgfqpoint{1.110429in}{0.250655in}}%
\pgfpathlineto{\pgfqpoint{1.112710in}{0.237728in}}%
\pgfpathlineto{\pgfqpoint{1.112007in}{0.224802in}}%
\pgfpathlineto{\pgfqpoint{1.107408in}{0.211875in}}%
\pgfpathlineto{\pgfqpoint{1.114163in}{0.211875in}}%
\pgfpathlineto{\pgfqpoint{1.117253in}{0.211875in}}%
\pgfpathlineto{\pgfqpoint{1.118536in}{0.224802in}}%
\pgfpathlineto{\pgfqpoint{1.118927in}{0.237728in}}%
\pgfpathlineto{\pgfqpoint{1.118852in}{0.250655in}}%
\pgfpathlineto{\pgfqpoint{1.117671in}{0.263581in}}%
\pgfpathlineto{\pgfqpoint{1.114163in}{0.269210in}}%
\pgfpathlineto{\pgfqpoint{1.100894in}{0.271621in}}%
\pgfpathlineto{\pgfqpoint{1.087625in}{0.271762in}}%
\pgfpathlineto{\pgfqpoint{1.074356in}{0.270909in}}%
\pgfpathlineto{\pgfqpoint{1.065235in}{0.263581in}}%
\pgfpathlineto{\pgfqpoint{1.063867in}{0.250655in}}%
\pgfpathlineto{\pgfqpoint{1.063761in}{0.237728in}}%
\pgfpathlineto{\pgfqpoint{1.064174in}{0.224802in}}%
\pgfpathlineto{\pgfqpoint{1.065555in}{0.211875in}}%
\pgfpathlineto{\pgfqpoint{1.074194in}{0.211875in}}%
\pgfpathclose%
\pgfusepath{fill}%
\end{pgfscope}%
\begin{pgfscope}%
\pgfpathrectangle{\pgfqpoint{0.211875in}{0.211875in}}{\pgfqpoint{1.313625in}{1.279725in}}%
\pgfusepath{clip}%
\pgfsetbuttcap%
\pgfsetroundjoin%
\definecolor{currentfill}{rgb}{0.901975,0.231521,0.249182}%
\pgfsetfillcolor{currentfill}%
\pgfsetlinewidth{0.000000pt}%
\definecolor{currentstroke}{rgb}{0.000000,0.000000,0.000000}%
\pgfsetstrokecolor{currentstroke}%
\pgfsetdash{}{0pt}%
\pgfpathmoveto{\pgfqpoint{1.188451in}{0.224802in}}%
\pgfpathlineto{\pgfqpoint{1.187868in}{0.237728in}}%
\pgfpathlineto{\pgfqpoint{1.189758in}{0.250655in}}%
\pgfpathlineto{\pgfqpoint{1.193777in}{0.258018in}}%
\pgfpathlineto{\pgfqpoint{1.207045in}{0.263568in}}%
\pgfpathlineto{\pgfqpoint{1.220314in}{0.261073in}}%
\pgfpathlineto{\pgfqpoint{1.228171in}{0.250655in}}%
\pgfpathlineto{\pgfqpoint{1.230326in}{0.237728in}}%
\pgfpathlineto{\pgfqpoint{1.229666in}{0.224802in}}%
\pgfpathlineto{\pgfqpoint{1.225341in}{0.211875in}}%
\pgfpathlineto{\pgfqpoint{1.233583in}{0.211875in}}%
\pgfpathlineto{\pgfqpoint{1.235091in}{0.211875in}}%
\pgfpathlineto{\pgfqpoint{1.236407in}{0.224802in}}%
\pgfpathlineto{\pgfqpoint{1.236800in}{0.237728in}}%
\pgfpathlineto{\pgfqpoint{1.236700in}{0.250655in}}%
\pgfpathlineto{\pgfqpoint{1.235395in}{0.263581in}}%
\pgfpathlineto{\pgfqpoint{1.233583in}{0.267474in}}%
\pgfpathlineto{\pgfqpoint{1.220314in}{0.271485in}}%
\pgfpathlineto{\pgfqpoint{1.207045in}{0.271791in}}%
\pgfpathlineto{\pgfqpoint{1.193777in}{0.271222in}}%
\pgfpathlineto{\pgfqpoint{1.182857in}{0.263581in}}%
\pgfpathlineto{\pgfqpoint{1.181575in}{0.250655in}}%
\pgfpathlineto{\pgfqpoint{1.181493in}{0.237728in}}%
\pgfpathlineto{\pgfqpoint{1.181918in}{0.224802in}}%
\pgfpathlineto{\pgfqpoint{1.183311in}{0.211875in}}%
\pgfpathlineto{\pgfqpoint{1.192259in}{0.211875in}}%
\pgfpathclose%
\pgfusepath{fill}%
\end{pgfscope}%
\begin{pgfscope}%
\pgfpathrectangle{\pgfqpoint{0.211875in}{0.211875in}}{\pgfqpoint{1.313625in}{1.279725in}}%
\pgfusepath{clip}%
\pgfsetbuttcap%
\pgfsetroundjoin%
\definecolor{currentfill}{rgb}{0.901975,0.231521,0.249182}%
\pgfsetfillcolor{currentfill}%
\pgfsetlinewidth{0.000000pt}%
\definecolor{currentstroke}{rgb}{0.000000,0.000000,0.000000}%
\pgfsetstrokecolor{currentstroke}%
\pgfsetdash{}{0pt}%
\pgfpathmoveto{\pgfqpoint{1.306817in}{0.224802in}}%
\pgfpathlineto{\pgfqpoint{1.306201in}{0.237728in}}%
\pgfpathlineto{\pgfqpoint{1.308274in}{0.250655in}}%
\pgfpathlineto{\pgfqpoint{1.313197in}{0.258598in}}%
\pgfpathlineto{\pgfqpoint{1.326466in}{0.262975in}}%
\pgfpathlineto{\pgfqpoint{1.339735in}{0.259207in}}%
\pgfpathlineto{\pgfqpoint{1.345453in}{0.250655in}}%
\pgfpathlineto{\pgfqpoint{1.347635in}{0.237728in}}%
\pgfpathlineto{\pgfqpoint{1.346999in}{0.224802in}}%
\pgfpathlineto{\pgfqpoint{1.342721in}{0.211875in}}%
\pgfpathlineto{\pgfqpoint{1.352241in}{0.211875in}}%
\pgfpathlineto{\pgfqpoint{1.353004in}{0.216615in}}%
\pgfpathlineto{\pgfqpoint{1.353843in}{0.224802in}}%
\pgfpathlineto{\pgfqpoint{1.354251in}{0.237728in}}%
\pgfpathlineto{\pgfqpoint{1.354014in}{0.250655in}}%
\pgfpathlineto{\pgfqpoint{1.353004in}{0.259522in}}%
\pgfpathlineto{\pgfqpoint{1.351887in}{0.263581in}}%
\pgfpathlineto{\pgfqpoint{1.339735in}{0.270608in}}%
\pgfpathlineto{\pgfqpoint{1.326466in}{0.271260in}}%
\pgfpathlineto{\pgfqpoint{1.313197in}{0.270716in}}%
\pgfpathlineto{\pgfqpoint{1.301567in}{0.263581in}}%
\pgfpathlineto{\pgfqpoint{1.299928in}{0.253254in}}%
\pgfpathlineto{\pgfqpoint{1.299752in}{0.250655in}}%
\pgfpathlineto{\pgfqpoint{1.299588in}{0.237728in}}%
\pgfpathlineto{\pgfqpoint{1.299928in}{0.226975in}}%
\pgfpathlineto{\pgfqpoint{1.300005in}{0.224802in}}%
\pgfpathlineto{\pgfqpoint{1.301588in}{0.211875in}}%
\pgfpathlineto{\pgfqpoint{1.310931in}{0.211875in}}%
\pgfpathclose%
\pgfusepath{fill}%
\end{pgfscope}%
\begin{pgfscope}%
\pgfpathrectangle{\pgfqpoint{0.211875in}{0.211875in}}{\pgfqpoint{1.313625in}{1.279725in}}%
\pgfusepath{clip}%
\pgfsetbuttcap%
\pgfsetroundjoin%
\definecolor{currentfill}{rgb}{0.901975,0.231521,0.249182}%
\pgfsetfillcolor{currentfill}%
\pgfsetlinewidth{0.000000pt}%
\definecolor{currentstroke}{rgb}{0.000000,0.000000,0.000000}%
\pgfsetstrokecolor{currentstroke}%
\pgfsetdash{}{0pt}%
\pgfpathmoveto{\pgfqpoint{1.419348in}{0.218018in}}%
\pgfpathlineto{\pgfqpoint{1.420445in}{0.211875in}}%
\pgfpathlineto{\pgfqpoint{1.430286in}{0.211875in}}%
\pgfpathlineto{\pgfqpoint{1.425630in}{0.224802in}}%
\pgfpathlineto{\pgfqpoint{1.424961in}{0.237728in}}%
\pgfpathlineto{\pgfqpoint{1.427385in}{0.250655in}}%
\pgfpathlineto{\pgfqpoint{1.432617in}{0.258122in}}%
\pgfpathlineto{\pgfqpoint{1.445886in}{0.261624in}}%
\pgfpathlineto{\pgfqpoint{1.459155in}{0.256001in}}%
\pgfpathlineto{\pgfqpoint{1.462318in}{0.250655in}}%
\pgfpathlineto{\pgfqpoint{1.464657in}{0.237728in}}%
\pgfpathlineto{\pgfqpoint{1.464027in}{0.224802in}}%
\pgfpathlineto{\pgfqpoint{1.459607in}{0.211875in}}%
\pgfpathlineto{\pgfqpoint{1.468673in}{0.211875in}}%
\pgfpathlineto{\pgfqpoint{1.470673in}{0.224802in}}%
\pgfpathlineto{\pgfqpoint{1.471144in}{0.237728in}}%
\pgfpathlineto{\pgfqpoint{1.470619in}{0.250655in}}%
\pgfpathlineto{\pgfqpoint{1.467134in}{0.263581in}}%
\pgfpathlineto{\pgfqpoint{1.459155in}{0.268817in}}%
\pgfpathlineto{\pgfqpoint{1.445886in}{0.270168in}}%
\pgfpathlineto{\pgfqpoint{1.432617in}{0.269565in}}%
\pgfpathlineto{\pgfqpoint{1.421530in}{0.263581in}}%
\pgfpathlineto{\pgfqpoint{1.419348in}{0.256914in}}%
\pgfpathlineto{\pgfqpoint{1.418533in}{0.250655in}}%
\pgfpathlineto{\pgfqpoint{1.418172in}{0.237728in}}%
\pgfpathlineto{\pgfqpoint{1.418595in}{0.224802in}}%
\pgfpathclose%
\pgfusepath{fill}%
\end{pgfscope}%
\begin{pgfscope}%
\pgfpathrectangle{\pgfqpoint{0.211875in}{0.211875in}}{\pgfqpoint{1.313625in}{1.279725in}}%
\pgfusepath{clip}%
\pgfsetbuttcap%
\pgfsetroundjoin%
\definecolor{currentfill}{rgb}{0.901975,0.231521,0.249182}%
\pgfsetfillcolor{currentfill}%
\pgfsetlinewidth{0.000000pt}%
\definecolor{currentstroke}{rgb}{0.000000,0.000000,0.000000}%
\pgfsetstrokecolor{currentstroke}%
\pgfsetdash{}{0pt}%
\pgfpathmoveto{\pgfqpoint{0.795708in}{0.276461in}}%
\pgfpathlineto{\pgfqpoint{0.796846in}{0.276508in}}%
\pgfpathlineto{\pgfqpoint{0.808977in}{0.277180in}}%
\pgfpathlineto{\pgfqpoint{0.822246in}{0.287583in}}%
\pgfpathlineto{\pgfqpoint{0.822752in}{0.289434in}}%
\pgfpathlineto{\pgfqpoint{0.824249in}{0.302361in}}%
\pgfpathlineto{\pgfqpoint{0.824726in}{0.315287in}}%
\pgfpathlineto{\pgfqpoint{0.824852in}{0.328214in}}%
\pgfpathlineto{\pgfqpoint{0.824595in}{0.341140in}}%
\pgfpathlineto{\pgfqpoint{0.822246in}{0.350758in}}%
\pgfpathlineto{\pgfqpoint{0.808977in}{0.353346in}}%
\pgfpathlineto{\pgfqpoint{0.795708in}{0.353385in}}%
\pgfpathlineto{\pgfqpoint{0.782439in}{0.352759in}}%
\pgfpathlineto{\pgfqpoint{0.769454in}{0.341140in}}%
\pgfpathlineto{\pgfqpoint{0.769170in}{0.337943in}}%
\pgfpathlineto{\pgfqpoint{0.768709in}{0.328214in}}%
\pgfpathlineto{\pgfqpoint{0.768737in}{0.315287in}}%
\pgfpathlineto{\pgfqpoint{0.769170in}{0.304546in}}%
\pgfpathlineto{\pgfqpoint{0.769276in}{0.302361in}}%
\pgfpathlineto{\pgfqpoint{0.771223in}{0.289434in}}%
\pgfpathlineto{\pgfqpoint{0.782439in}{0.277958in}}%
\pgfpathlineto{\pgfqpoint{0.795120in}{0.276508in}}%
\pgfpathclose%
\pgfpathmoveto{\pgfqpoint{0.781946in}{0.289434in}}%
\pgfpathlineto{\pgfqpoint{0.776316in}{0.302361in}}%
\pgfpathlineto{\pgfqpoint{0.775087in}{0.315287in}}%
\pgfpathlineto{\pgfqpoint{0.776182in}{0.328214in}}%
\pgfpathlineto{\pgfqpoint{0.782355in}{0.341140in}}%
\pgfpathlineto{\pgfqpoint{0.782439in}{0.341215in}}%
\pgfpathlineto{\pgfqpoint{0.795708in}{0.345003in}}%
\pgfpathlineto{\pgfqpoint{0.808977in}{0.342585in}}%
\pgfpathlineto{\pgfqpoint{0.810871in}{0.341140in}}%
\pgfpathlineto{\pgfqpoint{0.817211in}{0.328214in}}%
\pgfpathlineto{\pgfqpoint{0.818297in}{0.315287in}}%
\pgfpathlineto{\pgfqpoint{0.817032in}{0.302361in}}%
\pgfpathlineto{\pgfqpoint{0.811200in}{0.289434in}}%
\pgfpathlineto{\pgfqpoint{0.808977in}{0.287469in}}%
\pgfpathlineto{\pgfqpoint{0.795708in}{0.284618in}}%
\pgfpathlineto{\pgfqpoint{0.782439in}{0.288929in}}%
\pgfpathclose%
\pgfusepath{fill}%
\end{pgfscope}%
\begin{pgfscope}%
\pgfpathrectangle{\pgfqpoint{0.211875in}{0.211875in}}{\pgfqpoint{1.313625in}{1.279725in}}%
\pgfusepath{clip}%
\pgfsetbuttcap%
\pgfsetroundjoin%
\definecolor{currentfill}{rgb}{0.901975,0.231521,0.249182}%
\pgfsetfillcolor{currentfill}%
\pgfsetlinewidth{0.000000pt}%
\definecolor{currentstroke}{rgb}{0.000000,0.000000,0.000000}%
\pgfsetstrokecolor{currentstroke}%
\pgfsetdash{}{0pt}%
\pgfpathmoveto{\pgfqpoint{0.901860in}{0.275885in}}%
\pgfpathlineto{\pgfqpoint{0.915129in}{0.275188in}}%
\pgfpathlineto{\pgfqpoint{0.928398in}{0.275778in}}%
\pgfpathlineto{\pgfqpoint{0.932073in}{0.276508in}}%
\pgfpathlineto{\pgfqpoint{0.941667in}{0.286871in}}%
\pgfpathlineto{\pgfqpoint{0.942130in}{0.289434in}}%
\pgfpathlineto{\pgfqpoint{0.943115in}{0.302361in}}%
\pgfpathlineto{\pgfqpoint{0.943507in}{0.315287in}}%
\pgfpathlineto{\pgfqpoint{0.943793in}{0.328214in}}%
\pgfpathlineto{\pgfqpoint{0.944310in}{0.341140in}}%
\pgfpathlineto{\pgfqpoint{0.954936in}{0.353084in}}%
\pgfpathlineto{\pgfqpoint{0.968205in}{0.353504in}}%
\pgfpathlineto{\pgfqpoint{0.981473in}{0.353358in}}%
\pgfpathlineto{\pgfqpoint{0.994742in}{0.352084in}}%
\pgfpathlineto{\pgfqpoint{1.001805in}{0.341140in}}%
\pgfpathlineto{\pgfqpoint{1.002541in}{0.328214in}}%
\pgfpathlineto{\pgfqpoint{1.002869in}{0.315287in}}%
\pgfpathlineto{\pgfqpoint{1.003216in}{0.302361in}}%
\pgfpathlineto{\pgfqpoint{1.003992in}{0.289434in}}%
\pgfpathlineto{\pgfqpoint{1.008011in}{0.278154in}}%
\pgfpathlineto{\pgfqpoint{1.011192in}{0.276508in}}%
\pgfpathlineto{\pgfqpoint{1.021280in}{0.274762in}}%
\pgfpathlineto{\pgfqpoint{1.034549in}{0.274439in}}%
\pgfpathlineto{\pgfqpoint{1.047818in}{0.274977in}}%
\pgfpathlineto{\pgfqpoint{1.054493in}{0.276508in}}%
\pgfpathlineto{\pgfqpoint{1.060792in}{0.289434in}}%
\pgfpathlineto{\pgfqpoint{1.061087in}{0.293428in}}%
\pgfpathlineto{\pgfqpoint{1.061506in}{0.302361in}}%
\pgfpathlineto{\pgfqpoint{1.061853in}{0.315287in}}%
\pgfpathlineto{\pgfqpoint{1.062235in}{0.328214in}}%
\pgfpathlineto{\pgfqpoint{1.063194in}{0.341140in}}%
\pgfpathlineto{\pgfqpoint{1.074356in}{0.352260in}}%
\pgfpathlineto{\pgfqpoint{1.087625in}{0.352966in}}%
\pgfpathlineto{\pgfqpoint{1.100894in}{0.352763in}}%
\pgfpathlineto{\pgfqpoint{1.114163in}{0.350454in}}%
\pgfpathlineto{\pgfqpoint{1.119193in}{0.341140in}}%
\pgfpathlineto{\pgfqpoint{1.120178in}{0.328214in}}%
\pgfpathlineto{\pgfqpoint{1.120556in}{0.315287in}}%
\pgfpathlineto{\pgfqpoint{1.120875in}{0.302361in}}%
\pgfpathlineto{\pgfqpoint{1.121486in}{0.289434in}}%
\pgfpathlineto{\pgfqpoint{1.126583in}{0.276508in}}%
\pgfpathlineto{\pgfqpoint{1.127432in}{0.276109in}}%
\pgfpathlineto{\pgfqpoint{1.140701in}{0.274344in}}%
\pgfpathlineto{\pgfqpoint{1.153970in}{0.274211in}}%
\pgfpathlineto{\pgfqpoint{1.167239in}{0.274865in}}%
\pgfpathlineto{\pgfqpoint{1.173426in}{0.276508in}}%
\pgfpathlineto{\pgfqpoint{1.178777in}{0.289434in}}%
\pgfpathlineto{\pgfqpoint{1.179418in}{0.302361in}}%
\pgfpathlineto{\pgfqpoint{1.179752in}{0.315287in}}%
\pgfpathlineto{\pgfqpoint{1.180149in}{0.328214in}}%
\pgfpathlineto{\pgfqpoint{1.180508in}{0.334399in}}%
\pgfpathlineto{\pgfqpoint{1.181207in}{0.341140in}}%
\pgfpathlineto{\pgfqpoint{1.193777in}{0.352334in}}%
\pgfpathlineto{\pgfqpoint{1.207045in}{0.352970in}}%
\pgfpathlineto{\pgfqpoint{1.220314in}{0.352740in}}%
\pgfpathlineto{\pgfqpoint{1.233583in}{0.349813in}}%
\pgfpathlineto{\pgfqpoint{1.237343in}{0.341140in}}%
\pgfpathlineto{\pgfqpoint{1.238257in}{0.328214in}}%
\pgfpathlineto{\pgfqpoint{1.238621in}{0.315287in}}%
\pgfpathlineto{\pgfqpoint{1.238952in}{0.302361in}}%
\pgfpathlineto{\pgfqpoint{1.239625in}{0.289434in}}%
\pgfpathlineto{\pgfqpoint{1.245464in}{0.276508in}}%
\pgfpathlineto{\pgfqpoint{1.246852in}{0.275976in}}%
\pgfpathlineto{\pgfqpoint{1.260121in}{0.274532in}}%
\pgfpathlineto{\pgfqpoint{1.273390in}{0.274518in}}%
\pgfpathlineto{\pgfqpoint{1.286659in}{0.275654in}}%
\pgfpathlineto{\pgfqpoint{1.289415in}{0.276508in}}%
\pgfpathlineto{\pgfqpoint{1.296173in}{0.289434in}}%
\pgfpathlineto{\pgfqpoint{1.296966in}{0.302361in}}%
\pgfpathlineto{\pgfqpoint{1.297322in}{0.315287in}}%
\pgfpathlineto{\pgfqpoint{1.297657in}{0.328214in}}%
\pgfpathlineto{\pgfqpoint{1.298412in}{0.341140in}}%
\pgfpathlineto{\pgfqpoint{1.299928in}{0.347679in}}%
\pgfpathlineto{\pgfqpoint{1.313197in}{0.353096in}}%
\pgfpathlineto{\pgfqpoint{1.326466in}{0.353496in}}%
\pgfpathlineto{\pgfqpoint{1.339735in}{0.353379in}}%
\pgfpathlineto{\pgfqpoint{1.353004in}{0.351192in}}%
\pgfpathlineto{\pgfqpoint{1.356264in}{0.341140in}}%
\pgfpathlineto{\pgfqpoint{1.356771in}{0.328214in}}%
\pgfpathlineto{\pgfqpoint{1.357054in}{0.315287in}}%
\pgfpathlineto{\pgfqpoint{1.357439in}{0.302361in}}%
\pgfpathlineto{\pgfqpoint{1.358412in}{0.289434in}}%
\pgfpathlineto{\pgfqpoint{1.366273in}{0.276926in}}%
\pgfpathlineto{\pgfqpoint{1.368215in}{0.276508in}}%
\pgfpathlineto{\pgfqpoint{1.379542in}{0.275263in}}%
\pgfpathlineto{\pgfqpoint{1.392811in}{0.275399in}}%
\pgfpathlineto{\pgfqpoint{1.401246in}{0.276508in}}%
\pgfpathlineto{\pgfqpoint{1.406080in}{0.277784in}}%
\pgfpathlineto{\pgfqpoint{1.413001in}{0.289434in}}%
\pgfpathlineto{\pgfqpoint{1.414163in}{0.302361in}}%
\pgfpathlineto{\pgfqpoint{1.414580in}{0.315287in}}%
\pgfpathlineto{\pgfqpoint{1.414792in}{0.328214in}}%
\pgfpathlineto{\pgfqpoint{1.414963in}{0.341140in}}%
\pgfpathlineto{\pgfqpoint{1.419348in}{0.353982in}}%
\pgfpathlineto{\pgfqpoint{1.420213in}{0.354067in}}%
\pgfpathlineto{\pgfqpoint{1.432617in}{0.354414in}}%
\pgfpathlineto{\pgfqpoint{1.445886in}{0.354543in}}%
\pgfpathlineto{\pgfqpoint{1.459155in}{0.354815in}}%
\pgfpathlineto{\pgfqpoint{1.472424in}{0.358076in}}%
\pgfpathlineto{\pgfqpoint{1.474305in}{0.366993in}}%
\pgfpathlineto{\pgfqpoint{1.474745in}{0.379920in}}%
\pgfpathlineto{\pgfqpoint{1.474993in}{0.392846in}}%
\pgfpathlineto{\pgfqpoint{1.475335in}{0.405773in}}%
\pgfpathlineto{\pgfqpoint{1.476175in}{0.418699in}}%
\pgfpathlineto{\pgfqpoint{1.482480in}{0.431626in}}%
\pgfpathlineto{\pgfqpoint{1.485693in}{0.432786in}}%
\pgfpathlineto{\pgfqpoint{1.498962in}{0.434096in}}%
\pgfpathlineto{\pgfqpoint{1.512231in}{0.434184in}}%
\pgfpathlineto{\pgfqpoint{1.525500in}{0.432983in}}%
\pgfpathlineto{\pgfqpoint{1.525500in}{0.444552in}}%
\pgfpathlineto{\pgfqpoint{1.525500in}{0.452328in}}%
\pgfpathlineto{\pgfqpoint{1.517325in}{0.444552in}}%
\pgfpathlineto{\pgfqpoint{1.512231in}{0.442649in}}%
\pgfpathlineto{\pgfqpoint{1.498962in}{0.442031in}}%
\pgfpathlineto{\pgfqpoint{1.489242in}{0.444552in}}%
\pgfpathlineto{\pgfqpoint{1.485693in}{0.446396in}}%
\pgfpathlineto{\pgfqpoint{1.480111in}{0.457479in}}%
\pgfpathlineto{\pgfqpoint{1.478548in}{0.470405in}}%
\pgfpathlineto{\pgfqpoint{1.478344in}{0.483332in}}%
\pgfpathlineto{\pgfqpoint{1.479344in}{0.496258in}}%
\pgfpathlineto{\pgfqpoint{1.485693in}{0.509137in}}%
\pgfpathlineto{\pgfqpoint{1.485795in}{0.509185in}}%
\pgfpathlineto{\pgfqpoint{1.498962in}{0.512208in}}%
\pgfpathlineto{\pgfqpoint{1.512231in}{0.511687in}}%
\pgfpathlineto{\pgfqpoint{1.519652in}{0.509185in}}%
\pgfpathlineto{\pgfqpoint{1.525500in}{0.503852in}}%
\pgfpathlineto{\pgfqpoint{1.525500in}{0.509185in}}%
\pgfpathlineto{\pgfqpoint{1.525500in}{0.522111in}}%
\pgfpathlineto{\pgfqpoint{1.525500in}{0.525446in}}%
\pgfpathlineto{\pgfqpoint{1.519343in}{0.522111in}}%
\pgfpathlineto{\pgfqpoint{1.512231in}{0.520608in}}%
\pgfpathlineto{\pgfqpoint{1.498962in}{0.520540in}}%
\pgfpathlineto{\pgfqpoint{1.490471in}{0.522111in}}%
\pgfpathlineto{\pgfqpoint{1.485693in}{0.523831in}}%
\pgfpathlineto{\pgfqpoint{1.479463in}{0.535038in}}%
\pgfpathlineto{\pgfqpoint{1.478209in}{0.547964in}}%
\pgfpathlineto{\pgfqpoint{1.478219in}{0.560891in}}%
\pgfpathlineto{\pgfqpoint{1.479322in}{0.573817in}}%
\pgfpathlineto{\pgfqpoint{1.484056in}{0.586744in}}%
\pgfpathlineto{\pgfqpoint{1.485693in}{0.588365in}}%
\pgfpathlineto{\pgfqpoint{1.498962in}{0.592438in}}%
\pgfpathlineto{\pgfqpoint{1.512231in}{0.592259in}}%
\pgfpathlineto{\pgfqpoint{1.524825in}{0.586744in}}%
\pgfpathlineto{\pgfqpoint{1.525500in}{0.585948in}}%
\pgfpathlineto{\pgfqpoint{1.525500in}{0.586744in}}%
\pgfpathlineto{\pgfqpoint{1.525500in}{0.599670in}}%
\pgfpathlineto{\pgfqpoint{1.525500in}{0.604851in}}%
\pgfpathlineto{\pgfqpoint{1.512231in}{0.600925in}}%
\pgfpathlineto{\pgfqpoint{1.498962in}{0.600549in}}%
\pgfpathlineto{\pgfqpoint{1.485693in}{0.602054in}}%
\pgfpathlineto{\pgfqpoint{1.477236in}{0.612597in}}%
\pgfpathlineto{\pgfqpoint{1.475952in}{0.625523in}}%
\pgfpathlineto{\pgfqpoint{1.475582in}{0.638450in}}%
\pgfpathlineto{\pgfqpoint{1.475514in}{0.651377in}}%
\pgfpathlineto{\pgfqpoint{1.475792in}{0.664303in}}%
\pgfpathlineto{\pgfqpoint{1.481873in}{0.677230in}}%
\pgfpathlineto{\pgfqpoint{1.485693in}{0.677869in}}%
\pgfpathlineto{\pgfqpoint{1.498962in}{0.678286in}}%
\pgfpathlineto{\pgfqpoint{1.512231in}{0.678016in}}%
\pgfpathlineto{\pgfqpoint{1.519925in}{0.677230in}}%
\pgfpathlineto{\pgfqpoint{1.525500in}{0.675881in}}%
\pgfpathlineto{\pgfqpoint{1.525500in}{0.677230in}}%
\pgfpathlineto{\pgfqpoint{1.525500in}{0.690156in}}%
\pgfpathlineto{\pgfqpoint{1.525500in}{0.697983in}}%
\pgfpathlineto{\pgfqpoint{1.519369in}{0.690156in}}%
\pgfpathlineto{\pgfqpoint{1.512231in}{0.686998in}}%
\pgfpathlineto{\pgfqpoint{1.498962in}{0.686679in}}%
\pgfpathlineto{\pgfqpoint{1.489306in}{0.690156in}}%
\pgfpathlineto{\pgfqpoint{1.485693in}{0.693015in}}%
\pgfpathlineto{\pgfqpoint{1.481680in}{0.703083in}}%
\pgfpathlineto{\pgfqpoint{1.480466in}{0.716009in}}%
\pgfpathlineto{\pgfqpoint{1.480942in}{0.728936in}}%
\pgfpathlineto{\pgfqpoint{1.483851in}{0.741862in}}%
\pgfpathlineto{\pgfqpoint{1.485693in}{0.745000in}}%
\pgfpathlineto{\pgfqpoint{1.498962in}{0.751787in}}%
\pgfpathlineto{\pgfqpoint{1.512231in}{0.751368in}}%
\pgfpathlineto{\pgfqpoint{1.524636in}{0.741862in}}%
\pgfpathlineto{\pgfqpoint{1.525500in}{0.739655in}}%
\pgfpathlineto{\pgfqpoint{1.525500in}{0.741862in}}%
\pgfpathlineto{\pgfqpoint{1.525500in}{0.754789in}}%
\pgfpathlineto{\pgfqpoint{1.525500in}{0.760470in}}%
\pgfpathlineto{\pgfqpoint{1.512231in}{0.760274in}}%
\pgfpathlineto{\pgfqpoint{1.498962in}{0.760108in}}%
\pgfpathlineto{\pgfqpoint{1.485693in}{0.759564in}}%
\pgfpathlineto{\pgfqpoint{1.476474in}{0.754789in}}%
\pgfpathlineto{\pgfqpoint{1.474977in}{0.741862in}}%
\pgfpathlineto{\pgfqpoint{1.474689in}{0.728936in}}%
\pgfpathlineto{\pgfqpoint{1.474527in}{0.716009in}}%
\pgfpathlineto{\pgfqpoint{1.474328in}{0.703083in}}%
\pgfpathlineto{\pgfqpoint{1.473745in}{0.690156in}}%
\pgfpathlineto{\pgfqpoint{1.472424in}{0.684182in}}%
\pgfpathlineto{\pgfqpoint{1.459155in}{0.679889in}}%
\pgfpathlineto{\pgfqpoint{1.445886in}{0.679578in}}%
\pgfpathlineto{\pgfqpoint{1.432617in}{0.679476in}}%
\pgfpathlineto{\pgfqpoint{1.419348in}{0.679366in}}%
\pgfpathlineto{\pgfqpoint{1.414997in}{0.690156in}}%
\pgfpathlineto{\pgfqpoint{1.414982in}{0.703083in}}%
\pgfpathlineto{\pgfqpoint{1.414901in}{0.716009in}}%
\pgfpathlineto{\pgfqpoint{1.414713in}{0.728936in}}%
\pgfpathlineto{\pgfqpoint{1.414201in}{0.741862in}}%
\pgfpathlineto{\pgfqpoint{1.411051in}{0.754789in}}%
\pgfpathlineto{\pgfqpoint{1.406080in}{0.757642in}}%
\pgfpathlineto{\pgfqpoint{1.392811in}{0.758869in}}%
\pgfpathlineto{\pgfqpoint{1.379542in}{0.758758in}}%
\pgfpathlineto{\pgfqpoint{1.366273in}{0.757050in}}%
\pgfpathlineto{\pgfqpoint{1.362427in}{0.754789in}}%
\pgfpathlineto{\pgfqpoint{1.358361in}{0.741862in}}%
\pgfpathlineto{\pgfqpoint{1.357691in}{0.728936in}}%
\pgfpathlineto{\pgfqpoint{1.357485in}{0.716009in}}%
\pgfpathlineto{\pgfqpoint{1.357496in}{0.703083in}}%
\pgfpathlineto{\pgfqpoint{1.357918in}{0.690156in}}%
\pgfpathlineto{\pgfqpoint{1.366273in}{0.680257in}}%
\pgfpathlineto{\pgfqpoint{1.379542in}{0.679667in}}%
\pgfpathlineto{\pgfqpoint{1.392811in}{0.679541in}}%
\pgfpathlineto{\pgfqpoint{1.406080in}{0.679526in}}%
\pgfpathlineto{\pgfqpoint{1.415249in}{0.677230in}}%
\pgfpathlineto{\pgfqpoint{1.415052in}{0.664303in}}%
\pgfpathlineto{\pgfqpoint{1.414979in}{0.651377in}}%
\pgfpathlineto{\pgfqpoint{1.414847in}{0.638450in}}%
\pgfpathlineto{\pgfqpoint{1.414562in}{0.625523in}}%
\pgfpathlineto{\pgfqpoint{1.413698in}{0.612597in}}%
\pgfpathlineto{\pgfqpoint{1.406080in}{0.600922in}}%
\pgfpathlineto{\pgfqpoint{1.397135in}{0.599670in}}%
\pgfpathlineto{\pgfqpoint{1.392811in}{0.599363in}}%
\pgfpathlineto{\pgfqpoint{1.379542in}{0.599231in}}%
\pgfpathlineto{\pgfqpoint{1.370878in}{0.599670in}}%
\pgfpathlineto{\pgfqpoint{1.366273in}{0.600119in}}%
\pgfpathlineto{\pgfqpoint{1.357632in}{0.612597in}}%
\pgfpathlineto{\pgfqpoint{1.357019in}{0.625523in}}%
\pgfpathlineto{\pgfqpoint{1.356783in}{0.638450in}}%
\pgfpathlineto{\pgfqpoint{1.356598in}{0.651377in}}%
\pgfpathlineto{\pgfqpoint{1.356255in}{0.664303in}}%
\pgfpathlineto{\pgfqpoint{1.353004in}{0.676457in}}%
\pgfpathlineto{\pgfqpoint{1.351200in}{0.677230in}}%
\pgfpathlineto{\pgfqpoint{1.339735in}{0.678444in}}%
\pgfpathlineto{\pgfqpoint{1.326466in}{0.678535in}}%
\pgfpathlineto{\pgfqpoint{1.313197in}{0.678167in}}%
\pgfpathlineto{\pgfqpoint{1.306166in}{0.677230in}}%
\pgfpathlineto{\pgfqpoint{1.299928in}{0.672688in}}%
\pgfpathlineto{\pgfqpoint{1.298360in}{0.664303in}}%
\pgfpathlineto{\pgfqpoint{1.297818in}{0.651377in}}%
\pgfpathlineto{\pgfqpoint{1.297593in}{0.638450in}}%
\pgfpathlineto{\pgfqpoint{1.297398in}{0.625523in}}%
\pgfpathlineto{\pgfqpoint{1.297001in}{0.612597in}}%
\pgfpathlineto{\pgfqpoint{1.290940in}{0.599670in}}%
\pgfpathlineto{\pgfqpoint{1.286659in}{0.598979in}}%
\pgfpathlineto{\pgfqpoint{1.273390in}{0.598487in}}%
\pgfpathlineto{\pgfqpoint{1.260121in}{0.598452in}}%
\pgfpathlineto{\pgfqpoint{1.246852in}{0.598868in}}%
\pgfpathlineto{\pgfqpoint{1.242655in}{0.599670in}}%
\pgfpathlineto{\pgfqpoint{1.238774in}{0.612597in}}%
\pgfpathlineto{\pgfqpoint{1.238519in}{0.625523in}}%
\pgfpathlineto{\pgfqpoint{1.238355in}{0.638450in}}%
\pgfpathlineto{\pgfqpoint{1.238107in}{0.651377in}}%
\pgfpathlineto{\pgfqpoint{1.237437in}{0.664303in}}%
\pgfpathlineto{\pgfqpoint{1.233583in}{0.674899in}}%
\pgfpathlineto{\pgfqpoint{1.226043in}{0.677230in}}%
\pgfpathlineto{\pgfqpoint{1.220314in}{0.677800in}}%
\pgfpathlineto{\pgfqpoint{1.207045in}{0.678012in}}%
\pgfpathlineto{\pgfqpoint{1.193777in}{0.677414in}}%
\pgfpathlineto{\pgfqpoint{1.192484in}{0.677230in}}%
\pgfpathlineto{\pgfqpoint{1.181084in}{0.664303in}}%
\pgfpathlineto{\pgfqpoint{1.180508in}{0.656214in}}%
\pgfpathlineto{\pgfqpoint{1.180301in}{0.651377in}}%
\pgfpathlineto{\pgfqpoint{1.180030in}{0.638450in}}%
\pgfpathlineto{\pgfqpoint{1.179872in}{0.625523in}}%
\pgfpathlineto{\pgfqpoint{1.179680in}{0.612597in}}%
\pgfpathlineto{\pgfqpoint{1.177175in}{0.599670in}}%
\pgfpathlineto{\pgfqpoint{1.167239in}{0.598330in}}%
\pgfpathlineto{\pgfqpoint{1.153970in}{0.598191in}}%
\pgfpathlineto{\pgfqpoint{1.140701in}{0.598219in}}%
\pgfpathlineto{\pgfqpoint{1.127432in}{0.598600in}}%
\pgfpathlineto{\pgfqpoint{1.123014in}{0.599670in}}%
\pgfpathlineto{\pgfqpoint{1.120626in}{0.612597in}}%
\pgfpathlineto{\pgfqpoint{1.120442in}{0.625523in}}%
\pgfpathlineto{\pgfqpoint{1.120292in}{0.638450in}}%
\pgfpathlineto{\pgfqpoint{1.120034in}{0.651377in}}%
\pgfpathlineto{\pgfqpoint{1.119305in}{0.664303in}}%
\pgfpathlineto{\pgfqpoint{1.114163in}{0.675542in}}%
\pgfpathlineto{\pgfqpoint{1.107160in}{0.677230in}}%
\pgfpathlineto{\pgfqpoint{1.100894in}{0.677818in}}%
\pgfpathlineto{\pgfqpoint{1.087625in}{0.678010in}}%
\pgfpathlineto{\pgfqpoint{1.074356in}{0.677350in}}%
\pgfpathlineto{\pgfqpoint{1.073577in}{0.677230in}}%
\pgfpathlineto{\pgfqpoint{1.063095in}{0.664303in}}%
\pgfpathlineto{\pgfqpoint{1.062392in}{0.651377in}}%
\pgfpathlineto{\pgfqpoint{1.062132in}{0.638450in}}%
\pgfpathlineto{\pgfqpoint{1.061960in}{0.625523in}}%
\pgfpathlineto{\pgfqpoint{1.061693in}{0.612597in}}%
\pgfpathlineto{\pgfqpoint{1.061087in}{0.604814in}}%
\pgfpathlineto{\pgfqpoint{1.057523in}{0.599670in}}%
\pgfpathlineto{\pgfqpoint{1.047818in}{0.598573in}}%
\pgfpathlineto{\pgfqpoint{1.034549in}{0.598437in}}%
\pgfpathlineto{\pgfqpoint{1.021280in}{0.598596in}}%
\pgfpathlineto{\pgfqpoint{1.009358in}{0.599670in}}%
\pgfpathlineto{\pgfqpoint{1.008011in}{0.600032in}}%
\pgfpathlineto{\pgfqpoint{1.003183in}{0.612597in}}%
\pgfpathlineto{\pgfqpoint{1.002794in}{0.625523in}}%
\pgfpathlineto{\pgfqpoint{1.002604in}{0.638450in}}%
\pgfpathlineto{\pgfqpoint{1.002385in}{0.651377in}}%
\pgfpathlineto{\pgfqpoint{1.001856in}{0.664303in}}%
\pgfpathlineto{\pgfqpoint{0.994742in}{0.677227in}}%
\pgfpathlineto{\pgfqpoint{0.994731in}{0.677230in}}%
\pgfpathlineto{\pgfqpoint{0.981473in}{0.678407in}}%
\pgfpathlineto{\pgfqpoint{0.968205in}{0.678546in}}%
\pgfpathlineto{\pgfqpoint{0.954936in}{0.678191in}}%
\pgfpathlineto{\pgfqpoint{0.949206in}{0.677230in}}%
\pgfpathlineto{\pgfqpoint{0.944318in}{0.664303in}}%
\pgfpathlineto{\pgfqpoint{0.943969in}{0.651377in}}%
\pgfpathlineto{\pgfqpoint{0.943781in}{0.638450in}}%
\pgfpathlineto{\pgfqpoint{0.943542in}{0.625523in}}%
\pgfpathlineto{\pgfqpoint{0.942921in}{0.612597in}}%
\pgfpathlineto{\pgfqpoint{0.941667in}{0.605832in}}%
\pgfpathlineto{\pgfqpoint{0.930244in}{0.599670in}}%
\pgfpathlineto{\pgfqpoint{0.928398in}{0.599496in}}%
\pgfpathlineto{\pgfqpoint{0.915129in}{0.599208in}}%
\pgfpathlineto{\pgfqpoint{0.902205in}{0.599670in}}%
\pgfpathlineto{\pgfqpoint{0.901860in}{0.599687in}}%
\pgfpathlineto{\pgfqpoint{0.888591in}{0.605111in}}%
\pgfpathlineto{\pgfqpoint{0.886467in}{0.612597in}}%
\pgfpathlineto{\pgfqpoint{0.885601in}{0.625523in}}%
\pgfpathlineto{\pgfqpoint{0.885315in}{0.638450in}}%
\pgfpathlineto{\pgfqpoint{0.885184in}{0.651377in}}%
\pgfpathlineto{\pgfqpoint{0.885111in}{0.664303in}}%
\pgfpathlineto{\pgfqpoint{0.884928in}{0.677230in}}%
\pgfpathlineto{\pgfqpoint{0.875322in}{0.679434in}}%
\pgfpathlineto{\pgfqpoint{0.862053in}{0.679510in}}%
\pgfpathlineto{\pgfqpoint{0.848784in}{0.679664in}}%
\pgfpathlineto{\pgfqpoint{0.835515in}{0.680306in}}%
\pgfpathlineto{\pgfqpoint{0.826798in}{0.690156in}}%
\pgfpathlineto{\pgfqpoint{0.826226in}{0.703083in}}%
\pgfpathlineto{\pgfqpoint{0.826031in}{0.716009in}}%
\pgfpathlineto{\pgfqpoint{0.825872in}{0.728936in}}%
\pgfpathlineto{\pgfqpoint{0.825588in}{0.741862in}}%
\pgfpathlineto{\pgfqpoint{0.824111in}{0.754789in}}%
\pgfpathlineto{\pgfqpoint{0.822246in}{0.757509in}}%
\pgfpathlineto{\pgfqpoint{0.808977in}{0.759904in}}%
\pgfpathlineto{\pgfqpoint{0.795708in}{0.760194in}}%
\pgfpathlineto{\pgfqpoint{0.782439in}{0.760332in}}%
\pgfpathlineto{\pgfqpoint{0.769170in}{0.761072in}}%
\pgfpathlineto{\pgfqpoint{0.766885in}{0.767715in}}%
\pgfpathlineto{\pgfqpoint{0.766685in}{0.780642in}}%
\pgfpathlineto{\pgfqpoint{0.766598in}{0.793568in}}%
\pgfpathlineto{\pgfqpoint{0.766478in}{0.806495in}}%
\pgfpathlineto{\pgfqpoint{0.766213in}{0.819421in}}%
\pgfpathlineto{\pgfqpoint{0.765175in}{0.832348in}}%
\pgfpathlineto{\pgfqpoint{0.755902in}{0.840282in}}%
\pgfpathlineto{\pgfqpoint{0.742633in}{0.841050in}}%
\pgfpathlineto{\pgfqpoint{0.729364in}{0.841238in}}%
\pgfpathlineto{\pgfqpoint{0.716095in}{0.841281in}}%
\pgfpathlineto{\pgfqpoint{0.707780in}{0.845274in}}%
\pgfpathlineto{\pgfqpoint{0.707832in}{0.858201in}}%
\pgfpathlineto{\pgfqpoint{0.707807in}{0.871127in}}%
\pgfpathlineto{\pgfqpoint{0.707739in}{0.884054in}}%
\pgfpathlineto{\pgfqpoint{0.707577in}{0.896980in}}%
\pgfpathlineto{\pgfqpoint{0.707048in}{0.909907in}}%
\pgfpathlineto{\pgfqpoint{0.702826in}{0.920310in}}%
\pgfpathlineto{\pgfqpoint{0.689557in}{0.922313in}}%
\pgfpathlineto{\pgfqpoint{0.676288in}{0.922645in}}%
\pgfpathlineto{\pgfqpoint{0.664979in}{0.922833in}}%
\pgfpathlineto{\pgfqpoint{0.663019in}{0.922881in}}%
\pgfpathlineto{\pgfqpoint{0.649750in}{0.928093in}}%
\pgfpathlineto{\pgfqpoint{0.649227in}{0.935760in}}%
\pgfpathlineto{\pgfqpoint{0.649050in}{0.948686in}}%
\pgfpathlineto{\pgfqpoint{0.648977in}{0.961613in}}%
\pgfpathlineto{\pgfqpoint{0.648908in}{0.974539in}}%
\pgfpathlineto{\pgfqpoint{0.648766in}{0.987466in}}%
\pgfpathlineto{\pgfqpoint{0.647748in}{1.000392in}}%
\pgfpathlineto{\pgfqpoint{0.636481in}{1.004111in}}%
\pgfpathlineto{\pgfqpoint{0.623212in}{1.004396in}}%
\pgfpathlineto{\pgfqpoint{0.609943in}{1.004725in}}%
\pgfpathlineto{\pgfqpoint{0.596674in}{1.006337in}}%
\pgfpathlineto{\pgfqpoint{0.591799in}{1.013319in}}%
\pgfpathlineto{\pgfqpoint{0.590850in}{1.026245in}}%
\pgfpathlineto{\pgfqpoint{0.590648in}{1.039172in}}%
\pgfpathlineto{\pgfqpoint{0.590627in}{1.052098in}}%
\pgfpathlineto{\pgfqpoint{0.590776in}{1.065025in}}%
\pgfpathlineto{\pgfqpoint{0.591674in}{1.077952in}}%
\pgfpathlineto{\pgfqpoint{0.596674in}{1.083897in}}%
\pgfpathlineto{\pgfqpoint{0.609943in}{1.085079in}}%
\pgfpathlineto{\pgfqpoint{0.623212in}{1.085376in}}%
\pgfpathlineto{\pgfqpoint{0.636481in}{1.085814in}}%
\pgfpathlineto{\pgfqpoint{0.647159in}{1.090878in}}%
\pgfpathlineto{\pgfqpoint{0.648364in}{1.103805in}}%
\pgfpathlineto{\pgfqpoint{0.648548in}{1.116731in}}%
\pgfpathlineto{\pgfqpoint{0.648571in}{1.129658in}}%
\pgfpathlineto{\pgfqpoint{0.648475in}{1.142584in}}%
\pgfpathlineto{\pgfqpoint{0.648019in}{1.155511in}}%
\pgfpathlineto{\pgfqpoint{0.636481in}{1.166039in}}%
\pgfpathlineto{\pgfqpoint{0.623212in}{1.166484in}}%
\pgfpathlineto{\pgfqpoint{0.609943in}{1.166786in}}%
\pgfpathlineto{\pgfqpoint{0.596674in}{1.167984in}}%
\pgfpathlineto{\pgfqpoint{0.595391in}{1.168437in}}%
\pgfpathlineto{\pgfqpoint{0.590961in}{1.181364in}}%
\pgfpathlineto{\pgfqpoint{0.590662in}{1.194290in}}%
\pgfpathlineto{\pgfqpoint{0.590622in}{1.207217in}}%
\pgfpathlineto{\pgfqpoint{0.590730in}{1.220143in}}%
\pgfpathlineto{\pgfqpoint{0.591181in}{1.233070in}}%
\pgfpathlineto{\pgfqpoint{0.596674in}{1.245652in}}%
\pgfpathlineto{\pgfqpoint{0.597899in}{1.245996in}}%
\pgfpathlineto{\pgfqpoint{0.609943in}{1.247292in}}%
\pgfpathlineto{\pgfqpoint{0.623212in}{1.247625in}}%
\pgfpathlineto{\pgfqpoint{0.636481in}{1.247913in}}%
\pgfpathlineto{\pgfqpoint{0.648611in}{1.258923in}}%
\pgfpathlineto{\pgfqpoint{0.648864in}{1.271849in}}%
\pgfpathlineto{\pgfqpoint{0.648950in}{1.284776in}}%
\pgfpathlineto{\pgfqpoint{0.649015in}{1.297702in}}%
\pgfpathlineto{\pgfqpoint{0.649122in}{1.310629in}}%
\pgfpathlineto{\pgfqpoint{0.649682in}{1.323555in}}%
\pgfpathlineto{\pgfqpoint{0.649750in}{1.323997in}}%
\pgfpathlineto{\pgfqpoint{0.663019in}{1.329069in}}%
\pgfpathlineto{\pgfqpoint{0.676288in}{1.329298in}}%
\pgfpathlineto{\pgfqpoint{0.689557in}{1.329620in}}%
\pgfpathlineto{\pgfqpoint{0.702826in}{1.331562in}}%
\pgfpathlineto{\pgfqpoint{0.706252in}{1.336482in}}%
\pgfpathlineto{\pgfqpoint{0.707435in}{1.349408in}}%
\pgfpathlineto{\pgfqpoint{0.707688in}{1.362335in}}%
\pgfpathlineto{\pgfqpoint{0.707785in}{1.375261in}}%
\pgfpathlineto{\pgfqpoint{0.707826in}{1.388188in}}%
\pgfpathlineto{\pgfqpoint{0.707825in}{1.401114in}}%
\pgfpathlineto{\pgfqpoint{0.702826in}{1.410454in}}%
\pgfpathlineto{\pgfqpoint{0.689557in}{1.410555in}}%
\pgfpathlineto{\pgfqpoint{0.676288in}{1.410733in}}%
\pgfpathlineto{\pgfqpoint{0.663019in}{1.411342in}}%
\pgfpathlineto{\pgfqpoint{0.654194in}{1.414041in}}%
\pgfpathlineto{\pgfqpoint{0.650534in}{1.426967in}}%
\pgfpathlineto{\pgfqpoint{0.650151in}{1.439894in}}%
\pgfpathlineto{\pgfqpoint{0.650179in}{1.452820in}}%
\pgfpathlineto{\pgfqpoint{0.650557in}{1.465747in}}%
\pgfpathlineto{\pgfqpoint{0.652007in}{1.478673in}}%
\pgfpathlineto{\pgfqpoint{0.663019in}{1.488803in}}%
\pgfpathlineto{\pgfqpoint{0.676288in}{1.489879in}}%
\pgfpathlineto{\pgfqpoint{0.689557in}{1.489859in}}%
\pgfpathlineto{\pgfqpoint{0.702826in}{1.487531in}}%
\pgfpathlineto{\pgfqpoint{0.706750in}{1.478673in}}%
\pgfpathlineto{\pgfqpoint{0.707487in}{1.465747in}}%
\pgfpathlineto{\pgfqpoint{0.707707in}{1.452820in}}%
\pgfpathlineto{\pgfqpoint{0.707804in}{1.439894in}}%
\pgfpathlineto{\pgfqpoint{0.707858in}{1.426967in}}%
\pgfpathlineto{\pgfqpoint{0.707948in}{1.414041in}}%
\pgfpathlineto{\pgfqpoint{0.716095in}{1.410555in}}%
\pgfpathlineto{\pgfqpoint{0.729364in}{1.410597in}}%
\pgfpathlineto{\pgfqpoint{0.742633in}{1.410785in}}%
\pgfpathlineto{\pgfqpoint{0.755902in}{1.411553in}}%
\pgfpathlineto{\pgfqpoint{0.762646in}{1.414041in}}%
\pgfpathlineto{\pgfqpoint{0.765967in}{1.426967in}}%
\pgfpathlineto{\pgfqpoint{0.766394in}{1.439894in}}%
\pgfpathlineto{\pgfqpoint{0.766555in}{1.452820in}}%
\pgfpathlineto{\pgfqpoint{0.766647in}{1.465747in}}%
\pgfpathlineto{\pgfqpoint{0.766757in}{1.478673in}}%
\pgfpathlineto{\pgfqpoint{0.769170in}{1.490856in}}%
\pgfpathlineto{\pgfqpoint{0.781268in}{1.491600in}}%
\pgfpathlineto{\pgfqpoint{0.769170in}{1.491600in}}%
\pgfpathlineto{\pgfqpoint{0.755902in}{1.491600in}}%
\pgfpathlineto{\pgfqpoint{0.742633in}{1.491600in}}%
\pgfpathlineto{\pgfqpoint{0.729364in}{1.491600in}}%
\pgfpathlineto{\pgfqpoint{0.716095in}{1.491600in}}%
\pgfpathlineto{\pgfqpoint{0.702826in}{1.491600in}}%
\pgfpathlineto{\pgfqpoint{0.689557in}{1.491600in}}%
\pgfpathlineto{\pgfqpoint{0.676288in}{1.491600in}}%
\pgfpathlineto{\pgfqpoint{0.663019in}{1.491600in}}%
\pgfpathlineto{\pgfqpoint{0.649750in}{1.491600in}}%
\pgfpathlineto{\pgfqpoint{0.636481in}{1.491600in}}%
\pgfpathlineto{\pgfqpoint{0.623212in}{1.491600in}}%
\pgfpathlineto{\pgfqpoint{0.609943in}{1.491600in}}%
\pgfpathlineto{\pgfqpoint{0.596674in}{1.491600in}}%
\pgfpathlineto{\pgfqpoint{0.583405in}{1.491600in}}%
\pgfpathlineto{\pgfqpoint{0.570136in}{1.491600in}}%
\pgfpathlineto{\pgfqpoint{0.556867in}{1.491600in}}%
\pgfpathlineto{\pgfqpoint{0.543598in}{1.491600in}}%
\pgfpathlineto{\pgfqpoint{0.530330in}{1.491600in}}%
\pgfpathlineto{\pgfqpoint{0.517061in}{1.491600in}}%
\pgfpathlineto{\pgfqpoint{0.503792in}{1.491600in}}%
\pgfpathlineto{\pgfqpoint{0.490523in}{1.491600in}}%
\pgfpathlineto{\pgfqpoint{0.477254in}{1.491600in}}%
\pgfpathlineto{\pgfqpoint{0.463985in}{1.491600in}}%
\pgfpathlineto{\pgfqpoint{0.450716in}{1.491600in}}%
\pgfpathlineto{\pgfqpoint{0.437447in}{1.491600in}}%
\pgfpathlineto{\pgfqpoint{0.424178in}{1.491600in}}%
\pgfpathlineto{\pgfqpoint{0.410909in}{1.491600in}}%
\pgfpathlineto{\pgfqpoint{0.397640in}{1.491600in}}%
\pgfpathlineto{\pgfqpoint{0.395959in}{1.491600in}}%
\pgfpathlineto{\pgfqpoint{0.397640in}{1.491539in}}%
\pgfpathlineto{\pgfqpoint{0.410909in}{1.488483in}}%
\pgfpathlineto{\pgfqpoint{0.412489in}{1.478673in}}%
\pgfpathlineto{\pgfqpoint{0.412698in}{1.465747in}}%
\pgfpathlineto{\pgfqpoint{0.412674in}{1.452820in}}%
\pgfpathlineto{\pgfqpoint{0.412481in}{1.439894in}}%
\pgfpathlineto{\pgfqpoint{0.411866in}{1.426967in}}%
\pgfpathlineto{\pgfqpoint{0.410909in}{1.420669in}}%
\pgfpathlineto{\pgfqpoint{0.406841in}{1.414041in}}%
\pgfpathlineto{\pgfqpoint{0.397640in}{1.411516in}}%
\pgfpathlineto{\pgfqpoint{0.384371in}{1.410805in}}%
\pgfpathlineto{\pgfqpoint{0.371102in}{1.410489in}}%
\pgfpathlineto{\pgfqpoint{0.357833in}{1.409392in}}%
\pgfpathlineto{\pgfqpoint{0.354519in}{1.401114in}}%
\pgfpathlineto{\pgfqpoint{0.354210in}{1.388188in}}%
\pgfpathlineto{\pgfqpoint{0.354107in}{1.375261in}}%
\pgfpathlineto{\pgfqpoint{0.354013in}{1.362335in}}%
\pgfpathlineto{\pgfqpoint{0.353838in}{1.349408in}}%
\pgfpathlineto{\pgfqpoint{0.353100in}{1.336482in}}%
\pgfpathlineto{\pgfqpoint{0.344564in}{1.329480in}}%
\pgfpathlineto{\pgfqpoint{0.331295in}{1.328854in}}%
\pgfpathlineto{\pgfqpoint{0.318027in}{1.328422in}}%
\pgfpathlineto{\pgfqpoint{0.304758in}{1.327001in}}%
\pgfpathlineto{\pgfqpoint{0.299399in}{1.323555in}}%
\pgfpathlineto{\pgfqpoint{0.296689in}{1.310629in}}%
\pgfpathlineto{\pgfqpoint{0.296246in}{1.297702in}}%
\pgfpathlineto{\pgfqpoint{0.296141in}{1.284776in}}%
\pgfpathlineto{\pgfqpoint{0.296236in}{1.271849in}}%
\pgfpathlineto{\pgfqpoint{0.296827in}{1.258923in}}%
\pgfpathlineto{\pgfqpoint{0.304758in}{1.248879in}}%
\pgfpathlineto{\pgfqpoint{0.318027in}{1.247925in}}%
\pgfpathlineto{\pgfqpoint{0.331295in}{1.247483in}}%
\pgfpathlineto{\pgfqpoint{0.344564in}{1.246322in}}%
\pgfpathlineto{\pgfqpoint{0.345837in}{1.245996in}}%
\pgfpathlineto{\pgfqpoint{0.352740in}{1.233070in}}%
\pgfpathlineto{\pgfqpoint{0.353238in}{1.220143in}}%
\pgfpathlineto{\pgfqpoint{0.353319in}{1.207217in}}%
\pgfpathlineto{\pgfqpoint{0.353171in}{1.194290in}}%
\pgfpathlineto{\pgfqpoint{0.352528in}{1.181364in}}%
\pgfpathlineto{\pgfqpoint{0.344564in}{1.168566in}}%
\pgfpathlineto{\pgfqpoint{0.343913in}{1.168437in}}%
\pgfpathlineto{\pgfqpoint{0.331295in}{1.167168in}}%
\pgfpathlineto{\pgfqpoint{0.318027in}{1.166731in}}%
\pgfpathlineto{\pgfqpoint{0.304758in}{1.166054in}}%
\pgfpathlineto{\pgfqpoint{0.296278in}{1.155511in}}%
\pgfpathlineto{\pgfqpoint{0.295868in}{1.142584in}}%
\pgfpathlineto{\pgfqpoint{0.295781in}{1.129658in}}%
\pgfpathlineto{\pgfqpoint{0.295802in}{1.116731in}}%
\pgfpathlineto{\pgfqpoint{0.295968in}{1.103805in}}%
\pgfpathlineto{\pgfqpoint{0.297036in}{1.090878in}}%
\pgfpathlineto{\pgfqpoint{0.304758in}{1.085801in}}%
\pgfpathlineto{\pgfqpoint{0.318027in}{1.085134in}}%
\pgfpathlineto{\pgfqpoint{0.331295in}{1.084703in}}%
\pgfpathlineto{\pgfqpoint{0.344564in}{1.083332in}}%
\pgfpathlineto{\pgfqpoint{0.351087in}{1.077952in}}%
\pgfpathlineto{\pgfqpoint{0.352917in}{1.065025in}}%
\pgfpathlineto{\pgfqpoint{0.353267in}{1.052098in}}%
\pgfpathlineto{\pgfqpoint{0.353313in}{1.039172in}}%
\pgfpathlineto{\pgfqpoint{0.353110in}{1.026245in}}%
\pgfpathlineto{\pgfqpoint{0.352043in}{1.013319in}}%
\pgfpathlineto{\pgfqpoint{0.344564in}{1.005684in}}%
\pgfpathlineto{\pgfqpoint{0.331295in}{1.004536in}}%
\pgfpathlineto{\pgfqpoint{0.318027in}{1.004099in}}%
\pgfpathlineto{\pgfqpoint{0.304758in}{1.003156in}}%
\pgfpathlineto{\pgfqpoint{0.298950in}{1.000392in}}%
\pgfpathlineto{\pgfqpoint{0.296449in}{0.987466in}}%
\pgfpathlineto{\pgfqpoint{0.296167in}{0.974539in}}%
\pgfpathlineto{\pgfqpoint{0.296162in}{0.961613in}}%
\pgfpathlineto{\pgfqpoint{0.296370in}{0.948686in}}%
\pgfpathlineto{\pgfqpoint{0.297195in}{0.935760in}}%
\pgfpathlineto{\pgfqpoint{0.304758in}{0.925012in}}%
\pgfpathlineto{\pgfqpoint{0.318027in}{0.923548in}}%
\pgfpathlineto{\pgfqpoint{0.331295in}{0.923103in}}%
\pgfpathlineto{\pgfqpoint{0.338915in}{0.922833in}}%
\pgfpathlineto{\pgfqpoint{0.344564in}{0.922456in}}%
\pgfpathlineto{\pgfqpoint{0.353590in}{0.909907in}}%
\pgfpathlineto{\pgfqpoint{0.353933in}{0.896980in}}%
\pgfpathlineto{\pgfqpoint{0.354057in}{0.884054in}}%
\pgfpathlineto{\pgfqpoint{0.354145in}{0.871127in}}%
\pgfpathlineto{\pgfqpoint{0.354287in}{0.858201in}}%
\pgfpathlineto{\pgfqpoint{0.355238in}{0.845274in}}%
\pgfpathlineto{\pgfqpoint{0.344564in}{0.840479in}}%
\pgfpathlineto{\pgfqpoint{0.331295in}{0.840599in}}%
\pgfpathlineto{\pgfqpoint{0.318027in}{0.840151in}}%
\pgfpathlineto{\pgfqpoint{0.304758in}{0.837839in}}%
\pgfpathlineto{\pgfqpoint{0.299702in}{0.832348in}}%
\pgfpathlineto{\pgfqpoint{0.297628in}{0.819421in}}%
\pgfpathlineto{\pgfqpoint{0.297262in}{0.806495in}}%
\pgfpathlineto{\pgfqpoint{0.297424in}{0.793568in}}%
\pgfpathlineto{\pgfqpoint{0.298298in}{0.780642in}}%
\pgfpathlineto{\pgfqpoint{0.302819in}{0.767715in}}%
\pgfpathlineto{\pgfqpoint{0.304758in}{0.766130in}}%
\pgfpathlineto{\pgfqpoint{0.318027in}{0.762993in}}%
\pgfpathlineto{\pgfqpoint{0.331295in}{0.762553in}}%
\pgfpathlineto{\pgfqpoint{0.344564in}{0.763588in}}%
\pgfpathlineto{\pgfqpoint{0.350688in}{0.767715in}}%
\pgfpathlineto{\pgfqpoint{0.353067in}{0.780642in}}%
\pgfpathlineto{\pgfqpoint{0.353505in}{0.793568in}}%
\pgfpathlineto{\pgfqpoint{0.353647in}{0.806495in}}%
\pgfpathlineto{\pgfqpoint{0.353648in}{0.819421in}}%
\pgfpathlineto{\pgfqpoint{0.353339in}{0.832348in}}%
\pgfpathlineto{\pgfqpoint{0.357833in}{0.842446in}}%
\pgfpathlineto{\pgfqpoint{0.371102in}{0.841346in}}%
\pgfpathlineto{\pgfqpoint{0.384371in}{0.841030in}}%
\pgfpathlineto{\pgfqpoint{0.397640in}{0.840319in}}%
\pgfpathlineto{\pgfqpoint{0.410688in}{0.832348in}}%
\pgfpathlineto{\pgfqpoint{0.410909in}{0.831459in}}%
\pgfpathlineto{\pgfqpoint{0.412225in}{0.819421in}}%
\pgfpathlineto{\pgfqpoint{0.412590in}{0.806495in}}%
\pgfpathlineto{\pgfqpoint{0.412702in}{0.793568in}}%
\pgfpathlineto{\pgfqpoint{0.412654in}{0.780642in}}%
\pgfpathlineto{\pgfqpoint{0.412116in}{0.767715in}}%
\pgfpathlineto{\pgfqpoint{0.410909in}{0.763358in}}%
\pgfpathlineto{\pgfqpoint{0.397640in}{0.760402in}}%
\pgfpathlineto{\pgfqpoint{0.384371in}{0.760042in}}%
\pgfpathlineto{\pgfqpoint{0.371102in}{0.759423in}}%
\pgfpathlineto{\pgfqpoint{0.358554in}{0.754789in}}%
\pgfpathlineto{\pgfqpoint{0.357833in}{0.753455in}}%
\pgfpathlineto{\pgfqpoint{0.355896in}{0.741862in}}%
\pgfpathlineto{\pgfqpoint{0.355548in}{0.728936in}}%
\pgfpathlineto{\pgfqpoint{0.355630in}{0.716009in}}%
\pgfpathlineto{\pgfqpoint{0.356161in}{0.703083in}}%
\pgfpathlineto{\pgfqpoint{0.357833in}{0.691845in}}%
\pgfpathlineto{\pgfqpoint{0.358379in}{0.690156in}}%
\pgfpathlineto{\pgfqpoint{0.371102in}{0.681944in}}%
\pgfpathlineto{\pgfqpoint{0.384371in}{0.680993in}}%
\pgfpathlineto{\pgfqpoint{0.397640in}{0.681081in}}%
\pgfpathlineto{\pgfqpoint{0.410909in}{0.686272in}}%
\pgfpathlineto{\pgfqpoint{0.411954in}{0.690156in}}%
\pgfpathlineto{\pgfqpoint{0.412857in}{0.703083in}}%
\pgfpathlineto{\pgfqpoint{0.413117in}{0.716009in}}%
\pgfpathlineto{\pgfqpoint{0.413272in}{0.728936in}}%
\pgfpathlineto{\pgfqpoint{0.413473in}{0.741862in}}%
\pgfpathlineto{\pgfqpoint{0.414493in}{0.754789in}}%
\pgfpathlineto{\pgfqpoint{0.424178in}{0.759276in}}%
\pgfpathlineto{\pgfqpoint{0.437447in}{0.759416in}}%
\pgfpathlineto{\pgfqpoint{0.450716in}{0.758983in}}%
\pgfpathlineto{\pgfqpoint{0.463985in}{0.756546in}}%
\pgfpathlineto{\pgfqpoint{0.466382in}{0.754789in}}%
\pgfpathlineto{\pgfqpoint{0.470334in}{0.741862in}}%
\pgfpathlineto{\pgfqpoint{0.471011in}{0.728936in}}%
\pgfpathlineto{\pgfqpoint{0.471216in}{0.716009in}}%
\pgfpathlineto{\pgfqpoint{0.471190in}{0.703083in}}%
\pgfpathlineto{\pgfqpoint{0.470715in}{0.690156in}}%
\pgfpathlineto{\pgfqpoint{0.463985in}{0.680359in}}%
\pgfpathlineto{\pgfqpoint{0.450716in}{0.679384in}}%
\pgfpathlineto{\pgfqpoint{0.437447in}{0.679005in}}%
\pgfpathlineto{\pgfqpoint{0.424178in}{0.678117in}}%
\pgfpathlineto{\pgfqpoint{0.420592in}{0.677230in}}%
\pgfpathlineto{\pgfqpoint{0.414902in}{0.664303in}}%
\pgfpathlineto{\pgfqpoint{0.414521in}{0.651377in}}%
\pgfpathlineto{\pgfqpoint{0.414563in}{0.638450in}}%
\pgfpathlineto{\pgfqpoint{0.414944in}{0.625523in}}%
\pgfpathlineto{\pgfqpoint{0.416270in}{0.612597in}}%
\pgfpathlineto{\pgfqpoint{0.424178in}{0.601880in}}%
\pgfpathlineto{\pgfqpoint{0.437447in}{0.599841in}}%
\pgfpathlineto{\pgfqpoint{0.445502in}{0.599670in}}%
\pgfpathlineto{\pgfqpoint{0.450716in}{0.599558in}}%
\pgfpathlineto{\pgfqpoint{0.453598in}{0.599670in}}%
\pgfpathlineto{\pgfqpoint{0.463985in}{0.600551in}}%
\pgfpathlineto{\pgfqpoint{0.471253in}{0.612597in}}%
\pgfpathlineto{\pgfqpoint{0.471821in}{0.625523in}}%
\pgfpathlineto{\pgfqpoint{0.472048in}{0.638450in}}%
\pgfpathlineto{\pgfqpoint{0.472244in}{0.651377in}}%
\pgfpathlineto{\pgfqpoint{0.472646in}{0.664303in}}%
\pgfpathlineto{\pgfqpoint{0.477254in}{0.676524in}}%
\pgfpathlineto{\pgfqpoint{0.480377in}{0.677230in}}%
\pgfpathlineto{\pgfqpoint{0.490523in}{0.677972in}}%
\pgfpathlineto{\pgfqpoint{0.503792in}{0.677863in}}%
\pgfpathlineto{\pgfqpoint{0.512652in}{0.677230in}}%
\pgfpathlineto{\pgfqpoint{0.517061in}{0.676756in}}%
\pgfpathlineto{\pgfqpoint{0.528708in}{0.664303in}}%
\pgfpathlineto{\pgfqpoint{0.529858in}{0.651377in}}%
\pgfpathlineto{\pgfqpoint{0.530209in}{0.638450in}}%
\pgfpathlineto{\pgfqpoint{0.530287in}{0.625523in}}%
\pgfpathlineto{\pgfqpoint{0.530073in}{0.612597in}}%
\pgfpathlineto{\pgfqpoint{0.525990in}{0.599670in}}%
\pgfpathlineto{\pgfqpoint{0.517061in}{0.598241in}}%
\pgfpathlineto{\pgfqpoint{0.503792in}{0.597823in}}%
\pgfpathlineto{\pgfqpoint{0.490523in}{0.597389in}}%
\pgfpathlineto{\pgfqpoint{0.477254in}{0.594712in}}%
\pgfpathlineto{\pgfqpoint{0.474045in}{0.586744in}}%
\pgfpathlineto{\pgfqpoint{0.473401in}{0.573817in}}%
\pgfpathlineto{\pgfqpoint{0.473372in}{0.560891in}}%
\pgfpathlineto{\pgfqpoint{0.473670in}{0.547964in}}%
\pgfpathlineto{\pgfqpoint{0.474654in}{0.535038in}}%
\pgfpathlineto{\pgfqpoint{0.477254in}{0.526062in}}%
\pgfpathlineto{\pgfqpoint{0.481384in}{0.522111in}}%
\pgfpathlineto{\pgfqpoint{0.490523in}{0.519417in}}%
\pgfpathlineto{\pgfqpoint{0.503792in}{0.518671in}}%
\pgfpathlineto{\pgfqpoint{0.517061in}{0.519066in}}%
\pgfpathlineto{\pgfqpoint{0.526243in}{0.522111in}}%
\pgfpathlineto{\pgfqpoint{0.530142in}{0.535038in}}%
\pgfpathlineto{\pgfqpoint{0.530330in}{0.538133in}}%
\pgfpathlineto{\pgfqpoint{0.530726in}{0.547964in}}%
\pgfpathlineto{\pgfqpoint{0.530982in}{0.560891in}}%
\pgfpathlineto{\pgfqpoint{0.531205in}{0.573817in}}%
\pgfpathlineto{\pgfqpoint{0.531730in}{0.586744in}}%
\pgfpathlineto{\pgfqpoint{0.543598in}{0.596701in}}%
\pgfpathlineto{\pgfqpoint{0.556867in}{0.596860in}}%
\pgfpathlineto{\pgfqpoint{0.570136in}{0.596402in}}%
\pgfpathlineto{\pgfqpoint{0.583405in}{0.593414in}}%
\pgfpathlineto{\pgfqpoint{0.587480in}{0.586744in}}%
\pgfpathlineto{\pgfqpoint{0.588950in}{0.573817in}}%
\pgfpathlineto{\pgfqpoint{0.589368in}{0.560891in}}%
\pgfpathlineto{\pgfqpoint{0.589547in}{0.547964in}}%
\pgfpathlineto{\pgfqpoint{0.589636in}{0.535038in}}%
\pgfpathlineto{\pgfqpoint{0.589705in}{0.522111in}}%
\pgfpathlineto{\pgfqpoint{0.596674in}{0.516415in}}%
\pgfpathlineto{\pgfqpoint{0.609943in}{0.516319in}}%
\pgfpathlineto{\pgfqpoint{0.623212in}{0.516106in}}%
\pgfpathlineto{\pgfqpoint{0.636481in}{0.515342in}}%
\pgfpathlineto{\pgfqpoint{0.646546in}{0.509185in}}%
\pgfpathlineto{\pgfqpoint{0.648312in}{0.496258in}}%
\pgfpathlineto{\pgfqpoint{0.648743in}{0.483332in}}%
\pgfpathlineto{\pgfqpoint{0.649002in}{0.470405in}}%
\pgfpathlineto{\pgfqpoint{0.649338in}{0.457479in}}%
\pgfpathlineto{\pgfqpoint{0.649750in}{0.450310in}}%
\pgfpathlineto{\pgfqpoint{0.650481in}{0.444552in}}%
\pgfpathlineto{\pgfqpoint{0.663019in}{0.436538in}}%
\pgfpathlineto{\pgfqpoint{0.676288in}{0.436017in}}%
\pgfpathlineto{\pgfqpoint{0.689557in}{0.435825in}}%
\pgfpathlineto{\pgfqpoint{0.702826in}{0.435483in}}%
\pgfpathlineto{\pgfqpoint{0.708001in}{0.431626in}}%
\pgfpathlineto{\pgfqpoint{0.708474in}{0.418699in}}%
\pgfpathlineto{\pgfqpoint{0.708627in}{0.405773in}}%
\pgfpathlineto{\pgfqpoint{0.708849in}{0.392846in}}%
\pgfpathlineto{\pgfqpoint{0.709321in}{0.379920in}}%
\pgfpathlineto{\pgfqpoint{0.710848in}{0.366993in}}%
\pgfpathlineto{\pgfqpoint{0.716095in}{0.359096in}}%
\pgfpathlineto{\pgfqpoint{0.729364in}{0.356385in}}%
\pgfpathlineto{\pgfqpoint{0.742633in}{0.356061in}}%
\pgfpathlineto{\pgfqpoint{0.755902in}{0.356901in}}%
\pgfpathlineto{\pgfqpoint{0.765353in}{0.366993in}}%
\pgfpathlineto{\pgfqpoint{0.766370in}{0.379920in}}%
\pgfpathlineto{\pgfqpoint{0.766736in}{0.392846in}}%
\pgfpathlineto{\pgfqpoint{0.767032in}{0.405773in}}%
\pgfpathlineto{\pgfqpoint{0.767575in}{0.418699in}}%
\pgfpathlineto{\pgfqpoint{0.769170in}{0.428033in}}%
\pgfpathlineto{\pgfqpoint{0.772020in}{0.431626in}}%
\pgfpathlineto{\pgfqpoint{0.782439in}{0.433969in}}%
\pgfpathlineto{\pgfqpoint{0.795708in}{0.434218in}}%
\pgfpathlineto{\pgfqpoint{0.808977in}{0.433640in}}%
\pgfpathlineto{\pgfqpoint{0.817607in}{0.431626in}}%
\pgfpathlineto{\pgfqpoint{0.822246in}{0.427388in}}%
\pgfpathlineto{\pgfqpoint{0.824411in}{0.418699in}}%
\pgfpathlineto{\pgfqpoint{0.825238in}{0.405773in}}%
\pgfpathlineto{\pgfqpoint{0.825574in}{0.392846in}}%
\pgfpathlineto{\pgfqpoint{0.825817in}{0.379920in}}%
\pgfpathlineto{\pgfqpoint{0.826247in}{0.366993in}}%
\pgfpathlineto{\pgfqpoint{0.835515in}{0.355158in}}%
\pgfpathlineto{\pgfqpoint{0.848784in}{0.354622in}}%
\pgfpathlineto{\pgfqpoint{0.862053in}{0.354468in}}%
\pgfpathlineto{\pgfqpoint{0.875322in}{0.354296in}}%
\pgfpathlineto{\pgfqpoint{0.880121in}{0.354067in}}%
\pgfpathlineto{\pgfqpoint{0.885200in}{0.341140in}}%
\pgfpathlineto{\pgfqpoint{0.885371in}{0.328214in}}%
\pgfpathlineto{\pgfqpoint{0.885582in}{0.315287in}}%
\pgfpathlineto{\pgfqpoint{0.886000in}{0.302361in}}%
\pgfpathlineto{\pgfqpoint{0.887165in}{0.289434in}}%
\pgfpathlineto{\pgfqpoint{0.888591in}{0.284103in}}%
\pgfpathlineto{\pgfqpoint{0.898721in}{0.276508in}}%
\pgfpathclose%
\pgfpathmoveto{\pgfqpoint{0.898219in}{0.289434in}}%
\pgfpathlineto{\pgfqpoint{0.892980in}{0.302361in}}%
\pgfpathlineto{\pgfqpoint{0.891811in}{0.315287in}}%
\pgfpathlineto{\pgfqpoint{0.892752in}{0.328214in}}%
\pgfpathlineto{\pgfqpoint{0.898251in}{0.341140in}}%
\pgfpathlineto{\pgfqpoint{0.901860in}{0.343994in}}%
\pgfpathlineto{\pgfqpoint{0.915129in}{0.346399in}}%
\pgfpathlineto{\pgfqpoint{0.928398in}{0.343547in}}%
\pgfpathlineto{\pgfqpoint{0.931181in}{0.341140in}}%
\pgfpathlineto{\pgfqpoint{0.936318in}{0.328214in}}%
\pgfpathlineto{\pgfqpoint{0.937174in}{0.315287in}}%
\pgfpathlineto{\pgfqpoint{0.936072in}{0.302361in}}%
\pgfpathlineto{\pgfqpoint{0.931128in}{0.289434in}}%
\pgfpathlineto{\pgfqpoint{0.928398in}{0.286717in}}%
\pgfpathlineto{\pgfqpoint{0.915129in}{0.283259in}}%
\pgfpathlineto{\pgfqpoint{0.901860in}{0.286118in}}%
\pgfpathclose%
\pgfpathmoveto{\pgfqpoint{1.015027in}{0.289434in}}%
\pgfpathlineto{\pgfqpoint{1.009923in}{0.302361in}}%
\pgfpathlineto{\pgfqpoint{1.008764in}{0.315287in}}%
\pgfpathlineto{\pgfqpoint{1.009621in}{0.328214in}}%
\pgfpathlineto{\pgfqpoint{1.014795in}{0.341140in}}%
\pgfpathlineto{\pgfqpoint{1.021280in}{0.345681in}}%
\pgfpathlineto{\pgfqpoint{1.034549in}{0.347181in}}%
\pgfpathlineto{\pgfqpoint{1.047818in}{0.343744in}}%
\pgfpathlineto{\pgfqpoint{1.050463in}{0.341140in}}%
\pgfpathlineto{\pgfqpoint{1.054907in}{0.328214in}}%
\pgfpathlineto{\pgfqpoint{1.055632in}{0.315287in}}%
\pgfpathlineto{\pgfqpoint{1.054631in}{0.302361in}}%
\pgfpathlineto{\pgfqpoint{1.050215in}{0.289434in}}%
\pgfpathlineto{\pgfqpoint{1.047818in}{0.286744in}}%
\pgfpathlineto{\pgfqpoint{1.034549in}{0.282512in}}%
\pgfpathlineto{\pgfqpoint{1.021280in}{0.284361in}}%
\pgfpathclose%
\pgfpathmoveto{\pgfqpoint{1.132431in}{0.289434in}}%
\pgfpathlineto{\pgfqpoint{1.127432in}{0.301349in}}%
\pgfpathlineto{\pgfqpoint{1.127222in}{0.302361in}}%
\pgfpathlineto{\pgfqpoint{1.126311in}{0.315287in}}%
\pgfpathlineto{\pgfqpoint{1.126958in}{0.328214in}}%
\pgfpathlineto{\pgfqpoint{1.127432in}{0.330663in}}%
\pgfpathlineto{\pgfqpoint{1.132069in}{0.341140in}}%
\pgfpathlineto{\pgfqpoint{1.140701in}{0.346485in}}%
\pgfpathlineto{\pgfqpoint{1.153970in}{0.347354in}}%
\pgfpathlineto{\pgfqpoint{1.167239in}{0.342962in}}%
\pgfpathlineto{\pgfqpoint{1.168853in}{0.341140in}}%
\pgfpathlineto{\pgfqpoint{1.173033in}{0.328214in}}%
\pgfpathlineto{\pgfqpoint{1.173713in}{0.315287in}}%
\pgfpathlineto{\pgfqpoint{1.172756in}{0.302361in}}%
\pgfpathlineto{\pgfqpoint{1.168565in}{0.289434in}}%
\pgfpathlineto{\pgfqpoint{1.167239in}{0.287746in}}%
\pgfpathlineto{\pgfqpoint{1.153970in}{0.282373in}}%
\pgfpathlineto{\pgfqpoint{1.140701in}{0.283470in}}%
\pgfpathclose%
\pgfpathmoveto{\pgfqpoint{1.250540in}{0.289434in}}%
\pgfpathlineto{\pgfqpoint{1.246852in}{0.296263in}}%
\pgfpathlineto{\pgfqpoint{1.245345in}{0.302361in}}%
\pgfpathlineto{\pgfqpoint{1.244415in}{0.315287in}}%
\pgfpathlineto{\pgfqpoint{1.245088in}{0.328214in}}%
\pgfpathlineto{\pgfqpoint{1.246852in}{0.335517in}}%
\pgfpathlineto{\pgfqpoint{1.250212in}{0.341140in}}%
\pgfpathlineto{\pgfqpoint{1.260121in}{0.346546in}}%
\pgfpathlineto{\pgfqpoint{1.273390in}{0.346895in}}%
\pgfpathlineto{\pgfqpoint{1.286288in}{0.341140in}}%
\pgfpathlineto{\pgfqpoint{1.286659in}{0.340735in}}%
\pgfpathlineto{\pgfqpoint{1.290723in}{0.328214in}}%
\pgfpathlineto{\pgfqpoint{1.291434in}{0.315287in}}%
\pgfpathlineto{\pgfqpoint{1.290474in}{0.302361in}}%
\pgfpathlineto{\pgfqpoint{1.286659in}{0.290299in}}%
\pgfpathlineto{\pgfqpoint{1.285975in}{0.289434in}}%
\pgfpathlineto{\pgfqpoint{1.273390in}{0.282862in}}%
\pgfpathlineto{\pgfqpoint{1.260121in}{0.283318in}}%
\pgfpathclose%
\pgfpathmoveto{\pgfqpoint{1.369538in}{0.289434in}}%
\pgfpathlineto{\pgfqpoint{1.366273in}{0.294203in}}%
\pgfpathlineto{\pgfqpoint{1.363922in}{0.302361in}}%
\pgfpathlineto{\pgfqpoint{1.362927in}{0.315287in}}%
\pgfpathlineto{\pgfqpoint{1.363700in}{0.328214in}}%
\pgfpathlineto{\pgfqpoint{1.366273in}{0.337065in}}%
\pgfpathlineto{\pgfqpoint{1.369466in}{0.341140in}}%
\pgfpathlineto{\pgfqpoint{1.379542in}{0.345950in}}%
\pgfpathlineto{\pgfqpoint{1.392811in}{0.345756in}}%
\pgfpathlineto{\pgfqpoint{1.401831in}{0.341140in}}%
\pgfpathlineto{\pgfqpoint{1.406080in}{0.335153in}}%
\pgfpathlineto{\pgfqpoint{1.407988in}{0.328214in}}%
\pgfpathlineto{\pgfqpoint{1.408800in}{0.315287in}}%
\pgfpathlineto{\pgfqpoint{1.407792in}{0.302361in}}%
\pgfpathlineto{\pgfqpoint{1.406080in}{0.296128in}}%
\pgfpathlineto{\pgfqpoint{1.401872in}{0.289434in}}%
\pgfpathlineto{\pgfqpoint{1.392811in}{0.284025in}}%
\pgfpathlineto{\pgfqpoint{1.379542in}{0.283823in}}%
\pgfpathclose%
\pgfpathmoveto{\pgfqpoint{0.725460in}{0.366993in}}%
\pgfpathlineto{\pgfqpoint{0.716133in}{0.379920in}}%
\pgfpathlineto{\pgfqpoint{0.716095in}{0.380113in}}%
\pgfpathlineto{\pgfqpoint{0.714565in}{0.392846in}}%
\pgfpathlineto{\pgfqpoint{0.714856in}{0.405773in}}%
\pgfpathlineto{\pgfqpoint{0.716095in}{0.412524in}}%
\pgfpathlineto{\pgfqpoint{0.718464in}{0.418699in}}%
\pgfpathlineto{\pgfqpoint{0.729364in}{0.426817in}}%
\pgfpathlineto{\pgfqpoint{0.742633in}{0.427592in}}%
\pgfpathlineto{\pgfqpoint{0.755902in}{0.421559in}}%
\pgfpathlineto{\pgfqpoint{0.757655in}{0.418699in}}%
\pgfpathlineto{\pgfqpoint{0.760482in}{0.405773in}}%
\pgfpathlineto{\pgfqpoint{0.760739in}{0.392846in}}%
\pgfpathlineto{\pgfqpoint{0.759202in}{0.379920in}}%
\pgfpathlineto{\pgfqpoint{0.755902in}{0.371725in}}%
\pgfpathlineto{\pgfqpoint{0.750138in}{0.366993in}}%
\pgfpathlineto{\pgfqpoint{0.742633in}{0.364182in}}%
\pgfpathlineto{\pgfqpoint{0.729364in}{0.364982in}}%
\pgfpathclose%
\pgfpathmoveto{\pgfqpoint{0.839867in}{0.366993in}}%
\pgfpathlineto{\pgfqpoint{0.835515in}{0.371753in}}%
\pgfpathlineto{\pgfqpoint{0.832709in}{0.379920in}}%
\pgfpathlineto{\pgfqpoint{0.831363in}{0.392846in}}%
\pgfpathlineto{\pgfqpoint{0.831548in}{0.405773in}}%
\pgfpathlineto{\pgfqpoint{0.833864in}{0.418699in}}%
\pgfpathlineto{\pgfqpoint{0.835515in}{0.421939in}}%
\pgfpathlineto{\pgfqpoint{0.848784in}{0.429039in}}%
\pgfpathlineto{\pgfqpoint{0.862053in}{0.429208in}}%
\pgfpathlineto{\pgfqpoint{0.875322in}{0.422924in}}%
\pgfpathlineto{\pgfqpoint{0.877526in}{0.418699in}}%
\pgfpathlineto{\pgfqpoint{0.879685in}{0.405773in}}%
\pgfpathlineto{\pgfqpoint{0.879838in}{0.392846in}}%
\pgfpathlineto{\pgfqpoint{0.878542in}{0.379920in}}%
\pgfpathlineto{\pgfqpoint{0.875322in}{0.370699in}}%
\pgfpathlineto{\pgfqpoint{0.871769in}{0.366993in}}%
\pgfpathlineto{\pgfqpoint{0.862053in}{0.362797in}}%
\pgfpathlineto{\pgfqpoint{0.848784in}{0.362946in}}%
\pgfpathclose%
\pgfpathmoveto{\pgfqpoint{0.885580in}{0.366993in}}%
\pgfpathlineto{\pgfqpoint{0.885567in}{0.379920in}}%
\pgfpathlineto{\pgfqpoint{0.885720in}{0.392846in}}%
\pgfpathlineto{\pgfqpoint{0.886107in}{0.405773in}}%
\pgfpathlineto{\pgfqpoint{0.887232in}{0.418699in}}%
\pgfpathlineto{\pgfqpoint{0.888591in}{0.423970in}}%
\pgfpathlineto{\pgfqpoint{0.898155in}{0.431626in}}%
\pgfpathlineto{\pgfqpoint{0.901860in}{0.432389in}}%
\pgfpathlineto{\pgfqpoint{0.915129in}{0.432940in}}%
\pgfpathlineto{\pgfqpoint{0.928398in}{0.431856in}}%
\pgfpathlineto{\pgfqpoint{0.929305in}{0.431626in}}%
\pgfpathlineto{\pgfqpoint{0.940791in}{0.418699in}}%
\pgfpathlineto{\pgfqpoint{0.941667in}{0.412126in}}%
\pgfpathlineto{\pgfqpoint{0.942185in}{0.405773in}}%
\pgfpathlineto{\pgfqpoint{0.942607in}{0.392846in}}%
\pgfpathlineto{\pgfqpoint{0.942687in}{0.379920in}}%
\pgfpathlineto{\pgfqpoint{0.942327in}{0.366993in}}%
\pgfpathlineto{\pgfqpoint{0.941667in}{0.361918in}}%
\pgfpathlineto{\pgfqpoint{0.928398in}{0.355104in}}%
\pgfpathlineto{\pgfqpoint{0.915129in}{0.354777in}}%
\pgfpathlineto{\pgfqpoint{0.901860in}{0.354736in}}%
\pgfpathlineto{\pgfqpoint{0.888591in}{0.355334in}}%
\pgfpathclose%
\pgfpathmoveto{\pgfqpoint{0.954760in}{0.366993in}}%
\pgfpathlineto{\pgfqpoint{0.949718in}{0.379920in}}%
\pgfpathlineto{\pgfqpoint{0.948508in}{0.392846in}}%
\pgfpathlineto{\pgfqpoint{0.948620in}{0.405773in}}%
\pgfpathlineto{\pgfqpoint{0.950519in}{0.418699in}}%
\pgfpathlineto{\pgfqpoint{0.954936in}{0.426080in}}%
\pgfpathlineto{\pgfqpoint{0.968205in}{0.430468in}}%
\pgfpathlineto{\pgfqpoint{0.981473in}{0.430225in}}%
\pgfpathlineto{\pgfqpoint{0.994742in}{0.423190in}}%
\pgfpathlineto{\pgfqpoint{0.996689in}{0.418699in}}%
\pgfpathlineto{\pgfqpoint{0.998435in}{0.405773in}}%
\pgfpathlineto{\pgfqpoint{0.998523in}{0.392846in}}%
\pgfpathlineto{\pgfqpoint{0.997377in}{0.379920in}}%
\pgfpathlineto{\pgfqpoint{0.994742in}{0.371069in}}%
\pgfpathlineto{\pgfqpoint{0.991684in}{0.366993in}}%
\pgfpathlineto{\pgfqpoint{0.981473in}{0.361965in}}%
\pgfpathlineto{\pgfqpoint{0.968205in}{0.361623in}}%
\pgfpathlineto{\pgfqpoint{0.954936in}{0.366799in}}%
\pgfpathclose%
\pgfpathmoveto{\pgfqpoint{1.004946in}{0.366993in}}%
\pgfpathlineto{\pgfqpoint{1.004310in}{0.379920in}}%
\pgfpathlineto{\pgfqpoint{1.004331in}{0.392846in}}%
\pgfpathlineto{\pgfqpoint{1.004774in}{0.405773in}}%
\pgfpathlineto{\pgfqpoint{1.006251in}{0.418699in}}%
\pgfpathlineto{\pgfqpoint{1.008011in}{0.423849in}}%
\pgfpathlineto{\pgfqpoint{1.021280in}{0.431586in}}%
\pgfpathlineto{\pgfqpoint{1.022010in}{0.431626in}}%
\pgfpathlineto{\pgfqpoint{1.034549in}{0.432157in}}%
\pgfpathlineto{\pgfqpoint{1.040016in}{0.431626in}}%
\pgfpathlineto{\pgfqpoint{1.047818in}{0.430441in}}%
\pgfpathlineto{\pgfqpoint{1.057728in}{0.418699in}}%
\pgfpathlineto{\pgfqpoint{1.059407in}{0.405773in}}%
\pgfpathlineto{\pgfqpoint{1.059898in}{0.392846in}}%
\pgfpathlineto{\pgfqpoint{1.059884in}{0.379920in}}%
\pgfpathlineto{\pgfqpoint{1.059030in}{0.366993in}}%
\pgfpathlineto{\pgfqpoint{1.047818in}{0.356317in}}%
\pgfpathlineto{\pgfqpoint{1.034549in}{0.355611in}}%
\pgfpathlineto{\pgfqpoint{1.021280in}{0.355784in}}%
\pgfpathlineto{\pgfqpoint{1.008011in}{0.358793in}}%
\pgfpathclose%
\pgfpathmoveto{\pgfqpoint{1.071890in}{0.366993in}}%
\pgfpathlineto{\pgfqpoint{1.067137in}{0.379920in}}%
\pgfpathlineto{\pgfqpoint{1.065987in}{0.392846in}}%
\pgfpathlineto{\pgfqpoint{1.066058in}{0.405773in}}%
\pgfpathlineto{\pgfqpoint{1.067746in}{0.418699in}}%
\pgfpathlineto{\pgfqpoint{1.074356in}{0.428259in}}%
\pgfpathlineto{\pgfqpoint{1.087625in}{0.431193in}}%
\pgfpathlineto{\pgfqpoint{1.100894in}{0.430599in}}%
\pgfpathlineto{\pgfqpoint{1.114163in}{0.421625in}}%
\pgfpathlineto{\pgfqpoint{1.115178in}{0.418699in}}%
\pgfpathlineto{\pgfqpoint{1.116745in}{0.405773in}}%
\pgfpathlineto{\pgfqpoint{1.116805in}{0.392846in}}%
\pgfpathlineto{\pgfqpoint{1.115725in}{0.379920in}}%
\pgfpathlineto{\pgfqpoint{1.114163in}{0.373606in}}%
\pgfpathlineto{\pgfqpoint{1.110346in}{0.366993in}}%
\pgfpathlineto{\pgfqpoint{1.100894in}{0.361724in}}%
\pgfpathlineto{\pgfqpoint{1.087625in}{0.360940in}}%
\pgfpathlineto{\pgfqpoint{1.074356in}{0.364616in}}%
\pgfpathclose%
\pgfpathmoveto{\pgfqpoint{1.123501in}{0.366993in}}%
\pgfpathlineto{\pgfqpoint{1.122611in}{0.379920in}}%
\pgfpathlineto{\pgfqpoint{1.122578in}{0.392846in}}%
\pgfpathlineto{\pgfqpoint{1.123043in}{0.405773in}}%
\pgfpathlineto{\pgfqpoint{1.124663in}{0.418699in}}%
\pgfpathlineto{\pgfqpoint{1.127432in}{0.425230in}}%
\pgfpathlineto{\pgfqpoint{1.140701in}{0.431419in}}%
\pgfpathlineto{\pgfqpoint{1.146250in}{0.431626in}}%
\pgfpathlineto{\pgfqpoint{1.153970in}{0.431865in}}%
\pgfpathlineto{\pgfqpoint{1.156179in}{0.431626in}}%
\pgfpathlineto{\pgfqpoint{1.167239in}{0.429599in}}%
\pgfpathlineto{\pgfqpoint{1.175442in}{0.418699in}}%
\pgfpathlineto{\pgfqpoint{1.177142in}{0.405773in}}%
\pgfpathlineto{\pgfqpoint{1.177630in}{0.392846in}}%
\pgfpathlineto{\pgfqpoint{1.177595in}{0.379920in}}%
\pgfpathlineto{\pgfqpoint{1.176661in}{0.366993in}}%
\pgfpathlineto{\pgfqpoint{1.167239in}{0.356857in}}%
\pgfpathlineto{\pgfqpoint{1.153970in}{0.355897in}}%
\pgfpathlineto{\pgfqpoint{1.140701in}{0.356087in}}%
\pgfpathlineto{\pgfqpoint{1.127432in}{0.358907in}}%
\pgfpathclose%
\pgfpathmoveto{\pgfqpoint{1.189827in}{0.366993in}}%
\pgfpathlineto{\pgfqpoint{1.184969in}{0.379920in}}%
\pgfpathlineto{\pgfqpoint{1.183797in}{0.392846in}}%
\pgfpathlineto{\pgfqpoint{1.183862in}{0.405773in}}%
\pgfpathlineto{\pgfqpoint{1.185563in}{0.418699in}}%
\pgfpathlineto{\pgfqpoint{1.193777in}{0.429092in}}%
\pgfpathlineto{\pgfqpoint{1.207045in}{0.431266in}}%
\pgfpathlineto{\pgfqpoint{1.220314in}{0.430253in}}%
\pgfpathlineto{\pgfqpoint{1.232871in}{0.418699in}}%
\pgfpathlineto{\pgfqpoint{1.233583in}{0.415167in}}%
\pgfpathlineto{\pgfqpoint{1.234612in}{0.405773in}}%
\pgfpathlineto{\pgfqpoint{1.234679in}{0.392846in}}%
\pgfpathlineto{\pgfqpoint{1.233583in}{0.379921in}}%
\pgfpathlineto{\pgfqpoint{1.233583in}{0.379920in}}%
\pgfpathlineto{\pgfqpoint{1.228032in}{0.366993in}}%
\pgfpathlineto{\pgfqpoint{1.220314in}{0.362144in}}%
\pgfpathlineto{\pgfqpoint{1.207045in}{0.360849in}}%
\pgfpathlineto{\pgfqpoint{1.193777in}{0.363638in}}%
\pgfpathclose%
\pgfpathmoveto{\pgfqpoint{1.241261in}{0.366993in}}%
\pgfpathlineto{\pgfqpoint{1.240467in}{0.379920in}}%
\pgfpathlineto{\pgfqpoint{1.240454in}{0.392846in}}%
\pgfpathlineto{\pgfqpoint{1.240910in}{0.405773in}}%
\pgfpathlineto{\pgfqpoint{1.242470in}{0.418699in}}%
\pgfpathlineto{\pgfqpoint{1.246852in}{0.427365in}}%
\pgfpathlineto{\pgfqpoint{1.258827in}{0.431626in}}%
\pgfpathlineto{\pgfqpoint{1.260121in}{0.431832in}}%
\pgfpathlineto{\pgfqpoint{1.273390in}{0.432079in}}%
\pgfpathlineto{\pgfqpoint{1.277161in}{0.431626in}}%
\pgfpathlineto{\pgfqpoint{1.286659in}{0.429522in}}%
\pgfpathlineto{\pgfqpoint{1.293858in}{0.418699in}}%
\pgfpathlineto{\pgfqpoint{1.295371in}{0.405773in}}%
\pgfpathlineto{\pgfqpoint{1.295824in}{0.392846in}}%
\pgfpathlineto{\pgfqpoint{1.295845in}{0.379920in}}%
\pgfpathlineto{\pgfqpoint{1.295194in}{0.366993in}}%
\pgfpathlineto{\pgfqpoint{1.286659in}{0.356498in}}%
\pgfpathlineto{\pgfqpoint{1.273390in}{0.355618in}}%
\pgfpathlineto{\pgfqpoint{1.260121in}{0.355756in}}%
\pgfpathlineto{\pgfqpoint{1.246852in}{0.357623in}}%
\pgfpathclose%
\pgfpathmoveto{\pgfqpoint{1.308643in}{0.366993in}}%
\pgfpathlineto{\pgfqpoint{1.303239in}{0.379920in}}%
\pgfpathlineto{\pgfqpoint{1.301950in}{0.392846in}}%
\pgfpathlineto{\pgfqpoint{1.302049in}{0.405773in}}%
\pgfpathlineto{\pgfqpoint{1.304014in}{0.418699in}}%
\pgfpathlineto{\pgfqpoint{1.313197in}{0.428936in}}%
\pgfpathlineto{\pgfqpoint{1.326466in}{0.430715in}}%
\pgfpathlineto{\pgfqpoint{1.339735in}{0.429058in}}%
\pgfpathlineto{\pgfqpoint{1.349714in}{0.418699in}}%
\pgfpathlineto{\pgfqpoint{1.351865in}{0.405773in}}%
\pgfpathlineto{\pgfqpoint{1.351991in}{0.392846in}}%
\pgfpathlineto{\pgfqpoint{1.350621in}{0.379920in}}%
\pgfpathlineto{\pgfqpoint{1.344912in}{0.366993in}}%
\pgfpathlineto{\pgfqpoint{1.339735in}{0.363334in}}%
\pgfpathlineto{\pgfqpoint{1.326466in}{0.361329in}}%
\pgfpathlineto{\pgfqpoint{1.313197in}{0.363569in}}%
\pgfpathclose%
\pgfpathmoveto{\pgfqpoint{1.358211in}{0.366993in}}%
\pgfpathlineto{\pgfqpoint{1.357858in}{0.379920in}}%
\pgfpathlineto{\pgfqpoint{1.357938in}{0.392846in}}%
\pgfpathlineto{\pgfqpoint{1.358353in}{0.405773in}}%
\pgfpathlineto{\pgfqpoint{1.359655in}{0.418699in}}%
\pgfpathlineto{\pgfqpoint{1.366273in}{0.429957in}}%
\pgfpathlineto{\pgfqpoint{1.372002in}{0.431626in}}%
\pgfpathlineto{\pgfqpoint{1.379542in}{0.432730in}}%
\pgfpathlineto{\pgfqpoint{1.392811in}{0.432831in}}%
\pgfpathlineto{\pgfqpoint{1.401932in}{0.431626in}}%
\pgfpathlineto{\pgfqpoint{1.406080in}{0.430507in}}%
\pgfpathlineto{\pgfqpoint{1.412930in}{0.418699in}}%
\pgfpathlineto{\pgfqpoint{1.414054in}{0.405773in}}%
\pgfpathlineto{\pgfqpoint{1.414441in}{0.392846in}}%
\pgfpathlineto{\pgfqpoint{1.414592in}{0.379920in}}%
\pgfpathlineto{\pgfqpoint{1.414579in}{0.366993in}}%
\pgfpathlineto{\pgfqpoint{1.406080in}{0.354837in}}%
\pgfpathlineto{\pgfqpoint{1.392811in}{0.354731in}}%
\pgfpathlineto{\pgfqpoint{1.379542in}{0.354860in}}%
\pgfpathlineto{\pgfqpoint{1.366273in}{0.355590in}}%
\pgfpathclose%
\pgfpathmoveto{\pgfqpoint{1.428467in}{0.366993in}}%
\pgfpathlineto{\pgfqpoint{1.421990in}{0.379920in}}%
\pgfpathlineto{\pgfqpoint{1.420478in}{0.392846in}}%
\pgfpathlineto{\pgfqpoint{1.420655in}{0.405773in}}%
\pgfpathlineto{\pgfqpoint{1.423179in}{0.418699in}}%
\pgfpathlineto{\pgfqpoint{1.432617in}{0.428003in}}%
\pgfpathlineto{\pgfqpoint{1.445886in}{0.429539in}}%
\pgfpathlineto{\pgfqpoint{1.459155in}{0.426817in}}%
\pgfpathlineto{\pgfqpoint{1.466073in}{0.418699in}}%
\pgfpathlineto{\pgfqpoint{1.468623in}{0.405773in}}%
\pgfpathlineto{\pgfqpoint{1.468826in}{0.392846in}}%
\pgfpathlineto{\pgfqpoint{1.467344in}{0.379920in}}%
\pgfpathlineto{\pgfqpoint{1.461083in}{0.366993in}}%
\pgfpathlineto{\pgfqpoint{1.459155in}{0.365460in}}%
\pgfpathlineto{\pgfqpoint{1.445886in}{0.362379in}}%
\pgfpathlineto{\pgfqpoint{1.432617in}{0.364225in}}%
\pgfpathclose%
\pgfpathmoveto{\pgfqpoint{0.672693in}{0.444552in}}%
\pgfpathlineto{\pgfqpoint{0.663019in}{0.448342in}}%
\pgfpathlineto{\pgfqpoint{0.657461in}{0.457479in}}%
\pgfpathlineto{\pgfqpoint{0.655262in}{0.470405in}}%
\pgfpathlineto{\pgfqpoint{0.655081in}{0.483332in}}%
\pgfpathlineto{\pgfqpoint{0.656815in}{0.496258in}}%
\pgfpathlineto{\pgfqpoint{0.663019in}{0.506511in}}%
\pgfpathlineto{\pgfqpoint{0.670517in}{0.509185in}}%
\pgfpathlineto{\pgfqpoint{0.676288in}{0.510365in}}%
\pgfpathlineto{\pgfqpoint{0.689557in}{0.509380in}}%
\pgfpathlineto{\pgfqpoint{0.690079in}{0.509185in}}%
\pgfpathlineto{\pgfqpoint{0.701329in}{0.496258in}}%
\pgfpathlineto{\pgfqpoint{0.702826in}{0.486020in}}%
\pgfpathlineto{\pgfqpoint{0.703060in}{0.483332in}}%
\pgfpathlineto{\pgfqpoint{0.702878in}{0.470405in}}%
\pgfpathlineto{\pgfqpoint{0.702826in}{0.469946in}}%
\pgfpathlineto{\pgfqpoint{0.700511in}{0.457479in}}%
\pgfpathlineto{\pgfqpoint{0.689557in}{0.445074in}}%
\pgfpathlineto{\pgfqpoint{0.685431in}{0.444552in}}%
\pgfpathlineto{\pgfqpoint{0.676288in}{0.443739in}}%
\pgfpathclose%
\pgfpathmoveto{\pgfqpoint{0.708884in}{0.444552in}}%
\pgfpathlineto{\pgfqpoint{0.708774in}{0.457479in}}%
\pgfpathlineto{\pgfqpoint{0.708865in}{0.470405in}}%
\pgfpathlineto{\pgfqpoint{0.709130in}{0.483332in}}%
\pgfpathlineto{\pgfqpoint{0.709843in}{0.496258in}}%
\pgfpathlineto{\pgfqpoint{0.713377in}{0.509185in}}%
\pgfpathlineto{\pgfqpoint{0.716095in}{0.511512in}}%
\pgfpathlineto{\pgfqpoint{0.729364in}{0.514049in}}%
\pgfpathlineto{\pgfqpoint{0.742633in}{0.513994in}}%
\pgfpathlineto{\pgfqpoint{0.755902in}{0.511712in}}%
\pgfpathlineto{\pgfqpoint{0.759577in}{0.509185in}}%
\pgfpathlineto{\pgfqpoint{0.764166in}{0.496258in}}%
\pgfpathlineto{\pgfqpoint{0.765105in}{0.483332in}}%
\pgfpathlineto{\pgfqpoint{0.765380in}{0.470405in}}%
\pgfpathlineto{\pgfqpoint{0.765272in}{0.457479in}}%
\pgfpathlineto{\pgfqpoint{0.764152in}{0.444552in}}%
\pgfpathlineto{\pgfqpoint{0.755902in}{0.437250in}}%
\pgfpathlineto{\pgfqpoint{0.742633in}{0.436324in}}%
\pgfpathlineto{\pgfqpoint{0.729364in}{0.436136in}}%
\pgfpathlineto{\pgfqpoint{0.716095in}{0.436260in}}%
\pgfpathclose%
\pgfpathmoveto{\pgfqpoint{0.782021in}{0.444552in}}%
\pgfpathlineto{\pgfqpoint{0.773395in}{0.457479in}}%
\pgfpathlineto{\pgfqpoint{0.771576in}{0.470405in}}%
\pgfpathlineto{\pgfqpoint{0.771366in}{0.483332in}}%
\pgfpathlineto{\pgfqpoint{0.772618in}{0.496258in}}%
\pgfpathlineto{\pgfqpoint{0.780142in}{0.509185in}}%
\pgfpathlineto{\pgfqpoint{0.782439in}{0.510369in}}%
\pgfpathlineto{\pgfqpoint{0.795708in}{0.512289in}}%
\pgfpathlineto{\pgfqpoint{0.808977in}{0.511257in}}%
\pgfpathlineto{\pgfqpoint{0.813835in}{0.509185in}}%
\pgfpathlineto{\pgfqpoint{0.821156in}{0.496258in}}%
\pgfpathlineto{\pgfqpoint{0.822246in}{0.484072in}}%
\pgfpathlineto{\pgfqpoint{0.822288in}{0.483332in}}%
\pgfpathlineto{\pgfqpoint{0.822246in}{0.480645in}}%
\pgfpathlineto{\pgfqpoint{0.822062in}{0.470405in}}%
\pgfpathlineto{\pgfqpoint{0.820282in}{0.457479in}}%
\pgfpathlineto{\pgfqpoint{0.811539in}{0.444552in}}%
\pgfpathlineto{\pgfqpoint{0.808977in}{0.443336in}}%
\pgfpathlineto{\pgfqpoint{0.795708in}{0.441892in}}%
\pgfpathlineto{\pgfqpoint{0.782439in}{0.444316in}}%
\pgfpathclose%
\pgfpathmoveto{\pgfqpoint{0.830588in}{0.444552in}}%
\pgfpathlineto{\pgfqpoint{0.828411in}{0.457479in}}%
\pgfpathlineto{\pgfqpoint{0.828111in}{0.470405in}}%
\pgfpathlineto{\pgfqpoint{0.828396in}{0.483332in}}%
\pgfpathlineto{\pgfqpoint{0.829573in}{0.496258in}}%
\pgfpathlineto{\pgfqpoint{0.835515in}{0.508983in}}%
\pgfpathlineto{\pgfqpoint{0.835881in}{0.509185in}}%
\pgfpathlineto{\pgfqpoint{0.848784in}{0.512472in}}%
\pgfpathlineto{\pgfqpoint{0.862053in}{0.512318in}}%
\pgfpathlineto{\pgfqpoint{0.873315in}{0.509185in}}%
\pgfpathlineto{\pgfqpoint{0.875322in}{0.507970in}}%
\pgfpathlineto{\pgfqpoint{0.880651in}{0.496258in}}%
\pgfpathlineto{\pgfqpoint{0.881957in}{0.483332in}}%
\pgfpathlineto{\pgfqpoint{0.882243in}{0.470405in}}%
\pgfpathlineto{\pgfqpoint{0.881815in}{0.457479in}}%
\pgfpathlineto{\pgfqpoint{0.879060in}{0.444552in}}%
\pgfpathlineto{\pgfqpoint{0.875322in}{0.440728in}}%
\pgfpathlineto{\pgfqpoint{0.862053in}{0.438043in}}%
\pgfpathlineto{\pgfqpoint{0.848784in}{0.437882in}}%
\pgfpathlineto{\pgfqpoint{0.835515in}{0.439764in}}%
\pgfpathclose%
\pgfpathmoveto{\pgfqpoint{0.896904in}{0.444552in}}%
\pgfpathlineto{\pgfqpoint{0.889645in}{0.457479in}}%
\pgfpathlineto{\pgfqpoint{0.888591in}{0.465520in}}%
\pgfpathlineto{\pgfqpoint{0.888165in}{0.470405in}}%
\pgfpathlineto{\pgfqpoint{0.887960in}{0.483332in}}%
\pgfpathlineto{\pgfqpoint{0.888591in}{0.494470in}}%
\pgfpathlineto{\pgfqpoint{0.888749in}{0.496258in}}%
\pgfpathlineto{\pgfqpoint{0.894345in}{0.509185in}}%
\pgfpathlineto{\pgfqpoint{0.901860in}{0.512598in}}%
\pgfpathlineto{\pgfqpoint{0.915129in}{0.513629in}}%
\pgfpathlineto{\pgfqpoint{0.928398in}{0.512611in}}%
\pgfpathlineto{\pgfqpoint{0.935445in}{0.509185in}}%
\pgfpathlineto{\pgfqpoint{0.940291in}{0.496258in}}%
\pgfpathlineto{\pgfqpoint{0.941017in}{0.483332in}}%
\pgfpathlineto{\pgfqpoint{0.940782in}{0.470405in}}%
\pgfpathlineto{\pgfqpoint{0.939392in}{0.457479in}}%
\pgfpathlineto{\pgfqpoint{0.932739in}{0.444552in}}%
\pgfpathlineto{\pgfqpoint{0.928398in}{0.442220in}}%
\pgfpathlineto{\pgfqpoint{0.915129in}{0.440614in}}%
\pgfpathlineto{\pgfqpoint{0.901860in}{0.442071in}}%
\pgfpathclose%
\pgfpathmoveto{\pgfqpoint{0.951146in}{0.444552in}}%
\pgfpathlineto{\pgfqpoint{0.947531in}{0.457479in}}%
\pgfpathlineto{\pgfqpoint{0.946956in}{0.470405in}}%
\pgfpathlineto{\pgfqpoint{0.947257in}{0.483332in}}%
\pgfpathlineto{\pgfqpoint{0.948767in}{0.496258in}}%
\pgfpathlineto{\pgfqpoint{0.954936in}{0.507789in}}%
\pgfpathlineto{\pgfqpoint{0.958056in}{0.509185in}}%
\pgfpathlineto{\pgfqpoint{0.968205in}{0.511494in}}%
\pgfpathlineto{\pgfqpoint{0.981473in}{0.511083in}}%
\pgfpathlineto{\pgfqpoint{0.987640in}{0.509185in}}%
\pgfpathlineto{\pgfqpoint{0.994742in}{0.503838in}}%
\pgfpathlineto{\pgfqpoint{0.997698in}{0.496258in}}%
\pgfpathlineto{\pgfqpoint{0.999239in}{0.483332in}}%
\pgfpathlineto{\pgfqpoint{0.999533in}{0.470405in}}%
\pgfpathlineto{\pgfqpoint{0.998902in}{0.457479in}}%
\pgfpathlineto{\pgfqpoint{0.995086in}{0.444552in}}%
\pgfpathlineto{\pgfqpoint{0.994742in}{0.444136in}}%
\pgfpathlineto{\pgfqpoint{0.981473in}{0.439284in}}%
\pgfpathlineto{\pgfqpoint{0.968205in}{0.438971in}}%
\pgfpathlineto{\pgfqpoint{0.954936in}{0.441366in}}%
\pgfpathclose%
\pgfpathmoveto{\pgfqpoint{1.012692in}{0.444552in}}%
\pgfpathlineto{\pgfqpoint{1.008011in}{0.451570in}}%
\pgfpathlineto{\pgfqpoint{1.006550in}{0.457479in}}%
\pgfpathlineto{\pgfqpoint{1.005420in}{0.470405in}}%
\pgfpathlineto{\pgfqpoint{1.005208in}{0.483332in}}%
\pgfpathlineto{\pgfqpoint{1.005736in}{0.496258in}}%
\pgfpathlineto{\pgfqpoint{1.008011in}{0.506789in}}%
\pgfpathlineto{\pgfqpoint{1.009570in}{0.509185in}}%
\pgfpathlineto{\pgfqpoint{1.021280in}{0.513874in}}%
\pgfpathlineto{\pgfqpoint{1.034549in}{0.514411in}}%
\pgfpathlineto{\pgfqpoint{1.047818in}{0.513350in}}%
\pgfpathlineto{\pgfqpoint{1.055313in}{0.509185in}}%
\pgfpathlineto{\pgfqpoint{1.058836in}{0.496258in}}%
\pgfpathlineto{\pgfqpoint{1.059343in}{0.483332in}}%
\pgfpathlineto{\pgfqpoint{1.059109in}{0.470405in}}%
\pgfpathlineto{\pgfqpoint{1.057938in}{0.457479in}}%
\pgfpathlineto{\pgfqpoint{1.052448in}{0.444552in}}%
\pgfpathlineto{\pgfqpoint{1.047818in}{0.441730in}}%
\pgfpathlineto{\pgfqpoint{1.034549in}{0.439881in}}%
\pgfpathlineto{\pgfqpoint{1.021280in}{0.440742in}}%
\pgfpathclose%
\pgfpathmoveto{\pgfqpoint{1.070568in}{0.444552in}}%
\pgfpathlineto{\pgfqpoint{1.066119in}{0.457479in}}%
\pgfpathlineto{\pgfqpoint{1.065383in}{0.470405in}}%
\pgfpathlineto{\pgfqpoint{1.065696in}{0.483332in}}%
\pgfpathlineto{\pgfqpoint{1.067411in}{0.496258in}}%
\pgfpathlineto{\pgfqpoint{1.074356in}{0.507732in}}%
\pgfpathlineto{\pgfqpoint{1.078356in}{0.509185in}}%
\pgfpathlineto{\pgfqpoint{1.087625in}{0.511052in}}%
\pgfpathlineto{\pgfqpoint{1.100894in}{0.510308in}}%
\pgfpathlineto{\pgfqpoint{1.104203in}{0.509185in}}%
\pgfpathlineto{\pgfqpoint{1.114163in}{0.499641in}}%
\pgfpathlineto{\pgfqpoint{1.115263in}{0.496258in}}%
\pgfpathlineto{\pgfqpoint{1.116914in}{0.483332in}}%
\pgfpathlineto{\pgfqpoint{1.117212in}{0.470405in}}%
\pgfpathlineto{\pgfqpoint{1.116490in}{0.457479in}}%
\pgfpathlineto{\pgfqpoint{1.114163in}{0.448170in}}%
\pgfpathlineto{\pgfqpoint{1.111565in}{0.444552in}}%
\pgfpathlineto{\pgfqpoint{1.100894in}{0.440020in}}%
\pgfpathlineto{\pgfqpoint{1.087625in}{0.439476in}}%
\pgfpathlineto{\pgfqpoint{1.074356in}{0.441751in}}%
\pgfpathclose%
\pgfpathmoveto{\pgfqpoint{1.129513in}{0.444552in}}%
\pgfpathlineto{\pgfqpoint{1.127432in}{0.446868in}}%
\pgfpathlineto{\pgfqpoint{1.124149in}{0.457479in}}%
\pgfpathlineto{\pgfqpoint{1.123102in}{0.470405in}}%
\pgfpathlineto{\pgfqpoint{1.122886in}{0.483332in}}%
\pgfpathlineto{\pgfqpoint{1.123315in}{0.496258in}}%
\pgfpathlineto{\pgfqpoint{1.126324in}{0.509185in}}%
\pgfpathlineto{\pgfqpoint{1.127432in}{0.510528in}}%
\pgfpathlineto{\pgfqpoint{1.140701in}{0.514376in}}%
\pgfpathlineto{\pgfqpoint{1.153970in}{0.514638in}}%
\pgfpathlineto{\pgfqpoint{1.167239in}{0.513316in}}%
\pgfpathlineto{\pgfqpoint{1.173698in}{0.509185in}}%
\pgfpathlineto{\pgfqpoint{1.176856in}{0.496258in}}%
\pgfpathlineto{\pgfqpoint{1.177307in}{0.483332in}}%
\pgfpathlineto{\pgfqpoint{1.177080in}{0.470405in}}%
\pgfpathlineto{\pgfqpoint{1.175981in}{0.457479in}}%
\pgfpathlineto{\pgfqpoint{1.170883in}{0.444552in}}%
\pgfpathlineto{\pgfqpoint{1.167239in}{0.442020in}}%
\pgfpathlineto{\pgfqpoint{1.153970in}{0.439688in}}%
\pgfpathlineto{\pgfqpoint{1.140701in}{0.440163in}}%
\pgfpathclose%
\pgfpathmoveto{\pgfqpoint{1.188816in}{0.444552in}}%
\pgfpathlineto{\pgfqpoint{1.184138in}{0.457479in}}%
\pgfpathlineto{\pgfqpoint{1.183355in}{0.470405in}}%
\pgfpathlineto{\pgfqpoint{1.183678in}{0.483332in}}%
\pgfpathlineto{\pgfqpoint{1.185469in}{0.496258in}}%
\pgfpathlineto{\pgfqpoint{1.193777in}{0.508496in}}%
\pgfpathlineto{\pgfqpoint{1.196122in}{0.509185in}}%
\pgfpathlineto{\pgfqpoint{1.207045in}{0.511107in}}%
\pgfpathlineto{\pgfqpoint{1.220314in}{0.510041in}}%
\pgfpathlineto{\pgfqpoint{1.222603in}{0.509185in}}%
\pgfpathlineto{\pgfqpoint{1.233263in}{0.496258in}}%
\pgfpathlineto{\pgfqpoint{1.233583in}{0.494662in}}%
\pgfpathlineto{\pgfqpoint{1.234957in}{0.483332in}}%
\pgfpathlineto{\pgfqpoint{1.235256in}{0.470405in}}%
\pgfpathlineto{\pgfqpoint{1.234554in}{0.457479in}}%
\pgfpathlineto{\pgfqpoint{1.233583in}{0.452361in}}%
\pgfpathlineto{\pgfqpoint{1.229574in}{0.444552in}}%
\pgfpathlineto{\pgfqpoint{1.220314in}{0.440196in}}%
\pgfpathlineto{\pgfqpoint{1.207045in}{0.439440in}}%
\pgfpathlineto{\pgfqpoint{1.193777in}{0.441291in}}%
\pgfpathclose%
\pgfpathmoveto{\pgfqpoint{1.247587in}{0.444552in}}%
\pgfpathlineto{\pgfqpoint{1.246852in}{0.445179in}}%
\pgfpathlineto{\pgfqpoint{1.242275in}{0.457479in}}%
\pgfpathlineto{\pgfqpoint{1.241187in}{0.470405in}}%
\pgfpathlineto{\pgfqpoint{1.240970in}{0.483332in}}%
\pgfpathlineto{\pgfqpoint{1.241441in}{0.496258in}}%
\pgfpathlineto{\pgfqpoint{1.244710in}{0.509185in}}%
\pgfpathlineto{\pgfqpoint{1.246852in}{0.511271in}}%
\pgfpathlineto{\pgfqpoint{1.260121in}{0.514222in}}%
\pgfpathlineto{\pgfqpoint{1.273390in}{0.514294in}}%
\pgfpathlineto{\pgfqpoint{1.286659in}{0.512238in}}%
\pgfpathlineto{\pgfqpoint{1.290766in}{0.509185in}}%
\pgfpathlineto{\pgfqpoint{1.294387in}{0.496258in}}%
\pgfpathlineto{\pgfqpoint{1.294928in}{0.483332in}}%
\pgfpathlineto{\pgfqpoint{1.294710in}{0.470405in}}%
\pgfpathlineto{\pgfqpoint{1.293554in}{0.457479in}}%
\pgfpathlineto{\pgfqpoint{1.288175in}{0.444552in}}%
\pgfpathlineto{\pgfqpoint{1.286659in}{0.443341in}}%
\pgfpathlineto{\pgfqpoint{1.273390in}{0.440054in}}%
\pgfpathlineto{\pgfqpoint{1.260121in}{0.440226in}}%
\pgfpathclose%
\pgfpathmoveto{\pgfqpoint{1.305803in}{0.444552in}}%
\pgfpathlineto{\pgfqpoint{1.301523in}{0.457479in}}%
\pgfpathlineto{\pgfqpoint{1.300815in}{0.470405in}}%
\pgfpathlineto{\pgfqpoint{1.301145in}{0.483332in}}%
\pgfpathlineto{\pgfqpoint{1.302876in}{0.496258in}}%
\pgfpathlineto{\pgfqpoint{1.311891in}{0.509185in}}%
\pgfpathlineto{\pgfqpoint{1.313197in}{0.509795in}}%
\pgfpathlineto{\pgfqpoint{1.326466in}{0.511644in}}%
\pgfpathlineto{\pgfqpoint{1.339735in}{0.510358in}}%
\pgfpathlineto{\pgfqpoint{1.342578in}{0.509185in}}%
\pgfpathlineto{\pgfqpoint{1.351699in}{0.496258in}}%
\pgfpathlineto{\pgfqpoint{1.353004in}{0.487167in}}%
\pgfpathlineto{\pgfqpoint{1.353358in}{0.483332in}}%
\pgfpathlineto{\pgfqpoint{1.353654in}{0.470405in}}%
\pgfpathlineto{\pgfqpoint{1.353088in}{0.457479in}}%
\pgfpathlineto{\pgfqpoint{1.353004in}{0.456820in}}%
\pgfpathlineto{\pgfqpoint{1.348997in}{0.444552in}}%
\pgfpathlineto{\pgfqpoint{1.339735in}{0.439720in}}%
\pgfpathlineto{\pgfqpoint{1.326466in}{0.438884in}}%
\pgfpathlineto{\pgfqpoint{1.313197in}{0.440196in}}%
\pgfpathclose%
\pgfpathmoveto{\pgfqpoint{1.367287in}{0.444552in}}%
\pgfpathlineto{\pgfqpoint{1.366273in}{0.445226in}}%
\pgfpathlineto{\pgfqpoint{1.360923in}{0.457479in}}%
\pgfpathlineto{\pgfqpoint{1.359667in}{0.470405in}}%
\pgfpathlineto{\pgfqpoint{1.359454in}{0.483332in}}%
\pgfpathlineto{\pgfqpoint{1.360111in}{0.496258in}}%
\pgfpathlineto{\pgfqpoint{1.364482in}{0.509185in}}%
\pgfpathlineto{\pgfqpoint{1.366273in}{0.510633in}}%
\pgfpathlineto{\pgfqpoint{1.379542in}{0.513485in}}%
\pgfpathlineto{\pgfqpoint{1.392811in}{0.513334in}}%
\pgfpathlineto{\pgfqpoint{1.406080in}{0.509648in}}%
\pgfpathlineto{\pgfqpoint{1.406607in}{0.509185in}}%
\pgfpathlineto{\pgfqpoint{1.411439in}{0.496258in}}%
\pgfpathlineto{\pgfqpoint{1.412205in}{0.483332in}}%
\pgfpathlineto{\pgfqpoint{1.412001in}{0.470405in}}%
\pgfpathlineto{\pgfqpoint{1.410666in}{0.457479in}}%
\pgfpathlineto{\pgfqpoint{1.406080in}{0.446415in}}%
\pgfpathlineto{\pgfqpoint{1.403571in}{0.444552in}}%
\pgfpathlineto{\pgfqpoint{1.392811in}{0.441019in}}%
\pgfpathlineto{\pgfqpoint{1.379542in}{0.440862in}}%
\pgfpathclose%
\pgfpathmoveto{\pgfqpoint{1.421383in}{0.444552in}}%
\pgfpathlineto{\pgfqpoint{1.419348in}{0.450303in}}%
\pgfpathlineto{\pgfqpoint{1.418344in}{0.457479in}}%
\pgfpathlineto{\pgfqpoint{1.417916in}{0.470405in}}%
\pgfpathlineto{\pgfqpoint{1.418202in}{0.483332in}}%
\pgfpathlineto{\pgfqpoint{1.419348in}{0.495051in}}%
\pgfpathlineto{\pgfqpoint{1.419533in}{0.496258in}}%
\pgfpathlineto{\pgfqpoint{1.427275in}{0.509185in}}%
\pgfpathlineto{\pgfqpoint{1.432617in}{0.511447in}}%
\pgfpathlineto{\pgfqpoint{1.445886in}{0.512664in}}%
\pgfpathlineto{\pgfqpoint{1.459155in}{0.511379in}}%
\pgfpathlineto{\pgfqpoint{1.463967in}{0.509185in}}%
\pgfpathlineto{\pgfqpoint{1.470798in}{0.496258in}}%
\pgfpathlineto{\pgfqpoint{1.472096in}{0.483332in}}%
\pgfpathlineto{\pgfqpoint{1.472410in}{0.470405in}}%
\pgfpathlineto{\pgfqpoint{1.472080in}{0.457479in}}%
\pgfpathlineto{\pgfqpoint{1.469675in}{0.444552in}}%
\pgfpathlineto{\pgfqpoint{1.459155in}{0.438445in}}%
\pgfpathlineto{\pgfqpoint{1.445886in}{0.437806in}}%
\pgfpathlineto{\pgfqpoint{1.432617in}{0.438589in}}%
\pgfpathclose%
\pgfpathmoveto{\pgfqpoint{0.654323in}{0.522111in}}%
\pgfpathlineto{\pgfqpoint{0.650962in}{0.535038in}}%
\pgfpathlineto{\pgfqpoint{0.650604in}{0.547964in}}%
\pgfpathlineto{\pgfqpoint{0.650744in}{0.560891in}}%
\pgfpathlineto{\pgfqpoint{0.651441in}{0.573817in}}%
\pgfpathlineto{\pgfqpoint{0.654284in}{0.586744in}}%
\pgfpathlineto{\pgfqpoint{0.663019in}{0.593283in}}%
\pgfpathlineto{\pgfqpoint{0.676288in}{0.594502in}}%
\pgfpathlineto{\pgfqpoint{0.689557in}{0.593578in}}%
\pgfpathlineto{\pgfqpoint{0.701457in}{0.586744in}}%
\pgfpathlineto{\pgfqpoint{0.702826in}{0.583908in}}%
\pgfpathlineto{\pgfqpoint{0.704939in}{0.573817in}}%
\pgfpathlineto{\pgfqpoint{0.705726in}{0.560891in}}%
\pgfpathlineto{\pgfqpoint{0.705810in}{0.547964in}}%
\pgfpathlineto{\pgfqpoint{0.705162in}{0.535038in}}%
\pgfpathlineto{\pgfqpoint{0.702826in}{0.525571in}}%
\pgfpathlineto{\pgfqpoint{0.699659in}{0.522111in}}%
\pgfpathlineto{\pgfqpoint{0.689557in}{0.519083in}}%
\pgfpathlineto{\pgfqpoint{0.676288in}{0.518461in}}%
\pgfpathlineto{\pgfqpoint{0.663019in}{0.518920in}}%
\pgfpathclose%
\pgfpathmoveto{\pgfqpoint{0.738280in}{0.522111in}}%
\pgfpathlineto{\pgfqpoint{0.729364in}{0.522510in}}%
\pgfpathlineto{\pgfqpoint{0.716095in}{0.530060in}}%
\pgfpathlineto{\pgfqpoint{0.713888in}{0.535038in}}%
\pgfpathlineto{\pgfqpoint{0.711904in}{0.547964in}}%
\pgfpathlineto{\pgfqpoint{0.711476in}{0.560891in}}%
\pgfpathlineto{\pgfqpoint{0.711996in}{0.573817in}}%
\pgfpathlineto{\pgfqpoint{0.714991in}{0.586744in}}%
\pgfpathlineto{\pgfqpoint{0.716095in}{0.588417in}}%
\pgfpathlineto{\pgfqpoint{0.729364in}{0.593938in}}%
\pgfpathlineto{\pgfqpoint{0.742633in}{0.594393in}}%
\pgfpathlineto{\pgfqpoint{0.755902in}{0.592055in}}%
\pgfpathlineto{\pgfqpoint{0.761025in}{0.586744in}}%
\pgfpathlineto{\pgfqpoint{0.763578in}{0.573817in}}%
\pgfpathlineto{\pgfqpoint{0.763984in}{0.560891in}}%
\pgfpathlineto{\pgfqpoint{0.763571in}{0.547964in}}%
\pgfpathlineto{\pgfqpoint{0.761749in}{0.535038in}}%
\pgfpathlineto{\pgfqpoint{0.755902in}{0.525578in}}%
\pgfpathlineto{\pgfqpoint{0.743730in}{0.522111in}}%
\pgfpathlineto{\pgfqpoint{0.742633in}{0.521946in}}%
\pgfpathclose%
\pgfpathmoveto{\pgfqpoint{0.780334in}{0.522111in}}%
\pgfpathlineto{\pgfqpoint{0.771188in}{0.535038in}}%
\pgfpathlineto{\pgfqpoint{0.770049in}{0.547964in}}%
\pgfpathlineto{\pgfqpoint{0.770095in}{0.560891in}}%
\pgfpathlineto{\pgfqpoint{0.771194in}{0.573817in}}%
\pgfpathlineto{\pgfqpoint{0.775981in}{0.586744in}}%
\pgfpathlineto{\pgfqpoint{0.782439in}{0.591085in}}%
\pgfpathlineto{\pgfqpoint{0.795708in}{0.592695in}}%
\pgfpathlineto{\pgfqpoint{0.808977in}{0.591080in}}%
\pgfpathlineto{\pgfqpoint{0.815808in}{0.586744in}}%
\pgfpathlineto{\pgfqpoint{0.821176in}{0.573817in}}%
\pgfpathlineto{\pgfqpoint{0.822246in}{0.563131in}}%
\pgfpathlineto{\pgfqpoint{0.822404in}{0.560891in}}%
\pgfpathlineto{\pgfqpoint{0.822414in}{0.547964in}}%
\pgfpathlineto{\pgfqpoint{0.822246in}{0.545584in}}%
\pgfpathlineto{\pgfqpoint{0.821015in}{0.535038in}}%
\pgfpathlineto{\pgfqpoint{0.810648in}{0.522111in}}%
\pgfpathlineto{\pgfqpoint{0.808977in}{0.521562in}}%
\pgfpathlineto{\pgfqpoint{0.795708in}{0.520333in}}%
\pgfpathlineto{\pgfqpoint{0.782439in}{0.521415in}}%
\pgfpathclose%
\pgfpathmoveto{\pgfqpoint{0.841354in}{0.522111in}}%
\pgfpathlineto{\pgfqpoint{0.835515in}{0.524540in}}%
\pgfpathlineto{\pgfqpoint{0.830032in}{0.535038in}}%
\pgfpathlineto{\pgfqpoint{0.828587in}{0.547964in}}%
\pgfpathlineto{\pgfqpoint{0.828227in}{0.560891in}}%
\pgfpathlineto{\pgfqpoint{0.828472in}{0.573817in}}%
\pgfpathlineto{\pgfqpoint{0.830172in}{0.586744in}}%
\pgfpathlineto{\pgfqpoint{0.835515in}{0.593485in}}%
\pgfpathlineto{\pgfqpoint{0.848784in}{0.595847in}}%
\pgfpathlineto{\pgfqpoint{0.862053in}{0.596007in}}%
\pgfpathlineto{\pgfqpoint{0.875322in}{0.594420in}}%
\pgfpathlineto{\pgfqpoint{0.881621in}{0.586744in}}%
\pgfpathlineto{\pgfqpoint{0.882881in}{0.573817in}}%
\pgfpathlineto{\pgfqpoint{0.883027in}{0.560891in}}%
\pgfpathlineto{\pgfqpoint{0.882688in}{0.547964in}}%
\pgfpathlineto{\pgfqpoint{0.881416in}{0.535038in}}%
\pgfpathlineto{\pgfqpoint{0.875322in}{0.523622in}}%
\pgfpathlineto{\pgfqpoint{0.871247in}{0.522111in}}%
\pgfpathlineto{\pgfqpoint{0.862053in}{0.520476in}}%
\pgfpathlineto{\pgfqpoint{0.848784in}{0.520624in}}%
\pgfpathclose%
\pgfpathmoveto{\pgfqpoint{0.909567in}{0.522111in}}%
\pgfpathlineto{\pgfqpoint{0.901860in}{0.522978in}}%
\pgfpathlineto{\pgfqpoint{0.890794in}{0.535038in}}%
\pgfpathlineto{\pgfqpoint{0.889028in}{0.547964in}}%
\pgfpathlineto{\pgfqpoint{0.889000in}{0.560891in}}%
\pgfpathlineto{\pgfqpoint{0.890432in}{0.573817in}}%
\pgfpathlineto{\pgfqpoint{0.896798in}{0.586744in}}%
\pgfpathlineto{\pgfqpoint{0.901860in}{0.589810in}}%
\pgfpathlineto{\pgfqpoint{0.915129in}{0.591392in}}%
\pgfpathlineto{\pgfqpoint{0.928398in}{0.588921in}}%
\pgfpathlineto{\pgfqpoint{0.931493in}{0.586744in}}%
\pgfpathlineto{\pgfqpoint{0.937741in}{0.573817in}}%
\pgfpathlineto{\pgfqpoint{0.939167in}{0.560891in}}%
\pgfpathlineto{\pgfqpoint{0.939119in}{0.547964in}}%
\pgfpathlineto{\pgfqpoint{0.937302in}{0.535038in}}%
\pgfpathlineto{\pgfqpoint{0.928398in}{0.523858in}}%
\pgfpathlineto{\pgfqpoint{0.918818in}{0.522111in}}%
\pgfpathlineto{\pgfqpoint{0.915129in}{0.521673in}}%
\pgfpathclose%
\pgfpathmoveto{\pgfqpoint{0.953599in}{0.522111in}}%
\pgfpathlineto{\pgfqpoint{0.946694in}{0.535038in}}%
\pgfpathlineto{\pgfqpoint{0.945621in}{0.547964in}}%
\pgfpathlineto{\pgfqpoint{0.945306in}{0.560891in}}%
\pgfpathlineto{\pgfqpoint{0.945357in}{0.573817in}}%
\pgfpathlineto{\pgfqpoint{0.946139in}{0.586744in}}%
\pgfpathlineto{\pgfqpoint{0.954936in}{0.596199in}}%
\pgfpathlineto{\pgfqpoint{0.968205in}{0.597059in}}%
\pgfpathlineto{\pgfqpoint{0.981473in}{0.597101in}}%
\pgfpathlineto{\pgfqpoint{0.994742in}{0.596233in}}%
\pgfpathlineto{\pgfqpoint{1.001219in}{0.586744in}}%
\pgfpathlineto{\pgfqpoint{1.001690in}{0.573817in}}%
\pgfpathlineto{\pgfqpoint{1.001676in}{0.560891in}}%
\pgfpathlineto{\pgfqpoint{1.001383in}{0.547964in}}%
\pgfpathlineto{\pgfqpoint{1.000448in}{0.535038in}}%
\pgfpathlineto{\pgfqpoint{0.994742in}{0.522431in}}%
\pgfpathlineto{\pgfqpoint{0.994082in}{0.522111in}}%
\pgfpathlineto{\pgfqpoint{0.981473in}{0.519511in}}%
\pgfpathlineto{\pgfqpoint{0.968205in}{0.519442in}}%
\pgfpathlineto{\pgfqpoint{0.954936in}{0.521471in}}%
\pgfpathclose%
\pgfpathmoveto{\pgfqpoint{1.069431in}{0.522111in}}%
\pgfpathlineto{\pgfqpoint{1.063863in}{0.535038in}}%
\pgfpathlineto{\pgfqpoint{1.062989in}{0.547964in}}%
\pgfpathlineto{\pgfqpoint{1.062697in}{0.560891in}}%
\pgfpathlineto{\pgfqpoint{1.062634in}{0.573817in}}%
\pgfpathlineto{\pgfqpoint{1.062890in}{0.586744in}}%
\pgfpathlineto{\pgfqpoint{1.074356in}{0.597406in}}%
\pgfpathlineto{\pgfqpoint{1.087625in}{0.597650in}}%
\pgfpathlineto{\pgfqpoint{1.100894in}{0.597642in}}%
\pgfpathlineto{\pgfqpoint{1.114163in}{0.597178in}}%
\pgfpathlineto{\pgfqpoint{1.119880in}{0.586744in}}%
\pgfpathlineto{\pgfqpoint{1.120021in}{0.573817in}}%
\pgfpathlineto{\pgfqpoint{1.119941in}{0.560891in}}%
\pgfpathlineto{\pgfqpoint{1.119667in}{0.547964in}}%
\pgfpathlineto{\pgfqpoint{1.118876in}{0.535038in}}%
\pgfpathlineto{\pgfqpoint{1.114163in}{0.522416in}}%
\pgfpathlineto{\pgfqpoint{1.113688in}{0.522111in}}%
\pgfpathlineto{\pgfqpoint{1.100894in}{0.519084in}}%
\pgfpathlineto{\pgfqpoint{1.087625in}{0.518854in}}%
\pgfpathlineto{\pgfqpoint{1.074356in}{0.520068in}}%
\pgfpathclose%
\pgfpathmoveto{\pgfqpoint{1.187062in}{0.522111in}}%
\pgfpathlineto{\pgfqpoint{1.181549in}{0.535038in}}%
\pgfpathlineto{\pgfqpoint{1.180690in}{0.547964in}}%
\pgfpathlineto{\pgfqpoint{1.180508in}{0.555729in}}%
\pgfpathlineto{\pgfqpoint{1.180397in}{0.560891in}}%
\pgfpathlineto{\pgfqpoint{1.180312in}{0.573817in}}%
\pgfpathlineto{\pgfqpoint{1.180460in}{0.586744in}}%
\pgfpathlineto{\pgfqpoint{1.180508in}{0.587844in}}%
\pgfpathlineto{\pgfqpoint{1.193777in}{0.597567in}}%
\pgfpathlineto{\pgfqpoint{1.207045in}{0.597667in}}%
\pgfpathlineto{\pgfqpoint{1.220314in}{0.597567in}}%
\pgfpathlineto{\pgfqpoint{1.233583in}{0.596492in}}%
\pgfpathlineto{\pgfqpoint{1.237630in}{0.586744in}}%
\pgfpathlineto{\pgfqpoint{1.237874in}{0.573817in}}%
\pgfpathlineto{\pgfqpoint{1.237815in}{0.560891in}}%
\pgfpathlineto{\pgfqpoint{1.237536in}{0.547964in}}%
\pgfpathlineto{\pgfqpoint{1.236703in}{0.535038in}}%
\pgfpathlineto{\pgfqpoint{1.233583in}{0.524467in}}%
\pgfpathlineto{\pgfqpoint{1.230905in}{0.522111in}}%
\pgfpathlineto{\pgfqpoint{1.220314in}{0.519254in}}%
\pgfpathlineto{\pgfqpoint{1.207045in}{0.518816in}}%
\pgfpathlineto{\pgfqpoint{1.193777in}{0.519683in}}%
\pgfpathclose%
\pgfpathmoveto{\pgfqpoint{1.306740in}{0.522111in}}%
\pgfpathlineto{\pgfqpoint{1.299928in}{0.534213in}}%
\pgfpathlineto{\pgfqpoint{1.299797in}{0.535038in}}%
\pgfpathlineto{\pgfqpoint{1.298841in}{0.547964in}}%
\pgfpathlineto{\pgfqpoint{1.298541in}{0.560891in}}%
\pgfpathlineto{\pgfqpoint{1.298527in}{0.573817in}}%
\pgfpathlineto{\pgfqpoint{1.299008in}{0.586744in}}%
\pgfpathlineto{\pgfqpoint{1.299928in}{0.591728in}}%
\pgfpathlineto{\pgfqpoint{1.313197in}{0.596950in}}%
\pgfpathlineto{\pgfqpoint{1.326466in}{0.597132in}}%
\pgfpathlineto{\pgfqpoint{1.339735in}{0.596771in}}%
\pgfpathlineto{\pgfqpoint{1.353004in}{0.591890in}}%
\pgfpathlineto{\pgfqpoint{1.354456in}{0.586744in}}%
\pgfpathlineto{\pgfqpoint{1.355228in}{0.573817in}}%
\pgfpathlineto{\pgfqpoint{1.355278in}{0.560891in}}%
\pgfpathlineto{\pgfqpoint{1.354968in}{0.547964in}}%
\pgfpathlineto{\pgfqpoint{1.353910in}{0.535038in}}%
\pgfpathlineto{\pgfqpoint{1.353004in}{0.530856in}}%
\pgfpathlineto{\pgfqpoint{1.346234in}{0.522111in}}%
\pgfpathlineto{\pgfqpoint{1.339735in}{0.520118in}}%
\pgfpathlineto{\pgfqpoint{1.326466in}{0.519310in}}%
\pgfpathlineto{\pgfqpoint{1.313197in}{0.520067in}}%
\pgfpathclose%
\pgfpathmoveto{\pgfqpoint{1.379173in}{0.522111in}}%
\pgfpathlineto{\pgfqpoint{1.366273in}{0.527835in}}%
\pgfpathlineto{\pgfqpoint{1.362815in}{0.535038in}}%
\pgfpathlineto{\pgfqpoint{1.361170in}{0.547964in}}%
\pgfpathlineto{\pgfqpoint{1.361126in}{0.560891in}}%
\pgfpathlineto{\pgfqpoint{1.362417in}{0.573817in}}%
\pgfpathlineto{\pgfqpoint{1.366273in}{0.584180in}}%
\pgfpathlineto{\pgfqpoint{1.369024in}{0.586744in}}%
\pgfpathlineto{\pgfqpoint{1.379542in}{0.590946in}}%
\pgfpathlineto{\pgfqpoint{1.392811in}{0.591022in}}%
\pgfpathlineto{\pgfqpoint{1.403714in}{0.586744in}}%
\pgfpathlineto{\pgfqpoint{1.406080in}{0.584525in}}%
\pgfpathlineto{\pgfqpoint{1.409990in}{0.573817in}}%
\pgfpathlineto{\pgfqpoint{1.411221in}{0.560891in}}%
\pgfpathlineto{\pgfqpoint{1.411198in}{0.547964in}}%
\pgfpathlineto{\pgfqpoint{1.409678in}{0.535038in}}%
\pgfpathlineto{\pgfqpoint{1.406080in}{0.527401in}}%
\pgfpathlineto{\pgfqpoint{1.393676in}{0.522111in}}%
\pgfpathlineto{\pgfqpoint{1.392811in}{0.521946in}}%
\pgfpathlineto{\pgfqpoint{1.379542in}{0.522036in}}%
\pgfpathclose%
\pgfpathmoveto{\pgfqpoint{1.428859in}{0.522111in}}%
\pgfpathlineto{\pgfqpoint{1.419348in}{0.532486in}}%
\pgfpathlineto{\pgfqpoint{1.418737in}{0.535038in}}%
\pgfpathlineto{\pgfqpoint{1.417468in}{0.547964in}}%
\pgfpathlineto{\pgfqpoint{1.417130in}{0.560891in}}%
\pgfpathlineto{\pgfqpoint{1.417275in}{0.573817in}}%
\pgfpathlineto{\pgfqpoint{1.418530in}{0.586744in}}%
\pgfpathlineto{\pgfqpoint{1.419348in}{0.589281in}}%
\pgfpathlineto{\pgfqpoint{1.432617in}{0.595715in}}%
\pgfpathlineto{\pgfqpoint{1.445886in}{0.596043in}}%
\pgfpathlineto{\pgfqpoint{1.459155in}{0.595090in}}%
\pgfpathlineto{\pgfqpoint{1.470139in}{0.586744in}}%
\pgfpathlineto{\pgfqpoint{1.472011in}{0.573817in}}%
\pgfpathlineto{\pgfqpoint{1.472281in}{0.560891in}}%
\pgfpathlineto{\pgfqpoint{1.471885in}{0.547964in}}%
\pgfpathlineto{\pgfqpoint{1.470293in}{0.535038in}}%
\pgfpathlineto{\pgfqpoint{1.459975in}{0.522111in}}%
\pgfpathlineto{\pgfqpoint{1.459155in}{0.521826in}}%
\pgfpathlineto{\pgfqpoint{1.445886in}{0.520335in}}%
\pgfpathlineto{\pgfqpoint{1.432617in}{0.521068in}}%
\pgfpathclose%
\pgfpathmoveto{\pgfqpoint{0.484631in}{0.535038in}}%
\pgfpathlineto{\pgfqpoint{0.480274in}{0.547964in}}%
\pgfpathlineto{\pgfqpoint{0.479496in}{0.560891in}}%
\pgfpathlineto{\pgfqpoint{0.481104in}{0.573817in}}%
\pgfpathlineto{\pgfqpoint{0.489281in}{0.586744in}}%
\pgfpathlineto{\pgfqpoint{0.490523in}{0.587514in}}%
\pgfpathlineto{\pgfqpoint{0.503792in}{0.589622in}}%
\pgfpathlineto{\pgfqpoint{0.515489in}{0.586744in}}%
\pgfpathlineto{\pgfqpoint{0.517061in}{0.586015in}}%
\pgfpathlineto{\pgfqpoint{0.523309in}{0.573817in}}%
\pgfpathlineto{\pgfqpoint{0.524593in}{0.560891in}}%
\pgfpathlineto{\pgfqpoint{0.523931in}{0.547964in}}%
\pgfpathlineto{\pgfqpoint{0.520252in}{0.535038in}}%
\pgfpathlineto{\pgfqpoint{0.517061in}{0.531061in}}%
\pgfpathlineto{\pgfqpoint{0.503792in}{0.526852in}}%
\pgfpathlineto{\pgfqpoint{0.490523in}{0.529334in}}%
\pgfpathclose%
\pgfpathmoveto{\pgfqpoint{0.598785in}{0.535038in}}%
\pgfpathlineto{\pgfqpoint{0.596674in}{0.541679in}}%
\pgfpathlineto{\pgfqpoint{0.595602in}{0.547964in}}%
\pgfpathlineto{\pgfqpoint{0.595082in}{0.560891in}}%
\pgfpathlineto{\pgfqpoint{0.595958in}{0.573817in}}%
\pgfpathlineto{\pgfqpoint{0.596674in}{0.577096in}}%
\pgfpathlineto{\pgfqpoint{0.601807in}{0.586744in}}%
\pgfpathlineto{\pgfqpoint{0.609943in}{0.591216in}}%
\pgfpathlineto{\pgfqpoint{0.623212in}{0.592269in}}%
\pgfpathlineto{\pgfqpoint{0.636481in}{0.589285in}}%
\pgfpathlineto{\pgfqpoint{0.639319in}{0.586744in}}%
\pgfpathlineto{\pgfqpoint{0.643743in}{0.573817in}}%
\pgfpathlineto{\pgfqpoint{0.644522in}{0.560891in}}%
\pgfpathlineto{\pgfqpoint{0.644003in}{0.547964in}}%
\pgfpathlineto{\pgfqpoint{0.641392in}{0.535038in}}%
\pgfpathlineto{\pgfqpoint{0.636481in}{0.528093in}}%
\pgfpathlineto{\pgfqpoint{0.623212in}{0.524109in}}%
\pgfpathlineto{\pgfqpoint{0.609943in}{0.525413in}}%
\pgfpathclose%
\pgfpathmoveto{\pgfqpoint{1.009717in}{0.535038in}}%
\pgfpathlineto{\pgfqpoint{1.008011in}{0.543954in}}%
\pgfpathlineto{\pgfqpoint{1.007581in}{0.547964in}}%
\pgfpathlineto{\pgfqpoint{1.007518in}{0.560891in}}%
\pgfpathlineto{\pgfqpoint{1.008011in}{0.566186in}}%
\pgfpathlineto{\pgfqpoint{1.009095in}{0.573817in}}%
\pgfpathlineto{\pgfqpoint{1.016670in}{0.586744in}}%
\pgfpathlineto{\pgfqpoint{1.021280in}{0.589265in}}%
\pgfpathlineto{\pgfqpoint{1.034549in}{0.590570in}}%
\pgfpathlineto{\pgfqpoint{1.047818in}{0.587133in}}%
\pgfpathlineto{\pgfqpoint{1.048315in}{0.586744in}}%
\pgfpathlineto{\pgfqpoint{1.054984in}{0.573817in}}%
\pgfpathlineto{\pgfqpoint{1.056487in}{0.560891in}}%
\pgfpathlineto{\pgfqpoint{1.056409in}{0.547964in}}%
\pgfpathlineto{\pgfqpoint{1.054400in}{0.535038in}}%
\pgfpathlineto{\pgfqpoint{1.047818in}{0.525764in}}%
\pgfpathlineto{\pgfqpoint{1.034549in}{0.522555in}}%
\pgfpathlineto{\pgfqpoint{1.021280in}{0.523733in}}%
\pgfpathclose%
\pgfpathmoveto{\pgfqpoint{1.127850in}{0.535038in}}%
\pgfpathlineto{\pgfqpoint{1.127432in}{0.536593in}}%
\pgfpathlineto{\pgfqpoint{1.125827in}{0.547964in}}%
\pgfpathlineto{\pgfqpoint{1.125747in}{0.560891in}}%
\pgfpathlineto{\pgfqpoint{1.127170in}{0.573817in}}%
\pgfpathlineto{\pgfqpoint{1.127432in}{0.574798in}}%
\pgfpathlineto{\pgfqpoint{1.135480in}{0.586744in}}%
\pgfpathlineto{\pgfqpoint{1.140701in}{0.589322in}}%
\pgfpathlineto{\pgfqpoint{1.153970in}{0.590223in}}%
\pgfpathlineto{\pgfqpoint{1.165260in}{0.586744in}}%
\pgfpathlineto{\pgfqpoint{1.167239in}{0.585537in}}%
\pgfpathlineto{\pgfqpoint{1.172812in}{0.573817in}}%
\pgfpathlineto{\pgfqpoint{1.174304in}{0.560891in}}%
\pgfpathlineto{\pgfqpoint{1.174221in}{0.547964in}}%
\pgfpathlineto{\pgfqpoint{1.172205in}{0.535038in}}%
\pgfpathlineto{\pgfqpoint{1.167239in}{0.527124in}}%
\pgfpathlineto{\pgfqpoint{1.153970in}{0.522931in}}%
\pgfpathlineto{\pgfqpoint{1.140701in}{0.523763in}}%
\pgfpathclose%
\pgfpathmoveto{\pgfqpoint{1.245563in}{0.535038in}}%
\pgfpathlineto{\pgfqpoint{1.243696in}{0.547964in}}%
\pgfpathlineto{\pgfqpoint{1.243624in}{0.560891in}}%
\pgfpathlineto{\pgfqpoint{1.245020in}{0.573817in}}%
\pgfpathlineto{\pgfqpoint{1.246852in}{0.579553in}}%
\pgfpathlineto{\pgfqpoint{1.253031in}{0.586744in}}%
\pgfpathlineto{\pgfqpoint{1.260121in}{0.589899in}}%
\pgfpathlineto{\pgfqpoint{1.273390in}{0.590362in}}%
\pgfpathlineto{\pgfqpoint{1.283749in}{0.586744in}}%
\pgfpathlineto{\pgfqpoint{1.286659in}{0.584546in}}%
\pgfpathlineto{\pgfqpoint{1.291160in}{0.573817in}}%
\pgfpathlineto{\pgfqpoint{1.292562in}{0.560891in}}%
\pgfpathlineto{\pgfqpoint{1.292497in}{0.547964in}}%
\pgfpathlineto{\pgfqpoint{1.290645in}{0.535038in}}%
\pgfpathlineto{\pgfqpoint{1.286659in}{0.527770in}}%
\pgfpathlineto{\pgfqpoint{1.273390in}{0.522735in}}%
\pgfpathlineto{\pgfqpoint{1.260121in}{0.523173in}}%
\pgfpathlineto{\pgfqpoint{1.246852in}{0.531855in}}%
\pgfpathclose%
\pgfpathmoveto{\pgfqpoint{0.428557in}{0.612597in}}%
\pgfpathlineto{\pgfqpoint{0.424178in}{0.617647in}}%
\pgfpathlineto{\pgfqpoint{0.421541in}{0.625523in}}%
\pgfpathlineto{\pgfqpoint{0.420348in}{0.638450in}}%
\pgfpathlineto{\pgfqpoint{0.421072in}{0.651377in}}%
\pgfpathlineto{\pgfqpoint{0.424178in}{0.661867in}}%
\pgfpathlineto{\pgfqpoint{0.426023in}{0.664303in}}%
\pgfpathlineto{\pgfqpoint{0.437447in}{0.670293in}}%
\pgfpathlineto{\pgfqpoint{0.450716in}{0.670191in}}%
\pgfpathlineto{\pgfqpoint{0.460705in}{0.664303in}}%
\pgfpathlineto{\pgfqpoint{0.463985in}{0.658220in}}%
\pgfpathlineto{\pgfqpoint{0.465553in}{0.651377in}}%
\pgfpathlineto{\pgfqpoint{0.466169in}{0.638450in}}%
\pgfpathlineto{\pgfqpoint{0.465080in}{0.625523in}}%
\pgfpathlineto{\pgfqpoint{0.463985in}{0.621423in}}%
\pgfpathlineto{\pgfqpoint{0.458357in}{0.612597in}}%
\pgfpathlineto{\pgfqpoint{0.450716in}{0.608398in}}%
\pgfpathlineto{\pgfqpoint{0.437447in}{0.608232in}}%
\pgfpathclose%
\pgfpathmoveto{\pgfqpoint{0.541011in}{0.612597in}}%
\pgfpathlineto{\pgfqpoint{0.537056in}{0.625523in}}%
\pgfpathlineto{\pgfqpoint{0.536145in}{0.638450in}}%
\pgfpathlineto{\pgfqpoint{0.536593in}{0.651377in}}%
\pgfpathlineto{\pgfqpoint{0.539470in}{0.664303in}}%
\pgfpathlineto{\pgfqpoint{0.543598in}{0.669645in}}%
\pgfpathlineto{\pgfqpoint{0.556867in}{0.673652in}}%
\pgfpathlineto{\pgfqpoint{0.570136in}{0.673302in}}%
\pgfpathlineto{\pgfqpoint{0.583405in}{0.664575in}}%
\pgfpathlineto{\pgfqpoint{0.583523in}{0.664303in}}%
\pgfpathlineto{\pgfqpoint{0.585867in}{0.651377in}}%
\pgfpathlineto{\pgfqpoint{0.586198in}{0.638450in}}%
\pgfpathlineto{\pgfqpoint{0.585415in}{0.625523in}}%
\pgfpathlineto{\pgfqpoint{0.583405in}{0.616456in}}%
\pgfpathlineto{\pgfqpoint{0.581520in}{0.612597in}}%
\pgfpathlineto{\pgfqpoint{0.570136in}{0.605495in}}%
\pgfpathlineto{\pgfqpoint{0.556867in}{0.604997in}}%
\pgfpathlineto{\pgfqpoint{0.543598in}{0.609536in}}%
\pgfpathclose%
\pgfpathmoveto{\pgfqpoint{0.593057in}{0.612597in}}%
\pgfpathlineto{\pgfqpoint{0.592106in}{0.625523in}}%
\pgfpathlineto{\pgfqpoint{0.592036in}{0.638450in}}%
\pgfpathlineto{\pgfqpoint{0.592509in}{0.651377in}}%
\pgfpathlineto{\pgfqpoint{0.594356in}{0.664303in}}%
\pgfpathlineto{\pgfqpoint{0.596674in}{0.669198in}}%
\pgfpathlineto{\pgfqpoint{0.609943in}{0.674911in}}%
\pgfpathlineto{\pgfqpoint{0.623212in}{0.675125in}}%
\pgfpathlineto{\pgfqpoint{0.636481in}{0.672316in}}%
\pgfpathlineto{\pgfqpoint{0.643168in}{0.664303in}}%
\pgfpathlineto{\pgfqpoint{0.645556in}{0.651377in}}%
\pgfpathlineto{\pgfqpoint{0.646146in}{0.638450in}}%
\pgfpathlineto{\pgfqpoint{0.645959in}{0.625523in}}%
\pgfpathlineto{\pgfqpoint{0.644434in}{0.612597in}}%
\pgfpathlineto{\pgfqpoint{0.636481in}{0.602675in}}%
\pgfpathlineto{\pgfqpoint{0.623212in}{0.600606in}}%
\pgfpathlineto{\pgfqpoint{0.609943in}{0.600631in}}%
\pgfpathlineto{\pgfqpoint{0.596674in}{0.604222in}}%
\pgfpathclose%
\pgfpathmoveto{\pgfqpoint{0.655605in}{0.612597in}}%
\pgfpathlineto{\pgfqpoint{0.652847in}{0.625523in}}%
\pgfpathlineto{\pgfqpoint{0.652174in}{0.638450in}}%
\pgfpathlineto{\pgfqpoint{0.652384in}{0.651377in}}%
\pgfpathlineto{\pgfqpoint{0.654079in}{0.664303in}}%
\pgfpathlineto{\pgfqpoint{0.663019in}{0.674436in}}%
\pgfpathlineto{\pgfqpoint{0.676288in}{0.676290in}}%
\pgfpathlineto{\pgfqpoint{0.689557in}{0.675982in}}%
\pgfpathlineto{\pgfqpoint{0.702826in}{0.669461in}}%
\pgfpathlineto{\pgfqpoint{0.704536in}{0.664303in}}%
\pgfpathlineto{\pgfqpoint{0.705698in}{0.651377in}}%
\pgfpathlineto{\pgfqpoint{0.705804in}{0.638450in}}%
\pgfpathlineto{\pgfqpoint{0.705261in}{0.625523in}}%
\pgfpathlineto{\pgfqpoint{0.703074in}{0.612597in}}%
\pgfpathlineto{\pgfqpoint{0.702826in}{0.611967in}}%
\pgfpathlineto{\pgfqpoint{0.689557in}{0.603028in}}%
\pgfpathlineto{\pgfqpoint{0.676288in}{0.602457in}}%
\pgfpathlineto{\pgfqpoint{0.663019in}{0.604869in}}%
\pgfpathclose%
\pgfpathmoveto{\pgfqpoint{0.714104in}{0.612597in}}%
\pgfpathlineto{\pgfqpoint{0.711947in}{0.625523in}}%
\pgfpathlineto{\pgfqpoint{0.711640in}{0.638450in}}%
\pgfpathlineto{\pgfqpoint{0.712336in}{0.651377in}}%
\pgfpathlineto{\pgfqpoint{0.715336in}{0.664303in}}%
\pgfpathlineto{\pgfqpoint{0.716095in}{0.665636in}}%
\pgfpathlineto{\pgfqpoint{0.729364in}{0.672661in}}%
\pgfpathlineto{\pgfqpoint{0.742633in}{0.672818in}}%
\pgfpathlineto{\pgfqpoint{0.755902in}{0.667825in}}%
\pgfpathlineto{\pgfqpoint{0.758500in}{0.664303in}}%
\pgfpathlineto{\pgfqpoint{0.761841in}{0.651377in}}%
\pgfpathlineto{\pgfqpoint{0.762615in}{0.638450in}}%
\pgfpathlineto{\pgfqpoint{0.762224in}{0.625523in}}%
\pgfpathlineto{\pgfqpoint{0.759691in}{0.612597in}}%
\pgfpathlineto{\pgfqpoint{0.755902in}{0.607209in}}%
\pgfpathlineto{\pgfqpoint{0.742633in}{0.602971in}}%
\pgfpathlineto{\pgfqpoint{0.729364in}{0.603044in}}%
\pgfpathlineto{\pgfqpoint{0.716095in}{0.608843in}}%
\pgfpathclose%
\pgfpathmoveto{\pgfqpoint{0.770650in}{0.612597in}}%
\pgfpathlineto{\pgfqpoint{0.769170in}{0.622402in}}%
\pgfpathlineto{\pgfqpoint{0.768904in}{0.625523in}}%
\pgfpathlineto{\pgfqpoint{0.768464in}{0.638450in}}%
\pgfpathlineto{\pgfqpoint{0.768473in}{0.651377in}}%
\pgfpathlineto{\pgfqpoint{0.769120in}{0.664303in}}%
\pgfpathlineto{\pgfqpoint{0.769170in}{0.664703in}}%
\pgfpathlineto{\pgfqpoint{0.779967in}{0.677230in}}%
\pgfpathlineto{\pgfqpoint{0.782439in}{0.677608in}}%
\pgfpathlineto{\pgfqpoint{0.795708in}{0.678244in}}%
\pgfpathlineto{\pgfqpoint{0.808977in}{0.678180in}}%
\pgfpathlineto{\pgfqpoint{0.818245in}{0.677230in}}%
\pgfpathlineto{\pgfqpoint{0.822246in}{0.675366in}}%
\pgfpathlineto{\pgfqpoint{0.824795in}{0.664303in}}%
\pgfpathlineto{\pgfqpoint{0.825066in}{0.651377in}}%
\pgfpathlineto{\pgfqpoint{0.824999in}{0.638450in}}%
\pgfpathlineto{\pgfqpoint{0.824636in}{0.625523in}}%
\pgfpathlineto{\pgfqpoint{0.823377in}{0.612597in}}%
\pgfpathlineto{\pgfqpoint{0.822246in}{0.608699in}}%
\pgfpathlineto{\pgfqpoint{0.808977in}{0.601013in}}%
\pgfpathlineto{\pgfqpoint{0.795708in}{0.600538in}}%
\pgfpathlineto{\pgfqpoint{0.782439in}{0.601709in}}%
\pgfpathclose%
\pgfpathmoveto{\pgfqpoint{0.834446in}{0.612597in}}%
\pgfpathlineto{\pgfqpoint{0.831361in}{0.625523in}}%
\pgfpathlineto{\pgfqpoint{0.830872in}{0.638450in}}%
\pgfpathlineto{\pgfqpoint{0.831742in}{0.651377in}}%
\pgfpathlineto{\pgfqpoint{0.835515in}{0.664058in}}%
\pgfpathlineto{\pgfqpoint{0.835696in}{0.664303in}}%
\pgfpathlineto{\pgfqpoint{0.848784in}{0.671112in}}%
\pgfpathlineto{\pgfqpoint{0.862053in}{0.670946in}}%
\pgfpathlineto{\pgfqpoint{0.874224in}{0.664303in}}%
\pgfpathlineto{\pgfqpoint{0.875322in}{0.662744in}}%
\pgfpathlineto{\pgfqpoint{0.878631in}{0.651377in}}%
\pgfpathlineto{\pgfqpoint{0.879538in}{0.638450in}}%
\pgfpathlineto{\pgfqpoint{0.879002in}{0.625523in}}%
\pgfpathlineto{\pgfqpoint{0.875733in}{0.612597in}}%
\pgfpathlineto{\pgfqpoint{0.875322in}{0.611920in}}%
\pgfpathlineto{\pgfqpoint{0.862053in}{0.604876in}}%
\pgfpathlineto{\pgfqpoint{0.848784in}{0.604705in}}%
\pgfpathlineto{\pgfqpoint{0.835515in}{0.610895in}}%
\pgfpathclose%
\pgfpathmoveto{\pgfqpoint{0.954101in}{0.612597in}}%
\pgfpathlineto{\pgfqpoint{0.950351in}{0.625523in}}%
\pgfpathlineto{\pgfqpoint{0.949731in}{0.638450in}}%
\pgfpathlineto{\pgfqpoint{0.950729in}{0.651377in}}%
\pgfpathlineto{\pgfqpoint{0.954936in}{0.663702in}}%
\pgfpathlineto{\pgfqpoint{0.955490in}{0.664303in}}%
\pgfpathlineto{\pgfqpoint{0.968205in}{0.670168in}}%
\pgfpathlineto{\pgfqpoint{0.981473in}{0.669511in}}%
\pgfpathlineto{\pgfqpoint{0.990024in}{0.664303in}}%
\pgfpathlineto{\pgfqpoint{0.994742in}{0.655897in}}%
\pgfpathlineto{\pgfqpoint{0.995869in}{0.651377in}}%
\pgfpathlineto{\pgfqpoint{0.996862in}{0.638450in}}%
\pgfpathlineto{\pgfqpoint{0.996232in}{0.625523in}}%
\pgfpathlineto{\pgfqpoint{0.994742in}{0.618828in}}%
\pgfpathlineto{\pgfqpoint{0.991578in}{0.612597in}}%
\pgfpathlineto{\pgfqpoint{0.981473in}{0.606321in}}%
\pgfpathlineto{\pgfqpoint{0.968205in}{0.605722in}}%
\pgfpathlineto{\pgfqpoint{0.954936in}{0.611450in}}%
\pgfpathclose%
\pgfpathmoveto{\pgfqpoint{1.073063in}{0.612597in}}%
\pgfpathlineto{\pgfqpoint{1.068901in}{0.625523in}}%
\pgfpathlineto{\pgfqpoint{1.068200in}{0.638450in}}%
\pgfpathlineto{\pgfqpoint{1.069283in}{0.651377in}}%
\pgfpathlineto{\pgfqpoint{1.074232in}{0.664303in}}%
\pgfpathlineto{\pgfqpoint{1.074356in}{0.664447in}}%
\pgfpathlineto{\pgfqpoint{1.087625in}{0.669767in}}%
\pgfpathlineto{\pgfqpoint{1.100894in}{0.668529in}}%
\pgfpathlineto{\pgfqpoint{1.107136in}{0.664303in}}%
\pgfpathlineto{\pgfqpoint{1.113312in}{0.651377in}}%
\pgfpathlineto{\pgfqpoint{1.114163in}{0.643641in}}%
\pgfpathlineto{\pgfqpoint{1.114547in}{0.638450in}}%
\pgfpathlineto{\pgfqpoint{1.114163in}{0.630659in}}%
\pgfpathlineto{\pgfqpoint{1.113784in}{0.625523in}}%
\pgfpathlineto{\pgfqpoint{1.108576in}{0.612597in}}%
\pgfpathlineto{\pgfqpoint{1.100894in}{0.607284in}}%
\pgfpathlineto{\pgfqpoint{1.087625in}{0.606164in}}%
\pgfpathlineto{\pgfqpoint{1.074356in}{0.611038in}}%
\pgfpathclose%
\pgfpathmoveto{\pgfqpoint{1.191295in}{0.612597in}}%
\pgfpathlineto{\pgfqpoint{1.186977in}{0.625523in}}%
\pgfpathlineto{\pgfqpoint{1.186246in}{0.638450in}}%
\pgfpathlineto{\pgfqpoint{1.187368in}{0.651377in}}%
\pgfpathlineto{\pgfqpoint{1.192489in}{0.664303in}}%
\pgfpathlineto{\pgfqpoint{1.193777in}{0.665633in}}%
\pgfpathlineto{\pgfqpoint{1.207045in}{0.669869in}}%
\pgfpathlineto{\pgfqpoint{1.220314in}{0.668044in}}%
\pgfpathlineto{\pgfqpoint{1.225293in}{0.664303in}}%
\pgfpathlineto{\pgfqpoint{1.231076in}{0.651377in}}%
\pgfpathlineto{\pgfqpoint{1.232342in}{0.638450in}}%
\pgfpathlineto{\pgfqpoint{1.231522in}{0.625523in}}%
\pgfpathlineto{\pgfqpoint{1.226659in}{0.612597in}}%
\pgfpathlineto{\pgfqpoint{1.220314in}{0.607717in}}%
\pgfpathlineto{\pgfqpoint{1.207045in}{0.606075in}}%
\pgfpathlineto{\pgfqpoint{1.193777in}{0.609944in}}%
\pgfpathclose%
\pgfpathmoveto{\pgfqpoint{1.308728in}{0.612597in}}%
\pgfpathlineto{\pgfqpoint{1.304522in}{0.625523in}}%
\pgfpathlineto{\pgfqpoint{1.303815in}{0.638450in}}%
\pgfpathlineto{\pgfqpoint{1.304930in}{0.651377in}}%
\pgfpathlineto{\pgfqpoint{1.309974in}{0.664303in}}%
\pgfpathlineto{\pgfqpoint{1.313197in}{0.667292in}}%
\pgfpathlineto{\pgfqpoint{1.326466in}{0.670457in}}%
\pgfpathlineto{\pgfqpoint{1.339735in}{0.668127in}}%
\pgfpathlineto{\pgfqpoint{1.344321in}{0.664303in}}%
\pgfpathlineto{\pgfqpoint{1.349472in}{0.651377in}}%
\pgfpathlineto{\pgfqpoint{1.350605in}{0.638450in}}%
\pgfpathlineto{\pgfqpoint{1.349901in}{0.625523in}}%
\pgfpathlineto{\pgfqpoint{1.345645in}{0.612597in}}%
\pgfpathlineto{\pgfqpoint{1.339735in}{0.607537in}}%
\pgfpathlineto{\pgfqpoint{1.326466in}{0.605472in}}%
\pgfpathlineto{\pgfqpoint{1.313197in}{0.608328in}}%
\pgfpathclose%
\pgfpathmoveto{\pgfqpoint{1.425252in}{0.612597in}}%
\pgfpathlineto{\pgfqpoint{1.421452in}{0.625523in}}%
\pgfpathlineto{\pgfqpoint{1.420828in}{0.638450in}}%
\pgfpathlineto{\pgfqpoint{1.421883in}{0.651377in}}%
\pgfpathlineto{\pgfqpoint{1.426578in}{0.664303in}}%
\pgfpathlineto{\pgfqpoint{1.432617in}{0.669348in}}%
\pgfpathlineto{\pgfqpoint{1.445886in}{0.671536in}}%
\pgfpathlineto{\pgfqpoint{1.459155in}{0.668895in}}%
\pgfpathlineto{\pgfqpoint{1.464106in}{0.664303in}}%
\pgfpathlineto{\pgfqpoint{1.468404in}{0.651377in}}%
\pgfpathlineto{\pgfqpoint{1.469364in}{0.638450in}}%
\pgfpathlineto{\pgfqpoint{1.468824in}{0.625523in}}%
\pgfpathlineto{\pgfqpoint{1.465417in}{0.612597in}}%
\pgfpathlineto{\pgfqpoint{1.459155in}{0.606613in}}%
\pgfpathlineto{\pgfqpoint{1.445886in}{0.604353in}}%
\pgfpathlineto{\pgfqpoint{1.432617in}{0.606282in}}%
\pgfpathclose%
\pgfpathmoveto{\pgfqpoint{0.377819in}{0.690156in}}%
\pgfpathlineto{\pgfqpoint{0.371102in}{0.692349in}}%
\pgfpathlineto{\pgfqpoint{0.364081in}{0.703083in}}%
\pgfpathlineto{\pgfqpoint{0.361923in}{0.716009in}}%
\pgfpathlineto{\pgfqpoint{0.362174in}{0.728936in}}%
\pgfpathlineto{\pgfqpoint{0.365429in}{0.741862in}}%
\pgfpathlineto{\pgfqpoint{0.371102in}{0.748503in}}%
\pgfpathlineto{\pgfqpoint{0.384371in}{0.751749in}}%
\pgfpathlineto{\pgfqpoint{0.397640in}{0.748931in}}%
\pgfpathlineto{\pgfqpoint{0.403552in}{0.741862in}}%
\pgfpathlineto{\pgfqpoint{0.406440in}{0.728936in}}%
\pgfpathlineto{\pgfqpoint{0.406636in}{0.716009in}}%
\pgfpathlineto{\pgfqpoint{0.404691in}{0.703083in}}%
\pgfpathlineto{\pgfqpoint{0.397640in}{0.691967in}}%
\pgfpathlineto{\pgfqpoint{0.391736in}{0.690156in}}%
\pgfpathlineto{\pgfqpoint{0.384371in}{0.688816in}}%
\pgfpathclose%
\pgfpathmoveto{\pgfqpoint{0.485675in}{0.690156in}}%
\pgfpathlineto{\pgfqpoint{0.478487in}{0.703083in}}%
\pgfpathlineto{\pgfqpoint{0.477254in}{0.712434in}}%
\pgfpathlineto{\pgfqpoint{0.476941in}{0.716009in}}%
\pgfpathlineto{\pgfqpoint{0.477037in}{0.728936in}}%
\pgfpathlineto{\pgfqpoint{0.477254in}{0.731017in}}%
\pgfpathlineto{\pgfqpoint{0.479203in}{0.741862in}}%
\pgfpathlineto{\pgfqpoint{0.490523in}{0.753645in}}%
\pgfpathlineto{\pgfqpoint{0.499262in}{0.754789in}}%
\pgfpathlineto{\pgfqpoint{0.503792in}{0.755228in}}%
\pgfpathlineto{\pgfqpoint{0.507709in}{0.754789in}}%
\pgfpathlineto{\pgfqpoint{0.517061in}{0.753061in}}%
\pgfpathlineto{\pgfqpoint{0.525319in}{0.741862in}}%
\pgfpathlineto{\pgfqpoint{0.526948in}{0.728936in}}%
\pgfpathlineto{\pgfqpoint{0.527003in}{0.716009in}}%
\pgfpathlineto{\pgfqpoint{0.525745in}{0.703083in}}%
\pgfpathlineto{\pgfqpoint{0.519877in}{0.690156in}}%
\pgfpathlineto{\pgfqpoint{0.517061in}{0.688145in}}%
\pgfpathlineto{\pgfqpoint{0.503792in}{0.685731in}}%
\pgfpathlineto{\pgfqpoint{0.490523in}{0.687373in}}%
\pgfpathclose%
\pgfpathmoveto{\pgfqpoint{0.535612in}{0.690156in}}%
\pgfpathlineto{\pgfqpoint{0.533639in}{0.703083in}}%
\pgfpathlineto{\pgfqpoint{0.533325in}{0.716009in}}%
\pgfpathlineto{\pgfqpoint{0.533611in}{0.728936in}}%
\pgfpathlineto{\pgfqpoint{0.534898in}{0.741862in}}%
\pgfpathlineto{\pgfqpoint{0.543598in}{0.754705in}}%
\pgfpathlineto{\pgfqpoint{0.543951in}{0.754789in}}%
\pgfpathlineto{\pgfqpoint{0.556867in}{0.756526in}}%
\pgfpathlineto{\pgfqpoint{0.570136in}{0.755731in}}%
\pgfpathlineto{\pgfqpoint{0.573539in}{0.754789in}}%
\pgfpathlineto{\pgfqpoint{0.583405in}{0.747350in}}%
\pgfpathlineto{\pgfqpoint{0.585354in}{0.741862in}}%
\pgfpathlineto{\pgfqpoint{0.586899in}{0.728936in}}%
\pgfpathlineto{\pgfqpoint{0.587208in}{0.716009in}}%
\pgfpathlineto{\pgfqpoint{0.586714in}{0.703083in}}%
\pgfpathlineto{\pgfqpoint{0.584007in}{0.690156in}}%
\pgfpathlineto{\pgfqpoint{0.583405in}{0.689089in}}%
\pgfpathlineto{\pgfqpoint{0.570136in}{0.682746in}}%
\pgfpathlineto{\pgfqpoint{0.556867in}{0.682061in}}%
\pgfpathlineto{\pgfqpoint{0.543598in}{0.683251in}}%
\pgfpathclose%
\pgfpathmoveto{\pgfqpoint{0.597780in}{0.690156in}}%
\pgfpathlineto{\pgfqpoint{0.596674in}{0.691887in}}%
\pgfpathlineto{\pgfqpoint{0.593791in}{0.703083in}}%
\pgfpathlineto{\pgfqpoint{0.592932in}{0.716009in}}%
\pgfpathlineto{\pgfqpoint{0.592925in}{0.728936in}}%
\pgfpathlineto{\pgfqpoint{0.593883in}{0.741862in}}%
\pgfpathlineto{\pgfqpoint{0.596674in}{0.750669in}}%
\pgfpathlineto{\pgfqpoint{0.601545in}{0.754789in}}%
\pgfpathlineto{\pgfqpoint{0.609943in}{0.757205in}}%
\pgfpathlineto{\pgfqpoint{0.623212in}{0.757849in}}%
\pgfpathlineto{\pgfqpoint{0.636481in}{0.756714in}}%
\pgfpathlineto{\pgfqpoint{0.640923in}{0.754789in}}%
\pgfpathlineto{\pgfqpoint{0.646231in}{0.741862in}}%
\pgfpathlineto{\pgfqpoint{0.646907in}{0.728936in}}%
\pgfpathlineto{\pgfqpoint{0.646854in}{0.716009in}}%
\pgfpathlineto{\pgfqpoint{0.646119in}{0.703083in}}%
\pgfpathlineto{\pgfqpoint{0.642869in}{0.690156in}}%
\pgfpathlineto{\pgfqpoint{0.636481in}{0.684954in}}%
\pgfpathlineto{\pgfqpoint{0.623212in}{0.683169in}}%
\pgfpathlineto{\pgfqpoint{0.609943in}{0.683952in}}%
\pgfpathclose%
\pgfpathmoveto{\pgfqpoint{0.658397in}{0.690156in}}%
\pgfpathlineto{\pgfqpoint{0.653962in}{0.703083in}}%
\pgfpathlineto{\pgfqpoint{0.653153in}{0.716009in}}%
\pgfpathlineto{\pgfqpoint{0.653555in}{0.728936in}}%
\pgfpathlineto{\pgfqpoint{0.655783in}{0.741862in}}%
\pgfpathlineto{\pgfqpoint{0.663019in}{0.751361in}}%
\pgfpathlineto{\pgfqpoint{0.676288in}{0.754162in}}%
\pgfpathlineto{\pgfqpoint{0.689557in}{0.752536in}}%
\pgfpathlineto{\pgfqpoint{0.700442in}{0.741862in}}%
\pgfpathlineto{\pgfqpoint{0.702826in}{0.731431in}}%
\pgfpathlineto{\pgfqpoint{0.703153in}{0.728936in}}%
\pgfpathlineto{\pgfqpoint{0.703550in}{0.716009in}}%
\pgfpathlineto{\pgfqpoint{0.702826in}{0.705013in}}%
\pgfpathlineto{\pgfqpoint{0.702633in}{0.703083in}}%
\pgfpathlineto{\pgfqpoint{0.697029in}{0.690156in}}%
\pgfpathlineto{\pgfqpoint{0.689557in}{0.685795in}}%
\pgfpathlineto{\pgfqpoint{0.676288in}{0.684485in}}%
\pgfpathlineto{\pgfqpoint{0.663019in}{0.686610in}}%
\pgfpathclose%
\pgfpathmoveto{\pgfqpoint{0.711753in}{0.690156in}}%
\pgfpathlineto{\pgfqpoint{0.709795in}{0.703083in}}%
\pgfpathlineto{\pgfqpoint{0.709310in}{0.716009in}}%
\pgfpathlineto{\pgfqpoint{0.709217in}{0.728936in}}%
\pgfpathlineto{\pgfqpoint{0.709479in}{0.741862in}}%
\pgfpathlineto{\pgfqpoint{0.711575in}{0.754789in}}%
\pgfpathlineto{\pgfqpoint{0.716095in}{0.758288in}}%
\pgfpathlineto{\pgfqpoint{0.729364in}{0.759723in}}%
\pgfpathlineto{\pgfqpoint{0.742633in}{0.759981in}}%
\pgfpathlineto{\pgfqpoint{0.755902in}{0.759976in}}%
\pgfpathlineto{\pgfqpoint{0.766160in}{0.754789in}}%
\pgfpathlineto{\pgfqpoint{0.766417in}{0.741862in}}%
\pgfpathlineto{\pgfqpoint{0.766391in}{0.728936in}}%
\pgfpathlineto{\pgfqpoint{0.766258in}{0.716009in}}%
\pgfpathlineto{\pgfqpoint{0.765909in}{0.703083in}}%
\pgfpathlineto{\pgfqpoint{0.764573in}{0.690156in}}%
\pgfpathlineto{\pgfqpoint{0.755902in}{0.682038in}}%
\pgfpathlineto{\pgfqpoint{0.742633in}{0.681100in}}%
\pgfpathlineto{\pgfqpoint{0.729364in}{0.681432in}}%
\pgfpathlineto{\pgfqpoint{0.716095in}{0.684344in}}%
\pgfpathclose%
\pgfpathmoveto{\pgfqpoint{0.780331in}{0.690156in}}%
\pgfpathlineto{\pgfqpoint{0.773820in}{0.703083in}}%
\pgfpathlineto{\pgfqpoint{0.772590in}{0.716009in}}%
\pgfpathlineto{\pgfqpoint{0.773092in}{0.728936in}}%
\pgfpathlineto{\pgfqpoint{0.776128in}{0.741862in}}%
\pgfpathlineto{\pgfqpoint{0.782439in}{0.749276in}}%
\pgfpathlineto{\pgfqpoint{0.795708in}{0.752181in}}%
\pgfpathlineto{\pgfqpoint{0.808977in}{0.749545in}}%
\pgfpathlineto{\pgfqpoint{0.816036in}{0.741862in}}%
\pgfpathlineto{\pgfqpoint{0.819337in}{0.728936in}}%
\pgfpathlineto{\pgfqpoint{0.819878in}{0.716009in}}%
\pgfpathlineto{\pgfqpoint{0.818500in}{0.703083in}}%
\pgfpathlineto{\pgfqpoint{0.811470in}{0.690156in}}%
\pgfpathlineto{\pgfqpoint{0.808977in}{0.688544in}}%
\pgfpathlineto{\pgfqpoint{0.795708in}{0.686348in}}%
\pgfpathlineto{\pgfqpoint{0.782439in}{0.688710in}}%
\pgfpathclose%
\pgfpathmoveto{\pgfqpoint{1.374892in}{0.690156in}}%
\pgfpathlineto{\pgfqpoint{1.366273in}{0.698676in}}%
\pgfpathlineto{\pgfqpoint{1.364756in}{0.703083in}}%
\pgfpathlineto{\pgfqpoint{1.363345in}{0.716009in}}%
\pgfpathlineto{\pgfqpoint{1.363862in}{0.728936in}}%
\pgfpathlineto{\pgfqpoint{1.366273in}{0.739270in}}%
\pgfpathlineto{\pgfqpoint{1.367585in}{0.741862in}}%
\pgfpathlineto{\pgfqpoint{1.379542in}{0.750053in}}%
\pgfpathlineto{\pgfqpoint{1.392811in}{0.750114in}}%
\pgfpathlineto{\pgfqpoint{1.404871in}{0.741862in}}%
\pgfpathlineto{\pgfqpoint{1.406080in}{0.739418in}}%
\pgfpathlineto{\pgfqpoint{1.408474in}{0.728936in}}%
\pgfpathlineto{\pgfqpoint{1.408981in}{0.716009in}}%
\pgfpathlineto{\pgfqpoint{1.407624in}{0.703083in}}%
\pgfpathlineto{\pgfqpoint{1.406080in}{0.698499in}}%
\pgfpathlineto{\pgfqpoint{1.397715in}{0.690156in}}%
\pgfpathlineto{\pgfqpoint{1.392811in}{0.688214in}}%
\pgfpathlineto{\pgfqpoint{1.379542in}{0.688279in}}%
\pgfpathclose%
\pgfpathmoveto{\pgfqpoint{0.436446in}{0.767715in}}%
\pgfpathlineto{\pgfqpoint{0.424178in}{0.774005in}}%
\pgfpathlineto{\pgfqpoint{0.420806in}{0.780642in}}%
\pgfpathlineto{\pgfqpoint{0.418734in}{0.793568in}}%
\pgfpathlineto{\pgfqpoint{0.418503in}{0.806495in}}%
\pgfpathlineto{\pgfqpoint{0.419804in}{0.819421in}}%
\pgfpathlineto{\pgfqpoint{0.424178in}{0.829719in}}%
\pgfpathlineto{\pgfqpoint{0.427879in}{0.832348in}}%
\pgfpathlineto{\pgfqpoint{0.437447in}{0.835511in}}%
\pgfpathlineto{\pgfqpoint{0.450716in}{0.835557in}}%
\pgfpathlineto{\pgfqpoint{0.459881in}{0.832348in}}%
\pgfpathlineto{\pgfqpoint{0.463985in}{0.828516in}}%
\pgfpathlineto{\pgfqpoint{0.467017in}{0.819421in}}%
\pgfpathlineto{\pgfqpoint{0.468066in}{0.806495in}}%
\pgfpathlineto{\pgfqpoint{0.467851in}{0.793568in}}%
\pgfpathlineto{\pgfqpoint{0.466079in}{0.780642in}}%
\pgfpathlineto{\pgfqpoint{0.463985in}{0.775499in}}%
\pgfpathlineto{\pgfqpoint{0.451689in}{0.767715in}}%
\pgfpathlineto{\pgfqpoint{0.450716in}{0.767463in}}%
\pgfpathlineto{\pgfqpoint{0.437447in}{0.767458in}}%
\pgfpathclose%
\pgfpathmoveto{\pgfqpoint{0.478061in}{0.767715in}}%
\pgfpathlineto{\pgfqpoint{0.477254in}{0.768769in}}%
\pgfpathlineto{\pgfqpoint{0.474498in}{0.780642in}}%
\pgfpathlineto{\pgfqpoint{0.473991in}{0.793568in}}%
\pgfpathlineto{\pgfqpoint{0.474079in}{0.806495in}}%
\pgfpathlineto{\pgfqpoint{0.474814in}{0.819421in}}%
\pgfpathlineto{\pgfqpoint{0.477254in}{0.830083in}}%
\pgfpathlineto{\pgfqpoint{0.478676in}{0.832348in}}%
\pgfpathlineto{\pgfqpoint{0.490523in}{0.837546in}}%
\pgfpathlineto{\pgfqpoint{0.503792in}{0.837894in}}%
\pgfpathlineto{\pgfqpoint{0.517061in}{0.835758in}}%
\pgfpathlineto{\pgfqpoint{0.522090in}{0.832348in}}%
\pgfpathlineto{\pgfqpoint{0.527015in}{0.819421in}}%
\pgfpathlineto{\pgfqpoint{0.528071in}{0.806495in}}%
\pgfpathlineto{\pgfqpoint{0.528141in}{0.793568in}}%
\pgfpathlineto{\pgfqpoint{0.527240in}{0.780642in}}%
\pgfpathlineto{\pgfqpoint{0.521950in}{0.767715in}}%
\pgfpathlineto{\pgfqpoint{0.517061in}{0.764994in}}%
\pgfpathlineto{\pgfqpoint{0.503792in}{0.763234in}}%
\pgfpathlineto{\pgfqpoint{0.490523in}{0.763398in}}%
\pgfpathclose%
\pgfpathmoveto{\pgfqpoint{0.542681in}{0.767715in}}%
\pgfpathlineto{\pgfqpoint{0.535703in}{0.780642in}}%
\pgfpathlineto{\pgfqpoint{0.534395in}{0.793568in}}%
\pgfpathlineto{\pgfqpoint{0.534204in}{0.806495in}}%
\pgfpathlineto{\pgfqpoint{0.534895in}{0.819421in}}%
\pgfpathlineto{\pgfqpoint{0.538614in}{0.832348in}}%
\pgfpathlineto{\pgfqpoint{0.543598in}{0.836341in}}%
\pgfpathlineto{\pgfqpoint{0.556867in}{0.838637in}}%
\pgfpathlineto{\pgfqpoint{0.570136in}{0.838606in}}%
\pgfpathlineto{\pgfqpoint{0.583405in}{0.834748in}}%
\pgfpathlineto{\pgfqpoint{0.585156in}{0.832348in}}%
\pgfpathlineto{\pgfqpoint{0.587702in}{0.819421in}}%
\pgfpathlineto{\pgfqpoint{0.588129in}{0.806495in}}%
\pgfpathlineto{\pgfqpoint{0.587959in}{0.793568in}}%
\pgfpathlineto{\pgfqpoint{0.586985in}{0.780642in}}%
\pgfpathlineto{\pgfqpoint{0.583405in}{0.769975in}}%
\pgfpathlineto{\pgfqpoint{0.580720in}{0.767715in}}%
\pgfpathlineto{\pgfqpoint{0.570136in}{0.764556in}}%
\pgfpathlineto{\pgfqpoint{0.556867in}{0.764413in}}%
\pgfpathlineto{\pgfqpoint{0.543598in}{0.767158in}}%
\pgfpathclose%
\pgfpathmoveto{\pgfqpoint{0.605600in}{0.767715in}}%
\pgfpathlineto{\pgfqpoint{0.596674in}{0.776185in}}%
\pgfpathlineto{\pgfqpoint{0.595335in}{0.780642in}}%
\pgfpathlineto{\pgfqpoint{0.594056in}{0.793568in}}%
\pgfpathlineto{\pgfqpoint{0.594099in}{0.806495in}}%
\pgfpathlineto{\pgfqpoint{0.595437in}{0.819421in}}%
\pgfpathlineto{\pgfqpoint{0.596674in}{0.823704in}}%
\pgfpathlineto{\pgfqpoint{0.603969in}{0.832348in}}%
\pgfpathlineto{\pgfqpoint{0.609943in}{0.834711in}}%
\pgfpathlineto{\pgfqpoint{0.623212in}{0.835214in}}%
\pgfpathlineto{\pgfqpoint{0.633692in}{0.832348in}}%
\pgfpathlineto{\pgfqpoint{0.636481in}{0.830902in}}%
\pgfpathlineto{\pgfqpoint{0.642448in}{0.819421in}}%
\pgfpathlineto{\pgfqpoint{0.644022in}{0.806495in}}%
\pgfpathlineto{\pgfqpoint{0.644051in}{0.793568in}}%
\pgfpathlineto{\pgfqpoint{0.642480in}{0.780642in}}%
\pgfpathlineto{\pgfqpoint{0.636481in}{0.769946in}}%
\pgfpathlineto{\pgfqpoint{0.631084in}{0.767715in}}%
\pgfpathlineto{\pgfqpoint{0.623212in}{0.765958in}}%
\pgfpathlineto{\pgfqpoint{0.609943in}{0.766357in}}%
\pgfpathclose%
\pgfpathmoveto{\pgfqpoint{0.654267in}{0.767715in}}%
\pgfpathlineto{\pgfqpoint{0.650934in}{0.780642in}}%
\pgfpathlineto{\pgfqpoint{0.650285in}{0.793568in}}%
\pgfpathlineto{\pgfqpoint{0.650127in}{0.806495in}}%
\pgfpathlineto{\pgfqpoint{0.650289in}{0.819421in}}%
\pgfpathlineto{\pgfqpoint{0.651398in}{0.832348in}}%
\pgfpathlineto{\pgfqpoint{0.663019in}{0.840489in}}%
\pgfpathlineto{\pgfqpoint{0.676288in}{0.841099in}}%
\pgfpathlineto{\pgfqpoint{0.689557in}{0.841277in}}%
\pgfpathlineto{\pgfqpoint{0.702826in}{0.841375in}}%
\pgfpathlineto{\pgfqpoint{0.707887in}{0.832348in}}%
\pgfpathlineto{\pgfqpoint{0.707838in}{0.819421in}}%
\pgfpathlineto{\pgfqpoint{0.707771in}{0.806495in}}%
\pgfpathlineto{\pgfqpoint{0.707636in}{0.793568in}}%
\pgfpathlineto{\pgfqpoint{0.707294in}{0.780642in}}%
\pgfpathlineto{\pgfqpoint{0.705549in}{0.767715in}}%
\pgfpathlineto{\pgfqpoint{0.702826in}{0.764298in}}%
\pgfpathlineto{\pgfqpoint{0.689557in}{0.762035in}}%
\pgfpathlineto{\pgfqpoint{0.676288in}{0.762014in}}%
\pgfpathlineto{\pgfqpoint{0.663019in}{0.763060in}}%
\pgfpathclose%
\pgfpathmoveto{\pgfqpoint{0.307037in}{0.780642in}}%
\pgfpathlineto{\pgfqpoint{0.304758in}{0.786406in}}%
\pgfpathlineto{\pgfqpoint{0.303346in}{0.793568in}}%
\pgfpathlineto{\pgfqpoint{0.303067in}{0.806495in}}%
\pgfpathlineto{\pgfqpoint{0.304758in}{0.817824in}}%
\pgfpathlineto{\pgfqpoint{0.305212in}{0.819421in}}%
\pgfpathlineto{\pgfqpoint{0.318027in}{0.831459in}}%
\pgfpathlineto{\pgfqpoint{0.331295in}{0.832064in}}%
\pgfpathlineto{\pgfqpoint{0.344564in}{0.822367in}}%
\pgfpathlineto{\pgfqpoint{0.345735in}{0.819421in}}%
\pgfpathlineto{\pgfqpoint{0.347552in}{0.806495in}}%
\pgfpathlineto{\pgfqpoint{0.347280in}{0.793568in}}%
\pgfpathlineto{\pgfqpoint{0.344564in}{0.780796in}}%
\pgfpathlineto{\pgfqpoint{0.344494in}{0.780642in}}%
\pgfpathlineto{\pgfqpoint{0.331295in}{0.771242in}}%
\pgfpathlineto{\pgfqpoint{0.318027in}{0.771803in}}%
\pgfpathclose%
\pgfpathmoveto{\pgfqpoint{0.715629in}{0.780642in}}%
\pgfpathlineto{\pgfqpoint{0.713730in}{0.793568in}}%
\pgfpathlineto{\pgfqpoint{0.713739in}{0.806495in}}%
\pgfpathlineto{\pgfqpoint{0.715562in}{0.819421in}}%
\pgfpathlineto{\pgfqpoint{0.716095in}{0.820948in}}%
\pgfpathlineto{\pgfqpoint{0.728497in}{0.832348in}}%
\pgfpathlineto{\pgfqpoint{0.729364in}{0.832656in}}%
\pgfpathlineto{\pgfqpoint{0.742633in}{0.832953in}}%
\pgfpathlineto{\pgfqpoint{0.744601in}{0.832348in}}%
\pgfpathlineto{\pgfqpoint{0.755902in}{0.825150in}}%
\pgfpathlineto{\pgfqpoint{0.758527in}{0.819421in}}%
\pgfpathlineto{\pgfqpoint{0.760499in}{0.806495in}}%
\pgfpathlineto{\pgfqpoint{0.760496in}{0.793568in}}%
\pgfpathlineto{\pgfqpoint{0.758410in}{0.780642in}}%
\pgfpathlineto{\pgfqpoint{0.755902in}{0.775541in}}%
\pgfpathlineto{\pgfqpoint{0.742633in}{0.768335in}}%
\pgfpathlineto{\pgfqpoint{0.729364in}{0.768626in}}%
\pgfpathlineto{\pgfqpoint{0.716095in}{0.779378in}}%
\pgfpathclose%
\pgfpathmoveto{\pgfqpoint{0.422615in}{0.845274in}}%
\pgfpathlineto{\pgfqpoint{0.415426in}{0.858201in}}%
\pgfpathlineto{\pgfqpoint{0.414806in}{0.871127in}}%
\pgfpathlineto{\pgfqpoint{0.414769in}{0.884054in}}%
\pgfpathlineto{\pgfqpoint{0.415146in}{0.896980in}}%
\pgfpathlineto{\pgfqpoint{0.416741in}{0.909907in}}%
\pgfpathlineto{\pgfqpoint{0.424178in}{0.918323in}}%
\pgfpathlineto{\pgfqpoint{0.437447in}{0.919886in}}%
\pgfpathlineto{\pgfqpoint{0.450716in}{0.919330in}}%
\pgfpathlineto{\pgfqpoint{0.463985in}{0.914482in}}%
\pgfpathlineto{\pgfqpoint{0.466800in}{0.909907in}}%
\pgfpathlineto{\pgfqpoint{0.469130in}{0.896980in}}%
\pgfpathlineto{\pgfqpoint{0.469673in}{0.884054in}}%
\pgfpathlineto{\pgfqpoint{0.469535in}{0.871127in}}%
\pgfpathlineto{\pgfqpoint{0.468378in}{0.858201in}}%
\pgfpathlineto{\pgfqpoint{0.463985in}{0.848573in}}%
\pgfpathlineto{\pgfqpoint{0.456053in}{0.845274in}}%
\pgfpathlineto{\pgfqpoint{0.450716in}{0.844328in}}%
\pgfpathlineto{\pgfqpoint{0.437447in}{0.843816in}}%
\pgfpathlineto{\pgfqpoint{0.424178in}{0.844817in}}%
\pgfpathclose%
\pgfpathmoveto{\pgfqpoint{0.601269in}{0.845274in}}%
\pgfpathlineto{\pgfqpoint{0.596674in}{0.847574in}}%
\pgfpathlineto{\pgfqpoint{0.592492in}{0.858201in}}%
\pgfpathlineto{\pgfqpoint{0.591580in}{0.871127in}}%
\pgfpathlineto{\pgfqpoint{0.591358in}{0.884054in}}%
\pgfpathlineto{\pgfqpoint{0.591483in}{0.896980in}}%
\pgfpathlineto{\pgfqpoint{0.592322in}{0.909907in}}%
\pgfpathlineto{\pgfqpoint{0.596674in}{0.919011in}}%
\pgfpathlineto{\pgfqpoint{0.609943in}{0.921718in}}%
\pgfpathlineto{\pgfqpoint{0.623212in}{0.922132in}}%
\pgfpathlineto{\pgfqpoint{0.636481in}{0.922108in}}%
\pgfpathlineto{\pgfqpoint{0.648436in}{0.909907in}}%
\pgfpathlineto{\pgfqpoint{0.648580in}{0.896980in}}%
\pgfpathlineto{\pgfqpoint{0.648554in}{0.884054in}}%
\pgfpathlineto{\pgfqpoint{0.648400in}{0.871127in}}%
\pgfpathlineto{\pgfqpoint{0.647900in}{0.858201in}}%
\pgfpathlineto{\pgfqpoint{0.642951in}{0.845274in}}%
\pgfpathlineto{\pgfqpoint{0.636481in}{0.843620in}}%
\pgfpathlineto{\pgfqpoint{0.623212in}{0.843159in}}%
\pgfpathlineto{\pgfqpoint{0.609943in}{0.843649in}}%
\pgfpathclose%
\pgfpathmoveto{\pgfqpoint{0.364573in}{0.858201in}}%
\pgfpathlineto{\pgfqpoint{0.360792in}{0.871127in}}%
\pgfpathlineto{\pgfqpoint{0.360060in}{0.884054in}}%
\pgfpathlineto{\pgfqpoint{0.361090in}{0.896980in}}%
\pgfpathlineto{\pgfqpoint{0.366221in}{0.909907in}}%
\pgfpathlineto{\pgfqpoint{0.371102in}{0.913767in}}%
\pgfpathlineto{\pgfqpoint{0.384371in}{0.916178in}}%
\pgfpathlineto{\pgfqpoint{0.397640in}{0.914303in}}%
\pgfpathlineto{\pgfqpoint{0.403233in}{0.909907in}}%
\pgfpathlineto{\pgfqpoint{0.407675in}{0.896980in}}%
\pgfpathlineto{\pgfqpoint{0.408512in}{0.884054in}}%
\pgfpathlineto{\pgfqpoint{0.407882in}{0.871127in}}%
\pgfpathlineto{\pgfqpoint{0.404595in}{0.858201in}}%
\pgfpathlineto{\pgfqpoint{0.397640in}{0.851244in}}%
\pgfpathlineto{\pgfqpoint{0.384371in}{0.849094in}}%
\pgfpathlineto{\pgfqpoint{0.371102in}{0.851743in}}%
\pgfpathclose%
\pgfpathmoveto{\pgfqpoint{0.477871in}{0.858201in}}%
\pgfpathlineto{\pgfqpoint{0.477254in}{0.860573in}}%
\pgfpathlineto{\pgfqpoint{0.475760in}{0.871127in}}%
\pgfpathlineto{\pgfqpoint{0.475350in}{0.884054in}}%
\pgfpathlineto{\pgfqpoint{0.475827in}{0.896980in}}%
\pgfpathlineto{\pgfqpoint{0.477254in}{0.905971in}}%
\pgfpathlineto{\pgfqpoint{0.478564in}{0.909907in}}%
\pgfpathlineto{\pgfqpoint{0.490523in}{0.918311in}}%
\pgfpathlineto{\pgfqpoint{0.503792in}{0.919417in}}%
\pgfpathlineto{\pgfqpoint{0.517061in}{0.918286in}}%
\pgfpathlineto{\pgfqpoint{0.526442in}{0.909907in}}%
\pgfpathlineto{\pgfqpoint{0.528433in}{0.896980in}}%
\pgfpathlineto{\pgfqpoint{0.528779in}{0.884054in}}%
\pgfpathlineto{\pgfqpoint{0.528420in}{0.871127in}}%
\pgfpathlineto{\pgfqpoint{0.526719in}{0.858201in}}%
\pgfpathlineto{\pgfqpoint{0.517061in}{0.847268in}}%
\pgfpathlineto{\pgfqpoint{0.503792in}{0.845777in}}%
\pgfpathlineto{\pgfqpoint{0.490523in}{0.847056in}}%
\pgfpathclose%
\pgfpathmoveto{\pgfqpoint{0.537268in}{0.858201in}}%
\pgfpathlineto{\pgfqpoint{0.535260in}{0.871127in}}%
\pgfpathlineto{\pgfqpoint{0.534988in}{0.884054in}}%
\pgfpathlineto{\pgfqpoint{0.535808in}{0.896980in}}%
\pgfpathlineto{\pgfqpoint{0.539574in}{0.909907in}}%
\pgfpathlineto{\pgfqpoint{0.543598in}{0.913920in}}%
\pgfpathlineto{\pgfqpoint{0.556867in}{0.916922in}}%
\pgfpathlineto{\pgfqpoint{0.570136in}{0.915868in}}%
\pgfpathlineto{\pgfqpoint{0.579948in}{0.909907in}}%
\pgfpathlineto{\pgfqpoint{0.583405in}{0.903099in}}%
\pgfpathlineto{\pgfqpoint{0.584784in}{0.896980in}}%
\pgfpathlineto{\pgfqpoint{0.585682in}{0.884054in}}%
\pgfpathlineto{\pgfqpoint{0.585354in}{0.871127in}}%
\pgfpathlineto{\pgfqpoint{0.583405in}{0.859443in}}%
\pgfpathlineto{\pgfqpoint{0.582996in}{0.858201in}}%
\pgfpathlineto{\pgfqpoint{0.570136in}{0.847929in}}%
\pgfpathlineto{\pgfqpoint{0.556867in}{0.846889in}}%
\pgfpathlineto{\pgfqpoint{0.543598in}{0.849692in}}%
\pgfpathclose%
\pgfpathmoveto{\pgfqpoint{0.658533in}{0.858201in}}%
\pgfpathlineto{\pgfqpoint{0.655319in}{0.871127in}}%
\pgfpathlineto{\pgfqpoint{0.654844in}{0.884054in}}%
\pgfpathlineto{\pgfqpoint{0.656052in}{0.896980in}}%
\pgfpathlineto{\pgfqpoint{0.661703in}{0.909907in}}%
\pgfpathlineto{\pgfqpoint{0.663019in}{0.911075in}}%
\pgfpathlineto{\pgfqpoint{0.676288in}{0.914561in}}%
\pgfpathlineto{\pgfqpoint{0.689557in}{0.912687in}}%
\pgfpathlineto{\pgfqpoint{0.693690in}{0.909907in}}%
\pgfpathlineto{\pgfqpoint{0.700376in}{0.896980in}}%
\pgfpathlineto{\pgfqpoint{0.701846in}{0.884054in}}%
\pgfpathlineto{\pgfqpoint{0.701247in}{0.871127in}}%
\pgfpathlineto{\pgfqpoint{0.697338in}{0.858201in}}%
\pgfpathlineto{\pgfqpoint{0.689557in}{0.851298in}}%
\pgfpathlineto{\pgfqpoint{0.676288in}{0.849421in}}%
\pgfpathlineto{\pgfqpoint{0.663019in}{0.852858in}}%
\pgfpathclose%
\pgfpathmoveto{\pgfqpoint{0.310739in}{0.935760in}}%
\pgfpathlineto{\pgfqpoint{0.304758in}{0.943346in}}%
\pgfpathlineto{\pgfqpoint{0.303103in}{0.948686in}}%
\pgfpathlineto{\pgfqpoint{0.301857in}{0.961613in}}%
\pgfpathlineto{\pgfqpoint{0.302391in}{0.974539in}}%
\pgfpathlineto{\pgfqpoint{0.304758in}{0.984610in}}%
\pgfpathlineto{\pgfqpoint{0.306245in}{0.987466in}}%
\pgfpathlineto{\pgfqpoint{0.318027in}{0.995260in}}%
\pgfpathlineto{\pgfqpoint{0.331295in}{0.995782in}}%
\pgfpathlineto{\pgfqpoint{0.344564in}{0.988755in}}%
\pgfpathlineto{\pgfqpoint{0.345312in}{0.987466in}}%
\pgfpathlineto{\pgfqpoint{0.348342in}{0.974539in}}%
\pgfpathlineto{\pgfqpoint{0.348800in}{0.961613in}}%
\pgfpathlineto{\pgfqpoint{0.347687in}{0.948686in}}%
\pgfpathlineto{\pgfqpoint{0.344564in}{0.939408in}}%
\pgfpathlineto{\pgfqpoint{0.341088in}{0.935760in}}%
\pgfpathlineto{\pgfqpoint{0.331295in}{0.931624in}}%
\pgfpathlineto{\pgfqpoint{0.318027in}{0.932116in}}%
\pgfpathclose%
\pgfpathmoveto{\pgfqpoint{0.355166in}{0.935760in}}%
\pgfpathlineto{\pgfqpoint{0.354812in}{0.948686in}}%
\pgfpathlineto{\pgfqpoint{0.354776in}{0.961613in}}%
\pgfpathlineto{\pgfqpoint{0.354901in}{0.974539in}}%
\pgfpathlineto{\pgfqpoint{0.355381in}{0.987466in}}%
\pgfpathlineto{\pgfqpoint{0.357833in}{0.998138in}}%
\pgfpathlineto{\pgfqpoint{0.360735in}{1.000392in}}%
\pgfpathlineto{\pgfqpoint{0.371102in}{1.002236in}}%
\pgfpathlineto{\pgfqpoint{0.384371in}{1.002277in}}%
\pgfpathlineto{\pgfqpoint{0.397640in}{1.001106in}}%
\pgfpathlineto{\pgfqpoint{0.400251in}{1.000392in}}%
\pgfpathlineto{\pgfqpoint{0.410101in}{0.987466in}}%
\pgfpathlineto{\pgfqpoint{0.410909in}{0.980123in}}%
\pgfpathlineto{\pgfqpoint{0.411285in}{0.974539in}}%
\pgfpathlineto{\pgfqpoint{0.411519in}{0.961613in}}%
\pgfpathlineto{\pgfqpoint{0.411261in}{0.948686in}}%
\pgfpathlineto{\pgfqpoint{0.410909in}{0.943875in}}%
\pgfpathlineto{\pgfqpoint{0.409788in}{0.935760in}}%
\pgfpathlineto{\pgfqpoint{0.397640in}{0.925383in}}%
\pgfpathlineto{\pgfqpoint{0.384371in}{0.924343in}}%
\pgfpathlineto{\pgfqpoint{0.371102in}{0.924242in}}%
\pgfpathlineto{\pgfqpoint{0.357833in}{0.926418in}}%
\pgfpathclose%
\pgfpathmoveto{\pgfqpoint{0.421750in}{0.935760in}}%
\pgfpathlineto{\pgfqpoint{0.418122in}{0.948686in}}%
\pgfpathlineto{\pgfqpoint{0.417320in}{0.961613in}}%
\pgfpathlineto{\pgfqpoint{0.417626in}{0.974539in}}%
\pgfpathlineto{\pgfqpoint{0.419672in}{0.987466in}}%
\pgfpathlineto{\pgfqpoint{0.424178in}{0.994853in}}%
\pgfpathlineto{\pgfqpoint{0.437447in}{0.999384in}}%
\pgfpathlineto{\pgfqpoint{0.450716in}{0.999517in}}%
\pgfpathlineto{\pgfqpoint{0.463985in}{0.994633in}}%
\pgfpathlineto{\pgfqpoint{0.467481in}{0.987466in}}%
\pgfpathlineto{\pgfqpoint{0.469034in}{0.974539in}}%
\pgfpathlineto{\pgfqpoint{0.469249in}{0.961613in}}%
\pgfpathlineto{\pgfqpoint{0.468620in}{0.948686in}}%
\pgfpathlineto{\pgfqpoint{0.465692in}{0.935760in}}%
\pgfpathlineto{\pgfqpoint{0.463985in}{0.933219in}}%
\pgfpathlineto{\pgfqpoint{0.450716in}{0.928135in}}%
\pgfpathlineto{\pgfqpoint{0.437447in}{0.928233in}}%
\pgfpathlineto{\pgfqpoint{0.424178in}{0.932799in}}%
\pgfpathclose%
\pgfpathmoveto{\pgfqpoint{0.478519in}{0.935760in}}%
\pgfpathlineto{\pgfqpoint{0.477254in}{0.939142in}}%
\pgfpathlineto{\pgfqpoint{0.475643in}{0.948686in}}%
\pgfpathlineto{\pgfqpoint{0.475143in}{0.961613in}}%
\pgfpathlineto{\pgfqpoint{0.475502in}{0.974539in}}%
\pgfpathlineto{\pgfqpoint{0.477254in}{0.986845in}}%
\pgfpathlineto{\pgfqpoint{0.477421in}{0.987466in}}%
\pgfpathlineto{\pgfqpoint{0.490523in}{0.998359in}}%
\pgfpathlineto{\pgfqpoint{0.503792in}{0.999018in}}%
\pgfpathlineto{\pgfqpoint{0.517061in}{0.996054in}}%
\pgfpathlineto{\pgfqpoint{0.524064in}{0.987466in}}%
\pgfpathlineto{\pgfqpoint{0.526517in}{0.974539in}}%
\pgfpathlineto{\pgfqpoint{0.526983in}{0.961613in}}%
\pgfpathlineto{\pgfqpoint{0.526283in}{0.948686in}}%
\pgfpathlineto{\pgfqpoint{0.522805in}{0.935760in}}%
\pgfpathlineto{\pgfqpoint{0.517061in}{0.930286in}}%
\pgfpathlineto{\pgfqpoint{0.503792in}{0.927625in}}%
\pgfpathlineto{\pgfqpoint{0.490523in}{0.928168in}}%
\pgfpathclose%
\pgfpathmoveto{\pgfqpoint{0.535237in}{0.935760in}}%
\pgfpathlineto{\pgfqpoint{0.533447in}{0.948686in}}%
\pgfpathlineto{\pgfqpoint{0.533034in}{0.961613in}}%
\pgfpathlineto{\pgfqpoint{0.533139in}{0.974539in}}%
\pgfpathlineto{\pgfqpoint{0.534016in}{0.987466in}}%
\pgfpathlineto{\pgfqpoint{0.541643in}{1.000392in}}%
\pgfpathlineto{\pgfqpoint{0.543598in}{1.001067in}}%
\pgfpathlineto{\pgfqpoint{0.556867in}{1.002532in}}%
\pgfpathlineto{\pgfqpoint{0.570136in}{1.002666in}}%
\pgfpathlineto{\pgfqpoint{0.583405in}{1.001191in}}%
\pgfpathlineto{\pgfqpoint{0.584964in}{1.000392in}}%
\pgfpathlineto{\pgfqpoint{0.588887in}{0.987466in}}%
\pgfpathlineto{\pgfqpoint{0.589247in}{0.974539in}}%
\pgfpathlineto{\pgfqpoint{0.589263in}{0.961613in}}%
\pgfpathlineto{\pgfqpoint{0.589026in}{0.948686in}}%
\pgfpathlineto{\pgfqpoint{0.588034in}{0.935760in}}%
\pgfpathlineto{\pgfqpoint{0.583405in}{0.927348in}}%
\pgfpathlineto{\pgfqpoint{0.570136in}{0.925036in}}%
\pgfpathlineto{\pgfqpoint{0.556867in}{0.925110in}}%
\pgfpathlineto{\pgfqpoint{0.543598in}{0.926883in}}%
\pgfpathclose%
\pgfpathmoveto{\pgfqpoint{0.601719in}{0.935760in}}%
\pgfpathlineto{\pgfqpoint{0.596674in}{0.945571in}}%
\pgfpathlineto{\pgfqpoint{0.595997in}{0.948686in}}%
\pgfpathlineto{\pgfqpoint{0.595116in}{0.961613in}}%
\pgfpathlineto{\pgfqpoint{0.595668in}{0.974539in}}%
\pgfpathlineto{\pgfqpoint{0.596674in}{0.980110in}}%
\pgfpathlineto{\pgfqpoint{0.599357in}{0.987466in}}%
\pgfpathlineto{\pgfqpoint{0.609943in}{0.995369in}}%
\pgfpathlineto{\pgfqpoint{0.623212in}{0.996098in}}%
\pgfpathlineto{\pgfqpoint{0.636481in}{0.990846in}}%
\pgfpathlineto{\pgfqpoint{0.638934in}{0.987466in}}%
\pgfpathlineto{\pgfqpoint{0.642368in}{0.974539in}}%
\pgfpathlineto{\pgfqpoint{0.642998in}{0.961613in}}%
\pgfpathlineto{\pgfqpoint{0.641970in}{0.948686in}}%
\pgfpathlineto{\pgfqpoint{0.636919in}{0.935760in}}%
\pgfpathlineto{\pgfqpoint{0.636481in}{0.935291in}}%
\pgfpathlineto{\pgfqpoint{0.623212in}{0.930438in}}%
\pgfpathlineto{\pgfqpoint{0.609943in}{0.931086in}}%
\pgfpathclose%
\pgfpathmoveto{\pgfqpoint{0.369215in}{1.013319in}}%
\pgfpathlineto{\pgfqpoint{0.361021in}{1.026245in}}%
\pgfpathlineto{\pgfqpoint{0.359341in}{1.039172in}}%
\pgfpathlineto{\pgfqpoint{0.359406in}{1.052098in}}%
\pgfpathlineto{\pgfqpoint{0.361355in}{1.065025in}}%
\pgfpathlineto{\pgfqpoint{0.371102in}{1.077481in}}%
\pgfpathlineto{\pgfqpoint{0.373202in}{1.077952in}}%
\pgfpathlineto{\pgfqpoint{0.384371in}{1.079632in}}%
\pgfpathlineto{\pgfqpoint{0.397640in}{1.078123in}}%
\pgfpathlineto{\pgfqpoint{0.398041in}{1.077952in}}%
\pgfpathlineto{\pgfqpoint{0.407658in}{1.065025in}}%
\pgfpathlineto{\pgfqpoint{0.409221in}{1.052098in}}%
\pgfpathlineto{\pgfqpoint{0.409264in}{1.039172in}}%
\pgfpathlineto{\pgfqpoint{0.407902in}{1.026245in}}%
\pgfpathlineto{\pgfqpoint{0.400664in}{1.013319in}}%
\pgfpathlineto{\pgfqpoint{0.397640in}{1.011709in}}%
\pgfpathlineto{\pgfqpoint{0.384371in}{1.010126in}}%
\pgfpathlineto{\pgfqpoint{0.371102in}{1.012267in}}%
\pgfpathclose%
\pgfpathmoveto{\pgfqpoint{0.419656in}{1.013319in}}%
\pgfpathlineto{\pgfqpoint{0.416214in}{1.026245in}}%
\pgfpathlineto{\pgfqpoint{0.415585in}{1.039172in}}%
\pgfpathlineto{\pgfqpoint{0.415658in}{1.052098in}}%
\pgfpathlineto{\pgfqpoint{0.416545in}{1.065025in}}%
\pgfpathlineto{\pgfqpoint{0.421822in}{1.077952in}}%
\pgfpathlineto{\pgfqpoint{0.424178in}{1.079436in}}%
\pgfpathlineto{\pgfqpoint{0.437447in}{1.081582in}}%
\pgfpathlineto{\pgfqpoint{0.450716in}{1.080993in}}%
\pgfpathlineto{\pgfqpoint{0.460069in}{1.077952in}}%
\pgfpathlineto{\pgfqpoint{0.463985in}{1.074782in}}%
\pgfpathlineto{\pgfqpoint{0.467606in}{1.065025in}}%
\pgfpathlineto{\pgfqpoint{0.468797in}{1.052098in}}%
\pgfpathlineto{\pgfqpoint{0.468885in}{1.039172in}}%
\pgfpathlineto{\pgfqpoint{0.467998in}{1.026245in}}%
\pgfpathlineto{\pgfqpoint{0.463985in}{1.013991in}}%
\pgfpathlineto{\pgfqpoint{0.463342in}{1.013319in}}%
\pgfpathlineto{\pgfqpoint{0.450716in}{1.008282in}}%
\pgfpathlineto{\pgfqpoint{0.437447in}{1.007700in}}%
\pgfpathlineto{\pgfqpoint{0.424178in}{1.009696in}}%
\pgfpathclose%
\pgfpathmoveto{\pgfqpoint{0.479689in}{1.013319in}}%
\pgfpathlineto{\pgfqpoint{0.477254in}{1.017790in}}%
\pgfpathlineto{\pgfqpoint{0.475457in}{1.026245in}}%
\pgfpathlineto{\pgfqpoint{0.474680in}{1.039172in}}%
\pgfpathlineto{\pgfqpoint{0.474694in}{1.052098in}}%
\pgfpathlineto{\pgfqpoint{0.475550in}{1.065025in}}%
\pgfpathlineto{\pgfqpoint{0.477254in}{1.072432in}}%
\pgfpathlineto{\pgfqpoint{0.480967in}{1.077952in}}%
\pgfpathlineto{\pgfqpoint{0.490523in}{1.081851in}}%
\pgfpathlineto{\pgfqpoint{0.503792in}{1.082754in}}%
\pgfpathlineto{\pgfqpoint{0.517061in}{1.082036in}}%
\pgfpathlineto{\pgfqpoint{0.525433in}{1.077952in}}%
\pgfpathlineto{\pgfqpoint{0.528982in}{1.065025in}}%
\pgfpathlineto{\pgfqpoint{0.529557in}{1.052098in}}%
\pgfpathlineto{\pgfqpoint{0.529553in}{1.039172in}}%
\pgfpathlineto{\pgfqpoint{0.528995in}{1.026245in}}%
\pgfpathlineto{\pgfqpoint{0.526100in}{1.013319in}}%
\pgfpathlineto{\pgfqpoint{0.517061in}{1.007835in}}%
\pgfpathlineto{\pgfqpoint{0.503792in}{1.007011in}}%
\pgfpathlineto{\pgfqpoint{0.490523in}{1.007959in}}%
\pgfpathclose%
\pgfpathmoveto{\pgfqpoint{0.545319in}{1.013319in}}%
\pgfpathlineto{\pgfqpoint{0.543598in}{1.014098in}}%
\pgfpathlineto{\pgfqpoint{0.537226in}{1.026245in}}%
\pgfpathlineto{\pgfqpoint{0.535870in}{1.039172in}}%
\pgfpathlineto{\pgfqpoint{0.535988in}{1.052098in}}%
\pgfpathlineto{\pgfqpoint{0.537773in}{1.065025in}}%
\pgfpathlineto{\pgfqpoint{0.543598in}{1.074846in}}%
\pgfpathlineto{\pgfqpoint{0.552759in}{1.077952in}}%
\pgfpathlineto{\pgfqpoint{0.556867in}{1.078706in}}%
\pgfpathlineto{\pgfqpoint{0.566059in}{1.077952in}}%
\pgfpathlineto{\pgfqpoint{0.570136in}{1.077516in}}%
\pgfpathlineto{\pgfqpoint{0.582644in}{1.065025in}}%
\pgfpathlineto{\pgfqpoint{0.583405in}{1.061851in}}%
\pgfpathlineto{\pgfqpoint{0.584731in}{1.052098in}}%
\pgfpathlineto{\pgfqpoint{0.584853in}{1.039172in}}%
\pgfpathlineto{\pgfqpoint{0.583405in}{1.026382in}}%
\pgfpathlineto{\pgfqpoint{0.583379in}{1.026245in}}%
\pgfpathlineto{\pgfqpoint{0.573744in}{1.013319in}}%
\pgfpathlineto{\pgfqpoint{0.570136in}{1.011729in}}%
\pgfpathlineto{\pgfqpoint{0.556867in}{1.010613in}}%
\pgfpathclose%
\pgfpathmoveto{\pgfqpoint{0.359447in}{1.090878in}}%
\pgfpathlineto{\pgfqpoint{0.357833in}{1.092987in}}%
\pgfpathlineto{\pgfqpoint{0.355708in}{1.103805in}}%
\pgfpathlineto{\pgfqpoint{0.355243in}{1.116731in}}%
\pgfpathlineto{\pgfqpoint{0.355186in}{1.129658in}}%
\pgfpathlineto{\pgfqpoint{0.355426in}{1.142584in}}%
\pgfpathlineto{\pgfqpoint{0.356626in}{1.155511in}}%
\pgfpathlineto{\pgfqpoint{0.357833in}{1.158983in}}%
\pgfpathlineto{\pgfqpoint{0.371102in}{1.164217in}}%
\pgfpathlineto{\pgfqpoint{0.384371in}{1.164336in}}%
\pgfpathlineto{\pgfqpoint{0.397640in}{1.163045in}}%
\pgfpathlineto{\pgfqpoint{0.407951in}{1.155511in}}%
\pgfpathlineto{\pgfqpoint{0.410574in}{1.142584in}}%
\pgfpathlineto{\pgfqpoint{0.410909in}{1.135462in}}%
\pgfpathlineto{\pgfqpoint{0.411112in}{1.129658in}}%
\pgfpathlineto{\pgfqpoint{0.410990in}{1.116731in}}%
\pgfpathlineto{\pgfqpoint{0.410909in}{1.115259in}}%
\pgfpathlineto{\pgfqpoint{0.409932in}{1.103805in}}%
\pgfpathlineto{\pgfqpoint{0.403115in}{1.090878in}}%
\pgfpathlineto{\pgfqpoint{0.397640in}{1.088763in}}%
\pgfpathlineto{\pgfqpoint{0.384371in}{1.087492in}}%
\pgfpathlineto{\pgfqpoint{0.371102in}{1.087609in}}%
\pgfpathclose%
\pgfpathmoveto{\pgfqpoint{0.431871in}{1.090878in}}%
\pgfpathlineto{\pgfqpoint{0.424178in}{1.093965in}}%
\pgfpathlineto{\pgfqpoint{0.418760in}{1.103805in}}%
\pgfpathlineto{\pgfqpoint{0.417157in}{1.116731in}}%
\pgfpathlineto{\pgfqpoint{0.416956in}{1.129658in}}%
\pgfpathlineto{\pgfqpoint{0.417794in}{1.142584in}}%
\pgfpathlineto{\pgfqpoint{0.421752in}{1.155511in}}%
\pgfpathlineto{\pgfqpoint{0.424178in}{1.158093in}}%
\pgfpathlineto{\pgfqpoint{0.437447in}{1.162063in}}%
\pgfpathlineto{\pgfqpoint{0.450716in}{1.162204in}}%
\pgfpathlineto{\pgfqpoint{0.463985in}{1.158064in}}%
\pgfpathlineto{\pgfqpoint{0.465967in}{1.155511in}}%
\pgfpathlineto{\pgfqpoint{0.469015in}{1.142584in}}%
\pgfpathlineto{\pgfqpoint{0.469627in}{1.129658in}}%
\pgfpathlineto{\pgfqpoint{0.469481in}{1.116731in}}%
\pgfpathlineto{\pgfqpoint{0.468297in}{1.103805in}}%
\pgfpathlineto{\pgfqpoint{0.463985in}{1.093997in}}%
\pgfpathlineto{\pgfqpoint{0.457315in}{1.090878in}}%
\pgfpathlineto{\pgfqpoint{0.450716in}{1.089592in}}%
\pgfpathlineto{\pgfqpoint{0.437447in}{1.089730in}}%
\pgfpathclose%
\pgfpathmoveto{\pgfqpoint{0.498952in}{1.090878in}}%
\pgfpathlineto{\pgfqpoint{0.490523in}{1.091386in}}%
\pgfpathlineto{\pgfqpoint{0.477393in}{1.103805in}}%
\pgfpathlineto{\pgfqpoint{0.477254in}{1.104449in}}%
\pgfpathlineto{\pgfqpoint{0.475763in}{1.116731in}}%
\pgfpathlineto{\pgfqpoint{0.475566in}{1.129658in}}%
\pgfpathlineto{\pgfqpoint{0.476394in}{1.142584in}}%
\pgfpathlineto{\pgfqpoint{0.477254in}{1.147003in}}%
\pgfpathlineto{\pgfqpoint{0.481297in}{1.155511in}}%
\pgfpathlineto{\pgfqpoint{0.490523in}{1.160437in}}%
\pgfpathlineto{\pgfqpoint{0.503792in}{1.161120in}}%
\pgfpathlineto{\pgfqpoint{0.517061in}{1.158175in}}%
\pgfpathlineto{\pgfqpoint{0.520335in}{1.155511in}}%
\pgfpathlineto{\pgfqpoint{0.525453in}{1.142584in}}%
\pgfpathlineto{\pgfqpoint{0.526535in}{1.129658in}}%
\pgfpathlineto{\pgfqpoint{0.526276in}{1.116731in}}%
\pgfpathlineto{\pgfqpoint{0.524206in}{1.103805in}}%
\pgfpathlineto{\pgfqpoint{0.517061in}{1.093875in}}%
\pgfpathlineto{\pgfqpoint{0.505196in}{1.090878in}}%
\pgfpathlineto{\pgfqpoint{0.503792in}{1.090660in}}%
\pgfpathclose%
\pgfpathmoveto{\pgfqpoint{0.537595in}{1.090878in}}%
\pgfpathlineto{\pgfqpoint{0.533372in}{1.103805in}}%
\pgfpathlineto{\pgfqpoint{0.532725in}{1.116731in}}%
\pgfpathlineto{\pgfqpoint{0.532644in}{1.129658in}}%
\pgfpathlineto{\pgfqpoint{0.532982in}{1.142584in}}%
\pgfpathlineto{\pgfqpoint{0.534585in}{1.155511in}}%
\pgfpathlineto{\pgfqpoint{0.543598in}{1.163843in}}%
\pgfpathlineto{\pgfqpoint{0.556867in}{1.165145in}}%
\pgfpathlineto{\pgfqpoint{0.570136in}{1.165313in}}%
\pgfpathlineto{\pgfqpoint{0.583405in}{1.164312in}}%
\pgfpathlineto{\pgfqpoint{0.588980in}{1.155511in}}%
\pgfpathlineto{\pgfqpoint{0.589541in}{1.142584in}}%
\pgfpathlineto{\pgfqpoint{0.589653in}{1.129658in}}%
\pgfpathlineto{\pgfqpoint{0.589626in}{1.116731in}}%
\pgfpathlineto{\pgfqpoint{0.589408in}{1.103805in}}%
\pgfpathlineto{\pgfqpoint{0.587799in}{1.090878in}}%
\pgfpathlineto{\pgfqpoint{0.583405in}{1.087516in}}%
\pgfpathlineto{\pgfqpoint{0.570136in}{1.086530in}}%
\pgfpathlineto{\pgfqpoint{0.556867in}{1.086696in}}%
\pgfpathlineto{\pgfqpoint{0.543598in}{1.087978in}}%
\pgfpathclose%
\pgfpathmoveto{\pgfqpoint{0.304541in}{1.103805in}}%
\pgfpathlineto{\pgfqpoint{0.301855in}{1.116731in}}%
\pgfpathlineto{\pgfqpoint{0.301518in}{1.129658in}}%
\pgfpathlineto{\pgfqpoint{0.302924in}{1.142584in}}%
\pgfpathlineto{\pgfqpoint{0.304758in}{1.148000in}}%
\pgfpathlineto{\pgfqpoint{0.311784in}{1.155511in}}%
\pgfpathlineto{\pgfqpoint{0.318027in}{1.158232in}}%
\pgfpathlineto{\pgfqpoint{0.331295in}{1.158716in}}%
\pgfpathlineto{\pgfqpoint{0.340190in}{1.155511in}}%
\pgfpathlineto{\pgfqpoint{0.344564in}{1.151728in}}%
\pgfpathlineto{\pgfqpoint{0.347939in}{1.142584in}}%
\pgfpathlineto{\pgfqpoint{0.349164in}{1.129658in}}%
\pgfpathlineto{\pgfqpoint{0.348873in}{1.116731in}}%
\pgfpathlineto{\pgfqpoint{0.346496in}{1.103805in}}%
\pgfpathlineto{\pgfqpoint{0.344564in}{1.100101in}}%
\pgfpathlineto{\pgfqpoint{0.331295in}{1.093279in}}%
\pgfpathlineto{\pgfqpoint{0.318027in}{1.093812in}}%
\pgfpathlineto{\pgfqpoint{0.304758in}{1.103346in}}%
\pgfpathclose%
\pgfpathmoveto{\pgfqpoint{0.598952in}{1.103805in}}%
\pgfpathlineto{\pgfqpoint{0.596674in}{1.111512in}}%
\pgfpathlineto{\pgfqpoint{0.595864in}{1.116731in}}%
\pgfpathlineto{\pgfqpoint{0.595552in}{1.129658in}}%
\pgfpathlineto{\pgfqpoint{0.596674in}{1.141045in}}%
\pgfpathlineto{\pgfqpoint{0.596922in}{1.142584in}}%
\pgfpathlineto{\pgfqpoint{0.605536in}{1.155511in}}%
\pgfpathlineto{\pgfqpoint{0.609943in}{1.157628in}}%
\pgfpathlineto{\pgfqpoint{0.623212in}{1.158357in}}%
\pgfpathlineto{\pgfqpoint{0.631965in}{1.155511in}}%
\pgfpathlineto{\pgfqpoint{0.636481in}{1.152577in}}%
\pgfpathlineto{\pgfqpoint{0.641052in}{1.142584in}}%
\pgfpathlineto{\pgfqpoint{0.642548in}{1.129658in}}%
\pgfpathlineto{\pgfqpoint{0.642190in}{1.116731in}}%
\pgfpathlineto{\pgfqpoint{0.639323in}{1.103805in}}%
\pgfpathlineto{\pgfqpoint{0.636481in}{1.099360in}}%
\pgfpathlineto{\pgfqpoint{0.623212in}{1.093674in}}%
\pgfpathlineto{\pgfqpoint{0.609943in}{1.094477in}}%
\pgfpathclose%
\pgfpathmoveto{\pgfqpoint{0.363609in}{1.181364in}}%
\pgfpathlineto{\pgfqpoint{0.359914in}{1.194290in}}%
\pgfpathlineto{\pgfqpoint{0.359197in}{1.207217in}}%
\pgfpathlineto{\pgfqpoint{0.359999in}{1.220143in}}%
\pgfpathlineto{\pgfqpoint{0.363891in}{1.233070in}}%
\pgfpathlineto{\pgfqpoint{0.371102in}{1.239439in}}%
\pgfpathlineto{\pgfqpoint{0.384371in}{1.241682in}}%
\pgfpathlineto{\pgfqpoint{0.397640in}{1.240024in}}%
\pgfpathlineto{\pgfqpoint{0.405487in}{1.233070in}}%
\pgfpathlineto{\pgfqpoint{0.408734in}{1.220143in}}%
\pgfpathlineto{\pgfqpoint{0.409381in}{1.207217in}}%
\pgfpathlineto{\pgfqpoint{0.408820in}{1.194290in}}%
\pgfpathlineto{\pgfqpoint{0.405777in}{1.181364in}}%
\pgfpathlineto{\pgfqpoint{0.397640in}{1.174054in}}%
\pgfpathlineto{\pgfqpoint{0.384371in}{1.172463in}}%
\pgfpathlineto{\pgfqpoint{0.371102in}{1.174661in}}%
\pgfpathclose%
\pgfpathmoveto{\pgfqpoint{0.417586in}{1.181364in}}%
\pgfpathlineto{\pgfqpoint{0.415892in}{1.194290in}}%
\pgfpathlineto{\pgfqpoint{0.415546in}{1.207217in}}%
\pgfpathlineto{\pgfqpoint{0.415823in}{1.220143in}}%
\pgfpathlineto{\pgfqpoint{0.417362in}{1.233070in}}%
\pgfpathlineto{\pgfqpoint{0.424178in}{1.242131in}}%
\pgfpathlineto{\pgfqpoint{0.437447in}{1.244223in}}%
\pgfpathlineto{\pgfqpoint{0.450716in}{1.243613in}}%
\pgfpathlineto{\pgfqpoint{0.463985in}{1.237755in}}%
\pgfpathlineto{\pgfqpoint{0.466442in}{1.233070in}}%
\pgfpathlineto{\pgfqpoint{0.468545in}{1.220143in}}%
\pgfpathlineto{\pgfqpoint{0.468945in}{1.207217in}}%
\pgfpathlineto{\pgfqpoint{0.468481in}{1.194290in}}%
\pgfpathlineto{\pgfqpoint{0.466245in}{1.181364in}}%
\pgfpathlineto{\pgfqpoint{0.463985in}{1.177105in}}%
\pgfpathlineto{\pgfqpoint{0.450716in}{1.171029in}}%
\pgfpathlineto{\pgfqpoint{0.437447in}{1.170409in}}%
\pgfpathlineto{\pgfqpoint{0.424178in}{1.172668in}}%
\pgfpathclose%
\pgfpathmoveto{\pgfqpoint{0.476550in}{1.181364in}}%
\pgfpathlineto{\pgfqpoint{0.474915in}{1.194290in}}%
\pgfpathlineto{\pgfqpoint{0.474609in}{1.207217in}}%
\pgfpathlineto{\pgfqpoint{0.474986in}{1.220143in}}%
\pgfpathlineto{\pgfqpoint{0.476782in}{1.233070in}}%
\pgfpathlineto{\pgfqpoint{0.477254in}{1.234460in}}%
\pgfpathlineto{\pgfqpoint{0.490523in}{1.243952in}}%
\pgfpathlineto{\pgfqpoint{0.503792in}{1.244945in}}%
\pgfpathlineto{\pgfqpoint{0.517061in}{1.244082in}}%
\pgfpathlineto{\pgfqpoint{0.528023in}{1.233070in}}%
\pgfpathlineto{\pgfqpoint{0.529333in}{1.220143in}}%
\pgfpathlineto{\pgfqpoint{0.529607in}{1.207217in}}%
\pgfpathlineto{\pgfqpoint{0.529411in}{1.194290in}}%
\pgfpathlineto{\pgfqpoint{0.528284in}{1.181364in}}%
\pgfpathlineto{\pgfqpoint{0.517061in}{1.169929in}}%
\pgfpathlineto{\pgfqpoint{0.503792in}{1.169173in}}%
\pgfpathlineto{\pgfqpoint{0.490523in}{1.170125in}}%
\pgfpathlineto{\pgfqpoint{0.477254in}{1.179231in}}%
\pgfpathclose%
\pgfpathmoveto{\pgfqpoint{0.539878in}{1.181364in}}%
\pgfpathlineto{\pgfqpoint{0.536456in}{1.194290in}}%
\pgfpathlineto{\pgfqpoint{0.535775in}{1.207217in}}%
\pgfpathlineto{\pgfqpoint{0.536389in}{1.220143in}}%
\pgfpathlineto{\pgfqpoint{0.539666in}{1.233070in}}%
\pgfpathlineto{\pgfqpoint{0.543598in}{1.237662in}}%
\pgfpathlineto{\pgfqpoint{0.556867in}{1.241171in}}%
\pgfpathlineto{\pgfqpoint{0.570136in}{1.240001in}}%
\pgfpathlineto{\pgfqpoint{0.580023in}{1.233070in}}%
\pgfpathlineto{\pgfqpoint{0.583405in}{1.224721in}}%
\pgfpathlineto{\pgfqpoint{0.584288in}{1.220143in}}%
\pgfpathlineto{\pgfqpoint{0.584959in}{1.207217in}}%
\pgfpathlineto{\pgfqpoint{0.584227in}{1.194290in}}%
\pgfpathlineto{\pgfqpoint{0.583405in}{1.190124in}}%
\pgfpathlineto{\pgfqpoint{0.579781in}{1.181364in}}%
\pgfpathlineto{\pgfqpoint{0.570136in}{1.174626in}}%
\pgfpathlineto{\pgfqpoint{0.556867in}{1.173437in}}%
\pgfpathlineto{\pgfqpoint{0.543598in}{1.177047in}}%
\pgfpathclose%
\pgfpathmoveto{\pgfqpoint{0.312504in}{1.258923in}}%
\pgfpathlineto{\pgfqpoint{0.304758in}{1.267398in}}%
\pgfpathlineto{\pgfqpoint{0.303281in}{1.271849in}}%
\pgfpathlineto{\pgfqpoint{0.301877in}{1.284776in}}%
\pgfpathlineto{\pgfqpoint{0.302304in}{1.297702in}}%
\pgfpathlineto{\pgfqpoint{0.304758in}{1.308931in}}%
\pgfpathlineto{\pgfqpoint{0.305538in}{1.310629in}}%
\pgfpathlineto{\pgfqpoint{0.318027in}{1.319710in}}%
\pgfpathlineto{\pgfqpoint{0.331295in}{1.320243in}}%
\pgfpathlineto{\pgfqpoint{0.344564in}{1.312771in}}%
\pgfpathlineto{\pgfqpoint{0.345702in}{1.310629in}}%
\pgfpathlineto{\pgfqpoint{0.348401in}{1.297702in}}%
\pgfpathlineto{\pgfqpoint{0.348788in}{1.284776in}}%
\pgfpathlineto{\pgfqpoint{0.347558in}{1.271849in}}%
\pgfpathlineto{\pgfqpoint{0.344564in}{1.263573in}}%
\pgfpathlineto{\pgfqpoint{0.339313in}{1.258923in}}%
\pgfpathlineto{\pgfqpoint{0.331295in}{1.255997in}}%
\pgfpathlineto{\pgfqpoint{0.318027in}{1.256487in}}%
\pgfpathclose%
\pgfpathmoveto{\pgfqpoint{0.355996in}{1.258923in}}%
\pgfpathlineto{\pgfqpoint{0.355032in}{1.271849in}}%
\pgfpathlineto{\pgfqpoint{0.354807in}{1.284776in}}%
\pgfpathlineto{\pgfqpoint{0.354776in}{1.297702in}}%
\pgfpathlineto{\pgfqpoint{0.354938in}{1.310629in}}%
\pgfpathlineto{\pgfqpoint{0.356347in}{1.323555in}}%
\pgfpathlineto{\pgfqpoint{0.357833in}{1.325642in}}%
\pgfpathlineto{\pgfqpoint{0.371102in}{1.327751in}}%
\pgfpathlineto{\pgfqpoint{0.384371in}{1.327652in}}%
\pgfpathlineto{\pgfqpoint{0.397640in}{1.326643in}}%
\pgfpathlineto{\pgfqpoint{0.405639in}{1.323555in}}%
\pgfpathlineto{\pgfqpoint{0.410726in}{1.310629in}}%
\pgfpathlineto{\pgfqpoint{0.410909in}{1.308279in}}%
\pgfpathlineto{\pgfqpoint{0.411441in}{1.297702in}}%
\pgfpathlineto{\pgfqpoint{0.411480in}{1.284776in}}%
\pgfpathlineto{\pgfqpoint{0.410983in}{1.271849in}}%
\pgfpathlineto{\pgfqpoint{0.410909in}{1.271059in}}%
\pgfpathlineto{\pgfqpoint{0.408593in}{1.258923in}}%
\pgfpathlineto{\pgfqpoint{0.397640in}{1.250949in}}%
\pgfpathlineto{\pgfqpoint{0.384371in}{1.249765in}}%
\pgfpathlineto{\pgfqpoint{0.371102in}{1.249807in}}%
\pgfpathlineto{\pgfqpoint{0.357833in}{1.253785in}}%
\pgfpathclose%
\pgfpathmoveto{\pgfqpoint{0.422279in}{1.258923in}}%
\pgfpathlineto{\pgfqpoint{0.418163in}{1.271849in}}%
\pgfpathlineto{\pgfqpoint{0.417322in}{1.284776in}}%
\pgfpathlineto{\pgfqpoint{0.417612in}{1.297702in}}%
\pgfpathlineto{\pgfqpoint{0.419508in}{1.310629in}}%
\pgfpathlineto{\pgfqpoint{0.424178in}{1.318972in}}%
\pgfpathlineto{\pgfqpoint{0.435862in}{1.323555in}}%
\pgfpathlineto{\pgfqpoint{0.437447in}{1.323877in}}%
\pgfpathlineto{\pgfqpoint{0.450716in}{1.323972in}}%
\pgfpathlineto{\pgfqpoint{0.452892in}{1.323555in}}%
\pgfpathlineto{\pgfqpoint{0.463985in}{1.318519in}}%
\pgfpathlineto{\pgfqpoint{0.467525in}{1.310629in}}%
\pgfpathlineto{\pgfqpoint{0.469018in}{1.297702in}}%
\pgfpathlineto{\pgfqpoint{0.469254in}{1.284776in}}%
\pgfpathlineto{\pgfqpoint{0.468635in}{1.271849in}}%
\pgfpathlineto{\pgfqpoint{0.465392in}{1.258923in}}%
\pgfpathlineto{\pgfqpoint{0.463985in}{1.257073in}}%
\pgfpathlineto{\pgfqpoint{0.450716in}{1.252494in}}%
\pgfpathlineto{\pgfqpoint{0.437447in}{1.252619in}}%
\pgfpathlineto{\pgfqpoint{0.424178in}{1.256867in}}%
\pgfpathclose%
\pgfpathmoveto{\pgfqpoint{0.480493in}{1.258923in}}%
\pgfpathlineto{\pgfqpoint{0.477254in}{1.265581in}}%
\pgfpathlineto{\pgfqpoint{0.476001in}{1.271849in}}%
\pgfpathlineto{\pgfqpoint{0.475192in}{1.284776in}}%
\pgfpathlineto{\pgfqpoint{0.475305in}{1.297702in}}%
\pgfpathlineto{\pgfqpoint{0.476618in}{1.310629in}}%
\pgfpathlineto{\pgfqpoint{0.477254in}{1.312998in}}%
\pgfpathlineto{\pgfqpoint{0.489002in}{1.323555in}}%
\pgfpathlineto{\pgfqpoint{0.490523in}{1.323940in}}%
\pgfpathlineto{\pgfqpoint{0.503792in}{1.324468in}}%
\pgfpathlineto{\pgfqpoint{0.509687in}{1.323555in}}%
\pgfpathlineto{\pgfqpoint{0.517061in}{1.321691in}}%
\pgfpathlineto{\pgfqpoint{0.524966in}{1.310629in}}%
\pgfpathlineto{\pgfqpoint{0.526751in}{1.297702in}}%
\pgfpathlineto{\pgfqpoint{0.526925in}{1.284776in}}%
\pgfpathlineto{\pgfqpoint{0.525860in}{1.271849in}}%
\pgfpathlineto{\pgfqpoint{0.520985in}{1.258923in}}%
\pgfpathlineto{\pgfqpoint{0.517061in}{1.255744in}}%
\pgfpathlineto{\pgfqpoint{0.503792in}{1.252960in}}%
\pgfpathlineto{\pgfqpoint{0.490523in}{1.253580in}}%
\pgfpathclose%
\pgfpathmoveto{\pgfqpoint{0.535152in}{1.258923in}}%
\pgfpathlineto{\pgfqpoint{0.533365in}{1.271849in}}%
\pgfpathlineto{\pgfqpoint{0.533020in}{1.284776in}}%
\pgfpathlineto{\pgfqpoint{0.533191in}{1.297702in}}%
\pgfpathlineto{\pgfqpoint{0.534132in}{1.310629in}}%
\pgfpathlineto{\pgfqpoint{0.540158in}{1.323555in}}%
\pgfpathlineto{\pgfqpoint{0.543598in}{1.325185in}}%
\pgfpathlineto{\pgfqpoint{0.556867in}{1.326907in}}%
\pgfpathlineto{\pgfqpoint{0.570136in}{1.326978in}}%
\pgfpathlineto{\pgfqpoint{0.583405in}{1.324733in}}%
\pgfpathlineto{\pgfqpoint{0.584990in}{1.323555in}}%
\pgfpathlineto{\pgfqpoint{0.588651in}{1.310629in}}%
\pgfpathlineto{\pgfqpoint{0.589168in}{1.297702in}}%
\pgfpathlineto{\pgfqpoint{0.589283in}{1.284776in}}%
\pgfpathlineto{\pgfqpoint{0.589159in}{1.271849in}}%
\pgfpathlineto{\pgfqpoint{0.588386in}{1.258923in}}%
\pgfpathlineto{\pgfqpoint{0.583405in}{1.250868in}}%
\pgfpathlineto{\pgfqpoint{0.570136in}{1.249374in}}%
\pgfpathlineto{\pgfqpoint{0.556867in}{1.249510in}}%
\pgfpathlineto{\pgfqpoint{0.543598in}{1.250992in}}%
\pgfpathclose%
\pgfpathmoveto{\pgfqpoint{0.604653in}{1.258923in}}%
\pgfpathlineto{\pgfqpoint{0.596674in}{1.271038in}}%
\pgfpathlineto{\pgfqpoint{0.596468in}{1.271849in}}%
\pgfpathlineto{\pgfqpoint{0.595180in}{1.284776in}}%
\pgfpathlineto{\pgfqpoint{0.595411in}{1.297702in}}%
\pgfpathlineto{\pgfqpoint{0.596674in}{1.306077in}}%
\pgfpathlineto{\pgfqpoint{0.597983in}{1.310629in}}%
\pgfpathlineto{\pgfqpoint{0.609943in}{1.320824in}}%
\pgfpathlineto{\pgfqpoint{0.623212in}{1.321527in}}%
\pgfpathlineto{\pgfqpoint{0.636481in}{1.316267in}}%
\pgfpathlineto{\pgfqpoint{0.640059in}{1.310629in}}%
\pgfpathlineto{\pgfqpoint{0.642651in}{1.297702in}}%
\pgfpathlineto{\pgfqpoint{0.642928in}{1.284776in}}%
\pgfpathlineto{\pgfqpoint{0.641453in}{1.271849in}}%
\pgfpathlineto{\pgfqpoint{0.636481in}{1.261084in}}%
\pgfpathlineto{\pgfqpoint{0.633132in}{1.258923in}}%
\pgfpathlineto{\pgfqpoint{0.623212in}{1.255702in}}%
\pgfpathlineto{\pgfqpoint{0.609943in}{1.256386in}}%
\pgfpathclose%
\pgfpathmoveto{\pgfqpoint{0.378341in}{1.336482in}}%
\pgfpathlineto{\pgfqpoint{0.371102in}{1.338133in}}%
\pgfpathlineto{\pgfqpoint{0.362393in}{1.349408in}}%
\pgfpathlineto{\pgfqpoint{0.360258in}{1.362335in}}%
\pgfpathlineto{\pgfqpoint{0.360257in}{1.375261in}}%
\pgfpathlineto{\pgfqpoint{0.362300in}{1.388188in}}%
\pgfpathlineto{\pgfqpoint{0.371102in}{1.400423in}}%
\pgfpathlineto{\pgfqpoint{0.373649in}{1.401114in}}%
\pgfpathlineto{\pgfqpoint{0.384371in}{1.403007in}}%
\pgfpathlineto{\pgfqpoint{0.397098in}{1.401114in}}%
\pgfpathlineto{\pgfqpoint{0.397640in}{1.400983in}}%
\pgfpathlineto{\pgfqpoint{0.406591in}{1.388188in}}%
\pgfpathlineto{\pgfqpoint{0.408337in}{1.375261in}}%
\pgfpathlineto{\pgfqpoint{0.408356in}{1.362335in}}%
\pgfpathlineto{\pgfqpoint{0.406585in}{1.349408in}}%
\pgfpathlineto{\pgfqpoint{0.397640in}{1.337527in}}%
\pgfpathlineto{\pgfqpoint{0.392093in}{1.336482in}}%
\pgfpathlineto{\pgfqpoint{0.384371in}{1.335563in}}%
\pgfpathclose%
\pgfpathmoveto{\pgfqpoint{0.419412in}{1.336482in}}%
\pgfpathlineto{\pgfqpoint{0.415550in}{1.349408in}}%
\pgfpathlineto{\pgfqpoint{0.414867in}{1.362335in}}%
\pgfpathlineto{\pgfqpoint{0.414745in}{1.375261in}}%
\pgfpathlineto{\pgfqpoint{0.415039in}{1.388188in}}%
\pgfpathlineto{\pgfqpoint{0.416846in}{1.401114in}}%
\pgfpathlineto{\pgfqpoint{0.424178in}{1.407018in}}%
\pgfpathlineto{\pgfqpoint{0.437447in}{1.408019in}}%
\pgfpathlineto{\pgfqpoint{0.450716in}{1.407508in}}%
\pgfpathlineto{\pgfqpoint{0.463985in}{1.403497in}}%
\pgfpathlineto{\pgfqpoint{0.465959in}{1.401114in}}%
\pgfpathlineto{\pgfqpoint{0.469086in}{1.388188in}}%
\pgfpathlineto{\pgfqpoint{0.469673in}{1.375261in}}%
\pgfpathlineto{\pgfqpoint{0.469540in}{1.362335in}}%
\pgfpathlineto{\pgfqpoint{0.468529in}{1.349408in}}%
\pgfpathlineto{\pgfqpoint{0.463985in}{1.337332in}}%
\pgfpathlineto{\pgfqpoint{0.462925in}{1.336482in}}%
\pgfpathlineto{\pgfqpoint{0.450716in}{1.332512in}}%
\pgfpathlineto{\pgfqpoint{0.437447in}{1.331972in}}%
\pgfpathlineto{\pgfqpoint{0.424178in}{1.333489in}}%
\pgfpathclose%
\pgfpathmoveto{\pgfqpoint{0.483316in}{1.336482in}}%
\pgfpathlineto{\pgfqpoint{0.477254in}{1.345828in}}%
\pgfpathlineto{\pgfqpoint{0.476456in}{1.349408in}}%
\pgfpathlineto{\pgfqpoint{0.475433in}{1.362335in}}%
\pgfpathlineto{\pgfqpoint{0.475471in}{1.375261in}}%
\pgfpathlineto{\pgfqpoint{0.476560in}{1.388188in}}%
\pgfpathlineto{\pgfqpoint{0.477254in}{1.391354in}}%
\pgfpathlineto{\pgfqpoint{0.482758in}{1.401114in}}%
\pgfpathlineto{\pgfqpoint{0.490523in}{1.404904in}}%
\pgfpathlineto{\pgfqpoint{0.503792in}{1.406095in}}%
\pgfpathlineto{\pgfqpoint{0.517061in}{1.404707in}}%
\pgfpathlineto{\pgfqpoint{0.523076in}{1.401114in}}%
\pgfpathlineto{\pgfqpoint{0.527746in}{1.388188in}}%
\pgfpathlineto{\pgfqpoint{0.528666in}{1.375261in}}%
\pgfpathlineto{\pgfqpoint{0.528726in}{1.362335in}}%
\pgfpathlineto{\pgfqpoint{0.527947in}{1.349408in}}%
\pgfpathlineto{\pgfqpoint{0.523025in}{1.336482in}}%
\pgfpathlineto{\pgfqpoint{0.517061in}{1.333521in}}%
\pgfpathlineto{\pgfqpoint{0.503792in}{1.332426in}}%
\pgfpathlineto{\pgfqpoint{0.490523in}{1.333497in}}%
\pgfpathclose%
\pgfpathmoveto{\pgfqpoint{0.547977in}{1.336482in}}%
\pgfpathlineto{\pgfqpoint{0.543598in}{1.337963in}}%
\pgfpathlineto{\pgfqpoint{0.536752in}{1.349408in}}%
\pgfpathlineto{\pgfqpoint{0.535180in}{1.362335in}}%
\pgfpathlineto{\pgfqpoint{0.535015in}{1.375261in}}%
\pgfpathlineto{\pgfqpoint{0.536034in}{1.388188in}}%
\pgfpathlineto{\pgfqpoint{0.541706in}{1.401114in}}%
\pgfpathlineto{\pgfqpoint{0.543598in}{1.402456in}}%
\pgfpathlineto{\pgfqpoint{0.556867in}{1.405060in}}%
\pgfpathlineto{\pgfqpoint{0.570136in}{1.404094in}}%
\pgfpathlineto{\pgfqpoint{0.576891in}{1.401114in}}%
\pgfpathlineto{\pgfqpoint{0.583405in}{1.392211in}}%
\pgfpathlineto{\pgfqpoint{0.584470in}{1.388188in}}%
\pgfpathlineto{\pgfqpoint{0.585640in}{1.375261in}}%
\pgfpathlineto{\pgfqpoint{0.585475in}{1.362335in}}%
\pgfpathlineto{\pgfqpoint{0.583754in}{1.349408in}}%
\pgfpathlineto{\pgfqpoint{0.583405in}{1.348293in}}%
\pgfpathlineto{\pgfqpoint{0.571852in}{1.336482in}}%
\pgfpathlineto{\pgfqpoint{0.570136in}{1.335868in}}%
\pgfpathlineto{\pgfqpoint{0.556867in}{1.334845in}}%
\pgfpathclose%
\pgfpathmoveto{\pgfqpoint{0.593672in}{1.336482in}}%
\pgfpathlineto{\pgfqpoint{0.591689in}{1.349408in}}%
\pgfpathlineto{\pgfqpoint{0.591367in}{1.362335in}}%
\pgfpathlineto{\pgfqpoint{0.591437in}{1.375261in}}%
\pgfpathlineto{\pgfqpoint{0.591949in}{1.388188in}}%
\pgfpathlineto{\pgfqpoint{0.594310in}{1.401114in}}%
\pgfpathlineto{\pgfqpoint{0.596674in}{1.404420in}}%
\pgfpathlineto{\pgfqpoint{0.609943in}{1.408187in}}%
\pgfpathlineto{\pgfqpoint{0.623212in}{1.408677in}}%
\pgfpathlineto{\pgfqpoint{0.636481in}{1.408216in}}%
\pgfpathlineto{\pgfqpoint{0.646881in}{1.401114in}}%
\pgfpathlineto{\pgfqpoint{0.648195in}{1.388188in}}%
\pgfpathlineto{\pgfqpoint{0.648487in}{1.375261in}}%
\pgfpathlineto{\pgfqpoint{0.648578in}{1.362335in}}%
\pgfpathlineto{\pgfqpoint{0.648552in}{1.349408in}}%
\pgfpathlineto{\pgfqpoint{0.648150in}{1.336482in}}%
\pgfpathlineto{\pgfqpoint{0.636481in}{1.329821in}}%
\pgfpathlineto{\pgfqpoint{0.623212in}{1.329797in}}%
\pgfpathlineto{\pgfqpoint{0.609943in}{1.330197in}}%
\pgfpathlineto{\pgfqpoint{0.596674in}{1.332818in}}%
\pgfpathclose%
\pgfpathmoveto{\pgfqpoint{0.657467in}{1.349408in}}%
\pgfpathlineto{\pgfqpoint{0.655118in}{1.362335in}}%
\pgfpathlineto{\pgfqpoint{0.654915in}{1.375261in}}%
\pgfpathlineto{\pgfqpoint{0.656565in}{1.388188in}}%
\pgfpathlineto{\pgfqpoint{0.663019in}{1.399175in}}%
\pgfpathlineto{\pgfqpoint{0.668400in}{1.401114in}}%
\pgfpathlineto{\pgfqpoint{0.676288in}{1.402707in}}%
\pgfpathlineto{\pgfqpoint{0.688575in}{1.401114in}}%
\pgfpathlineto{\pgfqpoint{0.689557in}{1.400930in}}%
\pgfpathlineto{\pgfqpoint{0.699714in}{1.388188in}}%
\pgfpathlineto{\pgfqpoint{0.701751in}{1.375261in}}%
\pgfpathlineto{\pgfqpoint{0.701513in}{1.362335in}}%
\pgfpathlineto{\pgfqpoint{0.698669in}{1.349408in}}%
\pgfpathlineto{\pgfqpoint{0.689557in}{1.339351in}}%
\pgfpathlineto{\pgfqpoint{0.676288in}{1.337240in}}%
\pgfpathlineto{\pgfqpoint{0.663019in}{1.341165in}}%
\pgfpathclose%
\pgfpathmoveto{\pgfqpoint{0.499368in}{1.414041in}}%
\pgfpathlineto{\pgfqpoint{0.490523in}{1.414306in}}%
\pgfpathlineto{\pgfqpoint{0.477254in}{1.421911in}}%
\pgfpathlineto{\pgfqpoint{0.475614in}{1.426967in}}%
\pgfpathlineto{\pgfqpoint{0.474281in}{1.439894in}}%
\pgfpathlineto{\pgfqpoint{0.473970in}{1.452820in}}%
\pgfpathlineto{\pgfqpoint{0.474175in}{1.465747in}}%
\pgfpathlineto{\pgfqpoint{0.475522in}{1.478673in}}%
\pgfpathlineto{\pgfqpoint{0.477254in}{1.483065in}}%
\pgfpathlineto{\pgfqpoint{0.490523in}{1.488451in}}%
\pgfpathlineto{\pgfqpoint{0.503792in}{1.488620in}}%
\pgfpathlineto{\pgfqpoint{0.517061in}{1.486807in}}%
\pgfpathlineto{\pgfqpoint{0.525551in}{1.478673in}}%
\pgfpathlineto{\pgfqpoint{0.527800in}{1.465747in}}%
\pgfpathlineto{\pgfqpoint{0.528204in}{1.452820in}}%
\pgfpathlineto{\pgfqpoint{0.527786in}{1.439894in}}%
\pgfpathlineto{\pgfqpoint{0.525872in}{1.426967in}}%
\pgfpathlineto{\pgfqpoint{0.517061in}{1.416229in}}%
\pgfpathlineto{\pgfqpoint{0.504698in}{1.414041in}}%
\pgfpathlineto{\pgfqpoint{0.503792in}{1.413938in}}%
\pgfpathclose%
\pgfpathmoveto{\pgfqpoint{0.550433in}{1.414041in}}%
\pgfpathlineto{\pgfqpoint{0.543598in}{1.415601in}}%
\pgfpathlineto{\pgfqpoint{0.535741in}{1.426967in}}%
\pgfpathlineto{\pgfqpoint{0.534362in}{1.439894in}}%
\pgfpathlineto{\pgfqpoint{0.534219in}{1.452820in}}%
\pgfpathlineto{\pgfqpoint{0.534927in}{1.465747in}}%
\pgfpathlineto{\pgfqpoint{0.537948in}{1.478673in}}%
\pgfpathlineto{\pgfqpoint{0.543598in}{1.484582in}}%
\pgfpathlineto{\pgfqpoint{0.556867in}{1.487409in}}%
\pgfpathlineto{\pgfqpoint{0.570136in}{1.487261in}}%
\pgfpathlineto{\pgfqpoint{0.583405in}{1.481994in}}%
\pgfpathlineto{\pgfqpoint{0.585275in}{1.478673in}}%
\pgfpathlineto{\pgfqpoint{0.587562in}{1.465747in}}%
\pgfpathlineto{\pgfqpoint{0.588097in}{1.452820in}}%
\pgfpathlineto{\pgfqpoint{0.588038in}{1.439894in}}%
\pgfpathlineto{\pgfqpoint{0.587148in}{1.426967in}}%
\pgfpathlineto{\pgfqpoint{0.583405in}{1.417324in}}%
\pgfpathlineto{\pgfqpoint{0.575548in}{1.414041in}}%
\pgfpathlineto{\pgfqpoint{0.570136in}{1.413223in}}%
\pgfpathlineto{\pgfqpoint{0.556867in}{1.413193in}}%
\pgfpathclose%
\pgfpathmoveto{\pgfqpoint{0.421342in}{1.426967in}}%
\pgfpathlineto{\pgfqpoint{0.418816in}{1.439894in}}%
\pgfpathlineto{\pgfqpoint{0.418475in}{1.452820in}}%
\pgfpathlineto{\pgfqpoint{0.419569in}{1.465747in}}%
\pgfpathlineto{\pgfqpoint{0.424178in}{1.478318in}}%
\pgfpathlineto{\pgfqpoint{0.424540in}{1.478673in}}%
\pgfpathlineto{\pgfqpoint{0.437447in}{1.484272in}}%
\pgfpathlineto{\pgfqpoint{0.450716in}{1.484268in}}%
\pgfpathlineto{\pgfqpoint{0.462397in}{1.478673in}}%
\pgfpathlineto{\pgfqpoint{0.463985in}{1.476573in}}%
\pgfpathlineto{\pgfqpoint{0.467141in}{1.465747in}}%
\pgfpathlineto{\pgfqpoint{0.468075in}{1.452820in}}%
\pgfpathlineto{\pgfqpoint{0.467819in}{1.439894in}}%
\pgfpathlineto{\pgfqpoint{0.465736in}{1.426967in}}%
\pgfpathlineto{\pgfqpoint{0.463985in}{1.423304in}}%
\pgfpathlineto{\pgfqpoint{0.450716in}{1.416447in}}%
\pgfpathlineto{\pgfqpoint{0.437447in}{1.416497in}}%
\pgfpathlineto{\pgfqpoint{0.424178in}{1.422240in}}%
\pgfpathclose%
\pgfpathmoveto{\pgfqpoint{0.597020in}{1.426967in}}%
\pgfpathlineto{\pgfqpoint{0.596674in}{1.427895in}}%
\pgfpathlineto{\pgfqpoint{0.594453in}{1.439894in}}%
\pgfpathlineto{\pgfqpoint{0.593952in}{1.452820in}}%
\pgfpathlineto{\pgfqpoint{0.594543in}{1.465747in}}%
\pgfpathlineto{\pgfqpoint{0.596674in}{1.475774in}}%
\pgfpathlineto{\pgfqpoint{0.598136in}{1.478673in}}%
\pgfpathlineto{\pgfqpoint{0.609943in}{1.485402in}}%
\pgfpathlineto{\pgfqpoint{0.623212in}{1.485814in}}%
\pgfpathlineto{\pgfqpoint{0.636481in}{1.482023in}}%
\pgfpathlineto{\pgfqpoint{0.639589in}{1.478673in}}%
\pgfpathlineto{\pgfqpoint{0.643445in}{1.465747in}}%
\pgfpathlineto{\pgfqpoint{0.644187in}{1.452820in}}%
\pgfpathlineto{\pgfqpoint{0.643606in}{1.439894in}}%
\pgfpathlineto{\pgfqpoint{0.640709in}{1.426967in}}%
\pgfpathlineto{\pgfqpoint{0.636481in}{1.421184in}}%
\pgfpathlineto{\pgfqpoint{0.623212in}{1.416815in}}%
\pgfpathlineto{\pgfqpoint{0.609943in}{1.417356in}}%
\pgfpathclose%
\pgfpathmoveto{\pgfqpoint{0.718262in}{1.426967in}}%
\pgfpathlineto{\pgfqpoint{0.716095in}{1.431466in}}%
\pgfpathlineto{\pgfqpoint{0.714215in}{1.439894in}}%
\pgfpathlineto{\pgfqpoint{0.713560in}{1.452820in}}%
\pgfpathlineto{\pgfqpoint{0.714461in}{1.465747in}}%
\pgfpathlineto{\pgfqpoint{0.716095in}{1.472004in}}%
\pgfpathlineto{\pgfqpoint{0.720514in}{1.478673in}}%
\pgfpathlineto{\pgfqpoint{0.729364in}{1.483194in}}%
\pgfpathlineto{\pgfqpoint{0.742633in}{1.483453in}}%
\pgfpathlineto{\pgfqpoint{0.753759in}{1.478673in}}%
\pgfpathlineto{\pgfqpoint{0.755902in}{1.476532in}}%
\pgfpathlineto{\pgfqpoint{0.759687in}{1.465747in}}%
\pgfpathlineto{\pgfqpoint{0.760689in}{1.452820in}}%
\pgfpathlineto{\pgfqpoint{0.759982in}{1.439894in}}%
\pgfpathlineto{\pgfqpoint{0.756327in}{1.426967in}}%
\pgfpathlineto{\pgfqpoint{0.755902in}{1.426308in}}%
\pgfpathlineto{\pgfqpoint{0.742633in}{1.419248in}}%
\pgfpathlineto{\pgfqpoint{0.729364in}{1.419567in}}%
\pgfpathclose%
\pgfusepath{fill}%
\end{pgfscope}%
\begin{pgfscope}%
\pgfpathrectangle{\pgfqpoint{0.211875in}{0.211875in}}{\pgfqpoint{1.313625in}{1.279725in}}%
\pgfusepath{clip}%
\pgfsetbuttcap%
\pgfsetroundjoin%
\definecolor{currentfill}{rgb}{0.901975,0.231521,0.249182}%
\pgfsetfillcolor{currentfill}%
\pgfsetlinewidth{0.000000pt}%
\definecolor{currentstroke}{rgb}{0.000000,0.000000,0.000000}%
\pgfsetstrokecolor{currentstroke}%
\pgfsetdash{}{0pt}%
\pgfpathmoveto{\pgfqpoint{1.498962in}{0.276501in}}%
\pgfpathlineto{\pgfqpoint{1.499191in}{0.276508in}}%
\pgfpathlineto{\pgfqpoint{1.512231in}{0.276944in}}%
\pgfpathlineto{\pgfqpoint{1.525500in}{0.281986in}}%
\pgfpathlineto{\pgfqpoint{1.525500in}{0.289434in}}%
\pgfpathlineto{\pgfqpoint{1.525500in}{0.302361in}}%
\pgfpathlineto{\pgfqpoint{1.525500in}{0.311506in}}%
\pgfpathlineto{\pgfqpoint{1.524389in}{0.302361in}}%
\pgfpathlineto{\pgfqpoint{1.517401in}{0.289434in}}%
\pgfpathlineto{\pgfqpoint{1.512231in}{0.285939in}}%
\pgfpathlineto{\pgfqpoint{1.498962in}{0.284936in}}%
\pgfpathlineto{\pgfqpoint{1.489737in}{0.289434in}}%
\pgfpathlineto{\pgfqpoint{1.485693in}{0.294111in}}%
\pgfpathlineto{\pgfqpoint{1.482965in}{0.302361in}}%
\pgfpathlineto{\pgfqpoint{1.481854in}{0.315287in}}%
\pgfpathlineto{\pgfqpoint{1.482807in}{0.328214in}}%
\pgfpathlineto{\pgfqpoint{1.485693in}{0.336667in}}%
\pgfpathlineto{\pgfqpoint{1.490231in}{0.341140in}}%
\pgfpathlineto{\pgfqpoint{1.498962in}{0.344747in}}%
\pgfpathlineto{\pgfqpoint{1.512231in}{0.343855in}}%
\pgfpathlineto{\pgfqpoint{1.516899in}{0.341140in}}%
\pgfpathlineto{\pgfqpoint{1.524555in}{0.328214in}}%
\pgfpathlineto{\pgfqpoint{1.525500in}{0.319703in}}%
\pgfpathlineto{\pgfqpoint{1.525500in}{0.328214in}}%
\pgfpathlineto{\pgfqpoint{1.525500in}{0.341140in}}%
\pgfpathlineto{\pgfqpoint{1.525500in}{0.351094in}}%
\pgfpathlineto{\pgfqpoint{1.512231in}{0.353159in}}%
\pgfpathlineto{\pgfqpoint{1.498962in}{0.353432in}}%
\pgfpathlineto{\pgfqpoint{1.485693in}{0.353069in}}%
\pgfpathlineto{\pgfqpoint{1.475996in}{0.341140in}}%
\pgfpathlineto{\pgfqpoint{1.475732in}{0.328214in}}%
\pgfpathlineto{\pgfqpoint{1.475860in}{0.315287in}}%
\pgfpathlineto{\pgfqpoint{1.476346in}{0.302361in}}%
\pgfpathlineto{\pgfqpoint{1.477874in}{0.289434in}}%
\pgfpathlineto{\pgfqpoint{1.485693in}{0.278653in}}%
\pgfpathlineto{\pgfqpoint{1.498888in}{0.276508in}}%
\pgfpathclose%
\pgfusepath{fill}%
\end{pgfscope}%
\begin{pgfscope}%
\pgfpathrectangle{\pgfqpoint{0.211875in}{0.211875in}}{\pgfqpoint{1.313625in}{1.279725in}}%
\pgfusepath{clip}%
\pgfsetbuttcap%
\pgfsetroundjoin%
\definecolor{currentfill}{rgb}{0.901975,0.231521,0.249182}%
\pgfsetfillcolor{currentfill}%
\pgfsetlinewidth{0.000000pt}%
\definecolor{currentstroke}{rgb}{0.000000,0.000000,0.000000}%
\pgfsetstrokecolor{currentstroke}%
\pgfsetdash{}{0pt}%
\pgfpathmoveto{\pgfqpoint{0.318027in}{0.288360in}}%
\pgfpathlineto{\pgfqpoint{0.331295in}{0.287828in}}%
\pgfpathlineto{\pgfqpoint{0.334279in}{0.289434in}}%
\pgfpathlineto{\pgfqpoint{0.343873in}{0.302361in}}%
\pgfpathlineto{\pgfqpoint{0.344564in}{0.306578in}}%
\pgfpathlineto{\pgfqpoint{0.345395in}{0.315287in}}%
\pgfpathlineto{\pgfqpoint{0.344564in}{0.324396in}}%
\pgfpathlineto{\pgfqpoint{0.343924in}{0.328214in}}%
\pgfpathlineto{\pgfqpoint{0.332794in}{0.341140in}}%
\pgfpathlineto{\pgfqpoint{0.331295in}{0.341820in}}%
\pgfpathlineto{\pgfqpoint{0.318027in}{0.341282in}}%
\pgfpathlineto{\pgfqpoint{0.317761in}{0.341140in}}%
\pgfpathlineto{\pgfqpoint{0.307257in}{0.328214in}}%
\pgfpathlineto{\pgfqpoint{0.305271in}{0.315287in}}%
\pgfpathlineto{\pgfqpoint{0.307212in}{0.302361in}}%
\pgfpathlineto{\pgfqpoint{0.316293in}{0.289434in}}%
\pgfpathclose%
\pgfpathmoveto{\pgfqpoint{0.316546in}{0.302361in}}%
\pgfpathlineto{\pgfqpoint{0.313763in}{0.315287in}}%
\pgfpathlineto{\pgfqpoint{0.317204in}{0.328214in}}%
\pgfpathlineto{\pgfqpoint{0.318027in}{0.329216in}}%
\pgfpathlineto{\pgfqpoint{0.331295in}{0.330013in}}%
\pgfpathlineto{\pgfqpoint{0.332937in}{0.328214in}}%
\pgfpathlineto{\pgfqpoint{0.336589in}{0.315287in}}%
\pgfpathlineto{\pgfqpoint{0.333617in}{0.302361in}}%
\pgfpathlineto{\pgfqpoint{0.331295in}{0.299549in}}%
\pgfpathlineto{\pgfqpoint{0.318027in}{0.300380in}}%
\pgfpathclose%
\pgfusepath{fill}%
\end{pgfscope}%
\begin{pgfscope}%
\pgfpathrectangle{\pgfqpoint{0.211875in}{0.211875in}}{\pgfqpoint{1.313625in}{1.279725in}}%
\pgfusepath{clip}%
\pgfsetbuttcap%
\pgfsetroundjoin%
\definecolor{currentfill}{rgb}{0.901975,0.231521,0.249182}%
\pgfsetfillcolor{currentfill}%
\pgfsetlinewidth{0.000000pt}%
\definecolor{currentstroke}{rgb}{0.000000,0.000000,0.000000}%
\pgfsetstrokecolor{currentstroke}%
\pgfsetdash{}{0pt}%
\pgfpathmoveto{\pgfqpoint{0.437447in}{0.284265in}}%
\pgfpathlineto{\pgfqpoint{0.450716in}{0.284479in}}%
\pgfpathlineto{\pgfqpoint{0.458772in}{0.289434in}}%
\pgfpathlineto{\pgfqpoint{0.463985in}{0.298759in}}%
\pgfpathlineto{\pgfqpoint{0.464888in}{0.302361in}}%
\pgfpathlineto{\pgfqpoint{0.465887in}{0.315287in}}%
\pgfpathlineto{\pgfqpoint{0.465110in}{0.328214in}}%
\pgfpathlineto{\pgfqpoint{0.463985in}{0.332802in}}%
\pgfpathlineto{\pgfqpoint{0.458788in}{0.341140in}}%
\pgfpathlineto{\pgfqpoint{0.450716in}{0.345348in}}%
\pgfpathlineto{\pgfqpoint{0.437447in}{0.345449in}}%
\pgfpathlineto{\pgfqpoint{0.428264in}{0.341140in}}%
\pgfpathlineto{\pgfqpoint{0.424178in}{0.336461in}}%
\pgfpathlineto{\pgfqpoint{0.421543in}{0.328214in}}%
\pgfpathlineto{\pgfqpoint{0.420627in}{0.315287in}}%
\pgfpathlineto{\pgfqpoint{0.421697in}{0.302361in}}%
\pgfpathlineto{\pgfqpoint{0.424178in}{0.294397in}}%
\pgfpathlineto{\pgfqpoint{0.427951in}{0.289434in}}%
\pgfpathclose%
\pgfpathmoveto{\pgfqpoint{0.430499in}{0.302361in}}%
\pgfpathlineto{\pgfqpoint{0.427822in}{0.315287in}}%
\pgfpathlineto{\pgfqpoint{0.430933in}{0.328214in}}%
\pgfpathlineto{\pgfqpoint{0.437447in}{0.335175in}}%
\pgfpathlineto{\pgfqpoint{0.450716in}{0.334358in}}%
\pgfpathlineto{\pgfqpoint{0.455629in}{0.328214in}}%
\pgfpathlineto{\pgfqpoint{0.458288in}{0.315287in}}%
\pgfpathlineto{\pgfqpoint{0.455967in}{0.302361in}}%
\pgfpathlineto{\pgfqpoint{0.450716in}{0.295134in}}%
\pgfpathlineto{\pgfqpoint{0.437447in}{0.294183in}}%
\pgfpathclose%
\pgfusepath{fill}%
\end{pgfscope}%
\begin{pgfscope}%
\pgfpathrectangle{\pgfqpoint{0.211875in}{0.211875in}}{\pgfqpoint{1.313625in}{1.279725in}}%
\pgfusepath{clip}%
\pgfsetbuttcap%
\pgfsetroundjoin%
\definecolor{currentfill}{rgb}{0.901975,0.231521,0.249182}%
\pgfsetfillcolor{currentfill}%
\pgfsetlinewidth{0.000000pt}%
\definecolor{currentstroke}{rgb}{0.000000,0.000000,0.000000}%
\pgfsetstrokecolor{currentstroke}%
\pgfsetdash{}{0pt}%
\pgfpathmoveto{\pgfqpoint{0.543598in}{0.286028in}}%
\pgfpathlineto{\pgfqpoint{0.556867in}{0.280980in}}%
\pgfpathlineto{\pgfqpoint{0.570136in}{0.281586in}}%
\pgfpathlineto{\pgfqpoint{0.581394in}{0.289434in}}%
\pgfpathlineto{\pgfqpoint{0.583405in}{0.294081in}}%
\pgfpathlineto{\pgfqpoint{0.585144in}{0.302361in}}%
\pgfpathlineto{\pgfqpoint{0.585922in}{0.315287in}}%
\pgfpathlineto{\pgfqpoint{0.585521in}{0.328214in}}%
\pgfpathlineto{\pgfqpoint{0.583405in}{0.339014in}}%
\pgfpathlineto{\pgfqpoint{0.582429in}{0.341140in}}%
\pgfpathlineto{\pgfqpoint{0.570136in}{0.348441in}}%
\pgfpathlineto{\pgfqpoint{0.556867in}{0.348785in}}%
\pgfpathlineto{\pgfqpoint{0.543598in}{0.344794in}}%
\pgfpathlineto{\pgfqpoint{0.540459in}{0.341140in}}%
\pgfpathlineto{\pgfqpoint{0.536978in}{0.328214in}}%
\pgfpathlineto{\pgfqpoint{0.536428in}{0.315287in}}%
\pgfpathlineto{\pgfqpoint{0.537300in}{0.302361in}}%
\pgfpathlineto{\pgfqpoint{0.540989in}{0.289434in}}%
\pgfpathclose%
\pgfpathmoveto{\pgfqpoint{0.556573in}{0.289434in}}%
\pgfpathlineto{\pgfqpoint{0.544037in}{0.302361in}}%
\pgfpathlineto{\pgfqpoint{0.543598in}{0.304349in}}%
\pgfpathlineto{\pgfqpoint{0.542371in}{0.315287in}}%
\pgfpathlineto{\pgfqpoint{0.543598in}{0.325621in}}%
\pgfpathlineto{\pgfqpoint{0.544244in}{0.328214in}}%
\pgfpathlineto{\pgfqpoint{0.556867in}{0.339917in}}%
\pgfpathlineto{\pgfqpoint{0.570136in}{0.338000in}}%
\pgfpathlineto{\pgfqpoint{0.577039in}{0.328214in}}%
\pgfpathlineto{\pgfqpoint{0.578986in}{0.315287in}}%
\pgfpathlineto{\pgfqpoint{0.577137in}{0.302361in}}%
\pgfpathlineto{\pgfqpoint{0.570136in}{0.291490in}}%
\pgfpathlineto{\pgfqpoint{0.558163in}{0.289434in}}%
\pgfpathlineto{\pgfqpoint{0.556867in}{0.289295in}}%
\pgfpathclose%
\pgfusepath{fill}%
\end{pgfscope}%
\begin{pgfscope}%
\pgfpathrectangle{\pgfqpoint{0.211875in}{0.211875in}}{\pgfqpoint{1.313625in}{1.279725in}}%
\pgfusepath{clip}%
\pgfsetbuttcap%
\pgfsetroundjoin%
\definecolor{currentfill}{rgb}{0.901975,0.231521,0.249182}%
\pgfsetfillcolor{currentfill}%
\pgfsetlinewidth{0.000000pt}%
\definecolor{currentstroke}{rgb}{0.000000,0.000000,0.000000}%
\pgfsetstrokecolor{currentstroke}%
\pgfsetdash{}{0pt}%
\pgfpathmoveto{\pgfqpoint{0.663019in}{0.281230in}}%
\pgfpathlineto{\pgfqpoint{0.676288in}{0.278402in}}%
\pgfpathlineto{\pgfqpoint{0.689557in}{0.279148in}}%
\pgfpathlineto{\pgfqpoint{0.702643in}{0.289434in}}%
\pgfpathlineto{\pgfqpoint{0.702826in}{0.289997in}}%
\pgfpathlineto{\pgfqpoint{0.704925in}{0.302361in}}%
\pgfpathlineto{\pgfqpoint{0.705531in}{0.315287in}}%
\pgfpathlineto{\pgfqpoint{0.705429in}{0.328214in}}%
\pgfpathlineto{\pgfqpoint{0.704058in}{0.341140in}}%
\pgfpathlineto{\pgfqpoint{0.702826in}{0.344561in}}%
\pgfpathlineto{\pgfqpoint{0.689557in}{0.351111in}}%
\pgfpathlineto{\pgfqpoint{0.676288in}{0.351404in}}%
\pgfpathlineto{\pgfqpoint{0.663019in}{0.349584in}}%
\pgfpathlineto{\pgfqpoint{0.654713in}{0.341140in}}%
\pgfpathlineto{\pgfqpoint{0.652698in}{0.328214in}}%
\pgfpathlineto{\pgfqpoint{0.652463in}{0.315287in}}%
\pgfpathlineto{\pgfqpoint{0.653171in}{0.302361in}}%
\pgfpathlineto{\pgfqpoint{0.655900in}{0.289434in}}%
\pgfpathclose%
\pgfpathmoveto{\pgfqpoint{0.669287in}{0.289434in}}%
\pgfpathlineto{\pgfqpoint{0.663019in}{0.294638in}}%
\pgfpathlineto{\pgfqpoint{0.659933in}{0.302361in}}%
\pgfpathlineto{\pgfqpoint{0.658601in}{0.315287in}}%
\pgfpathlineto{\pgfqpoint{0.659914in}{0.328214in}}%
\pgfpathlineto{\pgfqpoint{0.663019in}{0.335551in}}%
\pgfpathlineto{\pgfqpoint{0.671011in}{0.341140in}}%
\pgfpathlineto{\pgfqpoint{0.676288in}{0.342966in}}%
\pgfpathlineto{\pgfqpoint{0.688673in}{0.341140in}}%
\pgfpathlineto{\pgfqpoint{0.689557in}{0.340929in}}%
\pgfpathlineto{\pgfqpoint{0.697497in}{0.328214in}}%
\pgfpathlineto{\pgfqpoint{0.698936in}{0.315287in}}%
\pgfpathlineto{\pgfqpoint{0.697428in}{0.302361in}}%
\pgfpathlineto{\pgfqpoint{0.690266in}{0.289434in}}%
\pgfpathlineto{\pgfqpoint{0.689557in}{0.288876in}}%
\pgfpathlineto{\pgfqpoint{0.676288in}{0.286614in}}%
\pgfpathclose%
\pgfusepath{fill}%
\end{pgfscope}%
\begin{pgfscope}%
\pgfpathrectangle{\pgfqpoint{0.211875in}{0.211875in}}{\pgfqpoint{1.313625in}{1.279725in}}%
\pgfusepath{clip}%
\pgfsetbuttcap%
\pgfsetroundjoin%
\definecolor{currentfill}{rgb}{0.901975,0.231521,0.249182}%
\pgfsetfillcolor{currentfill}%
\pgfsetlinewidth{0.000000pt}%
\definecolor{currentstroke}{rgb}{0.000000,0.000000,0.000000}%
\pgfsetstrokecolor{currentstroke}%
\pgfsetdash{}{0pt}%
\pgfpathmoveto{\pgfqpoint{0.221110in}{0.302361in}}%
\pgfpathlineto{\pgfqpoint{0.223868in}{0.315287in}}%
\pgfpathlineto{\pgfqpoint{0.220769in}{0.328214in}}%
\pgfpathlineto{\pgfqpoint{0.211875in}{0.336673in}}%
\pgfpathlineto{\pgfqpoint{0.211875in}{0.328214in}}%
\pgfpathlineto{\pgfqpoint{0.211875in}{0.319914in}}%
\pgfpathlineto{\pgfqpoint{0.213488in}{0.315287in}}%
\pgfpathlineto{\pgfqpoint{0.211875in}{0.309590in}}%
\pgfpathlineto{\pgfqpoint{0.211875in}{0.302361in}}%
\pgfpathlineto{\pgfqpoint{0.211875in}{0.292626in}}%
\pgfpathclose%
\pgfusepath{fill}%
\end{pgfscope}%
\begin{pgfscope}%
\pgfpathrectangle{\pgfqpoint{0.211875in}{0.211875in}}{\pgfqpoint{1.313625in}{1.279725in}}%
\pgfusepath{clip}%
\pgfsetbuttcap%
\pgfsetroundjoin%
\definecolor{currentfill}{rgb}{0.901975,0.231521,0.249182}%
\pgfsetfillcolor{currentfill}%
\pgfsetlinewidth{0.000000pt}%
\definecolor{currentstroke}{rgb}{0.000000,0.000000,0.000000}%
\pgfsetstrokecolor{currentstroke}%
\pgfsetdash{}{0pt}%
\pgfpathmoveto{\pgfqpoint{0.371102in}{0.366812in}}%
\pgfpathlineto{\pgfqpoint{0.384371in}{0.363790in}}%
\pgfpathlineto{\pgfqpoint{0.397640in}{0.366490in}}%
\pgfpathlineto{\pgfqpoint{0.398335in}{0.366993in}}%
\pgfpathlineto{\pgfqpoint{0.405576in}{0.379920in}}%
\pgfpathlineto{\pgfqpoint{0.407216in}{0.392846in}}%
\pgfpathlineto{\pgfqpoint{0.406963in}{0.405773in}}%
\pgfpathlineto{\pgfqpoint{0.403999in}{0.418699in}}%
\pgfpathlineto{\pgfqpoint{0.397640in}{0.425593in}}%
\pgfpathlineto{\pgfqpoint{0.384371in}{0.427955in}}%
\pgfpathlineto{\pgfqpoint{0.371102in}{0.425086in}}%
\pgfpathlineto{\pgfqpoint{0.365117in}{0.418699in}}%
\pgfpathlineto{\pgfqpoint{0.361690in}{0.405773in}}%
\pgfpathlineto{\pgfqpoint{0.361342in}{0.392846in}}%
\pgfpathlineto{\pgfqpoint{0.363157in}{0.379920in}}%
\pgfpathlineto{\pgfqpoint{0.370855in}{0.366993in}}%
\pgfpathclose%
\pgfpathmoveto{\pgfqpoint{0.370958in}{0.379920in}}%
\pgfpathlineto{\pgfqpoint{0.367914in}{0.392846in}}%
\pgfpathlineto{\pgfqpoint{0.368843in}{0.405773in}}%
\pgfpathlineto{\pgfqpoint{0.371102in}{0.411199in}}%
\pgfpathlineto{\pgfqpoint{0.382917in}{0.418699in}}%
\pgfpathlineto{\pgfqpoint{0.384371in}{0.419171in}}%
\pgfpathlineto{\pgfqpoint{0.385857in}{0.418699in}}%
\pgfpathlineto{\pgfqpoint{0.397640in}{0.410931in}}%
\pgfpathlineto{\pgfqpoint{0.399697in}{0.405773in}}%
\pgfpathlineto{\pgfqpoint{0.400569in}{0.392846in}}%
\pgfpathlineto{\pgfqpoint{0.397640in}{0.380018in}}%
\pgfpathlineto{\pgfqpoint{0.397574in}{0.379920in}}%
\pgfpathlineto{\pgfqpoint{0.384371in}{0.373025in}}%
\pgfpathlineto{\pgfqpoint{0.371102in}{0.379681in}}%
\pgfpathclose%
\pgfusepath{fill}%
\end{pgfscope}%
\begin{pgfscope}%
\pgfpathrectangle{\pgfqpoint{0.211875in}{0.211875in}}{\pgfqpoint{1.313625in}{1.279725in}}%
\pgfusepath{clip}%
\pgfsetbuttcap%
\pgfsetroundjoin%
\definecolor{currentfill}{rgb}{0.901975,0.231521,0.249182}%
\pgfsetfillcolor{currentfill}%
\pgfsetlinewidth{0.000000pt}%
\definecolor{currentstroke}{rgb}{0.000000,0.000000,0.000000}%
\pgfsetstrokecolor{currentstroke}%
\pgfsetdash{}{0pt}%
\pgfpathmoveto{\pgfqpoint{0.490523in}{0.362304in}}%
\pgfpathlineto{\pgfqpoint{0.503792in}{0.360701in}}%
\pgfpathlineto{\pgfqpoint{0.517061in}{0.363022in}}%
\pgfpathlineto{\pgfqpoint{0.521904in}{0.366993in}}%
\pgfpathlineto{\pgfqpoint{0.526448in}{0.379920in}}%
\pgfpathlineto{\pgfqpoint{0.527539in}{0.392846in}}%
\pgfpathlineto{\pgfqpoint{0.527523in}{0.405773in}}%
\pgfpathlineto{\pgfqpoint{0.526075in}{0.418699in}}%
\pgfpathlineto{\pgfqpoint{0.517061in}{0.429810in}}%
\pgfpathlineto{\pgfqpoint{0.503792in}{0.431433in}}%
\pgfpathlineto{\pgfqpoint{0.490523in}{0.430067in}}%
\pgfpathlineto{\pgfqpoint{0.478525in}{0.418699in}}%
\pgfpathlineto{\pgfqpoint{0.477254in}{0.412249in}}%
\pgfpathlineto{\pgfqpoint{0.476552in}{0.405773in}}%
\pgfpathlineto{\pgfqpoint{0.476456in}{0.392846in}}%
\pgfpathlineto{\pgfqpoint{0.477254in}{0.383330in}}%
\pgfpathlineto{\pgfqpoint{0.477669in}{0.379920in}}%
\pgfpathlineto{\pgfqpoint{0.483342in}{0.366993in}}%
\pgfpathclose%
\pgfpathmoveto{\pgfqpoint{0.485920in}{0.379920in}}%
\pgfpathlineto{\pgfqpoint{0.483227in}{0.392846in}}%
\pgfpathlineto{\pgfqpoint{0.483960in}{0.405773in}}%
\pgfpathlineto{\pgfqpoint{0.489776in}{0.418699in}}%
\pgfpathlineto{\pgfqpoint{0.490523in}{0.419407in}}%
\pgfpathlineto{\pgfqpoint{0.503792in}{0.422596in}}%
\pgfpathlineto{\pgfqpoint{0.513709in}{0.418699in}}%
\pgfpathlineto{\pgfqpoint{0.517061in}{0.415741in}}%
\pgfpathlineto{\pgfqpoint{0.520559in}{0.405773in}}%
\pgfpathlineto{\pgfqpoint{0.521166in}{0.392846in}}%
\pgfpathlineto{\pgfqpoint{0.518820in}{0.379920in}}%
\pgfpathlineto{\pgfqpoint{0.517061in}{0.376531in}}%
\pgfpathlineto{\pgfqpoint{0.503792in}{0.369057in}}%
\pgfpathlineto{\pgfqpoint{0.490523in}{0.373106in}}%
\pgfpathclose%
\pgfusepath{fill}%
\end{pgfscope}%
\begin{pgfscope}%
\pgfpathrectangle{\pgfqpoint{0.211875in}{0.211875in}}{\pgfqpoint{1.313625in}{1.279725in}}%
\pgfusepath{clip}%
\pgfsetbuttcap%
\pgfsetroundjoin%
\definecolor{currentfill}{rgb}{0.901975,0.231521,0.249182}%
\pgfsetfillcolor{currentfill}%
\pgfsetlinewidth{0.000000pt}%
\definecolor{currentstroke}{rgb}{0.000000,0.000000,0.000000}%
\pgfsetstrokecolor{currentstroke}%
\pgfsetdash{}{0pt}%
\pgfpathmoveto{\pgfqpoint{0.596674in}{0.366063in}}%
\pgfpathlineto{\pgfqpoint{0.609943in}{0.358898in}}%
\pgfpathlineto{\pgfqpoint{0.623212in}{0.358135in}}%
\pgfpathlineto{\pgfqpoint{0.636481in}{0.359822in}}%
\pgfpathlineto{\pgfqpoint{0.644174in}{0.366993in}}%
\pgfpathlineto{\pgfqpoint{0.646683in}{0.379920in}}%
\pgfpathlineto{\pgfqpoint{0.647357in}{0.392846in}}%
\pgfpathlineto{\pgfqpoint{0.647520in}{0.405773in}}%
\pgfpathlineto{\pgfqpoint{0.647220in}{0.418699in}}%
\pgfpathlineto{\pgfqpoint{0.643282in}{0.431626in}}%
\pgfpathlineto{\pgfqpoint{0.636481in}{0.433669in}}%
\pgfpathlineto{\pgfqpoint{0.623212in}{0.434131in}}%
\pgfpathlineto{\pgfqpoint{0.609943in}{0.433625in}}%
\pgfpathlineto{\pgfqpoint{0.600504in}{0.431626in}}%
\pgfpathlineto{\pgfqpoint{0.596674in}{0.429262in}}%
\pgfpathlineto{\pgfqpoint{0.593084in}{0.418699in}}%
\pgfpathlineto{\pgfqpoint{0.592384in}{0.405773in}}%
\pgfpathlineto{\pgfqpoint{0.592462in}{0.392846in}}%
\pgfpathlineto{\pgfqpoint{0.593231in}{0.379920in}}%
\pgfpathlineto{\pgfqpoint{0.596164in}{0.366993in}}%
\pgfpathclose%
\pgfpathmoveto{\pgfqpoint{0.616059in}{0.366993in}}%
\pgfpathlineto{\pgfqpoint{0.609943in}{0.368101in}}%
\pgfpathlineto{\pgfqpoint{0.600997in}{0.379920in}}%
\pgfpathlineto{\pgfqpoint{0.598593in}{0.392846in}}%
\pgfpathlineto{\pgfqpoint{0.599152in}{0.405773in}}%
\pgfpathlineto{\pgfqpoint{0.604034in}{0.418699in}}%
\pgfpathlineto{\pgfqpoint{0.609943in}{0.423671in}}%
\pgfpathlineto{\pgfqpoint{0.623212in}{0.425392in}}%
\pgfpathlineto{\pgfqpoint{0.636481in}{0.419443in}}%
\pgfpathlineto{\pgfqpoint{0.637009in}{0.418699in}}%
\pgfpathlineto{\pgfqpoint{0.640792in}{0.405773in}}%
\pgfpathlineto{\pgfqpoint{0.641198in}{0.392846in}}%
\pgfpathlineto{\pgfqpoint{0.639316in}{0.379920in}}%
\pgfpathlineto{\pgfqpoint{0.636481in}{0.373743in}}%
\pgfpathlineto{\pgfqpoint{0.625989in}{0.366993in}}%
\pgfpathlineto{\pgfqpoint{0.623212in}{0.366106in}}%
\pgfpathclose%
\pgfusepath{fill}%
\end{pgfscope}%
\begin{pgfscope}%
\pgfpathrectangle{\pgfqpoint{0.211875in}{0.211875in}}{\pgfqpoint{1.313625in}{1.279725in}}%
\pgfusepath{clip}%
\pgfsetbuttcap%
\pgfsetroundjoin%
\definecolor{currentfill}{rgb}{0.901975,0.231521,0.249182}%
\pgfsetfillcolor{currentfill}%
\pgfsetlinewidth{0.000000pt}%
\definecolor{currentstroke}{rgb}{0.000000,0.000000,0.000000}%
\pgfsetstrokecolor{currentstroke}%
\pgfsetdash{}{0pt}%
\pgfpathmoveto{\pgfqpoint{0.251682in}{0.374429in}}%
\pgfpathlineto{\pgfqpoint{0.264951in}{0.367590in}}%
\pgfpathlineto{\pgfqpoint{0.278220in}{0.371208in}}%
\pgfpathlineto{\pgfqpoint{0.283938in}{0.379920in}}%
\pgfpathlineto{\pgfqpoint{0.286288in}{0.392846in}}%
\pgfpathlineto{\pgfqpoint{0.285729in}{0.405773in}}%
\pgfpathlineto{\pgfqpoint{0.280788in}{0.418699in}}%
\pgfpathlineto{\pgfqpoint{0.278220in}{0.421157in}}%
\pgfpathlineto{\pgfqpoint{0.264951in}{0.423849in}}%
\pgfpathlineto{\pgfqpoint{0.252063in}{0.418699in}}%
\pgfpathlineto{\pgfqpoint{0.251682in}{0.418378in}}%
\pgfpathlineto{\pgfqpoint{0.247032in}{0.405773in}}%
\pgfpathlineto{\pgfqpoint{0.246439in}{0.392846in}}%
\pgfpathlineto{\pgfqpoint{0.248723in}{0.379920in}}%
\pgfpathclose%
\pgfpathmoveto{\pgfqpoint{0.261555in}{0.379920in}}%
\pgfpathlineto{\pgfqpoint{0.253960in}{0.392846in}}%
\pgfpathlineto{\pgfqpoint{0.256488in}{0.405773in}}%
\pgfpathlineto{\pgfqpoint{0.264951in}{0.413211in}}%
\pgfpathlineto{\pgfqpoint{0.277799in}{0.405773in}}%
\pgfpathlineto{\pgfqpoint{0.278220in}{0.404392in}}%
\pgfpathlineto{\pgfqpoint{0.279294in}{0.392846in}}%
\pgfpathlineto{\pgfqpoint{0.278220in}{0.388638in}}%
\pgfpathlineto{\pgfqpoint{0.270183in}{0.379920in}}%
\pgfpathlineto{\pgfqpoint{0.264951in}{0.377804in}}%
\pgfpathclose%
\pgfusepath{fill}%
\end{pgfscope}%
\begin{pgfscope}%
\pgfpathrectangle{\pgfqpoint{0.211875in}{0.211875in}}{\pgfqpoint{1.313625in}{1.279725in}}%
\pgfusepath{clip}%
\pgfsetbuttcap%
\pgfsetroundjoin%
\definecolor{currentfill}{rgb}{0.901975,0.231521,0.249182}%
\pgfsetfillcolor{currentfill}%
\pgfsetlinewidth{0.000000pt}%
\definecolor{currentstroke}{rgb}{0.000000,0.000000,0.000000}%
\pgfsetstrokecolor{currentstroke}%
\pgfsetdash{}{0pt}%
\pgfpathmoveto{\pgfqpoint{0.437447in}{0.441377in}}%
\pgfpathlineto{\pgfqpoint{0.450716in}{0.441314in}}%
\pgfpathlineto{\pgfqpoint{0.460752in}{0.444552in}}%
\pgfpathlineto{\pgfqpoint{0.463985in}{0.447137in}}%
\pgfpathlineto{\pgfqpoint{0.467904in}{0.457479in}}%
\pgfpathlineto{\pgfqpoint{0.469157in}{0.470405in}}%
\pgfpathlineto{\pgfqpoint{0.469371in}{0.483332in}}%
\pgfpathlineto{\pgfqpoint{0.468724in}{0.496258in}}%
\pgfpathlineto{\pgfqpoint{0.464262in}{0.509185in}}%
\pgfpathlineto{\pgfqpoint{0.463985in}{0.509441in}}%
\pgfpathlineto{\pgfqpoint{0.450716in}{0.513077in}}%
\pgfpathlineto{\pgfqpoint{0.437447in}{0.512904in}}%
\pgfpathlineto{\pgfqpoint{0.424178in}{0.509225in}}%
\pgfpathlineto{\pgfqpoint{0.424126in}{0.509185in}}%
\pgfpathlineto{\pgfqpoint{0.418227in}{0.496258in}}%
\pgfpathlineto{\pgfqpoint{0.417237in}{0.483332in}}%
\pgfpathlineto{\pgfqpoint{0.417433in}{0.470405in}}%
\pgfpathlineto{\pgfqpoint{0.418957in}{0.457479in}}%
\pgfpathlineto{\pgfqpoint{0.424178in}{0.446390in}}%
\pgfpathlineto{\pgfqpoint{0.427108in}{0.444552in}}%
\pgfpathclose%
\pgfpathmoveto{\pgfqpoint{0.428129in}{0.457479in}}%
\pgfpathlineto{\pgfqpoint{0.424178in}{0.466507in}}%
\pgfpathlineto{\pgfqpoint{0.423362in}{0.470405in}}%
\pgfpathlineto{\pgfqpoint{0.423246in}{0.483332in}}%
\pgfpathlineto{\pgfqpoint{0.424178in}{0.488155in}}%
\pgfpathlineto{\pgfqpoint{0.427565in}{0.496258in}}%
\pgfpathlineto{\pgfqpoint{0.437447in}{0.503725in}}%
\pgfpathlineto{\pgfqpoint{0.450716in}{0.503368in}}%
\pgfpathlineto{\pgfqpoint{0.458870in}{0.496258in}}%
\pgfpathlineto{\pgfqpoint{0.462943in}{0.483332in}}%
\pgfpathlineto{\pgfqpoint{0.462758in}{0.470405in}}%
\pgfpathlineto{\pgfqpoint{0.458294in}{0.457479in}}%
\pgfpathlineto{\pgfqpoint{0.450716in}{0.450735in}}%
\pgfpathlineto{\pgfqpoint{0.437447in}{0.450298in}}%
\pgfpathclose%
\pgfusepath{fill}%
\end{pgfscope}%
\begin{pgfscope}%
\pgfpathrectangle{\pgfqpoint{0.211875in}{0.211875in}}{\pgfqpoint{1.313625in}{1.279725in}}%
\pgfusepath{clip}%
\pgfsetbuttcap%
\pgfsetroundjoin%
\definecolor{currentfill}{rgb}{0.901975,0.231521,0.249182}%
\pgfsetfillcolor{currentfill}%
\pgfsetlinewidth{0.000000pt}%
\definecolor{currentstroke}{rgb}{0.000000,0.000000,0.000000}%
\pgfsetstrokecolor{currentstroke}%
\pgfsetdash{}{0pt}%
\pgfpathmoveto{\pgfqpoint{0.543598in}{0.440413in}}%
\pgfpathlineto{\pgfqpoint{0.556867in}{0.438381in}}%
\pgfpathlineto{\pgfqpoint{0.570136in}{0.438384in}}%
\pgfpathlineto{\pgfqpoint{0.583405in}{0.441504in}}%
\pgfpathlineto{\pgfqpoint{0.585978in}{0.444552in}}%
\pgfpathlineto{\pgfqpoint{0.588614in}{0.457479in}}%
\pgfpathlineto{\pgfqpoint{0.589218in}{0.470405in}}%
\pgfpathlineto{\pgfqpoint{0.589460in}{0.483332in}}%
\pgfpathlineto{\pgfqpoint{0.589570in}{0.496258in}}%
\pgfpathlineto{\pgfqpoint{0.589593in}{0.509185in}}%
\pgfpathlineto{\pgfqpoint{0.583405in}{0.516307in}}%
\pgfpathlineto{\pgfqpoint{0.570136in}{0.516256in}}%
\pgfpathlineto{\pgfqpoint{0.556867in}{0.516010in}}%
\pgfpathlineto{\pgfqpoint{0.543598in}{0.515047in}}%
\pgfpathlineto{\pgfqpoint{0.534909in}{0.509185in}}%
\pgfpathlineto{\pgfqpoint{0.533095in}{0.496258in}}%
\pgfpathlineto{\pgfqpoint{0.532868in}{0.483332in}}%
\pgfpathlineto{\pgfqpoint{0.533100in}{0.470405in}}%
\pgfpathlineto{\pgfqpoint{0.533992in}{0.457479in}}%
\pgfpathlineto{\pgfqpoint{0.537906in}{0.444552in}}%
\pgfpathclose%
\pgfpathmoveto{\pgfqpoint{0.541860in}{0.457479in}}%
\pgfpathlineto{\pgfqpoint{0.539181in}{0.470405in}}%
\pgfpathlineto{\pgfqpoint{0.539031in}{0.483332in}}%
\pgfpathlineto{\pgfqpoint{0.541357in}{0.496258in}}%
\pgfpathlineto{\pgfqpoint{0.543598in}{0.500489in}}%
\pgfpathlineto{\pgfqpoint{0.556867in}{0.507565in}}%
\pgfpathlineto{\pgfqpoint{0.570136in}{0.506656in}}%
\pgfpathlineto{\pgfqpoint{0.580649in}{0.496258in}}%
\pgfpathlineto{\pgfqpoint{0.583381in}{0.483332in}}%
\pgfpathlineto{\pgfqpoint{0.583172in}{0.470405in}}%
\pgfpathlineto{\pgfqpoint{0.579926in}{0.457479in}}%
\pgfpathlineto{\pgfqpoint{0.570136in}{0.447629in}}%
\pgfpathlineto{\pgfqpoint{0.556867in}{0.446528in}}%
\pgfpathlineto{\pgfqpoint{0.543598in}{0.454244in}}%
\pgfpathclose%
\pgfusepath{fill}%
\end{pgfscope}%
\begin{pgfscope}%
\pgfpathrectangle{\pgfqpoint{0.211875in}{0.211875in}}{\pgfqpoint{1.313625in}{1.279725in}}%
\pgfusepath{clip}%
\pgfsetbuttcap%
\pgfsetroundjoin%
\definecolor{currentfill}{rgb}{0.901975,0.231521,0.249182}%
\pgfsetfillcolor{currentfill}%
\pgfsetlinewidth{0.000000pt}%
\definecolor{currentstroke}{rgb}{0.000000,0.000000,0.000000}%
\pgfsetstrokecolor{currentstroke}%
\pgfsetdash{}{0pt}%
\pgfpathmoveto{\pgfqpoint{0.224234in}{0.457479in}}%
\pgfpathlineto{\pgfqpoint{0.225144in}{0.459277in}}%
\pgfpathlineto{\pgfqpoint{0.227647in}{0.470405in}}%
\pgfpathlineto{\pgfqpoint{0.227789in}{0.483332in}}%
\pgfpathlineto{\pgfqpoint{0.225144in}{0.496013in}}%
\pgfpathlineto{\pgfqpoint{0.225024in}{0.496258in}}%
\pgfpathlineto{\pgfqpoint{0.211875in}{0.504948in}}%
\pgfpathlineto{\pgfqpoint{0.211875in}{0.496258in}}%
\pgfpathlineto{\pgfqpoint{0.211875in}{0.494360in}}%
\pgfpathlineto{\pgfqpoint{0.218864in}{0.483332in}}%
\pgfpathlineto{\pgfqpoint{0.218774in}{0.470405in}}%
\pgfpathlineto{\pgfqpoint{0.211875in}{0.459486in}}%
\pgfpathlineto{\pgfqpoint{0.211875in}{0.457479in}}%
\pgfpathlineto{\pgfqpoint{0.211875in}{0.449101in}}%
\pgfpathclose%
\pgfusepath{fill}%
\end{pgfscope}%
\begin{pgfscope}%
\pgfpathrectangle{\pgfqpoint{0.211875in}{0.211875in}}{\pgfqpoint{1.313625in}{1.279725in}}%
\pgfusepath{clip}%
\pgfsetbuttcap%
\pgfsetroundjoin%
\definecolor{currentfill}{rgb}{0.901975,0.231521,0.249182}%
\pgfsetfillcolor{currentfill}%
\pgfsetlinewidth{0.000000pt}%
\definecolor{currentstroke}{rgb}{0.000000,0.000000,0.000000}%
\pgfsetstrokecolor{currentstroke}%
\pgfsetdash{}{0pt}%
\pgfpathmoveto{\pgfqpoint{0.304758in}{0.456278in}}%
\pgfpathlineto{\pgfqpoint{0.318027in}{0.445209in}}%
\pgfpathlineto{\pgfqpoint{0.331295in}{0.444642in}}%
\pgfpathlineto{\pgfqpoint{0.344564in}{0.452913in}}%
\pgfpathlineto{\pgfqpoint{0.346596in}{0.457479in}}%
\pgfpathlineto{\pgfqpoint{0.348650in}{0.470405in}}%
\pgfpathlineto{\pgfqpoint{0.348832in}{0.483332in}}%
\pgfpathlineto{\pgfqpoint{0.347252in}{0.496258in}}%
\pgfpathlineto{\pgfqpoint{0.344564in}{0.502518in}}%
\pgfpathlineto{\pgfqpoint{0.332695in}{0.509185in}}%
\pgfpathlineto{\pgfqpoint{0.331295in}{0.509527in}}%
\pgfpathlineto{\pgfqpoint{0.321961in}{0.509185in}}%
\pgfpathlineto{\pgfqpoint{0.318027in}{0.509014in}}%
\pgfpathlineto{\pgfqpoint{0.304758in}{0.498921in}}%
\pgfpathlineto{\pgfqpoint{0.303718in}{0.496258in}}%
\pgfpathlineto{\pgfqpoint{0.301866in}{0.483332in}}%
\pgfpathlineto{\pgfqpoint{0.302023in}{0.470405in}}%
\pgfpathlineto{\pgfqpoint{0.304269in}{0.457479in}}%
\pgfpathclose%
\pgfpathmoveto{\pgfqpoint{0.315211in}{0.457479in}}%
\pgfpathlineto{\pgfqpoint{0.309363in}{0.470405in}}%
\pgfpathlineto{\pgfqpoint{0.309244in}{0.483332in}}%
\pgfpathlineto{\pgfqpoint{0.314940in}{0.496258in}}%
\pgfpathlineto{\pgfqpoint{0.318027in}{0.498916in}}%
\pgfpathlineto{\pgfqpoint{0.331295in}{0.499546in}}%
\pgfpathlineto{\pgfqpoint{0.335603in}{0.496258in}}%
\pgfpathlineto{\pgfqpoint{0.341563in}{0.483332in}}%
\pgfpathlineto{\pgfqpoint{0.341417in}{0.470405in}}%
\pgfpathlineto{\pgfqpoint{0.335250in}{0.457479in}}%
\pgfpathlineto{\pgfqpoint{0.331295in}{0.454390in}}%
\pgfpathlineto{\pgfqpoint{0.318027in}{0.455010in}}%
\pgfpathclose%
\pgfusepath{fill}%
\end{pgfscope}%
\begin{pgfscope}%
\pgfpathrectangle{\pgfqpoint{0.211875in}{0.211875in}}{\pgfqpoint{1.313625in}{1.279725in}}%
\pgfusepath{clip}%
\pgfsetbuttcap%
\pgfsetroundjoin%
\definecolor{currentfill}{rgb}{0.901975,0.231521,0.249182}%
\pgfsetfillcolor{currentfill}%
\pgfsetlinewidth{0.000000pt}%
\definecolor{currentstroke}{rgb}{0.000000,0.000000,0.000000}%
\pgfsetstrokecolor{currentstroke}%
\pgfsetdash{}{0pt}%
\pgfpathmoveto{\pgfqpoint{0.384371in}{0.521719in}}%
\pgfpathlineto{\pgfqpoint{0.389590in}{0.522111in}}%
\pgfpathlineto{\pgfqpoint{0.397640in}{0.523013in}}%
\pgfpathlineto{\pgfqpoint{0.408534in}{0.535038in}}%
\pgfpathlineto{\pgfqpoint{0.410381in}{0.547964in}}%
\pgfpathlineto{\pgfqpoint{0.410815in}{0.560891in}}%
\pgfpathlineto{\pgfqpoint{0.410456in}{0.573817in}}%
\pgfpathlineto{\pgfqpoint{0.408031in}{0.586744in}}%
\pgfpathlineto{\pgfqpoint{0.397640in}{0.593867in}}%
\pgfpathlineto{\pgfqpoint{0.384371in}{0.594571in}}%
\pgfpathlineto{\pgfqpoint{0.371102in}{0.593086in}}%
\pgfpathlineto{\pgfqpoint{0.362026in}{0.586744in}}%
\pgfpathlineto{\pgfqpoint{0.358433in}{0.573817in}}%
\pgfpathlineto{\pgfqpoint{0.357833in}{0.561914in}}%
\pgfpathlineto{\pgfqpoint{0.357797in}{0.560891in}}%
\pgfpathlineto{\pgfqpoint{0.357833in}{0.559752in}}%
\pgfpathlineto{\pgfqpoint{0.358296in}{0.547964in}}%
\pgfpathlineto{\pgfqpoint{0.360650in}{0.535038in}}%
\pgfpathlineto{\pgfqpoint{0.371102in}{0.523693in}}%
\pgfpathlineto{\pgfqpoint{0.380931in}{0.522111in}}%
\pgfpathclose%
\pgfpathmoveto{\pgfqpoint{0.370616in}{0.535038in}}%
\pgfpathlineto{\pgfqpoint{0.365295in}{0.547964in}}%
\pgfpathlineto{\pgfqpoint{0.364400in}{0.560891in}}%
\pgfpathlineto{\pgfqpoint{0.366523in}{0.573817in}}%
\pgfpathlineto{\pgfqpoint{0.371102in}{0.581426in}}%
\pgfpathlineto{\pgfqpoint{0.384371in}{0.586328in}}%
\pgfpathlineto{\pgfqpoint{0.397640in}{0.581630in}}%
\pgfpathlineto{\pgfqpoint{0.402180in}{0.573817in}}%
\pgfpathlineto{\pgfqpoint{0.404128in}{0.560891in}}%
\pgfpathlineto{\pgfqpoint{0.403278in}{0.547964in}}%
\pgfpathlineto{\pgfqpoint{0.398192in}{0.535038in}}%
\pgfpathlineto{\pgfqpoint{0.397640in}{0.534429in}}%
\pgfpathlineto{\pgfqpoint{0.384371in}{0.530188in}}%
\pgfpathlineto{\pgfqpoint{0.371102in}{0.534510in}}%
\pgfpathclose%
\pgfusepath{fill}%
\end{pgfscope}%
\begin{pgfscope}%
\pgfpathrectangle{\pgfqpoint{0.211875in}{0.211875in}}{\pgfqpoint{1.313625in}{1.279725in}}%
\pgfusepath{clip}%
\pgfsetbuttcap%
\pgfsetroundjoin%
\definecolor{currentfill}{rgb}{0.901975,0.231521,0.249182}%
\pgfsetfillcolor{currentfill}%
\pgfsetlinewidth{0.000000pt}%
\definecolor{currentstroke}{rgb}{0.000000,0.000000,0.000000}%
\pgfsetstrokecolor{currentstroke}%
\pgfsetdash{}{0pt}%
\pgfpathmoveto{\pgfqpoint{0.251682in}{0.529706in}}%
\pgfpathlineto{\pgfqpoint{0.264951in}{0.525653in}}%
\pgfpathlineto{\pgfqpoint{0.278220in}{0.527486in}}%
\pgfpathlineto{\pgfqpoint{0.285927in}{0.535038in}}%
\pgfpathlineto{\pgfqpoint{0.289404in}{0.547964in}}%
\pgfpathlineto{\pgfqpoint{0.290056in}{0.560891in}}%
\pgfpathlineto{\pgfqpoint{0.288931in}{0.573817in}}%
\pgfpathlineto{\pgfqpoint{0.282596in}{0.586744in}}%
\pgfpathlineto{\pgfqpoint{0.278220in}{0.589384in}}%
\pgfpathlineto{\pgfqpoint{0.264951in}{0.590763in}}%
\pgfpathlineto{\pgfqpoint{0.251682in}{0.587377in}}%
\pgfpathlineto{\pgfqpoint{0.250880in}{0.586744in}}%
\pgfpathlineto{\pgfqpoint{0.244347in}{0.573817in}}%
\pgfpathlineto{\pgfqpoint{0.243067in}{0.560891in}}%
\pgfpathlineto{\pgfqpoint{0.243725in}{0.547964in}}%
\pgfpathlineto{\pgfqpoint{0.247305in}{0.535038in}}%
\pgfpathclose%
\pgfpathmoveto{\pgfqpoint{0.262844in}{0.535038in}}%
\pgfpathlineto{\pgfqpoint{0.251682in}{0.544579in}}%
\pgfpathlineto{\pgfqpoint{0.250404in}{0.547964in}}%
\pgfpathlineto{\pgfqpoint{0.249374in}{0.560891in}}%
\pgfpathlineto{\pgfqpoint{0.251682in}{0.572281in}}%
\pgfpathlineto{\pgfqpoint{0.252520in}{0.573817in}}%
\pgfpathlineto{\pgfqpoint{0.264951in}{0.581455in}}%
\pgfpathlineto{\pgfqpoint{0.278220in}{0.576850in}}%
\pgfpathlineto{\pgfqpoint{0.280212in}{0.573817in}}%
\pgfpathlineto{\pgfqpoint{0.283020in}{0.560891in}}%
\pgfpathlineto{\pgfqpoint{0.281928in}{0.547964in}}%
\pgfpathlineto{\pgfqpoint{0.278220in}{0.539846in}}%
\pgfpathlineto{\pgfqpoint{0.268366in}{0.535038in}}%
\pgfpathlineto{\pgfqpoint{0.264951in}{0.534171in}}%
\pgfpathclose%
\pgfusepath{fill}%
\end{pgfscope}%
\begin{pgfscope}%
\pgfpathrectangle{\pgfqpoint{0.211875in}{0.211875in}}{\pgfqpoint{1.313625in}{1.279725in}}%
\pgfusepath{clip}%
\pgfsetbuttcap%
\pgfsetroundjoin%
\definecolor{currentfill}{rgb}{0.901975,0.231521,0.249182}%
\pgfsetfillcolor{currentfill}%
\pgfsetlinewidth{0.000000pt}%
\definecolor{currentstroke}{rgb}{0.000000,0.000000,0.000000}%
\pgfsetstrokecolor{currentstroke}%
\pgfsetdash{}{0pt}%
\pgfpathmoveto{\pgfqpoint{0.224915in}{0.612597in}}%
\pgfpathlineto{\pgfqpoint{0.225144in}{0.612816in}}%
\pgfpathlineto{\pgfqpoint{0.229745in}{0.625523in}}%
\pgfpathlineto{\pgfqpoint{0.230779in}{0.638450in}}%
\pgfpathlineto{\pgfqpoint{0.230229in}{0.651377in}}%
\pgfpathlineto{\pgfqpoint{0.226685in}{0.664303in}}%
\pgfpathlineto{\pgfqpoint{0.225144in}{0.666360in}}%
\pgfpathlineto{\pgfqpoint{0.211875in}{0.671371in}}%
\pgfpathlineto{\pgfqpoint{0.211875in}{0.664303in}}%
\pgfpathlineto{\pgfqpoint{0.211875in}{0.661931in}}%
\pgfpathlineto{\pgfqpoint{0.221992in}{0.651377in}}%
\pgfpathlineto{\pgfqpoint{0.224383in}{0.638450in}}%
\pgfpathlineto{\pgfqpoint{0.221076in}{0.625523in}}%
\pgfpathlineto{\pgfqpoint{0.211875in}{0.616660in}}%
\pgfpathlineto{\pgfqpoint{0.211875in}{0.612597in}}%
\pgfpathlineto{\pgfqpoint{0.211875in}{0.607197in}}%
\pgfpathclose%
\pgfusepath{fill}%
\end{pgfscope}%
\begin{pgfscope}%
\pgfpathrectangle{\pgfqpoint{0.211875in}{0.211875in}}{\pgfqpoint{1.313625in}{1.279725in}}%
\pgfusepath{clip}%
\pgfsetbuttcap%
\pgfsetroundjoin%
\definecolor{currentfill}{rgb}{0.901975,0.231521,0.249182}%
\pgfsetfillcolor{currentfill}%
\pgfsetlinewidth{0.000000pt}%
\definecolor{currentstroke}{rgb}{0.000000,0.000000,0.000000}%
\pgfsetstrokecolor{currentstroke}%
\pgfsetdash{}{0pt}%
\pgfpathmoveto{\pgfqpoint{0.304758in}{0.609699in}}%
\pgfpathlineto{\pgfqpoint{0.318027in}{0.603653in}}%
\pgfpathlineto{\pgfqpoint{0.331295in}{0.603177in}}%
\pgfpathlineto{\pgfqpoint{0.344564in}{0.606914in}}%
\pgfpathlineto{\pgfqpoint{0.348611in}{0.612597in}}%
\pgfpathlineto{\pgfqpoint{0.351053in}{0.625523in}}%
\pgfpathlineto{\pgfqpoint{0.351644in}{0.638450in}}%
\pgfpathlineto{\pgfqpoint{0.351503in}{0.651377in}}%
\pgfpathlineto{\pgfqpoint{0.350128in}{0.664303in}}%
\pgfpathlineto{\pgfqpoint{0.344564in}{0.672998in}}%
\pgfpathlineto{\pgfqpoint{0.331295in}{0.675619in}}%
\pgfpathlineto{\pgfqpoint{0.318027in}{0.675129in}}%
\pgfpathlineto{\pgfqpoint{0.304758in}{0.670034in}}%
\pgfpathlineto{\pgfqpoint{0.301485in}{0.664303in}}%
\pgfpathlineto{\pgfqpoint{0.299434in}{0.651377in}}%
\pgfpathlineto{\pgfqpoint{0.299143in}{0.638450in}}%
\pgfpathlineto{\pgfqpoint{0.299870in}{0.625523in}}%
\pgfpathlineto{\pgfqpoint{0.302915in}{0.612597in}}%
\pgfpathclose%
\pgfpathmoveto{\pgfqpoint{0.317434in}{0.612597in}}%
\pgfpathlineto{\pgfqpoint{0.307122in}{0.625523in}}%
\pgfpathlineto{\pgfqpoint{0.304856in}{0.638450in}}%
\pgfpathlineto{\pgfqpoint{0.306406in}{0.651377in}}%
\pgfpathlineto{\pgfqpoint{0.315021in}{0.664303in}}%
\pgfpathlineto{\pgfqpoint{0.318027in}{0.666100in}}%
\pgfpathlineto{\pgfqpoint{0.331295in}{0.666641in}}%
\pgfpathlineto{\pgfqpoint{0.335840in}{0.664303in}}%
\pgfpathlineto{\pgfqpoint{0.344564in}{0.651932in}}%
\pgfpathlineto{\pgfqpoint{0.344717in}{0.651377in}}%
\pgfpathlineto{\pgfqpoint{0.345685in}{0.638450in}}%
\pgfpathlineto{\pgfqpoint{0.344564in}{0.628275in}}%
\pgfpathlineto{\pgfqpoint{0.344024in}{0.625523in}}%
\pgfpathlineto{\pgfqpoint{0.333075in}{0.612597in}}%
\pgfpathlineto{\pgfqpoint{0.331295in}{0.611743in}}%
\pgfpathlineto{\pgfqpoint{0.318027in}{0.612265in}}%
\pgfpathclose%
\pgfusepath{fill}%
\end{pgfscope}%
\begin{pgfscope}%
\pgfpathrectangle{\pgfqpoint{0.211875in}{0.211875in}}{\pgfqpoint{1.313625in}{1.279725in}}%
\pgfusepath{clip}%
\pgfsetbuttcap%
\pgfsetroundjoin%
\definecolor{currentfill}{rgb}{0.901975,0.231521,0.249182}%
\pgfsetfillcolor{currentfill}%
\pgfsetlinewidth{0.000000pt}%
\definecolor{currentstroke}{rgb}{0.000000,0.000000,0.000000}%
\pgfsetstrokecolor{currentstroke}%
\pgfsetdash{}{0pt}%
\pgfpathmoveto{\pgfqpoint{0.251682in}{0.687188in}}%
\pgfpathlineto{\pgfqpoint{0.264951in}{0.684646in}}%
\pgfpathlineto{\pgfqpoint{0.278220in}{0.685486in}}%
\pgfpathlineto{\pgfqpoint{0.286628in}{0.690156in}}%
\pgfpathlineto{\pgfqpoint{0.291489in}{0.702390in}}%
\pgfpathlineto{\pgfqpoint{0.291605in}{0.703083in}}%
\pgfpathlineto{\pgfqpoint{0.292526in}{0.716009in}}%
\pgfpathlineto{\pgfqpoint{0.292532in}{0.728936in}}%
\pgfpathlineto{\pgfqpoint{0.291489in}{0.741857in}}%
\pgfpathlineto{\pgfqpoint{0.291488in}{0.741862in}}%
\pgfpathlineto{\pgfqpoint{0.281687in}{0.754789in}}%
\pgfpathlineto{\pgfqpoint{0.278220in}{0.755769in}}%
\pgfpathlineto{\pgfqpoint{0.264951in}{0.756326in}}%
\pgfpathlineto{\pgfqpoint{0.254794in}{0.754789in}}%
\pgfpathlineto{\pgfqpoint{0.251682in}{0.754059in}}%
\pgfpathlineto{\pgfqpoint{0.242436in}{0.741862in}}%
\pgfpathlineto{\pgfqpoint{0.240763in}{0.728936in}}%
\pgfpathlineto{\pgfqpoint{0.240697in}{0.716009in}}%
\pgfpathlineto{\pgfqpoint{0.241979in}{0.703083in}}%
\pgfpathlineto{\pgfqpoint{0.247597in}{0.690156in}}%
\pgfpathclose%
\pgfpathmoveto{\pgfqpoint{0.249768in}{0.703083in}}%
\pgfpathlineto{\pgfqpoint{0.247016in}{0.716009in}}%
\pgfpathlineto{\pgfqpoint{0.247410in}{0.728936in}}%
\pgfpathlineto{\pgfqpoint{0.251682in}{0.741624in}}%
\pgfpathlineto{\pgfqpoint{0.251928in}{0.741862in}}%
\pgfpathlineto{\pgfqpoint{0.264951in}{0.747569in}}%
\pgfpathlineto{\pgfqpoint{0.278220in}{0.744536in}}%
\pgfpathlineto{\pgfqpoint{0.280749in}{0.741862in}}%
\pgfpathlineto{\pgfqpoint{0.285272in}{0.728936in}}%
\pgfpathlineto{\pgfqpoint{0.285653in}{0.716009in}}%
\pgfpathlineto{\pgfqpoint{0.282817in}{0.703083in}}%
\pgfpathlineto{\pgfqpoint{0.278220in}{0.696643in}}%
\pgfpathlineto{\pgfqpoint{0.264951in}{0.693063in}}%
\pgfpathlineto{\pgfqpoint{0.251682in}{0.699800in}}%
\pgfpathclose%
\pgfusepath{fill}%
\end{pgfscope}%
\begin{pgfscope}%
\pgfpathrectangle{\pgfqpoint{0.211875in}{0.211875in}}{\pgfqpoint{1.313625in}{1.279725in}}%
\pgfusepath{clip}%
\pgfsetbuttcap%
\pgfsetroundjoin%
\definecolor{currentfill}{rgb}{0.901975,0.231521,0.249182}%
\pgfsetfillcolor{currentfill}%
\pgfsetlinewidth{0.000000pt}%
\definecolor{currentstroke}{rgb}{0.000000,0.000000,0.000000}%
\pgfsetstrokecolor{currentstroke}%
\pgfsetdash{}{0pt}%
\pgfpathmoveto{\pgfqpoint{0.888591in}{0.679603in}}%
\pgfpathlineto{\pgfqpoint{0.901860in}{0.679523in}}%
\pgfpathlineto{\pgfqpoint{0.915129in}{0.679594in}}%
\pgfpathlineto{\pgfqpoint{0.928398in}{0.679869in}}%
\pgfpathlineto{\pgfqpoint{0.941667in}{0.684675in}}%
\pgfpathlineto{\pgfqpoint{0.942625in}{0.690156in}}%
\pgfpathlineto{\pgfqpoint{0.943055in}{0.703083in}}%
\pgfpathlineto{\pgfqpoint{0.943066in}{0.716009in}}%
\pgfpathlineto{\pgfqpoint{0.942857in}{0.728936in}}%
\pgfpathlineto{\pgfqpoint{0.942176in}{0.741862in}}%
\pgfpathlineto{\pgfqpoint{0.941667in}{0.745676in}}%
\pgfpathlineto{\pgfqpoint{0.937734in}{0.754789in}}%
\pgfpathlineto{\pgfqpoint{0.928398in}{0.758195in}}%
\pgfpathlineto{\pgfqpoint{0.915129in}{0.758905in}}%
\pgfpathlineto{\pgfqpoint{0.901860in}{0.758636in}}%
\pgfpathlineto{\pgfqpoint{0.889197in}{0.754789in}}%
\pgfpathlineto{\pgfqpoint{0.888591in}{0.754014in}}%
\pgfpathlineto{\pgfqpoint{0.885959in}{0.741862in}}%
\pgfpathlineto{\pgfqpoint{0.885447in}{0.728936in}}%
\pgfpathlineto{\pgfqpoint{0.885259in}{0.716009in}}%
\pgfpathlineto{\pgfqpoint{0.885178in}{0.703083in}}%
\pgfpathlineto{\pgfqpoint{0.885160in}{0.690156in}}%
\pgfpathclose%
\pgfpathmoveto{\pgfqpoint{0.901398in}{0.690156in}}%
\pgfpathlineto{\pgfqpoint{0.893183in}{0.703083in}}%
\pgfpathlineto{\pgfqpoint{0.891604in}{0.716009in}}%
\pgfpathlineto{\pgfqpoint{0.892194in}{0.728936in}}%
\pgfpathlineto{\pgfqpoint{0.895908in}{0.741862in}}%
\pgfpathlineto{\pgfqpoint{0.901860in}{0.748143in}}%
\pgfpathlineto{\pgfqpoint{0.915129in}{0.750739in}}%
\pgfpathlineto{\pgfqpoint{0.928398in}{0.746869in}}%
\pgfpathlineto{\pgfqpoint{0.932534in}{0.741862in}}%
\pgfpathlineto{\pgfqpoint{0.936145in}{0.728936in}}%
\pgfpathlineto{\pgfqpoint{0.936716in}{0.716009in}}%
\pgfpathlineto{\pgfqpoint{0.935158in}{0.703083in}}%
\pgfpathlineto{\pgfqpoint{0.928398in}{0.691212in}}%
\pgfpathlineto{\pgfqpoint{0.925766in}{0.690156in}}%
\pgfpathlineto{\pgfqpoint{0.915129in}{0.687699in}}%
\pgfpathlineto{\pgfqpoint{0.901860in}{0.689871in}}%
\pgfpathclose%
\pgfusepath{fill}%
\end{pgfscope}%
\begin{pgfscope}%
\pgfpathrectangle{\pgfqpoint{0.211875in}{0.211875in}}{\pgfqpoint{1.313625in}{1.279725in}}%
\pgfusepath{clip}%
\pgfsetbuttcap%
\pgfsetroundjoin%
\definecolor{currentfill}{rgb}{0.901975,0.231521,0.249182}%
\pgfsetfillcolor{currentfill}%
\pgfsetlinewidth{0.000000pt}%
\definecolor{currentstroke}{rgb}{0.000000,0.000000,0.000000}%
\pgfsetstrokecolor{currentstroke}%
\pgfsetdash{}{0pt}%
\pgfpathmoveto{\pgfqpoint{1.008011in}{0.683036in}}%
\pgfpathlineto{\pgfqpoint{1.021280in}{0.680548in}}%
\pgfpathlineto{\pgfqpoint{1.034549in}{0.680404in}}%
\pgfpathlineto{\pgfqpoint{1.047818in}{0.681027in}}%
\pgfpathlineto{\pgfqpoint{1.059185in}{0.690156in}}%
\pgfpathlineto{\pgfqpoint{1.060242in}{0.703083in}}%
\pgfpathlineto{\pgfqpoint{1.060378in}{0.716009in}}%
\pgfpathlineto{\pgfqpoint{1.060130in}{0.728936in}}%
\pgfpathlineto{\pgfqpoint{1.059181in}{0.741862in}}%
\pgfpathlineto{\pgfqpoint{1.053222in}{0.754789in}}%
\pgfpathlineto{\pgfqpoint{1.047818in}{0.756966in}}%
\pgfpathlineto{\pgfqpoint{1.034549in}{0.758123in}}%
\pgfpathlineto{\pgfqpoint{1.021280in}{0.757752in}}%
\pgfpathlineto{\pgfqpoint{1.010561in}{0.754789in}}%
\pgfpathlineto{\pgfqpoint{1.008011in}{0.752489in}}%
\pgfpathlineto{\pgfqpoint{1.004911in}{0.741862in}}%
\pgfpathlineto{\pgfqpoint{1.004103in}{0.728936in}}%
\pgfpathlineto{\pgfqpoint{1.003881in}{0.716009in}}%
\pgfpathlineto{\pgfqpoint{1.003964in}{0.703083in}}%
\pgfpathlineto{\pgfqpoint{1.004744in}{0.690156in}}%
\pgfpathclose%
\pgfpathmoveto{\pgfqpoint{1.022210in}{0.690156in}}%
\pgfpathlineto{\pgfqpoint{1.021280in}{0.690338in}}%
\pgfpathlineto{\pgfqpoint{1.011996in}{0.703083in}}%
\pgfpathlineto{\pgfqpoint{1.010141in}{0.716009in}}%
\pgfpathlineto{\pgfqpoint{1.010805in}{0.728936in}}%
\pgfpathlineto{\pgfqpoint{1.015066in}{0.741862in}}%
\pgfpathlineto{\pgfqpoint{1.021280in}{0.747762in}}%
\pgfpathlineto{\pgfqpoint{1.034549in}{0.749809in}}%
\pgfpathlineto{\pgfqpoint{1.047818in}{0.744530in}}%
\pgfpathlineto{\pgfqpoint{1.049792in}{0.741862in}}%
\pgfpathlineto{\pgfqpoint{1.053533in}{0.728936in}}%
\pgfpathlineto{\pgfqpoint{1.054115in}{0.716009in}}%
\pgfpathlineto{\pgfqpoint{1.052476in}{0.703083in}}%
\pgfpathlineto{\pgfqpoint{1.047818in}{0.693900in}}%
\pgfpathlineto{\pgfqpoint{1.040439in}{0.690156in}}%
\pgfpathlineto{\pgfqpoint{1.034549in}{0.688564in}}%
\pgfpathclose%
\pgfusepath{fill}%
\end{pgfscope}%
\begin{pgfscope}%
\pgfpathrectangle{\pgfqpoint{0.211875in}{0.211875in}}{\pgfqpoint{1.313625in}{1.279725in}}%
\pgfusepath{clip}%
\pgfsetbuttcap%
\pgfsetroundjoin%
\definecolor{currentfill}{rgb}{0.901975,0.231521,0.249182}%
\pgfsetfillcolor{currentfill}%
\pgfsetlinewidth{0.000000pt}%
\definecolor{currentstroke}{rgb}{0.000000,0.000000,0.000000}%
\pgfsetstrokecolor{currentstroke}%
\pgfsetdash{}{0pt}%
\pgfpathmoveto{\pgfqpoint{1.127432in}{0.683288in}}%
\pgfpathlineto{\pgfqpoint{1.140701in}{0.680850in}}%
\pgfpathlineto{\pgfqpoint{1.153970in}{0.680681in}}%
\pgfpathlineto{\pgfqpoint{1.167239in}{0.681528in}}%
\pgfpathlineto{\pgfqpoint{1.176780in}{0.690156in}}%
\pgfpathlineto{\pgfqpoint{1.177939in}{0.703083in}}%
\pgfpathlineto{\pgfqpoint{1.178097in}{0.716009in}}%
\pgfpathlineto{\pgfqpoint{1.177849in}{0.728936in}}%
\pgfpathlineto{\pgfqpoint{1.176875in}{0.741862in}}%
\pgfpathlineto{\pgfqpoint{1.170661in}{0.754789in}}%
\pgfpathlineto{\pgfqpoint{1.167239in}{0.756323in}}%
\pgfpathlineto{\pgfqpoint{1.153970in}{0.757843in}}%
\pgfpathlineto{\pgfqpoint{1.140701in}{0.757535in}}%
\pgfpathlineto{\pgfqpoint{1.129793in}{0.754789in}}%
\pgfpathlineto{\pgfqpoint{1.127432in}{0.753176in}}%
\pgfpathlineto{\pgfqpoint{1.123297in}{0.741862in}}%
\pgfpathlineto{\pgfqpoint{1.122369in}{0.728936in}}%
\pgfpathlineto{\pgfqpoint{1.122133in}{0.716009in}}%
\pgfpathlineto{\pgfqpoint{1.122283in}{0.703083in}}%
\pgfpathlineto{\pgfqpoint{1.123387in}{0.690156in}}%
\pgfpathclose%
\pgfpathmoveto{\pgfqpoint{1.140641in}{0.690156in}}%
\pgfpathlineto{\pgfqpoint{1.130167in}{0.703083in}}%
\pgfpathlineto{\pgfqpoint{1.128117in}{0.716009in}}%
\pgfpathlineto{\pgfqpoint{1.128839in}{0.728936in}}%
\pgfpathlineto{\pgfqpoint{1.133510in}{0.741862in}}%
\pgfpathlineto{\pgfqpoint{1.140701in}{0.748004in}}%
\pgfpathlineto{\pgfqpoint{1.153970in}{0.749382in}}%
\pgfpathlineto{\pgfqpoint{1.167239in}{0.742576in}}%
\pgfpathlineto{\pgfqpoint{1.167708in}{0.741862in}}%
\pgfpathlineto{\pgfqpoint{1.171418in}{0.728936in}}%
\pgfpathlineto{\pgfqpoint{1.171992in}{0.716009in}}%
\pgfpathlineto{\pgfqpoint{1.170363in}{0.703083in}}%
\pgfpathlineto{\pgfqpoint{1.167239in}{0.696113in}}%
\pgfpathlineto{\pgfqpoint{1.157852in}{0.690156in}}%
\pgfpathlineto{\pgfqpoint{1.153970in}{0.688950in}}%
\pgfpathlineto{\pgfqpoint{1.140701in}{0.690126in}}%
\pgfpathclose%
\pgfusepath{fill}%
\end{pgfscope}%
\begin{pgfscope}%
\pgfpathrectangle{\pgfqpoint{0.211875in}{0.211875in}}{\pgfqpoint{1.313625in}{1.279725in}}%
\pgfusepath{clip}%
\pgfsetbuttcap%
\pgfsetroundjoin%
\definecolor{currentfill}{rgb}{0.901975,0.231521,0.249182}%
\pgfsetfillcolor{currentfill}%
\pgfsetlinewidth{0.000000pt}%
\definecolor{currentstroke}{rgb}{0.000000,0.000000,0.000000}%
\pgfsetstrokecolor{currentstroke}%
\pgfsetdash{}{0pt}%
\pgfpathmoveto{\pgfqpoint{1.246852in}{0.682152in}}%
\pgfpathlineto{\pgfqpoint{1.260121in}{0.680534in}}%
\pgfpathlineto{\pgfqpoint{1.273390in}{0.680407in}}%
\pgfpathlineto{\pgfqpoint{1.286659in}{0.681154in}}%
\pgfpathlineto{\pgfqpoint{1.295400in}{0.690156in}}%
\pgfpathlineto{\pgfqpoint{1.296199in}{0.703083in}}%
\pgfpathlineto{\pgfqpoint{1.296284in}{0.716009in}}%
\pgfpathlineto{\pgfqpoint{1.296057in}{0.728936in}}%
\pgfpathlineto{\pgfqpoint{1.295230in}{0.741862in}}%
\pgfpathlineto{\pgfqpoint{1.289948in}{0.754789in}}%
\pgfpathlineto{\pgfqpoint{1.286659in}{0.756445in}}%
\pgfpathlineto{\pgfqpoint{1.273390in}{0.758079in}}%
\pgfpathlineto{\pgfqpoint{1.260121in}{0.757890in}}%
\pgfpathlineto{\pgfqpoint{1.246852in}{0.754917in}}%
\pgfpathlineto{\pgfqpoint{1.246664in}{0.754789in}}%
\pgfpathlineto{\pgfqpoint{1.241120in}{0.741862in}}%
\pgfpathlineto{\pgfqpoint{1.240238in}{0.728936in}}%
\pgfpathlineto{\pgfqpoint{1.240008in}{0.716009in}}%
\pgfpathlineto{\pgfqpoint{1.240134in}{0.703083in}}%
\pgfpathlineto{\pgfqpoint{1.241116in}{0.690156in}}%
\pgfpathclose%
\pgfpathmoveto{\pgfqpoint{1.258522in}{0.690156in}}%
\pgfpathlineto{\pgfqpoint{1.247558in}{0.703083in}}%
\pgfpathlineto{\pgfqpoint{1.246852in}{0.706725in}}%
\pgfpathlineto{\pgfqpoint{1.245828in}{0.716009in}}%
\pgfpathlineto{\pgfqpoint{1.246368in}{0.728936in}}%
\pgfpathlineto{\pgfqpoint{1.246852in}{0.731354in}}%
\pgfpathlineto{\pgfqpoint{1.251091in}{0.741862in}}%
\pgfpathlineto{\pgfqpoint{1.260121in}{0.748784in}}%
\pgfpathlineto{\pgfqpoint{1.273390in}{0.749473in}}%
\pgfpathlineto{\pgfqpoint{1.285933in}{0.741862in}}%
\pgfpathlineto{\pgfqpoint{1.286659in}{0.740685in}}%
\pgfpathlineto{\pgfqpoint{1.289744in}{0.728936in}}%
\pgfpathlineto{\pgfqpoint{1.290293in}{0.716009in}}%
\pgfpathlineto{\pgfqpoint{1.288758in}{0.703083in}}%
\pgfpathlineto{\pgfqpoint{1.286659in}{0.697725in}}%
\pgfpathlineto{\pgfqpoint{1.277112in}{0.690156in}}%
\pgfpathlineto{\pgfqpoint{1.273390in}{0.688845in}}%
\pgfpathlineto{\pgfqpoint{1.260121in}{0.689436in}}%
\pgfpathclose%
\pgfusepath{fill}%
\end{pgfscope}%
\begin{pgfscope}%
\pgfpathrectangle{\pgfqpoint{0.211875in}{0.211875in}}{\pgfqpoint{1.313625in}{1.279725in}}%
\pgfusepath{clip}%
\pgfsetbuttcap%
\pgfsetroundjoin%
\definecolor{currentfill}{rgb}{0.901975,0.231521,0.249182}%
\pgfsetfillcolor{currentfill}%
\pgfsetlinewidth{0.000000pt}%
\definecolor{currentstroke}{rgb}{0.000000,0.000000,0.000000}%
\pgfsetstrokecolor{currentstroke}%
\pgfsetdash{}{0pt}%
\pgfpathmoveto{\pgfqpoint{0.218434in}{0.767715in}}%
\pgfpathlineto{\pgfqpoint{0.225144in}{0.770195in}}%
\pgfpathlineto{\pgfqpoint{0.230962in}{0.780642in}}%
\pgfpathlineto{\pgfqpoint{0.232548in}{0.793568in}}%
\pgfpathlineto{\pgfqpoint{0.232757in}{0.806495in}}%
\pgfpathlineto{\pgfqpoint{0.231868in}{0.819421in}}%
\pgfpathlineto{\pgfqpoint{0.226772in}{0.832348in}}%
\pgfpathlineto{\pgfqpoint{0.225144in}{0.833633in}}%
\pgfpathlineto{\pgfqpoint{0.211875in}{0.836519in}}%
\pgfpathlineto{\pgfqpoint{0.211875in}{0.832348in}}%
\pgfpathlineto{\pgfqpoint{0.211875in}{0.827360in}}%
\pgfpathlineto{\pgfqpoint{0.222807in}{0.819421in}}%
\pgfpathlineto{\pgfqpoint{0.225144in}{0.813863in}}%
\pgfpathlineto{\pgfqpoint{0.226534in}{0.806495in}}%
\pgfpathlineto{\pgfqpoint{0.226194in}{0.793568in}}%
\pgfpathlineto{\pgfqpoint{0.225144in}{0.789293in}}%
\pgfpathlineto{\pgfqpoint{0.220127in}{0.780642in}}%
\pgfpathlineto{\pgfqpoint{0.211875in}{0.775548in}}%
\pgfpathlineto{\pgfqpoint{0.211875in}{0.767715in}}%
\pgfpathlineto{\pgfqpoint{0.211875in}{0.766478in}}%
\pgfpathclose%
\pgfusepath{fill}%
\end{pgfscope}%
\begin{pgfscope}%
\pgfpathrectangle{\pgfqpoint{0.211875in}{0.211875in}}{\pgfqpoint{1.313625in}{1.279725in}}%
\pgfusepath{clip}%
\pgfsetbuttcap%
\pgfsetroundjoin%
\definecolor{currentfill}{rgb}{0.901975,0.231521,0.249182}%
\pgfsetfillcolor{currentfill}%
\pgfsetlinewidth{0.000000pt}%
\definecolor{currentstroke}{rgb}{0.000000,0.000000,0.000000}%
\pgfsetstrokecolor{currentstroke}%
\pgfsetdash{}{0pt}%
\pgfpathmoveto{\pgfqpoint{0.835515in}{0.762111in}}%
\pgfpathlineto{\pgfqpoint{0.848784in}{0.761479in}}%
\pgfpathlineto{\pgfqpoint{0.862053in}{0.761637in}}%
\pgfpathlineto{\pgfqpoint{0.875322in}{0.763037in}}%
\pgfpathlineto{\pgfqpoint{0.881099in}{0.767715in}}%
\pgfpathlineto{\pgfqpoint{0.883122in}{0.780642in}}%
\pgfpathlineto{\pgfqpoint{0.883427in}{0.793568in}}%
\pgfpathlineto{\pgfqpoint{0.883330in}{0.806495in}}%
\pgfpathlineto{\pgfqpoint{0.882761in}{0.819421in}}%
\pgfpathlineto{\pgfqpoint{0.880181in}{0.832348in}}%
\pgfpathlineto{\pgfqpoint{0.875322in}{0.837107in}}%
\pgfpathlineto{\pgfqpoint{0.862053in}{0.839401in}}%
\pgfpathlineto{\pgfqpoint{0.848784in}{0.839556in}}%
\pgfpathlineto{\pgfqpoint{0.835515in}{0.837997in}}%
\pgfpathlineto{\pgfqpoint{0.829456in}{0.832348in}}%
\pgfpathlineto{\pgfqpoint{0.827451in}{0.819421in}}%
\pgfpathlineto{\pgfqpoint{0.827006in}{0.806495in}}%
\pgfpathlineto{\pgfqpoint{0.826898in}{0.793568in}}%
\pgfpathlineto{\pgfqpoint{0.827035in}{0.780642in}}%
\pgfpathlineto{\pgfqpoint{0.828150in}{0.767715in}}%
\pgfpathclose%
\pgfpathmoveto{\pgfqpoint{0.835411in}{0.780642in}}%
\pgfpathlineto{\pgfqpoint{0.833030in}{0.793568in}}%
\pgfpathlineto{\pgfqpoint{0.833012in}{0.806495in}}%
\pgfpathlineto{\pgfqpoint{0.835216in}{0.819421in}}%
\pgfpathlineto{\pgfqpoint{0.835515in}{0.820154in}}%
\pgfpathlineto{\pgfqpoint{0.848784in}{0.831023in}}%
\pgfpathlineto{\pgfqpoint{0.862053in}{0.830829in}}%
\pgfpathlineto{\pgfqpoint{0.875082in}{0.819421in}}%
\pgfpathlineto{\pgfqpoint{0.875322in}{0.818704in}}%
\pgfpathlineto{\pgfqpoint{0.877425in}{0.806495in}}%
\pgfpathlineto{\pgfqpoint{0.877400in}{0.793568in}}%
\pgfpathlineto{\pgfqpoint{0.875322in}{0.782186in}}%
\pgfpathlineto{\pgfqpoint{0.874744in}{0.780642in}}%
\pgfpathlineto{\pgfqpoint{0.862053in}{0.770516in}}%
\pgfpathlineto{\pgfqpoint{0.848784in}{0.770326in}}%
\pgfpathlineto{\pgfqpoint{0.835515in}{0.780403in}}%
\pgfpathclose%
\pgfusepath{fill}%
\end{pgfscope}%
\begin{pgfscope}%
\pgfpathrectangle{\pgfqpoint{0.211875in}{0.211875in}}{\pgfqpoint{1.313625in}{1.279725in}}%
\pgfusepath{clip}%
\pgfsetbuttcap%
\pgfsetroundjoin%
\definecolor{currentfill}{rgb}{0.901975,0.231521,0.249182}%
\pgfsetfillcolor{currentfill}%
\pgfsetlinewidth{0.000000pt}%
\definecolor{currentstroke}{rgb}{0.000000,0.000000,0.000000}%
\pgfsetstrokecolor{currentstroke}%
\pgfsetdash{}{0pt}%
\pgfpathmoveto{\pgfqpoint{0.954936in}{0.763984in}}%
\pgfpathlineto{\pgfqpoint{0.968205in}{0.762582in}}%
\pgfpathlineto{\pgfqpoint{0.981473in}{0.762811in}}%
\pgfpathlineto{\pgfqpoint{0.994742in}{0.765869in}}%
\pgfpathlineto{\pgfqpoint{0.996681in}{0.767715in}}%
\pgfpathlineto{\pgfqpoint{1.000145in}{0.780642in}}%
\pgfpathlineto{\pgfqpoint{1.000699in}{0.793568in}}%
\pgfpathlineto{\pgfqpoint{1.000616in}{0.806495in}}%
\pgfpathlineto{\pgfqpoint{0.999854in}{0.819421in}}%
\pgfpathlineto{\pgfqpoint{0.996276in}{0.832348in}}%
\pgfpathlineto{\pgfqpoint{0.994742in}{0.834102in}}%
\pgfpathlineto{\pgfqpoint{0.981473in}{0.838217in}}%
\pgfpathlineto{\pgfqpoint{0.968205in}{0.838503in}}%
\pgfpathlineto{\pgfqpoint{0.954936in}{0.836437in}}%
\pgfpathlineto{\pgfqpoint{0.949914in}{0.832348in}}%
\pgfpathlineto{\pgfqpoint{0.946546in}{0.819421in}}%
\pgfpathlineto{\pgfqpoint{0.945835in}{0.806495in}}%
\pgfpathlineto{\pgfqpoint{0.945744in}{0.793568in}}%
\pgfpathlineto{\pgfqpoint{0.946216in}{0.780642in}}%
\pgfpathlineto{\pgfqpoint{0.949301in}{0.767715in}}%
\pgfpathclose%
\pgfpathmoveto{\pgfqpoint{0.954690in}{0.780642in}}%
\pgfpathlineto{\pgfqpoint{0.951954in}{0.793568in}}%
\pgfpathlineto{\pgfqpoint{0.951918in}{0.806495in}}%
\pgfpathlineto{\pgfqpoint{0.954404in}{0.819421in}}%
\pgfpathlineto{\pgfqpoint{0.954936in}{0.820553in}}%
\pgfpathlineto{\pgfqpoint{0.968205in}{0.830005in}}%
\pgfpathlineto{\pgfqpoint{0.981473in}{0.829059in}}%
\pgfpathlineto{\pgfqpoint{0.991309in}{0.819421in}}%
\pgfpathlineto{\pgfqpoint{0.994742in}{0.806520in}}%
\pgfpathlineto{\pgfqpoint{0.994746in}{0.806495in}}%
\pgfpathlineto{\pgfqpoint{0.994742in}{0.805221in}}%
\pgfpathlineto{\pgfqpoint{0.994693in}{0.793568in}}%
\pgfpathlineto{\pgfqpoint{0.990899in}{0.780642in}}%
\pgfpathlineto{\pgfqpoint{0.981473in}{0.772221in}}%
\pgfpathlineto{\pgfqpoint{0.968205in}{0.771349in}}%
\pgfpathlineto{\pgfqpoint{0.954936in}{0.780155in}}%
\pgfpathclose%
\pgfusepath{fill}%
\end{pgfscope}%
\begin{pgfscope}%
\pgfpathrectangle{\pgfqpoint{0.211875in}{0.211875in}}{\pgfqpoint{1.313625in}{1.279725in}}%
\pgfusepath{clip}%
\pgfsetbuttcap%
\pgfsetroundjoin%
\definecolor{currentfill}{rgb}{0.901975,0.231521,0.249182}%
\pgfsetfillcolor{currentfill}%
\pgfsetlinewidth{0.000000pt}%
\definecolor{currentstroke}{rgb}{0.000000,0.000000,0.000000}%
\pgfsetstrokecolor{currentstroke}%
\pgfsetdash{}{0pt}%
\pgfpathmoveto{\pgfqpoint{1.074356in}{0.764597in}}%
\pgfpathlineto{\pgfqpoint{1.087625in}{0.763102in}}%
\pgfpathlineto{\pgfqpoint{1.100894in}{0.763476in}}%
\pgfpathlineto{\pgfqpoint{1.113398in}{0.767715in}}%
\pgfpathlineto{\pgfqpoint{1.114163in}{0.768506in}}%
\pgfpathlineto{\pgfqpoint{1.117709in}{0.780642in}}%
\pgfpathlineto{\pgfqpoint{1.118373in}{0.793568in}}%
\pgfpathlineto{\pgfqpoint{1.118297in}{0.806495in}}%
\pgfpathlineto{\pgfqpoint{1.117448in}{0.819421in}}%
\pgfpathlineto{\pgfqpoint{1.114163in}{0.831091in}}%
\pgfpathlineto{\pgfqpoint{1.113171in}{0.832348in}}%
\pgfpathlineto{\pgfqpoint{1.100894in}{0.837523in}}%
\pgfpathlineto{\pgfqpoint{1.087625in}{0.838013in}}%
\pgfpathlineto{\pgfqpoint{1.074356in}{0.836016in}}%
\pgfpathlineto{\pgfqpoint{1.069269in}{0.832348in}}%
\pgfpathlineto{\pgfqpoint{1.065108in}{0.819421in}}%
\pgfpathlineto{\pgfqpoint{1.064237in}{0.806495in}}%
\pgfpathlineto{\pgfqpoint{1.064155in}{0.793568in}}%
\pgfpathlineto{\pgfqpoint{1.064823in}{0.780642in}}%
\pgfpathlineto{\pgfqpoint{1.069026in}{0.767715in}}%
\pgfpathclose%
\pgfpathmoveto{\pgfqpoint{1.073452in}{0.780642in}}%
\pgfpathlineto{\pgfqpoint{1.070488in}{0.793568in}}%
\pgfpathlineto{\pgfqpoint{1.070441in}{0.806495in}}%
\pgfpathlineto{\pgfqpoint{1.073114in}{0.819421in}}%
\pgfpathlineto{\pgfqpoint{1.074356in}{0.821753in}}%
\pgfpathlineto{\pgfqpoint{1.087625in}{0.829593in}}%
\pgfpathlineto{\pgfqpoint{1.100894in}{0.827787in}}%
\pgfpathlineto{\pgfqpoint{1.108554in}{0.819421in}}%
\pgfpathlineto{\pgfqpoint{1.111887in}{0.806495in}}%
\pgfpathlineto{\pgfqpoint{1.111828in}{0.793568in}}%
\pgfpathlineto{\pgfqpoint{1.108128in}{0.780642in}}%
\pgfpathlineto{\pgfqpoint{1.100894in}{0.773432in}}%
\pgfpathlineto{\pgfqpoint{1.087625in}{0.771769in}}%
\pgfpathlineto{\pgfqpoint{1.074356in}{0.779073in}}%
\pgfpathclose%
\pgfusepath{fill}%
\end{pgfscope}%
\begin{pgfscope}%
\pgfpathrectangle{\pgfqpoint{0.211875in}{0.211875in}}{\pgfqpoint{1.313625in}{1.279725in}}%
\pgfusepath{clip}%
\pgfsetbuttcap%
\pgfsetroundjoin%
\definecolor{currentfill}{rgb}{0.901975,0.231521,0.249182}%
\pgfsetfillcolor{currentfill}%
\pgfsetlinewidth{0.000000pt}%
\definecolor{currentstroke}{rgb}{0.000000,0.000000,0.000000}%
\pgfsetstrokecolor{currentstroke}%
\pgfsetdash{}{0pt}%
\pgfpathmoveto{\pgfqpoint{1.193777in}{0.764329in}}%
\pgfpathlineto{\pgfqpoint{1.207045in}{0.763082in}}%
\pgfpathlineto{\pgfqpoint{1.220314in}{0.763576in}}%
\pgfpathlineto{\pgfqpoint{1.231376in}{0.767715in}}%
\pgfpathlineto{\pgfqpoint{1.233583in}{0.770845in}}%
\pgfpathlineto{\pgfqpoint{1.235790in}{0.780642in}}%
\pgfpathlineto{\pgfqpoint{1.236426in}{0.793568in}}%
\pgfpathlineto{\pgfqpoint{1.236348in}{0.806495in}}%
\pgfpathlineto{\pgfqpoint{1.235518in}{0.819421in}}%
\pgfpathlineto{\pgfqpoint{1.233583in}{0.828111in}}%
\pgfpathlineto{\pgfqpoint{1.231092in}{0.832348in}}%
\pgfpathlineto{\pgfqpoint{1.220314in}{0.837371in}}%
\pgfpathlineto{\pgfqpoint{1.207045in}{0.838044in}}%
\pgfpathlineto{\pgfqpoint{1.193777in}{0.836405in}}%
\pgfpathlineto{\pgfqpoint{1.187483in}{0.832348in}}%
\pgfpathlineto{\pgfqpoint{1.183099in}{0.819421in}}%
\pgfpathlineto{\pgfqpoint{1.182178in}{0.806495in}}%
\pgfpathlineto{\pgfqpoint{1.182095in}{0.793568in}}%
\pgfpathlineto{\pgfqpoint{1.182816in}{0.780642in}}%
\pgfpathlineto{\pgfqpoint{1.187295in}{0.767715in}}%
\pgfpathclose%
\pgfpathmoveto{\pgfqpoint{1.191666in}{0.780642in}}%
\pgfpathlineto{\pgfqpoint{1.188599in}{0.793568in}}%
\pgfpathlineto{\pgfqpoint{1.188549in}{0.806495in}}%
\pgfpathlineto{\pgfqpoint{1.191313in}{0.819421in}}%
\pgfpathlineto{\pgfqpoint{1.193777in}{0.823535in}}%
\pgfpathlineto{\pgfqpoint{1.207045in}{0.829747in}}%
\pgfpathlineto{\pgfqpoint{1.220314in}{0.827057in}}%
\pgfpathlineto{\pgfqpoint{1.226600in}{0.819421in}}%
\pgfpathlineto{\pgfqpoint{1.229723in}{0.806495in}}%
\pgfpathlineto{\pgfqpoint{1.229668in}{0.793568in}}%
\pgfpathlineto{\pgfqpoint{1.226204in}{0.780642in}}%
\pgfpathlineto{\pgfqpoint{1.220314in}{0.774103in}}%
\pgfpathlineto{\pgfqpoint{1.207045in}{0.771630in}}%
\pgfpathlineto{\pgfqpoint{1.193777in}{0.777396in}}%
\pgfpathclose%
\pgfusepath{fill}%
\end{pgfscope}%
\begin{pgfscope}%
\pgfpathrectangle{\pgfqpoint{0.211875in}{0.211875in}}{\pgfqpoint{1.313625in}{1.279725in}}%
\pgfusepath{clip}%
\pgfsetbuttcap%
\pgfsetroundjoin%
\definecolor{currentfill}{rgb}{0.901975,0.231521,0.249182}%
\pgfsetfillcolor{currentfill}%
\pgfsetlinewidth{0.000000pt}%
\definecolor{currentstroke}{rgb}{0.000000,0.000000,0.000000}%
\pgfsetstrokecolor{currentstroke}%
\pgfsetdash{}{0pt}%
\pgfpathmoveto{\pgfqpoint{1.313197in}{0.763398in}}%
\pgfpathlineto{\pgfqpoint{1.326466in}{0.762542in}}%
\pgfpathlineto{\pgfqpoint{1.339735in}{0.763018in}}%
\pgfpathlineto{\pgfqpoint{1.351093in}{0.767715in}}%
\pgfpathlineto{\pgfqpoint{1.353004in}{0.771830in}}%
\pgfpathlineto{\pgfqpoint{1.354384in}{0.780642in}}%
\pgfpathlineto{\pgfqpoint{1.354849in}{0.793568in}}%
\pgfpathlineto{\pgfqpoint{1.354759in}{0.806495in}}%
\pgfpathlineto{\pgfqpoint{1.354059in}{0.819421in}}%
\pgfpathlineto{\pgfqpoint{1.353004in}{0.825916in}}%
\pgfpathlineto{\pgfqpoint{1.350396in}{0.832348in}}%
\pgfpathlineto{\pgfqpoint{1.339735in}{0.837845in}}%
\pgfpathlineto{\pgfqpoint{1.326466in}{0.838578in}}%
\pgfpathlineto{\pgfqpoint{1.313197in}{0.837411in}}%
\pgfpathlineto{\pgfqpoint{1.304469in}{0.832348in}}%
\pgfpathlineto{\pgfqpoint{1.300455in}{0.819421in}}%
\pgfpathlineto{\pgfqpoint{1.299928in}{0.812261in}}%
\pgfpathlineto{\pgfqpoint{1.299628in}{0.806495in}}%
\pgfpathlineto{\pgfqpoint{1.299544in}{0.793568in}}%
\pgfpathlineto{\pgfqpoint{1.299928in}{0.783988in}}%
\pgfpathlineto{\pgfqpoint{1.300128in}{0.780642in}}%
\pgfpathlineto{\pgfqpoint{1.304013in}{0.767715in}}%
\pgfpathclose%
\pgfpathmoveto{\pgfqpoint{1.309272in}{0.780642in}}%
\pgfpathlineto{\pgfqpoint{1.306234in}{0.793568in}}%
\pgfpathlineto{\pgfqpoint{1.306191in}{0.806495in}}%
\pgfpathlineto{\pgfqpoint{1.308943in}{0.819421in}}%
\pgfpathlineto{\pgfqpoint{1.313197in}{0.825779in}}%
\pgfpathlineto{\pgfqpoint{1.326466in}{0.830447in}}%
\pgfpathlineto{\pgfqpoint{1.339735in}{0.826945in}}%
\pgfpathlineto{\pgfqpoint{1.345303in}{0.819421in}}%
\pgfpathlineto{\pgfqpoint{1.348124in}{0.806495in}}%
\pgfpathlineto{\pgfqpoint{1.348083in}{0.793568in}}%
\pgfpathlineto{\pgfqpoint{1.344978in}{0.780642in}}%
\pgfpathlineto{\pgfqpoint{1.339735in}{0.774155in}}%
\pgfpathlineto{\pgfqpoint{1.326466in}{0.770949in}}%
\pgfpathlineto{\pgfqpoint{1.313197in}{0.775253in}}%
\pgfpathclose%
\pgfusepath{fill}%
\end{pgfscope}%
\begin{pgfscope}%
\pgfpathrectangle{\pgfqpoint{0.211875in}{0.211875in}}{\pgfqpoint{1.313625in}{1.279725in}}%
\pgfusepath{clip}%
\pgfsetbuttcap%
\pgfsetroundjoin%
\definecolor{currentfill}{rgb}{0.901975,0.231521,0.249182}%
\pgfsetfillcolor{currentfill}%
\pgfsetlinewidth{0.000000pt}%
\definecolor{currentstroke}{rgb}{0.000000,0.000000,0.000000}%
\pgfsetstrokecolor{currentstroke}%
\pgfsetdash{}{0pt}%
\pgfpathmoveto{\pgfqpoint{1.419348in}{0.767142in}}%
\pgfpathlineto{\pgfqpoint{1.432617in}{0.761933in}}%
\pgfpathlineto{\pgfqpoint{1.445886in}{0.761482in}}%
\pgfpathlineto{\pgfqpoint{1.459155in}{0.761657in}}%
\pgfpathlineto{\pgfqpoint{1.472370in}{0.767715in}}%
\pgfpathlineto{\pgfqpoint{1.472424in}{0.767943in}}%
\pgfpathlineto{\pgfqpoint{1.473508in}{0.780642in}}%
\pgfpathlineto{\pgfqpoint{1.473647in}{0.793568in}}%
\pgfpathlineto{\pgfqpoint{1.473537in}{0.806495in}}%
\pgfpathlineto{\pgfqpoint{1.473085in}{0.819421in}}%
\pgfpathlineto{\pgfqpoint{1.472424in}{0.826013in}}%
\pgfpathlineto{\pgfqpoint{1.470926in}{0.832348in}}%
\pgfpathlineto{\pgfqpoint{1.459155in}{0.839079in}}%
\pgfpathlineto{\pgfqpoint{1.445886in}{0.839617in}}%
\pgfpathlineto{\pgfqpoint{1.432617in}{0.838919in}}%
\pgfpathlineto{\pgfqpoint{1.420082in}{0.832348in}}%
\pgfpathlineto{\pgfqpoint{1.419348in}{0.830638in}}%
\pgfpathlineto{\pgfqpoint{1.417399in}{0.819421in}}%
\pgfpathlineto{\pgfqpoint{1.416830in}{0.806495in}}%
\pgfpathlineto{\pgfqpoint{1.416733in}{0.793568in}}%
\pgfpathlineto{\pgfqpoint{1.417039in}{0.780642in}}%
\pgfpathlineto{\pgfqpoint{1.419064in}{0.767715in}}%
\pgfpathclose%
\pgfpathmoveto{\pgfqpoint{1.426179in}{0.780642in}}%
\pgfpathlineto{\pgfqpoint{1.423316in}{0.793568in}}%
\pgfpathlineto{\pgfqpoint{1.423288in}{0.806495in}}%
\pgfpathlineto{\pgfqpoint{1.425919in}{0.819421in}}%
\pgfpathlineto{\pgfqpoint{1.432617in}{0.828417in}}%
\pgfpathlineto{\pgfqpoint{1.445886in}{0.831697in}}%
\pgfpathlineto{\pgfqpoint{1.459155in}{0.827576in}}%
\pgfpathlineto{\pgfqpoint{1.464570in}{0.819421in}}%
\pgfpathlineto{\pgfqpoint{1.467003in}{0.806495in}}%
\pgfpathlineto{\pgfqpoint{1.466984in}{0.793568in}}%
\pgfpathlineto{\pgfqpoint{1.464354in}{0.780642in}}%
\pgfpathlineto{\pgfqpoint{1.459155in}{0.773456in}}%
\pgfpathlineto{\pgfqpoint{1.445886in}{0.769725in}}%
\pgfpathlineto{\pgfqpoint{1.432617in}{0.772717in}}%
\pgfpathclose%
\pgfusepath{fill}%
\end{pgfscope}%
\begin{pgfscope}%
\pgfpathrectangle{\pgfqpoint{0.211875in}{0.211875in}}{\pgfqpoint{1.313625in}{1.279725in}}%
\pgfusepath{clip}%
\pgfsetbuttcap%
\pgfsetroundjoin%
\definecolor{currentfill}{rgb}{0.901975,0.231521,0.249182}%
\pgfsetfillcolor{currentfill}%
\pgfsetlinewidth{0.000000pt}%
\definecolor{currentstroke}{rgb}{0.000000,0.000000,0.000000}%
\pgfsetstrokecolor{currentstroke}%
\pgfsetdash{}{0pt}%
\pgfpathmoveto{\pgfqpoint{0.264951in}{0.844672in}}%
\pgfpathlineto{\pgfqpoint{0.278220in}{0.844919in}}%
\pgfpathlineto{\pgfqpoint{0.280429in}{0.845274in}}%
\pgfpathlineto{\pgfqpoint{0.291489in}{0.852409in}}%
\pgfpathlineto{\pgfqpoint{0.292978in}{0.858201in}}%
\pgfpathlineto{\pgfqpoint{0.293950in}{0.871127in}}%
\pgfpathlineto{\pgfqpoint{0.294181in}{0.884054in}}%
\pgfpathlineto{\pgfqpoint{0.294054in}{0.896980in}}%
\pgfpathlineto{\pgfqpoint{0.293158in}{0.909907in}}%
\pgfpathlineto{\pgfqpoint{0.291489in}{0.915381in}}%
\pgfpathlineto{\pgfqpoint{0.278220in}{0.920472in}}%
\pgfpathlineto{\pgfqpoint{0.264951in}{0.920534in}}%
\pgfpathlineto{\pgfqpoint{0.251682in}{0.919012in}}%
\pgfpathlineto{\pgfqpoint{0.241470in}{0.909907in}}%
\pgfpathlineto{\pgfqpoint{0.239343in}{0.896980in}}%
\pgfpathlineto{\pgfqpoint{0.238947in}{0.884054in}}%
\pgfpathlineto{\pgfqpoint{0.239330in}{0.871127in}}%
\pgfpathlineto{\pgfqpoint{0.241124in}{0.858201in}}%
\pgfpathlineto{\pgfqpoint{0.251682in}{0.846462in}}%
\pgfpathlineto{\pgfqpoint{0.259290in}{0.845274in}}%
\pgfpathclose%
\pgfpathmoveto{\pgfqpoint{0.251420in}{0.858201in}}%
\pgfpathlineto{\pgfqpoint{0.246195in}{0.871127in}}%
\pgfpathlineto{\pgfqpoint{0.245212in}{0.884054in}}%
\pgfpathlineto{\pgfqpoint{0.246722in}{0.896980in}}%
\pgfpathlineto{\pgfqpoint{0.251682in}{0.907055in}}%
\pgfpathlineto{\pgfqpoint{0.256829in}{0.909907in}}%
\pgfpathlineto{\pgfqpoint{0.264951in}{0.912361in}}%
\pgfpathlineto{\pgfqpoint{0.278220in}{0.910121in}}%
\pgfpathlineto{\pgfqpoint{0.278528in}{0.909907in}}%
\pgfpathlineto{\pgfqpoint{0.286182in}{0.896980in}}%
\pgfpathlineto{\pgfqpoint{0.287655in}{0.884054in}}%
\pgfpathlineto{\pgfqpoint{0.286675in}{0.871127in}}%
\pgfpathlineto{\pgfqpoint{0.281319in}{0.858201in}}%
\pgfpathlineto{\pgfqpoint{0.278220in}{0.855454in}}%
\pgfpathlineto{\pgfqpoint{0.264951in}{0.853010in}}%
\pgfpathlineto{\pgfqpoint{0.251682in}{0.857910in}}%
\pgfpathclose%
\pgfusepath{fill}%
\end{pgfscope}%
\begin{pgfscope}%
\pgfpathrectangle{\pgfqpoint{0.211875in}{0.211875in}}{\pgfqpoint{1.313625in}{1.279725in}}%
\pgfusepath{clip}%
\pgfsetbuttcap%
\pgfsetroundjoin%
\definecolor{currentfill}{rgb}{0.901975,0.231521,0.249182}%
\pgfsetfillcolor{currentfill}%
\pgfsetlinewidth{0.000000pt}%
\definecolor{currentstroke}{rgb}{0.000000,0.000000,0.000000}%
\pgfsetstrokecolor{currentstroke}%
\pgfsetdash{}{0pt}%
\pgfpathmoveto{\pgfqpoint{0.782439in}{0.843214in}}%
\pgfpathlineto{\pgfqpoint{0.795708in}{0.842971in}}%
\pgfpathlineto{\pgfqpoint{0.808977in}{0.843564in}}%
\pgfpathlineto{\pgfqpoint{0.816562in}{0.845274in}}%
\pgfpathlineto{\pgfqpoint{0.822246in}{0.851093in}}%
\pgfpathlineto{\pgfqpoint{0.823724in}{0.858201in}}%
\pgfpathlineto{\pgfqpoint{0.824400in}{0.871127in}}%
\pgfpathlineto{\pgfqpoint{0.824452in}{0.884054in}}%
\pgfpathlineto{\pgfqpoint{0.824062in}{0.896980in}}%
\pgfpathlineto{\pgfqpoint{0.822447in}{0.909907in}}%
\pgfpathlineto{\pgfqpoint{0.822246in}{0.910582in}}%
\pgfpathlineto{\pgfqpoint{0.808977in}{0.919967in}}%
\pgfpathlineto{\pgfqpoint{0.795708in}{0.920767in}}%
\pgfpathlineto{\pgfqpoint{0.782439in}{0.920250in}}%
\pgfpathlineto{\pgfqpoint{0.769428in}{0.909907in}}%
\pgfpathlineto{\pgfqpoint{0.769170in}{0.908344in}}%
\pgfpathlineto{\pgfqpoint{0.768201in}{0.896980in}}%
\pgfpathlineto{\pgfqpoint{0.767900in}{0.884054in}}%
\pgfpathlineto{\pgfqpoint{0.767898in}{0.871127in}}%
\pgfpathlineto{\pgfqpoint{0.768279in}{0.858201in}}%
\pgfpathlineto{\pgfqpoint{0.769170in}{0.851110in}}%
\pgfpathlineto{\pgfqpoint{0.772858in}{0.845274in}}%
\pgfpathclose%
\pgfpathmoveto{\pgfqpoint{0.779225in}{0.858201in}}%
\pgfpathlineto{\pgfqpoint{0.774975in}{0.871127in}}%
\pgfpathlineto{\pgfqpoint{0.774324in}{0.884054in}}%
\pgfpathlineto{\pgfqpoint{0.775870in}{0.896980in}}%
\pgfpathlineto{\pgfqpoint{0.782439in}{0.909170in}}%
\pgfpathlineto{\pgfqpoint{0.784125in}{0.909907in}}%
\pgfpathlineto{\pgfqpoint{0.795708in}{0.912734in}}%
\pgfpathlineto{\pgfqpoint{0.808481in}{0.909907in}}%
\pgfpathlineto{\pgfqpoint{0.808977in}{0.909723in}}%
\pgfpathlineto{\pgfqpoint{0.816447in}{0.896980in}}%
\pgfpathlineto{\pgfqpoint{0.818110in}{0.884054in}}%
\pgfpathlineto{\pgfqpoint{0.817399in}{0.871127in}}%
\pgfpathlineto{\pgfqpoint{0.812848in}{0.858201in}}%
\pgfpathlineto{\pgfqpoint{0.808977in}{0.854385in}}%
\pgfpathlineto{\pgfqpoint{0.795708in}{0.851376in}}%
\pgfpathlineto{\pgfqpoint{0.782439in}{0.854784in}}%
\pgfpathclose%
\pgfusepath{fill}%
\end{pgfscope}%
\begin{pgfscope}%
\pgfpathrectangle{\pgfqpoint{0.211875in}{0.211875in}}{\pgfqpoint{1.313625in}{1.279725in}}%
\pgfusepath{clip}%
\pgfsetbuttcap%
\pgfsetroundjoin%
\definecolor{currentfill}{rgb}{0.901975,0.231521,0.249182}%
\pgfsetfillcolor{currentfill}%
\pgfsetlinewidth{0.000000pt}%
\definecolor{currentstroke}{rgb}{0.000000,0.000000,0.000000}%
\pgfsetstrokecolor{currentstroke}%
\pgfsetdash{}{0pt}%
\pgfpathmoveto{\pgfqpoint{0.901860in}{0.844879in}}%
\pgfpathlineto{\pgfqpoint{0.915129in}{0.844301in}}%
\pgfpathlineto{\pgfqpoint{0.926912in}{0.845274in}}%
\pgfpathlineto{\pgfqpoint{0.928398in}{0.845455in}}%
\pgfpathlineto{\pgfqpoint{0.940001in}{0.858201in}}%
\pgfpathlineto{\pgfqpoint{0.941294in}{0.871127in}}%
\pgfpathlineto{\pgfqpoint{0.941445in}{0.884054in}}%
\pgfpathlineto{\pgfqpoint{0.940839in}{0.896980in}}%
\pgfpathlineto{\pgfqpoint{0.938206in}{0.909907in}}%
\pgfpathlineto{\pgfqpoint{0.928398in}{0.918022in}}%
\pgfpathlineto{\pgfqpoint{0.915129in}{0.919428in}}%
\pgfpathlineto{\pgfqpoint{0.901860in}{0.918651in}}%
\pgfpathlineto{\pgfqpoint{0.889628in}{0.909907in}}%
\pgfpathlineto{\pgfqpoint{0.888591in}{0.906156in}}%
\pgfpathlineto{\pgfqpoint{0.887351in}{0.896980in}}%
\pgfpathlineto{\pgfqpoint{0.886860in}{0.884054in}}%
\pgfpathlineto{\pgfqpoint{0.886963in}{0.871127in}}%
\pgfpathlineto{\pgfqpoint{0.887953in}{0.858201in}}%
\pgfpathlineto{\pgfqpoint{0.888591in}{0.855241in}}%
\pgfpathlineto{\pgfqpoint{0.899851in}{0.845274in}}%
\pgfpathclose%
\pgfpathmoveto{\pgfqpoint{0.899323in}{0.858201in}}%
\pgfpathlineto{\pgfqpoint{0.894200in}{0.871127in}}%
\pgfpathlineto{\pgfqpoint{0.893402in}{0.884054in}}%
\pgfpathlineto{\pgfqpoint{0.895236in}{0.896980in}}%
\pgfpathlineto{\pgfqpoint{0.901860in}{0.908004in}}%
\pgfpathlineto{\pgfqpoint{0.907606in}{0.909907in}}%
\pgfpathlineto{\pgfqpoint{0.915129in}{0.911395in}}%
\pgfpathlineto{\pgfqpoint{0.920689in}{0.909907in}}%
\pgfpathlineto{\pgfqpoint{0.928398in}{0.906205in}}%
\pgfpathlineto{\pgfqpoint{0.933252in}{0.896980in}}%
\pgfpathlineto{\pgfqpoint{0.935028in}{0.884054in}}%
\pgfpathlineto{\pgfqpoint{0.934249in}{0.871127in}}%
\pgfpathlineto{\pgfqpoint{0.929311in}{0.858201in}}%
\pgfpathlineto{\pgfqpoint{0.928398in}{0.857197in}}%
\pgfpathlineto{\pgfqpoint{0.915129in}{0.852806in}}%
\pgfpathlineto{\pgfqpoint{0.901860in}{0.855783in}}%
\pgfpathclose%
\pgfusepath{fill}%
\end{pgfscope}%
\begin{pgfscope}%
\pgfpathrectangle{\pgfqpoint{0.211875in}{0.211875in}}{\pgfqpoint{1.313625in}{1.279725in}}%
\pgfusepath{clip}%
\pgfsetbuttcap%
\pgfsetroundjoin%
\definecolor{currentfill}{rgb}{0.901975,0.231521,0.249182}%
\pgfsetfillcolor{currentfill}%
\pgfsetlinewidth{0.000000pt}%
\definecolor{currentstroke}{rgb}{0.000000,0.000000,0.000000}%
\pgfsetstrokecolor{currentstroke}%
\pgfsetdash{}{0pt}%
\pgfpathmoveto{\pgfqpoint{1.034549in}{0.845115in}}%
\pgfpathlineto{\pgfqpoint{1.036265in}{0.845274in}}%
\pgfpathlineto{\pgfqpoint{1.047818in}{0.846974in}}%
\pgfpathlineto{\pgfqpoint{1.056946in}{0.858201in}}%
\pgfpathlineto{\pgfqpoint{1.058498in}{0.871127in}}%
\pgfpathlineto{\pgfqpoint{1.058695in}{0.884054in}}%
\pgfpathlineto{\pgfqpoint{1.058013in}{0.896980in}}%
\pgfpathlineto{\pgfqpoint{1.054997in}{0.909907in}}%
\pgfpathlineto{\pgfqpoint{1.047818in}{0.916531in}}%
\pgfpathlineto{\pgfqpoint{1.034549in}{0.918604in}}%
\pgfpathlineto{\pgfqpoint{1.021280in}{0.917864in}}%
\pgfpathlineto{\pgfqpoint{1.008927in}{0.909907in}}%
\pgfpathlineto{\pgfqpoint{1.008011in}{0.907608in}}%
\pgfpathlineto{\pgfqpoint{1.006062in}{0.896980in}}%
\pgfpathlineto{\pgfqpoint{1.005456in}{0.884054in}}%
\pgfpathlineto{\pgfqpoint{1.005623in}{0.871127in}}%
\pgfpathlineto{\pgfqpoint{1.006981in}{0.858201in}}%
\pgfpathlineto{\pgfqpoint{1.008011in}{0.854791in}}%
\pgfpathlineto{\pgfqpoint{1.021280in}{0.845749in}}%
\pgfpathlineto{\pgfqpoint{1.030558in}{0.845274in}}%
\pgfpathclose%
\pgfpathmoveto{\pgfqpoint{1.018774in}{0.858201in}}%
\pgfpathlineto{\pgfqpoint{1.012942in}{0.871127in}}%
\pgfpathlineto{\pgfqpoint{1.012025in}{0.884054in}}%
\pgfpathlineto{\pgfqpoint{1.014099in}{0.896980in}}%
\pgfpathlineto{\pgfqpoint{1.021280in}{0.907717in}}%
\pgfpathlineto{\pgfqpoint{1.030505in}{0.909907in}}%
\pgfpathlineto{\pgfqpoint{1.034549in}{0.910518in}}%
\pgfpathlineto{\pgfqpoint{1.036495in}{0.909907in}}%
\pgfpathlineto{\pgfqpoint{1.047818in}{0.903037in}}%
\pgfpathlineto{\pgfqpoint{1.050668in}{0.896980in}}%
\pgfpathlineto{\pgfqpoint{1.052486in}{0.884054in}}%
\pgfpathlineto{\pgfqpoint{1.051679in}{0.871127in}}%
\pgfpathlineto{\pgfqpoint{1.047818in}{0.860454in}}%
\pgfpathlineto{\pgfqpoint{1.045326in}{0.858201in}}%
\pgfpathlineto{\pgfqpoint{1.034549in}{0.853737in}}%
\pgfpathlineto{\pgfqpoint{1.021280in}{0.856055in}}%
\pgfpathclose%
\pgfusepath{fill}%
\end{pgfscope}%
\begin{pgfscope}%
\pgfpathrectangle{\pgfqpoint{0.211875in}{0.211875in}}{\pgfqpoint{1.313625in}{1.279725in}}%
\pgfusepath{clip}%
\pgfsetbuttcap%
\pgfsetroundjoin%
\definecolor{currentfill}{rgb}{0.901975,0.231521,0.249182}%
\pgfsetfillcolor{currentfill}%
\pgfsetlinewidth{0.000000pt}%
\definecolor{currentstroke}{rgb}{0.000000,0.000000,0.000000}%
\pgfsetstrokecolor{currentstroke}%
\pgfsetdash{}{0pt}%
\pgfpathmoveto{\pgfqpoint{1.273390in}{0.845198in}}%
\pgfpathlineto{\pgfqpoint{1.274061in}{0.845274in}}%
\pgfpathlineto{\pgfqpoint{1.286659in}{0.847994in}}%
\pgfpathlineto{\pgfqpoint{1.293111in}{0.858201in}}%
\pgfpathlineto{\pgfqpoint{1.294501in}{0.871127in}}%
\pgfpathlineto{\pgfqpoint{1.294672in}{0.884054in}}%
\pgfpathlineto{\pgfqpoint{1.294051in}{0.896980in}}%
\pgfpathlineto{\pgfqpoint{1.291298in}{0.909907in}}%
\pgfpathlineto{\pgfqpoint{1.286659in}{0.915361in}}%
\pgfpathlineto{\pgfqpoint{1.273390in}{0.918497in}}%
\pgfpathlineto{\pgfqpoint{1.260121in}{0.918196in}}%
\pgfpathlineto{\pgfqpoint{1.246852in}{0.912794in}}%
\pgfpathlineto{\pgfqpoint{1.245010in}{0.909907in}}%
\pgfpathlineto{\pgfqpoint{1.242206in}{0.896980in}}%
\pgfpathlineto{\pgfqpoint{1.241572in}{0.884054in}}%
\pgfpathlineto{\pgfqpoint{1.241755in}{0.871127in}}%
\pgfpathlineto{\pgfqpoint{1.243197in}{0.858201in}}%
\pgfpathlineto{\pgfqpoint{1.246852in}{0.850431in}}%
\pgfpathlineto{\pgfqpoint{1.260121in}{0.845470in}}%
\pgfpathlineto{\pgfqpoint{1.269328in}{0.845274in}}%
\pgfpathclose%
\pgfpathmoveto{\pgfqpoint{1.255319in}{0.858201in}}%
\pgfpathlineto{\pgfqpoint{1.248608in}{0.871127in}}%
\pgfpathlineto{\pgfqpoint{1.247545in}{0.884054in}}%
\pgfpathlineto{\pgfqpoint{1.249939in}{0.896980in}}%
\pgfpathlineto{\pgfqpoint{1.260121in}{0.909230in}}%
\pgfpathlineto{\pgfqpoint{1.269072in}{0.909907in}}%
\pgfpathlineto{\pgfqpoint{1.273390in}{0.910141in}}%
\pgfpathlineto{\pgfqpoint{1.273961in}{0.909907in}}%
\pgfpathlineto{\pgfqpoint{1.286659in}{0.897955in}}%
\pgfpathlineto{\pgfqpoint{1.287017in}{0.896980in}}%
\pgfpathlineto{\pgfqpoint{1.288734in}{0.884054in}}%
\pgfpathlineto{\pgfqpoint{1.287974in}{0.871127in}}%
\pgfpathlineto{\pgfqpoint{1.286659in}{0.866424in}}%
\pgfpathlineto{\pgfqpoint{1.280883in}{0.858201in}}%
\pgfpathlineto{\pgfqpoint{1.273390in}{0.854120in}}%
\pgfpathlineto{\pgfqpoint{1.260121in}{0.854886in}}%
\pgfpathclose%
\pgfusepath{fill}%
\end{pgfscope}%
\begin{pgfscope}%
\pgfpathrectangle{\pgfqpoint{0.211875in}{0.211875in}}{\pgfqpoint{1.313625in}{1.279725in}}%
\pgfusepath{clip}%
\pgfsetbuttcap%
\pgfsetroundjoin%
\definecolor{currentfill}{rgb}{0.901975,0.231521,0.249182}%
\pgfsetfillcolor{currentfill}%
\pgfsetlinewidth{0.000000pt}%
\definecolor{currentstroke}{rgb}{0.000000,0.000000,0.000000}%
\pgfsetstrokecolor{currentstroke}%
\pgfsetdash{}{0pt}%
\pgfpathmoveto{\pgfqpoint{1.379542in}{0.844520in}}%
\pgfpathlineto{\pgfqpoint{1.392811in}{0.844414in}}%
\pgfpathlineto{\pgfqpoint{1.399693in}{0.845274in}}%
\pgfpathlineto{\pgfqpoint{1.406080in}{0.846965in}}%
\pgfpathlineto{\pgfqpoint{1.412210in}{0.858201in}}%
\pgfpathlineto{\pgfqpoint{1.413199in}{0.871127in}}%
\pgfpathlineto{\pgfqpoint{1.413302in}{0.884054in}}%
\pgfpathlineto{\pgfqpoint{1.412812in}{0.896980in}}%
\pgfpathlineto{\pgfqpoint{1.410682in}{0.909907in}}%
\pgfpathlineto{\pgfqpoint{1.406080in}{0.916133in}}%
\pgfpathlineto{\pgfqpoint{1.392811in}{0.919265in}}%
\pgfpathlineto{\pgfqpoint{1.379542in}{0.919163in}}%
\pgfpathlineto{\pgfqpoint{1.366273in}{0.915640in}}%
\pgfpathlineto{\pgfqpoint{1.361997in}{0.909907in}}%
\pgfpathlineto{\pgfqpoint{1.359613in}{0.896980in}}%
\pgfpathlineto{\pgfqpoint{1.359064in}{0.884054in}}%
\pgfpathlineto{\pgfqpoint{1.359201in}{0.871127in}}%
\pgfpathlineto{\pgfqpoint{1.360371in}{0.858201in}}%
\pgfpathlineto{\pgfqpoint{1.366273in}{0.847553in}}%
\pgfpathlineto{\pgfqpoint{1.374161in}{0.845274in}}%
\pgfpathclose%
\pgfpathmoveto{\pgfqpoint{1.372039in}{0.858201in}}%
\pgfpathlineto{\pgfqpoint{1.366273in}{0.868309in}}%
\pgfpathlineto{\pgfqpoint{1.365579in}{0.871127in}}%
\pgfpathlineto{\pgfqpoint{1.364873in}{0.884054in}}%
\pgfpathlineto{\pgfqpoint{1.366273in}{0.895535in}}%
\pgfpathlineto{\pgfqpoint{1.366591in}{0.896980in}}%
\pgfpathlineto{\pgfqpoint{1.377973in}{0.909907in}}%
\pgfpathlineto{\pgfqpoint{1.379542in}{0.910641in}}%
\pgfpathlineto{\pgfqpoint{1.392811in}{0.910683in}}%
\pgfpathlineto{\pgfqpoint{1.394491in}{0.909907in}}%
\pgfpathlineto{\pgfqpoint{1.405744in}{0.896980in}}%
\pgfpathlineto{\pgfqpoint{1.406080in}{0.895411in}}%
\pgfpathlineto{\pgfqpoint{1.407435in}{0.884054in}}%
\pgfpathlineto{\pgfqpoint{1.406749in}{0.871127in}}%
\pgfpathlineto{\pgfqpoint{1.406080in}{0.868347in}}%
\pgfpathlineto{\pgfqpoint{1.400426in}{0.858201in}}%
\pgfpathlineto{\pgfqpoint{1.392811in}{0.853524in}}%
\pgfpathlineto{\pgfqpoint{1.379542in}{0.853574in}}%
\pgfpathclose%
\pgfusepath{fill}%
\end{pgfscope}%
\begin{pgfscope}%
\pgfpathrectangle{\pgfqpoint{0.211875in}{0.211875in}}{\pgfqpoint{1.313625in}{1.279725in}}%
\pgfusepath{clip}%
\pgfsetbuttcap%
\pgfsetroundjoin%
\definecolor{currentfill}{rgb}{0.901975,0.231521,0.249182}%
\pgfsetfillcolor{currentfill}%
\pgfsetlinewidth{0.000000pt}%
\definecolor{currentstroke}{rgb}{0.000000,0.000000,0.000000}%
\pgfsetstrokecolor{currentstroke}%
\pgfsetdash{}{0pt}%
\pgfpathmoveto{\pgfqpoint{1.485693in}{0.844463in}}%
\pgfpathlineto{\pgfqpoint{1.498962in}{0.843097in}}%
\pgfpathlineto{\pgfqpoint{1.512231in}{0.843001in}}%
\pgfpathlineto{\pgfqpoint{1.525500in}{0.844249in}}%
\pgfpathlineto{\pgfqpoint{1.525500in}{0.845274in}}%
\pgfpathlineto{\pgfqpoint{1.525500in}{0.858201in}}%
\pgfpathlineto{\pgfqpoint{1.525500in}{0.869097in}}%
\pgfpathlineto{\pgfqpoint{1.520761in}{0.858201in}}%
\pgfpathlineto{\pgfqpoint{1.512231in}{0.852331in}}%
\pgfpathlineto{\pgfqpoint{1.498962in}{0.851815in}}%
\pgfpathlineto{\pgfqpoint{1.487299in}{0.858201in}}%
\pgfpathlineto{\pgfqpoint{1.485693in}{0.860447in}}%
\pgfpathlineto{\pgfqpoint{1.482649in}{0.871127in}}%
\pgfpathlineto{\pgfqpoint{1.482023in}{0.884054in}}%
\pgfpathlineto{\pgfqpoint{1.483488in}{0.896980in}}%
\pgfpathlineto{\pgfqpoint{1.485693in}{0.902885in}}%
\pgfpathlineto{\pgfqpoint{1.493214in}{0.909907in}}%
\pgfpathlineto{\pgfqpoint{1.498962in}{0.912302in}}%
\pgfpathlineto{\pgfqpoint{1.512231in}{0.911777in}}%
\pgfpathlineto{\pgfqpoint{1.515851in}{0.909907in}}%
\pgfpathlineto{\pgfqpoint{1.524957in}{0.896980in}}%
\pgfpathlineto{\pgfqpoint{1.525500in}{0.893774in}}%
\pgfpathlineto{\pgfqpoint{1.525500in}{0.896980in}}%
\pgfpathlineto{\pgfqpoint{1.525500in}{0.909907in}}%
\pgfpathlineto{\pgfqpoint{1.525500in}{0.918428in}}%
\pgfpathlineto{\pgfqpoint{1.512231in}{0.920656in}}%
\pgfpathlineto{\pgfqpoint{1.498962in}{0.920614in}}%
\pgfpathlineto{\pgfqpoint{1.485693in}{0.918706in}}%
\pgfpathlineto{\pgfqpoint{1.478172in}{0.909907in}}%
\pgfpathlineto{\pgfqpoint{1.476531in}{0.896980in}}%
\pgfpathlineto{\pgfqpoint{1.476135in}{0.884054in}}%
\pgfpathlineto{\pgfqpoint{1.476188in}{0.871127in}}%
\pgfpathlineto{\pgfqpoint{1.476874in}{0.858201in}}%
\pgfpathlineto{\pgfqpoint{1.483410in}{0.845274in}}%
\pgfpathclose%
\pgfusepath{fill}%
\end{pgfscope}%
\begin{pgfscope}%
\pgfpathrectangle{\pgfqpoint{0.211875in}{0.211875in}}{\pgfqpoint{1.313625in}{1.279725in}}%
\pgfusepath{clip}%
\pgfsetbuttcap%
\pgfsetroundjoin%
\definecolor{currentfill}{rgb}{0.901975,0.231521,0.249182}%
\pgfsetfillcolor{currentfill}%
\pgfsetlinewidth{0.000000pt}%
\definecolor{currentstroke}{rgb}{0.000000,0.000000,0.000000}%
\pgfsetstrokecolor{currentstroke}%
\pgfsetdash{}{0pt}%
\pgfpathmoveto{\pgfqpoint{1.127432in}{0.852932in}}%
\pgfpathlineto{\pgfqpoint{1.140701in}{0.845920in}}%
\pgfpathlineto{\pgfqpoint{1.153970in}{0.845430in}}%
\pgfpathlineto{\pgfqpoint{1.167239in}{0.847884in}}%
\pgfpathlineto{\pgfqpoint{1.174675in}{0.858201in}}%
\pgfpathlineto{\pgfqpoint{1.176254in}{0.871127in}}%
\pgfpathlineto{\pgfqpoint{1.176457in}{0.884054in}}%
\pgfpathlineto{\pgfqpoint{1.175772in}{0.896980in}}%
\pgfpathlineto{\pgfqpoint{1.172722in}{0.909907in}}%
\pgfpathlineto{\pgfqpoint{1.167239in}{0.915589in}}%
\pgfpathlineto{\pgfqpoint{1.153970in}{0.918289in}}%
\pgfpathlineto{\pgfqpoint{1.140701in}{0.917744in}}%
\pgfpathlineto{\pgfqpoint{1.127432in}{0.910239in}}%
\pgfpathlineto{\pgfqpoint{1.127256in}{0.909907in}}%
\pgfpathlineto{\pgfqpoint{1.124349in}{0.896980in}}%
\pgfpathlineto{\pgfqpoint{1.123696in}{0.884054in}}%
\pgfpathlineto{\pgfqpoint{1.123889in}{0.871127in}}%
\pgfpathlineto{\pgfqpoint{1.125394in}{0.858201in}}%
\pgfpathclose%
\pgfpathmoveto{\pgfqpoint{1.137488in}{0.858201in}}%
\pgfpathlineto{\pgfqpoint{1.131121in}{0.871127in}}%
\pgfpathlineto{\pgfqpoint{1.130114in}{0.884054in}}%
\pgfpathlineto{\pgfqpoint{1.132376in}{0.896980in}}%
\pgfpathlineto{\pgfqpoint{1.140701in}{0.908157in}}%
\pgfpathlineto{\pgfqpoint{1.152136in}{0.909907in}}%
\pgfpathlineto{\pgfqpoint{1.153970in}{0.910097in}}%
\pgfpathlineto{\pgfqpoint{1.154495in}{0.909907in}}%
\pgfpathlineto{\pgfqpoint{1.167239in}{0.900259in}}%
\pgfpathlineto{\pgfqpoint{1.168609in}{0.896980in}}%
\pgfpathlineto{\pgfqpoint{1.170406in}{0.884054in}}%
\pgfpathlineto{\pgfqpoint{1.169606in}{0.871127in}}%
\pgfpathlineto{\pgfqpoint{1.167239in}{0.863725in}}%
\pgfpathlineto{\pgfqpoint{1.162377in}{0.858201in}}%
\pgfpathlineto{\pgfqpoint{1.153970in}{0.854180in}}%
\pgfpathlineto{\pgfqpoint{1.140701in}{0.855729in}}%
\pgfpathclose%
\pgfusepath{fill}%
\end{pgfscope}%
\begin{pgfscope}%
\pgfpathrectangle{\pgfqpoint{0.211875in}{0.211875in}}{\pgfqpoint{1.313625in}{1.279725in}}%
\pgfusepath{clip}%
\pgfsetbuttcap%
\pgfsetroundjoin%
\definecolor{currentfill}{rgb}{0.901975,0.231521,0.249182}%
\pgfsetfillcolor{currentfill}%
\pgfsetlinewidth{0.000000pt}%
\definecolor{currentstroke}{rgb}{0.000000,0.000000,0.000000}%
\pgfsetstrokecolor{currentstroke}%
\pgfsetdash{}{0pt}%
\pgfpathmoveto{\pgfqpoint{0.225144in}{0.929432in}}%
\pgfpathlineto{\pgfqpoint{0.231015in}{0.935760in}}%
\pgfpathlineto{\pgfqpoint{0.233443in}{0.948686in}}%
\pgfpathlineto{\pgfqpoint{0.233967in}{0.961613in}}%
\pgfpathlineto{\pgfqpoint{0.233810in}{0.974539in}}%
\pgfpathlineto{\pgfqpoint{0.232582in}{0.987466in}}%
\pgfpathlineto{\pgfqpoint{0.225144in}{0.998457in}}%
\pgfpathlineto{\pgfqpoint{0.212587in}{1.000392in}}%
\pgfpathlineto{\pgfqpoint{0.211875in}{1.000452in}}%
\pgfpathlineto{\pgfqpoint{0.211875in}{1.000392in}}%
\pgfpathlineto{\pgfqpoint{0.211875in}{0.991607in}}%
\pgfpathlineto{\pgfqpoint{0.220197in}{0.987466in}}%
\pgfpathlineto{\pgfqpoint{0.225144in}{0.981181in}}%
\pgfpathlineto{\pgfqpoint{0.227112in}{0.974539in}}%
\pgfpathlineto{\pgfqpoint{0.227868in}{0.961613in}}%
\pgfpathlineto{\pgfqpoint{0.226165in}{0.948686in}}%
\pgfpathlineto{\pgfqpoint{0.225144in}{0.946070in}}%
\pgfpathlineto{\pgfqpoint{0.212501in}{0.935760in}}%
\pgfpathlineto{\pgfqpoint{0.211875in}{0.935533in}}%
\pgfpathlineto{\pgfqpoint{0.211875in}{0.927224in}}%
\pgfpathclose%
\pgfusepath{fill}%
\end{pgfscope}%
\begin{pgfscope}%
\pgfpathrectangle{\pgfqpoint{0.211875in}{0.211875in}}{\pgfqpoint{1.313625in}{1.279725in}}%
\pgfusepath{clip}%
\pgfsetbuttcap%
\pgfsetroundjoin%
\definecolor{currentfill}{rgb}{0.901975,0.231521,0.249182}%
\pgfsetfillcolor{currentfill}%
\pgfsetlinewidth{0.000000pt}%
\definecolor{currentstroke}{rgb}{0.000000,0.000000,0.000000}%
\pgfsetstrokecolor{currentstroke}%
\pgfsetdash{}{0pt}%
\pgfpathmoveto{\pgfqpoint{0.716095in}{0.925236in}}%
\pgfpathlineto{\pgfqpoint{0.729364in}{0.924209in}}%
\pgfpathlineto{\pgfqpoint{0.742633in}{0.924354in}}%
\pgfpathlineto{\pgfqpoint{0.755902in}{0.925798in}}%
\pgfpathlineto{\pgfqpoint{0.764091in}{0.935760in}}%
\pgfpathlineto{\pgfqpoint{0.765191in}{0.948686in}}%
\pgfpathlineto{\pgfqpoint{0.765388in}{0.961613in}}%
\pgfpathlineto{\pgfqpoint{0.765183in}{0.974539in}}%
\pgfpathlineto{\pgfqpoint{0.764242in}{0.987466in}}%
\pgfpathlineto{\pgfqpoint{0.756186in}{1.000392in}}%
\pgfpathlineto{\pgfqpoint{0.755902in}{1.000497in}}%
\pgfpathlineto{\pgfqpoint{0.742633in}{1.002248in}}%
\pgfpathlineto{\pgfqpoint{0.729364in}{1.002357in}}%
\pgfpathlineto{\pgfqpoint{0.716095in}{1.000745in}}%
\pgfpathlineto{\pgfqpoint{0.715269in}{1.000392in}}%
\pgfpathlineto{\pgfqpoint{0.709601in}{0.987466in}}%
\pgfpathlineto{\pgfqpoint{0.709014in}{0.974539in}}%
\pgfpathlineto{\pgfqpoint{0.708871in}{0.961613in}}%
\pgfpathlineto{\pgfqpoint{0.708947in}{0.948686in}}%
\pgfpathlineto{\pgfqpoint{0.709483in}{0.935760in}}%
\pgfpathclose%
\pgfpathmoveto{\pgfqpoint{0.724326in}{0.935760in}}%
\pgfpathlineto{\pgfqpoint{0.716095in}{0.947996in}}%
\pgfpathlineto{\pgfqpoint{0.715911in}{0.948686in}}%
\pgfpathlineto{\pgfqpoint{0.714724in}{0.961613in}}%
\pgfpathlineto{\pgfqpoint{0.715432in}{0.974539in}}%
\pgfpathlineto{\pgfqpoint{0.716095in}{0.977572in}}%
\pgfpathlineto{\pgfqpoint{0.720777in}{0.987466in}}%
\pgfpathlineto{\pgfqpoint{0.729364in}{0.993210in}}%
\pgfpathlineto{\pgfqpoint{0.742633in}{0.993622in}}%
\pgfpathlineto{\pgfqpoint{0.753732in}{0.987466in}}%
\pgfpathlineto{\pgfqpoint{0.755902in}{0.984426in}}%
\pgfpathlineto{\pgfqpoint{0.758766in}{0.974539in}}%
\pgfpathlineto{\pgfqpoint{0.759522in}{0.961613in}}%
\pgfpathlineto{\pgfqpoint{0.758241in}{0.948686in}}%
\pgfpathlineto{\pgfqpoint{0.755902in}{0.942103in}}%
\pgfpathlineto{\pgfqpoint{0.749493in}{0.935760in}}%
\pgfpathlineto{\pgfqpoint{0.742633in}{0.932820in}}%
\pgfpathlineto{\pgfqpoint{0.729364in}{0.933193in}}%
\pgfpathclose%
\pgfusepath{fill}%
\end{pgfscope}%
\begin{pgfscope}%
\pgfpathrectangle{\pgfqpoint{0.211875in}{0.211875in}}{\pgfqpoint{1.313625in}{1.279725in}}%
\pgfusepath{clip}%
\pgfsetbuttcap%
\pgfsetroundjoin%
\definecolor{currentfill}{rgb}{0.901975,0.231521,0.249182}%
\pgfsetfillcolor{currentfill}%
\pgfsetlinewidth{0.000000pt}%
\definecolor{currentstroke}{rgb}{0.000000,0.000000,0.000000}%
\pgfsetstrokecolor{currentstroke}%
\pgfsetdash{}{0pt}%
\pgfpathmoveto{\pgfqpoint{0.835515in}{0.928444in}}%
\pgfpathlineto{\pgfqpoint{0.848784in}{0.925939in}}%
\pgfpathlineto{\pgfqpoint{0.862053in}{0.926102in}}%
\pgfpathlineto{\pgfqpoint{0.875322in}{0.929398in}}%
\pgfpathlineto{\pgfqpoint{0.879857in}{0.935760in}}%
\pgfpathlineto{\pgfqpoint{0.881884in}{0.948686in}}%
\pgfpathlineto{\pgfqpoint{0.882274in}{0.961613in}}%
\pgfpathlineto{\pgfqpoint{0.881973in}{0.974539in}}%
\pgfpathlineto{\pgfqpoint{0.880454in}{0.987466in}}%
\pgfpathlineto{\pgfqpoint{0.875322in}{0.996699in}}%
\pgfpathlineto{\pgfqpoint{0.863147in}{1.000392in}}%
\pgfpathlineto{\pgfqpoint{0.862053in}{1.000554in}}%
\pgfpathlineto{\pgfqpoint{0.848784in}{1.000711in}}%
\pgfpathlineto{\pgfqpoint{0.846211in}{1.000392in}}%
\pgfpathlineto{\pgfqpoint{0.835515in}{0.997676in}}%
\pgfpathlineto{\pgfqpoint{0.829664in}{0.987466in}}%
\pgfpathlineto{\pgfqpoint{0.828354in}{0.974539in}}%
\pgfpathlineto{\pgfqpoint{0.828089in}{0.961613in}}%
\pgfpathlineto{\pgfqpoint{0.828403in}{0.948686in}}%
\pgfpathlineto{\pgfqpoint{0.830100in}{0.935760in}}%
\pgfpathclose%
\pgfpathmoveto{\pgfqpoint{0.846291in}{0.935760in}}%
\pgfpathlineto{\pgfqpoint{0.835515in}{0.948342in}}%
\pgfpathlineto{\pgfqpoint{0.835407in}{0.948686in}}%
\pgfpathlineto{\pgfqpoint{0.833978in}{0.961613in}}%
\pgfpathlineto{\pgfqpoint{0.834810in}{0.974539in}}%
\pgfpathlineto{\pgfqpoint{0.835515in}{0.977286in}}%
\pgfpathlineto{\pgfqpoint{0.841619in}{0.987466in}}%
\pgfpathlineto{\pgfqpoint{0.848784in}{0.991742in}}%
\pgfpathlineto{\pgfqpoint{0.862053in}{0.991567in}}%
\pgfpathlineto{\pgfqpoint{0.868630in}{0.987466in}}%
\pgfpathlineto{\pgfqpoint{0.875322in}{0.975775in}}%
\pgfpathlineto{\pgfqpoint{0.875633in}{0.974539in}}%
\pgfpathlineto{\pgfqpoint{0.876482in}{0.961613in}}%
\pgfpathlineto{\pgfqpoint{0.875322in}{0.951034in}}%
\pgfpathlineto{\pgfqpoint{0.874861in}{0.948686in}}%
\pgfpathlineto{\pgfqpoint{0.864062in}{0.935760in}}%
\pgfpathlineto{\pgfqpoint{0.862053in}{0.934794in}}%
\pgfpathlineto{\pgfqpoint{0.848784in}{0.934626in}}%
\pgfpathclose%
\pgfusepath{fill}%
\end{pgfscope}%
\begin{pgfscope}%
\pgfpathrectangle{\pgfqpoint{0.211875in}{0.211875in}}{\pgfqpoint{1.313625in}{1.279725in}}%
\pgfusepath{clip}%
\pgfsetbuttcap%
\pgfsetroundjoin%
\definecolor{currentfill}{rgb}{0.901975,0.231521,0.249182}%
\pgfsetfillcolor{currentfill}%
\pgfsetlinewidth{0.000000pt}%
\definecolor{currentstroke}{rgb}{0.000000,0.000000,0.000000}%
\pgfsetstrokecolor{currentstroke}%
\pgfsetdash{}{0pt}%
\pgfpathmoveto{\pgfqpoint{0.954936in}{0.929873in}}%
\pgfpathlineto{\pgfqpoint{0.968205in}{0.927016in}}%
\pgfpathlineto{\pgfqpoint{0.981473in}{0.927372in}}%
\pgfpathlineto{\pgfqpoint{0.994742in}{0.932949in}}%
\pgfpathlineto{\pgfqpoint{0.996442in}{0.935760in}}%
\pgfpathlineto{\pgfqpoint{0.999067in}{0.948686in}}%
\pgfpathlineto{\pgfqpoint{0.999580in}{0.961613in}}%
\pgfpathlineto{\pgfqpoint{0.999216in}{0.974539in}}%
\pgfpathlineto{\pgfqpoint{0.997326in}{0.987466in}}%
\pgfpathlineto{\pgfqpoint{0.994742in}{0.992927in}}%
\pgfpathlineto{\pgfqpoint{0.981473in}{0.999233in}}%
\pgfpathlineto{\pgfqpoint{0.968205in}{0.999633in}}%
\pgfpathlineto{\pgfqpoint{0.954936in}{0.996330in}}%
\pgfpathlineto{\pgfqpoint{0.949090in}{0.987466in}}%
\pgfpathlineto{\pgfqpoint{0.947266in}{0.974539in}}%
\pgfpathlineto{\pgfqpoint{0.946914in}{0.961613in}}%
\pgfpathlineto{\pgfqpoint{0.947397in}{0.948686in}}%
\pgfpathlineto{\pgfqpoint{0.949911in}{0.935760in}}%
\pgfpathclose%
\pgfpathmoveto{\pgfqpoint{0.967507in}{0.935760in}}%
\pgfpathlineto{\pgfqpoint{0.954936in}{0.947454in}}%
\pgfpathlineto{\pgfqpoint{0.954488in}{0.948686in}}%
\pgfpathlineto{\pgfqpoint{0.952880in}{0.961613in}}%
\pgfpathlineto{\pgfqpoint{0.953804in}{0.974539in}}%
\pgfpathlineto{\pgfqpoint{0.954936in}{0.978364in}}%
\pgfpathlineto{\pgfqpoint{0.961765in}{0.987466in}}%
\pgfpathlineto{\pgfqpoint{0.968205in}{0.990870in}}%
\pgfpathlineto{\pgfqpoint{0.981473in}{0.989929in}}%
\pgfpathlineto{\pgfqpoint{0.985008in}{0.987466in}}%
\pgfpathlineto{\pgfqpoint{0.992175in}{0.974539in}}%
\pgfpathlineto{\pgfqpoint{0.993451in}{0.961613in}}%
\pgfpathlineto{\pgfqpoint{0.991228in}{0.948686in}}%
\pgfpathlineto{\pgfqpoint{0.981473in}{0.936560in}}%
\pgfpathlineto{\pgfqpoint{0.972601in}{0.935760in}}%
\pgfpathlineto{\pgfqpoint{0.968205in}{0.935478in}}%
\pgfpathclose%
\pgfusepath{fill}%
\end{pgfscope}%
\begin{pgfscope}%
\pgfpathrectangle{\pgfqpoint{0.211875in}{0.211875in}}{\pgfqpoint{1.313625in}{1.279725in}}%
\pgfusepath{clip}%
\pgfsetbuttcap%
\pgfsetroundjoin%
\definecolor{currentfill}{rgb}{0.901975,0.231521,0.249182}%
\pgfsetfillcolor{currentfill}%
\pgfsetlinewidth{0.000000pt}%
\definecolor{currentstroke}{rgb}{0.000000,0.000000,0.000000}%
\pgfsetstrokecolor{currentstroke}%
\pgfsetdash{}{0pt}%
\pgfpathmoveto{\pgfqpoint{1.074356in}{0.930145in}}%
\pgfpathlineto{\pgfqpoint{1.087625in}{0.927512in}}%
\pgfpathlineto{\pgfqpoint{1.100894in}{0.928140in}}%
\pgfpathlineto{\pgfqpoint{1.113681in}{0.935760in}}%
\pgfpathlineto{\pgfqpoint{1.114163in}{0.936746in}}%
\pgfpathlineto{\pgfqpoint{1.116697in}{0.948686in}}%
\pgfpathlineto{\pgfqpoint{1.117265in}{0.961613in}}%
\pgfpathlineto{\pgfqpoint{1.116874in}{0.974539in}}%
\pgfpathlineto{\pgfqpoint{1.114812in}{0.987466in}}%
\pgfpathlineto{\pgfqpoint{1.114163in}{0.989127in}}%
\pgfpathlineto{\pgfqpoint{1.100894in}{0.998415in}}%
\pgfpathlineto{\pgfqpoint{1.087625in}{0.999129in}}%
\pgfpathlineto{\pgfqpoint{1.074356in}{0.996136in}}%
\pgfpathlineto{\pgfqpoint{1.067867in}{0.987466in}}%
\pgfpathlineto{\pgfqpoint{1.065734in}{0.974539in}}%
\pgfpathlineto{\pgfqpoint{1.065329in}{0.961613in}}%
\pgfpathlineto{\pgfqpoint{1.065913in}{0.948686in}}%
\pgfpathlineto{\pgfqpoint{1.068911in}{0.935760in}}%
\pgfpathclose%
\pgfpathmoveto{\pgfqpoint{1.073140in}{0.948686in}}%
\pgfpathlineto{\pgfqpoint{1.071413in}{0.961613in}}%
\pgfpathlineto{\pgfqpoint{1.072400in}{0.974539in}}%
\pgfpathlineto{\pgfqpoint{1.074356in}{0.980358in}}%
\pgfpathlineto{\pgfqpoint{1.081011in}{0.987466in}}%
\pgfpathlineto{\pgfqpoint{1.087625in}{0.990530in}}%
\pgfpathlineto{\pgfqpoint{1.100894in}{0.988721in}}%
\pgfpathlineto{\pgfqpoint{1.102512in}{0.987466in}}%
\pgfpathlineto{\pgfqpoint{1.109455in}{0.974539in}}%
\pgfpathlineto{\pgfqpoint{1.110685in}{0.961613in}}%
\pgfpathlineto{\pgfqpoint{1.108532in}{0.948686in}}%
\pgfpathlineto{\pgfqpoint{1.100894in}{0.938085in}}%
\pgfpathlineto{\pgfqpoint{1.087625in}{0.935830in}}%
\pgfpathlineto{\pgfqpoint{1.074356in}{0.945752in}}%
\pgfpathclose%
\pgfusepath{fill}%
\end{pgfscope}%
\begin{pgfscope}%
\pgfpathrectangle{\pgfqpoint{0.211875in}{0.211875in}}{\pgfqpoint{1.313625in}{1.279725in}}%
\pgfusepath{clip}%
\pgfsetbuttcap%
\pgfsetroundjoin%
\definecolor{currentfill}{rgb}{0.901975,0.231521,0.249182}%
\pgfsetfillcolor{currentfill}%
\pgfsetlinewidth{0.000000pt}%
\definecolor{currentstroke}{rgb}{0.000000,0.000000,0.000000}%
\pgfsetstrokecolor{currentstroke}%
\pgfsetdash{}{0pt}%
\pgfpathmoveto{\pgfqpoint{1.193777in}{0.929601in}}%
\pgfpathlineto{\pgfqpoint{1.207045in}{0.927468in}}%
\pgfpathlineto{\pgfqpoint{1.220314in}{0.928353in}}%
\pgfpathlineto{\pgfqpoint{1.231511in}{0.935760in}}%
\pgfpathlineto{\pgfqpoint{1.233583in}{0.941587in}}%
\pgfpathlineto{\pgfqpoint{1.234750in}{0.948686in}}%
\pgfpathlineto{\pgfqpoint{1.235307in}{0.961613in}}%
\pgfpathlineto{\pgfqpoint{1.234921in}{0.974539in}}%
\pgfpathlineto{\pgfqpoint{1.233583in}{0.984165in}}%
\pgfpathlineto{\pgfqpoint{1.232730in}{0.987466in}}%
\pgfpathlineto{\pgfqpoint{1.220314in}{0.998167in}}%
\pgfpathlineto{\pgfqpoint{1.207045in}{0.999181in}}%
\pgfpathlineto{\pgfqpoint{1.193777in}{0.996760in}}%
\pgfpathlineto{\pgfqpoint{1.185958in}{0.987466in}}%
\pgfpathlineto{\pgfqpoint{1.183722in}{0.974539in}}%
\pgfpathlineto{\pgfqpoint{1.183297in}{0.961613in}}%
\pgfpathlineto{\pgfqpoint{1.183914in}{0.948686in}}%
\pgfpathlineto{\pgfqpoint{1.187061in}{0.935760in}}%
\pgfpathclose%
\pgfpathmoveto{\pgfqpoint{1.206744in}{0.935760in}}%
\pgfpathlineto{\pgfqpoint{1.193777in}{0.943458in}}%
\pgfpathlineto{\pgfqpoint{1.191331in}{0.948686in}}%
\pgfpathlineto{\pgfqpoint{1.189546in}{0.961613in}}%
\pgfpathlineto{\pgfqpoint{1.190566in}{0.974539in}}%
\pgfpathlineto{\pgfqpoint{1.193777in}{0.983023in}}%
\pgfpathlineto{\pgfqpoint{1.198995in}{0.987466in}}%
\pgfpathlineto{\pgfqpoint{1.207045in}{0.990686in}}%
\pgfpathlineto{\pgfqpoint{1.220314in}{0.987979in}}%
\pgfpathlineto{\pgfqpoint{1.220910in}{0.987466in}}%
\pgfpathlineto{\pgfqpoint{1.227434in}{0.974539in}}%
\pgfpathlineto{\pgfqpoint{1.228588in}{0.961613in}}%
\pgfpathlineto{\pgfqpoint{1.226570in}{0.948686in}}%
\pgfpathlineto{\pgfqpoint{1.220314in}{0.939010in}}%
\pgfpathlineto{\pgfqpoint{1.207621in}{0.935760in}}%
\pgfpathlineto{\pgfqpoint{1.207045in}{0.935667in}}%
\pgfpathclose%
\pgfusepath{fill}%
\end{pgfscope}%
\begin{pgfscope}%
\pgfpathrectangle{\pgfqpoint{0.211875in}{0.211875in}}{\pgfqpoint{1.313625in}{1.279725in}}%
\pgfusepath{clip}%
\pgfsetbuttcap%
\pgfsetroundjoin%
\definecolor{currentfill}{rgb}{0.901975,0.231521,0.249182}%
\pgfsetfillcolor{currentfill}%
\pgfsetlinewidth{0.000000pt}%
\definecolor{currentstroke}{rgb}{0.000000,0.000000,0.000000}%
\pgfsetstrokecolor{currentstroke}%
\pgfsetdash{}{0pt}%
\pgfpathmoveto{\pgfqpoint{1.313197in}{0.928440in}}%
\pgfpathlineto{\pgfqpoint{1.326466in}{0.926905in}}%
\pgfpathlineto{\pgfqpoint{1.339735in}{0.927921in}}%
\pgfpathlineto{\pgfqpoint{1.350400in}{0.935760in}}%
\pgfpathlineto{\pgfqpoint{1.353004in}{0.946816in}}%
\pgfpathlineto{\pgfqpoint{1.353220in}{0.948686in}}%
\pgfpathlineto{\pgfqpoint{1.353696in}{0.961613in}}%
\pgfpathlineto{\pgfqpoint{1.353349in}{0.974539in}}%
\pgfpathlineto{\pgfqpoint{1.353004in}{0.977970in}}%
\pgfpathlineto{\pgfqpoint{1.351332in}{0.987466in}}%
\pgfpathlineto{\pgfqpoint{1.339735in}{0.998577in}}%
\pgfpathlineto{\pgfqpoint{1.326466in}{0.999767in}}%
\pgfpathlineto{\pgfqpoint{1.313197in}{0.998009in}}%
\pgfpathlineto{\pgfqpoint{1.303293in}{0.987466in}}%
\pgfpathlineto{\pgfqpoint{1.301171in}{0.974539in}}%
\pgfpathlineto{\pgfqpoint{1.300763in}{0.961613in}}%
\pgfpathlineto{\pgfqpoint{1.301339in}{0.948686in}}%
\pgfpathlineto{\pgfqpoint{1.304284in}{0.935760in}}%
\pgfpathclose%
\pgfpathmoveto{\pgfqpoint{1.323763in}{0.935760in}}%
\pgfpathlineto{\pgfqpoint{1.313197in}{0.940695in}}%
\pgfpathlineto{\pgfqpoint{1.309008in}{0.948686in}}%
\pgfpathlineto{\pgfqpoint{1.307228in}{0.961613in}}%
\pgfpathlineto{\pgfqpoint{1.308250in}{0.974539in}}%
\pgfpathlineto{\pgfqpoint{1.313197in}{0.986224in}}%
\pgfpathlineto{\pgfqpoint{1.315054in}{0.987466in}}%
\pgfpathlineto{\pgfqpoint{1.326466in}{0.991321in}}%
\pgfpathlineto{\pgfqpoint{1.339735in}{0.987770in}}%
\pgfpathlineto{\pgfqpoint{1.340052in}{0.987466in}}%
\pgfpathlineto{\pgfqpoint{1.345984in}{0.974539in}}%
\pgfpathlineto{\pgfqpoint{1.347033in}{0.961613in}}%
\pgfpathlineto{\pgfqpoint{1.345208in}{0.948686in}}%
\pgfpathlineto{\pgfqpoint{1.339735in}{0.939247in}}%
\pgfpathlineto{\pgfqpoint{1.329855in}{0.935760in}}%
\pgfpathlineto{\pgfqpoint{1.326466in}{0.935057in}}%
\pgfpathclose%
\pgfusepath{fill}%
\end{pgfscope}%
\begin{pgfscope}%
\pgfpathrectangle{\pgfqpoint{0.211875in}{0.211875in}}{\pgfqpoint{1.313625in}{1.279725in}}%
\pgfusepath{clip}%
\pgfsetbuttcap%
\pgfsetroundjoin%
\definecolor{currentfill}{rgb}{0.901975,0.231521,0.249182}%
\pgfsetfillcolor{currentfill}%
\pgfsetlinewidth{0.000000pt}%
\definecolor{currentstroke}{rgb}{0.000000,0.000000,0.000000}%
\pgfsetstrokecolor{currentstroke}%
\pgfsetdash{}{0pt}%
\pgfpathmoveto{\pgfqpoint{1.432617in}{0.926777in}}%
\pgfpathlineto{\pgfqpoint{1.445886in}{0.925820in}}%
\pgfpathlineto{\pgfqpoint{1.459155in}{0.926706in}}%
\pgfpathlineto{\pgfqpoint{1.470215in}{0.935760in}}%
\pgfpathlineto{\pgfqpoint{1.472088in}{0.948686in}}%
\pgfpathlineto{\pgfqpoint{1.472424in}{0.961163in}}%
\pgfpathlineto{\pgfqpoint{1.472434in}{0.961613in}}%
\pgfpathlineto{\pgfqpoint{1.472424in}{0.962106in}}%
\pgfpathlineto{\pgfqpoint{1.472143in}{0.974539in}}%
\pgfpathlineto{\pgfqpoint{1.470697in}{0.987466in}}%
\pgfpathlineto{\pgfqpoint{1.459155in}{0.999785in}}%
\pgfpathlineto{\pgfqpoint{1.452912in}{1.000392in}}%
\pgfpathlineto{\pgfqpoint{1.445886in}{1.000856in}}%
\pgfpathlineto{\pgfqpoint{1.439232in}{1.000392in}}%
\pgfpathlineto{\pgfqpoint{1.432617in}{0.999768in}}%
\pgfpathlineto{\pgfqpoint{1.419763in}{0.987466in}}%
\pgfpathlineto{\pgfqpoint{1.419348in}{0.985285in}}%
\pgfpathlineto{\pgfqpoint{1.418186in}{0.974539in}}%
\pgfpathlineto{\pgfqpoint{1.417885in}{0.961613in}}%
\pgfpathlineto{\pgfqpoint{1.418274in}{0.948686in}}%
\pgfpathlineto{\pgfqpoint{1.419348in}{0.940109in}}%
\pgfpathlineto{\pgfqpoint{1.420457in}{0.935760in}}%
\pgfpathclose%
\pgfpathmoveto{\pgfqpoint{1.437549in}{0.935760in}}%
\pgfpathlineto{\pgfqpoint{1.432617in}{0.937525in}}%
\pgfpathlineto{\pgfqpoint{1.426090in}{0.948686in}}%
\pgfpathlineto{\pgfqpoint{1.424386in}{0.961613in}}%
\pgfpathlineto{\pgfqpoint{1.425374in}{0.974539in}}%
\pgfpathlineto{\pgfqpoint{1.430892in}{0.987466in}}%
\pgfpathlineto{\pgfqpoint{1.432617in}{0.989118in}}%
\pgfpathlineto{\pgfqpoint{1.445886in}{0.992438in}}%
\pgfpathlineto{\pgfqpoint{1.459155in}{0.988198in}}%
\pgfpathlineto{\pgfqpoint{1.459841in}{0.987466in}}%
\pgfpathlineto{\pgfqpoint{1.465019in}{0.974539in}}%
\pgfpathlineto{\pgfqpoint{1.465937in}{0.961613in}}%
\pgfpathlineto{\pgfqpoint{1.464360in}{0.948686in}}%
\pgfpathlineto{\pgfqpoint{1.459155in}{0.938657in}}%
\pgfpathlineto{\pgfqpoint{1.452905in}{0.935760in}}%
\pgfpathlineto{\pgfqpoint{1.445886in}{0.933978in}}%
\pgfpathclose%
\pgfusepath{fill}%
\end{pgfscope}%
\begin{pgfscope}%
\pgfpathrectangle{\pgfqpoint{0.211875in}{0.211875in}}{\pgfqpoint{1.313625in}{1.279725in}}%
\pgfusepath{clip}%
\pgfsetbuttcap%
\pgfsetroundjoin%
\definecolor{currentfill}{rgb}{0.901975,0.231521,0.249182}%
\pgfsetfillcolor{currentfill}%
\pgfsetlinewidth{0.000000pt}%
\definecolor{currentstroke}{rgb}{0.000000,0.000000,0.000000}%
\pgfsetstrokecolor{currentstroke}%
\pgfsetdash{}{0pt}%
\pgfpathmoveto{\pgfqpoint{0.251682in}{1.007194in}}%
\pgfpathlineto{\pgfqpoint{0.264951in}{1.005940in}}%
\pgfpathlineto{\pgfqpoint{0.278220in}{1.005856in}}%
\pgfpathlineto{\pgfqpoint{0.291489in}{1.008546in}}%
\pgfpathlineto{\pgfqpoint{0.293742in}{1.013319in}}%
\pgfpathlineto{\pgfqpoint{0.294732in}{1.026245in}}%
\pgfpathlineto{\pgfqpoint{0.294929in}{1.039172in}}%
\pgfpathlineto{\pgfqpoint{0.294952in}{1.052098in}}%
\pgfpathlineto{\pgfqpoint{0.294816in}{1.065025in}}%
\pgfpathlineto{\pgfqpoint{0.293896in}{1.077952in}}%
\pgfpathlineto{\pgfqpoint{0.291489in}{1.082127in}}%
\pgfpathlineto{\pgfqpoint{0.278220in}{1.083974in}}%
\pgfpathlineto{\pgfqpoint{0.264951in}{1.083824in}}%
\pgfpathlineto{\pgfqpoint{0.251682in}{1.082658in}}%
\pgfpathlineto{\pgfqpoint{0.242642in}{1.077952in}}%
\pgfpathlineto{\pgfqpoint{0.238881in}{1.065025in}}%
\pgfpathlineto{\pgfqpoint{0.238413in}{1.057064in}}%
\pgfpathlineto{\pgfqpoint{0.238212in}{1.052098in}}%
\pgfpathlineto{\pgfqpoint{0.238210in}{1.039172in}}%
\pgfpathlineto{\pgfqpoint{0.238413in}{1.034024in}}%
\pgfpathlineto{\pgfqpoint{0.238840in}{1.026245in}}%
\pgfpathlineto{\pgfqpoint{0.241938in}{1.013319in}}%
\pgfpathclose%
\pgfpathmoveto{\pgfqpoint{0.247050in}{1.026245in}}%
\pgfpathlineto{\pgfqpoint{0.244588in}{1.039172in}}%
\pgfpathlineto{\pgfqpoint{0.244702in}{1.052098in}}%
\pgfpathlineto{\pgfqpoint{0.247617in}{1.065025in}}%
\pgfpathlineto{\pgfqpoint{0.251682in}{1.070869in}}%
\pgfpathlineto{\pgfqpoint{0.264951in}{1.075634in}}%
\pgfpathlineto{\pgfqpoint{0.278220in}{1.073390in}}%
\pgfpathlineto{\pgfqpoint{0.285425in}{1.065025in}}%
\pgfpathlineto{\pgfqpoint{0.288274in}{1.052098in}}%
\pgfpathlineto{\pgfqpoint{0.288378in}{1.039172in}}%
\pgfpathlineto{\pgfqpoint{0.285971in}{1.026245in}}%
\pgfpathlineto{\pgfqpoint{0.278220in}{1.016280in}}%
\pgfpathlineto{\pgfqpoint{0.264951in}{1.013893in}}%
\pgfpathlineto{\pgfqpoint{0.251682in}{1.018912in}}%
\pgfpathclose%
\pgfusepath{fill}%
\end{pgfscope}%
\begin{pgfscope}%
\pgfpathrectangle{\pgfqpoint{0.211875in}{0.211875in}}{\pgfqpoint{1.313625in}{1.279725in}}%
\pgfusepath{clip}%
\pgfsetbuttcap%
\pgfsetroundjoin%
\definecolor{currentfill}{rgb}{0.901975,0.231521,0.249182}%
\pgfsetfillcolor{currentfill}%
\pgfsetlinewidth{0.000000pt}%
\definecolor{currentstroke}{rgb}{0.000000,0.000000,0.000000}%
\pgfsetstrokecolor{currentstroke}%
\pgfsetdash{}{0pt}%
\pgfpathmoveto{\pgfqpoint{0.649750in}{1.013239in}}%
\pgfpathlineto{\pgfqpoint{0.663019in}{1.004973in}}%
\pgfpathlineto{\pgfqpoint{0.676288in}{1.004959in}}%
\pgfpathlineto{\pgfqpoint{0.689557in}{1.005390in}}%
\pgfpathlineto{\pgfqpoint{0.702826in}{1.008760in}}%
\pgfpathlineto{\pgfqpoint{0.705310in}{1.013319in}}%
\pgfpathlineto{\pgfqpoint{0.706724in}{1.026245in}}%
\pgfpathlineto{\pgfqpoint{0.706991in}{1.039172in}}%
\pgfpathlineto{\pgfqpoint{0.706941in}{1.052098in}}%
\pgfpathlineto{\pgfqpoint{0.706511in}{1.065025in}}%
\pgfpathlineto{\pgfqpoint{0.704187in}{1.077952in}}%
\pgfpathlineto{\pgfqpoint{0.702826in}{1.079937in}}%
\pgfpathlineto{\pgfqpoint{0.689557in}{1.083825in}}%
\pgfpathlineto{\pgfqpoint{0.676288in}{1.084300in}}%
\pgfpathlineto{\pgfqpoint{0.663019in}{1.084168in}}%
\pgfpathlineto{\pgfqpoint{0.650531in}{1.077952in}}%
\pgfpathlineto{\pgfqpoint{0.649750in}{1.070157in}}%
\pgfpathlineto{\pgfqpoint{0.649557in}{1.065025in}}%
\pgfpathlineto{\pgfqpoint{0.649382in}{1.052098in}}%
\pgfpathlineto{\pgfqpoint{0.649343in}{1.039172in}}%
\pgfpathlineto{\pgfqpoint{0.649395in}{1.026245in}}%
\pgfpathlineto{\pgfqpoint{0.649744in}{1.013319in}}%
\pgfpathclose%
\pgfpathmoveto{\pgfqpoint{0.674386in}{1.013319in}}%
\pgfpathlineto{\pgfqpoint{0.663019in}{1.017401in}}%
\pgfpathlineto{\pgfqpoint{0.657782in}{1.026245in}}%
\pgfpathlineto{\pgfqpoint{0.655791in}{1.039172in}}%
\pgfpathlineto{\pgfqpoint{0.655949in}{1.052098in}}%
\pgfpathlineto{\pgfqpoint{0.658520in}{1.065025in}}%
\pgfpathlineto{\pgfqpoint{0.663019in}{1.071754in}}%
\pgfpathlineto{\pgfqpoint{0.676288in}{1.076165in}}%
\pgfpathlineto{\pgfqpoint{0.689557in}{1.073853in}}%
\pgfpathlineto{\pgfqpoint{0.697512in}{1.065025in}}%
\pgfpathlineto{\pgfqpoint{0.700591in}{1.052098in}}%
\pgfpathlineto{\pgfqpoint{0.700778in}{1.039172in}}%
\pgfpathlineto{\pgfqpoint{0.698377in}{1.026245in}}%
\pgfpathlineto{\pgfqpoint{0.689557in}{1.015246in}}%
\pgfpathlineto{\pgfqpoint{0.679255in}{1.013319in}}%
\pgfpathlineto{\pgfqpoint{0.676288in}{1.012935in}}%
\pgfpathclose%
\pgfusepath{fill}%
\end{pgfscope}%
\begin{pgfscope}%
\pgfpathrectangle{\pgfqpoint{0.211875in}{0.211875in}}{\pgfqpoint{1.313625in}{1.279725in}}%
\pgfusepath{clip}%
\pgfsetbuttcap%
\pgfsetroundjoin%
\definecolor{currentfill}{rgb}{0.901975,0.231521,0.249182}%
\pgfsetfillcolor{currentfill}%
\pgfsetlinewidth{0.000000pt}%
\definecolor{currentstroke}{rgb}{0.000000,0.000000,0.000000}%
\pgfsetstrokecolor{currentstroke}%
\pgfsetdash{}{0pt}%
\pgfpathmoveto{\pgfqpoint{0.782439in}{1.007521in}}%
\pgfpathlineto{\pgfqpoint{0.795708in}{1.006809in}}%
\pgfpathlineto{\pgfqpoint{0.808977in}{1.007751in}}%
\pgfpathlineto{\pgfqpoint{0.819340in}{1.013319in}}%
\pgfpathlineto{\pgfqpoint{0.822246in}{1.021090in}}%
\pgfpathlineto{\pgfqpoint{0.823028in}{1.026245in}}%
\pgfpathlineto{\pgfqpoint{0.823668in}{1.039172in}}%
\pgfpathlineto{\pgfqpoint{0.823595in}{1.052098in}}%
\pgfpathlineto{\pgfqpoint{0.822705in}{1.065025in}}%
\pgfpathlineto{\pgfqpoint{0.822246in}{1.067634in}}%
\pgfpathlineto{\pgfqpoint{0.817112in}{1.077952in}}%
\pgfpathlineto{\pgfqpoint{0.808977in}{1.081476in}}%
\pgfpathlineto{\pgfqpoint{0.795708in}{1.082472in}}%
\pgfpathlineto{\pgfqpoint{0.782439in}{1.081683in}}%
\pgfpathlineto{\pgfqpoint{0.774105in}{1.077952in}}%
\pgfpathlineto{\pgfqpoint{0.769520in}{1.065025in}}%
\pgfpathlineto{\pgfqpoint{0.769170in}{1.060271in}}%
\pgfpathlineto{\pgfqpoint{0.768766in}{1.052098in}}%
\pgfpathlineto{\pgfqpoint{0.768700in}{1.039172in}}%
\pgfpathlineto{\pgfqpoint{0.769170in}{1.026860in}}%
\pgfpathlineto{\pgfqpoint{0.769204in}{1.026245in}}%
\pgfpathlineto{\pgfqpoint{0.772155in}{1.013319in}}%
\pgfpathclose%
\pgfpathmoveto{\pgfqpoint{0.777881in}{1.026245in}}%
\pgfpathlineto{\pgfqpoint{0.775339in}{1.039172in}}%
\pgfpathlineto{\pgfqpoint{0.775531in}{1.052098in}}%
\pgfpathlineto{\pgfqpoint{0.778785in}{1.065025in}}%
\pgfpathlineto{\pgfqpoint{0.782439in}{1.069907in}}%
\pgfpathlineto{\pgfqpoint{0.795708in}{1.074100in}}%
\pgfpathlineto{\pgfqpoint{0.808977in}{1.070470in}}%
\pgfpathlineto{\pgfqpoint{0.813392in}{1.065025in}}%
\pgfpathlineto{\pgfqpoint{0.816859in}{1.052098in}}%
\pgfpathlineto{\pgfqpoint{0.817063in}{1.039172in}}%
\pgfpathlineto{\pgfqpoint{0.814344in}{1.026245in}}%
\pgfpathlineto{\pgfqpoint{0.808977in}{1.018800in}}%
\pgfpathlineto{\pgfqpoint{0.795708in}{1.015030in}}%
\pgfpathlineto{\pgfqpoint{0.782439in}{1.019377in}}%
\pgfpathclose%
\pgfusepath{fill}%
\end{pgfscope}%
\begin{pgfscope}%
\pgfpathrectangle{\pgfqpoint{0.211875in}{0.211875in}}{\pgfqpoint{1.313625in}{1.279725in}}%
\pgfusepath{clip}%
\pgfsetbuttcap%
\pgfsetroundjoin%
\definecolor{currentfill}{rgb}{0.901975,0.231521,0.249182}%
\pgfsetfillcolor{currentfill}%
\pgfsetlinewidth{0.000000pt}%
\definecolor{currentstroke}{rgb}{0.000000,0.000000,0.000000}%
\pgfsetstrokecolor{currentstroke}%
\pgfsetdash{}{0pt}%
\pgfpathmoveto{\pgfqpoint{0.901860in}{1.009053in}}%
\pgfpathlineto{\pgfqpoint{0.915129in}{1.008129in}}%
\pgfpathlineto{\pgfqpoint{0.928398in}{1.009730in}}%
\pgfpathlineto{\pgfqpoint{0.934425in}{1.013319in}}%
\pgfpathlineto{\pgfqpoint{0.939560in}{1.026245in}}%
\pgfpathlineto{\pgfqpoint{0.940555in}{1.039172in}}%
\pgfpathlineto{\pgfqpoint{0.940457in}{1.052098in}}%
\pgfpathlineto{\pgfqpoint{0.939117in}{1.065025in}}%
\pgfpathlineto{\pgfqpoint{0.931639in}{1.077952in}}%
\pgfpathlineto{\pgfqpoint{0.928398in}{1.079503in}}%
\pgfpathlineto{\pgfqpoint{0.915129in}{1.081166in}}%
\pgfpathlineto{\pgfqpoint{0.901860in}{1.080192in}}%
\pgfpathlineto{\pgfqpoint{0.896320in}{1.077952in}}%
\pgfpathlineto{\pgfqpoint{0.888899in}{1.065025in}}%
\pgfpathlineto{\pgfqpoint{0.888591in}{1.062585in}}%
\pgfpathlineto{\pgfqpoint{0.887751in}{1.052098in}}%
\pgfpathlineto{\pgfqpoint{0.887667in}{1.039172in}}%
\pgfpathlineto{\pgfqpoint{0.888478in}{1.026245in}}%
\pgfpathlineto{\pgfqpoint{0.888591in}{1.025526in}}%
\pgfpathlineto{\pgfqpoint{0.893459in}{1.013319in}}%
\pgfpathclose%
\pgfpathmoveto{\pgfqpoint{0.897498in}{1.026245in}}%
\pgfpathlineto{\pgfqpoint{0.894485in}{1.039172in}}%
\pgfpathlineto{\pgfqpoint{0.894708in}{1.052098in}}%
\pgfpathlineto{\pgfqpoint{0.898547in}{1.065025in}}%
\pgfpathlineto{\pgfqpoint{0.901860in}{1.068995in}}%
\pgfpathlineto{\pgfqpoint{0.915129in}{1.072583in}}%
\pgfpathlineto{\pgfqpoint{0.928398in}{1.067355in}}%
\pgfpathlineto{\pgfqpoint{0.930093in}{1.065025in}}%
\pgfpathlineto{\pgfqpoint{0.933788in}{1.052098in}}%
\pgfpathlineto{\pgfqpoint{0.934002in}{1.039172in}}%
\pgfpathlineto{\pgfqpoint{0.931095in}{1.026245in}}%
\pgfpathlineto{\pgfqpoint{0.928398in}{1.022072in}}%
\pgfpathlineto{\pgfqpoint{0.915129in}{1.016629in}}%
\pgfpathlineto{\pgfqpoint{0.901860in}{1.020357in}}%
\pgfpathclose%
\pgfusepath{fill}%
\end{pgfscope}%
\begin{pgfscope}%
\pgfpathrectangle{\pgfqpoint{0.211875in}{0.211875in}}{\pgfqpoint{1.313625in}{1.279725in}}%
\pgfusepath{clip}%
\pgfsetbuttcap%
\pgfsetroundjoin%
\definecolor{currentfill}{rgb}{0.901975,0.231521,0.249182}%
\pgfsetfillcolor{currentfill}%
\pgfsetlinewidth{0.000000pt}%
\definecolor{currentstroke}{rgb}{0.000000,0.000000,0.000000}%
\pgfsetstrokecolor{currentstroke}%
\pgfsetdash{}{0pt}%
\pgfpathmoveto{\pgfqpoint{1.021280in}{1.009789in}}%
\pgfpathlineto{\pgfqpoint{1.034549in}{1.008946in}}%
\pgfpathlineto{\pgfqpoint{1.047818in}{1.011278in}}%
\pgfpathlineto{\pgfqpoint{1.050896in}{1.013319in}}%
\pgfpathlineto{\pgfqpoint{1.056702in}{1.026245in}}%
\pgfpathlineto{\pgfqpoint{1.057822in}{1.039172in}}%
\pgfpathlineto{\pgfqpoint{1.057718in}{1.052098in}}%
\pgfpathlineto{\pgfqpoint{1.056227in}{1.065025in}}%
\pgfpathlineto{\pgfqpoint{1.047818in}{1.077952in}}%
\pgfpathlineto{\pgfqpoint{1.047818in}{1.077952in}}%
\pgfpathlineto{\pgfqpoint{1.034549in}{1.080357in}}%
\pgfpathlineto{\pgfqpoint{1.021280in}{1.079481in}}%
\pgfpathlineto{\pgfqpoint{1.017105in}{1.077952in}}%
\pgfpathlineto{\pgfqpoint{1.008011in}{1.066285in}}%
\pgfpathlineto{\pgfqpoint{1.007697in}{1.065025in}}%
\pgfpathlineto{\pgfqpoint{1.006362in}{1.052098in}}%
\pgfpathlineto{\pgfqpoint{1.006267in}{1.039172in}}%
\pgfpathlineto{\pgfqpoint{1.007266in}{1.026245in}}%
\pgfpathlineto{\pgfqpoint{1.008011in}{1.022815in}}%
\pgfpathlineto{\pgfqpoint{1.013582in}{1.013319in}}%
\pgfpathclose%
\pgfpathmoveto{\pgfqpoint{1.016585in}{1.026245in}}%
\pgfpathlineto{\pgfqpoint{1.013182in}{1.039172in}}%
\pgfpathlineto{\pgfqpoint{1.013430in}{1.052098in}}%
\pgfpathlineto{\pgfqpoint{1.017755in}{1.065025in}}%
\pgfpathlineto{\pgfqpoint{1.021280in}{1.068820in}}%
\pgfpathlineto{\pgfqpoint{1.034549in}{1.071583in}}%
\pgfpathlineto{\pgfqpoint{1.047145in}{1.065025in}}%
\pgfpathlineto{\pgfqpoint{1.047818in}{1.064161in}}%
\pgfpathlineto{\pgfqpoint{1.051265in}{1.052098in}}%
\pgfpathlineto{\pgfqpoint{1.051482in}{1.039172in}}%
\pgfpathlineto{\pgfqpoint{1.048503in}{1.026245in}}%
\pgfpathlineto{\pgfqpoint{1.047818in}{1.025058in}}%
\pgfpathlineto{\pgfqpoint{1.034549in}{1.017680in}}%
\pgfpathlineto{\pgfqpoint{1.021280in}{1.020555in}}%
\pgfpathclose%
\pgfusepath{fill}%
\end{pgfscope}%
\begin{pgfscope}%
\pgfpathrectangle{\pgfqpoint{0.211875in}{0.211875in}}{\pgfqpoint{1.313625in}{1.279725in}}%
\pgfusepath{clip}%
\pgfsetbuttcap%
\pgfsetroundjoin%
\definecolor{currentfill}{rgb}{0.901975,0.231521,0.249182}%
\pgfsetfillcolor{currentfill}%
\pgfsetlinewidth{0.000000pt}%
\definecolor{currentstroke}{rgb}{0.000000,0.000000,0.000000}%
\pgfsetstrokecolor{currentstroke}%
\pgfsetdash{}{0pt}%
\pgfpathmoveto{\pgfqpoint{1.140701in}{1.009874in}}%
\pgfpathlineto{\pgfqpoint{1.153970in}{1.009265in}}%
\pgfpathlineto{\pgfqpoint{1.167239in}{1.012304in}}%
\pgfpathlineto{\pgfqpoint{1.168602in}{1.013319in}}%
\pgfpathlineto{\pgfqpoint{1.174478in}{1.026245in}}%
\pgfpathlineto{\pgfqpoint{1.175604in}{1.039172in}}%
\pgfpathlineto{\pgfqpoint{1.175501in}{1.052098in}}%
\pgfpathlineto{\pgfqpoint{1.174004in}{1.065025in}}%
\pgfpathlineto{\pgfqpoint{1.167239in}{1.076741in}}%
\pgfpathlineto{\pgfqpoint{1.164071in}{1.077952in}}%
\pgfpathlineto{\pgfqpoint{1.153970in}{1.080039in}}%
\pgfpathlineto{\pgfqpoint{1.140701in}{1.079409in}}%
\pgfpathlineto{\pgfqpoint{1.136308in}{1.077952in}}%
\pgfpathlineto{\pgfqpoint{1.127432in}{1.069458in}}%
\pgfpathlineto{\pgfqpoint{1.126034in}{1.065025in}}%
\pgfpathlineto{\pgfqpoint{1.124607in}{1.052098in}}%
\pgfpathlineto{\pgfqpoint{1.124508in}{1.039172in}}%
\pgfpathlineto{\pgfqpoint{1.125582in}{1.026245in}}%
\pgfpathlineto{\pgfqpoint{1.127432in}{1.019561in}}%
\pgfpathlineto{\pgfqpoint{1.132384in}{1.013319in}}%
\pgfpathclose%
\pgfpathmoveto{\pgfqpoint{1.135058in}{1.026245in}}%
\pgfpathlineto{\pgfqpoint{1.131349in}{1.039172in}}%
\pgfpathlineto{\pgfqpoint{1.131619in}{1.052098in}}%
\pgfpathlineto{\pgfqpoint{1.136326in}{1.065025in}}%
\pgfpathlineto{\pgfqpoint{1.140701in}{1.069256in}}%
\pgfpathlineto{\pgfqpoint{1.153970in}{1.071093in}}%
\pgfpathlineto{\pgfqpoint{1.164045in}{1.065025in}}%
\pgfpathlineto{\pgfqpoint{1.167239in}{1.059880in}}%
\pgfpathlineto{\pgfqpoint{1.169210in}{1.052098in}}%
\pgfpathlineto{\pgfqpoint{1.169424in}{1.039172in}}%
\pgfpathlineto{\pgfqpoint{1.167239in}{1.028984in}}%
\pgfpathlineto{\pgfqpoint{1.165865in}{1.026245in}}%
\pgfpathlineto{\pgfqpoint{1.153970in}{1.018194in}}%
\pgfpathlineto{\pgfqpoint{1.140701in}{1.020106in}}%
\pgfpathclose%
\pgfusepath{fill}%
\end{pgfscope}%
\begin{pgfscope}%
\pgfpathrectangle{\pgfqpoint{0.211875in}{0.211875in}}{\pgfqpoint{1.313625in}{1.279725in}}%
\pgfusepath{clip}%
\pgfsetbuttcap%
\pgfsetroundjoin%
\definecolor{currentfill}{rgb}{0.901975,0.231521,0.249182}%
\pgfsetfillcolor{currentfill}%
\pgfsetlinewidth{0.000000pt}%
\definecolor{currentstroke}{rgb}{0.000000,0.000000,0.000000}%
\pgfsetstrokecolor{currentstroke}%
\pgfsetdash{}{0pt}%
\pgfpathmoveto{\pgfqpoint{1.260121in}{1.009401in}}%
\pgfpathlineto{\pgfqpoint{1.273390in}{1.009072in}}%
\pgfpathlineto{\pgfqpoint{1.286659in}{1.012653in}}%
\pgfpathlineto{\pgfqpoint{1.287449in}{1.013319in}}%
\pgfpathlineto{\pgfqpoint{1.292819in}{1.026245in}}%
\pgfpathlineto{\pgfqpoint{1.293842in}{1.039172in}}%
\pgfpathlineto{\pgfqpoint{1.293745in}{1.052098in}}%
\pgfpathlineto{\pgfqpoint{1.292378in}{1.065025in}}%
\pgfpathlineto{\pgfqpoint{1.286659in}{1.076298in}}%
\pgfpathlineto{\pgfqpoint{1.283156in}{1.077952in}}%
\pgfpathlineto{\pgfqpoint{1.273390in}{1.080225in}}%
\pgfpathlineto{\pgfqpoint{1.260121in}{1.079887in}}%
\pgfpathlineto{\pgfqpoint{1.253678in}{1.077952in}}%
\pgfpathlineto{\pgfqpoint{1.246852in}{1.072863in}}%
\pgfpathlineto{\pgfqpoint{1.243865in}{1.065025in}}%
\pgfpathlineto{\pgfqpoint{1.242480in}{1.052098in}}%
\pgfpathlineto{\pgfqpoint{1.242383in}{1.039172in}}%
\pgfpathlineto{\pgfqpoint{1.243424in}{1.026245in}}%
\pgfpathlineto{\pgfqpoint{1.246852in}{1.016035in}}%
\pgfpathlineto{\pgfqpoint{1.249637in}{1.013319in}}%
\pgfpathclose%
\pgfpathmoveto{\pgfqpoint{1.252788in}{1.026245in}}%
\pgfpathlineto{\pgfqpoint{1.248867in}{1.039172in}}%
\pgfpathlineto{\pgfqpoint{1.249153in}{1.052098in}}%
\pgfpathlineto{\pgfqpoint{1.254129in}{1.065025in}}%
\pgfpathlineto{\pgfqpoint{1.260121in}{1.070221in}}%
\pgfpathlineto{\pgfqpoint{1.273390in}{1.071123in}}%
\pgfpathlineto{\pgfqpoint{1.282272in}{1.065025in}}%
\pgfpathlineto{\pgfqpoint{1.286659in}{1.056196in}}%
\pgfpathlineto{\pgfqpoint{1.287571in}{1.052098in}}%
\pgfpathlineto{\pgfqpoint{1.287776in}{1.039172in}}%
\pgfpathlineto{\pgfqpoint{1.286659in}{1.033226in}}%
\pgfpathlineto{\pgfqpoint{1.283865in}{1.026245in}}%
\pgfpathlineto{\pgfqpoint{1.273390in}{1.018158in}}%
\pgfpathlineto{\pgfqpoint{1.260121in}{1.019098in}}%
\pgfpathclose%
\pgfusepath{fill}%
\end{pgfscope}%
\begin{pgfscope}%
\pgfpathrectangle{\pgfqpoint{0.211875in}{0.211875in}}{\pgfqpoint{1.313625in}{1.279725in}}%
\pgfusepath{clip}%
\pgfsetbuttcap%
\pgfsetroundjoin%
\definecolor{currentfill}{rgb}{0.901975,0.231521,0.249182}%
\pgfsetfillcolor{currentfill}%
\pgfsetlinewidth{0.000000pt}%
\definecolor{currentstroke}{rgb}{0.000000,0.000000,0.000000}%
\pgfsetstrokecolor{currentstroke}%
\pgfsetdash{}{0pt}%
\pgfpathmoveto{\pgfqpoint{1.366273in}{1.012494in}}%
\pgfpathlineto{\pgfqpoint{1.379542in}{1.008426in}}%
\pgfpathlineto{\pgfqpoint{1.392811in}{1.008330in}}%
\pgfpathlineto{\pgfqpoint{1.406080in}{1.012044in}}%
\pgfpathlineto{\pgfqpoint{1.407391in}{1.013319in}}%
\pgfpathlineto{\pgfqpoint{1.411686in}{1.026245in}}%
\pgfpathlineto{\pgfqpoint{1.412496in}{1.039172in}}%
\pgfpathlineto{\pgfqpoint{1.412412in}{1.052098in}}%
\pgfpathlineto{\pgfqpoint{1.411308in}{1.065025in}}%
\pgfpathlineto{\pgfqpoint{1.406080in}{1.076930in}}%
\pgfpathlineto{\pgfqpoint{1.404340in}{1.077952in}}%
\pgfpathlineto{\pgfqpoint{1.392811in}{1.080952in}}%
\pgfpathlineto{\pgfqpoint{1.379542in}{1.080859in}}%
\pgfpathlineto{\pgfqpoint{1.368821in}{1.077952in}}%
\pgfpathlineto{\pgfqpoint{1.366273in}{1.076433in}}%
\pgfpathlineto{\pgfqpoint{1.361172in}{1.065025in}}%
\pgfpathlineto{\pgfqpoint{1.359959in}{1.052098in}}%
\pgfpathlineto{\pgfqpoint{1.359870in}{1.039172in}}%
\pgfpathlineto{\pgfqpoint{1.360771in}{1.026245in}}%
\pgfpathlineto{\pgfqpoint{1.365425in}{1.013319in}}%
\pgfpathclose%
\pgfpathmoveto{\pgfqpoint{1.369575in}{1.026245in}}%
\pgfpathlineto{\pgfqpoint{1.366273in}{1.036332in}}%
\pgfpathlineto{\pgfqpoint{1.365802in}{1.039172in}}%
\pgfpathlineto{\pgfqpoint{1.365996in}{1.052098in}}%
\pgfpathlineto{\pgfqpoint{1.366273in}{1.053506in}}%
\pgfpathlineto{\pgfqpoint{1.370960in}{1.065025in}}%
\pgfpathlineto{\pgfqpoint{1.379542in}{1.071668in}}%
\pgfpathlineto{\pgfqpoint{1.392811in}{1.071710in}}%
\pgfpathlineto{\pgfqpoint{1.401437in}{1.065025in}}%
\pgfpathlineto{\pgfqpoint{1.406080in}{1.053302in}}%
\pgfpathlineto{\pgfqpoint{1.406311in}{1.052098in}}%
\pgfpathlineto{\pgfqpoint{1.406502in}{1.039172in}}%
\pgfpathlineto{\pgfqpoint{1.406080in}{1.036562in}}%
\pgfpathlineto{\pgfqpoint{1.402801in}{1.026245in}}%
\pgfpathlineto{\pgfqpoint{1.392811in}{1.017534in}}%
\pgfpathlineto{\pgfqpoint{1.379542in}{1.017579in}}%
\pgfpathclose%
\pgfusepath{fill}%
\end{pgfscope}%
\begin{pgfscope}%
\pgfpathrectangle{\pgfqpoint{0.211875in}{0.211875in}}{\pgfqpoint{1.313625in}{1.279725in}}%
\pgfusepath{clip}%
\pgfsetbuttcap%
\pgfsetroundjoin%
\definecolor{currentfill}{rgb}{0.901975,0.231521,0.249182}%
\pgfsetfillcolor{currentfill}%
\pgfsetlinewidth{0.000000pt}%
\definecolor{currentstroke}{rgb}{0.000000,0.000000,0.000000}%
\pgfsetstrokecolor{currentstroke}%
\pgfsetdash{}{0pt}%
\pgfpathmoveto{\pgfqpoint{1.485693in}{1.009288in}}%
\pgfpathlineto{\pgfqpoint{1.498962in}{1.006979in}}%
\pgfpathlineto{\pgfqpoint{1.512231in}{1.006977in}}%
\pgfpathlineto{\pgfqpoint{1.525500in}{1.009964in}}%
\pgfpathlineto{\pgfqpoint{1.525500in}{1.013319in}}%
\pgfpathlineto{\pgfqpoint{1.525500in}{1.026245in}}%
\pgfpathlineto{\pgfqpoint{1.525500in}{1.038548in}}%
\pgfpathlineto{\pgfqpoint{1.522443in}{1.026245in}}%
\pgfpathlineto{\pgfqpoint{1.512231in}{1.016258in}}%
\pgfpathlineto{\pgfqpoint{1.498962in}{1.015574in}}%
\pgfpathlineto{\pgfqpoint{1.485693in}{1.025474in}}%
\pgfpathlineto{\pgfqpoint{1.485342in}{1.026245in}}%
\pgfpathlineto{\pgfqpoint{1.482945in}{1.039172in}}%
\pgfpathlineto{\pgfqpoint{1.483125in}{1.052098in}}%
\pgfpathlineto{\pgfqpoint{1.485693in}{1.063432in}}%
\pgfpathlineto{\pgfqpoint{1.486503in}{1.065025in}}%
\pgfpathlineto{\pgfqpoint{1.498962in}{1.073577in}}%
\pgfpathlineto{\pgfqpoint{1.512231in}{1.072913in}}%
\pgfpathlineto{\pgfqpoint{1.521312in}{1.065025in}}%
\pgfpathlineto{\pgfqpoint{1.525380in}{1.052098in}}%
\pgfpathlineto{\pgfqpoint{1.525500in}{1.045663in}}%
\pgfpathlineto{\pgfqpoint{1.525500in}{1.052098in}}%
\pgfpathlineto{\pgfqpoint{1.525500in}{1.065025in}}%
\pgfpathlineto{\pgfqpoint{1.525500in}{1.077952in}}%
\pgfpathlineto{\pgfqpoint{1.525500in}{1.079033in}}%
\pgfpathlineto{\pgfqpoint{1.512231in}{1.082280in}}%
\pgfpathlineto{\pgfqpoint{1.498962in}{1.082295in}}%
\pgfpathlineto{\pgfqpoint{1.485693in}{1.079848in}}%
\pgfpathlineto{\pgfqpoint{1.482910in}{1.077952in}}%
\pgfpathlineto{\pgfqpoint{1.477912in}{1.065025in}}%
\pgfpathlineto{\pgfqpoint{1.477007in}{1.052098in}}%
\pgfpathlineto{\pgfqpoint{1.476933in}{1.039172in}}%
\pgfpathlineto{\pgfqpoint{1.477583in}{1.026245in}}%
\pgfpathlineto{\pgfqpoint{1.480941in}{1.013319in}}%
\pgfpathclose%
\pgfusepath{fill}%
\end{pgfscope}%
\begin{pgfscope}%
\pgfpathrectangle{\pgfqpoint{0.211875in}{0.211875in}}{\pgfqpoint{1.313625in}{1.279725in}}%
\pgfusepath{clip}%
\pgfsetbuttcap%
\pgfsetroundjoin%
\definecolor{currentfill}{rgb}{0.901975,0.231521,0.249182}%
\pgfsetfillcolor{currentfill}%
\pgfsetlinewidth{0.000000pt}%
\definecolor{currentstroke}{rgb}{0.000000,0.000000,0.000000}%
\pgfsetstrokecolor{currentstroke}%
\pgfsetdash{}{0pt}%
\pgfpathmoveto{\pgfqpoint{0.225144in}{1.090365in}}%
\pgfpathlineto{\pgfqpoint{0.226319in}{1.090878in}}%
\pgfpathlineto{\pgfqpoint{0.233337in}{1.103805in}}%
\pgfpathlineto{\pgfqpoint{0.234252in}{1.116731in}}%
\pgfpathlineto{\pgfqpoint{0.234363in}{1.129658in}}%
\pgfpathlineto{\pgfqpoint{0.233893in}{1.142584in}}%
\pgfpathlineto{\pgfqpoint{0.231519in}{1.155511in}}%
\pgfpathlineto{\pgfqpoint{0.225144in}{1.161418in}}%
\pgfpathlineto{\pgfqpoint{0.211875in}{1.163060in}}%
\pgfpathlineto{\pgfqpoint{0.211875in}{1.155511in}}%
\pgfpathlineto{\pgfqpoint{0.211875in}{1.154638in}}%
\pgfpathlineto{\pgfqpoint{0.225144in}{1.145148in}}%
\pgfpathlineto{\pgfqpoint{0.226244in}{1.142584in}}%
\pgfpathlineto{\pgfqpoint{0.228217in}{1.129658in}}%
\pgfpathlineto{\pgfqpoint{0.227748in}{1.116731in}}%
\pgfpathlineto{\pgfqpoint{0.225144in}{1.106980in}}%
\pgfpathlineto{\pgfqpoint{0.223029in}{1.103805in}}%
\pgfpathlineto{\pgfqpoint{0.211875in}{1.097567in}}%
\pgfpathlineto{\pgfqpoint{0.211875in}{1.090878in}}%
\pgfpathlineto{\pgfqpoint{0.211875in}{1.088748in}}%
\pgfpathclose%
\pgfusepath{fill}%
\end{pgfscope}%
\begin{pgfscope}%
\pgfpathrectangle{\pgfqpoint{0.211875in}{0.211875in}}{\pgfqpoint{1.313625in}{1.279725in}}%
\pgfusepath{clip}%
\pgfsetbuttcap%
\pgfsetroundjoin%
\definecolor{currentfill}{rgb}{0.901975,0.231521,0.249182}%
\pgfsetfillcolor{currentfill}%
\pgfsetlinewidth{0.000000pt}%
\definecolor{currentstroke}{rgb}{0.000000,0.000000,0.000000}%
\pgfsetstrokecolor{currentstroke}%
\pgfsetdash{}{0pt}%
\pgfpathmoveto{\pgfqpoint{0.716095in}{1.089418in}}%
\pgfpathlineto{\pgfqpoint{0.729364in}{1.087440in}}%
\pgfpathlineto{\pgfqpoint{0.742633in}{1.087532in}}%
\pgfpathlineto{\pgfqpoint{0.755902in}{1.089500in}}%
\pgfpathlineto{\pgfqpoint{0.758486in}{1.090878in}}%
\pgfpathlineto{\pgfqpoint{0.764040in}{1.103805in}}%
\pgfpathlineto{\pgfqpoint{0.764875in}{1.116731in}}%
\pgfpathlineto{\pgfqpoint{0.764979in}{1.129658in}}%
\pgfpathlineto{\pgfqpoint{0.764544in}{1.142584in}}%
\pgfpathlineto{\pgfqpoint{0.762466in}{1.155511in}}%
\pgfpathlineto{\pgfqpoint{0.755902in}{1.162298in}}%
\pgfpathlineto{\pgfqpoint{0.742633in}{1.164296in}}%
\pgfpathlineto{\pgfqpoint{0.729364in}{1.164389in}}%
\pgfpathlineto{\pgfqpoint{0.716095in}{1.162381in}}%
\pgfpathlineto{\pgfqpoint{0.710960in}{1.155511in}}%
\pgfpathlineto{\pgfqpoint{0.709559in}{1.142584in}}%
\pgfpathlineto{\pgfqpoint{0.709275in}{1.129658in}}%
\pgfpathlineto{\pgfqpoint{0.709342in}{1.116731in}}%
\pgfpathlineto{\pgfqpoint{0.709891in}{1.103805in}}%
\pgfpathlineto{\pgfqpoint{0.713864in}{1.090878in}}%
\pgfpathclose%
\pgfpathmoveto{\pgfqpoint{0.720022in}{1.103805in}}%
\pgfpathlineto{\pgfqpoint{0.716095in}{1.113989in}}%
\pgfpathlineto{\pgfqpoint{0.715577in}{1.116731in}}%
\pgfpathlineto{\pgfqpoint{0.715171in}{1.129658in}}%
\pgfpathlineto{\pgfqpoint{0.716095in}{1.137355in}}%
\pgfpathlineto{\pgfqpoint{0.717196in}{1.142584in}}%
\pgfpathlineto{\pgfqpoint{0.729154in}{1.155511in}}%
\pgfpathlineto{\pgfqpoint{0.729364in}{1.155601in}}%
\pgfpathlineto{\pgfqpoint{0.742633in}{1.156011in}}%
\pgfpathlineto{\pgfqpoint{0.743997in}{1.155511in}}%
\pgfpathlineto{\pgfqpoint{0.755902in}{1.145951in}}%
\pgfpathlineto{\pgfqpoint{0.757255in}{1.142584in}}%
\pgfpathlineto{\pgfqpoint{0.759070in}{1.129658in}}%
\pgfpathlineto{\pgfqpoint{0.758636in}{1.116731in}}%
\pgfpathlineto{\pgfqpoint{0.755902in}{1.105884in}}%
\pgfpathlineto{\pgfqpoint{0.754680in}{1.103805in}}%
\pgfpathlineto{\pgfqpoint{0.742633in}{1.096255in}}%
\pgfpathlineto{\pgfqpoint{0.729364in}{1.096706in}}%
\pgfpathclose%
\pgfusepath{fill}%
\end{pgfscope}%
\begin{pgfscope}%
\pgfpathrectangle{\pgfqpoint{0.211875in}{0.211875in}}{\pgfqpoint{1.313625in}{1.279725in}}%
\pgfusepath{clip}%
\pgfsetbuttcap%
\pgfsetroundjoin%
\definecolor{currentfill}{rgb}{0.901975,0.231521,0.249182}%
\pgfsetfillcolor{currentfill}%
\pgfsetlinewidth{0.000000pt}%
\definecolor{currentstroke}{rgb}{0.000000,0.000000,0.000000}%
\pgfsetstrokecolor{currentstroke}%
\pgfsetdash{}{0pt}%
\pgfpathmoveto{\pgfqpoint{0.848784in}{1.089079in}}%
\pgfpathlineto{\pgfqpoint{0.862053in}{1.089237in}}%
\pgfpathlineto{\pgfqpoint{0.869789in}{1.090878in}}%
\pgfpathlineto{\pgfqpoint{0.875322in}{1.093403in}}%
\pgfpathlineto{\pgfqpoint{0.880396in}{1.103805in}}%
\pgfpathlineto{\pgfqpoint{0.881700in}{1.116731in}}%
\pgfpathlineto{\pgfqpoint{0.881862in}{1.129658in}}%
\pgfpathlineto{\pgfqpoint{0.881183in}{1.142584in}}%
\pgfpathlineto{\pgfqpoint{0.877923in}{1.155511in}}%
\pgfpathlineto{\pgfqpoint{0.875322in}{1.158604in}}%
\pgfpathlineto{\pgfqpoint{0.862053in}{1.162565in}}%
\pgfpathlineto{\pgfqpoint{0.848784in}{1.162724in}}%
\pgfpathlineto{\pgfqpoint{0.835515in}{1.159536in}}%
\pgfpathlineto{\pgfqpoint{0.831983in}{1.155511in}}%
\pgfpathlineto{\pgfqpoint{0.829093in}{1.142584in}}%
\pgfpathlineto{\pgfqpoint{0.828504in}{1.129658in}}%
\pgfpathlineto{\pgfqpoint{0.828645in}{1.116731in}}%
\pgfpathlineto{\pgfqpoint{0.829781in}{1.103805in}}%
\pgfpathlineto{\pgfqpoint{0.835515in}{1.092378in}}%
\pgfpathlineto{\pgfqpoint{0.839229in}{1.090878in}}%
\pgfpathclose%
\pgfpathmoveto{\pgfqpoint{0.840538in}{1.103805in}}%
\pgfpathlineto{\pgfqpoint{0.835515in}{1.114046in}}%
\pgfpathlineto{\pgfqpoint{0.834917in}{1.116731in}}%
\pgfpathlineto{\pgfqpoint{0.834438in}{1.129658in}}%
\pgfpathlineto{\pgfqpoint{0.835515in}{1.137270in}}%
\pgfpathlineto{\pgfqpoint{0.836940in}{1.142584in}}%
\pgfpathlineto{\pgfqpoint{0.848784in}{1.153888in}}%
\pgfpathlineto{\pgfqpoint{0.862053in}{1.153678in}}%
\pgfpathlineto{\pgfqpoint{0.873293in}{1.142584in}}%
\pgfpathlineto{\pgfqpoint{0.875322in}{1.134741in}}%
\pgfpathlineto{\pgfqpoint{0.876027in}{1.129658in}}%
\pgfpathlineto{\pgfqpoint{0.875538in}{1.116731in}}%
\pgfpathlineto{\pgfqpoint{0.875322in}{1.115743in}}%
\pgfpathlineto{\pgfqpoint{0.869714in}{1.103805in}}%
\pgfpathlineto{\pgfqpoint{0.862053in}{1.098401in}}%
\pgfpathlineto{\pgfqpoint{0.848784in}{1.098219in}}%
\pgfpathclose%
\pgfusepath{fill}%
\end{pgfscope}%
\begin{pgfscope}%
\pgfpathrectangle{\pgfqpoint{0.211875in}{0.211875in}}{\pgfqpoint{1.313625in}{1.279725in}}%
\pgfusepath{clip}%
\pgfsetbuttcap%
\pgfsetroundjoin%
\definecolor{currentfill}{rgb}{0.901975,0.231521,0.249182}%
\pgfsetfillcolor{currentfill}%
\pgfsetlinewidth{0.000000pt}%
\definecolor{currentstroke}{rgb}{0.000000,0.000000,0.000000}%
\pgfsetstrokecolor{currentstroke}%
\pgfsetdash{}{0pt}%
\pgfpathmoveto{\pgfqpoint{0.968205in}{1.090097in}}%
\pgfpathlineto{\pgfqpoint{0.981473in}{1.090489in}}%
\pgfpathlineto{\pgfqpoint{0.983142in}{1.090878in}}%
\pgfpathlineto{\pgfqpoint{0.994742in}{1.097482in}}%
\pgfpathlineto{\pgfqpoint{0.997360in}{1.103805in}}%
\pgfpathlineto{\pgfqpoint{0.998966in}{1.116731in}}%
\pgfpathlineto{\pgfqpoint{0.999165in}{1.129658in}}%
\pgfpathlineto{\pgfqpoint{0.998330in}{1.142584in}}%
\pgfpathlineto{\pgfqpoint{0.994742in}{1.154733in}}%
\pgfpathlineto{\pgfqpoint{0.994126in}{1.155511in}}%
\pgfpathlineto{\pgfqpoint{0.981473in}{1.161294in}}%
\pgfpathlineto{\pgfqpoint{0.968205in}{1.161691in}}%
\pgfpathlineto{\pgfqpoint{0.954936in}{1.158334in}}%
\pgfpathlineto{\pgfqpoint{0.952086in}{1.155511in}}%
\pgfpathlineto{\pgfqpoint{0.948147in}{1.142584in}}%
\pgfpathlineto{\pgfqpoint{0.947341in}{1.129658in}}%
\pgfpathlineto{\pgfqpoint{0.947534in}{1.116731in}}%
\pgfpathlineto{\pgfqpoint{0.949087in}{1.103805in}}%
\pgfpathlineto{\pgfqpoint{0.954936in}{1.093700in}}%
\pgfpathlineto{\pgfqpoint{0.963583in}{1.090878in}}%
\pgfpathclose%
\pgfpathmoveto{\pgfqpoint{0.960383in}{1.103805in}}%
\pgfpathlineto{\pgfqpoint{0.954936in}{1.112652in}}%
\pgfpathlineto{\pgfqpoint{0.953886in}{1.116731in}}%
\pgfpathlineto{\pgfqpoint{0.953352in}{1.129658in}}%
\pgfpathlineto{\pgfqpoint{0.954936in}{1.139347in}}%
\pgfpathlineto{\pgfqpoint{0.956027in}{1.142584in}}%
\pgfpathlineto{\pgfqpoint{0.968205in}{1.152858in}}%
\pgfpathlineto{\pgfqpoint{0.981473in}{1.151706in}}%
\pgfpathlineto{\pgfqpoint{0.989724in}{1.142584in}}%
\pgfpathlineto{\pgfqpoint{0.992810in}{1.129658in}}%
\pgfpathlineto{\pgfqpoint{0.992072in}{1.116731in}}%
\pgfpathlineto{\pgfqpoint{0.986146in}{1.103805in}}%
\pgfpathlineto{\pgfqpoint{0.981473in}{1.100119in}}%
\pgfpathlineto{\pgfqpoint{0.968205in}{1.099116in}}%
\pgfpathclose%
\pgfusepath{fill}%
\end{pgfscope}%
\begin{pgfscope}%
\pgfpathrectangle{\pgfqpoint{0.211875in}{0.211875in}}{\pgfqpoint{1.313625in}{1.279725in}}%
\pgfusepath{clip}%
\pgfsetbuttcap%
\pgfsetroundjoin%
\definecolor{currentfill}{rgb}{0.901975,0.231521,0.249182}%
\pgfsetfillcolor{currentfill}%
\pgfsetlinewidth{0.000000pt}%
\definecolor{currentstroke}{rgb}{0.000000,0.000000,0.000000}%
\pgfsetstrokecolor{currentstroke}%
\pgfsetdash{}{0pt}%
\pgfpathmoveto{\pgfqpoint{1.087625in}{1.090560in}}%
\pgfpathlineto{\pgfqpoint{1.094021in}{1.090878in}}%
\pgfpathlineto{\pgfqpoint{1.100894in}{1.091308in}}%
\pgfpathlineto{\pgfqpoint{1.114163in}{1.101677in}}%
\pgfpathlineto{\pgfqpoint{1.114887in}{1.103805in}}%
\pgfpathlineto{\pgfqpoint{1.116632in}{1.116731in}}%
\pgfpathlineto{\pgfqpoint{1.116849in}{1.129658in}}%
\pgfpathlineto{\pgfqpoint{1.115942in}{1.142584in}}%
\pgfpathlineto{\pgfqpoint{1.114163in}{1.149917in}}%
\pgfpathlineto{\pgfqpoint{1.110745in}{1.155511in}}%
\pgfpathlineto{\pgfqpoint{1.100894in}{1.160507in}}%
\pgfpathlineto{\pgfqpoint{1.087625in}{1.161221in}}%
\pgfpathlineto{\pgfqpoint{1.074356in}{1.158208in}}%
\pgfpathlineto{\pgfqpoint{1.071270in}{1.155511in}}%
\pgfpathlineto{\pgfqpoint{1.066706in}{1.142584in}}%
\pgfpathlineto{\pgfqpoint{1.065768in}{1.129658in}}%
\pgfpathlineto{\pgfqpoint{1.065992in}{1.116731in}}%
\pgfpathlineto{\pgfqpoint{1.067799in}{1.103805in}}%
\pgfpathlineto{\pgfqpoint{1.074356in}{1.093838in}}%
\pgfpathlineto{\pgfqpoint{1.085514in}{1.090878in}}%
\pgfpathclose%
\pgfpathmoveto{\pgfqpoint{1.079351in}{1.103805in}}%
\pgfpathlineto{\pgfqpoint{1.074356in}{1.110290in}}%
\pgfpathlineto{\pgfqpoint{1.072469in}{1.116731in}}%
\pgfpathlineto{\pgfqpoint{1.071898in}{1.129658in}}%
\pgfpathlineto{\pgfqpoint{1.074291in}{1.142584in}}%
\pgfpathlineto{\pgfqpoint{1.074356in}{1.142723in}}%
\pgfpathlineto{\pgfqpoint{1.087625in}{1.152462in}}%
\pgfpathlineto{\pgfqpoint{1.100894in}{1.150242in}}%
\pgfpathlineto{\pgfqpoint{1.107102in}{1.142584in}}%
\pgfpathlineto{\pgfqpoint{1.110083in}{1.129658in}}%
\pgfpathlineto{\pgfqpoint{1.109372in}{1.116731in}}%
\pgfpathlineto{\pgfqpoint{1.103636in}{1.103805in}}%
\pgfpathlineto{\pgfqpoint{1.100894in}{1.101394in}}%
\pgfpathlineto{\pgfqpoint{1.087625in}{1.099461in}}%
\pgfpathclose%
\pgfusepath{fill}%
\end{pgfscope}%
\begin{pgfscope}%
\pgfpathrectangle{\pgfqpoint{0.211875in}{0.211875in}}{\pgfqpoint{1.313625in}{1.279725in}}%
\pgfusepath{clip}%
\pgfsetbuttcap%
\pgfsetroundjoin%
\definecolor{currentfill}{rgb}{0.901975,0.231521,0.249182}%
\pgfsetfillcolor{currentfill}%
\pgfsetlinewidth{0.000000pt}%
\definecolor{currentstroke}{rgb}{0.000000,0.000000,0.000000}%
\pgfsetstrokecolor{currentstroke}%
\pgfsetdash{}{0pt}%
\pgfpathmoveto{\pgfqpoint{1.207045in}{1.090509in}}%
\pgfpathlineto{\pgfqpoint{1.212505in}{1.090878in}}%
\pgfpathlineto{\pgfqpoint{1.220314in}{1.091586in}}%
\pgfpathlineto{\pgfqpoint{1.232811in}{1.103805in}}%
\pgfpathlineto{\pgfqpoint{1.233583in}{1.107518in}}%
\pgfpathlineto{\pgfqpoint{1.234675in}{1.116731in}}%
\pgfpathlineto{\pgfqpoint{1.234888in}{1.129658in}}%
\pgfpathlineto{\pgfqpoint{1.233994in}{1.142584in}}%
\pgfpathlineto{\pgfqpoint{1.233583in}{1.144754in}}%
\pgfpathlineto{\pgfqpoint{1.228753in}{1.155511in}}%
\pgfpathlineto{\pgfqpoint{1.220314in}{1.160255in}}%
\pgfpathlineto{\pgfqpoint{1.207045in}{1.161273in}}%
\pgfpathlineto{\pgfqpoint{1.193777in}{1.158845in}}%
\pgfpathlineto{\pgfqpoint{1.189496in}{1.155511in}}%
\pgfpathlineto{\pgfqpoint{1.184733in}{1.142584in}}%
\pgfpathlineto{\pgfqpoint{1.183749in}{1.129658in}}%
\pgfpathlineto{\pgfqpoint{1.183984in}{1.116731in}}%
\pgfpathlineto{\pgfqpoint{1.185876in}{1.103805in}}%
\pgfpathlineto{\pgfqpoint{1.193777in}{1.093138in}}%
\pgfpathlineto{\pgfqpoint{1.204263in}{1.090878in}}%
\pgfpathclose%
\pgfpathmoveto{\pgfqpoint{1.197085in}{1.103805in}}%
\pgfpathlineto{\pgfqpoint{1.193777in}{1.107220in}}%
\pgfpathlineto{\pgfqpoint{1.190635in}{1.116731in}}%
\pgfpathlineto{\pgfqpoint{1.190045in}{1.129658in}}%
\pgfpathlineto{\pgfqpoint{1.192516in}{1.142584in}}%
\pgfpathlineto{\pgfqpoint{1.193777in}{1.144972in}}%
\pgfpathlineto{\pgfqpoint{1.207045in}{1.152653in}}%
\pgfpathlineto{\pgfqpoint{1.220314in}{1.149329in}}%
\pgfpathlineto{\pgfqpoint{1.225224in}{1.142584in}}%
\pgfpathlineto{\pgfqpoint{1.228021in}{1.129658in}}%
\pgfpathlineto{\pgfqpoint{1.227354in}{1.116731in}}%
\pgfpathlineto{\pgfqpoint{1.221967in}{1.103805in}}%
\pgfpathlineto{\pgfqpoint{1.220314in}{1.102189in}}%
\pgfpathlineto{\pgfqpoint{1.207045in}{1.099295in}}%
\pgfpathclose%
\pgfusepath{fill}%
\end{pgfscope}%
\begin{pgfscope}%
\pgfpathrectangle{\pgfqpoint{0.211875in}{0.211875in}}{\pgfqpoint{1.313625in}{1.279725in}}%
\pgfusepath{clip}%
\pgfsetbuttcap%
\pgfsetroundjoin%
\definecolor{currentfill}{rgb}{0.901975,0.231521,0.249182}%
\pgfsetfillcolor{currentfill}%
\pgfsetlinewidth{0.000000pt}%
\definecolor{currentstroke}{rgb}{0.000000,0.000000,0.000000}%
\pgfsetstrokecolor{currentstroke}%
\pgfsetdash{}{0pt}%
\pgfpathmoveto{\pgfqpoint{1.326466in}{1.089962in}}%
\pgfpathlineto{\pgfqpoint{1.337228in}{1.090878in}}%
\pgfpathlineto{\pgfqpoint{1.339735in}{1.091184in}}%
\pgfpathlineto{\pgfqpoint{1.351335in}{1.103805in}}%
\pgfpathlineto{\pgfqpoint{1.353004in}{1.115766in}}%
\pgfpathlineto{\pgfqpoint{1.353085in}{1.116731in}}%
\pgfpathlineto{\pgfqpoint{1.353275in}{1.129658in}}%
\pgfpathlineto{\pgfqpoint{1.353004in}{1.134706in}}%
\pgfpathlineto{\pgfqpoint{1.352401in}{1.142584in}}%
\pgfpathlineto{\pgfqpoint{1.347930in}{1.155511in}}%
\pgfpathlineto{\pgfqpoint{1.339735in}{1.160621in}}%
\pgfpathlineto{\pgfqpoint{1.326466in}{1.161829in}}%
\pgfpathlineto{\pgfqpoint{1.313197in}{1.160062in}}%
\pgfpathlineto{\pgfqpoint{1.306684in}{1.155511in}}%
\pgfpathlineto{\pgfqpoint{1.302166in}{1.142584in}}%
\pgfpathlineto{\pgfqpoint{1.301228in}{1.129658in}}%
\pgfpathlineto{\pgfqpoint{1.301452in}{1.116731in}}%
\pgfpathlineto{\pgfqpoint{1.303254in}{1.103805in}}%
\pgfpathlineto{\pgfqpoint{1.313197in}{1.091799in}}%
\pgfpathlineto{\pgfqpoint{1.318500in}{1.090878in}}%
\pgfpathclose%
\pgfpathmoveto{\pgfqpoint{1.313080in}{1.103805in}}%
\pgfpathlineto{\pgfqpoint{1.308332in}{1.116731in}}%
\pgfpathlineto{\pgfqpoint{1.307742in}{1.129658in}}%
\pgfpathlineto{\pgfqpoint{1.310212in}{1.142584in}}%
\pgfpathlineto{\pgfqpoint{1.313197in}{1.147637in}}%
\pgfpathlineto{\pgfqpoint{1.326466in}{1.153413in}}%
\pgfpathlineto{\pgfqpoint{1.339735in}{1.149044in}}%
\pgfpathlineto{\pgfqpoint{1.343957in}{1.142584in}}%
\pgfpathlineto{\pgfqpoint{1.346497in}{1.129658in}}%
\pgfpathlineto{\pgfqpoint{1.345891in}{1.116731in}}%
\pgfpathlineto{\pgfqpoint{1.340991in}{1.103805in}}%
\pgfpathlineto{\pgfqpoint{1.339735in}{1.102437in}}%
\pgfpathlineto{\pgfqpoint{1.326466in}{1.098633in}}%
\pgfpathlineto{\pgfqpoint{1.313197in}{1.103663in}}%
\pgfpathclose%
\pgfusepath{fill}%
\end{pgfscope}%
\begin{pgfscope}%
\pgfpathrectangle{\pgfqpoint{0.211875in}{0.211875in}}{\pgfqpoint{1.313625in}{1.279725in}}%
\pgfusepath{clip}%
\pgfsetbuttcap%
\pgfsetroundjoin%
\definecolor{currentfill}{rgb}{0.901975,0.231521,0.249182}%
\pgfsetfillcolor{currentfill}%
\pgfsetlinewidth{0.000000pt}%
\definecolor{currentstroke}{rgb}{0.000000,0.000000,0.000000}%
\pgfsetstrokecolor{currentstroke}%
\pgfsetdash{}{0pt}%
\pgfpathmoveto{\pgfqpoint{1.432617in}{1.090036in}}%
\pgfpathlineto{\pgfqpoint{1.445886in}{1.088916in}}%
\pgfpathlineto{\pgfqpoint{1.459155in}{1.090052in}}%
\pgfpathlineto{\pgfqpoint{1.461576in}{1.090878in}}%
\pgfpathlineto{\pgfqpoint{1.470568in}{1.103805in}}%
\pgfpathlineto{\pgfqpoint{1.471822in}{1.116731in}}%
\pgfpathlineto{\pgfqpoint{1.471976in}{1.129658in}}%
\pgfpathlineto{\pgfqpoint{1.471327in}{1.142584in}}%
\pgfpathlineto{\pgfqpoint{1.468136in}{1.155511in}}%
\pgfpathlineto{\pgfqpoint{1.459155in}{1.161737in}}%
\pgfpathlineto{\pgfqpoint{1.445886in}{1.162890in}}%
\pgfpathlineto{\pgfqpoint{1.432617in}{1.161753in}}%
\pgfpathlineto{\pgfqpoint{1.422705in}{1.155511in}}%
\pgfpathlineto{\pgfqpoint{1.419348in}{1.145149in}}%
\pgfpathlineto{\pgfqpoint{1.418975in}{1.142584in}}%
\pgfpathlineto{\pgfqpoint{1.418296in}{1.129658in}}%
\pgfpathlineto{\pgfqpoint{1.418458in}{1.116731in}}%
\pgfpathlineto{\pgfqpoint{1.419348in}{1.106980in}}%
\pgfpathlineto{\pgfqpoint{1.419830in}{1.103805in}}%
\pgfpathlineto{\pgfqpoint{1.430014in}{1.090878in}}%
\pgfpathclose%
\pgfpathmoveto{\pgfqpoint{1.430055in}{1.103805in}}%
\pgfpathlineto{\pgfqpoint{1.425485in}{1.116731in}}%
\pgfpathlineto{\pgfqpoint{1.424915in}{1.129658in}}%
\pgfpathlineto{\pgfqpoint{1.427297in}{1.142584in}}%
\pgfpathlineto{\pgfqpoint{1.432617in}{1.150665in}}%
\pgfpathlineto{\pgfqpoint{1.445886in}{1.154746in}}%
\pgfpathlineto{\pgfqpoint{1.459155in}{1.149514in}}%
\pgfpathlineto{\pgfqpoint{1.463212in}{1.142584in}}%
\pgfpathlineto{\pgfqpoint{1.465429in}{1.129658in}}%
\pgfpathlineto{\pgfqpoint{1.464901in}{1.116731in}}%
\pgfpathlineto{\pgfqpoint{1.460619in}{1.103805in}}%
\pgfpathlineto{\pgfqpoint{1.459155in}{1.102029in}}%
\pgfpathlineto{\pgfqpoint{1.445886in}{1.097472in}}%
\pgfpathlineto{\pgfqpoint{1.432617in}{1.101026in}}%
\pgfpathclose%
\pgfusepath{fill}%
\end{pgfscope}%
\begin{pgfscope}%
\pgfpathrectangle{\pgfqpoint{0.211875in}{0.211875in}}{\pgfqpoint{1.313625in}{1.279725in}}%
\pgfusepath{clip}%
\pgfsetbuttcap%
\pgfsetroundjoin%
\definecolor{currentfill}{rgb}{0.901975,0.231521,0.249182}%
\pgfsetfillcolor{currentfill}%
\pgfsetlinewidth{0.000000pt}%
\definecolor{currentstroke}{rgb}{0.000000,0.000000,0.000000}%
\pgfsetstrokecolor{currentstroke}%
\pgfsetdash{}{0pt}%
\pgfpathmoveto{\pgfqpoint{0.264951in}{1.168059in}}%
\pgfpathlineto{\pgfqpoint{0.278220in}{1.167907in}}%
\pgfpathlineto{\pgfqpoint{0.286279in}{1.168437in}}%
\pgfpathlineto{\pgfqpoint{0.291489in}{1.169832in}}%
\pgfpathlineto{\pgfqpoint{0.294639in}{1.181364in}}%
\pgfpathlineto{\pgfqpoint{0.294921in}{1.194290in}}%
\pgfpathlineto{\pgfqpoint{0.294955in}{1.207217in}}%
\pgfpathlineto{\pgfqpoint{0.294849in}{1.220143in}}%
\pgfpathlineto{\pgfqpoint{0.294400in}{1.233070in}}%
\pgfpathlineto{\pgfqpoint{0.291489in}{1.243339in}}%
\pgfpathlineto{\pgfqpoint{0.280689in}{1.245996in}}%
\pgfpathlineto{\pgfqpoint{0.278220in}{1.246149in}}%
\pgfpathlineto{\pgfqpoint{0.264951in}{1.246064in}}%
\pgfpathlineto{\pgfqpoint{0.263997in}{1.245996in}}%
\pgfpathlineto{\pgfqpoint{0.251682in}{1.244754in}}%
\pgfpathlineto{\pgfqpoint{0.239929in}{1.233070in}}%
\pgfpathlineto{\pgfqpoint{0.238449in}{1.220143in}}%
\pgfpathlineto{\pgfqpoint{0.238413in}{1.218837in}}%
\pgfpathlineto{\pgfqpoint{0.238152in}{1.207217in}}%
\pgfpathlineto{\pgfqpoint{0.238375in}{1.194290in}}%
\pgfpathlineto{\pgfqpoint{0.238413in}{1.193656in}}%
\pgfpathlineto{\pgfqpoint{0.239680in}{1.181364in}}%
\pgfpathlineto{\pgfqpoint{0.251682in}{1.169274in}}%
\pgfpathlineto{\pgfqpoint{0.259757in}{1.168437in}}%
\pgfpathclose%
\pgfpathmoveto{\pgfqpoint{0.250973in}{1.181364in}}%
\pgfpathlineto{\pgfqpoint{0.245464in}{1.194290in}}%
\pgfpathlineto{\pgfqpoint{0.244382in}{1.207217in}}%
\pgfpathlineto{\pgfqpoint{0.245550in}{1.220143in}}%
\pgfpathlineto{\pgfqpoint{0.251259in}{1.233070in}}%
\pgfpathlineto{\pgfqpoint{0.251682in}{1.233490in}}%
\pgfpathlineto{\pgfqpoint{0.264951in}{1.237839in}}%
\pgfpathlineto{\pgfqpoint{0.278220in}{1.235772in}}%
\pgfpathlineto{\pgfqpoint{0.281666in}{1.233070in}}%
\pgfpathlineto{\pgfqpoint{0.287446in}{1.220143in}}%
\pgfpathlineto{\pgfqpoint{0.288577in}{1.207217in}}%
\pgfpathlineto{\pgfqpoint{0.287541in}{1.194290in}}%
\pgfpathlineto{\pgfqpoint{0.281995in}{1.181364in}}%
\pgfpathlineto{\pgfqpoint{0.278220in}{1.178366in}}%
\pgfpathlineto{\pgfqpoint{0.264951in}{1.176333in}}%
\pgfpathlineto{\pgfqpoint{0.251682in}{1.180649in}}%
\pgfpathclose%
\pgfusepath{fill}%
\end{pgfscope}%
\begin{pgfscope}%
\pgfpathrectangle{\pgfqpoint{0.211875in}{0.211875in}}{\pgfqpoint{1.313625in}{1.279725in}}%
\pgfusepath{clip}%
\pgfsetbuttcap%
\pgfsetroundjoin%
\definecolor{currentfill}{rgb}{0.901975,0.231521,0.249182}%
\pgfsetfillcolor{currentfill}%
\pgfsetlinewidth{0.000000pt}%
\definecolor{currentstroke}{rgb}{0.000000,0.000000,0.000000}%
\pgfsetstrokecolor{currentstroke}%
\pgfsetdash{}{0pt}%
\pgfpathmoveto{\pgfqpoint{0.663019in}{1.167713in}}%
\pgfpathlineto{\pgfqpoint{0.676288in}{1.167578in}}%
\pgfpathlineto{\pgfqpoint{0.689557in}{1.168060in}}%
\pgfpathlineto{\pgfqpoint{0.692775in}{1.168437in}}%
\pgfpathlineto{\pgfqpoint{0.702826in}{1.172139in}}%
\pgfpathlineto{\pgfqpoint{0.706025in}{1.181364in}}%
\pgfpathlineto{\pgfqpoint{0.706825in}{1.194290in}}%
\pgfpathlineto{\pgfqpoint{0.707003in}{1.207217in}}%
\pgfpathlineto{\pgfqpoint{0.706892in}{1.220143in}}%
\pgfpathlineto{\pgfqpoint{0.706237in}{1.233070in}}%
\pgfpathlineto{\pgfqpoint{0.702826in}{1.243110in}}%
\pgfpathlineto{\pgfqpoint{0.694977in}{1.245996in}}%
\pgfpathlineto{\pgfqpoint{0.689557in}{1.246620in}}%
\pgfpathlineto{\pgfqpoint{0.676288in}{1.247056in}}%
\pgfpathlineto{\pgfqpoint{0.663019in}{1.247042in}}%
\pgfpathlineto{\pgfqpoint{0.653083in}{1.245996in}}%
\pgfpathlineto{\pgfqpoint{0.649750in}{1.238407in}}%
\pgfpathlineto{\pgfqpoint{0.649510in}{1.233070in}}%
\pgfpathlineto{\pgfqpoint{0.649359in}{1.220143in}}%
\pgfpathlineto{\pgfqpoint{0.649349in}{1.207217in}}%
\pgfpathlineto{\pgfqpoint{0.649432in}{1.194290in}}%
\pgfpathlineto{\pgfqpoint{0.649749in}{1.181364in}}%
\pgfpathlineto{\pgfqpoint{0.649750in}{1.181345in}}%
\pgfpathlineto{\pgfqpoint{0.656330in}{1.168437in}}%
\pgfpathclose%
\pgfpathmoveto{\pgfqpoint{0.661551in}{1.181364in}}%
\pgfpathlineto{\pgfqpoint{0.656621in}{1.194290in}}%
\pgfpathlineto{\pgfqpoint{0.655647in}{1.207217in}}%
\pgfpathlineto{\pgfqpoint{0.656555in}{1.220143in}}%
\pgfpathlineto{\pgfqpoint{0.661347in}{1.233070in}}%
\pgfpathlineto{\pgfqpoint{0.663019in}{1.234803in}}%
\pgfpathlineto{\pgfqpoint{0.676288in}{1.238739in}}%
\pgfpathlineto{\pgfqpoint{0.689557in}{1.236668in}}%
\pgfpathlineto{\pgfqpoint{0.694184in}{1.233070in}}%
\pgfpathlineto{\pgfqpoint{0.699851in}{1.220143in}}%
\pgfpathlineto{\pgfqpoint{0.700956in}{1.207217in}}%
\pgfpathlineto{\pgfqpoint{0.699779in}{1.194290in}}%
\pgfpathlineto{\pgfqpoint{0.693967in}{1.181364in}}%
\pgfpathlineto{\pgfqpoint{0.689557in}{1.177948in}}%
\pgfpathlineto{\pgfqpoint{0.676288in}{1.175852in}}%
\pgfpathlineto{\pgfqpoint{0.663019in}{1.179850in}}%
\pgfpathclose%
\pgfusepath{fill}%
\end{pgfscope}%
\begin{pgfscope}%
\pgfpathrectangle{\pgfqpoint{0.211875in}{0.211875in}}{\pgfqpoint{1.313625in}{1.279725in}}%
\pgfusepath{clip}%
\pgfsetbuttcap%
\pgfsetroundjoin%
\definecolor{currentfill}{rgb}{0.901975,0.231521,0.249182}%
\pgfsetfillcolor{currentfill}%
\pgfsetlinewidth{0.000000pt}%
\definecolor{currentstroke}{rgb}{0.000000,0.000000,0.000000}%
\pgfsetstrokecolor{currentstroke}%
\pgfsetdash{}{0pt}%
\pgfpathmoveto{\pgfqpoint{0.782439in}{1.170302in}}%
\pgfpathlineto{\pgfqpoint{0.795708in}{1.169471in}}%
\pgfpathlineto{\pgfqpoint{0.808977in}{1.170520in}}%
\pgfpathlineto{\pgfqpoint{0.821597in}{1.181364in}}%
\pgfpathlineto{\pgfqpoint{0.822246in}{1.184452in}}%
\pgfpathlineto{\pgfqpoint{0.823358in}{1.194290in}}%
\pgfpathlineto{\pgfqpoint{0.823708in}{1.207217in}}%
\pgfpathlineto{\pgfqpoint{0.823424in}{1.220143in}}%
\pgfpathlineto{\pgfqpoint{0.822246in}{1.231037in}}%
\pgfpathlineto{\pgfqpoint{0.821836in}{1.233070in}}%
\pgfpathlineto{\pgfqpoint{0.808977in}{1.244168in}}%
\pgfpathlineto{\pgfqpoint{0.795708in}{1.245156in}}%
\pgfpathlineto{\pgfqpoint{0.782439in}{1.244410in}}%
\pgfpathlineto{\pgfqpoint{0.770198in}{1.233070in}}%
\pgfpathlineto{\pgfqpoint{0.769170in}{1.224392in}}%
\pgfpathlineto{\pgfqpoint{0.768889in}{1.220143in}}%
\pgfpathlineto{\pgfqpoint{0.768671in}{1.207217in}}%
\pgfpathlineto{\pgfqpoint{0.768958in}{1.194290in}}%
\pgfpathlineto{\pgfqpoint{0.769170in}{1.191321in}}%
\pgfpathlineto{\pgfqpoint{0.770440in}{1.181364in}}%
\pgfpathclose%
\pgfpathmoveto{\pgfqpoint{0.782860in}{1.181364in}}%
\pgfpathlineto{\pgfqpoint{0.782439in}{1.181589in}}%
\pgfpathlineto{\pgfqpoint{0.776382in}{1.194290in}}%
\pgfpathlineto{\pgfqpoint{0.775151in}{1.207217in}}%
\pgfpathlineto{\pgfqpoint{0.776316in}{1.220143in}}%
\pgfpathlineto{\pgfqpoint{0.782416in}{1.233070in}}%
\pgfpathlineto{\pgfqpoint{0.782439in}{1.233092in}}%
\pgfpathlineto{\pgfqpoint{0.795708in}{1.236854in}}%
\pgfpathlineto{\pgfqpoint{0.808977in}{1.233591in}}%
\pgfpathlineto{\pgfqpoint{0.809581in}{1.233070in}}%
\pgfpathlineto{\pgfqpoint{0.816014in}{1.220143in}}%
\pgfpathlineto{\pgfqpoint{0.817266in}{1.207217in}}%
\pgfpathlineto{\pgfqpoint{0.815947in}{1.194290in}}%
\pgfpathlineto{\pgfqpoint{0.809383in}{1.181364in}}%
\pgfpathlineto{\pgfqpoint{0.808977in}{1.181015in}}%
\pgfpathlineto{\pgfqpoint{0.795708in}{1.177723in}}%
\pgfpathclose%
\pgfusepath{fill}%
\end{pgfscope}%
\begin{pgfscope}%
\pgfpathrectangle{\pgfqpoint{0.211875in}{0.211875in}}{\pgfqpoint{1.313625in}{1.279725in}}%
\pgfusepath{clip}%
\pgfsetbuttcap%
\pgfsetroundjoin%
\definecolor{currentfill}{rgb}{0.901975,0.231521,0.249182}%
\pgfsetfillcolor{currentfill}%
\pgfsetlinewidth{0.000000pt}%
\definecolor{currentstroke}{rgb}{0.000000,0.000000,0.000000}%
\pgfsetstrokecolor{currentstroke}%
\pgfsetdash{}{0pt}%
\pgfpathmoveto{\pgfqpoint{0.901860in}{1.171872in}}%
\pgfpathlineto{\pgfqpoint{0.915129in}{1.170846in}}%
\pgfpathlineto{\pgfqpoint{0.928398in}{1.172598in}}%
\pgfpathlineto{\pgfqpoint{0.937571in}{1.181364in}}%
\pgfpathlineto{\pgfqpoint{0.940102in}{1.194290in}}%
\pgfpathlineto{\pgfqpoint{0.940623in}{1.207217in}}%
\pgfpathlineto{\pgfqpoint{0.940174in}{1.220143in}}%
\pgfpathlineto{\pgfqpoint{0.937796in}{1.233070in}}%
\pgfpathlineto{\pgfqpoint{0.928398in}{1.242096in}}%
\pgfpathlineto{\pgfqpoint{0.915129in}{1.243773in}}%
\pgfpathlineto{\pgfqpoint{0.901860in}{1.242805in}}%
\pgfpathlineto{\pgfqpoint{0.890151in}{1.233070in}}%
\pgfpathlineto{\pgfqpoint{0.888591in}{1.225672in}}%
\pgfpathlineto{\pgfqpoint{0.887977in}{1.220143in}}%
\pgfpathlineto{\pgfqpoint{0.887614in}{1.207217in}}%
\pgfpathlineto{\pgfqpoint{0.888043in}{1.194290in}}%
\pgfpathlineto{\pgfqpoint{0.888591in}{1.189569in}}%
\pgfpathlineto{\pgfqpoint{0.890396in}{1.181364in}}%
\pgfpathclose%
\pgfpathmoveto{\pgfqpoint{0.905255in}{1.181364in}}%
\pgfpathlineto{\pgfqpoint{0.901860in}{1.182739in}}%
\pgfpathlineto{\pgfqpoint{0.895712in}{1.194290in}}%
\pgfpathlineto{\pgfqpoint{0.894261in}{1.207217in}}%
\pgfpathlineto{\pgfqpoint{0.895645in}{1.220143in}}%
\pgfpathlineto{\pgfqpoint{0.901860in}{1.231918in}}%
\pgfpathlineto{\pgfqpoint{0.904687in}{1.233070in}}%
\pgfpathlineto{\pgfqpoint{0.915129in}{1.235471in}}%
\pgfpathlineto{\pgfqpoint{0.922912in}{1.233070in}}%
\pgfpathlineto{\pgfqpoint{0.928398in}{1.229850in}}%
\pgfpathlineto{\pgfqpoint{0.932880in}{1.220143in}}%
\pgfpathlineto{\pgfqpoint{0.934220in}{1.207217in}}%
\pgfpathlineto{\pgfqpoint{0.932817in}{1.194290in}}%
\pgfpathlineto{\pgfqpoint{0.928398in}{1.184807in}}%
\pgfpathlineto{\pgfqpoint{0.922490in}{1.181364in}}%
\pgfpathlineto{\pgfqpoint{0.915129in}{1.179099in}}%
\pgfpathclose%
\pgfusepath{fill}%
\end{pgfscope}%
\begin{pgfscope}%
\pgfpathrectangle{\pgfqpoint{0.211875in}{0.211875in}}{\pgfqpoint{1.313625in}{1.279725in}}%
\pgfusepath{clip}%
\pgfsetbuttcap%
\pgfsetroundjoin%
\definecolor{currentfill}{rgb}{0.901975,0.231521,0.249182}%
\pgfsetfillcolor{currentfill}%
\pgfsetlinewidth{0.000000pt}%
\definecolor{currentstroke}{rgb}{0.000000,0.000000,0.000000}%
\pgfsetstrokecolor{currentstroke}%
\pgfsetdash{}{0pt}%
\pgfpathmoveto{\pgfqpoint{1.021280in}{1.172621in}}%
\pgfpathlineto{\pgfqpoint{1.034549in}{1.171698in}}%
\pgfpathlineto{\pgfqpoint{1.047818in}{1.174231in}}%
\pgfpathlineto{\pgfqpoint{1.054500in}{1.181364in}}%
\pgfpathlineto{\pgfqpoint{1.057325in}{1.194290in}}%
\pgfpathlineto{\pgfqpoint{1.057900in}{1.207217in}}%
\pgfpathlineto{\pgfqpoint{1.057393in}{1.220143in}}%
\pgfpathlineto{\pgfqpoint{1.054715in}{1.233070in}}%
\pgfpathlineto{\pgfqpoint{1.047818in}{1.240475in}}%
\pgfpathlineto{\pgfqpoint{1.034549in}{1.242917in}}%
\pgfpathlineto{\pgfqpoint{1.021280in}{1.242034in}}%
\pgfpathlineto{\pgfqpoint{1.009295in}{1.233070in}}%
\pgfpathlineto{\pgfqpoint{1.008011in}{1.228952in}}%
\pgfpathlineto{\pgfqpoint{1.006649in}{1.220143in}}%
\pgfpathlineto{\pgfqpoint{1.006199in}{1.207217in}}%
\pgfpathlineto{\pgfqpoint{1.006714in}{1.194290in}}%
\pgfpathlineto{\pgfqpoint{1.008011in}{1.186150in}}%
\pgfpathlineto{\pgfqpoint{1.009546in}{1.181364in}}%
\pgfpathclose%
\pgfpathmoveto{\pgfqpoint{1.026786in}{1.181364in}}%
\pgfpathlineto{\pgfqpoint{1.021280in}{1.182959in}}%
\pgfpathlineto{\pgfqpoint{1.014562in}{1.194290in}}%
\pgfpathlineto{\pgfqpoint{1.012927in}{1.207217in}}%
\pgfpathlineto{\pgfqpoint{1.014492in}{1.220143in}}%
\pgfpathlineto{\pgfqpoint{1.021280in}{1.231679in}}%
\pgfpathlineto{\pgfqpoint{1.026053in}{1.233070in}}%
\pgfpathlineto{\pgfqpoint{1.034549in}{1.234560in}}%
\pgfpathlineto{\pgfqpoint{1.038655in}{1.233070in}}%
\pgfpathlineto{\pgfqpoint{1.047818in}{1.226248in}}%
\pgfpathlineto{\pgfqpoint{1.050333in}{1.220143in}}%
\pgfpathlineto{\pgfqpoint{1.051705in}{1.207217in}}%
\pgfpathlineto{\pgfqpoint{1.050273in}{1.194290in}}%
\pgfpathlineto{\pgfqpoint{1.047818in}{1.188392in}}%
\pgfpathlineto{\pgfqpoint{1.038302in}{1.181364in}}%
\pgfpathlineto{\pgfqpoint{1.034549in}{1.180005in}}%
\pgfpathclose%
\pgfusepath{fill}%
\end{pgfscope}%
\begin{pgfscope}%
\pgfpathrectangle{\pgfqpoint{0.211875in}{0.211875in}}{\pgfqpoint{1.313625in}{1.279725in}}%
\pgfusepath{clip}%
\pgfsetbuttcap%
\pgfsetroundjoin%
\definecolor{currentfill}{rgb}{0.901975,0.231521,0.249182}%
\pgfsetfillcolor{currentfill}%
\pgfsetlinewidth{0.000000pt}%
\definecolor{currentstroke}{rgb}{0.000000,0.000000,0.000000}%
\pgfsetstrokecolor{currentstroke}%
\pgfsetdash{}{0pt}%
\pgfpathmoveto{\pgfqpoint{1.140701in}{1.172696in}}%
\pgfpathlineto{\pgfqpoint{1.153970in}{1.172033in}}%
\pgfpathlineto{\pgfqpoint{1.167239in}{1.175328in}}%
\pgfpathlineto{\pgfqpoint{1.172264in}{1.181364in}}%
\pgfpathlineto{\pgfqpoint{1.175106in}{1.194290in}}%
\pgfpathlineto{\pgfqpoint{1.175683in}{1.207217in}}%
\pgfpathlineto{\pgfqpoint{1.175173in}{1.220143in}}%
\pgfpathlineto{\pgfqpoint{1.172473in}{1.233070in}}%
\pgfpathlineto{\pgfqpoint{1.167239in}{1.239399in}}%
\pgfpathlineto{\pgfqpoint{1.153970in}{1.242583in}}%
\pgfpathlineto{\pgfqpoint{1.140701in}{1.241945in}}%
\pgfpathlineto{\pgfqpoint{1.127512in}{1.233070in}}%
\pgfpathlineto{\pgfqpoint{1.127432in}{1.232881in}}%
\pgfpathlineto{\pgfqpoint{1.124919in}{1.220143in}}%
\pgfpathlineto{\pgfqpoint{1.124433in}{1.207217in}}%
\pgfpathlineto{\pgfqpoint{1.124983in}{1.194290in}}%
\pgfpathlineto{\pgfqpoint{1.127432in}{1.182157in}}%
\pgfpathlineto{\pgfqpoint{1.127775in}{1.181364in}}%
\pgfpathclose%
\pgfpathmoveto{\pgfqpoint{1.146322in}{1.181364in}}%
\pgfpathlineto{\pgfqpoint{1.140701in}{1.182409in}}%
\pgfpathlineto{\pgfqpoint{1.132852in}{1.194290in}}%
\pgfpathlineto{\pgfqpoint{1.131071in}{1.207217in}}%
\pgfpathlineto{\pgfqpoint{1.132779in}{1.220143in}}%
\pgfpathlineto{\pgfqpoint{1.140701in}{1.232220in}}%
\pgfpathlineto{\pgfqpoint{1.145244in}{1.233070in}}%
\pgfpathlineto{\pgfqpoint{1.153970in}{1.234115in}}%
\pgfpathlineto{\pgfqpoint{1.156467in}{1.233070in}}%
\pgfpathlineto{\pgfqpoint{1.167239in}{1.223022in}}%
\pgfpathlineto{\pgfqpoint{1.168289in}{1.220143in}}%
\pgfpathlineto{\pgfqpoint{1.169646in}{1.207217in}}%
\pgfpathlineto{\pgfqpoint{1.168230in}{1.194290in}}%
\pgfpathlineto{\pgfqpoint{1.167239in}{1.191603in}}%
\pgfpathlineto{\pgfqpoint{1.156158in}{1.181364in}}%
\pgfpathlineto{\pgfqpoint{1.153970in}{1.180450in}}%
\pgfpathclose%
\pgfusepath{fill}%
\end{pgfscope}%
\begin{pgfscope}%
\pgfpathrectangle{\pgfqpoint{0.211875in}{0.211875in}}{\pgfqpoint{1.313625in}{1.279725in}}%
\pgfusepath{clip}%
\pgfsetbuttcap%
\pgfsetroundjoin%
\definecolor{currentfill}{rgb}{0.901975,0.231521,0.249182}%
\pgfsetfillcolor{currentfill}%
\pgfsetlinewidth{0.000000pt}%
\definecolor{currentstroke}{rgb}{0.000000,0.000000,0.000000}%
\pgfsetstrokecolor{currentstroke}%
\pgfsetdash{}{0pt}%
\pgfpathmoveto{\pgfqpoint{1.246852in}{1.178846in}}%
\pgfpathlineto{\pgfqpoint{1.260121in}{1.172193in}}%
\pgfpathlineto{\pgfqpoint{1.273390in}{1.171837in}}%
\pgfpathlineto{\pgfqpoint{1.286659in}{1.175731in}}%
\pgfpathlineto{\pgfqpoint{1.290786in}{1.181364in}}%
\pgfpathlineto{\pgfqpoint{1.293385in}{1.194290in}}%
\pgfpathlineto{\pgfqpoint{1.293912in}{1.207217in}}%
\pgfpathlineto{\pgfqpoint{1.293450in}{1.220143in}}%
\pgfpathlineto{\pgfqpoint{1.290994in}{1.233070in}}%
\pgfpathlineto{\pgfqpoint{1.286659in}{1.239034in}}%
\pgfpathlineto{\pgfqpoint{1.273390in}{1.242786in}}%
\pgfpathlineto{\pgfqpoint{1.260121in}{1.242441in}}%
\pgfpathlineto{\pgfqpoint{1.246852in}{1.235985in}}%
\pgfpathlineto{\pgfqpoint{1.245271in}{1.233070in}}%
\pgfpathlineto{\pgfqpoint{1.242782in}{1.220143in}}%
\pgfpathlineto{\pgfqpoint{1.242311in}{1.207217in}}%
\pgfpathlineto{\pgfqpoint{1.242845in}{1.194290in}}%
\pgfpathlineto{\pgfqpoint{1.245471in}{1.181364in}}%
\pgfpathclose%
\pgfpathmoveto{\pgfqpoint{1.259917in}{1.181364in}}%
\pgfpathlineto{\pgfqpoint{1.250459in}{1.194290in}}%
\pgfpathlineto{\pgfqpoint{1.248573in}{1.207217in}}%
\pgfpathlineto{\pgfqpoint{1.250380in}{1.220143in}}%
\pgfpathlineto{\pgfqpoint{1.259685in}{1.233070in}}%
\pgfpathlineto{\pgfqpoint{1.260121in}{1.233333in}}%
\pgfpathlineto{\pgfqpoint{1.273390in}{1.234147in}}%
\pgfpathlineto{\pgfqpoint{1.275652in}{1.233070in}}%
\pgfpathlineto{\pgfqpoint{1.286659in}{1.220244in}}%
\pgfpathlineto{\pgfqpoint{1.286691in}{1.220143in}}%
\pgfpathlineto{\pgfqpoint{1.287987in}{1.207217in}}%
\pgfpathlineto{\pgfqpoint{1.286659in}{1.194504in}}%
\pgfpathlineto{\pgfqpoint{1.286617in}{1.194290in}}%
\pgfpathlineto{\pgfqpoint{1.275373in}{1.181364in}}%
\pgfpathlineto{\pgfqpoint{1.273390in}{1.180422in}}%
\pgfpathlineto{\pgfqpoint{1.260121in}{1.181240in}}%
\pgfpathclose%
\pgfusepath{fill}%
\end{pgfscope}%
\begin{pgfscope}%
\pgfpathrectangle{\pgfqpoint{0.211875in}{0.211875in}}{\pgfqpoint{1.313625in}{1.279725in}}%
\pgfusepath{clip}%
\pgfsetbuttcap%
\pgfsetroundjoin%
\definecolor{currentfill}{rgb}{0.901975,0.231521,0.249182}%
\pgfsetfillcolor{currentfill}%
\pgfsetlinewidth{0.000000pt}%
\definecolor{currentstroke}{rgb}{0.000000,0.000000,0.000000}%
\pgfsetstrokecolor{currentstroke}%
\pgfsetdash{}{0pt}%
\pgfpathmoveto{\pgfqpoint{1.366273in}{1.175607in}}%
\pgfpathlineto{\pgfqpoint{1.379542in}{1.171170in}}%
\pgfpathlineto{\pgfqpoint{1.392811in}{1.171072in}}%
\pgfpathlineto{\pgfqpoint{1.406080in}{1.175157in}}%
\pgfpathlineto{\pgfqpoint{1.410021in}{1.181364in}}%
\pgfpathlineto{\pgfqpoint{1.412121in}{1.194290in}}%
\pgfpathlineto{\pgfqpoint{1.412549in}{1.207217in}}%
\pgfpathlineto{\pgfqpoint{1.412187in}{1.220143in}}%
\pgfpathlineto{\pgfqpoint{1.410231in}{1.233070in}}%
\pgfpathlineto{\pgfqpoint{1.406080in}{1.239672in}}%
\pgfpathlineto{\pgfqpoint{1.392811in}{1.243562in}}%
\pgfpathlineto{\pgfqpoint{1.379542in}{1.243462in}}%
\pgfpathlineto{\pgfqpoint{1.366273in}{1.239200in}}%
\pgfpathlineto{\pgfqpoint{1.362368in}{1.233070in}}%
\pgfpathlineto{\pgfqpoint{1.360215in}{1.220143in}}%
\pgfpathlineto{\pgfqpoint{1.359809in}{1.207217in}}%
\pgfpathlineto{\pgfqpoint{1.360279in}{1.194290in}}%
\pgfpathlineto{\pgfqpoint{1.362572in}{1.181364in}}%
\pgfpathclose%
\pgfpathmoveto{\pgfqpoint{1.376878in}{1.181364in}}%
\pgfpathlineto{\pgfqpoint{1.367192in}{1.194290in}}%
\pgfpathlineto{\pgfqpoint{1.366273in}{1.199998in}}%
\pgfpathlineto{\pgfqpoint{1.365604in}{1.207217in}}%
\pgfpathlineto{\pgfqpoint{1.366273in}{1.214754in}}%
\pgfpathlineto{\pgfqpoint{1.367105in}{1.220143in}}%
\pgfpathlineto{\pgfqpoint{1.376623in}{1.233070in}}%
\pgfpathlineto{\pgfqpoint{1.379542in}{1.234648in}}%
\pgfpathlineto{\pgfqpoint{1.392811in}{1.234687in}}%
\pgfpathlineto{\pgfqpoint{1.395825in}{1.233070in}}%
\pgfpathlineto{\pgfqpoint{1.405212in}{1.220143in}}%
\pgfpathlineto{\pgfqpoint{1.406080in}{1.214332in}}%
\pgfpathlineto{\pgfqpoint{1.406695in}{1.207217in}}%
\pgfpathlineto{\pgfqpoint{1.406080in}{1.200414in}}%
\pgfpathlineto{\pgfqpoint{1.405124in}{1.194290in}}%
\pgfpathlineto{\pgfqpoint{1.395564in}{1.181364in}}%
\pgfpathlineto{\pgfqpoint{1.392811in}{1.179891in}}%
\pgfpathlineto{\pgfqpoint{1.379542in}{1.179928in}}%
\pgfpathclose%
\pgfusepath{fill}%
\end{pgfscope}%
\begin{pgfscope}%
\pgfpathrectangle{\pgfqpoint{0.211875in}{0.211875in}}{\pgfqpoint{1.313625in}{1.279725in}}%
\pgfusepath{clip}%
\pgfsetbuttcap%
\pgfsetroundjoin%
\definecolor{currentfill}{rgb}{0.901975,0.231521,0.249182}%
\pgfsetfillcolor{currentfill}%
\pgfsetlinewidth{0.000000pt}%
\definecolor{currentstroke}{rgb}{0.000000,0.000000,0.000000}%
\pgfsetstrokecolor{currentstroke}%
\pgfsetdash{}{0pt}%
\pgfpathmoveto{\pgfqpoint{1.485693in}{1.172234in}}%
\pgfpathlineto{\pgfqpoint{1.498962in}{1.169657in}}%
\pgfpathlineto{\pgfqpoint{1.512231in}{1.169673in}}%
\pgfpathlineto{\pgfqpoint{1.525500in}{1.173091in}}%
\pgfpathlineto{\pgfqpoint{1.525500in}{1.181364in}}%
\pgfpathlineto{\pgfqpoint{1.525500in}{1.194290in}}%
\pgfpathlineto{\pgfqpoint{1.525500in}{1.203928in}}%
\pgfpathlineto{\pgfqpoint{1.524317in}{1.194290in}}%
\pgfpathlineto{\pgfqpoint{1.516514in}{1.181364in}}%
\pgfpathlineto{\pgfqpoint{1.512231in}{1.178800in}}%
\pgfpathlineto{\pgfqpoint{1.498962in}{1.178198in}}%
\pgfpathlineto{\pgfqpoint{1.492355in}{1.181364in}}%
\pgfpathlineto{\pgfqpoint{1.485693in}{1.188938in}}%
\pgfpathlineto{\pgfqpoint{1.483928in}{1.194290in}}%
\pgfpathlineto{\pgfqpoint{1.482766in}{1.207217in}}%
\pgfpathlineto{\pgfqpoint{1.483870in}{1.220143in}}%
\pgfpathlineto{\pgfqpoint{1.485693in}{1.225746in}}%
\pgfpathlineto{\pgfqpoint{1.492066in}{1.233070in}}%
\pgfpathlineto{\pgfqpoint{1.498962in}{1.236384in}}%
\pgfpathlineto{\pgfqpoint{1.512231in}{1.235792in}}%
\pgfpathlineto{\pgfqpoint{1.516764in}{1.233070in}}%
\pgfpathlineto{\pgfqpoint{1.524399in}{1.220143in}}%
\pgfpathlineto{\pgfqpoint{1.525500in}{1.210689in}}%
\pgfpathlineto{\pgfqpoint{1.525500in}{1.220143in}}%
\pgfpathlineto{\pgfqpoint{1.525500in}{1.233070in}}%
\pgfpathlineto{\pgfqpoint{1.525500in}{1.241850in}}%
\pgfpathlineto{\pgfqpoint{1.512231in}{1.244980in}}%
\pgfpathlineto{\pgfqpoint{1.498962in}{1.244978in}}%
\pgfpathlineto{\pgfqpoint{1.485693in}{1.242559in}}%
\pgfpathlineto{\pgfqpoint{1.478739in}{1.233070in}}%
\pgfpathlineto{\pgfqpoint{1.477181in}{1.220143in}}%
\pgfpathlineto{\pgfqpoint{1.476892in}{1.207217in}}%
\pgfpathlineto{\pgfqpoint{1.477248in}{1.194290in}}%
\pgfpathlineto{\pgfqpoint{1.478950in}{1.181364in}}%
\pgfpathclose%
\pgfusepath{fill}%
\end{pgfscope}%
\begin{pgfscope}%
\pgfpathrectangle{\pgfqpoint{0.211875in}{0.211875in}}{\pgfqpoint{1.313625in}{1.279725in}}%
\pgfusepath{clip}%
\pgfsetbuttcap%
\pgfsetroundjoin%
\definecolor{currentfill}{rgb}{0.901975,0.231521,0.249182}%
\pgfsetfillcolor{currentfill}%
\pgfsetlinewidth{0.000000pt}%
\definecolor{currentstroke}{rgb}{0.000000,0.000000,0.000000}%
\pgfsetstrokecolor{currentstroke}%
\pgfsetdash{}{0pt}%
\pgfpathmoveto{\pgfqpoint{0.225144in}{1.253489in}}%
\pgfpathlineto{\pgfqpoint{0.230909in}{1.258923in}}%
\pgfpathlineto{\pgfqpoint{0.233497in}{1.271849in}}%
\pgfpathlineto{\pgfqpoint{0.233978in}{1.284776in}}%
\pgfpathlineto{\pgfqpoint{0.233773in}{1.297702in}}%
\pgfpathlineto{\pgfqpoint{0.232538in}{1.310629in}}%
\pgfpathlineto{\pgfqpoint{0.225144in}{1.322615in}}%
\pgfpathlineto{\pgfqpoint{0.221486in}{1.323555in}}%
\pgfpathlineto{\pgfqpoint{0.211875in}{1.324855in}}%
\pgfpathlineto{\pgfqpoint{0.211875in}{1.323555in}}%
\pgfpathlineto{\pgfqpoint{0.211875in}{1.316012in}}%
\pgfpathlineto{\pgfqpoint{0.221609in}{1.310629in}}%
\pgfpathlineto{\pgfqpoint{0.225144in}{1.305455in}}%
\pgfpathlineto{\pgfqpoint{0.227264in}{1.297702in}}%
\pgfpathlineto{\pgfqpoint{0.227833in}{1.284776in}}%
\pgfpathlineto{\pgfqpoint{0.225861in}{1.271849in}}%
\pgfpathlineto{\pgfqpoint{0.225144in}{1.270148in}}%
\pgfpathlineto{\pgfqpoint{0.211875in}{1.260178in}}%
\pgfpathlineto{\pgfqpoint{0.211875in}{1.258923in}}%
\pgfpathlineto{\pgfqpoint{0.211875in}{1.251613in}}%
\pgfpathclose%
\pgfusepath{fill}%
\end{pgfscope}%
\begin{pgfscope}%
\pgfpathrectangle{\pgfqpoint{0.211875in}{0.211875in}}{\pgfqpoint{1.313625in}{1.279725in}}%
\pgfusepath{clip}%
\pgfsetbuttcap%
\pgfsetroundjoin%
\definecolor{currentfill}{rgb}{0.901975,0.231521,0.249182}%
\pgfsetfillcolor{currentfill}%
\pgfsetlinewidth{0.000000pt}%
\definecolor{currentstroke}{rgb}{0.000000,0.000000,0.000000}%
\pgfsetstrokecolor{currentstroke}%
\pgfsetdash{}{0pt}%
\pgfpathmoveto{\pgfqpoint{0.716095in}{1.251311in}}%
\pgfpathlineto{\pgfqpoint{0.729364in}{1.249685in}}%
\pgfpathlineto{\pgfqpoint{0.742633in}{1.249795in}}%
\pgfpathlineto{\pgfqpoint{0.755902in}{1.251563in}}%
\pgfpathlineto{\pgfqpoint{0.763065in}{1.258923in}}%
\pgfpathlineto{\pgfqpoint{0.764929in}{1.271849in}}%
\pgfpathlineto{\pgfqpoint{0.765351in}{1.284776in}}%
\pgfpathlineto{\pgfqpoint{0.765331in}{1.297702in}}%
\pgfpathlineto{\pgfqpoint{0.764780in}{1.310629in}}%
\pgfpathlineto{\pgfqpoint{0.760929in}{1.323555in}}%
\pgfpathlineto{\pgfqpoint{0.755902in}{1.326241in}}%
\pgfpathlineto{\pgfqpoint{0.742633in}{1.327642in}}%
\pgfpathlineto{\pgfqpoint{0.729364in}{1.327783in}}%
\pgfpathlineto{\pgfqpoint{0.716095in}{1.326787in}}%
\pgfpathlineto{\pgfqpoint{0.711180in}{1.323555in}}%
\pgfpathlineto{\pgfqpoint{0.709142in}{1.310629in}}%
\pgfpathlineto{\pgfqpoint{0.708886in}{1.297702in}}%
\pgfpathlineto{\pgfqpoint{0.708904in}{1.284776in}}%
\pgfpathlineto{\pgfqpoint{0.709174in}{1.271849in}}%
\pgfpathlineto{\pgfqpoint{0.710349in}{1.258923in}}%
\pgfpathclose%
\pgfpathmoveto{\pgfqpoint{0.728182in}{1.258923in}}%
\pgfpathlineto{\pgfqpoint{0.716634in}{1.271849in}}%
\pgfpathlineto{\pgfqpoint{0.716095in}{1.274327in}}%
\pgfpathlineto{\pgfqpoint{0.714799in}{1.284776in}}%
\pgfpathlineto{\pgfqpoint{0.715125in}{1.297702in}}%
\pgfpathlineto{\pgfqpoint{0.716095in}{1.302952in}}%
\pgfpathlineto{\pgfqpoint{0.718993in}{1.310629in}}%
\pgfpathlineto{\pgfqpoint{0.729364in}{1.318542in}}%
\pgfpathlineto{\pgfqpoint{0.742633in}{1.318946in}}%
\pgfpathlineto{\pgfqpoint{0.755853in}{1.310629in}}%
\pgfpathlineto{\pgfqpoint{0.755902in}{1.310546in}}%
\pgfpathlineto{\pgfqpoint{0.759087in}{1.297702in}}%
\pgfpathlineto{\pgfqpoint{0.759444in}{1.284776in}}%
\pgfpathlineto{\pgfqpoint{0.757652in}{1.271849in}}%
\pgfpathlineto{\pgfqpoint{0.755902in}{1.267544in}}%
\pgfpathlineto{\pgfqpoint{0.745078in}{1.258923in}}%
\pgfpathlineto{\pgfqpoint{0.742633in}{1.258027in}}%
\pgfpathlineto{\pgfqpoint{0.729364in}{1.258414in}}%
\pgfpathclose%
\pgfusepath{fill}%
\end{pgfscope}%
\begin{pgfscope}%
\pgfpathrectangle{\pgfqpoint{0.211875in}{0.211875in}}{\pgfqpoint{1.313625in}{1.279725in}}%
\pgfusepath{clip}%
\pgfsetbuttcap%
\pgfsetroundjoin%
\definecolor{currentfill}{rgb}{0.901975,0.231521,0.249182}%
\pgfsetfillcolor{currentfill}%
\pgfsetlinewidth{0.000000pt}%
\definecolor{currentstroke}{rgb}{0.000000,0.000000,0.000000}%
\pgfsetstrokecolor{currentstroke}%
\pgfsetdash{}{0pt}%
\pgfpathmoveto{\pgfqpoint{0.835515in}{1.254220in}}%
\pgfpathlineto{\pgfqpoint{0.848784in}{1.251348in}}%
\pgfpathlineto{\pgfqpoint{0.862053in}{1.251507in}}%
\pgfpathlineto{\pgfqpoint{0.875322in}{1.255138in}}%
\pgfpathlineto{\pgfqpoint{0.878530in}{1.258923in}}%
\pgfpathlineto{\pgfqpoint{0.881566in}{1.271849in}}%
\pgfpathlineto{\pgfqpoint{0.882230in}{1.284776in}}%
\pgfpathlineto{\pgfqpoint{0.882150in}{1.297702in}}%
\pgfpathlineto{\pgfqpoint{0.881123in}{1.310629in}}%
\pgfpathlineto{\pgfqpoint{0.875322in}{1.322654in}}%
\pgfpathlineto{\pgfqpoint{0.873354in}{1.323555in}}%
\pgfpathlineto{\pgfqpoint{0.862053in}{1.325946in}}%
\pgfpathlineto{\pgfqpoint{0.848784in}{1.326104in}}%
\pgfpathlineto{\pgfqpoint{0.835515in}{1.323674in}}%
\pgfpathlineto{\pgfqpoint{0.835305in}{1.323555in}}%
\pgfpathlineto{\pgfqpoint{0.829033in}{1.310629in}}%
\pgfpathlineto{\pgfqpoint{0.828187in}{1.297702in}}%
\pgfpathlineto{\pgfqpoint{0.828131in}{1.284776in}}%
\pgfpathlineto{\pgfqpoint{0.828703in}{1.271849in}}%
\pgfpathlineto{\pgfqpoint{0.831356in}{1.258923in}}%
\pgfpathclose%
\pgfpathmoveto{\pgfqpoint{0.836326in}{1.271849in}}%
\pgfpathlineto{\pgfqpoint{0.835515in}{1.274792in}}%
\pgfpathlineto{\pgfqpoint{0.834063in}{1.284776in}}%
\pgfpathlineto{\pgfqpoint{0.834464in}{1.297702in}}%
\pgfpathlineto{\pgfqpoint{0.835515in}{1.302510in}}%
\pgfpathlineto{\pgfqpoint{0.839429in}{1.310629in}}%
\pgfpathlineto{\pgfqpoint{0.848784in}{1.316990in}}%
\pgfpathlineto{\pgfqpoint{0.862053in}{1.316807in}}%
\pgfpathlineto{\pgfqpoint{0.870778in}{1.310629in}}%
\pgfpathlineto{\pgfqpoint{0.875322in}{1.300786in}}%
\pgfpathlineto{\pgfqpoint{0.875982in}{1.297702in}}%
\pgfpathlineto{\pgfqpoint{0.876396in}{1.284776in}}%
\pgfpathlineto{\pgfqpoint{0.875322in}{1.277243in}}%
\pgfpathlineto{\pgfqpoint{0.873889in}{1.271849in}}%
\pgfpathlineto{\pgfqpoint{0.862053in}{1.260228in}}%
\pgfpathlineto{\pgfqpoint{0.848784in}{1.260021in}}%
\pgfpathclose%
\pgfusepath{fill}%
\end{pgfscope}%
\begin{pgfscope}%
\pgfpathrectangle{\pgfqpoint{0.211875in}{0.211875in}}{\pgfqpoint{1.313625in}{1.279725in}}%
\pgfusepath{clip}%
\pgfsetbuttcap%
\pgfsetroundjoin%
\definecolor{currentfill}{rgb}{0.901975,0.231521,0.249182}%
\pgfsetfillcolor{currentfill}%
\pgfsetlinewidth{0.000000pt}%
\definecolor{currentstroke}{rgb}{0.000000,0.000000,0.000000}%
\pgfsetstrokecolor{currentstroke}%
\pgfsetdash{}{0pt}%
\pgfpathmoveto{\pgfqpoint{0.954936in}{1.255484in}}%
\pgfpathlineto{\pgfqpoint{0.968205in}{1.252383in}}%
\pgfpathlineto{\pgfqpoint{0.981473in}{1.252759in}}%
\pgfpathlineto{\pgfqpoint{0.994742in}{1.258683in}}%
\pgfpathlineto{\pgfqpoint{0.994916in}{1.258923in}}%
\pgfpathlineto{\pgfqpoint{0.998712in}{1.271849in}}%
\pgfpathlineto{\pgfqpoint{0.999531in}{1.284776in}}%
\pgfpathlineto{\pgfqpoint{0.999412in}{1.297702in}}%
\pgfpathlineto{\pgfqpoint{0.998082in}{1.310629in}}%
\pgfpathlineto{\pgfqpoint{0.994742in}{1.318804in}}%
\pgfpathlineto{\pgfqpoint{0.986449in}{1.323555in}}%
\pgfpathlineto{\pgfqpoint{0.981473in}{1.324713in}}%
\pgfpathlineto{\pgfqpoint{0.968205in}{1.325059in}}%
\pgfpathlineto{\pgfqpoint{0.959272in}{1.323555in}}%
\pgfpathlineto{\pgfqpoint{0.954936in}{1.322139in}}%
\pgfpathlineto{\pgfqpoint{0.948335in}{1.310629in}}%
\pgfpathlineto{\pgfqpoint{0.947070in}{1.297702in}}%
\pgfpathlineto{\pgfqpoint{0.946963in}{1.284776in}}%
\pgfpathlineto{\pgfqpoint{0.947751in}{1.271849in}}%
\pgfpathlineto{\pgfqpoint{0.951444in}{1.258923in}}%
\pgfpathclose%
\pgfpathmoveto{\pgfqpoint{0.955346in}{1.271849in}}%
\pgfpathlineto{\pgfqpoint{0.954936in}{1.273040in}}%
\pgfpathlineto{\pgfqpoint{0.952972in}{1.284776in}}%
\pgfpathlineto{\pgfqpoint{0.953428in}{1.297702in}}%
\pgfpathlineto{\pgfqpoint{0.954936in}{1.303656in}}%
\pgfpathlineto{\pgfqpoint{0.959167in}{1.310629in}}%
\pgfpathlineto{\pgfqpoint{0.968205in}{1.316066in}}%
\pgfpathlineto{\pgfqpoint{0.981473in}{1.315107in}}%
\pgfpathlineto{\pgfqpoint{0.987126in}{1.310629in}}%
\pgfpathlineto{\pgfqpoint{0.992692in}{1.297702in}}%
\pgfpathlineto{\pgfqpoint{0.993325in}{1.284776in}}%
\pgfpathlineto{\pgfqpoint{0.990276in}{1.271849in}}%
\pgfpathlineto{\pgfqpoint{0.981473in}{1.262171in}}%
\pgfpathlineto{\pgfqpoint{0.968205in}{1.261055in}}%
\pgfpathclose%
\pgfusepath{fill}%
\end{pgfscope}%
\begin{pgfscope}%
\pgfpathrectangle{\pgfqpoint{0.211875in}{0.211875in}}{\pgfqpoint{1.313625in}{1.279725in}}%
\pgfusepath{clip}%
\pgfsetbuttcap%
\pgfsetroundjoin%
\definecolor{currentfill}{rgb}{0.901975,0.231521,0.249182}%
\pgfsetfillcolor{currentfill}%
\pgfsetlinewidth{0.000000pt}%
\definecolor{currentstroke}{rgb}{0.000000,0.000000,0.000000}%
\pgfsetstrokecolor{currentstroke}%
\pgfsetdash{}{0pt}%
\pgfpathmoveto{\pgfqpoint{1.074356in}{1.255667in}}%
\pgfpathlineto{\pgfqpoint{1.087625in}{1.252856in}}%
\pgfpathlineto{\pgfqpoint{1.100894in}{1.253526in}}%
\pgfpathlineto{\pgfqpoint{1.111556in}{1.258923in}}%
\pgfpathlineto{\pgfqpoint{1.114163in}{1.263116in}}%
\pgfpathlineto{\pgfqpoint{1.116325in}{1.271849in}}%
\pgfpathlineto{\pgfqpoint{1.117215in}{1.284776in}}%
\pgfpathlineto{\pgfqpoint{1.117079in}{1.297702in}}%
\pgfpathlineto{\pgfqpoint{1.115609in}{1.310629in}}%
\pgfpathlineto{\pgfqpoint{1.114163in}{1.314952in}}%
\pgfpathlineto{\pgfqpoint{1.102508in}{1.323555in}}%
\pgfpathlineto{\pgfqpoint{1.100894in}{1.323968in}}%
\pgfpathlineto{\pgfqpoint{1.087625in}{1.324578in}}%
\pgfpathlineto{\pgfqpoint{1.080806in}{1.323555in}}%
\pgfpathlineto{\pgfqpoint{1.074356in}{1.321844in}}%
\pgfpathlineto{\pgfqpoint{1.067034in}{1.310629in}}%
\pgfpathlineto{\pgfqpoint{1.065520in}{1.297702in}}%
\pgfpathlineto{\pgfqpoint{1.065382in}{1.284776in}}%
\pgfpathlineto{\pgfqpoint{1.066301in}{1.271849in}}%
\pgfpathlineto{\pgfqpoint{1.070613in}{1.258923in}}%
\pgfpathclose%
\pgfpathmoveto{\pgfqpoint{1.073874in}{1.271849in}}%
\pgfpathlineto{\pgfqpoint{1.071510in}{1.284776in}}%
\pgfpathlineto{\pgfqpoint{1.072002in}{1.297702in}}%
\pgfpathlineto{\pgfqpoint{1.074356in}{1.305849in}}%
\pgfpathlineto{\pgfqpoint{1.077993in}{1.310629in}}%
\pgfpathlineto{\pgfqpoint{1.087625in}{1.315703in}}%
\pgfpathlineto{\pgfqpoint{1.100894in}{1.313859in}}%
\pgfpathlineto{\pgfqpoint{1.104552in}{1.310629in}}%
\pgfpathlineto{\pgfqpoint{1.109949in}{1.297702in}}%
\pgfpathlineto{\pgfqpoint{1.110564in}{1.284776in}}%
\pgfpathlineto{\pgfqpoint{1.107619in}{1.271849in}}%
\pgfpathlineto{\pgfqpoint{1.100894in}{1.263603in}}%
\pgfpathlineto{\pgfqpoint{1.087625in}{1.261458in}}%
\pgfpathlineto{\pgfqpoint{1.074356in}{1.270829in}}%
\pgfpathclose%
\pgfusepath{fill}%
\end{pgfscope}%
\begin{pgfscope}%
\pgfpathrectangle{\pgfqpoint{0.211875in}{0.211875in}}{\pgfqpoint{1.313625in}{1.279725in}}%
\pgfusepath{clip}%
\pgfsetbuttcap%
\pgfsetroundjoin%
\definecolor{currentfill}{rgb}{0.901975,0.231521,0.249182}%
\pgfsetfillcolor{currentfill}%
\pgfsetlinewidth{0.000000pt}%
\definecolor{currentstroke}{rgb}{0.000000,0.000000,0.000000}%
\pgfsetstrokecolor{currentstroke}%
\pgfsetdash{}{0pt}%
\pgfpathmoveto{\pgfqpoint{1.193777in}{1.255081in}}%
\pgfpathlineto{\pgfqpoint{1.207045in}{1.252808in}}%
\pgfpathlineto{\pgfqpoint{1.220314in}{1.253759in}}%
\pgfpathlineto{\pgfqpoint{1.229522in}{1.258923in}}%
\pgfpathlineto{\pgfqpoint{1.233583in}{1.267754in}}%
\pgfpathlineto{\pgfqpoint{1.234380in}{1.271849in}}%
\pgfpathlineto{\pgfqpoint{1.235257in}{1.284776in}}%
\pgfpathlineto{\pgfqpoint{1.235125in}{1.297702in}}%
\pgfpathlineto{\pgfqpoint{1.233681in}{1.310629in}}%
\pgfpathlineto{\pgfqpoint{1.233583in}{1.311006in}}%
\pgfpathlineto{\pgfqpoint{1.221049in}{1.323555in}}%
\pgfpathlineto{\pgfqpoint{1.220314in}{1.323762in}}%
\pgfpathlineto{\pgfqpoint{1.207045in}{1.324620in}}%
\pgfpathlineto{\pgfqpoint{1.198989in}{1.323555in}}%
\pgfpathlineto{\pgfqpoint{1.193777in}{1.322433in}}%
\pgfpathlineto{\pgfqpoint{1.185093in}{1.310629in}}%
\pgfpathlineto{\pgfqpoint{1.183499in}{1.297702in}}%
\pgfpathlineto{\pgfqpoint{1.183352in}{1.284776in}}%
\pgfpathlineto{\pgfqpoint{1.184317in}{1.271849in}}%
\pgfpathlineto{\pgfqpoint{1.188823in}{1.258923in}}%
\pgfpathclose%
\pgfpathmoveto{\pgfqpoint{1.192088in}{1.271849in}}%
\pgfpathlineto{\pgfqpoint{1.189646in}{1.284776in}}%
\pgfpathlineto{\pgfqpoint{1.190156in}{1.297702in}}%
\pgfpathlineto{\pgfqpoint{1.193777in}{1.308800in}}%
\pgfpathlineto{\pgfqpoint{1.195529in}{1.310629in}}%
\pgfpathlineto{\pgfqpoint{1.207045in}{1.315861in}}%
\pgfpathlineto{\pgfqpoint{1.220314in}{1.313102in}}%
\pgfpathlineto{\pgfqpoint{1.222832in}{1.310629in}}%
\pgfpathlineto{\pgfqpoint{1.227899in}{1.297702in}}%
\pgfpathlineto{\pgfqpoint{1.228474in}{1.284776in}}%
\pgfpathlineto{\pgfqpoint{1.225712in}{1.271849in}}%
\pgfpathlineto{\pgfqpoint{1.220314in}{1.264482in}}%
\pgfpathlineto{\pgfqpoint{1.207045in}{1.261273in}}%
\pgfpathlineto{\pgfqpoint{1.193777in}{1.268677in}}%
\pgfpathclose%
\pgfusepath{fill}%
\end{pgfscope}%
\begin{pgfscope}%
\pgfpathrectangle{\pgfqpoint{0.211875in}{0.211875in}}{\pgfqpoint{1.313625in}{1.279725in}}%
\pgfusepath{clip}%
\pgfsetbuttcap%
\pgfsetroundjoin%
\definecolor{currentfill}{rgb}{0.901975,0.231521,0.249182}%
\pgfsetfillcolor{currentfill}%
\pgfsetlinewidth{0.000000pt}%
\definecolor{currentstroke}{rgb}{0.000000,0.000000,0.000000}%
\pgfsetstrokecolor{currentstroke}%
\pgfsetdash{}{0pt}%
\pgfpathmoveto{\pgfqpoint{1.313197in}{1.253908in}}%
\pgfpathlineto{\pgfqpoint{1.326466in}{1.252257in}}%
\pgfpathlineto{\pgfqpoint{1.339735in}{1.253374in}}%
\pgfpathlineto{\pgfqpoint{1.348660in}{1.258923in}}%
\pgfpathlineto{\pgfqpoint{1.352851in}{1.271849in}}%
\pgfpathlineto{\pgfqpoint{1.353004in}{1.273696in}}%
\pgfpathlineto{\pgfqpoint{1.353648in}{1.284776in}}%
\pgfpathlineto{\pgfqpoint{1.353542in}{1.297702in}}%
\pgfpathlineto{\pgfqpoint{1.353004in}{1.304451in}}%
\pgfpathlineto{\pgfqpoint{1.352189in}{1.310629in}}%
\pgfpathlineto{\pgfqpoint{1.341760in}{1.323555in}}%
\pgfpathlineto{\pgfqpoint{1.339735in}{1.324180in}}%
\pgfpathlineto{\pgfqpoint{1.326466in}{1.325167in}}%
\pgfpathlineto{\pgfqpoint{1.313197in}{1.323677in}}%
\pgfpathlineto{\pgfqpoint{1.312855in}{1.323555in}}%
\pgfpathlineto{\pgfqpoint{1.302444in}{1.310629in}}%
\pgfpathlineto{\pgfqpoint{1.300951in}{1.297702in}}%
\pgfpathlineto{\pgfqpoint{1.300817in}{1.284776in}}%
\pgfpathlineto{\pgfqpoint{1.301737in}{1.271849in}}%
\pgfpathlineto{\pgfqpoint{1.305996in}{1.258923in}}%
\pgfpathclose%
\pgfpathmoveto{\pgfqpoint{1.309770in}{1.271849in}}%
\pgfpathlineto{\pgfqpoint{1.307329in}{1.284776in}}%
\pgfpathlineto{\pgfqpoint{1.307836in}{1.297702in}}%
\pgfpathlineto{\pgfqpoint{1.312295in}{1.310629in}}%
\pgfpathlineto{\pgfqpoint{1.313197in}{1.311725in}}%
\pgfpathlineto{\pgfqpoint{1.326466in}{1.316523in}}%
\pgfpathlineto{\pgfqpoint{1.339735in}{1.312908in}}%
\pgfpathlineto{\pgfqpoint{1.341819in}{1.310629in}}%
\pgfpathlineto{\pgfqpoint{1.346411in}{1.297702in}}%
\pgfpathlineto{\pgfqpoint{1.346929in}{1.284776in}}%
\pgfpathlineto{\pgfqpoint{1.344420in}{1.271849in}}%
\pgfpathlineto{\pgfqpoint{1.339735in}{1.264730in}}%
\pgfpathlineto{\pgfqpoint{1.326466in}{1.260520in}}%
\pgfpathlineto{\pgfqpoint{1.313197in}{1.266092in}}%
\pgfpathclose%
\pgfusepath{fill}%
\end{pgfscope}%
\begin{pgfscope}%
\pgfpathrectangle{\pgfqpoint{0.211875in}{0.211875in}}{\pgfqpoint{1.313625in}{1.279725in}}%
\pgfusepath{clip}%
\pgfsetbuttcap%
\pgfsetroundjoin%
\definecolor{currentfill}{rgb}{0.901975,0.231521,0.249182}%
\pgfsetfillcolor{currentfill}%
\pgfsetlinewidth{0.000000pt}%
\definecolor{currentstroke}{rgb}{0.000000,0.000000,0.000000}%
\pgfsetstrokecolor{currentstroke}%
\pgfsetdash{}{0pt}%
\pgfpathmoveto{\pgfqpoint{1.432617in}{1.252256in}}%
\pgfpathlineto{\pgfqpoint{1.445886in}{1.251202in}}%
\pgfpathlineto{\pgfqpoint{1.459155in}{1.252240in}}%
\pgfpathlineto{\pgfqpoint{1.468828in}{1.258923in}}%
\pgfpathlineto{\pgfqpoint{1.471757in}{1.271849in}}%
\pgfpathlineto{\pgfqpoint{1.472389in}{1.284776in}}%
\pgfpathlineto{\pgfqpoint{1.472327in}{1.297702in}}%
\pgfpathlineto{\pgfqpoint{1.471393in}{1.310629in}}%
\pgfpathlineto{\pgfqpoint{1.464454in}{1.323555in}}%
\pgfpathlineto{\pgfqpoint{1.459155in}{1.325360in}}%
\pgfpathlineto{\pgfqpoint{1.445886in}{1.326219in}}%
\pgfpathlineto{\pgfqpoint{1.432617in}{1.325291in}}%
\pgfpathlineto{\pgfqpoint{1.427242in}{1.323555in}}%
\pgfpathlineto{\pgfqpoint{1.419348in}{1.312214in}}%
\pgfpathlineto{\pgfqpoint{1.419035in}{1.310629in}}%
\pgfpathlineto{\pgfqpoint{1.418009in}{1.297702in}}%
\pgfpathlineto{\pgfqpoint{1.417929in}{1.284776in}}%
\pgfpathlineto{\pgfqpoint{1.418593in}{1.271849in}}%
\pgfpathlineto{\pgfqpoint{1.419348in}{1.266843in}}%
\pgfpathlineto{\pgfqpoint{1.421999in}{1.258923in}}%
\pgfpathclose%
\pgfpathmoveto{\pgfqpoint{1.426838in}{1.271849in}}%
\pgfpathlineto{\pgfqpoint{1.424485in}{1.284776in}}%
\pgfpathlineto{\pgfqpoint{1.424967in}{1.297702in}}%
\pgfpathlineto{\pgfqpoint{1.429235in}{1.310629in}}%
\pgfpathlineto{\pgfqpoint{1.432617in}{1.314317in}}%
\pgfpathlineto{\pgfqpoint{1.445886in}{1.317692in}}%
\pgfpathlineto{\pgfqpoint{1.459155in}{1.313392in}}%
\pgfpathlineto{\pgfqpoint{1.461418in}{1.310629in}}%
\pgfpathlineto{\pgfqpoint{1.465401in}{1.297702in}}%
\pgfpathlineto{\pgfqpoint{1.465844in}{1.284776in}}%
\pgfpathlineto{\pgfqpoint{1.463655in}{1.271849in}}%
\pgfpathlineto{\pgfqpoint{1.459155in}{1.264221in}}%
\pgfpathlineto{\pgfqpoint{1.445886in}{1.259195in}}%
\pgfpathlineto{\pgfqpoint{1.432617in}{1.263132in}}%
\pgfpathclose%
\pgfusepath{fill}%
\end{pgfscope}%
\begin{pgfscope}%
\pgfpathrectangle{\pgfqpoint{0.211875in}{0.211875in}}{\pgfqpoint{1.313625in}{1.279725in}}%
\pgfusepath{clip}%
\pgfsetbuttcap%
\pgfsetroundjoin%
\definecolor{currentfill}{rgb}{0.901975,0.231521,0.249182}%
\pgfsetfillcolor{currentfill}%
\pgfsetlinewidth{0.000000pt}%
\definecolor{currentstroke}{rgb}{0.000000,0.000000,0.000000}%
\pgfsetstrokecolor{currentstroke}%
\pgfsetdash{}{0pt}%
\pgfpathmoveto{\pgfqpoint{0.251682in}{1.332818in}}%
\pgfpathlineto{\pgfqpoint{0.264951in}{1.331344in}}%
\pgfpathlineto{\pgfqpoint{0.278220in}{1.331404in}}%
\pgfpathlineto{\pgfqpoint{0.291489in}{1.336324in}}%
\pgfpathlineto{\pgfqpoint{0.291576in}{1.336482in}}%
\pgfpathlineto{\pgfqpoint{0.293840in}{1.349408in}}%
\pgfpathlineto{\pgfqpoint{0.294173in}{1.362335in}}%
\pgfpathlineto{\pgfqpoint{0.294098in}{1.375261in}}%
\pgfpathlineto{\pgfqpoint{0.293562in}{1.388188in}}%
\pgfpathlineto{\pgfqpoint{0.291489in}{1.399671in}}%
\pgfpathlineto{\pgfqpoint{0.290792in}{1.401114in}}%
\pgfpathlineto{\pgfqpoint{0.278220in}{1.406918in}}%
\pgfpathlineto{\pgfqpoint{0.264951in}{1.407165in}}%
\pgfpathlineto{\pgfqpoint{0.251682in}{1.405457in}}%
\pgfpathlineto{\pgfqpoint{0.244673in}{1.401114in}}%
\pgfpathlineto{\pgfqpoint{0.240051in}{1.388188in}}%
\pgfpathlineto{\pgfqpoint{0.239065in}{1.375261in}}%
\pgfpathlineto{\pgfqpoint{0.239012in}{1.362335in}}%
\pgfpathlineto{\pgfqpoint{0.239879in}{1.349408in}}%
\pgfpathlineto{\pgfqpoint{0.244754in}{1.336482in}}%
\pgfpathclose%
\pgfpathmoveto{\pgfqpoint{0.248597in}{1.349408in}}%
\pgfpathlineto{\pgfqpoint{0.245514in}{1.362335in}}%
\pgfpathlineto{\pgfqpoint{0.245463in}{1.375261in}}%
\pgfpathlineto{\pgfqpoint{0.248276in}{1.388188in}}%
\pgfpathlineto{\pgfqpoint{0.251682in}{1.393505in}}%
\pgfpathlineto{\pgfqpoint{0.264951in}{1.399002in}}%
\pgfpathlineto{\pgfqpoint{0.278220in}{1.396260in}}%
\pgfpathlineto{\pgfqpoint{0.284581in}{1.388188in}}%
\pgfpathlineto{\pgfqpoint{0.287401in}{1.375261in}}%
\pgfpathlineto{\pgfqpoint{0.287366in}{1.362335in}}%
\pgfpathlineto{\pgfqpoint{0.284304in}{1.349408in}}%
\pgfpathlineto{\pgfqpoint{0.278220in}{1.342253in}}%
\pgfpathlineto{\pgfqpoint{0.264951in}{1.339722in}}%
\pgfpathlineto{\pgfqpoint{0.251682in}{1.344913in}}%
\pgfpathclose%
\pgfusepath{fill}%
\end{pgfscope}%
\begin{pgfscope}%
\pgfpathrectangle{\pgfqpoint{0.211875in}{0.211875in}}{\pgfqpoint{1.313625in}{1.279725in}}%
\pgfusepath{clip}%
\pgfsetbuttcap%
\pgfsetroundjoin%
\definecolor{currentfill}{rgb}{0.901975,0.231521,0.249182}%
\pgfsetfillcolor{currentfill}%
\pgfsetlinewidth{0.000000pt}%
\definecolor{currentstroke}{rgb}{0.000000,0.000000,0.000000}%
\pgfsetstrokecolor{currentstroke}%
\pgfsetdash{}{0pt}%
\pgfpathmoveto{\pgfqpoint{0.782439in}{1.331620in}}%
\pgfpathlineto{\pgfqpoint{0.795708in}{1.331119in}}%
\pgfpathlineto{\pgfqpoint{0.808977in}{1.331895in}}%
\pgfpathlineto{\pgfqpoint{0.819573in}{1.336482in}}%
\pgfpathlineto{\pgfqpoint{0.822246in}{1.341704in}}%
\pgfpathlineto{\pgfqpoint{0.823646in}{1.349408in}}%
\pgfpathlineto{\pgfqpoint{0.824351in}{1.362335in}}%
\pgfpathlineto{\pgfqpoint{0.824471in}{1.375261in}}%
\pgfpathlineto{\pgfqpoint{0.824142in}{1.388188in}}%
\pgfpathlineto{\pgfqpoint{0.822265in}{1.401114in}}%
\pgfpathlineto{\pgfqpoint{0.822246in}{1.401163in}}%
\pgfpathlineto{\pgfqpoint{0.808977in}{1.408271in}}%
\pgfpathlineto{\pgfqpoint{0.795708in}{1.408863in}}%
\pgfpathlineto{\pgfqpoint{0.782439in}{1.408620in}}%
\pgfpathlineto{\pgfqpoint{0.769170in}{1.401159in}}%
\pgfpathlineto{\pgfqpoint{0.769160in}{1.401114in}}%
\pgfpathlineto{\pgfqpoint{0.768038in}{1.388188in}}%
\pgfpathlineto{\pgfqpoint{0.767867in}{1.375261in}}%
\pgfpathlineto{\pgfqpoint{0.767982in}{1.362335in}}%
\pgfpathlineto{\pgfqpoint{0.768510in}{1.349408in}}%
\pgfpathlineto{\pgfqpoint{0.769170in}{1.343794in}}%
\pgfpathlineto{\pgfqpoint{0.771572in}{1.336482in}}%
\pgfpathclose%
\pgfpathmoveto{\pgfqpoint{0.777693in}{1.349408in}}%
\pgfpathlineto{\pgfqpoint{0.774671in}{1.362335in}}%
\pgfpathlineto{\pgfqpoint{0.774433in}{1.375261in}}%
\pgfpathlineto{\pgfqpoint{0.776628in}{1.388188in}}%
\pgfpathlineto{\pgfqpoint{0.782439in}{1.397007in}}%
\pgfpathlineto{\pgfqpoint{0.795708in}{1.400842in}}%
\pgfpathlineto{\pgfqpoint{0.808977in}{1.397456in}}%
\pgfpathlineto{\pgfqpoint{0.815614in}{1.388188in}}%
\pgfpathlineto{\pgfqpoint{0.817989in}{1.375261in}}%
\pgfpathlineto{\pgfqpoint{0.817737in}{1.362335in}}%
\pgfpathlineto{\pgfqpoint{0.814499in}{1.349408in}}%
\pgfpathlineto{\pgfqpoint{0.808977in}{1.342637in}}%
\pgfpathlineto{\pgfqpoint{0.795708in}{1.339298in}}%
\pgfpathlineto{\pgfqpoint{0.782439in}{1.343108in}}%
\pgfpathclose%
\pgfusepath{fill}%
\end{pgfscope}%
\begin{pgfscope}%
\pgfpathrectangle{\pgfqpoint{0.211875in}{0.211875in}}{\pgfqpoint{1.313625in}{1.279725in}}%
\pgfusepath{clip}%
\pgfsetbuttcap%
\pgfsetroundjoin%
\definecolor{currentfill}{rgb}{0.901975,0.231521,0.249182}%
\pgfsetfillcolor{currentfill}%
\pgfsetlinewidth{0.000000pt}%
\definecolor{currentstroke}{rgb}{0.000000,0.000000,0.000000}%
\pgfsetstrokecolor{currentstroke}%
\pgfsetdash{}{0pt}%
\pgfpathmoveto{\pgfqpoint{0.901860in}{1.333170in}}%
\pgfpathlineto{\pgfqpoint{0.915129in}{1.332416in}}%
\pgfpathlineto{\pgfqpoint{0.928398in}{1.333780in}}%
\pgfpathlineto{\pgfqpoint{0.934050in}{1.336482in}}%
\pgfpathlineto{\pgfqpoint{0.940165in}{1.349408in}}%
\pgfpathlineto{\pgfqpoint{0.941297in}{1.362335in}}%
\pgfpathlineto{\pgfqpoint{0.941446in}{1.375261in}}%
\pgfpathlineto{\pgfqpoint{0.940795in}{1.388188in}}%
\pgfpathlineto{\pgfqpoint{0.937241in}{1.401114in}}%
\pgfpathlineto{\pgfqpoint{0.928398in}{1.406394in}}%
\pgfpathlineto{\pgfqpoint{0.915129in}{1.407534in}}%
\pgfpathlineto{\pgfqpoint{0.901860in}{1.406957in}}%
\pgfpathlineto{\pgfqpoint{0.890377in}{1.401114in}}%
\pgfpathlineto{\pgfqpoint{0.888591in}{1.396484in}}%
\pgfpathlineto{\pgfqpoint{0.887342in}{1.388188in}}%
\pgfpathlineto{\pgfqpoint{0.886851in}{1.375261in}}%
\pgfpathlineto{\pgfqpoint{0.886981in}{1.362335in}}%
\pgfpathlineto{\pgfqpoint{0.887892in}{1.349408in}}%
\pgfpathlineto{\pgfqpoint{0.888591in}{1.345680in}}%
\pgfpathlineto{\pgfqpoint{0.893669in}{1.336482in}}%
\pgfpathclose%
\pgfpathmoveto{\pgfqpoint{0.897406in}{1.349408in}}%
\pgfpathlineto{\pgfqpoint{0.893810in}{1.362335in}}%
\pgfpathlineto{\pgfqpoint{0.893541in}{1.375261in}}%
\pgfpathlineto{\pgfqpoint{0.896196in}{1.388188in}}%
\pgfpathlineto{\pgfqpoint{0.901860in}{1.395883in}}%
\pgfpathlineto{\pgfqpoint{0.915129in}{1.399233in}}%
\pgfpathlineto{\pgfqpoint{0.928398in}{1.394291in}}%
\pgfpathlineto{\pgfqpoint{0.932314in}{1.388188in}}%
\pgfpathlineto{\pgfqpoint{0.934890in}{1.375261in}}%
\pgfpathlineto{\pgfqpoint{0.934632in}{1.362335in}}%
\pgfpathlineto{\pgfqpoint{0.931162in}{1.349408in}}%
\pgfpathlineto{\pgfqpoint{0.928398in}{1.345636in}}%
\pgfpathlineto{\pgfqpoint{0.915129in}{1.340806in}}%
\pgfpathlineto{\pgfqpoint{0.901860in}{1.344102in}}%
\pgfpathclose%
\pgfusepath{fill}%
\end{pgfscope}%
\begin{pgfscope}%
\pgfpathrectangle{\pgfqpoint{0.211875in}{0.211875in}}{\pgfqpoint{1.313625in}{1.279725in}}%
\pgfusepath{clip}%
\pgfsetbuttcap%
\pgfsetroundjoin%
\definecolor{currentfill}{rgb}{0.901975,0.231521,0.249182}%
\pgfsetfillcolor{currentfill}%
\pgfsetlinewidth{0.000000pt}%
\definecolor{currentstroke}{rgb}{0.000000,0.000000,0.000000}%
\pgfsetstrokecolor{currentstroke}%
\pgfsetdash{}{0pt}%
\pgfpathmoveto{\pgfqpoint{1.021280in}{1.333933in}}%
\pgfpathlineto{\pgfqpoint{1.034549in}{1.333215in}}%
\pgfpathlineto{\pgfqpoint{1.047818in}{1.335226in}}%
\pgfpathlineto{\pgfqpoint{1.050187in}{1.336482in}}%
\pgfpathlineto{\pgfqpoint{1.057244in}{1.349408in}}%
\pgfpathlineto{\pgfqpoint{1.058531in}{1.362335in}}%
\pgfpathlineto{\pgfqpoint{1.058684in}{1.375261in}}%
\pgfpathlineto{\pgfqpoint{1.057897in}{1.388188in}}%
\pgfpathlineto{\pgfqpoint{1.053630in}{1.401114in}}%
\pgfpathlineto{\pgfqpoint{1.047818in}{1.404982in}}%
\pgfpathlineto{\pgfqpoint{1.034549in}{1.406721in}}%
\pgfpathlineto{\pgfqpoint{1.021280in}{1.406121in}}%
\pgfpathlineto{\pgfqpoint{1.010377in}{1.401114in}}%
\pgfpathlineto{\pgfqpoint{1.008011in}{1.396995in}}%
\pgfpathlineto{\pgfqpoint{1.006147in}{1.388188in}}%
\pgfpathlineto{\pgfqpoint{1.005462in}{1.375261in}}%
\pgfpathlineto{\pgfqpoint{1.005602in}{1.362335in}}%
\pgfpathlineto{\pgfqpoint{1.006745in}{1.349408in}}%
\pgfpathlineto{\pgfqpoint{1.008011in}{1.344436in}}%
\pgfpathlineto{\pgfqpoint{1.014321in}{1.336482in}}%
\pgfpathclose%
\pgfpathmoveto{\pgfqpoint{1.016554in}{1.349408in}}%
\pgfpathlineto{\pgfqpoint{1.012485in}{1.362335in}}%
\pgfpathlineto{\pgfqpoint{1.012189in}{1.375261in}}%
\pgfpathlineto{\pgfqpoint{1.015219in}{1.388188in}}%
\pgfpathlineto{\pgfqpoint{1.021280in}{1.395577in}}%
\pgfpathlineto{\pgfqpoint{1.034549in}{1.398185in}}%
\pgfpathlineto{\pgfqpoint{1.047818in}{1.391445in}}%
\pgfpathlineto{\pgfqpoint{1.049683in}{1.388188in}}%
\pgfpathlineto{\pgfqpoint{1.052341in}{1.375261in}}%
\pgfpathlineto{\pgfqpoint{1.052082in}{1.362335in}}%
\pgfpathlineto{\pgfqpoint{1.048520in}{1.349408in}}%
\pgfpathlineto{\pgfqpoint{1.047818in}{1.348337in}}%
\pgfpathlineto{\pgfqpoint{1.034549in}{1.341794in}}%
\pgfpathlineto{\pgfqpoint{1.021280in}{1.344347in}}%
\pgfpathclose%
\pgfusepath{fill}%
\end{pgfscope}%
\begin{pgfscope}%
\pgfpathrectangle{\pgfqpoint{0.211875in}{0.211875in}}{\pgfqpoint{1.313625in}{1.279725in}}%
\pgfusepath{clip}%
\pgfsetbuttcap%
\pgfsetroundjoin%
\definecolor{currentfill}{rgb}{0.901975,0.231521,0.249182}%
\pgfsetfillcolor{currentfill}%
\pgfsetlinewidth{0.000000pt}%
\definecolor{currentstroke}{rgb}{0.000000,0.000000,0.000000}%
\pgfsetstrokecolor{currentstroke}%
\pgfsetdash{}{0pt}%
\pgfpathmoveto{\pgfqpoint{1.140701in}{1.334049in}}%
\pgfpathlineto{\pgfqpoint{1.153970in}{1.333521in}}%
\pgfpathlineto{\pgfqpoint{1.167239in}{1.336139in}}%
\pgfpathlineto{\pgfqpoint{1.167817in}{1.336482in}}%
\pgfpathlineto{\pgfqpoint{1.174997in}{1.349408in}}%
\pgfpathlineto{\pgfqpoint{1.176293in}{1.362335in}}%
\pgfpathlineto{\pgfqpoint{1.176445in}{1.375261in}}%
\pgfpathlineto{\pgfqpoint{1.175643in}{1.388188in}}%
\pgfpathlineto{\pgfqpoint{1.171282in}{1.401114in}}%
\pgfpathlineto{\pgfqpoint{1.167239in}{1.404136in}}%
\pgfpathlineto{\pgfqpoint{1.153970in}{1.406417in}}%
\pgfpathlineto{\pgfqpoint{1.140701in}{1.405961in}}%
\pgfpathlineto{\pgfqpoint{1.129012in}{1.401114in}}%
\pgfpathlineto{\pgfqpoint{1.127432in}{1.399092in}}%
\pgfpathlineto{\pgfqpoint{1.124471in}{1.388188in}}%
\pgfpathlineto{\pgfqpoint{1.123707in}{1.375261in}}%
\pgfpathlineto{\pgfqpoint{1.123851in}{1.362335in}}%
\pgfpathlineto{\pgfqpoint{1.125087in}{1.349408in}}%
\pgfpathlineto{\pgfqpoint{1.127432in}{1.342099in}}%
\pgfpathlineto{\pgfqpoint{1.133373in}{1.336482in}}%
\pgfpathclose%
\pgfpathmoveto{\pgfqpoint{1.135052in}{1.349408in}}%
\pgfpathlineto{\pgfqpoint{1.130616in}{1.362335in}}%
\pgfpathlineto{\pgfqpoint{1.130296in}{1.375261in}}%
\pgfpathlineto{\pgfqpoint{1.133610in}{1.388188in}}%
\pgfpathlineto{\pgfqpoint{1.140701in}{1.395944in}}%
\pgfpathlineto{\pgfqpoint{1.153970in}{1.397688in}}%
\pgfpathlineto{\pgfqpoint{1.167239in}{1.388957in}}%
\pgfpathlineto{\pgfqpoint{1.167628in}{1.388188in}}%
\pgfpathlineto{\pgfqpoint{1.170261in}{1.375261in}}%
\pgfpathlineto{\pgfqpoint{1.170007in}{1.362335in}}%
\pgfpathlineto{\pgfqpoint{1.167239in}{1.351577in}}%
\pgfpathlineto{\pgfqpoint{1.165873in}{1.349408in}}%
\pgfpathlineto{\pgfqpoint{1.153970in}{1.342268in}}%
\pgfpathlineto{\pgfqpoint{1.140701in}{1.343972in}}%
\pgfpathclose%
\pgfusepath{fill}%
\end{pgfscope}%
\begin{pgfscope}%
\pgfpathrectangle{\pgfqpoint{0.211875in}{0.211875in}}{\pgfqpoint{1.313625in}{1.279725in}}%
\pgfusepath{clip}%
\pgfsetbuttcap%
\pgfsetroundjoin%
\definecolor{currentfill}{rgb}{0.901975,0.231521,0.249182}%
\pgfsetfillcolor{currentfill}%
\pgfsetlinewidth{0.000000pt}%
\definecolor{currentstroke}{rgb}{0.000000,0.000000,0.000000}%
\pgfsetstrokecolor{currentstroke}%
\pgfsetdash{}{0pt}%
\pgfpathmoveto{\pgfqpoint{1.260121in}{1.333611in}}%
\pgfpathlineto{\pgfqpoint{1.273390in}{1.333319in}}%
\pgfpathlineto{\pgfqpoint{1.286659in}{1.336361in}}%
\pgfpathlineto{\pgfqpoint{1.286839in}{1.336482in}}%
\pgfpathlineto{\pgfqpoint{1.293353in}{1.349408in}}%
\pgfpathlineto{\pgfqpoint{1.294523in}{1.362335in}}%
\pgfpathlineto{\pgfqpoint{1.294666in}{1.375261in}}%
\pgfpathlineto{\pgfqpoint{1.293965in}{1.388188in}}%
\pgfpathlineto{\pgfqpoint{1.290099in}{1.401114in}}%
\pgfpathlineto{\pgfqpoint{1.286659in}{1.404034in}}%
\pgfpathlineto{\pgfqpoint{1.273390in}{1.406638in}}%
\pgfpathlineto{\pgfqpoint{1.260121in}{1.406380in}}%
\pgfpathlineto{\pgfqpoint{1.246852in}{1.401771in}}%
\pgfpathlineto{\pgfqpoint{1.246281in}{1.401114in}}%
\pgfpathlineto{\pgfqpoint{1.242313in}{1.388188in}}%
\pgfpathlineto{\pgfqpoint{1.241582in}{1.375261in}}%
\pgfpathlineto{\pgfqpoint{1.241724in}{1.362335in}}%
\pgfpathlineto{\pgfqpoint{1.242920in}{1.349408in}}%
\pgfpathlineto{\pgfqpoint{1.246852in}{1.339229in}}%
\pgfpathlineto{\pgfqpoint{1.250569in}{1.336482in}}%
\pgfpathclose%
\pgfpathmoveto{\pgfqpoint{1.252765in}{1.349408in}}%
\pgfpathlineto{\pgfqpoint{1.248077in}{1.362335in}}%
\pgfpathlineto{\pgfqpoint{1.247736in}{1.375261in}}%
\pgfpathlineto{\pgfqpoint{1.251236in}{1.388188in}}%
\pgfpathlineto{\pgfqpoint{1.260121in}{1.396893in}}%
\pgfpathlineto{\pgfqpoint{1.273390in}{1.397755in}}%
\pgfpathlineto{\pgfqpoint{1.285724in}{1.388188in}}%
\pgfpathlineto{\pgfqpoint{1.286659in}{1.385804in}}%
\pgfpathlineto{\pgfqpoint{1.288598in}{1.375261in}}%
\pgfpathlineto{\pgfqpoint{1.288353in}{1.362335in}}%
\pgfpathlineto{\pgfqpoint{1.286659in}{1.354857in}}%
\pgfpathlineto{\pgfqpoint{1.283907in}{1.349408in}}%
\pgfpathlineto{\pgfqpoint{1.273390in}{1.342218in}}%
\pgfpathlineto{\pgfqpoint{1.260121in}{1.343058in}}%
\pgfpathclose%
\pgfusepath{fill}%
\end{pgfscope}%
\begin{pgfscope}%
\pgfpathrectangle{\pgfqpoint{0.211875in}{0.211875in}}{\pgfqpoint{1.313625in}{1.279725in}}%
\pgfusepath{clip}%
\pgfsetbuttcap%
\pgfsetroundjoin%
\definecolor{currentfill}{rgb}{0.901975,0.231521,0.249182}%
\pgfsetfillcolor{currentfill}%
\pgfsetlinewidth{0.000000pt}%
\definecolor{currentstroke}{rgb}{0.000000,0.000000,0.000000}%
\pgfsetstrokecolor{currentstroke}%
\pgfsetdash{}{0pt}%
\pgfpathmoveto{\pgfqpoint{1.366273in}{1.336091in}}%
\pgfpathlineto{\pgfqpoint{1.379542in}{1.332673in}}%
\pgfpathlineto{\pgfqpoint{1.392811in}{1.332575in}}%
\pgfpathlineto{\pgfqpoint{1.406080in}{1.335613in}}%
\pgfpathlineto{\pgfqpoint{1.407213in}{1.336482in}}%
\pgfpathlineto{\pgfqpoint{1.412271in}{1.349408in}}%
\pgfpathlineto{\pgfqpoint{1.413181in}{1.362335in}}%
\pgfpathlineto{\pgfqpoint{1.413311in}{1.375261in}}%
\pgfpathlineto{\pgfqpoint{1.412821in}{1.388188in}}%
\pgfpathlineto{\pgfqpoint{1.410038in}{1.401114in}}%
\pgfpathlineto{\pgfqpoint{1.406080in}{1.404991in}}%
\pgfpathlineto{\pgfqpoint{1.392811in}{1.407422in}}%
\pgfpathlineto{\pgfqpoint{1.379542in}{1.407316in}}%
\pgfpathlineto{\pgfqpoint{1.366273in}{1.404445in}}%
\pgfpathlineto{\pgfqpoint{1.362873in}{1.401114in}}%
\pgfpathlineto{\pgfqpoint{1.359653in}{1.388188in}}%
\pgfpathlineto{\pgfqpoint{1.359063in}{1.375261in}}%
\pgfpathlineto{\pgfqpoint{1.359198in}{1.362335in}}%
\pgfpathlineto{\pgfqpoint{1.360223in}{1.349408in}}%
\pgfpathlineto{\pgfqpoint{1.365768in}{1.336482in}}%
\pgfpathclose%
\pgfpathmoveto{\pgfqpoint{1.369483in}{1.349408in}}%
\pgfpathlineto{\pgfqpoint{1.366273in}{1.357151in}}%
\pgfpathlineto{\pgfqpoint{1.365231in}{1.362335in}}%
\pgfpathlineto{\pgfqpoint{1.364998in}{1.375261in}}%
\pgfpathlineto{\pgfqpoint{1.366273in}{1.383136in}}%
\pgfpathlineto{\pgfqpoint{1.367889in}{1.388188in}}%
\pgfpathlineto{\pgfqpoint{1.379542in}{1.398369in}}%
\pgfpathlineto{\pgfqpoint{1.392811in}{1.398425in}}%
\pgfpathlineto{\pgfqpoint{1.404495in}{1.388188in}}%
\pgfpathlineto{\pgfqpoint{1.406080in}{1.383082in}}%
\pgfpathlineto{\pgfqpoint{1.407315in}{1.375261in}}%
\pgfpathlineto{\pgfqpoint{1.407083in}{1.362335in}}%
\pgfpathlineto{\pgfqpoint{1.406080in}{1.357228in}}%
\pgfpathlineto{\pgfqpoint{1.402922in}{1.349408in}}%
\pgfpathlineto{\pgfqpoint{1.392811in}{1.341607in}}%
\pgfpathlineto{\pgfqpoint{1.379542in}{1.341655in}}%
\pgfpathclose%
\pgfusepath{fill}%
\end{pgfscope}%
\begin{pgfscope}%
\pgfpathrectangle{\pgfqpoint{0.211875in}{0.211875in}}{\pgfqpoint{1.313625in}{1.279725in}}%
\pgfusepath{clip}%
\pgfsetbuttcap%
\pgfsetroundjoin%
\definecolor{currentfill}{rgb}{0.901975,0.231521,0.249182}%
\pgfsetfillcolor{currentfill}%
\pgfsetlinewidth{0.000000pt}%
\definecolor{currentstroke}{rgb}{0.000000,0.000000,0.000000}%
\pgfsetstrokecolor{currentstroke}%
\pgfsetdash{}{0pt}%
\pgfpathmoveto{\pgfqpoint{1.485693in}{1.333117in}}%
\pgfpathlineto{\pgfqpoint{1.498962in}{1.331267in}}%
\pgfpathlineto{\pgfqpoint{1.512231in}{1.331227in}}%
\pgfpathlineto{\pgfqpoint{1.525500in}{1.333388in}}%
\pgfpathlineto{\pgfqpoint{1.525500in}{1.336482in}}%
\pgfpathlineto{\pgfqpoint{1.525500in}{1.349408in}}%
\pgfpathlineto{\pgfqpoint{1.525500in}{1.358260in}}%
\pgfpathlineto{\pgfqpoint{1.522678in}{1.349408in}}%
\pgfpathlineto{\pgfqpoint{1.512231in}{1.340375in}}%
\pgfpathlineto{\pgfqpoint{1.498962in}{1.339785in}}%
\pgfpathlineto{\pgfqpoint{1.485693in}{1.348468in}}%
\pgfpathlineto{\pgfqpoint{1.485205in}{1.349408in}}%
\pgfpathlineto{\pgfqpoint{1.482351in}{1.362335in}}%
\pgfpathlineto{\pgfqpoint{1.482130in}{1.375261in}}%
\pgfpathlineto{\pgfqpoint{1.484222in}{1.388188in}}%
\pgfpathlineto{\pgfqpoint{1.485693in}{1.391451in}}%
\pgfpathlineto{\pgfqpoint{1.498962in}{1.400348in}}%
\pgfpathlineto{\pgfqpoint{1.512231in}{1.399767in}}%
\pgfpathlineto{\pgfqpoint{1.524010in}{1.388188in}}%
\pgfpathlineto{\pgfqpoint{1.525500in}{1.382013in}}%
\pgfpathlineto{\pgfqpoint{1.525500in}{1.388188in}}%
\pgfpathlineto{\pgfqpoint{1.525500in}{1.401114in}}%
\pgfpathlineto{\pgfqpoint{1.525500in}{1.407585in}}%
\pgfpathlineto{\pgfqpoint{1.512231in}{1.408833in}}%
\pgfpathlineto{\pgfqpoint{1.498962in}{1.408737in}}%
\pgfpathlineto{\pgfqpoint{1.485693in}{1.407372in}}%
\pgfpathlineto{\pgfqpoint{1.478356in}{1.401114in}}%
\pgfpathlineto{\pgfqpoint{1.476449in}{1.388188in}}%
\pgfpathlineto{\pgfqpoint{1.476115in}{1.375261in}}%
\pgfpathlineto{\pgfqpoint{1.476237in}{1.362335in}}%
\pgfpathlineto{\pgfqpoint{1.476954in}{1.349408in}}%
\pgfpathlineto{\pgfqpoint{1.480735in}{1.336482in}}%
\pgfpathclose%
\pgfusepath{fill}%
\end{pgfscope}%
\begin{pgfscope}%
\pgfpathrectangle{\pgfqpoint{0.211875in}{0.211875in}}{\pgfqpoint{1.313625in}{1.279725in}}%
\pgfusepath{clip}%
\pgfsetbuttcap%
\pgfsetroundjoin%
\definecolor{currentfill}{rgb}{0.901975,0.231521,0.249182}%
\pgfsetfillcolor{currentfill}%
\pgfsetlinewidth{0.000000pt}%
\definecolor{currentstroke}{rgb}{0.000000,0.000000,0.000000}%
\pgfsetstrokecolor{currentstroke}%
\pgfsetdash{}{0pt}%
\pgfpathmoveto{\pgfqpoint{0.304758in}{1.413988in}}%
\pgfpathlineto{\pgfqpoint{0.318027in}{1.411680in}}%
\pgfpathlineto{\pgfqpoint{0.331295in}{1.411232in}}%
\pgfpathlineto{\pgfqpoint{0.344564in}{1.411351in}}%
\pgfpathlineto{\pgfqpoint{0.352317in}{1.414041in}}%
\pgfpathlineto{\pgfqpoint{0.353590in}{1.426967in}}%
\pgfpathlineto{\pgfqpoint{0.353664in}{1.439894in}}%
\pgfpathlineto{\pgfqpoint{0.353586in}{1.452820in}}%
\pgfpathlineto{\pgfqpoint{0.353318in}{1.465747in}}%
\pgfpathlineto{\pgfqpoint{0.352340in}{1.478673in}}%
\pgfpathlineto{\pgfqpoint{0.344564in}{1.488260in}}%
\pgfpathlineto{\pgfqpoint{0.331295in}{1.489325in}}%
\pgfpathlineto{\pgfqpoint{0.318027in}{1.488872in}}%
\pgfpathlineto{\pgfqpoint{0.304758in}{1.485642in}}%
\pgfpathlineto{\pgfqpoint{0.299767in}{1.478673in}}%
\pgfpathlineto{\pgfqpoint{0.297784in}{1.465747in}}%
\pgfpathlineto{\pgfqpoint{0.297295in}{1.452820in}}%
\pgfpathlineto{\pgfqpoint{0.297339in}{1.439894in}}%
\pgfpathlineto{\pgfqpoint{0.298100in}{1.426967in}}%
\pgfpathlineto{\pgfqpoint{0.304648in}{1.414041in}}%
\pgfpathclose%
\pgfpathmoveto{\pgfqpoint{0.308653in}{1.426967in}}%
\pgfpathlineto{\pgfqpoint{0.304758in}{1.434653in}}%
\pgfpathlineto{\pgfqpoint{0.303557in}{1.439894in}}%
\pgfpathlineto{\pgfqpoint{0.302992in}{1.452820in}}%
\pgfpathlineto{\pgfqpoint{0.304527in}{1.465747in}}%
\pgfpathlineto{\pgfqpoint{0.304758in}{1.466477in}}%
\pgfpathlineto{\pgfqpoint{0.314600in}{1.478673in}}%
\pgfpathlineto{\pgfqpoint{0.318027in}{1.480367in}}%
\pgfpathlineto{\pgfqpoint{0.331295in}{1.480867in}}%
\pgfpathlineto{\pgfqpoint{0.336550in}{1.478673in}}%
\pgfpathlineto{\pgfqpoint{0.344564in}{1.470459in}}%
\pgfpathlineto{\pgfqpoint{0.346183in}{1.465747in}}%
\pgfpathlineto{\pgfqpoint{0.347610in}{1.452820in}}%
\pgfpathlineto{\pgfqpoint{0.347112in}{1.439894in}}%
\pgfpathlineto{\pgfqpoint{0.344564in}{1.429630in}}%
\pgfpathlineto{\pgfqpoint{0.342979in}{1.426967in}}%
\pgfpathlineto{\pgfqpoint{0.331295in}{1.420161in}}%
\pgfpathlineto{\pgfqpoint{0.318027in}{1.420697in}}%
\pgfpathclose%
\pgfusepath{fill}%
\end{pgfscope}%
\begin{pgfscope}%
\pgfpathrectangle{\pgfqpoint{0.211875in}{0.211875in}}{\pgfqpoint{1.313625in}{1.279725in}}%
\pgfusepath{clip}%
\pgfsetbuttcap%
\pgfsetroundjoin%
\definecolor{currentfill}{rgb}{0.901975,0.231521,0.249182}%
\pgfsetfillcolor{currentfill}%
\pgfsetlinewidth{0.000000pt}%
\definecolor{currentstroke}{rgb}{0.000000,0.000000,0.000000}%
\pgfsetstrokecolor{currentstroke}%
\pgfsetdash{}{0pt}%
\pgfpathmoveto{\pgfqpoint{0.835515in}{1.413838in}}%
\pgfpathlineto{\pgfqpoint{0.848784in}{1.412277in}}%
\pgfpathlineto{\pgfqpoint{0.862053in}{1.412432in}}%
\pgfpathlineto{\pgfqpoint{0.872893in}{1.414041in}}%
\pgfpathlineto{\pgfqpoint{0.875322in}{1.414778in}}%
\pgfpathlineto{\pgfqpoint{0.882165in}{1.426967in}}%
\pgfpathlineto{\pgfqpoint{0.883170in}{1.439894in}}%
\pgfpathlineto{\pgfqpoint{0.883427in}{1.452820in}}%
\pgfpathlineto{\pgfqpoint{0.883321in}{1.465747in}}%
\pgfpathlineto{\pgfqpoint{0.882491in}{1.478673in}}%
\pgfpathlineto{\pgfqpoint{0.875322in}{1.488823in}}%
\pgfpathlineto{\pgfqpoint{0.862053in}{1.490266in}}%
\pgfpathlineto{\pgfqpoint{0.848784in}{1.490429in}}%
\pgfpathlineto{\pgfqpoint{0.835515in}{1.489776in}}%
\pgfpathlineto{\pgfqpoint{0.827370in}{1.478673in}}%
\pgfpathlineto{\pgfqpoint{0.826937in}{1.465747in}}%
\pgfpathlineto{\pgfqpoint{0.826917in}{1.452820in}}%
\pgfpathlineto{\pgfqpoint{0.827135in}{1.439894in}}%
\pgfpathlineto{\pgfqpoint{0.827908in}{1.426967in}}%
\pgfpathlineto{\pgfqpoint{0.834969in}{1.414041in}}%
\pgfpathclose%
\pgfpathmoveto{\pgfqpoint{0.838882in}{1.426967in}}%
\pgfpathlineto{\pgfqpoint{0.835515in}{1.432502in}}%
\pgfpathlineto{\pgfqpoint{0.833585in}{1.439894in}}%
\pgfpathlineto{\pgfqpoint{0.832807in}{1.452820in}}%
\pgfpathlineto{\pgfqpoint{0.833951in}{1.465747in}}%
\pgfpathlineto{\pgfqpoint{0.835515in}{1.470798in}}%
\pgfpathlineto{\pgfqpoint{0.842180in}{1.478673in}}%
\pgfpathlineto{\pgfqpoint{0.848784in}{1.481685in}}%
\pgfpathlineto{\pgfqpoint{0.862053in}{1.481517in}}%
\pgfpathlineto{\pgfqpoint{0.867955in}{1.478673in}}%
\pgfpathlineto{\pgfqpoint{0.875322in}{1.469454in}}%
\pgfpathlineto{\pgfqpoint{0.876444in}{1.465747in}}%
\pgfpathlineto{\pgfqpoint{0.877634in}{1.452820in}}%
\pgfpathlineto{\pgfqpoint{0.876836in}{1.439894in}}%
\pgfpathlineto{\pgfqpoint{0.875322in}{1.433972in}}%
\pgfpathlineto{\pgfqpoint{0.871253in}{1.426967in}}%
\pgfpathlineto{\pgfqpoint{0.862053in}{1.421251in}}%
\pgfpathlineto{\pgfqpoint{0.848784in}{1.421078in}}%
\pgfpathclose%
\pgfusepath{fill}%
\end{pgfscope}%
\begin{pgfscope}%
\pgfpathrectangle{\pgfqpoint{0.211875in}{0.211875in}}{\pgfqpoint{1.313625in}{1.279725in}}%
\pgfusepath{clip}%
\pgfsetbuttcap%
\pgfsetroundjoin%
\definecolor{currentfill}{rgb}{0.901975,0.231521,0.249182}%
\pgfsetfillcolor{currentfill}%
\pgfsetlinewidth{0.000000pt}%
\definecolor{currentstroke}{rgb}{0.000000,0.000000,0.000000}%
\pgfsetstrokecolor{currentstroke}%
\pgfsetdash{}{0pt}%
\pgfpathmoveto{\pgfqpoint{0.968205in}{1.413329in}}%
\pgfpathlineto{\pgfqpoint{0.981473in}{1.413616in}}%
\pgfpathlineto{\pgfqpoint{0.984108in}{1.414041in}}%
\pgfpathlineto{\pgfqpoint{0.994742in}{1.418008in}}%
\pgfpathlineto{\pgfqpoint{0.999034in}{1.426967in}}%
\pgfpathlineto{\pgfqpoint{1.000407in}{1.439894in}}%
\pgfpathlineto{\pgfqpoint{1.000725in}{1.452820in}}%
\pgfpathlineto{\pgfqpoint{1.000495in}{1.465747in}}%
\pgfpathlineto{\pgfqpoint{0.999064in}{1.478673in}}%
\pgfpathlineto{\pgfqpoint{0.994742in}{1.485902in}}%
\pgfpathlineto{\pgfqpoint{0.981473in}{1.489056in}}%
\pgfpathlineto{\pgfqpoint{0.968205in}{1.489293in}}%
\pgfpathlineto{\pgfqpoint{0.954936in}{1.487846in}}%
\pgfpathlineto{\pgfqpoint{0.947163in}{1.478673in}}%
\pgfpathlineto{\pgfqpoint{0.945915in}{1.465747in}}%
\pgfpathlineto{\pgfqpoint{0.945727in}{1.452820in}}%
\pgfpathlineto{\pgfqpoint{0.946031in}{1.439894in}}%
\pgfpathlineto{\pgfqpoint{0.947310in}{1.426967in}}%
\pgfpathlineto{\pgfqpoint{0.954936in}{1.415499in}}%
\pgfpathlineto{\pgfqpoint{0.961922in}{1.414041in}}%
\pgfpathclose%
\pgfpathmoveto{\pgfqpoint{0.958747in}{1.426967in}}%
\pgfpathlineto{\pgfqpoint{0.954936in}{1.431986in}}%
\pgfpathlineto{\pgfqpoint{0.952563in}{1.439894in}}%
\pgfpathlineto{\pgfqpoint{0.951694in}{1.452820in}}%
\pgfpathlineto{\pgfqpoint{0.953015in}{1.465747in}}%
\pgfpathlineto{\pgfqpoint{0.954936in}{1.471095in}}%
\pgfpathlineto{\pgfqpoint{0.963004in}{1.478673in}}%
\pgfpathlineto{\pgfqpoint{0.968205in}{1.480777in}}%
\pgfpathlineto{\pgfqpoint{0.981473in}{1.480004in}}%
\pgfpathlineto{\pgfqpoint{0.983948in}{1.478673in}}%
\pgfpathlineto{\pgfqpoint{0.993217in}{1.465747in}}%
\pgfpathlineto{\pgfqpoint{0.994742in}{1.455364in}}%
\pgfpathlineto{\pgfqpoint{0.994967in}{1.452820in}}%
\pgfpathlineto{\pgfqpoint{0.994742in}{1.449209in}}%
\pgfpathlineto{\pgfqpoint{0.993855in}{1.439894in}}%
\pgfpathlineto{\pgfqpoint{0.987440in}{1.426967in}}%
\pgfpathlineto{\pgfqpoint{0.981473in}{1.422826in}}%
\pgfpathlineto{\pgfqpoint{0.968205in}{1.421984in}}%
\pgfpathclose%
\pgfusepath{fill}%
\end{pgfscope}%
\begin{pgfscope}%
\pgfpathrectangle{\pgfqpoint{0.211875in}{0.211875in}}{\pgfqpoint{1.313625in}{1.279725in}}%
\pgfusepath{clip}%
\pgfsetbuttcap%
\pgfsetroundjoin%
\definecolor{currentfill}{rgb}{0.901975,0.231521,0.249182}%
\pgfsetfillcolor{currentfill}%
\pgfsetlinewidth{0.000000pt}%
\definecolor{currentstroke}{rgb}{0.000000,0.000000,0.000000}%
\pgfsetstrokecolor{currentstroke}%
\pgfsetdash{}{0pt}%
\pgfpathmoveto{\pgfqpoint{1.087625in}{1.413820in}}%
\pgfpathlineto{\pgfqpoint{1.094017in}{1.414041in}}%
\pgfpathlineto{\pgfqpoint{1.100894in}{1.414330in}}%
\pgfpathlineto{\pgfqpoint{1.114163in}{1.421012in}}%
\pgfpathlineto{\pgfqpoint{1.116526in}{1.426967in}}%
\pgfpathlineto{\pgfqpoint{1.118066in}{1.439894in}}%
\pgfpathlineto{\pgfqpoint{1.118411in}{1.452820in}}%
\pgfpathlineto{\pgfqpoint{1.118126in}{1.465747in}}%
\pgfpathlineto{\pgfqpoint{1.116424in}{1.478673in}}%
\pgfpathlineto{\pgfqpoint{1.114163in}{1.483299in}}%
\pgfpathlineto{\pgfqpoint{1.100894in}{1.488371in}}%
\pgfpathlineto{\pgfqpoint{1.087625in}{1.488757in}}%
\pgfpathlineto{\pgfqpoint{1.074356in}{1.487215in}}%
\pgfpathlineto{\pgfqpoint{1.066123in}{1.478673in}}%
\pgfpathlineto{\pgfqpoint{1.064402in}{1.465747in}}%
\pgfpathlineto{\pgfqpoint{1.064118in}{1.452820in}}%
\pgfpathlineto{\pgfqpoint{1.064475in}{1.439894in}}%
\pgfpathlineto{\pgfqpoint{1.066054in}{1.426967in}}%
\pgfpathlineto{\pgfqpoint{1.074356in}{1.415952in}}%
\pgfpathlineto{\pgfqpoint{1.085458in}{1.414041in}}%
\pgfpathclose%
\pgfpathmoveto{\pgfqpoint{1.077628in}{1.426967in}}%
\pgfpathlineto{\pgfqpoint{1.074356in}{1.430426in}}%
\pgfpathlineto{\pgfqpoint{1.071134in}{1.439894in}}%
\pgfpathlineto{\pgfqpoint{1.070203in}{1.452820in}}%
\pgfpathlineto{\pgfqpoint{1.071640in}{1.465747in}}%
\pgfpathlineto{\pgfqpoint{1.074356in}{1.472371in}}%
\pgfpathlineto{\pgfqpoint{1.082753in}{1.478673in}}%
\pgfpathlineto{\pgfqpoint{1.087625in}{1.480404in}}%
\pgfpathlineto{\pgfqpoint{1.100894in}{1.478929in}}%
\pgfpathlineto{\pgfqpoint{1.101322in}{1.478673in}}%
\pgfpathlineto{\pgfqpoint{1.110388in}{1.465747in}}%
\pgfpathlineto{\pgfqpoint{1.112184in}{1.452820in}}%
\pgfpathlineto{\pgfqpoint{1.111023in}{1.439894in}}%
\pgfpathlineto{\pgfqpoint{1.104790in}{1.426967in}}%
\pgfpathlineto{\pgfqpoint{1.100894in}{1.423957in}}%
\pgfpathlineto{\pgfqpoint{1.087625in}{1.422350in}}%
\pgfpathclose%
\pgfusepath{fill}%
\end{pgfscope}%
\begin{pgfscope}%
\pgfpathrectangle{\pgfqpoint{0.211875in}{0.211875in}}{\pgfqpoint{1.313625in}{1.279725in}}%
\pgfusepath{clip}%
\pgfsetbuttcap%
\pgfsetroundjoin%
\definecolor{currentfill}{rgb}{0.901975,0.231521,0.249182}%
\pgfsetfillcolor{currentfill}%
\pgfsetlinewidth{0.000000pt}%
\definecolor{currentstroke}{rgb}{0.000000,0.000000,0.000000}%
\pgfsetstrokecolor{currentstroke}%
\pgfsetdash{}{0pt}%
\pgfpathmoveto{\pgfqpoint{1.207045in}{1.413789in}}%
\pgfpathlineto{\pgfqpoint{1.212595in}{1.414041in}}%
\pgfpathlineto{\pgfqpoint{1.220314in}{1.414494in}}%
\pgfpathlineto{\pgfqpoint{1.233583in}{1.423674in}}%
\pgfpathlineto{\pgfqpoint{1.234616in}{1.426967in}}%
\pgfpathlineto{\pgfqpoint{1.236121in}{1.439894in}}%
\pgfpathlineto{\pgfqpoint{1.236461in}{1.452820in}}%
\pgfpathlineto{\pgfqpoint{1.236190in}{1.465747in}}%
\pgfpathlineto{\pgfqpoint{1.234550in}{1.478673in}}%
\pgfpathlineto{\pgfqpoint{1.233583in}{1.481230in}}%
\pgfpathlineto{\pgfqpoint{1.220314in}{1.488268in}}%
\pgfpathlineto{\pgfqpoint{1.207045in}{1.488778in}}%
\pgfpathlineto{\pgfqpoint{1.193777in}{1.487492in}}%
\pgfpathlineto{\pgfqpoint{1.184209in}{1.478673in}}%
\pgfpathlineto{\pgfqpoint{1.182363in}{1.465747in}}%
\pgfpathlineto{\pgfqpoint{1.182054in}{1.452820in}}%
\pgfpathlineto{\pgfqpoint{1.182429in}{1.439894in}}%
\pgfpathlineto{\pgfqpoint{1.184099in}{1.426967in}}%
\pgfpathlineto{\pgfqpoint{1.193777in}{1.415533in}}%
\pgfpathlineto{\pgfqpoint{1.204280in}{1.414041in}}%
\pgfpathclose%
\pgfpathmoveto{\pgfqpoint{1.195124in}{1.426967in}}%
\pgfpathlineto{\pgfqpoint{1.193777in}{1.428104in}}%
\pgfpathlineto{\pgfqpoint{1.189266in}{1.439894in}}%
\pgfpathlineto{\pgfqpoint{1.188303in}{1.452820in}}%
\pgfpathlineto{\pgfqpoint{1.189792in}{1.465747in}}%
\pgfpathlineto{\pgfqpoint{1.193777in}{1.474347in}}%
\pgfpathlineto{\pgfqpoint{1.201015in}{1.478673in}}%
\pgfpathlineto{\pgfqpoint{1.207045in}{1.480528in}}%
\pgfpathlineto{\pgfqpoint{1.218562in}{1.478673in}}%
\pgfpathlineto{\pgfqpoint{1.220314in}{1.478223in}}%
\pgfpathlineto{\pgfqpoint{1.228322in}{1.465747in}}%
\pgfpathlineto{\pgfqpoint{1.230001in}{1.452820in}}%
\pgfpathlineto{\pgfqpoint{1.228914in}{1.439894in}}%
\pgfpathlineto{\pgfqpoint{1.223064in}{1.426967in}}%
\pgfpathlineto{\pgfqpoint{1.220314in}{1.424608in}}%
\pgfpathlineto{\pgfqpoint{1.207045in}{1.422214in}}%
\pgfpathclose%
\pgfusepath{fill}%
\end{pgfscope}%
\begin{pgfscope}%
\pgfpathrectangle{\pgfqpoint{0.211875in}{0.211875in}}{\pgfqpoint{1.313625in}{1.279725in}}%
\pgfusepath{clip}%
\pgfsetbuttcap%
\pgfsetroundjoin%
\definecolor{currentfill}{rgb}{0.901975,0.231521,0.249182}%
\pgfsetfillcolor{currentfill}%
\pgfsetlinewidth{0.000000pt}%
\definecolor{currentstroke}{rgb}{0.000000,0.000000,0.000000}%
\pgfsetstrokecolor{currentstroke}%
\pgfsetdash{}{0pt}%
\pgfpathmoveto{\pgfqpoint{1.326466in}{1.413255in}}%
\pgfpathlineto{\pgfqpoint{1.339735in}{1.413988in}}%
\pgfpathlineto{\pgfqpoint{1.339991in}{1.414041in}}%
\pgfpathlineto{\pgfqpoint{1.353004in}{1.425645in}}%
\pgfpathlineto{\pgfqpoint{1.353306in}{1.426967in}}%
\pgfpathlineto{\pgfqpoint{1.354566in}{1.439894in}}%
\pgfpathlineto{\pgfqpoint{1.354865in}{1.452820in}}%
\pgfpathlineto{\pgfqpoint{1.354681in}{1.465747in}}%
\pgfpathlineto{\pgfqpoint{1.353451in}{1.478673in}}%
\pgfpathlineto{\pgfqpoint{1.353004in}{1.480366in}}%
\pgfpathlineto{\pgfqpoint{1.339735in}{1.488843in}}%
\pgfpathlineto{\pgfqpoint{1.326466in}{1.489334in}}%
\pgfpathlineto{\pgfqpoint{1.313197in}{1.488451in}}%
\pgfpathlineto{\pgfqpoint{1.301342in}{1.478673in}}%
\pgfpathlineto{\pgfqpoint{1.299928in}{1.468246in}}%
\pgfpathlineto{\pgfqpoint{1.299752in}{1.465747in}}%
\pgfpathlineto{\pgfqpoint{1.299516in}{1.452820in}}%
\pgfpathlineto{\pgfqpoint{1.299842in}{1.439894in}}%
\pgfpathlineto{\pgfqpoint{1.299928in}{1.438737in}}%
\pgfpathlineto{\pgfqpoint{1.301375in}{1.426967in}}%
\pgfpathlineto{\pgfqpoint{1.313197in}{1.414451in}}%
\pgfpathlineto{\pgfqpoint{1.316711in}{1.414041in}}%
\pgfpathclose%
\pgfpathmoveto{\pgfqpoint{1.312043in}{1.426967in}}%
\pgfpathlineto{\pgfqpoint{1.306905in}{1.439894in}}%
\pgfpathlineto{\pgfqpoint{1.305943in}{1.452820in}}%
\pgfpathlineto{\pgfqpoint{1.307416in}{1.465747in}}%
\pgfpathlineto{\pgfqpoint{1.313197in}{1.476870in}}%
\pgfpathlineto{\pgfqpoint{1.317036in}{1.478673in}}%
\pgfpathlineto{\pgfqpoint{1.326466in}{1.481132in}}%
\pgfpathlineto{\pgfqpoint{1.338296in}{1.478673in}}%
\pgfpathlineto{\pgfqpoint{1.339735in}{1.478163in}}%
\pgfpathlineto{\pgfqpoint{1.346879in}{1.465747in}}%
\pgfpathlineto{\pgfqpoint{1.348378in}{1.452820in}}%
\pgfpathlineto{\pgfqpoint{1.347393in}{1.439894in}}%
\pgfpathlineto{\pgfqpoint{1.342105in}{1.426967in}}%
\pgfpathlineto{\pgfqpoint{1.339735in}{1.424707in}}%
\pgfpathlineto{\pgfqpoint{1.326466in}{1.421591in}}%
\pgfpathlineto{\pgfqpoint{1.313197in}{1.425746in}}%
\pgfpathclose%
\pgfusepath{fill}%
\end{pgfscope}%
\begin{pgfscope}%
\pgfpathrectangle{\pgfqpoint{0.211875in}{0.211875in}}{\pgfqpoint{1.313625in}{1.279725in}}%
\pgfusepath{clip}%
\pgfsetbuttcap%
\pgfsetroundjoin%
\definecolor{currentfill}{rgb}{0.901975,0.231521,0.249182}%
\pgfsetfillcolor{currentfill}%
\pgfsetlinewidth{0.000000pt}%
\definecolor{currentstroke}{rgb}{0.000000,0.000000,0.000000}%
\pgfsetstrokecolor{currentstroke}%
\pgfsetdash{}{0pt}%
\pgfpathmoveto{\pgfqpoint{1.432617in}{1.412915in}}%
\pgfpathlineto{\pgfqpoint{1.445886in}{1.412216in}}%
\pgfpathlineto{\pgfqpoint{1.459155in}{1.412755in}}%
\pgfpathlineto{\pgfqpoint{1.464815in}{1.414041in}}%
\pgfpathlineto{\pgfqpoint{1.472424in}{1.425576in}}%
\pgfpathlineto{\pgfqpoint{1.472620in}{1.426967in}}%
\pgfpathlineto{\pgfqpoint{1.473406in}{1.439894in}}%
\pgfpathlineto{\pgfqpoint{1.473628in}{1.452820in}}%
\pgfpathlineto{\pgfqpoint{1.473608in}{1.465747in}}%
\pgfpathlineto{\pgfqpoint{1.473168in}{1.478673in}}%
\pgfpathlineto{\pgfqpoint{1.472424in}{1.483782in}}%
\pgfpathlineto{\pgfqpoint{1.459155in}{1.490246in}}%
\pgfpathlineto{\pgfqpoint{1.445886in}{1.490427in}}%
\pgfpathlineto{\pgfqpoint{1.432617in}{1.489961in}}%
\pgfpathlineto{\pgfqpoint{1.419348in}{1.484585in}}%
\pgfpathlineto{\pgfqpoint{1.417670in}{1.478673in}}%
\pgfpathlineto{\pgfqpoint{1.416839in}{1.465747in}}%
\pgfpathlineto{\pgfqpoint{1.416734in}{1.452820in}}%
\pgfpathlineto{\pgfqpoint{1.416991in}{1.439894in}}%
\pgfpathlineto{\pgfqpoint{1.417995in}{1.426967in}}%
\pgfpathlineto{\pgfqpoint{1.419348in}{1.421413in}}%
\pgfpathlineto{\pgfqpoint{1.427597in}{1.414041in}}%
\pgfpathclose%
\pgfpathmoveto{\pgfqpoint{1.428869in}{1.426967in}}%
\pgfpathlineto{\pgfqpoint{1.423973in}{1.439894in}}%
\pgfpathlineto{\pgfqpoint{1.423045in}{1.452820in}}%
\pgfpathlineto{\pgfqpoint{1.424429in}{1.465747in}}%
\pgfpathlineto{\pgfqpoint{1.431416in}{1.478673in}}%
\pgfpathlineto{\pgfqpoint{1.432617in}{1.479564in}}%
\pgfpathlineto{\pgfqpoint{1.445886in}{1.482219in}}%
\pgfpathlineto{\pgfqpoint{1.459155in}{1.478909in}}%
\pgfpathlineto{\pgfqpoint{1.459442in}{1.478673in}}%
\pgfpathlineto{\pgfqpoint{1.465966in}{1.465747in}}%
\pgfpathlineto{\pgfqpoint{1.467229in}{1.452820in}}%
\pgfpathlineto{\pgfqpoint{1.466371in}{1.439894in}}%
\pgfpathlineto{\pgfqpoint{1.461813in}{1.426967in}}%
\pgfpathlineto{\pgfqpoint{1.459155in}{1.424146in}}%
\pgfpathlineto{\pgfqpoint{1.445886in}{1.420479in}}%
\pgfpathlineto{\pgfqpoint{1.432617in}{1.423397in}}%
\pgfpathclose%
\pgfusepath{fill}%
\end{pgfscope}%
\begin{pgfscope}%
\pgfpathrectangle{\pgfqpoint{0.211875in}{0.211875in}}{\pgfqpoint{1.313625in}{1.279725in}}%
\pgfusepath{clip}%
\pgfsetbuttcap%
\pgfsetroundjoin%
\definecolor{currentfill}{rgb}{0.901975,0.231521,0.249182}%
\pgfsetfillcolor{currentfill}%
\pgfsetlinewidth{0.000000pt}%
\definecolor{currentstroke}{rgb}{0.000000,0.000000,0.000000}%
\pgfsetstrokecolor{currentstroke}%
\pgfsetdash{}{0pt}%
\pgfpathmoveto{\pgfqpoint{0.225144in}{1.418526in}}%
\pgfpathlineto{\pgfqpoint{0.230764in}{1.426967in}}%
\pgfpathlineto{\pgfqpoint{0.232551in}{1.439894in}}%
\pgfpathlineto{\pgfqpoint{0.232753in}{1.452820in}}%
\pgfpathlineto{\pgfqpoint{0.231912in}{1.465747in}}%
\pgfpathlineto{\pgfqpoint{0.228091in}{1.478673in}}%
\pgfpathlineto{\pgfqpoint{0.225144in}{1.481799in}}%
\pgfpathlineto{\pgfqpoint{0.211875in}{1.485282in}}%
\pgfpathlineto{\pgfqpoint{0.211875in}{1.478673in}}%
\pgfpathlineto{\pgfqpoint{0.211875in}{1.476517in}}%
\pgfpathlineto{\pgfqpoint{0.224254in}{1.465747in}}%
\pgfpathlineto{\pgfqpoint{0.225144in}{1.462851in}}%
\pgfpathlineto{\pgfqpoint{0.226653in}{1.452820in}}%
\pgfpathlineto{\pgfqpoint{0.225860in}{1.439894in}}%
\pgfpathlineto{\pgfqpoint{0.225144in}{1.437422in}}%
\pgfpathlineto{\pgfqpoint{0.217115in}{1.426967in}}%
\pgfpathlineto{\pgfqpoint{0.211875in}{1.424331in}}%
\pgfpathlineto{\pgfqpoint{0.211875in}{1.415409in}}%
\pgfpathclose%
\pgfusepath{fill}%
\end{pgfscope}%
\begin{pgfscope}%
\pgfpathrectangle{\pgfqpoint{0.211875in}{0.211875in}}{\pgfqpoint{1.313625in}{1.279725in}}%
\pgfusepath{clip}%
\pgfsetbuttcap%
\pgfsetroundjoin%
\definecolor{currentfill}{rgb}{0.901975,0.231521,0.249182}%
\pgfsetfillcolor{currentfill}%
\pgfsetlinewidth{0.000000pt}%
\definecolor{currentstroke}{rgb}{0.000000,0.000000,0.000000}%
\pgfsetstrokecolor{currentstroke}%
\pgfsetdash{}{0pt}%
\pgfpathmoveto{\pgfqpoint{1.525500in}{1.491471in}}%
\pgfpathlineto{\pgfqpoint{1.525500in}{1.491600in}}%
\pgfpathlineto{\pgfqpoint{1.519501in}{1.491600in}}%
\pgfpathclose%
\pgfusepath{fill}%
\end{pgfscope}%
\begin{pgfscope}%
\pgfpathrectangle{\pgfqpoint{0.211875in}{0.211875in}}{\pgfqpoint{1.313625in}{1.279725in}}%
\pgfusepath{clip}%
\pgfsetbuttcap%
\pgfsetroundjoin%
\definecolor{currentfill}{rgb}{0.947270,0.405591,0.279023}%
\pgfsetfillcolor{currentfill}%
\pgfsetlinewidth{0.000000pt}%
\definecolor{currentstroke}{rgb}{0.000000,0.000000,0.000000}%
\pgfsetstrokecolor{currentstroke}%
\pgfsetdash{}{0pt}%
\pgfpathmoveto{\pgfqpoint{0.384371in}{0.218614in}}%
\pgfpathlineto{\pgfqpoint{0.391807in}{0.224802in}}%
\pgfpathlineto{\pgfqpoint{0.393870in}{0.237728in}}%
\pgfpathlineto{\pgfqpoint{0.384371in}{0.248943in}}%
\pgfpathlineto{\pgfqpoint{0.374346in}{0.237728in}}%
\pgfpathlineto{\pgfqpoint{0.376548in}{0.224802in}}%
\pgfpathclose%
\pgfusepath{fill}%
\end{pgfscope}%
\begin{pgfscope}%
\pgfpathrectangle{\pgfqpoint{0.211875in}{0.211875in}}{\pgfqpoint{1.313625in}{1.279725in}}%
\pgfusepath{clip}%
\pgfsetbuttcap%
\pgfsetroundjoin%
\definecolor{currentfill}{rgb}{0.947270,0.405591,0.279023}%
\pgfsetfillcolor{currentfill}%
\pgfsetlinewidth{0.000000pt}%
\definecolor{currentstroke}{rgb}{0.000000,0.000000,0.000000}%
\pgfsetstrokecolor{currentstroke}%
\pgfsetdash{}{0pt}%
\pgfpathmoveto{\pgfqpoint{0.490523in}{0.220558in}}%
\pgfpathlineto{\pgfqpoint{0.503792in}{0.213188in}}%
\pgfpathlineto{\pgfqpoint{0.515207in}{0.224802in}}%
\pgfpathlineto{\pgfqpoint{0.516768in}{0.237728in}}%
\pgfpathlineto{\pgfqpoint{0.509017in}{0.250655in}}%
\pgfpathlineto{\pgfqpoint{0.503792in}{0.253650in}}%
\pgfpathlineto{\pgfqpoint{0.495255in}{0.250655in}}%
\pgfpathlineto{\pgfqpoint{0.490523in}{0.246723in}}%
\pgfpathlineto{\pgfqpoint{0.487866in}{0.237728in}}%
\pgfpathlineto{\pgfqpoint{0.488693in}{0.224802in}}%
\pgfpathclose%
\pgfusepath{fill}%
\end{pgfscope}%
\begin{pgfscope}%
\pgfpathrectangle{\pgfqpoint{0.211875in}{0.211875in}}{\pgfqpoint{1.313625in}{1.279725in}}%
\pgfusepath{clip}%
\pgfsetbuttcap%
\pgfsetroundjoin%
\definecolor{currentfill}{rgb}{0.947270,0.405591,0.279023}%
\pgfsetfillcolor{currentfill}%
\pgfsetlinewidth{0.000000pt}%
\definecolor{currentstroke}{rgb}{0.000000,0.000000,0.000000}%
\pgfsetstrokecolor{currentstroke}%
\pgfsetdash{}{0pt}%
\pgfpathmoveto{\pgfqpoint{0.609943in}{0.213156in}}%
\pgfpathlineto{\pgfqpoint{0.613598in}{0.211875in}}%
\pgfpathlineto{\pgfqpoint{0.623212in}{0.211875in}}%
\pgfpathlineto{\pgfqpoint{0.626676in}{0.211875in}}%
\pgfpathlineto{\pgfqpoint{0.636442in}{0.224802in}}%
\pgfpathlineto{\pgfqpoint{0.636481in}{0.225199in}}%
\pgfpathlineto{\pgfqpoint{0.637129in}{0.237728in}}%
\pgfpathlineto{\pgfqpoint{0.636481in}{0.241149in}}%
\pgfpathlineto{\pgfqpoint{0.632169in}{0.250655in}}%
\pgfpathlineto{\pgfqpoint{0.623212in}{0.256764in}}%
\pgfpathlineto{\pgfqpoint{0.609943in}{0.253931in}}%
\pgfpathlineto{\pgfqpoint{0.607224in}{0.250655in}}%
\pgfpathlineto{\pgfqpoint{0.603505in}{0.237728in}}%
\pgfpathlineto{\pgfqpoint{0.604326in}{0.224802in}}%
\pgfpathclose%
\pgfpathmoveto{\pgfqpoint{0.619359in}{0.224802in}}%
\pgfpathlineto{\pgfqpoint{0.614892in}{0.237728in}}%
\pgfpathlineto{\pgfqpoint{0.623212in}{0.242637in}}%
\pgfpathlineto{\pgfqpoint{0.626077in}{0.237728in}}%
\pgfpathlineto{\pgfqpoint{0.624543in}{0.224802in}}%
\pgfpathlineto{\pgfqpoint{0.623212in}{0.223198in}}%
\pgfpathclose%
\pgfusepath{fill}%
\end{pgfscope}%
\begin{pgfscope}%
\pgfpathrectangle{\pgfqpoint{0.211875in}{0.211875in}}{\pgfqpoint{1.313625in}{1.279725in}}%
\pgfusepath{clip}%
\pgfsetbuttcap%
\pgfsetroundjoin%
\definecolor{currentfill}{rgb}{0.947270,0.405591,0.279023}%
\pgfsetfillcolor{currentfill}%
\pgfsetlinewidth{0.000000pt}%
\definecolor{currentstroke}{rgb}{0.000000,0.000000,0.000000}%
\pgfsetstrokecolor{currentstroke}%
\pgfsetdash{}{0pt}%
\pgfpathmoveto{\pgfqpoint{0.729364in}{0.211875in}}%
\pgfpathlineto{\pgfqpoint{0.742633in}{0.211875in}}%
\pgfpathlineto{\pgfqpoint{0.748842in}{0.211875in}}%
\pgfpathlineto{\pgfqpoint{0.755902in}{0.223688in}}%
\pgfpathlineto{\pgfqpoint{0.756191in}{0.224802in}}%
\pgfpathlineto{\pgfqpoint{0.756803in}{0.237728in}}%
\pgfpathlineto{\pgfqpoint{0.755902in}{0.243282in}}%
\pgfpathlineto{\pgfqpoint{0.753324in}{0.250655in}}%
\pgfpathlineto{\pgfqpoint{0.742633in}{0.259151in}}%
\pgfpathlineto{\pgfqpoint{0.729364in}{0.257846in}}%
\pgfpathlineto{\pgfqpoint{0.722624in}{0.250655in}}%
\pgfpathlineto{\pgfqpoint{0.719219in}{0.237728in}}%
\pgfpathlineto{\pgfqpoint{0.720051in}{0.224802in}}%
\pgfpathlineto{\pgfqpoint{0.726217in}{0.211875in}}%
\pgfpathclose%
\pgfpathmoveto{\pgfqpoint{0.728402in}{0.224802in}}%
\pgfpathlineto{\pgfqpoint{0.727371in}{0.237728in}}%
\pgfpathlineto{\pgfqpoint{0.729364in}{0.243020in}}%
\pgfpathlineto{\pgfqpoint{0.742633in}{0.246866in}}%
\pgfpathlineto{\pgfqpoint{0.747219in}{0.237728in}}%
\pgfpathlineto{\pgfqpoint{0.745939in}{0.224802in}}%
\pgfpathlineto{\pgfqpoint{0.742633in}{0.220181in}}%
\pgfpathlineto{\pgfqpoint{0.729364in}{0.223026in}}%
\pgfpathclose%
\pgfusepath{fill}%
\end{pgfscope}%
\begin{pgfscope}%
\pgfpathrectangle{\pgfqpoint{0.211875in}{0.211875in}}{\pgfqpoint{1.313625in}{1.279725in}}%
\pgfusepath{clip}%
\pgfsetbuttcap%
\pgfsetroundjoin%
\definecolor{currentfill}{rgb}{0.947270,0.405591,0.279023}%
\pgfsetfillcolor{currentfill}%
\pgfsetlinewidth{0.000000pt}%
\definecolor{currentstroke}{rgb}{0.000000,0.000000,0.000000}%
\pgfsetstrokecolor{currentstroke}%
\pgfsetdash{}{0pt}%
\pgfpathmoveto{\pgfqpoint{0.848784in}{0.211875in}}%
\pgfpathlineto{\pgfqpoint{0.862053in}{0.211875in}}%
\pgfpathlineto{\pgfqpoint{0.869391in}{0.211875in}}%
\pgfpathlineto{\pgfqpoint{0.875322in}{0.224348in}}%
\pgfpathlineto{\pgfqpoint{0.875425in}{0.224802in}}%
\pgfpathlineto{\pgfqpoint{0.875998in}{0.237728in}}%
\pgfpathlineto{\pgfqpoint{0.875322in}{0.242683in}}%
\pgfpathlineto{\pgfqpoint{0.873186in}{0.250655in}}%
\pgfpathlineto{\pgfqpoint{0.862053in}{0.260818in}}%
\pgfpathlineto{\pgfqpoint{0.848784in}{0.260629in}}%
\pgfpathlineto{\pgfqpoint{0.838175in}{0.250655in}}%
\pgfpathlineto{\pgfqpoint{0.835515in}{0.240458in}}%
\pgfpathlineto{\pgfqpoint{0.835148in}{0.237728in}}%
\pgfpathlineto{\pgfqpoint{0.835515in}{0.229345in}}%
\pgfpathlineto{\pgfqpoint{0.835825in}{0.224802in}}%
\pgfpathlineto{\pgfqpoint{0.841899in}{0.211875in}}%
\pgfpathclose%
\pgfpathmoveto{\pgfqpoint{0.844968in}{0.224802in}}%
\pgfpathlineto{\pgfqpoint{0.843883in}{0.237728in}}%
\pgfpathlineto{\pgfqpoint{0.848784in}{0.249188in}}%
\pgfpathlineto{\pgfqpoint{0.862053in}{0.249541in}}%
\pgfpathlineto{\pgfqpoint{0.867219in}{0.237728in}}%
\pgfpathlineto{\pgfqpoint{0.866116in}{0.224802in}}%
\pgfpathlineto{\pgfqpoint{0.862053in}{0.218311in}}%
\pgfpathlineto{\pgfqpoint{0.848784in}{0.218562in}}%
\pgfpathclose%
\pgfusepath{fill}%
\end{pgfscope}%
\begin{pgfscope}%
\pgfpathrectangle{\pgfqpoint{0.211875in}{0.211875in}}{\pgfqpoint{1.313625in}{1.279725in}}%
\pgfusepath{clip}%
\pgfsetbuttcap%
\pgfsetroundjoin%
\definecolor{currentfill}{rgb}{0.947270,0.405591,0.279023}%
\pgfsetfillcolor{currentfill}%
\pgfsetlinewidth{0.000000pt}%
\definecolor{currentstroke}{rgb}{0.000000,0.000000,0.000000}%
\pgfsetstrokecolor{currentstroke}%
\pgfsetdash{}{0pt}%
\pgfpathmoveto{\pgfqpoint{0.954936in}{0.216900in}}%
\pgfpathlineto{\pgfqpoint{0.957836in}{0.211875in}}%
\pgfpathlineto{\pgfqpoint{0.968205in}{0.211875in}}%
\pgfpathlineto{\pgfqpoint{0.981473in}{0.211875in}}%
\pgfpathlineto{\pgfqpoint{0.988816in}{0.211875in}}%
\pgfpathlineto{\pgfqpoint{0.993972in}{0.224802in}}%
\pgfpathlineto{\pgfqpoint{0.994742in}{0.237700in}}%
\pgfpathlineto{\pgfqpoint{0.994744in}{0.237728in}}%
\pgfpathlineto{\pgfqpoint{0.994742in}{0.237738in}}%
\pgfpathlineto{\pgfqpoint{0.992146in}{0.250655in}}%
\pgfpathlineto{\pgfqpoint{0.981473in}{0.261742in}}%
\pgfpathlineto{\pgfqpoint{0.968205in}{0.262445in}}%
\pgfpathlineto{\pgfqpoint{0.954936in}{0.252245in}}%
\pgfpathlineto{\pgfqpoint{0.954280in}{0.250655in}}%
\pgfpathlineto{\pgfqpoint{0.952352in}{0.237728in}}%
\pgfpathlineto{\pgfqpoint{0.952914in}{0.224802in}}%
\pgfpathclose%
\pgfpathmoveto{\pgfqpoint{0.961767in}{0.224802in}}%
\pgfpathlineto{\pgfqpoint{0.960594in}{0.237728in}}%
\pgfpathlineto{\pgfqpoint{0.966562in}{0.250655in}}%
\pgfpathlineto{\pgfqpoint{0.968205in}{0.252004in}}%
\pgfpathlineto{\pgfqpoint{0.981180in}{0.250655in}}%
\pgfpathlineto{\pgfqpoint{0.981473in}{0.250591in}}%
\pgfpathlineto{\pgfqpoint{0.986417in}{0.237728in}}%
\pgfpathlineto{\pgfqpoint{0.985437in}{0.224802in}}%
\pgfpathlineto{\pgfqpoint{0.981473in}{0.217633in}}%
\pgfpathlineto{\pgfqpoint{0.968205in}{0.215569in}}%
\pgfpathclose%
\pgfusepath{fill}%
\end{pgfscope}%
\begin{pgfscope}%
\pgfpathrectangle{\pgfqpoint{0.211875in}{0.211875in}}{\pgfqpoint{1.313625in}{1.279725in}}%
\pgfusepath{clip}%
\pgfsetbuttcap%
\pgfsetroundjoin%
\definecolor{currentfill}{rgb}{0.947270,0.405591,0.279023}%
\pgfsetfillcolor{currentfill}%
\pgfsetlinewidth{0.000000pt}%
\definecolor{currentstroke}{rgb}{0.000000,0.000000,0.000000}%
\pgfsetstrokecolor{currentstroke}%
\pgfsetdash{}{0pt}%
\pgfpathmoveto{\pgfqpoint{1.074356in}{0.211875in}}%
\pgfpathlineto{\pgfqpoint{1.087625in}{0.211875in}}%
\pgfpathlineto{\pgfqpoint{1.100894in}{0.211875in}}%
\pgfpathlineto{\pgfqpoint{1.107408in}{0.211875in}}%
\pgfpathlineto{\pgfqpoint{1.112007in}{0.224802in}}%
\pgfpathlineto{\pgfqpoint{1.112710in}{0.237728in}}%
\pgfpathlineto{\pgfqpoint{1.110429in}{0.250655in}}%
\pgfpathlineto{\pgfqpoint{1.100894in}{0.261862in}}%
\pgfpathlineto{\pgfqpoint{1.087625in}{0.263402in}}%
\pgfpathlineto{\pgfqpoint{1.074356in}{0.256072in}}%
\pgfpathlineto{\pgfqpoint{1.071771in}{0.250655in}}%
\pgfpathlineto{\pgfqpoint{1.069925in}{0.237728in}}%
\pgfpathlineto{\pgfqpoint{1.070490in}{0.224802in}}%
\pgfpathlineto{\pgfqpoint{1.074194in}{0.211875in}}%
\pgfpathclose%
\pgfpathmoveto{\pgfqpoint{1.078838in}{0.224802in}}%
\pgfpathlineto{\pgfqpoint{1.077526in}{0.237728in}}%
\pgfpathlineto{\pgfqpoint{1.084079in}{0.250655in}}%
\pgfpathlineto{\pgfqpoint{1.087625in}{0.253166in}}%
\pgfpathlineto{\pgfqpoint{1.099239in}{0.250655in}}%
\pgfpathlineto{\pgfqpoint{1.100894in}{0.249862in}}%
\pgfpathlineto{\pgfqpoint{1.105013in}{0.237728in}}%
\pgfpathlineto{\pgfqpoint{1.104119in}{0.224802in}}%
\pgfpathlineto{\pgfqpoint{1.100894in}{0.218245in}}%
\pgfpathlineto{\pgfqpoint{1.087625in}{0.213896in}}%
\pgfpathclose%
\pgfusepath{fill}%
\end{pgfscope}%
\begin{pgfscope}%
\pgfpathrectangle{\pgfqpoint{0.211875in}{0.211875in}}{\pgfqpoint{1.313625in}{1.279725in}}%
\pgfusepath{clip}%
\pgfsetbuttcap%
\pgfsetroundjoin%
\definecolor{currentfill}{rgb}{0.947270,0.405591,0.279023}%
\pgfsetfillcolor{currentfill}%
\pgfsetlinewidth{0.000000pt}%
\definecolor{currentstroke}{rgb}{0.000000,0.000000,0.000000}%
\pgfsetstrokecolor{currentstroke}%
\pgfsetdash{}{0pt}%
\pgfpathmoveto{\pgfqpoint{1.193777in}{0.211875in}}%
\pgfpathlineto{\pgfqpoint{1.207045in}{0.211875in}}%
\pgfpathlineto{\pgfqpoint{1.220314in}{0.211875in}}%
\pgfpathlineto{\pgfqpoint{1.225341in}{0.211875in}}%
\pgfpathlineto{\pgfqpoint{1.229666in}{0.224802in}}%
\pgfpathlineto{\pgfqpoint{1.230326in}{0.237728in}}%
\pgfpathlineto{\pgfqpoint{1.228171in}{0.250655in}}%
\pgfpathlineto{\pgfqpoint{1.220314in}{0.261073in}}%
\pgfpathlineto{\pgfqpoint{1.207045in}{0.263568in}}%
\pgfpathlineto{\pgfqpoint{1.193777in}{0.258018in}}%
\pgfpathlineto{\pgfqpoint{1.189758in}{0.250655in}}%
\pgfpathlineto{\pgfqpoint{1.187868in}{0.237728in}}%
\pgfpathlineto{\pgfqpoint{1.188451in}{0.224802in}}%
\pgfpathlineto{\pgfqpoint{1.192259in}{0.211875in}}%
\pgfpathclose%
\pgfpathmoveto{\pgfqpoint{1.196265in}{0.224802in}}%
\pgfpathlineto{\pgfqpoint{1.194739in}{0.237728in}}%
\pgfpathlineto{\pgfqpoint{1.202359in}{0.250655in}}%
\pgfpathlineto{\pgfqpoint{1.207045in}{0.253463in}}%
\pgfpathlineto{\pgfqpoint{1.215687in}{0.250655in}}%
\pgfpathlineto{\pgfqpoint{1.220314in}{0.247089in}}%
\pgfpathlineto{\pgfqpoint{1.223127in}{0.237728in}}%
\pgfpathlineto{\pgfqpoint{1.222290in}{0.224802in}}%
\pgfpathlineto{\pgfqpoint{1.220314in}{0.220304in}}%
\pgfpathlineto{\pgfqpoint{1.207045in}{0.213450in}}%
\pgfpathclose%
\pgfusepath{fill}%
\end{pgfscope}%
\begin{pgfscope}%
\pgfpathrectangle{\pgfqpoint{0.211875in}{0.211875in}}{\pgfqpoint{1.313625in}{1.279725in}}%
\pgfusepath{clip}%
\pgfsetbuttcap%
\pgfsetroundjoin%
\definecolor{currentfill}{rgb}{0.947270,0.405591,0.279023}%
\pgfsetfillcolor{currentfill}%
\pgfsetlinewidth{0.000000pt}%
\definecolor{currentstroke}{rgb}{0.000000,0.000000,0.000000}%
\pgfsetstrokecolor{currentstroke}%
\pgfsetdash{}{0pt}%
\pgfpathmoveto{\pgfqpoint{1.313197in}{0.211875in}}%
\pgfpathlineto{\pgfqpoint{1.326466in}{0.211875in}}%
\pgfpathlineto{\pgfqpoint{1.339735in}{0.211875in}}%
\pgfpathlineto{\pgfqpoint{1.342721in}{0.211875in}}%
\pgfpathlineto{\pgfqpoint{1.346999in}{0.224802in}}%
\pgfpathlineto{\pgfqpoint{1.347635in}{0.237728in}}%
\pgfpathlineto{\pgfqpoint{1.345453in}{0.250655in}}%
\pgfpathlineto{\pgfqpoint{1.339735in}{0.259207in}}%
\pgfpathlineto{\pgfqpoint{1.326466in}{0.262975in}}%
\pgfpathlineto{\pgfqpoint{1.313197in}{0.258598in}}%
\pgfpathlineto{\pgfqpoint{1.308274in}{0.250655in}}%
\pgfpathlineto{\pgfqpoint{1.306201in}{0.237728in}}%
\pgfpathlineto{\pgfqpoint{1.306817in}{0.224802in}}%
\pgfpathlineto{\pgfqpoint{1.310931in}{0.211875in}}%
\pgfpathclose%
\pgfpathmoveto{\pgfqpoint{1.314227in}{0.224802in}}%
\pgfpathlineto{\pgfqpoint{1.313197in}{0.231861in}}%
\pgfpathlineto{\pgfqpoint{1.312847in}{0.237728in}}%
\pgfpathlineto{\pgfqpoint{1.313197in}{0.239158in}}%
\pgfpathlineto{\pgfqpoint{1.321842in}{0.250655in}}%
\pgfpathlineto{\pgfqpoint{1.326466in}{0.252930in}}%
\pgfpathlineto{\pgfqpoint{1.331737in}{0.250655in}}%
\pgfpathlineto{\pgfqpoint{1.339735in}{0.241857in}}%
\pgfpathlineto{\pgfqpoint{1.340832in}{0.237728in}}%
\pgfpathlineto{\pgfqpoint{1.340029in}{0.224802in}}%
\pgfpathlineto{\pgfqpoint{1.339735in}{0.224054in}}%
\pgfpathlineto{\pgfqpoint{1.326466in}{0.214183in}}%
\pgfpathclose%
\pgfusepath{fill}%
\end{pgfscope}%
\begin{pgfscope}%
\pgfpathrectangle{\pgfqpoint{0.211875in}{0.211875in}}{\pgfqpoint{1.313625in}{1.279725in}}%
\pgfusepath{clip}%
\pgfsetbuttcap%
\pgfsetroundjoin%
\definecolor{currentfill}{rgb}{0.947270,0.405591,0.279023}%
\pgfsetfillcolor{currentfill}%
\pgfsetlinewidth{0.000000pt}%
\definecolor{currentstroke}{rgb}{0.000000,0.000000,0.000000}%
\pgfsetstrokecolor{currentstroke}%
\pgfsetdash{}{0pt}%
\pgfpathmoveto{\pgfqpoint{1.432617in}{0.211875in}}%
\pgfpathlineto{\pgfqpoint{1.445886in}{0.211875in}}%
\pgfpathlineto{\pgfqpoint{1.459155in}{0.211875in}}%
\pgfpathlineto{\pgfqpoint{1.459607in}{0.211875in}}%
\pgfpathlineto{\pgfqpoint{1.464027in}{0.224802in}}%
\pgfpathlineto{\pgfqpoint{1.464657in}{0.237728in}}%
\pgfpathlineto{\pgfqpoint{1.462318in}{0.250655in}}%
\pgfpathlineto{\pgfqpoint{1.459155in}{0.256001in}}%
\pgfpathlineto{\pgfqpoint{1.445886in}{0.261624in}}%
\pgfpathlineto{\pgfqpoint{1.432617in}{0.258122in}}%
\pgfpathlineto{\pgfqpoint{1.427385in}{0.250655in}}%
\pgfpathlineto{\pgfqpoint{1.424961in}{0.237728in}}%
\pgfpathlineto{\pgfqpoint{1.425630in}{0.224802in}}%
\pgfpathlineto{\pgfqpoint{1.430286in}{0.211875in}}%
\pgfpathclose%
\pgfpathmoveto{\pgfqpoint{1.433126in}{0.224802in}}%
\pgfpathlineto{\pgfqpoint{1.432617in}{0.227449in}}%
\pgfpathlineto{\pgfqpoint{1.431950in}{0.237728in}}%
\pgfpathlineto{\pgfqpoint{1.432617in}{0.240133in}}%
\pgfpathlineto{\pgfqpoint{1.443509in}{0.250655in}}%
\pgfpathlineto{\pgfqpoint{1.445886in}{0.251572in}}%
\pgfpathlineto{\pgfqpoint{1.447591in}{0.250655in}}%
\pgfpathlineto{\pgfqpoint{1.456929in}{0.237728in}}%
\pgfpathlineto{\pgfqpoint{1.455150in}{0.224802in}}%
\pgfpathlineto{\pgfqpoint{1.445886in}{0.216091in}}%
\pgfpathclose%
\pgfusepath{fill}%
\end{pgfscope}%
\begin{pgfscope}%
\pgfpathrectangle{\pgfqpoint{0.211875in}{0.211875in}}{\pgfqpoint{1.313625in}{1.279725in}}%
\pgfusepath{clip}%
\pgfsetbuttcap%
\pgfsetroundjoin%
\definecolor{currentfill}{rgb}{0.947270,0.405591,0.279023}%
\pgfsetfillcolor{currentfill}%
\pgfsetlinewidth{0.000000pt}%
\definecolor{currentstroke}{rgb}{0.000000,0.000000,0.000000}%
\pgfsetstrokecolor{currentstroke}%
\pgfsetdash{}{0pt}%
\pgfpathmoveto{\pgfqpoint{0.264951in}{0.227600in}}%
\pgfpathlineto{\pgfqpoint{0.267220in}{0.237728in}}%
\pgfpathlineto{\pgfqpoint{0.264951in}{0.239813in}}%
\pgfpathlineto{\pgfqpoint{0.263440in}{0.237728in}}%
\pgfpathclose%
\pgfusepath{fill}%
\end{pgfscope}%
\begin{pgfscope}%
\pgfpathrectangle{\pgfqpoint{0.211875in}{0.211875in}}{\pgfqpoint{1.313625in}{1.279725in}}%
\pgfusepath{clip}%
\pgfsetbuttcap%
\pgfsetroundjoin%
\definecolor{currentfill}{rgb}{0.947270,0.405591,0.279023}%
\pgfsetfillcolor{currentfill}%
\pgfsetlinewidth{0.000000pt}%
\definecolor{currentstroke}{rgb}{0.000000,0.000000,0.000000}%
\pgfsetstrokecolor{currentstroke}%
\pgfsetdash{}{0pt}%
\pgfpathmoveto{\pgfqpoint{0.556867in}{0.289295in}}%
\pgfpathlineto{\pgfqpoint{0.558163in}{0.289434in}}%
\pgfpathlineto{\pgfqpoint{0.570136in}{0.291490in}}%
\pgfpathlineto{\pgfqpoint{0.577137in}{0.302361in}}%
\pgfpathlineto{\pgfqpoint{0.578986in}{0.315287in}}%
\pgfpathlineto{\pgfqpoint{0.577039in}{0.328214in}}%
\pgfpathlineto{\pgfqpoint{0.570136in}{0.338000in}}%
\pgfpathlineto{\pgfqpoint{0.556867in}{0.339917in}}%
\pgfpathlineto{\pgfqpoint{0.544244in}{0.328214in}}%
\pgfpathlineto{\pgfqpoint{0.543598in}{0.325621in}}%
\pgfpathlineto{\pgfqpoint{0.542371in}{0.315287in}}%
\pgfpathlineto{\pgfqpoint{0.543598in}{0.304349in}}%
\pgfpathlineto{\pgfqpoint{0.544037in}{0.302361in}}%
\pgfpathlineto{\pgfqpoint{0.556573in}{0.289434in}}%
\pgfpathclose%
\pgfpathmoveto{\pgfqpoint{0.555717in}{0.302361in}}%
\pgfpathlineto{\pgfqpoint{0.552026in}{0.315287in}}%
\pgfpathlineto{\pgfqpoint{0.556693in}{0.328214in}}%
\pgfpathlineto{\pgfqpoint{0.556867in}{0.328376in}}%
\pgfpathlineto{\pgfqpoint{0.557579in}{0.328214in}}%
\pgfpathlineto{\pgfqpoint{0.570136in}{0.319974in}}%
\pgfpathlineto{\pgfqpoint{0.571229in}{0.315287in}}%
\pgfpathlineto{\pgfqpoint{0.570136in}{0.309674in}}%
\pgfpathlineto{\pgfqpoint{0.561503in}{0.302361in}}%
\pgfpathlineto{\pgfqpoint{0.556867in}{0.301183in}}%
\pgfpathclose%
\pgfusepath{fill}%
\end{pgfscope}%
\begin{pgfscope}%
\pgfpathrectangle{\pgfqpoint{0.211875in}{0.211875in}}{\pgfqpoint{1.313625in}{1.279725in}}%
\pgfusepath{clip}%
\pgfsetbuttcap%
\pgfsetroundjoin%
\definecolor{currentfill}{rgb}{0.947270,0.405591,0.279023}%
\pgfsetfillcolor{currentfill}%
\pgfsetlinewidth{0.000000pt}%
\definecolor{currentstroke}{rgb}{0.000000,0.000000,0.000000}%
\pgfsetstrokecolor{currentstroke}%
\pgfsetdash{}{0pt}%
\pgfpathmoveto{\pgfqpoint{0.676288in}{0.286614in}}%
\pgfpathlineto{\pgfqpoint{0.689557in}{0.288876in}}%
\pgfpathlineto{\pgfqpoint{0.690266in}{0.289434in}}%
\pgfpathlineto{\pgfqpoint{0.697428in}{0.302361in}}%
\pgfpathlineto{\pgfqpoint{0.698936in}{0.315287in}}%
\pgfpathlineto{\pgfqpoint{0.697497in}{0.328214in}}%
\pgfpathlineto{\pgfqpoint{0.689557in}{0.340929in}}%
\pgfpathlineto{\pgfqpoint{0.688673in}{0.341140in}}%
\pgfpathlineto{\pgfqpoint{0.676288in}{0.342966in}}%
\pgfpathlineto{\pgfqpoint{0.671011in}{0.341140in}}%
\pgfpathlineto{\pgfqpoint{0.663019in}{0.335551in}}%
\pgfpathlineto{\pgfqpoint{0.659914in}{0.328214in}}%
\pgfpathlineto{\pgfqpoint{0.658601in}{0.315287in}}%
\pgfpathlineto{\pgfqpoint{0.659933in}{0.302361in}}%
\pgfpathlineto{\pgfqpoint{0.663019in}{0.294638in}}%
\pgfpathlineto{\pgfqpoint{0.669287in}{0.289434in}}%
\pgfpathclose%
\pgfpathmoveto{\pgfqpoint{0.670361in}{0.302361in}}%
\pgfpathlineto{\pgfqpoint{0.666462in}{0.315287in}}%
\pgfpathlineto{\pgfqpoint{0.671222in}{0.328214in}}%
\pgfpathlineto{\pgfqpoint{0.676288in}{0.332219in}}%
\pgfpathlineto{\pgfqpoint{0.687310in}{0.328214in}}%
\pgfpathlineto{\pgfqpoint{0.689557in}{0.325691in}}%
\pgfpathlineto{\pgfqpoint{0.691698in}{0.315287in}}%
\pgfpathlineto{\pgfqpoint{0.689557in}{0.303011in}}%
\pgfpathlineto{\pgfqpoint{0.689099in}{0.302361in}}%
\pgfpathlineto{\pgfqpoint{0.676288in}{0.297182in}}%
\pgfpathclose%
\pgfusepath{fill}%
\end{pgfscope}%
\begin{pgfscope}%
\pgfpathrectangle{\pgfqpoint{0.211875in}{0.211875in}}{\pgfqpoint{1.313625in}{1.279725in}}%
\pgfusepath{clip}%
\pgfsetbuttcap%
\pgfsetroundjoin%
\definecolor{currentfill}{rgb}{0.947270,0.405591,0.279023}%
\pgfsetfillcolor{currentfill}%
\pgfsetlinewidth{0.000000pt}%
\definecolor{currentstroke}{rgb}{0.000000,0.000000,0.000000}%
\pgfsetstrokecolor{currentstroke}%
\pgfsetdash{}{0pt}%
\pgfpathmoveto{\pgfqpoint{0.782439in}{0.288929in}}%
\pgfpathlineto{\pgfqpoint{0.795708in}{0.284618in}}%
\pgfpathlineto{\pgfqpoint{0.808977in}{0.287469in}}%
\pgfpathlineto{\pgfqpoint{0.811200in}{0.289434in}}%
\pgfpathlineto{\pgfqpoint{0.817032in}{0.302361in}}%
\pgfpathlineto{\pgfqpoint{0.818297in}{0.315287in}}%
\pgfpathlineto{\pgfqpoint{0.817211in}{0.328214in}}%
\pgfpathlineto{\pgfqpoint{0.810871in}{0.341140in}}%
\pgfpathlineto{\pgfqpoint{0.808977in}{0.342585in}}%
\pgfpathlineto{\pgfqpoint{0.795708in}{0.345003in}}%
\pgfpathlineto{\pgfqpoint{0.782439in}{0.341215in}}%
\pgfpathlineto{\pgfqpoint{0.782355in}{0.341140in}}%
\pgfpathlineto{\pgfqpoint{0.776182in}{0.328214in}}%
\pgfpathlineto{\pgfqpoint{0.775087in}{0.315287in}}%
\pgfpathlineto{\pgfqpoint{0.776316in}{0.302361in}}%
\pgfpathlineto{\pgfqpoint{0.781946in}{0.289434in}}%
\pgfpathclose%
\pgfpathmoveto{\pgfqpoint{0.784540in}{0.302361in}}%
\pgfpathlineto{\pgfqpoint{0.782439in}{0.308351in}}%
\pgfpathlineto{\pgfqpoint{0.781476in}{0.315287in}}%
\pgfpathlineto{\pgfqpoint{0.782439in}{0.321414in}}%
\pgfpathlineto{\pgfqpoint{0.785317in}{0.328214in}}%
\pgfpathlineto{\pgfqpoint{0.795708in}{0.335044in}}%
\pgfpathlineto{\pgfqpoint{0.808977in}{0.328527in}}%
\pgfpathlineto{\pgfqpoint{0.809151in}{0.328214in}}%
\pgfpathlineto{\pgfqpoint{0.811470in}{0.315287in}}%
\pgfpathlineto{\pgfqpoint{0.809493in}{0.302361in}}%
\pgfpathlineto{\pgfqpoint{0.808977in}{0.301352in}}%
\pgfpathlineto{\pgfqpoint{0.795708in}{0.294242in}}%
\pgfpathclose%
\pgfusepath{fill}%
\end{pgfscope}%
\begin{pgfscope}%
\pgfpathrectangle{\pgfqpoint{0.211875in}{0.211875in}}{\pgfqpoint{1.313625in}{1.279725in}}%
\pgfusepath{clip}%
\pgfsetbuttcap%
\pgfsetroundjoin%
\definecolor{currentfill}{rgb}{0.947270,0.405591,0.279023}%
\pgfsetfillcolor{currentfill}%
\pgfsetlinewidth{0.000000pt}%
\definecolor{currentstroke}{rgb}{0.000000,0.000000,0.000000}%
\pgfsetstrokecolor{currentstroke}%
\pgfsetdash{}{0pt}%
\pgfpathmoveto{\pgfqpoint{0.901860in}{0.286118in}}%
\pgfpathlineto{\pgfqpoint{0.915129in}{0.283259in}}%
\pgfpathlineto{\pgfqpoint{0.928398in}{0.286717in}}%
\pgfpathlineto{\pgfqpoint{0.931128in}{0.289434in}}%
\pgfpathlineto{\pgfqpoint{0.936072in}{0.302361in}}%
\pgfpathlineto{\pgfqpoint{0.937174in}{0.315287in}}%
\pgfpathlineto{\pgfqpoint{0.936318in}{0.328214in}}%
\pgfpathlineto{\pgfqpoint{0.931181in}{0.341140in}}%
\pgfpathlineto{\pgfqpoint{0.928398in}{0.343547in}}%
\pgfpathlineto{\pgfqpoint{0.915129in}{0.346399in}}%
\pgfpathlineto{\pgfqpoint{0.901860in}{0.343994in}}%
\pgfpathlineto{\pgfqpoint{0.898251in}{0.341140in}}%
\pgfpathlineto{\pgfqpoint{0.892752in}{0.328214in}}%
\pgfpathlineto{\pgfqpoint{0.891811in}{0.315287in}}%
\pgfpathlineto{\pgfqpoint{0.892980in}{0.302361in}}%
\pgfpathlineto{\pgfqpoint{0.898219in}{0.289434in}}%
\pgfpathclose%
\pgfpathmoveto{\pgfqpoint{0.900372in}{0.302361in}}%
\pgfpathlineto{\pgfqpoint{0.898517in}{0.315287in}}%
\pgfpathlineto{\pgfqpoint{0.900643in}{0.328214in}}%
\pgfpathlineto{\pgfqpoint{0.901860in}{0.330457in}}%
\pgfpathlineto{\pgfqpoint{0.915129in}{0.336926in}}%
\pgfpathlineto{\pgfqpoint{0.928398in}{0.328730in}}%
\pgfpathlineto{\pgfqpoint{0.928650in}{0.328214in}}%
\pgfpathlineto{\pgfqpoint{0.930674in}{0.315287in}}%
\pgfpathlineto{\pgfqpoint{0.928898in}{0.302361in}}%
\pgfpathlineto{\pgfqpoint{0.928398in}{0.301264in}}%
\pgfpathlineto{\pgfqpoint{0.915129in}{0.292289in}}%
\pgfpathlineto{\pgfqpoint{0.901860in}{0.299398in}}%
\pgfpathclose%
\pgfusepath{fill}%
\end{pgfscope}%
\begin{pgfscope}%
\pgfpathrectangle{\pgfqpoint{0.211875in}{0.211875in}}{\pgfqpoint{1.313625in}{1.279725in}}%
\pgfusepath{clip}%
\pgfsetbuttcap%
\pgfsetroundjoin%
\definecolor{currentfill}{rgb}{0.947270,0.405591,0.279023}%
\pgfsetfillcolor{currentfill}%
\pgfsetlinewidth{0.000000pt}%
\definecolor{currentstroke}{rgb}{0.000000,0.000000,0.000000}%
\pgfsetstrokecolor{currentstroke}%
\pgfsetdash{}{0pt}%
\pgfpathmoveto{\pgfqpoint{1.021280in}{0.284361in}}%
\pgfpathlineto{\pgfqpoint{1.034549in}{0.282512in}}%
\pgfpathlineto{\pgfqpoint{1.047818in}{0.286744in}}%
\pgfpathlineto{\pgfqpoint{1.050215in}{0.289434in}}%
\pgfpathlineto{\pgfqpoint{1.054631in}{0.302361in}}%
\pgfpathlineto{\pgfqpoint{1.055632in}{0.315287in}}%
\pgfpathlineto{\pgfqpoint{1.054907in}{0.328214in}}%
\pgfpathlineto{\pgfqpoint{1.050463in}{0.341140in}}%
\pgfpathlineto{\pgfqpoint{1.047818in}{0.343744in}}%
\pgfpathlineto{\pgfqpoint{1.034549in}{0.347181in}}%
\pgfpathlineto{\pgfqpoint{1.021280in}{0.345681in}}%
\pgfpathlineto{\pgfqpoint{1.014795in}{0.341140in}}%
\pgfpathlineto{\pgfqpoint{1.009621in}{0.328214in}}%
\pgfpathlineto{\pgfqpoint{1.008764in}{0.315287in}}%
\pgfpathlineto{\pgfqpoint{1.009923in}{0.302361in}}%
\pgfpathlineto{\pgfqpoint{1.015027in}{0.289434in}}%
\pgfpathclose%
\pgfpathmoveto{\pgfqpoint{1.017755in}{0.302361in}}%
\pgfpathlineto{\pgfqpoint{1.015868in}{0.315287in}}%
\pgfpathlineto{\pgfqpoint{1.017986in}{0.328214in}}%
\pgfpathlineto{\pgfqpoint{1.021280in}{0.333601in}}%
\pgfpathlineto{\pgfqpoint{1.034549in}{0.337904in}}%
\pgfpathlineto{\pgfqpoint{1.047272in}{0.328214in}}%
\pgfpathlineto{\pgfqpoint{1.047818in}{0.326578in}}%
\pgfpathlineto{\pgfqpoint{1.049391in}{0.315287in}}%
\pgfpathlineto{\pgfqpoint{1.047818in}{0.302886in}}%
\pgfpathlineto{\pgfqpoint{1.047669in}{0.302361in}}%
\pgfpathlineto{\pgfqpoint{1.034549in}{0.291289in}}%
\pgfpathlineto{\pgfqpoint{1.021280in}{0.296092in}}%
\pgfpathclose%
\pgfusepath{fill}%
\end{pgfscope}%
\begin{pgfscope}%
\pgfpathrectangle{\pgfqpoint{0.211875in}{0.211875in}}{\pgfqpoint{1.313625in}{1.279725in}}%
\pgfusepath{clip}%
\pgfsetbuttcap%
\pgfsetroundjoin%
\definecolor{currentfill}{rgb}{0.947270,0.405591,0.279023}%
\pgfsetfillcolor{currentfill}%
\pgfsetlinewidth{0.000000pt}%
\definecolor{currentstroke}{rgb}{0.000000,0.000000,0.000000}%
\pgfsetstrokecolor{currentstroke}%
\pgfsetdash{}{0pt}%
\pgfpathmoveto{\pgfqpoint{1.140701in}{0.283470in}}%
\pgfpathlineto{\pgfqpoint{1.153970in}{0.282373in}}%
\pgfpathlineto{\pgfqpoint{1.167239in}{0.287746in}}%
\pgfpathlineto{\pgfqpoint{1.168565in}{0.289434in}}%
\pgfpathlineto{\pgfqpoint{1.172756in}{0.302361in}}%
\pgfpathlineto{\pgfqpoint{1.173713in}{0.315287in}}%
\pgfpathlineto{\pgfqpoint{1.173033in}{0.328214in}}%
\pgfpathlineto{\pgfqpoint{1.168853in}{0.341140in}}%
\pgfpathlineto{\pgfqpoint{1.167239in}{0.342962in}}%
\pgfpathlineto{\pgfqpoint{1.153970in}{0.347354in}}%
\pgfpathlineto{\pgfqpoint{1.140701in}{0.346485in}}%
\pgfpathlineto{\pgfqpoint{1.132069in}{0.341140in}}%
\pgfpathlineto{\pgfqpoint{1.127432in}{0.330663in}}%
\pgfpathlineto{\pgfqpoint{1.126958in}{0.328214in}}%
\pgfpathlineto{\pgfqpoint{1.126311in}{0.315287in}}%
\pgfpathlineto{\pgfqpoint{1.127222in}{0.302361in}}%
\pgfpathlineto{\pgfqpoint{1.127432in}{0.301349in}}%
\pgfpathlineto{\pgfqpoint{1.132431in}{0.289434in}}%
\pgfpathclose%
\pgfpathmoveto{\pgfqpoint{1.135543in}{0.302361in}}%
\pgfpathlineto{\pgfqpoint{1.133555in}{0.315287in}}%
\pgfpathlineto{\pgfqpoint{1.135766in}{0.328214in}}%
\pgfpathlineto{\pgfqpoint{1.140701in}{0.335370in}}%
\pgfpathlineto{\pgfqpoint{1.153970in}{0.337981in}}%
\pgfpathlineto{\pgfqpoint{1.164849in}{0.328214in}}%
\pgfpathlineto{\pgfqpoint{1.167239in}{0.318891in}}%
\pgfpathlineto{\pgfqpoint{1.167673in}{0.315287in}}%
\pgfpathlineto{\pgfqpoint{1.167239in}{0.311476in}}%
\pgfpathlineto{\pgfqpoint{1.165170in}{0.302361in}}%
\pgfpathlineto{\pgfqpoint{1.153970in}{0.291239in}}%
\pgfpathlineto{\pgfqpoint{1.140701in}{0.294180in}}%
\pgfpathclose%
\pgfusepath{fill}%
\end{pgfscope}%
\begin{pgfscope}%
\pgfpathrectangle{\pgfqpoint{0.211875in}{0.211875in}}{\pgfqpoint{1.313625in}{1.279725in}}%
\pgfusepath{clip}%
\pgfsetbuttcap%
\pgfsetroundjoin%
\definecolor{currentfill}{rgb}{0.947270,0.405591,0.279023}%
\pgfsetfillcolor{currentfill}%
\pgfsetlinewidth{0.000000pt}%
\definecolor{currentstroke}{rgb}{0.000000,0.000000,0.000000}%
\pgfsetstrokecolor{currentstroke}%
\pgfsetdash{}{0pt}%
\pgfpathmoveto{\pgfqpoint{1.260121in}{0.283318in}}%
\pgfpathlineto{\pgfqpoint{1.273390in}{0.282862in}}%
\pgfpathlineto{\pgfqpoint{1.285975in}{0.289434in}}%
\pgfpathlineto{\pgfqpoint{1.286659in}{0.290299in}}%
\pgfpathlineto{\pgfqpoint{1.290474in}{0.302361in}}%
\pgfpathlineto{\pgfqpoint{1.291434in}{0.315287in}}%
\pgfpathlineto{\pgfqpoint{1.290723in}{0.328214in}}%
\pgfpathlineto{\pgfqpoint{1.286659in}{0.340735in}}%
\pgfpathlineto{\pgfqpoint{1.286288in}{0.341140in}}%
\pgfpathlineto{\pgfqpoint{1.273390in}{0.346895in}}%
\pgfpathlineto{\pgfqpoint{1.260121in}{0.346546in}}%
\pgfpathlineto{\pgfqpoint{1.250212in}{0.341140in}}%
\pgfpathlineto{\pgfqpoint{1.246852in}{0.335517in}}%
\pgfpathlineto{\pgfqpoint{1.245088in}{0.328214in}}%
\pgfpathlineto{\pgfqpoint{1.244415in}{0.315287in}}%
\pgfpathlineto{\pgfqpoint{1.245345in}{0.302361in}}%
\pgfpathlineto{\pgfqpoint{1.246852in}{0.296263in}}%
\pgfpathlineto{\pgfqpoint{1.250540in}{0.289434in}}%
\pgfpathclose%
\pgfpathmoveto{\pgfqpoint{1.253800in}{0.302361in}}%
\pgfpathlineto{\pgfqpoint{1.251626in}{0.315287in}}%
\pgfpathlineto{\pgfqpoint{1.254058in}{0.328214in}}%
\pgfpathlineto{\pgfqpoint{1.260121in}{0.335987in}}%
\pgfpathlineto{\pgfqpoint{1.273390in}{0.337121in}}%
\pgfpathlineto{\pgfqpoint{1.281963in}{0.328214in}}%
\pgfpathlineto{\pgfqpoint{1.284836in}{0.315287in}}%
\pgfpathlineto{\pgfqpoint{1.282276in}{0.302361in}}%
\pgfpathlineto{\pgfqpoint{1.273390in}{0.292173in}}%
\pgfpathlineto{\pgfqpoint{1.260121in}{0.293458in}}%
\pgfpathclose%
\pgfusepath{fill}%
\end{pgfscope}%
\begin{pgfscope}%
\pgfpathrectangle{\pgfqpoint{0.211875in}{0.211875in}}{\pgfqpoint{1.313625in}{1.279725in}}%
\pgfusepath{clip}%
\pgfsetbuttcap%
\pgfsetroundjoin%
\definecolor{currentfill}{rgb}{0.947270,0.405591,0.279023}%
\pgfsetfillcolor{currentfill}%
\pgfsetlinewidth{0.000000pt}%
\definecolor{currentstroke}{rgb}{0.000000,0.000000,0.000000}%
\pgfsetstrokecolor{currentstroke}%
\pgfsetdash{}{0pt}%
\pgfpathmoveto{\pgfqpoint{1.379542in}{0.283823in}}%
\pgfpathlineto{\pgfqpoint{1.392811in}{0.284025in}}%
\pgfpathlineto{\pgfqpoint{1.401872in}{0.289434in}}%
\pgfpathlineto{\pgfqpoint{1.406080in}{0.296128in}}%
\pgfpathlineto{\pgfqpoint{1.407792in}{0.302361in}}%
\pgfpathlineto{\pgfqpoint{1.408800in}{0.315287in}}%
\pgfpathlineto{\pgfqpoint{1.407988in}{0.328214in}}%
\pgfpathlineto{\pgfqpoint{1.406080in}{0.335153in}}%
\pgfpathlineto{\pgfqpoint{1.401831in}{0.341140in}}%
\pgfpathlineto{\pgfqpoint{1.392811in}{0.345756in}}%
\pgfpathlineto{\pgfqpoint{1.379542in}{0.345950in}}%
\pgfpathlineto{\pgfqpoint{1.369466in}{0.341140in}}%
\pgfpathlineto{\pgfqpoint{1.366273in}{0.337065in}}%
\pgfpathlineto{\pgfqpoint{1.363700in}{0.328214in}}%
\pgfpathlineto{\pgfqpoint{1.362927in}{0.315287in}}%
\pgfpathlineto{\pgfqpoint{1.363922in}{0.302361in}}%
\pgfpathlineto{\pgfqpoint{1.366273in}{0.294203in}}%
\pgfpathlineto{\pgfqpoint{1.369538in}{0.289434in}}%
\pgfpathclose%
\pgfpathmoveto{\pgfqpoint{1.372642in}{0.302361in}}%
\pgfpathlineto{\pgfqpoint{1.370168in}{0.315287in}}%
\pgfpathlineto{\pgfqpoint{1.372989in}{0.328214in}}%
\pgfpathlineto{\pgfqpoint{1.379542in}{0.335597in}}%
\pgfpathlineto{\pgfqpoint{1.392811in}{0.335249in}}%
\pgfpathlineto{\pgfqpoint{1.398735in}{0.328214in}}%
\pgfpathlineto{\pgfqpoint{1.401484in}{0.315287in}}%
\pgfpathlineto{\pgfqpoint{1.399085in}{0.302361in}}%
\pgfpathlineto{\pgfqpoint{1.392811in}{0.294163in}}%
\pgfpathlineto{\pgfqpoint{1.379542in}{0.293789in}}%
\pgfpathclose%
\pgfusepath{fill}%
\end{pgfscope}%
\begin{pgfscope}%
\pgfpathrectangle{\pgfqpoint{0.211875in}{0.211875in}}{\pgfqpoint{1.313625in}{1.279725in}}%
\pgfusepath{clip}%
\pgfsetbuttcap%
\pgfsetroundjoin%
\definecolor{currentfill}{rgb}{0.947270,0.405591,0.279023}%
\pgfsetfillcolor{currentfill}%
\pgfsetlinewidth{0.000000pt}%
\definecolor{currentstroke}{rgb}{0.000000,0.000000,0.000000}%
\pgfsetstrokecolor{currentstroke}%
\pgfsetdash{}{0pt}%
\pgfpathmoveto{\pgfqpoint{1.498962in}{0.284936in}}%
\pgfpathlineto{\pgfqpoint{1.512231in}{0.285939in}}%
\pgfpathlineto{\pgfqpoint{1.517401in}{0.289434in}}%
\pgfpathlineto{\pgfqpoint{1.524389in}{0.302361in}}%
\pgfpathlineto{\pgfqpoint{1.525500in}{0.311506in}}%
\pgfpathlineto{\pgfqpoint{1.525500in}{0.315287in}}%
\pgfpathlineto{\pgfqpoint{1.525500in}{0.319703in}}%
\pgfpathlineto{\pgfqpoint{1.524555in}{0.328214in}}%
\pgfpathlineto{\pgfqpoint{1.516899in}{0.341140in}}%
\pgfpathlineto{\pgfqpoint{1.512231in}{0.343855in}}%
\pgfpathlineto{\pgfqpoint{1.498962in}{0.344747in}}%
\pgfpathlineto{\pgfqpoint{1.490231in}{0.341140in}}%
\pgfpathlineto{\pgfqpoint{1.485693in}{0.336667in}}%
\pgfpathlineto{\pgfqpoint{1.482807in}{0.328214in}}%
\pgfpathlineto{\pgfqpoint{1.481854in}{0.315287in}}%
\pgfpathlineto{\pgfqpoint{1.482965in}{0.302361in}}%
\pgfpathlineto{\pgfqpoint{1.485693in}{0.294111in}}%
\pgfpathlineto{\pgfqpoint{1.489737in}{0.289434in}}%
\pgfpathclose%
\pgfpathmoveto{\pgfqpoint{1.492261in}{0.302361in}}%
\pgfpathlineto{\pgfqpoint{1.489329in}{0.315287in}}%
\pgfpathlineto{\pgfqpoint{1.492771in}{0.328214in}}%
\pgfpathlineto{\pgfqpoint{1.498962in}{0.334286in}}%
\pgfpathlineto{\pgfqpoint{1.512231in}{0.332239in}}%
\pgfpathlineto{\pgfqpoint{1.515221in}{0.328214in}}%
\pgfpathlineto{\pgfqpoint{1.517976in}{0.315287in}}%
\pgfpathlineto{\pgfqpoint{1.515642in}{0.302361in}}%
\pgfpathlineto{\pgfqpoint{1.512231in}{0.297327in}}%
\pgfpathlineto{\pgfqpoint{1.498962in}{0.295093in}}%
\pgfpathclose%
\pgfusepath{fill}%
\end{pgfscope}%
\begin{pgfscope}%
\pgfpathrectangle{\pgfqpoint{0.211875in}{0.211875in}}{\pgfqpoint{1.313625in}{1.279725in}}%
\pgfusepath{clip}%
\pgfsetbuttcap%
\pgfsetroundjoin%
\definecolor{currentfill}{rgb}{0.947270,0.405591,0.279023}%
\pgfsetfillcolor{currentfill}%
\pgfsetlinewidth{0.000000pt}%
\definecolor{currentstroke}{rgb}{0.000000,0.000000,0.000000}%
\pgfsetstrokecolor{currentstroke}%
\pgfsetdash{}{0pt}%
\pgfpathmoveto{\pgfqpoint{0.318027in}{0.300380in}}%
\pgfpathlineto{\pgfqpoint{0.331295in}{0.299549in}}%
\pgfpathlineto{\pgfqpoint{0.333617in}{0.302361in}}%
\pgfpathlineto{\pgfqpoint{0.336589in}{0.315287in}}%
\pgfpathlineto{\pgfqpoint{0.332937in}{0.328214in}}%
\pgfpathlineto{\pgfqpoint{0.331295in}{0.330013in}}%
\pgfpathlineto{\pgfqpoint{0.318027in}{0.329216in}}%
\pgfpathlineto{\pgfqpoint{0.317204in}{0.328214in}}%
\pgfpathlineto{\pgfqpoint{0.313763in}{0.315287in}}%
\pgfpathlineto{\pgfqpoint{0.316546in}{0.302361in}}%
\pgfpathclose%
\pgfusepath{fill}%
\end{pgfscope}%
\begin{pgfscope}%
\pgfpathrectangle{\pgfqpoint{0.211875in}{0.211875in}}{\pgfqpoint{1.313625in}{1.279725in}}%
\pgfusepath{clip}%
\pgfsetbuttcap%
\pgfsetroundjoin%
\definecolor{currentfill}{rgb}{0.947270,0.405591,0.279023}%
\pgfsetfillcolor{currentfill}%
\pgfsetlinewidth{0.000000pt}%
\definecolor{currentstroke}{rgb}{0.000000,0.000000,0.000000}%
\pgfsetstrokecolor{currentstroke}%
\pgfsetdash{}{0pt}%
\pgfpathmoveto{\pgfqpoint{0.437447in}{0.294183in}}%
\pgfpathlineto{\pgfqpoint{0.450716in}{0.295134in}}%
\pgfpathlineto{\pgfqpoint{0.455967in}{0.302361in}}%
\pgfpathlineto{\pgfqpoint{0.458288in}{0.315287in}}%
\pgfpathlineto{\pgfqpoint{0.455629in}{0.328214in}}%
\pgfpathlineto{\pgfqpoint{0.450716in}{0.334358in}}%
\pgfpathlineto{\pgfqpoint{0.437447in}{0.335175in}}%
\pgfpathlineto{\pgfqpoint{0.430933in}{0.328214in}}%
\pgfpathlineto{\pgfqpoint{0.427822in}{0.315287in}}%
\pgfpathlineto{\pgfqpoint{0.430499in}{0.302361in}}%
\pgfpathclose%
\pgfpathmoveto{\pgfqpoint{0.437230in}{0.315287in}}%
\pgfpathlineto{\pgfqpoint{0.437447in}{0.315976in}}%
\pgfpathlineto{\pgfqpoint{0.439926in}{0.315287in}}%
\pgfpathlineto{\pgfqpoint{0.437447in}{0.314439in}}%
\pgfpathclose%
\pgfusepath{fill}%
\end{pgfscope}%
\begin{pgfscope}%
\pgfpathrectangle{\pgfqpoint{0.211875in}{0.211875in}}{\pgfqpoint{1.313625in}{1.279725in}}%
\pgfusepath{clip}%
\pgfsetbuttcap%
\pgfsetroundjoin%
\definecolor{currentfill}{rgb}{0.947270,0.405591,0.279023}%
\pgfsetfillcolor{currentfill}%
\pgfsetlinewidth{0.000000pt}%
\definecolor{currentstroke}{rgb}{0.000000,0.000000,0.000000}%
\pgfsetstrokecolor{currentstroke}%
\pgfsetdash{}{0pt}%
\pgfpathmoveto{\pgfqpoint{0.213488in}{0.315287in}}%
\pgfpathlineto{\pgfqpoint{0.211875in}{0.319914in}}%
\pgfpathlineto{\pgfqpoint{0.211875in}{0.315287in}}%
\pgfpathlineto{\pgfqpoint{0.211875in}{0.309590in}}%
\pgfpathclose%
\pgfusepath{fill}%
\end{pgfscope}%
\begin{pgfscope}%
\pgfpathrectangle{\pgfqpoint{0.211875in}{0.211875in}}{\pgfqpoint{1.313625in}{1.279725in}}%
\pgfusepath{clip}%
\pgfsetbuttcap%
\pgfsetroundjoin%
\definecolor{currentfill}{rgb}{0.947270,0.405591,0.279023}%
\pgfsetfillcolor{currentfill}%
\pgfsetlinewidth{0.000000pt}%
\definecolor{currentstroke}{rgb}{0.000000,0.000000,0.000000}%
\pgfsetstrokecolor{currentstroke}%
\pgfsetdash{}{0pt}%
\pgfpathmoveto{\pgfqpoint{0.623212in}{0.366106in}}%
\pgfpathlineto{\pgfqpoint{0.625989in}{0.366993in}}%
\pgfpathlineto{\pgfqpoint{0.636481in}{0.373743in}}%
\pgfpathlineto{\pgfqpoint{0.639316in}{0.379920in}}%
\pgfpathlineto{\pgfqpoint{0.641198in}{0.392846in}}%
\pgfpathlineto{\pgfqpoint{0.640792in}{0.405773in}}%
\pgfpathlineto{\pgfqpoint{0.637009in}{0.418699in}}%
\pgfpathlineto{\pgfqpoint{0.636481in}{0.419443in}}%
\pgfpathlineto{\pgfqpoint{0.623212in}{0.425392in}}%
\pgfpathlineto{\pgfqpoint{0.609943in}{0.423671in}}%
\pgfpathlineto{\pgfqpoint{0.604034in}{0.418699in}}%
\pgfpathlineto{\pgfqpoint{0.599152in}{0.405773in}}%
\pgfpathlineto{\pgfqpoint{0.598593in}{0.392846in}}%
\pgfpathlineto{\pgfqpoint{0.600997in}{0.379920in}}%
\pgfpathlineto{\pgfqpoint{0.609943in}{0.368101in}}%
\pgfpathlineto{\pgfqpoint{0.616059in}{0.366993in}}%
\pgfpathclose%
\pgfpathmoveto{\pgfqpoint{0.609813in}{0.379920in}}%
\pgfpathlineto{\pgfqpoint{0.606013in}{0.392846in}}%
\pgfpathlineto{\pgfqpoint{0.607232in}{0.405773in}}%
\pgfpathlineto{\pgfqpoint{0.609943in}{0.410908in}}%
\pgfpathlineto{\pgfqpoint{0.623212in}{0.415292in}}%
\pgfpathlineto{\pgfqpoint{0.632001in}{0.405773in}}%
\pgfpathlineto{\pgfqpoint{0.633811in}{0.392846in}}%
\pgfpathlineto{\pgfqpoint{0.628072in}{0.379920in}}%
\pgfpathlineto{\pgfqpoint{0.623212in}{0.376226in}}%
\pgfpathlineto{\pgfqpoint{0.609943in}{0.379748in}}%
\pgfpathclose%
\pgfusepath{fill}%
\end{pgfscope}%
\begin{pgfscope}%
\pgfpathrectangle{\pgfqpoint{0.211875in}{0.211875in}}{\pgfqpoint{1.313625in}{1.279725in}}%
\pgfusepath{clip}%
\pgfsetbuttcap%
\pgfsetroundjoin%
\definecolor{currentfill}{rgb}{0.947270,0.405591,0.279023}%
\pgfsetfillcolor{currentfill}%
\pgfsetlinewidth{0.000000pt}%
\definecolor{currentstroke}{rgb}{0.000000,0.000000,0.000000}%
\pgfsetstrokecolor{currentstroke}%
\pgfsetdash{}{0pt}%
\pgfpathmoveto{\pgfqpoint{0.729364in}{0.364982in}}%
\pgfpathlineto{\pgfqpoint{0.742633in}{0.364182in}}%
\pgfpathlineto{\pgfqpoint{0.750138in}{0.366993in}}%
\pgfpathlineto{\pgfqpoint{0.755902in}{0.371725in}}%
\pgfpathlineto{\pgfqpoint{0.759202in}{0.379920in}}%
\pgfpathlineto{\pgfqpoint{0.760739in}{0.392846in}}%
\pgfpathlineto{\pgfqpoint{0.760482in}{0.405773in}}%
\pgfpathlineto{\pgfqpoint{0.757655in}{0.418699in}}%
\pgfpathlineto{\pgfqpoint{0.755902in}{0.421559in}}%
\pgfpathlineto{\pgfqpoint{0.742633in}{0.427592in}}%
\pgfpathlineto{\pgfqpoint{0.729364in}{0.426817in}}%
\pgfpathlineto{\pgfqpoint{0.718464in}{0.418699in}}%
\pgfpathlineto{\pgfqpoint{0.716095in}{0.412524in}}%
\pgfpathlineto{\pgfqpoint{0.714856in}{0.405773in}}%
\pgfpathlineto{\pgfqpoint{0.714565in}{0.392846in}}%
\pgfpathlineto{\pgfqpoint{0.716095in}{0.380113in}}%
\pgfpathlineto{\pgfqpoint{0.716133in}{0.379920in}}%
\pgfpathlineto{\pgfqpoint{0.725460in}{0.366993in}}%
\pgfpathclose%
\pgfpathmoveto{\pgfqpoint{0.725662in}{0.379920in}}%
\pgfpathlineto{\pgfqpoint{0.721965in}{0.392846in}}%
\pgfpathlineto{\pgfqpoint{0.723089in}{0.405773in}}%
\pgfpathlineto{\pgfqpoint{0.729364in}{0.416307in}}%
\pgfpathlineto{\pgfqpoint{0.742633in}{0.418342in}}%
\pgfpathlineto{\pgfqpoint{0.752613in}{0.405773in}}%
\pgfpathlineto{\pgfqpoint{0.753964in}{0.392846in}}%
\pgfpathlineto{\pgfqpoint{0.749436in}{0.379920in}}%
\pgfpathlineto{\pgfqpoint{0.742633in}{0.373912in}}%
\pgfpathlineto{\pgfqpoint{0.729364in}{0.375571in}}%
\pgfpathclose%
\pgfusepath{fill}%
\end{pgfscope}%
\begin{pgfscope}%
\pgfpathrectangle{\pgfqpoint{0.211875in}{0.211875in}}{\pgfqpoint{1.313625in}{1.279725in}}%
\pgfusepath{clip}%
\pgfsetbuttcap%
\pgfsetroundjoin%
\definecolor{currentfill}{rgb}{0.947270,0.405591,0.279023}%
\pgfsetfillcolor{currentfill}%
\pgfsetlinewidth{0.000000pt}%
\definecolor{currentstroke}{rgb}{0.000000,0.000000,0.000000}%
\pgfsetstrokecolor{currentstroke}%
\pgfsetdash{}{0pt}%
\pgfpathmoveto{\pgfqpoint{0.848784in}{0.362946in}}%
\pgfpathlineto{\pgfqpoint{0.862053in}{0.362797in}}%
\pgfpathlineto{\pgfqpoint{0.871769in}{0.366993in}}%
\pgfpathlineto{\pgfqpoint{0.875322in}{0.370699in}}%
\pgfpathlineto{\pgfqpoint{0.878542in}{0.379920in}}%
\pgfpathlineto{\pgfqpoint{0.879838in}{0.392846in}}%
\pgfpathlineto{\pgfqpoint{0.879685in}{0.405773in}}%
\pgfpathlineto{\pgfqpoint{0.877526in}{0.418699in}}%
\pgfpathlineto{\pgfqpoint{0.875322in}{0.422924in}}%
\pgfpathlineto{\pgfqpoint{0.862053in}{0.429208in}}%
\pgfpathlineto{\pgfqpoint{0.848784in}{0.429039in}}%
\pgfpathlineto{\pgfqpoint{0.835515in}{0.421939in}}%
\pgfpathlineto{\pgfqpoint{0.833864in}{0.418699in}}%
\pgfpathlineto{\pgfqpoint{0.831548in}{0.405773in}}%
\pgfpathlineto{\pgfqpoint{0.831363in}{0.392846in}}%
\pgfpathlineto{\pgfqpoint{0.832709in}{0.379920in}}%
\pgfpathlineto{\pgfqpoint{0.835515in}{0.371753in}}%
\pgfpathlineto{\pgfqpoint{0.839867in}{0.366993in}}%
\pgfpathclose%
\pgfpathmoveto{\pgfqpoint{0.841703in}{0.379920in}}%
\pgfpathlineto{\pgfqpoint{0.837996in}{0.392846in}}%
\pgfpathlineto{\pgfqpoint{0.839065in}{0.405773in}}%
\pgfpathlineto{\pgfqpoint{0.847310in}{0.418699in}}%
\pgfpathlineto{\pgfqpoint{0.848784in}{0.419666in}}%
\pgfpathlineto{\pgfqpoint{0.862053in}{0.419828in}}%
\pgfpathlineto{\pgfqpoint{0.863841in}{0.418699in}}%
\pgfpathlineto{\pgfqpoint{0.872181in}{0.405773in}}%
\pgfpathlineto{\pgfqpoint{0.873237in}{0.392846in}}%
\pgfpathlineto{\pgfqpoint{0.869511in}{0.379920in}}%
\pgfpathlineto{\pgfqpoint{0.862053in}{0.372377in}}%
\pgfpathlineto{\pgfqpoint{0.848784in}{0.372564in}}%
\pgfpathclose%
\pgfusepath{fill}%
\end{pgfscope}%
\begin{pgfscope}%
\pgfpathrectangle{\pgfqpoint{0.211875in}{0.211875in}}{\pgfqpoint{1.313625in}{1.279725in}}%
\pgfusepath{clip}%
\pgfsetbuttcap%
\pgfsetroundjoin%
\definecolor{currentfill}{rgb}{0.947270,0.405591,0.279023}%
\pgfsetfillcolor{currentfill}%
\pgfsetlinewidth{0.000000pt}%
\definecolor{currentstroke}{rgb}{0.000000,0.000000,0.000000}%
\pgfsetstrokecolor{currentstroke}%
\pgfsetdash{}{0pt}%
\pgfpathmoveto{\pgfqpoint{0.954936in}{0.366799in}}%
\pgfpathlineto{\pgfqpoint{0.968205in}{0.361623in}}%
\pgfpathlineto{\pgfqpoint{0.981473in}{0.361965in}}%
\pgfpathlineto{\pgfqpoint{0.991684in}{0.366993in}}%
\pgfpathlineto{\pgfqpoint{0.994742in}{0.371069in}}%
\pgfpathlineto{\pgfqpoint{0.997377in}{0.379920in}}%
\pgfpathlineto{\pgfqpoint{0.998523in}{0.392846in}}%
\pgfpathlineto{\pgfqpoint{0.998435in}{0.405773in}}%
\pgfpathlineto{\pgfqpoint{0.996689in}{0.418699in}}%
\pgfpathlineto{\pgfqpoint{0.994742in}{0.423190in}}%
\pgfpathlineto{\pgfqpoint{0.981473in}{0.430225in}}%
\pgfpathlineto{\pgfqpoint{0.968205in}{0.430468in}}%
\pgfpathlineto{\pgfqpoint{0.954936in}{0.426080in}}%
\pgfpathlineto{\pgfqpoint{0.950519in}{0.418699in}}%
\pgfpathlineto{\pgfqpoint{0.948620in}{0.405773in}}%
\pgfpathlineto{\pgfqpoint{0.948508in}{0.392846in}}%
\pgfpathlineto{\pgfqpoint{0.949718in}{0.379920in}}%
\pgfpathlineto{\pgfqpoint{0.954760in}{0.366993in}}%
\pgfpathclose%
\pgfpathmoveto{\pgfqpoint{0.957933in}{0.379920in}}%
\pgfpathlineto{\pgfqpoint{0.954936in}{0.389529in}}%
\pgfpathlineto{\pgfqpoint{0.954410in}{0.392846in}}%
\pgfpathlineto{\pgfqpoint{0.954936in}{0.403550in}}%
\pgfpathlineto{\pgfqpoint{0.955132in}{0.405773in}}%
\pgfpathlineto{\pgfqpoint{0.963586in}{0.418699in}}%
\pgfpathlineto{\pgfqpoint{0.968205in}{0.421343in}}%
\pgfpathlineto{\pgfqpoint{0.981473in}{0.420514in}}%
\pgfpathlineto{\pgfqpoint{0.983996in}{0.418699in}}%
\pgfpathlineto{\pgfqpoint{0.991014in}{0.405773in}}%
\pgfpathlineto{\pgfqpoint{0.991886in}{0.392846in}}%
\pgfpathlineto{\pgfqpoint{0.988685in}{0.379920in}}%
\pgfpathlineto{\pgfqpoint{0.981473in}{0.371650in}}%
\pgfpathlineto{\pgfqpoint{0.968205in}{0.370572in}}%
\pgfpathclose%
\pgfusepath{fill}%
\end{pgfscope}%
\begin{pgfscope}%
\pgfpathrectangle{\pgfqpoint{0.211875in}{0.211875in}}{\pgfqpoint{1.313625in}{1.279725in}}%
\pgfusepath{clip}%
\pgfsetbuttcap%
\pgfsetroundjoin%
\definecolor{currentfill}{rgb}{0.947270,0.405591,0.279023}%
\pgfsetfillcolor{currentfill}%
\pgfsetlinewidth{0.000000pt}%
\definecolor{currentstroke}{rgb}{0.000000,0.000000,0.000000}%
\pgfsetstrokecolor{currentstroke}%
\pgfsetdash{}{0pt}%
\pgfpathmoveto{\pgfqpoint{1.074356in}{0.364616in}}%
\pgfpathlineto{\pgfqpoint{1.087625in}{0.360940in}}%
\pgfpathlineto{\pgfqpoint{1.100894in}{0.361724in}}%
\pgfpathlineto{\pgfqpoint{1.110346in}{0.366993in}}%
\pgfpathlineto{\pgfqpoint{1.114163in}{0.373606in}}%
\pgfpathlineto{\pgfqpoint{1.115725in}{0.379920in}}%
\pgfpathlineto{\pgfqpoint{1.116805in}{0.392846in}}%
\pgfpathlineto{\pgfqpoint{1.116745in}{0.405773in}}%
\pgfpathlineto{\pgfqpoint{1.115178in}{0.418699in}}%
\pgfpathlineto{\pgfqpoint{1.114163in}{0.421625in}}%
\pgfpathlineto{\pgfqpoint{1.100894in}{0.430599in}}%
\pgfpathlineto{\pgfqpoint{1.087625in}{0.431193in}}%
\pgfpathlineto{\pgfqpoint{1.074356in}{0.428259in}}%
\pgfpathlineto{\pgfqpoint{1.067746in}{0.418699in}}%
\pgfpathlineto{\pgfqpoint{1.066058in}{0.405773in}}%
\pgfpathlineto{\pgfqpoint{1.065987in}{0.392846in}}%
\pgfpathlineto{\pgfqpoint{1.067137in}{0.379920in}}%
\pgfpathlineto{\pgfqpoint{1.071890in}{0.366993in}}%
\pgfpathclose%
\pgfpathmoveto{\pgfqpoint{1.074362in}{0.379920in}}%
\pgfpathlineto{\pgfqpoint{1.074356in}{0.379934in}}%
\pgfpathlineto{\pgfqpoint{1.072046in}{0.392846in}}%
\pgfpathlineto{\pgfqpoint{1.072666in}{0.405773in}}%
\pgfpathlineto{\pgfqpoint{1.074356in}{0.411372in}}%
\pgfpathlineto{\pgfqpoint{1.080432in}{0.418699in}}%
\pgfpathlineto{\pgfqpoint{1.087625in}{0.422245in}}%
\pgfpathlineto{\pgfqpoint{1.100894in}{0.420464in}}%
\pgfpathlineto{\pgfqpoint{1.103059in}{0.418699in}}%
\pgfpathlineto{\pgfqpoint{1.109294in}{0.405773in}}%
\pgfpathlineto{\pgfqpoint{1.110060in}{0.392846in}}%
\pgfpathlineto{\pgfqpoint{1.107186in}{0.379920in}}%
\pgfpathlineto{\pgfqpoint{1.100894in}{0.371792in}}%
\pgfpathlineto{\pgfqpoint{1.087625in}{0.369489in}}%
\pgfpathclose%
\pgfusepath{fill}%
\end{pgfscope}%
\begin{pgfscope}%
\pgfpathrectangle{\pgfqpoint{0.211875in}{0.211875in}}{\pgfqpoint{1.313625in}{1.279725in}}%
\pgfusepath{clip}%
\pgfsetbuttcap%
\pgfsetroundjoin%
\definecolor{currentfill}{rgb}{0.947270,0.405591,0.279023}%
\pgfsetfillcolor{currentfill}%
\pgfsetlinewidth{0.000000pt}%
\definecolor{currentstroke}{rgb}{0.000000,0.000000,0.000000}%
\pgfsetstrokecolor{currentstroke}%
\pgfsetdash{}{0pt}%
\pgfpathmoveto{\pgfqpoint{1.193777in}{0.363638in}}%
\pgfpathlineto{\pgfqpoint{1.207045in}{0.360849in}}%
\pgfpathlineto{\pgfqpoint{1.220314in}{0.362144in}}%
\pgfpathlineto{\pgfqpoint{1.228032in}{0.366993in}}%
\pgfpathlineto{\pgfqpoint{1.233583in}{0.379920in}}%
\pgfpathlineto{\pgfqpoint{1.233583in}{0.379921in}}%
\pgfpathlineto{\pgfqpoint{1.234679in}{0.392846in}}%
\pgfpathlineto{\pgfqpoint{1.234612in}{0.405773in}}%
\pgfpathlineto{\pgfqpoint{1.233583in}{0.415167in}}%
\pgfpathlineto{\pgfqpoint{1.232871in}{0.418699in}}%
\pgfpathlineto{\pgfqpoint{1.220314in}{0.430253in}}%
\pgfpathlineto{\pgfqpoint{1.207045in}{0.431266in}}%
\pgfpathlineto{\pgfqpoint{1.193777in}{0.429092in}}%
\pgfpathlineto{\pgfqpoint{1.185563in}{0.418699in}}%
\pgfpathlineto{\pgfqpoint{1.183862in}{0.405773in}}%
\pgfpathlineto{\pgfqpoint{1.183797in}{0.392846in}}%
\pgfpathlineto{\pgfqpoint{1.184969in}{0.379920in}}%
\pgfpathlineto{\pgfqpoint{1.189827in}{0.366993in}}%
\pgfpathclose%
\pgfpathmoveto{\pgfqpoint{1.192444in}{0.379920in}}%
\pgfpathlineto{\pgfqpoint{1.190063in}{0.392846in}}%
\pgfpathlineto{\pgfqpoint{1.190698in}{0.405773in}}%
\pgfpathlineto{\pgfqpoint{1.193777in}{0.414691in}}%
\pgfpathlineto{\pgfqpoint{1.198075in}{0.418699in}}%
\pgfpathlineto{\pgfqpoint{1.207045in}{0.422431in}}%
\pgfpathlineto{\pgfqpoint{1.220314in}{0.419579in}}%
\pgfpathlineto{\pgfqpoint{1.221270in}{0.418699in}}%
\pgfpathlineto{\pgfqpoint{1.227126in}{0.405773in}}%
\pgfpathlineto{\pgfqpoint{1.227850in}{0.392846in}}%
\pgfpathlineto{\pgfqpoint{1.225148in}{0.379920in}}%
\pgfpathlineto{\pgfqpoint{1.220314in}{0.372912in}}%
\pgfpathlineto{\pgfqpoint{1.207045in}{0.369248in}}%
\pgfpathlineto{\pgfqpoint{1.193777in}{0.377305in}}%
\pgfpathclose%
\pgfusepath{fill}%
\end{pgfscope}%
\begin{pgfscope}%
\pgfpathrectangle{\pgfqpoint{0.211875in}{0.211875in}}{\pgfqpoint{1.313625in}{1.279725in}}%
\pgfusepath{clip}%
\pgfsetbuttcap%
\pgfsetroundjoin%
\definecolor{currentfill}{rgb}{0.947270,0.405591,0.279023}%
\pgfsetfillcolor{currentfill}%
\pgfsetlinewidth{0.000000pt}%
\definecolor{currentstroke}{rgb}{0.000000,0.000000,0.000000}%
\pgfsetstrokecolor{currentstroke}%
\pgfsetdash{}{0pt}%
\pgfpathmoveto{\pgfqpoint{1.313197in}{0.363569in}}%
\pgfpathlineto{\pgfqpoint{1.326466in}{0.361329in}}%
\pgfpathlineto{\pgfqpoint{1.339735in}{0.363334in}}%
\pgfpathlineto{\pgfqpoint{1.344912in}{0.366993in}}%
\pgfpathlineto{\pgfqpoint{1.350621in}{0.379920in}}%
\pgfpathlineto{\pgfqpoint{1.351991in}{0.392846in}}%
\pgfpathlineto{\pgfqpoint{1.351865in}{0.405773in}}%
\pgfpathlineto{\pgfqpoint{1.349714in}{0.418699in}}%
\pgfpathlineto{\pgfqpoint{1.339735in}{0.429058in}}%
\pgfpathlineto{\pgfqpoint{1.326466in}{0.430715in}}%
\pgfpathlineto{\pgfqpoint{1.313197in}{0.428936in}}%
\pgfpathlineto{\pgfqpoint{1.304014in}{0.418699in}}%
\pgfpathlineto{\pgfqpoint{1.302049in}{0.405773in}}%
\pgfpathlineto{\pgfqpoint{1.301950in}{0.392846in}}%
\pgfpathlineto{\pgfqpoint{1.303239in}{0.379920in}}%
\pgfpathlineto{\pgfqpoint{1.308643in}{0.366993in}}%
\pgfpathclose%
\pgfpathmoveto{\pgfqpoint{1.311036in}{0.379920in}}%
\pgfpathlineto{\pgfqpoint{1.308482in}{0.392846in}}%
\pgfpathlineto{\pgfqpoint{1.309177in}{0.405773in}}%
\pgfpathlineto{\pgfqpoint{1.313197in}{0.416032in}}%
\pgfpathlineto{\pgfqpoint{1.316976in}{0.418699in}}%
\pgfpathlineto{\pgfqpoint{1.326466in}{0.421932in}}%
\pgfpathlineto{\pgfqpoint{1.337183in}{0.418699in}}%
\pgfpathlineto{\pgfqpoint{1.339735in}{0.417191in}}%
\pgfpathlineto{\pgfqpoint{1.344574in}{0.405773in}}%
\pgfpathlineto{\pgfqpoint{1.345304in}{0.392846in}}%
\pgfpathlineto{\pgfqpoint{1.342655in}{0.379920in}}%
\pgfpathlineto{\pgfqpoint{1.339735in}{0.375175in}}%
\pgfpathlineto{\pgfqpoint{1.326466in}{0.369817in}}%
\pgfpathlineto{\pgfqpoint{1.313197in}{0.376149in}}%
\pgfpathclose%
\pgfusepath{fill}%
\end{pgfscope}%
\begin{pgfscope}%
\pgfpathrectangle{\pgfqpoint{0.211875in}{0.211875in}}{\pgfqpoint{1.313625in}{1.279725in}}%
\pgfusepath{clip}%
\pgfsetbuttcap%
\pgfsetroundjoin%
\definecolor{currentfill}{rgb}{0.947270,0.405591,0.279023}%
\pgfsetfillcolor{currentfill}%
\pgfsetlinewidth{0.000000pt}%
\definecolor{currentstroke}{rgb}{0.000000,0.000000,0.000000}%
\pgfsetstrokecolor{currentstroke}%
\pgfsetdash{}{0pt}%
\pgfpathmoveto{\pgfqpoint{1.432617in}{0.364225in}}%
\pgfpathlineto{\pgfqpoint{1.445886in}{0.362379in}}%
\pgfpathlineto{\pgfqpoint{1.459155in}{0.365460in}}%
\pgfpathlineto{\pgfqpoint{1.461083in}{0.366993in}}%
\pgfpathlineto{\pgfqpoint{1.467344in}{0.379920in}}%
\pgfpathlineto{\pgfqpoint{1.468826in}{0.392846in}}%
\pgfpathlineto{\pgfqpoint{1.468623in}{0.405773in}}%
\pgfpathlineto{\pgfqpoint{1.466073in}{0.418699in}}%
\pgfpathlineto{\pgfqpoint{1.459155in}{0.426817in}}%
\pgfpathlineto{\pgfqpoint{1.445886in}{0.429539in}}%
\pgfpathlineto{\pgfqpoint{1.432617in}{0.428003in}}%
\pgfpathlineto{\pgfqpoint{1.423179in}{0.418699in}}%
\pgfpathlineto{\pgfqpoint{1.420655in}{0.405773in}}%
\pgfpathlineto{\pgfqpoint{1.420478in}{0.392846in}}%
\pgfpathlineto{\pgfqpoint{1.421990in}{0.379920in}}%
\pgfpathlineto{\pgfqpoint{1.428467in}{0.366993in}}%
\pgfpathclose%
\pgfpathmoveto{\pgfqpoint{1.430195in}{0.379920in}}%
\pgfpathlineto{\pgfqpoint{1.427346in}{0.392846in}}%
\pgfpathlineto{\pgfqpoint{1.428153in}{0.405773in}}%
\pgfpathlineto{\pgfqpoint{1.432617in}{0.415856in}}%
\pgfpathlineto{\pgfqpoint{1.438172in}{0.418699in}}%
\pgfpathlineto{\pgfqpoint{1.445886in}{0.420749in}}%
\pgfpathlineto{\pgfqpoint{1.451328in}{0.418699in}}%
\pgfpathlineto{\pgfqpoint{1.459155in}{0.412482in}}%
\pgfpathlineto{\pgfqpoint{1.461671in}{0.405773in}}%
\pgfpathlineto{\pgfqpoint{1.462449in}{0.392846in}}%
\pgfpathlineto{\pgfqpoint{1.459752in}{0.379920in}}%
\pgfpathlineto{\pgfqpoint{1.459155in}{0.378833in}}%
\pgfpathlineto{\pgfqpoint{1.445886in}{0.371193in}}%
\pgfpathlineto{\pgfqpoint{1.432617in}{0.376153in}}%
\pgfpathclose%
\pgfusepath{fill}%
\end{pgfscope}%
\begin{pgfscope}%
\pgfpathrectangle{\pgfqpoint{0.211875in}{0.211875in}}{\pgfqpoint{1.313625in}{1.279725in}}%
\pgfusepath{clip}%
\pgfsetbuttcap%
\pgfsetroundjoin%
\definecolor{currentfill}{rgb}{0.947270,0.405591,0.279023}%
\pgfsetfillcolor{currentfill}%
\pgfsetlinewidth{0.000000pt}%
\definecolor{currentstroke}{rgb}{0.000000,0.000000,0.000000}%
\pgfsetstrokecolor{currentstroke}%
\pgfsetdash{}{0pt}%
\pgfpathmoveto{\pgfqpoint{0.264951in}{0.377804in}}%
\pgfpathlineto{\pgfqpoint{0.270183in}{0.379920in}}%
\pgfpathlineto{\pgfqpoint{0.278220in}{0.388638in}}%
\pgfpathlineto{\pgfqpoint{0.279294in}{0.392846in}}%
\pgfpathlineto{\pgfqpoint{0.278220in}{0.404392in}}%
\pgfpathlineto{\pgfqpoint{0.277799in}{0.405773in}}%
\pgfpathlineto{\pgfqpoint{0.264951in}{0.413211in}}%
\pgfpathlineto{\pgfqpoint{0.256488in}{0.405773in}}%
\pgfpathlineto{\pgfqpoint{0.253960in}{0.392846in}}%
\pgfpathlineto{\pgfqpoint{0.261555in}{0.379920in}}%
\pgfpathclose%
\pgfusepath{fill}%
\end{pgfscope}%
\begin{pgfscope}%
\pgfpathrectangle{\pgfqpoint{0.211875in}{0.211875in}}{\pgfqpoint{1.313625in}{1.279725in}}%
\pgfusepath{clip}%
\pgfsetbuttcap%
\pgfsetroundjoin%
\definecolor{currentfill}{rgb}{0.947270,0.405591,0.279023}%
\pgfsetfillcolor{currentfill}%
\pgfsetlinewidth{0.000000pt}%
\definecolor{currentstroke}{rgb}{0.000000,0.000000,0.000000}%
\pgfsetstrokecolor{currentstroke}%
\pgfsetdash{}{0pt}%
\pgfpathmoveto{\pgfqpoint{0.371102in}{0.379681in}}%
\pgfpathlineto{\pgfqpoint{0.384371in}{0.373025in}}%
\pgfpathlineto{\pgfqpoint{0.397574in}{0.379920in}}%
\pgfpathlineto{\pgfqpoint{0.397640in}{0.380018in}}%
\pgfpathlineto{\pgfqpoint{0.400569in}{0.392846in}}%
\pgfpathlineto{\pgfqpoint{0.399697in}{0.405773in}}%
\pgfpathlineto{\pgfqpoint{0.397640in}{0.410931in}}%
\pgfpathlineto{\pgfqpoint{0.385857in}{0.418699in}}%
\pgfpathlineto{\pgfqpoint{0.384371in}{0.419171in}}%
\pgfpathlineto{\pgfqpoint{0.382917in}{0.418699in}}%
\pgfpathlineto{\pgfqpoint{0.371102in}{0.411199in}}%
\pgfpathlineto{\pgfqpoint{0.368843in}{0.405773in}}%
\pgfpathlineto{\pgfqpoint{0.367914in}{0.392846in}}%
\pgfpathlineto{\pgfqpoint{0.370958in}{0.379920in}}%
\pgfpathclose%
\pgfpathmoveto{\pgfqpoint{0.379925in}{0.392846in}}%
\pgfpathlineto{\pgfqpoint{0.383821in}{0.405773in}}%
\pgfpathlineto{\pgfqpoint{0.384371in}{0.406164in}}%
\pgfpathlineto{\pgfqpoint{0.384897in}{0.405773in}}%
\pgfpathlineto{\pgfqpoint{0.388575in}{0.392846in}}%
\pgfpathlineto{\pgfqpoint{0.384371in}{0.387104in}}%
\pgfpathclose%
\pgfusepath{fill}%
\end{pgfscope}%
\begin{pgfscope}%
\pgfpathrectangle{\pgfqpoint{0.211875in}{0.211875in}}{\pgfqpoint{1.313625in}{1.279725in}}%
\pgfusepath{clip}%
\pgfsetbuttcap%
\pgfsetroundjoin%
\definecolor{currentfill}{rgb}{0.947270,0.405591,0.279023}%
\pgfsetfillcolor{currentfill}%
\pgfsetlinewidth{0.000000pt}%
\definecolor{currentstroke}{rgb}{0.000000,0.000000,0.000000}%
\pgfsetstrokecolor{currentstroke}%
\pgfsetdash{}{0pt}%
\pgfpathmoveto{\pgfqpoint{0.490523in}{0.373106in}}%
\pgfpathlineto{\pgfqpoint{0.503792in}{0.369057in}}%
\pgfpathlineto{\pgfqpoint{0.517061in}{0.376531in}}%
\pgfpathlineto{\pgfqpoint{0.518820in}{0.379920in}}%
\pgfpathlineto{\pgfqpoint{0.521166in}{0.392846in}}%
\pgfpathlineto{\pgfqpoint{0.520559in}{0.405773in}}%
\pgfpathlineto{\pgfqpoint{0.517061in}{0.415741in}}%
\pgfpathlineto{\pgfqpoint{0.513709in}{0.418699in}}%
\pgfpathlineto{\pgfqpoint{0.503792in}{0.422596in}}%
\pgfpathlineto{\pgfqpoint{0.490523in}{0.419407in}}%
\pgfpathlineto{\pgfqpoint{0.489776in}{0.418699in}}%
\pgfpathlineto{\pgfqpoint{0.483960in}{0.405773in}}%
\pgfpathlineto{\pgfqpoint{0.483227in}{0.392846in}}%
\pgfpathlineto{\pgfqpoint{0.485920in}{0.379920in}}%
\pgfpathclose%
\pgfpathmoveto{\pgfqpoint{0.502248in}{0.379920in}}%
\pgfpathlineto{\pgfqpoint{0.490523in}{0.391530in}}%
\pgfpathlineto{\pgfqpoint{0.490174in}{0.392846in}}%
\pgfpathlineto{\pgfqpoint{0.490523in}{0.396396in}}%
\pgfpathlineto{\pgfqpoint{0.493757in}{0.405773in}}%
\pgfpathlineto{\pgfqpoint{0.503792in}{0.411248in}}%
\pgfpathlineto{\pgfqpoint{0.509792in}{0.405773in}}%
\pgfpathlineto{\pgfqpoint{0.512315in}{0.392846in}}%
\pgfpathlineto{\pgfqpoint{0.504725in}{0.379920in}}%
\pgfpathlineto{\pgfqpoint{0.503792in}{0.379322in}}%
\pgfpathclose%
\pgfusepath{fill}%
\end{pgfscope}%
\begin{pgfscope}%
\pgfpathrectangle{\pgfqpoint{0.211875in}{0.211875in}}{\pgfqpoint{1.313625in}{1.279725in}}%
\pgfusepath{clip}%
\pgfsetbuttcap%
\pgfsetroundjoin%
\definecolor{currentfill}{rgb}{0.947270,0.405591,0.279023}%
\pgfsetfillcolor{currentfill}%
\pgfsetlinewidth{0.000000pt}%
\definecolor{currentstroke}{rgb}{0.000000,0.000000,0.000000}%
\pgfsetstrokecolor{currentstroke}%
\pgfsetdash{}{0pt}%
\pgfpathmoveto{\pgfqpoint{0.676288in}{0.443739in}}%
\pgfpathlineto{\pgfqpoint{0.685431in}{0.444552in}}%
\pgfpathlineto{\pgfqpoint{0.689557in}{0.445074in}}%
\pgfpathlineto{\pgfqpoint{0.700511in}{0.457479in}}%
\pgfpathlineto{\pgfqpoint{0.702826in}{0.469946in}}%
\pgfpathlineto{\pgfqpoint{0.702878in}{0.470405in}}%
\pgfpathlineto{\pgfqpoint{0.703060in}{0.483332in}}%
\pgfpathlineto{\pgfqpoint{0.702826in}{0.486020in}}%
\pgfpathlineto{\pgfqpoint{0.701329in}{0.496258in}}%
\pgfpathlineto{\pgfqpoint{0.690079in}{0.509185in}}%
\pgfpathlineto{\pgfqpoint{0.689557in}{0.509380in}}%
\pgfpathlineto{\pgfqpoint{0.676288in}{0.510365in}}%
\pgfpathlineto{\pgfqpoint{0.670517in}{0.509185in}}%
\pgfpathlineto{\pgfqpoint{0.663019in}{0.506511in}}%
\pgfpathlineto{\pgfqpoint{0.656815in}{0.496258in}}%
\pgfpathlineto{\pgfqpoint{0.655081in}{0.483332in}}%
\pgfpathlineto{\pgfqpoint{0.655262in}{0.470405in}}%
\pgfpathlineto{\pgfqpoint{0.657461in}{0.457479in}}%
\pgfpathlineto{\pgfqpoint{0.663019in}{0.448342in}}%
\pgfpathlineto{\pgfqpoint{0.672693in}{0.444552in}}%
\pgfpathclose%
\pgfpathmoveto{\pgfqpoint{0.668149in}{0.457479in}}%
\pgfpathlineto{\pgfqpoint{0.663019in}{0.464900in}}%
\pgfpathlineto{\pgfqpoint{0.661543in}{0.470405in}}%
\pgfpathlineto{\pgfqpoint{0.661448in}{0.483332in}}%
\pgfpathlineto{\pgfqpoint{0.663019in}{0.489459in}}%
\pgfpathlineto{\pgfqpoint{0.667674in}{0.496258in}}%
\pgfpathlineto{\pgfqpoint{0.676288in}{0.501071in}}%
\pgfpathlineto{\pgfqpoint{0.689557in}{0.498021in}}%
\pgfpathlineto{\pgfqpoint{0.691134in}{0.496258in}}%
\pgfpathlineto{\pgfqpoint{0.695600in}{0.483332in}}%
\pgfpathlineto{\pgfqpoint{0.695479in}{0.470405in}}%
\pgfpathlineto{\pgfqpoint{0.690825in}{0.457479in}}%
\pgfpathlineto{\pgfqpoint{0.689557in}{0.456043in}}%
\pgfpathlineto{\pgfqpoint{0.676288in}{0.452830in}}%
\pgfpathclose%
\pgfusepath{fill}%
\end{pgfscope}%
\begin{pgfscope}%
\pgfpathrectangle{\pgfqpoint{0.211875in}{0.211875in}}{\pgfqpoint{1.313625in}{1.279725in}}%
\pgfusepath{clip}%
\pgfsetbuttcap%
\pgfsetroundjoin%
\definecolor{currentfill}{rgb}{0.947270,0.405591,0.279023}%
\pgfsetfillcolor{currentfill}%
\pgfsetlinewidth{0.000000pt}%
\definecolor{currentstroke}{rgb}{0.000000,0.000000,0.000000}%
\pgfsetstrokecolor{currentstroke}%
\pgfsetdash{}{0pt}%
\pgfpathmoveto{\pgfqpoint{0.782439in}{0.444316in}}%
\pgfpathlineto{\pgfqpoint{0.795708in}{0.441892in}}%
\pgfpathlineto{\pgfqpoint{0.808977in}{0.443336in}}%
\pgfpathlineto{\pgfqpoint{0.811539in}{0.444552in}}%
\pgfpathlineto{\pgfqpoint{0.820282in}{0.457479in}}%
\pgfpathlineto{\pgfqpoint{0.822062in}{0.470405in}}%
\pgfpathlineto{\pgfqpoint{0.822246in}{0.480645in}}%
\pgfpathlineto{\pgfqpoint{0.822288in}{0.483332in}}%
\pgfpathlineto{\pgfqpoint{0.822246in}{0.484072in}}%
\pgfpathlineto{\pgfqpoint{0.821156in}{0.496258in}}%
\pgfpathlineto{\pgfqpoint{0.813835in}{0.509185in}}%
\pgfpathlineto{\pgfqpoint{0.808977in}{0.511257in}}%
\pgfpathlineto{\pgfqpoint{0.795708in}{0.512289in}}%
\pgfpathlineto{\pgfqpoint{0.782439in}{0.510369in}}%
\pgfpathlineto{\pgfqpoint{0.780142in}{0.509185in}}%
\pgfpathlineto{\pgfqpoint{0.772618in}{0.496258in}}%
\pgfpathlineto{\pgfqpoint{0.771366in}{0.483332in}}%
\pgfpathlineto{\pgfqpoint{0.771576in}{0.470405in}}%
\pgfpathlineto{\pgfqpoint{0.773395in}{0.457479in}}%
\pgfpathlineto{\pgfqpoint{0.782021in}{0.444552in}}%
\pgfpathclose%
\pgfpathmoveto{\pgfqpoint{0.781870in}{0.457479in}}%
\pgfpathlineto{\pgfqpoint{0.778114in}{0.470405in}}%
\pgfpathlineto{\pgfqpoint{0.777993in}{0.483332in}}%
\pgfpathlineto{\pgfqpoint{0.781521in}{0.496258in}}%
\pgfpathlineto{\pgfqpoint{0.782439in}{0.497597in}}%
\pgfpathlineto{\pgfqpoint{0.795708in}{0.503391in}}%
\pgfpathlineto{\pgfqpoint{0.808977in}{0.499515in}}%
\pgfpathlineto{\pgfqpoint{0.811559in}{0.496258in}}%
\pgfpathlineto{\pgfqpoint{0.815208in}{0.483332in}}%
\pgfpathlineto{\pgfqpoint{0.815073in}{0.470405in}}%
\pgfpathlineto{\pgfqpoint{0.811161in}{0.457479in}}%
\pgfpathlineto{\pgfqpoint{0.808977in}{0.454700in}}%
\pgfpathlineto{\pgfqpoint{0.795708in}{0.450560in}}%
\pgfpathlineto{\pgfqpoint{0.782439in}{0.456647in}}%
\pgfpathclose%
\pgfusepath{fill}%
\end{pgfscope}%
\begin{pgfscope}%
\pgfpathrectangle{\pgfqpoint{0.211875in}{0.211875in}}{\pgfqpoint{1.313625in}{1.279725in}}%
\pgfusepath{clip}%
\pgfsetbuttcap%
\pgfsetroundjoin%
\definecolor{currentfill}{rgb}{0.947270,0.405591,0.279023}%
\pgfsetfillcolor{currentfill}%
\pgfsetlinewidth{0.000000pt}%
\definecolor{currentstroke}{rgb}{0.000000,0.000000,0.000000}%
\pgfsetstrokecolor{currentstroke}%
\pgfsetdash{}{0pt}%
\pgfpathmoveto{\pgfqpoint{0.901860in}{0.442071in}}%
\pgfpathlineto{\pgfqpoint{0.915129in}{0.440614in}}%
\pgfpathlineto{\pgfqpoint{0.928398in}{0.442220in}}%
\pgfpathlineto{\pgfqpoint{0.932739in}{0.444552in}}%
\pgfpathlineto{\pgfqpoint{0.939392in}{0.457479in}}%
\pgfpathlineto{\pgfqpoint{0.940782in}{0.470405in}}%
\pgfpathlineto{\pgfqpoint{0.941017in}{0.483332in}}%
\pgfpathlineto{\pgfqpoint{0.940291in}{0.496258in}}%
\pgfpathlineto{\pgfqpoint{0.935445in}{0.509185in}}%
\pgfpathlineto{\pgfqpoint{0.928398in}{0.512611in}}%
\pgfpathlineto{\pgfqpoint{0.915129in}{0.513629in}}%
\pgfpathlineto{\pgfqpoint{0.901860in}{0.512598in}}%
\pgfpathlineto{\pgfqpoint{0.894345in}{0.509185in}}%
\pgfpathlineto{\pgfqpoint{0.888749in}{0.496258in}}%
\pgfpathlineto{\pgfqpoint{0.888591in}{0.494470in}}%
\pgfpathlineto{\pgfqpoint{0.887960in}{0.483332in}}%
\pgfpathlineto{\pgfqpoint{0.888165in}{0.470405in}}%
\pgfpathlineto{\pgfqpoint{0.888591in}{0.465520in}}%
\pgfpathlineto{\pgfqpoint{0.889645in}{0.457479in}}%
\pgfpathlineto{\pgfqpoint{0.896904in}{0.444552in}}%
\pgfpathclose%
\pgfpathmoveto{\pgfqpoint{0.898550in}{0.457479in}}%
\pgfpathlineto{\pgfqpoint{0.894960in}{0.470405in}}%
\pgfpathlineto{\pgfqpoint{0.894817in}{0.483332in}}%
\pgfpathlineto{\pgfqpoint{0.898105in}{0.496258in}}%
\pgfpathlineto{\pgfqpoint{0.901860in}{0.501108in}}%
\pgfpathlineto{\pgfqpoint{0.915129in}{0.504962in}}%
\pgfpathlineto{\pgfqpoint{0.928398in}{0.500209in}}%
\pgfpathlineto{\pgfqpoint{0.931169in}{0.496258in}}%
\pgfpathlineto{\pgfqpoint{0.934271in}{0.483332in}}%
\pgfpathlineto{\pgfqpoint{0.934128in}{0.470405in}}%
\pgfpathlineto{\pgfqpoint{0.930720in}{0.457479in}}%
\pgfpathlineto{\pgfqpoint{0.928398in}{0.454159in}}%
\pgfpathlineto{\pgfqpoint{0.915129in}{0.449029in}}%
\pgfpathlineto{\pgfqpoint{0.901860in}{0.453171in}}%
\pgfpathclose%
\pgfusepath{fill}%
\end{pgfscope}%
\begin{pgfscope}%
\pgfpathrectangle{\pgfqpoint{0.211875in}{0.211875in}}{\pgfqpoint{1.313625in}{1.279725in}}%
\pgfusepath{clip}%
\pgfsetbuttcap%
\pgfsetroundjoin%
\definecolor{currentfill}{rgb}{0.947270,0.405591,0.279023}%
\pgfsetfillcolor{currentfill}%
\pgfsetlinewidth{0.000000pt}%
\definecolor{currentstroke}{rgb}{0.000000,0.000000,0.000000}%
\pgfsetstrokecolor{currentstroke}%
\pgfsetdash{}{0pt}%
\pgfpathmoveto{\pgfqpoint{1.021280in}{0.440742in}}%
\pgfpathlineto{\pgfqpoint{1.034549in}{0.439881in}}%
\pgfpathlineto{\pgfqpoint{1.047818in}{0.441730in}}%
\pgfpathlineto{\pgfqpoint{1.052448in}{0.444552in}}%
\pgfpathlineto{\pgfqpoint{1.057938in}{0.457479in}}%
\pgfpathlineto{\pgfqpoint{1.059109in}{0.470405in}}%
\pgfpathlineto{\pgfqpoint{1.059343in}{0.483332in}}%
\pgfpathlineto{\pgfqpoint{1.058836in}{0.496258in}}%
\pgfpathlineto{\pgfqpoint{1.055313in}{0.509185in}}%
\pgfpathlineto{\pgfqpoint{1.047818in}{0.513350in}}%
\pgfpathlineto{\pgfqpoint{1.034549in}{0.514411in}}%
\pgfpathlineto{\pgfqpoint{1.021280in}{0.513874in}}%
\pgfpathlineto{\pgfqpoint{1.009570in}{0.509185in}}%
\pgfpathlineto{\pgfqpoint{1.008011in}{0.506789in}}%
\pgfpathlineto{\pgfqpoint{1.005736in}{0.496258in}}%
\pgfpathlineto{\pgfqpoint{1.005208in}{0.483332in}}%
\pgfpathlineto{\pgfqpoint{1.005420in}{0.470405in}}%
\pgfpathlineto{\pgfqpoint{1.006550in}{0.457479in}}%
\pgfpathlineto{\pgfqpoint{1.008011in}{0.451570in}}%
\pgfpathlineto{\pgfqpoint{1.012692in}{0.444552in}}%
\pgfpathclose%
\pgfpathmoveto{\pgfqpoint{1.015651in}{0.457479in}}%
\pgfpathlineto{\pgfqpoint{1.012082in}{0.470405in}}%
\pgfpathlineto{\pgfqpoint{1.011920in}{0.483332in}}%
\pgfpathlineto{\pgfqpoint{1.015128in}{0.496258in}}%
\pgfpathlineto{\pgfqpoint{1.021280in}{0.503311in}}%
\pgfpathlineto{\pgfqpoint{1.034549in}{0.505815in}}%
\pgfpathlineto{\pgfqpoint{1.047818in}{0.499934in}}%
\pgfpathlineto{\pgfqpoint{1.050090in}{0.496258in}}%
\pgfpathlineto{\pgfqpoint{1.052866in}{0.483332in}}%
\pgfpathlineto{\pgfqpoint{1.052721in}{0.470405in}}%
\pgfpathlineto{\pgfqpoint{1.049622in}{0.457479in}}%
\pgfpathlineto{\pgfqpoint{1.047818in}{0.454574in}}%
\pgfpathlineto{\pgfqpoint{1.034549in}{0.448209in}}%
\pgfpathlineto{\pgfqpoint{1.021280in}{0.450947in}}%
\pgfpathclose%
\pgfusepath{fill}%
\end{pgfscope}%
\begin{pgfscope}%
\pgfpathrectangle{\pgfqpoint{0.211875in}{0.211875in}}{\pgfqpoint{1.313625in}{1.279725in}}%
\pgfusepath{clip}%
\pgfsetbuttcap%
\pgfsetroundjoin%
\definecolor{currentfill}{rgb}{0.947270,0.405591,0.279023}%
\pgfsetfillcolor{currentfill}%
\pgfsetlinewidth{0.000000pt}%
\definecolor{currentstroke}{rgb}{0.000000,0.000000,0.000000}%
\pgfsetstrokecolor{currentstroke}%
\pgfsetdash{}{0pt}%
\pgfpathmoveto{\pgfqpoint{1.140701in}{0.440163in}}%
\pgfpathlineto{\pgfqpoint{1.153970in}{0.439688in}}%
\pgfpathlineto{\pgfqpoint{1.167239in}{0.442020in}}%
\pgfpathlineto{\pgfqpoint{1.170883in}{0.444552in}}%
\pgfpathlineto{\pgfqpoint{1.175981in}{0.457479in}}%
\pgfpathlineto{\pgfqpoint{1.177080in}{0.470405in}}%
\pgfpathlineto{\pgfqpoint{1.177307in}{0.483332in}}%
\pgfpathlineto{\pgfqpoint{1.176856in}{0.496258in}}%
\pgfpathlineto{\pgfqpoint{1.173698in}{0.509185in}}%
\pgfpathlineto{\pgfqpoint{1.167239in}{0.513316in}}%
\pgfpathlineto{\pgfqpoint{1.153970in}{0.514638in}}%
\pgfpathlineto{\pgfqpoint{1.140701in}{0.514376in}}%
\pgfpathlineto{\pgfqpoint{1.127432in}{0.510528in}}%
\pgfpathlineto{\pgfqpoint{1.126324in}{0.509185in}}%
\pgfpathlineto{\pgfqpoint{1.123315in}{0.496258in}}%
\pgfpathlineto{\pgfqpoint{1.122886in}{0.483332in}}%
\pgfpathlineto{\pgfqpoint{1.123102in}{0.470405in}}%
\pgfpathlineto{\pgfqpoint{1.124149in}{0.457479in}}%
\pgfpathlineto{\pgfqpoint{1.127432in}{0.446868in}}%
\pgfpathlineto{\pgfqpoint{1.129513in}{0.444552in}}%
\pgfpathclose%
\pgfpathmoveto{\pgfqpoint{1.133216in}{0.457479in}}%
\pgfpathlineto{\pgfqpoint{1.129494in}{0.470405in}}%
\pgfpathlineto{\pgfqpoint{1.129315in}{0.483332in}}%
\pgfpathlineto{\pgfqpoint{1.132633in}{0.496258in}}%
\pgfpathlineto{\pgfqpoint{1.140701in}{0.504460in}}%
\pgfpathlineto{\pgfqpoint{1.153970in}{0.505953in}}%
\pgfpathlineto{\pgfqpoint{1.167239in}{0.498414in}}%
\pgfpathlineto{\pgfqpoint{1.168404in}{0.496258in}}%
\pgfpathlineto{\pgfqpoint{1.171040in}{0.483332in}}%
\pgfpathlineto{\pgfqpoint{1.170898in}{0.470405in}}%
\pgfpathlineto{\pgfqpoint{1.167942in}{0.457479in}}%
\pgfpathlineto{\pgfqpoint{1.167239in}{0.456196in}}%
\pgfpathlineto{\pgfqpoint{1.153970in}{0.448098in}}%
\pgfpathlineto{\pgfqpoint{1.140701in}{0.449749in}}%
\pgfpathclose%
\pgfusepath{fill}%
\end{pgfscope}%
\begin{pgfscope}%
\pgfpathrectangle{\pgfqpoint{0.211875in}{0.211875in}}{\pgfqpoint{1.313625in}{1.279725in}}%
\pgfusepath{clip}%
\pgfsetbuttcap%
\pgfsetroundjoin%
\definecolor{currentfill}{rgb}{0.947270,0.405591,0.279023}%
\pgfsetfillcolor{currentfill}%
\pgfsetlinewidth{0.000000pt}%
\definecolor{currentstroke}{rgb}{0.000000,0.000000,0.000000}%
\pgfsetstrokecolor{currentstroke}%
\pgfsetdash{}{0pt}%
\pgfpathmoveto{\pgfqpoint{1.260121in}{0.440226in}}%
\pgfpathlineto{\pgfqpoint{1.273390in}{0.440054in}}%
\pgfpathlineto{\pgfqpoint{1.286659in}{0.443341in}}%
\pgfpathlineto{\pgfqpoint{1.288175in}{0.444552in}}%
\pgfpathlineto{\pgfqpoint{1.293554in}{0.457479in}}%
\pgfpathlineto{\pgfqpoint{1.294710in}{0.470405in}}%
\pgfpathlineto{\pgfqpoint{1.294928in}{0.483332in}}%
\pgfpathlineto{\pgfqpoint{1.294387in}{0.496258in}}%
\pgfpathlineto{\pgfqpoint{1.290766in}{0.509185in}}%
\pgfpathlineto{\pgfqpoint{1.286659in}{0.512238in}}%
\pgfpathlineto{\pgfqpoint{1.273390in}{0.514294in}}%
\pgfpathlineto{\pgfqpoint{1.260121in}{0.514222in}}%
\pgfpathlineto{\pgfqpoint{1.246852in}{0.511271in}}%
\pgfpathlineto{\pgfqpoint{1.244710in}{0.509185in}}%
\pgfpathlineto{\pgfqpoint{1.241441in}{0.496258in}}%
\pgfpathlineto{\pgfqpoint{1.240970in}{0.483332in}}%
\pgfpathlineto{\pgfqpoint{1.241187in}{0.470405in}}%
\pgfpathlineto{\pgfqpoint{1.242275in}{0.457479in}}%
\pgfpathlineto{\pgfqpoint{1.246852in}{0.445179in}}%
\pgfpathlineto{\pgfqpoint{1.247587in}{0.444552in}}%
\pgfpathclose%
\pgfpathmoveto{\pgfqpoint{1.251323in}{0.457479in}}%
\pgfpathlineto{\pgfqpoint{1.247231in}{0.470405in}}%
\pgfpathlineto{\pgfqpoint{1.247040in}{0.483332in}}%
\pgfpathlineto{\pgfqpoint{1.250704in}{0.496258in}}%
\pgfpathlineto{\pgfqpoint{1.260121in}{0.504719in}}%
\pgfpathlineto{\pgfqpoint{1.273390in}{0.505350in}}%
\pgfpathlineto{\pgfqpoint{1.285840in}{0.496258in}}%
\pgfpathlineto{\pgfqpoint{1.286659in}{0.494359in}}%
\pgfpathlineto{\pgfqpoint{1.288818in}{0.483332in}}%
\pgfpathlineto{\pgfqpoint{1.288684in}{0.470405in}}%
\pgfpathlineto{\pgfqpoint{1.286659in}{0.460868in}}%
\pgfpathlineto{\pgfqpoint{1.285131in}{0.457479in}}%
\pgfpathlineto{\pgfqpoint{1.273390in}{0.448719in}}%
\pgfpathlineto{\pgfqpoint{1.260121in}{0.449424in}}%
\pgfpathclose%
\pgfusepath{fill}%
\end{pgfscope}%
\begin{pgfscope}%
\pgfpathrectangle{\pgfqpoint{0.211875in}{0.211875in}}{\pgfqpoint{1.313625in}{1.279725in}}%
\pgfusepath{clip}%
\pgfsetbuttcap%
\pgfsetroundjoin%
\definecolor{currentfill}{rgb}{0.947270,0.405591,0.279023}%
\pgfsetfillcolor{currentfill}%
\pgfsetlinewidth{0.000000pt}%
\definecolor{currentstroke}{rgb}{0.000000,0.000000,0.000000}%
\pgfsetstrokecolor{currentstroke}%
\pgfsetdash{}{0pt}%
\pgfpathmoveto{\pgfqpoint{1.379542in}{0.440862in}}%
\pgfpathlineto{\pgfqpoint{1.392811in}{0.441019in}}%
\pgfpathlineto{\pgfqpoint{1.403571in}{0.444552in}}%
\pgfpathlineto{\pgfqpoint{1.406080in}{0.446415in}}%
\pgfpathlineto{\pgfqpoint{1.410666in}{0.457479in}}%
\pgfpathlineto{\pgfqpoint{1.412001in}{0.470405in}}%
\pgfpathlineto{\pgfqpoint{1.412205in}{0.483332in}}%
\pgfpathlineto{\pgfqpoint{1.411439in}{0.496258in}}%
\pgfpathlineto{\pgfqpoint{1.406607in}{0.509185in}}%
\pgfpathlineto{\pgfqpoint{1.406080in}{0.509648in}}%
\pgfpathlineto{\pgfqpoint{1.392811in}{0.513334in}}%
\pgfpathlineto{\pgfqpoint{1.379542in}{0.513485in}}%
\pgfpathlineto{\pgfqpoint{1.366273in}{0.510633in}}%
\pgfpathlineto{\pgfqpoint{1.364482in}{0.509185in}}%
\pgfpathlineto{\pgfqpoint{1.360111in}{0.496258in}}%
\pgfpathlineto{\pgfqpoint{1.359454in}{0.483332in}}%
\pgfpathlineto{\pgfqpoint{1.359667in}{0.470405in}}%
\pgfpathlineto{\pgfqpoint{1.360923in}{0.457479in}}%
\pgfpathlineto{\pgfqpoint{1.366273in}{0.445226in}}%
\pgfpathlineto{\pgfqpoint{1.367287in}{0.444552in}}%
\pgfpathclose%
\pgfpathmoveto{\pgfqpoint{1.370106in}{0.457479in}}%
\pgfpathlineto{\pgfqpoint{1.366273in}{0.467338in}}%
\pgfpathlineto{\pgfqpoint{1.365680in}{0.470405in}}%
\pgfpathlineto{\pgfqpoint{1.365550in}{0.483332in}}%
\pgfpathlineto{\pgfqpoint{1.366273in}{0.487463in}}%
\pgfpathlineto{\pgfqpoint{1.369480in}{0.496258in}}%
\pgfpathlineto{\pgfqpoint{1.379542in}{0.504197in}}%
\pgfpathlineto{\pgfqpoint{1.392811in}{0.503948in}}%
\pgfpathlineto{\pgfqpoint{1.402018in}{0.496258in}}%
\pgfpathlineto{\pgfqpoint{1.406080in}{0.484110in}}%
\pgfpathlineto{\pgfqpoint{1.406208in}{0.483332in}}%
\pgfpathlineto{\pgfqpoint{1.406085in}{0.470405in}}%
\pgfpathlineto{\pgfqpoint{1.406080in}{0.470375in}}%
\pgfpathlineto{\pgfqpoint{1.401444in}{0.457479in}}%
\pgfpathlineto{\pgfqpoint{1.392811in}{0.450127in}}%
\pgfpathlineto{\pgfqpoint{1.379542in}{0.449874in}}%
\pgfpathclose%
\pgfusepath{fill}%
\end{pgfscope}%
\begin{pgfscope}%
\pgfpathrectangle{\pgfqpoint{0.211875in}{0.211875in}}{\pgfqpoint{1.313625in}{1.279725in}}%
\pgfusepath{clip}%
\pgfsetbuttcap%
\pgfsetroundjoin%
\definecolor{currentfill}{rgb}{0.947270,0.405591,0.279023}%
\pgfsetfillcolor{currentfill}%
\pgfsetlinewidth{0.000000pt}%
\definecolor{currentstroke}{rgb}{0.000000,0.000000,0.000000}%
\pgfsetstrokecolor{currentstroke}%
\pgfsetdash{}{0pt}%
\pgfpathmoveto{\pgfqpoint{1.498962in}{0.442031in}}%
\pgfpathlineto{\pgfqpoint{1.512231in}{0.442649in}}%
\pgfpathlineto{\pgfqpoint{1.517325in}{0.444552in}}%
\pgfpathlineto{\pgfqpoint{1.525500in}{0.452328in}}%
\pgfpathlineto{\pgfqpoint{1.525500in}{0.457479in}}%
\pgfpathlineto{\pgfqpoint{1.525500in}{0.470405in}}%
\pgfpathlineto{\pgfqpoint{1.525500in}{0.483332in}}%
\pgfpathlineto{\pgfqpoint{1.525500in}{0.496258in}}%
\pgfpathlineto{\pgfqpoint{1.525500in}{0.503852in}}%
\pgfpathlineto{\pgfqpoint{1.519652in}{0.509185in}}%
\pgfpathlineto{\pgfqpoint{1.512231in}{0.511687in}}%
\pgfpathlineto{\pgfqpoint{1.498962in}{0.512208in}}%
\pgfpathlineto{\pgfqpoint{1.485795in}{0.509185in}}%
\pgfpathlineto{\pgfqpoint{1.485693in}{0.509137in}}%
\pgfpathlineto{\pgfqpoint{1.479344in}{0.496258in}}%
\pgfpathlineto{\pgfqpoint{1.478344in}{0.483332in}}%
\pgfpathlineto{\pgfqpoint{1.478548in}{0.470405in}}%
\pgfpathlineto{\pgfqpoint{1.480111in}{0.457479in}}%
\pgfpathlineto{\pgfqpoint{1.485693in}{0.446396in}}%
\pgfpathlineto{\pgfqpoint{1.489242in}{0.444552in}}%
\pgfpathclose%
\pgfpathmoveto{\pgfqpoint{1.489791in}{0.457479in}}%
\pgfpathlineto{\pgfqpoint{1.485693in}{0.465857in}}%
\pgfpathlineto{\pgfqpoint{1.484685in}{0.470405in}}%
\pgfpathlineto{\pgfqpoint{1.484566in}{0.483332in}}%
\pgfpathlineto{\pgfqpoint{1.485693in}{0.488812in}}%
\pgfpathlineto{\pgfqpoint{1.489200in}{0.496258in}}%
\pgfpathlineto{\pgfqpoint{1.498962in}{0.502955in}}%
\pgfpathlineto{\pgfqpoint{1.512231in}{0.501651in}}%
\pgfpathlineto{\pgfqpoint{1.517925in}{0.496258in}}%
\pgfpathlineto{\pgfqpoint{1.522305in}{0.483332in}}%
\pgfpathlineto{\pgfqpoint{1.522155in}{0.470405in}}%
\pgfpathlineto{\pgfqpoint{1.517492in}{0.457479in}}%
\pgfpathlineto{\pgfqpoint{1.512231in}{0.452411in}}%
\pgfpathlineto{\pgfqpoint{1.498962in}{0.451038in}}%
\pgfpathclose%
\pgfusepath{fill}%
\end{pgfscope}%
\begin{pgfscope}%
\pgfpathrectangle{\pgfqpoint{0.211875in}{0.211875in}}{\pgfqpoint{1.313625in}{1.279725in}}%
\pgfusepath{clip}%
\pgfsetbuttcap%
\pgfsetroundjoin%
\definecolor{currentfill}{rgb}{0.947270,0.405591,0.279023}%
\pgfsetfillcolor{currentfill}%
\pgfsetlinewidth{0.000000pt}%
\definecolor{currentstroke}{rgb}{0.000000,0.000000,0.000000}%
\pgfsetstrokecolor{currentstroke}%
\pgfsetdash{}{0pt}%
\pgfpathmoveto{\pgfqpoint{0.318027in}{0.455010in}}%
\pgfpathlineto{\pgfqpoint{0.331295in}{0.454390in}}%
\pgfpathlineto{\pgfqpoint{0.335250in}{0.457479in}}%
\pgfpathlineto{\pgfqpoint{0.341417in}{0.470405in}}%
\pgfpathlineto{\pgfqpoint{0.341563in}{0.483332in}}%
\pgfpathlineto{\pgfqpoint{0.335603in}{0.496258in}}%
\pgfpathlineto{\pgfqpoint{0.331295in}{0.499546in}}%
\pgfpathlineto{\pgfqpoint{0.318027in}{0.498916in}}%
\pgfpathlineto{\pgfqpoint{0.314940in}{0.496258in}}%
\pgfpathlineto{\pgfqpoint{0.309244in}{0.483332in}}%
\pgfpathlineto{\pgfqpoint{0.309363in}{0.470405in}}%
\pgfpathlineto{\pgfqpoint{0.315211in}{0.457479in}}%
\pgfpathclose%
\pgfpathmoveto{\pgfqpoint{0.318498in}{0.470405in}}%
\pgfpathlineto{\pgfqpoint{0.318428in}{0.483332in}}%
\pgfpathlineto{\pgfqpoint{0.331295in}{0.484497in}}%
\pgfpathlineto{\pgfqpoint{0.331937in}{0.483332in}}%
\pgfpathlineto{\pgfqpoint{0.331927in}{0.470405in}}%
\pgfpathlineto{\pgfqpoint{0.331295in}{0.469259in}}%
\pgfpathclose%
\pgfusepath{fill}%
\end{pgfscope}%
\begin{pgfscope}%
\pgfpathrectangle{\pgfqpoint{0.211875in}{0.211875in}}{\pgfqpoint{1.313625in}{1.279725in}}%
\pgfusepath{clip}%
\pgfsetbuttcap%
\pgfsetroundjoin%
\definecolor{currentfill}{rgb}{0.947270,0.405591,0.279023}%
\pgfsetfillcolor{currentfill}%
\pgfsetlinewidth{0.000000pt}%
\definecolor{currentstroke}{rgb}{0.000000,0.000000,0.000000}%
\pgfsetstrokecolor{currentstroke}%
\pgfsetdash{}{0pt}%
\pgfpathmoveto{\pgfqpoint{0.437447in}{0.450298in}}%
\pgfpathlineto{\pgfqpoint{0.450716in}{0.450735in}}%
\pgfpathlineto{\pgfqpoint{0.458294in}{0.457479in}}%
\pgfpathlineto{\pgfqpoint{0.462758in}{0.470405in}}%
\pgfpathlineto{\pgfqpoint{0.462943in}{0.483332in}}%
\pgfpathlineto{\pgfqpoint{0.458870in}{0.496258in}}%
\pgfpathlineto{\pgfqpoint{0.450716in}{0.503368in}}%
\pgfpathlineto{\pgfqpoint{0.437447in}{0.503725in}}%
\pgfpathlineto{\pgfqpoint{0.427565in}{0.496258in}}%
\pgfpathlineto{\pgfqpoint{0.424178in}{0.488155in}}%
\pgfpathlineto{\pgfqpoint{0.423246in}{0.483332in}}%
\pgfpathlineto{\pgfqpoint{0.423362in}{0.470405in}}%
\pgfpathlineto{\pgfqpoint{0.424178in}{0.466507in}}%
\pgfpathlineto{\pgfqpoint{0.428129in}{0.457479in}}%
\pgfpathclose%
\pgfpathmoveto{\pgfqpoint{0.432476in}{0.470405in}}%
\pgfpathlineto{\pgfqpoint{0.432417in}{0.483332in}}%
\pgfpathlineto{\pgfqpoint{0.437447in}{0.492154in}}%
\pgfpathlineto{\pgfqpoint{0.450716in}{0.490547in}}%
\pgfpathlineto{\pgfqpoint{0.454200in}{0.483332in}}%
\pgfpathlineto{\pgfqpoint{0.454139in}{0.470405in}}%
\pgfpathlineto{\pgfqpoint{0.450716in}{0.463344in}}%
\pgfpathlineto{\pgfqpoint{0.437447in}{0.461674in}}%
\pgfpathclose%
\pgfusepath{fill}%
\end{pgfscope}%
\begin{pgfscope}%
\pgfpathrectangle{\pgfqpoint{0.211875in}{0.211875in}}{\pgfqpoint{1.313625in}{1.279725in}}%
\pgfusepath{clip}%
\pgfsetbuttcap%
\pgfsetroundjoin%
\definecolor{currentfill}{rgb}{0.947270,0.405591,0.279023}%
\pgfsetfillcolor{currentfill}%
\pgfsetlinewidth{0.000000pt}%
\definecolor{currentstroke}{rgb}{0.000000,0.000000,0.000000}%
\pgfsetstrokecolor{currentstroke}%
\pgfsetdash{}{0pt}%
\pgfpathmoveto{\pgfqpoint{0.543598in}{0.454244in}}%
\pgfpathlineto{\pgfqpoint{0.556867in}{0.446528in}}%
\pgfpathlineto{\pgfqpoint{0.570136in}{0.447629in}}%
\pgfpathlineto{\pgfqpoint{0.579926in}{0.457479in}}%
\pgfpathlineto{\pgfqpoint{0.583172in}{0.470405in}}%
\pgfpathlineto{\pgfqpoint{0.583381in}{0.483332in}}%
\pgfpathlineto{\pgfqpoint{0.580649in}{0.496258in}}%
\pgfpathlineto{\pgfqpoint{0.570136in}{0.506656in}}%
\pgfpathlineto{\pgfqpoint{0.556867in}{0.507565in}}%
\pgfpathlineto{\pgfqpoint{0.543598in}{0.500489in}}%
\pgfpathlineto{\pgfqpoint{0.541357in}{0.496258in}}%
\pgfpathlineto{\pgfqpoint{0.539031in}{0.483332in}}%
\pgfpathlineto{\pgfqpoint{0.539181in}{0.470405in}}%
\pgfpathlineto{\pgfqpoint{0.541860in}{0.457479in}}%
\pgfpathclose%
\pgfpathmoveto{\pgfqpoint{0.554505in}{0.457479in}}%
\pgfpathlineto{\pgfqpoint{0.546570in}{0.470405in}}%
\pgfpathlineto{\pgfqpoint{0.546449in}{0.483332in}}%
\pgfpathlineto{\pgfqpoint{0.554303in}{0.496258in}}%
\pgfpathlineto{\pgfqpoint{0.556867in}{0.497939in}}%
\pgfpathlineto{\pgfqpoint{0.567669in}{0.496258in}}%
\pgfpathlineto{\pgfqpoint{0.570136in}{0.495504in}}%
\pgfpathlineto{\pgfqpoint{0.575325in}{0.483332in}}%
\pgfpathlineto{\pgfqpoint{0.575229in}{0.470405in}}%
\pgfpathlineto{\pgfqpoint{0.570136in}{0.458562in}}%
\pgfpathlineto{\pgfqpoint{0.566715in}{0.457479in}}%
\pgfpathlineto{\pgfqpoint{0.556867in}{0.455897in}}%
\pgfpathclose%
\pgfusepath{fill}%
\end{pgfscope}%
\begin{pgfscope}%
\pgfpathrectangle{\pgfqpoint{0.211875in}{0.211875in}}{\pgfqpoint{1.313625in}{1.279725in}}%
\pgfusepath{clip}%
\pgfsetbuttcap%
\pgfsetroundjoin%
\definecolor{currentfill}{rgb}{0.947270,0.405591,0.279023}%
\pgfsetfillcolor{currentfill}%
\pgfsetlinewidth{0.000000pt}%
\definecolor{currentstroke}{rgb}{0.000000,0.000000,0.000000}%
\pgfsetstrokecolor{currentstroke}%
\pgfsetdash{}{0pt}%
\pgfpathmoveto{\pgfqpoint{0.218774in}{0.470405in}}%
\pgfpathlineto{\pgfqpoint{0.218864in}{0.483332in}}%
\pgfpathlineto{\pgfqpoint{0.211875in}{0.494360in}}%
\pgfpathlineto{\pgfqpoint{0.211875in}{0.483332in}}%
\pgfpathlineto{\pgfqpoint{0.211875in}{0.470405in}}%
\pgfpathlineto{\pgfqpoint{0.211875in}{0.459486in}}%
\pgfpathclose%
\pgfusepath{fill}%
\end{pgfscope}%
\begin{pgfscope}%
\pgfpathrectangle{\pgfqpoint{0.211875in}{0.211875in}}{\pgfqpoint{1.313625in}{1.279725in}}%
\pgfusepath{clip}%
\pgfsetbuttcap%
\pgfsetroundjoin%
\definecolor{currentfill}{rgb}{0.947270,0.405591,0.279023}%
\pgfsetfillcolor{currentfill}%
\pgfsetlinewidth{0.000000pt}%
\definecolor{currentstroke}{rgb}{0.000000,0.000000,0.000000}%
\pgfsetstrokecolor{currentstroke}%
\pgfsetdash{}{0pt}%
\pgfpathmoveto{\pgfqpoint{0.742633in}{0.521946in}}%
\pgfpathlineto{\pgfqpoint{0.743730in}{0.522111in}}%
\pgfpathlineto{\pgfqpoint{0.755902in}{0.525578in}}%
\pgfpathlineto{\pgfqpoint{0.761749in}{0.535038in}}%
\pgfpathlineto{\pgfqpoint{0.763571in}{0.547964in}}%
\pgfpathlineto{\pgfqpoint{0.763984in}{0.560891in}}%
\pgfpathlineto{\pgfqpoint{0.763578in}{0.573817in}}%
\pgfpathlineto{\pgfqpoint{0.761025in}{0.586744in}}%
\pgfpathlineto{\pgfqpoint{0.755902in}{0.592055in}}%
\pgfpathlineto{\pgfqpoint{0.742633in}{0.594393in}}%
\pgfpathlineto{\pgfqpoint{0.729364in}{0.593938in}}%
\pgfpathlineto{\pgfqpoint{0.716095in}{0.588417in}}%
\pgfpathlineto{\pgfqpoint{0.714991in}{0.586744in}}%
\pgfpathlineto{\pgfqpoint{0.711996in}{0.573817in}}%
\pgfpathlineto{\pgfqpoint{0.711476in}{0.560891in}}%
\pgfpathlineto{\pgfqpoint{0.711904in}{0.547964in}}%
\pgfpathlineto{\pgfqpoint{0.713888in}{0.535038in}}%
\pgfpathlineto{\pgfqpoint{0.716095in}{0.530060in}}%
\pgfpathlineto{\pgfqpoint{0.729364in}{0.522510in}}%
\pgfpathlineto{\pgfqpoint{0.738280in}{0.522111in}}%
\pgfpathclose%
\pgfpathmoveto{\pgfqpoint{0.725215in}{0.535038in}}%
\pgfpathlineto{\pgfqpoint{0.718762in}{0.547964in}}%
\pgfpathlineto{\pgfqpoint{0.717679in}{0.560891in}}%
\pgfpathlineto{\pgfqpoint{0.720238in}{0.573817in}}%
\pgfpathlineto{\pgfqpoint{0.729364in}{0.584439in}}%
\pgfpathlineto{\pgfqpoint{0.742633in}{0.585695in}}%
\pgfpathlineto{\pgfqpoint{0.755902in}{0.574412in}}%
\pgfpathlineto{\pgfqpoint{0.756133in}{0.573817in}}%
\pgfpathlineto{\pgfqpoint{0.757951in}{0.560891in}}%
\pgfpathlineto{\pgfqpoint{0.757167in}{0.547964in}}%
\pgfpathlineto{\pgfqpoint{0.755902in}{0.543457in}}%
\pgfpathlineto{\pgfqpoint{0.750166in}{0.535038in}}%
\pgfpathlineto{\pgfqpoint{0.742633in}{0.530757in}}%
\pgfpathlineto{\pgfqpoint{0.729364in}{0.531860in}}%
\pgfpathclose%
\pgfusepath{fill}%
\end{pgfscope}%
\begin{pgfscope}%
\pgfpathrectangle{\pgfqpoint{0.211875in}{0.211875in}}{\pgfqpoint{1.313625in}{1.279725in}}%
\pgfusepath{clip}%
\pgfsetbuttcap%
\pgfsetroundjoin%
\definecolor{currentfill}{rgb}{0.947270,0.405591,0.279023}%
\pgfsetfillcolor{currentfill}%
\pgfsetlinewidth{0.000000pt}%
\definecolor{currentstroke}{rgb}{0.000000,0.000000,0.000000}%
\pgfsetstrokecolor{currentstroke}%
\pgfsetdash{}{0pt}%
\pgfpathmoveto{\pgfqpoint{0.848784in}{0.520624in}}%
\pgfpathlineto{\pgfqpoint{0.862053in}{0.520476in}}%
\pgfpathlineto{\pgfqpoint{0.871247in}{0.522111in}}%
\pgfpathlineto{\pgfqpoint{0.875322in}{0.523622in}}%
\pgfpathlineto{\pgfqpoint{0.881416in}{0.535038in}}%
\pgfpathlineto{\pgfqpoint{0.882688in}{0.547964in}}%
\pgfpathlineto{\pgfqpoint{0.883027in}{0.560891in}}%
\pgfpathlineto{\pgfqpoint{0.882881in}{0.573817in}}%
\pgfpathlineto{\pgfqpoint{0.881621in}{0.586744in}}%
\pgfpathlineto{\pgfqpoint{0.875322in}{0.594420in}}%
\pgfpathlineto{\pgfqpoint{0.862053in}{0.596007in}}%
\pgfpathlineto{\pgfqpoint{0.848784in}{0.595847in}}%
\pgfpathlineto{\pgfqpoint{0.835515in}{0.593485in}}%
\pgfpathlineto{\pgfqpoint{0.830172in}{0.586744in}}%
\pgfpathlineto{\pgfqpoint{0.828472in}{0.573817in}}%
\pgfpathlineto{\pgfqpoint{0.828227in}{0.560891in}}%
\pgfpathlineto{\pgfqpoint{0.828587in}{0.547964in}}%
\pgfpathlineto{\pgfqpoint{0.830032in}{0.535038in}}%
\pgfpathlineto{\pgfqpoint{0.835515in}{0.524540in}}%
\pgfpathlineto{\pgfqpoint{0.841354in}{0.522111in}}%
\pgfpathclose%
\pgfpathmoveto{\pgfqpoint{0.840610in}{0.535038in}}%
\pgfpathlineto{\pgfqpoint{0.835515in}{0.544817in}}%
\pgfpathlineto{\pgfqpoint{0.834759in}{0.547964in}}%
\pgfpathlineto{\pgfqpoint{0.834049in}{0.560891in}}%
\pgfpathlineto{\pgfqpoint{0.835515in}{0.573079in}}%
\pgfpathlineto{\pgfqpoint{0.835680in}{0.573817in}}%
\pgfpathlineto{\pgfqpoint{0.847825in}{0.586744in}}%
\pgfpathlineto{\pgfqpoint{0.848784in}{0.587153in}}%
\pgfpathlineto{\pgfqpoint{0.862053in}{0.587307in}}%
\pgfpathlineto{\pgfqpoint{0.863447in}{0.586744in}}%
\pgfpathlineto{\pgfqpoint{0.875322in}{0.574612in}}%
\pgfpathlineto{\pgfqpoint{0.875585in}{0.573817in}}%
\pgfpathlineto{\pgfqpoint{0.877110in}{0.560891in}}%
\pgfpathlineto{\pgfqpoint{0.876409in}{0.547964in}}%
\pgfpathlineto{\pgfqpoint{0.875322in}{0.543501in}}%
\pgfpathlineto{\pgfqpoint{0.870765in}{0.535038in}}%
\pgfpathlineto{\pgfqpoint{0.862053in}{0.529353in}}%
\pgfpathlineto{\pgfqpoint{0.848784in}{0.529512in}}%
\pgfpathclose%
\pgfusepath{fill}%
\end{pgfscope}%
\begin{pgfscope}%
\pgfpathrectangle{\pgfqpoint{0.211875in}{0.211875in}}{\pgfqpoint{1.313625in}{1.279725in}}%
\pgfusepath{clip}%
\pgfsetbuttcap%
\pgfsetroundjoin%
\definecolor{currentfill}{rgb}{0.947270,0.405591,0.279023}%
\pgfsetfillcolor{currentfill}%
\pgfsetlinewidth{0.000000pt}%
\definecolor{currentstroke}{rgb}{0.000000,0.000000,0.000000}%
\pgfsetstrokecolor{currentstroke}%
\pgfsetdash{}{0pt}%
\pgfpathmoveto{\pgfqpoint{0.954936in}{0.521471in}}%
\pgfpathlineto{\pgfqpoint{0.968205in}{0.519442in}}%
\pgfpathlineto{\pgfqpoint{0.981473in}{0.519511in}}%
\pgfpathlineto{\pgfqpoint{0.994082in}{0.522111in}}%
\pgfpathlineto{\pgfqpoint{0.994742in}{0.522431in}}%
\pgfpathlineto{\pgfqpoint{1.000448in}{0.535038in}}%
\pgfpathlineto{\pgfqpoint{1.001383in}{0.547964in}}%
\pgfpathlineto{\pgfqpoint{1.001676in}{0.560891in}}%
\pgfpathlineto{\pgfqpoint{1.001690in}{0.573817in}}%
\pgfpathlineto{\pgfqpoint{1.001219in}{0.586744in}}%
\pgfpathlineto{\pgfqpoint{0.994742in}{0.596233in}}%
\pgfpathlineto{\pgfqpoint{0.981473in}{0.597101in}}%
\pgfpathlineto{\pgfqpoint{0.968205in}{0.597059in}}%
\pgfpathlineto{\pgfqpoint{0.954936in}{0.596199in}}%
\pgfpathlineto{\pgfqpoint{0.946139in}{0.586744in}}%
\pgfpathlineto{\pgfqpoint{0.945357in}{0.573817in}}%
\pgfpathlineto{\pgfqpoint{0.945306in}{0.560891in}}%
\pgfpathlineto{\pgfqpoint{0.945621in}{0.547964in}}%
\pgfpathlineto{\pgfqpoint{0.946694in}{0.535038in}}%
\pgfpathlineto{\pgfqpoint{0.953599in}{0.522111in}}%
\pgfpathclose%
\pgfpathmoveto{\pgfqpoint{0.956253in}{0.535038in}}%
\pgfpathlineto{\pgfqpoint{0.954936in}{0.537046in}}%
\pgfpathlineto{\pgfqpoint{0.951915in}{0.547964in}}%
\pgfpathlineto{\pgfqpoint{0.951242in}{0.560891in}}%
\pgfpathlineto{\pgfqpoint{0.952654in}{0.573817in}}%
\pgfpathlineto{\pgfqpoint{0.954936in}{0.579791in}}%
\pgfpathlineto{\pgfqpoint{0.963228in}{0.586744in}}%
\pgfpathlineto{\pgfqpoint{0.968205in}{0.588593in}}%
\pgfpathlineto{\pgfqpoint{0.981473in}{0.588097in}}%
\pgfpathlineto{\pgfqpoint{0.984403in}{0.586744in}}%
\pgfpathlineto{\pgfqpoint{0.994389in}{0.573817in}}%
\pgfpathlineto{\pgfqpoint{0.994742in}{0.571786in}}%
\pgfpathlineto{\pgfqpoint{0.995834in}{0.560891in}}%
\pgfpathlineto{\pgfqpoint{0.995185in}{0.547964in}}%
\pgfpathlineto{\pgfqpoint{0.994742in}{0.545836in}}%
\pgfpathlineto{\pgfqpoint{0.990177in}{0.535038in}}%
\pgfpathlineto{\pgfqpoint{0.981473in}{0.528582in}}%
\pgfpathlineto{\pgfqpoint{0.968205in}{0.527973in}}%
\pgfpathclose%
\pgfusepath{fill}%
\end{pgfscope}%
\begin{pgfscope}%
\pgfpathrectangle{\pgfqpoint{0.211875in}{0.211875in}}{\pgfqpoint{1.313625in}{1.279725in}}%
\pgfusepath{clip}%
\pgfsetbuttcap%
\pgfsetroundjoin%
\definecolor{currentfill}{rgb}{0.947270,0.405591,0.279023}%
\pgfsetfillcolor{currentfill}%
\pgfsetlinewidth{0.000000pt}%
\definecolor{currentstroke}{rgb}{0.000000,0.000000,0.000000}%
\pgfsetstrokecolor{currentstroke}%
\pgfsetdash{}{0pt}%
\pgfpathmoveto{\pgfqpoint{1.074356in}{0.520068in}}%
\pgfpathlineto{\pgfqpoint{1.087625in}{0.518854in}}%
\pgfpathlineto{\pgfqpoint{1.100894in}{0.519084in}}%
\pgfpathlineto{\pgfqpoint{1.113688in}{0.522111in}}%
\pgfpathlineto{\pgfqpoint{1.114163in}{0.522416in}}%
\pgfpathlineto{\pgfqpoint{1.118876in}{0.535038in}}%
\pgfpathlineto{\pgfqpoint{1.119667in}{0.547964in}}%
\pgfpathlineto{\pgfqpoint{1.119941in}{0.560891in}}%
\pgfpathlineto{\pgfqpoint{1.120021in}{0.573817in}}%
\pgfpathlineto{\pgfqpoint{1.119880in}{0.586744in}}%
\pgfpathlineto{\pgfqpoint{1.114163in}{0.597178in}}%
\pgfpathlineto{\pgfqpoint{1.100894in}{0.597642in}}%
\pgfpathlineto{\pgfqpoint{1.087625in}{0.597650in}}%
\pgfpathlineto{\pgfqpoint{1.074356in}{0.597406in}}%
\pgfpathlineto{\pgfqpoint{1.062890in}{0.586744in}}%
\pgfpathlineto{\pgfqpoint{1.062634in}{0.573817in}}%
\pgfpathlineto{\pgfqpoint{1.062697in}{0.560891in}}%
\pgfpathlineto{\pgfqpoint{1.062989in}{0.547964in}}%
\pgfpathlineto{\pgfqpoint{1.063863in}{0.535038in}}%
\pgfpathlineto{\pgfqpoint{1.069431in}{0.522111in}}%
\pgfpathclose%
\pgfpathmoveto{\pgfqpoint{1.073160in}{0.535038in}}%
\pgfpathlineto{\pgfqpoint{1.069453in}{0.547964in}}%
\pgfpathlineto{\pgfqpoint{1.068791in}{0.560891in}}%
\pgfpathlineto{\pgfqpoint{1.070132in}{0.573817in}}%
\pgfpathlineto{\pgfqpoint{1.074356in}{0.583407in}}%
\pgfpathlineto{\pgfqpoint{1.079461in}{0.586744in}}%
\pgfpathlineto{\pgfqpoint{1.087625in}{0.589347in}}%
\pgfpathlineto{\pgfqpoint{1.100894in}{0.588248in}}%
\pgfpathlineto{\pgfqpoint{1.103761in}{0.586744in}}%
\pgfpathlineto{\pgfqpoint{1.112472in}{0.573817in}}%
\pgfpathlineto{\pgfqpoint{1.114125in}{0.560891in}}%
\pgfpathlineto{\pgfqpoint{1.113305in}{0.547964in}}%
\pgfpathlineto{\pgfqpoint{1.108710in}{0.535038in}}%
\pgfpathlineto{\pgfqpoint{1.100894in}{0.528492in}}%
\pgfpathlineto{\pgfqpoint{1.087625in}{0.527157in}}%
\pgfpathlineto{\pgfqpoint{1.074356in}{0.533306in}}%
\pgfpathclose%
\pgfusepath{fill}%
\end{pgfscope}%
\begin{pgfscope}%
\pgfpathrectangle{\pgfqpoint{0.211875in}{0.211875in}}{\pgfqpoint{1.313625in}{1.279725in}}%
\pgfusepath{clip}%
\pgfsetbuttcap%
\pgfsetroundjoin%
\definecolor{currentfill}{rgb}{0.947270,0.405591,0.279023}%
\pgfsetfillcolor{currentfill}%
\pgfsetlinewidth{0.000000pt}%
\definecolor{currentstroke}{rgb}{0.000000,0.000000,0.000000}%
\pgfsetstrokecolor{currentstroke}%
\pgfsetdash{}{0pt}%
\pgfpathmoveto{\pgfqpoint{1.193777in}{0.519683in}}%
\pgfpathlineto{\pgfqpoint{1.207045in}{0.518816in}}%
\pgfpathlineto{\pgfqpoint{1.220314in}{0.519254in}}%
\pgfpathlineto{\pgfqpoint{1.230905in}{0.522111in}}%
\pgfpathlineto{\pgfqpoint{1.233583in}{0.524467in}}%
\pgfpathlineto{\pgfqpoint{1.236703in}{0.535038in}}%
\pgfpathlineto{\pgfqpoint{1.237536in}{0.547964in}}%
\pgfpathlineto{\pgfqpoint{1.237815in}{0.560891in}}%
\pgfpathlineto{\pgfqpoint{1.237874in}{0.573817in}}%
\pgfpathlineto{\pgfqpoint{1.237630in}{0.586744in}}%
\pgfpathlineto{\pgfqpoint{1.233583in}{0.596492in}}%
\pgfpathlineto{\pgfqpoint{1.220314in}{0.597567in}}%
\pgfpathlineto{\pgfqpoint{1.207045in}{0.597667in}}%
\pgfpathlineto{\pgfqpoint{1.193777in}{0.597567in}}%
\pgfpathlineto{\pgfqpoint{1.180508in}{0.587844in}}%
\pgfpathlineto{\pgfqpoint{1.180460in}{0.586744in}}%
\pgfpathlineto{\pgfqpoint{1.180312in}{0.573817in}}%
\pgfpathlineto{\pgfqpoint{1.180397in}{0.560891in}}%
\pgfpathlineto{\pgfqpoint{1.180508in}{0.555729in}}%
\pgfpathlineto{\pgfqpoint{1.180690in}{0.547964in}}%
\pgfpathlineto{\pgfqpoint{1.181549in}{0.535038in}}%
\pgfpathlineto{\pgfqpoint{1.187062in}{0.522111in}}%
\pgfpathclose%
\pgfpathmoveto{\pgfqpoint{1.191181in}{0.535038in}}%
\pgfpathlineto{\pgfqpoint{1.187376in}{0.547964in}}%
\pgfpathlineto{\pgfqpoint{1.186696in}{0.560891in}}%
\pgfpathlineto{\pgfqpoint{1.188066in}{0.573817in}}%
\pgfpathlineto{\pgfqpoint{1.193777in}{0.585180in}}%
\pgfpathlineto{\pgfqpoint{1.196877in}{0.586744in}}%
\pgfpathlineto{\pgfqpoint{1.207045in}{0.589468in}}%
\pgfpathlineto{\pgfqpoint{1.220314in}{0.587679in}}%
\pgfpathlineto{\pgfqpoint{1.221889in}{0.586744in}}%
\pgfpathlineto{\pgfqpoint{1.230085in}{0.573817in}}%
\pgfpathlineto{\pgfqpoint{1.231651in}{0.560891in}}%
\pgfpathlineto{\pgfqpoint{1.230878in}{0.547964in}}%
\pgfpathlineto{\pgfqpoint{1.226550in}{0.535038in}}%
\pgfpathlineto{\pgfqpoint{1.220314in}{0.529165in}}%
\pgfpathlineto{\pgfqpoint{1.207045in}{0.527013in}}%
\pgfpathlineto{\pgfqpoint{1.193777in}{0.531722in}}%
\pgfpathclose%
\pgfusepath{fill}%
\end{pgfscope}%
\begin{pgfscope}%
\pgfpathrectangle{\pgfqpoint{0.211875in}{0.211875in}}{\pgfqpoint{1.313625in}{1.279725in}}%
\pgfusepath{clip}%
\pgfsetbuttcap%
\pgfsetroundjoin%
\definecolor{currentfill}{rgb}{0.947270,0.405591,0.279023}%
\pgfsetfillcolor{currentfill}%
\pgfsetlinewidth{0.000000pt}%
\definecolor{currentstroke}{rgb}{0.000000,0.000000,0.000000}%
\pgfsetstrokecolor{currentstroke}%
\pgfsetdash{}{0pt}%
\pgfpathmoveto{\pgfqpoint{1.313197in}{0.520067in}}%
\pgfpathlineto{\pgfqpoint{1.326466in}{0.519310in}}%
\pgfpathlineto{\pgfqpoint{1.339735in}{0.520118in}}%
\pgfpathlineto{\pgfqpoint{1.346234in}{0.522111in}}%
\pgfpathlineto{\pgfqpoint{1.353004in}{0.530856in}}%
\pgfpathlineto{\pgfqpoint{1.353910in}{0.535038in}}%
\pgfpathlineto{\pgfqpoint{1.354968in}{0.547964in}}%
\pgfpathlineto{\pgfqpoint{1.355278in}{0.560891in}}%
\pgfpathlineto{\pgfqpoint{1.355228in}{0.573817in}}%
\pgfpathlineto{\pgfqpoint{1.354456in}{0.586744in}}%
\pgfpathlineto{\pgfqpoint{1.353004in}{0.591890in}}%
\pgfpathlineto{\pgfqpoint{1.339735in}{0.596771in}}%
\pgfpathlineto{\pgfqpoint{1.326466in}{0.597132in}}%
\pgfpathlineto{\pgfqpoint{1.313197in}{0.596950in}}%
\pgfpathlineto{\pgfqpoint{1.299928in}{0.591728in}}%
\pgfpathlineto{\pgfqpoint{1.299008in}{0.586744in}}%
\pgfpathlineto{\pgfqpoint{1.298527in}{0.573817in}}%
\pgfpathlineto{\pgfqpoint{1.298541in}{0.560891in}}%
\pgfpathlineto{\pgfqpoint{1.298841in}{0.547964in}}%
\pgfpathlineto{\pgfqpoint{1.299797in}{0.535038in}}%
\pgfpathlineto{\pgfqpoint{1.299928in}{0.534213in}}%
\pgfpathlineto{\pgfqpoint{1.306740in}{0.522111in}}%
\pgfpathclose%
\pgfpathmoveto{\pgfqpoint{1.309845in}{0.535038in}}%
\pgfpathlineto{\pgfqpoint{1.305705in}{0.547964in}}%
\pgfpathlineto{\pgfqpoint{1.304975in}{0.560891in}}%
\pgfpathlineto{\pgfqpoint{1.306485in}{0.573817in}}%
\pgfpathlineto{\pgfqpoint{1.313197in}{0.585603in}}%
\pgfpathlineto{\pgfqpoint{1.316199in}{0.586744in}}%
\pgfpathlineto{\pgfqpoint{1.326466in}{0.588981in}}%
\pgfpathlineto{\pgfqpoint{1.337829in}{0.586744in}}%
\pgfpathlineto{\pgfqpoint{1.339735in}{0.586124in}}%
\pgfpathlineto{\pgfqpoint{1.347294in}{0.573817in}}%
\pgfpathlineto{\pgfqpoint{1.348893in}{0.560891in}}%
\pgfpathlineto{\pgfqpoint{1.348131in}{0.547964in}}%
\pgfpathlineto{\pgfqpoint{1.343806in}{0.535038in}}%
\pgfpathlineto{\pgfqpoint{1.339735in}{0.530731in}}%
\pgfpathlineto{\pgfqpoint{1.326466in}{0.527513in}}%
\pgfpathlineto{\pgfqpoint{1.313197in}{0.531241in}}%
\pgfpathclose%
\pgfusepath{fill}%
\end{pgfscope}%
\begin{pgfscope}%
\pgfpathrectangle{\pgfqpoint{0.211875in}{0.211875in}}{\pgfqpoint{1.313625in}{1.279725in}}%
\pgfusepath{clip}%
\pgfsetbuttcap%
\pgfsetroundjoin%
\definecolor{currentfill}{rgb}{0.947270,0.405591,0.279023}%
\pgfsetfillcolor{currentfill}%
\pgfsetlinewidth{0.000000pt}%
\definecolor{currentstroke}{rgb}{0.000000,0.000000,0.000000}%
\pgfsetstrokecolor{currentstroke}%
\pgfsetdash{}{0pt}%
\pgfpathmoveto{\pgfqpoint{1.432617in}{0.521068in}}%
\pgfpathlineto{\pgfqpoint{1.445886in}{0.520335in}}%
\pgfpathlineto{\pgfqpoint{1.459155in}{0.521826in}}%
\pgfpathlineto{\pgfqpoint{1.459975in}{0.522111in}}%
\pgfpathlineto{\pgfqpoint{1.470293in}{0.535038in}}%
\pgfpathlineto{\pgfqpoint{1.471885in}{0.547964in}}%
\pgfpathlineto{\pgfqpoint{1.472281in}{0.560891in}}%
\pgfpathlineto{\pgfqpoint{1.472011in}{0.573817in}}%
\pgfpathlineto{\pgfqpoint{1.470139in}{0.586744in}}%
\pgfpathlineto{\pgfqpoint{1.459155in}{0.595090in}}%
\pgfpathlineto{\pgfqpoint{1.445886in}{0.596043in}}%
\pgfpathlineto{\pgfqpoint{1.432617in}{0.595715in}}%
\pgfpathlineto{\pgfqpoint{1.419348in}{0.589281in}}%
\pgfpathlineto{\pgfqpoint{1.418530in}{0.586744in}}%
\pgfpathlineto{\pgfqpoint{1.417275in}{0.573817in}}%
\pgfpathlineto{\pgfqpoint{1.417130in}{0.560891in}}%
\pgfpathlineto{\pgfqpoint{1.417468in}{0.547964in}}%
\pgfpathlineto{\pgfqpoint{1.418737in}{0.535038in}}%
\pgfpathlineto{\pgfqpoint{1.419348in}{0.532486in}}%
\pgfpathlineto{\pgfqpoint{1.428859in}{0.522111in}}%
\pgfpathclose%
\pgfpathmoveto{\pgfqpoint{1.429235in}{0.535038in}}%
\pgfpathlineto{\pgfqpoint{1.424481in}{0.547964in}}%
\pgfpathlineto{\pgfqpoint{1.423663in}{0.560891in}}%
\pgfpathlineto{\pgfqpoint{1.425444in}{0.573817in}}%
\pgfpathlineto{\pgfqpoint{1.432617in}{0.584975in}}%
\pgfpathlineto{\pgfqpoint{1.439094in}{0.586744in}}%
\pgfpathlineto{\pgfqpoint{1.445886in}{0.587886in}}%
\pgfpathlineto{\pgfqpoint{1.450515in}{0.586744in}}%
\pgfpathlineto{\pgfqpoint{1.459155in}{0.582963in}}%
\pgfpathlineto{\pgfqpoint{1.464132in}{0.573817in}}%
\pgfpathlineto{\pgfqpoint{1.465867in}{0.560891in}}%
\pgfpathlineto{\pgfqpoint{1.465085in}{0.547964in}}%
\pgfpathlineto{\pgfqpoint{1.460545in}{0.535038in}}%
\pgfpathlineto{\pgfqpoint{1.459155in}{0.533386in}}%
\pgfpathlineto{\pgfqpoint{1.445886in}{0.528658in}}%
\pgfpathlineto{\pgfqpoint{1.432617in}{0.531631in}}%
\pgfpathclose%
\pgfusepath{fill}%
\end{pgfscope}%
\begin{pgfscope}%
\pgfpathrectangle{\pgfqpoint{0.211875in}{0.211875in}}{\pgfqpoint{1.313625in}{1.279725in}}%
\pgfusepath{clip}%
\pgfsetbuttcap%
\pgfsetroundjoin%
\definecolor{currentfill}{rgb}{0.947270,0.405591,0.279023}%
\pgfsetfillcolor{currentfill}%
\pgfsetlinewidth{0.000000pt}%
\definecolor{currentstroke}{rgb}{0.000000,0.000000,0.000000}%
\pgfsetstrokecolor{currentstroke}%
\pgfsetdash{}{0pt}%
\pgfpathmoveto{\pgfqpoint{0.264951in}{0.534171in}}%
\pgfpathlineto{\pgfqpoint{0.268366in}{0.535038in}}%
\pgfpathlineto{\pgfqpoint{0.278220in}{0.539846in}}%
\pgfpathlineto{\pgfqpoint{0.281928in}{0.547964in}}%
\pgfpathlineto{\pgfqpoint{0.283020in}{0.560891in}}%
\pgfpathlineto{\pgfqpoint{0.280212in}{0.573817in}}%
\pgfpathlineto{\pgfqpoint{0.278220in}{0.576850in}}%
\pgfpathlineto{\pgfqpoint{0.264951in}{0.581455in}}%
\pgfpathlineto{\pgfqpoint{0.252520in}{0.573817in}}%
\pgfpathlineto{\pgfqpoint{0.251682in}{0.572281in}}%
\pgfpathlineto{\pgfqpoint{0.249374in}{0.560891in}}%
\pgfpathlineto{\pgfqpoint{0.250404in}{0.547964in}}%
\pgfpathlineto{\pgfqpoint{0.251682in}{0.544579in}}%
\pgfpathlineto{\pgfqpoint{0.262844in}{0.535038in}}%
\pgfpathclose%
\pgfpathmoveto{\pgfqpoint{0.263625in}{0.547964in}}%
\pgfpathlineto{\pgfqpoint{0.260543in}{0.560891in}}%
\pgfpathlineto{\pgfqpoint{0.264951in}{0.567886in}}%
\pgfpathlineto{\pgfqpoint{0.271560in}{0.560891in}}%
\pgfpathlineto{\pgfqpoint{0.266956in}{0.547964in}}%
\pgfpathlineto{\pgfqpoint{0.264951in}{0.546790in}}%
\pgfpathclose%
\pgfusepath{fill}%
\end{pgfscope}%
\begin{pgfscope}%
\pgfpathrectangle{\pgfqpoint{0.211875in}{0.211875in}}{\pgfqpoint{1.313625in}{1.279725in}}%
\pgfusepath{clip}%
\pgfsetbuttcap%
\pgfsetroundjoin%
\definecolor{currentfill}{rgb}{0.947270,0.405591,0.279023}%
\pgfsetfillcolor{currentfill}%
\pgfsetlinewidth{0.000000pt}%
\definecolor{currentstroke}{rgb}{0.000000,0.000000,0.000000}%
\pgfsetstrokecolor{currentstroke}%
\pgfsetdash{}{0pt}%
\pgfpathmoveto{\pgfqpoint{0.371102in}{0.534510in}}%
\pgfpathlineto{\pgfqpoint{0.384371in}{0.530188in}}%
\pgfpathlineto{\pgfqpoint{0.397640in}{0.534429in}}%
\pgfpathlineto{\pgfqpoint{0.398192in}{0.535038in}}%
\pgfpathlineto{\pgfqpoint{0.403278in}{0.547964in}}%
\pgfpathlineto{\pgfqpoint{0.404128in}{0.560891in}}%
\pgfpathlineto{\pgfqpoint{0.402180in}{0.573817in}}%
\pgfpathlineto{\pgfqpoint{0.397640in}{0.581630in}}%
\pgfpathlineto{\pgfqpoint{0.384371in}{0.586328in}}%
\pgfpathlineto{\pgfqpoint{0.371102in}{0.581426in}}%
\pgfpathlineto{\pgfqpoint{0.366523in}{0.573817in}}%
\pgfpathlineto{\pgfqpoint{0.364400in}{0.560891in}}%
\pgfpathlineto{\pgfqpoint{0.365295in}{0.547964in}}%
\pgfpathlineto{\pgfqpoint{0.370616in}{0.535038in}}%
\pgfpathclose%
\pgfpathmoveto{\pgfqpoint{0.374206in}{0.547964in}}%
\pgfpathlineto{\pgfqpoint{0.371102in}{0.559907in}}%
\pgfpathlineto{\pgfqpoint{0.371009in}{0.560891in}}%
\pgfpathlineto{\pgfqpoint{0.371102in}{0.561296in}}%
\pgfpathlineto{\pgfqpoint{0.380174in}{0.573817in}}%
\pgfpathlineto{\pgfqpoint{0.384371in}{0.575909in}}%
\pgfpathlineto{\pgfqpoint{0.388456in}{0.573817in}}%
\pgfpathlineto{\pgfqpoint{0.397155in}{0.560891in}}%
\pgfpathlineto{\pgfqpoint{0.394061in}{0.547964in}}%
\pgfpathlineto{\pgfqpoint{0.384371in}{0.540665in}}%
\pgfpathclose%
\pgfusepath{fill}%
\end{pgfscope}%
\begin{pgfscope}%
\pgfpathrectangle{\pgfqpoint{0.211875in}{0.211875in}}{\pgfqpoint{1.313625in}{1.279725in}}%
\pgfusepath{clip}%
\pgfsetbuttcap%
\pgfsetroundjoin%
\definecolor{currentfill}{rgb}{0.947270,0.405591,0.279023}%
\pgfsetfillcolor{currentfill}%
\pgfsetlinewidth{0.000000pt}%
\definecolor{currentstroke}{rgb}{0.000000,0.000000,0.000000}%
\pgfsetstrokecolor{currentstroke}%
\pgfsetdash{}{0pt}%
\pgfpathmoveto{\pgfqpoint{0.490523in}{0.529334in}}%
\pgfpathlineto{\pgfqpoint{0.503792in}{0.526852in}}%
\pgfpathlineto{\pgfqpoint{0.517061in}{0.531061in}}%
\pgfpathlineto{\pgfqpoint{0.520252in}{0.535038in}}%
\pgfpathlineto{\pgfqpoint{0.523931in}{0.547964in}}%
\pgfpathlineto{\pgfqpoint{0.524593in}{0.560891in}}%
\pgfpathlineto{\pgfqpoint{0.523309in}{0.573817in}}%
\pgfpathlineto{\pgfqpoint{0.517061in}{0.586015in}}%
\pgfpathlineto{\pgfqpoint{0.515489in}{0.586744in}}%
\pgfpathlineto{\pgfqpoint{0.503792in}{0.589622in}}%
\pgfpathlineto{\pgfqpoint{0.490523in}{0.587514in}}%
\pgfpathlineto{\pgfqpoint{0.489281in}{0.586744in}}%
\pgfpathlineto{\pgfqpoint{0.481104in}{0.573817in}}%
\pgfpathlineto{\pgfqpoint{0.479496in}{0.560891in}}%
\pgfpathlineto{\pgfqpoint{0.480274in}{0.547964in}}%
\pgfpathlineto{\pgfqpoint{0.484631in}{0.535038in}}%
\pgfpathclose%
\pgfpathmoveto{\pgfqpoint{0.487676in}{0.547964in}}%
\pgfpathlineto{\pgfqpoint{0.486484in}{0.560891in}}%
\pgfpathlineto{\pgfqpoint{0.489662in}{0.573817in}}%
\pgfpathlineto{\pgfqpoint{0.490523in}{0.575089in}}%
\pgfpathlineto{\pgfqpoint{0.503792in}{0.579940in}}%
\pgfpathlineto{\pgfqpoint{0.513532in}{0.573817in}}%
\pgfpathlineto{\pgfqpoint{0.517061in}{0.566742in}}%
\pgfpathlineto{\pgfqpoint{0.518183in}{0.560891in}}%
\pgfpathlineto{\pgfqpoint{0.517123in}{0.547964in}}%
\pgfpathlineto{\pgfqpoint{0.517061in}{0.547791in}}%
\pgfpathlineto{\pgfqpoint{0.503792in}{0.535615in}}%
\pgfpathlineto{\pgfqpoint{0.490523in}{0.541947in}}%
\pgfpathclose%
\pgfusepath{fill}%
\end{pgfscope}%
\begin{pgfscope}%
\pgfpathrectangle{\pgfqpoint{0.211875in}{0.211875in}}{\pgfqpoint{1.313625in}{1.279725in}}%
\pgfusepath{clip}%
\pgfsetbuttcap%
\pgfsetroundjoin%
\definecolor{currentfill}{rgb}{0.947270,0.405591,0.279023}%
\pgfsetfillcolor{currentfill}%
\pgfsetlinewidth{0.000000pt}%
\definecolor{currentstroke}{rgb}{0.000000,0.000000,0.000000}%
\pgfsetstrokecolor{currentstroke}%
\pgfsetdash{}{0pt}%
\pgfpathmoveto{\pgfqpoint{0.609943in}{0.525413in}}%
\pgfpathlineto{\pgfqpoint{0.623212in}{0.524109in}}%
\pgfpathlineto{\pgfqpoint{0.636481in}{0.528093in}}%
\pgfpathlineto{\pgfqpoint{0.641392in}{0.535038in}}%
\pgfpathlineto{\pgfqpoint{0.644003in}{0.547964in}}%
\pgfpathlineto{\pgfqpoint{0.644522in}{0.560891in}}%
\pgfpathlineto{\pgfqpoint{0.643743in}{0.573817in}}%
\pgfpathlineto{\pgfqpoint{0.639319in}{0.586744in}}%
\pgfpathlineto{\pgfqpoint{0.636481in}{0.589285in}}%
\pgfpathlineto{\pgfqpoint{0.623212in}{0.592269in}}%
\pgfpathlineto{\pgfqpoint{0.609943in}{0.591216in}}%
\pgfpathlineto{\pgfqpoint{0.601807in}{0.586744in}}%
\pgfpathlineto{\pgfqpoint{0.596674in}{0.577096in}}%
\pgfpathlineto{\pgfqpoint{0.595958in}{0.573817in}}%
\pgfpathlineto{\pgfqpoint{0.595082in}{0.560891in}}%
\pgfpathlineto{\pgfqpoint{0.595602in}{0.547964in}}%
\pgfpathlineto{\pgfqpoint{0.596674in}{0.541679in}}%
\pgfpathlineto{\pgfqpoint{0.598785in}{0.535038in}}%
\pgfpathclose%
\pgfpathmoveto{\pgfqpoint{0.610479in}{0.535038in}}%
\pgfpathlineto{\pgfqpoint{0.609943in}{0.535202in}}%
\pgfpathlineto{\pgfqpoint{0.603181in}{0.547964in}}%
\pgfpathlineto{\pgfqpoint{0.602057in}{0.560891in}}%
\pgfpathlineto{\pgfqpoint{0.604885in}{0.573817in}}%
\pgfpathlineto{\pgfqpoint{0.609943in}{0.580459in}}%
\pgfpathlineto{\pgfqpoint{0.623212in}{0.583190in}}%
\pgfpathlineto{\pgfqpoint{0.635754in}{0.573817in}}%
\pgfpathlineto{\pgfqpoint{0.636481in}{0.571922in}}%
\pgfpathlineto{\pgfqpoint{0.638326in}{0.560891in}}%
\pgfpathlineto{\pgfqpoint{0.637425in}{0.547964in}}%
\pgfpathlineto{\pgfqpoint{0.636481in}{0.545015in}}%
\pgfpathlineto{\pgfqpoint{0.627855in}{0.535038in}}%
\pgfpathlineto{\pgfqpoint{0.623212in}{0.532775in}}%
\pgfpathclose%
\pgfusepath{fill}%
\end{pgfscope}%
\begin{pgfscope}%
\pgfpathrectangle{\pgfqpoint{0.211875in}{0.211875in}}{\pgfqpoint{1.313625in}{1.279725in}}%
\pgfusepath{clip}%
\pgfsetbuttcap%
\pgfsetroundjoin%
\definecolor{currentfill}{rgb}{0.947270,0.405591,0.279023}%
\pgfsetfillcolor{currentfill}%
\pgfsetlinewidth{0.000000pt}%
\definecolor{currentstroke}{rgb}{0.000000,0.000000,0.000000}%
\pgfsetstrokecolor{currentstroke}%
\pgfsetdash{}{0pt}%
\pgfpathmoveto{\pgfqpoint{0.915129in}{0.599208in}}%
\pgfpathlineto{\pgfqpoint{0.928398in}{0.599496in}}%
\pgfpathlineto{\pgfqpoint{0.930244in}{0.599670in}}%
\pgfpathlineto{\pgfqpoint{0.941667in}{0.605832in}}%
\pgfpathlineto{\pgfqpoint{0.942921in}{0.612597in}}%
\pgfpathlineto{\pgfqpoint{0.943542in}{0.625523in}}%
\pgfpathlineto{\pgfqpoint{0.943781in}{0.638450in}}%
\pgfpathlineto{\pgfqpoint{0.943969in}{0.651377in}}%
\pgfpathlineto{\pgfqpoint{0.944318in}{0.664303in}}%
\pgfpathlineto{\pgfqpoint{0.949206in}{0.677230in}}%
\pgfpathlineto{\pgfqpoint{0.954936in}{0.678191in}}%
\pgfpathlineto{\pgfqpoint{0.968205in}{0.678546in}}%
\pgfpathlineto{\pgfqpoint{0.981473in}{0.678407in}}%
\pgfpathlineto{\pgfqpoint{0.994731in}{0.677230in}}%
\pgfpathlineto{\pgfqpoint{0.994742in}{0.677227in}}%
\pgfpathlineto{\pgfqpoint{1.001856in}{0.664303in}}%
\pgfpathlineto{\pgfqpoint{1.002385in}{0.651377in}}%
\pgfpathlineto{\pgfqpoint{1.002604in}{0.638450in}}%
\pgfpathlineto{\pgfqpoint{1.002794in}{0.625523in}}%
\pgfpathlineto{\pgfqpoint{1.003183in}{0.612597in}}%
\pgfpathlineto{\pgfqpoint{1.008011in}{0.600032in}}%
\pgfpathlineto{\pgfqpoint{1.009358in}{0.599670in}}%
\pgfpathlineto{\pgfqpoint{1.021280in}{0.598596in}}%
\pgfpathlineto{\pgfqpoint{1.034549in}{0.598437in}}%
\pgfpathlineto{\pgfqpoint{1.047818in}{0.598573in}}%
\pgfpathlineto{\pgfqpoint{1.057523in}{0.599670in}}%
\pgfpathlineto{\pgfqpoint{1.061087in}{0.604814in}}%
\pgfpathlineto{\pgfqpoint{1.061693in}{0.612597in}}%
\pgfpathlineto{\pgfqpoint{1.061960in}{0.625523in}}%
\pgfpathlineto{\pgfqpoint{1.062132in}{0.638450in}}%
\pgfpathlineto{\pgfqpoint{1.062392in}{0.651377in}}%
\pgfpathlineto{\pgfqpoint{1.063095in}{0.664303in}}%
\pgfpathlineto{\pgfqpoint{1.073577in}{0.677230in}}%
\pgfpathlineto{\pgfqpoint{1.074356in}{0.677350in}}%
\pgfpathlineto{\pgfqpoint{1.087625in}{0.678010in}}%
\pgfpathlineto{\pgfqpoint{1.100894in}{0.677818in}}%
\pgfpathlineto{\pgfqpoint{1.107160in}{0.677230in}}%
\pgfpathlineto{\pgfqpoint{1.114163in}{0.675542in}}%
\pgfpathlineto{\pgfqpoint{1.119305in}{0.664303in}}%
\pgfpathlineto{\pgfqpoint{1.120034in}{0.651377in}}%
\pgfpathlineto{\pgfqpoint{1.120292in}{0.638450in}}%
\pgfpathlineto{\pgfqpoint{1.120442in}{0.625523in}}%
\pgfpathlineto{\pgfqpoint{1.120626in}{0.612597in}}%
\pgfpathlineto{\pgfqpoint{1.123014in}{0.599670in}}%
\pgfpathlineto{\pgfqpoint{1.127432in}{0.598600in}}%
\pgfpathlineto{\pgfqpoint{1.140701in}{0.598219in}}%
\pgfpathlineto{\pgfqpoint{1.153970in}{0.598191in}}%
\pgfpathlineto{\pgfqpoint{1.167239in}{0.598330in}}%
\pgfpathlineto{\pgfqpoint{1.177175in}{0.599670in}}%
\pgfpathlineto{\pgfqpoint{1.179680in}{0.612597in}}%
\pgfpathlineto{\pgfqpoint{1.179872in}{0.625523in}}%
\pgfpathlineto{\pgfqpoint{1.180030in}{0.638450in}}%
\pgfpathlineto{\pgfqpoint{1.180301in}{0.651377in}}%
\pgfpathlineto{\pgfqpoint{1.180508in}{0.656214in}}%
\pgfpathlineto{\pgfqpoint{1.181084in}{0.664303in}}%
\pgfpathlineto{\pgfqpoint{1.192484in}{0.677230in}}%
\pgfpathlineto{\pgfqpoint{1.193777in}{0.677414in}}%
\pgfpathlineto{\pgfqpoint{1.207045in}{0.678012in}}%
\pgfpathlineto{\pgfqpoint{1.220314in}{0.677800in}}%
\pgfpathlineto{\pgfqpoint{1.226043in}{0.677230in}}%
\pgfpathlineto{\pgfqpoint{1.233583in}{0.674899in}}%
\pgfpathlineto{\pgfqpoint{1.237437in}{0.664303in}}%
\pgfpathlineto{\pgfqpoint{1.238107in}{0.651377in}}%
\pgfpathlineto{\pgfqpoint{1.238355in}{0.638450in}}%
\pgfpathlineto{\pgfqpoint{1.238519in}{0.625523in}}%
\pgfpathlineto{\pgfqpoint{1.238774in}{0.612597in}}%
\pgfpathlineto{\pgfqpoint{1.242655in}{0.599670in}}%
\pgfpathlineto{\pgfqpoint{1.246852in}{0.598868in}}%
\pgfpathlineto{\pgfqpoint{1.260121in}{0.598452in}}%
\pgfpathlineto{\pgfqpoint{1.273390in}{0.598487in}}%
\pgfpathlineto{\pgfqpoint{1.286659in}{0.598979in}}%
\pgfpathlineto{\pgfqpoint{1.290940in}{0.599670in}}%
\pgfpathlineto{\pgfqpoint{1.297001in}{0.612597in}}%
\pgfpathlineto{\pgfqpoint{1.297398in}{0.625523in}}%
\pgfpathlineto{\pgfqpoint{1.297593in}{0.638450in}}%
\pgfpathlineto{\pgfqpoint{1.297818in}{0.651377in}}%
\pgfpathlineto{\pgfqpoint{1.298360in}{0.664303in}}%
\pgfpathlineto{\pgfqpoint{1.299928in}{0.672688in}}%
\pgfpathlineto{\pgfqpoint{1.306166in}{0.677230in}}%
\pgfpathlineto{\pgfqpoint{1.313197in}{0.678167in}}%
\pgfpathlineto{\pgfqpoint{1.326466in}{0.678535in}}%
\pgfpathlineto{\pgfqpoint{1.339735in}{0.678444in}}%
\pgfpathlineto{\pgfqpoint{1.351200in}{0.677230in}}%
\pgfpathlineto{\pgfqpoint{1.353004in}{0.676457in}}%
\pgfpathlineto{\pgfqpoint{1.356255in}{0.664303in}}%
\pgfpathlineto{\pgfqpoint{1.356598in}{0.651377in}}%
\pgfpathlineto{\pgfqpoint{1.356783in}{0.638450in}}%
\pgfpathlineto{\pgfqpoint{1.357019in}{0.625523in}}%
\pgfpathlineto{\pgfqpoint{1.357632in}{0.612597in}}%
\pgfpathlineto{\pgfqpoint{1.366273in}{0.600119in}}%
\pgfpathlineto{\pgfqpoint{1.370878in}{0.599670in}}%
\pgfpathlineto{\pgfqpoint{1.379542in}{0.599231in}}%
\pgfpathlineto{\pgfqpoint{1.392811in}{0.599363in}}%
\pgfpathlineto{\pgfqpoint{1.397135in}{0.599670in}}%
\pgfpathlineto{\pgfqpoint{1.406080in}{0.600922in}}%
\pgfpathlineto{\pgfqpoint{1.413698in}{0.612597in}}%
\pgfpathlineto{\pgfqpoint{1.414562in}{0.625523in}}%
\pgfpathlineto{\pgfqpoint{1.414847in}{0.638450in}}%
\pgfpathlineto{\pgfqpoint{1.414979in}{0.651377in}}%
\pgfpathlineto{\pgfqpoint{1.415052in}{0.664303in}}%
\pgfpathlineto{\pgfqpoint{1.415249in}{0.677230in}}%
\pgfpathlineto{\pgfqpoint{1.406080in}{0.679526in}}%
\pgfpathlineto{\pgfqpoint{1.392811in}{0.679541in}}%
\pgfpathlineto{\pgfqpoint{1.379542in}{0.679667in}}%
\pgfpathlineto{\pgfqpoint{1.366273in}{0.680257in}}%
\pgfpathlineto{\pgfqpoint{1.357918in}{0.690156in}}%
\pgfpathlineto{\pgfqpoint{1.357496in}{0.703083in}}%
\pgfpathlineto{\pgfqpoint{1.357485in}{0.716009in}}%
\pgfpathlineto{\pgfqpoint{1.357691in}{0.728936in}}%
\pgfpathlineto{\pgfqpoint{1.358361in}{0.741862in}}%
\pgfpathlineto{\pgfqpoint{1.362427in}{0.754789in}}%
\pgfpathlineto{\pgfqpoint{1.366273in}{0.757050in}}%
\pgfpathlineto{\pgfqpoint{1.379542in}{0.758758in}}%
\pgfpathlineto{\pgfqpoint{1.392811in}{0.758869in}}%
\pgfpathlineto{\pgfqpoint{1.406080in}{0.757642in}}%
\pgfpathlineto{\pgfqpoint{1.411051in}{0.754789in}}%
\pgfpathlineto{\pgfqpoint{1.414201in}{0.741862in}}%
\pgfpathlineto{\pgfqpoint{1.414713in}{0.728936in}}%
\pgfpathlineto{\pgfqpoint{1.414901in}{0.716009in}}%
\pgfpathlineto{\pgfqpoint{1.414982in}{0.703083in}}%
\pgfpathlineto{\pgfqpoint{1.414997in}{0.690156in}}%
\pgfpathlineto{\pgfqpoint{1.419348in}{0.679366in}}%
\pgfpathlineto{\pgfqpoint{1.432617in}{0.679476in}}%
\pgfpathlineto{\pgfqpoint{1.445886in}{0.679578in}}%
\pgfpathlineto{\pgfqpoint{1.459155in}{0.679889in}}%
\pgfpathlineto{\pgfqpoint{1.472424in}{0.684182in}}%
\pgfpathlineto{\pgfqpoint{1.473745in}{0.690156in}}%
\pgfpathlineto{\pgfqpoint{1.474328in}{0.703083in}}%
\pgfpathlineto{\pgfqpoint{1.474527in}{0.716009in}}%
\pgfpathlineto{\pgfqpoint{1.474689in}{0.728936in}}%
\pgfpathlineto{\pgfqpoint{1.474977in}{0.741862in}}%
\pgfpathlineto{\pgfqpoint{1.476474in}{0.754789in}}%
\pgfpathlineto{\pgfqpoint{1.485693in}{0.759564in}}%
\pgfpathlineto{\pgfqpoint{1.498962in}{0.760108in}}%
\pgfpathlineto{\pgfqpoint{1.512231in}{0.760274in}}%
\pgfpathlineto{\pgfqpoint{1.525500in}{0.760470in}}%
\pgfpathlineto{\pgfqpoint{1.525500in}{0.767715in}}%
\pgfpathlineto{\pgfqpoint{1.525500in}{0.780642in}}%
\pgfpathlineto{\pgfqpoint{1.525500in}{0.780876in}}%
\pgfpathlineto{\pgfqpoint{1.525425in}{0.780642in}}%
\pgfpathlineto{\pgfqpoint{1.512231in}{0.768993in}}%
\pgfpathlineto{\pgfqpoint{1.498962in}{0.768174in}}%
\pgfpathlineto{\pgfqpoint{1.485693in}{0.773948in}}%
\pgfpathlineto{\pgfqpoint{1.482043in}{0.780642in}}%
\pgfpathlineto{\pgfqpoint{1.479894in}{0.793568in}}%
\pgfpathlineto{\pgfqpoint{1.479656in}{0.806495in}}%
\pgfpathlineto{\pgfqpoint{1.480997in}{0.819421in}}%
\pgfpathlineto{\pgfqpoint{1.485693in}{0.829742in}}%
\pgfpathlineto{\pgfqpoint{1.490109in}{0.832348in}}%
\pgfpathlineto{\pgfqpoint{1.498962in}{0.834831in}}%
\pgfpathlineto{\pgfqpoint{1.512231in}{0.834160in}}%
\pgfpathlineto{\pgfqpoint{1.516719in}{0.832348in}}%
\pgfpathlineto{\pgfqpoint{1.525500in}{0.822599in}}%
\pgfpathlineto{\pgfqpoint{1.525500in}{0.832348in}}%
\pgfpathlineto{\pgfqpoint{1.525500in}{0.844249in}}%
\pgfpathlineto{\pgfqpoint{1.512231in}{0.843001in}}%
\pgfpathlineto{\pgfqpoint{1.498962in}{0.843097in}}%
\pgfpathlineto{\pgfqpoint{1.485693in}{0.844463in}}%
\pgfpathlineto{\pgfqpoint{1.483410in}{0.845274in}}%
\pgfpathlineto{\pgfqpoint{1.476874in}{0.858201in}}%
\pgfpathlineto{\pgfqpoint{1.476188in}{0.871127in}}%
\pgfpathlineto{\pgfqpoint{1.476135in}{0.884054in}}%
\pgfpathlineto{\pgfqpoint{1.476531in}{0.896980in}}%
\pgfpathlineto{\pgfqpoint{1.478172in}{0.909907in}}%
\pgfpathlineto{\pgfqpoint{1.485693in}{0.918706in}}%
\pgfpathlineto{\pgfqpoint{1.498962in}{0.920614in}}%
\pgfpathlineto{\pgfqpoint{1.512231in}{0.920656in}}%
\pgfpathlineto{\pgfqpoint{1.525500in}{0.918428in}}%
\pgfpathlineto{\pgfqpoint{1.525500in}{0.922833in}}%
\pgfpathlineto{\pgfqpoint{1.525500in}{0.935760in}}%
\pgfpathlineto{\pgfqpoint{1.525500in}{0.938522in}}%
\pgfpathlineto{\pgfqpoint{1.523731in}{0.935760in}}%
\pgfpathlineto{\pgfqpoint{1.512231in}{0.929528in}}%
\pgfpathlineto{\pgfqpoint{1.498962in}{0.928920in}}%
\pgfpathlineto{\pgfqpoint{1.485693in}{0.932825in}}%
\pgfpathlineto{\pgfqpoint{1.483059in}{0.935760in}}%
\pgfpathlineto{\pgfqpoint{1.479250in}{0.948686in}}%
\pgfpathlineto{\pgfqpoint{1.478433in}{0.961613in}}%
\pgfpathlineto{\pgfqpoint{1.478742in}{0.974539in}}%
\pgfpathlineto{\pgfqpoint{1.480850in}{0.987466in}}%
\pgfpathlineto{\pgfqpoint{1.485693in}{0.994805in}}%
\pgfpathlineto{\pgfqpoint{1.498962in}{0.998649in}}%
\pgfpathlineto{\pgfqpoint{1.512231in}{0.998035in}}%
\pgfpathlineto{\pgfqpoint{1.525500in}{0.989633in}}%
\pgfpathlineto{\pgfqpoint{1.525500in}{1.000392in}}%
\pgfpathlineto{\pgfqpoint{1.525500in}{1.009964in}}%
\pgfpathlineto{\pgfqpoint{1.512231in}{1.006977in}}%
\pgfpathlineto{\pgfqpoint{1.498962in}{1.006979in}}%
\pgfpathlineto{\pgfqpoint{1.485693in}{1.009288in}}%
\pgfpathlineto{\pgfqpoint{1.480941in}{1.013319in}}%
\pgfpathlineto{\pgfqpoint{1.477583in}{1.026245in}}%
\pgfpathlineto{\pgfqpoint{1.476933in}{1.039172in}}%
\pgfpathlineto{\pgfqpoint{1.477007in}{1.052098in}}%
\pgfpathlineto{\pgfqpoint{1.477912in}{1.065025in}}%
\pgfpathlineto{\pgfqpoint{1.482910in}{1.077952in}}%
\pgfpathlineto{\pgfqpoint{1.485693in}{1.079848in}}%
\pgfpathlineto{\pgfqpoint{1.498962in}{1.082295in}}%
\pgfpathlineto{\pgfqpoint{1.512231in}{1.082280in}}%
\pgfpathlineto{\pgfqpoint{1.525500in}{1.079033in}}%
\pgfpathlineto{\pgfqpoint{1.525500in}{1.090878in}}%
\pgfpathlineto{\pgfqpoint{1.525500in}{1.099039in}}%
\pgfpathlineto{\pgfqpoint{1.512231in}{1.090955in}}%
\pgfpathlineto{\pgfqpoint{1.510639in}{1.090878in}}%
\pgfpathlineto{\pgfqpoint{1.498962in}{1.090405in}}%
\pgfpathlineto{\pgfqpoint{1.496008in}{1.090878in}}%
\pgfpathlineto{\pgfqpoint{1.485693in}{1.094030in}}%
\pgfpathlineto{\pgfqpoint{1.479895in}{1.103805in}}%
\pgfpathlineto{\pgfqpoint{1.478258in}{1.116731in}}%
\pgfpathlineto{\pgfqpoint{1.478056in}{1.129658in}}%
\pgfpathlineto{\pgfqpoint{1.478904in}{1.142584in}}%
\pgfpathlineto{\pgfqpoint{1.483074in}{1.155511in}}%
\pgfpathlineto{\pgfqpoint{1.485693in}{1.158033in}}%
\pgfpathlineto{\pgfqpoint{1.498962in}{1.161377in}}%
\pgfpathlineto{\pgfqpoint{1.512231in}{1.160827in}}%
\pgfpathlineto{\pgfqpoint{1.523547in}{1.155511in}}%
\pgfpathlineto{\pgfqpoint{1.525500in}{1.152947in}}%
\pgfpathlineto{\pgfqpoint{1.525500in}{1.155511in}}%
\pgfpathlineto{\pgfqpoint{1.525500in}{1.168437in}}%
\pgfpathlineto{\pgfqpoint{1.525500in}{1.173091in}}%
\pgfpathlineto{\pgfqpoint{1.512231in}{1.169673in}}%
\pgfpathlineto{\pgfqpoint{1.498962in}{1.169657in}}%
\pgfpathlineto{\pgfqpoint{1.485693in}{1.172234in}}%
\pgfpathlineto{\pgfqpoint{1.478950in}{1.181364in}}%
\pgfpathlineto{\pgfqpoint{1.477248in}{1.194290in}}%
\pgfpathlineto{\pgfqpoint{1.476892in}{1.207217in}}%
\pgfpathlineto{\pgfqpoint{1.477181in}{1.220143in}}%
\pgfpathlineto{\pgfqpoint{1.478739in}{1.233070in}}%
\pgfpathlineto{\pgfqpoint{1.485693in}{1.242559in}}%
\pgfpathlineto{\pgfqpoint{1.498962in}{1.244978in}}%
\pgfpathlineto{\pgfqpoint{1.512231in}{1.244980in}}%
\pgfpathlineto{\pgfqpoint{1.525500in}{1.241850in}}%
\pgfpathlineto{\pgfqpoint{1.525500in}{1.245996in}}%
\pgfpathlineto{\pgfqpoint{1.525500in}{1.258923in}}%
\pgfpathlineto{\pgfqpoint{1.525500in}{1.262528in}}%
\pgfpathlineto{\pgfqpoint{1.522826in}{1.258923in}}%
\pgfpathlineto{\pgfqpoint{1.512231in}{1.253884in}}%
\pgfpathlineto{\pgfqpoint{1.498962in}{1.253308in}}%
\pgfpathlineto{\pgfqpoint{1.485693in}{1.256913in}}%
\pgfpathlineto{\pgfqpoint{1.483641in}{1.258923in}}%
\pgfpathlineto{\pgfqpoint{1.479288in}{1.271849in}}%
\pgfpathlineto{\pgfqpoint{1.478435in}{1.284776in}}%
\pgfpathlineto{\pgfqpoint{1.478730in}{1.297702in}}%
\pgfpathlineto{\pgfqpoint{1.480681in}{1.310629in}}%
\pgfpathlineto{\pgfqpoint{1.485693in}{1.318943in}}%
\pgfpathlineto{\pgfqpoint{1.498962in}{1.323169in}}%
\pgfpathlineto{\pgfqpoint{1.512231in}{1.322511in}}%
\pgfpathlineto{\pgfqpoint{1.525500in}{1.313498in}}%
\pgfpathlineto{\pgfqpoint{1.525500in}{1.323555in}}%
\pgfpathlineto{\pgfqpoint{1.525500in}{1.333388in}}%
\pgfpathlineto{\pgfqpoint{1.512231in}{1.331227in}}%
\pgfpathlineto{\pgfqpoint{1.498962in}{1.331267in}}%
\pgfpathlineto{\pgfqpoint{1.485693in}{1.333117in}}%
\pgfpathlineto{\pgfqpoint{1.480735in}{1.336482in}}%
\pgfpathlineto{\pgfqpoint{1.476954in}{1.349408in}}%
\pgfpathlineto{\pgfqpoint{1.476237in}{1.362335in}}%
\pgfpathlineto{\pgfqpoint{1.476115in}{1.375261in}}%
\pgfpathlineto{\pgfqpoint{1.476449in}{1.388188in}}%
\pgfpathlineto{\pgfqpoint{1.478356in}{1.401114in}}%
\pgfpathlineto{\pgfqpoint{1.485693in}{1.407372in}}%
\pgfpathlineto{\pgfqpoint{1.498962in}{1.408737in}}%
\pgfpathlineto{\pgfqpoint{1.512231in}{1.408833in}}%
\pgfpathlineto{\pgfqpoint{1.525500in}{1.407585in}}%
\pgfpathlineto{\pgfqpoint{1.525500in}{1.414041in}}%
\pgfpathlineto{\pgfqpoint{1.525500in}{1.426967in}}%
\pgfpathlineto{\pgfqpoint{1.525500in}{1.429323in}}%
\pgfpathlineto{\pgfqpoint{1.524549in}{1.426967in}}%
\pgfpathlineto{\pgfqpoint{1.512231in}{1.417954in}}%
\pgfpathlineto{\pgfqpoint{1.498962in}{1.417230in}}%
\pgfpathlineto{\pgfqpoint{1.485693in}{1.422220in}}%
\pgfpathlineto{\pgfqpoint{1.482611in}{1.426967in}}%
\pgfpathlineto{\pgfqpoint{1.479976in}{1.439894in}}%
\pgfpathlineto{\pgfqpoint{1.479628in}{1.452820in}}%
\pgfpathlineto{\pgfqpoint{1.480754in}{1.465747in}}%
\pgfpathlineto{\pgfqpoint{1.485693in}{1.478385in}}%
\pgfpathlineto{\pgfqpoint{1.486034in}{1.478673in}}%
\pgfpathlineto{\pgfqpoint{1.498962in}{1.483598in}}%
\pgfpathlineto{\pgfqpoint{1.512231in}{1.482869in}}%
\pgfpathlineto{\pgfqpoint{1.520065in}{1.478673in}}%
\pgfpathlineto{\pgfqpoint{1.525500in}{1.470407in}}%
\pgfpathlineto{\pgfqpoint{1.525500in}{1.478673in}}%
\pgfpathlineto{\pgfqpoint{1.525500in}{1.491471in}}%
\pgfpathlineto{\pgfqpoint{1.519501in}{1.491600in}}%
\pgfpathlineto{\pgfqpoint{1.512231in}{1.491600in}}%
\pgfpathlineto{\pgfqpoint{1.498962in}{1.491600in}}%
\pgfpathlineto{\pgfqpoint{1.485693in}{1.491600in}}%
\pgfpathlineto{\pgfqpoint{1.472424in}{1.491600in}}%
\pgfpathlineto{\pgfqpoint{1.459155in}{1.491600in}}%
\pgfpathlineto{\pgfqpoint{1.445886in}{1.491600in}}%
\pgfpathlineto{\pgfqpoint{1.432617in}{1.491600in}}%
\pgfpathlineto{\pgfqpoint{1.419348in}{1.491600in}}%
\pgfpathlineto{\pgfqpoint{1.406080in}{1.491600in}}%
\pgfpathlineto{\pgfqpoint{1.392811in}{1.491600in}}%
\pgfpathlineto{\pgfqpoint{1.379542in}{1.491600in}}%
\pgfpathlineto{\pgfqpoint{1.366273in}{1.491600in}}%
\pgfpathlineto{\pgfqpoint{1.353004in}{1.491600in}}%
\pgfpathlineto{\pgfqpoint{1.339735in}{1.491600in}}%
\pgfpathlineto{\pgfqpoint{1.326466in}{1.491600in}}%
\pgfpathlineto{\pgfqpoint{1.313197in}{1.491600in}}%
\pgfpathlineto{\pgfqpoint{1.299928in}{1.491600in}}%
\pgfpathlineto{\pgfqpoint{1.286659in}{1.491600in}}%
\pgfpathlineto{\pgfqpoint{1.273390in}{1.491600in}}%
\pgfpathlineto{\pgfqpoint{1.260121in}{1.491600in}}%
\pgfpathlineto{\pgfqpoint{1.246852in}{1.491600in}}%
\pgfpathlineto{\pgfqpoint{1.233583in}{1.491600in}}%
\pgfpathlineto{\pgfqpoint{1.220314in}{1.491600in}}%
\pgfpathlineto{\pgfqpoint{1.207045in}{1.491600in}}%
\pgfpathlineto{\pgfqpoint{1.193777in}{1.491600in}}%
\pgfpathlineto{\pgfqpoint{1.180508in}{1.491600in}}%
\pgfpathlineto{\pgfqpoint{1.167239in}{1.491600in}}%
\pgfpathlineto{\pgfqpoint{1.153970in}{1.491600in}}%
\pgfpathlineto{\pgfqpoint{1.140701in}{1.491600in}}%
\pgfpathlineto{\pgfqpoint{1.127432in}{1.491600in}}%
\pgfpathlineto{\pgfqpoint{1.114163in}{1.491600in}}%
\pgfpathlineto{\pgfqpoint{1.100894in}{1.491600in}}%
\pgfpathlineto{\pgfqpoint{1.087625in}{1.491600in}}%
\pgfpathlineto{\pgfqpoint{1.074356in}{1.491600in}}%
\pgfpathlineto{\pgfqpoint{1.061087in}{1.491600in}}%
\pgfpathlineto{\pgfqpoint{1.047818in}{1.491600in}}%
\pgfpathlineto{\pgfqpoint{1.034549in}{1.491600in}}%
\pgfpathlineto{\pgfqpoint{1.021280in}{1.491600in}}%
\pgfpathlineto{\pgfqpoint{1.008011in}{1.491600in}}%
\pgfpathlineto{\pgfqpoint{0.994742in}{1.491600in}}%
\pgfpathlineto{\pgfqpoint{0.981473in}{1.491600in}}%
\pgfpathlineto{\pgfqpoint{0.968205in}{1.491600in}}%
\pgfpathlineto{\pgfqpoint{0.954936in}{1.491600in}}%
\pgfpathlineto{\pgfqpoint{0.941667in}{1.491600in}}%
\pgfpathlineto{\pgfqpoint{0.928398in}{1.491600in}}%
\pgfpathlineto{\pgfqpoint{0.915129in}{1.491600in}}%
\pgfpathlineto{\pgfqpoint{0.901860in}{1.491600in}}%
\pgfpathlineto{\pgfqpoint{0.888591in}{1.491600in}}%
\pgfpathlineto{\pgfqpoint{0.875322in}{1.491600in}}%
\pgfpathlineto{\pgfqpoint{0.862053in}{1.491600in}}%
\pgfpathlineto{\pgfqpoint{0.848784in}{1.491600in}}%
\pgfpathlineto{\pgfqpoint{0.835515in}{1.491600in}}%
\pgfpathlineto{\pgfqpoint{0.822246in}{1.491600in}}%
\pgfpathlineto{\pgfqpoint{0.808977in}{1.491600in}}%
\pgfpathlineto{\pgfqpoint{0.795708in}{1.491600in}}%
\pgfpathlineto{\pgfqpoint{0.782439in}{1.491600in}}%
\pgfpathlineto{\pgfqpoint{0.781268in}{1.491600in}}%
\pgfpathlineto{\pgfqpoint{0.769170in}{1.490856in}}%
\pgfpathlineto{\pgfqpoint{0.766757in}{1.478673in}}%
\pgfpathlineto{\pgfqpoint{0.766647in}{1.465747in}}%
\pgfpathlineto{\pgfqpoint{0.766555in}{1.452820in}}%
\pgfpathlineto{\pgfqpoint{0.766394in}{1.439894in}}%
\pgfpathlineto{\pgfqpoint{0.765967in}{1.426967in}}%
\pgfpathlineto{\pgfqpoint{0.762646in}{1.414041in}}%
\pgfpathlineto{\pgfqpoint{0.755902in}{1.411553in}}%
\pgfpathlineto{\pgfqpoint{0.742633in}{1.410785in}}%
\pgfpathlineto{\pgfqpoint{0.729364in}{1.410597in}}%
\pgfpathlineto{\pgfqpoint{0.716095in}{1.410555in}}%
\pgfpathlineto{\pgfqpoint{0.707948in}{1.414041in}}%
\pgfpathlineto{\pgfqpoint{0.707858in}{1.426967in}}%
\pgfpathlineto{\pgfqpoint{0.707804in}{1.439894in}}%
\pgfpathlineto{\pgfqpoint{0.707707in}{1.452820in}}%
\pgfpathlineto{\pgfqpoint{0.707487in}{1.465747in}}%
\pgfpathlineto{\pgfqpoint{0.706750in}{1.478673in}}%
\pgfpathlineto{\pgfqpoint{0.702826in}{1.487531in}}%
\pgfpathlineto{\pgfqpoint{0.689557in}{1.489859in}}%
\pgfpathlineto{\pgfqpoint{0.676288in}{1.489879in}}%
\pgfpathlineto{\pgfqpoint{0.663019in}{1.488803in}}%
\pgfpathlineto{\pgfqpoint{0.652007in}{1.478673in}}%
\pgfpathlineto{\pgfqpoint{0.650557in}{1.465747in}}%
\pgfpathlineto{\pgfqpoint{0.650179in}{1.452820in}}%
\pgfpathlineto{\pgfqpoint{0.650151in}{1.439894in}}%
\pgfpathlineto{\pgfqpoint{0.650534in}{1.426967in}}%
\pgfpathlineto{\pgfqpoint{0.654194in}{1.414041in}}%
\pgfpathlineto{\pgfqpoint{0.663019in}{1.411342in}}%
\pgfpathlineto{\pgfqpoint{0.676288in}{1.410733in}}%
\pgfpathlineto{\pgfqpoint{0.689557in}{1.410555in}}%
\pgfpathlineto{\pgfqpoint{0.702826in}{1.410454in}}%
\pgfpathlineto{\pgfqpoint{0.707825in}{1.401114in}}%
\pgfpathlineto{\pgfqpoint{0.707826in}{1.388188in}}%
\pgfpathlineto{\pgfqpoint{0.707785in}{1.375261in}}%
\pgfpathlineto{\pgfqpoint{0.707688in}{1.362335in}}%
\pgfpathlineto{\pgfqpoint{0.707435in}{1.349408in}}%
\pgfpathlineto{\pgfqpoint{0.706252in}{1.336482in}}%
\pgfpathlineto{\pgfqpoint{0.702826in}{1.331562in}}%
\pgfpathlineto{\pgfqpoint{0.689557in}{1.329620in}}%
\pgfpathlineto{\pgfqpoint{0.676288in}{1.329298in}}%
\pgfpathlineto{\pgfqpoint{0.663019in}{1.329069in}}%
\pgfpathlineto{\pgfqpoint{0.649750in}{1.323997in}}%
\pgfpathlineto{\pgfqpoint{0.649682in}{1.323555in}}%
\pgfpathlineto{\pgfqpoint{0.649122in}{1.310629in}}%
\pgfpathlineto{\pgfqpoint{0.649015in}{1.297702in}}%
\pgfpathlineto{\pgfqpoint{0.648950in}{1.284776in}}%
\pgfpathlineto{\pgfqpoint{0.648864in}{1.271849in}}%
\pgfpathlineto{\pgfqpoint{0.648611in}{1.258923in}}%
\pgfpathlineto{\pgfqpoint{0.636481in}{1.247913in}}%
\pgfpathlineto{\pgfqpoint{0.623212in}{1.247625in}}%
\pgfpathlineto{\pgfqpoint{0.609943in}{1.247292in}}%
\pgfpathlineto{\pgfqpoint{0.597899in}{1.245996in}}%
\pgfpathlineto{\pgfqpoint{0.596674in}{1.245652in}}%
\pgfpathlineto{\pgfqpoint{0.591181in}{1.233070in}}%
\pgfpathlineto{\pgfqpoint{0.590730in}{1.220143in}}%
\pgfpathlineto{\pgfqpoint{0.590622in}{1.207217in}}%
\pgfpathlineto{\pgfqpoint{0.590662in}{1.194290in}}%
\pgfpathlineto{\pgfqpoint{0.590961in}{1.181364in}}%
\pgfpathlineto{\pgfqpoint{0.595391in}{1.168437in}}%
\pgfpathlineto{\pgfqpoint{0.596674in}{1.167984in}}%
\pgfpathlineto{\pgfqpoint{0.609943in}{1.166786in}}%
\pgfpathlineto{\pgfqpoint{0.623212in}{1.166484in}}%
\pgfpathlineto{\pgfqpoint{0.636481in}{1.166039in}}%
\pgfpathlineto{\pgfqpoint{0.648019in}{1.155511in}}%
\pgfpathlineto{\pgfqpoint{0.648475in}{1.142584in}}%
\pgfpathlineto{\pgfqpoint{0.648571in}{1.129658in}}%
\pgfpathlineto{\pgfqpoint{0.648548in}{1.116731in}}%
\pgfpathlineto{\pgfqpoint{0.648364in}{1.103805in}}%
\pgfpathlineto{\pgfqpoint{0.647159in}{1.090878in}}%
\pgfpathlineto{\pgfqpoint{0.636481in}{1.085814in}}%
\pgfpathlineto{\pgfqpoint{0.623212in}{1.085376in}}%
\pgfpathlineto{\pgfqpoint{0.609943in}{1.085079in}}%
\pgfpathlineto{\pgfqpoint{0.596674in}{1.083897in}}%
\pgfpathlineto{\pgfqpoint{0.591674in}{1.077952in}}%
\pgfpathlineto{\pgfqpoint{0.590776in}{1.065025in}}%
\pgfpathlineto{\pgfqpoint{0.590627in}{1.052098in}}%
\pgfpathlineto{\pgfqpoint{0.590648in}{1.039172in}}%
\pgfpathlineto{\pgfqpoint{0.590850in}{1.026245in}}%
\pgfpathlineto{\pgfqpoint{0.591799in}{1.013319in}}%
\pgfpathlineto{\pgfqpoint{0.596674in}{1.006337in}}%
\pgfpathlineto{\pgfqpoint{0.609943in}{1.004725in}}%
\pgfpathlineto{\pgfqpoint{0.623212in}{1.004396in}}%
\pgfpathlineto{\pgfqpoint{0.636481in}{1.004111in}}%
\pgfpathlineto{\pgfqpoint{0.647748in}{1.000392in}}%
\pgfpathlineto{\pgfqpoint{0.648766in}{0.987466in}}%
\pgfpathlineto{\pgfqpoint{0.648908in}{0.974539in}}%
\pgfpathlineto{\pgfqpoint{0.648977in}{0.961613in}}%
\pgfpathlineto{\pgfqpoint{0.649050in}{0.948686in}}%
\pgfpathlineto{\pgfqpoint{0.649227in}{0.935760in}}%
\pgfpathlineto{\pgfqpoint{0.649750in}{0.928093in}}%
\pgfpathlineto{\pgfqpoint{0.663019in}{0.922881in}}%
\pgfpathlineto{\pgfqpoint{0.664979in}{0.922833in}}%
\pgfpathlineto{\pgfqpoint{0.676288in}{0.922645in}}%
\pgfpathlineto{\pgfqpoint{0.689557in}{0.922313in}}%
\pgfpathlineto{\pgfqpoint{0.702826in}{0.920310in}}%
\pgfpathlineto{\pgfqpoint{0.707048in}{0.909907in}}%
\pgfpathlineto{\pgfqpoint{0.707577in}{0.896980in}}%
\pgfpathlineto{\pgfqpoint{0.707739in}{0.884054in}}%
\pgfpathlineto{\pgfqpoint{0.707807in}{0.871127in}}%
\pgfpathlineto{\pgfqpoint{0.707832in}{0.858201in}}%
\pgfpathlineto{\pgfqpoint{0.707780in}{0.845274in}}%
\pgfpathlineto{\pgfqpoint{0.716095in}{0.841281in}}%
\pgfpathlineto{\pgfqpoint{0.729364in}{0.841238in}}%
\pgfpathlineto{\pgfqpoint{0.742633in}{0.841050in}}%
\pgfpathlineto{\pgfqpoint{0.755902in}{0.840282in}}%
\pgfpathlineto{\pgfqpoint{0.765175in}{0.832348in}}%
\pgfpathlineto{\pgfqpoint{0.766213in}{0.819421in}}%
\pgfpathlineto{\pgfqpoint{0.766478in}{0.806495in}}%
\pgfpathlineto{\pgfqpoint{0.766598in}{0.793568in}}%
\pgfpathlineto{\pgfqpoint{0.766685in}{0.780642in}}%
\pgfpathlineto{\pgfqpoint{0.766885in}{0.767715in}}%
\pgfpathlineto{\pgfqpoint{0.769170in}{0.761072in}}%
\pgfpathlineto{\pgfqpoint{0.782439in}{0.760332in}}%
\pgfpathlineto{\pgfqpoint{0.795708in}{0.760194in}}%
\pgfpathlineto{\pgfqpoint{0.808977in}{0.759904in}}%
\pgfpathlineto{\pgfqpoint{0.822246in}{0.757509in}}%
\pgfpathlineto{\pgfqpoint{0.824111in}{0.754789in}}%
\pgfpathlineto{\pgfqpoint{0.825588in}{0.741862in}}%
\pgfpathlineto{\pgfqpoint{0.825872in}{0.728936in}}%
\pgfpathlineto{\pgfqpoint{0.826031in}{0.716009in}}%
\pgfpathlineto{\pgfqpoint{0.826226in}{0.703083in}}%
\pgfpathlineto{\pgfqpoint{0.826798in}{0.690156in}}%
\pgfpathlineto{\pgfqpoint{0.835515in}{0.680306in}}%
\pgfpathlineto{\pgfqpoint{0.848784in}{0.679664in}}%
\pgfpathlineto{\pgfqpoint{0.862053in}{0.679510in}}%
\pgfpathlineto{\pgfqpoint{0.875322in}{0.679434in}}%
\pgfpathlineto{\pgfqpoint{0.884928in}{0.677230in}}%
\pgfpathlineto{\pgfqpoint{0.885111in}{0.664303in}}%
\pgfpathlineto{\pgfqpoint{0.885184in}{0.651377in}}%
\pgfpathlineto{\pgfqpoint{0.885315in}{0.638450in}}%
\pgfpathlineto{\pgfqpoint{0.885601in}{0.625523in}}%
\pgfpathlineto{\pgfqpoint{0.886467in}{0.612597in}}%
\pgfpathlineto{\pgfqpoint{0.888591in}{0.605111in}}%
\pgfpathlineto{\pgfqpoint{0.901860in}{0.599687in}}%
\pgfpathlineto{\pgfqpoint{0.902205in}{0.599670in}}%
\pgfpathclose%
\pgfpathmoveto{\pgfqpoint{0.898562in}{0.612597in}}%
\pgfpathlineto{\pgfqpoint{0.892774in}{0.625523in}}%
\pgfpathlineto{\pgfqpoint{0.891489in}{0.638450in}}%
\pgfpathlineto{\pgfqpoint{0.892235in}{0.651377in}}%
\pgfpathlineto{\pgfqpoint{0.896741in}{0.664303in}}%
\pgfpathlineto{\pgfqpoint{0.901860in}{0.668835in}}%
\pgfpathlineto{\pgfqpoint{0.915129in}{0.671249in}}%
\pgfpathlineto{\pgfqpoint{0.928398in}{0.668390in}}%
\pgfpathlineto{\pgfqpoint{0.932611in}{0.664303in}}%
\pgfpathlineto{\pgfqpoint{0.936806in}{0.651377in}}%
\pgfpathlineto{\pgfqpoint{0.937485in}{0.638450in}}%
\pgfpathlineto{\pgfqpoint{0.936285in}{0.625523in}}%
\pgfpathlineto{\pgfqpoint{0.930833in}{0.612597in}}%
\pgfpathlineto{\pgfqpoint{0.928398in}{0.610416in}}%
\pgfpathlineto{\pgfqpoint{0.915129in}{0.607268in}}%
\pgfpathlineto{\pgfqpoint{0.901860in}{0.609890in}}%
\pgfpathclose%
\pgfpathmoveto{\pgfqpoint{1.015259in}{0.612597in}}%
\pgfpathlineto{\pgfqpoint{1.009674in}{0.625523in}}%
\pgfpathlineto{\pgfqpoint{1.008425in}{0.638450in}}%
\pgfpathlineto{\pgfqpoint{1.009109in}{0.651377in}}%
\pgfpathlineto{\pgfqpoint{1.013353in}{0.664303in}}%
\pgfpathlineto{\pgfqpoint{1.021280in}{0.670527in}}%
\pgfpathlineto{\pgfqpoint{1.034549in}{0.672037in}}%
\pgfpathlineto{\pgfqpoint{1.047818in}{0.668591in}}%
\pgfpathlineto{\pgfqpoint{1.051711in}{0.664303in}}%
\pgfpathlineto{\pgfqpoint{1.055351in}{0.651377in}}%
\pgfpathlineto{\pgfqpoint{1.055930in}{0.638450in}}%
\pgfpathlineto{\pgfqpoint{1.054856in}{0.625523in}}%
\pgfpathlineto{\pgfqpoint{1.050032in}{0.612597in}}%
\pgfpathlineto{\pgfqpoint{1.047818in}{0.610357in}}%
\pgfpathlineto{\pgfqpoint{1.034549in}{0.606523in}}%
\pgfpathlineto{\pgfqpoint{1.021280in}{0.608202in}}%
\pgfpathclose%
\pgfpathmoveto{\pgfqpoint{1.132624in}{0.612597in}}%
\pgfpathlineto{\pgfqpoint{1.127432in}{0.623595in}}%
\pgfpathlineto{\pgfqpoint{1.127010in}{0.625523in}}%
\pgfpathlineto{\pgfqpoint{1.126036in}{0.638450in}}%
\pgfpathlineto{\pgfqpoint{1.126555in}{0.651377in}}%
\pgfpathlineto{\pgfqpoint{1.127432in}{0.656182in}}%
\pgfpathlineto{\pgfqpoint{1.130589in}{0.664303in}}%
\pgfpathlineto{\pgfqpoint{1.140701in}{0.671336in}}%
\pgfpathlineto{\pgfqpoint{1.153970in}{0.672211in}}%
\pgfpathlineto{\pgfqpoint{1.167239in}{0.667815in}}%
\pgfpathlineto{\pgfqpoint{1.170028in}{0.664303in}}%
\pgfpathlineto{\pgfqpoint{1.173456in}{0.651377in}}%
\pgfpathlineto{\pgfqpoint{1.174001in}{0.638450in}}%
\pgfpathlineto{\pgfqpoint{1.172978in}{0.625523in}}%
\pgfpathlineto{\pgfqpoint{1.168412in}{0.612597in}}%
\pgfpathlineto{\pgfqpoint{1.167239in}{0.611247in}}%
\pgfpathlineto{\pgfqpoint{1.153970in}{0.606375in}}%
\pgfpathlineto{\pgfqpoint{1.140701in}{0.607363in}}%
\pgfpathclose%
\pgfpathmoveto{\pgfqpoint{1.250783in}{0.612597in}}%
\pgfpathlineto{\pgfqpoint{1.246852in}{0.619003in}}%
\pgfpathlineto{\pgfqpoint{1.245137in}{0.625523in}}%
\pgfpathlineto{\pgfqpoint{1.244139in}{0.638450in}}%
\pgfpathlineto{\pgfqpoint{1.244677in}{0.651377in}}%
\pgfpathlineto{\pgfqpoint{1.246852in}{0.661009in}}%
\pgfpathlineto{\pgfqpoint{1.248563in}{0.664303in}}%
\pgfpathlineto{\pgfqpoint{1.260121in}{0.671397in}}%
\pgfpathlineto{\pgfqpoint{1.273390in}{0.671748in}}%
\pgfpathlineto{\pgfqpoint{1.286659in}{0.665713in}}%
\pgfpathlineto{\pgfqpoint{1.287628in}{0.664303in}}%
\pgfpathlineto{\pgfqpoint{1.291148in}{0.651377in}}%
\pgfpathlineto{\pgfqpoint{1.291715in}{0.638450in}}%
\pgfpathlineto{\pgfqpoint{1.290680in}{0.625523in}}%
\pgfpathlineto{\pgfqpoint{1.286659in}{0.613699in}}%
\pgfpathlineto{\pgfqpoint{1.285661in}{0.612597in}}%
\pgfpathlineto{\pgfqpoint{1.273390in}{0.606842in}}%
\pgfpathlineto{\pgfqpoint{1.260121in}{0.607249in}}%
\pgfpathclose%
\pgfpathmoveto{\pgfqpoint{1.369950in}{0.612597in}}%
\pgfpathlineto{\pgfqpoint{1.366273in}{0.617289in}}%
\pgfpathlineto{\pgfqpoint{1.363730in}{0.625523in}}%
\pgfpathlineto{\pgfqpoint{1.362646in}{0.638450in}}%
\pgfpathlineto{\pgfqpoint{1.363260in}{0.651377in}}%
\pgfpathlineto{\pgfqpoint{1.366273in}{0.662533in}}%
\pgfpathlineto{\pgfqpoint{1.367469in}{0.664303in}}%
\pgfpathlineto{\pgfqpoint{1.379542in}{0.670797in}}%
\pgfpathlineto{\pgfqpoint{1.392811in}{0.670602in}}%
\pgfpathlineto{\pgfqpoint{1.403782in}{0.664303in}}%
\pgfpathlineto{\pgfqpoint{1.406080in}{0.660571in}}%
\pgfpathlineto{\pgfqpoint{1.408434in}{0.651377in}}%
\pgfpathlineto{\pgfqpoint{1.409077in}{0.638450in}}%
\pgfpathlineto{\pgfqpoint{1.407969in}{0.625523in}}%
\pgfpathlineto{\pgfqpoint{1.406080in}{0.619092in}}%
\pgfpathlineto{\pgfqpoint{1.401429in}{0.612597in}}%
\pgfpathlineto{\pgfqpoint{1.392811in}{0.607972in}}%
\pgfpathlineto{\pgfqpoint{1.379542in}{0.607777in}}%
\pgfpathclose%
\pgfpathmoveto{\pgfqpoint{0.843322in}{0.690156in}}%
\pgfpathlineto{\pgfqpoint{0.835515in}{0.697469in}}%
\pgfpathlineto{\pgfqpoint{0.833449in}{0.703083in}}%
\pgfpathlineto{\pgfqpoint{0.831864in}{0.716009in}}%
\pgfpathlineto{\pgfqpoint{0.832014in}{0.728936in}}%
\pgfpathlineto{\pgfqpoint{0.834307in}{0.741862in}}%
\pgfpathlineto{\pgfqpoint{0.835515in}{0.744440in}}%
\pgfpathlineto{\pgfqpoint{0.848784in}{0.752773in}}%
\pgfpathlineto{\pgfqpoint{0.862053in}{0.752944in}}%
\pgfpathlineto{\pgfqpoint{0.875322in}{0.745421in}}%
\pgfpathlineto{\pgfqpoint{0.877026in}{0.741862in}}%
\pgfpathlineto{\pgfqpoint{0.879200in}{0.728936in}}%
\pgfpathlineto{\pgfqpoint{0.879331in}{0.716009in}}%
\pgfpathlineto{\pgfqpoint{0.877809in}{0.703083in}}%
\pgfpathlineto{\pgfqpoint{0.875322in}{0.696441in}}%
\pgfpathlineto{\pgfqpoint{0.868234in}{0.690156in}}%
\pgfpathlineto{\pgfqpoint{0.862053in}{0.687837in}}%
\pgfpathlineto{\pgfqpoint{0.848784in}{0.687986in}}%
\pgfpathclose%
\pgfpathmoveto{\pgfqpoint{0.885160in}{0.690156in}}%
\pgfpathlineto{\pgfqpoint{0.885178in}{0.703083in}}%
\pgfpathlineto{\pgfqpoint{0.885259in}{0.716009in}}%
\pgfpathlineto{\pgfqpoint{0.885447in}{0.728936in}}%
\pgfpathlineto{\pgfqpoint{0.885959in}{0.741862in}}%
\pgfpathlineto{\pgfqpoint{0.888591in}{0.754014in}}%
\pgfpathlineto{\pgfqpoint{0.889197in}{0.754789in}}%
\pgfpathlineto{\pgfqpoint{0.901860in}{0.758636in}}%
\pgfpathlineto{\pgfqpoint{0.915129in}{0.758905in}}%
\pgfpathlineto{\pgfqpoint{0.928398in}{0.758195in}}%
\pgfpathlineto{\pgfqpoint{0.937734in}{0.754789in}}%
\pgfpathlineto{\pgfqpoint{0.941667in}{0.745676in}}%
\pgfpathlineto{\pgfqpoint{0.942176in}{0.741862in}}%
\pgfpathlineto{\pgfqpoint{0.942857in}{0.728936in}}%
\pgfpathlineto{\pgfqpoint{0.943066in}{0.716009in}}%
\pgfpathlineto{\pgfqpoint{0.943055in}{0.703083in}}%
\pgfpathlineto{\pgfqpoint{0.942625in}{0.690156in}}%
\pgfpathlineto{\pgfqpoint{0.941667in}{0.684675in}}%
\pgfpathlineto{\pgfqpoint{0.928398in}{0.679869in}}%
\pgfpathlineto{\pgfqpoint{0.915129in}{0.679594in}}%
\pgfpathlineto{\pgfqpoint{0.901860in}{0.679523in}}%
\pgfpathlineto{\pgfqpoint{0.888591in}{0.679603in}}%
\pgfpathclose%
\pgfpathmoveto{\pgfqpoint{0.958099in}{0.690156in}}%
\pgfpathlineto{\pgfqpoint{0.954936in}{0.692491in}}%
\pgfpathlineto{\pgfqpoint{0.950424in}{0.703083in}}%
\pgfpathlineto{\pgfqpoint{0.949013in}{0.716009in}}%
\pgfpathlineto{\pgfqpoint{0.949120in}{0.728936in}}%
\pgfpathlineto{\pgfqpoint{0.951079in}{0.741862in}}%
\pgfpathlineto{\pgfqpoint{0.954936in}{0.748897in}}%
\pgfpathlineto{\pgfqpoint{0.968205in}{0.754260in}}%
\pgfpathlineto{\pgfqpoint{0.981473in}{0.753933in}}%
\pgfpathlineto{\pgfqpoint{0.994742in}{0.745269in}}%
\pgfpathlineto{\pgfqpoint{0.996105in}{0.741862in}}%
\pgfpathlineto{\pgfqpoint{0.997936in}{0.728936in}}%
\pgfpathlineto{\pgfqpoint{0.998028in}{0.716009in}}%
\pgfpathlineto{\pgfqpoint{0.996694in}{0.703083in}}%
\pgfpathlineto{\pgfqpoint{0.994742in}{0.696932in}}%
\pgfpathlineto{\pgfqpoint{0.988806in}{0.690156in}}%
\pgfpathlineto{\pgfqpoint{0.981473in}{0.687013in}}%
\pgfpathlineto{\pgfqpoint{0.968205in}{0.686658in}}%
\pgfpathclose%
\pgfpathmoveto{\pgfqpoint{1.004744in}{0.690156in}}%
\pgfpathlineto{\pgfqpoint{1.003964in}{0.703083in}}%
\pgfpathlineto{\pgfqpoint{1.003881in}{0.716009in}}%
\pgfpathlineto{\pgfqpoint{1.004103in}{0.728936in}}%
\pgfpathlineto{\pgfqpoint{1.004911in}{0.741862in}}%
\pgfpathlineto{\pgfqpoint{1.008011in}{0.752489in}}%
\pgfpathlineto{\pgfqpoint{1.010561in}{0.754789in}}%
\pgfpathlineto{\pgfqpoint{1.021280in}{0.757752in}}%
\pgfpathlineto{\pgfqpoint{1.034549in}{0.758123in}}%
\pgfpathlineto{\pgfqpoint{1.047818in}{0.756966in}}%
\pgfpathlineto{\pgfqpoint{1.053222in}{0.754789in}}%
\pgfpathlineto{\pgfqpoint{1.059181in}{0.741862in}}%
\pgfpathlineto{\pgfqpoint{1.060130in}{0.728936in}}%
\pgfpathlineto{\pgfqpoint{1.060378in}{0.716009in}}%
\pgfpathlineto{\pgfqpoint{1.060242in}{0.703083in}}%
\pgfpathlineto{\pgfqpoint{1.059185in}{0.690156in}}%
\pgfpathlineto{\pgfqpoint{1.047818in}{0.681027in}}%
\pgfpathlineto{\pgfqpoint{1.034549in}{0.680404in}}%
\pgfpathlineto{\pgfqpoint{1.021280in}{0.680548in}}%
\pgfpathlineto{\pgfqpoint{1.008011in}{0.683036in}}%
\pgfpathclose%
\pgfpathmoveto{\pgfqpoint{1.073911in}{0.690156in}}%
\pgfpathlineto{\pgfqpoint{1.067835in}{0.703083in}}%
\pgfpathlineto{\pgfqpoint{1.066502in}{0.716009in}}%
\pgfpathlineto{\pgfqpoint{1.066585in}{0.728936in}}%
\pgfpathlineto{\pgfqpoint{1.068384in}{0.741862in}}%
\pgfpathlineto{\pgfqpoint{1.074356in}{0.751321in}}%
\pgfpathlineto{\pgfqpoint{1.086343in}{0.754789in}}%
\pgfpathlineto{\pgfqpoint{1.087625in}{0.754996in}}%
\pgfpathlineto{\pgfqpoint{1.092013in}{0.754789in}}%
\pgfpathlineto{\pgfqpoint{1.100894in}{0.754256in}}%
\pgfpathlineto{\pgfqpoint{1.114163in}{0.743091in}}%
\pgfpathlineto{\pgfqpoint{1.114560in}{0.741862in}}%
\pgfpathlineto{\pgfqpoint{1.116240in}{0.728936in}}%
\pgfpathlineto{\pgfqpoint{1.116315in}{0.716009in}}%
\pgfpathlineto{\pgfqpoint{1.115064in}{0.703083in}}%
\pgfpathlineto{\pgfqpoint{1.114163in}{0.699637in}}%
\pgfpathlineto{\pgfqpoint{1.107835in}{0.690156in}}%
\pgfpathlineto{\pgfqpoint{1.100894in}{0.686781in}}%
\pgfpathlineto{\pgfqpoint{1.087625in}{0.685972in}}%
\pgfpathlineto{\pgfqpoint{1.074356in}{0.689779in}}%
\pgfpathclose%
\pgfpathmoveto{\pgfqpoint{1.123387in}{0.690156in}}%
\pgfpathlineto{\pgfqpoint{1.122283in}{0.703083in}}%
\pgfpathlineto{\pgfqpoint{1.122133in}{0.716009in}}%
\pgfpathlineto{\pgfqpoint{1.122369in}{0.728936in}}%
\pgfpathlineto{\pgfqpoint{1.123297in}{0.741862in}}%
\pgfpathlineto{\pgfqpoint{1.127432in}{0.753176in}}%
\pgfpathlineto{\pgfqpoint{1.129793in}{0.754789in}}%
\pgfpathlineto{\pgfqpoint{1.140701in}{0.757535in}}%
\pgfpathlineto{\pgfqpoint{1.153970in}{0.757843in}}%
\pgfpathlineto{\pgfqpoint{1.167239in}{0.756323in}}%
\pgfpathlineto{\pgfqpoint{1.170661in}{0.754789in}}%
\pgfpathlineto{\pgfqpoint{1.176875in}{0.741862in}}%
\pgfpathlineto{\pgfqpoint{1.177849in}{0.728936in}}%
\pgfpathlineto{\pgfqpoint{1.178097in}{0.716009in}}%
\pgfpathlineto{\pgfqpoint{1.177939in}{0.703083in}}%
\pgfpathlineto{\pgfqpoint{1.176780in}{0.690156in}}%
\pgfpathlineto{\pgfqpoint{1.167239in}{0.681528in}}%
\pgfpathlineto{\pgfqpoint{1.153970in}{0.680681in}}%
\pgfpathlineto{\pgfqpoint{1.140701in}{0.680850in}}%
\pgfpathlineto{\pgfqpoint{1.127432in}{0.683288in}}%
\pgfpathclose%
\pgfpathmoveto{\pgfqpoint{1.191905in}{0.690156in}}%
\pgfpathlineto{\pgfqpoint{1.185687in}{0.703083in}}%
\pgfpathlineto{\pgfqpoint{1.184328in}{0.716009in}}%
\pgfpathlineto{\pgfqpoint{1.184410in}{0.728936in}}%
\pgfpathlineto{\pgfqpoint{1.186235in}{0.741862in}}%
\pgfpathlineto{\pgfqpoint{1.193777in}{0.752339in}}%
\pgfpathlineto{\pgfqpoint{1.204893in}{0.754789in}}%
\pgfpathlineto{\pgfqpoint{1.207045in}{0.755077in}}%
\pgfpathlineto{\pgfqpoint{1.210884in}{0.754789in}}%
\pgfpathlineto{\pgfqpoint{1.220314in}{0.753831in}}%
\pgfpathlineto{\pgfqpoint{1.232127in}{0.741862in}}%
\pgfpathlineto{\pgfqpoint{1.233583in}{0.733926in}}%
\pgfpathlineto{\pgfqpoint{1.234109in}{0.728936in}}%
\pgfpathlineto{\pgfqpoint{1.234188in}{0.716009in}}%
\pgfpathlineto{\pgfqpoint{1.233583in}{0.709160in}}%
\pgfpathlineto{\pgfqpoint{1.232768in}{0.703083in}}%
\pgfpathlineto{\pgfqpoint{1.225674in}{0.690156in}}%
\pgfpathlineto{\pgfqpoint{1.220314in}{0.687213in}}%
\pgfpathlineto{\pgfqpoint{1.207045in}{0.685879in}}%
\pgfpathlineto{\pgfqpoint{1.193777in}{0.688764in}}%
\pgfpathclose%
\pgfpathmoveto{\pgfqpoint{1.241116in}{0.690156in}}%
\pgfpathlineto{\pgfqpoint{1.240134in}{0.703083in}}%
\pgfpathlineto{\pgfqpoint{1.240008in}{0.716009in}}%
\pgfpathlineto{\pgfqpoint{1.240238in}{0.728936in}}%
\pgfpathlineto{\pgfqpoint{1.241120in}{0.741862in}}%
\pgfpathlineto{\pgfqpoint{1.246664in}{0.754789in}}%
\pgfpathlineto{\pgfqpoint{1.246852in}{0.754917in}}%
\pgfpathlineto{\pgfqpoint{1.260121in}{0.757890in}}%
\pgfpathlineto{\pgfqpoint{1.273390in}{0.758079in}}%
\pgfpathlineto{\pgfqpoint{1.286659in}{0.756445in}}%
\pgfpathlineto{\pgfqpoint{1.289948in}{0.754789in}}%
\pgfpathlineto{\pgfqpoint{1.295230in}{0.741862in}}%
\pgfpathlineto{\pgfqpoint{1.296057in}{0.728936in}}%
\pgfpathlineto{\pgfqpoint{1.296284in}{0.716009in}}%
\pgfpathlineto{\pgfqpoint{1.296199in}{0.703083in}}%
\pgfpathlineto{\pgfqpoint{1.295400in}{0.690156in}}%
\pgfpathlineto{\pgfqpoint{1.286659in}{0.681154in}}%
\pgfpathlineto{\pgfqpoint{1.273390in}{0.680407in}}%
\pgfpathlineto{\pgfqpoint{1.260121in}{0.680534in}}%
\pgfpathlineto{\pgfqpoint{1.246852in}{0.682152in}}%
\pgfpathclose%
\pgfpathmoveto{\pgfqpoint{1.310936in}{0.690156in}}%
\pgfpathlineto{\pgfqpoint{1.304008in}{0.703083in}}%
\pgfpathlineto{\pgfqpoint{1.302507in}{0.716009in}}%
\pgfpathlineto{\pgfqpoint{1.302611in}{0.728936in}}%
\pgfpathlineto{\pgfqpoint{1.304670in}{0.741862in}}%
\pgfpathlineto{\pgfqpoint{1.313197in}{0.752318in}}%
\pgfpathlineto{\pgfqpoint{1.326466in}{0.754558in}}%
\pgfpathlineto{\pgfqpoint{1.339735in}{0.752527in}}%
\pgfpathlineto{\pgfqpoint{1.349079in}{0.741862in}}%
\pgfpathlineto{\pgfqpoint{1.351298in}{0.728936in}}%
\pgfpathlineto{\pgfqpoint{1.351419in}{0.716009in}}%
\pgfpathlineto{\pgfqpoint{1.349820in}{0.703083in}}%
\pgfpathlineto{\pgfqpoint{1.342539in}{0.690156in}}%
\pgfpathlineto{\pgfqpoint{1.339735in}{0.688420in}}%
\pgfpathlineto{\pgfqpoint{1.326466in}{0.686357in}}%
\pgfpathlineto{\pgfqpoint{1.313197in}{0.688670in}}%
\pgfpathclose%
\pgfpathmoveto{\pgfqpoint{1.431160in}{0.690156in}}%
\pgfpathlineto{\pgfqpoint{1.422847in}{0.703083in}}%
\pgfpathlineto{\pgfqpoint{1.421069in}{0.716009in}}%
\pgfpathlineto{\pgfqpoint{1.421223in}{0.728936in}}%
\pgfpathlineto{\pgfqpoint{1.423763in}{0.741862in}}%
\pgfpathlineto{\pgfqpoint{1.432617in}{0.751480in}}%
\pgfpathlineto{\pgfqpoint{1.445886in}{0.753360in}}%
\pgfpathlineto{\pgfqpoint{1.459155in}{0.750144in}}%
\pgfpathlineto{\pgfqpoint{1.465585in}{0.741862in}}%
\pgfpathlineto{\pgfqpoint{1.468109in}{0.728936in}}%
\pgfpathlineto{\pgfqpoint{1.468273in}{0.716009in}}%
\pgfpathlineto{\pgfqpoint{1.466529in}{0.703083in}}%
\pgfpathlineto{\pgfqpoint{1.459155in}{0.690678in}}%
\pgfpathlineto{\pgfqpoint{1.457829in}{0.690156in}}%
\pgfpathlineto{\pgfqpoint{1.445886in}{0.687407in}}%
\pgfpathlineto{\pgfqpoint{1.432617in}{0.689309in}}%
\pgfpathclose%
\pgfpathmoveto{\pgfqpoint{0.828150in}{0.767715in}}%
\pgfpathlineto{\pgfqpoint{0.827035in}{0.780642in}}%
\pgfpathlineto{\pgfqpoint{0.826898in}{0.793568in}}%
\pgfpathlineto{\pgfqpoint{0.827006in}{0.806495in}}%
\pgfpathlineto{\pgfqpoint{0.827451in}{0.819421in}}%
\pgfpathlineto{\pgfqpoint{0.829456in}{0.832348in}}%
\pgfpathlineto{\pgfqpoint{0.835515in}{0.837997in}}%
\pgfpathlineto{\pgfqpoint{0.848784in}{0.839556in}}%
\pgfpathlineto{\pgfqpoint{0.862053in}{0.839401in}}%
\pgfpathlineto{\pgfqpoint{0.875322in}{0.837107in}}%
\pgfpathlineto{\pgfqpoint{0.880181in}{0.832348in}}%
\pgfpathlineto{\pgfqpoint{0.882761in}{0.819421in}}%
\pgfpathlineto{\pgfqpoint{0.883330in}{0.806495in}}%
\pgfpathlineto{\pgfqpoint{0.883427in}{0.793568in}}%
\pgfpathlineto{\pgfqpoint{0.883122in}{0.780642in}}%
\pgfpathlineto{\pgfqpoint{0.881099in}{0.767715in}}%
\pgfpathlineto{\pgfqpoint{0.875322in}{0.763037in}}%
\pgfpathlineto{\pgfqpoint{0.862053in}{0.761637in}}%
\pgfpathlineto{\pgfqpoint{0.848784in}{0.761479in}}%
\pgfpathlineto{\pgfqpoint{0.835515in}{0.762111in}}%
\pgfpathclose%
\pgfpathmoveto{\pgfqpoint{0.906438in}{0.767715in}}%
\pgfpathlineto{\pgfqpoint{0.901860in}{0.768543in}}%
\pgfpathlineto{\pgfqpoint{0.891735in}{0.780642in}}%
\pgfpathlineto{\pgfqpoint{0.889591in}{0.793568in}}%
\pgfpathlineto{\pgfqpoint{0.889338in}{0.806495in}}%
\pgfpathlineto{\pgfqpoint{0.890640in}{0.819421in}}%
\pgfpathlineto{\pgfqpoint{0.897535in}{0.832348in}}%
\pgfpathlineto{\pgfqpoint{0.901860in}{0.834704in}}%
\pgfpathlineto{\pgfqpoint{0.915129in}{0.836318in}}%
\pgfpathlineto{\pgfqpoint{0.928398in}{0.834514in}}%
\pgfpathlineto{\pgfqpoint{0.932083in}{0.832348in}}%
\pgfpathlineto{\pgfqpoint{0.938418in}{0.819421in}}%
\pgfpathlineto{\pgfqpoint{0.939578in}{0.806495in}}%
\pgfpathlineto{\pgfqpoint{0.939340in}{0.793568in}}%
\pgfpathlineto{\pgfqpoint{0.937386in}{0.780642in}}%
\pgfpathlineto{\pgfqpoint{0.928398in}{0.768842in}}%
\pgfpathlineto{\pgfqpoint{0.923076in}{0.767715in}}%
\pgfpathlineto{\pgfqpoint{0.915129in}{0.766642in}}%
\pgfpathclose%
\pgfpathmoveto{\pgfqpoint{0.949301in}{0.767715in}}%
\pgfpathlineto{\pgfqpoint{0.946216in}{0.780642in}}%
\pgfpathlineto{\pgfqpoint{0.945744in}{0.793568in}}%
\pgfpathlineto{\pgfqpoint{0.945835in}{0.806495in}}%
\pgfpathlineto{\pgfqpoint{0.946546in}{0.819421in}}%
\pgfpathlineto{\pgfqpoint{0.949914in}{0.832348in}}%
\pgfpathlineto{\pgfqpoint{0.954936in}{0.836437in}}%
\pgfpathlineto{\pgfqpoint{0.968205in}{0.838503in}}%
\pgfpathlineto{\pgfqpoint{0.981473in}{0.838217in}}%
\pgfpathlineto{\pgfqpoint{0.994742in}{0.834102in}}%
\pgfpathlineto{\pgfqpoint{0.996276in}{0.832348in}}%
\pgfpathlineto{\pgfqpoint{0.999854in}{0.819421in}}%
\pgfpathlineto{\pgfqpoint{1.000616in}{0.806495in}}%
\pgfpathlineto{\pgfqpoint{1.000699in}{0.793568in}}%
\pgfpathlineto{\pgfqpoint{1.000145in}{0.780642in}}%
\pgfpathlineto{\pgfqpoint{0.996681in}{0.767715in}}%
\pgfpathlineto{\pgfqpoint{0.994742in}{0.765869in}}%
\pgfpathlineto{\pgfqpoint{0.981473in}{0.762811in}}%
\pgfpathlineto{\pgfqpoint{0.968205in}{0.762582in}}%
\pgfpathlineto{\pgfqpoint{0.954936in}{0.763984in}}%
\pgfpathclose%
\pgfpathmoveto{\pgfqpoint{1.019331in}{0.767715in}}%
\pgfpathlineto{\pgfqpoint{1.008355in}{0.780642in}}%
\pgfpathlineto{\pgfqpoint{1.008011in}{0.782376in}}%
\pgfpathlineto{\pgfqpoint{1.006690in}{0.793568in}}%
\pgfpathlineto{\pgfqpoint{1.006486in}{0.806495in}}%
\pgfpathlineto{\pgfqpoint{1.007415in}{0.819421in}}%
\pgfpathlineto{\pgfqpoint{1.008011in}{0.822322in}}%
\pgfpathlineto{\pgfqpoint{1.013442in}{0.832348in}}%
\pgfpathlineto{\pgfqpoint{1.021280in}{0.836124in}}%
\pgfpathlineto{\pgfqpoint{1.034549in}{0.837079in}}%
\pgfpathlineto{\pgfqpoint{1.047818in}{0.834990in}}%
\pgfpathlineto{\pgfqpoint{1.051766in}{0.832348in}}%
\pgfpathlineto{\pgfqpoint{1.056995in}{0.819421in}}%
\pgfpathlineto{\pgfqpoint{1.057952in}{0.806495in}}%
\pgfpathlineto{\pgfqpoint{1.057735in}{0.793568in}}%
\pgfpathlineto{\pgfqpoint{1.056061in}{0.780642in}}%
\pgfpathlineto{\pgfqpoint{1.047818in}{0.768424in}}%
\pgfpathlineto{\pgfqpoint{1.045313in}{0.767715in}}%
\pgfpathlineto{\pgfqpoint{1.034549in}{0.765910in}}%
\pgfpathlineto{\pgfqpoint{1.021280in}{0.767002in}}%
\pgfpathclose%
\pgfpathmoveto{\pgfqpoint{1.069026in}{0.767715in}}%
\pgfpathlineto{\pgfqpoint{1.064823in}{0.780642in}}%
\pgfpathlineto{\pgfqpoint{1.064155in}{0.793568in}}%
\pgfpathlineto{\pgfqpoint{1.064237in}{0.806495in}}%
\pgfpathlineto{\pgfqpoint{1.065108in}{0.819421in}}%
\pgfpathlineto{\pgfqpoint{1.069269in}{0.832348in}}%
\pgfpathlineto{\pgfqpoint{1.074356in}{0.836016in}}%
\pgfpathlineto{\pgfqpoint{1.087625in}{0.838013in}}%
\pgfpathlineto{\pgfqpoint{1.100894in}{0.837523in}}%
\pgfpathlineto{\pgfqpoint{1.113171in}{0.832348in}}%
\pgfpathlineto{\pgfqpoint{1.114163in}{0.831091in}}%
\pgfpathlineto{\pgfqpoint{1.117448in}{0.819421in}}%
\pgfpathlineto{\pgfqpoint{1.118297in}{0.806495in}}%
\pgfpathlineto{\pgfqpoint{1.118373in}{0.793568in}}%
\pgfpathlineto{\pgfqpoint{1.117709in}{0.780642in}}%
\pgfpathlineto{\pgfqpoint{1.114163in}{0.768506in}}%
\pgfpathlineto{\pgfqpoint{1.113398in}{0.767715in}}%
\pgfpathlineto{\pgfqpoint{1.100894in}{0.763476in}}%
\pgfpathlineto{\pgfqpoint{1.087625in}{0.763102in}}%
\pgfpathlineto{\pgfqpoint{1.074356in}{0.764597in}}%
\pgfpathclose%
\pgfpathmoveto{\pgfqpoint{1.136469in}{0.767715in}}%
\pgfpathlineto{\pgfqpoint{1.127432in}{0.775896in}}%
\pgfpathlineto{\pgfqpoint{1.125868in}{0.780642in}}%
\pgfpathlineto{\pgfqpoint{1.124368in}{0.793568in}}%
\pgfpathlineto{\pgfqpoint{1.124170in}{0.806495in}}%
\pgfpathlineto{\pgfqpoint{1.125019in}{0.819421in}}%
\pgfpathlineto{\pgfqpoint{1.127432in}{0.828476in}}%
\pgfpathlineto{\pgfqpoint{1.130356in}{0.832348in}}%
\pgfpathlineto{\pgfqpoint{1.140701in}{0.836747in}}%
\pgfpathlineto{\pgfqpoint{1.153970in}{0.837275in}}%
\pgfpathlineto{\pgfqpoint{1.167239in}{0.834629in}}%
\pgfpathlineto{\pgfqpoint{1.170213in}{0.832348in}}%
\pgfpathlineto{\pgfqpoint{1.175068in}{0.819421in}}%
\pgfpathlineto{\pgfqpoint{1.175959in}{0.806495in}}%
\pgfpathlineto{\pgfqpoint{1.175752in}{0.793568in}}%
\pgfpathlineto{\pgfqpoint{1.174177in}{0.780642in}}%
\pgfpathlineto{\pgfqpoint{1.167239in}{0.768963in}}%
\pgfpathlineto{\pgfqpoint{1.163840in}{0.767715in}}%
\pgfpathlineto{\pgfqpoint{1.153970in}{0.765732in}}%
\pgfpathlineto{\pgfqpoint{1.140701in}{0.766351in}}%
\pgfpathclose%
\pgfpathmoveto{\pgfqpoint{1.187295in}{0.767715in}}%
\pgfpathlineto{\pgfqpoint{1.182816in}{0.780642in}}%
\pgfpathlineto{\pgfqpoint{1.182095in}{0.793568in}}%
\pgfpathlineto{\pgfqpoint{1.182178in}{0.806495in}}%
\pgfpathlineto{\pgfqpoint{1.183099in}{0.819421in}}%
\pgfpathlineto{\pgfqpoint{1.187483in}{0.832348in}}%
\pgfpathlineto{\pgfqpoint{1.193777in}{0.836405in}}%
\pgfpathlineto{\pgfqpoint{1.207045in}{0.838044in}}%
\pgfpathlineto{\pgfqpoint{1.220314in}{0.837371in}}%
\pgfpathlineto{\pgfqpoint{1.231092in}{0.832348in}}%
\pgfpathlineto{\pgfqpoint{1.233583in}{0.828111in}}%
\pgfpathlineto{\pgfqpoint{1.235518in}{0.819421in}}%
\pgfpathlineto{\pgfqpoint{1.236348in}{0.806495in}}%
\pgfpathlineto{\pgfqpoint{1.236426in}{0.793568in}}%
\pgfpathlineto{\pgfqpoint{1.235790in}{0.780642in}}%
\pgfpathlineto{\pgfqpoint{1.233583in}{0.770845in}}%
\pgfpathlineto{\pgfqpoint{1.231376in}{0.767715in}}%
\pgfpathlineto{\pgfqpoint{1.220314in}{0.763576in}}%
\pgfpathlineto{\pgfqpoint{1.207045in}{0.763082in}}%
\pgfpathlineto{\pgfqpoint{1.193777in}{0.764329in}}%
\pgfpathclose%
\pgfpathmoveto{\pgfqpoint{1.255335in}{0.767715in}}%
\pgfpathlineto{\pgfqpoint{1.246852in}{0.773542in}}%
\pgfpathlineto{\pgfqpoint{1.244017in}{0.780642in}}%
\pgfpathlineto{\pgfqpoint{1.242463in}{0.793568in}}%
\pgfpathlineto{\pgfqpoint{1.242262in}{0.806495in}}%
\pgfpathlineto{\pgfqpoint{1.243150in}{0.819421in}}%
\pgfpathlineto{\pgfqpoint{1.246852in}{0.830721in}}%
\pgfpathlineto{\pgfqpoint{1.248491in}{0.832348in}}%
\pgfpathlineto{\pgfqpoint{1.260121in}{0.836693in}}%
\pgfpathlineto{\pgfqpoint{1.273390in}{0.836889in}}%
\pgfpathlineto{\pgfqpoint{1.286659in}{0.833141in}}%
\pgfpathlineto{\pgfqpoint{1.287553in}{0.832348in}}%
\pgfpathlineto{\pgfqpoint{1.292669in}{0.819421in}}%
\pgfpathlineto{\pgfqpoint{1.293619in}{0.806495in}}%
\pgfpathlineto{\pgfqpoint{1.293411in}{0.793568in}}%
\pgfpathlineto{\pgfqpoint{1.291773in}{0.780642in}}%
\pgfpathlineto{\pgfqpoint{1.286659in}{0.770776in}}%
\pgfpathlineto{\pgfqpoint{1.280150in}{0.767715in}}%
\pgfpathlineto{\pgfqpoint{1.273390in}{0.766125in}}%
\pgfpathlineto{\pgfqpoint{1.260121in}{0.766363in}}%
\pgfpathclose%
\pgfpathmoveto{\pgfqpoint{1.304013in}{0.767715in}}%
\pgfpathlineto{\pgfqpoint{1.300128in}{0.780642in}}%
\pgfpathlineto{\pgfqpoint{1.299928in}{0.783988in}}%
\pgfpathlineto{\pgfqpoint{1.299544in}{0.793568in}}%
\pgfpathlineto{\pgfqpoint{1.299628in}{0.806495in}}%
\pgfpathlineto{\pgfqpoint{1.299928in}{0.812261in}}%
\pgfpathlineto{\pgfqpoint{1.300455in}{0.819421in}}%
\pgfpathlineto{\pgfqpoint{1.304469in}{0.832348in}}%
\pgfpathlineto{\pgfqpoint{1.313197in}{0.837411in}}%
\pgfpathlineto{\pgfqpoint{1.326466in}{0.838578in}}%
\pgfpathlineto{\pgfqpoint{1.339735in}{0.837845in}}%
\pgfpathlineto{\pgfqpoint{1.350396in}{0.832348in}}%
\pgfpathlineto{\pgfqpoint{1.353004in}{0.825916in}}%
\pgfpathlineto{\pgfqpoint{1.354059in}{0.819421in}}%
\pgfpathlineto{\pgfqpoint{1.354759in}{0.806495in}}%
\pgfpathlineto{\pgfqpoint{1.354849in}{0.793568in}}%
\pgfpathlineto{\pgfqpoint{1.354384in}{0.780642in}}%
\pgfpathlineto{\pgfqpoint{1.353004in}{0.771830in}}%
\pgfpathlineto{\pgfqpoint{1.351093in}{0.767715in}}%
\pgfpathlineto{\pgfqpoint{1.339735in}{0.763018in}}%
\pgfpathlineto{\pgfqpoint{1.326466in}{0.762542in}}%
\pgfpathlineto{\pgfqpoint{1.313197in}{0.763398in}}%
\pgfpathclose%
\pgfpathmoveto{\pgfqpoint{1.376480in}{0.767715in}}%
\pgfpathlineto{\pgfqpoint{1.366273in}{0.773129in}}%
\pgfpathlineto{\pgfqpoint{1.362735in}{0.780642in}}%
\pgfpathlineto{\pgfqpoint{1.360969in}{0.793568in}}%
\pgfpathlineto{\pgfqpoint{1.360755in}{0.806495in}}%
\pgfpathlineto{\pgfqpoint{1.361803in}{0.819421in}}%
\pgfpathlineto{\pgfqpoint{1.366273in}{0.830875in}}%
\pgfpathlineto{\pgfqpoint{1.368205in}{0.832348in}}%
\pgfpathlineto{\pgfqpoint{1.379542in}{0.836041in}}%
\pgfpathlineto{\pgfqpoint{1.392811in}{0.835875in}}%
\pgfpathlineto{\pgfqpoint{1.402757in}{0.832348in}}%
\pgfpathlineto{\pgfqpoint{1.406080in}{0.829513in}}%
\pgfpathlineto{\pgfqpoint{1.409809in}{0.819421in}}%
\pgfpathlineto{\pgfqpoint{1.410931in}{0.806495in}}%
\pgfpathlineto{\pgfqpoint{1.410713in}{0.793568in}}%
\pgfpathlineto{\pgfqpoint{1.408865in}{0.780642in}}%
\pgfpathlineto{\pgfqpoint{1.406080in}{0.774400in}}%
\pgfpathlineto{\pgfqpoint{1.394965in}{0.767715in}}%
\pgfpathlineto{\pgfqpoint{1.392811in}{0.767131in}}%
\pgfpathlineto{\pgfqpoint{1.379542in}{0.766964in}}%
\pgfpathclose%
\pgfpathmoveto{\pgfqpoint{1.419064in}{0.767715in}}%
\pgfpathlineto{\pgfqpoint{1.417039in}{0.780642in}}%
\pgfpathlineto{\pgfqpoint{1.416733in}{0.793568in}}%
\pgfpathlineto{\pgfqpoint{1.416830in}{0.806495in}}%
\pgfpathlineto{\pgfqpoint{1.417399in}{0.819421in}}%
\pgfpathlineto{\pgfqpoint{1.419348in}{0.830638in}}%
\pgfpathlineto{\pgfqpoint{1.420082in}{0.832348in}}%
\pgfpathlineto{\pgfqpoint{1.432617in}{0.838919in}}%
\pgfpathlineto{\pgfqpoint{1.445886in}{0.839617in}}%
\pgfpathlineto{\pgfqpoint{1.459155in}{0.839079in}}%
\pgfpathlineto{\pgfqpoint{1.470926in}{0.832348in}}%
\pgfpathlineto{\pgfqpoint{1.472424in}{0.826013in}}%
\pgfpathlineto{\pgfqpoint{1.473085in}{0.819421in}}%
\pgfpathlineto{\pgfqpoint{1.473537in}{0.806495in}}%
\pgfpathlineto{\pgfqpoint{1.473647in}{0.793568in}}%
\pgfpathlineto{\pgfqpoint{1.473508in}{0.780642in}}%
\pgfpathlineto{\pgfqpoint{1.472424in}{0.767943in}}%
\pgfpathlineto{\pgfqpoint{1.472370in}{0.767715in}}%
\pgfpathlineto{\pgfqpoint{1.459155in}{0.761657in}}%
\pgfpathlineto{\pgfqpoint{1.445886in}{0.761482in}}%
\pgfpathlineto{\pgfqpoint{1.432617in}{0.761933in}}%
\pgfpathlineto{\pgfqpoint{1.419348in}{0.767142in}}%
\pgfpathclose%
\pgfpathmoveto{\pgfqpoint{0.775480in}{0.780642in}}%
\pgfpathlineto{\pgfqpoint{0.773019in}{0.793568in}}%
\pgfpathlineto{\pgfqpoint{0.772754in}{0.806495in}}%
\pgfpathlineto{\pgfqpoint{0.774325in}{0.819421in}}%
\pgfpathlineto{\pgfqpoint{0.782439in}{0.832295in}}%
\pgfpathlineto{\pgfqpoint{0.782597in}{0.832348in}}%
\pgfpathlineto{\pgfqpoint{0.795708in}{0.834988in}}%
\pgfpathlineto{\pgfqpoint{0.808977in}{0.833374in}}%
\pgfpathlineto{\pgfqpoint{0.810959in}{0.832348in}}%
\pgfpathlineto{\pgfqpoint{0.819273in}{0.819421in}}%
\pgfpathlineto{\pgfqpoint{0.820801in}{0.806495in}}%
\pgfpathlineto{\pgfqpoint{0.820529in}{0.793568in}}%
\pgfpathlineto{\pgfqpoint{0.818081in}{0.780642in}}%
\pgfpathlineto{\pgfqpoint{0.808977in}{0.770025in}}%
\pgfpathlineto{\pgfqpoint{0.795708in}{0.767964in}}%
\pgfpathlineto{\pgfqpoint{0.782439in}{0.771293in}}%
\pgfpathclose%
\pgfpathmoveto{\pgfqpoint{0.772858in}{0.845274in}}%
\pgfpathlineto{\pgfqpoint{0.769170in}{0.851110in}}%
\pgfpathlineto{\pgfqpoint{0.768279in}{0.858201in}}%
\pgfpathlineto{\pgfqpoint{0.767898in}{0.871127in}}%
\pgfpathlineto{\pgfqpoint{0.767900in}{0.884054in}}%
\pgfpathlineto{\pgfqpoint{0.768201in}{0.896980in}}%
\pgfpathlineto{\pgfqpoint{0.769170in}{0.908344in}}%
\pgfpathlineto{\pgfqpoint{0.769428in}{0.909907in}}%
\pgfpathlineto{\pgfqpoint{0.782439in}{0.920250in}}%
\pgfpathlineto{\pgfqpoint{0.795708in}{0.920767in}}%
\pgfpathlineto{\pgfqpoint{0.808977in}{0.919967in}}%
\pgfpathlineto{\pgfqpoint{0.822246in}{0.910582in}}%
\pgfpathlineto{\pgfqpoint{0.822447in}{0.909907in}}%
\pgfpathlineto{\pgfqpoint{0.824062in}{0.896980in}}%
\pgfpathlineto{\pgfqpoint{0.824452in}{0.884054in}}%
\pgfpathlineto{\pgfqpoint{0.824400in}{0.871127in}}%
\pgfpathlineto{\pgfqpoint{0.823724in}{0.858201in}}%
\pgfpathlineto{\pgfqpoint{0.822246in}{0.851093in}}%
\pgfpathlineto{\pgfqpoint{0.816562in}{0.845274in}}%
\pgfpathlineto{\pgfqpoint{0.808977in}{0.843564in}}%
\pgfpathlineto{\pgfqpoint{0.795708in}{0.842971in}}%
\pgfpathlineto{\pgfqpoint{0.782439in}{0.843214in}}%
\pgfpathclose%
\pgfpathmoveto{\pgfqpoint{0.899851in}{0.845274in}}%
\pgfpathlineto{\pgfqpoint{0.888591in}{0.855241in}}%
\pgfpathlineto{\pgfqpoint{0.887953in}{0.858201in}}%
\pgfpathlineto{\pgfqpoint{0.886963in}{0.871127in}}%
\pgfpathlineto{\pgfqpoint{0.886860in}{0.884054in}}%
\pgfpathlineto{\pgfqpoint{0.887351in}{0.896980in}}%
\pgfpathlineto{\pgfqpoint{0.888591in}{0.906156in}}%
\pgfpathlineto{\pgfqpoint{0.889628in}{0.909907in}}%
\pgfpathlineto{\pgfqpoint{0.901860in}{0.918651in}}%
\pgfpathlineto{\pgfqpoint{0.915129in}{0.919428in}}%
\pgfpathlineto{\pgfqpoint{0.928398in}{0.918022in}}%
\pgfpathlineto{\pgfqpoint{0.938206in}{0.909907in}}%
\pgfpathlineto{\pgfqpoint{0.940839in}{0.896980in}}%
\pgfpathlineto{\pgfqpoint{0.941445in}{0.884054in}}%
\pgfpathlineto{\pgfqpoint{0.941294in}{0.871127in}}%
\pgfpathlineto{\pgfqpoint{0.940001in}{0.858201in}}%
\pgfpathlineto{\pgfqpoint{0.928398in}{0.845455in}}%
\pgfpathlineto{\pgfqpoint{0.926912in}{0.845274in}}%
\pgfpathlineto{\pgfqpoint{0.915129in}{0.844301in}}%
\pgfpathlineto{\pgfqpoint{0.901860in}{0.844879in}}%
\pgfpathclose%
\pgfpathmoveto{\pgfqpoint{1.030558in}{0.845274in}}%
\pgfpathlineto{\pgfqpoint{1.021280in}{0.845749in}}%
\pgfpathlineto{\pgfqpoint{1.008011in}{0.854791in}}%
\pgfpathlineto{\pgfqpoint{1.006981in}{0.858201in}}%
\pgfpathlineto{\pgfqpoint{1.005623in}{0.871127in}}%
\pgfpathlineto{\pgfqpoint{1.005456in}{0.884054in}}%
\pgfpathlineto{\pgfqpoint{1.006062in}{0.896980in}}%
\pgfpathlineto{\pgfqpoint{1.008011in}{0.907608in}}%
\pgfpathlineto{\pgfqpoint{1.008927in}{0.909907in}}%
\pgfpathlineto{\pgfqpoint{1.021280in}{0.917864in}}%
\pgfpathlineto{\pgfqpoint{1.034549in}{0.918604in}}%
\pgfpathlineto{\pgfqpoint{1.047818in}{0.916531in}}%
\pgfpathlineto{\pgfqpoint{1.054997in}{0.909907in}}%
\pgfpathlineto{\pgfqpoint{1.058013in}{0.896980in}}%
\pgfpathlineto{\pgfqpoint{1.058695in}{0.884054in}}%
\pgfpathlineto{\pgfqpoint{1.058498in}{0.871127in}}%
\pgfpathlineto{\pgfqpoint{1.056946in}{0.858201in}}%
\pgfpathlineto{\pgfqpoint{1.047818in}{0.846974in}}%
\pgfpathlineto{\pgfqpoint{1.036265in}{0.845274in}}%
\pgfpathlineto{\pgfqpoint{1.034549in}{0.845115in}}%
\pgfpathclose%
\pgfpathmoveto{\pgfqpoint{1.269328in}{0.845274in}}%
\pgfpathlineto{\pgfqpoint{1.260121in}{0.845470in}}%
\pgfpathlineto{\pgfqpoint{1.246852in}{0.850431in}}%
\pgfpathlineto{\pgfqpoint{1.243197in}{0.858201in}}%
\pgfpathlineto{\pgfqpoint{1.241755in}{0.871127in}}%
\pgfpathlineto{\pgfqpoint{1.241572in}{0.884054in}}%
\pgfpathlineto{\pgfqpoint{1.242206in}{0.896980in}}%
\pgfpathlineto{\pgfqpoint{1.245010in}{0.909907in}}%
\pgfpathlineto{\pgfqpoint{1.246852in}{0.912794in}}%
\pgfpathlineto{\pgfqpoint{1.260121in}{0.918196in}}%
\pgfpathlineto{\pgfqpoint{1.273390in}{0.918497in}}%
\pgfpathlineto{\pgfqpoint{1.286659in}{0.915361in}}%
\pgfpathlineto{\pgfqpoint{1.291298in}{0.909907in}}%
\pgfpathlineto{\pgfqpoint{1.294051in}{0.896980in}}%
\pgfpathlineto{\pgfqpoint{1.294672in}{0.884054in}}%
\pgfpathlineto{\pgfqpoint{1.294501in}{0.871127in}}%
\pgfpathlineto{\pgfqpoint{1.293111in}{0.858201in}}%
\pgfpathlineto{\pgfqpoint{1.286659in}{0.847994in}}%
\pgfpathlineto{\pgfqpoint{1.274061in}{0.845274in}}%
\pgfpathlineto{\pgfqpoint{1.273390in}{0.845198in}}%
\pgfpathclose%
\pgfpathmoveto{\pgfqpoint{1.374161in}{0.845274in}}%
\pgfpathlineto{\pgfqpoint{1.366273in}{0.847553in}}%
\pgfpathlineto{\pgfqpoint{1.360371in}{0.858201in}}%
\pgfpathlineto{\pgfqpoint{1.359201in}{0.871127in}}%
\pgfpathlineto{\pgfqpoint{1.359064in}{0.884054in}}%
\pgfpathlineto{\pgfqpoint{1.359613in}{0.896980in}}%
\pgfpathlineto{\pgfqpoint{1.361997in}{0.909907in}}%
\pgfpathlineto{\pgfqpoint{1.366273in}{0.915640in}}%
\pgfpathlineto{\pgfqpoint{1.379542in}{0.919163in}}%
\pgfpathlineto{\pgfqpoint{1.392811in}{0.919265in}}%
\pgfpathlineto{\pgfqpoint{1.406080in}{0.916133in}}%
\pgfpathlineto{\pgfqpoint{1.410682in}{0.909907in}}%
\pgfpathlineto{\pgfqpoint{1.412812in}{0.896980in}}%
\pgfpathlineto{\pgfqpoint{1.413302in}{0.884054in}}%
\pgfpathlineto{\pgfqpoint{1.413199in}{0.871127in}}%
\pgfpathlineto{\pgfqpoint{1.412210in}{0.858201in}}%
\pgfpathlineto{\pgfqpoint{1.406080in}{0.846965in}}%
\pgfpathlineto{\pgfqpoint{1.399693in}{0.845274in}}%
\pgfpathlineto{\pgfqpoint{1.392811in}{0.844414in}}%
\pgfpathlineto{\pgfqpoint{1.379542in}{0.844520in}}%
\pgfpathclose%
\pgfpathmoveto{\pgfqpoint{0.717793in}{0.858201in}}%
\pgfpathlineto{\pgfqpoint{0.716095in}{0.861876in}}%
\pgfpathlineto{\pgfqpoint{0.714073in}{0.871127in}}%
\pgfpathlineto{\pgfqpoint{0.713450in}{0.884054in}}%
\pgfpathlineto{\pgfqpoint{0.714321in}{0.896980in}}%
\pgfpathlineto{\pgfqpoint{0.716095in}{0.903944in}}%
\pgfpathlineto{\pgfqpoint{0.719764in}{0.909907in}}%
\pgfpathlineto{\pgfqpoint{0.729364in}{0.915209in}}%
\pgfpathlineto{\pgfqpoint{0.742633in}{0.915874in}}%
\pgfpathlineto{\pgfqpoint{0.755902in}{0.911025in}}%
\pgfpathlineto{\pgfqpoint{0.756851in}{0.909907in}}%
\pgfpathlineto{\pgfqpoint{0.761094in}{0.896980in}}%
\pgfpathlineto{\pgfqpoint{0.761908in}{0.884054in}}%
\pgfpathlineto{\pgfqpoint{0.761308in}{0.871127in}}%
\pgfpathlineto{\pgfqpoint{0.758180in}{0.858201in}}%
\pgfpathlineto{\pgfqpoint{0.755902in}{0.854835in}}%
\pgfpathlineto{\pgfqpoint{0.742633in}{0.849432in}}%
\pgfpathlineto{\pgfqpoint{0.729364in}{0.850155in}}%
\pgfpathclose%
\pgfpathmoveto{\pgfqpoint{0.833342in}{0.858201in}}%
\pgfpathlineto{\pgfqpoint{0.830747in}{0.871127in}}%
\pgfpathlineto{\pgfqpoint{0.830235in}{0.884054in}}%
\pgfpathlineto{\pgfqpoint{0.830896in}{0.896980in}}%
\pgfpathlineto{\pgfqpoint{0.834310in}{0.909907in}}%
\pgfpathlineto{\pgfqpoint{0.835515in}{0.911617in}}%
\pgfpathlineto{\pgfqpoint{0.848784in}{0.917241in}}%
\pgfpathlineto{\pgfqpoint{0.862053in}{0.917398in}}%
\pgfpathlineto{\pgfqpoint{0.875322in}{0.912519in}}%
\pgfpathlineto{\pgfqpoint{0.877218in}{0.909907in}}%
\pgfpathlineto{\pgfqpoint{0.880386in}{0.896980in}}%
\pgfpathlineto{\pgfqpoint{0.880983in}{0.884054in}}%
\pgfpathlineto{\pgfqpoint{0.880503in}{0.871127in}}%
\pgfpathlineto{\pgfqpoint{0.878072in}{0.858201in}}%
\pgfpathlineto{\pgfqpoint{0.875322in}{0.853477in}}%
\pgfpathlineto{\pgfqpoint{0.862053in}{0.847889in}}%
\pgfpathlineto{\pgfqpoint{0.848784in}{0.848050in}}%
\pgfpathlineto{\pgfqpoint{0.835515in}{0.854379in}}%
\pgfpathclose%
\pgfpathmoveto{\pgfqpoint{0.949961in}{0.858201in}}%
\pgfpathlineto{\pgfqpoint{0.947798in}{0.871127in}}%
\pgfpathlineto{\pgfqpoint{0.947359in}{0.884054in}}%
\pgfpathlineto{\pgfqpoint{0.947877in}{0.896980in}}%
\pgfpathlineto{\pgfqpoint{0.950619in}{0.909907in}}%
\pgfpathlineto{\pgfqpoint{0.954936in}{0.915150in}}%
\pgfpathlineto{\pgfqpoint{0.968205in}{0.918546in}}%
\pgfpathlineto{\pgfqpoint{0.981473in}{0.918372in}}%
\pgfpathlineto{\pgfqpoint{0.994742in}{0.913105in}}%
\pgfpathlineto{\pgfqpoint{0.996685in}{0.909907in}}%
\pgfpathlineto{\pgfqpoint{0.999190in}{0.896980in}}%
\pgfpathlineto{\pgfqpoint{0.999654in}{0.884054in}}%
\pgfpathlineto{\pgfqpoint{0.999247in}{0.871127in}}%
\pgfpathlineto{\pgfqpoint{0.997247in}{0.858201in}}%
\pgfpathlineto{\pgfqpoint{0.994742in}{0.853101in}}%
\pgfpathlineto{\pgfqpoint{0.981473in}{0.846919in}}%
\pgfpathlineto{\pgfqpoint{0.968205in}{0.846694in}}%
\pgfpathlineto{\pgfqpoint{0.954936in}{0.850668in}}%
\pgfpathclose%
\pgfpathmoveto{\pgfqpoint{1.067159in}{0.858201in}}%
\pgfpathlineto{\pgfqpoint{1.065210in}{0.871127in}}%
\pgfpathlineto{\pgfqpoint{1.064808in}{0.884054in}}%
\pgfpathlineto{\pgfqpoint{1.065251in}{0.896980in}}%
\pgfpathlineto{\pgfqpoint{1.067649in}{0.909907in}}%
\pgfpathlineto{\pgfqpoint{1.074356in}{0.916979in}}%
\pgfpathlineto{\pgfqpoint{1.087625in}{0.919203in}}%
\pgfpathlineto{\pgfqpoint{1.100894in}{0.918757in}}%
\pgfpathlineto{\pgfqpoint{1.114163in}{0.912229in}}%
\pgfpathlineto{\pgfqpoint{1.115304in}{0.909907in}}%
\pgfpathlineto{\pgfqpoint{1.117522in}{0.896980in}}%
\pgfpathlineto{\pgfqpoint{1.117929in}{0.884054in}}%
\pgfpathlineto{\pgfqpoint{1.117553in}{0.871127in}}%
\pgfpathlineto{\pgfqpoint{1.115741in}{0.858201in}}%
\pgfpathlineto{\pgfqpoint{1.114163in}{0.854278in}}%
\pgfpathlineto{\pgfqpoint{1.100894in}{0.846559in}}%
\pgfpathlineto{\pgfqpoint{1.087625in}{0.846005in}}%
\pgfpathlineto{\pgfqpoint{1.074356in}{0.848692in}}%
\pgfpathclose%
\pgfpathmoveto{\pgfqpoint{1.125394in}{0.858201in}}%
\pgfpathlineto{\pgfqpoint{1.123889in}{0.871127in}}%
\pgfpathlineto{\pgfqpoint{1.123696in}{0.884054in}}%
\pgfpathlineto{\pgfqpoint{1.124349in}{0.896980in}}%
\pgfpathlineto{\pgfqpoint{1.127256in}{0.909907in}}%
\pgfpathlineto{\pgfqpoint{1.127432in}{0.910239in}}%
\pgfpathlineto{\pgfqpoint{1.140701in}{0.917744in}}%
\pgfpathlineto{\pgfqpoint{1.153970in}{0.918289in}}%
\pgfpathlineto{\pgfqpoint{1.167239in}{0.915589in}}%
\pgfpathlineto{\pgfqpoint{1.172722in}{0.909907in}}%
\pgfpathlineto{\pgfqpoint{1.175772in}{0.896980in}}%
\pgfpathlineto{\pgfqpoint{1.176457in}{0.884054in}}%
\pgfpathlineto{\pgfqpoint{1.176254in}{0.871127in}}%
\pgfpathlineto{\pgfqpoint{1.174675in}{0.858201in}}%
\pgfpathlineto{\pgfqpoint{1.167239in}{0.847884in}}%
\pgfpathlineto{\pgfqpoint{1.153970in}{0.845430in}}%
\pgfpathlineto{\pgfqpoint{1.140701in}{0.845920in}}%
\pgfpathlineto{\pgfqpoint{1.127432in}{0.852932in}}%
\pgfpathclose%
\pgfpathmoveto{\pgfqpoint{1.184953in}{0.858201in}}%
\pgfpathlineto{\pgfqpoint{1.182985in}{0.871127in}}%
\pgfpathlineto{\pgfqpoint{1.182577in}{0.884054in}}%
\pgfpathlineto{\pgfqpoint{1.183018in}{0.896980in}}%
\pgfpathlineto{\pgfqpoint{1.185427in}{0.909907in}}%
\pgfpathlineto{\pgfqpoint{1.193777in}{0.917623in}}%
\pgfpathlineto{\pgfqpoint{1.207045in}{0.919262in}}%
\pgfpathlineto{\pgfqpoint{1.220314in}{0.918483in}}%
\pgfpathlineto{\pgfqpoint{1.232985in}{0.909907in}}%
\pgfpathlineto{\pgfqpoint{1.233583in}{0.908166in}}%
\pgfpathlineto{\pgfqpoint{1.235381in}{0.896980in}}%
\pgfpathlineto{\pgfqpoint{1.235803in}{0.884054in}}%
\pgfpathlineto{\pgfqpoint{1.235419in}{0.871127in}}%
\pgfpathlineto{\pgfqpoint{1.233583in}{0.858301in}}%
\pgfpathlineto{\pgfqpoint{1.233557in}{0.858201in}}%
\pgfpathlineto{\pgfqpoint{1.220314in}{0.846881in}}%
\pgfpathlineto{\pgfqpoint{1.207045in}{0.845936in}}%
\pgfpathlineto{\pgfqpoint{1.193777in}{0.847939in}}%
\pgfpathclose%
\pgfpathmoveto{\pgfqpoint{1.303386in}{0.858201in}}%
\pgfpathlineto{\pgfqpoint{1.301136in}{0.871127in}}%
\pgfpathlineto{\pgfqpoint{1.300678in}{0.884054in}}%
\pgfpathlineto{\pgfqpoint{1.301200in}{0.896980in}}%
\pgfpathlineto{\pgfqpoint{1.304020in}{0.909907in}}%
\pgfpathlineto{\pgfqpoint{1.313197in}{0.917385in}}%
\pgfpathlineto{\pgfqpoint{1.326466in}{0.918745in}}%
\pgfpathlineto{\pgfqpoint{1.339735in}{0.917439in}}%
\pgfpathlineto{\pgfqpoint{1.349600in}{0.909907in}}%
\pgfpathlineto{\pgfqpoint{1.352706in}{0.896980in}}%
\pgfpathlineto{\pgfqpoint{1.353004in}{0.890965in}}%
\pgfpathlineto{\pgfqpoint{1.353255in}{0.884054in}}%
\pgfpathlineto{\pgfqpoint{1.353004in}{0.876240in}}%
\pgfpathlineto{\pgfqpoint{1.352796in}{0.871127in}}%
\pgfpathlineto{\pgfqpoint{1.350345in}{0.858201in}}%
\pgfpathlineto{\pgfqpoint{1.339735in}{0.847999in}}%
\pgfpathlineto{\pgfqpoint{1.326466in}{0.846462in}}%
\pgfpathlineto{\pgfqpoint{1.313197in}{0.848103in}}%
\pgfpathclose%
\pgfpathmoveto{\pgfqpoint{1.422540in}{0.858201in}}%
\pgfpathlineto{\pgfqpoint{1.419701in}{0.871127in}}%
\pgfpathlineto{\pgfqpoint{1.419348in}{0.878954in}}%
\pgfpathlineto{\pgfqpoint{1.419170in}{0.884054in}}%
\pgfpathlineto{\pgfqpoint{1.419348in}{0.888400in}}%
\pgfpathlineto{\pgfqpoint{1.419837in}{0.896980in}}%
\pgfpathlineto{\pgfqpoint{1.423540in}{0.909907in}}%
\pgfpathlineto{\pgfqpoint{1.432617in}{0.916448in}}%
\pgfpathlineto{\pgfqpoint{1.445886in}{0.917651in}}%
\pgfpathlineto{\pgfqpoint{1.459155in}{0.915451in}}%
\pgfpathlineto{\pgfqpoint{1.465583in}{0.909907in}}%
\pgfpathlineto{\pgfqpoint{1.469341in}{0.896980in}}%
\pgfpathlineto{\pgfqpoint{1.470069in}{0.884054in}}%
\pgfpathlineto{\pgfqpoint{1.469505in}{0.871127in}}%
\pgfpathlineto{\pgfqpoint{1.466647in}{0.858201in}}%
\pgfpathlineto{\pgfqpoint{1.459155in}{0.850089in}}%
\pgfpathlineto{\pgfqpoint{1.445886in}{0.847584in}}%
\pgfpathlineto{\pgfqpoint{1.432617in}{0.848997in}}%
\pgfpathclose%
\pgfpathmoveto{\pgfqpoint{0.661861in}{0.935760in}}%
\pgfpathlineto{\pgfqpoint{0.656306in}{0.948686in}}%
\pgfpathlineto{\pgfqpoint{0.655100in}{0.961613in}}%
\pgfpathlineto{\pgfqpoint{0.655605in}{0.974539in}}%
\pgfpathlineto{\pgfqpoint{0.658842in}{0.987466in}}%
\pgfpathlineto{\pgfqpoint{0.663019in}{0.992692in}}%
\pgfpathlineto{\pgfqpoint{0.676288in}{0.996716in}}%
\pgfpathlineto{\pgfqpoint{0.689557in}{0.995514in}}%
\pgfpathlineto{\pgfqpoint{0.699081in}{0.987466in}}%
\pgfpathlineto{\pgfqpoint{0.702542in}{0.974539in}}%
\pgfpathlineto{\pgfqpoint{0.702826in}{0.967826in}}%
\pgfpathlineto{\pgfqpoint{0.703019in}{0.961613in}}%
\pgfpathlineto{\pgfqpoint{0.702826in}{0.958840in}}%
\pgfpathlineto{\pgfqpoint{0.701781in}{0.948686in}}%
\pgfpathlineto{\pgfqpoint{0.695689in}{0.935760in}}%
\pgfpathlineto{\pgfqpoint{0.689557in}{0.931932in}}%
\pgfpathlineto{\pgfqpoint{0.676288in}{0.930723in}}%
\pgfpathlineto{\pgfqpoint{0.663019in}{0.934679in}}%
\pgfpathclose%
\pgfpathmoveto{\pgfqpoint{0.709483in}{0.935760in}}%
\pgfpathlineto{\pgfqpoint{0.708947in}{0.948686in}}%
\pgfpathlineto{\pgfqpoint{0.708871in}{0.961613in}}%
\pgfpathlineto{\pgfqpoint{0.709014in}{0.974539in}}%
\pgfpathlineto{\pgfqpoint{0.709601in}{0.987466in}}%
\pgfpathlineto{\pgfqpoint{0.715269in}{1.000392in}}%
\pgfpathlineto{\pgfqpoint{0.716095in}{1.000745in}}%
\pgfpathlineto{\pgfqpoint{0.729364in}{1.002357in}}%
\pgfpathlineto{\pgfqpoint{0.742633in}{1.002248in}}%
\pgfpathlineto{\pgfqpoint{0.755902in}{1.000497in}}%
\pgfpathlineto{\pgfqpoint{0.756186in}{1.000392in}}%
\pgfpathlineto{\pgfqpoint{0.764242in}{0.987466in}}%
\pgfpathlineto{\pgfqpoint{0.765183in}{0.974539in}}%
\pgfpathlineto{\pgfqpoint{0.765388in}{0.961613in}}%
\pgfpathlineto{\pgfqpoint{0.765191in}{0.948686in}}%
\pgfpathlineto{\pgfqpoint{0.764091in}{0.935760in}}%
\pgfpathlineto{\pgfqpoint{0.755902in}{0.925798in}}%
\pgfpathlineto{\pgfqpoint{0.742633in}{0.924354in}}%
\pgfpathlineto{\pgfqpoint{0.729364in}{0.924209in}}%
\pgfpathlineto{\pgfqpoint{0.716095in}{0.925236in}}%
\pgfpathclose%
\pgfpathmoveto{\pgfqpoint{0.776843in}{0.935760in}}%
\pgfpathlineto{\pgfqpoint{0.772409in}{0.948686in}}%
\pgfpathlineto{\pgfqpoint{0.771441in}{0.961613in}}%
\pgfpathlineto{\pgfqpoint{0.771821in}{0.974539in}}%
\pgfpathlineto{\pgfqpoint{0.774336in}{0.987466in}}%
\pgfpathlineto{\pgfqpoint{0.782439in}{0.996427in}}%
\pgfpathlineto{\pgfqpoint{0.795708in}{0.998766in}}%
\pgfpathlineto{\pgfqpoint{0.808977in}{0.997439in}}%
\pgfpathlineto{\pgfqpoint{0.819440in}{0.987466in}}%
\pgfpathlineto{\pgfqpoint{0.821842in}{0.974539in}}%
\pgfpathlineto{\pgfqpoint{0.822193in}{0.961613in}}%
\pgfpathlineto{\pgfqpoint{0.821263in}{0.948686in}}%
\pgfpathlineto{\pgfqpoint{0.816920in}{0.935760in}}%
\pgfpathlineto{\pgfqpoint{0.808977in}{0.930160in}}%
\pgfpathlineto{\pgfqpoint{0.795708in}{0.928795in}}%
\pgfpathlineto{\pgfqpoint{0.782439in}{0.931145in}}%
\pgfpathclose%
\pgfpathmoveto{\pgfqpoint{0.830100in}{0.935760in}}%
\pgfpathlineto{\pgfqpoint{0.828403in}{0.948686in}}%
\pgfpathlineto{\pgfqpoint{0.828089in}{0.961613in}}%
\pgfpathlineto{\pgfqpoint{0.828354in}{0.974539in}}%
\pgfpathlineto{\pgfqpoint{0.829664in}{0.987466in}}%
\pgfpathlineto{\pgfqpoint{0.835515in}{0.997676in}}%
\pgfpathlineto{\pgfqpoint{0.846211in}{1.000392in}}%
\pgfpathlineto{\pgfqpoint{0.848784in}{1.000711in}}%
\pgfpathlineto{\pgfqpoint{0.862053in}{1.000554in}}%
\pgfpathlineto{\pgfqpoint{0.863147in}{1.000392in}}%
\pgfpathlineto{\pgfqpoint{0.875322in}{0.996699in}}%
\pgfpathlineto{\pgfqpoint{0.880454in}{0.987466in}}%
\pgfpathlineto{\pgfqpoint{0.881973in}{0.974539in}}%
\pgfpathlineto{\pgfqpoint{0.882274in}{0.961613in}}%
\pgfpathlineto{\pgfqpoint{0.881884in}{0.948686in}}%
\pgfpathlineto{\pgfqpoint{0.879857in}{0.935760in}}%
\pgfpathlineto{\pgfqpoint{0.875322in}{0.929398in}}%
\pgfpathlineto{\pgfqpoint{0.862053in}{0.926102in}}%
\pgfpathlineto{\pgfqpoint{0.848784in}{0.925939in}}%
\pgfpathlineto{\pgfqpoint{0.835515in}{0.928444in}}%
\pgfpathclose%
\pgfpathmoveto{\pgfqpoint{0.892390in}{0.935760in}}%
\pgfpathlineto{\pgfqpoint{0.888778in}{0.948686in}}%
\pgfpathlineto{\pgfqpoint{0.888591in}{0.951348in}}%
\pgfpathlineto{\pgfqpoint{0.888066in}{0.961613in}}%
\pgfpathlineto{\pgfqpoint{0.888313in}{0.974539in}}%
\pgfpathlineto{\pgfqpoint{0.888591in}{0.977472in}}%
\pgfpathlineto{\pgfqpoint{0.890249in}{0.987466in}}%
\pgfpathlineto{\pgfqpoint{0.901860in}{0.998858in}}%
\pgfpathlineto{\pgfqpoint{0.915129in}{1.000189in}}%
\pgfpathlineto{\pgfqpoint{0.928398in}{0.998791in}}%
\pgfpathlineto{\pgfqpoint{0.938921in}{0.987466in}}%
\pgfpathlineto{\pgfqpoint{0.940643in}{0.974539in}}%
\pgfpathlineto{\pgfqpoint{0.940884in}{0.961613in}}%
\pgfpathlineto{\pgfqpoint{0.940183in}{0.948686in}}%
\pgfpathlineto{\pgfqpoint{0.936966in}{0.935760in}}%
\pgfpathlineto{\pgfqpoint{0.928398in}{0.928936in}}%
\pgfpathlineto{\pgfqpoint{0.915129in}{0.927458in}}%
\pgfpathlineto{\pgfqpoint{0.901860in}{0.928831in}}%
\pgfpathclose%
\pgfpathmoveto{\pgfqpoint{0.949911in}{0.935760in}}%
\pgfpathlineto{\pgfqpoint{0.947397in}{0.948686in}}%
\pgfpathlineto{\pgfqpoint{0.946914in}{0.961613in}}%
\pgfpathlineto{\pgfqpoint{0.947266in}{0.974539in}}%
\pgfpathlineto{\pgfqpoint{0.949090in}{0.987466in}}%
\pgfpathlineto{\pgfqpoint{0.954936in}{0.996330in}}%
\pgfpathlineto{\pgfqpoint{0.968205in}{0.999633in}}%
\pgfpathlineto{\pgfqpoint{0.981473in}{0.999233in}}%
\pgfpathlineto{\pgfqpoint{0.994742in}{0.992927in}}%
\pgfpathlineto{\pgfqpoint{0.997326in}{0.987466in}}%
\pgfpathlineto{\pgfqpoint{0.999216in}{0.974539in}}%
\pgfpathlineto{\pgfqpoint{0.999580in}{0.961613in}}%
\pgfpathlineto{\pgfqpoint{0.999067in}{0.948686in}}%
\pgfpathlineto{\pgfqpoint{0.996442in}{0.935760in}}%
\pgfpathlineto{\pgfqpoint{0.994742in}{0.932949in}}%
\pgfpathlineto{\pgfqpoint{0.981473in}{0.927372in}}%
\pgfpathlineto{\pgfqpoint{0.968205in}{0.927016in}}%
\pgfpathlineto{\pgfqpoint{0.954936in}{0.929873in}}%
\pgfpathclose%
\pgfpathmoveto{\pgfqpoint{1.008525in}{0.935760in}}%
\pgfpathlineto{\pgfqpoint{1.008011in}{0.937063in}}%
\pgfpathlineto{\pgfqpoint{1.005898in}{0.948686in}}%
\pgfpathlineto{\pgfqpoint{1.005337in}{0.961613in}}%
\pgfpathlineto{\pgfqpoint{1.005520in}{0.974539in}}%
\pgfpathlineto{\pgfqpoint{1.006858in}{0.987466in}}%
\pgfpathlineto{\pgfqpoint{1.008011in}{0.991285in}}%
\pgfpathlineto{\pgfqpoint{1.021280in}{1.000270in}}%
\pgfpathlineto{\pgfqpoint{1.023159in}{1.000392in}}%
\pgfpathlineto{\pgfqpoint{1.034549in}{1.000968in}}%
\pgfpathlineto{\pgfqpoint{1.041146in}{1.000392in}}%
\pgfpathlineto{\pgfqpoint{1.047818in}{0.999468in}}%
\pgfpathlineto{\pgfqpoint{1.057662in}{0.987466in}}%
\pgfpathlineto{\pgfqpoint{1.059015in}{0.974539in}}%
\pgfpathlineto{\pgfqpoint{1.059195in}{0.961613in}}%
\pgfpathlineto{\pgfqpoint{1.058622in}{0.948686in}}%
\pgfpathlineto{\pgfqpoint{1.056023in}{0.935760in}}%
\pgfpathlineto{\pgfqpoint{1.047818in}{0.928357in}}%
\pgfpathlineto{\pgfqpoint{1.034549in}{0.926687in}}%
\pgfpathlineto{\pgfqpoint{1.021280in}{0.927476in}}%
\pgfpathclose%
\pgfpathmoveto{\pgfqpoint{1.068911in}{0.935760in}}%
\pgfpathlineto{\pgfqpoint{1.065913in}{0.948686in}}%
\pgfpathlineto{\pgfqpoint{1.065329in}{0.961613in}}%
\pgfpathlineto{\pgfqpoint{1.065734in}{0.974539in}}%
\pgfpathlineto{\pgfqpoint{1.067867in}{0.987466in}}%
\pgfpathlineto{\pgfqpoint{1.074356in}{0.996136in}}%
\pgfpathlineto{\pgfqpoint{1.087625in}{0.999129in}}%
\pgfpathlineto{\pgfqpoint{1.100894in}{0.998415in}}%
\pgfpathlineto{\pgfqpoint{1.114163in}{0.989127in}}%
\pgfpathlineto{\pgfqpoint{1.114812in}{0.987466in}}%
\pgfpathlineto{\pgfqpoint{1.116874in}{0.974539in}}%
\pgfpathlineto{\pgfqpoint{1.117265in}{0.961613in}}%
\pgfpathlineto{\pgfqpoint{1.116697in}{0.948686in}}%
\pgfpathlineto{\pgfqpoint{1.114163in}{0.936746in}}%
\pgfpathlineto{\pgfqpoint{1.113681in}{0.935760in}}%
\pgfpathlineto{\pgfqpoint{1.100894in}{0.928140in}}%
\pgfpathlineto{\pgfqpoint{1.087625in}{0.927512in}}%
\pgfpathlineto{\pgfqpoint{1.074356in}{0.930145in}}%
\pgfpathclose%
\pgfpathmoveto{\pgfqpoint{1.125827in}{0.935760in}}%
\pgfpathlineto{\pgfqpoint{1.123535in}{0.948686in}}%
\pgfpathlineto{\pgfqpoint{1.123025in}{0.961613in}}%
\pgfpathlineto{\pgfqpoint{1.123182in}{0.974539in}}%
\pgfpathlineto{\pgfqpoint{1.124368in}{0.987466in}}%
\pgfpathlineto{\pgfqpoint{1.127432in}{0.995345in}}%
\pgfpathlineto{\pgfqpoint{1.138220in}{1.000392in}}%
\pgfpathlineto{\pgfqpoint{1.140701in}{1.000820in}}%
\pgfpathlineto{\pgfqpoint{1.153970in}{1.001181in}}%
\pgfpathlineto{\pgfqpoint{1.161327in}{1.000392in}}%
\pgfpathlineto{\pgfqpoint{1.167239in}{0.999299in}}%
\pgfpathlineto{\pgfqpoint{1.175751in}{0.987466in}}%
\pgfpathlineto{\pgfqpoint{1.176996in}{0.974539in}}%
\pgfpathlineto{\pgfqpoint{1.177161in}{0.961613in}}%
\pgfpathlineto{\pgfqpoint{1.176626in}{0.948686in}}%
\pgfpathlineto{\pgfqpoint{1.174220in}{0.935760in}}%
\pgfpathlineto{\pgfqpoint{1.167239in}{0.928581in}}%
\pgfpathlineto{\pgfqpoint{1.153970in}{0.926478in}}%
\pgfpathlineto{\pgfqpoint{1.140701in}{0.926902in}}%
\pgfpathlineto{\pgfqpoint{1.127432in}{0.932757in}}%
\pgfpathclose%
\pgfpathmoveto{\pgfqpoint{1.187061in}{0.935760in}}%
\pgfpathlineto{\pgfqpoint{1.183914in}{0.948686in}}%
\pgfpathlineto{\pgfqpoint{1.183297in}{0.961613in}}%
\pgfpathlineto{\pgfqpoint{1.183722in}{0.974539in}}%
\pgfpathlineto{\pgfqpoint{1.185958in}{0.987466in}}%
\pgfpathlineto{\pgfqpoint{1.193777in}{0.996760in}}%
\pgfpathlineto{\pgfqpoint{1.207045in}{0.999181in}}%
\pgfpathlineto{\pgfqpoint{1.220314in}{0.998167in}}%
\pgfpathlineto{\pgfqpoint{1.232730in}{0.987466in}}%
\pgfpathlineto{\pgfqpoint{1.233583in}{0.984165in}}%
\pgfpathlineto{\pgfqpoint{1.234921in}{0.974539in}}%
\pgfpathlineto{\pgfqpoint{1.235307in}{0.961613in}}%
\pgfpathlineto{\pgfqpoint{1.234750in}{0.948686in}}%
\pgfpathlineto{\pgfqpoint{1.233583in}{0.941587in}}%
\pgfpathlineto{\pgfqpoint{1.231511in}{0.935760in}}%
\pgfpathlineto{\pgfqpoint{1.220314in}{0.928353in}}%
\pgfpathlineto{\pgfqpoint{1.207045in}{0.927468in}}%
\pgfpathlineto{\pgfqpoint{1.193777in}{0.929601in}}%
\pgfpathclose%
\pgfpathmoveto{\pgfqpoint{1.244052in}{0.935760in}}%
\pgfpathlineto{\pgfqpoint{1.241640in}{0.948686in}}%
\pgfpathlineto{\pgfqpoint{1.241107in}{0.961613in}}%
\pgfpathlineto{\pgfqpoint{1.241275in}{0.974539in}}%
\pgfpathlineto{\pgfqpoint{1.242531in}{0.987466in}}%
\pgfpathlineto{\pgfqpoint{1.246852in}{0.996534in}}%
\pgfpathlineto{\pgfqpoint{1.258007in}{1.000392in}}%
\pgfpathlineto{\pgfqpoint{1.260121in}{1.000707in}}%
\pgfpathlineto{\pgfqpoint{1.273390in}{1.000825in}}%
\pgfpathlineto{\pgfqpoint{1.276773in}{1.000392in}}%
\pgfpathlineto{\pgfqpoint{1.286659in}{0.998000in}}%
\pgfpathlineto{\pgfqpoint{1.293239in}{0.987466in}}%
\pgfpathlineto{\pgfqpoint{1.294608in}{0.974539in}}%
\pgfpathlineto{\pgfqpoint{1.294795in}{0.961613in}}%
\pgfpathlineto{\pgfqpoint{1.294222in}{0.948686in}}%
\pgfpathlineto{\pgfqpoint{1.291633in}{0.935760in}}%
\pgfpathlineto{\pgfqpoint{1.286659in}{0.929877in}}%
\pgfpathlineto{\pgfqpoint{1.273390in}{0.926849in}}%
\pgfpathlineto{\pgfqpoint{1.260121in}{0.926995in}}%
\pgfpathlineto{\pgfqpoint{1.246852in}{0.931461in}}%
\pgfpathclose%
\pgfpathmoveto{\pgfqpoint{1.304284in}{0.935760in}}%
\pgfpathlineto{\pgfqpoint{1.301339in}{0.948686in}}%
\pgfpathlineto{\pgfqpoint{1.300763in}{0.961613in}}%
\pgfpathlineto{\pgfqpoint{1.301171in}{0.974539in}}%
\pgfpathlineto{\pgfqpoint{1.303293in}{0.987466in}}%
\pgfpathlineto{\pgfqpoint{1.313197in}{0.998009in}}%
\pgfpathlineto{\pgfqpoint{1.326466in}{0.999767in}}%
\pgfpathlineto{\pgfqpoint{1.339735in}{0.998577in}}%
\pgfpathlineto{\pgfqpoint{1.351332in}{0.987466in}}%
\pgfpathlineto{\pgfqpoint{1.353004in}{0.977970in}}%
\pgfpathlineto{\pgfqpoint{1.353349in}{0.974539in}}%
\pgfpathlineto{\pgfqpoint{1.353696in}{0.961613in}}%
\pgfpathlineto{\pgfqpoint{1.353220in}{0.948686in}}%
\pgfpathlineto{\pgfqpoint{1.353004in}{0.946816in}}%
\pgfpathlineto{\pgfqpoint{1.350400in}{0.935760in}}%
\pgfpathlineto{\pgfqpoint{1.339735in}{0.927921in}}%
\pgfpathlineto{\pgfqpoint{1.326466in}{0.926905in}}%
\pgfpathlineto{\pgfqpoint{1.313197in}{0.928440in}}%
\pgfpathclose%
\pgfpathmoveto{\pgfqpoint{1.363113in}{0.935760in}}%
\pgfpathlineto{\pgfqpoint{1.360208in}{0.948686in}}%
\pgfpathlineto{\pgfqpoint{1.359574in}{0.961613in}}%
\pgfpathlineto{\pgfqpoint{1.359792in}{0.974539in}}%
\pgfpathlineto{\pgfqpoint{1.361348in}{0.987466in}}%
\pgfpathlineto{\pgfqpoint{1.366273in}{0.996159in}}%
\pgfpathlineto{\pgfqpoint{1.379542in}{0.999982in}}%
\pgfpathlineto{\pgfqpoint{1.392811in}{0.999815in}}%
\pgfpathlineto{\pgfqpoint{1.406080in}{0.995076in}}%
\pgfpathlineto{\pgfqpoint{1.410146in}{0.987466in}}%
\pgfpathlineto{\pgfqpoint{1.411853in}{0.974539in}}%
\pgfpathlineto{\pgfqpoint{1.412099in}{0.961613in}}%
\pgfpathlineto{\pgfqpoint{1.411414in}{0.948686in}}%
\pgfpathlineto{\pgfqpoint{1.408299in}{0.935760in}}%
\pgfpathlineto{\pgfqpoint{1.406080in}{0.932698in}}%
\pgfpathlineto{\pgfqpoint{1.392811in}{0.927842in}}%
\pgfpathlineto{\pgfqpoint{1.379542in}{0.927682in}}%
\pgfpathlineto{\pgfqpoint{1.366273in}{0.931667in}}%
\pgfpathclose%
\pgfpathmoveto{\pgfqpoint{1.420457in}{0.935760in}}%
\pgfpathlineto{\pgfqpoint{1.419348in}{0.940109in}}%
\pgfpathlineto{\pgfqpoint{1.418274in}{0.948686in}}%
\pgfpathlineto{\pgfqpoint{1.417885in}{0.961613in}}%
\pgfpathlineto{\pgfqpoint{1.418186in}{0.974539in}}%
\pgfpathlineto{\pgfqpoint{1.419348in}{0.985285in}}%
\pgfpathlineto{\pgfqpoint{1.419763in}{0.987466in}}%
\pgfpathlineto{\pgfqpoint{1.432617in}{0.999768in}}%
\pgfpathlineto{\pgfqpoint{1.439232in}{1.000392in}}%
\pgfpathlineto{\pgfqpoint{1.445886in}{1.000856in}}%
\pgfpathlineto{\pgfqpoint{1.452912in}{1.000392in}}%
\pgfpathlineto{\pgfqpoint{1.459155in}{0.999785in}}%
\pgfpathlineto{\pgfqpoint{1.470697in}{0.987466in}}%
\pgfpathlineto{\pgfqpoint{1.472143in}{0.974539in}}%
\pgfpathlineto{\pgfqpoint{1.472424in}{0.962106in}}%
\pgfpathlineto{\pgfqpoint{1.472434in}{0.961613in}}%
\pgfpathlineto{\pgfqpoint{1.472424in}{0.961163in}}%
\pgfpathlineto{\pgfqpoint{1.472088in}{0.948686in}}%
\pgfpathlineto{\pgfqpoint{1.470215in}{0.935760in}}%
\pgfpathlineto{\pgfqpoint{1.459155in}{0.926706in}}%
\pgfpathlineto{\pgfqpoint{1.445886in}{0.925820in}}%
\pgfpathlineto{\pgfqpoint{1.432617in}{0.926777in}}%
\pgfpathclose%
\pgfpathmoveto{\pgfqpoint{0.613652in}{1.013319in}}%
\pgfpathlineto{\pgfqpoint{0.609943in}{1.013810in}}%
\pgfpathlineto{\pgfqpoint{0.598815in}{1.026245in}}%
\pgfpathlineto{\pgfqpoint{0.596674in}{1.037180in}}%
\pgfpathlineto{\pgfqpoint{0.596442in}{1.039172in}}%
\pgfpathlineto{\pgfqpoint{0.596523in}{1.052098in}}%
\pgfpathlineto{\pgfqpoint{0.596674in}{1.053267in}}%
\pgfpathlineto{\pgfqpoint{0.599352in}{1.065025in}}%
\pgfpathlineto{\pgfqpoint{0.609943in}{1.075748in}}%
\pgfpathlineto{\pgfqpoint{0.623212in}{1.077276in}}%
\pgfpathlineto{\pgfqpoint{0.636481in}{1.072368in}}%
\pgfpathlineto{\pgfqpoint{0.640829in}{1.065025in}}%
\pgfpathlineto{\pgfqpoint{0.642982in}{1.052098in}}%
\pgfpathlineto{\pgfqpoint{0.643057in}{1.039172in}}%
\pgfpathlineto{\pgfqpoint{0.641223in}{1.026245in}}%
\pgfpathlineto{\pgfqpoint{0.636481in}{1.017411in}}%
\pgfpathlineto{\pgfqpoint{0.627221in}{1.013319in}}%
\pgfpathlineto{\pgfqpoint{0.623212in}{1.012389in}}%
\pgfpathclose%
\pgfpathmoveto{\pgfqpoint{0.649744in}{1.013319in}}%
\pgfpathlineto{\pgfqpoint{0.649395in}{1.026245in}}%
\pgfpathlineto{\pgfqpoint{0.649343in}{1.039172in}}%
\pgfpathlineto{\pgfqpoint{0.649382in}{1.052098in}}%
\pgfpathlineto{\pgfqpoint{0.649557in}{1.065025in}}%
\pgfpathlineto{\pgfqpoint{0.649750in}{1.070157in}}%
\pgfpathlineto{\pgfqpoint{0.650531in}{1.077952in}}%
\pgfpathlineto{\pgfqpoint{0.663019in}{1.084168in}}%
\pgfpathlineto{\pgfqpoint{0.676288in}{1.084300in}}%
\pgfpathlineto{\pgfqpoint{0.689557in}{1.083825in}}%
\pgfpathlineto{\pgfqpoint{0.702826in}{1.079937in}}%
\pgfpathlineto{\pgfqpoint{0.704187in}{1.077952in}}%
\pgfpathlineto{\pgfqpoint{0.706511in}{1.065025in}}%
\pgfpathlineto{\pgfqpoint{0.706941in}{1.052098in}}%
\pgfpathlineto{\pgfqpoint{0.706991in}{1.039172in}}%
\pgfpathlineto{\pgfqpoint{0.706724in}{1.026245in}}%
\pgfpathlineto{\pgfqpoint{0.705310in}{1.013319in}}%
\pgfpathlineto{\pgfqpoint{0.702826in}{1.008760in}}%
\pgfpathlineto{\pgfqpoint{0.689557in}{1.005390in}}%
\pgfpathlineto{\pgfqpoint{0.676288in}{1.004959in}}%
\pgfpathlineto{\pgfqpoint{0.663019in}{1.004973in}}%
\pgfpathlineto{\pgfqpoint{0.649750in}{1.013239in}}%
\pgfpathclose%
\pgfpathmoveto{\pgfqpoint{0.723371in}{1.013319in}}%
\pgfpathlineto{\pgfqpoint{0.716095in}{1.020687in}}%
\pgfpathlineto{\pgfqpoint{0.714242in}{1.026245in}}%
\pgfpathlineto{\pgfqpoint{0.712822in}{1.039172in}}%
\pgfpathlineto{\pgfqpoint{0.712875in}{1.052098in}}%
\pgfpathlineto{\pgfqpoint{0.714522in}{1.065025in}}%
\pgfpathlineto{\pgfqpoint{0.716095in}{1.069337in}}%
\pgfpathlineto{\pgfqpoint{0.726750in}{1.077952in}}%
\pgfpathlineto{\pgfqpoint{0.729364in}{1.078784in}}%
\pgfpathlineto{\pgfqpoint{0.742633in}{1.079371in}}%
\pgfpathlineto{\pgfqpoint{0.749091in}{1.077952in}}%
\pgfpathlineto{\pgfqpoint{0.755902in}{1.074930in}}%
\pgfpathlineto{\pgfqpoint{0.761006in}{1.065025in}}%
\pgfpathlineto{\pgfqpoint{0.762535in}{1.052098in}}%
\pgfpathlineto{\pgfqpoint{0.762580in}{1.039172in}}%
\pgfpathlineto{\pgfqpoint{0.761253in}{1.026245in}}%
\pgfpathlineto{\pgfqpoint{0.755902in}{1.014807in}}%
\pgfpathlineto{\pgfqpoint{0.753297in}{1.013319in}}%
\pgfpathlineto{\pgfqpoint{0.742633in}{1.010395in}}%
\pgfpathlineto{\pgfqpoint{0.729364in}{1.010995in}}%
\pgfpathclose%
\pgfpathmoveto{\pgfqpoint{0.772155in}{1.013319in}}%
\pgfpathlineto{\pgfqpoint{0.769204in}{1.026245in}}%
\pgfpathlineto{\pgfqpoint{0.769170in}{1.026860in}}%
\pgfpathlineto{\pgfqpoint{0.768700in}{1.039172in}}%
\pgfpathlineto{\pgfqpoint{0.768766in}{1.052098in}}%
\pgfpathlineto{\pgfqpoint{0.769170in}{1.060271in}}%
\pgfpathlineto{\pgfqpoint{0.769520in}{1.065025in}}%
\pgfpathlineto{\pgfqpoint{0.774105in}{1.077952in}}%
\pgfpathlineto{\pgfqpoint{0.782439in}{1.081683in}}%
\pgfpathlineto{\pgfqpoint{0.795708in}{1.082472in}}%
\pgfpathlineto{\pgfqpoint{0.808977in}{1.081476in}}%
\pgfpathlineto{\pgfqpoint{0.817112in}{1.077952in}}%
\pgfpathlineto{\pgfqpoint{0.822246in}{1.067634in}}%
\pgfpathlineto{\pgfqpoint{0.822705in}{1.065025in}}%
\pgfpathlineto{\pgfqpoint{0.823595in}{1.052098in}}%
\pgfpathlineto{\pgfqpoint{0.823668in}{1.039172in}}%
\pgfpathlineto{\pgfqpoint{0.823028in}{1.026245in}}%
\pgfpathlineto{\pgfqpoint{0.822246in}{1.021090in}}%
\pgfpathlineto{\pgfqpoint{0.819340in}{1.013319in}}%
\pgfpathlineto{\pgfqpoint{0.808977in}{1.007751in}}%
\pgfpathlineto{\pgfqpoint{0.795708in}{1.006809in}}%
\pgfpathlineto{\pgfqpoint{0.782439in}{1.007521in}}%
\pgfpathclose%
\pgfpathmoveto{\pgfqpoint{0.836297in}{1.013319in}}%
\pgfpathlineto{\pgfqpoint{0.835515in}{1.013933in}}%
\pgfpathlineto{\pgfqpoint{0.830650in}{1.026245in}}%
\pgfpathlineto{\pgfqpoint{0.829573in}{1.039172in}}%
\pgfpathlineto{\pgfqpoint{0.829604in}{1.052098in}}%
\pgfpathlineto{\pgfqpoint{0.830828in}{1.065025in}}%
\pgfpathlineto{\pgfqpoint{0.835515in}{1.075834in}}%
\pgfpathlineto{\pgfqpoint{0.838906in}{1.077952in}}%
\pgfpathlineto{\pgfqpoint{0.848784in}{1.080708in}}%
\pgfpathlineto{\pgfqpoint{0.862053in}{1.080860in}}%
\pgfpathlineto{\pgfqpoint{0.873396in}{1.077952in}}%
\pgfpathlineto{\pgfqpoint{0.875322in}{1.076843in}}%
\pgfpathlineto{\pgfqpoint{0.880547in}{1.065025in}}%
\pgfpathlineto{\pgfqpoint{0.881641in}{1.052098in}}%
\pgfpathlineto{\pgfqpoint{0.881665in}{1.039172in}}%
\pgfpathlineto{\pgfqpoint{0.880693in}{1.026245in}}%
\pgfpathlineto{\pgfqpoint{0.875696in}{1.013319in}}%
\pgfpathlineto{\pgfqpoint{0.875322in}{1.012969in}}%
\pgfpathlineto{\pgfqpoint{0.862053in}{1.008916in}}%
\pgfpathlineto{\pgfqpoint{0.848784in}{1.009067in}}%
\pgfpathclose%
\pgfpathmoveto{\pgfqpoint{0.893459in}{1.013319in}}%
\pgfpathlineto{\pgfqpoint{0.888591in}{1.025526in}}%
\pgfpathlineto{\pgfqpoint{0.888478in}{1.026245in}}%
\pgfpathlineto{\pgfqpoint{0.887667in}{1.039172in}}%
\pgfpathlineto{\pgfqpoint{0.887751in}{1.052098in}}%
\pgfpathlineto{\pgfqpoint{0.888591in}{1.062585in}}%
\pgfpathlineto{\pgfqpoint{0.888899in}{1.065025in}}%
\pgfpathlineto{\pgfqpoint{0.896320in}{1.077952in}}%
\pgfpathlineto{\pgfqpoint{0.901860in}{1.080192in}}%
\pgfpathlineto{\pgfqpoint{0.915129in}{1.081166in}}%
\pgfpathlineto{\pgfqpoint{0.928398in}{1.079503in}}%
\pgfpathlineto{\pgfqpoint{0.931639in}{1.077952in}}%
\pgfpathlineto{\pgfqpoint{0.939117in}{1.065025in}}%
\pgfpathlineto{\pgfqpoint{0.940457in}{1.052098in}}%
\pgfpathlineto{\pgfqpoint{0.940555in}{1.039172in}}%
\pgfpathlineto{\pgfqpoint{0.939560in}{1.026245in}}%
\pgfpathlineto{\pgfqpoint{0.934425in}{1.013319in}}%
\pgfpathlineto{\pgfqpoint{0.928398in}{1.009730in}}%
\pgfpathlineto{\pgfqpoint{0.915129in}{1.008129in}}%
\pgfpathlineto{\pgfqpoint{0.901860in}{1.009053in}}%
\pgfpathclose%
\pgfpathmoveto{\pgfqpoint{0.951729in}{1.013319in}}%
\pgfpathlineto{\pgfqpoint{0.947509in}{1.026245in}}%
\pgfpathlineto{\pgfqpoint{0.946666in}{1.039172in}}%
\pgfpathlineto{\pgfqpoint{0.946683in}{1.052098in}}%
\pgfpathlineto{\pgfqpoint{0.947616in}{1.065025in}}%
\pgfpathlineto{\pgfqpoint{0.952973in}{1.077952in}}%
\pgfpathlineto{\pgfqpoint{0.954936in}{1.079299in}}%
\pgfpathlineto{\pgfqpoint{0.968205in}{1.081941in}}%
\pgfpathlineto{\pgfqpoint{0.981473in}{1.081825in}}%
\pgfpathlineto{\pgfqpoint{0.994599in}{1.077952in}}%
\pgfpathlineto{\pgfqpoint{0.994742in}{1.077844in}}%
\pgfpathlineto{\pgfqpoint{0.999504in}{1.065025in}}%
\pgfpathlineto{\pgfqpoint{1.000330in}{1.052098in}}%
\pgfpathlineto{\pgfqpoint{1.000341in}{1.039172in}}%
\pgfpathlineto{\pgfqpoint{0.999588in}{1.026245in}}%
\pgfpathlineto{\pgfqpoint{0.995752in}{1.013319in}}%
\pgfpathlineto{\pgfqpoint{0.994742in}{1.012190in}}%
\pgfpathlineto{\pgfqpoint{0.981473in}{1.007960in}}%
\pgfpathlineto{\pgfqpoint{0.968205in}{1.007830in}}%
\pgfpathlineto{\pgfqpoint{0.954936in}{1.010632in}}%
\pgfpathclose%
\pgfpathmoveto{\pgfqpoint{1.013582in}{1.013319in}}%
\pgfpathlineto{\pgfqpoint{1.008011in}{1.022815in}}%
\pgfpathlineto{\pgfqpoint{1.007266in}{1.026245in}}%
\pgfpathlineto{\pgfqpoint{1.006267in}{1.039172in}}%
\pgfpathlineto{\pgfqpoint{1.006362in}{1.052098in}}%
\pgfpathlineto{\pgfqpoint{1.007697in}{1.065025in}}%
\pgfpathlineto{\pgfqpoint{1.008011in}{1.066285in}}%
\pgfpathlineto{\pgfqpoint{1.017105in}{1.077952in}}%
\pgfpathlineto{\pgfqpoint{1.021280in}{1.079481in}}%
\pgfpathlineto{\pgfqpoint{1.034549in}{1.080357in}}%
\pgfpathlineto{\pgfqpoint{1.047818in}{1.077952in}}%
\pgfpathlineto{\pgfqpoint{1.047818in}{1.077952in}}%
\pgfpathlineto{\pgfqpoint{1.056227in}{1.065025in}}%
\pgfpathlineto{\pgfqpoint{1.057718in}{1.052098in}}%
\pgfpathlineto{\pgfqpoint{1.057822in}{1.039172in}}%
\pgfpathlineto{\pgfqpoint{1.056702in}{1.026245in}}%
\pgfpathlineto{\pgfqpoint{1.050896in}{1.013319in}}%
\pgfpathlineto{\pgfqpoint{1.047818in}{1.011278in}}%
\pgfpathlineto{\pgfqpoint{1.034549in}{1.008946in}}%
\pgfpathlineto{\pgfqpoint{1.021280in}{1.009789in}}%
\pgfpathclose%
\pgfpathmoveto{\pgfqpoint{1.068420in}{1.013319in}}%
\pgfpathlineto{\pgfqpoint{1.064806in}{1.026245in}}%
\pgfpathlineto{\pgfqpoint{1.064086in}{1.039172in}}%
\pgfpathlineto{\pgfqpoint{1.064094in}{1.052098in}}%
\pgfpathlineto{\pgfqpoint{1.064873in}{1.065025in}}%
\pgfpathlineto{\pgfqpoint{1.069397in}{1.077952in}}%
\pgfpathlineto{\pgfqpoint{1.074356in}{1.080898in}}%
\pgfpathlineto{\pgfqpoint{1.087625in}{1.082558in}}%
\pgfpathlineto{\pgfqpoint{1.100894in}{1.082233in}}%
\pgfpathlineto{\pgfqpoint{1.113604in}{1.077952in}}%
\pgfpathlineto{\pgfqpoint{1.114163in}{1.077391in}}%
\pgfpathlineto{\pgfqpoint{1.117901in}{1.065025in}}%
\pgfpathlineto{\pgfqpoint{1.118612in}{1.052098in}}%
\pgfpathlineto{\pgfqpoint{1.118618in}{1.039172in}}%
\pgfpathlineto{\pgfqpoint{1.117958in}{1.026245in}}%
\pgfpathlineto{\pgfqpoint{1.114627in}{1.013319in}}%
\pgfpathlineto{\pgfqpoint{1.114163in}{1.012681in}}%
\pgfpathlineto{\pgfqpoint{1.100894in}{1.007565in}}%
\pgfpathlineto{\pgfqpoint{1.087625in}{1.007209in}}%
\pgfpathlineto{\pgfqpoint{1.074356in}{1.009001in}}%
\pgfpathclose%
\pgfpathmoveto{\pgfqpoint{1.132384in}{1.013319in}}%
\pgfpathlineto{\pgfqpoint{1.127432in}{1.019561in}}%
\pgfpathlineto{\pgfqpoint{1.125582in}{1.026245in}}%
\pgfpathlineto{\pgfqpoint{1.124508in}{1.039172in}}%
\pgfpathlineto{\pgfqpoint{1.124607in}{1.052098in}}%
\pgfpathlineto{\pgfqpoint{1.126034in}{1.065025in}}%
\pgfpathlineto{\pgfqpoint{1.127432in}{1.069458in}}%
\pgfpathlineto{\pgfqpoint{1.136308in}{1.077952in}}%
\pgfpathlineto{\pgfqpoint{1.140701in}{1.079409in}}%
\pgfpathlineto{\pgfqpoint{1.153970in}{1.080039in}}%
\pgfpathlineto{\pgfqpoint{1.164071in}{1.077952in}}%
\pgfpathlineto{\pgfqpoint{1.167239in}{1.076741in}}%
\pgfpathlineto{\pgfqpoint{1.174004in}{1.065025in}}%
\pgfpathlineto{\pgfqpoint{1.175501in}{1.052098in}}%
\pgfpathlineto{\pgfqpoint{1.175604in}{1.039172in}}%
\pgfpathlineto{\pgfqpoint{1.174478in}{1.026245in}}%
\pgfpathlineto{\pgfqpoint{1.168602in}{1.013319in}}%
\pgfpathlineto{\pgfqpoint{1.167239in}{1.012304in}}%
\pgfpathlineto{\pgfqpoint{1.153970in}{1.009265in}}%
\pgfpathlineto{\pgfqpoint{1.140701in}{1.009874in}}%
\pgfpathclose%
\pgfpathmoveto{\pgfqpoint{1.186164in}{1.013319in}}%
\pgfpathlineto{\pgfqpoint{1.182545in}{1.026245in}}%
\pgfpathlineto{\pgfqpoint{1.181829in}{1.039172in}}%
\pgfpathlineto{\pgfqpoint{1.181835in}{1.052098in}}%
\pgfpathlineto{\pgfqpoint{1.182607in}{1.065025in}}%
\pgfpathlineto{\pgfqpoint{1.187130in}{1.077952in}}%
\pgfpathlineto{\pgfqpoint{1.193777in}{1.081398in}}%
\pgfpathlineto{\pgfqpoint{1.207045in}{1.082604in}}%
\pgfpathlineto{\pgfqpoint{1.220314in}{1.082018in}}%
\pgfpathlineto{\pgfqpoint{1.230944in}{1.077952in}}%
\pgfpathlineto{\pgfqpoint{1.233583in}{1.074254in}}%
\pgfpathlineto{\pgfqpoint{1.235740in}{1.065025in}}%
\pgfpathlineto{\pgfqpoint{1.236483in}{1.052098in}}%
\pgfpathlineto{\pgfqpoint{1.236491in}{1.039172in}}%
\pgfpathlineto{\pgfqpoint{1.235805in}{1.026245in}}%
\pgfpathlineto{\pgfqpoint{1.233583in}{1.015944in}}%
\pgfpathlineto{\pgfqpoint{1.232085in}{1.013319in}}%
\pgfpathlineto{\pgfqpoint{1.220314in}{1.007793in}}%
\pgfpathlineto{\pgfqpoint{1.207045in}{1.007160in}}%
\pgfpathlineto{\pgfqpoint{1.193777in}{1.008470in}}%
\pgfpathclose%
\pgfpathmoveto{\pgfqpoint{1.249637in}{1.013319in}}%
\pgfpathlineto{\pgfqpoint{1.246852in}{1.016035in}}%
\pgfpathlineto{\pgfqpoint{1.243424in}{1.026245in}}%
\pgfpathlineto{\pgfqpoint{1.242383in}{1.039172in}}%
\pgfpathlineto{\pgfqpoint{1.242480in}{1.052098in}}%
\pgfpathlineto{\pgfqpoint{1.243865in}{1.065025in}}%
\pgfpathlineto{\pgfqpoint{1.246852in}{1.072863in}}%
\pgfpathlineto{\pgfqpoint{1.253678in}{1.077952in}}%
\pgfpathlineto{\pgfqpoint{1.260121in}{1.079887in}}%
\pgfpathlineto{\pgfqpoint{1.273390in}{1.080225in}}%
\pgfpathlineto{\pgfqpoint{1.283156in}{1.077952in}}%
\pgfpathlineto{\pgfqpoint{1.286659in}{1.076298in}}%
\pgfpathlineto{\pgfqpoint{1.292378in}{1.065025in}}%
\pgfpathlineto{\pgfqpoint{1.293745in}{1.052098in}}%
\pgfpathlineto{\pgfqpoint{1.293842in}{1.039172in}}%
\pgfpathlineto{\pgfqpoint{1.292819in}{1.026245in}}%
\pgfpathlineto{\pgfqpoint{1.287449in}{1.013319in}}%
\pgfpathlineto{\pgfqpoint{1.286659in}{1.012653in}}%
\pgfpathlineto{\pgfqpoint{1.273390in}{1.009072in}}%
\pgfpathlineto{\pgfqpoint{1.260121in}{1.009401in}}%
\pgfpathclose%
\pgfpathmoveto{\pgfqpoint{1.305071in}{1.013319in}}%
\pgfpathlineto{\pgfqpoint{1.300752in}{1.026245in}}%
\pgfpathlineto{\pgfqpoint{1.299928in}{1.038723in}}%
\pgfpathlineto{\pgfqpoint{1.299907in}{1.039172in}}%
\pgfpathlineto{\pgfqpoint{1.299919in}{1.052098in}}%
\pgfpathlineto{\pgfqpoint{1.299928in}{1.052285in}}%
\pgfpathlineto{\pgfqpoint{1.300846in}{1.065025in}}%
\pgfpathlineto{\pgfqpoint{1.306325in}{1.077952in}}%
\pgfpathlineto{\pgfqpoint{1.313197in}{1.081079in}}%
\pgfpathlineto{\pgfqpoint{1.326466in}{1.082102in}}%
\pgfpathlineto{\pgfqpoint{1.339735in}{1.081075in}}%
\pgfpathlineto{\pgfqpoint{1.346938in}{1.077952in}}%
\pgfpathlineto{\pgfqpoint{1.353002in}{1.065025in}}%
\pgfpathlineto{\pgfqpoint{1.353004in}{1.065005in}}%
\pgfpathlineto{\pgfqpoint{1.353922in}{1.052098in}}%
\pgfpathlineto{\pgfqpoint{1.353938in}{1.039172in}}%
\pgfpathlineto{\pgfqpoint{1.353108in}{1.026245in}}%
\pgfpathlineto{\pgfqpoint{1.353004in}{1.025571in}}%
\pgfpathlineto{\pgfqpoint{1.348345in}{1.013319in}}%
\pgfpathlineto{\pgfqpoint{1.339735in}{1.008750in}}%
\pgfpathlineto{\pgfqpoint{1.326466in}{1.007660in}}%
\pgfpathlineto{\pgfqpoint{1.313197in}{1.008762in}}%
\pgfpathclose%
\pgfpathmoveto{\pgfqpoint{1.365425in}{1.013319in}}%
\pgfpathlineto{\pgfqpoint{1.360771in}{1.026245in}}%
\pgfpathlineto{\pgfqpoint{1.359870in}{1.039172in}}%
\pgfpathlineto{\pgfqpoint{1.359959in}{1.052098in}}%
\pgfpathlineto{\pgfqpoint{1.361172in}{1.065025in}}%
\pgfpathlineto{\pgfqpoint{1.366273in}{1.076433in}}%
\pgfpathlineto{\pgfqpoint{1.368821in}{1.077952in}}%
\pgfpathlineto{\pgfqpoint{1.379542in}{1.080859in}}%
\pgfpathlineto{\pgfqpoint{1.392811in}{1.080952in}}%
\pgfpathlineto{\pgfqpoint{1.404340in}{1.077952in}}%
\pgfpathlineto{\pgfqpoint{1.406080in}{1.076930in}}%
\pgfpathlineto{\pgfqpoint{1.411308in}{1.065025in}}%
\pgfpathlineto{\pgfqpoint{1.412412in}{1.052098in}}%
\pgfpathlineto{\pgfqpoint{1.412496in}{1.039172in}}%
\pgfpathlineto{\pgfqpoint{1.411686in}{1.026245in}}%
\pgfpathlineto{\pgfqpoint{1.407391in}{1.013319in}}%
\pgfpathlineto{\pgfqpoint{1.406080in}{1.012044in}}%
\pgfpathlineto{\pgfqpoint{1.392811in}{1.008330in}}%
\pgfpathlineto{\pgfqpoint{1.379542in}{1.008426in}}%
\pgfpathlineto{\pgfqpoint{1.366273in}{1.012494in}}%
\pgfpathclose%
\pgfpathmoveto{\pgfqpoint{1.425326in}{1.013319in}}%
\pgfpathlineto{\pgfqpoint{1.419478in}{1.026245in}}%
\pgfpathlineto{\pgfqpoint{1.419348in}{1.027413in}}%
\pgfpathlineto{\pgfqpoint{1.418489in}{1.039172in}}%
\pgfpathlineto{\pgfqpoint{1.418514in}{1.052098in}}%
\pgfpathlineto{\pgfqpoint{1.419348in}{1.062644in}}%
\pgfpathlineto{\pgfqpoint{1.419648in}{1.065025in}}%
\pgfpathlineto{\pgfqpoint{1.427231in}{1.077952in}}%
\pgfpathlineto{\pgfqpoint{1.432617in}{1.080108in}}%
\pgfpathlineto{\pgfqpoint{1.445886in}{1.081051in}}%
\pgfpathlineto{\pgfqpoint{1.459155in}{1.079242in}}%
\pgfpathlineto{\pgfqpoint{1.461779in}{1.077952in}}%
\pgfpathlineto{\pgfqpoint{1.469415in}{1.065025in}}%
\pgfpathlineto{\pgfqpoint{1.470763in}{1.052098in}}%
\pgfpathlineto{\pgfqpoint{1.470798in}{1.039172in}}%
\pgfpathlineto{\pgfqpoint{1.469612in}{1.026245in}}%
\pgfpathlineto{\pgfqpoint{1.463689in}{1.013319in}}%
\pgfpathlineto{\pgfqpoint{1.459155in}{1.010598in}}%
\pgfpathlineto{\pgfqpoint{1.445886in}{1.008710in}}%
\pgfpathlineto{\pgfqpoint{1.432617in}{1.009711in}}%
\pgfpathclose%
\pgfpathmoveto{\pgfqpoint{0.713864in}{1.090878in}}%
\pgfpathlineto{\pgfqpoint{0.709891in}{1.103805in}}%
\pgfpathlineto{\pgfqpoint{0.709342in}{1.116731in}}%
\pgfpathlineto{\pgfqpoint{0.709275in}{1.129658in}}%
\pgfpathlineto{\pgfqpoint{0.709559in}{1.142584in}}%
\pgfpathlineto{\pgfqpoint{0.710960in}{1.155511in}}%
\pgfpathlineto{\pgfqpoint{0.716095in}{1.162381in}}%
\pgfpathlineto{\pgfqpoint{0.729364in}{1.164389in}}%
\pgfpathlineto{\pgfqpoint{0.742633in}{1.164296in}}%
\pgfpathlineto{\pgfqpoint{0.755902in}{1.162298in}}%
\pgfpathlineto{\pgfqpoint{0.762466in}{1.155511in}}%
\pgfpathlineto{\pgfqpoint{0.764544in}{1.142584in}}%
\pgfpathlineto{\pgfqpoint{0.764979in}{1.129658in}}%
\pgfpathlineto{\pgfqpoint{0.764875in}{1.116731in}}%
\pgfpathlineto{\pgfqpoint{0.764040in}{1.103805in}}%
\pgfpathlineto{\pgfqpoint{0.758486in}{1.090878in}}%
\pgfpathlineto{\pgfqpoint{0.755902in}{1.089500in}}%
\pgfpathlineto{\pgfqpoint{0.742633in}{1.087532in}}%
\pgfpathlineto{\pgfqpoint{0.729364in}{1.087440in}}%
\pgfpathlineto{\pgfqpoint{0.716095in}{1.089418in}}%
\pgfpathclose%
\pgfpathmoveto{\pgfqpoint{0.791115in}{1.090878in}}%
\pgfpathlineto{\pgfqpoint{0.782439in}{1.092515in}}%
\pgfpathlineto{\pgfqpoint{0.773264in}{1.103805in}}%
\pgfpathlineto{\pgfqpoint{0.771289in}{1.116731in}}%
\pgfpathlineto{\pgfqpoint{0.771043in}{1.129658in}}%
\pgfpathlineto{\pgfqpoint{0.772073in}{1.142584in}}%
\pgfpathlineto{\pgfqpoint{0.776992in}{1.155511in}}%
\pgfpathlineto{\pgfqpoint{0.782439in}{1.159410in}}%
\pgfpathlineto{\pgfqpoint{0.795708in}{1.161476in}}%
\pgfpathlineto{\pgfqpoint{0.808977in}{1.160323in}}%
\pgfpathlineto{\pgfqpoint{0.816902in}{1.155511in}}%
\pgfpathlineto{\pgfqpoint{0.821658in}{1.142584in}}%
\pgfpathlineto{\pgfqpoint{0.822246in}{1.135375in}}%
\pgfpathlineto{\pgfqpoint{0.822571in}{1.129658in}}%
\pgfpathlineto{\pgfqpoint{0.822373in}{1.116731in}}%
\pgfpathlineto{\pgfqpoint{0.822246in}{1.115318in}}%
\pgfpathlineto{\pgfqpoint{0.820528in}{1.103805in}}%
\pgfpathlineto{\pgfqpoint{0.808977in}{1.091510in}}%
\pgfpathlineto{\pgfqpoint{0.803162in}{1.090878in}}%
\pgfpathlineto{\pgfqpoint{0.795708in}{1.090308in}}%
\pgfpathclose%
\pgfpathmoveto{\pgfqpoint{0.839229in}{1.090878in}}%
\pgfpathlineto{\pgfqpoint{0.835515in}{1.092378in}}%
\pgfpathlineto{\pgfqpoint{0.829781in}{1.103805in}}%
\pgfpathlineto{\pgfqpoint{0.828645in}{1.116731in}}%
\pgfpathlineto{\pgfqpoint{0.828504in}{1.129658in}}%
\pgfpathlineto{\pgfqpoint{0.829093in}{1.142584in}}%
\pgfpathlineto{\pgfqpoint{0.831983in}{1.155511in}}%
\pgfpathlineto{\pgfqpoint{0.835515in}{1.159536in}}%
\pgfpathlineto{\pgfqpoint{0.848784in}{1.162724in}}%
\pgfpathlineto{\pgfqpoint{0.862053in}{1.162565in}}%
\pgfpathlineto{\pgfqpoint{0.875322in}{1.158604in}}%
\pgfpathlineto{\pgfqpoint{0.877923in}{1.155511in}}%
\pgfpathlineto{\pgfqpoint{0.881183in}{1.142584in}}%
\pgfpathlineto{\pgfqpoint{0.881862in}{1.129658in}}%
\pgfpathlineto{\pgfqpoint{0.881700in}{1.116731in}}%
\pgfpathlineto{\pgfqpoint{0.880396in}{1.103805in}}%
\pgfpathlineto{\pgfqpoint{0.875322in}{1.093403in}}%
\pgfpathlineto{\pgfqpoint{0.869789in}{1.090878in}}%
\pgfpathlineto{\pgfqpoint{0.862053in}{1.089237in}}%
\pgfpathlineto{\pgfqpoint{0.848784in}{1.089079in}}%
\pgfpathclose%
\pgfpathmoveto{\pgfqpoint{0.899604in}{1.090878in}}%
\pgfpathlineto{\pgfqpoint{0.889281in}{1.103805in}}%
\pgfpathlineto{\pgfqpoint{0.888591in}{1.108514in}}%
\pgfpathlineto{\pgfqpoint{0.887863in}{1.116731in}}%
\pgfpathlineto{\pgfqpoint{0.887698in}{1.129658in}}%
\pgfpathlineto{\pgfqpoint{0.888387in}{1.142584in}}%
\pgfpathlineto{\pgfqpoint{0.888591in}{1.144024in}}%
\pgfpathlineto{\pgfqpoint{0.892188in}{1.155511in}}%
\pgfpathlineto{\pgfqpoint{0.901860in}{1.161641in}}%
\pgfpathlineto{\pgfqpoint{0.915129in}{1.162802in}}%
\pgfpathlineto{\pgfqpoint{0.928398in}{1.161606in}}%
\pgfpathlineto{\pgfqpoint{0.937266in}{1.155511in}}%
\pgfpathlineto{\pgfqpoint{0.940617in}{1.142584in}}%
\pgfpathlineto{\pgfqpoint{0.941300in}{1.129658in}}%
\pgfpathlineto{\pgfqpoint{0.941137in}{1.116731in}}%
\pgfpathlineto{\pgfqpoint{0.939818in}{1.103805in}}%
\pgfpathlineto{\pgfqpoint{0.930448in}{1.090878in}}%
\pgfpathlineto{\pgfqpoint{0.928398in}{1.090180in}}%
\pgfpathlineto{\pgfqpoint{0.915129in}{1.089003in}}%
\pgfpathlineto{\pgfqpoint{0.901860in}{1.090145in}}%
\pgfpathclose%
\pgfpathmoveto{\pgfqpoint{0.963583in}{1.090878in}}%
\pgfpathlineto{\pgfqpoint{0.954936in}{1.093700in}}%
\pgfpathlineto{\pgfqpoint{0.949087in}{1.103805in}}%
\pgfpathlineto{\pgfqpoint{0.947534in}{1.116731in}}%
\pgfpathlineto{\pgfqpoint{0.947341in}{1.129658in}}%
\pgfpathlineto{\pgfqpoint{0.948147in}{1.142584in}}%
\pgfpathlineto{\pgfqpoint{0.952086in}{1.155511in}}%
\pgfpathlineto{\pgfqpoint{0.954936in}{1.158334in}}%
\pgfpathlineto{\pgfqpoint{0.968205in}{1.161691in}}%
\pgfpathlineto{\pgfqpoint{0.981473in}{1.161294in}}%
\pgfpathlineto{\pgfqpoint{0.994126in}{1.155511in}}%
\pgfpathlineto{\pgfqpoint{0.994742in}{1.154733in}}%
\pgfpathlineto{\pgfqpoint{0.998330in}{1.142584in}}%
\pgfpathlineto{\pgfqpoint{0.999165in}{1.129658in}}%
\pgfpathlineto{\pgfqpoint{0.998966in}{1.116731in}}%
\pgfpathlineto{\pgfqpoint{0.997360in}{1.103805in}}%
\pgfpathlineto{\pgfqpoint{0.994742in}{1.097482in}}%
\pgfpathlineto{\pgfqpoint{0.983142in}{1.090878in}}%
\pgfpathlineto{\pgfqpoint{0.981473in}{1.090489in}}%
\pgfpathlineto{\pgfqpoint{0.968205in}{1.090097in}}%
\pgfpathclose%
\pgfpathmoveto{\pgfqpoint{1.014260in}{1.090878in}}%
\pgfpathlineto{\pgfqpoint{1.008011in}{1.096914in}}%
\pgfpathlineto{\pgfqpoint{1.006113in}{1.103805in}}%
\pgfpathlineto{\pgfqpoint{1.005093in}{1.116731in}}%
\pgfpathlineto{\pgfqpoint{1.004966in}{1.129658in}}%
\pgfpathlineto{\pgfqpoint{1.005497in}{1.142584in}}%
\pgfpathlineto{\pgfqpoint{1.008011in}{1.155388in}}%
\pgfpathlineto{\pgfqpoint{1.008068in}{1.155511in}}%
\pgfpathlineto{\pgfqpoint{1.021280in}{1.162934in}}%
\pgfpathlineto{\pgfqpoint{1.034549in}{1.163571in}}%
\pgfpathlineto{\pgfqpoint{1.047818in}{1.162269in}}%
\pgfpathlineto{\pgfqpoint{1.056480in}{1.155511in}}%
\pgfpathlineto{\pgfqpoint{1.059068in}{1.142584in}}%
\pgfpathlineto{\pgfqpoint{1.059599in}{1.129658in}}%
\pgfpathlineto{\pgfqpoint{1.059473in}{1.116731in}}%
\pgfpathlineto{\pgfqpoint{1.058449in}{1.103805in}}%
\pgfpathlineto{\pgfqpoint{1.051281in}{1.090878in}}%
\pgfpathlineto{\pgfqpoint{1.047818in}{1.089528in}}%
\pgfpathlineto{\pgfqpoint{1.034549in}{1.088245in}}%
\pgfpathlineto{\pgfqpoint{1.021280in}{1.088873in}}%
\pgfpathclose%
\pgfpathmoveto{\pgfqpoint{1.085514in}{1.090878in}}%
\pgfpathlineto{\pgfqpoint{1.074356in}{1.093838in}}%
\pgfpathlineto{\pgfqpoint{1.067799in}{1.103805in}}%
\pgfpathlineto{\pgfqpoint{1.065992in}{1.116731in}}%
\pgfpathlineto{\pgfqpoint{1.065768in}{1.129658in}}%
\pgfpathlineto{\pgfqpoint{1.066706in}{1.142584in}}%
\pgfpathlineto{\pgfqpoint{1.071270in}{1.155511in}}%
\pgfpathlineto{\pgfqpoint{1.074356in}{1.158208in}}%
\pgfpathlineto{\pgfqpoint{1.087625in}{1.161221in}}%
\pgfpathlineto{\pgfqpoint{1.100894in}{1.160507in}}%
\pgfpathlineto{\pgfqpoint{1.110745in}{1.155511in}}%
\pgfpathlineto{\pgfqpoint{1.114163in}{1.149917in}}%
\pgfpathlineto{\pgfqpoint{1.115942in}{1.142584in}}%
\pgfpathlineto{\pgfqpoint{1.116849in}{1.129658in}}%
\pgfpathlineto{\pgfqpoint{1.116632in}{1.116731in}}%
\pgfpathlineto{\pgfqpoint{1.114887in}{1.103805in}}%
\pgfpathlineto{\pgfqpoint{1.114163in}{1.101677in}}%
\pgfpathlineto{\pgfqpoint{1.100894in}{1.091308in}}%
\pgfpathlineto{\pgfqpoint{1.094021in}{1.090878in}}%
\pgfpathlineto{\pgfqpoint{1.087625in}{1.090560in}}%
\pgfpathclose%
\pgfpathmoveto{\pgfqpoint{1.130626in}{1.090878in}}%
\pgfpathlineto{\pgfqpoint{1.127432in}{1.093103in}}%
\pgfpathlineto{\pgfqpoint{1.123658in}{1.103805in}}%
\pgfpathlineto{\pgfqpoint{1.122763in}{1.116731in}}%
\pgfpathlineto{\pgfqpoint{1.122652in}{1.129658in}}%
\pgfpathlineto{\pgfqpoint{1.123117in}{1.142584in}}%
\pgfpathlineto{\pgfqpoint{1.125373in}{1.155511in}}%
\pgfpathlineto{\pgfqpoint{1.127432in}{1.158875in}}%
\pgfpathlineto{\pgfqpoint{1.140701in}{1.163460in}}%
\pgfpathlineto{\pgfqpoint{1.153970in}{1.163788in}}%
\pgfpathlineto{\pgfqpoint{1.167239in}{1.162155in}}%
\pgfpathlineto{\pgfqpoint{1.174697in}{1.155511in}}%
\pgfpathlineto{\pgfqpoint{1.177064in}{1.142584in}}%
\pgfpathlineto{\pgfqpoint{1.177552in}{1.129658in}}%
\pgfpathlineto{\pgfqpoint{1.177436in}{1.116731in}}%
\pgfpathlineto{\pgfqpoint{1.176496in}{1.103805in}}%
\pgfpathlineto{\pgfqpoint{1.169998in}{1.090878in}}%
\pgfpathlineto{\pgfqpoint{1.167239in}{1.089639in}}%
\pgfpathlineto{\pgfqpoint{1.153970in}{1.088031in}}%
\pgfpathlineto{\pgfqpoint{1.140701in}{1.088354in}}%
\pgfpathclose%
\pgfpathmoveto{\pgfqpoint{1.204263in}{1.090878in}}%
\pgfpathlineto{\pgfqpoint{1.193777in}{1.093138in}}%
\pgfpathlineto{\pgfqpoint{1.185876in}{1.103805in}}%
\pgfpathlineto{\pgfqpoint{1.183984in}{1.116731in}}%
\pgfpathlineto{\pgfqpoint{1.183749in}{1.129658in}}%
\pgfpathlineto{\pgfqpoint{1.184733in}{1.142584in}}%
\pgfpathlineto{\pgfqpoint{1.189496in}{1.155511in}}%
\pgfpathlineto{\pgfqpoint{1.193777in}{1.158845in}}%
\pgfpathlineto{\pgfqpoint{1.207045in}{1.161273in}}%
\pgfpathlineto{\pgfqpoint{1.220314in}{1.160255in}}%
\pgfpathlineto{\pgfqpoint{1.228753in}{1.155511in}}%
\pgfpathlineto{\pgfqpoint{1.233583in}{1.144754in}}%
\pgfpathlineto{\pgfqpoint{1.233994in}{1.142584in}}%
\pgfpathlineto{\pgfqpoint{1.234888in}{1.129658in}}%
\pgfpathlineto{\pgfqpoint{1.234675in}{1.116731in}}%
\pgfpathlineto{\pgfqpoint{1.233583in}{1.107518in}}%
\pgfpathlineto{\pgfqpoint{1.232811in}{1.103805in}}%
\pgfpathlineto{\pgfqpoint{1.220314in}{1.091586in}}%
\pgfpathlineto{\pgfqpoint{1.212505in}{1.090878in}}%
\pgfpathlineto{\pgfqpoint{1.207045in}{1.090509in}}%
\pgfpathclose%
\pgfpathmoveto{\pgfqpoint{1.249132in}{1.090878in}}%
\pgfpathlineto{\pgfqpoint{1.246852in}{1.092069in}}%
\pgfpathlineto{\pgfqpoint{1.241800in}{1.103805in}}%
\pgfpathlineto{\pgfqpoint{1.240850in}{1.116731in}}%
\pgfpathlineto{\pgfqpoint{1.240732in}{1.129658in}}%
\pgfpathlineto{\pgfqpoint{1.241225in}{1.142584in}}%
\pgfpathlineto{\pgfqpoint{1.243628in}{1.155511in}}%
\pgfpathlineto{\pgfqpoint{1.246852in}{1.159815in}}%
\pgfpathlineto{\pgfqpoint{1.260121in}{1.163334in}}%
\pgfpathlineto{\pgfqpoint{1.273390in}{1.163436in}}%
\pgfpathlineto{\pgfqpoint{1.286659in}{1.161008in}}%
\pgfpathlineto{\pgfqpoint{1.292011in}{1.155511in}}%
\pgfpathlineto{\pgfqpoint{1.294632in}{1.142584in}}%
\pgfpathlineto{\pgfqpoint{1.295175in}{1.129658in}}%
\pgfpathlineto{\pgfqpoint{1.295045in}{1.116731in}}%
\pgfpathlineto{\pgfqpoint{1.294002in}{1.103805in}}%
\pgfpathlineto{\pgfqpoint{1.286870in}{1.090878in}}%
\pgfpathlineto{\pgfqpoint{1.286659in}{1.090768in}}%
\pgfpathlineto{\pgfqpoint{1.273390in}{1.088378in}}%
\pgfpathlineto{\pgfqpoint{1.260121in}{1.088479in}}%
\pgfpathclose%
\pgfpathmoveto{\pgfqpoint{1.318500in}{1.090878in}}%
\pgfpathlineto{\pgfqpoint{1.313197in}{1.091799in}}%
\pgfpathlineto{\pgfqpoint{1.303254in}{1.103805in}}%
\pgfpathlineto{\pgfqpoint{1.301452in}{1.116731in}}%
\pgfpathlineto{\pgfqpoint{1.301228in}{1.129658in}}%
\pgfpathlineto{\pgfqpoint{1.302166in}{1.142584in}}%
\pgfpathlineto{\pgfqpoint{1.306684in}{1.155511in}}%
\pgfpathlineto{\pgfqpoint{1.313197in}{1.160062in}}%
\pgfpathlineto{\pgfqpoint{1.326466in}{1.161829in}}%
\pgfpathlineto{\pgfqpoint{1.339735in}{1.160621in}}%
\pgfpathlineto{\pgfqpoint{1.347930in}{1.155511in}}%
\pgfpathlineto{\pgfqpoint{1.352401in}{1.142584in}}%
\pgfpathlineto{\pgfqpoint{1.353004in}{1.134706in}}%
\pgfpathlineto{\pgfqpoint{1.353275in}{1.129658in}}%
\pgfpathlineto{\pgfqpoint{1.353085in}{1.116731in}}%
\pgfpathlineto{\pgfqpoint{1.353004in}{1.115766in}}%
\pgfpathlineto{\pgfqpoint{1.351335in}{1.103805in}}%
\pgfpathlineto{\pgfqpoint{1.339735in}{1.091184in}}%
\pgfpathlineto{\pgfqpoint{1.337228in}{1.090878in}}%
\pgfpathlineto{\pgfqpoint{1.326466in}{1.089962in}}%
\pgfpathclose%
\pgfpathmoveto{\pgfqpoint{1.370506in}{1.090878in}}%
\pgfpathlineto{\pgfqpoint{1.366273in}{1.092566in}}%
\pgfpathlineto{\pgfqpoint{1.360538in}{1.103805in}}%
\pgfpathlineto{\pgfqpoint{1.359345in}{1.116731in}}%
\pgfpathlineto{\pgfqpoint{1.359198in}{1.129658in}}%
\pgfpathlineto{\pgfqpoint{1.359816in}{1.142584in}}%
\pgfpathlineto{\pgfqpoint{1.362842in}{1.155511in}}%
\pgfpathlineto{\pgfqpoint{1.366273in}{1.159364in}}%
\pgfpathlineto{\pgfqpoint{1.379542in}{1.162626in}}%
\pgfpathlineto{\pgfqpoint{1.392811in}{1.162473in}}%
\pgfpathlineto{\pgfqpoint{1.406080in}{1.158385in}}%
\pgfpathlineto{\pgfqpoint{1.408471in}{1.155511in}}%
\pgfpathlineto{\pgfqpoint{1.411779in}{1.142584in}}%
\pgfpathlineto{\pgfqpoint{1.412467in}{1.129658in}}%
\pgfpathlineto{\pgfqpoint{1.412303in}{1.116731in}}%
\pgfpathlineto{\pgfqpoint{1.410981in}{1.103805in}}%
\pgfpathlineto{\pgfqpoint{1.406080in}{1.093643in}}%
\pgfpathlineto{\pgfqpoint{1.400101in}{1.090878in}}%
\pgfpathlineto{\pgfqpoint{1.392811in}{1.089327in}}%
\pgfpathlineto{\pgfqpoint{1.379542in}{1.089176in}}%
\pgfpathclose%
\pgfpathmoveto{\pgfqpoint{1.430014in}{1.090878in}}%
\pgfpathlineto{\pgfqpoint{1.419830in}{1.103805in}}%
\pgfpathlineto{\pgfqpoint{1.419348in}{1.106980in}}%
\pgfpathlineto{\pgfqpoint{1.418458in}{1.116731in}}%
\pgfpathlineto{\pgfqpoint{1.418296in}{1.129658in}}%
\pgfpathlineto{\pgfqpoint{1.418975in}{1.142584in}}%
\pgfpathlineto{\pgfqpoint{1.419348in}{1.145149in}}%
\pgfpathlineto{\pgfqpoint{1.422705in}{1.155511in}}%
\pgfpathlineto{\pgfqpoint{1.432617in}{1.161753in}}%
\pgfpathlineto{\pgfqpoint{1.445886in}{1.162890in}}%
\pgfpathlineto{\pgfqpoint{1.459155in}{1.161737in}}%
\pgfpathlineto{\pgfqpoint{1.468136in}{1.155511in}}%
\pgfpathlineto{\pgfqpoint{1.471327in}{1.142584in}}%
\pgfpathlineto{\pgfqpoint{1.471976in}{1.129658in}}%
\pgfpathlineto{\pgfqpoint{1.471822in}{1.116731in}}%
\pgfpathlineto{\pgfqpoint{1.470568in}{1.103805in}}%
\pgfpathlineto{\pgfqpoint{1.461576in}{1.090878in}}%
\pgfpathlineto{\pgfqpoint{1.459155in}{1.090052in}}%
\pgfpathlineto{\pgfqpoint{1.445886in}{1.088916in}}%
\pgfpathlineto{\pgfqpoint{1.432617in}{1.090036in}}%
\pgfpathclose%
\pgfpathmoveto{\pgfqpoint{0.657616in}{1.103805in}}%
\pgfpathlineto{\pgfqpoint{0.655049in}{1.116731in}}%
\pgfpathlineto{\pgfqpoint{0.654729in}{1.129658in}}%
\pgfpathlineto{\pgfqpoint{0.656068in}{1.142584in}}%
\pgfpathlineto{\pgfqpoint{0.662443in}{1.155511in}}%
\pgfpathlineto{\pgfqpoint{0.663019in}{1.155978in}}%
\pgfpathlineto{\pgfqpoint{0.676288in}{1.159569in}}%
\pgfpathlineto{\pgfqpoint{0.689557in}{1.158510in}}%
\pgfpathlineto{\pgfqpoint{0.695147in}{1.155511in}}%
\pgfpathlineto{\pgfqpoint{0.702106in}{1.142584in}}%
\pgfpathlineto{\pgfqpoint{0.702826in}{1.136651in}}%
\pgfpathlineto{\pgfqpoint{0.703378in}{1.129658in}}%
\pgfpathlineto{\pgfqpoint{0.703108in}{1.116731in}}%
\pgfpathlineto{\pgfqpoint{0.702826in}{1.114461in}}%
\pgfpathlineto{\pgfqpoint{0.700461in}{1.103805in}}%
\pgfpathlineto{\pgfqpoint{0.689557in}{1.093505in}}%
\pgfpathlineto{\pgfqpoint{0.676288in}{1.092340in}}%
\pgfpathlineto{\pgfqpoint{0.663019in}{1.096293in}}%
\pgfpathclose%
\pgfpathmoveto{\pgfqpoint{0.656330in}{1.168437in}}%
\pgfpathlineto{\pgfqpoint{0.649750in}{1.181345in}}%
\pgfpathlineto{\pgfqpoint{0.649749in}{1.181364in}}%
\pgfpathlineto{\pgfqpoint{0.649432in}{1.194290in}}%
\pgfpathlineto{\pgfqpoint{0.649349in}{1.207217in}}%
\pgfpathlineto{\pgfqpoint{0.649359in}{1.220143in}}%
\pgfpathlineto{\pgfqpoint{0.649510in}{1.233070in}}%
\pgfpathlineto{\pgfqpoint{0.649750in}{1.238407in}}%
\pgfpathlineto{\pgfqpoint{0.653083in}{1.245996in}}%
\pgfpathlineto{\pgfqpoint{0.663019in}{1.247042in}}%
\pgfpathlineto{\pgfqpoint{0.676288in}{1.247056in}}%
\pgfpathlineto{\pgfqpoint{0.689557in}{1.246620in}}%
\pgfpathlineto{\pgfqpoint{0.694977in}{1.245996in}}%
\pgfpathlineto{\pgfqpoint{0.702826in}{1.243110in}}%
\pgfpathlineto{\pgfqpoint{0.706237in}{1.233070in}}%
\pgfpathlineto{\pgfqpoint{0.706892in}{1.220143in}}%
\pgfpathlineto{\pgfqpoint{0.707003in}{1.207217in}}%
\pgfpathlineto{\pgfqpoint{0.706825in}{1.194290in}}%
\pgfpathlineto{\pgfqpoint{0.706025in}{1.181364in}}%
\pgfpathlineto{\pgfqpoint{0.702826in}{1.172139in}}%
\pgfpathlineto{\pgfqpoint{0.692775in}{1.168437in}}%
\pgfpathlineto{\pgfqpoint{0.689557in}{1.168060in}}%
\pgfpathlineto{\pgfqpoint{0.676288in}{1.167578in}}%
\pgfpathlineto{\pgfqpoint{0.663019in}{1.167713in}}%
\pgfpathclose%
\pgfpathmoveto{\pgfqpoint{0.602684in}{1.181364in}}%
\pgfpathlineto{\pgfqpoint{0.597227in}{1.194290in}}%
\pgfpathlineto{\pgfqpoint{0.596674in}{1.200607in}}%
\pgfpathlineto{\pgfqpoint{0.596285in}{1.207217in}}%
\pgfpathlineto{\pgfqpoint{0.596674in}{1.213299in}}%
\pgfpathlineto{\pgfqpoint{0.597325in}{1.220143in}}%
\pgfpathlineto{\pgfqpoint{0.603015in}{1.233070in}}%
\pgfpathlineto{\pgfqpoint{0.609943in}{1.237912in}}%
\pgfpathlineto{\pgfqpoint{0.623212in}{1.239312in}}%
\pgfpathlineto{\pgfqpoint{0.636481in}{1.234791in}}%
\pgfpathlineto{\pgfqpoint{0.637985in}{1.233070in}}%
\pgfpathlineto{\pgfqpoint{0.642345in}{1.220143in}}%
\pgfpathlineto{\pgfqpoint{0.643211in}{1.207217in}}%
\pgfpathlineto{\pgfqpoint{0.642426in}{1.194290in}}%
\pgfpathlineto{\pgfqpoint{0.638264in}{1.181364in}}%
\pgfpathlineto{\pgfqpoint{0.636481in}{1.179292in}}%
\pgfpathlineto{\pgfqpoint{0.623212in}{1.174847in}}%
\pgfpathlineto{\pgfqpoint{0.609943in}{1.176230in}}%
\pgfpathclose%
\pgfpathmoveto{\pgfqpoint{0.716585in}{1.181364in}}%
\pgfpathlineto{\pgfqpoint{0.716095in}{1.182301in}}%
\pgfpathlineto{\pgfqpoint{0.713303in}{1.194290in}}%
\pgfpathlineto{\pgfqpoint{0.712700in}{1.207217in}}%
\pgfpathlineto{\pgfqpoint{0.713377in}{1.220143in}}%
\pgfpathlineto{\pgfqpoint{0.716095in}{1.231517in}}%
\pgfpathlineto{\pgfqpoint{0.716931in}{1.233070in}}%
\pgfpathlineto{\pgfqpoint{0.729364in}{1.240771in}}%
\pgfpathlineto{\pgfqpoint{0.742633in}{1.241400in}}%
\pgfpathlineto{\pgfqpoint{0.755902in}{1.237047in}}%
\pgfpathlineto{\pgfqpoint{0.758924in}{1.233070in}}%
\pgfpathlineto{\pgfqpoint{0.762063in}{1.220143in}}%
\pgfpathlineto{\pgfqpoint{0.762694in}{1.207217in}}%
\pgfpathlineto{\pgfqpoint{0.762140in}{1.194290in}}%
\pgfpathlineto{\pgfqpoint{0.759186in}{1.181364in}}%
\pgfpathlineto{\pgfqpoint{0.755902in}{1.176971in}}%
\pgfpathlineto{\pgfqpoint{0.742633in}{1.172738in}}%
\pgfpathlineto{\pgfqpoint{0.729364in}{1.173357in}}%
\pgfpathclose%
\pgfpathmoveto{\pgfqpoint{0.770440in}{1.181364in}}%
\pgfpathlineto{\pgfqpoint{0.769170in}{1.191321in}}%
\pgfpathlineto{\pgfqpoint{0.768958in}{1.194290in}}%
\pgfpathlineto{\pgfqpoint{0.768671in}{1.207217in}}%
\pgfpathlineto{\pgfqpoint{0.768889in}{1.220143in}}%
\pgfpathlineto{\pgfqpoint{0.769170in}{1.224392in}}%
\pgfpathlineto{\pgfqpoint{0.770198in}{1.233070in}}%
\pgfpathlineto{\pgfqpoint{0.782439in}{1.244410in}}%
\pgfpathlineto{\pgfqpoint{0.795708in}{1.245156in}}%
\pgfpathlineto{\pgfqpoint{0.808977in}{1.244168in}}%
\pgfpathlineto{\pgfqpoint{0.821836in}{1.233070in}}%
\pgfpathlineto{\pgfqpoint{0.822246in}{1.231037in}}%
\pgfpathlineto{\pgfqpoint{0.823424in}{1.220143in}}%
\pgfpathlineto{\pgfqpoint{0.823708in}{1.207217in}}%
\pgfpathlineto{\pgfqpoint{0.823358in}{1.194290in}}%
\pgfpathlineto{\pgfqpoint{0.822246in}{1.184452in}}%
\pgfpathlineto{\pgfqpoint{0.821597in}{1.181364in}}%
\pgfpathlineto{\pgfqpoint{0.808977in}{1.170520in}}%
\pgfpathlineto{\pgfqpoint{0.795708in}{1.169471in}}%
\pgfpathlineto{\pgfqpoint{0.782439in}{1.170302in}}%
\pgfpathclose%
\pgfpathmoveto{\pgfqpoint{0.832264in}{1.181364in}}%
\pgfpathlineto{\pgfqpoint{0.829921in}{1.194290in}}%
\pgfpathlineto{\pgfqpoint{0.829478in}{1.207217in}}%
\pgfpathlineto{\pgfqpoint{0.829994in}{1.220143in}}%
\pgfpathlineto{\pgfqpoint{0.832508in}{1.233070in}}%
\pgfpathlineto{\pgfqpoint{0.835515in}{1.237805in}}%
\pgfpathlineto{\pgfqpoint{0.848784in}{1.242791in}}%
\pgfpathlineto{\pgfqpoint{0.862053in}{1.242950in}}%
\pgfpathlineto{\pgfqpoint{0.875322in}{1.238705in}}%
\pgfpathlineto{\pgfqpoint{0.878997in}{1.233070in}}%
\pgfpathlineto{\pgfqpoint{0.881285in}{1.220143in}}%
\pgfpathlineto{\pgfqpoint{0.881751in}{1.207217in}}%
\pgfpathlineto{\pgfqpoint{0.881360in}{1.194290in}}%
\pgfpathlineto{\pgfqpoint{0.879245in}{1.181364in}}%
\pgfpathlineto{\pgfqpoint{0.875322in}{1.175239in}}%
\pgfpathlineto{\pgfqpoint{0.862053in}{1.171169in}}%
\pgfpathlineto{\pgfqpoint{0.848784in}{1.171329in}}%
\pgfpathlineto{\pgfqpoint{0.835515in}{1.176153in}}%
\pgfpathclose%
\pgfpathmoveto{\pgfqpoint{0.890396in}{1.181364in}}%
\pgfpathlineto{\pgfqpoint{0.888591in}{1.189569in}}%
\pgfpathlineto{\pgfqpoint{0.888043in}{1.194290in}}%
\pgfpathlineto{\pgfqpoint{0.887614in}{1.207217in}}%
\pgfpathlineto{\pgfqpoint{0.887977in}{1.220143in}}%
\pgfpathlineto{\pgfqpoint{0.888591in}{1.225672in}}%
\pgfpathlineto{\pgfqpoint{0.890151in}{1.233070in}}%
\pgfpathlineto{\pgfqpoint{0.901860in}{1.242805in}}%
\pgfpathlineto{\pgfqpoint{0.915129in}{1.243773in}}%
\pgfpathlineto{\pgfqpoint{0.928398in}{1.242096in}}%
\pgfpathlineto{\pgfqpoint{0.937796in}{1.233070in}}%
\pgfpathlineto{\pgfqpoint{0.940174in}{1.220143in}}%
\pgfpathlineto{\pgfqpoint{0.940623in}{1.207217in}}%
\pgfpathlineto{\pgfqpoint{0.940102in}{1.194290in}}%
\pgfpathlineto{\pgfqpoint{0.937571in}{1.181364in}}%
\pgfpathlineto{\pgfqpoint{0.928398in}{1.172598in}}%
\pgfpathlineto{\pgfqpoint{0.915129in}{1.170846in}}%
\pgfpathlineto{\pgfqpoint{0.901860in}{1.171872in}}%
\pgfpathclose%
\pgfpathmoveto{\pgfqpoint{0.948717in}{1.181364in}}%
\pgfpathlineto{\pgfqpoint{0.946923in}{1.194290in}}%
\pgfpathlineto{\pgfqpoint{0.946590in}{1.207217in}}%
\pgfpathlineto{\pgfqpoint{0.946997in}{1.220143in}}%
\pgfpathlineto{\pgfqpoint{0.948961in}{1.233070in}}%
\pgfpathlineto{\pgfqpoint{0.954936in}{1.241153in}}%
\pgfpathlineto{\pgfqpoint{0.968205in}{1.244087in}}%
\pgfpathlineto{\pgfqpoint{0.981473in}{1.243951in}}%
\pgfpathlineto{\pgfqpoint{0.994742in}{1.239521in}}%
\pgfpathlineto{\pgfqpoint{0.998281in}{1.233070in}}%
\pgfpathlineto{\pgfqpoint{1.000046in}{1.220143in}}%
\pgfpathlineto{\pgfqpoint{1.000410in}{1.207217in}}%
\pgfpathlineto{\pgfqpoint{1.000118in}{1.194290in}}%
\pgfpathlineto{\pgfqpoint{0.998521in}{1.181364in}}%
\pgfpathlineto{\pgfqpoint{0.994742in}{1.174334in}}%
\pgfpathlineto{\pgfqpoint{0.981473in}{1.170152in}}%
\pgfpathlineto{\pgfqpoint{0.968205in}{1.170030in}}%
\pgfpathlineto{\pgfqpoint{0.954936in}{1.172814in}}%
\pgfpathclose%
\pgfpathmoveto{\pgfqpoint{1.009546in}{1.181364in}}%
\pgfpathlineto{\pgfqpoint{1.008011in}{1.186150in}}%
\pgfpathlineto{\pgfqpoint{1.006714in}{1.194290in}}%
\pgfpathlineto{\pgfqpoint{1.006199in}{1.207217in}}%
\pgfpathlineto{\pgfqpoint{1.006649in}{1.220143in}}%
\pgfpathlineto{\pgfqpoint{1.008011in}{1.228952in}}%
\pgfpathlineto{\pgfqpoint{1.009295in}{1.233070in}}%
\pgfpathlineto{\pgfqpoint{1.021280in}{1.242034in}}%
\pgfpathlineto{\pgfqpoint{1.034549in}{1.242917in}}%
\pgfpathlineto{\pgfqpoint{1.047818in}{1.240475in}}%
\pgfpathlineto{\pgfqpoint{1.054715in}{1.233070in}}%
\pgfpathlineto{\pgfqpoint{1.057393in}{1.220143in}}%
\pgfpathlineto{\pgfqpoint{1.057900in}{1.207217in}}%
\pgfpathlineto{\pgfqpoint{1.057325in}{1.194290in}}%
\pgfpathlineto{\pgfqpoint{1.054500in}{1.181364in}}%
\pgfpathlineto{\pgfqpoint{1.047818in}{1.174231in}}%
\pgfpathlineto{\pgfqpoint{1.034549in}{1.171698in}}%
\pgfpathlineto{\pgfqpoint{1.021280in}{1.172621in}}%
\pgfpathclose%
\pgfpathmoveto{\pgfqpoint{1.065798in}{1.181364in}}%
\pgfpathlineto{\pgfqpoint{1.064294in}{1.194290in}}%
\pgfpathlineto{\pgfqpoint{1.064019in}{1.207217in}}%
\pgfpathlineto{\pgfqpoint{1.064369in}{1.220143in}}%
\pgfpathlineto{\pgfqpoint{1.066046in}{1.233070in}}%
\pgfpathlineto{\pgfqpoint{1.074356in}{1.242861in}}%
\pgfpathlineto{\pgfqpoint{1.087625in}{1.244737in}}%
\pgfpathlineto{\pgfqpoint{1.100894in}{1.244365in}}%
\pgfpathlineto{\pgfqpoint{1.114163in}{1.239006in}}%
\pgfpathlineto{\pgfqpoint{1.116819in}{1.233070in}}%
\pgfpathlineto{\pgfqpoint{1.118359in}{1.220143in}}%
\pgfpathlineto{\pgfqpoint{1.118680in}{1.207217in}}%
\pgfpathlineto{\pgfqpoint{1.118430in}{1.194290in}}%
\pgfpathlineto{\pgfqpoint{1.117055in}{1.181364in}}%
\pgfpathlineto{\pgfqpoint{1.114163in}{1.174744in}}%
\pgfpathlineto{\pgfqpoint{1.100894in}{1.169722in}}%
\pgfpathlineto{\pgfqpoint{1.087625in}{1.169380in}}%
\pgfpathlineto{\pgfqpoint{1.074356in}{1.171129in}}%
\pgfpathclose%
\pgfpathmoveto{\pgfqpoint{1.127775in}{1.181364in}}%
\pgfpathlineto{\pgfqpoint{1.127432in}{1.182157in}}%
\pgfpathlineto{\pgfqpoint{1.124983in}{1.194290in}}%
\pgfpathlineto{\pgfqpoint{1.124433in}{1.207217in}}%
\pgfpathlineto{\pgfqpoint{1.124919in}{1.220143in}}%
\pgfpathlineto{\pgfqpoint{1.127432in}{1.232881in}}%
\pgfpathlineto{\pgfqpoint{1.127512in}{1.233070in}}%
\pgfpathlineto{\pgfqpoint{1.140701in}{1.241945in}}%
\pgfpathlineto{\pgfqpoint{1.153970in}{1.242583in}}%
\pgfpathlineto{\pgfqpoint{1.167239in}{1.239399in}}%
\pgfpathlineto{\pgfqpoint{1.172473in}{1.233070in}}%
\pgfpathlineto{\pgfqpoint{1.175173in}{1.220143in}}%
\pgfpathlineto{\pgfqpoint{1.175683in}{1.207217in}}%
\pgfpathlineto{\pgfqpoint{1.175106in}{1.194290in}}%
\pgfpathlineto{\pgfqpoint{1.172264in}{1.181364in}}%
\pgfpathlineto{\pgfqpoint{1.167239in}{1.175328in}}%
\pgfpathlineto{\pgfqpoint{1.153970in}{1.172033in}}%
\pgfpathlineto{\pgfqpoint{1.140701in}{1.172696in}}%
\pgfpathclose%
\pgfpathmoveto{\pgfqpoint{1.183526in}{1.181364in}}%
\pgfpathlineto{\pgfqpoint{1.182033in}{1.194290in}}%
\pgfpathlineto{\pgfqpoint{1.181762in}{1.207217in}}%
\pgfpathlineto{\pgfqpoint{1.182110in}{1.220143in}}%
\pgfpathlineto{\pgfqpoint{1.183782in}{1.233070in}}%
\pgfpathlineto{\pgfqpoint{1.193777in}{1.243417in}}%
\pgfpathlineto{\pgfqpoint{1.207045in}{1.244789in}}%
\pgfpathlineto{\pgfqpoint{1.220314in}{1.244126in}}%
\pgfpathlineto{\pgfqpoint{1.233583in}{1.236062in}}%
\pgfpathlineto{\pgfqpoint{1.234623in}{1.233070in}}%
\pgfpathlineto{\pgfqpoint{1.236221in}{1.220143in}}%
\pgfpathlineto{\pgfqpoint{1.236555in}{1.207217in}}%
\pgfpathlineto{\pgfqpoint{1.236293in}{1.194290in}}%
\pgfpathlineto{\pgfqpoint{1.234859in}{1.181364in}}%
\pgfpathlineto{\pgfqpoint{1.233583in}{1.177583in}}%
\pgfpathlineto{\pgfqpoint{1.220314in}{1.169949in}}%
\pgfpathlineto{\pgfqpoint{1.207045in}{1.169331in}}%
\pgfpathlineto{\pgfqpoint{1.193777in}{1.170602in}}%
\pgfpathclose%
\pgfpathmoveto{\pgfqpoint{1.245471in}{1.181364in}}%
\pgfpathlineto{\pgfqpoint{1.242845in}{1.194290in}}%
\pgfpathlineto{\pgfqpoint{1.242311in}{1.207217in}}%
\pgfpathlineto{\pgfqpoint{1.242782in}{1.220143in}}%
\pgfpathlineto{\pgfqpoint{1.245271in}{1.233070in}}%
\pgfpathlineto{\pgfqpoint{1.246852in}{1.235985in}}%
\pgfpathlineto{\pgfqpoint{1.260121in}{1.242441in}}%
\pgfpathlineto{\pgfqpoint{1.273390in}{1.242786in}}%
\pgfpathlineto{\pgfqpoint{1.286659in}{1.239034in}}%
\pgfpathlineto{\pgfqpoint{1.290994in}{1.233070in}}%
\pgfpathlineto{\pgfqpoint{1.293450in}{1.220143in}}%
\pgfpathlineto{\pgfqpoint{1.293912in}{1.207217in}}%
\pgfpathlineto{\pgfqpoint{1.293385in}{1.194290in}}%
\pgfpathlineto{\pgfqpoint{1.290786in}{1.181364in}}%
\pgfpathlineto{\pgfqpoint{1.286659in}{1.175731in}}%
\pgfpathlineto{\pgfqpoint{1.273390in}{1.171837in}}%
\pgfpathlineto{\pgfqpoint{1.260121in}{1.172193in}}%
\pgfpathlineto{\pgfqpoint{1.246852in}{1.178846in}}%
\pgfpathclose%
\pgfpathmoveto{\pgfqpoint{1.301953in}{1.181364in}}%
\pgfpathlineto{\pgfqpoint{1.300156in}{1.194290in}}%
\pgfpathlineto{\pgfqpoint{1.299928in}{1.202897in}}%
\pgfpathlineto{\pgfqpoint{1.299837in}{1.207217in}}%
\pgfpathlineto{\pgfqpoint{1.299928in}{1.210710in}}%
\pgfpathlineto{\pgfqpoint{1.300237in}{1.220143in}}%
\pgfpathlineto{\pgfqpoint{1.302223in}{1.233070in}}%
\pgfpathlineto{\pgfqpoint{1.313197in}{1.243111in}}%
\pgfpathlineto{\pgfqpoint{1.326466in}{1.244265in}}%
\pgfpathlineto{\pgfqpoint{1.339735in}{1.243123in}}%
\pgfpathlineto{\pgfqpoint{1.351478in}{1.233070in}}%
\pgfpathlineto{\pgfqpoint{1.353004in}{1.225651in}}%
\pgfpathlineto{\pgfqpoint{1.353612in}{1.220143in}}%
\pgfpathlineto{\pgfqpoint{1.354013in}{1.207217in}}%
\pgfpathlineto{\pgfqpoint{1.353685in}{1.194290in}}%
\pgfpathlineto{\pgfqpoint{1.353004in}{1.187787in}}%
\pgfpathlineto{\pgfqpoint{1.351754in}{1.181364in}}%
\pgfpathlineto{\pgfqpoint{1.339735in}{1.170943in}}%
\pgfpathlineto{\pgfqpoint{1.326466in}{1.169860in}}%
\pgfpathlineto{\pgfqpoint{1.313197in}{1.170938in}}%
\pgfpathclose%
\pgfpathmoveto{\pgfqpoint{1.362572in}{1.181364in}}%
\pgfpathlineto{\pgfqpoint{1.360279in}{1.194290in}}%
\pgfpathlineto{\pgfqpoint{1.359809in}{1.207217in}}%
\pgfpathlineto{\pgfqpoint{1.360215in}{1.220143in}}%
\pgfpathlineto{\pgfqpoint{1.362368in}{1.233070in}}%
\pgfpathlineto{\pgfqpoint{1.366273in}{1.239200in}}%
\pgfpathlineto{\pgfqpoint{1.379542in}{1.243462in}}%
\pgfpathlineto{\pgfqpoint{1.392811in}{1.243562in}}%
\pgfpathlineto{\pgfqpoint{1.406080in}{1.239672in}}%
\pgfpathlineto{\pgfqpoint{1.410231in}{1.233070in}}%
\pgfpathlineto{\pgfqpoint{1.412187in}{1.220143in}}%
\pgfpathlineto{\pgfqpoint{1.412549in}{1.207217in}}%
\pgfpathlineto{\pgfqpoint{1.412121in}{1.194290in}}%
\pgfpathlineto{\pgfqpoint{1.410021in}{1.181364in}}%
\pgfpathlineto{\pgfqpoint{1.406080in}{1.175157in}}%
\pgfpathlineto{\pgfqpoint{1.392811in}{1.171072in}}%
\pgfpathlineto{\pgfqpoint{1.379542in}{1.171170in}}%
\pgfpathlineto{\pgfqpoint{1.366273in}{1.175607in}}%
\pgfpathclose%
\pgfpathmoveto{\pgfqpoint{1.421169in}{1.181364in}}%
\pgfpathlineto{\pgfqpoint{1.419348in}{1.189538in}}%
\pgfpathlineto{\pgfqpoint{1.418794in}{1.194290in}}%
\pgfpathlineto{\pgfqpoint{1.418403in}{1.207217in}}%
\pgfpathlineto{\pgfqpoint{1.418868in}{1.220143in}}%
\pgfpathlineto{\pgfqpoint{1.419348in}{1.224063in}}%
\pgfpathlineto{\pgfqpoint{1.421460in}{1.233070in}}%
\pgfpathlineto{\pgfqpoint{1.432617in}{1.242117in}}%
\pgfpathlineto{\pgfqpoint{1.445886in}{1.243165in}}%
\pgfpathlineto{\pgfqpoint{1.459155in}{1.241188in}}%
\pgfpathlineto{\pgfqpoint{1.467567in}{1.233070in}}%
\pgfpathlineto{\pgfqpoint{1.470334in}{1.220143in}}%
\pgfpathlineto{\pgfqpoint{1.470903in}{1.207217in}}%
\pgfpathlineto{\pgfqpoint{1.470415in}{1.194290in}}%
\pgfpathlineto{\pgfqpoint{1.467835in}{1.181364in}}%
\pgfpathlineto{\pgfqpoint{1.459155in}{1.172875in}}%
\pgfpathlineto{\pgfqpoint{1.445886in}{1.170968in}}%
\pgfpathlineto{\pgfqpoint{1.432617in}{1.171962in}}%
\pgfpathclose%
\pgfpathmoveto{\pgfqpoint{0.662984in}{1.258923in}}%
\pgfpathlineto{\pgfqpoint{0.656456in}{1.271849in}}%
\pgfpathlineto{\pgfqpoint{0.655116in}{1.284776in}}%
\pgfpathlineto{\pgfqpoint{0.655534in}{1.297702in}}%
\pgfpathlineto{\pgfqpoint{0.658418in}{1.310629in}}%
\pgfpathlineto{\pgfqpoint{0.663019in}{1.316936in}}%
\pgfpathlineto{\pgfqpoint{0.676288in}{1.321218in}}%
\pgfpathlineto{\pgfqpoint{0.689557in}{1.319909in}}%
\pgfpathlineto{\pgfqpoint{0.699502in}{1.310629in}}%
\pgfpathlineto{\pgfqpoint{0.702604in}{1.297702in}}%
\pgfpathlineto{\pgfqpoint{0.702826in}{1.291482in}}%
\pgfpathlineto{\pgfqpoint{0.703009in}{1.284776in}}%
\pgfpathlineto{\pgfqpoint{0.702826in}{1.282295in}}%
\pgfpathlineto{\pgfqpoint{0.701645in}{1.271849in}}%
\pgfpathlineto{\pgfqpoint{0.694476in}{1.258923in}}%
\pgfpathlineto{\pgfqpoint{0.689557in}{1.256248in}}%
\pgfpathlineto{\pgfqpoint{0.676288in}{1.255121in}}%
\pgfpathlineto{\pgfqpoint{0.663019in}{1.258894in}}%
\pgfpathclose%
\pgfpathmoveto{\pgfqpoint{0.710349in}{1.258923in}}%
\pgfpathlineto{\pgfqpoint{0.709174in}{1.271849in}}%
\pgfpathlineto{\pgfqpoint{0.708904in}{1.284776in}}%
\pgfpathlineto{\pgfqpoint{0.708886in}{1.297702in}}%
\pgfpathlineto{\pgfqpoint{0.709142in}{1.310629in}}%
\pgfpathlineto{\pgfqpoint{0.711180in}{1.323555in}}%
\pgfpathlineto{\pgfqpoint{0.716095in}{1.326787in}}%
\pgfpathlineto{\pgfqpoint{0.729364in}{1.327783in}}%
\pgfpathlineto{\pgfqpoint{0.742633in}{1.327642in}}%
\pgfpathlineto{\pgfqpoint{0.755902in}{1.326241in}}%
\pgfpathlineto{\pgfqpoint{0.760929in}{1.323555in}}%
\pgfpathlineto{\pgfqpoint{0.764780in}{1.310629in}}%
\pgfpathlineto{\pgfqpoint{0.765331in}{1.297702in}}%
\pgfpathlineto{\pgfqpoint{0.765351in}{1.284776in}}%
\pgfpathlineto{\pgfqpoint{0.764929in}{1.271849in}}%
\pgfpathlineto{\pgfqpoint{0.763065in}{1.258923in}}%
\pgfpathlineto{\pgfqpoint{0.755902in}{1.251563in}}%
\pgfpathlineto{\pgfqpoint{0.742633in}{1.249795in}}%
\pgfpathlineto{\pgfqpoint{0.729364in}{1.249685in}}%
\pgfpathlineto{\pgfqpoint{0.716095in}{1.251311in}}%
\pgfpathclose%
\pgfpathmoveto{\pgfqpoint{0.777574in}{1.258923in}}%
\pgfpathlineto{\pgfqpoint{0.772479in}{1.271849in}}%
\pgfpathlineto{\pgfqpoint{0.771447in}{1.284776in}}%
\pgfpathlineto{\pgfqpoint{0.771792in}{1.297702in}}%
\pgfpathlineto{\pgfqpoint{0.774094in}{1.310629in}}%
\pgfpathlineto{\pgfqpoint{0.782439in}{1.320762in}}%
\pgfpathlineto{\pgfqpoint{0.795708in}{1.323306in}}%
\pgfpathlineto{\pgfqpoint{0.808977in}{1.321827in}}%
\pgfpathlineto{\pgfqpoint{0.819633in}{1.310629in}}%
\pgfpathlineto{\pgfqpoint{0.821855in}{1.297702in}}%
\pgfpathlineto{\pgfqpoint{0.822191in}{1.284776in}}%
\pgfpathlineto{\pgfqpoint{0.821220in}{1.271849in}}%
\pgfpathlineto{\pgfqpoint{0.816255in}{1.258923in}}%
\pgfpathlineto{\pgfqpoint{0.808977in}{1.254443in}}%
\pgfpathlineto{\pgfqpoint{0.795708in}{1.253198in}}%
\pgfpathlineto{\pgfqpoint{0.782439in}{1.255391in}}%
\pgfpathclose%
\pgfpathmoveto{\pgfqpoint{0.831356in}{1.258923in}}%
\pgfpathlineto{\pgfqpoint{0.828703in}{1.271849in}}%
\pgfpathlineto{\pgfqpoint{0.828131in}{1.284776in}}%
\pgfpathlineto{\pgfqpoint{0.828187in}{1.297702in}}%
\pgfpathlineto{\pgfqpoint{0.829033in}{1.310629in}}%
\pgfpathlineto{\pgfqpoint{0.835305in}{1.323555in}}%
\pgfpathlineto{\pgfqpoint{0.835515in}{1.323674in}}%
\pgfpathlineto{\pgfqpoint{0.848784in}{1.326104in}}%
\pgfpathlineto{\pgfqpoint{0.862053in}{1.325946in}}%
\pgfpathlineto{\pgfqpoint{0.873354in}{1.323555in}}%
\pgfpathlineto{\pgfqpoint{0.875322in}{1.322654in}}%
\pgfpathlineto{\pgfqpoint{0.881123in}{1.310629in}}%
\pgfpathlineto{\pgfqpoint{0.882150in}{1.297702in}}%
\pgfpathlineto{\pgfqpoint{0.882230in}{1.284776in}}%
\pgfpathlineto{\pgfqpoint{0.881566in}{1.271849in}}%
\pgfpathlineto{\pgfqpoint{0.878530in}{1.258923in}}%
\pgfpathlineto{\pgfqpoint{0.875322in}{1.255138in}}%
\pgfpathlineto{\pgfqpoint{0.862053in}{1.251507in}}%
\pgfpathlineto{\pgfqpoint{0.848784in}{1.251348in}}%
\pgfpathlineto{\pgfqpoint{0.835515in}{1.254220in}}%
\pgfpathclose%
\pgfpathmoveto{\pgfqpoint{0.892816in}{1.258923in}}%
\pgfpathlineto{\pgfqpoint{0.888784in}{1.271849in}}%
\pgfpathlineto{\pgfqpoint{0.888591in}{1.274493in}}%
\pgfpathlineto{\pgfqpoint{0.888064in}{1.284776in}}%
\pgfpathlineto{\pgfqpoint{0.888318in}{1.297702in}}%
\pgfpathlineto{\pgfqpoint{0.888591in}{1.300612in}}%
\pgfpathlineto{\pgfqpoint{0.890149in}{1.310629in}}%
\pgfpathlineto{\pgfqpoint{0.901860in}{1.323265in}}%
\pgfpathlineto{\pgfqpoint{0.903988in}{1.323555in}}%
\pgfpathlineto{\pgfqpoint{0.915129in}{1.324628in}}%
\pgfpathlineto{\pgfqpoint{0.925761in}{1.323555in}}%
\pgfpathlineto{\pgfqpoint{0.928398in}{1.323152in}}%
\pgfpathlineto{\pgfqpoint{0.938971in}{1.310629in}}%
\pgfpathlineto{\pgfqpoint{0.940626in}{1.297702in}}%
\pgfpathlineto{\pgfqpoint{0.940889in}{1.284776in}}%
\pgfpathlineto{\pgfqpoint{0.940198in}{1.271849in}}%
\pgfpathlineto{\pgfqpoint{0.936642in}{1.258923in}}%
\pgfpathlineto{\pgfqpoint{0.928398in}{1.253176in}}%
\pgfpathlineto{\pgfqpoint{0.915129in}{1.251864in}}%
\pgfpathlineto{\pgfqpoint{0.901860in}{1.253112in}}%
\pgfpathclose%
\pgfpathmoveto{\pgfqpoint{0.951444in}{1.258923in}}%
\pgfpathlineto{\pgfqpoint{0.947751in}{1.271849in}}%
\pgfpathlineto{\pgfqpoint{0.946963in}{1.284776in}}%
\pgfpathlineto{\pgfqpoint{0.947070in}{1.297702in}}%
\pgfpathlineto{\pgfqpoint{0.948335in}{1.310629in}}%
\pgfpathlineto{\pgfqpoint{0.954936in}{1.322139in}}%
\pgfpathlineto{\pgfqpoint{0.959272in}{1.323555in}}%
\pgfpathlineto{\pgfqpoint{0.968205in}{1.325059in}}%
\pgfpathlineto{\pgfqpoint{0.981473in}{1.324713in}}%
\pgfpathlineto{\pgfqpoint{0.986449in}{1.323555in}}%
\pgfpathlineto{\pgfqpoint{0.994742in}{1.318804in}}%
\pgfpathlineto{\pgfqpoint{0.998082in}{1.310629in}}%
\pgfpathlineto{\pgfqpoint{0.999412in}{1.297702in}}%
\pgfpathlineto{\pgfqpoint{0.999531in}{1.284776in}}%
\pgfpathlineto{\pgfqpoint{0.998712in}{1.271849in}}%
\pgfpathlineto{\pgfqpoint{0.994916in}{1.258923in}}%
\pgfpathlineto{\pgfqpoint{0.994742in}{1.258683in}}%
\pgfpathlineto{\pgfqpoint{0.981473in}{1.252759in}}%
\pgfpathlineto{\pgfqpoint{0.968205in}{1.252383in}}%
\pgfpathlineto{\pgfqpoint{0.954936in}{1.255484in}}%
\pgfpathclose%
\pgfpathmoveto{\pgfqpoint{1.008745in}{1.258923in}}%
\pgfpathlineto{\pgfqpoint{1.008011in}{1.260558in}}%
\pgfpathlineto{\pgfqpoint{1.005867in}{1.271849in}}%
\pgfpathlineto{\pgfqpoint{1.005330in}{1.284776in}}%
\pgfpathlineto{\pgfqpoint{1.005544in}{1.297702in}}%
\pgfpathlineto{\pgfqpoint{1.006857in}{1.310629in}}%
\pgfpathlineto{\pgfqpoint{1.008011in}{1.314699in}}%
\pgfpathlineto{\pgfqpoint{1.017519in}{1.323555in}}%
\pgfpathlineto{\pgfqpoint{1.021280in}{1.324611in}}%
\pgfpathlineto{\pgfqpoint{1.034549in}{1.325376in}}%
\pgfpathlineto{\pgfqpoint{1.047818in}{1.323756in}}%
\pgfpathlineto{\pgfqpoint{1.048343in}{1.323555in}}%
\pgfpathlineto{\pgfqpoint{1.057640in}{1.310629in}}%
\pgfpathlineto{\pgfqpoint{1.058983in}{1.297702in}}%
\pgfpathlineto{\pgfqpoint{1.059204in}{1.284776in}}%
\pgfpathlineto{\pgfqpoint{1.058666in}{1.271849in}}%
\pgfpathlineto{\pgfqpoint{1.055877in}{1.258923in}}%
\pgfpathlineto{\pgfqpoint{1.047818in}{1.252541in}}%
\pgfpathlineto{\pgfqpoint{1.034549in}{1.251091in}}%
\pgfpathlineto{\pgfqpoint{1.021280in}{1.251788in}}%
\pgfpathclose%
\pgfpathmoveto{\pgfqpoint{1.070613in}{1.258923in}}%
\pgfpathlineto{\pgfqpoint{1.066301in}{1.271849in}}%
\pgfpathlineto{\pgfqpoint{1.065382in}{1.284776in}}%
\pgfpathlineto{\pgfqpoint{1.065520in}{1.297702in}}%
\pgfpathlineto{\pgfqpoint{1.067034in}{1.310629in}}%
\pgfpathlineto{\pgfqpoint{1.074356in}{1.321844in}}%
\pgfpathlineto{\pgfqpoint{1.080806in}{1.323555in}}%
\pgfpathlineto{\pgfqpoint{1.087625in}{1.324578in}}%
\pgfpathlineto{\pgfqpoint{1.100894in}{1.323968in}}%
\pgfpathlineto{\pgfqpoint{1.102508in}{1.323555in}}%
\pgfpathlineto{\pgfqpoint{1.114163in}{1.314952in}}%
\pgfpathlineto{\pgfqpoint{1.115609in}{1.310629in}}%
\pgfpathlineto{\pgfqpoint{1.117079in}{1.297702in}}%
\pgfpathlineto{\pgfqpoint{1.117215in}{1.284776in}}%
\pgfpathlineto{\pgfqpoint{1.116325in}{1.271849in}}%
\pgfpathlineto{\pgfqpoint{1.114163in}{1.263116in}}%
\pgfpathlineto{\pgfqpoint{1.111556in}{1.258923in}}%
\pgfpathlineto{\pgfqpoint{1.100894in}{1.253526in}}%
\pgfpathlineto{\pgfqpoint{1.087625in}{1.252856in}}%
\pgfpathlineto{\pgfqpoint{1.074356in}{1.255667in}}%
\pgfpathclose%
\pgfpathmoveto{\pgfqpoint{1.125927in}{1.258923in}}%
\pgfpathlineto{\pgfqpoint{1.123488in}{1.271849in}}%
\pgfpathlineto{\pgfqpoint{1.123016in}{1.284776in}}%
\pgfpathlineto{\pgfqpoint{1.123215in}{1.297702in}}%
\pgfpathlineto{\pgfqpoint{1.124403in}{1.310629in}}%
\pgfpathlineto{\pgfqpoint{1.127432in}{1.319019in}}%
\pgfpathlineto{\pgfqpoint{1.134151in}{1.323555in}}%
\pgfpathlineto{\pgfqpoint{1.140701in}{1.325168in}}%
\pgfpathlineto{\pgfqpoint{1.153970in}{1.325579in}}%
\pgfpathlineto{\pgfqpoint{1.167168in}{1.323555in}}%
\pgfpathlineto{\pgfqpoint{1.167239in}{1.323536in}}%
\pgfpathlineto{\pgfqpoint{1.175715in}{1.310629in}}%
\pgfpathlineto{\pgfqpoint{1.176962in}{1.297702in}}%
\pgfpathlineto{\pgfqpoint{1.177170in}{1.284776in}}%
\pgfpathlineto{\pgfqpoint{1.176675in}{1.271849in}}%
\pgfpathlineto{\pgfqpoint{1.174115in}{1.258923in}}%
\pgfpathlineto{\pgfqpoint{1.167239in}{1.252699in}}%
\pgfpathlineto{\pgfqpoint{1.153970in}{1.250876in}}%
\pgfpathlineto{\pgfqpoint{1.140701in}{1.251241in}}%
\pgfpathlineto{\pgfqpoint{1.127432in}{1.256405in}}%
\pgfpathclose%
\pgfpathmoveto{\pgfqpoint{1.188823in}{1.258923in}}%
\pgfpathlineto{\pgfqpoint{1.184317in}{1.271849in}}%
\pgfpathlineto{\pgfqpoint{1.183352in}{1.284776in}}%
\pgfpathlineto{\pgfqpoint{1.183499in}{1.297702in}}%
\pgfpathlineto{\pgfqpoint{1.185093in}{1.310629in}}%
\pgfpathlineto{\pgfqpoint{1.193777in}{1.322433in}}%
\pgfpathlineto{\pgfqpoint{1.198989in}{1.323555in}}%
\pgfpathlineto{\pgfqpoint{1.207045in}{1.324620in}}%
\pgfpathlineto{\pgfqpoint{1.220314in}{1.323762in}}%
\pgfpathlineto{\pgfqpoint{1.221049in}{1.323555in}}%
\pgfpathlineto{\pgfqpoint{1.233583in}{1.311006in}}%
\pgfpathlineto{\pgfqpoint{1.233681in}{1.310629in}}%
\pgfpathlineto{\pgfqpoint{1.235125in}{1.297702in}}%
\pgfpathlineto{\pgfqpoint{1.235257in}{1.284776in}}%
\pgfpathlineto{\pgfqpoint{1.234380in}{1.271849in}}%
\pgfpathlineto{\pgfqpoint{1.233583in}{1.267754in}}%
\pgfpathlineto{\pgfqpoint{1.229522in}{1.258923in}}%
\pgfpathlineto{\pgfqpoint{1.220314in}{1.253759in}}%
\pgfpathlineto{\pgfqpoint{1.207045in}{1.252808in}}%
\pgfpathlineto{\pgfqpoint{1.193777in}{1.255081in}}%
\pgfpathclose%
\pgfpathmoveto{\pgfqpoint{1.244187in}{1.258923in}}%
\pgfpathlineto{\pgfqpoint{1.241599in}{1.271849in}}%
\pgfpathlineto{\pgfqpoint{1.241099in}{1.284776in}}%
\pgfpathlineto{\pgfqpoint{1.241305in}{1.297702in}}%
\pgfpathlineto{\pgfqpoint{1.242551in}{1.310629in}}%
\pgfpathlineto{\pgfqpoint{1.246852in}{1.320420in}}%
\pgfpathlineto{\pgfqpoint{1.253027in}{1.323555in}}%
\pgfpathlineto{\pgfqpoint{1.260121in}{1.325077in}}%
\pgfpathlineto{\pgfqpoint{1.273390in}{1.325220in}}%
\pgfpathlineto{\pgfqpoint{1.282601in}{1.323555in}}%
\pgfpathlineto{\pgfqpoint{1.286659in}{1.322134in}}%
\pgfpathlineto{\pgfqpoint{1.293240in}{1.310629in}}%
\pgfpathlineto{\pgfqpoint{1.294583in}{1.297702in}}%
\pgfpathlineto{\pgfqpoint{1.294803in}{1.284776in}}%
\pgfpathlineto{\pgfqpoint{1.294253in}{1.271849in}}%
\pgfpathlineto{\pgfqpoint{1.291450in}{1.258923in}}%
\pgfpathlineto{\pgfqpoint{1.286659in}{1.253917in}}%
\pgfpathlineto{\pgfqpoint{1.273390in}{1.251236in}}%
\pgfpathlineto{\pgfqpoint{1.260121in}{1.251356in}}%
\pgfpathlineto{\pgfqpoint{1.246852in}{1.255291in}}%
\pgfpathclose%
\pgfpathmoveto{\pgfqpoint{1.305996in}{1.258923in}}%
\pgfpathlineto{\pgfqpoint{1.301737in}{1.271849in}}%
\pgfpathlineto{\pgfqpoint{1.300817in}{1.284776in}}%
\pgfpathlineto{\pgfqpoint{1.300951in}{1.297702in}}%
\pgfpathlineto{\pgfqpoint{1.302444in}{1.310629in}}%
\pgfpathlineto{\pgfqpoint{1.312855in}{1.323555in}}%
\pgfpathlineto{\pgfqpoint{1.313197in}{1.323677in}}%
\pgfpathlineto{\pgfqpoint{1.326466in}{1.325167in}}%
\pgfpathlineto{\pgfqpoint{1.339735in}{1.324180in}}%
\pgfpathlineto{\pgfqpoint{1.341760in}{1.323555in}}%
\pgfpathlineto{\pgfqpoint{1.352189in}{1.310629in}}%
\pgfpathlineto{\pgfqpoint{1.353004in}{1.304451in}}%
\pgfpathlineto{\pgfqpoint{1.353542in}{1.297702in}}%
\pgfpathlineto{\pgfqpoint{1.353648in}{1.284776in}}%
\pgfpathlineto{\pgfqpoint{1.353004in}{1.273696in}}%
\pgfpathlineto{\pgfqpoint{1.352851in}{1.271849in}}%
\pgfpathlineto{\pgfqpoint{1.348660in}{1.258923in}}%
\pgfpathlineto{\pgfqpoint{1.339735in}{1.253374in}}%
\pgfpathlineto{\pgfqpoint{1.326466in}{1.252257in}}%
\pgfpathlineto{\pgfqpoint{1.313197in}{1.253908in}}%
\pgfpathclose%
\pgfpathmoveto{\pgfqpoint{1.363405in}{1.258923in}}%
\pgfpathlineto{\pgfqpoint{1.360194in}{1.271849in}}%
\pgfpathlineto{\pgfqpoint{1.359569in}{1.284776in}}%
\pgfpathlineto{\pgfqpoint{1.359807in}{1.297702in}}%
\pgfpathlineto{\pgfqpoint{1.361303in}{1.310629in}}%
\pgfpathlineto{\pgfqpoint{1.366273in}{1.320197in}}%
\pgfpathlineto{\pgfqpoint{1.374918in}{1.323555in}}%
\pgfpathlineto{\pgfqpoint{1.379542in}{1.324411in}}%
\pgfpathlineto{\pgfqpoint{1.392811in}{1.324256in}}%
\pgfpathlineto{\pgfqpoint{1.396158in}{1.323555in}}%
\pgfpathlineto{\pgfqpoint{1.406080in}{1.319082in}}%
\pgfpathlineto{\pgfqpoint{1.410232in}{1.310629in}}%
\pgfpathlineto{\pgfqpoint{1.411848in}{1.297702in}}%
\pgfpathlineto{\pgfqpoint{1.412102in}{1.284776in}}%
\pgfpathlineto{\pgfqpoint{1.411409in}{1.271849in}}%
\pgfpathlineto{\pgfqpoint{1.407930in}{1.258923in}}%
\pgfpathlineto{\pgfqpoint{1.406080in}{1.256657in}}%
\pgfpathlineto{\pgfqpoint{1.392811in}{1.252215in}}%
\pgfpathlineto{\pgfqpoint{1.379542in}{1.252058in}}%
\pgfpathlineto{\pgfqpoint{1.366273in}{1.255643in}}%
\pgfpathclose%
\pgfpathmoveto{\pgfqpoint{1.421999in}{1.258923in}}%
\pgfpathlineto{\pgfqpoint{1.419348in}{1.266843in}}%
\pgfpathlineto{\pgfqpoint{1.418593in}{1.271849in}}%
\pgfpathlineto{\pgfqpoint{1.417929in}{1.284776in}}%
\pgfpathlineto{\pgfqpoint{1.418009in}{1.297702in}}%
\pgfpathlineto{\pgfqpoint{1.419035in}{1.310629in}}%
\pgfpathlineto{\pgfqpoint{1.419348in}{1.312214in}}%
\pgfpathlineto{\pgfqpoint{1.427242in}{1.323555in}}%
\pgfpathlineto{\pgfqpoint{1.432617in}{1.325291in}}%
\pgfpathlineto{\pgfqpoint{1.445886in}{1.326219in}}%
\pgfpathlineto{\pgfqpoint{1.459155in}{1.325360in}}%
\pgfpathlineto{\pgfqpoint{1.464454in}{1.323555in}}%
\pgfpathlineto{\pgfqpoint{1.471393in}{1.310629in}}%
\pgfpathlineto{\pgfqpoint{1.472327in}{1.297702in}}%
\pgfpathlineto{\pgfqpoint{1.472389in}{1.284776in}}%
\pgfpathlineto{\pgfqpoint{1.471757in}{1.271849in}}%
\pgfpathlineto{\pgfqpoint{1.468828in}{1.258923in}}%
\pgfpathlineto{\pgfqpoint{1.459155in}{1.252240in}}%
\pgfpathlineto{\pgfqpoint{1.445886in}{1.251202in}}%
\pgfpathlineto{\pgfqpoint{1.432617in}{1.252256in}}%
\pgfpathclose%
\pgfpathmoveto{\pgfqpoint{0.729731in}{1.336482in}}%
\pgfpathlineto{\pgfqpoint{0.729364in}{1.336504in}}%
\pgfpathlineto{\pgfqpoint{0.716095in}{1.347544in}}%
\pgfpathlineto{\pgfqpoint{0.715430in}{1.349408in}}%
\pgfpathlineto{\pgfqpoint{0.713617in}{1.362335in}}%
\pgfpathlineto{\pgfqpoint{0.713618in}{1.375261in}}%
\pgfpathlineto{\pgfqpoint{0.715359in}{1.388188in}}%
\pgfpathlineto{\pgfqpoint{0.716095in}{1.390349in}}%
\pgfpathlineto{\pgfqpoint{0.726998in}{1.401114in}}%
\pgfpathlineto{\pgfqpoint{0.729364in}{1.402020in}}%
\pgfpathlineto{\pgfqpoint{0.742633in}{1.402693in}}%
\pgfpathlineto{\pgfqpoint{0.748466in}{1.401114in}}%
\pgfpathlineto{\pgfqpoint{0.755902in}{1.396955in}}%
\pgfpathlineto{\pgfqpoint{0.760075in}{1.388188in}}%
\pgfpathlineto{\pgfqpoint{0.761744in}{1.375261in}}%
\pgfpathlineto{\pgfqpoint{0.761755in}{1.362335in}}%
\pgfpathlineto{\pgfqpoint{0.760043in}{1.349408in}}%
\pgfpathlineto{\pgfqpoint{0.755902in}{1.341233in}}%
\pgfpathlineto{\pgfqpoint{0.745432in}{1.336482in}}%
\pgfpathlineto{\pgfqpoint{0.742633in}{1.335857in}}%
\pgfpathclose%
\pgfpathmoveto{\pgfqpoint{0.771572in}{1.336482in}}%
\pgfpathlineto{\pgfqpoint{0.769170in}{1.343794in}}%
\pgfpathlineto{\pgfqpoint{0.768510in}{1.349408in}}%
\pgfpathlineto{\pgfqpoint{0.767982in}{1.362335in}}%
\pgfpathlineto{\pgfqpoint{0.767867in}{1.375261in}}%
\pgfpathlineto{\pgfqpoint{0.768038in}{1.388188in}}%
\pgfpathlineto{\pgfqpoint{0.769160in}{1.401114in}}%
\pgfpathlineto{\pgfqpoint{0.769170in}{1.401159in}}%
\pgfpathlineto{\pgfqpoint{0.782439in}{1.408620in}}%
\pgfpathlineto{\pgfqpoint{0.795708in}{1.408863in}}%
\pgfpathlineto{\pgfqpoint{0.808977in}{1.408271in}}%
\pgfpathlineto{\pgfqpoint{0.822246in}{1.401163in}}%
\pgfpathlineto{\pgfqpoint{0.822265in}{1.401114in}}%
\pgfpathlineto{\pgfqpoint{0.824142in}{1.388188in}}%
\pgfpathlineto{\pgfqpoint{0.824471in}{1.375261in}}%
\pgfpathlineto{\pgfqpoint{0.824351in}{1.362335in}}%
\pgfpathlineto{\pgfqpoint{0.823646in}{1.349408in}}%
\pgfpathlineto{\pgfqpoint{0.822246in}{1.341704in}}%
\pgfpathlineto{\pgfqpoint{0.819573in}{1.336482in}}%
\pgfpathlineto{\pgfqpoint{0.808977in}{1.331895in}}%
\pgfpathlineto{\pgfqpoint{0.795708in}{1.331119in}}%
\pgfpathlineto{\pgfqpoint{0.782439in}{1.331620in}}%
\pgfpathclose%
\pgfpathmoveto{\pgfqpoint{0.841899in}{1.336482in}}%
\pgfpathlineto{\pgfqpoint{0.835515in}{1.340564in}}%
\pgfpathlineto{\pgfqpoint{0.831752in}{1.349408in}}%
\pgfpathlineto{\pgfqpoint{0.830357in}{1.362335in}}%
\pgfpathlineto{\pgfqpoint{0.830379in}{1.375261in}}%
\pgfpathlineto{\pgfqpoint{0.831779in}{1.388188in}}%
\pgfpathlineto{\pgfqpoint{0.835515in}{1.397467in}}%
\pgfpathlineto{\pgfqpoint{0.840263in}{1.401114in}}%
\pgfpathlineto{\pgfqpoint{0.848784in}{1.403979in}}%
\pgfpathlineto{\pgfqpoint{0.862053in}{1.404129in}}%
\pgfpathlineto{\pgfqpoint{0.871655in}{1.401114in}}%
\pgfpathlineto{\pgfqpoint{0.875322in}{1.398478in}}%
\pgfpathlineto{\pgfqpoint{0.879541in}{1.388188in}}%
\pgfpathlineto{\pgfqpoint{0.880846in}{1.375261in}}%
\pgfpathlineto{\pgfqpoint{0.880875in}{1.362335in}}%
\pgfpathlineto{\pgfqpoint{0.879601in}{1.349408in}}%
\pgfpathlineto{\pgfqpoint{0.875322in}{1.339545in}}%
\pgfpathlineto{\pgfqpoint{0.870127in}{1.336482in}}%
\pgfpathlineto{\pgfqpoint{0.862053in}{1.334381in}}%
\pgfpathlineto{\pgfqpoint{0.848784in}{1.334534in}}%
\pgfpathclose%
\pgfpathmoveto{\pgfqpoint{0.893669in}{1.336482in}}%
\pgfpathlineto{\pgfqpoint{0.888591in}{1.345680in}}%
\pgfpathlineto{\pgfqpoint{0.887892in}{1.349408in}}%
\pgfpathlineto{\pgfqpoint{0.886981in}{1.362335in}}%
\pgfpathlineto{\pgfqpoint{0.886851in}{1.375261in}}%
\pgfpathlineto{\pgfqpoint{0.887342in}{1.388188in}}%
\pgfpathlineto{\pgfqpoint{0.888591in}{1.396484in}}%
\pgfpathlineto{\pgfqpoint{0.890377in}{1.401114in}}%
\pgfpathlineto{\pgfqpoint{0.901860in}{1.406957in}}%
\pgfpathlineto{\pgfqpoint{0.915129in}{1.407534in}}%
\pgfpathlineto{\pgfqpoint{0.928398in}{1.406394in}}%
\pgfpathlineto{\pgfqpoint{0.937241in}{1.401114in}}%
\pgfpathlineto{\pgfqpoint{0.940795in}{1.388188in}}%
\pgfpathlineto{\pgfqpoint{0.941446in}{1.375261in}}%
\pgfpathlineto{\pgfqpoint{0.941297in}{1.362335in}}%
\pgfpathlineto{\pgfqpoint{0.940165in}{1.349408in}}%
\pgfpathlineto{\pgfqpoint{0.934050in}{1.336482in}}%
\pgfpathlineto{\pgfqpoint{0.928398in}{1.333780in}}%
\pgfpathlineto{\pgfqpoint{0.915129in}{1.332416in}}%
\pgfpathlineto{\pgfqpoint{0.901860in}{1.333170in}}%
\pgfpathclose%
\pgfpathmoveto{\pgfqpoint{0.955112in}{1.336482in}}%
\pgfpathlineto{\pgfqpoint{0.954936in}{1.336569in}}%
\pgfpathlineto{\pgfqpoint{0.948561in}{1.349408in}}%
\pgfpathlineto{\pgfqpoint{0.947451in}{1.362335in}}%
\pgfpathlineto{\pgfqpoint{0.947488in}{1.375261in}}%
\pgfpathlineto{\pgfqpoint{0.948659in}{1.388188in}}%
\pgfpathlineto{\pgfqpoint{0.954418in}{1.401114in}}%
\pgfpathlineto{\pgfqpoint{0.954936in}{1.401541in}}%
\pgfpathlineto{\pgfqpoint{0.968205in}{1.405242in}}%
\pgfpathlineto{\pgfqpoint{0.981473in}{1.405032in}}%
\pgfpathlineto{\pgfqpoint{0.992363in}{1.401114in}}%
\pgfpathlineto{\pgfqpoint{0.994742in}{1.398899in}}%
\pgfpathlineto{\pgfqpoint{0.998453in}{1.388188in}}%
\pgfpathlineto{\pgfqpoint{0.999533in}{1.375261in}}%
\pgfpathlineto{\pgfqpoint{0.999573in}{1.362335in}}%
\pgfpathlineto{\pgfqpoint{0.998569in}{1.349408in}}%
\pgfpathlineto{\pgfqpoint{0.994742in}{1.338881in}}%
\pgfpathlineto{\pgfqpoint{0.991634in}{1.336482in}}%
\pgfpathlineto{\pgfqpoint{0.981473in}{1.333438in}}%
\pgfpathlineto{\pgfqpoint{0.968205in}{1.333270in}}%
\pgfpathclose%
\pgfpathmoveto{\pgfqpoint{1.014321in}{1.336482in}}%
\pgfpathlineto{\pgfqpoint{1.008011in}{1.344436in}}%
\pgfpathlineto{\pgfqpoint{1.006745in}{1.349408in}}%
\pgfpathlineto{\pgfqpoint{1.005602in}{1.362335in}}%
\pgfpathlineto{\pgfqpoint{1.005462in}{1.375261in}}%
\pgfpathlineto{\pgfqpoint{1.006147in}{1.388188in}}%
\pgfpathlineto{\pgfqpoint{1.008011in}{1.396995in}}%
\pgfpathlineto{\pgfqpoint{1.010377in}{1.401114in}}%
\pgfpathlineto{\pgfqpoint{1.021280in}{1.406121in}}%
\pgfpathlineto{\pgfqpoint{1.034549in}{1.406721in}}%
\pgfpathlineto{\pgfqpoint{1.047818in}{1.404982in}}%
\pgfpathlineto{\pgfqpoint{1.053630in}{1.401114in}}%
\pgfpathlineto{\pgfqpoint{1.057897in}{1.388188in}}%
\pgfpathlineto{\pgfqpoint{1.058684in}{1.375261in}}%
\pgfpathlineto{\pgfqpoint{1.058531in}{1.362335in}}%
\pgfpathlineto{\pgfqpoint{1.057244in}{1.349408in}}%
\pgfpathlineto{\pgfqpoint{1.050187in}{1.336482in}}%
\pgfpathlineto{\pgfqpoint{1.047818in}{1.335226in}}%
\pgfpathlineto{\pgfqpoint{1.034549in}{1.333215in}}%
\pgfpathlineto{\pgfqpoint{1.021280in}{1.333933in}}%
\pgfpathclose%
\pgfpathmoveto{\pgfqpoint{1.071552in}{1.336482in}}%
\pgfpathlineto{\pgfqpoint{1.065847in}{1.349408in}}%
\pgfpathlineto{\pgfqpoint{1.064883in}{1.362335in}}%
\pgfpathlineto{\pgfqpoint{1.064929in}{1.375261in}}%
\pgfpathlineto{\pgfqpoint{1.065986in}{1.388188in}}%
\pgfpathlineto{\pgfqpoint{1.071190in}{1.401114in}}%
\pgfpathlineto{\pgfqpoint{1.074356in}{1.403380in}}%
\pgfpathlineto{\pgfqpoint{1.087625in}{1.405882in}}%
\pgfpathlineto{\pgfqpoint{1.100894in}{1.405367in}}%
\pgfpathlineto{\pgfqpoint{1.111289in}{1.401114in}}%
\pgfpathlineto{\pgfqpoint{1.114163in}{1.397581in}}%
\pgfpathlineto{\pgfqpoint{1.116833in}{1.388188in}}%
\pgfpathlineto{\pgfqpoint{1.117815in}{1.375261in}}%
\pgfpathlineto{\pgfqpoint{1.117860in}{1.362335in}}%
\pgfpathlineto{\pgfqpoint{1.116972in}{1.349408in}}%
\pgfpathlineto{\pgfqpoint{1.114163in}{1.339874in}}%
\pgfpathlineto{\pgfqpoint{1.110886in}{1.336482in}}%
\pgfpathlineto{\pgfqpoint{1.100894in}{1.333065in}}%
\pgfpathlineto{\pgfqpoint{1.087625in}{1.332633in}}%
\pgfpathlineto{\pgfqpoint{1.074356in}{1.334786in}}%
\pgfpathclose%
\pgfpathmoveto{\pgfqpoint{1.133373in}{1.336482in}}%
\pgfpathlineto{\pgfqpoint{1.127432in}{1.342099in}}%
\pgfpathlineto{\pgfqpoint{1.125087in}{1.349408in}}%
\pgfpathlineto{\pgfqpoint{1.123851in}{1.362335in}}%
\pgfpathlineto{\pgfqpoint{1.123707in}{1.375261in}}%
\pgfpathlineto{\pgfqpoint{1.124471in}{1.388188in}}%
\pgfpathlineto{\pgfqpoint{1.127432in}{1.399092in}}%
\pgfpathlineto{\pgfqpoint{1.129012in}{1.401114in}}%
\pgfpathlineto{\pgfqpoint{1.140701in}{1.405961in}}%
\pgfpathlineto{\pgfqpoint{1.153970in}{1.406417in}}%
\pgfpathlineto{\pgfqpoint{1.167239in}{1.404136in}}%
\pgfpathlineto{\pgfqpoint{1.171282in}{1.401114in}}%
\pgfpathlineto{\pgfqpoint{1.175643in}{1.388188in}}%
\pgfpathlineto{\pgfqpoint{1.176445in}{1.375261in}}%
\pgfpathlineto{\pgfqpoint{1.176293in}{1.362335in}}%
\pgfpathlineto{\pgfqpoint{1.174997in}{1.349408in}}%
\pgfpathlineto{\pgfqpoint{1.167817in}{1.336482in}}%
\pgfpathlineto{\pgfqpoint{1.167239in}{1.336139in}}%
\pgfpathlineto{\pgfqpoint{1.153970in}{1.333521in}}%
\pgfpathlineto{\pgfqpoint{1.140701in}{1.334049in}}%
\pgfpathclose%
\pgfpathmoveto{\pgfqpoint{1.189380in}{1.336482in}}%
\pgfpathlineto{\pgfqpoint{1.183615in}{1.349408in}}%
\pgfpathlineto{\pgfqpoint{1.182651in}{1.362335in}}%
\pgfpathlineto{\pgfqpoint{1.182701in}{1.375261in}}%
\pgfpathlineto{\pgfqpoint{1.183767in}{1.388188in}}%
\pgfpathlineto{\pgfqpoint{1.189046in}{1.401114in}}%
\pgfpathlineto{\pgfqpoint{1.193777in}{1.404082in}}%
\pgfpathlineto{\pgfqpoint{1.207045in}{1.405947in}}%
\pgfpathlineto{\pgfqpoint{1.220314in}{1.405067in}}%
\pgfpathlineto{\pgfqpoint{1.228851in}{1.401114in}}%
\pgfpathlineto{\pgfqpoint{1.233583in}{1.393115in}}%
\pgfpathlineto{\pgfqpoint{1.234680in}{1.388188in}}%
\pgfpathlineto{\pgfqpoint{1.235687in}{1.375261in}}%
\pgfpathlineto{\pgfqpoint{1.235731in}{1.362335in}}%
\pgfpathlineto{\pgfqpoint{1.234813in}{1.349408in}}%
\pgfpathlineto{\pgfqpoint{1.233583in}{1.343977in}}%
\pgfpathlineto{\pgfqpoint{1.228429in}{1.336482in}}%
\pgfpathlineto{\pgfqpoint{1.220314in}{1.333330in}}%
\pgfpathlineto{\pgfqpoint{1.207045in}{1.332576in}}%
\pgfpathlineto{\pgfqpoint{1.193777in}{1.334163in}}%
\pgfpathclose%
\pgfpathmoveto{\pgfqpoint{1.250569in}{1.336482in}}%
\pgfpathlineto{\pgfqpoint{1.246852in}{1.339229in}}%
\pgfpathlineto{\pgfqpoint{1.242920in}{1.349408in}}%
\pgfpathlineto{\pgfqpoint{1.241724in}{1.362335in}}%
\pgfpathlineto{\pgfqpoint{1.241582in}{1.375261in}}%
\pgfpathlineto{\pgfqpoint{1.242313in}{1.388188in}}%
\pgfpathlineto{\pgfqpoint{1.246281in}{1.401114in}}%
\pgfpathlineto{\pgfqpoint{1.246852in}{1.401771in}}%
\pgfpathlineto{\pgfqpoint{1.260121in}{1.406380in}}%
\pgfpathlineto{\pgfqpoint{1.273390in}{1.406638in}}%
\pgfpathlineto{\pgfqpoint{1.286659in}{1.404034in}}%
\pgfpathlineto{\pgfqpoint{1.290099in}{1.401114in}}%
\pgfpathlineto{\pgfqpoint{1.293965in}{1.388188in}}%
\pgfpathlineto{\pgfqpoint{1.294666in}{1.375261in}}%
\pgfpathlineto{\pgfqpoint{1.294523in}{1.362335in}}%
\pgfpathlineto{\pgfqpoint{1.293353in}{1.349408in}}%
\pgfpathlineto{\pgfqpoint{1.286839in}{1.336482in}}%
\pgfpathlineto{\pgfqpoint{1.286659in}{1.336361in}}%
\pgfpathlineto{\pgfqpoint{1.273390in}{1.333319in}}%
\pgfpathlineto{\pgfqpoint{1.260121in}{1.333611in}}%
\pgfpathclose%
\pgfpathmoveto{\pgfqpoint{1.308683in}{1.336482in}}%
\pgfpathlineto{\pgfqpoint{1.301898in}{1.349408in}}%
\pgfpathlineto{\pgfqpoint{1.300769in}{1.362335in}}%
\pgfpathlineto{\pgfqpoint{1.300814in}{1.375261in}}%
\pgfpathlineto{\pgfqpoint{1.302029in}{1.388188in}}%
\pgfpathlineto{\pgfqpoint{1.308103in}{1.401114in}}%
\pgfpathlineto{\pgfqpoint{1.313197in}{1.403929in}}%
\pgfpathlineto{\pgfqpoint{1.326466in}{1.405457in}}%
\pgfpathlineto{\pgfqpoint{1.339735in}{1.404026in}}%
\pgfpathlineto{\pgfqpoint{1.345301in}{1.401114in}}%
\pgfpathlineto{\pgfqpoint{1.351821in}{1.388188in}}%
\pgfpathlineto{\pgfqpoint{1.353004in}{1.377015in}}%
\pgfpathlineto{\pgfqpoint{1.353129in}{1.375261in}}%
\pgfpathlineto{\pgfqpoint{1.353165in}{1.362335in}}%
\pgfpathlineto{\pgfqpoint{1.353004in}{1.359888in}}%
\pgfpathlineto{\pgfqpoint{1.351931in}{1.349408in}}%
\pgfpathlineto{\pgfqpoint{1.344596in}{1.336482in}}%
\pgfpathlineto{\pgfqpoint{1.339735in}{1.334341in}}%
\pgfpathlineto{\pgfqpoint{1.326466in}{1.333077in}}%
\pgfpathlineto{\pgfqpoint{1.313197in}{1.334393in}}%
\pgfpathclose%
\pgfpathmoveto{\pgfqpoint{1.365768in}{1.336482in}}%
\pgfpathlineto{\pgfqpoint{1.360223in}{1.349408in}}%
\pgfpathlineto{\pgfqpoint{1.359198in}{1.362335in}}%
\pgfpathlineto{\pgfqpoint{1.359063in}{1.375261in}}%
\pgfpathlineto{\pgfqpoint{1.359653in}{1.388188in}}%
\pgfpathlineto{\pgfqpoint{1.362873in}{1.401114in}}%
\pgfpathlineto{\pgfqpoint{1.366273in}{1.404445in}}%
\pgfpathlineto{\pgfqpoint{1.379542in}{1.407316in}}%
\pgfpathlineto{\pgfqpoint{1.392811in}{1.407422in}}%
\pgfpathlineto{\pgfqpoint{1.406080in}{1.404991in}}%
\pgfpathlineto{\pgfqpoint{1.410038in}{1.401114in}}%
\pgfpathlineto{\pgfqpoint{1.412821in}{1.388188in}}%
\pgfpathlineto{\pgfqpoint{1.413311in}{1.375261in}}%
\pgfpathlineto{\pgfqpoint{1.413181in}{1.362335in}}%
\pgfpathlineto{\pgfqpoint{1.412271in}{1.349408in}}%
\pgfpathlineto{\pgfqpoint{1.407213in}{1.336482in}}%
\pgfpathlineto{\pgfqpoint{1.406080in}{1.335613in}}%
\pgfpathlineto{\pgfqpoint{1.392811in}{1.332575in}}%
\pgfpathlineto{\pgfqpoint{1.379542in}{1.332673in}}%
\pgfpathlineto{\pgfqpoint{1.366273in}{1.336091in}}%
\pgfpathclose%
\pgfpathmoveto{\pgfqpoint{1.429713in}{1.336482in}}%
\pgfpathlineto{\pgfqpoint{1.420754in}{1.349408in}}%
\pgfpathlineto{\pgfqpoint{1.419348in}{1.361380in}}%
\pgfpathlineto{\pgfqpoint{1.419278in}{1.362335in}}%
\pgfpathlineto{\pgfqpoint{1.419307in}{1.375261in}}%
\pgfpathlineto{\pgfqpoint{1.419348in}{1.375787in}}%
\pgfpathlineto{\pgfqpoint{1.420823in}{1.388188in}}%
\pgfpathlineto{\pgfqpoint{1.428553in}{1.401114in}}%
\pgfpathlineto{\pgfqpoint{1.432617in}{1.403097in}}%
\pgfpathlineto{\pgfqpoint{1.445886in}{1.404413in}}%
\pgfpathlineto{\pgfqpoint{1.459155in}{1.402080in}}%
\pgfpathlineto{\pgfqpoint{1.460789in}{1.401114in}}%
\pgfpathlineto{\pgfqpoint{1.468369in}{1.388188in}}%
\pgfpathlineto{\pgfqpoint{1.469910in}{1.375261in}}%
\pgfpathlineto{\pgfqpoint{1.469934in}{1.362335in}}%
\pgfpathlineto{\pgfqpoint{1.468398in}{1.349408in}}%
\pgfpathlineto{\pgfqpoint{1.459586in}{1.336482in}}%
\pgfpathlineto{\pgfqpoint{1.459155in}{1.336266in}}%
\pgfpathlineto{\pgfqpoint{1.445886in}{1.334136in}}%
\pgfpathlineto{\pgfqpoint{1.432617in}{1.335301in}}%
\pgfpathclose%
\pgfpathmoveto{\pgfqpoint{0.834969in}{1.414041in}}%
\pgfpathlineto{\pgfqpoint{0.827908in}{1.426967in}}%
\pgfpathlineto{\pgfqpoint{0.827135in}{1.439894in}}%
\pgfpathlineto{\pgfqpoint{0.826917in}{1.452820in}}%
\pgfpathlineto{\pgfqpoint{0.826937in}{1.465747in}}%
\pgfpathlineto{\pgfqpoint{0.827370in}{1.478673in}}%
\pgfpathlineto{\pgfqpoint{0.835515in}{1.489776in}}%
\pgfpathlineto{\pgfqpoint{0.848784in}{1.490429in}}%
\pgfpathlineto{\pgfqpoint{0.862053in}{1.490266in}}%
\pgfpathlineto{\pgfqpoint{0.875322in}{1.488823in}}%
\pgfpathlineto{\pgfqpoint{0.882491in}{1.478673in}}%
\pgfpathlineto{\pgfqpoint{0.883321in}{1.465747in}}%
\pgfpathlineto{\pgfqpoint{0.883427in}{1.452820in}}%
\pgfpathlineto{\pgfqpoint{0.883170in}{1.439894in}}%
\pgfpathlineto{\pgfqpoint{0.882165in}{1.426967in}}%
\pgfpathlineto{\pgfqpoint{0.875322in}{1.414778in}}%
\pgfpathlineto{\pgfqpoint{0.872893in}{1.414041in}}%
\pgfpathlineto{\pgfqpoint{0.862053in}{1.412432in}}%
\pgfpathlineto{\pgfqpoint{0.848784in}{1.412277in}}%
\pgfpathlineto{\pgfqpoint{0.835515in}{1.413838in}}%
\pgfpathclose%
\pgfpathmoveto{\pgfqpoint{0.961922in}{1.414041in}}%
\pgfpathlineto{\pgfqpoint{0.954936in}{1.415499in}}%
\pgfpathlineto{\pgfqpoint{0.947310in}{1.426967in}}%
\pgfpathlineto{\pgfqpoint{0.946031in}{1.439894in}}%
\pgfpathlineto{\pgfqpoint{0.945727in}{1.452820in}}%
\pgfpathlineto{\pgfqpoint{0.945915in}{1.465747in}}%
\pgfpathlineto{\pgfqpoint{0.947163in}{1.478673in}}%
\pgfpathlineto{\pgfqpoint{0.954936in}{1.487846in}}%
\pgfpathlineto{\pgfqpoint{0.968205in}{1.489293in}}%
\pgfpathlineto{\pgfqpoint{0.981473in}{1.489056in}}%
\pgfpathlineto{\pgfqpoint{0.994742in}{1.485902in}}%
\pgfpathlineto{\pgfqpoint{0.999064in}{1.478673in}}%
\pgfpathlineto{\pgfqpoint{1.000495in}{1.465747in}}%
\pgfpathlineto{\pgfqpoint{1.000725in}{1.452820in}}%
\pgfpathlineto{\pgfqpoint{1.000407in}{1.439894in}}%
\pgfpathlineto{\pgfqpoint{0.999034in}{1.426967in}}%
\pgfpathlineto{\pgfqpoint{0.994742in}{1.418008in}}%
\pgfpathlineto{\pgfqpoint{0.984108in}{1.414041in}}%
\pgfpathlineto{\pgfqpoint{0.981473in}{1.413616in}}%
\pgfpathlineto{\pgfqpoint{0.968205in}{1.413329in}}%
\pgfpathclose%
\pgfpathmoveto{\pgfqpoint{1.085458in}{1.414041in}}%
\pgfpathlineto{\pgfqpoint{1.074356in}{1.415952in}}%
\pgfpathlineto{\pgfqpoint{1.066054in}{1.426967in}}%
\pgfpathlineto{\pgfqpoint{1.064475in}{1.439894in}}%
\pgfpathlineto{\pgfqpoint{1.064118in}{1.452820in}}%
\pgfpathlineto{\pgfqpoint{1.064402in}{1.465747in}}%
\pgfpathlineto{\pgfqpoint{1.066123in}{1.478673in}}%
\pgfpathlineto{\pgfqpoint{1.074356in}{1.487215in}}%
\pgfpathlineto{\pgfqpoint{1.087625in}{1.488757in}}%
\pgfpathlineto{\pgfqpoint{1.100894in}{1.488371in}}%
\pgfpathlineto{\pgfqpoint{1.114163in}{1.483299in}}%
\pgfpathlineto{\pgfqpoint{1.116424in}{1.478673in}}%
\pgfpathlineto{\pgfqpoint{1.118126in}{1.465747in}}%
\pgfpathlineto{\pgfqpoint{1.118411in}{1.452820in}}%
\pgfpathlineto{\pgfqpoint{1.118066in}{1.439894in}}%
\pgfpathlineto{\pgfqpoint{1.116526in}{1.426967in}}%
\pgfpathlineto{\pgfqpoint{1.114163in}{1.421012in}}%
\pgfpathlineto{\pgfqpoint{1.100894in}{1.414330in}}%
\pgfpathlineto{\pgfqpoint{1.094017in}{1.414041in}}%
\pgfpathlineto{\pgfqpoint{1.087625in}{1.413820in}}%
\pgfpathclose%
\pgfpathmoveto{\pgfqpoint{1.204280in}{1.414041in}}%
\pgfpathlineto{\pgfqpoint{1.193777in}{1.415533in}}%
\pgfpathlineto{\pgfqpoint{1.184099in}{1.426967in}}%
\pgfpathlineto{\pgfqpoint{1.182429in}{1.439894in}}%
\pgfpathlineto{\pgfqpoint{1.182054in}{1.452820in}}%
\pgfpathlineto{\pgfqpoint{1.182363in}{1.465747in}}%
\pgfpathlineto{\pgfqpoint{1.184209in}{1.478673in}}%
\pgfpathlineto{\pgfqpoint{1.193777in}{1.487492in}}%
\pgfpathlineto{\pgfqpoint{1.207045in}{1.488778in}}%
\pgfpathlineto{\pgfqpoint{1.220314in}{1.488268in}}%
\pgfpathlineto{\pgfqpoint{1.233583in}{1.481230in}}%
\pgfpathlineto{\pgfqpoint{1.234550in}{1.478673in}}%
\pgfpathlineto{\pgfqpoint{1.236190in}{1.465747in}}%
\pgfpathlineto{\pgfqpoint{1.236461in}{1.452820in}}%
\pgfpathlineto{\pgfqpoint{1.236121in}{1.439894in}}%
\pgfpathlineto{\pgfqpoint{1.234616in}{1.426967in}}%
\pgfpathlineto{\pgfqpoint{1.233583in}{1.423674in}}%
\pgfpathlineto{\pgfqpoint{1.220314in}{1.414494in}}%
\pgfpathlineto{\pgfqpoint{1.212595in}{1.414041in}}%
\pgfpathlineto{\pgfqpoint{1.207045in}{1.413789in}}%
\pgfpathclose%
\pgfpathmoveto{\pgfqpoint{1.316711in}{1.414041in}}%
\pgfpathlineto{\pgfqpoint{1.313197in}{1.414451in}}%
\pgfpathlineto{\pgfqpoint{1.301375in}{1.426967in}}%
\pgfpathlineto{\pgfqpoint{1.299928in}{1.438737in}}%
\pgfpathlineto{\pgfqpoint{1.299842in}{1.439894in}}%
\pgfpathlineto{\pgfqpoint{1.299516in}{1.452820in}}%
\pgfpathlineto{\pgfqpoint{1.299752in}{1.465747in}}%
\pgfpathlineto{\pgfqpoint{1.299928in}{1.468246in}}%
\pgfpathlineto{\pgfqpoint{1.301342in}{1.478673in}}%
\pgfpathlineto{\pgfqpoint{1.313197in}{1.488451in}}%
\pgfpathlineto{\pgfqpoint{1.326466in}{1.489334in}}%
\pgfpathlineto{\pgfqpoint{1.339735in}{1.488843in}}%
\pgfpathlineto{\pgfqpoint{1.353004in}{1.480366in}}%
\pgfpathlineto{\pgfqpoint{1.353451in}{1.478673in}}%
\pgfpathlineto{\pgfqpoint{1.354681in}{1.465747in}}%
\pgfpathlineto{\pgfqpoint{1.354865in}{1.452820in}}%
\pgfpathlineto{\pgfqpoint{1.354566in}{1.439894in}}%
\pgfpathlineto{\pgfqpoint{1.353306in}{1.426967in}}%
\pgfpathlineto{\pgfqpoint{1.353004in}{1.425645in}}%
\pgfpathlineto{\pgfqpoint{1.339991in}{1.414041in}}%
\pgfpathlineto{\pgfqpoint{1.339735in}{1.413988in}}%
\pgfpathlineto{\pgfqpoint{1.326466in}{1.413255in}}%
\pgfpathclose%
\pgfpathmoveto{\pgfqpoint{1.427597in}{1.414041in}}%
\pgfpathlineto{\pgfqpoint{1.419348in}{1.421413in}}%
\pgfpathlineto{\pgfqpoint{1.417995in}{1.426967in}}%
\pgfpathlineto{\pgfqpoint{1.416991in}{1.439894in}}%
\pgfpathlineto{\pgfqpoint{1.416734in}{1.452820in}}%
\pgfpathlineto{\pgfqpoint{1.416839in}{1.465747in}}%
\pgfpathlineto{\pgfqpoint{1.417670in}{1.478673in}}%
\pgfpathlineto{\pgfqpoint{1.419348in}{1.484585in}}%
\pgfpathlineto{\pgfqpoint{1.432617in}{1.489961in}}%
\pgfpathlineto{\pgfqpoint{1.445886in}{1.490427in}}%
\pgfpathlineto{\pgfqpoint{1.459155in}{1.490246in}}%
\pgfpathlineto{\pgfqpoint{1.472424in}{1.483782in}}%
\pgfpathlineto{\pgfqpoint{1.473168in}{1.478673in}}%
\pgfpathlineto{\pgfqpoint{1.473608in}{1.465747in}}%
\pgfpathlineto{\pgfqpoint{1.473628in}{1.452820in}}%
\pgfpathlineto{\pgfqpoint{1.473406in}{1.439894in}}%
\pgfpathlineto{\pgfqpoint{1.472620in}{1.426967in}}%
\pgfpathlineto{\pgfqpoint{1.472424in}{1.425576in}}%
\pgfpathlineto{\pgfqpoint{1.464815in}{1.414041in}}%
\pgfpathlineto{\pgfqpoint{1.459155in}{1.412755in}}%
\pgfpathlineto{\pgfqpoint{1.445886in}{1.412216in}}%
\pgfpathlineto{\pgfqpoint{1.432617in}{1.412915in}}%
\pgfpathclose%
\pgfpathmoveto{\pgfqpoint{0.660608in}{1.426967in}}%
\pgfpathlineto{\pgfqpoint{0.656865in}{1.439894in}}%
\pgfpathlineto{\pgfqpoint{0.656324in}{1.452820in}}%
\pgfpathlineto{\pgfqpoint{0.657842in}{1.465747in}}%
\pgfpathlineto{\pgfqpoint{0.663019in}{1.476638in}}%
\pgfpathlineto{\pgfqpoint{0.666344in}{1.478673in}}%
\pgfpathlineto{\pgfqpoint{0.676288in}{1.481859in}}%
\pgfpathlineto{\pgfqpoint{0.689557in}{1.480320in}}%
\pgfpathlineto{\pgfqpoint{0.692227in}{1.478673in}}%
\pgfpathlineto{\pgfqpoint{0.699960in}{1.465747in}}%
\pgfpathlineto{\pgfqpoint{0.701622in}{1.452820in}}%
\pgfpathlineto{\pgfqpoint{0.701052in}{1.439894in}}%
\pgfpathlineto{\pgfqpoint{0.696955in}{1.426967in}}%
\pgfpathlineto{\pgfqpoint{0.689557in}{1.420639in}}%
\pgfpathlineto{\pgfqpoint{0.676288in}{1.419136in}}%
\pgfpathlineto{\pgfqpoint{0.663019in}{1.423908in}}%
\pgfpathclose%
\pgfpathmoveto{\pgfqpoint{0.776185in}{1.426967in}}%
\pgfpathlineto{\pgfqpoint{0.773134in}{1.439894in}}%
\pgfpathlineto{\pgfqpoint{0.772715in}{1.452820in}}%
\pgfpathlineto{\pgfqpoint{0.774007in}{1.465747in}}%
\pgfpathlineto{\pgfqpoint{0.779801in}{1.478673in}}%
\pgfpathlineto{\pgfqpoint{0.782439in}{1.480821in}}%
\pgfpathlineto{\pgfqpoint{0.795708in}{1.483785in}}%
\pgfpathlineto{\pgfqpoint{0.808977in}{1.481950in}}%
\pgfpathlineto{\pgfqpoint{0.813686in}{1.478673in}}%
\pgfpathlineto{\pgfqpoint{0.819550in}{1.465747in}}%
\pgfpathlineto{\pgfqpoint{0.820832in}{1.452820in}}%
\pgfpathlineto{\pgfqpoint{0.820436in}{1.439894in}}%
\pgfpathlineto{\pgfqpoint{0.817433in}{1.426967in}}%
\pgfpathlineto{\pgfqpoint{0.808977in}{1.418804in}}%
\pgfpathlineto{\pgfqpoint{0.795708in}{1.417061in}}%
\pgfpathlineto{\pgfqpoint{0.782439in}{1.419958in}}%
\pgfpathclose%
\pgfpathmoveto{\pgfqpoint{0.892199in}{1.426967in}}%
\pgfpathlineto{\pgfqpoint{0.889648in}{1.439894in}}%
\pgfpathlineto{\pgfqpoint{0.889320in}{1.452820in}}%
\pgfpathlineto{\pgfqpoint{0.890454in}{1.465747in}}%
\pgfpathlineto{\pgfqpoint{0.895495in}{1.478673in}}%
\pgfpathlineto{\pgfqpoint{0.901860in}{1.483271in}}%
\pgfpathlineto{\pgfqpoint{0.915129in}{1.485113in}}%
\pgfpathlineto{\pgfqpoint{0.928398in}{1.483004in}}%
\pgfpathlineto{\pgfqpoint{0.933910in}{1.478673in}}%
\pgfpathlineto{\pgfqpoint{0.938555in}{1.465747in}}%
\pgfpathlineto{\pgfqpoint{0.939588in}{1.452820in}}%
\pgfpathlineto{\pgfqpoint{0.939304in}{1.439894in}}%
\pgfpathlineto{\pgfqpoint{0.937010in}{1.426967in}}%
\pgfpathlineto{\pgfqpoint{0.928398in}{1.417573in}}%
\pgfpathlineto{\pgfqpoint{0.915129in}{1.415626in}}%
\pgfpathlineto{\pgfqpoint{0.901860in}{1.417368in}}%
\pgfpathclose%
\pgfpathmoveto{\pgfqpoint{1.008662in}{1.426967in}}%
\pgfpathlineto{\pgfqpoint{1.008011in}{1.429693in}}%
\pgfpathlineto{\pgfqpoint{1.006705in}{1.439894in}}%
\pgfpathlineto{\pgfqpoint{1.006484in}{1.452820in}}%
\pgfpathlineto{\pgfqpoint{1.007335in}{1.465747in}}%
\pgfpathlineto{\pgfqpoint{1.008011in}{1.469343in}}%
\pgfpathlineto{\pgfqpoint{1.011820in}{1.478673in}}%
\pgfpathlineto{\pgfqpoint{1.021280in}{1.484742in}}%
\pgfpathlineto{\pgfqpoint{1.034549in}{1.485866in}}%
\pgfpathlineto{\pgfqpoint{1.047818in}{1.483377in}}%
\pgfpathlineto{\pgfqpoint{1.053105in}{1.478673in}}%
\pgfpathlineto{\pgfqpoint{1.057060in}{1.465747in}}%
\pgfpathlineto{\pgfqpoint{1.057951in}{1.452820in}}%
\pgfpathlineto{\pgfqpoint{1.057728in}{1.439894in}}%
\pgfpathlineto{\pgfqpoint{1.055830in}{1.426967in}}%
\pgfpathlineto{\pgfqpoint{1.047818in}{1.417060in}}%
\pgfpathlineto{\pgfqpoint{1.034549in}{1.414805in}}%
\pgfpathlineto{\pgfqpoint{1.021280in}{1.415836in}}%
\pgfpathclose%
\pgfpathmoveto{\pgfqpoint{1.126054in}{1.426967in}}%
\pgfpathlineto{\pgfqpoint{1.124368in}{1.439894in}}%
\pgfpathlineto{\pgfqpoint{1.124172in}{1.452820in}}%
\pgfpathlineto{\pgfqpoint{1.124973in}{1.465747in}}%
\pgfpathlineto{\pgfqpoint{1.127432in}{1.476108in}}%
\pgfpathlineto{\pgfqpoint{1.128851in}{1.478673in}}%
\pgfpathlineto{\pgfqpoint{1.140701in}{1.485412in}}%
\pgfpathlineto{\pgfqpoint{1.153970in}{1.486050in}}%
\pgfpathlineto{\pgfqpoint{1.167239in}{1.482897in}}%
\pgfpathlineto{\pgfqpoint{1.171408in}{1.478673in}}%
\pgfpathlineto{\pgfqpoint{1.175116in}{1.465747in}}%
\pgfpathlineto{\pgfqpoint{1.175957in}{1.452820in}}%
\pgfpathlineto{\pgfqpoint{1.175752in}{1.439894in}}%
\pgfpathlineto{\pgfqpoint{1.173982in}{1.426967in}}%
\pgfpathlineto{\pgfqpoint{1.167239in}{1.417450in}}%
\pgfpathlineto{\pgfqpoint{1.153970in}{1.414594in}}%
\pgfpathlineto{\pgfqpoint{1.140701in}{1.415164in}}%
\pgfpathlineto{\pgfqpoint{1.127432in}{1.423338in}}%
\pgfpathclose%
\pgfpathmoveto{\pgfqpoint{1.244232in}{1.426967in}}%
\pgfpathlineto{\pgfqpoint{1.242469in}{1.439894in}}%
\pgfpathlineto{\pgfqpoint{1.242262in}{1.452820in}}%
\pgfpathlineto{\pgfqpoint{1.243090in}{1.465747in}}%
\pgfpathlineto{\pgfqpoint{1.246760in}{1.478673in}}%
\pgfpathlineto{\pgfqpoint{1.246852in}{1.478813in}}%
\pgfpathlineto{\pgfqpoint{1.260121in}{1.485400in}}%
\pgfpathlineto{\pgfqpoint{1.273390in}{1.485645in}}%
\pgfpathlineto{\pgfqpoint{1.286659in}{1.481280in}}%
\pgfpathlineto{\pgfqpoint{1.288899in}{1.478673in}}%
\pgfpathlineto{\pgfqpoint{1.292750in}{1.465747in}}%
\pgfpathlineto{\pgfqpoint{1.293622in}{1.452820in}}%
\pgfpathlineto{\pgfqpoint{1.293395in}{1.439894in}}%
\pgfpathlineto{\pgfqpoint{1.291519in}{1.426967in}}%
\pgfpathlineto{\pgfqpoint{1.286659in}{1.419057in}}%
\pgfpathlineto{\pgfqpoint{1.273390in}{1.415011in}}%
\pgfpathlineto{\pgfqpoint{1.260121in}{1.415221in}}%
\pgfpathlineto{\pgfqpoint{1.246852in}{1.421355in}}%
\pgfpathclose%
\pgfpathmoveto{\pgfqpoint{1.363074in}{1.426967in}}%
\pgfpathlineto{\pgfqpoint{1.361002in}{1.439894in}}%
\pgfpathlineto{\pgfqpoint{1.360745in}{1.452820in}}%
\pgfpathlineto{\pgfqpoint{1.361679in}{1.465747in}}%
\pgfpathlineto{\pgfqpoint{1.365873in}{1.478673in}}%
\pgfpathlineto{\pgfqpoint{1.366273in}{1.479182in}}%
\pgfpathlineto{\pgfqpoint{1.379542in}{1.484781in}}%
\pgfpathlineto{\pgfqpoint{1.392811in}{1.484609in}}%
\pgfpathlineto{\pgfqpoint{1.405390in}{1.478673in}}%
\pgfpathlineto{\pgfqpoint{1.406080in}{1.477856in}}%
\pgfpathlineto{\pgfqpoint{1.409969in}{1.465747in}}%
\pgfpathlineto{\pgfqpoint{1.410947in}{1.452820in}}%
\pgfpathlineto{\pgfqpoint{1.410664in}{1.439894in}}%
\pgfpathlineto{\pgfqpoint{1.408464in}{1.426967in}}%
\pgfpathlineto{\pgfqpoint{1.406080in}{1.422422in}}%
\pgfpathlineto{\pgfqpoint{1.392811in}{1.416105in}}%
\pgfpathlineto{\pgfqpoint{1.379542in}{1.415926in}}%
\pgfpathlineto{\pgfqpoint{1.366273in}{1.421218in}}%
\pgfpathclose%
\pgfusepath{fill}%
\end{pgfscope}%
\begin{pgfscope}%
\pgfpathrectangle{\pgfqpoint{0.211875in}{0.211875in}}{\pgfqpoint{1.313625in}{1.279725in}}%
\pgfusepath{clip}%
\pgfsetbuttcap%
\pgfsetroundjoin%
\definecolor{currentfill}{rgb}{0.947270,0.405591,0.279023}%
\pgfsetfillcolor{currentfill}%
\pgfsetlinewidth{0.000000pt}%
\definecolor{currentstroke}{rgb}{0.000000,0.000000,0.000000}%
\pgfsetstrokecolor{currentstroke}%
\pgfsetdash{}{0pt}%
\pgfpathmoveto{\pgfqpoint{0.318027in}{0.612265in}}%
\pgfpathlineto{\pgfqpoint{0.331295in}{0.611743in}}%
\pgfpathlineto{\pgfqpoint{0.333075in}{0.612597in}}%
\pgfpathlineto{\pgfqpoint{0.344024in}{0.625523in}}%
\pgfpathlineto{\pgfqpoint{0.344564in}{0.628275in}}%
\pgfpathlineto{\pgfqpoint{0.345685in}{0.638450in}}%
\pgfpathlineto{\pgfqpoint{0.344717in}{0.651377in}}%
\pgfpathlineto{\pgfqpoint{0.344564in}{0.651932in}}%
\pgfpathlineto{\pgfqpoint{0.335840in}{0.664303in}}%
\pgfpathlineto{\pgfqpoint{0.331295in}{0.666641in}}%
\pgfpathlineto{\pgfqpoint{0.318027in}{0.666100in}}%
\pgfpathlineto{\pgfqpoint{0.315021in}{0.664303in}}%
\pgfpathlineto{\pgfqpoint{0.306406in}{0.651377in}}%
\pgfpathlineto{\pgfqpoint{0.304856in}{0.638450in}}%
\pgfpathlineto{\pgfqpoint{0.307122in}{0.625523in}}%
\pgfpathlineto{\pgfqpoint{0.317434in}{0.612597in}}%
\pgfpathclose%
\pgfpathmoveto{\pgfqpoint{0.316770in}{0.625523in}}%
\pgfpathlineto{\pgfqpoint{0.313333in}{0.638450in}}%
\pgfpathlineto{\pgfqpoint{0.315986in}{0.651377in}}%
\pgfpathlineto{\pgfqpoint{0.318027in}{0.654091in}}%
\pgfpathlineto{\pgfqpoint{0.331295in}{0.654928in}}%
\pgfpathlineto{\pgfqpoint{0.334252in}{0.651377in}}%
\pgfpathlineto{\pgfqpoint{0.337057in}{0.638450in}}%
\pgfpathlineto{\pgfqpoint{0.333394in}{0.625523in}}%
\pgfpathlineto{\pgfqpoint{0.331295in}{0.623198in}}%
\pgfpathlineto{\pgfqpoint{0.318027in}{0.623981in}}%
\pgfpathclose%
\pgfusepath{fill}%
\end{pgfscope}%
\begin{pgfscope}%
\pgfpathrectangle{\pgfqpoint{0.211875in}{0.211875in}}{\pgfqpoint{1.313625in}{1.279725in}}%
\pgfusepath{clip}%
\pgfsetbuttcap%
\pgfsetroundjoin%
\definecolor{currentfill}{rgb}{0.947270,0.405591,0.279023}%
\pgfsetfillcolor{currentfill}%
\pgfsetlinewidth{0.000000pt}%
\definecolor{currentstroke}{rgb}{0.000000,0.000000,0.000000}%
\pgfsetstrokecolor{currentstroke}%
\pgfsetdash{}{0pt}%
\pgfpathmoveto{\pgfqpoint{0.437447in}{0.608232in}}%
\pgfpathlineto{\pgfqpoint{0.450716in}{0.608398in}}%
\pgfpathlineto{\pgfqpoint{0.458357in}{0.612597in}}%
\pgfpathlineto{\pgfqpoint{0.463985in}{0.621423in}}%
\pgfpathlineto{\pgfqpoint{0.465080in}{0.625523in}}%
\pgfpathlineto{\pgfqpoint{0.466169in}{0.638450in}}%
\pgfpathlineto{\pgfqpoint{0.465553in}{0.651377in}}%
\pgfpathlineto{\pgfqpoint{0.463985in}{0.658220in}}%
\pgfpathlineto{\pgfqpoint{0.460705in}{0.664303in}}%
\pgfpathlineto{\pgfqpoint{0.450716in}{0.670191in}}%
\pgfpathlineto{\pgfqpoint{0.437447in}{0.670293in}}%
\pgfpathlineto{\pgfqpoint{0.426023in}{0.664303in}}%
\pgfpathlineto{\pgfqpoint{0.424178in}{0.661867in}}%
\pgfpathlineto{\pgfqpoint{0.421072in}{0.651377in}}%
\pgfpathlineto{\pgfqpoint{0.420348in}{0.638450in}}%
\pgfpathlineto{\pgfqpoint{0.421541in}{0.625523in}}%
\pgfpathlineto{\pgfqpoint{0.424178in}{0.617647in}}%
\pgfpathlineto{\pgfqpoint{0.428557in}{0.612597in}}%
\pgfpathclose%
\pgfpathmoveto{\pgfqpoint{0.430596in}{0.625523in}}%
\pgfpathlineto{\pgfqpoint{0.427352in}{0.638450in}}%
\pgfpathlineto{\pgfqpoint{0.429760in}{0.651377in}}%
\pgfpathlineto{\pgfqpoint{0.437447in}{0.660355in}}%
\pgfpathlineto{\pgfqpoint{0.450716in}{0.659509in}}%
\pgfpathlineto{\pgfqpoint{0.456655in}{0.651377in}}%
\pgfpathlineto{\pgfqpoint{0.458706in}{0.638450in}}%
\pgfpathlineto{\pgfqpoint{0.455907in}{0.625523in}}%
\pgfpathlineto{\pgfqpoint{0.450716in}{0.618981in}}%
\pgfpathlineto{\pgfqpoint{0.437447in}{0.618131in}}%
\pgfpathclose%
\pgfusepath{fill}%
\end{pgfscope}%
\begin{pgfscope}%
\pgfpathrectangle{\pgfqpoint{0.211875in}{0.211875in}}{\pgfqpoint{1.313625in}{1.279725in}}%
\pgfusepath{clip}%
\pgfsetbuttcap%
\pgfsetroundjoin%
\definecolor{currentfill}{rgb}{0.947270,0.405591,0.279023}%
\pgfsetfillcolor{currentfill}%
\pgfsetlinewidth{0.000000pt}%
\definecolor{currentstroke}{rgb}{0.000000,0.000000,0.000000}%
\pgfsetstrokecolor{currentstroke}%
\pgfsetdash{}{0pt}%
\pgfpathmoveto{\pgfqpoint{0.543598in}{0.609536in}}%
\pgfpathlineto{\pgfqpoint{0.556867in}{0.604997in}}%
\pgfpathlineto{\pgfqpoint{0.570136in}{0.605495in}}%
\pgfpathlineto{\pgfqpoint{0.581520in}{0.612597in}}%
\pgfpathlineto{\pgfqpoint{0.583405in}{0.616456in}}%
\pgfpathlineto{\pgfqpoint{0.585415in}{0.625523in}}%
\pgfpathlineto{\pgfqpoint{0.586198in}{0.638450in}}%
\pgfpathlineto{\pgfqpoint{0.585867in}{0.651377in}}%
\pgfpathlineto{\pgfqpoint{0.583523in}{0.664303in}}%
\pgfpathlineto{\pgfqpoint{0.583405in}{0.664575in}}%
\pgfpathlineto{\pgfqpoint{0.570136in}{0.673302in}}%
\pgfpathlineto{\pgfqpoint{0.556867in}{0.673652in}}%
\pgfpathlineto{\pgfqpoint{0.543598in}{0.669645in}}%
\pgfpathlineto{\pgfqpoint{0.539470in}{0.664303in}}%
\pgfpathlineto{\pgfqpoint{0.536593in}{0.651377in}}%
\pgfpathlineto{\pgfqpoint{0.536145in}{0.638450in}}%
\pgfpathlineto{\pgfqpoint{0.537056in}{0.625523in}}%
\pgfpathlineto{\pgfqpoint{0.541011in}{0.612597in}}%
\pgfpathclose%
\pgfpathmoveto{\pgfqpoint{0.544001in}{0.625523in}}%
\pgfpathlineto{\pgfqpoint{0.543598in}{0.627013in}}%
\pgfpathlineto{\pgfqpoint{0.542078in}{0.638450in}}%
\pgfpathlineto{\pgfqpoint{0.543313in}{0.651377in}}%
\pgfpathlineto{\pgfqpoint{0.543598in}{0.652213in}}%
\pgfpathlineto{\pgfqpoint{0.555245in}{0.664303in}}%
\pgfpathlineto{\pgfqpoint{0.556867in}{0.665040in}}%
\pgfpathlineto{\pgfqpoint{0.564169in}{0.664303in}}%
\pgfpathlineto{\pgfqpoint{0.570136in}{0.663347in}}%
\pgfpathlineto{\pgfqpoint{0.577857in}{0.651377in}}%
\pgfpathlineto{\pgfqpoint{0.579367in}{0.638450in}}%
\pgfpathlineto{\pgfqpoint{0.577191in}{0.625523in}}%
\pgfpathlineto{\pgfqpoint{0.570136in}{0.615481in}}%
\pgfpathlineto{\pgfqpoint{0.556867in}{0.613463in}}%
\pgfpathclose%
\pgfusepath{fill}%
\end{pgfscope}%
\begin{pgfscope}%
\pgfpathrectangle{\pgfqpoint{0.211875in}{0.211875in}}{\pgfqpoint{1.313625in}{1.279725in}}%
\pgfusepath{clip}%
\pgfsetbuttcap%
\pgfsetroundjoin%
\definecolor{currentfill}{rgb}{0.947270,0.405591,0.279023}%
\pgfsetfillcolor{currentfill}%
\pgfsetlinewidth{0.000000pt}%
\definecolor{currentstroke}{rgb}{0.000000,0.000000,0.000000}%
\pgfsetstrokecolor{currentstroke}%
\pgfsetdash{}{0pt}%
\pgfpathmoveto{\pgfqpoint{0.663019in}{0.604869in}}%
\pgfpathlineto{\pgfqpoint{0.676288in}{0.602457in}}%
\pgfpathlineto{\pgfqpoint{0.689557in}{0.603028in}}%
\pgfpathlineto{\pgfqpoint{0.702826in}{0.611967in}}%
\pgfpathlineto{\pgfqpoint{0.703074in}{0.612597in}}%
\pgfpathlineto{\pgfqpoint{0.705261in}{0.625523in}}%
\pgfpathlineto{\pgfqpoint{0.705804in}{0.638450in}}%
\pgfpathlineto{\pgfqpoint{0.705698in}{0.651377in}}%
\pgfpathlineto{\pgfqpoint{0.704536in}{0.664303in}}%
\pgfpathlineto{\pgfqpoint{0.702826in}{0.669461in}}%
\pgfpathlineto{\pgfqpoint{0.689557in}{0.675982in}}%
\pgfpathlineto{\pgfqpoint{0.676288in}{0.676290in}}%
\pgfpathlineto{\pgfqpoint{0.663019in}{0.674436in}}%
\pgfpathlineto{\pgfqpoint{0.654079in}{0.664303in}}%
\pgfpathlineto{\pgfqpoint{0.652384in}{0.651377in}}%
\pgfpathlineto{\pgfqpoint{0.652174in}{0.638450in}}%
\pgfpathlineto{\pgfqpoint{0.652847in}{0.625523in}}%
\pgfpathlineto{\pgfqpoint{0.655605in}{0.612597in}}%
\pgfpathclose%
\pgfpathmoveto{\pgfqpoint{0.670739in}{0.612597in}}%
\pgfpathlineto{\pgfqpoint{0.663019in}{0.618175in}}%
\pgfpathlineto{\pgfqpoint{0.659841in}{0.625523in}}%
\pgfpathlineto{\pgfqpoint{0.658302in}{0.638450in}}%
\pgfpathlineto{\pgfqpoint{0.659327in}{0.651377in}}%
\pgfpathlineto{\pgfqpoint{0.663019in}{0.660847in}}%
\pgfpathlineto{\pgfqpoint{0.667278in}{0.664303in}}%
\pgfpathlineto{\pgfqpoint{0.676288in}{0.667792in}}%
\pgfpathlineto{\pgfqpoint{0.689557in}{0.665809in}}%
\pgfpathlineto{\pgfqpoint{0.691541in}{0.664303in}}%
\pgfpathlineto{\pgfqpoint{0.698163in}{0.651377in}}%
\pgfpathlineto{\pgfqpoint{0.699287in}{0.638450in}}%
\pgfpathlineto{\pgfqpoint{0.697560in}{0.625523in}}%
\pgfpathlineto{\pgfqpoint{0.689557in}{0.612712in}}%
\pgfpathlineto{\pgfqpoint{0.689110in}{0.612597in}}%
\pgfpathlineto{\pgfqpoint{0.676288in}{0.610583in}}%
\pgfpathclose%
\pgfusepath{fill}%
\end{pgfscope}%
\begin{pgfscope}%
\pgfpathrectangle{\pgfqpoint{0.211875in}{0.211875in}}{\pgfqpoint{1.313625in}{1.279725in}}%
\pgfusepath{clip}%
\pgfsetbuttcap%
\pgfsetroundjoin%
\definecolor{currentfill}{rgb}{0.947270,0.405591,0.279023}%
\pgfsetfillcolor{currentfill}%
\pgfsetlinewidth{0.000000pt}%
\definecolor{currentstroke}{rgb}{0.000000,0.000000,0.000000}%
\pgfsetstrokecolor{currentstroke}%
\pgfsetdash{}{0pt}%
\pgfpathmoveto{\pgfqpoint{0.782439in}{0.601709in}}%
\pgfpathlineto{\pgfqpoint{0.795708in}{0.600538in}}%
\pgfpathlineto{\pgfqpoint{0.808977in}{0.601013in}}%
\pgfpathlineto{\pgfqpoint{0.822246in}{0.608699in}}%
\pgfpathlineto{\pgfqpoint{0.823377in}{0.612597in}}%
\pgfpathlineto{\pgfqpoint{0.824636in}{0.625523in}}%
\pgfpathlineto{\pgfqpoint{0.824999in}{0.638450in}}%
\pgfpathlineto{\pgfqpoint{0.825066in}{0.651377in}}%
\pgfpathlineto{\pgfqpoint{0.824795in}{0.664303in}}%
\pgfpathlineto{\pgfqpoint{0.822246in}{0.675366in}}%
\pgfpathlineto{\pgfqpoint{0.818245in}{0.677230in}}%
\pgfpathlineto{\pgfqpoint{0.808977in}{0.678180in}}%
\pgfpathlineto{\pgfqpoint{0.795708in}{0.678244in}}%
\pgfpathlineto{\pgfqpoint{0.782439in}{0.677608in}}%
\pgfpathlineto{\pgfqpoint{0.779967in}{0.677230in}}%
\pgfpathlineto{\pgfqpoint{0.769170in}{0.664703in}}%
\pgfpathlineto{\pgfqpoint{0.769120in}{0.664303in}}%
\pgfpathlineto{\pgfqpoint{0.768473in}{0.651377in}}%
\pgfpathlineto{\pgfqpoint{0.768464in}{0.638450in}}%
\pgfpathlineto{\pgfqpoint{0.768904in}{0.625523in}}%
\pgfpathlineto{\pgfqpoint{0.769170in}{0.622402in}}%
\pgfpathlineto{\pgfqpoint{0.770650in}{0.612597in}}%
\pgfpathclose%
\pgfpathmoveto{\pgfqpoint{0.782488in}{0.612597in}}%
\pgfpathlineto{\pgfqpoint{0.782439in}{0.612624in}}%
\pgfpathlineto{\pgfqpoint{0.776162in}{0.625523in}}%
\pgfpathlineto{\pgfqpoint{0.774778in}{0.638450in}}%
\pgfpathlineto{\pgfqpoint{0.775640in}{0.651377in}}%
\pgfpathlineto{\pgfqpoint{0.780690in}{0.664303in}}%
\pgfpathlineto{\pgfqpoint{0.782439in}{0.666052in}}%
\pgfpathlineto{\pgfqpoint{0.795708in}{0.669843in}}%
\pgfpathlineto{\pgfqpoint{0.808977in}{0.667420in}}%
\pgfpathlineto{\pgfqpoint{0.812613in}{0.664303in}}%
\pgfpathlineto{\pgfqpoint{0.817770in}{0.651377in}}%
\pgfpathlineto{\pgfqpoint{0.818626in}{0.638450in}}%
\pgfpathlineto{\pgfqpoint{0.817215in}{0.625523in}}%
\pgfpathlineto{\pgfqpoint{0.810698in}{0.612597in}}%
\pgfpathlineto{\pgfqpoint{0.808977in}{0.611230in}}%
\pgfpathlineto{\pgfqpoint{0.795708in}{0.608613in}}%
\pgfpathclose%
\pgfusepath{fill}%
\end{pgfscope}%
\begin{pgfscope}%
\pgfpathrectangle{\pgfqpoint{0.211875in}{0.211875in}}{\pgfqpoint{1.313625in}{1.279725in}}%
\pgfusepath{clip}%
\pgfsetbuttcap%
\pgfsetroundjoin%
\definecolor{currentfill}{rgb}{0.947270,0.405591,0.279023}%
\pgfsetfillcolor{currentfill}%
\pgfsetlinewidth{0.000000pt}%
\definecolor{currentstroke}{rgb}{0.000000,0.000000,0.000000}%
\pgfsetstrokecolor{currentstroke}%
\pgfsetdash{}{0pt}%
\pgfpathmoveto{\pgfqpoint{1.485693in}{0.602054in}}%
\pgfpathlineto{\pgfqpoint{1.498962in}{0.600549in}}%
\pgfpathlineto{\pgfqpoint{1.512231in}{0.600925in}}%
\pgfpathlineto{\pgfqpoint{1.525500in}{0.604851in}}%
\pgfpathlineto{\pgfqpoint{1.525500in}{0.612597in}}%
\pgfpathlineto{\pgfqpoint{1.525500in}{0.625523in}}%
\pgfpathlineto{\pgfqpoint{1.525500in}{0.632009in}}%
\pgfpathlineto{\pgfqpoint{1.524580in}{0.625523in}}%
\pgfpathlineto{\pgfqpoint{1.516765in}{0.612597in}}%
\pgfpathlineto{\pgfqpoint{1.512231in}{0.609838in}}%
\pgfpathlineto{\pgfqpoint{1.498962in}{0.608901in}}%
\pgfpathlineto{\pgfqpoint{1.490486in}{0.612597in}}%
\pgfpathlineto{\pgfqpoint{1.485693in}{0.617412in}}%
\pgfpathlineto{\pgfqpoint{1.482804in}{0.625523in}}%
\pgfpathlineto{\pgfqpoint{1.481565in}{0.638450in}}%
\pgfpathlineto{\pgfqpoint{1.482316in}{0.651377in}}%
\pgfpathlineto{\pgfqpoint{1.485693in}{0.662071in}}%
\pgfpathlineto{\pgfqpoint{1.487635in}{0.664303in}}%
\pgfpathlineto{\pgfqpoint{1.498962in}{0.669586in}}%
\pgfpathlineto{\pgfqpoint{1.512231in}{0.668690in}}%
\pgfpathlineto{\pgfqpoint{1.518961in}{0.664303in}}%
\pgfpathlineto{\pgfqpoint{1.525229in}{0.651377in}}%
\pgfpathlineto{\pgfqpoint{1.525500in}{0.648402in}}%
\pgfpathlineto{\pgfqpoint{1.525500in}{0.651377in}}%
\pgfpathlineto{\pgfqpoint{1.525500in}{0.664303in}}%
\pgfpathlineto{\pgfqpoint{1.525500in}{0.675881in}}%
\pgfpathlineto{\pgfqpoint{1.519925in}{0.677230in}}%
\pgfpathlineto{\pgfqpoint{1.512231in}{0.678016in}}%
\pgfpathlineto{\pgfqpoint{1.498962in}{0.678286in}}%
\pgfpathlineto{\pgfqpoint{1.485693in}{0.677869in}}%
\pgfpathlineto{\pgfqpoint{1.481873in}{0.677230in}}%
\pgfpathlineto{\pgfqpoint{1.475792in}{0.664303in}}%
\pgfpathlineto{\pgfqpoint{1.475514in}{0.651377in}}%
\pgfpathlineto{\pgfqpoint{1.475582in}{0.638450in}}%
\pgfpathlineto{\pgfqpoint{1.475952in}{0.625523in}}%
\pgfpathlineto{\pgfqpoint{1.477236in}{0.612597in}}%
\pgfpathclose%
\pgfusepath{fill}%
\end{pgfscope}%
\begin{pgfscope}%
\pgfpathrectangle{\pgfqpoint{0.211875in}{0.211875in}}{\pgfqpoint{1.313625in}{1.279725in}}%
\pgfusepath{clip}%
\pgfsetbuttcap%
\pgfsetroundjoin%
\definecolor{currentfill}{rgb}{0.947270,0.405591,0.279023}%
\pgfsetfillcolor{currentfill}%
\pgfsetlinewidth{0.000000pt}%
\definecolor{currentstroke}{rgb}{0.000000,0.000000,0.000000}%
\pgfsetstrokecolor{currentstroke}%
\pgfsetdash{}{0pt}%
\pgfpathmoveto{\pgfqpoint{0.221076in}{0.625523in}}%
\pgfpathlineto{\pgfqpoint{0.224383in}{0.638450in}}%
\pgfpathlineto{\pgfqpoint{0.221992in}{0.651377in}}%
\pgfpathlineto{\pgfqpoint{0.211875in}{0.661931in}}%
\pgfpathlineto{\pgfqpoint{0.211875in}{0.651377in}}%
\pgfpathlineto{\pgfqpoint{0.211875in}{0.646049in}}%
\pgfpathlineto{\pgfqpoint{0.214022in}{0.638450in}}%
\pgfpathlineto{\pgfqpoint{0.211875in}{0.632303in}}%
\pgfpathlineto{\pgfqpoint{0.211875in}{0.625523in}}%
\pgfpathlineto{\pgfqpoint{0.211875in}{0.616660in}}%
\pgfpathclose%
\pgfusepath{fill}%
\end{pgfscope}%
\begin{pgfscope}%
\pgfpathrectangle{\pgfqpoint{0.211875in}{0.211875in}}{\pgfqpoint{1.313625in}{1.279725in}}%
\pgfusepath{clip}%
\pgfsetbuttcap%
\pgfsetroundjoin%
\definecolor{currentfill}{rgb}{0.947270,0.405591,0.279023}%
\pgfsetfillcolor{currentfill}%
\pgfsetlinewidth{0.000000pt}%
\definecolor{currentstroke}{rgb}{0.000000,0.000000,0.000000}%
\pgfsetstrokecolor{currentstroke}%
\pgfsetdash{}{0pt}%
\pgfpathmoveto{\pgfqpoint{0.384371in}{0.688816in}}%
\pgfpathlineto{\pgfqpoint{0.391736in}{0.690156in}}%
\pgfpathlineto{\pgfqpoint{0.397640in}{0.691967in}}%
\pgfpathlineto{\pgfqpoint{0.404691in}{0.703083in}}%
\pgfpathlineto{\pgfqpoint{0.406636in}{0.716009in}}%
\pgfpathlineto{\pgfqpoint{0.406440in}{0.728936in}}%
\pgfpathlineto{\pgfqpoint{0.403552in}{0.741862in}}%
\pgfpathlineto{\pgfqpoint{0.397640in}{0.748931in}}%
\pgfpathlineto{\pgfqpoint{0.384371in}{0.751749in}}%
\pgfpathlineto{\pgfqpoint{0.371102in}{0.748503in}}%
\pgfpathlineto{\pgfqpoint{0.365429in}{0.741862in}}%
\pgfpathlineto{\pgfqpoint{0.362174in}{0.728936in}}%
\pgfpathlineto{\pgfqpoint{0.361923in}{0.716009in}}%
\pgfpathlineto{\pgfqpoint{0.364081in}{0.703083in}}%
\pgfpathlineto{\pgfqpoint{0.371102in}{0.692349in}}%
\pgfpathlineto{\pgfqpoint{0.377819in}{0.690156in}}%
\pgfpathclose%
\pgfpathmoveto{\pgfqpoint{0.374062in}{0.703083in}}%
\pgfpathlineto{\pgfqpoint{0.371102in}{0.706526in}}%
\pgfpathlineto{\pgfqpoint{0.368545in}{0.716009in}}%
\pgfpathlineto{\pgfqpoint{0.369141in}{0.728936in}}%
\pgfpathlineto{\pgfqpoint{0.371102in}{0.734107in}}%
\pgfpathlineto{\pgfqpoint{0.381711in}{0.741862in}}%
\pgfpathlineto{\pgfqpoint{0.384371in}{0.742809in}}%
\pgfpathlineto{\pgfqpoint{0.387047in}{0.741862in}}%
\pgfpathlineto{\pgfqpoint{0.397640in}{0.733706in}}%
\pgfpathlineto{\pgfqpoint{0.399373in}{0.728936in}}%
\pgfpathlineto{\pgfqpoint{0.399936in}{0.716009in}}%
\pgfpathlineto{\pgfqpoint{0.397640in}{0.707138in}}%
\pgfpathlineto{\pgfqpoint{0.394418in}{0.703083in}}%
\pgfpathlineto{\pgfqpoint{0.384371in}{0.698288in}}%
\pgfpathclose%
\pgfusepath{fill}%
\end{pgfscope}%
\begin{pgfscope}%
\pgfpathrectangle{\pgfqpoint{0.211875in}{0.211875in}}{\pgfqpoint{1.313625in}{1.279725in}}%
\pgfusepath{clip}%
\pgfsetbuttcap%
\pgfsetroundjoin%
\definecolor{currentfill}{rgb}{0.947270,0.405591,0.279023}%
\pgfsetfillcolor{currentfill}%
\pgfsetlinewidth{0.000000pt}%
\definecolor{currentstroke}{rgb}{0.000000,0.000000,0.000000}%
\pgfsetstrokecolor{currentstroke}%
\pgfsetdash{}{0pt}%
\pgfpathmoveto{\pgfqpoint{0.490523in}{0.687373in}}%
\pgfpathlineto{\pgfqpoint{0.503792in}{0.685731in}}%
\pgfpathlineto{\pgfqpoint{0.517061in}{0.688145in}}%
\pgfpathlineto{\pgfqpoint{0.519877in}{0.690156in}}%
\pgfpathlineto{\pgfqpoint{0.525745in}{0.703083in}}%
\pgfpathlineto{\pgfqpoint{0.527003in}{0.716009in}}%
\pgfpathlineto{\pgfqpoint{0.526948in}{0.728936in}}%
\pgfpathlineto{\pgfqpoint{0.525319in}{0.741862in}}%
\pgfpathlineto{\pgfqpoint{0.517061in}{0.753061in}}%
\pgfpathlineto{\pgfqpoint{0.507709in}{0.754789in}}%
\pgfpathlineto{\pgfqpoint{0.503792in}{0.755228in}}%
\pgfpathlineto{\pgfqpoint{0.499262in}{0.754789in}}%
\pgfpathlineto{\pgfqpoint{0.490523in}{0.753645in}}%
\pgfpathlineto{\pgfqpoint{0.479203in}{0.741862in}}%
\pgfpathlineto{\pgfqpoint{0.477254in}{0.731017in}}%
\pgfpathlineto{\pgfqpoint{0.477037in}{0.728936in}}%
\pgfpathlineto{\pgfqpoint{0.476941in}{0.716009in}}%
\pgfpathlineto{\pgfqpoint{0.477254in}{0.712434in}}%
\pgfpathlineto{\pgfqpoint{0.478487in}{0.703083in}}%
\pgfpathlineto{\pgfqpoint{0.485675in}{0.690156in}}%
\pgfpathclose%
\pgfpathmoveto{\pgfqpoint{0.487128in}{0.703083in}}%
\pgfpathlineto{\pgfqpoint{0.483872in}{0.716009in}}%
\pgfpathlineto{\pgfqpoint{0.484355in}{0.728936in}}%
\pgfpathlineto{\pgfqpoint{0.489598in}{0.741862in}}%
\pgfpathlineto{\pgfqpoint{0.490523in}{0.742825in}}%
\pgfpathlineto{\pgfqpoint{0.503792in}{0.746288in}}%
\pgfpathlineto{\pgfqpoint{0.513933in}{0.741862in}}%
\pgfpathlineto{\pgfqpoint{0.517061in}{0.738654in}}%
\pgfpathlineto{\pgfqpoint{0.520175in}{0.728936in}}%
\pgfpathlineto{\pgfqpoint{0.520580in}{0.716009in}}%
\pgfpathlineto{\pgfqpoint{0.517738in}{0.703083in}}%
\pgfpathlineto{\pgfqpoint{0.517061in}{0.701878in}}%
\pgfpathlineto{\pgfqpoint{0.503792in}{0.694475in}}%
\pgfpathlineto{\pgfqpoint{0.490523in}{0.698452in}}%
\pgfpathclose%
\pgfusepath{fill}%
\end{pgfscope}%
\begin{pgfscope}%
\pgfpathrectangle{\pgfqpoint{0.211875in}{0.211875in}}{\pgfqpoint{1.313625in}{1.279725in}}%
\pgfusepath{clip}%
\pgfsetbuttcap%
\pgfsetroundjoin%
\definecolor{currentfill}{rgb}{0.947270,0.405591,0.279023}%
\pgfsetfillcolor{currentfill}%
\pgfsetlinewidth{0.000000pt}%
\definecolor{currentstroke}{rgb}{0.000000,0.000000,0.000000}%
\pgfsetstrokecolor{currentstroke}%
\pgfsetdash{}{0pt}%
\pgfpathmoveto{\pgfqpoint{0.609943in}{0.683952in}}%
\pgfpathlineto{\pgfqpoint{0.623212in}{0.683169in}}%
\pgfpathlineto{\pgfqpoint{0.636481in}{0.684954in}}%
\pgfpathlineto{\pgfqpoint{0.642869in}{0.690156in}}%
\pgfpathlineto{\pgfqpoint{0.646119in}{0.703083in}}%
\pgfpathlineto{\pgfqpoint{0.646854in}{0.716009in}}%
\pgfpathlineto{\pgfqpoint{0.646907in}{0.728936in}}%
\pgfpathlineto{\pgfqpoint{0.646231in}{0.741862in}}%
\pgfpathlineto{\pgfqpoint{0.640923in}{0.754789in}}%
\pgfpathlineto{\pgfqpoint{0.636481in}{0.756714in}}%
\pgfpathlineto{\pgfqpoint{0.623212in}{0.757849in}}%
\pgfpathlineto{\pgfqpoint{0.609943in}{0.757205in}}%
\pgfpathlineto{\pgfqpoint{0.601545in}{0.754789in}}%
\pgfpathlineto{\pgfqpoint{0.596674in}{0.750669in}}%
\pgfpathlineto{\pgfqpoint{0.593883in}{0.741862in}}%
\pgfpathlineto{\pgfqpoint{0.592925in}{0.728936in}}%
\pgfpathlineto{\pgfqpoint{0.592932in}{0.716009in}}%
\pgfpathlineto{\pgfqpoint{0.593791in}{0.703083in}}%
\pgfpathlineto{\pgfqpoint{0.596674in}{0.691887in}}%
\pgfpathlineto{\pgfqpoint{0.597780in}{0.690156in}}%
\pgfpathclose%
\pgfpathmoveto{\pgfqpoint{0.602147in}{0.703083in}}%
\pgfpathlineto{\pgfqpoint{0.599264in}{0.716009in}}%
\pgfpathlineto{\pgfqpoint{0.599646in}{0.728936in}}%
\pgfpathlineto{\pgfqpoint{0.604153in}{0.741862in}}%
\pgfpathlineto{\pgfqpoint{0.609943in}{0.747219in}}%
\pgfpathlineto{\pgfqpoint{0.623212in}{0.749119in}}%
\pgfpathlineto{\pgfqpoint{0.636481in}{0.742420in}}%
\pgfpathlineto{\pgfqpoint{0.636842in}{0.741862in}}%
\pgfpathlineto{\pgfqpoint{0.640362in}{0.728936in}}%
\pgfpathlineto{\pgfqpoint{0.640646in}{0.716009in}}%
\pgfpathlineto{\pgfqpoint{0.638388in}{0.703083in}}%
\pgfpathlineto{\pgfqpoint{0.636481in}{0.699236in}}%
\pgfpathlineto{\pgfqpoint{0.623212in}{0.691383in}}%
\pgfpathlineto{\pgfqpoint{0.609943in}{0.693603in}}%
\pgfpathclose%
\pgfusepath{fill}%
\end{pgfscope}%
\begin{pgfscope}%
\pgfpathrectangle{\pgfqpoint{0.211875in}{0.211875in}}{\pgfqpoint{1.313625in}{1.279725in}}%
\pgfusepath{clip}%
\pgfsetbuttcap%
\pgfsetroundjoin%
\definecolor{currentfill}{rgb}{0.947270,0.405591,0.279023}%
\pgfsetfillcolor{currentfill}%
\pgfsetlinewidth{0.000000pt}%
\definecolor{currentstroke}{rgb}{0.000000,0.000000,0.000000}%
\pgfsetstrokecolor{currentstroke}%
\pgfsetdash{}{0pt}%
\pgfpathmoveto{\pgfqpoint{0.716095in}{0.684344in}}%
\pgfpathlineto{\pgfqpoint{0.729364in}{0.681432in}}%
\pgfpathlineto{\pgfqpoint{0.742633in}{0.681100in}}%
\pgfpathlineto{\pgfqpoint{0.755902in}{0.682038in}}%
\pgfpathlineto{\pgfqpoint{0.764573in}{0.690156in}}%
\pgfpathlineto{\pgfqpoint{0.765909in}{0.703083in}}%
\pgfpathlineto{\pgfqpoint{0.766258in}{0.716009in}}%
\pgfpathlineto{\pgfqpoint{0.766391in}{0.728936in}}%
\pgfpathlineto{\pgfqpoint{0.766417in}{0.741862in}}%
\pgfpathlineto{\pgfqpoint{0.766160in}{0.754789in}}%
\pgfpathlineto{\pgfqpoint{0.755902in}{0.759976in}}%
\pgfpathlineto{\pgfqpoint{0.742633in}{0.759981in}}%
\pgfpathlineto{\pgfqpoint{0.729364in}{0.759723in}}%
\pgfpathlineto{\pgfqpoint{0.716095in}{0.758288in}}%
\pgfpathlineto{\pgfqpoint{0.711575in}{0.754789in}}%
\pgfpathlineto{\pgfqpoint{0.709479in}{0.741862in}}%
\pgfpathlineto{\pgfqpoint{0.709217in}{0.728936in}}%
\pgfpathlineto{\pgfqpoint{0.709310in}{0.716009in}}%
\pgfpathlineto{\pgfqpoint{0.709795in}{0.703083in}}%
\pgfpathlineto{\pgfqpoint{0.711753in}{0.690156in}}%
\pgfpathclose%
\pgfpathmoveto{\pgfqpoint{0.729085in}{0.690156in}}%
\pgfpathlineto{\pgfqpoint{0.717252in}{0.703083in}}%
\pgfpathlineto{\pgfqpoint{0.716095in}{0.708183in}}%
\pgfpathlineto{\pgfqpoint{0.715070in}{0.716009in}}%
\pgfpathlineto{\pgfqpoint{0.715281in}{0.728936in}}%
\pgfpathlineto{\pgfqpoint{0.716095in}{0.733692in}}%
\pgfpathlineto{\pgfqpoint{0.718865in}{0.741862in}}%
\pgfpathlineto{\pgfqpoint{0.729364in}{0.750471in}}%
\pgfpathlineto{\pgfqpoint{0.742633in}{0.751334in}}%
\pgfpathlineto{\pgfqpoint{0.755902in}{0.744342in}}%
\pgfpathlineto{\pgfqpoint{0.757291in}{0.741862in}}%
\pgfpathlineto{\pgfqpoint{0.760019in}{0.728936in}}%
\pgfpathlineto{\pgfqpoint{0.760213in}{0.716009in}}%
\pgfpathlineto{\pgfqpoint{0.758388in}{0.703083in}}%
\pgfpathlineto{\pgfqpoint{0.755902in}{0.697348in}}%
\pgfpathlineto{\pgfqpoint{0.745530in}{0.690156in}}%
\pgfpathlineto{\pgfqpoint{0.742633in}{0.689216in}}%
\pgfpathlineto{\pgfqpoint{0.729364in}{0.690030in}}%
\pgfpathclose%
\pgfusepath{fill}%
\end{pgfscope}%
\begin{pgfscope}%
\pgfpathrectangle{\pgfqpoint{0.211875in}{0.211875in}}{\pgfqpoint{1.313625in}{1.279725in}}%
\pgfusepath{clip}%
\pgfsetbuttcap%
\pgfsetroundjoin%
\definecolor{currentfill}{rgb}{0.947270,0.405591,0.279023}%
\pgfsetfillcolor{currentfill}%
\pgfsetlinewidth{0.000000pt}%
\definecolor{currentstroke}{rgb}{0.000000,0.000000,0.000000}%
\pgfsetstrokecolor{currentstroke}%
\pgfsetdash{}{0pt}%
\pgfpathmoveto{\pgfqpoint{0.251682in}{0.699800in}}%
\pgfpathlineto{\pgfqpoint{0.264951in}{0.693063in}}%
\pgfpathlineto{\pgfqpoint{0.278220in}{0.696643in}}%
\pgfpathlineto{\pgfqpoint{0.282817in}{0.703083in}}%
\pgfpathlineto{\pgfqpoint{0.285653in}{0.716009in}}%
\pgfpathlineto{\pgfqpoint{0.285272in}{0.728936in}}%
\pgfpathlineto{\pgfqpoint{0.280749in}{0.741862in}}%
\pgfpathlineto{\pgfqpoint{0.278220in}{0.744536in}}%
\pgfpathlineto{\pgfqpoint{0.264951in}{0.747569in}}%
\pgfpathlineto{\pgfqpoint{0.251928in}{0.741862in}}%
\pgfpathlineto{\pgfqpoint{0.251682in}{0.741624in}}%
\pgfpathlineto{\pgfqpoint{0.247410in}{0.728936in}}%
\pgfpathlineto{\pgfqpoint{0.247016in}{0.716009in}}%
\pgfpathlineto{\pgfqpoint{0.249768in}{0.703083in}}%
\pgfpathclose%
\pgfpathmoveto{\pgfqpoint{0.264602in}{0.703083in}}%
\pgfpathlineto{\pgfqpoint{0.255343in}{0.716009in}}%
\pgfpathlineto{\pgfqpoint{0.256936in}{0.728936in}}%
\pgfpathlineto{\pgfqpoint{0.264951in}{0.736704in}}%
\pgfpathlineto{\pgfqpoint{0.277067in}{0.728936in}}%
\pgfpathlineto{\pgfqpoint{0.278220in}{0.722628in}}%
\pgfpathlineto{\pgfqpoint{0.278603in}{0.716009in}}%
\pgfpathlineto{\pgfqpoint{0.278220in}{0.714689in}}%
\pgfpathlineto{\pgfqpoint{0.265493in}{0.703083in}}%
\pgfpathlineto{\pgfqpoint{0.264951in}{0.702882in}}%
\pgfpathclose%
\pgfusepath{fill}%
\end{pgfscope}%
\begin{pgfscope}%
\pgfpathrectangle{\pgfqpoint{0.211875in}{0.211875in}}{\pgfqpoint{1.313625in}{1.279725in}}%
\pgfusepath{clip}%
\pgfsetbuttcap%
\pgfsetroundjoin%
\definecolor{currentfill}{rgb}{0.947270,0.405591,0.279023}%
\pgfsetfillcolor{currentfill}%
\pgfsetlinewidth{0.000000pt}%
\definecolor{currentstroke}{rgb}{0.000000,0.000000,0.000000}%
\pgfsetstrokecolor{currentstroke}%
\pgfsetdash{}{0pt}%
\pgfpathmoveto{\pgfqpoint{0.437447in}{0.767458in}}%
\pgfpathlineto{\pgfqpoint{0.450716in}{0.767463in}}%
\pgfpathlineto{\pgfqpoint{0.451689in}{0.767715in}}%
\pgfpathlineto{\pgfqpoint{0.463985in}{0.775499in}}%
\pgfpathlineto{\pgfqpoint{0.466079in}{0.780642in}}%
\pgfpathlineto{\pgfqpoint{0.467851in}{0.793568in}}%
\pgfpathlineto{\pgfqpoint{0.468066in}{0.806495in}}%
\pgfpathlineto{\pgfqpoint{0.467017in}{0.819421in}}%
\pgfpathlineto{\pgfqpoint{0.463985in}{0.828516in}}%
\pgfpathlineto{\pgfqpoint{0.459881in}{0.832348in}}%
\pgfpathlineto{\pgfqpoint{0.450716in}{0.835557in}}%
\pgfpathlineto{\pgfqpoint{0.437447in}{0.835511in}}%
\pgfpathlineto{\pgfqpoint{0.427879in}{0.832348in}}%
\pgfpathlineto{\pgfqpoint{0.424178in}{0.829719in}}%
\pgfpathlineto{\pgfqpoint{0.419804in}{0.819421in}}%
\pgfpathlineto{\pgfqpoint{0.418503in}{0.806495in}}%
\pgfpathlineto{\pgfqpoint{0.418734in}{0.793568in}}%
\pgfpathlineto{\pgfqpoint{0.420806in}{0.780642in}}%
\pgfpathlineto{\pgfqpoint{0.424178in}{0.774005in}}%
\pgfpathlineto{\pgfqpoint{0.436446in}{0.767715in}}%
\pgfpathclose%
\pgfpathmoveto{\pgfqpoint{0.431898in}{0.780642in}}%
\pgfpathlineto{\pgfqpoint{0.425133in}{0.793568in}}%
\pgfpathlineto{\pgfqpoint{0.424563in}{0.806495in}}%
\pgfpathlineto{\pgfqpoint{0.429362in}{0.819421in}}%
\pgfpathlineto{\pgfqpoint{0.437447in}{0.826089in}}%
\pgfpathlineto{\pgfqpoint{0.450716in}{0.825577in}}%
\pgfpathlineto{\pgfqpoint{0.457158in}{0.819421in}}%
\pgfpathlineto{\pgfqpoint{0.461224in}{0.806495in}}%
\pgfpathlineto{\pgfqpoint{0.460735in}{0.793568in}}%
\pgfpathlineto{\pgfqpoint{0.454913in}{0.780642in}}%
\pgfpathlineto{\pgfqpoint{0.450716in}{0.777233in}}%
\pgfpathlineto{\pgfqpoint{0.437447in}{0.776721in}}%
\pgfpathclose%
\pgfusepath{fill}%
\end{pgfscope}%
\begin{pgfscope}%
\pgfpathrectangle{\pgfqpoint{0.211875in}{0.211875in}}{\pgfqpoint{1.313625in}{1.279725in}}%
\pgfusepath{clip}%
\pgfsetbuttcap%
\pgfsetroundjoin%
\definecolor{currentfill}{rgb}{0.947270,0.405591,0.279023}%
\pgfsetfillcolor{currentfill}%
\pgfsetlinewidth{0.000000pt}%
\definecolor{currentstroke}{rgb}{0.000000,0.000000,0.000000}%
\pgfsetstrokecolor{currentstroke}%
\pgfsetdash{}{0pt}%
\pgfpathmoveto{\pgfqpoint{0.543598in}{0.767158in}}%
\pgfpathlineto{\pgfqpoint{0.556867in}{0.764413in}}%
\pgfpathlineto{\pgfqpoint{0.570136in}{0.764556in}}%
\pgfpathlineto{\pgfqpoint{0.580720in}{0.767715in}}%
\pgfpathlineto{\pgfqpoint{0.583405in}{0.769975in}}%
\pgfpathlineto{\pgfqpoint{0.586985in}{0.780642in}}%
\pgfpathlineto{\pgfqpoint{0.587959in}{0.793568in}}%
\pgfpathlineto{\pgfqpoint{0.588129in}{0.806495in}}%
\pgfpathlineto{\pgfqpoint{0.587702in}{0.819421in}}%
\pgfpathlineto{\pgfqpoint{0.585156in}{0.832348in}}%
\pgfpathlineto{\pgfqpoint{0.583405in}{0.834748in}}%
\pgfpathlineto{\pgfqpoint{0.570136in}{0.838606in}}%
\pgfpathlineto{\pgfqpoint{0.556867in}{0.838637in}}%
\pgfpathlineto{\pgfqpoint{0.543598in}{0.836341in}}%
\pgfpathlineto{\pgfqpoint{0.538614in}{0.832348in}}%
\pgfpathlineto{\pgfqpoint{0.534895in}{0.819421in}}%
\pgfpathlineto{\pgfqpoint{0.534204in}{0.806495in}}%
\pgfpathlineto{\pgfqpoint{0.534395in}{0.793568in}}%
\pgfpathlineto{\pgfqpoint{0.535703in}{0.780642in}}%
\pgfpathlineto{\pgfqpoint{0.542681in}{0.767715in}}%
\pgfpathclose%
\pgfpathmoveto{\pgfqpoint{0.544439in}{0.780642in}}%
\pgfpathlineto{\pgfqpoint{0.543598in}{0.781984in}}%
\pgfpathlineto{\pgfqpoint{0.540582in}{0.793568in}}%
\pgfpathlineto{\pgfqpoint{0.540267in}{0.806495in}}%
\pgfpathlineto{\pgfqpoint{0.542675in}{0.819421in}}%
\pgfpathlineto{\pgfqpoint{0.543598in}{0.821311in}}%
\pgfpathlineto{\pgfqpoint{0.556867in}{0.830094in}}%
\pgfpathlineto{\pgfqpoint{0.570136in}{0.828842in}}%
\pgfpathlineto{\pgfqpoint{0.578837in}{0.819421in}}%
\pgfpathlineto{\pgfqpoint{0.581750in}{0.806495in}}%
\pgfpathlineto{\pgfqpoint{0.581358in}{0.793568in}}%
\pgfpathlineto{\pgfqpoint{0.577065in}{0.780642in}}%
\pgfpathlineto{\pgfqpoint{0.570136in}{0.774269in}}%
\pgfpathlineto{\pgfqpoint{0.556867in}{0.773019in}}%
\pgfpathclose%
\pgfusepath{fill}%
\end{pgfscope}%
\begin{pgfscope}%
\pgfpathrectangle{\pgfqpoint{0.211875in}{0.211875in}}{\pgfqpoint{1.313625in}{1.279725in}}%
\pgfusepath{clip}%
\pgfsetbuttcap%
\pgfsetroundjoin%
\definecolor{currentfill}{rgb}{0.947270,0.405591,0.279023}%
\pgfsetfillcolor{currentfill}%
\pgfsetlinewidth{0.000000pt}%
\definecolor{currentstroke}{rgb}{0.000000,0.000000,0.000000}%
\pgfsetstrokecolor{currentstroke}%
\pgfsetdash{}{0pt}%
\pgfpathmoveto{\pgfqpoint{0.663019in}{0.763060in}}%
\pgfpathlineto{\pgfqpoint{0.676288in}{0.762014in}}%
\pgfpathlineto{\pgfqpoint{0.689557in}{0.762035in}}%
\pgfpathlineto{\pgfqpoint{0.702826in}{0.764298in}}%
\pgfpathlineto{\pgfqpoint{0.705549in}{0.767715in}}%
\pgfpathlineto{\pgfqpoint{0.707294in}{0.780642in}}%
\pgfpathlineto{\pgfqpoint{0.707636in}{0.793568in}}%
\pgfpathlineto{\pgfqpoint{0.707771in}{0.806495in}}%
\pgfpathlineto{\pgfqpoint{0.707838in}{0.819421in}}%
\pgfpathlineto{\pgfqpoint{0.707887in}{0.832348in}}%
\pgfpathlineto{\pgfqpoint{0.702826in}{0.841375in}}%
\pgfpathlineto{\pgfqpoint{0.689557in}{0.841277in}}%
\pgfpathlineto{\pgfqpoint{0.676288in}{0.841099in}}%
\pgfpathlineto{\pgfqpoint{0.663019in}{0.840489in}}%
\pgfpathlineto{\pgfqpoint{0.651398in}{0.832348in}}%
\pgfpathlineto{\pgfqpoint{0.650289in}{0.819421in}}%
\pgfpathlineto{\pgfqpoint{0.650127in}{0.806495in}}%
\pgfpathlineto{\pgfqpoint{0.650285in}{0.793568in}}%
\pgfpathlineto{\pgfqpoint{0.650934in}{0.780642in}}%
\pgfpathlineto{\pgfqpoint{0.654267in}{0.767715in}}%
\pgfpathclose%
\pgfpathmoveto{\pgfqpoint{0.659588in}{0.780642in}}%
\pgfpathlineto{\pgfqpoint{0.656676in}{0.793568in}}%
\pgfpathlineto{\pgfqpoint{0.656390in}{0.806495in}}%
\pgfpathlineto{\pgfqpoint{0.658331in}{0.819421in}}%
\pgfpathlineto{\pgfqpoint{0.663019in}{0.827835in}}%
\pgfpathlineto{\pgfqpoint{0.673342in}{0.832348in}}%
\pgfpathlineto{\pgfqpoint{0.676288in}{0.833064in}}%
\pgfpathlineto{\pgfqpoint{0.683709in}{0.832348in}}%
\pgfpathlineto{\pgfqpoint{0.689557in}{0.831526in}}%
\pgfpathlineto{\pgfqpoint{0.699464in}{0.819421in}}%
\pgfpathlineto{\pgfqpoint{0.701558in}{0.806495in}}%
\pgfpathlineto{\pgfqpoint{0.701237in}{0.793568in}}%
\pgfpathlineto{\pgfqpoint{0.698032in}{0.780642in}}%
\pgfpathlineto{\pgfqpoint{0.689557in}{0.771856in}}%
\pgfpathlineto{\pgfqpoint{0.676288in}{0.770128in}}%
\pgfpathlineto{\pgfqpoint{0.663019in}{0.775444in}}%
\pgfpathclose%
\pgfusepath{fill}%
\end{pgfscope}%
\begin{pgfscope}%
\pgfpathrectangle{\pgfqpoint{0.211875in}{0.211875in}}{\pgfqpoint{1.313625in}{1.279725in}}%
\pgfusepath{clip}%
\pgfsetbuttcap%
\pgfsetroundjoin%
\definecolor{currentfill}{rgb}{0.947270,0.405591,0.279023}%
\pgfsetfillcolor{currentfill}%
\pgfsetlinewidth{0.000000pt}%
\definecolor{currentstroke}{rgb}{0.000000,0.000000,0.000000}%
\pgfsetstrokecolor{currentstroke}%
\pgfsetdash{}{0pt}%
\pgfpathmoveto{\pgfqpoint{0.220127in}{0.780642in}}%
\pgfpathlineto{\pgfqpoint{0.225144in}{0.789293in}}%
\pgfpathlineto{\pgfqpoint{0.226194in}{0.793568in}}%
\pgfpathlineto{\pgfqpoint{0.226534in}{0.806495in}}%
\pgfpathlineto{\pgfqpoint{0.225144in}{0.813863in}}%
\pgfpathlineto{\pgfqpoint{0.222807in}{0.819421in}}%
\pgfpathlineto{\pgfqpoint{0.211875in}{0.827360in}}%
\pgfpathlineto{\pgfqpoint{0.211875in}{0.819421in}}%
\pgfpathlineto{\pgfqpoint{0.211875in}{0.815440in}}%
\pgfpathlineto{\pgfqpoint{0.216910in}{0.806495in}}%
\pgfpathlineto{\pgfqpoint{0.216100in}{0.793568in}}%
\pgfpathlineto{\pgfqpoint{0.211875in}{0.787568in}}%
\pgfpathlineto{\pgfqpoint{0.211875in}{0.780642in}}%
\pgfpathlineto{\pgfqpoint{0.211875in}{0.775548in}}%
\pgfpathclose%
\pgfusepath{fill}%
\end{pgfscope}%
\begin{pgfscope}%
\pgfpathrectangle{\pgfqpoint{0.211875in}{0.211875in}}{\pgfqpoint{1.313625in}{1.279725in}}%
\pgfusepath{clip}%
\pgfsetbuttcap%
\pgfsetroundjoin%
\definecolor{currentfill}{rgb}{0.947270,0.405591,0.279023}%
\pgfsetfillcolor{currentfill}%
\pgfsetlinewidth{0.000000pt}%
\definecolor{currentstroke}{rgb}{0.000000,0.000000,0.000000}%
\pgfsetstrokecolor{currentstroke}%
\pgfsetdash{}{0pt}%
\pgfpathmoveto{\pgfqpoint{0.318027in}{0.771803in}}%
\pgfpathlineto{\pgfqpoint{0.331295in}{0.771242in}}%
\pgfpathlineto{\pgfqpoint{0.344494in}{0.780642in}}%
\pgfpathlineto{\pgfqpoint{0.344564in}{0.780796in}}%
\pgfpathlineto{\pgfqpoint{0.347280in}{0.793568in}}%
\pgfpathlineto{\pgfqpoint{0.347552in}{0.806495in}}%
\pgfpathlineto{\pgfqpoint{0.345735in}{0.819421in}}%
\pgfpathlineto{\pgfqpoint{0.344564in}{0.822367in}}%
\pgfpathlineto{\pgfqpoint{0.331295in}{0.832064in}}%
\pgfpathlineto{\pgfqpoint{0.318027in}{0.831459in}}%
\pgfpathlineto{\pgfqpoint{0.305212in}{0.819421in}}%
\pgfpathlineto{\pgfqpoint{0.304758in}{0.817824in}}%
\pgfpathlineto{\pgfqpoint{0.303067in}{0.806495in}}%
\pgfpathlineto{\pgfqpoint{0.303346in}{0.793568in}}%
\pgfpathlineto{\pgfqpoint{0.304758in}{0.786406in}}%
\pgfpathlineto{\pgfqpoint{0.307037in}{0.780642in}}%
\pgfpathclose%
\pgfpathmoveto{\pgfqpoint{0.311485in}{0.793568in}}%
\pgfpathlineto{\pgfqpoint{0.310895in}{0.806495in}}%
\pgfpathlineto{\pgfqpoint{0.316269in}{0.819421in}}%
\pgfpathlineto{\pgfqpoint{0.318027in}{0.821072in}}%
\pgfpathlineto{\pgfqpoint{0.331295in}{0.821735in}}%
\pgfpathlineto{\pgfqpoint{0.334058in}{0.819421in}}%
\pgfpathlineto{\pgfqpoint{0.339740in}{0.806495in}}%
\pgfpathlineto{\pgfqpoint{0.339117in}{0.793568in}}%
\pgfpathlineto{\pgfqpoint{0.331295in}{0.780807in}}%
\pgfpathlineto{\pgfqpoint{0.318027in}{0.781826in}}%
\pgfpathclose%
\pgfusepath{fill}%
\end{pgfscope}%
\begin{pgfscope}%
\pgfpathrectangle{\pgfqpoint{0.211875in}{0.211875in}}{\pgfqpoint{1.313625in}{1.279725in}}%
\pgfusepath{clip}%
\pgfsetbuttcap%
\pgfsetroundjoin%
\definecolor{currentfill}{rgb}{0.947270,0.405591,0.279023}%
\pgfsetfillcolor{currentfill}%
\pgfsetlinewidth{0.000000pt}%
\definecolor{currentstroke}{rgb}{0.000000,0.000000,0.000000}%
\pgfsetstrokecolor{currentstroke}%
\pgfsetdash{}{0pt}%
\pgfpathmoveto{\pgfqpoint{0.609943in}{0.843649in}}%
\pgfpathlineto{\pgfqpoint{0.623212in}{0.843159in}}%
\pgfpathlineto{\pgfqpoint{0.636481in}{0.843620in}}%
\pgfpathlineto{\pgfqpoint{0.642951in}{0.845274in}}%
\pgfpathlineto{\pgfqpoint{0.647900in}{0.858201in}}%
\pgfpathlineto{\pgfqpoint{0.648400in}{0.871127in}}%
\pgfpathlineto{\pgfqpoint{0.648554in}{0.884054in}}%
\pgfpathlineto{\pgfqpoint{0.648580in}{0.896980in}}%
\pgfpathlineto{\pgfqpoint{0.648436in}{0.909907in}}%
\pgfpathlineto{\pgfqpoint{0.636481in}{0.922108in}}%
\pgfpathlineto{\pgfqpoint{0.623212in}{0.922132in}}%
\pgfpathlineto{\pgfqpoint{0.609943in}{0.921718in}}%
\pgfpathlineto{\pgfqpoint{0.596674in}{0.919011in}}%
\pgfpathlineto{\pgfqpoint{0.592322in}{0.909907in}}%
\pgfpathlineto{\pgfqpoint{0.591483in}{0.896980in}}%
\pgfpathlineto{\pgfqpoint{0.591358in}{0.884054in}}%
\pgfpathlineto{\pgfqpoint{0.591580in}{0.871127in}}%
\pgfpathlineto{\pgfqpoint{0.592492in}{0.858201in}}%
\pgfpathlineto{\pgfqpoint{0.596674in}{0.847574in}}%
\pgfpathlineto{\pgfqpoint{0.601269in}{0.845274in}}%
\pgfpathclose%
\pgfpathmoveto{\pgfqpoint{0.603470in}{0.858201in}}%
\pgfpathlineto{\pgfqpoint{0.598151in}{0.871127in}}%
\pgfpathlineto{\pgfqpoint{0.597143in}{0.884054in}}%
\pgfpathlineto{\pgfqpoint{0.598642in}{0.896980in}}%
\pgfpathlineto{\pgfqpoint{0.606054in}{0.909907in}}%
\pgfpathlineto{\pgfqpoint{0.609943in}{0.912333in}}%
\pgfpathlineto{\pgfqpoint{0.623212in}{0.913811in}}%
\pgfpathlineto{\pgfqpoint{0.634620in}{0.909907in}}%
\pgfpathlineto{\pgfqpoint{0.636481in}{0.908557in}}%
\pgfpathlineto{\pgfqpoint{0.641277in}{0.896980in}}%
\pgfpathlineto{\pgfqpoint{0.642400in}{0.884054in}}%
\pgfpathlineto{\pgfqpoint{0.641630in}{0.871127in}}%
\pgfpathlineto{\pgfqpoint{0.637502in}{0.858201in}}%
\pgfpathlineto{\pgfqpoint{0.636481in}{0.856886in}}%
\pgfpathlineto{\pgfqpoint{0.623212in}{0.851533in}}%
\pgfpathlineto{\pgfqpoint{0.609943in}{0.853126in}}%
\pgfpathclose%
\pgfusepath{fill}%
\end{pgfscope}%
\begin{pgfscope}%
\pgfpathrectangle{\pgfqpoint{0.211875in}{0.211875in}}{\pgfqpoint{1.313625in}{1.279725in}}%
\pgfusepath{clip}%
\pgfsetbuttcap%
\pgfsetroundjoin%
\definecolor{currentfill}{rgb}{0.947270,0.405591,0.279023}%
\pgfsetfillcolor{currentfill}%
\pgfsetlinewidth{0.000000pt}%
\definecolor{currentstroke}{rgb}{0.000000,0.000000,0.000000}%
\pgfsetstrokecolor{currentstroke}%
\pgfsetdash{}{0pt}%
\pgfpathmoveto{\pgfqpoint{0.251682in}{0.857910in}}%
\pgfpathlineto{\pgfqpoint{0.264951in}{0.853010in}}%
\pgfpathlineto{\pgfqpoint{0.278220in}{0.855454in}}%
\pgfpathlineto{\pgfqpoint{0.281319in}{0.858201in}}%
\pgfpathlineto{\pgfqpoint{0.286675in}{0.871127in}}%
\pgfpathlineto{\pgfqpoint{0.287655in}{0.884054in}}%
\pgfpathlineto{\pgfqpoint{0.286182in}{0.896980in}}%
\pgfpathlineto{\pgfqpoint{0.278528in}{0.909907in}}%
\pgfpathlineto{\pgfqpoint{0.278220in}{0.910121in}}%
\pgfpathlineto{\pgfqpoint{0.264951in}{0.912361in}}%
\pgfpathlineto{\pgfqpoint{0.256829in}{0.909907in}}%
\pgfpathlineto{\pgfqpoint{0.251682in}{0.907055in}}%
\pgfpathlineto{\pgfqpoint{0.246722in}{0.896980in}}%
\pgfpathlineto{\pgfqpoint{0.245212in}{0.884054in}}%
\pgfpathlineto{\pgfqpoint{0.246195in}{0.871127in}}%
\pgfpathlineto{\pgfqpoint{0.251420in}{0.858201in}}%
\pgfpathclose%
\pgfpathmoveto{\pgfqpoint{0.254725in}{0.871127in}}%
\pgfpathlineto{\pgfqpoint{0.251682in}{0.882247in}}%
\pgfpathlineto{\pgfqpoint{0.251477in}{0.884054in}}%
\pgfpathlineto{\pgfqpoint{0.251682in}{0.885223in}}%
\pgfpathlineto{\pgfqpoint{0.257013in}{0.896980in}}%
\pgfpathlineto{\pgfqpoint{0.264951in}{0.902292in}}%
\pgfpathlineto{\pgfqpoint{0.277164in}{0.896980in}}%
\pgfpathlineto{\pgfqpoint{0.278220in}{0.895591in}}%
\pgfpathlineto{\pgfqpoint{0.280668in}{0.884054in}}%
\pgfpathlineto{\pgfqpoint{0.278978in}{0.871127in}}%
\pgfpathlineto{\pgfqpoint{0.278220in}{0.869608in}}%
\pgfpathlineto{\pgfqpoint{0.264951in}{0.862866in}}%
\pgfpathclose%
\pgfusepath{fill}%
\end{pgfscope}%
\begin{pgfscope}%
\pgfpathrectangle{\pgfqpoint{0.211875in}{0.211875in}}{\pgfqpoint{1.313625in}{1.279725in}}%
\pgfusepath{clip}%
\pgfsetbuttcap%
\pgfsetroundjoin%
\definecolor{currentfill}{rgb}{0.947270,0.405591,0.279023}%
\pgfsetfillcolor{currentfill}%
\pgfsetlinewidth{0.000000pt}%
\definecolor{currentstroke}{rgb}{0.000000,0.000000,0.000000}%
\pgfsetstrokecolor{currentstroke}%
\pgfsetdash{}{0pt}%
\pgfpathmoveto{\pgfqpoint{0.371102in}{0.851743in}}%
\pgfpathlineto{\pgfqpoint{0.384371in}{0.849094in}}%
\pgfpathlineto{\pgfqpoint{0.397640in}{0.851244in}}%
\pgfpathlineto{\pgfqpoint{0.404595in}{0.858201in}}%
\pgfpathlineto{\pgfqpoint{0.407882in}{0.871127in}}%
\pgfpathlineto{\pgfqpoint{0.408512in}{0.884054in}}%
\pgfpathlineto{\pgfqpoint{0.407675in}{0.896980in}}%
\pgfpathlineto{\pgfqpoint{0.403233in}{0.909907in}}%
\pgfpathlineto{\pgfqpoint{0.397640in}{0.914303in}}%
\pgfpathlineto{\pgfqpoint{0.384371in}{0.916178in}}%
\pgfpathlineto{\pgfqpoint{0.371102in}{0.913767in}}%
\pgfpathlineto{\pgfqpoint{0.366221in}{0.909907in}}%
\pgfpathlineto{\pgfqpoint{0.361090in}{0.896980in}}%
\pgfpathlineto{\pgfqpoint{0.360060in}{0.884054in}}%
\pgfpathlineto{\pgfqpoint{0.360792in}{0.871127in}}%
\pgfpathlineto{\pgfqpoint{0.364573in}{0.858201in}}%
\pgfpathclose%
\pgfpathmoveto{\pgfqpoint{0.381981in}{0.858201in}}%
\pgfpathlineto{\pgfqpoint{0.371102in}{0.864374in}}%
\pgfpathlineto{\pgfqpoint{0.367987in}{0.871127in}}%
\pgfpathlineto{\pgfqpoint{0.366625in}{0.884054in}}%
\pgfpathlineto{\pgfqpoint{0.368826in}{0.896980in}}%
\pgfpathlineto{\pgfqpoint{0.371102in}{0.901089in}}%
\pgfpathlineto{\pgfqpoint{0.384371in}{0.907381in}}%
\pgfpathlineto{\pgfqpoint{0.397640in}{0.900990in}}%
\pgfpathlineto{\pgfqpoint{0.399782in}{0.896980in}}%
\pgfpathlineto{\pgfqpoint{0.401871in}{0.884054in}}%
\pgfpathlineto{\pgfqpoint{0.400571in}{0.871127in}}%
\pgfpathlineto{\pgfqpoint{0.397640in}{0.864523in}}%
\pgfpathlineto{\pgfqpoint{0.386822in}{0.858201in}}%
\pgfpathlineto{\pgfqpoint{0.384371in}{0.857472in}}%
\pgfpathclose%
\pgfusepath{fill}%
\end{pgfscope}%
\begin{pgfscope}%
\pgfpathrectangle{\pgfqpoint{0.211875in}{0.211875in}}{\pgfqpoint{1.313625in}{1.279725in}}%
\pgfusepath{clip}%
\pgfsetbuttcap%
\pgfsetroundjoin%
\definecolor{currentfill}{rgb}{0.947270,0.405591,0.279023}%
\pgfsetfillcolor{currentfill}%
\pgfsetlinewidth{0.000000pt}%
\definecolor{currentstroke}{rgb}{0.000000,0.000000,0.000000}%
\pgfsetstrokecolor{currentstroke}%
\pgfsetdash{}{0pt}%
\pgfpathmoveto{\pgfqpoint{0.490523in}{0.847056in}}%
\pgfpathlineto{\pgfqpoint{0.503792in}{0.845777in}}%
\pgfpathlineto{\pgfqpoint{0.517061in}{0.847268in}}%
\pgfpathlineto{\pgfqpoint{0.526719in}{0.858201in}}%
\pgfpathlineto{\pgfqpoint{0.528420in}{0.871127in}}%
\pgfpathlineto{\pgfqpoint{0.528779in}{0.884054in}}%
\pgfpathlineto{\pgfqpoint{0.528433in}{0.896980in}}%
\pgfpathlineto{\pgfqpoint{0.526442in}{0.909907in}}%
\pgfpathlineto{\pgfqpoint{0.517061in}{0.918286in}}%
\pgfpathlineto{\pgfqpoint{0.503792in}{0.919417in}}%
\pgfpathlineto{\pgfqpoint{0.490523in}{0.918311in}}%
\pgfpathlineto{\pgfqpoint{0.478564in}{0.909907in}}%
\pgfpathlineto{\pgfqpoint{0.477254in}{0.905971in}}%
\pgfpathlineto{\pgfqpoint{0.475827in}{0.896980in}}%
\pgfpathlineto{\pgfqpoint{0.475350in}{0.884054in}}%
\pgfpathlineto{\pgfqpoint{0.475760in}{0.871127in}}%
\pgfpathlineto{\pgfqpoint{0.477254in}{0.860573in}}%
\pgfpathlineto{\pgfqpoint{0.477871in}{0.858201in}}%
\pgfpathclose%
\pgfpathmoveto{\pgfqpoint{0.489315in}{0.858201in}}%
\pgfpathlineto{\pgfqpoint{0.483037in}{0.871127in}}%
\pgfpathlineto{\pgfqpoint{0.481867in}{0.884054in}}%
\pgfpathlineto{\pgfqpoint{0.483692in}{0.896980in}}%
\pgfpathlineto{\pgfqpoint{0.490523in}{0.907955in}}%
\pgfpathlineto{\pgfqpoint{0.496963in}{0.909907in}}%
\pgfpathlineto{\pgfqpoint{0.503792in}{0.911200in}}%
\pgfpathlineto{\pgfqpoint{0.508331in}{0.909907in}}%
\pgfpathlineto{\pgfqpoint{0.517061in}{0.905056in}}%
\pgfpathlineto{\pgfqpoint{0.520872in}{0.896980in}}%
\pgfpathlineto{\pgfqpoint{0.522413in}{0.884054in}}%
\pgfpathlineto{\pgfqpoint{0.521413in}{0.871127in}}%
\pgfpathlineto{\pgfqpoint{0.517061in}{0.860064in}}%
\pgfpathlineto{\pgfqpoint{0.514662in}{0.858201in}}%
\pgfpathlineto{\pgfqpoint{0.503792in}{0.854203in}}%
\pgfpathlineto{\pgfqpoint{0.490523in}{0.857136in}}%
\pgfpathclose%
\pgfusepath{fill}%
\end{pgfscope}%
\begin{pgfscope}%
\pgfpathrectangle{\pgfqpoint{0.211875in}{0.211875in}}{\pgfqpoint{1.313625in}{1.279725in}}%
\pgfusepath{clip}%
\pgfsetbuttcap%
\pgfsetroundjoin%
\definecolor{currentfill}{rgb}{0.947270,0.405591,0.279023}%
\pgfsetfillcolor{currentfill}%
\pgfsetlinewidth{0.000000pt}%
\definecolor{currentstroke}{rgb}{0.000000,0.000000,0.000000}%
\pgfsetstrokecolor{currentstroke}%
\pgfsetdash{}{0pt}%
\pgfpathmoveto{\pgfqpoint{0.212501in}{0.935760in}}%
\pgfpathlineto{\pgfqpoint{0.225144in}{0.946070in}}%
\pgfpathlineto{\pgfqpoint{0.226165in}{0.948686in}}%
\pgfpathlineto{\pgfqpoint{0.227868in}{0.961613in}}%
\pgfpathlineto{\pgfqpoint{0.227112in}{0.974539in}}%
\pgfpathlineto{\pgfqpoint{0.225144in}{0.981181in}}%
\pgfpathlineto{\pgfqpoint{0.220197in}{0.987466in}}%
\pgfpathlineto{\pgfqpoint{0.211875in}{0.991607in}}%
\pgfpathlineto{\pgfqpoint{0.211875in}{0.987466in}}%
\pgfpathlineto{\pgfqpoint{0.211875in}{0.980497in}}%
\pgfpathlineto{\pgfqpoint{0.217073in}{0.974539in}}%
\pgfpathlineto{\pgfqpoint{0.219395in}{0.961613in}}%
\pgfpathlineto{\pgfqpoint{0.214447in}{0.948686in}}%
\pgfpathlineto{\pgfqpoint{0.211875in}{0.946421in}}%
\pgfpathlineto{\pgfqpoint{0.211875in}{0.935760in}}%
\pgfpathlineto{\pgfqpoint{0.211875in}{0.935533in}}%
\pgfpathclose%
\pgfusepath{fill}%
\end{pgfscope}%
\begin{pgfscope}%
\pgfpathrectangle{\pgfqpoint{0.211875in}{0.211875in}}{\pgfqpoint{1.313625in}{1.279725in}}%
\pgfusepath{clip}%
\pgfsetbuttcap%
\pgfsetroundjoin%
\definecolor{currentfill}{rgb}{0.947270,0.405591,0.279023}%
\pgfsetfillcolor{currentfill}%
\pgfsetlinewidth{0.000000pt}%
\definecolor{currentstroke}{rgb}{0.000000,0.000000,0.000000}%
\pgfsetstrokecolor{currentstroke}%
\pgfsetdash{}{0pt}%
\pgfpathmoveto{\pgfqpoint{0.318027in}{0.932116in}}%
\pgfpathlineto{\pgfqpoint{0.331295in}{0.931624in}}%
\pgfpathlineto{\pgfqpoint{0.341088in}{0.935760in}}%
\pgfpathlineto{\pgfqpoint{0.344564in}{0.939408in}}%
\pgfpathlineto{\pgfqpoint{0.347687in}{0.948686in}}%
\pgfpathlineto{\pgfqpoint{0.348800in}{0.961613in}}%
\pgfpathlineto{\pgfqpoint{0.348342in}{0.974539in}}%
\pgfpathlineto{\pgfqpoint{0.345312in}{0.987466in}}%
\pgfpathlineto{\pgfqpoint{0.344564in}{0.988755in}}%
\pgfpathlineto{\pgfqpoint{0.331295in}{0.995782in}}%
\pgfpathlineto{\pgfqpoint{0.318027in}{0.995260in}}%
\pgfpathlineto{\pgfqpoint{0.306245in}{0.987466in}}%
\pgfpathlineto{\pgfqpoint{0.304758in}{0.984610in}}%
\pgfpathlineto{\pgfqpoint{0.302391in}{0.974539in}}%
\pgfpathlineto{\pgfqpoint{0.301857in}{0.961613in}}%
\pgfpathlineto{\pgfqpoint{0.303103in}{0.948686in}}%
\pgfpathlineto{\pgfqpoint{0.304758in}{0.943346in}}%
\pgfpathlineto{\pgfqpoint{0.310739in}{0.935760in}}%
\pgfpathclose%
\pgfpathmoveto{\pgfqpoint{0.312322in}{0.948686in}}%
\pgfpathlineto{\pgfqpoint{0.308928in}{0.961613in}}%
\pgfpathlineto{\pgfqpoint{0.310510in}{0.974539in}}%
\pgfpathlineto{\pgfqpoint{0.318027in}{0.985488in}}%
\pgfpathlineto{\pgfqpoint{0.331295in}{0.986338in}}%
\pgfpathlineto{\pgfqpoint{0.340242in}{0.974539in}}%
\pgfpathlineto{\pgfqpoint{0.341866in}{0.961613in}}%
\pgfpathlineto{\pgfqpoint{0.338339in}{0.948686in}}%
\pgfpathlineto{\pgfqpoint{0.331295in}{0.941544in}}%
\pgfpathlineto{\pgfqpoint{0.318027in}{0.942256in}}%
\pgfpathclose%
\pgfusepath{fill}%
\end{pgfscope}%
\begin{pgfscope}%
\pgfpathrectangle{\pgfqpoint{0.211875in}{0.211875in}}{\pgfqpoint{1.313625in}{1.279725in}}%
\pgfusepath{clip}%
\pgfsetbuttcap%
\pgfsetroundjoin%
\definecolor{currentfill}{rgb}{0.947270,0.405591,0.279023}%
\pgfsetfillcolor{currentfill}%
\pgfsetlinewidth{0.000000pt}%
\definecolor{currentstroke}{rgb}{0.000000,0.000000,0.000000}%
\pgfsetstrokecolor{currentstroke}%
\pgfsetdash{}{0pt}%
\pgfpathmoveto{\pgfqpoint{0.424178in}{0.932799in}}%
\pgfpathlineto{\pgfqpoint{0.437447in}{0.928233in}}%
\pgfpathlineto{\pgfqpoint{0.450716in}{0.928135in}}%
\pgfpathlineto{\pgfqpoint{0.463985in}{0.933219in}}%
\pgfpathlineto{\pgfqpoint{0.465692in}{0.935760in}}%
\pgfpathlineto{\pgfqpoint{0.468620in}{0.948686in}}%
\pgfpathlineto{\pgfqpoint{0.469249in}{0.961613in}}%
\pgfpathlineto{\pgfqpoint{0.469034in}{0.974539in}}%
\pgfpathlineto{\pgfqpoint{0.467481in}{0.987466in}}%
\pgfpathlineto{\pgfqpoint{0.463985in}{0.994633in}}%
\pgfpathlineto{\pgfqpoint{0.450716in}{0.999517in}}%
\pgfpathlineto{\pgfqpoint{0.437447in}{0.999384in}}%
\pgfpathlineto{\pgfqpoint{0.424178in}{0.994853in}}%
\pgfpathlineto{\pgfqpoint{0.419672in}{0.987466in}}%
\pgfpathlineto{\pgfqpoint{0.417626in}{0.974539in}}%
\pgfpathlineto{\pgfqpoint{0.417320in}{0.961613in}}%
\pgfpathlineto{\pgfqpoint{0.418122in}{0.948686in}}%
\pgfpathlineto{\pgfqpoint{0.421750in}{0.935760in}}%
\pgfpathclose%
\pgfpathmoveto{\pgfqpoint{0.425483in}{0.948686in}}%
\pgfpathlineto{\pgfqpoint{0.424178in}{0.953750in}}%
\pgfpathlineto{\pgfqpoint{0.423120in}{0.961613in}}%
\pgfpathlineto{\pgfqpoint{0.423967in}{0.974539in}}%
\pgfpathlineto{\pgfqpoint{0.424178in}{0.975307in}}%
\pgfpathlineto{\pgfqpoint{0.432227in}{0.987466in}}%
\pgfpathlineto{\pgfqpoint{0.437447in}{0.990497in}}%
\pgfpathlineto{\pgfqpoint{0.450716in}{0.990135in}}%
\pgfpathlineto{\pgfqpoint{0.454767in}{0.987466in}}%
\pgfpathlineto{\pgfqpoint{0.461956in}{0.974539in}}%
\pgfpathlineto{\pgfqpoint{0.463084in}{0.961613in}}%
\pgfpathlineto{\pgfqpoint{0.460566in}{0.948686in}}%
\pgfpathlineto{\pgfqpoint{0.450716in}{0.937312in}}%
\pgfpathlineto{\pgfqpoint{0.437447in}{0.936837in}}%
\pgfpathclose%
\pgfusepath{fill}%
\end{pgfscope}%
\begin{pgfscope}%
\pgfpathrectangle{\pgfqpoint{0.211875in}{0.211875in}}{\pgfqpoint{1.313625in}{1.279725in}}%
\pgfusepath{clip}%
\pgfsetbuttcap%
\pgfsetroundjoin%
\definecolor{currentfill}{rgb}{0.947270,0.405591,0.279023}%
\pgfsetfillcolor{currentfill}%
\pgfsetlinewidth{0.000000pt}%
\definecolor{currentstroke}{rgb}{0.000000,0.000000,0.000000}%
\pgfsetstrokecolor{currentstroke}%
\pgfsetdash{}{0pt}%
\pgfpathmoveto{\pgfqpoint{0.543598in}{0.926883in}}%
\pgfpathlineto{\pgfqpoint{0.556867in}{0.925110in}}%
\pgfpathlineto{\pgfqpoint{0.570136in}{0.925036in}}%
\pgfpathlineto{\pgfqpoint{0.583405in}{0.927348in}}%
\pgfpathlineto{\pgfqpoint{0.588034in}{0.935760in}}%
\pgfpathlineto{\pgfqpoint{0.589026in}{0.948686in}}%
\pgfpathlineto{\pgfqpoint{0.589263in}{0.961613in}}%
\pgfpathlineto{\pgfqpoint{0.589247in}{0.974539in}}%
\pgfpathlineto{\pgfqpoint{0.588887in}{0.987466in}}%
\pgfpathlineto{\pgfqpoint{0.584964in}{1.000392in}}%
\pgfpathlineto{\pgfqpoint{0.583405in}{1.001191in}}%
\pgfpathlineto{\pgfqpoint{0.570136in}{1.002666in}}%
\pgfpathlineto{\pgfqpoint{0.556867in}{1.002532in}}%
\pgfpathlineto{\pgfqpoint{0.543598in}{1.001067in}}%
\pgfpathlineto{\pgfqpoint{0.541643in}{1.000392in}}%
\pgfpathlineto{\pgfqpoint{0.534016in}{0.987466in}}%
\pgfpathlineto{\pgfqpoint{0.533139in}{0.974539in}}%
\pgfpathlineto{\pgfqpoint{0.533034in}{0.961613in}}%
\pgfpathlineto{\pgfqpoint{0.533447in}{0.948686in}}%
\pgfpathlineto{\pgfqpoint{0.535237in}{0.935760in}}%
\pgfpathclose%
\pgfpathmoveto{\pgfqpoint{0.550377in}{0.935760in}}%
\pgfpathlineto{\pgfqpoint{0.543598in}{0.941139in}}%
\pgfpathlineto{\pgfqpoint{0.540487in}{0.948686in}}%
\pgfpathlineto{\pgfqpoint{0.538983in}{0.961613in}}%
\pgfpathlineto{\pgfqpoint{0.539644in}{0.974539in}}%
\pgfpathlineto{\pgfqpoint{0.543598in}{0.987085in}}%
\pgfpathlineto{\pgfqpoint{0.543918in}{0.987466in}}%
\pgfpathlineto{\pgfqpoint{0.556867in}{0.993987in}}%
\pgfpathlineto{\pgfqpoint{0.570136in}{0.993071in}}%
\pgfpathlineto{\pgfqpoint{0.577633in}{0.987466in}}%
\pgfpathlineto{\pgfqpoint{0.582635in}{0.974539in}}%
\pgfpathlineto{\pgfqpoint{0.583405in}{0.961694in}}%
\pgfpathlineto{\pgfqpoint{0.583409in}{0.961613in}}%
\pgfpathlineto{\pgfqpoint{0.583405in}{0.961575in}}%
\pgfpathlineto{\pgfqpoint{0.581613in}{0.948686in}}%
\pgfpathlineto{\pgfqpoint{0.572967in}{0.935760in}}%
\pgfpathlineto{\pgfqpoint{0.570136in}{0.934199in}}%
\pgfpathlineto{\pgfqpoint{0.556867in}{0.933292in}}%
\pgfpathclose%
\pgfusepath{fill}%
\end{pgfscope}%
\begin{pgfscope}%
\pgfpathrectangle{\pgfqpoint{0.211875in}{0.211875in}}{\pgfqpoint{1.313625in}{1.279725in}}%
\pgfusepath{clip}%
\pgfsetbuttcap%
\pgfsetroundjoin%
\definecolor{currentfill}{rgb}{0.947270,0.405591,0.279023}%
\pgfsetfillcolor{currentfill}%
\pgfsetlinewidth{0.000000pt}%
\definecolor{currentstroke}{rgb}{0.000000,0.000000,0.000000}%
\pgfsetstrokecolor{currentstroke}%
\pgfsetdash{}{0pt}%
\pgfpathmoveto{\pgfqpoint{0.371102in}{1.012267in}}%
\pgfpathlineto{\pgfqpoint{0.384371in}{1.010126in}}%
\pgfpathlineto{\pgfqpoint{0.397640in}{1.011709in}}%
\pgfpathlineto{\pgfqpoint{0.400664in}{1.013319in}}%
\pgfpathlineto{\pgfqpoint{0.407902in}{1.026245in}}%
\pgfpathlineto{\pgfqpoint{0.409264in}{1.039172in}}%
\pgfpathlineto{\pgfqpoint{0.409221in}{1.052098in}}%
\pgfpathlineto{\pgfqpoint{0.407658in}{1.065025in}}%
\pgfpathlineto{\pgfqpoint{0.398041in}{1.077952in}}%
\pgfpathlineto{\pgfqpoint{0.397640in}{1.078123in}}%
\pgfpathlineto{\pgfqpoint{0.384371in}{1.079632in}}%
\pgfpathlineto{\pgfqpoint{0.373202in}{1.077952in}}%
\pgfpathlineto{\pgfqpoint{0.371102in}{1.077481in}}%
\pgfpathlineto{\pgfqpoint{0.361355in}{1.065025in}}%
\pgfpathlineto{\pgfqpoint{0.359406in}{1.052098in}}%
\pgfpathlineto{\pgfqpoint{0.359341in}{1.039172in}}%
\pgfpathlineto{\pgfqpoint{0.361021in}{1.026245in}}%
\pgfpathlineto{\pgfqpoint{0.369215in}{1.013319in}}%
\pgfpathclose%
\pgfpathmoveto{\pgfqpoint{0.369631in}{1.026245in}}%
\pgfpathlineto{\pgfqpoint{0.366043in}{1.039172in}}%
\pgfpathlineto{\pgfqpoint{0.366223in}{1.052098in}}%
\pgfpathlineto{\pgfqpoint{0.370519in}{1.065025in}}%
\pgfpathlineto{\pgfqpoint{0.371102in}{1.065770in}}%
\pgfpathlineto{\pgfqpoint{0.384371in}{1.070762in}}%
\pgfpathlineto{\pgfqpoint{0.397640in}{1.065766in}}%
\pgfpathlineto{\pgfqpoint{0.398206in}{1.065025in}}%
\pgfpathlineto{\pgfqpoint{0.402312in}{1.052098in}}%
\pgfpathlineto{\pgfqpoint{0.402479in}{1.039172in}}%
\pgfpathlineto{\pgfqpoint{0.399059in}{1.026245in}}%
\pgfpathlineto{\pgfqpoint{0.397640in}{1.024189in}}%
\pgfpathlineto{\pgfqpoint{0.384371in}{1.018948in}}%
\pgfpathlineto{\pgfqpoint{0.371102in}{1.024173in}}%
\pgfpathclose%
\pgfusepath{fill}%
\end{pgfscope}%
\begin{pgfscope}%
\pgfpathrectangle{\pgfqpoint{0.211875in}{0.211875in}}{\pgfqpoint{1.313625in}{1.279725in}}%
\pgfusepath{clip}%
\pgfsetbuttcap%
\pgfsetroundjoin%
\definecolor{currentfill}{rgb}{0.947270,0.405591,0.279023}%
\pgfsetfillcolor{currentfill}%
\pgfsetlinewidth{0.000000pt}%
\definecolor{currentstroke}{rgb}{0.000000,0.000000,0.000000}%
\pgfsetstrokecolor{currentstroke}%
\pgfsetdash{}{0pt}%
\pgfpathmoveto{\pgfqpoint{0.490523in}{1.007959in}}%
\pgfpathlineto{\pgfqpoint{0.503792in}{1.007011in}}%
\pgfpathlineto{\pgfqpoint{0.517061in}{1.007835in}}%
\pgfpathlineto{\pgfqpoint{0.526100in}{1.013319in}}%
\pgfpathlineto{\pgfqpoint{0.528995in}{1.026245in}}%
\pgfpathlineto{\pgfqpoint{0.529553in}{1.039172in}}%
\pgfpathlineto{\pgfqpoint{0.529557in}{1.052098in}}%
\pgfpathlineto{\pgfqpoint{0.528982in}{1.065025in}}%
\pgfpathlineto{\pgfqpoint{0.525433in}{1.077952in}}%
\pgfpathlineto{\pgfqpoint{0.517061in}{1.082036in}}%
\pgfpathlineto{\pgfqpoint{0.503792in}{1.082754in}}%
\pgfpathlineto{\pgfqpoint{0.490523in}{1.081851in}}%
\pgfpathlineto{\pgfqpoint{0.480967in}{1.077952in}}%
\pgfpathlineto{\pgfqpoint{0.477254in}{1.072432in}}%
\pgfpathlineto{\pgfqpoint{0.475550in}{1.065025in}}%
\pgfpathlineto{\pgfqpoint{0.474694in}{1.052098in}}%
\pgfpathlineto{\pgfqpoint{0.474680in}{1.039172in}}%
\pgfpathlineto{\pgfqpoint{0.475457in}{1.026245in}}%
\pgfpathlineto{\pgfqpoint{0.477254in}{1.017790in}}%
\pgfpathlineto{\pgfqpoint{0.479689in}{1.013319in}}%
\pgfpathclose%
\pgfpathmoveto{\pgfqpoint{0.484170in}{1.026245in}}%
\pgfpathlineto{\pgfqpoint{0.481193in}{1.039172in}}%
\pgfpathlineto{\pgfqpoint{0.481334in}{1.052098in}}%
\pgfpathlineto{\pgfqpoint{0.484872in}{1.065025in}}%
\pgfpathlineto{\pgfqpoint{0.490523in}{1.071454in}}%
\pgfpathlineto{\pgfqpoint{0.503792in}{1.074340in}}%
\pgfpathlineto{\pgfqpoint{0.517061in}{1.069291in}}%
\pgfpathlineto{\pgfqpoint{0.519937in}{1.065025in}}%
\pgfpathlineto{\pgfqpoint{0.522934in}{1.052098in}}%
\pgfpathlineto{\pgfqpoint{0.523048in}{1.039172in}}%
\pgfpathlineto{\pgfqpoint{0.520529in}{1.026245in}}%
\pgfpathlineto{\pgfqpoint{0.517061in}{1.020562in}}%
\pgfpathlineto{\pgfqpoint{0.503792in}{1.015237in}}%
\pgfpathlineto{\pgfqpoint{0.490523in}{1.018272in}}%
\pgfpathclose%
\pgfusepath{fill}%
\end{pgfscope}%
\begin{pgfscope}%
\pgfpathrectangle{\pgfqpoint{0.211875in}{0.211875in}}{\pgfqpoint{1.313625in}{1.279725in}}%
\pgfusepath{clip}%
\pgfsetbuttcap%
\pgfsetroundjoin%
\definecolor{currentfill}{rgb}{0.947270,0.405591,0.279023}%
\pgfsetfillcolor{currentfill}%
\pgfsetlinewidth{0.000000pt}%
\definecolor{currentstroke}{rgb}{0.000000,0.000000,0.000000}%
\pgfsetstrokecolor{currentstroke}%
\pgfsetdash{}{0pt}%
\pgfpathmoveto{\pgfqpoint{0.251682in}{1.018912in}}%
\pgfpathlineto{\pgfqpoint{0.264951in}{1.013893in}}%
\pgfpathlineto{\pgfqpoint{0.278220in}{1.016280in}}%
\pgfpathlineto{\pgfqpoint{0.285971in}{1.026245in}}%
\pgfpathlineto{\pgfqpoint{0.288378in}{1.039172in}}%
\pgfpathlineto{\pgfqpoint{0.288274in}{1.052098in}}%
\pgfpathlineto{\pgfqpoint{0.285425in}{1.065025in}}%
\pgfpathlineto{\pgfqpoint{0.278220in}{1.073390in}}%
\pgfpathlineto{\pgfqpoint{0.264951in}{1.075634in}}%
\pgfpathlineto{\pgfqpoint{0.251682in}{1.070869in}}%
\pgfpathlineto{\pgfqpoint{0.247617in}{1.065025in}}%
\pgfpathlineto{\pgfqpoint{0.244702in}{1.052098in}}%
\pgfpathlineto{\pgfqpoint{0.244588in}{1.039172in}}%
\pgfpathlineto{\pgfqpoint{0.247050in}{1.026245in}}%
\pgfpathclose%
\pgfpathmoveto{\pgfqpoint{0.259531in}{1.026245in}}%
\pgfpathlineto{\pgfqpoint{0.251682in}{1.036580in}}%
\pgfpathlineto{\pgfqpoint{0.250983in}{1.039172in}}%
\pgfpathlineto{\pgfqpoint{0.251208in}{1.052098in}}%
\pgfpathlineto{\pgfqpoint{0.251682in}{1.053649in}}%
\pgfpathlineto{\pgfqpoint{0.261904in}{1.065025in}}%
\pgfpathlineto{\pgfqpoint{0.264951in}{1.066483in}}%
\pgfpathlineto{\pgfqpoint{0.269796in}{1.065025in}}%
\pgfpathlineto{\pgfqpoint{0.278220in}{1.059597in}}%
\pgfpathlineto{\pgfqpoint{0.281003in}{1.052098in}}%
\pgfpathlineto{\pgfqpoint{0.281238in}{1.039172in}}%
\pgfpathlineto{\pgfqpoint{0.278220in}{1.029927in}}%
\pgfpathlineto{\pgfqpoint{0.273460in}{1.026245in}}%
\pgfpathlineto{\pgfqpoint{0.264951in}{1.023388in}}%
\pgfpathclose%
\pgfusepath{fill}%
\end{pgfscope}%
\begin{pgfscope}%
\pgfpathrectangle{\pgfqpoint{0.211875in}{0.211875in}}{\pgfqpoint{1.313625in}{1.279725in}}%
\pgfusepath{clip}%
\pgfsetbuttcap%
\pgfsetroundjoin%
\definecolor{currentfill}{rgb}{0.947270,0.405591,0.279023}%
\pgfsetfillcolor{currentfill}%
\pgfsetlinewidth{0.000000pt}%
\definecolor{currentstroke}{rgb}{0.000000,0.000000,0.000000}%
\pgfsetstrokecolor{currentstroke}%
\pgfsetdash{}{0pt}%
\pgfpathmoveto{\pgfqpoint{0.437447in}{1.089730in}}%
\pgfpathlineto{\pgfqpoint{0.450716in}{1.089592in}}%
\pgfpathlineto{\pgfqpoint{0.457315in}{1.090878in}}%
\pgfpathlineto{\pgfqpoint{0.463985in}{1.093997in}}%
\pgfpathlineto{\pgfqpoint{0.468297in}{1.103805in}}%
\pgfpathlineto{\pgfqpoint{0.469481in}{1.116731in}}%
\pgfpathlineto{\pgfqpoint{0.469627in}{1.129658in}}%
\pgfpathlineto{\pgfqpoint{0.469015in}{1.142584in}}%
\pgfpathlineto{\pgfqpoint{0.465967in}{1.155511in}}%
\pgfpathlineto{\pgfqpoint{0.463985in}{1.158064in}}%
\pgfpathlineto{\pgfqpoint{0.450716in}{1.162204in}}%
\pgfpathlineto{\pgfqpoint{0.437447in}{1.162063in}}%
\pgfpathlineto{\pgfqpoint{0.424178in}{1.158093in}}%
\pgfpathlineto{\pgfqpoint{0.421752in}{1.155511in}}%
\pgfpathlineto{\pgfqpoint{0.417794in}{1.142584in}}%
\pgfpathlineto{\pgfqpoint{0.416956in}{1.129658in}}%
\pgfpathlineto{\pgfqpoint{0.417157in}{1.116731in}}%
\pgfpathlineto{\pgfqpoint{0.418760in}{1.103805in}}%
\pgfpathlineto{\pgfqpoint{0.424178in}{1.093965in}}%
\pgfpathlineto{\pgfqpoint{0.431871in}{1.090878in}}%
\pgfpathclose%
\pgfpathmoveto{\pgfqpoint{0.429545in}{1.103805in}}%
\pgfpathlineto{\pgfqpoint{0.424178in}{1.113278in}}%
\pgfpathlineto{\pgfqpoint{0.423323in}{1.116731in}}%
\pgfpathlineto{\pgfqpoint{0.422799in}{1.129658in}}%
\pgfpathlineto{\pgfqpoint{0.424178in}{1.138414in}}%
\pgfpathlineto{\pgfqpoint{0.425486in}{1.142584in}}%
\pgfpathlineto{\pgfqpoint{0.437447in}{1.153333in}}%
\pgfpathlineto{\pgfqpoint{0.450716in}{1.152932in}}%
\pgfpathlineto{\pgfqpoint{0.460622in}{1.142584in}}%
\pgfpathlineto{\pgfqpoint{0.463559in}{1.129658in}}%
\pgfpathlineto{\pgfqpoint{0.462862in}{1.116731in}}%
\pgfpathlineto{\pgfqpoint{0.457154in}{1.103805in}}%
\pgfpathlineto{\pgfqpoint{0.450716in}{1.099052in}}%
\pgfpathlineto{\pgfqpoint{0.437447in}{1.098703in}}%
\pgfpathclose%
\pgfusepath{fill}%
\end{pgfscope}%
\begin{pgfscope}%
\pgfpathrectangle{\pgfqpoint{0.211875in}{0.211875in}}{\pgfqpoint{1.313625in}{1.279725in}}%
\pgfusepath{clip}%
\pgfsetbuttcap%
\pgfsetroundjoin%
\definecolor{currentfill}{rgb}{0.947270,0.405591,0.279023}%
\pgfsetfillcolor{currentfill}%
\pgfsetlinewidth{0.000000pt}%
\definecolor{currentstroke}{rgb}{0.000000,0.000000,0.000000}%
\pgfsetstrokecolor{currentstroke}%
\pgfsetdash{}{0pt}%
\pgfpathmoveto{\pgfqpoint{0.543598in}{1.087978in}}%
\pgfpathlineto{\pgfqpoint{0.556867in}{1.086696in}}%
\pgfpathlineto{\pgfqpoint{0.570136in}{1.086530in}}%
\pgfpathlineto{\pgfqpoint{0.583405in}{1.087516in}}%
\pgfpathlineto{\pgfqpoint{0.587799in}{1.090878in}}%
\pgfpathlineto{\pgfqpoint{0.589408in}{1.103805in}}%
\pgfpathlineto{\pgfqpoint{0.589626in}{1.116731in}}%
\pgfpathlineto{\pgfqpoint{0.589653in}{1.129658in}}%
\pgfpathlineto{\pgfqpoint{0.589541in}{1.142584in}}%
\pgfpathlineto{\pgfqpoint{0.588980in}{1.155511in}}%
\pgfpathlineto{\pgfqpoint{0.583405in}{1.164312in}}%
\pgfpathlineto{\pgfqpoint{0.570136in}{1.165313in}}%
\pgfpathlineto{\pgfqpoint{0.556867in}{1.165145in}}%
\pgfpathlineto{\pgfqpoint{0.543598in}{1.163843in}}%
\pgfpathlineto{\pgfqpoint{0.534585in}{1.155511in}}%
\pgfpathlineto{\pgfqpoint{0.532982in}{1.142584in}}%
\pgfpathlineto{\pgfqpoint{0.532644in}{1.129658in}}%
\pgfpathlineto{\pgfqpoint{0.532725in}{1.116731in}}%
\pgfpathlineto{\pgfqpoint{0.533372in}{1.103805in}}%
\pgfpathlineto{\pgfqpoint{0.537595in}{1.090878in}}%
\pgfpathclose%
\pgfpathmoveto{\pgfqpoint{0.542353in}{1.103805in}}%
\pgfpathlineto{\pgfqpoint{0.539050in}{1.116731in}}%
\pgfpathlineto{\pgfqpoint{0.538637in}{1.129658in}}%
\pgfpathlineto{\pgfqpoint{0.540362in}{1.142584in}}%
\pgfpathlineto{\pgfqpoint{0.543598in}{1.149735in}}%
\pgfpathlineto{\pgfqpoint{0.552276in}{1.155511in}}%
\pgfpathlineto{\pgfqpoint{0.556867in}{1.157031in}}%
\pgfpathlineto{\pgfqpoint{0.570136in}{1.156219in}}%
\pgfpathlineto{\pgfqpoint{0.571634in}{1.155511in}}%
\pgfpathlineto{\pgfqpoint{0.581829in}{1.142584in}}%
\pgfpathlineto{\pgfqpoint{0.583405in}{1.133115in}}%
\pgfpathlineto{\pgfqpoint{0.583754in}{1.129658in}}%
\pgfpathlineto{\pgfqpoint{0.583405in}{1.117278in}}%
\pgfpathlineto{\pgfqpoint{0.583384in}{1.116731in}}%
\pgfpathlineto{\pgfqpoint{0.579430in}{1.103805in}}%
\pgfpathlineto{\pgfqpoint{0.570136in}{1.096027in}}%
\pgfpathlineto{\pgfqpoint{0.556867in}{1.095133in}}%
\pgfpathlineto{\pgfqpoint{0.543598in}{1.101835in}}%
\pgfpathclose%
\pgfusepath{fill}%
\end{pgfscope}%
\begin{pgfscope}%
\pgfpathrectangle{\pgfqpoint{0.211875in}{0.211875in}}{\pgfqpoint{1.313625in}{1.279725in}}%
\pgfusepath{clip}%
\pgfsetbuttcap%
\pgfsetroundjoin%
\definecolor{currentfill}{rgb}{0.947270,0.405591,0.279023}%
\pgfsetfillcolor{currentfill}%
\pgfsetlinewidth{0.000000pt}%
\definecolor{currentstroke}{rgb}{0.000000,0.000000,0.000000}%
\pgfsetstrokecolor{currentstroke}%
\pgfsetdash{}{0pt}%
\pgfpathmoveto{\pgfqpoint{0.223029in}{1.103805in}}%
\pgfpathlineto{\pgfqpoint{0.225144in}{1.106980in}}%
\pgfpathlineto{\pgfqpoint{0.227748in}{1.116731in}}%
\pgfpathlineto{\pgfqpoint{0.228217in}{1.129658in}}%
\pgfpathlineto{\pgfqpoint{0.226244in}{1.142584in}}%
\pgfpathlineto{\pgfqpoint{0.225144in}{1.145148in}}%
\pgfpathlineto{\pgfqpoint{0.211875in}{1.154638in}}%
\pgfpathlineto{\pgfqpoint{0.211875in}{1.144221in}}%
\pgfpathlineto{\pgfqpoint{0.213933in}{1.142584in}}%
\pgfpathlineto{\pgfqpoint{0.219908in}{1.129658in}}%
\pgfpathlineto{\pgfqpoint{0.218494in}{1.116731in}}%
\pgfpathlineto{\pgfqpoint{0.211875in}{1.108245in}}%
\pgfpathlineto{\pgfqpoint{0.211875in}{1.103805in}}%
\pgfpathlineto{\pgfqpoint{0.211875in}{1.097567in}}%
\pgfpathclose%
\pgfusepath{fill}%
\end{pgfscope}%
\begin{pgfscope}%
\pgfpathrectangle{\pgfqpoint{0.211875in}{0.211875in}}{\pgfqpoint{1.313625in}{1.279725in}}%
\pgfusepath{clip}%
\pgfsetbuttcap%
\pgfsetroundjoin%
\definecolor{currentfill}{rgb}{0.947270,0.405591,0.279023}%
\pgfsetfillcolor{currentfill}%
\pgfsetlinewidth{0.000000pt}%
\definecolor{currentstroke}{rgb}{0.000000,0.000000,0.000000}%
\pgfsetstrokecolor{currentstroke}%
\pgfsetdash{}{0pt}%
\pgfpathmoveto{\pgfqpoint{0.304758in}{1.103346in}}%
\pgfpathlineto{\pgfqpoint{0.318027in}{1.093812in}}%
\pgfpathlineto{\pgfqpoint{0.331295in}{1.093279in}}%
\pgfpathlineto{\pgfqpoint{0.344564in}{1.100101in}}%
\pgfpathlineto{\pgfqpoint{0.346496in}{1.103805in}}%
\pgfpathlineto{\pgfqpoint{0.348873in}{1.116731in}}%
\pgfpathlineto{\pgfqpoint{0.349164in}{1.129658in}}%
\pgfpathlineto{\pgfqpoint{0.347939in}{1.142584in}}%
\pgfpathlineto{\pgfqpoint{0.344564in}{1.151728in}}%
\pgfpathlineto{\pgfqpoint{0.340190in}{1.155511in}}%
\pgfpathlineto{\pgfqpoint{0.331295in}{1.158716in}}%
\pgfpathlineto{\pgfqpoint{0.318027in}{1.158232in}}%
\pgfpathlineto{\pgfqpoint{0.311784in}{1.155511in}}%
\pgfpathlineto{\pgfqpoint{0.304758in}{1.148000in}}%
\pgfpathlineto{\pgfqpoint{0.302924in}{1.142584in}}%
\pgfpathlineto{\pgfqpoint{0.301518in}{1.129658in}}%
\pgfpathlineto{\pgfqpoint{0.301855in}{1.116731in}}%
\pgfpathlineto{\pgfqpoint{0.304541in}{1.103805in}}%
\pgfpathclose%
\pgfpathmoveto{\pgfqpoint{0.317159in}{1.103805in}}%
\pgfpathlineto{\pgfqpoint{0.309456in}{1.116731in}}%
\pgfpathlineto{\pgfqpoint{0.308485in}{1.129658in}}%
\pgfpathlineto{\pgfqpoint{0.312531in}{1.142584in}}%
\pgfpathlineto{\pgfqpoint{0.318027in}{1.148206in}}%
\pgfpathlineto{\pgfqpoint{0.331295in}{1.148879in}}%
\pgfpathlineto{\pgfqpoint{0.338163in}{1.142584in}}%
\pgfpathlineto{\pgfqpoint{0.342358in}{1.129658in}}%
\pgfpathlineto{\pgfqpoint{0.341364in}{1.116731in}}%
\pgfpathlineto{\pgfqpoint{0.333188in}{1.103805in}}%
\pgfpathlineto{\pgfqpoint{0.331295in}{1.102581in}}%
\pgfpathlineto{\pgfqpoint{0.318027in}{1.103167in}}%
\pgfpathclose%
\pgfusepath{fill}%
\end{pgfscope}%
\begin{pgfscope}%
\pgfpathrectangle{\pgfqpoint{0.211875in}{0.211875in}}{\pgfqpoint{1.313625in}{1.279725in}}%
\pgfusepath{clip}%
\pgfsetbuttcap%
\pgfsetroundjoin%
\definecolor{currentfill}{rgb}{0.947270,0.405591,0.279023}%
\pgfsetfillcolor{currentfill}%
\pgfsetlinewidth{0.000000pt}%
\definecolor{currentstroke}{rgb}{0.000000,0.000000,0.000000}%
\pgfsetstrokecolor{currentstroke}%
\pgfsetdash{}{0pt}%
\pgfpathmoveto{\pgfqpoint{0.251682in}{1.180649in}}%
\pgfpathlineto{\pgfqpoint{0.264951in}{1.176333in}}%
\pgfpathlineto{\pgfqpoint{0.278220in}{1.178366in}}%
\pgfpathlineto{\pgfqpoint{0.281995in}{1.181364in}}%
\pgfpathlineto{\pgfqpoint{0.287541in}{1.194290in}}%
\pgfpathlineto{\pgfqpoint{0.288577in}{1.207217in}}%
\pgfpathlineto{\pgfqpoint{0.287446in}{1.220143in}}%
\pgfpathlineto{\pgfqpoint{0.281666in}{1.233070in}}%
\pgfpathlineto{\pgfqpoint{0.278220in}{1.235772in}}%
\pgfpathlineto{\pgfqpoint{0.264951in}{1.237839in}}%
\pgfpathlineto{\pgfqpoint{0.251682in}{1.233490in}}%
\pgfpathlineto{\pgfqpoint{0.251259in}{1.233070in}}%
\pgfpathlineto{\pgfqpoint{0.245550in}{1.220143in}}%
\pgfpathlineto{\pgfqpoint{0.244382in}{1.207217in}}%
\pgfpathlineto{\pgfqpoint{0.245464in}{1.194290in}}%
\pgfpathlineto{\pgfqpoint{0.250973in}{1.181364in}}%
\pgfpathclose%
\pgfpathmoveto{\pgfqpoint{0.253613in}{1.194290in}}%
\pgfpathlineto{\pgfqpoint{0.251682in}{1.199769in}}%
\pgfpathlineto{\pgfqpoint{0.250634in}{1.207217in}}%
\pgfpathlineto{\pgfqpoint{0.251682in}{1.214345in}}%
\pgfpathlineto{\pgfqpoint{0.253819in}{1.220143in}}%
\pgfpathlineto{\pgfqpoint{0.264951in}{1.228265in}}%
\pgfpathlineto{\pgfqpoint{0.278220in}{1.222399in}}%
\pgfpathlineto{\pgfqpoint{0.279467in}{1.220143in}}%
\pgfpathlineto{\pgfqpoint{0.281607in}{1.207217in}}%
\pgfpathlineto{\pgfqpoint{0.279571in}{1.194290in}}%
\pgfpathlineto{\pgfqpoint{0.278220in}{1.191809in}}%
\pgfpathlineto{\pgfqpoint{0.264951in}{1.185911in}}%
\pgfpathclose%
\pgfusepath{fill}%
\end{pgfscope}%
\begin{pgfscope}%
\pgfpathrectangle{\pgfqpoint{0.211875in}{0.211875in}}{\pgfqpoint{1.313625in}{1.279725in}}%
\pgfusepath{clip}%
\pgfsetbuttcap%
\pgfsetroundjoin%
\definecolor{currentfill}{rgb}{0.947270,0.405591,0.279023}%
\pgfsetfillcolor{currentfill}%
\pgfsetlinewidth{0.000000pt}%
\definecolor{currentstroke}{rgb}{0.000000,0.000000,0.000000}%
\pgfsetstrokecolor{currentstroke}%
\pgfsetdash{}{0pt}%
\pgfpathmoveto{\pgfqpoint{0.371102in}{1.174661in}}%
\pgfpathlineto{\pgfqpoint{0.384371in}{1.172463in}}%
\pgfpathlineto{\pgfqpoint{0.397640in}{1.174054in}}%
\pgfpathlineto{\pgfqpoint{0.405777in}{1.181364in}}%
\pgfpathlineto{\pgfqpoint{0.408820in}{1.194290in}}%
\pgfpathlineto{\pgfqpoint{0.409381in}{1.207217in}}%
\pgfpathlineto{\pgfqpoint{0.408734in}{1.220143in}}%
\pgfpathlineto{\pgfqpoint{0.405487in}{1.233070in}}%
\pgfpathlineto{\pgfqpoint{0.397640in}{1.240024in}}%
\pgfpathlineto{\pgfqpoint{0.384371in}{1.241682in}}%
\pgfpathlineto{\pgfqpoint{0.371102in}{1.239439in}}%
\pgfpathlineto{\pgfqpoint{0.363891in}{1.233070in}}%
\pgfpathlineto{\pgfqpoint{0.359999in}{1.220143in}}%
\pgfpathlineto{\pgfqpoint{0.359197in}{1.207217in}}%
\pgfpathlineto{\pgfqpoint{0.359914in}{1.194290in}}%
\pgfpathlineto{\pgfqpoint{0.363609in}{1.181364in}}%
\pgfpathclose%
\pgfpathmoveto{\pgfqpoint{0.382136in}{1.181364in}}%
\pgfpathlineto{\pgfqpoint{0.371102in}{1.186813in}}%
\pgfpathlineto{\pgfqpoint{0.367348in}{1.194290in}}%
\pgfpathlineto{\pgfqpoint{0.365748in}{1.207217in}}%
\pgfpathlineto{\pgfqpoint{0.367442in}{1.220143in}}%
\pgfpathlineto{\pgfqpoint{0.371102in}{1.227321in}}%
\pgfpathlineto{\pgfqpoint{0.382945in}{1.233070in}}%
\pgfpathlineto{\pgfqpoint{0.384371in}{1.233460in}}%
\pgfpathlineto{\pgfqpoint{0.385865in}{1.233070in}}%
\pgfpathlineto{\pgfqpoint{0.397640in}{1.227302in}}%
\pgfpathlineto{\pgfqpoint{0.401156in}{1.220143in}}%
\pgfpathlineto{\pgfqpoint{0.402757in}{1.207217in}}%
\pgfpathlineto{\pgfqpoint{0.401251in}{1.194290in}}%
\pgfpathlineto{\pgfqpoint{0.397640in}{1.186818in}}%
\pgfpathlineto{\pgfqpoint{0.386711in}{1.181364in}}%
\pgfpathlineto{\pgfqpoint{0.384371in}{1.180746in}}%
\pgfpathclose%
\pgfusepath{fill}%
\end{pgfscope}%
\begin{pgfscope}%
\pgfpathrectangle{\pgfqpoint{0.211875in}{0.211875in}}{\pgfqpoint{1.313625in}{1.279725in}}%
\pgfusepath{clip}%
\pgfsetbuttcap%
\pgfsetroundjoin%
\definecolor{currentfill}{rgb}{0.947270,0.405591,0.279023}%
\pgfsetfillcolor{currentfill}%
\pgfsetlinewidth{0.000000pt}%
\definecolor{currentstroke}{rgb}{0.000000,0.000000,0.000000}%
\pgfsetstrokecolor{currentstroke}%
\pgfsetdash{}{0pt}%
\pgfpathmoveto{\pgfqpoint{0.477254in}{1.179231in}}%
\pgfpathlineto{\pgfqpoint{0.490523in}{1.170125in}}%
\pgfpathlineto{\pgfqpoint{0.503792in}{1.169173in}}%
\pgfpathlineto{\pgfqpoint{0.517061in}{1.169929in}}%
\pgfpathlineto{\pgfqpoint{0.528284in}{1.181364in}}%
\pgfpathlineto{\pgfqpoint{0.529411in}{1.194290in}}%
\pgfpathlineto{\pgfqpoint{0.529607in}{1.207217in}}%
\pgfpathlineto{\pgfqpoint{0.529333in}{1.220143in}}%
\pgfpathlineto{\pgfqpoint{0.528023in}{1.233070in}}%
\pgfpathlineto{\pgfqpoint{0.517061in}{1.244082in}}%
\pgfpathlineto{\pgfqpoint{0.503792in}{1.244945in}}%
\pgfpathlineto{\pgfqpoint{0.490523in}{1.243952in}}%
\pgfpathlineto{\pgfqpoint{0.477254in}{1.234460in}}%
\pgfpathlineto{\pgfqpoint{0.476782in}{1.233070in}}%
\pgfpathlineto{\pgfqpoint{0.474986in}{1.220143in}}%
\pgfpathlineto{\pgfqpoint{0.474609in}{1.207217in}}%
\pgfpathlineto{\pgfqpoint{0.474915in}{1.194290in}}%
\pgfpathlineto{\pgfqpoint{0.476550in}{1.181364in}}%
\pgfpathclose%
\pgfpathmoveto{\pgfqpoint{0.488958in}{1.181364in}}%
\pgfpathlineto{\pgfqpoint{0.482259in}{1.194290in}}%
\pgfpathlineto{\pgfqpoint{0.480946in}{1.207217in}}%
\pgfpathlineto{\pgfqpoint{0.482354in}{1.220143in}}%
\pgfpathlineto{\pgfqpoint{0.489282in}{1.233070in}}%
\pgfpathlineto{\pgfqpoint{0.490523in}{1.234045in}}%
\pgfpathlineto{\pgfqpoint{0.503792in}{1.236675in}}%
\pgfpathlineto{\pgfqpoint{0.514918in}{1.233070in}}%
\pgfpathlineto{\pgfqpoint{0.517061in}{1.231668in}}%
\pgfpathlineto{\pgfqpoint{0.522072in}{1.220143in}}%
\pgfpathlineto{\pgfqpoint{0.523256in}{1.207217in}}%
\pgfpathlineto{\pgfqpoint{0.522160in}{1.194290in}}%
\pgfpathlineto{\pgfqpoint{0.517061in}{1.182361in}}%
\pgfpathlineto{\pgfqpoint{0.515566in}{1.181364in}}%
\pgfpathlineto{\pgfqpoint{0.503792in}{1.177505in}}%
\pgfpathlineto{\pgfqpoint{0.490523in}{1.180119in}}%
\pgfpathclose%
\pgfusepath{fill}%
\end{pgfscope}%
\begin{pgfscope}%
\pgfpathrectangle{\pgfqpoint{0.211875in}{0.211875in}}{\pgfqpoint{1.313625in}{1.279725in}}%
\pgfusepath{clip}%
\pgfsetbuttcap%
\pgfsetroundjoin%
\definecolor{currentfill}{rgb}{0.947270,0.405591,0.279023}%
\pgfsetfillcolor{currentfill}%
\pgfsetlinewidth{0.000000pt}%
\definecolor{currentstroke}{rgb}{0.000000,0.000000,0.000000}%
\pgfsetstrokecolor{currentstroke}%
\pgfsetdash{}{0pt}%
\pgfpathmoveto{\pgfqpoint{0.318027in}{1.256487in}}%
\pgfpathlineto{\pgfqpoint{0.331295in}{1.255997in}}%
\pgfpathlineto{\pgfqpoint{0.339313in}{1.258923in}}%
\pgfpathlineto{\pgfqpoint{0.344564in}{1.263573in}}%
\pgfpathlineto{\pgfqpoint{0.347558in}{1.271849in}}%
\pgfpathlineto{\pgfqpoint{0.348788in}{1.284776in}}%
\pgfpathlineto{\pgfqpoint{0.348401in}{1.297702in}}%
\pgfpathlineto{\pgfqpoint{0.345702in}{1.310629in}}%
\pgfpathlineto{\pgfqpoint{0.344564in}{1.312771in}}%
\pgfpathlineto{\pgfqpoint{0.331295in}{1.320243in}}%
\pgfpathlineto{\pgfqpoint{0.318027in}{1.319710in}}%
\pgfpathlineto{\pgfqpoint{0.305538in}{1.310629in}}%
\pgfpathlineto{\pgfqpoint{0.304758in}{1.308931in}}%
\pgfpathlineto{\pgfqpoint{0.302304in}{1.297702in}}%
\pgfpathlineto{\pgfqpoint{0.301877in}{1.284776in}}%
\pgfpathlineto{\pgfqpoint{0.303281in}{1.271849in}}%
\pgfpathlineto{\pgfqpoint{0.304758in}{1.267398in}}%
\pgfpathlineto{\pgfqpoint{0.312504in}{1.258923in}}%
\pgfpathclose%
\pgfpathmoveto{\pgfqpoint{0.313045in}{1.271849in}}%
\pgfpathlineto{\pgfqpoint{0.309017in}{1.284776in}}%
\pgfpathlineto{\pgfqpoint{0.310134in}{1.297702in}}%
\pgfpathlineto{\pgfqpoint{0.318027in}{1.310328in}}%
\pgfpathlineto{\pgfqpoint{0.322393in}{1.310629in}}%
\pgfpathlineto{\pgfqpoint{0.331295in}{1.311020in}}%
\pgfpathlineto{\pgfqpoint{0.331908in}{1.310629in}}%
\pgfpathlineto{\pgfqpoint{0.340622in}{1.297702in}}%
\pgfpathlineto{\pgfqpoint{0.341777in}{1.284776in}}%
\pgfpathlineto{\pgfqpoint{0.337590in}{1.271849in}}%
\pgfpathlineto{\pgfqpoint{0.331295in}{1.266009in}}%
\pgfpathlineto{\pgfqpoint{0.318027in}{1.266692in}}%
\pgfpathclose%
\pgfusepath{fill}%
\end{pgfscope}%
\begin{pgfscope}%
\pgfpathrectangle{\pgfqpoint{0.211875in}{0.211875in}}{\pgfqpoint{1.313625in}{1.279725in}}%
\pgfusepath{clip}%
\pgfsetbuttcap%
\pgfsetroundjoin%
\definecolor{currentfill}{rgb}{0.947270,0.405591,0.279023}%
\pgfsetfillcolor{currentfill}%
\pgfsetlinewidth{0.000000pt}%
\definecolor{currentstroke}{rgb}{0.000000,0.000000,0.000000}%
\pgfsetstrokecolor{currentstroke}%
\pgfsetdash{}{0pt}%
\pgfpathmoveto{\pgfqpoint{0.424178in}{1.256867in}}%
\pgfpathlineto{\pgfqpoint{0.437447in}{1.252619in}}%
\pgfpathlineto{\pgfqpoint{0.450716in}{1.252494in}}%
\pgfpathlineto{\pgfqpoint{0.463985in}{1.257073in}}%
\pgfpathlineto{\pgfqpoint{0.465392in}{1.258923in}}%
\pgfpathlineto{\pgfqpoint{0.468635in}{1.271849in}}%
\pgfpathlineto{\pgfqpoint{0.469254in}{1.284776in}}%
\pgfpathlineto{\pgfqpoint{0.469018in}{1.297702in}}%
\pgfpathlineto{\pgfqpoint{0.467525in}{1.310629in}}%
\pgfpathlineto{\pgfqpoint{0.463985in}{1.318519in}}%
\pgfpathlineto{\pgfqpoint{0.452892in}{1.323555in}}%
\pgfpathlineto{\pgfqpoint{0.450716in}{1.323972in}}%
\pgfpathlineto{\pgfqpoint{0.437447in}{1.323877in}}%
\pgfpathlineto{\pgfqpoint{0.435862in}{1.323555in}}%
\pgfpathlineto{\pgfqpoint{0.424178in}{1.318972in}}%
\pgfpathlineto{\pgfqpoint{0.419508in}{1.310629in}}%
\pgfpathlineto{\pgfqpoint{0.417612in}{1.297702in}}%
\pgfpathlineto{\pgfqpoint{0.417322in}{1.284776in}}%
\pgfpathlineto{\pgfqpoint{0.418163in}{1.271849in}}%
\pgfpathlineto{\pgfqpoint{0.422279in}{1.258923in}}%
\pgfpathclose%
\pgfpathmoveto{\pgfqpoint{0.426064in}{1.271849in}}%
\pgfpathlineto{\pgfqpoint{0.424178in}{1.278094in}}%
\pgfpathlineto{\pgfqpoint{0.423164in}{1.284776in}}%
\pgfpathlineto{\pgfqpoint{0.423784in}{1.297702in}}%
\pgfpathlineto{\pgfqpoint{0.424178in}{1.299257in}}%
\pgfpathlineto{\pgfqpoint{0.430786in}{1.310629in}}%
\pgfpathlineto{\pgfqpoint{0.437447in}{1.314882in}}%
\pgfpathlineto{\pgfqpoint{0.450716in}{1.314493in}}%
\pgfpathlineto{\pgfqpoint{0.456011in}{1.310629in}}%
\pgfpathlineto{\pgfqpoint{0.462192in}{1.297702in}}%
\pgfpathlineto{\pgfqpoint{0.463029in}{1.284776in}}%
\pgfpathlineto{\pgfqpoint{0.460095in}{1.271849in}}%
\pgfpathlineto{\pgfqpoint{0.450716in}{1.261929in}}%
\pgfpathlineto{\pgfqpoint{0.437447in}{1.261498in}}%
\pgfpathclose%
\pgfusepath{fill}%
\end{pgfscope}%
\begin{pgfscope}%
\pgfpathrectangle{\pgfqpoint{0.211875in}{0.211875in}}{\pgfqpoint{1.313625in}{1.279725in}}%
\pgfusepath{clip}%
\pgfsetbuttcap%
\pgfsetroundjoin%
\definecolor{currentfill}{rgb}{0.947270,0.405591,0.279023}%
\pgfsetfillcolor{currentfill}%
\pgfsetlinewidth{0.000000pt}%
\definecolor{currentstroke}{rgb}{0.000000,0.000000,0.000000}%
\pgfsetstrokecolor{currentstroke}%
\pgfsetdash{}{0pt}%
\pgfpathmoveto{\pgfqpoint{0.543598in}{1.250992in}}%
\pgfpathlineto{\pgfqpoint{0.556867in}{1.249510in}}%
\pgfpathlineto{\pgfqpoint{0.570136in}{1.249374in}}%
\pgfpathlineto{\pgfqpoint{0.583405in}{1.250868in}}%
\pgfpathlineto{\pgfqpoint{0.588386in}{1.258923in}}%
\pgfpathlineto{\pgfqpoint{0.589159in}{1.271849in}}%
\pgfpathlineto{\pgfqpoint{0.589283in}{1.284776in}}%
\pgfpathlineto{\pgfqpoint{0.589168in}{1.297702in}}%
\pgfpathlineto{\pgfqpoint{0.588651in}{1.310629in}}%
\pgfpathlineto{\pgfqpoint{0.584990in}{1.323555in}}%
\pgfpathlineto{\pgfqpoint{0.583405in}{1.324733in}}%
\pgfpathlineto{\pgfqpoint{0.570136in}{1.326978in}}%
\pgfpathlineto{\pgfqpoint{0.556867in}{1.326907in}}%
\pgfpathlineto{\pgfqpoint{0.543598in}{1.325185in}}%
\pgfpathlineto{\pgfqpoint{0.540158in}{1.323555in}}%
\pgfpathlineto{\pgfqpoint{0.534132in}{1.310629in}}%
\pgfpathlineto{\pgfqpoint{0.533191in}{1.297702in}}%
\pgfpathlineto{\pgfqpoint{0.533020in}{1.284776in}}%
\pgfpathlineto{\pgfqpoint{0.533365in}{1.271849in}}%
\pgfpathlineto{\pgfqpoint{0.535152in}{1.258923in}}%
\pgfpathclose%
\pgfpathmoveto{\pgfqpoint{0.553159in}{1.258923in}}%
\pgfpathlineto{\pgfqpoint{0.543598in}{1.265411in}}%
\pgfpathlineto{\pgfqpoint{0.540734in}{1.271849in}}%
\pgfpathlineto{\pgfqpoint{0.539011in}{1.284776in}}%
\pgfpathlineto{\pgfqpoint{0.539520in}{1.297702in}}%
\pgfpathlineto{\pgfqpoint{0.543135in}{1.310629in}}%
\pgfpathlineto{\pgfqpoint{0.543598in}{1.311350in}}%
\pgfpathlineto{\pgfqpoint{0.556867in}{1.318438in}}%
\pgfpathlineto{\pgfqpoint{0.570136in}{1.317456in}}%
\pgfpathlineto{\pgfqpoint{0.578392in}{1.310629in}}%
\pgfpathlineto{\pgfqpoint{0.582769in}{1.297702in}}%
\pgfpathlineto{\pgfqpoint{0.583379in}{1.284776in}}%
\pgfpathlineto{\pgfqpoint{0.581339in}{1.271849in}}%
\pgfpathlineto{\pgfqpoint{0.570937in}{1.258923in}}%
\pgfpathlineto{\pgfqpoint{0.570136in}{1.258539in}}%
\pgfpathlineto{\pgfqpoint{0.556867in}{1.257681in}}%
\pgfpathclose%
\pgfusepath{fill}%
\end{pgfscope}%
\begin{pgfscope}%
\pgfpathrectangle{\pgfqpoint{0.211875in}{0.211875in}}{\pgfqpoint{1.313625in}{1.279725in}}%
\pgfusepath{clip}%
\pgfsetbuttcap%
\pgfsetroundjoin%
\definecolor{currentfill}{rgb}{0.947270,0.405591,0.279023}%
\pgfsetfillcolor{currentfill}%
\pgfsetlinewidth{0.000000pt}%
\definecolor{currentstroke}{rgb}{0.000000,0.000000,0.000000}%
\pgfsetstrokecolor{currentstroke}%
\pgfsetdash{}{0pt}%
\pgfpathmoveto{\pgfqpoint{0.225144in}{1.270148in}}%
\pgfpathlineto{\pgfqpoint{0.225861in}{1.271849in}}%
\pgfpathlineto{\pgfqpoint{0.227833in}{1.284776in}}%
\pgfpathlineto{\pgfqpoint{0.227264in}{1.297702in}}%
\pgfpathlineto{\pgfqpoint{0.225144in}{1.305455in}}%
\pgfpathlineto{\pgfqpoint{0.221609in}{1.310629in}}%
\pgfpathlineto{\pgfqpoint{0.211875in}{1.316012in}}%
\pgfpathlineto{\pgfqpoint{0.211875in}{1.310629in}}%
\pgfpathlineto{\pgfqpoint{0.211875in}{1.305003in}}%
\pgfpathlineto{\pgfqpoint{0.217658in}{1.297702in}}%
\pgfpathlineto{\pgfqpoint{0.219258in}{1.284776in}}%
\pgfpathlineto{\pgfqpoint{0.213301in}{1.271849in}}%
\pgfpathlineto{\pgfqpoint{0.211875in}{1.270701in}}%
\pgfpathlineto{\pgfqpoint{0.211875in}{1.260178in}}%
\pgfpathclose%
\pgfusepath{fill}%
\end{pgfscope}%
\begin{pgfscope}%
\pgfpathrectangle{\pgfqpoint{0.211875in}{0.211875in}}{\pgfqpoint{1.313625in}{1.279725in}}%
\pgfusepath{clip}%
\pgfsetbuttcap%
\pgfsetroundjoin%
\definecolor{currentfill}{rgb}{0.947270,0.405591,0.279023}%
\pgfsetfillcolor{currentfill}%
\pgfsetlinewidth{0.000000pt}%
\definecolor{currentstroke}{rgb}{0.000000,0.000000,0.000000}%
\pgfsetstrokecolor{currentstroke}%
\pgfsetdash{}{0pt}%
\pgfpathmoveto{\pgfqpoint{0.384371in}{1.335563in}}%
\pgfpathlineto{\pgfqpoint{0.392093in}{1.336482in}}%
\pgfpathlineto{\pgfqpoint{0.397640in}{1.337527in}}%
\pgfpathlineto{\pgfqpoint{0.406585in}{1.349408in}}%
\pgfpathlineto{\pgfqpoint{0.408356in}{1.362335in}}%
\pgfpathlineto{\pgfqpoint{0.408337in}{1.375261in}}%
\pgfpathlineto{\pgfqpoint{0.406591in}{1.388188in}}%
\pgfpathlineto{\pgfqpoint{0.397640in}{1.400983in}}%
\pgfpathlineto{\pgfqpoint{0.397098in}{1.401114in}}%
\pgfpathlineto{\pgfqpoint{0.384371in}{1.403007in}}%
\pgfpathlineto{\pgfqpoint{0.373649in}{1.401114in}}%
\pgfpathlineto{\pgfqpoint{0.371102in}{1.400423in}}%
\pgfpathlineto{\pgfqpoint{0.362300in}{1.388188in}}%
\pgfpathlineto{\pgfqpoint{0.360257in}{1.375261in}}%
\pgfpathlineto{\pgfqpoint{0.360258in}{1.362335in}}%
\pgfpathlineto{\pgfqpoint{0.362393in}{1.349408in}}%
\pgfpathlineto{\pgfqpoint{0.371102in}{1.338133in}}%
\pgfpathlineto{\pgfqpoint{0.378341in}{1.336482in}}%
\pgfpathclose%
\pgfpathmoveto{\pgfqpoint{0.372219in}{1.349408in}}%
\pgfpathlineto{\pgfqpoint{0.371102in}{1.350370in}}%
\pgfpathlineto{\pgfqpoint{0.367072in}{1.362335in}}%
\pgfpathlineto{\pgfqpoint{0.366962in}{1.375261in}}%
\pgfpathlineto{\pgfqpoint{0.370926in}{1.388188in}}%
\pgfpathlineto{\pgfqpoint{0.371102in}{1.388433in}}%
\pgfpathlineto{\pgfqpoint{0.384371in}{1.393994in}}%
\pgfpathlineto{\pgfqpoint{0.397640in}{1.388317in}}%
\pgfpathlineto{\pgfqpoint{0.397731in}{1.388188in}}%
\pgfpathlineto{\pgfqpoint{0.401549in}{1.375261in}}%
\pgfpathlineto{\pgfqpoint{0.401451in}{1.362335in}}%
\pgfpathlineto{\pgfqpoint{0.397640in}{1.350512in}}%
\pgfpathlineto{\pgfqpoint{0.396438in}{1.349408in}}%
\pgfpathlineto{\pgfqpoint{0.384371in}{1.344636in}}%
\pgfpathclose%
\pgfusepath{fill}%
\end{pgfscope}%
\begin{pgfscope}%
\pgfpathrectangle{\pgfqpoint{0.211875in}{0.211875in}}{\pgfqpoint{1.313625in}{1.279725in}}%
\pgfusepath{clip}%
\pgfsetbuttcap%
\pgfsetroundjoin%
\definecolor{currentfill}{rgb}{0.947270,0.405591,0.279023}%
\pgfsetfillcolor{currentfill}%
\pgfsetlinewidth{0.000000pt}%
\definecolor{currentstroke}{rgb}{0.000000,0.000000,0.000000}%
\pgfsetstrokecolor{currentstroke}%
\pgfsetdash{}{0pt}%
\pgfpathmoveto{\pgfqpoint{0.490523in}{1.333497in}}%
\pgfpathlineto{\pgfqpoint{0.503792in}{1.332426in}}%
\pgfpathlineto{\pgfqpoint{0.517061in}{1.333521in}}%
\pgfpathlineto{\pgfqpoint{0.523025in}{1.336482in}}%
\pgfpathlineto{\pgfqpoint{0.527947in}{1.349408in}}%
\pgfpathlineto{\pgfqpoint{0.528726in}{1.362335in}}%
\pgfpathlineto{\pgfqpoint{0.528666in}{1.375261in}}%
\pgfpathlineto{\pgfqpoint{0.527746in}{1.388188in}}%
\pgfpathlineto{\pgfqpoint{0.523076in}{1.401114in}}%
\pgfpathlineto{\pgfqpoint{0.517061in}{1.404707in}}%
\pgfpathlineto{\pgfqpoint{0.503792in}{1.406095in}}%
\pgfpathlineto{\pgfqpoint{0.490523in}{1.404904in}}%
\pgfpathlineto{\pgfqpoint{0.482758in}{1.401114in}}%
\pgfpathlineto{\pgfqpoint{0.477254in}{1.391354in}}%
\pgfpathlineto{\pgfqpoint{0.476560in}{1.388188in}}%
\pgfpathlineto{\pgfqpoint{0.475471in}{1.375261in}}%
\pgfpathlineto{\pgfqpoint{0.475433in}{1.362335in}}%
\pgfpathlineto{\pgfqpoint{0.476456in}{1.349408in}}%
\pgfpathlineto{\pgfqpoint{0.477254in}{1.345828in}}%
\pgfpathlineto{\pgfqpoint{0.483316in}{1.336482in}}%
\pgfpathclose%
\pgfpathmoveto{\pgfqpoint{0.485958in}{1.349408in}}%
\pgfpathlineto{\pgfqpoint{0.482233in}{1.362335in}}%
\pgfpathlineto{\pgfqpoint{0.482163in}{1.375261in}}%
\pgfpathlineto{\pgfqpoint{0.485533in}{1.388188in}}%
\pgfpathlineto{\pgfqpoint{0.490523in}{1.394372in}}%
\pgfpathlineto{\pgfqpoint{0.503792in}{1.397663in}}%
\pgfpathlineto{\pgfqpoint{0.517061in}{1.391746in}}%
\pgfpathlineto{\pgfqpoint{0.519264in}{1.388188in}}%
\pgfpathlineto{\pgfqpoint{0.522158in}{1.375261in}}%
\pgfpathlineto{\pgfqpoint{0.522107in}{1.362335in}}%
\pgfpathlineto{\pgfqpoint{0.518922in}{1.349408in}}%
\pgfpathlineto{\pgfqpoint{0.517061in}{1.346607in}}%
\pgfpathlineto{\pgfqpoint{0.503792in}{1.341032in}}%
\pgfpathlineto{\pgfqpoint{0.490523in}{1.344149in}}%
\pgfpathclose%
\pgfusepath{fill}%
\end{pgfscope}%
\begin{pgfscope}%
\pgfpathrectangle{\pgfqpoint{0.211875in}{0.211875in}}{\pgfqpoint{1.313625in}{1.279725in}}%
\pgfusepath{clip}%
\pgfsetbuttcap%
\pgfsetroundjoin%
\definecolor{currentfill}{rgb}{0.947270,0.405591,0.279023}%
\pgfsetfillcolor{currentfill}%
\pgfsetlinewidth{0.000000pt}%
\definecolor{currentstroke}{rgb}{0.000000,0.000000,0.000000}%
\pgfsetstrokecolor{currentstroke}%
\pgfsetdash{}{0pt}%
\pgfpathmoveto{\pgfqpoint{0.596674in}{1.332818in}}%
\pgfpathlineto{\pgfqpoint{0.609943in}{1.330197in}}%
\pgfpathlineto{\pgfqpoint{0.623212in}{1.329797in}}%
\pgfpathlineto{\pgfqpoint{0.636481in}{1.329821in}}%
\pgfpathlineto{\pgfqpoint{0.648150in}{1.336482in}}%
\pgfpathlineto{\pgfqpoint{0.648552in}{1.349408in}}%
\pgfpathlineto{\pgfqpoint{0.648578in}{1.362335in}}%
\pgfpathlineto{\pgfqpoint{0.648487in}{1.375261in}}%
\pgfpathlineto{\pgfqpoint{0.648195in}{1.388188in}}%
\pgfpathlineto{\pgfqpoint{0.646881in}{1.401114in}}%
\pgfpathlineto{\pgfqpoint{0.636481in}{1.408216in}}%
\pgfpathlineto{\pgfqpoint{0.623212in}{1.408677in}}%
\pgfpathlineto{\pgfqpoint{0.609943in}{1.408187in}}%
\pgfpathlineto{\pgfqpoint{0.596674in}{1.404420in}}%
\pgfpathlineto{\pgfqpoint{0.594310in}{1.401114in}}%
\pgfpathlineto{\pgfqpoint{0.591949in}{1.388188in}}%
\pgfpathlineto{\pgfqpoint{0.591437in}{1.375261in}}%
\pgfpathlineto{\pgfqpoint{0.591367in}{1.362335in}}%
\pgfpathlineto{\pgfqpoint{0.591689in}{1.349408in}}%
\pgfpathlineto{\pgfqpoint{0.593672in}{1.336482in}}%
\pgfpathclose%
\pgfpathmoveto{\pgfqpoint{0.600520in}{1.349408in}}%
\pgfpathlineto{\pgfqpoint{0.597439in}{1.362335in}}%
\pgfpathlineto{\pgfqpoint{0.597406in}{1.375261in}}%
\pgfpathlineto{\pgfqpoint{0.600266in}{1.388188in}}%
\pgfpathlineto{\pgfqpoint{0.609943in}{1.398871in}}%
\pgfpathlineto{\pgfqpoint{0.623212in}{1.400659in}}%
\pgfpathlineto{\pgfqpoint{0.636481in}{1.394655in}}%
\pgfpathlineto{\pgfqpoint{0.640007in}{1.388188in}}%
\pgfpathlineto{\pgfqpoint{0.642198in}{1.375261in}}%
\pgfpathlineto{\pgfqpoint{0.642182in}{1.362335in}}%
\pgfpathlineto{\pgfqpoint{0.639844in}{1.349408in}}%
\pgfpathlineto{\pgfqpoint{0.636481in}{1.343640in}}%
\pgfpathlineto{\pgfqpoint{0.623212in}{1.338083in}}%
\pgfpathlineto{\pgfqpoint{0.609943in}{1.339753in}}%
\pgfpathclose%
\pgfusepath{fill}%
\end{pgfscope}%
\begin{pgfscope}%
\pgfpathrectangle{\pgfqpoint{0.211875in}{0.211875in}}{\pgfqpoint{1.313625in}{1.279725in}}%
\pgfusepath{clip}%
\pgfsetbuttcap%
\pgfsetroundjoin%
\definecolor{currentfill}{rgb}{0.947270,0.405591,0.279023}%
\pgfsetfillcolor{currentfill}%
\pgfsetlinewidth{0.000000pt}%
\definecolor{currentstroke}{rgb}{0.000000,0.000000,0.000000}%
\pgfsetstrokecolor{currentstroke}%
\pgfsetdash{}{0pt}%
\pgfpathmoveto{\pgfqpoint{0.251682in}{1.344913in}}%
\pgfpathlineto{\pgfqpoint{0.264951in}{1.339722in}}%
\pgfpathlineto{\pgfqpoint{0.278220in}{1.342253in}}%
\pgfpathlineto{\pgfqpoint{0.284304in}{1.349408in}}%
\pgfpathlineto{\pgfqpoint{0.287366in}{1.362335in}}%
\pgfpathlineto{\pgfqpoint{0.287401in}{1.375261in}}%
\pgfpathlineto{\pgfqpoint{0.284581in}{1.388188in}}%
\pgfpathlineto{\pgfqpoint{0.278220in}{1.396260in}}%
\pgfpathlineto{\pgfqpoint{0.264951in}{1.399002in}}%
\pgfpathlineto{\pgfqpoint{0.251682in}{1.393505in}}%
\pgfpathlineto{\pgfqpoint{0.248276in}{1.388188in}}%
\pgfpathlineto{\pgfqpoint{0.245463in}{1.375261in}}%
\pgfpathlineto{\pgfqpoint{0.245514in}{1.362335in}}%
\pgfpathlineto{\pgfqpoint{0.248597in}{1.349408in}}%
\pgfpathclose%
\pgfpathmoveto{\pgfqpoint{0.264010in}{1.349408in}}%
\pgfpathlineto{\pgfqpoint{0.252420in}{1.362335in}}%
\pgfpathlineto{\pgfqpoint{0.252079in}{1.375261in}}%
\pgfpathlineto{\pgfqpoint{0.262255in}{1.388188in}}%
\pgfpathlineto{\pgfqpoint{0.264951in}{1.389594in}}%
\pgfpathlineto{\pgfqpoint{0.269185in}{1.388188in}}%
\pgfpathlineto{\pgfqpoint{0.278220in}{1.381366in}}%
\pgfpathlineto{\pgfqpoint{0.280258in}{1.375261in}}%
\pgfpathlineto{\pgfqpoint{0.280100in}{1.362335in}}%
\pgfpathlineto{\pgfqpoint{0.278220in}{1.357165in}}%
\pgfpathlineto{\pgfqpoint{0.266447in}{1.349408in}}%
\pgfpathlineto{\pgfqpoint{0.264951in}{1.348953in}}%
\pgfpathclose%
\pgfusepath{fill}%
\end{pgfscope}%
\begin{pgfscope}%
\pgfpathrectangle{\pgfqpoint{0.211875in}{0.211875in}}{\pgfqpoint{1.313625in}{1.279725in}}%
\pgfusepath{clip}%
\pgfsetbuttcap%
\pgfsetroundjoin%
\definecolor{currentfill}{rgb}{0.947270,0.405591,0.279023}%
\pgfsetfillcolor{currentfill}%
\pgfsetlinewidth{0.000000pt}%
\definecolor{currentstroke}{rgb}{0.000000,0.000000,0.000000}%
\pgfsetstrokecolor{currentstroke}%
\pgfsetdash{}{0pt}%
\pgfpathmoveto{\pgfqpoint{0.556867in}{1.413193in}}%
\pgfpathlineto{\pgfqpoint{0.570136in}{1.413223in}}%
\pgfpathlineto{\pgfqpoint{0.575548in}{1.414041in}}%
\pgfpathlineto{\pgfqpoint{0.583405in}{1.417324in}}%
\pgfpathlineto{\pgfqpoint{0.587148in}{1.426967in}}%
\pgfpathlineto{\pgfqpoint{0.588038in}{1.439894in}}%
\pgfpathlineto{\pgfqpoint{0.588097in}{1.452820in}}%
\pgfpathlineto{\pgfqpoint{0.587562in}{1.465747in}}%
\pgfpathlineto{\pgfqpoint{0.585275in}{1.478673in}}%
\pgfpathlineto{\pgfqpoint{0.583405in}{1.481994in}}%
\pgfpathlineto{\pgfqpoint{0.570136in}{1.487261in}}%
\pgfpathlineto{\pgfqpoint{0.556867in}{1.487409in}}%
\pgfpathlineto{\pgfqpoint{0.543598in}{1.484582in}}%
\pgfpathlineto{\pgfqpoint{0.537948in}{1.478673in}}%
\pgfpathlineto{\pgfqpoint{0.534927in}{1.465747in}}%
\pgfpathlineto{\pgfqpoint{0.534219in}{1.452820in}}%
\pgfpathlineto{\pgfqpoint{0.534362in}{1.439894in}}%
\pgfpathlineto{\pgfqpoint{0.535741in}{1.426967in}}%
\pgfpathlineto{\pgfqpoint{0.543598in}{1.415601in}}%
\pgfpathlineto{\pgfqpoint{0.550433in}{1.414041in}}%
\pgfpathclose%
\pgfpathmoveto{\pgfqpoint{0.546927in}{1.426967in}}%
\pgfpathlineto{\pgfqpoint{0.543598in}{1.431028in}}%
\pgfpathlineto{\pgfqpoint{0.540861in}{1.439894in}}%
\pgfpathlineto{\pgfqpoint{0.540168in}{1.452820in}}%
\pgfpathlineto{\pgfqpoint{0.541977in}{1.465747in}}%
\pgfpathlineto{\pgfqpoint{0.543598in}{1.469613in}}%
\pgfpathlineto{\pgfqpoint{0.555240in}{1.478673in}}%
\pgfpathlineto{\pgfqpoint{0.556867in}{1.479285in}}%
\pgfpathlineto{\pgfqpoint{0.564521in}{1.478673in}}%
\pgfpathlineto{\pgfqpoint{0.570136in}{1.478013in}}%
\pgfpathlineto{\pgfqpoint{0.579655in}{1.465747in}}%
\pgfpathlineto{\pgfqpoint{0.581863in}{1.452820in}}%
\pgfpathlineto{\pgfqpoint{0.581040in}{1.439894in}}%
\pgfpathlineto{\pgfqpoint{0.575357in}{1.426967in}}%
\pgfpathlineto{\pgfqpoint{0.570136in}{1.423017in}}%
\pgfpathlineto{\pgfqpoint{0.556867in}{1.421907in}}%
\pgfpathclose%
\pgfusepath{fill}%
\end{pgfscope}%
\begin{pgfscope}%
\pgfpathrectangle{\pgfqpoint{0.211875in}{0.211875in}}{\pgfqpoint{1.313625in}{1.279725in}}%
\pgfusepath{clip}%
\pgfsetbuttcap%
\pgfsetroundjoin%
\definecolor{currentfill}{rgb}{0.947270,0.405591,0.279023}%
\pgfsetfillcolor{currentfill}%
\pgfsetlinewidth{0.000000pt}%
\definecolor{currentstroke}{rgb}{0.000000,0.000000,0.000000}%
\pgfsetstrokecolor{currentstroke}%
\pgfsetdash{}{0pt}%
\pgfpathmoveto{\pgfqpoint{0.217115in}{1.426967in}}%
\pgfpathlineto{\pgfqpoint{0.225144in}{1.437422in}}%
\pgfpathlineto{\pgfqpoint{0.225860in}{1.439894in}}%
\pgfpathlineto{\pgfqpoint{0.226653in}{1.452820in}}%
\pgfpathlineto{\pgfqpoint{0.225144in}{1.462851in}}%
\pgfpathlineto{\pgfqpoint{0.224254in}{1.465747in}}%
\pgfpathlineto{\pgfqpoint{0.211875in}{1.476517in}}%
\pgfpathlineto{\pgfqpoint{0.211875in}{1.465747in}}%
\pgfpathlineto{\pgfqpoint{0.211875in}{1.465585in}}%
\pgfpathlineto{\pgfqpoint{0.217323in}{1.452820in}}%
\pgfpathlineto{\pgfqpoint{0.214948in}{1.439894in}}%
\pgfpathlineto{\pgfqpoint{0.211875in}{1.436320in}}%
\pgfpathlineto{\pgfqpoint{0.211875in}{1.426967in}}%
\pgfpathlineto{\pgfqpoint{0.211875in}{1.424331in}}%
\pgfpathclose%
\pgfusepath{fill}%
\end{pgfscope}%
\begin{pgfscope}%
\pgfpathrectangle{\pgfqpoint{0.211875in}{0.211875in}}{\pgfqpoint{1.313625in}{1.279725in}}%
\pgfusepath{clip}%
\pgfsetbuttcap%
\pgfsetroundjoin%
\definecolor{currentfill}{rgb}{0.947270,0.405591,0.279023}%
\pgfsetfillcolor{currentfill}%
\pgfsetlinewidth{0.000000pt}%
\definecolor{currentstroke}{rgb}{0.000000,0.000000,0.000000}%
\pgfsetstrokecolor{currentstroke}%
\pgfsetdash{}{0pt}%
\pgfpathmoveto{\pgfqpoint{0.318027in}{1.420697in}}%
\pgfpathlineto{\pgfqpoint{0.331295in}{1.420161in}}%
\pgfpathlineto{\pgfqpoint{0.342979in}{1.426967in}}%
\pgfpathlineto{\pgfqpoint{0.344564in}{1.429630in}}%
\pgfpathlineto{\pgfqpoint{0.347112in}{1.439894in}}%
\pgfpathlineto{\pgfqpoint{0.347610in}{1.452820in}}%
\pgfpathlineto{\pgfqpoint{0.346183in}{1.465747in}}%
\pgfpathlineto{\pgfqpoint{0.344564in}{1.470459in}}%
\pgfpathlineto{\pgfqpoint{0.336550in}{1.478673in}}%
\pgfpathlineto{\pgfqpoint{0.331295in}{1.480867in}}%
\pgfpathlineto{\pgfqpoint{0.318027in}{1.480367in}}%
\pgfpathlineto{\pgfqpoint{0.314600in}{1.478673in}}%
\pgfpathlineto{\pgfqpoint{0.304758in}{1.466477in}}%
\pgfpathlineto{\pgfqpoint{0.304527in}{1.465747in}}%
\pgfpathlineto{\pgfqpoint{0.302992in}{1.452820in}}%
\pgfpathlineto{\pgfqpoint{0.303557in}{1.439894in}}%
\pgfpathlineto{\pgfqpoint{0.304758in}{1.434653in}}%
\pgfpathlineto{\pgfqpoint{0.308653in}{1.426967in}}%
\pgfpathclose%
\pgfpathmoveto{\pgfqpoint{0.312238in}{1.439894in}}%
\pgfpathlineto{\pgfqpoint{0.310622in}{1.452820in}}%
\pgfpathlineto{\pgfqpoint{0.314455in}{1.465747in}}%
\pgfpathlineto{\pgfqpoint{0.318027in}{1.469722in}}%
\pgfpathlineto{\pgfqpoint{0.331295in}{1.470440in}}%
\pgfpathlineto{\pgfqpoint{0.335982in}{1.465747in}}%
\pgfpathlineto{\pgfqpoint{0.340017in}{1.452820in}}%
\pgfpathlineto{\pgfqpoint{0.338342in}{1.439894in}}%
\pgfpathlineto{\pgfqpoint{0.331295in}{1.430463in}}%
\pgfpathlineto{\pgfqpoint{0.318027in}{1.431338in}}%
\pgfpathclose%
\pgfusepath{fill}%
\end{pgfscope}%
\begin{pgfscope}%
\pgfpathrectangle{\pgfqpoint{0.211875in}{0.211875in}}{\pgfqpoint{1.313625in}{1.279725in}}%
\pgfusepath{clip}%
\pgfsetbuttcap%
\pgfsetroundjoin%
\definecolor{currentfill}{rgb}{0.947270,0.405591,0.279023}%
\pgfsetfillcolor{currentfill}%
\pgfsetlinewidth{0.000000pt}%
\definecolor{currentstroke}{rgb}{0.000000,0.000000,0.000000}%
\pgfsetstrokecolor{currentstroke}%
\pgfsetdash{}{0pt}%
\pgfpathmoveto{\pgfqpoint{0.424178in}{1.422240in}}%
\pgfpathlineto{\pgfqpoint{0.437447in}{1.416497in}}%
\pgfpathlineto{\pgfqpoint{0.450716in}{1.416447in}}%
\pgfpathlineto{\pgfqpoint{0.463985in}{1.423304in}}%
\pgfpathlineto{\pgfqpoint{0.465736in}{1.426967in}}%
\pgfpathlineto{\pgfqpoint{0.467819in}{1.439894in}}%
\pgfpathlineto{\pgfqpoint{0.468075in}{1.452820in}}%
\pgfpathlineto{\pgfqpoint{0.467141in}{1.465747in}}%
\pgfpathlineto{\pgfqpoint{0.463985in}{1.476573in}}%
\pgfpathlineto{\pgfqpoint{0.462397in}{1.478673in}}%
\pgfpathlineto{\pgfqpoint{0.450716in}{1.484268in}}%
\pgfpathlineto{\pgfqpoint{0.437447in}{1.484272in}}%
\pgfpathlineto{\pgfqpoint{0.424540in}{1.478673in}}%
\pgfpathlineto{\pgfqpoint{0.424178in}{1.478318in}}%
\pgfpathlineto{\pgfqpoint{0.419569in}{1.465747in}}%
\pgfpathlineto{\pgfqpoint{0.418475in}{1.452820in}}%
\pgfpathlineto{\pgfqpoint{0.418816in}{1.439894in}}%
\pgfpathlineto{\pgfqpoint{0.421342in}{1.426967in}}%
\pgfpathclose%
\pgfpathmoveto{\pgfqpoint{0.434873in}{1.426967in}}%
\pgfpathlineto{\pgfqpoint{0.425757in}{1.439894in}}%
\pgfpathlineto{\pgfqpoint{0.424338in}{1.452820in}}%
\pgfpathlineto{\pgfqpoint{0.427841in}{1.465747in}}%
\pgfpathlineto{\pgfqpoint{0.437447in}{1.475145in}}%
\pgfpathlineto{\pgfqpoint{0.450716in}{1.474546in}}%
\pgfpathlineto{\pgfqpoint{0.458434in}{1.465747in}}%
\pgfpathlineto{\pgfqpoint{0.461405in}{1.452820in}}%
\pgfpathlineto{\pgfqpoint{0.460227in}{1.439894in}}%
\pgfpathlineto{\pgfqpoint{0.452302in}{1.426967in}}%
\pgfpathlineto{\pgfqpoint{0.450716in}{1.425910in}}%
\pgfpathlineto{\pgfqpoint{0.437447in}{1.425457in}}%
\pgfpathclose%
\pgfusepath{fill}%
\end{pgfscope}%
\begin{pgfscope}%
\pgfpathrectangle{\pgfqpoint{0.211875in}{0.211875in}}{\pgfqpoint{1.313625in}{1.279725in}}%
\pgfusepath{clip}%
\pgfsetbuttcap%
\pgfsetroundjoin%
\definecolor{currentfill}{rgb}{0.961115,0.566634,0.405693}%
\pgfsetfillcolor{currentfill}%
\pgfsetlinewidth{0.000000pt}%
\definecolor{currentstroke}{rgb}{0.000000,0.000000,0.000000}%
\pgfsetstrokecolor{currentstroke}%
\pgfsetdash{}{0pt}%
\pgfpathmoveto{\pgfqpoint{0.623212in}{0.223198in}}%
\pgfpathlineto{\pgfqpoint{0.624543in}{0.224802in}}%
\pgfpathlineto{\pgfqpoint{0.626077in}{0.237728in}}%
\pgfpathlineto{\pgfqpoint{0.623212in}{0.242637in}}%
\pgfpathlineto{\pgfqpoint{0.614892in}{0.237728in}}%
\pgfpathlineto{\pgfqpoint{0.619359in}{0.224802in}}%
\pgfpathclose%
\pgfusepath{fill}%
\end{pgfscope}%
\begin{pgfscope}%
\pgfpathrectangle{\pgfqpoint{0.211875in}{0.211875in}}{\pgfqpoint{1.313625in}{1.279725in}}%
\pgfusepath{clip}%
\pgfsetbuttcap%
\pgfsetroundjoin%
\definecolor{currentfill}{rgb}{0.961115,0.566634,0.405693}%
\pgfsetfillcolor{currentfill}%
\pgfsetlinewidth{0.000000pt}%
\definecolor{currentstroke}{rgb}{0.000000,0.000000,0.000000}%
\pgfsetstrokecolor{currentstroke}%
\pgfsetdash{}{0pt}%
\pgfpathmoveto{\pgfqpoint{0.729364in}{0.223026in}}%
\pgfpathlineto{\pgfqpoint{0.742633in}{0.220181in}}%
\pgfpathlineto{\pgfqpoint{0.745939in}{0.224802in}}%
\pgfpathlineto{\pgfqpoint{0.747219in}{0.237728in}}%
\pgfpathlineto{\pgfqpoint{0.742633in}{0.246866in}}%
\pgfpathlineto{\pgfqpoint{0.729364in}{0.243020in}}%
\pgfpathlineto{\pgfqpoint{0.727371in}{0.237728in}}%
\pgfpathlineto{\pgfqpoint{0.728402in}{0.224802in}}%
\pgfpathclose%
\pgfusepath{fill}%
\end{pgfscope}%
\begin{pgfscope}%
\pgfpathrectangle{\pgfqpoint{0.211875in}{0.211875in}}{\pgfqpoint{1.313625in}{1.279725in}}%
\pgfusepath{clip}%
\pgfsetbuttcap%
\pgfsetroundjoin%
\definecolor{currentfill}{rgb}{0.961115,0.566634,0.405693}%
\pgfsetfillcolor{currentfill}%
\pgfsetlinewidth{0.000000pt}%
\definecolor{currentstroke}{rgb}{0.000000,0.000000,0.000000}%
\pgfsetstrokecolor{currentstroke}%
\pgfsetdash{}{0pt}%
\pgfpathmoveto{\pgfqpoint{0.848784in}{0.218562in}}%
\pgfpathlineto{\pgfqpoint{0.862053in}{0.218311in}}%
\pgfpathlineto{\pgfqpoint{0.866116in}{0.224802in}}%
\pgfpathlineto{\pgfqpoint{0.867219in}{0.237728in}}%
\pgfpathlineto{\pgfqpoint{0.862053in}{0.249541in}}%
\pgfpathlineto{\pgfqpoint{0.848784in}{0.249188in}}%
\pgfpathlineto{\pgfqpoint{0.843883in}{0.237728in}}%
\pgfpathlineto{\pgfqpoint{0.844968in}{0.224802in}}%
\pgfpathclose%
\pgfusepath{fill}%
\end{pgfscope}%
\begin{pgfscope}%
\pgfpathrectangle{\pgfqpoint{0.211875in}{0.211875in}}{\pgfqpoint{1.313625in}{1.279725in}}%
\pgfusepath{clip}%
\pgfsetbuttcap%
\pgfsetroundjoin%
\definecolor{currentfill}{rgb}{0.961115,0.566634,0.405693}%
\pgfsetfillcolor{currentfill}%
\pgfsetlinewidth{0.000000pt}%
\definecolor{currentstroke}{rgb}{0.000000,0.000000,0.000000}%
\pgfsetstrokecolor{currentstroke}%
\pgfsetdash{}{0pt}%
\pgfpathmoveto{\pgfqpoint{0.968205in}{0.215569in}}%
\pgfpathlineto{\pgfqpoint{0.981473in}{0.217633in}}%
\pgfpathlineto{\pgfqpoint{0.985437in}{0.224802in}}%
\pgfpathlineto{\pgfqpoint{0.986417in}{0.237728in}}%
\pgfpathlineto{\pgfqpoint{0.981473in}{0.250591in}}%
\pgfpathlineto{\pgfqpoint{0.981180in}{0.250655in}}%
\pgfpathlineto{\pgfqpoint{0.968205in}{0.252004in}}%
\pgfpathlineto{\pgfqpoint{0.966562in}{0.250655in}}%
\pgfpathlineto{\pgfqpoint{0.960594in}{0.237728in}}%
\pgfpathlineto{\pgfqpoint{0.961767in}{0.224802in}}%
\pgfpathclose%
\pgfusepath{fill}%
\end{pgfscope}%
\begin{pgfscope}%
\pgfpathrectangle{\pgfqpoint{0.211875in}{0.211875in}}{\pgfqpoint{1.313625in}{1.279725in}}%
\pgfusepath{clip}%
\pgfsetbuttcap%
\pgfsetroundjoin%
\definecolor{currentfill}{rgb}{0.961115,0.566634,0.405693}%
\pgfsetfillcolor{currentfill}%
\pgfsetlinewidth{0.000000pt}%
\definecolor{currentstroke}{rgb}{0.000000,0.000000,0.000000}%
\pgfsetstrokecolor{currentstroke}%
\pgfsetdash{}{0pt}%
\pgfpathmoveto{\pgfqpoint{1.087625in}{0.213896in}}%
\pgfpathlineto{\pgfqpoint{1.100894in}{0.218245in}}%
\pgfpathlineto{\pgfqpoint{1.104119in}{0.224802in}}%
\pgfpathlineto{\pgfqpoint{1.105013in}{0.237728in}}%
\pgfpathlineto{\pgfqpoint{1.100894in}{0.249862in}}%
\pgfpathlineto{\pgfqpoint{1.099239in}{0.250655in}}%
\pgfpathlineto{\pgfqpoint{1.087625in}{0.253166in}}%
\pgfpathlineto{\pgfqpoint{1.084079in}{0.250655in}}%
\pgfpathlineto{\pgfqpoint{1.077526in}{0.237728in}}%
\pgfpathlineto{\pgfqpoint{1.078838in}{0.224802in}}%
\pgfpathclose%
\pgfusepath{fill}%
\end{pgfscope}%
\begin{pgfscope}%
\pgfpathrectangle{\pgfqpoint{0.211875in}{0.211875in}}{\pgfqpoint{1.313625in}{1.279725in}}%
\pgfusepath{clip}%
\pgfsetbuttcap%
\pgfsetroundjoin%
\definecolor{currentfill}{rgb}{0.961115,0.566634,0.405693}%
\pgfsetfillcolor{currentfill}%
\pgfsetlinewidth{0.000000pt}%
\definecolor{currentstroke}{rgb}{0.000000,0.000000,0.000000}%
\pgfsetstrokecolor{currentstroke}%
\pgfsetdash{}{0pt}%
\pgfpathmoveto{\pgfqpoint{1.207045in}{0.213450in}}%
\pgfpathlineto{\pgfqpoint{1.220314in}{0.220304in}}%
\pgfpathlineto{\pgfqpoint{1.222290in}{0.224802in}}%
\pgfpathlineto{\pgfqpoint{1.223127in}{0.237728in}}%
\pgfpathlineto{\pgfqpoint{1.220314in}{0.247089in}}%
\pgfpathlineto{\pgfqpoint{1.215687in}{0.250655in}}%
\pgfpathlineto{\pgfqpoint{1.207045in}{0.253463in}}%
\pgfpathlineto{\pgfqpoint{1.202359in}{0.250655in}}%
\pgfpathlineto{\pgfqpoint{1.194739in}{0.237728in}}%
\pgfpathlineto{\pgfqpoint{1.196265in}{0.224802in}}%
\pgfpathclose%
\pgfusepath{fill}%
\end{pgfscope}%
\begin{pgfscope}%
\pgfpathrectangle{\pgfqpoint{0.211875in}{0.211875in}}{\pgfqpoint{1.313625in}{1.279725in}}%
\pgfusepath{clip}%
\pgfsetbuttcap%
\pgfsetroundjoin%
\definecolor{currentfill}{rgb}{0.961115,0.566634,0.405693}%
\pgfsetfillcolor{currentfill}%
\pgfsetlinewidth{0.000000pt}%
\definecolor{currentstroke}{rgb}{0.000000,0.000000,0.000000}%
\pgfsetstrokecolor{currentstroke}%
\pgfsetdash{}{0pt}%
\pgfpathmoveto{\pgfqpoint{1.326466in}{0.214183in}}%
\pgfpathlineto{\pgfqpoint{1.339735in}{0.224054in}}%
\pgfpathlineto{\pgfqpoint{1.340029in}{0.224802in}}%
\pgfpathlineto{\pgfqpoint{1.340832in}{0.237728in}}%
\pgfpathlineto{\pgfqpoint{1.339735in}{0.241857in}}%
\pgfpathlineto{\pgfqpoint{1.331737in}{0.250655in}}%
\pgfpathlineto{\pgfqpoint{1.326466in}{0.252930in}}%
\pgfpathlineto{\pgfqpoint{1.321842in}{0.250655in}}%
\pgfpathlineto{\pgfqpoint{1.313197in}{0.239158in}}%
\pgfpathlineto{\pgfqpoint{1.312847in}{0.237728in}}%
\pgfpathlineto{\pgfqpoint{1.313197in}{0.231861in}}%
\pgfpathlineto{\pgfqpoint{1.314227in}{0.224802in}}%
\pgfpathclose%
\pgfusepath{fill}%
\end{pgfscope}%
\begin{pgfscope}%
\pgfpathrectangle{\pgfqpoint{0.211875in}{0.211875in}}{\pgfqpoint{1.313625in}{1.279725in}}%
\pgfusepath{clip}%
\pgfsetbuttcap%
\pgfsetroundjoin%
\definecolor{currentfill}{rgb}{0.961115,0.566634,0.405693}%
\pgfsetfillcolor{currentfill}%
\pgfsetlinewidth{0.000000pt}%
\definecolor{currentstroke}{rgb}{0.000000,0.000000,0.000000}%
\pgfsetstrokecolor{currentstroke}%
\pgfsetdash{}{0pt}%
\pgfpathmoveto{\pgfqpoint{1.445886in}{0.216091in}}%
\pgfpathlineto{\pgfqpoint{1.455150in}{0.224802in}}%
\pgfpathlineto{\pgfqpoint{1.456929in}{0.237728in}}%
\pgfpathlineto{\pgfqpoint{1.447591in}{0.250655in}}%
\pgfpathlineto{\pgfqpoint{1.445886in}{0.251572in}}%
\pgfpathlineto{\pgfqpoint{1.443509in}{0.250655in}}%
\pgfpathlineto{\pgfqpoint{1.432617in}{0.240133in}}%
\pgfpathlineto{\pgfqpoint{1.431950in}{0.237728in}}%
\pgfpathlineto{\pgfqpoint{1.432617in}{0.227449in}}%
\pgfpathlineto{\pgfqpoint{1.433126in}{0.224802in}}%
\pgfpathclose%
\pgfusepath{fill}%
\end{pgfscope}%
\begin{pgfscope}%
\pgfpathrectangle{\pgfqpoint{0.211875in}{0.211875in}}{\pgfqpoint{1.313625in}{1.279725in}}%
\pgfusepath{clip}%
\pgfsetbuttcap%
\pgfsetroundjoin%
\definecolor{currentfill}{rgb}{0.961115,0.566634,0.405693}%
\pgfsetfillcolor{currentfill}%
\pgfsetlinewidth{0.000000pt}%
\definecolor{currentstroke}{rgb}{0.000000,0.000000,0.000000}%
\pgfsetstrokecolor{currentstroke}%
\pgfsetdash{}{0pt}%
\pgfpathmoveto{\pgfqpoint{0.556867in}{0.301183in}}%
\pgfpathlineto{\pgfqpoint{0.561503in}{0.302361in}}%
\pgfpathlineto{\pgfqpoint{0.570136in}{0.309674in}}%
\pgfpathlineto{\pgfqpoint{0.571229in}{0.315287in}}%
\pgfpathlineto{\pgfqpoint{0.570136in}{0.319974in}}%
\pgfpathlineto{\pgfqpoint{0.557579in}{0.328214in}}%
\pgfpathlineto{\pgfqpoint{0.556867in}{0.328376in}}%
\pgfpathlineto{\pgfqpoint{0.556693in}{0.328214in}}%
\pgfpathlineto{\pgfqpoint{0.552026in}{0.315287in}}%
\pgfpathlineto{\pgfqpoint{0.555717in}{0.302361in}}%
\pgfpathclose%
\pgfusepath{fill}%
\end{pgfscope}%
\begin{pgfscope}%
\pgfpathrectangle{\pgfqpoint{0.211875in}{0.211875in}}{\pgfqpoint{1.313625in}{1.279725in}}%
\pgfusepath{clip}%
\pgfsetbuttcap%
\pgfsetroundjoin%
\definecolor{currentfill}{rgb}{0.961115,0.566634,0.405693}%
\pgfsetfillcolor{currentfill}%
\pgfsetlinewidth{0.000000pt}%
\definecolor{currentstroke}{rgb}{0.000000,0.000000,0.000000}%
\pgfsetstrokecolor{currentstroke}%
\pgfsetdash{}{0pt}%
\pgfpathmoveto{\pgfqpoint{0.676288in}{0.297182in}}%
\pgfpathlineto{\pgfqpoint{0.689099in}{0.302361in}}%
\pgfpathlineto{\pgfqpoint{0.689557in}{0.303011in}}%
\pgfpathlineto{\pgfqpoint{0.691698in}{0.315287in}}%
\pgfpathlineto{\pgfqpoint{0.689557in}{0.325691in}}%
\pgfpathlineto{\pgfqpoint{0.687310in}{0.328214in}}%
\pgfpathlineto{\pgfqpoint{0.676288in}{0.332219in}}%
\pgfpathlineto{\pgfqpoint{0.671222in}{0.328214in}}%
\pgfpathlineto{\pgfqpoint{0.666462in}{0.315287in}}%
\pgfpathlineto{\pgfqpoint{0.670361in}{0.302361in}}%
\pgfpathclose%
\pgfusepath{fill}%
\end{pgfscope}%
\begin{pgfscope}%
\pgfpathrectangle{\pgfqpoint{0.211875in}{0.211875in}}{\pgfqpoint{1.313625in}{1.279725in}}%
\pgfusepath{clip}%
\pgfsetbuttcap%
\pgfsetroundjoin%
\definecolor{currentfill}{rgb}{0.961115,0.566634,0.405693}%
\pgfsetfillcolor{currentfill}%
\pgfsetlinewidth{0.000000pt}%
\definecolor{currentstroke}{rgb}{0.000000,0.000000,0.000000}%
\pgfsetstrokecolor{currentstroke}%
\pgfsetdash{}{0pt}%
\pgfpathmoveto{\pgfqpoint{0.795708in}{0.294242in}}%
\pgfpathlineto{\pgfqpoint{0.808977in}{0.301352in}}%
\pgfpathlineto{\pgfqpoint{0.809493in}{0.302361in}}%
\pgfpathlineto{\pgfqpoint{0.811470in}{0.315287in}}%
\pgfpathlineto{\pgfqpoint{0.809151in}{0.328214in}}%
\pgfpathlineto{\pgfqpoint{0.808977in}{0.328527in}}%
\pgfpathlineto{\pgfqpoint{0.795708in}{0.335044in}}%
\pgfpathlineto{\pgfqpoint{0.785317in}{0.328214in}}%
\pgfpathlineto{\pgfqpoint{0.782439in}{0.321414in}}%
\pgfpathlineto{\pgfqpoint{0.781476in}{0.315287in}}%
\pgfpathlineto{\pgfqpoint{0.782439in}{0.308351in}}%
\pgfpathlineto{\pgfqpoint{0.784540in}{0.302361in}}%
\pgfpathclose%
\pgfpathmoveto{\pgfqpoint{0.794896in}{0.315287in}}%
\pgfpathlineto{\pgfqpoint{0.795708in}{0.316874in}}%
\pgfpathlineto{\pgfqpoint{0.796774in}{0.315287in}}%
\pgfpathlineto{\pgfqpoint{0.795708in}{0.313315in}}%
\pgfpathclose%
\pgfusepath{fill}%
\end{pgfscope}%
\begin{pgfscope}%
\pgfpathrectangle{\pgfqpoint{0.211875in}{0.211875in}}{\pgfqpoint{1.313625in}{1.279725in}}%
\pgfusepath{clip}%
\pgfsetbuttcap%
\pgfsetroundjoin%
\definecolor{currentfill}{rgb}{0.961115,0.566634,0.405693}%
\pgfsetfillcolor{currentfill}%
\pgfsetlinewidth{0.000000pt}%
\definecolor{currentstroke}{rgb}{0.000000,0.000000,0.000000}%
\pgfsetstrokecolor{currentstroke}%
\pgfsetdash{}{0pt}%
\pgfpathmoveto{\pgfqpoint{0.901860in}{0.299398in}}%
\pgfpathlineto{\pgfqpoint{0.915129in}{0.292289in}}%
\pgfpathlineto{\pgfqpoint{0.928398in}{0.301264in}}%
\pgfpathlineto{\pgfqpoint{0.928898in}{0.302361in}}%
\pgfpathlineto{\pgfqpoint{0.930674in}{0.315287in}}%
\pgfpathlineto{\pgfqpoint{0.928650in}{0.328214in}}%
\pgfpathlineto{\pgfqpoint{0.928398in}{0.328730in}}%
\pgfpathlineto{\pgfqpoint{0.915129in}{0.336926in}}%
\pgfpathlineto{\pgfqpoint{0.901860in}{0.330457in}}%
\pgfpathlineto{\pgfqpoint{0.900643in}{0.328214in}}%
\pgfpathlineto{\pgfqpoint{0.898517in}{0.315287in}}%
\pgfpathlineto{\pgfqpoint{0.900372in}{0.302361in}}%
\pgfpathclose%
\pgfpathmoveto{\pgfqpoint{0.911061in}{0.315287in}}%
\pgfpathlineto{\pgfqpoint{0.915129in}{0.321637in}}%
\pgfpathlineto{\pgfqpoint{0.918505in}{0.315287in}}%
\pgfpathlineto{\pgfqpoint{0.915129in}{0.307396in}}%
\pgfpathclose%
\pgfusepath{fill}%
\end{pgfscope}%
\begin{pgfscope}%
\pgfpathrectangle{\pgfqpoint{0.211875in}{0.211875in}}{\pgfqpoint{1.313625in}{1.279725in}}%
\pgfusepath{clip}%
\pgfsetbuttcap%
\pgfsetroundjoin%
\definecolor{currentfill}{rgb}{0.961115,0.566634,0.405693}%
\pgfsetfillcolor{currentfill}%
\pgfsetlinewidth{0.000000pt}%
\definecolor{currentstroke}{rgb}{0.000000,0.000000,0.000000}%
\pgfsetstrokecolor{currentstroke}%
\pgfsetdash{}{0pt}%
\pgfpathmoveto{\pgfqpoint{1.021280in}{0.296092in}}%
\pgfpathlineto{\pgfqpoint{1.034549in}{0.291289in}}%
\pgfpathlineto{\pgfqpoint{1.047669in}{0.302361in}}%
\pgfpathlineto{\pgfqpoint{1.047818in}{0.302886in}}%
\pgfpathlineto{\pgfqpoint{1.049391in}{0.315287in}}%
\pgfpathlineto{\pgfqpoint{1.047818in}{0.326578in}}%
\pgfpathlineto{\pgfqpoint{1.047272in}{0.328214in}}%
\pgfpathlineto{\pgfqpoint{1.034549in}{0.337904in}}%
\pgfpathlineto{\pgfqpoint{1.021280in}{0.333601in}}%
\pgfpathlineto{\pgfqpoint{1.017986in}{0.328214in}}%
\pgfpathlineto{\pgfqpoint{1.015868in}{0.315287in}}%
\pgfpathlineto{\pgfqpoint{1.017755in}{0.302361in}}%
\pgfpathclose%
\pgfpathmoveto{\pgfqpoint{1.027163in}{0.315287in}}%
\pgfpathlineto{\pgfqpoint{1.034549in}{0.323915in}}%
\pgfpathlineto{\pgfqpoint{1.038340in}{0.315287in}}%
\pgfpathlineto{\pgfqpoint{1.034549in}{0.304583in}}%
\pgfpathclose%
\pgfusepath{fill}%
\end{pgfscope}%
\begin{pgfscope}%
\pgfpathrectangle{\pgfqpoint{0.211875in}{0.211875in}}{\pgfqpoint{1.313625in}{1.279725in}}%
\pgfusepath{clip}%
\pgfsetbuttcap%
\pgfsetroundjoin%
\definecolor{currentfill}{rgb}{0.961115,0.566634,0.405693}%
\pgfsetfillcolor{currentfill}%
\pgfsetlinewidth{0.000000pt}%
\definecolor{currentstroke}{rgb}{0.000000,0.000000,0.000000}%
\pgfsetstrokecolor{currentstroke}%
\pgfsetdash{}{0pt}%
\pgfpathmoveto{\pgfqpoint{1.140701in}{0.294180in}}%
\pgfpathlineto{\pgfqpoint{1.153970in}{0.291239in}}%
\pgfpathlineto{\pgfqpoint{1.165170in}{0.302361in}}%
\pgfpathlineto{\pgfqpoint{1.167239in}{0.311476in}}%
\pgfpathlineto{\pgfqpoint{1.167673in}{0.315287in}}%
\pgfpathlineto{\pgfqpoint{1.167239in}{0.318891in}}%
\pgfpathlineto{\pgfqpoint{1.164849in}{0.328214in}}%
\pgfpathlineto{\pgfqpoint{1.153970in}{0.337981in}}%
\pgfpathlineto{\pgfqpoint{1.140701in}{0.335370in}}%
\pgfpathlineto{\pgfqpoint{1.135766in}{0.328214in}}%
\pgfpathlineto{\pgfqpoint{1.133555in}{0.315287in}}%
\pgfpathlineto{\pgfqpoint{1.135543in}{0.302361in}}%
\pgfpathclose%
\pgfpathmoveto{\pgfqpoint{1.143001in}{0.315287in}}%
\pgfpathlineto{\pgfqpoint{1.153970in}{0.323706in}}%
\pgfpathlineto{\pgfqpoint{1.157109in}{0.315287in}}%
\pgfpathlineto{\pgfqpoint{1.153970in}{0.304879in}}%
\pgfpathclose%
\pgfusepath{fill}%
\end{pgfscope}%
\begin{pgfscope}%
\pgfpathrectangle{\pgfqpoint{0.211875in}{0.211875in}}{\pgfqpoint{1.313625in}{1.279725in}}%
\pgfusepath{clip}%
\pgfsetbuttcap%
\pgfsetroundjoin%
\definecolor{currentfill}{rgb}{0.961115,0.566634,0.405693}%
\pgfsetfillcolor{currentfill}%
\pgfsetlinewidth{0.000000pt}%
\definecolor{currentstroke}{rgb}{0.000000,0.000000,0.000000}%
\pgfsetstrokecolor{currentstroke}%
\pgfsetdash{}{0pt}%
\pgfpathmoveto{\pgfqpoint{1.260121in}{0.293458in}}%
\pgfpathlineto{\pgfqpoint{1.273390in}{0.292173in}}%
\pgfpathlineto{\pgfqpoint{1.282276in}{0.302361in}}%
\pgfpathlineto{\pgfqpoint{1.284836in}{0.315287in}}%
\pgfpathlineto{\pgfqpoint{1.281963in}{0.328214in}}%
\pgfpathlineto{\pgfqpoint{1.273390in}{0.337121in}}%
\pgfpathlineto{\pgfqpoint{1.260121in}{0.335987in}}%
\pgfpathlineto{\pgfqpoint{1.254058in}{0.328214in}}%
\pgfpathlineto{\pgfqpoint{1.251626in}{0.315287in}}%
\pgfpathlineto{\pgfqpoint{1.253800in}{0.302361in}}%
\pgfpathclose%
\pgfpathmoveto{\pgfqpoint{1.259864in}{0.315287in}}%
\pgfpathlineto{\pgfqpoint{1.260121in}{0.316280in}}%
\pgfpathlineto{\pgfqpoint{1.273390in}{0.320900in}}%
\pgfpathlineto{\pgfqpoint{1.275198in}{0.315287in}}%
\pgfpathlineto{\pgfqpoint{1.273390in}{0.308385in}}%
\pgfpathlineto{\pgfqpoint{1.260121in}{0.314084in}}%
\pgfpathclose%
\pgfusepath{fill}%
\end{pgfscope}%
\begin{pgfscope}%
\pgfpathrectangle{\pgfqpoint{0.211875in}{0.211875in}}{\pgfqpoint{1.313625in}{1.279725in}}%
\pgfusepath{clip}%
\pgfsetbuttcap%
\pgfsetroundjoin%
\definecolor{currentfill}{rgb}{0.961115,0.566634,0.405693}%
\pgfsetfillcolor{currentfill}%
\pgfsetlinewidth{0.000000pt}%
\definecolor{currentstroke}{rgb}{0.000000,0.000000,0.000000}%
\pgfsetstrokecolor{currentstroke}%
\pgfsetdash{}{0pt}%
\pgfpathmoveto{\pgfqpoint{1.379542in}{0.293789in}}%
\pgfpathlineto{\pgfqpoint{1.392811in}{0.294163in}}%
\pgfpathlineto{\pgfqpoint{1.399085in}{0.302361in}}%
\pgfpathlineto{\pgfqpoint{1.401484in}{0.315287in}}%
\pgfpathlineto{\pgfqpoint{1.398735in}{0.328214in}}%
\pgfpathlineto{\pgfqpoint{1.392811in}{0.335249in}}%
\pgfpathlineto{\pgfqpoint{1.379542in}{0.335597in}}%
\pgfpathlineto{\pgfqpoint{1.372989in}{0.328214in}}%
\pgfpathlineto{\pgfqpoint{1.370168in}{0.315287in}}%
\pgfpathlineto{\pgfqpoint{1.372642in}{0.302361in}}%
\pgfpathclose%
\pgfpathmoveto{\pgfqpoint{1.379218in}{0.315287in}}%
\pgfpathlineto{\pgfqpoint{1.379542in}{0.316380in}}%
\pgfpathlineto{\pgfqpoint{1.392547in}{0.315287in}}%
\pgfpathlineto{\pgfqpoint{1.379542in}{0.313950in}}%
\pgfpathclose%
\pgfusepath{fill}%
\end{pgfscope}%
\begin{pgfscope}%
\pgfpathrectangle{\pgfqpoint{0.211875in}{0.211875in}}{\pgfqpoint{1.313625in}{1.279725in}}%
\pgfusepath{clip}%
\pgfsetbuttcap%
\pgfsetroundjoin%
\definecolor{currentfill}{rgb}{0.961115,0.566634,0.405693}%
\pgfsetfillcolor{currentfill}%
\pgfsetlinewidth{0.000000pt}%
\definecolor{currentstroke}{rgb}{0.000000,0.000000,0.000000}%
\pgfsetstrokecolor{currentstroke}%
\pgfsetdash{}{0pt}%
\pgfpathmoveto{\pgfqpoint{1.498962in}{0.295093in}}%
\pgfpathlineto{\pgfqpoint{1.512231in}{0.297327in}}%
\pgfpathlineto{\pgfqpoint{1.515642in}{0.302361in}}%
\pgfpathlineto{\pgfqpoint{1.517976in}{0.315287in}}%
\pgfpathlineto{\pgfqpoint{1.515221in}{0.328214in}}%
\pgfpathlineto{\pgfqpoint{1.512231in}{0.332239in}}%
\pgfpathlineto{\pgfqpoint{1.498962in}{0.334286in}}%
\pgfpathlineto{\pgfqpoint{1.492771in}{0.328214in}}%
\pgfpathlineto{\pgfqpoint{1.489329in}{0.315287in}}%
\pgfpathlineto{\pgfqpoint{1.492261in}{0.302361in}}%
\pgfpathclose%
\pgfusepath{fill}%
\end{pgfscope}%
\begin{pgfscope}%
\pgfpathrectangle{\pgfqpoint{0.211875in}{0.211875in}}{\pgfqpoint{1.313625in}{1.279725in}}%
\pgfusepath{clip}%
\pgfsetbuttcap%
\pgfsetroundjoin%
\definecolor{currentfill}{rgb}{0.961115,0.566634,0.405693}%
\pgfsetfillcolor{currentfill}%
\pgfsetlinewidth{0.000000pt}%
\definecolor{currentstroke}{rgb}{0.000000,0.000000,0.000000}%
\pgfsetstrokecolor{currentstroke}%
\pgfsetdash{}{0pt}%
\pgfpathmoveto{\pgfqpoint{0.437447in}{0.314439in}}%
\pgfpathlineto{\pgfqpoint{0.439926in}{0.315287in}}%
\pgfpathlineto{\pgfqpoint{0.437447in}{0.315976in}}%
\pgfpathlineto{\pgfqpoint{0.437230in}{0.315287in}}%
\pgfpathclose%
\pgfusepath{fill}%
\end{pgfscope}%
\begin{pgfscope}%
\pgfpathrectangle{\pgfqpoint{0.211875in}{0.211875in}}{\pgfqpoint{1.313625in}{1.279725in}}%
\pgfusepath{clip}%
\pgfsetbuttcap%
\pgfsetroundjoin%
\definecolor{currentfill}{rgb}{0.961115,0.566634,0.405693}%
\pgfsetfillcolor{currentfill}%
\pgfsetlinewidth{0.000000pt}%
\definecolor{currentstroke}{rgb}{0.000000,0.000000,0.000000}%
\pgfsetstrokecolor{currentstroke}%
\pgfsetdash{}{0pt}%
\pgfpathmoveto{\pgfqpoint{0.503792in}{0.379322in}}%
\pgfpathlineto{\pgfqpoint{0.504725in}{0.379920in}}%
\pgfpathlineto{\pgfqpoint{0.512315in}{0.392846in}}%
\pgfpathlineto{\pgfqpoint{0.509792in}{0.405773in}}%
\pgfpathlineto{\pgfqpoint{0.503792in}{0.411248in}}%
\pgfpathlineto{\pgfqpoint{0.493757in}{0.405773in}}%
\pgfpathlineto{\pgfqpoint{0.490523in}{0.396396in}}%
\pgfpathlineto{\pgfqpoint{0.490174in}{0.392846in}}%
\pgfpathlineto{\pgfqpoint{0.490523in}{0.391530in}}%
\pgfpathlineto{\pgfqpoint{0.502248in}{0.379920in}}%
\pgfpathclose%
\pgfusepath{fill}%
\end{pgfscope}%
\begin{pgfscope}%
\pgfpathrectangle{\pgfqpoint{0.211875in}{0.211875in}}{\pgfqpoint{1.313625in}{1.279725in}}%
\pgfusepath{clip}%
\pgfsetbuttcap%
\pgfsetroundjoin%
\definecolor{currentfill}{rgb}{0.961115,0.566634,0.405693}%
\pgfsetfillcolor{currentfill}%
\pgfsetlinewidth{0.000000pt}%
\definecolor{currentstroke}{rgb}{0.000000,0.000000,0.000000}%
\pgfsetstrokecolor{currentstroke}%
\pgfsetdash{}{0pt}%
\pgfpathmoveto{\pgfqpoint{0.609943in}{0.379748in}}%
\pgfpathlineto{\pgfqpoint{0.623212in}{0.376226in}}%
\pgfpathlineto{\pgfqpoint{0.628072in}{0.379920in}}%
\pgfpathlineto{\pgfqpoint{0.633811in}{0.392846in}}%
\pgfpathlineto{\pgfqpoint{0.632001in}{0.405773in}}%
\pgfpathlineto{\pgfqpoint{0.623212in}{0.415292in}}%
\pgfpathlineto{\pgfqpoint{0.609943in}{0.410908in}}%
\pgfpathlineto{\pgfqpoint{0.607232in}{0.405773in}}%
\pgfpathlineto{\pgfqpoint{0.606013in}{0.392846in}}%
\pgfpathlineto{\pgfqpoint{0.609813in}{0.379920in}}%
\pgfpathclose%
\pgfusepath{fill}%
\end{pgfscope}%
\begin{pgfscope}%
\pgfpathrectangle{\pgfqpoint{0.211875in}{0.211875in}}{\pgfqpoint{1.313625in}{1.279725in}}%
\pgfusepath{clip}%
\pgfsetbuttcap%
\pgfsetroundjoin%
\definecolor{currentfill}{rgb}{0.961115,0.566634,0.405693}%
\pgfsetfillcolor{currentfill}%
\pgfsetlinewidth{0.000000pt}%
\definecolor{currentstroke}{rgb}{0.000000,0.000000,0.000000}%
\pgfsetstrokecolor{currentstroke}%
\pgfsetdash{}{0pt}%
\pgfpathmoveto{\pgfqpoint{0.729364in}{0.375571in}}%
\pgfpathlineto{\pgfqpoint{0.742633in}{0.373912in}}%
\pgfpathlineto{\pgfqpoint{0.749436in}{0.379920in}}%
\pgfpathlineto{\pgfqpoint{0.753964in}{0.392846in}}%
\pgfpathlineto{\pgfqpoint{0.752613in}{0.405773in}}%
\pgfpathlineto{\pgfqpoint{0.742633in}{0.418342in}}%
\pgfpathlineto{\pgfqpoint{0.729364in}{0.416307in}}%
\pgfpathlineto{\pgfqpoint{0.723089in}{0.405773in}}%
\pgfpathlineto{\pgfqpoint{0.721965in}{0.392846in}}%
\pgfpathlineto{\pgfqpoint{0.725662in}{0.379920in}}%
\pgfpathclose%
\pgfpathmoveto{\pgfqpoint{0.734249in}{0.392846in}}%
\pgfpathlineto{\pgfqpoint{0.742633in}{0.400613in}}%
\pgfpathlineto{\pgfqpoint{0.743953in}{0.392846in}}%
\pgfpathlineto{\pgfqpoint{0.742633in}{0.389823in}}%
\pgfpathclose%
\pgfusepath{fill}%
\end{pgfscope}%
\begin{pgfscope}%
\pgfpathrectangle{\pgfqpoint{0.211875in}{0.211875in}}{\pgfqpoint{1.313625in}{1.279725in}}%
\pgfusepath{clip}%
\pgfsetbuttcap%
\pgfsetroundjoin%
\definecolor{currentfill}{rgb}{0.961115,0.566634,0.405693}%
\pgfsetfillcolor{currentfill}%
\pgfsetlinewidth{0.000000pt}%
\definecolor{currentstroke}{rgb}{0.000000,0.000000,0.000000}%
\pgfsetstrokecolor{currentstroke}%
\pgfsetdash{}{0pt}%
\pgfpathmoveto{\pgfqpoint{0.848784in}{0.372564in}}%
\pgfpathlineto{\pgfqpoint{0.862053in}{0.372377in}}%
\pgfpathlineto{\pgfqpoint{0.869511in}{0.379920in}}%
\pgfpathlineto{\pgfqpoint{0.873237in}{0.392846in}}%
\pgfpathlineto{\pgfqpoint{0.872181in}{0.405773in}}%
\pgfpathlineto{\pgfqpoint{0.863841in}{0.418699in}}%
\pgfpathlineto{\pgfqpoint{0.862053in}{0.419828in}}%
\pgfpathlineto{\pgfqpoint{0.848784in}{0.419666in}}%
\pgfpathlineto{\pgfqpoint{0.847310in}{0.418699in}}%
\pgfpathlineto{\pgfqpoint{0.839065in}{0.405773in}}%
\pgfpathlineto{\pgfqpoint{0.837996in}{0.392846in}}%
\pgfpathlineto{\pgfqpoint{0.841703in}{0.379920in}}%
\pgfpathclose%
\pgfpathmoveto{\pgfqpoint{0.846770in}{0.392846in}}%
\pgfpathlineto{\pgfqpoint{0.848626in}{0.405773in}}%
\pgfpathlineto{\pgfqpoint{0.848784in}{0.406007in}}%
\pgfpathlineto{\pgfqpoint{0.862053in}{0.406243in}}%
\pgfpathlineto{\pgfqpoint{0.862379in}{0.405773in}}%
\pgfpathlineto{\pgfqpoint{0.864261in}{0.392846in}}%
\pgfpathlineto{\pgfqpoint{0.862053in}{0.387075in}}%
\pgfpathlineto{\pgfqpoint{0.848784in}{0.387464in}}%
\pgfpathclose%
\pgfusepath{fill}%
\end{pgfscope}%
\begin{pgfscope}%
\pgfpathrectangle{\pgfqpoint{0.211875in}{0.211875in}}{\pgfqpoint{1.313625in}{1.279725in}}%
\pgfusepath{clip}%
\pgfsetbuttcap%
\pgfsetroundjoin%
\definecolor{currentfill}{rgb}{0.961115,0.566634,0.405693}%
\pgfsetfillcolor{currentfill}%
\pgfsetlinewidth{0.000000pt}%
\definecolor{currentstroke}{rgb}{0.000000,0.000000,0.000000}%
\pgfsetstrokecolor{currentstroke}%
\pgfsetdash{}{0pt}%
\pgfpathmoveto{\pgfqpoint{0.968205in}{0.370572in}}%
\pgfpathlineto{\pgfqpoint{0.981473in}{0.371650in}}%
\pgfpathlineto{\pgfqpoint{0.988685in}{0.379920in}}%
\pgfpathlineto{\pgfqpoint{0.991886in}{0.392846in}}%
\pgfpathlineto{\pgfqpoint{0.991014in}{0.405773in}}%
\pgfpathlineto{\pgfqpoint{0.983996in}{0.418699in}}%
\pgfpathlineto{\pgfqpoint{0.981473in}{0.420514in}}%
\pgfpathlineto{\pgfqpoint{0.968205in}{0.421343in}}%
\pgfpathlineto{\pgfqpoint{0.963586in}{0.418699in}}%
\pgfpathlineto{\pgfqpoint{0.955132in}{0.405773in}}%
\pgfpathlineto{\pgfqpoint{0.954936in}{0.403550in}}%
\pgfpathlineto{\pgfqpoint{0.954410in}{0.392846in}}%
\pgfpathlineto{\pgfqpoint{0.954936in}{0.389529in}}%
\pgfpathlineto{\pgfqpoint{0.957933in}{0.379920in}}%
\pgfpathclose%
\pgfpathmoveto{\pgfqpoint{0.963829in}{0.392846in}}%
\pgfpathlineto{\pgfqpoint{0.965773in}{0.405773in}}%
\pgfpathlineto{\pgfqpoint{0.968205in}{0.408919in}}%
\pgfpathlineto{\pgfqpoint{0.981473in}{0.406772in}}%
\pgfpathlineto{\pgfqpoint{0.982082in}{0.405773in}}%
\pgfpathlineto{\pgfqpoint{0.983702in}{0.392846in}}%
\pgfpathlineto{\pgfqpoint{0.981473in}{0.386258in}}%
\pgfpathlineto{\pgfqpoint{0.968205in}{0.382582in}}%
\pgfpathclose%
\pgfusepath{fill}%
\end{pgfscope}%
\begin{pgfscope}%
\pgfpathrectangle{\pgfqpoint{0.211875in}{0.211875in}}{\pgfqpoint{1.313625in}{1.279725in}}%
\pgfusepath{clip}%
\pgfsetbuttcap%
\pgfsetroundjoin%
\definecolor{currentfill}{rgb}{0.961115,0.566634,0.405693}%
\pgfsetfillcolor{currentfill}%
\pgfsetlinewidth{0.000000pt}%
\definecolor{currentstroke}{rgb}{0.000000,0.000000,0.000000}%
\pgfsetstrokecolor{currentstroke}%
\pgfsetdash{}{0pt}%
\pgfpathmoveto{\pgfqpoint{1.087625in}{0.369489in}}%
\pgfpathlineto{\pgfqpoint{1.100894in}{0.371792in}}%
\pgfpathlineto{\pgfqpoint{1.107186in}{0.379920in}}%
\pgfpathlineto{\pgfqpoint{1.110060in}{0.392846in}}%
\pgfpathlineto{\pgfqpoint{1.109294in}{0.405773in}}%
\pgfpathlineto{\pgfqpoint{1.103059in}{0.418699in}}%
\pgfpathlineto{\pgfqpoint{1.100894in}{0.420464in}}%
\pgfpathlineto{\pgfqpoint{1.087625in}{0.422245in}}%
\pgfpathlineto{\pgfqpoint{1.080432in}{0.418699in}}%
\pgfpathlineto{\pgfqpoint{1.074356in}{0.411372in}}%
\pgfpathlineto{\pgfqpoint{1.072666in}{0.405773in}}%
\pgfpathlineto{\pgfqpoint{1.072046in}{0.392846in}}%
\pgfpathlineto{\pgfqpoint{1.074356in}{0.379934in}}%
\pgfpathlineto{\pgfqpoint{1.074362in}{0.379920in}}%
\pgfpathclose%
\pgfpathmoveto{\pgfqpoint{1.087567in}{0.379920in}}%
\pgfpathlineto{\pgfqpoint{1.081212in}{0.392846in}}%
\pgfpathlineto{\pgfqpoint{1.083349in}{0.405773in}}%
\pgfpathlineto{\pgfqpoint{1.087625in}{0.410549in}}%
\pgfpathlineto{\pgfqpoint{1.100894in}{0.406044in}}%
\pgfpathlineto{\pgfqpoint{1.101040in}{0.405773in}}%
\pgfpathlineto{\pgfqpoint{1.102495in}{0.392846in}}%
\pgfpathlineto{\pgfqpoint{1.100894in}{0.387528in}}%
\pgfpathlineto{\pgfqpoint{1.087814in}{0.379920in}}%
\pgfpathlineto{\pgfqpoint{1.087625in}{0.379874in}}%
\pgfpathclose%
\pgfusepath{fill}%
\end{pgfscope}%
\begin{pgfscope}%
\pgfpathrectangle{\pgfqpoint{0.211875in}{0.211875in}}{\pgfqpoint{1.313625in}{1.279725in}}%
\pgfusepath{clip}%
\pgfsetbuttcap%
\pgfsetroundjoin%
\definecolor{currentfill}{rgb}{0.961115,0.566634,0.405693}%
\pgfsetfillcolor{currentfill}%
\pgfsetlinewidth{0.000000pt}%
\definecolor{currentstroke}{rgb}{0.000000,0.000000,0.000000}%
\pgfsetstrokecolor{currentstroke}%
\pgfsetdash{}{0pt}%
\pgfpathmoveto{\pgfqpoint{1.193777in}{0.377305in}}%
\pgfpathlineto{\pgfqpoint{1.207045in}{0.369248in}}%
\pgfpathlineto{\pgfqpoint{1.220314in}{0.372912in}}%
\pgfpathlineto{\pgfqpoint{1.225148in}{0.379920in}}%
\pgfpathlineto{\pgfqpoint{1.227850in}{0.392846in}}%
\pgfpathlineto{\pgfqpoint{1.227126in}{0.405773in}}%
\pgfpathlineto{\pgfqpoint{1.221270in}{0.418699in}}%
\pgfpathlineto{\pgfqpoint{1.220314in}{0.419579in}}%
\pgfpathlineto{\pgfqpoint{1.207045in}{0.422431in}}%
\pgfpathlineto{\pgfqpoint{1.198075in}{0.418699in}}%
\pgfpathlineto{\pgfqpoint{1.193777in}{0.414691in}}%
\pgfpathlineto{\pgfqpoint{1.190698in}{0.405773in}}%
\pgfpathlineto{\pgfqpoint{1.190063in}{0.392846in}}%
\pgfpathlineto{\pgfqpoint{1.192444in}{0.379920in}}%
\pgfpathclose%
\pgfpathmoveto{\pgfqpoint{1.206433in}{0.379920in}}%
\pgfpathlineto{\pgfqpoint{1.199034in}{0.392846in}}%
\pgfpathlineto{\pgfqpoint{1.201518in}{0.405773in}}%
\pgfpathlineto{\pgfqpoint{1.207045in}{0.411002in}}%
\pgfpathlineto{\pgfqpoint{1.217259in}{0.405773in}}%
\pgfpathlineto{\pgfqpoint{1.220314in}{0.397446in}}%
\pgfpathlineto{\pgfqpoint{1.220773in}{0.392846in}}%
\pgfpathlineto{\pgfqpoint{1.220314in}{0.391141in}}%
\pgfpathlineto{\pgfqpoint{1.208177in}{0.379920in}}%
\pgfpathlineto{\pgfqpoint{1.207045in}{0.379511in}}%
\pgfpathclose%
\pgfusepath{fill}%
\end{pgfscope}%
\begin{pgfscope}%
\pgfpathrectangle{\pgfqpoint{0.211875in}{0.211875in}}{\pgfqpoint{1.313625in}{1.279725in}}%
\pgfusepath{clip}%
\pgfsetbuttcap%
\pgfsetroundjoin%
\definecolor{currentfill}{rgb}{0.961115,0.566634,0.405693}%
\pgfsetfillcolor{currentfill}%
\pgfsetlinewidth{0.000000pt}%
\definecolor{currentstroke}{rgb}{0.000000,0.000000,0.000000}%
\pgfsetstrokecolor{currentstroke}%
\pgfsetdash{}{0pt}%
\pgfpathmoveto{\pgfqpoint{1.313197in}{0.376149in}}%
\pgfpathlineto{\pgfqpoint{1.326466in}{0.369817in}}%
\pgfpathlineto{\pgfqpoint{1.339735in}{0.375175in}}%
\pgfpathlineto{\pgfqpoint{1.342655in}{0.379920in}}%
\pgfpathlineto{\pgfqpoint{1.345304in}{0.392846in}}%
\pgfpathlineto{\pgfqpoint{1.344574in}{0.405773in}}%
\pgfpathlineto{\pgfqpoint{1.339735in}{0.417191in}}%
\pgfpathlineto{\pgfqpoint{1.337183in}{0.418699in}}%
\pgfpathlineto{\pgfqpoint{1.326466in}{0.421932in}}%
\pgfpathlineto{\pgfqpoint{1.316976in}{0.418699in}}%
\pgfpathlineto{\pgfqpoint{1.313197in}{0.416032in}}%
\pgfpathlineto{\pgfqpoint{1.309177in}{0.405773in}}%
\pgfpathlineto{\pgfqpoint{1.308482in}{0.392846in}}%
\pgfpathlineto{\pgfqpoint{1.311036in}{0.379920in}}%
\pgfpathclose%
\pgfpathmoveto{\pgfqpoint{1.317530in}{0.392846in}}%
\pgfpathlineto{\pgfqpoint{1.320615in}{0.405773in}}%
\pgfpathlineto{\pgfqpoint{1.326466in}{0.410329in}}%
\pgfpathlineto{\pgfqpoint{1.333175in}{0.405773in}}%
\pgfpathlineto{\pgfqpoint{1.336740in}{0.392846in}}%
\pgfpathlineto{\pgfqpoint{1.326466in}{0.380145in}}%
\pgfpathclose%
\pgfusepath{fill}%
\end{pgfscope}%
\begin{pgfscope}%
\pgfpathrectangle{\pgfqpoint{0.211875in}{0.211875in}}{\pgfqpoint{1.313625in}{1.279725in}}%
\pgfusepath{clip}%
\pgfsetbuttcap%
\pgfsetroundjoin%
\definecolor{currentfill}{rgb}{0.961115,0.566634,0.405693}%
\pgfsetfillcolor{currentfill}%
\pgfsetlinewidth{0.000000pt}%
\definecolor{currentstroke}{rgb}{0.000000,0.000000,0.000000}%
\pgfsetstrokecolor{currentstroke}%
\pgfsetdash{}{0pt}%
\pgfpathmoveto{\pgfqpoint{1.432617in}{0.376153in}}%
\pgfpathlineto{\pgfqpoint{1.445886in}{0.371193in}}%
\pgfpathlineto{\pgfqpoint{1.459155in}{0.378833in}}%
\pgfpathlineto{\pgfqpoint{1.459752in}{0.379920in}}%
\pgfpathlineto{\pgfqpoint{1.462449in}{0.392846in}}%
\pgfpathlineto{\pgfqpoint{1.461671in}{0.405773in}}%
\pgfpathlineto{\pgfqpoint{1.459155in}{0.412482in}}%
\pgfpathlineto{\pgfqpoint{1.451328in}{0.418699in}}%
\pgfpathlineto{\pgfqpoint{1.445886in}{0.420749in}}%
\pgfpathlineto{\pgfqpoint{1.438172in}{0.418699in}}%
\pgfpathlineto{\pgfqpoint{1.432617in}{0.415856in}}%
\pgfpathlineto{\pgfqpoint{1.428153in}{0.405773in}}%
\pgfpathlineto{\pgfqpoint{1.427346in}{0.392846in}}%
\pgfpathlineto{\pgfqpoint{1.430195in}{0.379920in}}%
\pgfpathclose%
\pgfpathmoveto{\pgfqpoint{1.437218in}{0.392846in}}%
\pgfpathlineto{\pgfqpoint{1.441378in}{0.405773in}}%
\pgfpathlineto{\pgfqpoint{1.445886in}{0.408534in}}%
\pgfpathlineto{\pgfqpoint{1.449152in}{0.405773in}}%
\pgfpathlineto{\pgfqpoint{1.452192in}{0.392846in}}%
\pgfpathlineto{\pgfqpoint{1.445886in}{0.383146in}}%
\pgfpathclose%
\pgfusepath{fill}%
\end{pgfscope}%
\begin{pgfscope}%
\pgfpathrectangle{\pgfqpoint{0.211875in}{0.211875in}}{\pgfqpoint{1.313625in}{1.279725in}}%
\pgfusepath{clip}%
\pgfsetbuttcap%
\pgfsetroundjoin%
\definecolor{currentfill}{rgb}{0.961115,0.566634,0.405693}%
\pgfsetfillcolor{currentfill}%
\pgfsetlinewidth{0.000000pt}%
\definecolor{currentstroke}{rgb}{0.000000,0.000000,0.000000}%
\pgfsetstrokecolor{currentstroke}%
\pgfsetdash{}{0pt}%
\pgfpathmoveto{\pgfqpoint{0.384371in}{0.387104in}}%
\pgfpathlineto{\pgfqpoint{0.388575in}{0.392846in}}%
\pgfpathlineto{\pgfqpoint{0.384897in}{0.405773in}}%
\pgfpathlineto{\pgfqpoint{0.384371in}{0.406164in}}%
\pgfpathlineto{\pgfqpoint{0.383821in}{0.405773in}}%
\pgfpathlineto{\pgfqpoint{0.379925in}{0.392846in}}%
\pgfpathclose%
\pgfusepath{fill}%
\end{pgfscope}%
\begin{pgfscope}%
\pgfpathrectangle{\pgfqpoint{0.211875in}{0.211875in}}{\pgfqpoint{1.313625in}{1.279725in}}%
\pgfusepath{clip}%
\pgfsetbuttcap%
\pgfsetroundjoin%
\definecolor{currentfill}{rgb}{0.961115,0.566634,0.405693}%
\pgfsetfillcolor{currentfill}%
\pgfsetlinewidth{0.000000pt}%
\definecolor{currentstroke}{rgb}{0.000000,0.000000,0.000000}%
\pgfsetstrokecolor{currentstroke}%
\pgfsetdash{}{0pt}%
\pgfpathmoveto{\pgfqpoint{0.556867in}{0.455897in}}%
\pgfpathlineto{\pgfqpoint{0.566715in}{0.457479in}}%
\pgfpathlineto{\pgfqpoint{0.570136in}{0.458562in}}%
\pgfpathlineto{\pgfqpoint{0.575229in}{0.470405in}}%
\pgfpathlineto{\pgfqpoint{0.575325in}{0.483332in}}%
\pgfpathlineto{\pgfqpoint{0.570136in}{0.495504in}}%
\pgfpathlineto{\pgfqpoint{0.567669in}{0.496258in}}%
\pgfpathlineto{\pgfqpoint{0.556867in}{0.497939in}}%
\pgfpathlineto{\pgfqpoint{0.554303in}{0.496258in}}%
\pgfpathlineto{\pgfqpoint{0.546449in}{0.483332in}}%
\pgfpathlineto{\pgfqpoint{0.546570in}{0.470405in}}%
\pgfpathlineto{\pgfqpoint{0.554505in}{0.457479in}}%
\pgfpathclose%
\pgfusepath{fill}%
\end{pgfscope}%
\begin{pgfscope}%
\pgfpathrectangle{\pgfqpoint{0.211875in}{0.211875in}}{\pgfqpoint{1.313625in}{1.279725in}}%
\pgfusepath{clip}%
\pgfsetbuttcap%
\pgfsetroundjoin%
\definecolor{currentfill}{rgb}{0.961115,0.566634,0.405693}%
\pgfsetfillcolor{currentfill}%
\pgfsetlinewidth{0.000000pt}%
\definecolor{currentstroke}{rgb}{0.000000,0.000000,0.000000}%
\pgfsetstrokecolor{currentstroke}%
\pgfsetdash{}{0pt}%
\pgfpathmoveto{\pgfqpoint{0.676288in}{0.452830in}}%
\pgfpathlineto{\pgfqpoint{0.689557in}{0.456043in}}%
\pgfpathlineto{\pgfqpoint{0.690825in}{0.457479in}}%
\pgfpathlineto{\pgfqpoint{0.695479in}{0.470405in}}%
\pgfpathlineto{\pgfqpoint{0.695600in}{0.483332in}}%
\pgfpathlineto{\pgfqpoint{0.691134in}{0.496258in}}%
\pgfpathlineto{\pgfqpoint{0.689557in}{0.498021in}}%
\pgfpathlineto{\pgfqpoint{0.676288in}{0.501071in}}%
\pgfpathlineto{\pgfqpoint{0.667674in}{0.496258in}}%
\pgfpathlineto{\pgfqpoint{0.663019in}{0.489459in}}%
\pgfpathlineto{\pgfqpoint{0.661448in}{0.483332in}}%
\pgfpathlineto{\pgfqpoint{0.661543in}{0.470405in}}%
\pgfpathlineto{\pgfqpoint{0.663019in}{0.464900in}}%
\pgfpathlineto{\pgfqpoint{0.668149in}{0.457479in}}%
\pgfpathclose%
\pgfpathmoveto{\pgfqpoint{0.672630in}{0.470405in}}%
\pgfpathlineto{\pgfqpoint{0.672609in}{0.483332in}}%
\pgfpathlineto{\pgfqpoint{0.676288in}{0.488107in}}%
\pgfpathlineto{\pgfqpoint{0.684195in}{0.483332in}}%
\pgfpathlineto{\pgfqpoint{0.684139in}{0.470405in}}%
\pgfpathlineto{\pgfqpoint{0.676288in}{0.465634in}}%
\pgfpathclose%
\pgfusepath{fill}%
\end{pgfscope}%
\begin{pgfscope}%
\pgfpathrectangle{\pgfqpoint{0.211875in}{0.211875in}}{\pgfqpoint{1.313625in}{1.279725in}}%
\pgfusepath{clip}%
\pgfsetbuttcap%
\pgfsetroundjoin%
\definecolor{currentfill}{rgb}{0.961115,0.566634,0.405693}%
\pgfsetfillcolor{currentfill}%
\pgfsetlinewidth{0.000000pt}%
\definecolor{currentstroke}{rgb}{0.000000,0.000000,0.000000}%
\pgfsetstrokecolor{currentstroke}%
\pgfsetdash{}{0pt}%
\pgfpathmoveto{\pgfqpoint{0.782439in}{0.456647in}}%
\pgfpathlineto{\pgfqpoint{0.795708in}{0.450560in}}%
\pgfpathlineto{\pgfqpoint{0.808977in}{0.454700in}}%
\pgfpathlineto{\pgfqpoint{0.811161in}{0.457479in}}%
\pgfpathlineto{\pgfqpoint{0.815073in}{0.470405in}}%
\pgfpathlineto{\pgfqpoint{0.815208in}{0.483332in}}%
\pgfpathlineto{\pgfqpoint{0.811559in}{0.496258in}}%
\pgfpathlineto{\pgfqpoint{0.808977in}{0.499515in}}%
\pgfpathlineto{\pgfqpoint{0.795708in}{0.503391in}}%
\pgfpathlineto{\pgfqpoint{0.782439in}{0.497597in}}%
\pgfpathlineto{\pgfqpoint{0.781521in}{0.496258in}}%
\pgfpathlineto{\pgfqpoint{0.777993in}{0.483332in}}%
\pgfpathlineto{\pgfqpoint{0.778114in}{0.470405in}}%
\pgfpathlineto{\pgfqpoint{0.781870in}{0.457479in}}%
\pgfpathclose%
\pgfpathmoveto{\pgfqpoint{0.787516in}{0.470405in}}%
\pgfpathlineto{\pgfqpoint{0.787443in}{0.483332in}}%
\pgfpathlineto{\pgfqpoint{0.795708in}{0.492253in}}%
\pgfpathlineto{\pgfqpoint{0.806565in}{0.483332in}}%
\pgfpathlineto{\pgfqpoint{0.806460in}{0.470405in}}%
\pgfpathlineto{\pgfqpoint{0.795708in}{0.461511in}}%
\pgfpathclose%
\pgfusepath{fill}%
\end{pgfscope}%
\begin{pgfscope}%
\pgfpathrectangle{\pgfqpoint{0.211875in}{0.211875in}}{\pgfqpoint{1.313625in}{1.279725in}}%
\pgfusepath{clip}%
\pgfsetbuttcap%
\pgfsetroundjoin%
\definecolor{currentfill}{rgb}{0.961115,0.566634,0.405693}%
\pgfsetfillcolor{currentfill}%
\pgfsetlinewidth{0.000000pt}%
\definecolor{currentstroke}{rgb}{0.000000,0.000000,0.000000}%
\pgfsetstrokecolor{currentstroke}%
\pgfsetdash{}{0pt}%
\pgfpathmoveto{\pgfqpoint{0.901860in}{0.453171in}}%
\pgfpathlineto{\pgfqpoint{0.915129in}{0.449029in}}%
\pgfpathlineto{\pgfqpoint{0.928398in}{0.454159in}}%
\pgfpathlineto{\pgfqpoint{0.930720in}{0.457479in}}%
\pgfpathlineto{\pgfqpoint{0.934128in}{0.470405in}}%
\pgfpathlineto{\pgfqpoint{0.934271in}{0.483332in}}%
\pgfpathlineto{\pgfqpoint{0.931169in}{0.496258in}}%
\pgfpathlineto{\pgfqpoint{0.928398in}{0.500209in}}%
\pgfpathlineto{\pgfqpoint{0.915129in}{0.504962in}}%
\pgfpathlineto{\pgfqpoint{0.901860in}{0.501108in}}%
\pgfpathlineto{\pgfqpoint{0.898105in}{0.496258in}}%
\pgfpathlineto{\pgfqpoint{0.894817in}{0.483332in}}%
\pgfpathlineto{\pgfqpoint{0.894960in}{0.470405in}}%
\pgfpathlineto{\pgfqpoint{0.898550in}{0.457479in}}%
\pgfpathclose%
\pgfpathmoveto{\pgfqpoint{0.901823in}{0.470405in}}%
\pgfpathlineto{\pgfqpoint{0.901774in}{0.483332in}}%
\pgfpathlineto{\pgfqpoint{0.901860in}{0.483593in}}%
\pgfpathlineto{\pgfqpoint{0.915129in}{0.494984in}}%
\pgfpathlineto{\pgfqpoint{0.926352in}{0.483332in}}%
\pgfpathlineto{\pgfqpoint{0.926235in}{0.470405in}}%
\pgfpathlineto{\pgfqpoint{0.915129in}{0.458801in}}%
\pgfpathlineto{\pgfqpoint{0.901860in}{0.470296in}}%
\pgfpathclose%
\pgfusepath{fill}%
\end{pgfscope}%
\begin{pgfscope}%
\pgfpathrectangle{\pgfqpoint{0.211875in}{0.211875in}}{\pgfqpoint{1.313625in}{1.279725in}}%
\pgfusepath{clip}%
\pgfsetbuttcap%
\pgfsetroundjoin%
\definecolor{currentfill}{rgb}{0.961115,0.566634,0.405693}%
\pgfsetfillcolor{currentfill}%
\pgfsetlinewidth{0.000000pt}%
\definecolor{currentstroke}{rgb}{0.000000,0.000000,0.000000}%
\pgfsetstrokecolor{currentstroke}%
\pgfsetdash{}{0pt}%
\pgfpathmoveto{\pgfqpoint{1.021280in}{0.450947in}}%
\pgfpathlineto{\pgfqpoint{1.034549in}{0.448209in}}%
\pgfpathlineto{\pgfqpoint{1.047818in}{0.454574in}}%
\pgfpathlineto{\pgfqpoint{1.049622in}{0.457479in}}%
\pgfpathlineto{\pgfqpoint{1.052721in}{0.470405in}}%
\pgfpathlineto{\pgfqpoint{1.052866in}{0.483332in}}%
\pgfpathlineto{\pgfqpoint{1.050090in}{0.496258in}}%
\pgfpathlineto{\pgfqpoint{1.047818in}{0.499934in}}%
\pgfpathlineto{\pgfqpoint{1.034549in}{0.505815in}}%
\pgfpathlineto{\pgfqpoint{1.021280in}{0.503311in}}%
\pgfpathlineto{\pgfqpoint{1.015128in}{0.496258in}}%
\pgfpathlineto{\pgfqpoint{1.011920in}{0.483332in}}%
\pgfpathlineto{\pgfqpoint{1.012082in}{0.470405in}}%
\pgfpathlineto{\pgfqpoint{1.015651in}{0.457479in}}%
\pgfpathclose%
\pgfpathmoveto{\pgfqpoint{1.034497in}{0.457479in}}%
\pgfpathlineto{\pgfqpoint{1.021280in}{0.465283in}}%
\pgfpathlineto{\pgfqpoint{1.019353in}{0.470405in}}%
\pgfpathlineto{\pgfqpoint{1.019290in}{0.483332in}}%
\pgfpathlineto{\pgfqpoint{1.021280in}{0.488700in}}%
\pgfpathlineto{\pgfqpoint{1.034346in}{0.496258in}}%
\pgfpathlineto{\pgfqpoint{1.034549in}{0.496314in}}%
\pgfpathlineto{\pgfqpoint{1.034654in}{0.496258in}}%
\pgfpathlineto{\pgfqpoint{1.044915in}{0.483332in}}%
\pgfpathlineto{\pgfqpoint{1.044801in}{0.470405in}}%
\pgfpathlineto{\pgfqpoint{1.034576in}{0.457479in}}%
\pgfpathlineto{\pgfqpoint{1.034549in}{0.457464in}}%
\pgfpathclose%
\pgfusepath{fill}%
\end{pgfscope}%
\begin{pgfscope}%
\pgfpathrectangle{\pgfqpoint{0.211875in}{0.211875in}}{\pgfqpoint{1.313625in}{1.279725in}}%
\pgfusepath{clip}%
\pgfsetbuttcap%
\pgfsetroundjoin%
\definecolor{currentfill}{rgb}{0.961115,0.566634,0.405693}%
\pgfsetfillcolor{currentfill}%
\pgfsetlinewidth{0.000000pt}%
\definecolor{currentstroke}{rgb}{0.000000,0.000000,0.000000}%
\pgfsetstrokecolor{currentstroke}%
\pgfsetdash{}{0pt}%
\pgfpathmoveto{\pgfqpoint{1.140701in}{0.449749in}}%
\pgfpathlineto{\pgfqpoint{1.153970in}{0.448098in}}%
\pgfpathlineto{\pgfqpoint{1.167239in}{0.456196in}}%
\pgfpathlineto{\pgfqpoint{1.167942in}{0.457479in}}%
\pgfpathlineto{\pgfqpoint{1.170898in}{0.470405in}}%
\pgfpathlineto{\pgfqpoint{1.171040in}{0.483332in}}%
\pgfpathlineto{\pgfqpoint{1.168404in}{0.496258in}}%
\pgfpathlineto{\pgfqpoint{1.167239in}{0.498414in}}%
\pgfpathlineto{\pgfqpoint{1.153970in}{0.505953in}}%
\pgfpathlineto{\pgfqpoint{1.140701in}{0.504460in}}%
\pgfpathlineto{\pgfqpoint{1.132633in}{0.496258in}}%
\pgfpathlineto{\pgfqpoint{1.129315in}{0.483332in}}%
\pgfpathlineto{\pgfqpoint{1.129494in}{0.470405in}}%
\pgfpathlineto{\pgfqpoint{1.133216in}{0.457479in}}%
\pgfpathclose%
\pgfpathmoveto{\pgfqpoint{1.153927in}{0.457479in}}%
\pgfpathlineto{\pgfqpoint{1.140701in}{0.462288in}}%
\pgfpathlineto{\pgfqpoint{1.137278in}{0.470405in}}%
\pgfpathlineto{\pgfqpoint{1.137206in}{0.483332in}}%
\pgfpathlineto{\pgfqpoint{1.140701in}{0.491692in}}%
\pgfpathlineto{\pgfqpoint{1.153611in}{0.496258in}}%
\pgfpathlineto{\pgfqpoint{1.153970in}{0.496323in}}%
\pgfpathlineto{\pgfqpoint{1.154072in}{0.496258in}}%
\pgfpathlineto{\pgfqpoint{1.162782in}{0.483332in}}%
\pgfpathlineto{\pgfqpoint{1.162680in}{0.470405in}}%
\pgfpathlineto{\pgfqpoint{1.153982in}{0.457479in}}%
\pgfpathlineto{\pgfqpoint{1.153970in}{0.457471in}}%
\pgfpathclose%
\pgfusepath{fill}%
\end{pgfscope}%
\begin{pgfscope}%
\pgfpathrectangle{\pgfqpoint{0.211875in}{0.211875in}}{\pgfqpoint{1.313625in}{1.279725in}}%
\pgfusepath{clip}%
\pgfsetbuttcap%
\pgfsetroundjoin%
\definecolor{currentfill}{rgb}{0.961115,0.566634,0.405693}%
\pgfsetfillcolor{currentfill}%
\pgfsetlinewidth{0.000000pt}%
\definecolor{currentstroke}{rgb}{0.000000,0.000000,0.000000}%
\pgfsetstrokecolor{currentstroke}%
\pgfsetdash{}{0pt}%
\pgfpathmoveto{\pgfqpoint{1.260121in}{0.449424in}}%
\pgfpathlineto{\pgfqpoint{1.273390in}{0.448719in}}%
\pgfpathlineto{\pgfqpoint{1.285131in}{0.457479in}}%
\pgfpathlineto{\pgfqpoint{1.286659in}{0.460868in}}%
\pgfpathlineto{\pgfqpoint{1.288684in}{0.470405in}}%
\pgfpathlineto{\pgfqpoint{1.288818in}{0.483332in}}%
\pgfpathlineto{\pgfqpoint{1.286659in}{0.494359in}}%
\pgfpathlineto{\pgfqpoint{1.285840in}{0.496258in}}%
\pgfpathlineto{\pgfqpoint{1.273390in}{0.505350in}}%
\pgfpathlineto{\pgfqpoint{1.260121in}{0.504719in}}%
\pgfpathlineto{\pgfqpoint{1.250704in}{0.496258in}}%
\pgfpathlineto{\pgfqpoint{1.247040in}{0.483332in}}%
\pgfpathlineto{\pgfqpoint{1.247231in}{0.470405in}}%
\pgfpathlineto{\pgfqpoint{1.251323in}{0.457479in}}%
\pgfpathclose%
\pgfpathmoveto{\pgfqpoint{1.255664in}{0.470405in}}%
\pgfpathlineto{\pgfqpoint{1.255590in}{0.483332in}}%
\pgfpathlineto{\pgfqpoint{1.260121in}{0.492919in}}%
\pgfpathlineto{\pgfqpoint{1.273390in}{0.494974in}}%
\pgfpathlineto{\pgfqpoint{1.280194in}{0.483332in}}%
\pgfpathlineto{\pgfqpoint{1.280109in}{0.470405in}}%
\pgfpathlineto{\pgfqpoint{1.273390in}{0.458887in}}%
\pgfpathlineto{\pgfqpoint{1.260121in}{0.461012in}}%
\pgfpathclose%
\pgfusepath{fill}%
\end{pgfscope}%
\begin{pgfscope}%
\pgfpathrectangle{\pgfqpoint{0.211875in}{0.211875in}}{\pgfqpoint{1.313625in}{1.279725in}}%
\pgfusepath{clip}%
\pgfsetbuttcap%
\pgfsetroundjoin%
\definecolor{currentfill}{rgb}{0.961115,0.566634,0.405693}%
\pgfsetfillcolor{currentfill}%
\pgfsetlinewidth{0.000000pt}%
\definecolor{currentstroke}{rgb}{0.000000,0.000000,0.000000}%
\pgfsetstrokecolor{currentstroke}%
\pgfsetdash{}{0pt}%
\pgfpathmoveto{\pgfqpoint{1.379542in}{0.449874in}}%
\pgfpathlineto{\pgfqpoint{1.392811in}{0.450127in}}%
\pgfpathlineto{\pgfqpoint{1.401444in}{0.457479in}}%
\pgfpathlineto{\pgfqpoint{1.406080in}{0.470375in}}%
\pgfpathlineto{\pgfqpoint{1.406085in}{0.470405in}}%
\pgfpathlineto{\pgfqpoint{1.406208in}{0.483332in}}%
\pgfpathlineto{\pgfqpoint{1.406080in}{0.484110in}}%
\pgfpathlineto{\pgfqpoint{1.402018in}{0.496258in}}%
\pgfpathlineto{\pgfqpoint{1.392811in}{0.503948in}}%
\pgfpathlineto{\pgfqpoint{1.379542in}{0.504197in}}%
\pgfpathlineto{\pgfqpoint{1.369480in}{0.496258in}}%
\pgfpathlineto{\pgfqpoint{1.366273in}{0.487463in}}%
\pgfpathlineto{\pgfqpoint{1.365550in}{0.483332in}}%
\pgfpathlineto{\pgfqpoint{1.365680in}{0.470405in}}%
\pgfpathlineto{\pgfqpoint{1.366273in}{0.467338in}}%
\pgfpathlineto{\pgfqpoint{1.370106in}{0.457479in}}%
\pgfpathclose%
\pgfpathmoveto{\pgfqpoint{1.374624in}{0.470405in}}%
\pgfpathlineto{\pgfqpoint{1.374554in}{0.483332in}}%
\pgfpathlineto{\pgfqpoint{1.379542in}{0.492610in}}%
\pgfpathlineto{\pgfqpoint{1.392811in}{0.492050in}}%
\pgfpathlineto{\pgfqpoint{1.397269in}{0.483332in}}%
\pgfpathlineto{\pgfqpoint{1.397206in}{0.470405in}}%
\pgfpathlineto{\pgfqpoint{1.392811in}{0.461822in}}%
\pgfpathlineto{\pgfqpoint{1.379542in}{0.461259in}}%
\pgfpathclose%
\pgfusepath{fill}%
\end{pgfscope}%
\begin{pgfscope}%
\pgfpathrectangle{\pgfqpoint{0.211875in}{0.211875in}}{\pgfqpoint{1.313625in}{1.279725in}}%
\pgfusepath{clip}%
\pgfsetbuttcap%
\pgfsetroundjoin%
\definecolor{currentfill}{rgb}{0.961115,0.566634,0.405693}%
\pgfsetfillcolor{currentfill}%
\pgfsetlinewidth{0.000000pt}%
\definecolor{currentstroke}{rgb}{0.000000,0.000000,0.000000}%
\pgfsetstrokecolor{currentstroke}%
\pgfsetdash{}{0pt}%
\pgfpathmoveto{\pgfqpoint{1.498962in}{0.451038in}}%
\pgfpathlineto{\pgfqpoint{1.512231in}{0.452411in}}%
\pgfpathlineto{\pgfqpoint{1.517492in}{0.457479in}}%
\pgfpathlineto{\pgfqpoint{1.522155in}{0.470405in}}%
\pgfpathlineto{\pgfqpoint{1.522305in}{0.483332in}}%
\pgfpathlineto{\pgfqpoint{1.517925in}{0.496258in}}%
\pgfpathlineto{\pgfqpoint{1.512231in}{0.501651in}}%
\pgfpathlineto{\pgfqpoint{1.498962in}{0.502955in}}%
\pgfpathlineto{\pgfqpoint{1.489200in}{0.496258in}}%
\pgfpathlineto{\pgfqpoint{1.485693in}{0.488812in}}%
\pgfpathlineto{\pgfqpoint{1.484566in}{0.483332in}}%
\pgfpathlineto{\pgfqpoint{1.484685in}{0.470405in}}%
\pgfpathlineto{\pgfqpoint{1.485693in}{0.465857in}}%
\pgfpathlineto{\pgfqpoint{1.489791in}{0.457479in}}%
\pgfpathclose%
\pgfpathmoveto{\pgfqpoint{1.494347in}{0.470405in}}%
\pgfpathlineto{\pgfqpoint{1.494292in}{0.483332in}}%
\pgfpathlineto{\pgfqpoint{1.498962in}{0.490899in}}%
\pgfpathlineto{\pgfqpoint{1.512231in}{0.487406in}}%
\pgfpathlineto{\pgfqpoint{1.514069in}{0.483332in}}%
\pgfpathlineto{\pgfqpoint{1.514030in}{0.470405in}}%
\pgfpathlineto{\pgfqpoint{1.512231in}{0.466439in}}%
\pgfpathlineto{\pgfqpoint{1.498962in}{0.462907in}}%
\pgfpathclose%
\pgfusepath{fill}%
\end{pgfscope}%
\begin{pgfscope}%
\pgfpathrectangle{\pgfqpoint{0.211875in}{0.211875in}}{\pgfqpoint{1.313625in}{1.279725in}}%
\pgfusepath{clip}%
\pgfsetbuttcap%
\pgfsetroundjoin%
\definecolor{currentfill}{rgb}{0.961115,0.566634,0.405693}%
\pgfsetfillcolor{currentfill}%
\pgfsetlinewidth{0.000000pt}%
\definecolor{currentstroke}{rgb}{0.000000,0.000000,0.000000}%
\pgfsetstrokecolor{currentstroke}%
\pgfsetdash{}{0pt}%
\pgfpathmoveto{\pgfqpoint{0.331295in}{0.469259in}}%
\pgfpathlineto{\pgfqpoint{0.331927in}{0.470405in}}%
\pgfpathlineto{\pgfqpoint{0.331937in}{0.483332in}}%
\pgfpathlineto{\pgfqpoint{0.331295in}{0.484497in}}%
\pgfpathlineto{\pgfqpoint{0.318428in}{0.483332in}}%
\pgfpathlineto{\pgfqpoint{0.318498in}{0.470405in}}%
\pgfpathclose%
\pgfusepath{fill}%
\end{pgfscope}%
\begin{pgfscope}%
\pgfpathrectangle{\pgfqpoint{0.211875in}{0.211875in}}{\pgfqpoint{1.313625in}{1.279725in}}%
\pgfusepath{clip}%
\pgfsetbuttcap%
\pgfsetroundjoin%
\definecolor{currentfill}{rgb}{0.961115,0.566634,0.405693}%
\pgfsetfillcolor{currentfill}%
\pgfsetlinewidth{0.000000pt}%
\definecolor{currentstroke}{rgb}{0.000000,0.000000,0.000000}%
\pgfsetstrokecolor{currentstroke}%
\pgfsetdash{}{0pt}%
\pgfpathmoveto{\pgfqpoint{0.437447in}{0.461674in}}%
\pgfpathlineto{\pgfqpoint{0.450716in}{0.463344in}}%
\pgfpathlineto{\pgfqpoint{0.454139in}{0.470405in}}%
\pgfpathlineto{\pgfqpoint{0.454200in}{0.483332in}}%
\pgfpathlineto{\pgfqpoint{0.450716in}{0.490547in}}%
\pgfpathlineto{\pgfqpoint{0.437447in}{0.492154in}}%
\pgfpathlineto{\pgfqpoint{0.432417in}{0.483332in}}%
\pgfpathlineto{\pgfqpoint{0.432476in}{0.470405in}}%
\pgfpathclose%
\pgfusepath{fill}%
\end{pgfscope}%
\begin{pgfscope}%
\pgfpathrectangle{\pgfqpoint{0.211875in}{0.211875in}}{\pgfqpoint{1.313625in}{1.279725in}}%
\pgfusepath{clip}%
\pgfsetbuttcap%
\pgfsetroundjoin%
\definecolor{currentfill}{rgb}{0.961115,0.566634,0.405693}%
\pgfsetfillcolor{currentfill}%
\pgfsetlinewidth{0.000000pt}%
\definecolor{currentstroke}{rgb}{0.000000,0.000000,0.000000}%
\pgfsetstrokecolor{currentstroke}%
\pgfsetdash{}{0pt}%
\pgfpathmoveto{\pgfqpoint{0.623212in}{0.532775in}}%
\pgfpathlineto{\pgfqpoint{0.627855in}{0.535038in}}%
\pgfpathlineto{\pgfqpoint{0.636481in}{0.545015in}}%
\pgfpathlineto{\pgfqpoint{0.637425in}{0.547964in}}%
\pgfpathlineto{\pgfqpoint{0.638326in}{0.560891in}}%
\pgfpathlineto{\pgfqpoint{0.636481in}{0.571922in}}%
\pgfpathlineto{\pgfqpoint{0.635754in}{0.573817in}}%
\pgfpathlineto{\pgfqpoint{0.623212in}{0.583190in}}%
\pgfpathlineto{\pgfqpoint{0.609943in}{0.580459in}}%
\pgfpathlineto{\pgfqpoint{0.604885in}{0.573817in}}%
\pgfpathlineto{\pgfqpoint{0.602057in}{0.560891in}}%
\pgfpathlineto{\pgfqpoint{0.603181in}{0.547964in}}%
\pgfpathlineto{\pgfqpoint{0.609943in}{0.535202in}}%
\pgfpathlineto{\pgfqpoint{0.610479in}{0.535038in}}%
\pgfpathclose%
\pgfpathmoveto{\pgfqpoint{0.615035in}{0.547964in}}%
\pgfpathlineto{\pgfqpoint{0.609943in}{0.557247in}}%
\pgfpathlineto{\pgfqpoint{0.609519in}{0.560891in}}%
\pgfpathlineto{\pgfqpoint{0.609943in}{0.562349in}}%
\pgfpathlineto{\pgfqpoint{0.623212in}{0.571167in}}%
\pgfpathlineto{\pgfqpoint{0.628427in}{0.560891in}}%
\pgfpathlineto{\pgfqpoint{0.626044in}{0.547964in}}%
\pgfpathlineto{\pgfqpoint{0.623212in}{0.544872in}}%
\pgfpathclose%
\pgfusepath{fill}%
\end{pgfscope}%
\begin{pgfscope}%
\pgfpathrectangle{\pgfqpoint{0.211875in}{0.211875in}}{\pgfqpoint{1.313625in}{1.279725in}}%
\pgfusepath{clip}%
\pgfsetbuttcap%
\pgfsetroundjoin%
\definecolor{currentfill}{rgb}{0.961115,0.566634,0.405693}%
\pgfsetfillcolor{currentfill}%
\pgfsetlinewidth{0.000000pt}%
\definecolor{currentstroke}{rgb}{0.000000,0.000000,0.000000}%
\pgfsetstrokecolor{currentstroke}%
\pgfsetdash{}{0pt}%
\pgfpathmoveto{\pgfqpoint{0.729364in}{0.531860in}}%
\pgfpathlineto{\pgfqpoint{0.742633in}{0.530757in}}%
\pgfpathlineto{\pgfqpoint{0.750166in}{0.535038in}}%
\pgfpathlineto{\pgfqpoint{0.755902in}{0.543457in}}%
\pgfpathlineto{\pgfqpoint{0.757167in}{0.547964in}}%
\pgfpathlineto{\pgfqpoint{0.757951in}{0.560891in}}%
\pgfpathlineto{\pgfqpoint{0.756133in}{0.573817in}}%
\pgfpathlineto{\pgfqpoint{0.755902in}{0.574412in}}%
\pgfpathlineto{\pgfqpoint{0.742633in}{0.585695in}}%
\pgfpathlineto{\pgfqpoint{0.729364in}{0.584439in}}%
\pgfpathlineto{\pgfqpoint{0.720238in}{0.573817in}}%
\pgfpathlineto{\pgfqpoint{0.717679in}{0.560891in}}%
\pgfpathlineto{\pgfqpoint{0.718762in}{0.547964in}}%
\pgfpathlineto{\pgfqpoint{0.725215in}{0.535038in}}%
\pgfpathclose%
\pgfpathmoveto{\pgfqpoint{0.727304in}{0.547964in}}%
\pgfpathlineto{\pgfqpoint{0.725740in}{0.560891in}}%
\pgfpathlineto{\pgfqpoint{0.729364in}{0.571929in}}%
\pgfpathlineto{\pgfqpoint{0.735237in}{0.573817in}}%
\pgfpathlineto{\pgfqpoint{0.742633in}{0.574866in}}%
\pgfpathlineto{\pgfqpoint{0.743838in}{0.573817in}}%
\pgfpathlineto{\pgfqpoint{0.749251in}{0.560891in}}%
\pgfpathlineto{\pgfqpoint{0.747321in}{0.547964in}}%
\pgfpathlineto{\pgfqpoint{0.742633in}{0.542021in}}%
\pgfpathlineto{\pgfqpoint{0.729364in}{0.544507in}}%
\pgfpathclose%
\pgfusepath{fill}%
\end{pgfscope}%
\begin{pgfscope}%
\pgfpathrectangle{\pgfqpoint{0.211875in}{0.211875in}}{\pgfqpoint{1.313625in}{1.279725in}}%
\pgfusepath{clip}%
\pgfsetbuttcap%
\pgfsetroundjoin%
\definecolor{currentfill}{rgb}{0.961115,0.566634,0.405693}%
\pgfsetfillcolor{currentfill}%
\pgfsetlinewidth{0.000000pt}%
\definecolor{currentstroke}{rgb}{0.000000,0.000000,0.000000}%
\pgfsetstrokecolor{currentstroke}%
\pgfsetdash{}{0pt}%
\pgfpathmoveto{\pgfqpoint{0.848784in}{0.529512in}}%
\pgfpathlineto{\pgfqpoint{0.862053in}{0.529353in}}%
\pgfpathlineto{\pgfqpoint{0.870765in}{0.535038in}}%
\pgfpathlineto{\pgfqpoint{0.875322in}{0.543501in}}%
\pgfpathlineto{\pgfqpoint{0.876409in}{0.547964in}}%
\pgfpathlineto{\pgfqpoint{0.877110in}{0.560891in}}%
\pgfpathlineto{\pgfqpoint{0.875585in}{0.573817in}}%
\pgfpathlineto{\pgfqpoint{0.875322in}{0.574612in}}%
\pgfpathlineto{\pgfqpoint{0.863447in}{0.586744in}}%
\pgfpathlineto{\pgfqpoint{0.862053in}{0.587307in}}%
\pgfpathlineto{\pgfqpoint{0.848784in}{0.587153in}}%
\pgfpathlineto{\pgfqpoint{0.847825in}{0.586744in}}%
\pgfpathlineto{\pgfqpoint{0.835680in}{0.573817in}}%
\pgfpathlineto{\pgfqpoint{0.835515in}{0.573079in}}%
\pgfpathlineto{\pgfqpoint{0.834049in}{0.560891in}}%
\pgfpathlineto{\pgfqpoint{0.834759in}{0.547964in}}%
\pgfpathlineto{\pgfqpoint{0.835515in}{0.544817in}}%
\pgfpathlineto{\pgfqpoint{0.840610in}{0.535038in}}%
\pgfpathclose%
\pgfpathmoveto{\pgfqpoint{0.843723in}{0.547964in}}%
\pgfpathlineto{\pgfqpoint{0.842119in}{0.560891in}}%
\pgfpathlineto{\pgfqpoint{0.846511in}{0.573817in}}%
\pgfpathlineto{\pgfqpoint{0.848784in}{0.576150in}}%
\pgfpathlineto{\pgfqpoint{0.862053in}{0.576339in}}%
\pgfpathlineto{\pgfqpoint{0.864579in}{0.573817in}}%
\pgfpathlineto{\pgfqpoint{0.869021in}{0.560891in}}%
\pgfpathlineto{\pgfqpoint{0.867397in}{0.547964in}}%
\pgfpathlineto{\pgfqpoint{0.862053in}{0.540213in}}%
\pgfpathlineto{\pgfqpoint{0.848784in}{0.540448in}}%
\pgfpathclose%
\pgfusepath{fill}%
\end{pgfscope}%
\begin{pgfscope}%
\pgfpathrectangle{\pgfqpoint{0.211875in}{0.211875in}}{\pgfqpoint{1.313625in}{1.279725in}}%
\pgfusepath{clip}%
\pgfsetbuttcap%
\pgfsetroundjoin%
\definecolor{currentfill}{rgb}{0.961115,0.566634,0.405693}%
\pgfsetfillcolor{currentfill}%
\pgfsetlinewidth{0.000000pt}%
\definecolor{currentstroke}{rgb}{0.000000,0.000000,0.000000}%
\pgfsetstrokecolor{currentstroke}%
\pgfsetdash{}{0pt}%
\pgfpathmoveto{\pgfqpoint{0.968205in}{0.527973in}}%
\pgfpathlineto{\pgfqpoint{0.981473in}{0.528582in}}%
\pgfpathlineto{\pgfqpoint{0.990177in}{0.535038in}}%
\pgfpathlineto{\pgfqpoint{0.994742in}{0.545836in}}%
\pgfpathlineto{\pgfqpoint{0.995185in}{0.547964in}}%
\pgfpathlineto{\pgfqpoint{0.995834in}{0.560891in}}%
\pgfpathlineto{\pgfqpoint{0.994742in}{0.571786in}}%
\pgfpathlineto{\pgfqpoint{0.994389in}{0.573817in}}%
\pgfpathlineto{\pgfqpoint{0.984403in}{0.586744in}}%
\pgfpathlineto{\pgfqpoint{0.981473in}{0.588097in}}%
\pgfpathlineto{\pgfqpoint{0.968205in}{0.588593in}}%
\pgfpathlineto{\pgfqpoint{0.963228in}{0.586744in}}%
\pgfpathlineto{\pgfqpoint{0.954936in}{0.579791in}}%
\pgfpathlineto{\pgfqpoint{0.952654in}{0.573817in}}%
\pgfpathlineto{\pgfqpoint{0.951242in}{0.560891in}}%
\pgfpathlineto{\pgfqpoint{0.951915in}{0.547964in}}%
\pgfpathlineto{\pgfqpoint{0.954936in}{0.537046in}}%
\pgfpathlineto{\pgfqpoint{0.956253in}{0.535038in}}%
\pgfpathclose%
\pgfpathmoveto{\pgfqpoint{0.960350in}{0.547964in}}%
\pgfpathlineto{\pgfqpoint{0.958646in}{0.560891in}}%
\pgfpathlineto{\pgfqpoint{0.963226in}{0.573817in}}%
\pgfpathlineto{\pgfqpoint{0.968205in}{0.578285in}}%
\pgfpathlineto{\pgfqpoint{0.981473in}{0.576959in}}%
\pgfpathlineto{\pgfqpoint{0.984242in}{0.573817in}}%
\pgfpathlineto{\pgfqpoint{0.988048in}{0.560891in}}%
\pgfpathlineto{\pgfqpoint{0.986631in}{0.547964in}}%
\pgfpathlineto{\pgfqpoint{0.981473in}{0.539489in}}%
\pgfpathlineto{\pgfqpoint{0.968205in}{0.537740in}}%
\pgfpathclose%
\pgfusepath{fill}%
\end{pgfscope}%
\begin{pgfscope}%
\pgfpathrectangle{\pgfqpoint{0.211875in}{0.211875in}}{\pgfqpoint{1.313625in}{1.279725in}}%
\pgfusepath{clip}%
\pgfsetbuttcap%
\pgfsetroundjoin%
\definecolor{currentfill}{rgb}{0.961115,0.566634,0.405693}%
\pgfsetfillcolor{currentfill}%
\pgfsetlinewidth{0.000000pt}%
\definecolor{currentstroke}{rgb}{0.000000,0.000000,0.000000}%
\pgfsetstrokecolor{currentstroke}%
\pgfsetdash{}{0pt}%
\pgfpathmoveto{\pgfqpoint{1.074356in}{0.533306in}}%
\pgfpathlineto{\pgfqpoint{1.087625in}{0.527157in}}%
\pgfpathlineto{\pgfqpoint{1.100894in}{0.528492in}}%
\pgfpathlineto{\pgfqpoint{1.108710in}{0.535038in}}%
\pgfpathlineto{\pgfqpoint{1.113305in}{0.547964in}}%
\pgfpathlineto{\pgfqpoint{1.114125in}{0.560891in}}%
\pgfpathlineto{\pgfqpoint{1.112472in}{0.573817in}}%
\pgfpathlineto{\pgfqpoint{1.103761in}{0.586744in}}%
\pgfpathlineto{\pgfqpoint{1.100894in}{0.588248in}}%
\pgfpathlineto{\pgfqpoint{1.087625in}{0.589347in}}%
\pgfpathlineto{\pgfqpoint{1.079461in}{0.586744in}}%
\pgfpathlineto{\pgfqpoint{1.074356in}{0.583407in}}%
\pgfpathlineto{\pgfqpoint{1.070132in}{0.573817in}}%
\pgfpathlineto{\pgfqpoint{1.068791in}{0.560891in}}%
\pgfpathlineto{\pgfqpoint{1.069453in}{0.547964in}}%
\pgfpathlineto{\pgfqpoint{1.073160in}{0.535038in}}%
\pgfpathclose%
\pgfpathmoveto{\pgfqpoint{1.077209in}{0.547964in}}%
\pgfpathlineto{\pgfqpoint{1.075324in}{0.560891in}}%
\pgfpathlineto{\pgfqpoint{1.080340in}{0.573817in}}%
\pgfpathlineto{\pgfqpoint{1.087625in}{0.579458in}}%
\pgfpathlineto{\pgfqpoint{1.100894in}{0.576650in}}%
\pgfpathlineto{\pgfqpoint{1.103102in}{0.573817in}}%
\pgfpathlineto{\pgfqpoint{1.106516in}{0.560891in}}%
\pgfpathlineto{\pgfqpoint{1.105232in}{0.547964in}}%
\pgfpathlineto{\pgfqpoint{1.100894in}{0.539937in}}%
\pgfpathlineto{\pgfqpoint{1.087625in}{0.536239in}}%
\pgfpathclose%
\pgfusepath{fill}%
\end{pgfscope}%
\begin{pgfscope}%
\pgfpathrectangle{\pgfqpoint{0.211875in}{0.211875in}}{\pgfqpoint{1.313625in}{1.279725in}}%
\pgfusepath{clip}%
\pgfsetbuttcap%
\pgfsetroundjoin%
\definecolor{currentfill}{rgb}{0.961115,0.566634,0.405693}%
\pgfsetfillcolor{currentfill}%
\pgfsetlinewidth{0.000000pt}%
\definecolor{currentstroke}{rgb}{0.000000,0.000000,0.000000}%
\pgfsetstrokecolor{currentstroke}%
\pgfsetdash{}{0pt}%
\pgfpathmoveto{\pgfqpoint{1.193777in}{0.531722in}}%
\pgfpathlineto{\pgfqpoint{1.207045in}{0.527013in}}%
\pgfpathlineto{\pgfqpoint{1.220314in}{0.529165in}}%
\pgfpathlineto{\pgfqpoint{1.226550in}{0.535038in}}%
\pgfpathlineto{\pgfqpoint{1.230878in}{0.547964in}}%
\pgfpathlineto{\pgfqpoint{1.231651in}{0.560891in}}%
\pgfpathlineto{\pgfqpoint{1.230085in}{0.573817in}}%
\pgfpathlineto{\pgfqpoint{1.221889in}{0.586744in}}%
\pgfpathlineto{\pgfqpoint{1.220314in}{0.587679in}}%
\pgfpathlineto{\pgfqpoint{1.207045in}{0.589468in}}%
\pgfpathlineto{\pgfqpoint{1.196877in}{0.586744in}}%
\pgfpathlineto{\pgfqpoint{1.193777in}{0.585180in}}%
\pgfpathlineto{\pgfqpoint{1.188066in}{0.573817in}}%
\pgfpathlineto{\pgfqpoint{1.186696in}{0.560891in}}%
\pgfpathlineto{\pgfqpoint{1.187376in}{0.547964in}}%
\pgfpathlineto{\pgfqpoint{1.191181in}{0.535038in}}%
\pgfpathclose%
\pgfpathmoveto{\pgfqpoint{1.194365in}{0.547964in}}%
\pgfpathlineto{\pgfqpoint{1.193777in}{0.551289in}}%
\pgfpathlineto{\pgfqpoint{1.192999in}{0.560891in}}%
\pgfpathlineto{\pgfqpoint{1.193777in}{0.565006in}}%
\pgfpathlineto{\pgfqpoint{1.198000in}{0.573817in}}%
\pgfpathlineto{\pgfqpoint{1.207045in}{0.579746in}}%
\pgfpathlineto{\pgfqpoint{1.220314in}{0.575285in}}%
\pgfpathlineto{\pgfqpoint{1.221330in}{0.573817in}}%
\pgfpathlineto{\pgfqpoint{1.224533in}{0.560891in}}%
\pgfpathlineto{\pgfqpoint{1.223330in}{0.547964in}}%
\pgfpathlineto{\pgfqpoint{1.220314in}{0.541706in}}%
\pgfpathlineto{\pgfqpoint{1.207045in}{0.535859in}}%
\pgfpathclose%
\pgfusepath{fill}%
\end{pgfscope}%
\begin{pgfscope}%
\pgfpathrectangle{\pgfqpoint{0.211875in}{0.211875in}}{\pgfqpoint{1.313625in}{1.279725in}}%
\pgfusepath{clip}%
\pgfsetbuttcap%
\pgfsetroundjoin%
\definecolor{currentfill}{rgb}{0.961115,0.566634,0.405693}%
\pgfsetfillcolor{currentfill}%
\pgfsetlinewidth{0.000000pt}%
\definecolor{currentstroke}{rgb}{0.000000,0.000000,0.000000}%
\pgfsetstrokecolor{currentstroke}%
\pgfsetdash{}{0pt}%
\pgfpathmoveto{\pgfqpoint{1.313197in}{0.531241in}}%
\pgfpathlineto{\pgfqpoint{1.326466in}{0.527513in}}%
\pgfpathlineto{\pgfqpoint{1.339735in}{0.530731in}}%
\pgfpathlineto{\pgfqpoint{1.343806in}{0.535038in}}%
\pgfpathlineto{\pgfqpoint{1.348131in}{0.547964in}}%
\pgfpathlineto{\pgfqpoint{1.348893in}{0.560891in}}%
\pgfpathlineto{\pgfqpoint{1.347294in}{0.573817in}}%
\pgfpathlineto{\pgfqpoint{1.339735in}{0.586124in}}%
\pgfpathlineto{\pgfqpoint{1.337829in}{0.586744in}}%
\pgfpathlineto{\pgfqpoint{1.326466in}{0.588981in}}%
\pgfpathlineto{\pgfqpoint{1.316199in}{0.586744in}}%
\pgfpathlineto{\pgfqpoint{1.313197in}{0.585603in}}%
\pgfpathlineto{\pgfqpoint{1.306485in}{0.573817in}}%
\pgfpathlineto{\pgfqpoint{1.304975in}{0.560891in}}%
\pgfpathlineto{\pgfqpoint{1.305705in}{0.547964in}}%
\pgfpathlineto{\pgfqpoint{1.309845in}{0.535038in}}%
\pgfpathclose%
\pgfpathmoveto{\pgfqpoint{1.312676in}{0.547964in}}%
\pgfpathlineto{\pgfqpoint{1.311545in}{0.560891in}}%
\pgfpathlineto{\pgfqpoint{1.313197in}{0.568585in}}%
\pgfpathlineto{\pgfqpoint{1.316504in}{0.573817in}}%
\pgfpathlineto{\pgfqpoint{1.326466in}{0.579184in}}%
\pgfpathlineto{\pgfqpoint{1.337844in}{0.573817in}}%
\pgfpathlineto{\pgfqpoint{1.339735in}{0.571342in}}%
\pgfpathlineto{\pgfqpoint{1.342167in}{0.560891in}}%
\pgfpathlineto{\pgfqpoint{1.340999in}{0.547964in}}%
\pgfpathlineto{\pgfqpoint{1.339735in}{0.545025in}}%
\pgfpathlineto{\pgfqpoint{1.326466in}{0.536555in}}%
\pgfpathlineto{\pgfqpoint{1.313197in}{0.546662in}}%
\pgfpathclose%
\pgfusepath{fill}%
\end{pgfscope}%
\begin{pgfscope}%
\pgfpathrectangle{\pgfqpoint{0.211875in}{0.211875in}}{\pgfqpoint{1.313625in}{1.279725in}}%
\pgfusepath{clip}%
\pgfsetbuttcap%
\pgfsetroundjoin%
\definecolor{currentfill}{rgb}{0.961115,0.566634,0.405693}%
\pgfsetfillcolor{currentfill}%
\pgfsetlinewidth{0.000000pt}%
\definecolor{currentstroke}{rgb}{0.000000,0.000000,0.000000}%
\pgfsetstrokecolor{currentstroke}%
\pgfsetdash{}{0pt}%
\pgfpathmoveto{\pgfqpoint{1.432617in}{0.531631in}}%
\pgfpathlineto{\pgfqpoint{1.445886in}{0.528658in}}%
\pgfpathlineto{\pgfqpoint{1.459155in}{0.533386in}}%
\pgfpathlineto{\pgfqpoint{1.460545in}{0.535038in}}%
\pgfpathlineto{\pgfqpoint{1.465085in}{0.547964in}}%
\pgfpathlineto{\pgfqpoint{1.465867in}{0.560891in}}%
\pgfpathlineto{\pgfqpoint{1.464132in}{0.573817in}}%
\pgfpathlineto{\pgfqpoint{1.459155in}{0.582963in}}%
\pgfpathlineto{\pgfqpoint{1.450515in}{0.586744in}}%
\pgfpathlineto{\pgfqpoint{1.445886in}{0.587886in}}%
\pgfpathlineto{\pgfqpoint{1.439094in}{0.586744in}}%
\pgfpathlineto{\pgfqpoint{1.432617in}{0.584975in}}%
\pgfpathlineto{\pgfqpoint{1.425444in}{0.573817in}}%
\pgfpathlineto{\pgfqpoint{1.423663in}{0.560891in}}%
\pgfpathlineto{\pgfqpoint{1.424481in}{0.547964in}}%
\pgfpathlineto{\pgfqpoint{1.429235in}{0.535038in}}%
\pgfpathclose%
\pgfpathmoveto{\pgfqpoint{1.431813in}{0.547964in}}%
\pgfpathlineto{\pgfqpoint{1.430571in}{0.560891in}}%
\pgfpathlineto{\pgfqpoint{1.432617in}{0.569317in}}%
\pgfpathlineto{\pgfqpoint{1.436520in}{0.573817in}}%
\pgfpathlineto{\pgfqpoint{1.445886in}{0.577776in}}%
\pgfpathlineto{\pgfqpoint{1.452622in}{0.573817in}}%
\pgfpathlineto{\pgfqpoint{1.459155in}{0.562333in}}%
\pgfpathlineto{\pgfqpoint{1.459452in}{0.560891in}}%
\pgfpathlineto{\pgfqpoint{1.459155in}{0.557461in}}%
\pgfpathlineto{\pgfqpoint{1.457191in}{0.547964in}}%
\pgfpathlineto{\pgfqpoint{1.445886in}{0.538322in}}%
\pgfpathlineto{\pgfqpoint{1.432617in}{0.546170in}}%
\pgfpathclose%
\pgfusepath{fill}%
\end{pgfscope}%
\begin{pgfscope}%
\pgfpathrectangle{\pgfqpoint{0.211875in}{0.211875in}}{\pgfqpoint{1.313625in}{1.279725in}}%
\pgfusepath{clip}%
\pgfsetbuttcap%
\pgfsetroundjoin%
\definecolor{currentfill}{rgb}{0.961115,0.566634,0.405693}%
\pgfsetfillcolor{currentfill}%
\pgfsetlinewidth{0.000000pt}%
\definecolor{currentstroke}{rgb}{0.000000,0.000000,0.000000}%
\pgfsetstrokecolor{currentstroke}%
\pgfsetdash{}{0pt}%
\pgfpathmoveto{\pgfqpoint{0.264951in}{0.546790in}}%
\pgfpathlineto{\pgfqpoint{0.266956in}{0.547964in}}%
\pgfpathlineto{\pgfqpoint{0.271560in}{0.560891in}}%
\pgfpathlineto{\pgfqpoint{0.264951in}{0.567886in}}%
\pgfpathlineto{\pgfqpoint{0.260543in}{0.560891in}}%
\pgfpathlineto{\pgfqpoint{0.263625in}{0.547964in}}%
\pgfpathclose%
\pgfusepath{fill}%
\end{pgfscope}%
\begin{pgfscope}%
\pgfpathrectangle{\pgfqpoint{0.211875in}{0.211875in}}{\pgfqpoint{1.313625in}{1.279725in}}%
\pgfusepath{clip}%
\pgfsetbuttcap%
\pgfsetroundjoin%
\definecolor{currentfill}{rgb}{0.961115,0.566634,0.405693}%
\pgfsetfillcolor{currentfill}%
\pgfsetlinewidth{0.000000pt}%
\definecolor{currentstroke}{rgb}{0.000000,0.000000,0.000000}%
\pgfsetstrokecolor{currentstroke}%
\pgfsetdash{}{0pt}%
\pgfpathmoveto{\pgfqpoint{0.384371in}{0.540665in}}%
\pgfpathlineto{\pgfqpoint{0.394061in}{0.547964in}}%
\pgfpathlineto{\pgfqpoint{0.397155in}{0.560891in}}%
\pgfpathlineto{\pgfqpoint{0.388456in}{0.573817in}}%
\pgfpathlineto{\pgfqpoint{0.384371in}{0.575909in}}%
\pgfpathlineto{\pgfqpoint{0.380174in}{0.573817in}}%
\pgfpathlineto{\pgfqpoint{0.371102in}{0.561296in}}%
\pgfpathlineto{\pgfqpoint{0.371009in}{0.560891in}}%
\pgfpathlineto{\pgfqpoint{0.371102in}{0.559907in}}%
\pgfpathlineto{\pgfqpoint{0.374206in}{0.547964in}}%
\pgfpathclose%
\pgfusepath{fill}%
\end{pgfscope}%
\begin{pgfscope}%
\pgfpathrectangle{\pgfqpoint{0.211875in}{0.211875in}}{\pgfqpoint{1.313625in}{1.279725in}}%
\pgfusepath{clip}%
\pgfsetbuttcap%
\pgfsetroundjoin%
\definecolor{currentfill}{rgb}{0.961115,0.566634,0.405693}%
\pgfsetfillcolor{currentfill}%
\pgfsetlinewidth{0.000000pt}%
\definecolor{currentstroke}{rgb}{0.000000,0.000000,0.000000}%
\pgfsetstrokecolor{currentstroke}%
\pgfsetdash{}{0pt}%
\pgfpathmoveto{\pgfqpoint{0.490523in}{0.541947in}}%
\pgfpathlineto{\pgfqpoint{0.503792in}{0.535615in}}%
\pgfpathlineto{\pgfqpoint{0.517061in}{0.547791in}}%
\pgfpathlineto{\pgfqpoint{0.517123in}{0.547964in}}%
\pgfpathlineto{\pgfqpoint{0.518183in}{0.560891in}}%
\pgfpathlineto{\pgfqpoint{0.517061in}{0.566742in}}%
\pgfpathlineto{\pgfqpoint{0.513532in}{0.573817in}}%
\pgfpathlineto{\pgfqpoint{0.503792in}{0.579940in}}%
\pgfpathlineto{\pgfqpoint{0.490523in}{0.575089in}}%
\pgfpathlineto{\pgfqpoint{0.489662in}{0.573817in}}%
\pgfpathlineto{\pgfqpoint{0.486484in}{0.560891in}}%
\pgfpathlineto{\pgfqpoint{0.487676in}{0.547964in}}%
\pgfpathclose%
\pgfpathmoveto{\pgfqpoint{0.500083in}{0.560891in}}%
\pgfpathlineto{\pgfqpoint{0.503792in}{0.564542in}}%
\pgfpathlineto{\pgfqpoint{0.505989in}{0.560891in}}%
\pgfpathlineto{\pgfqpoint{0.503792in}{0.551472in}}%
\pgfpathclose%
\pgfusepath{fill}%
\end{pgfscope}%
\begin{pgfscope}%
\pgfpathrectangle{\pgfqpoint{0.211875in}{0.211875in}}{\pgfqpoint{1.313625in}{1.279725in}}%
\pgfusepath{clip}%
\pgfsetbuttcap%
\pgfsetroundjoin%
\definecolor{currentfill}{rgb}{0.961115,0.566634,0.405693}%
\pgfsetfillcolor{currentfill}%
\pgfsetlinewidth{0.000000pt}%
\definecolor{currentstroke}{rgb}{0.000000,0.000000,0.000000}%
\pgfsetstrokecolor{currentstroke}%
\pgfsetdash{}{0pt}%
\pgfpathmoveto{\pgfqpoint{0.676288in}{0.610583in}}%
\pgfpathlineto{\pgfqpoint{0.689110in}{0.612597in}}%
\pgfpathlineto{\pgfqpoint{0.689557in}{0.612712in}}%
\pgfpathlineto{\pgfqpoint{0.697560in}{0.625523in}}%
\pgfpathlineto{\pgfqpoint{0.699287in}{0.638450in}}%
\pgfpathlineto{\pgfqpoint{0.698163in}{0.651377in}}%
\pgfpathlineto{\pgfqpoint{0.691541in}{0.664303in}}%
\pgfpathlineto{\pgfqpoint{0.689557in}{0.665809in}}%
\pgfpathlineto{\pgfqpoint{0.676288in}{0.667792in}}%
\pgfpathlineto{\pgfqpoint{0.667278in}{0.664303in}}%
\pgfpathlineto{\pgfqpoint{0.663019in}{0.660847in}}%
\pgfpathlineto{\pgfqpoint{0.659327in}{0.651377in}}%
\pgfpathlineto{\pgfqpoint{0.658302in}{0.638450in}}%
\pgfpathlineto{\pgfqpoint{0.659841in}{0.625523in}}%
\pgfpathlineto{\pgfqpoint{0.663019in}{0.618175in}}%
\pgfpathlineto{\pgfqpoint{0.670739in}{0.612597in}}%
\pgfpathclose%
\pgfpathmoveto{\pgfqpoint{0.670638in}{0.625523in}}%
\pgfpathlineto{\pgfqpoint{0.665842in}{0.638450in}}%
\pgfpathlineto{\pgfqpoint{0.669514in}{0.651377in}}%
\pgfpathlineto{\pgfqpoint{0.676288in}{0.657234in}}%
\pgfpathlineto{\pgfqpoint{0.689557in}{0.652052in}}%
\pgfpathlineto{\pgfqpoint{0.689943in}{0.651377in}}%
\pgfpathlineto{\pgfqpoint{0.692062in}{0.638450in}}%
\pgfpathlineto{\pgfqpoint{0.689557in}{0.626663in}}%
\pgfpathlineto{\pgfqpoint{0.688541in}{0.625523in}}%
\pgfpathlineto{\pgfqpoint{0.676288in}{0.621002in}}%
\pgfpathclose%
\pgfusepath{fill}%
\end{pgfscope}%
\begin{pgfscope}%
\pgfpathrectangle{\pgfqpoint{0.211875in}{0.211875in}}{\pgfqpoint{1.313625in}{1.279725in}}%
\pgfusepath{clip}%
\pgfsetbuttcap%
\pgfsetroundjoin%
\definecolor{currentfill}{rgb}{0.961115,0.566634,0.405693}%
\pgfsetfillcolor{currentfill}%
\pgfsetlinewidth{0.000000pt}%
\definecolor{currentstroke}{rgb}{0.000000,0.000000,0.000000}%
\pgfsetstrokecolor{currentstroke}%
\pgfsetdash{}{0pt}%
\pgfpathmoveto{\pgfqpoint{0.795708in}{0.608613in}}%
\pgfpathlineto{\pgfqpoint{0.808977in}{0.611230in}}%
\pgfpathlineto{\pgfqpoint{0.810698in}{0.612597in}}%
\pgfpathlineto{\pgfqpoint{0.817215in}{0.625523in}}%
\pgfpathlineto{\pgfqpoint{0.818626in}{0.638450in}}%
\pgfpathlineto{\pgfqpoint{0.817770in}{0.651377in}}%
\pgfpathlineto{\pgfqpoint{0.812613in}{0.664303in}}%
\pgfpathlineto{\pgfqpoint{0.808977in}{0.667420in}}%
\pgfpathlineto{\pgfqpoint{0.795708in}{0.669843in}}%
\pgfpathlineto{\pgfqpoint{0.782439in}{0.666052in}}%
\pgfpathlineto{\pgfqpoint{0.780690in}{0.664303in}}%
\pgfpathlineto{\pgfqpoint{0.775640in}{0.651377in}}%
\pgfpathlineto{\pgfqpoint{0.774778in}{0.638450in}}%
\pgfpathlineto{\pgfqpoint{0.776162in}{0.625523in}}%
\pgfpathlineto{\pgfqpoint{0.782439in}{0.612624in}}%
\pgfpathlineto{\pgfqpoint{0.782488in}{0.612597in}}%
\pgfpathclose%
\pgfpathmoveto{\pgfqpoint{0.784742in}{0.625523in}}%
\pgfpathlineto{\pgfqpoint{0.782439in}{0.630784in}}%
\pgfpathlineto{\pgfqpoint{0.781157in}{0.638450in}}%
\pgfpathlineto{\pgfqpoint{0.782439in}{0.648421in}}%
\pgfpathlineto{\pgfqpoint{0.783427in}{0.651377in}}%
\pgfpathlineto{\pgfqpoint{0.795708in}{0.660208in}}%
\pgfpathlineto{\pgfqpoint{0.808977in}{0.653437in}}%
\pgfpathlineto{\pgfqpoint{0.810022in}{0.651377in}}%
\pgfpathlineto{\pgfqpoint{0.811812in}{0.638450in}}%
\pgfpathlineto{\pgfqpoint{0.809410in}{0.625523in}}%
\pgfpathlineto{\pgfqpoint{0.808977in}{0.624746in}}%
\pgfpathlineto{\pgfqpoint{0.795708in}{0.618223in}}%
\pgfpathclose%
\pgfusepath{fill}%
\end{pgfscope}%
\begin{pgfscope}%
\pgfpathrectangle{\pgfqpoint{0.211875in}{0.211875in}}{\pgfqpoint{1.313625in}{1.279725in}}%
\pgfusepath{clip}%
\pgfsetbuttcap%
\pgfsetroundjoin%
\definecolor{currentfill}{rgb}{0.961115,0.566634,0.405693}%
\pgfsetfillcolor{currentfill}%
\pgfsetlinewidth{0.000000pt}%
\definecolor{currentstroke}{rgb}{0.000000,0.000000,0.000000}%
\pgfsetstrokecolor{currentstroke}%
\pgfsetdash{}{0pt}%
\pgfpathmoveto{\pgfqpoint{0.901860in}{0.609890in}}%
\pgfpathlineto{\pgfqpoint{0.915129in}{0.607268in}}%
\pgfpathlineto{\pgfqpoint{0.928398in}{0.610416in}}%
\pgfpathlineto{\pgfqpoint{0.930833in}{0.612597in}}%
\pgfpathlineto{\pgfqpoint{0.936285in}{0.625523in}}%
\pgfpathlineto{\pgfqpoint{0.937485in}{0.638450in}}%
\pgfpathlineto{\pgfqpoint{0.936806in}{0.651377in}}%
\pgfpathlineto{\pgfqpoint{0.932611in}{0.664303in}}%
\pgfpathlineto{\pgfqpoint{0.928398in}{0.668390in}}%
\pgfpathlineto{\pgfqpoint{0.915129in}{0.671249in}}%
\pgfpathlineto{\pgfqpoint{0.901860in}{0.668835in}}%
\pgfpathlineto{\pgfqpoint{0.896741in}{0.664303in}}%
\pgfpathlineto{\pgfqpoint{0.892235in}{0.651377in}}%
\pgfpathlineto{\pgfqpoint{0.891489in}{0.638450in}}%
\pgfpathlineto{\pgfqpoint{0.892774in}{0.625523in}}%
\pgfpathlineto{\pgfqpoint{0.898562in}{0.612597in}}%
\pgfpathclose%
\pgfpathmoveto{\pgfqpoint{0.900421in}{0.625523in}}%
\pgfpathlineto{\pgfqpoint{0.898183in}{0.638450in}}%
\pgfpathlineto{\pgfqpoint{0.899827in}{0.651377in}}%
\pgfpathlineto{\pgfqpoint{0.901860in}{0.655460in}}%
\pgfpathlineto{\pgfqpoint{0.915129in}{0.662190in}}%
\pgfpathlineto{\pgfqpoint{0.928398in}{0.653681in}}%
\pgfpathlineto{\pgfqpoint{0.929433in}{0.651377in}}%
\pgfpathlineto{\pgfqpoint{0.930997in}{0.638450in}}%
\pgfpathlineto{\pgfqpoint{0.928859in}{0.625523in}}%
\pgfpathlineto{\pgfqpoint{0.928398in}{0.624593in}}%
\pgfpathlineto{\pgfqpoint{0.915129in}{0.616375in}}%
\pgfpathlineto{\pgfqpoint{0.901860in}{0.622887in}}%
\pgfpathclose%
\pgfusepath{fill}%
\end{pgfscope}%
\begin{pgfscope}%
\pgfpathrectangle{\pgfqpoint{0.211875in}{0.211875in}}{\pgfqpoint{1.313625in}{1.279725in}}%
\pgfusepath{clip}%
\pgfsetbuttcap%
\pgfsetroundjoin%
\definecolor{currentfill}{rgb}{0.961115,0.566634,0.405693}%
\pgfsetfillcolor{currentfill}%
\pgfsetlinewidth{0.000000pt}%
\definecolor{currentstroke}{rgb}{0.000000,0.000000,0.000000}%
\pgfsetstrokecolor{currentstroke}%
\pgfsetdash{}{0pt}%
\pgfpathmoveto{\pgfqpoint{1.021280in}{0.608202in}}%
\pgfpathlineto{\pgfqpoint{1.034549in}{0.606523in}}%
\pgfpathlineto{\pgfqpoint{1.047818in}{0.610357in}}%
\pgfpathlineto{\pgfqpoint{1.050032in}{0.612597in}}%
\pgfpathlineto{\pgfqpoint{1.054856in}{0.625523in}}%
\pgfpathlineto{\pgfqpoint{1.055930in}{0.638450in}}%
\pgfpathlineto{\pgfqpoint{1.055351in}{0.651377in}}%
\pgfpathlineto{\pgfqpoint{1.051711in}{0.664303in}}%
\pgfpathlineto{\pgfqpoint{1.047818in}{0.668591in}}%
\pgfpathlineto{\pgfqpoint{1.034549in}{0.672037in}}%
\pgfpathlineto{\pgfqpoint{1.021280in}{0.670527in}}%
\pgfpathlineto{\pgfqpoint{1.013353in}{0.664303in}}%
\pgfpathlineto{\pgfqpoint{1.009109in}{0.651377in}}%
\pgfpathlineto{\pgfqpoint{1.008425in}{0.638450in}}%
\pgfpathlineto{\pgfqpoint{1.009674in}{0.625523in}}%
\pgfpathlineto{\pgfqpoint{1.015259in}{0.612597in}}%
\pgfpathclose%
\pgfpathmoveto{\pgfqpoint{1.017778in}{0.625523in}}%
\pgfpathlineto{\pgfqpoint{1.015516in}{0.638450in}}%
\pgfpathlineto{\pgfqpoint{1.017156in}{0.651377in}}%
\pgfpathlineto{\pgfqpoint{1.021280in}{0.658738in}}%
\pgfpathlineto{\pgfqpoint{1.034549in}{0.663222in}}%
\pgfpathlineto{\pgfqpoint{1.047818in}{0.652529in}}%
\pgfpathlineto{\pgfqpoint{1.048275in}{0.651377in}}%
\pgfpathlineto{\pgfqpoint{1.049701in}{0.638450in}}%
\pgfpathlineto{\pgfqpoint{1.047818in}{0.626041in}}%
\pgfpathlineto{\pgfqpoint{1.047636in}{0.625523in}}%
\pgfpathlineto{\pgfqpoint{1.034549in}{0.615422in}}%
\pgfpathlineto{\pgfqpoint{1.021280in}{0.619802in}}%
\pgfpathclose%
\pgfusepath{fill}%
\end{pgfscope}%
\begin{pgfscope}%
\pgfpathrectangle{\pgfqpoint{0.211875in}{0.211875in}}{\pgfqpoint{1.313625in}{1.279725in}}%
\pgfusepath{clip}%
\pgfsetbuttcap%
\pgfsetroundjoin%
\definecolor{currentfill}{rgb}{0.961115,0.566634,0.405693}%
\pgfsetfillcolor{currentfill}%
\pgfsetlinewidth{0.000000pt}%
\definecolor{currentstroke}{rgb}{0.000000,0.000000,0.000000}%
\pgfsetstrokecolor{currentstroke}%
\pgfsetdash{}{0pt}%
\pgfpathmoveto{\pgfqpoint{1.140701in}{0.607363in}}%
\pgfpathlineto{\pgfqpoint{1.153970in}{0.606375in}}%
\pgfpathlineto{\pgfqpoint{1.167239in}{0.611247in}}%
\pgfpathlineto{\pgfqpoint{1.168412in}{0.612597in}}%
\pgfpathlineto{\pgfqpoint{1.172978in}{0.625523in}}%
\pgfpathlineto{\pgfqpoint{1.174001in}{0.638450in}}%
\pgfpathlineto{\pgfqpoint{1.173456in}{0.651377in}}%
\pgfpathlineto{\pgfqpoint{1.170028in}{0.664303in}}%
\pgfpathlineto{\pgfqpoint{1.167239in}{0.667815in}}%
\pgfpathlineto{\pgfqpoint{1.153970in}{0.672211in}}%
\pgfpathlineto{\pgfqpoint{1.140701in}{0.671336in}}%
\pgfpathlineto{\pgfqpoint{1.130589in}{0.664303in}}%
\pgfpathlineto{\pgfqpoint{1.127432in}{0.656182in}}%
\pgfpathlineto{\pgfqpoint{1.126555in}{0.651377in}}%
\pgfpathlineto{\pgfqpoint{1.126036in}{0.638450in}}%
\pgfpathlineto{\pgfqpoint{1.127010in}{0.625523in}}%
\pgfpathlineto{\pgfqpoint{1.127432in}{0.623595in}}%
\pgfpathlineto{\pgfqpoint{1.132624in}{0.612597in}}%
\pgfpathclose%
\pgfpathmoveto{\pgfqpoint{1.135555in}{0.625523in}}%
\pgfpathlineto{\pgfqpoint{1.133179in}{0.638450in}}%
\pgfpathlineto{\pgfqpoint{1.134891in}{0.651377in}}%
\pgfpathlineto{\pgfqpoint{1.140701in}{0.660583in}}%
\pgfpathlineto{\pgfqpoint{1.153970in}{0.663305in}}%
\pgfpathlineto{\pgfqpoint{1.166105in}{0.651377in}}%
\pgfpathlineto{\pgfqpoint{1.167239in}{0.645820in}}%
\pgfpathlineto{\pgfqpoint{1.167971in}{0.638450in}}%
\pgfpathlineto{\pgfqpoint{1.167239in}{0.633014in}}%
\pgfpathlineto{\pgfqpoint{1.165152in}{0.625523in}}%
\pgfpathlineto{\pgfqpoint{1.153970in}{0.615362in}}%
\pgfpathlineto{\pgfqpoint{1.140701in}{0.618036in}}%
\pgfpathclose%
\pgfusepath{fill}%
\end{pgfscope}%
\begin{pgfscope}%
\pgfpathrectangle{\pgfqpoint{0.211875in}{0.211875in}}{\pgfqpoint{1.313625in}{1.279725in}}%
\pgfusepath{clip}%
\pgfsetbuttcap%
\pgfsetroundjoin%
\definecolor{currentfill}{rgb}{0.961115,0.566634,0.405693}%
\pgfsetfillcolor{currentfill}%
\pgfsetlinewidth{0.000000pt}%
\definecolor{currentstroke}{rgb}{0.000000,0.000000,0.000000}%
\pgfsetstrokecolor{currentstroke}%
\pgfsetdash{}{0pt}%
\pgfpathmoveto{\pgfqpoint{1.260121in}{0.607249in}}%
\pgfpathlineto{\pgfqpoint{1.273390in}{0.606842in}}%
\pgfpathlineto{\pgfqpoint{1.285661in}{0.612597in}}%
\pgfpathlineto{\pgfqpoint{1.286659in}{0.613699in}}%
\pgfpathlineto{\pgfqpoint{1.290680in}{0.625523in}}%
\pgfpathlineto{\pgfqpoint{1.291715in}{0.638450in}}%
\pgfpathlineto{\pgfqpoint{1.291148in}{0.651377in}}%
\pgfpathlineto{\pgfqpoint{1.287628in}{0.664303in}}%
\pgfpathlineto{\pgfqpoint{1.286659in}{0.665713in}}%
\pgfpathlineto{\pgfqpoint{1.273390in}{0.671748in}}%
\pgfpathlineto{\pgfqpoint{1.260121in}{0.671397in}}%
\pgfpathlineto{\pgfqpoint{1.248563in}{0.664303in}}%
\pgfpathlineto{\pgfqpoint{1.246852in}{0.661009in}}%
\pgfpathlineto{\pgfqpoint{1.244677in}{0.651377in}}%
\pgfpathlineto{\pgfqpoint{1.244139in}{0.638450in}}%
\pgfpathlineto{\pgfqpoint{1.245137in}{0.625523in}}%
\pgfpathlineto{\pgfqpoint{1.246852in}{0.619003in}}%
\pgfpathlineto{\pgfqpoint{1.250783in}{0.612597in}}%
\pgfpathclose%
\pgfpathmoveto{\pgfqpoint{1.253822in}{0.625523in}}%
\pgfpathlineto{\pgfqpoint{1.251218in}{0.638450in}}%
\pgfpathlineto{\pgfqpoint{1.253100in}{0.651377in}}%
\pgfpathlineto{\pgfqpoint{1.260121in}{0.661222in}}%
\pgfpathlineto{\pgfqpoint{1.273390in}{0.662403in}}%
\pgfpathlineto{\pgfqpoint{1.283090in}{0.651377in}}%
\pgfpathlineto{\pgfqpoint{1.285314in}{0.638450in}}%
\pgfpathlineto{\pgfqpoint{1.282245in}{0.625523in}}%
\pgfpathlineto{\pgfqpoint{1.273390in}{0.616227in}}%
\pgfpathlineto{\pgfqpoint{1.260121in}{0.617392in}}%
\pgfpathclose%
\pgfusepath{fill}%
\end{pgfscope}%
\begin{pgfscope}%
\pgfpathrectangle{\pgfqpoint{0.211875in}{0.211875in}}{\pgfqpoint{1.313625in}{1.279725in}}%
\pgfusepath{clip}%
\pgfsetbuttcap%
\pgfsetroundjoin%
\definecolor{currentfill}{rgb}{0.961115,0.566634,0.405693}%
\pgfsetfillcolor{currentfill}%
\pgfsetlinewidth{0.000000pt}%
\definecolor{currentstroke}{rgb}{0.000000,0.000000,0.000000}%
\pgfsetstrokecolor{currentstroke}%
\pgfsetdash{}{0pt}%
\pgfpathmoveto{\pgfqpoint{1.379542in}{0.607777in}}%
\pgfpathlineto{\pgfqpoint{1.392811in}{0.607972in}}%
\pgfpathlineto{\pgfqpoint{1.401429in}{0.612597in}}%
\pgfpathlineto{\pgfqpoint{1.406080in}{0.619092in}}%
\pgfpathlineto{\pgfqpoint{1.407969in}{0.625523in}}%
\pgfpathlineto{\pgfqpoint{1.409077in}{0.638450in}}%
\pgfpathlineto{\pgfqpoint{1.408434in}{0.651377in}}%
\pgfpathlineto{\pgfqpoint{1.406080in}{0.660571in}}%
\pgfpathlineto{\pgfqpoint{1.403782in}{0.664303in}}%
\pgfpathlineto{\pgfqpoint{1.392811in}{0.670602in}}%
\pgfpathlineto{\pgfqpoint{1.379542in}{0.670797in}}%
\pgfpathlineto{\pgfqpoint{1.367469in}{0.664303in}}%
\pgfpathlineto{\pgfqpoint{1.366273in}{0.662533in}}%
\pgfpathlineto{\pgfqpoint{1.363260in}{0.651377in}}%
\pgfpathlineto{\pgfqpoint{1.362646in}{0.638450in}}%
\pgfpathlineto{\pgfqpoint{1.363730in}{0.625523in}}%
\pgfpathlineto{\pgfqpoint{1.366273in}{0.617289in}}%
\pgfpathlineto{\pgfqpoint{1.369950in}{0.612597in}}%
\pgfpathclose%
\pgfpathmoveto{\pgfqpoint{1.372698in}{0.625523in}}%
\pgfpathlineto{\pgfqpoint{1.369718in}{0.638450in}}%
\pgfpathlineto{\pgfqpoint{1.371898in}{0.651377in}}%
\pgfpathlineto{\pgfqpoint{1.379542in}{0.660804in}}%
\pgfpathlineto{\pgfqpoint{1.392811in}{0.660439in}}%
\pgfpathlineto{\pgfqpoint{1.399789in}{0.651377in}}%
\pgfpathlineto{\pgfqpoint{1.401916in}{0.638450in}}%
\pgfpathlineto{\pgfqpoint{1.399022in}{0.625523in}}%
\pgfpathlineto{\pgfqpoint{1.392811in}{0.618086in}}%
\pgfpathlineto{\pgfqpoint{1.379542in}{0.617739in}}%
\pgfpathclose%
\pgfusepath{fill}%
\end{pgfscope}%
\begin{pgfscope}%
\pgfpathrectangle{\pgfqpoint{0.211875in}{0.211875in}}{\pgfqpoint{1.313625in}{1.279725in}}%
\pgfusepath{clip}%
\pgfsetbuttcap%
\pgfsetroundjoin%
\definecolor{currentfill}{rgb}{0.961115,0.566634,0.405693}%
\pgfsetfillcolor{currentfill}%
\pgfsetlinewidth{0.000000pt}%
\definecolor{currentstroke}{rgb}{0.000000,0.000000,0.000000}%
\pgfsetstrokecolor{currentstroke}%
\pgfsetdash{}{0pt}%
\pgfpathmoveto{\pgfqpoint{1.498962in}{0.608901in}}%
\pgfpathlineto{\pgfqpoint{1.512231in}{0.609838in}}%
\pgfpathlineto{\pgfqpoint{1.516765in}{0.612597in}}%
\pgfpathlineto{\pgfqpoint{1.524580in}{0.625523in}}%
\pgfpathlineto{\pgfqpoint{1.525500in}{0.632009in}}%
\pgfpathlineto{\pgfqpoint{1.525500in}{0.638450in}}%
\pgfpathlineto{\pgfqpoint{1.525500in}{0.648402in}}%
\pgfpathlineto{\pgfqpoint{1.525229in}{0.651377in}}%
\pgfpathlineto{\pgfqpoint{1.518961in}{0.664303in}}%
\pgfpathlineto{\pgfqpoint{1.512231in}{0.668690in}}%
\pgfpathlineto{\pgfqpoint{1.498962in}{0.669586in}}%
\pgfpathlineto{\pgfqpoint{1.487635in}{0.664303in}}%
\pgfpathlineto{\pgfqpoint{1.485693in}{0.662071in}}%
\pgfpathlineto{\pgfqpoint{1.482316in}{0.651377in}}%
\pgfpathlineto{\pgfqpoint{1.481565in}{0.638450in}}%
\pgfpathlineto{\pgfqpoint{1.482804in}{0.625523in}}%
\pgfpathlineto{\pgfqpoint{1.485693in}{0.617412in}}%
\pgfpathlineto{\pgfqpoint{1.490486in}{0.612597in}}%
\pgfpathclose%
\pgfpathmoveto{\pgfqpoint{1.492386in}{0.625523in}}%
\pgfpathlineto{\pgfqpoint{1.488824in}{0.638450in}}%
\pgfpathlineto{\pgfqpoint{1.491477in}{0.651377in}}%
\pgfpathlineto{\pgfqpoint{1.498962in}{0.659418in}}%
\pgfpathlineto{\pgfqpoint{1.512231in}{0.657283in}}%
\pgfpathlineto{\pgfqpoint{1.516245in}{0.651377in}}%
\pgfpathlineto{\pgfqpoint{1.518373in}{0.638450in}}%
\pgfpathlineto{\pgfqpoint{1.515532in}{0.625523in}}%
\pgfpathlineto{\pgfqpoint{1.512231in}{0.621054in}}%
\pgfpathlineto{\pgfqpoint{1.498962in}{0.618998in}}%
\pgfpathclose%
\pgfusepath{fill}%
\end{pgfscope}%
\begin{pgfscope}%
\pgfpathrectangle{\pgfqpoint{0.211875in}{0.211875in}}{\pgfqpoint{1.313625in}{1.279725in}}%
\pgfusepath{clip}%
\pgfsetbuttcap%
\pgfsetroundjoin%
\definecolor{currentfill}{rgb}{0.961115,0.566634,0.405693}%
\pgfsetfillcolor{currentfill}%
\pgfsetlinewidth{0.000000pt}%
\definecolor{currentstroke}{rgb}{0.000000,0.000000,0.000000}%
\pgfsetstrokecolor{currentstroke}%
\pgfsetdash{}{0pt}%
\pgfpathmoveto{\pgfqpoint{0.318027in}{0.623981in}}%
\pgfpathlineto{\pgfqpoint{0.331295in}{0.623198in}}%
\pgfpathlineto{\pgfqpoint{0.333394in}{0.625523in}}%
\pgfpathlineto{\pgfqpoint{0.337057in}{0.638450in}}%
\pgfpathlineto{\pgfqpoint{0.334252in}{0.651377in}}%
\pgfpathlineto{\pgfqpoint{0.331295in}{0.654928in}}%
\pgfpathlineto{\pgfqpoint{0.318027in}{0.654091in}}%
\pgfpathlineto{\pgfqpoint{0.315986in}{0.651377in}}%
\pgfpathlineto{\pgfqpoint{0.313333in}{0.638450in}}%
\pgfpathlineto{\pgfqpoint{0.316770in}{0.625523in}}%
\pgfpathclose%
\pgfusepath{fill}%
\end{pgfscope}%
\begin{pgfscope}%
\pgfpathrectangle{\pgfqpoint{0.211875in}{0.211875in}}{\pgfqpoint{1.313625in}{1.279725in}}%
\pgfusepath{clip}%
\pgfsetbuttcap%
\pgfsetroundjoin%
\definecolor{currentfill}{rgb}{0.961115,0.566634,0.405693}%
\pgfsetfillcolor{currentfill}%
\pgfsetlinewidth{0.000000pt}%
\definecolor{currentstroke}{rgb}{0.000000,0.000000,0.000000}%
\pgfsetstrokecolor{currentstroke}%
\pgfsetdash{}{0pt}%
\pgfpathmoveto{\pgfqpoint{0.437447in}{0.618131in}}%
\pgfpathlineto{\pgfqpoint{0.450716in}{0.618981in}}%
\pgfpathlineto{\pgfqpoint{0.455907in}{0.625523in}}%
\pgfpathlineto{\pgfqpoint{0.458706in}{0.638450in}}%
\pgfpathlineto{\pgfqpoint{0.456655in}{0.651377in}}%
\pgfpathlineto{\pgfqpoint{0.450716in}{0.659509in}}%
\pgfpathlineto{\pgfqpoint{0.437447in}{0.660355in}}%
\pgfpathlineto{\pgfqpoint{0.429760in}{0.651377in}}%
\pgfpathlineto{\pgfqpoint{0.427352in}{0.638450in}}%
\pgfpathlineto{\pgfqpoint{0.430596in}{0.625523in}}%
\pgfpathclose%
\pgfpathmoveto{\pgfqpoint{0.436744in}{0.638450in}}%
\pgfpathlineto{\pgfqpoint{0.437447in}{0.641207in}}%
\pgfpathlineto{\pgfqpoint{0.445477in}{0.638450in}}%
\pgfpathlineto{\pgfqpoint{0.437447in}{0.636221in}}%
\pgfpathclose%
\pgfusepath{fill}%
\end{pgfscope}%
\begin{pgfscope}%
\pgfpathrectangle{\pgfqpoint{0.211875in}{0.211875in}}{\pgfqpoint{1.313625in}{1.279725in}}%
\pgfusepath{clip}%
\pgfsetbuttcap%
\pgfsetroundjoin%
\definecolor{currentfill}{rgb}{0.961115,0.566634,0.405693}%
\pgfsetfillcolor{currentfill}%
\pgfsetlinewidth{0.000000pt}%
\definecolor{currentstroke}{rgb}{0.000000,0.000000,0.000000}%
\pgfsetstrokecolor{currentstroke}%
\pgfsetdash{}{0pt}%
\pgfpathmoveto{\pgfqpoint{0.556867in}{0.613463in}}%
\pgfpathlineto{\pgfqpoint{0.570136in}{0.615481in}}%
\pgfpathlineto{\pgfqpoint{0.577191in}{0.625523in}}%
\pgfpathlineto{\pgfqpoint{0.579367in}{0.638450in}}%
\pgfpathlineto{\pgfqpoint{0.577857in}{0.651377in}}%
\pgfpathlineto{\pgfqpoint{0.570136in}{0.663347in}}%
\pgfpathlineto{\pgfqpoint{0.564169in}{0.664303in}}%
\pgfpathlineto{\pgfqpoint{0.556867in}{0.665040in}}%
\pgfpathlineto{\pgfqpoint{0.555245in}{0.664303in}}%
\pgfpathlineto{\pgfqpoint{0.543598in}{0.652213in}}%
\pgfpathlineto{\pgfqpoint{0.543313in}{0.651377in}}%
\pgfpathlineto{\pgfqpoint{0.542078in}{0.638450in}}%
\pgfpathlineto{\pgfqpoint{0.543598in}{0.627013in}}%
\pgfpathlineto{\pgfqpoint{0.544001in}{0.625523in}}%
\pgfpathclose%
\pgfpathmoveto{\pgfqpoint{0.556074in}{0.625523in}}%
\pgfpathlineto{\pgfqpoint{0.551484in}{0.638450in}}%
\pgfpathlineto{\pgfqpoint{0.555078in}{0.651377in}}%
\pgfpathlineto{\pgfqpoint{0.556867in}{0.653191in}}%
\pgfpathlineto{\pgfqpoint{0.564108in}{0.651377in}}%
\pgfpathlineto{\pgfqpoint{0.570136in}{0.646299in}}%
\pgfpathlineto{\pgfqpoint{0.571624in}{0.638450in}}%
\pgfpathlineto{\pgfqpoint{0.570136in}{0.632206in}}%
\pgfpathlineto{\pgfqpoint{0.560079in}{0.625523in}}%
\pgfpathlineto{\pgfqpoint{0.556867in}{0.624780in}}%
\pgfpathclose%
\pgfusepath{fill}%
\end{pgfscope}%
\begin{pgfscope}%
\pgfpathrectangle{\pgfqpoint{0.211875in}{0.211875in}}{\pgfqpoint{1.313625in}{1.279725in}}%
\pgfusepath{clip}%
\pgfsetbuttcap%
\pgfsetroundjoin%
\definecolor{currentfill}{rgb}{0.961115,0.566634,0.405693}%
\pgfsetfillcolor{currentfill}%
\pgfsetlinewidth{0.000000pt}%
\definecolor{currentstroke}{rgb}{0.000000,0.000000,0.000000}%
\pgfsetstrokecolor{currentstroke}%
\pgfsetdash{}{0pt}%
\pgfpathmoveto{\pgfqpoint{0.214022in}{0.638450in}}%
\pgfpathlineto{\pgfqpoint{0.211875in}{0.646049in}}%
\pgfpathlineto{\pgfqpoint{0.211875in}{0.638450in}}%
\pgfpathlineto{\pgfqpoint{0.211875in}{0.632303in}}%
\pgfpathclose%
\pgfusepath{fill}%
\end{pgfscope}%
\begin{pgfscope}%
\pgfpathrectangle{\pgfqpoint{0.211875in}{0.211875in}}{\pgfqpoint{1.313625in}{1.279725in}}%
\pgfusepath{clip}%
\pgfsetbuttcap%
\pgfsetroundjoin%
\definecolor{currentfill}{rgb}{0.961115,0.566634,0.405693}%
\pgfsetfillcolor{currentfill}%
\pgfsetlinewidth{0.000000pt}%
\definecolor{currentstroke}{rgb}{0.000000,0.000000,0.000000}%
\pgfsetstrokecolor{currentstroke}%
\pgfsetdash{}{0pt}%
\pgfpathmoveto{\pgfqpoint{0.729364in}{0.690030in}}%
\pgfpathlineto{\pgfqpoint{0.742633in}{0.689216in}}%
\pgfpathlineto{\pgfqpoint{0.745530in}{0.690156in}}%
\pgfpathlineto{\pgfqpoint{0.755902in}{0.697348in}}%
\pgfpathlineto{\pgfqpoint{0.758388in}{0.703083in}}%
\pgfpathlineto{\pgfqpoint{0.760213in}{0.716009in}}%
\pgfpathlineto{\pgfqpoint{0.760019in}{0.728936in}}%
\pgfpathlineto{\pgfqpoint{0.757291in}{0.741862in}}%
\pgfpathlineto{\pgfqpoint{0.755902in}{0.744342in}}%
\pgfpathlineto{\pgfqpoint{0.742633in}{0.751334in}}%
\pgfpathlineto{\pgfqpoint{0.729364in}{0.750471in}}%
\pgfpathlineto{\pgfqpoint{0.718865in}{0.741862in}}%
\pgfpathlineto{\pgfqpoint{0.716095in}{0.733692in}}%
\pgfpathlineto{\pgfqpoint{0.715281in}{0.728936in}}%
\pgfpathlineto{\pgfqpoint{0.715070in}{0.716009in}}%
\pgfpathlineto{\pgfqpoint{0.716095in}{0.708183in}}%
\pgfpathlineto{\pgfqpoint{0.717252in}{0.703083in}}%
\pgfpathlineto{\pgfqpoint{0.729085in}{0.690156in}}%
\pgfpathclose%
\pgfpathmoveto{\pgfqpoint{0.727234in}{0.703083in}}%
\pgfpathlineto{\pgfqpoint{0.722734in}{0.716009in}}%
\pgfpathlineto{\pgfqpoint{0.723455in}{0.728936in}}%
\pgfpathlineto{\pgfqpoint{0.729364in}{0.739865in}}%
\pgfpathlineto{\pgfqpoint{0.740838in}{0.741862in}}%
\pgfpathlineto{\pgfqpoint{0.742633in}{0.742047in}}%
\pgfpathlineto{\pgfqpoint{0.742936in}{0.741862in}}%
\pgfpathlineto{\pgfqpoint{0.752137in}{0.728936in}}%
\pgfpathlineto{\pgfqpoint{0.753008in}{0.716009in}}%
\pgfpathlineto{\pgfqpoint{0.747482in}{0.703083in}}%
\pgfpathlineto{\pgfqpoint{0.742633in}{0.699157in}}%
\pgfpathlineto{\pgfqpoint{0.729364in}{0.700782in}}%
\pgfpathclose%
\pgfusepath{fill}%
\end{pgfscope}%
\begin{pgfscope}%
\pgfpathrectangle{\pgfqpoint{0.211875in}{0.211875in}}{\pgfqpoint{1.313625in}{1.279725in}}%
\pgfusepath{clip}%
\pgfsetbuttcap%
\pgfsetroundjoin%
\definecolor{currentfill}{rgb}{0.961115,0.566634,0.405693}%
\pgfsetfillcolor{currentfill}%
\pgfsetlinewidth{0.000000pt}%
\definecolor{currentstroke}{rgb}{0.000000,0.000000,0.000000}%
\pgfsetstrokecolor{currentstroke}%
\pgfsetdash{}{0pt}%
\pgfpathmoveto{\pgfqpoint{0.848784in}{0.687986in}}%
\pgfpathlineto{\pgfqpoint{0.862053in}{0.687837in}}%
\pgfpathlineto{\pgfqpoint{0.868234in}{0.690156in}}%
\pgfpathlineto{\pgfqpoint{0.875322in}{0.696441in}}%
\pgfpathlineto{\pgfqpoint{0.877809in}{0.703083in}}%
\pgfpathlineto{\pgfqpoint{0.879331in}{0.716009in}}%
\pgfpathlineto{\pgfqpoint{0.879200in}{0.728936in}}%
\pgfpathlineto{\pgfqpoint{0.877026in}{0.741862in}}%
\pgfpathlineto{\pgfqpoint{0.875322in}{0.745421in}}%
\pgfpathlineto{\pgfqpoint{0.862053in}{0.752944in}}%
\pgfpathlineto{\pgfqpoint{0.848784in}{0.752773in}}%
\pgfpathlineto{\pgfqpoint{0.835515in}{0.744440in}}%
\pgfpathlineto{\pgfqpoint{0.834307in}{0.741862in}}%
\pgfpathlineto{\pgfqpoint{0.832014in}{0.728936in}}%
\pgfpathlineto{\pgfqpoint{0.831864in}{0.716009in}}%
\pgfpathlineto{\pgfqpoint{0.833449in}{0.703083in}}%
\pgfpathlineto{\pgfqpoint{0.835515in}{0.697469in}}%
\pgfpathlineto{\pgfqpoint{0.843322in}{0.690156in}}%
\pgfpathclose%
\pgfpathmoveto{\pgfqpoint{0.843323in}{0.703083in}}%
\pgfpathlineto{\pgfqpoint{0.838824in}{0.716009in}}%
\pgfpathlineto{\pgfqpoint{0.839518in}{0.728936in}}%
\pgfpathlineto{\pgfqpoint{0.846876in}{0.741862in}}%
\pgfpathlineto{\pgfqpoint{0.848784in}{0.743242in}}%
\pgfpathlineto{\pgfqpoint{0.862053in}{0.743406in}}%
\pgfpathlineto{\pgfqpoint{0.864265in}{0.741862in}}%
\pgfpathlineto{\pgfqpoint{0.871704in}{0.728936in}}%
\pgfpathlineto{\pgfqpoint{0.872393in}{0.716009in}}%
\pgfpathlineto{\pgfqpoint{0.867863in}{0.703083in}}%
\pgfpathlineto{\pgfqpoint{0.862053in}{0.697692in}}%
\pgfpathlineto{\pgfqpoint{0.848784in}{0.697871in}}%
\pgfpathclose%
\pgfusepath{fill}%
\end{pgfscope}%
\begin{pgfscope}%
\pgfpathrectangle{\pgfqpoint{0.211875in}{0.211875in}}{\pgfqpoint{1.313625in}{1.279725in}}%
\pgfusepath{clip}%
\pgfsetbuttcap%
\pgfsetroundjoin%
\definecolor{currentfill}{rgb}{0.961115,0.566634,0.405693}%
\pgfsetfillcolor{currentfill}%
\pgfsetlinewidth{0.000000pt}%
\definecolor{currentstroke}{rgb}{0.000000,0.000000,0.000000}%
\pgfsetstrokecolor{currentstroke}%
\pgfsetdash{}{0pt}%
\pgfpathmoveto{\pgfqpoint{0.968205in}{0.686658in}}%
\pgfpathlineto{\pgfqpoint{0.981473in}{0.687013in}}%
\pgfpathlineto{\pgfqpoint{0.988806in}{0.690156in}}%
\pgfpathlineto{\pgfqpoint{0.994742in}{0.696932in}}%
\pgfpathlineto{\pgfqpoint{0.996694in}{0.703083in}}%
\pgfpathlineto{\pgfqpoint{0.998028in}{0.716009in}}%
\pgfpathlineto{\pgfqpoint{0.997936in}{0.728936in}}%
\pgfpathlineto{\pgfqpoint{0.996105in}{0.741862in}}%
\pgfpathlineto{\pgfqpoint{0.994742in}{0.745269in}}%
\pgfpathlineto{\pgfqpoint{0.981473in}{0.753933in}}%
\pgfpathlineto{\pgfqpoint{0.968205in}{0.754260in}}%
\pgfpathlineto{\pgfqpoint{0.954936in}{0.748897in}}%
\pgfpathlineto{\pgfqpoint{0.951079in}{0.741862in}}%
\pgfpathlineto{\pgfqpoint{0.949120in}{0.728936in}}%
\pgfpathlineto{\pgfqpoint{0.949013in}{0.716009in}}%
\pgfpathlineto{\pgfqpoint{0.950424in}{0.703083in}}%
\pgfpathlineto{\pgfqpoint{0.954936in}{0.692491in}}%
\pgfpathlineto{\pgfqpoint{0.958099in}{0.690156in}}%
\pgfpathclose%
\pgfpathmoveto{\pgfqpoint{0.959659in}{0.703083in}}%
\pgfpathlineto{\pgfqpoint{0.954977in}{0.716009in}}%
\pgfpathlineto{\pgfqpoint{0.955676in}{0.728936in}}%
\pgfpathlineto{\pgfqpoint{0.963270in}{0.741862in}}%
\pgfpathlineto{\pgfqpoint{0.968205in}{0.744977in}}%
\pgfpathlineto{\pgfqpoint{0.981473in}{0.744064in}}%
\pgfpathlineto{\pgfqpoint{0.984244in}{0.741862in}}%
\pgfpathlineto{\pgfqpoint{0.990551in}{0.728936in}}%
\pgfpathlineto{\pgfqpoint{0.991124in}{0.716009in}}%
\pgfpathlineto{\pgfqpoint{0.987245in}{0.703083in}}%
\pgfpathlineto{\pgfqpoint{0.981473in}{0.697006in}}%
\pgfpathlineto{\pgfqpoint{0.968205in}{0.695944in}}%
\pgfpathclose%
\pgfusepath{fill}%
\end{pgfscope}%
\begin{pgfscope}%
\pgfpathrectangle{\pgfqpoint{0.211875in}{0.211875in}}{\pgfqpoint{1.313625in}{1.279725in}}%
\pgfusepath{clip}%
\pgfsetbuttcap%
\pgfsetroundjoin%
\definecolor{currentfill}{rgb}{0.961115,0.566634,0.405693}%
\pgfsetfillcolor{currentfill}%
\pgfsetlinewidth{0.000000pt}%
\definecolor{currentstroke}{rgb}{0.000000,0.000000,0.000000}%
\pgfsetstrokecolor{currentstroke}%
\pgfsetdash{}{0pt}%
\pgfpathmoveto{\pgfqpoint{1.074356in}{0.689779in}}%
\pgfpathlineto{\pgfqpoint{1.087625in}{0.685972in}}%
\pgfpathlineto{\pgfqpoint{1.100894in}{0.686781in}}%
\pgfpathlineto{\pgfqpoint{1.107835in}{0.690156in}}%
\pgfpathlineto{\pgfqpoint{1.114163in}{0.699637in}}%
\pgfpathlineto{\pgfqpoint{1.115064in}{0.703083in}}%
\pgfpathlineto{\pgfqpoint{1.116315in}{0.716009in}}%
\pgfpathlineto{\pgfqpoint{1.116240in}{0.728936in}}%
\pgfpathlineto{\pgfqpoint{1.114560in}{0.741862in}}%
\pgfpathlineto{\pgfqpoint{1.114163in}{0.743091in}}%
\pgfpathlineto{\pgfqpoint{1.100894in}{0.754256in}}%
\pgfpathlineto{\pgfqpoint{1.092013in}{0.754789in}}%
\pgfpathlineto{\pgfqpoint{1.087625in}{0.754996in}}%
\pgfpathlineto{\pgfqpoint{1.086343in}{0.754789in}}%
\pgfpathlineto{\pgfqpoint{1.074356in}{0.751321in}}%
\pgfpathlineto{\pgfqpoint{1.068384in}{0.741862in}}%
\pgfpathlineto{\pgfqpoint{1.066585in}{0.728936in}}%
\pgfpathlineto{\pgfqpoint{1.066502in}{0.716009in}}%
\pgfpathlineto{\pgfqpoint{1.067835in}{0.703083in}}%
\pgfpathlineto{\pgfqpoint{1.073911in}{0.690156in}}%
\pgfpathclose%
\pgfpathmoveto{\pgfqpoint{1.076277in}{0.703083in}}%
\pgfpathlineto{\pgfqpoint{1.074356in}{0.707302in}}%
\pgfpathlineto{\pgfqpoint{1.072608in}{0.716009in}}%
\pgfpathlineto{\pgfqpoint{1.073017in}{0.728936in}}%
\pgfpathlineto{\pgfqpoint{1.074356in}{0.733777in}}%
\pgfpathlineto{\pgfqpoint{1.080174in}{0.741862in}}%
\pgfpathlineto{\pgfqpoint{1.087625in}{0.745917in}}%
\pgfpathlineto{\pgfqpoint{1.100894in}{0.743961in}}%
\pgfpathlineto{\pgfqpoint{1.103229in}{0.741862in}}%
\pgfpathlineto{\pgfqpoint{1.108853in}{0.728936in}}%
\pgfpathlineto{\pgfqpoint{1.109360in}{0.716009in}}%
\pgfpathlineto{\pgfqpoint{1.105880in}{0.703083in}}%
\pgfpathlineto{\pgfqpoint{1.100894in}{0.697161in}}%
\pgfpathlineto{\pgfqpoint{1.087625in}{0.694895in}}%
\pgfpathclose%
\pgfusepath{fill}%
\end{pgfscope}%
\begin{pgfscope}%
\pgfpathrectangle{\pgfqpoint{0.211875in}{0.211875in}}{\pgfqpoint{1.313625in}{1.279725in}}%
\pgfusepath{clip}%
\pgfsetbuttcap%
\pgfsetroundjoin%
\definecolor{currentfill}{rgb}{0.961115,0.566634,0.405693}%
\pgfsetfillcolor{currentfill}%
\pgfsetlinewidth{0.000000pt}%
\definecolor{currentstroke}{rgb}{0.000000,0.000000,0.000000}%
\pgfsetstrokecolor{currentstroke}%
\pgfsetdash{}{0pt}%
\pgfpathmoveto{\pgfqpoint{1.193777in}{0.688764in}}%
\pgfpathlineto{\pgfqpoint{1.207045in}{0.685879in}}%
\pgfpathlineto{\pgfqpoint{1.220314in}{0.687213in}}%
\pgfpathlineto{\pgfqpoint{1.225674in}{0.690156in}}%
\pgfpathlineto{\pgfqpoint{1.232768in}{0.703083in}}%
\pgfpathlineto{\pgfqpoint{1.233583in}{0.709160in}}%
\pgfpathlineto{\pgfqpoint{1.234188in}{0.716009in}}%
\pgfpathlineto{\pgfqpoint{1.234109in}{0.728936in}}%
\pgfpathlineto{\pgfqpoint{1.233583in}{0.733926in}}%
\pgfpathlineto{\pgfqpoint{1.232127in}{0.741862in}}%
\pgfpathlineto{\pgfqpoint{1.220314in}{0.753831in}}%
\pgfpathlineto{\pgfqpoint{1.210884in}{0.754789in}}%
\pgfpathlineto{\pgfqpoint{1.207045in}{0.755077in}}%
\pgfpathlineto{\pgfqpoint{1.204893in}{0.754789in}}%
\pgfpathlineto{\pgfqpoint{1.193777in}{0.752339in}}%
\pgfpathlineto{\pgfqpoint{1.186235in}{0.741862in}}%
\pgfpathlineto{\pgfqpoint{1.184410in}{0.728936in}}%
\pgfpathlineto{\pgfqpoint{1.184328in}{0.716009in}}%
\pgfpathlineto{\pgfqpoint{1.185687in}{0.703083in}}%
\pgfpathlineto{\pgfqpoint{1.191905in}{0.690156in}}%
\pgfpathclose%
\pgfpathmoveto{\pgfqpoint{1.193525in}{0.703083in}}%
\pgfpathlineto{\pgfqpoint{1.190643in}{0.716009in}}%
\pgfpathlineto{\pgfqpoint{1.191063in}{0.728936in}}%
\pgfpathlineto{\pgfqpoint{1.193777in}{0.737539in}}%
\pgfpathlineto{\pgfqpoint{1.197784in}{0.741862in}}%
\pgfpathlineto{\pgfqpoint{1.207045in}{0.746120in}}%
\pgfpathlineto{\pgfqpoint{1.220314in}{0.742998in}}%
\pgfpathlineto{\pgfqpoint{1.221435in}{0.741862in}}%
\pgfpathlineto{\pgfqpoint{1.226716in}{0.728936in}}%
\pgfpathlineto{\pgfqpoint{1.227194in}{0.716009in}}%
\pgfpathlineto{\pgfqpoint{1.223925in}{0.703083in}}%
\pgfpathlineto{\pgfqpoint{1.220314in}{0.698265in}}%
\pgfpathlineto{\pgfqpoint{1.207045in}{0.694659in}}%
\pgfpathlineto{\pgfqpoint{1.193777in}{0.702626in}}%
\pgfpathclose%
\pgfusepath{fill}%
\end{pgfscope}%
\begin{pgfscope}%
\pgfpathrectangle{\pgfqpoint{0.211875in}{0.211875in}}{\pgfqpoint{1.313625in}{1.279725in}}%
\pgfusepath{clip}%
\pgfsetbuttcap%
\pgfsetroundjoin%
\definecolor{currentfill}{rgb}{0.961115,0.566634,0.405693}%
\pgfsetfillcolor{currentfill}%
\pgfsetlinewidth{0.000000pt}%
\definecolor{currentstroke}{rgb}{0.000000,0.000000,0.000000}%
\pgfsetstrokecolor{currentstroke}%
\pgfsetdash{}{0pt}%
\pgfpathmoveto{\pgfqpoint{1.313197in}{0.688670in}}%
\pgfpathlineto{\pgfqpoint{1.326466in}{0.686357in}}%
\pgfpathlineto{\pgfqpoint{1.339735in}{0.688420in}}%
\pgfpathlineto{\pgfqpoint{1.342539in}{0.690156in}}%
\pgfpathlineto{\pgfqpoint{1.349820in}{0.703083in}}%
\pgfpathlineto{\pgfqpoint{1.351419in}{0.716009in}}%
\pgfpathlineto{\pgfqpoint{1.351298in}{0.728936in}}%
\pgfpathlineto{\pgfqpoint{1.349079in}{0.741862in}}%
\pgfpathlineto{\pgfqpoint{1.339735in}{0.752527in}}%
\pgfpathlineto{\pgfqpoint{1.326466in}{0.754558in}}%
\pgfpathlineto{\pgfqpoint{1.313197in}{0.752318in}}%
\pgfpathlineto{\pgfqpoint{1.304670in}{0.741862in}}%
\pgfpathlineto{\pgfqpoint{1.302611in}{0.728936in}}%
\pgfpathlineto{\pgfqpoint{1.302507in}{0.716009in}}%
\pgfpathlineto{\pgfqpoint{1.304008in}{0.703083in}}%
\pgfpathlineto{\pgfqpoint{1.310936in}{0.690156in}}%
\pgfpathclose%
\pgfpathmoveto{\pgfqpoint{1.312185in}{0.703083in}}%
\pgfpathlineto{\pgfqpoint{1.309090in}{0.716009in}}%
\pgfpathlineto{\pgfqpoint{1.309547in}{0.728936in}}%
\pgfpathlineto{\pgfqpoint{1.313197in}{0.739156in}}%
\pgfpathlineto{\pgfqpoint{1.316501in}{0.741862in}}%
\pgfpathlineto{\pgfqpoint{1.326466in}{0.745619in}}%
\pgfpathlineto{\pgfqpoint{1.337760in}{0.741862in}}%
\pgfpathlineto{\pgfqpoint{1.339735in}{0.740511in}}%
\pgfpathlineto{\pgfqpoint{1.344201in}{0.728936in}}%
\pgfpathlineto{\pgfqpoint{1.344680in}{0.716009in}}%
\pgfpathlineto{\pgfqpoint{1.341471in}{0.703083in}}%
\pgfpathlineto{\pgfqpoint{1.339735in}{0.700482in}}%
\pgfpathlineto{\pgfqpoint{1.326466in}{0.695203in}}%
\pgfpathlineto{\pgfqpoint{1.313197in}{0.701455in}}%
\pgfpathclose%
\pgfusepath{fill}%
\end{pgfscope}%
\begin{pgfscope}%
\pgfpathrectangle{\pgfqpoint{0.211875in}{0.211875in}}{\pgfqpoint{1.313625in}{1.279725in}}%
\pgfusepath{clip}%
\pgfsetbuttcap%
\pgfsetroundjoin%
\definecolor{currentfill}{rgb}{0.961115,0.566634,0.405693}%
\pgfsetfillcolor{currentfill}%
\pgfsetlinewidth{0.000000pt}%
\definecolor{currentstroke}{rgb}{0.000000,0.000000,0.000000}%
\pgfsetstrokecolor{currentstroke}%
\pgfsetdash{}{0pt}%
\pgfpathmoveto{\pgfqpoint{1.432617in}{0.689309in}}%
\pgfpathlineto{\pgfqpoint{1.445886in}{0.687407in}}%
\pgfpathlineto{\pgfqpoint{1.457829in}{0.690156in}}%
\pgfpathlineto{\pgfqpoint{1.459155in}{0.690678in}}%
\pgfpathlineto{\pgfqpoint{1.466529in}{0.703083in}}%
\pgfpathlineto{\pgfqpoint{1.468273in}{0.716009in}}%
\pgfpathlineto{\pgfqpoint{1.468109in}{0.728936in}}%
\pgfpathlineto{\pgfqpoint{1.465585in}{0.741862in}}%
\pgfpathlineto{\pgfqpoint{1.459155in}{0.750144in}}%
\pgfpathlineto{\pgfqpoint{1.445886in}{0.753360in}}%
\pgfpathlineto{\pgfqpoint{1.432617in}{0.751480in}}%
\pgfpathlineto{\pgfqpoint{1.423763in}{0.741862in}}%
\pgfpathlineto{\pgfqpoint{1.421223in}{0.728936in}}%
\pgfpathlineto{\pgfqpoint{1.421069in}{0.716009in}}%
\pgfpathlineto{\pgfqpoint{1.422847in}{0.703083in}}%
\pgfpathlineto{\pgfqpoint{1.431160in}{0.690156in}}%
\pgfpathclose%
\pgfpathmoveto{\pgfqpoint{1.431454in}{0.703083in}}%
\pgfpathlineto{\pgfqpoint{1.427991in}{0.716009in}}%
\pgfpathlineto{\pgfqpoint{1.428518in}{0.728936in}}%
\pgfpathlineto{\pgfqpoint{1.432617in}{0.739114in}}%
\pgfpathlineto{\pgfqpoint{1.437225in}{0.741862in}}%
\pgfpathlineto{\pgfqpoint{1.445886in}{0.744414in}}%
\pgfpathlineto{\pgfqpoint{1.452034in}{0.741862in}}%
\pgfpathlineto{\pgfqpoint{1.459155in}{0.735328in}}%
\pgfpathlineto{\pgfqpoint{1.461342in}{0.728936in}}%
\pgfpathlineto{\pgfqpoint{1.461847in}{0.716009in}}%
\pgfpathlineto{\pgfqpoint{1.459155in}{0.705006in}}%
\pgfpathlineto{\pgfqpoint{1.457844in}{0.703083in}}%
\pgfpathlineto{\pgfqpoint{1.445886in}{0.696527in}}%
\pgfpathlineto{\pgfqpoint{1.432617in}{0.701417in}}%
\pgfpathclose%
\pgfusepath{fill}%
\end{pgfscope}%
\begin{pgfscope}%
\pgfpathrectangle{\pgfqpoint{0.211875in}{0.211875in}}{\pgfqpoint{1.313625in}{1.279725in}}%
\pgfusepath{clip}%
\pgfsetbuttcap%
\pgfsetroundjoin%
\definecolor{currentfill}{rgb}{0.961115,0.566634,0.405693}%
\pgfsetfillcolor{currentfill}%
\pgfsetlinewidth{0.000000pt}%
\definecolor{currentstroke}{rgb}{0.000000,0.000000,0.000000}%
\pgfsetstrokecolor{currentstroke}%
\pgfsetdash{}{0pt}%
\pgfpathmoveto{\pgfqpoint{0.264951in}{0.702882in}}%
\pgfpathlineto{\pgfqpoint{0.265493in}{0.703083in}}%
\pgfpathlineto{\pgfqpoint{0.278220in}{0.714689in}}%
\pgfpathlineto{\pgfqpoint{0.278603in}{0.716009in}}%
\pgfpathlineto{\pgfqpoint{0.278220in}{0.722628in}}%
\pgfpathlineto{\pgfqpoint{0.277067in}{0.728936in}}%
\pgfpathlineto{\pgfqpoint{0.264951in}{0.736704in}}%
\pgfpathlineto{\pgfqpoint{0.256936in}{0.728936in}}%
\pgfpathlineto{\pgfqpoint{0.255343in}{0.716009in}}%
\pgfpathlineto{\pgfqpoint{0.264602in}{0.703083in}}%
\pgfpathclose%
\pgfusepath{fill}%
\end{pgfscope}%
\begin{pgfscope}%
\pgfpathrectangle{\pgfqpoint{0.211875in}{0.211875in}}{\pgfqpoint{1.313625in}{1.279725in}}%
\pgfusepath{clip}%
\pgfsetbuttcap%
\pgfsetroundjoin%
\definecolor{currentfill}{rgb}{0.961115,0.566634,0.405693}%
\pgfsetfillcolor{currentfill}%
\pgfsetlinewidth{0.000000pt}%
\definecolor{currentstroke}{rgb}{0.000000,0.000000,0.000000}%
\pgfsetstrokecolor{currentstroke}%
\pgfsetdash{}{0pt}%
\pgfpathmoveto{\pgfqpoint{0.384371in}{0.698288in}}%
\pgfpathlineto{\pgfqpoint{0.394418in}{0.703083in}}%
\pgfpathlineto{\pgfqpoint{0.397640in}{0.707138in}}%
\pgfpathlineto{\pgfqpoint{0.399936in}{0.716009in}}%
\pgfpathlineto{\pgfqpoint{0.399373in}{0.728936in}}%
\pgfpathlineto{\pgfqpoint{0.397640in}{0.733706in}}%
\pgfpathlineto{\pgfqpoint{0.387047in}{0.741862in}}%
\pgfpathlineto{\pgfqpoint{0.384371in}{0.742809in}}%
\pgfpathlineto{\pgfqpoint{0.381711in}{0.741862in}}%
\pgfpathlineto{\pgfqpoint{0.371102in}{0.734107in}}%
\pgfpathlineto{\pgfqpoint{0.369141in}{0.728936in}}%
\pgfpathlineto{\pgfqpoint{0.368545in}{0.716009in}}%
\pgfpathlineto{\pgfqpoint{0.371102in}{0.706526in}}%
\pgfpathlineto{\pgfqpoint{0.374062in}{0.703083in}}%
\pgfpathclose%
\pgfpathmoveto{\pgfqpoint{0.381698in}{0.716009in}}%
\pgfpathlineto{\pgfqpoint{0.384124in}{0.728936in}}%
\pgfpathlineto{\pgfqpoint{0.384371in}{0.729130in}}%
\pgfpathlineto{\pgfqpoint{0.384607in}{0.728936in}}%
\pgfpathlineto{\pgfqpoint{0.386901in}{0.716009in}}%
\pgfpathlineto{\pgfqpoint{0.384371in}{0.713005in}}%
\pgfpathclose%
\pgfusepath{fill}%
\end{pgfscope}%
\begin{pgfscope}%
\pgfpathrectangle{\pgfqpoint{0.211875in}{0.211875in}}{\pgfqpoint{1.313625in}{1.279725in}}%
\pgfusepath{clip}%
\pgfsetbuttcap%
\pgfsetroundjoin%
\definecolor{currentfill}{rgb}{0.961115,0.566634,0.405693}%
\pgfsetfillcolor{currentfill}%
\pgfsetlinewidth{0.000000pt}%
\definecolor{currentstroke}{rgb}{0.000000,0.000000,0.000000}%
\pgfsetstrokecolor{currentstroke}%
\pgfsetdash{}{0pt}%
\pgfpathmoveto{\pgfqpoint{0.490523in}{0.698452in}}%
\pgfpathlineto{\pgfqpoint{0.503792in}{0.694475in}}%
\pgfpathlineto{\pgfqpoint{0.517061in}{0.701878in}}%
\pgfpathlineto{\pgfqpoint{0.517738in}{0.703083in}}%
\pgfpathlineto{\pgfqpoint{0.520580in}{0.716009in}}%
\pgfpathlineto{\pgfqpoint{0.520175in}{0.728936in}}%
\pgfpathlineto{\pgfqpoint{0.517061in}{0.738654in}}%
\pgfpathlineto{\pgfqpoint{0.513933in}{0.741862in}}%
\pgfpathlineto{\pgfqpoint{0.503792in}{0.746288in}}%
\pgfpathlineto{\pgfqpoint{0.490523in}{0.742825in}}%
\pgfpathlineto{\pgfqpoint{0.489598in}{0.741862in}}%
\pgfpathlineto{\pgfqpoint{0.484355in}{0.728936in}}%
\pgfpathlineto{\pgfqpoint{0.483872in}{0.716009in}}%
\pgfpathlineto{\pgfqpoint{0.487128in}{0.703083in}}%
\pgfpathclose%
\pgfpathmoveto{\pgfqpoint{0.491658in}{0.716009in}}%
\pgfpathlineto{\pgfqpoint{0.494398in}{0.728936in}}%
\pgfpathlineto{\pgfqpoint{0.503792in}{0.734586in}}%
\pgfpathlineto{\pgfqpoint{0.509391in}{0.728936in}}%
\pgfpathlineto{\pgfqpoint{0.510981in}{0.716009in}}%
\pgfpathlineto{\pgfqpoint{0.503792in}{0.705578in}}%
\pgfpathclose%
\pgfusepath{fill}%
\end{pgfscope}%
\begin{pgfscope}%
\pgfpathrectangle{\pgfqpoint{0.211875in}{0.211875in}}{\pgfqpoint{1.313625in}{1.279725in}}%
\pgfusepath{clip}%
\pgfsetbuttcap%
\pgfsetroundjoin%
\definecolor{currentfill}{rgb}{0.961115,0.566634,0.405693}%
\pgfsetfillcolor{currentfill}%
\pgfsetlinewidth{0.000000pt}%
\definecolor{currentstroke}{rgb}{0.000000,0.000000,0.000000}%
\pgfsetstrokecolor{currentstroke}%
\pgfsetdash{}{0pt}%
\pgfpathmoveto{\pgfqpoint{0.609943in}{0.693603in}}%
\pgfpathlineto{\pgfqpoint{0.623212in}{0.691383in}}%
\pgfpathlineto{\pgfqpoint{0.636481in}{0.699236in}}%
\pgfpathlineto{\pgfqpoint{0.638388in}{0.703083in}}%
\pgfpathlineto{\pgfqpoint{0.640646in}{0.716009in}}%
\pgfpathlineto{\pgfqpoint{0.640362in}{0.728936in}}%
\pgfpathlineto{\pgfqpoint{0.636842in}{0.741862in}}%
\pgfpathlineto{\pgfqpoint{0.636481in}{0.742420in}}%
\pgfpathlineto{\pgfqpoint{0.623212in}{0.749119in}}%
\pgfpathlineto{\pgfqpoint{0.609943in}{0.747219in}}%
\pgfpathlineto{\pgfqpoint{0.604153in}{0.741862in}}%
\pgfpathlineto{\pgfqpoint{0.599646in}{0.728936in}}%
\pgfpathlineto{\pgfqpoint{0.599264in}{0.716009in}}%
\pgfpathlineto{\pgfqpoint{0.602147in}{0.703083in}}%
\pgfpathclose%
\pgfpathmoveto{\pgfqpoint{0.616265in}{0.703083in}}%
\pgfpathlineto{\pgfqpoint{0.609943in}{0.706526in}}%
\pgfpathlineto{\pgfqpoint{0.606740in}{0.716009in}}%
\pgfpathlineto{\pgfqpoint{0.607514in}{0.728936in}}%
\pgfpathlineto{\pgfqpoint{0.609943in}{0.734002in}}%
\pgfpathlineto{\pgfqpoint{0.623212in}{0.738912in}}%
\pgfpathlineto{\pgfqpoint{0.631547in}{0.728936in}}%
\pgfpathlineto{\pgfqpoint{0.632700in}{0.716009in}}%
\pgfpathlineto{\pgfqpoint{0.625667in}{0.703083in}}%
\pgfpathlineto{\pgfqpoint{0.623212in}{0.701374in}}%
\pgfpathclose%
\pgfusepath{fill}%
\end{pgfscope}%
\begin{pgfscope}%
\pgfpathrectangle{\pgfqpoint{0.211875in}{0.211875in}}{\pgfqpoint{1.313625in}{1.279725in}}%
\pgfusepath{clip}%
\pgfsetbuttcap%
\pgfsetroundjoin%
\definecolor{currentfill}{rgb}{0.961115,0.566634,0.405693}%
\pgfsetfillcolor{currentfill}%
\pgfsetlinewidth{0.000000pt}%
\definecolor{currentstroke}{rgb}{0.000000,0.000000,0.000000}%
\pgfsetstrokecolor{currentstroke}%
\pgfsetdash{}{0pt}%
\pgfpathmoveto{\pgfqpoint{0.915129in}{0.766642in}}%
\pgfpathlineto{\pgfqpoint{0.923076in}{0.767715in}}%
\pgfpathlineto{\pgfqpoint{0.928398in}{0.768842in}}%
\pgfpathlineto{\pgfqpoint{0.937386in}{0.780642in}}%
\pgfpathlineto{\pgfqpoint{0.939340in}{0.793568in}}%
\pgfpathlineto{\pgfqpoint{0.939578in}{0.806495in}}%
\pgfpathlineto{\pgfqpoint{0.938418in}{0.819421in}}%
\pgfpathlineto{\pgfqpoint{0.932083in}{0.832348in}}%
\pgfpathlineto{\pgfqpoint{0.928398in}{0.834514in}}%
\pgfpathlineto{\pgfqpoint{0.915129in}{0.836318in}}%
\pgfpathlineto{\pgfqpoint{0.901860in}{0.834704in}}%
\pgfpathlineto{\pgfqpoint{0.897535in}{0.832348in}}%
\pgfpathlineto{\pgfqpoint{0.890640in}{0.819421in}}%
\pgfpathlineto{\pgfqpoint{0.889338in}{0.806495in}}%
\pgfpathlineto{\pgfqpoint{0.889591in}{0.793568in}}%
\pgfpathlineto{\pgfqpoint{0.891735in}{0.780642in}}%
\pgfpathlineto{\pgfqpoint{0.901860in}{0.768543in}}%
\pgfpathlineto{\pgfqpoint{0.906438in}{0.767715in}}%
\pgfpathclose%
\pgfpathmoveto{\pgfqpoint{0.901214in}{0.780642in}}%
\pgfpathlineto{\pgfqpoint{0.896574in}{0.793568in}}%
\pgfpathlineto{\pgfqpoint{0.896181in}{0.806495in}}%
\pgfpathlineto{\pgfqpoint{0.899443in}{0.819421in}}%
\pgfpathlineto{\pgfqpoint{0.901860in}{0.822823in}}%
\pgfpathlineto{\pgfqpoint{0.915129in}{0.827464in}}%
\pgfpathlineto{\pgfqpoint{0.928398in}{0.821671in}}%
\pgfpathlineto{\pgfqpoint{0.929846in}{0.819421in}}%
\pgfpathlineto{\pgfqpoint{0.932944in}{0.806495in}}%
\pgfpathlineto{\pgfqpoint{0.932568in}{0.793568in}}%
\pgfpathlineto{\pgfqpoint{0.928398in}{0.781197in}}%
\pgfpathlineto{\pgfqpoint{0.927796in}{0.780642in}}%
\pgfpathlineto{\pgfqpoint{0.915129in}{0.775426in}}%
\pgfpathlineto{\pgfqpoint{0.901860in}{0.779870in}}%
\pgfpathclose%
\pgfusepath{fill}%
\end{pgfscope}%
\begin{pgfscope}%
\pgfpathrectangle{\pgfqpoint{0.211875in}{0.211875in}}{\pgfqpoint{1.313625in}{1.279725in}}%
\pgfusepath{clip}%
\pgfsetbuttcap%
\pgfsetroundjoin%
\definecolor{currentfill}{rgb}{0.961115,0.566634,0.405693}%
\pgfsetfillcolor{currentfill}%
\pgfsetlinewidth{0.000000pt}%
\definecolor{currentstroke}{rgb}{0.000000,0.000000,0.000000}%
\pgfsetstrokecolor{currentstroke}%
\pgfsetdash{}{0pt}%
\pgfpathmoveto{\pgfqpoint{1.021280in}{0.767002in}}%
\pgfpathlineto{\pgfqpoint{1.034549in}{0.765910in}}%
\pgfpathlineto{\pgfqpoint{1.045313in}{0.767715in}}%
\pgfpathlineto{\pgfqpoint{1.047818in}{0.768424in}}%
\pgfpathlineto{\pgfqpoint{1.056061in}{0.780642in}}%
\pgfpathlineto{\pgfqpoint{1.057735in}{0.793568in}}%
\pgfpathlineto{\pgfqpoint{1.057952in}{0.806495in}}%
\pgfpathlineto{\pgfqpoint{1.056995in}{0.819421in}}%
\pgfpathlineto{\pgfqpoint{1.051766in}{0.832348in}}%
\pgfpathlineto{\pgfqpoint{1.047818in}{0.834990in}}%
\pgfpathlineto{\pgfqpoint{1.034549in}{0.837079in}}%
\pgfpathlineto{\pgfqpoint{1.021280in}{0.836124in}}%
\pgfpathlineto{\pgfqpoint{1.013442in}{0.832348in}}%
\pgfpathlineto{\pgfqpoint{1.008011in}{0.822322in}}%
\pgfpathlineto{\pgfqpoint{1.007415in}{0.819421in}}%
\pgfpathlineto{\pgfqpoint{1.006486in}{0.806495in}}%
\pgfpathlineto{\pgfqpoint{1.006690in}{0.793568in}}%
\pgfpathlineto{\pgfqpoint{1.008011in}{0.782376in}}%
\pgfpathlineto{\pgfqpoint{1.008355in}{0.780642in}}%
\pgfpathlineto{\pgfqpoint{1.019331in}{0.767715in}}%
\pgfpathclose%
\pgfpathmoveto{\pgfqpoint{1.018412in}{0.780642in}}%
\pgfpathlineto{\pgfqpoint{1.013778in}{0.793568in}}%
\pgfpathlineto{\pgfqpoint{1.013378in}{0.806495in}}%
\pgfpathlineto{\pgfqpoint{1.016611in}{0.819421in}}%
\pgfpathlineto{\pgfqpoint{1.021280in}{0.825264in}}%
\pgfpathlineto{\pgfqpoint{1.034549in}{0.828325in}}%
\pgfpathlineto{\pgfqpoint{1.047818in}{0.821104in}}%
\pgfpathlineto{\pgfqpoint{1.048775in}{0.819421in}}%
\pgfpathlineto{\pgfqpoint{1.051582in}{0.806495in}}%
\pgfpathlineto{\pgfqpoint{1.051233in}{0.793568in}}%
\pgfpathlineto{\pgfqpoint{1.047818in}{0.782160in}}%
\pgfpathlineto{\pgfqpoint{1.046559in}{0.780642in}}%
\pgfpathlineto{\pgfqpoint{1.034549in}{0.774638in}}%
\pgfpathlineto{\pgfqpoint{1.021280in}{0.777590in}}%
\pgfpathclose%
\pgfusepath{fill}%
\end{pgfscope}%
\begin{pgfscope}%
\pgfpathrectangle{\pgfqpoint{0.211875in}{0.211875in}}{\pgfqpoint{1.313625in}{1.279725in}}%
\pgfusepath{clip}%
\pgfsetbuttcap%
\pgfsetroundjoin%
\definecolor{currentfill}{rgb}{0.961115,0.566634,0.405693}%
\pgfsetfillcolor{currentfill}%
\pgfsetlinewidth{0.000000pt}%
\definecolor{currentstroke}{rgb}{0.000000,0.000000,0.000000}%
\pgfsetstrokecolor{currentstroke}%
\pgfsetdash{}{0pt}%
\pgfpathmoveto{\pgfqpoint{1.140701in}{0.766351in}}%
\pgfpathlineto{\pgfqpoint{1.153970in}{0.765732in}}%
\pgfpathlineto{\pgfqpoint{1.163840in}{0.767715in}}%
\pgfpathlineto{\pgfqpoint{1.167239in}{0.768963in}}%
\pgfpathlineto{\pgfqpoint{1.174177in}{0.780642in}}%
\pgfpathlineto{\pgfqpoint{1.175752in}{0.793568in}}%
\pgfpathlineto{\pgfqpoint{1.175959in}{0.806495in}}%
\pgfpathlineto{\pgfqpoint{1.175068in}{0.819421in}}%
\pgfpathlineto{\pgfqpoint{1.170213in}{0.832348in}}%
\pgfpathlineto{\pgfqpoint{1.167239in}{0.834629in}}%
\pgfpathlineto{\pgfqpoint{1.153970in}{0.837275in}}%
\pgfpathlineto{\pgfqpoint{1.140701in}{0.836747in}}%
\pgfpathlineto{\pgfqpoint{1.130356in}{0.832348in}}%
\pgfpathlineto{\pgfqpoint{1.127432in}{0.828476in}}%
\pgfpathlineto{\pgfqpoint{1.125019in}{0.819421in}}%
\pgfpathlineto{\pgfqpoint{1.124170in}{0.806495in}}%
\pgfpathlineto{\pgfqpoint{1.124368in}{0.793568in}}%
\pgfpathlineto{\pgfqpoint{1.125868in}{0.780642in}}%
\pgfpathlineto{\pgfqpoint{1.127432in}{0.775896in}}%
\pgfpathlineto{\pgfqpoint{1.136469in}{0.767715in}}%
\pgfpathclose%
\pgfpathmoveto{\pgfqpoint{1.136148in}{0.780642in}}%
\pgfpathlineto{\pgfqpoint{1.131304in}{0.793568in}}%
\pgfpathlineto{\pgfqpoint{1.130882in}{0.806495in}}%
\pgfpathlineto{\pgfqpoint{1.134249in}{0.819421in}}%
\pgfpathlineto{\pgfqpoint{1.140701in}{0.826588in}}%
\pgfpathlineto{\pgfqpoint{1.153970in}{0.828433in}}%
\pgfpathlineto{\pgfqpoint{1.167026in}{0.819421in}}%
\pgfpathlineto{\pgfqpoint{1.167239in}{0.818976in}}%
\pgfpathlineto{\pgfqpoint{1.169795in}{0.806495in}}%
\pgfpathlineto{\pgfqpoint{1.169460in}{0.793568in}}%
\pgfpathlineto{\pgfqpoint{1.167239in}{0.785162in}}%
\pgfpathlineto{\pgfqpoint{1.164300in}{0.780642in}}%
\pgfpathlineto{\pgfqpoint{1.153970in}{0.774548in}}%
\pgfpathlineto{\pgfqpoint{1.140701in}{0.776336in}}%
\pgfpathclose%
\pgfusepath{fill}%
\end{pgfscope}%
\begin{pgfscope}%
\pgfpathrectangle{\pgfqpoint{0.211875in}{0.211875in}}{\pgfqpoint{1.313625in}{1.279725in}}%
\pgfusepath{clip}%
\pgfsetbuttcap%
\pgfsetroundjoin%
\definecolor{currentfill}{rgb}{0.961115,0.566634,0.405693}%
\pgfsetfillcolor{currentfill}%
\pgfsetlinewidth{0.000000pt}%
\definecolor{currentstroke}{rgb}{0.000000,0.000000,0.000000}%
\pgfsetstrokecolor{currentstroke}%
\pgfsetdash{}{0pt}%
\pgfpathmoveto{\pgfqpoint{1.260121in}{0.766363in}}%
\pgfpathlineto{\pgfqpoint{1.273390in}{0.766125in}}%
\pgfpathlineto{\pgfqpoint{1.280150in}{0.767715in}}%
\pgfpathlineto{\pgfqpoint{1.286659in}{0.770776in}}%
\pgfpathlineto{\pgfqpoint{1.291773in}{0.780642in}}%
\pgfpathlineto{\pgfqpoint{1.293411in}{0.793568in}}%
\pgfpathlineto{\pgfqpoint{1.293619in}{0.806495in}}%
\pgfpathlineto{\pgfqpoint{1.292669in}{0.819421in}}%
\pgfpathlineto{\pgfqpoint{1.287553in}{0.832348in}}%
\pgfpathlineto{\pgfqpoint{1.286659in}{0.833141in}}%
\pgfpathlineto{\pgfqpoint{1.273390in}{0.836889in}}%
\pgfpathlineto{\pgfqpoint{1.260121in}{0.836693in}}%
\pgfpathlineto{\pgfqpoint{1.248491in}{0.832348in}}%
\pgfpathlineto{\pgfqpoint{1.246852in}{0.830721in}}%
\pgfpathlineto{\pgfqpoint{1.243150in}{0.819421in}}%
\pgfpathlineto{\pgfqpoint{1.242262in}{0.806495in}}%
\pgfpathlineto{\pgfqpoint{1.242463in}{0.793568in}}%
\pgfpathlineto{\pgfqpoint{1.244017in}{0.780642in}}%
\pgfpathlineto{\pgfqpoint{1.246852in}{0.773542in}}%
\pgfpathlineto{\pgfqpoint{1.255335in}{0.767715in}}%
\pgfpathclose%
\pgfpathmoveto{\pgfqpoint{1.254523in}{0.780642in}}%
\pgfpathlineto{\pgfqpoint{1.249195in}{0.793568in}}%
\pgfpathlineto{\pgfqpoint{1.248734in}{0.806495in}}%
\pgfpathlineto{\pgfqpoint{1.252441in}{0.819421in}}%
\pgfpathlineto{\pgfqpoint{1.260121in}{0.826970in}}%
\pgfpathlineto{\pgfqpoint{1.273390in}{0.827760in}}%
\pgfpathlineto{\pgfqpoint{1.283829in}{0.819421in}}%
\pgfpathlineto{\pgfqpoint{1.286659in}{0.811852in}}%
\pgfpathlineto{\pgfqpoint{1.287610in}{0.806495in}}%
\pgfpathlineto{\pgfqpoint{1.287278in}{0.793568in}}%
\pgfpathlineto{\pgfqpoint{1.286659in}{0.790894in}}%
\pgfpathlineto{\pgfqpoint{1.281387in}{0.780642in}}%
\pgfpathlineto{\pgfqpoint{1.273390in}{0.775183in}}%
\pgfpathlineto{\pgfqpoint{1.260121in}{0.775953in}}%
\pgfpathclose%
\pgfusepath{fill}%
\end{pgfscope}%
\begin{pgfscope}%
\pgfpathrectangle{\pgfqpoint{0.211875in}{0.211875in}}{\pgfqpoint{1.313625in}{1.279725in}}%
\pgfusepath{clip}%
\pgfsetbuttcap%
\pgfsetroundjoin%
\definecolor{currentfill}{rgb}{0.961115,0.566634,0.405693}%
\pgfsetfillcolor{currentfill}%
\pgfsetlinewidth{0.000000pt}%
\definecolor{currentstroke}{rgb}{0.000000,0.000000,0.000000}%
\pgfsetstrokecolor{currentstroke}%
\pgfsetdash{}{0pt}%
\pgfpathmoveto{\pgfqpoint{1.379542in}{0.766964in}}%
\pgfpathlineto{\pgfqpoint{1.392811in}{0.767131in}}%
\pgfpathlineto{\pgfqpoint{1.394965in}{0.767715in}}%
\pgfpathlineto{\pgfqpoint{1.406080in}{0.774400in}}%
\pgfpathlineto{\pgfqpoint{1.408865in}{0.780642in}}%
\pgfpathlineto{\pgfqpoint{1.410713in}{0.793568in}}%
\pgfpathlineto{\pgfqpoint{1.410931in}{0.806495in}}%
\pgfpathlineto{\pgfqpoint{1.409809in}{0.819421in}}%
\pgfpathlineto{\pgfqpoint{1.406080in}{0.829513in}}%
\pgfpathlineto{\pgfqpoint{1.402757in}{0.832348in}}%
\pgfpathlineto{\pgfqpoint{1.392811in}{0.835875in}}%
\pgfpathlineto{\pgfqpoint{1.379542in}{0.836041in}}%
\pgfpathlineto{\pgfqpoint{1.368205in}{0.832348in}}%
\pgfpathlineto{\pgfqpoint{1.366273in}{0.830875in}}%
\pgfpathlineto{\pgfqpoint{1.361803in}{0.819421in}}%
\pgfpathlineto{\pgfqpoint{1.360755in}{0.806495in}}%
\pgfpathlineto{\pgfqpoint{1.360969in}{0.793568in}}%
\pgfpathlineto{\pgfqpoint{1.362735in}{0.780642in}}%
\pgfpathlineto{\pgfqpoint{1.366273in}{0.773129in}}%
\pgfpathlineto{\pgfqpoint{1.376480in}{0.767715in}}%
\pgfpathclose%
\pgfpathmoveto{\pgfqpoint{1.373701in}{0.780642in}}%
\pgfpathlineto{\pgfqpoint{1.367531in}{0.793568in}}%
\pgfpathlineto{\pgfqpoint{1.367008in}{0.806495in}}%
\pgfpathlineto{\pgfqpoint{1.371324in}{0.819421in}}%
\pgfpathlineto{\pgfqpoint{1.379542in}{0.826521in}}%
\pgfpathlineto{\pgfqpoint{1.392811in}{0.826247in}}%
\pgfpathlineto{\pgfqpoint{1.400287in}{0.819421in}}%
\pgfpathlineto{\pgfqpoint{1.404508in}{0.806495in}}%
\pgfpathlineto{\pgfqpoint{1.403999in}{0.793568in}}%
\pgfpathlineto{\pgfqpoint{1.397998in}{0.780642in}}%
\pgfpathlineto{\pgfqpoint{1.392811in}{0.776597in}}%
\pgfpathlineto{\pgfqpoint{1.379542in}{0.776339in}}%
\pgfpathclose%
\pgfusepath{fill}%
\end{pgfscope}%
\begin{pgfscope}%
\pgfpathrectangle{\pgfqpoint{0.211875in}{0.211875in}}{\pgfqpoint{1.313625in}{1.279725in}}%
\pgfusepath{clip}%
\pgfsetbuttcap%
\pgfsetroundjoin%
\definecolor{currentfill}{rgb}{0.961115,0.566634,0.405693}%
\pgfsetfillcolor{currentfill}%
\pgfsetlinewidth{0.000000pt}%
\definecolor{currentstroke}{rgb}{0.000000,0.000000,0.000000}%
\pgfsetstrokecolor{currentstroke}%
\pgfsetdash{}{0pt}%
\pgfpathmoveto{\pgfqpoint{0.437447in}{0.776721in}}%
\pgfpathlineto{\pgfqpoint{0.450716in}{0.777233in}}%
\pgfpathlineto{\pgfqpoint{0.454913in}{0.780642in}}%
\pgfpathlineto{\pgfqpoint{0.460735in}{0.793568in}}%
\pgfpathlineto{\pgfqpoint{0.461224in}{0.806495in}}%
\pgfpathlineto{\pgfqpoint{0.457158in}{0.819421in}}%
\pgfpathlineto{\pgfqpoint{0.450716in}{0.825577in}}%
\pgfpathlineto{\pgfqpoint{0.437447in}{0.826089in}}%
\pgfpathlineto{\pgfqpoint{0.429362in}{0.819421in}}%
\pgfpathlineto{\pgfqpoint{0.424563in}{0.806495in}}%
\pgfpathlineto{\pgfqpoint{0.425133in}{0.793568in}}%
\pgfpathlineto{\pgfqpoint{0.431898in}{0.780642in}}%
\pgfpathclose%
\pgfpathmoveto{\pgfqpoint{0.434920in}{0.793568in}}%
\pgfpathlineto{\pgfqpoint{0.434158in}{0.806495in}}%
\pgfpathlineto{\pgfqpoint{0.437447in}{0.812980in}}%
\pgfpathlineto{\pgfqpoint{0.450716in}{0.810951in}}%
\pgfpathlineto{\pgfqpoint{0.452631in}{0.806495in}}%
\pgfpathlineto{\pgfqpoint{0.451958in}{0.793568in}}%
\pgfpathlineto{\pgfqpoint{0.450716in}{0.791264in}}%
\pgfpathlineto{\pgfqpoint{0.437447in}{0.789580in}}%
\pgfpathclose%
\pgfusepath{fill}%
\end{pgfscope}%
\begin{pgfscope}%
\pgfpathrectangle{\pgfqpoint{0.211875in}{0.211875in}}{\pgfqpoint{1.313625in}{1.279725in}}%
\pgfusepath{clip}%
\pgfsetbuttcap%
\pgfsetroundjoin%
\definecolor{currentfill}{rgb}{0.961115,0.566634,0.405693}%
\pgfsetfillcolor{currentfill}%
\pgfsetlinewidth{0.000000pt}%
\definecolor{currentstroke}{rgb}{0.000000,0.000000,0.000000}%
\pgfsetstrokecolor{currentstroke}%
\pgfsetdash{}{0pt}%
\pgfpathmoveto{\pgfqpoint{0.556867in}{0.773019in}}%
\pgfpathlineto{\pgfqpoint{0.570136in}{0.774269in}}%
\pgfpathlineto{\pgfqpoint{0.577065in}{0.780642in}}%
\pgfpathlineto{\pgfqpoint{0.581358in}{0.793568in}}%
\pgfpathlineto{\pgfqpoint{0.581750in}{0.806495in}}%
\pgfpathlineto{\pgfqpoint{0.578837in}{0.819421in}}%
\pgfpathlineto{\pgfqpoint{0.570136in}{0.828842in}}%
\pgfpathlineto{\pgfqpoint{0.556867in}{0.830094in}}%
\pgfpathlineto{\pgfqpoint{0.543598in}{0.821311in}}%
\pgfpathlineto{\pgfqpoint{0.542675in}{0.819421in}}%
\pgfpathlineto{\pgfqpoint{0.540267in}{0.806495in}}%
\pgfpathlineto{\pgfqpoint{0.540582in}{0.793568in}}%
\pgfpathlineto{\pgfqpoint{0.543598in}{0.781984in}}%
\pgfpathlineto{\pgfqpoint{0.544439in}{0.780642in}}%
\pgfpathclose%
\pgfpathmoveto{\pgfqpoint{0.549260in}{0.793568in}}%
\pgfpathlineto{\pgfqpoint{0.548480in}{0.806495in}}%
\pgfpathlineto{\pgfqpoint{0.555800in}{0.819421in}}%
\pgfpathlineto{\pgfqpoint{0.556867in}{0.820185in}}%
\pgfpathlineto{\pgfqpoint{0.561307in}{0.819421in}}%
\pgfpathlineto{\pgfqpoint{0.570136in}{0.816222in}}%
\pgfpathlineto{\pgfqpoint{0.573830in}{0.806495in}}%
\pgfpathlineto{\pgfqpoint{0.573271in}{0.793568in}}%
\pgfpathlineto{\pgfqpoint{0.570136in}{0.787005in}}%
\pgfpathlineto{\pgfqpoint{0.556867in}{0.783136in}}%
\pgfpathclose%
\pgfusepath{fill}%
\end{pgfscope}%
\begin{pgfscope}%
\pgfpathrectangle{\pgfqpoint{0.211875in}{0.211875in}}{\pgfqpoint{1.313625in}{1.279725in}}%
\pgfusepath{clip}%
\pgfsetbuttcap%
\pgfsetroundjoin%
\definecolor{currentfill}{rgb}{0.961115,0.566634,0.405693}%
\pgfsetfillcolor{currentfill}%
\pgfsetlinewidth{0.000000pt}%
\definecolor{currentstroke}{rgb}{0.000000,0.000000,0.000000}%
\pgfsetstrokecolor{currentstroke}%
\pgfsetdash{}{0pt}%
\pgfpathmoveto{\pgfqpoint{0.663019in}{0.775444in}}%
\pgfpathlineto{\pgfqpoint{0.676288in}{0.770128in}}%
\pgfpathlineto{\pgfqpoint{0.689557in}{0.771856in}}%
\pgfpathlineto{\pgfqpoint{0.698032in}{0.780642in}}%
\pgfpathlineto{\pgfqpoint{0.701237in}{0.793568in}}%
\pgfpathlineto{\pgfqpoint{0.701558in}{0.806495in}}%
\pgfpathlineto{\pgfqpoint{0.699464in}{0.819421in}}%
\pgfpathlineto{\pgfqpoint{0.689557in}{0.831526in}}%
\pgfpathlineto{\pgfqpoint{0.683709in}{0.832348in}}%
\pgfpathlineto{\pgfqpoint{0.676288in}{0.833064in}}%
\pgfpathlineto{\pgfqpoint{0.673342in}{0.832348in}}%
\pgfpathlineto{\pgfqpoint{0.663019in}{0.827835in}}%
\pgfpathlineto{\pgfqpoint{0.658331in}{0.819421in}}%
\pgfpathlineto{\pgfqpoint{0.656390in}{0.806495in}}%
\pgfpathlineto{\pgfqpoint{0.656676in}{0.793568in}}%
\pgfpathlineto{\pgfqpoint{0.659588in}{0.780642in}}%
\pgfpathclose%
\pgfpathmoveto{\pgfqpoint{0.673411in}{0.780642in}}%
\pgfpathlineto{\pgfqpoint{0.663114in}{0.793568in}}%
\pgfpathlineto{\pgfqpoint{0.663019in}{0.795025in}}%
\pgfpathlineto{\pgfqpoint{0.662653in}{0.806495in}}%
\pgfpathlineto{\pgfqpoint{0.663019in}{0.808082in}}%
\pgfpathlineto{\pgfqpoint{0.669702in}{0.819421in}}%
\pgfpathlineto{\pgfqpoint{0.676288in}{0.823441in}}%
\pgfpathlineto{\pgfqpoint{0.689557in}{0.819830in}}%
\pgfpathlineto{\pgfqpoint{0.689891in}{0.819421in}}%
\pgfpathlineto{\pgfqpoint{0.694168in}{0.806495in}}%
\pgfpathlineto{\pgfqpoint{0.693691in}{0.793568in}}%
\pgfpathlineto{\pgfqpoint{0.689557in}{0.783836in}}%
\pgfpathlineto{\pgfqpoint{0.682678in}{0.780642in}}%
\pgfpathlineto{\pgfqpoint{0.676288in}{0.779137in}}%
\pgfpathclose%
\pgfusepath{fill}%
\end{pgfscope}%
\begin{pgfscope}%
\pgfpathrectangle{\pgfqpoint{0.211875in}{0.211875in}}{\pgfqpoint{1.313625in}{1.279725in}}%
\pgfusepath{clip}%
\pgfsetbuttcap%
\pgfsetroundjoin%
\definecolor{currentfill}{rgb}{0.961115,0.566634,0.405693}%
\pgfsetfillcolor{currentfill}%
\pgfsetlinewidth{0.000000pt}%
\definecolor{currentstroke}{rgb}{0.000000,0.000000,0.000000}%
\pgfsetstrokecolor{currentstroke}%
\pgfsetdash{}{0pt}%
\pgfpathmoveto{\pgfqpoint{0.782439in}{0.771293in}}%
\pgfpathlineto{\pgfqpoint{0.795708in}{0.767964in}}%
\pgfpathlineto{\pgfqpoint{0.808977in}{0.770025in}}%
\pgfpathlineto{\pgfqpoint{0.818081in}{0.780642in}}%
\pgfpathlineto{\pgfqpoint{0.820529in}{0.793568in}}%
\pgfpathlineto{\pgfqpoint{0.820801in}{0.806495in}}%
\pgfpathlineto{\pgfqpoint{0.819273in}{0.819421in}}%
\pgfpathlineto{\pgfqpoint{0.810959in}{0.832348in}}%
\pgfpathlineto{\pgfqpoint{0.808977in}{0.833374in}}%
\pgfpathlineto{\pgfqpoint{0.795708in}{0.834988in}}%
\pgfpathlineto{\pgfqpoint{0.782597in}{0.832348in}}%
\pgfpathlineto{\pgfqpoint{0.782439in}{0.832295in}}%
\pgfpathlineto{\pgfqpoint{0.774325in}{0.819421in}}%
\pgfpathlineto{\pgfqpoint{0.772754in}{0.806495in}}%
\pgfpathlineto{\pgfqpoint{0.773019in}{0.793568in}}%
\pgfpathlineto{\pgfqpoint{0.775480in}{0.780642in}}%
\pgfpathclose%
\pgfpathmoveto{\pgfqpoint{0.787142in}{0.780642in}}%
\pgfpathlineto{\pgfqpoint{0.782439in}{0.785223in}}%
\pgfpathlineto{\pgfqpoint{0.779672in}{0.793568in}}%
\pgfpathlineto{\pgfqpoint{0.779273in}{0.806495in}}%
\pgfpathlineto{\pgfqpoint{0.782439in}{0.818632in}}%
\pgfpathlineto{\pgfqpoint{0.783044in}{0.819421in}}%
\pgfpathlineto{\pgfqpoint{0.795708in}{0.825846in}}%
\pgfpathlineto{\pgfqpoint{0.808977in}{0.821185in}}%
\pgfpathlineto{\pgfqpoint{0.810258in}{0.819421in}}%
\pgfpathlineto{\pgfqpoint{0.813832in}{0.806495in}}%
\pgfpathlineto{\pgfqpoint{0.813415in}{0.793568in}}%
\pgfpathlineto{\pgfqpoint{0.808977in}{0.781846in}}%
\pgfpathlineto{\pgfqpoint{0.807207in}{0.780642in}}%
\pgfpathlineto{\pgfqpoint{0.795708in}{0.776916in}}%
\pgfpathclose%
\pgfusepath{fill}%
\end{pgfscope}%
\begin{pgfscope}%
\pgfpathrectangle{\pgfqpoint{0.211875in}{0.211875in}}{\pgfqpoint{1.313625in}{1.279725in}}%
\pgfusepath{clip}%
\pgfsetbuttcap%
\pgfsetroundjoin%
\definecolor{currentfill}{rgb}{0.961115,0.566634,0.405693}%
\pgfsetfillcolor{currentfill}%
\pgfsetlinewidth{0.000000pt}%
\definecolor{currentstroke}{rgb}{0.000000,0.000000,0.000000}%
\pgfsetstrokecolor{currentstroke}%
\pgfsetdash{}{0pt}%
\pgfpathmoveto{\pgfqpoint{1.485693in}{0.773948in}}%
\pgfpathlineto{\pgfqpoint{1.498962in}{0.768174in}}%
\pgfpathlineto{\pgfqpoint{1.512231in}{0.768993in}}%
\pgfpathlineto{\pgfqpoint{1.525425in}{0.780642in}}%
\pgfpathlineto{\pgfqpoint{1.525500in}{0.780876in}}%
\pgfpathlineto{\pgfqpoint{1.525500in}{0.793568in}}%
\pgfpathlineto{\pgfqpoint{1.525500in}{0.806495in}}%
\pgfpathlineto{\pgfqpoint{1.525500in}{0.819421in}}%
\pgfpathlineto{\pgfqpoint{1.525500in}{0.822599in}}%
\pgfpathlineto{\pgfqpoint{1.516719in}{0.832348in}}%
\pgfpathlineto{\pgfqpoint{1.512231in}{0.834160in}}%
\pgfpathlineto{\pgfqpoint{1.498962in}{0.834831in}}%
\pgfpathlineto{\pgfqpoint{1.490109in}{0.832348in}}%
\pgfpathlineto{\pgfqpoint{1.485693in}{0.829742in}}%
\pgfpathlineto{\pgfqpoint{1.480997in}{0.819421in}}%
\pgfpathlineto{\pgfqpoint{1.479656in}{0.806495in}}%
\pgfpathlineto{\pgfqpoint{1.479894in}{0.793568in}}%
\pgfpathlineto{\pgfqpoint{1.482043in}{0.780642in}}%
\pgfpathclose%
\pgfpathmoveto{\pgfqpoint{1.493960in}{0.780642in}}%
\pgfpathlineto{\pgfqpoint{1.486448in}{0.793568in}}%
\pgfpathlineto{\pgfqpoint{1.485830in}{0.806495in}}%
\pgfpathlineto{\pgfqpoint{1.491130in}{0.819421in}}%
\pgfpathlineto{\pgfqpoint{1.498962in}{0.825310in}}%
\pgfpathlineto{\pgfqpoint{1.512231in}{0.823791in}}%
\pgfpathlineto{\pgfqpoint{1.516455in}{0.819421in}}%
\pgfpathlineto{\pgfqpoint{1.520714in}{0.806495in}}%
\pgfpathlineto{\pgfqpoint{1.520219in}{0.793568in}}%
\pgfpathlineto{\pgfqpoint{1.514227in}{0.780642in}}%
\pgfpathlineto{\pgfqpoint{1.512231in}{0.778879in}}%
\pgfpathlineto{\pgfqpoint{1.498962in}{0.777435in}}%
\pgfpathclose%
\pgfusepath{fill}%
\end{pgfscope}%
\begin{pgfscope}%
\pgfpathrectangle{\pgfqpoint{0.211875in}{0.211875in}}{\pgfqpoint{1.313625in}{1.279725in}}%
\pgfusepath{clip}%
\pgfsetbuttcap%
\pgfsetroundjoin%
\definecolor{currentfill}{rgb}{0.961115,0.566634,0.405693}%
\pgfsetfillcolor{currentfill}%
\pgfsetlinewidth{0.000000pt}%
\definecolor{currentstroke}{rgb}{0.000000,0.000000,0.000000}%
\pgfsetstrokecolor{currentstroke}%
\pgfsetdash{}{0pt}%
\pgfpathmoveto{\pgfqpoint{0.216100in}{0.793568in}}%
\pgfpathlineto{\pgfqpoint{0.216910in}{0.806495in}}%
\pgfpathlineto{\pgfqpoint{0.211875in}{0.815440in}}%
\pgfpathlineto{\pgfqpoint{0.211875in}{0.806495in}}%
\pgfpathlineto{\pgfqpoint{0.211875in}{0.793568in}}%
\pgfpathlineto{\pgfqpoint{0.211875in}{0.787568in}}%
\pgfpathclose%
\pgfusepath{fill}%
\end{pgfscope}%
\begin{pgfscope}%
\pgfpathrectangle{\pgfqpoint{0.211875in}{0.211875in}}{\pgfqpoint{1.313625in}{1.279725in}}%
\pgfusepath{clip}%
\pgfsetbuttcap%
\pgfsetroundjoin%
\definecolor{currentfill}{rgb}{0.961115,0.566634,0.405693}%
\pgfsetfillcolor{currentfill}%
\pgfsetlinewidth{0.000000pt}%
\definecolor{currentstroke}{rgb}{0.000000,0.000000,0.000000}%
\pgfsetstrokecolor{currentstroke}%
\pgfsetdash{}{0pt}%
\pgfpathmoveto{\pgfqpoint{0.318027in}{0.781826in}}%
\pgfpathlineto{\pgfqpoint{0.331295in}{0.780807in}}%
\pgfpathlineto{\pgfqpoint{0.339117in}{0.793568in}}%
\pgfpathlineto{\pgfqpoint{0.339740in}{0.806495in}}%
\pgfpathlineto{\pgfqpoint{0.334058in}{0.819421in}}%
\pgfpathlineto{\pgfqpoint{0.331295in}{0.821735in}}%
\pgfpathlineto{\pgfqpoint{0.318027in}{0.821072in}}%
\pgfpathlineto{\pgfqpoint{0.316269in}{0.819421in}}%
\pgfpathlineto{\pgfqpoint{0.310895in}{0.806495in}}%
\pgfpathlineto{\pgfqpoint{0.311485in}{0.793568in}}%
\pgfpathclose%
\pgfusepath{fill}%
\end{pgfscope}%
\begin{pgfscope}%
\pgfpathrectangle{\pgfqpoint{0.211875in}{0.211875in}}{\pgfqpoint{1.313625in}{1.279725in}}%
\pgfusepath{clip}%
\pgfsetbuttcap%
\pgfsetroundjoin%
\definecolor{currentfill}{rgb}{0.961115,0.566634,0.405693}%
\pgfsetfillcolor{currentfill}%
\pgfsetlinewidth{0.000000pt}%
\definecolor{currentstroke}{rgb}{0.000000,0.000000,0.000000}%
\pgfsetstrokecolor{currentstroke}%
\pgfsetdash{}{0pt}%
\pgfpathmoveto{\pgfqpoint{0.384371in}{0.857472in}}%
\pgfpathlineto{\pgfqpoint{0.386822in}{0.858201in}}%
\pgfpathlineto{\pgfqpoint{0.397640in}{0.864523in}}%
\pgfpathlineto{\pgfqpoint{0.400571in}{0.871127in}}%
\pgfpathlineto{\pgfqpoint{0.401871in}{0.884054in}}%
\pgfpathlineto{\pgfqpoint{0.399782in}{0.896980in}}%
\pgfpathlineto{\pgfqpoint{0.397640in}{0.900990in}}%
\pgfpathlineto{\pgfqpoint{0.384371in}{0.907381in}}%
\pgfpathlineto{\pgfqpoint{0.371102in}{0.901089in}}%
\pgfpathlineto{\pgfqpoint{0.368826in}{0.896980in}}%
\pgfpathlineto{\pgfqpoint{0.366625in}{0.884054in}}%
\pgfpathlineto{\pgfqpoint{0.367987in}{0.871127in}}%
\pgfpathlineto{\pgfqpoint{0.371102in}{0.864374in}}%
\pgfpathlineto{\pgfqpoint{0.381981in}{0.858201in}}%
\pgfpathclose%
\pgfpathmoveto{\pgfqpoint{0.381704in}{0.871127in}}%
\pgfpathlineto{\pgfqpoint{0.376548in}{0.884054in}}%
\pgfpathlineto{\pgfqpoint{0.384371in}{0.895856in}}%
\pgfpathlineto{\pgfqpoint{0.391767in}{0.884054in}}%
\pgfpathlineto{\pgfqpoint{0.386923in}{0.871127in}}%
\pgfpathlineto{\pgfqpoint{0.384371in}{0.869380in}}%
\pgfpathclose%
\pgfusepath{fill}%
\end{pgfscope}%
\begin{pgfscope}%
\pgfpathrectangle{\pgfqpoint{0.211875in}{0.211875in}}{\pgfqpoint{1.313625in}{1.279725in}}%
\pgfusepath{clip}%
\pgfsetbuttcap%
\pgfsetroundjoin%
\definecolor{currentfill}{rgb}{0.961115,0.566634,0.405693}%
\pgfsetfillcolor{currentfill}%
\pgfsetlinewidth{0.000000pt}%
\definecolor{currentstroke}{rgb}{0.000000,0.000000,0.000000}%
\pgfsetstrokecolor{currentstroke}%
\pgfsetdash{}{0pt}%
\pgfpathmoveto{\pgfqpoint{0.490523in}{0.857136in}}%
\pgfpathlineto{\pgfqpoint{0.503792in}{0.854203in}}%
\pgfpathlineto{\pgfqpoint{0.514662in}{0.858201in}}%
\pgfpathlineto{\pgfqpoint{0.517061in}{0.860064in}}%
\pgfpathlineto{\pgfqpoint{0.521413in}{0.871127in}}%
\pgfpathlineto{\pgfqpoint{0.522413in}{0.884054in}}%
\pgfpathlineto{\pgfqpoint{0.520872in}{0.896980in}}%
\pgfpathlineto{\pgfqpoint{0.517061in}{0.905056in}}%
\pgfpathlineto{\pgfqpoint{0.508331in}{0.909907in}}%
\pgfpathlineto{\pgfqpoint{0.503792in}{0.911200in}}%
\pgfpathlineto{\pgfqpoint{0.496963in}{0.909907in}}%
\pgfpathlineto{\pgfqpoint{0.490523in}{0.907955in}}%
\pgfpathlineto{\pgfqpoint{0.483692in}{0.896980in}}%
\pgfpathlineto{\pgfqpoint{0.481867in}{0.884054in}}%
\pgfpathlineto{\pgfqpoint{0.483037in}{0.871127in}}%
\pgfpathlineto{\pgfqpoint{0.489315in}{0.858201in}}%
\pgfpathclose%
\pgfpathmoveto{\pgfqpoint{0.490921in}{0.871127in}}%
\pgfpathlineto{\pgfqpoint{0.490523in}{0.871923in}}%
\pgfpathlineto{\pgfqpoint{0.488808in}{0.884054in}}%
\pgfpathlineto{\pgfqpoint{0.490523in}{0.891799in}}%
\pgfpathlineto{\pgfqpoint{0.494871in}{0.896980in}}%
\pgfpathlineto{\pgfqpoint{0.503792in}{0.900690in}}%
\pgfpathlineto{\pgfqpoint{0.509179in}{0.896980in}}%
\pgfpathlineto{\pgfqpoint{0.514938in}{0.884054in}}%
\pgfpathlineto{\pgfqpoint{0.511493in}{0.871127in}}%
\pgfpathlineto{\pgfqpoint{0.503792in}{0.864674in}}%
\pgfpathclose%
\pgfusepath{fill}%
\end{pgfscope}%
\begin{pgfscope}%
\pgfpathrectangle{\pgfqpoint{0.211875in}{0.211875in}}{\pgfqpoint{1.313625in}{1.279725in}}%
\pgfusepath{clip}%
\pgfsetbuttcap%
\pgfsetroundjoin%
\definecolor{currentfill}{rgb}{0.961115,0.566634,0.405693}%
\pgfsetfillcolor{currentfill}%
\pgfsetlinewidth{0.000000pt}%
\definecolor{currentstroke}{rgb}{0.000000,0.000000,0.000000}%
\pgfsetstrokecolor{currentstroke}%
\pgfsetdash{}{0pt}%
\pgfpathmoveto{\pgfqpoint{0.609943in}{0.853126in}}%
\pgfpathlineto{\pgfqpoint{0.623212in}{0.851533in}}%
\pgfpathlineto{\pgfqpoint{0.636481in}{0.856886in}}%
\pgfpathlineto{\pgfqpoint{0.637502in}{0.858201in}}%
\pgfpathlineto{\pgfqpoint{0.641630in}{0.871127in}}%
\pgfpathlineto{\pgfqpoint{0.642400in}{0.884054in}}%
\pgfpathlineto{\pgfqpoint{0.641277in}{0.896980in}}%
\pgfpathlineto{\pgfqpoint{0.636481in}{0.908557in}}%
\pgfpathlineto{\pgfqpoint{0.634620in}{0.909907in}}%
\pgfpathlineto{\pgfqpoint{0.623212in}{0.913811in}}%
\pgfpathlineto{\pgfqpoint{0.609943in}{0.912333in}}%
\pgfpathlineto{\pgfqpoint{0.606054in}{0.909907in}}%
\pgfpathlineto{\pgfqpoint{0.598642in}{0.896980in}}%
\pgfpathlineto{\pgfqpoint{0.597143in}{0.884054in}}%
\pgfpathlineto{\pgfqpoint{0.598151in}{0.871127in}}%
\pgfpathlineto{\pgfqpoint{0.603470in}{0.858201in}}%
\pgfpathclose%
\pgfpathmoveto{\pgfqpoint{0.606278in}{0.871127in}}%
\pgfpathlineto{\pgfqpoint{0.604556in}{0.884054in}}%
\pgfpathlineto{\pgfqpoint{0.607385in}{0.896980in}}%
\pgfpathlineto{\pgfqpoint{0.609943in}{0.900639in}}%
\pgfpathlineto{\pgfqpoint{0.623212in}{0.904027in}}%
\pgfpathlineto{\pgfqpoint{0.631831in}{0.896980in}}%
\pgfpathlineto{\pgfqpoint{0.636048in}{0.884054in}}%
\pgfpathlineto{\pgfqpoint{0.633479in}{0.871127in}}%
\pgfpathlineto{\pgfqpoint{0.623212in}{0.860924in}}%
\pgfpathlineto{\pgfqpoint{0.609943in}{0.864807in}}%
\pgfpathclose%
\pgfusepath{fill}%
\end{pgfscope}%
\begin{pgfscope}%
\pgfpathrectangle{\pgfqpoint{0.211875in}{0.211875in}}{\pgfqpoint{1.313625in}{1.279725in}}%
\pgfusepath{clip}%
\pgfsetbuttcap%
\pgfsetroundjoin%
\definecolor{currentfill}{rgb}{0.961115,0.566634,0.405693}%
\pgfsetfillcolor{currentfill}%
\pgfsetlinewidth{0.000000pt}%
\definecolor{currentstroke}{rgb}{0.000000,0.000000,0.000000}%
\pgfsetstrokecolor{currentstroke}%
\pgfsetdash{}{0pt}%
\pgfpathmoveto{\pgfqpoint{0.729364in}{0.850155in}}%
\pgfpathlineto{\pgfqpoint{0.742633in}{0.849432in}}%
\pgfpathlineto{\pgfqpoint{0.755902in}{0.854835in}}%
\pgfpathlineto{\pgfqpoint{0.758180in}{0.858201in}}%
\pgfpathlineto{\pgfqpoint{0.761308in}{0.871127in}}%
\pgfpathlineto{\pgfqpoint{0.761908in}{0.884054in}}%
\pgfpathlineto{\pgfqpoint{0.761094in}{0.896980in}}%
\pgfpathlineto{\pgfqpoint{0.756851in}{0.909907in}}%
\pgfpathlineto{\pgfqpoint{0.755902in}{0.911025in}}%
\pgfpathlineto{\pgfqpoint{0.742633in}{0.915874in}}%
\pgfpathlineto{\pgfqpoint{0.729364in}{0.915209in}}%
\pgfpathlineto{\pgfqpoint{0.719764in}{0.909907in}}%
\pgfpathlineto{\pgfqpoint{0.716095in}{0.903944in}}%
\pgfpathlineto{\pgfqpoint{0.714321in}{0.896980in}}%
\pgfpathlineto{\pgfqpoint{0.713450in}{0.884054in}}%
\pgfpathlineto{\pgfqpoint{0.714073in}{0.871127in}}%
\pgfpathlineto{\pgfqpoint{0.716095in}{0.861876in}}%
\pgfpathlineto{\pgfqpoint{0.717793in}{0.858201in}}%
\pgfpathclose%
\pgfpathmoveto{\pgfqpoint{0.741790in}{0.858201in}}%
\pgfpathlineto{\pgfqpoint{0.729364in}{0.859908in}}%
\pgfpathlineto{\pgfqpoint{0.722044in}{0.871127in}}%
\pgfpathlineto{\pgfqpoint{0.720393in}{0.884054in}}%
\pgfpathlineto{\pgfqpoint{0.723059in}{0.896980in}}%
\pgfpathlineto{\pgfqpoint{0.729364in}{0.904980in}}%
\pgfpathlineto{\pgfqpoint{0.742633in}{0.906557in}}%
\pgfpathlineto{\pgfqpoint{0.752703in}{0.896980in}}%
\pgfpathlineto{\pgfqpoint{0.755902in}{0.884170in}}%
\pgfpathlineto{\pgfqpoint{0.755916in}{0.884054in}}%
\pgfpathlineto{\pgfqpoint{0.755902in}{0.883880in}}%
\pgfpathlineto{\pgfqpoint{0.753926in}{0.871127in}}%
\pgfpathlineto{\pgfqpoint{0.742778in}{0.858201in}}%
\pgfpathlineto{\pgfqpoint{0.742633in}{0.858126in}}%
\pgfpathclose%
\pgfusepath{fill}%
\end{pgfscope}%
\begin{pgfscope}%
\pgfpathrectangle{\pgfqpoint{0.211875in}{0.211875in}}{\pgfqpoint{1.313625in}{1.279725in}}%
\pgfusepath{clip}%
\pgfsetbuttcap%
\pgfsetroundjoin%
\definecolor{currentfill}{rgb}{0.961115,0.566634,0.405693}%
\pgfsetfillcolor{currentfill}%
\pgfsetlinewidth{0.000000pt}%
\definecolor{currentstroke}{rgb}{0.000000,0.000000,0.000000}%
\pgfsetstrokecolor{currentstroke}%
\pgfsetdash{}{0pt}%
\pgfpathmoveto{\pgfqpoint{0.835515in}{0.854379in}}%
\pgfpathlineto{\pgfqpoint{0.848784in}{0.848050in}}%
\pgfpathlineto{\pgfqpoint{0.862053in}{0.847889in}}%
\pgfpathlineto{\pgfqpoint{0.875322in}{0.853477in}}%
\pgfpathlineto{\pgfqpoint{0.878072in}{0.858201in}}%
\pgfpathlineto{\pgfqpoint{0.880503in}{0.871127in}}%
\pgfpathlineto{\pgfqpoint{0.880983in}{0.884054in}}%
\pgfpathlineto{\pgfqpoint{0.880386in}{0.896980in}}%
\pgfpathlineto{\pgfqpoint{0.877218in}{0.909907in}}%
\pgfpathlineto{\pgfqpoint{0.875322in}{0.912519in}}%
\pgfpathlineto{\pgfqpoint{0.862053in}{0.917398in}}%
\pgfpathlineto{\pgfqpoint{0.848784in}{0.917241in}}%
\pgfpathlineto{\pgfqpoint{0.835515in}{0.911617in}}%
\pgfpathlineto{\pgfqpoint{0.834310in}{0.909907in}}%
\pgfpathlineto{\pgfqpoint{0.830896in}{0.896980in}}%
\pgfpathlineto{\pgfqpoint{0.830235in}{0.884054in}}%
\pgfpathlineto{\pgfqpoint{0.830747in}{0.871127in}}%
\pgfpathlineto{\pgfqpoint{0.833342in}{0.858201in}}%
\pgfpathclose%
\pgfpathmoveto{\pgfqpoint{0.846770in}{0.858201in}}%
\pgfpathlineto{\pgfqpoint{0.837908in}{0.871127in}}%
\pgfpathlineto{\pgfqpoint{0.836277in}{0.884054in}}%
\pgfpathlineto{\pgfqpoint{0.838869in}{0.896980in}}%
\pgfpathlineto{\pgfqpoint{0.848784in}{0.908082in}}%
\pgfpathlineto{\pgfqpoint{0.862053in}{0.908284in}}%
\pgfpathlineto{\pgfqpoint{0.872419in}{0.896980in}}%
\pgfpathlineto{\pgfqpoint{0.874994in}{0.884054in}}%
\pgfpathlineto{\pgfqpoint{0.873369in}{0.871127in}}%
\pgfpathlineto{\pgfqpoint{0.864409in}{0.858201in}}%
\pgfpathlineto{\pgfqpoint{0.862053in}{0.856811in}}%
\pgfpathlineto{\pgfqpoint{0.848784in}{0.856966in}}%
\pgfpathclose%
\pgfusepath{fill}%
\end{pgfscope}%
\begin{pgfscope}%
\pgfpathrectangle{\pgfqpoint{0.211875in}{0.211875in}}{\pgfqpoint{1.313625in}{1.279725in}}%
\pgfusepath{clip}%
\pgfsetbuttcap%
\pgfsetroundjoin%
\definecolor{currentfill}{rgb}{0.961115,0.566634,0.405693}%
\pgfsetfillcolor{currentfill}%
\pgfsetlinewidth{0.000000pt}%
\definecolor{currentstroke}{rgb}{0.000000,0.000000,0.000000}%
\pgfsetstrokecolor{currentstroke}%
\pgfsetdash{}{0pt}%
\pgfpathmoveto{\pgfqpoint{0.954936in}{0.850668in}}%
\pgfpathlineto{\pgfqpoint{0.968205in}{0.846694in}}%
\pgfpathlineto{\pgfqpoint{0.981473in}{0.846919in}}%
\pgfpathlineto{\pgfqpoint{0.994742in}{0.853101in}}%
\pgfpathlineto{\pgfqpoint{0.997247in}{0.858201in}}%
\pgfpathlineto{\pgfqpoint{0.999247in}{0.871127in}}%
\pgfpathlineto{\pgfqpoint{0.999654in}{0.884054in}}%
\pgfpathlineto{\pgfqpoint{0.999190in}{0.896980in}}%
\pgfpathlineto{\pgfqpoint{0.996685in}{0.909907in}}%
\pgfpathlineto{\pgfqpoint{0.994742in}{0.913105in}}%
\pgfpathlineto{\pgfqpoint{0.981473in}{0.918372in}}%
\pgfpathlineto{\pgfqpoint{0.968205in}{0.918546in}}%
\pgfpathlineto{\pgfqpoint{0.954936in}{0.915150in}}%
\pgfpathlineto{\pgfqpoint{0.950619in}{0.909907in}}%
\pgfpathlineto{\pgfqpoint{0.947877in}{0.896980in}}%
\pgfpathlineto{\pgfqpoint{0.947359in}{0.884054in}}%
\pgfpathlineto{\pgfqpoint{0.947798in}{0.871127in}}%
\pgfpathlineto{\pgfqpoint{0.949961in}{0.858201in}}%
\pgfpathclose%
\pgfpathmoveto{\pgfqpoint{0.962945in}{0.858201in}}%
\pgfpathlineto{\pgfqpoint{0.954936in}{0.868902in}}%
\pgfpathlineto{\pgfqpoint{0.954271in}{0.871127in}}%
\pgfpathlineto{\pgfqpoint{0.953255in}{0.884054in}}%
\pgfpathlineto{\pgfqpoint{0.954848in}{0.896980in}}%
\pgfpathlineto{\pgfqpoint{0.954936in}{0.897227in}}%
\pgfpathlineto{\pgfqpoint{0.967827in}{0.909907in}}%
\pgfpathlineto{\pgfqpoint{0.968205in}{0.910067in}}%
\pgfpathlineto{\pgfqpoint{0.971359in}{0.909907in}}%
\pgfpathlineto{\pgfqpoint{0.981473in}{0.909181in}}%
\pgfpathlineto{\pgfqpoint{0.991324in}{0.896980in}}%
\pgfpathlineto{\pgfqpoint{0.993487in}{0.884054in}}%
\pgfpathlineto{\pgfqpoint{0.992104in}{0.871127in}}%
\pgfpathlineto{\pgfqpoint{0.984542in}{0.858201in}}%
\pgfpathlineto{\pgfqpoint{0.981473in}{0.856141in}}%
\pgfpathlineto{\pgfqpoint{0.968205in}{0.855383in}}%
\pgfpathclose%
\pgfusepath{fill}%
\end{pgfscope}%
\begin{pgfscope}%
\pgfpathrectangle{\pgfqpoint{0.211875in}{0.211875in}}{\pgfqpoint{1.313625in}{1.279725in}}%
\pgfusepath{clip}%
\pgfsetbuttcap%
\pgfsetroundjoin%
\definecolor{currentfill}{rgb}{0.961115,0.566634,0.405693}%
\pgfsetfillcolor{currentfill}%
\pgfsetlinewidth{0.000000pt}%
\definecolor{currentstroke}{rgb}{0.000000,0.000000,0.000000}%
\pgfsetstrokecolor{currentstroke}%
\pgfsetdash{}{0pt}%
\pgfpathmoveto{\pgfqpoint{1.074356in}{0.848692in}}%
\pgfpathlineto{\pgfqpoint{1.087625in}{0.846005in}}%
\pgfpathlineto{\pgfqpoint{1.100894in}{0.846559in}}%
\pgfpathlineto{\pgfqpoint{1.114163in}{0.854278in}}%
\pgfpathlineto{\pgfqpoint{1.115741in}{0.858201in}}%
\pgfpathlineto{\pgfqpoint{1.117553in}{0.871127in}}%
\pgfpathlineto{\pgfqpoint{1.117929in}{0.884054in}}%
\pgfpathlineto{\pgfqpoint{1.117522in}{0.896980in}}%
\pgfpathlineto{\pgfqpoint{1.115304in}{0.909907in}}%
\pgfpathlineto{\pgfqpoint{1.114163in}{0.912229in}}%
\pgfpathlineto{\pgfqpoint{1.100894in}{0.918757in}}%
\pgfpathlineto{\pgfqpoint{1.087625in}{0.919203in}}%
\pgfpathlineto{\pgfqpoint{1.074356in}{0.916979in}}%
\pgfpathlineto{\pgfqpoint{1.067649in}{0.909907in}}%
\pgfpathlineto{\pgfqpoint{1.065251in}{0.896980in}}%
\pgfpathlineto{\pgfqpoint{1.064808in}{0.884054in}}%
\pgfpathlineto{\pgfqpoint{1.065210in}{0.871127in}}%
\pgfpathlineto{\pgfqpoint{1.067159in}{0.858201in}}%
\pgfpathclose%
\pgfpathmoveto{\pgfqpoint{1.079680in}{0.858201in}}%
\pgfpathlineto{\pgfqpoint{1.074356in}{0.863811in}}%
\pgfpathlineto{\pgfqpoint{1.071858in}{0.871127in}}%
\pgfpathlineto{\pgfqpoint{1.070861in}{0.884054in}}%
\pgfpathlineto{\pgfqpoint{1.072412in}{0.896980in}}%
\pgfpathlineto{\pgfqpoint{1.074356in}{0.901756in}}%
\pgfpathlineto{\pgfqpoint{1.084938in}{0.909907in}}%
\pgfpathlineto{\pgfqpoint{1.087625in}{0.910885in}}%
\pgfpathlineto{\pgfqpoint{1.096514in}{0.909907in}}%
\pgfpathlineto{\pgfqpoint{1.100894in}{0.909181in}}%
\pgfpathlineto{\pgfqpoint{1.109618in}{0.896980in}}%
\pgfpathlineto{\pgfqpoint{1.111540in}{0.884054in}}%
\pgfpathlineto{\pgfqpoint{1.110303in}{0.871127in}}%
\pgfpathlineto{\pgfqpoint{1.103575in}{0.858201in}}%
\pgfpathlineto{\pgfqpoint{1.100894in}{0.856165in}}%
\pgfpathlineto{\pgfqpoint{1.087625in}{0.854533in}}%
\pgfpathclose%
\pgfusepath{fill}%
\end{pgfscope}%
\begin{pgfscope}%
\pgfpathrectangle{\pgfqpoint{0.211875in}{0.211875in}}{\pgfqpoint{1.313625in}{1.279725in}}%
\pgfusepath{clip}%
\pgfsetbuttcap%
\pgfsetroundjoin%
\definecolor{currentfill}{rgb}{0.961115,0.566634,0.405693}%
\pgfsetfillcolor{currentfill}%
\pgfsetlinewidth{0.000000pt}%
\definecolor{currentstroke}{rgb}{0.000000,0.000000,0.000000}%
\pgfsetstrokecolor{currentstroke}%
\pgfsetdash{}{0pt}%
\pgfpathmoveto{\pgfqpoint{1.193777in}{0.847939in}}%
\pgfpathlineto{\pgfqpoint{1.207045in}{0.845936in}}%
\pgfpathlineto{\pgfqpoint{1.220314in}{0.846881in}}%
\pgfpathlineto{\pgfqpoint{1.233557in}{0.858201in}}%
\pgfpathlineto{\pgfqpoint{1.233583in}{0.858301in}}%
\pgfpathlineto{\pgfqpoint{1.235419in}{0.871127in}}%
\pgfpathlineto{\pgfqpoint{1.235803in}{0.884054in}}%
\pgfpathlineto{\pgfqpoint{1.235381in}{0.896980in}}%
\pgfpathlineto{\pgfqpoint{1.233583in}{0.908166in}}%
\pgfpathlineto{\pgfqpoint{1.232985in}{0.909907in}}%
\pgfpathlineto{\pgfqpoint{1.220314in}{0.918483in}}%
\pgfpathlineto{\pgfqpoint{1.207045in}{0.919262in}}%
\pgfpathlineto{\pgfqpoint{1.193777in}{0.917623in}}%
\pgfpathlineto{\pgfqpoint{1.185427in}{0.909907in}}%
\pgfpathlineto{\pgfqpoint{1.183018in}{0.896980in}}%
\pgfpathlineto{\pgfqpoint{1.182577in}{0.884054in}}%
\pgfpathlineto{\pgfqpoint{1.182985in}{0.871127in}}%
\pgfpathlineto{\pgfqpoint{1.184953in}{0.858201in}}%
\pgfpathclose%
\pgfpathmoveto{\pgfqpoint{1.197193in}{0.858201in}}%
\pgfpathlineto{\pgfqpoint{1.193777in}{0.861010in}}%
\pgfpathlineto{\pgfqpoint{1.189862in}{0.871127in}}%
\pgfpathlineto{\pgfqpoint{1.188837in}{0.884054in}}%
\pgfpathlineto{\pgfqpoint{1.190429in}{0.896980in}}%
\pgfpathlineto{\pgfqpoint{1.193777in}{0.904204in}}%
\pgfpathlineto{\pgfqpoint{1.203328in}{0.909907in}}%
\pgfpathlineto{\pgfqpoint{1.207045in}{0.911047in}}%
\pgfpathlineto{\pgfqpoint{1.213846in}{0.909907in}}%
\pgfpathlineto{\pgfqpoint{1.220314in}{0.908169in}}%
\pgfpathlineto{\pgfqpoint{1.227423in}{0.896980in}}%
\pgfpathlineto{\pgfqpoint{1.229234in}{0.884054in}}%
\pgfpathlineto{\pgfqpoint{1.228069in}{0.871127in}}%
\pgfpathlineto{\pgfqpoint{1.221750in}{0.858201in}}%
\pgfpathlineto{\pgfqpoint{1.220314in}{0.856973in}}%
\pgfpathlineto{\pgfqpoint{1.207045in}{0.854360in}}%
\pgfpathclose%
\pgfusepath{fill}%
\end{pgfscope}%
\begin{pgfscope}%
\pgfpathrectangle{\pgfqpoint{0.211875in}{0.211875in}}{\pgfqpoint{1.313625in}{1.279725in}}%
\pgfusepath{clip}%
\pgfsetbuttcap%
\pgfsetroundjoin%
\definecolor{currentfill}{rgb}{0.961115,0.566634,0.405693}%
\pgfsetfillcolor{currentfill}%
\pgfsetlinewidth{0.000000pt}%
\definecolor{currentstroke}{rgb}{0.000000,0.000000,0.000000}%
\pgfsetstrokecolor{currentstroke}%
\pgfsetdash{}{0pt}%
\pgfpathmoveto{\pgfqpoint{1.313197in}{0.848103in}}%
\pgfpathlineto{\pgfqpoint{1.326466in}{0.846462in}}%
\pgfpathlineto{\pgfqpoint{1.339735in}{0.847999in}}%
\pgfpathlineto{\pgfqpoint{1.350345in}{0.858201in}}%
\pgfpathlineto{\pgfqpoint{1.352796in}{0.871127in}}%
\pgfpathlineto{\pgfqpoint{1.353004in}{0.876240in}}%
\pgfpathlineto{\pgfqpoint{1.353255in}{0.884054in}}%
\pgfpathlineto{\pgfqpoint{1.353004in}{0.890965in}}%
\pgfpathlineto{\pgfqpoint{1.352706in}{0.896980in}}%
\pgfpathlineto{\pgfqpoint{1.349600in}{0.909907in}}%
\pgfpathlineto{\pgfqpoint{1.339735in}{0.917439in}}%
\pgfpathlineto{\pgfqpoint{1.326466in}{0.918745in}}%
\pgfpathlineto{\pgfqpoint{1.313197in}{0.917385in}}%
\pgfpathlineto{\pgfqpoint{1.304020in}{0.909907in}}%
\pgfpathlineto{\pgfqpoint{1.301200in}{0.896980in}}%
\pgfpathlineto{\pgfqpoint{1.300678in}{0.884054in}}%
\pgfpathlineto{\pgfqpoint{1.301136in}{0.871127in}}%
\pgfpathlineto{\pgfqpoint{1.303386in}{0.858201in}}%
\pgfpathclose%
\pgfpathmoveto{\pgfqpoint{1.315932in}{0.858201in}}%
\pgfpathlineto{\pgfqpoint{1.313197in}{0.859915in}}%
\pgfpathlineto{\pgfqpoint{1.308308in}{0.871127in}}%
\pgfpathlineto{\pgfqpoint{1.307204in}{0.884054in}}%
\pgfpathlineto{\pgfqpoint{1.308930in}{0.896980in}}%
\pgfpathlineto{\pgfqpoint{1.313197in}{0.905123in}}%
\pgfpathlineto{\pgfqpoint{1.323787in}{0.909907in}}%
\pgfpathlineto{\pgfqpoint{1.326466in}{0.910577in}}%
\pgfpathlineto{\pgfqpoint{1.329456in}{0.909907in}}%
\pgfpathlineto{\pgfqpoint{1.339735in}{0.905962in}}%
\pgfpathlineto{\pgfqpoint{1.344808in}{0.896980in}}%
\pgfpathlineto{\pgfqpoint{1.346612in}{0.884054in}}%
\pgfpathlineto{\pgfqpoint{1.345462in}{0.871127in}}%
\pgfpathlineto{\pgfqpoint{1.339735in}{0.858930in}}%
\pgfpathlineto{\pgfqpoint{1.338351in}{0.858201in}}%
\pgfpathlineto{\pgfqpoint{1.326466in}{0.854839in}}%
\pgfpathclose%
\pgfusepath{fill}%
\end{pgfscope}%
\begin{pgfscope}%
\pgfpathrectangle{\pgfqpoint{0.211875in}{0.211875in}}{\pgfqpoint{1.313625in}{1.279725in}}%
\pgfusepath{clip}%
\pgfsetbuttcap%
\pgfsetroundjoin%
\definecolor{currentfill}{rgb}{0.961115,0.566634,0.405693}%
\pgfsetfillcolor{currentfill}%
\pgfsetlinewidth{0.000000pt}%
\definecolor{currentstroke}{rgb}{0.000000,0.000000,0.000000}%
\pgfsetstrokecolor{currentstroke}%
\pgfsetdash{}{0pt}%
\pgfpathmoveto{\pgfqpoint{1.432617in}{0.848997in}}%
\pgfpathlineto{\pgfqpoint{1.445886in}{0.847584in}}%
\pgfpathlineto{\pgfqpoint{1.459155in}{0.850089in}}%
\pgfpathlineto{\pgfqpoint{1.466647in}{0.858201in}}%
\pgfpathlineto{\pgfqpoint{1.469505in}{0.871127in}}%
\pgfpathlineto{\pgfqpoint{1.470069in}{0.884054in}}%
\pgfpathlineto{\pgfqpoint{1.469341in}{0.896980in}}%
\pgfpathlineto{\pgfqpoint{1.465583in}{0.909907in}}%
\pgfpathlineto{\pgfqpoint{1.459155in}{0.915451in}}%
\pgfpathlineto{\pgfqpoint{1.445886in}{0.917651in}}%
\pgfpathlineto{\pgfqpoint{1.432617in}{0.916448in}}%
\pgfpathlineto{\pgfqpoint{1.423540in}{0.909907in}}%
\pgfpathlineto{\pgfqpoint{1.419837in}{0.896980in}}%
\pgfpathlineto{\pgfqpoint{1.419348in}{0.888400in}}%
\pgfpathlineto{\pgfqpoint{1.419170in}{0.884054in}}%
\pgfpathlineto{\pgfqpoint{1.419348in}{0.878954in}}%
\pgfpathlineto{\pgfqpoint{1.419701in}{0.871127in}}%
\pgfpathlineto{\pgfqpoint{1.422540in}{0.858201in}}%
\pgfpathclose%
\pgfpathmoveto{\pgfqpoint{1.436909in}{0.858201in}}%
\pgfpathlineto{\pgfqpoint{1.432617in}{0.860162in}}%
\pgfpathlineto{\pgfqpoint{1.427245in}{0.871127in}}%
\pgfpathlineto{\pgfqpoint{1.426001in}{0.884054in}}%
\pgfpathlineto{\pgfqpoint{1.427971in}{0.896980in}}%
\pgfpathlineto{\pgfqpoint{1.432617in}{0.904849in}}%
\pgfpathlineto{\pgfqpoint{1.445886in}{0.909335in}}%
\pgfpathlineto{\pgfqpoint{1.459155in}{0.902282in}}%
\pgfpathlineto{\pgfqpoint{1.461813in}{0.896980in}}%
\pgfpathlineto{\pgfqpoint{1.463698in}{0.884054in}}%
\pgfpathlineto{\pgfqpoint{1.462512in}{0.871127in}}%
\pgfpathlineto{\pgfqpoint{1.459155in}{0.863096in}}%
\pgfpathlineto{\pgfqpoint{1.452210in}{0.858201in}}%
\pgfpathlineto{\pgfqpoint{1.445886in}{0.855967in}}%
\pgfpathclose%
\pgfusepath{fill}%
\end{pgfscope}%
\begin{pgfscope}%
\pgfpathrectangle{\pgfqpoint{0.211875in}{0.211875in}}{\pgfqpoint{1.313625in}{1.279725in}}%
\pgfusepath{clip}%
\pgfsetbuttcap%
\pgfsetroundjoin%
\definecolor{currentfill}{rgb}{0.961115,0.566634,0.405693}%
\pgfsetfillcolor{currentfill}%
\pgfsetlinewidth{0.000000pt}%
\definecolor{currentstroke}{rgb}{0.000000,0.000000,0.000000}%
\pgfsetstrokecolor{currentstroke}%
\pgfsetdash{}{0pt}%
\pgfpathmoveto{\pgfqpoint{0.264951in}{0.862866in}}%
\pgfpathlineto{\pgfqpoint{0.278220in}{0.869608in}}%
\pgfpathlineto{\pgfqpoint{0.278978in}{0.871127in}}%
\pgfpathlineto{\pgfqpoint{0.280668in}{0.884054in}}%
\pgfpathlineto{\pgfqpoint{0.278220in}{0.895591in}}%
\pgfpathlineto{\pgfqpoint{0.277164in}{0.896980in}}%
\pgfpathlineto{\pgfqpoint{0.264951in}{0.902292in}}%
\pgfpathlineto{\pgfqpoint{0.257013in}{0.896980in}}%
\pgfpathlineto{\pgfqpoint{0.251682in}{0.885223in}}%
\pgfpathlineto{\pgfqpoint{0.251477in}{0.884054in}}%
\pgfpathlineto{\pgfqpoint{0.251682in}{0.882247in}}%
\pgfpathlineto{\pgfqpoint{0.254725in}{0.871127in}}%
\pgfpathclose%
\pgfusepath{fill}%
\end{pgfscope}%
\begin{pgfscope}%
\pgfpathrectangle{\pgfqpoint{0.211875in}{0.211875in}}{\pgfqpoint{1.313625in}{1.279725in}}%
\pgfusepath{clip}%
\pgfsetbuttcap%
\pgfsetroundjoin%
\definecolor{currentfill}{rgb}{0.961115,0.566634,0.405693}%
\pgfsetfillcolor{currentfill}%
\pgfsetlinewidth{0.000000pt}%
\definecolor{currentstroke}{rgb}{0.000000,0.000000,0.000000}%
\pgfsetstrokecolor{currentstroke}%
\pgfsetdash{}{0pt}%
\pgfpathmoveto{\pgfqpoint{0.556867in}{0.933292in}}%
\pgfpathlineto{\pgfqpoint{0.570136in}{0.934199in}}%
\pgfpathlineto{\pgfqpoint{0.572967in}{0.935760in}}%
\pgfpathlineto{\pgfqpoint{0.581613in}{0.948686in}}%
\pgfpathlineto{\pgfqpoint{0.583405in}{0.961575in}}%
\pgfpathlineto{\pgfqpoint{0.583409in}{0.961613in}}%
\pgfpathlineto{\pgfqpoint{0.583405in}{0.961694in}}%
\pgfpathlineto{\pgfqpoint{0.582635in}{0.974539in}}%
\pgfpathlineto{\pgfqpoint{0.577633in}{0.987466in}}%
\pgfpathlineto{\pgfqpoint{0.570136in}{0.993071in}}%
\pgfpathlineto{\pgfqpoint{0.556867in}{0.993987in}}%
\pgfpathlineto{\pgfqpoint{0.543918in}{0.987466in}}%
\pgfpathlineto{\pgfqpoint{0.543598in}{0.987085in}}%
\pgfpathlineto{\pgfqpoint{0.539644in}{0.974539in}}%
\pgfpathlineto{\pgfqpoint{0.538983in}{0.961613in}}%
\pgfpathlineto{\pgfqpoint{0.540487in}{0.948686in}}%
\pgfpathlineto{\pgfqpoint{0.543598in}{0.941139in}}%
\pgfpathlineto{\pgfqpoint{0.550377in}{0.935760in}}%
\pgfpathclose%
\pgfpathmoveto{\pgfqpoint{0.550602in}{0.948686in}}%
\pgfpathlineto{\pgfqpoint{0.545980in}{0.961613in}}%
\pgfpathlineto{\pgfqpoint{0.548150in}{0.974539in}}%
\pgfpathlineto{\pgfqpoint{0.556867in}{0.984222in}}%
\pgfpathlineto{\pgfqpoint{0.570136in}{0.981294in}}%
\pgfpathlineto{\pgfqpoint{0.574114in}{0.974539in}}%
\pgfpathlineto{\pgfqpoint{0.575646in}{0.961613in}}%
\pgfpathlineto{\pgfqpoint{0.572358in}{0.948686in}}%
\pgfpathlineto{\pgfqpoint{0.570136in}{0.945786in}}%
\pgfpathlineto{\pgfqpoint{0.556867in}{0.943298in}}%
\pgfpathclose%
\pgfusepath{fill}%
\end{pgfscope}%
\begin{pgfscope}%
\pgfpathrectangle{\pgfqpoint{0.211875in}{0.211875in}}{\pgfqpoint{1.313625in}{1.279725in}}%
\pgfusepath{clip}%
\pgfsetbuttcap%
\pgfsetroundjoin%
\definecolor{currentfill}{rgb}{0.961115,0.566634,0.405693}%
\pgfsetfillcolor{currentfill}%
\pgfsetlinewidth{0.000000pt}%
\definecolor{currentstroke}{rgb}{0.000000,0.000000,0.000000}%
\pgfsetstrokecolor{currentstroke}%
\pgfsetdash{}{0pt}%
\pgfpathmoveto{\pgfqpoint{0.663019in}{0.934679in}}%
\pgfpathlineto{\pgfqpoint{0.676288in}{0.930723in}}%
\pgfpathlineto{\pgfqpoint{0.689557in}{0.931932in}}%
\pgfpathlineto{\pgfqpoint{0.695689in}{0.935760in}}%
\pgfpathlineto{\pgfqpoint{0.701781in}{0.948686in}}%
\pgfpathlineto{\pgfqpoint{0.702826in}{0.958840in}}%
\pgfpathlineto{\pgfqpoint{0.703019in}{0.961613in}}%
\pgfpathlineto{\pgfqpoint{0.702826in}{0.967826in}}%
\pgfpathlineto{\pgfqpoint{0.702542in}{0.974539in}}%
\pgfpathlineto{\pgfqpoint{0.699081in}{0.987466in}}%
\pgfpathlineto{\pgfqpoint{0.689557in}{0.995514in}}%
\pgfpathlineto{\pgfqpoint{0.676288in}{0.996716in}}%
\pgfpathlineto{\pgfqpoint{0.663019in}{0.992692in}}%
\pgfpathlineto{\pgfqpoint{0.658842in}{0.987466in}}%
\pgfpathlineto{\pgfqpoint{0.655605in}{0.974539in}}%
\pgfpathlineto{\pgfqpoint{0.655100in}{0.961613in}}%
\pgfpathlineto{\pgfqpoint{0.656306in}{0.948686in}}%
\pgfpathlineto{\pgfqpoint{0.661861in}{0.935760in}}%
\pgfpathclose%
\pgfpathmoveto{\pgfqpoint{0.664141in}{0.948686in}}%
\pgfpathlineto{\pgfqpoint{0.663019in}{0.951413in}}%
\pgfpathlineto{\pgfqpoint{0.661244in}{0.961613in}}%
\pgfpathlineto{\pgfqpoint{0.662326in}{0.974539in}}%
\pgfpathlineto{\pgfqpoint{0.663019in}{0.976473in}}%
\pgfpathlineto{\pgfqpoint{0.674771in}{0.987466in}}%
\pgfpathlineto{\pgfqpoint{0.676288in}{0.988117in}}%
\pgfpathlineto{\pgfqpoint{0.679729in}{0.987466in}}%
\pgfpathlineto{\pgfqpoint{0.689557in}{0.984179in}}%
\pgfpathlineto{\pgfqpoint{0.694594in}{0.974539in}}%
\pgfpathlineto{\pgfqpoint{0.695820in}{0.961613in}}%
\pgfpathlineto{\pgfqpoint{0.693154in}{0.948686in}}%
\pgfpathlineto{\pgfqpoint{0.689557in}{0.943397in}}%
\pgfpathlineto{\pgfqpoint{0.676288in}{0.939771in}}%
\pgfpathclose%
\pgfusepath{fill}%
\end{pgfscope}%
\begin{pgfscope}%
\pgfpathrectangle{\pgfqpoint{0.211875in}{0.211875in}}{\pgfqpoint{1.313625in}{1.279725in}}%
\pgfusepath{clip}%
\pgfsetbuttcap%
\pgfsetroundjoin%
\definecolor{currentfill}{rgb}{0.961115,0.566634,0.405693}%
\pgfsetfillcolor{currentfill}%
\pgfsetlinewidth{0.000000pt}%
\definecolor{currentstroke}{rgb}{0.000000,0.000000,0.000000}%
\pgfsetstrokecolor{currentstroke}%
\pgfsetdash{}{0pt}%
\pgfpathmoveto{\pgfqpoint{0.782439in}{0.931145in}}%
\pgfpathlineto{\pgfqpoint{0.795708in}{0.928795in}}%
\pgfpathlineto{\pgfqpoint{0.808977in}{0.930160in}}%
\pgfpathlineto{\pgfqpoint{0.816920in}{0.935760in}}%
\pgfpathlineto{\pgfqpoint{0.821263in}{0.948686in}}%
\pgfpathlineto{\pgfqpoint{0.822193in}{0.961613in}}%
\pgfpathlineto{\pgfqpoint{0.821842in}{0.974539in}}%
\pgfpathlineto{\pgfqpoint{0.819440in}{0.987466in}}%
\pgfpathlineto{\pgfqpoint{0.808977in}{0.997439in}}%
\pgfpathlineto{\pgfqpoint{0.795708in}{0.998766in}}%
\pgfpathlineto{\pgfqpoint{0.782439in}{0.996427in}}%
\pgfpathlineto{\pgfqpoint{0.774336in}{0.987466in}}%
\pgfpathlineto{\pgfqpoint{0.771821in}{0.974539in}}%
\pgfpathlineto{\pgfqpoint{0.771441in}{0.961613in}}%
\pgfpathlineto{\pgfqpoint{0.772409in}{0.948686in}}%
\pgfpathlineto{\pgfqpoint{0.776843in}{0.935760in}}%
\pgfpathclose%
\pgfpathmoveto{\pgfqpoint{0.779986in}{0.948686in}}%
\pgfpathlineto{\pgfqpoint{0.777836in}{0.961613in}}%
\pgfpathlineto{\pgfqpoint{0.778818in}{0.974539in}}%
\pgfpathlineto{\pgfqpoint{0.782439in}{0.983477in}}%
\pgfpathlineto{\pgfqpoint{0.787976in}{0.987466in}}%
\pgfpathlineto{\pgfqpoint{0.795708in}{0.990222in}}%
\pgfpathlineto{\pgfqpoint{0.806232in}{0.987466in}}%
\pgfpathlineto{\pgfqpoint{0.808977in}{0.986108in}}%
\pgfpathlineto{\pgfqpoint{0.814350in}{0.974539in}}%
\pgfpathlineto{\pgfqpoint{0.815360in}{0.961613in}}%
\pgfpathlineto{\pgfqpoint{0.813134in}{0.948686in}}%
\pgfpathlineto{\pgfqpoint{0.808977in}{0.941815in}}%
\pgfpathlineto{\pgfqpoint{0.795708in}{0.937159in}}%
\pgfpathlineto{\pgfqpoint{0.782439in}{0.944037in}}%
\pgfpathclose%
\pgfusepath{fill}%
\end{pgfscope}%
\begin{pgfscope}%
\pgfpathrectangle{\pgfqpoint{0.211875in}{0.211875in}}{\pgfqpoint{1.313625in}{1.279725in}}%
\pgfusepath{clip}%
\pgfsetbuttcap%
\pgfsetroundjoin%
\definecolor{currentfill}{rgb}{0.961115,0.566634,0.405693}%
\pgfsetfillcolor{currentfill}%
\pgfsetlinewidth{0.000000pt}%
\definecolor{currentstroke}{rgb}{0.000000,0.000000,0.000000}%
\pgfsetstrokecolor{currentstroke}%
\pgfsetdash{}{0pt}%
\pgfpathmoveto{\pgfqpoint{0.901860in}{0.928831in}}%
\pgfpathlineto{\pgfqpoint{0.915129in}{0.927458in}}%
\pgfpathlineto{\pgfqpoint{0.928398in}{0.928936in}}%
\pgfpathlineto{\pgfqpoint{0.936966in}{0.935760in}}%
\pgfpathlineto{\pgfqpoint{0.940183in}{0.948686in}}%
\pgfpathlineto{\pgfqpoint{0.940884in}{0.961613in}}%
\pgfpathlineto{\pgfqpoint{0.940643in}{0.974539in}}%
\pgfpathlineto{\pgfqpoint{0.938921in}{0.987466in}}%
\pgfpathlineto{\pgfqpoint{0.928398in}{0.998791in}}%
\pgfpathlineto{\pgfqpoint{0.915129in}{1.000189in}}%
\pgfpathlineto{\pgfqpoint{0.901860in}{0.998858in}}%
\pgfpathlineto{\pgfqpoint{0.890249in}{0.987466in}}%
\pgfpathlineto{\pgfqpoint{0.888591in}{0.977472in}}%
\pgfpathlineto{\pgfqpoint{0.888313in}{0.974539in}}%
\pgfpathlineto{\pgfqpoint{0.888066in}{0.961613in}}%
\pgfpathlineto{\pgfqpoint{0.888591in}{0.951348in}}%
\pgfpathlineto{\pgfqpoint{0.888778in}{0.948686in}}%
\pgfpathlineto{\pgfqpoint{0.892390in}{0.935760in}}%
\pgfpathclose%
\pgfpathmoveto{\pgfqpoint{0.913841in}{0.935760in}}%
\pgfpathlineto{\pgfqpoint{0.901860in}{0.940044in}}%
\pgfpathlineto{\pgfqpoint{0.896735in}{0.948686in}}%
\pgfpathlineto{\pgfqpoint{0.894695in}{0.961613in}}%
\pgfpathlineto{\pgfqpoint{0.895615in}{0.974539in}}%
\pgfpathlineto{\pgfqpoint{0.901332in}{0.987466in}}%
\pgfpathlineto{\pgfqpoint{0.901860in}{0.987984in}}%
\pgfpathlineto{\pgfqpoint{0.915129in}{0.991644in}}%
\pgfpathlineto{\pgfqpoint{0.927636in}{0.987466in}}%
\pgfpathlineto{\pgfqpoint{0.928398in}{0.986951in}}%
\pgfpathlineto{\pgfqpoint{0.933513in}{0.974539in}}%
\pgfpathlineto{\pgfqpoint{0.934379in}{0.961613in}}%
\pgfpathlineto{\pgfqpoint{0.932450in}{0.948686in}}%
\pgfpathlineto{\pgfqpoint{0.928398in}{0.941149in}}%
\pgfpathlineto{\pgfqpoint{0.916242in}{0.935760in}}%
\pgfpathlineto{\pgfqpoint{0.915129in}{0.935484in}}%
\pgfpathclose%
\pgfusepath{fill}%
\end{pgfscope}%
\begin{pgfscope}%
\pgfpathrectangle{\pgfqpoint{0.211875in}{0.211875in}}{\pgfqpoint{1.313625in}{1.279725in}}%
\pgfusepath{clip}%
\pgfsetbuttcap%
\pgfsetroundjoin%
\definecolor{currentfill}{rgb}{0.961115,0.566634,0.405693}%
\pgfsetfillcolor{currentfill}%
\pgfsetlinewidth{0.000000pt}%
\definecolor{currentstroke}{rgb}{0.000000,0.000000,0.000000}%
\pgfsetstrokecolor{currentstroke}%
\pgfsetdash{}{0pt}%
\pgfpathmoveto{\pgfqpoint{1.021280in}{0.927476in}}%
\pgfpathlineto{\pgfqpoint{1.034549in}{0.926687in}}%
\pgfpathlineto{\pgfqpoint{1.047818in}{0.928357in}}%
\pgfpathlineto{\pgfqpoint{1.056023in}{0.935760in}}%
\pgfpathlineto{\pgfqpoint{1.058622in}{0.948686in}}%
\pgfpathlineto{\pgfqpoint{1.059195in}{0.961613in}}%
\pgfpathlineto{\pgfqpoint{1.059015in}{0.974539in}}%
\pgfpathlineto{\pgfqpoint{1.057662in}{0.987466in}}%
\pgfpathlineto{\pgfqpoint{1.047818in}{0.999468in}}%
\pgfpathlineto{\pgfqpoint{1.041146in}{1.000392in}}%
\pgfpathlineto{\pgfqpoint{1.034549in}{1.000968in}}%
\pgfpathlineto{\pgfqpoint{1.023159in}{1.000392in}}%
\pgfpathlineto{\pgfqpoint{1.021280in}{1.000270in}}%
\pgfpathlineto{\pgfqpoint{1.008011in}{0.991285in}}%
\pgfpathlineto{\pgfqpoint{1.006858in}{0.987466in}}%
\pgfpathlineto{\pgfqpoint{1.005520in}{0.974539in}}%
\pgfpathlineto{\pgfqpoint{1.005337in}{0.961613in}}%
\pgfpathlineto{\pgfqpoint{1.005898in}{0.948686in}}%
\pgfpathlineto{\pgfqpoint{1.008011in}{0.937063in}}%
\pgfpathlineto{\pgfqpoint{1.008525in}{0.935760in}}%
\pgfpathclose%
\pgfpathmoveto{\pgfqpoint{1.028290in}{0.935760in}}%
\pgfpathlineto{\pgfqpoint{1.021280in}{0.937505in}}%
\pgfpathlineto{\pgfqpoint{1.013836in}{0.948686in}}%
\pgfpathlineto{\pgfqpoint{1.011819in}{0.961613in}}%
\pgfpathlineto{\pgfqpoint{1.012720in}{0.974539in}}%
\pgfpathlineto{\pgfqpoint{1.018355in}{0.987466in}}%
\pgfpathlineto{\pgfqpoint{1.021280in}{0.990015in}}%
\pgfpathlineto{\pgfqpoint{1.034549in}{0.992412in}}%
\pgfpathlineto{\pgfqpoint{1.046709in}{0.987466in}}%
\pgfpathlineto{\pgfqpoint{1.047818in}{0.986481in}}%
\pgfpathlineto{\pgfqpoint{1.052171in}{0.974539in}}%
\pgfpathlineto{\pgfqpoint{1.052949in}{0.961613in}}%
\pgfpathlineto{\pgfqpoint{1.051201in}{0.948686in}}%
\pgfpathlineto{\pgfqpoint{1.047818in}{0.941583in}}%
\pgfpathlineto{\pgfqpoint{1.037835in}{0.935760in}}%
\pgfpathlineto{\pgfqpoint{1.034549in}{0.934766in}}%
\pgfpathclose%
\pgfusepath{fill}%
\end{pgfscope}%
\begin{pgfscope}%
\pgfpathrectangle{\pgfqpoint{0.211875in}{0.211875in}}{\pgfqpoint{1.313625in}{1.279725in}}%
\pgfusepath{clip}%
\pgfsetbuttcap%
\pgfsetroundjoin%
\definecolor{currentfill}{rgb}{0.961115,0.566634,0.405693}%
\pgfsetfillcolor{currentfill}%
\pgfsetlinewidth{0.000000pt}%
\definecolor{currentstroke}{rgb}{0.000000,0.000000,0.000000}%
\pgfsetstrokecolor{currentstroke}%
\pgfsetdash{}{0pt}%
\pgfpathmoveto{\pgfqpoint{1.127432in}{0.932757in}}%
\pgfpathlineto{\pgfqpoint{1.140701in}{0.926902in}}%
\pgfpathlineto{\pgfqpoint{1.153970in}{0.926478in}}%
\pgfpathlineto{\pgfqpoint{1.167239in}{0.928581in}}%
\pgfpathlineto{\pgfqpoint{1.174220in}{0.935760in}}%
\pgfpathlineto{\pgfqpoint{1.176626in}{0.948686in}}%
\pgfpathlineto{\pgfqpoint{1.177161in}{0.961613in}}%
\pgfpathlineto{\pgfqpoint{1.176996in}{0.974539in}}%
\pgfpathlineto{\pgfqpoint{1.175751in}{0.987466in}}%
\pgfpathlineto{\pgfqpoint{1.167239in}{0.999299in}}%
\pgfpathlineto{\pgfqpoint{1.161327in}{1.000392in}}%
\pgfpathlineto{\pgfqpoint{1.153970in}{1.001181in}}%
\pgfpathlineto{\pgfqpoint{1.140701in}{1.000820in}}%
\pgfpathlineto{\pgfqpoint{1.138220in}{1.000392in}}%
\pgfpathlineto{\pgfqpoint{1.127432in}{0.995345in}}%
\pgfpathlineto{\pgfqpoint{1.124368in}{0.987466in}}%
\pgfpathlineto{\pgfqpoint{1.123182in}{0.974539in}}%
\pgfpathlineto{\pgfqpoint{1.123025in}{0.961613in}}%
\pgfpathlineto{\pgfqpoint{1.123535in}{0.948686in}}%
\pgfpathlineto{\pgfqpoint{1.125827in}{0.935760in}}%
\pgfpathclose%
\pgfpathmoveto{\pgfqpoint{1.143279in}{0.935760in}}%
\pgfpathlineto{\pgfqpoint{1.140701in}{0.936150in}}%
\pgfpathlineto{\pgfqpoint{1.131318in}{0.948686in}}%
\pgfpathlineto{\pgfqpoint{1.129220in}{0.961613in}}%
\pgfpathlineto{\pgfqpoint{1.130152in}{0.974539in}}%
\pgfpathlineto{\pgfqpoint{1.136012in}{0.987466in}}%
\pgfpathlineto{\pgfqpoint{1.140701in}{0.991090in}}%
\pgfpathlineto{\pgfqpoint{1.153970in}{0.992527in}}%
\pgfpathlineto{\pgfqpoint{1.164496in}{0.987466in}}%
\pgfpathlineto{\pgfqpoint{1.167239in}{0.984336in}}%
\pgfpathlineto{\pgfqpoint{1.170375in}{0.974539in}}%
\pgfpathlineto{\pgfqpoint{1.171115in}{0.961613in}}%
\pgfpathlineto{\pgfqpoint{1.169449in}{0.948686in}}%
\pgfpathlineto{\pgfqpoint{1.167239in}{0.943415in}}%
\pgfpathlineto{\pgfqpoint{1.157029in}{0.935760in}}%
\pgfpathlineto{\pgfqpoint{1.153970in}{0.934663in}}%
\pgfpathclose%
\pgfusepath{fill}%
\end{pgfscope}%
\begin{pgfscope}%
\pgfpathrectangle{\pgfqpoint{0.211875in}{0.211875in}}{\pgfqpoint{1.313625in}{1.279725in}}%
\pgfusepath{clip}%
\pgfsetbuttcap%
\pgfsetroundjoin%
\definecolor{currentfill}{rgb}{0.961115,0.566634,0.405693}%
\pgfsetfillcolor{currentfill}%
\pgfsetlinewidth{0.000000pt}%
\definecolor{currentstroke}{rgb}{0.000000,0.000000,0.000000}%
\pgfsetstrokecolor{currentstroke}%
\pgfsetdash{}{0pt}%
\pgfpathmoveto{\pgfqpoint{1.246852in}{0.931461in}}%
\pgfpathlineto{\pgfqpoint{1.260121in}{0.926995in}}%
\pgfpathlineto{\pgfqpoint{1.273390in}{0.926849in}}%
\pgfpathlineto{\pgfqpoint{1.286659in}{0.929877in}}%
\pgfpathlineto{\pgfqpoint{1.291633in}{0.935760in}}%
\pgfpathlineto{\pgfqpoint{1.294222in}{0.948686in}}%
\pgfpathlineto{\pgfqpoint{1.294795in}{0.961613in}}%
\pgfpathlineto{\pgfqpoint{1.294608in}{0.974539in}}%
\pgfpathlineto{\pgfqpoint{1.293239in}{0.987466in}}%
\pgfpathlineto{\pgfqpoint{1.286659in}{0.998000in}}%
\pgfpathlineto{\pgfqpoint{1.276773in}{1.000392in}}%
\pgfpathlineto{\pgfqpoint{1.273390in}{1.000825in}}%
\pgfpathlineto{\pgfqpoint{1.260121in}{1.000707in}}%
\pgfpathlineto{\pgfqpoint{1.258007in}{1.000392in}}%
\pgfpathlineto{\pgfqpoint{1.246852in}{0.996534in}}%
\pgfpathlineto{\pgfqpoint{1.242531in}{0.987466in}}%
\pgfpathlineto{\pgfqpoint{1.241275in}{0.974539in}}%
\pgfpathlineto{\pgfqpoint{1.241107in}{0.961613in}}%
\pgfpathlineto{\pgfqpoint{1.241640in}{0.948686in}}%
\pgfpathlineto{\pgfqpoint{1.244052in}{0.935760in}}%
\pgfpathclose%
\pgfpathmoveto{\pgfqpoint{1.260756in}{0.935760in}}%
\pgfpathlineto{\pgfqpoint{1.260121in}{0.935799in}}%
\pgfpathlineto{\pgfqpoint{1.249237in}{0.948686in}}%
\pgfpathlineto{\pgfqpoint{1.246930in}{0.961613in}}%
\pgfpathlineto{\pgfqpoint{1.247957in}{0.974539in}}%
\pgfpathlineto{\pgfqpoint{1.254426in}{0.987466in}}%
\pgfpathlineto{\pgfqpoint{1.260121in}{0.991357in}}%
\pgfpathlineto{\pgfqpoint{1.273390in}{0.991967in}}%
\pgfpathlineto{\pgfqpoint{1.281464in}{0.987466in}}%
\pgfpathlineto{\pgfqpoint{1.286659in}{0.979915in}}%
\pgfpathlineto{\pgfqpoint{1.288155in}{0.974539in}}%
\pgfpathlineto{\pgfqpoint{1.288902in}{0.961613in}}%
\pgfpathlineto{\pgfqpoint{1.287229in}{0.948686in}}%
\pgfpathlineto{\pgfqpoint{1.286659in}{0.947128in}}%
\pgfpathlineto{\pgfqpoint{1.274742in}{0.935760in}}%
\pgfpathlineto{\pgfqpoint{1.273390in}{0.935197in}}%
\pgfpathclose%
\pgfusepath{fill}%
\end{pgfscope}%
\begin{pgfscope}%
\pgfpathrectangle{\pgfqpoint{0.211875in}{0.211875in}}{\pgfqpoint{1.313625in}{1.279725in}}%
\pgfusepath{clip}%
\pgfsetbuttcap%
\pgfsetroundjoin%
\definecolor{currentfill}{rgb}{0.961115,0.566634,0.405693}%
\pgfsetfillcolor{currentfill}%
\pgfsetlinewidth{0.000000pt}%
\definecolor{currentstroke}{rgb}{0.000000,0.000000,0.000000}%
\pgfsetstrokecolor{currentstroke}%
\pgfsetdash{}{0pt}%
\pgfpathmoveto{\pgfqpoint{1.366273in}{0.931667in}}%
\pgfpathlineto{\pgfqpoint{1.379542in}{0.927682in}}%
\pgfpathlineto{\pgfqpoint{1.392811in}{0.927842in}}%
\pgfpathlineto{\pgfqpoint{1.406080in}{0.932698in}}%
\pgfpathlineto{\pgfqpoint{1.408299in}{0.935760in}}%
\pgfpathlineto{\pgfqpoint{1.411414in}{0.948686in}}%
\pgfpathlineto{\pgfqpoint{1.412099in}{0.961613in}}%
\pgfpathlineto{\pgfqpoint{1.411853in}{0.974539in}}%
\pgfpathlineto{\pgfqpoint{1.410146in}{0.987466in}}%
\pgfpathlineto{\pgfqpoint{1.406080in}{0.995076in}}%
\pgfpathlineto{\pgfqpoint{1.392811in}{0.999815in}}%
\pgfpathlineto{\pgfqpoint{1.379542in}{0.999982in}}%
\pgfpathlineto{\pgfqpoint{1.366273in}{0.996159in}}%
\pgfpathlineto{\pgfqpoint{1.361348in}{0.987466in}}%
\pgfpathlineto{\pgfqpoint{1.359792in}{0.974539in}}%
\pgfpathlineto{\pgfqpoint{1.359574in}{0.961613in}}%
\pgfpathlineto{\pgfqpoint{1.360208in}{0.948686in}}%
\pgfpathlineto{\pgfqpoint{1.363113in}{0.935760in}}%
\pgfpathclose%
\pgfpathmoveto{\pgfqpoint{1.367695in}{0.948686in}}%
\pgfpathlineto{\pgfqpoint{1.366273in}{0.954975in}}%
\pgfpathlineto{\pgfqpoint{1.365453in}{0.961613in}}%
\pgfpathlineto{\pgfqpoint{1.366235in}{0.974539in}}%
\pgfpathlineto{\pgfqpoint{1.366273in}{0.974691in}}%
\pgfpathlineto{\pgfqpoint{1.373798in}{0.987466in}}%
\pgfpathlineto{\pgfqpoint{1.379542in}{0.990909in}}%
\pgfpathlineto{\pgfqpoint{1.392811in}{0.990679in}}%
\pgfpathlineto{\pgfqpoint{1.397849in}{0.987466in}}%
\pgfpathlineto{\pgfqpoint{1.405240in}{0.974539in}}%
\pgfpathlineto{\pgfqpoint{1.406080in}{0.965675in}}%
\pgfpathlineto{\pgfqpoint{1.406314in}{0.961613in}}%
\pgfpathlineto{\pgfqpoint{1.406080in}{0.959623in}}%
\pgfpathlineto{\pgfqpoint{1.403790in}{0.948686in}}%
\pgfpathlineto{\pgfqpoint{1.392811in}{0.936625in}}%
\pgfpathlineto{\pgfqpoint{1.379542in}{0.936336in}}%
\pgfpathclose%
\pgfusepath{fill}%
\end{pgfscope}%
\begin{pgfscope}%
\pgfpathrectangle{\pgfqpoint{0.211875in}{0.211875in}}{\pgfqpoint{1.313625in}{1.279725in}}%
\pgfusepath{clip}%
\pgfsetbuttcap%
\pgfsetroundjoin%
\definecolor{currentfill}{rgb}{0.961115,0.566634,0.405693}%
\pgfsetfillcolor{currentfill}%
\pgfsetlinewidth{0.000000pt}%
\definecolor{currentstroke}{rgb}{0.000000,0.000000,0.000000}%
\pgfsetstrokecolor{currentstroke}%
\pgfsetdash{}{0pt}%
\pgfpathmoveto{\pgfqpoint{1.485693in}{0.932825in}}%
\pgfpathlineto{\pgfqpoint{1.498962in}{0.928920in}}%
\pgfpathlineto{\pgfqpoint{1.512231in}{0.929528in}}%
\pgfpathlineto{\pgfqpoint{1.523731in}{0.935760in}}%
\pgfpathlineto{\pgfqpoint{1.525500in}{0.938522in}}%
\pgfpathlineto{\pgfqpoint{1.525500in}{0.948686in}}%
\pgfpathlineto{\pgfqpoint{1.525500in}{0.961613in}}%
\pgfpathlineto{\pgfqpoint{1.525500in}{0.974539in}}%
\pgfpathlineto{\pgfqpoint{1.525500in}{0.987466in}}%
\pgfpathlineto{\pgfqpoint{1.525500in}{0.989633in}}%
\pgfpathlineto{\pgfqpoint{1.512231in}{0.998035in}}%
\pgfpathlineto{\pgfqpoint{1.498962in}{0.998649in}}%
\pgfpathlineto{\pgfqpoint{1.485693in}{0.994805in}}%
\pgfpathlineto{\pgfqpoint{1.480850in}{0.987466in}}%
\pgfpathlineto{\pgfqpoint{1.478742in}{0.974539in}}%
\pgfpathlineto{\pgfqpoint{1.478433in}{0.961613in}}%
\pgfpathlineto{\pgfqpoint{1.479250in}{0.948686in}}%
\pgfpathlineto{\pgfqpoint{1.483059in}{0.935760in}}%
\pgfpathclose%
\pgfpathmoveto{\pgfqpoint{1.486864in}{0.948686in}}%
\pgfpathlineto{\pgfqpoint{1.485693in}{0.952773in}}%
\pgfpathlineto{\pgfqpoint{1.484433in}{0.961613in}}%
\pgfpathlineto{\pgfqpoint{1.485319in}{0.974539in}}%
\pgfpathlineto{\pgfqpoint{1.485693in}{0.975821in}}%
\pgfpathlineto{\pgfqpoint{1.494471in}{0.987466in}}%
\pgfpathlineto{\pgfqpoint{1.498962in}{0.989804in}}%
\pgfpathlineto{\pgfqpoint{1.512231in}{0.988578in}}%
\pgfpathlineto{\pgfqpoint{1.513769in}{0.987466in}}%
\pgfpathlineto{\pgfqpoint{1.521280in}{0.974539in}}%
\pgfpathlineto{\pgfqpoint{1.522500in}{0.961613in}}%
\pgfpathlineto{\pgfqpoint{1.519829in}{0.948686in}}%
\pgfpathlineto{\pgfqpoint{1.512231in}{0.939241in}}%
\pgfpathlineto{\pgfqpoint{1.498962in}{0.937693in}}%
\pgfpathclose%
\pgfusepath{fill}%
\end{pgfscope}%
\begin{pgfscope}%
\pgfpathrectangle{\pgfqpoint{0.211875in}{0.211875in}}{\pgfqpoint{1.313625in}{1.279725in}}%
\pgfusepath{clip}%
\pgfsetbuttcap%
\pgfsetroundjoin%
\definecolor{currentfill}{rgb}{0.961115,0.566634,0.405693}%
\pgfsetfillcolor{currentfill}%
\pgfsetlinewidth{0.000000pt}%
\definecolor{currentstroke}{rgb}{0.000000,0.000000,0.000000}%
\pgfsetstrokecolor{currentstroke}%
\pgfsetdash{}{0pt}%
\pgfpathmoveto{\pgfqpoint{0.214447in}{0.948686in}}%
\pgfpathlineto{\pgfqpoint{0.219395in}{0.961613in}}%
\pgfpathlineto{\pgfqpoint{0.217073in}{0.974539in}}%
\pgfpathlineto{\pgfqpoint{0.211875in}{0.980497in}}%
\pgfpathlineto{\pgfqpoint{0.211875in}{0.974539in}}%
\pgfpathlineto{\pgfqpoint{0.211875in}{0.961613in}}%
\pgfpathlineto{\pgfqpoint{0.211875in}{0.948686in}}%
\pgfpathlineto{\pgfqpoint{0.211875in}{0.946421in}}%
\pgfpathclose%
\pgfusepath{fill}%
\end{pgfscope}%
\begin{pgfscope}%
\pgfpathrectangle{\pgfqpoint{0.211875in}{0.211875in}}{\pgfqpoint{1.313625in}{1.279725in}}%
\pgfusepath{clip}%
\pgfsetbuttcap%
\pgfsetroundjoin%
\definecolor{currentfill}{rgb}{0.961115,0.566634,0.405693}%
\pgfsetfillcolor{currentfill}%
\pgfsetlinewidth{0.000000pt}%
\definecolor{currentstroke}{rgb}{0.000000,0.000000,0.000000}%
\pgfsetstrokecolor{currentstroke}%
\pgfsetdash{}{0pt}%
\pgfpathmoveto{\pgfqpoint{0.318027in}{0.942256in}}%
\pgfpathlineto{\pgfqpoint{0.331295in}{0.941544in}}%
\pgfpathlineto{\pgfqpoint{0.338339in}{0.948686in}}%
\pgfpathlineto{\pgfqpoint{0.341866in}{0.961613in}}%
\pgfpathlineto{\pgfqpoint{0.340242in}{0.974539in}}%
\pgfpathlineto{\pgfqpoint{0.331295in}{0.986338in}}%
\pgfpathlineto{\pgfqpoint{0.318027in}{0.985488in}}%
\pgfpathlineto{\pgfqpoint{0.310510in}{0.974539in}}%
\pgfpathlineto{\pgfqpoint{0.308928in}{0.961613in}}%
\pgfpathlineto{\pgfqpoint{0.312322in}{0.948686in}}%
\pgfpathclose%
\pgfpathmoveto{\pgfqpoint{0.317427in}{0.961613in}}%
\pgfpathlineto{\pgfqpoint{0.318027in}{0.965115in}}%
\pgfpathlineto{\pgfqpoint{0.331295in}{0.968515in}}%
\pgfpathlineto{\pgfqpoint{0.332593in}{0.961613in}}%
\pgfpathlineto{\pgfqpoint{0.331295in}{0.958027in}}%
\pgfpathlineto{\pgfqpoint{0.318027in}{0.959794in}}%
\pgfpathclose%
\pgfusepath{fill}%
\end{pgfscope}%
\begin{pgfscope}%
\pgfpathrectangle{\pgfqpoint{0.211875in}{0.211875in}}{\pgfqpoint{1.313625in}{1.279725in}}%
\pgfusepath{clip}%
\pgfsetbuttcap%
\pgfsetroundjoin%
\definecolor{currentfill}{rgb}{0.961115,0.566634,0.405693}%
\pgfsetfillcolor{currentfill}%
\pgfsetlinewidth{0.000000pt}%
\definecolor{currentstroke}{rgb}{0.000000,0.000000,0.000000}%
\pgfsetstrokecolor{currentstroke}%
\pgfsetdash{}{0pt}%
\pgfpathmoveto{\pgfqpoint{0.437447in}{0.936837in}}%
\pgfpathlineto{\pgfqpoint{0.450716in}{0.937312in}}%
\pgfpathlineto{\pgfqpoint{0.460566in}{0.948686in}}%
\pgfpathlineto{\pgfqpoint{0.463084in}{0.961613in}}%
\pgfpathlineto{\pgfqpoint{0.461956in}{0.974539in}}%
\pgfpathlineto{\pgfqpoint{0.454767in}{0.987466in}}%
\pgfpathlineto{\pgfqpoint{0.450716in}{0.990135in}}%
\pgfpathlineto{\pgfqpoint{0.437447in}{0.990497in}}%
\pgfpathlineto{\pgfqpoint{0.432227in}{0.987466in}}%
\pgfpathlineto{\pgfqpoint{0.424178in}{0.975307in}}%
\pgfpathlineto{\pgfqpoint{0.423967in}{0.974539in}}%
\pgfpathlineto{\pgfqpoint{0.423120in}{0.961613in}}%
\pgfpathlineto{\pgfqpoint{0.424178in}{0.953750in}}%
\pgfpathlineto{\pgfqpoint{0.425483in}{0.948686in}}%
\pgfpathclose%
\pgfpathmoveto{\pgfqpoint{0.436597in}{0.948686in}}%
\pgfpathlineto{\pgfqpoint{0.431877in}{0.961613in}}%
\pgfpathlineto{\pgfqpoint{0.434119in}{0.974539in}}%
\pgfpathlineto{\pgfqpoint{0.437447in}{0.978797in}}%
\pgfpathlineto{\pgfqpoint{0.450716in}{0.977526in}}%
\pgfpathlineto{\pgfqpoint{0.452705in}{0.974539in}}%
\pgfpathlineto{\pgfqpoint{0.454660in}{0.961613in}}%
\pgfpathlineto{\pgfqpoint{0.450716in}{0.949222in}}%
\pgfpathlineto{\pgfqpoint{0.447939in}{0.948686in}}%
\pgfpathlineto{\pgfqpoint{0.437447in}{0.947844in}}%
\pgfpathclose%
\pgfusepath{fill}%
\end{pgfscope}%
\begin{pgfscope}%
\pgfpathrectangle{\pgfqpoint{0.211875in}{0.211875in}}{\pgfqpoint{1.313625in}{1.279725in}}%
\pgfusepath{clip}%
\pgfsetbuttcap%
\pgfsetroundjoin%
\definecolor{currentfill}{rgb}{0.961115,0.566634,0.405693}%
\pgfsetfillcolor{currentfill}%
\pgfsetlinewidth{0.000000pt}%
\definecolor{currentstroke}{rgb}{0.000000,0.000000,0.000000}%
\pgfsetstrokecolor{currentstroke}%
\pgfsetdash{}{0pt}%
\pgfpathmoveto{\pgfqpoint{0.623212in}{1.012389in}}%
\pgfpathlineto{\pgfqpoint{0.627221in}{1.013319in}}%
\pgfpathlineto{\pgfqpoint{0.636481in}{1.017411in}}%
\pgfpathlineto{\pgfqpoint{0.641223in}{1.026245in}}%
\pgfpathlineto{\pgfqpoint{0.643057in}{1.039172in}}%
\pgfpathlineto{\pgfqpoint{0.642982in}{1.052098in}}%
\pgfpathlineto{\pgfqpoint{0.640829in}{1.065025in}}%
\pgfpathlineto{\pgfqpoint{0.636481in}{1.072368in}}%
\pgfpathlineto{\pgfqpoint{0.623212in}{1.077276in}}%
\pgfpathlineto{\pgfqpoint{0.609943in}{1.075748in}}%
\pgfpathlineto{\pgfqpoint{0.599352in}{1.065025in}}%
\pgfpathlineto{\pgfqpoint{0.596674in}{1.053267in}}%
\pgfpathlineto{\pgfqpoint{0.596523in}{1.052098in}}%
\pgfpathlineto{\pgfqpoint{0.596442in}{1.039172in}}%
\pgfpathlineto{\pgfqpoint{0.596674in}{1.037180in}}%
\pgfpathlineto{\pgfqpoint{0.598815in}{1.026245in}}%
\pgfpathlineto{\pgfqpoint{0.609943in}{1.013810in}}%
\pgfpathlineto{\pgfqpoint{0.613652in}{1.013319in}}%
\pgfpathclose%
\pgfpathmoveto{\pgfqpoint{0.608552in}{1.026245in}}%
\pgfpathlineto{\pgfqpoint{0.603938in}{1.039172in}}%
\pgfpathlineto{\pgfqpoint{0.604176in}{1.052098in}}%
\pgfpathlineto{\pgfqpoint{0.609721in}{1.065025in}}%
\pgfpathlineto{\pgfqpoint{0.609943in}{1.065250in}}%
\pgfpathlineto{\pgfqpoint{0.623212in}{1.067961in}}%
\pgfpathlineto{\pgfqpoint{0.628332in}{1.065025in}}%
\pgfpathlineto{\pgfqpoint{0.636481in}{1.052494in}}%
\pgfpathlineto{\pgfqpoint{0.636582in}{1.052098in}}%
\pgfpathlineto{\pgfqpoint{0.636771in}{1.039172in}}%
\pgfpathlineto{\pgfqpoint{0.636481in}{1.037893in}}%
\pgfpathlineto{\pgfqpoint{0.630108in}{1.026245in}}%
\pgfpathlineto{\pgfqpoint{0.623212in}{1.021858in}}%
\pgfpathlineto{\pgfqpoint{0.609943in}{1.024691in}}%
\pgfpathclose%
\pgfusepath{fill}%
\end{pgfscope}%
\begin{pgfscope}%
\pgfpathrectangle{\pgfqpoint{0.211875in}{0.211875in}}{\pgfqpoint{1.313625in}{1.279725in}}%
\pgfusepath{clip}%
\pgfsetbuttcap%
\pgfsetroundjoin%
\definecolor{currentfill}{rgb}{0.961115,0.566634,0.405693}%
\pgfsetfillcolor{currentfill}%
\pgfsetlinewidth{0.000000pt}%
\definecolor{currentstroke}{rgb}{0.000000,0.000000,0.000000}%
\pgfsetstrokecolor{currentstroke}%
\pgfsetdash{}{0pt}%
\pgfpathmoveto{\pgfqpoint{0.729364in}{1.010995in}}%
\pgfpathlineto{\pgfqpoint{0.742633in}{1.010395in}}%
\pgfpathlineto{\pgfqpoint{0.753297in}{1.013319in}}%
\pgfpathlineto{\pgfqpoint{0.755902in}{1.014807in}}%
\pgfpathlineto{\pgfqpoint{0.761253in}{1.026245in}}%
\pgfpathlineto{\pgfqpoint{0.762580in}{1.039172in}}%
\pgfpathlineto{\pgfqpoint{0.762535in}{1.052098in}}%
\pgfpathlineto{\pgfqpoint{0.761006in}{1.065025in}}%
\pgfpathlineto{\pgfqpoint{0.755902in}{1.074930in}}%
\pgfpathlineto{\pgfqpoint{0.749091in}{1.077952in}}%
\pgfpathlineto{\pgfqpoint{0.742633in}{1.079371in}}%
\pgfpathlineto{\pgfqpoint{0.729364in}{1.078784in}}%
\pgfpathlineto{\pgfqpoint{0.726750in}{1.077952in}}%
\pgfpathlineto{\pgfqpoint{0.716095in}{1.069337in}}%
\pgfpathlineto{\pgfqpoint{0.714522in}{1.065025in}}%
\pgfpathlineto{\pgfqpoint{0.712875in}{1.052098in}}%
\pgfpathlineto{\pgfqpoint{0.712822in}{1.039172in}}%
\pgfpathlineto{\pgfqpoint{0.714242in}{1.026245in}}%
\pgfpathlineto{\pgfqpoint{0.716095in}{1.020687in}}%
\pgfpathlineto{\pgfqpoint{0.723371in}{1.013319in}}%
\pgfpathclose%
\pgfpathmoveto{\pgfqpoint{0.724029in}{1.026245in}}%
\pgfpathlineto{\pgfqpoint{0.719681in}{1.039172in}}%
\pgfpathlineto{\pgfqpoint{0.719899in}{1.052098in}}%
\pgfpathlineto{\pgfqpoint{0.725108in}{1.065025in}}%
\pgfpathlineto{\pgfqpoint{0.729364in}{1.068848in}}%
\pgfpathlineto{\pgfqpoint{0.742633in}{1.070108in}}%
\pgfpathlineto{\pgfqpoint{0.750239in}{1.065025in}}%
\pgfpathlineto{\pgfqpoint{0.755902in}{1.053899in}}%
\pgfpathlineto{\pgfqpoint{0.756303in}{1.052098in}}%
\pgfpathlineto{\pgfqpoint{0.756459in}{1.039172in}}%
\pgfpathlineto{\pgfqpoint{0.755902in}{1.036354in}}%
\pgfpathlineto{\pgfqpoint{0.751558in}{1.026245in}}%
\pgfpathlineto{\pgfqpoint{0.742633in}{1.019634in}}%
\pgfpathlineto{\pgfqpoint{0.729364in}{1.020953in}}%
\pgfpathclose%
\pgfusepath{fill}%
\end{pgfscope}%
\begin{pgfscope}%
\pgfpathrectangle{\pgfqpoint{0.211875in}{0.211875in}}{\pgfqpoint{1.313625in}{1.279725in}}%
\pgfusepath{clip}%
\pgfsetbuttcap%
\pgfsetroundjoin%
\definecolor{currentfill}{rgb}{0.961115,0.566634,0.405693}%
\pgfsetfillcolor{currentfill}%
\pgfsetlinewidth{0.000000pt}%
\definecolor{currentstroke}{rgb}{0.000000,0.000000,0.000000}%
\pgfsetstrokecolor{currentstroke}%
\pgfsetdash{}{0pt}%
\pgfpathmoveto{\pgfqpoint{0.848784in}{1.009067in}}%
\pgfpathlineto{\pgfqpoint{0.862053in}{1.008916in}}%
\pgfpathlineto{\pgfqpoint{0.875322in}{1.012969in}}%
\pgfpathlineto{\pgfqpoint{0.875696in}{1.013319in}}%
\pgfpathlineto{\pgfqpoint{0.880693in}{1.026245in}}%
\pgfpathlineto{\pgfqpoint{0.881665in}{1.039172in}}%
\pgfpathlineto{\pgfqpoint{0.881641in}{1.052098in}}%
\pgfpathlineto{\pgfqpoint{0.880547in}{1.065025in}}%
\pgfpathlineto{\pgfqpoint{0.875322in}{1.076843in}}%
\pgfpathlineto{\pgfqpoint{0.873396in}{1.077952in}}%
\pgfpathlineto{\pgfqpoint{0.862053in}{1.080860in}}%
\pgfpathlineto{\pgfqpoint{0.848784in}{1.080708in}}%
\pgfpathlineto{\pgfqpoint{0.838906in}{1.077952in}}%
\pgfpathlineto{\pgfqpoint{0.835515in}{1.075834in}}%
\pgfpathlineto{\pgfqpoint{0.830828in}{1.065025in}}%
\pgfpathlineto{\pgfqpoint{0.829604in}{1.052098in}}%
\pgfpathlineto{\pgfqpoint{0.829573in}{1.039172in}}%
\pgfpathlineto{\pgfqpoint{0.830650in}{1.026245in}}%
\pgfpathlineto{\pgfqpoint{0.835515in}{1.013933in}}%
\pgfpathlineto{\pgfqpoint{0.836297in}{1.013319in}}%
\pgfpathclose%
\pgfpathmoveto{\pgfqpoint{0.839689in}{1.026245in}}%
\pgfpathlineto{\pgfqpoint{0.835515in}{1.038950in}}%
\pgfpathlineto{\pgfqpoint{0.835478in}{1.039172in}}%
\pgfpathlineto{\pgfqpoint{0.835515in}{1.042769in}}%
\pgfpathlineto{\pgfqpoint{0.835665in}{1.052098in}}%
\pgfpathlineto{\pgfqpoint{0.840716in}{1.065025in}}%
\pgfpathlineto{\pgfqpoint{0.848784in}{1.071418in}}%
\pgfpathlineto{\pgfqpoint{0.862053in}{1.071588in}}%
\pgfpathlineto{\pgfqpoint{0.870606in}{1.065025in}}%
\pgfpathlineto{\pgfqpoint{0.875322in}{1.053187in}}%
\pgfpathlineto{\pgfqpoint{0.875531in}{1.052098in}}%
\pgfpathlineto{\pgfqpoint{0.875663in}{1.039172in}}%
\pgfpathlineto{\pgfqpoint{0.875322in}{1.037171in}}%
\pgfpathlineto{\pgfqpoint{0.871633in}{1.026245in}}%
\pgfpathlineto{\pgfqpoint{0.862053in}{1.018105in}}%
\pgfpathlineto{\pgfqpoint{0.848784in}{1.018282in}}%
\pgfpathclose%
\pgfusepath{fill}%
\end{pgfscope}%
\begin{pgfscope}%
\pgfpathrectangle{\pgfqpoint{0.211875in}{0.211875in}}{\pgfqpoint{1.313625in}{1.279725in}}%
\pgfusepath{clip}%
\pgfsetbuttcap%
\pgfsetroundjoin%
\definecolor{currentfill}{rgb}{0.961115,0.566634,0.405693}%
\pgfsetfillcolor{currentfill}%
\pgfsetlinewidth{0.000000pt}%
\definecolor{currentstroke}{rgb}{0.000000,0.000000,0.000000}%
\pgfsetstrokecolor{currentstroke}%
\pgfsetdash{}{0pt}%
\pgfpathmoveto{\pgfqpoint{0.954936in}{1.010632in}}%
\pgfpathlineto{\pgfqpoint{0.968205in}{1.007830in}}%
\pgfpathlineto{\pgfqpoint{0.981473in}{1.007960in}}%
\pgfpathlineto{\pgfqpoint{0.994742in}{1.012190in}}%
\pgfpathlineto{\pgfqpoint{0.995752in}{1.013319in}}%
\pgfpathlineto{\pgfqpoint{0.999588in}{1.026245in}}%
\pgfpathlineto{\pgfqpoint{1.000341in}{1.039172in}}%
\pgfpathlineto{\pgfqpoint{1.000330in}{1.052098in}}%
\pgfpathlineto{\pgfqpoint{0.999504in}{1.065025in}}%
\pgfpathlineto{\pgfqpoint{0.994742in}{1.077844in}}%
\pgfpathlineto{\pgfqpoint{0.994599in}{1.077952in}}%
\pgfpathlineto{\pgfqpoint{0.981473in}{1.081825in}}%
\pgfpathlineto{\pgfqpoint{0.968205in}{1.081941in}}%
\pgfpathlineto{\pgfqpoint{0.954936in}{1.079299in}}%
\pgfpathlineto{\pgfqpoint{0.952973in}{1.077952in}}%
\pgfpathlineto{\pgfqpoint{0.947616in}{1.065025in}}%
\pgfpathlineto{\pgfqpoint{0.946683in}{1.052098in}}%
\pgfpathlineto{\pgfqpoint{0.946666in}{1.039172in}}%
\pgfpathlineto{\pgfqpoint{0.947509in}{1.026245in}}%
\pgfpathlineto{\pgfqpoint{0.951729in}{1.013319in}}%
\pgfpathclose%
\pgfpathmoveto{\pgfqpoint{0.955517in}{1.026245in}}%
\pgfpathlineto{\pgfqpoint{0.954936in}{1.027641in}}%
\pgfpathlineto{\pgfqpoint{0.952687in}{1.039172in}}%
\pgfpathlineto{\pgfqpoint{0.952810in}{1.052098in}}%
\pgfpathlineto{\pgfqpoint{0.954936in}{1.061757in}}%
\pgfpathlineto{\pgfqpoint{0.956543in}{1.065025in}}%
\pgfpathlineto{\pgfqpoint{0.968205in}{1.073103in}}%
\pgfpathlineto{\pgfqpoint{0.981473in}{1.072378in}}%
\pgfpathlineto{\pgfqpoint{0.989903in}{1.065025in}}%
\pgfpathlineto{\pgfqpoint{0.994115in}{1.052098in}}%
\pgfpathlineto{\pgfqpoint{0.994281in}{1.039172in}}%
\pgfpathlineto{\pgfqpoint{0.990747in}{1.026245in}}%
\pgfpathlineto{\pgfqpoint{0.981473in}{1.017293in}}%
\pgfpathlineto{\pgfqpoint{0.968205in}{1.016528in}}%
\pgfpathclose%
\pgfusepath{fill}%
\end{pgfscope}%
\begin{pgfscope}%
\pgfpathrectangle{\pgfqpoint{0.211875in}{0.211875in}}{\pgfqpoint{1.313625in}{1.279725in}}%
\pgfusepath{clip}%
\pgfsetbuttcap%
\pgfsetroundjoin%
\definecolor{currentfill}{rgb}{0.961115,0.566634,0.405693}%
\pgfsetfillcolor{currentfill}%
\pgfsetlinewidth{0.000000pt}%
\definecolor{currentstroke}{rgb}{0.000000,0.000000,0.000000}%
\pgfsetstrokecolor{currentstroke}%
\pgfsetdash{}{0pt}%
\pgfpathmoveto{\pgfqpoint{1.074356in}{1.009001in}}%
\pgfpathlineto{\pgfqpoint{1.087625in}{1.007209in}}%
\pgfpathlineto{\pgfqpoint{1.100894in}{1.007565in}}%
\pgfpathlineto{\pgfqpoint{1.114163in}{1.012681in}}%
\pgfpathlineto{\pgfqpoint{1.114627in}{1.013319in}}%
\pgfpathlineto{\pgfqpoint{1.117958in}{1.026245in}}%
\pgfpathlineto{\pgfqpoint{1.118618in}{1.039172in}}%
\pgfpathlineto{\pgfqpoint{1.118612in}{1.052098in}}%
\pgfpathlineto{\pgfqpoint{1.117901in}{1.065025in}}%
\pgfpathlineto{\pgfqpoint{1.114163in}{1.077391in}}%
\pgfpathlineto{\pgfqpoint{1.113604in}{1.077952in}}%
\pgfpathlineto{\pgfqpoint{1.100894in}{1.082233in}}%
\pgfpathlineto{\pgfqpoint{1.087625in}{1.082558in}}%
\pgfpathlineto{\pgfqpoint{1.074356in}{1.080898in}}%
\pgfpathlineto{\pgfqpoint{1.069397in}{1.077952in}}%
\pgfpathlineto{\pgfqpoint{1.064873in}{1.065025in}}%
\pgfpathlineto{\pgfqpoint{1.064094in}{1.052098in}}%
\pgfpathlineto{\pgfqpoint{1.064086in}{1.039172in}}%
\pgfpathlineto{\pgfqpoint{1.064806in}{1.026245in}}%
\pgfpathlineto{\pgfqpoint{1.068420in}{1.013319in}}%
\pgfpathclose%
\pgfpathmoveto{\pgfqpoint{1.072800in}{1.026245in}}%
\pgfpathlineto{\pgfqpoint{1.070267in}{1.039172in}}%
\pgfpathlineto{\pgfqpoint{1.070385in}{1.052098in}}%
\pgfpathlineto{\pgfqpoint{1.073398in}{1.065025in}}%
\pgfpathlineto{\pgfqpoint{1.074356in}{1.066677in}}%
\pgfpathlineto{\pgfqpoint{1.087625in}{1.074000in}}%
\pgfpathlineto{\pgfqpoint{1.100894in}{1.072426in}}%
\pgfpathlineto{\pgfqpoint{1.108402in}{1.065025in}}%
\pgfpathlineto{\pgfqpoint{1.112137in}{1.052098in}}%
\pgfpathlineto{\pgfqpoint{1.112283in}{1.039172in}}%
\pgfpathlineto{\pgfqpoint{1.109144in}{1.026245in}}%
\pgfpathlineto{\pgfqpoint{1.100894in}{1.017254in}}%
\pgfpathlineto{\pgfqpoint{1.087625in}{1.015593in}}%
\pgfpathlineto{\pgfqpoint{1.074356in}{1.023288in}}%
\pgfpathclose%
\pgfusepath{fill}%
\end{pgfscope}%
\begin{pgfscope}%
\pgfpathrectangle{\pgfqpoint{0.211875in}{0.211875in}}{\pgfqpoint{1.313625in}{1.279725in}}%
\pgfusepath{clip}%
\pgfsetbuttcap%
\pgfsetroundjoin%
\definecolor{currentfill}{rgb}{0.961115,0.566634,0.405693}%
\pgfsetfillcolor{currentfill}%
\pgfsetlinewidth{0.000000pt}%
\definecolor{currentstroke}{rgb}{0.000000,0.000000,0.000000}%
\pgfsetstrokecolor{currentstroke}%
\pgfsetdash{}{0pt}%
\pgfpathmoveto{\pgfqpoint{1.193777in}{1.008470in}}%
\pgfpathlineto{\pgfqpoint{1.207045in}{1.007160in}}%
\pgfpathlineto{\pgfqpoint{1.220314in}{1.007793in}}%
\pgfpathlineto{\pgfqpoint{1.232085in}{1.013319in}}%
\pgfpathlineto{\pgfqpoint{1.233583in}{1.015944in}}%
\pgfpathlineto{\pgfqpoint{1.235805in}{1.026245in}}%
\pgfpathlineto{\pgfqpoint{1.236491in}{1.039172in}}%
\pgfpathlineto{\pgfqpoint{1.236483in}{1.052098in}}%
\pgfpathlineto{\pgfqpoint{1.235740in}{1.065025in}}%
\pgfpathlineto{\pgfqpoint{1.233583in}{1.074254in}}%
\pgfpathlineto{\pgfqpoint{1.230944in}{1.077952in}}%
\pgfpathlineto{\pgfqpoint{1.220314in}{1.082018in}}%
\pgfpathlineto{\pgfqpoint{1.207045in}{1.082604in}}%
\pgfpathlineto{\pgfqpoint{1.193777in}{1.081398in}}%
\pgfpathlineto{\pgfqpoint{1.187130in}{1.077952in}}%
\pgfpathlineto{\pgfqpoint{1.182607in}{1.065025in}}%
\pgfpathlineto{\pgfqpoint{1.181835in}{1.052098in}}%
\pgfpathlineto{\pgfqpoint{1.181829in}{1.039172in}}%
\pgfpathlineto{\pgfqpoint{1.182545in}{1.026245in}}%
\pgfpathlineto{\pgfqpoint{1.186164in}{1.013319in}}%
\pgfpathclose%
\pgfpathmoveto{\pgfqpoint{1.190822in}{1.026245in}}%
\pgfpathlineto{\pgfqpoint{1.188222in}{1.039172in}}%
\pgfpathlineto{\pgfqpoint{1.188343in}{1.052098in}}%
\pgfpathlineto{\pgfqpoint{1.191436in}{1.065025in}}%
\pgfpathlineto{\pgfqpoint{1.193777in}{1.068574in}}%
\pgfpathlineto{\pgfqpoint{1.207045in}{1.074168in}}%
\pgfpathlineto{\pgfqpoint{1.220314in}{1.071635in}}%
\pgfpathlineto{\pgfqpoint{1.226271in}{1.065025in}}%
\pgfpathlineto{\pgfqpoint{1.229789in}{1.052098in}}%
\pgfpathlineto{\pgfqpoint{1.229927in}{1.039172in}}%
\pgfpathlineto{\pgfqpoint{1.226970in}{1.026245in}}%
\pgfpathlineto{\pgfqpoint{1.220314in}{1.018085in}}%
\pgfpathlineto{\pgfqpoint{1.207045in}{1.015416in}}%
\pgfpathlineto{\pgfqpoint{1.193777in}{1.021303in}}%
\pgfpathclose%
\pgfusepath{fill}%
\end{pgfscope}%
\begin{pgfscope}%
\pgfpathrectangle{\pgfqpoint{0.211875in}{0.211875in}}{\pgfqpoint{1.313625in}{1.279725in}}%
\pgfusepath{clip}%
\pgfsetbuttcap%
\pgfsetroundjoin%
\definecolor{currentfill}{rgb}{0.961115,0.566634,0.405693}%
\pgfsetfillcolor{currentfill}%
\pgfsetlinewidth{0.000000pt}%
\definecolor{currentstroke}{rgb}{0.000000,0.000000,0.000000}%
\pgfsetstrokecolor{currentstroke}%
\pgfsetdash{}{0pt}%
\pgfpathmoveto{\pgfqpoint{1.313197in}{1.008762in}}%
\pgfpathlineto{\pgfqpoint{1.326466in}{1.007660in}}%
\pgfpathlineto{\pgfqpoint{1.339735in}{1.008750in}}%
\pgfpathlineto{\pgfqpoint{1.348345in}{1.013319in}}%
\pgfpathlineto{\pgfqpoint{1.353004in}{1.025571in}}%
\pgfpathlineto{\pgfqpoint{1.353108in}{1.026245in}}%
\pgfpathlineto{\pgfqpoint{1.353938in}{1.039172in}}%
\pgfpathlineto{\pgfqpoint{1.353922in}{1.052098in}}%
\pgfpathlineto{\pgfqpoint{1.353004in}{1.065005in}}%
\pgfpathlineto{\pgfqpoint{1.353002in}{1.065025in}}%
\pgfpathlineto{\pgfqpoint{1.346938in}{1.077952in}}%
\pgfpathlineto{\pgfqpoint{1.339735in}{1.081075in}}%
\pgfpathlineto{\pgfqpoint{1.326466in}{1.082102in}}%
\pgfpathlineto{\pgfqpoint{1.313197in}{1.081079in}}%
\pgfpathlineto{\pgfqpoint{1.306325in}{1.077952in}}%
\pgfpathlineto{\pgfqpoint{1.300846in}{1.065025in}}%
\pgfpathlineto{\pgfqpoint{1.299928in}{1.052285in}}%
\pgfpathlineto{\pgfqpoint{1.299919in}{1.052098in}}%
\pgfpathlineto{\pgfqpoint{1.299907in}{1.039172in}}%
\pgfpathlineto{\pgfqpoint{1.299928in}{1.038723in}}%
\pgfpathlineto{\pgfqpoint{1.300752in}{1.026245in}}%
\pgfpathlineto{\pgfqpoint{1.305071in}{1.013319in}}%
\pgfpathclose%
\pgfpathmoveto{\pgfqpoint{1.309390in}{1.026245in}}%
\pgfpathlineto{\pgfqpoint{1.306570in}{1.039172in}}%
\pgfpathlineto{\pgfqpoint{1.306703in}{1.052098in}}%
\pgfpathlineto{\pgfqpoint{1.310064in}{1.065025in}}%
\pgfpathlineto{\pgfqpoint{1.313197in}{1.069227in}}%
\pgfpathlineto{\pgfqpoint{1.326466in}{1.073637in}}%
\pgfpathlineto{\pgfqpoint{1.339735in}{1.069859in}}%
\pgfpathlineto{\pgfqpoint{1.343607in}{1.065025in}}%
\pgfpathlineto{\pgfqpoint{1.347117in}{1.052098in}}%
\pgfpathlineto{\pgfqpoint{1.347256in}{1.039172in}}%
\pgfpathlineto{\pgfqpoint{1.344311in}{1.026245in}}%
\pgfpathlineto{\pgfqpoint{1.339735in}{1.019939in}}%
\pgfpathlineto{\pgfqpoint{1.326466in}{1.015966in}}%
\pgfpathlineto{\pgfqpoint{1.313197in}{1.020607in}}%
\pgfpathclose%
\pgfusepath{fill}%
\end{pgfscope}%
\begin{pgfscope}%
\pgfpathrectangle{\pgfqpoint{0.211875in}{0.211875in}}{\pgfqpoint{1.313625in}{1.279725in}}%
\pgfusepath{clip}%
\pgfsetbuttcap%
\pgfsetroundjoin%
\definecolor{currentfill}{rgb}{0.961115,0.566634,0.405693}%
\pgfsetfillcolor{currentfill}%
\pgfsetlinewidth{0.000000pt}%
\definecolor{currentstroke}{rgb}{0.000000,0.000000,0.000000}%
\pgfsetstrokecolor{currentstroke}%
\pgfsetdash{}{0pt}%
\pgfpathmoveto{\pgfqpoint{1.432617in}{1.009711in}}%
\pgfpathlineto{\pgfqpoint{1.445886in}{1.008710in}}%
\pgfpathlineto{\pgfqpoint{1.459155in}{1.010598in}}%
\pgfpathlineto{\pgfqpoint{1.463689in}{1.013319in}}%
\pgfpathlineto{\pgfqpoint{1.469612in}{1.026245in}}%
\pgfpathlineto{\pgfqpoint{1.470798in}{1.039172in}}%
\pgfpathlineto{\pgfqpoint{1.470763in}{1.052098in}}%
\pgfpathlineto{\pgfqpoint{1.469415in}{1.065025in}}%
\pgfpathlineto{\pgfqpoint{1.461779in}{1.077952in}}%
\pgfpathlineto{\pgfqpoint{1.459155in}{1.079242in}}%
\pgfpathlineto{\pgfqpoint{1.445886in}{1.081051in}}%
\pgfpathlineto{\pgfqpoint{1.432617in}{1.080108in}}%
\pgfpathlineto{\pgfqpoint{1.427231in}{1.077952in}}%
\pgfpathlineto{\pgfqpoint{1.419648in}{1.065025in}}%
\pgfpathlineto{\pgfqpoint{1.419348in}{1.062644in}}%
\pgfpathlineto{\pgfqpoint{1.418514in}{1.052098in}}%
\pgfpathlineto{\pgfqpoint{1.418489in}{1.039172in}}%
\pgfpathlineto{\pgfqpoint{1.419348in}{1.027413in}}%
\pgfpathlineto{\pgfqpoint{1.419478in}{1.026245in}}%
\pgfpathlineto{\pgfqpoint{1.425326in}{1.013319in}}%
\pgfpathclose%
\pgfpathmoveto{\pgfqpoint{1.428572in}{1.026245in}}%
\pgfpathlineto{\pgfqpoint{1.425352in}{1.039172in}}%
\pgfpathlineto{\pgfqpoint{1.425507in}{1.052098in}}%
\pgfpathlineto{\pgfqpoint{1.429357in}{1.065025in}}%
\pgfpathlineto{\pgfqpoint{1.432617in}{1.068906in}}%
\pgfpathlineto{\pgfqpoint{1.445886in}{1.072406in}}%
\pgfpathlineto{\pgfqpoint{1.459155in}{1.066871in}}%
\pgfpathlineto{\pgfqpoint{1.460468in}{1.065025in}}%
\pgfpathlineto{\pgfqpoint{1.464143in}{1.052098in}}%
\pgfpathlineto{\pgfqpoint{1.464293in}{1.039172in}}%
\pgfpathlineto{\pgfqpoint{1.461217in}{1.026245in}}%
\pgfpathlineto{\pgfqpoint{1.459155in}{1.023049in}}%
\pgfpathlineto{\pgfqpoint{1.445886in}{1.017243in}}%
\pgfpathlineto{\pgfqpoint{1.432617in}{1.020923in}}%
\pgfpathclose%
\pgfusepath{fill}%
\end{pgfscope}%
\begin{pgfscope}%
\pgfpathrectangle{\pgfqpoint{0.211875in}{0.211875in}}{\pgfqpoint{1.313625in}{1.279725in}}%
\pgfusepath{clip}%
\pgfsetbuttcap%
\pgfsetroundjoin%
\definecolor{currentfill}{rgb}{0.961115,0.566634,0.405693}%
\pgfsetfillcolor{currentfill}%
\pgfsetlinewidth{0.000000pt}%
\definecolor{currentstroke}{rgb}{0.000000,0.000000,0.000000}%
\pgfsetstrokecolor{currentstroke}%
\pgfsetdash{}{0pt}%
\pgfpathmoveto{\pgfqpoint{0.264951in}{1.023388in}}%
\pgfpathlineto{\pgfqpoint{0.273460in}{1.026245in}}%
\pgfpathlineto{\pgfqpoint{0.278220in}{1.029927in}}%
\pgfpathlineto{\pgfqpoint{0.281238in}{1.039172in}}%
\pgfpathlineto{\pgfqpoint{0.281003in}{1.052098in}}%
\pgfpathlineto{\pgfqpoint{0.278220in}{1.059597in}}%
\pgfpathlineto{\pgfqpoint{0.269796in}{1.065025in}}%
\pgfpathlineto{\pgfqpoint{0.264951in}{1.066483in}}%
\pgfpathlineto{\pgfqpoint{0.261904in}{1.065025in}}%
\pgfpathlineto{\pgfqpoint{0.251682in}{1.053649in}}%
\pgfpathlineto{\pgfqpoint{0.251208in}{1.052098in}}%
\pgfpathlineto{\pgfqpoint{0.250983in}{1.039172in}}%
\pgfpathlineto{\pgfqpoint{0.251682in}{1.036580in}}%
\pgfpathlineto{\pgfqpoint{0.259531in}{1.026245in}}%
\pgfpathclose%
\pgfpathmoveto{\pgfqpoint{0.264297in}{1.039172in}}%
\pgfpathlineto{\pgfqpoint{0.264951in}{1.050688in}}%
\pgfpathlineto{\pgfqpoint{0.265933in}{1.039172in}}%
\pgfpathlineto{\pgfqpoint{0.264951in}{1.038371in}}%
\pgfpathclose%
\pgfusepath{fill}%
\end{pgfscope}%
\begin{pgfscope}%
\pgfpathrectangle{\pgfqpoint{0.211875in}{0.211875in}}{\pgfqpoint{1.313625in}{1.279725in}}%
\pgfusepath{clip}%
\pgfsetbuttcap%
\pgfsetroundjoin%
\definecolor{currentfill}{rgb}{0.961115,0.566634,0.405693}%
\pgfsetfillcolor{currentfill}%
\pgfsetlinewidth{0.000000pt}%
\definecolor{currentstroke}{rgb}{0.000000,0.000000,0.000000}%
\pgfsetstrokecolor{currentstroke}%
\pgfsetdash{}{0pt}%
\pgfpathmoveto{\pgfqpoint{0.371102in}{1.024173in}}%
\pgfpathlineto{\pgfqpoint{0.384371in}{1.018948in}}%
\pgfpathlineto{\pgfqpoint{0.397640in}{1.024189in}}%
\pgfpathlineto{\pgfqpoint{0.399059in}{1.026245in}}%
\pgfpathlineto{\pgfqpoint{0.402479in}{1.039172in}}%
\pgfpathlineto{\pgfqpoint{0.402312in}{1.052098in}}%
\pgfpathlineto{\pgfqpoint{0.398206in}{1.065025in}}%
\pgfpathlineto{\pgfqpoint{0.397640in}{1.065766in}}%
\pgfpathlineto{\pgfqpoint{0.384371in}{1.070762in}}%
\pgfpathlineto{\pgfqpoint{0.371102in}{1.065770in}}%
\pgfpathlineto{\pgfqpoint{0.370519in}{1.065025in}}%
\pgfpathlineto{\pgfqpoint{0.366223in}{1.052098in}}%
\pgfpathlineto{\pgfqpoint{0.366043in}{1.039172in}}%
\pgfpathlineto{\pgfqpoint{0.369631in}{1.026245in}}%
\pgfpathclose%
\pgfpathmoveto{\pgfqpoint{0.375382in}{1.039172in}}%
\pgfpathlineto{\pgfqpoint{0.376152in}{1.052098in}}%
\pgfpathlineto{\pgfqpoint{0.384371in}{1.059260in}}%
\pgfpathlineto{\pgfqpoint{0.392178in}{1.052098in}}%
\pgfpathlineto{\pgfqpoint{0.392892in}{1.039172in}}%
\pgfpathlineto{\pgfqpoint{0.384371in}{1.030250in}}%
\pgfpathclose%
\pgfusepath{fill}%
\end{pgfscope}%
\begin{pgfscope}%
\pgfpathrectangle{\pgfqpoint{0.211875in}{0.211875in}}{\pgfqpoint{1.313625in}{1.279725in}}%
\pgfusepath{clip}%
\pgfsetbuttcap%
\pgfsetroundjoin%
\definecolor{currentfill}{rgb}{0.961115,0.566634,0.405693}%
\pgfsetfillcolor{currentfill}%
\pgfsetlinewidth{0.000000pt}%
\definecolor{currentstroke}{rgb}{0.000000,0.000000,0.000000}%
\pgfsetstrokecolor{currentstroke}%
\pgfsetdash{}{0pt}%
\pgfpathmoveto{\pgfqpoint{0.490523in}{1.018272in}}%
\pgfpathlineto{\pgfqpoint{0.503792in}{1.015237in}}%
\pgfpathlineto{\pgfqpoint{0.517061in}{1.020562in}}%
\pgfpathlineto{\pgfqpoint{0.520529in}{1.026245in}}%
\pgfpathlineto{\pgfqpoint{0.523048in}{1.039172in}}%
\pgfpathlineto{\pgfqpoint{0.522934in}{1.052098in}}%
\pgfpathlineto{\pgfqpoint{0.519937in}{1.065025in}}%
\pgfpathlineto{\pgfqpoint{0.517061in}{1.069291in}}%
\pgfpathlineto{\pgfqpoint{0.503792in}{1.074340in}}%
\pgfpathlineto{\pgfqpoint{0.490523in}{1.071454in}}%
\pgfpathlineto{\pgfqpoint{0.484872in}{1.065025in}}%
\pgfpathlineto{\pgfqpoint{0.481334in}{1.052098in}}%
\pgfpathlineto{\pgfqpoint{0.481193in}{1.039172in}}%
\pgfpathlineto{\pgfqpoint{0.484170in}{1.026245in}}%
\pgfpathclose%
\pgfpathmoveto{\pgfqpoint{0.499329in}{1.026245in}}%
\pgfpathlineto{\pgfqpoint{0.490523in}{1.032561in}}%
\pgfpathlineto{\pgfqpoint{0.488278in}{1.039172in}}%
\pgfpathlineto{\pgfqpoint{0.488543in}{1.052098in}}%
\pgfpathlineto{\pgfqpoint{0.490523in}{1.057228in}}%
\pgfpathlineto{\pgfqpoint{0.503402in}{1.065025in}}%
\pgfpathlineto{\pgfqpoint{0.503792in}{1.065141in}}%
\pgfpathlineto{\pgfqpoint{0.504033in}{1.065025in}}%
\pgfpathlineto{\pgfqpoint{0.515490in}{1.052098in}}%
\pgfpathlineto{\pgfqpoint{0.515979in}{1.039172in}}%
\pgfpathlineto{\pgfqpoint{0.506528in}{1.026245in}}%
\pgfpathlineto{\pgfqpoint{0.503792in}{1.024782in}}%
\pgfpathclose%
\pgfusepath{fill}%
\end{pgfscope}%
\begin{pgfscope}%
\pgfpathrectangle{\pgfqpoint{0.211875in}{0.211875in}}{\pgfqpoint{1.313625in}{1.279725in}}%
\pgfusepath{clip}%
\pgfsetbuttcap%
\pgfsetroundjoin%
\definecolor{currentfill}{rgb}{0.961115,0.566634,0.405693}%
\pgfsetfillcolor{currentfill}%
\pgfsetlinewidth{0.000000pt}%
\definecolor{currentstroke}{rgb}{0.000000,0.000000,0.000000}%
\pgfsetstrokecolor{currentstroke}%
\pgfsetdash{}{0pt}%
\pgfpathmoveto{\pgfqpoint{0.795708in}{1.090308in}}%
\pgfpathlineto{\pgfqpoint{0.803162in}{1.090878in}}%
\pgfpathlineto{\pgfqpoint{0.808977in}{1.091510in}}%
\pgfpathlineto{\pgfqpoint{0.820528in}{1.103805in}}%
\pgfpathlineto{\pgfqpoint{0.822246in}{1.115318in}}%
\pgfpathlineto{\pgfqpoint{0.822373in}{1.116731in}}%
\pgfpathlineto{\pgfqpoint{0.822571in}{1.129658in}}%
\pgfpathlineto{\pgfqpoint{0.822246in}{1.135375in}}%
\pgfpathlineto{\pgfqpoint{0.821658in}{1.142584in}}%
\pgfpathlineto{\pgfqpoint{0.816902in}{1.155511in}}%
\pgfpathlineto{\pgfqpoint{0.808977in}{1.160323in}}%
\pgfpathlineto{\pgfqpoint{0.795708in}{1.161476in}}%
\pgfpathlineto{\pgfqpoint{0.782439in}{1.159410in}}%
\pgfpathlineto{\pgfqpoint{0.776992in}{1.155511in}}%
\pgfpathlineto{\pgfqpoint{0.772073in}{1.142584in}}%
\pgfpathlineto{\pgfqpoint{0.771043in}{1.129658in}}%
\pgfpathlineto{\pgfqpoint{0.771289in}{1.116731in}}%
\pgfpathlineto{\pgfqpoint{0.773264in}{1.103805in}}%
\pgfpathlineto{\pgfqpoint{0.782439in}{1.092515in}}%
\pgfpathlineto{\pgfqpoint{0.791115in}{1.090878in}}%
\pgfpathclose%
\pgfpathmoveto{\pgfqpoint{0.783600in}{1.103805in}}%
\pgfpathlineto{\pgfqpoint{0.782439in}{1.104787in}}%
\pgfpathlineto{\pgfqpoint{0.778092in}{1.116731in}}%
\pgfpathlineto{\pgfqpoint{0.777486in}{1.129658in}}%
\pgfpathlineto{\pgfqpoint{0.780019in}{1.142584in}}%
\pgfpathlineto{\pgfqpoint{0.782439in}{1.146754in}}%
\pgfpathlineto{\pgfqpoint{0.795708in}{1.152992in}}%
\pgfpathlineto{\pgfqpoint{0.808977in}{1.148797in}}%
\pgfpathlineto{\pgfqpoint{0.813122in}{1.142584in}}%
\pgfpathlineto{\pgfqpoint{0.815738in}{1.129658in}}%
\pgfpathlineto{\pgfqpoint{0.815115in}{1.116731in}}%
\pgfpathlineto{\pgfqpoint{0.810060in}{1.103805in}}%
\pgfpathlineto{\pgfqpoint{0.808977in}{1.102652in}}%
\pgfpathlineto{\pgfqpoint{0.795708in}{1.099000in}}%
\pgfpathclose%
\pgfusepath{fill}%
\end{pgfscope}%
\begin{pgfscope}%
\pgfpathrectangle{\pgfqpoint{0.211875in}{0.211875in}}{\pgfqpoint{1.313625in}{1.279725in}}%
\pgfusepath{clip}%
\pgfsetbuttcap%
\pgfsetroundjoin%
\definecolor{currentfill}{rgb}{0.961115,0.566634,0.405693}%
\pgfsetfillcolor{currentfill}%
\pgfsetlinewidth{0.000000pt}%
\definecolor{currentstroke}{rgb}{0.000000,0.000000,0.000000}%
\pgfsetstrokecolor{currentstroke}%
\pgfsetdash{}{0pt}%
\pgfpathmoveto{\pgfqpoint{0.901860in}{1.090145in}}%
\pgfpathlineto{\pgfqpoint{0.915129in}{1.089003in}}%
\pgfpathlineto{\pgfqpoint{0.928398in}{1.090180in}}%
\pgfpathlineto{\pgfqpoint{0.930448in}{1.090878in}}%
\pgfpathlineto{\pgfqpoint{0.939818in}{1.103805in}}%
\pgfpathlineto{\pgfqpoint{0.941137in}{1.116731in}}%
\pgfpathlineto{\pgfqpoint{0.941300in}{1.129658in}}%
\pgfpathlineto{\pgfqpoint{0.940617in}{1.142584in}}%
\pgfpathlineto{\pgfqpoint{0.937266in}{1.155511in}}%
\pgfpathlineto{\pgfqpoint{0.928398in}{1.161606in}}%
\pgfpathlineto{\pgfqpoint{0.915129in}{1.162802in}}%
\pgfpathlineto{\pgfqpoint{0.901860in}{1.161641in}}%
\pgfpathlineto{\pgfqpoint{0.892188in}{1.155511in}}%
\pgfpathlineto{\pgfqpoint{0.888591in}{1.144024in}}%
\pgfpathlineto{\pgfqpoint{0.888387in}{1.142584in}}%
\pgfpathlineto{\pgfqpoint{0.887698in}{1.129658in}}%
\pgfpathlineto{\pgfqpoint{0.887863in}{1.116731in}}%
\pgfpathlineto{\pgfqpoint{0.888591in}{1.108514in}}%
\pgfpathlineto{\pgfqpoint{0.889281in}{1.103805in}}%
\pgfpathlineto{\pgfqpoint{0.899604in}{1.090878in}}%
\pgfpathclose%
\pgfpathmoveto{\pgfqpoint{0.899463in}{1.103805in}}%
\pgfpathlineto{\pgfqpoint{0.894888in}{1.116731in}}%
\pgfpathlineto{\pgfqpoint{0.894319in}{1.129658in}}%
\pgfpathlineto{\pgfqpoint{0.896702in}{1.142584in}}%
\pgfpathlineto{\pgfqpoint{0.901860in}{1.150482in}}%
\pgfpathlineto{\pgfqpoint{0.915129in}{1.154666in}}%
\pgfpathlineto{\pgfqpoint{0.928398in}{1.149508in}}%
\pgfpathlineto{\pgfqpoint{0.932498in}{1.142584in}}%
\pgfpathlineto{\pgfqpoint{0.934746in}{1.129658in}}%
\pgfpathlineto{\pgfqpoint{0.934210in}{1.116731in}}%
\pgfpathlineto{\pgfqpoint{0.929873in}{1.103805in}}%
\pgfpathlineto{\pgfqpoint{0.928398in}{1.102034in}}%
\pgfpathlineto{\pgfqpoint{0.915129in}{1.097542in}}%
\pgfpathlineto{\pgfqpoint{0.901860in}{1.101186in}}%
\pgfpathclose%
\pgfusepath{fill}%
\end{pgfscope}%
\begin{pgfscope}%
\pgfpathrectangle{\pgfqpoint{0.211875in}{0.211875in}}{\pgfqpoint{1.313625in}{1.279725in}}%
\pgfusepath{clip}%
\pgfsetbuttcap%
\pgfsetroundjoin%
\definecolor{currentfill}{rgb}{0.961115,0.566634,0.405693}%
\pgfsetfillcolor{currentfill}%
\pgfsetlinewidth{0.000000pt}%
\definecolor{currentstroke}{rgb}{0.000000,0.000000,0.000000}%
\pgfsetstrokecolor{currentstroke}%
\pgfsetdash{}{0pt}%
\pgfpathmoveto{\pgfqpoint{1.021280in}{1.088873in}}%
\pgfpathlineto{\pgfqpoint{1.034549in}{1.088245in}}%
\pgfpathlineto{\pgfqpoint{1.047818in}{1.089528in}}%
\pgfpathlineto{\pgfqpoint{1.051281in}{1.090878in}}%
\pgfpathlineto{\pgfqpoint{1.058449in}{1.103805in}}%
\pgfpathlineto{\pgfqpoint{1.059473in}{1.116731in}}%
\pgfpathlineto{\pgfqpoint{1.059599in}{1.129658in}}%
\pgfpathlineto{\pgfqpoint{1.059068in}{1.142584in}}%
\pgfpathlineto{\pgfqpoint{1.056480in}{1.155511in}}%
\pgfpathlineto{\pgfqpoint{1.047818in}{1.162269in}}%
\pgfpathlineto{\pgfqpoint{1.034549in}{1.163571in}}%
\pgfpathlineto{\pgfqpoint{1.021280in}{1.162934in}}%
\pgfpathlineto{\pgfqpoint{1.008068in}{1.155511in}}%
\pgfpathlineto{\pgfqpoint{1.008011in}{1.155388in}}%
\pgfpathlineto{\pgfqpoint{1.005497in}{1.142584in}}%
\pgfpathlineto{\pgfqpoint{1.004966in}{1.129658in}}%
\pgfpathlineto{\pgfqpoint{1.005093in}{1.116731in}}%
\pgfpathlineto{\pgfqpoint{1.006113in}{1.103805in}}%
\pgfpathlineto{\pgfqpoint{1.008011in}{1.096914in}}%
\pgfpathlineto{\pgfqpoint{1.014260in}{1.090878in}}%
\pgfpathclose%
\pgfpathmoveto{\pgfqpoint{1.016474in}{1.103805in}}%
\pgfpathlineto{\pgfqpoint{1.011974in}{1.116731in}}%
\pgfpathlineto{\pgfqpoint{1.011414in}{1.129658in}}%
\pgfpathlineto{\pgfqpoint{1.013756in}{1.142584in}}%
\pgfpathlineto{\pgfqpoint{1.021280in}{1.152837in}}%
\pgfpathlineto{\pgfqpoint{1.034192in}{1.155511in}}%
\pgfpathlineto{\pgfqpoint{1.034549in}{1.155560in}}%
\pgfpathlineto{\pgfqpoint{1.034738in}{1.155511in}}%
\pgfpathlineto{\pgfqpoint{1.047818in}{1.149190in}}%
\pgfpathlineto{\pgfqpoint{1.051280in}{1.142584in}}%
\pgfpathlineto{\pgfqpoint{1.053305in}{1.129658in}}%
\pgfpathlineto{\pgfqpoint{1.052822in}{1.116731in}}%
\pgfpathlineto{\pgfqpoint{1.048919in}{1.103805in}}%
\pgfpathlineto{\pgfqpoint{1.047818in}{1.102310in}}%
\pgfpathlineto{\pgfqpoint{1.034549in}{1.096753in}}%
\pgfpathlineto{\pgfqpoint{1.021280in}{1.099135in}}%
\pgfpathclose%
\pgfusepath{fill}%
\end{pgfscope}%
\begin{pgfscope}%
\pgfpathrectangle{\pgfqpoint{0.211875in}{0.211875in}}{\pgfqpoint{1.313625in}{1.279725in}}%
\pgfusepath{clip}%
\pgfsetbuttcap%
\pgfsetroundjoin%
\definecolor{currentfill}{rgb}{0.961115,0.566634,0.405693}%
\pgfsetfillcolor{currentfill}%
\pgfsetlinewidth{0.000000pt}%
\definecolor{currentstroke}{rgb}{0.000000,0.000000,0.000000}%
\pgfsetstrokecolor{currentstroke}%
\pgfsetdash{}{0pt}%
\pgfpathmoveto{\pgfqpoint{1.140701in}{1.088354in}}%
\pgfpathlineto{\pgfqpoint{1.153970in}{1.088031in}}%
\pgfpathlineto{\pgfqpoint{1.167239in}{1.089639in}}%
\pgfpathlineto{\pgfqpoint{1.169998in}{1.090878in}}%
\pgfpathlineto{\pgfqpoint{1.176496in}{1.103805in}}%
\pgfpathlineto{\pgfqpoint{1.177436in}{1.116731in}}%
\pgfpathlineto{\pgfqpoint{1.177552in}{1.129658in}}%
\pgfpathlineto{\pgfqpoint{1.177064in}{1.142584in}}%
\pgfpathlineto{\pgfqpoint{1.174697in}{1.155511in}}%
\pgfpathlineto{\pgfqpoint{1.167239in}{1.162155in}}%
\pgfpathlineto{\pgfqpoint{1.153970in}{1.163788in}}%
\pgfpathlineto{\pgfqpoint{1.140701in}{1.163460in}}%
\pgfpathlineto{\pgfqpoint{1.127432in}{1.158875in}}%
\pgfpathlineto{\pgfqpoint{1.125373in}{1.155511in}}%
\pgfpathlineto{\pgfqpoint{1.123117in}{1.142584in}}%
\pgfpathlineto{\pgfqpoint{1.122652in}{1.129658in}}%
\pgfpathlineto{\pgfqpoint{1.122763in}{1.116731in}}%
\pgfpathlineto{\pgfqpoint{1.123658in}{1.103805in}}%
\pgfpathlineto{\pgfqpoint{1.127432in}{1.093103in}}%
\pgfpathlineto{\pgfqpoint{1.130626in}{1.090878in}}%
\pgfpathclose%
\pgfpathmoveto{\pgfqpoint{1.134037in}{1.103805in}}%
\pgfpathlineto{\pgfqpoint{1.129364in}{1.116731in}}%
\pgfpathlineto{\pgfqpoint{1.128785in}{1.129658in}}%
\pgfpathlineto{\pgfqpoint{1.131212in}{1.142584in}}%
\pgfpathlineto{\pgfqpoint{1.140701in}{1.154076in}}%
\pgfpathlineto{\pgfqpoint{1.152153in}{1.155511in}}%
\pgfpathlineto{\pgfqpoint{1.153970in}{1.155671in}}%
\pgfpathlineto{\pgfqpoint{1.154490in}{1.155511in}}%
\pgfpathlineto{\pgfqpoint{1.167239in}{1.147565in}}%
\pgfpathlineto{\pgfqpoint{1.169533in}{1.142584in}}%
\pgfpathlineto{\pgfqpoint{1.171461in}{1.129658in}}%
\pgfpathlineto{\pgfqpoint{1.171001in}{1.116731in}}%
\pgfpathlineto{\pgfqpoint{1.167290in}{1.103805in}}%
\pgfpathlineto{\pgfqpoint{1.167239in}{1.103725in}}%
\pgfpathlineto{\pgfqpoint{1.153970in}{1.096630in}}%
\pgfpathlineto{\pgfqpoint{1.140701in}{1.098056in}}%
\pgfpathclose%
\pgfusepath{fill}%
\end{pgfscope}%
\begin{pgfscope}%
\pgfpathrectangle{\pgfqpoint{0.211875in}{0.211875in}}{\pgfqpoint{1.313625in}{1.279725in}}%
\pgfusepath{clip}%
\pgfsetbuttcap%
\pgfsetroundjoin%
\definecolor{currentfill}{rgb}{0.961115,0.566634,0.405693}%
\pgfsetfillcolor{currentfill}%
\pgfsetlinewidth{0.000000pt}%
\definecolor{currentstroke}{rgb}{0.000000,0.000000,0.000000}%
\pgfsetstrokecolor{currentstroke}%
\pgfsetdash{}{0pt}%
\pgfpathmoveto{\pgfqpoint{1.260121in}{1.088479in}}%
\pgfpathlineto{\pgfqpoint{1.273390in}{1.088378in}}%
\pgfpathlineto{\pgfqpoint{1.286659in}{1.090768in}}%
\pgfpathlineto{\pgfqpoint{1.286870in}{1.090878in}}%
\pgfpathlineto{\pgfqpoint{1.294002in}{1.103805in}}%
\pgfpathlineto{\pgfqpoint{1.295045in}{1.116731in}}%
\pgfpathlineto{\pgfqpoint{1.295175in}{1.129658in}}%
\pgfpathlineto{\pgfqpoint{1.294632in}{1.142584in}}%
\pgfpathlineto{\pgfqpoint{1.292011in}{1.155511in}}%
\pgfpathlineto{\pgfqpoint{1.286659in}{1.161008in}}%
\pgfpathlineto{\pgfqpoint{1.273390in}{1.163436in}}%
\pgfpathlineto{\pgfqpoint{1.260121in}{1.163334in}}%
\pgfpathlineto{\pgfqpoint{1.246852in}{1.159815in}}%
\pgfpathlineto{\pgfqpoint{1.243628in}{1.155511in}}%
\pgfpathlineto{\pgfqpoint{1.241225in}{1.142584in}}%
\pgfpathlineto{\pgfqpoint{1.240732in}{1.129658in}}%
\pgfpathlineto{\pgfqpoint{1.240850in}{1.116731in}}%
\pgfpathlineto{\pgfqpoint{1.241800in}{1.103805in}}%
\pgfpathlineto{\pgfqpoint{1.246852in}{1.092069in}}%
\pgfpathlineto{\pgfqpoint{1.249132in}{1.090878in}}%
\pgfpathclose%
\pgfpathmoveto{\pgfqpoint{1.252252in}{1.103805in}}%
\pgfpathlineto{\pgfqpoint{1.247097in}{1.116731in}}%
\pgfpathlineto{\pgfqpoint{1.246852in}{1.121532in}}%
\pgfpathlineto{\pgfqpoint{1.246576in}{1.129658in}}%
\pgfpathlineto{\pgfqpoint{1.246852in}{1.131927in}}%
\pgfpathlineto{\pgfqpoint{1.249133in}{1.142584in}}%
\pgfpathlineto{\pgfqpoint{1.260121in}{1.154369in}}%
\pgfpathlineto{\pgfqpoint{1.273390in}{1.155063in}}%
\pgfpathlineto{\pgfqpoint{1.286659in}{1.144172in}}%
\pgfpathlineto{\pgfqpoint{1.287296in}{1.142584in}}%
\pgfpathlineto{\pgfqpoint{1.289237in}{1.129658in}}%
\pgfpathlineto{\pgfqpoint{1.288774in}{1.116731in}}%
\pgfpathlineto{\pgfqpoint{1.286659in}{1.108312in}}%
\pgfpathlineto{\pgfqpoint{1.284015in}{1.103805in}}%
\pgfpathlineto{\pgfqpoint{1.273390in}{1.097196in}}%
\pgfpathlineto{\pgfqpoint{1.260121in}{1.097800in}}%
\pgfpathclose%
\pgfusepath{fill}%
\end{pgfscope}%
\begin{pgfscope}%
\pgfpathrectangle{\pgfqpoint{0.211875in}{0.211875in}}{\pgfqpoint{1.313625in}{1.279725in}}%
\pgfusepath{clip}%
\pgfsetbuttcap%
\pgfsetroundjoin%
\definecolor{currentfill}{rgb}{0.961115,0.566634,0.405693}%
\pgfsetfillcolor{currentfill}%
\pgfsetlinewidth{0.000000pt}%
\definecolor{currentstroke}{rgb}{0.000000,0.000000,0.000000}%
\pgfsetstrokecolor{currentstroke}%
\pgfsetdash{}{0pt}%
\pgfpathmoveto{\pgfqpoint{1.379542in}{1.089176in}}%
\pgfpathlineto{\pgfqpoint{1.392811in}{1.089327in}}%
\pgfpathlineto{\pgfqpoint{1.400101in}{1.090878in}}%
\pgfpathlineto{\pgfqpoint{1.406080in}{1.093643in}}%
\pgfpathlineto{\pgfqpoint{1.410981in}{1.103805in}}%
\pgfpathlineto{\pgfqpoint{1.412303in}{1.116731in}}%
\pgfpathlineto{\pgfqpoint{1.412467in}{1.129658in}}%
\pgfpathlineto{\pgfqpoint{1.411779in}{1.142584in}}%
\pgfpathlineto{\pgfqpoint{1.408471in}{1.155511in}}%
\pgfpathlineto{\pgfqpoint{1.406080in}{1.158385in}}%
\pgfpathlineto{\pgfqpoint{1.392811in}{1.162473in}}%
\pgfpathlineto{\pgfqpoint{1.379542in}{1.162626in}}%
\pgfpathlineto{\pgfqpoint{1.366273in}{1.159364in}}%
\pgfpathlineto{\pgfqpoint{1.362842in}{1.155511in}}%
\pgfpathlineto{\pgfqpoint{1.359816in}{1.142584in}}%
\pgfpathlineto{\pgfqpoint{1.359198in}{1.129658in}}%
\pgfpathlineto{\pgfqpoint{1.359345in}{1.116731in}}%
\pgfpathlineto{\pgfqpoint{1.360538in}{1.103805in}}%
\pgfpathlineto{\pgfqpoint{1.366273in}{1.092566in}}%
\pgfpathlineto{\pgfqpoint{1.370506in}{1.090878in}}%
\pgfpathclose%
\pgfpathmoveto{\pgfqpoint{1.371288in}{1.103805in}}%
\pgfpathlineto{\pgfqpoint{1.366273in}{1.113789in}}%
\pgfpathlineto{\pgfqpoint{1.365606in}{1.116731in}}%
\pgfpathlineto{\pgfqpoint{1.365122in}{1.129658in}}%
\pgfpathlineto{\pgfqpoint{1.366273in}{1.137654in}}%
\pgfpathlineto{\pgfqpoint{1.367628in}{1.142584in}}%
\pgfpathlineto{\pgfqpoint{1.379542in}{1.153828in}}%
\pgfpathlineto{\pgfqpoint{1.392811in}{1.153560in}}%
\pgfpathlineto{\pgfqpoint{1.403834in}{1.142584in}}%
\pgfpathlineto{\pgfqpoint{1.406080in}{1.133734in}}%
\pgfpathlineto{\pgfqpoint{1.406638in}{1.129658in}}%
\pgfpathlineto{\pgfqpoint{1.406147in}{1.116731in}}%
\pgfpathlineto{\pgfqpoint{1.406080in}{1.116418in}}%
\pgfpathlineto{\pgfqpoint{1.400263in}{1.103805in}}%
\pgfpathlineto{\pgfqpoint{1.392811in}{1.098505in}}%
\pgfpathlineto{\pgfqpoint{1.379542in}{1.098272in}}%
\pgfpathclose%
\pgfusepath{fill}%
\end{pgfscope}%
\begin{pgfscope}%
\pgfpathrectangle{\pgfqpoint{0.211875in}{0.211875in}}{\pgfqpoint{1.313625in}{1.279725in}}%
\pgfusepath{clip}%
\pgfsetbuttcap%
\pgfsetroundjoin%
\definecolor{currentfill}{rgb}{0.961115,0.566634,0.405693}%
\pgfsetfillcolor{currentfill}%
\pgfsetlinewidth{0.000000pt}%
\definecolor{currentstroke}{rgb}{0.000000,0.000000,0.000000}%
\pgfsetstrokecolor{currentstroke}%
\pgfsetdash{}{0pt}%
\pgfpathmoveto{\pgfqpoint{1.498962in}{1.090405in}}%
\pgfpathlineto{\pgfqpoint{1.510639in}{1.090878in}}%
\pgfpathlineto{\pgfqpoint{1.512231in}{1.090955in}}%
\pgfpathlineto{\pgfqpoint{1.525500in}{1.099039in}}%
\pgfpathlineto{\pgfqpoint{1.525500in}{1.103805in}}%
\pgfpathlineto{\pgfqpoint{1.525500in}{1.116731in}}%
\pgfpathlineto{\pgfqpoint{1.525500in}{1.129658in}}%
\pgfpathlineto{\pgfqpoint{1.525500in}{1.142584in}}%
\pgfpathlineto{\pgfqpoint{1.525500in}{1.152947in}}%
\pgfpathlineto{\pgfqpoint{1.523547in}{1.155511in}}%
\pgfpathlineto{\pgfqpoint{1.512231in}{1.160827in}}%
\pgfpathlineto{\pgfqpoint{1.498962in}{1.161377in}}%
\pgfpathlineto{\pgfqpoint{1.485693in}{1.158033in}}%
\pgfpathlineto{\pgfqpoint{1.483074in}{1.155511in}}%
\pgfpathlineto{\pgfqpoint{1.478904in}{1.142584in}}%
\pgfpathlineto{\pgfqpoint{1.478056in}{1.129658in}}%
\pgfpathlineto{\pgfqpoint{1.478258in}{1.116731in}}%
\pgfpathlineto{\pgfqpoint{1.479895in}{1.103805in}}%
\pgfpathlineto{\pgfqpoint{1.485693in}{1.094030in}}%
\pgfpathlineto{\pgfqpoint{1.496008in}{1.090878in}}%
\pgfpathclose%
\pgfpathmoveto{\pgfqpoint{1.491429in}{1.103805in}}%
\pgfpathlineto{\pgfqpoint{1.485693in}{1.112760in}}%
\pgfpathlineto{\pgfqpoint{1.484648in}{1.116731in}}%
\pgfpathlineto{\pgfqpoint{1.484101in}{1.129658in}}%
\pgfpathlineto{\pgfqpoint{1.485693in}{1.139187in}}%
\pgfpathlineto{\pgfqpoint{1.486882in}{1.142584in}}%
\pgfpathlineto{\pgfqpoint{1.498962in}{1.152515in}}%
\pgfpathlineto{\pgfqpoint{1.512231in}{1.151105in}}%
\pgfpathlineto{\pgfqpoint{1.519790in}{1.142584in}}%
\pgfpathlineto{\pgfqpoint{1.522935in}{1.129658in}}%
\pgfpathlineto{\pgfqpoint{1.522183in}{1.116731in}}%
\pgfpathlineto{\pgfqpoint{1.516155in}{1.103805in}}%
\pgfpathlineto{\pgfqpoint{1.512231in}{1.100643in}}%
\pgfpathlineto{\pgfqpoint{1.498962in}{1.099415in}}%
\pgfpathclose%
\pgfusepath{fill}%
\end{pgfscope}%
\begin{pgfscope}%
\pgfpathrectangle{\pgfqpoint{0.211875in}{0.211875in}}{\pgfqpoint{1.313625in}{1.279725in}}%
\pgfusepath{clip}%
\pgfsetbuttcap%
\pgfsetroundjoin%
\definecolor{currentfill}{rgb}{0.961115,0.566634,0.405693}%
\pgfsetfillcolor{currentfill}%
\pgfsetlinewidth{0.000000pt}%
\definecolor{currentstroke}{rgb}{0.000000,0.000000,0.000000}%
\pgfsetstrokecolor{currentstroke}%
\pgfsetdash{}{0pt}%
\pgfpathmoveto{\pgfqpoint{0.318027in}{1.103167in}}%
\pgfpathlineto{\pgfqpoint{0.331295in}{1.102581in}}%
\pgfpathlineto{\pgfqpoint{0.333188in}{1.103805in}}%
\pgfpathlineto{\pgfqpoint{0.341364in}{1.116731in}}%
\pgfpathlineto{\pgfqpoint{0.342358in}{1.129658in}}%
\pgfpathlineto{\pgfqpoint{0.338163in}{1.142584in}}%
\pgfpathlineto{\pgfqpoint{0.331295in}{1.148879in}}%
\pgfpathlineto{\pgfqpoint{0.318027in}{1.148206in}}%
\pgfpathlineto{\pgfqpoint{0.312531in}{1.142584in}}%
\pgfpathlineto{\pgfqpoint{0.308485in}{1.129658in}}%
\pgfpathlineto{\pgfqpoint{0.309456in}{1.116731in}}%
\pgfpathlineto{\pgfqpoint{0.317159in}{1.103805in}}%
\pgfpathclose%
\pgfpathmoveto{\pgfqpoint{0.327858in}{1.116731in}}%
\pgfpathlineto{\pgfqpoint{0.318027in}{1.120691in}}%
\pgfpathlineto{\pgfqpoint{0.317046in}{1.129658in}}%
\pgfpathlineto{\pgfqpoint{0.318027in}{1.132162in}}%
\pgfpathlineto{\pgfqpoint{0.331295in}{1.133653in}}%
\pgfpathlineto{\pgfqpoint{0.333013in}{1.129658in}}%
\pgfpathlineto{\pgfqpoint{0.331470in}{1.116731in}}%
\pgfpathlineto{\pgfqpoint{0.331295in}{1.116473in}}%
\pgfpathclose%
\pgfusepath{fill}%
\end{pgfscope}%
\begin{pgfscope}%
\pgfpathrectangle{\pgfqpoint{0.211875in}{0.211875in}}{\pgfqpoint{1.313625in}{1.279725in}}%
\pgfusepath{clip}%
\pgfsetbuttcap%
\pgfsetroundjoin%
\definecolor{currentfill}{rgb}{0.961115,0.566634,0.405693}%
\pgfsetfillcolor{currentfill}%
\pgfsetlinewidth{0.000000pt}%
\definecolor{currentstroke}{rgb}{0.000000,0.000000,0.000000}%
\pgfsetstrokecolor{currentstroke}%
\pgfsetdash{}{0pt}%
\pgfpathmoveto{\pgfqpoint{0.437447in}{1.098703in}}%
\pgfpathlineto{\pgfqpoint{0.450716in}{1.099052in}}%
\pgfpathlineto{\pgfqpoint{0.457154in}{1.103805in}}%
\pgfpathlineto{\pgfqpoint{0.462862in}{1.116731in}}%
\pgfpathlineto{\pgfqpoint{0.463559in}{1.129658in}}%
\pgfpathlineto{\pgfqpoint{0.460622in}{1.142584in}}%
\pgfpathlineto{\pgfqpoint{0.450716in}{1.152932in}}%
\pgfpathlineto{\pgfqpoint{0.437447in}{1.153333in}}%
\pgfpathlineto{\pgfqpoint{0.425486in}{1.142584in}}%
\pgfpathlineto{\pgfqpoint{0.424178in}{1.138414in}}%
\pgfpathlineto{\pgfqpoint{0.422799in}{1.129658in}}%
\pgfpathlineto{\pgfqpoint{0.423323in}{1.116731in}}%
\pgfpathlineto{\pgfqpoint{0.424178in}{1.113278in}}%
\pgfpathlineto{\pgfqpoint{0.429545in}{1.103805in}}%
\pgfpathclose%
\pgfpathmoveto{\pgfqpoint{0.432794in}{1.116731in}}%
\pgfpathlineto{\pgfqpoint{0.431425in}{1.129658in}}%
\pgfpathlineto{\pgfqpoint{0.437126in}{1.142584in}}%
\pgfpathlineto{\pgfqpoint{0.437447in}{1.142872in}}%
\pgfpathlineto{\pgfqpoint{0.441501in}{1.142584in}}%
\pgfpathlineto{\pgfqpoint{0.450716in}{1.141178in}}%
\pgfpathlineto{\pgfqpoint{0.455070in}{1.129658in}}%
\pgfpathlineto{\pgfqpoint{0.453879in}{1.116731in}}%
\pgfpathlineto{\pgfqpoint{0.450716in}{1.111424in}}%
\pgfpathlineto{\pgfqpoint{0.437447in}{1.110087in}}%
\pgfpathclose%
\pgfusepath{fill}%
\end{pgfscope}%
\begin{pgfscope}%
\pgfpathrectangle{\pgfqpoint{0.211875in}{0.211875in}}{\pgfqpoint{1.313625in}{1.279725in}}%
\pgfusepath{clip}%
\pgfsetbuttcap%
\pgfsetroundjoin%
\definecolor{currentfill}{rgb}{0.961115,0.566634,0.405693}%
\pgfsetfillcolor{currentfill}%
\pgfsetlinewidth{0.000000pt}%
\definecolor{currentstroke}{rgb}{0.000000,0.000000,0.000000}%
\pgfsetstrokecolor{currentstroke}%
\pgfsetdash{}{0pt}%
\pgfpathmoveto{\pgfqpoint{0.543598in}{1.101835in}}%
\pgfpathlineto{\pgfqpoint{0.556867in}{1.095133in}}%
\pgfpathlineto{\pgfqpoint{0.570136in}{1.096027in}}%
\pgfpathlineto{\pgfqpoint{0.579430in}{1.103805in}}%
\pgfpathlineto{\pgfqpoint{0.583384in}{1.116731in}}%
\pgfpathlineto{\pgfqpoint{0.583405in}{1.117278in}}%
\pgfpathlineto{\pgfqpoint{0.583754in}{1.129658in}}%
\pgfpathlineto{\pgfqpoint{0.583405in}{1.133115in}}%
\pgfpathlineto{\pgfqpoint{0.581829in}{1.142584in}}%
\pgfpathlineto{\pgfqpoint{0.571634in}{1.155511in}}%
\pgfpathlineto{\pgfqpoint{0.570136in}{1.156219in}}%
\pgfpathlineto{\pgfqpoint{0.556867in}{1.157031in}}%
\pgfpathlineto{\pgfqpoint{0.552276in}{1.155511in}}%
\pgfpathlineto{\pgfqpoint{0.543598in}{1.149735in}}%
\pgfpathlineto{\pgfqpoint{0.540362in}{1.142584in}}%
\pgfpathlineto{\pgfqpoint{0.538637in}{1.129658in}}%
\pgfpathlineto{\pgfqpoint{0.539050in}{1.116731in}}%
\pgfpathlineto{\pgfqpoint{0.542353in}{1.103805in}}%
\pgfpathclose%
\pgfpathmoveto{\pgfqpoint{0.546770in}{1.116731in}}%
\pgfpathlineto{\pgfqpoint{0.545441in}{1.129658in}}%
\pgfpathlineto{\pgfqpoint{0.550981in}{1.142584in}}%
\pgfpathlineto{\pgfqpoint{0.556867in}{1.147176in}}%
\pgfpathlineto{\pgfqpoint{0.570136in}{1.144911in}}%
\pgfpathlineto{\pgfqpoint{0.572105in}{1.142584in}}%
\pgfpathlineto{\pgfqpoint{0.576044in}{1.129658in}}%
\pgfpathlineto{\pgfqpoint{0.575108in}{1.116731in}}%
\pgfpathlineto{\pgfqpoint{0.570136in}{1.107303in}}%
\pgfpathlineto{\pgfqpoint{0.556867in}{1.104211in}}%
\pgfpathclose%
\pgfusepath{fill}%
\end{pgfscope}%
\begin{pgfscope}%
\pgfpathrectangle{\pgfqpoint{0.211875in}{0.211875in}}{\pgfqpoint{1.313625in}{1.279725in}}%
\pgfusepath{clip}%
\pgfsetbuttcap%
\pgfsetroundjoin%
\definecolor{currentfill}{rgb}{0.961115,0.566634,0.405693}%
\pgfsetfillcolor{currentfill}%
\pgfsetlinewidth{0.000000pt}%
\definecolor{currentstroke}{rgb}{0.000000,0.000000,0.000000}%
\pgfsetstrokecolor{currentstroke}%
\pgfsetdash{}{0pt}%
\pgfpathmoveto{\pgfqpoint{0.663019in}{1.096293in}}%
\pgfpathlineto{\pgfqpoint{0.676288in}{1.092340in}}%
\pgfpathlineto{\pgfqpoint{0.689557in}{1.093505in}}%
\pgfpathlineto{\pgfqpoint{0.700461in}{1.103805in}}%
\pgfpathlineto{\pgfqpoint{0.702826in}{1.114461in}}%
\pgfpathlineto{\pgfqpoint{0.703108in}{1.116731in}}%
\pgfpathlineto{\pgfqpoint{0.703378in}{1.129658in}}%
\pgfpathlineto{\pgfqpoint{0.702826in}{1.136651in}}%
\pgfpathlineto{\pgfqpoint{0.702106in}{1.142584in}}%
\pgfpathlineto{\pgfqpoint{0.695147in}{1.155511in}}%
\pgfpathlineto{\pgfqpoint{0.689557in}{1.158510in}}%
\pgfpathlineto{\pgfqpoint{0.676288in}{1.159569in}}%
\pgfpathlineto{\pgfqpoint{0.663019in}{1.155978in}}%
\pgfpathlineto{\pgfqpoint{0.662443in}{1.155511in}}%
\pgfpathlineto{\pgfqpoint{0.656068in}{1.142584in}}%
\pgfpathlineto{\pgfqpoint{0.654729in}{1.129658in}}%
\pgfpathlineto{\pgfqpoint{0.655049in}{1.116731in}}%
\pgfpathlineto{\pgfqpoint{0.657616in}{1.103805in}}%
\pgfpathclose%
\pgfpathmoveto{\pgfqpoint{0.670742in}{1.103805in}}%
\pgfpathlineto{\pgfqpoint{0.663019in}{1.112280in}}%
\pgfpathlineto{\pgfqpoint{0.661583in}{1.116731in}}%
\pgfpathlineto{\pgfqpoint{0.660918in}{1.129658in}}%
\pgfpathlineto{\pgfqpoint{0.663019in}{1.139902in}}%
\pgfpathlineto{\pgfqpoint{0.664370in}{1.142584in}}%
\pgfpathlineto{\pgfqpoint{0.676288in}{1.150516in}}%
\pgfpathlineto{\pgfqpoint{0.689557in}{1.147231in}}%
\pgfpathlineto{\pgfqpoint{0.693045in}{1.142584in}}%
\pgfpathlineto{\pgfqpoint{0.696208in}{1.129658in}}%
\pgfpathlineto{\pgfqpoint{0.695456in}{1.116731in}}%
\pgfpathlineto{\pgfqpoint{0.689557in}{1.104136in}}%
\pgfpathlineto{\pgfqpoint{0.688724in}{1.103805in}}%
\pgfpathlineto{\pgfqpoint{0.676288in}{1.101156in}}%
\pgfpathclose%
\pgfusepath{fill}%
\end{pgfscope}%
\begin{pgfscope}%
\pgfpathrectangle{\pgfqpoint{0.211875in}{0.211875in}}{\pgfqpoint{1.313625in}{1.279725in}}%
\pgfusepath{clip}%
\pgfsetbuttcap%
\pgfsetroundjoin%
\definecolor{currentfill}{rgb}{0.961115,0.566634,0.405693}%
\pgfsetfillcolor{currentfill}%
\pgfsetlinewidth{0.000000pt}%
\definecolor{currentstroke}{rgb}{0.000000,0.000000,0.000000}%
\pgfsetstrokecolor{currentstroke}%
\pgfsetdash{}{0pt}%
\pgfpathmoveto{\pgfqpoint{0.218494in}{1.116731in}}%
\pgfpathlineto{\pgfqpoint{0.219908in}{1.129658in}}%
\pgfpathlineto{\pgfqpoint{0.213933in}{1.142584in}}%
\pgfpathlineto{\pgfqpoint{0.211875in}{1.144221in}}%
\pgfpathlineto{\pgfqpoint{0.211875in}{1.142584in}}%
\pgfpathlineto{\pgfqpoint{0.211875in}{1.129658in}}%
\pgfpathlineto{\pgfqpoint{0.211875in}{1.116731in}}%
\pgfpathlineto{\pgfqpoint{0.211875in}{1.108245in}}%
\pgfpathclose%
\pgfusepath{fill}%
\end{pgfscope}%
\begin{pgfscope}%
\pgfpathrectangle{\pgfqpoint{0.211875in}{0.211875in}}{\pgfqpoint{1.313625in}{1.279725in}}%
\pgfusepath{clip}%
\pgfsetbuttcap%
\pgfsetroundjoin%
\definecolor{currentfill}{rgb}{0.961115,0.566634,0.405693}%
\pgfsetfillcolor{currentfill}%
\pgfsetlinewidth{0.000000pt}%
\definecolor{currentstroke}{rgb}{0.000000,0.000000,0.000000}%
\pgfsetstrokecolor{currentstroke}%
\pgfsetdash{}{0pt}%
\pgfpathmoveto{\pgfqpoint{0.384371in}{1.180746in}}%
\pgfpathlineto{\pgfqpoint{0.386711in}{1.181364in}}%
\pgfpathlineto{\pgfqpoint{0.397640in}{1.186818in}}%
\pgfpathlineto{\pgfqpoint{0.401251in}{1.194290in}}%
\pgfpathlineto{\pgfqpoint{0.402757in}{1.207217in}}%
\pgfpathlineto{\pgfqpoint{0.401156in}{1.220143in}}%
\pgfpathlineto{\pgfqpoint{0.397640in}{1.227302in}}%
\pgfpathlineto{\pgfqpoint{0.385865in}{1.233070in}}%
\pgfpathlineto{\pgfqpoint{0.384371in}{1.233460in}}%
\pgfpathlineto{\pgfqpoint{0.382945in}{1.233070in}}%
\pgfpathlineto{\pgfqpoint{0.371102in}{1.227321in}}%
\pgfpathlineto{\pgfqpoint{0.367442in}{1.220143in}}%
\pgfpathlineto{\pgfqpoint{0.365748in}{1.207217in}}%
\pgfpathlineto{\pgfqpoint{0.367348in}{1.194290in}}%
\pgfpathlineto{\pgfqpoint{0.371102in}{1.186813in}}%
\pgfpathlineto{\pgfqpoint{0.382136in}{1.181364in}}%
\pgfpathclose%
\pgfpathmoveto{\pgfqpoint{0.380652in}{1.194290in}}%
\pgfpathlineto{\pgfqpoint{0.374221in}{1.207217in}}%
\pgfpathlineto{\pgfqpoint{0.380918in}{1.220143in}}%
\pgfpathlineto{\pgfqpoint{0.384371in}{1.222186in}}%
\pgfpathlineto{\pgfqpoint{0.387690in}{1.220143in}}%
\pgfpathlineto{\pgfqpoint{0.393964in}{1.207217in}}%
\pgfpathlineto{\pgfqpoint{0.387946in}{1.194290in}}%
\pgfpathlineto{\pgfqpoint{0.384371in}{1.192061in}}%
\pgfpathclose%
\pgfusepath{fill}%
\end{pgfscope}%
\begin{pgfscope}%
\pgfpathrectangle{\pgfqpoint{0.211875in}{0.211875in}}{\pgfqpoint{1.313625in}{1.279725in}}%
\pgfusepath{clip}%
\pgfsetbuttcap%
\pgfsetroundjoin%
\definecolor{currentfill}{rgb}{0.961115,0.566634,0.405693}%
\pgfsetfillcolor{currentfill}%
\pgfsetlinewidth{0.000000pt}%
\definecolor{currentstroke}{rgb}{0.000000,0.000000,0.000000}%
\pgfsetstrokecolor{currentstroke}%
\pgfsetdash{}{0pt}%
\pgfpathmoveto{\pgfqpoint{0.490523in}{1.180119in}}%
\pgfpathlineto{\pgfqpoint{0.503792in}{1.177505in}}%
\pgfpathlineto{\pgfqpoint{0.515566in}{1.181364in}}%
\pgfpathlineto{\pgfqpoint{0.517061in}{1.182361in}}%
\pgfpathlineto{\pgfqpoint{0.522160in}{1.194290in}}%
\pgfpathlineto{\pgfqpoint{0.523256in}{1.207217in}}%
\pgfpathlineto{\pgfqpoint{0.522072in}{1.220143in}}%
\pgfpathlineto{\pgfqpoint{0.517061in}{1.231668in}}%
\pgfpathlineto{\pgfqpoint{0.514918in}{1.233070in}}%
\pgfpathlineto{\pgfqpoint{0.503792in}{1.236675in}}%
\pgfpathlineto{\pgfqpoint{0.490523in}{1.234045in}}%
\pgfpathlineto{\pgfqpoint{0.489282in}{1.233070in}}%
\pgfpathlineto{\pgfqpoint{0.482354in}{1.220143in}}%
\pgfpathlineto{\pgfqpoint{0.480946in}{1.207217in}}%
\pgfpathlineto{\pgfqpoint{0.482259in}{1.194290in}}%
\pgfpathlineto{\pgfqpoint{0.488958in}{1.181364in}}%
\pgfpathclose%
\pgfpathmoveto{\pgfqpoint{0.490121in}{1.194290in}}%
\pgfpathlineto{\pgfqpoint{0.487871in}{1.207217in}}%
\pgfpathlineto{\pgfqpoint{0.490226in}{1.220143in}}%
\pgfpathlineto{\pgfqpoint{0.490523in}{1.220661in}}%
\pgfpathlineto{\pgfqpoint{0.503792in}{1.226587in}}%
\pgfpathlineto{\pgfqpoint{0.512339in}{1.220143in}}%
\pgfpathlineto{\pgfqpoint{0.516734in}{1.207217in}}%
\pgfpathlineto{\pgfqpoint{0.512541in}{1.194290in}}%
\pgfpathlineto{\pgfqpoint{0.503792in}{1.187607in}}%
\pgfpathlineto{\pgfqpoint{0.490523in}{1.193578in}}%
\pgfpathclose%
\pgfusepath{fill}%
\end{pgfscope}%
\begin{pgfscope}%
\pgfpathrectangle{\pgfqpoint{0.211875in}{0.211875in}}{\pgfqpoint{1.313625in}{1.279725in}}%
\pgfusepath{clip}%
\pgfsetbuttcap%
\pgfsetroundjoin%
\definecolor{currentfill}{rgb}{0.961115,0.566634,0.405693}%
\pgfsetfillcolor{currentfill}%
\pgfsetlinewidth{0.000000pt}%
\definecolor{currentstroke}{rgb}{0.000000,0.000000,0.000000}%
\pgfsetstrokecolor{currentstroke}%
\pgfsetdash{}{0pt}%
\pgfpathmoveto{\pgfqpoint{0.609943in}{1.176230in}}%
\pgfpathlineto{\pgfqpoint{0.623212in}{1.174847in}}%
\pgfpathlineto{\pgfqpoint{0.636481in}{1.179292in}}%
\pgfpathlineto{\pgfqpoint{0.638264in}{1.181364in}}%
\pgfpathlineto{\pgfqpoint{0.642426in}{1.194290in}}%
\pgfpathlineto{\pgfqpoint{0.643211in}{1.207217in}}%
\pgfpathlineto{\pgfqpoint{0.642345in}{1.220143in}}%
\pgfpathlineto{\pgfqpoint{0.637985in}{1.233070in}}%
\pgfpathlineto{\pgfqpoint{0.636481in}{1.234791in}}%
\pgfpathlineto{\pgfqpoint{0.623212in}{1.239312in}}%
\pgfpathlineto{\pgfqpoint{0.609943in}{1.237912in}}%
\pgfpathlineto{\pgfqpoint{0.603015in}{1.233070in}}%
\pgfpathlineto{\pgfqpoint{0.597325in}{1.220143in}}%
\pgfpathlineto{\pgfqpoint{0.596674in}{1.213299in}}%
\pgfpathlineto{\pgfqpoint{0.596285in}{1.207217in}}%
\pgfpathlineto{\pgfqpoint{0.596674in}{1.200607in}}%
\pgfpathlineto{\pgfqpoint{0.597227in}{1.194290in}}%
\pgfpathlineto{\pgfqpoint{0.602684in}{1.181364in}}%
\pgfpathclose%
\pgfpathmoveto{\pgfqpoint{0.605626in}{1.194290in}}%
\pgfpathlineto{\pgfqpoint{0.603561in}{1.207217in}}%
\pgfpathlineto{\pgfqpoint{0.605735in}{1.220143in}}%
\pgfpathlineto{\pgfqpoint{0.609943in}{1.226698in}}%
\pgfpathlineto{\pgfqpoint{0.623212in}{1.230108in}}%
\pgfpathlineto{\pgfqpoint{0.634349in}{1.220143in}}%
\pgfpathlineto{\pgfqpoint{0.636481in}{1.211998in}}%
\pgfpathlineto{\pgfqpoint{0.637072in}{1.207217in}}%
\pgfpathlineto{\pgfqpoint{0.636481in}{1.202190in}}%
\pgfpathlineto{\pgfqpoint{0.634516in}{1.194290in}}%
\pgfpathlineto{\pgfqpoint{0.623212in}{1.184043in}}%
\pgfpathlineto{\pgfqpoint{0.609943in}{1.187470in}}%
\pgfpathclose%
\pgfusepath{fill}%
\end{pgfscope}%
\begin{pgfscope}%
\pgfpathrectangle{\pgfqpoint{0.211875in}{0.211875in}}{\pgfqpoint{1.313625in}{1.279725in}}%
\pgfusepath{clip}%
\pgfsetbuttcap%
\pgfsetroundjoin%
\definecolor{currentfill}{rgb}{0.961115,0.566634,0.405693}%
\pgfsetfillcolor{currentfill}%
\pgfsetlinewidth{0.000000pt}%
\definecolor{currentstroke}{rgb}{0.000000,0.000000,0.000000}%
\pgfsetstrokecolor{currentstroke}%
\pgfsetdash{}{0pt}%
\pgfpathmoveto{\pgfqpoint{0.729364in}{1.173357in}}%
\pgfpathlineto{\pgfqpoint{0.742633in}{1.172738in}}%
\pgfpathlineto{\pgfqpoint{0.755902in}{1.176971in}}%
\pgfpathlineto{\pgfqpoint{0.759186in}{1.181364in}}%
\pgfpathlineto{\pgfqpoint{0.762140in}{1.194290in}}%
\pgfpathlineto{\pgfqpoint{0.762694in}{1.207217in}}%
\pgfpathlineto{\pgfqpoint{0.762063in}{1.220143in}}%
\pgfpathlineto{\pgfqpoint{0.758924in}{1.233070in}}%
\pgfpathlineto{\pgfqpoint{0.755902in}{1.237047in}}%
\pgfpathlineto{\pgfqpoint{0.742633in}{1.241400in}}%
\pgfpathlineto{\pgfqpoint{0.729364in}{1.240771in}}%
\pgfpathlineto{\pgfqpoint{0.716931in}{1.233070in}}%
\pgfpathlineto{\pgfqpoint{0.716095in}{1.231517in}}%
\pgfpathlineto{\pgfqpoint{0.713377in}{1.220143in}}%
\pgfpathlineto{\pgfqpoint{0.712700in}{1.207217in}}%
\pgfpathlineto{\pgfqpoint{0.713303in}{1.194290in}}%
\pgfpathlineto{\pgfqpoint{0.716095in}{1.182301in}}%
\pgfpathlineto{\pgfqpoint{0.716585in}{1.181364in}}%
\pgfpathclose%
\pgfpathmoveto{\pgfqpoint{0.742321in}{1.181364in}}%
\pgfpathlineto{\pgfqpoint{0.729364in}{1.182922in}}%
\pgfpathlineto{\pgfqpoint{0.721259in}{1.194290in}}%
\pgfpathlineto{\pgfqpoint{0.719324in}{1.207217in}}%
\pgfpathlineto{\pgfqpoint{0.721374in}{1.220143in}}%
\pgfpathlineto{\pgfqpoint{0.729364in}{1.231197in}}%
\pgfpathlineto{\pgfqpoint{0.742633in}{1.232785in}}%
\pgfpathlineto{\pgfqpoint{0.754791in}{1.220143in}}%
\pgfpathlineto{\pgfqpoint{0.755902in}{1.214754in}}%
\pgfpathlineto{\pgfqpoint{0.756716in}{1.207217in}}%
\pgfpathlineto{\pgfqpoint{0.755902in}{1.199255in}}%
\pgfpathlineto{\pgfqpoint{0.754934in}{1.194290in}}%
\pgfpathlineto{\pgfqpoint{0.742688in}{1.181364in}}%
\pgfpathlineto{\pgfqpoint{0.742633in}{1.181338in}}%
\pgfpathclose%
\pgfusepath{fill}%
\end{pgfscope}%
\begin{pgfscope}%
\pgfpathrectangle{\pgfqpoint{0.211875in}{0.211875in}}{\pgfqpoint{1.313625in}{1.279725in}}%
\pgfusepath{clip}%
\pgfsetbuttcap%
\pgfsetroundjoin%
\definecolor{currentfill}{rgb}{0.961115,0.566634,0.405693}%
\pgfsetfillcolor{currentfill}%
\pgfsetlinewidth{0.000000pt}%
\definecolor{currentstroke}{rgb}{0.000000,0.000000,0.000000}%
\pgfsetstrokecolor{currentstroke}%
\pgfsetdash{}{0pt}%
\pgfpathmoveto{\pgfqpoint{0.835515in}{1.176153in}}%
\pgfpathlineto{\pgfqpoint{0.848784in}{1.171329in}}%
\pgfpathlineto{\pgfqpoint{0.862053in}{1.171169in}}%
\pgfpathlineto{\pgfqpoint{0.875322in}{1.175239in}}%
\pgfpathlineto{\pgfqpoint{0.879245in}{1.181364in}}%
\pgfpathlineto{\pgfqpoint{0.881360in}{1.194290in}}%
\pgfpathlineto{\pgfqpoint{0.881751in}{1.207217in}}%
\pgfpathlineto{\pgfqpoint{0.881285in}{1.220143in}}%
\pgfpathlineto{\pgfqpoint{0.878997in}{1.233070in}}%
\pgfpathlineto{\pgfqpoint{0.875322in}{1.238705in}}%
\pgfpathlineto{\pgfqpoint{0.862053in}{1.242950in}}%
\pgfpathlineto{\pgfqpoint{0.848784in}{1.242791in}}%
\pgfpathlineto{\pgfqpoint{0.835515in}{1.237805in}}%
\pgfpathlineto{\pgfqpoint{0.832508in}{1.233070in}}%
\pgfpathlineto{\pgfqpoint{0.829994in}{1.220143in}}%
\pgfpathlineto{\pgfqpoint{0.829478in}{1.207217in}}%
\pgfpathlineto{\pgfqpoint{0.829921in}{1.194290in}}%
\pgfpathlineto{\pgfqpoint{0.832264in}{1.181364in}}%
\pgfpathclose%
\pgfpathmoveto{\pgfqpoint{0.846590in}{1.181364in}}%
\pgfpathlineto{\pgfqpoint{0.836982in}{1.194290in}}%
\pgfpathlineto{\pgfqpoint{0.835515in}{1.204129in}}%
\pgfpathlineto{\pgfqpoint{0.835248in}{1.207217in}}%
\pgfpathlineto{\pgfqpoint{0.835515in}{1.210122in}}%
\pgfpathlineto{\pgfqpoint{0.837106in}{1.220143in}}%
\pgfpathlineto{\pgfqpoint{0.847011in}{1.233070in}}%
\pgfpathlineto{\pgfqpoint{0.848784in}{1.234037in}}%
\pgfpathlineto{\pgfqpoint{0.862053in}{1.234190in}}%
\pgfpathlineto{\pgfqpoint{0.864196in}{1.233070in}}%
\pgfpathlineto{\pgfqpoint{0.874210in}{1.220143in}}%
\pgfpathlineto{\pgfqpoint{0.875322in}{1.213298in}}%
\pgfpathlineto{\pgfqpoint{0.875889in}{1.207217in}}%
\pgfpathlineto{\pgfqpoint{0.875322in}{1.200752in}}%
\pgfpathlineto{\pgfqpoint{0.874336in}{1.194290in}}%
\pgfpathlineto{\pgfqpoint{0.864635in}{1.181364in}}%
\pgfpathlineto{\pgfqpoint{0.862053in}{1.179998in}}%
\pgfpathlineto{\pgfqpoint{0.848784in}{1.180152in}}%
\pgfpathclose%
\pgfusepath{fill}%
\end{pgfscope}%
\begin{pgfscope}%
\pgfpathrectangle{\pgfqpoint{0.211875in}{0.211875in}}{\pgfqpoint{1.313625in}{1.279725in}}%
\pgfusepath{clip}%
\pgfsetbuttcap%
\pgfsetroundjoin%
\definecolor{currentfill}{rgb}{0.961115,0.566634,0.405693}%
\pgfsetfillcolor{currentfill}%
\pgfsetlinewidth{0.000000pt}%
\definecolor{currentstroke}{rgb}{0.000000,0.000000,0.000000}%
\pgfsetstrokecolor{currentstroke}%
\pgfsetdash{}{0pt}%
\pgfpathmoveto{\pgfqpoint{0.954936in}{1.172814in}}%
\pgfpathlineto{\pgfqpoint{0.968205in}{1.170030in}}%
\pgfpathlineto{\pgfqpoint{0.981473in}{1.170152in}}%
\pgfpathlineto{\pgfqpoint{0.994742in}{1.174334in}}%
\pgfpathlineto{\pgfqpoint{0.998521in}{1.181364in}}%
\pgfpathlineto{\pgfqpoint{1.000118in}{1.194290in}}%
\pgfpathlineto{\pgfqpoint{1.000410in}{1.207217in}}%
\pgfpathlineto{\pgfqpoint{1.000046in}{1.220143in}}%
\pgfpathlineto{\pgfqpoint{0.998281in}{1.233070in}}%
\pgfpathlineto{\pgfqpoint{0.994742in}{1.239521in}}%
\pgfpathlineto{\pgfqpoint{0.981473in}{1.243951in}}%
\pgfpathlineto{\pgfqpoint{0.968205in}{1.244087in}}%
\pgfpathlineto{\pgfqpoint{0.954936in}{1.241153in}}%
\pgfpathlineto{\pgfqpoint{0.948961in}{1.233070in}}%
\pgfpathlineto{\pgfqpoint{0.946997in}{1.220143in}}%
\pgfpathlineto{\pgfqpoint{0.946590in}{1.207217in}}%
\pgfpathlineto{\pgfqpoint{0.946923in}{1.194290in}}%
\pgfpathlineto{\pgfqpoint{0.948717in}{1.181364in}}%
\pgfpathclose%
\pgfpathmoveto{\pgfqpoint{0.962521in}{1.181364in}}%
\pgfpathlineto{\pgfqpoint{0.954936in}{1.190196in}}%
\pgfpathlineto{\pgfqpoint{0.953617in}{1.194290in}}%
\pgfpathlineto{\pgfqpoint{0.952472in}{1.207217in}}%
\pgfpathlineto{\pgfqpoint{0.953698in}{1.220143in}}%
\pgfpathlineto{\pgfqpoint{0.954936in}{1.223907in}}%
\pgfpathlineto{\pgfqpoint{0.962984in}{1.233070in}}%
\pgfpathlineto{\pgfqpoint{0.968205in}{1.235557in}}%
\pgfpathlineto{\pgfqpoint{0.981473in}{1.234894in}}%
\pgfpathlineto{\pgfqpoint{0.984535in}{1.233070in}}%
\pgfpathlineto{\pgfqpoint{0.992909in}{1.220143in}}%
\pgfpathlineto{\pgfqpoint{0.994572in}{1.207217in}}%
\pgfpathlineto{\pgfqpoint{0.993023in}{1.194290in}}%
\pgfpathlineto{\pgfqpoint{0.984927in}{1.181364in}}%
\pgfpathlineto{\pgfqpoint{0.981473in}{1.179282in}}%
\pgfpathlineto{\pgfqpoint{0.968205in}{1.178626in}}%
\pgfpathclose%
\pgfusepath{fill}%
\end{pgfscope}%
\begin{pgfscope}%
\pgfpathrectangle{\pgfqpoint{0.211875in}{0.211875in}}{\pgfqpoint{1.313625in}{1.279725in}}%
\pgfusepath{clip}%
\pgfsetbuttcap%
\pgfsetroundjoin%
\definecolor{currentfill}{rgb}{0.961115,0.566634,0.405693}%
\pgfsetfillcolor{currentfill}%
\pgfsetlinewidth{0.000000pt}%
\definecolor{currentstroke}{rgb}{0.000000,0.000000,0.000000}%
\pgfsetstrokecolor{currentstroke}%
\pgfsetdash{}{0pt}%
\pgfpathmoveto{\pgfqpoint{1.074356in}{1.171129in}}%
\pgfpathlineto{\pgfqpoint{1.087625in}{1.169380in}}%
\pgfpathlineto{\pgfqpoint{1.100894in}{1.169722in}}%
\pgfpathlineto{\pgfqpoint{1.114163in}{1.174744in}}%
\pgfpathlineto{\pgfqpoint{1.117055in}{1.181364in}}%
\pgfpathlineto{\pgfqpoint{1.118430in}{1.194290in}}%
\pgfpathlineto{\pgfqpoint{1.118680in}{1.207217in}}%
\pgfpathlineto{\pgfqpoint{1.118359in}{1.220143in}}%
\pgfpathlineto{\pgfqpoint{1.116819in}{1.233070in}}%
\pgfpathlineto{\pgfqpoint{1.114163in}{1.239006in}}%
\pgfpathlineto{\pgfqpoint{1.100894in}{1.244365in}}%
\pgfpathlineto{\pgfqpoint{1.087625in}{1.244737in}}%
\pgfpathlineto{\pgfqpoint{1.074356in}{1.242861in}}%
\pgfpathlineto{\pgfqpoint{1.066046in}{1.233070in}}%
\pgfpathlineto{\pgfqpoint{1.064369in}{1.220143in}}%
\pgfpathlineto{\pgfqpoint{1.064019in}{1.207217in}}%
\pgfpathlineto{\pgfqpoint{1.064294in}{1.194290in}}%
\pgfpathlineto{\pgfqpoint{1.065798in}{1.181364in}}%
\pgfpathclose%
\pgfpathmoveto{\pgfqpoint{1.079063in}{1.181364in}}%
\pgfpathlineto{\pgfqpoint{1.074356in}{1.185667in}}%
\pgfpathlineto{\pgfqpoint{1.071169in}{1.194290in}}%
\pgfpathlineto{\pgfqpoint{1.070058in}{1.207217in}}%
\pgfpathlineto{\pgfqpoint{1.071252in}{1.220143in}}%
\pgfpathlineto{\pgfqpoint{1.074356in}{1.228387in}}%
\pgfpathlineto{\pgfqpoint{1.079585in}{1.233070in}}%
\pgfpathlineto{\pgfqpoint{1.087625in}{1.236367in}}%
\pgfpathlineto{\pgfqpoint{1.100894in}{1.234928in}}%
\pgfpathlineto{\pgfqpoint{1.103649in}{1.233070in}}%
\pgfpathlineto{\pgfqpoint{1.111063in}{1.220143in}}%
\pgfpathlineto{\pgfqpoint{1.112543in}{1.207217in}}%
\pgfpathlineto{\pgfqpoint{1.111168in}{1.194290in}}%
\pgfpathlineto{\pgfqpoint{1.104007in}{1.181364in}}%
\pgfpathlineto{\pgfqpoint{1.100894in}{1.179239in}}%
\pgfpathlineto{\pgfqpoint{1.087625in}{1.177814in}}%
\pgfpathclose%
\pgfusepath{fill}%
\end{pgfscope}%
\begin{pgfscope}%
\pgfpathrectangle{\pgfqpoint{0.211875in}{0.211875in}}{\pgfqpoint{1.313625in}{1.279725in}}%
\pgfusepath{clip}%
\pgfsetbuttcap%
\pgfsetroundjoin%
\definecolor{currentfill}{rgb}{0.961115,0.566634,0.405693}%
\pgfsetfillcolor{currentfill}%
\pgfsetlinewidth{0.000000pt}%
\definecolor{currentstroke}{rgb}{0.000000,0.000000,0.000000}%
\pgfsetstrokecolor{currentstroke}%
\pgfsetdash{}{0pt}%
\pgfpathmoveto{\pgfqpoint{1.193777in}{1.170602in}}%
\pgfpathlineto{\pgfqpoint{1.207045in}{1.169331in}}%
\pgfpathlineto{\pgfqpoint{1.220314in}{1.169949in}}%
\pgfpathlineto{\pgfqpoint{1.233583in}{1.177583in}}%
\pgfpathlineto{\pgfqpoint{1.234859in}{1.181364in}}%
\pgfpathlineto{\pgfqpoint{1.236293in}{1.194290in}}%
\pgfpathlineto{\pgfqpoint{1.236555in}{1.207217in}}%
\pgfpathlineto{\pgfqpoint{1.236221in}{1.220143in}}%
\pgfpathlineto{\pgfqpoint{1.234623in}{1.233070in}}%
\pgfpathlineto{\pgfqpoint{1.233583in}{1.236062in}}%
\pgfpathlineto{\pgfqpoint{1.220314in}{1.244126in}}%
\pgfpathlineto{\pgfqpoint{1.207045in}{1.244789in}}%
\pgfpathlineto{\pgfqpoint{1.193777in}{1.243417in}}%
\pgfpathlineto{\pgfqpoint{1.183782in}{1.233070in}}%
\pgfpathlineto{\pgfqpoint{1.182110in}{1.220143in}}%
\pgfpathlineto{\pgfqpoint{1.181762in}{1.207217in}}%
\pgfpathlineto{\pgfqpoint{1.182033in}{1.194290in}}%
\pgfpathlineto{\pgfqpoint{1.183526in}{1.181364in}}%
\pgfpathclose%
\pgfpathmoveto{\pgfqpoint{1.196455in}{1.181364in}}%
\pgfpathlineto{\pgfqpoint{1.193777in}{1.183267in}}%
\pgfpathlineto{\pgfqpoint{1.189146in}{1.194290in}}%
\pgfpathlineto{\pgfqpoint{1.188007in}{1.207217in}}%
\pgfpathlineto{\pgfqpoint{1.189232in}{1.220143in}}%
\pgfpathlineto{\pgfqpoint{1.193777in}{1.230776in}}%
\pgfpathlineto{\pgfqpoint{1.197067in}{1.233070in}}%
\pgfpathlineto{\pgfqpoint{1.207045in}{1.236521in}}%
\pgfpathlineto{\pgfqpoint{1.220314in}{1.234207in}}%
\pgfpathlineto{\pgfqpoint{1.221811in}{1.233070in}}%
\pgfpathlineto{\pgfqpoint{1.228777in}{1.220143in}}%
\pgfpathlineto{\pgfqpoint{1.230171in}{1.207217in}}%
\pgfpathlineto{\pgfqpoint{1.228874in}{1.194290in}}%
\pgfpathlineto{\pgfqpoint{1.222145in}{1.181364in}}%
\pgfpathlineto{\pgfqpoint{1.220314in}{1.179955in}}%
\pgfpathlineto{\pgfqpoint{1.207045in}{1.177661in}}%
\pgfpathclose%
\pgfusepath{fill}%
\end{pgfscope}%
\begin{pgfscope}%
\pgfpathrectangle{\pgfqpoint{0.211875in}{0.211875in}}{\pgfqpoint{1.313625in}{1.279725in}}%
\pgfusepath{clip}%
\pgfsetbuttcap%
\pgfsetroundjoin%
\definecolor{currentfill}{rgb}{0.961115,0.566634,0.405693}%
\pgfsetfillcolor{currentfill}%
\pgfsetlinewidth{0.000000pt}%
\definecolor{currentstroke}{rgb}{0.000000,0.000000,0.000000}%
\pgfsetstrokecolor{currentstroke}%
\pgfsetdash{}{0pt}%
\pgfpathmoveto{\pgfqpoint{1.313197in}{1.170938in}}%
\pgfpathlineto{\pgfqpoint{1.326466in}{1.169860in}}%
\pgfpathlineto{\pgfqpoint{1.339735in}{1.170943in}}%
\pgfpathlineto{\pgfqpoint{1.351754in}{1.181364in}}%
\pgfpathlineto{\pgfqpoint{1.353004in}{1.187787in}}%
\pgfpathlineto{\pgfqpoint{1.353685in}{1.194290in}}%
\pgfpathlineto{\pgfqpoint{1.354013in}{1.207217in}}%
\pgfpathlineto{\pgfqpoint{1.353612in}{1.220143in}}%
\pgfpathlineto{\pgfqpoint{1.353004in}{1.225651in}}%
\pgfpathlineto{\pgfqpoint{1.351478in}{1.233070in}}%
\pgfpathlineto{\pgfqpoint{1.339735in}{1.243123in}}%
\pgfpathlineto{\pgfqpoint{1.326466in}{1.244265in}}%
\pgfpathlineto{\pgfqpoint{1.313197in}{1.243111in}}%
\pgfpathlineto{\pgfqpoint{1.302223in}{1.233070in}}%
\pgfpathlineto{\pgfqpoint{1.300237in}{1.220143in}}%
\pgfpathlineto{\pgfqpoint{1.299928in}{1.210710in}}%
\pgfpathlineto{\pgfqpoint{1.299837in}{1.207217in}}%
\pgfpathlineto{\pgfqpoint{1.299928in}{1.202897in}}%
\pgfpathlineto{\pgfqpoint{1.300156in}{1.194290in}}%
\pgfpathlineto{\pgfqpoint{1.301953in}{1.181364in}}%
\pgfpathclose%
\pgfpathmoveto{\pgfqpoint{1.315195in}{1.181364in}}%
\pgfpathlineto{\pgfqpoint{1.313197in}{1.182442in}}%
\pgfpathlineto{\pgfqpoint{1.307575in}{1.194290in}}%
\pgfpathlineto{\pgfqpoint{1.306338in}{1.207217in}}%
\pgfpathlineto{\pgfqpoint{1.307665in}{1.220143in}}%
\pgfpathlineto{\pgfqpoint{1.313197in}{1.231614in}}%
\pgfpathlineto{\pgfqpoint{1.315945in}{1.233070in}}%
\pgfpathlineto{\pgfqpoint{1.326466in}{1.236044in}}%
\pgfpathlineto{\pgfqpoint{1.338279in}{1.233070in}}%
\pgfpathlineto{\pgfqpoint{1.339735in}{1.232418in}}%
\pgfpathlineto{\pgfqpoint{1.346111in}{1.220143in}}%
\pgfpathlineto{\pgfqpoint{1.347500in}{1.207217in}}%
\pgfpathlineto{\pgfqpoint{1.346203in}{1.194290in}}%
\pgfpathlineto{\pgfqpoint{1.339735in}{1.181643in}}%
\pgfpathlineto{\pgfqpoint{1.339123in}{1.181364in}}%
\pgfpathlineto{\pgfqpoint{1.326466in}{1.178143in}}%
\pgfpathclose%
\pgfusepath{fill}%
\end{pgfscope}%
\begin{pgfscope}%
\pgfpathrectangle{\pgfqpoint{0.211875in}{0.211875in}}{\pgfqpoint{1.313625in}{1.279725in}}%
\pgfusepath{clip}%
\pgfsetbuttcap%
\pgfsetroundjoin%
\definecolor{currentfill}{rgb}{0.961115,0.566634,0.405693}%
\pgfsetfillcolor{currentfill}%
\pgfsetlinewidth{0.000000pt}%
\definecolor{currentstroke}{rgb}{0.000000,0.000000,0.000000}%
\pgfsetstrokecolor{currentstroke}%
\pgfsetdash{}{0pt}%
\pgfpathmoveto{\pgfqpoint{1.432617in}{1.171962in}}%
\pgfpathlineto{\pgfqpoint{1.445886in}{1.170968in}}%
\pgfpathlineto{\pgfqpoint{1.459155in}{1.172875in}}%
\pgfpathlineto{\pgfqpoint{1.467835in}{1.181364in}}%
\pgfpathlineto{\pgfqpoint{1.470415in}{1.194290in}}%
\pgfpathlineto{\pgfqpoint{1.470903in}{1.207217in}}%
\pgfpathlineto{\pgfqpoint{1.470334in}{1.220143in}}%
\pgfpathlineto{\pgfqpoint{1.467567in}{1.233070in}}%
\pgfpathlineto{\pgfqpoint{1.459155in}{1.241188in}}%
\pgfpathlineto{\pgfqpoint{1.445886in}{1.243165in}}%
\pgfpathlineto{\pgfqpoint{1.432617in}{1.242117in}}%
\pgfpathlineto{\pgfqpoint{1.421460in}{1.233070in}}%
\pgfpathlineto{\pgfqpoint{1.419348in}{1.224063in}}%
\pgfpathlineto{\pgfqpoint{1.418868in}{1.220143in}}%
\pgfpathlineto{\pgfqpoint{1.418403in}{1.207217in}}%
\pgfpathlineto{\pgfqpoint{1.418794in}{1.194290in}}%
\pgfpathlineto{\pgfqpoint{1.419348in}{1.189538in}}%
\pgfpathlineto{\pgfqpoint{1.421169in}{1.181364in}}%
\pgfpathclose%
\pgfpathmoveto{\pgfqpoint{1.436407in}{1.181364in}}%
\pgfpathlineto{\pgfqpoint{1.432617in}{1.182848in}}%
\pgfpathlineto{\pgfqpoint{1.426505in}{1.194290in}}%
\pgfpathlineto{\pgfqpoint{1.425088in}{1.207217in}}%
\pgfpathlineto{\pgfqpoint{1.426601in}{1.220143in}}%
\pgfpathlineto{\pgfqpoint{1.432617in}{1.231234in}}%
\pgfpathlineto{\pgfqpoint{1.437389in}{1.233070in}}%
\pgfpathlineto{\pgfqpoint{1.445886in}{1.234937in}}%
\pgfpathlineto{\pgfqpoint{1.451822in}{1.233070in}}%
\pgfpathlineto{\pgfqpoint{1.459155in}{1.228675in}}%
\pgfpathlineto{\pgfqpoint{1.463096in}{1.220143in}}%
\pgfpathlineto{\pgfqpoint{1.464546in}{1.207217in}}%
\pgfpathlineto{\pgfqpoint{1.463185in}{1.194290in}}%
\pgfpathlineto{\pgfqpoint{1.459155in}{1.185421in}}%
\pgfpathlineto{\pgfqpoint{1.452510in}{1.181364in}}%
\pgfpathlineto{\pgfqpoint{1.445886in}{1.179257in}}%
\pgfpathclose%
\pgfusepath{fill}%
\end{pgfscope}%
\begin{pgfscope}%
\pgfpathrectangle{\pgfqpoint{0.211875in}{0.211875in}}{\pgfqpoint{1.313625in}{1.279725in}}%
\pgfusepath{clip}%
\pgfsetbuttcap%
\pgfsetroundjoin%
\definecolor{currentfill}{rgb}{0.961115,0.566634,0.405693}%
\pgfsetfillcolor{currentfill}%
\pgfsetlinewidth{0.000000pt}%
\definecolor{currentstroke}{rgb}{0.000000,0.000000,0.000000}%
\pgfsetstrokecolor{currentstroke}%
\pgfsetdash{}{0pt}%
\pgfpathmoveto{\pgfqpoint{0.264951in}{1.185911in}}%
\pgfpathlineto{\pgfqpoint{0.278220in}{1.191809in}}%
\pgfpathlineto{\pgfqpoint{0.279571in}{1.194290in}}%
\pgfpathlineto{\pgfqpoint{0.281607in}{1.207217in}}%
\pgfpathlineto{\pgfqpoint{0.279467in}{1.220143in}}%
\pgfpathlineto{\pgfqpoint{0.278220in}{1.222399in}}%
\pgfpathlineto{\pgfqpoint{0.264951in}{1.228265in}}%
\pgfpathlineto{\pgfqpoint{0.253819in}{1.220143in}}%
\pgfpathlineto{\pgfqpoint{0.251682in}{1.214345in}}%
\pgfpathlineto{\pgfqpoint{0.250634in}{1.207217in}}%
\pgfpathlineto{\pgfqpoint{0.251682in}{1.199769in}}%
\pgfpathlineto{\pgfqpoint{0.253613in}{1.194290in}}%
\pgfpathclose%
\pgfpathmoveto{\pgfqpoint{0.263213in}{1.207217in}}%
\pgfpathlineto{\pgfqpoint{0.264951in}{1.211123in}}%
\pgfpathlineto{\pgfqpoint{0.267553in}{1.207217in}}%
\pgfpathlineto{\pgfqpoint{0.264951in}{1.203173in}}%
\pgfpathclose%
\pgfusepath{fill}%
\end{pgfscope}%
\begin{pgfscope}%
\pgfpathrectangle{\pgfqpoint{0.211875in}{0.211875in}}{\pgfqpoint{1.313625in}{1.279725in}}%
\pgfusepath{clip}%
\pgfsetbuttcap%
\pgfsetroundjoin%
\definecolor{currentfill}{rgb}{0.961115,0.566634,0.405693}%
\pgfsetfillcolor{currentfill}%
\pgfsetlinewidth{0.000000pt}%
\definecolor{currentstroke}{rgb}{0.000000,0.000000,0.000000}%
\pgfsetstrokecolor{currentstroke}%
\pgfsetdash{}{0pt}%
\pgfpathmoveto{\pgfqpoint{0.556867in}{1.257681in}}%
\pgfpathlineto{\pgfqpoint{0.570136in}{1.258539in}}%
\pgfpathlineto{\pgfqpoint{0.570937in}{1.258923in}}%
\pgfpathlineto{\pgfqpoint{0.581339in}{1.271849in}}%
\pgfpathlineto{\pgfqpoint{0.583379in}{1.284776in}}%
\pgfpathlineto{\pgfqpoint{0.582769in}{1.297702in}}%
\pgfpathlineto{\pgfqpoint{0.578392in}{1.310629in}}%
\pgfpathlineto{\pgfqpoint{0.570136in}{1.317456in}}%
\pgfpathlineto{\pgfqpoint{0.556867in}{1.318438in}}%
\pgfpathlineto{\pgfqpoint{0.543598in}{1.311350in}}%
\pgfpathlineto{\pgfqpoint{0.543135in}{1.310629in}}%
\pgfpathlineto{\pgfqpoint{0.539520in}{1.297702in}}%
\pgfpathlineto{\pgfqpoint{0.539011in}{1.284776in}}%
\pgfpathlineto{\pgfqpoint{0.540734in}{1.271849in}}%
\pgfpathlineto{\pgfqpoint{0.543598in}{1.265411in}}%
\pgfpathlineto{\pgfqpoint{0.553159in}{1.258923in}}%
\pgfpathclose%
\pgfpathmoveto{\pgfqpoint{0.551621in}{1.271849in}}%
\pgfpathlineto{\pgfqpoint{0.546106in}{1.284776in}}%
\pgfpathlineto{\pgfqpoint{0.547619in}{1.297702in}}%
\pgfpathlineto{\pgfqpoint{0.556867in}{1.308996in}}%
\pgfpathlineto{\pgfqpoint{0.570136in}{1.305815in}}%
\pgfpathlineto{\pgfqpoint{0.574486in}{1.297702in}}%
\pgfpathlineto{\pgfqpoint{0.575558in}{1.284776in}}%
\pgfpathlineto{\pgfqpoint{0.571631in}{1.271849in}}%
\pgfpathlineto{\pgfqpoint{0.570136in}{1.270060in}}%
\pgfpathlineto{\pgfqpoint{0.556867in}{1.267709in}}%
\pgfpathclose%
\pgfusepath{fill}%
\end{pgfscope}%
\begin{pgfscope}%
\pgfpathrectangle{\pgfqpoint{0.211875in}{0.211875in}}{\pgfqpoint{1.313625in}{1.279725in}}%
\pgfusepath{clip}%
\pgfsetbuttcap%
\pgfsetroundjoin%
\definecolor{currentfill}{rgb}{0.961115,0.566634,0.405693}%
\pgfsetfillcolor{currentfill}%
\pgfsetlinewidth{0.000000pt}%
\definecolor{currentstroke}{rgb}{0.000000,0.000000,0.000000}%
\pgfsetstrokecolor{currentstroke}%
\pgfsetdash{}{0pt}%
\pgfpathmoveto{\pgfqpoint{0.663019in}{1.258894in}}%
\pgfpathlineto{\pgfqpoint{0.676288in}{1.255121in}}%
\pgfpathlineto{\pgfqpoint{0.689557in}{1.256248in}}%
\pgfpathlineto{\pgfqpoint{0.694476in}{1.258923in}}%
\pgfpathlineto{\pgfqpoint{0.701645in}{1.271849in}}%
\pgfpathlineto{\pgfqpoint{0.702826in}{1.282295in}}%
\pgfpathlineto{\pgfqpoint{0.703009in}{1.284776in}}%
\pgfpathlineto{\pgfqpoint{0.702826in}{1.291482in}}%
\pgfpathlineto{\pgfqpoint{0.702604in}{1.297702in}}%
\pgfpathlineto{\pgfqpoint{0.699502in}{1.310629in}}%
\pgfpathlineto{\pgfqpoint{0.689557in}{1.319909in}}%
\pgfpathlineto{\pgfqpoint{0.676288in}{1.321218in}}%
\pgfpathlineto{\pgfqpoint{0.663019in}{1.316936in}}%
\pgfpathlineto{\pgfqpoint{0.658418in}{1.310629in}}%
\pgfpathlineto{\pgfqpoint{0.655534in}{1.297702in}}%
\pgfpathlineto{\pgfqpoint{0.655116in}{1.284776in}}%
\pgfpathlineto{\pgfqpoint{0.656456in}{1.271849in}}%
\pgfpathlineto{\pgfqpoint{0.662984in}{1.258923in}}%
\pgfpathclose%
\pgfpathmoveto{\pgfqpoint{0.665118in}{1.271849in}}%
\pgfpathlineto{\pgfqpoint{0.663019in}{1.276150in}}%
\pgfpathlineto{\pgfqpoint{0.661304in}{1.284776in}}%
\pgfpathlineto{\pgfqpoint{0.662073in}{1.297702in}}%
\pgfpathlineto{\pgfqpoint{0.663019in}{1.300577in}}%
\pgfpathlineto{\pgfqpoint{0.672381in}{1.310629in}}%
\pgfpathlineto{\pgfqpoint{0.676288in}{1.312475in}}%
\pgfpathlineto{\pgfqpoint{0.685051in}{1.310629in}}%
\pgfpathlineto{\pgfqpoint{0.689557in}{1.308868in}}%
\pgfpathlineto{\pgfqpoint{0.694877in}{1.297702in}}%
\pgfpathlineto{\pgfqpoint{0.695754in}{1.284776in}}%
\pgfpathlineto{\pgfqpoint{0.692598in}{1.271849in}}%
\pgfpathlineto{\pgfqpoint{0.689557in}{1.267743in}}%
\pgfpathlineto{\pgfqpoint{0.676288in}{1.264328in}}%
\pgfpathclose%
\pgfusepath{fill}%
\end{pgfscope}%
\begin{pgfscope}%
\pgfpathrectangle{\pgfqpoint{0.211875in}{0.211875in}}{\pgfqpoint{1.313625in}{1.279725in}}%
\pgfusepath{clip}%
\pgfsetbuttcap%
\pgfsetroundjoin%
\definecolor{currentfill}{rgb}{0.961115,0.566634,0.405693}%
\pgfsetfillcolor{currentfill}%
\pgfsetlinewidth{0.000000pt}%
\definecolor{currentstroke}{rgb}{0.000000,0.000000,0.000000}%
\pgfsetstrokecolor{currentstroke}%
\pgfsetdash{}{0pt}%
\pgfpathmoveto{\pgfqpoint{0.782439in}{1.255391in}}%
\pgfpathlineto{\pgfqpoint{0.795708in}{1.253198in}}%
\pgfpathlineto{\pgfqpoint{0.808977in}{1.254443in}}%
\pgfpathlineto{\pgfqpoint{0.816255in}{1.258923in}}%
\pgfpathlineto{\pgfqpoint{0.821220in}{1.271849in}}%
\pgfpathlineto{\pgfqpoint{0.822191in}{1.284776in}}%
\pgfpathlineto{\pgfqpoint{0.821855in}{1.297702in}}%
\pgfpathlineto{\pgfqpoint{0.819633in}{1.310629in}}%
\pgfpathlineto{\pgfqpoint{0.808977in}{1.321827in}}%
\pgfpathlineto{\pgfqpoint{0.795708in}{1.323306in}}%
\pgfpathlineto{\pgfqpoint{0.782439in}{1.320762in}}%
\pgfpathlineto{\pgfqpoint{0.774094in}{1.310629in}}%
\pgfpathlineto{\pgfqpoint{0.771792in}{1.297702in}}%
\pgfpathlineto{\pgfqpoint{0.771447in}{1.284776in}}%
\pgfpathlineto{\pgfqpoint{0.772479in}{1.271849in}}%
\pgfpathlineto{\pgfqpoint{0.777574in}{1.258923in}}%
\pgfpathclose%
\pgfpathmoveto{\pgfqpoint{0.780412in}{1.271849in}}%
\pgfpathlineto{\pgfqpoint{0.777887in}{1.284776in}}%
\pgfpathlineto{\pgfqpoint{0.778600in}{1.297702in}}%
\pgfpathlineto{\pgfqpoint{0.782439in}{1.308051in}}%
\pgfpathlineto{\pgfqpoint{0.785548in}{1.310629in}}%
\pgfpathlineto{\pgfqpoint{0.795708in}{1.314618in}}%
\pgfpathlineto{\pgfqpoint{0.808977in}{1.310797in}}%
\pgfpathlineto{\pgfqpoint{0.809137in}{1.310629in}}%
\pgfpathlineto{\pgfqpoint{0.814571in}{1.297702in}}%
\pgfpathlineto{\pgfqpoint{0.815309in}{1.284776in}}%
\pgfpathlineto{\pgfqpoint{0.812698in}{1.271849in}}%
\pgfpathlineto{\pgfqpoint{0.808977in}{1.266194in}}%
\pgfpathlineto{\pgfqpoint{0.795708in}{1.261825in}}%
\pgfpathlineto{\pgfqpoint{0.782439in}{1.268307in}}%
\pgfpathclose%
\pgfusepath{fill}%
\end{pgfscope}%
\begin{pgfscope}%
\pgfpathrectangle{\pgfqpoint{0.211875in}{0.211875in}}{\pgfqpoint{1.313625in}{1.279725in}}%
\pgfusepath{clip}%
\pgfsetbuttcap%
\pgfsetroundjoin%
\definecolor{currentfill}{rgb}{0.961115,0.566634,0.405693}%
\pgfsetfillcolor{currentfill}%
\pgfsetlinewidth{0.000000pt}%
\definecolor{currentstroke}{rgb}{0.000000,0.000000,0.000000}%
\pgfsetstrokecolor{currentstroke}%
\pgfsetdash{}{0pt}%
\pgfpathmoveto{\pgfqpoint{0.901860in}{1.253112in}}%
\pgfpathlineto{\pgfqpoint{0.915129in}{1.251864in}}%
\pgfpathlineto{\pgfqpoint{0.928398in}{1.253176in}}%
\pgfpathlineto{\pgfqpoint{0.936642in}{1.258923in}}%
\pgfpathlineto{\pgfqpoint{0.940198in}{1.271849in}}%
\pgfpathlineto{\pgfqpoint{0.940889in}{1.284776in}}%
\pgfpathlineto{\pgfqpoint{0.940626in}{1.297702in}}%
\pgfpathlineto{\pgfqpoint{0.938971in}{1.310629in}}%
\pgfpathlineto{\pgfqpoint{0.928398in}{1.323152in}}%
\pgfpathlineto{\pgfqpoint{0.925761in}{1.323555in}}%
\pgfpathlineto{\pgfqpoint{0.915129in}{1.324628in}}%
\pgfpathlineto{\pgfqpoint{0.903988in}{1.323555in}}%
\pgfpathlineto{\pgfqpoint{0.901860in}{1.323265in}}%
\pgfpathlineto{\pgfqpoint{0.890149in}{1.310629in}}%
\pgfpathlineto{\pgfqpoint{0.888591in}{1.300612in}}%
\pgfpathlineto{\pgfqpoint{0.888318in}{1.297702in}}%
\pgfpathlineto{\pgfqpoint{0.888064in}{1.284776in}}%
\pgfpathlineto{\pgfqpoint{0.888591in}{1.274493in}}%
\pgfpathlineto{\pgfqpoint{0.888784in}{1.271849in}}%
\pgfpathlineto{\pgfqpoint{0.892816in}{1.258923in}}%
\pgfpathclose%
\pgfpathmoveto{\pgfqpoint{0.897118in}{1.271849in}}%
\pgfpathlineto{\pgfqpoint{0.894741in}{1.284776in}}%
\pgfpathlineto{\pgfqpoint{0.895421in}{1.297702in}}%
\pgfpathlineto{\pgfqpoint{0.900357in}{1.310629in}}%
\pgfpathlineto{\pgfqpoint{0.901860in}{1.312250in}}%
\pgfpathlineto{\pgfqpoint{0.915129in}{1.316064in}}%
\pgfpathlineto{\pgfqpoint{0.928398in}{1.311343in}}%
\pgfpathlineto{\pgfqpoint{0.929001in}{1.310629in}}%
\pgfpathlineto{\pgfqpoint{0.933693in}{1.297702in}}%
\pgfpathlineto{\pgfqpoint{0.934337in}{1.284776in}}%
\pgfpathlineto{\pgfqpoint{0.932093in}{1.271849in}}%
\pgfpathlineto{\pgfqpoint{0.928398in}{1.265518in}}%
\pgfpathlineto{\pgfqpoint{0.915129in}{1.260133in}}%
\pgfpathlineto{\pgfqpoint{0.901860in}{1.264487in}}%
\pgfpathclose%
\pgfusepath{fill}%
\end{pgfscope}%
\begin{pgfscope}%
\pgfpathrectangle{\pgfqpoint{0.211875in}{0.211875in}}{\pgfqpoint{1.313625in}{1.279725in}}%
\pgfusepath{clip}%
\pgfsetbuttcap%
\pgfsetroundjoin%
\definecolor{currentfill}{rgb}{0.961115,0.566634,0.405693}%
\pgfsetfillcolor{currentfill}%
\pgfsetlinewidth{0.000000pt}%
\definecolor{currentstroke}{rgb}{0.000000,0.000000,0.000000}%
\pgfsetstrokecolor{currentstroke}%
\pgfsetdash{}{0pt}%
\pgfpathmoveto{\pgfqpoint{1.021280in}{1.251788in}}%
\pgfpathlineto{\pgfqpoint{1.034549in}{1.251091in}}%
\pgfpathlineto{\pgfqpoint{1.047818in}{1.252541in}}%
\pgfpathlineto{\pgfqpoint{1.055877in}{1.258923in}}%
\pgfpathlineto{\pgfqpoint{1.058666in}{1.271849in}}%
\pgfpathlineto{\pgfqpoint{1.059204in}{1.284776in}}%
\pgfpathlineto{\pgfqpoint{1.058983in}{1.297702in}}%
\pgfpathlineto{\pgfqpoint{1.057640in}{1.310629in}}%
\pgfpathlineto{\pgfqpoint{1.048343in}{1.323555in}}%
\pgfpathlineto{\pgfqpoint{1.047818in}{1.323756in}}%
\pgfpathlineto{\pgfqpoint{1.034549in}{1.325376in}}%
\pgfpathlineto{\pgfqpoint{1.021280in}{1.324611in}}%
\pgfpathlineto{\pgfqpoint{1.017519in}{1.323555in}}%
\pgfpathlineto{\pgfqpoint{1.008011in}{1.314699in}}%
\pgfpathlineto{\pgfqpoint{1.006857in}{1.310629in}}%
\pgfpathlineto{\pgfqpoint{1.005544in}{1.297702in}}%
\pgfpathlineto{\pgfqpoint{1.005330in}{1.284776in}}%
\pgfpathlineto{\pgfqpoint{1.005867in}{1.271849in}}%
\pgfpathlineto{\pgfqpoint{1.008011in}{1.260558in}}%
\pgfpathlineto{\pgfqpoint{1.008745in}{1.258923in}}%
\pgfpathclose%
\pgfpathmoveto{\pgfqpoint{1.014198in}{1.271849in}}%
\pgfpathlineto{\pgfqpoint{1.011861in}{1.284776in}}%
\pgfpathlineto{\pgfqpoint{1.012537in}{1.297702in}}%
\pgfpathlineto{\pgfqpoint{1.017420in}{1.310629in}}%
\pgfpathlineto{\pgfqpoint{1.021280in}{1.314334in}}%
\pgfpathlineto{\pgfqpoint{1.034549in}{1.316842in}}%
\pgfpathlineto{\pgfqpoint{1.047818in}{1.310986in}}%
\pgfpathlineto{\pgfqpoint{1.048085in}{1.310629in}}%
\pgfpathlineto{\pgfqpoint{1.052327in}{1.297702in}}%
\pgfpathlineto{\pgfqpoint{1.052913in}{1.284776in}}%
\pgfpathlineto{\pgfqpoint{1.050890in}{1.271849in}}%
\pgfpathlineto{\pgfqpoint{1.047818in}{1.265895in}}%
\pgfpathlineto{\pgfqpoint{1.034549in}{1.259221in}}%
\pgfpathlineto{\pgfqpoint{1.021280in}{1.262071in}}%
\pgfpathclose%
\pgfusepath{fill}%
\end{pgfscope}%
\begin{pgfscope}%
\pgfpathrectangle{\pgfqpoint{0.211875in}{0.211875in}}{\pgfqpoint{1.313625in}{1.279725in}}%
\pgfusepath{clip}%
\pgfsetbuttcap%
\pgfsetroundjoin%
\definecolor{currentfill}{rgb}{0.961115,0.566634,0.405693}%
\pgfsetfillcolor{currentfill}%
\pgfsetlinewidth{0.000000pt}%
\definecolor{currentstroke}{rgb}{0.000000,0.000000,0.000000}%
\pgfsetstrokecolor{currentstroke}%
\pgfsetdash{}{0pt}%
\pgfpathmoveto{\pgfqpoint{1.127432in}{1.256405in}}%
\pgfpathlineto{\pgfqpoint{1.140701in}{1.251241in}}%
\pgfpathlineto{\pgfqpoint{1.153970in}{1.250876in}}%
\pgfpathlineto{\pgfqpoint{1.167239in}{1.252699in}}%
\pgfpathlineto{\pgfqpoint{1.174115in}{1.258923in}}%
\pgfpathlineto{\pgfqpoint{1.176675in}{1.271849in}}%
\pgfpathlineto{\pgfqpoint{1.177170in}{1.284776in}}%
\pgfpathlineto{\pgfqpoint{1.176962in}{1.297702in}}%
\pgfpathlineto{\pgfqpoint{1.175715in}{1.310629in}}%
\pgfpathlineto{\pgfqpoint{1.167239in}{1.323536in}}%
\pgfpathlineto{\pgfqpoint{1.167168in}{1.323555in}}%
\pgfpathlineto{\pgfqpoint{1.153970in}{1.325579in}}%
\pgfpathlineto{\pgfqpoint{1.140701in}{1.325168in}}%
\pgfpathlineto{\pgfqpoint{1.134151in}{1.323555in}}%
\pgfpathlineto{\pgfqpoint{1.127432in}{1.319019in}}%
\pgfpathlineto{\pgfqpoint{1.124403in}{1.310629in}}%
\pgfpathlineto{\pgfqpoint{1.123215in}{1.297702in}}%
\pgfpathlineto{\pgfqpoint{1.123016in}{1.284776in}}%
\pgfpathlineto{\pgfqpoint{1.123488in}{1.271849in}}%
\pgfpathlineto{\pgfqpoint{1.125927in}{1.258923in}}%
\pgfpathclose%
\pgfpathmoveto{\pgfqpoint{1.131687in}{1.271849in}}%
\pgfpathlineto{\pgfqpoint{1.129263in}{1.284776in}}%
\pgfpathlineto{\pgfqpoint{1.129967in}{1.297702in}}%
\pgfpathlineto{\pgfqpoint{1.135050in}{1.310629in}}%
\pgfpathlineto{\pgfqpoint{1.140701in}{1.315446in}}%
\pgfpathlineto{\pgfqpoint{1.153970in}{1.316954in}}%
\pgfpathlineto{\pgfqpoint{1.165877in}{1.310629in}}%
\pgfpathlineto{\pgfqpoint{1.167239in}{1.308845in}}%
\pgfpathlineto{\pgfqpoint{1.170522in}{1.297702in}}%
\pgfpathlineto{\pgfqpoint{1.171081in}{1.284776in}}%
\pgfpathlineto{\pgfqpoint{1.169156in}{1.271849in}}%
\pgfpathlineto{\pgfqpoint{1.167239in}{1.267617in}}%
\pgfpathlineto{\pgfqpoint{1.153970in}{1.259084in}}%
\pgfpathlineto{\pgfqpoint{1.140701in}{1.260792in}}%
\pgfpathclose%
\pgfusepath{fill}%
\end{pgfscope}%
\begin{pgfscope}%
\pgfpathrectangle{\pgfqpoint{0.211875in}{0.211875in}}{\pgfqpoint{1.313625in}{1.279725in}}%
\pgfusepath{clip}%
\pgfsetbuttcap%
\pgfsetroundjoin%
\definecolor{currentfill}{rgb}{0.961115,0.566634,0.405693}%
\pgfsetfillcolor{currentfill}%
\pgfsetlinewidth{0.000000pt}%
\definecolor{currentstroke}{rgb}{0.000000,0.000000,0.000000}%
\pgfsetstrokecolor{currentstroke}%
\pgfsetdash{}{0pt}%
\pgfpathmoveto{\pgfqpoint{1.246852in}{1.255291in}}%
\pgfpathlineto{\pgfqpoint{1.260121in}{1.251356in}}%
\pgfpathlineto{\pgfqpoint{1.273390in}{1.251236in}}%
\pgfpathlineto{\pgfqpoint{1.286659in}{1.253917in}}%
\pgfpathlineto{\pgfqpoint{1.291450in}{1.258923in}}%
\pgfpathlineto{\pgfqpoint{1.294253in}{1.271849in}}%
\pgfpathlineto{\pgfqpoint{1.294803in}{1.284776in}}%
\pgfpathlineto{\pgfqpoint{1.294583in}{1.297702in}}%
\pgfpathlineto{\pgfqpoint{1.293240in}{1.310629in}}%
\pgfpathlineto{\pgfqpoint{1.286659in}{1.322134in}}%
\pgfpathlineto{\pgfqpoint{1.282601in}{1.323555in}}%
\pgfpathlineto{\pgfqpoint{1.273390in}{1.325220in}}%
\pgfpathlineto{\pgfqpoint{1.260121in}{1.325077in}}%
\pgfpathlineto{\pgfqpoint{1.253027in}{1.323555in}}%
\pgfpathlineto{\pgfqpoint{1.246852in}{1.320420in}}%
\pgfpathlineto{\pgfqpoint{1.242551in}{1.310629in}}%
\pgfpathlineto{\pgfqpoint{1.241305in}{1.297702in}}%
\pgfpathlineto{\pgfqpoint{1.241099in}{1.284776in}}%
\pgfpathlineto{\pgfqpoint{1.241599in}{1.271849in}}%
\pgfpathlineto{\pgfqpoint{1.244187in}{1.258923in}}%
\pgfpathclose%
\pgfpathmoveto{\pgfqpoint{1.249648in}{1.271849in}}%
\pgfpathlineto{\pgfqpoint{1.246978in}{1.284776in}}%
\pgfpathlineto{\pgfqpoint{1.247751in}{1.297702in}}%
\pgfpathlineto{\pgfqpoint{1.253353in}{1.310629in}}%
\pgfpathlineto{\pgfqpoint{1.260121in}{1.315734in}}%
\pgfpathlineto{\pgfqpoint{1.273390in}{1.316376in}}%
\pgfpathlineto{\pgfqpoint{1.282732in}{1.310629in}}%
\pgfpathlineto{\pgfqpoint{1.286659in}{1.304105in}}%
\pgfpathlineto{\pgfqpoint{1.288306in}{1.297702in}}%
\pgfpathlineto{\pgfqpoint{1.288867in}{1.284776in}}%
\pgfpathlineto{\pgfqpoint{1.286929in}{1.271849in}}%
\pgfpathlineto{\pgfqpoint{1.286659in}{1.271163in}}%
\pgfpathlineto{\pgfqpoint{1.273390in}{1.259750in}}%
\pgfpathlineto{\pgfqpoint{1.260121in}{1.260475in}}%
\pgfpathclose%
\pgfusepath{fill}%
\end{pgfscope}%
\begin{pgfscope}%
\pgfpathrectangle{\pgfqpoint{0.211875in}{0.211875in}}{\pgfqpoint{1.313625in}{1.279725in}}%
\pgfusepath{clip}%
\pgfsetbuttcap%
\pgfsetroundjoin%
\definecolor{currentfill}{rgb}{0.961115,0.566634,0.405693}%
\pgfsetfillcolor{currentfill}%
\pgfsetlinewidth{0.000000pt}%
\definecolor{currentstroke}{rgb}{0.000000,0.000000,0.000000}%
\pgfsetstrokecolor{currentstroke}%
\pgfsetdash{}{0pt}%
\pgfpathmoveto{\pgfqpoint{1.366273in}{1.255643in}}%
\pgfpathlineto{\pgfqpoint{1.379542in}{1.252058in}}%
\pgfpathlineto{\pgfqpoint{1.392811in}{1.252215in}}%
\pgfpathlineto{\pgfqpoint{1.406080in}{1.256657in}}%
\pgfpathlineto{\pgfqpoint{1.407930in}{1.258923in}}%
\pgfpathlineto{\pgfqpoint{1.411409in}{1.271849in}}%
\pgfpathlineto{\pgfqpoint{1.412102in}{1.284776in}}%
\pgfpathlineto{\pgfqpoint{1.411848in}{1.297702in}}%
\pgfpathlineto{\pgfqpoint{1.410232in}{1.310629in}}%
\pgfpathlineto{\pgfqpoint{1.406080in}{1.319082in}}%
\pgfpathlineto{\pgfqpoint{1.396158in}{1.323555in}}%
\pgfpathlineto{\pgfqpoint{1.392811in}{1.324256in}}%
\pgfpathlineto{\pgfqpoint{1.379542in}{1.324411in}}%
\pgfpathlineto{\pgfqpoint{1.374918in}{1.323555in}}%
\pgfpathlineto{\pgfqpoint{1.366273in}{1.320197in}}%
\pgfpathlineto{\pgfqpoint{1.361303in}{1.310629in}}%
\pgfpathlineto{\pgfqpoint{1.359807in}{1.297702in}}%
\pgfpathlineto{\pgfqpoint{1.359569in}{1.284776in}}%
\pgfpathlineto{\pgfqpoint{1.360194in}{1.271849in}}%
\pgfpathlineto{\pgfqpoint{1.363405in}{1.258923in}}%
\pgfpathclose%
\pgfpathmoveto{\pgfqpoint{1.368193in}{1.271849in}}%
\pgfpathlineto{\pgfqpoint{1.366273in}{1.279126in}}%
\pgfpathlineto{\pgfqpoint{1.365491in}{1.284776in}}%
\pgfpathlineto{\pgfqpoint{1.366072in}{1.297702in}}%
\pgfpathlineto{\pgfqpoint{1.366273in}{1.298563in}}%
\pgfpathlineto{\pgfqpoint{1.372505in}{1.310629in}}%
\pgfpathlineto{\pgfqpoint{1.379542in}{1.315293in}}%
\pgfpathlineto{\pgfqpoint{1.392811in}{1.315057in}}%
\pgfpathlineto{\pgfqpoint{1.399107in}{1.310629in}}%
\pgfpathlineto{\pgfqpoint{1.405491in}{1.297702in}}%
\pgfpathlineto{\pgfqpoint{1.406080in}{1.289231in}}%
\pgfpathlineto{\pgfqpoint{1.406274in}{1.284776in}}%
\pgfpathlineto{\pgfqpoint{1.406080in}{1.283293in}}%
\pgfpathlineto{\pgfqpoint{1.403296in}{1.271849in}}%
\pgfpathlineto{\pgfqpoint{1.392811in}{1.261281in}}%
\pgfpathlineto{\pgfqpoint{1.379542in}{1.261007in}}%
\pgfpathclose%
\pgfusepath{fill}%
\end{pgfscope}%
\begin{pgfscope}%
\pgfpathrectangle{\pgfqpoint{0.211875in}{0.211875in}}{\pgfqpoint{1.313625in}{1.279725in}}%
\pgfusepath{clip}%
\pgfsetbuttcap%
\pgfsetroundjoin%
\definecolor{currentfill}{rgb}{0.961115,0.566634,0.405693}%
\pgfsetfillcolor{currentfill}%
\pgfsetlinewidth{0.000000pt}%
\definecolor{currentstroke}{rgb}{0.000000,0.000000,0.000000}%
\pgfsetstrokecolor{currentstroke}%
\pgfsetdash{}{0pt}%
\pgfpathmoveto{\pgfqpoint{1.485693in}{1.256913in}}%
\pgfpathlineto{\pgfqpoint{1.498962in}{1.253308in}}%
\pgfpathlineto{\pgfqpoint{1.512231in}{1.253884in}}%
\pgfpathlineto{\pgfqpoint{1.522826in}{1.258923in}}%
\pgfpathlineto{\pgfqpoint{1.525500in}{1.262528in}}%
\pgfpathlineto{\pgfqpoint{1.525500in}{1.271849in}}%
\pgfpathlineto{\pgfqpoint{1.525500in}{1.284776in}}%
\pgfpathlineto{\pgfqpoint{1.525500in}{1.297702in}}%
\pgfpathlineto{\pgfqpoint{1.525500in}{1.310629in}}%
\pgfpathlineto{\pgfqpoint{1.525500in}{1.313498in}}%
\pgfpathlineto{\pgfqpoint{1.512231in}{1.322511in}}%
\pgfpathlineto{\pgfqpoint{1.498962in}{1.323169in}}%
\pgfpathlineto{\pgfqpoint{1.485693in}{1.318943in}}%
\pgfpathlineto{\pgfqpoint{1.480681in}{1.310629in}}%
\pgfpathlineto{\pgfqpoint{1.478730in}{1.297702in}}%
\pgfpathlineto{\pgfqpoint{1.478435in}{1.284776in}}%
\pgfpathlineto{\pgfqpoint{1.479288in}{1.271849in}}%
\pgfpathlineto{\pgfqpoint{1.483641in}{1.258923in}}%
\pgfpathclose%
\pgfpathmoveto{\pgfqpoint{1.487511in}{1.271849in}}%
\pgfpathlineto{\pgfqpoint{1.485693in}{1.277238in}}%
\pgfpathlineto{\pgfqpoint{1.484478in}{1.284776in}}%
\pgfpathlineto{\pgfqpoint{1.485126in}{1.297702in}}%
\pgfpathlineto{\pgfqpoint{1.485693in}{1.299809in}}%
\pgfpathlineto{\pgfqpoint{1.492800in}{1.310629in}}%
\pgfpathlineto{\pgfqpoint{1.498962in}{1.314180in}}%
\pgfpathlineto{\pgfqpoint{1.512231in}{1.312910in}}%
\pgfpathlineto{\pgfqpoint{1.515094in}{1.310629in}}%
\pgfpathlineto{\pgfqpoint{1.521551in}{1.297702in}}%
\pgfpathlineto{\pgfqpoint{1.522436in}{1.284776in}}%
\pgfpathlineto{\pgfqpoint{1.519301in}{1.271849in}}%
\pgfpathlineto{\pgfqpoint{1.512231in}{1.263780in}}%
\pgfpathlineto{\pgfqpoint{1.498962in}{1.262322in}}%
\pgfpathclose%
\pgfusepath{fill}%
\end{pgfscope}%
\begin{pgfscope}%
\pgfpathrectangle{\pgfqpoint{0.211875in}{0.211875in}}{\pgfqpoint{1.313625in}{1.279725in}}%
\pgfusepath{clip}%
\pgfsetbuttcap%
\pgfsetroundjoin%
\definecolor{currentfill}{rgb}{0.961115,0.566634,0.405693}%
\pgfsetfillcolor{currentfill}%
\pgfsetlinewidth{0.000000pt}%
\definecolor{currentstroke}{rgb}{0.000000,0.000000,0.000000}%
\pgfsetstrokecolor{currentstroke}%
\pgfsetdash{}{0pt}%
\pgfpathmoveto{\pgfqpoint{0.213301in}{1.271849in}}%
\pgfpathlineto{\pgfqpoint{0.219258in}{1.284776in}}%
\pgfpathlineto{\pgfqpoint{0.217658in}{1.297702in}}%
\pgfpathlineto{\pgfqpoint{0.211875in}{1.305003in}}%
\pgfpathlineto{\pgfqpoint{0.211875in}{1.297702in}}%
\pgfpathlineto{\pgfqpoint{0.211875in}{1.284776in}}%
\pgfpathlineto{\pgfqpoint{0.211875in}{1.271849in}}%
\pgfpathlineto{\pgfqpoint{0.211875in}{1.270701in}}%
\pgfpathclose%
\pgfusepath{fill}%
\end{pgfscope}%
\begin{pgfscope}%
\pgfpathrectangle{\pgfqpoint{0.211875in}{0.211875in}}{\pgfqpoint{1.313625in}{1.279725in}}%
\pgfusepath{clip}%
\pgfsetbuttcap%
\pgfsetroundjoin%
\definecolor{currentfill}{rgb}{0.961115,0.566634,0.405693}%
\pgfsetfillcolor{currentfill}%
\pgfsetlinewidth{0.000000pt}%
\definecolor{currentstroke}{rgb}{0.000000,0.000000,0.000000}%
\pgfsetstrokecolor{currentstroke}%
\pgfsetdash{}{0pt}%
\pgfpathmoveto{\pgfqpoint{0.318027in}{1.266692in}}%
\pgfpathlineto{\pgfqpoint{0.331295in}{1.266009in}}%
\pgfpathlineto{\pgfqpoint{0.337590in}{1.271849in}}%
\pgfpathlineto{\pgfqpoint{0.341777in}{1.284776in}}%
\pgfpathlineto{\pgfqpoint{0.340622in}{1.297702in}}%
\pgfpathlineto{\pgfqpoint{0.331908in}{1.310629in}}%
\pgfpathlineto{\pgfqpoint{0.331295in}{1.311020in}}%
\pgfpathlineto{\pgfqpoint{0.322393in}{1.310629in}}%
\pgfpathlineto{\pgfqpoint{0.318027in}{1.310328in}}%
\pgfpathlineto{\pgfqpoint{0.310134in}{1.297702in}}%
\pgfpathlineto{\pgfqpoint{0.309017in}{1.284776in}}%
\pgfpathlineto{\pgfqpoint{0.313045in}{1.271849in}}%
\pgfpathclose%
\pgfpathmoveto{\pgfqpoint{0.317576in}{1.284776in}}%
\pgfpathlineto{\pgfqpoint{0.318027in}{1.288581in}}%
\pgfpathlineto{\pgfqpoint{0.331295in}{1.293532in}}%
\pgfpathlineto{\pgfqpoint{0.332434in}{1.284776in}}%
\pgfpathlineto{\pgfqpoint{0.331295in}{1.282063in}}%
\pgfpathlineto{\pgfqpoint{0.318027in}{1.283597in}}%
\pgfpathclose%
\pgfusepath{fill}%
\end{pgfscope}%
\begin{pgfscope}%
\pgfpathrectangle{\pgfqpoint{0.211875in}{0.211875in}}{\pgfqpoint{1.313625in}{1.279725in}}%
\pgfusepath{clip}%
\pgfsetbuttcap%
\pgfsetroundjoin%
\definecolor{currentfill}{rgb}{0.961115,0.566634,0.405693}%
\pgfsetfillcolor{currentfill}%
\pgfsetlinewidth{0.000000pt}%
\definecolor{currentstroke}{rgb}{0.000000,0.000000,0.000000}%
\pgfsetstrokecolor{currentstroke}%
\pgfsetdash{}{0pt}%
\pgfpathmoveto{\pgfqpoint{0.437447in}{1.261498in}}%
\pgfpathlineto{\pgfqpoint{0.450716in}{1.261929in}}%
\pgfpathlineto{\pgfqpoint{0.460095in}{1.271849in}}%
\pgfpathlineto{\pgfqpoint{0.463029in}{1.284776in}}%
\pgfpathlineto{\pgfqpoint{0.462192in}{1.297702in}}%
\pgfpathlineto{\pgfqpoint{0.456011in}{1.310629in}}%
\pgfpathlineto{\pgfqpoint{0.450716in}{1.314493in}}%
\pgfpathlineto{\pgfqpoint{0.437447in}{1.314882in}}%
\pgfpathlineto{\pgfqpoint{0.430786in}{1.310629in}}%
\pgfpathlineto{\pgfqpoint{0.424178in}{1.299257in}}%
\pgfpathlineto{\pgfqpoint{0.423784in}{1.297702in}}%
\pgfpathlineto{\pgfqpoint{0.423164in}{1.284776in}}%
\pgfpathlineto{\pgfqpoint{0.424178in}{1.278094in}}%
\pgfpathlineto{\pgfqpoint{0.426064in}{1.271849in}}%
\pgfpathclose%
\pgfpathmoveto{\pgfqpoint{0.432014in}{1.284776in}}%
\pgfpathlineto{\pgfqpoint{0.433549in}{1.297702in}}%
\pgfpathlineto{\pgfqpoint{0.437447in}{1.303183in}}%
\pgfpathlineto{\pgfqpoint{0.450716in}{1.301806in}}%
\pgfpathlineto{\pgfqpoint{0.453201in}{1.297702in}}%
\pgfpathlineto{\pgfqpoint{0.454543in}{1.284776in}}%
\pgfpathlineto{\pgfqpoint{0.450716in}{1.274402in}}%
\pgfpathlineto{\pgfqpoint{0.437447in}{1.272298in}}%
\pgfpathclose%
\pgfusepath{fill}%
\end{pgfscope}%
\begin{pgfscope}%
\pgfpathrectangle{\pgfqpoint{0.211875in}{0.211875in}}{\pgfqpoint{1.313625in}{1.279725in}}%
\pgfusepath{clip}%
\pgfsetbuttcap%
\pgfsetroundjoin%
\definecolor{currentfill}{rgb}{0.961115,0.566634,0.405693}%
\pgfsetfillcolor{currentfill}%
\pgfsetlinewidth{0.000000pt}%
\definecolor{currentstroke}{rgb}{0.000000,0.000000,0.000000}%
\pgfsetstrokecolor{currentstroke}%
\pgfsetdash{}{0pt}%
\pgfpathmoveto{\pgfqpoint{0.742633in}{1.335857in}}%
\pgfpathlineto{\pgfqpoint{0.745432in}{1.336482in}}%
\pgfpathlineto{\pgfqpoint{0.755902in}{1.341233in}}%
\pgfpathlineto{\pgfqpoint{0.760043in}{1.349408in}}%
\pgfpathlineto{\pgfqpoint{0.761755in}{1.362335in}}%
\pgfpathlineto{\pgfqpoint{0.761744in}{1.375261in}}%
\pgfpathlineto{\pgfqpoint{0.760075in}{1.388188in}}%
\pgfpathlineto{\pgfqpoint{0.755902in}{1.396955in}}%
\pgfpathlineto{\pgfqpoint{0.748466in}{1.401114in}}%
\pgfpathlineto{\pgfqpoint{0.742633in}{1.402693in}}%
\pgfpathlineto{\pgfqpoint{0.729364in}{1.402020in}}%
\pgfpathlineto{\pgfqpoint{0.726998in}{1.401114in}}%
\pgfpathlineto{\pgfqpoint{0.716095in}{1.390349in}}%
\pgfpathlineto{\pgfqpoint{0.715359in}{1.388188in}}%
\pgfpathlineto{\pgfqpoint{0.713618in}{1.375261in}}%
\pgfpathlineto{\pgfqpoint{0.713617in}{1.362335in}}%
\pgfpathlineto{\pgfqpoint{0.715430in}{1.349408in}}%
\pgfpathlineto{\pgfqpoint{0.716095in}{1.347544in}}%
\pgfpathlineto{\pgfqpoint{0.729364in}{1.336504in}}%
\pgfpathlineto{\pgfqpoint{0.729731in}{1.336482in}}%
\pgfpathclose%
\pgfpathmoveto{\pgfqpoint{0.726353in}{1.349408in}}%
\pgfpathlineto{\pgfqpoint{0.720934in}{1.362335in}}%
\pgfpathlineto{\pgfqpoint{0.720801in}{1.375261in}}%
\pgfpathlineto{\pgfqpoint{0.725612in}{1.388188in}}%
\pgfpathlineto{\pgfqpoint{0.729364in}{1.391866in}}%
\pgfpathlineto{\pgfqpoint{0.742633in}{1.393261in}}%
\pgfpathlineto{\pgfqpoint{0.749560in}{1.388188in}}%
\pgfpathlineto{\pgfqpoint{0.755431in}{1.375261in}}%
\pgfpathlineto{\pgfqpoint{0.755277in}{1.362335in}}%
\pgfpathlineto{\pgfqpoint{0.748660in}{1.349408in}}%
\pgfpathlineto{\pgfqpoint{0.742633in}{1.345335in}}%
\pgfpathlineto{\pgfqpoint{0.729364in}{1.346673in}}%
\pgfpathclose%
\pgfusepath{fill}%
\end{pgfscope}%
\begin{pgfscope}%
\pgfpathrectangle{\pgfqpoint{0.211875in}{0.211875in}}{\pgfqpoint{1.313625in}{1.279725in}}%
\pgfusepath{clip}%
\pgfsetbuttcap%
\pgfsetroundjoin%
\definecolor{currentfill}{rgb}{0.961115,0.566634,0.405693}%
\pgfsetfillcolor{currentfill}%
\pgfsetlinewidth{0.000000pt}%
\definecolor{currentstroke}{rgb}{0.000000,0.000000,0.000000}%
\pgfsetstrokecolor{currentstroke}%
\pgfsetdash{}{0pt}%
\pgfpathmoveto{\pgfqpoint{0.848784in}{1.334534in}}%
\pgfpathlineto{\pgfqpoint{0.862053in}{1.334381in}}%
\pgfpathlineto{\pgfqpoint{0.870127in}{1.336482in}}%
\pgfpathlineto{\pgfqpoint{0.875322in}{1.339545in}}%
\pgfpathlineto{\pgfqpoint{0.879601in}{1.349408in}}%
\pgfpathlineto{\pgfqpoint{0.880875in}{1.362335in}}%
\pgfpathlineto{\pgfqpoint{0.880846in}{1.375261in}}%
\pgfpathlineto{\pgfqpoint{0.879541in}{1.388188in}}%
\pgfpathlineto{\pgfqpoint{0.875322in}{1.398478in}}%
\pgfpathlineto{\pgfqpoint{0.871655in}{1.401114in}}%
\pgfpathlineto{\pgfqpoint{0.862053in}{1.404129in}}%
\pgfpathlineto{\pgfqpoint{0.848784in}{1.403979in}}%
\pgfpathlineto{\pgfqpoint{0.840263in}{1.401114in}}%
\pgfpathlineto{\pgfqpoint{0.835515in}{1.397467in}}%
\pgfpathlineto{\pgfqpoint{0.831779in}{1.388188in}}%
\pgfpathlineto{\pgfqpoint{0.830379in}{1.375261in}}%
\pgfpathlineto{\pgfqpoint{0.830357in}{1.362335in}}%
\pgfpathlineto{\pgfqpoint{0.831752in}{1.349408in}}%
\pgfpathlineto{\pgfqpoint{0.835515in}{1.340564in}}%
\pgfpathlineto{\pgfqpoint{0.841899in}{1.336482in}}%
\pgfpathclose%
\pgfpathmoveto{\pgfqpoint{0.842087in}{1.349408in}}%
\pgfpathlineto{\pgfqpoint{0.836799in}{1.362335in}}%
\pgfpathlineto{\pgfqpoint{0.836685in}{1.375261in}}%
\pgfpathlineto{\pgfqpoint{0.841419in}{1.388188in}}%
\pgfpathlineto{\pgfqpoint{0.848784in}{1.394562in}}%
\pgfpathlineto{\pgfqpoint{0.862053in}{1.394736in}}%
\pgfpathlineto{\pgfqpoint{0.869850in}{1.388188in}}%
\pgfpathlineto{\pgfqpoint{0.874587in}{1.375261in}}%
\pgfpathlineto{\pgfqpoint{0.874478in}{1.362335in}}%
\pgfpathlineto{\pgfqpoint{0.869189in}{1.349408in}}%
\pgfpathlineto{\pgfqpoint{0.862053in}{1.343870in}}%
\pgfpathlineto{\pgfqpoint{0.848784in}{1.344041in}}%
\pgfpathclose%
\pgfusepath{fill}%
\end{pgfscope}%
\begin{pgfscope}%
\pgfpathrectangle{\pgfqpoint{0.211875in}{0.211875in}}{\pgfqpoint{1.313625in}{1.279725in}}%
\pgfusepath{clip}%
\pgfsetbuttcap%
\pgfsetroundjoin%
\definecolor{currentfill}{rgb}{0.961115,0.566634,0.405693}%
\pgfsetfillcolor{currentfill}%
\pgfsetlinewidth{0.000000pt}%
\definecolor{currentstroke}{rgb}{0.000000,0.000000,0.000000}%
\pgfsetstrokecolor{currentstroke}%
\pgfsetdash{}{0pt}%
\pgfpathmoveto{\pgfqpoint{0.968205in}{1.333270in}}%
\pgfpathlineto{\pgfqpoint{0.981473in}{1.333438in}}%
\pgfpathlineto{\pgfqpoint{0.991634in}{1.336482in}}%
\pgfpathlineto{\pgfqpoint{0.994742in}{1.338881in}}%
\pgfpathlineto{\pgfqpoint{0.998569in}{1.349408in}}%
\pgfpathlineto{\pgfqpoint{0.999573in}{1.362335in}}%
\pgfpathlineto{\pgfqpoint{0.999533in}{1.375261in}}%
\pgfpathlineto{\pgfqpoint{0.998453in}{1.388188in}}%
\pgfpathlineto{\pgfqpoint{0.994742in}{1.398899in}}%
\pgfpathlineto{\pgfqpoint{0.992363in}{1.401114in}}%
\pgfpathlineto{\pgfqpoint{0.981473in}{1.405032in}}%
\pgfpathlineto{\pgfqpoint{0.968205in}{1.405242in}}%
\pgfpathlineto{\pgfqpoint{0.954936in}{1.401541in}}%
\pgfpathlineto{\pgfqpoint{0.954418in}{1.401114in}}%
\pgfpathlineto{\pgfqpoint{0.948659in}{1.388188in}}%
\pgfpathlineto{\pgfqpoint{0.947488in}{1.375261in}}%
\pgfpathlineto{\pgfqpoint{0.947451in}{1.362335in}}%
\pgfpathlineto{\pgfqpoint{0.948561in}{1.349408in}}%
\pgfpathlineto{\pgfqpoint{0.954936in}{1.336569in}}%
\pgfpathlineto{\pgfqpoint{0.955112in}{1.336482in}}%
\pgfpathclose%
\pgfpathmoveto{\pgfqpoint{0.958075in}{1.349408in}}%
\pgfpathlineto{\pgfqpoint{0.954936in}{1.355977in}}%
\pgfpathlineto{\pgfqpoint{0.953574in}{1.362335in}}%
\pgfpathlineto{\pgfqpoint{0.953511in}{1.375261in}}%
\pgfpathlineto{\pgfqpoint{0.954936in}{1.382353in}}%
\pgfpathlineto{\pgfqpoint{0.957441in}{1.388188in}}%
\pgfpathlineto{\pgfqpoint{0.968205in}{1.396339in}}%
\pgfpathlineto{\pgfqpoint{0.981473in}{1.395489in}}%
\pgfpathlineto{\pgfqpoint{0.989127in}{1.388188in}}%
\pgfpathlineto{\pgfqpoint{0.993138in}{1.375261in}}%
\pgfpathlineto{\pgfqpoint{0.993055in}{1.362335in}}%
\pgfpathlineto{\pgfqpoint{0.988610in}{1.349408in}}%
\pgfpathlineto{\pgfqpoint{0.981473in}{1.343109in}}%
\pgfpathlineto{\pgfqpoint{0.968205in}{1.342313in}}%
\pgfpathclose%
\pgfusepath{fill}%
\end{pgfscope}%
\begin{pgfscope}%
\pgfpathrectangle{\pgfqpoint{0.211875in}{0.211875in}}{\pgfqpoint{1.313625in}{1.279725in}}%
\pgfusepath{clip}%
\pgfsetbuttcap%
\pgfsetroundjoin%
\definecolor{currentfill}{rgb}{0.961115,0.566634,0.405693}%
\pgfsetfillcolor{currentfill}%
\pgfsetlinewidth{0.000000pt}%
\definecolor{currentstroke}{rgb}{0.000000,0.000000,0.000000}%
\pgfsetstrokecolor{currentstroke}%
\pgfsetdash{}{0pt}%
\pgfpathmoveto{\pgfqpoint{1.074356in}{1.334786in}}%
\pgfpathlineto{\pgfqpoint{1.087625in}{1.332633in}}%
\pgfpathlineto{\pgfqpoint{1.100894in}{1.333065in}}%
\pgfpathlineto{\pgfqpoint{1.110886in}{1.336482in}}%
\pgfpathlineto{\pgfqpoint{1.114163in}{1.339874in}}%
\pgfpathlineto{\pgfqpoint{1.116972in}{1.349408in}}%
\pgfpathlineto{\pgfqpoint{1.117860in}{1.362335in}}%
\pgfpathlineto{\pgfqpoint{1.117815in}{1.375261in}}%
\pgfpathlineto{\pgfqpoint{1.116833in}{1.388188in}}%
\pgfpathlineto{\pgfqpoint{1.114163in}{1.397581in}}%
\pgfpathlineto{\pgfqpoint{1.111289in}{1.401114in}}%
\pgfpathlineto{\pgfqpoint{1.100894in}{1.405367in}}%
\pgfpathlineto{\pgfqpoint{1.087625in}{1.405882in}}%
\pgfpathlineto{\pgfqpoint{1.074356in}{1.403380in}}%
\pgfpathlineto{\pgfqpoint{1.071190in}{1.401114in}}%
\pgfpathlineto{\pgfqpoint{1.065986in}{1.388188in}}%
\pgfpathlineto{\pgfqpoint{1.064929in}{1.375261in}}%
\pgfpathlineto{\pgfqpoint{1.064883in}{1.362335in}}%
\pgfpathlineto{\pgfqpoint{1.065847in}{1.349408in}}%
\pgfpathlineto{\pgfqpoint{1.071552in}{1.336482in}}%
\pgfpathclose%
\pgfpathmoveto{\pgfqpoint{1.074353in}{1.349408in}}%
\pgfpathlineto{\pgfqpoint{1.071170in}{1.362335in}}%
\pgfpathlineto{\pgfqpoint{1.071114in}{1.375261in}}%
\pgfpathlineto{\pgfqpoint{1.073995in}{1.388188in}}%
\pgfpathlineto{\pgfqpoint{1.074356in}{1.388863in}}%
\pgfpathlineto{\pgfqpoint{1.087625in}{1.397293in}}%
\pgfpathlineto{\pgfqpoint{1.100894in}{1.395461in}}%
\pgfpathlineto{\pgfqpoint{1.107650in}{1.388188in}}%
\pgfpathlineto{\pgfqpoint{1.111227in}{1.375261in}}%
\pgfpathlineto{\pgfqpoint{1.111157in}{1.362335in}}%
\pgfpathlineto{\pgfqpoint{1.107208in}{1.349408in}}%
\pgfpathlineto{\pgfqpoint{1.100894in}{1.343109in}}%
\pgfpathlineto{\pgfqpoint{1.087625in}{1.341388in}}%
\pgfpathlineto{\pgfqpoint{1.074356in}{1.349403in}}%
\pgfpathclose%
\pgfusepath{fill}%
\end{pgfscope}%
\begin{pgfscope}%
\pgfpathrectangle{\pgfqpoint{0.211875in}{0.211875in}}{\pgfqpoint{1.313625in}{1.279725in}}%
\pgfusepath{clip}%
\pgfsetbuttcap%
\pgfsetroundjoin%
\definecolor{currentfill}{rgb}{0.961115,0.566634,0.405693}%
\pgfsetfillcolor{currentfill}%
\pgfsetlinewidth{0.000000pt}%
\definecolor{currentstroke}{rgb}{0.000000,0.000000,0.000000}%
\pgfsetstrokecolor{currentstroke}%
\pgfsetdash{}{0pt}%
\pgfpathmoveto{\pgfqpoint{1.193777in}{1.334163in}}%
\pgfpathlineto{\pgfqpoint{1.207045in}{1.332576in}}%
\pgfpathlineto{\pgfqpoint{1.220314in}{1.333330in}}%
\pgfpathlineto{\pgfqpoint{1.228429in}{1.336482in}}%
\pgfpathlineto{\pgfqpoint{1.233583in}{1.343977in}}%
\pgfpathlineto{\pgfqpoint{1.234813in}{1.349408in}}%
\pgfpathlineto{\pgfqpoint{1.235731in}{1.362335in}}%
\pgfpathlineto{\pgfqpoint{1.235687in}{1.375261in}}%
\pgfpathlineto{\pgfqpoint{1.234680in}{1.388188in}}%
\pgfpathlineto{\pgfqpoint{1.233583in}{1.393115in}}%
\pgfpathlineto{\pgfqpoint{1.228851in}{1.401114in}}%
\pgfpathlineto{\pgfqpoint{1.220314in}{1.405067in}}%
\pgfpathlineto{\pgfqpoint{1.207045in}{1.405947in}}%
\pgfpathlineto{\pgfqpoint{1.193777in}{1.404082in}}%
\pgfpathlineto{\pgfqpoint{1.189046in}{1.401114in}}%
\pgfpathlineto{\pgfqpoint{1.183767in}{1.388188in}}%
\pgfpathlineto{\pgfqpoint{1.182701in}{1.375261in}}%
\pgfpathlineto{\pgfqpoint{1.182651in}{1.362335in}}%
\pgfpathlineto{\pgfqpoint{1.183615in}{1.349408in}}%
\pgfpathlineto{\pgfqpoint{1.189380in}{1.336482in}}%
\pgfpathclose%
\pgfpathmoveto{\pgfqpoint{1.192425in}{1.349408in}}%
\pgfpathlineto{\pgfqpoint{1.189155in}{1.362335in}}%
\pgfpathlineto{\pgfqpoint{1.189097in}{1.375261in}}%
\pgfpathlineto{\pgfqpoint{1.192060in}{1.388188in}}%
\pgfpathlineto{\pgfqpoint{1.193777in}{1.391018in}}%
\pgfpathlineto{\pgfqpoint{1.207045in}{1.397486in}}%
\pgfpathlineto{\pgfqpoint{1.220314in}{1.394555in}}%
\pgfpathlineto{\pgfqpoint{1.225574in}{1.388188in}}%
\pgfpathlineto{\pgfqpoint{1.228939in}{1.375261in}}%
\pgfpathlineto{\pgfqpoint{1.228872in}{1.362335in}}%
\pgfpathlineto{\pgfqpoint{1.225156in}{1.349408in}}%
\pgfpathlineto{\pgfqpoint{1.220314in}{1.343968in}}%
\pgfpathlineto{\pgfqpoint{1.207045in}{1.341205in}}%
\pgfpathlineto{\pgfqpoint{1.193777in}{1.347329in}}%
\pgfpathclose%
\pgfusepath{fill}%
\end{pgfscope}%
\begin{pgfscope}%
\pgfpathrectangle{\pgfqpoint{0.211875in}{0.211875in}}{\pgfqpoint{1.313625in}{1.279725in}}%
\pgfusepath{clip}%
\pgfsetbuttcap%
\pgfsetroundjoin%
\definecolor{currentfill}{rgb}{0.961115,0.566634,0.405693}%
\pgfsetfillcolor{currentfill}%
\pgfsetlinewidth{0.000000pt}%
\definecolor{currentstroke}{rgb}{0.000000,0.000000,0.000000}%
\pgfsetstrokecolor{currentstroke}%
\pgfsetdash{}{0pt}%
\pgfpathmoveto{\pgfqpoint{1.313197in}{1.334393in}}%
\pgfpathlineto{\pgfqpoint{1.326466in}{1.333077in}}%
\pgfpathlineto{\pgfqpoint{1.339735in}{1.334341in}}%
\pgfpathlineto{\pgfqpoint{1.344596in}{1.336482in}}%
\pgfpathlineto{\pgfqpoint{1.351931in}{1.349408in}}%
\pgfpathlineto{\pgfqpoint{1.353004in}{1.359888in}}%
\pgfpathlineto{\pgfqpoint{1.353165in}{1.362335in}}%
\pgfpathlineto{\pgfqpoint{1.353129in}{1.375261in}}%
\pgfpathlineto{\pgfqpoint{1.353004in}{1.377015in}}%
\pgfpathlineto{\pgfqpoint{1.351821in}{1.388188in}}%
\pgfpathlineto{\pgfqpoint{1.345301in}{1.401114in}}%
\pgfpathlineto{\pgfqpoint{1.339735in}{1.404026in}}%
\pgfpathlineto{\pgfqpoint{1.326466in}{1.405457in}}%
\pgfpathlineto{\pgfqpoint{1.313197in}{1.403929in}}%
\pgfpathlineto{\pgfqpoint{1.308103in}{1.401114in}}%
\pgfpathlineto{\pgfqpoint{1.302029in}{1.388188in}}%
\pgfpathlineto{\pgfqpoint{1.300814in}{1.375261in}}%
\pgfpathlineto{\pgfqpoint{1.300769in}{1.362335in}}%
\pgfpathlineto{\pgfqpoint{1.301898in}{1.349408in}}%
\pgfpathlineto{\pgfqpoint{1.308683in}{1.336482in}}%
\pgfpathclose%
\pgfpathmoveto{\pgfqpoint{1.311095in}{1.349408in}}%
\pgfpathlineto{\pgfqpoint{1.307549in}{1.362335in}}%
\pgfpathlineto{\pgfqpoint{1.307483in}{1.375261in}}%
\pgfpathlineto{\pgfqpoint{1.310683in}{1.388188in}}%
\pgfpathlineto{\pgfqpoint{1.313197in}{1.391861in}}%
\pgfpathlineto{\pgfqpoint{1.326466in}{1.396949in}}%
\pgfpathlineto{\pgfqpoint{1.339735in}{1.392617in}}%
\pgfpathlineto{\pgfqpoint{1.342992in}{1.388188in}}%
\pgfpathlineto{\pgfqpoint{1.346322in}{1.375261in}}%
\pgfpathlineto{\pgfqpoint{1.346251in}{1.362335in}}%
\pgfpathlineto{\pgfqpoint{1.342557in}{1.349408in}}%
\pgfpathlineto{\pgfqpoint{1.339735in}{1.345839in}}%
\pgfpathlineto{\pgfqpoint{1.326466in}{1.341736in}}%
\pgfpathlineto{\pgfqpoint{1.313197in}{1.346551in}}%
\pgfpathclose%
\pgfusepath{fill}%
\end{pgfscope}%
\begin{pgfscope}%
\pgfpathrectangle{\pgfqpoint{0.211875in}{0.211875in}}{\pgfqpoint{1.313625in}{1.279725in}}%
\pgfusepath{clip}%
\pgfsetbuttcap%
\pgfsetroundjoin%
\definecolor{currentfill}{rgb}{0.961115,0.566634,0.405693}%
\pgfsetfillcolor{currentfill}%
\pgfsetlinewidth{0.000000pt}%
\definecolor{currentstroke}{rgb}{0.000000,0.000000,0.000000}%
\pgfsetstrokecolor{currentstroke}%
\pgfsetdash{}{0pt}%
\pgfpathmoveto{\pgfqpoint{1.432617in}{1.335301in}}%
\pgfpathlineto{\pgfqpoint{1.445886in}{1.334136in}}%
\pgfpathlineto{\pgfqpoint{1.459155in}{1.336266in}}%
\pgfpathlineto{\pgfqpoint{1.459586in}{1.336482in}}%
\pgfpathlineto{\pgfqpoint{1.468398in}{1.349408in}}%
\pgfpathlineto{\pgfqpoint{1.469934in}{1.362335in}}%
\pgfpathlineto{\pgfqpoint{1.469910in}{1.375261in}}%
\pgfpathlineto{\pgfqpoint{1.468369in}{1.388188in}}%
\pgfpathlineto{\pgfqpoint{1.460789in}{1.401114in}}%
\pgfpathlineto{\pgfqpoint{1.459155in}{1.402080in}}%
\pgfpathlineto{\pgfqpoint{1.445886in}{1.404413in}}%
\pgfpathlineto{\pgfqpoint{1.432617in}{1.403097in}}%
\pgfpathlineto{\pgfqpoint{1.428553in}{1.401114in}}%
\pgfpathlineto{\pgfqpoint{1.420823in}{1.388188in}}%
\pgfpathlineto{\pgfqpoint{1.419348in}{1.375787in}}%
\pgfpathlineto{\pgfqpoint{1.419307in}{1.375261in}}%
\pgfpathlineto{\pgfqpoint{1.419278in}{1.362335in}}%
\pgfpathlineto{\pgfqpoint{1.419348in}{1.361380in}}%
\pgfpathlineto{\pgfqpoint{1.420754in}{1.349408in}}%
\pgfpathlineto{\pgfqpoint{1.429713in}{1.336482in}}%
\pgfpathclose%
\pgfpathmoveto{\pgfqpoint{1.430440in}{1.349408in}}%
\pgfpathlineto{\pgfqpoint{1.426396in}{1.362335in}}%
\pgfpathlineto{\pgfqpoint{1.426312in}{1.375261in}}%
\pgfpathlineto{\pgfqpoint{1.429936in}{1.388188in}}%
\pgfpathlineto{\pgfqpoint{1.432617in}{1.391671in}}%
\pgfpathlineto{\pgfqpoint{1.445886in}{1.395683in}}%
\pgfpathlineto{\pgfqpoint{1.459155in}{1.389415in}}%
\pgfpathlineto{\pgfqpoint{1.459958in}{1.388188in}}%
\pgfpathlineto{\pgfqpoint{1.463401in}{1.375261in}}%
\pgfpathlineto{\pgfqpoint{1.463318in}{1.362335in}}%
\pgfpathlineto{\pgfqpoint{1.459470in}{1.349408in}}%
\pgfpathlineto{\pgfqpoint{1.459155in}{1.348959in}}%
\pgfpathlineto{\pgfqpoint{1.445886in}{1.342978in}}%
\pgfpathlineto{\pgfqpoint{1.432617in}{1.346783in}}%
\pgfpathclose%
\pgfusepath{fill}%
\end{pgfscope}%
\begin{pgfscope}%
\pgfpathrectangle{\pgfqpoint{0.211875in}{0.211875in}}{\pgfqpoint{1.313625in}{1.279725in}}%
\pgfusepath{clip}%
\pgfsetbuttcap%
\pgfsetroundjoin%
\definecolor{currentfill}{rgb}{0.961115,0.566634,0.405693}%
\pgfsetfillcolor{currentfill}%
\pgfsetlinewidth{0.000000pt}%
\definecolor{currentstroke}{rgb}{0.000000,0.000000,0.000000}%
\pgfsetstrokecolor{currentstroke}%
\pgfsetdash{}{0pt}%
\pgfpathmoveto{\pgfqpoint{0.264951in}{1.348953in}}%
\pgfpathlineto{\pgfqpoint{0.266447in}{1.349408in}}%
\pgfpathlineto{\pgfqpoint{0.278220in}{1.357165in}}%
\pgfpathlineto{\pgfqpoint{0.280100in}{1.362335in}}%
\pgfpathlineto{\pgfqpoint{0.280258in}{1.375261in}}%
\pgfpathlineto{\pgfqpoint{0.278220in}{1.381366in}}%
\pgfpathlineto{\pgfqpoint{0.269185in}{1.388188in}}%
\pgfpathlineto{\pgfqpoint{0.264951in}{1.389594in}}%
\pgfpathlineto{\pgfqpoint{0.262255in}{1.388188in}}%
\pgfpathlineto{\pgfqpoint{0.252079in}{1.375261in}}%
\pgfpathlineto{\pgfqpoint{0.252420in}{1.362335in}}%
\pgfpathlineto{\pgfqpoint{0.264010in}{1.349408in}}%
\pgfpathclose%
\pgfusepath{fill}%
\end{pgfscope}%
\begin{pgfscope}%
\pgfpathrectangle{\pgfqpoint{0.211875in}{0.211875in}}{\pgfqpoint{1.313625in}{1.279725in}}%
\pgfusepath{clip}%
\pgfsetbuttcap%
\pgfsetroundjoin%
\definecolor{currentfill}{rgb}{0.961115,0.566634,0.405693}%
\pgfsetfillcolor{currentfill}%
\pgfsetlinewidth{0.000000pt}%
\definecolor{currentstroke}{rgb}{0.000000,0.000000,0.000000}%
\pgfsetstrokecolor{currentstroke}%
\pgfsetdash{}{0pt}%
\pgfpathmoveto{\pgfqpoint{0.384371in}{1.344636in}}%
\pgfpathlineto{\pgfqpoint{0.396438in}{1.349408in}}%
\pgfpathlineto{\pgfqpoint{0.397640in}{1.350512in}}%
\pgfpathlineto{\pgfqpoint{0.401451in}{1.362335in}}%
\pgfpathlineto{\pgfqpoint{0.401549in}{1.375261in}}%
\pgfpathlineto{\pgfqpoint{0.397731in}{1.388188in}}%
\pgfpathlineto{\pgfqpoint{0.397640in}{1.388317in}}%
\pgfpathlineto{\pgfqpoint{0.384371in}{1.393994in}}%
\pgfpathlineto{\pgfqpoint{0.371102in}{1.388433in}}%
\pgfpathlineto{\pgfqpoint{0.370926in}{1.388188in}}%
\pgfpathlineto{\pgfqpoint{0.366962in}{1.375261in}}%
\pgfpathlineto{\pgfqpoint{0.367072in}{1.362335in}}%
\pgfpathlineto{\pgfqpoint{0.371102in}{1.350370in}}%
\pgfpathlineto{\pgfqpoint{0.372219in}{1.349408in}}%
\pgfpathclose%
\pgfpathmoveto{\pgfqpoint{0.378352in}{1.362335in}}%
\pgfpathlineto{\pgfqpoint{0.377784in}{1.375261in}}%
\pgfpathlineto{\pgfqpoint{0.384371in}{1.381678in}}%
\pgfpathlineto{\pgfqpoint{0.390615in}{1.375261in}}%
\pgfpathlineto{\pgfqpoint{0.390089in}{1.362335in}}%
\pgfpathlineto{\pgfqpoint{0.384371in}{1.357000in}}%
\pgfpathclose%
\pgfusepath{fill}%
\end{pgfscope}%
\begin{pgfscope}%
\pgfpathrectangle{\pgfqpoint{0.211875in}{0.211875in}}{\pgfqpoint{1.313625in}{1.279725in}}%
\pgfusepath{clip}%
\pgfsetbuttcap%
\pgfsetroundjoin%
\definecolor{currentfill}{rgb}{0.961115,0.566634,0.405693}%
\pgfsetfillcolor{currentfill}%
\pgfsetlinewidth{0.000000pt}%
\definecolor{currentstroke}{rgb}{0.000000,0.000000,0.000000}%
\pgfsetstrokecolor{currentstroke}%
\pgfsetdash{}{0pt}%
\pgfpathmoveto{\pgfqpoint{0.490523in}{1.344149in}}%
\pgfpathlineto{\pgfqpoint{0.503792in}{1.341032in}}%
\pgfpathlineto{\pgfqpoint{0.517061in}{1.346607in}}%
\pgfpathlineto{\pgfqpoint{0.518922in}{1.349408in}}%
\pgfpathlineto{\pgfqpoint{0.522107in}{1.362335in}}%
\pgfpathlineto{\pgfqpoint{0.522158in}{1.375261in}}%
\pgfpathlineto{\pgfqpoint{0.519264in}{1.388188in}}%
\pgfpathlineto{\pgfqpoint{0.517061in}{1.391746in}}%
\pgfpathlineto{\pgfqpoint{0.503792in}{1.397663in}}%
\pgfpathlineto{\pgfqpoint{0.490523in}{1.394372in}}%
\pgfpathlineto{\pgfqpoint{0.485533in}{1.388188in}}%
\pgfpathlineto{\pgfqpoint{0.482163in}{1.375261in}}%
\pgfpathlineto{\pgfqpoint{0.482233in}{1.362335in}}%
\pgfpathlineto{\pgfqpoint{0.485958in}{1.349408in}}%
\pgfpathclose%
\pgfpathmoveto{\pgfqpoint{0.489438in}{1.362335in}}%
\pgfpathlineto{\pgfqpoint{0.489252in}{1.375261in}}%
\pgfpathlineto{\pgfqpoint{0.490523in}{1.378925in}}%
\pgfpathlineto{\pgfqpoint{0.503737in}{1.388188in}}%
\pgfpathlineto{\pgfqpoint{0.503792in}{1.388206in}}%
\pgfpathlineto{\pgfqpoint{0.503825in}{1.388188in}}%
\pgfpathlineto{\pgfqpoint{0.514108in}{1.375261in}}%
\pgfpathlineto{\pgfqpoint{0.513768in}{1.362335in}}%
\pgfpathlineto{\pgfqpoint{0.503792in}{1.350950in}}%
\pgfpathlineto{\pgfqpoint{0.490523in}{1.359467in}}%
\pgfpathclose%
\pgfusepath{fill}%
\end{pgfscope}%
\begin{pgfscope}%
\pgfpathrectangle{\pgfqpoint{0.211875in}{0.211875in}}{\pgfqpoint{1.313625in}{1.279725in}}%
\pgfusepath{clip}%
\pgfsetbuttcap%
\pgfsetroundjoin%
\definecolor{currentfill}{rgb}{0.961115,0.566634,0.405693}%
\pgfsetfillcolor{currentfill}%
\pgfsetlinewidth{0.000000pt}%
\definecolor{currentstroke}{rgb}{0.000000,0.000000,0.000000}%
\pgfsetstrokecolor{currentstroke}%
\pgfsetdash{}{0pt}%
\pgfpathmoveto{\pgfqpoint{0.609943in}{1.339753in}}%
\pgfpathlineto{\pgfqpoint{0.623212in}{1.338083in}}%
\pgfpathlineto{\pgfqpoint{0.636481in}{1.343640in}}%
\pgfpathlineto{\pgfqpoint{0.639844in}{1.349408in}}%
\pgfpathlineto{\pgfqpoint{0.642182in}{1.362335in}}%
\pgfpathlineto{\pgfqpoint{0.642198in}{1.375261in}}%
\pgfpathlineto{\pgfqpoint{0.640007in}{1.388188in}}%
\pgfpathlineto{\pgfqpoint{0.636481in}{1.394655in}}%
\pgfpathlineto{\pgfqpoint{0.623212in}{1.400659in}}%
\pgfpathlineto{\pgfqpoint{0.609943in}{1.398871in}}%
\pgfpathlineto{\pgfqpoint{0.600266in}{1.388188in}}%
\pgfpathlineto{\pgfqpoint{0.597406in}{1.375261in}}%
\pgfpathlineto{\pgfqpoint{0.597439in}{1.362335in}}%
\pgfpathlineto{\pgfqpoint{0.600520in}{1.349408in}}%
\pgfpathclose%
\pgfpathmoveto{\pgfqpoint{0.613966in}{1.349408in}}%
\pgfpathlineto{\pgfqpoint{0.609943in}{1.351023in}}%
\pgfpathlineto{\pgfqpoint{0.605133in}{1.362335in}}%
\pgfpathlineto{\pgfqpoint{0.604977in}{1.375261in}}%
\pgfpathlineto{\pgfqpoint{0.609943in}{1.388033in}}%
\pgfpathlineto{\pgfqpoint{0.610284in}{1.388188in}}%
\pgfpathlineto{\pgfqpoint{0.623212in}{1.391085in}}%
\pgfpathlineto{\pgfqpoint{0.627817in}{1.388188in}}%
\pgfpathlineto{\pgfqpoint{0.635421in}{1.375261in}}%
\pgfpathlineto{\pgfqpoint{0.635195in}{1.362335in}}%
\pgfpathlineto{\pgfqpoint{0.626536in}{1.349408in}}%
\pgfpathlineto{\pgfqpoint{0.623212in}{1.347481in}}%
\pgfpathclose%
\pgfusepath{fill}%
\end{pgfscope}%
\begin{pgfscope}%
\pgfpathrectangle{\pgfqpoint{0.211875in}{0.211875in}}{\pgfqpoint{1.313625in}{1.279725in}}%
\pgfusepath{clip}%
\pgfsetbuttcap%
\pgfsetroundjoin%
\definecolor{currentfill}{rgb}{0.961115,0.566634,0.405693}%
\pgfsetfillcolor{currentfill}%
\pgfsetlinewidth{0.000000pt}%
\definecolor{currentstroke}{rgb}{0.000000,0.000000,0.000000}%
\pgfsetstrokecolor{currentstroke}%
\pgfsetdash{}{0pt}%
\pgfpathmoveto{\pgfqpoint{0.437447in}{1.425457in}}%
\pgfpathlineto{\pgfqpoint{0.450716in}{1.425910in}}%
\pgfpathlineto{\pgfqpoint{0.452302in}{1.426967in}}%
\pgfpathlineto{\pgfqpoint{0.460227in}{1.439894in}}%
\pgfpathlineto{\pgfqpoint{0.461405in}{1.452820in}}%
\pgfpathlineto{\pgfqpoint{0.458434in}{1.465747in}}%
\pgfpathlineto{\pgfqpoint{0.450716in}{1.474546in}}%
\pgfpathlineto{\pgfqpoint{0.437447in}{1.475145in}}%
\pgfpathlineto{\pgfqpoint{0.427841in}{1.465747in}}%
\pgfpathlineto{\pgfqpoint{0.424338in}{1.452820in}}%
\pgfpathlineto{\pgfqpoint{0.425757in}{1.439894in}}%
\pgfpathlineto{\pgfqpoint{0.434873in}{1.426967in}}%
\pgfpathclose%
\pgfpathmoveto{\pgfqpoint{0.436029in}{1.439894in}}%
\pgfpathlineto{\pgfqpoint{0.433756in}{1.452820in}}%
\pgfpathlineto{\pgfqpoint{0.437447in}{1.462402in}}%
\pgfpathlineto{\pgfqpoint{0.450716in}{1.459739in}}%
\pgfpathlineto{\pgfqpoint{0.452980in}{1.452820in}}%
\pgfpathlineto{\pgfqpoint{0.450986in}{1.439894in}}%
\pgfpathlineto{\pgfqpoint{0.450716in}{1.439482in}}%
\pgfpathlineto{\pgfqpoint{0.437447in}{1.438054in}}%
\pgfpathclose%
\pgfusepath{fill}%
\end{pgfscope}%
\begin{pgfscope}%
\pgfpathrectangle{\pgfqpoint{0.211875in}{0.211875in}}{\pgfqpoint{1.313625in}{1.279725in}}%
\pgfusepath{clip}%
\pgfsetbuttcap%
\pgfsetroundjoin%
\definecolor{currentfill}{rgb}{0.961115,0.566634,0.405693}%
\pgfsetfillcolor{currentfill}%
\pgfsetlinewidth{0.000000pt}%
\definecolor{currentstroke}{rgb}{0.000000,0.000000,0.000000}%
\pgfsetstrokecolor{currentstroke}%
\pgfsetdash{}{0pt}%
\pgfpathmoveto{\pgfqpoint{0.556867in}{1.421907in}}%
\pgfpathlineto{\pgfqpoint{0.570136in}{1.423017in}}%
\pgfpathlineto{\pgfqpoint{0.575357in}{1.426967in}}%
\pgfpathlineto{\pgfqpoint{0.581040in}{1.439894in}}%
\pgfpathlineto{\pgfqpoint{0.581863in}{1.452820in}}%
\pgfpathlineto{\pgfqpoint{0.579655in}{1.465747in}}%
\pgfpathlineto{\pgfqpoint{0.570136in}{1.478013in}}%
\pgfpathlineto{\pgfqpoint{0.564521in}{1.478673in}}%
\pgfpathlineto{\pgfqpoint{0.556867in}{1.479285in}}%
\pgfpathlineto{\pgfqpoint{0.555240in}{1.478673in}}%
\pgfpathlineto{\pgfqpoint{0.543598in}{1.469613in}}%
\pgfpathlineto{\pgfqpoint{0.541977in}{1.465747in}}%
\pgfpathlineto{\pgfqpoint{0.540168in}{1.452820in}}%
\pgfpathlineto{\pgfqpoint{0.540861in}{1.439894in}}%
\pgfpathlineto{\pgfqpoint{0.543598in}{1.431028in}}%
\pgfpathlineto{\pgfqpoint{0.546927in}{1.426967in}}%
\pgfpathclose%
\pgfpathmoveto{\pgfqpoint{0.550311in}{1.439894in}}%
\pgfpathlineto{\pgfqpoint{0.548100in}{1.452820in}}%
\pgfpathlineto{\pgfqpoint{0.553276in}{1.465747in}}%
\pgfpathlineto{\pgfqpoint{0.556867in}{1.468799in}}%
\pgfpathlineto{\pgfqpoint{0.570136in}{1.466070in}}%
\pgfpathlineto{\pgfqpoint{0.570387in}{1.465747in}}%
\pgfpathlineto{\pgfqpoint{0.574097in}{1.452820in}}%
\pgfpathlineto{\pgfqpoint{0.572527in}{1.439894in}}%
\pgfpathlineto{\pgfqpoint{0.570136in}{1.435769in}}%
\pgfpathlineto{\pgfqpoint{0.556867in}{1.432507in}}%
\pgfpathclose%
\pgfusepath{fill}%
\end{pgfscope}%
\begin{pgfscope}%
\pgfpathrectangle{\pgfqpoint{0.211875in}{0.211875in}}{\pgfqpoint{1.313625in}{1.279725in}}%
\pgfusepath{clip}%
\pgfsetbuttcap%
\pgfsetroundjoin%
\definecolor{currentfill}{rgb}{0.961115,0.566634,0.405693}%
\pgfsetfillcolor{currentfill}%
\pgfsetlinewidth{0.000000pt}%
\definecolor{currentstroke}{rgb}{0.000000,0.000000,0.000000}%
\pgfsetstrokecolor{currentstroke}%
\pgfsetdash{}{0pt}%
\pgfpathmoveto{\pgfqpoint{0.663019in}{1.423908in}}%
\pgfpathlineto{\pgfqpoint{0.676288in}{1.419136in}}%
\pgfpathlineto{\pgfqpoint{0.689557in}{1.420639in}}%
\pgfpathlineto{\pgfqpoint{0.696955in}{1.426967in}}%
\pgfpathlineto{\pgfqpoint{0.701052in}{1.439894in}}%
\pgfpathlineto{\pgfqpoint{0.701622in}{1.452820in}}%
\pgfpathlineto{\pgfqpoint{0.699960in}{1.465747in}}%
\pgfpathlineto{\pgfqpoint{0.692227in}{1.478673in}}%
\pgfpathlineto{\pgfqpoint{0.689557in}{1.480320in}}%
\pgfpathlineto{\pgfqpoint{0.676288in}{1.481859in}}%
\pgfpathlineto{\pgfqpoint{0.666344in}{1.478673in}}%
\pgfpathlineto{\pgfqpoint{0.663019in}{1.476638in}}%
\pgfpathlineto{\pgfqpoint{0.657842in}{1.465747in}}%
\pgfpathlineto{\pgfqpoint{0.656324in}{1.452820in}}%
\pgfpathlineto{\pgfqpoint{0.656865in}{1.439894in}}%
\pgfpathlineto{\pgfqpoint{0.660608in}{1.426967in}}%
\pgfpathclose%
\pgfpathmoveto{\pgfqpoint{0.664137in}{1.439894in}}%
\pgfpathlineto{\pgfqpoint{0.663019in}{1.446126in}}%
\pgfpathlineto{\pgfqpoint{0.662469in}{1.452820in}}%
\pgfpathlineto{\pgfqpoint{0.663019in}{1.455874in}}%
\pgfpathlineto{\pgfqpoint{0.667227in}{1.465747in}}%
\pgfpathlineto{\pgfqpoint{0.676288in}{1.472318in}}%
\pgfpathlineto{\pgfqpoint{0.689557in}{1.468305in}}%
\pgfpathlineto{\pgfqpoint{0.691320in}{1.465747in}}%
\pgfpathlineto{\pgfqpoint{0.694376in}{1.452820in}}%
\pgfpathlineto{\pgfqpoint{0.693112in}{1.439894in}}%
\pgfpathlineto{\pgfqpoint{0.689557in}{1.432979in}}%
\pgfpathlineto{\pgfqpoint{0.676288in}{1.428213in}}%
\pgfpathclose%
\pgfusepath{fill}%
\end{pgfscope}%
\begin{pgfscope}%
\pgfpathrectangle{\pgfqpoint{0.211875in}{0.211875in}}{\pgfqpoint{1.313625in}{1.279725in}}%
\pgfusepath{clip}%
\pgfsetbuttcap%
\pgfsetroundjoin%
\definecolor{currentfill}{rgb}{0.961115,0.566634,0.405693}%
\pgfsetfillcolor{currentfill}%
\pgfsetlinewidth{0.000000pt}%
\definecolor{currentstroke}{rgb}{0.000000,0.000000,0.000000}%
\pgfsetstrokecolor{currentstroke}%
\pgfsetdash{}{0pt}%
\pgfpathmoveto{\pgfqpoint{0.782439in}{1.419958in}}%
\pgfpathlineto{\pgfqpoint{0.795708in}{1.417061in}}%
\pgfpathlineto{\pgfqpoint{0.808977in}{1.418804in}}%
\pgfpathlineto{\pgfqpoint{0.817433in}{1.426967in}}%
\pgfpathlineto{\pgfqpoint{0.820436in}{1.439894in}}%
\pgfpathlineto{\pgfqpoint{0.820832in}{1.452820in}}%
\pgfpathlineto{\pgfqpoint{0.819550in}{1.465747in}}%
\pgfpathlineto{\pgfqpoint{0.813686in}{1.478673in}}%
\pgfpathlineto{\pgfqpoint{0.808977in}{1.481950in}}%
\pgfpathlineto{\pgfqpoint{0.795708in}{1.483785in}}%
\pgfpathlineto{\pgfqpoint{0.782439in}{1.480821in}}%
\pgfpathlineto{\pgfqpoint{0.779801in}{1.478673in}}%
\pgfpathlineto{\pgfqpoint{0.774007in}{1.465747in}}%
\pgfpathlineto{\pgfqpoint{0.772715in}{1.452820in}}%
\pgfpathlineto{\pgfqpoint{0.773134in}{1.439894in}}%
\pgfpathlineto{\pgfqpoint{0.776185in}{1.426967in}}%
\pgfpathclose%
\pgfpathmoveto{\pgfqpoint{0.792114in}{1.426967in}}%
\pgfpathlineto{\pgfqpoint{0.782439in}{1.434076in}}%
\pgfpathlineto{\pgfqpoint{0.780124in}{1.439894in}}%
\pgfpathlineto{\pgfqpoint{0.779111in}{1.452820in}}%
\pgfpathlineto{\pgfqpoint{0.781595in}{1.465747in}}%
\pgfpathlineto{\pgfqpoint{0.782439in}{1.467324in}}%
\pgfpathlineto{\pgfqpoint{0.795708in}{1.474916in}}%
\pgfpathlineto{\pgfqpoint{0.808977in}{1.469710in}}%
\pgfpathlineto{\pgfqpoint{0.811409in}{1.465747in}}%
\pgfpathlineto{\pgfqpoint{0.813998in}{1.452820in}}%
\pgfpathlineto{\pgfqpoint{0.812951in}{1.439894in}}%
\pgfpathlineto{\pgfqpoint{0.808977in}{1.431190in}}%
\pgfpathlineto{\pgfqpoint{0.800599in}{1.426967in}}%
\pgfpathlineto{\pgfqpoint{0.795708in}{1.425672in}}%
\pgfpathclose%
\pgfusepath{fill}%
\end{pgfscope}%
\begin{pgfscope}%
\pgfpathrectangle{\pgfqpoint{0.211875in}{0.211875in}}{\pgfqpoint{1.313625in}{1.279725in}}%
\pgfusepath{clip}%
\pgfsetbuttcap%
\pgfsetroundjoin%
\definecolor{currentfill}{rgb}{0.961115,0.566634,0.405693}%
\pgfsetfillcolor{currentfill}%
\pgfsetlinewidth{0.000000pt}%
\definecolor{currentstroke}{rgb}{0.000000,0.000000,0.000000}%
\pgfsetstrokecolor{currentstroke}%
\pgfsetdash{}{0pt}%
\pgfpathmoveto{\pgfqpoint{0.901860in}{1.417368in}}%
\pgfpathlineto{\pgfqpoint{0.915129in}{1.415626in}}%
\pgfpathlineto{\pgfqpoint{0.928398in}{1.417573in}}%
\pgfpathlineto{\pgfqpoint{0.937010in}{1.426967in}}%
\pgfpathlineto{\pgfqpoint{0.939304in}{1.439894in}}%
\pgfpathlineto{\pgfqpoint{0.939588in}{1.452820in}}%
\pgfpathlineto{\pgfqpoint{0.938555in}{1.465747in}}%
\pgfpathlineto{\pgfqpoint{0.933910in}{1.478673in}}%
\pgfpathlineto{\pgfqpoint{0.928398in}{1.483004in}}%
\pgfpathlineto{\pgfqpoint{0.915129in}{1.485113in}}%
\pgfpathlineto{\pgfqpoint{0.901860in}{1.483271in}}%
\pgfpathlineto{\pgfqpoint{0.895495in}{1.478673in}}%
\pgfpathlineto{\pgfqpoint{0.890454in}{1.465747in}}%
\pgfpathlineto{\pgfqpoint{0.889320in}{1.452820in}}%
\pgfpathlineto{\pgfqpoint{0.889648in}{1.439894in}}%
\pgfpathlineto{\pgfqpoint{0.892199in}{1.426967in}}%
\pgfpathclose%
\pgfpathmoveto{\pgfqpoint{0.905631in}{1.426967in}}%
\pgfpathlineto{\pgfqpoint{0.901860in}{1.429027in}}%
\pgfpathlineto{\pgfqpoint{0.896987in}{1.439894in}}%
\pgfpathlineto{\pgfqpoint{0.896033in}{1.452820in}}%
\pgfpathlineto{\pgfqpoint{0.898423in}{1.465747in}}%
\pgfpathlineto{\pgfqpoint{0.901860in}{1.471462in}}%
\pgfpathlineto{\pgfqpoint{0.915129in}{1.476660in}}%
\pgfpathlineto{\pgfqpoint{0.928398in}{1.470168in}}%
\pgfpathlineto{\pgfqpoint{0.930810in}{1.465747in}}%
\pgfpathlineto{\pgfqpoint{0.933082in}{1.452820in}}%
\pgfpathlineto{\pgfqpoint{0.932181in}{1.439894in}}%
\pgfpathlineto{\pgfqpoint{0.928398in}{1.430549in}}%
\pgfpathlineto{\pgfqpoint{0.923207in}{1.426967in}}%
\pgfpathlineto{\pgfqpoint{0.915129in}{1.424239in}}%
\pgfpathclose%
\pgfusepath{fill}%
\end{pgfscope}%
\begin{pgfscope}%
\pgfpathrectangle{\pgfqpoint{0.211875in}{0.211875in}}{\pgfqpoint{1.313625in}{1.279725in}}%
\pgfusepath{clip}%
\pgfsetbuttcap%
\pgfsetroundjoin%
\definecolor{currentfill}{rgb}{0.961115,0.566634,0.405693}%
\pgfsetfillcolor{currentfill}%
\pgfsetlinewidth{0.000000pt}%
\definecolor{currentstroke}{rgb}{0.000000,0.000000,0.000000}%
\pgfsetstrokecolor{currentstroke}%
\pgfsetdash{}{0pt}%
\pgfpathmoveto{\pgfqpoint{1.021280in}{1.415836in}}%
\pgfpathlineto{\pgfqpoint{1.034549in}{1.414805in}}%
\pgfpathlineto{\pgfqpoint{1.047818in}{1.417060in}}%
\pgfpathlineto{\pgfqpoint{1.055830in}{1.426967in}}%
\pgfpathlineto{\pgfqpoint{1.057728in}{1.439894in}}%
\pgfpathlineto{\pgfqpoint{1.057951in}{1.452820in}}%
\pgfpathlineto{\pgfqpoint{1.057060in}{1.465747in}}%
\pgfpathlineto{\pgfqpoint{1.053105in}{1.478673in}}%
\pgfpathlineto{\pgfqpoint{1.047818in}{1.483377in}}%
\pgfpathlineto{\pgfqpoint{1.034549in}{1.485866in}}%
\pgfpathlineto{\pgfqpoint{1.021280in}{1.484742in}}%
\pgfpathlineto{\pgfqpoint{1.011820in}{1.478673in}}%
\pgfpathlineto{\pgfqpoint{1.008011in}{1.469343in}}%
\pgfpathlineto{\pgfqpoint{1.007335in}{1.465747in}}%
\pgfpathlineto{\pgfqpoint{1.006484in}{1.452820in}}%
\pgfpathlineto{\pgfqpoint{1.006705in}{1.439894in}}%
\pgfpathlineto{\pgfqpoint{1.008011in}{1.429693in}}%
\pgfpathlineto{\pgfqpoint{1.008662in}{1.426967in}}%
\pgfpathclose%
\pgfpathmoveto{\pgfqpoint{1.020396in}{1.426967in}}%
\pgfpathlineto{\pgfqpoint{1.014175in}{1.439894in}}%
\pgfpathlineto{\pgfqpoint{1.013236in}{1.452820in}}%
\pgfpathlineto{\pgfqpoint{1.015624in}{1.465747in}}%
\pgfpathlineto{\pgfqpoint{1.021280in}{1.474128in}}%
\pgfpathlineto{\pgfqpoint{1.034549in}{1.477583in}}%
\pgfpathlineto{\pgfqpoint{1.047818in}{1.469489in}}%
\pgfpathlineto{\pgfqpoint{1.049629in}{1.465747in}}%
\pgfpathlineto{\pgfqpoint{1.051704in}{1.452820in}}%
\pgfpathlineto{\pgfqpoint{1.050891in}{1.439894in}}%
\pgfpathlineto{\pgfqpoint{1.047818in}{1.431299in}}%
\pgfpathlineto{\pgfqpoint{1.043042in}{1.426967in}}%
\pgfpathlineto{\pgfqpoint{1.034549in}{1.423475in}}%
\pgfpathlineto{\pgfqpoint{1.021280in}{1.426187in}}%
\pgfpathclose%
\pgfusepath{fill}%
\end{pgfscope}%
\begin{pgfscope}%
\pgfpathrectangle{\pgfqpoint{0.211875in}{0.211875in}}{\pgfqpoint{1.313625in}{1.279725in}}%
\pgfusepath{clip}%
\pgfsetbuttcap%
\pgfsetroundjoin%
\definecolor{currentfill}{rgb}{0.961115,0.566634,0.405693}%
\pgfsetfillcolor{currentfill}%
\pgfsetlinewidth{0.000000pt}%
\definecolor{currentstroke}{rgb}{0.000000,0.000000,0.000000}%
\pgfsetstrokecolor{currentstroke}%
\pgfsetdash{}{0pt}%
\pgfpathmoveto{\pgfqpoint{1.127432in}{1.423338in}}%
\pgfpathlineto{\pgfqpoint{1.140701in}{1.415164in}}%
\pgfpathlineto{\pgfqpoint{1.153970in}{1.414594in}}%
\pgfpathlineto{\pgfqpoint{1.167239in}{1.417450in}}%
\pgfpathlineto{\pgfqpoint{1.173982in}{1.426967in}}%
\pgfpathlineto{\pgfqpoint{1.175752in}{1.439894in}}%
\pgfpathlineto{\pgfqpoint{1.175957in}{1.452820in}}%
\pgfpathlineto{\pgfqpoint{1.175116in}{1.465747in}}%
\pgfpathlineto{\pgfqpoint{1.171408in}{1.478673in}}%
\pgfpathlineto{\pgfqpoint{1.167239in}{1.482897in}}%
\pgfpathlineto{\pgfqpoint{1.153970in}{1.486050in}}%
\pgfpathlineto{\pgfqpoint{1.140701in}{1.485412in}}%
\pgfpathlineto{\pgfqpoint{1.128851in}{1.478673in}}%
\pgfpathlineto{\pgfqpoint{1.127432in}{1.476108in}}%
\pgfpathlineto{\pgfqpoint{1.124973in}{1.465747in}}%
\pgfpathlineto{\pgfqpoint{1.124172in}{1.452820in}}%
\pgfpathlineto{\pgfqpoint{1.124368in}{1.439894in}}%
\pgfpathlineto{\pgfqpoint{1.126054in}{1.426967in}}%
\pgfpathclose%
\pgfpathmoveto{\pgfqpoint{1.138204in}{1.426967in}}%
\pgfpathlineto{\pgfqpoint{1.131711in}{1.439894in}}%
\pgfpathlineto{\pgfqpoint{1.130737in}{1.452820in}}%
\pgfpathlineto{\pgfqpoint{1.133232in}{1.465747in}}%
\pgfpathlineto{\pgfqpoint{1.140701in}{1.475595in}}%
\pgfpathlineto{\pgfqpoint{1.153970in}{1.477688in}}%
\pgfpathlineto{\pgfqpoint{1.167239in}{1.467365in}}%
\pgfpathlineto{\pgfqpoint{1.167929in}{1.465747in}}%
\pgfpathlineto{\pgfqpoint{1.169911in}{1.452820in}}%
\pgfpathlineto{\pgfqpoint{1.169137in}{1.439894in}}%
\pgfpathlineto{\pgfqpoint{1.167239in}{1.433840in}}%
\pgfpathlineto{\pgfqpoint{1.161351in}{1.426967in}}%
\pgfpathlineto{\pgfqpoint{1.153970in}{1.423379in}}%
\pgfpathlineto{\pgfqpoint{1.140701in}{1.425014in}}%
\pgfpathclose%
\pgfusepath{fill}%
\end{pgfscope}%
\begin{pgfscope}%
\pgfpathrectangle{\pgfqpoint{0.211875in}{0.211875in}}{\pgfqpoint{1.313625in}{1.279725in}}%
\pgfusepath{clip}%
\pgfsetbuttcap%
\pgfsetroundjoin%
\definecolor{currentfill}{rgb}{0.961115,0.566634,0.405693}%
\pgfsetfillcolor{currentfill}%
\pgfsetlinewidth{0.000000pt}%
\definecolor{currentstroke}{rgb}{0.000000,0.000000,0.000000}%
\pgfsetstrokecolor{currentstroke}%
\pgfsetdash{}{0pt}%
\pgfpathmoveto{\pgfqpoint{1.246852in}{1.421355in}}%
\pgfpathlineto{\pgfqpoint{1.260121in}{1.415221in}}%
\pgfpathlineto{\pgfqpoint{1.273390in}{1.415011in}}%
\pgfpathlineto{\pgfqpoint{1.286659in}{1.419057in}}%
\pgfpathlineto{\pgfqpoint{1.291519in}{1.426967in}}%
\pgfpathlineto{\pgfqpoint{1.293395in}{1.439894in}}%
\pgfpathlineto{\pgfqpoint{1.293622in}{1.452820in}}%
\pgfpathlineto{\pgfqpoint{1.292750in}{1.465747in}}%
\pgfpathlineto{\pgfqpoint{1.288899in}{1.478673in}}%
\pgfpathlineto{\pgfqpoint{1.286659in}{1.481280in}}%
\pgfpathlineto{\pgfqpoint{1.273390in}{1.485645in}}%
\pgfpathlineto{\pgfqpoint{1.260121in}{1.485400in}}%
\pgfpathlineto{\pgfqpoint{1.246852in}{1.478813in}}%
\pgfpathlineto{\pgfqpoint{1.246760in}{1.478673in}}%
\pgfpathlineto{\pgfqpoint{1.243090in}{1.465747in}}%
\pgfpathlineto{\pgfqpoint{1.242262in}{1.452820in}}%
\pgfpathlineto{\pgfqpoint{1.242469in}{1.439894in}}%
\pgfpathlineto{\pgfqpoint{1.244232in}{1.426967in}}%
\pgfpathclose%
\pgfpathmoveto{\pgfqpoint{1.256806in}{1.426967in}}%
\pgfpathlineto{\pgfqpoint{1.249646in}{1.439894in}}%
\pgfpathlineto{\pgfqpoint{1.248574in}{1.452820in}}%
\pgfpathlineto{\pgfqpoint{1.251313in}{1.465747in}}%
\pgfpathlineto{\pgfqpoint{1.260121in}{1.476044in}}%
\pgfpathlineto{\pgfqpoint{1.273390in}{1.476944in}}%
\pgfpathlineto{\pgfqpoint{1.285168in}{1.465747in}}%
\pgfpathlineto{\pgfqpoint{1.286659in}{1.460343in}}%
\pgfpathlineto{\pgfqpoint{1.287727in}{1.452820in}}%
\pgfpathlineto{\pgfqpoint{1.286949in}{1.439894in}}%
\pgfpathlineto{\pgfqpoint{1.286659in}{1.438829in}}%
\pgfpathlineto{\pgfqpoint{1.278698in}{1.426967in}}%
\pgfpathlineto{\pgfqpoint{1.273390in}{1.423976in}}%
\pgfpathlineto{\pgfqpoint{1.260121in}{1.424676in}}%
\pgfpathclose%
\pgfusepath{fill}%
\end{pgfscope}%
\begin{pgfscope}%
\pgfpathrectangle{\pgfqpoint{0.211875in}{0.211875in}}{\pgfqpoint{1.313625in}{1.279725in}}%
\pgfusepath{clip}%
\pgfsetbuttcap%
\pgfsetroundjoin%
\definecolor{currentfill}{rgb}{0.961115,0.566634,0.405693}%
\pgfsetfillcolor{currentfill}%
\pgfsetlinewidth{0.000000pt}%
\definecolor{currentstroke}{rgb}{0.000000,0.000000,0.000000}%
\pgfsetstrokecolor{currentstroke}%
\pgfsetdash{}{0pt}%
\pgfpathmoveto{\pgfqpoint{1.366273in}{1.421218in}}%
\pgfpathlineto{\pgfqpoint{1.379542in}{1.415926in}}%
\pgfpathlineto{\pgfqpoint{1.392811in}{1.416105in}}%
\pgfpathlineto{\pgfqpoint{1.406080in}{1.422422in}}%
\pgfpathlineto{\pgfqpoint{1.408464in}{1.426967in}}%
\pgfpathlineto{\pgfqpoint{1.410664in}{1.439894in}}%
\pgfpathlineto{\pgfqpoint{1.410947in}{1.452820in}}%
\pgfpathlineto{\pgfqpoint{1.409969in}{1.465747in}}%
\pgfpathlineto{\pgfqpoint{1.406080in}{1.477856in}}%
\pgfpathlineto{\pgfqpoint{1.405390in}{1.478673in}}%
\pgfpathlineto{\pgfqpoint{1.392811in}{1.484609in}}%
\pgfpathlineto{\pgfqpoint{1.379542in}{1.484781in}}%
\pgfpathlineto{\pgfqpoint{1.366273in}{1.479182in}}%
\pgfpathlineto{\pgfqpoint{1.365873in}{1.478673in}}%
\pgfpathlineto{\pgfqpoint{1.361679in}{1.465747in}}%
\pgfpathlineto{\pgfqpoint{1.360745in}{1.452820in}}%
\pgfpathlineto{\pgfqpoint{1.361002in}{1.439894in}}%
\pgfpathlineto{\pgfqpoint{1.363074in}{1.426967in}}%
\pgfpathclose%
\pgfpathmoveto{\pgfqpoint{1.376419in}{1.426967in}}%
\pgfpathlineto{\pgfqpoint{1.368070in}{1.439894in}}%
\pgfpathlineto{\pgfqpoint{1.366816in}{1.452820in}}%
\pgfpathlineto{\pgfqpoint{1.369979in}{1.465747in}}%
\pgfpathlineto{\pgfqpoint{1.379542in}{1.475592in}}%
\pgfpathlineto{\pgfqpoint{1.392811in}{1.475290in}}%
\pgfpathlineto{\pgfqpoint{1.401607in}{1.465747in}}%
\pgfpathlineto{\pgfqpoint{1.404699in}{1.452820in}}%
\pgfpathlineto{\pgfqpoint{1.403464in}{1.439894in}}%
\pgfpathlineto{\pgfqpoint{1.395369in}{1.426967in}}%
\pgfpathlineto{\pgfqpoint{1.392811in}{1.425317in}}%
\pgfpathlineto{\pgfqpoint{1.379542in}{1.425074in}}%
\pgfpathclose%
\pgfusepath{fill}%
\end{pgfscope}%
\begin{pgfscope}%
\pgfpathrectangle{\pgfqpoint{0.211875in}{0.211875in}}{\pgfqpoint{1.313625in}{1.279725in}}%
\pgfusepath{clip}%
\pgfsetbuttcap%
\pgfsetroundjoin%
\definecolor{currentfill}{rgb}{0.961115,0.566634,0.405693}%
\pgfsetfillcolor{currentfill}%
\pgfsetlinewidth{0.000000pt}%
\definecolor{currentstroke}{rgb}{0.000000,0.000000,0.000000}%
\pgfsetstrokecolor{currentstroke}%
\pgfsetdash{}{0pt}%
\pgfpathmoveto{\pgfqpoint{1.485693in}{1.422220in}}%
\pgfpathlineto{\pgfqpoint{1.498962in}{1.417230in}}%
\pgfpathlineto{\pgfqpoint{1.512231in}{1.417954in}}%
\pgfpathlineto{\pgfqpoint{1.524549in}{1.426967in}}%
\pgfpathlineto{\pgfqpoint{1.525500in}{1.429323in}}%
\pgfpathlineto{\pgfqpoint{1.525500in}{1.439894in}}%
\pgfpathlineto{\pgfqpoint{1.525500in}{1.452820in}}%
\pgfpathlineto{\pgfqpoint{1.525500in}{1.465747in}}%
\pgfpathlineto{\pgfqpoint{1.525500in}{1.470407in}}%
\pgfpathlineto{\pgfqpoint{1.520065in}{1.478673in}}%
\pgfpathlineto{\pgfqpoint{1.512231in}{1.482869in}}%
\pgfpathlineto{\pgfqpoint{1.498962in}{1.483598in}}%
\pgfpathlineto{\pgfqpoint{1.486034in}{1.478673in}}%
\pgfpathlineto{\pgfqpoint{1.485693in}{1.478385in}}%
\pgfpathlineto{\pgfqpoint{1.480754in}{1.465747in}}%
\pgfpathlineto{\pgfqpoint{1.479628in}{1.452820in}}%
\pgfpathlineto{\pgfqpoint{1.479976in}{1.439894in}}%
\pgfpathlineto{\pgfqpoint{1.482611in}{1.426967in}}%
\pgfpathclose%
\pgfpathmoveto{\pgfqpoint{1.497405in}{1.426967in}}%
\pgfpathlineto{\pgfqpoint{1.487135in}{1.439894in}}%
\pgfpathlineto{\pgfqpoint{1.485693in}{1.451835in}}%
\pgfpathlineto{\pgfqpoint{1.485629in}{1.452820in}}%
\pgfpathlineto{\pgfqpoint{1.485693in}{1.453254in}}%
\pgfpathlineto{\pgfqpoint{1.489422in}{1.465747in}}%
\pgfpathlineto{\pgfqpoint{1.498962in}{1.474310in}}%
\pgfpathlineto{\pgfqpoint{1.512231in}{1.472620in}}%
\pgfpathlineto{\pgfqpoint{1.517831in}{1.465747in}}%
\pgfpathlineto{\pgfqpoint{1.520916in}{1.452820in}}%
\pgfpathlineto{\pgfqpoint{1.519658in}{1.439894in}}%
\pgfpathlineto{\pgfqpoint{1.512231in}{1.427750in}}%
\pgfpathlineto{\pgfqpoint{1.507258in}{1.426967in}}%
\pgfpathlineto{\pgfqpoint{1.498962in}{1.426147in}}%
\pgfpathclose%
\pgfusepath{fill}%
\end{pgfscope}%
\begin{pgfscope}%
\pgfpathrectangle{\pgfqpoint{0.211875in}{0.211875in}}{\pgfqpoint{1.313625in}{1.279725in}}%
\pgfusepath{clip}%
\pgfsetbuttcap%
\pgfsetroundjoin%
\definecolor{currentfill}{rgb}{0.961115,0.566634,0.405693}%
\pgfsetfillcolor{currentfill}%
\pgfsetlinewidth{0.000000pt}%
\definecolor{currentstroke}{rgb}{0.000000,0.000000,0.000000}%
\pgfsetstrokecolor{currentstroke}%
\pgfsetdash{}{0pt}%
\pgfpathmoveto{\pgfqpoint{0.214948in}{1.439894in}}%
\pgfpathlineto{\pgfqpoint{0.217323in}{1.452820in}}%
\pgfpathlineto{\pgfqpoint{0.211875in}{1.465585in}}%
\pgfpathlineto{\pgfqpoint{0.211875in}{1.452820in}}%
\pgfpathlineto{\pgfqpoint{0.211875in}{1.439894in}}%
\pgfpathlineto{\pgfqpoint{0.211875in}{1.436320in}}%
\pgfpathclose%
\pgfusepath{fill}%
\end{pgfscope}%
\begin{pgfscope}%
\pgfpathrectangle{\pgfqpoint{0.211875in}{0.211875in}}{\pgfqpoint{1.313625in}{1.279725in}}%
\pgfusepath{clip}%
\pgfsetbuttcap%
\pgfsetroundjoin%
\definecolor{currentfill}{rgb}{0.961115,0.566634,0.405693}%
\pgfsetfillcolor{currentfill}%
\pgfsetlinewidth{0.000000pt}%
\definecolor{currentstroke}{rgb}{0.000000,0.000000,0.000000}%
\pgfsetstrokecolor{currentstroke}%
\pgfsetdash{}{0pt}%
\pgfpathmoveto{\pgfqpoint{0.318027in}{1.431338in}}%
\pgfpathlineto{\pgfqpoint{0.331295in}{1.430463in}}%
\pgfpathlineto{\pgfqpoint{0.338342in}{1.439894in}}%
\pgfpathlineto{\pgfqpoint{0.340017in}{1.452820in}}%
\pgfpathlineto{\pgfqpoint{0.335982in}{1.465747in}}%
\pgfpathlineto{\pgfqpoint{0.331295in}{1.470440in}}%
\pgfpathlineto{\pgfqpoint{0.318027in}{1.469722in}}%
\pgfpathlineto{\pgfqpoint{0.314455in}{1.465747in}}%
\pgfpathlineto{\pgfqpoint{0.310622in}{1.452820in}}%
\pgfpathlineto{\pgfqpoint{0.312238in}{1.439894in}}%
\pgfpathclose%
\pgfusepath{fill}%
\end{pgfscope}%
\begin{pgfscope}%
\pgfpathrectangle{\pgfqpoint{0.211875in}{0.211875in}}{\pgfqpoint{1.313625in}{1.279725in}}%
\pgfusepath{clip}%
\pgfsetbuttcap%
\pgfsetroundjoin%
\definecolor{currentfill}{rgb}{0.965440,0.720101,0.576404}%
\pgfsetfillcolor{currentfill}%
\pgfsetlinewidth{0.000000pt}%
\definecolor{currentstroke}{rgb}{0.000000,0.000000,0.000000}%
\pgfsetstrokecolor{currentstroke}%
\pgfsetdash{}{0pt}%
\pgfpathmoveto{\pgfqpoint{0.795708in}{0.313315in}}%
\pgfpathlineto{\pgfqpoint{0.796774in}{0.315287in}}%
\pgfpathlineto{\pgfqpoint{0.795708in}{0.316874in}}%
\pgfpathlineto{\pgfqpoint{0.794896in}{0.315287in}}%
\pgfpathclose%
\pgfusepath{fill}%
\end{pgfscope}%
\begin{pgfscope}%
\pgfpathrectangle{\pgfqpoint{0.211875in}{0.211875in}}{\pgfqpoint{1.313625in}{1.279725in}}%
\pgfusepath{clip}%
\pgfsetbuttcap%
\pgfsetroundjoin%
\definecolor{currentfill}{rgb}{0.965440,0.720101,0.576404}%
\pgfsetfillcolor{currentfill}%
\pgfsetlinewidth{0.000000pt}%
\definecolor{currentstroke}{rgb}{0.000000,0.000000,0.000000}%
\pgfsetstrokecolor{currentstroke}%
\pgfsetdash{}{0pt}%
\pgfpathmoveto{\pgfqpoint{0.915129in}{0.307396in}}%
\pgfpathlineto{\pgfqpoint{0.918505in}{0.315287in}}%
\pgfpathlineto{\pgfqpoint{0.915129in}{0.321637in}}%
\pgfpathlineto{\pgfqpoint{0.911061in}{0.315287in}}%
\pgfpathclose%
\pgfusepath{fill}%
\end{pgfscope}%
\begin{pgfscope}%
\pgfpathrectangle{\pgfqpoint{0.211875in}{0.211875in}}{\pgfqpoint{1.313625in}{1.279725in}}%
\pgfusepath{clip}%
\pgfsetbuttcap%
\pgfsetroundjoin%
\definecolor{currentfill}{rgb}{0.965440,0.720101,0.576404}%
\pgfsetfillcolor{currentfill}%
\pgfsetlinewidth{0.000000pt}%
\definecolor{currentstroke}{rgb}{0.000000,0.000000,0.000000}%
\pgfsetstrokecolor{currentstroke}%
\pgfsetdash{}{0pt}%
\pgfpathmoveto{\pgfqpoint{1.034549in}{0.304583in}}%
\pgfpathlineto{\pgfqpoint{1.038340in}{0.315287in}}%
\pgfpathlineto{\pgfqpoint{1.034549in}{0.323915in}}%
\pgfpathlineto{\pgfqpoint{1.027163in}{0.315287in}}%
\pgfpathclose%
\pgfusepath{fill}%
\end{pgfscope}%
\begin{pgfscope}%
\pgfpathrectangle{\pgfqpoint{0.211875in}{0.211875in}}{\pgfqpoint{1.313625in}{1.279725in}}%
\pgfusepath{clip}%
\pgfsetbuttcap%
\pgfsetroundjoin%
\definecolor{currentfill}{rgb}{0.965440,0.720101,0.576404}%
\pgfsetfillcolor{currentfill}%
\pgfsetlinewidth{0.000000pt}%
\definecolor{currentstroke}{rgb}{0.000000,0.000000,0.000000}%
\pgfsetstrokecolor{currentstroke}%
\pgfsetdash{}{0pt}%
\pgfpathmoveto{\pgfqpoint{1.153970in}{0.304879in}}%
\pgfpathlineto{\pgfqpoint{1.157109in}{0.315287in}}%
\pgfpathlineto{\pgfqpoint{1.153970in}{0.323706in}}%
\pgfpathlineto{\pgfqpoint{1.143001in}{0.315287in}}%
\pgfpathclose%
\pgfusepath{fill}%
\end{pgfscope}%
\begin{pgfscope}%
\pgfpathrectangle{\pgfqpoint{0.211875in}{0.211875in}}{\pgfqpoint{1.313625in}{1.279725in}}%
\pgfusepath{clip}%
\pgfsetbuttcap%
\pgfsetroundjoin%
\definecolor{currentfill}{rgb}{0.965440,0.720101,0.576404}%
\pgfsetfillcolor{currentfill}%
\pgfsetlinewidth{0.000000pt}%
\definecolor{currentstroke}{rgb}{0.000000,0.000000,0.000000}%
\pgfsetstrokecolor{currentstroke}%
\pgfsetdash{}{0pt}%
\pgfpathmoveto{\pgfqpoint{1.260121in}{0.314084in}}%
\pgfpathlineto{\pgfqpoint{1.273390in}{0.308385in}}%
\pgfpathlineto{\pgfqpoint{1.275198in}{0.315287in}}%
\pgfpathlineto{\pgfqpoint{1.273390in}{0.320900in}}%
\pgfpathlineto{\pgfqpoint{1.260121in}{0.316280in}}%
\pgfpathlineto{\pgfqpoint{1.259864in}{0.315287in}}%
\pgfpathclose%
\pgfusepath{fill}%
\end{pgfscope}%
\begin{pgfscope}%
\pgfpathrectangle{\pgfqpoint{0.211875in}{0.211875in}}{\pgfqpoint{1.313625in}{1.279725in}}%
\pgfusepath{clip}%
\pgfsetbuttcap%
\pgfsetroundjoin%
\definecolor{currentfill}{rgb}{0.965440,0.720101,0.576404}%
\pgfsetfillcolor{currentfill}%
\pgfsetlinewidth{0.000000pt}%
\definecolor{currentstroke}{rgb}{0.000000,0.000000,0.000000}%
\pgfsetstrokecolor{currentstroke}%
\pgfsetdash{}{0pt}%
\pgfpathmoveto{\pgfqpoint{1.379542in}{0.313950in}}%
\pgfpathlineto{\pgfqpoint{1.392547in}{0.315287in}}%
\pgfpathlineto{\pgfqpoint{1.379542in}{0.316380in}}%
\pgfpathlineto{\pgfqpoint{1.379218in}{0.315287in}}%
\pgfpathclose%
\pgfusepath{fill}%
\end{pgfscope}%
\begin{pgfscope}%
\pgfpathrectangle{\pgfqpoint{0.211875in}{0.211875in}}{\pgfqpoint{1.313625in}{1.279725in}}%
\pgfusepath{clip}%
\pgfsetbuttcap%
\pgfsetroundjoin%
\definecolor{currentfill}{rgb}{0.965440,0.720101,0.576404}%
\pgfsetfillcolor{currentfill}%
\pgfsetlinewidth{0.000000pt}%
\definecolor{currentstroke}{rgb}{0.000000,0.000000,0.000000}%
\pgfsetstrokecolor{currentstroke}%
\pgfsetdash{}{0pt}%
\pgfpathmoveto{\pgfqpoint{1.087625in}{0.379874in}}%
\pgfpathlineto{\pgfqpoint{1.087814in}{0.379920in}}%
\pgfpathlineto{\pgfqpoint{1.100894in}{0.387528in}}%
\pgfpathlineto{\pgfqpoint{1.102495in}{0.392846in}}%
\pgfpathlineto{\pgfqpoint{1.101040in}{0.405773in}}%
\pgfpathlineto{\pgfqpoint{1.100894in}{0.406044in}}%
\pgfpathlineto{\pgfqpoint{1.087625in}{0.410549in}}%
\pgfpathlineto{\pgfqpoint{1.083349in}{0.405773in}}%
\pgfpathlineto{\pgfqpoint{1.081212in}{0.392846in}}%
\pgfpathlineto{\pgfqpoint{1.087567in}{0.379920in}}%
\pgfpathclose%
\pgfusepath{fill}%
\end{pgfscope}%
\begin{pgfscope}%
\pgfpathrectangle{\pgfqpoint{0.211875in}{0.211875in}}{\pgfqpoint{1.313625in}{1.279725in}}%
\pgfusepath{clip}%
\pgfsetbuttcap%
\pgfsetroundjoin%
\definecolor{currentfill}{rgb}{0.965440,0.720101,0.576404}%
\pgfsetfillcolor{currentfill}%
\pgfsetlinewidth{0.000000pt}%
\definecolor{currentstroke}{rgb}{0.000000,0.000000,0.000000}%
\pgfsetstrokecolor{currentstroke}%
\pgfsetdash{}{0pt}%
\pgfpathmoveto{\pgfqpoint{1.207045in}{0.379511in}}%
\pgfpathlineto{\pgfqpoint{1.208177in}{0.379920in}}%
\pgfpathlineto{\pgfqpoint{1.220314in}{0.391141in}}%
\pgfpathlineto{\pgfqpoint{1.220773in}{0.392846in}}%
\pgfpathlineto{\pgfqpoint{1.220314in}{0.397446in}}%
\pgfpathlineto{\pgfqpoint{1.217259in}{0.405773in}}%
\pgfpathlineto{\pgfqpoint{1.207045in}{0.411002in}}%
\pgfpathlineto{\pgfqpoint{1.201518in}{0.405773in}}%
\pgfpathlineto{\pgfqpoint{1.199034in}{0.392846in}}%
\pgfpathlineto{\pgfqpoint{1.206433in}{0.379920in}}%
\pgfpathclose%
\pgfusepath{fill}%
\end{pgfscope}%
\begin{pgfscope}%
\pgfpathrectangle{\pgfqpoint{0.211875in}{0.211875in}}{\pgfqpoint{1.313625in}{1.279725in}}%
\pgfusepath{clip}%
\pgfsetbuttcap%
\pgfsetroundjoin%
\definecolor{currentfill}{rgb}{0.965440,0.720101,0.576404}%
\pgfsetfillcolor{currentfill}%
\pgfsetlinewidth{0.000000pt}%
\definecolor{currentstroke}{rgb}{0.000000,0.000000,0.000000}%
\pgfsetstrokecolor{currentstroke}%
\pgfsetdash{}{0pt}%
\pgfpathmoveto{\pgfqpoint{0.742633in}{0.389823in}}%
\pgfpathlineto{\pgfqpoint{0.743953in}{0.392846in}}%
\pgfpathlineto{\pgfqpoint{0.742633in}{0.400613in}}%
\pgfpathlineto{\pgfqpoint{0.734249in}{0.392846in}}%
\pgfpathclose%
\pgfusepath{fill}%
\end{pgfscope}%
\begin{pgfscope}%
\pgfpathrectangle{\pgfqpoint{0.211875in}{0.211875in}}{\pgfqpoint{1.313625in}{1.279725in}}%
\pgfusepath{clip}%
\pgfsetbuttcap%
\pgfsetroundjoin%
\definecolor{currentfill}{rgb}{0.965440,0.720101,0.576404}%
\pgfsetfillcolor{currentfill}%
\pgfsetlinewidth{0.000000pt}%
\definecolor{currentstroke}{rgb}{0.000000,0.000000,0.000000}%
\pgfsetstrokecolor{currentstroke}%
\pgfsetdash{}{0pt}%
\pgfpathmoveto{\pgfqpoint{0.848784in}{0.387464in}}%
\pgfpathlineto{\pgfqpoint{0.862053in}{0.387075in}}%
\pgfpathlineto{\pgfqpoint{0.864261in}{0.392846in}}%
\pgfpathlineto{\pgfqpoint{0.862379in}{0.405773in}}%
\pgfpathlineto{\pgfqpoint{0.862053in}{0.406243in}}%
\pgfpathlineto{\pgfqpoint{0.848784in}{0.406007in}}%
\pgfpathlineto{\pgfqpoint{0.848626in}{0.405773in}}%
\pgfpathlineto{\pgfqpoint{0.846770in}{0.392846in}}%
\pgfpathclose%
\pgfusepath{fill}%
\end{pgfscope}%
\begin{pgfscope}%
\pgfpathrectangle{\pgfqpoint{0.211875in}{0.211875in}}{\pgfqpoint{1.313625in}{1.279725in}}%
\pgfusepath{clip}%
\pgfsetbuttcap%
\pgfsetroundjoin%
\definecolor{currentfill}{rgb}{0.965440,0.720101,0.576404}%
\pgfsetfillcolor{currentfill}%
\pgfsetlinewidth{0.000000pt}%
\definecolor{currentstroke}{rgb}{0.000000,0.000000,0.000000}%
\pgfsetstrokecolor{currentstroke}%
\pgfsetdash{}{0pt}%
\pgfpathmoveto{\pgfqpoint{0.968205in}{0.382582in}}%
\pgfpathlineto{\pgfqpoint{0.981473in}{0.386258in}}%
\pgfpathlineto{\pgfqpoint{0.983702in}{0.392846in}}%
\pgfpathlineto{\pgfqpoint{0.982082in}{0.405773in}}%
\pgfpathlineto{\pgfqpoint{0.981473in}{0.406772in}}%
\pgfpathlineto{\pgfqpoint{0.968205in}{0.408919in}}%
\pgfpathlineto{\pgfqpoint{0.965773in}{0.405773in}}%
\pgfpathlineto{\pgfqpoint{0.963829in}{0.392846in}}%
\pgfpathclose%
\pgfusepath{fill}%
\end{pgfscope}%
\begin{pgfscope}%
\pgfpathrectangle{\pgfqpoint{0.211875in}{0.211875in}}{\pgfqpoint{1.313625in}{1.279725in}}%
\pgfusepath{clip}%
\pgfsetbuttcap%
\pgfsetroundjoin%
\definecolor{currentfill}{rgb}{0.965440,0.720101,0.576404}%
\pgfsetfillcolor{currentfill}%
\pgfsetlinewidth{0.000000pt}%
\definecolor{currentstroke}{rgb}{0.000000,0.000000,0.000000}%
\pgfsetstrokecolor{currentstroke}%
\pgfsetdash{}{0pt}%
\pgfpathmoveto{\pgfqpoint{1.326466in}{0.380145in}}%
\pgfpathlineto{\pgfqpoint{1.336740in}{0.392846in}}%
\pgfpathlineto{\pgfqpoint{1.333175in}{0.405773in}}%
\pgfpathlineto{\pgfqpoint{1.326466in}{0.410329in}}%
\pgfpathlineto{\pgfqpoint{1.320615in}{0.405773in}}%
\pgfpathlineto{\pgfqpoint{1.317530in}{0.392846in}}%
\pgfpathclose%
\pgfusepath{fill}%
\end{pgfscope}%
\begin{pgfscope}%
\pgfpathrectangle{\pgfqpoint{0.211875in}{0.211875in}}{\pgfqpoint{1.313625in}{1.279725in}}%
\pgfusepath{clip}%
\pgfsetbuttcap%
\pgfsetroundjoin%
\definecolor{currentfill}{rgb}{0.965440,0.720101,0.576404}%
\pgfsetfillcolor{currentfill}%
\pgfsetlinewidth{0.000000pt}%
\definecolor{currentstroke}{rgb}{0.000000,0.000000,0.000000}%
\pgfsetstrokecolor{currentstroke}%
\pgfsetdash{}{0pt}%
\pgfpathmoveto{\pgfqpoint{1.445886in}{0.383146in}}%
\pgfpathlineto{\pgfqpoint{1.452192in}{0.392846in}}%
\pgfpathlineto{\pgfqpoint{1.449152in}{0.405773in}}%
\pgfpathlineto{\pgfqpoint{1.445886in}{0.408534in}}%
\pgfpathlineto{\pgfqpoint{1.441378in}{0.405773in}}%
\pgfpathlineto{\pgfqpoint{1.437218in}{0.392846in}}%
\pgfpathclose%
\pgfusepath{fill}%
\end{pgfscope}%
\begin{pgfscope}%
\pgfpathrectangle{\pgfqpoint{0.211875in}{0.211875in}}{\pgfqpoint{1.313625in}{1.279725in}}%
\pgfusepath{clip}%
\pgfsetbuttcap%
\pgfsetroundjoin%
\definecolor{currentfill}{rgb}{0.965440,0.720101,0.576404}%
\pgfsetfillcolor{currentfill}%
\pgfsetlinewidth{0.000000pt}%
\definecolor{currentstroke}{rgb}{0.000000,0.000000,0.000000}%
\pgfsetstrokecolor{currentstroke}%
\pgfsetdash{}{0pt}%
\pgfpathmoveto{\pgfqpoint{1.034549in}{0.457464in}}%
\pgfpathlineto{\pgfqpoint{1.034576in}{0.457479in}}%
\pgfpathlineto{\pgfqpoint{1.044801in}{0.470405in}}%
\pgfpathlineto{\pgfqpoint{1.044915in}{0.483332in}}%
\pgfpathlineto{\pgfqpoint{1.034654in}{0.496258in}}%
\pgfpathlineto{\pgfqpoint{1.034549in}{0.496314in}}%
\pgfpathlineto{\pgfqpoint{1.034346in}{0.496258in}}%
\pgfpathlineto{\pgfqpoint{1.021280in}{0.488700in}}%
\pgfpathlineto{\pgfqpoint{1.019290in}{0.483332in}}%
\pgfpathlineto{\pgfqpoint{1.019353in}{0.470405in}}%
\pgfpathlineto{\pgfqpoint{1.021280in}{0.465283in}}%
\pgfpathlineto{\pgfqpoint{1.034497in}{0.457479in}}%
\pgfpathclose%
\pgfusepath{fill}%
\end{pgfscope}%
\begin{pgfscope}%
\pgfpathrectangle{\pgfqpoint{0.211875in}{0.211875in}}{\pgfqpoint{1.313625in}{1.279725in}}%
\pgfusepath{clip}%
\pgfsetbuttcap%
\pgfsetroundjoin%
\definecolor{currentfill}{rgb}{0.965440,0.720101,0.576404}%
\pgfsetfillcolor{currentfill}%
\pgfsetlinewidth{0.000000pt}%
\definecolor{currentstroke}{rgb}{0.000000,0.000000,0.000000}%
\pgfsetstrokecolor{currentstroke}%
\pgfsetdash{}{0pt}%
\pgfpathmoveto{\pgfqpoint{1.153970in}{0.457471in}}%
\pgfpathlineto{\pgfqpoint{1.153982in}{0.457479in}}%
\pgfpathlineto{\pgfqpoint{1.162680in}{0.470405in}}%
\pgfpathlineto{\pgfqpoint{1.162782in}{0.483332in}}%
\pgfpathlineto{\pgfqpoint{1.154072in}{0.496258in}}%
\pgfpathlineto{\pgfqpoint{1.153970in}{0.496323in}}%
\pgfpathlineto{\pgfqpoint{1.153611in}{0.496258in}}%
\pgfpathlineto{\pgfqpoint{1.140701in}{0.491692in}}%
\pgfpathlineto{\pgfqpoint{1.137206in}{0.483332in}}%
\pgfpathlineto{\pgfqpoint{1.137278in}{0.470405in}}%
\pgfpathlineto{\pgfqpoint{1.140701in}{0.462288in}}%
\pgfpathlineto{\pgfqpoint{1.153927in}{0.457479in}}%
\pgfpathclose%
\pgfusepath{fill}%
\end{pgfscope}%
\begin{pgfscope}%
\pgfpathrectangle{\pgfqpoint{0.211875in}{0.211875in}}{\pgfqpoint{1.313625in}{1.279725in}}%
\pgfusepath{clip}%
\pgfsetbuttcap%
\pgfsetroundjoin%
\definecolor{currentfill}{rgb}{0.965440,0.720101,0.576404}%
\pgfsetfillcolor{currentfill}%
\pgfsetlinewidth{0.000000pt}%
\definecolor{currentstroke}{rgb}{0.000000,0.000000,0.000000}%
\pgfsetstrokecolor{currentstroke}%
\pgfsetdash{}{0pt}%
\pgfpathmoveto{\pgfqpoint{0.676288in}{0.465634in}}%
\pgfpathlineto{\pgfqpoint{0.684139in}{0.470405in}}%
\pgfpathlineto{\pgfqpoint{0.684195in}{0.483332in}}%
\pgfpathlineto{\pgfqpoint{0.676288in}{0.488107in}}%
\pgfpathlineto{\pgfqpoint{0.672609in}{0.483332in}}%
\pgfpathlineto{\pgfqpoint{0.672630in}{0.470405in}}%
\pgfpathclose%
\pgfusepath{fill}%
\end{pgfscope}%
\begin{pgfscope}%
\pgfpathrectangle{\pgfqpoint{0.211875in}{0.211875in}}{\pgfqpoint{1.313625in}{1.279725in}}%
\pgfusepath{clip}%
\pgfsetbuttcap%
\pgfsetroundjoin%
\definecolor{currentfill}{rgb}{0.965440,0.720101,0.576404}%
\pgfsetfillcolor{currentfill}%
\pgfsetlinewidth{0.000000pt}%
\definecolor{currentstroke}{rgb}{0.000000,0.000000,0.000000}%
\pgfsetstrokecolor{currentstroke}%
\pgfsetdash{}{0pt}%
\pgfpathmoveto{\pgfqpoint{0.795708in}{0.461511in}}%
\pgfpathlineto{\pgfqpoint{0.806460in}{0.470405in}}%
\pgfpathlineto{\pgfqpoint{0.806565in}{0.483332in}}%
\pgfpathlineto{\pgfqpoint{0.795708in}{0.492253in}}%
\pgfpathlineto{\pgfqpoint{0.787443in}{0.483332in}}%
\pgfpathlineto{\pgfqpoint{0.787516in}{0.470405in}}%
\pgfpathclose%
\pgfusepath{fill}%
\end{pgfscope}%
\begin{pgfscope}%
\pgfpathrectangle{\pgfqpoint{0.211875in}{0.211875in}}{\pgfqpoint{1.313625in}{1.279725in}}%
\pgfusepath{clip}%
\pgfsetbuttcap%
\pgfsetroundjoin%
\definecolor{currentfill}{rgb}{0.965440,0.720101,0.576404}%
\pgfsetfillcolor{currentfill}%
\pgfsetlinewidth{0.000000pt}%
\definecolor{currentstroke}{rgb}{0.000000,0.000000,0.000000}%
\pgfsetstrokecolor{currentstroke}%
\pgfsetdash{}{0pt}%
\pgfpathmoveto{\pgfqpoint{0.901860in}{0.470296in}}%
\pgfpathlineto{\pgfqpoint{0.915129in}{0.458801in}}%
\pgfpathlineto{\pgfqpoint{0.926235in}{0.470405in}}%
\pgfpathlineto{\pgfqpoint{0.926352in}{0.483332in}}%
\pgfpathlineto{\pgfqpoint{0.915129in}{0.494984in}}%
\pgfpathlineto{\pgfqpoint{0.901860in}{0.483593in}}%
\pgfpathlineto{\pgfqpoint{0.901774in}{0.483332in}}%
\pgfpathlineto{\pgfqpoint{0.901823in}{0.470405in}}%
\pgfpathclose%
\pgfusepath{fill}%
\end{pgfscope}%
\begin{pgfscope}%
\pgfpathrectangle{\pgfqpoint{0.211875in}{0.211875in}}{\pgfqpoint{1.313625in}{1.279725in}}%
\pgfusepath{clip}%
\pgfsetbuttcap%
\pgfsetroundjoin%
\definecolor{currentfill}{rgb}{0.965440,0.720101,0.576404}%
\pgfsetfillcolor{currentfill}%
\pgfsetlinewidth{0.000000pt}%
\definecolor{currentstroke}{rgb}{0.000000,0.000000,0.000000}%
\pgfsetstrokecolor{currentstroke}%
\pgfsetdash{}{0pt}%
\pgfpathmoveto{\pgfqpoint{1.260121in}{0.461012in}}%
\pgfpathlineto{\pgfqpoint{1.273390in}{0.458887in}}%
\pgfpathlineto{\pgfqpoint{1.280109in}{0.470405in}}%
\pgfpathlineto{\pgfqpoint{1.280194in}{0.483332in}}%
\pgfpathlineto{\pgfqpoint{1.273390in}{0.494974in}}%
\pgfpathlineto{\pgfqpoint{1.260121in}{0.492919in}}%
\pgfpathlineto{\pgfqpoint{1.255590in}{0.483332in}}%
\pgfpathlineto{\pgfqpoint{1.255664in}{0.470405in}}%
\pgfpathclose%
\pgfusepath{fill}%
\end{pgfscope}%
\begin{pgfscope}%
\pgfpathrectangle{\pgfqpoint{0.211875in}{0.211875in}}{\pgfqpoint{1.313625in}{1.279725in}}%
\pgfusepath{clip}%
\pgfsetbuttcap%
\pgfsetroundjoin%
\definecolor{currentfill}{rgb}{0.965440,0.720101,0.576404}%
\pgfsetfillcolor{currentfill}%
\pgfsetlinewidth{0.000000pt}%
\definecolor{currentstroke}{rgb}{0.000000,0.000000,0.000000}%
\pgfsetstrokecolor{currentstroke}%
\pgfsetdash{}{0pt}%
\pgfpathmoveto{\pgfqpoint{1.379542in}{0.461259in}}%
\pgfpathlineto{\pgfqpoint{1.392811in}{0.461822in}}%
\pgfpathlineto{\pgfqpoint{1.397206in}{0.470405in}}%
\pgfpathlineto{\pgfqpoint{1.397269in}{0.483332in}}%
\pgfpathlineto{\pgfqpoint{1.392811in}{0.492050in}}%
\pgfpathlineto{\pgfqpoint{1.379542in}{0.492610in}}%
\pgfpathlineto{\pgfqpoint{1.374554in}{0.483332in}}%
\pgfpathlineto{\pgfqpoint{1.374624in}{0.470405in}}%
\pgfpathclose%
\pgfusepath{fill}%
\end{pgfscope}%
\begin{pgfscope}%
\pgfpathrectangle{\pgfqpoint{0.211875in}{0.211875in}}{\pgfqpoint{1.313625in}{1.279725in}}%
\pgfusepath{clip}%
\pgfsetbuttcap%
\pgfsetroundjoin%
\definecolor{currentfill}{rgb}{0.965440,0.720101,0.576404}%
\pgfsetfillcolor{currentfill}%
\pgfsetlinewidth{0.000000pt}%
\definecolor{currentstroke}{rgb}{0.000000,0.000000,0.000000}%
\pgfsetstrokecolor{currentstroke}%
\pgfsetdash{}{0pt}%
\pgfpathmoveto{\pgfqpoint{1.498962in}{0.462907in}}%
\pgfpathlineto{\pgfqpoint{1.512231in}{0.466439in}}%
\pgfpathlineto{\pgfqpoint{1.514030in}{0.470405in}}%
\pgfpathlineto{\pgfqpoint{1.514069in}{0.483332in}}%
\pgfpathlineto{\pgfqpoint{1.512231in}{0.487406in}}%
\pgfpathlineto{\pgfqpoint{1.498962in}{0.490899in}}%
\pgfpathlineto{\pgfqpoint{1.494292in}{0.483332in}}%
\pgfpathlineto{\pgfqpoint{1.494347in}{0.470405in}}%
\pgfpathclose%
\pgfusepath{fill}%
\end{pgfscope}%
\begin{pgfscope}%
\pgfpathrectangle{\pgfqpoint{0.211875in}{0.211875in}}{\pgfqpoint{1.313625in}{1.279725in}}%
\pgfusepath{clip}%
\pgfsetbuttcap%
\pgfsetroundjoin%
\definecolor{currentfill}{rgb}{0.965440,0.720101,0.576404}%
\pgfsetfillcolor{currentfill}%
\pgfsetlinewidth{0.000000pt}%
\definecolor{currentstroke}{rgb}{0.000000,0.000000,0.000000}%
\pgfsetstrokecolor{currentstroke}%
\pgfsetdash{}{0pt}%
\pgfpathmoveto{\pgfqpoint{0.623212in}{0.544872in}}%
\pgfpathlineto{\pgfqpoint{0.626044in}{0.547964in}}%
\pgfpathlineto{\pgfqpoint{0.628427in}{0.560891in}}%
\pgfpathlineto{\pgfqpoint{0.623212in}{0.571167in}}%
\pgfpathlineto{\pgfqpoint{0.609943in}{0.562349in}}%
\pgfpathlineto{\pgfqpoint{0.609519in}{0.560891in}}%
\pgfpathlineto{\pgfqpoint{0.609943in}{0.557247in}}%
\pgfpathlineto{\pgfqpoint{0.615035in}{0.547964in}}%
\pgfpathclose%
\pgfusepath{fill}%
\end{pgfscope}%
\begin{pgfscope}%
\pgfpathrectangle{\pgfqpoint{0.211875in}{0.211875in}}{\pgfqpoint{1.313625in}{1.279725in}}%
\pgfusepath{clip}%
\pgfsetbuttcap%
\pgfsetroundjoin%
\definecolor{currentfill}{rgb}{0.965440,0.720101,0.576404}%
\pgfsetfillcolor{currentfill}%
\pgfsetlinewidth{0.000000pt}%
\definecolor{currentstroke}{rgb}{0.000000,0.000000,0.000000}%
\pgfsetstrokecolor{currentstroke}%
\pgfsetdash{}{0pt}%
\pgfpathmoveto{\pgfqpoint{0.729364in}{0.544507in}}%
\pgfpathlineto{\pgfqpoint{0.742633in}{0.542021in}}%
\pgfpathlineto{\pgfqpoint{0.747321in}{0.547964in}}%
\pgfpathlineto{\pgfqpoint{0.749251in}{0.560891in}}%
\pgfpathlineto{\pgfqpoint{0.743838in}{0.573817in}}%
\pgfpathlineto{\pgfqpoint{0.742633in}{0.574866in}}%
\pgfpathlineto{\pgfqpoint{0.735237in}{0.573817in}}%
\pgfpathlineto{\pgfqpoint{0.729364in}{0.571929in}}%
\pgfpathlineto{\pgfqpoint{0.725740in}{0.560891in}}%
\pgfpathlineto{\pgfqpoint{0.727304in}{0.547964in}}%
\pgfpathclose%
\pgfusepath{fill}%
\end{pgfscope}%
\begin{pgfscope}%
\pgfpathrectangle{\pgfqpoint{0.211875in}{0.211875in}}{\pgfqpoint{1.313625in}{1.279725in}}%
\pgfusepath{clip}%
\pgfsetbuttcap%
\pgfsetroundjoin%
\definecolor{currentfill}{rgb}{0.965440,0.720101,0.576404}%
\pgfsetfillcolor{currentfill}%
\pgfsetlinewidth{0.000000pt}%
\definecolor{currentstroke}{rgb}{0.000000,0.000000,0.000000}%
\pgfsetstrokecolor{currentstroke}%
\pgfsetdash{}{0pt}%
\pgfpathmoveto{\pgfqpoint{0.848784in}{0.540448in}}%
\pgfpathlineto{\pgfqpoint{0.862053in}{0.540213in}}%
\pgfpathlineto{\pgfqpoint{0.867397in}{0.547964in}}%
\pgfpathlineto{\pgfqpoint{0.869021in}{0.560891in}}%
\pgfpathlineto{\pgfqpoint{0.864579in}{0.573817in}}%
\pgfpathlineto{\pgfqpoint{0.862053in}{0.576339in}}%
\pgfpathlineto{\pgfqpoint{0.848784in}{0.576150in}}%
\pgfpathlineto{\pgfqpoint{0.846511in}{0.573817in}}%
\pgfpathlineto{\pgfqpoint{0.842119in}{0.560891in}}%
\pgfpathlineto{\pgfqpoint{0.843723in}{0.547964in}}%
\pgfpathclose%
\pgfusepath{fill}%
\end{pgfscope}%
\begin{pgfscope}%
\pgfpathrectangle{\pgfqpoint{0.211875in}{0.211875in}}{\pgfqpoint{1.313625in}{1.279725in}}%
\pgfusepath{clip}%
\pgfsetbuttcap%
\pgfsetroundjoin%
\definecolor{currentfill}{rgb}{0.965440,0.720101,0.576404}%
\pgfsetfillcolor{currentfill}%
\pgfsetlinewidth{0.000000pt}%
\definecolor{currentstroke}{rgb}{0.000000,0.000000,0.000000}%
\pgfsetstrokecolor{currentstroke}%
\pgfsetdash{}{0pt}%
\pgfpathmoveto{\pgfqpoint{0.968205in}{0.537740in}}%
\pgfpathlineto{\pgfqpoint{0.981473in}{0.539489in}}%
\pgfpathlineto{\pgfqpoint{0.986631in}{0.547964in}}%
\pgfpathlineto{\pgfqpoint{0.988048in}{0.560891in}}%
\pgfpathlineto{\pgfqpoint{0.984242in}{0.573817in}}%
\pgfpathlineto{\pgfqpoint{0.981473in}{0.576959in}}%
\pgfpathlineto{\pgfqpoint{0.968205in}{0.578285in}}%
\pgfpathlineto{\pgfqpoint{0.963226in}{0.573817in}}%
\pgfpathlineto{\pgfqpoint{0.958646in}{0.560891in}}%
\pgfpathlineto{\pgfqpoint{0.960350in}{0.547964in}}%
\pgfpathclose%
\pgfusepath{fill}%
\end{pgfscope}%
\begin{pgfscope}%
\pgfpathrectangle{\pgfqpoint{0.211875in}{0.211875in}}{\pgfqpoint{1.313625in}{1.279725in}}%
\pgfusepath{clip}%
\pgfsetbuttcap%
\pgfsetroundjoin%
\definecolor{currentfill}{rgb}{0.965440,0.720101,0.576404}%
\pgfsetfillcolor{currentfill}%
\pgfsetlinewidth{0.000000pt}%
\definecolor{currentstroke}{rgb}{0.000000,0.000000,0.000000}%
\pgfsetstrokecolor{currentstroke}%
\pgfsetdash{}{0pt}%
\pgfpathmoveto{\pgfqpoint{1.087625in}{0.536239in}}%
\pgfpathlineto{\pgfqpoint{1.100894in}{0.539937in}}%
\pgfpathlineto{\pgfqpoint{1.105232in}{0.547964in}}%
\pgfpathlineto{\pgfqpoint{1.106516in}{0.560891in}}%
\pgfpathlineto{\pgfqpoint{1.103102in}{0.573817in}}%
\pgfpathlineto{\pgfqpoint{1.100894in}{0.576650in}}%
\pgfpathlineto{\pgfqpoint{1.087625in}{0.579458in}}%
\pgfpathlineto{\pgfqpoint{1.080340in}{0.573817in}}%
\pgfpathlineto{\pgfqpoint{1.075324in}{0.560891in}}%
\pgfpathlineto{\pgfqpoint{1.077209in}{0.547964in}}%
\pgfpathclose%
\pgfpathmoveto{\pgfqpoint{1.086469in}{0.560891in}}%
\pgfpathlineto{\pgfqpoint{1.087625in}{0.563233in}}%
\pgfpathlineto{\pgfqpoint{1.091395in}{0.560891in}}%
\pgfpathlineto{\pgfqpoint{1.087625in}{0.554865in}}%
\pgfpathclose%
\pgfusepath{fill}%
\end{pgfscope}%
\begin{pgfscope}%
\pgfpathrectangle{\pgfqpoint{0.211875in}{0.211875in}}{\pgfqpoint{1.313625in}{1.279725in}}%
\pgfusepath{clip}%
\pgfsetbuttcap%
\pgfsetroundjoin%
\definecolor{currentfill}{rgb}{0.965440,0.720101,0.576404}%
\pgfsetfillcolor{currentfill}%
\pgfsetlinewidth{0.000000pt}%
\definecolor{currentstroke}{rgb}{0.000000,0.000000,0.000000}%
\pgfsetstrokecolor{currentstroke}%
\pgfsetdash{}{0pt}%
\pgfpathmoveto{\pgfqpoint{1.207045in}{0.535859in}}%
\pgfpathlineto{\pgfqpoint{1.220314in}{0.541706in}}%
\pgfpathlineto{\pgfqpoint{1.223330in}{0.547964in}}%
\pgfpathlineto{\pgfqpoint{1.224533in}{0.560891in}}%
\pgfpathlineto{\pgfqpoint{1.221330in}{0.573817in}}%
\pgfpathlineto{\pgfqpoint{1.220314in}{0.575285in}}%
\pgfpathlineto{\pgfqpoint{1.207045in}{0.579746in}}%
\pgfpathlineto{\pgfqpoint{1.198000in}{0.573817in}}%
\pgfpathlineto{\pgfqpoint{1.193777in}{0.565006in}}%
\pgfpathlineto{\pgfqpoint{1.192999in}{0.560891in}}%
\pgfpathlineto{\pgfqpoint{1.193777in}{0.551289in}}%
\pgfpathlineto{\pgfqpoint{1.194365in}{0.547964in}}%
\pgfpathclose%
\pgfpathmoveto{\pgfqpoint{1.205158in}{0.560891in}}%
\pgfpathlineto{\pgfqpoint{1.207045in}{0.564133in}}%
\pgfpathlineto{\pgfqpoint{1.210536in}{0.560891in}}%
\pgfpathlineto{\pgfqpoint{1.207045in}{0.552527in}}%
\pgfpathclose%
\pgfusepath{fill}%
\end{pgfscope}%
\begin{pgfscope}%
\pgfpathrectangle{\pgfqpoint{0.211875in}{0.211875in}}{\pgfqpoint{1.313625in}{1.279725in}}%
\pgfusepath{clip}%
\pgfsetbuttcap%
\pgfsetroundjoin%
\definecolor{currentfill}{rgb}{0.965440,0.720101,0.576404}%
\pgfsetfillcolor{currentfill}%
\pgfsetlinewidth{0.000000pt}%
\definecolor{currentstroke}{rgb}{0.000000,0.000000,0.000000}%
\pgfsetstrokecolor{currentstroke}%
\pgfsetdash{}{0pt}%
\pgfpathmoveto{\pgfqpoint{1.313197in}{0.546662in}}%
\pgfpathlineto{\pgfqpoint{1.326466in}{0.536555in}}%
\pgfpathlineto{\pgfqpoint{1.339735in}{0.545025in}}%
\pgfpathlineto{\pgfqpoint{1.340999in}{0.547964in}}%
\pgfpathlineto{\pgfqpoint{1.342167in}{0.560891in}}%
\pgfpathlineto{\pgfqpoint{1.339735in}{0.571342in}}%
\pgfpathlineto{\pgfqpoint{1.337844in}{0.573817in}}%
\pgfpathlineto{\pgfqpoint{1.326466in}{0.579184in}}%
\pgfpathlineto{\pgfqpoint{1.316504in}{0.573817in}}%
\pgfpathlineto{\pgfqpoint{1.313197in}{0.568585in}}%
\pgfpathlineto{\pgfqpoint{1.311545in}{0.560891in}}%
\pgfpathlineto{\pgfqpoint{1.312676in}{0.547964in}}%
\pgfpathclose%
\pgfpathmoveto{\pgfqpoint{1.324925in}{0.560891in}}%
\pgfpathlineto{\pgfqpoint{1.326466in}{0.563071in}}%
\pgfpathlineto{\pgfqpoint{1.328237in}{0.560891in}}%
\pgfpathlineto{\pgfqpoint{1.326466in}{0.555259in}}%
\pgfpathclose%
\pgfusepath{fill}%
\end{pgfscope}%
\begin{pgfscope}%
\pgfpathrectangle{\pgfqpoint{0.211875in}{0.211875in}}{\pgfqpoint{1.313625in}{1.279725in}}%
\pgfusepath{clip}%
\pgfsetbuttcap%
\pgfsetroundjoin%
\definecolor{currentfill}{rgb}{0.965440,0.720101,0.576404}%
\pgfsetfillcolor{currentfill}%
\pgfsetlinewidth{0.000000pt}%
\definecolor{currentstroke}{rgb}{0.000000,0.000000,0.000000}%
\pgfsetstrokecolor{currentstroke}%
\pgfsetdash{}{0pt}%
\pgfpathmoveto{\pgfqpoint{1.432617in}{0.546170in}}%
\pgfpathlineto{\pgfqpoint{1.445886in}{0.538322in}}%
\pgfpathlineto{\pgfqpoint{1.457191in}{0.547964in}}%
\pgfpathlineto{\pgfqpoint{1.459155in}{0.557461in}}%
\pgfpathlineto{\pgfqpoint{1.459452in}{0.560891in}}%
\pgfpathlineto{\pgfqpoint{1.459155in}{0.562333in}}%
\pgfpathlineto{\pgfqpoint{1.452622in}{0.573817in}}%
\pgfpathlineto{\pgfqpoint{1.445886in}{0.577776in}}%
\pgfpathlineto{\pgfqpoint{1.436520in}{0.573817in}}%
\pgfpathlineto{\pgfqpoint{1.432617in}{0.569317in}}%
\pgfpathlineto{\pgfqpoint{1.430571in}{0.560891in}}%
\pgfpathlineto{\pgfqpoint{1.431813in}{0.547964in}}%
\pgfpathclose%
\pgfusepath{fill}%
\end{pgfscope}%
\begin{pgfscope}%
\pgfpathrectangle{\pgfqpoint{0.211875in}{0.211875in}}{\pgfqpoint{1.313625in}{1.279725in}}%
\pgfusepath{clip}%
\pgfsetbuttcap%
\pgfsetroundjoin%
\definecolor{currentfill}{rgb}{0.965440,0.720101,0.576404}%
\pgfsetfillcolor{currentfill}%
\pgfsetlinewidth{0.000000pt}%
\definecolor{currentstroke}{rgb}{0.000000,0.000000,0.000000}%
\pgfsetstrokecolor{currentstroke}%
\pgfsetdash{}{0pt}%
\pgfpathmoveto{\pgfqpoint{0.503792in}{0.551472in}}%
\pgfpathlineto{\pgfqpoint{0.505989in}{0.560891in}}%
\pgfpathlineto{\pgfqpoint{0.503792in}{0.564542in}}%
\pgfpathlineto{\pgfqpoint{0.500083in}{0.560891in}}%
\pgfpathclose%
\pgfusepath{fill}%
\end{pgfscope}%
\begin{pgfscope}%
\pgfpathrectangle{\pgfqpoint{0.211875in}{0.211875in}}{\pgfqpoint{1.313625in}{1.279725in}}%
\pgfusepath{clip}%
\pgfsetbuttcap%
\pgfsetroundjoin%
\definecolor{currentfill}{rgb}{0.965440,0.720101,0.576404}%
\pgfsetfillcolor{currentfill}%
\pgfsetlinewidth{0.000000pt}%
\definecolor{currentstroke}{rgb}{0.000000,0.000000,0.000000}%
\pgfsetstrokecolor{currentstroke}%
\pgfsetdash{}{0pt}%
\pgfpathmoveto{\pgfqpoint{0.556867in}{0.624780in}}%
\pgfpathlineto{\pgfqpoint{0.560079in}{0.625523in}}%
\pgfpathlineto{\pgfqpoint{0.570136in}{0.632206in}}%
\pgfpathlineto{\pgfqpoint{0.571624in}{0.638450in}}%
\pgfpathlineto{\pgfqpoint{0.570136in}{0.646299in}}%
\pgfpathlineto{\pgfqpoint{0.564108in}{0.651377in}}%
\pgfpathlineto{\pgfqpoint{0.556867in}{0.653191in}}%
\pgfpathlineto{\pgfqpoint{0.555078in}{0.651377in}}%
\pgfpathlineto{\pgfqpoint{0.551484in}{0.638450in}}%
\pgfpathlineto{\pgfqpoint{0.556074in}{0.625523in}}%
\pgfpathclose%
\pgfusepath{fill}%
\end{pgfscope}%
\begin{pgfscope}%
\pgfpathrectangle{\pgfqpoint{0.211875in}{0.211875in}}{\pgfqpoint{1.313625in}{1.279725in}}%
\pgfusepath{clip}%
\pgfsetbuttcap%
\pgfsetroundjoin%
\definecolor{currentfill}{rgb}{0.965440,0.720101,0.576404}%
\pgfsetfillcolor{currentfill}%
\pgfsetlinewidth{0.000000pt}%
\definecolor{currentstroke}{rgb}{0.000000,0.000000,0.000000}%
\pgfsetstrokecolor{currentstroke}%
\pgfsetdash{}{0pt}%
\pgfpathmoveto{\pgfqpoint{0.676288in}{0.621002in}}%
\pgfpathlineto{\pgfqpoint{0.688541in}{0.625523in}}%
\pgfpathlineto{\pgfqpoint{0.689557in}{0.626663in}}%
\pgfpathlineto{\pgfqpoint{0.692062in}{0.638450in}}%
\pgfpathlineto{\pgfqpoint{0.689943in}{0.651377in}}%
\pgfpathlineto{\pgfqpoint{0.689557in}{0.652052in}}%
\pgfpathlineto{\pgfqpoint{0.676288in}{0.657234in}}%
\pgfpathlineto{\pgfqpoint{0.669514in}{0.651377in}}%
\pgfpathlineto{\pgfqpoint{0.665842in}{0.638450in}}%
\pgfpathlineto{\pgfqpoint{0.670638in}{0.625523in}}%
\pgfpathclose%
\pgfusepath{fill}%
\end{pgfscope}%
\begin{pgfscope}%
\pgfpathrectangle{\pgfqpoint{0.211875in}{0.211875in}}{\pgfqpoint{1.313625in}{1.279725in}}%
\pgfusepath{clip}%
\pgfsetbuttcap%
\pgfsetroundjoin%
\definecolor{currentfill}{rgb}{0.965440,0.720101,0.576404}%
\pgfsetfillcolor{currentfill}%
\pgfsetlinewidth{0.000000pt}%
\definecolor{currentstroke}{rgb}{0.000000,0.000000,0.000000}%
\pgfsetstrokecolor{currentstroke}%
\pgfsetdash{}{0pt}%
\pgfpathmoveto{\pgfqpoint{0.795708in}{0.618223in}}%
\pgfpathlineto{\pgfqpoint{0.808977in}{0.624746in}}%
\pgfpathlineto{\pgfqpoint{0.809410in}{0.625523in}}%
\pgfpathlineto{\pgfqpoint{0.811812in}{0.638450in}}%
\pgfpathlineto{\pgfqpoint{0.810022in}{0.651377in}}%
\pgfpathlineto{\pgfqpoint{0.808977in}{0.653437in}}%
\pgfpathlineto{\pgfqpoint{0.795708in}{0.660208in}}%
\pgfpathlineto{\pgfqpoint{0.783427in}{0.651377in}}%
\pgfpathlineto{\pgfqpoint{0.782439in}{0.648421in}}%
\pgfpathlineto{\pgfqpoint{0.781157in}{0.638450in}}%
\pgfpathlineto{\pgfqpoint{0.782439in}{0.630784in}}%
\pgfpathlineto{\pgfqpoint{0.784742in}{0.625523in}}%
\pgfpathclose%
\pgfpathmoveto{\pgfqpoint{0.794135in}{0.638450in}}%
\pgfpathlineto{\pgfqpoint{0.795708in}{0.642243in}}%
\pgfpathlineto{\pgfqpoint{0.797770in}{0.638450in}}%
\pgfpathlineto{\pgfqpoint{0.795708in}{0.635365in}}%
\pgfpathclose%
\pgfusepath{fill}%
\end{pgfscope}%
\begin{pgfscope}%
\pgfpathrectangle{\pgfqpoint{0.211875in}{0.211875in}}{\pgfqpoint{1.313625in}{1.279725in}}%
\pgfusepath{clip}%
\pgfsetbuttcap%
\pgfsetroundjoin%
\definecolor{currentfill}{rgb}{0.965440,0.720101,0.576404}%
\pgfsetfillcolor{currentfill}%
\pgfsetlinewidth{0.000000pt}%
\definecolor{currentstroke}{rgb}{0.000000,0.000000,0.000000}%
\pgfsetstrokecolor{currentstroke}%
\pgfsetdash{}{0pt}%
\pgfpathmoveto{\pgfqpoint{0.901860in}{0.622887in}}%
\pgfpathlineto{\pgfqpoint{0.915129in}{0.616375in}}%
\pgfpathlineto{\pgfqpoint{0.928398in}{0.624593in}}%
\pgfpathlineto{\pgfqpoint{0.928859in}{0.625523in}}%
\pgfpathlineto{\pgfqpoint{0.930997in}{0.638450in}}%
\pgfpathlineto{\pgfqpoint{0.929433in}{0.651377in}}%
\pgfpathlineto{\pgfqpoint{0.928398in}{0.653681in}}%
\pgfpathlineto{\pgfqpoint{0.915129in}{0.662190in}}%
\pgfpathlineto{\pgfqpoint{0.901860in}{0.655460in}}%
\pgfpathlineto{\pgfqpoint{0.899827in}{0.651377in}}%
\pgfpathlineto{\pgfqpoint{0.898183in}{0.638450in}}%
\pgfpathlineto{\pgfqpoint{0.900421in}{0.625523in}}%
\pgfpathclose%
\pgfpathmoveto{\pgfqpoint{0.910115in}{0.638450in}}%
\pgfpathlineto{\pgfqpoint{0.915129in}{0.648117in}}%
\pgfpathlineto{\pgfqpoint{0.919290in}{0.638450in}}%
\pgfpathlineto{\pgfqpoint{0.915129in}{0.630587in}}%
\pgfpathclose%
\pgfusepath{fill}%
\end{pgfscope}%
\begin{pgfscope}%
\pgfpathrectangle{\pgfqpoint{0.211875in}{0.211875in}}{\pgfqpoint{1.313625in}{1.279725in}}%
\pgfusepath{clip}%
\pgfsetbuttcap%
\pgfsetroundjoin%
\definecolor{currentfill}{rgb}{0.965440,0.720101,0.576404}%
\pgfsetfillcolor{currentfill}%
\pgfsetlinewidth{0.000000pt}%
\definecolor{currentstroke}{rgb}{0.000000,0.000000,0.000000}%
\pgfsetstrokecolor{currentstroke}%
\pgfsetdash{}{0pt}%
\pgfpathmoveto{\pgfqpoint{1.021280in}{0.619802in}}%
\pgfpathlineto{\pgfqpoint{1.034549in}{0.615422in}}%
\pgfpathlineto{\pgfqpoint{1.047636in}{0.625523in}}%
\pgfpathlineto{\pgfqpoint{1.047818in}{0.626041in}}%
\pgfpathlineto{\pgfqpoint{1.049701in}{0.638450in}}%
\pgfpathlineto{\pgfqpoint{1.048275in}{0.651377in}}%
\pgfpathlineto{\pgfqpoint{1.047818in}{0.652529in}}%
\pgfpathlineto{\pgfqpoint{1.034549in}{0.663222in}}%
\pgfpathlineto{\pgfqpoint{1.021280in}{0.658738in}}%
\pgfpathlineto{\pgfqpoint{1.017156in}{0.651377in}}%
\pgfpathlineto{\pgfqpoint{1.015516in}{0.638450in}}%
\pgfpathlineto{\pgfqpoint{1.017778in}{0.625523in}}%
\pgfpathclose%
\pgfpathmoveto{\pgfqpoint{1.025895in}{0.638450in}}%
\pgfpathlineto{\pgfqpoint{1.034549in}{0.650936in}}%
\pgfpathlineto{\pgfqpoint{1.038991in}{0.638450in}}%
\pgfpathlineto{\pgfqpoint{1.034549in}{0.628305in}}%
\pgfpathclose%
\pgfusepath{fill}%
\end{pgfscope}%
\begin{pgfscope}%
\pgfpathrectangle{\pgfqpoint{0.211875in}{0.211875in}}{\pgfqpoint{1.313625in}{1.279725in}}%
\pgfusepath{clip}%
\pgfsetbuttcap%
\pgfsetroundjoin%
\definecolor{currentfill}{rgb}{0.965440,0.720101,0.576404}%
\pgfsetfillcolor{currentfill}%
\pgfsetlinewidth{0.000000pt}%
\definecolor{currentstroke}{rgb}{0.000000,0.000000,0.000000}%
\pgfsetstrokecolor{currentstroke}%
\pgfsetdash{}{0pt}%
\pgfpathmoveto{\pgfqpoint{1.140701in}{0.618036in}}%
\pgfpathlineto{\pgfqpoint{1.153970in}{0.615362in}}%
\pgfpathlineto{\pgfqpoint{1.165152in}{0.625523in}}%
\pgfpathlineto{\pgfqpoint{1.167239in}{0.633014in}}%
\pgfpathlineto{\pgfqpoint{1.167971in}{0.638450in}}%
\pgfpathlineto{\pgfqpoint{1.167239in}{0.645820in}}%
\pgfpathlineto{\pgfqpoint{1.166105in}{0.651377in}}%
\pgfpathlineto{\pgfqpoint{1.153970in}{0.663305in}}%
\pgfpathlineto{\pgfqpoint{1.140701in}{0.660583in}}%
\pgfpathlineto{\pgfqpoint{1.134891in}{0.651377in}}%
\pgfpathlineto{\pgfqpoint{1.133179in}{0.638450in}}%
\pgfpathlineto{\pgfqpoint{1.135555in}{0.625523in}}%
\pgfpathclose%
\pgfpathmoveto{\pgfqpoint{1.141045in}{0.638450in}}%
\pgfpathlineto{\pgfqpoint{1.153970in}{0.650698in}}%
\pgfpathlineto{\pgfqpoint{1.157668in}{0.638450in}}%
\pgfpathlineto{\pgfqpoint{1.153970in}{0.628519in}}%
\pgfpathclose%
\pgfusepath{fill}%
\end{pgfscope}%
\begin{pgfscope}%
\pgfpathrectangle{\pgfqpoint{0.211875in}{0.211875in}}{\pgfqpoint{1.313625in}{1.279725in}}%
\pgfusepath{clip}%
\pgfsetbuttcap%
\pgfsetroundjoin%
\definecolor{currentfill}{rgb}{0.965440,0.720101,0.576404}%
\pgfsetfillcolor{currentfill}%
\pgfsetlinewidth{0.000000pt}%
\definecolor{currentstroke}{rgb}{0.000000,0.000000,0.000000}%
\pgfsetstrokecolor{currentstroke}%
\pgfsetdash{}{0pt}%
\pgfpathmoveto{\pgfqpoint{1.260121in}{0.617392in}}%
\pgfpathlineto{\pgfqpoint{1.273390in}{0.616227in}}%
\pgfpathlineto{\pgfqpoint{1.282245in}{0.625523in}}%
\pgfpathlineto{\pgfqpoint{1.285314in}{0.638450in}}%
\pgfpathlineto{\pgfqpoint{1.283090in}{0.651377in}}%
\pgfpathlineto{\pgfqpoint{1.273390in}{0.662403in}}%
\pgfpathlineto{\pgfqpoint{1.260121in}{0.661222in}}%
\pgfpathlineto{\pgfqpoint{1.253100in}{0.651377in}}%
\pgfpathlineto{\pgfqpoint{1.251218in}{0.638450in}}%
\pgfpathlineto{\pgfqpoint{1.253822in}{0.625523in}}%
\pgfpathclose%
\pgfpathmoveto{\pgfqpoint{1.259441in}{0.638450in}}%
\pgfpathlineto{\pgfqpoint{1.260121in}{0.641680in}}%
\pgfpathlineto{\pgfqpoint{1.273390in}{0.647272in}}%
\pgfpathlineto{\pgfqpoint{1.275693in}{0.638450in}}%
\pgfpathlineto{\pgfqpoint{1.273390in}{0.631320in}}%
\pgfpathlineto{\pgfqpoint{1.260121in}{0.635862in}}%
\pgfpathclose%
\pgfusepath{fill}%
\end{pgfscope}%
\begin{pgfscope}%
\pgfpathrectangle{\pgfqpoint{0.211875in}{0.211875in}}{\pgfqpoint{1.313625in}{1.279725in}}%
\pgfusepath{clip}%
\pgfsetbuttcap%
\pgfsetroundjoin%
\definecolor{currentfill}{rgb}{0.965440,0.720101,0.576404}%
\pgfsetfillcolor{currentfill}%
\pgfsetlinewidth{0.000000pt}%
\definecolor{currentstroke}{rgb}{0.000000,0.000000,0.000000}%
\pgfsetstrokecolor{currentstroke}%
\pgfsetdash{}{0pt}%
\pgfpathmoveto{\pgfqpoint{1.379542in}{0.617739in}}%
\pgfpathlineto{\pgfqpoint{1.392811in}{0.618086in}}%
\pgfpathlineto{\pgfqpoint{1.399022in}{0.625523in}}%
\pgfpathlineto{\pgfqpoint{1.401916in}{0.638450in}}%
\pgfpathlineto{\pgfqpoint{1.399789in}{0.651377in}}%
\pgfpathlineto{\pgfqpoint{1.392811in}{0.660439in}}%
\pgfpathlineto{\pgfqpoint{1.379542in}{0.660804in}}%
\pgfpathlineto{\pgfqpoint{1.371898in}{0.651377in}}%
\pgfpathlineto{\pgfqpoint{1.369718in}{0.638450in}}%
\pgfpathlineto{\pgfqpoint{1.372698in}{0.625523in}}%
\pgfpathclose%
\pgfpathmoveto{\pgfqpoint{1.378752in}{0.638450in}}%
\pgfpathlineto{\pgfqpoint{1.379542in}{0.641743in}}%
\pgfpathlineto{\pgfqpoint{1.392811in}{0.640383in}}%
\pgfpathlineto{\pgfqpoint{1.393252in}{0.638450in}}%
\pgfpathlineto{\pgfqpoint{1.392811in}{0.636895in}}%
\pgfpathlineto{\pgfqpoint{1.379542in}{0.635797in}}%
\pgfpathclose%
\pgfusepath{fill}%
\end{pgfscope}%
\begin{pgfscope}%
\pgfpathrectangle{\pgfqpoint{0.211875in}{0.211875in}}{\pgfqpoint{1.313625in}{1.279725in}}%
\pgfusepath{clip}%
\pgfsetbuttcap%
\pgfsetroundjoin%
\definecolor{currentfill}{rgb}{0.965440,0.720101,0.576404}%
\pgfsetfillcolor{currentfill}%
\pgfsetlinewidth{0.000000pt}%
\definecolor{currentstroke}{rgb}{0.000000,0.000000,0.000000}%
\pgfsetstrokecolor{currentstroke}%
\pgfsetdash{}{0pt}%
\pgfpathmoveto{\pgfqpoint{1.498962in}{0.618998in}}%
\pgfpathlineto{\pgfqpoint{1.512231in}{0.621054in}}%
\pgfpathlineto{\pgfqpoint{1.515532in}{0.625523in}}%
\pgfpathlineto{\pgfqpoint{1.518373in}{0.638450in}}%
\pgfpathlineto{\pgfqpoint{1.516245in}{0.651377in}}%
\pgfpathlineto{\pgfqpoint{1.512231in}{0.657283in}}%
\pgfpathlineto{\pgfqpoint{1.498962in}{0.659418in}}%
\pgfpathlineto{\pgfqpoint{1.491477in}{0.651377in}}%
\pgfpathlineto{\pgfqpoint{1.488824in}{0.638450in}}%
\pgfpathlineto{\pgfqpoint{1.492386in}{0.625523in}}%
\pgfpathclose%
\pgfpathmoveto{\pgfqpoint{1.498919in}{0.638450in}}%
\pgfpathlineto{\pgfqpoint{1.498962in}{0.638606in}}%
\pgfpathlineto{\pgfqpoint{1.499208in}{0.638450in}}%
\pgfpathlineto{\pgfqpoint{1.498962in}{0.638324in}}%
\pgfpathclose%
\pgfusepath{fill}%
\end{pgfscope}%
\begin{pgfscope}%
\pgfpathrectangle{\pgfqpoint{0.211875in}{0.211875in}}{\pgfqpoint{1.313625in}{1.279725in}}%
\pgfusepath{clip}%
\pgfsetbuttcap%
\pgfsetroundjoin%
\definecolor{currentfill}{rgb}{0.965440,0.720101,0.576404}%
\pgfsetfillcolor{currentfill}%
\pgfsetlinewidth{0.000000pt}%
\definecolor{currentstroke}{rgb}{0.000000,0.000000,0.000000}%
\pgfsetstrokecolor{currentstroke}%
\pgfsetdash{}{0pt}%
\pgfpathmoveto{\pgfqpoint{0.437447in}{0.636221in}}%
\pgfpathlineto{\pgfqpoint{0.445477in}{0.638450in}}%
\pgfpathlineto{\pgfqpoint{0.437447in}{0.641207in}}%
\pgfpathlineto{\pgfqpoint{0.436744in}{0.638450in}}%
\pgfpathclose%
\pgfusepath{fill}%
\end{pgfscope}%
\begin{pgfscope}%
\pgfpathrectangle{\pgfqpoint{0.211875in}{0.211875in}}{\pgfqpoint{1.313625in}{1.279725in}}%
\pgfusepath{clip}%
\pgfsetbuttcap%
\pgfsetroundjoin%
\definecolor{currentfill}{rgb}{0.965440,0.720101,0.576404}%
\pgfsetfillcolor{currentfill}%
\pgfsetlinewidth{0.000000pt}%
\definecolor{currentstroke}{rgb}{0.000000,0.000000,0.000000}%
\pgfsetstrokecolor{currentstroke}%
\pgfsetdash{}{0pt}%
\pgfpathmoveto{\pgfqpoint{0.623212in}{0.701374in}}%
\pgfpathlineto{\pgfqpoint{0.625667in}{0.703083in}}%
\pgfpathlineto{\pgfqpoint{0.632700in}{0.716009in}}%
\pgfpathlineto{\pgfqpoint{0.631547in}{0.728936in}}%
\pgfpathlineto{\pgfqpoint{0.623212in}{0.738912in}}%
\pgfpathlineto{\pgfqpoint{0.609943in}{0.734002in}}%
\pgfpathlineto{\pgfqpoint{0.607514in}{0.728936in}}%
\pgfpathlineto{\pgfqpoint{0.606740in}{0.716009in}}%
\pgfpathlineto{\pgfqpoint{0.609943in}{0.706526in}}%
\pgfpathlineto{\pgfqpoint{0.616265in}{0.703083in}}%
\pgfpathclose%
\pgfusepath{fill}%
\end{pgfscope}%
\begin{pgfscope}%
\pgfpathrectangle{\pgfqpoint{0.211875in}{0.211875in}}{\pgfqpoint{1.313625in}{1.279725in}}%
\pgfusepath{clip}%
\pgfsetbuttcap%
\pgfsetroundjoin%
\definecolor{currentfill}{rgb}{0.965440,0.720101,0.576404}%
\pgfsetfillcolor{currentfill}%
\pgfsetlinewidth{0.000000pt}%
\definecolor{currentstroke}{rgb}{0.000000,0.000000,0.000000}%
\pgfsetstrokecolor{currentstroke}%
\pgfsetdash{}{0pt}%
\pgfpathmoveto{\pgfqpoint{0.729364in}{0.700782in}}%
\pgfpathlineto{\pgfqpoint{0.742633in}{0.699157in}}%
\pgfpathlineto{\pgfqpoint{0.747482in}{0.703083in}}%
\pgfpathlineto{\pgfqpoint{0.753008in}{0.716009in}}%
\pgfpathlineto{\pgfqpoint{0.752137in}{0.728936in}}%
\pgfpathlineto{\pgfqpoint{0.742936in}{0.741862in}}%
\pgfpathlineto{\pgfqpoint{0.742633in}{0.742047in}}%
\pgfpathlineto{\pgfqpoint{0.740838in}{0.741862in}}%
\pgfpathlineto{\pgfqpoint{0.729364in}{0.739865in}}%
\pgfpathlineto{\pgfqpoint{0.723455in}{0.728936in}}%
\pgfpathlineto{\pgfqpoint{0.722734in}{0.716009in}}%
\pgfpathlineto{\pgfqpoint{0.727234in}{0.703083in}}%
\pgfpathclose%
\pgfpathmoveto{\pgfqpoint{0.740819in}{0.716009in}}%
\pgfpathlineto{\pgfqpoint{0.742633in}{0.718723in}}%
\pgfpathlineto{\pgfqpoint{0.742919in}{0.716009in}}%
\pgfpathlineto{\pgfqpoint{0.742633in}{0.715439in}}%
\pgfpathclose%
\pgfusepath{fill}%
\end{pgfscope}%
\begin{pgfscope}%
\pgfpathrectangle{\pgfqpoint{0.211875in}{0.211875in}}{\pgfqpoint{1.313625in}{1.279725in}}%
\pgfusepath{clip}%
\pgfsetbuttcap%
\pgfsetroundjoin%
\definecolor{currentfill}{rgb}{0.965440,0.720101,0.576404}%
\pgfsetfillcolor{currentfill}%
\pgfsetlinewidth{0.000000pt}%
\definecolor{currentstroke}{rgb}{0.000000,0.000000,0.000000}%
\pgfsetstrokecolor{currentstroke}%
\pgfsetdash{}{0pt}%
\pgfpathmoveto{\pgfqpoint{0.848784in}{0.697871in}}%
\pgfpathlineto{\pgfqpoint{0.862053in}{0.697692in}}%
\pgfpathlineto{\pgfqpoint{0.867863in}{0.703083in}}%
\pgfpathlineto{\pgfqpoint{0.872393in}{0.716009in}}%
\pgfpathlineto{\pgfqpoint{0.871704in}{0.728936in}}%
\pgfpathlineto{\pgfqpoint{0.864265in}{0.741862in}}%
\pgfpathlineto{\pgfqpoint{0.862053in}{0.743406in}}%
\pgfpathlineto{\pgfqpoint{0.848784in}{0.743242in}}%
\pgfpathlineto{\pgfqpoint{0.846876in}{0.741862in}}%
\pgfpathlineto{\pgfqpoint{0.839518in}{0.728936in}}%
\pgfpathlineto{\pgfqpoint{0.838824in}{0.716009in}}%
\pgfpathlineto{\pgfqpoint{0.843323in}{0.703083in}}%
\pgfpathclose%
\pgfpathmoveto{\pgfqpoint{0.847665in}{0.716009in}}%
\pgfpathlineto{\pgfqpoint{0.848784in}{0.728493in}}%
\pgfpathlineto{\pgfqpoint{0.852373in}{0.728936in}}%
\pgfpathlineto{\pgfqpoint{0.862053in}{0.729119in}}%
\pgfpathlineto{\pgfqpoint{0.862168in}{0.728936in}}%
\pgfpathlineto{\pgfqpoint{0.863347in}{0.716009in}}%
\pgfpathlineto{\pgfqpoint{0.862053in}{0.713060in}}%
\pgfpathlineto{\pgfqpoint{0.848784in}{0.713402in}}%
\pgfpathclose%
\pgfusepath{fill}%
\end{pgfscope}%
\begin{pgfscope}%
\pgfpathrectangle{\pgfqpoint{0.211875in}{0.211875in}}{\pgfqpoint{1.313625in}{1.279725in}}%
\pgfusepath{clip}%
\pgfsetbuttcap%
\pgfsetroundjoin%
\definecolor{currentfill}{rgb}{0.965440,0.720101,0.576404}%
\pgfsetfillcolor{currentfill}%
\pgfsetlinewidth{0.000000pt}%
\definecolor{currentstroke}{rgb}{0.000000,0.000000,0.000000}%
\pgfsetstrokecolor{currentstroke}%
\pgfsetdash{}{0pt}%
\pgfpathmoveto{\pgfqpoint{0.968205in}{0.695944in}}%
\pgfpathlineto{\pgfqpoint{0.981473in}{0.697006in}}%
\pgfpathlineto{\pgfqpoint{0.987245in}{0.703083in}}%
\pgfpathlineto{\pgfqpoint{0.991124in}{0.716009in}}%
\pgfpathlineto{\pgfqpoint{0.990551in}{0.728936in}}%
\pgfpathlineto{\pgfqpoint{0.984244in}{0.741862in}}%
\pgfpathlineto{\pgfqpoint{0.981473in}{0.744064in}}%
\pgfpathlineto{\pgfqpoint{0.968205in}{0.744977in}}%
\pgfpathlineto{\pgfqpoint{0.963270in}{0.741862in}}%
\pgfpathlineto{\pgfqpoint{0.955676in}{0.728936in}}%
\pgfpathlineto{\pgfqpoint{0.954977in}{0.716009in}}%
\pgfpathlineto{\pgfqpoint{0.959659in}{0.703083in}}%
\pgfpathclose%
\pgfpathmoveto{\pgfqpoint{0.964814in}{0.716009in}}%
\pgfpathlineto{\pgfqpoint{0.966035in}{0.728936in}}%
\pgfpathlineto{\pgfqpoint{0.968205in}{0.732034in}}%
\pgfpathlineto{\pgfqpoint{0.981473in}{0.729635in}}%
\pgfpathlineto{\pgfqpoint{0.981859in}{0.728936in}}%
\pgfpathlineto{\pgfqpoint{0.982877in}{0.716009in}}%
\pgfpathlineto{\pgfqpoint{0.981473in}{0.712386in}}%
\pgfpathlineto{\pgfqpoint{0.968205in}{0.709082in}}%
\pgfpathclose%
\pgfusepath{fill}%
\end{pgfscope}%
\begin{pgfscope}%
\pgfpathrectangle{\pgfqpoint{0.211875in}{0.211875in}}{\pgfqpoint{1.313625in}{1.279725in}}%
\pgfusepath{clip}%
\pgfsetbuttcap%
\pgfsetroundjoin%
\definecolor{currentfill}{rgb}{0.965440,0.720101,0.576404}%
\pgfsetfillcolor{currentfill}%
\pgfsetlinewidth{0.000000pt}%
\definecolor{currentstroke}{rgb}{0.000000,0.000000,0.000000}%
\pgfsetstrokecolor{currentstroke}%
\pgfsetdash{}{0pt}%
\pgfpathmoveto{\pgfqpoint{1.087625in}{0.694895in}}%
\pgfpathlineto{\pgfqpoint{1.100894in}{0.697161in}}%
\pgfpathlineto{\pgfqpoint{1.105880in}{0.703083in}}%
\pgfpathlineto{\pgfqpoint{1.109360in}{0.716009in}}%
\pgfpathlineto{\pgfqpoint{1.108853in}{0.728936in}}%
\pgfpathlineto{\pgfqpoint{1.103229in}{0.741862in}}%
\pgfpathlineto{\pgfqpoint{1.100894in}{0.743961in}}%
\pgfpathlineto{\pgfqpoint{1.087625in}{0.745917in}}%
\pgfpathlineto{\pgfqpoint{1.080174in}{0.741862in}}%
\pgfpathlineto{\pgfqpoint{1.074356in}{0.733777in}}%
\pgfpathlineto{\pgfqpoint{1.073017in}{0.728936in}}%
\pgfpathlineto{\pgfqpoint{1.072608in}{0.716009in}}%
\pgfpathlineto{\pgfqpoint{1.074356in}{0.707302in}}%
\pgfpathlineto{\pgfqpoint{1.076277in}{0.703083in}}%
\pgfpathclose%
\pgfpathmoveto{\pgfqpoint{1.082325in}{0.716009in}}%
\pgfpathlineto{\pgfqpoint{1.083668in}{0.728936in}}%
\pgfpathlineto{\pgfqpoint{1.087625in}{0.733816in}}%
\pgfpathlineto{\pgfqpoint{1.100544in}{0.728936in}}%
\pgfpathlineto{\pgfqpoint{1.100894in}{0.727950in}}%
\pgfpathlineto{\pgfqpoint{1.101736in}{0.716009in}}%
\pgfpathlineto{\pgfqpoint{1.100894in}{0.713562in}}%
\pgfpathlineto{\pgfqpoint{1.087625in}{0.706640in}}%
\pgfpathclose%
\pgfusepath{fill}%
\end{pgfscope}%
\begin{pgfscope}%
\pgfpathrectangle{\pgfqpoint{0.211875in}{0.211875in}}{\pgfqpoint{1.313625in}{1.279725in}}%
\pgfusepath{clip}%
\pgfsetbuttcap%
\pgfsetroundjoin%
\definecolor{currentfill}{rgb}{0.965440,0.720101,0.576404}%
\pgfsetfillcolor{currentfill}%
\pgfsetlinewidth{0.000000pt}%
\definecolor{currentstroke}{rgb}{0.000000,0.000000,0.000000}%
\pgfsetstrokecolor{currentstroke}%
\pgfsetdash{}{0pt}%
\pgfpathmoveto{\pgfqpoint{1.193777in}{0.702626in}}%
\pgfpathlineto{\pgfqpoint{1.207045in}{0.694659in}}%
\pgfpathlineto{\pgfqpoint{1.220314in}{0.698265in}}%
\pgfpathlineto{\pgfqpoint{1.223925in}{0.703083in}}%
\pgfpathlineto{\pgfqpoint{1.227194in}{0.716009in}}%
\pgfpathlineto{\pgfqpoint{1.226716in}{0.728936in}}%
\pgfpathlineto{\pgfqpoint{1.221435in}{0.741862in}}%
\pgfpathlineto{\pgfqpoint{1.220314in}{0.742998in}}%
\pgfpathlineto{\pgfqpoint{1.207045in}{0.746120in}}%
\pgfpathlineto{\pgfqpoint{1.197784in}{0.741862in}}%
\pgfpathlineto{\pgfqpoint{1.193777in}{0.737539in}}%
\pgfpathlineto{\pgfqpoint{1.191063in}{0.728936in}}%
\pgfpathlineto{\pgfqpoint{1.190643in}{0.716009in}}%
\pgfpathlineto{\pgfqpoint{1.193525in}{0.703083in}}%
\pgfpathclose%
\pgfpathmoveto{\pgfqpoint{1.200330in}{0.716009in}}%
\pgfpathlineto{\pgfqpoint{1.201892in}{0.728936in}}%
\pgfpathlineto{\pgfqpoint{1.207045in}{0.734322in}}%
\pgfpathlineto{\pgfqpoint{1.216571in}{0.728936in}}%
\pgfpathlineto{\pgfqpoint{1.219465in}{0.716009in}}%
\pgfpathlineto{\pgfqpoint{1.207045in}{0.705938in}}%
\pgfpathclose%
\pgfusepath{fill}%
\end{pgfscope}%
\begin{pgfscope}%
\pgfpathrectangle{\pgfqpoint{0.211875in}{0.211875in}}{\pgfqpoint{1.313625in}{1.279725in}}%
\pgfusepath{clip}%
\pgfsetbuttcap%
\pgfsetroundjoin%
\definecolor{currentfill}{rgb}{0.965440,0.720101,0.576404}%
\pgfsetfillcolor{currentfill}%
\pgfsetlinewidth{0.000000pt}%
\definecolor{currentstroke}{rgb}{0.000000,0.000000,0.000000}%
\pgfsetstrokecolor{currentstroke}%
\pgfsetdash{}{0pt}%
\pgfpathmoveto{\pgfqpoint{1.313197in}{0.701455in}}%
\pgfpathlineto{\pgfqpoint{1.326466in}{0.695203in}}%
\pgfpathlineto{\pgfqpoint{1.339735in}{0.700482in}}%
\pgfpathlineto{\pgfqpoint{1.341471in}{0.703083in}}%
\pgfpathlineto{\pgfqpoint{1.344680in}{0.716009in}}%
\pgfpathlineto{\pgfqpoint{1.344201in}{0.728936in}}%
\pgfpathlineto{\pgfqpoint{1.339735in}{0.740511in}}%
\pgfpathlineto{\pgfqpoint{1.337760in}{0.741862in}}%
\pgfpathlineto{\pgfqpoint{1.326466in}{0.745619in}}%
\pgfpathlineto{\pgfqpoint{1.316501in}{0.741862in}}%
\pgfpathlineto{\pgfqpoint{1.313197in}{0.739156in}}%
\pgfpathlineto{\pgfqpoint{1.309547in}{0.728936in}}%
\pgfpathlineto{\pgfqpoint{1.309090in}{0.716009in}}%
\pgfpathlineto{\pgfqpoint{1.312185in}{0.703083in}}%
\pgfpathclose%
\pgfpathmoveto{\pgfqpoint{1.319100in}{0.716009in}}%
\pgfpathlineto{\pgfqpoint{1.321038in}{0.728936in}}%
\pgfpathlineto{\pgfqpoint{1.326466in}{0.733606in}}%
\pgfpathlineto{\pgfqpoint{1.332696in}{0.728936in}}%
\pgfpathlineto{\pgfqpoint{1.334932in}{0.716009in}}%
\pgfpathlineto{\pgfqpoint{1.326466in}{0.706906in}}%
\pgfpathclose%
\pgfusepath{fill}%
\end{pgfscope}%
\begin{pgfscope}%
\pgfpathrectangle{\pgfqpoint{0.211875in}{0.211875in}}{\pgfqpoint{1.313625in}{1.279725in}}%
\pgfusepath{clip}%
\pgfsetbuttcap%
\pgfsetroundjoin%
\definecolor{currentfill}{rgb}{0.965440,0.720101,0.576404}%
\pgfsetfillcolor{currentfill}%
\pgfsetlinewidth{0.000000pt}%
\definecolor{currentstroke}{rgb}{0.000000,0.000000,0.000000}%
\pgfsetstrokecolor{currentstroke}%
\pgfsetdash{}{0pt}%
\pgfpathmoveto{\pgfqpoint{1.432617in}{0.701417in}}%
\pgfpathlineto{\pgfqpoint{1.445886in}{0.696527in}}%
\pgfpathlineto{\pgfqpoint{1.457844in}{0.703083in}}%
\pgfpathlineto{\pgfqpoint{1.459155in}{0.705006in}}%
\pgfpathlineto{\pgfqpoint{1.461847in}{0.716009in}}%
\pgfpathlineto{\pgfqpoint{1.461342in}{0.728936in}}%
\pgfpathlineto{\pgfqpoint{1.459155in}{0.735328in}}%
\pgfpathlineto{\pgfqpoint{1.452034in}{0.741862in}}%
\pgfpathlineto{\pgfqpoint{1.445886in}{0.744414in}}%
\pgfpathlineto{\pgfqpoint{1.437225in}{0.741862in}}%
\pgfpathlineto{\pgfqpoint{1.432617in}{0.739114in}}%
\pgfpathlineto{\pgfqpoint{1.428518in}{0.728936in}}%
\pgfpathlineto{\pgfqpoint{1.427991in}{0.716009in}}%
\pgfpathlineto{\pgfqpoint{1.431454in}{0.703083in}}%
\pgfpathclose%
\pgfpathmoveto{\pgfqpoint{1.439235in}{0.716009in}}%
\pgfpathlineto{\pgfqpoint{1.441839in}{0.728936in}}%
\pgfpathlineto{\pgfqpoint{1.445886in}{0.731676in}}%
\pgfpathlineto{\pgfqpoint{1.448822in}{0.728936in}}%
\pgfpathlineto{\pgfqpoint{1.450723in}{0.716009in}}%
\pgfpathlineto{\pgfqpoint{1.445886in}{0.709537in}}%
\pgfpathclose%
\pgfusepath{fill}%
\end{pgfscope}%
\begin{pgfscope}%
\pgfpathrectangle{\pgfqpoint{0.211875in}{0.211875in}}{\pgfqpoint{1.313625in}{1.279725in}}%
\pgfusepath{clip}%
\pgfsetbuttcap%
\pgfsetroundjoin%
\definecolor{currentfill}{rgb}{0.965440,0.720101,0.576404}%
\pgfsetfillcolor{currentfill}%
\pgfsetlinewidth{0.000000pt}%
\definecolor{currentstroke}{rgb}{0.000000,0.000000,0.000000}%
\pgfsetstrokecolor{currentstroke}%
\pgfsetdash{}{0pt}%
\pgfpathmoveto{\pgfqpoint{0.384371in}{0.713005in}}%
\pgfpathlineto{\pgfqpoint{0.386901in}{0.716009in}}%
\pgfpathlineto{\pgfqpoint{0.384607in}{0.728936in}}%
\pgfpathlineto{\pgfqpoint{0.384371in}{0.729130in}}%
\pgfpathlineto{\pgfqpoint{0.384124in}{0.728936in}}%
\pgfpathlineto{\pgfqpoint{0.381698in}{0.716009in}}%
\pgfpathclose%
\pgfusepath{fill}%
\end{pgfscope}%
\begin{pgfscope}%
\pgfpathrectangle{\pgfqpoint{0.211875in}{0.211875in}}{\pgfqpoint{1.313625in}{1.279725in}}%
\pgfusepath{clip}%
\pgfsetbuttcap%
\pgfsetroundjoin%
\definecolor{currentfill}{rgb}{0.965440,0.720101,0.576404}%
\pgfsetfillcolor{currentfill}%
\pgfsetlinewidth{0.000000pt}%
\definecolor{currentstroke}{rgb}{0.000000,0.000000,0.000000}%
\pgfsetstrokecolor{currentstroke}%
\pgfsetdash{}{0pt}%
\pgfpathmoveto{\pgfqpoint{0.503792in}{0.705578in}}%
\pgfpathlineto{\pgfqpoint{0.510981in}{0.716009in}}%
\pgfpathlineto{\pgfqpoint{0.509391in}{0.728936in}}%
\pgfpathlineto{\pgfqpoint{0.503792in}{0.734586in}}%
\pgfpathlineto{\pgfqpoint{0.494398in}{0.728936in}}%
\pgfpathlineto{\pgfqpoint{0.491658in}{0.716009in}}%
\pgfpathclose%
\pgfusepath{fill}%
\end{pgfscope}%
\begin{pgfscope}%
\pgfpathrectangle{\pgfqpoint{0.211875in}{0.211875in}}{\pgfqpoint{1.313625in}{1.279725in}}%
\pgfusepath{clip}%
\pgfsetbuttcap%
\pgfsetroundjoin%
\definecolor{currentfill}{rgb}{0.965440,0.720101,0.576404}%
\pgfsetfillcolor{currentfill}%
\pgfsetlinewidth{0.000000pt}%
\definecolor{currentstroke}{rgb}{0.000000,0.000000,0.000000}%
\pgfsetstrokecolor{currentstroke}%
\pgfsetdash{}{0pt}%
\pgfpathmoveto{\pgfqpoint{0.676288in}{0.779137in}}%
\pgfpathlineto{\pgfqpoint{0.682678in}{0.780642in}}%
\pgfpathlineto{\pgfqpoint{0.689557in}{0.783836in}}%
\pgfpathlineto{\pgfqpoint{0.693691in}{0.793568in}}%
\pgfpathlineto{\pgfqpoint{0.694168in}{0.806495in}}%
\pgfpathlineto{\pgfqpoint{0.689891in}{0.819421in}}%
\pgfpathlineto{\pgfqpoint{0.689557in}{0.819830in}}%
\pgfpathlineto{\pgfqpoint{0.676288in}{0.823441in}}%
\pgfpathlineto{\pgfqpoint{0.669702in}{0.819421in}}%
\pgfpathlineto{\pgfqpoint{0.663019in}{0.808082in}}%
\pgfpathlineto{\pgfqpoint{0.662653in}{0.806495in}}%
\pgfpathlineto{\pgfqpoint{0.663019in}{0.795025in}}%
\pgfpathlineto{\pgfqpoint{0.663114in}{0.793568in}}%
\pgfpathlineto{\pgfqpoint{0.673411in}{0.780642in}}%
\pgfpathclose%
\pgfpathmoveto{\pgfqpoint{0.675893in}{0.793568in}}%
\pgfpathlineto{\pgfqpoint{0.674814in}{0.806495in}}%
\pgfpathlineto{\pgfqpoint{0.676288in}{0.808647in}}%
\pgfpathlineto{\pgfqpoint{0.679450in}{0.806495in}}%
\pgfpathlineto{\pgfqpoint{0.677138in}{0.793568in}}%
\pgfpathlineto{\pgfqpoint{0.676288in}{0.793105in}}%
\pgfpathclose%
\pgfusepath{fill}%
\end{pgfscope}%
\begin{pgfscope}%
\pgfpathrectangle{\pgfqpoint{0.211875in}{0.211875in}}{\pgfqpoint{1.313625in}{1.279725in}}%
\pgfusepath{clip}%
\pgfsetbuttcap%
\pgfsetroundjoin%
\definecolor{currentfill}{rgb}{0.965440,0.720101,0.576404}%
\pgfsetfillcolor{currentfill}%
\pgfsetlinewidth{0.000000pt}%
\definecolor{currentstroke}{rgb}{0.000000,0.000000,0.000000}%
\pgfsetstrokecolor{currentstroke}%
\pgfsetdash{}{0pt}%
\pgfpathmoveto{\pgfqpoint{0.795708in}{0.776916in}}%
\pgfpathlineto{\pgfqpoint{0.807207in}{0.780642in}}%
\pgfpathlineto{\pgfqpoint{0.808977in}{0.781846in}}%
\pgfpathlineto{\pgfqpoint{0.813415in}{0.793568in}}%
\pgfpathlineto{\pgfqpoint{0.813832in}{0.806495in}}%
\pgfpathlineto{\pgfqpoint{0.810258in}{0.819421in}}%
\pgfpathlineto{\pgfqpoint{0.808977in}{0.821185in}}%
\pgfpathlineto{\pgfqpoint{0.795708in}{0.825846in}}%
\pgfpathlineto{\pgfqpoint{0.783044in}{0.819421in}}%
\pgfpathlineto{\pgfqpoint{0.782439in}{0.818632in}}%
\pgfpathlineto{\pgfqpoint{0.779273in}{0.806495in}}%
\pgfpathlineto{\pgfqpoint{0.779672in}{0.793568in}}%
\pgfpathlineto{\pgfqpoint{0.782439in}{0.785223in}}%
\pgfpathlineto{\pgfqpoint{0.787142in}{0.780642in}}%
\pgfpathclose%
\pgfpathmoveto{\pgfqpoint{0.791352in}{0.793568in}}%
\pgfpathlineto{\pgfqpoint{0.790135in}{0.806495in}}%
\pgfpathlineto{\pgfqpoint{0.795708in}{0.813266in}}%
\pgfpathlineto{\pgfqpoint{0.803023in}{0.806495in}}%
\pgfpathlineto{\pgfqpoint{0.801432in}{0.793568in}}%
\pgfpathlineto{\pgfqpoint{0.795708in}{0.789324in}}%
\pgfpathclose%
\pgfusepath{fill}%
\end{pgfscope}%
\begin{pgfscope}%
\pgfpathrectangle{\pgfqpoint{0.211875in}{0.211875in}}{\pgfqpoint{1.313625in}{1.279725in}}%
\pgfusepath{clip}%
\pgfsetbuttcap%
\pgfsetroundjoin%
\definecolor{currentfill}{rgb}{0.965440,0.720101,0.576404}%
\pgfsetfillcolor{currentfill}%
\pgfsetlinewidth{0.000000pt}%
\definecolor{currentstroke}{rgb}{0.000000,0.000000,0.000000}%
\pgfsetstrokecolor{currentstroke}%
\pgfsetdash{}{0pt}%
\pgfpathmoveto{\pgfqpoint{0.901860in}{0.779870in}}%
\pgfpathlineto{\pgfqpoint{0.915129in}{0.775426in}}%
\pgfpathlineto{\pgfqpoint{0.927796in}{0.780642in}}%
\pgfpathlineto{\pgfqpoint{0.928398in}{0.781197in}}%
\pgfpathlineto{\pgfqpoint{0.932568in}{0.793568in}}%
\pgfpathlineto{\pgfqpoint{0.932944in}{0.806495in}}%
\pgfpathlineto{\pgfqpoint{0.929846in}{0.819421in}}%
\pgfpathlineto{\pgfqpoint{0.928398in}{0.821671in}}%
\pgfpathlineto{\pgfqpoint{0.915129in}{0.827464in}}%
\pgfpathlineto{\pgfqpoint{0.901860in}{0.822823in}}%
\pgfpathlineto{\pgfqpoint{0.899443in}{0.819421in}}%
\pgfpathlineto{\pgfqpoint{0.896181in}{0.806495in}}%
\pgfpathlineto{\pgfqpoint{0.896574in}{0.793568in}}%
\pgfpathlineto{\pgfqpoint{0.901214in}{0.780642in}}%
\pgfpathclose%
\pgfpathmoveto{\pgfqpoint{0.906503in}{0.793568in}}%
\pgfpathlineto{\pgfqpoint{0.905044in}{0.806495in}}%
\pgfpathlineto{\pgfqpoint{0.915129in}{0.816290in}}%
\pgfpathlineto{\pgfqpoint{0.923506in}{0.806495in}}%
\pgfpathlineto{\pgfqpoint{0.922299in}{0.793568in}}%
\pgfpathlineto{\pgfqpoint{0.915129in}{0.786850in}}%
\pgfpathclose%
\pgfusepath{fill}%
\end{pgfscope}%
\begin{pgfscope}%
\pgfpathrectangle{\pgfqpoint{0.211875in}{0.211875in}}{\pgfqpoint{1.313625in}{1.279725in}}%
\pgfusepath{clip}%
\pgfsetbuttcap%
\pgfsetroundjoin%
\definecolor{currentfill}{rgb}{0.965440,0.720101,0.576404}%
\pgfsetfillcolor{currentfill}%
\pgfsetlinewidth{0.000000pt}%
\definecolor{currentstroke}{rgb}{0.000000,0.000000,0.000000}%
\pgfsetstrokecolor{currentstroke}%
\pgfsetdash{}{0pt}%
\pgfpathmoveto{\pgfqpoint{1.021280in}{0.777590in}}%
\pgfpathlineto{\pgfqpoint{1.034549in}{0.774638in}}%
\pgfpathlineto{\pgfqpoint{1.046559in}{0.780642in}}%
\pgfpathlineto{\pgfqpoint{1.047818in}{0.782160in}}%
\pgfpathlineto{\pgfqpoint{1.051233in}{0.793568in}}%
\pgfpathlineto{\pgfqpoint{1.051582in}{0.806495in}}%
\pgfpathlineto{\pgfqpoint{1.048775in}{0.819421in}}%
\pgfpathlineto{\pgfqpoint{1.047818in}{0.821104in}}%
\pgfpathlineto{\pgfqpoint{1.034549in}{0.828325in}}%
\pgfpathlineto{\pgfqpoint{1.021280in}{0.825264in}}%
\pgfpathlineto{\pgfqpoint{1.016611in}{0.819421in}}%
\pgfpathlineto{\pgfqpoint{1.013378in}{0.806495in}}%
\pgfpathlineto{\pgfqpoint{1.013778in}{0.793568in}}%
\pgfpathlineto{\pgfqpoint{1.018412in}{0.780642in}}%
\pgfpathclose%
\pgfpathmoveto{\pgfqpoint{1.021178in}{0.793568in}}%
\pgfpathlineto{\pgfqpoint{1.020628in}{0.806495in}}%
\pgfpathlineto{\pgfqpoint{1.021280in}{0.808468in}}%
\pgfpathlineto{\pgfqpoint{1.034549in}{0.817780in}}%
\pgfpathlineto{\pgfqpoint{1.042526in}{0.806495in}}%
\pgfpathlineto{\pgfqpoint{1.041548in}{0.793568in}}%
\pgfpathlineto{\pgfqpoint{1.034549in}{0.785636in}}%
\pgfpathlineto{\pgfqpoint{1.021280in}{0.793322in}}%
\pgfpathclose%
\pgfusepath{fill}%
\end{pgfscope}%
\begin{pgfscope}%
\pgfpathrectangle{\pgfqpoint{0.211875in}{0.211875in}}{\pgfqpoint{1.313625in}{1.279725in}}%
\pgfusepath{clip}%
\pgfsetbuttcap%
\pgfsetroundjoin%
\definecolor{currentfill}{rgb}{0.965440,0.720101,0.576404}%
\pgfsetfillcolor{currentfill}%
\pgfsetlinewidth{0.000000pt}%
\definecolor{currentstroke}{rgb}{0.000000,0.000000,0.000000}%
\pgfsetstrokecolor{currentstroke}%
\pgfsetdash{}{0pt}%
\pgfpathmoveto{\pgfqpoint{1.140701in}{0.776336in}}%
\pgfpathlineto{\pgfqpoint{1.153970in}{0.774548in}}%
\pgfpathlineto{\pgfqpoint{1.164300in}{0.780642in}}%
\pgfpathlineto{\pgfqpoint{1.167239in}{0.785162in}}%
\pgfpathlineto{\pgfqpoint{1.169460in}{0.793568in}}%
\pgfpathlineto{\pgfqpoint{1.169795in}{0.806495in}}%
\pgfpathlineto{\pgfqpoint{1.167239in}{0.818976in}}%
\pgfpathlineto{\pgfqpoint{1.167026in}{0.819421in}}%
\pgfpathlineto{\pgfqpoint{1.153970in}{0.828433in}}%
\pgfpathlineto{\pgfqpoint{1.140701in}{0.826588in}}%
\pgfpathlineto{\pgfqpoint{1.134249in}{0.819421in}}%
\pgfpathlineto{\pgfqpoint{1.130882in}{0.806495in}}%
\pgfpathlineto{\pgfqpoint{1.131304in}{0.793568in}}%
\pgfpathlineto{\pgfqpoint{1.136148in}{0.780642in}}%
\pgfpathclose%
\pgfpathmoveto{\pgfqpoint{1.139226in}{0.793568in}}%
\pgfpathlineto{\pgfqpoint{1.138643in}{0.806495in}}%
\pgfpathlineto{\pgfqpoint{1.140701in}{0.812019in}}%
\pgfpathlineto{\pgfqpoint{1.153970in}{0.817736in}}%
\pgfpathlineto{\pgfqpoint{1.160720in}{0.806495in}}%
\pgfpathlineto{\pgfqpoint{1.159884in}{0.793568in}}%
\pgfpathlineto{\pgfqpoint{1.153970in}{0.785682in}}%
\pgfpathlineto{\pgfqpoint{1.140701in}{0.790417in}}%
\pgfpathclose%
\pgfusepath{fill}%
\end{pgfscope}%
\begin{pgfscope}%
\pgfpathrectangle{\pgfqpoint{0.211875in}{0.211875in}}{\pgfqpoint{1.313625in}{1.279725in}}%
\pgfusepath{clip}%
\pgfsetbuttcap%
\pgfsetroundjoin%
\definecolor{currentfill}{rgb}{0.965440,0.720101,0.576404}%
\pgfsetfillcolor{currentfill}%
\pgfsetlinewidth{0.000000pt}%
\definecolor{currentstroke}{rgb}{0.000000,0.000000,0.000000}%
\pgfsetstrokecolor{currentstroke}%
\pgfsetdash{}{0pt}%
\pgfpathmoveto{\pgfqpoint{1.260121in}{0.775953in}}%
\pgfpathlineto{\pgfqpoint{1.273390in}{0.775183in}}%
\pgfpathlineto{\pgfqpoint{1.281387in}{0.780642in}}%
\pgfpathlineto{\pgfqpoint{1.286659in}{0.790894in}}%
\pgfpathlineto{\pgfqpoint{1.287278in}{0.793568in}}%
\pgfpathlineto{\pgfqpoint{1.287610in}{0.806495in}}%
\pgfpathlineto{\pgfqpoint{1.286659in}{0.811852in}}%
\pgfpathlineto{\pgfqpoint{1.283829in}{0.819421in}}%
\pgfpathlineto{\pgfqpoint{1.273390in}{0.827760in}}%
\pgfpathlineto{\pgfqpoint{1.260121in}{0.826970in}}%
\pgfpathlineto{\pgfqpoint{1.252441in}{0.819421in}}%
\pgfpathlineto{\pgfqpoint{1.248734in}{0.806495in}}%
\pgfpathlineto{\pgfqpoint{1.249195in}{0.793568in}}%
\pgfpathlineto{\pgfqpoint{1.254523in}{0.780642in}}%
\pgfpathclose%
\pgfpathmoveto{\pgfqpoint{1.257778in}{0.793568in}}%
\pgfpathlineto{\pgfqpoint{1.257142in}{0.806495in}}%
\pgfpathlineto{\pgfqpoint{1.260121in}{0.813576in}}%
\pgfpathlineto{\pgfqpoint{1.273390in}{0.816098in}}%
\pgfpathlineto{\pgfqpoint{1.278379in}{0.806495in}}%
\pgfpathlineto{\pgfqpoint{1.277633in}{0.793568in}}%
\pgfpathlineto{\pgfqpoint{1.273390in}{0.787035in}}%
\pgfpathlineto{\pgfqpoint{1.260121in}{0.789127in}}%
\pgfpathclose%
\pgfusepath{fill}%
\end{pgfscope}%
\begin{pgfscope}%
\pgfpathrectangle{\pgfqpoint{0.211875in}{0.211875in}}{\pgfqpoint{1.313625in}{1.279725in}}%
\pgfusepath{clip}%
\pgfsetbuttcap%
\pgfsetroundjoin%
\definecolor{currentfill}{rgb}{0.965440,0.720101,0.576404}%
\pgfsetfillcolor{currentfill}%
\pgfsetlinewidth{0.000000pt}%
\definecolor{currentstroke}{rgb}{0.000000,0.000000,0.000000}%
\pgfsetstrokecolor{currentstroke}%
\pgfsetdash{}{0pt}%
\pgfpathmoveto{\pgfqpoint{1.379542in}{0.776339in}}%
\pgfpathlineto{\pgfqpoint{1.392811in}{0.776597in}}%
\pgfpathlineto{\pgfqpoint{1.397998in}{0.780642in}}%
\pgfpathlineto{\pgfqpoint{1.403999in}{0.793568in}}%
\pgfpathlineto{\pgfqpoint{1.404508in}{0.806495in}}%
\pgfpathlineto{\pgfqpoint{1.400287in}{0.819421in}}%
\pgfpathlineto{\pgfqpoint{1.392811in}{0.826247in}}%
\pgfpathlineto{\pgfqpoint{1.379542in}{0.826521in}}%
\pgfpathlineto{\pgfqpoint{1.371324in}{0.819421in}}%
\pgfpathlineto{\pgfqpoint{1.367008in}{0.806495in}}%
\pgfpathlineto{\pgfqpoint{1.367531in}{0.793568in}}%
\pgfpathlineto{\pgfqpoint{1.373701in}{0.780642in}}%
\pgfpathclose%
\pgfpathmoveto{\pgfqpoint{1.376963in}{0.793568in}}%
\pgfpathlineto{\pgfqpoint{1.376246in}{0.806495in}}%
\pgfpathlineto{\pgfqpoint{1.379542in}{0.813389in}}%
\pgfpathlineto{\pgfqpoint{1.392811in}{0.812738in}}%
\pgfpathlineto{\pgfqpoint{1.395651in}{0.806495in}}%
\pgfpathlineto{\pgfqpoint{1.394960in}{0.793568in}}%
\pgfpathlineto{\pgfqpoint{1.392811in}{0.789794in}}%
\pgfpathlineto{\pgfqpoint{1.379542in}{0.789259in}}%
\pgfpathclose%
\pgfusepath{fill}%
\end{pgfscope}%
\begin{pgfscope}%
\pgfpathrectangle{\pgfqpoint{0.211875in}{0.211875in}}{\pgfqpoint{1.313625in}{1.279725in}}%
\pgfusepath{clip}%
\pgfsetbuttcap%
\pgfsetroundjoin%
\definecolor{currentfill}{rgb}{0.965440,0.720101,0.576404}%
\pgfsetfillcolor{currentfill}%
\pgfsetlinewidth{0.000000pt}%
\definecolor{currentstroke}{rgb}{0.000000,0.000000,0.000000}%
\pgfsetstrokecolor{currentstroke}%
\pgfsetdash{}{0pt}%
\pgfpathmoveto{\pgfqpoint{1.498962in}{0.777435in}}%
\pgfpathlineto{\pgfqpoint{1.512231in}{0.778879in}}%
\pgfpathlineto{\pgfqpoint{1.514227in}{0.780642in}}%
\pgfpathlineto{\pgfqpoint{1.520219in}{0.793568in}}%
\pgfpathlineto{\pgfqpoint{1.520714in}{0.806495in}}%
\pgfpathlineto{\pgfqpoint{1.516455in}{0.819421in}}%
\pgfpathlineto{\pgfqpoint{1.512231in}{0.823791in}}%
\pgfpathlineto{\pgfqpoint{1.498962in}{0.825310in}}%
\pgfpathlineto{\pgfqpoint{1.491130in}{0.819421in}}%
\pgfpathlineto{\pgfqpoint{1.485830in}{0.806495in}}%
\pgfpathlineto{\pgfqpoint{1.486448in}{0.793568in}}%
\pgfpathlineto{\pgfqpoint{1.493960in}{0.780642in}}%
\pgfpathclose%
\pgfpathmoveto{\pgfqpoint{1.496991in}{0.793568in}}%
\pgfpathlineto{\pgfqpoint{1.496155in}{0.806495in}}%
\pgfpathlineto{\pgfqpoint{1.498962in}{0.811613in}}%
\pgfpathlineto{\pgfqpoint{1.512231in}{0.807442in}}%
\pgfpathlineto{\pgfqpoint{1.512612in}{0.806495in}}%
\pgfpathlineto{\pgfqpoint{1.512231in}{0.798976in}}%
\pgfpathlineto{\pgfqpoint{1.510200in}{0.793568in}}%
\pgfpathlineto{\pgfqpoint{1.498962in}{0.790693in}}%
\pgfpathclose%
\pgfusepath{fill}%
\end{pgfscope}%
\begin{pgfscope}%
\pgfpathrectangle{\pgfqpoint{0.211875in}{0.211875in}}{\pgfqpoint{1.313625in}{1.279725in}}%
\pgfusepath{clip}%
\pgfsetbuttcap%
\pgfsetroundjoin%
\definecolor{currentfill}{rgb}{0.965440,0.720101,0.576404}%
\pgfsetfillcolor{currentfill}%
\pgfsetlinewidth{0.000000pt}%
\definecolor{currentstroke}{rgb}{0.000000,0.000000,0.000000}%
\pgfsetstrokecolor{currentstroke}%
\pgfsetdash{}{0pt}%
\pgfpathmoveto{\pgfqpoint{0.437447in}{0.789580in}}%
\pgfpathlineto{\pgfqpoint{0.450716in}{0.791264in}}%
\pgfpathlineto{\pgfqpoint{0.451958in}{0.793568in}}%
\pgfpathlineto{\pgfqpoint{0.452631in}{0.806495in}}%
\pgfpathlineto{\pgfqpoint{0.450716in}{0.810951in}}%
\pgfpathlineto{\pgfqpoint{0.437447in}{0.812980in}}%
\pgfpathlineto{\pgfqpoint{0.434158in}{0.806495in}}%
\pgfpathlineto{\pgfqpoint{0.434920in}{0.793568in}}%
\pgfpathclose%
\pgfusepath{fill}%
\end{pgfscope}%
\begin{pgfscope}%
\pgfpathrectangle{\pgfqpoint{0.211875in}{0.211875in}}{\pgfqpoint{1.313625in}{1.279725in}}%
\pgfusepath{clip}%
\pgfsetbuttcap%
\pgfsetroundjoin%
\definecolor{currentfill}{rgb}{0.965440,0.720101,0.576404}%
\pgfsetfillcolor{currentfill}%
\pgfsetlinewidth{0.000000pt}%
\definecolor{currentstroke}{rgb}{0.000000,0.000000,0.000000}%
\pgfsetstrokecolor{currentstroke}%
\pgfsetdash{}{0pt}%
\pgfpathmoveto{\pgfqpoint{0.556867in}{0.783136in}}%
\pgfpathlineto{\pgfqpoint{0.570136in}{0.787005in}}%
\pgfpathlineto{\pgfqpoint{0.573271in}{0.793568in}}%
\pgfpathlineto{\pgfqpoint{0.573830in}{0.806495in}}%
\pgfpathlineto{\pgfqpoint{0.570136in}{0.816222in}}%
\pgfpathlineto{\pgfqpoint{0.561307in}{0.819421in}}%
\pgfpathlineto{\pgfqpoint{0.556867in}{0.820185in}}%
\pgfpathlineto{\pgfqpoint{0.555800in}{0.819421in}}%
\pgfpathlineto{\pgfqpoint{0.548480in}{0.806495in}}%
\pgfpathlineto{\pgfqpoint{0.549260in}{0.793568in}}%
\pgfpathclose%
\pgfusepath{fill}%
\end{pgfscope}%
\begin{pgfscope}%
\pgfpathrectangle{\pgfqpoint{0.211875in}{0.211875in}}{\pgfqpoint{1.313625in}{1.279725in}}%
\pgfusepath{clip}%
\pgfsetbuttcap%
\pgfsetroundjoin%
\definecolor{currentfill}{rgb}{0.965440,0.720101,0.576404}%
\pgfsetfillcolor{currentfill}%
\pgfsetlinewidth{0.000000pt}%
\definecolor{currentstroke}{rgb}{0.000000,0.000000,0.000000}%
\pgfsetstrokecolor{currentstroke}%
\pgfsetdash{}{0pt}%
\pgfpathmoveto{\pgfqpoint{0.742633in}{0.858126in}}%
\pgfpathlineto{\pgfqpoint{0.742778in}{0.858201in}}%
\pgfpathlineto{\pgfqpoint{0.753926in}{0.871127in}}%
\pgfpathlineto{\pgfqpoint{0.755902in}{0.883880in}}%
\pgfpathlineto{\pgfqpoint{0.755916in}{0.884054in}}%
\pgfpathlineto{\pgfqpoint{0.755902in}{0.884170in}}%
\pgfpathlineto{\pgfqpoint{0.752703in}{0.896980in}}%
\pgfpathlineto{\pgfqpoint{0.742633in}{0.906557in}}%
\pgfpathlineto{\pgfqpoint{0.729364in}{0.904980in}}%
\pgfpathlineto{\pgfqpoint{0.723059in}{0.896980in}}%
\pgfpathlineto{\pgfqpoint{0.720393in}{0.884054in}}%
\pgfpathlineto{\pgfqpoint{0.722044in}{0.871127in}}%
\pgfpathlineto{\pgfqpoint{0.729364in}{0.859908in}}%
\pgfpathlineto{\pgfqpoint{0.741790in}{0.858201in}}%
\pgfpathclose%
\pgfpathmoveto{\pgfqpoint{0.740835in}{0.871127in}}%
\pgfpathlineto{\pgfqpoint{0.729364in}{0.878633in}}%
\pgfpathlineto{\pgfqpoint{0.728401in}{0.884054in}}%
\pgfpathlineto{\pgfqpoint{0.729364in}{0.887486in}}%
\pgfpathlineto{\pgfqpoint{0.742633in}{0.892896in}}%
\pgfpathlineto{\pgfqpoint{0.745926in}{0.884054in}}%
\pgfpathlineto{\pgfqpoint{0.742920in}{0.871127in}}%
\pgfpathlineto{\pgfqpoint{0.742633in}{0.870796in}}%
\pgfpathclose%
\pgfusepath{fill}%
\end{pgfscope}%
\begin{pgfscope}%
\pgfpathrectangle{\pgfqpoint{0.211875in}{0.211875in}}{\pgfqpoint{1.313625in}{1.279725in}}%
\pgfusepath{clip}%
\pgfsetbuttcap%
\pgfsetroundjoin%
\definecolor{currentfill}{rgb}{0.965440,0.720101,0.576404}%
\pgfsetfillcolor{currentfill}%
\pgfsetlinewidth{0.000000pt}%
\definecolor{currentstroke}{rgb}{0.000000,0.000000,0.000000}%
\pgfsetstrokecolor{currentstroke}%
\pgfsetdash{}{0pt}%
\pgfpathmoveto{\pgfqpoint{0.848784in}{0.856966in}}%
\pgfpathlineto{\pgfqpoint{0.862053in}{0.856811in}}%
\pgfpathlineto{\pgfqpoint{0.864409in}{0.858201in}}%
\pgfpathlineto{\pgfqpoint{0.873369in}{0.871127in}}%
\pgfpathlineto{\pgfqpoint{0.874994in}{0.884054in}}%
\pgfpathlineto{\pgfqpoint{0.872419in}{0.896980in}}%
\pgfpathlineto{\pgfqpoint{0.862053in}{0.908284in}}%
\pgfpathlineto{\pgfqpoint{0.848784in}{0.908082in}}%
\pgfpathlineto{\pgfqpoint{0.838869in}{0.896980in}}%
\pgfpathlineto{\pgfqpoint{0.836277in}{0.884054in}}%
\pgfpathlineto{\pgfqpoint{0.837908in}{0.871127in}}%
\pgfpathlineto{\pgfqpoint{0.846770in}{0.858201in}}%
\pgfpathclose%
\pgfpathmoveto{\pgfqpoint{0.847526in}{0.871127in}}%
\pgfpathlineto{\pgfqpoint{0.845043in}{0.884054in}}%
\pgfpathlineto{\pgfqpoint{0.848784in}{0.895818in}}%
\pgfpathlineto{\pgfqpoint{0.862053in}{0.896280in}}%
\pgfpathlineto{\pgfqpoint{0.866027in}{0.884054in}}%
\pgfpathlineto{\pgfqpoint{0.863507in}{0.871127in}}%
\pgfpathlineto{\pgfqpoint{0.862053in}{0.869205in}}%
\pgfpathlineto{\pgfqpoint{0.848784in}{0.869423in}}%
\pgfpathclose%
\pgfusepath{fill}%
\end{pgfscope}%
\begin{pgfscope}%
\pgfpathrectangle{\pgfqpoint{0.211875in}{0.211875in}}{\pgfqpoint{1.313625in}{1.279725in}}%
\pgfusepath{clip}%
\pgfsetbuttcap%
\pgfsetroundjoin%
\definecolor{currentfill}{rgb}{0.965440,0.720101,0.576404}%
\pgfsetfillcolor{currentfill}%
\pgfsetlinewidth{0.000000pt}%
\definecolor{currentstroke}{rgb}{0.000000,0.000000,0.000000}%
\pgfsetstrokecolor{currentstroke}%
\pgfsetdash{}{0pt}%
\pgfpathmoveto{\pgfqpoint{0.968205in}{0.855383in}}%
\pgfpathlineto{\pgfqpoint{0.981473in}{0.856141in}}%
\pgfpathlineto{\pgfqpoint{0.984542in}{0.858201in}}%
\pgfpathlineto{\pgfqpoint{0.992104in}{0.871127in}}%
\pgfpathlineto{\pgfqpoint{0.993487in}{0.884054in}}%
\pgfpathlineto{\pgfqpoint{0.991324in}{0.896980in}}%
\pgfpathlineto{\pgfqpoint{0.981473in}{0.909181in}}%
\pgfpathlineto{\pgfqpoint{0.971359in}{0.909907in}}%
\pgfpathlineto{\pgfqpoint{0.968205in}{0.910067in}}%
\pgfpathlineto{\pgfqpoint{0.967827in}{0.909907in}}%
\pgfpathlineto{\pgfqpoint{0.954936in}{0.897227in}}%
\pgfpathlineto{\pgfqpoint{0.954848in}{0.896980in}}%
\pgfpathlineto{\pgfqpoint{0.953255in}{0.884054in}}%
\pgfpathlineto{\pgfqpoint{0.954271in}{0.871127in}}%
\pgfpathlineto{\pgfqpoint{0.954936in}{0.868902in}}%
\pgfpathlineto{\pgfqpoint{0.962945in}{0.858201in}}%
\pgfpathclose%
\pgfpathmoveto{\pgfqpoint{0.964541in}{0.871127in}}%
\pgfpathlineto{\pgfqpoint{0.961909in}{0.884054in}}%
\pgfpathlineto{\pgfqpoint{0.966316in}{0.896980in}}%
\pgfpathlineto{\pgfqpoint{0.968205in}{0.898831in}}%
\pgfpathlineto{\pgfqpoint{0.981473in}{0.897182in}}%
\pgfpathlineto{\pgfqpoint{0.981636in}{0.896980in}}%
\pgfpathlineto{\pgfqpoint{0.985312in}{0.884054in}}%
\pgfpathlineto{\pgfqpoint{0.983117in}{0.871127in}}%
\pgfpathlineto{\pgfqpoint{0.981473in}{0.868663in}}%
\pgfpathlineto{\pgfqpoint{0.968205in}{0.866782in}}%
\pgfpathclose%
\pgfusepath{fill}%
\end{pgfscope}%
\begin{pgfscope}%
\pgfpathrectangle{\pgfqpoint{0.211875in}{0.211875in}}{\pgfqpoint{1.313625in}{1.279725in}}%
\pgfusepath{clip}%
\pgfsetbuttcap%
\pgfsetroundjoin%
\definecolor{currentfill}{rgb}{0.965440,0.720101,0.576404}%
\pgfsetfillcolor{currentfill}%
\pgfsetlinewidth{0.000000pt}%
\definecolor{currentstroke}{rgb}{0.000000,0.000000,0.000000}%
\pgfsetstrokecolor{currentstroke}%
\pgfsetdash{}{0pt}%
\pgfpathmoveto{\pgfqpoint{1.087625in}{0.854533in}}%
\pgfpathlineto{\pgfqpoint{1.100894in}{0.856165in}}%
\pgfpathlineto{\pgfqpoint{1.103575in}{0.858201in}}%
\pgfpathlineto{\pgfqpoint{1.110303in}{0.871127in}}%
\pgfpathlineto{\pgfqpoint{1.111540in}{0.884054in}}%
\pgfpathlineto{\pgfqpoint{1.109618in}{0.896980in}}%
\pgfpathlineto{\pgfqpoint{1.100894in}{0.909181in}}%
\pgfpathlineto{\pgfqpoint{1.096514in}{0.909907in}}%
\pgfpathlineto{\pgfqpoint{1.087625in}{0.910885in}}%
\pgfpathlineto{\pgfqpoint{1.084938in}{0.909907in}}%
\pgfpathlineto{\pgfqpoint{1.074356in}{0.901756in}}%
\pgfpathlineto{\pgfqpoint{1.072412in}{0.896980in}}%
\pgfpathlineto{\pgfqpoint{1.070861in}{0.884054in}}%
\pgfpathlineto{\pgfqpoint{1.071858in}{0.871127in}}%
\pgfpathlineto{\pgfqpoint{1.074356in}{0.863811in}}%
\pgfpathlineto{\pgfqpoint{1.079680in}{0.858201in}}%
\pgfpathclose%
\pgfpathmoveto{\pgfqpoint{1.081944in}{0.871127in}}%
\pgfpathlineto{\pgfqpoint{1.079033in}{0.884054in}}%
\pgfpathlineto{\pgfqpoint{1.083894in}{0.896980in}}%
\pgfpathlineto{\pgfqpoint{1.087625in}{0.900136in}}%
\pgfpathlineto{\pgfqpoint{1.099826in}{0.896980in}}%
\pgfpathlineto{\pgfqpoint{1.100894in}{0.896237in}}%
\pgfpathlineto{\pgfqpoint{1.103983in}{0.884054in}}%
\pgfpathlineto{\pgfqpoint{1.101999in}{0.871127in}}%
\pgfpathlineto{\pgfqpoint{1.100894in}{0.869259in}}%
\pgfpathlineto{\pgfqpoint{1.087625in}{0.865303in}}%
\pgfpathclose%
\pgfusepath{fill}%
\end{pgfscope}%
\begin{pgfscope}%
\pgfpathrectangle{\pgfqpoint{0.211875in}{0.211875in}}{\pgfqpoint{1.313625in}{1.279725in}}%
\pgfusepath{clip}%
\pgfsetbuttcap%
\pgfsetroundjoin%
\definecolor{currentfill}{rgb}{0.965440,0.720101,0.576404}%
\pgfsetfillcolor{currentfill}%
\pgfsetlinewidth{0.000000pt}%
\definecolor{currentstroke}{rgb}{0.000000,0.000000,0.000000}%
\pgfsetstrokecolor{currentstroke}%
\pgfsetdash{}{0pt}%
\pgfpathmoveto{\pgfqpoint{1.207045in}{0.854360in}}%
\pgfpathlineto{\pgfqpoint{1.220314in}{0.856973in}}%
\pgfpathlineto{\pgfqpoint{1.221750in}{0.858201in}}%
\pgfpathlineto{\pgfqpoint{1.228069in}{0.871127in}}%
\pgfpathlineto{\pgfqpoint{1.229234in}{0.884054in}}%
\pgfpathlineto{\pgfqpoint{1.227423in}{0.896980in}}%
\pgfpathlineto{\pgfqpoint{1.220314in}{0.908169in}}%
\pgfpathlineto{\pgfqpoint{1.213846in}{0.909907in}}%
\pgfpathlineto{\pgfqpoint{1.207045in}{0.911047in}}%
\pgfpathlineto{\pgfqpoint{1.203328in}{0.909907in}}%
\pgfpathlineto{\pgfqpoint{1.193777in}{0.904204in}}%
\pgfpathlineto{\pgfqpoint{1.190429in}{0.896980in}}%
\pgfpathlineto{\pgfqpoint{1.188837in}{0.884054in}}%
\pgfpathlineto{\pgfqpoint{1.189862in}{0.871127in}}%
\pgfpathlineto{\pgfqpoint{1.193777in}{0.861010in}}%
\pgfpathlineto{\pgfqpoint{1.197193in}{0.858201in}}%
\pgfpathclose%
\pgfpathmoveto{\pgfqpoint{1.199881in}{0.871127in}}%
\pgfpathlineto{\pgfqpoint{1.196497in}{0.884054in}}%
\pgfpathlineto{\pgfqpoint{1.202149in}{0.896980in}}%
\pgfpathlineto{\pgfqpoint{1.207045in}{0.900487in}}%
\pgfpathlineto{\pgfqpoint{1.216085in}{0.896980in}}%
\pgfpathlineto{\pgfqpoint{1.220314in}{0.892271in}}%
\pgfpathlineto{\pgfqpoint{1.222165in}{0.884054in}}%
\pgfpathlineto{\pgfqpoint{1.220314in}{0.871185in}}%
\pgfpathlineto{\pgfqpoint{1.220284in}{0.871127in}}%
\pgfpathlineto{\pgfqpoint{1.207045in}{0.864901in}}%
\pgfpathclose%
\pgfusepath{fill}%
\end{pgfscope}%
\begin{pgfscope}%
\pgfpathrectangle{\pgfqpoint{0.211875in}{0.211875in}}{\pgfqpoint{1.313625in}{1.279725in}}%
\pgfusepath{clip}%
\pgfsetbuttcap%
\pgfsetroundjoin%
\definecolor{currentfill}{rgb}{0.965440,0.720101,0.576404}%
\pgfsetfillcolor{currentfill}%
\pgfsetlinewidth{0.000000pt}%
\definecolor{currentstroke}{rgb}{0.000000,0.000000,0.000000}%
\pgfsetstrokecolor{currentstroke}%
\pgfsetdash{}{0pt}%
\pgfpathmoveto{\pgfqpoint{1.326466in}{0.854839in}}%
\pgfpathlineto{\pgfqpoint{1.338351in}{0.858201in}}%
\pgfpathlineto{\pgfqpoint{1.339735in}{0.858930in}}%
\pgfpathlineto{\pgfqpoint{1.345462in}{0.871127in}}%
\pgfpathlineto{\pgfqpoint{1.346612in}{0.884054in}}%
\pgfpathlineto{\pgfqpoint{1.344808in}{0.896980in}}%
\pgfpathlineto{\pgfqpoint{1.339735in}{0.905962in}}%
\pgfpathlineto{\pgfqpoint{1.329456in}{0.909907in}}%
\pgfpathlineto{\pgfqpoint{1.326466in}{0.910577in}}%
\pgfpathlineto{\pgfqpoint{1.323787in}{0.909907in}}%
\pgfpathlineto{\pgfqpoint{1.313197in}{0.905123in}}%
\pgfpathlineto{\pgfqpoint{1.308930in}{0.896980in}}%
\pgfpathlineto{\pgfqpoint{1.307204in}{0.884054in}}%
\pgfpathlineto{\pgfqpoint{1.308308in}{0.871127in}}%
\pgfpathlineto{\pgfqpoint{1.313197in}{0.859915in}}%
\pgfpathlineto{\pgfqpoint{1.315932in}{0.858201in}}%
\pgfpathclose%
\pgfpathmoveto{\pgfqpoint{1.318644in}{0.871127in}}%
\pgfpathlineto{\pgfqpoint{1.314466in}{0.884054in}}%
\pgfpathlineto{\pgfqpoint{1.321467in}{0.896980in}}%
\pgfpathlineto{\pgfqpoint{1.326466in}{0.899925in}}%
\pgfpathlineto{\pgfqpoint{1.332184in}{0.896980in}}%
\pgfpathlineto{\pgfqpoint{1.339735in}{0.885038in}}%
\pgfpathlineto{\pgfqpoint{1.339932in}{0.884054in}}%
\pgfpathlineto{\pgfqpoint{1.339735in}{0.882521in}}%
\pgfpathlineto{\pgfqpoint{1.335434in}{0.871127in}}%
\pgfpathlineto{\pgfqpoint{1.326466in}{0.865531in}}%
\pgfpathclose%
\pgfusepath{fill}%
\end{pgfscope}%
\begin{pgfscope}%
\pgfpathrectangle{\pgfqpoint{0.211875in}{0.211875in}}{\pgfqpoint{1.313625in}{1.279725in}}%
\pgfusepath{clip}%
\pgfsetbuttcap%
\pgfsetroundjoin%
\definecolor{currentfill}{rgb}{0.965440,0.720101,0.576404}%
\pgfsetfillcolor{currentfill}%
\pgfsetlinewidth{0.000000pt}%
\definecolor{currentstroke}{rgb}{0.000000,0.000000,0.000000}%
\pgfsetstrokecolor{currentstroke}%
\pgfsetdash{}{0pt}%
\pgfpathmoveto{\pgfqpoint{1.445886in}{0.855967in}}%
\pgfpathlineto{\pgfqpoint{1.452210in}{0.858201in}}%
\pgfpathlineto{\pgfqpoint{1.459155in}{0.863096in}}%
\pgfpathlineto{\pgfqpoint{1.462512in}{0.871127in}}%
\pgfpathlineto{\pgfqpoint{1.463698in}{0.884054in}}%
\pgfpathlineto{\pgfqpoint{1.461813in}{0.896980in}}%
\pgfpathlineto{\pgfqpoint{1.459155in}{0.902282in}}%
\pgfpathlineto{\pgfqpoint{1.445886in}{0.909335in}}%
\pgfpathlineto{\pgfqpoint{1.432617in}{0.904849in}}%
\pgfpathlineto{\pgfqpoint{1.427971in}{0.896980in}}%
\pgfpathlineto{\pgfqpoint{1.426001in}{0.884054in}}%
\pgfpathlineto{\pgfqpoint{1.427245in}{0.871127in}}%
\pgfpathlineto{\pgfqpoint{1.432617in}{0.860162in}}%
\pgfpathlineto{\pgfqpoint{1.436909in}{0.858201in}}%
\pgfpathclose%
\pgfpathmoveto{\pgfqpoint{1.438886in}{0.871127in}}%
\pgfpathlineto{\pgfqpoint{1.433322in}{0.884054in}}%
\pgfpathlineto{\pgfqpoint{1.442703in}{0.896980in}}%
\pgfpathlineto{\pgfqpoint{1.445886in}{0.898453in}}%
\pgfpathlineto{\pgfqpoint{1.448182in}{0.896980in}}%
\pgfpathlineto{\pgfqpoint{1.455026in}{0.884054in}}%
\pgfpathlineto{\pgfqpoint{1.450955in}{0.871127in}}%
\pgfpathlineto{\pgfqpoint{1.445886in}{0.867189in}}%
\pgfpathclose%
\pgfusepath{fill}%
\end{pgfscope}%
\begin{pgfscope}%
\pgfpathrectangle{\pgfqpoint{0.211875in}{0.211875in}}{\pgfqpoint{1.313625in}{1.279725in}}%
\pgfusepath{clip}%
\pgfsetbuttcap%
\pgfsetroundjoin%
\definecolor{currentfill}{rgb}{0.965440,0.720101,0.576404}%
\pgfsetfillcolor{currentfill}%
\pgfsetlinewidth{0.000000pt}%
\definecolor{currentstroke}{rgb}{0.000000,0.000000,0.000000}%
\pgfsetstrokecolor{currentstroke}%
\pgfsetdash{}{0pt}%
\pgfpathmoveto{\pgfqpoint{0.384371in}{0.869380in}}%
\pgfpathlineto{\pgfqpoint{0.386923in}{0.871127in}}%
\pgfpathlineto{\pgfqpoint{0.391767in}{0.884054in}}%
\pgfpathlineto{\pgfqpoint{0.384371in}{0.895856in}}%
\pgfpathlineto{\pgfqpoint{0.376548in}{0.884054in}}%
\pgfpathlineto{\pgfqpoint{0.381704in}{0.871127in}}%
\pgfpathclose%
\pgfusepath{fill}%
\end{pgfscope}%
\begin{pgfscope}%
\pgfpathrectangle{\pgfqpoint{0.211875in}{0.211875in}}{\pgfqpoint{1.313625in}{1.279725in}}%
\pgfusepath{clip}%
\pgfsetbuttcap%
\pgfsetroundjoin%
\definecolor{currentfill}{rgb}{0.965440,0.720101,0.576404}%
\pgfsetfillcolor{currentfill}%
\pgfsetlinewidth{0.000000pt}%
\definecolor{currentstroke}{rgb}{0.000000,0.000000,0.000000}%
\pgfsetstrokecolor{currentstroke}%
\pgfsetdash{}{0pt}%
\pgfpathmoveto{\pgfqpoint{0.503792in}{0.864674in}}%
\pgfpathlineto{\pgfqpoint{0.511493in}{0.871127in}}%
\pgfpathlineto{\pgfqpoint{0.514938in}{0.884054in}}%
\pgfpathlineto{\pgfqpoint{0.509179in}{0.896980in}}%
\pgfpathlineto{\pgfqpoint{0.503792in}{0.900690in}}%
\pgfpathlineto{\pgfqpoint{0.494871in}{0.896980in}}%
\pgfpathlineto{\pgfqpoint{0.490523in}{0.891799in}}%
\pgfpathlineto{\pgfqpoint{0.488808in}{0.884054in}}%
\pgfpathlineto{\pgfqpoint{0.490523in}{0.871923in}}%
\pgfpathlineto{\pgfqpoint{0.490921in}{0.871127in}}%
\pgfpathclose%
\pgfusepath{fill}%
\end{pgfscope}%
\begin{pgfscope}%
\pgfpathrectangle{\pgfqpoint{0.211875in}{0.211875in}}{\pgfqpoint{1.313625in}{1.279725in}}%
\pgfusepath{clip}%
\pgfsetbuttcap%
\pgfsetroundjoin%
\definecolor{currentfill}{rgb}{0.965440,0.720101,0.576404}%
\pgfsetfillcolor{currentfill}%
\pgfsetlinewidth{0.000000pt}%
\definecolor{currentstroke}{rgb}{0.000000,0.000000,0.000000}%
\pgfsetstrokecolor{currentstroke}%
\pgfsetdash{}{0pt}%
\pgfpathmoveto{\pgfqpoint{0.609943in}{0.864807in}}%
\pgfpathlineto{\pgfqpoint{0.623212in}{0.860924in}}%
\pgfpathlineto{\pgfqpoint{0.633479in}{0.871127in}}%
\pgfpathlineto{\pgfqpoint{0.636048in}{0.884054in}}%
\pgfpathlineto{\pgfqpoint{0.631831in}{0.896980in}}%
\pgfpathlineto{\pgfqpoint{0.623212in}{0.904027in}}%
\pgfpathlineto{\pgfqpoint{0.609943in}{0.900639in}}%
\pgfpathlineto{\pgfqpoint{0.607385in}{0.896980in}}%
\pgfpathlineto{\pgfqpoint{0.604556in}{0.884054in}}%
\pgfpathlineto{\pgfqpoint{0.606278in}{0.871127in}}%
\pgfpathclose%
\pgfpathmoveto{\pgfqpoint{0.618998in}{0.884054in}}%
\pgfpathlineto{\pgfqpoint{0.623212in}{0.887402in}}%
\pgfpathlineto{\pgfqpoint{0.624660in}{0.884054in}}%
\pgfpathlineto{\pgfqpoint{0.623212in}{0.878730in}}%
\pgfpathclose%
\pgfusepath{fill}%
\end{pgfscope}%
\begin{pgfscope}%
\pgfpathrectangle{\pgfqpoint{0.211875in}{0.211875in}}{\pgfqpoint{1.313625in}{1.279725in}}%
\pgfusepath{clip}%
\pgfsetbuttcap%
\pgfsetroundjoin%
\definecolor{currentfill}{rgb}{0.965440,0.720101,0.576404}%
\pgfsetfillcolor{currentfill}%
\pgfsetlinewidth{0.000000pt}%
\definecolor{currentstroke}{rgb}{0.000000,0.000000,0.000000}%
\pgfsetstrokecolor{currentstroke}%
\pgfsetdash{}{0pt}%
\pgfpathmoveto{\pgfqpoint{0.915129in}{0.935484in}}%
\pgfpathlineto{\pgfqpoint{0.916242in}{0.935760in}}%
\pgfpathlineto{\pgfqpoint{0.928398in}{0.941149in}}%
\pgfpathlineto{\pgfqpoint{0.932450in}{0.948686in}}%
\pgfpathlineto{\pgfqpoint{0.934379in}{0.961613in}}%
\pgfpathlineto{\pgfqpoint{0.933513in}{0.974539in}}%
\pgfpathlineto{\pgfqpoint{0.928398in}{0.986951in}}%
\pgfpathlineto{\pgfqpoint{0.927636in}{0.987466in}}%
\pgfpathlineto{\pgfqpoint{0.915129in}{0.991644in}}%
\pgfpathlineto{\pgfqpoint{0.901860in}{0.987984in}}%
\pgfpathlineto{\pgfqpoint{0.901332in}{0.987466in}}%
\pgfpathlineto{\pgfqpoint{0.895615in}{0.974539in}}%
\pgfpathlineto{\pgfqpoint{0.894695in}{0.961613in}}%
\pgfpathlineto{\pgfqpoint{0.896735in}{0.948686in}}%
\pgfpathlineto{\pgfqpoint{0.901860in}{0.940044in}}%
\pgfpathlineto{\pgfqpoint{0.913841in}{0.935760in}}%
\pgfpathclose%
\pgfpathmoveto{\pgfqpoint{0.909591in}{0.948686in}}%
\pgfpathlineto{\pgfqpoint{0.901860in}{0.959555in}}%
\pgfpathlineto{\pgfqpoint{0.901407in}{0.961613in}}%
\pgfpathlineto{\pgfqpoint{0.901860in}{0.965621in}}%
\pgfpathlineto{\pgfqpoint{0.904869in}{0.974539in}}%
\pgfpathlineto{\pgfqpoint{0.915129in}{0.981004in}}%
\pgfpathlineto{\pgfqpoint{0.923674in}{0.974539in}}%
\pgfpathlineto{\pgfqpoint{0.927169in}{0.961613in}}%
\pgfpathlineto{\pgfqpoint{0.919758in}{0.948686in}}%
\pgfpathlineto{\pgfqpoint{0.915129in}{0.945986in}}%
\pgfpathclose%
\pgfusepath{fill}%
\end{pgfscope}%
\begin{pgfscope}%
\pgfpathrectangle{\pgfqpoint{0.211875in}{0.211875in}}{\pgfqpoint{1.313625in}{1.279725in}}%
\pgfusepath{clip}%
\pgfsetbuttcap%
\pgfsetroundjoin%
\definecolor{currentfill}{rgb}{0.965440,0.720101,0.576404}%
\pgfsetfillcolor{currentfill}%
\pgfsetlinewidth{0.000000pt}%
\definecolor{currentstroke}{rgb}{0.000000,0.000000,0.000000}%
\pgfsetstrokecolor{currentstroke}%
\pgfsetdash{}{0pt}%
\pgfpathmoveto{\pgfqpoint{1.034549in}{0.934766in}}%
\pgfpathlineto{\pgfqpoint{1.037835in}{0.935760in}}%
\pgfpathlineto{\pgfqpoint{1.047818in}{0.941583in}}%
\pgfpathlineto{\pgfqpoint{1.051201in}{0.948686in}}%
\pgfpathlineto{\pgfqpoint{1.052949in}{0.961613in}}%
\pgfpathlineto{\pgfqpoint{1.052171in}{0.974539in}}%
\pgfpathlineto{\pgfqpoint{1.047818in}{0.986481in}}%
\pgfpathlineto{\pgfqpoint{1.046709in}{0.987466in}}%
\pgfpathlineto{\pgfqpoint{1.034549in}{0.992412in}}%
\pgfpathlineto{\pgfqpoint{1.021280in}{0.990015in}}%
\pgfpathlineto{\pgfqpoint{1.018355in}{0.987466in}}%
\pgfpathlineto{\pgfqpoint{1.012720in}{0.974539in}}%
\pgfpathlineto{\pgfqpoint{1.011819in}{0.961613in}}%
\pgfpathlineto{\pgfqpoint{1.013836in}{0.948686in}}%
\pgfpathlineto{\pgfqpoint{1.021280in}{0.937505in}}%
\pgfpathlineto{\pgfqpoint{1.028290in}{0.935760in}}%
\pgfpathclose%
\pgfpathmoveto{\pgfqpoint{1.024720in}{0.948686in}}%
\pgfpathlineto{\pgfqpoint{1.021280in}{0.952077in}}%
\pgfpathlineto{\pgfqpoint{1.018930in}{0.961613in}}%
\pgfpathlineto{\pgfqpoint{1.020505in}{0.974539in}}%
\pgfpathlineto{\pgfqpoint{1.021280in}{0.976052in}}%
\pgfpathlineto{\pgfqpoint{1.034549in}{0.982056in}}%
\pgfpathlineto{\pgfqpoint{1.042757in}{0.974539in}}%
\pgfpathlineto{\pgfqpoint{1.045552in}{0.961613in}}%
\pgfpathlineto{\pgfqpoint{1.039613in}{0.948686in}}%
\pgfpathlineto{\pgfqpoint{1.034549in}{0.945107in}}%
\pgfpathclose%
\pgfusepath{fill}%
\end{pgfscope}%
\begin{pgfscope}%
\pgfpathrectangle{\pgfqpoint{0.211875in}{0.211875in}}{\pgfqpoint{1.313625in}{1.279725in}}%
\pgfusepath{clip}%
\pgfsetbuttcap%
\pgfsetroundjoin%
\definecolor{currentfill}{rgb}{0.965440,0.720101,0.576404}%
\pgfsetfillcolor{currentfill}%
\pgfsetlinewidth{0.000000pt}%
\definecolor{currentstroke}{rgb}{0.000000,0.000000,0.000000}%
\pgfsetstrokecolor{currentstroke}%
\pgfsetdash{}{0pt}%
\pgfpathmoveto{\pgfqpoint{1.153970in}{0.934663in}}%
\pgfpathlineto{\pgfqpoint{1.157029in}{0.935760in}}%
\pgfpathlineto{\pgfqpoint{1.167239in}{0.943415in}}%
\pgfpathlineto{\pgfqpoint{1.169449in}{0.948686in}}%
\pgfpathlineto{\pgfqpoint{1.171115in}{0.961613in}}%
\pgfpathlineto{\pgfqpoint{1.170375in}{0.974539in}}%
\pgfpathlineto{\pgfqpoint{1.167239in}{0.984336in}}%
\pgfpathlineto{\pgfqpoint{1.164496in}{0.987466in}}%
\pgfpathlineto{\pgfqpoint{1.153970in}{0.992527in}}%
\pgfpathlineto{\pgfqpoint{1.140701in}{0.991090in}}%
\pgfpathlineto{\pgfqpoint{1.136012in}{0.987466in}}%
\pgfpathlineto{\pgfqpoint{1.130152in}{0.974539in}}%
\pgfpathlineto{\pgfqpoint{1.129220in}{0.961613in}}%
\pgfpathlineto{\pgfqpoint{1.131318in}{0.948686in}}%
\pgfpathlineto{\pgfqpoint{1.140701in}{0.936150in}}%
\pgfpathlineto{\pgfqpoint{1.143279in}{0.935760in}}%
\pgfpathclose%
\pgfpathmoveto{\pgfqpoint{1.140354in}{0.948686in}}%
\pgfpathlineto{\pgfqpoint{1.136831in}{0.961613in}}%
\pgfpathlineto{\pgfqpoint{1.138489in}{0.974539in}}%
\pgfpathlineto{\pgfqpoint{1.140701in}{0.978375in}}%
\pgfpathlineto{\pgfqpoint{1.153970in}{0.982057in}}%
\pgfpathlineto{\pgfqpoint{1.160941in}{0.974539in}}%
\pgfpathlineto{\pgfqpoint{1.163320in}{0.961613in}}%
\pgfpathlineto{\pgfqpoint{1.158264in}{0.948686in}}%
\pgfpathlineto{\pgfqpoint{1.153970in}{0.945111in}}%
\pgfpathlineto{\pgfqpoint{1.140701in}{0.948223in}}%
\pgfpathclose%
\pgfusepath{fill}%
\end{pgfscope}%
\begin{pgfscope}%
\pgfpathrectangle{\pgfqpoint{0.211875in}{0.211875in}}{\pgfqpoint{1.313625in}{1.279725in}}%
\pgfusepath{clip}%
\pgfsetbuttcap%
\pgfsetroundjoin%
\definecolor{currentfill}{rgb}{0.965440,0.720101,0.576404}%
\pgfsetfillcolor{currentfill}%
\pgfsetlinewidth{0.000000pt}%
\definecolor{currentstroke}{rgb}{0.000000,0.000000,0.000000}%
\pgfsetstrokecolor{currentstroke}%
\pgfsetdash{}{0pt}%
\pgfpathmoveto{\pgfqpoint{1.273390in}{0.935197in}}%
\pgfpathlineto{\pgfqpoint{1.274742in}{0.935760in}}%
\pgfpathlineto{\pgfqpoint{1.286659in}{0.947128in}}%
\pgfpathlineto{\pgfqpoint{1.287229in}{0.948686in}}%
\pgfpathlineto{\pgfqpoint{1.288902in}{0.961613in}}%
\pgfpathlineto{\pgfqpoint{1.288155in}{0.974539in}}%
\pgfpathlineto{\pgfqpoint{1.286659in}{0.979915in}}%
\pgfpathlineto{\pgfqpoint{1.281464in}{0.987466in}}%
\pgfpathlineto{\pgfqpoint{1.273390in}{0.991967in}}%
\pgfpathlineto{\pgfqpoint{1.260121in}{0.991357in}}%
\pgfpathlineto{\pgfqpoint{1.254426in}{0.987466in}}%
\pgfpathlineto{\pgfqpoint{1.247957in}{0.974539in}}%
\pgfpathlineto{\pgfqpoint{1.246930in}{0.961613in}}%
\pgfpathlineto{\pgfqpoint{1.249237in}{0.948686in}}%
\pgfpathlineto{\pgfqpoint{1.260121in}{0.935799in}}%
\pgfpathlineto{\pgfqpoint{1.260756in}{0.935760in}}%
\pgfpathclose%
\pgfpathmoveto{\pgfqpoint{1.259035in}{0.948686in}}%
\pgfpathlineto{\pgfqpoint{1.255176in}{0.961613in}}%
\pgfpathlineto{\pgfqpoint{1.256992in}{0.974539in}}%
\pgfpathlineto{\pgfqpoint{1.260121in}{0.979347in}}%
\pgfpathlineto{\pgfqpoint{1.273390in}{0.980966in}}%
\pgfpathlineto{\pgfqpoint{1.278546in}{0.974539in}}%
\pgfpathlineto{\pgfqpoint{1.280683in}{0.961613in}}%
\pgfpathlineto{\pgfqpoint{1.276150in}{0.948686in}}%
\pgfpathlineto{\pgfqpoint{1.273390in}{0.946030in}}%
\pgfpathlineto{\pgfqpoint{1.260121in}{0.947400in}}%
\pgfpathclose%
\pgfusepath{fill}%
\end{pgfscope}%
\begin{pgfscope}%
\pgfpathrectangle{\pgfqpoint{0.211875in}{0.211875in}}{\pgfqpoint{1.313625in}{1.279725in}}%
\pgfusepath{clip}%
\pgfsetbuttcap%
\pgfsetroundjoin%
\definecolor{currentfill}{rgb}{0.965440,0.720101,0.576404}%
\pgfsetfillcolor{currentfill}%
\pgfsetlinewidth{0.000000pt}%
\definecolor{currentstroke}{rgb}{0.000000,0.000000,0.000000}%
\pgfsetstrokecolor{currentstroke}%
\pgfsetdash{}{0pt}%
\pgfpathmoveto{\pgfqpoint{0.437447in}{0.947844in}}%
\pgfpathlineto{\pgfqpoint{0.447939in}{0.948686in}}%
\pgfpathlineto{\pgfqpoint{0.450716in}{0.949222in}}%
\pgfpathlineto{\pgfqpoint{0.454660in}{0.961613in}}%
\pgfpathlineto{\pgfqpoint{0.452705in}{0.974539in}}%
\pgfpathlineto{\pgfqpoint{0.450716in}{0.977526in}}%
\pgfpathlineto{\pgfqpoint{0.437447in}{0.978797in}}%
\pgfpathlineto{\pgfqpoint{0.434119in}{0.974539in}}%
\pgfpathlineto{\pgfqpoint{0.431877in}{0.961613in}}%
\pgfpathlineto{\pgfqpoint{0.436597in}{0.948686in}}%
\pgfpathclose%
\pgfusepath{fill}%
\end{pgfscope}%
\begin{pgfscope}%
\pgfpathrectangle{\pgfqpoint{0.211875in}{0.211875in}}{\pgfqpoint{1.313625in}{1.279725in}}%
\pgfusepath{clip}%
\pgfsetbuttcap%
\pgfsetroundjoin%
\definecolor{currentfill}{rgb}{0.965440,0.720101,0.576404}%
\pgfsetfillcolor{currentfill}%
\pgfsetlinewidth{0.000000pt}%
\definecolor{currentstroke}{rgb}{0.000000,0.000000,0.000000}%
\pgfsetstrokecolor{currentstroke}%
\pgfsetdash{}{0pt}%
\pgfpathmoveto{\pgfqpoint{0.556867in}{0.943298in}}%
\pgfpathlineto{\pgfqpoint{0.570136in}{0.945786in}}%
\pgfpathlineto{\pgfqpoint{0.572358in}{0.948686in}}%
\pgfpathlineto{\pgfqpoint{0.575646in}{0.961613in}}%
\pgfpathlineto{\pgfqpoint{0.574114in}{0.974539in}}%
\pgfpathlineto{\pgfqpoint{0.570136in}{0.981294in}}%
\pgfpathlineto{\pgfqpoint{0.556867in}{0.984222in}}%
\pgfpathlineto{\pgfqpoint{0.548150in}{0.974539in}}%
\pgfpathlineto{\pgfqpoint{0.545980in}{0.961613in}}%
\pgfpathlineto{\pgfqpoint{0.550602in}{0.948686in}}%
\pgfpathclose%
\pgfpathmoveto{\pgfqpoint{0.556610in}{0.961613in}}%
\pgfpathlineto{\pgfqpoint{0.556867in}{0.962758in}}%
\pgfpathlineto{\pgfqpoint{0.557889in}{0.961613in}}%
\pgfpathlineto{\pgfqpoint{0.556867in}{0.961017in}}%
\pgfpathclose%
\pgfusepath{fill}%
\end{pgfscope}%
\begin{pgfscope}%
\pgfpathrectangle{\pgfqpoint{0.211875in}{0.211875in}}{\pgfqpoint{1.313625in}{1.279725in}}%
\pgfusepath{clip}%
\pgfsetbuttcap%
\pgfsetroundjoin%
\definecolor{currentfill}{rgb}{0.965440,0.720101,0.576404}%
\pgfsetfillcolor{currentfill}%
\pgfsetlinewidth{0.000000pt}%
\definecolor{currentstroke}{rgb}{0.000000,0.000000,0.000000}%
\pgfsetstrokecolor{currentstroke}%
\pgfsetdash{}{0pt}%
\pgfpathmoveto{\pgfqpoint{0.676288in}{0.939771in}}%
\pgfpathlineto{\pgfqpoint{0.689557in}{0.943397in}}%
\pgfpathlineto{\pgfqpoint{0.693154in}{0.948686in}}%
\pgfpathlineto{\pgfqpoint{0.695820in}{0.961613in}}%
\pgfpathlineto{\pgfqpoint{0.694594in}{0.974539in}}%
\pgfpathlineto{\pgfqpoint{0.689557in}{0.984179in}}%
\pgfpathlineto{\pgfqpoint{0.679729in}{0.987466in}}%
\pgfpathlineto{\pgfqpoint{0.676288in}{0.988117in}}%
\pgfpathlineto{\pgfqpoint{0.674771in}{0.987466in}}%
\pgfpathlineto{\pgfqpoint{0.663019in}{0.976473in}}%
\pgfpathlineto{\pgfqpoint{0.662326in}{0.974539in}}%
\pgfpathlineto{\pgfqpoint{0.661244in}{0.961613in}}%
\pgfpathlineto{\pgfqpoint{0.663019in}{0.951413in}}%
\pgfpathlineto{\pgfqpoint{0.664141in}{0.948686in}}%
\pgfpathclose%
\pgfpathmoveto{\pgfqpoint{0.671760in}{0.961613in}}%
\pgfpathlineto{\pgfqpoint{0.675062in}{0.974539in}}%
\pgfpathlineto{\pgfqpoint{0.676288in}{0.975701in}}%
\pgfpathlineto{\pgfqpoint{0.678936in}{0.974539in}}%
\pgfpathlineto{\pgfqpoint{0.685990in}{0.961613in}}%
\pgfpathlineto{\pgfqpoint{0.676288in}{0.952657in}}%
\pgfpathclose%
\pgfusepath{fill}%
\end{pgfscope}%
\begin{pgfscope}%
\pgfpathrectangle{\pgfqpoint{0.211875in}{0.211875in}}{\pgfqpoint{1.313625in}{1.279725in}}%
\pgfusepath{clip}%
\pgfsetbuttcap%
\pgfsetroundjoin%
\definecolor{currentfill}{rgb}{0.965440,0.720101,0.576404}%
\pgfsetfillcolor{currentfill}%
\pgfsetlinewidth{0.000000pt}%
\definecolor{currentstroke}{rgb}{0.000000,0.000000,0.000000}%
\pgfsetstrokecolor{currentstroke}%
\pgfsetdash{}{0pt}%
\pgfpathmoveto{\pgfqpoint{0.782439in}{0.944037in}}%
\pgfpathlineto{\pgfqpoint{0.795708in}{0.937159in}}%
\pgfpathlineto{\pgfqpoint{0.808977in}{0.941815in}}%
\pgfpathlineto{\pgfqpoint{0.813134in}{0.948686in}}%
\pgfpathlineto{\pgfqpoint{0.815360in}{0.961613in}}%
\pgfpathlineto{\pgfqpoint{0.814350in}{0.974539in}}%
\pgfpathlineto{\pgfqpoint{0.808977in}{0.986108in}}%
\pgfpathlineto{\pgfqpoint{0.806232in}{0.987466in}}%
\pgfpathlineto{\pgfqpoint{0.795708in}{0.990222in}}%
\pgfpathlineto{\pgfqpoint{0.787976in}{0.987466in}}%
\pgfpathlineto{\pgfqpoint{0.782439in}{0.983477in}}%
\pgfpathlineto{\pgfqpoint{0.778818in}{0.974539in}}%
\pgfpathlineto{\pgfqpoint{0.777836in}{0.961613in}}%
\pgfpathlineto{\pgfqpoint{0.779986in}{0.948686in}}%
\pgfpathclose%
\pgfpathmoveto{\pgfqpoint{0.794171in}{0.948686in}}%
\pgfpathlineto{\pgfqpoint{0.786552in}{0.961613in}}%
\pgfpathlineto{\pgfqpoint{0.790178in}{0.974539in}}%
\pgfpathlineto{\pgfqpoint{0.795708in}{0.978899in}}%
\pgfpathlineto{\pgfqpoint{0.802997in}{0.974539in}}%
\pgfpathlineto{\pgfqpoint{0.807710in}{0.961613in}}%
\pgfpathlineto{\pgfqpoint{0.797746in}{0.948686in}}%
\pgfpathlineto{\pgfqpoint{0.795708in}{0.947748in}}%
\pgfpathclose%
\pgfusepath{fill}%
\end{pgfscope}%
\begin{pgfscope}%
\pgfpathrectangle{\pgfqpoint{0.211875in}{0.211875in}}{\pgfqpoint{1.313625in}{1.279725in}}%
\pgfusepath{clip}%
\pgfsetbuttcap%
\pgfsetroundjoin%
\definecolor{currentfill}{rgb}{0.965440,0.720101,0.576404}%
\pgfsetfillcolor{currentfill}%
\pgfsetlinewidth{0.000000pt}%
\definecolor{currentstroke}{rgb}{0.000000,0.000000,0.000000}%
\pgfsetstrokecolor{currentstroke}%
\pgfsetdash{}{0pt}%
\pgfpathmoveto{\pgfqpoint{1.379542in}{0.936336in}}%
\pgfpathlineto{\pgfqpoint{1.392811in}{0.936625in}}%
\pgfpathlineto{\pgfqpoint{1.403790in}{0.948686in}}%
\pgfpathlineto{\pgfqpoint{1.406080in}{0.959623in}}%
\pgfpathlineto{\pgfqpoint{1.406314in}{0.961613in}}%
\pgfpathlineto{\pgfqpoint{1.406080in}{0.965675in}}%
\pgfpathlineto{\pgfqpoint{1.405240in}{0.974539in}}%
\pgfpathlineto{\pgfqpoint{1.397849in}{0.987466in}}%
\pgfpathlineto{\pgfqpoint{1.392811in}{0.990679in}}%
\pgfpathlineto{\pgfqpoint{1.379542in}{0.990909in}}%
\pgfpathlineto{\pgfqpoint{1.373798in}{0.987466in}}%
\pgfpathlineto{\pgfqpoint{1.366273in}{0.974691in}}%
\pgfpathlineto{\pgfqpoint{1.366235in}{0.974539in}}%
\pgfpathlineto{\pgfqpoint{1.365453in}{0.961613in}}%
\pgfpathlineto{\pgfqpoint{1.366273in}{0.954975in}}%
\pgfpathlineto{\pgfqpoint{1.367695in}{0.948686in}}%
\pgfpathclose%
\pgfpathmoveto{\pgfqpoint{1.378470in}{0.948686in}}%
\pgfpathlineto{\pgfqpoint{1.374069in}{0.961613in}}%
\pgfpathlineto{\pgfqpoint{1.376145in}{0.974539in}}%
\pgfpathlineto{\pgfqpoint{1.379542in}{0.979133in}}%
\pgfpathlineto{\pgfqpoint{1.392811in}{0.978698in}}%
\pgfpathlineto{\pgfqpoint{1.395733in}{0.974539in}}%
\pgfpathlineto{\pgfqpoint{1.397743in}{0.961613in}}%
\pgfpathlineto{\pgfqpoint{1.393495in}{0.948686in}}%
\pgfpathlineto{\pgfqpoint{1.392811in}{0.947935in}}%
\pgfpathlineto{\pgfqpoint{1.379542in}{0.947569in}}%
\pgfpathclose%
\pgfusepath{fill}%
\end{pgfscope}%
\begin{pgfscope}%
\pgfpathrectangle{\pgfqpoint{0.211875in}{0.211875in}}{\pgfqpoint{1.313625in}{1.279725in}}%
\pgfusepath{clip}%
\pgfsetbuttcap%
\pgfsetroundjoin%
\definecolor{currentfill}{rgb}{0.965440,0.720101,0.576404}%
\pgfsetfillcolor{currentfill}%
\pgfsetlinewidth{0.000000pt}%
\definecolor{currentstroke}{rgb}{0.000000,0.000000,0.000000}%
\pgfsetstrokecolor{currentstroke}%
\pgfsetdash{}{0pt}%
\pgfpathmoveto{\pgfqpoint{1.498962in}{0.937693in}}%
\pgfpathlineto{\pgfqpoint{1.512231in}{0.939241in}}%
\pgfpathlineto{\pgfqpoint{1.519829in}{0.948686in}}%
\pgfpathlineto{\pgfqpoint{1.522500in}{0.961613in}}%
\pgfpathlineto{\pgfqpoint{1.521280in}{0.974539in}}%
\pgfpathlineto{\pgfqpoint{1.513769in}{0.987466in}}%
\pgfpathlineto{\pgfqpoint{1.512231in}{0.988578in}}%
\pgfpathlineto{\pgfqpoint{1.498962in}{0.989804in}}%
\pgfpathlineto{\pgfqpoint{1.494471in}{0.987466in}}%
\pgfpathlineto{\pgfqpoint{1.485693in}{0.975821in}}%
\pgfpathlineto{\pgfqpoint{1.485319in}{0.974539in}}%
\pgfpathlineto{\pgfqpoint{1.484433in}{0.961613in}}%
\pgfpathlineto{\pgfqpoint{1.485693in}{0.952773in}}%
\pgfpathlineto{\pgfqpoint{1.486864in}{0.948686in}}%
\pgfpathclose%
\pgfpathmoveto{\pgfqpoint{1.498920in}{0.948686in}}%
\pgfpathlineto{\pgfqpoint{1.493690in}{0.961613in}}%
\pgfpathlineto{\pgfqpoint{1.496167in}{0.974539in}}%
\pgfpathlineto{\pgfqpoint{1.498962in}{0.977834in}}%
\pgfpathlineto{\pgfqpoint{1.512231in}{0.975111in}}%
\pgfpathlineto{\pgfqpoint{1.512586in}{0.974539in}}%
\pgfpathlineto{\pgfqpoint{1.514552in}{0.961613in}}%
\pgfpathlineto{\pgfqpoint{1.512231in}{0.953826in}}%
\pgfpathlineto{\pgfqpoint{1.499198in}{0.948686in}}%
\pgfpathlineto{\pgfqpoint{1.498962in}{0.948648in}}%
\pgfpathclose%
\pgfusepath{fill}%
\end{pgfscope}%
\begin{pgfscope}%
\pgfpathrectangle{\pgfqpoint{0.211875in}{0.211875in}}{\pgfqpoint{1.313625in}{1.279725in}}%
\pgfusepath{clip}%
\pgfsetbuttcap%
\pgfsetroundjoin%
\definecolor{currentfill}{rgb}{0.965440,0.720101,0.576404}%
\pgfsetfillcolor{currentfill}%
\pgfsetlinewidth{0.000000pt}%
\definecolor{currentstroke}{rgb}{0.000000,0.000000,0.000000}%
\pgfsetstrokecolor{currentstroke}%
\pgfsetdash{}{0pt}%
\pgfpathmoveto{\pgfqpoint{0.318027in}{0.959794in}}%
\pgfpathlineto{\pgfqpoint{0.331295in}{0.958027in}}%
\pgfpathlineto{\pgfqpoint{0.332593in}{0.961613in}}%
\pgfpathlineto{\pgfqpoint{0.331295in}{0.968515in}}%
\pgfpathlineto{\pgfqpoint{0.318027in}{0.965115in}}%
\pgfpathlineto{\pgfqpoint{0.317427in}{0.961613in}}%
\pgfpathclose%
\pgfusepath{fill}%
\end{pgfscope}%
\begin{pgfscope}%
\pgfpathrectangle{\pgfqpoint{0.211875in}{0.211875in}}{\pgfqpoint{1.313625in}{1.279725in}}%
\pgfusepath{clip}%
\pgfsetbuttcap%
\pgfsetroundjoin%
\definecolor{currentfill}{rgb}{0.965440,0.720101,0.576404}%
\pgfsetfillcolor{currentfill}%
\pgfsetlinewidth{0.000000pt}%
\definecolor{currentstroke}{rgb}{0.000000,0.000000,0.000000}%
\pgfsetstrokecolor{currentstroke}%
\pgfsetdash{}{0pt}%
\pgfpathmoveto{\pgfqpoint{0.503792in}{1.024782in}}%
\pgfpathlineto{\pgfqpoint{0.506528in}{1.026245in}}%
\pgfpathlineto{\pgfqpoint{0.515979in}{1.039172in}}%
\pgfpathlineto{\pgfqpoint{0.515490in}{1.052098in}}%
\pgfpathlineto{\pgfqpoint{0.504033in}{1.065025in}}%
\pgfpathlineto{\pgfqpoint{0.503792in}{1.065141in}}%
\pgfpathlineto{\pgfqpoint{0.503402in}{1.065025in}}%
\pgfpathlineto{\pgfqpoint{0.490523in}{1.057228in}}%
\pgfpathlineto{\pgfqpoint{0.488543in}{1.052098in}}%
\pgfpathlineto{\pgfqpoint{0.488278in}{1.039172in}}%
\pgfpathlineto{\pgfqpoint{0.490523in}{1.032561in}}%
\pgfpathlineto{\pgfqpoint{0.499329in}{1.026245in}}%
\pgfpathclose%
\pgfusepath{fill}%
\end{pgfscope}%
\begin{pgfscope}%
\pgfpathrectangle{\pgfqpoint{0.211875in}{0.211875in}}{\pgfqpoint{1.313625in}{1.279725in}}%
\pgfusepath{clip}%
\pgfsetbuttcap%
\pgfsetroundjoin%
\definecolor{currentfill}{rgb}{0.965440,0.720101,0.576404}%
\pgfsetfillcolor{currentfill}%
\pgfsetlinewidth{0.000000pt}%
\definecolor{currentstroke}{rgb}{0.000000,0.000000,0.000000}%
\pgfsetstrokecolor{currentstroke}%
\pgfsetdash{}{0pt}%
\pgfpathmoveto{\pgfqpoint{0.609943in}{1.024691in}}%
\pgfpathlineto{\pgfqpoint{0.623212in}{1.021858in}}%
\pgfpathlineto{\pgfqpoint{0.630108in}{1.026245in}}%
\pgfpathlineto{\pgfqpoint{0.636481in}{1.037893in}}%
\pgfpathlineto{\pgfqpoint{0.636771in}{1.039172in}}%
\pgfpathlineto{\pgfqpoint{0.636582in}{1.052098in}}%
\pgfpathlineto{\pgfqpoint{0.636481in}{1.052494in}}%
\pgfpathlineto{\pgfqpoint{0.628332in}{1.065025in}}%
\pgfpathlineto{\pgfqpoint{0.623212in}{1.067961in}}%
\pgfpathlineto{\pgfqpoint{0.609943in}{1.065250in}}%
\pgfpathlineto{\pgfqpoint{0.609721in}{1.065025in}}%
\pgfpathlineto{\pgfqpoint{0.604176in}{1.052098in}}%
\pgfpathlineto{\pgfqpoint{0.603938in}{1.039172in}}%
\pgfpathlineto{\pgfqpoint{0.608552in}{1.026245in}}%
\pgfpathclose%
\pgfpathmoveto{\pgfqpoint{0.616918in}{1.039172in}}%
\pgfpathlineto{\pgfqpoint{0.618560in}{1.052098in}}%
\pgfpathlineto{\pgfqpoint{0.623212in}{1.054234in}}%
\pgfpathlineto{\pgfqpoint{0.624818in}{1.052098in}}%
\pgfpathlineto{\pgfqpoint{0.625380in}{1.039172in}}%
\pgfpathlineto{\pgfqpoint{0.623212in}{1.035883in}}%
\pgfpathclose%
\pgfusepath{fill}%
\end{pgfscope}%
\begin{pgfscope}%
\pgfpathrectangle{\pgfqpoint{0.211875in}{0.211875in}}{\pgfqpoint{1.313625in}{1.279725in}}%
\pgfusepath{clip}%
\pgfsetbuttcap%
\pgfsetroundjoin%
\definecolor{currentfill}{rgb}{0.965440,0.720101,0.576404}%
\pgfsetfillcolor{currentfill}%
\pgfsetlinewidth{0.000000pt}%
\definecolor{currentstroke}{rgb}{0.000000,0.000000,0.000000}%
\pgfsetstrokecolor{currentstroke}%
\pgfsetdash{}{0pt}%
\pgfpathmoveto{\pgfqpoint{0.729364in}{1.020953in}}%
\pgfpathlineto{\pgfqpoint{0.742633in}{1.019634in}}%
\pgfpathlineto{\pgfqpoint{0.751558in}{1.026245in}}%
\pgfpathlineto{\pgfqpoint{0.755902in}{1.036354in}}%
\pgfpathlineto{\pgfqpoint{0.756459in}{1.039172in}}%
\pgfpathlineto{\pgfqpoint{0.756303in}{1.052098in}}%
\pgfpathlineto{\pgfqpoint{0.755902in}{1.053899in}}%
\pgfpathlineto{\pgfqpoint{0.750239in}{1.065025in}}%
\pgfpathlineto{\pgfqpoint{0.742633in}{1.070108in}}%
\pgfpathlineto{\pgfqpoint{0.729364in}{1.068848in}}%
\pgfpathlineto{\pgfqpoint{0.725108in}{1.065025in}}%
\pgfpathlineto{\pgfqpoint{0.719899in}{1.052098in}}%
\pgfpathlineto{\pgfqpoint{0.719681in}{1.039172in}}%
\pgfpathlineto{\pgfqpoint{0.724029in}{1.026245in}}%
\pgfpathclose%
\pgfpathmoveto{\pgfqpoint{0.727855in}{1.039172in}}%
\pgfpathlineto{\pgfqpoint{0.728217in}{1.052098in}}%
\pgfpathlineto{\pgfqpoint{0.729364in}{1.054449in}}%
\pgfpathlineto{\pgfqpoint{0.742633in}{1.057560in}}%
\pgfpathlineto{\pgfqpoint{0.746169in}{1.052098in}}%
\pgfpathlineto{\pgfqpoint{0.746614in}{1.039172in}}%
\pgfpathlineto{\pgfqpoint{0.742633in}{1.032161in}}%
\pgfpathlineto{\pgfqpoint{0.729364in}{1.035653in}}%
\pgfpathclose%
\pgfusepath{fill}%
\end{pgfscope}%
\begin{pgfscope}%
\pgfpathrectangle{\pgfqpoint{0.211875in}{0.211875in}}{\pgfqpoint{1.313625in}{1.279725in}}%
\pgfusepath{clip}%
\pgfsetbuttcap%
\pgfsetroundjoin%
\definecolor{currentfill}{rgb}{0.965440,0.720101,0.576404}%
\pgfsetfillcolor{currentfill}%
\pgfsetlinewidth{0.000000pt}%
\definecolor{currentstroke}{rgb}{0.000000,0.000000,0.000000}%
\pgfsetstrokecolor{currentstroke}%
\pgfsetdash{}{0pt}%
\pgfpathmoveto{\pgfqpoint{0.848784in}{1.018282in}}%
\pgfpathlineto{\pgfqpoint{0.862053in}{1.018105in}}%
\pgfpathlineto{\pgfqpoint{0.871633in}{1.026245in}}%
\pgfpathlineto{\pgfqpoint{0.875322in}{1.037171in}}%
\pgfpathlineto{\pgfqpoint{0.875663in}{1.039172in}}%
\pgfpathlineto{\pgfqpoint{0.875531in}{1.052098in}}%
\pgfpathlineto{\pgfqpoint{0.875322in}{1.053187in}}%
\pgfpathlineto{\pgfqpoint{0.870606in}{1.065025in}}%
\pgfpathlineto{\pgfqpoint{0.862053in}{1.071588in}}%
\pgfpathlineto{\pgfqpoint{0.848784in}{1.071418in}}%
\pgfpathlineto{\pgfqpoint{0.840716in}{1.065025in}}%
\pgfpathlineto{\pgfqpoint{0.835665in}{1.052098in}}%
\pgfpathlineto{\pgfqpoint{0.835515in}{1.042769in}}%
\pgfpathlineto{\pgfqpoint{0.835478in}{1.039172in}}%
\pgfpathlineto{\pgfqpoint{0.835515in}{1.038950in}}%
\pgfpathlineto{\pgfqpoint{0.839689in}{1.026245in}}%
\pgfpathclose%
\pgfpathmoveto{\pgfqpoint{0.844409in}{1.039172in}}%
\pgfpathlineto{\pgfqpoint{0.844772in}{1.052098in}}%
\pgfpathlineto{\pgfqpoint{0.848784in}{1.059359in}}%
\pgfpathlineto{\pgfqpoint{0.862053in}{1.059637in}}%
\pgfpathlineto{\pgfqpoint{0.866315in}{1.052098in}}%
\pgfpathlineto{\pgfqpoint{0.866682in}{1.039172in}}%
\pgfpathlineto{\pgfqpoint{0.862053in}{1.029841in}}%
\pgfpathlineto{\pgfqpoint{0.848784in}{1.030152in}}%
\pgfpathclose%
\pgfusepath{fill}%
\end{pgfscope}%
\begin{pgfscope}%
\pgfpathrectangle{\pgfqpoint{0.211875in}{0.211875in}}{\pgfqpoint{1.313625in}{1.279725in}}%
\pgfusepath{clip}%
\pgfsetbuttcap%
\pgfsetroundjoin%
\definecolor{currentfill}{rgb}{0.965440,0.720101,0.576404}%
\pgfsetfillcolor{currentfill}%
\pgfsetlinewidth{0.000000pt}%
\definecolor{currentstroke}{rgb}{0.000000,0.000000,0.000000}%
\pgfsetstrokecolor{currentstroke}%
\pgfsetdash{}{0pt}%
\pgfpathmoveto{\pgfqpoint{0.968205in}{1.016528in}}%
\pgfpathlineto{\pgfqpoint{0.981473in}{1.017293in}}%
\pgfpathlineto{\pgfqpoint{0.990747in}{1.026245in}}%
\pgfpathlineto{\pgfqpoint{0.994281in}{1.039172in}}%
\pgfpathlineto{\pgfqpoint{0.994115in}{1.052098in}}%
\pgfpathlineto{\pgfqpoint{0.989903in}{1.065025in}}%
\pgfpathlineto{\pgfqpoint{0.981473in}{1.072378in}}%
\pgfpathlineto{\pgfqpoint{0.968205in}{1.073103in}}%
\pgfpathlineto{\pgfqpoint{0.956543in}{1.065025in}}%
\pgfpathlineto{\pgfqpoint{0.954936in}{1.061757in}}%
\pgfpathlineto{\pgfqpoint{0.952810in}{1.052098in}}%
\pgfpathlineto{\pgfqpoint{0.952687in}{1.039172in}}%
\pgfpathlineto{\pgfqpoint{0.954936in}{1.027641in}}%
\pgfpathlineto{\pgfqpoint{0.955517in}{1.026245in}}%
\pgfpathclose%
\pgfpathmoveto{\pgfqpoint{0.961174in}{1.039172in}}%
\pgfpathlineto{\pgfqpoint{0.961555in}{1.052098in}}%
\pgfpathlineto{\pgfqpoint{0.968205in}{1.062635in}}%
\pgfpathlineto{\pgfqpoint{0.981473in}{1.060410in}}%
\pgfpathlineto{\pgfqpoint{0.985615in}{1.052098in}}%
\pgfpathlineto{\pgfqpoint{0.985931in}{1.039172in}}%
\pgfpathlineto{\pgfqpoint{0.981473in}{1.028984in}}%
\pgfpathlineto{\pgfqpoint{0.968205in}{1.026478in}}%
\pgfpathclose%
\pgfusepath{fill}%
\end{pgfscope}%
\begin{pgfscope}%
\pgfpathrectangle{\pgfqpoint{0.211875in}{0.211875in}}{\pgfqpoint{1.313625in}{1.279725in}}%
\pgfusepath{clip}%
\pgfsetbuttcap%
\pgfsetroundjoin%
\definecolor{currentfill}{rgb}{0.965440,0.720101,0.576404}%
\pgfsetfillcolor{currentfill}%
\pgfsetlinewidth{0.000000pt}%
\definecolor{currentstroke}{rgb}{0.000000,0.000000,0.000000}%
\pgfsetstrokecolor{currentstroke}%
\pgfsetdash{}{0pt}%
\pgfpathmoveto{\pgfqpoint{1.074356in}{1.023288in}}%
\pgfpathlineto{\pgfqpoint{1.087625in}{1.015593in}}%
\pgfpathlineto{\pgfqpoint{1.100894in}{1.017254in}}%
\pgfpathlineto{\pgfqpoint{1.109144in}{1.026245in}}%
\pgfpathlineto{\pgfqpoint{1.112283in}{1.039172in}}%
\pgfpathlineto{\pgfqpoint{1.112137in}{1.052098in}}%
\pgfpathlineto{\pgfqpoint{1.108402in}{1.065025in}}%
\pgfpathlineto{\pgfqpoint{1.100894in}{1.072426in}}%
\pgfpathlineto{\pgfqpoint{1.087625in}{1.074000in}}%
\pgfpathlineto{\pgfqpoint{1.074356in}{1.066677in}}%
\pgfpathlineto{\pgfqpoint{1.073398in}{1.065025in}}%
\pgfpathlineto{\pgfqpoint{1.070385in}{1.052098in}}%
\pgfpathlineto{\pgfqpoint{1.070267in}{1.039172in}}%
\pgfpathlineto{\pgfqpoint{1.072800in}{1.026245in}}%
\pgfpathclose%
\pgfpathmoveto{\pgfqpoint{1.086124in}{1.026245in}}%
\pgfpathlineto{\pgfqpoint{1.078183in}{1.039172in}}%
\pgfpathlineto{\pgfqpoint{1.078600in}{1.052098in}}%
\pgfpathlineto{\pgfqpoint{1.087625in}{1.064456in}}%
\pgfpathlineto{\pgfqpoint{1.100894in}{1.059765in}}%
\pgfpathlineto{\pgfqpoint{1.104281in}{1.052098in}}%
\pgfpathlineto{\pgfqpoint{1.104565in}{1.039172in}}%
\pgfpathlineto{\pgfqpoint{1.100894in}{1.029715in}}%
\pgfpathlineto{\pgfqpoint{1.092542in}{1.026245in}}%
\pgfpathlineto{\pgfqpoint{1.087625in}{1.025253in}}%
\pgfpathclose%
\pgfusepath{fill}%
\end{pgfscope}%
\begin{pgfscope}%
\pgfpathrectangle{\pgfqpoint{0.211875in}{0.211875in}}{\pgfqpoint{1.313625in}{1.279725in}}%
\pgfusepath{clip}%
\pgfsetbuttcap%
\pgfsetroundjoin%
\definecolor{currentfill}{rgb}{0.965440,0.720101,0.576404}%
\pgfsetfillcolor{currentfill}%
\pgfsetlinewidth{0.000000pt}%
\definecolor{currentstroke}{rgb}{0.000000,0.000000,0.000000}%
\pgfsetstrokecolor{currentstroke}%
\pgfsetdash{}{0pt}%
\pgfpathmoveto{\pgfqpoint{1.193777in}{1.021303in}}%
\pgfpathlineto{\pgfqpoint{1.207045in}{1.015416in}}%
\pgfpathlineto{\pgfqpoint{1.220314in}{1.018085in}}%
\pgfpathlineto{\pgfqpoint{1.226970in}{1.026245in}}%
\pgfpathlineto{\pgfqpoint{1.229927in}{1.039172in}}%
\pgfpathlineto{\pgfqpoint{1.229789in}{1.052098in}}%
\pgfpathlineto{\pgfqpoint{1.226271in}{1.065025in}}%
\pgfpathlineto{\pgfqpoint{1.220314in}{1.071635in}}%
\pgfpathlineto{\pgfqpoint{1.207045in}{1.074168in}}%
\pgfpathlineto{\pgfqpoint{1.193777in}{1.068574in}}%
\pgfpathlineto{\pgfqpoint{1.191436in}{1.065025in}}%
\pgfpathlineto{\pgfqpoint{1.188343in}{1.052098in}}%
\pgfpathlineto{\pgfqpoint{1.188222in}{1.039172in}}%
\pgfpathlineto{\pgfqpoint{1.190822in}{1.026245in}}%
\pgfpathclose%
\pgfpathmoveto{\pgfqpoint{1.204745in}{1.026245in}}%
\pgfpathlineto{\pgfqpoint{1.195504in}{1.039172in}}%
\pgfpathlineto{\pgfqpoint{1.195988in}{1.052098in}}%
\pgfpathlineto{\pgfqpoint{1.207045in}{1.064934in}}%
\pgfpathlineto{\pgfqpoint{1.220314in}{1.057512in}}%
\pgfpathlineto{\pgfqpoint{1.222441in}{1.052098in}}%
\pgfpathlineto{\pgfqpoint{1.222708in}{1.039172in}}%
\pgfpathlineto{\pgfqpoint{1.220314in}{1.032244in}}%
\pgfpathlineto{\pgfqpoint{1.211286in}{1.026245in}}%
\pgfpathlineto{\pgfqpoint{1.207045in}{1.024958in}}%
\pgfpathclose%
\pgfusepath{fill}%
\end{pgfscope}%
\begin{pgfscope}%
\pgfpathrectangle{\pgfqpoint{0.211875in}{0.211875in}}{\pgfqpoint{1.313625in}{1.279725in}}%
\pgfusepath{clip}%
\pgfsetbuttcap%
\pgfsetroundjoin%
\definecolor{currentfill}{rgb}{0.965440,0.720101,0.576404}%
\pgfsetfillcolor{currentfill}%
\pgfsetlinewidth{0.000000pt}%
\definecolor{currentstroke}{rgb}{0.000000,0.000000,0.000000}%
\pgfsetstrokecolor{currentstroke}%
\pgfsetdash{}{0pt}%
\pgfpathmoveto{\pgfqpoint{1.313197in}{1.020607in}}%
\pgfpathlineto{\pgfqpoint{1.326466in}{1.015966in}}%
\pgfpathlineto{\pgfqpoint{1.339735in}{1.019939in}}%
\pgfpathlineto{\pgfqpoint{1.344311in}{1.026245in}}%
\pgfpathlineto{\pgfqpoint{1.347256in}{1.039172in}}%
\pgfpathlineto{\pgfqpoint{1.347117in}{1.052098in}}%
\pgfpathlineto{\pgfqpoint{1.343607in}{1.065025in}}%
\pgfpathlineto{\pgfqpoint{1.339735in}{1.069859in}}%
\pgfpathlineto{\pgfqpoint{1.326466in}{1.073637in}}%
\pgfpathlineto{\pgfqpoint{1.313197in}{1.069227in}}%
\pgfpathlineto{\pgfqpoint{1.310064in}{1.065025in}}%
\pgfpathlineto{\pgfqpoint{1.306703in}{1.052098in}}%
\pgfpathlineto{\pgfqpoint{1.306570in}{1.039172in}}%
\pgfpathlineto{\pgfqpoint{1.309390in}{1.026245in}}%
\pgfpathclose%
\pgfpathmoveto{\pgfqpoint{1.324745in}{1.026245in}}%
\pgfpathlineto{\pgfqpoint{1.313288in}{1.039172in}}%
\pgfpathlineto{\pgfqpoint{1.313889in}{1.052098in}}%
\pgfpathlineto{\pgfqpoint{1.326466in}{1.064123in}}%
\pgfpathlineto{\pgfqpoint{1.339735in}{1.053354in}}%
\pgfpathlineto{\pgfqpoint{1.340174in}{1.052098in}}%
\pgfpathlineto{\pgfqpoint{1.340435in}{1.039172in}}%
\pgfpathlineto{\pgfqpoint{1.339735in}{1.036899in}}%
\pgfpathlineto{\pgfqpoint{1.328427in}{1.026245in}}%
\pgfpathlineto{\pgfqpoint{1.326466in}{1.025455in}}%
\pgfpathclose%
\pgfusepath{fill}%
\end{pgfscope}%
\begin{pgfscope}%
\pgfpathrectangle{\pgfqpoint{0.211875in}{0.211875in}}{\pgfqpoint{1.313625in}{1.279725in}}%
\pgfusepath{clip}%
\pgfsetbuttcap%
\pgfsetroundjoin%
\definecolor{currentfill}{rgb}{0.965440,0.720101,0.576404}%
\pgfsetfillcolor{currentfill}%
\pgfsetlinewidth{0.000000pt}%
\definecolor{currentstroke}{rgb}{0.000000,0.000000,0.000000}%
\pgfsetstrokecolor{currentstroke}%
\pgfsetdash{}{0pt}%
\pgfpathmoveto{\pgfqpoint{1.432617in}{1.020923in}}%
\pgfpathlineto{\pgfqpoint{1.445886in}{1.017243in}}%
\pgfpathlineto{\pgfqpoint{1.459155in}{1.023049in}}%
\pgfpathlineto{\pgfqpoint{1.461217in}{1.026245in}}%
\pgfpathlineto{\pgfqpoint{1.464293in}{1.039172in}}%
\pgfpathlineto{\pgfqpoint{1.464143in}{1.052098in}}%
\pgfpathlineto{\pgfqpoint{1.460468in}{1.065025in}}%
\pgfpathlineto{\pgfqpoint{1.459155in}{1.066871in}}%
\pgfpathlineto{\pgfqpoint{1.445886in}{1.072406in}}%
\pgfpathlineto{\pgfqpoint{1.432617in}{1.068906in}}%
\pgfpathlineto{\pgfqpoint{1.429357in}{1.065025in}}%
\pgfpathlineto{\pgfqpoint{1.425507in}{1.052098in}}%
\pgfpathlineto{\pgfqpoint{1.425352in}{1.039172in}}%
\pgfpathlineto{\pgfqpoint{1.428572in}{1.026245in}}%
\pgfpathclose%
\pgfpathmoveto{\pgfqpoint{1.432361in}{1.039172in}}%
\pgfpathlineto{\pgfqpoint{1.432617in}{1.051019in}}%
\pgfpathlineto{\pgfqpoint{1.432686in}{1.052098in}}%
\pgfpathlineto{\pgfqpoint{1.445886in}{1.062030in}}%
\pgfpathlineto{\pgfqpoint{1.455471in}{1.052098in}}%
\pgfpathlineto{\pgfqpoint{1.456067in}{1.039172in}}%
\pgfpathlineto{\pgfqpoint{1.445886in}{1.027146in}}%
\pgfpathlineto{\pgfqpoint{1.432617in}{1.038371in}}%
\pgfpathclose%
\pgfusepath{fill}%
\end{pgfscope}%
\begin{pgfscope}%
\pgfpathrectangle{\pgfqpoint{0.211875in}{0.211875in}}{\pgfqpoint{1.313625in}{1.279725in}}%
\pgfusepath{clip}%
\pgfsetbuttcap%
\pgfsetroundjoin%
\definecolor{currentfill}{rgb}{0.965440,0.720101,0.576404}%
\pgfsetfillcolor{currentfill}%
\pgfsetlinewidth{0.000000pt}%
\definecolor{currentstroke}{rgb}{0.000000,0.000000,0.000000}%
\pgfsetstrokecolor{currentstroke}%
\pgfsetdash{}{0pt}%
\pgfpathmoveto{\pgfqpoint{0.264951in}{1.038371in}}%
\pgfpathlineto{\pgfqpoint{0.265933in}{1.039172in}}%
\pgfpathlineto{\pgfqpoint{0.264951in}{1.050688in}}%
\pgfpathlineto{\pgfqpoint{0.264297in}{1.039172in}}%
\pgfpathclose%
\pgfusepath{fill}%
\end{pgfscope}%
\begin{pgfscope}%
\pgfpathrectangle{\pgfqpoint{0.211875in}{0.211875in}}{\pgfqpoint{1.313625in}{1.279725in}}%
\pgfusepath{clip}%
\pgfsetbuttcap%
\pgfsetroundjoin%
\definecolor{currentfill}{rgb}{0.965440,0.720101,0.576404}%
\pgfsetfillcolor{currentfill}%
\pgfsetlinewidth{0.000000pt}%
\definecolor{currentstroke}{rgb}{0.000000,0.000000,0.000000}%
\pgfsetstrokecolor{currentstroke}%
\pgfsetdash{}{0pt}%
\pgfpathmoveto{\pgfqpoint{0.384371in}{1.030250in}}%
\pgfpathlineto{\pgfqpoint{0.392892in}{1.039172in}}%
\pgfpathlineto{\pgfqpoint{0.392178in}{1.052098in}}%
\pgfpathlineto{\pgfqpoint{0.384371in}{1.059260in}}%
\pgfpathlineto{\pgfqpoint{0.376152in}{1.052098in}}%
\pgfpathlineto{\pgfqpoint{0.375382in}{1.039172in}}%
\pgfpathclose%
\pgfusepath{fill}%
\end{pgfscope}%
\begin{pgfscope}%
\pgfpathrectangle{\pgfqpoint{0.211875in}{0.211875in}}{\pgfqpoint{1.313625in}{1.279725in}}%
\pgfusepath{clip}%
\pgfsetbuttcap%
\pgfsetroundjoin%
\definecolor{currentfill}{rgb}{0.965440,0.720101,0.576404}%
\pgfsetfillcolor{currentfill}%
\pgfsetlinewidth{0.000000pt}%
\definecolor{currentstroke}{rgb}{0.000000,0.000000,0.000000}%
\pgfsetstrokecolor{currentstroke}%
\pgfsetdash{}{0pt}%
\pgfpathmoveto{\pgfqpoint{0.676288in}{1.101156in}}%
\pgfpathlineto{\pgfqpoint{0.688724in}{1.103805in}}%
\pgfpathlineto{\pgfqpoint{0.689557in}{1.104136in}}%
\pgfpathlineto{\pgfqpoint{0.695456in}{1.116731in}}%
\pgfpathlineto{\pgfqpoint{0.696208in}{1.129658in}}%
\pgfpathlineto{\pgfqpoint{0.693045in}{1.142584in}}%
\pgfpathlineto{\pgfqpoint{0.689557in}{1.147231in}}%
\pgfpathlineto{\pgfqpoint{0.676288in}{1.150516in}}%
\pgfpathlineto{\pgfqpoint{0.664370in}{1.142584in}}%
\pgfpathlineto{\pgfqpoint{0.663019in}{1.139902in}}%
\pgfpathlineto{\pgfqpoint{0.660918in}{1.129658in}}%
\pgfpathlineto{\pgfqpoint{0.661583in}{1.116731in}}%
\pgfpathlineto{\pgfqpoint{0.663019in}{1.112280in}}%
\pgfpathlineto{\pgfqpoint{0.670742in}{1.103805in}}%
\pgfpathclose%
\pgfpathmoveto{\pgfqpoint{0.673210in}{1.116731in}}%
\pgfpathlineto{\pgfqpoint{0.671199in}{1.129658in}}%
\pgfpathlineto{\pgfqpoint{0.676288in}{1.138124in}}%
\pgfpathlineto{\pgfqpoint{0.687198in}{1.129658in}}%
\pgfpathlineto{\pgfqpoint{0.682920in}{1.116731in}}%
\pgfpathlineto{\pgfqpoint{0.676288in}{1.113473in}}%
\pgfpathclose%
\pgfusepath{fill}%
\end{pgfscope}%
\begin{pgfscope}%
\pgfpathrectangle{\pgfqpoint{0.211875in}{0.211875in}}{\pgfqpoint{1.313625in}{1.279725in}}%
\pgfusepath{clip}%
\pgfsetbuttcap%
\pgfsetroundjoin%
\definecolor{currentfill}{rgb}{0.965440,0.720101,0.576404}%
\pgfsetfillcolor{currentfill}%
\pgfsetlinewidth{0.000000pt}%
\definecolor{currentstroke}{rgb}{0.000000,0.000000,0.000000}%
\pgfsetstrokecolor{currentstroke}%
\pgfsetdash{}{0pt}%
\pgfpathmoveto{\pgfqpoint{0.795708in}{1.099000in}}%
\pgfpathlineto{\pgfqpoint{0.808977in}{1.102652in}}%
\pgfpathlineto{\pgfqpoint{0.810060in}{1.103805in}}%
\pgfpathlineto{\pgfqpoint{0.815115in}{1.116731in}}%
\pgfpathlineto{\pgfqpoint{0.815738in}{1.129658in}}%
\pgfpathlineto{\pgfqpoint{0.813122in}{1.142584in}}%
\pgfpathlineto{\pgfqpoint{0.808977in}{1.148797in}}%
\pgfpathlineto{\pgfqpoint{0.795708in}{1.152992in}}%
\pgfpathlineto{\pgfqpoint{0.782439in}{1.146754in}}%
\pgfpathlineto{\pgfqpoint{0.780019in}{1.142584in}}%
\pgfpathlineto{\pgfqpoint{0.777486in}{1.129658in}}%
\pgfpathlineto{\pgfqpoint{0.778092in}{1.116731in}}%
\pgfpathlineto{\pgfqpoint{0.782439in}{1.104787in}}%
\pgfpathlineto{\pgfqpoint{0.783600in}{1.103805in}}%
\pgfpathclose%
\pgfpathmoveto{\pgfqpoint{0.788068in}{1.116731in}}%
\pgfpathlineto{\pgfqpoint{0.785858in}{1.129658in}}%
\pgfpathlineto{\pgfqpoint{0.795079in}{1.142584in}}%
\pgfpathlineto{\pgfqpoint{0.795708in}{1.142932in}}%
\pgfpathlineto{\pgfqpoint{0.796545in}{1.142584in}}%
\pgfpathlineto{\pgfqpoint{0.808626in}{1.129658in}}%
\pgfpathlineto{\pgfqpoint{0.805760in}{1.116731in}}%
\pgfpathlineto{\pgfqpoint{0.795708in}{1.110005in}}%
\pgfpathclose%
\pgfusepath{fill}%
\end{pgfscope}%
\begin{pgfscope}%
\pgfpathrectangle{\pgfqpoint{0.211875in}{0.211875in}}{\pgfqpoint{1.313625in}{1.279725in}}%
\pgfusepath{clip}%
\pgfsetbuttcap%
\pgfsetroundjoin%
\definecolor{currentfill}{rgb}{0.965440,0.720101,0.576404}%
\pgfsetfillcolor{currentfill}%
\pgfsetlinewidth{0.000000pt}%
\definecolor{currentstroke}{rgb}{0.000000,0.000000,0.000000}%
\pgfsetstrokecolor{currentstroke}%
\pgfsetdash{}{0pt}%
\pgfpathmoveto{\pgfqpoint{0.901860in}{1.101186in}}%
\pgfpathlineto{\pgfqpoint{0.915129in}{1.097542in}}%
\pgfpathlineto{\pgfqpoint{0.928398in}{1.102034in}}%
\pgfpathlineto{\pgfqpoint{0.929873in}{1.103805in}}%
\pgfpathlineto{\pgfqpoint{0.934210in}{1.116731in}}%
\pgfpathlineto{\pgfqpoint{0.934746in}{1.129658in}}%
\pgfpathlineto{\pgfqpoint{0.932498in}{1.142584in}}%
\pgfpathlineto{\pgfqpoint{0.928398in}{1.149508in}}%
\pgfpathlineto{\pgfqpoint{0.915129in}{1.154666in}}%
\pgfpathlineto{\pgfqpoint{0.901860in}{1.150482in}}%
\pgfpathlineto{\pgfqpoint{0.896702in}{1.142584in}}%
\pgfpathlineto{\pgfqpoint{0.894319in}{1.129658in}}%
\pgfpathlineto{\pgfqpoint{0.894888in}{1.116731in}}%
\pgfpathlineto{\pgfqpoint{0.899463in}{1.103805in}}%
\pgfpathclose%
\pgfpathmoveto{\pgfqpoint{0.902323in}{1.116731in}}%
\pgfpathlineto{\pgfqpoint{0.901860in}{1.118941in}}%
\pgfpathlineto{\pgfqpoint{0.901081in}{1.129658in}}%
\pgfpathlineto{\pgfqpoint{0.901860in}{1.132651in}}%
\pgfpathlineto{\pgfqpoint{0.910554in}{1.142584in}}%
\pgfpathlineto{\pgfqpoint{0.915129in}{1.144606in}}%
\pgfpathlineto{\pgfqpoint{0.918961in}{1.142584in}}%
\pgfpathlineto{\pgfqpoint{0.927913in}{1.129658in}}%
\pgfpathlineto{\pgfqpoint{0.925783in}{1.116731in}}%
\pgfpathlineto{\pgfqpoint{0.915129in}{1.107719in}}%
\pgfpathclose%
\pgfusepath{fill}%
\end{pgfscope}%
\begin{pgfscope}%
\pgfpathrectangle{\pgfqpoint{0.211875in}{0.211875in}}{\pgfqpoint{1.313625in}{1.279725in}}%
\pgfusepath{clip}%
\pgfsetbuttcap%
\pgfsetroundjoin%
\definecolor{currentfill}{rgb}{0.965440,0.720101,0.576404}%
\pgfsetfillcolor{currentfill}%
\pgfsetlinewidth{0.000000pt}%
\definecolor{currentstroke}{rgb}{0.000000,0.000000,0.000000}%
\pgfsetstrokecolor{currentstroke}%
\pgfsetdash{}{0pt}%
\pgfpathmoveto{\pgfqpoint{1.021280in}{1.099135in}}%
\pgfpathlineto{\pgfqpoint{1.034549in}{1.096753in}}%
\pgfpathlineto{\pgfqpoint{1.047818in}{1.102310in}}%
\pgfpathlineto{\pgfqpoint{1.048919in}{1.103805in}}%
\pgfpathlineto{\pgfqpoint{1.052822in}{1.116731in}}%
\pgfpathlineto{\pgfqpoint{1.053305in}{1.129658in}}%
\pgfpathlineto{\pgfqpoint{1.051280in}{1.142584in}}%
\pgfpathlineto{\pgfqpoint{1.047818in}{1.149190in}}%
\pgfpathlineto{\pgfqpoint{1.034738in}{1.155511in}}%
\pgfpathlineto{\pgfqpoint{1.034549in}{1.155560in}}%
\pgfpathlineto{\pgfqpoint{1.034192in}{1.155511in}}%
\pgfpathlineto{\pgfqpoint{1.021280in}{1.152837in}}%
\pgfpathlineto{\pgfqpoint{1.013756in}{1.142584in}}%
\pgfpathlineto{\pgfqpoint{1.011414in}{1.129658in}}%
\pgfpathlineto{\pgfqpoint{1.011974in}{1.116731in}}%
\pgfpathlineto{\pgfqpoint{1.016474in}{1.103805in}}%
\pgfpathclose%
\pgfpathmoveto{\pgfqpoint{1.019540in}{1.116731in}}%
\pgfpathlineto{\pgfqpoint{1.018578in}{1.129658in}}%
\pgfpathlineto{\pgfqpoint{1.021280in}{1.138911in}}%
\pgfpathlineto{\pgfqpoint{1.025875in}{1.142584in}}%
\pgfpathlineto{\pgfqpoint{1.034549in}{1.145446in}}%
\pgfpathlineto{\pgfqpoint{1.039024in}{1.142584in}}%
\pgfpathlineto{\pgfqpoint{1.046180in}{1.129658in}}%
\pgfpathlineto{\pgfqpoint{1.044474in}{1.116731in}}%
\pgfpathlineto{\pgfqpoint{1.034549in}{1.106572in}}%
\pgfpathlineto{\pgfqpoint{1.021280in}{1.112945in}}%
\pgfpathclose%
\pgfusepath{fill}%
\end{pgfscope}%
\begin{pgfscope}%
\pgfpathrectangle{\pgfqpoint{0.211875in}{0.211875in}}{\pgfqpoint{1.313625in}{1.279725in}}%
\pgfusepath{clip}%
\pgfsetbuttcap%
\pgfsetroundjoin%
\definecolor{currentfill}{rgb}{0.965440,0.720101,0.576404}%
\pgfsetfillcolor{currentfill}%
\pgfsetlinewidth{0.000000pt}%
\definecolor{currentstroke}{rgb}{0.000000,0.000000,0.000000}%
\pgfsetstrokecolor{currentstroke}%
\pgfsetdash{}{0pt}%
\pgfpathmoveto{\pgfqpoint{1.140701in}{1.098056in}}%
\pgfpathlineto{\pgfqpoint{1.153970in}{1.096630in}}%
\pgfpathlineto{\pgfqpoint{1.167239in}{1.103725in}}%
\pgfpathlineto{\pgfqpoint{1.167290in}{1.103805in}}%
\pgfpathlineto{\pgfqpoint{1.171001in}{1.116731in}}%
\pgfpathlineto{\pgfqpoint{1.171461in}{1.129658in}}%
\pgfpathlineto{\pgfqpoint{1.169533in}{1.142584in}}%
\pgfpathlineto{\pgfqpoint{1.167239in}{1.147565in}}%
\pgfpathlineto{\pgfqpoint{1.154490in}{1.155511in}}%
\pgfpathlineto{\pgfqpoint{1.153970in}{1.155671in}}%
\pgfpathlineto{\pgfqpoint{1.152153in}{1.155511in}}%
\pgfpathlineto{\pgfqpoint{1.140701in}{1.154076in}}%
\pgfpathlineto{\pgfqpoint{1.131212in}{1.142584in}}%
\pgfpathlineto{\pgfqpoint{1.128785in}{1.129658in}}%
\pgfpathlineto{\pgfqpoint{1.129364in}{1.116731in}}%
\pgfpathlineto{\pgfqpoint{1.134037in}{1.103805in}}%
\pgfpathclose%
\pgfpathmoveto{\pgfqpoint{1.137465in}{1.116731in}}%
\pgfpathlineto{\pgfqpoint{1.136453in}{1.129658in}}%
\pgfpathlineto{\pgfqpoint{1.140694in}{1.142584in}}%
\pgfpathlineto{\pgfqpoint{1.140701in}{1.142592in}}%
\pgfpathlineto{\pgfqpoint{1.153970in}{1.145454in}}%
\pgfpathlineto{\pgfqpoint{1.157776in}{1.142584in}}%
\pgfpathlineto{\pgfqpoint{1.163863in}{1.129658in}}%
\pgfpathlineto{\pgfqpoint{1.162410in}{1.116731in}}%
\pgfpathlineto{\pgfqpoint{1.153970in}{1.106562in}}%
\pgfpathlineto{\pgfqpoint{1.140701in}{1.110470in}}%
\pgfpathclose%
\pgfusepath{fill}%
\end{pgfscope}%
\begin{pgfscope}%
\pgfpathrectangle{\pgfqpoint{0.211875in}{0.211875in}}{\pgfqpoint{1.313625in}{1.279725in}}%
\pgfusepath{clip}%
\pgfsetbuttcap%
\pgfsetroundjoin%
\definecolor{currentfill}{rgb}{0.965440,0.720101,0.576404}%
\pgfsetfillcolor{currentfill}%
\pgfsetlinewidth{0.000000pt}%
\definecolor{currentstroke}{rgb}{0.000000,0.000000,0.000000}%
\pgfsetstrokecolor{currentstroke}%
\pgfsetdash{}{0pt}%
\pgfpathmoveto{\pgfqpoint{1.260121in}{1.097800in}}%
\pgfpathlineto{\pgfqpoint{1.273390in}{1.097196in}}%
\pgfpathlineto{\pgfqpoint{1.284015in}{1.103805in}}%
\pgfpathlineto{\pgfqpoint{1.286659in}{1.108312in}}%
\pgfpathlineto{\pgfqpoint{1.288774in}{1.116731in}}%
\pgfpathlineto{\pgfqpoint{1.289237in}{1.129658in}}%
\pgfpathlineto{\pgfqpoint{1.287296in}{1.142584in}}%
\pgfpathlineto{\pgfqpoint{1.286659in}{1.144172in}}%
\pgfpathlineto{\pgfqpoint{1.273390in}{1.155063in}}%
\pgfpathlineto{\pgfqpoint{1.260121in}{1.154369in}}%
\pgfpathlineto{\pgfqpoint{1.249133in}{1.142584in}}%
\pgfpathlineto{\pgfqpoint{1.246852in}{1.131927in}}%
\pgfpathlineto{\pgfqpoint{1.246576in}{1.129658in}}%
\pgfpathlineto{\pgfqpoint{1.246852in}{1.121532in}}%
\pgfpathlineto{\pgfqpoint{1.247097in}{1.116731in}}%
\pgfpathlineto{\pgfqpoint{1.252252in}{1.103805in}}%
\pgfpathclose%
\pgfpathmoveto{\pgfqpoint{1.255876in}{1.116731in}}%
\pgfpathlineto{\pgfqpoint{1.254767in}{1.129658in}}%
\pgfpathlineto{\pgfqpoint{1.259417in}{1.142584in}}%
\pgfpathlineto{\pgfqpoint{1.260121in}{1.143339in}}%
\pgfpathlineto{\pgfqpoint{1.273390in}{1.144597in}}%
\pgfpathlineto{\pgfqpoint{1.275699in}{1.142584in}}%
\pgfpathlineto{\pgfqpoint{1.281159in}{1.129658in}}%
\pgfpathlineto{\pgfqpoint{1.279855in}{1.116731in}}%
\pgfpathlineto{\pgfqpoint{1.273390in}{1.107732in}}%
\pgfpathlineto{\pgfqpoint{1.260121in}{1.109449in}}%
\pgfpathclose%
\pgfusepath{fill}%
\end{pgfscope}%
\begin{pgfscope}%
\pgfpathrectangle{\pgfqpoint{0.211875in}{0.211875in}}{\pgfqpoint{1.313625in}{1.279725in}}%
\pgfusepath{clip}%
\pgfsetbuttcap%
\pgfsetroundjoin%
\definecolor{currentfill}{rgb}{0.965440,0.720101,0.576404}%
\pgfsetfillcolor{currentfill}%
\pgfsetlinewidth{0.000000pt}%
\definecolor{currentstroke}{rgb}{0.000000,0.000000,0.000000}%
\pgfsetstrokecolor{currentstroke}%
\pgfsetdash{}{0pt}%
\pgfpathmoveto{\pgfqpoint{1.379542in}{1.098272in}}%
\pgfpathlineto{\pgfqpoint{1.392811in}{1.098505in}}%
\pgfpathlineto{\pgfqpoint{1.400263in}{1.103805in}}%
\pgfpathlineto{\pgfqpoint{1.406080in}{1.116418in}}%
\pgfpathlineto{\pgfqpoint{1.406147in}{1.116731in}}%
\pgfpathlineto{\pgfqpoint{1.406638in}{1.129658in}}%
\pgfpathlineto{\pgfqpoint{1.406080in}{1.133734in}}%
\pgfpathlineto{\pgfqpoint{1.403834in}{1.142584in}}%
\pgfpathlineto{\pgfqpoint{1.392811in}{1.153560in}}%
\pgfpathlineto{\pgfqpoint{1.379542in}{1.153828in}}%
\pgfpathlineto{\pgfqpoint{1.367628in}{1.142584in}}%
\pgfpathlineto{\pgfqpoint{1.366273in}{1.137654in}}%
\pgfpathlineto{\pgfqpoint{1.365122in}{1.129658in}}%
\pgfpathlineto{\pgfqpoint{1.365606in}{1.116731in}}%
\pgfpathlineto{\pgfqpoint{1.366273in}{1.113789in}}%
\pgfpathlineto{\pgfqpoint{1.371288in}{1.103805in}}%
\pgfpathclose%
\pgfpathmoveto{\pgfqpoint{1.374893in}{1.116731in}}%
\pgfpathlineto{\pgfqpoint{1.373627in}{1.129658in}}%
\pgfpathlineto{\pgfqpoint{1.378941in}{1.142584in}}%
\pgfpathlineto{\pgfqpoint{1.379542in}{1.143151in}}%
\pgfpathlineto{\pgfqpoint{1.392811in}{1.142809in}}%
\pgfpathlineto{\pgfqpoint{1.393036in}{1.142584in}}%
\pgfpathlineto{\pgfqpoint{1.398165in}{1.129658in}}%
\pgfpathlineto{\pgfqpoint{1.396939in}{1.116731in}}%
\pgfpathlineto{\pgfqpoint{1.392811in}{1.110173in}}%
\pgfpathlineto{\pgfqpoint{1.379542in}{1.109707in}}%
\pgfpathclose%
\pgfusepath{fill}%
\end{pgfscope}%
\begin{pgfscope}%
\pgfpathrectangle{\pgfqpoint{0.211875in}{0.211875in}}{\pgfqpoint{1.313625in}{1.279725in}}%
\pgfusepath{clip}%
\pgfsetbuttcap%
\pgfsetroundjoin%
\definecolor{currentfill}{rgb}{0.965440,0.720101,0.576404}%
\pgfsetfillcolor{currentfill}%
\pgfsetlinewidth{0.000000pt}%
\definecolor{currentstroke}{rgb}{0.000000,0.000000,0.000000}%
\pgfsetstrokecolor{currentstroke}%
\pgfsetdash{}{0pt}%
\pgfpathmoveto{\pgfqpoint{1.498962in}{1.099415in}}%
\pgfpathlineto{\pgfqpoint{1.512231in}{1.100643in}}%
\pgfpathlineto{\pgfqpoint{1.516155in}{1.103805in}}%
\pgfpathlineto{\pgfqpoint{1.522183in}{1.116731in}}%
\pgfpathlineto{\pgfqpoint{1.522935in}{1.129658in}}%
\pgfpathlineto{\pgfqpoint{1.519790in}{1.142584in}}%
\pgfpathlineto{\pgfqpoint{1.512231in}{1.151105in}}%
\pgfpathlineto{\pgfqpoint{1.498962in}{1.152515in}}%
\pgfpathlineto{\pgfqpoint{1.486882in}{1.142584in}}%
\pgfpathlineto{\pgfqpoint{1.485693in}{1.139187in}}%
\pgfpathlineto{\pgfqpoint{1.484101in}{1.129658in}}%
\pgfpathlineto{\pgfqpoint{1.484648in}{1.116731in}}%
\pgfpathlineto{\pgfqpoint{1.485693in}{1.112760in}}%
\pgfpathlineto{\pgfqpoint{1.491429in}{1.103805in}}%
\pgfpathclose%
\pgfpathmoveto{\pgfqpoint{1.494716in}{1.116731in}}%
\pgfpathlineto{\pgfqpoint{1.493208in}{1.129658in}}%
\pgfpathlineto{\pgfqpoint{1.498962in}{1.141608in}}%
\pgfpathlineto{\pgfqpoint{1.512231in}{1.137283in}}%
\pgfpathlineto{\pgfqpoint{1.514928in}{1.129658in}}%
\pgfpathlineto{\pgfqpoint{1.513730in}{1.116731in}}%
\pgfpathlineto{\pgfqpoint{1.512231in}{1.114037in}}%
\pgfpathlineto{\pgfqpoint{1.498962in}{1.111135in}}%
\pgfpathclose%
\pgfusepath{fill}%
\end{pgfscope}%
\begin{pgfscope}%
\pgfpathrectangle{\pgfqpoint{0.211875in}{0.211875in}}{\pgfqpoint{1.313625in}{1.279725in}}%
\pgfusepath{clip}%
\pgfsetbuttcap%
\pgfsetroundjoin%
\definecolor{currentfill}{rgb}{0.965440,0.720101,0.576404}%
\pgfsetfillcolor{currentfill}%
\pgfsetlinewidth{0.000000pt}%
\definecolor{currentstroke}{rgb}{0.000000,0.000000,0.000000}%
\pgfsetstrokecolor{currentstroke}%
\pgfsetdash{}{0pt}%
\pgfpathmoveto{\pgfqpoint{0.331295in}{1.116473in}}%
\pgfpathlineto{\pgfqpoint{0.331470in}{1.116731in}}%
\pgfpathlineto{\pgfqpoint{0.333013in}{1.129658in}}%
\pgfpathlineto{\pgfqpoint{0.331295in}{1.133653in}}%
\pgfpathlineto{\pgfqpoint{0.318027in}{1.132162in}}%
\pgfpathlineto{\pgfqpoint{0.317046in}{1.129658in}}%
\pgfpathlineto{\pgfqpoint{0.318027in}{1.120691in}}%
\pgfpathlineto{\pgfqpoint{0.327858in}{1.116731in}}%
\pgfpathclose%
\pgfusepath{fill}%
\end{pgfscope}%
\begin{pgfscope}%
\pgfpathrectangle{\pgfqpoint{0.211875in}{0.211875in}}{\pgfqpoint{1.313625in}{1.279725in}}%
\pgfusepath{clip}%
\pgfsetbuttcap%
\pgfsetroundjoin%
\definecolor{currentfill}{rgb}{0.965440,0.720101,0.576404}%
\pgfsetfillcolor{currentfill}%
\pgfsetlinewidth{0.000000pt}%
\definecolor{currentstroke}{rgb}{0.000000,0.000000,0.000000}%
\pgfsetstrokecolor{currentstroke}%
\pgfsetdash{}{0pt}%
\pgfpathmoveto{\pgfqpoint{0.437447in}{1.110087in}}%
\pgfpathlineto{\pgfqpoint{0.450716in}{1.111424in}}%
\pgfpathlineto{\pgfqpoint{0.453879in}{1.116731in}}%
\pgfpathlineto{\pgfqpoint{0.455070in}{1.129658in}}%
\pgfpathlineto{\pgfqpoint{0.450716in}{1.141178in}}%
\pgfpathlineto{\pgfqpoint{0.441501in}{1.142584in}}%
\pgfpathlineto{\pgfqpoint{0.437447in}{1.142872in}}%
\pgfpathlineto{\pgfqpoint{0.437126in}{1.142584in}}%
\pgfpathlineto{\pgfqpoint{0.431425in}{1.129658in}}%
\pgfpathlineto{\pgfqpoint{0.432794in}{1.116731in}}%
\pgfpathclose%
\pgfusepath{fill}%
\end{pgfscope}%
\begin{pgfscope}%
\pgfpathrectangle{\pgfqpoint{0.211875in}{0.211875in}}{\pgfqpoint{1.313625in}{1.279725in}}%
\pgfusepath{clip}%
\pgfsetbuttcap%
\pgfsetroundjoin%
\definecolor{currentfill}{rgb}{0.965440,0.720101,0.576404}%
\pgfsetfillcolor{currentfill}%
\pgfsetlinewidth{0.000000pt}%
\definecolor{currentstroke}{rgb}{0.000000,0.000000,0.000000}%
\pgfsetstrokecolor{currentstroke}%
\pgfsetdash{}{0pt}%
\pgfpathmoveto{\pgfqpoint{0.556867in}{1.104211in}}%
\pgfpathlineto{\pgfqpoint{0.570136in}{1.107303in}}%
\pgfpathlineto{\pgfqpoint{0.575108in}{1.116731in}}%
\pgfpathlineto{\pgfqpoint{0.576044in}{1.129658in}}%
\pgfpathlineto{\pgfqpoint{0.572105in}{1.142584in}}%
\pgfpathlineto{\pgfqpoint{0.570136in}{1.144911in}}%
\pgfpathlineto{\pgfqpoint{0.556867in}{1.147176in}}%
\pgfpathlineto{\pgfqpoint{0.550981in}{1.142584in}}%
\pgfpathlineto{\pgfqpoint{0.545441in}{1.129658in}}%
\pgfpathlineto{\pgfqpoint{0.546770in}{1.116731in}}%
\pgfpathclose%
\pgfpathmoveto{\pgfqpoint{0.556149in}{1.129658in}}%
\pgfpathlineto{\pgfqpoint{0.556867in}{1.131058in}}%
\pgfpathlineto{\pgfqpoint{0.559721in}{1.129658in}}%
\pgfpathlineto{\pgfqpoint{0.556867in}{1.124644in}}%
\pgfpathclose%
\pgfusepath{fill}%
\end{pgfscope}%
\begin{pgfscope}%
\pgfpathrectangle{\pgfqpoint{0.211875in}{0.211875in}}{\pgfqpoint{1.313625in}{1.279725in}}%
\pgfusepath{clip}%
\pgfsetbuttcap%
\pgfsetroundjoin%
\definecolor{currentfill}{rgb}{0.965440,0.720101,0.576404}%
\pgfsetfillcolor{currentfill}%
\pgfsetlinewidth{0.000000pt}%
\definecolor{currentstroke}{rgb}{0.000000,0.000000,0.000000}%
\pgfsetstrokecolor{currentstroke}%
\pgfsetdash{}{0pt}%
\pgfpathmoveto{\pgfqpoint{0.742633in}{1.181338in}}%
\pgfpathlineto{\pgfqpoint{0.742688in}{1.181364in}}%
\pgfpathlineto{\pgfqpoint{0.754934in}{1.194290in}}%
\pgfpathlineto{\pgfqpoint{0.755902in}{1.199255in}}%
\pgfpathlineto{\pgfqpoint{0.756716in}{1.207217in}}%
\pgfpathlineto{\pgfqpoint{0.755902in}{1.214754in}}%
\pgfpathlineto{\pgfqpoint{0.754791in}{1.220143in}}%
\pgfpathlineto{\pgfqpoint{0.742633in}{1.232785in}}%
\pgfpathlineto{\pgfqpoint{0.729364in}{1.231197in}}%
\pgfpathlineto{\pgfqpoint{0.721374in}{1.220143in}}%
\pgfpathlineto{\pgfqpoint{0.719324in}{1.207217in}}%
\pgfpathlineto{\pgfqpoint{0.721259in}{1.194290in}}%
\pgfpathlineto{\pgfqpoint{0.729364in}{1.182922in}}%
\pgfpathlineto{\pgfqpoint{0.742321in}{1.181364in}}%
\pgfpathclose%
\pgfpathmoveto{\pgfqpoint{0.736966in}{1.194290in}}%
\pgfpathlineto{\pgfqpoint{0.729364in}{1.198097in}}%
\pgfpathlineto{\pgfqpoint{0.727313in}{1.207217in}}%
\pgfpathlineto{\pgfqpoint{0.729364in}{1.216006in}}%
\pgfpathlineto{\pgfqpoint{0.737945in}{1.220143in}}%
\pgfpathlineto{\pgfqpoint{0.742633in}{1.220926in}}%
\pgfpathlineto{\pgfqpoint{0.743386in}{1.220143in}}%
\pgfpathlineto{\pgfqpoint{0.747284in}{1.207217in}}%
\pgfpathlineto{\pgfqpoint{0.743543in}{1.194290in}}%
\pgfpathlineto{\pgfqpoint{0.742633in}{1.193331in}}%
\pgfpathclose%
\pgfusepath{fill}%
\end{pgfscope}%
\begin{pgfscope}%
\pgfpathrectangle{\pgfqpoint{0.211875in}{0.211875in}}{\pgfqpoint{1.313625in}{1.279725in}}%
\pgfusepath{clip}%
\pgfsetbuttcap%
\pgfsetroundjoin%
\definecolor{currentfill}{rgb}{0.965440,0.720101,0.576404}%
\pgfsetfillcolor{currentfill}%
\pgfsetlinewidth{0.000000pt}%
\definecolor{currentstroke}{rgb}{0.000000,0.000000,0.000000}%
\pgfsetstrokecolor{currentstroke}%
\pgfsetdash{}{0pt}%
\pgfpathmoveto{\pgfqpoint{0.848784in}{1.180152in}}%
\pgfpathlineto{\pgfqpoint{0.862053in}{1.179998in}}%
\pgfpathlineto{\pgfqpoint{0.864635in}{1.181364in}}%
\pgfpathlineto{\pgfqpoint{0.874336in}{1.194290in}}%
\pgfpathlineto{\pgfqpoint{0.875322in}{1.200752in}}%
\pgfpathlineto{\pgfqpoint{0.875889in}{1.207217in}}%
\pgfpathlineto{\pgfqpoint{0.875322in}{1.213298in}}%
\pgfpathlineto{\pgfqpoint{0.874210in}{1.220143in}}%
\pgfpathlineto{\pgfqpoint{0.864196in}{1.233070in}}%
\pgfpathlineto{\pgfqpoint{0.862053in}{1.234190in}}%
\pgfpathlineto{\pgfqpoint{0.848784in}{1.234037in}}%
\pgfpathlineto{\pgfqpoint{0.847011in}{1.233070in}}%
\pgfpathlineto{\pgfqpoint{0.837106in}{1.220143in}}%
\pgfpathlineto{\pgfqpoint{0.835515in}{1.210122in}}%
\pgfpathlineto{\pgfqpoint{0.835248in}{1.207217in}}%
\pgfpathlineto{\pgfqpoint{0.835515in}{1.204129in}}%
\pgfpathlineto{\pgfqpoint{0.836982in}{1.194290in}}%
\pgfpathlineto{\pgfqpoint{0.846590in}{1.181364in}}%
\pgfpathclose%
\pgfpathmoveto{\pgfqpoint{0.846925in}{1.194290in}}%
\pgfpathlineto{\pgfqpoint{0.843856in}{1.207217in}}%
\pgfpathlineto{\pgfqpoint{0.847060in}{1.220143in}}%
\pgfpathlineto{\pgfqpoint{0.848784in}{1.222250in}}%
\pgfpathlineto{\pgfqpoint{0.862053in}{1.222455in}}%
\pgfpathlineto{\pgfqpoint{0.863994in}{1.220143in}}%
\pgfpathlineto{\pgfqpoint{0.867241in}{1.207217in}}%
\pgfpathlineto{\pgfqpoint{0.864132in}{1.194290in}}%
\pgfpathlineto{\pgfqpoint{0.862053in}{1.191780in}}%
\pgfpathlineto{\pgfqpoint{0.848784in}{1.191987in}}%
\pgfpathclose%
\pgfusepath{fill}%
\end{pgfscope}%
\begin{pgfscope}%
\pgfpathrectangle{\pgfqpoint{0.211875in}{0.211875in}}{\pgfqpoint{1.313625in}{1.279725in}}%
\pgfusepath{clip}%
\pgfsetbuttcap%
\pgfsetroundjoin%
\definecolor{currentfill}{rgb}{0.965440,0.720101,0.576404}%
\pgfsetfillcolor{currentfill}%
\pgfsetlinewidth{0.000000pt}%
\definecolor{currentstroke}{rgb}{0.000000,0.000000,0.000000}%
\pgfsetstrokecolor{currentstroke}%
\pgfsetdash{}{0pt}%
\pgfpathmoveto{\pgfqpoint{0.968205in}{1.178626in}}%
\pgfpathlineto{\pgfqpoint{0.981473in}{1.179282in}}%
\pgfpathlineto{\pgfqpoint{0.984927in}{1.181364in}}%
\pgfpathlineto{\pgfqpoint{0.993023in}{1.194290in}}%
\pgfpathlineto{\pgfqpoint{0.994572in}{1.207217in}}%
\pgfpathlineto{\pgfqpoint{0.992909in}{1.220143in}}%
\pgfpathlineto{\pgfqpoint{0.984535in}{1.233070in}}%
\pgfpathlineto{\pgfqpoint{0.981473in}{1.234894in}}%
\pgfpathlineto{\pgfqpoint{0.968205in}{1.235557in}}%
\pgfpathlineto{\pgfqpoint{0.962984in}{1.233070in}}%
\pgfpathlineto{\pgfqpoint{0.954936in}{1.223907in}}%
\pgfpathlineto{\pgfqpoint{0.953698in}{1.220143in}}%
\pgfpathlineto{\pgfqpoint{0.952472in}{1.207217in}}%
\pgfpathlineto{\pgfqpoint{0.953617in}{1.194290in}}%
\pgfpathlineto{\pgfqpoint{0.954936in}{1.190196in}}%
\pgfpathlineto{\pgfqpoint{0.962521in}{1.181364in}}%
\pgfpathclose%
\pgfpathmoveto{\pgfqpoint{0.963822in}{1.194290in}}%
\pgfpathlineto{\pgfqpoint{0.960591in}{1.207217in}}%
\pgfpathlineto{\pgfqpoint{0.963971in}{1.220143in}}%
\pgfpathlineto{\pgfqpoint{0.968205in}{1.224672in}}%
\pgfpathlineto{\pgfqpoint{0.981473in}{1.223020in}}%
\pgfpathlineto{\pgfqpoint{0.983602in}{1.220143in}}%
\pgfpathlineto{\pgfqpoint{0.986416in}{1.207217in}}%
\pgfpathlineto{\pgfqpoint{0.983726in}{1.194290in}}%
\pgfpathlineto{\pgfqpoint{0.981473in}{1.191202in}}%
\pgfpathlineto{\pgfqpoint{0.968205in}{1.189540in}}%
\pgfpathclose%
\pgfusepath{fill}%
\end{pgfscope}%
\begin{pgfscope}%
\pgfpathrectangle{\pgfqpoint{0.211875in}{0.211875in}}{\pgfqpoint{1.313625in}{1.279725in}}%
\pgfusepath{clip}%
\pgfsetbuttcap%
\pgfsetroundjoin%
\definecolor{currentfill}{rgb}{0.965440,0.720101,0.576404}%
\pgfsetfillcolor{currentfill}%
\pgfsetlinewidth{0.000000pt}%
\definecolor{currentstroke}{rgb}{0.000000,0.000000,0.000000}%
\pgfsetstrokecolor{currentstroke}%
\pgfsetdash{}{0pt}%
\pgfpathmoveto{\pgfqpoint{1.087625in}{1.177814in}}%
\pgfpathlineto{\pgfqpoint{1.100894in}{1.179239in}}%
\pgfpathlineto{\pgfqpoint{1.104007in}{1.181364in}}%
\pgfpathlineto{\pgfqpoint{1.111168in}{1.194290in}}%
\pgfpathlineto{\pgfqpoint{1.112543in}{1.207217in}}%
\pgfpathlineto{\pgfqpoint{1.111063in}{1.220143in}}%
\pgfpathlineto{\pgfqpoint{1.103649in}{1.233070in}}%
\pgfpathlineto{\pgfqpoint{1.100894in}{1.234928in}}%
\pgfpathlineto{\pgfqpoint{1.087625in}{1.236367in}}%
\pgfpathlineto{\pgfqpoint{1.079585in}{1.233070in}}%
\pgfpathlineto{\pgfqpoint{1.074356in}{1.228387in}}%
\pgfpathlineto{\pgfqpoint{1.071252in}{1.220143in}}%
\pgfpathlineto{\pgfqpoint{1.070058in}{1.207217in}}%
\pgfpathlineto{\pgfqpoint{1.071169in}{1.194290in}}%
\pgfpathlineto{\pgfqpoint{1.074356in}{1.185667in}}%
\pgfpathlineto{\pgfqpoint{1.079063in}{1.181364in}}%
\pgfpathclose%
\pgfpathmoveto{\pgfqpoint{1.081098in}{1.194290in}}%
\pgfpathlineto{\pgfqpoint{1.077540in}{1.207217in}}%
\pgfpathlineto{\pgfqpoint{1.081266in}{1.220143in}}%
\pgfpathlineto{\pgfqpoint{1.087625in}{1.226020in}}%
\pgfpathlineto{\pgfqpoint{1.100894in}{1.222538in}}%
\pgfpathlineto{\pgfqpoint{1.102465in}{1.220143in}}%
\pgfpathlineto{\pgfqpoint{1.105002in}{1.207217in}}%
\pgfpathlineto{\pgfqpoint{1.102579in}{1.194290in}}%
\pgfpathlineto{\pgfqpoint{1.100894in}{1.191683in}}%
\pgfpathlineto{\pgfqpoint{1.087625in}{1.188179in}}%
\pgfpathclose%
\pgfusepath{fill}%
\end{pgfscope}%
\begin{pgfscope}%
\pgfpathrectangle{\pgfqpoint{0.211875in}{0.211875in}}{\pgfqpoint{1.313625in}{1.279725in}}%
\pgfusepath{clip}%
\pgfsetbuttcap%
\pgfsetroundjoin%
\definecolor{currentfill}{rgb}{0.965440,0.720101,0.576404}%
\pgfsetfillcolor{currentfill}%
\pgfsetlinewidth{0.000000pt}%
\definecolor{currentstroke}{rgb}{0.000000,0.000000,0.000000}%
\pgfsetstrokecolor{currentstroke}%
\pgfsetdash{}{0pt}%
\pgfpathmoveto{\pgfqpoint{1.207045in}{1.177661in}}%
\pgfpathlineto{\pgfqpoint{1.220314in}{1.179955in}}%
\pgfpathlineto{\pgfqpoint{1.222145in}{1.181364in}}%
\pgfpathlineto{\pgfqpoint{1.228874in}{1.194290in}}%
\pgfpathlineto{\pgfqpoint{1.230171in}{1.207217in}}%
\pgfpathlineto{\pgfqpoint{1.228777in}{1.220143in}}%
\pgfpathlineto{\pgfqpoint{1.221811in}{1.233070in}}%
\pgfpathlineto{\pgfqpoint{1.220314in}{1.234207in}}%
\pgfpathlineto{\pgfqpoint{1.207045in}{1.236521in}}%
\pgfpathlineto{\pgfqpoint{1.197067in}{1.233070in}}%
\pgfpathlineto{\pgfqpoint{1.193777in}{1.230776in}}%
\pgfpathlineto{\pgfqpoint{1.189232in}{1.220143in}}%
\pgfpathlineto{\pgfqpoint{1.188007in}{1.207217in}}%
\pgfpathlineto{\pgfqpoint{1.189146in}{1.194290in}}%
\pgfpathlineto{\pgfqpoint{1.193777in}{1.183267in}}%
\pgfpathlineto{\pgfqpoint{1.196455in}{1.181364in}}%
\pgfpathclose%
\pgfpathmoveto{\pgfqpoint{1.198892in}{1.194290in}}%
\pgfpathlineto{\pgfqpoint{1.194757in}{1.207217in}}%
\pgfpathlineto{\pgfqpoint{1.199087in}{1.220143in}}%
\pgfpathlineto{\pgfqpoint{1.207045in}{1.226375in}}%
\pgfpathlineto{\pgfqpoint{1.220314in}{1.220870in}}%
\pgfpathlineto{\pgfqpoint{1.220739in}{1.220143in}}%
\pgfpathlineto{\pgfqpoint{1.223119in}{1.207217in}}%
\pgfpathlineto{\pgfqpoint{1.220846in}{1.194290in}}%
\pgfpathlineto{\pgfqpoint{1.220314in}{1.193366in}}%
\pgfpathlineto{\pgfqpoint{1.207045in}{1.187822in}}%
\pgfpathclose%
\pgfusepath{fill}%
\end{pgfscope}%
\begin{pgfscope}%
\pgfpathrectangle{\pgfqpoint{0.211875in}{0.211875in}}{\pgfqpoint{1.313625in}{1.279725in}}%
\pgfusepath{clip}%
\pgfsetbuttcap%
\pgfsetroundjoin%
\definecolor{currentfill}{rgb}{0.965440,0.720101,0.576404}%
\pgfsetfillcolor{currentfill}%
\pgfsetlinewidth{0.000000pt}%
\definecolor{currentstroke}{rgb}{0.000000,0.000000,0.000000}%
\pgfsetstrokecolor{currentstroke}%
\pgfsetdash{}{0pt}%
\pgfpathmoveto{\pgfqpoint{1.326466in}{1.178143in}}%
\pgfpathlineto{\pgfqpoint{1.339123in}{1.181364in}}%
\pgfpathlineto{\pgfqpoint{1.339735in}{1.181643in}}%
\pgfpathlineto{\pgfqpoint{1.346203in}{1.194290in}}%
\pgfpathlineto{\pgfqpoint{1.347500in}{1.207217in}}%
\pgfpathlineto{\pgfqpoint{1.346111in}{1.220143in}}%
\pgfpathlineto{\pgfqpoint{1.339735in}{1.232418in}}%
\pgfpathlineto{\pgfqpoint{1.338279in}{1.233070in}}%
\pgfpathlineto{\pgfqpoint{1.326466in}{1.236044in}}%
\pgfpathlineto{\pgfqpoint{1.315945in}{1.233070in}}%
\pgfpathlineto{\pgfqpoint{1.313197in}{1.231614in}}%
\pgfpathlineto{\pgfqpoint{1.307665in}{1.220143in}}%
\pgfpathlineto{\pgfqpoint{1.306338in}{1.207217in}}%
\pgfpathlineto{\pgfqpoint{1.307575in}{1.194290in}}%
\pgfpathlineto{\pgfqpoint{1.313197in}{1.182442in}}%
\pgfpathlineto{\pgfqpoint{1.315195in}{1.181364in}}%
\pgfpathclose%
\pgfpathmoveto{\pgfqpoint{1.317485in}{1.194290in}}%
\pgfpathlineto{\pgfqpoint{1.313197in}{1.204875in}}%
\pgfpathlineto{\pgfqpoint{1.312848in}{1.207217in}}%
\pgfpathlineto{\pgfqpoint{1.313197in}{1.209459in}}%
\pgfpathlineto{\pgfqpoint{1.317723in}{1.220143in}}%
\pgfpathlineto{\pgfqpoint{1.326466in}{1.225778in}}%
\pgfpathlineto{\pgfqpoint{1.336480in}{1.220143in}}%
\pgfpathlineto{\pgfqpoint{1.339735in}{1.213773in}}%
\pgfpathlineto{\pgfqpoint{1.340834in}{1.207217in}}%
\pgfpathlineto{\pgfqpoint{1.339735in}{1.200383in}}%
\pgfpathlineto{\pgfqpoint{1.336752in}{1.194290in}}%
\pgfpathlineto{\pgfqpoint{1.326466in}{1.188428in}}%
\pgfpathclose%
\pgfusepath{fill}%
\end{pgfscope}%
\begin{pgfscope}%
\pgfpathrectangle{\pgfqpoint{0.211875in}{0.211875in}}{\pgfqpoint{1.313625in}{1.279725in}}%
\pgfusepath{clip}%
\pgfsetbuttcap%
\pgfsetroundjoin%
\definecolor{currentfill}{rgb}{0.965440,0.720101,0.576404}%
\pgfsetfillcolor{currentfill}%
\pgfsetlinewidth{0.000000pt}%
\definecolor{currentstroke}{rgb}{0.000000,0.000000,0.000000}%
\pgfsetstrokecolor{currentstroke}%
\pgfsetdash{}{0pt}%
\pgfpathmoveto{\pgfqpoint{1.445886in}{1.179257in}}%
\pgfpathlineto{\pgfqpoint{1.452510in}{1.181364in}}%
\pgfpathlineto{\pgfqpoint{1.459155in}{1.185421in}}%
\pgfpathlineto{\pgfqpoint{1.463185in}{1.194290in}}%
\pgfpathlineto{\pgfqpoint{1.464546in}{1.207217in}}%
\pgfpathlineto{\pgfqpoint{1.463096in}{1.220143in}}%
\pgfpathlineto{\pgfqpoint{1.459155in}{1.228675in}}%
\pgfpathlineto{\pgfqpoint{1.451822in}{1.233070in}}%
\pgfpathlineto{\pgfqpoint{1.445886in}{1.234937in}}%
\pgfpathlineto{\pgfqpoint{1.437389in}{1.233070in}}%
\pgfpathlineto{\pgfqpoint{1.432617in}{1.231234in}}%
\pgfpathlineto{\pgfqpoint{1.426601in}{1.220143in}}%
\pgfpathlineto{\pgfqpoint{1.425088in}{1.207217in}}%
\pgfpathlineto{\pgfqpoint{1.426505in}{1.194290in}}%
\pgfpathlineto{\pgfqpoint{1.432617in}{1.182848in}}%
\pgfpathlineto{\pgfqpoint{1.436407in}{1.181364in}}%
\pgfpathclose%
\pgfpathmoveto{\pgfqpoint{1.437507in}{1.194290in}}%
\pgfpathlineto{\pgfqpoint{1.432617in}{1.203138in}}%
\pgfpathlineto{\pgfqpoint{1.431933in}{1.207217in}}%
\pgfpathlineto{\pgfqpoint{1.432617in}{1.211131in}}%
\pgfpathlineto{\pgfqpoint{1.437812in}{1.220143in}}%
\pgfpathlineto{\pgfqpoint{1.445886in}{1.224232in}}%
\pgfpathlineto{\pgfqpoint{1.451721in}{1.220143in}}%
\pgfpathlineto{\pgfqpoint{1.456974in}{1.207217in}}%
\pgfpathlineto{\pgfqpoint{1.451943in}{1.194290in}}%
\pgfpathlineto{\pgfqpoint{1.445886in}{1.189992in}}%
\pgfpathclose%
\pgfusepath{fill}%
\end{pgfscope}%
\begin{pgfscope}%
\pgfpathrectangle{\pgfqpoint{0.211875in}{0.211875in}}{\pgfqpoint{1.313625in}{1.279725in}}%
\pgfusepath{clip}%
\pgfsetbuttcap%
\pgfsetroundjoin%
\definecolor{currentfill}{rgb}{0.965440,0.720101,0.576404}%
\pgfsetfillcolor{currentfill}%
\pgfsetlinewidth{0.000000pt}%
\definecolor{currentstroke}{rgb}{0.000000,0.000000,0.000000}%
\pgfsetstrokecolor{currentstroke}%
\pgfsetdash{}{0pt}%
\pgfpathmoveto{\pgfqpoint{0.384371in}{1.192061in}}%
\pgfpathlineto{\pgfqpoint{0.387946in}{1.194290in}}%
\pgfpathlineto{\pgfqpoint{0.393964in}{1.207217in}}%
\pgfpathlineto{\pgfqpoint{0.387690in}{1.220143in}}%
\pgfpathlineto{\pgfqpoint{0.384371in}{1.222186in}}%
\pgfpathlineto{\pgfqpoint{0.380918in}{1.220143in}}%
\pgfpathlineto{\pgfqpoint{0.374221in}{1.207217in}}%
\pgfpathlineto{\pgfqpoint{0.380652in}{1.194290in}}%
\pgfpathclose%
\pgfusepath{fill}%
\end{pgfscope}%
\begin{pgfscope}%
\pgfpathrectangle{\pgfqpoint{0.211875in}{0.211875in}}{\pgfqpoint{1.313625in}{1.279725in}}%
\pgfusepath{clip}%
\pgfsetbuttcap%
\pgfsetroundjoin%
\definecolor{currentfill}{rgb}{0.965440,0.720101,0.576404}%
\pgfsetfillcolor{currentfill}%
\pgfsetlinewidth{0.000000pt}%
\definecolor{currentstroke}{rgb}{0.000000,0.000000,0.000000}%
\pgfsetstrokecolor{currentstroke}%
\pgfsetdash{}{0pt}%
\pgfpathmoveto{\pgfqpoint{0.490523in}{1.193578in}}%
\pgfpathlineto{\pgfqpoint{0.503792in}{1.187607in}}%
\pgfpathlineto{\pgfqpoint{0.512541in}{1.194290in}}%
\pgfpathlineto{\pgfqpoint{0.516734in}{1.207217in}}%
\pgfpathlineto{\pgfqpoint{0.512339in}{1.220143in}}%
\pgfpathlineto{\pgfqpoint{0.503792in}{1.226587in}}%
\pgfpathlineto{\pgfqpoint{0.490523in}{1.220661in}}%
\pgfpathlineto{\pgfqpoint{0.490226in}{1.220143in}}%
\pgfpathlineto{\pgfqpoint{0.487871in}{1.207217in}}%
\pgfpathlineto{\pgfqpoint{0.490121in}{1.194290in}}%
\pgfpathclose%
\pgfusepath{fill}%
\end{pgfscope}%
\begin{pgfscope}%
\pgfpathrectangle{\pgfqpoint{0.211875in}{0.211875in}}{\pgfqpoint{1.313625in}{1.279725in}}%
\pgfusepath{clip}%
\pgfsetbuttcap%
\pgfsetroundjoin%
\definecolor{currentfill}{rgb}{0.965440,0.720101,0.576404}%
\pgfsetfillcolor{currentfill}%
\pgfsetlinewidth{0.000000pt}%
\definecolor{currentstroke}{rgb}{0.000000,0.000000,0.000000}%
\pgfsetstrokecolor{currentstroke}%
\pgfsetdash{}{0pt}%
\pgfpathmoveto{\pgfqpoint{0.609943in}{1.187470in}}%
\pgfpathlineto{\pgfqpoint{0.623212in}{1.184043in}}%
\pgfpathlineto{\pgfqpoint{0.634516in}{1.194290in}}%
\pgfpathlineto{\pgfqpoint{0.636481in}{1.202190in}}%
\pgfpathlineto{\pgfqpoint{0.637072in}{1.207217in}}%
\pgfpathlineto{\pgfqpoint{0.636481in}{1.211998in}}%
\pgfpathlineto{\pgfqpoint{0.634349in}{1.220143in}}%
\pgfpathlineto{\pgfqpoint{0.623212in}{1.230108in}}%
\pgfpathlineto{\pgfqpoint{0.609943in}{1.226698in}}%
\pgfpathlineto{\pgfqpoint{0.605735in}{1.220143in}}%
\pgfpathlineto{\pgfqpoint{0.603561in}{1.207217in}}%
\pgfpathlineto{\pgfqpoint{0.605626in}{1.194290in}}%
\pgfpathclose%
\pgfpathmoveto{\pgfqpoint{0.614475in}{1.207217in}}%
\pgfpathlineto{\pgfqpoint{0.623212in}{1.215592in}}%
\pgfpathlineto{\pgfqpoint{0.626214in}{1.207217in}}%
\pgfpathlineto{\pgfqpoint{0.623212in}{1.198545in}}%
\pgfpathclose%
\pgfusepath{fill}%
\end{pgfscope}%
\begin{pgfscope}%
\pgfpathrectangle{\pgfqpoint{0.211875in}{0.211875in}}{\pgfqpoint{1.313625in}{1.279725in}}%
\pgfusepath{clip}%
\pgfsetbuttcap%
\pgfsetroundjoin%
\definecolor{currentfill}{rgb}{0.965440,0.720101,0.576404}%
\pgfsetfillcolor{currentfill}%
\pgfsetlinewidth{0.000000pt}%
\definecolor{currentstroke}{rgb}{0.000000,0.000000,0.000000}%
\pgfsetstrokecolor{currentstroke}%
\pgfsetdash{}{0pt}%
\pgfpathmoveto{\pgfqpoint{0.264951in}{1.203173in}}%
\pgfpathlineto{\pgfqpoint{0.267553in}{1.207217in}}%
\pgfpathlineto{\pgfqpoint{0.264951in}{1.211123in}}%
\pgfpathlineto{\pgfqpoint{0.263213in}{1.207217in}}%
\pgfpathclose%
\pgfusepath{fill}%
\end{pgfscope}%
\begin{pgfscope}%
\pgfpathrectangle{\pgfqpoint{0.211875in}{0.211875in}}{\pgfqpoint{1.313625in}{1.279725in}}%
\pgfusepath{clip}%
\pgfsetbuttcap%
\pgfsetroundjoin%
\definecolor{currentfill}{rgb}{0.965440,0.720101,0.576404}%
\pgfsetfillcolor{currentfill}%
\pgfsetlinewidth{0.000000pt}%
\definecolor{currentstroke}{rgb}{0.000000,0.000000,0.000000}%
\pgfsetstrokecolor{currentstroke}%
\pgfsetdash{}{0pt}%
\pgfpathmoveto{\pgfqpoint{0.556867in}{1.267709in}}%
\pgfpathlineto{\pgfqpoint{0.570136in}{1.270060in}}%
\pgfpathlineto{\pgfqpoint{0.571631in}{1.271849in}}%
\pgfpathlineto{\pgfqpoint{0.575558in}{1.284776in}}%
\pgfpathlineto{\pgfqpoint{0.574486in}{1.297702in}}%
\pgfpathlineto{\pgfqpoint{0.570136in}{1.305815in}}%
\pgfpathlineto{\pgfqpoint{0.556867in}{1.308996in}}%
\pgfpathlineto{\pgfqpoint{0.547619in}{1.297702in}}%
\pgfpathlineto{\pgfqpoint{0.546106in}{1.284776in}}%
\pgfpathlineto{\pgfqpoint{0.551621in}{1.271849in}}%
\pgfpathclose%
\pgfpathmoveto{\pgfqpoint{0.556811in}{1.284776in}}%
\pgfpathlineto{\pgfqpoint{0.556867in}{1.285140in}}%
\pgfpathlineto{\pgfqpoint{0.557092in}{1.284776in}}%
\pgfpathlineto{\pgfqpoint{0.556867in}{1.284663in}}%
\pgfpathclose%
\pgfusepath{fill}%
\end{pgfscope}%
\begin{pgfscope}%
\pgfpathrectangle{\pgfqpoint{0.211875in}{0.211875in}}{\pgfqpoint{1.313625in}{1.279725in}}%
\pgfusepath{clip}%
\pgfsetbuttcap%
\pgfsetroundjoin%
\definecolor{currentfill}{rgb}{0.965440,0.720101,0.576404}%
\pgfsetfillcolor{currentfill}%
\pgfsetlinewidth{0.000000pt}%
\definecolor{currentstroke}{rgb}{0.000000,0.000000,0.000000}%
\pgfsetstrokecolor{currentstroke}%
\pgfsetdash{}{0pt}%
\pgfpathmoveto{\pgfqpoint{0.676288in}{1.264328in}}%
\pgfpathlineto{\pgfqpoint{0.689557in}{1.267743in}}%
\pgfpathlineto{\pgfqpoint{0.692598in}{1.271849in}}%
\pgfpathlineto{\pgfqpoint{0.695754in}{1.284776in}}%
\pgfpathlineto{\pgfqpoint{0.694877in}{1.297702in}}%
\pgfpathlineto{\pgfqpoint{0.689557in}{1.308868in}}%
\pgfpathlineto{\pgfqpoint{0.685051in}{1.310629in}}%
\pgfpathlineto{\pgfqpoint{0.676288in}{1.312475in}}%
\pgfpathlineto{\pgfqpoint{0.672381in}{1.310629in}}%
\pgfpathlineto{\pgfqpoint{0.663019in}{1.300577in}}%
\pgfpathlineto{\pgfqpoint{0.662073in}{1.297702in}}%
\pgfpathlineto{\pgfqpoint{0.661304in}{1.284776in}}%
\pgfpathlineto{\pgfqpoint{0.663019in}{1.276150in}}%
\pgfpathlineto{\pgfqpoint{0.665118in}{1.271849in}}%
\pgfpathclose%
\pgfpathmoveto{\pgfqpoint{0.671966in}{1.284776in}}%
\pgfpathlineto{\pgfqpoint{0.674199in}{1.297702in}}%
\pgfpathlineto{\pgfqpoint{0.676288in}{1.299880in}}%
\pgfpathlineto{\pgfqpoint{0.680788in}{1.297702in}}%
\pgfpathlineto{\pgfqpoint{0.685553in}{1.284776in}}%
\pgfpathlineto{\pgfqpoint{0.676288in}{1.277421in}}%
\pgfpathclose%
\pgfusepath{fill}%
\end{pgfscope}%
\begin{pgfscope}%
\pgfpathrectangle{\pgfqpoint{0.211875in}{0.211875in}}{\pgfqpoint{1.313625in}{1.279725in}}%
\pgfusepath{clip}%
\pgfsetbuttcap%
\pgfsetroundjoin%
\definecolor{currentfill}{rgb}{0.965440,0.720101,0.576404}%
\pgfsetfillcolor{currentfill}%
\pgfsetlinewidth{0.000000pt}%
\definecolor{currentstroke}{rgb}{0.000000,0.000000,0.000000}%
\pgfsetstrokecolor{currentstroke}%
\pgfsetdash{}{0pt}%
\pgfpathmoveto{\pgfqpoint{0.782439in}{1.268307in}}%
\pgfpathlineto{\pgfqpoint{0.795708in}{1.261825in}}%
\pgfpathlineto{\pgfqpoint{0.808977in}{1.266194in}}%
\pgfpathlineto{\pgfqpoint{0.812698in}{1.271849in}}%
\pgfpathlineto{\pgfqpoint{0.815309in}{1.284776in}}%
\pgfpathlineto{\pgfqpoint{0.814571in}{1.297702in}}%
\pgfpathlineto{\pgfqpoint{0.809137in}{1.310629in}}%
\pgfpathlineto{\pgfqpoint{0.808977in}{1.310797in}}%
\pgfpathlineto{\pgfqpoint{0.795708in}{1.314618in}}%
\pgfpathlineto{\pgfqpoint{0.785548in}{1.310629in}}%
\pgfpathlineto{\pgfqpoint{0.782439in}{1.308051in}}%
\pgfpathlineto{\pgfqpoint{0.778600in}{1.297702in}}%
\pgfpathlineto{\pgfqpoint{0.777887in}{1.284776in}}%
\pgfpathlineto{\pgfqpoint{0.780412in}{1.271849in}}%
\pgfpathclose%
\pgfpathmoveto{\pgfqpoint{0.786775in}{1.284776in}}%
\pgfpathlineto{\pgfqpoint{0.789247in}{1.297702in}}%
\pgfpathlineto{\pgfqpoint{0.795708in}{1.303306in}}%
\pgfpathlineto{\pgfqpoint{0.804209in}{1.297702in}}%
\pgfpathlineto{\pgfqpoint{0.807424in}{1.284776in}}%
\pgfpathlineto{\pgfqpoint{0.795708in}{1.272129in}}%
\pgfpathclose%
\pgfusepath{fill}%
\end{pgfscope}%
\begin{pgfscope}%
\pgfpathrectangle{\pgfqpoint{0.211875in}{0.211875in}}{\pgfqpoint{1.313625in}{1.279725in}}%
\pgfusepath{clip}%
\pgfsetbuttcap%
\pgfsetroundjoin%
\definecolor{currentfill}{rgb}{0.965440,0.720101,0.576404}%
\pgfsetfillcolor{currentfill}%
\pgfsetlinewidth{0.000000pt}%
\definecolor{currentstroke}{rgb}{0.000000,0.000000,0.000000}%
\pgfsetstrokecolor{currentstroke}%
\pgfsetdash{}{0pt}%
\pgfpathmoveto{\pgfqpoint{0.901860in}{1.264487in}}%
\pgfpathlineto{\pgfqpoint{0.915129in}{1.260133in}}%
\pgfpathlineto{\pgfqpoint{0.928398in}{1.265518in}}%
\pgfpathlineto{\pgfqpoint{0.932093in}{1.271849in}}%
\pgfpathlineto{\pgfqpoint{0.934337in}{1.284776in}}%
\pgfpathlineto{\pgfqpoint{0.933693in}{1.297702in}}%
\pgfpathlineto{\pgfqpoint{0.929001in}{1.310629in}}%
\pgfpathlineto{\pgfqpoint{0.928398in}{1.311343in}}%
\pgfpathlineto{\pgfqpoint{0.915129in}{1.316064in}}%
\pgfpathlineto{\pgfqpoint{0.901860in}{1.312250in}}%
\pgfpathlineto{\pgfqpoint{0.900357in}{1.310629in}}%
\pgfpathlineto{\pgfqpoint{0.895421in}{1.297702in}}%
\pgfpathlineto{\pgfqpoint{0.894741in}{1.284776in}}%
\pgfpathlineto{\pgfqpoint{0.897118in}{1.271849in}}%
\pgfpathclose%
\pgfpathmoveto{\pgfqpoint{0.911651in}{1.271849in}}%
\pgfpathlineto{\pgfqpoint{0.901860in}{1.283358in}}%
\pgfpathlineto{\pgfqpoint{0.901501in}{1.284776in}}%
\pgfpathlineto{\pgfqpoint{0.901860in}{1.289285in}}%
\pgfpathlineto{\pgfqpoint{0.903796in}{1.297702in}}%
\pgfpathlineto{\pgfqpoint{0.915129in}{1.305559in}}%
\pgfpathlineto{\pgfqpoint{0.924558in}{1.297702in}}%
\pgfpathlineto{\pgfqpoint{0.926960in}{1.284776in}}%
\pgfpathlineto{\pgfqpoint{0.918042in}{1.271849in}}%
\pgfpathlineto{\pgfqpoint{0.915129in}{1.270294in}}%
\pgfpathclose%
\pgfusepath{fill}%
\end{pgfscope}%
\begin{pgfscope}%
\pgfpathrectangle{\pgfqpoint{0.211875in}{0.211875in}}{\pgfqpoint{1.313625in}{1.279725in}}%
\pgfusepath{clip}%
\pgfsetbuttcap%
\pgfsetroundjoin%
\definecolor{currentfill}{rgb}{0.965440,0.720101,0.576404}%
\pgfsetfillcolor{currentfill}%
\pgfsetlinewidth{0.000000pt}%
\definecolor{currentstroke}{rgb}{0.000000,0.000000,0.000000}%
\pgfsetstrokecolor{currentstroke}%
\pgfsetdash{}{0pt}%
\pgfpathmoveto{\pgfqpoint{1.021280in}{1.262071in}}%
\pgfpathlineto{\pgfqpoint{1.034549in}{1.259221in}}%
\pgfpathlineto{\pgfqpoint{1.047818in}{1.265895in}}%
\pgfpathlineto{\pgfqpoint{1.050890in}{1.271849in}}%
\pgfpathlineto{\pgfqpoint{1.052913in}{1.284776in}}%
\pgfpathlineto{\pgfqpoint{1.052327in}{1.297702in}}%
\pgfpathlineto{\pgfqpoint{1.048085in}{1.310629in}}%
\pgfpathlineto{\pgfqpoint{1.047818in}{1.310986in}}%
\pgfpathlineto{\pgfqpoint{1.034549in}{1.316842in}}%
\pgfpathlineto{\pgfqpoint{1.021280in}{1.314334in}}%
\pgfpathlineto{\pgfqpoint{1.017420in}{1.310629in}}%
\pgfpathlineto{\pgfqpoint{1.012537in}{1.297702in}}%
\pgfpathlineto{\pgfqpoint{1.011861in}{1.284776in}}%
\pgfpathlineto{\pgfqpoint{1.014198in}{1.271849in}}%
\pgfpathclose%
\pgfpathmoveto{\pgfqpoint{1.027360in}{1.271849in}}%
\pgfpathlineto{\pgfqpoint{1.021280in}{1.276843in}}%
\pgfpathlineto{\pgfqpoint{1.019023in}{1.284776in}}%
\pgfpathlineto{\pgfqpoint{1.020110in}{1.297702in}}%
\pgfpathlineto{\pgfqpoint{1.021280in}{1.300204in}}%
\pgfpathlineto{\pgfqpoint{1.034549in}{1.306683in}}%
\pgfpathlineto{\pgfqpoint{1.043457in}{1.297702in}}%
\pgfpathlineto{\pgfqpoint{1.045386in}{1.284776in}}%
\pgfpathlineto{\pgfqpoint{1.038258in}{1.271849in}}%
\pgfpathlineto{\pgfqpoint{1.034549in}{1.269449in}}%
\pgfpathclose%
\pgfusepath{fill}%
\end{pgfscope}%
\begin{pgfscope}%
\pgfpathrectangle{\pgfqpoint{0.211875in}{0.211875in}}{\pgfqpoint{1.313625in}{1.279725in}}%
\pgfusepath{clip}%
\pgfsetbuttcap%
\pgfsetroundjoin%
\definecolor{currentfill}{rgb}{0.965440,0.720101,0.576404}%
\pgfsetfillcolor{currentfill}%
\pgfsetlinewidth{0.000000pt}%
\definecolor{currentstroke}{rgb}{0.000000,0.000000,0.000000}%
\pgfsetstrokecolor{currentstroke}%
\pgfsetdash{}{0pt}%
\pgfpathmoveto{\pgfqpoint{1.140701in}{1.260792in}}%
\pgfpathlineto{\pgfqpoint{1.153970in}{1.259084in}}%
\pgfpathlineto{\pgfqpoint{1.167239in}{1.267617in}}%
\pgfpathlineto{\pgfqpoint{1.169156in}{1.271849in}}%
\pgfpathlineto{\pgfqpoint{1.171081in}{1.284776in}}%
\pgfpathlineto{\pgfqpoint{1.170522in}{1.297702in}}%
\pgfpathlineto{\pgfqpoint{1.167239in}{1.308845in}}%
\pgfpathlineto{\pgfqpoint{1.165877in}{1.310629in}}%
\pgfpathlineto{\pgfqpoint{1.153970in}{1.316954in}}%
\pgfpathlineto{\pgfqpoint{1.140701in}{1.315446in}}%
\pgfpathlineto{\pgfqpoint{1.135050in}{1.310629in}}%
\pgfpathlineto{\pgfqpoint{1.129967in}{1.297702in}}%
\pgfpathlineto{\pgfqpoint{1.129263in}{1.284776in}}%
\pgfpathlineto{\pgfqpoint{1.131687in}{1.271849in}}%
\pgfpathclose%
\pgfpathmoveto{\pgfqpoint{1.142972in}{1.271849in}}%
\pgfpathlineto{\pgfqpoint{1.140701in}{1.272995in}}%
\pgfpathlineto{\pgfqpoint{1.136929in}{1.284776in}}%
\pgfpathlineto{\pgfqpoint{1.138076in}{1.297702in}}%
\pgfpathlineto{\pgfqpoint{1.140701in}{1.302699in}}%
\pgfpathlineto{\pgfqpoint{1.153970in}{1.306678in}}%
\pgfpathlineto{\pgfqpoint{1.161535in}{1.297702in}}%
\pgfpathlineto{\pgfqpoint{1.163180in}{1.284776in}}%
\pgfpathlineto{\pgfqpoint{1.157117in}{1.271849in}}%
\pgfpathlineto{\pgfqpoint{1.153970in}{1.269448in}}%
\pgfpathclose%
\pgfusepath{fill}%
\end{pgfscope}%
\begin{pgfscope}%
\pgfpathrectangle{\pgfqpoint{0.211875in}{0.211875in}}{\pgfqpoint{1.313625in}{1.279725in}}%
\pgfusepath{clip}%
\pgfsetbuttcap%
\pgfsetroundjoin%
\definecolor{currentfill}{rgb}{0.965440,0.720101,0.576404}%
\pgfsetfillcolor{currentfill}%
\pgfsetlinewidth{0.000000pt}%
\definecolor{currentstroke}{rgb}{0.000000,0.000000,0.000000}%
\pgfsetstrokecolor{currentstroke}%
\pgfsetdash{}{0pt}%
\pgfpathmoveto{\pgfqpoint{1.260121in}{1.260475in}}%
\pgfpathlineto{\pgfqpoint{1.273390in}{1.259750in}}%
\pgfpathlineto{\pgfqpoint{1.286659in}{1.271163in}}%
\pgfpathlineto{\pgfqpoint{1.286929in}{1.271849in}}%
\pgfpathlineto{\pgfqpoint{1.288867in}{1.284776in}}%
\pgfpathlineto{\pgfqpoint{1.288306in}{1.297702in}}%
\pgfpathlineto{\pgfqpoint{1.286659in}{1.304105in}}%
\pgfpathlineto{\pgfqpoint{1.282732in}{1.310629in}}%
\pgfpathlineto{\pgfqpoint{1.273390in}{1.316376in}}%
\pgfpathlineto{\pgfqpoint{1.260121in}{1.315734in}}%
\pgfpathlineto{\pgfqpoint{1.253353in}{1.310629in}}%
\pgfpathlineto{\pgfqpoint{1.247751in}{1.297702in}}%
\pgfpathlineto{\pgfqpoint{1.246978in}{1.284776in}}%
\pgfpathlineto{\pgfqpoint{1.249648in}{1.271849in}}%
\pgfpathclose%
\pgfpathmoveto{\pgfqpoint{1.259915in}{1.271849in}}%
\pgfpathlineto{\pgfqpoint{1.255283in}{1.284776in}}%
\pgfpathlineto{\pgfqpoint{1.256538in}{1.297702in}}%
\pgfpathlineto{\pgfqpoint{1.260121in}{1.303752in}}%
\pgfpathlineto{\pgfqpoint{1.273390in}{1.305503in}}%
\pgfpathlineto{\pgfqpoint{1.279081in}{1.297702in}}%
\pgfpathlineto{\pgfqpoint{1.280556in}{1.284776in}}%
\pgfpathlineto{\pgfqpoint{1.275118in}{1.271849in}}%
\pgfpathlineto{\pgfqpoint{1.273390in}{1.270324in}}%
\pgfpathlineto{\pgfqpoint{1.260121in}{1.271625in}}%
\pgfpathclose%
\pgfusepath{fill}%
\end{pgfscope}%
\begin{pgfscope}%
\pgfpathrectangle{\pgfqpoint{0.211875in}{0.211875in}}{\pgfqpoint{1.313625in}{1.279725in}}%
\pgfusepath{clip}%
\pgfsetbuttcap%
\pgfsetroundjoin%
\definecolor{currentfill}{rgb}{0.965440,0.720101,0.576404}%
\pgfsetfillcolor{currentfill}%
\pgfsetlinewidth{0.000000pt}%
\definecolor{currentstroke}{rgb}{0.000000,0.000000,0.000000}%
\pgfsetstrokecolor{currentstroke}%
\pgfsetdash{}{0pt}%
\pgfpathmoveto{\pgfqpoint{1.379542in}{1.261007in}}%
\pgfpathlineto{\pgfqpoint{1.392811in}{1.261281in}}%
\pgfpathlineto{\pgfqpoint{1.403296in}{1.271849in}}%
\pgfpathlineto{\pgfqpoint{1.406080in}{1.283293in}}%
\pgfpathlineto{\pgfqpoint{1.406274in}{1.284776in}}%
\pgfpathlineto{\pgfqpoint{1.406080in}{1.289231in}}%
\pgfpathlineto{\pgfqpoint{1.405491in}{1.297702in}}%
\pgfpathlineto{\pgfqpoint{1.399107in}{1.310629in}}%
\pgfpathlineto{\pgfqpoint{1.392811in}{1.315057in}}%
\pgfpathlineto{\pgfqpoint{1.379542in}{1.315293in}}%
\pgfpathlineto{\pgfqpoint{1.372505in}{1.310629in}}%
\pgfpathlineto{\pgfqpoint{1.366273in}{1.298563in}}%
\pgfpathlineto{\pgfqpoint{1.366072in}{1.297702in}}%
\pgfpathlineto{\pgfqpoint{1.365491in}{1.284776in}}%
\pgfpathlineto{\pgfqpoint{1.366273in}{1.279126in}}%
\pgfpathlineto{\pgfqpoint{1.368193in}{1.271849in}}%
\pgfpathclose%
\pgfpathmoveto{\pgfqpoint{1.379487in}{1.271849in}}%
\pgfpathlineto{\pgfqpoint{1.374193in}{1.284776in}}%
\pgfpathlineto{\pgfqpoint{1.375620in}{1.297702in}}%
\pgfpathlineto{\pgfqpoint{1.379542in}{1.303535in}}%
\pgfpathlineto{\pgfqpoint{1.392811in}{1.303067in}}%
\pgfpathlineto{\pgfqpoint{1.396242in}{1.297702in}}%
\pgfpathlineto{\pgfqpoint{1.397622in}{1.284776in}}%
\pgfpathlineto{\pgfqpoint{1.392811in}{1.272462in}}%
\pgfpathlineto{\pgfqpoint{1.381537in}{1.271849in}}%
\pgfpathlineto{\pgfqpoint{1.379542in}{1.271797in}}%
\pgfpathclose%
\pgfusepath{fill}%
\end{pgfscope}%
\begin{pgfscope}%
\pgfpathrectangle{\pgfqpoint{0.211875in}{0.211875in}}{\pgfqpoint{1.313625in}{1.279725in}}%
\pgfusepath{clip}%
\pgfsetbuttcap%
\pgfsetroundjoin%
\definecolor{currentfill}{rgb}{0.965440,0.720101,0.576404}%
\pgfsetfillcolor{currentfill}%
\pgfsetlinewidth{0.000000pt}%
\definecolor{currentstroke}{rgb}{0.000000,0.000000,0.000000}%
\pgfsetstrokecolor{currentstroke}%
\pgfsetdash{}{0pt}%
\pgfpathmoveto{\pgfqpoint{1.498962in}{1.262322in}}%
\pgfpathlineto{\pgfqpoint{1.512231in}{1.263780in}}%
\pgfpathlineto{\pgfqpoint{1.519301in}{1.271849in}}%
\pgfpathlineto{\pgfqpoint{1.522436in}{1.284776in}}%
\pgfpathlineto{\pgfqpoint{1.521551in}{1.297702in}}%
\pgfpathlineto{\pgfqpoint{1.515094in}{1.310629in}}%
\pgfpathlineto{\pgfqpoint{1.512231in}{1.312910in}}%
\pgfpathlineto{\pgfqpoint{1.498962in}{1.314180in}}%
\pgfpathlineto{\pgfqpoint{1.492800in}{1.310629in}}%
\pgfpathlineto{\pgfqpoint{1.485693in}{1.299809in}}%
\pgfpathlineto{\pgfqpoint{1.485126in}{1.297702in}}%
\pgfpathlineto{\pgfqpoint{1.484478in}{1.284776in}}%
\pgfpathlineto{\pgfqpoint{1.485693in}{1.277238in}}%
\pgfpathlineto{\pgfqpoint{1.487511in}{1.271849in}}%
\pgfpathclose%
\pgfpathmoveto{\pgfqpoint{1.493840in}{1.284776in}}%
\pgfpathlineto{\pgfqpoint{1.495531in}{1.297702in}}%
\pgfpathlineto{\pgfqpoint{1.498962in}{1.302155in}}%
\pgfpathlineto{\pgfqpoint{1.512231in}{1.299223in}}%
\pgfpathlineto{\pgfqpoint{1.513091in}{1.297702in}}%
\pgfpathlineto{\pgfqpoint{1.514432in}{1.284776in}}%
\pgfpathlineto{\pgfqpoint{1.512231in}{1.278401in}}%
\pgfpathlineto{\pgfqpoint{1.498962in}{1.273891in}}%
\pgfpathclose%
\pgfusepath{fill}%
\end{pgfscope}%
\begin{pgfscope}%
\pgfpathrectangle{\pgfqpoint{0.211875in}{0.211875in}}{\pgfqpoint{1.313625in}{1.279725in}}%
\pgfusepath{clip}%
\pgfsetbuttcap%
\pgfsetroundjoin%
\definecolor{currentfill}{rgb}{0.965440,0.720101,0.576404}%
\pgfsetfillcolor{currentfill}%
\pgfsetlinewidth{0.000000pt}%
\definecolor{currentstroke}{rgb}{0.000000,0.000000,0.000000}%
\pgfsetstrokecolor{currentstroke}%
\pgfsetdash{}{0pt}%
\pgfpathmoveto{\pgfqpoint{0.318027in}{1.283597in}}%
\pgfpathlineto{\pgfqpoint{0.331295in}{1.282063in}}%
\pgfpathlineto{\pgfqpoint{0.332434in}{1.284776in}}%
\pgfpathlineto{\pgfqpoint{0.331295in}{1.293532in}}%
\pgfpathlineto{\pgfqpoint{0.318027in}{1.288581in}}%
\pgfpathlineto{\pgfqpoint{0.317576in}{1.284776in}}%
\pgfpathclose%
\pgfusepath{fill}%
\end{pgfscope}%
\begin{pgfscope}%
\pgfpathrectangle{\pgfqpoint{0.211875in}{0.211875in}}{\pgfqpoint{1.313625in}{1.279725in}}%
\pgfusepath{clip}%
\pgfsetbuttcap%
\pgfsetroundjoin%
\definecolor{currentfill}{rgb}{0.965440,0.720101,0.576404}%
\pgfsetfillcolor{currentfill}%
\pgfsetlinewidth{0.000000pt}%
\definecolor{currentstroke}{rgb}{0.000000,0.000000,0.000000}%
\pgfsetstrokecolor{currentstroke}%
\pgfsetdash{}{0pt}%
\pgfpathmoveto{\pgfqpoint{0.437447in}{1.272298in}}%
\pgfpathlineto{\pgfqpoint{0.450716in}{1.274402in}}%
\pgfpathlineto{\pgfqpoint{0.454543in}{1.284776in}}%
\pgfpathlineto{\pgfqpoint{0.453201in}{1.297702in}}%
\pgfpathlineto{\pgfqpoint{0.450716in}{1.301806in}}%
\pgfpathlineto{\pgfqpoint{0.437447in}{1.303183in}}%
\pgfpathlineto{\pgfqpoint{0.433549in}{1.297702in}}%
\pgfpathlineto{\pgfqpoint{0.432014in}{1.284776in}}%
\pgfpathclose%
\pgfusepath{fill}%
\end{pgfscope}%
\begin{pgfscope}%
\pgfpathrectangle{\pgfqpoint{0.211875in}{0.211875in}}{\pgfqpoint{1.313625in}{1.279725in}}%
\pgfusepath{clip}%
\pgfsetbuttcap%
\pgfsetroundjoin%
\definecolor{currentfill}{rgb}{0.965440,0.720101,0.576404}%
\pgfsetfillcolor{currentfill}%
\pgfsetlinewidth{0.000000pt}%
\definecolor{currentstroke}{rgb}{0.000000,0.000000,0.000000}%
\pgfsetstrokecolor{currentstroke}%
\pgfsetdash{}{0pt}%
\pgfpathmoveto{\pgfqpoint{0.623212in}{1.347481in}}%
\pgfpathlineto{\pgfqpoint{0.626536in}{1.349408in}}%
\pgfpathlineto{\pgfqpoint{0.635195in}{1.362335in}}%
\pgfpathlineto{\pgfqpoint{0.635421in}{1.375261in}}%
\pgfpathlineto{\pgfqpoint{0.627817in}{1.388188in}}%
\pgfpathlineto{\pgfqpoint{0.623212in}{1.391085in}}%
\pgfpathlineto{\pgfqpoint{0.610284in}{1.388188in}}%
\pgfpathlineto{\pgfqpoint{0.609943in}{1.388033in}}%
\pgfpathlineto{\pgfqpoint{0.604977in}{1.375261in}}%
\pgfpathlineto{\pgfqpoint{0.605133in}{1.362335in}}%
\pgfpathlineto{\pgfqpoint{0.609943in}{1.351023in}}%
\pgfpathlineto{\pgfqpoint{0.613966in}{1.349408in}}%
\pgfpathclose%
\pgfpathmoveto{\pgfqpoint{0.622807in}{1.362335in}}%
\pgfpathlineto{\pgfqpoint{0.621570in}{1.375261in}}%
\pgfpathlineto{\pgfqpoint{0.623212in}{1.376103in}}%
\pgfpathlineto{\pgfqpoint{0.623778in}{1.375261in}}%
\pgfpathlineto{\pgfqpoint{0.623352in}{1.362335in}}%
\pgfpathlineto{\pgfqpoint{0.623212in}{1.362146in}}%
\pgfpathclose%
\pgfusepath{fill}%
\end{pgfscope}%
\begin{pgfscope}%
\pgfpathrectangle{\pgfqpoint{0.211875in}{0.211875in}}{\pgfqpoint{1.313625in}{1.279725in}}%
\pgfusepath{clip}%
\pgfsetbuttcap%
\pgfsetroundjoin%
\definecolor{currentfill}{rgb}{0.965440,0.720101,0.576404}%
\pgfsetfillcolor{currentfill}%
\pgfsetlinewidth{0.000000pt}%
\definecolor{currentstroke}{rgb}{0.000000,0.000000,0.000000}%
\pgfsetstrokecolor{currentstroke}%
\pgfsetdash{}{0pt}%
\pgfpathmoveto{\pgfqpoint{0.729364in}{1.346673in}}%
\pgfpathlineto{\pgfqpoint{0.742633in}{1.345335in}}%
\pgfpathlineto{\pgfqpoint{0.748660in}{1.349408in}}%
\pgfpathlineto{\pgfqpoint{0.755277in}{1.362335in}}%
\pgfpathlineto{\pgfqpoint{0.755431in}{1.375261in}}%
\pgfpathlineto{\pgfqpoint{0.749560in}{1.388188in}}%
\pgfpathlineto{\pgfqpoint{0.742633in}{1.393261in}}%
\pgfpathlineto{\pgfqpoint{0.729364in}{1.391866in}}%
\pgfpathlineto{\pgfqpoint{0.725612in}{1.388188in}}%
\pgfpathlineto{\pgfqpoint{0.720801in}{1.375261in}}%
\pgfpathlineto{\pgfqpoint{0.720934in}{1.362335in}}%
\pgfpathlineto{\pgfqpoint{0.726353in}{1.349408in}}%
\pgfpathclose%
\pgfpathmoveto{\pgfqpoint{0.729247in}{1.362335in}}%
\pgfpathlineto{\pgfqpoint{0.728980in}{1.375261in}}%
\pgfpathlineto{\pgfqpoint{0.729364in}{1.376138in}}%
\pgfpathlineto{\pgfqpoint{0.742633in}{1.379711in}}%
\pgfpathlineto{\pgfqpoint{0.745209in}{1.375261in}}%
\pgfpathlineto{\pgfqpoint{0.744881in}{1.362335in}}%
\pgfpathlineto{\pgfqpoint{0.742633in}{1.358801in}}%
\pgfpathlineto{\pgfqpoint{0.729364in}{1.362091in}}%
\pgfpathclose%
\pgfusepath{fill}%
\end{pgfscope}%
\begin{pgfscope}%
\pgfpathrectangle{\pgfqpoint{0.211875in}{0.211875in}}{\pgfqpoint{1.313625in}{1.279725in}}%
\pgfusepath{clip}%
\pgfsetbuttcap%
\pgfsetroundjoin%
\definecolor{currentfill}{rgb}{0.965440,0.720101,0.576404}%
\pgfsetfillcolor{currentfill}%
\pgfsetlinewidth{0.000000pt}%
\definecolor{currentstroke}{rgb}{0.000000,0.000000,0.000000}%
\pgfsetstrokecolor{currentstroke}%
\pgfsetdash{}{0pt}%
\pgfpathmoveto{\pgfqpoint{0.848784in}{1.344041in}}%
\pgfpathlineto{\pgfqpoint{0.862053in}{1.343870in}}%
\pgfpathlineto{\pgfqpoint{0.869189in}{1.349408in}}%
\pgfpathlineto{\pgfqpoint{0.874478in}{1.362335in}}%
\pgfpathlineto{\pgfqpoint{0.874587in}{1.375261in}}%
\pgfpathlineto{\pgfqpoint{0.869850in}{1.388188in}}%
\pgfpathlineto{\pgfqpoint{0.862053in}{1.394736in}}%
\pgfpathlineto{\pgfqpoint{0.848784in}{1.394562in}}%
\pgfpathlineto{\pgfqpoint{0.841419in}{1.388188in}}%
\pgfpathlineto{\pgfqpoint{0.836685in}{1.375261in}}%
\pgfpathlineto{\pgfqpoint{0.836799in}{1.362335in}}%
\pgfpathlineto{\pgfqpoint{0.842087in}{1.349408in}}%
\pgfpathclose%
\pgfpathmoveto{\pgfqpoint{0.845901in}{1.362335in}}%
\pgfpathlineto{\pgfqpoint{0.845639in}{1.375261in}}%
\pgfpathlineto{\pgfqpoint{0.848784in}{1.381619in}}%
\pgfpathlineto{\pgfqpoint{0.862053in}{1.381922in}}%
\pgfpathlineto{\pgfqpoint{0.865423in}{1.375261in}}%
\pgfpathlineto{\pgfqpoint{0.865159in}{1.362335in}}%
\pgfpathlineto{\pgfqpoint{0.862053in}{1.356742in}}%
\pgfpathlineto{\pgfqpoint{0.848784in}{1.357023in}}%
\pgfpathclose%
\pgfusepath{fill}%
\end{pgfscope}%
\begin{pgfscope}%
\pgfpathrectangle{\pgfqpoint{0.211875in}{0.211875in}}{\pgfqpoint{1.313625in}{1.279725in}}%
\pgfusepath{clip}%
\pgfsetbuttcap%
\pgfsetroundjoin%
\definecolor{currentfill}{rgb}{0.965440,0.720101,0.576404}%
\pgfsetfillcolor{currentfill}%
\pgfsetlinewidth{0.000000pt}%
\definecolor{currentstroke}{rgb}{0.000000,0.000000,0.000000}%
\pgfsetstrokecolor{currentstroke}%
\pgfsetdash{}{0pt}%
\pgfpathmoveto{\pgfqpoint{0.968205in}{1.342313in}}%
\pgfpathlineto{\pgfqpoint{0.981473in}{1.343109in}}%
\pgfpathlineto{\pgfqpoint{0.988610in}{1.349408in}}%
\pgfpathlineto{\pgfqpoint{0.993055in}{1.362335in}}%
\pgfpathlineto{\pgfqpoint{0.993138in}{1.375261in}}%
\pgfpathlineto{\pgfqpoint{0.989127in}{1.388188in}}%
\pgfpathlineto{\pgfqpoint{0.981473in}{1.395489in}}%
\pgfpathlineto{\pgfqpoint{0.968205in}{1.396339in}}%
\pgfpathlineto{\pgfqpoint{0.957441in}{1.388188in}}%
\pgfpathlineto{\pgfqpoint{0.954936in}{1.382353in}}%
\pgfpathlineto{\pgfqpoint{0.953511in}{1.375261in}}%
\pgfpathlineto{\pgfqpoint{0.953574in}{1.362335in}}%
\pgfpathlineto{\pgfqpoint{0.954936in}{1.355977in}}%
\pgfpathlineto{\pgfqpoint{0.958075in}{1.349408in}}%
\pgfpathclose%
\pgfpathmoveto{\pgfqpoint{0.962812in}{1.362335in}}%
\pgfpathlineto{\pgfqpoint{0.962543in}{1.375261in}}%
\pgfpathlineto{\pgfqpoint{0.968205in}{1.385290in}}%
\pgfpathlineto{\pgfqpoint{0.981473in}{1.382676in}}%
\pgfpathlineto{\pgfqpoint{0.984783in}{1.375261in}}%
\pgfpathlineto{\pgfqpoint{0.984560in}{1.362335in}}%
\pgfpathlineto{\pgfqpoint{0.981473in}{1.356025in}}%
\pgfpathlineto{\pgfqpoint{0.968205in}{1.353639in}}%
\pgfpathclose%
\pgfusepath{fill}%
\end{pgfscope}%
\begin{pgfscope}%
\pgfpathrectangle{\pgfqpoint{0.211875in}{0.211875in}}{\pgfqpoint{1.313625in}{1.279725in}}%
\pgfusepath{clip}%
\pgfsetbuttcap%
\pgfsetroundjoin%
\definecolor{currentfill}{rgb}{0.965440,0.720101,0.576404}%
\pgfsetfillcolor{currentfill}%
\pgfsetlinewidth{0.000000pt}%
\definecolor{currentstroke}{rgb}{0.000000,0.000000,0.000000}%
\pgfsetstrokecolor{currentstroke}%
\pgfsetdash{}{0pt}%
\pgfpathmoveto{\pgfqpoint{1.074356in}{1.349403in}}%
\pgfpathlineto{\pgfqpoint{1.087625in}{1.341388in}}%
\pgfpathlineto{\pgfqpoint{1.100894in}{1.343109in}}%
\pgfpathlineto{\pgfqpoint{1.107208in}{1.349408in}}%
\pgfpathlineto{\pgfqpoint{1.111157in}{1.362335in}}%
\pgfpathlineto{\pgfqpoint{1.111227in}{1.375261in}}%
\pgfpathlineto{\pgfqpoint{1.107650in}{1.388188in}}%
\pgfpathlineto{\pgfqpoint{1.100894in}{1.395461in}}%
\pgfpathlineto{\pgfqpoint{1.087625in}{1.397293in}}%
\pgfpathlineto{\pgfqpoint{1.074356in}{1.388863in}}%
\pgfpathlineto{\pgfqpoint{1.073995in}{1.388188in}}%
\pgfpathlineto{\pgfqpoint{1.071114in}{1.375261in}}%
\pgfpathlineto{\pgfqpoint{1.071170in}{1.362335in}}%
\pgfpathlineto{\pgfqpoint{1.074353in}{1.349408in}}%
\pgfpathclose%
\pgfpathmoveto{\pgfqpoint{1.080028in}{1.362335in}}%
\pgfpathlineto{\pgfqpoint{1.079736in}{1.375261in}}%
\pgfpathlineto{\pgfqpoint{1.087625in}{1.387344in}}%
\pgfpathlineto{\pgfqpoint{1.100894in}{1.381848in}}%
\pgfpathlineto{\pgfqpoint{1.103504in}{1.375261in}}%
\pgfpathlineto{\pgfqpoint{1.103306in}{1.362335in}}%
\pgfpathlineto{\pgfqpoint{1.100894in}{1.356771in}}%
\pgfpathlineto{\pgfqpoint{1.087625in}{1.351751in}}%
\pgfpathclose%
\pgfusepath{fill}%
\end{pgfscope}%
\begin{pgfscope}%
\pgfpathrectangle{\pgfqpoint{0.211875in}{0.211875in}}{\pgfqpoint{1.313625in}{1.279725in}}%
\pgfusepath{clip}%
\pgfsetbuttcap%
\pgfsetroundjoin%
\definecolor{currentfill}{rgb}{0.965440,0.720101,0.576404}%
\pgfsetfillcolor{currentfill}%
\pgfsetlinewidth{0.000000pt}%
\definecolor{currentstroke}{rgb}{0.000000,0.000000,0.000000}%
\pgfsetstrokecolor{currentstroke}%
\pgfsetdash{}{0pt}%
\pgfpathmoveto{\pgfqpoint{1.193777in}{1.347329in}}%
\pgfpathlineto{\pgfqpoint{1.207045in}{1.341205in}}%
\pgfpathlineto{\pgfqpoint{1.220314in}{1.343968in}}%
\pgfpathlineto{\pgfqpoint{1.225156in}{1.349408in}}%
\pgfpathlineto{\pgfqpoint{1.228872in}{1.362335in}}%
\pgfpathlineto{\pgfqpoint{1.228939in}{1.375261in}}%
\pgfpathlineto{\pgfqpoint{1.225574in}{1.388188in}}%
\pgfpathlineto{\pgfqpoint{1.220314in}{1.394555in}}%
\pgfpathlineto{\pgfqpoint{1.207045in}{1.397486in}}%
\pgfpathlineto{\pgfqpoint{1.193777in}{1.391018in}}%
\pgfpathlineto{\pgfqpoint{1.192060in}{1.388188in}}%
\pgfpathlineto{\pgfqpoint{1.189097in}{1.375261in}}%
\pgfpathlineto{\pgfqpoint{1.189155in}{1.362335in}}%
\pgfpathlineto{\pgfqpoint{1.192425in}{1.349408in}}%
\pgfpathclose%
\pgfpathmoveto{\pgfqpoint{1.197652in}{1.362335in}}%
\pgfpathlineto{\pgfqpoint{1.197313in}{1.375261in}}%
\pgfpathlineto{\pgfqpoint{1.207045in}{1.387904in}}%
\pgfpathlineto{\pgfqpoint{1.220314in}{1.379232in}}%
\pgfpathlineto{\pgfqpoint{1.221716in}{1.375261in}}%
\pgfpathlineto{\pgfqpoint{1.221529in}{1.362335in}}%
\pgfpathlineto{\pgfqpoint{1.220314in}{1.359180in}}%
\pgfpathlineto{\pgfqpoint{1.207045in}{1.351242in}}%
\pgfpathclose%
\pgfusepath{fill}%
\end{pgfscope}%
\begin{pgfscope}%
\pgfpathrectangle{\pgfqpoint{0.211875in}{0.211875in}}{\pgfqpoint{1.313625in}{1.279725in}}%
\pgfusepath{clip}%
\pgfsetbuttcap%
\pgfsetroundjoin%
\definecolor{currentfill}{rgb}{0.965440,0.720101,0.576404}%
\pgfsetfillcolor{currentfill}%
\pgfsetlinewidth{0.000000pt}%
\definecolor{currentstroke}{rgb}{0.000000,0.000000,0.000000}%
\pgfsetstrokecolor{currentstroke}%
\pgfsetdash{}{0pt}%
\pgfpathmoveto{\pgfqpoint{1.313197in}{1.346551in}}%
\pgfpathlineto{\pgfqpoint{1.326466in}{1.341736in}}%
\pgfpathlineto{\pgfqpoint{1.339735in}{1.345839in}}%
\pgfpathlineto{\pgfqpoint{1.342557in}{1.349408in}}%
\pgfpathlineto{\pgfqpoint{1.346251in}{1.362335in}}%
\pgfpathlineto{\pgfqpoint{1.346322in}{1.375261in}}%
\pgfpathlineto{\pgfqpoint{1.342992in}{1.388188in}}%
\pgfpathlineto{\pgfqpoint{1.339735in}{1.392617in}}%
\pgfpathlineto{\pgfqpoint{1.326466in}{1.396949in}}%
\pgfpathlineto{\pgfqpoint{1.313197in}{1.391861in}}%
\pgfpathlineto{\pgfqpoint{1.310683in}{1.388188in}}%
\pgfpathlineto{\pgfqpoint{1.307483in}{1.375261in}}%
\pgfpathlineto{\pgfqpoint{1.307549in}{1.362335in}}%
\pgfpathlineto{\pgfqpoint{1.311095in}{1.349408in}}%
\pgfpathclose%
\pgfpathmoveto{\pgfqpoint{1.315897in}{1.362335in}}%
\pgfpathlineto{\pgfqpoint{1.315472in}{1.375261in}}%
\pgfpathlineto{\pgfqpoint{1.326466in}{1.387028in}}%
\pgfpathlineto{\pgfqpoint{1.339098in}{1.375261in}}%
\pgfpathlineto{\pgfqpoint{1.338605in}{1.362335in}}%
\pgfpathlineto{\pgfqpoint{1.326466in}{1.352055in}}%
\pgfpathclose%
\pgfusepath{fill}%
\end{pgfscope}%
\begin{pgfscope}%
\pgfpathrectangle{\pgfqpoint{0.211875in}{0.211875in}}{\pgfqpoint{1.313625in}{1.279725in}}%
\pgfusepath{clip}%
\pgfsetbuttcap%
\pgfsetroundjoin%
\definecolor{currentfill}{rgb}{0.965440,0.720101,0.576404}%
\pgfsetfillcolor{currentfill}%
\pgfsetlinewidth{0.000000pt}%
\definecolor{currentstroke}{rgb}{0.000000,0.000000,0.000000}%
\pgfsetstrokecolor{currentstroke}%
\pgfsetdash{}{0pt}%
\pgfpathmoveto{\pgfqpoint{1.432617in}{1.346783in}}%
\pgfpathlineto{\pgfqpoint{1.445886in}{1.342978in}}%
\pgfpathlineto{\pgfqpoint{1.459155in}{1.348959in}}%
\pgfpathlineto{\pgfqpoint{1.459470in}{1.349408in}}%
\pgfpathlineto{\pgfqpoint{1.463318in}{1.362335in}}%
\pgfpathlineto{\pgfqpoint{1.463401in}{1.375261in}}%
\pgfpathlineto{\pgfqpoint{1.459958in}{1.388188in}}%
\pgfpathlineto{\pgfqpoint{1.459155in}{1.389415in}}%
\pgfpathlineto{\pgfqpoint{1.445886in}{1.395683in}}%
\pgfpathlineto{\pgfqpoint{1.432617in}{1.391671in}}%
\pgfpathlineto{\pgfqpoint{1.429936in}{1.388188in}}%
\pgfpathlineto{\pgfqpoint{1.426312in}{1.375261in}}%
\pgfpathlineto{\pgfqpoint{1.426396in}{1.362335in}}%
\pgfpathlineto{\pgfqpoint{1.430440in}{1.349408in}}%
\pgfpathclose%
\pgfpathmoveto{\pgfqpoint{1.435238in}{1.362335in}}%
\pgfpathlineto{\pgfqpoint{1.434655in}{1.375261in}}%
\pgfpathlineto{\pgfqpoint{1.445886in}{1.384723in}}%
\pgfpathlineto{\pgfqpoint{1.454048in}{1.375261in}}%
\pgfpathlineto{\pgfqpoint{1.453618in}{1.362335in}}%
\pgfpathlineto{\pgfqpoint{1.445886in}{1.354185in}}%
\pgfpathclose%
\pgfusepath{fill}%
\end{pgfscope}%
\begin{pgfscope}%
\pgfpathrectangle{\pgfqpoint{0.211875in}{0.211875in}}{\pgfqpoint{1.313625in}{1.279725in}}%
\pgfusepath{clip}%
\pgfsetbuttcap%
\pgfsetroundjoin%
\definecolor{currentfill}{rgb}{0.965440,0.720101,0.576404}%
\pgfsetfillcolor{currentfill}%
\pgfsetlinewidth{0.000000pt}%
\definecolor{currentstroke}{rgb}{0.000000,0.000000,0.000000}%
\pgfsetstrokecolor{currentstroke}%
\pgfsetdash{}{0pt}%
\pgfpathmoveto{\pgfqpoint{0.384371in}{1.357000in}}%
\pgfpathlineto{\pgfqpoint{0.390089in}{1.362335in}}%
\pgfpathlineto{\pgfqpoint{0.390615in}{1.375261in}}%
\pgfpathlineto{\pgfqpoint{0.384371in}{1.381678in}}%
\pgfpathlineto{\pgfqpoint{0.377784in}{1.375261in}}%
\pgfpathlineto{\pgfqpoint{0.378352in}{1.362335in}}%
\pgfpathclose%
\pgfusepath{fill}%
\end{pgfscope}%
\begin{pgfscope}%
\pgfpathrectangle{\pgfqpoint{0.211875in}{0.211875in}}{\pgfqpoint{1.313625in}{1.279725in}}%
\pgfusepath{clip}%
\pgfsetbuttcap%
\pgfsetroundjoin%
\definecolor{currentfill}{rgb}{0.965440,0.720101,0.576404}%
\pgfsetfillcolor{currentfill}%
\pgfsetlinewidth{0.000000pt}%
\definecolor{currentstroke}{rgb}{0.000000,0.000000,0.000000}%
\pgfsetstrokecolor{currentstroke}%
\pgfsetdash{}{0pt}%
\pgfpathmoveto{\pgfqpoint{0.490523in}{1.359467in}}%
\pgfpathlineto{\pgfqpoint{0.503792in}{1.350950in}}%
\pgfpathlineto{\pgfqpoint{0.513768in}{1.362335in}}%
\pgfpathlineto{\pgfqpoint{0.514108in}{1.375261in}}%
\pgfpathlineto{\pgfqpoint{0.503825in}{1.388188in}}%
\pgfpathlineto{\pgfqpoint{0.503792in}{1.388206in}}%
\pgfpathlineto{\pgfqpoint{0.503737in}{1.388188in}}%
\pgfpathlineto{\pgfqpoint{0.490523in}{1.378925in}}%
\pgfpathlineto{\pgfqpoint{0.489252in}{1.375261in}}%
\pgfpathlineto{\pgfqpoint{0.489438in}{1.362335in}}%
\pgfpathclose%
\pgfusepath{fill}%
\end{pgfscope}%
\begin{pgfscope}%
\pgfpathrectangle{\pgfqpoint{0.211875in}{0.211875in}}{\pgfqpoint{1.313625in}{1.279725in}}%
\pgfusepath{clip}%
\pgfsetbuttcap%
\pgfsetroundjoin%
\definecolor{currentfill}{rgb}{0.965440,0.720101,0.576404}%
\pgfsetfillcolor{currentfill}%
\pgfsetlinewidth{0.000000pt}%
\definecolor{currentstroke}{rgb}{0.000000,0.000000,0.000000}%
\pgfsetstrokecolor{currentstroke}%
\pgfsetdash{}{0pt}%
\pgfpathmoveto{\pgfqpoint{0.795708in}{1.425672in}}%
\pgfpathlineto{\pgfqpoint{0.800599in}{1.426967in}}%
\pgfpathlineto{\pgfqpoint{0.808977in}{1.431190in}}%
\pgfpathlineto{\pgfqpoint{0.812951in}{1.439894in}}%
\pgfpathlineto{\pgfqpoint{0.813998in}{1.452820in}}%
\pgfpathlineto{\pgfqpoint{0.811409in}{1.465747in}}%
\pgfpathlineto{\pgfqpoint{0.808977in}{1.469710in}}%
\pgfpathlineto{\pgfqpoint{0.795708in}{1.474916in}}%
\pgfpathlineto{\pgfqpoint{0.782439in}{1.467324in}}%
\pgfpathlineto{\pgfqpoint{0.781595in}{1.465747in}}%
\pgfpathlineto{\pgfqpoint{0.779111in}{1.452820in}}%
\pgfpathlineto{\pgfqpoint{0.780124in}{1.439894in}}%
\pgfpathlineto{\pgfqpoint{0.782439in}{1.434076in}}%
\pgfpathlineto{\pgfqpoint{0.792114in}{1.426967in}}%
\pgfpathclose%
\pgfpathmoveto{\pgfqpoint{0.793156in}{1.439894in}}%
\pgfpathlineto{\pgfqpoint{0.789481in}{1.452820in}}%
\pgfpathlineto{\pgfqpoint{0.795708in}{1.462807in}}%
\pgfpathlineto{\pgfqpoint{0.803871in}{1.452820in}}%
\pgfpathlineto{\pgfqpoint{0.799072in}{1.439894in}}%
\pgfpathlineto{\pgfqpoint{0.795708in}{1.437853in}}%
\pgfpathclose%
\pgfusepath{fill}%
\end{pgfscope}%
\begin{pgfscope}%
\pgfpathrectangle{\pgfqpoint{0.211875in}{0.211875in}}{\pgfqpoint{1.313625in}{1.279725in}}%
\pgfusepath{clip}%
\pgfsetbuttcap%
\pgfsetroundjoin%
\definecolor{currentfill}{rgb}{0.965440,0.720101,0.576404}%
\pgfsetfillcolor{currentfill}%
\pgfsetlinewidth{0.000000pt}%
\definecolor{currentstroke}{rgb}{0.000000,0.000000,0.000000}%
\pgfsetstrokecolor{currentstroke}%
\pgfsetdash{}{0pt}%
\pgfpathmoveto{\pgfqpoint{0.915129in}{1.424239in}}%
\pgfpathlineto{\pgfqpoint{0.923207in}{1.426967in}}%
\pgfpathlineto{\pgfqpoint{0.928398in}{1.430549in}}%
\pgfpathlineto{\pgfqpoint{0.932181in}{1.439894in}}%
\pgfpathlineto{\pgfqpoint{0.933082in}{1.452820in}}%
\pgfpathlineto{\pgfqpoint{0.930810in}{1.465747in}}%
\pgfpathlineto{\pgfqpoint{0.928398in}{1.470168in}}%
\pgfpathlineto{\pgfqpoint{0.915129in}{1.476660in}}%
\pgfpathlineto{\pgfqpoint{0.901860in}{1.471462in}}%
\pgfpathlineto{\pgfqpoint{0.898423in}{1.465747in}}%
\pgfpathlineto{\pgfqpoint{0.896033in}{1.452820in}}%
\pgfpathlineto{\pgfqpoint{0.896987in}{1.439894in}}%
\pgfpathlineto{\pgfqpoint{0.901860in}{1.429027in}}%
\pgfpathlineto{\pgfqpoint{0.905631in}{1.426967in}}%
\pgfpathclose%
\pgfpathmoveto{\pgfqpoint{0.908599in}{1.439894in}}%
\pgfpathlineto{\pgfqpoint{0.904285in}{1.452820in}}%
\pgfpathlineto{\pgfqpoint{0.914226in}{1.465747in}}%
\pgfpathlineto{\pgfqpoint{0.915129in}{1.466182in}}%
\pgfpathlineto{\pgfqpoint{0.915884in}{1.465747in}}%
\pgfpathlineto{\pgfqpoint{0.924129in}{1.452820in}}%
\pgfpathlineto{\pgfqpoint{0.920567in}{1.439894in}}%
\pgfpathlineto{\pgfqpoint{0.915129in}{1.435721in}}%
\pgfpathclose%
\pgfusepath{fill}%
\end{pgfscope}%
\begin{pgfscope}%
\pgfpathrectangle{\pgfqpoint{0.211875in}{0.211875in}}{\pgfqpoint{1.313625in}{1.279725in}}%
\pgfusepath{clip}%
\pgfsetbuttcap%
\pgfsetroundjoin%
\definecolor{currentfill}{rgb}{0.965440,0.720101,0.576404}%
\pgfsetfillcolor{currentfill}%
\pgfsetlinewidth{0.000000pt}%
\definecolor{currentstroke}{rgb}{0.000000,0.000000,0.000000}%
\pgfsetstrokecolor{currentstroke}%
\pgfsetdash{}{0pt}%
\pgfpathmoveto{\pgfqpoint{1.021280in}{1.426187in}}%
\pgfpathlineto{\pgfqpoint{1.034549in}{1.423475in}}%
\pgfpathlineto{\pgfqpoint{1.043042in}{1.426967in}}%
\pgfpathlineto{\pgfqpoint{1.047818in}{1.431299in}}%
\pgfpathlineto{\pgfqpoint{1.050891in}{1.439894in}}%
\pgfpathlineto{\pgfqpoint{1.051704in}{1.452820in}}%
\pgfpathlineto{\pgfqpoint{1.049629in}{1.465747in}}%
\pgfpathlineto{\pgfqpoint{1.047818in}{1.469489in}}%
\pgfpathlineto{\pgfqpoint{1.034549in}{1.477583in}}%
\pgfpathlineto{\pgfqpoint{1.021280in}{1.474128in}}%
\pgfpathlineto{\pgfqpoint{1.015624in}{1.465747in}}%
\pgfpathlineto{\pgfqpoint{1.013236in}{1.452820in}}%
\pgfpathlineto{\pgfqpoint{1.014175in}{1.439894in}}%
\pgfpathlineto{\pgfqpoint{1.020396in}{1.426967in}}%
\pgfpathclose%
\pgfpathmoveto{\pgfqpoint{1.023614in}{1.439894in}}%
\pgfpathlineto{\pgfqpoint{1.021280in}{1.445029in}}%
\pgfpathlineto{\pgfqpoint{1.020348in}{1.452820in}}%
\pgfpathlineto{\pgfqpoint{1.021280in}{1.456493in}}%
\pgfpathlineto{\pgfqpoint{1.030963in}{1.465747in}}%
\pgfpathlineto{\pgfqpoint{1.034549in}{1.467037in}}%
\pgfpathlineto{\pgfqpoint{1.036397in}{1.465747in}}%
\pgfpathlineto{\pgfqpoint{1.043022in}{1.452820in}}%
\pgfpathlineto{\pgfqpoint{1.040172in}{1.439894in}}%
\pgfpathlineto{\pgfqpoint{1.034549in}{1.434671in}}%
\pgfpathclose%
\pgfusepath{fill}%
\end{pgfscope}%
\begin{pgfscope}%
\pgfpathrectangle{\pgfqpoint{0.211875in}{0.211875in}}{\pgfqpoint{1.313625in}{1.279725in}}%
\pgfusepath{clip}%
\pgfsetbuttcap%
\pgfsetroundjoin%
\definecolor{currentfill}{rgb}{0.965440,0.720101,0.576404}%
\pgfsetfillcolor{currentfill}%
\pgfsetlinewidth{0.000000pt}%
\definecolor{currentstroke}{rgb}{0.000000,0.000000,0.000000}%
\pgfsetstrokecolor{currentstroke}%
\pgfsetdash{}{0pt}%
\pgfpathmoveto{\pgfqpoint{1.140701in}{1.425014in}}%
\pgfpathlineto{\pgfqpoint{1.153970in}{1.423379in}}%
\pgfpathlineto{\pgfqpoint{1.161351in}{1.426967in}}%
\pgfpathlineto{\pgfqpoint{1.167239in}{1.433840in}}%
\pgfpathlineto{\pgfqpoint{1.169137in}{1.439894in}}%
\pgfpathlineto{\pgfqpoint{1.169911in}{1.452820in}}%
\pgfpathlineto{\pgfqpoint{1.167929in}{1.465747in}}%
\pgfpathlineto{\pgfqpoint{1.167239in}{1.467365in}}%
\pgfpathlineto{\pgfqpoint{1.153970in}{1.477688in}}%
\pgfpathlineto{\pgfqpoint{1.140701in}{1.475595in}}%
\pgfpathlineto{\pgfqpoint{1.133232in}{1.465747in}}%
\pgfpathlineto{\pgfqpoint{1.130737in}{1.452820in}}%
\pgfpathlineto{\pgfqpoint{1.131711in}{1.439894in}}%
\pgfpathlineto{\pgfqpoint{1.138204in}{1.426967in}}%
\pgfpathclose%
\pgfpathmoveto{\pgfqpoint{1.140039in}{1.439894in}}%
\pgfpathlineto{\pgfqpoint{1.138350in}{1.452820in}}%
\pgfpathlineto{\pgfqpoint{1.140701in}{1.461078in}}%
\pgfpathlineto{\pgfqpoint{1.148626in}{1.465747in}}%
\pgfpathlineto{\pgfqpoint{1.153970in}{1.467005in}}%
\pgfpathlineto{\pgfqpoint{1.155499in}{1.465747in}}%
\pgfpathlineto{\pgfqpoint{1.161141in}{1.452820in}}%
\pgfpathlineto{\pgfqpoint{1.158716in}{1.439894in}}%
\pgfpathlineto{\pgfqpoint{1.153970in}{1.434702in}}%
\pgfpathlineto{\pgfqpoint{1.140701in}{1.438729in}}%
\pgfpathclose%
\pgfusepath{fill}%
\end{pgfscope}%
\begin{pgfscope}%
\pgfpathrectangle{\pgfqpoint{0.211875in}{0.211875in}}{\pgfqpoint{1.313625in}{1.279725in}}%
\pgfusepath{clip}%
\pgfsetbuttcap%
\pgfsetroundjoin%
\definecolor{currentfill}{rgb}{0.965440,0.720101,0.576404}%
\pgfsetfillcolor{currentfill}%
\pgfsetlinewidth{0.000000pt}%
\definecolor{currentstroke}{rgb}{0.000000,0.000000,0.000000}%
\pgfsetstrokecolor{currentstroke}%
\pgfsetdash{}{0pt}%
\pgfpathmoveto{\pgfqpoint{1.260121in}{1.424676in}}%
\pgfpathlineto{\pgfqpoint{1.273390in}{1.423976in}}%
\pgfpathlineto{\pgfqpoint{1.278698in}{1.426967in}}%
\pgfpathlineto{\pgfqpoint{1.286659in}{1.438829in}}%
\pgfpathlineto{\pgfqpoint{1.286949in}{1.439894in}}%
\pgfpathlineto{\pgfqpoint{1.287727in}{1.452820in}}%
\pgfpathlineto{\pgfqpoint{1.286659in}{1.460343in}}%
\pgfpathlineto{\pgfqpoint{1.285168in}{1.465747in}}%
\pgfpathlineto{\pgfqpoint{1.273390in}{1.476944in}}%
\pgfpathlineto{\pgfqpoint{1.260121in}{1.476044in}}%
\pgfpathlineto{\pgfqpoint{1.251313in}{1.465747in}}%
\pgfpathlineto{\pgfqpoint{1.248574in}{1.452820in}}%
\pgfpathlineto{\pgfqpoint{1.249646in}{1.439894in}}%
\pgfpathlineto{\pgfqpoint{1.256806in}{1.426967in}}%
\pgfpathclose%
\pgfpathmoveto{\pgfqpoint{1.258672in}{1.439894in}}%
\pgfpathlineto{\pgfqpoint{1.256820in}{1.452820in}}%
\pgfpathlineto{\pgfqpoint{1.260121in}{1.463117in}}%
\pgfpathlineto{\pgfqpoint{1.270545in}{1.465747in}}%
\pgfpathlineto{\pgfqpoint{1.273390in}{1.466051in}}%
\pgfpathlineto{\pgfqpoint{1.273710in}{1.465747in}}%
\pgfpathlineto{\pgfqpoint{1.278759in}{1.452820in}}%
\pgfpathlineto{\pgfqpoint{1.276583in}{1.439894in}}%
\pgfpathlineto{\pgfqpoint{1.273390in}{1.435857in}}%
\pgfpathlineto{\pgfqpoint{1.260121in}{1.437634in}}%
\pgfpathclose%
\pgfusepath{fill}%
\end{pgfscope}%
\begin{pgfscope}%
\pgfpathrectangle{\pgfqpoint{0.211875in}{0.211875in}}{\pgfqpoint{1.313625in}{1.279725in}}%
\pgfusepath{clip}%
\pgfsetbuttcap%
\pgfsetroundjoin%
\definecolor{currentfill}{rgb}{0.965440,0.720101,0.576404}%
\pgfsetfillcolor{currentfill}%
\pgfsetlinewidth{0.000000pt}%
\definecolor{currentstroke}{rgb}{0.000000,0.000000,0.000000}%
\pgfsetstrokecolor{currentstroke}%
\pgfsetdash{}{0pt}%
\pgfpathmoveto{\pgfqpoint{1.379542in}{1.425074in}}%
\pgfpathlineto{\pgfqpoint{1.392811in}{1.425317in}}%
\pgfpathlineto{\pgfqpoint{1.395369in}{1.426967in}}%
\pgfpathlineto{\pgfqpoint{1.403464in}{1.439894in}}%
\pgfpathlineto{\pgfqpoint{1.404699in}{1.452820in}}%
\pgfpathlineto{\pgfqpoint{1.401607in}{1.465747in}}%
\pgfpathlineto{\pgfqpoint{1.392811in}{1.475290in}}%
\pgfpathlineto{\pgfqpoint{1.379542in}{1.475592in}}%
\pgfpathlineto{\pgfqpoint{1.369979in}{1.465747in}}%
\pgfpathlineto{\pgfqpoint{1.366816in}{1.452820in}}%
\pgfpathlineto{\pgfqpoint{1.368070in}{1.439894in}}%
\pgfpathlineto{\pgfqpoint{1.376419in}{1.426967in}}%
\pgfpathclose%
\pgfpathmoveto{\pgfqpoint{1.377991in}{1.439894in}}%
\pgfpathlineto{\pgfqpoint{1.375875in}{1.452820in}}%
\pgfpathlineto{\pgfqpoint{1.379542in}{1.462910in}}%
\pgfpathlineto{\pgfqpoint{1.392811in}{1.462063in}}%
\pgfpathlineto{\pgfqpoint{1.396010in}{1.452820in}}%
\pgfpathlineto{\pgfqpoint{1.393967in}{1.439894in}}%
\pgfpathlineto{\pgfqpoint{1.392811in}{1.438224in}}%
\pgfpathlineto{\pgfqpoint{1.379542in}{1.437766in}}%
\pgfpathclose%
\pgfusepath{fill}%
\end{pgfscope}%
\begin{pgfscope}%
\pgfpathrectangle{\pgfqpoint{0.211875in}{0.211875in}}{\pgfqpoint{1.313625in}{1.279725in}}%
\pgfusepath{clip}%
\pgfsetbuttcap%
\pgfsetroundjoin%
\definecolor{currentfill}{rgb}{0.965440,0.720101,0.576404}%
\pgfsetfillcolor{currentfill}%
\pgfsetlinewidth{0.000000pt}%
\definecolor{currentstroke}{rgb}{0.000000,0.000000,0.000000}%
\pgfsetstrokecolor{currentstroke}%
\pgfsetdash{}{0pt}%
\pgfpathmoveto{\pgfqpoint{1.498962in}{1.426147in}}%
\pgfpathlineto{\pgfqpoint{1.507258in}{1.426967in}}%
\pgfpathlineto{\pgfqpoint{1.512231in}{1.427750in}}%
\pgfpathlineto{\pgfqpoint{1.519658in}{1.439894in}}%
\pgfpathlineto{\pgfqpoint{1.520916in}{1.452820in}}%
\pgfpathlineto{\pgfqpoint{1.517831in}{1.465747in}}%
\pgfpathlineto{\pgfqpoint{1.512231in}{1.472620in}}%
\pgfpathlineto{\pgfqpoint{1.498962in}{1.474310in}}%
\pgfpathlineto{\pgfqpoint{1.489422in}{1.465747in}}%
\pgfpathlineto{\pgfqpoint{1.485693in}{1.453254in}}%
\pgfpathlineto{\pgfqpoint{1.485629in}{1.452820in}}%
\pgfpathlineto{\pgfqpoint{1.485693in}{1.451835in}}%
\pgfpathlineto{\pgfqpoint{1.487135in}{1.439894in}}%
\pgfpathlineto{\pgfqpoint{1.497405in}{1.426967in}}%
\pgfpathclose%
\pgfpathmoveto{\pgfqpoint{1.498229in}{1.439894in}}%
\pgfpathlineto{\pgfqpoint{1.495709in}{1.452820in}}%
\pgfpathlineto{\pgfqpoint{1.498962in}{1.460639in}}%
\pgfpathlineto{\pgfqpoint{1.512231in}{1.455219in}}%
\pgfpathlineto{\pgfqpoint{1.512966in}{1.452820in}}%
\pgfpathlineto{\pgfqpoint{1.512231in}{1.447786in}}%
\pgfpathlineto{\pgfqpoint{1.503115in}{1.439894in}}%
\pgfpathlineto{\pgfqpoint{1.498962in}{1.439017in}}%
\pgfpathclose%
\pgfusepath{fill}%
\end{pgfscope}%
\begin{pgfscope}%
\pgfpathrectangle{\pgfqpoint{0.211875in}{0.211875in}}{\pgfqpoint{1.313625in}{1.279725in}}%
\pgfusepath{clip}%
\pgfsetbuttcap%
\pgfsetroundjoin%
\definecolor{currentfill}{rgb}{0.965440,0.720101,0.576404}%
\pgfsetfillcolor{currentfill}%
\pgfsetlinewidth{0.000000pt}%
\definecolor{currentstroke}{rgb}{0.000000,0.000000,0.000000}%
\pgfsetstrokecolor{currentstroke}%
\pgfsetdash{}{0pt}%
\pgfpathmoveto{\pgfqpoint{0.437447in}{1.438054in}}%
\pgfpathlineto{\pgfqpoint{0.450716in}{1.439482in}}%
\pgfpathlineto{\pgfqpoint{0.450986in}{1.439894in}}%
\pgfpathlineto{\pgfqpoint{0.452980in}{1.452820in}}%
\pgfpathlineto{\pgfqpoint{0.450716in}{1.459739in}}%
\pgfpathlineto{\pgfqpoint{0.437447in}{1.462402in}}%
\pgfpathlineto{\pgfqpoint{0.433756in}{1.452820in}}%
\pgfpathlineto{\pgfqpoint{0.436029in}{1.439894in}}%
\pgfpathclose%
\pgfusepath{fill}%
\end{pgfscope}%
\begin{pgfscope}%
\pgfpathrectangle{\pgfqpoint{0.211875in}{0.211875in}}{\pgfqpoint{1.313625in}{1.279725in}}%
\pgfusepath{clip}%
\pgfsetbuttcap%
\pgfsetroundjoin%
\definecolor{currentfill}{rgb}{0.965440,0.720101,0.576404}%
\pgfsetfillcolor{currentfill}%
\pgfsetlinewidth{0.000000pt}%
\definecolor{currentstroke}{rgb}{0.000000,0.000000,0.000000}%
\pgfsetstrokecolor{currentstroke}%
\pgfsetdash{}{0pt}%
\pgfpathmoveto{\pgfqpoint{0.556867in}{1.432507in}}%
\pgfpathlineto{\pgfqpoint{0.570136in}{1.435769in}}%
\pgfpathlineto{\pgfqpoint{0.572527in}{1.439894in}}%
\pgfpathlineto{\pgfqpoint{0.574097in}{1.452820in}}%
\pgfpathlineto{\pgfqpoint{0.570387in}{1.465747in}}%
\pgfpathlineto{\pgfqpoint{0.570136in}{1.466070in}}%
\pgfpathlineto{\pgfqpoint{0.556867in}{1.468799in}}%
\pgfpathlineto{\pgfqpoint{0.553276in}{1.465747in}}%
\pgfpathlineto{\pgfqpoint{0.548100in}{1.452820in}}%
\pgfpathlineto{\pgfqpoint{0.550311in}{1.439894in}}%
\pgfpathclose%
\pgfusepath{fill}%
\end{pgfscope}%
\begin{pgfscope}%
\pgfpathrectangle{\pgfqpoint{0.211875in}{0.211875in}}{\pgfqpoint{1.313625in}{1.279725in}}%
\pgfusepath{clip}%
\pgfsetbuttcap%
\pgfsetroundjoin%
\definecolor{currentfill}{rgb}{0.965440,0.720101,0.576404}%
\pgfsetfillcolor{currentfill}%
\pgfsetlinewidth{0.000000pt}%
\definecolor{currentstroke}{rgb}{0.000000,0.000000,0.000000}%
\pgfsetstrokecolor{currentstroke}%
\pgfsetdash{}{0pt}%
\pgfpathmoveto{\pgfqpoint{0.676288in}{1.428213in}}%
\pgfpathlineto{\pgfqpoint{0.689557in}{1.432979in}}%
\pgfpathlineto{\pgfqpoint{0.693112in}{1.439894in}}%
\pgfpathlineto{\pgfqpoint{0.694376in}{1.452820in}}%
\pgfpathlineto{\pgfqpoint{0.691320in}{1.465747in}}%
\pgfpathlineto{\pgfqpoint{0.689557in}{1.468305in}}%
\pgfpathlineto{\pgfqpoint{0.676288in}{1.472318in}}%
\pgfpathlineto{\pgfqpoint{0.667227in}{1.465747in}}%
\pgfpathlineto{\pgfqpoint{0.663019in}{1.455874in}}%
\pgfpathlineto{\pgfqpoint{0.662469in}{1.452820in}}%
\pgfpathlineto{\pgfqpoint{0.663019in}{1.446126in}}%
\pgfpathlineto{\pgfqpoint{0.664137in}{1.439894in}}%
\pgfpathclose%
\pgfpathmoveto{\pgfqpoint{0.674212in}{1.452820in}}%
\pgfpathlineto{\pgfqpoint{0.676288in}{1.456820in}}%
\pgfpathlineto{\pgfqpoint{0.680735in}{1.452820in}}%
\pgfpathlineto{\pgfqpoint{0.676288in}{1.444525in}}%
\pgfpathclose%
\pgfusepath{fill}%
\end{pgfscope}%
\begin{pgfscope}%
\pgfpathrectangle{\pgfqpoint{0.211875in}{0.211875in}}{\pgfqpoint{1.313625in}{1.279725in}}%
\pgfusepath{clip}%
\pgfsetbuttcap%
\pgfsetroundjoin%
\definecolor{currentfill}{rgb}{0.973832,0.856556,0.771584}%
\pgfsetfillcolor{currentfill}%
\pgfsetlinewidth{0.000000pt}%
\definecolor{currentstroke}{rgb}{0.000000,0.000000,0.000000}%
\pgfsetstrokecolor{currentstroke}%
\pgfsetdash{}{0pt}%
\pgfpathmoveto{\pgfqpoint{1.087625in}{0.554865in}}%
\pgfpathlineto{\pgfqpoint{1.091395in}{0.560891in}}%
\pgfpathlineto{\pgfqpoint{1.087625in}{0.563233in}}%
\pgfpathlineto{\pgfqpoint{1.086469in}{0.560891in}}%
\pgfpathclose%
\pgfusepath{fill}%
\end{pgfscope}%
\begin{pgfscope}%
\pgfpathrectangle{\pgfqpoint{0.211875in}{0.211875in}}{\pgfqpoint{1.313625in}{1.279725in}}%
\pgfusepath{clip}%
\pgfsetbuttcap%
\pgfsetroundjoin%
\definecolor{currentfill}{rgb}{0.973832,0.856556,0.771584}%
\pgfsetfillcolor{currentfill}%
\pgfsetlinewidth{0.000000pt}%
\definecolor{currentstroke}{rgb}{0.000000,0.000000,0.000000}%
\pgfsetstrokecolor{currentstroke}%
\pgfsetdash{}{0pt}%
\pgfpathmoveto{\pgfqpoint{1.207045in}{0.552527in}}%
\pgfpathlineto{\pgfqpoint{1.210536in}{0.560891in}}%
\pgfpathlineto{\pgfqpoint{1.207045in}{0.564133in}}%
\pgfpathlineto{\pgfqpoint{1.205158in}{0.560891in}}%
\pgfpathclose%
\pgfusepath{fill}%
\end{pgfscope}%
\begin{pgfscope}%
\pgfpathrectangle{\pgfqpoint{0.211875in}{0.211875in}}{\pgfqpoint{1.313625in}{1.279725in}}%
\pgfusepath{clip}%
\pgfsetbuttcap%
\pgfsetroundjoin%
\definecolor{currentfill}{rgb}{0.973832,0.856556,0.771584}%
\pgfsetfillcolor{currentfill}%
\pgfsetlinewidth{0.000000pt}%
\definecolor{currentstroke}{rgb}{0.000000,0.000000,0.000000}%
\pgfsetstrokecolor{currentstroke}%
\pgfsetdash{}{0pt}%
\pgfpathmoveto{\pgfqpoint{1.326466in}{0.555259in}}%
\pgfpathlineto{\pgfqpoint{1.328237in}{0.560891in}}%
\pgfpathlineto{\pgfqpoint{1.326466in}{0.563071in}}%
\pgfpathlineto{\pgfqpoint{1.324925in}{0.560891in}}%
\pgfpathclose%
\pgfusepath{fill}%
\end{pgfscope}%
\begin{pgfscope}%
\pgfpathrectangle{\pgfqpoint{0.211875in}{0.211875in}}{\pgfqpoint{1.313625in}{1.279725in}}%
\pgfusepath{clip}%
\pgfsetbuttcap%
\pgfsetroundjoin%
\definecolor{currentfill}{rgb}{0.973832,0.856556,0.771584}%
\pgfsetfillcolor{currentfill}%
\pgfsetlinewidth{0.000000pt}%
\definecolor{currentstroke}{rgb}{0.000000,0.000000,0.000000}%
\pgfsetstrokecolor{currentstroke}%
\pgfsetdash{}{0pt}%
\pgfpathmoveto{\pgfqpoint{0.795708in}{0.635365in}}%
\pgfpathlineto{\pgfqpoint{0.797770in}{0.638450in}}%
\pgfpathlineto{\pgfqpoint{0.795708in}{0.642243in}}%
\pgfpathlineto{\pgfqpoint{0.794135in}{0.638450in}}%
\pgfpathclose%
\pgfusepath{fill}%
\end{pgfscope}%
\begin{pgfscope}%
\pgfpathrectangle{\pgfqpoint{0.211875in}{0.211875in}}{\pgfqpoint{1.313625in}{1.279725in}}%
\pgfusepath{clip}%
\pgfsetbuttcap%
\pgfsetroundjoin%
\definecolor{currentfill}{rgb}{0.973832,0.856556,0.771584}%
\pgfsetfillcolor{currentfill}%
\pgfsetlinewidth{0.000000pt}%
\definecolor{currentstroke}{rgb}{0.000000,0.000000,0.000000}%
\pgfsetstrokecolor{currentstroke}%
\pgfsetdash{}{0pt}%
\pgfpathmoveto{\pgfqpoint{0.915129in}{0.630587in}}%
\pgfpathlineto{\pgfqpoint{0.919290in}{0.638450in}}%
\pgfpathlineto{\pgfqpoint{0.915129in}{0.648117in}}%
\pgfpathlineto{\pgfqpoint{0.910115in}{0.638450in}}%
\pgfpathclose%
\pgfusepath{fill}%
\end{pgfscope}%
\begin{pgfscope}%
\pgfpathrectangle{\pgfqpoint{0.211875in}{0.211875in}}{\pgfqpoint{1.313625in}{1.279725in}}%
\pgfusepath{clip}%
\pgfsetbuttcap%
\pgfsetroundjoin%
\definecolor{currentfill}{rgb}{0.973832,0.856556,0.771584}%
\pgfsetfillcolor{currentfill}%
\pgfsetlinewidth{0.000000pt}%
\definecolor{currentstroke}{rgb}{0.000000,0.000000,0.000000}%
\pgfsetstrokecolor{currentstroke}%
\pgfsetdash{}{0pt}%
\pgfpathmoveto{\pgfqpoint{1.034549in}{0.628305in}}%
\pgfpathlineto{\pgfqpoint{1.038991in}{0.638450in}}%
\pgfpathlineto{\pgfqpoint{1.034549in}{0.650936in}}%
\pgfpathlineto{\pgfqpoint{1.025895in}{0.638450in}}%
\pgfpathclose%
\pgfusepath{fill}%
\end{pgfscope}%
\begin{pgfscope}%
\pgfpathrectangle{\pgfqpoint{0.211875in}{0.211875in}}{\pgfqpoint{1.313625in}{1.279725in}}%
\pgfusepath{clip}%
\pgfsetbuttcap%
\pgfsetroundjoin%
\definecolor{currentfill}{rgb}{0.973832,0.856556,0.771584}%
\pgfsetfillcolor{currentfill}%
\pgfsetlinewidth{0.000000pt}%
\definecolor{currentstroke}{rgb}{0.000000,0.000000,0.000000}%
\pgfsetstrokecolor{currentstroke}%
\pgfsetdash{}{0pt}%
\pgfpathmoveto{\pgfqpoint{1.153970in}{0.628519in}}%
\pgfpathlineto{\pgfqpoint{1.157668in}{0.638450in}}%
\pgfpathlineto{\pgfqpoint{1.153970in}{0.650698in}}%
\pgfpathlineto{\pgfqpoint{1.141045in}{0.638450in}}%
\pgfpathclose%
\pgfusepath{fill}%
\end{pgfscope}%
\begin{pgfscope}%
\pgfpathrectangle{\pgfqpoint{0.211875in}{0.211875in}}{\pgfqpoint{1.313625in}{1.279725in}}%
\pgfusepath{clip}%
\pgfsetbuttcap%
\pgfsetroundjoin%
\definecolor{currentfill}{rgb}{0.973832,0.856556,0.771584}%
\pgfsetfillcolor{currentfill}%
\pgfsetlinewidth{0.000000pt}%
\definecolor{currentstroke}{rgb}{0.000000,0.000000,0.000000}%
\pgfsetstrokecolor{currentstroke}%
\pgfsetdash{}{0pt}%
\pgfpathmoveto{\pgfqpoint{1.260121in}{0.635862in}}%
\pgfpathlineto{\pgfqpoint{1.273390in}{0.631320in}}%
\pgfpathlineto{\pgfqpoint{1.275693in}{0.638450in}}%
\pgfpathlineto{\pgfqpoint{1.273390in}{0.647272in}}%
\pgfpathlineto{\pgfqpoint{1.260121in}{0.641680in}}%
\pgfpathlineto{\pgfqpoint{1.259441in}{0.638450in}}%
\pgfpathclose%
\pgfusepath{fill}%
\end{pgfscope}%
\begin{pgfscope}%
\pgfpathrectangle{\pgfqpoint{0.211875in}{0.211875in}}{\pgfqpoint{1.313625in}{1.279725in}}%
\pgfusepath{clip}%
\pgfsetbuttcap%
\pgfsetroundjoin%
\definecolor{currentfill}{rgb}{0.973832,0.856556,0.771584}%
\pgfsetfillcolor{currentfill}%
\pgfsetlinewidth{0.000000pt}%
\definecolor{currentstroke}{rgb}{0.000000,0.000000,0.000000}%
\pgfsetstrokecolor{currentstroke}%
\pgfsetdash{}{0pt}%
\pgfpathmoveto{\pgfqpoint{1.379542in}{0.635797in}}%
\pgfpathlineto{\pgfqpoint{1.392811in}{0.636895in}}%
\pgfpathlineto{\pgfqpoint{1.393252in}{0.638450in}}%
\pgfpathlineto{\pgfqpoint{1.392811in}{0.640383in}}%
\pgfpathlineto{\pgfqpoint{1.379542in}{0.641743in}}%
\pgfpathlineto{\pgfqpoint{1.378752in}{0.638450in}}%
\pgfpathclose%
\pgfusepath{fill}%
\end{pgfscope}%
\begin{pgfscope}%
\pgfpathrectangle{\pgfqpoint{0.211875in}{0.211875in}}{\pgfqpoint{1.313625in}{1.279725in}}%
\pgfusepath{clip}%
\pgfsetbuttcap%
\pgfsetroundjoin%
\definecolor{currentfill}{rgb}{0.973832,0.856556,0.771584}%
\pgfsetfillcolor{currentfill}%
\pgfsetlinewidth{0.000000pt}%
\definecolor{currentstroke}{rgb}{0.000000,0.000000,0.000000}%
\pgfsetstrokecolor{currentstroke}%
\pgfsetdash{}{0pt}%
\pgfpathmoveto{\pgfqpoint{1.498962in}{0.638324in}}%
\pgfpathlineto{\pgfqpoint{1.499208in}{0.638450in}}%
\pgfpathlineto{\pgfqpoint{1.498962in}{0.638606in}}%
\pgfpathlineto{\pgfqpoint{1.498919in}{0.638450in}}%
\pgfpathclose%
\pgfusepath{fill}%
\end{pgfscope}%
\begin{pgfscope}%
\pgfpathrectangle{\pgfqpoint{0.211875in}{0.211875in}}{\pgfqpoint{1.313625in}{1.279725in}}%
\pgfusepath{clip}%
\pgfsetbuttcap%
\pgfsetroundjoin%
\definecolor{currentfill}{rgb}{0.973832,0.856556,0.771584}%
\pgfsetfillcolor{currentfill}%
\pgfsetlinewidth{0.000000pt}%
\definecolor{currentstroke}{rgb}{0.000000,0.000000,0.000000}%
\pgfsetstrokecolor{currentstroke}%
\pgfsetdash{}{0pt}%
\pgfpathmoveto{\pgfqpoint{0.742633in}{0.715439in}}%
\pgfpathlineto{\pgfqpoint{0.742919in}{0.716009in}}%
\pgfpathlineto{\pgfqpoint{0.742633in}{0.718723in}}%
\pgfpathlineto{\pgfqpoint{0.740819in}{0.716009in}}%
\pgfpathclose%
\pgfusepath{fill}%
\end{pgfscope}%
\begin{pgfscope}%
\pgfpathrectangle{\pgfqpoint{0.211875in}{0.211875in}}{\pgfqpoint{1.313625in}{1.279725in}}%
\pgfusepath{clip}%
\pgfsetbuttcap%
\pgfsetroundjoin%
\definecolor{currentfill}{rgb}{0.973832,0.856556,0.771584}%
\pgfsetfillcolor{currentfill}%
\pgfsetlinewidth{0.000000pt}%
\definecolor{currentstroke}{rgb}{0.000000,0.000000,0.000000}%
\pgfsetstrokecolor{currentstroke}%
\pgfsetdash{}{0pt}%
\pgfpathmoveto{\pgfqpoint{0.848784in}{0.713402in}}%
\pgfpathlineto{\pgfqpoint{0.862053in}{0.713060in}}%
\pgfpathlineto{\pgfqpoint{0.863347in}{0.716009in}}%
\pgfpathlineto{\pgfqpoint{0.862168in}{0.728936in}}%
\pgfpathlineto{\pgfqpoint{0.862053in}{0.729119in}}%
\pgfpathlineto{\pgfqpoint{0.852373in}{0.728936in}}%
\pgfpathlineto{\pgfqpoint{0.848784in}{0.728493in}}%
\pgfpathlineto{\pgfqpoint{0.847665in}{0.716009in}}%
\pgfpathclose%
\pgfusepath{fill}%
\end{pgfscope}%
\begin{pgfscope}%
\pgfpathrectangle{\pgfqpoint{0.211875in}{0.211875in}}{\pgfqpoint{1.313625in}{1.279725in}}%
\pgfusepath{clip}%
\pgfsetbuttcap%
\pgfsetroundjoin%
\definecolor{currentfill}{rgb}{0.973832,0.856556,0.771584}%
\pgfsetfillcolor{currentfill}%
\pgfsetlinewidth{0.000000pt}%
\definecolor{currentstroke}{rgb}{0.000000,0.000000,0.000000}%
\pgfsetstrokecolor{currentstroke}%
\pgfsetdash{}{0pt}%
\pgfpathmoveto{\pgfqpoint{0.968205in}{0.709082in}}%
\pgfpathlineto{\pgfqpoint{0.981473in}{0.712386in}}%
\pgfpathlineto{\pgfqpoint{0.982877in}{0.716009in}}%
\pgfpathlineto{\pgfqpoint{0.981859in}{0.728936in}}%
\pgfpathlineto{\pgfqpoint{0.981473in}{0.729635in}}%
\pgfpathlineto{\pgfqpoint{0.968205in}{0.732034in}}%
\pgfpathlineto{\pgfqpoint{0.966035in}{0.728936in}}%
\pgfpathlineto{\pgfqpoint{0.964814in}{0.716009in}}%
\pgfpathclose%
\pgfusepath{fill}%
\end{pgfscope}%
\begin{pgfscope}%
\pgfpathrectangle{\pgfqpoint{0.211875in}{0.211875in}}{\pgfqpoint{1.313625in}{1.279725in}}%
\pgfusepath{clip}%
\pgfsetbuttcap%
\pgfsetroundjoin%
\definecolor{currentfill}{rgb}{0.973832,0.856556,0.771584}%
\pgfsetfillcolor{currentfill}%
\pgfsetlinewidth{0.000000pt}%
\definecolor{currentstroke}{rgb}{0.000000,0.000000,0.000000}%
\pgfsetstrokecolor{currentstroke}%
\pgfsetdash{}{0pt}%
\pgfpathmoveto{\pgfqpoint{1.087625in}{0.706640in}}%
\pgfpathlineto{\pgfqpoint{1.100894in}{0.713562in}}%
\pgfpathlineto{\pgfqpoint{1.101736in}{0.716009in}}%
\pgfpathlineto{\pgfqpoint{1.100894in}{0.727950in}}%
\pgfpathlineto{\pgfqpoint{1.100544in}{0.728936in}}%
\pgfpathlineto{\pgfqpoint{1.087625in}{0.733816in}}%
\pgfpathlineto{\pgfqpoint{1.083668in}{0.728936in}}%
\pgfpathlineto{\pgfqpoint{1.082325in}{0.716009in}}%
\pgfpathclose%
\pgfusepath{fill}%
\end{pgfscope}%
\begin{pgfscope}%
\pgfpathrectangle{\pgfqpoint{0.211875in}{0.211875in}}{\pgfqpoint{1.313625in}{1.279725in}}%
\pgfusepath{clip}%
\pgfsetbuttcap%
\pgfsetroundjoin%
\definecolor{currentfill}{rgb}{0.973832,0.856556,0.771584}%
\pgfsetfillcolor{currentfill}%
\pgfsetlinewidth{0.000000pt}%
\definecolor{currentstroke}{rgb}{0.000000,0.000000,0.000000}%
\pgfsetstrokecolor{currentstroke}%
\pgfsetdash{}{0pt}%
\pgfpathmoveto{\pgfqpoint{1.207045in}{0.705938in}}%
\pgfpathlineto{\pgfqpoint{1.219465in}{0.716009in}}%
\pgfpathlineto{\pgfqpoint{1.216571in}{0.728936in}}%
\pgfpathlineto{\pgfqpoint{1.207045in}{0.734322in}}%
\pgfpathlineto{\pgfqpoint{1.201892in}{0.728936in}}%
\pgfpathlineto{\pgfqpoint{1.200330in}{0.716009in}}%
\pgfpathclose%
\pgfusepath{fill}%
\end{pgfscope}%
\begin{pgfscope}%
\pgfpathrectangle{\pgfqpoint{0.211875in}{0.211875in}}{\pgfqpoint{1.313625in}{1.279725in}}%
\pgfusepath{clip}%
\pgfsetbuttcap%
\pgfsetroundjoin%
\definecolor{currentfill}{rgb}{0.973832,0.856556,0.771584}%
\pgfsetfillcolor{currentfill}%
\pgfsetlinewidth{0.000000pt}%
\definecolor{currentstroke}{rgb}{0.000000,0.000000,0.000000}%
\pgfsetstrokecolor{currentstroke}%
\pgfsetdash{}{0pt}%
\pgfpathmoveto{\pgfqpoint{1.326466in}{0.706906in}}%
\pgfpathlineto{\pgfqpoint{1.334932in}{0.716009in}}%
\pgfpathlineto{\pgfqpoint{1.332696in}{0.728936in}}%
\pgfpathlineto{\pgfqpoint{1.326466in}{0.733606in}}%
\pgfpathlineto{\pgfqpoint{1.321038in}{0.728936in}}%
\pgfpathlineto{\pgfqpoint{1.319100in}{0.716009in}}%
\pgfpathclose%
\pgfusepath{fill}%
\end{pgfscope}%
\begin{pgfscope}%
\pgfpathrectangle{\pgfqpoint{0.211875in}{0.211875in}}{\pgfqpoint{1.313625in}{1.279725in}}%
\pgfusepath{clip}%
\pgfsetbuttcap%
\pgfsetroundjoin%
\definecolor{currentfill}{rgb}{0.973832,0.856556,0.771584}%
\pgfsetfillcolor{currentfill}%
\pgfsetlinewidth{0.000000pt}%
\definecolor{currentstroke}{rgb}{0.000000,0.000000,0.000000}%
\pgfsetstrokecolor{currentstroke}%
\pgfsetdash{}{0pt}%
\pgfpathmoveto{\pgfqpoint{1.445886in}{0.709537in}}%
\pgfpathlineto{\pgfqpoint{1.450723in}{0.716009in}}%
\pgfpathlineto{\pgfqpoint{1.448822in}{0.728936in}}%
\pgfpathlineto{\pgfqpoint{1.445886in}{0.731676in}}%
\pgfpathlineto{\pgfqpoint{1.441839in}{0.728936in}}%
\pgfpathlineto{\pgfqpoint{1.439235in}{0.716009in}}%
\pgfpathclose%
\pgfusepath{fill}%
\end{pgfscope}%
\begin{pgfscope}%
\pgfpathrectangle{\pgfqpoint{0.211875in}{0.211875in}}{\pgfqpoint{1.313625in}{1.279725in}}%
\pgfusepath{clip}%
\pgfsetbuttcap%
\pgfsetroundjoin%
\definecolor{currentfill}{rgb}{0.973832,0.856556,0.771584}%
\pgfsetfillcolor{currentfill}%
\pgfsetlinewidth{0.000000pt}%
\definecolor{currentstroke}{rgb}{0.000000,0.000000,0.000000}%
\pgfsetstrokecolor{currentstroke}%
\pgfsetdash{}{0pt}%
\pgfpathmoveto{\pgfqpoint{0.676288in}{0.793105in}}%
\pgfpathlineto{\pgfqpoint{0.677138in}{0.793568in}}%
\pgfpathlineto{\pgfqpoint{0.679450in}{0.806495in}}%
\pgfpathlineto{\pgfqpoint{0.676288in}{0.808647in}}%
\pgfpathlineto{\pgfqpoint{0.674814in}{0.806495in}}%
\pgfpathlineto{\pgfqpoint{0.675893in}{0.793568in}}%
\pgfpathclose%
\pgfusepath{fill}%
\end{pgfscope}%
\begin{pgfscope}%
\pgfpathrectangle{\pgfqpoint{0.211875in}{0.211875in}}{\pgfqpoint{1.313625in}{1.279725in}}%
\pgfusepath{clip}%
\pgfsetbuttcap%
\pgfsetroundjoin%
\definecolor{currentfill}{rgb}{0.973832,0.856556,0.771584}%
\pgfsetfillcolor{currentfill}%
\pgfsetlinewidth{0.000000pt}%
\definecolor{currentstroke}{rgb}{0.000000,0.000000,0.000000}%
\pgfsetstrokecolor{currentstroke}%
\pgfsetdash{}{0pt}%
\pgfpathmoveto{\pgfqpoint{0.795708in}{0.789324in}}%
\pgfpathlineto{\pgfqpoint{0.801432in}{0.793568in}}%
\pgfpathlineto{\pgfqpoint{0.803023in}{0.806495in}}%
\pgfpathlineto{\pgfqpoint{0.795708in}{0.813266in}}%
\pgfpathlineto{\pgfqpoint{0.790135in}{0.806495in}}%
\pgfpathlineto{\pgfqpoint{0.791352in}{0.793568in}}%
\pgfpathclose%
\pgfusepath{fill}%
\end{pgfscope}%
\begin{pgfscope}%
\pgfpathrectangle{\pgfqpoint{0.211875in}{0.211875in}}{\pgfqpoint{1.313625in}{1.279725in}}%
\pgfusepath{clip}%
\pgfsetbuttcap%
\pgfsetroundjoin%
\definecolor{currentfill}{rgb}{0.973832,0.856556,0.771584}%
\pgfsetfillcolor{currentfill}%
\pgfsetlinewidth{0.000000pt}%
\definecolor{currentstroke}{rgb}{0.000000,0.000000,0.000000}%
\pgfsetstrokecolor{currentstroke}%
\pgfsetdash{}{0pt}%
\pgfpathmoveto{\pgfqpoint{0.915129in}{0.786850in}}%
\pgfpathlineto{\pgfqpoint{0.922299in}{0.793568in}}%
\pgfpathlineto{\pgfqpoint{0.923506in}{0.806495in}}%
\pgfpathlineto{\pgfqpoint{0.915129in}{0.816290in}}%
\pgfpathlineto{\pgfqpoint{0.905044in}{0.806495in}}%
\pgfpathlineto{\pgfqpoint{0.906503in}{0.793568in}}%
\pgfpathclose%
\pgfusepath{fill}%
\end{pgfscope}%
\begin{pgfscope}%
\pgfpathrectangle{\pgfqpoint{0.211875in}{0.211875in}}{\pgfqpoint{1.313625in}{1.279725in}}%
\pgfusepath{clip}%
\pgfsetbuttcap%
\pgfsetroundjoin%
\definecolor{currentfill}{rgb}{0.973832,0.856556,0.771584}%
\pgfsetfillcolor{currentfill}%
\pgfsetlinewidth{0.000000pt}%
\definecolor{currentstroke}{rgb}{0.000000,0.000000,0.000000}%
\pgfsetstrokecolor{currentstroke}%
\pgfsetdash{}{0pt}%
\pgfpathmoveto{\pgfqpoint{1.021280in}{0.793322in}}%
\pgfpathlineto{\pgfqpoint{1.034549in}{0.785636in}}%
\pgfpathlineto{\pgfqpoint{1.041548in}{0.793568in}}%
\pgfpathlineto{\pgfqpoint{1.042526in}{0.806495in}}%
\pgfpathlineto{\pgfqpoint{1.034549in}{0.817780in}}%
\pgfpathlineto{\pgfqpoint{1.021280in}{0.808468in}}%
\pgfpathlineto{\pgfqpoint{1.020628in}{0.806495in}}%
\pgfpathlineto{\pgfqpoint{1.021178in}{0.793568in}}%
\pgfpathclose%
\pgfusepath{fill}%
\end{pgfscope}%
\begin{pgfscope}%
\pgfpathrectangle{\pgfqpoint{0.211875in}{0.211875in}}{\pgfqpoint{1.313625in}{1.279725in}}%
\pgfusepath{clip}%
\pgfsetbuttcap%
\pgfsetroundjoin%
\definecolor{currentfill}{rgb}{0.973832,0.856556,0.771584}%
\pgfsetfillcolor{currentfill}%
\pgfsetlinewidth{0.000000pt}%
\definecolor{currentstroke}{rgb}{0.000000,0.000000,0.000000}%
\pgfsetstrokecolor{currentstroke}%
\pgfsetdash{}{0pt}%
\pgfpathmoveto{\pgfqpoint{1.140701in}{0.790417in}}%
\pgfpathlineto{\pgfqpoint{1.153970in}{0.785682in}}%
\pgfpathlineto{\pgfqpoint{1.159884in}{0.793568in}}%
\pgfpathlineto{\pgfqpoint{1.160720in}{0.806495in}}%
\pgfpathlineto{\pgfqpoint{1.153970in}{0.817736in}}%
\pgfpathlineto{\pgfqpoint{1.140701in}{0.812019in}}%
\pgfpathlineto{\pgfqpoint{1.138643in}{0.806495in}}%
\pgfpathlineto{\pgfqpoint{1.139226in}{0.793568in}}%
\pgfpathclose%
\pgfusepath{fill}%
\end{pgfscope}%
\begin{pgfscope}%
\pgfpathrectangle{\pgfqpoint{0.211875in}{0.211875in}}{\pgfqpoint{1.313625in}{1.279725in}}%
\pgfusepath{clip}%
\pgfsetbuttcap%
\pgfsetroundjoin%
\definecolor{currentfill}{rgb}{0.973832,0.856556,0.771584}%
\pgfsetfillcolor{currentfill}%
\pgfsetlinewidth{0.000000pt}%
\definecolor{currentstroke}{rgb}{0.000000,0.000000,0.000000}%
\pgfsetstrokecolor{currentstroke}%
\pgfsetdash{}{0pt}%
\pgfpathmoveto{\pgfqpoint{1.260121in}{0.789127in}}%
\pgfpathlineto{\pgfqpoint{1.273390in}{0.787035in}}%
\pgfpathlineto{\pgfqpoint{1.277633in}{0.793568in}}%
\pgfpathlineto{\pgfqpoint{1.278379in}{0.806495in}}%
\pgfpathlineto{\pgfqpoint{1.273390in}{0.816098in}}%
\pgfpathlineto{\pgfqpoint{1.260121in}{0.813576in}}%
\pgfpathlineto{\pgfqpoint{1.257142in}{0.806495in}}%
\pgfpathlineto{\pgfqpoint{1.257778in}{0.793568in}}%
\pgfpathclose%
\pgfusepath{fill}%
\end{pgfscope}%
\begin{pgfscope}%
\pgfpathrectangle{\pgfqpoint{0.211875in}{0.211875in}}{\pgfqpoint{1.313625in}{1.279725in}}%
\pgfusepath{clip}%
\pgfsetbuttcap%
\pgfsetroundjoin%
\definecolor{currentfill}{rgb}{0.973832,0.856556,0.771584}%
\pgfsetfillcolor{currentfill}%
\pgfsetlinewidth{0.000000pt}%
\definecolor{currentstroke}{rgb}{0.000000,0.000000,0.000000}%
\pgfsetstrokecolor{currentstroke}%
\pgfsetdash{}{0pt}%
\pgfpathmoveto{\pgfqpoint{1.379542in}{0.789259in}}%
\pgfpathlineto{\pgfqpoint{1.392811in}{0.789794in}}%
\pgfpathlineto{\pgfqpoint{1.394960in}{0.793568in}}%
\pgfpathlineto{\pgfqpoint{1.395651in}{0.806495in}}%
\pgfpathlineto{\pgfqpoint{1.392811in}{0.812738in}}%
\pgfpathlineto{\pgfqpoint{1.379542in}{0.813389in}}%
\pgfpathlineto{\pgfqpoint{1.376246in}{0.806495in}}%
\pgfpathlineto{\pgfqpoint{1.376963in}{0.793568in}}%
\pgfpathclose%
\pgfusepath{fill}%
\end{pgfscope}%
\begin{pgfscope}%
\pgfpathrectangle{\pgfqpoint{0.211875in}{0.211875in}}{\pgfqpoint{1.313625in}{1.279725in}}%
\pgfusepath{clip}%
\pgfsetbuttcap%
\pgfsetroundjoin%
\definecolor{currentfill}{rgb}{0.973832,0.856556,0.771584}%
\pgfsetfillcolor{currentfill}%
\pgfsetlinewidth{0.000000pt}%
\definecolor{currentstroke}{rgb}{0.000000,0.000000,0.000000}%
\pgfsetstrokecolor{currentstroke}%
\pgfsetdash{}{0pt}%
\pgfpathmoveto{\pgfqpoint{1.498962in}{0.790693in}}%
\pgfpathlineto{\pgfqpoint{1.510200in}{0.793568in}}%
\pgfpathlineto{\pgfqpoint{1.512231in}{0.798976in}}%
\pgfpathlineto{\pgfqpoint{1.512612in}{0.806495in}}%
\pgfpathlineto{\pgfqpoint{1.512231in}{0.807442in}}%
\pgfpathlineto{\pgfqpoint{1.498962in}{0.811613in}}%
\pgfpathlineto{\pgfqpoint{1.496155in}{0.806495in}}%
\pgfpathlineto{\pgfqpoint{1.496991in}{0.793568in}}%
\pgfpathclose%
\pgfusepath{fill}%
\end{pgfscope}%
\begin{pgfscope}%
\pgfpathrectangle{\pgfqpoint{0.211875in}{0.211875in}}{\pgfqpoint{1.313625in}{1.279725in}}%
\pgfusepath{clip}%
\pgfsetbuttcap%
\pgfsetroundjoin%
\definecolor{currentfill}{rgb}{0.973832,0.856556,0.771584}%
\pgfsetfillcolor{currentfill}%
\pgfsetlinewidth{0.000000pt}%
\definecolor{currentstroke}{rgb}{0.000000,0.000000,0.000000}%
\pgfsetstrokecolor{currentstroke}%
\pgfsetdash{}{0pt}%
\pgfpathmoveto{\pgfqpoint{0.742633in}{0.870796in}}%
\pgfpathlineto{\pgfqpoint{0.742920in}{0.871127in}}%
\pgfpathlineto{\pgfqpoint{0.745926in}{0.884054in}}%
\pgfpathlineto{\pgfqpoint{0.742633in}{0.892896in}}%
\pgfpathlineto{\pgfqpoint{0.729364in}{0.887486in}}%
\pgfpathlineto{\pgfqpoint{0.728401in}{0.884054in}}%
\pgfpathlineto{\pgfqpoint{0.729364in}{0.878633in}}%
\pgfpathlineto{\pgfqpoint{0.740835in}{0.871127in}}%
\pgfpathclose%
\pgfusepath{fill}%
\end{pgfscope}%
\begin{pgfscope}%
\pgfpathrectangle{\pgfqpoint{0.211875in}{0.211875in}}{\pgfqpoint{1.313625in}{1.279725in}}%
\pgfusepath{clip}%
\pgfsetbuttcap%
\pgfsetroundjoin%
\definecolor{currentfill}{rgb}{0.973832,0.856556,0.771584}%
\pgfsetfillcolor{currentfill}%
\pgfsetlinewidth{0.000000pt}%
\definecolor{currentstroke}{rgb}{0.000000,0.000000,0.000000}%
\pgfsetstrokecolor{currentstroke}%
\pgfsetdash{}{0pt}%
\pgfpathmoveto{\pgfqpoint{0.848784in}{0.869423in}}%
\pgfpathlineto{\pgfqpoint{0.862053in}{0.869205in}}%
\pgfpathlineto{\pgfqpoint{0.863507in}{0.871127in}}%
\pgfpathlineto{\pgfqpoint{0.866027in}{0.884054in}}%
\pgfpathlineto{\pgfqpoint{0.862053in}{0.896280in}}%
\pgfpathlineto{\pgfqpoint{0.848784in}{0.895818in}}%
\pgfpathlineto{\pgfqpoint{0.845043in}{0.884054in}}%
\pgfpathlineto{\pgfqpoint{0.847526in}{0.871127in}}%
\pgfpathclose%
\pgfusepath{fill}%
\end{pgfscope}%
\begin{pgfscope}%
\pgfpathrectangle{\pgfqpoint{0.211875in}{0.211875in}}{\pgfqpoint{1.313625in}{1.279725in}}%
\pgfusepath{clip}%
\pgfsetbuttcap%
\pgfsetroundjoin%
\definecolor{currentfill}{rgb}{0.973832,0.856556,0.771584}%
\pgfsetfillcolor{currentfill}%
\pgfsetlinewidth{0.000000pt}%
\definecolor{currentstroke}{rgb}{0.000000,0.000000,0.000000}%
\pgfsetstrokecolor{currentstroke}%
\pgfsetdash{}{0pt}%
\pgfpathmoveto{\pgfqpoint{0.968205in}{0.866782in}}%
\pgfpathlineto{\pgfqpoint{0.981473in}{0.868663in}}%
\pgfpathlineto{\pgfqpoint{0.983117in}{0.871127in}}%
\pgfpathlineto{\pgfqpoint{0.985312in}{0.884054in}}%
\pgfpathlineto{\pgfqpoint{0.981636in}{0.896980in}}%
\pgfpathlineto{\pgfqpoint{0.981473in}{0.897182in}}%
\pgfpathlineto{\pgfqpoint{0.968205in}{0.898831in}}%
\pgfpathlineto{\pgfqpoint{0.966316in}{0.896980in}}%
\pgfpathlineto{\pgfqpoint{0.961909in}{0.884054in}}%
\pgfpathlineto{\pgfqpoint{0.964541in}{0.871127in}}%
\pgfpathclose%
\pgfusepath{fill}%
\end{pgfscope}%
\begin{pgfscope}%
\pgfpathrectangle{\pgfqpoint{0.211875in}{0.211875in}}{\pgfqpoint{1.313625in}{1.279725in}}%
\pgfusepath{clip}%
\pgfsetbuttcap%
\pgfsetroundjoin%
\definecolor{currentfill}{rgb}{0.973832,0.856556,0.771584}%
\pgfsetfillcolor{currentfill}%
\pgfsetlinewidth{0.000000pt}%
\definecolor{currentstroke}{rgb}{0.000000,0.000000,0.000000}%
\pgfsetstrokecolor{currentstroke}%
\pgfsetdash{}{0pt}%
\pgfpathmoveto{\pgfqpoint{1.087625in}{0.865303in}}%
\pgfpathlineto{\pgfqpoint{1.100894in}{0.869259in}}%
\pgfpathlineto{\pgfqpoint{1.101999in}{0.871127in}}%
\pgfpathlineto{\pgfqpoint{1.103983in}{0.884054in}}%
\pgfpathlineto{\pgfqpoint{1.100894in}{0.896237in}}%
\pgfpathlineto{\pgfqpoint{1.099826in}{0.896980in}}%
\pgfpathlineto{\pgfqpoint{1.087625in}{0.900136in}}%
\pgfpathlineto{\pgfqpoint{1.083894in}{0.896980in}}%
\pgfpathlineto{\pgfqpoint{1.079033in}{0.884054in}}%
\pgfpathlineto{\pgfqpoint{1.081944in}{0.871127in}}%
\pgfpathclose%
\pgfusepath{fill}%
\end{pgfscope}%
\begin{pgfscope}%
\pgfpathrectangle{\pgfqpoint{0.211875in}{0.211875in}}{\pgfqpoint{1.313625in}{1.279725in}}%
\pgfusepath{clip}%
\pgfsetbuttcap%
\pgfsetroundjoin%
\definecolor{currentfill}{rgb}{0.973832,0.856556,0.771584}%
\pgfsetfillcolor{currentfill}%
\pgfsetlinewidth{0.000000pt}%
\definecolor{currentstroke}{rgb}{0.000000,0.000000,0.000000}%
\pgfsetstrokecolor{currentstroke}%
\pgfsetdash{}{0pt}%
\pgfpathmoveto{\pgfqpoint{1.207045in}{0.864901in}}%
\pgfpathlineto{\pgfqpoint{1.220284in}{0.871127in}}%
\pgfpathlineto{\pgfqpoint{1.220314in}{0.871185in}}%
\pgfpathlineto{\pgfqpoint{1.222165in}{0.884054in}}%
\pgfpathlineto{\pgfqpoint{1.220314in}{0.892271in}}%
\pgfpathlineto{\pgfqpoint{1.216085in}{0.896980in}}%
\pgfpathlineto{\pgfqpoint{1.207045in}{0.900487in}}%
\pgfpathlineto{\pgfqpoint{1.202149in}{0.896980in}}%
\pgfpathlineto{\pgfqpoint{1.196497in}{0.884054in}}%
\pgfpathlineto{\pgfqpoint{1.199881in}{0.871127in}}%
\pgfpathclose%
\pgfusepath{fill}%
\end{pgfscope}%
\begin{pgfscope}%
\pgfpathrectangle{\pgfqpoint{0.211875in}{0.211875in}}{\pgfqpoint{1.313625in}{1.279725in}}%
\pgfusepath{clip}%
\pgfsetbuttcap%
\pgfsetroundjoin%
\definecolor{currentfill}{rgb}{0.973832,0.856556,0.771584}%
\pgfsetfillcolor{currentfill}%
\pgfsetlinewidth{0.000000pt}%
\definecolor{currentstroke}{rgb}{0.000000,0.000000,0.000000}%
\pgfsetstrokecolor{currentstroke}%
\pgfsetdash{}{0pt}%
\pgfpathmoveto{\pgfqpoint{1.326466in}{0.865531in}}%
\pgfpathlineto{\pgfqpoint{1.335434in}{0.871127in}}%
\pgfpathlineto{\pgfqpoint{1.339735in}{0.882521in}}%
\pgfpathlineto{\pgfqpoint{1.339932in}{0.884054in}}%
\pgfpathlineto{\pgfqpoint{1.339735in}{0.885038in}}%
\pgfpathlineto{\pgfqpoint{1.332184in}{0.896980in}}%
\pgfpathlineto{\pgfqpoint{1.326466in}{0.899925in}}%
\pgfpathlineto{\pgfqpoint{1.321467in}{0.896980in}}%
\pgfpathlineto{\pgfqpoint{1.314466in}{0.884054in}}%
\pgfpathlineto{\pgfqpoint{1.318644in}{0.871127in}}%
\pgfpathclose%
\pgfusepath{fill}%
\end{pgfscope}%
\begin{pgfscope}%
\pgfpathrectangle{\pgfqpoint{0.211875in}{0.211875in}}{\pgfqpoint{1.313625in}{1.279725in}}%
\pgfusepath{clip}%
\pgfsetbuttcap%
\pgfsetroundjoin%
\definecolor{currentfill}{rgb}{0.973832,0.856556,0.771584}%
\pgfsetfillcolor{currentfill}%
\pgfsetlinewidth{0.000000pt}%
\definecolor{currentstroke}{rgb}{0.000000,0.000000,0.000000}%
\pgfsetstrokecolor{currentstroke}%
\pgfsetdash{}{0pt}%
\pgfpathmoveto{\pgfqpoint{1.445886in}{0.867189in}}%
\pgfpathlineto{\pgfqpoint{1.450955in}{0.871127in}}%
\pgfpathlineto{\pgfqpoint{1.455026in}{0.884054in}}%
\pgfpathlineto{\pgfqpoint{1.448182in}{0.896980in}}%
\pgfpathlineto{\pgfqpoint{1.445886in}{0.898453in}}%
\pgfpathlineto{\pgfqpoint{1.442703in}{0.896980in}}%
\pgfpathlineto{\pgfqpoint{1.433322in}{0.884054in}}%
\pgfpathlineto{\pgfqpoint{1.438886in}{0.871127in}}%
\pgfpathclose%
\pgfusepath{fill}%
\end{pgfscope}%
\begin{pgfscope}%
\pgfpathrectangle{\pgfqpoint{0.211875in}{0.211875in}}{\pgfqpoint{1.313625in}{1.279725in}}%
\pgfusepath{clip}%
\pgfsetbuttcap%
\pgfsetroundjoin%
\definecolor{currentfill}{rgb}{0.973832,0.856556,0.771584}%
\pgfsetfillcolor{currentfill}%
\pgfsetlinewidth{0.000000pt}%
\definecolor{currentstroke}{rgb}{0.000000,0.000000,0.000000}%
\pgfsetstrokecolor{currentstroke}%
\pgfsetdash{}{0pt}%
\pgfpathmoveto{\pgfqpoint{0.623212in}{0.878730in}}%
\pgfpathlineto{\pgfqpoint{0.624660in}{0.884054in}}%
\pgfpathlineto{\pgfqpoint{0.623212in}{0.887402in}}%
\pgfpathlineto{\pgfqpoint{0.618998in}{0.884054in}}%
\pgfpathclose%
\pgfusepath{fill}%
\end{pgfscope}%
\begin{pgfscope}%
\pgfpathrectangle{\pgfqpoint{0.211875in}{0.211875in}}{\pgfqpoint{1.313625in}{1.279725in}}%
\pgfusepath{clip}%
\pgfsetbuttcap%
\pgfsetroundjoin%
\definecolor{currentfill}{rgb}{0.973832,0.856556,0.771584}%
\pgfsetfillcolor{currentfill}%
\pgfsetlinewidth{0.000000pt}%
\definecolor{currentstroke}{rgb}{0.000000,0.000000,0.000000}%
\pgfsetstrokecolor{currentstroke}%
\pgfsetdash{}{0pt}%
\pgfpathmoveto{\pgfqpoint{0.795708in}{0.947748in}}%
\pgfpathlineto{\pgfqpoint{0.797746in}{0.948686in}}%
\pgfpathlineto{\pgfqpoint{0.807710in}{0.961613in}}%
\pgfpathlineto{\pgfqpoint{0.802997in}{0.974539in}}%
\pgfpathlineto{\pgfqpoint{0.795708in}{0.978899in}}%
\pgfpathlineto{\pgfqpoint{0.790178in}{0.974539in}}%
\pgfpathlineto{\pgfqpoint{0.786552in}{0.961613in}}%
\pgfpathlineto{\pgfqpoint{0.794171in}{0.948686in}}%
\pgfpathclose%
\pgfusepath{fill}%
\end{pgfscope}%
\begin{pgfscope}%
\pgfpathrectangle{\pgfqpoint{0.211875in}{0.211875in}}{\pgfqpoint{1.313625in}{1.279725in}}%
\pgfusepath{clip}%
\pgfsetbuttcap%
\pgfsetroundjoin%
\definecolor{currentfill}{rgb}{0.973832,0.856556,0.771584}%
\pgfsetfillcolor{currentfill}%
\pgfsetlinewidth{0.000000pt}%
\definecolor{currentstroke}{rgb}{0.000000,0.000000,0.000000}%
\pgfsetstrokecolor{currentstroke}%
\pgfsetdash{}{0pt}%
\pgfpathmoveto{\pgfqpoint{0.915129in}{0.945986in}}%
\pgfpathlineto{\pgfqpoint{0.919758in}{0.948686in}}%
\pgfpathlineto{\pgfqpoint{0.927169in}{0.961613in}}%
\pgfpathlineto{\pgfqpoint{0.923674in}{0.974539in}}%
\pgfpathlineto{\pgfqpoint{0.915129in}{0.981004in}}%
\pgfpathlineto{\pgfqpoint{0.904869in}{0.974539in}}%
\pgfpathlineto{\pgfqpoint{0.901860in}{0.965621in}}%
\pgfpathlineto{\pgfqpoint{0.901407in}{0.961613in}}%
\pgfpathlineto{\pgfqpoint{0.901860in}{0.959555in}}%
\pgfpathlineto{\pgfqpoint{0.909591in}{0.948686in}}%
\pgfpathclose%
\pgfusepath{fill}%
\end{pgfscope}%
\begin{pgfscope}%
\pgfpathrectangle{\pgfqpoint{0.211875in}{0.211875in}}{\pgfqpoint{1.313625in}{1.279725in}}%
\pgfusepath{clip}%
\pgfsetbuttcap%
\pgfsetroundjoin%
\definecolor{currentfill}{rgb}{0.973832,0.856556,0.771584}%
\pgfsetfillcolor{currentfill}%
\pgfsetlinewidth{0.000000pt}%
\definecolor{currentstroke}{rgb}{0.000000,0.000000,0.000000}%
\pgfsetstrokecolor{currentstroke}%
\pgfsetdash{}{0pt}%
\pgfpathmoveto{\pgfqpoint{1.034549in}{0.945107in}}%
\pgfpathlineto{\pgfqpoint{1.039613in}{0.948686in}}%
\pgfpathlineto{\pgfqpoint{1.045552in}{0.961613in}}%
\pgfpathlineto{\pgfqpoint{1.042757in}{0.974539in}}%
\pgfpathlineto{\pgfqpoint{1.034549in}{0.982056in}}%
\pgfpathlineto{\pgfqpoint{1.021280in}{0.976052in}}%
\pgfpathlineto{\pgfqpoint{1.020505in}{0.974539in}}%
\pgfpathlineto{\pgfqpoint{1.018930in}{0.961613in}}%
\pgfpathlineto{\pgfqpoint{1.021280in}{0.952077in}}%
\pgfpathlineto{\pgfqpoint{1.024720in}{0.948686in}}%
\pgfpathclose%
\pgfusepath{fill}%
\end{pgfscope}%
\begin{pgfscope}%
\pgfpathrectangle{\pgfqpoint{0.211875in}{0.211875in}}{\pgfqpoint{1.313625in}{1.279725in}}%
\pgfusepath{clip}%
\pgfsetbuttcap%
\pgfsetroundjoin%
\definecolor{currentfill}{rgb}{0.973832,0.856556,0.771584}%
\pgfsetfillcolor{currentfill}%
\pgfsetlinewidth{0.000000pt}%
\definecolor{currentstroke}{rgb}{0.000000,0.000000,0.000000}%
\pgfsetstrokecolor{currentstroke}%
\pgfsetdash{}{0pt}%
\pgfpathmoveto{\pgfqpoint{1.140701in}{0.948223in}}%
\pgfpathlineto{\pgfqpoint{1.153970in}{0.945111in}}%
\pgfpathlineto{\pgfqpoint{1.158264in}{0.948686in}}%
\pgfpathlineto{\pgfqpoint{1.163320in}{0.961613in}}%
\pgfpathlineto{\pgfqpoint{1.160941in}{0.974539in}}%
\pgfpathlineto{\pgfqpoint{1.153970in}{0.982057in}}%
\pgfpathlineto{\pgfqpoint{1.140701in}{0.978375in}}%
\pgfpathlineto{\pgfqpoint{1.138489in}{0.974539in}}%
\pgfpathlineto{\pgfqpoint{1.136831in}{0.961613in}}%
\pgfpathlineto{\pgfqpoint{1.140354in}{0.948686in}}%
\pgfpathclose%
\pgfusepath{fill}%
\end{pgfscope}%
\begin{pgfscope}%
\pgfpathrectangle{\pgfqpoint{0.211875in}{0.211875in}}{\pgfqpoint{1.313625in}{1.279725in}}%
\pgfusepath{clip}%
\pgfsetbuttcap%
\pgfsetroundjoin%
\definecolor{currentfill}{rgb}{0.973832,0.856556,0.771584}%
\pgfsetfillcolor{currentfill}%
\pgfsetlinewidth{0.000000pt}%
\definecolor{currentstroke}{rgb}{0.000000,0.000000,0.000000}%
\pgfsetstrokecolor{currentstroke}%
\pgfsetdash{}{0pt}%
\pgfpathmoveto{\pgfqpoint{1.260121in}{0.947400in}}%
\pgfpathlineto{\pgfqpoint{1.273390in}{0.946030in}}%
\pgfpathlineto{\pgfqpoint{1.276150in}{0.948686in}}%
\pgfpathlineto{\pgfqpoint{1.280683in}{0.961613in}}%
\pgfpathlineto{\pgfqpoint{1.278546in}{0.974539in}}%
\pgfpathlineto{\pgfqpoint{1.273390in}{0.980966in}}%
\pgfpathlineto{\pgfqpoint{1.260121in}{0.979347in}}%
\pgfpathlineto{\pgfqpoint{1.256992in}{0.974539in}}%
\pgfpathlineto{\pgfqpoint{1.255176in}{0.961613in}}%
\pgfpathlineto{\pgfqpoint{1.259035in}{0.948686in}}%
\pgfpathclose%
\pgfusepath{fill}%
\end{pgfscope}%
\begin{pgfscope}%
\pgfpathrectangle{\pgfqpoint{0.211875in}{0.211875in}}{\pgfqpoint{1.313625in}{1.279725in}}%
\pgfusepath{clip}%
\pgfsetbuttcap%
\pgfsetroundjoin%
\definecolor{currentfill}{rgb}{0.973832,0.856556,0.771584}%
\pgfsetfillcolor{currentfill}%
\pgfsetlinewidth{0.000000pt}%
\definecolor{currentstroke}{rgb}{0.000000,0.000000,0.000000}%
\pgfsetstrokecolor{currentstroke}%
\pgfsetdash{}{0pt}%
\pgfpathmoveto{\pgfqpoint{1.379542in}{0.947569in}}%
\pgfpathlineto{\pgfqpoint{1.392811in}{0.947935in}}%
\pgfpathlineto{\pgfqpoint{1.393495in}{0.948686in}}%
\pgfpathlineto{\pgfqpoint{1.397743in}{0.961613in}}%
\pgfpathlineto{\pgfqpoint{1.395733in}{0.974539in}}%
\pgfpathlineto{\pgfqpoint{1.392811in}{0.978698in}}%
\pgfpathlineto{\pgfqpoint{1.379542in}{0.979133in}}%
\pgfpathlineto{\pgfqpoint{1.376145in}{0.974539in}}%
\pgfpathlineto{\pgfqpoint{1.374069in}{0.961613in}}%
\pgfpathlineto{\pgfqpoint{1.378470in}{0.948686in}}%
\pgfpathclose%
\pgfusepath{fill}%
\end{pgfscope}%
\begin{pgfscope}%
\pgfpathrectangle{\pgfqpoint{0.211875in}{0.211875in}}{\pgfqpoint{1.313625in}{1.279725in}}%
\pgfusepath{clip}%
\pgfsetbuttcap%
\pgfsetroundjoin%
\definecolor{currentfill}{rgb}{0.973832,0.856556,0.771584}%
\pgfsetfillcolor{currentfill}%
\pgfsetlinewidth{0.000000pt}%
\definecolor{currentstroke}{rgb}{0.000000,0.000000,0.000000}%
\pgfsetstrokecolor{currentstroke}%
\pgfsetdash{}{0pt}%
\pgfpathmoveto{\pgfqpoint{1.498962in}{0.948648in}}%
\pgfpathlineto{\pgfqpoint{1.499198in}{0.948686in}}%
\pgfpathlineto{\pgfqpoint{1.512231in}{0.953826in}}%
\pgfpathlineto{\pgfqpoint{1.514552in}{0.961613in}}%
\pgfpathlineto{\pgfqpoint{1.512586in}{0.974539in}}%
\pgfpathlineto{\pgfqpoint{1.512231in}{0.975111in}}%
\pgfpathlineto{\pgfqpoint{1.498962in}{0.977834in}}%
\pgfpathlineto{\pgfqpoint{1.496167in}{0.974539in}}%
\pgfpathlineto{\pgfqpoint{1.493690in}{0.961613in}}%
\pgfpathlineto{\pgfqpoint{1.498920in}{0.948686in}}%
\pgfpathclose%
\pgfusepath{fill}%
\end{pgfscope}%
\begin{pgfscope}%
\pgfpathrectangle{\pgfqpoint{0.211875in}{0.211875in}}{\pgfqpoint{1.313625in}{1.279725in}}%
\pgfusepath{clip}%
\pgfsetbuttcap%
\pgfsetroundjoin%
\definecolor{currentfill}{rgb}{0.973832,0.856556,0.771584}%
\pgfsetfillcolor{currentfill}%
\pgfsetlinewidth{0.000000pt}%
\definecolor{currentstroke}{rgb}{0.000000,0.000000,0.000000}%
\pgfsetstrokecolor{currentstroke}%
\pgfsetdash{}{0pt}%
\pgfpathmoveto{\pgfqpoint{0.556867in}{0.961017in}}%
\pgfpathlineto{\pgfqpoint{0.557889in}{0.961613in}}%
\pgfpathlineto{\pgfqpoint{0.556867in}{0.962758in}}%
\pgfpathlineto{\pgfqpoint{0.556610in}{0.961613in}}%
\pgfpathclose%
\pgfusepath{fill}%
\end{pgfscope}%
\begin{pgfscope}%
\pgfpathrectangle{\pgfqpoint{0.211875in}{0.211875in}}{\pgfqpoint{1.313625in}{1.279725in}}%
\pgfusepath{clip}%
\pgfsetbuttcap%
\pgfsetroundjoin%
\definecolor{currentfill}{rgb}{0.973832,0.856556,0.771584}%
\pgfsetfillcolor{currentfill}%
\pgfsetlinewidth{0.000000pt}%
\definecolor{currentstroke}{rgb}{0.000000,0.000000,0.000000}%
\pgfsetstrokecolor{currentstroke}%
\pgfsetdash{}{0pt}%
\pgfpathmoveto{\pgfqpoint{0.676288in}{0.952657in}}%
\pgfpathlineto{\pgfqpoint{0.685990in}{0.961613in}}%
\pgfpathlineto{\pgfqpoint{0.678936in}{0.974539in}}%
\pgfpathlineto{\pgfqpoint{0.676288in}{0.975701in}}%
\pgfpathlineto{\pgfqpoint{0.675062in}{0.974539in}}%
\pgfpathlineto{\pgfqpoint{0.671760in}{0.961613in}}%
\pgfpathclose%
\pgfusepath{fill}%
\end{pgfscope}%
\begin{pgfscope}%
\pgfpathrectangle{\pgfqpoint{0.211875in}{0.211875in}}{\pgfqpoint{1.313625in}{1.279725in}}%
\pgfusepath{clip}%
\pgfsetbuttcap%
\pgfsetroundjoin%
\definecolor{currentfill}{rgb}{0.973832,0.856556,0.771584}%
\pgfsetfillcolor{currentfill}%
\pgfsetlinewidth{0.000000pt}%
\definecolor{currentstroke}{rgb}{0.000000,0.000000,0.000000}%
\pgfsetstrokecolor{currentstroke}%
\pgfsetdash{}{0pt}%
\pgfpathmoveto{\pgfqpoint{1.087625in}{1.025253in}}%
\pgfpathlineto{\pgfqpoint{1.092542in}{1.026245in}}%
\pgfpathlineto{\pgfqpoint{1.100894in}{1.029715in}}%
\pgfpathlineto{\pgfqpoint{1.104565in}{1.039172in}}%
\pgfpathlineto{\pgfqpoint{1.104281in}{1.052098in}}%
\pgfpathlineto{\pgfqpoint{1.100894in}{1.059765in}}%
\pgfpathlineto{\pgfqpoint{1.087625in}{1.064456in}}%
\pgfpathlineto{\pgfqpoint{1.078600in}{1.052098in}}%
\pgfpathlineto{\pgfqpoint{1.078183in}{1.039172in}}%
\pgfpathlineto{\pgfqpoint{1.086124in}{1.026245in}}%
\pgfpathclose%
\pgfusepath{fill}%
\end{pgfscope}%
\begin{pgfscope}%
\pgfpathrectangle{\pgfqpoint{0.211875in}{0.211875in}}{\pgfqpoint{1.313625in}{1.279725in}}%
\pgfusepath{clip}%
\pgfsetbuttcap%
\pgfsetroundjoin%
\definecolor{currentfill}{rgb}{0.973832,0.856556,0.771584}%
\pgfsetfillcolor{currentfill}%
\pgfsetlinewidth{0.000000pt}%
\definecolor{currentstroke}{rgb}{0.000000,0.000000,0.000000}%
\pgfsetstrokecolor{currentstroke}%
\pgfsetdash{}{0pt}%
\pgfpathmoveto{\pgfqpoint{1.207045in}{1.024958in}}%
\pgfpathlineto{\pgfqpoint{1.211286in}{1.026245in}}%
\pgfpathlineto{\pgfqpoint{1.220314in}{1.032244in}}%
\pgfpathlineto{\pgfqpoint{1.222708in}{1.039172in}}%
\pgfpathlineto{\pgfqpoint{1.222441in}{1.052098in}}%
\pgfpathlineto{\pgfqpoint{1.220314in}{1.057512in}}%
\pgfpathlineto{\pgfqpoint{1.207045in}{1.064934in}}%
\pgfpathlineto{\pgfqpoint{1.195988in}{1.052098in}}%
\pgfpathlineto{\pgfqpoint{1.195504in}{1.039172in}}%
\pgfpathlineto{\pgfqpoint{1.204745in}{1.026245in}}%
\pgfpathclose%
\pgfusepath{fill}%
\end{pgfscope}%
\begin{pgfscope}%
\pgfpathrectangle{\pgfqpoint{0.211875in}{0.211875in}}{\pgfqpoint{1.313625in}{1.279725in}}%
\pgfusepath{clip}%
\pgfsetbuttcap%
\pgfsetroundjoin%
\definecolor{currentfill}{rgb}{0.973832,0.856556,0.771584}%
\pgfsetfillcolor{currentfill}%
\pgfsetlinewidth{0.000000pt}%
\definecolor{currentstroke}{rgb}{0.000000,0.000000,0.000000}%
\pgfsetstrokecolor{currentstroke}%
\pgfsetdash{}{0pt}%
\pgfpathmoveto{\pgfqpoint{1.326466in}{1.025455in}}%
\pgfpathlineto{\pgfqpoint{1.328427in}{1.026245in}}%
\pgfpathlineto{\pgfqpoint{1.339735in}{1.036899in}}%
\pgfpathlineto{\pgfqpoint{1.340435in}{1.039172in}}%
\pgfpathlineto{\pgfqpoint{1.340174in}{1.052098in}}%
\pgfpathlineto{\pgfqpoint{1.339735in}{1.053354in}}%
\pgfpathlineto{\pgfqpoint{1.326466in}{1.064123in}}%
\pgfpathlineto{\pgfqpoint{1.313889in}{1.052098in}}%
\pgfpathlineto{\pgfqpoint{1.313288in}{1.039172in}}%
\pgfpathlineto{\pgfqpoint{1.324745in}{1.026245in}}%
\pgfpathclose%
\pgfusepath{fill}%
\end{pgfscope}%
\begin{pgfscope}%
\pgfpathrectangle{\pgfqpoint{0.211875in}{0.211875in}}{\pgfqpoint{1.313625in}{1.279725in}}%
\pgfusepath{clip}%
\pgfsetbuttcap%
\pgfsetroundjoin%
\definecolor{currentfill}{rgb}{0.973832,0.856556,0.771584}%
\pgfsetfillcolor{currentfill}%
\pgfsetlinewidth{0.000000pt}%
\definecolor{currentstroke}{rgb}{0.000000,0.000000,0.000000}%
\pgfsetstrokecolor{currentstroke}%
\pgfsetdash{}{0pt}%
\pgfpathmoveto{\pgfqpoint{0.623212in}{1.035883in}}%
\pgfpathlineto{\pgfqpoint{0.625380in}{1.039172in}}%
\pgfpathlineto{\pgfqpoint{0.624818in}{1.052098in}}%
\pgfpathlineto{\pgfqpoint{0.623212in}{1.054234in}}%
\pgfpathlineto{\pgfqpoint{0.618560in}{1.052098in}}%
\pgfpathlineto{\pgfqpoint{0.616918in}{1.039172in}}%
\pgfpathclose%
\pgfusepath{fill}%
\end{pgfscope}%
\begin{pgfscope}%
\pgfpathrectangle{\pgfqpoint{0.211875in}{0.211875in}}{\pgfqpoint{1.313625in}{1.279725in}}%
\pgfusepath{clip}%
\pgfsetbuttcap%
\pgfsetroundjoin%
\definecolor{currentfill}{rgb}{0.973832,0.856556,0.771584}%
\pgfsetfillcolor{currentfill}%
\pgfsetlinewidth{0.000000pt}%
\definecolor{currentstroke}{rgb}{0.000000,0.000000,0.000000}%
\pgfsetstrokecolor{currentstroke}%
\pgfsetdash{}{0pt}%
\pgfpathmoveto{\pgfqpoint{0.729364in}{1.035653in}}%
\pgfpathlineto{\pgfqpoint{0.742633in}{1.032161in}}%
\pgfpathlineto{\pgfqpoint{0.746614in}{1.039172in}}%
\pgfpathlineto{\pgfqpoint{0.746169in}{1.052098in}}%
\pgfpathlineto{\pgfqpoint{0.742633in}{1.057560in}}%
\pgfpathlineto{\pgfqpoint{0.729364in}{1.054449in}}%
\pgfpathlineto{\pgfqpoint{0.728217in}{1.052098in}}%
\pgfpathlineto{\pgfqpoint{0.727855in}{1.039172in}}%
\pgfpathclose%
\pgfusepath{fill}%
\end{pgfscope}%
\begin{pgfscope}%
\pgfpathrectangle{\pgfqpoint{0.211875in}{0.211875in}}{\pgfqpoint{1.313625in}{1.279725in}}%
\pgfusepath{clip}%
\pgfsetbuttcap%
\pgfsetroundjoin%
\definecolor{currentfill}{rgb}{0.973832,0.856556,0.771584}%
\pgfsetfillcolor{currentfill}%
\pgfsetlinewidth{0.000000pt}%
\definecolor{currentstroke}{rgb}{0.000000,0.000000,0.000000}%
\pgfsetstrokecolor{currentstroke}%
\pgfsetdash{}{0pt}%
\pgfpathmoveto{\pgfqpoint{0.848784in}{1.030152in}}%
\pgfpathlineto{\pgfqpoint{0.862053in}{1.029841in}}%
\pgfpathlineto{\pgfqpoint{0.866682in}{1.039172in}}%
\pgfpathlineto{\pgfqpoint{0.866315in}{1.052098in}}%
\pgfpathlineto{\pgfqpoint{0.862053in}{1.059637in}}%
\pgfpathlineto{\pgfqpoint{0.848784in}{1.059359in}}%
\pgfpathlineto{\pgfqpoint{0.844772in}{1.052098in}}%
\pgfpathlineto{\pgfqpoint{0.844409in}{1.039172in}}%
\pgfpathclose%
\pgfusepath{fill}%
\end{pgfscope}%
\begin{pgfscope}%
\pgfpathrectangle{\pgfqpoint{0.211875in}{0.211875in}}{\pgfqpoint{1.313625in}{1.279725in}}%
\pgfusepath{clip}%
\pgfsetbuttcap%
\pgfsetroundjoin%
\definecolor{currentfill}{rgb}{0.973832,0.856556,0.771584}%
\pgfsetfillcolor{currentfill}%
\pgfsetlinewidth{0.000000pt}%
\definecolor{currentstroke}{rgb}{0.000000,0.000000,0.000000}%
\pgfsetstrokecolor{currentstroke}%
\pgfsetdash{}{0pt}%
\pgfpathmoveto{\pgfqpoint{0.968205in}{1.026478in}}%
\pgfpathlineto{\pgfqpoint{0.981473in}{1.028984in}}%
\pgfpathlineto{\pgfqpoint{0.985931in}{1.039172in}}%
\pgfpathlineto{\pgfqpoint{0.985615in}{1.052098in}}%
\pgfpathlineto{\pgfqpoint{0.981473in}{1.060410in}}%
\pgfpathlineto{\pgfqpoint{0.968205in}{1.062635in}}%
\pgfpathlineto{\pgfqpoint{0.961555in}{1.052098in}}%
\pgfpathlineto{\pgfqpoint{0.961174in}{1.039172in}}%
\pgfpathclose%
\pgfusepath{fill}%
\end{pgfscope}%
\begin{pgfscope}%
\pgfpathrectangle{\pgfqpoint{0.211875in}{0.211875in}}{\pgfqpoint{1.313625in}{1.279725in}}%
\pgfusepath{clip}%
\pgfsetbuttcap%
\pgfsetroundjoin%
\definecolor{currentfill}{rgb}{0.973832,0.856556,0.771584}%
\pgfsetfillcolor{currentfill}%
\pgfsetlinewidth{0.000000pt}%
\definecolor{currentstroke}{rgb}{0.000000,0.000000,0.000000}%
\pgfsetstrokecolor{currentstroke}%
\pgfsetdash{}{0pt}%
\pgfpathmoveto{\pgfqpoint{1.432617in}{1.038371in}}%
\pgfpathlineto{\pgfqpoint{1.445886in}{1.027146in}}%
\pgfpathlineto{\pgfqpoint{1.456067in}{1.039172in}}%
\pgfpathlineto{\pgfqpoint{1.455471in}{1.052098in}}%
\pgfpathlineto{\pgfqpoint{1.445886in}{1.062030in}}%
\pgfpathlineto{\pgfqpoint{1.432686in}{1.052098in}}%
\pgfpathlineto{\pgfqpoint{1.432617in}{1.051019in}}%
\pgfpathlineto{\pgfqpoint{1.432361in}{1.039172in}}%
\pgfpathclose%
\pgfusepath{fill}%
\end{pgfscope}%
\begin{pgfscope}%
\pgfpathrectangle{\pgfqpoint{0.211875in}{0.211875in}}{\pgfqpoint{1.313625in}{1.279725in}}%
\pgfusepath{clip}%
\pgfsetbuttcap%
\pgfsetroundjoin%
\definecolor{currentfill}{rgb}{0.973832,0.856556,0.771584}%
\pgfsetfillcolor{currentfill}%
\pgfsetlinewidth{0.000000pt}%
\definecolor{currentstroke}{rgb}{0.000000,0.000000,0.000000}%
\pgfsetstrokecolor{currentstroke}%
\pgfsetdash{}{0pt}%
\pgfpathmoveto{\pgfqpoint{0.676288in}{1.113473in}}%
\pgfpathlineto{\pgfqpoint{0.682920in}{1.116731in}}%
\pgfpathlineto{\pgfqpoint{0.687198in}{1.129658in}}%
\pgfpathlineto{\pgfqpoint{0.676288in}{1.138124in}}%
\pgfpathlineto{\pgfqpoint{0.671199in}{1.129658in}}%
\pgfpathlineto{\pgfqpoint{0.673210in}{1.116731in}}%
\pgfpathclose%
\pgfusepath{fill}%
\end{pgfscope}%
\begin{pgfscope}%
\pgfpathrectangle{\pgfqpoint{0.211875in}{0.211875in}}{\pgfqpoint{1.313625in}{1.279725in}}%
\pgfusepath{clip}%
\pgfsetbuttcap%
\pgfsetroundjoin%
\definecolor{currentfill}{rgb}{0.973832,0.856556,0.771584}%
\pgfsetfillcolor{currentfill}%
\pgfsetlinewidth{0.000000pt}%
\definecolor{currentstroke}{rgb}{0.000000,0.000000,0.000000}%
\pgfsetstrokecolor{currentstroke}%
\pgfsetdash{}{0pt}%
\pgfpathmoveto{\pgfqpoint{0.795708in}{1.110005in}}%
\pgfpathlineto{\pgfqpoint{0.805760in}{1.116731in}}%
\pgfpathlineto{\pgfqpoint{0.808626in}{1.129658in}}%
\pgfpathlineto{\pgfqpoint{0.796545in}{1.142584in}}%
\pgfpathlineto{\pgfqpoint{0.795708in}{1.142932in}}%
\pgfpathlineto{\pgfqpoint{0.795079in}{1.142584in}}%
\pgfpathlineto{\pgfqpoint{0.785858in}{1.129658in}}%
\pgfpathlineto{\pgfqpoint{0.788068in}{1.116731in}}%
\pgfpathclose%
\pgfusepath{fill}%
\end{pgfscope}%
\begin{pgfscope}%
\pgfpathrectangle{\pgfqpoint{0.211875in}{0.211875in}}{\pgfqpoint{1.313625in}{1.279725in}}%
\pgfusepath{clip}%
\pgfsetbuttcap%
\pgfsetroundjoin%
\definecolor{currentfill}{rgb}{0.973832,0.856556,0.771584}%
\pgfsetfillcolor{currentfill}%
\pgfsetlinewidth{0.000000pt}%
\definecolor{currentstroke}{rgb}{0.000000,0.000000,0.000000}%
\pgfsetstrokecolor{currentstroke}%
\pgfsetdash{}{0pt}%
\pgfpathmoveto{\pgfqpoint{0.915129in}{1.107719in}}%
\pgfpathlineto{\pgfqpoint{0.925783in}{1.116731in}}%
\pgfpathlineto{\pgfqpoint{0.927913in}{1.129658in}}%
\pgfpathlineto{\pgfqpoint{0.918961in}{1.142584in}}%
\pgfpathlineto{\pgfqpoint{0.915129in}{1.144606in}}%
\pgfpathlineto{\pgfqpoint{0.910554in}{1.142584in}}%
\pgfpathlineto{\pgfqpoint{0.901860in}{1.132651in}}%
\pgfpathlineto{\pgfqpoint{0.901081in}{1.129658in}}%
\pgfpathlineto{\pgfqpoint{0.901860in}{1.118941in}}%
\pgfpathlineto{\pgfqpoint{0.902323in}{1.116731in}}%
\pgfpathclose%
\pgfusepath{fill}%
\end{pgfscope}%
\begin{pgfscope}%
\pgfpathrectangle{\pgfqpoint{0.211875in}{0.211875in}}{\pgfqpoint{1.313625in}{1.279725in}}%
\pgfusepath{clip}%
\pgfsetbuttcap%
\pgfsetroundjoin%
\definecolor{currentfill}{rgb}{0.973832,0.856556,0.771584}%
\pgfsetfillcolor{currentfill}%
\pgfsetlinewidth{0.000000pt}%
\definecolor{currentstroke}{rgb}{0.000000,0.000000,0.000000}%
\pgfsetstrokecolor{currentstroke}%
\pgfsetdash{}{0pt}%
\pgfpathmoveto{\pgfqpoint{1.021280in}{1.112945in}}%
\pgfpathlineto{\pgfqpoint{1.034549in}{1.106572in}}%
\pgfpathlineto{\pgfqpoint{1.044474in}{1.116731in}}%
\pgfpathlineto{\pgfqpoint{1.046180in}{1.129658in}}%
\pgfpathlineto{\pgfqpoint{1.039024in}{1.142584in}}%
\pgfpathlineto{\pgfqpoint{1.034549in}{1.145446in}}%
\pgfpathlineto{\pgfqpoint{1.025875in}{1.142584in}}%
\pgfpathlineto{\pgfqpoint{1.021280in}{1.138911in}}%
\pgfpathlineto{\pgfqpoint{1.018578in}{1.129658in}}%
\pgfpathlineto{\pgfqpoint{1.019540in}{1.116731in}}%
\pgfpathclose%
\pgfusepath{fill}%
\end{pgfscope}%
\begin{pgfscope}%
\pgfpathrectangle{\pgfqpoint{0.211875in}{0.211875in}}{\pgfqpoint{1.313625in}{1.279725in}}%
\pgfusepath{clip}%
\pgfsetbuttcap%
\pgfsetroundjoin%
\definecolor{currentfill}{rgb}{0.973832,0.856556,0.771584}%
\pgfsetfillcolor{currentfill}%
\pgfsetlinewidth{0.000000pt}%
\definecolor{currentstroke}{rgb}{0.000000,0.000000,0.000000}%
\pgfsetstrokecolor{currentstroke}%
\pgfsetdash{}{0pt}%
\pgfpathmoveto{\pgfqpoint{1.140701in}{1.110470in}}%
\pgfpathlineto{\pgfqpoint{1.153970in}{1.106562in}}%
\pgfpathlineto{\pgfqpoint{1.162410in}{1.116731in}}%
\pgfpathlineto{\pgfqpoint{1.163863in}{1.129658in}}%
\pgfpathlineto{\pgfqpoint{1.157776in}{1.142584in}}%
\pgfpathlineto{\pgfqpoint{1.153970in}{1.145454in}}%
\pgfpathlineto{\pgfqpoint{1.140701in}{1.142592in}}%
\pgfpathlineto{\pgfqpoint{1.140694in}{1.142584in}}%
\pgfpathlineto{\pgfqpoint{1.136453in}{1.129658in}}%
\pgfpathlineto{\pgfqpoint{1.137465in}{1.116731in}}%
\pgfpathclose%
\pgfusepath{fill}%
\end{pgfscope}%
\begin{pgfscope}%
\pgfpathrectangle{\pgfqpoint{0.211875in}{0.211875in}}{\pgfqpoint{1.313625in}{1.279725in}}%
\pgfusepath{clip}%
\pgfsetbuttcap%
\pgfsetroundjoin%
\definecolor{currentfill}{rgb}{0.973832,0.856556,0.771584}%
\pgfsetfillcolor{currentfill}%
\pgfsetlinewidth{0.000000pt}%
\definecolor{currentstroke}{rgb}{0.000000,0.000000,0.000000}%
\pgfsetstrokecolor{currentstroke}%
\pgfsetdash{}{0pt}%
\pgfpathmoveto{\pgfqpoint{1.260121in}{1.109449in}}%
\pgfpathlineto{\pgfqpoint{1.273390in}{1.107732in}}%
\pgfpathlineto{\pgfqpoint{1.279855in}{1.116731in}}%
\pgfpathlineto{\pgfqpoint{1.281159in}{1.129658in}}%
\pgfpathlineto{\pgfqpoint{1.275699in}{1.142584in}}%
\pgfpathlineto{\pgfqpoint{1.273390in}{1.144597in}}%
\pgfpathlineto{\pgfqpoint{1.260121in}{1.143339in}}%
\pgfpathlineto{\pgfqpoint{1.259417in}{1.142584in}}%
\pgfpathlineto{\pgfqpoint{1.254767in}{1.129658in}}%
\pgfpathlineto{\pgfqpoint{1.255876in}{1.116731in}}%
\pgfpathclose%
\pgfusepath{fill}%
\end{pgfscope}%
\begin{pgfscope}%
\pgfpathrectangle{\pgfqpoint{0.211875in}{0.211875in}}{\pgfqpoint{1.313625in}{1.279725in}}%
\pgfusepath{clip}%
\pgfsetbuttcap%
\pgfsetroundjoin%
\definecolor{currentfill}{rgb}{0.973832,0.856556,0.771584}%
\pgfsetfillcolor{currentfill}%
\pgfsetlinewidth{0.000000pt}%
\definecolor{currentstroke}{rgb}{0.000000,0.000000,0.000000}%
\pgfsetstrokecolor{currentstroke}%
\pgfsetdash{}{0pt}%
\pgfpathmoveto{\pgfqpoint{1.379542in}{1.109707in}}%
\pgfpathlineto{\pgfqpoint{1.392811in}{1.110173in}}%
\pgfpathlineto{\pgfqpoint{1.396939in}{1.116731in}}%
\pgfpathlineto{\pgfqpoint{1.398165in}{1.129658in}}%
\pgfpathlineto{\pgfqpoint{1.393036in}{1.142584in}}%
\pgfpathlineto{\pgfqpoint{1.392811in}{1.142809in}}%
\pgfpathlineto{\pgfqpoint{1.379542in}{1.143151in}}%
\pgfpathlineto{\pgfqpoint{1.378941in}{1.142584in}}%
\pgfpathlineto{\pgfqpoint{1.373627in}{1.129658in}}%
\pgfpathlineto{\pgfqpoint{1.374893in}{1.116731in}}%
\pgfpathclose%
\pgfusepath{fill}%
\end{pgfscope}%
\begin{pgfscope}%
\pgfpathrectangle{\pgfqpoint{0.211875in}{0.211875in}}{\pgfqpoint{1.313625in}{1.279725in}}%
\pgfusepath{clip}%
\pgfsetbuttcap%
\pgfsetroundjoin%
\definecolor{currentfill}{rgb}{0.973832,0.856556,0.771584}%
\pgfsetfillcolor{currentfill}%
\pgfsetlinewidth{0.000000pt}%
\definecolor{currentstroke}{rgb}{0.000000,0.000000,0.000000}%
\pgfsetstrokecolor{currentstroke}%
\pgfsetdash{}{0pt}%
\pgfpathmoveto{\pgfqpoint{1.498962in}{1.111135in}}%
\pgfpathlineto{\pgfqpoint{1.512231in}{1.114037in}}%
\pgfpathlineto{\pgfqpoint{1.513730in}{1.116731in}}%
\pgfpathlineto{\pgfqpoint{1.514928in}{1.129658in}}%
\pgfpathlineto{\pgfqpoint{1.512231in}{1.137283in}}%
\pgfpathlineto{\pgfqpoint{1.498962in}{1.141608in}}%
\pgfpathlineto{\pgfqpoint{1.493208in}{1.129658in}}%
\pgfpathlineto{\pgfqpoint{1.494716in}{1.116731in}}%
\pgfpathclose%
\pgfusepath{fill}%
\end{pgfscope}%
\begin{pgfscope}%
\pgfpathrectangle{\pgfqpoint{0.211875in}{0.211875in}}{\pgfqpoint{1.313625in}{1.279725in}}%
\pgfusepath{clip}%
\pgfsetbuttcap%
\pgfsetroundjoin%
\definecolor{currentfill}{rgb}{0.973832,0.856556,0.771584}%
\pgfsetfillcolor{currentfill}%
\pgfsetlinewidth{0.000000pt}%
\definecolor{currentstroke}{rgb}{0.000000,0.000000,0.000000}%
\pgfsetstrokecolor{currentstroke}%
\pgfsetdash{}{0pt}%
\pgfpathmoveto{\pgfqpoint{0.556867in}{1.124644in}}%
\pgfpathlineto{\pgfqpoint{0.559721in}{1.129658in}}%
\pgfpathlineto{\pgfqpoint{0.556867in}{1.131058in}}%
\pgfpathlineto{\pgfqpoint{0.556149in}{1.129658in}}%
\pgfpathclose%
\pgfusepath{fill}%
\end{pgfscope}%
\begin{pgfscope}%
\pgfpathrectangle{\pgfqpoint{0.211875in}{0.211875in}}{\pgfqpoint{1.313625in}{1.279725in}}%
\pgfusepath{clip}%
\pgfsetbuttcap%
\pgfsetroundjoin%
\definecolor{currentfill}{rgb}{0.973832,0.856556,0.771584}%
\pgfsetfillcolor{currentfill}%
\pgfsetlinewidth{0.000000pt}%
\definecolor{currentstroke}{rgb}{0.000000,0.000000,0.000000}%
\pgfsetstrokecolor{currentstroke}%
\pgfsetdash{}{0pt}%
\pgfpathmoveto{\pgfqpoint{0.742633in}{1.193331in}}%
\pgfpathlineto{\pgfqpoint{0.743543in}{1.194290in}}%
\pgfpathlineto{\pgfqpoint{0.747284in}{1.207217in}}%
\pgfpathlineto{\pgfqpoint{0.743386in}{1.220143in}}%
\pgfpathlineto{\pgfqpoint{0.742633in}{1.220926in}}%
\pgfpathlineto{\pgfqpoint{0.737945in}{1.220143in}}%
\pgfpathlineto{\pgfqpoint{0.729364in}{1.216006in}}%
\pgfpathlineto{\pgfqpoint{0.727313in}{1.207217in}}%
\pgfpathlineto{\pgfqpoint{0.729364in}{1.198097in}}%
\pgfpathlineto{\pgfqpoint{0.736966in}{1.194290in}}%
\pgfpathclose%
\pgfusepath{fill}%
\end{pgfscope}%
\begin{pgfscope}%
\pgfpathrectangle{\pgfqpoint{0.211875in}{0.211875in}}{\pgfqpoint{1.313625in}{1.279725in}}%
\pgfusepath{clip}%
\pgfsetbuttcap%
\pgfsetroundjoin%
\definecolor{currentfill}{rgb}{0.973832,0.856556,0.771584}%
\pgfsetfillcolor{currentfill}%
\pgfsetlinewidth{0.000000pt}%
\definecolor{currentstroke}{rgb}{0.000000,0.000000,0.000000}%
\pgfsetstrokecolor{currentstroke}%
\pgfsetdash{}{0pt}%
\pgfpathmoveto{\pgfqpoint{0.848784in}{1.191987in}}%
\pgfpathlineto{\pgfqpoint{0.862053in}{1.191780in}}%
\pgfpathlineto{\pgfqpoint{0.864132in}{1.194290in}}%
\pgfpathlineto{\pgfqpoint{0.867241in}{1.207217in}}%
\pgfpathlineto{\pgfqpoint{0.863994in}{1.220143in}}%
\pgfpathlineto{\pgfqpoint{0.862053in}{1.222455in}}%
\pgfpathlineto{\pgfqpoint{0.848784in}{1.222250in}}%
\pgfpathlineto{\pgfqpoint{0.847060in}{1.220143in}}%
\pgfpathlineto{\pgfqpoint{0.843856in}{1.207217in}}%
\pgfpathlineto{\pgfqpoint{0.846925in}{1.194290in}}%
\pgfpathclose%
\pgfusepath{fill}%
\end{pgfscope}%
\begin{pgfscope}%
\pgfpathrectangle{\pgfqpoint{0.211875in}{0.211875in}}{\pgfqpoint{1.313625in}{1.279725in}}%
\pgfusepath{clip}%
\pgfsetbuttcap%
\pgfsetroundjoin%
\definecolor{currentfill}{rgb}{0.973832,0.856556,0.771584}%
\pgfsetfillcolor{currentfill}%
\pgfsetlinewidth{0.000000pt}%
\definecolor{currentstroke}{rgb}{0.000000,0.000000,0.000000}%
\pgfsetstrokecolor{currentstroke}%
\pgfsetdash{}{0pt}%
\pgfpathmoveto{\pgfqpoint{0.968205in}{1.189540in}}%
\pgfpathlineto{\pgfqpoint{0.981473in}{1.191202in}}%
\pgfpathlineto{\pgfqpoint{0.983726in}{1.194290in}}%
\pgfpathlineto{\pgfqpoint{0.986416in}{1.207217in}}%
\pgfpathlineto{\pgfqpoint{0.983602in}{1.220143in}}%
\pgfpathlineto{\pgfqpoint{0.981473in}{1.223020in}}%
\pgfpathlineto{\pgfqpoint{0.968205in}{1.224672in}}%
\pgfpathlineto{\pgfqpoint{0.963971in}{1.220143in}}%
\pgfpathlineto{\pgfqpoint{0.960591in}{1.207217in}}%
\pgfpathlineto{\pgfqpoint{0.963822in}{1.194290in}}%
\pgfpathclose%
\pgfusepath{fill}%
\end{pgfscope}%
\begin{pgfscope}%
\pgfpathrectangle{\pgfqpoint{0.211875in}{0.211875in}}{\pgfqpoint{1.313625in}{1.279725in}}%
\pgfusepath{clip}%
\pgfsetbuttcap%
\pgfsetroundjoin%
\definecolor{currentfill}{rgb}{0.973832,0.856556,0.771584}%
\pgfsetfillcolor{currentfill}%
\pgfsetlinewidth{0.000000pt}%
\definecolor{currentstroke}{rgb}{0.000000,0.000000,0.000000}%
\pgfsetstrokecolor{currentstroke}%
\pgfsetdash{}{0pt}%
\pgfpathmoveto{\pgfqpoint{1.087625in}{1.188179in}}%
\pgfpathlineto{\pgfqpoint{1.100894in}{1.191683in}}%
\pgfpathlineto{\pgfqpoint{1.102579in}{1.194290in}}%
\pgfpathlineto{\pgfqpoint{1.105002in}{1.207217in}}%
\pgfpathlineto{\pgfqpoint{1.102465in}{1.220143in}}%
\pgfpathlineto{\pgfqpoint{1.100894in}{1.222538in}}%
\pgfpathlineto{\pgfqpoint{1.087625in}{1.226020in}}%
\pgfpathlineto{\pgfqpoint{1.081266in}{1.220143in}}%
\pgfpathlineto{\pgfqpoint{1.077540in}{1.207217in}}%
\pgfpathlineto{\pgfqpoint{1.081098in}{1.194290in}}%
\pgfpathclose%
\pgfusepath{fill}%
\end{pgfscope}%
\begin{pgfscope}%
\pgfpathrectangle{\pgfqpoint{0.211875in}{0.211875in}}{\pgfqpoint{1.313625in}{1.279725in}}%
\pgfusepath{clip}%
\pgfsetbuttcap%
\pgfsetroundjoin%
\definecolor{currentfill}{rgb}{0.973832,0.856556,0.771584}%
\pgfsetfillcolor{currentfill}%
\pgfsetlinewidth{0.000000pt}%
\definecolor{currentstroke}{rgb}{0.000000,0.000000,0.000000}%
\pgfsetstrokecolor{currentstroke}%
\pgfsetdash{}{0pt}%
\pgfpathmoveto{\pgfqpoint{1.207045in}{1.187822in}}%
\pgfpathlineto{\pgfqpoint{1.220314in}{1.193366in}}%
\pgfpathlineto{\pgfqpoint{1.220846in}{1.194290in}}%
\pgfpathlineto{\pgfqpoint{1.223119in}{1.207217in}}%
\pgfpathlineto{\pgfqpoint{1.220739in}{1.220143in}}%
\pgfpathlineto{\pgfqpoint{1.220314in}{1.220870in}}%
\pgfpathlineto{\pgfqpoint{1.207045in}{1.226375in}}%
\pgfpathlineto{\pgfqpoint{1.199087in}{1.220143in}}%
\pgfpathlineto{\pgfqpoint{1.194757in}{1.207217in}}%
\pgfpathlineto{\pgfqpoint{1.198892in}{1.194290in}}%
\pgfpathclose%
\pgfusepath{fill}%
\end{pgfscope}%
\begin{pgfscope}%
\pgfpathrectangle{\pgfqpoint{0.211875in}{0.211875in}}{\pgfqpoint{1.313625in}{1.279725in}}%
\pgfusepath{clip}%
\pgfsetbuttcap%
\pgfsetroundjoin%
\definecolor{currentfill}{rgb}{0.973832,0.856556,0.771584}%
\pgfsetfillcolor{currentfill}%
\pgfsetlinewidth{0.000000pt}%
\definecolor{currentstroke}{rgb}{0.000000,0.000000,0.000000}%
\pgfsetstrokecolor{currentstroke}%
\pgfsetdash{}{0pt}%
\pgfpathmoveto{\pgfqpoint{1.326466in}{1.188428in}}%
\pgfpathlineto{\pgfqpoint{1.336752in}{1.194290in}}%
\pgfpathlineto{\pgfqpoint{1.339735in}{1.200383in}}%
\pgfpathlineto{\pgfqpoint{1.340834in}{1.207217in}}%
\pgfpathlineto{\pgfqpoint{1.339735in}{1.213773in}}%
\pgfpathlineto{\pgfqpoint{1.336480in}{1.220143in}}%
\pgfpathlineto{\pgfqpoint{1.326466in}{1.225778in}}%
\pgfpathlineto{\pgfqpoint{1.317723in}{1.220143in}}%
\pgfpathlineto{\pgfqpoint{1.313197in}{1.209459in}}%
\pgfpathlineto{\pgfqpoint{1.312848in}{1.207217in}}%
\pgfpathlineto{\pgfqpoint{1.313197in}{1.204875in}}%
\pgfpathlineto{\pgfqpoint{1.317485in}{1.194290in}}%
\pgfpathclose%
\pgfusepath{fill}%
\end{pgfscope}%
\begin{pgfscope}%
\pgfpathrectangle{\pgfqpoint{0.211875in}{0.211875in}}{\pgfqpoint{1.313625in}{1.279725in}}%
\pgfusepath{clip}%
\pgfsetbuttcap%
\pgfsetroundjoin%
\definecolor{currentfill}{rgb}{0.973832,0.856556,0.771584}%
\pgfsetfillcolor{currentfill}%
\pgfsetlinewidth{0.000000pt}%
\definecolor{currentstroke}{rgb}{0.000000,0.000000,0.000000}%
\pgfsetstrokecolor{currentstroke}%
\pgfsetdash{}{0pt}%
\pgfpathmoveto{\pgfqpoint{1.445886in}{1.189992in}}%
\pgfpathlineto{\pgfqpoint{1.451943in}{1.194290in}}%
\pgfpathlineto{\pgfqpoint{1.456974in}{1.207217in}}%
\pgfpathlineto{\pgfqpoint{1.451721in}{1.220143in}}%
\pgfpathlineto{\pgfqpoint{1.445886in}{1.224232in}}%
\pgfpathlineto{\pgfqpoint{1.437812in}{1.220143in}}%
\pgfpathlineto{\pgfqpoint{1.432617in}{1.211131in}}%
\pgfpathlineto{\pgfqpoint{1.431933in}{1.207217in}}%
\pgfpathlineto{\pgfqpoint{1.432617in}{1.203138in}}%
\pgfpathlineto{\pgfqpoint{1.437507in}{1.194290in}}%
\pgfpathclose%
\pgfusepath{fill}%
\end{pgfscope}%
\begin{pgfscope}%
\pgfpathrectangle{\pgfqpoint{0.211875in}{0.211875in}}{\pgfqpoint{1.313625in}{1.279725in}}%
\pgfusepath{clip}%
\pgfsetbuttcap%
\pgfsetroundjoin%
\definecolor{currentfill}{rgb}{0.973832,0.856556,0.771584}%
\pgfsetfillcolor{currentfill}%
\pgfsetlinewidth{0.000000pt}%
\definecolor{currentstroke}{rgb}{0.000000,0.000000,0.000000}%
\pgfsetstrokecolor{currentstroke}%
\pgfsetdash{}{0pt}%
\pgfpathmoveto{\pgfqpoint{0.623212in}{1.198545in}}%
\pgfpathlineto{\pgfqpoint{0.626214in}{1.207217in}}%
\pgfpathlineto{\pgfqpoint{0.623212in}{1.215592in}}%
\pgfpathlineto{\pgfqpoint{0.614475in}{1.207217in}}%
\pgfpathclose%
\pgfusepath{fill}%
\end{pgfscope}%
\begin{pgfscope}%
\pgfpathrectangle{\pgfqpoint{0.211875in}{0.211875in}}{\pgfqpoint{1.313625in}{1.279725in}}%
\pgfusepath{clip}%
\pgfsetbuttcap%
\pgfsetroundjoin%
\definecolor{currentfill}{rgb}{0.973832,0.856556,0.771584}%
\pgfsetfillcolor{currentfill}%
\pgfsetlinewidth{0.000000pt}%
\definecolor{currentstroke}{rgb}{0.000000,0.000000,0.000000}%
\pgfsetstrokecolor{currentstroke}%
\pgfsetdash{}{0pt}%
\pgfpathmoveto{\pgfqpoint{0.915129in}{1.270294in}}%
\pgfpathlineto{\pgfqpoint{0.918042in}{1.271849in}}%
\pgfpathlineto{\pgfqpoint{0.926960in}{1.284776in}}%
\pgfpathlineto{\pgfqpoint{0.924558in}{1.297702in}}%
\pgfpathlineto{\pgfqpoint{0.915129in}{1.305559in}}%
\pgfpathlineto{\pgfqpoint{0.903796in}{1.297702in}}%
\pgfpathlineto{\pgfqpoint{0.901860in}{1.289285in}}%
\pgfpathlineto{\pgfqpoint{0.901501in}{1.284776in}}%
\pgfpathlineto{\pgfqpoint{0.901860in}{1.283358in}}%
\pgfpathlineto{\pgfqpoint{0.911651in}{1.271849in}}%
\pgfpathclose%
\pgfusepath{fill}%
\end{pgfscope}%
\begin{pgfscope}%
\pgfpathrectangle{\pgfqpoint{0.211875in}{0.211875in}}{\pgfqpoint{1.313625in}{1.279725in}}%
\pgfusepath{clip}%
\pgfsetbuttcap%
\pgfsetroundjoin%
\definecolor{currentfill}{rgb}{0.973832,0.856556,0.771584}%
\pgfsetfillcolor{currentfill}%
\pgfsetlinewidth{0.000000pt}%
\definecolor{currentstroke}{rgb}{0.000000,0.000000,0.000000}%
\pgfsetstrokecolor{currentstroke}%
\pgfsetdash{}{0pt}%
\pgfpathmoveto{\pgfqpoint{1.034549in}{1.269449in}}%
\pgfpathlineto{\pgfqpoint{1.038258in}{1.271849in}}%
\pgfpathlineto{\pgfqpoint{1.045386in}{1.284776in}}%
\pgfpathlineto{\pgfqpoint{1.043457in}{1.297702in}}%
\pgfpathlineto{\pgfqpoint{1.034549in}{1.306683in}}%
\pgfpathlineto{\pgfqpoint{1.021280in}{1.300204in}}%
\pgfpathlineto{\pgfqpoint{1.020110in}{1.297702in}}%
\pgfpathlineto{\pgfqpoint{1.019023in}{1.284776in}}%
\pgfpathlineto{\pgfqpoint{1.021280in}{1.276843in}}%
\pgfpathlineto{\pgfqpoint{1.027360in}{1.271849in}}%
\pgfpathclose%
\pgfusepath{fill}%
\end{pgfscope}%
\begin{pgfscope}%
\pgfpathrectangle{\pgfqpoint{0.211875in}{0.211875in}}{\pgfqpoint{1.313625in}{1.279725in}}%
\pgfusepath{clip}%
\pgfsetbuttcap%
\pgfsetroundjoin%
\definecolor{currentfill}{rgb}{0.973832,0.856556,0.771584}%
\pgfsetfillcolor{currentfill}%
\pgfsetlinewidth{0.000000pt}%
\definecolor{currentstroke}{rgb}{0.000000,0.000000,0.000000}%
\pgfsetstrokecolor{currentstroke}%
\pgfsetdash{}{0pt}%
\pgfpathmoveto{\pgfqpoint{1.153970in}{1.269448in}}%
\pgfpathlineto{\pgfqpoint{1.157117in}{1.271849in}}%
\pgfpathlineto{\pgfqpoint{1.163180in}{1.284776in}}%
\pgfpathlineto{\pgfqpoint{1.161535in}{1.297702in}}%
\pgfpathlineto{\pgfqpoint{1.153970in}{1.306678in}}%
\pgfpathlineto{\pgfqpoint{1.140701in}{1.302699in}}%
\pgfpathlineto{\pgfqpoint{1.138076in}{1.297702in}}%
\pgfpathlineto{\pgfqpoint{1.136929in}{1.284776in}}%
\pgfpathlineto{\pgfqpoint{1.140701in}{1.272995in}}%
\pgfpathlineto{\pgfqpoint{1.142972in}{1.271849in}}%
\pgfpathclose%
\pgfusepath{fill}%
\end{pgfscope}%
\begin{pgfscope}%
\pgfpathrectangle{\pgfqpoint{0.211875in}{0.211875in}}{\pgfqpoint{1.313625in}{1.279725in}}%
\pgfusepath{clip}%
\pgfsetbuttcap%
\pgfsetroundjoin%
\definecolor{currentfill}{rgb}{0.973832,0.856556,0.771584}%
\pgfsetfillcolor{currentfill}%
\pgfsetlinewidth{0.000000pt}%
\definecolor{currentstroke}{rgb}{0.000000,0.000000,0.000000}%
\pgfsetstrokecolor{currentstroke}%
\pgfsetdash{}{0pt}%
\pgfpathmoveto{\pgfqpoint{1.260121in}{1.271625in}}%
\pgfpathlineto{\pgfqpoint{1.273390in}{1.270324in}}%
\pgfpathlineto{\pgfqpoint{1.275118in}{1.271849in}}%
\pgfpathlineto{\pgfqpoint{1.280556in}{1.284776in}}%
\pgfpathlineto{\pgfqpoint{1.279081in}{1.297702in}}%
\pgfpathlineto{\pgfqpoint{1.273390in}{1.305503in}}%
\pgfpathlineto{\pgfqpoint{1.260121in}{1.303752in}}%
\pgfpathlineto{\pgfqpoint{1.256538in}{1.297702in}}%
\pgfpathlineto{\pgfqpoint{1.255283in}{1.284776in}}%
\pgfpathlineto{\pgfqpoint{1.259915in}{1.271849in}}%
\pgfpathclose%
\pgfusepath{fill}%
\end{pgfscope}%
\begin{pgfscope}%
\pgfpathrectangle{\pgfqpoint{0.211875in}{0.211875in}}{\pgfqpoint{1.313625in}{1.279725in}}%
\pgfusepath{clip}%
\pgfsetbuttcap%
\pgfsetroundjoin%
\definecolor{currentfill}{rgb}{0.973832,0.856556,0.771584}%
\pgfsetfillcolor{currentfill}%
\pgfsetlinewidth{0.000000pt}%
\definecolor{currentstroke}{rgb}{0.000000,0.000000,0.000000}%
\pgfsetstrokecolor{currentstroke}%
\pgfsetdash{}{0pt}%
\pgfpathmoveto{\pgfqpoint{1.379542in}{1.271797in}}%
\pgfpathlineto{\pgfqpoint{1.381537in}{1.271849in}}%
\pgfpathlineto{\pgfqpoint{1.392811in}{1.272462in}}%
\pgfpathlineto{\pgfqpoint{1.397622in}{1.284776in}}%
\pgfpathlineto{\pgfqpoint{1.396242in}{1.297702in}}%
\pgfpathlineto{\pgfqpoint{1.392811in}{1.303067in}}%
\pgfpathlineto{\pgfqpoint{1.379542in}{1.303535in}}%
\pgfpathlineto{\pgfqpoint{1.375620in}{1.297702in}}%
\pgfpathlineto{\pgfqpoint{1.374193in}{1.284776in}}%
\pgfpathlineto{\pgfqpoint{1.379487in}{1.271849in}}%
\pgfpathclose%
\pgfusepath{fill}%
\end{pgfscope}%
\begin{pgfscope}%
\pgfpathrectangle{\pgfqpoint{0.211875in}{0.211875in}}{\pgfqpoint{1.313625in}{1.279725in}}%
\pgfusepath{clip}%
\pgfsetbuttcap%
\pgfsetroundjoin%
\definecolor{currentfill}{rgb}{0.973832,0.856556,0.771584}%
\pgfsetfillcolor{currentfill}%
\pgfsetlinewidth{0.000000pt}%
\definecolor{currentstroke}{rgb}{0.000000,0.000000,0.000000}%
\pgfsetstrokecolor{currentstroke}%
\pgfsetdash{}{0pt}%
\pgfpathmoveto{\pgfqpoint{0.556867in}{1.284663in}}%
\pgfpathlineto{\pgfqpoint{0.557092in}{1.284776in}}%
\pgfpathlineto{\pgfqpoint{0.556867in}{1.285140in}}%
\pgfpathlineto{\pgfqpoint{0.556811in}{1.284776in}}%
\pgfpathclose%
\pgfusepath{fill}%
\end{pgfscope}%
\begin{pgfscope}%
\pgfpathrectangle{\pgfqpoint{0.211875in}{0.211875in}}{\pgfqpoint{1.313625in}{1.279725in}}%
\pgfusepath{clip}%
\pgfsetbuttcap%
\pgfsetroundjoin%
\definecolor{currentfill}{rgb}{0.973832,0.856556,0.771584}%
\pgfsetfillcolor{currentfill}%
\pgfsetlinewidth{0.000000pt}%
\definecolor{currentstroke}{rgb}{0.000000,0.000000,0.000000}%
\pgfsetstrokecolor{currentstroke}%
\pgfsetdash{}{0pt}%
\pgfpathmoveto{\pgfqpoint{0.676288in}{1.277421in}}%
\pgfpathlineto{\pgfqpoint{0.685553in}{1.284776in}}%
\pgfpathlineto{\pgfqpoint{0.680788in}{1.297702in}}%
\pgfpathlineto{\pgfqpoint{0.676288in}{1.299880in}}%
\pgfpathlineto{\pgfqpoint{0.674199in}{1.297702in}}%
\pgfpathlineto{\pgfqpoint{0.671966in}{1.284776in}}%
\pgfpathclose%
\pgfusepath{fill}%
\end{pgfscope}%
\begin{pgfscope}%
\pgfpathrectangle{\pgfqpoint{0.211875in}{0.211875in}}{\pgfqpoint{1.313625in}{1.279725in}}%
\pgfusepath{clip}%
\pgfsetbuttcap%
\pgfsetroundjoin%
\definecolor{currentfill}{rgb}{0.973832,0.856556,0.771584}%
\pgfsetfillcolor{currentfill}%
\pgfsetlinewidth{0.000000pt}%
\definecolor{currentstroke}{rgb}{0.000000,0.000000,0.000000}%
\pgfsetstrokecolor{currentstroke}%
\pgfsetdash{}{0pt}%
\pgfpathmoveto{\pgfqpoint{0.795708in}{1.272129in}}%
\pgfpathlineto{\pgfqpoint{0.807424in}{1.284776in}}%
\pgfpathlineto{\pgfqpoint{0.804209in}{1.297702in}}%
\pgfpathlineto{\pgfqpoint{0.795708in}{1.303306in}}%
\pgfpathlineto{\pgfqpoint{0.789247in}{1.297702in}}%
\pgfpathlineto{\pgfqpoint{0.786775in}{1.284776in}}%
\pgfpathclose%
\pgfusepath{fill}%
\end{pgfscope}%
\begin{pgfscope}%
\pgfpathrectangle{\pgfqpoint{0.211875in}{0.211875in}}{\pgfqpoint{1.313625in}{1.279725in}}%
\pgfusepath{clip}%
\pgfsetbuttcap%
\pgfsetroundjoin%
\definecolor{currentfill}{rgb}{0.973832,0.856556,0.771584}%
\pgfsetfillcolor{currentfill}%
\pgfsetlinewidth{0.000000pt}%
\definecolor{currentstroke}{rgb}{0.000000,0.000000,0.000000}%
\pgfsetstrokecolor{currentstroke}%
\pgfsetdash{}{0pt}%
\pgfpathmoveto{\pgfqpoint{1.498962in}{1.273891in}}%
\pgfpathlineto{\pgfqpoint{1.512231in}{1.278401in}}%
\pgfpathlineto{\pgfqpoint{1.514432in}{1.284776in}}%
\pgfpathlineto{\pgfqpoint{1.513091in}{1.297702in}}%
\pgfpathlineto{\pgfqpoint{1.512231in}{1.299223in}}%
\pgfpathlineto{\pgfqpoint{1.498962in}{1.302155in}}%
\pgfpathlineto{\pgfqpoint{1.495531in}{1.297702in}}%
\pgfpathlineto{\pgfqpoint{1.493840in}{1.284776in}}%
\pgfpathclose%
\pgfusepath{fill}%
\end{pgfscope}%
\begin{pgfscope}%
\pgfpathrectangle{\pgfqpoint{0.211875in}{0.211875in}}{\pgfqpoint{1.313625in}{1.279725in}}%
\pgfusepath{clip}%
\pgfsetbuttcap%
\pgfsetroundjoin%
\definecolor{currentfill}{rgb}{0.973832,0.856556,0.771584}%
\pgfsetfillcolor{currentfill}%
\pgfsetlinewidth{0.000000pt}%
\definecolor{currentstroke}{rgb}{0.000000,0.000000,0.000000}%
\pgfsetstrokecolor{currentstroke}%
\pgfsetdash{}{0pt}%
\pgfpathmoveto{\pgfqpoint{0.623212in}{1.362146in}}%
\pgfpathlineto{\pgfqpoint{0.623352in}{1.362335in}}%
\pgfpathlineto{\pgfqpoint{0.623778in}{1.375261in}}%
\pgfpathlineto{\pgfqpoint{0.623212in}{1.376103in}}%
\pgfpathlineto{\pgfqpoint{0.621570in}{1.375261in}}%
\pgfpathlineto{\pgfqpoint{0.622807in}{1.362335in}}%
\pgfpathclose%
\pgfusepath{fill}%
\end{pgfscope}%
\begin{pgfscope}%
\pgfpathrectangle{\pgfqpoint{0.211875in}{0.211875in}}{\pgfqpoint{1.313625in}{1.279725in}}%
\pgfusepath{clip}%
\pgfsetbuttcap%
\pgfsetroundjoin%
\definecolor{currentfill}{rgb}{0.973832,0.856556,0.771584}%
\pgfsetfillcolor{currentfill}%
\pgfsetlinewidth{0.000000pt}%
\definecolor{currentstroke}{rgb}{0.000000,0.000000,0.000000}%
\pgfsetstrokecolor{currentstroke}%
\pgfsetdash{}{0pt}%
\pgfpathmoveto{\pgfqpoint{0.729364in}{1.362091in}}%
\pgfpathlineto{\pgfqpoint{0.742633in}{1.358801in}}%
\pgfpathlineto{\pgfqpoint{0.744881in}{1.362335in}}%
\pgfpathlineto{\pgfqpoint{0.745209in}{1.375261in}}%
\pgfpathlineto{\pgfqpoint{0.742633in}{1.379711in}}%
\pgfpathlineto{\pgfqpoint{0.729364in}{1.376138in}}%
\pgfpathlineto{\pgfqpoint{0.728980in}{1.375261in}}%
\pgfpathlineto{\pgfqpoint{0.729247in}{1.362335in}}%
\pgfpathclose%
\pgfusepath{fill}%
\end{pgfscope}%
\begin{pgfscope}%
\pgfpathrectangle{\pgfqpoint{0.211875in}{0.211875in}}{\pgfqpoint{1.313625in}{1.279725in}}%
\pgfusepath{clip}%
\pgfsetbuttcap%
\pgfsetroundjoin%
\definecolor{currentfill}{rgb}{0.973832,0.856556,0.771584}%
\pgfsetfillcolor{currentfill}%
\pgfsetlinewidth{0.000000pt}%
\definecolor{currentstroke}{rgb}{0.000000,0.000000,0.000000}%
\pgfsetstrokecolor{currentstroke}%
\pgfsetdash{}{0pt}%
\pgfpathmoveto{\pgfqpoint{0.848784in}{1.357023in}}%
\pgfpathlineto{\pgfqpoint{0.862053in}{1.356742in}}%
\pgfpathlineto{\pgfqpoint{0.865159in}{1.362335in}}%
\pgfpathlineto{\pgfqpoint{0.865423in}{1.375261in}}%
\pgfpathlineto{\pgfqpoint{0.862053in}{1.381922in}}%
\pgfpathlineto{\pgfqpoint{0.848784in}{1.381619in}}%
\pgfpathlineto{\pgfqpoint{0.845639in}{1.375261in}}%
\pgfpathlineto{\pgfqpoint{0.845901in}{1.362335in}}%
\pgfpathclose%
\pgfusepath{fill}%
\end{pgfscope}%
\begin{pgfscope}%
\pgfpathrectangle{\pgfqpoint{0.211875in}{0.211875in}}{\pgfqpoint{1.313625in}{1.279725in}}%
\pgfusepath{clip}%
\pgfsetbuttcap%
\pgfsetroundjoin%
\definecolor{currentfill}{rgb}{0.973832,0.856556,0.771584}%
\pgfsetfillcolor{currentfill}%
\pgfsetlinewidth{0.000000pt}%
\definecolor{currentstroke}{rgb}{0.000000,0.000000,0.000000}%
\pgfsetstrokecolor{currentstroke}%
\pgfsetdash{}{0pt}%
\pgfpathmoveto{\pgfqpoint{0.968205in}{1.353639in}}%
\pgfpathlineto{\pgfqpoint{0.981473in}{1.356025in}}%
\pgfpathlineto{\pgfqpoint{0.984560in}{1.362335in}}%
\pgfpathlineto{\pgfqpoint{0.984783in}{1.375261in}}%
\pgfpathlineto{\pgfqpoint{0.981473in}{1.382676in}}%
\pgfpathlineto{\pgfqpoint{0.968205in}{1.385290in}}%
\pgfpathlineto{\pgfqpoint{0.962543in}{1.375261in}}%
\pgfpathlineto{\pgfqpoint{0.962812in}{1.362335in}}%
\pgfpathclose%
\pgfusepath{fill}%
\end{pgfscope}%
\begin{pgfscope}%
\pgfpathrectangle{\pgfqpoint{0.211875in}{0.211875in}}{\pgfqpoint{1.313625in}{1.279725in}}%
\pgfusepath{clip}%
\pgfsetbuttcap%
\pgfsetroundjoin%
\definecolor{currentfill}{rgb}{0.973832,0.856556,0.771584}%
\pgfsetfillcolor{currentfill}%
\pgfsetlinewidth{0.000000pt}%
\definecolor{currentstroke}{rgb}{0.000000,0.000000,0.000000}%
\pgfsetstrokecolor{currentstroke}%
\pgfsetdash{}{0pt}%
\pgfpathmoveto{\pgfqpoint{1.087625in}{1.351751in}}%
\pgfpathlineto{\pgfqpoint{1.100894in}{1.356771in}}%
\pgfpathlineto{\pgfqpoint{1.103306in}{1.362335in}}%
\pgfpathlineto{\pgfqpoint{1.103504in}{1.375261in}}%
\pgfpathlineto{\pgfqpoint{1.100894in}{1.381848in}}%
\pgfpathlineto{\pgfqpoint{1.087625in}{1.387344in}}%
\pgfpathlineto{\pgfqpoint{1.079736in}{1.375261in}}%
\pgfpathlineto{\pgfqpoint{1.080028in}{1.362335in}}%
\pgfpathclose%
\pgfusepath{fill}%
\end{pgfscope}%
\begin{pgfscope}%
\pgfpathrectangle{\pgfqpoint{0.211875in}{0.211875in}}{\pgfqpoint{1.313625in}{1.279725in}}%
\pgfusepath{clip}%
\pgfsetbuttcap%
\pgfsetroundjoin%
\definecolor{currentfill}{rgb}{0.973832,0.856556,0.771584}%
\pgfsetfillcolor{currentfill}%
\pgfsetlinewidth{0.000000pt}%
\definecolor{currentstroke}{rgb}{0.000000,0.000000,0.000000}%
\pgfsetstrokecolor{currentstroke}%
\pgfsetdash{}{0pt}%
\pgfpathmoveto{\pgfqpoint{1.207045in}{1.351242in}}%
\pgfpathlineto{\pgfqpoint{1.220314in}{1.359180in}}%
\pgfpathlineto{\pgfqpoint{1.221529in}{1.362335in}}%
\pgfpathlineto{\pgfqpoint{1.221716in}{1.375261in}}%
\pgfpathlineto{\pgfqpoint{1.220314in}{1.379232in}}%
\pgfpathlineto{\pgfqpoint{1.207045in}{1.387904in}}%
\pgfpathlineto{\pgfqpoint{1.197313in}{1.375261in}}%
\pgfpathlineto{\pgfqpoint{1.197652in}{1.362335in}}%
\pgfpathclose%
\pgfusepath{fill}%
\end{pgfscope}%
\begin{pgfscope}%
\pgfpathrectangle{\pgfqpoint{0.211875in}{0.211875in}}{\pgfqpoint{1.313625in}{1.279725in}}%
\pgfusepath{clip}%
\pgfsetbuttcap%
\pgfsetroundjoin%
\definecolor{currentfill}{rgb}{0.973832,0.856556,0.771584}%
\pgfsetfillcolor{currentfill}%
\pgfsetlinewidth{0.000000pt}%
\definecolor{currentstroke}{rgb}{0.000000,0.000000,0.000000}%
\pgfsetstrokecolor{currentstroke}%
\pgfsetdash{}{0pt}%
\pgfpathmoveto{\pgfqpoint{1.326466in}{1.352055in}}%
\pgfpathlineto{\pgfqpoint{1.338605in}{1.362335in}}%
\pgfpathlineto{\pgfqpoint{1.339098in}{1.375261in}}%
\pgfpathlineto{\pgfqpoint{1.326466in}{1.387028in}}%
\pgfpathlineto{\pgfqpoint{1.315472in}{1.375261in}}%
\pgfpathlineto{\pgfqpoint{1.315897in}{1.362335in}}%
\pgfpathclose%
\pgfusepath{fill}%
\end{pgfscope}%
\begin{pgfscope}%
\pgfpathrectangle{\pgfqpoint{0.211875in}{0.211875in}}{\pgfqpoint{1.313625in}{1.279725in}}%
\pgfusepath{clip}%
\pgfsetbuttcap%
\pgfsetroundjoin%
\definecolor{currentfill}{rgb}{0.973832,0.856556,0.771584}%
\pgfsetfillcolor{currentfill}%
\pgfsetlinewidth{0.000000pt}%
\definecolor{currentstroke}{rgb}{0.000000,0.000000,0.000000}%
\pgfsetstrokecolor{currentstroke}%
\pgfsetdash{}{0pt}%
\pgfpathmoveto{\pgfqpoint{1.445886in}{1.354185in}}%
\pgfpathlineto{\pgfqpoint{1.453618in}{1.362335in}}%
\pgfpathlineto{\pgfqpoint{1.454048in}{1.375261in}}%
\pgfpathlineto{\pgfqpoint{1.445886in}{1.384723in}}%
\pgfpathlineto{\pgfqpoint{1.434655in}{1.375261in}}%
\pgfpathlineto{\pgfqpoint{1.435238in}{1.362335in}}%
\pgfpathclose%
\pgfusepath{fill}%
\end{pgfscope}%
\begin{pgfscope}%
\pgfpathrectangle{\pgfqpoint{0.211875in}{0.211875in}}{\pgfqpoint{1.313625in}{1.279725in}}%
\pgfusepath{clip}%
\pgfsetbuttcap%
\pgfsetroundjoin%
\definecolor{currentfill}{rgb}{0.973832,0.856556,0.771584}%
\pgfsetfillcolor{currentfill}%
\pgfsetlinewidth{0.000000pt}%
\definecolor{currentstroke}{rgb}{0.000000,0.000000,0.000000}%
\pgfsetstrokecolor{currentstroke}%
\pgfsetdash{}{0pt}%
\pgfpathmoveto{\pgfqpoint{0.795708in}{1.437853in}}%
\pgfpathlineto{\pgfqpoint{0.799072in}{1.439894in}}%
\pgfpathlineto{\pgfqpoint{0.803871in}{1.452820in}}%
\pgfpathlineto{\pgfqpoint{0.795708in}{1.462807in}}%
\pgfpathlineto{\pgfqpoint{0.789481in}{1.452820in}}%
\pgfpathlineto{\pgfqpoint{0.793156in}{1.439894in}}%
\pgfpathclose%
\pgfusepath{fill}%
\end{pgfscope}%
\begin{pgfscope}%
\pgfpathrectangle{\pgfqpoint{0.211875in}{0.211875in}}{\pgfqpoint{1.313625in}{1.279725in}}%
\pgfusepath{clip}%
\pgfsetbuttcap%
\pgfsetroundjoin%
\definecolor{currentfill}{rgb}{0.973832,0.856556,0.771584}%
\pgfsetfillcolor{currentfill}%
\pgfsetlinewidth{0.000000pt}%
\definecolor{currentstroke}{rgb}{0.000000,0.000000,0.000000}%
\pgfsetstrokecolor{currentstroke}%
\pgfsetdash{}{0pt}%
\pgfpathmoveto{\pgfqpoint{0.915129in}{1.435721in}}%
\pgfpathlineto{\pgfqpoint{0.920567in}{1.439894in}}%
\pgfpathlineto{\pgfqpoint{0.924129in}{1.452820in}}%
\pgfpathlineto{\pgfqpoint{0.915884in}{1.465747in}}%
\pgfpathlineto{\pgfqpoint{0.915129in}{1.466182in}}%
\pgfpathlineto{\pgfqpoint{0.914226in}{1.465747in}}%
\pgfpathlineto{\pgfqpoint{0.904285in}{1.452820in}}%
\pgfpathlineto{\pgfqpoint{0.908599in}{1.439894in}}%
\pgfpathclose%
\pgfusepath{fill}%
\end{pgfscope}%
\begin{pgfscope}%
\pgfpathrectangle{\pgfqpoint{0.211875in}{0.211875in}}{\pgfqpoint{1.313625in}{1.279725in}}%
\pgfusepath{clip}%
\pgfsetbuttcap%
\pgfsetroundjoin%
\definecolor{currentfill}{rgb}{0.973832,0.856556,0.771584}%
\pgfsetfillcolor{currentfill}%
\pgfsetlinewidth{0.000000pt}%
\definecolor{currentstroke}{rgb}{0.000000,0.000000,0.000000}%
\pgfsetstrokecolor{currentstroke}%
\pgfsetdash{}{0pt}%
\pgfpathmoveto{\pgfqpoint{1.034549in}{1.434671in}}%
\pgfpathlineto{\pgfqpoint{1.040172in}{1.439894in}}%
\pgfpathlineto{\pgfqpoint{1.043022in}{1.452820in}}%
\pgfpathlineto{\pgfqpoint{1.036397in}{1.465747in}}%
\pgfpathlineto{\pgfqpoint{1.034549in}{1.467037in}}%
\pgfpathlineto{\pgfqpoint{1.030963in}{1.465747in}}%
\pgfpathlineto{\pgfqpoint{1.021280in}{1.456493in}}%
\pgfpathlineto{\pgfqpoint{1.020348in}{1.452820in}}%
\pgfpathlineto{\pgfqpoint{1.021280in}{1.445029in}}%
\pgfpathlineto{\pgfqpoint{1.023614in}{1.439894in}}%
\pgfpathclose%
\pgfusepath{fill}%
\end{pgfscope}%
\begin{pgfscope}%
\pgfpathrectangle{\pgfqpoint{0.211875in}{0.211875in}}{\pgfqpoint{1.313625in}{1.279725in}}%
\pgfusepath{clip}%
\pgfsetbuttcap%
\pgfsetroundjoin%
\definecolor{currentfill}{rgb}{0.973832,0.856556,0.771584}%
\pgfsetfillcolor{currentfill}%
\pgfsetlinewidth{0.000000pt}%
\definecolor{currentstroke}{rgb}{0.000000,0.000000,0.000000}%
\pgfsetstrokecolor{currentstroke}%
\pgfsetdash{}{0pt}%
\pgfpathmoveto{\pgfqpoint{1.140701in}{1.438729in}}%
\pgfpathlineto{\pgfqpoint{1.153970in}{1.434702in}}%
\pgfpathlineto{\pgfqpoint{1.158716in}{1.439894in}}%
\pgfpathlineto{\pgfqpoint{1.161141in}{1.452820in}}%
\pgfpathlineto{\pgfqpoint{1.155499in}{1.465747in}}%
\pgfpathlineto{\pgfqpoint{1.153970in}{1.467005in}}%
\pgfpathlineto{\pgfqpoint{1.148626in}{1.465747in}}%
\pgfpathlineto{\pgfqpoint{1.140701in}{1.461078in}}%
\pgfpathlineto{\pgfqpoint{1.138350in}{1.452820in}}%
\pgfpathlineto{\pgfqpoint{1.140039in}{1.439894in}}%
\pgfpathclose%
\pgfusepath{fill}%
\end{pgfscope}%
\begin{pgfscope}%
\pgfpathrectangle{\pgfqpoint{0.211875in}{0.211875in}}{\pgfqpoint{1.313625in}{1.279725in}}%
\pgfusepath{clip}%
\pgfsetbuttcap%
\pgfsetroundjoin%
\definecolor{currentfill}{rgb}{0.973832,0.856556,0.771584}%
\pgfsetfillcolor{currentfill}%
\pgfsetlinewidth{0.000000pt}%
\definecolor{currentstroke}{rgb}{0.000000,0.000000,0.000000}%
\pgfsetstrokecolor{currentstroke}%
\pgfsetdash{}{0pt}%
\pgfpathmoveto{\pgfqpoint{1.260121in}{1.437634in}}%
\pgfpathlineto{\pgfqpoint{1.273390in}{1.435857in}}%
\pgfpathlineto{\pgfqpoint{1.276583in}{1.439894in}}%
\pgfpathlineto{\pgfqpoint{1.278759in}{1.452820in}}%
\pgfpathlineto{\pgfqpoint{1.273710in}{1.465747in}}%
\pgfpathlineto{\pgfqpoint{1.273390in}{1.466051in}}%
\pgfpathlineto{\pgfqpoint{1.270545in}{1.465747in}}%
\pgfpathlineto{\pgfqpoint{1.260121in}{1.463117in}}%
\pgfpathlineto{\pgfqpoint{1.256820in}{1.452820in}}%
\pgfpathlineto{\pgfqpoint{1.258672in}{1.439894in}}%
\pgfpathclose%
\pgfusepath{fill}%
\end{pgfscope}%
\begin{pgfscope}%
\pgfpathrectangle{\pgfqpoint{0.211875in}{0.211875in}}{\pgfqpoint{1.313625in}{1.279725in}}%
\pgfusepath{clip}%
\pgfsetbuttcap%
\pgfsetroundjoin%
\definecolor{currentfill}{rgb}{0.973832,0.856556,0.771584}%
\pgfsetfillcolor{currentfill}%
\pgfsetlinewidth{0.000000pt}%
\definecolor{currentstroke}{rgb}{0.000000,0.000000,0.000000}%
\pgfsetstrokecolor{currentstroke}%
\pgfsetdash{}{0pt}%
\pgfpathmoveto{\pgfqpoint{1.379542in}{1.437766in}}%
\pgfpathlineto{\pgfqpoint{1.392811in}{1.438224in}}%
\pgfpathlineto{\pgfqpoint{1.393967in}{1.439894in}}%
\pgfpathlineto{\pgfqpoint{1.396010in}{1.452820in}}%
\pgfpathlineto{\pgfqpoint{1.392811in}{1.462063in}}%
\pgfpathlineto{\pgfqpoint{1.379542in}{1.462910in}}%
\pgfpathlineto{\pgfqpoint{1.375875in}{1.452820in}}%
\pgfpathlineto{\pgfqpoint{1.377991in}{1.439894in}}%
\pgfpathclose%
\pgfusepath{fill}%
\end{pgfscope}%
\begin{pgfscope}%
\pgfpathrectangle{\pgfqpoint{0.211875in}{0.211875in}}{\pgfqpoint{1.313625in}{1.279725in}}%
\pgfusepath{clip}%
\pgfsetbuttcap%
\pgfsetroundjoin%
\definecolor{currentfill}{rgb}{0.973832,0.856556,0.771584}%
\pgfsetfillcolor{currentfill}%
\pgfsetlinewidth{0.000000pt}%
\definecolor{currentstroke}{rgb}{0.000000,0.000000,0.000000}%
\pgfsetstrokecolor{currentstroke}%
\pgfsetdash{}{0pt}%
\pgfpathmoveto{\pgfqpoint{1.498962in}{1.439017in}}%
\pgfpathlineto{\pgfqpoint{1.503115in}{1.439894in}}%
\pgfpathlineto{\pgfqpoint{1.512231in}{1.447786in}}%
\pgfpathlineto{\pgfqpoint{1.512966in}{1.452820in}}%
\pgfpathlineto{\pgfqpoint{1.512231in}{1.455219in}}%
\pgfpathlineto{\pgfqpoint{1.498962in}{1.460639in}}%
\pgfpathlineto{\pgfqpoint{1.495709in}{1.452820in}}%
\pgfpathlineto{\pgfqpoint{1.498229in}{1.439894in}}%
\pgfpathclose%
\pgfusepath{fill}%
\end{pgfscope}%
\begin{pgfscope}%
\pgfpathrectangle{\pgfqpoint{0.211875in}{0.211875in}}{\pgfqpoint{1.313625in}{1.279725in}}%
\pgfusepath{clip}%
\pgfsetbuttcap%
\pgfsetroundjoin%
\definecolor{currentfill}{rgb}{0.973832,0.856556,0.771584}%
\pgfsetfillcolor{currentfill}%
\pgfsetlinewidth{0.000000pt}%
\definecolor{currentstroke}{rgb}{0.000000,0.000000,0.000000}%
\pgfsetstrokecolor{currentstroke}%
\pgfsetdash{}{0pt}%
\pgfpathmoveto{\pgfqpoint{0.676288in}{1.444525in}}%
\pgfpathlineto{\pgfqpoint{0.680735in}{1.452820in}}%
\pgfpathlineto{\pgfqpoint{0.676288in}{1.456820in}}%
\pgfpathlineto{\pgfqpoint{0.674212in}{1.452820in}}%
\pgfpathclose%
\pgfusepath{fill}%
\end{pgfscope}%
\begin{pgfscope}%
\pgfsetrectcap%
\pgfsetmiterjoin%
\pgfsetlinewidth{0.000000pt}%
\definecolor{currentstroke}{rgb}{1.000000,1.000000,1.000000}%
\pgfsetstrokecolor{currentstroke}%
\pgfsetdash{}{0pt}%
\pgfpathmoveto{\pgfqpoint{0.211875in}{0.211875in}}%
\pgfpathlineto{\pgfqpoint{0.211875in}{1.491600in}}%
\pgfusepath{}%
\end{pgfscope}%
\begin{pgfscope}%
\pgfsetrectcap%
\pgfsetmiterjoin%
\pgfsetlinewidth{0.000000pt}%
\definecolor{currentstroke}{rgb}{1.000000,1.000000,1.000000}%
\pgfsetstrokecolor{currentstroke}%
\pgfsetdash{}{0pt}%
\pgfpathmoveto{\pgfqpoint{1.525500in}{0.211875in}}%
\pgfpathlineto{\pgfqpoint{1.525500in}{1.491600in}}%
\pgfusepath{}%
\end{pgfscope}%
\begin{pgfscope}%
\pgfsetrectcap%
\pgfsetmiterjoin%
\pgfsetlinewidth{0.000000pt}%
\definecolor{currentstroke}{rgb}{1.000000,1.000000,1.000000}%
\pgfsetstrokecolor{currentstroke}%
\pgfsetdash{}{0pt}%
\pgfpathmoveto{\pgfqpoint{0.211875in}{0.211875in}}%
\pgfpathlineto{\pgfqpoint{1.525500in}{0.211875in}}%
\pgfusepath{}%
\end{pgfscope}%
\begin{pgfscope}%
\pgfsetrectcap%
\pgfsetmiterjoin%
\pgfsetlinewidth{0.000000pt}%
\definecolor{currentstroke}{rgb}{1.000000,1.000000,1.000000}%
\pgfsetstrokecolor{currentstroke}%
\pgfsetdash{}{0pt}%
\pgfpathmoveto{\pgfqpoint{0.211875in}{1.491600in}}%
\pgfpathlineto{\pgfqpoint{1.525500in}{1.491600in}}%
\pgfusepath{}%
\end{pgfscope}%
\end{pgfpicture}%
\makeatother%
\endgroup%

            \end{minipage}
            \begin{minipage}{0.45\linewidth}
                %% Creator: Matplotlib, PGF backend
%%
%% To include the figure in your LaTeX document, write
%%   \input{<filename>.pgf}
%%
%% Make sure the required packages are loaded in your preamble
%%   \usepackage{pgf}
%%
%% Figures using additional raster images can only be included by \input if
%% they are in the same directory as the main LaTeX file. For loading figures
%% from other directories you can use the `import` package
%%   \usepackage{import}
%% and then include the figures with
%%   \import{<path to file>}{<filename>.pgf}
%%
%% Matplotlib used the following preamble
%%   \usepackage{gensymb}
%%   \usepackage{fontspec}
%%   \setmainfont{DejaVu Serif}
%%   \setsansfont{Arial}
%%   \setmonofont{DejaVu Sans Mono}
%%
\begingroup%
\makeatletter%
\begin{pgfpicture}%
\pgfpathrectangle{\pgfpointorigin}{\pgfqpoint{1.695000in}{1.695000in}}%
\pgfusepath{use as bounding box, clip}%
\begin{pgfscope}%
\pgfsetbuttcap%
\pgfsetmiterjoin%
\definecolor{currentfill}{rgb}{1.000000,1.000000,1.000000}%
\pgfsetfillcolor{currentfill}%
\pgfsetlinewidth{0.000000pt}%
\definecolor{currentstroke}{rgb}{1.000000,1.000000,1.000000}%
\pgfsetstrokecolor{currentstroke}%
\pgfsetdash{}{0pt}%
\pgfpathmoveto{\pgfqpoint{0.000000in}{0.000000in}}%
\pgfpathlineto{\pgfqpoint{1.695000in}{0.000000in}}%
\pgfpathlineto{\pgfqpoint{1.695000in}{1.695000in}}%
\pgfpathlineto{\pgfqpoint{0.000000in}{1.695000in}}%
\pgfpathclose%
\pgfusepath{fill}%
\end{pgfscope}%
\begin{pgfscope}%
\pgfsetbuttcap%
\pgfsetmiterjoin%
\definecolor{currentfill}{rgb}{0.917647,0.917647,0.949020}%
\pgfsetfillcolor{currentfill}%
\pgfsetlinewidth{0.000000pt}%
\definecolor{currentstroke}{rgb}{0.000000,0.000000,0.000000}%
\pgfsetstrokecolor{currentstroke}%
\pgfsetstrokeopacity{0.000000}%
\pgfsetdash{}{0pt}%
\pgfpathmoveto{\pgfqpoint{0.211875in}{0.211875in}}%
\pgfpathlineto{\pgfqpoint{1.525500in}{0.211875in}}%
\pgfpathlineto{\pgfqpoint{1.525500in}{1.491600in}}%
\pgfpathlineto{\pgfqpoint{0.211875in}{1.491600in}}%
\pgfpathclose%
\pgfusepath{fill}%
\end{pgfscope}%
\begin{pgfscope}%
\pgfpathrectangle{\pgfqpoint{0.211875in}{0.211875in}}{\pgfqpoint{1.313625in}{1.279725in}}%
\pgfusepath{clip}%
\pgfsetroundcap%
\pgfsetroundjoin%
\pgfsetlinewidth{0.803000pt}%
\definecolor{currentstroke}{rgb}{1.000000,1.000000,1.000000}%
\pgfsetstrokecolor{currentstroke}%
\pgfsetdash{}{0pt}%
\pgfpathmoveto{\pgfqpoint{0.294999in}{0.211875in}}%
\pgfpathlineto{\pgfqpoint{0.294999in}{1.491600in}}%
\pgfusepath{stroke}%
\end{pgfscope}%
\begin{pgfscope}%
\definecolor{textcolor}{rgb}{0.150000,0.150000,0.150000}%
\pgfsetstrokecolor{textcolor}%
\pgfsetfillcolor{textcolor}%
\pgftext[x=0.294999in,y=0.163264in,,top]{\color{textcolor}\rmfamily\fontsize{8.000000}{9.600000}\selectfont \(\displaystyle -50\)}%
\end{pgfscope}%
\begin{pgfscope}%
\pgfpathrectangle{\pgfqpoint{0.211875in}{0.211875in}}{\pgfqpoint{1.313625in}{1.279725in}}%
\pgfusepath{clip}%
\pgfsetroundcap%
\pgfsetroundjoin%
\pgfsetlinewidth{0.803000pt}%
\definecolor{currentstroke}{rgb}{1.000000,1.000000,1.000000}%
\pgfsetstrokecolor{currentstroke}%
\pgfsetdash{}{0pt}%
\pgfpathmoveto{\pgfqpoint{0.734464in}{0.211875in}}%
\pgfpathlineto{\pgfqpoint{0.734464in}{1.491600in}}%
\pgfusepath{stroke}%
\end{pgfscope}%
\begin{pgfscope}%
\definecolor{textcolor}{rgb}{0.150000,0.150000,0.150000}%
\pgfsetstrokecolor{textcolor}%
\pgfsetfillcolor{textcolor}%
\pgftext[x=0.734464in,y=0.163264in,,top]{\color{textcolor}\rmfamily\fontsize{8.000000}{9.600000}\selectfont \(\displaystyle -25\)}%
\end{pgfscope}%
\begin{pgfscope}%
\pgfpathrectangle{\pgfqpoint{0.211875in}{0.211875in}}{\pgfqpoint{1.313625in}{1.279725in}}%
\pgfusepath{clip}%
\pgfsetroundcap%
\pgfsetroundjoin%
\pgfsetlinewidth{0.803000pt}%
\definecolor{currentstroke}{rgb}{1.000000,1.000000,1.000000}%
\pgfsetstrokecolor{currentstroke}%
\pgfsetdash{}{0pt}%
\pgfpathmoveto{\pgfqpoint{1.173928in}{0.211875in}}%
\pgfpathlineto{\pgfqpoint{1.173928in}{1.491600in}}%
\pgfusepath{stroke}%
\end{pgfscope}%
\begin{pgfscope}%
\definecolor{textcolor}{rgb}{0.150000,0.150000,0.150000}%
\pgfsetstrokecolor{textcolor}%
\pgfsetfillcolor{textcolor}%
\pgftext[x=1.173928in,y=0.163264in,,top]{\color{textcolor}\rmfamily\fontsize{8.000000}{9.600000}\selectfont \(\displaystyle 0\)}%
\end{pgfscope}%
\begin{pgfscope}%
\pgfpathrectangle{\pgfqpoint{0.211875in}{0.211875in}}{\pgfqpoint{1.313625in}{1.279725in}}%
\pgfusepath{clip}%
\pgfsetroundcap%
\pgfsetroundjoin%
\pgfsetlinewidth{0.803000pt}%
\definecolor{currentstroke}{rgb}{1.000000,1.000000,1.000000}%
\pgfsetstrokecolor{currentstroke}%
\pgfsetdash{}{0pt}%
\pgfpathmoveto{\pgfqpoint{0.211875in}{0.433488in}}%
\pgfpathlineto{\pgfqpoint{1.525500in}{0.433488in}}%
\pgfusepath{stroke}%
\end{pgfscope}%
\begin{pgfscope}%
\definecolor{textcolor}{rgb}{0.150000,0.150000,0.150000}%
\pgfsetstrokecolor{textcolor}%
\pgfsetfillcolor{textcolor}%
\pgftext[x=-0.046616in,y=0.391279in,left,base]{\color{textcolor}\rmfamily\fontsize{8.000000}{9.600000}\selectfont \(\displaystyle -40\)}%
\end{pgfscope}%
\begin{pgfscope}%
\pgfpathrectangle{\pgfqpoint{0.211875in}{0.211875in}}{\pgfqpoint{1.313625in}{1.279725in}}%
\pgfusepath{clip}%
\pgfsetroundcap%
\pgfsetroundjoin%
\pgfsetlinewidth{0.803000pt}%
\definecolor{currentstroke}{rgb}{1.000000,1.000000,1.000000}%
\pgfsetstrokecolor{currentstroke}%
\pgfsetdash{}{0pt}%
\pgfpathmoveto{\pgfqpoint{0.211875in}{0.760260in}}%
\pgfpathlineto{\pgfqpoint{1.525500in}{0.760260in}}%
\pgfusepath{stroke}%
\end{pgfscope}%
\begin{pgfscope}%
\definecolor{textcolor}{rgb}{0.150000,0.150000,0.150000}%
\pgfsetstrokecolor{textcolor}%
\pgfsetfillcolor{textcolor}%
\pgftext[x=-0.046616in,y=0.718051in,left,base]{\color{textcolor}\rmfamily\fontsize{8.000000}{9.600000}\selectfont \(\displaystyle -20\)}%
\end{pgfscope}%
\begin{pgfscope}%
\pgfpathrectangle{\pgfqpoint{0.211875in}{0.211875in}}{\pgfqpoint{1.313625in}{1.279725in}}%
\pgfusepath{clip}%
\pgfsetroundcap%
\pgfsetroundjoin%
\pgfsetlinewidth{0.803000pt}%
\definecolor{currentstroke}{rgb}{1.000000,1.000000,1.000000}%
\pgfsetstrokecolor{currentstroke}%
\pgfsetdash{}{0pt}%
\pgfpathmoveto{\pgfqpoint{0.211875in}{1.087032in}}%
\pgfpathlineto{\pgfqpoint{1.525500in}{1.087032in}}%
\pgfusepath{stroke}%
\end{pgfscope}%
\begin{pgfscope}%
\definecolor{textcolor}{rgb}{0.150000,0.150000,0.150000}%
\pgfsetstrokecolor{textcolor}%
\pgfsetfillcolor{textcolor}%
\pgftext[x=0.104235in,y=1.044823in,left,base]{\color{textcolor}\rmfamily\fontsize{8.000000}{9.600000}\selectfont \(\displaystyle 0\)}%
\end{pgfscope}%
\begin{pgfscope}%
\pgfpathrectangle{\pgfqpoint{0.211875in}{0.211875in}}{\pgfqpoint{1.313625in}{1.279725in}}%
\pgfusepath{clip}%
\pgfsetroundcap%
\pgfsetroundjoin%
\pgfsetlinewidth{0.803000pt}%
\definecolor{currentstroke}{rgb}{1.000000,1.000000,1.000000}%
\pgfsetstrokecolor{currentstroke}%
\pgfsetdash{}{0pt}%
\pgfpathmoveto{\pgfqpoint{0.211875in}{1.413805in}}%
\pgfpathlineto{\pgfqpoint{1.525500in}{1.413805in}}%
\pgfusepath{stroke}%
\end{pgfscope}%
\begin{pgfscope}%
\definecolor{textcolor}{rgb}{0.150000,0.150000,0.150000}%
\pgfsetstrokecolor{textcolor}%
\pgfsetfillcolor{textcolor}%
\pgftext[x=0.045207in,y=1.371595in,left,base]{\color{textcolor}\rmfamily\fontsize{8.000000}{9.600000}\selectfont \(\displaystyle 20\)}%
\end{pgfscope}%
\begin{pgfscope}%
\pgfpathrectangle{\pgfqpoint{0.211875in}{0.211875in}}{\pgfqpoint{1.313625in}{1.279725in}}%
\pgfusepath{clip}%
\pgfsetbuttcap%
\pgfsetroundjoin%
\definecolor{currentfill}{rgb}{0.067555,0.047782,0.142002}%
\pgfsetfillcolor{currentfill}%
\pgfsetlinewidth{0.000000pt}%
\definecolor{currentstroke}{rgb}{0.000000,0.000000,0.000000}%
\pgfsetstrokecolor{currentstroke}%
\pgfsetdash{}{0pt}%
\pgfpathmoveto{\pgfqpoint{0.332287in}{0.277118in}}%
\pgfpathlineto{\pgfqpoint{0.333990in}{0.270102in}}%
\pgfpathlineto{\pgfqpoint{0.344716in}{0.270102in}}%
\pgfpathlineto{\pgfqpoint{0.357146in}{0.270102in}}%
\pgfpathlineto{\pgfqpoint{0.369575in}{0.270102in}}%
\pgfpathlineto{\pgfqpoint{0.382004in}{0.270102in}}%
\pgfpathlineto{\pgfqpoint{0.394434in}{0.270102in}}%
\pgfpathlineto{\pgfqpoint{0.406863in}{0.270102in}}%
\pgfpathlineto{\pgfqpoint{0.419292in}{0.270102in}}%
\pgfpathlineto{\pgfqpoint{0.431722in}{0.270102in}}%
\pgfpathlineto{\pgfqpoint{0.444151in}{0.270102in}}%
\pgfpathlineto{\pgfqpoint{0.456580in}{0.270102in}}%
\pgfpathlineto{\pgfqpoint{0.469009in}{0.270102in}}%
\pgfpathlineto{\pgfqpoint{0.481439in}{0.270102in}}%
\pgfpathlineto{\pgfqpoint{0.493868in}{0.270102in}}%
\pgfpathlineto{\pgfqpoint{0.506297in}{0.270102in}}%
\pgfpathlineto{\pgfqpoint{0.518727in}{0.270102in}}%
\pgfpathlineto{\pgfqpoint{0.531156in}{0.270102in}}%
\pgfpathlineto{\pgfqpoint{0.543585in}{0.270102in}}%
\pgfpathlineto{\pgfqpoint{0.556015in}{0.270102in}}%
\pgfpathlineto{\pgfqpoint{0.568444in}{0.270102in}}%
\pgfpathlineto{\pgfqpoint{0.580873in}{0.270102in}}%
\pgfpathlineto{\pgfqpoint{0.593302in}{0.270102in}}%
\pgfpathlineto{\pgfqpoint{0.605732in}{0.270102in}}%
\pgfpathlineto{\pgfqpoint{0.618161in}{0.270102in}}%
\pgfpathlineto{\pgfqpoint{0.630590in}{0.270102in}}%
\pgfpathlineto{\pgfqpoint{0.643020in}{0.270102in}}%
\pgfpathlineto{\pgfqpoint{0.655449in}{0.270102in}}%
\pgfpathlineto{\pgfqpoint{0.667878in}{0.270102in}}%
\pgfpathlineto{\pgfqpoint{0.680308in}{0.270102in}}%
\pgfpathlineto{\pgfqpoint{0.692737in}{0.270102in}}%
\pgfpathlineto{\pgfqpoint{0.705166in}{0.270102in}}%
\pgfpathlineto{\pgfqpoint{0.717595in}{0.270102in}}%
\pgfpathlineto{\pgfqpoint{0.730025in}{0.270102in}}%
\pgfpathlineto{\pgfqpoint{0.742454in}{0.270102in}}%
\pgfpathlineto{\pgfqpoint{0.750318in}{0.270102in}}%
\pgfpathlineto{\pgfqpoint{0.742454in}{0.275569in}}%
\pgfpathlineto{\pgfqpoint{0.733885in}{0.281654in}}%
\pgfpathlineto{\pgfqpoint{0.730025in}{0.284267in}}%
\pgfpathlineto{\pgfqpoint{0.717595in}{0.292839in}}%
\pgfpathlineto{\pgfqpoint{0.717066in}{0.293207in}}%
\pgfpathlineto{\pgfqpoint{0.705166in}{0.301129in}}%
\pgfpathlineto{\pgfqpoint{0.699805in}{0.304760in}}%
\pgfpathlineto{\pgfqpoint{0.692737in}{0.309371in}}%
\pgfpathlineto{\pgfqpoint{0.682260in}{0.316312in}}%
\pgfpathlineto{\pgfqpoint{0.680308in}{0.317564in}}%
\pgfpathlineto{\pgfqpoint{0.667878in}{0.325564in}}%
\pgfpathlineto{\pgfqpoint{0.664323in}{0.327865in}}%
\pgfpathlineto{\pgfqpoint{0.655449in}{0.333446in}}%
\pgfpathlineto{\pgfqpoint{0.646041in}{0.339417in}}%
\pgfpathlineto{\pgfqpoint{0.643020in}{0.341289in}}%
\pgfpathlineto{\pgfqpoint{0.630590in}{0.348966in}}%
\pgfpathlineto{\pgfqpoint{0.627327in}{0.350970in}}%
\pgfpathlineto{\pgfqpoint{0.618161in}{0.356482in}}%
\pgfpathlineto{\pgfqpoint{0.608122in}{0.362522in}}%
\pgfpathlineto{\pgfqpoint{0.605732in}{0.363936in}}%
\pgfpathlineto{\pgfqpoint{0.593302in}{0.371156in}}%
\pgfpathlineto{\pgfqpoint{0.588204in}{0.374075in}}%
\pgfpathlineto{\pgfqpoint{0.580873in}{0.378211in}}%
\pgfpathlineto{\pgfqpoint{0.568444in}{0.385126in}}%
\pgfpathlineto{\pgfqpoint{0.567505in}{0.385627in}}%
\pgfpathlineto{\pgfqpoint{0.556015in}{0.391691in}}%
\pgfpathlineto{\pgfqpoint{0.545378in}{0.397180in}}%
\pgfpathlineto{\pgfqpoint{0.543585in}{0.398097in}}%
\pgfpathlineto{\pgfqpoint{0.531156in}{0.404096in}}%
\pgfpathlineto{\pgfqpoint{0.521120in}{0.408733in}}%
\pgfpathlineto{\pgfqpoint{0.518727in}{0.409831in}}%
\pgfpathlineto{\pgfqpoint{0.506297in}{0.415071in}}%
\pgfpathlineto{\pgfqpoint{0.493868in}{0.419947in}}%
\pgfpathlineto{\pgfqpoint{0.492877in}{0.420285in}}%
\pgfpathlineto{\pgfqpoint{0.481439in}{0.424173in}}%
\pgfpathlineto{\pgfqpoint{0.469009in}{0.427831in}}%
\pgfpathlineto{\pgfqpoint{0.456580in}{0.430820in}}%
\pgfpathlineto{\pgfqpoint{0.450641in}{0.431838in}}%
\pgfpathlineto{\pgfqpoint{0.444151in}{0.432945in}}%
\pgfpathlineto{\pgfqpoint{0.431722in}{0.434097in}}%
\pgfpathlineto{\pgfqpoint{0.419292in}{0.434152in}}%
\pgfpathlineto{\pgfqpoint{0.406863in}{0.432915in}}%
\pgfpathlineto{\pgfqpoint{0.401862in}{0.431838in}}%
\pgfpathlineto{\pgfqpoint{0.394434in}{0.430050in}}%
\pgfpathlineto{\pgfqpoint{0.382004in}{0.425197in}}%
\pgfpathlineto{\pgfqpoint{0.373292in}{0.420285in}}%
\pgfpathlineto{\pgfqpoint{0.369575in}{0.417887in}}%
\pgfpathlineto{\pgfqpoint{0.358830in}{0.408733in}}%
\pgfpathlineto{\pgfqpoint{0.357146in}{0.407050in}}%
\pgfpathlineto{\pgfqpoint{0.349184in}{0.397180in}}%
\pgfpathlineto{\pgfqpoint{0.344716in}{0.390468in}}%
\pgfpathlineto{\pgfqpoint{0.342013in}{0.385627in}}%
\pgfpathlineto{\pgfqpoint{0.336754in}{0.374075in}}%
\pgfpathlineto{\pgfqpoint{0.332721in}{0.362522in}}%
\pgfpathlineto{\pgfqpoint{0.332287in}{0.360859in}}%
\pgfpathlineto{\pgfqpoint{0.330030in}{0.350970in}}%
\pgfpathlineto{\pgfqpoint{0.328286in}{0.339417in}}%
\pgfpathlineto{\pgfqpoint{0.327385in}{0.327865in}}%
\pgfpathlineto{\pgfqpoint{0.327274in}{0.316312in}}%
\pgfpathlineto{\pgfqpoint{0.327906in}{0.304760in}}%
\pgfpathlineto{\pgfqpoint{0.329239in}{0.293207in}}%
\pgfpathlineto{\pgfqpoint{0.331238in}{0.281654in}}%
\pgfpathclose%
\pgfusepath{fill}%
\end{pgfscope}%
\begin{pgfscope}%
\pgfpathrectangle{\pgfqpoint{0.211875in}{0.211875in}}{\pgfqpoint{1.313625in}{1.279725in}}%
\pgfusepath{clip}%
\pgfsetbuttcap%
\pgfsetroundjoin%
\definecolor{currentfill}{rgb}{0.198046,0.094652,0.234785}%
\pgfsetfillcolor{currentfill}%
\pgfsetlinewidth{0.000000pt}%
\definecolor{currentstroke}{rgb}{0.000000,0.000000,0.000000}%
\pgfsetstrokecolor{currentstroke}%
\pgfsetdash{}{0pt}%
\pgfpathmoveto{\pgfqpoint{0.307429in}{0.270102in}}%
\pgfpathlineto{\pgfqpoint{0.319858in}{0.270102in}}%
\pgfpathlineto{\pgfqpoint{0.332287in}{0.270102in}}%
\pgfpathlineto{\pgfqpoint{0.333990in}{0.270102in}}%
\pgfpathlineto{\pgfqpoint{0.332287in}{0.277118in}}%
\pgfpathlineto{\pgfqpoint{0.331238in}{0.281654in}}%
\pgfpathlineto{\pgfqpoint{0.329239in}{0.293207in}}%
\pgfpathlineto{\pgfqpoint{0.327906in}{0.304760in}}%
\pgfpathlineto{\pgfqpoint{0.327274in}{0.316312in}}%
\pgfpathlineto{\pgfqpoint{0.327385in}{0.327865in}}%
\pgfpathlineto{\pgfqpoint{0.328286in}{0.339417in}}%
\pgfpathlineto{\pgfqpoint{0.330030in}{0.350970in}}%
\pgfpathlineto{\pgfqpoint{0.332287in}{0.360859in}}%
\pgfpathlineto{\pgfqpoint{0.332721in}{0.362522in}}%
\pgfpathlineto{\pgfqpoint{0.336754in}{0.374075in}}%
\pgfpathlineto{\pgfqpoint{0.342013in}{0.385627in}}%
\pgfpathlineto{\pgfqpoint{0.344716in}{0.390468in}}%
\pgfpathlineto{\pgfqpoint{0.349184in}{0.397180in}}%
\pgfpathlineto{\pgfqpoint{0.357146in}{0.407050in}}%
\pgfpathlineto{\pgfqpoint{0.358830in}{0.408733in}}%
\pgfpathlineto{\pgfqpoint{0.369575in}{0.417887in}}%
\pgfpathlineto{\pgfqpoint{0.373292in}{0.420285in}}%
\pgfpathlineto{\pgfqpoint{0.382004in}{0.425197in}}%
\pgfpathlineto{\pgfqpoint{0.394434in}{0.430050in}}%
\pgfpathlineto{\pgfqpoint{0.401862in}{0.431838in}}%
\pgfpathlineto{\pgfqpoint{0.406863in}{0.432915in}}%
\pgfpathlineto{\pgfqpoint{0.419292in}{0.434152in}}%
\pgfpathlineto{\pgfqpoint{0.431722in}{0.434097in}}%
\pgfpathlineto{\pgfqpoint{0.444151in}{0.432945in}}%
\pgfpathlineto{\pgfqpoint{0.450641in}{0.431838in}}%
\pgfpathlineto{\pgfqpoint{0.456580in}{0.430820in}}%
\pgfpathlineto{\pgfqpoint{0.469009in}{0.427831in}}%
\pgfpathlineto{\pgfqpoint{0.481439in}{0.424173in}}%
\pgfpathlineto{\pgfqpoint{0.492877in}{0.420285in}}%
\pgfpathlineto{\pgfqpoint{0.493868in}{0.419947in}}%
\pgfpathlineto{\pgfqpoint{0.506297in}{0.415071in}}%
\pgfpathlineto{\pgfqpoint{0.518727in}{0.409831in}}%
\pgfpathlineto{\pgfqpoint{0.521120in}{0.408733in}}%
\pgfpathlineto{\pgfqpoint{0.531156in}{0.404096in}}%
\pgfpathlineto{\pgfqpoint{0.543585in}{0.398097in}}%
\pgfpathlineto{\pgfqpoint{0.545378in}{0.397180in}}%
\pgfpathlineto{\pgfqpoint{0.556015in}{0.391691in}}%
\pgfpathlineto{\pgfqpoint{0.567505in}{0.385627in}}%
\pgfpathlineto{\pgfqpoint{0.568444in}{0.385126in}}%
\pgfpathlineto{\pgfqpoint{0.580873in}{0.378211in}}%
\pgfpathlineto{\pgfqpoint{0.588204in}{0.374075in}}%
\pgfpathlineto{\pgfqpoint{0.593302in}{0.371156in}}%
\pgfpathlineto{\pgfqpoint{0.605732in}{0.363936in}}%
\pgfpathlineto{\pgfqpoint{0.608122in}{0.362522in}}%
\pgfpathlineto{\pgfqpoint{0.618161in}{0.356482in}}%
\pgfpathlineto{\pgfqpoint{0.627327in}{0.350970in}}%
\pgfpathlineto{\pgfqpoint{0.630590in}{0.348966in}}%
\pgfpathlineto{\pgfqpoint{0.643020in}{0.341289in}}%
\pgfpathlineto{\pgfqpoint{0.646041in}{0.339417in}}%
\pgfpathlineto{\pgfqpoint{0.655449in}{0.333446in}}%
\pgfpathlineto{\pgfqpoint{0.664323in}{0.327865in}}%
\pgfpathlineto{\pgfqpoint{0.667878in}{0.325564in}}%
\pgfpathlineto{\pgfqpoint{0.680308in}{0.317564in}}%
\pgfpathlineto{\pgfqpoint{0.682260in}{0.316312in}}%
\pgfpathlineto{\pgfqpoint{0.692737in}{0.309371in}}%
\pgfpathlineto{\pgfqpoint{0.699805in}{0.304760in}}%
\pgfpathlineto{\pgfqpoint{0.705166in}{0.301129in}}%
\pgfpathlineto{\pgfqpoint{0.717066in}{0.293207in}}%
\pgfpathlineto{\pgfqpoint{0.717595in}{0.292839in}}%
\pgfpathlineto{\pgfqpoint{0.730025in}{0.284267in}}%
\pgfpathlineto{\pgfqpoint{0.733885in}{0.281654in}}%
\pgfpathlineto{\pgfqpoint{0.742454in}{0.275569in}}%
\pgfpathlineto{\pgfqpoint{0.750318in}{0.270102in}}%
\pgfpathlineto{\pgfqpoint{0.754883in}{0.270102in}}%
\pgfpathlineto{\pgfqpoint{0.767313in}{0.270102in}}%
\pgfpathlineto{\pgfqpoint{0.779742in}{0.270102in}}%
\pgfpathlineto{\pgfqpoint{0.792171in}{0.270102in}}%
\pgfpathlineto{\pgfqpoint{0.804601in}{0.270102in}}%
\pgfpathlineto{\pgfqpoint{0.817030in}{0.270102in}}%
\pgfpathlineto{\pgfqpoint{0.829459in}{0.270102in}}%
\pgfpathlineto{\pgfqpoint{0.841888in}{0.270102in}}%
\pgfpathlineto{\pgfqpoint{0.854318in}{0.270102in}}%
\pgfpathlineto{\pgfqpoint{0.866747in}{0.270102in}}%
\pgfpathlineto{\pgfqpoint{0.879176in}{0.270102in}}%
\pgfpathlineto{\pgfqpoint{0.891606in}{0.270102in}}%
\pgfpathlineto{\pgfqpoint{0.904035in}{0.270102in}}%
\pgfpathlineto{\pgfqpoint{0.913460in}{0.270102in}}%
\pgfpathlineto{\pgfqpoint{0.904035in}{0.275107in}}%
\pgfpathlineto{\pgfqpoint{0.892159in}{0.281654in}}%
\pgfpathlineto{\pgfqpoint{0.891606in}{0.281944in}}%
\pgfpathlineto{\pgfqpoint{0.879176in}{0.288655in}}%
\pgfpathlineto{\pgfqpoint{0.871046in}{0.293207in}}%
\pgfpathlineto{\pgfqpoint{0.866747in}{0.295510in}}%
\pgfpathlineto{\pgfqpoint{0.854318in}{0.302384in}}%
\pgfpathlineto{\pgfqpoint{0.850149in}{0.304760in}}%
\pgfpathlineto{\pgfqpoint{0.841888in}{0.309292in}}%
\pgfpathlineto{\pgfqpoint{0.829531in}{0.316312in}}%
\pgfpathlineto{\pgfqpoint{0.829459in}{0.316352in}}%
\pgfpathlineto{\pgfqpoint{0.817030in}{0.323341in}}%
\pgfpathlineto{\pgfqpoint{0.809241in}{0.327865in}}%
\pgfpathlineto{\pgfqpoint{0.804601in}{0.330483in}}%
\pgfpathlineto{\pgfqpoint{0.792171in}{0.337690in}}%
\pgfpathlineto{\pgfqpoint{0.789261in}{0.339417in}}%
\pgfpathlineto{\pgfqpoint{0.779742in}{0.344928in}}%
\pgfpathlineto{\pgfqpoint{0.769602in}{0.350970in}}%
\pgfpathlineto{\pgfqpoint{0.767313in}{0.352306in}}%
\pgfpathlineto{\pgfqpoint{0.754883in}{0.359702in}}%
\pgfpathlineto{\pgfqpoint{0.750245in}{0.362522in}}%
\pgfpathlineto{\pgfqpoint{0.742454in}{0.367180in}}%
\pgfpathlineto{\pgfqpoint{0.731194in}{0.374075in}}%
\pgfpathlineto{\pgfqpoint{0.730025in}{0.374781in}}%
\pgfpathlineto{\pgfqpoint{0.717595in}{0.382384in}}%
\pgfpathlineto{\pgfqpoint{0.712389in}{0.385627in}}%
\pgfpathlineto{\pgfqpoint{0.705166in}{0.390079in}}%
\pgfpathlineto{\pgfqpoint{0.693858in}{0.397180in}}%
\pgfpathlineto{\pgfqpoint{0.692737in}{0.397878in}}%
\pgfpathlineto{\pgfqpoint{0.680308in}{0.405679in}}%
\pgfpathlineto{\pgfqpoint{0.675502in}{0.408733in}}%
\pgfpathlineto{\pgfqpoint{0.667878in}{0.413553in}}%
\pgfpathlineto{\pgfqpoint{0.657359in}{0.420285in}}%
\pgfpathlineto{\pgfqpoint{0.655449in}{0.421505in}}%
\pgfpathlineto{\pgfqpoint{0.643020in}{0.429460in}}%
\pgfpathlineto{\pgfqpoint{0.639318in}{0.431838in}}%
\pgfpathlineto{\pgfqpoint{0.630590in}{0.437445in}}%
\pgfpathlineto{\pgfqpoint{0.621380in}{0.443390in}}%
\pgfpathlineto{\pgfqpoint{0.618161in}{0.445474in}}%
\pgfpathlineto{\pgfqpoint{0.605732in}{0.453495in}}%
\pgfpathlineto{\pgfqpoint{0.603469in}{0.454943in}}%
\pgfpathlineto{\pgfqpoint{0.593302in}{0.461483in}}%
\pgfpathlineto{\pgfqpoint{0.585474in}{0.466495in}}%
\pgfpathlineto{\pgfqpoint{0.580873in}{0.469463in}}%
\pgfpathlineto{\pgfqpoint{0.568444in}{0.477395in}}%
\pgfpathlineto{\pgfqpoint{0.567397in}{0.478048in}}%
\pgfpathlineto{\pgfqpoint{0.556015in}{0.485215in}}%
\pgfpathlineto{\pgfqpoint{0.548931in}{0.489600in}}%
\pgfpathlineto{\pgfqpoint{0.543585in}{0.492949in}}%
\pgfpathlineto{\pgfqpoint{0.531156in}{0.500551in}}%
\pgfpathlineto{\pgfqpoint{0.530131in}{0.501153in}}%
\pgfpathlineto{\pgfqpoint{0.518727in}{0.507943in}}%
\pgfpathlineto{\pgfqpoint{0.510456in}{0.512705in}}%
\pgfpathlineto{\pgfqpoint{0.506297in}{0.515137in}}%
\pgfpathlineto{\pgfqpoint{0.493868in}{0.522054in}}%
\pgfpathlineto{\pgfqpoint{0.489664in}{0.524258in}}%
\pgfpathlineto{\pgfqpoint{0.481439in}{0.528643in}}%
\pgfpathlineto{\pgfqpoint{0.469009in}{0.534866in}}%
\pgfpathlineto{\pgfqpoint{0.466937in}{0.535811in}}%
\pgfpathlineto{\pgfqpoint{0.456580in}{0.540621in}}%
\pgfpathlineto{\pgfqpoint{0.444151in}{0.545876in}}%
\pgfpathlineto{\pgfqpoint{0.440121in}{0.547363in}}%
\pgfpathlineto{\pgfqpoint{0.431722in}{0.550523in}}%
\pgfpathlineto{\pgfqpoint{0.419292in}{0.554493in}}%
\pgfpathlineto{\pgfqpoint{0.406863in}{0.557703in}}%
\pgfpathlineto{\pgfqpoint{0.400344in}{0.558916in}}%
\pgfpathlineto{\pgfqpoint{0.394434in}{0.560036in}}%
\pgfpathlineto{\pgfqpoint{0.382004in}{0.561386in}}%
\pgfpathlineto{\pgfqpoint{0.369575in}{0.561647in}}%
\pgfpathlineto{\pgfqpoint{0.357146in}{0.560688in}}%
\pgfpathlineto{\pgfqpoint{0.347559in}{0.558916in}}%
\pgfpathlineto{\pgfqpoint{0.344716in}{0.558363in}}%
\pgfpathlineto{\pgfqpoint{0.332287in}{0.554469in}}%
\pgfpathlineto{\pgfqpoint{0.319858in}{0.548863in}}%
\pgfpathlineto{\pgfqpoint{0.317301in}{0.547363in}}%
\pgfpathlineto{\pgfqpoint{0.307429in}{0.541212in}}%
\pgfpathlineto{\pgfqpoint{0.300504in}{0.535811in}}%
\pgfpathlineto{\pgfqpoint{0.294999in}{0.531215in}}%
\pgfpathlineto{\pgfqpoint{0.294999in}{0.524258in}}%
\pgfpathlineto{\pgfqpoint{0.294999in}{0.512705in}}%
\pgfpathlineto{\pgfqpoint{0.294999in}{0.501153in}}%
\pgfpathlineto{\pgfqpoint{0.294999in}{0.489600in}}%
\pgfpathlineto{\pgfqpoint{0.294999in}{0.478048in}}%
\pgfpathlineto{\pgfqpoint{0.294999in}{0.466495in}}%
\pgfpathlineto{\pgfqpoint{0.294999in}{0.454943in}}%
\pgfpathlineto{\pgfqpoint{0.294999in}{0.443390in}}%
\pgfpathlineto{\pgfqpoint{0.294999in}{0.431838in}}%
\pgfpathlineto{\pgfqpoint{0.294999in}{0.420285in}}%
\pgfpathlineto{\pgfqpoint{0.294999in}{0.408733in}}%
\pgfpathlineto{\pgfqpoint{0.294999in}{0.397180in}}%
\pgfpathlineto{\pgfqpoint{0.294999in}{0.385627in}}%
\pgfpathlineto{\pgfqpoint{0.294999in}{0.374075in}}%
\pgfpathlineto{\pgfqpoint{0.294999in}{0.362522in}}%
\pgfpathlineto{\pgfqpoint{0.294999in}{0.350970in}}%
\pgfpathlineto{\pgfqpoint{0.294999in}{0.339417in}}%
\pgfpathlineto{\pgfqpoint{0.294999in}{0.327865in}}%
\pgfpathlineto{\pgfqpoint{0.294999in}{0.316312in}}%
\pgfpathlineto{\pgfqpoint{0.294999in}{0.304760in}}%
\pgfpathlineto{\pgfqpoint{0.294999in}{0.293207in}}%
\pgfpathlineto{\pgfqpoint{0.294999in}{0.281654in}}%
\pgfpathlineto{\pgfqpoint{0.294999in}{0.270102in}}%
\pgfpathclose%
\pgfusepath{fill}%
\end{pgfscope}%
\begin{pgfscope}%
\pgfpathrectangle{\pgfqpoint{0.211875in}{0.211875in}}{\pgfqpoint{1.313625in}{1.279725in}}%
\pgfusepath{clip}%
\pgfsetbuttcap%
\pgfsetroundjoin%
\definecolor{currentfill}{rgb}{0.343142,0.118134,0.311397}%
\pgfsetfillcolor{currentfill}%
\pgfsetlinewidth{0.000000pt}%
\definecolor{currentstroke}{rgb}{0.000000,0.000000,0.000000}%
\pgfsetstrokecolor{currentstroke}%
\pgfsetdash{}{0pt}%
\pgfpathmoveto{\pgfqpoint{0.904035in}{0.275107in}}%
\pgfpathlineto{\pgfqpoint{0.913460in}{0.270102in}}%
\pgfpathlineto{\pgfqpoint{0.916464in}{0.270102in}}%
\pgfpathlineto{\pgfqpoint{0.928894in}{0.270102in}}%
\pgfpathlineto{\pgfqpoint{0.941323in}{0.270102in}}%
\pgfpathlineto{\pgfqpoint{0.953752in}{0.270102in}}%
\pgfpathlineto{\pgfqpoint{0.966181in}{0.270102in}}%
\pgfpathlineto{\pgfqpoint{0.978611in}{0.270102in}}%
\pgfpathlineto{\pgfqpoint{0.991040in}{0.270102in}}%
\pgfpathlineto{\pgfqpoint{1.003469in}{0.270102in}}%
\pgfpathlineto{\pgfqpoint{1.015899in}{0.270102in}}%
\pgfpathlineto{\pgfqpoint{1.028328in}{0.270102in}}%
\pgfpathlineto{\pgfqpoint{1.040757in}{0.270102in}}%
\pgfpathlineto{\pgfqpoint{1.053187in}{0.270102in}}%
\pgfpathlineto{\pgfqpoint{1.065616in}{0.270102in}}%
\pgfpathlineto{\pgfqpoint{1.078045in}{0.270102in}}%
\pgfpathlineto{\pgfqpoint{1.090474in}{0.270102in}}%
\pgfpathlineto{\pgfqpoint{1.102904in}{0.270102in}}%
\pgfpathlineto{\pgfqpoint{1.115333in}{0.270102in}}%
\pgfpathlineto{\pgfqpoint{1.127762in}{0.270102in}}%
\pgfpathlineto{\pgfqpoint{1.140192in}{0.270102in}}%
\pgfpathlineto{\pgfqpoint{1.152621in}{0.270102in}}%
\pgfpathlineto{\pgfqpoint{1.158419in}{0.270102in}}%
\pgfpathlineto{\pgfqpoint{1.152621in}{0.271282in}}%
\pgfpathlineto{\pgfqpoint{1.140192in}{0.273951in}}%
\pgfpathlineto{\pgfqpoint{1.127762in}{0.276746in}}%
\pgfpathlineto{\pgfqpoint{1.115333in}{0.279685in}}%
\pgfpathlineto{\pgfqpoint{1.107474in}{0.281654in}}%
\pgfpathlineto{\pgfqpoint{1.102904in}{0.282716in}}%
\pgfpathlineto{\pgfqpoint{1.090474in}{0.285799in}}%
\pgfpathlineto{\pgfqpoint{1.078045in}{0.289060in}}%
\pgfpathlineto{\pgfqpoint{1.065616in}{0.292512in}}%
\pgfpathlineto{\pgfqpoint{1.063267in}{0.293207in}}%
\pgfpathlineto{\pgfqpoint{1.053187in}{0.296013in}}%
\pgfpathlineto{\pgfqpoint{1.040757in}{0.299682in}}%
\pgfpathlineto{\pgfqpoint{1.028328in}{0.303564in}}%
\pgfpathlineto{\pgfqpoint{1.024715in}{0.304760in}}%
\pgfpathlineto{\pgfqpoint{1.015899in}{0.307532in}}%
\pgfpathlineto{\pgfqpoint{1.003469in}{0.311666in}}%
\pgfpathlineto{\pgfqpoint{0.991040in}{0.316030in}}%
\pgfpathlineto{\pgfqpoint{0.990278in}{0.316312in}}%
\pgfpathlineto{\pgfqpoint{0.978611in}{0.320448in}}%
\pgfpathlineto{\pgfqpoint{0.966181in}{0.325090in}}%
\pgfpathlineto{\pgfqpoint{0.959113in}{0.327865in}}%
\pgfpathlineto{\pgfqpoint{0.953752in}{0.329893in}}%
\pgfpathlineto{\pgfqpoint{0.941323in}{0.334828in}}%
\pgfpathlineto{\pgfqpoint{0.930312in}{0.339417in}}%
\pgfpathlineto{\pgfqpoint{0.928894in}{0.339990in}}%
\pgfpathlineto{\pgfqpoint{0.916464in}{0.345232in}}%
\pgfpathlineto{\pgfqpoint{0.904035in}{0.350725in}}%
\pgfpathlineto{\pgfqpoint{0.903503in}{0.350970in}}%
\pgfpathlineto{\pgfqpoint{0.891606in}{0.356294in}}%
\pgfpathlineto{\pgfqpoint{0.879176in}{0.362108in}}%
\pgfpathlineto{\pgfqpoint{0.878322in}{0.362522in}}%
\pgfpathlineto{\pgfqpoint{0.866747in}{0.368009in}}%
\pgfpathlineto{\pgfqpoint{0.854478in}{0.374075in}}%
\pgfpathlineto{\pgfqpoint{0.854318in}{0.374152in}}%
\pgfpathlineto{\pgfqpoint{0.841888in}{0.380378in}}%
\pgfpathlineto{\pgfqpoint{0.831802in}{0.385627in}}%
\pgfpathlineto{\pgfqpoint{0.829459in}{0.386830in}}%
\pgfpathlineto{\pgfqpoint{0.817030in}{0.393399in}}%
\pgfpathlineto{\pgfqpoint{0.810108in}{0.397180in}}%
\pgfpathlineto{\pgfqpoint{0.804601in}{0.400155in}}%
\pgfpathlineto{\pgfqpoint{0.792171in}{0.407075in}}%
\pgfpathlineto{\pgfqpoint{0.789270in}{0.408733in}}%
\pgfpathlineto{\pgfqpoint{0.779742in}{0.414132in}}%
\pgfpathlineto{\pgfqpoint{0.769205in}{0.420285in}}%
\pgfpathlineto{\pgfqpoint{0.767313in}{0.421385in}}%
\pgfpathlineto{\pgfqpoint{0.754883in}{0.428763in}}%
\pgfpathlineto{\pgfqpoint{0.749825in}{0.431838in}}%
\pgfpathlineto{\pgfqpoint{0.742454in}{0.436309in}}%
\pgfpathlineto{\pgfqpoint{0.731061in}{0.443390in}}%
\pgfpathlineto{\pgfqpoint{0.730025in}{0.444035in}}%
\pgfpathlineto{\pgfqpoint{0.717595in}{0.451884in}}%
\pgfpathlineto{\pgfqpoint{0.712841in}{0.454943in}}%
\pgfpathlineto{\pgfqpoint{0.705166in}{0.459898in}}%
\pgfpathlineto{\pgfqpoint{0.695130in}{0.466495in}}%
\pgfpathlineto{\pgfqpoint{0.692737in}{0.468077in}}%
\pgfpathlineto{\pgfqpoint{0.680308in}{0.476395in}}%
\pgfpathlineto{\pgfqpoint{0.677862in}{0.478048in}}%
\pgfpathlineto{\pgfqpoint{0.667878in}{0.484852in}}%
\pgfpathlineto{\pgfqpoint{0.660990in}{0.489600in}}%
\pgfpathlineto{\pgfqpoint{0.655449in}{0.493460in}}%
\pgfpathlineto{\pgfqpoint{0.644503in}{0.501153in}}%
\pgfpathlineto{\pgfqpoint{0.643020in}{0.502209in}}%
\pgfpathlineto{\pgfqpoint{0.630590in}{0.511077in}}%
\pgfpathlineto{\pgfqpoint{0.628310in}{0.512705in}}%
\pgfpathlineto{\pgfqpoint{0.618161in}{0.520064in}}%
\pgfpathlineto{\pgfqpoint{0.612381in}{0.524258in}}%
\pgfpathlineto{\pgfqpoint{0.605732in}{0.529165in}}%
\pgfpathlineto{\pgfqpoint{0.596706in}{0.535811in}}%
\pgfpathlineto{\pgfqpoint{0.593302in}{0.538364in}}%
\pgfpathlineto{\pgfqpoint{0.581241in}{0.547363in}}%
\pgfpathlineto{\pgfqpoint{0.580873in}{0.547643in}}%
\pgfpathlineto{\pgfqpoint{0.568444in}{0.556981in}}%
\pgfpathlineto{\pgfqpoint{0.565834in}{0.558916in}}%
\pgfpathlineto{\pgfqpoint{0.556015in}{0.566361in}}%
\pgfpathlineto{\pgfqpoint{0.550506in}{0.570468in}}%
\pgfpathlineto{\pgfqpoint{0.543585in}{0.575754in}}%
\pgfpathlineto{\pgfqpoint{0.535204in}{0.582021in}}%
\pgfpathlineto{\pgfqpoint{0.531156in}{0.585127in}}%
\pgfpathlineto{\pgfqpoint{0.519855in}{0.593573in}}%
\pgfpathlineto{\pgfqpoint{0.518727in}{0.594441in}}%
\pgfpathlineto{\pgfqpoint{0.506297in}{0.603659in}}%
\pgfpathlineto{\pgfqpoint{0.504240in}{0.605126in}}%
\pgfpathlineto{\pgfqpoint{0.493868in}{0.612734in}}%
\pgfpathlineto{\pgfqpoint{0.488249in}{0.616678in}}%
\pgfpathlineto{\pgfqpoint{0.481439in}{0.621604in}}%
\pgfpathlineto{\pgfqpoint{0.471789in}{0.628231in}}%
\pgfpathlineto{\pgfqpoint{0.469009in}{0.630201in}}%
\pgfpathlineto{\pgfqpoint{0.456580in}{0.638463in}}%
\pgfpathlineto{\pgfqpoint{0.454436in}{0.639784in}}%
\pgfpathlineto{\pgfqpoint{0.444151in}{0.646324in}}%
\pgfpathlineto{\pgfqpoint{0.435588in}{0.651336in}}%
\pgfpathlineto{\pgfqpoint{0.431722in}{0.653675in}}%
\pgfpathlineto{\pgfqpoint{0.419292in}{0.660443in}}%
\pgfpathlineto{\pgfqpoint{0.414201in}{0.662889in}}%
\pgfpathlineto{\pgfqpoint{0.406863in}{0.666533in}}%
\pgfpathlineto{\pgfqpoint{0.394434in}{0.671835in}}%
\pgfpathlineto{\pgfqpoint{0.387001in}{0.674441in}}%
\pgfpathlineto{\pgfqpoint{0.382004in}{0.676253in}}%
\pgfpathlineto{\pgfqpoint{0.369575in}{0.679694in}}%
\pgfpathlineto{\pgfqpoint{0.357146in}{0.682033in}}%
\pgfpathlineto{\pgfqpoint{0.344716in}{0.683177in}}%
\pgfpathlineto{\pgfqpoint{0.332287in}{0.683027in}}%
\pgfpathlineto{\pgfqpoint{0.319858in}{0.681480in}}%
\pgfpathlineto{\pgfqpoint{0.307429in}{0.678429in}}%
\pgfpathlineto{\pgfqpoint{0.296766in}{0.674441in}}%
\pgfpathlineto{\pgfqpoint{0.294999in}{0.673768in}}%
\pgfpathlineto{\pgfqpoint{0.294999in}{0.662889in}}%
\pgfpathlineto{\pgfqpoint{0.294999in}{0.651336in}}%
\pgfpathlineto{\pgfqpoint{0.294999in}{0.639784in}}%
\pgfpathlineto{\pgfqpoint{0.294999in}{0.628231in}}%
\pgfpathlineto{\pgfqpoint{0.294999in}{0.616678in}}%
\pgfpathlineto{\pgfqpoint{0.294999in}{0.605126in}}%
\pgfpathlineto{\pgfqpoint{0.294999in}{0.593573in}}%
\pgfpathlineto{\pgfqpoint{0.294999in}{0.582021in}}%
\pgfpathlineto{\pgfqpoint{0.294999in}{0.570468in}}%
\pgfpathlineto{\pgfqpoint{0.294999in}{0.558916in}}%
\pgfpathlineto{\pgfqpoint{0.294999in}{0.547363in}}%
\pgfpathlineto{\pgfqpoint{0.294999in}{0.535811in}}%
\pgfpathlineto{\pgfqpoint{0.294999in}{0.531215in}}%
\pgfpathlineto{\pgfqpoint{0.300504in}{0.535811in}}%
\pgfpathlineto{\pgfqpoint{0.307429in}{0.541212in}}%
\pgfpathlineto{\pgfqpoint{0.317301in}{0.547363in}}%
\pgfpathlineto{\pgfqpoint{0.319858in}{0.548863in}}%
\pgfpathlineto{\pgfqpoint{0.332287in}{0.554469in}}%
\pgfpathlineto{\pgfqpoint{0.344716in}{0.558363in}}%
\pgfpathlineto{\pgfqpoint{0.347559in}{0.558916in}}%
\pgfpathlineto{\pgfqpoint{0.357146in}{0.560688in}}%
\pgfpathlineto{\pgfqpoint{0.369575in}{0.561647in}}%
\pgfpathlineto{\pgfqpoint{0.382004in}{0.561386in}}%
\pgfpathlineto{\pgfqpoint{0.394434in}{0.560036in}}%
\pgfpathlineto{\pgfqpoint{0.400344in}{0.558916in}}%
\pgfpathlineto{\pgfqpoint{0.406863in}{0.557703in}}%
\pgfpathlineto{\pgfqpoint{0.419292in}{0.554493in}}%
\pgfpathlineto{\pgfqpoint{0.431722in}{0.550523in}}%
\pgfpathlineto{\pgfqpoint{0.440121in}{0.547363in}}%
\pgfpathlineto{\pgfqpoint{0.444151in}{0.545876in}}%
\pgfpathlineto{\pgfqpoint{0.456580in}{0.540621in}}%
\pgfpathlineto{\pgfqpoint{0.466937in}{0.535811in}}%
\pgfpathlineto{\pgfqpoint{0.469009in}{0.534866in}}%
\pgfpathlineto{\pgfqpoint{0.481439in}{0.528643in}}%
\pgfpathlineto{\pgfqpoint{0.489664in}{0.524258in}}%
\pgfpathlineto{\pgfqpoint{0.493868in}{0.522054in}}%
\pgfpathlineto{\pgfqpoint{0.506297in}{0.515137in}}%
\pgfpathlineto{\pgfqpoint{0.510456in}{0.512705in}}%
\pgfpathlineto{\pgfqpoint{0.518727in}{0.507943in}}%
\pgfpathlineto{\pgfqpoint{0.530131in}{0.501153in}}%
\pgfpathlineto{\pgfqpoint{0.531156in}{0.500551in}}%
\pgfpathlineto{\pgfqpoint{0.543585in}{0.492949in}}%
\pgfpathlineto{\pgfqpoint{0.548931in}{0.489600in}}%
\pgfpathlineto{\pgfqpoint{0.556015in}{0.485215in}}%
\pgfpathlineto{\pgfqpoint{0.567397in}{0.478048in}}%
\pgfpathlineto{\pgfqpoint{0.568444in}{0.477395in}}%
\pgfpathlineto{\pgfqpoint{0.580873in}{0.469463in}}%
\pgfpathlineto{\pgfqpoint{0.585474in}{0.466495in}}%
\pgfpathlineto{\pgfqpoint{0.593302in}{0.461483in}}%
\pgfpathlineto{\pgfqpoint{0.603469in}{0.454943in}}%
\pgfpathlineto{\pgfqpoint{0.605732in}{0.453495in}}%
\pgfpathlineto{\pgfqpoint{0.618161in}{0.445474in}}%
\pgfpathlineto{\pgfqpoint{0.621380in}{0.443390in}}%
\pgfpathlineto{\pgfqpoint{0.630590in}{0.437445in}}%
\pgfpathlineto{\pgfqpoint{0.639318in}{0.431838in}}%
\pgfpathlineto{\pgfqpoint{0.643020in}{0.429460in}}%
\pgfpathlineto{\pgfqpoint{0.655449in}{0.421505in}}%
\pgfpathlineto{\pgfqpoint{0.657359in}{0.420285in}}%
\pgfpathlineto{\pgfqpoint{0.667878in}{0.413553in}}%
\pgfpathlineto{\pgfqpoint{0.675502in}{0.408733in}}%
\pgfpathlineto{\pgfqpoint{0.680308in}{0.405679in}}%
\pgfpathlineto{\pgfqpoint{0.692737in}{0.397878in}}%
\pgfpathlineto{\pgfqpoint{0.693858in}{0.397180in}}%
\pgfpathlineto{\pgfqpoint{0.705166in}{0.390079in}}%
\pgfpathlineto{\pgfqpoint{0.712389in}{0.385627in}}%
\pgfpathlineto{\pgfqpoint{0.717595in}{0.382384in}}%
\pgfpathlineto{\pgfqpoint{0.730025in}{0.374781in}}%
\pgfpathlineto{\pgfqpoint{0.731194in}{0.374075in}}%
\pgfpathlineto{\pgfqpoint{0.742454in}{0.367180in}}%
\pgfpathlineto{\pgfqpoint{0.750245in}{0.362522in}}%
\pgfpathlineto{\pgfqpoint{0.754883in}{0.359702in}}%
\pgfpathlineto{\pgfqpoint{0.767313in}{0.352306in}}%
\pgfpathlineto{\pgfqpoint{0.769602in}{0.350970in}}%
\pgfpathlineto{\pgfqpoint{0.779742in}{0.344928in}}%
\pgfpathlineto{\pgfqpoint{0.789261in}{0.339417in}}%
\pgfpathlineto{\pgfqpoint{0.792171in}{0.337690in}}%
\pgfpathlineto{\pgfqpoint{0.804601in}{0.330483in}}%
\pgfpathlineto{\pgfqpoint{0.809241in}{0.327865in}}%
\pgfpathlineto{\pgfqpoint{0.817030in}{0.323341in}}%
\pgfpathlineto{\pgfqpoint{0.829459in}{0.316352in}}%
\pgfpathlineto{\pgfqpoint{0.829531in}{0.316312in}}%
\pgfpathlineto{\pgfqpoint{0.841888in}{0.309292in}}%
\pgfpathlineto{\pgfqpoint{0.850149in}{0.304760in}}%
\pgfpathlineto{\pgfqpoint{0.854318in}{0.302384in}}%
\pgfpathlineto{\pgfqpoint{0.866747in}{0.295510in}}%
\pgfpathlineto{\pgfqpoint{0.871046in}{0.293207in}}%
\pgfpathlineto{\pgfqpoint{0.879176in}{0.288655in}}%
\pgfpathlineto{\pgfqpoint{0.891606in}{0.281944in}}%
\pgfpathlineto{\pgfqpoint{0.892159in}{0.281654in}}%
\pgfpathclose%
\pgfusepath{fill}%
\end{pgfscope}%
\begin{pgfscope}%
\pgfpathrectangle{\pgfqpoint{0.211875in}{0.211875in}}{\pgfqpoint{1.313625in}{1.279725in}}%
\pgfusepath{clip}%
\pgfsetbuttcap%
\pgfsetroundjoin%
\definecolor{currentfill}{rgb}{0.490838,0.119982,0.351115}%
\pgfsetfillcolor{currentfill}%
\pgfsetlinewidth{0.000000pt}%
\definecolor{currentstroke}{rgb}{0.000000,0.000000,0.000000}%
\pgfsetstrokecolor{currentstroke}%
\pgfsetdash{}{0pt}%
\pgfpathmoveto{\pgfqpoint{1.115333in}{0.279685in}}%
\pgfpathlineto{\pgfqpoint{1.127762in}{0.276746in}}%
\pgfpathlineto{\pgfqpoint{1.140192in}{0.273951in}}%
\pgfpathlineto{\pgfqpoint{1.152621in}{0.271282in}}%
\pgfpathlineto{\pgfqpoint{1.158419in}{0.270102in}}%
\pgfpathlineto{\pgfqpoint{1.165050in}{0.270102in}}%
\pgfpathlineto{\pgfqpoint{1.177480in}{0.270102in}}%
\pgfpathlineto{\pgfqpoint{1.189909in}{0.270102in}}%
\pgfpathlineto{\pgfqpoint{1.202338in}{0.270102in}}%
\pgfpathlineto{\pgfqpoint{1.214767in}{0.270102in}}%
\pgfpathlineto{\pgfqpoint{1.227197in}{0.270102in}}%
\pgfpathlineto{\pgfqpoint{1.239626in}{0.270102in}}%
\pgfpathlineto{\pgfqpoint{1.252055in}{0.270102in}}%
\pgfpathlineto{\pgfqpoint{1.264485in}{0.270102in}}%
\pgfpathlineto{\pgfqpoint{1.276914in}{0.270102in}}%
\pgfpathlineto{\pgfqpoint{1.289343in}{0.270102in}}%
\pgfpathlineto{\pgfqpoint{1.301773in}{0.270102in}}%
\pgfpathlineto{\pgfqpoint{1.314202in}{0.270102in}}%
\pgfpathlineto{\pgfqpoint{1.326631in}{0.270102in}}%
\pgfpathlineto{\pgfqpoint{1.339060in}{0.270102in}}%
\pgfpathlineto{\pgfqpoint{1.351490in}{0.270102in}}%
\pgfpathlineto{\pgfqpoint{1.363919in}{0.270102in}}%
\pgfpathlineto{\pgfqpoint{1.376348in}{0.270102in}}%
\pgfpathlineto{\pgfqpoint{1.388778in}{0.270102in}}%
\pgfpathlineto{\pgfqpoint{1.401207in}{0.270102in}}%
\pgfpathlineto{\pgfqpoint{1.413636in}{0.270102in}}%
\pgfpathlineto{\pgfqpoint{1.426066in}{0.270102in}}%
\pgfpathlineto{\pgfqpoint{1.438495in}{0.270102in}}%
\pgfpathlineto{\pgfqpoint{1.450924in}{0.270102in}}%
\pgfpathlineto{\pgfqpoint{1.463353in}{0.270102in}}%
\pgfpathlineto{\pgfqpoint{1.475783in}{0.270102in}}%
\pgfpathlineto{\pgfqpoint{1.488212in}{0.270102in}}%
\pgfpathlineto{\pgfqpoint{1.500641in}{0.270102in}}%
\pgfpathlineto{\pgfqpoint{1.513071in}{0.270102in}}%
\pgfpathlineto{\pgfqpoint{1.525500in}{0.270102in}}%
\pgfpathlineto{\pgfqpoint{1.525500in}{0.281654in}}%
\pgfpathlineto{\pgfqpoint{1.525500in}{0.293207in}}%
\pgfpathlineto{\pgfqpoint{1.525500in}{0.304760in}}%
\pgfpathlineto{\pgfqpoint{1.525500in}{0.316312in}}%
\pgfpathlineto{\pgfqpoint{1.525500in}{0.327865in}}%
\pgfpathlineto{\pgfqpoint{1.525500in}{0.339417in}}%
\pgfpathlineto{\pgfqpoint{1.525500in}{0.350970in}}%
\pgfpathlineto{\pgfqpoint{1.525500in}{0.361305in}}%
\pgfpathlineto{\pgfqpoint{1.524387in}{0.362522in}}%
\pgfpathlineto{\pgfqpoint{1.513071in}{0.372214in}}%
\pgfpathlineto{\pgfqpoint{1.510108in}{0.374075in}}%
\pgfpathlineto{\pgfqpoint{1.500641in}{0.378963in}}%
\pgfpathlineto{\pgfqpoint{1.488212in}{0.383255in}}%
\pgfpathlineto{\pgfqpoint{1.476776in}{0.385627in}}%
\pgfpathlineto{\pgfqpoint{1.475783in}{0.385804in}}%
\pgfpathlineto{\pgfqpoint{1.463353in}{0.386968in}}%
\pgfpathlineto{\pgfqpoint{1.450924in}{0.387194in}}%
\pgfpathlineto{\pgfqpoint{1.438495in}{0.386696in}}%
\pgfpathlineto{\pgfqpoint{1.426066in}{0.385648in}}%
\pgfpathlineto{\pgfqpoint{1.425903in}{0.385627in}}%
\pgfpathlineto{\pgfqpoint{1.413636in}{0.384111in}}%
\pgfpathlineto{\pgfqpoint{1.401207in}{0.382265in}}%
\pgfpathlineto{\pgfqpoint{1.388778in}{0.380208in}}%
\pgfpathlineto{\pgfqpoint{1.376348in}{0.378017in}}%
\pgfpathlineto{\pgfqpoint{1.363919in}{0.375756in}}%
\pgfpathlineto{\pgfqpoint{1.354859in}{0.374075in}}%
\pgfpathlineto{\pgfqpoint{1.351490in}{0.373451in}}%
\pgfpathlineto{\pgfqpoint{1.339060in}{0.371102in}}%
\pgfpathlineto{\pgfqpoint{1.326631in}{0.368823in}}%
\pgfpathlineto{\pgfqpoint{1.314202in}{0.366646in}}%
\pgfpathlineto{\pgfqpoint{1.301773in}{0.364592in}}%
\pgfpathlineto{\pgfqpoint{1.289343in}{0.362684in}}%
\pgfpathlineto{\pgfqpoint{1.288231in}{0.362522in}}%
\pgfpathlineto{\pgfqpoint{1.276914in}{0.360873in}}%
\pgfpathlineto{\pgfqpoint{1.264485in}{0.359242in}}%
\pgfpathlineto{\pgfqpoint{1.252055in}{0.357807in}}%
\pgfpathlineto{\pgfqpoint{1.239626in}{0.356578in}}%
\pgfpathlineto{\pgfqpoint{1.227197in}{0.355562in}}%
\pgfpathlineto{\pgfqpoint{1.214767in}{0.354766in}}%
\pgfpathlineto{\pgfqpoint{1.202338in}{0.354193in}}%
\pgfpathlineto{\pgfqpoint{1.189909in}{0.353849in}}%
\pgfpathlineto{\pgfqpoint{1.177480in}{0.353738in}}%
\pgfpathlineto{\pgfqpoint{1.165050in}{0.353862in}}%
\pgfpathlineto{\pgfqpoint{1.152621in}{0.354225in}}%
\pgfpathlineto{\pgfqpoint{1.140192in}{0.354830in}}%
\pgfpathlineto{\pgfqpoint{1.127762in}{0.355678in}}%
\pgfpathlineto{\pgfqpoint{1.115333in}{0.356772in}}%
\pgfpathlineto{\pgfqpoint{1.102904in}{0.358114in}}%
\pgfpathlineto{\pgfqpoint{1.090474in}{0.359706in}}%
\pgfpathlineto{\pgfqpoint{1.078045in}{0.361550in}}%
\pgfpathlineto{\pgfqpoint{1.072316in}{0.362522in}}%
\pgfpathlineto{\pgfqpoint{1.065616in}{0.363618in}}%
\pgfpathlineto{\pgfqpoint{1.053187in}{0.365908in}}%
\pgfpathlineto{\pgfqpoint{1.040757in}{0.368452in}}%
\pgfpathlineto{\pgfqpoint{1.028328in}{0.371250in}}%
\pgfpathlineto{\pgfqpoint{1.016844in}{0.374075in}}%
\pgfpathlineto{\pgfqpoint{1.015899in}{0.374301in}}%
\pgfpathlineto{\pgfqpoint{1.003469in}{0.377535in}}%
\pgfpathlineto{\pgfqpoint{0.991040in}{0.381026in}}%
\pgfpathlineto{\pgfqpoint{0.978611in}{0.384778in}}%
\pgfpathlineto{\pgfqpoint{0.975978in}{0.385627in}}%
\pgfpathlineto{\pgfqpoint{0.966181in}{0.388721in}}%
\pgfpathlineto{\pgfqpoint{0.953752in}{0.392905in}}%
\pgfpathlineto{\pgfqpoint{0.941801in}{0.397180in}}%
\pgfpathlineto{\pgfqpoint{0.941323in}{0.397348in}}%
\pgfpathlineto{\pgfqpoint{0.928894in}{0.401966in}}%
\pgfpathlineto{\pgfqpoint{0.916464in}{0.406846in}}%
\pgfpathlineto{\pgfqpoint{0.911895in}{0.408733in}}%
\pgfpathlineto{\pgfqpoint{0.904035in}{0.411935in}}%
\pgfpathlineto{\pgfqpoint{0.891606in}{0.417253in}}%
\pgfpathlineto{\pgfqpoint{0.884836in}{0.420285in}}%
\pgfpathlineto{\pgfqpoint{0.879176in}{0.422797in}}%
\pgfpathlineto{\pgfqpoint{0.866747in}{0.428557in}}%
\pgfpathlineto{\pgfqpoint{0.859957in}{0.431838in}}%
\pgfpathlineto{\pgfqpoint{0.854318in}{0.434546in}}%
\pgfpathlineto{\pgfqpoint{0.841888in}{0.440753in}}%
\pgfpathlineto{\pgfqpoint{0.836799in}{0.443390in}}%
\pgfpathlineto{\pgfqpoint{0.829459in}{0.447182in}}%
\pgfpathlineto{\pgfqpoint{0.817030in}{0.453844in}}%
\pgfpathlineto{\pgfqpoint{0.815041in}{0.454943in}}%
\pgfpathlineto{\pgfqpoint{0.804601in}{0.460712in}}%
\pgfpathlineto{\pgfqpoint{0.794488in}{0.466495in}}%
\pgfpathlineto{\pgfqpoint{0.792171in}{0.467824in}}%
\pgfpathlineto{\pgfqpoint{0.779742in}{0.475148in}}%
\pgfpathlineto{\pgfqpoint{0.774956in}{0.478048in}}%
\pgfpathlineto{\pgfqpoint{0.767313in}{0.482704in}}%
\pgfpathlineto{\pgfqpoint{0.756305in}{0.489600in}}%
\pgfpathlineto{\pgfqpoint{0.754883in}{0.490498in}}%
\pgfpathlineto{\pgfqpoint{0.742454in}{0.498514in}}%
\pgfpathlineto{\pgfqpoint{0.738452in}{0.501153in}}%
\pgfpathlineto{\pgfqpoint{0.730025in}{0.506768in}}%
\pgfpathlineto{\pgfqpoint{0.721304in}{0.512705in}}%
\pgfpathlineto{\pgfqpoint{0.717595in}{0.515262in}}%
\pgfpathlineto{\pgfqpoint{0.705166in}{0.523995in}}%
\pgfpathlineto{\pgfqpoint{0.704796in}{0.524258in}}%
\pgfpathlineto{\pgfqpoint{0.692737in}{0.532970in}}%
\pgfpathlineto{\pgfqpoint{0.688862in}{0.535811in}}%
\pgfpathlineto{\pgfqpoint{0.680308in}{0.542193in}}%
\pgfpathlineto{\pgfqpoint{0.673462in}{0.547363in}}%
\pgfpathlineto{\pgfqpoint{0.667878in}{0.551663in}}%
\pgfpathlineto{\pgfqpoint{0.658553in}{0.558916in}}%
\pgfpathlineto{\pgfqpoint{0.655449in}{0.561382in}}%
\pgfpathlineto{\pgfqpoint{0.644097in}{0.570468in}}%
\pgfpathlineto{\pgfqpoint{0.643020in}{0.571351in}}%
\pgfpathlineto{\pgfqpoint{0.630590in}{0.581573in}}%
\pgfpathlineto{\pgfqpoint{0.630047in}{0.582021in}}%
\pgfpathlineto{\pgfqpoint{0.618161in}{0.592054in}}%
\pgfpathlineto{\pgfqpoint{0.616359in}{0.593573in}}%
\pgfpathlineto{\pgfqpoint{0.605732in}{0.602784in}}%
\pgfpathlineto{\pgfqpoint{0.603019in}{0.605126in}}%
\pgfpathlineto{\pgfqpoint{0.593302in}{0.613761in}}%
\pgfpathlineto{\pgfqpoint{0.589994in}{0.616678in}}%
\pgfpathlineto{\pgfqpoint{0.580873in}{0.624977in}}%
\pgfpathlineto{\pgfqpoint{0.577256in}{0.628231in}}%
\pgfpathlineto{\pgfqpoint{0.568444in}{0.636423in}}%
\pgfpathlineto{\pgfqpoint{0.564774in}{0.639784in}}%
\pgfpathlineto{\pgfqpoint{0.556015in}{0.648086in}}%
\pgfpathlineto{\pgfqpoint{0.552519in}{0.651336in}}%
\pgfpathlineto{\pgfqpoint{0.543585in}{0.659946in}}%
\pgfpathlineto{\pgfqpoint{0.540459in}{0.662889in}}%
\pgfpathlineto{\pgfqpoint{0.531156in}{0.671979in}}%
\pgfpathlineto{\pgfqpoint{0.528563in}{0.674441in}}%
\pgfpathlineto{\pgfqpoint{0.518727in}{0.684152in}}%
\pgfpathlineto{\pgfqpoint{0.516798in}{0.685994in}}%
\pgfpathlineto{\pgfqpoint{0.506297in}{0.696427in}}%
\pgfpathlineto{\pgfqpoint{0.505125in}{0.697546in}}%
\pgfpathlineto{\pgfqpoint{0.493868in}{0.708752in}}%
\pgfpathlineto{\pgfqpoint{0.493503in}{0.709099in}}%
\pgfpathlineto{\pgfqpoint{0.481851in}{0.720651in}}%
\pgfpathlineto{\pgfqpoint{0.481439in}{0.721079in}}%
\pgfpathlineto{\pgfqpoint{0.470107in}{0.732204in}}%
\pgfpathlineto{\pgfqpoint{0.469009in}{0.733331in}}%
\pgfpathlineto{\pgfqpoint{0.458216in}{0.743757in}}%
\pgfpathlineto{\pgfqpoint{0.456580in}{0.745411in}}%
\pgfpathlineto{\pgfqpoint{0.446079in}{0.755309in}}%
\pgfpathlineto{\pgfqpoint{0.444151in}{0.757213in}}%
\pgfpathlineto{\pgfqpoint{0.433559in}{0.766862in}}%
\pgfpathlineto{\pgfqpoint{0.431722in}{0.768616in}}%
\pgfpathlineto{\pgfqpoint{0.420460in}{0.778414in}}%
\pgfpathlineto{\pgfqpoint{0.419292in}{0.779480in}}%
\pgfpathlineto{\pgfqpoint{0.406863in}{0.789659in}}%
\pgfpathlineto{\pgfqpoint{0.406436in}{0.789967in}}%
\pgfpathlineto{\pgfqpoint{0.394434in}{0.799038in}}%
\pgfpathlineto{\pgfqpoint{0.390617in}{0.801519in}}%
\pgfpathlineto{\pgfqpoint{0.382004in}{0.807392in}}%
\pgfpathlineto{\pgfqpoint{0.371995in}{0.813072in}}%
\pgfpathlineto{\pgfqpoint{0.369575in}{0.814513in}}%
\pgfpathlineto{\pgfqpoint{0.357146in}{0.820345in}}%
\pgfpathlineto{\pgfqpoint{0.344716in}{0.824601in}}%
\pgfpathlineto{\pgfqpoint{0.344603in}{0.824624in}}%
\pgfpathlineto{\pgfqpoint{0.332287in}{0.827318in}}%
\pgfpathlineto{\pgfqpoint{0.319858in}{0.828271in}}%
\pgfpathlineto{\pgfqpoint{0.307429in}{0.827401in}}%
\pgfpathlineto{\pgfqpoint{0.294999in}{0.824646in}}%
\pgfpathlineto{\pgfqpoint{0.294999in}{0.824624in}}%
\pgfpathlineto{\pgfqpoint{0.294999in}{0.813072in}}%
\pgfpathlineto{\pgfqpoint{0.294999in}{0.801519in}}%
\pgfpathlineto{\pgfqpoint{0.294999in}{0.789967in}}%
\pgfpathlineto{\pgfqpoint{0.294999in}{0.778414in}}%
\pgfpathlineto{\pgfqpoint{0.294999in}{0.766862in}}%
\pgfpathlineto{\pgfqpoint{0.294999in}{0.755309in}}%
\pgfpathlineto{\pgfqpoint{0.294999in}{0.743757in}}%
\pgfpathlineto{\pgfqpoint{0.294999in}{0.732204in}}%
\pgfpathlineto{\pgfqpoint{0.294999in}{0.720651in}}%
\pgfpathlineto{\pgfqpoint{0.294999in}{0.709099in}}%
\pgfpathlineto{\pgfqpoint{0.294999in}{0.697546in}}%
\pgfpathlineto{\pgfqpoint{0.294999in}{0.685994in}}%
\pgfpathlineto{\pgfqpoint{0.294999in}{0.674441in}}%
\pgfpathlineto{\pgfqpoint{0.294999in}{0.673768in}}%
\pgfpathlineto{\pgfqpoint{0.296766in}{0.674441in}}%
\pgfpathlineto{\pgfqpoint{0.307429in}{0.678429in}}%
\pgfpathlineto{\pgfqpoint{0.319858in}{0.681480in}}%
\pgfpathlineto{\pgfqpoint{0.332287in}{0.683027in}}%
\pgfpathlineto{\pgfqpoint{0.344716in}{0.683177in}}%
\pgfpathlineto{\pgfqpoint{0.357146in}{0.682033in}}%
\pgfpathlineto{\pgfqpoint{0.369575in}{0.679694in}}%
\pgfpathlineto{\pgfqpoint{0.382004in}{0.676253in}}%
\pgfpathlineto{\pgfqpoint{0.387001in}{0.674441in}}%
\pgfpathlineto{\pgfqpoint{0.394434in}{0.671835in}}%
\pgfpathlineto{\pgfqpoint{0.406863in}{0.666533in}}%
\pgfpathlineto{\pgfqpoint{0.414201in}{0.662889in}}%
\pgfpathlineto{\pgfqpoint{0.419292in}{0.660443in}}%
\pgfpathlineto{\pgfqpoint{0.431722in}{0.653675in}}%
\pgfpathlineto{\pgfqpoint{0.435588in}{0.651336in}}%
\pgfpathlineto{\pgfqpoint{0.444151in}{0.646324in}}%
\pgfpathlineto{\pgfqpoint{0.454436in}{0.639784in}}%
\pgfpathlineto{\pgfqpoint{0.456580in}{0.638463in}}%
\pgfpathlineto{\pgfqpoint{0.469009in}{0.630201in}}%
\pgfpathlineto{\pgfqpoint{0.471789in}{0.628231in}}%
\pgfpathlineto{\pgfqpoint{0.481439in}{0.621604in}}%
\pgfpathlineto{\pgfqpoint{0.488249in}{0.616678in}}%
\pgfpathlineto{\pgfqpoint{0.493868in}{0.612734in}}%
\pgfpathlineto{\pgfqpoint{0.504240in}{0.605126in}}%
\pgfpathlineto{\pgfqpoint{0.506297in}{0.603659in}}%
\pgfpathlineto{\pgfqpoint{0.518727in}{0.594441in}}%
\pgfpathlineto{\pgfqpoint{0.519855in}{0.593573in}}%
\pgfpathlineto{\pgfqpoint{0.531156in}{0.585127in}}%
\pgfpathlineto{\pgfqpoint{0.535204in}{0.582021in}}%
\pgfpathlineto{\pgfqpoint{0.543585in}{0.575754in}}%
\pgfpathlineto{\pgfqpoint{0.550506in}{0.570468in}}%
\pgfpathlineto{\pgfqpoint{0.556015in}{0.566361in}}%
\pgfpathlineto{\pgfqpoint{0.565834in}{0.558916in}}%
\pgfpathlineto{\pgfqpoint{0.568444in}{0.556981in}}%
\pgfpathlineto{\pgfqpoint{0.580873in}{0.547643in}}%
\pgfpathlineto{\pgfqpoint{0.581241in}{0.547363in}}%
\pgfpathlineto{\pgfqpoint{0.593302in}{0.538364in}}%
\pgfpathlineto{\pgfqpoint{0.596706in}{0.535811in}}%
\pgfpathlineto{\pgfqpoint{0.605732in}{0.529165in}}%
\pgfpathlineto{\pgfqpoint{0.612381in}{0.524258in}}%
\pgfpathlineto{\pgfqpoint{0.618161in}{0.520064in}}%
\pgfpathlineto{\pgfqpoint{0.628310in}{0.512705in}}%
\pgfpathlineto{\pgfqpoint{0.630590in}{0.511077in}}%
\pgfpathlineto{\pgfqpoint{0.643020in}{0.502209in}}%
\pgfpathlineto{\pgfqpoint{0.644503in}{0.501153in}}%
\pgfpathlineto{\pgfqpoint{0.655449in}{0.493460in}}%
\pgfpathlineto{\pgfqpoint{0.660990in}{0.489600in}}%
\pgfpathlineto{\pgfqpoint{0.667878in}{0.484852in}}%
\pgfpathlineto{\pgfqpoint{0.677862in}{0.478048in}}%
\pgfpathlineto{\pgfqpoint{0.680308in}{0.476395in}}%
\pgfpathlineto{\pgfqpoint{0.692737in}{0.468077in}}%
\pgfpathlineto{\pgfqpoint{0.695130in}{0.466495in}}%
\pgfpathlineto{\pgfqpoint{0.705166in}{0.459898in}}%
\pgfpathlineto{\pgfqpoint{0.712841in}{0.454943in}}%
\pgfpathlineto{\pgfqpoint{0.717595in}{0.451884in}}%
\pgfpathlineto{\pgfqpoint{0.730025in}{0.444035in}}%
\pgfpathlineto{\pgfqpoint{0.731061in}{0.443390in}}%
\pgfpathlineto{\pgfqpoint{0.742454in}{0.436309in}}%
\pgfpathlineto{\pgfqpoint{0.749825in}{0.431838in}}%
\pgfpathlineto{\pgfqpoint{0.754883in}{0.428763in}}%
\pgfpathlineto{\pgfqpoint{0.767313in}{0.421385in}}%
\pgfpathlineto{\pgfqpoint{0.769205in}{0.420285in}}%
\pgfpathlineto{\pgfqpoint{0.779742in}{0.414132in}}%
\pgfpathlineto{\pgfqpoint{0.789270in}{0.408733in}}%
\pgfpathlineto{\pgfqpoint{0.792171in}{0.407075in}}%
\pgfpathlineto{\pgfqpoint{0.804601in}{0.400155in}}%
\pgfpathlineto{\pgfqpoint{0.810108in}{0.397180in}}%
\pgfpathlineto{\pgfqpoint{0.817030in}{0.393399in}}%
\pgfpathlineto{\pgfqpoint{0.829459in}{0.386830in}}%
\pgfpathlineto{\pgfqpoint{0.831802in}{0.385627in}}%
\pgfpathlineto{\pgfqpoint{0.841888in}{0.380378in}}%
\pgfpathlineto{\pgfqpoint{0.854318in}{0.374152in}}%
\pgfpathlineto{\pgfqpoint{0.854478in}{0.374075in}}%
\pgfpathlineto{\pgfqpoint{0.866747in}{0.368009in}}%
\pgfpathlineto{\pgfqpoint{0.878322in}{0.362522in}}%
\pgfpathlineto{\pgfqpoint{0.879176in}{0.362108in}}%
\pgfpathlineto{\pgfqpoint{0.891606in}{0.356294in}}%
\pgfpathlineto{\pgfqpoint{0.903503in}{0.350970in}}%
\pgfpathlineto{\pgfqpoint{0.904035in}{0.350725in}}%
\pgfpathlineto{\pgfqpoint{0.916464in}{0.345232in}}%
\pgfpathlineto{\pgfqpoint{0.928894in}{0.339990in}}%
\pgfpathlineto{\pgfqpoint{0.930312in}{0.339417in}}%
\pgfpathlineto{\pgfqpoint{0.941323in}{0.334828in}}%
\pgfpathlineto{\pgfqpoint{0.953752in}{0.329893in}}%
\pgfpathlineto{\pgfqpoint{0.959113in}{0.327865in}}%
\pgfpathlineto{\pgfqpoint{0.966181in}{0.325090in}}%
\pgfpathlineto{\pgfqpoint{0.978611in}{0.320448in}}%
\pgfpathlineto{\pgfqpoint{0.990278in}{0.316312in}}%
\pgfpathlineto{\pgfqpoint{0.991040in}{0.316030in}}%
\pgfpathlineto{\pgfqpoint{1.003469in}{0.311666in}}%
\pgfpathlineto{\pgfqpoint{1.015899in}{0.307532in}}%
\pgfpathlineto{\pgfqpoint{1.024715in}{0.304760in}}%
\pgfpathlineto{\pgfqpoint{1.028328in}{0.303564in}}%
\pgfpathlineto{\pgfqpoint{1.040757in}{0.299682in}}%
\pgfpathlineto{\pgfqpoint{1.053187in}{0.296013in}}%
\pgfpathlineto{\pgfqpoint{1.063267in}{0.293207in}}%
\pgfpathlineto{\pgfqpoint{1.065616in}{0.292512in}}%
\pgfpathlineto{\pgfqpoint{1.078045in}{0.289060in}}%
\pgfpathlineto{\pgfqpoint{1.090474in}{0.285799in}}%
\pgfpathlineto{\pgfqpoint{1.102904in}{0.282716in}}%
\pgfpathlineto{\pgfqpoint{1.107474in}{0.281654in}}%
\pgfpathclose%
\pgfusepath{fill}%
\end{pgfscope}%
\begin{pgfscope}%
\pgfpathrectangle{\pgfqpoint{0.211875in}{0.211875in}}{\pgfqpoint{1.313625in}{1.279725in}}%
\pgfusepath{clip}%
\pgfsetbuttcap%
\pgfsetroundjoin%
\definecolor{currentfill}{rgb}{0.644838,0.098089,0.355336}%
\pgfsetfillcolor{currentfill}%
\pgfsetlinewidth{0.000000pt}%
\definecolor{currentstroke}{rgb}{0.000000,0.000000,0.000000}%
\pgfsetstrokecolor{currentstroke}%
\pgfsetdash{}{0pt}%
\pgfpathmoveto{\pgfqpoint{1.078045in}{0.361550in}}%
\pgfpathlineto{\pgfqpoint{1.090474in}{0.359706in}}%
\pgfpathlineto{\pgfqpoint{1.102904in}{0.358114in}}%
\pgfpathlineto{\pgfqpoint{1.115333in}{0.356772in}}%
\pgfpathlineto{\pgfqpoint{1.127762in}{0.355678in}}%
\pgfpathlineto{\pgfqpoint{1.140192in}{0.354830in}}%
\pgfpathlineto{\pgfqpoint{1.152621in}{0.354225in}}%
\pgfpathlineto{\pgfqpoint{1.165050in}{0.353862in}}%
\pgfpathlineto{\pgfqpoint{1.177480in}{0.353738in}}%
\pgfpathlineto{\pgfqpoint{1.189909in}{0.353849in}}%
\pgfpathlineto{\pgfqpoint{1.202338in}{0.354193in}}%
\pgfpathlineto{\pgfqpoint{1.214767in}{0.354766in}}%
\pgfpathlineto{\pgfqpoint{1.227197in}{0.355562in}}%
\pgfpathlineto{\pgfqpoint{1.239626in}{0.356578in}}%
\pgfpathlineto{\pgfqpoint{1.252055in}{0.357807in}}%
\pgfpathlineto{\pgfqpoint{1.264485in}{0.359242in}}%
\pgfpathlineto{\pgfqpoint{1.276914in}{0.360873in}}%
\pgfpathlineto{\pgfqpoint{1.288231in}{0.362522in}}%
\pgfpathlineto{\pgfqpoint{1.289343in}{0.362684in}}%
\pgfpathlineto{\pgfqpoint{1.301773in}{0.364592in}}%
\pgfpathlineto{\pgfqpoint{1.314202in}{0.366646in}}%
\pgfpathlineto{\pgfqpoint{1.326631in}{0.368823in}}%
\pgfpathlineto{\pgfqpoint{1.339060in}{0.371102in}}%
\pgfpathlineto{\pgfqpoint{1.351490in}{0.373451in}}%
\pgfpathlineto{\pgfqpoint{1.354859in}{0.374075in}}%
\pgfpathlineto{\pgfqpoint{1.363919in}{0.375756in}}%
\pgfpathlineto{\pgfqpoint{1.376348in}{0.378017in}}%
\pgfpathlineto{\pgfqpoint{1.388778in}{0.380208in}}%
\pgfpathlineto{\pgfqpoint{1.401207in}{0.382265in}}%
\pgfpathlineto{\pgfqpoint{1.413636in}{0.384111in}}%
\pgfpathlineto{\pgfqpoint{1.425903in}{0.385627in}}%
\pgfpathlineto{\pgfqpoint{1.426066in}{0.385648in}}%
\pgfpathlineto{\pgfqpoint{1.438495in}{0.386696in}}%
\pgfpathlineto{\pgfqpoint{1.450924in}{0.387194in}}%
\pgfpathlineto{\pgfqpoint{1.463353in}{0.386968in}}%
\pgfpathlineto{\pgfqpoint{1.475783in}{0.385804in}}%
\pgfpathlineto{\pgfqpoint{1.476776in}{0.385627in}}%
\pgfpathlineto{\pgfqpoint{1.488212in}{0.383255in}}%
\pgfpathlineto{\pgfqpoint{1.500641in}{0.378963in}}%
\pgfpathlineto{\pgfqpoint{1.510108in}{0.374075in}}%
\pgfpathlineto{\pgfqpoint{1.513071in}{0.372214in}}%
\pgfpathlineto{\pgfqpoint{1.524387in}{0.362522in}}%
\pgfpathlineto{\pgfqpoint{1.525500in}{0.361305in}}%
\pgfpathlineto{\pgfqpoint{1.525500in}{0.362522in}}%
\pgfpathlineto{\pgfqpoint{1.525500in}{0.374075in}}%
\pgfpathlineto{\pgfqpoint{1.525500in}{0.385627in}}%
\pgfpathlineto{\pgfqpoint{1.525500in}{0.397180in}}%
\pgfpathlineto{\pgfqpoint{1.525500in}{0.408733in}}%
\pgfpathlineto{\pgfqpoint{1.525500in}{0.420285in}}%
\pgfpathlineto{\pgfqpoint{1.525500in}{0.431838in}}%
\pgfpathlineto{\pgfqpoint{1.525500in}{0.443390in}}%
\pgfpathlineto{\pgfqpoint{1.525500in}{0.454943in}}%
\pgfpathlineto{\pgfqpoint{1.525500in}{0.466495in}}%
\pgfpathlineto{\pgfqpoint{1.525500in}{0.478048in}}%
\pgfpathlineto{\pgfqpoint{1.525500in}{0.489600in}}%
\pgfpathlineto{\pgfqpoint{1.525500in}{0.501153in}}%
\pgfpathlineto{\pgfqpoint{1.525500in}{0.512705in}}%
\pgfpathlineto{\pgfqpoint{1.525500in}{0.524258in}}%
\pgfpathlineto{\pgfqpoint{1.525500in}{0.535811in}}%
\pgfpathlineto{\pgfqpoint{1.525500in}{0.545140in}}%
\pgfpathlineto{\pgfqpoint{1.513071in}{0.541204in}}%
\pgfpathlineto{\pgfqpoint{1.500641in}{0.536710in}}%
\pgfpathlineto{\pgfqpoint{1.498456in}{0.535811in}}%
\pgfpathlineto{\pgfqpoint{1.488212in}{0.531723in}}%
\pgfpathlineto{\pgfqpoint{1.475783in}{0.526390in}}%
\pgfpathlineto{\pgfqpoint{1.471155in}{0.524258in}}%
\pgfpathlineto{\pgfqpoint{1.463353in}{0.520771in}}%
\pgfpathlineto{\pgfqpoint{1.450924in}{0.514968in}}%
\pgfpathlineto{\pgfqpoint{1.446269in}{0.512705in}}%
\pgfpathlineto{\pgfqpoint{1.438495in}{0.509035in}}%
\pgfpathlineto{\pgfqpoint{1.426066in}{0.503058in}}%
\pgfpathlineto{\pgfqpoint{1.422187in}{0.501153in}}%
\pgfpathlineto{\pgfqpoint{1.413636in}{0.497067in}}%
\pgfpathlineto{\pgfqpoint{1.401207in}{0.491143in}}%
\pgfpathlineto{\pgfqpoint{1.397989in}{0.489600in}}%
\pgfpathlineto{\pgfqpoint{1.388778in}{0.485295in}}%
\pgfpathlineto{\pgfqpoint{1.376348in}{0.479596in}}%
\pgfpathlineto{\pgfqpoint{1.372939in}{0.478048in}}%
\pgfpathlineto{\pgfqpoint{1.363919in}{0.474043in}}%
\pgfpathlineto{\pgfqpoint{1.351490in}{0.468698in}}%
\pgfpathlineto{\pgfqpoint{1.346224in}{0.466495in}}%
\pgfpathlineto{\pgfqpoint{1.339060in}{0.463558in}}%
\pgfpathlineto{\pgfqpoint{1.326631in}{0.458655in}}%
\pgfpathlineto{\pgfqpoint{1.316714in}{0.454943in}}%
\pgfpathlineto{\pgfqpoint{1.314202in}{0.454018in}}%
\pgfpathlineto{\pgfqpoint{1.301773in}{0.449622in}}%
\pgfpathlineto{\pgfqpoint{1.289343in}{0.445527in}}%
\pgfpathlineto{\pgfqpoint{1.282420in}{0.443390in}}%
\pgfpathlineto{\pgfqpoint{1.276914in}{0.441712in}}%
\pgfpathlineto{\pgfqpoint{1.264485in}{0.438181in}}%
\pgfpathlineto{\pgfqpoint{1.252055in}{0.434966in}}%
\pgfpathlineto{\pgfqpoint{1.239626in}{0.432067in}}%
\pgfpathlineto{\pgfqpoint{1.238545in}{0.431838in}}%
\pgfpathlineto{\pgfqpoint{1.227197in}{0.429448in}}%
\pgfpathlineto{\pgfqpoint{1.214767in}{0.427147in}}%
\pgfpathlineto{\pgfqpoint{1.202338in}{0.425165in}}%
\pgfpathlineto{\pgfqpoint{1.189909in}{0.423500in}}%
\pgfpathlineto{\pgfqpoint{1.177480in}{0.422150in}}%
\pgfpathlineto{\pgfqpoint{1.165050in}{0.421112in}}%
\pgfpathlineto{\pgfqpoint{1.152621in}{0.420384in}}%
\pgfpathlineto{\pgfqpoint{1.149733in}{0.420285in}}%
\pgfpathlineto{\pgfqpoint{1.140192in}{0.419959in}}%
\pgfpathlineto{\pgfqpoint{1.127762in}{0.419842in}}%
\pgfpathlineto{\pgfqpoint{1.115333in}{0.420034in}}%
\pgfpathlineto{\pgfqpoint{1.109086in}{0.420285in}}%
\pgfpathlineto{\pgfqpoint{1.102904in}{0.420528in}}%
\pgfpathlineto{\pgfqpoint{1.090474in}{0.421318in}}%
\pgfpathlineto{\pgfqpoint{1.078045in}{0.422406in}}%
\pgfpathlineto{\pgfqpoint{1.065616in}{0.423791in}}%
\pgfpathlineto{\pgfqpoint{1.053187in}{0.425472in}}%
\pgfpathlineto{\pgfqpoint{1.040757in}{0.427448in}}%
\pgfpathlineto{\pgfqpoint{1.028328in}{0.429717in}}%
\pgfpathlineto{\pgfqpoint{1.018053in}{0.431838in}}%
\pgfpathlineto{\pgfqpoint{1.015899in}{0.432275in}}%
\pgfpathlineto{\pgfqpoint{1.003469in}{0.435099in}}%
\pgfpathlineto{\pgfqpoint{0.991040in}{0.438212in}}%
\pgfpathlineto{\pgfqpoint{0.978611in}{0.441616in}}%
\pgfpathlineto{\pgfqpoint{0.972640in}{0.443390in}}%
\pgfpathlineto{\pgfqpoint{0.966181in}{0.445291in}}%
\pgfpathlineto{\pgfqpoint{0.953752in}{0.449237in}}%
\pgfpathlineto{\pgfqpoint{0.941323in}{0.453471in}}%
\pgfpathlineto{\pgfqpoint{0.937272in}{0.454943in}}%
\pgfpathlineto{\pgfqpoint{0.928894in}{0.457971in}}%
\pgfpathlineto{\pgfqpoint{0.916464in}{0.462746in}}%
\pgfpathlineto{\pgfqpoint{0.907252in}{0.466495in}}%
\pgfpathlineto{\pgfqpoint{0.904035in}{0.467802in}}%
\pgfpathlineto{\pgfqpoint{0.891606in}{0.473122in}}%
\pgfpathlineto{\pgfqpoint{0.880682in}{0.478048in}}%
\pgfpathlineto{\pgfqpoint{0.879176in}{0.478728in}}%
\pgfpathlineto{\pgfqpoint{0.866747in}{0.484599in}}%
\pgfpathlineto{\pgfqpoint{0.856643in}{0.489600in}}%
\pgfpathlineto{\pgfqpoint{0.854318in}{0.490756in}}%
\pgfpathlineto{\pgfqpoint{0.841888in}{0.497187in}}%
\pgfpathlineto{\pgfqpoint{0.834530in}{0.501153in}}%
\pgfpathlineto{\pgfqpoint{0.829459in}{0.503905in}}%
\pgfpathlineto{\pgfqpoint{0.817030in}{0.510907in}}%
\pgfpathlineto{\pgfqpoint{0.813943in}{0.512705in}}%
\pgfpathlineto{\pgfqpoint{0.804601in}{0.518202in}}%
\pgfpathlineto{\pgfqpoint{0.794656in}{0.524258in}}%
\pgfpathlineto{\pgfqpoint{0.792171in}{0.525790in}}%
\pgfpathlineto{\pgfqpoint{0.779742in}{0.533681in}}%
\pgfpathlineto{\pgfqpoint{0.776479in}{0.535811in}}%
\pgfpathlineto{\pgfqpoint{0.767313in}{0.541884in}}%
\pgfpathlineto{\pgfqpoint{0.759266in}{0.547363in}}%
\pgfpathlineto{\pgfqpoint{0.754883in}{0.550399in}}%
\pgfpathlineto{\pgfqpoint{0.742895in}{0.558916in}}%
\pgfpathlineto{\pgfqpoint{0.742454in}{0.559235in}}%
\pgfpathlineto{\pgfqpoint{0.730025in}{0.568418in}}%
\pgfpathlineto{\pgfqpoint{0.727306in}{0.570468in}}%
\pgfpathlineto{\pgfqpoint{0.717595in}{0.577950in}}%
\pgfpathlineto{\pgfqpoint{0.712407in}{0.582021in}}%
\pgfpathlineto{\pgfqpoint{0.705166in}{0.587840in}}%
\pgfpathlineto{\pgfqpoint{0.698143in}{0.593573in}}%
\pgfpathlineto{\pgfqpoint{0.692737in}{0.598104in}}%
\pgfpathlineto{\pgfqpoint{0.684466in}{0.605126in}}%
\pgfpathlineto{\pgfqpoint{0.680308in}{0.608757in}}%
\pgfpathlineto{\pgfqpoint{0.671332in}{0.616678in}}%
\pgfpathlineto{\pgfqpoint{0.667878in}{0.619820in}}%
\pgfpathlineto{\pgfqpoint{0.658705in}{0.628231in}}%
\pgfpathlineto{\pgfqpoint{0.655449in}{0.631314in}}%
\pgfpathlineto{\pgfqpoint{0.646550in}{0.639784in}}%
\pgfpathlineto{\pgfqpoint{0.643020in}{0.643261in}}%
\pgfpathlineto{\pgfqpoint{0.634840in}{0.651336in}}%
\pgfpathlineto{\pgfqpoint{0.630590in}{0.655686in}}%
\pgfpathlineto{\pgfqpoint{0.623548in}{0.662889in}}%
\pgfpathlineto{\pgfqpoint{0.618161in}{0.668614in}}%
\pgfpathlineto{\pgfqpoint{0.612655in}{0.674441in}}%
\pgfpathlineto{\pgfqpoint{0.605732in}{0.682068in}}%
\pgfpathlineto{\pgfqpoint{0.602142in}{0.685994in}}%
\pgfpathlineto{\pgfqpoint{0.593302in}{0.696075in}}%
\pgfpathlineto{\pgfqpoint{0.591998in}{0.697546in}}%
\pgfpathlineto{\pgfqpoint{0.582181in}{0.709099in}}%
\pgfpathlineto{\pgfqpoint{0.580873in}{0.710707in}}%
\pgfpathlineto{\pgfqpoint{0.572664in}{0.720651in}}%
\pgfpathlineto{\pgfqpoint{0.568444in}{0.726011in}}%
\pgfpathlineto{\pgfqpoint{0.563473in}{0.732204in}}%
\pgfpathlineto{\pgfqpoint{0.556015in}{0.741962in}}%
\pgfpathlineto{\pgfqpoint{0.554610in}{0.743757in}}%
\pgfpathlineto{\pgfqpoint{0.545996in}{0.755309in}}%
\pgfpathlineto{\pgfqpoint{0.543585in}{0.758711in}}%
\pgfpathlineto{\pgfqpoint{0.537642in}{0.766862in}}%
\pgfpathlineto{\pgfqpoint{0.531156in}{0.776249in}}%
\pgfpathlineto{\pgfqpoint{0.529609in}{0.778414in}}%
\pgfpathlineto{\pgfqpoint{0.521778in}{0.789967in}}%
\pgfpathlineto{\pgfqpoint{0.518727in}{0.794729in}}%
\pgfpathlineto{\pgfqpoint{0.514199in}{0.801519in}}%
\pgfpathlineto{\pgfqpoint{0.506930in}{0.813072in}}%
\pgfpathlineto{\pgfqpoint{0.506297in}{0.814138in}}%
\pgfpathlineto{\pgfqpoint{0.499777in}{0.824624in}}%
\pgfpathlineto{\pgfqpoint{0.493868in}{0.834730in}}%
\pgfpathlineto{\pgfqpoint{0.492976in}{0.836177in}}%
\pgfpathlineto{\pgfqpoint{0.486280in}{0.847730in}}%
\pgfpathlineto{\pgfqpoint{0.481439in}{0.856639in}}%
\pgfpathlineto{\pgfqpoint{0.479912in}{0.859282in}}%
\pgfpathlineto{\pgfqpoint{0.473666in}{0.870835in}}%
\pgfpathlineto{\pgfqpoint{0.469009in}{0.880052in}}%
\pgfpathlineto{\pgfqpoint{0.467744in}{0.882387in}}%
\pgfpathlineto{\pgfqpoint{0.461913in}{0.893940in}}%
\pgfpathlineto{\pgfqpoint{0.456580in}{0.905283in}}%
\pgfpathlineto{\pgfqpoint{0.456474in}{0.905492in}}%
\pgfpathlineto{\pgfqpoint{0.451026in}{0.917045in}}%
\pgfpathlineto{\pgfqpoint{0.445975in}{0.928597in}}%
\pgfpathlineto{\pgfqpoint{0.444151in}{0.933129in}}%
\pgfpathlineto{\pgfqpoint{0.441046in}{0.940150in}}%
\pgfpathlineto{\pgfqpoint{0.436350in}{0.951702in}}%
\pgfpathlineto{\pgfqpoint{0.432034in}{0.963255in}}%
\pgfpathlineto{\pgfqpoint{0.431722in}{0.964178in}}%
\pgfpathlineto{\pgfqpoint{0.427703in}{0.974808in}}%
\pgfpathlineto{\pgfqpoint{0.423723in}{0.986360in}}%
\pgfpathlineto{\pgfqpoint{0.420110in}{0.997913in}}%
\pgfpathlineto{\pgfqpoint{0.419292in}{1.000845in}}%
\pgfpathlineto{\pgfqpoint{0.416548in}{1.009465in}}%
\pgfpathlineto{\pgfqpoint{0.413259in}{1.021018in}}%
\pgfpathlineto{\pgfqpoint{0.410324in}{1.032570in}}%
\pgfpathlineto{\pgfqpoint{0.407728in}{1.044123in}}%
\pgfpathlineto{\pgfqpoint{0.406863in}{1.048571in}}%
\pgfpathlineto{\pgfqpoint{0.405235in}{1.055675in}}%
\pgfpathlineto{\pgfqpoint{0.402948in}{1.067228in}}%
\pgfpathlineto{\pgfqpoint{0.400974in}{1.078781in}}%
\pgfpathlineto{\pgfqpoint{0.399285in}{1.090333in}}%
\pgfpathlineto{\pgfqpoint{0.397853in}{1.101886in}}%
\pgfpathlineto{\pgfqpoint{0.396642in}{1.113438in}}%
\pgfpathlineto{\pgfqpoint{0.395615in}{1.124991in}}%
\pgfpathlineto{\pgfqpoint{0.394726in}{1.136543in}}%
\pgfpathlineto{\pgfqpoint{0.394434in}{1.140828in}}%
\pgfpathlineto{\pgfqpoint{0.393807in}{1.148096in}}%
\pgfpathlineto{\pgfqpoint{0.392845in}{1.159648in}}%
\pgfpathlineto{\pgfqpoint{0.391814in}{1.171201in}}%
\pgfpathlineto{\pgfqpoint{0.390610in}{1.182754in}}%
\pgfpathlineto{\pgfqpoint{0.389111in}{1.194306in}}%
\pgfpathlineto{\pgfqpoint{0.387170in}{1.205859in}}%
\pgfpathlineto{\pgfqpoint{0.384614in}{1.217411in}}%
\pgfpathlineto{\pgfqpoint{0.382004in}{1.226444in}}%
\pgfpathlineto{\pgfqpoint{0.380931in}{1.228964in}}%
\pgfpathlineto{\pgfqpoint{0.374585in}{1.240516in}}%
\pgfpathlineto{\pgfqpoint{0.369575in}{1.247501in}}%
\pgfpathlineto{\pgfqpoint{0.363059in}{1.252069in}}%
\pgfpathlineto{\pgfqpoint{0.357146in}{1.255381in}}%
\pgfpathlineto{\pgfqpoint{0.344716in}{1.256018in}}%
\pgfpathlineto{\pgfqpoint{0.334739in}{1.252069in}}%
\pgfpathlineto{\pgfqpoint{0.332287in}{1.251037in}}%
\pgfpathlineto{\pgfqpoint{0.320126in}{1.240516in}}%
\pgfpathlineto{\pgfqpoint{0.319858in}{1.240273in}}%
\pgfpathlineto{\pgfqpoint{0.311706in}{1.228964in}}%
\pgfpathlineto{\pgfqpoint{0.307429in}{1.222811in}}%
\pgfpathlineto{\pgfqpoint{0.304651in}{1.217411in}}%
\pgfpathlineto{\pgfqpoint{0.298943in}{1.205859in}}%
\pgfpathlineto{\pgfqpoint{0.294999in}{1.197741in}}%
\pgfpathlineto{\pgfqpoint{0.294999in}{1.194306in}}%
\pgfpathlineto{\pgfqpoint{0.294999in}{1.182754in}}%
\pgfpathlineto{\pgfqpoint{0.294999in}{1.171201in}}%
\pgfpathlineto{\pgfqpoint{0.294999in}{1.159648in}}%
\pgfpathlineto{\pgfqpoint{0.294999in}{1.148096in}}%
\pgfpathlineto{\pgfqpoint{0.294999in}{1.136543in}}%
\pgfpathlineto{\pgfqpoint{0.294999in}{1.124991in}}%
\pgfpathlineto{\pgfqpoint{0.294999in}{1.113438in}}%
\pgfpathlineto{\pgfqpoint{0.294999in}{1.101886in}}%
\pgfpathlineto{\pgfqpoint{0.294999in}{1.090333in}}%
\pgfpathlineto{\pgfqpoint{0.294999in}{1.078781in}}%
\pgfpathlineto{\pgfqpoint{0.294999in}{1.067228in}}%
\pgfpathlineto{\pgfqpoint{0.294999in}{1.055675in}}%
\pgfpathlineto{\pgfqpoint{0.294999in}{1.044123in}}%
\pgfpathlineto{\pgfqpoint{0.294999in}{1.032570in}}%
\pgfpathlineto{\pgfqpoint{0.294999in}{1.021018in}}%
\pgfpathlineto{\pgfqpoint{0.294999in}{1.009465in}}%
\pgfpathlineto{\pgfqpoint{0.294999in}{0.997913in}}%
\pgfpathlineto{\pgfqpoint{0.294999in}{0.986360in}}%
\pgfpathlineto{\pgfqpoint{0.294999in}{0.974808in}}%
\pgfpathlineto{\pgfqpoint{0.294999in}{0.963255in}}%
\pgfpathlineto{\pgfqpoint{0.294999in}{0.951702in}}%
\pgfpathlineto{\pgfqpoint{0.294999in}{0.940150in}}%
\pgfpathlineto{\pgfqpoint{0.294999in}{0.928597in}}%
\pgfpathlineto{\pgfqpoint{0.294999in}{0.917045in}}%
\pgfpathlineto{\pgfqpoint{0.294999in}{0.905492in}}%
\pgfpathlineto{\pgfqpoint{0.294999in}{0.893940in}}%
\pgfpathlineto{\pgfqpoint{0.294999in}{0.882387in}}%
\pgfpathlineto{\pgfqpoint{0.294999in}{0.870835in}}%
\pgfpathlineto{\pgfqpoint{0.294999in}{0.859282in}}%
\pgfpathlineto{\pgfqpoint{0.294999in}{0.847730in}}%
\pgfpathlineto{\pgfqpoint{0.294999in}{0.836177in}}%
\pgfpathlineto{\pgfqpoint{0.294999in}{0.824646in}}%
\pgfpathlineto{\pgfqpoint{0.307429in}{0.827401in}}%
\pgfpathlineto{\pgfqpoint{0.319858in}{0.828271in}}%
\pgfpathlineto{\pgfqpoint{0.332287in}{0.827318in}}%
\pgfpathlineto{\pgfqpoint{0.344603in}{0.824624in}}%
\pgfpathlineto{\pgfqpoint{0.344716in}{0.824601in}}%
\pgfpathlineto{\pgfqpoint{0.357146in}{0.820345in}}%
\pgfpathlineto{\pgfqpoint{0.369575in}{0.814513in}}%
\pgfpathlineto{\pgfqpoint{0.371995in}{0.813072in}}%
\pgfpathlineto{\pgfqpoint{0.382004in}{0.807392in}}%
\pgfpathlineto{\pgfqpoint{0.390617in}{0.801519in}}%
\pgfpathlineto{\pgfqpoint{0.394434in}{0.799038in}}%
\pgfpathlineto{\pgfqpoint{0.406436in}{0.789967in}}%
\pgfpathlineto{\pgfqpoint{0.406863in}{0.789659in}}%
\pgfpathlineto{\pgfqpoint{0.419292in}{0.779480in}}%
\pgfpathlineto{\pgfqpoint{0.420460in}{0.778414in}}%
\pgfpathlineto{\pgfqpoint{0.431722in}{0.768616in}}%
\pgfpathlineto{\pgfqpoint{0.433559in}{0.766862in}}%
\pgfpathlineto{\pgfqpoint{0.444151in}{0.757213in}}%
\pgfpathlineto{\pgfqpoint{0.446079in}{0.755309in}}%
\pgfpathlineto{\pgfqpoint{0.456580in}{0.745411in}}%
\pgfpathlineto{\pgfqpoint{0.458216in}{0.743757in}}%
\pgfpathlineto{\pgfqpoint{0.469009in}{0.733331in}}%
\pgfpathlineto{\pgfqpoint{0.470107in}{0.732204in}}%
\pgfpathlineto{\pgfqpoint{0.481439in}{0.721079in}}%
\pgfpathlineto{\pgfqpoint{0.481851in}{0.720651in}}%
\pgfpathlineto{\pgfqpoint{0.493503in}{0.709099in}}%
\pgfpathlineto{\pgfqpoint{0.493868in}{0.708752in}}%
\pgfpathlineto{\pgfqpoint{0.505125in}{0.697546in}}%
\pgfpathlineto{\pgfqpoint{0.506297in}{0.696427in}}%
\pgfpathlineto{\pgfqpoint{0.516798in}{0.685994in}}%
\pgfpathlineto{\pgfqpoint{0.518727in}{0.684152in}}%
\pgfpathlineto{\pgfqpoint{0.528563in}{0.674441in}}%
\pgfpathlineto{\pgfqpoint{0.531156in}{0.671979in}}%
\pgfpathlineto{\pgfqpoint{0.540459in}{0.662889in}}%
\pgfpathlineto{\pgfqpoint{0.543585in}{0.659946in}}%
\pgfpathlineto{\pgfqpoint{0.552519in}{0.651336in}}%
\pgfpathlineto{\pgfqpoint{0.556015in}{0.648086in}}%
\pgfpathlineto{\pgfqpoint{0.564774in}{0.639784in}}%
\pgfpathlineto{\pgfqpoint{0.568444in}{0.636423in}}%
\pgfpathlineto{\pgfqpoint{0.577256in}{0.628231in}}%
\pgfpathlineto{\pgfqpoint{0.580873in}{0.624977in}}%
\pgfpathlineto{\pgfqpoint{0.589994in}{0.616678in}}%
\pgfpathlineto{\pgfqpoint{0.593302in}{0.613761in}}%
\pgfpathlineto{\pgfqpoint{0.603019in}{0.605126in}}%
\pgfpathlineto{\pgfqpoint{0.605732in}{0.602784in}}%
\pgfpathlineto{\pgfqpoint{0.616359in}{0.593573in}}%
\pgfpathlineto{\pgfqpoint{0.618161in}{0.592054in}}%
\pgfpathlineto{\pgfqpoint{0.630047in}{0.582021in}}%
\pgfpathlineto{\pgfqpoint{0.630590in}{0.581573in}}%
\pgfpathlineto{\pgfqpoint{0.643020in}{0.571351in}}%
\pgfpathlineto{\pgfqpoint{0.644097in}{0.570468in}}%
\pgfpathlineto{\pgfqpoint{0.655449in}{0.561382in}}%
\pgfpathlineto{\pgfqpoint{0.658553in}{0.558916in}}%
\pgfpathlineto{\pgfqpoint{0.667878in}{0.551663in}}%
\pgfpathlineto{\pgfqpoint{0.673462in}{0.547363in}}%
\pgfpathlineto{\pgfqpoint{0.680308in}{0.542193in}}%
\pgfpathlineto{\pgfqpoint{0.688862in}{0.535811in}}%
\pgfpathlineto{\pgfqpoint{0.692737in}{0.532970in}}%
\pgfpathlineto{\pgfqpoint{0.704796in}{0.524258in}}%
\pgfpathlineto{\pgfqpoint{0.705166in}{0.523995in}}%
\pgfpathlineto{\pgfqpoint{0.717595in}{0.515262in}}%
\pgfpathlineto{\pgfqpoint{0.721304in}{0.512705in}}%
\pgfpathlineto{\pgfqpoint{0.730025in}{0.506768in}}%
\pgfpathlineto{\pgfqpoint{0.738452in}{0.501153in}}%
\pgfpathlineto{\pgfqpoint{0.742454in}{0.498514in}}%
\pgfpathlineto{\pgfqpoint{0.754883in}{0.490498in}}%
\pgfpathlineto{\pgfqpoint{0.756305in}{0.489600in}}%
\pgfpathlineto{\pgfqpoint{0.767313in}{0.482704in}}%
\pgfpathlineto{\pgfqpoint{0.774956in}{0.478048in}}%
\pgfpathlineto{\pgfqpoint{0.779742in}{0.475148in}}%
\pgfpathlineto{\pgfqpoint{0.792171in}{0.467824in}}%
\pgfpathlineto{\pgfqpoint{0.794488in}{0.466495in}}%
\pgfpathlineto{\pgfqpoint{0.804601in}{0.460712in}}%
\pgfpathlineto{\pgfqpoint{0.815041in}{0.454943in}}%
\pgfpathlineto{\pgfqpoint{0.817030in}{0.453844in}}%
\pgfpathlineto{\pgfqpoint{0.829459in}{0.447182in}}%
\pgfpathlineto{\pgfqpoint{0.836799in}{0.443390in}}%
\pgfpathlineto{\pgfqpoint{0.841888in}{0.440753in}}%
\pgfpathlineto{\pgfqpoint{0.854318in}{0.434546in}}%
\pgfpathlineto{\pgfqpoint{0.859957in}{0.431838in}}%
\pgfpathlineto{\pgfqpoint{0.866747in}{0.428557in}}%
\pgfpathlineto{\pgfqpoint{0.879176in}{0.422797in}}%
\pgfpathlineto{\pgfqpoint{0.884836in}{0.420285in}}%
\pgfpathlineto{\pgfqpoint{0.891606in}{0.417253in}}%
\pgfpathlineto{\pgfqpoint{0.904035in}{0.411935in}}%
\pgfpathlineto{\pgfqpoint{0.911895in}{0.408733in}}%
\pgfpathlineto{\pgfqpoint{0.916464in}{0.406846in}}%
\pgfpathlineto{\pgfqpoint{0.928894in}{0.401966in}}%
\pgfpathlineto{\pgfqpoint{0.941323in}{0.397348in}}%
\pgfpathlineto{\pgfqpoint{0.941801in}{0.397180in}}%
\pgfpathlineto{\pgfqpoint{0.953752in}{0.392905in}}%
\pgfpathlineto{\pgfqpoint{0.966181in}{0.388721in}}%
\pgfpathlineto{\pgfqpoint{0.975978in}{0.385627in}}%
\pgfpathlineto{\pgfqpoint{0.978611in}{0.384778in}}%
\pgfpathlineto{\pgfqpoint{0.991040in}{0.381026in}}%
\pgfpathlineto{\pgfqpoint{1.003469in}{0.377535in}}%
\pgfpathlineto{\pgfqpoint{1.015899in}{0.374301in}}%
\pgfpathlineto{\pgfqpoint{1.016844in}{0.374075in}}%
\pgfpathlineto{\pgfqpoint{1.028328in}{0.371250in}}%
\pgfpathlineto{\pgfqpoint{1.040757in}{0.368452in}}%
\pgfpathlineto{\pgfqpoint{1.053187in}{0.365908in}}%
\pgfpathlineto{\pgfqpoint{1.065616in}{0.363618in}}%
\pgfpathlineto{\pgfqpoint{1.072316in}{0.362522in}}%
\pgfpathclose%
\pgfusepath{fill}%
\end{pgfscope}%
\begin{pgfscope}%
\pgfpathrectangle{\pgfqpoint{0.211875in}{0.211875in}}{\pgfqpoint{1.313625in}{1.279725in}}%
\pgfusepath{clip}%
\pgfsetbuttcap%
\pgfsetroundjoin%
\definecolor{currentfill}{rgb}{0.796501,0.105066,0.310630}%
\pgfsetfillcolor{currentfill}%
\pgfsetlinewidth{0.000000pt}%
\definecolor{currentstroke}{rgb}{0.000000,0.000000,0.000000}%
\pgfsetstrokecolor{currentstroke}%
\pgfsetdash{}{0pt}%
\pgfpathmoveto{\pgfqpoint{1.115333in}{0.420034in}}%
\pgfpathlineto{\pgfqpoint{1.127762in}{0.419842in}}%
\pgfpathlineto{\pgfqpoint{1.140192in}{0.419959in}}%
\pgfpathlineto{\pgfqpoint{1.149733in}{0.420285in}}%
\pgfpathlineto{\pgfqpoint{1.152621in}{0.420384in}}%
\pgfpathlineto{\pgfqpoint{1.165050in}{0.421112in}}%
\pgfpathlineto{\pgfqpoint{1.177480in}{0.422150in}}%
\pgfpathlineto{\pgfqpoint{1.189909in}{0.423500in}}%
\pgfpathlineto{\pgfqpoint{1.202338in}{0.425165in}}%
\pgfpathlineto{\pgfqpoint{1.214767in}{0.427147in}}%
\pgfpathlineto{\pgfqpoint{1.227197in}{0.429448in}}%
\pgfpathlineto{\pgfqpoint{1.238545in}{0.431838in}}%
\pgfpathlineto{\pgfqpoint{1.239626in}{0.432067in}}%
\pgfpathlineto{\pgfqpoint{1.252055in}{0.434966in}}%
\pgfpathlineto{\pgfqpoint{1.264485in}{0.438181in}}%
\pgfpathlineto{\pgfqpoint{1.276914in}{0.441712in}}%
\pgfpathlineto{\pgfqpoint{1.282420in}{0.443390in}}%
\pgfpathlineto{\pgfqpoint{1.289343in}{0.445527in}}%
\pgfpathlineto{\pgfqpoint{1.301773in}{0.449622in}}%
\pgfpathlineto{\pgfqpoint{1.314202in}{0.454018in}}%
\pgfpathlineto{\pgfqpoint{1.316714in}{0.454943in}}%
\pgfpathlineto{\pgfqpoint{1.326631in}{0.458655in}}%
\pgfpathlineto{\pgfqpoint{1.339060in}{0.463558in}}%
\pgfpathlineto{\pgfqpoint{1.346224in}{0.466495in}}%
\pgfpathlineto{\pgfqpoint{1.351490in}{0.468698in}}%
\pgfpathlineto{\pgfqpoint{1.363919in}{0.474043in}}%
\pgfpathlineto{\pgfqpoint{1.372939in}{0.478048in}}%
\pgfpathlineto{\pgfqpoint{1.376348in}{0.479596in}}%
\pgfpathlineto{\pgfqpoint{1.388778in}{0.485295in}}%
\pgfpathlineto{\pgfqpoint{1.397989in}{0.489600in}}%
\pgfpathlineto{\pgfqpoint{1.401207in}{0.491143in}}%
\pgfpathlineto{\pgfqpoint{1.413636in}{0.497067in}}%
\pgfpathlineto{\pgfqpoint{1.422187in}{0.501153in}}%
\pgfpathlineto{\pgfqpoint{1.426066in}{0.503058in}}%
\pgfpathlineto{\pgfqpoint{1.438495in}{0.509035in}}%
\pgfpathlineto{\pgfqpoint{1.446269in}{0.512705in}}%
\pgfpathlineto{\pgfqpoint{1.450924in}{0.514968in}}%
\pgfpathlineto{\pgfqpoint{1.463353in}{0.520771in}}%
\pgfpathlineto{\pgfqpoint{1.471155in}{0.524258in}}%
\pgfpathlineto{\pgfqpoint{1.475783in}{0.526390in}}%
\pgfpathlineto{\pgfqpoint{1.488212in}{0.531723in}}%
\pgfpathlineto{\pgfqpoint{1.498456in}{0.535811in}}%
\pgfpathlineto{\pgfqpoint{1.500641in}{0.536710in}}%
\pgfpathlineto{\pgfqpoint{1.513071in}{0.541204in}}%
\pgfpathlineto{\pgfqpoint{1.525500in}{0.545140in}}%
\pgfpathlineto{\pgfqpoint{1.525500in}{0.547363in}}%
\pgfpathlineto{\pgfqpoint{1.525500in}{0.558916in}}%
\pgfpathlineto{\pgfqpoint{1.525500in}{0.570468in}}%
\pgfpathlineto{\pgfqpoint{1.525500in}{0.582021in}}%
\pgfpathlineto{\pgfqpoint{1.525500in}{0.593573in}}%
\pgfpathlineto{\pgfqpoint{1.525500in}{0.605126in}}%
\pgfpathlineto{\pgfqpoint{1.525500in}{0.616678in}}%
\pgfpathlineto{\pgfqpoint{1.525500in}{0.628231in}}%
\pgfpathlineto{\pgfqpoint{1.525500in}{0.639784in}}%
\pgfpathlineto{\pgfqpoint{1.525500in}{0.651336in}}%
\pgfpathlineto{\pgfqpoint{1.525500in}{0.662889in}}%
\pgfpathlineto{\pgfqpoint{1.525500in}{0.674441in}}%
\pgfpathlineto{\pgfqpoint{1.525500in}{0.681583in}}%
\pgfpathlineto{\pgfqpoint{1.515402in}{0.674441in}}%
\pgfpathlineto{\pgfqpoint{1.513071in}{0.672852in}}%
\pgfpathlineto{\pgfqpoint{1.500641in}{0.663691in}}%
\pgfpathlineto{\pgfqpoint{1.499624in}{0.662889in}}%
\pgfpathlineto{\pgfqpoint{1.488212in}{0.654227in}}%
\pgfpathlineto{\pgfqpoint{1.484598in}{0.651336in}}%
\pgfpathlineto{\pgfqpoint{1.475783in}{0.644554in}}%
\pgfpathlineto{\pgfqpoint{1.469841in}{0.639784in}}%
\pgfpathlineto{\pgfqpoint{1.463353in}{0.634775in}}%
\pgfpathlineto{\pgfqpoint{1.455158in}{0.628231in}}%
\pgfpathlineto{\pgfqpoint{1.450924in}{0.624980in}}%
\pgfpathlineto{\pgfqpoint{1.440382in}{0.616678in}}%
\pgfpathlineto{\pgfqpoint{1.438495in}{0.615249in}}%
\pgfpathlineto{\pgfqpoint{1.426066in}{0.605655in}}%
\pgfpathlineto{\pgfqpoint{1.425391in}{0.605126in}}%
\pgfpathlineto{\pgfqpoint{1.413636in}{0.596262in}}%
\pgfpathlineto{\pgfqpoint{1.410084in}{0.593573in}}%
\pgfpathlineto{\pgfqpoint{1.401207in}{0.587104in}}%
\pgfpathlineto{\pgfqpoint{1.394206in}{0.582021in}}%
\pgfpathlineto{\pgfqpoint{1.388778in}{0.578220in}}%
\pgfpathlineto{\pgfqpoint{1.377582in}{0.570468in}}%
\pgfpathlineto{\pgfqpoint{1.376348in}{0.569644in}}%
\pgfpathlineto{\pgfqpoint{1.363919in}{0.561415in}}%
\pgfpathlineto{\pgfqpoint{1.360064in}{0.558916in}}%
\pgfpathlineto{\pgfqpoint{1.351490in}{0.553540in}}%
\pgfpathlineto{\pgfqpoint{1.341323in}{0.547363in}}%
\pgfpathlineto{\pgfqpoint{1.339060in}{0.546031in}}%
\pgfpathlineto{\pgfqpoint{1.326631in}{0.538914in}}%
\pgfpathlineto{\pgfqpoint{1.320987in}{0.535811in}}%
\pgfpathlineto{\pgfqpoint{1.314202in}{0.532187in}}%
\pgfpathlineto{\pgfqpoint{1.301773in}{0.525857in}}%
\pgfpathlineto{\pgfqpoint{1.298483in}{0.524258in}}%
\pgfpathlineto{\pgfqpoint{1.289343in}{0.519931in}}%
\pgfpathlineto{\pgfqpoint{1.276914in}{0.514408in}}%
\pgfpathlineto{\pgfqpoint{1.272845in}{0.512705in}}%
\pgfpathlineto{\pgfqpoint{1.264485in}{0.509289in}}%
\pgfpathlineto{\pgfqpoint{1.252055in}{0.504574in}}%
\pgfpathlineto{\pgfqpoint{1.242241in}{0.501153in}}%
\pgfpathlineto{\pgfqpoint{1.239626in}{0.500260in}}%
\pgfpathlineto{\pgfqpoint{1.227197in}{0.496345in}}%
\pgfpathlineto{\pgfqpoint{1.214767in}{0.492826in}}%
\pgfpathlineto{\pgfqpoint{1.202338in}{0.489700in}}%
\pgfpathlineto{\pgfqpoint{1.201892in}{0.489600in}}%
\pgfpathlineto{\pgfqpoint{1.189909in}{0.486960in}}%
\pgfpathlineto{\pgfqpoint{1.177480in}{0.484605in}}%
\pgfpathlineto{\pgfqpoint{1.165050in}{0.482631in}}%
\pgfpathlineto{\pgfqpoint{1.152621in}{0.481032in}}%
\pgfpathlineto{\pgfqpoint{1.140192in}{0.479805in}}%
\pgfpathlineto{\pgfqpoint{1.127762in}{0.478944in}}%
\pgfpathlineto{\pgfqpoint{1.115333in}{0.478445in}}%
\pgfpathlineto{\pgfqpoint{1.102904in}{0.478304in}}%
\pgfpathlineto{\pgfqpoint{1.090474in}{0.478517in}}%
\pgfpathlineto{\pgfqpoint{1.078045in}{0.479079in}}%
\pgfpathlineto{\pgfqpoint{1.065616in}{0.479987in}}%
\pgfpathlineto{\pgfqpoint{1.053187in}{0.481238in}}%
\pgfpathlineto{\pgfqpoint{1.040757in}{0.482828in}}%
\pgfpathlineto{\pgfqpoint{1.028328in}{0.484754in}}%
\pgfpathlineto{\pgfqpoint{1.015899in}{0.487014in}}%
\pgfpathlineto{\pgfqpoint{1.003496in}{0.489600in}}%
\pgfpathlineto{\pgfqpoint{1.003469in}{0.489606in}}%
\pgfpathlineto{\pgfqpoint{0.991040in}{0.492527in}}%
\pgfpathlineto{\pgfqpoint{0.978611in}{0.495774in}}%
\pgfpathlineto{\pgfqpoint{0.966181in}{0.499347in}}%
\pgfpathlineto{\pgfqpoint{0.960421in}{0.501153in}}%
\pgfpathlineto{\pgfqpoint{0.953752in}{0.503247in}}%
\pgfpathlineto{\pgfqpoint{0.941323in}{0.507473in}}%
\pgfpathlineto{\pgfqpoint{0.928894in}{0.512020in}}%
\pgfpathlineto{\pgfqpoint{0.927141in}{0.512705in}}%
\pgfpathlineto{\pgfqpoint{0.916464in}{0.516902in}}%
\pgfpathlineto{\pgfqpoint{0.904035in}{0.522106in}}%
\pgfpathlineto{\pgfqpoint{0.899181in}{0.524258in}}%
\pgfpathlineto{\pgfqpoint{0.891606in}{0.527646in}}%
\pgfpathlineto{\pgfqpoint{0.879176in}{0.533516in}}%
\pgfpathlineto{\pgfqpoint{0.874555in}{0.535811in}}%
\pgfpathlineto{\pgfqpoint{0.866747in}{0.539731in}}%
\pgfpathlineto{\pgfqpoint{0.854318in}{0.546281in}}%
\pgfpathlineto{\pgfqpoint{0.852351in}{0.547363in}}%
\pgfpathlineto{\pgfqpoint{0.841888in}{0.553198in}}%
\pgfpathlineto{\pgfqpoint{0.832069in}{0.558916in}}%
\pgfpathlineto{\pgfqpoint{0.829459in}{0.560460in}}%
\pgfpathlineto{\pgfqpoint{0.817030in}{0.568101in}}%
\pgfpathlineto{\pgfqpoint{0.813314in}{0.570468in}}%
\pgfpathlineto{\pgfqpoint{0.804601in}{0.576128in}}%
\pgfpathlineto{\pgfqpoint{0.795835in}{0.582021in}}%
\pgfpathlineto{\pgfqpoint{0.792171in}{0.584537in}}%
\pgfpathlineto{\pgfqpoint{0.779742in}{0.593350in}}%
\pgfpathlineto{\pgfqpoint{0.779435in}{0.593573in}}%
\pgfpathlineto{\pgfqpoint{0.767313in}{0.602619in}}%
\pgfpathlineto{\pgfqpoint{0.764041in}{0.605126in}}%
\pgfpathlineto{\pgfqpoint{0.754883in}{0.612333in}}%
\pgfpathlineto{\pgfqpoint{0.749494in}{0.616678in}}%
\pgfpathlineto{\pgfqpoint{0.742454in}{0.622520in}}%
\pgfpathlineto{\pgfqpoint{0.735719in}{0.628231in}}%
\pgfpathlineto{\pgfqpoint{0.730025in}{0.633212in}}%
\pgfpathlineto{\pgfqpoint{0.722653in}{0.639784in}}%
\pgfpathlineto{\pgfqpoint{0.717595in}{0.644445in}}%
\pgfpathlineto{\pgfqpoint{0.710241in}{0.651336in}}%
\pgfpathlineto{\pgfqpoint{0.705166in}{0.656263in}}%
\pgfpathlineto{\pgfqpoint{0.698435in}{0.662889in}}%
\pgfpathlineto{\pgfqpoint{0.692737in}{0.668713in}}%
\pgfpathlineto{\pgfqpoint{0.687195in}{0.674441in}}%
\pgfpathlineto{\pgfqpoint{0.680308in}{0.681852in}}%
\pgfpathlineto{\pgfqpoint{0.676489in}{0.685994in}}%
\pgfpathlineto{\pgfqpoint{0.667878in}{0.695739in}}%
\pgfpathlineto{\pgfqpoint{0.666290in}{0.697546in}}%
\pgfpathlineto{\pgfqpoint{0.656569in}{0.709099in}}%
\pgfpathlineto{\pgfqpoint{0.655449in}{0.710492in}}%
\pgfpathlineto{\pgfqpoint{0.647297in}{0.720651in}}%
\pgfpathlineto{\pgfqpoint{0.643020in}{0.726249in}}%
\pgfpathlineto{\pgfqpoint{0.638464in}{0.732204in}}%
\pgfpathlineto{\pgfqpoint{0.630590in}{0.743039in}}%
\pgfpathlineto{\pgfqpoint{0.630066in}{0.743757in}}%
\pgfpathlineto{\pgfqpoint{0.622049in}{0.755309in}}%
\pgfpathlineto{\pgfqpoint{0.618161in}{0.761230in}}%
\pgfpathlineto{\pgfqpoint{0.614433in}{0.766862in}}%
\pgfpathlineto{\pgfqpoint{0.607210in}{0.778414in}}%
\pgfpathlineto{\pgfqpoint{0.605732in}{0.780921in}}%
\pgfpathlineto{\pgfqpoint{0.600330in}{0.789967in}}%
\pgfpathlineto{\pgfqpoint{0.593851in}{0.801519in}}%
\pgfpathlineto{\pgfqpoint{0.593302in}{0.802562in}}%
\pgfpathlineto{\pgfqpoint{0.587677in}{0.813072in}}%
\pgfpathlineto{\pgfqpoint{0.581899in}{0.824624in}}%
\pgfpathlineto{\pgfqpoint{0.580873in}{0.826824in}}%
\pgfpathlineto{\pgfqpoint{0.576417in}{0.836177in}}%
\pgfpathlineto{\pgfqpoint{0.571303in}{0.847730in}}%
\pgfpathlineto{\pgfqpoint{0.568444in}{0.854712in}}%
\pgfpathlineto{\pgfqpoint{0.566521in}{0.859282in}}%
\pgfpathlineto{\pgfqpoint{0.562044in}{0.870835in}}%
\pgfpathlineto{\pgfqpoint{0.557935in}{0.882387in}}%
\pgfpathlineto{\pgfqpoint{0.556015in}{0.888312in}}%
\pgfpathlineto{\pgfqpoint{0.554130in}{0.893940in}}%
\pgfpathlineto{\pgfqpoint{0.550626in}{0.905492in}}%
\pgfpathlineto{\pgfqpoint{0.547473in}{0.917045in}}%
\pgfpathlineto{\pgfqpoint{0.544662in}{0.928597in}}%
\pgfpathlineto{\pgfqpoint{0.543585in}{0.933636in}}%
\pgfpathlineto{\pgfqpoint{0.542136in}{0.940150in}}%
\pgfpathlineto{\pgfqpoint{0.539908in}{0.951702in}}%
\pgfpathlineto{\pgfqpoint{0.538009in}{0.963255in}}%
\pgfpathlineto{\pgfqpoint{0.536431in}{0.974808in}}%
\pgfpathlineto{\pgfqpoint{0.535165in}{0.986360in}}%
\pgfpathlineto{\pgfqpoint{0.534205in}{0.997913in}}%
\pgfpathlineto{\pgfqpoint{0.533542in}{1.009465in}}%
\pgfpathlineto{\pgfqpoint{0.533168in}{1.021018in}}%
\pgfpathlineto{\pgfqpoint{0.533075in}{1.032570in}}%
\pgfpathlineto{\pgfqpoint{0.533254in}{1.044123in}}%
\pgfpathlineto{\pgfqpoint{0.533695in}{1.055675in}}%
\pgfpathlineto{\pgfqpoint{0.534387in}{1.067228in}}%
\pgfpathlineto{\pgfqpoint{0.535319in}{1.078781in}}%
\pgfpathlineto{\pgfqpoint{0.536479in}{1.090333in}}%
\pgfpathlineto{\pgfqpoint{0.537852in}{1.101886in}}%
\pgfpathlineto{\pgfqpoint{0.539423in}{1.113438in}}%
\pgfpathlineto{\pgfqpoint{0.541176in}{1.124991in}}%
\pgfpathlineto{\pgfqpoint{0.543092in}{1.136543in}}%
\pgfpathlineto{\pgfqpoint{0.543585in}{1.139350in}}%
\pgfpathlineto{\pgfqpoint{0.545091in}{1.148096in}}%
\pgfpathlineto{\pgfqpoint{0.547186in}{1.159648in}}%
\pgfpathlineto{\pgfqpoint{0.549372in}{1.171201in}}%
\pgfpathlineto{\pgfqpoint{0.551623in}{1.182754in}}%
\pgfpathlineto{\pgfqpoint{0.553910in}{1.194306in}}%
\pgfpathlineto{\pgfqpoint{0.556015in}{1.204925in}}%
\pgfpathlineto{\pgfqpoint{0.556197in}{1.205859in}}%
\pgfpathlineto{\pgfqpoint{0.558383in}{1.217411in}}%
\pgfpathlineto{\pgfqpoint{0.560507in}{1.228964in}}%
\pgfpathlineto{\pgfqpoint{0.562532in}{1.240516in}}%
\pgfpathlineto{\pgfqpoint{0.564418in}{1.252069in}}%
\pgfpathlineto{\pgfqpoint{0.566122in}{1.263621in}}%
\pgfpathlineto{\pgfqpoint{0.567597in}{1.275174in}}%
\pgfpathlineto{\pgfqpoint{0.568444in}{1.283405in}}%
\pgfpathlineto{\pgfqpoint{0.568783in}{1.286727in}}%
\pgfpathlineto{\pgfqpoint{0.569621in}{1.298279in}}%
\pgfpathlineto{\pgfqpoint{0.570085in}{1.309832in}}%
\pgfpathlineto{\pgfqpoint{0.570116in}{1.321384in}}%
\pgfpathlineto{\pgfqpoint{0.569652in}{1.332937in}}%
\pgfpathlineto{\pgfqpoint{0.568628in}{1.344489in}}%
\pgfpathlineto{\pgfqpoint{0.568444in}{1.345805in}}%
\pgfpathlineto{\pgfqpoint{0.566922in}{1.356042in}}%
\pgfpathlineto{\pgfqpoint{0.564479in}{1.367594in}}%
\pgfpathlineto{\pgfqpoint{0.561223in}{1.379147in}}%
\pgfpathlineto{\pgfqpoint{0.557064in}{1.390699in}}%
\pgfpathlineto{\pgfqpoint{0.556015in}{1.393104in}}%
\pgfpathlineto{\pgfqpoint{0.551743in}{1.402252in}}%
\pgfpathlineto{\pgfqpoint{0.545238in}{1.413805in}}%
\pgfpathlineto{\pgfqpoint{0.543585in}{1.413805in}}%
\pgfpathlineto{\pgfqpoint{0.531156in}{1.413805in}}%
\pgfpathlineto{\pgfqpoint{0.518727in}{1.413805in}}%
\pgfpathlineto{\pgfqpoint{0.506297in}{1.413805in}}%
\pgfpathlineto{\pgfqpoint{0.493868in}{1.413805in}}%
\pgfpathlineto{\pgfqpoint{0.481439in}{1.413805in}}%
\pgfpathlineto{\pgfqpoint{0.469009in}{1.413805in}}%
\pgfpathlineto{\pgfqpoint{0.456580in}{1.413805in}}%
\pgfpathlineto{\pgfqpoint{0.444151in}{1.413805in}}%
\pgfpathlineto{\pgfqpoint{0.431722in}{1.413805in}}%
\pgfpathlineto{\pgfqpoint{0.419292in}{1.413805in}}%
\pgfpathlineto{\pgfqpoint{0.406863in}{1.413805in}}%
\pgfpathlineto{\pgfqpoint{0.394434in}{1.413805in}}%
\pgfpathlineto{\pgfqpoint{0.382004in}{1.413805in}}%
\pgfpathlineto{\pgfqpoint{0.369575in}{1.413805in}}%
\pgfpathlineto{\pgfqpoint{0.357146in}{1.413805in}}%
\pgfpathlineto{\pgfqpoint{0.344716in}{1.413805in}}%
\pgfpathlineto{\pgfqpoint{0.332287in}{1.413805in}}%
\pgfpathlineto{\pgfqpoint{0.319858in}{1.413805in}}%
\pgfpathlineto{\pgfqpoint{0.307429in}{1.413805in}}%
\pgfpathlineto{\pgfqpoint{0.294999in}{1.413805in}}%
\pgfpathlineto{\pgfqpoint{0.294999in}{1.402252in}}%
\pgfpathlineto{\pgfqpoint{0.294999in}{1.390699in}}%
\pgfpathlineto{\pgfqpoint{0.294999in}{1.379147in}}%
\pgfpathlineto{\pgfqpoint{0.294999in}{1.367594in}}%
\pgfpathlineto{\pgfqpoint{0.294999in}{1.356042in}}%
\pgfpathlineto{\pgfqpoint{0.294999in}{1.344489in}}%
\pgfpathlineto{\pgfqpoint{0.294999in}{1.332937in}}%
\pgfpathlineto{\pgfqpoint{0.294999in}{1.321384in}}%
\pgfpathlineto{\pgfqpoint{0.294999in}{1.309832in}}%
\pgfpathlineto{\pgfqpoint{0.294999in}{1.298279in}}%
\pgfpathlineto{\pgfqpoint{0.294999in}{1.286727in}}%
\pgfpathlineto{\pgfqpoint{0.294999in}{1.275174in}}%
\pgfpathlineto{\pgfqpoint{0.294999in}{1.263621in}}%
\pgfpathlineto{\pgfqpoint{0.294999in}{1.252069in}}%
\pgfpathlineto{\pgfqpoint{0.294999in}{1.240516in}}%
\pgfpathlineto{\pgfqpoint{0.294999in}{1.228964in}}%
\pgfpathlineto{\pgfqpoint{0.294999in}{1.217411in}}%
\pgfpathlineto{\pgfqpoint{0.294999in}{1.205859in}}%
\pgfpathlineto{\pgfqpoint{0.294999in}{1.197741in}}%
\pgfpathlineto{\pgfqpoint{0.298943in}{1.205859in}}%
\pgfpathlineto{\pgfqpoint{0.304651in}{1.217411in}}%
\pgfpathlineto{\pgfqpoint{0.307429in}{1.222811in}}%
\pgfpathlineto{\pgfqpoint{0.311706in}{1.228964in}}%
\pgfpathlineto{\pgfqpoint{0.319858in}{1.240273in}}%
\pgfpathlineto{\pgfqpoint{0.320126in}{1.240516in}}%
\pgfpathlineto{\pgfqpoint{0.332287in}{1.251037in}}%
\pgfpathlineto{\pgfqpoint{0.334739in}{1.252069in}}%
\pgfpathlineto{\pgfqpoint{0.344716in}{1.256018in}}%
\pgfpathlineto{\pgfqpoint{0.357146in}{1.255381in}}%
\pgfpathlineto{\pgfqpoint{0.363059in}{1.252069in}}%
\pgfpathlineto{\pgfqpoint{0.369575in}{1.247501in}}%
\pgfpathlineto{\pgfqpoint{0.374585in}{1.240516in}}%
\pgfpathlineto{\pgfqpoint{0.380931in}{1.228964in}}%
\pgfpathlineto{\pgfqpoint{0.382004in}{1.226444in}}%
\pgfpathlineto{\pgfqpoint{0.384614in}{1.217411in}}%
\pgfpathlineto{\pgfqpoint{0.387170in}{1.205859in}}%
\pgfpathlineto{\pgfqpoint{0.389111in}{1.194306in}}%
\pgfpathlineto{\pgfqpoint{0.390610in}{1.182754in}}%
\pgfpathlineto{\pgfqpoint{0.391814in}{1.171201in}}%
\pgfpathlineto{\pgfqpoint{0.392845in}{1.159648in}}%
\pgfpathlineto{\pgfqpoint{0.393807in}{1.148096in}}%
\pgfpathlineto{\pgfqpoint{0.394434in}{1.140828in}}%
\pgfpathlineto{\pgfqpoint{0.394726in}{1.136543in}}%
\pgfpathlineto{\pgfqpoint{0.395615in}{1.124991in}}%
\pgfpathlineto{\pgfqpoint{0.396642in}{1.113438in}}%
\pgfpathlineto{\pgfqpoint{0.397853in}{1.101886in}}%
\pgfpathlineto{\pgfqpoint{0.399285in}{1.090333in}}%
\pgfpathlineto{\pgfqpoint{0.400974in}{1.078781in}}%
\pgfpathlineto{\pgfqpoint{0.402948in}{1.067228in}}%
\pgfpathlineto{\pgfqpoint{0.405235in}{1.055675in}}%
\pgfpathlineto{\pgfqpoint{0.406863in}{1.048571in}}%
\pgfpathlineto{\pgfqpoint{0.407728in}{1.044123in}}%
\pgfpathlineto{\pgfqpoint{0.410324in}{1.032570in}}%
\pgfpathlineto{\pgfqpoint{0.413259in}{1.021018in}}%
\pgfpathlineto{\pgfqpoint{0.416548in}{1.009465in}}%
\pgfpathlineto{\pgfqpoint{0.419292in}{1.000845in}}%
\pgfpathlineto{\pgfqpoint{0.420110in}{0.997913in}}%
\pgfpathlineto{\pgfqpoint{0.423723in}{0.986360in}}%
\pgfpathlineto{\pgfqpoint{0.427703in}{0.974808in}}%
\pgfpathlineto{\pgfqpoint{0.431722in}{0.964178in}}%
\pgfpathlineto{\pgfqpoint{0.432034in}{0.963255in}}%
\pgfpathlineto{\pgfqpoint{0.436350in}{0.951702in}}%
\pgfpathlineto{\pgfqpoint{0.441046in}{0.940150in}}%
\pgfpathlineto{\pgfqpoint{0.444151in}{0.933129in}}%
\pgfpathlineto{\pgfqpoint{0.445975in}{0.928597in}}%
\pgfpathlineto{\pgfqpoint{0.451026in}{0.917045in}}%
\pgfpathlineto{\pgfqpoint{0.456474in}{0.905492in}}%
\pgfpathlineto{\pgfqpoint{0.456580in}{0.905283in}}%
\pgfpathlineto{\pgfqpoint{0.461913in}{0.893940in}}%
\pgfpathlineto{\pgfqpoint{0.467744in}{0.882387in}}%
\pgfpathlineto{\pgfqpoint{0.469009in}{0.880052in}}%
\pgfpathlineto{\pgfqpoint{0.473666in}{0.870835in}}%
\pgfpathlineto{\pgfqpoint{0.479912in}{0.859282in}}%
\pgfpathlineto{\pgfqpoint{0.481439in}{0.856639in}}%
\pgfpathlineto{\pgfqpoint{0.486280in}{0.847730in}}%
\pgfpathlineto{\pgfqpoint{0.492976in}{0.836177in}}%
\pgfpathlineto{\pgfqpoint{0.493868in}{0.834730in}}%
\pgfpathlineto{\pgfqpoint{0.499777in}{0.824624in}}%
\pgfpathlineto{\pgfqpoint{0.506297in}{0.814138in}}%
\pgfpathlineto{\pgfqpoint{0.506930in}{0.813072in}}%
\pgfpathlineto{\pgfqpoint{0.514199in}{0.801519in}}%
\pgfpathlineto{\pgfqpoint{0.518727in}{0.794729in}}%
\pgfpathlineto{\pgfqpoint{0.521778in}{0.789967in}}%
\pgfpathlineto{\pgfqpoint{0.529609in}{0.778414in}}%
\pgfpathlineto{\pgfqpoint{0.531156in}{0.776249in}}%
\pgfpathlineto{\pgfqpoint{0.537642in}{0.766862in}}%
\pgfpathlineto{\pgfqpoint{0.543585in}{0.758711in}}%
\pgfpathlineto{\pgfqpoint{0.545996in}{0.755309in}}%
\pgfpathlineto{\pgfqpoint{0.554610in}{0.743757in}}%
\pgfpathlineto{\pgfqpoint{0.556015in}{0.741962in}}%
\pgfpathlineto{\pgfqpoint{0.563473in}{0.732204in}}%
\pgfpathlineto{\pgfqpoint{0.568444in}{0.726011in}}%
\pgfpathlineto{\pgfqpoint{0.572664in}{0.720651in}}%
\pgfpathlineto{\pgfqpoint{0.580873in}{0.710707in}}%
\pgfpathlineto{\pgfqpoint{0.582181in}{0.709099in}}%
\pgfpathlineto{\pgfqpoint{0.591998in}{0.697546in}}%
\pgfpathlineto{\pgfqpoint{0.593302in}{0.696075in}}%
\pgfpathlineto{\pgfqpoint{0.602142in}{0.685994in}}%
\pgfpathlineto{\pgfqpoint{0.605732in}{0.682068in}}%
\pgfpathlineto{\pgfqpoint{0.612655in}{0.674441in}}%
\pgfpathlineto{\pgfqpoint{0.618161in}{0.668614in}}%
\pgfpathlineto{\pgfqpoint{0.623548in}{0.662889in}}%
\pgfpathlineto{\pgfqpoint{0.630590in}{0.655686in}}%
\pgfpathlineto{\pgfqpoint{0.634840in}{0.651336in}}%
\pgfpathlineto{\pgfqpoint{0.643020in}{0.643261in}}%
\pgfpathlineto{\pgfqpoint{0.646550in}{0.639784in}}%
\pgfpathlineto{\pgfqpoint{0.655449in}{0.631314in}}%
\pgfpathlineto{\pgfqpoint{0.658705in}{0.628231in}}%
\pgfpathlineto{\pgfqpoint{0.667878in}{0.619820in}}%
\pgfpathlineto{\pgfqpoint{0.671332in}{0.616678in}}%
\pgfpathlineto{\pgfqpoint{0.680308in}{0.608757in}}%
\pgfpathlineto{\pgfqpoint{0.684466in}{0.605126in}}%
\pgfpathlineto{\pgfqpoint{0.692737in}{0.598104in}}%
\pgfpathlineto{\pgfqpoint{0.698143in}{0.593573in}}%
\pgfpathlineto{\pgfqpoint{0.705166in}{0.587840in}}%
\pgfpathlineto{\pgfqpoint{0.712407in}{0.582021in}}%
\pgfpathlineto{\pgfqpoint{0.717595in}{0.577950in}}%
\pgfpathlineto{\pgfqpoint{0.727306in}{0.570468in}}%
\pgfpathlineto{\pgfqpoint{0.730025in}{0.568418in}}%
\pgfpathlineto{\pgfqpoint{0.742454in}{0.559235in}}%
\pgfpathlineto{\pgfqpoint{0.742895in}{0.558916in}}%
\pgfpathlineto{\pgfqpoint{0.754883in}{0.550399in}}%
\pgfpathlineto{\pgfqpoint{0.759266in}{0.547363in}}%
\pgfpathlineto{\pgfqpoint{0.767313in}{0.541884in}}%
\pgfpathlineto{\pgfqpoint{0.776479in}{0.535811in}}%
\pgfpathlineto{\pgfqpoint{0.779742in}{0.533681in}}%
\pgfpathlineto{\pgfqpoint{0.792171in}{0.525790in}}%
\pgfpathlineto{\pgfqpoint{0.794656in}{0.524258in}}%
\pgfpathlineto{\pgfqpoint{0.804601in}{0.518202in}}%
\pgfpathlineto{\pgfqpoint{0.813943in}{0.512705in}}%
\pgfpathlineto{\pgfqpoint{0.817030in}{0.510907in}}%
\pgfpathlineto{\pgfqpoint{0.829459in}{0.503905in}}%
\pgfpathlineto{\pgfqpoint{0.834530in}{0.501153in}}%
\pgfpathlineto{\pgfqpoint{0.841888in}{0.497187in}}%
\pgfpathlineto{\pgfqpoint{0.854318in}{0.490756in}}%
\pgfpathlineto{\pgfqpoint{0.856643in}{0.489600in}}%
\pgfpathlineto{\pgfqpoint{0.866747in}{0.484599in}}%
\pgfpathlineto{\pgfqpoint{0.879176in}{0.478728in}}%
\pgfpathlineto{\pgfqpoint{0.880682in}{0.478048in}}%
\pgfpathlineto{\pgfqpoint{0.891606in}{0.473122in}}%
\pgfpathlineto{\pgfqpoint{0.904035in}{0.467802in}}%
\pgfpathlineto{\pgfqpoint{0.907252in}{0.466495in}}%
\pgfpathlineto{\pgfqpoint{0.916464in}{0.462746in}}%
\pgfpathlineto{\pgfqpoint{0.928894in}{0.457971in}}%
\pgfpathlineto{\pgfqpoint{0.937272in}{0.454943in}}%
\pgfpathlineto{\pgfqpoint{0.941323in}{0.453471in}}%
\pgfpathlineto{\pgfqpoint{0.953752in}{0.449237in}}%
\pgfpathlineto{\pgfqpoint{0.966181in}{0.445291in}}%
\pgfpathlineto{\pgfqpoint{0.972640in}{0.443390in}}%
\pgfpathlineto{\pgfqpoint{0.978611in}{0.441616in}}%
\pgfpathlineto{\pgfqpoint{0.991040in}{0.438212in}}%
\pgfpathlineto{\pgfqpoint{1.003469in}{0.435099in}}%
\pgfpathlineto{\pgfqpoint{1.015899in}{0.432275in}}%
\pgfpathlineto{\pgfqpoint{1.018053in}{0.431838in}}%
\pgfpathlineto{\pgfqpoint{1.028328in}{0.429717in}}%
\pgfpathlineto{\pgfqpoint{1.040757in}{0.427448in}}%
\pgfpathlineto{\pgfqpoint{1.053187in}{0.425472in}}%
\pgfpathlineto{\pgfqpoint{1.065616in}{0.423791in}}%
\pgfpathlineto{\pgfqpoint{1.078045in}{0.422406in}}%
\pgfpathlineto{\pgfqpoint{1.090474in}{0.421318in}}%
\pgfpathlineto{\pgfqpoint{1.102904in}{0.420528in}}%
\pgfpathlineto{\pgfqpoint{1.109086in}{0.420285in}}%
\pgfpathclose%
\pgfusepath{fill}%
\end{pgfscope}%
\begin{pgfscope}%
\pgfpathrectangle{\pgfqpoint{0.211875in}{0.211875in}}{\pgfqpoint{1.313625in}{1.279725in}}%
\pgfusepath{clip}%
\pgfsetbuttcap%
\pgfsetroundjoin%
\definecolor{currentfill}{rgb}{0.901975,0.231521,0.249182}%
\pgfsetfillcolor{currentfill}%
\pgfsetlinewidth{0.000000pt}%
\definecolor{currentstroke}{rgb}{0.000000,0.000000,0.000000}%
\pgfsetstrokecolor{currentstroke}%
\pgfsetdash{}{0pt}%
\pgfpathmoveto{\pgfqpoint{1.015899in}{0.487014in}}%
\pgfpathlineto{\pgfqpoint{1.028328in}{0.484754in}}%
\pgfpathlineto{\pgfqpoint{1.040757in}{0.482828in}}%
\pgfpathlineto{\pgfqpoint{1.053187in}{0.481238in}}%
\pgfpathlineto{\pgfqpoint{1.065616in}{0.479987in}}%
\pgfpathlineto{\pgfqpoint{1.078045in}{0.479079in}}%
\pgfpathlineto{\pgfqpoint{1.090474in}{0.478517in}}%
\pgfpathlineto{\pgfqpoint{1.102904in}{0.478304in}}%
\pgfpathlineto{\pgfqpoint{1.115333in}{0.478445in}}%
\pgfpathlineto{\pgfqpoint{1.127762in}{0.478944in}}%
\pgfpathlineto{\pgfqpoint{1.140192in}{0.479805in}}%
\pgfpathlineto{\pgfqpoint{1.152621in}{0.481032in}}%
\pgfpathlineto{\pgfqpoint{1.165050in}{0.482631in}}%
\pgfpathlineto{\pgfqpoint{1.177480in}{0.484605in}}%
\pgfpathlineto{\pgfqpoint{1.189909in}{0.486960in}}%
\pgfpathlineto{\pgfqpoint{1.201892in}{0.489600in}}%
\pgfpathlineto{\pgfqpoint{1.202338in}{0.489700in}}%
\pgfpathlineto{\pgfqpoint{1.214767in}{0.492826in}}%
\pgfpathlineto{\pgfqpoint{1.227197in}{0.496345in}}%
\pgfpathlineto{\pgfqpoint{1.239626in}{0.500260in}}%
\pgfpathlineto{\pgfqpoint{1.242241in}{0.501153in}}%
\pgfpathlineto{\pgfqpoint{1.252055in}{0.504574in}}%
\pgfpathlineto{\pgfqpoint{1.264485in}{0.509289in}}%
\pgfpathlineto{\pgfqpoint{1.272845in}{0.512705in}}%
\pgfpathlineto{\pgfqpoint{1.276914in}{0.514408in}}%
\pgfpathlineto{\pgfqpoint{1.289343in}{0.519931in}}%
\pgfpathlineto{\pgfqpoint{1.298483in}{0.524258in}}%
\pgfpathlineto{\pgfqpoint{1.301773in}{0.525857in}}%
\pgfpathlineto{\pgfqpoint{1.314202in}{0.532187in}}%
\pgfpathlineto{\pgfqpoint{1.320987in}{0.535811in}}%
\pgfpathlineto{\pgfqpoint{1.326631in}{0.538914in}}%
\pgfpathlineto{\pgfqpoint{1.339060in}{0.546031in}}%
\pgfpathlineto{\pgfqpoint{1.341323in}{0.547363in}}%
\pgfpathlineto{\pgfqpoint{1.351490in}{0.553540in}}%
\pgfpathlineto{\pgfqpoint{1.360064in}{0.558916in}}%
\pgfpathlineto{\pgfqpoint{1.363919in}{0.561415in}}%
\pgfpathlineto{\pgfqpoint{1.376348in}{0.569644in}}%
\pgfpathlineto{\pgfqpoint{1.377582in}{0.570468in}}%
\pgfpathlineto{\pgfqpoint{1.388778in}{0.578220in}}%
\pgfpathlineto{\pgfqpoint{1.394206in}{0.582021in}}%
\pgfpathlineto{\pgfqpoint{1.401207in}{0.587104in}}%
\pgfpathlineto{\pgfqpoint{1.410084in}{0.593573in}}%
\pgfpathlineto{\pgfqpoint{1.413636in}{0.596262in}}%
\pgfpathlineto{\pgfqpoint{1.425391in}{0.605126in}}%
\pgfpathlineto{\pgfqpoint{1.426066in}{0.605655in}}%
\pgfpathlineto{\pgfqpoint{1.438495in}{0.615249in}}%
\pgfpathlineto{\pgfqpoint{1.440382in}{0.616678in}}%
\pgfpathlineto{\pgfqpoint{1.450924in}{0.624980in}}%
\pgfpathlineto{\pgfqpoint{1.455158in}{0.628231in}}%
\pgfpathlineto{\pgfqpoint{1.463353in}{0.634775in}}%
\pgfpathlineto{\pgfqpoint{1.469841in}{0.639784in}}%
\pgfpathlineto{\pgfqpoint{1.475783in}{0.644554in}}%
\pgfpathlineto{\pgfqpoint{1.484598in}{0.651336in}}%
\pgfpathlineto{\pgfqpoint{1.488212in}{0.654227in}}%
\pgfpathlineto{\pgfqpoint{1.499624in}{0.662889in}}%
\pgfpathlineto{\pgfqpoint{1.500641in}{0.663691in}}%
\pgfpathlineto{\pgfqpoint{1.513071in}{0.672852in}}%
\pgfpathlineto{\pgfqpoint{1.515402in}{0.674441in}}%
\pgfpathlineto{\pgfqpoint{1.525500in}{0.681583in}}%
\pgfpathlineto{\pgfqpoint{1.525500in}{0.685994in}}%
\pgfpathlineto{\pgfqpoint{1.525500in}{0.697546in}}%
\pgfpathlineto{\pgfqpoint{1.525500in}{0.709099in}}%
\pgfpathlineto{\pgfqpoint{1.525500in}{0.720651in}}%
\pgfpathlineto{\pgfqpoint{1.525500in}{0.732204in}}%
\pgfpathlineto{\pgfqpoint{1.525500in}{0.743757in}}%
\pgfpathlineto{\pgfqpoint{1.525500in}{0.755309in}}%
\pgfpathlineto{\pgfqpoint{1.525500in}{0.766862in}}%
\pgfpathlineto{\pgfqpoint{1.525500in}{0.778414in}}%
\pgfpathlineto{\pgfqpoint{1.525500in}{0.789967in}}%
\pgfpathlineto{\pgfqpoint{1.525500in}{0.801519in}}%
\pgfpathlineto{\pgfqpoint{1.525500in}{0.813072in}}%
\pgfpathlineto{\pgfqpoint{1.525500in}{0.824624in}}%
\pgfpathlineto{\pgfqpoint{1.525500in}{0.836177in}}%
\pgfpathlineto{\pgfqpoint{1.525500in}{0.837936in}}%
\pgfpathlineto{\pgfqpoint{1.523979in}{0.836177in}}%
\pgfpathlineto{\pgfqpoint{1.513715in}{0.824624in}}%
\pgfpathlineto{\pgfqpoint{1.513071in}{0.823920in}}%
\pgfpathlineto{\pgfqpoint{1.503988in}{0.813072in}}%
\pgfpathlineto{\pgfqpoint{1.500641in}{0.809201in}}%
\pgfpathlineto{\pgfqpoint{1.494486in}{0.801519in}}%
\pgfpathlineto{\pgfqpoint{1.488212in}{0.793963in}}%
\pgfpathlineto{\pgfqpoint{1.485104in}{0.789967in}}%
\pgfpathlineto{\pgfqpoint{1.475783in}{0.778433in}}%
\pgfpathlineto{\pgfqpoint{1.475768in}{0.778414in}}%
\pgfpathlineto{\pgfqpoint{1.466589in}{0.766862in}}%
\pgfpathlineto{\pgfqpoint{1.463353in}{0.762956in}}%
\pgfpathlineto{\pgfqpoint{1.457305in}{0.755309in}}%
\pgfpathlineto{\pgfqpoint{1.450924in}{0.747586in}}%
\pgfpathlineto{\pgfqpoint{1.447879in}{0.743757in}}%
\pgfpathlineto{\pgfqpoint{1.438495in}{0.732477in}}%
\pgfpathlineto{\pgfqpoint{1.438275in}{0.732204in}}%
\pgfpathlineto{\pgfqpoint{1.428543in}{0.720651in}}%
\pgfpathlineto{\pgfqpoint{1.426066in}{0.717842in}}%
\pgfpathlineto{\pgfqpoint{1.418529in}{0.709099in}}%
\pgfpathlineto{\pgfqpoint{1.413636in}{0.703681in}}%
\pgfpathlineto{\pgfqpoint{1.408183in}{0.697546in}}%
\pgfpathlineto{\pgfqpoint{1.401207in}{0.690057in}}%
\pgfpathlineto{\pgfqpoint{1.397457in}{0.685994in}}%
\pgfpathlineto{\pgfqpoint{1.388778in}{0.677017in}}%
\pgfpathlineto{\pgfqpoint{1.386294in}{0.674441in}}%
\pgfpathlineto{\pgfqpoint{1.376348in}{0.664594in}}%
\pgfpathlineto{\pgfqpoint{1.374620in}{0.662889in}}%
\pgfpathlineto{\pgfqpoint{1.363919in}{0.652801in}}%
\pgfpathlineto{\pgfqpoint{1.362349in}{0.651336in}}%
\pgfpathlineto{\pgfqpoint{1.351490in}{0.641645in}}%
\pgfpathlineto{\pgfqpoint{1.349369in}{0.639784in}}%
\pgfpathlineto{\pgfqpoint{1.339060in}{0.631119in}}%
\pgfpathlineto{\pgfqpoint{1.335543in}{0.628231in}}%
\pgfpathlineto{\pgfqpoint{1.326631in}{0.621214in}}%
\pgfpathlineto{\pgfqpoint{1.320694in}{0.616678in}}%
\pgfpathlineto{\pgfqpoint{1.314202in}{0.611915in}}%
\pgfpathlineto{\pgfqpoint{1.304596in}{0.605126in}}%
\pgfpathlineto{\pgfqpoint{1.301773in}{0.603206in}}%
\pgfpathlineto{\pgfqpoint{1.289343in}{0.595095in}}%
\pgfpathlineto{\pgfqpoint{1.286902in}{0.593573in}}%
\pgfpathlineto{\pgfqpoint{1.276914in}{0.587573in}}%
\pgfpathlineto{\pgfqpoint{1.267108in}{0.582021in}}%
\pgfpathlineto{\pgfqpoint{1.264485in}{0.580586in}}%
\pgfpathlineto{\pgfqpoint{1.252055in}{0.574170in}}%
\pgfpathlineto{\pgfqpoint{1.244345in}{0.570468in}}%
\pgfpathlineto{\pgfqpoint{1.239626in}{0.568274in}}%
\pgfpathlineto{\pgfqpoint{1.227197in}{0.562911in}}%
\pgfpathlineto{\pgfqpoint{1.217056in}{0.558916in}}%
\pgfpathlineto{\pgfqpoint{1.214767in}{0.558040in}}%
\pgfpathlineto{\pgfqpoint{1.202338in}{0.553690in}}%
\pgfpathlineto{\pgfqpoint{1.189909in}{0.549809in}}%
\pgfpathlineto{\pgfqpoint{1.181048in}{0.547363in}}%
\pgfpathlineto{\pgfqpoint{1.177480in}{0.546403in}}%
\pgfpathlineto{\pgfqpoint{1.165050in}{0.543476in}}%
\pgfpathlineto{\pgfqpoint{1.152621in}{0.540999in}}%
\pgfpathlineto{\pgfqpoint{1.140192in}{0.538963in}}%
\pgfpathlineto{\pgfqpoint{1.127762in}{0.537364in}}%
\pgfpathlineto{\pgfqpoint{1.115333in}{0.536195in}}%
\pgfpathlineto{\pgfqpoint{1.108942in}{0.535811in}}%
\pgfpathlineto{\pgfqpoint{1.102904in}{0.535454in}}%
\pgfpathlineto{\pgfqpoint{1.090474in}{0.535131in}}%
\pgfpathlineto{\pgfqpoint{1.078045in}{0.535216in}}%
\pgfpathlineto{\pgfqpoint{1.065616in}{0.535704in}}%
\pgfpathlineto{\pgfqpoint{1.064136in}{0.535811in}}%
\pgfpathlineto{\pgfqpoint{1.053187in}{0.536598in}}%
\pgfpathlineto{\pgfqpoint{1.040757in}{0.537889in}}%
\pgfpathlineto{\pgfqpoint{1.028328in}{0.539570in}}%
\pgfpathlineto{\pgfqpoint{1.015899in}{0.541639in}}%
\pgfpathlineto{\pgfqpoint{1.003469in}{0.544090in}}%
\pgfpathlineto{\pgfqpoint{0.991040in}{0.546919in}}%
\pgfpathlineto{\pgfqpoint{0.989319in}{0.547363in}}%
\pgfpathlineto{\pgfqpoint{0.978611in}{0.550152in}}%
\pgfpathlineto{\pgfqpoint{0.966181in}{0.553763in}}%
\pgfpathlineto{\pgfqpoint{0.953752in}{0.557746in}}%
\pgfpathlineto{\pgfqpoint{0.950410in}{0.558916in}}%
\pgfpathlineto{\pgfqpoint{0.941323in}{0.562135in}}%
\pgfpathlineto{\pgfqpoint{0.928894in}{0.566907in}}%
\pgfpathlineto{\pgfqpoint{0.920277in}{0.570468in}}%
\pgfpathlineto{\pgfqpoint{0.916464in}{0.572068in}}%
\pgfpathlineto{\pgfqpoint{0.904035in}{0.577646in}}%
\pgfpathlineto{\pgfqpoint{0.894879in}{0.582021in}}%
\pgfpathlineto{\pgfqpoint{0.891606in}{0.583613in}}%
\pgfpathlineto{\pgfqpoint{0.879176in}{0.590017in}}%
\pgfpathlineto{\pgfqpoint{0.872636in}{0.593573in}}%
\pgfpathlineto{\pgfqpoint{0.866747in}{0.596843in}}%
\pgfpathlineto{\pgfqpoint{0.854318in}{0.604098in}}%
\pgfpathlineto{\pgfqpoint{0.852635in}{0.605126in}}%
\pgfpathlineto{\pgfqpoint{0.841888in}{0.611847in}}%
\pgfpathlineto{\pgfqpoint{0.834492in}{0.616678in}}%
\pgfpathlineto{\pgfqpoint{0.829459in}{0.620054in}}%
\pgfpathlineto{\pgfqpoint{0.817748in}{0.628231in}}%
\pgfpathlineto{\pgfqpoint{0.817030in}{0.628747in}}%
\pgfpathlineto{\pgfqpoint{0.804601in}{0.638008in}}%
\pgfpathlineto{\pgfqpoint{0.802297in}{0.639784in}}%
\pgfpathlineto{\pgfqpoint{0.792171in}{0.647839in}}%
\pgfpathlineto{\pgfqpoint{0.787911in}{0.651336in}}%
\pgfpathlineto{\pgfqpoint{0.779742in}{0.658274in}}%
\pgfpathlineto{\pgfqpoint{0.774462in}{0.662889in}}%
\pgfpathlineto{\pgfqpoint{0.767313in}{0.669369in}}%
\pgfpathlineto{\pgfqpoint{0.761859in}{0.674441in}}%
\pgfpathlineto{\pgfqpoint{0.754883in}{0.681187in}}%
\pgfpathlineto{\pgfqpoint{0.750026in}{0.685994in}}%
\pgfpathlineto{\pgfqpoint{0.742454in}{0.693805in}}%
\pgfpathlineto{\pgfqpoint{0.738900in}{0.697546in}}%
\pgfpathlineto{\pgfqpoint{0.730025in}{0.707310in}}%
\pgfpathlineto{\pgfqpoint{0.728427in}{0.709099in}}%
\pgfpathlineto{\pgfqpoint{0.718574in}{0.720651in}}%
\pgfpathlineto{\pgfqpoint{0.717595in}{0.721856in}}%
\pgfpathlineto{\pgfqpoint{0.709306in}{0.732204in}}%
\pgfpathlineto{\pgfqpoint{0.705166in}{0.737650in}}%
\pgfpathlineto{\pgfqpoint{0.700576in}{0.743757in}}%
\pgfpathlineto{\pgfqpoint{0.692737in}{0.754774in}}%
\pgfpathlineto{\pgfqpoint{0.692359in}{0.755309in}}%
\pgfpathlineto{\pgfqpoint{0.684650in}{0.766862in}}%
\pgfpathlineto{\pgfqpoint{0.680308in}{0.773777in}}%
\pgfpathlineto{\pgfqpoint{0.677409in}{0.778414in}}%
\pgfpathlineto{\pgfqpoint{0.670630in}{0.789967in}}%
\pgfpathlineto{\pgfqpoint{0.667878in}{0.794980in}}%
\pgfpathlineto{\pgfqpoint{0.664292in}{0.801519in}}%
\pgfpathlineto{\pgfqpoint{0.658387in}{0.813072in}}%
\pgfpathlineto{\pgfqpoint{0.655449in}{0.819264in}}%
\pgfpathlineto{\pgfqpoint{0.652898in}{0.824624in}}%
\pgfpathlineto{\pgfqpoint{0.647816in}{0.836177in}}%
\pgfpathlineto{\pgfqpoint{0.643143in}{0.847730in}}%
\pgfpathlineto{\pgfqpoint{0.643020in}{0.848063in}}%
\pgfpathlineto{\pgfqpoint{0.638843in}{0.859282in}}%
\pgfpathlineto{\pgfqpoint{0.634936in}{0.870835in}}%
\pgfpathlineto{\pgfqpoint{0.631413in}{0.882387in}}%
\pgfpathlineto{\pgfqpoint{0.630590in}{0.885417in}}%
\pgfpathlineto{\pgfqpoint{0.628247in}{0.893940in}}%
\pgfpathlineto{\pgfqpoint{0.625446in}{0.905492in}}%
\pgfpathlineto{\pgfqpoint{0.623010in}{0.917045in}}%
\pgfpathlineto{\pgfqpoint{0.620933in}{0.928597in}}%
\pgfpathlineto{\pgfqpoint{0.619208in}{0.940150in}}%
\pgfpathlineto{\pgfqpoint{0.618161in}{0.948924in}}%
\pgfpathlineto{\pgfqpoint{0.617823in}{0.951702in}}%
\pgfpathlineto{\pgfqpoint{0.616768in}{0.963255in}}%
\pgfpathlineto{\pgfqpoint{0.616049in}{0.974808in}}%
\pgfpathlineto{\pgfqpoint{0.615659in}{0.986360in}}%
\pgfpathlineto{\pgfqpoint{0.615592in}{0.997913in}}%
\pgfpathlineto{\pgfqpoint{0.615841in}{1.009465in}}%
\pgfpathlineto{\pgfqpoint{0.616399in}{1.021018in}}%
\pgfpathlineto{\pgfqpoint{0.617257in}{1.032570in}}%
\pgfpathlineto{\pgfqpoint{0.618161in}{1.041661in}}%
\pgfpathlineto{\pgfqpoint{0.618406in}{1.044123in}}%
\pgfpathlineto{\pgfqpoint{0.619823in}{1.055675in}}%
\pgfpathlineto{\pgfqpoint{0.621511in}{1.067228in}}%
\pgfpathlineto{\pgfqpoint{0.623462in}{1.078781in}}%
\pgfpathlineto{\pgfqpoint{0.625663in}{1.090333in}}%
\pgfpathlineto{\pgfqpoint{0.628103in}{1.101886in}}%
\pgfpathlineto{\pgfqpoint{0.630590in}{1.112671in}}%
\pgfpathlineto{\pgfqpoint{0.630768in}{1.113438in}}%
\pgfpathlineto{\pgfqpoint{0.633616in}{1.124991in}}%
\pgfpathlineto{\pgfqpoint{0.636659in}{1.136543in}}%
\pgfpathlineto{\pgfqpoint{0.639882in}{1.148096in}}%
\pgfpathlineto{\pgfqpoint{0.643020in}{1.158816in}}%
\pgfpathlineto{\pgfqpoint{0.643265in}{1.159648in}}%
\pgfpathlineto{\pgfqpoint{0.646763in}{1.171201in}}%
\pgfpathlineto{\pgfqpoint{0.650383in}{1.182754in}}%
\pgfpathlineto{\pgfqpoint{0.654104in}{1.194306in}}%
\pgfpathlineto{\pgfqpoint{0.655449in}{1.198448in}}%
\pgfpathlineto{\pgfqpoint{0.657887in}{1.205859in}}%
\pgfpathlineto{\pgfqpoint{0.661714in}{1.217411in}}%
\pgfpathlineto{\pgfqpoint{0.665567in}{1.228964in}}%
\pgfpathlineto{\pgfqpoint{0.667878in}{1.235952in}}%
\pgfpathlineto{\pgfqpoint{0.669413in}{1.240516in}}%
\pgfpathlineto{\pgfqpoint{0.673219in}{1.252069in}}%
\pgfpathlineto{\pgfqpoint{0.676961in}{1.263621in}}%
\pgfpathlineto{\pgfqpoint{0.680308in}{1.274247in}}%
\pgfpathlineto{\pgfqpoint{0.680606in}{1.275174in}}%
\pgfpathlineto{\pgfqpoint{0.684114in}{1.286727in}}%
\pgfpathlineto{\pgfqpoint{0.687450in}{1.298279in}}%
\pgfpathlineto{\pgfqpoint{0.690574in}{1.309832in}}%
\pgfpathlineto{\pgfqpoint{0.692737in}{1.318591in}}%
\pgfpathlineto{\pgfqpoint{0.693444in}{1.321384in}}%
\pgfpathlineto{\pgfqpoint{0.696015in}{1.332937in}}%
\pgfpathlineto{\pgfqpoint{0.698234in}{1.344489in}}%
\pgfpathlineto{\pgfqpoint{0.700049in}{1.356042in}}%
\pgfpathlineto{\pgfqpoint{0.701404in}{1.367594in}}%
\pgfpathlineto{\pgfqpoint{0.702240in}{1.379147in}}%
\pgfpathlineto{\pgfqpoint{0.702492in}{1.390699in}}%
\pgfpathlineto{\pgfqpoint{0.702093in}{1.402252in}}%
\pgfpathlineto{\pgfqpoint{0.700968in}{1.413805in}}%
\pgfpathlineto{\pgfqpoint{0.692737in}{1.413805in}}%
\pgfpathlineto{\pgfqpoint{0.680308in}{1.413805in}}%
\pgfpathlineto{\pgfqpoint{0.667878in}{1.413805in}}%
\pgfpathlineto{\pgfqpoint{0.655449in}{1.413805in}}%
\pgfpathlineto{\pgfqpoint{0.643020in}{1.413805in}}%
\pgfpathlineto{\pgfqpoint{0.630590in}{1.413805in}}%
\pgfpathlineto{\pgfqpoint{0.618161in}{1.413805in}}%
\pgfpathlineto{\pgfqpoint{0.605732in}{1.413805in}}%
\pgfpathlineto{\pgfqpoint{0.593302in}{1.413805in}}%
\pgfpathlineto{\pgfqpoint{0.580873in}{1.413805in}}%
\pgfpathlineto{\pgfqpoint{0.568444in}{1.413805in}}%
\pgfpathlineto{\pgfqpoint{0.556015in}{1.413805in}}%
\pgfpathlineto{\pgfqpoint{0.545238in}{1.413805in}}%
\pgfpathlineto{\pgfqpoint{0.551743in}{1.402252in}}%
\pgfpathlineto{\pgfqpoint{0.556015in}{1.393104in}}%
\pgfpathlineto{\pgfqpoint{0.557064in}{1.390699in}}%
\pgfpathlineto{\pgfqpoint{0.561223in}{1.379147in}}%
\pgfpathlineto{\pgfqpoint{0.564479in}{1.367594in}}%
\pgfpathlineto{\pgfqpoint{0.566922in}{1.356042in}}%
\pgfpathlineto{\pgfqpoint{0.568444in}{1.345805in}}%
\pgfpathlineto{\pgfqpoint{0.568628in}{1.344489in}}%
\pgfpathlineto{\pgfqpoint{0.569652in}{1.332937in}}%
\pgfpathlineto{\pgfqpoint{0.570116in}{1.321384in}}%
\pgfpathlineto{\pgfqpoint{0.570085in}{1.309832in}}%
\pgfpathlineto{\pgfqpoint{0.569621in}{1.298279in}}%
\pgfpathlineto{\pgfqpoint{0.568783in}{1.286727in}}%
\pgfpathlineto{\pgfqpoint{0.568444in}{1.283405in}}%
\pgfpathlineto{\pgfqpoint{0.567597in}{1.275174in}}%
\pgfpathlineto{\pgfqpoint{0.566122in}{1.263621in}}%
\pgfpathlineto{\pgfqpoint{0.564418in}{1.252069in}}%
\pgfpathlineto{\pgfqpoint{0.562532in}{1.240516in}}%
\pgfpathlineto{\pgfqpoint{0.560507in}{1.228964in}}%
\pgfpathlineto{\pgfqpoint{0.558383in}{1.217411in}}%
\pgfpathlineto{\pgfqpoint{0.556197in}{1.205859in}}%
\pgfpathlineto{\pgfqpoint{0.556015in}{1.204925in}}%
\pgfpathlineto{\pgfqpoint{0.553910in}{1.194306in}}%
\pgfpathlineto{\pgfqpoint{0.551623in}{1.182754in}}%
\pgfpathlineto{\pgfqpoint{0.549372in}{1.171201in}}%
\pgfpathlineto{\pgfqpoint{0.547186in}{1.159648in}}%
\pgfpathlineto{\pgfqpoint{0.545091in}{1.148096in}}%
\pgfpathlineto{\pgfqpoint{0.543585in}{1.139350in}}%
\pgfpathlineto{\pgfqpoint{0.543092in}{1.136543in}}%
\pgfpathlineto{\pgfqpoint{0.541176in}{1.124991in}}%
\pgfpathlineto{\pgfqpoint{0.539423in}{1.113438in}}%
\pgfpathlineto{\pgfqpoint{0.537852in}{1.101886in}}%
\pgfpathlineto{\pgfqpoint{0.536479in}{1.090333in}}%
\pgfpathlineto{\pgfqpoint{0.535319in}{1.078781in}}%
\pgfpathlineto{\pgfqpoint{0.534387in}{1.067228in}}%
\pgfpathlineto{\pgfqpoint{0.533695in}{1.055675in}}%
\pgfpathlineto{\pgfqpoint{0.533254in}{1.044123in}}%
\pgfpathlineto{\pgfqpoint{0.533075in}{1.032570in}}%
\pgfpathlineto{\pgfqpoint{0.533168in}{1.021018in}}%
\pgfpathlineto{\pgfqpoint{0.533542in}{1.009465in}}%
\pgfpathlineto{\pgfqpoint{0.534205in}{0.997913in}}%
\pgfpathlineto{\pgfqpoint{0.535165in}{0.986360in}}%
\pgfpathlineto{\pgfqpoint{0.536431in}{0.974808in}}%
\pgfpathlineto{\pgfqpoint{0.538009in}{0.963255in}}%
\pgfpathlineto{\pgfqpoint{0.539908in}{0.951702in}}%
\pgfpathlineto{\pgfqpoint{0.542136in}{0.940150in}}%
\pgfpathlineto{\pgfqpoint{0.543585in}{0.933636in}}%
\pgfpathlineto{\pgfqpoint{0.544662in}{0.928597in}}%
\pgfpathlineto{\pgfqpoint{0.547473in}{0.917045in}}%
\pgfpathlineto{\pgfqpoint{0.550626in}{0.905492in}}%
\pgfpathlineto{\pgfqpoint{0.554130in}{0.893940in}}%
\pgfpathlineto{\pgfqpoint{0.556015in}{0.888312in}}%
\pgfpathlineto{\pgfqpoint{0.557935in}{0.882387in}}%
\pgfpathlineto{\pgfqpoint{0.562044in}{0.870835in}}%
\pgfpathlineto{\pgfqpoint{0.566521in}{0.859282in}}%
\pgfpathlineto{\pgfqpoint{0.568444in}{0.854712in}}%
\pgfpathlineto{\pgfqpoint{0.571303in}{0.847730in}}%
\pgfpathlineto{\pgfqpoint{0.576417in}{0.836177in}}%
\pgfpathlineto{\pgfqpoint{0.580873in}{0.826824in}}%
\pgfpathlineto{\pgfqpoint{0.581899in}{0.824624in}}%
\pgfpathlineto{\pgfqpoint{0.587677in}{0.813072in}}%
\pgfpathlineto{\pgfqpoint{0.593302in}{0.802562in}}%
\pgfpathlineto{\pgfqpoint{0.593851in}{0.801519in}}%
\pgfpathlineto{\pgfqpoint{0.600330in}{0.789967in}}%
\pgfpathlineto{\pgfqpoint{0.605732in}{0.780921in}}%
\pgfpathlineto{\pgfqpoint{0.607210in}{0.778414in}}%
\pgfpathlineto{\pgfqpoint{0.614433in}{0.766862in}}%
\pgfpathlineto{\pgfqpoint{0.618161in}{0.761230in}}%
\pgfpathlineto{\pgfqpoint{0.622049in}{0.755309in}}%
\pgfpathlineto{\pgfqpoint{0.630066in}{0.743757in}}%
\pgfpathlineto{\pgfqpoint{0.630590in}{0.743039in}}%
\pgfpathlineto{\pgfqpoint{0.638464in}{0.732204in}}%
\pgfpathlineto{\pgfqpoint{0.643020in}{0.726249in}}%
\pgfpathlineto{\pgfqpoint{0.647297in}{0.720651in}}%
\pgfpathlineto{\pgfqpoint{0.655449in}{0.710492in}}%
\pgfpathlineto{\pgfqpoint{0.656569in}{0.709099in}}%
\pgfpathlineto{\pgfqpoint{0.666290in}{0.697546in}}%
\pgfpathlineto{\pgfqpoint{0.667878in}{0.695739in}}%
\pgfpathlineto{\pgfqpoint{0.676489in}{0.685994in}}%
\pgfpathlineto{\pgfqpoint{0.680308in}{0.681852in}}%
\pgfpathlineto{\pgfqpoint{0.687195in}{0.674441in}}%
\pgfpathlineto{\pgfqpoint{0.692737in}{0.668713in}}%
\pgfpathlineto{\pgfqpoint{0.698435in}{0.662889in}}%
\pgfpathlineto{\pgfqpoint{0.705166in}{0.656263in}}%
\pgfpathlineto{\pgfqpoint{0.710241in}{0.651336in}}%
\pgfpathlineto{\pgfqpoint{0.717595in}{0.644445in}}%
\pgfpathlineto{\pgfqpoint{0.722653in}{0.639784in}}%
\pgfpathlineto{\pgfqpoint{0.730025in}{0.633212in}}%
\pgfpathlineto{\pgfqpoint{0.735719in}{0.628231in}}%
\pgfpathlineto{\pgfqpoint{0.742454in}{0.622520in}}%
\pgfpathlineto{\pgfqpoint{0.749494in}{0.616678in}}%
\pgfpathlineto{\pgfqpoint{0.754883in}{0.612333in}}%
\pgfpathlineto{\pgfqpoint{0.764041in}{0.605126in}}%
\pgfpathlineto{\pgfqpoint{0.767313in}{0.602619in}}%
\pgfpathlineto{\pgfqpoint{0.779435in}{0.593573in}}%
\pgfpathlineto{\pgfqpoint{0.779742in}{0.593350in}}%
\pgfpathlineto{\pgfqpoint{0.792171in}{0.584537in}}%
\pgfpathlineto{\pgfqpoint{0.795835in}{0.582021in}}%
\pgfpathlineto{\pgfqpoint{0.804601in}{0.576128in}}%
\pgfpathlineto{\pgfqpoint{0.813314in}{0.570468in}}%
\pgfpathlineto{\pgfqpoint{0.817030in}{0.568101in}}%
\pgfpathlineto{\pgfqpoint{0.829459in}{0.560460in}}%
\pgfpathlineto{\pgfqpoint{0.832069in}{0.558916in}}%
\pgfpathlineto{\pgfqpoint{0.841888in}{0.553198in}}%
\pgfpathlineto{\pgfqpoint{0.852351in}{0.547363in}}%
\pgfpathlineto{\pgfqpoint{0.854318in}{0.546281in}}%
\pgfpathlineto{\pgfqpoint{0.866747in}{0.539731in}}%
\pgfpathlineto{\pgfqpoint{0.874555in}{0.535811in}}%
\pgfpathlineto{\pgfqpoint{0.879176in}{0.533516in}}%
\pgfpathlineto{\pgfqpoint{0.891606in}{0.527646in}}%
\pgfpathlineto{\pgfqpoint{0.899181in}{0.524258in}}%
\pgfpathlineto{\pgfqpoint{0.904035in}{0.522106in}}%
\pgfpathlineto{\pgfqpoint{0.916464in}{0.516902in}}%
\pgfpathlineto{\pgfqpoint{0.927141in}{0.512705in}}%
\pgfpathlineto{\pgfqpoint{0.928894in}{0.512020in}}%
\pgfpathlineto{\pgfqpoint{0.941323in}{0.507473in}}%
\pgfpathlineto{\pgfqpoint{0.953752in}{0.503247in}}%
\pgfpathlineto{\pgfqpoint{0.960421in}{0.501153in}}%
\pgfpathlineto{\pgfqpoint{0.966181in}{0.499347in}}%
\pgfpathlineto{\pgfqpoint{0.978611in}{0.495774in}}%
\pgfpathlineto{\pgfqpoint{0.991040in}{0.492527in}}%
\pgfpathlineto{\pgfqpoint{1.003469in}{0.489606in}}%
\pgfpathlineto{\pgfqpoint{1.003496in}{0.489600in}}%
\pgfpathclose%
\pgfusepath{fill}%
\end{pgfscope}%
\begin{pgfscope}%
\pgfpathrectangle{\pgfqpoint{0.211875in}{0.211875in}}{\pgfqpoint{1.313625in}{1.279725in}}%
\pgfusepath{clip}%
\pgfsetbuttcap%
\pgfsetroundjoin%
\definecolor{currentfill}{rgb}{0.947270,0.405591,0.279023}%
\pgfsetfillcolor{currentfill}%
\pgfsetlinewidth{0.000000pt}%
\definecolor{currentstroke}{rgb}{0.000000,0.000000,0.000000}%
\pgfsetstrokecolor{currentstroke}%
\pgfsetdash{}{0pt}%
\pgfpathmoveto{\pgfqpoint{1.065616in}{0.535704in}}%
\pgfpathlineto{\pgfqpoint{1.078045in}{0.535216in}}%
\pgfpathlineto{\pgfqpoint{1.090474in}{0.535131in}}%
\pgfpathlineto{\pgfqpoint{1.102904in}{0.535454in}}%
\pgfpathlineto{\pgfqpoint{1.108942in}{0.535811in}}%
\pgfpathlineto{\pgfqpoint{1.115333in}{0.536195in}}%
\pgfpathlineto{\pgfqpoint{1.127762in}{0.537364in}}%
\pgfpathlineto{\pgfqpoint{1.140192in}{0.538963in}}%
\pgfpathlineto{\pgfqpoint{1.152621in}{0.540999in}}%
\pgfpathlineto{\pgfqpoint{1.165050in}{0.543476in}}%
\pgfpathlineto{\pgfqpoint{1.177480in}{0.546403in}}%
\pgfpathlineto{\pgfqpoint{1.181048in}{0.547363in}}%
\pgfpathlineto{\pgfqpoint{1.189909in}{0.549809in}}%
\pgfpathlineto{\pgfqpoint{1.202338in}{0.553690in}}%
\pgfpathlineto{\pgfqpoint{1.214767in}{0.558040in}}%
\pgfpathlineto{\pgfqpoint{1.217056in}{0.558916in}}%
\pgfpathlineto{\pgfqpoint{1.227197in}{0.562911in}}%
\pgfpathlineto{\pgfqpoint{1.239626in}{0.568274in}}%
\pgfpathlineto{\pgfqpoint{1.244345in}{0.570468in}}%
\pgfpathlineto{\pgfqpoint{1.252055in}{0.574170in}}%
\pgfpathlineto{\pgfqpoint{1.264485in}{0.580586in}}%
\pgfpathlineto{\pgfqpoint{1.267108in}{0.582021in}}%
\pgfpathlineto{\pgfqpoint{1.276914in}{0.587573in}}%
\pgfpathlineto{\pgfqpoint{1.286902in}{0.593573in}}%
\pgfpathlineto{\pgfqpoint{1.289343in}{0.595095in}}%
\pgfpathlineto{\pgfqpoint{1.301773in}{0.603206in}}%
\pgfpathlineto{\pgfqpoint{1.304596in}{0.605126in}}%
\pgfpathlineto{\pgfqpoint{1.314202in}{0.611915in}}%
\pgfpathlineto{\pgfqpoint{1.320694in}{0.616678in}}%
\pgfpathlineto{\pgfqpoint{1.326631in}{0.621214in}}%
\pgfpathlineto{\pgfqpoint{1.335543in}{0.628231in}}%
\pgfpathlineto{\pgfqpoint{1.339060in}{0.631119in}}%
\pgfpathlineto{\pgfqpoint{1.349369in}{0.639784in}}%
\pgfpathlineto{\pgfqpoint{1.351490in}{0.641645in}}%
\pgfpathlineto{\pgfqpoint{1.362349in}{0.651336in}}%
\pgfpathlineto{\pgfqpoint{1.363919in}{0.652801in}}%
\pgfpathlineto{\pgfqpoint{1.374620in}{0.662889in}}%
\pgfpathlineto{\pgfqpoint{1.376348in}{0.664594in}}%
\pgfpathlineto{\pgfqpoint{1.386294in}{0.674441in}}%
\pgfpathlineto{\pgfqpoint{1.388778in}{0.677017in}}%
\pgfpathlineto{\pgfqpoint{1.397457in}{0.685994in}}%
\pgfpathlineto{\pgfqpoint{1.401207in}{0.690057in}}%
\pgfpathlineto{\pgfqpoint{1.408183in}{0.697546in}}%
\pgfpathlineto{\pgfqpoint{1.413636in}{0.703681in}}%
\pgfpathlineto{\pgfqpoint{1.418529in}{0.709099in}}%
\pgfpathlineto{\pgfqpoint{1.426066in}{0.717842in}}%
\pgfpathlineto{\pgfqpoint{1.428543in}{0.720651in}}%
\pgfpathlineto{\pgfqpoint{1.438275in}{0.732204in}}%
\pgfpathlineto{\pgfqpoint{1.438495in}{0.732477in}}%
\pgfpathlineto{\pgfqpoint{1.447879in}{0.743757in}}%
\pgfpathlineto{\pgfqpoint{1.450924in}{0.747586in}}%
\pgfpathlineto{\pgfqpoint{1.457305in}{0.755309in}}%
\pgfpathlineto{\pgfqpoint{1.463353in}{0.762956in}}%
\pgfpathlineto{\pgfqpoint{1.466589in}{0.766862in}}%
\pgfpathlineto{\pgfqpoint{1.475768in}{0.778414in}}%
\pgfpathlineto{\pgfqpoint{1.475783in}{0.778433in}}%
\pgfpathlineto{\pgfqpoint{1.485104in}{0.789967in}}%
\pgfpathlineto{\pgfqpoint{1.488212in}{0.793963in}}%
\pgfpathlineto{\pgfqpoint{1.494486in}{0.801519in}}%
\pgfpathlineto{\pgfqpoint{1.500641in}{0.809201in}}%
\pgfpathlineto{\pgfqpoint{1.503988in}{0.813072in}}%
\pgfpathlineto{\pgfqpoint{1.513071in}{0.823920in}}%
\pgfpathlineto{\pgfqpoint{1.513715in}{0.824624in}}%
\pgfpathlineto{\pgfqpoint{1.523979in}{0.836177in}}%
\pgfpathlineto{\pgfqpoint{1.525500in}{0.837936in}}%
\pgfpathlineto{\pgfqpoint{1.525500in}{0.847730in}}%
\pgfpathlineto{\pgfqpoint{1.525500in}{0.859282in}}%
\pgfpathlineto{\pgfqpoint{1.525500in}{0.870835in}}%
\pgfpathlineto{\pgfqpoint{1.525500in}{0.882387in}}%
\pgfpathlineto{\pgfqpoint{1.525500in}{0.893940in}}%
\pgfpathlineto{\pgfqpoint{1.525500in}{0.905492in}}%
\pgfpathlineto{\pgfqpoint{1.525500in}{0.917045in}}%
\pgfpathlineto{\pgfqpoint{1.525500in}{0.928597in}}%
\pgfpathlineto{\pgfqpoint{1.525500in}{0.940150in}}%
\pgfpathlineto{\pgfqpoint{1.525500in}{0.951702in}}%
\pgfpathlineto{\pgfqpoint{1.525500in}{0.963255in}}%
\pgfpathlineto{\pgfqpoint{1.525500in}{0.974808in}}%
\pgfpathlineto{\pgfqpoint{1.525500in}{0.986360in}}%
\pgfpathlineto{\pgfqpoint{1.525500in}{0.997913in}}%
\pgfpathlineto{\pgfqpoint{1.525500in}{1.009465in}}%
\pgfpathlineto{\pgfqpoint{1.525500in}{1.021018in}}%
\pgfpathlineto{\pgfqpoint{1.525500in}{1.032570in}}%
\pgfpathlineto{\pgfqpoint{1.525500in}{1.044123in}}%
\pgfpathlineto{\pgfqpoint{1.525500in}{1.055675in}}%
\pgfpathlineto{\pgfqpoint{1.525500in}{1.067228in}}%
\pgfpathlineto{\pgfqpoint{1.525500in}{1.078781in}}%
\pgfpathlineto{\pgfqpoint{1.525500in}{1.090333in}}%
\pgfpathlineto{\pgfqpoint{1.525500in}{1.092921in}}%
\pgfpathlineto{\pgfqpoint{1.522924in}{1.090333in}}%
\pgfpathlineto{\pgfqpoint{1.513071in}{1.079605in}}%
\pgfpathlineto{\pgfqpoint{1.512453in}{1.078781in}}%
\pgfpathlineto{\pgfqpoint{1.504255in}{1.067228in}}%
\pgfpathlineto{\pgfqpoint{1.500641in}{1.061767in}}%
\pgfpathlineto{\pgfqpoint{1.497201in}{1.055675in}}%
\pgfpathlineto{\pgfqpoint{1.491000in}{1.044123in}}%
\pgfpathlineto{\pgfqpoint{1.488212in}{1.038725in}}%
\pgfpathlineto{\pgfqpoint{1.485415in}{1.032570in}}%
\pgfpathlineto{\pgfqpoint{1.480287in}{1.021018in}}%
\pgfpathlineto{\pgfqpoint{1.475783in}{1.010576in}}%
\pgfpathlineto{\pgfqpoint{1.475352in}{1.009465in}}%
\pgfpathlineto{\pgfqpoint{1.470805in}{0.997913in}}%
\pgfpathlineto{\pgfqpoint{1.466294in}{0.986360in}}%
\pgfpathlineto{\pgfqpoint{1.463353in}{0.978910in}}%
\pgfpathlineto{\pgfqpoint{1.461866in}{0.974808in}}%
\pgfpathlineto{\pgfqpoint{1.457558in}{0.963255in}}%
\pgfpathlineto{\pgfqpoint{1.453173in}{0.951702in}}%
\pgfpathlineto{\pgfqpoint{1.450924in}{0.945974in}}%
\pgfpathlineto{\pgfqpoint{1.448789in}{0.940150in}}%
\pgfpathlineto{\pgfqpoint{1.444382in}{0.928597in}}%
\pgfpathlineto{\pgfqpoint{1.439821in}{0.917045in}}%
\pgfpathlineto{\pgfqpoint{1.438495in}{0.913844in}}%
\pgfpathlineto{\pgfqpoint{1.435219in}{0.905492in}}%
\pgfpathlineto{\pgfqpoint{1.430476in}{0.893940in}}%
\pgfpathlineto{\pgfqpoint{1.426066in}{0.883678in}}%
\pgfpathlineto{\pgfqpoint{1.425535in}{0.882387in}}%
\pgfpathlineto{\pgfqpoint{1.420511in}{0.870835in}}%
\pgfpathlineto{\pgfqpoint{1.415235in}{0.859282in}}%
\pgfpathlineto{\pgfqpoint{1.413636in}{0.855970in}}%
\pgfpathlineto{\pgfqpoint{1.409794in}{0.847730in}}%
\pgfpathlineto{\pgfqpoint{1.404106in}{0.836177in}}%
\pgfpathlineto{\pgfqpoint{1.401207in}{0.830605in}}%
\pgfpathlineto{\pgfqpoint{1.398175in}{0.824624in}}%
\pgfpathlineto{\pgfqpoint{1.391979in}{0.813072in}}%
\pgfpathlineto{\pgfqpoint{1.388778in}{0.807428in}}%
\pgfpathlineto{\pgfqpoint{1.385486in}{0.801519in}}%
\pgfpathlineto{\pgfqpoint{1.378673in}{0.789967in}}%
\pgfpathlineto{\pgfqpoint{1.376348in}{0.786240in}}%
\pgfpathlineto{\pgfqpoint{1.371518in}{0.778414in}}%
\pgfpathlineto{\pgfqpoint{1.363963in}{0.766862in}}%
\pgfpathlineto{\pgfqpoint{1.363919in}{0.766798in}}%
\pgfpathlineto{\pgfqpoint{1.356013in}{0.755309in}}%
\pgfpathlineto{\pgfqpoint{1.351490in}{0.749101in}}%
\pgfpathlineto{\pgfqpoint{1.347584in}{0.743757in}}%
\pgfpathlineto{\pgfqpoint{1.339060in}{0.732742in}}%
\pgfpathlineto{\pgfqpoint{1.338640in}{0.732204in}}%
\pgfpathlineto{\pgfqpoint{1.329117in}{0.720651in}}%
\pgfpathlineto{\pgfqpoint{1.326631in}{0.717796in}}%
\pgfpathlineto{\pgfqpoint{1.318932in}{0.709099in}}%
\pgfpathlineto{\pgfqpoint{1.314202in}{0.704037in}}%
\pgfpathlineto{\pgfqpoint{1.307994in}{0.697546in}}%
\pgfpathlineto{\pgfqpoint{1.301773in}{0.691376in}}%
\pgfpathlineto{\pgfqpoint{1.296180in}{0.685994in}}%
\pgfpathlineto{\pgfqpoint{1.289343in}{0.679743in}}%
\pgfpathlineto{\pgfqpoint{1.283327in}{0.674441in}}%
\pgfpathlineto{\pgfqpoint{1.276914in}{0.669064in}}%
\pgfpathlineto{\pgfqpoint{1.269215in}{0.662889in}}%
\pgfpathlineto{\pgfqpoint{1.264485in}{0.659272in}}%
\pgfpathlineto{\pgfqpoint{1.253549in}{0.651336in}}%
\pgfpathlineto{\pgfqpoint{1.252055in}{0.650301in}}%
\pgfpathlineto{\pgfqpoint{1.239626in}{0.642158in}}%
\pgfpathlineto{\pgfqpoint{1.235754in}{0.639784in}}%
\pgfpathlineto{\pgfqpoint{1.227197in}{0.634760in}}%
\pgfpathlineto{\pgfqpoint{1.215163in}{0.628231in}}%
\pgfpathlineto{\pgfqpoint{1.214767in}{0.628025in}}%
\pgfpathlineto{\pgfqpoint{1.202338in}{0.622037in}}%
\pgfpathlineto{\pgfqpoint{1.190029in}{0.616678in}}%
\pgfpathlineto{\pgfqpoint{1.189909in}{0.616628in}}%
\pgfpathlineto{\pgfqpoint{1.177480in}{0.611909in}}%
\pgfpathlineto{\pgfqpoint{1.165050in}{0.607748in}}%
\pgfpathlineto{\pgfqpoint{1.156053in}{0.605126in}}%
\pgfpathlineto{\pgfqpoint{1.152621in}{0.604160in}}%
\pgfpathlineto{\pgfqpoint{1.140192in}{0.601164in}}%
\pgfpathlineto{\pgfqpoint{1.127762in}{0.598696in}}%
\pgfpathlineto{\pgfqpoint{1.115333in}{0.596747in}}%
\pgfpathlineto{\pgfqpoint{1.102904in}{0.595312in}}%
\pgfpathlineto{\pgfqpoint{1.090474in}{0.594382in}}%
\pgfpathlineto{\pgfqpoint{1.078045in}{0.593949in}}%
\pgfpathlineto{\pgfqpoint{1.065616in}{0.594006in}}%
\pgfpathlineto{\pgfqpoint{1.053187in}{0.594547in}}%
\pgfpathlineto{\pgfqpoint{1.040757in}{0.595562in}}%
\pgfpathlineto{\pgfqpoint{1.028328in}{0.597046in}}%
\pgfpathlineto{\pgfqpoint{1.015899in}{0.598991in}}%
\pgfpathlineto{\pgfqpoint{1.003469in}{0.601391in}}%
\pgfpathlineto{\pgfqpoint{0.991040in}{0.604238in}}%
\pgfpathlineto{\pgfqpoint{0.987695in}{0.605126in}}%
\pgfpathlineto{\pgfqpoint{0.978611in}{0.607581in}}%
\pgfpathlineto{\pgfqpoint{0.966181in}{0.611389in}}%
\pgfpathlineto{\pgfqpoint{0.953752in}{0.615636in}}%
\pgfpathlineto{\pgfqpoint{0.950986in}{0.616678in}}%
\pgfpathlineto{\pgfqpoint{0.941323in}{0.620404in}}%
\pgfpathlineto{\pgfqpoint{0.928894in}{0.625634in}}%
\pgfpathlineto{\pgfqpoint{0.923193in}{0.628231in}}%
\pgfpathlineto{\pgfqpoint{0.916464in}{0.631376in}}%
\pgfpathlineto{\pgfqpoint{0.904035in}{0.637622in}}%
\pgfpathlineto{\pgfqpoint{0.900010in}{0.639784in}}%
\pgfpathlineto{\pgfqpoint{0.891606in}{0.644427in}}%
\pgfpathlineto{\pgfqpoint{0.879844in}{0.651336in}}%
\pgfpathlineto{\pgfqpoint{0.879176in}{0.651741in}}%
\pgfpathlineto{\pgfqpoint{0.866747in}{0.659708in}}%
\pgfpathlineto{\pgfqpoint{0.862035in}{0.662889in}}%
\pgfpathlineto{\pgfqpoint{0.854318in}{0.668281in}}%
\pgfpathlineto{\pgfqpoint{0.845915in}{0.674441in}}%
\pgfpathlineto{\pgfqpoint{0.841888in}{0.677505in}}%
\pgfpathlineto{\pgfqpoint{0.831214in}{0.685994in}}%
\pgfpathlineto{\pgfqpoint{0.829459in}{0.687446in}}%
\pgfpathlineto{\pgfqpoint{0.817742in}{0.697546in}}%
\pgfpathlineto{\pgfqpoint{0.817030in}{0.698187in}}%
\pgfpathlineto{\pgfqpoint{0.805345in}{0.709099in}}%
\pgfpathlineto{\pgfqpoint{0.804601in}{0.709827in}}%
\pgfpathlineto{\pgfqpoint{0.793898in}{0.720651in}}%
\pgfpathlineto{\pgfqpoint{0.792171in}{0.722486in}}%
\pgfpathlineto{\pgfqpoint{0.783298in}{0.732204in}}%
\pgfpathlineto{\pgfqpoint{0.779742in}{0.736308in}}%
\pgfpathlineto{\pgfqpoint{0.773461in}{0.743757in}}%
\pgfpathlineto{\pgfqpoint{0.767313in}{0.751466in}}%
\pgfpathlineto{\pgfqpoint{0.764321in}{0.755309in}}%
\pgfpathlineto{\pgfqpoint{0.755842in}{0.766862in}}%
\pgfpathlineto{\pgfqpoint{0.754883in}{0.768252in}}%
\pgfpathlineto{\pgfqpoint{0.748018in}{0.778414in}}%
\pgfpathlineto{\pgfqpoint{0.742454in}{0.787220in}}%
\pgfpathlineto{\pgfqpoint{0.740748in}{0.789967in}}%
\pgfpathlineto{\pgfqpoint{0.734062in}{0.801519in}}%
\pgfpathlineto{\pgfqpoint{0.730025in}{0.809041in}}%
\pgfpathlineto{\pgfqpoint{0.727891in}{0.813072in}}%
\pgfpathlineto{\pgfqpoint{0.722249in}{0.824624in}}%
\pgfpathlineto{\pgfqpoint{0.717595in}{0.835008in}}%
\pgfpathlineto{\pgfqpoint{0.717077in}{0.836177in}}%
\pgfpathlineto{\pgfqpoint{0.712411in}{0.847730in}}%
\pgfpathlineto{\pgfqpoint{0.708191in}{0.859282in}}%
\pgfpathlineto{\pgfqpoint{0.705166in}{0.868529in}}%
\pgfpathlineto{\pgfqpoint{0.704417in}{0.870835in}}%
\pgfpathlineto{\pgfqpoint{0.701097in}{0.882387in}}%
\pgfpathlineto{\pgfqpoint{0.698200in}{0.893940in}}%
\pgfpathlineto{\pgfqpoint{0.695718in}{0.905492in}}%
\pgfpathlineto{\pgfqpoint{0.693646in}{0.917045in}}%
\pgfpathlineto{\pgfqpoint{0.692737in}{0.923348in}}%
\pgfpathlineto{\pgfqpoint{0.691980in}{0.928597in}}%
\pgfpathlineto{\pgfqpoint{0.690715in}{0.940150in}}%
\pgfpathlineto{\pgfqpoint{0.689838in}{0.951702in}}%
\pgfpathlineto{\pgfqpoint{0.689343in}{0.963255in}}%
\pgfpathlineto{\pgfqpoint{0.689223in}{0.974808in}}%
\pgfpathlineto{\pgfqpoint{0.689474in}{0.986360in}}%
\pgfpathlineto{\pgfqpoint{0.690087in}{0.997913in}}%
\pgfpathlineto{\pgfqpoint{0.691057in}{1.009465in}}%
\pgfpathlineto{\pgfqpoint{0.692376in}{1.021018in}}%
\pgfpathlineto{\pgfqpoint{0.692737in}{1.023545in}}%
\pgfpathlineto{\pgfqpoint{0.694045in}{1.032570in}}%
\pgfpathlineto{\pgfqpoint{0.696052in}{1.044123in}}%
\pgfpathlineto{\pgfqpoint{0.698386in}{1.055675in}}%
\pgfpathlineto{\pgfqpoint{0.701039in}{1.067228in}}%
\pgfpathlineto{\pgfqpoint{0.704001in}{1.078781in}}%
\pgfpathlineto{\pgfqpoint{0.705166in}{1.082944in}}%
\pgfpathlineto{\pgfqpoint{0.707278in}{1.090333in}}%
\pgfpathlineto{\pgfqpoint{0.710854in}{1.101886in}}%
\pgfpathlineto{\pgfqpoint{0.714708in}{1.113438in}}%
\pgfpathlineto{\pgfqpoint{0.717595in}{1.121573in}}%
\pgfpathlineto{\pgfqpoint{0.718839in}{1.124991in}}%
\pgfpathlineto{\pgfqpoint{0.723254in}{1.136543in}}%
\pgfpathlineto{\pgfqpoint{0.727907in}{1.148096in}}%
\pgfpathlineto{\pgfqpoint{0.730025in}{1.153160in}}%
\pgfpathlineto{\pgfqpoint{0.732818in}{1.159648in}}%
\pgfpathlineto{\pgfqpoint{0.737963in}{1.171201in}}%
\pgfpathlineto{\pgfqpoint{0.742454in}{1.180953in}}%
\pgfpathlineto{\pgfqpoint{0.743310in}{1.182754in}}%
\pgfpathlineto{\pgfqpoint{0.748899in}{1.194306in}}%
\pgfpathlineto{\pgfqpoint{0.754635in}{1.205859in}}%
\pgfpathlineto{\pgfqpoint{0.754883in}{1.206356in}}%
\pgfpathlineto{\pgfqpoint{0.760602in}{1.217411in}}%
\pgfpathlineto{\pgfqpoint{0.766674in}{1.228964in}}%
\pgfpathlineto{\pgfqpoint{0.767313in}{1.230184in}}%
\pgfpathlineto{\pgfqpoint{0.772939in}{1.240516in}}%
\pgfpathlineto{\pgfqpoint{0.779264in}{1.252069in}}%
\pgfpathlineto{\pgfqpoint{0.779742in}{1.252956in}}%
\pgfpathlineto{\pgfqpoint{0.785746in}{1.263621in}}%
\pgfpathlineto{\pgfqpoint{0.792171in}{1.275088in}}%
\pgfpathlineto{\pgfqpoint{0.792221in}{1.275174in}}%
\pgfpathlineto{\pgfqpoint{0.798813in}{1.286727in}}%
\pgfpathlineto{\pgfqpoint{0.804601in}{1.297043in}}%
\pgfpathlineto{\pgfqpoint{0.805329in}{1.298279in}}%
\pgfpathlineto{\pgfqpoint{0.811873in}{1.309832in}}%
\pgfpathlineto{\pgfqpoint{0.817030in}{1.319242in}}%
\pgfpathlineto{\pgfqpoint{0.818266in}{1.321384in}}%
\pgfpathlineto{\pgfqpoint{0.824579in}{1.332937in}}%
\pgfpathlineto{\pgfqpoint{0.829459in}{1.342343in}}%
\pgfpathlineto{\pgfqpoint{0.830635in}{1.344489in}}%
\pgfpathlineto{\pgfqpoint{0.836490in}{1.356042in}}%
\pgfpathlineto{\pgfqpoint{0.841888in}{1.367520in}}%
\pgfpathlineto{\pgfqpoint{0.841925in}{1.367594in}}%
\pgfpathlineto{\pgfqpoint{0.847048in}{1.379147in}}%
\pgfpathlineto{\pgfqpoint{0.851602in}{1.390699in}}%
\pgfpathlineto{\pgfqpoint{0.854318in}{1.398758in}}%
\pgfpathlineto{\pgfqpoint{0.855564in}{1.402252in}}%
\pgfpathlineto{\pgfqpoint{0.858888in}{1.413805in}}%
\pgfpathlineto{\pgfqpoint{0.854318in}{1.413805in}}%
\pgfpathlineto{\pgfqpoint{0.841888in}{1.413805in}}%
\pgfpathlineto{\pgfqpoint{0.829459in}{1.413805in}}%
\pgfpathlineto{\pgfqpoint{0.817030in}{1.413805in}}%
\pgfpathlineto{\pgfqpoint{0.804601in}{1.413805in}}%
\pgfpathlineto{\pgfqpoint{0.792171in}{1.413805in}}%
\pgfpathlineto{\pgfqpoint{0.779742in}{1.413805in}}%
\pgfpathlineto{\pgfqpoint{0.767313in}{1.413805in}}%
\pgfpathlineto{\pgfqpoint{0.754883in}{1.413805in}}%
\pgfpathlineto{\pgfqpoint{0.742454in}{1.413805in}}%
\pgfpathlineto{\pgfqpoint{0.730025in}{1.413805in}}%
\pgfpathlineto{\pgfqpoint{0.717595in}{1.413805in}}%
\pgfpathlineto{\pgfqpoint{0.705166in}{1.413805in}}%
\pgfpathlineto{\pgfqpoint{0.700968in}{1.413805in}}%
\pgfpathlineto{\pgfqpoint{0.702093in}{1.402252in}}%
\pgfpathlineto{\pgfqpoint{0.702492in}{1.390699in}}%
\pgfpathlineto{\pgfqpoint{0.702240in}{1.379147in}}%
\pgfpathlineto{\pgfqpoint{0.701404in}{1.367594in}}%
\pgfpathlineto{\pgfqpoint{0.700049in}{1.356042in}}%
\pgfpathlineto{\pgfqpoint{0.698234in}{1.344489in}}%
\pgfpathlineto{\pgfqpoint{0.696015in}{1.332937in}}%
\pgfpathlineto{\pgfqpoint{0.693444in}{1.321384in}}%
\pgfpathlineto{\pgfqpoint{0.692737in}{1.318591in}}%
\pgfpathlineto{\pgfqpoint{0.690574in}{1.309832in}}%
\pgfpathlineto{\pgfqpoint{0.687450in}{1.298279in}}%
\pgfpathlineto{\pgfqpoint{0.684114in}{1.286727in}}%
\pgfpathlineto{\pgfqpoint{0.680606in}{1.275174in}}%
\pgfpathlineto{\pgfqpoint{0.680308in}{1.274247in}}%
\pgfpathlineto{\pgfqpoint{0.676961in}{1.263621in}}%
\pgfpathlineto{\pgfqpoint{0.673219in}{1.252069in}}%
\pgfpathlineto{\pgfqpoint{0.669413in}{1.240516in}}%
\pgfpathlineto{\pgfqpoint{0.667878in}{1.235952in}}%
\pgfpathlineto{\pgfqpoint{0.665567in}{1.228964in}}%
\pgfpathlineto{\pgfqpoint{0.661714in}{1.217411in}}%
\pgfpathlineto{\pgfqpoint{0.657887in}{1.205859in}}%
\pgfpathlineto{\pgfqpoint{0.655449in}{1.198448in}}%
\pgfpathlineto{\pgfqpoint{0.654104in}{1.194306in}}%
\pgfpathlineto{\pgfqpoint{0.650383in}{1.182754in}}%
\pgfpathlineto{\pgfqpoint{0.646763in}{1.171201in}}%
\pgfpathlineto{\pgfqpoint{0.643265in}{1.159648in}}%
\pgfpathlineto{\pgfqpoint{0.643020in}{1.158816in}}%
\pgfpathlineto{\pgfqpoint{0.639882in}{1.148096in}}%
\pgfpathlineto{\pgfqpoint{0.636659in}{1.136543in}}%
\pgfpathlineto{\pgfqpoint{0.633616in}{1.124991in}}%
\pgfpathlineto{\pgfqpoint{0.630768in}{1.113438in}}%
\pgfpathlineto{\pgfqpoint{0.630590in}{1.112671in}}%
\pgfpathlineto{\pgfqpoint{0.628103in}{1.101886in}}%
\pgfpathlineto{\pgfqpoint{0.625663in}{1.090333in}}%
\pgfpathlineto{\pgfqpoint{0.623462in}{1.078781in}}%
\pgfpathlineto{\pgfqpoint{0.621511in}{1.067228in}}%
\pgfpathlineto{\pgfqpoint{0.619823in}{1.055675in}}%
\pgfpathlineto{\pgfqpoint{0.618406in}{1.044123in}}%
\pgfpathlineto{\pgfqpoint{0.618161in}{1.041661in}}%
\pgfpathlineto{\pgfqpoint{0.617257in}{1.032570in}}%
\pgfpathlineto{\pgfqpoint{0.616399in}{1.021018in}}%
\pgfpathlineto{\pgfqpoint{0.615841in}{1.009465in}}%
\pgfpathlineto{\pgfqpoint{0.615592in}{0.997913in}}%
\pgfpathlineto{\pgfqpoint{0.615659in}{0.986360in}}%
\pgfpathlineto{\pgfqpoint{0.616049in}{0.974808in}}%
\pgfpathlineto{\pgfqpoint{0.616768in}{0.963255in}}%
\pgfpathlineto{\pgfqpoint{0.617823in}{0.951702in}}%
\pgfpathlineto{\pgfqpoint{0.618161in}{0.948924in}}%
\pgfpathlineto{\pgfqpoint{0.619208in}{0.940150in}}%
\pgfpathlineto{\pgfqpoint{0.620933in}{0.928597in}}%
\pgfpathlineto{\pgfqpoint{0.623010in}{0.917045in}}%
\pgfpathlineto{\pgfqpoint{0.625446in}{0.905492in}}%
\pgfpathlineto{\pgfqpoint{0.628247in}{0.893940in}}%
\pgfpathlineto{\pgfqpoint{0.630590in}{0.885417in}}%
\pgfpathlineto{\pgfqpoint{0.631413in}{0.882387in}}%
\pgfpathlineto{\pgfqpoint{0.634936in}{0.870835in}}%
\pgfpathlineto{\pgfqpoint{0.638843in}{0.859282in}}%
\pgfpathlineto{\pgfqpoint{0.643020in}{0.848063in}}%
\pgfpathlineto{\pgfqpoint{0.643143in}{0.847730in}}%
\pgfpathlineto{\pgfqpoint{0.647816in}{0.836177in}}%
\pgfpathlineto{\pgfqpoint{0.652898in}{0.824624in}}%
\pgfpathlineto{\pgfqpoint{0.655449in}{0.819264in}}%
\pgfpathlineto{\pgfqpoint{0.658387in}{0.813072in}}%
\pgfpathlineto{\pgfqpoint{0.664292in}{0.801519in}}%
\pgfpathlineto{\pgfqpoint{0.667878in}{0.794980in}}%
\pgfpathlineto{\pgfqpoint{0.670630in}{0.789967in}}%
\pgfpathlineto{\pgfqpoint{0.677409in}{0.778414in}}%
\pgfpathlineto{\pgfqpoint{0.680308in}{0.773777in}}%
\pgfpathlineto{\pgfqpoint{0.684650in}{0.766862in}}%
\pgfpathlineto{\pgfqpoint{0.692359in}{0.755309in}}%
\pgfpathlineto{\pgfqpoint{0.692737in}{0.754774in}}%
\pgfpathlineto{\pgfqpoint{0.700576in}{0.743757in}}%
\pgfpathlineto{\pgfqpoint{0.705166in}{0.737650in}}%
\pgfpathlineto{\pgfqpoint{0.709306in}{0.732204in}}%
\pgfpathlineto{\pgfqpoint{0.717595in}{0.721856in}}%
\pgfpathlineto{\pgfqpoint{0.718574in}{0.720651in}}%
\pgfpathlineto{\pgfqpoint{0.728427in}{0.709099in}}%
\pgfpathlineto{\pgfqpoint{0.730025in}{0.707310in}}%
\pgfpathlineto{\pgfqpoint{0.738900in}{0.697546in}}%
\pgfpathlineto{\pgfqpoint{0.742454in}{0.693805in}}%
\pgfpathlineto{\pgfqpoint{0.750026in}{0.685994in}}%
\pgfpathlineto{\pgfqpoint{0.754883in}{0.681187in}}%
\pgfpathlineto{\pgfqpoint{0.761859in}{0.674441in}}%
\pgfpathlineto{\pgfqpoint{0.767313in}{0.669369in}}%
\pgfpathlineto{\pgfqpoint{0.774462in}{0.662889in}}%
\pgfpathlineto{\pgfqpoint{0.779742in}{0.658274in}}%
\pgfpathlineto{\pgfqpoint{0.787911in}{0.651336in}}%
\pgfpathlineto{\pgfqpoint{0.792171in}{0.647839in}}%
\pgfpathlineto{\pgfqpoint{0.802297in}{0.639784in}}%
\pgfpathlineto{\pgfqpoint{0.804601in}{0.638008in}}%
\pgfpathlineto{\pgfqpoint{0.817030in}{0.628747in}}%
\pgfpathlineto{\pgfqpoint{0.817748in}{0.628231in}}%
\pgfpathlineto{\pgfqpoint{0.829459in}{0.620054in}}%
\pgfpathlineto{\pgfqpoint{0.834492in}{0.616678in}}%
\pgfpathlineto{\pgfqpoint{0.841888in}{0.611847in}}%
\pgfpathlineto{\pgfqpoint{0.852635in}{0.605126in}}%
\pgfpathlineto{\pgfqpoint{0.854318in}{0.604098in}}%
\pgfpathlineto{\pgfqpoint{0.866747in}{0.596843in}}%
\pgfpathlineto{\pgfqpoint{0.872636in}{0.593573in}}%
\pgfpathlineto{\pgfqpoint{0.879176in}{0.590017in}}%
\pgfpathlineto{\pgfqpoint{0.891606in}{0.583613in}}%
\pgfpathlineto{\pgfqpoint{0.894879in}{0.582021in}}%
\pgfpathlineto{\pgfqpoint{0.904035in}{0.577646in}}%
\pgfpathlineto{\pgfqpoint{0.916464in}{0.572068in}}%
\pgfpathlineto{\pgfqpoint{0.920277in}{0.570468in}}%
\pgfpathlineto{\pgfqpoint{0.928894in}{0.566907in}}%
\pgfpathlineto{\pgfqpoint{0.941323in}{0.562135in}}%
\pgfpathlineto{\pgfqpoint{0.950410in}{0.558916in}}%
\pgfpathlineto{\pgfqpoint{0.953752in}{0.557746in}}%
\pgfpathlineto{\pgfqpoint{0.966181in}{0.553763in}}%
\pgfpathlineto{\pgfqpoint{0.978611in}{0.550152in}}%
\pgfpathlineto{\pgfqpoint{0.989319in}{0.547363in}}%
\pgfpathlineto{\pgfqpoint{0.991040in}{0.546919in}}%
\pgfpathlineto{\pgfqpoint{1.003469in}{0.544090in}}%
\pgfpathlineto{\pgfqpoint{1.015899in}{0.541639in}}%
\pgfpathlineto{\pgfqpoint{1.028328in}{0.539570in}}%
\pgfpathlineto{\pgfqpoint{1.040757in}{0.537889in}}%
\pgfpathlineto{\pgfqpoint{1.053187in}{0.536598in}}%
\pgfpathlineto{\pgfqpoint{1.064136in}{0.535811in}}%
\pgfpathclose%
\pgfusepath{fill}%
\end{pgfscope}%
\begin{pgfscope}%
\pgfpathrectangle{\pgfqpoint{0.211875in}{0.211875in}}{\pgfqpoint{1.313625in}{1.279725in}}%
\pgfusepath{clip}%
\pgfsetbuttcap%
\pgfsetroundjoin%
\definecolor{currentfill}{rgb}{0.961115,0.566634,0.405693}%
\pgfsetfillcolor{currentfill}%
\pgfsetlinewidth{0.000000pt}%
\definecolor{currentstroke}{rgb}{0.000000,0.000000,0.000000}%
\pgfsetstrokecolor{currentstroke}%
\pgfsetdash{}{0pt}%
\pgfpathmoveto{\pgfqpoint{0.991040in}{0.604238in}}%
\pgfpathlineto{\pgfqpoint{1.003469in}{0.601391in}}%
\pgfpathlineto{\pgfqpoint{1.015899in}{0.598991in}}%
\pgfpathlineto{\pgfqpoint{1.028328in}{0.597046in}}%
\pgfpathlineto{\pgfqpoint{1.040757in}{0.595562in}}%
\pgfpathlineto{\pgfqpoint{1.053187in}{0.594547in}}%
\pgfpathlineto{\pgfqpoint{1.065616in}{0.594006in}}%
\pgfpathlineto{\pgfqpoint{1.078045in}{0.593949in}}%
\pgfpathlineto{\pgfqpoint{1.090474in}{0.594382in}}%
\pgfpathlineto{\pgfqpoint{1.102904in}{0.595312in}}%
\pgfpathlineto{\pgfqpoint{1.115333in}{0.596747in}}%
\pgfpathlineto{\pgfqpoint{1.127762in}{0.598696in}}%
\pgfpathlineto{\pgfqpoint{1.140192in}{0.601164in}}%
\pgfpathlineto{\pgfqpoint{1.152621in}{0.604160in}}%
\pgfpathlineto{\pgfqpoint{1.156053in}{0.605126in}}%
\pgfpathlineto{\pgfqpoint{1.165050in}{0.607748in}}%
\pgfpathlineto{\pgfqpoint{1.177480in}{0.611909in}}%
\pgfpathlineto{\pgfqpoint{1.189909in}{0.616628in}}%
\pgfpathlineto{\pgfqpoint{1.190029in}{0.616678in}}%
\pgfpathlineto{\pgfqpoint{1.202338in}{0.622037in}}%
\pgfpathlineto{\pgfqpoint{1.214767in}{0.628025in}}%
\pgfpathlineto{\pgfqpoint{1.215163in}{0.628231in}}%
\pgfpathlineto{\pgfqpoint{1.227197in}{0.634760in}}%
\pgfpathlineto{\pgfqpoint{1.235754in}{0.639784in}}%
\pgfpathlineto{\pgfqpoint{1.239626in}{0.642158in}}%
\pgfpathlineto{\pgfqpoint{1.252055in}{0.650301in}}%
\pgfpathlineto{\pgfqpoint{1.253549in}{0.651336in}}%
\pgfpathlineto{\pgfqpoint{1.264485in}{0.659272in}}%
\pgfpathlineto{\pgfqpoint{1.269215in}{0.662889in}}%
\pgfpathlineto{\pgfqpoint{1.276914in}{0.669064in}}%
\pgfpathlineto{\pgfqpoint{1.283327in}{0.674441in}}%
\pgfpathlineto{\pgfqpoint{1.289343in}{0.679743in}}%
\pgfpathlineto{\pgfqpoint{1.296180in}{0.685994in}}%
\pgfpathlineto{\pgfqpoint{1.301773in}{0.691376in}}%
\pgfpathlineto{\pgfqpoint{1.307994in}{0.697546in}}%
\pgfpathlineto{\pgfqpoint{1.314202in}{0.704037in}}%
\pgfpathlineto{\pgfqpoint{1.318932in}{0.709099in}}%
\pgfpathlineto{\pgfqpoint{1.326631in}{0.717796in}}%
\pgfpathlineto{\pgfqpoint{1.329117in}{0.720651in}}%
\pgfpathlineto{\pgfqpoint{1.338640in}{0.732204in}}%
\pgfpathlineto{\pgfqpoint{1.339060in}{0.732742in}}%
\pgfpathlineto{\pgfqpoint{1.347584in}{0.743757in}}%
\pgfpathlineto{\pgfqpoint{1.351490in}{0.749101in}}%
\pgfpathlineto{\pgfqpoint{1.356013in}{0.755309in}}%
\pgfpathlineto{\pgfqpoint{1.363919in}{0.766798in}}%
\pgfpathlineto{\pgfqpoint{1.363963in}{0.766862in}}%
\pgfpathlineto{\pgfqpoint{1.371518in}{0.778414in}}%
\pgfpathlineto{\pgfqpoint{1.376348in}{0.786240in}}%
\pgfpathlineto{\pgfqpoint{1.378673in}{0.789967in}}%
\pgfpathlineto{\pgfqpoint{1.385486in}{0.801519in}}%
\pgfpathlineto{\pgfqpoint{1.388778in}{0.807428in}}%
\pgfpathlineto{\pgfqpoint{1.391979in}{0.813072in}}%
\pgfpathlineto{\pgfqpoint{1.398175in}{0.824624in}}%
\pgfpathlineto{\pgfqpoint{1.401207in}{0.830605in}}%
\pgfpathlineto{\pgfqpoint{1.404106in}{0.836177in}}%
\pgfpathlineto{\pgfqpoint{1.409794in}{0.847730in}}%
\pgfpathlineto{\pgfqpoint{1.413636in}{0.855970in}}%
\pgfpathlineto{\pgfqpoint{1.415235in}{0.859282in}}%
\pgfpathlineto{\pgfqpoint{1.420511in}{0.870835in}}%
\pgfpathlineto{\pgfqpoint{1.425535in}{0.882387in}}%
\pgfpathlineto{\pgfqpoint{1.426066in}{0.883678in}}%
\pgfpathlineto{\pgfqpoint{1.430476in}{0.893940in}}%
\pgfpathlineto{\pgfqpoint{1.435219in}{0.905492in}}%
\pgfpathlineto{\pgfqpoint{1.438495in}{0.913844in}}%
\pgfpathlineto{\pgfqpoint{1.439821in}{0.917045in}}%
\pgfpathlineto{\pgfqpoint{1.444382in}{0.928597in}}%
\pgfpathlineto{\pgfqpoint{1.448789in}{0.940150in}}%
\pgfpathlineto{\pgfqpoint{1.450924in}{0.945974in}}%
\pgfpathlineto{\pgfqpoint{1.453173in}{0.951702in}}%
\pgfpathlineto{\pgfqpoint{1.457558in}{0.963255in}}%
\pgfpathlineto{\pgfqpoint{1.461866in}{0.974808in}}%
\pgfpathlineto{\pgfqpoint{1.463353in}{0.978910in}}%
\pgfpathlineto{\pgfqpoint{1.466294in}{0.986360in}}%
\pgfpathlineto{\pgfqpoint{1.470805in}{0.997913in}}%
\pgfpathlineto{\pgfqpoint{1.475352in}{1.009465in}}%
\pgfpathlineto{\pgfqpoint{1.475783in}{1.010576in}}%
\pgfpathlineto{\pgfqpoint{1.480287in}{1.021018in}}%
\pgfpathlineto{\pgfqpoint{1.485415in}{1.032570in}}%
\pgfpathlineto{\pgfqpoint{1.488212in}{1.038725in}}%
\pgfpathlineto{\pgfqpoint{1.491000in}{1.044123in}}%
\pgfpathlineto{\pgfqpoint{1.497201in}{1.055675in}}%
\pgfpathlineto{\pgfqpoint{1.500641in}{1.061767in}}%
\pgfpathlineto{\pgfqpoint{1.504255in}{1.067228in}}%
\pgfpathlineto{\pgfqpoint{1.512453in}{1.078781in}}%
\pgfpathlineto{\pgfqpoint{1.513071in}{1.079605in}}%
\pgfpathlineto{\pgfqpoint{1.522924in}{1.090333in}}%
\pgfpathlineto{\pgfqpoint{1.525500in}{1.092921in}}%
\pgfpathlineto{\pgfqpoint{1.525500in}{1.101886in}}%
\pgfpathlineto{\pgfqpoint{1.525500in}{1.113438in}}%
\pgfpathlineto{\pgfqpoint{1.525500in}{1.124991in}}%
\pgfpathlineto{\pgfqpoint{1.525500in}{1.136543in}}%
\pgfpathlineto{\pgfqpoint{1.525500in}{1.148096in}}%
\pgfpathlineto{\pgfqpoint{1.525500in}{1.159648in}}%
\pgfpathlineto{\pgfqpoint{1.525500in}{1.171201in}}%
\pgfpathlineto{\pgfqpoint{1.525500in}{1.182754in}}%
\pgfpathlineto{\pgfqpoint{1.525500in}{1.194306in}}%
\pgfpathlineto{\pgfqpoint{1.525500in}{1.205859in}}%
\pgfpathlineto{\pgfqpoint{1.525500in}{1.217411in}}%
\pgfpathlineto{\pgfqpoint{1.525500in}{1.228964in}}%
\pgfpathlineto{\pgfqpoint{1.525500in}{1.240516in}}%
\pgfpathlineto{\pgfqpoint{1.525500in}{1.252069in}}%
\pgfpathlineto{\pgfqpoint{1.525500in}{1.263621in}}%
\pgfpathlineto{\pgfqpoint{1.525500in}{1.275174in}}%
\pgfpathlineto{\pgfqpoint{1.525500in}{1.286727in}}%
\pgfpathlineto{\pgfqpoint{1.525500in}{1.298279in}}%
\pgfpathlineto{\pgfqpoint{1.525500in}{1.309832in}}%
\pgfpathlineto{\pgfqpoint{1.525500in}{1.321384in}}%
\pgfpathlineto{\pgfqpoint{1.525500in}{1.332937in}}%
\pgfpathlineto{\pgfqpoint{1.525500in}{1.344489in}}%
\pgfpathlineto{\pgfqpoint{1.525500in}{1.356042in}}%
\pgfpathlineto{\pgfqpoint{1.525500in}{1.356979in}}%
\pgfpathlineto{\pgfqpoint{1.513071in}{1.362421in}}%
\pgfpathlineto{\pgfqpoint{1.500836in}{1.367594in}}%
\pgfpathlineto{\pgfqpoint{1.500641in}{1.367677in}}%
\pgfpathlineto{\pgfqpoint{1.488212in}{1.372576in}}%
\pgfpathlineto{\pgfqpoint{1.475783in}{1.377307in}}%
\pgfpathlineto{\pgfqpoint{1.470651in}{1.379147in}}%
\pgfpathlineto{\pgfqpoint{1.463353in}{1.381773in}}%
\pgfpathlineto{\pgfqpoint{1.450924in}{1.386014in}}%
\pgfpathlineto{\pgfqpoint{1.438495in}{1.390128in}}%
\pgfpathlineto{\pgfqpoint{1.436637in}{1.390699in}}%
\pgfpathlineto{\pgfqpoint{1.426066in}{1.393955in}}%
\pgfpathlineto{\pgfqpoint{1.413636in}{1.397649in}}%
\pgfpathlineto{\pgfqpoint{1.401207in}{1.401271in}}%
\pgfpathlineto{\pgfqpoint{1.397637in}{1.402252in}}%
\pgfpathlineto{\pgfqpoint{1.388778in}{1.404681in}}%
\pgfpathlineto{\pgfqpoint{1.376348in}{1.407995in}}%
\pgfpathlineto{\pgfqpoint{1.363919in}{1.411322in}}%
\pgfpathlineto{\pgfqpoint{1.354662in}{1.413805in}}%
\pgfpathlineto{\pgfqpoint{1.351490in}{1.413805in}}%
\pgfpathlineto{\pgfqpoint{1.339060in}{1.413805in}}%
\pgfpathlineto{\pgfqpoint{1.326631in}{1.413805in}}%
\pgfpathlineto{\pgfqpoint{1.314202in}{1.413805in}}%
\pgfpathlineto{\pgfqpoint{1.301773in}{1.413805in}}%
\pgfpathlineto{\pgfqpoint{1.289343in}{1.413805in}}%
\pgfpathlineto{\pgfqpoint{1.276914in}{1.413805in}}%
\pgfpathlineto{\pgfqpoint{1.264485in}{1.413805in}}%
\pgfpathlineto{\pgfqpoint{1.252055in}{1.413805in}}%
\pgfpathlineto{\pgfqpoint{1.239626in}{1.413805in}}%
\pgfpathlineto{\pgfqpoint{1.227197in}{1.413805in}}%
\pgfpathlineto{\pgfqpoint{1.214767in}{1.413805in}}%
\pgfpathlineto{\pgfqpoint{1.202338in}{1.413805in}}%
\pgfpathlineto{\pgfqpoint{1.189909in}{1.413805in}}%
\pgfpathlineto{\pgfqpoint{1.177480in}{1.413805in}}%
\pgfpathlineto{\pgfqpoint{1.165050in}{1.413805in}}%
\pgfpathlineto{\pgfqpoint{1.152621in}{1.413805in}}%
\pgfpathlineto{\pgfqpoint{1.140192in}{1.413805in}}%
\pgfpathlineto{\pgfqpoint{1.127762in}{1.413805in}}%
\pgfpathlineto{\pgfqpoint{1.115333in}{1.413805in}}%
\pgfpathlineto{\pgfqpoint{1.102904in}{1.413805in}}%
\pgfpathlineto{\pgfqpoint{1.090474in}{1.413805in}}%
\pgfpathlineto{\pgfqpoint{1.078045in}{1.413805in}}%
\pgfpathlineto{\pgfqpoint{1.065616in}{1.413805in}}%
\pgfpathlineto{\pgfqpoint{1.053187in}{1.413805in}}%
\pgfpathlineto{\pgfqpoint{1.040757in}{1.413805in}}%
\pgfpathlineto{\pgfqpoint{1.028328in}{1.413805in}}%
\pgfpathlineto{\pgfqpoint{1.015899in}{1.413805in}}%
\pgfpathlineto{\pgfqpoint{1.003469in}{1.413805in}}%
\pgfpathlineto{\pgfqpoint{0.991040in}{1.413805in}}%
\pgfpathlineto{\pgfqpoint{0.978611in}{1.413805in}}%
\pgfpathlineto{\pgfqpoint{0.966181in}{1.413805in}}%
\pgfpathlineto{\pgfqpoint{0.953752in}{1.413805in}}%
\pgfpathlineto{\pgfqpoint{0.941323in}{1.413805in}}%
\pgfpathlineto{\pgfqpoint{0.928894in}{1.413805in}}%
\pgfpathlineto{\pgfqpoint{0.916464in}{1.413805in}}%
\pgfpathlineto{\pgfqpoint{0.904035in}{1.413805in}}%
\pgfpathlineto{\pgfqpoint{0.891606in}{1.413805in}}%
\pgfpathlineto{\pgfqpoint{0.879176in}{1.413805in}}%
\pgfpathlineto{\pgfqpoint{0.866747in}{1.413805in}}%
\pgfpathlineto{\pgfqpoint{0.858888in}{1.413805in}}%
\pgfpathlineto{\pgfqpoint{0.855564in}{1.402252in}}%
\pgfpathlineto{\pgfqpoint{0.854318in}{1.398758in}}%
\pgfpathlineto{\pgfqpoint{0.851602in}{1.390699in}}%
\pgfpathlineto{\pgfqpoint{0.847048in}{1.379147in}}%
\pgfpathlineto{\pgfqpoint{0.841925in}{1.367594in}}%
\pgfpathlineto{\pgfqpoint{0.841888in}{1.367520in}}%
\pgfpathlineto{\pgfqpoint{0.836490in}{1.356042in}}%
\pgfpathlineto{\pgfqpoint{0.830635in}{1.344489in}}%
\pgfpathlineto{\pgfqpoint{0.829459in}{1.342343in}}%
\pgfpathlineto{\pgfqpoint{0.824579in}{1.332937in}}%
\pgfpathlineto{\pgfqpoint{0.818266in}{1.321384in}}%
\pgfpathlineto{\pgfqpoint{0.817030in}{1.319242in}}%
\pgfpathlineto{\pgfqpoint{0.811873in}{1.309832in}}%
\pgfpathlineto{\pgfqpoint{0.805329in}{1.298279in}}%
\pgfpathlineto{\pgfqpoint{0.804601in}{1.297043in}}%
\pgfpathlineto{\pgfqpoint{0.798813in}{1.286727in}}%
\pgfpathlineto{\pgfqpoint{0.792221in}{1.275174in}}%
\pgfpathlineto{\pgfqpoint{0.792171in}{1.275088in}}%
\pgfpathlineto{\pgfqpoint{0.785746in}{1.263621in}}%
\pgfpathlineto{\pgfqpoint{0.779742in}{1.252956in}}%
\pgfpathlineto{\pgfqpoint{0.779264in}{1.252069in}}%
\pgfpathlineto{\pgfqpoint{0.772939in}{1.240516in}}%
\pgfpathlineto{\pgfqpoint{0.767313in}{1.230184in}}%
\pgfpathlineto{\pgfqpoint{0.766674in}{1.228964in}}%
\pgfpathlineto{\pgfqpoint{0.760602in}{1.217411in}}%
\pgfpathlineto{\pgfqpoint{0.754883in}{1.206356in}}%
\pgfpathlineto{\pgfqpoint{0.754635in}{1.205859in}}%
\pgfpathlineto{\pgfqpoint{0.748899in}{1.194306in}}%
\pgfpathlineto{\pgfqpoint{0.743310in}{1.182754in}}%
\pgfpathlineto{\pgfqpoint{0.742454in}{1.180953in}}%
\pgfpathlineto{\pgfqpoint{0.737963in}{1.171201in}}%
\pgfpathlineto{\pgfqpoint{0.732818in}{1.159648in}}%
\pgfpathlineto{\pgfqpoint{0.730025in}{1.153160in}}%
\pgfpathlineto{\pgfqpoint{0.727907in}{1.148096in}}%
\pgfpathlineto{\pgfqpoint{0.723254in}{1.136543in}}%
\pgfpathlineto{\pgfqpoint{0.718839in}{1.124991in}}%
\pgfpathlineto{\pgfqpoint{0.717595in}{1.121573in}}%
\pgfpathlineto{\pgfqpoint{0.714708in}{1.113438in}}%
\pgfpathlineto{\pgfqpoint{0.710854in}{1.101886in}}%
\pgfpathlineto{\pgfqpoint{0.707278in}{1.090333in}}%
\pgfpathlineto{\pgfqpoint{0.705166in}{1.082944in}}%
\pgfpathlineto{\pgfqpoint{0.704001in}{1.078781in}}%
\pgfpathlineto{\pgfqpoint{0.701039in}{1.067228in}}%
\pgfpathlineto{\pgfqpoint{0.698386in}{1.055675in}}%
\pgfpathlineto{\pgfqpoint{0.696052in}{1.044123in}}%
\pgfpathlineto{\pgfqpoint{0.694045in}{1.032570in}}%
\pgfpathlineto{\pgfqpoint{0.692737in}{1.023545in}}%
\pgfpathlineto{\pgfqpoint{0.692376in}{1.021018in}}%
\pgfpathlineto{\pgfqpoint{0.691057in}{1.009465in}}%
\pgfpathlineto{\pgfqpoint{0.690087in}{0.997913in}}%
\pgfpathlineto{\pgfqpoint{0.689474in}{0.986360in}}%
\pgfpathlineto{\pgfqpoint{0.689223in}{0.974808in}}%
\pgfpathlineto{\pgfqpoint{0.689343in}{0.963255in}}%
\pgfpathlineto{\pgfqpoint{0.689838in}{0.951702in}}%
\pgfpathlineto{\pgfqpoint{0.690715in}{0.940150in}}%
\pgfpathlineto{\pgfqpoint{0.691980in}{0.928597in}}%
\pgfpathlineto{\pgfqpoint{0.692737in}{0.923348in}}%
\pgfpathlineto{\pgfqpoint{0.693646in}{0.917045in}}%
\pgfpathlineto{\pgfqpoint{0.695718in}{0.905492in}}%
\pgfpathlineto{\pgfqpoint{0.698200in}{0.893940in}}%
\pgfpathlineto{\pgfqpoint{0.701097in}{0.882387in}}%
\pgfpathlineto{\pgfqpoint{0.704417in}{0.870835in}}%
\pgfpathlineto{\pgfqpoint{0.705166in}{0.868529in}}%
\pgfpathlineto{\pgfqpoint{0.708191in}{0.859282in}}%
\pgfpathlineto{\pgfqpoint{0.712411in}{0.847730in}}%
\pgfpathlineto{\pgfqpoint{0.717077in}{0.836177in}}%
\pgfpathlineto{\pgfqpoint{0.717595in}{0.835008in}}%
\pgfpathlineto{\pgfqpoint{0.722249in}{0.824624in}}%
\pgfpathlineto{\pgfqpoint{0.727891in}{0.813072in}}%
\pgfpathlineto{\pgfqpoint{0.730025in}{0.809041in}}%
\pgfpathlineto{\pgfqpoint{0.734062in}{0.801519in}}%
\pgfpathlineto{\pgfqpoint{0.740748in}{0.789967in}}%
\pgfpathlineto{\pgfqpoint{0.742454in}{0.787220in}}%
\pgfpathlineto{\pgfqpoint{0.748018in}{0.778414in}}%
\pgfpathlineto{\pgfqpoint{0.754883in}{0.768252in}}%
\pgfpathlineto{\pgfqpoint{0.755842in}{0.766862in}}%
\pgfpathlineto{\pgfqpoint{0.764321in}{0.755309in}}%
\pgfpathlineto{\pgfqpoint{0.767313in}{0.751466in}}%
\pgfpathlineto{\pgfqpoint{0.773461in}{0.743757in}}%
\pgfpathlineto{\pgfqpoint{0.779742in}{0.736308in}}%
\pgfpathlineto{\pgfqpoint{0.783298in}{0.732204in}}%
\pgfpathlineto{\pgfqpoint{0.792171in}{0.722486in}}%
\pgfpathlineto{\pgfqpoint{0.793898in}{0.720651in}}%
\pgfpathlineto{\pgfqpoint{0.804601in}{0.709827in}}%
\pgfpathlineto{\pgfqpoint{0.805345in}{0.709099in}}%
\pgfpathlineto{\pgfqpoint{0.817030in}{0.698187in}}%
\pgfpathlineto{\pgfqpoint{0.817742in}{0.697546in}}%
\pgfpathlineto{\pgfqpoint{0.829459in}{0.687446in}}%
\pgfpathlineto{\pgfqpoint{0.831214in}{0.685994in}}%
\pgfpathlineto{\pgfqpoint{0.841888in}{0.677505in}}%
\pgfpathlineto{\pgfqpoint{0.845915in}{0.674441in}}%
\pgfpathlineto{\pgfqpoint{0.854318in}{0.668281in}}%
\pgfpathlineto{\pgfqpoint{0.862035in}{0.662889in}}%
\pgfpathlineto{\pgfqpoint{0.866747in}{0.659708in}}%
\pgfpathlineto{\pgfqpoint{0.879176in}{0.651741in}}%
\pgfpathlineto{\pgfqpoint{0.879844in}{0.651336in}}%
\pgfpathlineto{\pgfqpoint{0.891606in}{0.644427in}}%
\pgfpathlineto{\pgfqpoint{0.900010in}{0.639784in}}%
\pgfpathlineto{\pgfqpoint{0.904035in}{0.637622in}}%
\pgfpathlineto{\pgfqpoint{0.916464in}{0.631376in}}%
\pgfpathlineto{\pgfqpoint{0.923193in}{0.628231in}}%
\pgfpathlineto{\pgfqpoint{0.928894in}{0.625634in}}%
\pgfpathlineto{\pgfqpoint{0.941323in}{0.620404in}}%
\pgfpathlineto{\pgfqpoint{0.950986in}{0.616678in}}%
\pgfpathlineto{\pgfqpoint{0.953752in}{0.615636in}}%
\pgfpathlineto{\pgfqpoint{0.966181in}{0.611389in}}%
\pgfpathlineto{\pgfqpoint{0.978611in}{0.607581in}}%
\pgfpathlineto{\pgfqpoint{0.987695in}{0.605126in}}%
\pgfpathclose%
\pgfpathmoveto{\pgfqpoint{1.017737in}{0.662889in}}%
\pgfpathlineto{\pgfqpoint{1.015899in}{0.663172in}}%
\pgfpathlineto{\pgfqpoint{1.003469in}{0.665678in}}%
\pgfpathlineto{\pgfqpoint{0.991040in}{0.668755in}}%
\pgfpathlineto{\pgfqpoint{0.978611in}{0.672395in}}%
\pgfpathlineto{\pgfqpoint{0.972554in}{0.674441in}}%
\pgfpathlineto{\pgfqpoint{0.966181in}{0.676669in}}%
\pgfpathlineto{\pgfqpoint{0.953752in}{0.681583in}}%
\pgfpathlineto{\pgfqpoint{0.943727in}{0.685994in}}%
\pgfpathlineto{\pgfqpoint{0.941323in}{0.687092in}}%
\pgfpathlineto{\pgfqpoint{0.928894in}{0.693344in}}%
\pgfpathlineto{\pgfqpoint{0.921223in}{0.697546in}}%
\pgfpathlineto{\pgfqpoint{0.916464in}{0.700262in}}%
\pgfpathlineto{\pgfqpoint{0.904035in}{0.707927in}}%
\pgfpathlineto{\pgfqpoint{0.902264in}{0.709099in}}%
\pgfpathlineto{\pgfqpoint{0.891606in}{0.716472in}}%
\pgfpathlineto{\pgfqpoint{0.885938in}{0.720651in}}%
\pgfpathlineto{\pgfqpoint{0.879176in}{0.725883in}}%
\pgfpathlineto{\pgfqpoint{0.871470in}{0.732204in}}%
\pgfpathlineto{\pgfqpoint{0.866747in}{0.736284in}}%
\pgfpathlineto{\pgfqpoint{0.858546in}{0.743757in}}%
\pgfpathlineto{\pgfqpoint{0.854318in}{0.747829in}}%
\pgfpathlineto{\pgfqpoint{0.846922in}{0.755309in}}%
\pgfpathlineto{\pgfqpoint{0.841888in}{0.760712in}}%
\pgfpathlineto{\pgfqpoint{0.836409in}{0.766862in}}%
\pgfpathlineto{\pgfqpoint{0.829459in}{0.775175in}}%
\pgfpathlineto{\pgfqpoint{0.826858in}{0.778414in}}%
\pgfpathlineto{\pgfqpoint{0.818197in}{0.789967in}}%
\pgfpathlineto{\pgfqpoint{0.817030in}{0.791642in}}%
\pgfpathlineto{\pgfqpoint{0.810397in}{0.801519in}}%
\pgfpathlineto{\pgfqpoint{0.804601in}{0.810860in}}%
\pgfpathlineto{\pgfqpoint{0.803272in}{0.813072in}}%
\pgfpathlineto{\pgfqpoint{0.796905in}{0.824624in}}%
\pgfpathlineto{\pgfqpoint{0.792171in}{0.834046in}}%
\pgfpathlineto{\pgfqpoint{0.791131in}{0.836177in}}%
\pgfpathlineto{\pgfqpoint{0.786041in}{0.847730in}}%
\pgfpathlineto{\pgfqpoint{0.781486in}{0.859282in}}%
\pgfpathlineto{\pgfqpoint{0.779742in}{0.864298in}}%
\pgfpathlineto{\pgfqpoint{0.777524in}{0.870835in}}%
\pgfpathlineto{\pgfqpoint{0.774122in}{0.882387in}}%
\pgfpathlineto{\pgfqpoint{0.771226in}{0.893940in}}%
\pgfpathlineto{\pgfqpoint{0.768830in}{0.905492in}}%
\pgfpathlineto{\pgfqpoint{0.767313in}{0.914728in}}%
\pgfpathlineto{\pgfqpoint{0.766939in}{0.917045in}}%
\pgfpathlineto{\pgfqpoint{0.765564in}{0.928597in}}%
\pgfpathlineto{\pgfqpoint{0.764660in}{0.940150in}}%
\pgfpathlineto{\pgfqpoint{0.764222in}{0.951702in}}%
\pgfpathlineto{\pgfqpoint{0.764245in}{0.963255in}}%
\pgfpathlineto{\pgfqpoint{0.764721in}{0.974808in}}%
\pgfpathlineto{\pgfqpoint{0.765645in}{0.986360in}}%
\pgfpathlineto{\pgfqpoint{0.767010in}{0.997913in}}%
\pgfpathlineto{\pgfqpoint{0.767313in}{0.999873in}}%
\pgfpathlineto{\pgfqpoint{0.768845in}{1.009465in}}%
\pgfpathlineto{\pgfqpoint{0.771125in}{1.021018in}}%
\pgfpathlineto{\pgfqpoint{0.773833in}{1.032570in}}%
\pgfpathlineto{\pgfqpoint{0.776959in}{1.044123in}}%
\pgfpathlineto{\pgfqpoint{0.779742in}{1.053233in}}%
\pgfpathlineto{\pgfqpoint{0.780517in}{1.055675in}}%
\pgfpathlineto{\pgfqpoint{0.784565in}{1.067228in}}%
\pgfpathlineto{\pgfqpoint{0.789012in}{1.078781in}}%
\pgfpathlineto{\pgfqpoint{0.792171in}{1.086362in}}%
\pgfpathlineto{\pgfqpoint{0.793899in}{1.090333in}}%
\pgfpathlineto{\pgfqpoint{0.799272in}{1.101886in}}%
\pgfpathlineto{\pgfqpoint{0.804601in}{1.112613in}}%
\pgfpathlineto{\pgfqpoint{0.805031in}{1.113438in}}%
\pgfpathlineto{\pgfqpoint{0.811344in}{1.124991in}}%
\pgfpathlineto{\pgfqpoint{0.817030in}{1.134869in}}%
\pgfpathlineto{\pgfqpoint{0.818044in}{1.136543in}}%
\pgfpathlineto{\pgfqpoint{0.825308in}{1.148096in}}%
\pgfpathlineto{\pgfqpoint{0.829459in}{1.154459in}}%
\pgfpathlineto{\pgfqpoint{0.833038in}{1.159648in}}%
\pgfpathlineto{\pgfqpoint{0.841253in}{1.171201in}}%
\pgfpathlineto{\pgfqpoint{0.841888in}{1.172076in}}%
\pgfpathlineto{\pgfqpoint{0.850116in}{1.182754in}}%
\pgfpathlineto{\pgfqpoint{0.854318in}{1.188097in}}%
\pgfpathlineto{\pgfqpoint{0.859522in}{1.194306in}}%
\pgfpathlineto{\pgfqpoint{0.866747in}{1.202807in}}%
\pgfpathlineto{\pgfqpoint{0.869523in}{1.205859in}}%
\pgfpathlineto{\pgfqpoint{0.879176in}{1.216386in}}%
\pgfpathlineto{\pgfqpoint{0.880187in}{1.217411in}}%
\pgfpathlineto{\pgfqpoint{0.891594in}{1.228964in}}%
\pgfpathlineto{\pgfqpoint{0.891606in}{1.228975in}}%
\pgfpathlineto{\pgfqpoint{0.903855in}{1.240516in}}%
\pgfpathlineto{\pgfqpoint{0.904035in}{1.240687in}}%
\pgfpathlineto{\pgfqpoint{0.916464in}{1.251598in}}%
\pgfpathlineto{\pgfqpoint{0.917048in}{1.252069in}}%
\pgfpathlineto{\pgfqpoint{0.928894in}{1.261782in}}%
\pgfpathlineto{\pgfqpoint{0.931348in}{1.263621in}}%
\pgfpathlineto{\pgfqpoint{0.941323in}{1.271281in}}%
\pgfpathlineto{\pgfqpoint{0.946891in}{1.275174in}}%
\pgfpathlineto{\pgfqpoint{0.953752in}{1.280124in}}%
\pgfpathlineto{\pgfqpoint{0.963841in}{1.286727in}}%
\pgfpathlineto{\pgfqpoint{0.966181in}{1.288319in}}%
\pgfpathlineto{\pgfqpoint{0.978611in}{1.295967in}}%
\pgfpathlineto{\pgfqpoint{0.982794in}{1.298279in}}%
\pgfpathlineto{\pgfqpoint{0.991040in}{1.303071in}}%
\pgfpathlineto{\pgfqpoint{1.003469in}{1.309542in}}%
\pgfpathlineto{\pgfqpoint{1.004097in}{1.309832in}}%
\pgfpathlineto{\pgfqpoint{1.015899in}{1.315625in}}%
\pgfpathlineto{\pgfqpoint{1.028328in}{1.321041in}}%
\pgfpathlineto{\pgfqpoint{1.029236in}{1.321384in}}%
\pgfpathlineto{\pgfqpoint{1.040757in}{1.326103in}}%
\pgfpathlineto{\pgfqpoint{1.053187in}{1.330546in}}%
\pgfpathlineto{\pgfqpoint{1.061020in}{1.332937in}}%
\pgfpathlineto{\pgfqpoint{1.065616in}{1.334487in}}%
\pgfpathlineto{\pgfqpoint{1.078045in}{1.337986in}}%
\pgfpathlineto{\pgfqpoint{1.090474in}{1.340874in}}%
\pgfpathlineto{\pgfqpoint{1.102904in}{1.343153in}}%
\pgfpathlineto{\pgfqpoint{1.112987in}{1.344489in}}%
\pgfpathlineto{\pgfqpoint{1.115333in}{1.344849in}}%
\pgfpathlineto{\pgfqpoint{1.127762in}{1.345987in}}%
\pgfpathlineto{\pgfqpoint{1.140192in}{1.346380in}}%
\pgfpathlineto{\pgfqpoint{1.152621in}{1.345942in}}%
\pgfpathlineto{\pgfqpoint{1.165050in}{1.344538in}}%
\pgfpathlineto{\pgfqpoint{1.165308in}{1.344489in}}%
\pgfpathlineto{\pgfqpoint{1.177480in}{1.342281in}}%
\pgfpathlineto{\pgfqpoint{1.189909in}{1.338789in}}%
\pgfpathlineto{\pgfqpoint{1.202338in}{1.333674in}}%
\pgfpathlineto{\pgfqpoint{1.203743in}{1.332937in}}%
\pgfpathlineto{\pgfqpoint{1.214767in}{1.327268in}}%
\pgfpathlineto{\pgfqpoint{1.223418in}{1.321384in}}%
\pgfpathlineto{\pgfqpoint{1.227197in}{1.318832in}}%
\pgfpathlineto{\pgfqpoint{1.237929in}{1.309832in}}%
\pgfpathlineto{\pgfqpoint{1.239626in}{1.308399in}}%
\pgfpathlineto{\pgfqpoint{1.249691in}{1.298279in}}%
\pgfpathlineto{\pgfqpoint{1.252055in}{1.295855in}}%
\pgfpathlineto{\pgfqpoint{1.259751in}{1.286727in}}%
\pgfpathlineto{\pgfqpoint{1.264485in}{1.280925in}}%
\pgfpathlineto{\pgfqpoint{1.268631in}{1.275174in}}%
\pgfpathlineto{\pgfqpoint{1.276606in}{1.263621in}}%
\pgfpathlineto{\pgfqpoint{1.276914in}{1.263174in}}%
\pgfpathlineto{\pgfqpoint{1.283839in}{1.252069in}}%
\pgfpathlineto{\pgfqpoint{1.289343in}{1.242633in}}%
\pgfpathlineto{\pgfqpoint{1.290477in}{1.240516in}}%
\pgfpathlineto{\pgfqpoint{1.296586in}{1.228964in}}%
\pgfpathlineto{\pgfqpoint{1.301773in}{1.218409in}}%
\pgfpathlineto{\pgfqpoint{1.302230in}{1.217411in}}%
\pgfpathlineto{\pgfqpoint{1.307460in}{1.205859in}}%
\pgfpathlineto{\pgfqpoint{1.312290in}{1.194306in}}%
\pgfpathlineto{\pgfqpoint{1.314202in}{1.189524in}}%
\pgfpathlineto{\pgfqpoint{1.316766in}{1.182754in}}%
\pgfpathlineto{\pgfqpoint{1.320896in}{1.171201in}}%
\pgfpathlineto{\pgfqpoint{1.324686in}{1.159648in}}%
\pgfpathlineto{\pgfqpoint{1.326631in}{1.153311in}}%
\pgfpathlineto{\pgfqpoint{1.328166in}{1.148096in}}%
\pgfpathlineto{\pgfqpoint{1.331344in}{1.136543in}}%
\pgfpathlineto{\pgfqpoint{1.334213in}{1.124991in}}%
\pgfpathlineto{\pgfqpoint{1.336781in}{1.113438in}}%
\pgfpathlineto{\pgfqpoint{1.339057in}{1.101886in}}%
\pgfpathlineto{\pgfqpoint{1.339060in}{1.101866in}}%
\pgfpathlineto{\pgfqpoint{1.341063in}{1.090333in}}%
\pgfpathlineto{\pgfqpoint{1.342781in}{1.078781in}}%
\pgfpathlineto{\pgfqpoint{1.344213in}{1.067228in}}%
\pgfpathlineto{\pgfqpoint{1.345362in}{1.055675in}}%
\pgfpathlineto{\pgfqpoint{1.346230in}{1.044123in}}%
\pgfpathlineto{\pgfqpoint{1.346817in}{1.032570in}}%
\pgfpathlineto{\pgfqpoint{1.347123in}{1.021018in}}%
\pgfpathlineto{\pgfqpoint{1.347146in}{1.009465in}}%
\pgfpathlineto{\pgfqpoint{1.346886in}{0.997913in}}%
\pgfpathlineto{\pgfqpoint{1.346338in}{0.986360in}}%
\pgfpathlineto{\pgfqpoint{1.345500in}{0.974808in}}%
\pgfpathlineto{\pgfqpoint{1.344367in}{0.963255in}}%
\pgfpathlineto{\pgfqpoint{1.342934in}{0.951702in}}%
\pgfpathlineto{\pgfqpoint{1.341195in}{0.940150in}}%
\pgfpathlineto{\pgfqpoint{1.339144in}{0.928597in}}%
\pgfpathlineto{\pgfqpoint{1.339060in}{0.928197in}}%
\pgfpathlineto{\pgfqpoint{1.336776in}{0.917045in}}%
\pgfpathlineto{\pgfqpoint{1.334077in}{0.905492in}}%
\pgfpathlineto{\pgfqpoint{1.331039in}{0.893940in}}%
\pgfpathlineto{\pgfqpoint{1.327649in}{0.882387in}}%
\pgfpathlineto{\pgfqpoint{1.326631in}{0.879273in}}%
\pgfpathlineto{\pgfqpoint{1.323885in}{0.870835in}}%
\pgfpathlineto{\pgfqpoint{1.319737in}{0.859282in}}%
\pgfpathlineto{\pgfqpoint{1.315193in}{0.847730in}}%
\pgfpathlineto{\pgfqpoint{1.314202in}{0.845426in}}%
\pgfpathlineto{\pgfqpoint{1.310201in}{0.836177in}}%
\pgfpathlineto{\pgfqpoint{1.304763in}{0.824624in}}%
\pgfpathlineto{\pgfqpoint{1.301773in}{0.818768in}}%
\pgfpathlineto{\pgfqpoint{1.298826in}{0.813072in}}%
\pgfpathlineto{\pgfqpoint{1.292358in}{0.801519in}}%
\pgfpathlineto{\pgfqpoint{1.289343in}{0.796527in}}%
\pgfpathlineto{\pgfqpoint{1.285298in}{0.789967in}}%
\pgfpathlineto{\pgfqpoint{1.277621in}{0.778414in}}%
\pgfpathlineto{\pgfqpoint{1.276914in}{0.777421in}}%
\pgfpathlineto{\pgfqpoint{1.269168in}{0.766862in}}%
\pgfpathlineto{\pgfqpoint{1.264485in}{0.760910in}}%
\pgfpathlineto{\pgfqpoint{1.259916in}{0.755309in}}%
\pgfpathlineto{\pgfqpoint{1.252055in}{0.746309in}}%
\pgfpathlineto{\pgfqpoint{1.249726in}{0.743757in}}%
\pgfpathlineto{\pgfqpoint{1.239626in}{0.733393in}}%
\pgfpathlineto{\pgfqpoint{1.238406in}{0.732204in}}%
\pgfpathlineto{\pgfqpoint{1.227197in}{0.721956in}}%
\pgfpathlineto{\pgfqpoint{1.225681in}{0.720651in}}%
\pgfpathlineto{\pgfqpoint{1.214767in}{0.711817in}}%
\pgfpathlineto{\pgfqpoint{1.211164in}{0.709099in}}%
\pgfpathlineto{\pgfqpoint{1.202338in}{0.702818in}}%
\pgfpathlineto{\pgfqpoint{1.194299in}{0.697546in}}%
\pgfpathlineto{\pgfqpoint{1.189909in}{0.694823in}}%
\pgfpathlineto{\pgfqpoint{1.177480in}{0.687791in}}%
\pgfpathlineto{\pgfqpoint{1.173960in}{0.685994in}}%
\pgfpathlineto{\pgfqpoint{1.165050in}{0.681678in}}%
\pgfpathlineto{\pgfqpoint{1.152621in}{0.676349in}}%
\pgfpathlineto{\pgfqpoint{1.147523in}{0.674441in}}%
\pgfpathlineto{\pgfqpoint{1.140192in}{0.671829in}}%
\pgfpathlineto{\pgfqpoint{1.127762in}{0.668058in}}%
\pgfpathlineto{\pgfqpoint{1.115333in}{0.664956in}}%
\pgfpathlineto{\pgfqpoint{1.104822in}{0.662889in}}%
\pgfpathlineto{\pgfqpoint{1.102904in}{0.662528in}}%
\pgfpathlineto{\pgfqpoint{1.090474in}{0.660804in}}%
\pgfpathlineto{\pgfqpoint{1.078045in}{0.659700in}}%
\pgfpathlineto{\pgfqpoint{1.065616in}{0.659208in}}%
\pgfpathlineto{\pgfqpoint{1.053187in}{0.659319in}}%
\pgfpathlineto{\pgfqpoint{1.040757in}{0.660021in}}%
\pgfpathlineto{\pgfqpoint{1.028328in}{0.661305in}}%
\pgfpathclose%
\pgfusepath{fill}%
\end{pgfscope}%
\begin{pgfscope}%
\pgfpathrectangle{\pgfqpoint{0.211875in}{0.211875in}}{\pgfqpoint{1.313625in}{1.279725in}}%
\pgfusepath{clip}%
\pgfsetbuttcap%
\pgfsetroundjoin%
\definecolor{currentfill}{rgb}{0.965440,0.720101,0.576404}%
\pgfsetfillcolor{currentfill}%
\pgfsetlinewidth{0.000000pt}%
\definecolor{currentstroke}{rgb}{0.000000,0.000000,0.000000}%
\pgfsetstrokecolor{currentstroke}%
\pgfsetdash{}{0pt}%
\pgfpathmoveto{\pgfqpoint{1.028328in}{0.661305in}}%
\pgfpathlineto{\pgfqpoint{1.040757in}{0.660021in}}%
\pgfpathlineto{\pgfqpoint{1.053187in}{0.659319in}}%
\pgfpathlineto{\pgfqpoint{1.065616in}{0.659208in}}%
\pgfpathlineto{\pgfqpoint{1.078045in}{0.659700in}}%
\pgfpathlineto{\pgfqpoint{1.090474in}{0.660804in}}%
\pgfpathlineto{\pgfqpoint{1.102904in}{0.662528in}}%
\pgfpathlineto{\pgfqpoint{1.104822in}{0.662889in}}%
\pgfpathlineto{\pgfqpoint{1.115333in}{0.664956in}}%
\pgfpathlineto{\pgfqpoint{1.127762in}{0.668058in}}%
\pgfpathlineto{\pgfqpoint{1.140192in}{0.671829in}}%
\pgfpathlineto{\pgfqpoint{1.147523in}{0.674441in}}%
\pgfpathlineto{\pgfqpoint{1.152621in}{0.676349in}}%
\pgfpathlineto{\pgfqpoint{1.165050in}{0.681678in}}%
\pgfpathlineto{\pgfqpoint{1.173960in}{0.685994in}}%
\pgfpathlineto{\pgfqpoint{1.177480in}{0.687791in}}%
\pgfpathlineto{\pgfqpoint{1.189909in}{0.694823in}}%
\pgfpathlineto{\pgfqpoint{1.194299in}{0.697546in}}%
\pgfpathlineto{\pgfqpoint{1.202338in}{0.702818in}}%
\pgfpathlineto{\pgfqpoint{1.211164in}{0.709099in}}%
\pgfpathlineto{\pgfqpoint{1.214767in}{0.711817in}}%
\pgfpathlineto{\pgfqpoint{1.225681in}{0.720651in}}%
\pgfpathlineto{\pgfqpoint{1.227197in}{0.721956in}}%
\pgfpathlineto{\pgfqpoint{1.238406in}{0.732204in}}%
\pgfpathlineto{\pgfqpoint{1.239626in}{0.733393in}}%
\pgfpathlineto{\pgfqpoint{1.249726in}{0.743757in}}%
\pgfpathlineto{\pgfqpoint{1.252055in}{0.746309in}}%
\pgfpathlineto{\pgfqpoint{1.259916in}{0.755309in}}%
\pgfpathlineto{\pgfqpoint{1.264485in}{0.760910in}}%
\pgfpathlineto{\pgfqpoint{1.269168in}{0.766862in}}%
\pgfpathlineto{\pgfqpoint{1.276914in}{0.777421in}}%
\pgfpathlineto{\pgfqpoint{1.277621in}{0.778414in}}%
\pgfpathlineto{\pgfqpoint{1.285298in}{0.789967in}}%
\pgfpathlineto{\pgfqpoint{1.289343in}{0.796527in}}%
\pgfpathlineto{\pgfqpoint{1.292358in}{0.801519in}}%
\pgfpathlineto{\pgfqpoint{1.298826in}{0.813072in}}%
\pgfpathlineto{\pgfqpoint{1.301773in}{0.818768in}}%
\pgfpathlineto{\pgfqpoint{1.304763in}{0.824624in}}%
\pgfpathlineto{\pgfqpoint{1.310201in}{0.836177in}}%
\pgfpathlineto{\pgfqpoint{1.314202in}{0.845426in}}%
\pgfpathlineto{\pgfqpoint{1.315193in}{0.847730in}}%
\pgfpathlineto{\pgfqpoint{1.319737in}{0.859282in}}%
\pgfpathlineto{\pgfqpoint{1.323885in}{0.870835in}}%
\pgfpathlineto{\pgfqpoint{1.326631in}{0.879273in}}%
\pgfpathlineto{\pgfqpoint{1.327649in}{0.882387in}}%
\pgfpathlineto{\pgfqpoint{1.331039in}{0.893940in}}%
\pgfpathlineto{\pgfqpoint{1.334077in}{0.905492in}}%
\pgfpathlineto{\pgfqpoint{1.336776in}{0.917045in}}%
\pgfpathlineto{\pgfqpoint{1.339060in}{0.928197in}}%
\pgfpathlineto{\pgfqpoint{1.339144in}{0.928597in}}%
\pgfpathlineto{\pgfqpoint{1.341195in}{0.940150in}}%
\pgfpathlineto{\pgfqpoint{1.342934in}{0.951702in}}%
\pgfpathlineto{\pgfqpoint{1.344367in}{0.963255in}}%
\pgfpathlineto{\pgfqpoint{1.345500in}{0.974808in}}%
\pgfpathlineto{\pgfqpoint{1.346338in}{0.986360in}}%
\pgfpathlineto{\pgfqpoint{1.346886in}{0.997913in}}%
\pgfpathlineto{\pgfqpoint{1.347146in}{1.009465in}}%
\pgfpathlineto{\pgfqpoint{1.347123in}{1.021018in}}%
\pgfpathlineto{\pgfqpoint{1.346817in}{1.032570in}}%
\pgfpathlineto{\pgfqpoint{1.346230in}{1.044123in}}%
\pgfpathlineto{\pgfqpoint{1.345362in}{1.055675in}}%
\pgfpathlineto{\pgfqpoint{1.344213in}{1.067228in}}%
\pgfpathlineto{\pgfqpoint{1.342781in}{1.078781in}}%
\pgfpathlineto{\pgfqpoint{1.341063in}{1.090333in}}%
\pgfpathlineto{\pgfqpoint{1.339060in}{1.101866in}}%
\pgfpathlineto{\pgfqpoint{1.339057in}{1.101886in}}%
\pgfpathlineto{\pgfqpoint{1.336781in}{1.113438in}}%
\pgfpathlineto{\pgfqpoint{1.334213in}{1.124991in}}%
\pgfpathlineto{\pgfqpoint{1.331344in}{1.136543in}}%
\pgfpathlineto{\pgfqpoint{1.328166in}{1.148096in}}%
\pgfpathlineto{\pgfqpoint{1.326631in}{1.153311in}}%
\pgfpathlineto{\pgfqpoint{1.324686in}{1.159648in}}%
\pgfpathlineto{\pgfqpoint{1.320896in}{1.171201in}}%
\pgfpathlineto{\pgfqpoint{1.316766in}{1.182754in}}%
\pgfpathlineto{\pgfqpoint{1.314202in}{1.189524in}}%
\pgfpathlineto{\pgfqpoint{1.312290in}{1.194306in}}%
\pgfpathlineto{\pgfqpoint{1.307460in}{1.205859in}}%
\pgfpathlineto{\pgfqpoint{1.302230in}{1.217411in}}%
\pgfpathlineto{\pgfqpoint{1.301773in}{1.218409in}}%
\pgfpathlineto{\pgfqpoint{1.296586in}{1.228964in}}%
\pgfpathlineto{\pgfqpoint{1.290477in}{1.240516in}}%
\pgfpathlineto{\pgfqpoint{1.289343in}{1.242633in}}%
\pgfpathlineto{\pgfqpoint{1.283839in}{1.252069in}}%
\pgfpathlineto{\pgfqpoint{1.276914in}{1.263174in}}%
\pgfpathlineto{\pgfqpoint{1.276606in}{1.263621in}}%
\pgfpathlineto{\pgfqpoint{1.268631in}{1.275174in}}%
\pgfpathlineto{\pgfqpoint{1.264485in}{1.280925in}}%
\pgfpathlineto{\pgfqpoint{1.259751in}{1.286727in}}%
\pgfpathlineto{\pgfqpoint{1.252055in}{1.295855in}}%
\pgfpathlineto{\pgfqpoint{1.249691in}{1.298279in}}%
\pgfpathlineto{\pgfqpoint{1.239626in}{1.308399in}}%
\pgfpathlineto{\pgfqpoint{1.237929in}{1.309832in}}%
\pgfpathlineto{\pgfqpoint{1.227197in}{1.318832in}}%
\pgfpathlineto{\pgfqpoint{1.223418in}{1.321384in}}%
\pgfpathlineto{\pgfqpoint{1.214767in}{1.327268in}}%
\pgfpathlineto{\pgfqpoint{1.203743in}{1.332937in}}%
\pgfpathlineto{\pgfqpoint{1.202338in}{1.333674in}}%
\pgfpathlineto{\pgfqpoint{1.189909in}{1.338789in}}%
\pgfpathlineto{\pgfqpoint{1.177480in}{1.342281in}}%
\pgfpathlineto{\pgfqpoint{1.165308in}{1.344489in}}%
\pgfpathlineto{\pgfqpoint{1.165050in}{1.344538in}}%
\pgfpathlineto{\pgfqpoint{1.152621in}{1.345942in}}%
\pgfpathlineto{\pgfqpoint{1.140192in}{1.346380in}}%
\pgfpathlineto{\pgfqpoint{1.127762in}{1.345987in}}%
\pgfpathlineto{\pgfqpoint{1.115333in}{1.344849in}}%
\pgfpathlineto{\pgfqpoint{1.112987in}{1.344489in}}%
\pgfpathlineto{\pgfqpoint{1.102904in}{1.343153in}}%
\pgfpathlineto{\pgfqpoint{1.090474in}{1.340874in}}%
\pgfpathlineto{\pgfqpoint{1.078045in}{1.337986in}}%
\pgfpathlineto{\pgfqpoint{1.065616in}{1.334487in}}%
\pgfpathlineto{\pgfqpoint{1.061020in}{1.332937in}}%
\pgfpathlineto{\pgfqpoint{1.053187in}{1.330546in}}%
\pgfpathlineto{\pgfqpoint{1.040757in}{1.326103in}}%
\pgfpathlineto{\pgfqpoint{1.029236in}{1.321384in}}%
\pgfpathlineto{\pgfqpoint{1.028328in}{1.321041in}}%
\pgfpathlineto{\pgfqpoint{1.015899in}{1.315625in}}%
\pgfpathlineto{\pgfqpoint{1.004097in}{1.309832in}}%
\pgfpathlineto{\pgfqpoint{1.003469in}{1.309542in}}%
\pgfpathlineto{\pgfqpoint{0.991040in}{1.303071in}}%
\pgfpathlineto{\pgfqpoint{0.982794in}{1.298279in}}%
\pgfpathlineto{\pgfqpoint{0.978611in}{1.295967in}}%
\pgfpathlineto{\pgfqpoint{0.966181in}{1.288319in}}%
\pgfpathlineto{\pgfqpoint{0.963841in}{1.286727in}}%
\pgfpathlineto{\pgfqpoint{0.953752in}{1.280124in}}%
\pgfpathlineto{\pgfqpoint{0.946891in}{1.275174in}}%
\pgfpathlineto{\pgfqpoint{0.941323in}{1.271281in}}%
\pgfpathlineto{\pgfqpoint{0.931348in}{1.263621in}}%
\pgfpathlineto{\pgfqpoint{0.928894in}{1.261782in}}%
\pgfpathlineto{\pgfqpoint{0.917048in}{1.252069in}}%
\pgfpathlineto{\pgfqpoint{0.916464in}{1.251598in}}%
\pgfpathlineto{\pgfqpoint{0.904035in}{1.240687in}}%
\pgfpathlineto{\pgfqpoint{0.903855in}{1.240516in}}%
\pgfpathlineto{\pgfqpoint{0.891606in}{1.228975in}}%
\pgfpathlineto{\pgfqpoint{0.891594in}{1.228964in}}%
\pgfpathlineto{\pgfqpoint{0.880187in}{1.217411in}}%
\pgfpathlineto{\pgfqpoint{0.879176in}{1.216386in}}%
\pgfpathlineto{\pgfqpoint{0.869523in}{1.205859in}}%
\pgfpathlineto{\pgfqpoint{0.866747in}{1.202807in}}%
\pgfpathlineto{\pgfqpoint{0.859522in}{1.194306in}}%
\pgfpathlineto{\pgfqpoint{0.854318in}{1.188097in}}%
\pgfpathlineto{\pgfqpoint{0.850116in}{1.182754in}}%
\pgfpathlineto{\pgfqpoint{0.841888in}{1.172076in}}%
\pgfpathlineto{\pgfqpoint{0.841253in}{1.171201in}}%
\pgfpathlineto{\pgfqpoint{0.833038in}{1.159648in}}%
\pgfpathlineto{\pgfqpoint{0.829459in}{1.154459in}}%
\pgfpathlineto{\pgfqpoint{0.825308in}{1.148096in}}%
\pgfpathlineto{\pgfqpoint{0.818044in}{1.136543in}}%
\pgfpathlineto{\pgfqpoint{0.817030in}{1.134869in}}%
\pgfpathlineto{\pgfqpoint{0.811344in}{1.124991in}}%
\pgfpathlineto{\pgfqpoint{0.805031in}{1.113438in}}%
\pgfpathlineto{\pgfqpoint{0.804601in}{1.112613in}}%
\pgfpathlineto{\pgfqpoint{0.799272in}{1.101886in}}%
\pgfpathlineto{\pgfqpoint{0.793899in}{1.090333in}}%
\pgfpathlineto{\pgfqpoint{0.792171in}{1.086362in}}%
\pgfpathlineto{\pgfqpoint{0.789012in}{1.078781in}}%
\pgfpathlineto{\pgfqpoint{0.784565in}{1.067228in}}%
\pgfpathlineto{\pgfqpoint{0.780517in}{1.055675in}}%
\pgfpathlineto{\pgfqpoint{0.779742in}{1.053233in}}%
\pgfpathlineto{\pgfqpoint{0.776959in}{1.044123in}}%
\pgfpathlineto{\pgfqpoint{0.773833in}{1.032570in}}%
\pgfpathlineto{\pgfqpoint{0.771125in}{1.021018in}}%
\pgfpathlineto{\pgfqpoint{0.768845in}{1.009465in}}%
\pgfpathlineto{\pgfqpoint{0.767313in}{0.999873in}}%
\pgfpathlineto{\pgfqpoint{0.767010in}{0.997913in}}%
\pgfpathlineto{\pgfqpoint{0.765645in}{0.986360in}}%
\pgfpathlineto{\pgfqpoint{0.764721in}{0.974808in}}%
\pgfpathlineto{\pgfqpoint{0.764245in}{0.963255in}}%
\pgfpathlineto{\pgfqpoint{0.764222in}{0.951702in}}%
\pgfpathlineto{\pgfqpoint{0.764660in}{0.940150in}}%
\pgfpathlineto{\pgfqpoint{0.765564in}{0.928597in}}%
\pgfpathlineto{\pgfqpoint{0.766939in}{0.917045in}}%
\pgfpathlineto{\pgfqpoint{0.767313in}{0.914728in}}%
\pgfpathlineto{\pgfqpoint{0.768830in}{0.905492in}}%
\pgfpathlineto{\pgfqpoint{0.771226in}{0.893940in}}%
\pgfpathlineto{\pgfqpoint{0.774122in}{0.882387in}}%
\pgfpathlineto{\pgfqpoint{0.777524in}{0.870835in}}%
\pgfpathlineto{\pgfqpoint{0.779742in}{0.864298in}}%
\pgfpathlineto{\pgfqpoint{0.781486in}{0.859282in}}%
\pgfpathlineto{\pgfqpoint{0.786041in}{0.847730in}}%
\pgfpathlineto{\pgfqpoint{0.791131in}{0.836177in}}%
\pgfpathlineto{\pgfqpoint{0.792171in}{0.834046in}}%
\pgfpathlineto{\pgfqpoint{0.796905in}{0.824624in}}%
\pgfpathlineto{\pgfqpoint{0.803272in}{0.813072in}}%
\pgfpathlineto{\pgfqpoint{0.804601in}{0.810860in}}%
\pgfpathlineto{\pgfqpoint{0.810397in}{0.801519in}}%
\pgfpathlineto{\pgfqpoint{0.817030in}{0.791642in}}%
\pgfpathlineto{\pgfqpoint{0.818197in}{0.789967in}}%
\pgfpathlineto{\pgfqpoint{0.826858in}{0.778414in}}%
\pgfpathlineto{\pgfqpoint{0.829459in}{0.775175in}}%
\pgfpathlineto{\pgfqpoint{0.836409in}{0.766862in}}%
\pgfpathlineto{\pgfqpoint{0.841888in}{0.760712in}}%
\pgfpathlineto{\pgfqpoint{0.846922in}{0.755309in}}%
\pgfpathlineto{\pgfqpoint{0.854318in}{0.747829in}}%
\pgfpathlineto{\pgfqpoint{0.858546in}{0.743757in}}%
\pgfpathlineto{\pgfqpoint{0.866747in}{0.736284in}}%
\pgfpathlineto{\pgfqpoint{0.871470in}{0.732204in}}%
\pgfpathlineto{\pgfqpoint{0.879176in}{0.725883in}}%
\pgfpathlineto{\pgfqpoint{0.885938in}{0.720651in}}%
\pgfpathlineto{\pgfqpoint{0.891606in}{0.716472in}}%
\pgfpathlineto{\pgfqpoint{0.902264in}{0.709099in}}%
\pgfpathlineto{\pgfqpoint{0.904035in}{0.707927in}}%
\pgfpathlineto{\pgfqpoint{0.916464in}{0.700262in}}%
\pgfpathlineto{\pgfqpoint{0.921223in}{0.697546in}}%
\pgfpathlineto{\pgfqpoint{0.928894in}{0.693344in}}%
\pgfpathlineto{\pgfqpoint{0.941323in}{0.687092in}}%
\pgfpathlineto{\pgfqpoint{0.943727in}{0.685994in}}%
\pgfpathlineto{\pgfqpoint{0.953752in}{0.681583in}}%
\pgfpathlineto{\pgfqpoint{0.966181in}{0.676669in}}%
\pgfpathlineto{\pgfqpoint{0.972554in}{0.674441in}}%
\pgfpathlineto{\pgfqpoint{0.978611in}{0.672395in}}%
\pgfpathlineto{\pgfqpoint{0.991040in}{0.668755in}}%
\pgfpathlineto{\pgfqpoint{1.003469in}{0.665678in}}%
\pgfpathlineto{\pgfqpoint{1.015899in}{0.663172in}}%
\pgfpathlineto{\pgfqpoint{1.017737in}{0.662889in}}%
\pgfpathclose%
\pgfpathmoveto{\pgfqpoint{1.017665in}{0.743757in}}%
\pgfpathlineto{\pgfqpoint{1.015899in}{0.744050in}}%
\pgfpathlineto{\pgfqpoint{1.003469in}{0.746991in}}%
\pgfpathlineto{\pgfqpoint{0.991040in}{0.750778in}}%
\pgfpathlineto{\pgfqpoint{0.978835in}{0.755309in}}%
\pgfpathlineto{\pgfqpoint{0.978611in}{0.755397in}}%
\pgfpathlineto{\pgfqpoint{0.966181in}{0.761179in}}%
\pgfpathlineto{\pgfqpoint{0.955518in}{0.766862in}}%
\pgfpathlineto{\pgfqpoint{0.953752in}{0.767865in}}%
\pgfpathlineto{\pgfqpoint{0.941323in}{0.775829in}}%
\pgfpathlineto{\pgfqpoint{0.937687in}{0.778414in}}%
\pgfpathlineto{\pgfqpoint{0.928894in}{0.785118in}}%
\pgfpathlineto{\pgfqpoint{0.923094in}{0.789967in}}%
\pgfpathlineto{\pgfqpoint{0.916464in}{0.795944in}}%
\pgfpathlineto{\pgfqpoint{0.910772in}{0.801519in}}%
\pgfpathlineto{\pgfqpoint{0.904035in}{0.808683in}}%
\pgfpathlineto{\pgfqpoint{0.900204in}{0.813072in}}%
\pgfpathlineto{\pgfqpoint{0.891606in}{0.823847in}}%
\pgfpathlineto{\pgfqpoint{0.891025in}{0.824624in}}%
\pgfpathlineto{\pgfqpoint{0.883230in}{0.836177in}}%
\pgfpathlineto{\pgfqpoint{0.879176in}{0.842879in}}%
\pgfpathlineto{\pgfqpoint{0.876416in}{0.847730in}}%
\pgfpathlineto{\pgfqpoint{0.870608in}{0.859282in}}%
\pgfpathlineto{\pgfqpoint{0.866747in}{0.868111in}}%
\pgfpathlineto{\pgfqpoint{0.865619in}{0.870835in}}%
\pgfpathlineto{\pgfqpoint{0.861563in}{0.882387in}}%
\pgfpathlineto{\pgfqpoint{0.858214in}{0.893940in}}%
\pgfpathlineto{\pgfqpoint{0.855569in}{0.905492in}}%
\pgfpathlineto{\pgfqpoint{0.854318in}{0.912939in}}%
\pgfpathlineto{\pgfqpoint{0.853661in}{0.917045in}}%
\pgfpathlineto{\pgfqpoint{0.852479in}{0.928597in}}%
\pgfpathlineto{\pgfqpoint{0.851952in}{0.940150in}}%
\pgfpathlineto{\pgfqpoint{0.852074in}{0.951702in}}%
\pgfpathlineto{\pgfqpoint{0.852837in}{0.963255in}}%
\pgfpathlineto{\pgfqpoint{0.854236in}{0.974808in}}%
\pgfpathlineto{\pgfqpoint{0.854318in}{0.975279in}}%
\pgfpathlineto{\pgfqpoint{0.856372in}{0.986360in}}%
\pgfpathlineto{\pgfqpoint{0.859166in}{0.997913in}}%
\pgfpathlineto{\pgfqpoint{0.862603in}{1.009465in}}%
\pgfpathlineto{\pgfqpoint{0.866670in}{1.021018in}}%
\pgfpathlineto{\pgfqpoint{0.866747in}{1.021210in}}%
\pgfpathlineto{\pgfqpoint{0.871645in}{1.032570in}}%
\pgfpathlineto{\pgfqpoint{0.877265in}{1.044123in}}%
\pgfpathlineto{\pgfqpoint{0.879176in}{1.047683in}}%
\pgfpathlineto{\pgfqpoint{0.883810in}{1.055675in}}%
\pgfpathlineto{\pgfqpoint{0.891132in}{1.067228in}}%
\pgfpathlineto{\pgfqpoint{0.891606in}{1.067924in}}%
\pgfpathlineto{\pgfqpoint{0.899647in}{1.078781in}}%
\pgfpathlineto{\pgfqpoint{0.904035in}{1.084327in}}%
\pgfpathlineto{\pgfqpoint{0.909241in}{1.090333in}}%
\pgfpathlineto{\pgfqpoint{0.916464in}{1.098217in}}%
\pgfpathlineto{\pgfqpoint{0.920177in}{1.101886in}}%
\pgfpathlineto{\pgfqpoint{0.928894in}{1.110110in}}%
\pgfpathlineto{\pgfqpoint{0.932827in}{1.113438in}}%
\pgfpathlineto{\pgfqpoint{0.941323in}{1.120363in}}%
\pgfpathlineto{\pgfqpoint{0.947724in}{1.124991in}}%
\pgfpathlineto{\pgfqpoint{0.953752in}{1.129223in}}%
\pgfpathlineto{\pgfqpoint{0.965651in}{1.136543in}}%
\pgfpathlineto{\pgfqpoint{0.966181in}{1.136862in}}%
\pgfpathlineto{\pgfqpoint{0.978611in}{1.143324in}}%
\pgfpathlineto{\pgfqpoint{0.989475in}{1.148096in}}%
\pgfpathlineto{\pgfqpoint{0.991040in}{1.148775in}}%
\pgfpathlineto{\pgfqpoint{1.003469in}{1.153199in}}%
\pgfpathlineto{\pgfqpoint{1.015899in}{1.156685in}}%
\pgfpathlineto{\pgfqpoint{1.028328in}{1.159245in}}%
\pgfpathlineto{\pgfqpoint{1.031408in}{1.159648in}}%
\pgfpathlineto{\pgfqpoint{1.040757in}{1.160874in}}%
\pgfpathlineto{\pgfqpoint{1.053187in}{1.161584in}}%
\pgfpathlineto{\pgfqpoint{1.065616in}{1.161369in}}%
\pgfpathlineto{\pgfqpoint{1.078045in}{1.160207in}}%
\pgfpathlineto{\pgfqpoint{1.081340in}{1.159648in}}%
\pgfpathlineto{\pgfqpoint{1.090474in}{1.158057in}}%
\pgfpathlineto{\pgfqpoint{1.102904in}{1.154877in}}%
\pgfpathlineto{\pgfqpoint{1.115333in}{1.150617in}}%
\pgfpathlineto{\pgfqpoint{1.121207in}{1.148096in}}%
\pgfpathlineto{\pgfqpoint{1.127762in}{1.145172in}}%
\pgfpathlineto{\pgfqpoint{1.140192in}{1.138445in}}%
\pgfpathlineto{\pgfqpoint{1.143186in}{1.136543in}}%
\pgfpathlineto{\pgfqpoint{1.152621in}{1.130252in}}%
\pgfpathlineto{\pgfqpoint{1.159437in}{1.124991in}}%
\pgfpathlineto{\pgfqpoint{1.165050in}{1.120402in}}%
\pgfpathlineto{\pgfqpoint{1.172562in}{1.113438in}}%
\pgfpathlineto{\pgfqpoint{1.177480in}{1.108563in}}%
\pgfpathlineto{\pgfqpoint{1.183515in}{1.101886in}}%
\pgfpathlineto{\pgfqpoint{1.189909in}{1.094240in}}%
\pgfpathlineto{\pgfqpoint{1.192876in}{1.090333in}}%
\pgfpathlineto{\pgfqpoint{1.200935in}{1.078781in}}%
\pgfpathlineto{\pgfqpoint{1.202338in}{1.076584in}}%
\pgfpathlineto{\pgfqpoint{1.207838in}{1.067228in}}%
\pgfpathlineto{\pgfqpoint{1.213904in}{1.055675in}}%
\pgfpathlineto{\pgfqpoint{1.214767in}{1.053849in}}%
\pgfpathlineto{\pgfqpoint{1.219044in}{1.044123in}}%
\pgfpathlineto{\pgfqpoint{1.223482in}{1.032570in}}%
\pgfpathlineto{\pgfqpoint{1.227197in}{1.021283in}}%
\pgfpathlineto{\pgfqpoint{1.227279in}{1.021018in}}%
\pgfpathlineto{\pgfqpoint{1.230336in}{1.009465in}}%
\pgfpathlineto{\pgfqpoint{1.232813in}{0.997913in}}%
\pgfpathlineto{\pgfqpoint{1.234724in}{0.986360in}}%
\pgfpathlineto{\pgfqpoint{1.236076in}{0.974808in}}%
\pgfpathlineto{\pgfqpoint{1.236877in}{0.963255in}}%
\pgfpathlineto{\pgfqpoint{1.237129in}{0.951702in}}%
\pgfpathlineto{\pgfqpoint{1.236835in}{0.940150in}}%
\pgfpathlineto{\pgfqpoint{1.235993in}{0.928597in}}%
\pgfpathlineto{\pgfqpoint{1.234599in}{0.917045in}}%
\pgfpathlineto{\pgfqpoint{1.232647in}{0.905492in}}%
\pgfpathlineto{\pgfqpoint{1.230129in}{0.893940in}}%
\pgfpathlineto{\pgfqpoint{1.227197in}{0.882998in}}%
\pgfpathlineto{\pgfqpoint{1.227027in}{0.882387in}}%
\pgfpathlineto{\pgfqpoint{1.223180in}{0.870835in}}%
\pgfpathlineto{\pgfqpoint{1.218697in}{0.859282in}}%
\pgfpathlineto{\pgfqpoint{1.214767in}{0.850450in}}%
\pgfpathlineto{\pgfqpoint{1.213497in}{0.847730in}}%
\pgfpathlineto{\pgfqpoint{1.207377in}{0.836177in}}%
\pgfpathlineto{\pgfqpoint{1.202338in}{0.827698in}}%
\pgfpathlineto{\pgfqpoint{1.200400in}{0.824624in}}%
\pgfpathlineto{\pgfqpoint{1.192288in}{0.813072in}}%
\pgfpathlineto{\pgfqpoint{1.189909in}{0.810006in}}%
\pgfpathlineto{\pgfqpoint{1.182834in}{0.801519in}}%
\pgfpathlineto{\pgfqpoint{1.177480in}{0.795678in}}%
\pgfpathlineto{\pgfqpoint{1.171785in}{0.789967in}}%
\pgfpathlineto{\pgfqpoint{1.165050in}{0.783791in}}%
\pgfpathlineto{\pgfqpoint{1.158578in}{0.778414in}}%
\pgfpathlineto{\pgfqpoint{1.152621in}{0.773866in}}%
\pgfpathlineto{\pgfqpoint{1.142313in}{0.766862in}}%
\pgfpathlineto{\pgfqpoint{1.140192in}{0.765530in}}%
\pgfpathlineto{\pgfqpoint{1.127762in}{0.758711in}}%
\pgfpathlineto{\pgfqpoint{1.120473in}{0.755309in}}%
\pgfpathlineto{\pgfqpoint{1.115333in}{0.753084in}}%
\pgfpathlineto{\pgfqpoint{1.102904in}{0.748649in}}%
\pgfpathlineto{\pgfqpoint{1.090474in}{0.745182in}}%
\pgfpathlineto{\pgfqpoint{1.083435in}{0.743757in}}%
\pgfpathlineto{\pgfqpoint{1.078045in}{0.742739in}}%
\pgfpathlineto{\pgfqpoint{1.065616in}{0.741272in}}%
\pgfpathlineto{\pgfqpoint{1.053187in}{0.740682in}}%
\pgfpathlineto{\pgfqpoint{1.040757in}{0.740956in}}%
\pgfpathlineto{\pgfqpoint{1.028328in}{0.742079in}}%
\pgfpathclose%
\pgfusepath{fill}%
\end{pgfscope}%
\begin{pgfscope}%
\pgfpathrectangle{\pgfqpoint{0.211875in}{0.211875in}}{\pgfqpoint{1.313625in}{1.279725in}}%
\pgfusepath{clip}%
\pgfsetbuttcap%
\pgfsetroundjoin%
\definecolor{currentfill}{rgb}{0.965440,0.720101,0.576404}%
\pgfsetfillcolor{currentfill}%
\pgfsetlinewidth{0.000000pt}%
\definecolor{currentstroke}{rgb}{0.000000,0.000000,0.000000}%
\pgfsetstrokecolor{currentstroke}%
\pgfsetdash{}{0pt}%
\pgfpathmoveto{\pgfqpoint{1.513071in}{1.362421in}}%
\pgfpathlineto{\pgfqpoint{1.525500in}{1.356979in}}%
\pgfpathlineto{\pgfqpoint{1.525500in}{1.367594in}}%
\pgfpathlineto{\pgfqpoint{1.525500in}{1.379147in}}%
\pgfpathlineto{\pgfqpoint{1.525500in}{1.390699in}}%
\pgfpathlineto{\pgfqpoint{1.525500in}{1.402252in}}%
\pgfpathlineto{\pgfqpoint{1.525500in}{1.413805in}}%
\pgfpathlineto{\pgfqpoint{1.513071in}{1.413805in}}%
\pgfpathlineto{\pgfqpoint{1.500641in}{1.413805in}}%
\pgfpathlineto{\pgfqpoint{1.488212in}{1.413805in}}%
\pgfpathlineto{\pgfqpoint{1.475783in}{1.413805in}}%
\pgfpathlineto{\pgfqpoint{1.463353in}{1.413805in}}%
\pgfpathlineto{\pgfqpoint{1.450924in}{1.413805in}}%
\pgfpathlineto{\pgfqpoint{1.438495in}{1.413805in}}%
\pgfpathlineto{\pgfqpoint{1.426066in}{1.413805in}}%
\pgfpathlineto{\pgfqpoint{1.413636in}{1.413805in}}%
\pgfpathlineto{\pgfqpoint{1.401207in}{1.413805in}}%
\pgfpathlineto{\pgfqpoint{1.388778in}{1.413805in}}%
\pgfpathlineto{\pgfqpoint{1.376348in}{1.413805in}}%
\pgfpathlineto{\pgfqpoint{1.363919in}{1.413805in}}%
\pgfpathlineto{\pgfqpoint{1.354662in}{1.413805in}}%
\pgfpathlineto{\pgfqpoint{1.363919in}{1.411322in}}%
\pgfpathlineto{\pgfqpoint{1.376348in}{1.407995in}}%
\pgfpathlineto{\pgfqpoint{1.388778in}{1.404681in}}%
\pgfpathlineto{\pgfqpoint{1.397637in}{1.402252in}}%
\pgfpathlineto{\pgfqpoint{1.401207in}{1.401271in}}%
\pgfpathlineto{\pgfqpoint{1.413636in}{1.397649in}}%
\pgfpathlineto{\pgfqpoint{1.426066in}{1.393955in}}%
\pgfpathlineto{\pgfqpoint{1.436637in}{1.390699in}}%
\pgfpathlineto{\pgfqpoint{1.438495in}{1.390128in}}%
\pgfpathlineto{\pgfqpoint{1.450924in}{1.386014in}}%
\pgfpathlineto{\pgfqpoint{1.463353in}{1.381773in}}%
\pgfpathlineto{\pgfqpoint{1.470651in}{1.379147in}}%
\pgfpathlineto{\pgfqpoint{1.475783in}{1.377307in}}%
\pgfpathlineto{\pgfqpoint{1.488212in}{1.372576in}}%
\pgfpathlineto{\pgfqpoint{1.500641in}{1.367677in}}%
\pgfpathlineto{\pgfqpoint{1.500836in}{1.367594in}}%
\pgfpathclose%
\pgfusepath{fill}%
\end{pgfscope}%
\begin{pgfscope}%
\pgfpathrectangle{\pgfqpoint{0.211875in}{0.211875in}}{\pgfqpoint{1.313625in}{1.279725in}}%
\pgfusepath{clip}%
\pgfsetbuttcap%
\pgfsetroundjoin%
\definecolor{currentfill}{rgb}{0.973832,0.856556,0.771584}%
\pgfsetfillcolor{currentfill}%
\pgfsetlinewidth{0.000000pt}%
\definecolor{currentstroke}{rgb}{0.000000,0.000000,0.000000}%
\pgfsetstrokecolor{currentstroke}%
\pgfsetdash{}{0pt}%
\pgfpathmoveto{\pgfqpoint{1.028328in}{0.742079in}}%
\pgfpathlineto{\pgfqpoint{1.040757in}{0.740956in}}%
\pgfpathlineto{\pgfqpoint{1.053187in}{0.740682in}}%
\pgfpathlineto{\pgfqpoint{1.065616in}{0.741272in}}%
\pgfpathlineto{\pgfqpoint{1.078045in}{0.742739in}}%
\pgfpathlineto{\pgfqpoint{1.083435in}{0.743757in}}%
\pgfpathlineto{\pgfqpoint{1.090474in}{0.745182in}}%
\pgfpathlineto{\pgfqpoint{1.102904in}{0.748649in}}%
\pgfpathlineto{\pgfqpoint{1.115333in}{0.753084in}}%
\pgfpathlineto{\pgfqpoint{1.120473in}{0.755309in}}%
\pgfpathlineto{\pgfqpoint{1.127762in}{0.758711in}}%
\pgfpathlineto{\pgfqpoint{1.140192in}{0.765530in}}%
\pgfpathlineto{\pgfqpoint{1.142313in}{0.766862in}}%
\pgfpathlineto{\pgfqpoint{1.152621in}{0.773866in}}%
\pgfpathlineto{\pgfqpoint{1.158578in}{0.778414in}}%
\pgfpathlineto{\pgfqpoint{1.165050in}{0.783791in}}%
\pgfpathlineto{\pgfqpoint{1.171785in}{0.789967in}}%
\pgfpathlineto{\pgfqpoint{1.177480in}{0.795678in}}%
\pgfpathlineto{\pgfqpoint{1.182834in}{0.801519in}}%
\pgfpathlineto{\pgfqpoint{1.189909in}{0.810006in}}%
\pgfpathlineto{\pgfqpoint{1.192288in}{0.813072in}}%
\pgfpathlineto{\pgfqpoint{1.200400in}{0.824624in}}%
\pgfpathlineto{\pgfqpoint{1.202338in}{0.827698in}}%
\pgfpathlineto{\pgfqpoint{1.207377in}{0.836177in}}%
\pgfpathlineto{\pgfqpoint{1.213497in}{0.847730in}}%
\pgfpathlineto{\pgfqpoint{1.214767in}{0.850450in}}%
\pgfpathlineto{\pgfqpoint{1.218697in}{0.859282in}}%
\pgfpathlineto{\pgfqpoint{1.223180in}{0.870835in}}%
\pgfpathlineto{\pgfqpoint{1.227027in}{0.882387in}}%
\pgfpathlineto{\pgfqpoint{1.227197in}{0.882998in}}%
\pgfpathlineto{\pgfqpoint{1.230129in}{0.893940in}}%
\pgfpathlineto{\pgfqpoint{1.232647in}{0.905492in}}%
\pgfpathlineto{\pgfqpoint{1.234599in}{0.917045in}}%
\pgfpathlineto{\pgfqpoint{1.235993in}{0.928597in}}%
\pgfpathlineto{\pgfqpoint{1.236835in}{0.940150in}}%
\pgfpathlineto{\pgfqpoint{1.237129in}{0.951702in}}%
\pgfpathlineto{\pgfqpoint{1.236877in}{0.963255in}}%
\pgfpathlineto{\pgfqpoint{1.236076in}{0.974808in}}%
\pgfpathlineto{\pgfqpoint{1.234724in}{0.986360in}}%
\pgfpathlineto{\pgfqpoint{1.232813in}{0.997913in}}%
\pgfpathlineto{\pgfqpoint{1.230336in}{1.009465in}}%
\pgfpathlineto{\pgfqpoint{1.227279in}{1.021018in}}%
\pgfpathlineto{\pgfqpoint{1.227197in}{1.021283in}}%
\pgfpathlineto{\pgfqpoint{1.223482in}{1.032570in}}%
\pgfpathlineto{\pgfqpoint{1.219044in}{1.044123in}}%
\pgfpathlineto{\pgfqpoint{1.214767in}{1.053849in}}%
\pgfpathlineto{\pgfqpoint{1.213904in}{1.055675in}}%
\pgfpathlineto{\pgfqpoint{1.207838in}{1.067228in}}%
\pgfpathlineto{\pgfqpoint{1.202338in}{1.076584in}}%
\pgfpathlineto{\pgfqpoint{1.200935in}{1.078781in}}%
\pgfpathlineto{\pgfqpoint{1.192876in}{1.090333in}}%
\pgfpathlineto{\pgfqpoint{1.189909in}{1.094240in}}%
\pgfpathlineto{\pgfqpoint{1.183515in}{1.101886in}}%
\pgfpathlineto{\pgfqpoint{1.177480in}{1.108563in}}%
\pgfpathlineto{\pgfqpoint{1.172562in}{1.113438in}}%
\pgfpathlineto{\pgfqpoint{1.165050in}{1.120402in}}%
\pgfpathlineto{\pgfqpoint{1.159437in}{1.124991in}}%
\pgfpathlineto{\pgfqpoint{1.152621in}{1.130252in}}%
\pgfpathlineto{\pgfqpoint{1.143186in}{1.136543in}}%
\pgfpathlineto{\pgfqpoint{1.140192in}{1.138445in}}%
\pgfpathlineto{\pgfqpoint{1.127762in}{1.145172in}}%
\pgfpathlineto{\pgfqpoint{1.121207in}{1.148096in}}%
\pgfpathlineto{\pgfqpoint{1.115333in}{1.150617in}}%
\pgfpathlineto{\pgfqpoint{1.102904in}{1.154877in}}%
\pgfpathlineto{\pgfqpoint{1.090474in}{1.158057in}}%
\pgfpathlineto{\pgfqpoint{1.081340in}{1.159648in}}%
\pgfpathlineto{\pgfqpoint{1.078045in}{1.160207in}}%
\pgfpathlineto{\pgfqpoint{1.065616in}{1.161369in}}%
\pgfpathlineto{\pgfqpoint{1.053187in}{1.161584in}}%
\pgfpathlineto{\pgfqpoint{1.040757in}{1.160874in}}%
\pgfpathlineto{\pgfqpoint{1.031408in}{1.159648in}}%
\pgfpathlineto{\pgfqpoint{1.028328in}{1.159245in}}%
\pgfpathlineto{\pgfqpoint{1.015899in}{1.156685in}}%
\pgfpathlineto{\pgfqpoint{1.003469in}{1.153199in}}%
\pgfpathlineto{\pgfqpoint{0.991040in}{1.148775in}}%
\pgfpathlineto{\pgfqpoint{0.989475in}{1.148096in}}%
\pgfpathlineto{\pgfqpoint{0.978611in}{1.143324in}}%
\pgfpathlineto{\pgfqpoint{0.966181in}{1.136862in}}%
\pgfpathlineto{\pgfqpoint{0.965651in}{1.136543in}}%
\pgfpathlineto{\pgfqpoint{0.953752in}{1.129223in}}%
\pgfpathlineto{\pgfqpoint{0.947724in}{1.124991in}}%
\pgfpathlineto{\pgfqpoint{0.941323in}{1.120363in}}%
\pgfpathlineto{\pgfqpoint{0.932827in}{1.113438in}}%
\pgfpathlineto{\pgfqpoint{0.928894in}{1.110110in}}%
\pgfpathlineto{\pgfqpoint{0.920177in}{1.101886in}}%
\pgfpathlineto{\pgfqpoint{0.916464in}{1.098217in}}%
\pgfpathlineto{\pgfqpoint{0.909241in}{1.090333in}}%
\pgfpathlineto{\pgfqpoint{0.904035in}{1.084327in}}%
\pgfpathlineto{\pgfqpoint{0.899647in}{1.078781in}}%
\pgfpathlineto{\pgfqpoint{0.891606in}{1.067924in}}%
\pgfpathlineto{\pgfqpoint{0.891132in}{1.067228in}}%
\pgfpathlineto{\pgfqpoint{0.883810in}{1.055675in}}%
\pgfpathlineto{\pgfqpoint{0.879176in}{1.047683in}}%
\pgfpathlineto{\pgfqpoint{0.877265in}{1.044123in}}%
\pgfpathlineto{\pgfqpoint{0.871645in}{1.032570in}}%
\pgfpathlineto{\pgfqpoint{0.866747in}{1.021210in}}%
\pgfpathlineto{\pgfqpoint{0.866670in}{1.021018in}}%
\pgfpathlineto{\pgfqpoint{0.862603in}{1.009465in}}%
\pgfpathlineto{\pgfqpoint{0.859166in}{0.997913in}}%
\pgfpathlineto{\pgfqpoint{0.856372in}{0.986360in}}%
\pgfpathlineto{\pgfqpoint{0.854318in}{0.975279in}}%
\pgfpathlineto{\pgfqpoint{0.854236in}{0.974808in}}%
\pgfpathlineto{\pgfqpoint{0.852837in}{0.963255in}}%
\pgfpathlineto{\pgfqpoint{0.852074in}{0.951702in}}%
\pgfpathlineto{\pgfqpoint{0.851952in}{0.940150in}}%
\pgfpathlineto{\pgfqpoint{0.852479in}{0.928597in}}%
\pgfpathlineto{\pgfqpoint{0.853661in}{0.917045in}}%
\pgfpathlineto{\pgfqpoint{0.854318in}{0.912939in}}%
\pgfpathlineto{\pgfqpoint{0.855569in}{0.905492in}}%
\pgfpathlineto{\pgfqpoint{0.858214in}{0.893940in}}%
\pgfpathlineto{\pgfqpoint{0.861563in}{0.882387in}}%
\pgfpathlineto{\pgfqpoint{0.865619in}{0.870835in}}%
\pgfpathlineto{\pgfqpoint{0.866747in}{0.868111in}}%
\pgfpathlineto{\pgfqpoint{0.870608in}{0.859282in}}%
\pgfpathlineto{\pgfqpoint{0.876416in}{0.847730in}}%
\pgfpathlineto{\pgfqpoint{0.879176in}{0.842879in}}%
\pgfpathlineto{\pgfqpoint{0.883230in}{0.836177in}}%
\pgfpathlineto{\pgfqpoint{0.891025in}{0.824624in}}%
\pgfpathlineto{\pgfqpoint{0.891606in}{0.823847in}}%
\pgfpathlineto{\pgfqpoint{0.900204in}{0.813072in}}%
\pgfpathlineto{\pgfqpoint{0.904035in}{0.808683in}}%
\pgfpathlineto{\pgfqpoint{0.910772in}{0.801519in}}%
\pgfpathlineto{\pgfqpoint{0.916464in}{0.795944in}}%
\pgfpathlineto{\pgfqpoint{0.923094in}{0.789967in}}%
\pgfpathlineto{\pgfqpoint{0.928894in}{0.785118in}}%
\pgfpathlineto{\pgfqpoint{0.937687in}{0.778414in}}%
\pgfpathlineto{\pgfqpoint{0.941323in}{0.775829in}}%
\pgfpathlineto{\pgfqpoint{0.953752in}{0.767865in}}%
\pgfpathlineto{\pgfqpoint{0.955518in}{0.766862in}}%
\pgfpathlineto{\pgfqpoint{0.966181in}{0.761179in}}%
\pgfpathlineto{\pgfqpoint{0.978611in}{0.755397in}}%
\pgfpathlineto{\pgfqpoint{0.978835in}{0.755309in}}%
\pgfpathlineto{\pgfqpoint{0.991040in}{0.750778in}}%
\pgfpathlineto{\pgfqpoint{1.003469in}{0.746991in}}%
\pgfpathlineto{\pgfqpoint{1.015899in}{0.744050in}}%
\pgfpathlineto{\pgfqpoint{1.017665in}{0.743757in}}%
\pgfpathclose%
\pgfusepath{fill}%
\end{pgfscope}%
\begin{pgfscope}%
\pgfpathrectangle{\pgfqpoint{0.211875in}{0.211875in}}{\pgfqpoint{1.313625in}{1.279725in}}%
\pgfusepath{clip}%
\pgfsetbuttcap%
\pgfsetroundjoin%
\definecolor{currentfill}{rgb}{0.121569,0.466667,0.705882}%
\pgfsetfillcolor{currentfill}%
\pgfsetlinewidth{1.003750pt}%
\definecolor{currentstroke}{rgb}{0.121569,0.466667,0.705882}%
\pgfsetstrokecolor{currentstroke}%
\pgfsetdash{}{0pt}%
\pgfpathmoveto{\pgfqpoint{1.457098in}{0.324827in}}%
\pgfpathcurveto{\pgfqpoint{1.462922in}{0.324827in}}{\pgfqpoint{1.468508in}{0.327140in}}{\pgfqpoint{1.472627in}{0.331259in}}%
\pgfpathcurveto{\pgfqpoint{1.476745in}{0.335377in}}{\pgfqpoint{1.479059in}{0.340963in}}{\pgfqpoint{1.479059in}{0.346787in}}%
\pgfpathcurveto{\pgfqpoint{1.479059in}{0.352611in}}{\pgfqpoint{1.476745in}{0.358197in}}{\pgfqpoint{1.472627in}{0.362315in}}%
\pgfpathcurveto{\pgfqpoint{1.468508in}{0.366433in}}{\pgfqpoint{1.462922in}{0.368747in}}{\pgfqpoint{1.457098in}{0.368747in}}%
\pgfpathcurveto{\pgfqpoint{1.451274in}{0.368747in}}{\pgfqpoint{1.445688in}{0.366433in}}{\pgfqpoint{1.441570in}{0.362315in}}%
\pgfpathcurveto{\pgfqpoint{1.437452in}{0.358197in}}{\pgfqpoint{1.435138in}{0.352611in}}{\pgfqpoint{1.435138in}{0.346787in}}%
\pgfpathcurveto{\pgfqpoint{1.435138in}{0.340963in}}{\pgfqpoint{1.437452in}{0.335377in}}{\pgfqpoint{1.441570in}{0.331259in}}%
\pgfpathcurveto{\pgfqpoint{1.445688in}{0.327140in}}{\pgfqpoint{1.451274in}{0.324827in}}{\pgfqpoint{1.457098in}{0.324827in}}%
\pgfpathclose%
\pgfusepath{stroke,fill}%
\end{pgfscope}%
\begin{pgfscope}%
\pgfpathrectangle{\pgfqpoint{0.211875in}{0.211875in}}{\pgfqpoint{1.313625in}{1.279725in}}%
\pgfusepath{clip}%
\pgfsetbuttcap%
\pgfsetroundjoin%
\definecolor{currentfill}{rgb}{0.121569,0.466667,0.705882}%
\pgfsetfillcolor{currentfill}%
\pgfsetlinewidth{1.003750pt}%
\definecolor{currentstroke}{rgb}{0.121569,0.466667,0.705882}%
\pgfsetstrokecolor{currentstroke}%
\pgfsetdash{}{0pt}%
\pgfpathmoveto{\pgfqpoint{1.208389in}{0.364851in}}%
\pgfpathcurveto{\pgfqpoint{1.214213in}{0.364851in}}{\pgfqpoint{1.219800in}{0.367165in}}{\pgfqpoint{1.223918in}{0.371283in}}%
\pgfpathcurveto{\pgfqpoint{1.228036in}{0.375401in}}{\pgfqpoint{1.230350in}{0.380987in}}{\pgfqpoint{1.230350in}{0.386811in}}%
\pgfpathcurveto{\pgfqpoint{1.230350in}{0.392635in}}{\pgfqpoint{1.228036in}{0.398221in}}{\pgfqpoint{1.223918in}{0.402339in}}%
\pgfpathcurveto{\pgfqpoint{1.219800in}{0.406457in}}{\pgfqpoint{1.214213in}{0.408771in}}{\pgfqpoint{1.208389in}{0.408771in}}%
\pgfpathcurveto{\pgfqpoint{1.202566in}{0.408771in}}{\pgfqpoint{1.196979in}{0.406457in}}{\pgfqpoint{1.192861in}{0.402339in}}%
\pgfpathcurveto{\pgfqpoint{1.188743in}{0.398221in}}{\pgfqpoint{1.186429in}{0.392635in}}{\pgfqpoint{1.186429in}{0.386811in}}%
\pgfpathcurveto{\pgfqpoint{1.186429in}{0.380987in}}{\pgfqpoint{1.188743in}{0.375401in}}{\pgfqpoint{1.192861in}{0.371283in}}%
\pgfpathcurveto{\pgfqpoint{1.196979in}{0.367165in}}{\pgfqpoint{1.202566in}{0.364851in}}{\pgfqpoint{1.208389in}{0.364851in}}%
\pgfpathclose%
\pgfusepath{stroke,fill}%
\end{pgfscope}%
\begin{pgfscope}%
\pgfpathrectangle{\pgfqpoint{0.211875in}{0.211875in}}{\pgfqpoint{1.313625in}{1.279725in}}%
\pgfusepath{clip}%
\pgfsetbuttcap%
\pgfsetroundjoin%
\definecolor{currentfill}{rgb}{0.121569,0.466667,0.705882}%
\pgfsetfillcolor{currentfill}%
\pgfsetlinewidth{1.003750pt}%
\definecolor{currentstroke}{rgb}{0.121569,0.466667,0.705882}%
\pgfsetstrokecolor{currentstroke}%
\pgfsetdash{}{0pt}%
\pgfpathmoveto{\pgfqpoint{0.366972in}{0.725007in}}%
\pgfpathcurveto{\pgfqpoint{0.372795in}{0.725007in}}{\pgfqpoint{0.378382in}{0.727321in}}{\pgfqpoint{0.382500in}{0.731439in}}%
\pgfpathcurveto{\pgfqpoint{0.386618in}{0.735557in}}{\pgfqpoint{0.388932in}{0.741143in}}{\pgfqpoint{0.388932in}{0.746967in}}%
\pgfpathcurveto{\pgfqpoint{0.388932in}{0.752791in}}{\pgfqpoint{0.386618in}{0.758377in}}{\pgfqpoint{0.382500in}{0.762496in}}%
\pgfpathcurveto{\pgfqpoint{0.378382in}{0.766614in}}{\pgfqpoint{0.372795in}{0.768928in}}{\pgfqpoint{0.366972in}{0.768928in}}%
\pgfpathcurveto{\pgfqpoint{0.361148in}{0.768928in}}{\pgfqpoint{0.355561in}{0.766614in}}{\pgfqpoint{0.351443in}{0.762496in}}%
\pgfpathcurveto{\pgfqpoint{0.347325in}{0.758377in}}{\pgfqpoint{0.345011in}{0.752791in}}{\pgfqpoint{0.345011in}{0.746967in}}%
\pgfpathcurveto{\pgfqpoint{0.345011in}{0.741143in}}{\pgfqpoint{0.347325in}{0.735557in}}{\pgfqpoint{0.351443in}{0.731439in}}%
\pgfpathcurveto{\pgfqpoint{0.355561in}{0.727321in}}{\pgfqpoint{0.361148in}{0.725007in}}{\pgfqpoint{0.366972in}{0.725007in}}%
\pgfpathclose%
\pgfusepath{stroke,fill}%
\end{pgfscope}%
\begin{pgfscope}%
\pgfpathrectangle{\pgfqpoint{0.211875in}{0.211875in}}{\pgfqpoint{1.313625in}{1.279725in}}%
\pgfusepath{clip}%
\pgfsetbuttcap%
\pgfsetroundjoin%
\definecolor{currentfill}{rgb}{0.121569,0.466667,0.705882}%
\pgfsetfillcolor{currentfill}%
\pgfsetlinewidth{1.003750pt}%
\definecolor{currentstroke}{rgb}{0.121569,0.466667,0.705882}%
\pgfsetstrokecolor{currentstroke}%
\pgfsetdash{}{0pt}%
\pgfpathmoveto{\pgfqpoint{1.181756in}{0.779119in}}%
\pgfpathcurveto{\pgfqpoint{1.187580in}{0.779119in}}{\pgfqpoint{1.193166in}{0.781433in}}{\pgfqpoint{1.197284in}{0.785551in}}%
\pgfpathcurveto{\pgfqpoint{1.201403in}{0.789669in}}{\pgfqpoint{1.203716in}{0.795255in}}{\pgfqpoint{1.203716in}{0.801079in}}%
\pgfpathcurveto{\pgfqpoint{1.203716in}{0.806903in}}{\pgfqpoint{1.201403in}{0.812489in}}{\pgfqpoint{1.197284in}{0.816607in}}%
\pgfpathcurveto{\pgfqpoint{1.193166in}{0.820725in}}{\pgfqpoint{1.187580in}{0.823039in}}{\pgfqpoint{1.181756in}{0.823039in}}%
\pgfpathcurveto{\pgfqpoint{1.175932in}{0.823039in}}{\pgfqpoint{1.170346in}{0.820725in}}{\pgfqpoint{1.166228in}{0.816607in}}%
\pgfpathcurveto{\pgfqpoint{1.162110in}{0.812489in}}{\pgfqpoint{1.159796in}{0.806903in}}{\pgfqpoint{1.159796in}{0.801079in}}%
\pgfpathcurveto{\pgfqpoint{1.159796in}{0.795255in}}{\pgfqpoint{1.162110in}{0.789669in}}{\pgfqpoint{1.166228in}{0.785551in}}%
\pgfpathcurveto{\pgfqpoint{1.170346in}{0.781433in}}{\pgfqpoint{1.175932in}{0.779119in}}{\pgfqpoint{1.181756in}{0.779119in}}%
\pgfpathclose%
\pgfusepath{stroke,fill}%
\end{pgfscope}%
\begin{pgfscope}%
\pgfpathrectangle{\pgfqpoint{0.211875in}{0.211875in}}{\pgfqpoint{1.313625in}{1.279725in}}%
\pgfusepath{clip}%
\pgfsetbuttcap%
\pgfsetroundjoin%
\definecolor{currentfill}{rgb}{0.121569,0.466667,0.705882}%
\pgfsetfillcolor{currentfill}%
\pgfsetlinewidth{1.003750pt}%
\definecolor{currentstroke}{rgb}{0.121569,0.466667,0.705882}%
\pgfsetstrokecolor{currentstroke}%
\pgfsetdash{}{0pt}%
\pgfpathmoveto{\pgfqpoint{1.387520in}{1.166547in}}%
\pgfpathcurveto{\pgfqpoint{1.393344in}{1.166547in}}{\pgfqpoint{1.398930in}{1.168861in}}{\pgfqpoint{1.403048in}{1.172979in}}%
\pgfpathcurveto{\pgfqpoint{1.407167in}{1.177097in}}{\pgfqpoint{1.409480in}{1.182683in}}{\pgfqpoint{1.409480in}{1.188507in}}%
\pgfpathcurveto{\pgfqpoint{1.409480in}{1.194331in}}{\pgfqpoint{1.407167in}{1.199917in}}{\pgfqpoint{1.403048in}{1.204035in}}%
\pgfpathcurveto{\pgfqpoint{1.398930in}{1.208154in}}{\pgfqpoint{1.393344in}{1.210467in}}{\pgfqpoint{1.387520in}{1.210467in}}%
\pgfpathcurveto{\pgfqpoint{1.381696in}{1.210467in}}{\pgfqpoint{1.376110in}{1.208154in}}{\pgfqpoint{1.371992in}{1.204035in}}%
\pgfpathcurveto{\pgfqpoint{1.367874in}{1.199917in}}{\pgfqpoint{1.365560in}{1.194331in}}{\pgfqpoint{1.365560in}{1.188507in}}%
\pgfpathcurveto{\pgfqpoint{1.365560in}{1.182683in}}{\pgfqpoint{1.367874in}{1.177097in}}{\pgfqpoint{1.371992in}{1.172979in}}%
\pgfpathcurveto{\pgfqpoint{1.376110in}{1.168861in}}{\pgfqpoint{1.381696in}{1.166547in}}{\pgfqpoint{1.387520in}{1.166547in}}%
\pgfpathclose%
\pgfusepath{stroke,fill}%
\end{pgfscope}%
\begin{pgfscope}%
\pgfpathrectangle{\pgfqpoint{0.211875in}{0.211875in}}{\pgfqpoint{1.313625in}{1.279725in}}%
\pgfusepath{clip}%
\pgfsetbuttcap%
\pgfsetroundjoin%
\definecolor{currentfill}{rgb}{0.121569,0.466667,0.705882}%
\pgfsetfillcolor{currentfill}%
\pgfsetlinewidth{1.003750pt}%
\definecolor{currentstroke}{rgb}{0.121569,0.466667,0.705882}%
\pgfsetstrokecolor{currentstroke}%
\pgfsetdash{}{0pt}%
\pgfpathmoveto{\pgfqpoint{0.507191in}{0.825189in}}%
\pgfpathcurveto{\pgfqpoint{0.513015in}{0.825189in}}{\pgfqpoint{0.518601in}{0.827502in}}{\pgfqpoint{0.522720in}{0.831621in}}%
\pgfpathcurveto{\pgfqpoint{0.526838in}{0.835739in}}{\pgfqpoint{0.529152in}{0.841325in}}{\pgfqpoint{0.529152in}{0.847149in}}%
\pgfpathcurveto{\pgfqpoint{0.529152in}{0.852973in}}{\pgfqpoint{0.526838in}{0.858559in}}{\pgfqpoint{0.522720in}{0.862677in}}%
\pgfpathcurveto{\pgfqpoint{0.518601in}{0.866795in}}{\pgfqpoint{0.513015in}{0.869109in}}{\pgfqpoint{0.507191in}{0.869109in}}%
\pgfpathcurveto{\pgfqpoint{0.501367in}{0.869109in}}{\pgfqpoint{0.495781in}{0.866795in}}{\pgfqpoint{0.491663in}{0.862677in}}%
\pgfpathcurveto{\pgfqpoint{0.487545in}{0.858559in}}{\pgfqpoint{0.485231in}{0.852973in}}{\pgfqpoint{0.485231in}{0.847149in}}%
\pgfpathcurveto{\pgfqpoint{0.485231in}{0.841325in}}{\pgfqpoint{0.487545in}{0.835739in}}{\pgfqpoint{0.491663in}{0.831621in}}%
\pgfpathcurveto{\pgfqpoint{0.495781in}{0.827502in}}{\pgfqpoint{0.501367in}{0.825189in}}{\pgfqpoint{0.507191in}{0.825189in}}%
\pgfpathclose%
\pgfusepath{stroke,fill}%
\end{pgfscope}%
\begin{pgfscope}%
\pgfpathrectangle{\pgfqpoint{0.211875in}{0.211875in}}{\pgfqpoint{1.313625in}{1.279725in}}%
\pgfusepath{clip}%
\pgfsetbuttcap%
\pgfsetroundjoin%
\definecolor{currentfill}{rgb}{0.121569,0.466667,0.705882}%
\pgfsetfillcolor{currentfill}%
\pgfsetlinewidth{1.003750pt}%
\definecolor{currentstroke}{rgb}{0.121569,0.466667,0.705882}%
\pgfsetstrokecolor{currentstroke}%
\pgfsetdash{}{0pt}%
\pgfpathmoveto{\pgfqpoint{1.053779in}{0.267710in}}%
\pgfpathcurveto{\pgfqpoint{1.059603in}{0.267710in}}{\pgfqpoint{1.065189in}{0.270024in}}{\pgfqpoint{1.069308in}{0.274142in}}%
\pgfpathcurveto{\pgfqpoint{1.073426in}{0.278260in}}{\pgfqpoint{1.075740in}{0.283846in}}{\pgfqpoint{1.075740in}{0.289670in}}%
\pgfpathcurveto{\pgfqpoint{1.075740in}{0.295494in}}{\pgfqpoint{1.073426in}{0.301081in}}{\pgfqpoint{1.069308in}{0.305199in}}%
\pgfpathcurveto{\pgfqpoint{1.065189in}{0.309317in}}{\pgfqpoint{1.059603in}{0.311631in}}{\pgfqpoint{1.053779in}{0.311631in}}%
\pgfpathcurveto{\pgfqpoint{1.047955in}{0.311631in}}{\pgfqpoint{1.042369in}{0.309317in}}{\pgfqpoint{1.038251in}{0.305199in}}%
\pgfpathcurveto{\pgfqpoint{1.034133in}{0.301081in}}{\pgfqpoint{1.031819in}{0.295494in}}{\pgfqpoint{1.031819in}{0.289670in}}%
\pgfpathcurveto{\pgfqpoint{1.031819in}{0.283846in}}{\pgfqpoint{1.034133in}{0.278260in}}{\pgfqpoint{1.038251in}{0.274142in}}%
\pgfpathcurveto{\pgfqpoint{1.042369in}{0.270024in}}{\pgfqpoint{1.047955in}{0.267710in}}{\pgfqpoint{1.053779in}{0.267710in}}%
\pgfpathclose%
\pgfusepath{stroke,fill}%
\end{pgfscope}%
\begin{pgfscope}%
\pgfpathrectangle{\pgfqpoint{0.211875in}{0.211875in}}{\pgfqpoint{1.313625in}{1.279725in}}%
\pgfusepath{clip}%
\pgfsetbuttcap%
\pgfsetroundjoin%
\definecolor{currentfill}{rgb}{0.121569,0.466667,0.705882}%
\pgfsetfillcolor{currentfill}%
\pgfsetlinewidth{1.003750pt}%
\definecolor{currentstroke}{rgb}{0.121569,0.466667,0.705882}%
\pgfsetstrokecolor{currentstroke}%
\pgfsetdash{}{0pt}%
\pgfpathmoveto{\pgfqpoint{1.245651in}{0.581996in}}%
\pgfpathcurveto{\pgfqpoint{1.251475in}{0.581996in}}{\pgfqpoint{1.257061in}{0.584310in}}{\pgfqpoint{1.261179in}{0.588428in}}%
\pgfpathcurveto{\pgfqpoint{1.265297in}{0.592547in}}{\pgfqpoint{1.267611in}{0.598133in}}{\pgfqpoint{1.267611in}{0.603957in}}%
\pgfpathcurveto{\pgfqpoint{1.267611in}{0.609781in}}{\pgfqpoint{1.265297in}{0.615367in}}{\pgfqpoint{1.261179in}{0.619485in}}%
\pgfpathcurveto{\pgfqpoint{1.257061in}{0.623603in}}{\pgfqpoint{1.251475in}{0.625917in}}{\pgfqpoint{1.245651in}{0.625917in}}%
\pgfpathcurveto{\pgfqpoint{1.239827in}{0.625917in}}{\pgfqpoint{1.234241in}{0.623603in}}{\pgfqpoint{1.230122in}{0.619485in}}%
\pgfpathcurveto{\pgfqpoint{1.226004in}{0.615367in}}{\pgfqpoint{1.223690in}{0.609781in}}{\pgfqpoint{1.223690in}{0.603957in}}%
\pgfpathcurveto{\pgfqpoint{1.223690in}{0.598133in}}{\pgfqpoint{1.226004in}{0.592547in}}{\pgfqpoint{1.230122in}{0.588428in}}%
\pgfpathcurveto{\pgfqpoint{1.234241in}{0.584310in}}{\pgfqpoint{1.239827in}{0.581996in}}{\pgfqpoint{1.245651in}{0.581996in}}%
\pgfpathclose%
\pgfusepath{stroke,fill}%
\end{pgfscope}%
\begin{pgfscope}%
\pgfpathrectangle{\pgfqpoint{0.211875in}{0.211875in}}{\pgfqpoint{1.313625in}{1.279725in}}%
\pgfusepath{clip}%
\pgfsetbuttcap%
\pgfsetroundjoin%
\definecolor{currentfill}{rgb}{0.121569,0.466667,0.705882}%
\pgfsetfillcolor{currentfill}%
\pgfsetlinewidth{1.003750pt}%
\definecolor{currentstroke}{rgb}{0.121569,0.466667,0.705882}%
\pgfsetstrokecolor{currentstroke}%
\pgfsetdash{}{0pt}%
\pgfpathmoveto{\pgfqpoint{1.094108in}{0.949752in}}%
\pgfpathcurveto{\pgfqpoint{1.099932in}{0.949752in}}{\pgfqpoint{1.105518in}{0.952066in}}{\pgfqpoint{1.109636in}{0.956184in}}%
\pgfpathcurveto{\pgfqpoint{1.113754in}{0.960302in}}{\pgfqpoint{1.116068in}{0.965888in}}{\pgfqpoint{1.116068in}{0.971712in}}%
\pgfpathcurveto{\pgfqpoint{1.116068in}{0.977536in}}{\pgfqpoint{1.113754in}{0.983123in}}{\pgfqpoint{1.109636in}{0.987241in}}%
\pgfpathcurveto{\pgfqpoint{1.105518in}{0.991359in}}{\pgfqpoint{1.099932in}{0.993673in}}{\pgfqpoint{1.094108in}{0.993673in}}%
\pgfpathcurveto{\pgfqpoint{1.088284in}{0.993673in}}{\pgfqpoint{1.082698in}{0.991359in}}{\pgfqpoint{1.078580in}{0.987241in}}%
\pgfpathcurveto{\pgfqpoint{1.074462in}{0.983123in}}{\pgfqpoint{1.072148in}{0.977536in}}{\pgfqpoint{1.072148in}{0.971712in}}%
\pgfpathcurveto{\pgfqpoint{1.072148in}{0.965888in}}{\pgfqpoint{1.074462in}{0.960302in}}{\pgfqpoint{1.078580in}{0.956184in}}%
\pgfpathcurveto{\pgfqpoint{1.082698in}{0.952066in}}{\pgfqpoint{1.088284in}{0.949752in}}{\pgfqpoint{1.094108in}{0.949752in}}%
\pgfpathclose%
\pgfusepath{stroke,fill}%
\end{pgfscope}%
\begin{pgfscope}%
\pgfpathrectangle{\pgfqpoint{0.211875in}{0.211875in}}{\pgfqpoint{1.313625in}{1.279725in}}%
\pgfusepath{clip}%
\pgfsetbuttcap%
\pgfsetroundjoin%
\definecolor{currentfill}{rgb}{0.121569,0.466667,0.705882}%
\pgfsetfillcolor{currentfill}%
\pgfsetlinewidth{1.003750pt}%
\definecolor{currentstroke}{rgb}{0.121569,0.466667,0.705882}%
\pgfsetstrokecolor{currentstroke}%
\pgfsetdash{}{0pt}%
\pgfpathmoveto{\pgfqpoint{0.324270in}{0.304875in}}%
\pgfpathcurveto{\pgfqpoint{0.330094in}{0.304875in}}{\pgfqpoint{0.335680in}{0.307189in}}{\pgfqpoint{0.339798in}{0.311307in}}%
\pgfpathcurveto{\pgfqpoint{0.343916in}{0.315425in}}{\pgfqpoint{0.346230in}{0.321011in}}{\pgfqpoint{0.346230in}{0.326835in}}%
\pgfpathcurveto{\pgfqpoint{0.346230in}{0.332659in}}{\pgfqpoint{0.343916in}{0.338245in}}{\pgfqpoint{0.339798in}{0.342363in}}%
\pgfpathcurveto{\pgfqpoint{0.335680in}{0.346481in}}{\pgfqpoint{0.330094in}{0.348795in}}{\pgfqpoint{0.324270in}{0.348795in}}%
\pgfpathcurveto{\pgfqpoint{0.318446in}{0.348795in}}{\pgfqpoint{0.312860in}{0.346481in}}{\pgfqpoint{0.308741in}{0.342363in}}%
\pgfpathcurveto{\pgfqpoint{0.304623in}{0.338245in}}{\pgfqpoint{0.302309in}{0.332659in}}{\pgfqpoint{0.302309in}{0.326835in}}%
\pgfpathcurveto{\pgfqpoint{0.302309in}{0.321011in}}{\pgfqpoint{0.304623in}{0.315425in}}{\pgfqpoint{0.308741in}{0.311307in}}%
\pgfpathcurveto{\pgfqpoint{0.312860in}{0.307189in}}{\pgfqpoint{0.318446in}{0.304875in}}{\pgfqpoint{0.324270in}{0.304875in}}%
\pgfpathclose%
\pgfusepath{stroke,fill}%
\end{pgfscope}%
\begin{pgfscope}%
\pgfpathrectangle{\pgfqpoint{0.211875in}{0.211875in}}{\pgfqpoint{1.313625in}{1.279725in}}%
\pgfusepath{clip}%
\pgfsetbuttcap%
\pgfsetroundjoin%
\definecolor{currentfill}{rgb}{0.121569,0.466667,0.705882}%
\pgfsetfillcolor{currentfill}%
\pgfsetlinewidth{1.003750pt}%
\definecolor{currentstroke}{rgb}{0.121569,0.466667,0.705882}%
\pgfsetstrokecolor{currentstroke}%
\pgfsetdash{}{0pt}%
\pgfpathmoveto{\pgfqpoint{1.293518in}{0.312685in}}%
\pgfpathcurveto{\pgfqpoint{1.299342in}{0.312685in}}{\pgfqpoint{1.304928in}{0.314999in}}{\pgfqpoint{1.309046in}{0.319117in}}%
\pgfpathcurveto{\pgfqpoint{1.313165in}{0.323235in}}{\pgfqpoint{1.315478in}{0.328821in}}{\pgfqpoint{1.315478in}{0.334645in}}%
\pgfpathcurveto{\pgfqpoint{1.315478in}{0.340469in}}{\pgfqpoint{1.313165in}{0.346055in}}{\pgfqpoint{1.309046in}{0.350173in}}%
\pgfpathcurveto{\pgfqpoint{1.304928in}{0.354292in}}{\pgfqpoint{1.299342in}{0.356605in}}{\pgfqpoint{1.293518in}{0.356605in}}%
\pgfpathcurveto{\pgfqpoint{1.287694in}{0.356605in}}{\pgfqpoint{1.282108in}{0.354292in}}{\pgfqpoint{1.277990in}{0.350173in}}%
\pgfpathcurveto{\pgfqpoint{1.273872in}{0.346055in}}{\pgfqpoint{1.271558in}{0.340469in}}{\pgfqpoint{1.271558in}{0.334645in}}%
\pgfpathcurveto{\pgfqpoint{1.271558in}{0.328821in}}{\pgfqpoint{1.273872in}{0.323235in}}{\pgfqpoint{1.277990in}{0.319117in}}%
\pgfpathcurveto{\pgfqpoint{1.282108in}{0.314999in}}{\pgfqpoint{1.287694in}{0.312685in}}{\pgfqpoint{1.293518in}{0.312685in}}%
\pgfpathclose%
\pgfusepath{stroke,fill}%
\end{pgfscope}%
\begin{pgfscope}%
\pgfpathrectangle{\pgfqpoint{0.211875in}{0.211875in}}{\pgfqpoint{1.313625in}{1.279725in}}%
\pgfusepath{clip}%
\pgfsetbuttcap%
\pgfsetroundjoin%
\definecolor{currentfill}{rgb}{0.121569,0.466667,0.705882}%
\pgfsetfillcolor{currentfill}%
\pgfsetlinewidth{1.003750pt}%
\definecolor{currentstroke}{rgb}{0.121569,0.466667,0.705882}%
\pgfsetstrokecolor{currentstroke}%
\pgfsetdash{}{0pt}%
\pgfpathmoveto{\pgfqpoint{0.810637in}{0.902619in}}%
\pgfpathcurveto{\pgfqpoint{0.816461in}{0.902619in}}{\pgfqpoint{0.822047in}{0.904933in}}{\pgfqpoint{0.826165in}{0.909051in}}%
\pgfpathcurveto{\pgfqpoint{0.830283in}{0.913169in}}{\pgfqpoint{0.832597in}{0.918755in}}{\pgfqpoint{0.832597in}{0.924579in}}%
\pgfpathcurveto{\pgfqpoint{0.832597in}{0.930403in}}{\pgfqpoint{0.830283in}{0.935990in}}{\pgfqpoint{0.826165in}{0.940108in}}%
\pgfpathcurveto{\pgfqpoint{0.822047in}{0.944226in}}{\pgfqpoint{0.816461in}{0.946540in}}{\pgfqpoint{0.810637in}{0.946540in}}%
\pgfpathcurveto{\pgfqpoint{0.804813in}{0.946540in}}{\pgfqpoint{0.799226in}{0.944226in}}{\pgfqpoint{0.795108in}{0.940108in}}%
\pgfpathcurveto{\pgfqpoint{0.790990in}{0.935990in}}{\pgfqpoint{0.788676in}{0.930403in}}{\pgfqpoint{0.788676in}{0.924579in}}%
\pgfpathcurveto{\pgfqpoint{0.788676in}{0.918755in}}{\pgfqpoint{0.790990in}{0.913169in}}{\pgfqpoint{0.795108in}{0.909051in}}%
\pgfpathcurveto{\pgfqpoint{0.799226in}{0.904933in}}{\pgfqpoint{0.804813in}{0.902619in}}{\pgfqpoint{0.810637in}{0.902619in}}%
\pgfpathclose%
\pgfusepath{stroke,fill}%
\end{pgfscope}%
\begin{pgfscope}%
\pgfpathrectangle{\pgfqpoint{0.211875in}{0.211875in}}{\pgfqpoint{1.313625in}{1.279725in}}%
\pgfusepath{clip}%
\pgfsetbuttcap%
\pgfsetroundjoin%
\definecolor{currentfill}{rgb}{0.121569,0.466667,0.705882}%
\pgfsetfillcolor{currentfill}%
\pgfsetlinewidth{1.003750pt}%
\definecolor{currentstroke}{rgb}{0.121569,0.466667,0.705882}%
\pgfsetstrokecolor{currentstroke}%
\pgfsetdash{}{0pt}%
\pgfpathmoveto{\pgfqpoint{1.179343in}{1.087960in}}%
\pgfpathcurveto{\pgfqpoint{1.185167in}{1.087960in}}{\pgfqpoint{1.190753in}{1.090274in}}{\pgfqpoint{1.194871in}{1.094392in}}%
\pgfpathcurveto{\pgfqpoint{1.198989in}{1.098510in}}{\pgfqpoint{1.201303in}{1.104096in}}{\pgfqpoint{1.201303in}{1.109920in}}%
\pgfpathcurveto{\pgfqpoint{1.201303in}{1.115744in}}{\pgfqpoint{1.198989in}{1.121330in}}{\pgfqpoint{1.194871in}{1.125448in}}%
\pgfpathcurveto{\pgfqpoint{1.190753in}{1.129567in}}{\pgfqpoint{1.185167in}{1.131880in}}{\pgfqpoint{1.179343in}{1.131880in}}%
\pgfpathcurveto{\pgfqpoint{1.173519in}{1.131880in}}{\pgfqpoint{1.167933in}{1.129567in}}{\pgfqpoint{1.163815in}{1.125448in}}%
\pgfpathcurveto{\pgfqpoint{1.159697in}{1.121330in}}{\pgfqpoint{1.157383in}{1.115744in}}{\pgfqpoint{1.157383in}{1.109920in}}%
\pgfpathcurveto{\pgfqpoint{1.157383in}{1.104096in}}{\pgfqpoint{1.159697in}{1.098510in}}{\pgfqpoint{1.163815in}{1.094392in}}%
\pgfpathcurveto{\pgfqpoint{1.167933in}{1.090274in}}{\pgfqpoint{1.173519in}{1.087960in}}{\pgfqpoint{1.179343in}{1.087960in}}%
\pgfpathclose%
\pgfusepath{stroke,fill}%
\end{pgfscope}%
\begin{pgfscope}%
\pgfpathrectangle{\pgfqpoint{0.211875in}{0.211875in}}{\pgfqpoint{1.313625in}{1.279725in}}%
\pgfusepath{clip}%
\pgfsetbuttcap%
\pgfsetroundjoin%
\definecolor{currentfill}{rgb}{0.121569,0.466667,0.705882}%
\pgfsetfillcolor{currentfill}%
\pgfsetlinewidth{1.003750pt}%
\definecolor{currentstroke}{rgb}{0.121569,0.466667,0.705882}%
\pgfsetstrokecolor{currentstroke}%
\pgfsetdash{}{0pt}%
\pgfpathmoveto{\pgfqpoint{0.516604in}{0.562950in}}%
\pgfpathcurveto{\pgfqpoint{0.522428in}{0.562950in}}{\pgfqpoint{0.528014in}{0.565264in}}{\pgfqpoint{0.532132in}{0.569382in}}%
\pgfpathcurveto{\pgfqpoint{0.536251in}{0.573501in}}{\pgfqpoint{0.538564in}{0.579087in}}{\pgfqpoint{0.538564in}{0.584911in}}%
\pgfpathcurveto{\pgfqpoint{0.538564in}{0.590735in}}{\pgfqpoint{0.536251in}{0.596321in}}{\pgfqpoint{0.532132in}{0.600439in}}%
\pgfpathcurveto{\pgfqpoint{0.528014in}{0.604557in}}{\pgfqpoint{0.522428in}{0.606871in}}{\pgfqpoint{0.516604in}{0.606871in}}%
\pgfpathcurveto{\pgfqpoint{0.510780in}{0.606871in}}{\pgfqpoint{0.505194in}{0.604557in}}{\pgfqpoint{0.501076in}{0.600439in}}%
\pgfpathcurveto{\pgfqpoint{0.496958in}{0.596321in}}{\pgfqpoint{0.494644in}{0.590735in}}{\pgfqpoint{0.494644in}{0.584911in}}%
\pgfpathcurveto{\pgfqpoint{0.494644in}{0.579087in}}{\pgfqpoint{0.496958in}{0.573501in}}{\pgfqpoint{0.501076in}{0.569382in}}%
\pgfpathcurveto{\pgfqpoint{0.505194in}{0.565264in}}{\pgfqpoint{0.510780in}{0.562950in}}{\pgfqpoint{0.516604in}{0.562950in}}%
\pgfpathclose%
\pgfusepath{stroke,fill}%
\end{pgfscope}%
\begin{pgfscope}%
\pgfpathrectangle{\pgfqpoint{0.211875in}{0.211875in}}{\pgfqpoint{1.313625in}{1.279725in}}%
\pgfusepath{clip}%
\pgfsetbuttcap%
\pgfsetroundjoin%
\definecolor{currentfill}{rgb}{0.121569,0.466667,0.705882}%
\pgfsetfillcolor{currentfill}%
\pgfsetlinewidth{1.003750pt}%
\definecolor{currentstroke}{rgb}{0.121569,0.466667,0.705882}%
\pgfsetstrokecolor{currentstroke}%
\pgfsetdash{}{0pt}%
\pgfpathmoveto{\pgfqpoint{1.334683in}{0.385023in}}%
\pgfpathcurveto{\pgfqpoint{1.340507in}{0.385023in}}{\pgfqpoint{1.346093in}{0.387336in}}{\pgfqpoint{1.350211in}{0.391455in}}%
\pgfpathcurveto{\pgfqpoint{1.354329in}{0.395573in}}{\pgfqpoint{1.356643in}{0.401159in}}{\pgfqpoint{1.356643in}{0.406983in}}%
\pgfpathcurveto{\pgfqpoint{1.356643in}{0.412807in}}{\pgfqpoint{1.354329in}{0.418393in}}{\pgfqpoint{1.350211in}{0.422511in}}%
\pgfpathcurveto{\pgfqpoint{1.346093in}{0.426629in}}{\pgfqpoint{1.340507in}{0.428943in}}{\pgfqpoint{1.334683in}{0.428943in}}%
\pgfpathcurveto{\pgfqpoint{1.328859in}{0.428943in}}{\pgfqpoint{1.323273in}{0.426629in}}{\pgfqpoint{1.319155in}{0.422511in}}%
\pgfpathcurveto{\pgfqpoint{1.315036in}{0.418393in}}{\pgfqpoint{1.312723in}{0.412807in}}{\pgfqpoint{1.312723in}{0.406983in}}%
\pgfpathcurveto{\pgfqpoint{1.312723in}{0.401159in}}{\pgfqpoint{1.315036in}{0.395573in}}{\pgfqpoint{1.319155in}{0.391455in}}%
\pgfpathcurveto{\pgfqpoint{1.323273in}{0.387336in}}{\pgfqpoint{1.328859in}{0.385023in}}{\pgfqpoint{1.334683in}{0.385023in}}%
\pgfpathclose%
\pgfusepath{stroke,fill}%
\end{pgfscope}%
\begin{pgfscope}%
\pgfpathrectangle{\pgfqpoint{0.211875in}{0.211875in}}{\pgfqpoint{1.313625in}{1.279725in}}%
\pgfusepath{clip}%
\pgfsetbuttcap%
\pgfsetroundjoin%
\definecolor{currentfill}{rgb}{0.121569,0.466667,0.705882}%
\pgfsetfillcolor{currentfill}%
\pgfsetlinewidth{1.003750pt}%
\definecolor{currentstroke}{rgb}{0.121569,0.466667,0.705882}%
\pgfsetstrokecolor{currentstroke}%
\pgfsetdash{}{0pt}%
\pgfpathmoveto{\pgfqpoint{1.209717in}{1.067106in}}%
\pgfpathcurveto{\pgfqpoint{1.215540in}{1.067106in}}{\pgfqpoint{1.221127in}{1.069420in}}{\pgfqpoint{1.225245in}{1.073538in}}%
\pgfpathcurveto{\pgfqpoint{1.229363in}{1.077656in}}{\pgfqpoint{1.231677in}{1.083242in}}{\pgfqpoint{1.231677in}{1.089066in}}%
\pgfpathcurveto{\pgfqpoint{1.231677in}{1.094890in}}{\pgfqpoint{1.229363in}{1.100476in}}{\pgfqpoint{1.225245in}{1.104594in}}%
\pgfpathcurveto{\pgfqpoint{1.221127in}{1.108712in}}{\pgfqpoint{1.215540in}{1.111026in}}{\pgfqpoint{1.209717in}{1.111026in}}%
\pgfpathcurveto{\pgfqpoint{1.203893in}{1.111026in}}{\pgfqpoint{1.198306in}{1.108712in}}{\pgfqpoint{1.194188in}{1.104594in}}%
\pgfpathcurveto{\pgfqpoint{1.190070in}{1.100476in}}{\pgfqpoint{1.187756in}{1.094890in}}{\pgfqpoint{1.187756in}{1.089066in}}%
\pgfpathcurveto{\pgfqpoint{1.187756in}{1.083242in}}{\pgfqpoint{1.190070in}{1.077656in}}{\pgfqpoint{1.194188in}{1.073538in}}%
\pgfpathcurveto{\pgfqpoint{1.198306in}{1.069420in}}{\pgfqpoint{1.203893in}{1.067106in}}{\pgfqpoint{1.209717in}{1.067106in}}%
\pgfpathclose%
\pgfusepath{stroke,fill}%
\end{pgfscope}%
\begin{pgfscope}%
\pgfpathrectangle{\pgfqpoint{0.211875in}{0.211875in}}{\pgfqpoint{1.313625in}{1.279725in}}%
\pgfusepath{clip}%
\pgfsetbuttcap%
\pgfsetroundjoin%
\definecolor{currentfill}{rgb}{0.121569,0.466667,0.705882}%
\pgfsetfillcolor{currentfill}%
\pgfsetlinewidth{1.003750pt}%
\definecolor{currentstroke}{rgb}{0.121569,0.466667,0.705882}%
\pgfsetstrokecolor{currentstroke}%
\pgfsetdash{}{0pt}%
\pgfpathmoveto{\pgfqpoint{1.191166in}{0.591019in}}%
\pgfpathcurveto{\pgfqpoint{1.196990in}{0.591019in}}{\pgfqpoint{1.202576in}{0.593332in}}{\pgfqpoint{1.206694in}{0.597451in}}%
\pgfpathcurveto{\pgfqpoint{1.210813in}{0.601569in}}{\pgfqpoint{1.213126in}{0.607155in}}{\pgfqpoint{1.213126in}{0.612979in}}%
\pgfpathcurveto{\pgfqpoint{1.213126in}{0.618803in}}{\pgfqpoint{1.210813in}{0.624389in}}{\pgfqpoint{1.206694in}{0.628507in}}%
\pgfpathcurveto{\pgfqpoint{1.202576in}{0.632625in}}{\pgfqpoint{1.196990in}{0.634939in}}{\pgfqpoint{1.191166in}{0.634939in}}%
\pgfpathcurveto{\pgfqpoint{1.185342in}{0.634939in}}{\pgfqpoint{1.179756in}{0.632625in}}{\pgfqpoint{1.175638in}{0.628507in}}%
\pgfpathcurveto{\pgfqpoint{1.171520in}{0.624389in}}{\pgfqpoint{1.169206in}{0.618803in}}{\pgfqpoint{1.169206in}{0.612979in}}%
\pgfpathcurveto{\pgfqpoint{1.169206in}{0.607155in}}{\pgfqpoint{1.171520in}{0.601569in}}{\pgfqpoint{1.175638in}{0.597451in}}%
\pgfpathcurveto{\pgfqpoint{1.179756in}{0.593332in}}{\pgfqpoint{1.185342in}{0.591019in}}{\pgfqpoint{1.191166in}{0.591019in}}%
\pgfpathclose%
\pgfusepath{stroke,fill}%
\end{pgfscope}%
\begin{pgfscope}%
\pgfpathrectangle{\pgfqpoint{0.211875in}{0.211875in}}{\pgfqpoint{1.313625in}{1.279725in}}%
\pgfusepath{clip}%
\pgfsetbuttcap%
\pgfsetroundjoin%
\definecolor{currentfill}{rgb}{0.121569,0.466667,0.705882}%
\pgfsetfillcolor{currentfill}%
\pgfsetlinewidth{1.003750pt}%
\definecolor{currentstroke}{rgb}{0.121569,0.466667,0.705882}%
\pgfsetstrokecolor{currentstroke}%
\pgfsetdash{}{0pt}%
\pgfpathmoveto{\pgfqpoint{0.810413in}{1.113956in}}%
\pgfpathcurveto{\pgfqpoint{0.816237in}{1.113956in}}{\pgfqpoint{0.821823in}{1.116270in}}{\pgfqpoint{0.825942in}{1.120388in}}%
\pgfpathcurveto{\pgfqpoint{0.830060in}{1.124506in}}{\pgfqpoint{0.832374in}{1.130092in}}{\pgfqpoint{0.832374in}{1.135916in}}%
\pgfpathcurveto{\pgfqpoint{0.832374in}{1.141740in}}{\pgfqpoint{0.830060in}{1.147326in}}{\pgfqpoint{0.825942in}{1.151445in}}%
\pgfpathcurveto{\pgfqpoint{0.821823in}{1.155563in}}{\pgfqpoint{0.816237in}{1.157877in}}{\pgfqpoint{0.810413in}{1.157877in}}%
\pgfpathcurveto{\pgfqpoint{0.804589in}{1.157877in}}{\pgfqpoint{0.799003in}{1.155563in}}{\pgfqpoint{0.794885in}{1.151445in}}%
\pgfpathcurveto{\pgfqpoint{0.790767in}{1.147326in}}{\pgfqpoint{0.788453in}{1.141740in}}{\pgfqpoint{0.788453in}{1.135916in}}%
\pgfpathcurveto{\pgfqpoint{0.788453in}{1.130092in}}{\pgfqpoint{0.790767in}{1.124506in}}{\pgfqpoint{0.794885in}{1.120388in}}%
\pgfpathcurveto{\pgfqpoint{0.799003in}{1.116270in}}{\pgfqpoint{0.804589in}{1.113956in}}{\pgfqpoint{0.810413in}{1.113956in}}%
\pgfpathclose%
\pgfusepath{stroke,fill}%
\end{pgfscope}%
\begin{pgfscope}%
\pgfpathrectangle{\pgfqpoint{0.211875in}{0.211875in}}{\pgfqpoint{1.313625in}{1.279725in}}%
\pgfusepath{clip}%
\pgfsetbuttcap%
\pgfsetroundjoin%
\definecolor{currentfill}{rgb}{0.121569,0.466667,0.705882}%
\pgfsetfillcolor{currentfill}%
\pgfsetlinewidth{1.003750pt}%
\definecolor{currentstroke}{rgb}{0.121569,0.466667,0.705882}%
\pgfsetstrokecolor{currentstroke}%
\pgfsetdash{}{0pt}%
\pgfpathmoveto{\pgfqpoint{1.146674in}{1.181452in}}%
\pgfpathcurveto{\pgfqpoint{1.152497in}{1.181452in}}{\pgfqpoint{1.158084in}{1.183765in}}{\pgfqpoint{1.162202in}{1.187884in}}%
\pgfpathcurveto{\pgfqpoint{1.166320in}{1.192002in}}{\pgfqpoint{1.168634in}{1.197588in}}{\pgfqpoint{1.168634in}{1.203412in}}%
\pgfpathcurveto{\pgfqpoint{1.168634in}{1.209236in}}{\pgfqpoint{1.166320in}{1.214822in}}{\pgfqpoint{1.162202in}{1.218940in}}%
\pgfpathcurveto{\pgfqpoint{1.158084in}{1.223058in}}{\pgfqpoint{1.152497in}{1.225372in}}{\pgfqpoint{1.146674in}{1.225372in}}%
\pgfpathcurveto{\pgfqpoint{1.140850in}{1.225372in}}{\pgfqpoint{1.135263in}{1.223058in}}{\pgfqpoint{1.131145in}{1.218940in}}%
\pgfpathcurveto{\pgfqpoint{1.127027in}{1.214822in}}{\pgfqpoint{1.124713in}{1.209236in}}{\pgfqpoint{1.124713in}{1.203412in}}%
\pgfpathcurveto{\pgfqpoint{1.124713in}{1.197588in}}{\pgfqpoint{1.127027in}{1.192002in}}{\pgfqpoint{1.131145in}{1.187884in}}%
\pgfpathcurveto{\pgfqpoint{1.135263in}{1.183765in}}{\pgfqpoint{1.140850in}{1.181452in}}{\pgfqpoint{1.146674in}{1.181452in}}%
\pgfpathclose%
\pgfusepath{stroke,fill}%
\end{pgfscope}%
\begin{pgfscope}%
\pgfpathrectangle{\pgfqpoint{0.211875in}{0.211875in}}{\pgfqpoint{1.313625in}{1.279725in}}%
\pgfusepath{clip}%
\pgfsetbuttcap%
\pgfsetroundjoin%
\definecolor{currentfill}{rgb}{0.121569,0.466667,0.705882}%
\pgfsetfillcolor{currentfill}%
\pgfsetlinewidth{1.003750pt}%
\definecolor{currentstroke}{rgb}{0.121569,0.466667,0.705882}%
\pgfsetstrokecolor{currentstroke}%
\pgfsetdash{}{0pt}%
\pgfpathmoveto{\pgfqpoint{1.143334in}{0.699775in}}%
\pgfpathcurveto{\pgfqpoint{1.149157in}{0.699775in}}{\pgfqpoint{1.154744in}{0.702089in}}{\pgfqpoint{1.158862in}{0.706207in}}%
\pgfpathcurveto{\pgfqpoint{1.162980in}{0.710325in}}{\pgfqpoint{1.165294in}{0.715911in}}{\pgfqpoint{1.165294in}{0.721735in}}%
\pgfpathcurveto{\pgfqpoint{1.165294in}{0.727559in}}{\pgfqpoint{1.162980in}{0.733145in}}{\pgfqpoint{1.158862in}{0.737263in}}%
\pgfpathcurveto{\pgfqpoint{1.154744in}{0.741381in}}{\pgfqpoint{1.149157in}{0.743695in}}{\pgfqpoint{1.143334in}{0.743695in}}%
\pgfpathcurveto{\pgfqpoint{1.137510in}{0.743695in}}{\pgfqpoint{1.131923in}{0.741381in}}{\pgfqpoint{1.127805in}{0.737263in}}%
\pgfpathcurveto{\pgfqpoint{1.123687in}{0.733145in}}{\pgfqpoint{1.121373in}{0.727559in}}{\pgfqpoint{1.121373in}{0.721735in}}%
\pgfpathcurveto{\pgfqpoint{1.121373in}{0.715911in}}{\pgfqpoint{1.123687in}{0.710325in}}{\pgfqpoint{1.127805in}{0.706207in}}%
\pgfpathcurveto{\pgfqpoint{1.131923in}{0.702089in}}{\pgfqpoint{1.137510in}{0.699775in}}{\pgfqpoint{1.143334in}{0.699775in}}%
\pgfpathclose%
\pgfusepath{stroke,fill}%
\end{pgfscope}%
\begin{pgfscope}%
\pgfpathrectangle{\pgfqpoint{0.211875in}{0.211875in}}{\pgfqpoint{1.313625in}{1.279725in}}%
\pgfusepath{clip}%
\pgfsetbuttcap%
\pgfsetroundjoin%
\definecolor{currentfill}{rgb}{0.121569,0.466667,0.705882}%
\pgfsetfillcolor{currentfill}%
\pgfsetlinewidth{1.003750pt}%
\definecolor{currentstroke}{rgb}{0.121569,0.466667,0.705882}%
\pgfsetstrokecolor{currentstroke}%
\pgfsetdash{}{0pt}%
\pgfpathmoveto{\pgfqpoint{1.312810in}{0.830696in}}%
\pgfpathcurveto{\pgfqpoint{1.318634in}{0.830696in}}{\pgfqpoint{1.324220in}{0.833010in}}{\pgfqpoint{1.328338in}{0.837128in}}%
\pgfpathcurveto{\pgfqpoint{1.332457in}{0.841246in}}{\pgfqpoint{1.334770in}{0.846832in}}{\pgfqpoint{1.334770in}{0.852656in}}%
\pgfpathcurveto{\pgfqpoint{1.334770in}{0.858480in}}{\pgfqpoint{1.332457in}{0.864066in}}{\pgfqpoint{1.328338in}{0.868184in}}%
\pgfpathcurveto{\pgfqpoint{1.324220in}{0.872303in}}{\pgfqpoint{1.318634in}{0.874616in}}{\pgfqpoint{1.312810in}{0.874616in}}%
\pgfpathcurveto{\pgfqpoint{1.306986in}{0.874616in}}{\pgfqpoint{1.301400in}{0.872303in}}{\pgfqpoint{1.297282in}{0.868184in}}%
\pgfpathcurveto{\pgfqpoint{1.293164in}{0.864066in}}{\pgfqpoint{1.290850in}{0.858480in}}{\pgfqpoint{1.290850in}{0.852656in}}%
\pgfpathcurveto{\pgfqpoint{1.290850in}{0.846832in}}{\pgfqpoint{1.293164in}{0.841246in}}{\pgfqpoint{1.297282in}{0.837128in}}%
\pgfpathcurveto{\pgfqpoint{1.301400in}{0.833010in}}{\pgfqpoint{1.306986in}{0.830696in}}{\pgfqpoint{1.312810in}{0.830696in}}%
\pgfpathclose%
\pgfusepath{stroke,fill}%
\end{pgfscope}%
\begin{pgfscope}%
\pgfpathrectangle{\pgfqpoint{0.211875in}{0.211875in}}{\pgfqpoint{1.313625in}{1.279725in}}%
\pgfusepath{clip}%
\pgfsetbuttcap%
\pgfsetroundjoin%
\definecolor{currentfill}{rgb}{0.121569,0.466667,0.705882}%
\pgfsetfillcolor{currentfill}%
\pgfsetlinewidth{1.003750pt}%
\definecolor{currentstroke}{rgb}{0.121569,0.466667,0.705882}%
\pgfsetstrokecolor{currentstroke}%
\pgfsetdash{}{0pt}%
\pgfpathmoveto{\pgfqpoint{1.245626in}{0.847431in}}%
\pgfpathcurveto{\pgfqpoint{1.251450in}{0.847431in}}{\pgfqpoint{1.257036in}{0.849745in}}{\pgfqpoint{1.261154in}{0.853863in}}%
\pgfpathcurveto{\pgfqpoint{1.265273in}{0.857981in}}{\pgfqpoint{1.267586in}{0.863567in}}{\pgfqpoint{1.267586in}{0.869391in}}%
\pgfpathcurveto{\pgfqpoint{1.267586in}{0.875215in}}{\pgfqpoint{1.265273in}{0.880801in}}{\pgfqpoint{1.261154in}{0.884919in}}%
\pgfpathcurveto{\pgfqpoint{1.257036in}{0.889037in}}{\pgfqpoint{1.251450in}{0.891351in}}{\pgfqpoint{1.245626in}{0.891351in}}%
\pgfpathcurveto{\pgfqpoint{1.239802in}{0.891351in}}{\pgfqpoint{1.234216in}{0.889037in}}{\pgfqpoint{1.230098in}{0.884919in}}%
\pgfpathcurveto{\pgfqpoint{1.225980in}{0.880801in}}{\pgfqpoint{1.223666in}{0.875215in}}{\pgfqpoint{1.223666in}{0.869391in}}%
\pgfpathcurveto{\pgfqpoint{1.223666in}{0.863567in}}{\pgfqpoint{1.225980in}{0.857981in}}{\pgfqpoint{1.230098in}{0.853863in}}%
\pgfpathcurveto{\pgfqpoint{1.234216in}{0.849745in}}{\pgfqpoint{1.239802in}{0.847431in}}{\pgfqpoint{1.245626in}{0.847431in}}%
\pgfpathclose%
\pgfusepath{stroke,fill}%
\end{pgfscope}%
\begin{pgfscope}%
\pgfpathrectangle{\pgfqpoint{0.211875in}{0.211875in}}{\pgfqpoint{1.313625in}{1.279725in}}%
\pgfusepath{clip}%
\pgfsetbuttcap%
\pgfsetroundjoin%
\definecolor{currentfill}{rgb}{0.121569,0.466667,0.705882}%
\pgfsetfillcolor{currentfill}%
\pgfsetlinewidth{1.003750pt}%
\definecolor{currentstroke}{rgb}{0.121569,0.466667,0.705882}%
\pgfsetstrokecolor{currentstroke}%
\pgfsetdash{}{0pt}%
\pgfpathmoveto{\pgfqpoint{1.281773in}{0.834945in}}%
\pgfpathcurveto{\pgfqpoint{1.287597in}{0.834945in}}{\pgfqpoint{1.293184in}{0.837259in}}{\pgfqpoint{1.297302in}{0.841377in}}%
\pgfpathcurveto{\pgfqpoint{1.301420in}{0.845495in}}{\pgfqpoint{1.303734in}{0.851081in}}{\pgfqpoint{1.303734in}{0.856905in}}%
\pgfpathcurveto{\pgfqpoint{1.303734in}{0.862729in}}{\pgfqpoint{1.301420in}{0.868315in}}{\pgfqpoint{1.297302in}{0.872433in}}%
\pgfpathcurveto{\pgfqpoint{1.293184in}{0.876551in}}{\pgfqpoint{1.287597in}{0.878865in}}{\pgfqpoint{1.281773in}{0.878865in}}%
\pgfpathcurveto{\pgfqpoint{1.275950in}{0.878865in}}{\pgfqpoint{1.270363in}{0.876551in}}{\pgfqpoint{1.266245in}{0.872433in}}%
\pgfpathcurveto{\pgfqpoint{1.262127in}{0.868315in}}{\pgfqpoint{1.259813in}{0.862729in}}{\pgfqpoint{1.259813in}{0.856905in}}%
\pgfpathcurveto{\pgfqpoint{1.259813in}{0.851081in}}{\pgfqpoint{1.262127in}{0.845495in}}{\pgfqpoint{1.266245in}{0.841377in}}%
\pgfpathcurveto{\pgfqpoint{1.270363in}{0.837259in}}{\pgfqpoint{1.275950in}{0.834945in}}{\pgfqpoint{1.281773in}{0.834945in}}%
\pgfpathclose%
\pgfusepath{stroke,fill}%
\end{pgfscope}%
\begin{pgfscope}%
\pgfpathrectangle{\pgfqpoint{0.211875in}{0.211875in}}{\pgfqpoint{1.313625in}{1.279725in}}%
\pgfusepath{clip}%
\pgfsetbuttcap%
\pgfsetroundjoin%
\definecolor{currentfill}{rgb}{0.121569,0.466667,0.705882}%
\pgfsetfillcolor{currentfill}%
\pgfsetlinewidth{1.003750pt}%
\definecolor{currentstroke}{rgb}{0.121569,0.466667,0.705882}%
\pgfsetstrokecolor{currentstroke}%
\pgfsetdash{}{0pt}%
\pgfpathmoveto{\pgfqpoint{1.141051in}{0.914774in}}%
\pgfpathcurveto{\pgfqpoint{1.146875in}{0.914774in}}{\pgfqpoint{1.152461in}{0.917088in}}{\pgfqpoint{1.156579in}{0.921206in}}%
\pgfpathcurveto{\pgfqpoint{1.160697in}{0.925324in}}{\pgfqpoint{1.163011in}{0.930910in}}{\pgfqpoint{1.163011in}{0.936734in}}%
\pgfpathcurveto{\pgfqpoint{1.163011in}{0.942558in}}{\pgfqpoint{1.160697in}{0.948144in}}{\pgfqpoint{1.156579in}{0.952262in}}%
\pgfpathcurveto{\pgfqpoint{1.152461in}{0.956381in}}{\pgfqpoint{1.146875in}{0.958694in}}{\pgfqpoint{1.141051in}{0.958694in}}%
\pgfpathcurveto{\pgfqpoint{1.135227in}{0.958694in}}{\pgfqpoint{1.129641in}{0.956381in}}{\pgfqpoint{1.125523in}{0.952262in}}%
\pgfpathcurveto{\pgfqpoint{1.121405in}{0.948144in}}{\pgfqpoint{1.119091in}{0.942558in}}{\pgfqpoint{1.119091in}{0.936734in}}%
\pgfpathcurveto{\pgfqpoint{1.119091in}{0.930910in}}{\pgfqpoint{1.121405in}{0.925324in}}{\pgfqpoint{1.125523in}{0.921206in}}%
\pgfpathcurveto{\pgfqpoint{1.129641in}{0.917088in}}{\pgfqpoint{1.135227in}{0.914774in}}{\pgfqpoint{1.141051in}{0.914774in}}%
\pgfpathclose%
\pgfusepath{stroke,fill}%
\end{pgfscope}%
\begin{pgfscope}%
\pgfpathrectangle{\pgfqpoint{0.211875in}{0.211875in}}{\pgfqpoint{1.313625in}{1.279725in}}%
\pgfusepath{clip}%
\pgfsetbuttcap%
\pgfsetroundjoin%
\definecolor{currentfill}{rgb}{0.121569,0.466667,0.705882}%
\pgfsetfillcolor{currentfill}%
\pgfsetlinewidth{1.003750pt}%
\definecolor{currentstroke}{rgb}{0.121569,0.466667,0.705882}%
\pgfsetstrokecolor{currentstroke}%
\pgfsetdash{}{0pt}%
\pgfpathmoveto{\pgfqpoint{0.878082in}{1.391844in}}%
\pgfpathcurveto{\pgfqpoint{0.883906in}{1.391844in}}{\pgfqpoint{0.889492in}{1.394158in}}{\pgfqpoint{0.893610in}{1.398276in}}%
\pgfpathcurveto{\pgfqpoint{0.897728in}{1.402394in}}{\pgfqpoint{0.900042in}{1.407981in}}{\pgfqpoint{0.900042in}{1.413805in}}%
\pgfpathcurveto{\pgfqpoint{0.900042in}{1.419629in}}{\pgfqpoint{0.897728in}{1.425215in}}{\pgfqpoint{0.893610in}{1.429333in}}%
\pgfpathcurveto{\pgfqpoint{0.889492in}{1.433451in}}{\pgfqpoint{0.883906in}{1.435765in}}{\pgfqpoint{0.878082in}{1.435765in}}%
\pgfpathcurveto{\pgfqpoint{0.872258in}{1.435765in}}{\pgfqpoint{0.866672in}{1.433451in}}{\pgfqpoint{0.862554in}{1.429333in}}%
\pgfpathcurveto{\pgfqpoint{0.858435in}{1.425215in}}{\pgfqpoint{0.856122in}{1.419629in}}{\pgfqpoint{0.856122in}{1.413805in}}%
\pgfpathcurveto{\pgfqpoint{0.856122in}{1.407981in}}{\pgfqpoint{0.858435in}{1.402394in}}{\pgfqpoint{0.862554in}{1.398276in}}%
\pgfpathcurveto{\pgfqpoint{0.866672in}{1.394158in}}{\pgfqpoint{0.872258in}{1.391844in}}{\pgfqpoint{0.878082in}{1.391844in}}%
\pgfpathclose%
\pgfusepath{stroke,fill}%
\end{pgfscope}%
\begin{pgfscope}%
\pgfpathrectangle{\pgfqpoint{0.211875in}{0.211875in}}{\pgfqpoint{1.313625in}{1.279725in}}%
\pgfusepath{clip}%
\pgfsetbuttcap%
\pgfsetroundjoin%
\definecolor{currentfill}{rgb}{0.121569,0.466667,0.705882}%
\pgfsetfillcolor{currentfill}%
\pgfsetlinewidth{1.003750pt}%
\definecolor{currentstroke}{rgb}{0.121569,0.466667,0.705882}%
\pgfsetstrokecolor{currentstroke}%
\pgfsetdash{}{0pt}%
\pgfpathmoveto{\pgfqpoint{1.142920in}{0.942780in}}%
\pgfpathcurveto{\pgfqpoint{1.148744in}{0.942780in}}{\pgfqpoint{1.154330in}{0.945094in}}{\pgfqpoint{1.158449in}{0.949212in}}%
\pgfpathcurveto{\pgfqpoint{1.162567in}{0.953330in}}{\pgfqpoint{1.164881in}{0.958916in}}{\pgfqpoint{1.164881in}{0.964740in}}%
\pgfpathcurveto{\pgfqpoint{1.164881in}{0.970564in}}{\pgfqpoint{1.162567in}{0.976150in}}{\pgfqpoint{1.158449in}{0.980268in}}%
\pgfpathcurveto{\pgfqpoint{1.154330in}{0.984386in}}{\pgfqpoint{1.148744in}{0.986700in}}{\pgfqpoint{1.142920in}{0.986700in}}%
\pgfpathcurveto{\pgfqpoint{1.137096in}{0.986700in}}{\pgfqpoint{1.131510in}{0.984386in}}{\pgfqpoint{1.127392in}{0.980268in}}%
\pgfpathcurveto{\pgfqpoint{1.123274in}{0.976150in}}{\pgfqpoint{1.120960in}{0.970564in}}{\pgfqpoint{1.120960in}{0.964740in}}%
\pgfpathcurveto{\pgfqpoint{1.120960in}{0.958916in}}{\pgfqpoint{1.123274in}{0.953330in}}{\pgfqpoint{1.127392in}{0.949212in}}%
\pgfpathcurveto{\pgfqpoint{1.131510in}{0.945094in}}{\pgfqpoint{1.137096in}{0.942780in}}{\pgfqpoint{1.142920in}{0.942780in}}%
\pgfpathclose%
\pgfusepath{stroke,fill}%
\end{pgfscope}%
\begin{pgfscope}%
\pgfpathrectangle{\pgfqpoint{0.211875in}{0.211875in}}{\pgfqpoint{1.313625in}{1.279725in}}%
\pgfusepath{clip}%
\pgfsetbuttcap%
\pgfsetroundjoin%
\definecolor{currentfill}{rgb}{0.121569,0.466667,0.705882}%
\pgfsetfillcolor{currentfill}%
\pgfsetlinewidth{1.003750pt}%
\definecolor{currentstroke}{rgb}{0.121569,0.466667,0.705882}%
\pgfsetstrokecolor{currentstroke}%
\pgfsetdash{}{0pt}%
\pgfpathmoveto{\pgfqpoint{1.151719in}{0.937355in}}%
\pgfpathcurveto{\pgfqpoint{1.157543in}{0.937355in}}{\pgfqpoint{1.163129in}{0.939669in}}{\pgfqpoint{1.167247in}{0.943787in}}%
\pgfpathcurveto{\pgfqpoint{1.171365in}{0.947905in}}{\pgfqpoint{1.173679in}{0.953491in}}{\pgfqpoint{1.173679in}{0.959315in}}%
\pgfpathcurveto{\pgfqpoint{1.173679in}{0.965139in}}{\pgfqpoint{1.171365in}{0.970725in}}{\pgfqpoint{1.167247in}{0.974844in}}%
\pgfpathcurveto{\pgfqpoint{1.163129in}{0.978962in}}{\pgfqpoint{1.157543in}{0.981276in}}{\pgfqpoint{1.151719in}{0.981276in}}%
\pgfpathcurveto{\pgfqpoint{1.145895in}{0.981276in}}{\pgfqpoint{1.140309in}{0.978962in}}{\pgfqpoint{1.136191in}{0.974844in}}%
\pgfpathcurveto{\pgfqpoint{1.132073in}{0.970725in}}{\pgfqpoint{1.129759in}{0.965139in}}{\pgfqpoint{1.129759in}{0.959315in}}%
\pgfpathcurveto{\pgfqpoint{1.129759in}{0.953491in}}{\pgfqpoint{1.132073in}{0.947905in}}{\pgfqpoint{1.136191in}{0.943787in}}%
\pgfpathcurveto{\pgfqpoint{1.140309in}{0.939669in}}{\pgfqpoint{1.145895in}{0.937355in}}{\pgfqpoint{1.151719in}{0.937355in}}%
\pgfpathclose%
\pgfusepath{stroke,fill}%
\end{pgfscope}%
\begin{pgfscope}%
\pgfpathrectangle{\pgfqpoint{0.211875in}{0.211875in}}{\pgfqpoint{1.313625in}{1.279725in}}%
\pgfusepath{clip}%
\pgfsetbuttcap%
\pgfsetroundjoin%
\definecolor{currentfill}{rgb}{0.121569,0.466667,0.705882}%
\pgfsetfillcolor{currentfill}%
\pgfsetlinewidth{1.003750pt}%
\definecolor{currentstroke}{rgb}{0.121569,0.466667,0.705882}%
\pgfsetstrokecolor{currentstroke}%
\pgfsetdash{}{0pt}%
\pgfpathmoveto{\pgfqpoint{1.146424in}{0.938779in}}%
\pgfpathcurveto{\pgfqpoint{1.152248in}{0.938779in}}{\pgfqpoint{1.157835in}{0.941093in}}{\pgfqpoint{1.161953in}{0.945211in}}%
\pgfpathcurveto{\pgfqpoint{1.166071in}{0.949329in}}{\pgfqpoint{1.168385in}{0.954915in}}{\pgfqpoint{1.168385in}{0.960739in}}%
\pgfpathcurveto{\pgfqpoint{1.168385in}{0.966563in}}{\pgfqpoint{1.166071in}{0.972149in}}{\pgfqpoint{1.161953in}{0.976268in}}%
\pgfpathcurveto{\pgfqpoint{1.157835in}{0.980386in}}{\pgfqpoint{1.152248in}{0.982700in}}{\pgfqpoint{1.146424in}{0.982700in}}%
\pgfpathcurveto{\pgfqpoint{1.140601in}{0.982700in}}{\pgfqpoint{1.135014in}{0.980386in}}{\pgfqpoint{1.130896in}{0.976268in}}%
\pgfpathcurveto{\pgfqpoint{1.126778in}{0.972149in}}{\pgfqpoint{1.124464in}{0.966563in}}{\pgfqpoint{1.124464in}{0.960739in}}%
\pgfpathcurveto{\pgfqpoint{1.124464in}{0.954915in}}{\pgfqpoint{1.126778in}{0.949329in}}{\pgfqpoint{1.130896in}{0.945211in}}%
\pgfpathcurveto{\pgfqpoint{1.135014in}{0.941093in}}{\pgfqpoint{1.140601in}{0.938779in}}{\pgfqpoint{1.146424in}{0.938779in}}%
\pgfpathclose%
\pgfusepath{stroke,fill}%
\end{pgfscope}%
\begin{pgfscope}%
\pgfpathrectangle{\pgfqpoint{0.211875in}{0.211875in}}{\pgfqpoint{1.313625in}{1.279725in}}%
\pgfusepath{clip}%
\pgfsetbuttcap%
\pgfsetroundjoin%
\definecolor{currentfill}{rgb}{0.121569,0.466667,0.705882}%
\pgfsetfillcolor{currentfill}%
\pgfsetlinewidth{1.003750pt}%
\definecolor{currentstroke}{rgb}{0.121569,0.466667,0.705882}%
\pgfsetstrokecolor{currentstroke}%
\pgfsetdash{}{0pt}%
\pgfpathmoveto{\pgfqpoint{1.149102in}{0.935223in}}%
\pgfpathcurveto{\pgfqpoint{1.154926in}{0.935223in}}{\pgfqpoint{1.160512in}{0.937537in}}{\pgfqpoint{1.164630in}{0.941655in}}%
\pgfpathcurveto{\pgfqpoint{1.168748in}{0.945773in}}{\pgfqpoint{1.171062in}{0.951359in}}{\pgfqpoint{1.171062in}{0.957183in}}%
\pgfpathcurveto{\pgfqpoint{1.171062in}{0.963007in}}{\pgfqpoint{1.168748in}{0.968593in}}{\pgfqpoint{1.164630in}{0.972711in}}%
\pgfpathcurveto{\pgfqpoint{1.160512in}{0.976829in}}{\pgfqpoint{1.154926in}{0.979143in}}{\pgfqpoint{1.149102in}{0.979143in}}%
\pgfpathcurveto{\pgfqpoint{1.143278in}{0.979143in}}{\pgfqpoint{1.137692in}{0.976829in}}{\pgfqpoint{1.133573in}{0.972711in}}%
\pgfpathcurveto{\pgfqpoint{1.129455in}{0.968593in}}{\pgfqpoint{1.127141in}{0.963007in}}{\pgfqpoint{1.127141in}{0.957183in}}%
\pgfpathcurveto{\pgfqpoint{1.127141in}{0.951359in}}{\pgfqpoint{1.129455in}{0.945773in}}{\pgfqpoint{1.133573in}{0.941655in}}%
\pgfpathcurveto{\pgfqpoint{1.137692in}{0.937537in}}{\pgfqpoint{1.143278in}{0.935223in}}{\pgfqpoint{1.149102in}{0.935223in}}%
\pgfpathclose%
\pgfusepath{stroke,fill}%
\end{pgfscope}%
\begin{pgfscope}%
\pgfpathrectangle{\pgfqpoint{0.211875in}{0.211875in}}{\pgfqpoint{1.313625in}{1.279725in}}%
\pgfusepath{clip}%
\pgfsetbuttcap%
\pgfsetroundjoin%
\definecolor{currentfill}{rgb}{0.121569,0.466667,0.705882}%
\pgfsetfillcolor{currentfill}%
\pgfsetlinewidth{1.003750pt}%
\definecolor{currentstroke}{rgb}{0.121569,0.466667,0.705882}%
\pgfsetstrokecolor{currentstroke}%
\pgfsetdash{}{0pt}%
\pgfpathmoveto{\pgfqpoint{0.294999in}{1.391844in}}%
\pgfpathcurveto{\pgfqpoint{0.300823in}{1.391844in}}{\pgfqpoint{0.306409in}{1.394158in}}{\pgfqpoint{0.310527in}{1.398276in}}%
\pgfpathcurveto{\pgfqpoint{0.314646in}{1.402394in}}{\pgfqpoint{0.316959in}{1.407981in}}{\pgfqpoint{0.316959in}{1.413805in}}%
\pgfpathcurveto{\pgfqpoint{0.316959in}{1.419629in}}{\pgfqpoint{0.314646in}{1.425215in}}{\pgfqpoint{0.310527in}{1.429333in}}%
\pgfpathcurveto{\pgfqpoint{0.306409in}{1.433451in}}{\pgfqpoint{0.300823in}{1.435765in}}{\pgfqpoint{0.294999in}{1.435765in}}%
\pgfpathcurveto{\pgfqpoint{0.289175in}{1.435765in}}{\pgfqpoint{0.283589in}{1.433451in}}{\pgfqpoint{0.279471in}{1.429333in}}%
\pgfpathcurveto{\pgfqpoint{0.275353in}{1.425215in}}{\pgfqpoint{0.273039in}{1.419629in}}{\pgfqpoint{0.273039in}{1.413805in}}%
\pgfpathcurveto{\pgfqpoint{0.273039in}{1.407981in}}{\pgfqpoint{0.275353in}{1.402394in}}{\pgfqpoint{0.279471in}{1.398276in}}%
\pgfpathcurveto{\pgfqpoint{0.283589in}{1.394158in}}{\pgfqpoint{0.289175in}{1.391844in}}{\pgfqpoint{0.294999in}{1.391844in}}%
\pgfpathclose%
\pgfusepath{stroke,fill}%
\end{pgfscope}%
\begin{pgfscope}%
\pgfpathrectangle{\pgfqpoint{0.211875in}{0.211875in}}{\pgfqpoint{1.313625in}{1.279725in}}%
\pgfusepath{clip}%
\pgfsetbuttcap%
\pgfsetroundjoin%
\definecolor{currentfill}{rgb}{0.121569,0.466667,0.705882}%
\pgfsetfillcolor{currentfill}%
\pgfsetlinewidth{1.003750pt}%
\definecolor{currentstroke}{rgb}{0.121569,0.466667,0.705882}%
\pgfsetstrokecolor{currentstroke}%
\pgfsetdash{}{0pt}%
\pgfpathmoveto{\pgfqpoint{1.151139in}{0.938215in}}%
\pgfpathcurveto{\pgfqpoint{1.156963in}{0.938215in}}{\pgfqpoint{1.162550in}{0.940529in}}{\pgfqpoint{1.166668in}{0.944647in}}%
\pgfpathcurveto{\pgfqpoint{1.170786in}{0.948765in}}{\pgfqpoint{1.173100in}{0.954351in}}{\pgfqpoint{1.173100in}{0.960175in}}%
\pgfpathcurveto{\pgfqpoint{1.173100in}{0.965999in}}{\pgfqpoint{1.170786in}{0.971585in}}{\pgfqpoint{1.166668in}{0.975703in}}%
\pgfpathcurveto{\pgfqpoint{1.162550in}{0.979822in}}{\pgfqpoint{1.156963in}{0.982135in}}{\pgfqpoint{1.151139in}{0.982135in}}%
\pgfpathcurveto{\pgfqpoint{1.145316in}{0.982135in}}{\pgfqpoint{1.139729in}{0.979822in}}{\pgfqpoint{1.135611in}{0.975703in}}%
\pgfpathcurveto{\pgfqpoint{1.131493in}{0.971585in}}{\pgfqpoint{1.129179in}{0.965999in}}{\pgfqpoint{1.129179in}{0.960175in}}%
\pgfpathcurveto{\pgfqpoint{1.129179in}{0.954351in}}{\pgfqpoint{1.131493in}{0.948765in}}{\pgfqpoint{1.135611in}{0.944647in}}%
\pgfpathcurveto{\pgfqpoint{1.139729in}{0.940529in}}{\pgfqpoint{1.145316in}{0.938215in}}{\pgfqpoint{1.151139in}{0.938215in}}%
\pgfpathclose%
\pgfusepath{stroke,fill}%
\end{pgfscope}%
\begin{pgfscope}%
\pgfpathrectangle{\pgfqpoint{0.211875in}{0.211875in}}{\pgfqpoint{1.313625in}{1.279725in}}%
\pgfusepath{clip}%
\pgfsetbuttcap%
\pgfsetroundjoin%
\definecolor{currentfill}{rgb}{0.121569,0.466667,0.705882}%
\pgfsetfillcolor{currentfill}%
\pgfsetlinewidth{1.003750pt}%
\definecolor{currentstroke}{rgb}{0.121569,0.466667,0.705882}%
\pgfsetstrokecolor{currentstroke}%
\pgfsetdash{}{0pt}%
\pgfpathmoveto{\pgfqpoint{1.147381in}{0.937785in}}%
\pgfpathcurveto{\pgfqpoint{1.153205in}{0.937785in}}{\pgfqpoint{1.158791in}{0.940099in}}{\pgfqpoint{1.162909in}{0.944217in}}%
\pgfpathcurveto{\pgfqpoint{1.167027in}{0.948335in}}{\pgfqpoint{1.169341in}{0.953921in}}{\pgfqpoint{1.169341in}{0.959745in}}%
\pgfpathcurveto{\pgfqpoint{1.169341in}{0.965569in}}{\pgfqpoint{1.167027in}{0.971156in}}{\pgfqpoint{1.162909in}{0.975274in}}%
\pgfpathcurveto{\pgfqpoint{1.158791in}{0.979392in}}{\pgfqpoint{1.153205in}{0.981706in}}{\pgfqpoint{1.147381in}{0.981706in}}%
\pgfpathcurveto{\pgfqpoint{1.141557in}{0.981706in}}{\pgfqpoint{1.135971in}{0.979392in}}{\pgfqpoint{1.131852in}{0.975274in}}%
\pgfpathcurveto{\pgfqpoint{1.127734in}{0.971156in}}{\pgfqpoint{1.125420in}{0.965569in}}{\pgfqpoint{1.125420in}{0.959745in}}%
\pgfpathcurveto{\pgfqpoint{1.125420in}{0.953921in}}{\pgfqpoint{1.127734in}{0.948335in}}{\pgfqpoint{1.131852in}{0.944217in}}%
\pgfpathcurveto{\pgfqpoint{1.135971in}{0.940099in}}{\pgfqpoint{1.141557in}{0.937785in}}{\pgfqpoint{1.147381in}{0.937785in}}%
\pgfpathclose%
\pgfusepath{stroke,fill}%
\end{pgfscope}%
\begin{pgfscope}%
\pgfpathrectangle{\pgfqpoint{0.211875in}{0.211875in}}{\pgfqpoint{1.313625in}{1.279725in}}%
\pgfusepath{clip}%
\pgfsetbuttcap%
\pgfsetroundjoin%
\definecolor{currentfill}{rgb}{0.121569,0.466667,0.705882}%
\pgfsetfillcolor{currentfill}%
\pgfsetlinewidth{1.003750pt}%
\definecolor{currentstroke}{rgb}{0.121569,0.466667,0.705882}%
\pgfsetstrokecolor{currentstroke}%
\pgfsetdash{}{0pt}%
\pgfpathmoveto{\pgfqpoint{1.147720in}{0.934672in}}%
\pgfpathcurveto{\pgfqpoint{1.153544in}{0.934672in}}{\pgfqpoint{1.159131in}{0.936986in}}{\pgfqpoint{1.163249in}{0.941104in}}%
\pgfpathcurveto{\pgfqpoint{1.167367in}{0.945223in}}{\pgfqpoint{1.169681in}{0.950809in}}{\pgfqpoint{1.169681in}{0.956633in}}%
\pgfpathcurveto{\pgfqpoint{1.169681in}{0.962457in}}{\pgfqpoint{1.167367in}{0.968043in}}{\pgfqpoint{1.163249in}{0.972161in}}%
\pgfpathcurveto{\pgfqpoint{1.159131in}{0.976279in}}{\pgfqpoint{1.153544in}{0.978593in}}{\pgfqpoint{1.147720in}{0.978593in}}%
\pgfpathcurveto{\pgfqpoint{1.141897in}{0.978593in}}{\pgfqpoint{1.136310in}{0.976279in}}{\pgfqpoint{1.132192in}{0.972161in}}%
\pgfpathcurveto{\pgfqpoint{1.128074in}{0.968043in}}{\pgfqpoint{1.125760in}{0.962457in}}{\pgfqpoint{1.125760in}{0.956633in}}%
\pgfpathcurveto{\pgfqpoint{1.125760in}{0.950809in}}{\pgfqpoint{1.128074in}{0.945223in}}{\pgfqpoint{1.132192in}{0.941104in}}%
\pgfpathcurveto{\pgfqpoint{1.136310in}{0.936986in}}{\pgfqpoint{1.141897in}{0.934672in}}{\pgfqpoint{1.147720in}{0.934672in}}%
\pgfpathclose%
\pgfusepath{stroke,fill}%
\end{pgfscope}%
\begin{pgfscope}%
\pgfpathrectangle{\pgfqpoint{0.211875in}{0.211875in}}{\pgfqpoint{1.313625in}{1.279725in}}%
\pgfusepath{clip}%
\pgfsetbuttcap%
\pgfsetroundjoin%
\definecolor{currentfill}{rgb}{0.121569,0.466667,0.705882}%
\pgfsetfillcolor{currentfill}%
\pgfsetlinewidth{1.003750pt}%
\definecolor{currentstroke}{rgb}{0.121569,0.466667,0.705882}%
\pgfsetstrokecolor{currentstroke}%
\pgfsetdash{}{0pt}%
\pgfpathmoveto{\pgfqpoint{1.147283in}{0.931549in}}%
\pgfpathcurveto{\pgfqpoint{1.153107in}{0.931549in}}{\pgfqpoint{1.158693in}{0.933863in}}{\pgfqpoint{1.162811in}{0.937981in}}%
\pgfpathcurveto{\pgfqpoint{1.166929in}{0.942099in}}{\pgfqpoint{1.169243in}{0.947686in}}{\pgfqpoint{1.169243in}{0.953510in}}%
\pgfpathcurveto{\pgfqpoint{1.169243in}{0.959333in}}{\pgfqpoint{1.166929in}{0.964920in}}{\pgfqpoint{1.162811in}{0.969038in}}%
\pgfpathcurveto{\pgfqpoint{1.158693in}{0.973156in}}{\pgfqpoint{1.153107in}{0.975470in}}{\pgfqpoint{1.147283in}{0.975470in}}%
\pgfpathcurveto{\pgfqpoint{1.141459in}{0.975470in}}{\pgfqpoint{1.135873in}{0.973156in}}{\pgfqpoint{1.131755in}{0.969038in}}%
\pgfpathcurveto{\pgfqpoint{1.127637in}{0.964920in}}{\pgfqpoint{1.125323in}{0.959333in}}{\pgfqpoint{1.125323in}{0.953510in}}%
\pgfpathcurveto{\pgfqpoint{1.125323in}{0.947686in}}{\pgfqpoint{1.127637in}{0.942099in}}{\pgfqpoint{1.131755in}{0.937981in}}%
\pgfpathcurveto{\pgfqpoint{1.135873in}{0.933863in}}{\pgfqpoint{1.141459in}{0.931549in}}{\pgfqpoint{1.147283in}{0.931549in}}%
\pgfpathclose%
\pgfusepath{stroke,fill}%
\end{pgfscope}%
\begin{pgfscope}%
\pgfpathrectangle{\pgfqpoint{0.211875in}{0.211875in}}{\pgfqpoint{1.313625in}{1.279725in}}%
\pgfusepath{clip}%
\pgfsetbuttcap%
\pgfsetroundjoin%
\definecolor{currentfill}{rgb}{0.121569,0.466667,0.705882}%
\pgfsetfillcolor{currentfill}%
\pgfsetlinewidth{1.003750pt}%
\definecolor{currentstroke}{rgb}{0.121569,0.466667,0.705882}%
\pgfsetstrokecolor{currentstroke}%
\pgfsetdash{}{0pt}%
\pgfpathmoveto{\pgfqpoint{1.147298in}{0.927987in}}%
\pgfpathcurveto{\pgfqpoint{1.153122in}{0.927987in}}{\pgfqpoint{1.158708in}{0.930301in}}{\pgfqpoint{1.162826in}{0.934419in}}%
\pgfpathcurveto{\pgfqpoint{1.166944in}{0.938537in}}{\pgfqpoint{1.169258in}{0.944123in}}{\pgfqpoint{1.169258in}{0.949947in}}%
\pgfpathcurveto{\pgfqpoint{1.169258in}{0.955771in}}{\pgfqpoint{1.166944in}{0.961357in}}{\pgfqpoint{1.162826in}{0.965475in}}%
\pgfpathcurveto{\pgfqpoint{1.158708in}{0.969594in}}{\pgfqpoint{1.153122in}{0.971907in}}{\pgfqpoint{1.147298in}{0.971907in}}%
\pgfpathcurveto{\pgfqpoint{1.141474in}{0.971907in}}{\pgfqpoint{1.135888in}{0.969594in}}{\pgfqpoint{1.131770in}{0.965475in}}%
\pgfpathcurveto{\pgfqpoint{1.127652in}{0.961357in}}{\pgfqpoint{1.125338in}{0.955771in}}{\pgfqpoint{1.125338in}{0.949947in}}%
\pgfpathcurveto{\pgfqpoint{1.125338in}{0.944123in}}{\pgfqpoint{1.127652in}{0.938537in}}{\pgfqpoint{1.131770in}{0.934419in}}%
\pgfpathcurveto{\pgfqpoint{1.135888in}{0.930301in}}{\pgfqpoint{1.141474in}{0.927987in}}{\pgfqpoint{1.147298in}{0.927987in}}%
\pgfpathclose%
\pgfusepath{stroke,fill}%
\end{pgfscope}%
\begin{pgfscope}%
\pgfpathrectangle{\pgfqpoint{0.211875in}{0.211875in}}{\pgfqpoint{1.313625in}{1.279725in}}%
\pgfusepath{clip}%
\pgfsetbuttcap%
\pgfsetroundjoin%
\definecolor{currentfill}{rgb}{0.121569,0.466667,0.705882}%
\pgfsetfillcolor{currentfill}%
\pgfsetlinewidth{1.003750pt}%
\definecolor{currentstroke}{rgb}{0.121569,0.466667,0.705882}%
\pgfsetstrokecolor{currentstroke}%
\pgfsetdash{}{0pt}%
\pgfpathmoveto{\pgfqpoint{1.147157in}{0.924210in}}%
\pgfpathcurveto{\pgfqpoint{1.152981in}{0.924210in}}{\pgfqpoint{1.158567in}{0.926523in}}{\pgfqpoint{1.162686in}{0.930642in}}%
\pgfpathcurveto{\pgfqpoint{1.166804in}{0.934760in}}{\pgfqpoint{1.169118in}{0.940346in}}{\pgfqpoint{1.169118in}{0.946170in}}%
\pgfpathcurveto{\pgfqpoint{1.169118in}{0.951994in}}{\pgfqpoint{1.166804in}{0.957580in}}{\pgfqpoint{1.162686in}{0.961698in}}%
\pgfpathcurveto{\pgfqpoint{1.158567in}{0.965816in}}{\pgfqpoint{1.152981in}{0.968130in}}{\pgfqpoint{1.147157in}{0.968130in}}%
\pgfpathcurveto{\pgfqpoint{1.141333in}{0.968130in}}{\pgfqpoint{1.135747in}{0.965816in}}{\pgfqpoint{1.131629in}{0.961698in}}%
\pgfpathcurveto{\pgfqpoint{1.127511in}{0.957580in}}{\pgfqpoint{1.125197in}{0.951994in}}{\pgfqpoint{1.125197in}{0.946170in}}%
\pgfpathcurveto{\pgfqpoint{1.125197in}{0.940346in}}{\pgfqpoint{1.127511in}{0.934760in}}{\pgfqpoint{1.131629in}{0.930642in}}%
\pgfpathcurveto{\pgfqpoint{1.135747in}{0.926523in}}{\pgfqpoint{1.141333in}{0.924210in}}{\pgfqpoint{1.147157in}{0.924210in}}%
\pgfpathclose%
\pgfusepath{stroke,fill}%
\end{pgfscope}%
\begin{pgfscope}%
\pgfpathrectangle{\pgfqpoint{0.211875in}{0.211875in}}{\pgfqpoint{1.313625in}{1.279725in}}%
\pgfusepath{clip}%
\pgfsetbuttcap%
\pgfsetroundjoin%
\definecolor{currentfill}{rgb}{0.121569,0.466667,0.705882}%
\pgfsetfillcolor{currentfill}%
\pgfsetlinewidth{1.003750pt}%
\definecolor{currentstroke}{rgb}{0.121569,0.466667,0.705882}%
\pgfsetstrokecolor{currentstroke}%
\pgfsetdash{}{0pt}%
\pgfpathmoveto{\pgfqpoint{1.147187in}{0.920190in}}%
\pgfpathcurveto{\pgfqpoint{1.153011in}{0.920190in}}{\pgfqpoint{1.158597in}{0.922504in}}{\pgfqpoint{1.162715in}{0.926622in}}%
\pgfpathcurveto{\pgfqpoint{1.166833in}{0.930740in}}{\pgfqpoint{1.169147in}{0.936326in}}{\pgfqpoint{1.169147in}{0.942150in}}%
\pgfpathcurveto{\pgfqpoint{1.169147in}{0.947974in}}{\pgfqpoint{1.166833in}{0.953560in}}{\pgfqpoint{1.162715in}{0.957678in}}%
\pgfpathcurveto{\pgfqpoint{1.158597in}{0.961796in}}{\pgfqpoint{1.153011in}{0.964110in}}{\pgfqpoint{1.147187in}{0.964110in}}%
\pgfpathcurveto{\pgfqpoint{1.141363in}{0.964110in}}{\pgfqpoint{1.135776in}{0.961796in}}{\pgfqpoint{1.131658in}{0.957678in}}%
\pgfpathcurveto{\pgfqpoint{1.127540in}{0.953560in}}{\pgfqpoint{1.125226in}{0.947974in}}{\pgfqpoint{1.125226in}{0.942150in}}%
\pgfpathcurveto{\pgfqpoint{1.125226in}{0.936326in}}{\pgfqpoint{1.127540in}{0.930740in}}{\pgfqpoint{1.131658in}{0.926622in}}%
\pgfpathcurveto{\pgfqpoint{1.135776in}{0.922504in}}{\pgfqpoint{1.141363in}{0.920190in}}{\pgfqpoint{1.147187in}{0.920190in}}%
\pgfpathclose%
\pgfusepath{stroke,fill}%
\end{pgfscope}%
\begin{pgfscope}%
\pgfpathrectangle{\pgfqpoint{0.211875in}{0.211875in}}{\pgfqpoint{1.313625in}{1.279725in}}%
\pgfusepath{clip}%
\pgfsetbuttcap%
\pgfsetroundjoin%
\definecolor{currentfill}{rgb}{0.121569,0.466667,0.705882}%
\pgfsetfillcolor{currentfill}%
\pgfsetlinewidth{1.003750pt}%
\definecolor{currentstroke}{rgb}{0.121569,0.466667,0.705882}%
\pgfsetstrokecolor{currentstroke}%
\pgfsetdash{}{0pt}%
\pgfpathmoveto{\pgfqpoint{1.147185in}{0.915950in}}%
\pgfpathcurveto{\pgfqpoint{1.153009in}{0.915950in}}{\pgfqpoint{1.158595in}{0.918263in}}{\pgfqpoint{1.162713in}{0.922382in}}%
\pgfpathcurveto{\pgfqpoint{1.166831in}{0.926500in}}{\pgfqpoint{1.169145in}{0.932086in}}{\pgfqpoint{1.169145in}{0.937910in}}%
\pgfpathcurveto{\pgfqpoint{1.169145in}{0.943734in}}{\pgfqpoint{1.166831in}{0.949320in}}{\pgfqpoint{1.162713in}{0.953438in}}%
\pgfpathcurveto{\pgfqpoint{1.158595in}{0.957556in}}{\pgfqpoint{1.153009in}{0.959870in}}{\pgfqpoint{1.147185in}{0.959870in}}%
\pgfpathcurveto{\pgfqpoint{1.141361in}{0.959870in}}{\pgfqpoint{1.135775in}{0.957556in}}{\pgfqpoint{1.131657in}{0.953438in}}%
\pgfpathcurveto{\pgfqpoint{1.127539in}{0.949320in}}{\pgfqpoint{1.125225in}{0.943734in}}{\pgfqpoint{1.125225in}{0.937910in}}%
\pgfpathcurveto{\pgfqpoint{1.125225in}{0.932086in}}{\pgfqpoint{1.127539in}{0.926500in}}{\pgfqpoint{1.131657in}{0.922382in}}%
\pgfpathcurveto{\pgfqpoint{1.135775in}{0.918263in}}{\pgfqpoint{1.141361in}{0.915950in}}{\pgfqpoint{1.147185in}{0.915950in}}%
\pgfpathclose%
\pgfusepath{stroke,fill}%
\end{pgfscope}%
\begin{pgfscope}%
\pgfpathrectangle{\pgfqpoint{0.211875in}{0.211875in}}{\pgfqpoint{1.313625in}{1.279725in}}%
\pgfusepath{clip}%
\pgfsetbuttcap%
\pgfsetroundjoin%
\definecolor{currentfill}{rgb}{0.121569,0.466667,0.705882}%
\pgfsetfillcolor{currentfill}%
\pgfsetlinewidth{1.003750pt}%
\definecolor{currentstroke}{rgb}{0.121569,0.466667,0.705882}%
\pgfsetstrokecolor{currentstroke}%
\pgfsetdash{}{0pt}%
\pgfpathmoveto{\pgfqpoint{1.147275in}{0.911506in}}%
\pgfpathcurveto{\pgfqpoint{1.153099in}{0.911506in}}{\pgfqpoint{1.158685in}{0.913820in}}{\pgfqpoint{1.162803in}{0.917938in}}%
\pgfpathcurveto{\pgfqpoint{1.166921in}{0.922056in}}{\pgfqpoint{1.169235in}{0.927642in}}{\pgfqpoint{1.169235in}{0.933466in}}%
\pgfpathcurveto{\pgfqpoint{1.169235in}{0.939290in}}{\pgfqpoint{1.166921in}{0.944876in}}{\pgfqpoint{1.162803in}{0.948995in}}%
\pgfpathcurveto{\pgfqpoint{1.158685in}{0.953113in}}{\pgfqpoint{1.153099in}{0.955427in}}{\pgfqpoint{1.147275in}{0.955427in}}%
\pgfpathcurveto{\pgfqpoint{1.141451in}{0.955427in}}{\pgfqpoint{1.135865in}{0.953113in}}{\pgfqpoint{1.131747in}{0.948995in}}%
\pgfpathcurveto{\pgfqpoint{1.127629in}{0.944876in}}{\pgfqpoint{1.125315in}{0.939290in}}{\pgfqpoint{1.125315in}{0.933466in}}%
\pgfpathcurveto{\pgfqpoint{1.125315in}{0.927642in}}{\pgfqpoint{1.127629in}{0.922056in}}{\pgfqpoint{1.131747in}{0.917938in}}%
\pgfpathcurveto{\pgfqpoint{1.135865in}{0.913820in}}{\pgfqpoint{1.141451in}{0.911506in}}{\pgfqpoint{1.147275in}{0.911506in}}%
\pgfpathclose%
\pgfusepath{stroke,fill}%
\end{pgfscope}%
\begin{pgfscope}%
\pgfpathrectangle{\pgfqpoint{0.211875in}{0.211875in}}{\pgfqpoint{1.313625in}{1.279725in}}%
\pgfusepath{clip}%
\pgfsetbuttcap%
\pgfsetroundjoin%
\definecolor{currentfill}{rgb}{0.121569,0.466667,0.705882}%
\pgfsetfillcolor{currentfill}%
\pgfsetlinewidth{1.003750pt}%
\definecolor{currentstroke}{rgb}{0.121569,0.466667,0.705882}%
\pgfsetstrokecolor{currentstroke}%
\pgfsetdash{}{0pt}%
\pgfpathmoveto{\pgfqpoint{1.147411in}{0.906838in}}%
\pgfpathcurveto{\pgfqpoint{1.153235in}{0.906838in}}{\pgfqpoint{1.158821in}{0.909152in}}{\pgfqpoint{1.162939in}{0.913270in}}%
\pgfpathcurveto{\pgfqpoint{1.167057in}{0.917388in}}{\pgfqpoint{1.169371in}{0.922975in}}{\pgfqpoint{1.169371in}{0.928799in}}%
\pgfpathcurveto{\pgfqpoint{1.169371in}{0.934623in}}{\pgfqpoint{1.167057in}{0.940209in}}{\pgfqpoint{1.162939in}{0.944327in}}%
\pgfpathcurveto{\pgfqpoint{1.158821in}{0.948445in}}{\pgfqpoint{1.153235in}{0.950759in}}{\pgfqpoint{1.147411in}{0.950759in}}%
\pgfpathcurveto{\pgfqpoint{1.141587in}{0.950759in}}{\pgfqpoint{1.136001in}{0.948445in}}{\pgfqpoint{1.131883in}{0.944327in}}%
\pgfpathcurveto{\pgfqpoint{1.127764in}{0.940209in}}{\pgfqpoint{1.125451in}{0.934623in}}{\pgfqpoint{1.125451in}{0.928799in}}%
\pgfpathcurveto{\pgfqpoint{1.125451in}{0.922975in}}{\pgfqpoint{1.127764in}{0.917388in}}{\pgfqpoint{1.131883in}{0.913270in}}%
\pgfpathcurveto{\pgfqpoint{1.136001in}{0.909152in}}{\pgfqpoint{1.141587in}{0.906838in}}{\pgfqpoint{1.147411in}{0.906838in}}%
\pgfpathclose%
\pgfusepath{stroke,fill}%
\end{pgfscope}%
\begin{pgfscope}%
\pgfpathrectangle{\pgfqpoint{0.211875in}{0.211875in}}{\pgfqpoint{1.313625in}{1.279725in}}%
\pgfusepath{clip}%
\pgfsetbuttcap%
\pgfsetroundjoin%
\definecolor{currentfill}{rgb}{0.121569,0.466667,0.705882}%
\pgfsetfillcolor{currentfill}%
\pgfsetlinewidth{1.003750pt}%
\definecolor{currentstroke}{rgb}{0.121569,0.466667,0.705882}%
\pgfsetstrokecolor{currentstroke}%
\pgfsetdash{}{0pt}%
\pgfpathmoveto{\pgfqpoint{1.147645in}{0.901959in}}%
\pgfpathcurveto{\pgfqpoint{1.153469in}{0.901959in}}{\pgfqpoint{1.159055in}{0.904273in}}{\pgfqpoint{1.163173in}{0.908391in}}%
\pgfpathcurveto{\pgfqpoint{1.167291in}{0.912509in}}{\pgfqpoint{1.169605in}{0.918095in}}{\pgfqpoint{1.169605in}{0.923919in}}%
\pgfpathcurveto{\pgfqpoint{1.169605in}{0.929743in}}{\pgfqpoint{1.167291in}{0.935329in}}{\pgfqpoint{1.163173in}{0.939447in}}%
\pgfpathcurveto{\pgfqpoint{1.159055in}{0.943566in}}{\pgfqpoint{1.153469in}{0.945879in}}{\pgfqpoint{1.147645in}{0.945879in}}%
\pgfpathcurveto{\pgfqpoint{1.141821in}{0.945879in}}{\pgfqpoint{1.136234in}{0.943566in}}{\pgfqpoint{1.132116in}{0.939447in}}%
\pgfpathcurveto{\pgfqpoint{1.127998in}{0.935329in}}{\pgfqpoint{1.125684in}{0.929743in}}{\pgfqpoint{1.125684in}{0.923919in}}%
\pgfpathcurveto{\pgfqpoint{1.125684in}{0.918095in}}{\pgfqpoint{1.127998in}{0.912509in}}{\pgfqpoint{1.132116in}{0.908391in}}%
\pgfpathcurveto{\pgfqpoint{1.136234in}{0.904273in}}{\pgfqpoint{1.141821in}{0.901959in}}{\pgfqpoint{1.147645in}{0.901959in}}%
\pgfpathclose%
\pgfusepath{stroke,fill}%
\end{pgfscope}%
\begin{pgfscope}%
\pgfpathrectangle{\pgfqpoint{0.211875in}{0.211875in}}{\pgfqpoint{1.313625in}{1.279725in}}%
\pgfusepath{clip}%
\pgfsetbuttcap%
\pgfsetroundjoin%
\definecolor{currentfill}{rgb}{0.121569,0.466667,0.705882}%
\pgfsetfillcolor{currentfill}%
\pgfsetlinewidth{1.003750pt}%
\definecolor{currentstroke}{rgb}{0.121569,0.466667,0.705882}%
\pgfsetstrokecolor{currentstroke}%
\pgfsetdash{}{0pt}%
\pgfpathmoveto{\pgfqpoint{1.147995in}{0.896870in}}%
\pgfpathcurveto{\pgfqpoint{1.153819in}{0.896870in}}{\pgfqpoint{1.159405in}{0.899184in}}{\pgfqpoint{1.163523in}{0.903302in}}%
\pgfpathcurveto{\pgfqpoint{1.167641in}{0.907420in}}{\pgfqpoint{1.169955in}{0.913006in}}{\pgfqpoint{1.169955in}{0.918830in}}%
\pgfpathcurveto{\pgfqpoint{1.169955in}{0.924654in}}{\pgfqpoint{1.167641in}{0.930240in}}{\pgfqpoint{1.163523in}{0.934359in}}%
\pgfpathcurveto{\pgfqpoint{1.159405in}{0.938477in}}{\pgfqpoint{1.153819in}{0.940791in}}{\pgfqpoint{1.147995in}{0.940791in}}%
\pgfpathcurveto{\pgfqpoint{1.142171in}{0.940791in}}{\pgfqpoint{1.136585in}{0.938477in}}{\pgfqpoint{1.132467in}{0.934359in}}%
\pgfpathcurveto{\pgfqpoint{1.128348in}{0.930240in}}{\pgfqpoint{1.126035in}{0.924654in}}{\pgfqpoint{1.126035in}{0.918830in}}%
\pgfpathcurveto{\pgfqpoint{1.126035in}{0.913006in}}{\pgfqpoint{1.128348in}{0.907420in}}{\pgfqpoint{1.132467in}{0.903302in}}%
\pgfpathcurveto{\pgfqpoint{1.136585in}{0.899184in}}{\pgfqpoint{1.142171in}{0.896870in}}{\pgfqpoint{1.147995in}{0.896870in}}%
\pgfpathclose%
\pgfusepath{stroke,fill}%
\end{pgfscope}%
\begin{pgfscope}%
\pgfpathrectangle{\pgfqpoint{0.211875in}{0.211875in}}{\pgfqpoint{1.313625in}{1.279725in}}%
\pgfusepath{clip}%
\pgfsetbuttcap%
\pgfsetroundjoin%
\definecolor{currentfill}{rgb}{0.121569,0.466667,0.705882}%
\pgfsetfillcolor{currentfill}%
\pgfsetlinewidth{1.003750pt}%
\definecolor{currentstroke}{rgb}{0.121569,0.466667,0.705882}%
\pgfsetstrokecolor{currentstroke}%
\pgfsetdash{}{0pt}%
\pgfpathmoveto{\pgfqpoint{1.148514in}{0.891696in}}%
\pgfpathcurveto{\pgfqpoint{1.154338in}{0.891696in}}{\pgfqpoint{1.159925in}{0.894010in}}{\pgfqpoint{1.164043in}{0.898128in}}%
\pgfpathcurveto{\pgfqpoint{1.168161in}{0.902247in}}{\pgfqpoint{1.170475in}{0.907833in}}{\pgfqpoint{1.170475in}{0.913657in}}%
\pgfpathcurveto{\pgfqpoint{1.170475in}{0.919481in}}{\pgfqpoint{1.168161in}{0.925067in}}{\pgfqpoint{1.164043in}{0.929185in}}%
\pgfpathcurveto{\pgfqpoint{1.159925in}{0.933303in}}{\pgfqpoint{1.154338in}{0.935617in}}{\pgfqpoint{1.148514in}{0.935617in}}%
\pgfpathcurveto{\pgfqpoint{1.142691in}{0.935617in}}{\pgfqpoint{1.137104in}{0.933303in}}{\pgfqpoint{1.132986in}{0.929185in}}%
\pgfpathcurveto{\pgfqpoint{1.128868in}{0.925067in}}{\pgfqpoint{1.126554in}{0.919481in}}{\pgfqpoint{1.126554in}{0.913657in}}%
\pgfpathcurveto{\pgfqpoint{1.126554in}{0.907833in}}{\pgfqpoint{1.128868in}{0.902247in}}{\pgfqpoint{1.132986in}{0.898128in}}%
\pgfpathcurveto{\pgfqpoint{1.137104in}{0.894010in}}{\pgfqpoint{1.142691in}{0.891696in}}{\pgfqpoint{1.148514in}{0.891696in}}%
\pgfpathclose%
\pgfusepath{stroke,fill}%
\end{pgfscope}%
\begin{pgfscope}%
\pgfpathrectangle{\pgfqpoint{0.211875in}{0.211875in}}{\pgfqpoint{1.313625in}{1.279725in}}%
\pgfusepath{clip}%
\pgfsetbuttcap%
\pgfsetroundjoin%
\definecolor{currentfill}{rgb}{0.121569,0.466667,0.705882}%
\pgfsetfillcolor{currentfill}%
\pgfsetlinewidth{1.003750pt}%
\definecolor{currentstroke}{rgb}{0.121569,0.466667,0.705882}%
\pgfsetstrokecolor{currentstroke}%
\pgfsetdash{}{0pt}%
\pgfpathmoveto{\pgfqpoint{1.149191in}{0.886803in}}%
\pgfpathcurveto{\pgfqpoint{1.155015in}{0.886803in}}{\pgfqpoint{1.160601in}{0.889117in}}{\pgfqpoint{1.164720in}{0.893235in}}%
\pgfpathcurveto{\pgfqpoint{1.168838in}{0.897353in}}{\pgfqpoint{1.171152in}{0.902939in}}{\pgfqpoint{1.171152in}{0.908763in}}%
\pgfpathcurveto{\pgfqpoint{1.171152in}{0.914587in}}{\pgfqpoint{1.168838in}{0.920173in}}{\pgfqpoint{1.164720in}{0.924291in}}%
\pgfpathcurveto{\pgfqpoint{1.160601in}{0.928410in}}{\pgfqpoint{1.155015in}{0.930724in}}{\pgfqpoint{1.149191in}{0.930724in}}%
\pgfpathcurveto{\pgfqpoint{1.143367in}{0.930724in}}{\pgfqpoint{1.137781in}{0.928410in}}{\pgfqpoint{1.133663in}{0.924291in}}%
\pgfpathcurveto{\pgfqpoint{1.129545in}{0.920173in}}{\pgfqpoint{1.127231in}{0.914587in}}{\pgfqpoint{1.127231in}{0.908763in}}%
\pgfpathcurveto{\pgfqpoint{1.127231in}{0.902939in}}{\pgfqpoint{1.129545in}{0.897353in}}{\pgfqpoint{1.133663in}{0.893235in}}%
\pgfpathcurveto{\pgfqpoint{1.137781in}{0.889117in}}{\pgfqpoint{1.143367in}{0.886803in}}{\pgfqpoint{1.149191in}{0.886803in}}%
\pgfpathclose%
\pgfusepath{stroke,fill}%
\end{pgfscope}%
\begin{pgfscope}%
\pgfpathrectangle{\pgfqpoint{0.211875in}{0.211875in}}{\pgfqpoint{1.313625in}{1.279725in}}%
\pgfusepath{clip}%
\pgfsetbuttcap%
\pgfsetroundjoin%
\definecolor{currentfill}{rgb}{0.121569,0.466667,0.705882}%
\pgfsetfillcolor{currentfill}%
\pgfsetlinewidth{1.003750pt}%
\definecolor{currentstroke}{rgb}{0.121569,0.466667,0.705882}%
\pgfsetstrokecolor{currentstroke}%
\pgfsetdash{}{0pt}%
\pgfpathmoveto{\pgfqpoint{1.078255in}{1.391844in}}%
\pgfpathcurveto{\pgfqpoint{1.084079in}{1.391844in}}{\pgfqpoint{1.089665in}{1.394158in}}{\pgfqpoint{1.093783in}{1.398276in}}%
\pgfpathcurveto{\pgfqpoint{1.097901in}{1.402394in}}{\pgfqpoint{1.100215in}{1.407981in}}{\pgfqpoint{1.100215in}{1.413805in}}%
\pgfpathcurveto{\pgfqpoint{1.100215in}{1.419629in}}{\pgfqpoint{1.097901in}{1.425215in}}{\pgfqpoint{1.093783in}{1.429333in}}%
\pgfpathcurveto{\pgfqpoint{1.089665in}{1.433451in}}{\pgfqpoint{1.084079in}{1.435765in}}{\pgfqpoint{1.078255in}{1.435765in}}%
\pgfpathcurveto{\pgfqpoint{1.072431in}{1.435765in}}{\pgfqpoint{1.066845in}{1.433451in}}{\pgfqpoint{1.062727in}{1.429333in}}%
\pgfpathcurveto{\pgfqpoint{1.058609in}{1.425215in}}{\pgfqpoint{1.056295in}{1.419629in}}{\pgfqpoint{1.056295in}{1.413805in}}%
\pgfpathcurveto{\pgfqpoint{1.056295in}{1.407981in}}{\pgfqpoint{1.058609in}{1.402394in}}{\pgfqpoint{1.062727in}{1.398276in}}%
\pgfpathcurveto{\pgfqpoint{1.066845in}{1.394158in}}{\pgfqpoint{1.072431in}{1.391844in}}{\pgfqpoint{1.078255in}{1.391844in}}%
\pgfpathclose%
\pgfusepath{stroke,fill}%
\end{pgfscope}%
\begin{pgfscope}%
\pgfpathrectangle{\pgfqpoint{0.211875in}{0.211875in}}{\pgfqpoint{1.313625in}{1.279725in}}%
\pgfusepath{clip}%
\pgfsetbuttcap%
\pgfsetroundjoin%
\definecolor{currentfill}{rgb}{0.121569,0.466667,0.705882}%
\pgfsetfillcolor{currentfill}%
\pgfsetlinewidth{1.003750pt}%
\definecolor{currentstroke}{rgb}{0.121569,0.466667,0.705882}%
\pgfsetstrokecolor{currentstroke}%
\pgfsetdash{}{0pt}%
\pgfpathmoveto{\pgfqpoint{1.127513in}{0.919505in}}%
\pgfpathcurveto{\pgfqpoint{1.133337in}{0.919505in}}{\pgfqpoint{1.138923in}{0.921819in}}{\pgfqpoint{1.143041in}{0.925937in}}%
\pgfpathcurveto{\pgfqpoint{1.147159in}{0.930055in}}{\pgfqpoint{1.149473in}{0.935641in}}{\pgfqpoint{1.149473in}{0.941465in}}%
\pgfpathcurveto{\pgfqpoint{1.149473in}{0.947289in}}{\pgfqpoint{1.147159in}{0.952875in}}{\pgfqpoint{1.143041in}{0.956993in}}%
\pgfpathcurveto{\pgfqpoint{1.138923in}{0.961112in}}{\pgfqpoint{1.133337in}{0.963425in}}{\pgfqpoint{1.127513in}{0.963425in}}%
\pgfpathcurveto{\pgfqpoint{1.121689in}{0.963425in}}{\pgfqpoint{1.116103in}{0.961112in}}{\pgfqpoint{1.111985in}{0.956993in}}%
\pgfpathcurveto{\pgfqpoint{1.107867in}{0.952875in}}{\pgfqpoint{1.105553in}{0.947289in}}{\pgfqpoint{1.105553in}{0.941465in}}%
\pgfpathcurveto{\pgfqpoint{1.105553in}{0.935641in}}{\pgfqpoint{1.107867in}{0.930055in}}{\pgfqpoint{1.111985in}{0.925937in}}%
\pgfpathcurveto{\pgfqpoint{1.116103in}{0.921819in}}{\pgfqpoint{1.121689in}{0.919505in}}{\pgfqpoint{1.127513in}{0.919505in}}%
\pgfpathclose%
\pgfusepath{stroke,fill}%
\end{pgfscope}%
\begin{pgfscope}%
\pgfpathrectangle{\pgfqpoint{0.211875in}{0.211875in}}{\pgfqpoint{1.313625in}{1.279725in}}%
\pgfusepath{clip}%
\pgfsetbuttcap%
\pgfsetroundjoin%
\definecolor{currentfill}{rgb}{0.121569,0.466667,0.705882}%
\pgfsetfillcolor{currentfill}%
\pgfsetlinewidth{1.003750pt}%
\definecolor{currentstroke}{rgb}{0.121569,0.466667,0.705882}%
\pgfsetstrokecolor{currentstroke}%
\pgfsetdash{}{0pt}%
\pgfpathmoveto{\pgfqpoint{1.204028in}{0.915665in}}%
\pgfpathcurveto{\pgfqpoint{1.209852in}{0.915665in}}{\pgfqpoint{1.215438in}{0.917979in}}{\pgfqpoint{1.219556in}{0.922097in}}%
\pgfpathcurveto{\pgfqpoint{1.223674in}{0.926215in}}{\pgfqpoint{1.225988in}{0.931801in}}{\pgfqpoint{1.225988in}{0.937625in}}%
\pgfpathcurveto{\pgfqpoint{1.225988in}{0.943449in}}{\pgfqpoint{1.223674in}{0.949035in}}{\pgfqpoint{1.219556in}{0.953153in}}%
\pgfpathcurveto{\pgfqpoint{1.215438in}{0.957271in}}{\pgfqpoint{1.209852in}{0.959585in}}{\pgfqpoint{1.204028in}{0.959585in}}%
\pgfpathcurveto{\pgfqpoint{1.198204in}{0.959585in}}{\pgfqpoint{1.192618in}{0.957271in}}{\pgfqpoint{1.188499in}{0.953153in}}%
\pgfpathcurveto{\pgfqpoint{1.184381in}{0.949035in}}{\pgfqpoint{1.182067in}{0.943449in}}{\pgfqpoint{1.182067in}{0.937625in}}%
\pgfpathcurveto{\pgfqpoint{1.182067in}{0.931801in}}{\pgfqpoint{1.184381in}{0.926215in}}{\pgfqpoint{1.188499in}{0.922097in}}%
\pgfpathcurveto{\pgfqpoint{1.192618in}{0.917979in}}{\pgfqpoint{1.198204in}{0.915665in}}{\pgfqpoint{1.204028in}{0.915665in}}%
\pgfpathclose%
\pgfusepath{stroke,fill}%
\end{pgfscope}%
\begin{pgfscope}%
\pgfpathrectangle{\pgfqpoint{0.211875in}{0.211875in}}{\pgfqpoint{1.313625in}{1.279725in}}%
\pgfusepath{clip}%
\pgfsetbuttcap%
\pgfsetroundjoin%
\definecolor{currentfill}{rgb}{0.121569,0.466667,0.705882}%
\pgfsetfillcolor{currentfill}%
\pgfsetlinewidth{1.003750pt}%
\definecolor{currentstroke}{rgb}{0.121569,0.466667,0.705882}%
\pgfsetstrokecolor{currentstroke}%
\pgfsetdash{}{0pt}%
\pgfpathmoveto{\pgfqpoint{1.178291in}{0.926102in}}%
\pgfpathcurveto{\pgfqpoint{1.184115in}{0.926102in}}{\pgfqpoint{1.189701in}{0.928416in}}{\pgfqpoint{1.193819in}{0.932534in}}%
\pgfpathcurveto{\pgfqpoint{1.197938in}{0.936652in}}{\pgfqpoint{1.200251in}{0.942238in}}{\pgfqpoint{1.200251in}{0.948062in}}%
\pgfpathcurveto{\pgfqpoint{1.200251in}{0.953886in}}{\pgfqpoint{1.197938in}{0.959472in}}{\pgfqpoint{1.193819in}{0.963591in}}%
\pgfpathcurveto{\pgfqpoint{1.189701in}{0.967709in}}{\pgfqpoint{1.184115in}{0.970023in}}{\pgfqpoint{1.178291in}{0.970023in}}%
\pgfpathcurveto{\pgfqpoint{1.172467in}{0.970023in}}{\pgfqpoint{1.166881in}{0.967709in}}{\pgfqpoint{1.162763in}{0.963591in}}%
\pgfpathcurveto{\pgfqpoint{1.158645in}{0.959472in}}{\pgfqpoint{1.156331in}{0.953886in}}{\pgfqpoint{1.156331in}{0.948062in}}%
\pgfpathcurveto{\pgfqpoint{1.156331in}{0.942238in}}{\pgfqpoint{1.158645in}{0.936652in}}{\pgfqpoint{1.162763in}{0.932534in}}%
\pgfpathcurveto{\pgfqpoint{1.166881in}{0.928416in}}{\pgfqpoint{1.172467in}{0.926102in}}{\pgfqpoint{1.178291in}{0.926102in}}%
\pgfpathclose%
\pgfusepath{stroke,fill}%
\end{pgfscope}%
\begin{pgfscope}%
\pgfpathrectangle{\pgfqpoint{0.211875in}{0.211875in}}{\pgfqpoint{1.313625in}{1.279725in}}%
\pgfusepath{clip}%
\pgfsetbuttcap%
\pgfsetroundjoin%
\definecolor{currentfill}{rgb}{0.121569,0.466667,0.705882}%
\pgfsetfillcolor{currentfill}%
\pgfsetlinewidth{1.003750pt}%
\definecolor{currentstroke}{rgb}{0.121569,0.466667,0.705882}%
\pgfsetstrokecolor{currentstroke}%
\pgfsetdash{}{0pt}%
\pgfpathmoveto{\pgfqpoint{1.223010in}{0.937422in}}%
\pgfpathcurveto{\pgfqpoint{1.228834in}{0.937422in}}{\pgfqpoint{1.234420in}{0.939736in}}{\pgfqpoint{1.238539in}{0.943854in}}%
\pgfpathcurveto{\pgfqpoint{1.242657in}{0.947972in}}{\pgfqpoint{1.244971in}{0.953559in}}{\pgfqpoint{1.244971in}{0.959382in}}%
\pgfpathcurveto{\pgfqpoint{1.244971in}{0.965206in}}{\pgfqpoint{1.242657in}{0.970793in}}{\pgfqpoint{1.238539in}{0.974911in}}%
\pgfpathcurveto{\pgfqpoint{1.234420in}{0.979029in}}{\pgfqpoint{1.228834in}{0.981343in}}{\pgfqpoint{1.223010in}{0.981343in}}%
\pgfpathcurveto{\pgfqpoint{1.217186in}{0.981343in}}{\pgfqpoint{1.211600in}{0.979029in}}{\pgfqpoint{1.207482in}{0.974911in}}%
\pgfpathcurveto{\pgfqpoint{1.203364in}{0.970793in}}{\pgfqpoint{1.201050in}{0.965206in}}{\pgfqpoint{1.201050in}{0.959382in}}%
\pgfpathcurveto{\pgfqpoint{1.201050in}{0.953559in}}{\pgfqpoint{1.203364in}{0.947972in}}{\pgfqpoint{1.207482in}{0.943854in}}%
\pgfpathcurveto{\pgfqpoint{1.211600in}{0.939736in}}{\pgfqpoint{1.217186in}{0.937422in}}{\pgfqpoint{1.223010in}{0.937422in}}%
\pgfpathclose%
\pgfusepath{stroke,fill}%
\end{pgfscope}%
\begin{pgfscope}%
\pgfpathrectangle{\pgfqpoint{0.211875in}{0.211875in}}{\pgfqpoint{1.313625in}{1.279725in}}%
\pgfusepath{clip}%
\pgfsetbuttcap%
\pgfsetroundjoin%
\definecolor{currentfill}{rgb}{0.121569,0.466667,0.705882}%
\pgfsetfillcolor{currentfill}%
\pgfsetlinewidth{1.003750pt}%
\definecolor{currentstroke}{rgb}{0.121569,0.466667,0.705882}%
\pgfsetstrokecolor{currentstroke}%
\pgfsetdash{}{0pt}%
\pgfpathmoveto{\pgfqpoint{1.118091in}{0.992145in}}%
\pgfpathcurveto{\pgfqpoint{1.123915in}{0.992145in}}{\pgfqpoint{1.129501in}{0.994459in}}{\pgfqpoint{1.133620in}{0.998577in}}%
\pgfpathcurveto{\pgfqpoint{1.137738in}{1.002695in}}{\pgfqpoint{1.140052in}{1.008281in}}{\pgfqpoint{1.140052in}{1.014105in}}%
\pgfpathcurveto{\pgfqpoint{1.140052in}{1.019929in}}{\pgfqpoint{1.137738in}{1.025515in}}{\pgfqpoint{1.133620in}{1.029633in}}%
\pgfpathcurveto{\pgfqpoint{1.129501in}{1.033752in}}{\pgfqpoint{1.123915in}{1.036065in}}{\pgfqpoint{1.118091in}{1.036065in}}%
\pgfpathcurveto{\pgfqpoint{1.112267in}{1.036065in}}{\pgfqpoint{1.106681in}{1.033752in}}{\pgfqpoint{1.102563in}{1.029633in}}%
\pgfpathcurveto{\pgfqpoint{1.098445in}{1.025515in}}{\pgfqpoint{1.096131in}{1.019929in}}{\pgfqpoint{1.096131in}{1.014105in}}%
\pgfpathcurveto{\pgfqpoint{1.096131in}{1.008281in}}{\pgfqpoint{1.098445in}{1.002695in}}{\pgfqpoint{1.102563in}{0.998577in}}%
\pgfpathcurveto{\pgfqpoint{1.106681in}{0.994459in}}{\pgfqpoint{1.112267in}{0.992145in}}{\pgfqpoint{1.118091in}{0.992145in}}%
\pgfpathclose%
\pgfusepath{stroke,fill}%
\end{pgfscope}%
\begin{pgfscope}%
\pgfpathrectangle{\pgfqpoint{0.211875in}{0.211875in}}{\pgfqpoint{1.313625in}{1.279725in}}%
\pgfusepath{clip}%
\pgfsetbuttcap%
\pgfsetroundjoin%
\definecolor{currentfill}{rgb}{0.121569,0.466667,0.705882}%
\pgfsetfillcolor{currentfill}%
\pgfsetlinewidth{1.003750pt}%
\definecolor{currentstroke}{rgb}{0.121569,0.466667,0.705882}%
\pgfsetstrokecolor{currentstroke}%
\pgfsetdash{}{0pt}%
\pgfpathmoveto{\pgfqpoint{1.079823in}{1.075949in}}%
\pgfpathcurveto{\pgfqpoint{1.085647in}{1.075949in}}{\pgfqpoint{1.091233in}{1.078262in}}{\pgfqpoint{1.095351in}{1.082381in}}%
\pgfpathcurveto{\pgfqpoint{1.099469in}{1.086499in}}{\pgfqpoint{1.101783in}{1.092085in}}{\pgfqpoint{1.101783in}{1.097909in}}%
\pgfpathcurveto{\pgfqpoint{1.101783in}{1.103733in}}{\pgfqpoint{1.099469in}{1.109319in}}{\pgfqpoint{1.095351in}{1.113437in}}%
\pgfpathcurveto{\pgfqpoint{1.091233in}{1.117555in}}{\pgfqpoint{1.085647in}{1.119869in}}{\pgfqpoint{1.079823in}{1.119869in}}%
\pgfpathcurveto{\pgfqpoint{1.073999in}{1.119869in}}{\pgfqpoint{1.068412in}{1.117555in}}{\pgfqpoint{1.064294in}{1.113437in}}%
\pgfpathcurveto{\pgfqpoint{1.060176in}{1.109319in}}{\pgfqpoint{1.057862in}{1.103733in}}{\pgfqpoint{1.057862in}{1.097909in}}%
\pgfpathcurveto{\pgfqpoint{1.057862in}{1.092085in}}{\pgfqpoint{1.060176in}{1.086499in}}{\pgfqpoint{1.064294in}{1.082381in}}%
\pgfpathcurveto{\pgfqpoint{1.068412in}{1.078262in}}{\pgfqpoint{1.073999in}{1.075949in}}{\pgfqpoint{1.079823in}{1.075949in}}%
\pgfpathclose%
\pgfusepath{stroke,fill}%
\end{pgfscope}%
\begin{pgfscope}%
\pgfpathrectangle{\pgfqpoint{0.211875in}{0.211875in}}{\pgfqpoint{1.313625in}{1.279725in}}%
\pgfusepath{clip}%
\pgfsetbuttcap%
\pgfsetroundjoin%
\definecolor{currentfill}{rgb}{0.121569,0.466667,0.705882}%
\pgfsetfillcolor{currentfill}%
\pgfsetlinewidth{1.003750pt}%
\definecolor{currentstroke}{rgb}{0.121569,0.466667,0.705882}%
\pgfsetstrokecolor{currentstroke}%
\pgfsetdash{}{0pt}%
\pgfpathmoveto{\pgfqpoint{1.056447in}{1.120021in}}%
\pgfpathcurveto{\pgfqpoint{1.062271in}{1.120021in}}{\pgfqpoint{1.067857in}{1.122335in}}{\pgfqpoint{1.071975in}{1.126453in}}%
\pgfpathcurveto{\pgfqpoint{1.076093in}{1.130571in}}{\pgfqpoint{1.078407in}{1.136157in}}{\pgfqpoint{1.078407in}{1.141981in}}%
\pgfpathcurveto{\pgfqpoint{1.078407in}{1.147805in}}{\pgfqpoint{1.076093in}{1.153391in}}{\pgfqpoint{1.071975in}{1.157509in}}%
\pgfpathcurveto{\pgfqpoint{1.067857in}{1.161628in}}{\pgfqpoint{1.062271in}{1.163941in}}{\pgfqpoint{1.056447in}{1.163941in}}%
\pgfpathcurveto{\pgfqpoint{1.050623in}{1.163941in}}{\pgfqpoint{1.045037in}{1.161628in}}{\pgfqpoint{1.040919in}{1.157509in}}%
\pgfpathcurveto{\pgfqpoint{1.036800in}{1.153391in}}{\pgfqpoint{1.034487in}{1.147805in}}{\pgfqpoint{1.034487in}{1.141981in}}%
\pgfpathcurveto{\pgfqpoint{1.034487in}{1.136157in}}{\pgfqpoint{1.036800in}{1.130571in}}{\pgfqpoint{1.040919in}{1.126453in}}%
\pgfpathcurveto{\pgfqpoint{1.045037in}{1.122335in}}{\pgfqpoint{1.050623in}{1.120021in}}{\pgfqpoint{1.056447in}{1.120021in}}%
\pgfpathclose%
\pgfusepath{stroke,fill}%
\end{pgfscope}%
\begin{pgfscope}%
\pgfpathrectangle{\pgfqpoint{0.211875in}{0.211875in}}{\pgfqpoint{1.313625in}{1.279725in}}%
\pgfusepath{clip}%
\pgfsetbuttcap%
\pgfsetroundjoin%
\definecolor{currentfill}{rgb}{0.121569,0.466667,0.705882}%
\pgfsetfillcolor{currentfill}%
\pgfsetlinewidth{1.003750pt}%
\definecolor{currentstroke}{rgb}{0.121569,0.466667,0.705882}%
\pgfsetstrokecolor{currentstroke}%
\pgfsetdash{}{0pt}%
\pgfpathmoveto{\pgfqpoint{1.132409in}{1.008338in}}%
\pgfpathcurveto{\pgfqpoint{1.138233in}{1.008338in}}{\pgfqpoint{1.143819in}{1.010652in}}{\pgfqpoint{1.147937in}{1.014770in}}%
\pgfpathcurveto{\pgfqpoint{1.152055in}{1.018889in}}{\pgfqpoint{1.154369in}{1.024475in}}{\pgfqpoint{1.154369in}{1.030299in}}%
\pgfpathcurveto{\pgfqpoint{1.154369in}{1.036123in}}{\pgfqpoint{1.152055in}{1.041709in}}{\pgfqpoint{1.147937in}{1.045827in}}%
\pgfpathcurveto{\pgfqpoint{1.143819in}{1.049945in}}{\pgfqpoint{1.138233in}{1.052259in}}{\pgfqpoint{1.132409in}{1.052259in}}%
\pgfpathcurveto{\pgfqpoint{1.126585in}{1.052259in}}{\pgfqpoint{1.120998in}{1.049945in}}{\pgfqpoint{1.116880in}{1.045827in}}%
\pgfpathcurveto{\pgfqpoint{1.112762in}{1.041709in}}{\pgfqpoint{1.110448in}{1.036123in}}{\pgfqpoint{1.110448in}{1.030299in}}%
\pgfpathcurveto{\pgfqpoint{1.110448in}{1.024475in}}{\pgfqpoint{1.112762in}{1.018889in}}{\pgfqpoint{1.116880in}{1.014770in}}%
\pgfpathcurveto{\pgfqpoint{1.120998in}{1.010652in}}{\pgfqpoint{1.126585in}{1.008338in}}{\pgfqpoint{1.132409in}{1.008338in}}%
\pgfpathclose%
\pgfusepath{stroke,fill}%
\end{pgfscope}%
\begin{pgfscope}%
\pgfpathrectangle{\pgfqpoint{0.211875in}{0.211875in}}{\pgfqpoint{1.313625in}{1.279725in}}%
\pgfusepath{clip}%
\pgfsetbuttcap%
\pgfsetroundjoin%
\definecolor{currentfill}{rgb}{0.121569,0.466667,0.705882}%
\pgfsetfillcolor{currentfill}%
\pgfsetlinewidth{1.003750pt}%
\definecolor{currentstroke}{rgb}{0.121569,0.466667,0.705882}%
\pgfsetstrokecolor{currentstroke}%
\pgfsetdash{}{0pt}%
\pgfpathmoveto{\pgfqpoint{1.124331in}{1.033386in}}%
\pgfpathcurveto{\pgfqpoint{1.130155in}{1.033386in}}{\pgfqpoint{1.135741in}{1.035700in}}{\pgfqpoint{1.139859in}{1.039818in}}%
\pgfpathcurveto{\pgfqpoint{1.143978in}{1.043936in}}{\pgfqpoint{1.146291in}{1.049522in}}{\pgfqpoint{1.146291in}{1.055346in}}%
\pgfpathcurveto{\pgfqpoint{1.146291in}{1.061170in}}{\pgfqpoint{1.143978in}{1.066756in}}{\pgfqpoint{1.139859in}{1.070874in}}%
\pgfpathcurveto{\pgfqpoint{1.135741in}{1.074993in}}{\pgfqpoint{1.130155in}{1.077306in}}{\pgfqpoint{1.124331in}{1.077306in}}%
\pgfpathcurveto{\pgfqpoint{1.118507in}{1.077306in}}{\pgfqpoint{1.112921in}{1.074993in}}{\pgfqpoint{1.108803in}{1.070874in}}%
\pgfpathcurveto{\pgfqpoint{1.104685in}{1.066756in}}{\pgfqpoint{1.102371in}{1.061170in}}{\pgfqpoint{1.102371in}{1.055346in}}%
\pgfpathcurveto{\pgfqpoint{1.102371in}{1.049522in}}{\pgfqpoint{1.104685in}{1.043936in}}{\pgfqpoint{1.108803in}{1.039818in}}%
\pgfpathcurveto{\pgfqpoint{1.112921in}{1.035700in}}{\pgfqpoint{1.118507in}{1.033386in}}{\pgfqpoint{1.124331in}{1.033386in}}%
\pgfpathclose%
\pgfusepath{stroke,fill}%
\end{pgfscope}%
\begin{pgfscope}%
\pgfpathrectangle{\pgfqpoint{0.211875in}{0.211875in}}{\pgfqpoint{1.313625in}{1.279725in}}%
\pgfusepath{clip}%
\pgfsetbuttcap%
\pgfsetroundjoin%
\definecolor{currentfill}{rgb}{0.121569,0.466667,0.705882}%
\pgfsetfillcolor{currentfill}%
\pgfsetlinewidth{1.003750pt}%
\definecolor{currentstroke}{rgb}{0.121569,0.466667,0.705882}%
\pgfsetstrokecolor{currentstroke}%
\pgfsetdash{}{0pt}%
\pgfpathmoveto{\pgfqpoint{1.130361in}{1.015524in}}%
\pgfpathcurveto{\pgfqpoint{1.136185in}{1.015524in}}{\pgfqpoint{1.141771in}{1.017838in}}{\pgfqpoint{1.145889in}{1.021956in}}%
\pgfpathcurveto{\pgfqpoint{1.150007in}{1.026074in}}{\pgfqpoint{1.152321in}{1.031660in}}{\pgfqpoint{1.152321in}{1.037484in}}%
\pgfpathcurveto{\pgfqpoint{1.152321in}{1.043308in}}{\pgfqpoint{1.150007in}{1.048894in}}{\pgfqpoint{1.145889in}{1.053012in}}%
\pgfpathcurveto{\pgfqpoint{1.141771in}{1.057130in}}{\pgfqpoint{1.136185in}{1.059444in}}{\pgfqpoint{1.130361in}{1.059444in}}%
\pgfpathcurveto{\pgfqpoint{1.124537in}{1.059444in}}{\pgfqpoint{1.118951in}{1.057130in}}{\pgfqpoint{1.114833in}{1.053012in}}%
\pgfpathcurveto{\pgfqpoint{1.110715in}{1.048894in}}{\pgfqpoint{1.108401in}{1.043308in}}{\pgfqpoint{1.108401in}{1.037484in}}%
\pgfpathcurveto{\pgfqpoint{1.108401in}{1.031660in}}{\pgfqpoint{1.110715in}{1.026074in}}{\pgfqpoint{1.114833in}{1.021956in}}%
\pgfpathcurveto{\pgfqpoint{1.118951in}{1.017838in}}{\pgfqpoint{1.124537in}{1.015524in}}{\pgfqpoint{1.130361in}{1.015524in}}%
\pgfpathclose%
\pgfusepath{stroke,fill}%
\end{pgfscope}%
\begin{pgfscope}%
\pgfpathrectangle{\pgfqpoint{0.211875in}{0.211875in}}{\pgfqpoint{1.313625in}{1.279725in}}%
\pgfusepath{clip}%
\pgfsetbuttcap%
\pgfsetroundjoin%
\definecolor{currentfill}{rgb}{0.121569,0.466667,0.705882}%
\pgfsetfillcolor{currentfill}%
\pgfsetlinewidth{1.003750pt}%
\definecolor{currentstroke}{rgb}{0.121569,0.466667,0.705882}%
\pgfsetstrokecolor{currentstroke}%
\pgfsetdash{}{0pt}%
\pgfpathmoveto{\pgfqpoint{1.123692in}{1.032777in}}%
\pgfpathcurveto{\pgfqpoint{1.129515in}{1.032777in}}{\pgfqpoint{1.135102in}{1.035091in}}{\pgfqpoint{1.139220in}{1.039209in}}%
\pgfpathcurveto{\pgfqpoint{1.143338in}{1.043327in}}{\pgfqpoint{1.145652in}{1.048914in}}{\pgfqpoint{1.145652in}{1.054738in}}%
\pgfpathcurveto{\pgfqpoint{1.145652in}{1.060561in}}{\pgfqpoint{1.143338in}{1.066148in}}{\pgfqpoint{1.139220in}{1.070266in}}%
\pgfpathcurveto{\pgfqpoint{1.135102in}{1.074384in}}{\pgfqpoint{1.129515in}{1.076698in}}{\pgfqpoint{1.123692in}{1.076698in}}%
\pgfpathcurveto{\pgfqpoint{1.117868in}{1.076698in}}{\pgfqpoint{1.112281in}{1.074384in}}{\pgfqpoint{1.108163in}{1.070266in}}%
\pgfpathcurveto{\pgfqpoint{1.104045in}{1.066148in}}{\pgfqpoint{1.101731in}{1.060561in}}{\pgfqpoint{1.101731in}{1.054738in}}%
\pgfpathcurveto{\pgfqpoint{1.101731in}{1.048914in}}{\pgfqpoint{1.104045in}{1.043327in}}{\pgfqpoint{1.108163in}{1.039209in}}%
\pgfpathcurveto{\pgfqpoint{1.112281in}{1.035091in}}{\pgfqpoint{1.117868in}{1.032777in}}{\pgfqpoint{1.123692in}{1.032777in}}%
\pgfpathclose%
\pgfusepath{stroke,fill}%
\end{pgfscope}%
\begin{pgfscope}%
\pgfpathrectangle{\pgfqpoint{0.211875in}{0.211875in}}{\pgfqpoint{1.313625in}{1.279725in}}%
\pgfusepath{clip}%
\pgfsetbuttcap%
\pgfsetroundjoin%
\definecolor{currentfill}{rgb}{0.121569,0.466667,0.705882}%
\pgfsetfillcolor{currentfill}%
\pgfsetlinewidth{1.003750pt}%
\definecolor{currentstroke}{rgb}{0.121569,0.466667,0.705882}%
\pgfsetstrokecolor{currentstroke}%
\pgfsetdash{}{0pt}%
\pgfpathmoveto{\pgfqpoint{1.128652in}{1.017797in}}%
\pgfpathcurveto{\pgfqpoint{1.134476in}{1.017797in}}{\pgfqpoint{1.140062in}{1.020111in}}{\pgfqpoint{1.144180in}{1.024229in}}%
\pgfpathcurveto{\pgfqpoint{1.148298in}{1.028348in}}{\pgfqpoint{1.150612in}{1.033934in}}{\pgfqpoint{1.150612in}{1.039758in}}%
\pgfpathcurveto{\pgfqpoint{1.150612in}{1.045582in}}{\pgfqpoint{1.148298in}{1.051168in}}{\pgfqpoint{1.144180in}{1.055286in}}%
\pgfpathcurveto{\pgfqpoint{1.140062in}{1.059404in}}{\pgfqpoint{1.134476in}{1.061718in}}{\pgfqpoint{1.128652in}{1.061718in}}%
\pgfpathcurveto{\pgfqpoint{1.122828in}{1.061718in}}{\pgfqpoint{1.117242in}{1.059404in}}{\pgfqpoint{1.113124in}{1.055286in}}%
\pgfpathcurveto{\pgfqpoint{1.109005in}{1.051168in}}{\pgfqpoint{1.106692in}{1.045582in}}{\pgfqpoint{1.106692in}{1.039758in}}%
\pgfpathcurveto{\pgfqpoint{1.106692in}{1.033934in}}{\pgfqpoint{1.109005in}{1.028348in}}{\pgfqpoint{1.113124in}{1.024229in}}%
\pgfpathcurveto{\pgfqpoint{1.117242in}{1.020111in}}{\pgfqpoint{1.122828in}{1.017797in}}{\pgfqpoint{1.128652in}{1.017797in}}%
\pgfpathclose%
\pgfusepath{stroke,fill}%
\end{pgfscope}%
\begin{pgfscope}%
\pgfpathrectangle{\pgfqpoint{0.211875in}{0.211875in}}{\pgfqpoint{1.313625in}{1.279725in}}%
\pgfusepath{clip}%
\pgfsetbuttcap%
\pgfsetroundjoin%
\definecolor{currentfill}{rgb}{0.121569,0.466667,0.705882}%
\pgfsetfillcolor{currentfill}%
\pgfsetlinewidth{1.003750pt}%
\definecolor{currentstroke}{rgb}{0.121569,0.466667,0.705882}%
\pgfsetstrokecolor{currentstroke}%
\pgfsetdash{}{0pt}%
\pgfpathmoveto{\pgfqpoint{1.122452in}{1.031560in}}%
\pgfpathcurveto{\pgfqpoint{1.128276in}{1.031560in}}{\pgfqpoint{1.133862in}{1.033873in}}{\pgfqpoint{1.137980in}{1.037992in}}%
\pgfpathcurveto{\pgfqpoint{1.142098in}{1.042110in}}{\pgfqpoint{1.144412in}{1.047696in}}{\pgfqpoint{1.144412in}{1.053520in}}%
\pgfpathcurveto{\pgfqpoint{1.144412in}{1.059344in}}{\pgfqpoint{1.142098in}{1.064930in}}{\pgfqpoint{1.137980in}{1.069048in}}%
\pgfpathcurveto{\pgfqpoint{1.133862in}{1.073166in}}{\pgfqpoint{1.128276in}{1.075480in}}{\pgfqpoint{1.122452in}{1.075480in}}%
\pgfpathcurveto{\pgfqpoint{1.116628in}{1.075480in}}{\pgfqpoint{1.111042in}{1.073166in}}{\pgfqpoint{1.106923in}{1.069048in}}%
\pgfpathcurveto{\pgfqpoint{1.102805in}{1.064930in}}{\pgfqpoint{1.100491in}{1.059344in}}{\pgfqpoint{1.100491in}{1.053520in}}%
\pgfpathcurveto{\pgfqpoint{1.100491in}{1.047696in}}{\pgfqpoint{1.102805in}{1.042110in}}{\pgfqpoint{1.106923in}{1.037992in}}%
\pgfpathcurveto{\pgfqpoint{1.111042in}{1.033873in}}{\pgfqpoint{1.116628in}{1.031560in}}{\pgfqpoint{1.122452in}{1.031560in}}%
\pgfpathclose%
\pgfusepath{stroke,fill}%
\end{pgfscope}%
\begin{pgfscope}%
\pgfpathrectangle{\pgfqpoint{0.211875in}{0.211875in}}{\pgfqpoint{1.313625in}{1.279725in}}%
\pgfusepath{clip}%
\pgfsetbuttcap%
\pgfsetroundjoin%
\definecolor{currentfill}{rgb}{0.121569,0.466667,0.705882}%
\pgfsetfillcolor{currentfill}%
\pgfsetlinewidth{1.003750pt}%
\definecolor{currentstroke}{rgb}{0.121569,0.466667,0.705882}%
\pgfsetstrokecolor{currentstroke}%
\pgfsetdash{}{0pt}%
\pgfpathmoveto{\pgfqpoint{1.125954in}{1.020313in}}%
\pgfpathcurveto{\pgfqpoint{1.131778in}{1.020313in}}{\pgfqpoint{1.137364in}{1.022627in}}{\pgfqpoint{1.141482in}{1.026745in}}%
\pgfpathcurveto{\pgfqpoint{1.145600in}{1.030864in}}{\pgfqpoint{1.147914in}{1.036450in}}{\pgfqpoint{1.147914in}{1.042274in}}%
\pgfpathcurveto{\pgfqpoint{1.147914in}{1.048098in}}{\pgfqpoint{1.145600in}{1.053684in}}{\pgfqpoint{1.141482in}{1.057802in}}%
\pgfpathcurveto{\pgfqpoint{1.137364in}{1.061920in}}{\pgfqpoint{1.131778in}{1.064234in}}{\pgfqpoint{1.125954in}{1.064234in}}%
\pgfpathcurveto{\pgfqpoint{1.120130in}{1.064234in}}{\pgfqpoint{1.114544in}{1.061920in}}{\pgfqpoint{1.110426in}{1.057802in}}%
\pgfpathcurveto{\pgfqpoint{1.106308in}{1.053684in}}{\pgfqpoint{1.103994in}{1.048098in}}{\pgfqpoint{1.103994in}{1.042274in}}%
\pgfpathcurveto{\pgfqpoint{1.103994in}{1.036450in}}{\pgfqpoint{1.106308in}{1.030864in}}{\pgfqpoint{1.110426in}{1.026745in}}%
\pgfpathcurveto{\pgfqpoint{1.114544in}{1.022627in}}{\pgfqpoint{1.120130in}{1.020313in}}{\pgfqpoint{1.125954in}{1.020313in}}%
\pgfpathclose%
\pgfusepath{stroke,fill}%
\end{pgfscope}%
\begin{pgfscope}%
\pgfpathrectangle{\pgfqpoint{0.211875in}{0.211875in}}{\pgfqpoint{1.313625in}{1.279725in}}%
\pgfusepath{clip}%
\pgfsetbuttcap%
\pgfsetroundjoin%
\definecolor{currentfill}{rgb}{0.121569,0.466667,0.705882}%
\pgfsetfillcolor{currentfill}%
\pgfsetlinewidth{1.003750pt}%
\definecolor{currentstroke}{rgb}{0.121569,0.466667,0.705882}%
\pgfsetstrokecolor{currentstroke}%
\pgfsetdash{}{0pt}%
\pgfpathmoveto{\pgfqpoint{1.120243in}{1.030422in}}%
\pgfpathcurveto{\pgfqpoint{1.126067in}{1.030422in}}{\pgfqpoint{1.131653in}{1.032736in}}{\pgfqpoint{1.135771in}{1.036854in}}%
\pgfpathcurveto{\pgfqpoint{1.139890in}{1.040972in}}{\pgfqpoint{1.142203in}{1.046559in}}{\pgfqpoint{1.142203in}{1.052383in}}%
\pgfpathcurveto{\pgfqpoint{1.142203in}{1.058206in}}{\pgfqpoint{1.139890in}{1.063793in}}{\pgfqpoint{1.135771in}{1.067911in}}%
\pgfpathcurveto{\pgfqpoint{1.131653in}{1.072029in}}{\pgfqpoint{1.126067in}{1.074343in}}{\pgfqpoint{1.120243in}{1.074343in}}%
\pgfpathcurveto{\pgfqpoint{1.114419in}{1.074343in}}{\pgfqpoint{1.108833in}{1.072029in}}{\pgfqpoint{1.104715in}{1.067911in}}%
\pgfpathcurveto{\pgfqpoint{1.100597in}{1.063793in}}{\pgfqpoint{1.098283in}{1.058206in}}{\pgfqpoint{1.098283in}{1.052383in}}%
\pgfpathcurveto{\pgfqpoint{1.098283in}{1.046559in}}{\pgfqpoint{1.100597in}{1.040972in}}{\pgfqpoint{1.104715in}{1.036854in}}%
\pgfpathcurveto{\pgfqpoint{1.108833in}{1.032736in}}{\pgfqpoint{1.114419in}{1.030422in}}{\pgfqpoint{1.120243in}{1.030422in}}%
\pgfpathclose%
\pgfusepath{stroke,fill}%
\end{pgfscope}%
\begin{pgfscope}%
\pgfpathrectangle{\pgfqpoint{0.211875in}{0.211875in}}{\pgfqpoint{1.313625in}{1.279725in}}%
\pgfusepath{clip}%
\pgfsetbuttcap%
\pgfsetroundjoin%
\definecolor{currentfill}{rgb}{0.121569,0.466667,0.705882}%
\pgfsetfillcolor{currentfill}%
\pgfsetlinewidth{1.003750pt}%
\definecolor{currentstroke}{rgb}{0.121569,0.466667,0.705882}%
\pgfsetstrokecolor{currentstroke}%
\pgfsetdash{}{0pt}%
\pgfpathmoveto{\pgfqpoint{1.122631in}{1.022171in}}%
\pgfpathcurveto{\pgfqpoint{1.128455in}{1.022171in}}{\pgfqpoint{1.134041in}{1.024485in}}{\pgfqpoint{1.138159in}{1.028603in}}%
\pgfpathcurveto{\pgfqpoint{1.142277in}{1.032721in}}{\pgfqpoint{1.144591in}{1.038307in}}{\pgfqpoint{1.144591in}{1.044131in}}%
\pgfpathcurveto{\pgfqpoint{1.144591in}{1.049955in}}{\pgfqpoint{1.142277in}{1.055541in}}{\pgfqpoint{1.138159in}{1.059660in}}%
\pgfpathcurveto{\pgfqpoint{1.134041in}{1.063778in}}{\pgfqpoint{1.128455in}{1.066092in}}{\pgfqpoint{1.122631in}{1.066092in}}%
\pgfpathcurveto{\pgfqpoint{1.116807in}{1.066092in}}{\pgfqpoint{1.111221in}{1.063778in}}{\pgfqpoint{1.107102in}{1.059660in}}%
\pgfpathcurveto{\pgfqpoint{1.102984in}{1.055541in}}{\pgfqpoint{1.100670in}{1.049955in}}{\pgfqpoint{1.100670in}{1.044131in}}%
\pgfpathcurveto{\pgfqpoint{1.100670in}{1.038307in}}{\pgfqpoint{1.102984in}{1.032721in}}{\pgfqpoint{1.107102in}{1.028603in}}%
\pgfpathcurveto{\pgfqpoint{1.111221in}{1.024485in}}{\pgfqpoint{1.116807in}{1.022171in}}{\pgfqpoint{1.122631in}{1.022171in}}%
\pgfpathclose%
\pgfusepath{stroke,fill}%
\end{pgfscope}%
\begin{pgfscope}%
\pgfpathrectangle{\pgfqpoint{0.211875in}{0.211875in}}{\pgfqpoint{1.313625in}{1.279725in}}%
\pgfusepath{clip}%
\pgfsetbuttcap%
\pgfsetroundjoin%
\definecolor{currentfill}{rgb}{0.121569,0.466667,0.705882}%
\pgfsetfillcolor{currentfill}%
\pgfsetlinewidth{1.003750pt}%
\definecolor{currentstroke}{rgb}{0.121569,0.466667,0.705882}%
\pgfsetstrokecolor{currentstroke}%
\pgfsetdash{}{0pt}%
\pgfpathmoveto{\pgfqpoint{1.117424in}{1.029251in}}%
\pgfpathcurveto{\pgfqpoint{1.123248in}{1.029251in}}{\pgfqpoint{1.128834in}{1.031565in}}{\pgfqpoint{1.132952in}{1.035683in}}%
\pgfpathcurveto{\pgfqpoint{1.137071in}{1.039801in}}{\pgfqpoint{1.139384in}{1.045388in}}{\pgfqpoint{1.139384in}{1.051212in}}%
\pgfpathcurveto{\pgfqpoint{1.139384in}{1.057036in}}{\pgfqpoint{1.137071in}{1.062622in}}{\pgfqpoint{1.132952in}{1.066740in}}%
\pgfpathcurveto{\pgfqpoint{1.128834in}{1.070858in}}{\pgfqpoint{1.123248in}{1.073172in}}{\pgfqpoint{1.117424in}{1.073172in}}%
\pgfpathcurveto{\pgfqpoint{1.111600in}{1.073172in}}{\pgfqpoint{1.106014in}{1.070858in}}{\pgfqpoint{1.101896in}{1.066740in}}%
\pgfpathcurveto{\pgfqpoint{1.097778in}{1.062622in}}{\pgfqpoint{1.095464in}{1.057036in}}{\pgfqpoint{1.095464in}{1.051212in}}%
\pgfpathcurveto{\pgfqpoint{1.095464in}{1.045388in}}{\pgfqpoint{1.097778in}{1.039801in}}{\pgfqpoint{1.101896in}{1.035683in}}%
\pgfpathcurveto{\pgfqpoint{1.106014in}{1.031565in}}{\pgfqpoint{1.111600in}{1.029251in}}{\pgfqpoint{1.117424in}{1.029251in}}%
\pgfpathclose%
\pgfusepath{stroke,fill}%
\end{pgfscope}%
\begin{pgfscope}%
\pgfpathrectangle{\pgfqpoint{0.211875in}{0.211875in}}{\pgfqpoint{1.313625in}{1.279725in}}%
\pgfusepath{clip}%
\pgfsetbuttcap%
\pgfsetroundjoin%
\definecolor{currentfill}{rgb}{0.121569,0.466667,0.705882}%
\pgfsetfillcolor{currentfill}%
\pgfsetlinewidth{1.003750pt}%
\definecolor{currentstroke}{rgb}{0.121569,0.466667,0.705882}%
\pgfsetstrokecolor{currentstroke}%
\pgfsetdash{}{0pt}%
\pgfpathmoveto{\pgfqpoint{1.118663in}{1.023678in}}%
\pgfpathcurveto{\pgfqpoint{1.124487in}{1.023678in}}{\pgfqpoint{1.130073in}{1.025992in}}{\pgfqpoint{1.134191in}{1.030110in}}%
\pgfpathcurveto{\pgfqpoint{1.138310in}{1.034228in}}{\pgfqpoint{1.140623in}{1.039814in}}{\pgfqpoint{1.140623in}{1.045638in}}%
\pgfpathcurveto{\pgfqpoint{1.140623in}{1.051462in}}{\pgfqpoint{1.138310in}{1.057048in}}{\pgfqpoint{1.134191in}{1.061166in}}%
\pgfpathcurveto{\pgfqpoint{1.130073in}{1.065284in}}{\pgfqpoint{1.124487in}{1.067598in}}{\pgfqpoint{1.118663in}{1.067598in}}%
\pgfpathcurveto{\pgfqpoint{1.112839in}{1.067598in}}{\pgfqpoint{1.107253in}{1.065284in}}{\pgfqpoint{1.103135in}{1.061166in}}%
\pgfpathcurveto{\pgfqpoint{1.099017in}{1.057048in}}{\pgfqpoint{1.096703in}{1.051462in}}{\pgfqpoint{1.096703in}{1.045638in}}%
\pgfpathcurveto{\pgfqpoint{1.096703in}{1.039814in}}{\pgfqpoint{1.099017in}{1.034228in}}{\pgfqpoint{1.103135in}{1.030110in}}%
\pgfpathcurveto{\pgfqpoint{1.107253in}{1.025992in}}{\pgfqpoint{1.112839in}{1.023678in}}{\pgfqpoint{1.118663in}{1.023678in}}%
\pgfpathclose%
\pgfusepath{stroke,fill}%
\end{pgfscope}%
\begin{pgfscope}%
\pgfpathrectangle{\pgfqpoint{0.211875in}{0.211875in}}{\pgfqpoint{1.313625in}{1.279725in}}%
\pgfusepath{clip}%
\pgfsetbuttcap%
\pgfsetroundjoin%
\definecolor{currentfill}{rgb}{0.121569,0.466667,0.705882}%
\pgfsetfillcolor{currentfill}%
\pgfsetlinewidth{1.003750pt}%
\definecolor{currentstroke}{rgb}{0.121569,0.466667,0.705882}%
\pgfsetstrokecolor{currentstroke}%
\pgfsetdash{}{0pt}%
\pgfpathmoveto{\pgfqpoint{1.114224in}{1.027952in}}%
\pgfpathcurveto{\pgfqpoint{1.120048in}{1.027952in}}{\pgfqpoint{1.125634in}{1.030266in}}{\pgfqpoint{1.129752in}{1.034384in}}%
\pgfpathcurveto{\pgfqpoint{1.133871in}{1.038502in}}{\pgfqpoint{1.136184in}{1.044089in}}{\pgfqpoint{1.136184in}{1.049913in}}%
\pgfpathcurveto{\pgfqpoint{1.136184in}{1.055736in}}{\pgfqpoint{1.133871in}{1.061323in}}{\pgfqpoint{1.129752in}{1.065441in}}%
\pgfpathcurveto{\pgfqpoint{1.125634in}{1.069559in}}{\pgfqpoint{1.120048in}{1.071873in}}{\pgfqpoint{1.114224in}{1.071873in}}%
\pgfpathcurveto{\pgfqpoint{1.108400in}{1.071873in}}{\pgfqpoint{1.102814in}{1.069559in}}{\pgfqpoint{1.098696in}{1.065441in}}%
\pgfpathcurveto{\pgfqpoint{1.094578in}{1.061323in}}{\pgfqpoint{1.092264in}{1.055736in}}{\pgfqpoint{1.092264in}{1.049913in}}%
\pgfpathcurveto{\pgfqpoint{1.092264in}{1.044089in}}{\pgfqpoint{1.094578in}{1.038502in}}{\pgfqpoint{1.098696in}{1.034384in}}%
\pgfpathcurveto{\pgfqpoint{1.102814in}{1.030266in}}{\pgfqpoint{1.108400in}{1.027952in}}{\pgfqpoint{1.114224in}{1.027952in}}%
\pgfpathclose%
\pgfusepath{stroke,fill}%
\end{pgfscope}%
\begin{pgfscope}%
\pgfpathrectangle{\pgfqpoint{0.211875in}{0.211875in}}{\pgfqpoint{1.313625in}{1.279725in}}%
\pgfusepath{clip}%
\pgfsetbuttcap%
\pgfsetroundjoin%
\definecolor{currentfill}{rgb}{0.121569,0.466667,0.705882}%
\pgfsetfillcolor{currentfill}%
\pgfsetlinewidth{1.003750pt}%
\definecolor{currentstroke}{rgb}{0.121569,0.466667,0.705882}%
\pgfsetstrokecolor{currentstroke}%
\pgfsetdash{}{0pt}%
\pgfpathmoveto{\pgfqpoint{1.114157in}{1.024836in}}%
\pgfpathcurveto{\pgfqpoint{1.119981in}{1.024836in}}{\pgfqpoint{1.125567in}{1.027150in}}{\pgfqpoint{1.129686in}{1.031268in}}%
\pgfpathcurveto{\pgfqpoint{1.133804in}{1.035387in}}{\pgfqpoint{1.136118in}{1.040973in}}{\pgfqpoint{1.136118in}{1.046797in}}%
\pgfpathcurveto{\pgfqpoint{1.136118in}{1.052621in}}{\pgfqpoint{1.133804in}{1.058207in}}{\pgfqpoint{1.129686in}{1.062325in}}%
\pgfpathcurveto{\pgfqpoint{1.125567in}{1.066443in}}{\pgfqpoint{1.119981in}{1.068757in}}{\pgfqpoint{1.114157in}{1.068757in}}%
\pgfpathcurveto{\pgfqpoint{1.108333in}{1.068757in}}{\pgfqpoint{1.102747in}{1.066443in}}{\pgfqpoint{1.098629in}{1.062325in}}%
\pgfpathcurveto{\pgfqpoint{1.094511in}{1.058207in}}{\pgfqpoint{1.092197in}{1.052621in}}{\pgfqpoint{1.092197in}{1.046797in}}%
\pgfpathcurveto{\pgfqpoint{1.092197in}{1.040973in}}{\pgfqpoint{1.094511in}{1.035387in}}{\pgfqpoint{1.098629in}{1.031268in}}%
\pgfpathcurveto{\pgfqpoint{1.102747in}{1.027150in}}{\pgfqpoint{1.108333in}{1.024836in}}{\pgfqpoint{1.114157in}{1.024836in}}%
\pgfpathclose%
\pgfusepath{stroke,fill}%
\end{pgfscope}%
\begin{pgfscope}%
\pgfpathrectangle{\pgfqpoint{0.211875in}{0.211875in}}{\pgfqpoint{1.313625in}{1.279725in}}%
\pgfusepath{clip}%
\pgfsetbuttcap%
\pgfsetroundjoin%
\definecolor{currentfill}{rgb}{0.121569,0.466667,0.705882}%
\pgfsetfillcolor{currentfill}%
\pgfsetlinewidth{1.003750pt}%
\definecolor{currentstroke}{rgb}{0.121569,0.466667,0.705882}%
\pgfsetstrokecolor{currentstroke}%
\pgfsetdash{}{0pt}%
\pgfpathmoveto{\pgfqpoint{1.110777in}{1.026655in}}%
\pgfpathcurveto{\pgfqpoint{1.116601in}{1.026655in}}{\pgfqpoint{1.122187in}{1.028968in}}{\pgfqpoint{1.126305in}{1.033087in}}%
\pgfpathcurveto{\pgfqpoint{1.130423in}{1.037205in}}{\pgfqpoint{1.132737in}{1.042791in}}{\pgfqpoint{1.132737in}{1.048615in}}%
\pgfpathcurveto{\pgfqpoint{1.132737in}{1.054439in}}{\pgfqpoint{1.130423in}{1.060025in}}{\pgfqpoint{1.126305in}{1.064143in}}%
\pgfpathcurveto{\pgfqpoint{1.122187in}{1.068261in}}{\pgfqpoint{1.116601in}{1.070575in}}{\pgfqpoint{1.110777in}{1.070575in}}%
\pgfpathcurveto{\pgfqpoint{1.104953in}{1.070575in}}{\pgfqpoint{1.099367in}{1.068261in}}{\pgfqpoint{1.095248in}{1.064143in}}%
\pgfpathcurveto{\pgfqpoint{1.091130in}{1.060025in}}{\pgfqpoint{1.088816in}{1.054439in}}{\pgfqpoint{1.088816in}{1.048615in}}%
\pgfpathcurveto{\pgfqpoint{1.088816in}{1.042791in}}{\pgfqpoint{1.091130in}{1.037205in}}{\pgfqpoint{1.095248in}{1.033087in}}%
\pgfpathcurveto{\pgfqpoint{1.099367in}{1.028968in}}{\pgfqpoint{1.104953in}{1.026655in}}{\pgfqpoint{1.110777in}{1.026655in}}%
\pgfpathclose%
\pgfusepath{stroke,fill}%
\end{pgfscope}%
\begin{pgfscope}%
\pgfpathrectangle{\pgfqpoint{0.211875in}{0.211875in}}{\pgfqpoint{1.313625in}{1.279725in}}%
\pgfusepath{clip}%
\pgfsetbuttcap%
\pgfsetroundjoin%
\definecolor{currentfill}{rgb}{0.121569,0.466667,0.705882}%
\pgfsetfillcolor{currentfill}%
\pgfsetlinewidth{1.003750pt}%
\definecolor{currentstroke}{rgb}{0.121569,0.466667,0.705882}%
\pgfsetstrokecolor{currentstroke}%
\pgfsetdash{}{0pt}%
\pgfpathmoveto{\pgfqpoint{1.213565in}{0.820615in}}%
\pgfpathcurveto{\pgfqpoint{1.219389in}{0.820615in}}{\pgfqpoint{1.224975in}{0.822929in}}{\pgfqpoint{1.229093in}{0.827047in}}%
\pgfpathcurveto{\pgfqpoint{1.233211in}{0.831166in}}{\pgfqpoint{1.235525in}{0.836752in}}{\pgfqpoint{1.235525in}{0.842576in}}%
\pgfpathcurveto{\pgfqpoint{1.235525in}{0.848400in}}{\pgfqpoint{1.233211in}{0.853986in}}{\pgfqpoint{1.229093in}{0.858104in}}%
\pgfpathcurveto{\pgfqpoint{1.224975in}{0.862222in}}{\pgfqpoint{1.219389in}{0.864536in}}{\pgfqpoint{1.213565in}{0.864536in}}%
\pgfpathcurveto{\pgfqpoint{1.207741in}{0.864536in}}{\pgfqpoint{1.202155in}{0.862222in}}{\pgfqpoint{1.198037in}{0.858104in}}%
\pgfpathcurveto{\pgfqpoint{1.193918in}{0.853986in}}{\pgfqpoint{1.191605in}{0.848400in}}{\pgfqpoint{1.191605in}{0.842576in}}%
\pgfpathcurveto{\pgfqpoint{1.191605in}{0.836752in}}{\pgfqpoint{1.193918in}{0.831166in}}{\pgfqpoint{1.198037in}{0.827047in}}%
\pgfpathcurveto{\pgfqpoint{1.202155in}{0.822929in}}{\pgfqpoint{1.207741in}{0.820615in}}{\pgfqpoint{1.213565in}{0.820615in}}%
\pgfpathclose%
\pgfusepath{stroke,fill}%
\end{pgfscope}%
\begin{pgfscope}%
\pgfpathrectangle{\pgfqpoint{0.211875in}{0.211875in}}{\pgfqpoint{1.313625in}{1.279725in}}%
\pgfusepath{clip}%
\pgfsetbuttcap%
\pgfsetroundjoin%
\definecolor{currentfill}{rgb}{0.121569,0.466667,0.705882}%
\pgfsetfillcolor{currentfill}%
\pgfsetlinewidth{1.003750pt}%
\definecolor{currentstroke}{rgb}{0.121569,0.466667,0.705882}%
\pgfsetstrokecolor{currentstroke}%
\pgfsetdash{}{0pt}%
\pgfpathmoveto{\pgfqpoint{1.117640in}{1.007767in}}%
\pgfpathcurveto{\pgfqpoint{1.123464in}{1.007767in}}{\pgfqpoint{1.129050in}{1.010081in}}{\pgfqpoint{1.133168in}{1.014199in}}%
\pgfpathcurveto{\pgfqpoint{1.137286in}{1.018317in}}{\pgfqpoint{1.139600in}{1.023903in}}{\pgfqpoint{1.139600in}{1.029727in}}%
\pgfpathcurveto{\pgfqpoint{1.139600in}{1.035551in}}{\pgfqpoint{1.137286in}{1.041137in}}{\pgfqpoint{1.133168in}{1.045256in}}%
\pgfpathcurveto{\pgfqpoint{1.129050in}{1.049374in}}{\pgfqpoint{1.123464in}{1.051688in}}{\pgfqpoint{1.117640in}{1.051688in}}%
\pgfpathcurveto{\pgfqpoint{1.111816in}{1.051688in}}{\pgfqpoint{1.106229in}{1.049374in}}{\pgfqpoint{1.102111in}{1.045256in}}%
\pgfpathcurveto{\pgfqpoint{1.097993in}{1.041137in}}{\pgfqpoint{1.095679in}{1.035551in}}{\pgfqpoint{1.095679in}{1.029727in}}%
\pgfpathcurveto{\pgfqpoint{1.095679in}{1.023903in}}{\pgfqpoint{1.097993in}{1.018317in}}{\pgfqpoint{1.102111in}{1.014199in}}%
\pgfpathcurveto{\pgfqpoint{1.106229in}{1.010081in}}{\pgfqpoint{1.111816in}{1.007767in}}{\pgfqpoint{1.117640in}{1.007767in}}%
\pgfpathclose%
\pgfusepath{stroke,fill}%
\end{pgfscope}%
\begin{pgfscope}%
\pgfpathrectangle{\pgfqpoint{0.211875in}{0.211875in}}{\pgfqpoint{1.313625in}{1.279725in}}%
\pgfusepath{clip}%
\pgfsetbuttcap%
\pgfsetroundjoin%
\definecolor{currentfill}{rgb}{0.121569,0.466667,0.705882}%
\pgfsetfillcolor{currentfill}%
\pgfsetlinewidth{1.003750pt}%
\definecolor{currentstroke}{rgb}{0.121569,0.466667,0.705882}%
\pgfsetstrokecolor{currentstroke}%
\pgfsetdash{}{0pt}%
\pgfpathmoveto{\pgfqpoint{1.105659in}{1.025688in}}%
\pgfpathcurveto{\pgfqpoint{1.111483in}{1.025688in}}{\pgfqpoint{1.117069in}{1.028002in}}{\pgfqpoint{1.121188in}{1.032120in}}%
\pgfpathcurveto{\pgfqpoint{1.125306in}{1.036238in}}{\pgfqpoint{1.127620in}{1.041824in}}{\pgfqpoint{1.127620in}{1.047648in}}%
\pgfpathcurveto{\pgfqpoint{1.127620in}{1.053472in}}{\pgfqpoint{1.125306in}{1.059059in}}{\pgfqpoint{1.121188in}{1.063177in}}%
\pgfpathcurveto{\pgfqpoint{1.117069in}{1.067295in}}{\pgfqpoint{1.111483in}{1.069609in}}{\pgfqpoint{1.105659in}{1.069609in}}%
\pgfpathcurveto{\pgfqpoint{1.099835in}{1.069609in}}{\pgfqpoint{1.094249in}{1.067295in}}{\pgfqpoint{1.090131in}{1.063177in}}%
\pgfpathcurveto{\pgfqpoint{1.086013in}{1.059059in}}{\pgfqpoint{1.083699in}{1.053472in}}{\pgfqpoint{1.083699in}{1.047648in}}%
\pgfpathcurveto{\pgfqpoint{1.083699in}{1.041824in}}{\pgfqpoint{1.086013in}{1.036238in}}{\pgfqpoint{1.090131in}{1.032120in}}%
\pgfpathcurveto{\pgfqpoint{1.094249in}{1.028002in}}{\pgfqpoint{1.099835in}{1.025688in}}{\pgfqpoint{1.105659in}{1.025688in}}%
\pgfpathclose%
\pgfusepath{stroke,fill}%
\end{pgfscope}%
\begin{pgfscope}%
\pgfpathrectangle{\pgfqpoint{0.211875in}{0.211875in}}{\pgfqpoint{1.313625in}{1.279725in}}%
\pgfusepath{clip}%
\pgfsetbuttcap%
\pgfsetroundjoin%
\definecolor{currentfill}{rgb}{0.121569,0.466667,0.705882}%
\pgfsetfillcolor{currentfill}%
\pgfsetlinewidth{1.003750pt}%
\definecolor{currentstroke}{rgb}{0.121569,0.466667,0.705882}%
\pgfsetstrokecolor{currentstroke}%
\pgfsetdash{}{0pt}%
\pgfpathmoveto{\pgfqpoint{1.104643in}{1.024434in}}%
\pgfpathcurveto{\pgfqpoint{1.110467in}{1.024434in}}{\pgfqpoint{1.116053in}{1.026748in}}{\pgfqpoint{1.120171in}{1.030866in}}%
\pgfpathcurveto{\pgfqpoint{1.124289in}{1.034985in}}{\pgfqpoint{1.126603in}{1.040571in}}{\pgfqpoint{1.126603in}{1.046395in}}%
\pgfpathcurveto{\pgfqpoint{1.126603in}{1.052219in}}{\pgfqpoint{1.124289in}{1.057805in}}{\pgfqpoint{1.120171in}{1.061923in}}%
\pgfpathcurveto{\pgfqpoint{1.116053in}{1.066041in}}{\pgfqpoint{1.110467in}{1.068355in}}{\pgfqpoint{1.104643in}{1.068355in}}%
\pgfpathcurveto{\pgfqpoint{1.098819in}{1.068355in}}{\pgfqpoint{1.093233in}{1.066041in}}{\pgfqpoint{1.089115in}{1.061923in}}%
\pgfpathcurveto{\pgfqpoint{1.084996in}{1.057805in}}{\pgfqpoint{1.082683in}{1.052219in}}{\pgfqpoint{1.082683in}{1.046395in}}%
\pgfpathcurveto{\pgfqpoint{1.082683in}{1.040571in}}{\pgfqpoint{1.084996in}{1.034985in}}{\pgfqpoint{1.089115in}{1.030866in}}%
\pgfpathcurveto{\pgfqpoint{1.093233in}{1.026748in}}{\pgfqpoint{1.098819in}{1.024434in}}{\pgfqpoint{1.104643in}{1.024434in}}%
\pgfpathclose%
\pgfusepath{stroke,fill}%
\end{pgfscope}%
\begin{pgfscope}%
\pgfpathrectangle{\pgfqpoint{0.211875in}{0.211875in}}{\pgfqpoint{1.313625in}{1.279725in}}%
\pgfusepath{clip}%
\pgfsetbuttcap%
\pgfsetroundjoin%
\definecolor{currentfill}{rgb}{0.121569,0.466667,0.705882}%
\pgfsetfillcolor{currentfill}%
\pgfsetlinewidth{1.003750pt}%
\definecolor{currentstroke}{rgb}{0.121569,0.466667,0.705882}%
\pgfsetstrokecolor{currentstroke}%
\pgfsetdash{}{0pt}%
\pgfpathmoveto{\pgfqpoint{1.102734in}{1.024122in}}%
\pgfpathcurveto{\pgfqpoint{1.108558in}{1.024122in}}{\pgfqpoint{1.114144in}{1.026436in}}{\pgfqpoint{1.118262in}{1.030554in}}%
\pgfpathcurveto{\pgfqpoint{1.122380in}{1.034673in}}{\pgfqpoint{1.124694in}{1.040259in}}{\pgfqpoint{1.124694in}{1.046083in}}%
\pgfpathcurveto{\pgfqpoint{1.124694in}{1.051907in}}{\pgfqpoint{1.122380in}{1.057493in}}{\pgfqpoint{1.118262in}{1.061611in}}%
\pgfpathcurveto{\pgfqpoint{1.114144in}{1.065729in}}{\pgfqpoint{1.108558in}{1.068043in}}{\pgfqpoint{1.102734in}{1.068043in}}%
\pgfpathcurveto{\pgfqpoint{1.096910in}{1.068043in}}{\pgfqpoint{1.091324in}{1.065729in}}{\pgfqpoint{1.087206in}{1.061611in}}%
\pgfpathcurveto{\pgfqpoint{1.083088in}{1.057493in}}{\pgfqpoint{1.080774in}{1.051907in}}{\pgfqpoint{1.080774in}{1.046083in}}%
\pgfpathcurveto{\pgfqpoint{1.080774in}{1.040259in}}{\pgfqpoint{1.083088in}{1.034673in}}{\pgfqpoint{1.087206in}{1.030554in}}%
\pgfpathcurveto{\pgfqpoint{1.091324in}{1.026436in}}{\pgfqpoint{1.096910in}{1.024122in}}{\pgfqpoint{1.102734in}{1.024122in}}%
\pgfpathclose%
\pgfusepath{stroke,fill}%
\end{pgfscope}%
\begin{pgfscope}%
\pgfpathrectangle{\pgfqpoint{0.211875in}{0.211875in}}{\pgfqpoint{1.313625in}{1.279725in}}%
\pgfusepath{clip}%
\pgfsetbuttcap%
\pgfsetroundjoin%
\definecolor{currentfill}{rgb}{0.121569,0.466667,0.705882}%
\pgfsetfillcolor{currentfill}%
\pgfsetlinewidth{1.003750pt}%
\definecolor{currentstroke}{rgb}{0.121569,0.466667,0.705882}%
\pgfsetstrokecolor{currentstroke}%
\pgfsetdash{}{0pt}%
\pgfpathmoveto{\pgfqpoint{1.101085in}{1.023554in}}%
\pgfpathcurveto{\pgfqpoint{1.106909in}{1.023554in}}{\pgfqpoint{1.112495in}{1.025868in}}{\pgfqpoint{1.116613in}{1.029986in}}%
\pgfpathcurveto{\pgfqpoint{1.120731in}{1.034105in}}{\pgfqpoint{1.123045in}{1.039691in}}{\pgfqpoint{1.123045in}{1.045515in}}%
\pgfpathcurveto{\pgfqpoint{1.123045in}{1.051339in}}{\pgfqpoint{1.120731in}{1.056925in}}{\pgfqpoint{1.116613in}{1.061043in}}%
\pgfpathcurveto{\pgfqpoint{1.112495in}{1.065161in}}{\pgfqpoint{1.106909in}{1.067475in}}{\pgfqpoint{1.101085in}{1.067475in}}%
\pgfpathcurveto{\pgfqpoint{1.095261in}{1.067475in}}{\pgfqpoint{1.089675in}{1.065161in}}{\pgfqpoint{1.085557in}{1.061043in}}%
\pgfpathcurveto{\pgfqpoint{1.081439in}{1.056925in}}{\pgfqpoint{1.079125in}{1.051339in}}{\pgfqpoint{1.079125in}{1.045515in}}%
\pgfpathcurveto{\pgfqpoint{1.079125in}{1.039691in}}{\pgfqpoint{1.081439in}{1.034105in}}{\pgfqpoint{1.085557in}{1.029986in}}%
\pgfpathcurveto{\pgfqpoint{1.089675in}{1.025868in}}{\pgfqpoint{1.095261in}{1.023554in}}{\pgfqpoint{1.101085in}{1.023554in}}%
\pgfpathclose%
\pgfusepath{stroke,fill}%
\end{pgfscope}%
\begin{pgfscope}%
\pgfpathrectangle{\pgfqpoint{0.211875in}{0.211875in}}{\pgfqpoint{1.313625in}{1.279725in}}%
\pgfusepath{clip}%
\pgfsetbuttcap%
\pgfsetroundjoin%
\definecolor{currentfill}{rgb}{0.121569,0.466667,0.705882}%
\pgfsetfillcolor{currentfill}%
\pgfsetlinewidth{1.003750pt}%
\definecolor{currentstroke}{rgb}{0.121569,0.466667,0.705882}%
\pgfsetstrokecolor{currentstroke}%
\pgfsetdash{}{0pt}%
\pgfpathmoveto{\pgfqpoint{1.099561in}{1.022887in}}%
\pgfpathcurveto{\pgfqpoint{1.105385in}{1.022887in}}{\pgfqpoint{1.110971in}{1.025201in}}{\pgfqpoint{1.115089in}{1.029319in}}%
\pgfpathcurveto{\pgfqpoint{1.119207in}{1.033437in}}{\pgfqpoint{1.121521in}{1.039023in}}{\pgfqpoint{1.121521in}{1.044847in}}%
\pgfpathcurveto{\pgfqpoint{1.121521in}{1.050671in}}{\pgfqpoint{1.119207in}{1.056257in}}{\pgfqpoint{1.115089in}{1.060376in}}%
\pgfpathcurveto{\pgfqpoint{1.110971in}{1.064494in}}{\pgfqpoint{1.105385in}{1.066808in}}{\pgfqpoint{1.099561in}{1.066808in}}%
\pgfpathcurveto{\pgfqpoint{1.093737in}{1.066808in}}{\pgfqpoint{1.088151in}{1.064494in}}{\pgfqpoint{1.084033in}{1.060376in}}%
\pgfpathcurveto{\pgfqpoint{1.079914in}{1.056257in}}{\pgfqpoint{1.077601in}{1.050671in}}{\pgfqpoint{1.077601in}{1.044847in}}%
\pgfpathcurveto{\pgfqpoint{1.077601in}{1.039023in}}{\pgfqpoint{1.079914in}{1.033437in}}{\pgfqpoint{1.084033in}{1.029319in}}%
\pgfpathcurveto{\pgfqpoint{1.088151in}{1.025201in}}{\pgfqpoint{1.093737in}{1.022887in}}{\pgfqpoint{1.099561in}{1.022887in}}%
\pgfpathclose%
\pgfusepath{stroke,fill}%
\end{pgfscope}%
\begin{pgfscope}%
\pgfpathrectangle{\pgfqpoint{0.211875in}{0.211875in}}{\pgfqpoint{1.313625in}{1.279725in}}%
\pgfusepath{clip}%
\pgfsetbuttcap%
\pgfsetroundjoin%
\definecolor{currentfill}{rgb}{0.121569,0.466667,0.705882}%
\pgfsetfillcolor{currentfill}%
\pgfsetlinewidth{1.003750pt}%
\definecolor{currentstroke}{rgb}{0.121569,0.466667,0.705882}%
\pgfsetstrokecolor{currentstroke}%
\pgfsetdash{}{0pt}%
\pgfpathmoveto{\pgfqpoint{1.098206in}{1.022093in}}%
\pgfpathcurveto{\pgfqpoint{1.104030in}{1.022093in}}{\pgfqpoint{1.109616in}{1.024407in}}{\pgfqpoint{1.113735in}{1.028525in}}%
\pgfpathcurveto{\pgfqpoint{1.117853in}{1.032643in}}{\pgfqpoint{1.120167in}{1.038229in}}{\pgfqpoint{1.120167in}{1.044053in}}%
\pgfpathcurveto{\pgfqpoint{1.120167in}{1.049877in}}{\pgfqpoint{1.117853in}{1.055463in}}{\pgfqpoint{1.113735in}{1.059582in}}%
\pgfpathcurveto{\pgfqpoint{1.109616in}{1.063700in}}{\pgfqpoint{1.104030in}{1.066014in}}{\pgfqpoint{1.098206in}{1.066014in}}%
\pgfpathcurveto{\pgfqpoint{1.092382in}{1.066014in}}{\pgfqpoint{1.086796in}{1.063700in}}{\pgfqpoint{1.082678in}{1.059582in}}%
\pgfpathcurveto{\pgfqpoint{1.078560in}{1.055463in}}{\pgfqpoint{1.076246in}{1.049877in}}{\pgfqpoint{1.076246in}{1.044053in}}%
\pgfpathcurveto{\pgfqpoint{1.076246in}{1.038229in}}{\pgfqpoint{1.078560in}{1.032643in}}{\pgfqpoint{1.082678in}{1.028525in}}%
\pgfpathcurveto{\pgfqpoint{1.086796in}{1.024407in}}{\pgfqpoint{1.092382in}{1.022093in}}{\pgfqpoint{1.098206in}{1.022093in}}%
\pgfpathclose%
\pgfusepath{stroke,fill}%
\end{pgfscope}%
\begin{pgfscope}%
\pgfpathrectangle{\pgfqpoint{0.211875in}{0.211875in}}{\pgfqpoint{1.313625in}{1.279725in}}%
\pgfusepath{clip}%
\pgfsetbuttcap%
\pgfsetroundjoin%
\definecolor{currentfill}{rgb}{0.121569,0.466667,0.705882}%
\pgfsetfillcolor{currentfill}%
\pgfsetlinewidth{1.003750pt}%
\definecolor{currentstroke}{rgb}{0.121569,0.466667,0.705882}%
\pgfsetstrokecolor{currentstroke}%
\pgfsetdash{}{0pt}%
\pgfpathmoveto{\pgfqpoint{1.152426in}{0.765321in}}%
\pgfpathcurveto{\pgfqpoint{1.158250in}{0.765321in}}{\pgfqpoint{1.163836in}{0.767635in}}{\pgfqpoint{1.167954in}{0.771753in}}%
\pgfpathcurveto{\pgfqpoint{1.172072in}{0.775871in}}{\pgfqpoint{1.174386in}{0.781457in}}{\pgfqpoint{1.174386in}{0.787281in}}%
\pgfpathcurveto{\pgfqpoint{1.174386in}{0.793105in}}{\pgfqpoint{1.172072in}{0.798691in}}{\pgfqpoint{1.167954in}{0.802809in}}%
\pgfpathcurveto{\pgfqpoint{1.163836in}{0.806927in}}{\pgfqpoint{1.158250in}{0.809241in}}{\pgfqpoint{1.152426in}{0.809241in}}%
\pgfpathcurveto{\pgfqpoint{1.146602in}{0.809241in}}{\pgfqpoint{1.141016in}{0.806927in}}{\pgfqpoint{1.136898in}{0.802809in}}%
\pgfpathcurveto{\pgfqpoint{1.132779in}{0.798691in}}{\pgfqpoint{1.130466in}{0.793105in}}{\pgfqpoint{1.130466in}{0.787281in}}%
\pgfpathcurveto{\pgfqpoint{1.130466in}{0.781457in}}{\pgfqpoint{1.132779in}{0.775871in}}{\pgfqpoint{1.136898in}{0.771753in}}%
\pgfpathcurveto{\pgfqpoint{1.141016in}{0.767635in}}{\pgfqpoint{1.146602in}{0.765321in}}{\pgfqpoint{1.152426in}{0.765321in}}%
\pgfpathclose%
\pgfusepath{stroke,fill}%
\end{pgfscope}%
\begin{pgfscope}%
\pgfpathrectangle{\pgfqpoint{0.211875in}{0.211875in}}{\pgfqpoint{1.313625in}{1.279725in}}%
\pgfusepath{clip}%
\pgfsetbuttcap%
\pgfsetroundjoin%
\definecolor{currentfill}{rgb}{0.121569,0.466667,0.705882}%
\pgfsetfillcolor{currentfill}%
\pgfsetlinewidth{1.003750pt}%
\definecolor{currentstroke}{rgb}{0.121569,0.466667,0.705882}%
\pgfsetstrokecolor{currentstroke}%
\pgfsetdash{}{0pt}%
\pgfpathmoveto{\pgfqpoint{1.094619in}{0.996958in}}%
\pgfpathcurveto{\pgfqpoint{1.100443in}{0.996958in}}{\pgfqpoint{1.106029in}{0.999272in}}{\pgfqpoint{1.110147in}{1.003390in}}%
\pgfpathcurveto{\pgfqpoint{1.114265in}{1.007508in}}{\pgfqpoint{1.116579in}{1.013094in}}{\pgfqpoint{1.116579in}{1.018918in}}%
\pgfpathcurveto{\pgfqpoint{1.116579in}{1.024742in}}{\pgfqpoint{1.114265in}{1.030328in}}{\pgfqpoint{1.110147in}{1.034447in}}%
\pgfpathcurveto{\pgfqpoint{1.106029in}{1.038565in}}{\pgfqpoint{1.100443in}{1.040879in}}{\pgfqpoint{1.094619in}{1.040879in}}%
\pgfpathcurveto{\pgfqpoint{1.088795in}{1.040879in}}{\pgfqpoint{1.083209in}{1.038565in}}{\pgfqpoint{1.079091in}{1.034447in}}%
\pgfpathcurveto{\pgfqpoint{1.074973in}{1.030328in}}{\pgfqpoint{1.072659in}{1.024742in}}{\pgfqpoint{1.072659in}{1.018918in}}%
\pgfpathcurveto{\pgfqpoint{1.072659in}{1.013094in}}{\pgfqpoint{1.074973in}{1.007508in}}{\pgfqpoint{1.079091in}{1.003390in}}%
\pgfpathcurveto{\pgfqpoint{1.083209in}{0.999272in}}{\pgfqpoint{1.088795in}{0.996958in}}{\pgfqpoint{1.094619in}{0.996958in}}%
\pgfpathclose%
\pgfusepath{stroke,fill}%
\end{pgfscope}%
\begin{pgfscope}%
\pgfpathrectangle{\pgfqpoint{0.211875in}{0.211875in}}{\pgfqpoint{1.313625in}{1.279725in}}%
\pgfusepath{clip}%
\pgfsetbuttcap%
\pgfsetroundjoin%
\definecolor{currentfill}{rgb}{0.121569,0.466667,0.705882}%
\pgfsetfillcolor{currentfill}%
\pgfsetlinewidth{1.003750pt}%
\definecolor{currentstroke}{rgb}{0.121569,0.466667,0.705882}%
\pgfsetstrokecolor{currentstroke}%
\pgfsetdash{}{0pt}%
\pgfpathmoveto{\pgfqpoint{1.091136in}{0.996927in}}%
\pgfpathcurveto{\pgfqpoint{1.096960in}{0.996927in}}{\pgfqpoint{1.102547in}{0.999241in}}{\pgfqpoint{1.106665in}{1.003359in}}%
\pgfpathcurveto{\pgfqpoint{1.110783in}{1.007477in}}{\pgfqpoint{1.113097in}{1.013063in}}{\pgfqpoint{1.113097in}{1.018887in}}%
\pgfpathcurveto{\pgfqpoint{1.113097in}{1.024711in}}{\pgfqpoint{1.110783in}{1.030297in}}{\pgfqpoint{1.106665in}{1.034416in}}%
\pgfpathcurveto{\pgfqpoint{1.102547in}{1.038534in}}{\pgfqpoint{1.096960in}{1.040848in}}{\pgfqpoint{1.091136in}{1.040848in}}%
\pgfpathcurveto{\pgfqpoint{1.085312in}{1.040848in}}{\pgfqpoint{1.079726in}{1.038534in}}{\pgfqpoint{1.075608in}{1.034416in}}%
\pgfpathcurveto{\pgfqpoint{1.071490in}{1.030297in}}{\pgfqpoint{1.069176in}{1.024711in}}{\pgfqpoint{1.069176in}{1.018887in}}%
\pgfpathcurveto{\pgfqpoint{1.069176in}{1.013063in}}{\pgfqpoint{1.071490in}{1.007477in}}{\pgfqpoint{1.075608in}{1.003359in}}%
\pgfpathcurveto{\pgfqpoint{1.079726in}{0.999241in}}{\pgfqpoint{1.085312in}{0.996927in}}{\pgfqpoint{1.091136in}{0.996927in}}%
\pgfpathclose%
\pgfusepath{stroke,fill}%
\end{pgfscope}%
\begin{pgfscope}%
\pgfpathrectangle{\pgfqpoint{0.211875in}{0.211875in}}{\pgfqpoint{1.313625in}{1.279725in}}%
\pgfusepath{clip}%
\pgfsetbuttcap%
\pgfsetroundjoin%
\definecolor{currentfill}{rgb}{0.121569,0.466667,0.705882}%
\pgfsetfillcolor{currentfill}%
\pgfsetlinewidth{1.003750pt}%
\definecolor{currentstroke}{rgb}{0.121569,0.466667,0.705882}%
\pgfsetstrokecolor{currentstroke}%
\pgfsetdash{}{0pt}%
\pgfpathmoveto{\pgfqpoint{1.094370in}{0.994785in}}%
\pgfpathcurveto{\pgfqpoint{1.100194in}{0.994785in}}{\pgfqpoint{1.105780in}{0.997099in}}{\pgfqpoint{1.109898in}{1.001217in}}%
\pgfpathcurveto{\pgfqpoint{1.114016in}{1.005335in}}{\pgfqpoint{1.116330in}{1.010921in}}{\pgfqpoint{1.116330in}{1.016745in}}%
\pgfpathcurveto{\pgfqpoint{1.116330in}{1.022569in}}{\pgfqpoint{1.114016in}{1.028155in}}{\pgfqpoint{1.109898in}{1.032273in}}%
\pgfpathcurveto{\pgfqpoint{1.105780in}{1.036391in}}{\pgfqpoint{1.100194in}{1.038705in}}{\pgfqpoint{1.094370in}{1.038705in}}%
\pgfpathcurveto{\pgfqpoint{1.088546in}{1.038705in}}{\pgfqpoint{1.082960in}{1.036391in}}{\pgfqpoint{1.078842in}{1.032273in}}%
\pgfpathcurveto{\pgfqpoint{1.074724in}{1.028155in}}{\pgfqpoint{1.072410in}{1.022569in}}{\pgfqpoint{1.072410in}{1.016745in}}%
\pgfpathcurveto{\pgfqpoint{1.072410in}{1.010921in}}{\pgfqpoint{1.074724in}{1.005335in}}{\pgfqpoint{1.078842in}{1.001217in}}%
\pgfpathcurveto{\pgfqpoint{1.082960in}{0.997099in}}{\pgfqpoint{1.088546in}{0.994785in}}{\pgfqpoint{1.094370in}{0.994785in}}%
\pgfpathclose%
\pgfusepath{stroke,fill}%
\end{pgfscope}%
\begin{pgfscope}%
\pgfpathrectangle{\pgfqpoint{0.211875in}{0.211875in}}{\pgfqpoint{1.313625in}{1.279725in}}%
\pgfusepath{clip}%
\pgfsetbuttcap%
\pgfsetroundjoin%
\definecolor{currentfill}{rgb}{0.121569,0.466667,0.705882}%
\pgfsetfillcolor{currentfill}%
\pgfsetlinewidth{1.003750pt}%
\definecolor{currentstroke}{rgb}{0.121569,0.466667,0.705882}%
\pgfsetstrokecolor{currentstroke}%
\pgfsetdash{}{0pt}%
\pgfpathmoveto{\pgfqpoint{1.091385in}{0.994535in}}%
\pgfpathcurveto{\pgfqpoint{1.097209in}{0.994535in}}{\pgfqpoint{1.102795in}{0.996849in}}{\pgfqpoint{1.106913in}{1.000967in}}%
\pgfpathcurveto{\pgfqpoint{1.111031in}{1.005086in}}{\pgfqpoint{1.113345in}{1.010672in}}{\pgfqpoint{1.113345in}{1.016496in}}%
\pgfpathcurveto{\pgfqpoint{1.113345in}{1.022320in}}{\pgfqpoint{1.111031in}{1.027906in}}{\pgfqpoint{1.106913in}{1.032024in}}%
\pgfpathcurveto{\pgfqpoint{1.102795in}{1.036142in}}{\pgfqpoint{1.097209in}{1.038456in}}{\pgfqpoint{1.091385in}{1.038456in}}%
\pgfpathcurveto{\pgfqpoint{1.085561in}{1.038456in}}{\pgfqpoint{1.079975in}{1.036142in}}{\pgfqpoint{1.075856in}{1.032024in}}%
\pgfpathcurveto{\pgfqpoint{1.071738in}{1.027906in}}{\pgfqpoint{1.069424in}{1.022320in}}{\pgfqpoint{1.069424in}{1.016496in}}%
\pgfpathcurveto{\pgfqpoint{1.069424in}{1.010672in}}{\pgfqpoint{1.071738in}{1.005086in}}{\pgfqpoint{1.075856in}{1.000967in}}%
\pgfpathcurveto{\pgfqpoint{1.079975in}{0.996849in}}{\pgfqpoint{1.085561in}{0.994535in}}{\pgfqpoint{1.091385in}{0.994535in}}%
\pgfpathclose%
\pgfusepath{stroke,fill}%
\end{pgfscope}%
\begin{pgfscope}%
\pgfpathrectangle{\pgfqpoint{0.211875in}{0.211875in}}{\pgfqpoint{1.313625in}{1.279725in}}%
\pgfusepath{clip}%
\pgfsetbuttcap%
\pgfsetroundjoin%
\definecolor{currentfill}{rgb}{0.121569,0.466667,0.705882}%
\pgfsetfillcolor{currentfill}%
\pgfsetlinewidth{1.003750pt}%
\definecolor{currentstroke}{rgb}{0.121569,0.466667,0.705882}%
\pgfsetstrokecolor{currentstroke}%
\pgfsetdash{}{0pt}%
\pgfpathmoveto{\pgfqpoint{1.093945in}{0.992699in}}%
\pgfpathcurveto{\pgfqpoint{1.099769in}{0.992699in}}{\pgfqpoint{1.105355in}{0.995013in}}{\pgfqpoint{1.109473in}{0.999131in}}%
\pgfpathcurveto{\pgfqpoint{1.113591in}{1.003249in}}{\pgfqpoint{1.115905in}{1.008835in}}{\pgfqpoint{1.115905in}{1.014659in}}%
\pgfpathcurveto{\pgfqpoint{1.115905in}{1.020483in}}{\pgfqpoint{1.113591in}{1.026069in}}{\pgfqpoint{1.109473in}{1.030187in}}%
\pgfpathcurveto{\pgfqpoint{1.105355in}{1.034305in}}{\pgfqpoint{1.099769in}{1.036619in}}{\pgfqpoint{1.093945in}{1.036619in}}%
\pgfpathcurveto{\pgfqpoint{1.088121in}{1.036619in}}{\pgfqpoint{1.082535in}{1.034305in}}{\pgfqpoint{1.078417in}{1.030187in}}%
\pgfpathcurveto{\pgfqpoint{1.074299in}{1.026069in}}{\pgfqpoint{1.071985in}{1.020483in}}{\pgfqpoint{1.071985in}{1.014659in}}%
\pgfpathcurveto{\pgfqpoint{1.071985in}{1.008835in}}{\pgfqpoint{1.074299in}{1.003249in}}{\pgfqpoint{1.078417in}{0.999131in}}%
\pgfpathcurveto{\pgfqpoint{1.082535in}{0.995013in}}{\pgfqpoint{1.088121in}{0.992699in}}{\pgfqpoint{1.093945in}{0.992699in}}%
\pgfpathclose%
\pgfusepath{stroke,fill}%
\end{pgfscope}%
\begin{pgfscope}%
\pgfpathrectangle{\pgfqpoint{0.211875in}{0.211875in}}{\pgfqpoint{1.313625in}{1.279725in}}%
\pgfusepath{clip}%
\pgfsetbuttcap%
\pgfsetroundjoin%
\definecolor{currentfill}{rgb}{0.121569,0.466667,0.705882}%
\pgfsetfillcolor{currentfill}%
\pgfsetlinewidth{1.003750pt}%
\definecolor{currentstroke}{rgb}{0.121569,0.466667,0.705882}%
\pgfsetstrokecolor{currentstroke}%
\pgfsetdash{}{0pt}%
\pgfpathmoveto{\pgfqpoint{1.201229in}{0.748789in}}%
\pgfpathcurveto{\pgfqpoint{1.207053in}{0.748789in}}{\pgfqpoint{1.212639in}{0.751103in}}{\pgfqpoint{1.216757in}{0.755221in}}%
\pgfpathcurveto{\pgfqpoint{1.220875in}{0.759340in}}{\pgfqpoint{1.223189in}{0.764926in}}{\pgfqpoint{1.223189in}{0.770750in}}%
\pgfpathcurveto{\pgfqpoint{1.223189in}{0.776574in}}{\pgfqpoint{1.220875in}{0.782160in}}{\pgfqpoint{1.216757in}{0.786278in}}%
\pgfpathcurveto{\pgfqpoint{1.212639in}{0.790396in}}{\pgfqpoint{1.207053in}{0.792710in}}{\pgfqpoint{1.201229in}{0.792710in}}%
\pgfpathcurveto{\pgfqpoint{1.195405in}{0.792710in}}{\pgfqpoint{1.189819in}{0.790396in}}{\pgfqpoint{1.185701in}{0.786278in}}%
\pgfpathcurveto{\pgfqpoint{1.181583in}{0.782160in}}{\pgfqpoint{1.179269in}{0.776574in}}{\pgfqpoint{1.179269in}{0.770750in}}%
\pgfpathcurveto{\pgfqpoint{1.179269in}{0.764926in}}{\pgfqpoint{1.181583in}{0.759340in}}{\pgfqpoint{1.185701in}{0.755221in}}%
\pgfpathcurveto{\pgfqpoint{1.189819in}{0.751103in}}{\pgfqpoint{1.195405in}{0.748789in}}{\pgfqpoint{1.201229in}{0.748789in}}%
\pgfpathclose%
\pgfusepath{stroke,fill}%
\end{pgfscope}%
\begin{pgfscope}%
\pgfpathrectangle{\pgfqpoint{0.211875in}{0.211875in}}{\pgfqpoint{1.313625in}{1.279725in}}%
\pgfusepath{clip}%
\pgfsetbuttcap%
\pgfsetroundjoin%
\definecolor{currentfill}{rgb}{0.121569,0.466667,0.705882}%
\pgfsetfillcolor{currentfill}%
\pgfsetlinewidth{1.003750pt}%
\definecolor{currentstroke}{rgb}{0.121569,0.466667,0.705882}%
\pgfsetstrokecolor{currentstroke}%
\pgfsetdash{}{0pt}%
\pgfpathmoveto{\pgfqpoint{1.089852in}{0.998443in}}%
\pgfpathcurveto{\pgfqpoint{1.095676in}{0.998443in}}{\pgfqpoint{1.101262in}{1.000757in}}{\pgfqpoint{1.105380in}{1.004875in}}%
\pgfpathcurveto{\pgfqpoint{1.109499in}{1.008993in}}{\pgfqpoint{1.111812in}{1.014579in}}{\pgfqpoint{1.111812in}{1.020403in}}%
\pgfpathcurveto{\pgfqpoint{1.111812in}{1.026227in}}{\pgfqpoint{1.109499in}{1.031814in}}{\pgfqpoint{1.105380in}{1.035932in}}%
\pgfpathcurveto{\pgfqpoint{1.101262in}{1.040050in}}{\pgfqpoint{1.095676in}{1.042364in}}{\pgfqpoint{1.089852in}{1.042364in}}%
\pgfpathcurveto{\pgfqpoint{1.084028in}{1.042364in}}{\pgfqpoint{1.078442in}{1.040050in}}{\pgfqpoint{1.074324in}{1.035932in}}%
\pgfpathcurveto{\pgfqpoint{1.070206in}{1.031814in}}{\pgfqpoint{1.067892in}{1.026227in}}{\pgfqpoint{1.067892in}{1.020403in}}%
\pgfpathcurveto{\pgfqpoint{1.067892in}{1.014579in}}{\pgfqpoint{1.070206in}{1.008993in}}{\pgfqpoint{1.074324in}{1.004875in}}%
\pgfpathcurveto{\pgfqpoint{1.078442in}{1.000757in}}{\pgfqpoint{1.084028in}{0.998443in}}{\pgfqpoint{1.089852in}{0.998443in}}%
\pgfpathclose%
\pgfusepath{stroke,fill}%
\end{pgfscope}%
\begin{pgfscope}%
\pgfpathrectangle{\pgfqpoint{0.211875in}{0.211875in}}{\pgfqpoint{1.313625in}{1.279725in}}%
\pgfusepath{clip}%
\pgfsetbuttcap%
\pgfsetroundjoin%
\definecolor{currentfill}{rgb}{0.121569,0.466667,0.705882}%
\pgfsetfillcolor{currentfill}%
\pgfsetlinewidth{1.003750pt}%
\definecolor{currentstroke}{rgb}{0.121569,0.466667,0.705882}%
\pgfsetstrokecolor{currentstroke}%
\pgfsetdash{}{0pt}%
\pgfpathmoveto{\pgfqpoint{1.095206in}{0.995699in}}%
\pgfpathcurveto{\pgfqpoint{1.101030in}{0.995699in}}{\pgfqpoint{1.106616in}{0.998013in}}{\pgfqpoint{1.110734in}{1.002131in}}%
\pgfpathcurveto{\pgfqpoint{1.114852in}{1.006249in}}{\pgfqpoint{1.117166in}{1.011835in}}{\pgfqpoint{1.117166in}{1.017659in}}%
\pgfpathcurveto{\pgfqpoint{1.117166in}{1.023483in}}{\pgfqpoint{1.114852in}{1.029070in}}{\pgfqpoint{1.110734in}{1.033188in}}%
\pgfpathcurveto{\pgfqpoint{1.106616in}{1.037306in}}{\pgfqpoint{1.101030in}{1.039620in}}{\pgfqpoint{1.095206in}{1.039620in}}%
\pgfpathcurveto{\pgfqpoint{1.089382in}{1.039620in}}{\pgfqpoint{1.083796in}{1.037306in}}{\pgfqpoint{1.079678in}{1.033188in}}%
\pgfpathcurveto{\pgfqpoint{1.075559in}{1.029070in}}{\pgfqpoint{1.073246in}{1.023483in}}{\pgfqpoint{1.073246in}{1.017659in}}%
\pgfpathcurveto{\pgfqpoint{1.073246in}{1.011835in}}{\pgfqpoint{1.075559in}{1.006249in}}{\pgfqpoint{1.079678in}{1.002131in}}%
\pgfpathcurveto{\pgfqpoint{1.083796in}{0.998013in}}{\pgfqpoint{1.089382in}{0.995699in}}{\pgfqpoint{1.095206in}{0.995699in}}%
\pgfpathclose%
\pgfusepath{stroke,fill}%
\end{pgfscope}%
\begin{pgfscope}%
\pgfpathrectangle{\pgfqpoint{0.211875in}{0.211875in}}{\pgfqpoint{1.313625in}{1.279725in}}%
\pgfusepath{clip}%
\pgfsetbuttcap%
\pgfsetroundjoin%
\definecolor{currentfill}{rgb}{0.121569,0.466667,0.705882}%
\pgfsetfillcolor{currentfill}%
\pgfsetlinewidth{1.003750pt}%
\definecolor{currentstroke}{rgb}{0.121569,0.466667,0.705882}%
\pgfsetstrokecolor{currentstroke}%
\pgfsetdash{}{0pt}%
\pgfpathmoveto{\pgfqpoint{1.091247in}{0.996101in}}%
\pgfpathcurveto{\pgfqpoint{1.097071in}{0.996101in}}{\pgfqpoint{1.102657in}{0.998415in}}{\pgfqpoint{1.106775in}{1.002533in}}%
\pgfpathcurveto{\pgfqpoint{1.110893in}{1.006651in}}{\pgfqpoint{1.113207in}{1.012237in}}{\pgfqpoint{1.113207in}{1.018061in}}%
\pgfpathcurveto{\pgfqpoint{1.113207in}{1.023885in}}{\pgfqpoint{1.110893in}{1.029471in}}{\pgfqpoint{1.106775in}{1.033589in}}%
\pgfpathcurveto{\pgfqpoint{1.102657in}{1.037707in}}{\pgfqpoint{1.097071in}{1.040021in}}{\pgfqpoint{1.091247in}{1.040021in}}%
\pgfpathcurveto{\pgfqpoint{1.085423in}{1.040021in}}{\pgfqpoint{1.079837in}{1.037707in}}{\pgfqpoint{1.075718in}{1.033589in}}%
\pgfpathcurveto{\pgfqpoint{1.071600in}{1.029471in}}{\pgfqpoint{1.069286in}{1.023885in}}{\pgfqpoint{1.069286in}{1.018061in}}%
\pgfpathcurveto{\pgfqpoint{1.069286in}{1.012237in}}{\pgfqpoint{1.071600in}{1.006651in}}{\pgfqpoint{1.075718in}{1.002533in}}%
\pgfpathcurveto{\pgfqpoint{1.079837in}{0.998415in}}{\pgfqpoint{1.085423in}{0.996101in}}{\pgfqpoint{1.091247in}{0.996101in}}%
\pgfpathclose%
\pgfusepath{stroke,fill}%
\end{pgfscope}%
\begin{pgfscope}%
\pgfpathrectangle{\pgfqpoint{0.211875in}{0.211875in}}{\pgfqpoint{1.313625in}{1.279725in}}%
\pgfusepath{clip}%
\pgfsetbuttcap%
\pgfsetroundjoin%
\definecolor{currentfill}{rgb}{0.121569,0.466667,0.705882}%
\pgfsetfillcolor{currentfill}%
\pgfsetlinewidth{1.003750pt}%
\definecolor{currentstroke}{rgb}{0.121569,0.466667,0.705882}%
\pgfsetstrokecolor{currentstroke}%
\pgfsetdash{}{0pt}%
\pgfpathmoveto{\pgfqpoint{1.093728in}{0.994299in}}%
\pgfpathcurveto{\pgfqpoint{1.099552in}{0.994299in}}{\pgfqpoint{1.105138in}{0.996613in}}{\pgfqpoint{1.109256in}{1.000731in}}%
\pgfpathcurveto{\pgfqpoint{1.113374in}{1.004849in}}{\pgfqpoint{1.115688in}{1.010436in}}{\pgfqpoint{1.115688in}{1.016259in}}%
\pgfpathcurveto{\pgfqpoint{1.115688in}{1.022083in}}{\pgfqpoint{1.113374in}{1.027670in}}{\pgfqpoint{1.109256in}{1.031788in}}%
\pgfpathcurveto{\pgfqpoint{1.105138in}{1.035906in}}{\pgfqpoint{1.099552in}{1.038220in}}{\pgfqpoint{1.093728in}{1.038220in}}%
\pgfpathcurveto{\pgfqpoint{1.087904in}{1.038220in}}{\pgfqpoint{1.082318in}{1.035906in}}{\pgfqpoint{1.078200in}{1.031788in}}%
\pgfpathcurveto{\pgfqpoint{1.074081in}{1.027670in}}{\pgfqpoint{1.071768in}{1.022083in}}{\pgfqpoint{1.071768in}{1.016259in}}%
\pgfpathcurveto{\pgfqpoint{1.071768in}{1.010436in}}{\pgfqpoint{1.074081in}{1.004849in}}{\pgfqpoint{1.078200in}{1.000731in}}%
\pgfpathcurveto{\pgfqpoint{1.082318in}{0.996613in}}{\pgfqpoint{1.087904in}{0.994299in}}{\pgfqpoint{1.093728in}{0.994299in}}%
\pgfpathclose%
\pgfusepath{stroke,fill}%
\end{pgfscope}%
\begin{pgfscope}%
\pgfpathrectangle{\pgfqpoint{0.211875in}{0.211875in}}{\pgfqpoint{1.313625in}{1.279725in}}%
\pgfusepath{clip}%
\pgfsetbuttcap%
\pgfsetroundjoin%
\definecolor{currentfill}{rgb}{0.121569,0.466667,0.705882}%
\pgfsetfillcolor{currentfill}%
\pgfsetlinewidth{1.003750pt}%
\definecolor{currentstroke}{rgb}{0.121569,0.466667,0.705882}%
\pgfsetstrokecolor{currentstroke}%
\pgfsetdash{}{0pt}%
\pgfpathmoveto{\pgfqpoint{1.091997in}{0.993816in}}%
\pgfpathcurveto{\pgfqpoint{1.097821in}{0.993816in}}{\pgfqpoint{1.103408in}{0.996130in}}{\pgfqpoint{1.107526in}{1.000248in}}%
\pgfpathcurveto{\pgfqpoint{1.111644in}{1.004366in}}{\pgfqpoint{1.113958in}{1.009952in}}{\pgfqpoint{1.113958in}{1.015776in}}%
\pgfpathcurveto{\pgfqpoint{1.113958in}{1.021600in}}{\pgfqpoint{1.111644in}{1.027186in}}{\pgfqpoint{1.107526in}{1.031304in}}%
\pgfpathcurveto{\pgfqpoint{1.103408in}{1.035423in}}{\pgfqpoint{1.097821in}{1.037737in}}{\pgfqpoint{1.091997in}{1.037737in}}%
\pgfpathcurveto{\pgfqpoint{1.086174in}{1.037737in}}{\pgfqpoint{1.080587in}{1.035423in}}{\pgfqpoint{1.076469in}{1.031304in}}%
\pgfpathcurveto{\pgfqpoint{1.072351in}{1.027186in}}{\pgfqpoint{1.070037in}{1.021600in}}{\pgfqpoint{1.070037in}{1.015776in}}%
\pgfpathcurveto{\pgfqpoint{1.070037in}{1.009952in}}{\pgfqpoint{1.072351in}{1.004366in}}{\pgfqpoint{1.076469in}{1.000248in}}%
\pgfpathcurveto{\pgfqpoint{1.080587in}{0.996130in}}{\pgfqpoint{1.086174in}{0.993816in}}{\pgfqpoint{1.091997in}{0.993816in}}%
\pgfpathclose%
\pgfusepath{stroke,fill}%
\end{pgfscope}%
\begin{pgfscope}%
\pgfpathrectangle{\pgfqpoint{0.211875in}{0.211875in}}{\pgfqpoint{1.313625in}{1.279725in}}%
\pgfusepath{clip}%
\pgfsetbuttcap%
\pgfsetroundjoin%
\definecolor{currentfill}{rgb}{0.121569,0.466667,0.705882}%
\pgfsetfillcolor{currentfill}%
\pgfsetlinewidth{1.003750pt}%
\definecolor{currentstroke}{rgb}{0.121569,0.466667,0.705882}%
\pgfsetstrokecolor{currentstroke}%
\pgfsetdash{}{0pt}%
\pgfpathmoveto{\pgfqpoint{1.355606in}{0.825141in}}%
\pgfpathcurveto{\pgfqpoint{1.361430in}{0.825141in}}{\pgfqpoint{1.367016in}{0.827455in}}{\pgfqpoint{1.371135in}{0.831573in}}%
\pgfpathcurveto{\pgfqpoint{1.375253in}{0.835691in}}{\pgfqpoint{1.377567in}{0.841277in}}{\pgfqpoint{1.377567in}{0.847101in}}%
\pgfpathcurveto{\pgfqpoint{1.377567in}{0.852925in}}{\pgfqpoint{1.375253in}{0.858511in}}{\pgfqpoint{1.371135in}{0.862629in}}%
\pgfpathcurveto{\pgfqpoint{1.367016in}{0.866748in}}{\pgfqpoint{1.361430in}{0.869061in}}{\pgfqpoint{1.355606in}{0.869061in}}%
\pgfpathcurveto{\pgfqpoint{1.349782in}{0.869061in}}{\pgfqpoint{1.344196in}{0.866748in}}{\pgfqpoint{1.340078in}{0.862629in}}%
\pgfpathcurveto{\pgfqpoint{1.335960in}{0.858511in}}{\pgfqpoint{1.333646in}{0.852925in}}{\pgfqpoint{1.333646in}{0.847101in}}%
\pgfpathcurveto{\pgfqpoint{1.333646in}{0.841277in}}{\pgfqpoint{1.335960in}{0.835691in}}{\pgfqpoint{1.340078in}{0.831573in}}%
\pgfpathcurveto{\pgfqpoint{1.344196in}{0.827455in}}{\pgfqpoint{1.349782in}{0.825141in}}{\pgfqpoint{1.355606in}{0.825141in}}%
\pgfpathclose%
\pgfusepath{stroke,fill}%
\end{pgfscope}%
\begin{pgfscope}%
\pgfpathrectangle{\pgfqpoint{0.211875in}{0.211875in}}{\pgfqpoint{1.313625in}{1.279725in}}%
\pgfusepath{clip}%
\pgfsetbuttcap%
\pgfsetroundjoin%
\definecolor{currentfill}{rgb}{0.121569,0.466667,0.705882}%
\pgfsetfillcolor{currentfill}%
\pgfsetlinewidth{1.003750pt}%
\definecolor{currentstroke}{rgb}{0.121569,0.466667,0.705882}%
\pgfsetstrokecolor{currentstroke}%
\pgfsetdash{}{0pt}%
\pgfpathmoveto{\pgfqpoint{1.139159in}{0.978277in}}%
\pgfpathcurveto{\pgfqpoint{1.144983in}{0.978277in}}{\pgfqpoint{1.150569in}{0.980591in}}{\pgfqpoint{1.154687in}{0.984709in}}%
\pgfpathcurveto{\pgfqpoint{1.158805in}{0.988827in}}{\pgfqpoint{1.161119in}{0.994413in}}{\pgfqpoint{1.161119in}{1.000237in}}%
\pgfpathcurveto{\pgfqpoint{1.161119in}{1.006061in}}{\pgfqpoint{1.158805in}{1.011647in}}{\pgfqpoint{1.154687in}{1.015765in}}%
\pgfpathcurveto{\pgfqpoint{1.150569in}{1.019884in}}{\pgfqpoint{1.144983in}{1.022197in}}{\pgfqpoint{1.139159in}{1.022197in}}%
\pgfpathcurveto{\pgfqpoint{1.133335in}{1.022197in}}{\pgfqpoint{1.127749in}{1.019884in}}{\pgfqpoint{1.123630in}{1.015765in}}%
\pgfpathcurveto{\pgfqpoint{1.119512in}{1.011647in}}{\pgfqpoint{1.117198in}{1.006061in}}{\pgfqpoint{1.117198in}{1.000237in}}%
\pgfpathcurveto{\pgfqpoint{1.117198in}{0.994413in}}{\pgfqpoint{1.119512in}{0.988827in}}{\pgfqpoint{1.123630in}{0.984709in}}%
\pgfpathcurveto{\pgfqpoint{1.127749in}{0.980591in}}{\pgfqpoint{1.133335in}{0.978277in}}{\pgfqpoint{1.139159in}{0.978277in}}%
\pgfpathclose%
\pgfusepath{stroke,fill}%
\end{pgfscope}%
\begin{pgfscope}%
\pgfpathrectangle{\pgfqpoint{0.211875in}{0.211875in}}{\pgfqpoint{1.313625in}{1.279725in}}%
\pgfusepath{clip}%
\pgfsetbuttcap%
\pgfsetroundjoin%
\definecolor{currentfill}{rgb}{0.121569,0.466667,0.705882}%
\pgfsetfillcolor{currentfill}%
\pgfsetlinewidth{1.003750pt}%
\definecolor{currentstroke}{rgb}{0.121569,0.466667,0.705882}%
\pgfsetstrokecolor{currentstroke}%
\pgfsetdash{}{0pt}%
\pgfpathmoveto{\pgfqpoint{1.139247in}{0.978379in}}%
\pgfpathcurveto{\pgfqpoint{1.145071in}{0.978379in}}{\pgfqpoint{1.150658in}{0.980693in}}{\pgfqpoint{1.154776in}{0.984811in}}%
\pgfpathcurveto{\pgfqpoint{1.158894in}{0.988929in}}{\pgfqpoint{1.161208in}{0.994515in}}{\pgfqpoint{1.161208in}{1.000339in}}%
\pgfpathcurveto{\pgfqpoint{1.161208in}{1.006163in}}{\pgfqpoint{1.158894in}{1.011749in}}{\pgfqpoint{1.154776in}{1.015868in}}%
\pgfpathcurveto{\pgfqpoint{1.150658in}{1.019986in}}{\pgfqpoint{1.145071in}{1.022300in}}{\pgfqpoint{1.139247in}{1.022300in}}%
\pgfpathcurveto{\pgfqpoint{1.133423in}{1.022300in}}{\pgfqpoint{1.127837in}{1.019986in}}{\pgfqpoint{1.123719in}{1.015868in}}%
\pgfpathcurveto{\pgfqpoint{1.119601in}{1.011749in}}{\pgfqpoint{1.117287in}{1.006163in}}{\pgfqpoint{1.117287in}{1.000339in}}%
\pgfpathcurveto{\pgfqpoint{1.117287in}{0.994515in}}{\pgfqpoint{1.119601in}{0.988929in}}{\pgfqpoint{1.123719in}{0.984811in}}%
\pgfpathcurveto{\pgfqpoint{1.127837in}{0.980693in}}{\pgfqpoint{1.133423in}{0.978379in}}{\pgfqpoint{1.139247in}{0.978379in}}%
\pgfpathclose%
\pgfusepath{stroke,fill}%
\end{pgfscope}%
\begin{pgfscope}%
\pgfpathrectangle{\pgfqpoint{0.211875in}{0.211875in}}{\pgfqpoint{1.313625in}{1.279725in}}%
\pgfusepath{clip}%
\pgfsetbuttcap%
\pgfsetroundjoin%
\definecolor{currentfill}{rgb}{0.121569,0.466667,0.705882}%
\pgfsetfillcolor{currentfill}%
\pgfsetlinewidth{1.003750pt}%
\definecolor{currentstroke}{rgb}{0.121569,0.466667,0.705882}%
\pgfsetstrokecolor{currentstroke}%
\pgfsetdash{}{0pt}%
\pgfpathmoveto{\pgfqpoint{1.139296in}{0.978447in}}%
\pgfpathcurveto{\pgfqpoint{1.145120in}{0.978447in}}{\pgfqpoint{1.150706in}{0.980761in}}{\pgfqpoint{1.154824in}{0.984880in}}%
\pgfpathcurveto{\pgfqpoint{1.158942in}{0.988998in}}{\pgfqpoint{1.161256in}{0.994584in}}{\pgfqpoint{1.161256in}{1.000408in}}%
\pgfpathcurveto{\pgfqpoint{1.161256in}{1.006232in}}{\pgfqpoint{1.158942in}{1.011818in}}{\pgfqpoint{1.154824in}{1.015936in}}%
\pgfpathcurveto{\pgfqpoint{1.150706in}{1.020054in}}{\pgfqpoint{1.145120in}{1.022368in}}{\pgfqpoint{1.139296in}{1.022368in}}%
\pgfpathcurveto{\pgfqpoint{1.133472in}{1.022368in}}{\pgfqpoint{1.127886in}{1.020054in}}{\pgfqpoint{1.123768in}{1.015936in}}%
\pgfpathcurveto{\pgfqpoint{1.119649in}{1.011818in}}{\pgfqpoint{1.117336in}{1.006232in}}{\pgfqpoint{1.117336in}{1.000408in}}%
\pgfpathcurveto{\pgfqpoint{1.117336in}{0.994584in}}{\pgfqpoint{1.119649in}{0.988998in}}{\pgfqpoint{1.123768in}{0.984880in}}%
\pgfpathcurveto{\pgfqpoint{1.127886in}{0.980761in}}{\pgfqpoint{1.133472in}{0.978447in}}{\pgfqpoint{1.139296in}{0.978447in}}%
\pgfpathclose%
\pgfusepath{stroke,fill}%
\end{pgfscope}%
\begin{pgfscope}%
\pgfpathrectangle{\pgfqpoint{0.211875in}{0.211875in}}{\pgfqpoint{1.313625in}{1.279725in}}%
\pgfusepath{clip}%
\pgfsetbuttcap%
\pgfsetroundjoin%
\definecolor{currentfill}{rgb}{0.121569,0.466667,0.705882}%
\pgfsetfillcolor{currentfill}%
\pgfsetlinewidth{1.003750pt}%
\definecolor{currentstroke}{rgb}{0.121569,0.466667,0.705882}%
\pgfsetstrokecolor{currentstroke}%
\pgfsetdash{}{0pt}%
\pgfpathmoveto{\pgfqpoint{1.139313in}{0.978492in}}%
\pgfpathcurveto{\pgfqpoint{1.145137in}{0.978492in}}{\pgfqpoint{1.150723in}{0.980806in}}{\pgfqpoint{1.154841in}{0.984924in}}%
\pgfpathcurveto{\pgfqpoint{1.158959in}{0.989042in}}{\pgfqpoint{1.161273in}{0.994628in}}{\pgfqpoint{1.161273in}{1.000452in}}%
\pgfpathcurveto{\pgfqpoint{1.161273in}{1.006276in}}{\pgfqpoint{1.158959in}{1.011862in}}{\pgfqpoint{1.154841in}{1.015980in}}%
\pgfpathcurveto{\pgfqpoint{1.150723in}{1.020099in}}{\pgfqpoint{1.145137in}{1.022412in}}{\pgfqpoint{1.139313in}{1.022412in}}%
\pgfpathcurveto{\pgfqpoint{1.133489in}{1.022412in}}{\pgfqpoint{1.127903in}{1.020099in}}{\pgfqpoint{1.123785in}{1.015980in}}%
\pgfpathcurveto{\pgfqpoint{1.119666in}{1.011862in}}{\pgfqpoint{1.117353in}{1.006276in}}{\pgfqpoint{1.117353in}{1.000452in}}%
\pgfpathcurveto{\pgfqpoint{1.117353in}{0.994628in}}{\pgfqpoint{1.119666in}{0.989042in}}{\pgfqpoint{1.123785in}{0.984924in}}%
\pgfpathcurveto{\pgfqpoint{1.127903in}{0.980806in}}{\pgfqpoint{1.133489in}{0.978492in}}{\pgfqpoint{1.139313in}{0.978492in}}%
\pgfpathclose%
\pgfusepath{stroke,fill}%
\end{pgfscope}%
\begin{pgfscope}%
\pgfpathrectangle{\pgfqpoint{0.211875in}{0.211875in}}{\pgfqpoint{1.313625in}{1.279725in}}%
\pgfusepath{clip}%
\pgfsetbuttcap%
\pgfsetroundjoin%
\definecolor{currentfill}{rgb}{0.121569,0.466667,0.705882}%
\pgfsetfillcolor{currentfill}%
\pgfsetlinewidth{1.003750pt}%
\definecolor{currentstroke}{rgb}{0.121569,0.466667,0.705882}%
\pgfsetstrokecolor{currentstroke}%
\pgfsetdash{}{0pt}%
\pgfpathmoveto{\pgfqpoint{1.139303in}{0.978514in}}%
\pgfpathcurveto{\pgfqpoint{1.145126in}{0.978514in}}{\pgfqpoint{1.150713in}{0.980828in}}{\pgfqpoint{1.154831in}{0.984946in}}%
\pgfpathcurveto{\pgfqpoint{1.158949in}{0.989064in}}{\pgfqpoint{1.161263in}{0.994651in}}{\pgfqpoint{1.161263in}{1.000475in}}%
\pgfpathcurveto{\pgfqpoint{1.161263in}{1.006298in}}{\pgfqpoint{1.158949in}{1.011885in}}{\pgfqpoint{1.154831in}{1.016003in}}%
\pgfpathcurveto{\pgfqpoint{1.150713in}{1.020121in}}{\pgfqpoint{1.145126in}{1.022435in}}{\pgfqpoint{1.139303in}{1.022435in}}%
\pgfpathcurveto{\pgfqpoint{1.133479in}{1.022435in}}{\pgfqpoint{1.127892in}{1.020121in}}{\pgfqpoint{1.123774in}{1.016003in}}%
\pgfpathcurveto{\pgfqpoint{1.119656in}{1.011885in}}{\pgfqpoint{1.117342in}{1.006298in}}{\pgfqpoint{1.117342in}{1.000475in}}%
\pgfpathcurveto{\pgfqpoint{1.117342in}{0.994651in}}{\pgfqpoint{1.119656in}{0.989064in}}{\pgfqpoint{1.123774in}{0.984946in}}%
\pgfpathcurveto{\pgfqpoint{1.127892in}{0.980828in}}{\pgfqpoint{1.133479in}{0.978514in}}{\pgfqpoint{1.139303in}{0.978514in}}%
\pgfpathclose%
\pgfusepath{stroke,fill}%
\end{pgfscope}%
\begin{pgfscope}%
\pgfpathrectangle{\pgfqpoint{0.211875in}{0.211875in}}{\pgfqpoint{1.313625in}{1.279725in}}%
\pgfusepath{clip}%
\pgfsetbuttcap%
\pgfsetroundjoin%
\definecolor{currentfill}{rgb}{0.121569,0.466667,0.705882}%
\pgfsetfillcolor{currentfill}%
\pgfsetlinewidth{1.003750pt}%
\definecolor{currentstroke}{rgb}{0.121569,0.466667,0.705882}%
\pgfsetstrokecolor{currentstroke}%
\pgfsetdash{}{0pt}%
\pgfpathmoveto{\pgfqpoint{1.139268in}{0.978524in}}%
\pgfpathcurveto{\pgfqpoint{1.145092in}{0.978524in}}{\pgfqpoint{1.150678in}{0.980838in}}{\pgfqpoint{1.154796in}{0.984956in}}%
\pgfpathcurveto{\pgfqpoint{1.158914in}{0.989074in}}{\pgfqpoint{1.161228in}{0.994661in}}{\pgfqpoint{1.161228in}{1.000484in}}%
\pgfpathcurveto{\pgfqpoint{1.161228in}{1.006308in}}{\pgfqpoint{1.158914in}{1.011895in}}{\pgfqpoint{1.154796in}{1.016013in}}%
\pgfpathcurveto{\pgfqpoint{1.150678in}{1.020131in}}{\pgfqpoint{1.145092in}{1.022445in}}{\pgfqpoint{1.139268in}{1.022445in}}%
\pgfpathcurveto{\pgfqpoint{1.133444in}{1.022445in}}{\pgfqpoint{1.127858in}{1.020131in}}{\pgfqpoint{1.123739in}{1.016013in}}%
\pgfpathcurveto{\pgfqpoint{1.119621in}{1.011895in}}{\pgfqpoint{1.117307in}{1.006308in}}{\pgfqpoint{1.117307in}{1.000484in}}%
\pgfpathcurveto{\pgfqpoint{1.117307in}{0.994661in}}{\pgfqpoint{1.119621in}{0.989074in}}{\pgfqpoint{1.123739in}{0.984956in}}%
\pgfpathcurveto{\pgfqpoint{1.127858in}{0.980838in}}{\pgfqpoint{1.133444in}{0.978524in}}{\pgfqpoint{1.139268in}{0.978524in}}%
\pgfpathclose%
\pgfusepath{stroke,fill}%
\end{pgfscope}%
\begin{pgfscope}%
\pgfpathrectangle{\pgfqpoint{0.211875in}{0.211875in}}{\pgfqpoint{1.313625in}{1.279725in}}%
\pgfusepath{clip}%
\pgfsetbuttcap%
\pgfsetroundjoin%
\definecolor{currentfill}{rgb}{0.121569,0.466667,0.705882}%
\pgfsetfillcolor{currentfill}%
\pgfsetlinewidth{1.003750pt}%
\definecolor{currentstroke}{rgb}{0.121569,0.466667,0.705882}%
\pgfsetstrokecolor{currentstroke}%
\pgfsetdash{}{0pt}%
\pgfpathmoveto{\pgfqpoint{1.139216in}{0.978525in}}%
\pgfpathcurveto{\pgfqpoint{1.145040in}{0.978525in}}{\pgfqpoint{1.150626in}{0.980839in}}{\pgfqpoint{1.154744in}{0.984957in}}%
\pgfpathcurveto{\pgfqpoint{1.158862in}{0.989076in}}{\pgfqpoint{1.161176in}{0.994662in}}{\pgfqpoint{1.161176in}{1.000486in}}%
\pgfpathcurveto{\pgfqpoint{1.161176in}{1.006310in}}{\pgfqpoint{1.158862in}{1.011896in}}{\pgfqpoint{1.154744in}{1.016014in}}%
\pgfpathcurveto{\pgfqpoint{1.150626in}{1.020132in}}{\pgfqpoint{1.145040in}{1.022446in}}{\pgfqpoint{1.139216in}{1.022446in}}%
\pgfpathcurveto{\pgfqpoint{1.133392in}{1.022446in}}{\pgfqpoint{1.127806in}{1.020132in}}{\pgfqpoint{1.123688in}{1.016014in}}%
\pgfpathcurveto{\pgfqpoint{1.119570in}{1.011896in}}{\pgfqpoint{1.117256in}{1.006310in}}{\pgfqpoint{1.117256in}{1.000486in}}%
\pgfpathcurveto{\pgfqpoint{1.117256in}{0.994662in}}{\pgfqpoint{1.119570in}{0.989076in}}{\pgfqpoint{1.123688in}{0.984957in}}%
\pgfpathcurveto{\pgfqpoint{1.127806in}{0.980839in}}{\pgfqpoint{1.133392in}{0.978525in}}{\pgfqpoint{1.139216in}{0.978525in}}%
\pgfpathclose%
\pgfusepath{stroke,fill}%
\end{pgfscope}%
\begin{pgfscope}%
\pgfpathrectangle{\pgfqpoint{0.211875in}{0.211875in}}{\pgfqpoint{1.313625in}{1.279725in}}%
\pgfusepath{clip}%
\pgfsetbuttcap%
\pgfsetroundjoin%
\definecolor{currentfill}{rgb}{0.121569,0.466667,0.705882}%
\pgfsetfillcolor{currentfill}%
\pgfsetlinewidth{1.003750pt}%
\definecolor{currentstroke}{rgb}{0.121569,0.466667,0.705882}%
\pgfsetstrokecolor{currentstroke}%
\pgfsetdash{}{0pt}%
\pgfpathmoveto{\pgfqpoint{1.139145in}{0.978521in}}%
\pgfpathcurveto{\pgfqpoint{1.144969in}{0.978521in}}{\pgfqpoint{1.150555in}{0.980835in}}{\pgfqpoint{1.154673in}{0.984953in}}%
\pgfpathcurveto{\pgfqpoint{1.158792in}{0.989071in}}{\pgfqpoint{1.161105in}{0.994657in}}{\pgfqpoint{1.161105in}{1.000481in}}%
\pgfpathcurveto{\pgfqpoint{1.161105in}{1.006305in}}{\pgfqpoint{1.158792in}{1.011891in}}{\pgfqpoint{1.154673in}{1.016009in}}%
\pgfpathcurveto{\pgfqpoint{1.150555in}{1.020128in}}{\pgfqpoint{1.144969in}{1.022442in}}{\pgfqpoint{1.139145in}{1.022442in}}%
\pgfpathcurveto{\pgfqpoint{1.133321in}{1.022442in}}{\pgfqpoint{1.127735in}{1.020128in}}{\pgfqpoint{1.123617in}{1.016009in}}%
\pgfpathcurveto{\pgfqpoint{1.119499in}{1.011891in}}{\pgfqpoint{1.117185in}{1.006305in}}{\pgfqpoint{1.117185in}{1.000481in}}%
\pgfpathcurveto{\pgfqpoint{1.117185in}{0.994657in}}{\pgfqpoint{1.119499in}{0.989071in}}{\pgfqpoint{1.123617in}{0.984953in}}%
\pgfpathcurveto{\pgfqpoint{1.127735in}{0.980835in}}{\pgfqpoint{1.133321in}{0.978521in}}{\pgfqpoint{1.139145in}{0.978521in}}%
\pgfpathclose%
\pgfusepath{stroke,fill}%
\end{pgfscope}%
\begin{pgfscope}%
\pgfpathrectangle{\pgfqpoint{0.211875in}{0.211875in}}{\pgfqpoint{1.313625in}{1.279725in}}%
\pgfusepath{clip}%
\pgfsetbuttcap%
\pgfsetroundjoin%
\definecolor{currentfill}{rgb}{0.121569,0.466667,0.705882}%
\pgfsetfillcolor{currentfill}%
\pgfsetlinewidth{1.003750pt}%
\definecolor{currentstroke}{rgb}{0.121569,0.466667,0.705882}%
\pgfsetstrokecolor{currentstroke}%
\pgfsetdash{}{0pt}%
\pgfpathmoveto{\pgfqpoint{1.139064in}{0.978516in}}%
\pgfpathcurveto{\pgfqpoint{1.144888in}{0.978516in}}{\pgfqpoint{1.150474in}{0.980830in}}{\pgfqpoint{1.154592in}{0.984948in}}%
\pgfpathcurveto{\pgfqpoint{1.158710in}{0.989066in}}{\pgfqpoint{1.161024in}{0.994652in}}{\pgfqpoint{1.161024in}{1.000476in}}%
\pgfpathcurveto{\pgfqpoint{1.161024in}{1.006300in}}{\pgfqpoint{1.158710in}{1.011886in}}{\pgfqpoint{1.154592in}{1.016004in}}%
\pgfpathcurveto{\pgfqpoint{1.150474in}{1.020123in}}{\pgfqpoint{1.144888in}{1.022436in}}{\pgfqpoint{1.139064in}{1.022436in}}%
\pgfpathcurveto{\pgfqpoint{1.133240in}{1.022436in}}{\pgfqpoint{1.127654in}{1.020123in}}{\pgfqpoint{1.123536in}{1.016004in}}%
\pgfpathcurveto{\pgfqpoint{1.119417in}{1.011886in}}{\pgfqpoint{1.117104in}{1.006300in}}{\pgfqpoint{1.117104in}{1.000476in}}%
\pgfpathcurveto{\pgfqpoint{1.117104in}{0.994652in}}{\pgfqpoint{1.119417in}{0.989066in}}{\pgfqpoint{1.123536in}{0.984948in}}%
\pgfpathcurveto{\pgfqpoint{1.127654in}{0.980830in}}{\pgfqpoint{1.133240in}{0.978516in}}{\pgfqpoint{1.139064in}{0.978516in}}%
\pgfpathclose%
\pgfusepath{stroke,fill}%
\end{pgfscope}%
\begin{pgfscope}%
\pgfpathrectangle{\pgfqpoint{0.211875in}{0.211875in}}{\pgfqpoint{1.313625in}{1.279725in}}%
\pgfusepath{clip}%
\pgfsetbuttcap%
\pgfsetroundjoin%
\definecolor{currentfill}{rgb}{0.121569,0.466667,0.705882}%
\pgfsetfillcolor{currentfill}%
\pgfsetlinewidth{1.003750pt}%
\definecolor{currentstroke}{rgb}{0.121569,0.466667,0.705882}%
\pgfsetstrokecolor{currentstroke}%
\pgfsetdash{}{0pt}%
\pgfpathmoveto{\pgfqpoint{1.138968in}{0.978509in}}%
\pgfpathcurveto{\pgfqpoint{1.144792in}{0.978509in}}{\pgfqpoint{1.150378in}{0.980823in}}{\pgfqpoint{1.154496in}{0.984941in}}%
\pgfpathcurveto{\pgfqpoint{1.158614in}{0.989059in}}{\pgfqpoint{1.160928in}{0.994645in}}{\pgfqpoint{1.160928in}{1.000469in}}%
\pgfpathcurveto{\pgfqpoint{1.160928in}{1.006293in}}{\pgfqpoint{1.158614in}{1.011879in}}{\pgfqpoint{1.154496in}{1.015997in}}%
\pgfpathcurveto{\pgfqpoint{1.150378in}{1.020115in}}{\pgfqpoint{1.144792in}{1.022429in}}{\pgfqpoint{1.138968in}{1.022429in}}%
\pgfpathcurveto{\pgfqpoint{1.133144in}{1.022429in}}{\pgfqpoint{1.127558in}{1.020115in}}{\pgfqpoint{1.123440in}{1.015997in}}%
\pgfpathcurveto{\pgfqpoint{1.119322in}{1.011879in}}{\pgfqpoint{1.117008in}{1.006293in}}{\pgfqpoint{1.117008in}{1.000469in}}%
\pgfpathcurveto{\pgfqpoint{1.117008in}{0.994645in}}{\pgfqpoint{1.119322in}{0.989059in}}{\pgfqpoint{1.123440in}{0.984941in}}%
\pgfpathcurveto{\pgfqpoint{1.127558in}{0.980823in}}{\pgfqpoint{1.133144in}{0.978509in}}{\pgfqpoint{1.138968in}{0.978509in}}%
\pgfpathclose%
\pgfusepath{stroke,fill}%
\end{pgfscope}%
\begin{pgfscope}%
\pgfpathrectangle{\pgfqpoint{0.211875in}{0.211875in}}{\pgfqpoint{1.313625in}{1.279725in}}%
\pgfusepath{clip}%
\pgfsetbuttcap%
\pgfsetroundjoin%
\definecolor{currentfill}{rgb}{0.121569,0.466667,0.705882}%
\pgfsetfillcolor{currentfill}%
\pgfsetlinewidth{1.003750pt}%
\definecolor{currentstroke}{rgb}{0.121569,0.466667,0.705882}%
\pgfsetstrokecolor{currentstroke}%
\pgfsetdash{}{0pt}%
\pgfpathmoveto{\pgfqpoint{1.138862in}{0.978506in}}%
\pgfpathcurveto{\pgfqpoint{1.144686in}{0.978506in}}{\pgfqpoint{1.150272in}{0.980820in}}{\pgfqpoint{1.154390in}{0.984938in}}%
\pgfpathcurveto{\pgfqpoint{1.158508in}{0.989057in}}{\pgfqpoint{1.160822in}{0.994643in}}{\pgfqpoint{1.160822in}{1.000467in}}%
\pgfpathcurveto{\pgfqpoint{1.160822in}{1.006291in}}{\pgfqpoint{1.158508in}{1.011877in}}{\pgfqpoint{1.154390in}{1.015995in}}%
\pgfpathcurveto{\pgfqpoint{1.150272in}{1.020113in}}{\pgfqpoint{1.144686in}{1.022427in}}{\pgfqpoint{1.138862in}{1.022427in}}%
\pgfpathcurveto{\pgfqpoint{1.133038in}{1.022427in}}{\pgfqpoint{1.127452in}{1.020113in}}{\pgfqpoint{1.123334in}{1.015995in}}%
\pgfpathcurveto{\pgfqpoint{1.119216in}{1.011877in}}{\pgfqpoint{1.116902in}{1.006291in}}{\pgfqpoint{1.116902in}{1.000467in}}%
\pgfpathcurveto{\pgfqpoint{1.116902in}{0.994643in}}{\pgfqpoint{1.119216in}{0.989057in}}{\pgfqpoint{1.123334in}{0.984938in}}%
\pgfpathcurveto{\pgfqpoint{1.127452in}{0.980820in}}{\pgfqpoint{1.133038in}{0.978506in}}{\pgfqpoint{1.138862in}{0.978506in}}%
\pgfpathclose%
\pgfusepath{stroke,fill}%
\end{pgfscope}%
\begin{pgfscope}%
\pgfpathrectangle{\pgfqpoint{0.211875in}{0.211875in}}{\pgfqpoint{1.313625in}{1.279725in}}%
\pgfusepath{clip}%
\pgfsetbuttcap%
\pgfsetroundjoin%
\definecolor{currentfill}{rgb}{0.121569,0.466667,0.705882}%
\pgfsetfillcolor{currentfill}%
\pgfsetlinewidth{1.003750pt}%
\definecolor{currentstroke}{rgb}{0.121569,0.466667,0.705882}%
\pgfsetstrokecolor{currentstroke}%
\pgfsetdash{}{0pt}%
\pgfpathmoveto{\pgfqpoint{1.048126in}{0.910516in}}%
\pgfpathcurveto{\pgfqpoint{1.053950in}{0.910516in}}{\pgfqpoint{1.059536in}{0.912830in}}{\pgfqpoint{1.063654in}{0.916948in}}%
\pgfpathcurveto{\pgfqpoint{1.067772in}{0.921066in}}{\pgfqpoint{1.070086in}{0.926652in}}{\pgfqpoint{1.070086in}{0.932476in}}%
\pgfpathcurveto{\pgfqpoint{1.070086in}{0.938300in}}{\pgfqpoint{1.067772in}{0.943886in}}{\pgfqpoint{1.063654in}{0.948004in}}%
\pgfpathcurveto{\pgfqpoint{1.059536in}{0.952122in}}{\pgfqpoint{1.053950in}{0.954436in}}{\pgfqpoint{1.048126in}{0.954436in}}%
\pgfpathcurveto{\pgfqpoint{1.042302in}{0.954436in}}{\pgfqpoint{1.036716in}{0.952122in}}{\pgfqpoint{1.032598in}{0.948004in}}%
\pgfpathcurveto{\pgfqpoint{1.028480in}{0.943886in}}{\pgfqpoint{1.026166in}{0.938300in}}{\pgfqpoint{1.026166in}{0.932476in}}%
\pgfpathcurveto{\pgfqpoint{1.026166in}{0.926652in}}{\pgfqpoint{1.028480in}{0.921066in}}{\pgfqpoint{1.032598in}{0.916948in}}%
\pgfpathcurveto{\pgfqpoint{1.036716in}{0.912830in}}{\pgfqpoint{1.042302in}{0.910516in}}{\pgfqpoint{1.048126in}{0.910516in}}%
\pgfpathclose%
\pgfusepath{stroke,fill}%
\end{pgfscope}%
\begin{pgfscope}%
\pgfpathrectangle{\pgfqpoint{0.211875in}{0.211875in}}{\pgfqpoint{1.313625in}{1.279725in}}%
\pgfusepath{clip}%
\pgfsetbuttcap%
\pgfsetroundjoin%
\definecolor{currentfill}{rgb}{0.121569,0.466667,0.705882}%
\pgfsetfillcolor{currentfill}%
\pgfsetlinewidth{1.003750pt}%
\definecolor{currentstroke}{rgb}{0.121569,0.466667,0.705882}%
\pgfsetstrokecolor{currentstroke}%
\pgfsetdash{}{0pt}%
\pgfpathmoveto{\pgfqpoint{1.112822in}{0.966974in}}%
\pgfpathcurveto{\pgfqpoint{1.118646in}{0.966974in}}{\pgfqpoint{1.124232in}{0.969288in}}{\pgfqpoint{1.128350in}{0.973406in}}%
\pgfpathcurveto{\pgfqpoint{1.132468in}{0.977524in}}{\pgfqpoint{1.134782in}{0.983110in}}{\pgfqpoint{1.134782in}{0.988934in}}%
\pgfpathcurveto{\pgfqpoint{1.134782in}{0.994758in}}{\pgfqpoint{1.132468in}{1.000344in}}{\pgfqpoint{1.128350in}{1.004462in}}%
\pgfpathcurveto{\pgfqpoint{1.124232in}{1.008580in}}{\pgfqpoint{1.118646in}{1.010894in}}{\pgfqpoint{1.112822in}{1.010894in}}%
\pgfpathcurveto{\pgfqpoint{1.106998in}{1.010894in}}{\pgfqpoint{1.101412in}{1.008580in}}{\pgfqpoint{1.097293in}{1.004462in}}%
\pgfpathcurveto{\pgfqpoint{1.093175in}{1.000344in}}{\pgfqpoint{1.090861in}{0.994758in}}{\pgfqpoint{1.090861in}{0.988934in}}%
\pgfpathcurveto{\pgfqpoint{1.090861in}{0.983110in}}{\pgfqpoint{1.093175in}{0.977524in}}{\pgfqpoint{1.097293in}{0.973406in}}%
\pgfpathcurveto{\pgfqpoint{1.101412in}{0.969288in}}{\pgfqpoint{1.106998in}{0.966974in}}{\pgfqpoint{1.112822in}{0.966974in}}%
\pgfpathclose%
\pgfusepath{stroke,fill}%
\end{pgfscope}%
\begin{pgfscope}%
\pgfpathrectangle{\pgfqpoint{0.211875in}{0.211875in}}{\pgfqpoint{1.313625in}{1.279725in}}%
\pgfusepath{clip}%
\pgfsetbuttcap%
\pgfsetroundjoin%
\definecolor{currentfill}{rgb}{0.121569,0.466667,0.705882}%
\pgfsetfillcolor{currentfill}%
\pgfsetlinewidth{1.003750pt}%
\definecolor{currentstroke}{rgb}{0.121569,0.466667,0.705882}%
\pgfsetstrokecolor{currentstroke}%
\pgfsetdash{}{0pt}%
\pgfpathmoveto{\pgfqpoint{1.040182in}{1.062505in}}%
\pgfpathcurveto{\pgfqpoint{1.046006in}{1.062505in}}{\pgfqpoint{1.051592in}{1.064819in}}{\pgfqpoint{1.055710in}{1.068937in}}%
\pgfpathcurveto{\pgfqpoint{1.059829in}{1.073055in}}{\pgfqpoint{1.062142in}{1.078641in}}{\pgfqpoint{1.062142in}{1.084465in}}%
\pgfpathcurveto{\pgfqpoint{1.062142in}{1.090289in}}{\pgfqpoint{1.059829in}{1.095875in}}{\pgfqpoint{1.055710in}{1.099993in}}%
\pgfpathcurveto{\pgfqpoint{1.051592in}{1.104111in}}{\pgfqpoint{1.046006in}{1.106425in}}{\pgfqpoint{1.040182in}{1.106425in}}%
\pgfpathcurveto{\pgfqpoint{1.034358in}{1.106425in}}{\pgfqpoint{1.028772in}{1.104111in}}{\pgfqpoint{1.024654in}{1.099993in}}%
\pgfpathcurveto{\pgfqpoint{1.020536in}{1.095875in}}{\pgfqpoint{1.018222in}{1.090289in}}{\pgfqpoint{1.018222in}{1.084465in}}%
\pgfpathcurveto{\pgfqpoint{1.018222in}{1.078641in}}{\pgfqpoint{1.020536in}{1.073055in}}{\pgfqpoint{1.024654in}{1.068937in}}%
\pgfpathcurveto{\pgfqpoint{1.028772in}{1.064819in}}{\pgfqpoint{1.034358in}{1.062505in}}{\pgfqpoint{1.040182in}{1.062505in}}%
\pgfpathclose%
\pgfusepath{stroke,fill}%
\end{pgfscope}%
\begin{pgfscope}%
\pgfpathrectangle{\pgfqpoint{0.211875in}{0.211875in}}{\pgfqpoint{1.313625in}{1.279725in}}%
\pgfusepath{clip}%
\pgfsetbuttcap%
\pgfsetroundjoin%
\definecolor{currentfill}{rgb}{0.121569,0.466667,0.705882}%
\pgfsetfillcolor{currentfill}%
\pgfsetlinewidth{1.003750pt}%
\definecolor{currentstroke}{rgb}{0.121569,0.466667,0.705882}%
\pgfsetstrokecolor{currentstroke}%
\pgfsetdash{}{0pt}%
\pgfpathmoveto{\pgfqpoint{1.103055in}{0.969801in}}%
\pgfpathcurveto{\pgfqpoint{1.108879in}{0.969801in}}{\pgfqpoint{1.114465in}{0.972115in}}{\pgfqpoint{1.118583in}{0.976233in}}%
\pgfpathcurveto{\pgfqpoint{1.122702in}{0.980351in}}{\pgfqpoint{1.125015in}{0.985937in}}{\pgfqpoint{1.125015in}{0.991761in}}%
\pgfpathcurveto{\pgfqpoint{1.125015in}{0.997585in}}{\pgfqpoint{1.122702in}{1.003171in}}{\pgfqpoint{1.118583in}{1.007289in}}%
\pgfpathcurveto{\pgfqpoint{1.114465in}{1.011407in}}{\pgfqpoint{1.108879in}{1.013721in}}{\pgfqpoint{1.103055in}{1.013721in}}%
\pgfpathcurveto{\pgfqpoint{1.097231in}{1.013721in}}{\pgfqpoint{1.091645in}{1.011407in}}{\pgfqpoint{1.087527in}{1.007289in}}%
\pgfpathcurveto{\pgfqpoint{1.083409in}{1.003171in}}{\pgfqpoint{1.081095in}{0.997585in}}{\pgfqpoint{1.081095in}{0.991761in}}%
\pgfpathcurveto{\pgfqpoint{1.081095in}{0.985937in}}{\pgfqpoint{1.083409in}{0.980351in}}{\pgfqpoint{1.087527in}{0.976233in}}%
\pgfpathcurveto{\pgfqpoint{1.091645in}{0.972115in}}{\pgfqpoint{1.097231in}{0.969801in}}{\pgfqpoint{1.103055in}{0.969801in}}%
\pgfpathclose%
\pgfusepath{stroke,fill}%
\end{pgfscope}%
\begin{pgfscope}%
\pgfpathrectangle{\pgfqpoint{0.211875in}{0.211875in}}{\pgfqpoint{1.313625in}{1.279725in}}%
\pgfusepath{clip}%
\pgfsetbuttcap%
\pgfsetroundjoin%
\definecolor{currentfill}{rgb}{0.121569,0.466667,0.705882}%
\pgfsetfillcolor{currentfill}%
\pgfsetlinewidth{1.003750pt}%
\definecolor{currentstroke}{rgb}{0.121569,0.466667,0.705882}%
\pgfsetstrokecolor{currentstroke}%
\pgfsetdash{}{0pt}%
\pgfpathmoveto{\pgfqpoint{1.341652in}{0.883967in}}%
\pgfpathcurveto{\pgfqpoint{1.347476in}{0.883967in}}{\pgfqpoint{1.353062in}{0.886281in}}{\pgfqpoint{1.357180in}{0.890399in}}%
\pgfpathcurveto{\pgfqpoint{1.361298in}{0.894517in}}{\pgfqpoint{1.363612in}{0.900103in}}{\pgfqpoint{1.363612in}{0.905927in}}%
\pgfpathcurveto{\pgfqpoint{1.363612in}{0.911751in}}{\pgfqpoint{1.361298in}{0.917337in}}{\pgfqpoint{1.357180in}{0.921455in}}%
\pgfpathcurveto{\pgfqpoint{1.353062in}{0.925573in}}{\pgfqpoint{1.347476in}{0.927887in}}{\pgfqpoint{1.341652in}{0.927887in}}%
\pgfpathcurveto{\pgfqpoint{1.335828in}{0.927887in}}{\pgfqpoint{1.330242in}{0.925573in}}{\pgfqpoint{1.326124in}{0.921455in}}%
\pgfpathcurveto{\pgfqpoint{1.322006in}{0.917337in}}{\pgfqpoint{1.319692in}{0.911751in}}{\pgfqpoint{1.319692in}{0.905927in}}%
\pgfpathcurveto{\pgfqpoint{1.319692in}{0.900103in}}{\pgfqpoint{1.322006in}{0.894517in}}{\pgfqpoint{1.326124in}{0.890399in}}%
\pgfpathcurveto{\pgfqpoint{1.330242in}{0.886281in}}{\pgfqpoint{1.335828in}{0.883967in}}{\pgfqpoint{1.341652in}{0.883967in}}%
\pgfpathclose%
\pgfusepath{stroke,fill}%
\end{pgfscope}%
\begin{pgfscope}%
\pgfpathrectangle{\pgfqpoint{0.211875in}{0.211875in}}{\pgfqpoint{1.313625in}{1.279725in}}%
\pgfusepath{clip}%
\pgfsetbuttcap%
\pgfsetroundjoin%
\definecolor{currentfill}{rgb}{0.121569,0.466667,0.705882}%
\pgfsetfillcolor{currentfill}%
\pgfsetlinewidth{1.003750pt}%
\definecolor{currentstroke}{rgb}{0.121569,0.466667,0.705882}%
\pgfsetstrokecolor{currentstroke}%
\pgfsetdash{}{0pt}%
\pgfpathmoveto{\pgfqpoint{1.095317in}{0.969037in}}%
\pgfpathcurveto{\pgfqpoint{1.101141in}{0.969037in}}{\pgfqpoint{1.106727in}{0.971351in}}{\pgfqpoint{1.110845in}{0.975469in}}%
\pgfpathcurveto{\pgfqpoint{1.114963in}{0.979588in}}{\pgfqpoint{1.117277in}{0.985174in}}{\pgfqpoint{1.117277in}{0.990998in}}%
\pgfpathcurveto{\pgfqpoint{1.117277in}{0.996822in}}{\pgfqpoint{1.114963in}{1.002408in}}{\pgfqpoint{1.110845in}{1.006526in}}%
\pgfpathcurveto{\pgfqpoint{1.106727in}{1.010644in}}{\pgfqpoint{1.101141in}{1.012958in}}{\pgfqpoint{1.095317in}{1.012958in}}%
\pgfpathcurveto{\pgfqpoint{1.089493in}{1.012958in}}{\pgfqpoint{1.083907in}{1.010644in}}{\pgfqpoint{1.079789in}{1.006526in}}%
\pgfpathcurveto{\pgfqpoint{1.075671in}{1.002408in}}{\pgfqpoint{1.073357in}{0.996822in}}{\pgfqpoint{1.073357in}{0.990998in}}%
\pgfpathcurveto{\pgfqpoint{1.073357in}{0.985174in}}{\pgfqpoint{1.075671in}{0.979588in}}{\pgfqpoint{1.079789in}{0.975469in}}%
\pgfpathcurveto{\pgfqpoint{1.083907in}{0.971351in}}{\pgfqpoint{1.089493in}{0.969037in}}{\pgfqpoint{1.095317in}{0.969037in}}%
\pgfpathclose%
\pgfusepath{stroke,fill}%
\end{pgfscope}%
\begin{pgfscope}%
\pgfpathrectangle{\pgfqpoint{0.211875in}{0.211875in}}{\pgfqpoint{1.313625in}{1.279725in}}%
\pgfusepath{clip}%
\pgfsetbuttcap%
\pgfsetroundjoin%
\definecolor{currentfill}{rgb}{0.121569,0.466667,0.705882}%
\pgfsetfillcolor{currentfill}%
\pgfsetlinewidth{1.003750pt}%
\definecolor{currentstroke}{rgb}{0.121569,0.466667,0.705882}%
\pgfsetstrokecolor{currentstroke}%
\pgfsetdash{}{0pt}%
\pgfpathmoveto{\pgfqpoint{1.095800in}{0.968759in}}%
\pgfpathcurveto{\pgfqpoint{1.101624in}{0.968759in}}{\pgfqpoint{1.107210in}{0.971073in}}{\pgfqpoint{1.111328in}{0.975191in}}%
\pgfpathcurveto{\pgfqpoint{1.115447in}{0.979309in}}{\pgfqpoint{1.117760in}{0.984895in}}{\pgfqpoint{1.117760in}{0.990719in}}%
\pgfpathcurveto{\pgfqpoint{1.117760in}{0.996543in}}{\pgfqpoint{1.115447in}{1.002129in}}{\pgfqpoint{1.111328in}{1.006247in}}%
\pgfpathcurveto{\pgfqpoint{1.107210in}{1.010365in}}{\pgfqpoint{1.101624in}{1.012679in}}{\pgfqpoint{1.095800in}{1.012679in}}%
\pgfpathcurveto{\pgfqpoint{1.089976in}{1.012679in}}{\pgfqpoint{1.084390in}{1.010365in}}{\pgfqpoint{1.080272in}{1.006247in}}%
\pgfpathcurveto{\pgfqpoint{1.076154in}{1.002129in}}{\pgfqpoint{1.073840in}{0.996543in}}{\pgfqpoint{1.073840in}{0.990719in}}%
\pgfpathcurveto{\pgfqpoint{1.073840in}{0.984895in}}{\pgfqpoint{1.076154in}{0.979309in}}{\pgfqpoint{1.080272in}{0.975191in}}%
\pgfpathcurveto{\pgfqpoint{1.084390in}{0.971073in}}{\pgfqpoint{1.089976in}{0.968759in}}{\pgfqpoint{1.095800in}{0.968759in}}%
\pgfpathclose%
\pgfusepath{stroke,fill}%
\end{pgfscope}%
\begin{pgfscope}%
\pgfpathrectangle{\pgfqpoint{0.211875in}{0.211875in}}{\pgfqpoint{1.313625in}{1.279725in}}%
\pgfusepath{clip}%
\pgfsetbuttcap%
\pgfsetroundjoin%
\definecolor{currentfill}{rgb}{0.121569,0.466667,0.705882}%
\pgfsetfillcolor{currentfill}%
\pgfsetlinewidth{1.003750pt}%
\definecolor{currentstroke}{rgb}{0.121569,0.466667,0.705882}%
\pgfsetstrokecolor{currentstroke}%
\pgfsetdash{}{0pt}%
\pgfpathmoveto{\pgfqpoint{1.331336in}{0.771124in}}%
\pgfpathcurveto{\pgfqpoint{1.337160in}{0.771124in}}{\pgfqpoint{1.342746in}{0.773438in}}{\pgfqpoint{1.346864in}{0.777556in}}%
\pgfpathcurveto{\pgfqpoint{1.350982in}{0.781675in}}{\pgfqpoint{1.353296in}{0.787261in}}{\pgfqpoint{1.353296in}{0.793085in}}%
\pgfpathcurveto{\pgfqpoint{1.353296in}{0.798909in}}{\pgfqpoint{1.350982in}{0.804495in}}{\pgfqpoint{1.346864in}{0.808613in}}%
\pgfpathcurveto{\pgfqpoint{1.342746in}{0.812731in}}{\pgfqpoint{1.337160in}{0.815045in}}{\pgfqpoint{1.331336in}{0.815045in}}%
\pgfpathcurveto{\pgfqpoint{1.325512in}{0.815045in}}{\pgfqpoint{1.319926in}{0.812731in}}{\pgfqpoint{1.315807in}{0.808613in}}%
\pgfpathcurveto{\pgfqpoint{1.311689in}{0.804495in}}{\pgfqpoint{1.309375in}{0.798909in}}{\pgfqpoint{1.309375in}{0.793085in}}%
\pgfpathcurveto{\pgfqpoint{1.309375in}{0.787261in}}{\pgfqpoint{1.311689in}{0.781675in}}{\pgfqpoint{1.315807in}{0.777556in}}%
\pgfpathcurveto{\pgfqpoint{1.319926in}{0.773438in}}{\pgfqpoint{1.325512in}{0.771124in}}{\pgfqpoint{1.331336in}{0.771124in}}%
\pgfpathclose%
\pgfusepath{stroke,fill}%
\end{pgfscope}%
\begin{pgfscope}%
\pgfpathrectangle{\pgfqpoint{0.211875in}{0.211875in}}{\pgfqpoint{1.313625in}{1.279725in}}%
\pgfusepath{clip}%
\pgfsetbuttcap%
\pgfsetroundjoin%
\definecolor{currentfill}{rgb}{0.121569,0.466667,0.705882}%
\pgfsetfillcolor{currentfill}%
\pgfsetlinewidth{1.003750pt}%
\definecolor{currentstroke}{rgb}{0.121569,0.466667,0.705882}%
\pgfsetstrokecolor{currentstroke}%
\pgfsetdash{}{0pt}%
\pgfpathmoveto{\pgfqpoint{1.082273in}{0.840797in}}%
\pgfpathcurveto{\pgfqpoint{1.088097in}{0.840797in}}{\pgfqpoint{1.093683in}{0.843110in}}{\pgfqpoint{1.097801in}{0.847229in}}%
\pgfpathcurveto{\pgfqpoint{1.101920in}{0.851347in}}{\pgfqpoint{1.104233in}{0.856933in}}{\pgfqpoint{1.104233in}{0.862757in}}%
\pgfpathcurveto{\pgfqpoint{1.104233in}{0.868581in}}{\pgfqpoint{1.101920in}{0.874167in}}{\pgfqpoint{1.097801in}{0.878285in}}%
\pgfpathcurveto{\pgfqpoint{1.093683in}{0.882403in}}{\pgfqpoint{1.088097in}{0.884717in}}{\pgfqpoint{1.082273in}{0.884717in}}%
\pgfpathcurveto{\pgfqpoint{1.076449in}{0.884717in}}{\pgfqpoint{1.070863in}{0.882403in}}{\pgfqpoint{1.066745in}{0.878285in}}%
\pgfpathcurveto{\pgfqpoint{1.062627in}{0.874167in}}{\pgfqpoint{1.060313in}{0.868581in}}{\pgfqpoint{1.060313in}{0.862757in}}%
\pgfpathcurveto{\pgfqpoint{1.060313in}{0.856933in}}{\pgfqpoint{1.062627in}{0.851347in}}{\pgfqpoint{1.066745in}{0.847229in}}%
\pgfpathcurveto{\pgfqpoint{1.070863in}{0.843110in}}{\pgfqpoint{1.076449in}{0.840797in}}{\pgfqpoint{1.082273in}{0.840797in}}%
\pgfpathclose%
\pgfusepath{stroke,fill}%
\end{pgfscope}%
\begin{pgfscope}%
\pgfpathrectangle{\pgfqpoint{0.211875in}{0.211875in}}{\pgfqpoint{1.313625in}{1.279725in}}%
\pgfusepath{clip}%
\pgfsetbuttcap%
\pgfsetroundjoin%
\definecolor{currentfill}{rgb}{0.121569,0.466667,0.705882}%
\pgfsetfillcolor{currentfill}%
\pgfsetlinewidth{1.003750pt}%
\definecolor{currentstroke}{rgb}{0.121569,0.466667,0.705882}%
\pgfsetstrokecolor{currentstroke}%
\pgfsetdash{}{0pt}%
\pgfpathmoveto{\pgfqpoint{1.074460in}{1.026226in}}%
\pgfpathcurveto{\pgfqpoint{1.080284in}{1.026226in}}{\pgfqpoint{1.085870in}{1.028540in}}{\pgfqpoint{1.089988in}{1.032658in}}%
\pgfpathcurveto{\pgfqpoint{1.094107in}{1.036776in}}{\pgfqpoint{1.096420in}{1.042362in}}{\pgfqpoint{1.096420in}{1.048186in}}%
\pgfpathcurveto{\pgfqpoint{1.096420in}{1.054010in}}{\pgfqpoint{1.094107in}{1.059596in}}{\pgfqpoint{1.089988in}{1.063714in}}%
\pgfpathcurveto{\pgfqpoint{1.085870in}{1.067833in}}{\pgfqpoint{1.080284in}{1.070146in}}{\pgfqpoint{1.074460in}{1.070146in}}%
\pgfpathcurveto{\pgfqpoint{1.068636in}{1.070146in}}{\pgfqpoint{1.063050in}{1.067833in}}{\pgfqpoint{1.058932in}{1.063714in}}%
\pgfpathcurveto{\pgfqpoint{1.054814in}{1.059596in}}{\pgfqpoint{1.052500in}{1.054010in}}{\pgfqpoint{1.052500in}{1.048186in}}%
\pgfpathcurveto{\pgfqpoint{1.052500in}{1.042362in}}{\pgfqpoint{1.054814in}{1.036776in}}{\pgfqpoint{1.058932in}{1.032658in}}%
\pgfpathcurveto{\pgfqpoint{1.063050in}{1.028540in}}{\pgfqpoint{1.068636in}{1.026226in}}{\pgfqpoint{1.074460in}{1.026226in}}%
\pgfpathclose%
\pgfusepath{stroke,fill}%
\end{pgfscope}%
\begin{pgfscope}%
\pgfpathrectangle{\pgfqpoint{0.211875in}{0.211875in}}{\pgfqpoint{1.313625in}{1.279725in}}%
\pgfusepath{clip}%
\pgfsetbuttcap%
\pgfsetroundjoin%
\definecolor{currentfill}{rgb}{0.121569,0.466667,0.705882}%
\pgfsetfillcolor{currentfill}%
\pgfsetlinewidth{1.003750pt}%
\definecolor{currentstroke}{rgb}{0.121569,0.466667,0.705882}%
\pgfsetstrokecolor{currentstroke}%
\pgfsetdash{}{0pt}%
\pgfpathmoveto{\pgfqpoint{1.079617in}{1.022072in}}%
\pgfpathcurveto{\pgfqpoint{1.085441in}{1.022072in}}{\pgfqpoint{1.091028in}{1.024386in}}{\pgfqpoint{1.095146in}{1.028504in}}%
\pgfpathcurveto{\pgfqpoint{1.099264in}{1.032623in}}{\pgfqpoint{1.101578in}{1.038209in}}{\pgfqpoint{1.101578in}{1.044033in}}%
\pgfpathcurveto{\pgfqpoint{1.101578in}{1.049857in}}{\pgfqpoint{1.099264in}{1.055443in}}{\pgfqpoint{1.095146in}{1.059561in}}%
\pgfpathcurveto{\pgfqpoint{1.091028in}{1.063679in}}{\pgfqpoint{1.085441in}{1.065993in}}{\pgfqpoint{1.079617in}{1.065993in}}%
\pgfpathcurveto{\pgfqpoint{1.073794in}{1.065993in}}{\pgfqpoint{1.068207in}{1.063679in}}{\pgfqpoint{1.064089in}{1.059561in}}%
\pgfpathcurveto{\pgfqpoint{1.059971in}{1.055443in}}{\pgfqpoint{1.057657in}{1.049857in}}{\pgfqpoint{1.057657in}{1.044033in}}%
\pgfpathcurveto{\pgfqpoint{1.057657in}{1.038209in}}{\pgfqpoint{1.059971in}{1.032623in}}{\pgfqpoint{1.064089in}{1.028504in}}%
\pgfpathcurveto{\pgfqpoint{1.068207in}{1.024386in}}{\pgfqpoint{1.073794in}{1.022072in}}{\pgfqpoint{1.079617in}{1.022072in}}%
\pgfpathclose%
\pgfusepath{stroke,fill}%
\end{pgfscope}%
\begin{pgfscope}%
\pgfpathrectangle{\pgfqpoint{0.211875in}{0.211875in}}{\pgfqpoint{1.313625in}{1.279725in}}%
\pgfusepath{clip}%
\pgfsetbuttcap%
\pgfsetroundjoin%
\definecolor{currentfill}{rgb}{0.121569,0.466667,0.705882}%
\pgfsetfillcolor{currentfill}%
\pgfsetlinewidth{1.003750pt}%
\definecolor{currentstroke}{rgb}{0.121569,0.466667,0.705882}%
\pgfsetstrokecolor{currentstroke}%
\pgfsetdash{}{0pt}%
\pgfpathmoveto{\pgfqpoint{1.085694in}{1.017473in}}%
\pgfpathcurveto{\pgfqpoint{1.091518in}{1.017473in}}{\pgfqpoint{1.097104in}{1.019787in}}{\pgfqpoint{1.101222in}{1.023905in}}%
\pgfpathcurveto{\pgfqpoint{1.105340in}{1.028023in}}{\pgfqpoint{1.107654in}{1.033610in}}{\pgfqpoint{1.107654in}{1.039433in}}%
\pgfpathcurveto{\pgfqpoint{1.107654in}{1.045257in}}{\pgfqpoint{1.105340in}{1.050844in}}{\pgfqpoint{1.101222in}{1.054962in}}%
\pgfpathcurveto{\pgfqpoint{1.097104in}{1.059080in}}{\pgfqpoint{1.091518in}{1.061394in}}{\pgfqpoint{1.085694in}{1.061394in}}%
\pgfpathcurveto{\pgfqpoint{1.079870in}{1.061394in}}{\pgfqpoint{1.074284in}{1.059080in}}{\pgfqpoint{1.070166in}{1.054962in}}%
\pgfpathcurveto{\pgfqpoint{1.066048in}{1.050844in}}{\pgfqpoint{1.063734in}{1.045257in}}{\pgfqpoint{1.063734in}{1.039433in}}%
\pgfpathcurveto{\pgfqpoint{1.063734in}{1.033610in}}{\pgfqpoint{1.066048in}{1.028023in}}{\pgfqpoint{1.070166in}{1.023905in}}%
\pgfpathcurveto{\pgfqpoint{1.074284in}{1.019787in}}{\pgfqpoint{1.079870in}{1.017473in}}{\pgfqpoint{1.085694in}{1.017473in}}%
\pgfpathclose%
\pgfusepath{stroke,fill}%
\end{pgfscope}%
\begin{pgfscope}%
\pgfpathrectangle{\pgfqpoint{0.211875in}{0.211875in}}{\pgfqpoint{1.313625in}{1.279725in}}%
\pgfusepath{clip}%
\pgfsetbuttcap%
\pgfsetroundjoin%
\definecolor{currentfill}{rgb}{0.121569,0.466667,0.705882}%
\pgfsetfillcolor{currentfill}%
\pgfsetlinewidth{1.003750pt}%
\definecolor{currentstroke}{rgb}{0.121569,0.466667,0.705882}%
\pgfsetstrokecolor{currentstroke}%
\pgfsetdash{}{0pt}%
\pgfpathmoveto{\pgfqpoint{1.090256in}{1.013818in}}%
\pgfpathcurveto{\pgfqpoint{1.096079in}{1.013818in}}{\pgfqpoint{1.101666in}{1.016132in}}{\pgfqpoint{1.105784in}{1.020250in}}%
\pgfpathcurveto{\pgfqpoint{1.109902in}{1.024369in}}{\pgfqpoint{1.112216in}{1.029955in}}{\pgfqpoint{1.112216in}{1.035779in}}%
\pgfpathcurveto{\pgfqpoint{1.112216in}{1.041603in}}{\pgfqpoint{1.109902in}{1.047189in}}{\pgfqpoint{1.105784in}{1.051307in}}%
\pgfpathcurveto{\pgfqpoint{1.101666in}{1.055425in}}{\pgfqpoint{1.096079in}{1.057739in}}{\pgfqpoint{1.090256in}{1.057739in}}%
\pgfpathcurveto{\pgfqpoint{1.084432in}{1.057739in}}{\pgfqpoint{1.078845in}{1.055425in}}{\pgfqpoint{1.074727in}{1.051307in}}%
\pgfpathcurveto{\pgfqpoint{1.070609in}{1.047189in}}{\pgfqpoint{1.068295in}{1.041603in}}{\pgfqpoint{1.068295in}{1.035779in}}%
\pgfpathcurveto{\pgfqpoint{1.068295in}{1.029955in}}{\pgfqpoint{1.070609in}{1.024369in}}{\pgfqpoint{1.074727in}{1.020250in}}%
\pgfpathcurveto{\pgfqpoint{1.078845in}{1.016132in}}{\pgfqpoint{1.084432in}{1.013818in}}{\pgfqpoint{1.090256in}{1.013818in}}%
\pgfpathclose%
\pgfusepath{stroke,fill}%
\end{pgfscope}%
\begin{pgfscope}%
\pgfpathrectangle{\pgfqpoint{0.211875in}{0.211875in}}{\pgfqpoint{1.313625in}{1.279725in}}%
\pgfusepath{clip}%
\pgfsetbuttcap%
\pgfsetroundjoin%
\definecolor{currentfill}{rgb}{0.121569,0.466667,0.705882}%
\pgfsetfillcolor{currentfill}%
\pgfsetlinewidth{1.003750pt}%
\definecolor{currentstroke}{rgb}{0.121569,0.466667,0.705882}%
\pgfsetstrokecolor{currentstroke}%
\pgfsetdash{}{0pt}%
\pgfpathmoveto{\pgfqpoint{1.092503in}{1.011561in}}%
\pgfpathcurveto{\pgfqpoint{1.098327in}{1.011561in}}{\pgfqpoint{1.103913in}{1.013874in}}{\pgfqpoint{1.108032in}{1.017993in}}%
\pgfpathcurveto{\pgfqpoint{1.112150in}{1.022111in}}{\pgfqpoint{1.114464in}{1.027697in}}{\pgfqpoint{1.114464in}{1.033521in}}%
\pgfpathcurveto{\pgfqpoint{1.114464in}{1.039345in}}{\pgfqpoint{1.112150in}{1.044931in}}{\pgfqpoint{1.108032in}{1.049049in}}%
\pgfpathcurveto{\pgfqpoint{1.103913in}{1.053167in}}{\pgfqpoint{1.098327in}{1.055481in}}{\pgfqpoint{1.092503in}{1.055481in}}%
\pgfpathcurveto{\pgfqpoint{1.086679in}{1.055481in}}{\pgfqpoint{1.081093in}{1.053167in}}{\pgfqpoint{1.076975in}{1.049049in}}%
\pgfpathcurveto{\pgfqpoint{1.072857in}{1.044931in}}{\pgfqpoint{1.070543in}{1.039345in}}{\pgfqpoint{1.070543in}{1.033521in}}%
\pgfpathcurveto{\pgfqpoint{1.070543in}{1.027697in}}{\pgfqpoint{1.072857in}{1.022111in}}{\pgfqpoint{1.076975in}{1.017993in}}%
\pgfpathcurveto{\pgfqpoint{1.081093in}{1.013874in}}{\pgfqpoint{1.086679in}{1.011561in}}{\pgfqpoint{1.092503in}{1.011561in}}%
\pgfpathclose%
\pgfusepath{stroke,fill}%
\end{pgfscope}%
\begin{pgfscope}%
\pgfpathrectangle{\pgfqpoint{0.211875in}{0.211875in}}{\pgfqpoint{1.313625in}{1.279725in}}%
\pgfusepath{clip}%
\pgfsetbuttcap%
\pgfsetroundjoin%
\definecolor{currentfill}{rgb}{0.121569,0.466667,0.705882}%
\pgfsetfillcolor{currentfill}%
\pgfsetlinewidth{1.003750pt}%
\definecolor{currentstroke}{rgb}{0.121569,0.466667,0.705882}%
\pgfsetstrokecolor{currentstroke}%
\pgfsetdash{}{0pt}%
\pgfpathmoveto{\pgfqpoint{1.093432in}{1.010184in}}%
\pgfpathcurveto{\pgfqpoint{1.099256in}{1.010184in}}{\pgfqpoint{1.104842in}{1.012498in}}{\pgfqpoint{1.108960in}{1.016616in}}%
\pgfpathcurveto{\pgfqpoint{1.113078in}{1.020734in}}{\pgfqpoint{1.115392in}{1.026320in}}{\pgfqpoint{1.115392in}{1.032144in}}%
\pgfpathcurveto{\pgfqpoint{1.115392in}{1.037968in}}{\pgfqpoint{1.113078in}{1.043554in}}{\pgfqpoint{1.108960in}{1.047672in}}%
\pgfpathcurveto{\pgfqpoint{1.104842in}{1.051790in}}{\pgfqpoint{1.099256in}{1.054104in}}{\pgfqpoint{1.093432in}{1.054104in}}%
\pgfpathcurveto{\pgfqpoint{1.087608in}{1.054104in}}{\pgfqpoint{1.082022in}{1.051790in}}{\pgfqpoint{1.077904in}{1.047672in}}%
\pgfpathcurveto{\pgfqpoint{1.073786in}{1.043554in}}{\pgfqpoint{1.071472in}{1.037968in}}{\pgfqpoint{1.071472in}{1.032144in}}%
\pgfpathcurveto{\pgfqpoint{1.071472in}{1.026320in}}{\pgfqpoint{1.073786in}{1.020734in}}{\pgfqpoint{1.077904in}{1.016616in}}%
\pgfpathcurveto{\pgfqpoint{1.082022in}{1.012498in}}{\pgfqpoint{1.087608in}{1.010184in}}{\pgfqpoint{1.093432in}{1.010184in}}%
\pgfpathclose%
\pgfusepath{stroke,fill}%
\end{pgfscope}%
\begin{pgfscope}%
\pgfpathrectangle{\pgfqpoint{0.211875in}{0.211875in}}{\pgfqpoint{1.313625in}{1.279725in}}%
\pgfusepath{clip}%
\pgfsetbuttcap%
\pgfsetroundjoin%
\definecolor{currentfill}{rgb}{0.121569,0.466667,0.705882}%
\pgfsetfillcolor{currentfill}%
\pgfsetlinewidth{1.003750pt}%
\definecolor{currentstroke}{rgb}{0.121569,0.466667,0.705882}%
\pgfsetstrokecolor{currentstroke}%
\pgfsetdash{}{0pt}%
\pgfpathmoveto{\pgfqpoint{1.085929in}{0.654149in}}%
\pgfpathcurveto{\pgfqpoint{1.091752in}{0.654149in}}{\pgfqpoint{1.097339in}{0.656463in}}{\pgfqpoint{1.101457in}{0.660581in}}%
\pgfpathcurveto{\pgfqpoint{1.105575in}{0.664699in}}{\pgfqpoint{1.107889in}{0.670286in}}{\pgfqpoint{1.107889in}{0.676110in}}%
\pgfpathcurveto{\pgfqpoint{1.107889in}{0.681934in}}{\pgfqpoint{1.105575in}{0.687520in}}{\pgfqpoint{1.101457in}{0.691638in}}%
\pgfpathcurveto{\pgfqpoint{1.097339in}{0.695756in}}{\pgfqpoint{1.091752in}{0.698070in}}{\pgfqpoint{1.085929in}{0.698070in}}%
\pgfpathcurveto{\pgfqpoint{1.080105in}{0.698070in}}{\pgfqpoint{1.074518in}{0.695756in}}{\pgfqpoint{1.070400in}{0.691638in}}%
\pgfpathcurveto{\pgfqpoint{1.066282in}{0.687520in}}{\pgfqpoint{1.063968in}{0.681934in}}{\pgfqpoint{1.063968in}{0.676110in}}%
\pgfpathcurveto{\pgfqpoint{1.063968in}{0.670286in}}{\pgfqpoint{1.066282in}{0.664699in}}{\pgfqpoint{1.070400in}{0.660581in}}%
\pgfpathcurveto{\pgfqpoint{1.074518in}{0.656463in}}{\pgfqpoint{1.080105in}{0.654149in}}{\pgfqpoint{1.085929in}{0.654149in}}%
\pgfpathclose%
\pgfusepath{stroke,fill}%
\end{pgfscope}%
\begin{pgfscope}%
\pgfpathrectangle{\pgfqpoint{0.211875in}{0.211875in}}{\pgfqpoint{1.313625in}{1.279725in}}%
\pgfusepath{clip}%
\pgfsetbuttcap%
\pgfsetroundjoin%
\definecolor{currentfill}{rgb}{0.121569,0.466667,0.705882}%
\pgfsetfillcolor{currentfill}%
\pgfsetlinewidth{1.003750pt}%
\definecolor{currentstroke}{rgb}{0.121569,0.466667,0.705882}%
\pgfsetstrokecolor{currentstroke}%
\pgfsetdash{}{0pt}%
\pgfpathmoveto{\pgfqpoint{1.323230in}{0.568216in}}%
\pgfpathcurveto{\pgfqpoint{1.329054in}{0.568216in}}{\pgfqpoint{1.334640in}{0.570530in}}{\pgfqpoint{1.338759in}{0.574648in}}%
\pgfpathcurveto{\pgfqpoint{1.342877in}{0.578766in}}{\pgfqpoint{1.345191in}{0.584352in}}{\pgfqpoint{1.345191in}{0.590176in}}%
\pgfpathcurveto{\pgfqpoint{1.345191in}{0.596000in}}{\pgfqpoint{1.342877in}{0.601586in}}{\pgfqpoint{1.338759in}{0.605704in}}%
\pgfpathcurveto{\pgfqpoint{1.334640in}{0.609822in}}{\pgfqpoint{1.329054in}{0.612136in}}{\pgfqpoint{1.323230in}{0.612136in}}%
\pgfpathcurveto{\pgfqpoint{1.317406in}{0.612136in}}{\pgfqpoint{1.311820in}{0.609822in}}{\pgfqpoint{1.307702in}{0.605704in}}%
\pgfpathcurveto{\pgfqpoint{1.303584in}{0.601586in}}{\pgfqpoint{1.301270in}{0.596000in}}{\pgfqpoint{1.301270in}{0.590176in}}%
\pgfpathcurveto{\pgfqpoint{1.301270in}{0.584352in}}{\pgfqpoint{1.303584in}{0.578766in}}{\pgfqpoint{1.307702in}{0.574648in}}%
\pgfpathcurveto{\pgfqpoint{1.311820in}{0.570530in}}{\pgfqpoint{1.317406in}{0.568216in}}{\pgfqpoint{1.323230in}{0.568216in}}%
\pgfpathclose%
\pgfusepath{stroke,fill}%
\end{pgfscope}%
\begin{pgfscope}%
\pgfpathrectangle{\pgfqpoint{0.211875in}{0.211875in}}{\pgfqpoint{1.313625in}{1.279725in}}%
\pgfusepath{clip}%
\pgfsetbuttcap%
\pgfsetroundjoin%
\definecolor{currentfill}{rgb}{0.121569,0.466667,0.705882}%
\pgfsetfillcolor{currentfill}%
\pgfsetlinewidth{1.003750pt}%
\definecolor{currentstroke}{rgb}{0.121569,0.466667,0.705882}%
\pgfsetstrokecolor{currentstroke}%
\pgfsetdash{}{0pt}%
\pgfpathmoveto{\pgfqpoint{1.096568in}{0.989028in}}%
\pgfpathcurveto{\pgfqpoint{1.102392in}{0.989028in}}{\pgfqpoint{1.107978in}{0.991342in}}{\pgfqpoint{1.112096in}{0.995460in}}%
\pgfpathcurveto{\pgfqpoint{1.116214in}{0.999578in}}{\pgfqpoint{1.118528in}{1.005164in}}{\pgfqpoint{1.118528in}{1.010988in}}%
\pgfpathcurveto{\pgfqpoint{1.118528in}{1.016812in}}{\pgfqpoint{1.116214in}{1.022398in}}{\pgfqpoint{1.112096in}{1.026516in}}%
\pgfpathcurveto{\pgfqpoint{1.107978in}{1.030634in}}{\pgfqpoint{1.102392in}{1.032948in}}{\pgfqpoint{1.096568in}{1.032948in}}%
\pgfpathcurveto{\pgfqpoint{1.090744in}{1.032948in}}{\pgfqpoint{1.085158in}{1.030634in}}{\pgfqpoint{1.081040in}{1.026516in}}%
\pgfpathcurveto{\pgfqpoint{1.076922in}{1.022398in}}{\pgfqpoint{1.074608in}{1.016812in}}{\pgfqpoint{1.074608in}{1.010988in}}%
\pgfpathcurveto{\pgfqpoint{1.074608in}{1.005164in}}{\pgfqpoint{1.076922in}{0.999578in}}{\pgfqpoint{1.081040in}{0.995460in}}%
\pgfpathcurveto{\pgfqpoint{1.085158in}{0.991342in}}{\pgfqpoint{1.090744in}{0.989028in}}{\pgfqpoint{1.096568in}{0.989028in}}%
\pgfpathclose%
\pgfusepath{stroke,fill}%
\end{pgfscope}%
\begin{pgfscope}%
\pgfpathrectangle{\pgfqpoint{0.211875in}{0.211875in}}{\pgfqpoint{1.313625in}{1.279725in}}%
\pgfusepath{clip}%
\pgfsetbuttcap%
\pgfsetroundjoin%
\definecolor{currentfill}{rgb}{0.121569,0.466667,0.705882}%
\pgfsetfillcolor{currentfill}%
\pgfsetlinewidth{1.003750pt}%
\definecolor{currentstroke}{rgb}{0.121569,0.466667,0.705882}%
\pgfsetstrokecolor{currentstroke}%
\pgfsetdash{}{0pt}%
\pgfpathmoveto{\pgfqpoint{1.287587in}{0.507512in}}%
\pgfpathcurveto{\pgfqpoint{1.293411in}{0.507512in}}{\pgfqpoint{1.298997in}{0.509826in}}{\pgfqpoint{1.303115in}{0.513944in}}%
\pgfpathcurveto{\pgfqpoint{1.307233in}{0.518062in}}{\pgfqpoint{1.309547in}{0.523648in}}{\pgfqpoint{1.309547in}{0.529472in}}%
\pgfpathcurveto{\pgfqpoint{1.309547in}{0.535296in}}{\pgfqpoint{1.307233in}{0.540882in}}{\pgfqpoint{1.303115in}{0.545000in}}%
\pgfpathcurveto{\pgfqpoint{1.298997in}{0.549119in}}{\pgfqpoint{1.293411in}{0.551432in}}{\pgfqpoint{1.287587in}{0.551432in}}%
\pgfpathcurveto{\pgfqpoint{1.281763in}{0.551432in}}{\pgfqpoint{1.276177in}{0.549119in}}{\pgfqpoint{1.272059in}{0.545000in}}%
\pgfpathcurveto{\pgfqpoint{1.267940in}{0.540882in}}{\pgfqpoint{1.265627in}{0.535296in}}{\pgfqpoint{1.265627in}{0.529472in}}%
\pgfpathcurveto{\pgfqpoint{1.265627in}{0.523648in}}{\pgfqpoint{1.267940in}{0.518062in}}{\pgfqpoint{1.272059in}{0.513944in}}%
\pgfpathcurveto{\pgfqpoint{1.276177in}{0.509826in}}{\pgfqpoint{1.281763in}{0.507512in}}{\pgfqpoint{1.287587in}{0.507512in}}%
\pgfpathclose%
\pgfusepath{stroke,fill}%
\end{pgfscope}%
\begin{pgfscope}%
\pgfpathrectangle{\pgfqpoint{0.211875in}{0.211875in}}{\pgfqpoint{1.313625in}{1.279725in}}%
\pgfusepath{clip}%
\pgfsetbuttcap%
\pgfsetroundjoin%
\definecolor{currentfill}{rgb}{0.121569,0.466667,0.705882}%
\pgfsetfillcolor{currentfill}%
\pgfsetlinewidth{1.003750pt}%
\definecolor{currentstroke}{rgb}{0.121569,0.466667,0.705882}%
\pgfsetstrokecolor{currentstroke}%
\pgfsetdash{}{0pt}%
\pgfpathmoveto{\pgfqpoint{1.094220in}{0.978185in}}%
\pgfpathcurveto{\pgfqpoint{1.100044in}{0.978185in}}{\pgfqpoint{1.105630in}{0.980499in}}{\pgfqpoint{1.109749in}{0.984617in}}%
\pgfpathcurveto{\pgfqpoint{1.113867in}{0.988736in}}{\pgfqpoint{1.116181in}{0.994322in}}{\pgfqpoint{1.116181in}{1.000146in}}%
\pgfpathcurveto{\pgfqpoint{1.116181in}{1.005970in}}{\pgfqpoint{1.113867in}{1.011556in}}{\pgfqpoint{1.109749in}{1.015674in}}%
\pgfpathcurveto{\pgfqpoint{1.105630in}{1.019792in}}{\pgfqpoint{1.100044in}{1.022106in}}{\pgfqpoint{1.094220in}{1.022106in}}%
\pgfpathcurveto{\pgfqpoint{1.088396in}{1.022106in}}{\pgfqpoint{1.082810in}{1.019792in}}{\pgfqpoint{1.078692in}{1.015674in}}%
\pgfpathcurveto{\pgfqpoint{1.074574in}{1.011556in}}{\pgfqpoint{1.072260in}{1.005970in}}{\pgfqpoint{1.072260in}{1.000146in}}%
\pgfpathcurveto{\pgfqpoint{1.072260in}{0.994322in}}{\pgfqpoint{1.074574in}{0.988736in}}{\pgfqpoint{1.078692in}{0.984617in}}%
\pgfpathcurveto{\pgfqpoint{1.082810in}{0.980499in}}{\pgfqpoint{1.088396in}{0.978185in}}{\pgfqpoint{1.094220in}{0.978185in}}%
\pgfpathclose%
\pgfusepath{stroke,fill}%
\end{pgfscope}%
\begin{pgfscope}%
\pgfpathrectangle{\pgfqpoint{0.211875in}{0.211875in}}{\pgfqpoint{1.313625in}{1.279725in}}%
\pgfusepath{clip}%
\pgfsetbuttcap%
\pgfsetroundjoin%
\definecolor{currentfill}{rgb}{0.121569,0.466667,0.705882}%
\pgfsetfillcolor{currentfill}%
\pgfsetlinewidth{1.003750pt}%
\definecolor{currentstroke}{rgb}{0.121569,0.466667,0.705882}%
\pgfsetstrokecolor{currentstroke}%
\pgfsetdash{}{0pt}%
\pgfpathmoveto{\pgfqpoint{1.093877in}{0.977821in}}%
\pgfpathcurveto{\pgfqpoint{1.099701in}{0.977821in}}{\pgfqpoint{1.105287in}{0.980135in}}{\pgfqpoint{1.109406in}{0.984253in}}%
\pgfpathcurveto{\pgfqpoint{1.113524in}{0.988371in}}{\pgfqpoint{1.115838in}{0.993958in}}{\pgfqpoint{1.115838in}{0.999782in}}%
\pgfpathcurveto{\pgfqpoint{1.115838in}{1.005606in}}{\pgfqpoint{1.113524in}{1.011192in}}{\pgfqpoint{1.109406in}{1.015310in}}%
\pgfpathcurveto{\pgfqpoint{1.105287in}{1.019428in}}{\pgfqpoint{1.099701in}{1.021742in}}{\pgfqpoint{1.093877in}{1.021742in}}%
\pgfpathcurveto{\pgfqpoint{1.088053in}{1.021742in}}{\pgfqpoint{1.082467in}{1.019428in}}{\pgfqpoint{1.078349in}{1.015310in}}%
\pgfpathcurveto{\pgfqpoint{1.074231in}{1.011192in}}{\pgfqpoint{1.071917in}{1.005606in}}{\pgfqpoint{1.071917in}{0.999782in}}%
\pgfpathcurveto{\pgfqpoint{1.071917in}{0.993958in}}{\pgfqpoint{1.074231in}{0.988371in}}{\pgfqpoint{1.078349in}{0.984253in}}%
\pgfpathcurveto{\pgfqpoint{1.082467in}{0.980135in}}{\pgfqpoint{1.088053in}{0.977821in}}{\pgfqpoint{1.093877in}{0.977821in}}%
\pgfpathclose%
\pgfusepath{stroke,fill}%
\end{pgfscope}%
\begin{pgfscope}%
\pgfpathrectangle{\pgfqpoint{0.211875in}{0.211875in}}{\pgfqpoint{1.313625in}{1.279725in}}%
\pgfusepath{clip}%
\pgfsetbuttcap%
\pgfsetroundjoin%
\definecolor{currentfill}{rgb}{0.121569,0.466667,0.705882}%
\pgfsetfillcolor{currentfill}%
\pgfsetlinewidth{1.003750pt}%
\definecolor{currentstroke}{rgb}{0.121569,0.466667,0.705882}%
\pgfsetstrokecolor{currentstroke}%
\pgfsetdash{}{0pt}%
\pgfpathmoveto{\pgfqpoint{1.093788in}{0.977465in}}%
\pgfpathcurveto{\pgfqpoint{1.099612in}{0.977465in}}{\pgfqpoint{1.105198in}{0.979779in}}{\pgfqpoint{1.109317in}{0.983897in}}%
\pgfpathcurveto{\pgfqpoint{1.113435in}{0.988015in}}{\pgfqpoint{1.115749in}{0.993601in}}{\pgfqpoint{1.115749in}{0.999425in}}%
\pgfpathcurveto{\pgfqpoint{1.115749in}{1.005249in}}{\pgfqpoint{1.113435in}{1.010835in}}{\pgfqpoint{1.109317in}{1.014954in}}%
\pgfpathcurveto{\pgfqpoint{1.105198in}{1.019072in}}{\pgfqpoint{1.099612in}{1.021386in}}{\pgfqpoint{1.093788in}{1.021386in}}%
\pgfpathcurveto{\pgfqpoint{1.087964in}{1.021386in}}{\pgfqpoint{1.082378in}{1.019072in}}{\pgfqpoint{1.078260in}{1.014954in}}%
\pgfpathcurveto{\pgfqpoint{1.074142in}{1.010835in}}{\pgfqpoint{1.071828in}{1.005249in}}{\pgfqpoint{1.071828in}{0.999425in}}%
\pgfpathcurveto{\pgfqpoint{1.071828in}{0.993601in}}{\pgfqpoint{1.074142in}{0.988015in}}{\pgfqpoint{1.078260in}{0.983897in}}%
\pgfpathcurveto{\pgfqpoint{1.082378in}{0.979779in}}{\pgfqpoint{1.087964in}{0.977465in}}{\pgfqpoint{1.093788in}{0.977465in}}%
\pgfpathclose%
\pgfusepath{stroke,fill}%
\end{pgfscope}%
\begin{pgfscope}%
\pgfpathrectangle{\pgfqpoint{0.211875in}{0.211875in}}{\pgfqpoint{1.313625in}{1.279725in}}%
\pgfusepath{clip}%
\pgfsetbuttcap%
\pgfsetroundjoin%
\definecolor{currentfill}{rgb}{0.121569,0.466667,0.705882}%
\pgfsetfillcolor{currentfill}%
\pgfsetlinewidth{1.003750pt}%
\definecolor{currentstroke}{rgb}{0.121569,0.466667,0.705882}%
\pgfsetstrokecolor{currentstroke}%
\pgfsetdash{}{0pt}%
\pgfpathmoveto{\pgfqpoint{1.093775in}{0.977128in}}%
\pgfpathcurveto{\pgfqpoint{1.099599in}{0.977128in}}{\pgfqpoint{1.105185in}{0.979442in}}{\pgfqpoint{1.109303in}{0.983560in}}%
\pgfpathcurveto{\pgfqpoint{1.113421in}{0.987678in}}{\pgfqpoint{1.115735in}{0.993264in}}{\pgfqpoint{1.115735in}{0.999088in}}%
\pgfpathcurveto{\pgfqpoint{1.115735in}{1.004912in}}{\pgfqpoint{1.113421in}{1.010498in}}{\pgfqpoint{1.109303in}{1.014616in}}%
\pgfpathcurveto{\pgfqpoint{1.105185in}{1.018735in}}{\pgfqpoint{1.099599in}{1.021048in}}{\pgfqpoint{1.093775in}{1.021048in}}%
\pgfpathcurveto{\pgfqpoint{1.087951in}{1.021048in}}{\pgfqpoint{1.082365in}{1.018735in}}{\pgfqpoint{1.078247in}{1.014616in}}%
\pgfpathcurveto{\pgfqpoint{1.074128in}{1.010498in}}{\pgfqpoint{1.071815in}{1.004912in}}{\pgfqpoint{1.071815in}{0.999088in}}%
\pgfpathcurveto{\pgfqpoint{1.071815in}{0.993264in}}{\pgfqpoint{1.074128in}{0.987678in}}{\pgfqpoint{1.078247in}{0.983560in}}%
\pgfpathcurveto{\pgfqpoint{1.082365in}{0.979442in}}{\pgfqpoint{1.087951in}{0.977128in}}{\pgfqpoint{1.093775in}{0.977128in}}%
\pgfpathclose%
\pgfusepath{stroke,fill}%
\end{pgfscope}%
\begin{pgfscope}%
\pgfpathrectangle{\pgfqpoint{0.211875in}{0.211875in}}{\pgfqpoint{1.313625in}{1.279725in}}%
\pgfusepath{clip}%
\pgfsetbuttcap%
\pgfsetroundjoin%
\definecolor{currentfill}{rgb}{0.121569,0.466667,0.705882}%
\pgfsetfillcolor{currentfill}%
\pgfsetlinewidth{1.003750pt}%
\definecolor{currentstroke}{rgb}{0.121569,0.466667,0.705882}%
\pgfsetstrokecolor{currentstroke}%
\pgfsetdash{}{0pt}%
\pgfpathmoveto{\pgfqpoint{1.358886in}{0.504767in}}%
\pgfpathcurveto{\pgfqpoint{1.364710in}{0.504767in}}{\pgfqpoint{1.370296in}{0.507081in}}{\pgfqpoint{1.374414in}{0.511199in}}%
\pgfpathcurveto{\pgfqpoint{1.378532in}{0.515317in}}{\pgfqpoint{1.380846in}{0.520903in}}{\pgfqpoint{1.380846in}{0.526727in}}%
\pgfpathcurveto{\pgfqpoint{1.380846in}{0.532551in}}{\pgfqpoint{1.378532in}{0.538137in}}{\pgfqpoint{1.374414in}{0.542255in}}%
\pgfpathcurveto{\pgfqpoint{1.370296in}{0.546374in}}{\pgfqpoint{1.364710in}{0.548687in}}{\pgfqpoint{1.358886in}{0.548687in}}%
\pgfpathcurveto{\pgfqpoint{1.353062in}{0.548687in}}{\pgfqpoint{1.347476in}{0.546374in}}{\pgfqpoint{1.343358in}{0.542255in}}%
\pgfpathcurveto{\pgfqpoint{1.339239in}{0.538137in}}{\pgfqpoint{1.336926in}{0.532551in}}{\pgfqpoint{1.336926in}{0.526727in}}%
\pgfpathcurveto{\pgfqpoint{1.336926in}{0.520903in}}{\pgfqpoint{1.339239in}{0.515317in}}{\pgfqpoint{1.343358in}{0.511199in}}%
\pgfpathcurveto{\pgfqpoint{1.347476in}{0.507081in}}{\pgfqpoint{1.353062in}{0.504767in}}{\pgfqpoint{1.358886in}{0.504767in}}%
\pgfpathclose%
\pgfusepath{stroke,fill}%
\end{pgfscope}%
\begin{pgfscope}%
\pgfpathrectangle{\pgfqpoint{0.211875in}{0.211875in}}{\pgfqpoint{1.313625in}{1.279725in}}%
\pgfusepath{clip}%
\pgfsetbuttcap%
\pgfsetroundjoin%
\definecolor{currentfill}{rgb}{0.121569,0.466667,0.705882}%
\pgfsetfillcolor{currentfill}%
\pgfsetlinewidth{1.003750pt}%
\definecolor{currentstroke}{rgb}{0.121569,0.466667,0.705882}%
\pgfsetstrokecolor{currentstroke}%
\pgfsetdash{}{0pt}%
\pgfpathmoveto{\pgfqpoint{1.095755in}{0.976824in}}%
\pgfpathcurveto{\pgfqpoint{1.101579in}{0.976824in}}{\pgfqpoint{1.107165in}{0.979138in}}{\pgfqpoint{1.111283in}{0.983256in}}%
\pgfpathcurveto{\pgfqpoint{1.115401in}{0.987374in}}{\pgfqpoint{1.117715in}{0.992960in}}{\pgfqpoint{1.117715in}{0.998784in}}%
\pgfpathcurveto{\pgfqpoint{1.117715in}{1.004608in}}{\pgfqpoint{1.115401in}{1.010194in}}{\pgfqpoint{1.111283in}{1.014312in}}%
\pgfpathcurveto{\pgfqpoint{1.107165in}{1.018430in}}{\pgfqpoint{1.101579in}{1.020744in}}{\pgfqpoint{1.095755in}{1.020744in}}%
\pgfpathcurveto{\pgfqpoint{1.089931in}{1.020744in}}{\pgfqpoint{1.084345in}{1.018430in}}{\pgfqpoint{1.080227in}{1.014312in}}%
\pgfpathcurveto{\pgfqpoint{1.076109in}{1.010194in}}{\pgfqpoint{1.073795in}{1.004608in}}{\pgfqpoint{1.073795in}{0.998784in}}%
\pgfpathcurveto{\pgfqpoint{1.073795in}{0.992960in}}{\pgfqpoint{1.076109in}{0.987374in}}{\pgfqpoint{1.080227in}{0.983256in}}%
\pgfpathcurveto{\pgfqpoint{1.084345in}{0.979138in}}{\pgfqpoint{1.089931in}{0.976824in}}{\pgfqpoint{1.095755in}{0.976824in}}%
\pgfpathclose%
\pgfusepath{stroke,fill}%
\end{pgfscope}%
\begin{pgfscope}%
\pgfpathrectangle{\pgfqpoint{0.211875in}{0.211875in}}{\pgfqpoint{1.313625in}{1.279725in}}%
\pgfusepath{clip}%
\pgfsetbuttcap%
\pgfsetroundjoin%
\definecolor{currentfill}{rgb}{0.121569,0.466667,0.705882}%
\pgfsetfillcolor{currentfill}%
\pgfsetlinewidth{1.003750pt}%
\definecolor{currentstroke}{rgb}{0.121569,0.466667,0.705882}%
\pgfsetstrokecolor{currentstroke}%
\pgfsetdash{}{0pt}%
\pgfpathmoveto{\pgfqpoint{1.094901in}{0.976226in}}%
\pgfpathcurveto{\pgfqpoint{1.100725in}{0.976226in}}{\pgfqpoint{1.106311in}{0.978540in}}{\pgfqpoint{1.110430in}{0.982658in}}%
\pgfpathcurveto{\pgfqpoint{1.114548in}{0.986776in}}{\pgfqpoint{1.116862in}{0.992363in}}{\pgfqpoint{1.116862in}{0.998186in}}%
\pgfpathcurveto{\pgfqpoint{1.116862in}{1.004010in}}{\pgfqpoint{1.114548in}{1.009597in}}{\pgfqpoint{1.110430in}{1.013715in}}%
\pgfpathcurveto{\pgfqpoint{1.106311in}{1.017833in}}{\pgfqpoint{1.100725in}{1.020147in}}{\pgfqpoint{1.094901in}{1.020147in}}%
\pgfpathcurveto{\pgfqpoint{1.089077in}{1.020147in}}{\pgfqpoint{1.083491in}{1.017833in}}{\pgfqpoint{1.079373in}{1.013715in}}%
\pgfpathcurveto{\pgfqpoint{1.075255in}{1.009597in}}{\pgfqpoint{1.072941in}{1.004010in}}{\pgfqpoint{1.072941in}{0.998186in}}%
\pgfpathcurveto{\pgfqpoint{1.072941in}{0.992363in}}{\pgfqpoint{1.075255in}{0.986776in}}{\pgfqpoint{1.079373in}{0.982658in}}%
\pgfpathcurveto{\pgfqpoint{1.083491in}{0.978540in}}{\pgfqpoint{1.089077in}{0.976226in}}{\pgfqpoint{1.094901in}{0.976226in}}%
\pgfpathclose%
\pgfusepath{stroke,fill}%
\end{pgfscope}%
\begin{pgfscope}%
\pgfpathrectangle{\pgfqpoint{0.211875in}{0.211875in}}{\pgfqpoint{1.313625in}{1.279725in}}%
\pgfusepath{clip}%
\pgfsetbuttcap%
\pgfsetroundjoin%
\definecolor{currentfill}{rgb}{0.121569,0.466667,0.705882}%
\pgfsetfillcolor{currentfill}%
\pgfsetlinewidth{1.003750pt}%
\definecolor{currentstroke}{rgb}{0.121569,0.466667,0.705882}%
\pgfsetstrokecolor{currentstroke}%
\pgfsetdash{}{0pt}%
\pgfpathmoveto{\pgfqpoint{1.094525in}{0.975707in}}%
\pgfpathcurveto{\pgfqpoint{1.100349in}{0.975707in}}{\pgfqpoint{1.105935in}{0.978020in}}{\pgfqpoint{1.110053in}{0.982139in}}%
\pgfpathcurveto{\pgfqpoint{1.114172in}{0.986257in}}{\pgfqpoint{1.116485in}{0.991843in}}{\pgfqpoint{1.116485in}{0.997667in}}%
\pgfpathcurveto{\pgfqpoint{1.116485in}{1.003491in}}{\pgfqpoint{1.114172in}{1.009077in}}{\pgfqpoint{1.110053in}{1.013195in}}%
\pgfpathcurveto{\pgfqpoint{1.105935in}{1.017313in}}{\pgfqpoint{1.100349in}{1.019627in}}{\pgfqpoint{1.094525in}{1.019627in}}%
\pgfpathcurveto{\pgfqpoint{1.088701in}{1.019627in}}{\pgfqpoint{1.083115in}{1.017313in}}{\pgfqpoint{1.078997in}{1.013195in}}%
\pgfpathcurveto{\pgfqpoint{1.074879in}{1.009077in}}{\pgfqpoint{1.072565in}{1.003491in}}{\pgfqpoint{1.072565in}{0.997667in}}%
\pgfpathcurveto{\pgfqpoint{1.072565in}{0.991843in}}{\pgfqpoint{1.074879in}{0.986257in}}{\pgfqpoint{1.078997in}{0.982139in}}%
\pgfpathcurveto{\pgfqpoint{1.083115in}{0.978020in}}{\pgfqpoint{1.088701in}{0.975707in}}{\pgfqpoint{1.094525in}{0.975707in}}%
\pgfpathclose%
\pgfusepath{stroke,fill}%
\end{pgfscope}%
\begin{pgfscope}%
\pgfpathrectangle{\pgfqpoint{0.211875in}{0.211875in}}{\pgfqpoint{1.313625in}{1.279725in}}%
\pgfusepath{clip}%
\pgfsetbuttcap%
\pgfsetroundjoin%
\definecolor{currentfill}{rgb}{0.121569,0.466667,0.705882}%
\pgfsetfillcolor{currentfill}%
\pgfsetlinewidth{1.003750pt}%
\definecolor{currentstroke}{rgb}{0.121569,0.466667,0.705882}%
\pgfsetstrokecolor{currentstroke}%
\pgfsetdash{}{0pt}%
\pgfpathmoveto{\pgfqpoint{1.094375in}{0.975231in}}%
\pgfpathcurveto{\pgfqpoint{1.100199in}{0.975231in}}{\pgfqpoint{1.105785in}{0.977544in}}{\pgfqpoint{1.109904in}{0.981663in}}%
\pgfpathcurveto{\pgfqpoint{1.114022in}{0.985781in}}{\pgfqpoint{1.116336in}{0.991367in}}{\pgfqpoint{1.116336in}{0.997191in}}%
\pgfpathcurveto{\pgfqpoint{1.116336in}{1.003015in}}{\pgfqpoint{1.114022in}{1.008601in}}{\pgfqpoint{1.109904in}{1.012719in}}%
\pgfpathcurveto{\pgfqpoint{1.105785in}{1.016837in}}{\pgfqpoint{1.100199in}{1.019151in}}{\pgfqpoint{1.094375in}{1.019151in}}%
\pgfpathcurveto{\pgfqpoint{1.088551in}{1.019151in}}{\pgfqpoint{1.082965in}{1.016837in}}{\pgfqpoint{1.078847in}{1.012719in}}%
\pgfpathcurveto{\pgfqpoint{1.074729in}{1.008601in}}{\pgfqpoint{1.072415in}{1.003015in}}{\pgfqpoint{1.072415in}{0.997191in}}%
\pgfpathcurveto{\pgfqpoint{1.072415in}{0.991367in}}{\pgfqpoint{1.074729in}{0.985781in}}{\pgfqpoint{1.078847in}{0.981663in}}%
\pgfpathcurveto{\pgfqpoint{1.082965in}{0.977544in}}{\pgfqpoint{1.088551in}{0.975231in}}{\pgfqpoint{1.094375in}{0.975231in}}%
\pgfpathclose%
\pgfusepath{stroke,fill}%
\end{pgfscope}%
\begin{pgfscope}%
\pgfpathrectangle{\pgfqpoint{0.211875in}{0.211875in}}{\pgfqpoint{1.313625in}{1.279725in}}%
\pgfusepath{clip}%
\pgfsetbuttcap%
\pgfsetroundjoin%
\definecolor{currentfill}{rgb}{0.121569,0.466667,0.705882}%
\pgfsetfillcolor{currentfill}%
\pgfsetlinewidth{1.003750pt}%
\definecolor{currentstroke}{rgb}{0.121569,0.466667,0.705882}%
\pgfsetstrokecolor{currentstroke}%
\pgfsetdash{}{0pt}%
\pgfpathmoveto{\pgfqpoint{1.094334in}{0.974788in}}%
\pgfpathcurveto{\pgfqpoint{1.100158in}{0.974788in}}{\pgfqpoint{1.105744in}{0.977102in}}{\pgfqpoint{1.109862in}{0.981220in}}%
\pgfpathcurveto{\pgfqpoint{1.113980in}{0.985338in}}{\pgfqpoint{1.116294in}{0.990925in}}{\pgfqpoint{1.116294in}{0.996749in}}%
\pgfpathcurveto{\pgfqpoint{1.116294in}{1.002572in}}{\pgfqpoint{1.113980in}{1.008159in}}{\pgfqpoint{1.109862in}{1.012277in}}%
\pgfpathcurveto{\pgfqpoint{1.105744in}{1.016395in}}{\pgfqpoint{1.100158in}{1.018709in}}{\pgfqpoint{1.094334in}{1.018709in}}%
\pgfpathcurveto{\pgfqpoint{1.088510in}{1.018709in}}{\pgfqpoint{1.082924in}{1.016395in}}{\pgfqpoint{1.078805in}{1.012277in}}%
\pgfpathcurveto{\pgfqpoint{1.074687in}{1.008159in}}{\pgfqpoint{1.072373in}{1.002572in}}{\pgfqpoint{1.072373in}{0.996749in}}%
\pgfpathcurveto{\pgfqpoint{1.072373in}{0.990925in}}{\pgfqpoint{1.074687in}{0.985338in}}{\pgfqpoint{1.078805in}{0.981220in}}%
\pgfpathcurveto{\pgfqpoint{1.082924in}{0.977102in}}{\pgfqpoint{1.088510in}{0.974788in}}{\pgfqpoint{1.094334in}{0.974788in}}%
\pgfpathclose%
\pgfusepath{stroke,fill}%
\end{pgfscope}%
\begin{pgfscope}%
\pgfpathrectangle{\pgfqpoint{0.211875in}{0.211875in}}{\pgfqpoint{1.313625in}{1.279725in}}%
\pgfusepath{clip}%
\pgfsetbuttcap%
\pgfsetroundjoin%
\definecolor{currentfill}{rgb}{0.121569,0.466667,0.705882}%
\pgfsetfillcolor{currentfill}%
\pgfsetlinewidth{1.003750pt}%
\definecolor{currentstroke}{rgb}{0.121569,0.466667,0.705882}%
\pgfsetstrokecolor{currentstroke}%
\pgfsetdash{}{0pt}%
\pgfpathmoveto{\pgfqpoint{1.094342in}{0.974370in}}%
\pgfpathcurveto{\pgfqpoint{1.100166in}{0.974370in}}{\pgfqpoint{1.105752in}{0.976684in}}{\pgfqpoint{1.109870in}{0.980802in}}%
\pgfpathcurveto{\pgfqpoint{1.113988in}{0.984920in}}{\pgfqpoint{1.116302in}{0.990506in}}{\pgfqpoint{1.116302in}{0.996330in}}%
\pgfpathcurveto{\pgfqpoint{1.116302in}{1.002154in}}{\pgfqpoint{1.113988in}{1.007740in}}{\pgfqpoint{1.109870in}{1.011858in}}%
\pgfpathcurveto{\pgfqpoint{1.105752in}{1.015976in}}{\pgfqpoint{1.100166in}{1.018290in}}{\pgfqpoint{1.094342in}{1.018290in}}%
\pgfpathcurveto{\pgfqpoint{1.088518in}{1.018290in}}{\pgfqpoint{1.082932in}{1.015976in}}{\pgfqpoint{1.078814in}{1.011858in}}%
\pgfpathcurveto{\pgfqpoint{1.074696in}{1.007740in}}{\pgfqpoint{1.072382in}{1.002154in}}{\pgfqpoint{1.072382in}{0.996330in}}%
\pgfpathcurveto{\pgfqpoint{1.072382in}{0.990506in}}{\pgfqpoint{1.074696in}{0.984920in}}{\pgfqpoint{1.078814in}{0.980802in}}%
\pgfpathcurveto{\pgfqpoint{1.082932in}{0.976684in}}{\pgfqpoint{1.088518in}{0.974370in}}{\pgfqpoint{1.094342in}{0.974370in}}%
\pgfpathclose%
\pgfusepath{stroke,fill}%
\end{pgfscope}%
\begin{pgfscope}%
\pgfpathrectangle{\pgfqpoint{0.211875in}{0.211875in}}{\pgfqpoint{1.313625in}{1.279725in}}%
\pgfusepath{clip}%
\pgfsetbuttcap%
\pgfsetroundjoin%
\definecolor{currentfill}{rgb}{0.121569,0.466667,0.705882}%
\pgfsetfillcolor{currentfill}%
\pgfsetlinewidth{1.003750pt}%
\definecolor{currentstroke}{rgb}{0.121569,0.466667,0.705882}%
\pgfsetstrokecolor{currentstroke}%
\pgfsetdash{}{0pt}%
\pgfpathmoveto{\pgfqpoint{1.094375in}{0.973971in}}%
\pgfpathcurveto{\pgfqpoint{1.100199in}{0.973971in}}{\pgfqpoint{1.105785in}{0.976285in}}{\pgfqpoint{1.109903in}{0.980403in}}%
\pgfpathcurveto{\pgfqpoint{1.114021in}{0.984521in}}{\pgfqpoint{1.116335in}{0.990107in}}{\pgfqpoint{1.116335in}{0.995931in}}%
\pgfpathcurveto{\pgfqpoint{1.116335in}{1.001755in}}{\pgfqpoint{1.114021in}{1.007341in}}{\pgfqpoint{1.109903in}{1.011459in}}%
\pgfpathcurveto{\pgfqpoint{1.105785in}{1.015577in}}{\pgfqpoint{1.100199in}{1.017891in}}{\pgfqpoint{1.094375in}{1.017891in}}%
\pgfpathcurveto{\pgfqpoint{1.088551in}{1.017891in}}{\pgfqpoint{1.082965in}{1.015577in}}{\pgfqpoint{1.078846in}{1.011459in}}%
\pgfpathcurveto{\pgfqpoint{1.074728in}{1.007341in}}{\pgfqpoint{1.072414in}{1.001755in}}{\pgfqpoint{1.072414in}{0.995931in}}%
\pgfpathcurveto{\pgfqpoint{1.072414in}{0.990107in}}{\pgfqpoint{1.074728in}{0.984521in}}{\pgfqpoint{1.078846in}{0.980403in}}%
\pgfpathcurveto{\pgfqpoint{1.082965in}{0.976285in}}{\pgfqpoint{1.088551in}{0.973971in}}{\pgfqpoint{1.094375in}{0.973971in}}%
\pgfpathclose%
\pgfusepath{stroke,fill}%
\end{pgfscope}%
\begin{pgfscope}%
\pgfpathrectangle{\pgfqpoint{0.211875in}{0.211875in}}{\pgfqpoint{1.313625in}{1.279725in}}%
\pgfusepath{clip}%
\pgfsetbuttcap%
\pgfsetroundjoin%
\definecolor{currentfill}{rgb}{0.121569,0.466667,0.705882}%
\pgfsetfillcolor{currentfill}%
\pgfsetlinewidth{1.003750pt}%
\definecolor{currentstroke}{rgb}{0.121569,0.466667,0.705882}%
\pgfsetstrokecolor{currentstroke}%
\pgfsetdash{}{0pt}%
\pgfpathmoveto{\pgfqpoint{1.308462in}{0.636742in}}%
\pgfpathcurveto{\pgfqpoint{1.314286in}{0.636742in}}{\pgfqpoint{1.319872in}{0.639056in}}{\pgfqpoint{1.323990in}{0.643174in}}%
\pgfpathcurveto{\pgfqpoint{1.328109in}{0.647292in}}{\pgfqpoint{1.330422in}{0.652878in}}{\pgfqpoint{1.330422in}{0.658702in}}%
\pgfpathcurveto{\pgfqpoint{1.330422in}{0.664526in}}{\pgfqpoint{1.328109in}{0.670112in}}{\pgfqpoint{1.323990in}{0.674230in}}%
\pgfpathcurveto{\pgfqpoint{1.319872in}{0.678349in}}{\pgfqpoint{1.314286in}{0.680662in}}{\pgfqpoint{1.308462in}{0.680662in}}%
\pgfpathcurveto{\pgfqpoint{1.302638in}{0.680662in}}{\pgfqpoint{1.297052in}{0.678349in}}{\pgfqpoint{1.292934in}{0.674230in}}%
\pgfpathcurveto{\pgfqpoint{1.288816in}{0.670112in}}{\pgfqpoint{1.286502in}{0.664526in}}{\pgfqpoint{1.286502in}{0.658702in}}%
\pgfpathcurveto{\pgfqpoint{1.286502in}{0.652878in}}{\pgfqpoint{1.288816in}{0.647292in}}{\pgfqpoint{1.292934in}{0.643174in}}%
\pgfpathcurveto{\pgfqpoint{1.297052in}{0.639056in}}{\pgfqpoint{1.302638in}{0.636742in}}{\pgfqpoint{1.308462in}{0.636742in}}%
\pgfpathclose%
\pgfusepath{stroke,fill}%
\end{pgfscope}%
\begin{pgfscope}%
\pgfpathrectangle{\pgfqpoint{0.211875in}{0.211875in}}{\pgfqpoint{1.313625in}{1.279725in}}%
\pgfusepath{clip}%
\pgfsetbuttcap%
\pgfsetroundjoin%
\definecolor{currentfill}{rgb}{0.121569,0.466667,0.705882}%
\pgfsetfillcolor{currentfill}%
\pgfsetlinewidth{1.003750pt}%
\definecolor{currentstroke}{rgb}{0.121569,0.466667,0.705882}%
\pgfsetstrokecolor{currentstroke}%
\pgfsetdash{}{0pt}%
\pgfpathmoveto{\pgfqpoint{1.096714in}{0.968373in}}%
\pgfpathcurveto{\pgfqpoint{1.102537in}{0.968373in}}{\pgfqpoint{1.108124in}{0.970686in}}{\pgfqpoint{1.112242in}{0.974805in}}%
\pgfpathcurveto{\pgfqpoint{1.116360in}{0.978923in}}{\pgfqpoint{1.118674in}{0.984509in}}{\pgfqpoint{1.118674in}{0.990333in}}%
\pgfpathcurveto{\pgfqpoint{1.118674in}{0.996157in}}{\pgfqpoint{1.116360in}{1.001743in}}{\pgfqpoint{1.112242in}{1.005861in}}%
\pgfpathcurveto{\pgfqpoint{1.108124in}{1.009979in}}{\pgfqpoint{1.102537in}{1.012293in}}{\pgfqpoint{1.096714in}{1.012293in}}%
\pgfpathcurveto{\pgfqpoint{1.090890in}{1.012293in}}{\pgfqpoint{1.085303in}{1.009979in}}{\pgfqpoint{1.081185in}{1.005861in}}%
\pgfpathcurveto{\pgfqpoint{1.077067in}{1.001743in}}{\pgfqpoint{1.074753in}{0.996157in}}{\pgfqpoint{1.074753in}{0.990333in}}%
\pgfpathcurveto{\pgfqpoint{1.074753in}{0.984509in}}{\pgfqpoint{1.077067in}{0.978923in}}{\pgfqpoint{1.081185in}{0.974805in}}%
\pgfpathcurveto{\pgfqpoint{1.085303in}{0.970686in}}{\pgfqpoint{1.090890in}{0.968373in}}{\pgfqpoint{1.096714in}{0.968373in}}%
\pgfpathclose%
\pgfusepath{stroke,fill}%
\end{pgfscope}%
\begin{pgfscope}%
\pgfpathrectangle{\pgfqpoint{0.211875in}{0.211875in}}{\pgfqpoint{1.313625in}{1.279725in}}%
\pgfusepath{clip}%
\pgfsetbuttcap%
\pgfsetroundjoin%
\definecolor{currentfill}{rgb}{0.121569,0.466667,0.705882}%
\pgfsetfillcolor{currentfill}%
\pgfsetlinewidth{1.003750pt}%
\definecolor{currentstroke}{rgb}{0.121569,0.466667,0.705882}%
\pgfsetstrokecolor{currentstroke}%
\pgfsetdash{}{0pt}%
\pgfpathmoveto{\pgfqpoint{1.096442in}{0.967959in}}%
\pgfpathcurveto{\pgfqpoint{1.102266in}{0.967959in}}{\pgfqpoint{1.107852in}{0.970273in}}{\pgfqpoint{1.111971in}{0.974391in}}%
\pgfpathcurveto{\pgfqpoint{1.116089in}{0.978509in}}{\pgfqpoint{1.118403in}{0.984095in}}{\pgfqpoint{1.118403in}{0.989919in}}%
\pgfpathcurveto{\pgfqpoint{1.118403in}{0.995743in}}{\pgfqpoint{1.116089in}{1.001329in}}{\pgfqpoint{1.111971in}{1.005448in}}%
\pgfpathcurveto{\pgfqpoint{1.107852in}{1.009566in}}{\pgfqpoint{1.102266in}{1.011880in}}{\pgfqpoint{1.096442in}{1.011880in}}%
\pgfpathcurveto{\pgfqpoint{1.090618in}{1.011880in}}{\pgfqpoint{1.085032in}{1.009566in}}{\pgfqpoint{1.080914in}{1.005448in}}%
\pgfpathcurveto{\pgfqpoint{1.076796in}{1.001329in}}{\pgfqpoint{1.074482in}{0.995743in}}{\pgfqpoint{1.074482in}{0.989919in}}%
\pgfpathcurveto{\pgfqpoint{1.074482in}{0.984095in}}{\pgfqpoint{1.076796in}{0.978509in}}{\pgfqpoint{1.080914in}{0.974391in}}%
\pgfpathcurveto{\pgfqpoint{1.085032in}{0.970273in}}{\pgfqpoint{1.090618in}{0.967959in}}{\pgfqpoint{1.096442in}{0.967959in}}%
\pgfpathclose%
\pgfusepath{stroke,fill}%
\end{pgfscope}%
\begin{pgfscope}%
\pgfpathrectangle{\pgfqpoint{0.211875in}{0.211875in}}{\pgfqpoint{1.313625in}{1.279725in}}%
\pgfusepath{clip}%
\pgfsetbuttcap%
\pgfsetroundjoin%
\definecolor{currentfill}{rgb}{0.121569,0.466667,0.705882}%
\pgfsetfillcolor{currentfill}%
\pgfsetlinewidth{1.003750pt}%
\definecolor{currentstroke}{rgb}{0.121569,0.466667,0.705882}%
\pgfsetstrokecolor{currentstroke}%
\pgfsetdash{}{0pt}%
\pgfpathmoveto{\pgfqpoint{1.096302in}{0.967624in}}%
\pgfpathcurveto{\pgfqpoint{1.102126in}{0.967624in}}{\pgfqpoint{1.107712in}{0.969937in}}{\pgfqpoint{1.111830in}{0.974056in}}%
\pgfpathcurveto{\pgfqpoint{1.115948in}{0.978174in}}{\pgfqpoint{1.118262in}{0.983760in}}{\pgfqpoint{1.118262in}{0.989584in}}%
\pgfpathcurveto{\pgfqpoint{1.118262in}{0.995408in}}{\pgfqpoint{1.115948in}{1.000994in}}{\pgfqpoint{1.111830in}{1.005112in}}%
\pgfpathcurveto{\pgfqpoint{1.107712in}{1.009230in}}{\pgfqpoint{1.102126in}{1.011544in}}{\pgfqpoint{1.096302in}{1.011544in}}%
\pgfpathcurveto{\pgfqpoint{1.090478in}{1.011544in}}{\pgfqpoint{1.084892in}{1.009230in}}{\pgfqpoint{1.080774in}{1.005112in}}%
\pgfpathcurveto{\pgfqpoint{1.076656in}{1.000994in}}{\pgfqpoint{1.074342in}{0.995408in}}{\pgfqpoint{1.074342in}{0.989584in}}%
\pgfpathcurveto{\pgfqpoint{1.074342in}{0.983760in}}{\pgfqpoint{1.076656in}{0.978174in}}{\pgfqpoint{1.080774in}{0.974056in}}%
\pgfpathcurveto{\pgfqpoint{1.084892in}{0.969937in}}{\pgfqpoint{1.090478in}{0.967624in}}{\pgfqpoint{1.096302in}{0.967624in}}%
\pgfpathclose%
\pgfusepath{stroke,fill}%
\end{pgfscope}%
\begin{pgfscope}%
\pgfpathrectangle{\pgfqpoint{0.211875in}{0.211875in}}{\pgfqpoint{1.313625in}{1.279725in}}%
\pgfusepath{clip}%
\pgfsetbuttcap%
\pgfsetroundjoin%
\definecolor{currentfill}{rgb}{0.121569,0.466667,0.705882}%
\pgfsetfillcolor{currentfill}%
\pgfsetlinewidth{1.003750pt}%
\definecolor{currentstroke}{rgb}{0.121569,0.466667,0.705882}%
\pgfsetstrokecolor{currentstroke}%
\pgfsetdash{}{0pt}%
\pgfpathmoveto{\pgfqpoint{1.096231in}{0.967340in}}%
\pgfpathcurveto{\pgfqpoint{1.102055in}{0.967340in}}{\pgfqpoint{1.107642in}{0.969654in}}{\pgfqpoint{1.111760in}{0.973772in}}%
\pgfpathcurveto{\pgfqpoint{1.115878in}{0.977890in}}{\pgfqpoint{1.118192in}{0.983476in}}{\pgfqpoint{1.118192in}{0.989300in}}%
\pgfpathcurveto{\pgfqpoint{1.118192in}{0.995124in}}{\pgfqpoint{1.115878in}{1.000710in}}{\pgfqpoint{1.111760in}{1.004828in}}%
\pgfpathcurveto{\pgfqpoint{1.107642in}{1.008947in}}{\pgfqpoint{1.102055in}{1.011260in}}{\pgfqpoint{1.096231in}{1.011260in}}%
\pgfpathcurveto{\pgfqpoint{1.090407in}{1.011260in}}{\pgfqpoint{1.084821in}{1.008947in}}{\pgfqpoint{1.080703in}{1.004828in}}%
\pgfpathcurveto{\pgfqpoint{1.076585in}{1.000710in}}{\pgfqpoint{1.074271in}{0.995124in}}{\pgfqpoint{1.074271in}{0.989300in}}%
\pgfpathcurveto{\pgfqpoint{1.074271in}{0.983476in}}{\pgfqpoint{1.076585in}{0.977890in}}{\pgfqpoint{1.080703in}{0.973772in}}%
\pgfpathcurveto{\pgfqpoint{1.084821in}{0.969654in}}{\pgfqpoint{1.090407in}{0.967340in}}{\pgfqpoint{1.096231in}{0.967340in}}%
\pgfpathclose%
\pgfusepath{stroke,fill}%
\end{pgfscope}%
\begin{pgfscope}%
\pgfpathrectangle{\pgfqpoint{0.211875in}{0.211875in}}{\pgfqpoint{1.313625in}{1.279725in}}%
\pgfusepath{clip}%
\pgfsetbuttcap%
\pgfsetroundjoin%
\definecolor{currentfill}{rgb}{0.121569,0.466667,0.705882}%
\pgfsetfillcolor{currentfill}%
\pgfsetlinewidth{1.003750pt}%
\definecolor{currentstroke}{rgb}{0.121569,0.466667,0.705882}%
\pgfsetstrokecolor{currentstroke}%
\pgfsetdash{}{0pt}%
\pgfpathmoveto{\pgfqpoint{1.387138in}{0.615330in}}%
\pgfpathcurveto{\pgfqpoint{1.392962in}{0.615330in}}{\pgfqpoint{1.398548in}{0.617644in}}{\pgfqpoint{1.402666in}{0.621762in}}%
\pgfpathcurveto{\pgfqpoint{1.406784in}{0.625880in}}{\pgfqpoint{1.409098in}{0.631466in}}{\pgfqpoint{1.409098in}{0.637290in}}%
\pgfpathcurveto{\pgfqpoint{1.409098in}{0.643114in}}{\pgfqpoint{1.406784in}{0.648700in}}{\pgfqpoint{1.402666in}{0.652818in}}%
\pgfpathcurveto{\pgfqpoint{1.398548in}{0.656936in}}{\pgfqpoint{1.392962in}{0.659250in}}{\pgfqpoint{1.387138in}{0.659250in}}%
\pgfpathcurveto{\pgfqpoint{1.381314in}{0.659250in}}{\pgfqpoint{1.375728in}{0.656936in}}{\pgfqpoint{1.371610in}{0.652818in}}%
\pgfpathcurveto{\pgfqpoint{1.367492in}{0.648700in}}{\pgfqpoint{1.365178in}{0.643114in}}{\pgfqpoint{1.365178in}{0.637290in}}%
\pgfpathcurveto{\pgfqpoint{1.365178in}{0.631466in}}{\pgfqpoint{1.367492in}{0.625880in}}{\pgfqpoint{1.371610in}{0.621762in}}%
\pgfpathcurveto{\pgfqpoint{1.375728in}{0.617644in}}{\pgfqpoint{1.381314in}{0.615330in}}{\pgfqpoint{1.387138in}{0.615330in}}%
\pgfpathclose%
\pgfusepath{stroke,fill}%
\end{pgfscope}%
\begin{pgfscope}%
\pgfpathrectangle{\pgfqpoint{0.211875in}{0.211875in}}{\pgfqpoint{1.313625in}{1.279725in}}%
\pgfusepath{clip}%
\pgfsetbuttcap%
\pgfsetroundjoin%
\definecolor{currentfill}{rgb}{0.121569,0.466667,0.705882}%
\pgfsetfillcolor{currentfill}%
\pgfsetlinewidth{1.003750pt}%
\definecolor{currentstroke}{rgb}{0.121569,0.466667,0.705882}%
\pgfsetstrokecolor{currentstroke}%
\pgfsetdash{}{0pt}%
\pgfpathmoveto{\pgfqpoint{1.102918in}{0.965688in}}%
\pgfpathcurveto{\pgfqpoint{1.108741in}{0.965688in}}{\pgfqpoint{1.114328in}{0.968002in}}{\pgfqpoint{1.118446in}{0.972120in}}%
\pgfpathcurveto{\pgfqpoint{1.122564in}{0.976238in}}{\pgfqpoint{1.124878in}{0.981824in}}{\pgfqpoint{1.124878in}{0.987648in}}%
\pgfpathcurveto{\pgfqpoint{1.124878in}{0.993472in}}{\pgfqpoint{1.122564in}{0.999059in}}{\pgfqpoint{1.118446in}{1.003177in}}%
\pgfpathcurveto{\pgfqpoint{1.114328in}{1.007295in}}{\pgfqpoint{1.108741in}{1.009609in}}{\pgfqpoint{1.102918in}{1.009609in}}%
\pgfpathcurveto{\pgfqpoint{1.097094in}{1.009609in}}{\pgfqpoint{1.091507in}{1.007295in}}{\pgfqpoint{1.087389in}{1.003177in}}%
\pgfpathcurveto{\pgfqpoint{1.083271in}{0.999059in}}{\pgfqpoint{1.080957in}{0.993472in}}{\pgfqpoint{1.080957in}{0.987648in}}%
\pgfpathcurveto{\pgfqpoint{1.080957in}{0.981824in}}{\pgfqpoint{1.083271in}{0.976238in}}{\pgfqpoint{1.087389in}{0.972120in}}%
\pgfpathcurveto{\pgfqpoint{1.091507in}{0.968002in}}{\pgfqpoint{1.097094in}{0.965688in}}{\pgfqpoint{1.102918in}{0.965688in}}%
\pgfpathclose%
\pgfusepath{stroke,fill}%
\end{pgfscope}%
\begin{pgfscope}%
\pgfpathrectangle{\pgfqpoint{0.211875in}{0.211875in}}{\pgfqpoint{1.313625in}{1.279725in}}%
\pgfusepath{clip}%
\pgfsetbuttcap%
\pgfsetroundjoin%
\definecolor{currentfill}{rgb}{0.121569,0.466667,0.705882}%
\pgfsetfillcolor{currentfill}%
\pgfsetlinewidth{1.003750pt}%
\definecolor{currentstroke}{rgb}{0.121569,0.466667,0.705882}%
\pgfsetstrokecolor{currentstroke}%
\pgfsetdash{}{0pt}%
\pgfpathmoveto{\pgfqpoint{1.440033in}{0.638011in}}%
\pgfpathcurveto{\pgfqpoint{1.445857in}{0.638011in}}{\pgfqpoint{1.451444in}{0.640325in}}{\pgfqpoint{1.455562in}{0.644443in}}%
\pgfpathcurveto{\pgfqpoint{1.459680in}{0.648562in}}{\pgfqpoint{1.461994in}{0.654148in}}{\pgfqpoint{1.461994in}{0.659972in}}%
\pgfpathcurveto{\pgfqpoint{1.461994in}{0.665796in}}{\pgfqpoint{1.459680in}{0.671382in}}{\pgfqpoint{1.455562in}{0.675500in}}%
\pgfpathcurveto{\pgfqpoint{1.451444in}{0.679618in}}{\pgfqpoint{1.445857in}{0.681932in}}{\pgfqpoint{1.440033in}{0.681932in}}%
\pgfpathcurveto{\pgfqpoint{1.434210in}{0.681932in}}{\pgfqpoint{1.428623in}{0.679618in}}{\pgfqpoint{1.424505in}{0.675500in}}%
\pgfpathcurveto{\pgfqpoint{1.420387in}{0.671382in}}{\pgfqpoint{1.418073in}{0.665796in}}{\pgfqpoint{1.418073in}{0.659972in}}%
\pgfpathcurveto{\pgfqpoint{1.418073in}{0.654148in}}{\pgfqpoint{1.420387in}{0.648562in}}{\pgfqpoint{1.424505in}{0.644443in}}%
\pgfpathcurveto{\pgfqpoint{1.428623in}{0.640325in}}{\pgfqpoint{1.434210in}{0.638011in}}{\pgfqpoint{1.440033in}{0.638011in}}%
\pgfpathclose%
\pgfusepath{stroke,fill}%
\end{pgfscope}%
\begin{pgfscope}%
\pgfpathrectangle{\pgfqpoint{0.211875in}{0.211875in}}{\pgfqpoint{1.313625in}{1.279725in}}%
\pgfusepath{clip}%
\pgfsetbuttcap%
\pgfsetroundjoin%
\definecolor{currentfill}{rgb}{0.121569,0.466667,0.705882}%
\pgfsetfillcolor{currentfill}%
\pgfsetlinewidth{1.003750pt}%
\definecolor{currentstroke}{rgb}{0.121569,0.466667,0.705882}%
\pgfsetstrokecolor{currentstroke}%
\pgfsetdash{}{0pt}%
\pgfpathmoveto{\pgfqpoint{1.098796in}{0.954447in}}%
\pgfpathcurveto{\pgfqpoint{1.104620in}{0.954447in}}{\pgfqpoint{1.110206in}{0.956761in}}{\pgfqpoint{1.114324in}{0.960879in}}%
\pgfpathcurveto{\pgfqpoint{1.118443in}{0.964997in}}{\pgfqpoint{1.120756in}{0.970583in}}{\pgfqpoint{1.120756in}{0.976407in}}%
\pgfpathcurveto{\pgfqpoint{1.120756in}{0.982231in}}{\pgfqpoint{1.118443in}{0.987817in}}{\pgfqpoint{1.114324in}{0.991935in}}%
\pgfpathcurveto{\pgfqpoint{1.110206in}{0.996054in}}{\pgfqpoint{1.104620in}{0.998367in}}{\pgfqpoint{1.098796in}{0.998367in}}%
\pgfpathcurveto{\pgfqpoint{1.092972in}{0.998367in}}{\pgfqpoint{1.087386in}{0.996054in}}{\pgfqpoint{1.083268in}{0.991935in}}%
\pgfpathcurveto{\pgfqpoint{1.079150in}{0.987817in}}{\pgfqpoint{1.076836in}{0.982231in}}{\pgfqpoint{1.076836in}{0.976407in}}%
\pgfpathcurveto{\pgfqpoint{1.076836in}{0.970583in}}{\pgfqpoint{1.079150in}{0.964997in}}{\pgfqpoint{1.083268in}{0.960879in}}%
\pgfpathcurveto{\pgfqpoint{1.087386in}{0.956761in}}{\pgfqpoint{1.092972in}{0.954447in}}{\pgfqpoint{1.098796in}{0.954447in}}%
\pgfpathclose%
\pgfusepath{stroke,fill}%
\end{pgfscope}%
\begin{pgfscope}%
\pgfpathrectangle{\pgfqpoint{0.211875in}{0.211875in}}{\pgfqpoint{1.313625in}{1.279725in}}%
\pgfusepath{clip}%
\pgfsetbuttcap%
\pgfsetroundjoin%
\definecolor{currentfill}{rgb}{0.121569,0.466667,0.705882}%
\pgfsetfillcolor{currentfill}%
\pgfsetlinewidth{1.003750pt}%
\definecolor{currentstroke}{rgb}{0.121569,0.466667,0.705882}%
\pgfsetstrokecolor{currentstroke}%
\pgfsetdash{}{0pt}%
\pgfpathmoveto{\pgfqpoint{1.099923in}{0.956493in}}%
\pgfpathcurveto{\pgfqpoint{1.105747in}{0.956493in}}{\pgfqpoint{1.111333in}{0.958807in}}{\pgfqpoint{1.115451in}{0.962925in}}%
\pgfpathcurveto{\pgfqpoint{1.119570in}{0.967043in}}{\pgfqpoint{1.121883in}{0.972629in}}{\pgfqpoint{1.121883in}{0.978453in}}%
\pgfpathcurveto{\pgfqpoint{1.121883in}{0.984277in}}{\pgfqpoint{1.119570in}{0.989863in}}{\pgfqpoint{1.115451in}{0.993981in}}%
\pgfpathcurveto{\pgfqpoint{1.111333in}{0.998100in}}{\pgfqpoint{1.105747in}{1.000413in}}{\pgfqpoint{1.099923in}{1.000413in}}%
\pgfpathcurveto{\pgfqpoint{1.094099in}{1.000413in}}{\pgfqpoint{1.088513in}{0.998100in}}{\pgfqpoint{1.084395in}{0.993981in}}%
\pgfpathcurveto{\pgfqpoint{1.080277in}{0.989863in}}{\pgfqpoint{1.077963in}{0.984277in}}{\pgfqpoint{1.077963in}{0.978453in}}%
\pgfpathcurveto{\pgfqpoint{1.077963in}{0.972629in}}{\pgfqpoint{1.080277in}{0.967043in}}{\pgfqpoint{1.084395in}{0.962925in}}%
\pgfpathcurveto{\pgfqpoint{1.088513in}{0.958807in}}{\pgfqpoint{1.094099in}{0.956493in}}{\pgfqpoint{1.099923in}{0.956493in}}%
\pgfpathclose%
\pgfusepath{stroke,fill}%
\end{pgfscope}%
\begin{pgfscope}%
\pgfpathrectangle{\pgfqpoint{0.211875in}{0.211875in}}{\pgfqpoint{1.313625in}{1.279725in}}%
\pgfusepath{clip}%
\pgfsetbuttcap%
\pgfsetroundjoin%
\definecolor{currentfill}{rgb}{0.121569,0.466667,0.705882}%
\pgfsetfillcolor{currentfill}%
\pgfsetlinewidth{1.003750pt}%
\definecolor{currentstroke}{rgb}{0.121569,0.466667,0.705882}%
\pgfsetstrokecolor{currentstroke}%
\pgfsetdash{}{0pt}%
\pgfpathmoveto{\pgfqpoint{1.221813in}{0.487313in}}%
\pgfpathcurveto{\pgfqpoint{1.227637in}{0.487313in}}{\pgfqpoint{1.233223in}{0.489627in}}{\pgfqpoint{1.237341in}{0.493745in}}%
\pgfpathcurveto{\pgfqpoint{1.241459in}{0.497863in}}{\pgfqpoint{1.243773in}{0.503450in}}{\pgfqpoint{1.243773in}{0.509274in}}%
\pgfpathcurveto{\pgfqpoint{1.243773in}{0.515097in}}{\pgfqpoint{1.241459in}{0.520684in}}{\pgfqpoint{1.237341in}{0.524802in}}%
\pgfpathcurveto{\pgfqpoint{1.233223in}{0.528920in}}{\pgfqpoint{1.227637in}{0.531234in}}{\pgfqpoint{1.221813in}{0.531234in}}%
\pgfpathcurveto{\pgfqpoint{1.215989in}{0.531234in}}{\pgfqpoint{1.210403in}{0.528920in}}{\pgfqpoint{1.206285in}{0.524802in}}%
\pgfpathcurveto{\pgfqpoint{1.202166in}{0.520684in}}{\pgfqpoint{1.199852in}{0.515097in}}{\pgfqpoint{1.199852in}{0.509274in}}%
\pgfpathcurveto{\pgfqpoint{1.199852in}{0.503450in}}{\pgfqpoint{1.202166in}{0.497863in}}{\pgfqpoint{1.206285in}{0.493745in}}%
\pgfpathcurveto{\pgfqpoint{1.210403in}{0.489627in}}{\pgfqpoint{1.215989in}{0.487313in}}{\pgfqpoint{1.221813in}{0.487313in}}%
\pgfpathclose%
\pgfusepath{stroke,fill}%
\end{pgfscope}%
\begin{pgfscope}%
\pgfpathrectangle{\pgfqpoint{0.211875in}{0.211875in}}{\pgfqpoint{1.313625in}{1.279725in}}%
\pgfusepath{clip}%
\pgfsetbuttcap%
\pgfsetroundjoin%
\definecolor{currentfill}{rgb}{0.121569,0.466667,0.705882}%
\pgfsetfillcolor{currentfill}%
\pgfsetlinewidth{1.003750pt}%
\definecolor{currentstroke}{rgb}{0.121569,0.466667,0.705882}%
\pgfsetstrokecolor{currentstroke}%
\pgfsetdash{}{0pt}%
\pgfpathmoveto{\pgfqpoint{1.102516in}{0.992374in}}%
\pgfpathcurveto{\pgfqpoint{1.108340in}{0.992374in}}{\pgfqpoint{1.113927in}{0.994688in}}{\pgfqpoint{1.118045in}{0.998806in}}%
\pgfpathcurveto{\pgfqpoint{1.122163in}{1.002924in}}{\pgfqpoint{1.124477in}{1.008511in}}{\pgfqpoint{1.124477in}{1.014335in}}%
\pgfpathcurveto{\pgfqpoint{1.124477in}{1.020158in}}{\pgfqpoint{1.122163in}{1.025745in}}{\pgfqpoint{1.118045in}{1.029863in}}%
\pgfpathcurveto{\pgfqpoint{1.113927in}{1.033981in}}{\pgfqpoint{1.108340in}{1.036295in}}{\pgfqpoint{1.102516in}{1.036295in}}%
\pgfpathcurveto{\pgfqpoint{1.096692in}{1.036295in}}{\pgfqpoint{1.091106in}{1.033981in}}{\pgfqpoint{1.086988in}{1.029863in}}%
\pgfpathcurveto{\pgfqpoint{1.082870in}{1.025745in}}{\pgfqpoint{1.080556in}{1.020158in}}{\pgfqpoint{1.080556in}{1.014335in}}%
\pgfpathcurveto{\pgfqpoint{1.080556in}{1.008511in}}{\pgfqpoint{1.082870in}{1.002924in}}{\pgfqpoint{1.086988in}{0.998806in}}%
\pgfpathcurveto{\pgfqpoint{1.091106in}{0.994688in}}{\pgfqpoint{1.096692in}{0.992374in}}{\pgfqpoint{1.102516in}{0.992374in}}%
\pgfpathclose%
\pgfusepath{stroke,fill}%
\end{pgfscope}%
\begin{pgfscope}%
\pgfpathrectangle{\pgfqpoint{0.211875in}{0.211875in}}{\pgfqpoint{1.313625in}{1.279725in}}%
\pgfusepath{clip}%
\pgfsetbuttcap%
\pgfsetroundjoin%
\definecolor{currentfill}{rgb}{0.121569,0.466667,0.705882}%
\pgfsetfillcolor{currentfill}%
\pgfsetlinewidth{1.003750pt}%
\definecolor{currentstroke}{rgb}{0.121569,0.466667,0.705882}%
\pgfsetstrokecolor{currentstroke}%
\pgfsetdash{}{0pt}%
\pgfpathmoveto{\pgfqpoint{1.099440in}{0.995383in}}%
\pgfpathcurveto{\pgfqpoint{1.105264in}{0.995383in}}{\pgfqpoint{1.110850in}{0.997697in}}{\pgfqpoint{1.114969in}{1.001815in}}%
\pgfpathcurveto{\pgfqpoint{1.119087in}{1.005933in}}{\pgfqpoint{1.121401in}{1.011520in}}{\pgfqpoint{1.121401in}{1.017343in}}%
\pgfpathcurveto{\pgfqpoint{1.121401in}{1.023167in}}{\pgfqpoint{1.119087in}{1.028754in}}{\pgfqpoint{1.114969in}{1.032872in}}%
\pgfpathcurveto{\pgfqpoint{1.110850in}{1.036990in}}{\pgfqpoint{1.105264in}{1.039304in}}{\pgfqpoint{1.099440in}{1.039304in}}%
\pgfpathcurveto{\pgfqpoint{1.093616in}{1.039304in}}{\pgfqpoint{1.088030in}{1.036990in}}{\pgfqpoint{1.083912in}{1.032872in}}%
\pgfpathcurveto{\pgfqpoint{1.079794in}{1.028754in}}{\pgfqpoint{1.077480in}{1.023167in}}{\pgfqpoint{1.077480in}{1.017343in}}%
\pgfpathcurveto{\pgfqpoint{1.077480in}{1.011520in}}{\pgfqpoint{1.079794in}{1.005933in}}{\pgfqpoint{1.083912in}{1.001815in}}%
\pgfpathcurveto{\pgfqpoint{1.088030in}{0.997697in}}{\pgfqpoint{1.093616in}{0.995383in}}{\pgfqpoint{1.099440in}{0.995383in}}%
\pgfpathclose%
\pgfusepath{stroke,fill}%
\end{pgfscope}%
\begin{pgfscope}%
\pgfpathrectangle{\pgfqpoint{0.211875in}{0.211875in}}{\pgfqpoint{1.313625in}{1.279725in}}%
\pgfusepath{clip}%
\pgfsetbuttcap%
\pgfsetroundjoin%
\definecolor{currentfill}{rgb}{0.121569,0.466667,0.705882}%
\pgfsetfillcolor{currentfill}%
\pgfsetlinewidth{1.003750pt}%
\definecolor{currentstroke}{rgb}{0.121569,0.466667,0.705882}%
\pgfsetstrokecolor{currentstroke}%
\pgfsetdash{}{0pt}%
\pgfpathmoveto{\pgfqpoint{1.097005in}{0.997404in}}%
\pgfpathcurveto{\pgfqpoint{1.102828in}{0.997404in}}{\pgfqpoint{1.108415in}{0.999718in}}{\pgfqpoint{1.112533in}{1.003836in}}%
\pgfpathcurveto{\pgfqpoint{1.116651in}{1.007954in}}{\pgfqpoint{1.118965in}{1.013540in}}{\pgfqpoint{1.118965in}{1.019364in}}%
\pgfpathcurveto{\pgfqpoint{1.118965in}{1.025188in}}{\pgfqpoint{1.116651in}{1.030774in}}{\pgfqpoint{1.112533in}{1.034892in}}%
\pgfpathcurveto{\pgfqpoint{1.108415in}{1.039010in}}{\pgfqpoint{1.102828in}{1.041324in}}{\pgfqpoint{1.097005in}{1.041324in}}%
\pgfpathcurveto{\pgfqpoint{1.091181in}{1.041324in}}{\pgfqpoint{1.085594in}{1.039010in}}{\pgfqpoint{1.081476in}{1.034892in}}%
\pgfpathcurveto{\pgfqpoint{1.077358in}{1.030774in}}{\pgfqpoint{1.075044in}{1.025188in}}{\pgfqpoint{1.075044in}{1.019364in}}%
\pgfpathcurveto{\pgfqpoint{1.075044in}{1.013540in}}{\pgfqpoint{1.077358in}{1.007954in}}{\pgfqpoint{1.081476in}{1.003836in}}%
\pgfpathcurveto{\pgfqpoint{1.085594in}{0.999718in}}{\pgfqpoint{1.091181in}{0.997404in}}{\pgfqpoint{1.097005in}{0.997404in}}%
\pgfpathclose%
\pgfusepath{stroke,fill}%
\end{pgfscope}%
\begin{pgfscope}%
\pgfpathrectangle{\pgfqpoint{0.211875in}{0.211875in}}{\pgfqpoint{1.313625in}{1.279725in}}%
\pgfusepath{clip}%
\pgfsetbuttcap%
\pgfsetroundjoin%
\definecolor{currentfill}{rgb}{0.121569,0.466667,0.705882}%
\pgfsetfillcolor{currentfill}%
\pgfsetlinewidth{1.003750pt}%
\definecolor{currentstroke}{rgb}{0.121569,0.466667,0.705882}%
\pgfsetstrokecolor{currentstroke}%
\pgfsetdash{}{0pt}%
\pgfpathmoveto{\pgfqpoint{0.758316in}{1.226042in}}%
\pgfpathcurveto{\pgfqpoint{0.764140in}{1.226042in}}{\pgfqpoint{0.769726in}{1.228355in}}{\pgfqpoint{0.773844in}{1.232474in}}%
\pgfpathcurveto{\pgfqpoint{0.777962in}{1.236592in}}{\pgfqpoint{0.780276in}{1.242178in}}{\pgfqpoint{0.780276in}{1.248002in}}%
\pgfpathcurveto{\pgfqpoint{0.780276in}{1.253826in}}{\pgfqpoint{0.777962in}{1.259412in}}{\pgfqpoint{0.773844in}{1.263530in}}%
\pgfpathcurveto{\pgfqpoint{0.769726in}{1.267648in}}{\pgfqpoint{0.764140in}{1.269962in}}{\pgfqpoint{0.758316in}{1.269962in}}%
\pgfpathcurveto{\pgfqpoint{0.752492in}{1.269962in}}{\pgfqpoint{0.746906in}{1.267648in}}{\pgfqpoint{0.742787in}{1.263530in}}%
\pgfpathcurveto{\pgfqpoint{0.738669in}{1.259412in}}{\pgfqpoint{0.736355in}{1.253826in}}{\pgfqpoint{0.736355in}{1.248002in}}%
\pgfpathcurveto{\pgfqpoint{0.736355in}{1.242178in}}{\pgfqpoint{0.738669in}{1.236592in}}{\pgfqpoint{0.742787in}{1.232474in}}%
\pgfpathcurveto{\pgfqpoint{0.746906in}{1.228355in}}{\pgfqpoint{0.752492in}{1.226042in}}{\pgfqpoint{0.758316in}{1.226042in}}%
\pgfpathclose%
\pgfusepath{stroke,fill}%
\end{pgfscope}%
\begin{pgfscope}%
\pgfpathrectangle{\pgfqpoint{0.211875in}{0.211875in}}{\pgfqpoint{1.313625in}{1.279725in}}%
\pgfusepath{clip}%
\pgfsetbuttcap%
\pgfsetroundjoin%
\definecolor{currentfill}{rgb}{0.121569,0.466667,0.705882}%
\pgfsetfillcolor{currentfill}%
\pgfsetlinewidth{1.003750pt}%
\definecolor{currentstroke}{rgb}{0.121569,0.466667,0.705882}%
\pgfsetstrokecolor{currentstroke}%
\pgfsetdash{}{0pt}%
\pgfpathmoveto{\pgfqpoint{1.089749in}{1.004974in}}%
\pgfpathcurveto{\pgfqpoint{1.095573in}{1.004974in}}{\pgfqpoint{1.101159in}{1.007288in}}{\pgfqpoint{1.105277in}{1.011406in}}%
\pgfpathcurveto{\pgfqpoint{1.109395in}{1.015524in}}{\pgfqpoint{1.111709in}{1.021110in}}{\pgfqpoint{1.111709in}{1.026934in}}%
\pgfpathcurveto{\pgfqpoint{1.111709in}{1.032758in}}{\pgfqpoint{1.109395in}{1.038344in}}{\pgfqpoint{1.105277in}{1.042462in}}%
\pgfpathcurveto{\pgfqpoint{1.101159in}{1.046581in}}{\pgfqpoint{1.095573in}{1.048894in}}{\pgfqpoint{1.089749in}{1.048894in}}%
\pgfpathcurveto{\pgfqpoint{1.083925in}{1.048894in}}{\pgfqpoint{1.078338in}{1.046581in}}{\pgfqpoint{1.074220in}{1.042462in}}%
\pgfpathcurveto{\pgfqpoint{1.070102in}{1.038344in}}{\pgfqpoint{1.067788in}{1.032758in}}{\pgfqpoint{1.067788in}{1.026934in}}%
\pgfpathcurveto{\pgfqpoint{1.067788in}{1.021110in}}{\pgfqpoint{1.070102in}{1.015524in}}{\pgfqpoint{1.074220in}{1.011406in}}%
\pgfpathcurveto{\pgfqpoint{1.078338in}{1.007288in}}{\pgfqpoint{1.083925in}{1.004974in}}{\pgfqpoint{1.089749in}{1.004974in}}%
\pgfpathclose%
\pgfusepath{stroke,fill}%
\end{pgfscope}%
\begin{pgfscope}%
\pgfpathrectangle{\pgfqpoint{0.211875in}{0.211875in}}{\pgfqpoint{1.313625in}{1.279725in}}%
\pgfusepath{clip}%
\pgfsetbuttcap%
\pgfsetroundjoin%
\definecolor{currentfill}{rgb}{0.121569,0.466667,0.705882}%
\pgfsetfillcolor{currentfill}%
\pgfsetlinewidth{1.003750pt}%
\definecolor{currentstroke}{rgb}{0.121569,0.466667,0.705882}%
\pgfsetstrokecolor{currentstroke}%
\pgfsetdash{}{0pt}%
\pgfpathmoveto{\pgfqpoint{0.712758in}{1.056827in}}%
\pgfpathcurveto{\pgfqpoint{0.718582in}{1.056827in}}{\pgfqpoint{0.724168in}{1.059141in}}{\pgfqpoint{0.728286in}{1.063259in}}%
\pgfpathcurveto{\pgfqpoint{0.732404in}{1.067377in}}{\pgfqpoint{0.734718in}{1.072963in}}{\pgfqpoint{0.734718in}{1.078787in}}%
\pgfpathcurveto{\pgfqpoint{0.734718in}{1.084611in}}{\pgfqpoint{0.732404in}{1.090197in}}{\pgfqpoint{0.728286in}{1.094316in}}%
\pgfpathcurveto{\pgfqpoint{0.724168in}{1.098434in}}{\pgfqpoint{0.718582in}{1.100748in}}{\pgfqpoint{0.712758in}{1.100748in}}%
\pgfpathcurveto{\pgfqpoint{0.706934in}{1.100748in}}{\pgfqpoint{0.701348in}{1.098434in}}{\pgfqpoint{0.697230in}{1.094316in}}%
\pgfpathcurveto{\pgfqpoint{0.693112in}{1.090197in}}{\pgfqpoint{0.690798in}{1.084611in}}{\pgfqpoint{0.690798in}{1.078787in}}%
\pgfpathcurveto{\pgfqpoint{0.690798in}{1.072963in}}{\pgfqpoint{0.693112in}{1.067377in}}{\pgfqpoint{0.697230in}{1.063259in}}%
\pgfpathcurveto{\pgfqpoint{0.701348in}{1.059141in}}{\pgfqpoint{0.706934in}{1.056827in}}{\pgfqpoint{0.712758in}{1.056827in}}%
\pgfpathclose%
\pgfusepath{stroke,fill}%
\end{pgfscope}%
\begin{pgfscope}%
\pgfpathrectangle{\pgfqpoint{0.211875in}{0.211875in}}{\pgfqpoint{1.313625in}{1.279725in}}%
\pgfusepath{clip}%
\pgfsetbuttcap%
\pgfsetroundjoin%
\definecolor{currentfill}{rgb}{0.121569,0.466667,0.705882}%
\pgfsetfillcolor{currentfill}%
\pgfsetlinewidth{1.003750pt}%
\definecolor{currentstroke}{rgb}{0.121569,0.466667,0.705882}%
\pgfsetstrokecolor{currentstroke}%
\pgfsetdash{}{0pt}%
\pgfpathmoveto{\pgfqpoint{1.055746in}{1.016362in}}%
\pgfpathcurveto{\pgfqpoint{1.061570in}{1.016362in}}{\pgfqpoint{1.067156in}{1.018676in}}{\pgfqpoint{1.071274in}{1.022794in}}%
\pgfpathcurveto{\pgfqpoint{1.075393in}{1.026912in}}{\pgfqpoint{1.077706in}{1.032498in}}{\pgfqpoint{1.077706in}{1.038322in}}%
\pgfpathcurveto{\pgfqpoint{1.077706in}{1.044146in}}{\pgfqpoint{1.075393in}{1.049732in}}{\pgfqpoint{1.071274in}{1.053851in}}%
\pgfpathcurveto{\pgfqpoint{1.067156in}{1.057969in}}{\pgfqpoint{1.061570in}{1.060283in}}{\pgfqpoint{1.055746in}{1.060283in}}%
\pgfpathcurveto{\pgfqpoint{1.049922in}{1.060283in}}{\pgfqpoint{1.044336in}{1.057969in}}{\pgfqpoint{1.040218in}{1.053851in}}%
\pgfpathcurveto{\pgfqpoint{1.036100in}{1.049732in}}{\pgfqpoint{1.033786in}{1.044146in}}{\pgfqpoint{1.033786in}{1.038322in}}%
\pgfpathcurveto{\pgfqpoint{1.033786in}{1.032498in}}{\pgfqpoint{1.036100in}{1.026912in}}{\pgfqpoint{1.040218in}{1.022794in}}%
\pgfpathcurveto{\pgfqpoint{1.044336in}{1.018676in}}{\pgfqpoint{1.049922in}{1.016362in}}{\pgfqpoint{1.055746in}{1.016362in}}%
\pgfpathclose%
\pgfusepath{stroke,fill}%
\end{pgfscope}%
\begin{pgfscope}%
\pgfpathrectangle{\pgfqpoint{0.211875in}{0.211875in}}{\pgfqpoint{1.313625in}{1.279725in}}%
\pgfusepath{clip}%
\pgfsetbuttcap%
\pgfsetroundjoin%
\definecolor{currentfill}{rgb}{0.121569,0.466667,0.705882}%
\pgfsetfillcolor{currentfill}%
\pgfsetlinewidth{1.003750pt}%
\definecolor{currentstroke}{rgb}{0.121569,0.466667,0.705882}%
\pgfsetstrokecolor{currentstroke}%
\pgfsetdash{}{0pt}%
\pgfpathmoveto{\pgfqpoint{1.072134in}{1.006575in}}%
\pgfpathcurveto{\pgfqpoint{1.077958in}{1.006575in}}{\pgfqpoint{1.083544in}{1.008889in}}{\pgfqpoint{1.087662in}{1.013007in}}%
\pgfpathcurveto{\pgfqpoint{1.091780in}{1.017125in}}{\pgfqpoint{1.094094in}{1.022711in}}{\pgfqpoint{1.094094in}{1.028535in}}%
\pgfpathcurveto{\pgfqpoint{1.094094in}{1.034359in}}{\pgfqpoint{1.091780in}{1.039945in}}{\pgfqpoint{1.087662in}{1.044064in}}%
\pgfpathcurveto{\pgfqpoint{1.083544in}{1.048182in}}{\pgfqpoint{1.077958in}{1.050496in}}{\pgfqpoint{1.072134in}{1.050496in}}%
\pgfpathcurveto{\pgfqpoint{1.066310in}{1.050496in}}{\pgfqpoint{1.060724in}{1.048182in}}{\pgfqpoint{1.056605in}{1.044064in}}%
\pgfpathcurveto{\pgfqpoint{1.052487in}{1.039945in}}{\pgfqpoint{1.050173in}{1.034359in}}{\pgfqpoint{1.050173in}{1.028535in}}%
\pgfpathcurveto{\pgfqpoint{1.050173in}{1.022711in}}{\pgfqpoint{1.052487in}{1.017125in}}{\pgfqpoint{1.056605in}{1.013007in}}%
\pgfpathcurveto{\pgfqpoint{1.060724in}{1.008889in}}{\pgfqpoint{1.066310in}{1.006575in}}{\pgfqpoint{1.072134in}{1.006575in}}%
\pgfpathclose%
\pgfusepath{stroke,fill}%
\end{pgfscope}%
\begin{pgfscope}%
\pgfpathrectangle{\pgfqpoint{0.211875in}{0.211875in}}{\pgfqpoint{1.313625in}{1.279725in}}%
\pgfusepath{clip}%
\pgfsetbuttcap%
\pgfsetroundjoin%
\definecolor{currentfill}{rgb}{0.121569,0.466667,0.705882}%
\pgfsetfillcolor{currentfill}%
\pgfsetlinewidth{1.003750pt}%
\definecolor{currentstroke}{rgb}{0.121569,0.466667,0.705882}%
\pgfsetstrokecolor{currentstroke}%
\pgfsetdash{}{0pt}%
\pgfpathmoveto{\pgfqpoint{0.642815in}{1.046751in}}%
\pgfpathcurveto{\pgfqpoint{0.648639in}{1.046751in}}{\pgfqpoint{0.654225in}{1.049065in}}{\pgfqpoint{0.658343in}{1.053183in}}%
\pgfpathcurveto{\pgfqpoint{0.662461in}{1.057301in}}{\pgfqpoint{0.664775in}{1.062887in}}{\pgfqpoint{0.664775in}{1.068711in}}%
\pgfpathcurveto{\pgfqpoint{0.664775in}{1.074535in}}{\pgfqpoint{0.662461in}{1.080121in}}{\pgfqpoint{0.658343in}{1.084239in}}%
\pgfpathcurveto{\pgfqpoint{0.654225in}{1.088357in}}{\pgfqpoint{0.648639in}{1.090671in}}{\pgfqpoint{0.642815in}{1.090671in}}%
\pgfpathcurveto{\pgfqpoint{0.636991in}{1.090671in}}{\pgfqpoint{0.631405in}{1.088357in}}{\pgfqpoint{0.627287in}{1.084239in}}%
\pgfpathcurveto{\pgfqpoint{0.623168in}{1.080121in}}{\pgfqpoint{0.620855in}{1.074535in}}{\pgfqpoint{0.620855in}{1.068711in}}%
\pgfpathcurveto{\pgfqpoint{0.620855in}{1.062887in}}{\pgfqpoint{0.623168in}{1.057301in}}{\pgfqpoint{0.627287in}{1.053183in}}%
\pgfpathcurveto{\pgfqpoint{0.631405in}{1.049065in}}{\pgfqpoint{0.636991in}{1.046751in}}{\pgfqpoint{0.642815in}{1.046751in}}%
\pgfpathclose%
\pgfusepath{stroke,fill}%
\end{pgfscope}%
\begin{pgfscope}%
\pgfpathrectangle{\pgfqpoint{0.211875in}{0.211875in}}{\pgfqpoint{1.313625in}{1.279725in}}%
\pgfusepath{clip}%
\pgfsetbuttcap%
\pgfsetroundjoin%
\definecolor{currentfill}{rgb}{0.121569,0.466667,0.705882}%
\pgfsetfillcolor{currentfill}%
\pgfsetlinewidth{1.003750pt}%
\definecolor{currentstroke}{rgb}{0.121569,0.466667,0.705882}%
\pgfsetstrokecolor{currentstroke}%
\pgfsetdash{}{0pt}%
\pgfpathmoveto{\pgfqpoint{1.070340in}{1.002578in}}%
\pgfpathcurveto{\pgfqpoint{1.076164in}{1.002578in}}{\pgfqpoint{1.081750in}{1.004892in}}{\pgfqpoint{1.085868in}{1.009010in}}%
\pgfpathcurveto{\pgfqpoint{1.089986in}{1.013128in}}{\pgfqpoint{1.092300in}{1.018715in}}{\pgfqpoint{1.092300in}{1.024539in}}%
\pgfpathcurveto{\pgfqpoint{1.092300in}{1.030362in}}{\pgfqpoint{1.089986in}{1.035949in}}{\pgfqpoint{1.085868in}{1.040067in}}%
\pgfpathcurveto{\pgfqpoint{1.081750in}{1.044185in}}{\pgfqpoint{1.076164in}{1.046499in}}{\pgfqpoint{1.070340in}{1.046499in}}%
\pgfpathcurveto{\pgfqpoint{1.064516in}{1.046499in}}{\pgfqpoint{1.058930in}{1.044185in}}{\pgfqpoint{1.054812in}{1.040067in}}%
\pgfpathcurveto{\pgfqpoint{1.050693in}{1.035949in}}{\pgfqpoint{1.048379in}{1.030362in}}{\pgfqpoint{1.048379in}{1.024539in}}%
\pgfpathcurveto{\pgfqpoint{1.048379in}{1.018715in}}{\pgfqpoint{1.050693in}{1.013128in}}{\pgfqpoint{1.054812in}{1.009010in}}%
\pgfpathcurveto{\pgfqpoint{1.058930in}{1.004892in}}{\pgfqpoint{1.064516in}{1.002578in}}{\pgfqpoint{1.070340in}{1.002578in}}%
\pgfpathclose%
\pgfusepath{stroke,fill}%
\end{pgfscope}%
\begin{pgfscope}%
\pgfpathrectangle{\pgfqpoint{0.211875in}{0.211875in}}{\pgfqpoint{1.313625in}{1.279725in}}%
\pgfusepath{clip}%
\pgfsetbuttcap%
\pgfsetroundjoin%
\definecolor{currentfill}{rgb}{0.121569,0.466667,0.705882}%
\pgfsetfillcolor{currentfill}%
\pgfsetlinewidth{1.003750pt}%
\definecolor{currentstroke}{rgb}{0.121569,0.466667,0.705882}%
\pgfsetstrokecolor{currentstroke}%
\pgfsetdash{}{0pt}%
\pgfpathmoveto{\pgfqpoint{1.091470in}{0.994922in}}%
\pgfpathcurveto{\pgfqpoint{1.097294in}{0.994922in}}{\pgfqpoint{1.102880in}{0.997236in}}{\pgfqpoint{1.106998in}{1.001354in}}%
\pgfpathcurveto{\pgfqpoint{1.111116in}{1.005473in}}{\pgfqpoint{1.113430in}{1.011059in}}{\pgfqpoint{1.113430in}{1.016883in}}%
\pgfpathcurveto{\pgfqpoint{1.113430in}{1.022707in}}{\pgfqpoint{1.111116in}{1.028293in}}{\pgfqpoint{1.106998in}{1.032411in}}%
\pgfpathcurveto{\pgfqpoint{1.102880in}{1.036529in}}{\pgfqpoint{1.097294in}{1.038843in}}{\pgfqpoint{1.091470in}{1.038843in}}%
\pgfpathcurveto{\pgfqpoint{1.085646in}{1.038843in}}{\pgfqpoint{1.080060in}{1.036529in}}{\pgfqpoint{1.075942in}{1.032411in}}%
\pgfpathcurveto{\pgfqpoint{1.071824in}{1.028293in}}{\pgfqpoint{1.069510in}{1.022707in}}{\pgfqpoint{1.069510in}{1.016883in}}%
\pgfpathcurveto{\pgfqpoint{1.069510in}{1.011059in}}{\pgfqpoint{1.071824in}{1.005473in}}{\pgfqpoint{1.075942in}{1.001354in}}%
\pgfpathcurveto{\pgfqpoint{1.080060in}{0.997236in}}{\pgfqpoint{1.085646in}{0.994922in}}{\pgfqpoint{1.091470in}{0.994922in}}%
\pgfpathclose%
\pgfusepath{stroke,fill}%
\end{pgfscope}%
\begin{pgfscope}%
\pgfpathrectangle{\pgfqpoint{0.211875in}{0.211875in}}{\pgfqpoint{1.313625in}{1.279725in}}%
\pgfusepath{clip}%
\pgfsetbuttcap%
\pgfsetroundjoin%
\definecolor{currentfill}{rgb}{0.121569,0.466667,0.705882}%
\pgfsetfillcolor{currentfill}%
\pgfsetlinewidth{1.003750pt}%
\definecolor{currentstroke}{rgb}{0.121569,0.466667,0.705882}%
\pgfsetstrokecolor{currentstroke}%
\pgfsetdash{}{0pt}%
\pgfpathmoveto{\pgfqpoint{0.988593in}{0.977228in}}%
\pgfpathcurveto{\pgfqpoint{0.994417in}{0.977228in}}{\pgfqpoint{1.000003in}{0.979542in}}{\pgfqpoint{1.004122in}{0.983660in}}%
\pgfpathcurveto{\pgfqpoint{1.008240in}{0.987778in}}{\pgfqpoint{1.010554in}{0.993365in}}{\pgfqpoint{1.010554in}{0.999189in}}%
\pgfpathcurveto{\pgfqpoint{1.010554in}{1.005012in}}{\pgfqpoint{1.008240in}{1.010599in}}{\pgfqpoint{1.004122in}{1.014717in}}%
\pgfpathcurveto{\pgfqpoint{1.000003in}{1.018835in}}{\pgfqpoint{0.994417in}{1.021149in}}{\pgfqpoint{0.988593in}{1.021149in}}%
\pgfpathcurveto{\pgfqpoint{0.982769in}{1.021149in}}{\pgfqpoint{0.977183in}{1.018835in}}{\pgfqpoint{0.973065in}{1.014717in}}%
\pgfpathcurveto{\pgfqpoint{0.968947in}{1.010599in}}{\pgfqpoint{0.966633in}{1.005012in}}{\pgfqpoint{0.966633in}{0.999189in}}%
\pgfpathcurveto{\pgfqpoint{0.966633in}{0.993365in}}{\pgfqpoint{0.968947in}{0.987778in}}{\pgfqpoint{0.973065in}{0.983660in}}%
\pgfpathcurveto{\pgfqpoint{0.977183in}{0.979542in}}{\pgfqpoint{0.982769in}{0.977228in}}{\pgfqpoint{0.988593in}{0.977228in}}%
\pgfpathclose%
\pgfusepath{stroke,fill}%
\end{pgfscope}%
\begin{pgfscope}%
\pgfpathrectangle{\pgfqpoint{0.211875in}{0.211875in}}{\pgfqpoint{1.313625in}{1.279725in}}%
\pgfusepath{clip}%
\pgfsetbuttcap%
\pgfsetroundjoin%
\definecolor{currentfill}{rgb}{0.121569,0.466667,0.705882}%
\pgfsetfillcolor{currentfill}%
\pgfsetlinewidth{1.003750pt}%
\definecolor{currentstroke}{rgb}{0.121569,0.466667,0.705882}%
\pgfsetstrokecolor{currentstroke}%
\pgfsetdash{}{0pt}%
\pgfpathmoveto{\pgfqpoint{0.711619in}{0.979862in}}%
\pgfpathcurveto{\pgfqpoint{0.717443in}{0.979862in}}{\pgfqpoint{0.723029in}{0.982176in}}{\pgfqpoint{0.727147in}{0.986294in}}%
\pgfpathcurveto{\pgfqpoint{0.731265in}{0.990413in}}{\pgfqpoint{0.733579in}{0.995999in}}{\pgfqpoint{0.733579in}{1.001823in}}%
\pgfpathcurveto{\pgfqpoint{0.733579in}{1.007647in}}{\pgfqpoint{0.731265in}{1.013233in}}{\pgfqpoint{0.727147in}{1.017351in}}%
\pgfpathcurveto{\pgfqpoint{0.723029in}{1.021469in}}{\pgfqpoint{0.717443in}{1.023783in}}{\pgfqpoint{0.711619in}{1.023783in}}%
\pgfpathcurveto{\pgfqpoint{0.705795in}{1.023783in}}{\pgfqpoint{0.700209in}{1.021469in}}{\pgfqpoint{0.696091in}{1.017351in}}%
\pgfpathcurveto{\pgfqpoint{0.691972in}{1.013233in}}{\pgfqpoint{0.689659in}{1.007647in}}{\pgfqpoint{0.689659in}{1.001823in}}%
\pgfpathcurveto{\pgfqpoint{0.689659in}{0.995999in}}{\pgfqpoint{0.691972in}{0.990413in}}{\pgfqpoint{0.696091in}{0.986294in}}%
\pgfpathcurveto{\pgfqpoint{0.700209in}{0.982176in}}{\pgfqpoint{0.705795in}{0.979862in}}{\pgfqpoint{0.711619in}{0.979862in}}%
\pgfpathclose%
\pgfusepath{stroke,fill}%
\end{pgfscope}%
\begin{pgfscope}%
\pgfpathrectangle{\pgfqpoint{0.211875in}{0.211875in}}{\pgfqpoint{1.313625in}{1.279725in}}%
\pgfusepath{clip}%
\pgfsetbuttcap%
\pgfsetroundjoin%
\definecolor{currentfill}{rgb}{0.121569,0.466667,0.705882}%
\pgfsetfillcolor{currentfill}%
\pgfsetlinewidth{1.003750pt}%
\definecolor{currentstroke}{rgb}{0.121569,0.466667,0.705882}%
\pgfsetstrokecolor{currentstroke}%
\pgfsetdash{}{0pt}%
\pgfpathmoveto{\pgfqpoint{0.697818in}{1.125633in}}%
\pgfpathcurveto{\pgfqpoint{0.703642in}{1.125633in}}{\pgfqpoint{0.709228in}{1.127946in}}{\pgfqpoint{0.713346in}{1.132065in}}%
\pgfpathcurveto{\pgfqpoint{0.717464in}{1.136183in}}{\pgfqpoint{0.719778in}{1.141769in}}{\pgfqpoint{0.719778in}{1.147593in}}%
\pgfpathcurveto{\pgfqpoint{0.719778in}{1.153417in}}{\pgfqpoint{0.717464in}{1.159003in}}{\pgfqpoint{0.713346in}{1.163121in}}%
\pgfpathcurveto{\pgfqpoint{0.709228in}{1.167239in}}{\pgfqpoint{0.703642in}{1.169553in}}{\pgfqpoint{0.697818in}{1.169553in}}%
\pgfpathcurveto{\pgfqpoint{0.691994in}{1.169553in}}{\pgfqpoint{0.686407in}{1.167239in}}{\pgfqpoint{0.682289in}{1.163121in}}%
\pgfpathcurveto{\pgfqpoint{0.678171in}{1.159003in}}{\pgfqpoint{0.675857in}{1.153417in}}{\pgfqpoint{0.675857in}{1.147593in}}%
\pgfpathcurveto{\pgfqpoint{0.675857in}{1.141769in}}{\pgfqpoint{0.678171in}{1.136183in}}{\pgfqpoint{0.682289in}{1.132065in}}%
\pgfpathcurveto{\pgfqpoint{0.686407in}{1.127946in}}{\pgfqpoint{0.691994in}{1.125633in}}{\pgfqpoint{0.697818in}{1.125633in}}%
\pgfpathclose%
\pgfusepath{stroke,fill}%
\end{pgfscope}%
\begin{pgfscope}%
\pgfpathrectangle{\pgfqpoint{0.211875in}{0.211875in}}{\pgfqpoint{1.313625in}{1.279725in}}%
\pgfusepath{clip}%
\pgfsetbuttcap%
\pgfsetroundjoin%
\definecolor{currentfill}{rgb}{0.121569,0.466667,0.705882}%
\pgfsetfillcolor{currentfill}%
\pgfsetlinewidth{1.003750pt}%
\definecolor{currentstroke}{rgb}{0.121569,0.466667,0.705882}%
\pgfsetstrokecolor{currentstroke}%
\pgfsetdash{}{0pt}%
\pgfpathmoveto{\pgfqpoint{1.083328in}{1.001059in}}%
\pgfpathcurveto{\pgfqpoint{1.089152in}{1.001059in}}{\pgfqpoint{1.094738in}{1.003373in}}{\pgfqpoint{1.098856in}{1.007491in}}%
\pgfpathcurveto{\pgfqpoint{1.102974in}{1.011609in}}{\pgfqpoint{1.105288in}{1.017196in}}{\pgfqpoint{1.105288in}{1.023019in}}%
\pgfpathcurveto{\pgfqpoint{1.105288in}{1.028843in}}{\pgfqpoint{1.102974in}{1.034430in}}{\pgfqpoint{1.098856in}{1.038548in}}%
\pgfpathcurveto{\pgfqpoint{1.094738in}{1.042666in}}{\pgfqpoint{1.089152in}{1.044980in}}{\pgfqpoint{1.083328in}{1.044980in}}%
\pgfpathcurveto{\pgfqpoint{1.077504in}{1.044980in}}{\pgfqpoint{1.071918in}{1.042666in}}{\pgfqpoint{1.067799in}{1.038548in}}%
\pgfpathcurveto{\pgfqpoint{1.063681in}{1.034430in}}{\pgfqpoint{1.061367in}{1.028843in}}{\pgfqpoint{1.061367in}{1.023019in}}%
\pgfpathcurveto{\pgfqpoint{1.061367in}{1.017196in}}{\pgfqpoint{1.063681in}{1.011609in}}{\pgfqpoint{1.067799in}{1.007491in}}%
\pgfpathcurveto{\pgfqpoint{1.071918in}{1.003373in}}{\pgfqpoint{1.077504in}{1.001059in}}{\pgfqpoint{1.083328in}{1.001059in}}%
\pgfpathclose%
\pgfusepath{stroke,fill}%
\end{pgfscope}%
\begin{pgfscope}%
\pgfpathrectangle{\pgfqpoint{0.211875in}{0.211875in}}{\pgfqpoint{1.313625in}{1.279725in}}%
\pgfusepath{clip}%
\pgfsetbuttcap%
\pgfsetroundjoin%
\definecolor{currentfill}{rgb}{0.121569,0.466667,0.705882}%
\pgfsetfillcolor{currentfill}%
\pgfsetlinewidth{1.003750pt}%
\definecolor{currentstroke}{rgb}{0.121569,0.466667,0.705882}%
\pgfsetstrokecolor{currentstroke}%
\pgfsetdash{}{0pt}%
\pgfpathmoveto{\pgfqpoint{1.091043in}{0.995814in}}%
\pgfpathcurveto{\pgfqpoint{1.096867in}{0.995814in}}{\pgfqpoint{1.102453in}{0.998128in}}{\pgfqpoint{1.106571in}{1.002246in}}%
\pgfpathcurveto{\pgfqpoint{1.110689in}{1.006364in}}{\pgfqpoint{1.113003in}{1.011950in}}{\pgfqpoint{1.113003in}{1.017774in}}%
\pgfpathcurveto{\pgfqpoint{1.113003in}{1.023598in}}{\pgfqpoint{1.110689in}{1.029184in}}{\pgfqpoint{1.106571in}{1.033302in}}%
\pgfpathcurveto{\pgfqpoint{1.102453in}{1.037420in}}{\pgfqpoint{1.096867in}{1.039734in}}{\pgfqpoint{1.091043in}{1.039734in}}%
\pgfpathcurveto{\pgfqpoint{1.085219in}{1.039734in}}{\pgfqpoint{1.079633in}{1.037420in}}{\pgfqpoint{1.075515in}{1.033302in}}%
\pgfpathcurveto{\pgfqpoint{1.071397in}{1.029184in}}{\pgfqpoint{1.069083in}{1.023598in}}{\pgfqpoint{1.069083in}{1.017774in}}%
\pgfpathcurveto{\pgfqpoint{1.069083in}{1.011950in}}{\pgfqpoint{1.071397in}{1.006364in}}{\pgfqpoint{1.075515in}{1.002246in}}%
\pgfpathcurveto{\pgfqpoint{1.079633in}{0.998128in}}{\pgfqpoint{1.085219in}{0.995814in}}{\pgfqpoint{1.091043in}{0.995814in}}%
\pgfpathclose%
\pgfusepath{stroke,fill}%
\end{pgfscope}%
\begin{pgfscope}%
\pgfpathrectangle{\pgfqpoint{0.211875in}{0.211875in}}{\pgfqpoint{1.313625in}{1.279725in}}%
\pgfusepath{clip}%
\pgfsetbuttcap%
\pgfsetroundjoin%
\definecolor{currentfill}{rgb}{0.121569,0.466667,0.705882}%
\pgfsetfillcolor{currentfill}%
\pgfsetlinewidth{1.003750pt}%
\definecolor{currentstroke}{rgb}{0.121569,0.466667,0.705882}%
\pgfsetstrokecolor{currentstroke}%
\pgfsetdash{}{0pt}%
\pgfpathmoveto{\pgfqpoint{1.092518in}{0.994152in}}%
\pgfpathcurveto{\pgfqpoint{1.098342in}{0.994152in}}{\pgfqpoint{1.103928in}{0.996466in}}{\pgfqpoint{1.108046in}{1.000584in}}%
\pgfpathcurveto{\pgfqpoint{1.112165in}{1.004702in}}{\pgfqpoint{1.114478in}{1.010288in}}{\pgfqpoint{1.114478in}{1.016112in}}%
\pgfpathcurveto{\pgfqpoint{1.114478in}{1.021936in}}{\pgfqpoint{1.112165in}{1.027522in}}{\pgfqpoint{1.108046in}{1.031640in}}%
\pgfpathcurveto{\pgfqpoint{1.103928in}{1.035758in}}{\pgfqpoint{1.098342in}{1.038072in}}{\pgfqpoint{1.092518in}{1.038072in}}%
\pgfpathcurveto{\pgfqpoint{1.086694in}{1.038072in}}{\pgfqpoint{1.081108in}{1.035758in}}{\pgfqpoint{1.076990in}{1.031640in}}%
\pgfpathcurveto{\pgfqpoint{1.072872in}{1.027522in}}{\pgfqpoint{1.070558in}{1.021936in}}{\pgfqpoint{1.070558in}{1.016112in}}%
\pgfpathcurveto{\pgfqpoint{1.070558in}{1.010288in}}{\pgfqpoint{1.072872in}{1.004702in}}{\pgfqpoint{1.076990in}{1.000584in}}%
\pgfpathcurveto{\pgfqpoint{1.081108in}{0.996466in}}{\pgfqpoint{1.086694in}{0.994152in}}{\pgfqpoint{1.092518in}{0.994152in}}%
\pgfpathclose%
\pgfusepath{stroke,fill}%
\end{pgfscope}%
\begin{pgfscope}%
\pgfpathrectangle{\pgfqpoint{0.211875in}{0.211875in}}{\pgfqpoint{1.313625in}{1.279725in}}%
\pgfusepath{clip}%
\pgfsetbuttcap%
\pgfsetroundjoin%
\definecolor{currentfill}{rgb}{0.121569,0.466667,0.705882}%
\pgfsetfillcolor{currentfill}%
\pgfsetlinewidth{1.003750pt}%
\definecolor{currentstroke}{rgb}{0.121569,0.466667,0.705882}%
\pgfsetstrokecolor{currentstroke}%
\pgfsetdash{}{0pt}%
\pgfpathmoveto{\pgfqpoint{1.364791in}{0.663190in}}%
\pgfpathcurveto{\pgfqpoint{1.370615in}{0.663190in}}{\pgfqpoint{1.376201in}{0.665504in}}{\pgfqpoint{1.380320in}{0.669622in}}%
\pgfpathcurveto{\pgfqpoint{1.384438in}{0.673741in}}{\pgfqpoint{1.386752in}{0.679327in}}{\pgfqpoint{1.386752in}{0.685151in}}%
\pgfpathcurveto{\pgfqpoint{1.386752in}{0.690975in}}{\pgfqpoint{1.384438in}{0.696561in}}{\pgfqpoint{1.380320in}{0.700679in}}%
\pgfpathcurveto{\pgfqpoint{1.376201in}{0.704797in}}{\pgfqpoint{1.370615in}{0.707111in}}{\pgfqpoint{1.364791in}{0.707111in}}%
\pgfpathcurveto{\pgfqpoint{1.358967in}{0.707111in}}{\pgfqpoint{1.353381in}{0.704797in}}{\pgfqpoint{1.349263in}{0.700679in}}%
\pgfpathcurveto{\pgfqpoint{1.345145in}{0.696561in}}{\pgfqpoint{1.342831in}{0.690975in}}{\pgfqpoint{1.342831in}{0.685151in}}%
\pgfpathcurveto{\pgfqpoint{1.342831in}{0.679327in}}{\pgfqpoint{1.345145in}{0.673741in}}{\pgfqpoint{1.349263in}{0.669622in}}%
\pgfpathcurveto{\pgfqpoint{1.353381in}{0.665504in}}{\pgfqpoint{1.358967in}{0.663190in}}{\pgfqpoint{1.364791in}{0.663190in}}%
\pgfpathclose%
\pgfusepath{stroke,fill}%
\end{pgfscope}%
\begin{pgfscope}%
\pgfpathrectangle{\pgfqpoint{0.211875in}{0.211875in}}{\pgfqpoint{1.313625in}{1.279725in}}%
\pgfusepath{clip}%
\pgfsetbuttcap%
\pgfsetroundjoin%
\definecolor{currentfill}{rgb}{0.121569,0.466667,0.705882}%
\pgfsetfillcolor{currentfill}%
\pgfsetlinewidth{1.003750pt}%
\definecolor{currentstroke}{rgb}{0.121569,0.466667,0.705882}%
\pgfsetstrokecolor{currentstroke}%
\pgfsetdash{}{0pt}%
\pgfpathmoveto{\pgfqpoint{1.391870in}{0.570725in}}%
\pgfpathcurveto{\pgfqpoint{1.397694in}{0.570725in}}{\pgfqpoint{1.403280in}{0.573039in}}{\pgfqpoint{1.407399in}{0.577157in}}%
\pgfpathcurveto{\pgfqpoint{1.411517in}{0.581275in}}{\pgfqpoint{1.413831in}{0.586861in}}{\pgfqpoint{1.413831in}{0.592685in}}%
\pgfpathcurveto{\pgfqpoint{1.413831in}{0.598509in}}{\pgfqpoint{1.411517in}{0.604095in}}{\pgfqpoint{1.407399in}{0.608214in}}%
\pgfpathcurveto{\pgfqpoint{1.403280in}{0.612332in}}{\pgfqpoint{1.397694in}{0.614646in}}{\pgfqpoint{1.391870in}{0.614646in}}%
\pgfpathcurveto{\pgfqpoint{1.386046in}{0.614646in}}{\pgfqpoint{1.380460in}{0.612332in}}{\pgfqpoint{1.376342in}{0.608214in}}%
\pgfpathcurveto{\pgfqpoint{1.372224in}{0.604095in}}{\pgfqpoint{1.369910in}{0.598509in}}{\pgfqpoint{1.369910in}{0.592685in}}%
\pgfpathcurveto{\pgfqpoint{1.369910in}{0.586861in}}{\pgfqpoint{1.372224in}{0.581275in}}{\pgfqpoint{1.376342in}{0.577157in}}%
\pgfpathcurveto{\pgfqpoint{1.380460in}{0.573039in}}{\pgfqpoint{1.386046in}{0.570725in}}{\pgfqpoint{1.391870in}{0.570725in}}%
\pgfpathclose%
\pgfusepath{stroke,fill}%
\end{pgfscope}%
\begin{pgfscope}%
\pgfpathrectangle{\pgfqpoint{0.211875in}{0.211875in}}{\pgfqpoint{1.313625in}{1.279725in}}%
\pgfusepath{clip}%
\pgfsetbuttcap%
\pgfsetroundjoin%
\definecolor{currentfill}{rgb}{0.121569,0.466667,0.705882}%
\pgfsetfillcolor{currentfill}%
\pgfsetlinewidth{1.003750pt}%
\definecolor{currentstroke}{rgb}{0.121569,0.466667,0.705882}%
\pgfsetstrokecolor{currentstroke}%
\pgfsetdash{}{0pt}%
\pgfpathmoveto{\pgfqpoint{1.090361in}{0.988256in}}%
\pgfpathcurveto{\pgfqpoint{1.096185in}{0.988256in}}{\pgfqpoint{1.101771in}{0.990570in}}{\pgfqpoint{1.105889in}{0.994688in}}%
\pgfpathcurveto{\pgfqpoint{1.110007in}{0.998806in}}{\pgfqpoint{1.112321in}{1.004392in}}{\pgfqpoint{1.112321in}{1.010216in}}%
\pgfpathcurveto{\pgfqpoint{1.112321in}{1.016040in}}{\pgfqpoint{1.110007in}{1.021626in}}{\pgfqpoint{1.105889in}{1.025744in}}%
\pgfpathcurveto{\pgfqpoint{1.101771in}{1.029863in}}{\pgfqpoint{1.096185in}{1.032176in}}{\pgfqpoint{1.090361in}{1.032176in}}%
\pgfpathcurveto{\pgfqpoint{1.084537in}{1.032176in}}{\pgfqpoint{1.078951in}{1.029863in}}{\pgfqpoint{1.074833in}{1.025744in}}%
\pgfpathcurveto{\pgfqpoint{1.070715in}{1.021626in}}{\pgfqpoint{1.068401in}{1.016040in}}{\pgfqpoint{1.068401in}{1.010216in}}%
\pgfpathcurveto{\pgfqpoint{1.068401in}{1.004392in}}{\pgfqpoint{1.070715in}{0.998806in}}{\pgfqpoint{1.074833in}{0.994688in}}%
\pgfpathcurveto{\pgfqpoint{1.078951in}{0.990570in}}{\pgfqpoint{1.084537in}{0.988256in}}{\pgfqpoint{1.090361in}{0.988256in}}%
\pgfpathclose%
\pgfusepath{stroke,fill}%
\end{pgfscope}%
\begin{pgfscope}%
\pgfpathrectangle{\pgfqpoint{0.211875in}{0.211875in}}{\pgfqpoint{1.313625in}{1.279725in}}%
\pgfusepath{clip}%
\pgfsetbuttcap%
\pgfsetroundjoin%
\definecolor{currentfill}{rgb}{0.121569,0.466667,0.705882}%
\pgfsetfillcolor{currentfill}%
\pgfsetlinewidth{1.003750pt}%
\definecolor{currentstroke}{rgb}{0.121569,0.466667,0.705882}%
\pgfsetstrokecolor{currentstroke}%
\pgfsetdash{}{0pt}%
\pgfpathmoveto{\pgfqpoint{1.092265in}{0.986777in}}%
\pgfpathcurveto{\pgfqpoint{1.098089in}{0.986777in}}{\pgfqpoint{1.103675in}{0.989091in}}{\pgfqpoint{1.107793in}{0.993209in}}%
\pgfpathcurveto{\pgfqpoint{1.111911in}{0.997327in}}{\pgfqpoint{1.114225in}{1.002913in}}{\pgfqpoint{1.114225in}{1.008737in}}%
\pgfpathcurveto{\pgfqpoint{1.114225in}{1.014561in}}{\pgfqpoint{1.111911in}{1.020147in}}{\pgfqpoint{1.107793in}{1.024265in}}%
\pgfpathcurveto{\pgfqpoint{1.103675in}{1.028383in}}{\pgfqpoint{1.098089in}{1.030697in}}{\pgfqpoint{1.092265in}{1.030697in}}%
\pgfpathcurveto{\pgfqpoint{1.086441in}{1.030697in}}{\pgfqpoint{1.080855in}{1.028383in}}{\pgfqpoint{1.076737in}{1.024265in}}%
\pgfpathcurveto{\pgfqpoint{1.072619in}{1.020147in}}{\pgfqpoint{1.070305in}{1.014561in}}{\pgfqpoint{1.070305in}{1.008737in}}%
\pgfpathcurveto{\pgfqpoint{1.070305in}{1.002913in}}{\pgfqpoint{1.072619in}{0.997327in}}{\pgfqpoint{1.076737in}{0.993209in}}%
\pgfpathcurveto{\pgfqpoint{1.080855in}{0.989091in}}{\pgfqpoint{1.086441in}{0.986777in}}{\pgfqpoint{1.092265in}{0.986777in}}%
\pgfpathclose%
\pgfusepath{stroke,fill}%
\end{pgfscope}%
\begin{pgfscope}%
\pgfpathrectangle{\pgfqpoint{0.211875in}{0.211875in}}{\pgfqpoint{1.313625in}{1.279725in}}%
\pgfusepath{clip}%
\pgfsetbuttcap%
\pgfsetroundjoin%
\definecolor{currentfill}{rgb}{0.121569,0.466667,0.705882}%
\pgfsetfillcolor{currentfill}%
\pgfsetlinewidth{1.003750pt}%
\definecolor{currentstroke}{rgb}{0.121569,0.466667,0.705882}%
\pgfsetstrokecolor{currentstroke}%
\pgfsetdash{}{0pt}%
\pgfpathmoveto{\pgfqpoint{1.093145in}{0.985653in}}%
\pgfpathcurveto{\pgfqpoint{1.098969in}{0.985653in}}{\pgfqpoint{1.104555in}{0.987967in}}{\pgfqpoint{1.108673in}{0.992085in}}%
\pgfpathcurveto{\pgfqpoint{1.112791in}{0.996203in}}{\pgfqpoint{1.115105in}{1.001789in}}{\pgfqpoint{1.115105in}{1.007613in}}%
\pgfpathcurveto{\pgfqpoint{1.115105in}{1.013437in}}{\pgfqpoint{1.112791in}{1.019023in}}{\pgfqpoint{1.108673in}{1.023141in}}%
\pgfpathcurveto{\pgfqpoint{1.104555in}{1.027259in}}{\pgfqpoint{1.098969in}{1.029573in}}{\pgfqpoint{1.093145in}{1.029573in}}%
\pgfpathcurveto{\pgfqpoint{1.087321in}{1.029573in}}{\pgfqpoint{1.081735in}{1.027259in}}{\pgfqpoint{1.077617in}{1.023141in}}%
\pgfpathcurveto{\pgfqpoint{1.073498in}{1.019023in}}{\pgfqpoint{1.071185in}{1.013437in}}{\pgfqpoint{1.071185in}{1.007613in}}%
\pgfpathcurveto{\pgfqpoint{1.071185in}{1.001789in}}{\pgfqpoint{1.073498in}{0.996203in}}{\pgfqpoint{1.077617in}{0.992085in}}%
\pgfpathcurveto{\pgfqpoint{1.081735in}{0.987967in}}{\pgfqpoint{1.087321in}{0.985653in}}{\pgfqpoint{1.093145in}{0.985653in}}%
\pgfpathclose%
\pgfusepath{stroke,fill}%
\end{pgfscope}%
\begin{pgfscope}%
\pgfpathrectangle{\pgfqpoint{0.211875in}{0.211875in}}{\pgfqpoint{1.313625in}{1.279725in}}%
\pgfusepath{clip}%
\pgfsetbuttcap%
\pgfsetroundjoin%
\definecolor{currentfill}{rgb}{0.121569,0.466667,0.705882}%
\pgfsetfillcolor{currentfill}%
\pgfsetlinewidth{1.003750pt}%
\definecolor{currentstroke}{rgb}{0.121569,0.466667,0.705882}%
\pgfsetstrokecolor{currentstroke}%
\pgfsetdash{}{0pt}%
\pgfpathmoveto{\pgfqpoint{1.093615in}{0.984695in}}%
\pgfpathcurveto{\pgfqpoint{1.099439in}{0.984695in}}{\pgfqpoint{1.105025in}{0.987009in}}{\pgfqpoint{1.109144in}{0.991127in}}%
\pgfpathcurveto{\pgfqpoint{1.113262in}{0.995245in}}{\pgfqpoint{1.115576in}{1.000831in}}{\pgfqpoint{1.115576in}{1.006655in}}%
\pgfpathcurveto{\pgfqpoint{1.115576in}{1.012479in}}{\pgfqpoint{1.113262in}{1.018065in}}{\pgfqpoint{1.109144in}{1.022183in}}%
\pgfpathcurveto{\pgfqpoint{1.105025in}{1.026301in}}{\pgfqpoint{1.099439in}{1.028615in}}{\pgfqpoint{1.093615in}{1.028615in}}%
\pgfpathcurveto{\pgfqpoint{1.087791in}{1.028615in}}{\pgfqpoint{1.082205in}{1.026301in}}{\pgfqpoint{1.078087in}{1.022183in}}%
\pgfpathcurveto{\pgfqpoint{1.073969in}{1.018065in}}{\pgfqpoint{1.071655in}{1.012479in}}{\pgfqpoint{1.071655in}{1.006655in}}%
\pgfpathcurveto{\pgfqpoint{1.071655in}{1.000831in}}{\pgfqpoint{1.073969in}{0.995245in}}{\pgfqpoint{1.078087in}{0.991127in}}%
\pgfpathcurveto{\pgfqpoint{1.082205in}{0.987009in}}{\pgfqpoint{1.087791in}{0.984695in}}{\pgfqpoint{1.093615in}{0.984695in}}%
\pgfpathclose%
\pgfusepath{stroke,fill}%
\end{pgfscope}%
\begin{pgfscope}%
\pgfpathrectangle{\pgfqpoint{0.211875in}{0.211875in}}{\pgfqpoint{1.313625in}{1.279725in}}%
\pgfusepath{clip}%
\pgfsetbuttcap%
\pgfsetroundjoin%
\definecolor{currentfill}{rgb}{0.121569,0.466667,0.705882}%
\pgfsetfillcolor{currentfill}%
\pgfsetlinewidth{1.003750pt}%
\definecolor{currentstroke}{rgb}{0.121569,0.466667,0.705882}%
\pgfsetstrokecolor{currentstroke}%
\pgfsetdash{}{0pt}%
\pgfpathmoveto{\pgfqpoint{1.093901in}{0.983812in}}%
\pgfpathcurveto{\pgfqpoint{1.099725in}{0.983812in}}{\pgfqpoint{1.105311in}{0.986125in}}{\pgfqpoint{1.109429in}{0.990244in}}%
\pgfpathcurveto{\pgfqpoint{1.113547in}{0.994362in}}{\pgfqpoint{1.115861in}{0.999948in}}{\pgfqpoint{1.115861in}{1.005772in}}%
\pgfpathcurveto{\pgfqpoint{1.115861in}{1.011596in}}{\pgfqpoint{1.113547in}{1.017182in}}{\pgfqpoint{1.109429in}{1.021300in}}%
\pgfpathcurveto{\pgfqpoint{1.105311in}{1.025418in}}{\pgfqpoint{1.099725in}{1.027732in}}{\pgfqpoint{1.093901in}{1.027732in}}%
\pgfpathcurveto{\pgfqpoint{1.088077in}{1.027732in}}{\pgfqpoint{1.082491in}{1.025418in}}{\pgfqpoint{1.078373in}{1.021300in}}%
\pgfpathcurveto{\pgfqpoint{1.074254in}{1.017182in}}{\pgfqpoint{1.071941in}{1.011596in}}{\pgfqpoint{1.071941in}{1.005772in}}%
\pgfpathcurveto{\pgfqpoint{1.071941in}{0.999948in}}{\pgfqpoint{1.074254in}{0.994362in}}{\pgfqpoint{1.078373in}{0.990244in}}%
\pgfpathcurveto{\pgfqpoint{1.082491in}{0.986125in}}{\pgfqpoint{1.088077in}{0.983812in}}{\pgfqpoint{1.093901in}{0.983812in}}%
\pgfpathclose%
\pgfusepath{stroke,fill}%
\end{pgfscope}%
\begin{pgfscope}%
\pgfpathrectangle{\pgfqpoint{0.211875in}{0.211875in}}{\pgfqpoint{1.313625in}{1.279725in}}%
\pgfusepath{clip}%
\pgfsetbuttcap%
\pgfsetroundjoin%
\definecolor{currentfill}{rgb}{0.121569,0.466667,0.705882}%
\pgfsetfillcolor{currentfill}%
\pgfsetlinewidth{1.003750pt}%
\definecolor{currentstroke}{rgb}{0.121569,0.466667,0.705882}%
\pgfsetstrokecolor{currentstroke}%
\pgfsetdash{}{0pt}%
\pgfpathmoveto{\pgfqpoint{0.614593in}{1.157771in}}%
\pgfpathcurveto{\pgfqpoint{0.620417in}{1.157771in}}{\pgfqpoint{0.626003in}{1.160084in}}{\pgfqpoint{0.630122in}{1.164203in}}%
\pgfpathcurveto{\pgfqpoint{0.634240in}{1.168321in}}{\pgfqpoint{0.636554in}{1.173907in}}{\pgfqpoint{0.636554in}{1.179731in}}%
\pgfpathcurveto{\pgfqpoint{0.636554in}{1.185555in}}{\pgfqpoint{0.634240in}{1.191141in}}{\pgfqpoint{0.630122in}{1.195259in}}%
\pgfpathcurveto{\pgfqpoint{0.626003in}{1.199377in}}{\pgfqpoint{0.620417in}{1.201691in}}{\pgfqpoint{0.614593in}{1.201691in}}%
\pgfpathcurveto{\pgfqpoint{0.608769in}{1.201691in}}{\pgfqpoint{0.603183in}{1.199377in}}{\pgfqpoint{0.599065in}{1.195259in}}%
\pgfpathcurveto{\pgfqpoint{0.594947in}{1.191141in}}{\pgfqpoint{0.592633in}{1.185555in}}{\pgfqpoint{0.592633in}{1.179731in}}%
\pgfpathcurveto{\pgfqpoint{0.592633in}{1.173907in}}{\pgfqpoint{0.594947in}{1.168321in}}{\pgfqpoint{0.599065in}{1.164203in}}%
\pgfpathcurveto{\pgfqpoint{0.603183in}{1.160084in}}{\pgfqpoint{0.608769in}{1.157771in}}{\pgfqpoint{0.614593in}{1.157771in}}%
\pgfpathclose%
\pgfusepath{stroke,fill}%
\end{pgfscope}%
\begin{pgfscope}%
\pgfpathrectangle{\pgfqpoint{0.211875in}{0.211875in}}{\pgfqpoint{1.313625in}{1.279725in}}%
\pgfusepath{clip}%
\pgfsetbuttcap%
\pgfsetroundjoin%
\definecolor{currentfill}{rgb}{0.121569,0.466667,0.705882}%
\pgfsetfillcolor{currentfill}%
\pgfsetlinewidth{1.003750pt}%
\definecolor{currentstroke}{rgb}{0.121569,0.466667,0.705882}%
\pgfsetstrokecolor{currentstroke}%
\pgfsetdash{}{0pt}%
\pgfpathmoveto{\pgfqpoint{1.099969in}{0.988837in}}%
\pgfpathcurveto{\pgfqpoint{1.105793in}{0.988837in}}{\pgfqpoint{1.111379in}{0.991150in}}{\pgfqpoint{1.115497in}{0.995269in}}%
\pgfpathcurveto{\pgfqpoint{1.119616in}{0.999387in}}{\pgfqpoint{1.121929in}{1.004973in}}{\pgfqpoint{1.121929in}{1.010797in}}%
\pgfpathcurveto{\pgfqpoint{1.121929in}{1.016621in}}{\pgfqpoint{1.119616in}{1.022207in}}{\pgfqpoint{1.115497in}{1.026325in}}%
\pgfpathcurveto{\pgfqpoint{1.111379in}{1.030443in}}{\pgfqpoint{1.105793in}{1.032757in}}{\pgfqpoint{1.099969in}{1.032757in}}%
\pgfpathcurveto{\pgfqpoint{1.094145in}{1.032757in}}{\pgfqpoint{1.088559in}{1.030443in}}{\pgfqpoint{1.084441in}{1.026325in}}%
\pgfpathcurveto{\pgfqpoint{1.080323in}{1.022207in}}{\pgfqpoint{1.078009in}{1.016621in}}{\pgfqpoint{1.078009in}{1.010797in}}%
\pgfpathcurveto{\pgfqpoint{1.078009in}{1.004973in}}{\pgfqpoint{1.080323in}{0.999387in}}{\pgfqpoint{1.084441in}{0.995269in}}%
\pgfpathcurveto{\pgfqpoint{1.088559in}{0.991150in}}{\pgfqpoint{1.094145in}{0.988837in}}{\pgfqpoint{1.099969in}{0.988837in}}%
\pgfpathclose%
\pgfusepath{stroke,fill}%
\end{pgfscope}%
\begin{pgfscope}%
\pgfpathrectangle{\pgfqpoint{0.211875in}{0.211875in}}{\pgfqpoint{1.313625in}{1.279725in}}%
\pgfusepath{clip}%
\pgfsetbuttcap%
\pgfsetroundjoin%
\definecolor{currentfill}{rgb}{0.121569,0.466667,0.705882}%
\pgfsetfillcolor{currentfill}%
\pgfsetlinewidth{1.003750pt}%
\definecolor{currentstroke}{rgb}{0.121569,0.466667,0.705882}%
\pgfsetstrokecolor{currentstroke}%
\pgfsetdash{}{0pt}%
\pgfpathmoveto{\pgfqpoint{1.098364in}{0.989524in}}%
\pgfpathcurveto{\pgfqpoint{1.104188in}{0.989524in}}{\pgfqpoint{1.109774in}{0.991838in}}{\pgfqpoint{1.113893in}{0.995956in}}%
\pgfpathcurveto{\pgfqpoint{1.118011in}{1.000074in}}{\pgfqpoint{1.120325in}{1.005660in}}{\pgfqpoint{1.120325in}{1.011484in}}%
\pgfpathcurveto{\pgfqpoint{1.120325in}{1.017308in}}{\pgfqpoint{1.118011in}{1.022894in}}{\pgfqpoint{1.113893in}{1.027012in}}%
\pgfpathcurveto{\pgfqpoint{1.109774in}{1.031131in}}{\pgfqpoint{1.104188in}{1.033444in}}{\pgfqpoint{1.098364in}{1.033444in}}%
\pgfpathcurveto{\pgfqpoint{1.092540in}{1.033444in}}{\pgfqpoint{1.086954in}{1.031131in}}{\pgfqpoint{1.082836in}{1.027012in}}%
\pgfpathcurveto{\pgfqpoint{1.078718in}{1.022894in}}{\pgfqpoint{1.076404in}{1.017308in}}{\pgfqpoint{1.076404in}{1.011484in}}%
\pgfpathcurveto{\pgfqpoint{1.076404in}{1.005660in}}{\pgfqpoint{1.078718in}{1.000074in}}{\pgfqpoint{1.082836in}{0.995956in}}%
\pgfpathcurveto{\pgfqpoint{1.086954in}{0.991838in}}{\pgfqpoint{1.092540in}{0.989524in}}{\pgfqpoint{1.098364in}{0.989524in}}%
\pgfpathclose%
\pgfusepath{stroke,fill}%
\end{pgfscope}%
\begin{pgfscope}%
\pgfpathrectangle{\pgfqpoint{0.211875in}{0.211875in}}{\pgfqpoint{1.313625in}{1.279725in}}%
\pgfusepath{clip}%
\pgfsetbuttcap%
\pgfsetroundjoin%
\definecolor{currentfill}{rgb}{0.121569,0.466667,0.705882}%
\pgfsetfillcolor{currentfill}%
\pgfsetlinewidth{1.003750pt}%
\definecolor{currentstroke}{rgb}{0.121569,0.466667,0.705882}%
\pgfsetstrokecolor{currentstroke}%
\pgfsetdash{}{0pt}%
\pgfpathmoveto{\pgfqpoint{1.096984in}{0.989854in}}%
\pgfpathcurveto{\pgfqpoint{1.102808in}{0.989854in}}{\pgfqpoint{1.108394in}{0.992168in}}{\pgfqpoint{1.112512in}{0.996286in}}%
\pgfpathcurveto{\pgfqpoint{1.116630in}{1.000404in}}{\pgfqpoint{1.118944in}{1.005990in}}{\pgfqpoint{1.118944in}{1.011814in}}%
\pgfpathcurveto{\pgfqpoint{1.118944in}{1.017638in}}{\pgfqpoint{1.116630in}{1.023224in}}{\pgfqpoint{1.112512in}{1.027342in}}%
\pgfpathcurveto{\pgfqpoint{1.108394in}{1.031460in}}{\pgfqpoint{1.102808in}{1.033774in}}{\pgfqpoint{1.096984in}{1.033774in}}%
\pgfpathcurveto{\pgfqpoint{1.091160in}{1.033774in}}{\pgfqpoint{1.085574in}{1.031460in}}{\pgfqpoint{1.081456in}{1.027342in}}%
\pgfpathcurveto{\pgfqpoint{1.077337in}{1.023224in}}{\pgfqpoint{1.075024in}{1.017638in}}{\pgfqpoint{1.075024in}{1.011814in}}%
\pgfpathcurveto{\pgfqpoint{1.075024in}{1.005990in}}{\pgfqpoint{1.077337in}{1.000404in}}{\pgfqpoint{1.081456in}{0.996286in}}%
\pgfpathcurveto{\pgfqpoint{1.085574in}{0.992168in}}{\pgfqpoint{1.091160in}{0.989854in}}{\pgfqpoint{1.096984in}{0.989854in}}%
\pgfpathclose%
\pgfusepath{stroke,fill}%
\end{pgfscope}%
\begin{pgfscope}%
\pgfpathrectangle{\pgfqpoint{0.211875in}{0.211875in}}{\pgfqpoint{1.313625in}{1.279725in}}%
\pgfusepath{clip}%
\pgfsetbuttcap%
\pgfsetroundjoin%
\definecolor{currentfill}{rgb}{0.121569,0.466667,0.705882}%
\pgfsetfillcolor{currentfill}%
\pgfsetlinewidth{1.003750pt}%
\definecolor{currentstroke}{rgb}{0.121569,0.466667,0.705882}%
\pgfsetstrokecolor{currentstroke}%
\pgfsetdash{}{0pt}%
\pgfpathmoveto{\pgfqpoint{1.095874in}{0.989890in}}%
\pgfpathcurveto{\pgfqpoint{1.101698in}{0.989890in}}{\pgfqpoint{1.107284in}{0.992204in}}{\pgfqpoint{1.111402in}{0.996322in}}%
\pgfpathcurveto{\pgfqpoint{1.115520in}{1.000440in}}{\pgfqpoint{1.117834in}{1.006026in}}{\pgfqpoint{1.117834in}{1.011850in}}%
\pgfpathcurveto{\pgfqpoint{1.117834in}{1.017674in}}{\pgfqpoint{1.115520in}{1.023260in}}{\pgfqpoint{1.111402in}{1.027378in}}%
\pgfpathcurveto{\pgfqpoint{1.107284in}{1.031496in}}{\pgfqpoint{1.101698in}{1.033810in}}{\pgfqpoint{1.095874in}{1.033810in}}%
\pgfpathcurveto{\pgfqpoint{1.090050in}{1.033810in}}{\pgfqpoint{1.084464in}{1.031496in}}{\pgfqpoint{1.080345in}{1.027378in}}%
\pgfpathcurveto{\pgfqpoint{1.076227in}{1.023260in}}{\pgfqpoint{1.073913in}{1.017674in}}{\pgfqpoint{1.073913in}{1.011850in}}%
\pgfpathcurveto{\pgfqpoint{1.073913in}{1.006026in}}{\pgfqpoint{1.076227in}{1.000440in}}{\pgfqpoint{1.080345in}{0.996322in}}%
\pgfpathcurveto{\pgfqpoint{1.084464in}{0.992204in}}{\pgfqpoint{1.090050in}{0.989890in}}{\pgfqpoint{1.095874in}{0.989890in}}%
\pgfpathclose%
\pgfusepath{stroke,fill}%
\end{pgfscope}%
\begin{pgfscope}%
\pgfpathrectangle{\pgfqpoint{0.211875in}{0.211875in}}{\pgfqpoint{1.313625in}{1.279725in}}%
\pgfusepath{clip}%
\pgfsetbuttcap%
\pgfsetroundjoin%
\definecolor{currentfill}{rgb}{0.121569,0.466667,0.705882}%
\pgfsetfillcolor{currentfill}%
\pgfsetlinewidth{1.003750pt}%
\definecolor{currentstroke}{rgb}{0.121569,0.466667,0.705882}%
\pgfsetstrokecolor{currentstroke}%
\pgfsetdash{}{0pt}%
\pgfpathmoveto{\pgfqpoint{0.895048in}{1.179568in}}%
\pgfpathcurveto{\pgfqpoint{0.900872in}{1.179568in}}{\pgfqpoint{0.906458in}{1.181882in}}{\pgfqpoint{0.910576in}{1.186000in}}%
\pgfpathcurveto{\pgfqpoint{0.914694in}{1.190118in}}{\pgfqpoint{0.917008in}{1.195704in}}{\pgfqpoint{0.917008in}{1.201528in}}%
\pgfpathcurveto{\pgfqpoint{0.917008in}{1.207352in}}{\pgfqpoint{0.914694in}{1.212938in}}{\pgfqpoint{0.910576in}{1.217056in}}%
\pgfpathcurveto{\pgfqpoint{0.906458in}{1.221175in}}{\pgfqpoint{0.900872in}{1.223488in}}{\pgfqpoint{0.895048in}{1.223488in}}%
\pgfpathcurveto{\pgfqpoint{0.889224in}{1.223488in}}{\pgfqpoint{0.883638in}{1.221175in}}{\pgfqpoint{0.879520in}{1.217056in}}%
\pgfpathcurveto{\pgfqpoint{0.875402in}{1.212938in}}{\pgfqpoint{0.873088in}{1.207352in}}{\pgfqpoint{0.873088in}{1.201528in}}%
\pgfpathcurveto{\pgfqpoint{0.873088in}{1.195704in}}{\pgfqpoint{0.875402in}{1.190118in}}{\pgfqpoint{0.879520in}{1.186000in}}%
\pgfpathcurveto{\pgfqpoint{0.883638in}{1.181882in}}{\pgfqpoint{0.889224in}{1.179568in}}{\pgfqpoint{0.895048in}{1.179568in}}%
\pgfpathclose%
\pgfusepath{stroke,fill}%
\end{pgfscope}%
\begin{pgfscope}%
\pgfpathrectangle{\pgfqpoint{0.211875in}{0.211875in}}{\pgfqpoint{1.313625in}{1.279725in}}%
\pgfusepath{clip}%
\pgfsetbuttcap%
\pgfsetroundjoin%
\definecolor{currentfill}{rgb}{0.121569,0.466667,0.705882}%
\pgfsetfillcolor{currentfill}%
\pgfsetlinewidth{1.003750pt}%
\definecolor{currentstroke}{rgb}{0.121569,0.466667,0.705882}%
\pgfsetstrokecolor{currentstroke}%
\pgfsetdash{}{0pt}%
\pgfpathmoveto{\pgfqpoint{1.105360in}{0.976357in}}%
\pgfpathcurveto{\pgfqpoint{1.111184in}{0.976357in}}{\pgfqpoint{1.116770in}{0.978671in}}{\pgfqpoint{1.120888in}{0.982789in}}%
\pgfpathcurveto{\pgfqpoint{1.125006in}{0.986907in}}{\pgfqpoint{1.127320in}{0.992493in}}{\pgfqpoint{1.127320in}{0.998317in}}%
\pgfpathcurveto{\pgfqpoint{1.127320in}{1.004141in}}{\pgfqpoint{1.125006in}{1.009727in}}{\pgfqpoint{1.120888in}{1.013845in}}%
\pgfpathcurveto{\pgfqpoint{1.116770in}{1.017964in}}{\pgfqpoint{1.111184in}{1.020277in}}{\pgfqpoint{1.105360in}{1.020277in}}%
\pgfpathcurveto{\pgfqpoint{1.099536in}{1.020277in}}{\pgfqpoint{1.093950in}{1.017964in}}{\pgfqpoint{1.089831in}{1.013845in}}%
\pgfpathcurveto{\pgfqpoint{1.085713in}{1.009727in}}{\pgfqpoint{1.083399in}{1.004141in}}{\pgfqpoint{1.083399in}{0.998317in}}%
\pgfpathcurveto{\pgfqpoint{1.083399in}{0.992493in}}{\pgfqpoint{1.085713in}{0.986907in}}{\pgfqpoint{1.089831in}{0.982789in}}%
\pgfpathcurveto{\pgfqpoint{1.093950in}{0.978671in}}{\pgfqpoint{1.099536in}{0.976357in}}{\pgfqpoint{1.105360in}{0.976357in}}%
\pgfpathclose%
\pgfusepath{stroke,fill}%
\end{pgfscope}%
\begin{pgfscope}%
\pgfpathrectangle{\pgfqpoint{0.211875in}{0.211875in}}{\pgfqpoint{1.313625in}{1.279725in}}%
\pgfusepath{clip}%
\pgfsetbuttcap%
\pgfsetroundjoin%
\definecolor{currentfill}{rgb}{0.121569,0.466667,0.705882}%
\pgfsetfillcolor{currentfill}%
\pgfsetlinewidth{1.003750pt}%
\definecolor{currentstroke}{rgb}{0.121569,0.466667,0.705882}%
\pgfsetstrokecolor{currentstroke}%
\pgfsetdash{}{0pt}%
\pgfpathmoveto{\pgfqpoint{1.104066in}{0.975330in}}%
\pgfpathcurveto{\pgfqpoint{1.109890in}{0.975330in}}{\pgfqpoint{1.115476in}{0.977644in}}{\pgfqpoint{1.119595in}{0.981762in}}%
\pgfpathcurveto{\pgfqpoint{1.123713in}{0.985881in}}{\pgfqpoint{1.126027in}{0.991467in}}{\pgfqpoint{1.126027in}{0.997291in}}%
\pgfpathcurveto{\pgfqpoint{1.126027in}{1.003115in}}{\pgfqpoint{1.123713in}{1.008701in}}{\pgfqpoint{1.119595in}{1.012819in}}%
\pgfpathcurveto{\pgfqpoint{1.115476in}{1.016937in}}{\pgfqpoint{1.109890in}{1.019251in}}{\pgfqpoint{1.104066in}{1.019251in}}%
\pgfpathcurveto{\pgfqpoint{1.098242in}{1.019251in}}{\pgfqpoint{1.092656in}{1.016937in}}{\pgfqpoint{1.088538in}{1.012819in}}%
\pgfpathcurveto{\pgfqpoint{1.084420in}{1.008701in}}{\pgfqpoint{1.082106in}{1.003115in}}{\pgfqpoint{1.082106in}{0.997291in}}%
\pgfpathcurveto{\pgfqpoint{1.082106in}{0.991467in}}{\pgfqpoint{1.084420in}{0.985881in}}{\pgfqpoint{1.088538in}{0.981762in}}%
\pgfpathcurveto{\pgfqpoint{1.092656in}{0.977644in}}{\pgfqpoint{1.098242in}{0.975330in}}{\pgfqpoint{1.104066in}{0.975330in}}%
\pgfpathclose%
\pgfusepath{stroke,fill}%
\end{pgfscope}%
\begin{pgfscope}%
\pgfpathrectangle{\pgfqpoint{0.211875in}{0.211875in}}{\pgfqpoint{1.313625in}{1.279725in}}%
\pgfusepath{clip}%
\pgfsetbuttcap%
\pgfsetroundjoin%
\definecolor{currentfill}{rgb}{0.121569,0.466667,0.705882}%
\pgfsetfillcolor{currentfill}%
\pgfsetlinewidth{1.003750pt}%
\definecolor{currentstroke}{rgb}{0.121569,0.466667,0.705882}%
\pgfsetstrokecolor{currentstroke}%
\pgfsetdash{}{0pt}%
\pgfpathmoveto{\pgfqpoint{0.718472in}{0.283233in}}%
\pgfpathcurveto{\pgfqpoint{0.724296in}{0.283233in}}{\pgfqpoint{0.729882in}{0.285547in}}{\pgfqpoint{0.734001in}{0.289665in}}%
\pgfpathcurveto{\pgfqpoint{0.738119in}{0.293784in}}{\pgfqpoint{0.740433in}{0.299370in}}{\pgfqpoint{0.740433in}{0.305194in}}%
\pgfpathcurveto{\pgfqpoint{0.740433in}{0.311018in}}{\pgfqpoint{0.738119in}{0.316604in}}{\pgfqpoint{0.734001in}{0.320722in}}%
\pgfpathcurveto{\pgfqpoint{0.729882in}{0.324840in}}{\pgfqpoint{0.724296in}{0.327154in}}{\pgfqpoint{0.718472in}{0.327154in}}%
\pgfpathcurveto{\pgfqpoint{0.712648in}{0.327154in}}{\pgfqpoint{0.707062in}{0.324840in}}{\pgfqpoint{0.702944in}{0.320722in}}%
\pgfpathcurveto{\pgfqpoint{0.698826in}{0.316604in}}{\pgfqpoint{0.696512in}{0.311018in}}{\pgfqpoint{0.696512in}{0.305194in}}%
\pgfpathcurveto{\pgfqpoint{0.696512in}{0.299370in}}{\pgfqpoint{0.698826in}{0.293784in}}{\pgfqpoint{0.702944in}{0.289665in}}%
\pgfpathcurveto{\pgfqpoint{0.707062in}{0.285547in}}{\pgfqpoint{0.712648in}{0.283233in}}{\pgfqpoint{0.718472in}{0.283233in}}%
\pgfpathclose%
\pgfusepath{stroke,fill}%
\end{pgfscope}%
\begin{pgfscope}%
\pgfpathrectangle{\pgfqpoint{0.211875in}{0.211875in}}{\pgfqpoint{1.313625in}{1.279725in}}%
\pgfusepath{clip}%
\pgfsetbuttcap%
\pgfsetroundjoin%
\definecolor{currentfill}{rgb}{0.121569,0.466667,0.705882}%
\pgfsetfillcolor{currentfill}%
\pgfsetlinewidth{1.003750pt}%
\definecolor{currentstroke}{rgb}{0.121569,0.466667,0.705882}%
\pgfsetstrokecolor{currentstroke}%
\pgfsetdash{}{0pt}%
\pgfpathmoveto{\pgfqpoint{1.103324in}{0.976262in}}%
\pgfpathcurveto{\pgfqpoint{1.109148in}{0.976262in}}{\pgfqpoint{1.114735in}{0.978575in}}{\pgfqpoint{1.118853in}{0.982694in}}%
\pgfpathcurveto{\pgfqpoint{1.122971in}{0.986812in}}{\pgfqpoint{1.125285in}{0.992398in}}{\pgfqpoint{1.125285in}{0.998222in}}%
\pgfpathcurveto{\pgfqpoint{1.125285in}{1.004046in}}{\pgfqpoint{1.122971in}{1.009632in}}{\pgfqpoint{1.118853in}{1.013750in}}%
\pgfpathcurveto{\pgfqpoint{1.114735in}{1.017868in}}{\pgfqpoint{1.109148in}{1.020182in}}{\pgfqpoint{1.103324in}{1.020182in}}%
\pgfpathcurveto{\pgfqpoint{1.097501in}{1.020182in}}{\pgfqpoint{1.091914in}{1.017868in}}{\pgfqpoint{1.087796in}{1.013750in}}%
\pgfpathcurveto{\pgfqpoint{1.083678in}{1.009632in}}{\pgfqpoint{1.081364in}{1.004046in}}{\pgfqpoint{1.081364in}{0.998222in}}%
\pgfpathcurveto{\pgfqpoint{1.081364in}{0.992398in}}{\pgfqpoint{1.083678in}{0.986812in}}{\pgfqpoint{1.087796in}{0.982694in}}%
\pgfpathcurveto{\pgfqpoint{1.091914in}{0.978575in}}{\pgfqpoint{1.097501in}{0.976262in}}{\pgfqpoint{1.103324in}{0.976262in}}%
\pgfpathclose%
\pgfusepath{stroke,fill}%
\end{pgfscope}%
\begin{pgfscope}%
\pgfpathrectangle{\pgfqpoint{0.211875in}{0.211875in}}{\pgfqpoint{1.313625in}{1.279725in}}%
\pgfusepath{clip}%
\pgfsetbuttcap%
\pgfsetroundjoin%
\definecolor{currentfill}{rgb}{0.121569,0.466667,0.705882}%
\pgfsetfillcolor{currentfill}%
\pgfsetlinewidth{1.003750pt}%
\definecolor{currentstroke}{rgb}{0.121569,0.466667,0.705882}%
\pgfsetstrokecolor{currentstroke}%
\pgfsetdash{}{0pt}%
\pgfpathmoveto{\pgfqpoint{1.102109in}{0.975311in}}%
\pgfpathcurveto{\pgfqpoint{1.107933in}{0.975311in}}{\pgfqpoint{1.113519in}{0.977625in}}{\pgfqpoint{1.117637in}{0.981743in}}%
\pgfpathcurveto{\pgfqpoint{1.121755in}{0.985861in}}{\pgfqpoint{1.124069in}{0.991447in}}{\pgfqpoint{1.124069in}{0.997271in}}%
\pgfpathcurveto{\pgfqpoint{1.124069in}{1.003095in}}{\pgfqpoint{1.121755in}{1.008681in}}{\pgfqpoint{1.117637in}{1.012800in}}%
\pgfpathcurveto{\pgfqpoint{1.113519in}{1.016918in}}{\pgfqpoint{1.107933in}{1.019232in}}{\pgfqpoint{1.102109in}{1.019232in}}%
\pgfpathcurveto{\pgfqpoint{1.096285in}{1.019232in}}{\pgfqpoint{1.090699in}{1.016918in}}{\pgfqpoint{1.086581in}{1.012800in}}%
\pgfpathcurveto{\pgfqpoint{1.082463in}{1.008681in}}{\pgfqpoint{1.080149in}{1.003095in}}{\pgfqpoint{1.080149in}{0.997271in}}%
\pgfpathcurveto{\pgfqpoint{1.080149in}{0.991447in}}{\pgfqpoint{1.082463in}{0.985861in}}{\pgfqpoint{1.086581in}{0.981743in}}%
\pgfpathcurveto{\pgfqpoint{1.090699in}{0.977625in}}{\pgfqpoint{1.096285in}{0.975311in}}{\pgfqpoint{1.102109in}{0.975311in}}%
\pgfpathclose%
\pgfusepath{stroke,fill}%
\end{pgfscope}%
\begin{pgfscope}%
\pgfpathrectangle{\pgfqpoint{0.211875in}{0.211875in}}{\pgfqpoint{1.313625in}{1.279725in}}%
\pgfusepath{clip}%
\pgfsetbuttcap%
\pgfsetroundjoin%
\definecolor{currentfill}{rgb}{0.121569,0.466667,0.705882}%
\pgfsetfillcolor{currentfill}%
\pgfsetlinewidth{1.003750pt}%
\definecolor{currentstroke}{rgb}{0.121569,0.466667,0.705882}%
\pgfsetstrokecolor{currentstroke}%
\pgfsetdash{}{0pt}%
\pgfpathmoveto{\pgfqpoint{1.100994in}{0.974288in}}%
\pgfpathcurveto{\pgfqpoint{1.106818in}{0.974288in}}{\pgfqpoint{1.112404in}{0.976602in}}{\pgfqpoint{1.116523in}{0.980720in}}%
\pgfpathcurveto{\pgfqpoint{1.120641in}{0.984838in}}{\pgfqpoint{1.122955in}{0.990425in}}{\pgfqpoint{1.122955in}{0.996249in}}%
\pgfpathcurveto{\pgfqpoint{1.122955in}{1.002072in}}{\pgfqpoint{1.120641in}{1.007659in}}{\pgfqpoint{1.116523in}{1.011777in}}%
\pgfpathcurveto{\pgfqpoint{1.112404in}{1.015895in}}{\pgfqpoint{1.106818in}{1.018209in}}{\pgfqpoint{1.100994in}{1.018209in}}%
\pgfpathcurveto{\pgfqpoint{1.095170in}{1.018209in}}{\pgfqpoint{1.089584in}{1.015895in}}{\pgfqpoint{1.085466in}{1.011777in}}%
\pgfpathcurveto{\pgfqpoint{1.081348in}{1.007659in}}{\pgfqpoint{1.079034in}{1.002072in}}{\pgfqpoint{1.079034in}{0.996249in}}%
\pgfpathcurveto{\pgfqpoint{1.079034in}{0.990425in}}{\pgfqpoint{1.081348in}{0.984838in}}{\pgfqpoint{1.085466in}{0.980720in}}%
\pgfpathcurveto{\pgfqpoint{1.089584in}{0.976602in}}{\pgfqpoint{1.095170in}{0.974288in}}{\pgfqpoint{1.100994in}{0.974288in}}%
\pgfpathclose%
\pgfusepath{stroke,fill}%
\end{pgfscope}%
\begin{pgfscope}%
\pgfpathrectangle{\pgfqpoint{0.211875in}{0.211875in}}{\pgfqpoint{1.313625in}{1.279725in}}%
\pgfusepath{clip}%
\pgfsetbuttcap%
\pgfsetroundjoin%
\definecolor{currentfill}{rgb}{0.121569,0.466667,0.705882}%
\pgfsetfillcolor{currentfill}%
\pgfsetlinewidth{1.003750pt}%
\definecolor{currentstroke}{rgb}{0.121569,0.466667,0.705882}%
\pgfsetstrokecolor{currentstroke}%
\pgfsetdash{}{0pt}%
\pgfpathmoveto{\pgfqpoint{1.100018in}{0.973260in}}%
\pgfpathcurveto{\pgfqpoint{1.105842in}{0.973260in}}{\pgfqpoint{1.111428in}{0.975574in}}{\pgfqpoint{1.115547in}{0.979692in}}%
\pgfpathcurveto{\pgfqpoint{1.119665in}{0.983810in}}{\pgfqpoint{1.121979in}{0.989396in}}{\pgfqpoint{1.121979in}{0.995220in}}%
\pgfpathcurveto{\pgfqpoint{1.121979in}{1.001044in}}{\pgfqpoint{1.119665in}{1.006630in}}{\pgfqpoint{1.115547in}{1.010749in}}%
\pgfpathcurveto{\pgfqpoint{1.111428in}{1.014867in}}{\pgfqpoint{1.105842in}{1.017181in}}{\pgfqpoint{1.100018in}{1.017181in}}%
\pgfpathcurveto{\pgfqpoint{1.094194in}{1.017181in}}{\pgfqpoint{1.088608in}{1.014867in}}{\pgfqpoint{1.084490in}{1.010749in}}%
\pgfpathcurveto{\pgfqpoint{1.080372in}{1.006630in}}{\pgfqpoint{1.078058in}{1.001044in}}{\pgfqpoint{1.078058in}{0.995220in}}%
\pgfpathcurveto{\pgfqpoint{1.078058in}{0.989396in}}{\pgfqpoint{1.080372in}{0.983810in}}{\pgfqpoint{1.084490in}{0.979692in}}%
\pgfpathcurveto{\pgfqpoint{1.088608in}{0.975574in}}{\pgfqpoint{1.094194in}{0.973260in}}{\pgfqpoint{1.100018in}{0.973260in}}%
\pgfpathclose%
\pgfusepath{stroke,fill}%
\end{pgfscope}%
\begin{pgfscope}%
\pgfpathrectangle{\pgfqpoint{0.211875in}{0.211875in}}{\pgfqpoint{1.313625in}{1.279725in}}%
\pgfusepath{clip}%
\pgfsetbuttcap%
\pgfsetroundjoin%
\definecolor{currentfill}{rgb}{0.121569,0.466667,0.705882}%
\pgfsetfillcolor{currentfill}%
\pgfsetlinewidth{1.003750pt}%
\definecolor{currentstroke}{rgb}{0.121569,0.466667,0.705882}%
\pgfsetstrokecolor{currentstroke}%
\pgfsetdash{}{0pt}%
\pgfpathmoveto{\pgfqpoint{1.099199in}{0.972287in}}%
\pgfpathcurveto{\pgfqpoint{1.105023in}{0.972287in}}{\pgfqpoint{1.110609in}{0.974601in}}{\pgfqpoint{1.114727in}{0.978719in}}%
\pgfpathcurveto{\pgfqpoint{1.118845in}{0.982837in}}{\pgfqpoint{1.121159in}{0.988423in}}{\pgfqpoint{1.121159in}{0.994247in}}%
\pgfpathcurveto{\pgfqpoint{1.121159in}{1.000071in}}{\pgfqpoint{1.118845in}{1.005657in}}{\pgfqpoint{1.114727in}{1.009775in}}%
\pgfpathcurveto{\pgfqpoint{1.110609in}{1.013894in}}{\pgfqpoint{1.105023in}{1.016208in}}{\pgfqpoint{1.099199in}{1.016208in}}%
\pgfpathcurveto{\pgfqpoint{1.093375in}{1.016208in}}{\pgfqpoint{1.087789in}{1.013894in}}{\pgfqpoint{1.083671in}{1.009775in}}%
\pgfpathcurveto{\pgfqpoint{1.079553in}{1.005657in}}{\pgfqpoint{1.077239in}{1.000071in}}{\pgfqpoint{1.077239in}{0.994247in}}%
\pgfpathcurveto{\pgfqpoint{1.077239in}{0.988423in}}{\pgfqpoint{1.079553in}{0.982837in}}{\pgfqpoint{1.083671in}{0.978719in}}%
\pgfpathcurveto{\pgfqpoint{1.087789in}{0.974601in}}{\pgfqpoint{1.093375in}{0.972287in}}{\pgfqpoint{1.099199in}{0.972287in}}%
\pgfpathclose%
\pgfusepath{stroke,fill}%
\end{pgfscope}%
\begin{pgfscope}%
\pgfpathrectangle{\pgfqpoint{0.211875in}{0.211875in}}{\pgfqpoint{1.313625in}{1.279725in}}%
\pgfusepath{clip}%
\pgfsetbuttcap%
\pgfsetroundjoin%
\definecolor{currentfill}{rgb}{0.121569,0.466667,0.705882}%
\pgfsetfillcolor{currentfill}%
\pgfsetlinewidth{1.003750pt}%
\definecolor{currentstroke}{rgb}{0.121569,0.466667,0.705882}%
\pgfsetstrokecolor{currentstroke}%
\pgfsetdash{}{0pt}%
\pgfpathmoveto{\pgfqpoint{0.823200in}{0.611189in}}%
\pgfpathcurveto{\pgfqpoint{0.829024in}{0.611189in}}{\pgfqpoint{0.834610in}{0.613503in}}{\pgfqpoint{0.838728in}{0.617621in}}%
\pgfpathcurveto{\pgfqpoint{0.842846in}{0.621739in}}{\pgfqpoint{0.845160in}{0.627325in}}{\pgfqpoint{0.845160in}{0.633149in}}%
\pgfpathcurveto{\pgfqpoint{0.845160in}{0.638973in}}{\pgfqpoint{0.842846in}{0.644559in}}{\pgfqpoint{0.838728in}{0.648677in}}%
\pgfpathcurveto{\pgfqpoint{0.834610in}{0.652795in}}{\pgfqpoint{0.829024in}{0.655109in}}{\pgfqpoint{0.823200in}{0.655109in}}%
\pgfpathcurveto{\pgfqpoint{0.817376in}{0.655109in}}{\pgfqpoint{0.811790in}{0.652795in}}{\pgfqpoint{0.807671in}{0.648677in}}%
\pgfpathcurveto{\pgfqpoint{0.803553in}{0.644559in}}{\pgfqpoint{0.801239in}{0.638973in}}{\pgfqpoint{0.801239in}{0.633149in}}%
\pgfpathcurveto{\pgfqpoint{0.801239in}{0.627325in}}{\pgfqpoint{0.803553in}{0.621739in}}{\pgfqpoint{0.807671in}{0.617621in}}%
\pgfpathcurveto{\pgfqpoint{0.811790in}{0.613503in}}{\pgfqpoint{0.817376in}{0.611189in}}{\pgfqpoint{0.823200in}{0.611189in}}%
\pgfpathclose%
\pgfusepath{stroke,fill}%
\end{pgfscope}%
\begin{pgfscope}%
\pgfpathrectangle{\pgfqpoint{0.211875in}{0.211875in}}{\pgfqpoint{1.313625in}{1.279725in}}%
\pgfusepath{clip}%
\pgfsetbuttcap%
\pgfsetroundjoin%
\definecolor{currentfill}{rgb}{0.121569,0.466667,0.705882}%
\pgfsetfillcolor{currentfill}%
\pgfsetlinewidth{1.003750pt}%
\definecolor{currentstroke}{rgb}{0.121569,0.466667,0.705882}%
\pgfsetstrokecolor{currentstroke}%
\pgfsetdash{}{0pt}%
\pgfpathmoveto{\pgfqpoint{1.081181in}{0.957173in}}%
\pgfpathcurveto{\pgfqpoint{1.087005in}{0.957173in}}{\pgfqpoint{1.092592in}{0.959487in}}{\pgfqpoint{1.096710in}{0.963605in}}%
\pgfpathcurveto{\pgfqpoint{1.100828in}{0.967723in}}{\pgfqpoint{1.103142in}{0.973309in}}{\pgfqpoint{1.103142in}{0.979133in}}%
\pgfpathcurveto{\pgfqpoint{1.103142in}{0.984957in}}{\pgfqpoint{1.100828in}{0.990543in}}{\pgfqpoint{1.096710in}{0.994661in}}%
\pgfpathcurveto{\pgfqpoint{1.092592in}{0.998779in}}{\pgfqpoint{1.087005in}{1.001093in}}{\pgfqpoint{1.081181in}{1.001093in}}%
\pgfpathcurveto{\pgfqpoint{1.075357in}{1.001093in}}{\pgfqpoint{1.069771in}{0.998779in}}{\pgfqpoint{1.065653in}{0.994661in}}%
\pgfpathcurveto{\pgfqpoint{1.061535in}{0.990543in}}{\pgfqpoint{1.059221in}{0.984957in}}{\pgfqpoint{1.059221in}{0.979133in}}%
\pgfpathcurveto{\pgfqpoint{1.059221in}{0.973309in}}{\pgfqpoint{1.061535in}{0.967723in}}{\pgfqpoint{1.065653in}{0.963605in}}%
\pgfpathcurveto{\pgfqpoint{1.069771in}{0.959487in}}{\pgfqpoint{1.075357in}{0.957173in}}{\pgfqpoint{1.081181in}{0.957173in}}%
\pgfpathclose%
\pgfusepath{stroke,fill}%
\end{pgfscope}%
\begin{pgfscope}%
\pgfpathrectangle{\pgfqpoint{0.211875in}{0.211875in}}{\pgfqpoint{1.313625in}{1.279725in}}%
\pgfusepath{clip}%
\pgfsetbuttcap%
\pgfsetroundjoin%
\definecolor{currentfill}{rgb}{0.121569,0.466667,0.705882}%
\pgfsetfillcolor{currentfill}%
\pgfsetlinewidth{1.003750pt}%
\definecolor{currentstroke}{rgb}{0.121569,0.466667,0.705882}%
\pgfsetstrokecolor{currentstroke}%
\pgfsetdash{}{0pt}%
\pgfpathmoveto{\pgfqpoint{1.082816in}{0.959007in}}%
\pgfpathcurveto{\pgfqpoint{1.088640in}{0.959007in}}{\pgfqpoint{1.094226in}{0.961321in}}{\pgfqpoint{1.098344in}{0.965439in}}%
\pgfpathcurveto{\pgfqpoint{1.102463in}{0.969557in}}{\pgfqpoint{1.104776in}{0.975143in}}{\pgfqpoint{1.104776in}{0.980967in}}%
\pgfpathcurveto{\pgfqpoint{1.104776in}{0.986791in}}{\pgfqpoint{1.102463in}{0.992377in}}{\pgfqpoint{1.098344in}{0.996495in}}%
\pgfpathcurveto{\pgfqpoint{1.094226in}{1.000614in}}{\pgfqpoint{1.088640in}{1.002927in}}{\pgfqpoint{1.082816in}{1.002927in}}%
\pgfpathcurveto{\pgfqpoint{1.076992in}{1.002927in}}{\pgfqpoint{1.071406in}{1.000614in}}{\pgfqpoint{1.067288in}{0.996495in}}%
\pgfpathcurveto{\pgfqpoint{1.063170in}{0.992377in}}{\pgfqpoint{1.060856in}{0.986791in}}{\pgfqpoint{1.060856in}{0.980967in}}%
\pgfpathcurveto{\pgfqpoint{1.060856in}{0.975143in}}{\pgfqpoint{1.063170in}{0.969557in}}{\pgfqpoint{1.067288in}{0.965439in}}%
\pgfpathcurveto{\pgfqpoint{1.071406in}{0.961321in}}{\pgfqpoint{1.076992in}{0.959007in}}{\pgfqpoint{1.082816in}{0.959007in}}%
\pgfpathclose%
\pgfusepath{stroke,fill}%
\end{pgfscope}%
\begin{pgfscope}%
\pgfpathrectangle{\pgfqpoint{0.211875in}{0.211875in}}{\pgfqpoint{1.313625in}{1.279725in}}%
\pgfusepath{clip}%
\pgfsetbuttcap%
\pgfsetroundjoin%
\definecolor{currentfill}{rgb}{0.121569,0.466667,0.705882}%
\pgfsetfillcolor{currentfill}%
\pgfsetlinewidth{1.003750pt}%
\definecolor{currentstroke}{rgb}{0.121569,0.466667,0.705882}%
\pgfsetstrokecolor{currentstroke}%
\pgfsetdash{}{0pt}%
\pgfpathmoveto{\pgfqpoint{0.754859in}{0.522762in}}%
\pgfpathcurveto{\pgfqpoint{0.760683in}{0.522762in}}{\pgfqpoint{0.766270in}{0.525076in}}{\pgfqpoint{0.770388in}{0.529194in}}%
\pgfpathcurveto{\pgfqpoint{0.774506in}{0.533312in}}{\pgfqpoint{0.776820in}{0.538899in}}{\pgfqpoint{0.776820in}{0.544722in}}%
\pgfpathcurveto{\pgfqpoint{0.776820in}{0.550546in}}{\pgfqpoint{0.774506in}{0.556133in}}{\pgfqpoint{0.770388in}{0.560251in}}%
\pgfpathcurveto{\pgfqpoint{0.766270in}{0.564369in}}{\pgfqpoint{0.760683in}{0.566683in}}{\pgfqpoint{0.754859in}{0.566683in}}%
\pgfpathcurveto{\pgfqpoint{0.749036in}{0.566683in}}{\pgfqpoint{0.743449in}{0.564369in}}{\pgfqpoint{0.739331in}{0.560251in}}%
\pgfpathcurveto{\pgfqpoint{0.735213in}{0.556133in}}{\pgfqpoint{0.732899in}{0.550546in}}{\pgfqpoint{0.732899in}{0.544722in}}%
\pgfpathcurveto{\pgfqpoint{0.732899in}{0.538899in}}{\pgfqpoint{0.735213in}{0.533312in}}{\pgfqpoint{0.739331in}{0.529194in}}%
\pgfpathcurveto{\pgfqpoint{0.743449in}{0.525076in}}{\pgfqpoint{0.749036in}{0.522762in}}{\pgfqpoint{0.754859in}{0.522762in}}%
\pgfpathclose%
\pgfusepath{stroke,fill}%
\end{pgfscope}%
\begin{pgfscope}%
\pgfpathrectangle{\pgfqpoint{0.211875in}{0.211875in}}{\pgfqpoint{1.313625in}{1.279725in}}%
\pgfusepath{clip}%
\pgfsetbuttcap%
\pgfsetroundjoin%
\definecolor{currentfill}{rgb}{0.121569,0.466667,0.705882}%
\pgfsetfillcolor{currentfill}%
\pgfsetlinewidth{1.003750pt}%
\definecolor{currentstroke}{rgb}{0.121569,0.466667,0.705882}%
\pgfsetstrokecolor{currentstroke}%
\pgfsetdash{}{0pt}%
\pgfpathmoveto{\pgfqpoint{1.080784in}{0.958996in}}%
\pgfpathcurveto{\pgfqpoint{1.086608in}{0.958996in}}{\pgfqpoint{1.092195in}{0.961310in}}{\pgfqpoint{1.096313in}{0.965428in}}%
\pgfpathcurveto{\pgfqpoint{1.100431in}{0.969546in}}{\pgfqpoint{1.102745in}{0.975132in}}{\pgfqpoint{1.102745in}{0.980956in}}%
\pgfpathcurveto{\pgfqpoint{1.102745in}{0.986780in}}{\pgfqpoint{1.100431in}{0.992366in}}{\pgfqpoint{1.096313in}{0.996485in}}%
\pgfpathcurveto{\pgfqpoint{1.092195in}{1.000603in}}{\pgfqpoint{1.086608in}{1.002917in}}{\pgfqpoint{1.080784in}{1.002917in}}%
\pgfpathcurveto{\pgfqpoint{1.074961in}{1.002917in}}{\pgfqpoint{1.069374in}{1.000603in}}{\pgfqpoint{1.065256in}{0.996485in}}%
\pgfpathcurveto{\pgfqpoint{1.061138in}{0.992366in}}{\pgfqpoint{1.058824in}{0.986780in}}{\pgfqpoint{1.058824in}{0.980956in}}%
\pgfpathcurveto{\pgfqpoint{1.058824in}{0.975132in}}{\pgfqpoint{1.061138in}{0.969546in}}{\pgfqpoint{1.065256in}{0.965428in}}%
\pgfpathcurveto{\pgfqpoint{1.069374in}{0.961310in}}{\pgfqpoint{1.074961in}{0.958996in}}{\pgfqpoint{1.080784in}{0.958996in}}%
\pgfpathclose%
\pgfusepath{stroke,fill}%
\end{pgfscope}%
\begin{pgfscope}%
\pgfpathrectangle{\pgfqpoint{0.211875in}{0.211875in}}{\pgfqpoint{1.313625in}{1.279725in}}%
\pgfusepath{clip}%
\pgfsetbuttcap%
\pgfsetroundjoin%
\definecolor{currentfill}{rgb}{0.121569,0.466667,0.705882}%
\pgfsetfillcolor{currentfill}%
\pgfsetlinewidth{1.003750pt}%
\definecolor{currentstroke}{rgb}{0.121569,0.466667,0.705882}%
\pgfsetstrokecolor{currentstroke}%
\pgfsetdash{}{0pt}%
\pgfpathmoveto{\pgfqpoint{1.082856in}{0.961348in}}%
\pgfpathcurveto{\pgfqpoint{1.088680in}{0.961348in}}{\pgfqpoint{1.094266in}{0.963662in}}{\pgfqpoint{1.098384in}{0.967780in}}%
\pgfpathcurveto{\pgfqpoint{1.102502in}{0.971898in}}{\pgfqpoint{1.104816in}{0.977485in}}{\pgfqpoint{1.104816in}{0.983309in}}%
\pgfpathcurveto{\pgfqpoint{1.104816in}{0.989132in}}{\pgfqpoint{1.102502in}{0.994719in}}{\pgfqpoint{1.098384in}{0.998837in}}%
\pgfpathcurveto{\pgfqpoint{1.094266in}{1.002955in}}{\pgfqpoint{1.088680in}{1.005269in}}{\pgfqpoint{1.082856in}{1.005269in}}%
\pgfpathcurveto{\pgfqpoint{1.077032in}{1.005269in}}{\pgfqpoint{1.071446in}{1.002955in}}{\pgfqpoint{1.067328in}{0.998837in}}%
\pgfpathcurveto{\pgfqpoint{1.063209in}{0.994719in}}{\pgfqpoint{1.060896in}{0.989132in}}{\pgfqpoint{1.060896in}{0.983309in}}%
\pgfpathcurveto{\pgfqpoint{1.060896in}{0.977485in}}{\pgfqpoint{1.063209in}{0.971898in}}{\pgfqpoint{1.067328in}{0.967780in}}%
\pgfpathcurveto{\pgfqpoint{1.071446in}{0.963662in}}{\pgfqpoint{1.077032in}{0.961348in}}{\pgfqpoint{1.082856in}{0.961348in}}%
\pgfpathclose%
\pgfusepath{stroke,fill}%
\end{pgfscope}%
\begin{pgfscope}%
\pgfpathrectangle{\pgfqpoint{0.211875in}{0.211875in}}{\pgfqpoint{1.313625in}{1.279725in}}%
\pgfusepath{clip}%
\pgfsetbuttcap%
\pgfsetroundjoin%
\definecolor{currentfill}{rgb}{0.121569,0.466667,0.705882}%
\pgfsetfillcolor{currentfill}%
\pgfsetlinewidth{1.003750pt}%
\definecolor{currentstroke}{rgb}{0.121569,0.466667,0.705882}%
\pgfsetstrokecolor{currentstroke}%
\pgfsetdash{}{0pt}%
\pgfpathmoveto{\pgfqpoint{1.085085in}{0.963850in}}%
\pgfpathcurveto{\pgfqpoint{1.090909in}{0.963850in}}{\pgfqpoint{1.096495in}{0.966164in}}{\pgfqpoint{1.100614in}{0.970282in}}%
\pgfpathcurveto{\pgfqpoint{1.104732in}{0.974401in}}{\pgfqpoint{1.107046in}{0.979987in}}{\pgfqpoint{1.107046in}{0.985811in}}%
\pgfpathcurveto{\pgfqpoint{1.107046in}{0.991635in}}{\pgfqpoint{1.104732in}{0.997221in}}{\pgfqpoint{1.100614in}{1.001339in}}%
\pgfpathcurveto{\pgfqpoint{1.096495in}{1.005457in}}{\pgfqpoint{1.090909in}{1.007771in}}{\pgfqpoint{1.085085in}{1.007771in}}%
\pgfpathcurveto{\pgfqpoint{1.079261in}{1.007771in}}{\pgfqpoint{1.073675in}{1.005457in}}{\pgfqpoint{1.069557in}{1.001339in}}%
\pgfpathcurveto{\pgfqpoint{1.065439in}{0.997221in}}{\pgfqpoint{1.063125in}{0.991635in}}{\pgfqpoint{1.063125in}{0.985811in}}%
\pgfpathcurveto{\pgfqpoint{1.063125in}{0.979987in}}{\pgfqpoint{1.065439in}{0.974401in}}{\pgfqpoint{1.069557in}{0.970282in}}%
\pgfpathcurveto{\pgfqpoint{1.073675in}{0.966164in}}{\pgfqpoint{1.079261in}{0.963850in}}{\pgfqpoint{1.085085in}{0.963850in}}%
\pgfpathclose%
\pgfusepath{stroke,fill}%
\end{pgfscope}%
\begin{pgfscope}%
\pgfpathrectangle{\pgfqpoint{0.211875in}{0.211875in}}{\pgfqpoint{1.313625in}{1.279725in}}%
\pgfusepath{clip}%
\pgfsetbuttcap%
\pgfsetroundjoin%
\definecolor{currentfill}{rgb}{0.121569,0.466667,0.705882}%
\pgfsetfillcolor{currentfill}%
\pgfsetlinewidth{1.003750pt}%
\definecolor{currentstroke}{rgb}{0.121569,0.466667,0.705882}%
\pgfsetstrokecolor{currentstroke}%
\pgfsetdash{}{0pt}%
\pgfpathmoveto{\pgfqpoint{1.087200in}{0.966156in}}%
\pgfpathcurveto{\pgfqpoint{1.093024in}{0.966156in}}{\pgfqpoint{1.098610in}{0.968469in}}{\pgfqpoint{1.102728in}{0.972588in}}%
\pgfpathcurveto{\pgfqpoint{1.106846in}{0.976706in}}{\pgfqpoint{1.109160in}{0.982292in}}{\pgfqpoint{1.109160in}{0.988116in}}%
\pgfpathcurveto{\pgfqpoint{1.109160in}{0.993940in}}{\pgfqpoint{1.106846in}{0.999526in}}{\pgfqpoint{1.102728in}{1.003644in}}%
\pgfpathcurveto{\pgfqpoint{1.098610in}{1.007762in}}{\pgfqpoint{1.093024in}{1.010076in}}{\pgfqpoint{1.087200in}{1.010076in}}%
\pgfpathcurveto{\pgfqpoint{1.081376in}{1.010076in}}{\pgfqpoint{1.075789in}{1.007762in}}{\pgfqpoint{1.071671in}{1.003644in}}%
\pgfpathcurveto{\pgfqpoint{1.067553in}{0.999526in}}{\pgfqpoint{1.065239in}{0.993940in}}{\pgfqpoint{1.065239in}{0.988116in}}%
\pgfpathcurveto{\pgfqpoint{1.065239in}{0.982292in}}{\pgfqpoint{1.067553in}{0.976706in}}{\pgfqpoint{1.071671in}{0.972588in}}%
\pgfpathcurveto{\pgfqpoint{1.075789in}{0.968469in}}{\pgfqpoint{1.081376in}{0.966156in}}{\pgfqpoint{1.087200in}{0.966156in}}%
\pgfpathclose%
\pgfusepath{stroke,fill}%
\end{pgfscope}%
\begin{pgfscope}%
\pgfpathrectangle{\pgfqpoint{0.211875in}{0.211875in}}{\pgfqpoint{1.313625in}{1.279725in}}%
\pgfusepath{clip}%
\pgfsetbuttcap%
\pgfsetroundjoin%
\definecolor{currentfill}{rgb}{0.121569,0.466667,0.705882}%
\pgfsetfillcolor{currentfill}%
\pgfsetlinewidth{1.003750pt}%
\definecolor{currentstroke}{rgb}{0.121569,0.466667,0.705882}%
\pgfsetstrokecolor{currentstroke}%
\pgfsetdash{}{0pt}%
\pgfpathmoveto{\pgfqpoint{1.088970in}{0.967994in}}%
\pgfpathcurveto{\pgfqpoint{1.094794in}{0.967994in}}{\pgfqpoint{1.100380in}{0.970308in}}{\pgfqpoint{1.104498in}{0.974426in}}%
\pgfpathcurveto{\pgfqpoint{1.108616in}{0.978545in}}{\pgfqpoint{1.110930in}{0.984131in}}{\pgfqpoint{1.110930in}{0.989955in}}%
\pgfpathcurveto{\pgfqpoint{1.110930in}{0.995779in}}{\pgfqpoint{1.108616in}{1.001365in}}{\pgfqpoint{1.104498in}{1.005483in}}%
\pgfpathcurveto{\pgfqpoint{1.100380in}{1.009601in}}{\pgfqpoint{1.094794in}{1.011915in}}{\pgfqpoint{1.088970in}{1.011915in}}%
\pgfpathcurveto{\pgfqpoint{1.083146in}{1.011915in}}{\pgfqpoint{1.077560in}{1.009601in}}{\pgfqpoint{1.073441in}{1.005483in}}%
\pgfpathcurveto{\pgfqpoint{1.069323in}{1.001365in}}{\pgfqpoint{1.067009in}{0.995779in}}{\pgfqpoint{1.067009in}{0.989955in}}%
\pgfpathcurveto{\pgfqpoint{1.067009in}{0.984131in}}{\pgfqpoint{1.069323in}{0.978545in}}{\pgfqpoint{1.073441in}{0.974426in}}%
\pgfpathcurveto{\pgfqpoint{1.077560in}{0.970308in}}{\pgfqpoint{1.083146in}{0.967994in}}{\pgfqpoint{1.088970in}{0.967994in}}%
\pgfpathclose%
\pgfusepath{stroke,fill}%
\end{pgfscope}%
\begin{pgfscope}%
\pgfpathrectangle{\pgfqpoint{0.211875in}{0.211875in}}{\pgfqpoint{1.313625in}{1.279725in}}%
\pgfusepath{clip}%
\pgfsetbuttcap%
\pgfsetroundjoin%
\definecolor{currentfill}{rgb}{0.121569,0.466667,0.705882}%
\pgfsetfillcolor{currentfill}%
\pgfsetlinewidth{1.003750pt}%
\definecolor{currentstroke}{rgb}{0.121569,0.466667,0.705882}%
\pgfsetstrokecolor{currentstroke}%
\pgfsetdash{}{0pt}%
\pgfpathmoveto{\pgfqpoint{1.090319in}{0.969306in}}%
\pgfpathcurveto{\pgfqpoint{1.096143in}{0.969306in}}{\pgfqpoint{1.101729in}{0.971620in}}{\pgfqpoint{1.105847in}{0.975738in}}%
\pgfpathcurveto{\pgfqpoint{1.109965in}{0.979856in}}{\pgfqpoint{1.112279in}{0.985442in}}{\pgfqpoint{1.112279in}{0.991266in}}%
\pgfpathcurveto{\pgfqpoint{1.112279in}{0.997090in}}{\pgfqpoint{1.109965in}{1.002676in}}{\pgfqpoint{1.105847in}{1.006794in}}%
\pgfpathcurveto{\pgfqpoint{1.101729in}{1.010912in}}{\pgfqpoint{1.096143in}{1.013226in}}{\pgfqpoint{1.090319in}{1.013226in}}%
\pgfpathcurveto{\pgfqpoint{1.084495in}{1.013226in}}{\pgfqpoint{1.078909in}{1.010912in}}{\pgfqpoint{1.074791in}{1.006794in}}%
\pgfpathcurveto{\pgfqpoint{1.070672in}{1.002676in}}{\pgfqpoint{1.068358in}{0.997090in}}{\pgfqpoint{1.068358in}{0.991266in}}%
\pgfpathcurveto{\pgfqpoint{1.068358in}{0.985442in}}{\pgfqpoint{1.070672in}{0.979856in}}{\pgfqpoint{1.074791in}{0.975738in}}%
\pgfpathcurveto{\pgfqpoint{1.078909in}{0.971620in}}{\pgfqpoint{1.084495in}{0.969306in}}{\pgfqpoint{1.090319in}{0.969306in}}%
\pgfpathclose%
\pgfusepath{stroke,fill}%
\end{pgfscope}%
\begin{pgfscope}%
\pgfpathrectangle{\pgfqpoint{0.211875in}{0.211875in}}{\pgfqpoint{1.313625in}{1.279725in}}%
\pgfusepath{clip}%
\pgfsetbuttcap%
\pgfsetroundjoin%
\definecolor{currentfill}{rgb}{0.121569,0.466667,0.705882}%
\pgfsetfillcolor{currentfill}%
\pgfsetlinewidth{1.003750pt}%
\definecolor{currentstroke}{rgb}{0.121569,0.466667,0.705882}%
\pgfsetstrokecolor{currentstroke}%
\pgfsetdash{}{0pt}%
\pgfpathmoveto{\pgfqpoint{1.091300in}{0.970175in}}%
\pgfpathcurveto{\pgfqpoint{1.097124in}{0.970175in}}{\pgfqpoint{1.102710in}{0.972489in}}{\pgfqpoint{1.106828in}{0.976607in}}%
\pgfpathcurveto{\pgfqpoint{1.110947in}{0.980725in}}{\pgfqpoint{1.113260in}{0.986311in}}{\pgfqpoint{1.113260in}{0.992135in}}%
\pgfpathcurveto{\pgfqpoint{1.113260in}{0.997959in}}{\pgfqpoint{1.110947in}{1.003545in}}{\pgfqpoint{1.106828in}{1.007663in}}%
\pgfpathcurveto{\pgfqpoint{1.102710in}{1.011781in}}{\pgfqpoint{1.097124in}{1.014095in}}{\pgfqpoint{1.091300in}{1.014095in}}%
\pgfpathcurveto{\pgfqpoint{1.085476in}{1.014095in}}{\pgfqpoint{1.079890in}{1.011781in}}{\pgfqpoint{1.075772in}{1.007663in}}%
\pgfpathcurveto{\pgfqpoint{1.071654in}{1.003545in}}{\pgfqpoint{1.069340in}{0.997959in}}{\pgfqpoint{1.069340in}{0.992135in}}%
\pgfpathcurveto{\pgfqpoint{1.069340in}{0.986311in}}{\pgfqpoint{1.071654in}{0.980725in}}{\pgfqpoint{1.075772in}{0.976607in}}%
\pgfpathcurveto{\pgfqpoint{1.079890in}{0.972489in}}{\pgfqpoint{1.085476in}{0.970175in}}{\pgfqpoint{1.091300in}{0.970175in}}%
\pgfpathclose%
\pgfusepath{stroke,fill}%
\end{pgfscope}%
\begin{pgfscope}%
\pgfpathrectangle{\pgfqpoint{0.211875in}{0.211875in}}{\pgfqpoint{1.313625in}{1.279725in}}%
\pgfusepath{clip}%
\pgfsetbuttcap%
\pgfsetroundjoin%
\definecolor{currentfill}{rgb}{0.121569,0.466667,0.705882}%
\pgfsetfillcolor{currentfill}%
\pgfsetlinewidth{1.003750pt}%
\definecolor{currentstroke}{rgb}{0.121569,0.466667,0.705882}%
\pgfsetstrokecolor{currentstroke}%
\pgfsetdash{}{0pt}%
\pgfpathmoveto{\pgfqpoint{1.092002in}{0.970723in}}%
\pgfpathcurveto{\pgfqpoint{1.097826in}{0.970723in}}{\pgfqpoint{1.103412in}{0.973037in}}{\pgfqpoint{1.107530in}{0.977155in}}%
\pgfpathcurveto{\pgfqpoint{1.111648in}{0.981273in}}{\pgfqpoint{1.113962in}{0.986859in}}{\pgfqpoint{1.113962in}{0.992683in}}%
\pgfpathcurveto{\pgfqpoint{1.113962in}{0.998507in}}{\pgfqpoint{1.111648in}{1.004093in}}{\pgfqpoint{1.107530in}{1.008211in}}%
\pgfpathcurveto{\pgfqpoint{1.103412in}{1.012330in}}{\pgfqpoint{1.097826in}{1.014643in}}{\pgfqpoint{1.092002in}{1.014643in}}%
\pgfpathcurveto{\pgfqpoint{1.086178in}{1.014643in}}{\pgfqpoint{1.080592in}{1.012330in}}{\pgfqpoint{1.076473in}{1.008211in}}%
\pgfpathcurveto{\pgfqpoint{1.072355in}{1.004093in}}{\pgfqpoint{1.070041in}{0.998507in}}{\pgfqpoint{1.070041in}{0.992683in}}%
\pgfpathcurveto{\pgfqpoint{1.070041in}{0.986859in}}{\pgfqpoint{1.072355in}{0.981273in}}{\pgfqpoint{1.076473in}{0.977155in}}%
\pgfpathcurveto{\pgfqpoint{1.080592in}{0.973037in}}{\pgfqpoint{1.086178in}{0.970723in}}{\pgfqpoint{1.092002in}{0.970723in}}%
\pgfpathclose%
\pgfusepath{stroke,fill}%
\end{pgfscope}%
\begin{pgfscope}%
\pgfpathrectangle{\pgfqpoint{0.211875in}{0.211875in}}{\pgfqpoint{1.313625in}{1.279725in}}%
\pgfusepath{clip}%
\pgfsetbuttcap%
\pgfsetroundjoin%
\definecolor{currentfill}{rgb}{0.121569,0.466667,0.705882}%
\pgfsetfillcolor{currentfill}%
\pgfsetlinewidth{1.003750pt}%
\definecolor{currentstroke}{rgb}{0.121569,0.466667,0.705882}%
\pgfsetstrokecolor{currentstroke}%
\pgfsetdash{}{0pt}%
\pgfpathmoveto{\pgfqpoint{1.092502in}{0.971046in}}%
\pgfpathcurveto{\pgfqpoint{1.098326in}{0.971046in}}{\pgfqpoint{1.103912in}{0.973360in}}{\pgfqpoint{1.108030in}{0.977478in}}%
\pgfpathcurveto{\pgfqpoint{1.112149in}{0.981597in}}{\pgfqpoint{1.114462in}{0.987183in}}{\pgfqpoint{1.114462in}{0.993007in}}%
\pgfpathcurveto{\pgfqpoint{1.114462in}{0.998831in}}{\pgfqpoint{1.112149in}{1.004417in}}{\pgfqpoint{1.108030in}{1.008535in}}%
\pgfpathcurveto{\pgfqpoint{1.103912in}{1.012653in}}{\pgfqpoint{1.098326in}{1.014967in}}{\pgfqpoint{1.092502in}{1.014967in}}%
\pgfpathcurveto{\pgfqpoint{1.086678in}{1.014967in}}{\pgfqpoint{1.081092in}{1.012653in}}{\pgfqpoint{1.076974in}{1.008535in}}%
\pgfpathcurveto{\pgfqpoint{1.072856in}{1.004417in}}{\pgfqpoint{1.070542in}{0.998831in}}{\pgfqpoint{1.070542in}{0.993007in}}%
\pgfpathcurveto{\pgfqpoint{1.070542in}{0.987183in}}{\pgfqpoint{1.072856in}{0.981597in}}{\pgfqpoint{1.076974in}{0.977478in}}%
\pgfpathcurveto{\pgfqpoint{1.081092in}{0.973360in}}{\pgfqpoint{1.086678in}{0.971046in}}{\pgfqpoint{1.092502in}{0.971046in}}%
\pgfpathclose%
\pgfusepath{stroke,fill}%
\end{pgfscope}%
\begin{pgfscope}%
\pgfpathrectangle{\pgfqpoint{0.211875in}{0.211875in}}{\pgfqpoint{1.313625in}{1.279725in}}%
\pgfusepath{clip}%
\pgfsetbuttcap%
\pgfsetroundjoin%
\definecolor{currentfill}{rgb}{0.121569,0.466667,0.705882}%
\pgfsetfillcolor{currentfill}%
\pgfsetlinewidth{1.003750pt}%
\definecolor{currentstroke}{rgb}{0.121569,0.466667,0.705882}%
\pgfsetstrokecolor{currentstroke}%
\pgfsetdash{}{0pt}%
\pgfpathmoveto{\pgfqpoint{0.888747in}{0.695399in}}%
\pgfpathcurveto{\pgfqpoint{0.894571in}{0.695399in}}{\pgfqpoint{0.900157in}{0.697712in}}{\pgfqpoint{0.904275in}{0.701831in}}%
\pgfpathcurveto{\pgfqpoint{0.908393in}{0.705949in}}{\pgfqpoint{0.910707in}{0.711535in}}{\pgfqpoint{0.910707in}{0.717359in}}%
\pgfpathcurveto{\pgfqpoint{0.910707in}{0.723183in}}{\pgfqpoint{0.908393in}{0.728769in}}{\pgfqpoint{0.904275in}{0.732887in}}%
\pgfpathcurveto{\pgfqpoint{0.900157in}{0.737005in}}{\pgfqpoint{0.894571in}{0.739319in}}{\pgfqpoint{0.888747in}{0.739319in}}%
\pgfpathcurveto{\pgfqpoint{0.882923in}{0.739319in}}{\pgfqpoint{0.877337in}{0.737005in}}{\pgfqpoint{0.873219in}{0.732887in}}%
\pgfpathcurveto{\pgfqpoint{0.869101in}{0.728769in}}{\pgfqpoint{0.866787in}{0.723183in}}{\pgfqpoint{0.866787in}{0.717359in}}%
\pgfpathcurveto{\pgfqpoint{0.866787in}{0.711535in}}{\pgfqpoint{0.869101in}{0.705949in}}{\pgfqpoint{0.873219in}{0.701831in}}%
\pgfpathcurveto{\pgfqpoint{0.877337in}{0.697712in}}{\pgfqpoint{0.882923in}{0.695399in}}{\pgfqpoint{0.888747in}{0.695399in}}%
\pgfpathclose%
\pgfusepath{stroke,fill}%
\end{pgfscope}%
\begin{pgfscope}%
\pgfpathrectangle{\pgfqpoint{0.211875in}{0.211875in}}{\pgfqpoint{1.313625in}{1.279725in}}%
\pgfusepath{clip}%
\pgfsetbuttcap%
\pgfsetroundjoin%
\definecolor{currentfill}{rgb}{0.121569,0.466667,0.705882}%
\pgfsetfillcolor{currentfill}%
\pgfsetlinewidth{1.003750pt}%
\definecolor{currentstroke}{rgb}{0.121569,0.466667,0.705882}%
\pgfsetstrokecolor{currentstroke}%
\pgfsetdash{}{0pt}%
\pgfpathmoveto{\pgfqpoint{1.057923in}{0.916897in}}%
\pgfpathcurveto{\pgfqpoint{1.063747in}{0.916897in}}{\pgfqpoint{1.069333in}{0.919211in}}{\pgfqpoint{1.073452in}{0.923329in}}%
\pgfpathcurveto{\pgfqpoint{1.077570in}{0.927447in}}{\pgfqpoint{1.079884in}{0.933034in}}{\pgfqpoint{1.079884in}{0.938858in}}%
\pgfpathcurveto{\pgfqpoint{1.079884in}{0.944681in}}{\pgfqpoint{1.077570in}{0.950268in}}{\pgfqpoint{1.073452in}{0.954386in}}%
\pgfpathcurveto{\pgfqpoint{1.069333in}{0.958504in}}{\pgfqpoint{1.063747in}{0.960818in}}{\pgfqpoint{1.057923in}{0.960818in}}%
\pgfpathcurveto{\pgfqpoint{1.052099in}{0.960818in}}{\pgfqpoint{1.046513in}{0.958504in}}{\pgfqpoint{1.042395in}{0.954386in}}%
\pgfpathcurveto{\pgfqpoint{1.038277in}{0.950268in}}{\pgfqpoint{1.035963in}{0.944681in}}{\pgfqpoint{1.035963in}{0.938858in}}%
\pgfpathcurveto{\pgfqpoint{1.035963in}{0.933034in}}{\pgfqpoint{1.038277in}{0.927447in}}{\pgfqpoint{1.042395in}{0.923329in}}%
\pgfpathcurveto{\pgfqpoint{1.046513in}{0.919211in}}{\pgfqpoint{1.052099in}{0.916897in}}{\pgfqpoint{1.057923in}{0.916897in}}%
\pgfpathclose%
\pgfusepath{stroke,fill}%
\end{pgfscope}%
\begin{pgfscope}%
\pgfpathrectangle{\pgfqpoint{0.211875in}{0.211875in}}{\pgfqpoint{1.313625in}{1.279725in}}%
\pgfusepath{clip}%
\pgfsetbuttcap%
\pgfsetroundjoin%
\definecolor{currentfill}{rgb}{0.121569,0.466667,0.705882}%
\pgfsetfillcolor{currentfill}%
\pgfsetlinewidth{1.003750pt}%
\definecolor{currentstroke}{rgb}{0.121569,0.466667,0.705882}%
\pgfsetstrokecolor{currentstroke}%
\pgfsetdash{}{0pt}%
\pgfpathmoveto{\pgfqpoint{0.922801in}{0.650400in}}%
\pgfpathcurveto{\pgfqpoint{0.928625in}{0.650400in}}{\pgfqpoint{0.934211in}{0.652714in}}{\pgfqpoint{0.938329in}{0.656832in}}%
\pgfpathcurveto{\pgfqpoint{0.942447in}{0.660950in}}{\pgfqpoint{0.944761in}{0.666536in}}{\pgfqpoint{0.944761in}{0.672360in}}%
\pgfpathcurveto{\pgfqpoint{0.944761in}{0.678184in}}{\pgfqpoint{0.942447in}{0.683770in}}{\pgfqpoint{0.938329in}{0.687888in}}%
\pgfpathcurveto{\pgfqpoint{0.934211in}{0.692007in}}{\pgfqpoint{0.928625in}{0.694320in}}{\pgfqpoint{0.922801in}{0.694320in}}%
\pgfpathcurveto{\pgfqpoint{0.916977in}{0.694320in}}{\pgfqpoint{0.911391in}{0.692007in}}{\pgfqpoint{0.907273in}{0.687888in}}%
\pgfpathcurveto{\pgfqpoint{0.903155in}{0.683770in}}{\pgfqpoint{0.900841in}{0.678184in}}{\pgfqpoint{0.900841in}{0.672360in}}%
\pgfpathcurveto{\pgfqpoint{0.900841in}{0.666536in}}{\pgfqpoint{0.903155in}{0.660950in}}{\pgfqpoint{0.907273in}{0.656832in}}%
\pgfpathcurveto{\pgfqpoint{0.911391in}{0.652714in}}{\pgfqpoint{0.916977in}{0.650400in}}{\pgfqpoint{0.922801in}{0.650400in}}%
\pgfpathclose%
\pgfusepath{stroke,fill}%
\end{pgfscope}%
\begin{pgfscope}%
\pgfpathrectangle{\pgfqpoint{0.211875in}{0.211875in}}{\pgfqpoint{1.313625in}{1.279725in}}%
\pgfusepath{clip}%
\pgfsetbuttcap%
\pgfsetroundjoin%
\definecolor{currentfill}{rgb}{0.121569,0.466667,0.705882}%
\pgfsetfillcolor{currentfill}%
\pgfsetlinewidth{1.003750pt}%
\definecolor{currentstroke}{rgb}{0.121569,0.466667,0.705882}%
\pgfsetstrokecolor{currentstroke}%
\pgfsetdash{}{0pt}%
\pgfpathmoveto{\pgfqpoint{1.052527in}{0.911646in}}%
\pgfpathcurveto{\pgfqpoint{1.058351in}{0.911646in}}{\pgfqpoint{1.063937in}{0.913960in}}{\pgfqpoint{1.068056in}{0.918078in}}%
\pgfpathcurveto{\pgfqpoint{1.072174in}{0.922196in}}{\pgfqpoint{1.074488in}{0.927782in}}{\pgfqpoint{1.074488in}{0.933606in}}%
\pgfpathcurveto{\pgfqpoint{1.074488in}{0.939430in}}{\pgfqpoint{1.072174in}{0.945016in}}{\pgfqpoint{1.068056in}{0.949134in}}%
\pgfpathcurveto{\pgfqpoint{1.063937in}{0.953252in}}{\pgfqpoint{1.058351in}{0.955566in}}{\pgfqpoint{1.052527in}{0.955566in}}%
\pgfpathcurveto{\pgfqpoint{1.046703in}{0.955566in}}{\pgfqpoint{1.041117in}{0.953252in}}{\pgfqpoint{1.036999in}{0.949134in}}%
\pgfpathcurveto{\pgfqpoint{1.032881in}{0.945016in}}{\pgfqpoint{1.030567in}{0.939430in}}{\pgfqpoint{1.030567in}{0.933606in}}%
\pgfpathcurveto{\pgfqpoint{1.030567in}{0.927782in}}{\pgfqpoint{1.032881in}{0.922196in}}{\pgfqpoint{1.036999in}{0.918078in}}%
\pgfpathcurveto{\pgfqpoint{1.041117in}{0.913960in}}{\pgfqpoint{1.046703in}{0.911646in}}{\pgfqpoint{1.052527in}{0.911646in}}%
\pgfpathclose%
\pgfusepath{stroke,fill}%
\end{pgfscope}%
\begin{pgfscope}%
\pgfpathrectangle{\pgfqpoint{0.211875in}{0.211875in}}{\pgfqpoint{1.313625in}{1.279725in}}%
\pgfusepath{clip}%
\pgfsetbuttcap%
\pgfsetroundjoin%
\definecolor{currentfill}{rgb}{0.121569,0.466667,0.705882}%
\pgfsetfillcolor{currentfill}%
\pgfsetlinewidth{1.003750pt}%
\definecolor{currentstroke}{rgb}{0.121569,0.466667,0.705882}%
\pgfsetstrokecolor{currentstroke}%
\pgfsetdash{}{0pt}%
\pgfpathmoveto{\pgfqpoint{0.854883in}{0.736754in}}%
\pgfpathcurveto{\pgfqpoint{0.860707in}{0.736754in}}{\pgfqpoint{0.866293in}{0.739068in}}{\pgfqpoint{0.870411in}{0.743186in}}%
\pgfpathcurveto{\pgfqpoint{0.874529in}{0.747304in}}{\pgfqpoint{0.876843in}{0.752890in}}{\pgfqpoint{0.876843in}{0.758714in}}%
\pgfpathcurveto{\pgfqpoint{0.876843in}{0.764538in}}{\pgfqpoint{0.874529in}{0.770125in}}{\pgfqpoint{0.870411in}{0.774243in}}%
\pgfpathcurveto{\pgfqpoint{0.866293in}{0.778361in}}{\pgfqpoint{0.860707in}{0.780675in}}{\pgfqpoint{0.854883in}{0.780675in}}%
\pgfpathcurveto{\pgfqpoint{0.849059in}{0.780675in}}{\pgfqpoint{0.843473in}{0.778361in}}{\pgfqpoint{0.839355in}{0.774243in}}%
\pgfpathcurveto{\pgfqpoint{0.835237in}{0.770125in}}{\pgfqpoint{0.832923in}{0.764538in}}{\pgfqpoint{0.832923in}{0.758714in}}%
\pgfpathcurveto{\pgfqpoint{0.832923in}{0.752890in}}{\pgfqpoint{0.835237in}{0.747304in}}{\pgfqpoint{0.839355in}{0.743186in}}%
\pgfpathcurveto{\pgfqpoint{0.843473in}{0.739068in}}{\pgfqpoint{0.849059in}{0.736754in}}{\pgfqpoint{0.854883in}{0.736754in}}%
\pgfpathclose%
\pgfusepath{stroke,fill}%
\end{pgfscope}%
\begin{pgfscope}%
\pgfpathrectangle{\pgfqpoint{0.211875in}{0.211875in}}{\pgfqpoint{1.313625in}{1.279725in}}%
\pgfusepath{clip}%
\pgfsetbuttcap%
\pgfsetroundjoin%
\definecolor{currentfill}{rgb}{0.121569,0.466667,0.705882}%
\pgfsetfillcolor{currentfill}%
\pgfsetlinewidth{1.003750pt}%
\definecolor{currentstroke}{rgb}{0.121569,0.466667,0.705882}%
\pgfsetstrokecolor{currentstroke}%
\pgfsetdash{}{0pt}%
\pgfpathmoveto{\pgfqpoint{0.929239in}{0.732598in}}%
\pgfpathcurveto{\pgfqpoint{0.935063in}{0.732598in}}{\pgfqpoint{0.940649in}{0.734912in}}{\pgfqpoint{0.944767in}{0.739030in}}%
\pgfpathcurveto{\pgfqpoint{0.948885in}{0.743148in}}{\pgfqpoint{0.951199in}{0.748734in}}{\pgfqpoint{0.951199in}{0.754558in}}%
\pgfpathcurveto{\pgfqpoint{0.951199in}{0.760382in}}{\pgfqpoint{0.948885in}{0.765968in}}{\pgfqpoint{0.944767in}{0.770086in}}%
\pgfpathcurveto{\pgfqpoint{0.940649in}{0.774204in}}{\pgfqpoint{0.935063in}{0.776518in}}{\pgfqpoint{0.929239in}{0.776518in}}%
\pgfpathcurveto{\pgfqpoint{0.923415in}{0.776518in}}{\pgfqpoint{0.917829in}{0.774204in}}{\pgfqpoint{0.913711in}{0.770086in}}%
\pgfpathcurveto{\pgfqpoint{0.909593in}{0.765968in}}{\pgfqpoint{0.907279in}{0.760382in}}{\pgfqpoint{0.907279in}{0.754558in}}%
\pgfpathcurveto{\pgfqpoint{0.907279in}{0.748734in}}{\pgfqpoint{0.909593in}{0.743148in}}{\pgfqpoint{0.913711in}{0.739030in}}%
\pgfpathcurveto{\pgfqpoint{0.917829in}{0.734912in}}{\pgfqpoint{0.923415in}{0.732598in}}{\pgfqpoint{0.929239in}{0.732598in}}%
\pgfpathclose%
\pgfusepath{stroke,fill}%
\end{pgfscope}%
\begin{pgfscope}%
\pgfpathrectangle{\pgfqpoint{0.211875in}{0.211875in}}{\pgfqpoint{1.313625in}{1.279725in}}%
\pgfusepath{clip}%
\pgfsetbuttcap%
\pgfsetroundjoin%
\definecolor{currentfill}{rgb}{0.121569,0.466667,0.705882}%
\pgfsetfillcolor{currentfill}%
\pgfsetlinewidth{1.003750pt}%
\definecolor{currentstroke}{rgb}{0.121569,0.466667,0.705882}%
\pgfsetstrokecolor{currentstroke}%
\pgfsetdash{}{0pt}%
\pgfpathmoveto{\pgfqpoint{1.055361in}{0.915838in}}%
\pgfpathcurveto{\pgfqpoint{1.061185in}{0.915838in}}{\pgfqpoint{1.066772in}{0.918152in}}{\pgfqpoint{1.070890in}{0.922270in}}%
\pgfpathcurveto{\pgfqpoint{1.075008in}{0.926388in}}{\pgfqpoint{1.077322in}{0.931974in}}{\pgfqpoint{1.077322in}{0.937798in}}%
\pgfpathcurveto{\pgfqpoint{1.077322in}{0.943622in}}{\pgfqpoint{1.075008in}{0.949208in}}{\pgfqpoint{1.070890in}{0.953327in}}%
\pgfpathcurveto{\pgfqpoint{1.066772in}{0.957445in}}{\pgfqpoint{1.061185in}{0.959759in}}{\pgfqpoint{1.055361in}{0.959759in}}%
\pgfpathcurveto{\pgfqpoint{1.049537in}{0.959759in}}{\pgfqpoint{1.043951in}{0.957445in}}{\pgfqpoint{1.039833in}{0.953327in}}%
\pgfpathcurveto{\pgfqpoint{1.035715in}{0.949208in}}{\pgfqpoint{1.033401in}{0.943622in}}{\pgfqpoint{1.033401in}{0.937798in}}%
\pgfpathcurveto{\pgfqpoint{1.033401in}{0.931974in}}{\pgfqpoint{1.035715in}{0.926388in}}{\pgfqpoint{1.039833in}{0.922270in}}%
\pgfpathcurveto{\pgfqpoint{1.043951in}{0.918152in}}{\pgfqpoint{1.049537in}{0.915838in}}{\pgfqpoint{1.055361in}{0.915838in}}%
\pgfpathclose%
\pgfusepath{stroke,fill}%
\end{pgfscope}%
\begin{pgfscope}%
\pgfpathrectangle{\pgfqpoint{0.211875in}{0.211875in}}{\pgfqpoint{1.313625in}{1.279725in}}%
\pgfusepath{clip}%
\pgfsetbuttcap%
\pgfsetroundjoin%
\definecolor{currentfill}{rgb}{0.121569,0.466667,0.705882}%
\pgfsetfillcolor{currentfill}%
\pgfsetlinewidth{1.003750pt}%
\definecolor{currentstroke}{rgb}{0.121569,0.466667,0.705882}%
\pgfsetstrokecolor{currentstroke}%
\pgfsetdash{}{0pt}%
\pgfpathmoveto{\pgfqpoint{1.047794in}{0.904676in}}%
\pgfpathcurveto{\pgfqpoint{1.053617in}{0.904676in}}{\pgfqpoint{1.059204in}{0.906990in}}{\pgfqpoint{1.063322in}{0.911108in}}%
\pgfpathcurveto{\pgfqpoint{1.067440in}{0.915227in}}{\pgfqpoint{1.069754in}{0.920813in}}{\pgfqpoint{1.069754in}{0.926637in}}%
\pgfpathcurveto{\pgfqpoint{1.069754in}{0.932461in}}{\pgfqpoint{1.067440in}{0.938047in}}{\pgfqpoint{1.063322in}{0.942165in}}%
\pgfpathcurveto{\pgfqpoint{1.059204in}{0.946283in}}{\pgfqpoint{1.053617in}{0.948597in}}{\pgfqpoint{1.047794in}{0.948597in}}%
\pgfpathcurveto{\pgfqpoint{1.041970in}{0.948597in}}{\pgfqpoint{1.036383in}{0.946283in}}{\pgfqpoint{1.032265in}{0.942165in}}%
\pgfpathcurveto{\pgfqpoint{1.028147in}{0.938047in}}{\pgfqpoint{1.025833in}{0.932461in}}{\pgfqpoint{1.025833in}{0.926637in}}%
\pgfpathcurveto{\pgfqpoint{1.025833in}{0.920813in}}{\pgfqpoint{1.028147in}{0.915227in}}{\pgfqpoint{1.032265in}{0.911108in}}%
\pgfpathcurveto{\pgfqpoint{1.036383in}{0.906990in}}{\pgfqpoint{1.041970in}{0.904676in}}{\pgfqpoint{1.047794in}{0.904676in}}%
\pgfpathclose%
\pgfusepath{stroke,fill}%
\end{pgfscope}%
\begin{pgfscope}%
\pgfpathrectangle{\pgfqpoint{0.211875in}{0.211875in}}{\pgfqpoint{1.313625in}{1.279725in}}%
\pgfusepath{clip}%
\pgfsetbuttcap%
\pgfsetroundjoin%
\definecolor{currentfill}{rgb}{0.121569,0.466667,0.705882}%
\pgfsetfillcolor{currentfill}%
\pgfsetlinewidth{1.003750pt}%
\definecolor{currentstroke}{rgb}{0.121569,0.466667,0.705882}%
\pgfsetstrokecolor{currentstroke}%
\pgfsetdash{}{0pt}%
\pgfpathmoveto{\pgfqpoint{1.044310in}{0.899102in}}%
\pgfpathcurveto{\pgfqpoint{1.050134in}{0.899102in}}{\pgfqpoint{1.055720in}{0.901416in}}{\pgfqpoint{1.059838in}{0.905534in}}%
\pgfpathcurveto{\pgfqpoint{1.063956in}{0.909652in}}{\pgfqpoint{1.066270in}{0.915238in}}{\pgfqpoint{1.066270in}{0.921062in}}%
\pgfpathcurveto{\pgfqpoint{1.066270in}{0.926886in}}{\pgfqpoint{1.063956in}{0.932473in}}{\pgfqpoint{1.059838in}{0.936591in}}%
\pgfpathcurveto{\pgfqpoint{1.055720in}{0.940709in}}{\pgfqpoint{1.050134in}{0.943023in}}{\pgfqpoint{1.044310in}{0.943023in}}%
\pgfpathcurveto{\pgfqpoint{1.038486in}{0.943023in}}{\pgfqpoint{1.032900in}{0.940709in}}{\pgfqpoint{1.028781in}{0.936591in}}%
\pgfpathcurveto{\pgfqpoint{1.024663in}{0.932473in}}{\pgfqpoint{1.022349in}{0.926886in}}{\pgfqpoint{1.022349in}{0.921062in}}%
\pgfpathcurveto{\pgfqpoint{1.022349in}{0.915238in}}{\pgfqpoint{1.024663in}{0.909652in}}{\pgfqpoint{1.028781in}{0.905534in}}%
\pgfpathcurveto{\pgfqpoint{1.032900in}{0.901416in}}{\pgfqpoint{1.038486in}{0.899102in}}{\pgfqpoint{1.044310in}{0.899102in}}%
\pgfpathclose%
\pgfusepath{stroke,fill}%
\end{pgfscope}%
\begin{pgfscope}%
\pgfpathrectangle{\pgfqpoint{0.211875in}{0.211875in}}{\pgfqpoint{1.313625in}{1.279725in}}%
\pgfusepath{clip}%
\pgfsetbuttcap%
\pgfsetroundjoin%
\definecolor{currentfill}{rgb}{0.121569,0.466667,0.705882}%
\pgfsetfillcolor{currentfill}%
\pgfsetlinewidth{1.003750pt}%
\definecolor{currentstroke}{rgb}{0.121569,0.466667,0.705882}%
\pgfsetstrokecolor{currentstroke}%
\pgfsetdash{}{0pt}%
\pgfpathmoveto{\pgfqpoint{1.043334in}{0.897361in}}%
\pgfpathcurveto{\pgfqpoint{1.049158in}{0.897361in}}{\pgfqpoint{1.054744in}{0.899675in}}{\pgfqpoint{1.058863in}{0.903793in}}%
\pgfpathcurveto{\pgfqpoint{1.062981in}{0.907911in}}{\pgfqpoint{1.065295in}{0.913497in}}{\pgfqpoint{1.065295in}{0.919321in}}%
\pgfpathcurveto{\pgfqpoint{1.065295in}{0.925145in}}{\pgfqpoint{1.062981in}{0.930731in}}{\pgfqpoint{1.058863in}{0.934849in}}%
\pgfpathcurveto{\pgfqpoint{1.054744in}{0.938968in}}{\pgfqpoint{1.049158in}{0.941281in}}{\pgfqpoint{1.043334in}{0.941281in}}%
\pgfpathcurveto{\pgfqpoint{1.037510in}{0.941281in}}{\pgfqpoint{1.031924in}{0.938968in}}{\pgfqpoint{1.027806in}{0.934849in}}%
\pgfpathcurveto{\pgfqpoint{1.023688in}{0.930731in}}{\pgfqpoint{1.021374in}{0.925145in}}{\pgfqpoint{1.021374in}{0.919321in}}%
\pgfpathcurveto{\pgfqpoint{1.021374in}{0.913497in}}{\pgfqpoint{1.023688in}{0.907911in}}{\pgfqpoint{1.027806in}{0.903793in}}%
\pgfpathcurveto{\pgfqpoint{1.031924in}{0.899675in}}{\pgfqpoint{1.037510in}{0.897361in}}{\pgfqpoint{1.043334in}{0.897361in}}%
\pgfpathclose%
\pgfusepath{stroke,fill}%
\end{pgfscope}%
\begin{pgfscope}%
\pgfpathrectangle{\pgfqpoint{0.211875in}{0.211875in}}{\pgfqpoint{1.313625in}{1.279725in}}%
\pgfusepath{clip}%
\pgfsetbuttcap%
\pgfsetroundjoin%
\definecolor{currentfill}{rgb}{0.121569,0.466667,0.705882}%
\pgfsetfillcolor{currentfill}%
\pgfsetlinewidth{1.003750pt}%
\definecolor{currentstroke}{rgb}{0.121569,0.466667,0.705882}%
\pgfsetstrokecolor{currentstroke}%
\pgfsetdash{}{0pt}%
\pgfpathmoveto{\pgfqpoint{0.973647in}{0.848268in}}%
\pgfpathcurveto{\pgfqpoint{0.979471in}{0.848268in}}{\pgfqpoint{0.985057in}{0.850582in}}{\pgfqpoint{0.989175in}{0.854700in}}%
\pgfpathcurveto{\pgfqpoint{0.993293in}{0.858818in}}{\pgfqpoint{0.995607in}{0.864404in}}{\pgfqpoint{0.995607in}{0.870228in}}%
\pgfpathcurveto{\pgfqpoint{0.995607in}{0.876052in}}{\pgfqpoint{0.993293in}{0.881638in}}{\pgfqpoint{0.989175in}{0.885756in}}%
\pgfpathcurveto{\pgfqpoint{0.985057in}{0.889874in}}{\pgfqpoint{0.979471in}{0.892188in}}{\pgfqpoint{0.973647in}{0.892188in}}%
\pgfpathcurveto{\pgfqpoint{0.967823in}{0.892188in}}{\pgfqpoint{0.962237in}{0.889874in}}{\pgfqpoint{0.958119in}{0.885756in}}%
\pgfpathcurveto{\pgfqpoint{0.954000in}{0.881638in}}{\pgfqpoint{0.951687in}{0.876052in}}{\pgfqpoint{0.951687in}{0.870228in}}%
\pgfpathcurveto{\pgfqpoint{0.951687in}{0.864404in}}{\pgfqpoint{0.954000in}{0.858818in}}{\pgfqpoint{0.958119in}{0.854700in}}%
\pgfpathcurveto{\pgfqpoint{0.962237in}{0.850582in}}{\pgfqpoint{0.967823in}{0.848268in}}{\pgfqpoint{0.973647in}{0.848268in}}%
\pgfpathclose%
\pgfusepath{stroke,fill}%
\end{pgfscope}%
\begin{pgfscope}%
\pgfpathrectangle{\pgfqpoint{0.211875in}{0.211875in}}{\pgfqpoint{1.313625in}{1.279725in}}%
\pgfusepath{clip}%
\pgfsetbuttcap%
\pgfsetroundjoin%
\definecolor{currentfill}{rgb}{0.121569,0.466667,0.705882}%
\pgfsetfillcolor{currentfill}%
\pgfsetlinewidth{1.003750pt}%
\definecolor{currentstroke}{rgb}{0.121569,0.466667,0.705882}%
\pgfsetstrokecolor{currentstroke}%
\pgfsetdash{}{0pt}%
\pgfpathmoveto{\pgfqpoint{0.852669in}{0.667954in}}%
\pgfpathcurveto{\pgfqpoint{0.858493in}{0.667954in}}{\pgfqpoint{0.864079in}{0.670268in}}{\pgfqpoint{0.868197in}{0.674386in}}%
\pgfpathcurveto{\pgfqpoint{0.872315in}{0.678504in}}{\pgfqpoint{0.874629in}{0.684090in}}{\pgfqpoint{0.874629in}{0.689914in}}%
\pgfpathcurveto{\pgfqpoint{0.874629in}{0.695738in}}{\pgfqpoint{0.872315in}{0.701324in}}{\pgfqpoint{0.868197in}{0.705442in}}%
\pgfpathcurveto{\pgfqpoint{0.864079in}{0.709561in}}{\pgfqpoint{0.858493in}{0.711874in}}{\pgfqpoint{0.852669in}{0.711874in}}%
\pgfpathcurveto{\pgfqpoint{0.846845in}{0.711874in}}{\pgfqpoint{0.841259in}{0.709561in}}{\pgfqpoint{0.837141in}{0.705442in}}%
\pgfpathcurveto{\pgfqpoint{0.833023in}{0.701324in}}{\pgfqpoint{0.830709in}{0.695738in}}{\pgfqpoint{0.830709in}{0.689914in}}%
\pgfpathcurveto{\pgfqpoint{0.830709in}{0.684090in}}{\pgfqpoint{0.833023in}{0.678504in}}{\pgfqpoint{0.837141in}{0.674386in}}%
\pgfpathcurveto{\pgfqpoint{0.841259in}{0.670268in}}{\pgfqpoint{0.846845in}{0.667954in}}{\pgfqpoint{0.852669in}{0.667954in}}%
\pgfpathclose%
\pgfusepath{stroke,fill}%
\end{pgfscope}%
\begin{pgfscope}%
\pgfpathrectangle{\pgfqpoint{0.211875in}{0.211875in}}{\pgfqpoint{1.313625in}{1.279725in}}%
\pgfusepath{clip}%
\pgfsetbuttcap%
\pgfsetroundjoin%
\definecolor{currentfill}{rgb}{0.121569,0.466667,0.705882}%
\pgfsetfillcolor{currentfill}%
\pgfsetlinewidth{1.003750pt}%
\definecolor{currentstroke}{rgb}{0.121569,0.466667,0.705882}%
\pgfsetstrokecolor{currentstroke}%
\pgfsetdash{}{0pt}%
\pgfpathmoveto{\pgfqpoint{1.058090in}{0.917262in}}%
\pgfpathcurveto{\pgfqpoint{1.063913in}{0.917262in}}{\pgfqpoint{1.069500in}{0.919576in}}{\pgfqpoint{1.073618in}{0.923694in}}%
\pgfpathcurveto{\pgfqpoint{1.077736in}{0.927812in}}{\pgfqpoint{1.080050in}{0.933398in}}{\pgfqpoint{1.080050in}{0.939222in}}%
\pgfpathcurveto{\pgfqpoint{1.080050in}{0.945046in}}{\pgfqpoint{1.077736in}{0.950632in}}{\pgfqpoint{1.073618in}{0.954751in}}%
\pgfpathcurveto{\pgfqpoint{1.069500in}{0.958869in}}{\pgfqpoint{1.063913in}{0.961183in}}{\pgfqpoint{1.058090in}{0.961183in}}%
\pgfpathcurveto{\pgfqpoint{1.052266in}{0.961183in}}{\pgfqpoint{1.046679in}{0.958869in}}{\pgfqpoint{1.042561in}{0.954751in}}%
\pgfpathcurveto{\pgfqpoint{1.038443in}{0.950632in}}{\pgfqpoint{1.036129in}{0.945046in}}{\pgfqpoint{1.036129in}{0.939222in}}%
\pgfpathcurveto{\pgfqpoint{1.036129in}{0.933398in}}{\pgfqpoint{1.038443in}{0.927812in}}{\pgfqpoint{1.042561in}{0.923694in}}%
\pgfpathcurveto{\pgfqpoint{1.046679in}{0.919576in}}{\pgfqpoint{1.052266in}{0.917262in}}{\pgfqpoint{1.058090in}{0.917262in}}%
\pgfpathclose%
\pgfusepath{stroke,fill}%
\end{pgfscope}%
\begin{pgfscope}%
\pgfpathrectangle{\pgfqpoint{0.211875in}{0.211875in}}{\pgfqpoint{1.313625in}{1.279725in}}%
\pgfusepath{clip}%
\pgfsetbuttcap%
\pgfsetroundjoin%
\definecolor{currentfill}{rgb}{0.121569,0.466667,0.705882}%
\pgfsetfillcolor{currentfill}%
\pgfsetlinewidth{1.003750pt}%
\definecolor{currentstroke}{rgb}{0.121569,0.466667,0.705882}%
\pgfsetstrokecolor{currentstroke}%
\pgfsetdash{}{0pt}%
\pgfpathmoveto{\pgfqpoint{1.051859in}{0.907795in}}%
\pgfpathcurveto{\pgfqpoint{1.057683in}{0.907795in}}{\pgfqpoint{1.063269in}{0.910109in}}{\pgfqpoint{1.067387in}{0.914227in}}%
\pgfpathcurveto{\pgfqpoint{1.071506in}{0.918345in}}{\pgfqpoint{1.073819in}{0.923931in}}{\pgfqpoint{1.073819in}{0.929755in}}%
\pgfpathcurveto{\pgfqpoint{1.073819in}{0.935579in}}{\pgfqpoint{1.071506in}{0.941165in}}{\pgfqpoint{1.067387in}{0.945284in}}%
\pgfpathcurveto{\pgfqpoint{1.063269in}{0.949402in}}{\pgfqpoint{1.057683in}{0.951716in}}{\pgfqpoint{1.051859in}{0.951716in}}%
\pgfpathcurveto{\pgfqpoint{1.046035in}{0.951716in}}{\pgfqpoint{1.040449in}{0.949402in}}{\pgfqpoint{1.036331in}{0.945284in}}%
\pgfpathcurveto{\pgfqpoint{1.032213in}{0.941165in}}{\pgfqpoint{1.029899in}{0.935579in}}{\pgfqpoint{1.029899in}{0.929755in}}%
\pgfpathcurveto{\pgfqpoint{1.029899in}{0.923931in}}{\pgfqpoint{1.032213in}{0.918345in}}{\pgfqpoint{1.036331in}{0.914227in}}%
\pgfpathcurveto{\pgfqpoint{1.040449in}{0.910109in}}{\pgfqpoint{1.046035in}{0.907795in}}{\pgfqpoint{1.051859in}{0.907795in}}%
\pgfpathclose%
\pgfusepath{stroke,fill}%
\end{pgfscope}%
\begin{pgfscope}%
\pgfpathrectangle{\pgfqpoint{0.211875in}{0.211875in}}{\pgfqpoint{1.313625in}{1.279725in}}%
\pgfusepath{clip}%
\pgfsetbuttcap%
\pgfsetroundjoin%
\definecolor{currentfill}{rgb}{0.121569,0.466667,0.705882}%
\pgfsetfillcolor{currentfill}%
\pgfsetlinewidth{1.003750pt}%
\definecolor{currentstroke}{rgb}{0.121569,0.466667,0.705882}%
\pgfsetstrokecolor{currentstroke}%
\pgfsetdash{}{0pt}%
\pgfpathmoveto{\pgfqpoint{0.866652in}{0.534055in}}%
\pgfpathcurveto{\pgfqpoint{0.872476in}{0.534055in}}{\pgfqpoint{0.878062in}{0.536369in}}{\pgfqpoint{0.882180in}{0.540487in}}%
\pgfpathcurveto{\pgfqpoint{0.886298in}{0.544606in}}{\pgfqpoint{0.888612in}{0.550192in}}{\pgfqpoint{0.888612in}{0.556016in}}%
\pgfpathcurveto{\pgfqpoint{0.888612in}{0.561840in}}{\pgfqpoint{0.886298in}{0.567426in}}{\pgfqpoint{0.882180in}{0.571544in}}%
\pgfpathcurveto{\pgfqpoint{0.878062in}{0.575662in}}{\pgfqpoint{0.872476in}{0.577976in}}{\pgfqpoint{0.866652in}{0.577976in}}%
\pgfpathcurveto{\pgfqpoint{0.860828in}{0.577976in}}{\pgfqpoint{0.855242in}{0.575662in}}{\pgfqpoint{0.851123in}{0.571544in}}%
\pgfpathcurveto{\pgfqpoint{0.847005in}{0.567426in}}{\pgfqpoint{0.844691in}{0.561840in}}{\pgfqpoint{0.844691in}{0.556016in}}%
\pgfpathcurveto{\pgfqpoint{0.844691in}{0.550192in}}{\pgfqpoint{0.847005in}{0.544606in}}{\pgfqpoint{0.851123in}{0.540487in}}%
\pgfpathcurveto{\pgfqpoint{0.855242in}{0.536369in}}{\pgfqpoint{0.860828in}{0.534055in}}{\pgfqpoint{0.866652in}{0.534055in}}%
\pgfpathclose%
\pgfusepath{stroke,fill}%
\end{pgfscope}%
\begin{pgfscope}%
\pgfpathrectangle{\pgfqpoint{0.211875in}{0.211875in}}{\pgfqpoint{1.313625in}{1.279725in}}%
\pgfusepath{clip}%
\pgfsetbuttcap%
\pgfsetroundjoin%
\definecolor{currentfill}{rgb}{0.121569,0.466667,0.705882}%
\pgfsetfillcolor{currentfill}%
\pgfsetlinewidth{1.003750pt}%
\definecolor{currentstroke}{rgb}{0.121569,0.466667,0.705882}%
\pgfsetstrokecolor{currentstroke}%
\pgfsetdash{}{0pt}%
\pgfpathmoveto{\pgfqpoint{1.025255in}{0.796417in}}%
\pgfpathcurveto{\pgfqpoint{1.031079in}{0.796417in}}{\pgfqpoint{1.036665in}{0.798731in}}{\pgfqpoint{1.040783in}{0.802849in}}%
\pgfpathcurveto{\pgfqpoint{1.044902in}{0.806968in}}{\pgfqpoint{1.047215in}{0.812554in}}{\pgfqpoint{1.047215in}{0.818378in}}%
\pgfpathcurveto{\pgfqpoint{1.047215in}{0.824202in}}{\pgfqpoint{1.044902in}{0.829788in}}{\pgfqpoint{1.040783in}{0.833906in}}%
\pgfpathcurveto{\pgfqpoint{1.036665in}{0.838024in}}{\pgfqpoint{1.031079in}{0.840338in}}{\pgfqpoint{1.025255in}{0.840338in}}%
\pgfpathcurveto{\pgfqpoint{1.019431in}{0.840338in}}{\pgfqpoint{1.013845in}{0.838024in}}{\pgfqpoint{1.009727in}{0.833906in}}%
\pgfpathcurveto{\pgfqpoint{1.005609in}{0.829788in}}{\pgfqpoint{1.003295in}{0.824202in}}{\pgfqpoint{1.003295in}{0.818378in}}%
\pgfpathcurveto{\pgfqpoint{1.003295in}{0.812554in}}{\pgfqpoint{1.005609in}{0.806968in}}{\pgfqpoint{1.009727in}{0.802849in}}%
\pgfpathcurveto{\pgfqpoint{1.013845in}{0.798731in}}{\pgfqpoint{1.019431in}{0.796417in}}{\pgfqpoint{1.025255in}{0.796417in}}%
\pgfpathclose%
\pgfusepath{stroke,fill}%
\end{pgfscope}%
\begin{pgfscope}%
\pgfpathrectangle{\pgfqpoint{0.211875in}{0.211875in}}{\pgfqpoint{1.313625in}{1.279725in}}%
\pgfusepath{clip}%
\pgfsetbuttcap%
\pgfsetroundjoin%
\definecolor{currentfill}{rgb}{0.121569,0.466667,0.705882}%
\pgfsetfillcolor{currentfill}%
\pgfsetlinewidth{1.003750pt}%
\definecolor{currentstroke}{rgb}{0.121569,0.466667,0.705882}%
\pgfsetstrokecolor{currentstroke}%
\pgfsetdash{}{0pt}%
\pgfpathmoveto{\pgfqpoint{0.907517in}{1.010175in}}%
\pgfpathcurveto{\pgfqpoint{0.913341in}{1.010175in}}{\pgfqpoint{0.918927in}{1.012489in}}{\pgfqpoint{0.923045in}{1.016607in}}%
\pgfpathcurveto{\pgfqpoint{0.927164in}{1.020725in}}{\pgfqpoint{0.929477in}{1.026311in}}{\pgfqpoint{0.929477in}{1.032135in}}%
\pgfpathcurveto{\pgfqpoint{0.929477in}{1.037959in}}{\pgfqpoint{0.927164in}{1.043545in}}{\pgfqpoint{0.923045in}{1.047663in}}%
\pgfpathcurveto{\pgfqpoint{0.918927in}{1.051781in}}{\pgfqpoint{0.913341in}{1.054095in}}{\pgfqpoint{0.907517in}{1.054095in}}%
\pgfpathcurveto{\pgfqpoint{0.901693in}{1.054095in}}{\pgfqpoint{0.896107in}{1.051781in}}{\pgfqpoint{0.891989in}{1.047663in}}%
\pgfpathcurveto{\pgfqpoint{0.887871in}{1.043545in}}{\pgfqpoint{0.885557in}{1.037959in}}{\pgfqpoint{0.885557in}{1.032135in}}%
\pgfpathcurveto{\pgfqpoint{0.885557in}{1.026311in}}{\pgfqpoint{0.887871in}{1.020725in}}{\pgfqpoint{0.891989in}{1.016607in}}%
\pgfpathcurveto{\pgfqpoint{0.896107in}{1.012489in}}{\pgfqpoint{0.901693in}{1.010175in}}{\pgfqpoint{0.907517in}{1.010175in}}%
\pgfpathclose%
\pgfusepath{stroke,fill}%
\end{pgfscope}%
\begin{pgfscope}%
\pgfpathrectangle{\pgfqpoint{0.211875in}{0.211875in}}{\pgfqpoint{1.313625in}{1.279725in}}%
\pgfusepath{clip}%
\pgfsetbuttcap%
\pgfsetroundjoin%
\definecolor{currentfill}{rgb}{0.121569,0.466667,0.705882}%
\pgfsetfillcolor{currentfill}%
\pgfsetlinewidth{1.003750pt}%
\definecolor{currentstroke}{rgb}{0.121569,0.466667,0.705882}%
\pgfsetstrokecolor{currentstroke}%
\pgfsetdash{}{0pt}%
\pgfpathmoveto{\pgfqpoint{1.042207in}{0.923619in}}%
\pgfpathcurveto{\pgfqpoint{1.048031in}{0.923619in}}{\pgfqpoint{1.053617in}{0.925933in}}{\pgfqpoint{1.057735in}{0.930051in}}%
\pgfpathcurveto{\pgfqpoint{1.061853in}{0.934169in}}{\pgfqpoint{1.064167in}{0.939755in}}{\pgfqpoint{1.064167in}{0.945579in}}%
\pgfpathcurveto{\pgfqpoint{1.064167in}{0.951403in}}{\pgfqpoint{1.061853in}{0.956989in}}{\pgfqpoint{1.057735in}{0.961107in}}%
\pgfpathcurveto{\pgfqpoint{1.053617in}{0.965226in}}{\pgfqpoint{1.048031in}{0.967539in}}{\pgfqpoint{1.042207in}{0.967539in}}%
\pgfpathcurveto{\pgfqpoint{1.036383in}{0.967539in}}{\pgfqpoint{1.030797in}{0.965226in}}{\pgfqpoint{1.026679in}{0.961107in}}%
\pgfpathcurveto{\pgfqpoint{1.022560in}{0.956989in}}{\pgfqpoint{1.020247in}{0.951403in}}{\pgfqpoint{1.020247in}{0.945579in}}%
\pgfpathcurveto{\pgfqpoint{1.020247in}{0.939755in}}{\pgfqpoint{1.022560in}{0.934169in}}{\pgfqpoint{1.026679in}{0.930051in}}%
\pgfpathcurveto{\pgfqpoint{1.030797in}{0.925933in}}{\pgfqpoint{1.036383in}{0.923619in}}{\pgfqpoint{1.042207in}{0.923619in}}%
\pgfpathclose%
\pgfusepath{stroke,fill}%
\end{pgfscope}%
\begin{pgfscope}%
\pgfpathrectangle{\pgfqpoint{0.211875in}{0.211875in}}{\pgfqpoint{1.313625in}{1.279725in}}%
\pgfusepath{clip}%
\pgfsetbuttcap%
\pgfsetroundjoin%
\definecolor{currentfill}{rgb}{0.121569,0.466667,0.705882}%
\pgfsetfillcolor{currentfill}%
\pgfsetlinewidth{1.003750pt}%
\definecolor{currentstroke}{rgb}{0.121569,0.466667,0.705882}%
\pgfsetstrokecolor{currentstroke}%
\pgfsetdash{}{0pt}%
\pgfpathmoveto{\pgfqpoint{1.040780in}{0.921872in}}%
\pgfpathcurveto{\pgfqpoint{1.046604in}{0.921872in}}{\pgfqpoint{1.052191in}{0.924186in}}{\pgfqpoint{1.056309in}{0.928304in}}%
\pgfpathcurveto{\pgfqpoint{1.060427in}{0.932422in}}{\pgfqpoint{1.062741in}{0.938008in}}{\pgfqpoint{1.062741in}{0.943832in}}%
\pgfpathcurveto{\pgfqpoint{1.062741in}{0.949656in}}{\pgfqpoint{1.060427in}{0.955242in}}{\pgfqpoint{1.056309in}{0.959360in}}%
\pgfpathcurveto{\pgfqpoint{1.052191in}{0.963478in}}{\pgfqpoint{1.046604in}{0.965792in}}{\pgfqpoint{1.040780in}{0.965792in}}%
\pgfpathcurveto{\pgfqpoint{1.034957in}{0.965792in}}{\pgfqpoint{1.029370in}{0.963478in}}{\pgfqpoint{1.025252in}{0.959360in}}%
\pgfpathcurveto{\pgfqpoint{1.021134in}{0.955242in}}{\pgfqpoint{1.018820in}{0.949656in}}{\pgfqpoint{1.018820in}{0.943832in}}%
\pgfpathcurveto{\pgfqpoint{1.018820in}{0.938008in}}{\pgfqpoint{1.021134in}{0.932422in}}{\pgfqpoint{1.025252in}{0.928304in}}%
\pgfpathcurveto{\pgfqpoint{1.029370in}{0.924186in}}{\pgfqpoint{1.034957in}{0.921872in}}{\pgfqpoint{1.040780in}{0.921872in}}%
\pgfpathclose%
\pgfusepath{stroke,fill}%
\end{pgfscope}%
\begin{pgfscope}%
\pgfpathrectangle{\pgfqpoint{0.211875in}{0.211875in}}{\pgfqpoint{1.313625in}{1.279725in}}%
\pgfusepath{clip}%
\pgfsetbuttcap%
\pgfsetroundjoin%
\definecolor{currentfill}{rgb}{0.121569,0.466667,0.705882}%
\pgfsetfillcolor{currentfill}%
\pgfsetlinewidth{1.003750pt}%
\definecolor{currentstroke}{rgb}{0.121569,0.466667,0.705882}%
\pgfsetstrokecolor{currentstroke}%
\pgfsetdash{}{0pt}%
\pgfpathmoveto{\pgfqpoint{0.851989in}{1.024348in}}%
\pgfpathcurveto{\pgfqpoint{0.857813in}{1.024348in}}{\pgfqpoint{0.863399in}{1.026662in}}{\pgfqpoint{0.867517in}{1.030780in}}%
\pgfpathcurveto{\pgfqpoint{0.871635in}{1.034898in}}{\pgfqpoint{0.873949in}{1.040484in}}{\pgfqpoint{0.873949in}{1.046308in}}%
\pgfpathcurveto{\pgfqpoint{0.873949in}{1.052132in}}{\pgfqpoint{0.871635in}{1.057718in}}{\pgfqpoint{0.867517in}{1.061836in}}%
\pgfpathcurveto{\pgfqpoint{0.863399in}{1.065955in}}{\pgfqpoint{0.857813in}{1.068268in}}{\pgfqpoint{0.851989in}{1.068268in}}%
\pgfpathcurveto{\pgfqpoint{0.846165in}{1.068268in}}{\pgfqpoint{0.840579in}{1.065955in}}{\pgfqpoint{0.836461in}{1.061836in}}%
\pgfpathcurveto{\pgfqpoint{0.832343in}{1.057718in}}{\pgfqpoint{0.830029in}{1.052132in}}{\pgfqpoint{0.830029in}{1.046308in}}%
\pgfpathcurveto{\pgfqpoint{0.830029in}{1.040484in}}{\pgfqpoint{0.832343in}{1.034898in}}{\pgfqpoint{0.836461in}{1.030780in}}%
\pgfpathcurveto{\pgfqpoint{0.840579in}{1.026662in}}{\pgfqpoint{0.846165in}{1.024348in}}{\pgfqpoint{0.851989in}{1.024348in}}%
\pgfpathclose%
\pgfusepath{stroke,fill}%
\end{pgfscope}%
\begin{pgfscope}%
\pgfpathrectangle{\pgfqpoint{0.211875in}{0.211875in}}{\pgfqpoint{1.313625in}{1.279725in}}%
\pgfusepath{clip}%
\pgfsetbuttcap%
\pgfsetroundjoin%
\definecolor{currentfill}{rgb}{0.121569,0.466667,0.705882}%
\pgfsetfillcolor{currentfill}%
\pgfsetlinewidth{1.003750pt}%
\definecolor{currentstroke}{rgb}{0.121569,0.466667,0.705882}%
\pgfsetstrokecolor{currentstroke}%
\pgfsetdash{}{0pt}%
\pgfpathmoveto{\pgfqpoint{0.925154in}{1.065683in}}%
\pgfpathcurveto{\pgfqpoint{0.930978in}{1.065683in}}{\pgfqpoint{0.936564in}{1.067996in}}{\pgfqpoint{0.940682in}{1.072115in}}%
\pgfpathcurveto{\pgfqpoint{0.944800in}{1.076233in}}{\pgfqpoint{0.947114in}{1.081819in}}{\pgfqpoint{0.947114in}{1.087643in}}%
\pgfpathcurveto{\pgfqpoint{0.947114in}{1.093467in}}{\pgfqpoint{0.944800in}{1.099053in}}{\pgfqpoint{0.940682in}{1.103171in}}%
\pgfpathcurveto{\pgfqpoint{0.936564in}{1.107289in}}{\pgfqpoint{0.930978in}{1.109603in}}{\pgfqpoint{0.925154in}{1.109603in}}%
\pgfpathcurveto{\pgfqpoint{0.919330in}{1.109603in}}{\pgfqpoint{0.913744in}{1.107289in}}{\pgfqpoint{0.909626in}{1.103171in}}%
\pgfpathcurveto{\pgfqpoint{0.905508in}{1.099053in}}{\pgfqpoint{0.903194in}{1.093467in}}{\pgfqpoint{0.903194in}{1.087643in}}%
\pgfpathcurveto{\pgfqpoint{0.903194in}{1.081819in}}{\pgfqpoint{0.905508in}{1.076233in}}{\pgfqpoint{0.909626in}{1.072115in}}%
\pgfpathcurveto{\pgfqpoint{0.913744in}{1.067996in}}{\pgfqpoint{0.919330in}{1.065683in}}{\pgfqpoint{0.925154in}{1.065683in}}%
\pgfpathclose%
\pgfusepath{stroke,fill}%
\end{pgfscope}%
\begin{pgfscope}%
\pgfpathrectangle{\pgfqpoint{0.211875in}{0.211875in}}{\pgfqpoint{1.313625in}{1.279725in}}%
\pgfusepath{clip}%
\pgfsetbuttcap%
\pgfsetroundjoin%
\definecolor{currentfill}{rgb}{0.121569,0.466667,0.705882}%
\pgfsetfillcolor{currentfill}%
\pgfsetlinewidth{1.003750pt}%
\definecolor{currentstroke}{rgb}{0.121569,0.466667,0.705882}%
\pgfsetstrokecolor{currentstroke}%
\pgfsetdash{}{0pt}%
\pgfpathmoveto{\pgfqpoint{0.739814in}{0.642961in}}%
\pgfpathcurveto{\pgfqpoint{0.745638in}{0.642961in}}{\pgfqpoint{0.751224in}{0.645275in}}{\pgfqpoint{0.755342in}{0.649393in}}%
\pgfpathcurveto{\pgfqpoint{0.759460in}{0.653511in}}{\pgfqpoint{0.761774in}{0.659097in}}{\pgfqpoint{0.761774in}{0.664921in}}%
\pgfpathcurveto{\pgfqpoint{0.761774in}{0.670745in}}{\pgfqpoint{0.759460in}{0.676331in}}{\pgfqpoint{0.755342in}{0.680450in}}%
\pgfpathcurveto{\pgfqpoint{0.751224in}{0.684568in}}{\pgfqpoint{0.745638in}{0.686882in}}{\pgfqpoint{0.739814in}{0.686882in}}%
\pgfpathcurveto{\pgfqpoint{0.733990in}{0.686882in}}{\pgfqpoint{0.728404in}{0.684568in}}{\pgfqpoint{0.724286in}{0.680450in}}%
\pgfpathcurveto{\pgfqpoint{0.720168in}{0.676331in}}{\pgfqpoint{0.717854in}{0.670745in}}{\pgfqpoint{0.717854in}{0.664921in}}%
\pgfpathcurveto{\pgfqpoint{0.717854in}{0.659097in}}{\pgfqpoint{0.720168in}{0.653511in}}{\pgfqpoint{0.724286in}{0.649393in}}%
\pgfpathcurveto{\pgfqpoint{0.728404in}{0.645275in}}{\pgfqpoint{0.733990in}{0.642961in}}{\pgfqpoint{0.739814in}{0.642961in}}%
\pgfpathclose%
\pgfusepath{stroke,fill}%
\end{pgfscope}%
\begin{pgfscope}%
\pgfpathrectangle{\pgfqpoint{0.211875in}{0.211875in}}{\pgfqpoint{1.313625in}{1.279725in}}%
\pgfusepath{clip}%
\pgfsetbuttcap%
\pgfsetroundjoin%
\definecolor{currentfill}{rgb}{0.121569,0.466667,0.705882}%
\pgfsetfillcolor{currentfill}%
\pgfsetlinewidth{1.003750pt}%
\definecolor{currentstroke}{rgb}{0.121569,0.466667,0.705882}%
\pgfsetstrokecolor{currentstroke}%
\pgfsetdash{}{0pt}%
\pgfpathmoveto{\pgfqpoint{0.902789in}{0.952515in}}%
\pgfpathcurveto{\pgfqpoint{0.908613in}{0.952515in}}{\pgfqpoint{0.914199in}{0.954829in}}{\pgfqpoint{0.918317in}{0.958947in}}%
\pgfpathcurveto{\pgfqpoint{0.922436in}{0.963065in}}{\pgfqpoint{0.924749in}{0.968651in}}{\pgfqpoint{0.924749in}{0.974475in}}%
\pgfpathcurveto{\pgfqpoint{0.924749in}{0.980299in}}{\pgfqpoint{0.922436in}{0.985885in}}{\pgfqpoint{0.918317in}{0.990003in}}%
\pgfpathcurveto{\pgfqpoint{0.914199in}{0.994121in}}{\pgfqpoint{0.908613in}{0.996435in}}{\pgfqpoint{0.902789in}{0.996435in}}%
\pgfpathcurveto{\pgfqpoint{0.896965in}{0.996435in}}{\pgfqpoint{0.891379in}{0.994121in}}{\pgfqpoint{0.887261in}{0.990003in}}%
\pgfpathcurveto{\pgfqpoint{0.883143in}{0.985885in}}{\pgfqpoint{0.880829in}{0.980299in}}{\pgfqpoint{0.880829in}{0.974475in}}%
\pgfpathcurveto{\pgfqpoint{0.880829in}{0.968651in}}{\pgfqpoint{0.883143in}{0.963065in}}{\pgfqpoint{0.887261in}{0.958947in}}%
\pgfpathcurveto{\pgfqpoint{0.891379in}{0.954829in}}{\pgfqpoint{0.896965in}{0.952515in}}{\pgfqpoint{0.902789in}{0.952515in}}%
\pgfpathclose%
\pgfusepath{stroke,fill}%
\end{pgfscope}%
\begin{pgfscope}%
\pgfpathrectangle{\pgfqpoint{0.211875in}{0.211875in}}{\pgfqpoint{1.313625in}{1.279725in}}%
\pgfusepath{clip}%
\pgfsetbuttcap%
\pgfsetroundjoin%
\definecolor{currentfill}{rgb}{0.121569,0.466667,0.705882}%
\pgfsetfillcolor{currentfill}%
\pgfsetlinewidth{1.003750pt}%
\definecolor{currentstroke}{rgb}{0.121569,0.466667,0.705882}%
\pgfsetstrokecolor{currentstroke}%
\pgfsetdash{}{0pt}%
\pgfpathmoveto{\pgfqpoint{1.044894in}{0.909070in}}%
\pgfpathcurveto{\pgfqpoint{1.050718in}{0.909070in}}{\pgfqpoint{1.056304in}{0.911384in}}{\pgfqpoint{1.060422in}{0.915502in}}%
\pgfpathcurveto{\pgfqpoint{1.064540in}{0.919621in}}{\pgfqpoint{1.066854in}{0.925207in}}{\pgfqpoint{1.066854in}{0.931031in}}%
\pgfpathcurveto{\pgfqpoint{1.066854in}{0.936855in}}{\pgfqpoint{1.064540in}{0.942441in}}{\pgfqpoint{1.060422in}{0.946559in}}%
\pgfpathcurveto{\pgfqpoint{1.056304in}{0.950677in}}{\pgfqpoint{1.050718in}{0.952991in}}{\pgfqpoint{1.044894in}{0.952991in}}%
\pgfpathcurveto{\pgfqpoint{1.039070in}{0.952991in}}{\pgfqpoint{1.033484in}{0.950677in}}{\pgfqpoint{1.029366in}{0.946559in}}%
\pgfpathcurveto{\pgfqpoint{1.025248in}{0.942441in}}{\pgfqpoint{1.022934in}{0.936855in}}{\pgfqpoint{1.022934in}{0.931031in}}%
\pgfpathcurveto{\pgfqpoint{1.022934in}{0.925207in}}{\pgfqpoint{1.025248in}{0.919621in}}{\pgfqpoint{1.029366in}{0.915502in}}%
\pgfpathcurveto{\pgfqpoint{1.033484in}{0.911384in}}{\pgfqpoint{1.039070in}{0.909070in}}{\pgfqpoint{1.044894in}{0.909070in}}%
\pgfpathclose%
\pgfusepath{stroke,fill}%
\end{pgfscope}%
\begin{pgfscope}%
\pgfpathrectangle{\pgfqpoint{0.211875in}{0.211875in}}{\pgfqpoint{1.313625in}{1.279725in}}%
\pgfusepath{clip}%
\pgfsetbuttcap%
\pgfsetroundjoin%
\definecolor{currentfill}{rgb}{0.121569,0.466667,0.705882}%
\pgfsetfillcolor{currentfill}%
\pgfsetlinewidth{1.003750pt}%
\definecolor{currentstroke}{rgb}{0.121569,0.466667,0.705882}%
\pgfsetstrokecolor{currentstroke}%
\pgfsetdash{}{0pt}%
\pgfpathmoveto{\pgfqpoint{0.687521in}{0.680080in}}%
\pgfpathcurveto{\pgfqpoint{0.693345in}{0.680080in}}{\pgfqpoint{0.698931in}{0.682394in}}{\pgfqpoint{0.703049in}{0.686512in}}%
\pgfpathcurveto{\pgfqpoint{0.707167in}{0.690630in}}{\pgfqpoint{0.709481in}{0.696216in}}{\pgfqpoint{0.709481in}{0.702040in}}%
\pgfpathcurveto{\pgfqpoint{0.709481in}{0.707864in}}{\pgfqpoint{0.707167in}{0.713450in}}{\pgfqpoint{0.703049in}{0.717568in}}%
\pgfpathcurveto{\pgfqpoint{0.698931in}{0.721686in}}{\pgfqpoint{0.693345in}{0.724000in}}{\pgfqpoint{0.687521in}{0.724000in}}%
\pgfpathcurveto{\pgfqpoint{0.681697in}{0.724000in}}{\pgfqpoint{0.676111in}{0.721686in}}{\pgfqpoint{0.671993in}{0.717568in}}%
\pgfpathcurveto{\pgfqpoint{0.667875in}{0.713450in}}{\pgfqpoint{0.665561in}{0.707864in}}{\pgfqpoint{0.665561in}{0.702040in}}%
\pgfpathcurveto{\pgfqpoint{0.665561in}{0.696216in}}{\pgfqpoint{0.667875in}{0.690630in}}{\pgfqpoint{0.671993in}{0.686512in}}%
\pgfpathcurveto{\pgfqpoint{0.676111in}{0.682394in}}{\pgfqpoint{0.681697in}{0.680080in}}{\pgfqpoint{0.687521in}{0.680080in}}%
\pgfpathclose%
\pgfusepath{stroke,fill}%
\end{pgfscope}%
\begin{pgfscope}%
\pgfpathrectangle{\pgfqpoint{0.211875in}{0.211875in}}{\pgfqpoint{1.313625in}{1.279725in}}%
\pgfusepath{clip}%
\pgfsetbuttcap%
\pgfsetroundjoin%
\definecolor{currentfill}{rgb}{0.121569,0.466667,0.705882}%
\pgfsetfillcolor{currentfill}%
\pgfsetlinewidth{1.003750pt}%
\definecolor{currentstroke}{rgb}{0.121569,0.466667,0.705882}%
\pgfsetstrokecolor{currentstroke}%
\pgfsetdash{}{0pt}%
\pgfpathmoveto{\pgfqpoint{1.033753in}{0.901046in}}%
\pgfpathcurveto{\pgfqpoint{1.039577in}{0.901046in}}{\pgfqpoint{1.045163in}{0.903360in}}{\pgfqpoint{1.049281in}{0.907478in}}%
\pgfpathcurveto{\pgfqpoint{1.053400in}{0.911596in}}{\pgfqpoint{1.055713in}{0.917183in}}{\pgfqpoint{1.055713in}{0.923006in}}%
\pgfpathcurveto{\pgfqpoint{1.055713in}{0.928830in}}{\pgfqpoint{1.053400in}{0.934417in}}{\pgfqpoint{1.049281in}{0.938535in}}%
\pgfpathcurveto{\pgfqpoint{1.045163in}{0.942653in}}{\pgfqpoint{1.039577in}{0.944967in}}{\pgfqpoint{1.033753in}{0.944967in}}%
\pgfpathcurveto{\pgfqpoint{1.027929in}{0.944967in}}{\pgfqpoint{1.022343in}{0.942653in}}{\pgfqpoint{1.018225in}{0.938535in}}%
\pgfpathcurveto{\pgfqpoint{1.014107in}{0.934417in}}{\pgfqpoint{1.011793in}{0.928830in}}{\pgfqpoint{1.011793in}{0.923006in}}%
\pgfpathcurveto{\pgfqpoint{1.011793in}{0.917183in}}{\pgfqpoint{1.014107in}{0.911596in}}{\pgfqpoint{1.018225in}{0.907478in}}%
\pgfpathcurveto{\pgfqpoint{1.022343in}{0.903360in}}{\pgfqpoint{1.027929in}{0.901046in}}{\pgfqpoint{1.033753in}{0.901046in}}%
\pgfpathclose%
\pgfusepath{stroke,fill}%
\end{pgfscope}%
\begin{pgfscope}%
\pgfpathrectangle{\pgfqpoint{0.211875in}{0.211875in}}{\pgfqpoint{1.313625in}{1.279725in}}%
\pgfusepath{clip}%
\pgfsetbuttcap%
\pgfsetroundjoin%
\definecolor{currentfill}{rgb}{0.121569,0.466667,0.705882}%
\pgfsetfillcolor{currentfill}%
\pgfsetlinewidth{1.003750pt}%
\definecolor{currentstroke}{rgb}{0.121569,0.466667,0.705882}%
\pgfsetstrokecolor{currentstroke}%
\pgfsetdash{}{0pt}%
\pgfpathmoveto{\pgfqpoint{1.036642in}{0.904902in}}%
\pgfpathcurveto{\pgfqpoint{1.042466in}{0.904902in}}{\pgfqpoint{1.048052in}{0.907216in}}{\pgfqpoint{1.052170in}{0.911334in}}%
\pgfpathcurveto{\pgfqpoint{1.056288in}{0.915452in}}{\pgfqpoint{1.058602in}{0.921039in}}{\pgfqpoint{1.058602in}{0.926863in}}%
\pgfpathcurveto{\pgfqpoint{1.058602in}{0.932687in}}{\pgfqpoint{1.056288in}{0.938273in}}{\pgfqpoint{1.052170in}{0.942391in}}%
\pgfpathcurveto{\pgfqpoint{1.048052in}{0.946509in}}{\pgfqpoint{1.042466in}{0.948823in}}{\pgfqpoint{1.036642in}{0.948823in}}%
\pgfpathcurveto{\pgfqpoint{1.030818in}{0.948823in}}{\pgfqpoint{1.025232in}{0.946509in}}{\pgfqpoint{1.021114in}{0.942391in}}%
\pgfpathcurveto{\pgfqpoint{1.016995in}{0.938273in}}{\pgfqpoint{1.014682in}{0.932687in}}{\pgfqpoint{1.014682in}{0.926863in}}%
\pgfpathcurveto{\pgfqpoint{1.014682in}{0.921039in}}{\pgfqpoint{1.016995in}{0.915452in}}{\pgfqpoint{1.021114in}{0.911334in}}%
\pgfpathcurveto{\pgfqpoint{1.025232in}{0.907216in}}{\pgfqpoint{1.030818in}{0.904902in}}{\pgfqpoint{1.036642in}{0.904902in}}%
\pgfpathclose%
\pgfusepath{stroke,fill}%
\end{pgfscope}%
\begin{pgfscope}%
\pgfpathrectangle{\pgfqpoint{0.211875in}{0.211875in}}{\pgfqpoint{1.313625in}{1.279725in}}%
\pgfusepath{clip}%
\pgfsetbuttcap%
\pgfsetroundjoin%
\definecolor{currentfill}{rgb}{0.121569,0.466667,0.705882}%
\pgfsetfillcolor{currentfill}%
\pgfsetlinewidth{1.003750pt}%
\definecolor{currentstroke}{rgb}{0.121569,0.466667,0.705882}%
\pgfsetstrokecolor{currentstroke}%
\pgfsetdash{}{0pt}%
\pgfpathmoveto{\pgfqpoint{1.038227in}{0.906931in}}%
\pgfpathcurveto{\pgfqpoint{1.044051in}{0.906931in}}{\pgfqpoint{1.049637in}{0.909245in}}{\pgfqpoint{1.053755in}{0.913363in}}%
\pgfpathcurveto{\pgfqpoint{1.057873in}{0.917481in}}{\pgfqpoint{1.060187in}{0.923068in}}{\pgfqpoint{1.060187in}{0.928891in}}%
\pgfpathcurveto{\pgfqpoint{1.060187in}{0.934715in}}{\pgfqpoint{1.057873in}{0.940302in}}{\pgfqpoint{1.053755in}{0.944420in}}%
\pgfpathcurveto{\pgfqpoint{1.049637in}{0.948538in}}{\pgfqpoint{1.044051in}{0.950852in}}{\pgfqpoint{1.038227in}{0.950852in}}%
\pgfpathcurveto{\pgfqpoint{1.032403in}{0.950852in}}{\pgfqpoint{1.026817in}{0.948538in}}{\pgfqpoint{1.022699in}{0.944420in}}%
\pgfpathcurveto{\pgfqpoint{1.018581in}{0.940302in}}{\pgfqpoint{1.016267in}{0.934715in}}{\pgfqpoint{1.016267in}{0.928891in}}%
\pgfpathcurveto{\pgfqpoint{1.016267in}{0.923068in}}{\pgfqpoint{1.018581in}{0.917481in}}{\pgfqpoint{1.022699in}{0.913363in}}%
\pgfpathcurveto{\pgfqpoint{1.026817in}{0.909245in}}{\pgfqpoint{1.032403in}{0.906931in}}{\pgfqpoint{1.038227in}{0.906931in}}%
\pgfpathclose%
\pgfusepath{stroke,fill}%
\end{pgfscope}%
\begin{pgfscope}%
\pgfpathrectangle{\pgfqpoint{0.211875in}{0.211875in}}{\pgfqpoint{1.313625in}{1.279725in}}%
\pgfusepath{clip}%
\pgfsetbuttcap%
\pgfsetroundjoin%
\definecolor{currentfill}{rgb}{0.121569,0.466667,0.705882}%
\pgfsetfillcolor{currentfill}%
\pgfsetlinewidth{1.003750pt}%
\definecolor{currentstroke}{rgb}{0.121569,0.466667,0.705882}%
\pgfsetstrokecolor{currentstroke}%
\pgfsetdash{}{0pt}%
\pgfpathmoveto{\pgfqpoint{1.039000in}{0.907849in}}%
\pgfpathcurveto{\pgfqpoint{1.044824in}{0.907849in}}{\pgfqpoint{1.050410in}{0.910163in}}{\pgfqpoint{1.054528in}{0.914281in}}%
\pgfpathcurveto{\pgfqpoint{1.058646in}{0.918399in}}{\pgfqpoint{1.060960in}{0.923986in}}{\pgfqpoint{1.060960in}{0.929810in}}%
\pgfpathcurveto{\pgfqpoint{1.060960in}{0.935633in}}{\pgfqpoint{1.058646in}{0.941220in}}{\pgfqpoint{1.054528in}{0.945338in}}%
\pgfpathcurveto{\pgfqpoint{1.050410in}{0.949456in}}{\pgfqpoint{1.044824in}{0.951770in}}{\pgfqpoint{1.039000in}{0.951770in}}%
\pgfpathcurveto{\pgfqpoint{1.033176in}{0.951770in}}{\pgfqpoint{1.027590in}{0.949456in}}{\pgfqpoint{1.023472in}{0.945338in}}%
\pgfpathcurveto{\pgfqpoint{1.019353in}{0.941220in}}{\pgfqpoint{1.017040in}{0.935633in}}{\pgfqpoint{1.017040in}{0.929810in}}%
\pgfpathcurveto{\pgfqpoint{1.017040in}{0.923986in}}{\pgfqpoint{1.019353in}{0.918399in}}{\pgfqpoint{1.023472in}{0.914281in}}%
\pgfpathcurveto{\pgfqpoint{1.027590in}{0.910163in}}{\pgfqpoint{1.033176in}{0.907849in}}{\pgfqpoint{1.039000in}{0.907849in}}%
\pgfpathclose%
\pgfusepath{stroke,fill}%
\end{pgfscope}%
\begin{pgfscope}%
\pgfsetrectcap%
\pgfsetmiterjoin%
\pgfsetlinewidth{0.000000pt}%
\definecolor{currentstroke}{rgb}{1.000000,1.000000,1.000000}%
\pgfsetstrokecolor{currentstroke}%
\pgfsetdash{}{0pt}%
\pgfpathmoveto{\pgfqpoint{0.211875in}{0.211875in}}%
\pgfpathlineto{\pgfqpoint{0.211875in}{1.491600in}}%
\pgfusepath{}%
\end{pgfscope}%
\begin{pgfscope}%
\pgfsetrectcap%
\pgfsetmiterjoin%
\pgfsetlinewidth{0.000000pt}%
\definecolor{currentstroke}{rgb}{1.000000,1.000000,1.000000}%
\pgfsetstrokecolor{currentstroke}%
\pgfsetdash{}{0pt}%
\pgfpathmoveto{\pgfqpoint{1.525500in}{0.211875in}}%
\pgfpathlineto{\pgfqpoint{1.525500in}{1.491600in}}%
\pgfusepath{}%
\end{pgfscope}%
\begin{pgfscope}%
\pgfsetrectcap%
\pgfsetmiterjoin%
\pgfsetlinewidth{0.000000pt}%
\definecolor{currentstroke}{rgb}{1.000000,1.000000,1.000000}%
\pgfsetstrokecolor{currentstroke}%
\pgfsetdash{}{0pt}%
\pgfpathmoveto{\pgfqpoint{0.211875in}{0.211875in}}%
\pgfpathlineto{\pgfqpoint{1.525500in}{0.211875in}}%
\pgfusepath{}%
\end{pgfscope}%
\begin{pgfscope}%
\pgfsetrectcap%
\pgfsetmiterjoin%
\pgfsetlinewidth{0.000000pt}%
\definecolor{currentstroke}{rgb}{1.000000,1.000000,1.000000}%
\pgfsetstrokecolor{currentstroke}%
\pgfsetdash{}{0pt}%
\pgfpathmoveto{\pgfqpoint{0.211875in}{1.491600in}}%
\pgfpathlineto{\pgfqpoint{1.525500in}{1.491600in}}%
\pgfusepath{}%
\end{pgfscope}%
\end{pgfpicture}%
\makeatother%
\endgroup%

            \end{minipage}
            \begin{minipage}{0.45\linewidth}
                %% Creator: Matplotlib, PGF backend
%%
%% To include the figure in your LaTeX document, write
%%   \input{<filename>.pgf}
%%
%% Make sure the required packages are loaded in your preamble
%%   \usepackage{pgf}
%%
%% Figures using additional raster images can only be included by \input if
%% they are in the same directory as the main LaTeX file. For loading figures
%% from other directories you can use the `import` package
%%   \usepackage{import}
%% and then include the figures with
%%   \import{<path to file>}{<filename>.pgf}
%%
%% Matplotlib used the following preamble
%%   \usepackage{gensymb}
%%   \usepackage{fontspec}
%%   \setmainfont{DejaVu Serif}
%%   \setsansfont{Arial}
%%   \setmonofont{DejaVu Sans Mono}
%%
\begingroup%
\makeatletter%
\begin{pgfpicture}%
\pgfpathrectangle{\pgfpointorigin}{\pgfqpoint{1.695000in}{1.695000in}}%
\pgfusepath{use as bounding box, clip}%
\begin{pgfscope}%
\pgfsetbuttcap%
\pgfsetmiterjoin%
\definecolor{currentfill}{rgb}{1.000000,1.000000,1.000000}%
\pgfsetfillcolor{currentfill}%
\pgfsetlinewidth{0.000000pt}%
\definecolor{currentstroke}{rgb}{1.000000,1.000000,1.000000}%
\pgfsetstrokecolor{currentstroke}%
\pgfsetdash{}{0pt}%
\pgfpathmoveto{\pgfqpoint{0.000000in}{0.000000in}}%
\pgfpathlineto{\pgfqpoint{1.695000in}{0.000000in}}%
\pgfpathlineto{\pgfqpoint{1.695000in}{1.695000in}}%
\pgfpathlineto{\pgfqpoint{0.000000in}{1.695000in}}%
\pgfpathclose%
\pgfusepath{fill}%
\end{pgfscope}%
\begin{pgfscope}%
\pgfsetbuttcap%
\pgfsetmiterjoin%
\definecolor{currentfill}{rgb}{0.917647,0.917647,0.949020}%
\pgfsetfillcolor{currentfill}%
\pgfsetlinewidth{0.000000pt}%
\definecolor{currentstroke}{rgb}{0.000000,0.000000,0.000000}%
\pgfsetstrokecolor{currentstroke}%
\pgfsetstrokeopacity{0.000000}%
\pgfsetdash{}{0pt}%
\pgfpathmoveto{\pgfqpoint{0.211875in}{0.211875in}}%
\pgfpathlineto{\pgfqpoint{1.525500in}{0.211875in}}%
\pgfpathlineto{\pgfqpoint{1.525500in}{1.491600in}}%
\pgfpathlineto{\pgfqpoint{0.211875in}{1.491600in}}%
\pgfpathclose%
\pgfusepath{fill}%
\end{pgfscope}%
\begin{pgfscope}%
\pgfpathrectangle{\pgfqpoint{0.211875in}{0.211875in}}{\pgfqpoint{1.313625in}{1.279725in}}%
\pgfusepath{clip}%
\pgfsetroundcap%
\pgfsetroundjoin%
\pgfsetlinewidth{0.803000pt}%
\definecolor{currentstroke}{rgb}{1.000000,1.000000,1.000000}%
\pgfsetstrokecolor{currentstroke}%
\pgfsetdash{}{0pt}%
\pgfpathmoveto{\pgfqpoint{0.284884in}{0.211875in}}%
\pgfpathlineto{\pgfqpoint{0.284884in}{1.491600in}}%
\pgfusepath{stroke}%
\end{pgfscope}%
\begin{pgfscope}%
\definecolor{textcolor}{rgb}{0.150000,0.150000,0.150000}%
\pgfsetstrokecolor{textcolor}%
\pgfsetfillcolor{textcolor}%
\pgftext[x=0.284884in,y=0.163264in,,top]{\color{textcolor}\rmfamily\fontsize{8.000000}{9.600000}\selectfont \(\displaystyle -50\)}%
\end{pgfscope}%
\begin{pgfscope}%
\pgfpathrectangle{\pgfqpoint{0.211875in}{0.211875in}}{\pgfqpoint{1.313625in}{1.279725in}}%
\pgfusepath{clip}%
\pgfsetroundcap%
\pgfsetroundjoin%
\pgfsetlinewidth{0.803000pt}%
\definecolor{currentstroke}{rgb}{1.000000,1.000000,1.000000}%
\pgfsetstrokecolor{currentstroke}%
\pgfsetdash{}{0pt}%
\pgfpathmoveto{\pgfqpoint{1.114519in}{0.211875in}}%
\pgfpathlineto{\pgfqpoint{1.114519in}{1.491600in}}%
\pgfusepath{stroke}%
\end{pgfscope}%
\begin{pgfscope}%
\definecolor{textcolor}{rgb}{0.150000,0.150000,0.150000}%
\pgfsetstrokecolor{textcolor}%
\pgfsetfillcolor{textcolor}%
\pgftext[x=1.114519in,y=0.163264in,,top]{\color{textcolor}\rmfamily\fontsize{8.000000}{9.600000}\selectfont \(\displaystyle 0\)}%
\end{pgfscope}%
\begin{pgfscope}%
\pgfpathrectangle{\pgfqpoint{0.211875in}{0.211875in}}{\pgfqpoint{1.313625in}{1.279725in}}%
\pgfusepath{clip}%
\pgfsetroundcap%
\pgfsetroundjoin%
\pgfsetlinewidth{0.803000pt}%
\definecolor{currentstroke}{rgb}{1.000000,1.000000,1.000000}%
\pgfsetstrokecolor{currentstroke}%
\pgfsetdash{}{0pt}%
\pgfpathmoveto{\pgfqpoint{0.211875in}{0.445759in}}%
\pgfpathlineto{\pgfqpoint{1.525500in}{0.445759in}}%
\pgfusepath{stroke}%
\end{pgfscope}%
\begin{pgfscope}%
\definecolor{textcolor}{rgb}{0.150000,0.150000,0.150000}%
\pgfsetstrokecolor{textcolor}%
\pgfsetfillcolor{textcolor}%
\pgftext[x=-0.046616in,y=0.403550in,left,base]{\color{textcolor}\rmfamily\fontsize{8.000000}{9.600000}\selectfont \(\displaystyle -40\)}%
\end{pgfscope}%
\begin{pgfscope}%
\pgfpathrectangle{\pgfqpoint{0.211875in}{0.211875in}}{\pgfqpoint{1.313625in}{1.279725in}}%
\pgfusepath{clip}%
\pgfsetroundcap%
\pgfsetroundjoin%
\pgfsetlinewidth{0.803000pt}%
\definecolor{currentstroke}{rgb}{1.000000,1.000000,1.000000}%
\pgfsetstrokecolor{currentstroke}%
\pgfsetdash{}{0pt}%
\pgfpathmoveto{\pgfqpoint{0.211875in}{0.768521in}}%
\pgfpathlineto{\pgfqpoint{1.525500in}{0.768521in}}%
\pgfusepath{stroke}%
\end{pgfscope}%
\begin{pgfscope}%
\definecolor{textcolor}{rgb}{0.150000,0.150000,0.150000}%
\pgfsetstrokecolor{textcolor}%
\pgfsetfillcolor{textcolor}%
\pgftext[x=-0.046616in,y=0.726312in,left,base]{\color{textcolor}\rmfamily\fontsize{8.000000}{9.600000}\selectfont \(\displaystyle -20\)}%
\end{pgfscope}%
\begin{pgfscope}%
\pgfpathrectangle{\pgfqpoint{0.211875in}{0.211875in}}{\pgfqpoint{1.313625in}{1.279725in}}%
\pgfusepath{clip}%
\pgfsetroundcap%
\pgfsetroundjoin%
\pgfsetlinewidth{0.803000pt}%
\definecolor{currentstroke}{rgb}{1.000000,1.000000,1.000000}%
\pgfsetstrokecolor{currentstroke}%
\pgfsetdash{}{0pt}%
\pgfpathmoveto{\pgfqpoint{0.211875in}{1.091283in}}%
\pgfpathlineto{\pgfqpoint{1.525500in}{1.091283in}}%
\pgfusepath{stroke}%
\end{pgfscope}%
\begin{pgfscope}%
\definecolor{textcolor}{rgb}{0.150000,0.150000,0.150000}%
\pgfsetstrokecolor{textcolor}%
\pgfsetfillcolor{textcolor}%
\pgftext[x=0.104235in,y=1.049074in,left,base]{\color{textcolor}\rmfamily\fontsize{8.000000}{9.600000}\selectfont \(\displaystyle 0\)}%
\end{pgfscope}%
\begin{pgfscope}%
\pgfpathrectangle{\pgfqpoint{0.211875in}{0.211875in}}{\pgfqpoint{1.313625in}{1.279725in}}%
\pgfusepath{clip}%
\pgfsetroundcap%
\pgfsetroundjoin%
\pgfsetlinewidth{0.803000pt}%
\definecolor{currentstroke}{rgb}{1.000000,1.000000,1.000000}%
\pgfsetstrokecolor{currentstroke}%
\pgfsetdash{}{0pt}%
\pgfpathmoveto{\pgfqpoint{0.211875in}{1.414045in}}%
\pgfpathlineto{\pgfqpoint{1.525500in}{1.414045in}}%
\pgfusepath{stroke}%
\end{pgfscope}%
\begin{pgfscope}%
\definecolor{textcolor}{rgb}{0.150000,0.150000,0.150000}%
\pgfsetstrokecolor{textcolor}%
\pgfsetfillcolor{textcolor}%
\pgftext[x=0.045207in,y=1.371836in,left,base]{\color{textcolor}\rmfamily\fontsize{8.000000}{9.600000}\selectfont \(\displaystyle 20\)}%
\end{pgfscope}%
\begin{pgfscope}%
\pgfpathrectangle{\pgfqpoint{0.211875in}{0.211875in}}{\pgfqpoint{1.313625in}{1.279725in}}%
\pgfusepath{clip}%
\pgfsetbuttcap%
\pgfsetroundjoin%
\definecolor{currentfill}{rgb}{0.061853,0.045049,0.138049}%
\pgfsetfillcolor{currentfill}%
\pgfsetlinewidth{0.000000pt}%
\definecolor{currentstroke}{rgb}{0.000000,0.000000,0.000000}%
\pgfsetstrokecolor{currentstroke}%
\pgfsetdash{}{0pt}%
\pgfpathmoveto{\pgfqpoint{0.287266in}{0.706577in}}%
\pgfpathlineto{\pgfqpoint{0.284884in}{0.708156in}}%
\pgfpathlineto{\pgfqpoint{0.284884in}{0.706577in}}%
\pgfpathlineto{\pgfqpoint{0.284884in}{0.705576in}}%
\pgfpathclose%
\pgfusepath{fill}%
\end{pgfscope}%
\begin{pgfscope}%
\pgfpathrectangle{\pgfqpoint{0.211875in}{0.211875in}}{\pgfqpoint{1.313625in}{1.279725in}}%
\pgfusepath{clip}%
\pgfsetbuttcap%
\pgfsetroundjoin%
\definecolor{currentfill}{rgb}{0.061853,0.045049,0.138049}%
\pgfsetfillcolor{currentfill}%
\pgfsetlinewidth{0.000000pt}%
\definecolor{currentstroke}{rgb}{0.000000,0.000000,0.000000}%
\pgfsetstrokecolor{currentstroke}%
\pgfsetdash{}{0pt}%
\pgfpathmoveto{\pgfqpoint{0.801101in}{0.966912in}}%
\pgfpathlineto{\pgfqpoint{0.804302in}{0.969025in}}%
\pgfpathlineto{\pgfqpoint{0.803548in}{0.980436in}}%
\pgfpathlineto{\pgfqpoint{0.801101in}{0.984429in}}%
\pgfpathlineto{\pgfqpoint{0.797158in}{0.980436in}}%
\pgfpathlineto{\pgfqpoint{0.799205in}{0.969025in}}%
\pgfpathclose%
\pgfusepath{fill}%
\end{pgfscope}%
\begin{pgfscope}%
\pgfpathrectangle{\pgfqpoint{0.211875in}{0.211875in}}{\pgfqpoint{1.313625in}{1.279725in}}%
\pgfusepath{clip}%
\pgfsetbuttcap%
\pgfsetroundjoin%
\definecolor{currentfill}{rgb}{0.061853,0.045049,0.138049}%
\pgfsetfillcolor{currentfill}%
\pgfsetlinewidth{0.000000pt}%
\definecolor{currentstroke}{rgb}{0.000000,0.000000,0.000000}%
\pgfsetstrokecolor{currentstroke}%
\pgfsetdash{}{0pt}%
\pgfpathmoveto{\pgfqpoint{1.235193in}{1.058017in}}%
\pgfpathlineto{\pgfqpoint{1.246925in}{1.055637in}}%
\pgfpathlineto{\pgfqpoint{1.258658in}{1.055710in}}%
\pgfpathlineto{\pgfqpoint{1.267918in}{1.060311in}}%
\pgfpathlineto{\pgfqpoint{1.258781in}{1.071722in}}%
\pgfpathlineto{\pgfqpoint{1.258658in}{1.071767in}}%
\pgfpathlineto{\pgfqpoint{1.246925in}{1.075459in}}%
\pgfpathlineto{\pgfqpoint{1.235193in}{1.079815in}}%
\pgfpathlineto{\pgfqpoint{1.226024in}{1.083133in}}%
\pgfpathlineto{\pgfqpoint{1.223461in}{1.083710in}}%
\pgfpathlineto{\pgfqpoint{1.211729in}{1.085470in}}%
\pgfpathlineto{\pgfqpoint{1.199996in}{1.085321in}}%
\pgfpathlineto{\pgfqpoint{1.195114in}{1.083133in}}%
\pgfpathlineto{\pgfqpoint{1.199996in}{1.077013in}}%
\pgfpathlineto{\pgfqpoint{1.203912in}{1.071722in}}%
\pgfpathlineto{\pgfqpoint{1.211729in}{1.068719in}}%
\pgfpathlineto{\pgfqpoint{1.223461in}{1.062992in}}%
\pgfpathlineto{\pgfqpoint{1.227547in}{1.060311in}}%
\pgfpathclose%
\pgfusepath{fill}%
\end{pgfscope}%
\begin{pgfscope}%
\pgfpathrectangle{\pgfqpoint{0.211875in}{0.211875in}}{\pgfqpoint{1.313625in}{1.279725in}}%
\pgfusepath{clip}%
\pgfsetbuttcap%
\pgfsetroundjoin%
\definecolor{currentfill}{rgb}{0.186433,0.091654,0.226997}%
\pgfsetfillcolor{currentfill}%
\pgfsetlinewidth{0.000000pt}%
\definecolor{currentstroke}{rgb}{0.000000,0.000000,0.000000}%
\pgfsetstrokecolor{currentstroke}%
\pgfsetdash{}{0pt}%
\pgfpathmoveto{\pgfqpoint{0.496064in}{0.284378in}}%
\pgfpathlineto{\pgfqpoint{0.507796in}{0.284378in}}%
\pgfpathlineto{\pgfqpoint{0.519528in}{0.284378in}}%
\pgfpathlineto{\pgfqpoint{0.531261in}{0.284378in}}%
\pgfpathlineto{\pgfqpoint{0.542993in}{0.284378in}}%
\pgfpathlineto{\pgfqpoint{0.554725in}{0.284378in}}%
\pgfpathlineto{\pgfqpoint{0.566457in}{0.284378in}}%
\pgfpathlineto{\pgfqpoint{0.578190in}{0.284378in}}%
\pgfpathlineto{\pgfqpoint{0.589922in}{0.284378in}}%
\pgfpathlineto{\pgfqpoint{0.601654in}{0.284378in}}%
\pgfpathlineto{\pgfqpoint{0.613386in}{0.284378in}}%
\pgfpathlineto{\pgfqpoint{0.625118in}{0.284378in}}%
\pgfpathlineto{\pgfqpoint{0.636851in}{0.284378in}}%
\pgfpathlineto{\pgfqpoint{0.648583in}{0.284378in}}%
\pgfpathlineto{\pgfqpoint{0.660315in}{0.284378in}}%
\pgfpathlineto{\pgfqpoint{0.672047in}{0.284378in}}%
\pgfpathlineto{\pgfqpoint{0.683779in}{0.284378in}}%
\pgfpathlineto{\pgfqpoint{0.695512in}{0.284378in}}%
\pgfpathlineto{\pgfqpoint{0.707244in}{0.284378in}}%
\pgfpathlineto{\pgfqpoint{0.718976in}{0.284378in}}%
\pgfpathlineto{\pgfqpoint{0.730708in}{0.284378in}}%
\pgfpathlineto{\pgfqpoint{0.742440in}{0.284378in}}%
\pgfpathlineto{\pgfqpoint{0.754173in}{0.284378in}}%
\pgfpathlineto{\pgfqpoint{0.764017in}{0.284378in}}%
\pgfpathlineto{\pgfqpoint{0.763582in}{0.295789in}}%
\pgfpathlineto{\pgfqpoint{0.762945in}{0.307200in}}%
\pgfpathlineto{\pgfqpoint{0.761759in}{0.318611in}}%
\pgfpathlineto{\pgfqpoint{0.760267in}{0.330022in}}%
\pgfpathlineto{\pgfqpoint{0.758722in}{0.341432in}}%
\pgfpathlineto{\pgfqpoint{0.757194in}{0.352843in}}%
\pgfpathlineto{\pgfqpoint{0.755639in}{0.364254in}}%
\pgfpathlineto{\pgfqpoint{0.754173in}{0.374740in}}%
\pgfpathlineto{\pgfqpoint{0.754054in}{0.375665in}}%
\pgfpathlineto{\pgfqpoint{0.752778in}{0.387075in}}%
\pgfpathlineto{\pgfqpoint{0.751990in}{0.398486in}}%
\pgfpathlineto{\pgfqpoint{0.751106in}{0.409897in}}%
\pgfpathlineto{\pgfqpoint{0.750132in}{0.421308in}}%
\pgfpathlineto{\pgfqpoint{0.749057in}{0.432719in}}%
\pgfpathlineto{\pgfqpoint{0.747877in}{0.444129in}}%
\pgfpathlineto{\pgfqpoint{0.746596in}{0.455540in}}%
\pgfpathlineto{\pgfqpoint{0.745209in}{0.466951in}}%
\pgfpathlineto{\pgfqpoint{0.743693in}{0.478362in}}%
\pgfpathlineto{\pgfqpoint{0.742440in}{0.487053in}}%
\pgfpathlineto{\pgfqpoint{0.742025in}{0.489772in}}%
\pgfpathlineto{\pgfqpoint{0.740281in}{0.501183in}}%
\pgfpathlineto{\pgfqpoint{0.738949in}{0.512594in}}%
\pgfpathlineto{\pgfqpoint{0.737599in}{0.524005in}}%
\pgfpathlineto{\pgfqpoint{0.736020in}{0.535416in}}%
\pgfpathlineto{\pgfqpoint{0.734189in}{0.546826in}}%
\pgfpathlineto{\pgfqpoint{0.731485in}{0.558237in}}%
\pgfpathlineto{\pgfqpoint{0.730708in}{0.560205in}}%
\pgfpathlineto{\pgfqpoint{0.724915in}{0.569648in}}%
\pgfpathlineto{\pgfqpoint{0.718976in}{0.576331in}}%
\pgfpathlineto{\pgfqpoint{0.711012in}{0.581059in}}%
\pgfpathlineto{\pgfqpoint{0.707244in}{0.583185in}}%
\pgfpathlineto{\pgfqpoint{0.703602in}{0.592469in}}%
\pgfpathlineto{\pgfqpoint{0.699302in}{0.603880in}}%
\pgfpathlineto{\pgfqpoint{0.695512in}{0.610706in}}%
\pgfpathlineto{\pgfqpoint{0.687139in}{0.615291in}}%
\pgfpathlineto{\pgfqpoint{0.683779in}{0.616188in}}%
\pgfpathlineto{\pgfqpoint{0.682681in}{0.615291in}}%
\pgfpathlineto{\pgfqpoint{0.672047in}{0.608532in}}%
\pgfpathlineto{\pgfqpoint{0.668322in}{0.603880in}}%
\pgfpathlineto{\pgfqpoint{0.660315in}{0.596082in}}%
\pgfpathlineto{\pgfqpoint{0.658029in}{0.592469in}}%
\pgfpathlineto{\pgfqpoint{0.648832in}{0.581059in}}%
\pgfpathlineto{\pgfqpoint{0.648583in}{0.580789in}}%
\pgfpathlineto{\pgfqpoint{0.642205in}{0.569648in}}%
\pgfpathlineto{\pgfqpoint{0.636851in}{0.562676in}}%
\pgfpathlineto{\pgfqpoint{0.634549in}{0.558237in}}%
\pgfpathlineto{\pgfqpoint{0.627143in}{0.546826in}}%
\pgfpathlineto{\pgfqpoint{0.625118in}{0.544100in}}%
\pgfpathlineto{\pgfqpoint{0.620540in}{0.535416in}}%
\pgfpathlineto{\pgfqpoint{0.613386in}{0.524851in}}%
\pgfpathlineto{\pgfqpoint{0.612958in}{0.524005in}}%
\pgfpathlineto{\pgfqpoint{0.607035in}{0.512594in}}%
\pgfpathlineto{\pgfqpoint{0.601654in}{0.504319in}}%
\pgfpathlineto{\pgfqpoint{0.600111in}{0.501183in}}%
\pgfpathlineto{\pgfqpoint{0.594027in}{0.489772in}}%
\pgfpathlineto{\pgfqpoint{0.589922in}{0.483222in}}%
\pgfpathlineto{\pgfqpoint{0.587588in}{0.478362in}}%
\pgfpathlineto{\pgfqpoint{0.581443in}{0.466951in}}%
\pgfpathlineto{\pgfqpoint{0.578190in}{0.461591in}}%
\pgfpathlineto{\pgfqpoint{0.575341in}{0.455540in}}%
\pgfpathlineto{\pgfqpoint{0.569205in}{0.444129in}}%
\pgfpathlineto{\pgfqpoint{0.566457in}{0.439476in}}%
\pgfpathlineto{\pgfqpoint{0.563326in}{0.432719in}}%
\pgfpathlineto{\pgfqpoint{0.557243in}{0.421308in}}%
\pgfpathlineto{\pgfqpoint{0.554725in}{0.416943in}}%
\pgfpathlineto{\pgfqpoint{0.551498in}{0.409897in}}%
\pgfpathlineto{\pgfqpoint{0.545500in}{0.398486in}}%
\pgfpathlineto{\pgfqpoint{0.542993in}{0.394057in}}%
\pgfpathlineto{\pgfqpoint{0.539824in}{0.387075in}}%
\pgfpathlineto{\pgfqpoint{0.533925in}{0.375665in}}%
\pgfpathlineto{\pgfqpoint{0.531261in}{0.370882in}}%
\pgfpathlineto{\pgfqpoint{0.528270in}{0.364254in}}%
\pgfpathlineto{\pgfqpoint{0.522482in}{0.352843in}}%
\pgfpathlineto{\pgfqpoint{0.519528in}{0.347472in}}%
\pgfpathlineto{\pgfqpoint{0.516814in}{0.341432in}}%
\pgfpathlineto{\pgfqpoint{0.511141in}{0.330022in}}%
\pgfpathlineto{\pgfqpoint{0.507796in}{0.323874in}}%
\pgfpathlineto{\pgfqpoint{0.505437in}{0.318611in}}%
\pgfpathlineto{\pgfqpoint{0.499881in}{0.307200in}}%
\pgfpathlineto{\pgfqpoint{0.496064in}{0.300122in}}%
\pgfpathlineto{\pgfqpoint{0.494124in}{0.295789in}}%
\pgfpathlineto{\pgfqpoint{0.488684in}{0.284378in}}%
\pgfpathclose%
\pgfusepath{fill}%
\end{pgfscope}%
\begin{pgfscope}%
\pgfpathrectangle{\pgfqpoint{0.211875in}{0.211875in}}{\pgfqpoint{1.313625in}{1.279725in}}%
\pgfusepath{clip}%
\pgfsetbuttcap%
\pgfsetroundjoin%
\definecolor{currentfill}{rgb}{0.186433,0.091654,0.226997}%
\pgfsetfillcolor{currentfill}%
\pgfsetlinewidth{0.000000pt}%
\definecolor{currentstroke}{rgb}{0.000000,0.000000,0.000000}%
\pgfsetstrokecolor{currentstroke}%
\pgfsetdash{}{0pt}%
\pgfpathmoveto{\pgfqpoint{0.294228in}{0.672345in}}%
\pgfpathlineto{\pgfqpoint{0.296617in}{0.673599in}}%
\pgfpathlineto{\pgfqpoint{0.308349in}{0.679587in}}%
\pgfpathlineto{\pgfqpoint{0.317123in}{0.683756in}}%
\pgfpathlineto{\pgfqpoint{0.320081in}{0.685324in}}%
\pgfpathlineto{\pgfqpoint{0.331813in}{0.691329in}}%
\pgfpathlineto{\pgfqpoint{0.339815in}{0.695166in}}%
\pgfpathlineto{\pgfqpoint{0.343545in}{0.697171in}}%
\pgfpathlineto{\pgfqpoint{0.355278in}{0.703200in}}%
\pgfpathlineto{\pgfqpoint{0.362236in}{0.706577in}}%
\pgfpathlineto{\pgfqpoint{0.367010in}{0.709190in}}%
\pgfpathlineto{\pgfqpoint{0.378742in}{0.715246in}}%
\pgfpathlineto{\pgfqpoint{0.384303in}{0.717988in}}%
\pgfpathlineto{\pgfqpoint{0.390474in}{0.721449in}}%
\pgfpathlineto{\pgfqpoint{0.402206in}{0.727535in}}%
\pgfpathlineto{\pgfqpoint{0.405929in}{0.729399in}}%
\pgfpathlineto{\pgfqpoint{0.413939in}{0.734034in}}%
\pgfpathlineto{\pgfqpoint{0.425671in}{0.740125in}}%
\pgfpathlineto{\pgfqpoint{0.427020in}{0.740810in}}%
\pgfpathlineto{\pgfqpoint{0.437403in}{0.747059in}}%
\pgfpathlineto{\pgfqpoint{0.447206in}{0.752220in}}%
\pgfpathlineto{\pgfqpoint{0.449135in}{0.753515in}}%
\pgfpathlineto{\pgfqpoint{0.460867in}{0.760678in}}%
\pgfpathlineto{\pgfqpoint{0.466262in}{0.763631in}}%
\pgfpathlineto{\pgfqpoint{0.472600in}{0.768257in}}%
\pgfpathlineto{\pgfqpoint{0.484189in}{0.775042in}}%
\pgfpathlineto{\pgfqpoint{0.484332in}{0.775194in}}%
\pgfpathlineto{\pgfqpoint{0.496064in}{0.784698in}}%
\pgfpathlineto{\pgfqpoint{0.498831in}{0.786453in}}%
\pgfpathlineto{\pgfqpoint{0.501944in}{0.797863in}}%
\pgfpathlineto{\pgfqpoint{0.496064in}{0.798382in}}%
\pgfpathlineto{\pgfqpoint{0.489255in}{0.797863in}}%
\pgfpathlineto{\pgfqpoint{0.484332in}{0.796920in}}%
\pgfpathlineto{\pgfqpoint{0.472600in}{0.793660in}}%
\pgfpathlineto{\pgfqpoint{0.460867in}{0.790984in}}%
\pgfpathlineto{\pgfqpoint{0.449135in}{0.788486in}}%
\pgfpathlineto{\pgfqpoint{0.440196in}{0.786453in}}%
\pgfpathlineto{\pgfqpoint{0.437403in}{0.785426in}}%
\pgfpathlineto{\pgfqpoint{0.425671in}{0.781322in}}%
\pgfpathlineto{\pgfqpoint{0.413939in}{0.778060in}}%
\pgfpathlineto{\pgfqpoint{0.402391in}{0.775042in}}%
\pgfpathlineto{\pgfqpoint{0.402206in}{0.774969in}}%
\pgfpathlineto{\pgfqpoint{0.390474in}{0.770429in}}%
\pgfpathlineto{\pgfqpoint{0.378742in}{0.766844in}}%
\pgfpathlineto{\pgfqpoint{0.367500in}{0.763631in}}%
\pgfpathlineto{\pgfqpoint{0.367010in}{0.763431in}}%
\pgfpathlineto{\pgfqpoint{0.355278in}{0.758849in}}%
\pgfpathlineto{\pgfqpoint{0.343545in}{0.755143in}}%
\pgfpathlineto{\pgfqpoint{0.333922in}{0.752220in}}%
\pgfpathlineto{\pgfqpoint{0.331813in}{0.751350in}}%
\pgfpathlineto{\pgfqpoint{0.320081in}{0.746880in}}%
\pgfpathlineto{\pgfqpoint{0.308349in}{0.743141in}}%
\pgfpathlineto{\pgfqpoint{0.301004in}{0.740810in}}%
\pgfpathlineto{\pgfqpoint{0.296617in}{0.738996in}}%
\pgfpathlineto{\pgfqpoint{0.284884in}{0.734687in}}%
\pgfpathlineto{\pgfqpoint{0.284884in}{0.729399in}}%
\pgfpathlineto{\pgfqpoint{0.284884in}{0.717988in}}%
\pgfpathlineto{\pgfqpoint{0.284884in}{0.708156in}}%
\pgfpathlineto{\pgfqpoint{0.287266in}{0.706577in}}%
\pgfpathlineto{\pgfqpoint{0.284884in}{0.705576in}}%
\pgfpathlineto{\pgfqpoint{0.284884in}{0.695166in}}%
\pgfpathlineto{\pgfqpoint{0.284884in}{0.683756in}}%
\pgfpathlineto{\pgfqpoint{0.284884in}{0.672345in}}%
\pgfpathlineto{\pgfqpoint{0.284884in}{0.667936in}}%
\pgfpathclose%
\pgfusepath{fill}%
\end{pgfscope}%
\begin{pgfscope}%
\pgfpathrectangle{\pgfqpoint{0.211875in}{0.211875in}}{\pgfqpoint{1.313625in}{1.279725in}}%
\pgfusepath{clip}%
\pgfsetbuttcap%
\pgfsetroundjoin%
\definecolor{currentfill}{rgb}{0.186433,0.091654,0.226997}%
\pgfsetfillcolor{currentfill}%
\pgfsetlinewidth{0.000000pt}%
\definecolor{currentstroke}{rgb}{0.000000,0.000000,0.000000}%
\pgfsetstrokecolor{currentstroke}%
\pgfsetdash{}{0pt}%
\pgfpathmoveto{\pgfqpoint{0.801101in}{0.772751in}}%
\pgfpathlineto{\pgfqpoint{0.811033in}{0.775042in}}%
\pgfpathlineto{\pgfqpoint{0.812834in}{0.775354in}}%
\pgfpathlineto{\pgfqpoint{0.824260in}{0.786453in}}%
\pgfpathlineto{\pgfqpoint{0.824566in}{0.786707in}}%
\pgfpathlineto{\pgfqpoint{0.829619in}{0.797863in}}%
\pgfpathlineto{\pgfqpoint{0.834975in}{0.809274in}}%
\pgfpathlineto{\pgfqpoint{0.836298in}{0.812398in}}%
\pgfpathlineto{\pgfqpoint{0.837807in}{0.820685in}}%
\pgfpathlineto{\pgfqpoint{0.839245in}{0.832096in}}%
\pgfpathlineto{\pgfqpoint{0.839979in}{0.843507in}}%
\pgfpathlineto{\pgfqpoint{0.839749in}{0.854917in}}%
\pgfpathlineto{\pgfqpoint{0.837157in}{0.866328in}}%
\pgfpathlineto{\pgfqpoint{0.836298in}{0.867417in}}%
\pgfpathlineto{\pgfqpoint{0.834515in}{0.866328in}}%
\pgfpathlineto{\pgfqpoint{0.824566in}{0.861432in}}%
\pgfpathlineto{\pgfqpoint{0.816880in}{0.854917in}}%
\pgfpathlineto{\pgfqpoint{0.812834in}{0.850829in}}%
\pgfpathlineto{\pgfqpoint{0.806115in}{0.843507in}}%
\pgfpathlineto{\pgfqpoint{0.801101in}{0.837000in}}%
\pgfpathlineto{\pgfqpoint{0.797490in}{0.832096in}}%
\pgfpathlineto{\pgfqpoint{0.790598in}{0.820685in}}%
\pgfpathlineto{\pgfqpoint{0.789369in}{0.817761in}}%
\pgfpathlineto{\pgfqpoint{0.786192in}{0.809274in}}%
\pgfpathlineto{\pgfqpoint{0.783786in}{0.797863in}}%
\pgfpathlineto{\pgfqpoint{0.784666in}{0.786453in}}%
\pgfpathlineto{\pgfqpoint{0.789369in}{0.777692in}}%
\pgfpathlineto{\pgfqpoint{0.794999in}{0.775042in}}%
\pgfpathclose%
\pgfusepath{fill}%
\end{pgfscope}%
\begin{pgfscope}%
\pgfpathrectangle{\pgfqpoint{0.211875in}{0.211875in}}{\pgfqpoint{1.313625in}{1.279725in}}%
\pgfusepath{clip}%
\pgfsetbuttcap%
\pgfsetroundjoin%
\definecolor{currentfill}{rgb}{0.186433,0.091654,0.226997}%
\pgfsetfillcolor{currentfill}%
\pgfsetlinewidth{0.000000pt}%
\definecolor{currentstroke}{rgb}{0.000000,0.000000,0.000000}%
\pgfsetstrokecolor{currentstroke}%
\pgfsetdash{}{0pt}%
\pgfpathmoveto{\pgfqpoint{0.871495in}{0.888106in}}%
\pgfpathlineto{\pgfqpoint{0.872710in}{0.889150in}}%
\pgfpathlineto{\pgfqpoint{0.871935in}{0.900560in}}%
\pgfpathlineto{\pgfqpoint{0.871495in}{0.900959in}}%
\pgfpathlineto{\pgfqpoint{0.868146in}{0.900560in}}%
\pgfpathlineto{\pgfqpoint{0.869692in}{0.889150in}}%
\pgfpathclose%
\pgfusepath{fill}%
\end{pgfscope}%
\begin{pgfscope}%
\pgfpathrectangle{\pgfqpoint{0.211875in}{0.211875in}}{\pgfqpoint{1.313625in}{1.279725in}}%
\pgfusepath{clip}%
\pgfsetbuttcap%
\pgfsetroundjoin%
\definecolor{currentfill}{rgb}{0.186433,0.091654,0.226997}%
\pgfsetfillcolor{currentfill}%
\pgfsetlinewidth{0.000000pt}%
\definecolor{currentstroke}{rgb}{0.000000,0.000000,0.000000}%
\pgfsetstrokecolor{currentstroke}%
\pgfsetdash{}{0pt}%
\pgfpathmoveto{\pgfqpoint{0.801101in}{0.956594in}}%
\pgfpathlineto{\pgfqpoint{0.812834in}{0.952770in}}%
\pgfpathlineto{\pgfqpoint{0.818593in}{0.957614in}}%
\pgfpathlineto{\pgfqpoint{0.817001in}{0.969025in}}%
\pgfpathlineto{\pgfqpoint{0.814382in}{0.980436in}}%
\pgfpathlineto{\pgfqpoint{0.812834in}{0.986657in}}%
\pgfpathlineto{\pgfqpoint{0.810341in}{0.991847in}}%
\pgfpathlineto{\pgfqpoint{0.801101in}{1.001528in}}%
\pgfpathlineto{\pgfqpoint{0.789369in}{0.996305in}}%
\pgfpathlineto{\pgfqpoint{0.777637in}{0.993215in}}%
\pgfpathlineto{\pgfqpoint{0.776940in}{0.991847in}}%
\pgfpathlineto{\pgfqpoint{0.777637in}{0.990839in}}%
\pgfpathlineto{\pgfqpoint{0.782858in}{0.980436in}}%
\pgfpathlineto{\pgfqpoint{0.789369in}{0.970017in}}%
\pgfpathlineto{\pgfqpoint{0.789885in}{0.969025in}}%
\pgfpathlineto{\pgfqpoint{0.799822in}{0.957614in}}%
\pgfpathclose%
\pgfpathmoveto{\pgfqpoint{0.799205in}{0.969025in}}%
\pgfpathlineto{\pgfqpoint{0.797158in}{0.980436in}}%
\pgfpathlineto{\pgfqpoint{0.801101in}{0.984429in}}%
\pgfpathlineto{\pgfqpoint{0.803548in}{0.980436in}}%
\pgfpathlineto{\pgfqpoint{0.804302in}{0.969025in}}%
\pgfpathlineto{\pgfqpoint{0.801101in}{0.966912in}}%
\pgfpathclose%
\pgfusepath{fill}%
\end{pgfscope}%
\begin{pgfscope}%
\pgfpathrectangle{\pgfqpoint{0.211875in}{0.211875in}}{\pgfqpoint{1.313625in}{1.279725in}}%
\pgfusepath{clip}%
\pgfsetbuttcap%
\pgfsetroundjoin%
\definecolor{currentfill}{rgb}{0.186433,0.091654,0.226997}%
\pgfsetfillcolor{currentfill}%
\pgfsetlinewidth{0.000000pt}%
\definecolor{currentstroke}{rgb}{0.000000,0.000000,0.000000}%
\pgfsetstrokecolor{currentstroke}%
\pgfsetdash{}{0pt}%
\pgfpathmoveto{\pgfqpoint{1.446373in}{1.026071in}}%
\pgfpathlineto{\pgfqpoint{1.446373in}{1.026079in}}%
\pgfpathlineto{\pgfqpoint{1.446373in}{1.026126in}}%
\pgfpathlineto{\pgfqpoint{1.446367in}{1.026079in}}%
\pgfpathclose%
\pgfusepath{fill}%
\end{pgfscope}%
\begin{pgfscope}%
\pgfpathrectangle{\pgfqpoint{0.211875in}{0.211875in}}{\pgfqpoint{1.313625in}{1.279725in}}%
\pgfusepath{clip}%
\pgfsetbuttcap%
\pgfsetroundjoin%
\definecolor{currentfill}{rgb}{0.186433,0.091654,0.226997}%
\pgfsetfillcolor{currentfill}%
\pgfsetlinewidth{0.000000pt}%
\definecolor{currentstroke}{rgb}{0.000000,0.000000,0.000000}%
\pgfsetstrokecolor{currentstroke}%
\pgfsetdash{}{0pt}%
\pgfpathmoveto{\pgfqpoint{1.246925in}{1.046792in}}%
\pgfpathlineto{\pgfqpoint{1.258658in}{1.046491in}}%
\pgfpathlineto{\pgfqpoint{1.270390in}{1.047583in}}%
\pgfpathlineto{\pgfqpoint{1.276069in}{1.048901in}}%
\pgfpathlineto{\pgfqpoint{1.282122in}{1.057045in}}%
\pgfpathlineto{\pgfqpoint{1.282957in}{1.060311in}}%
\pgfpathlineto{\pgfqpoint{1.282122in}{1.061462in}}%
\pgfpathlineto{\pgfqpoint{1.275390in}{1.071722in}}%
\pgfpathlineto{\pgfqpoint{1.270390in}{1.075122in}}%
\pgfpathlineto{\pgfqpoint{1.258658in}{1.080046in}}%
\pgfpathlineto{\pgfqpoint{1.251702in}{1.083133in}}%
\pgfpathlineto{\pgfqpoint{1.246925in}{1.084806in}}%
\pgfpathlineto{\pgfqpoint{1.235193in}{1.088920in}}%
\pgfpathlineto{\pgfqpoint{1.223461in}{1.092636in}}%
\pgfpathlineto{\pgfqpoint{1.216969in}{1.094544in}}%
\pgfpathlineto{\pgfqpoint{1.211729in}{1.095586in}}%
\pgfpathlineto{\pgfqpoint{1.199996in}{1.096201in}}%
\pgfpathlineto{\pgfqpoint{1.188264in}{1.095308in}}%
\pgfpathlineto{\pgfqpoint{1.185562in}{1.094544in}}%
\pgfpathlineto{\pgfqpoint{1.176532in}{1.085290in}}%
\pgfpathlineto{\pgfqpoint{1.175810in}{1.083133in}}%
\pgfpathlineto{\pgfqpoint{1.176532in}{1.082376in}}%
\pgfpathlineto{\pgfqpoint{1.186428in}{1.071722in}}%
\pgfpathlineto{\pgfqpoint{1.188264in}{1.070798in}}%
\pgfpathlineto{\pgfqpoint{1.199996in}{1.066426in}}%
\pgfpathlineto{\pgfqpoint{1.210139in}{1.060311in}}%
\pgfpathlineto{\pgfqpoint{1.211729in}{1.059537in}}%
\pgfpathlineto{\pgfqpoint{1.223461in}{1.055599in}}%
\pgfpathlineto{\pgfqpoint{1.235193in}{1.050438in}}%
\pgfpathlineto{\pgfqpoint{1.240275in}{1.048901in}}%
\pgfpathclose%
\pgfpathmoveto{\pgfqpoint{1.227547in}{1.060311in}}%
\pgfpathlineto{\pgfqpoint{1.223461in}{1.062992in}}%
\pgfpathlineto{\pgfqpoint{1.211729in}{1.068719in}}%
\pgfpathlineto{\pgfqpoint{1.203912in}{1.071722in}}%
\pgfpathlineto{\pgfqpoint{1.199996in}{1.077013in}}%
\pgfpathlineto{\pgfqpoint{1.195114in}{1.083133in}}%
\pgfpathlineto{\pgfqpoint{1.199996in}{1.085321in}}%
\pgfpathlineto{\pgfqpoint{1.211729in}{1.085470in}}%
\pgfpathlineto{\pgfqpoint{1.223461in}{1.083710in}}%
\pgfpathlineto{\pgfqpoint{1.226024in}{1.083133in}}%
\pgfpathlineto{\pgfqpoint{1.235193in}{1.079815in}}%
\pgfpathlineto{\pgfqpoint{1.246925in}{1.075459in}}%
\pgfpathlineto{\pgfqpoint{1.258658in}{1.071767in}}%
\pgfpathlineto{\pgfqpoint{1.258781in}{1.071722in}}%
\pgfpathlineto{\pgfqpoint{1.267918in}{1.060311in}}%
\pgfpathlineto{\pgfqpoint{1.258658in}{1.055710in}}%
\pgfpathlineto{\pgfqpoint{1.246925in}{1.055637in}}%
\pgfpathlineto{\pgfqpoint{1.235193in}{1.058017in}}%
\pgfpathclose%
\pgfusepath{fill}%
\end{pgfscope}%
\begin{pgfscope}%
\pgfpathrectangle{\pgfqpoint{0.211875in}{0.211875in}}{\pgfqpoint{1.313625in}{1.279725in}}%
\pgfusepath{clip}%
\pgfsetbuttcap%
\pgfsetroundjoin%
\definecolor{currentfill}{rgb}{0.312084,0.115053,0.298245}%
\pgfsetfillcolor{currentfill}%
\pgfsetlinewidth{0.000000pt}%
\definecolor{currentstroke}{rgb}{0.000000,0.000000,0.000000}%
\pgfsetstrokecolor{currentstroke}%
\pgfsetdash{}{0pt}%
\pgfpathmoveto{\pgfqpoint{0.460867in}{0.284378in}}%
\pgfpathlineto{\pgfqpoint{0.472600in}{0.284378in}}%
\pgfpathlineto{\pgfqpoint{0.484332in}{0.284378in}}%
\pgfpathlineto{\pgfqpoint{0.488684in}{0.284378in}}%
\pgfpathlineto{\pgfqpoint{0.494124in}{0.295789in}}%
\pgfpathlineto{\pgfqpoint{0.496064in}{0.300122in}}%
\pgfpathlineto{\pgfqpoint{0.499881in}{0.307200in}}%
\pgfpathlineto{\pgfqpoint{0.505437in}{0.318611in}}%
\pgfpathlineto{\pgfqpoint{0.507796in}{0.323874in}}%
\pgfpathlineto{\pgfqpoint{0.511141in}{0.330022in}}%
\pgfpathlineto{\pgfqpoint{0.516814in}{0.341432in}}%
\pgfpathlineto{\pgfqpoint{0.519528in}{0.347472in}}%
\pgfpathlineto{\pgfqpoint{0.522482in}{0.352843in}}%
\pgfpathlineto{\pgfqpoint{0.528270in}{0.364254in}}%
\pgfpathlineto{\pgfqpoint{0.531261in}{0.370882in}}%
\pgfpathlineto{\pgfqpoint{0.533925in}{0.375665in}}%
\pgfpathlineto{\pgfqpoint{0.539824in}{0.387075in}}%
\pgfpathlineto{\pgfqpoint{0.542993in}{0.394057in}}%
\pgfpathlineto{\pgfqpoint{0.545500in}{0.398486in}}%
\pgfpathlineto{\pgfqpoint{0.551498in}{0.409897in}}%
\pgfpathlineto{\pgfqpoint{0.554725in}{0.416943in}}%
\pgfpathlineto{\pgfqpoint{0.557243in}{0.421308in}}%
\pgfpathlineto{\pgfqpoint{0.563326in}{0.432719in}}%
\pgfpathlineto{\pgfqpoint{0.566457in}{0.439476in}}%
\pgfpathlineto{\pgfqpoint{0.569205in}{0.444129in}}%
\pgfpathlineto{\pgfqpoint{0.575341in}{0.455540in}}%
\pgfpathlineto{\pgfqpoint{0.578190in}{0.461591in}}%
\pgfpathlineto{\pgfqpoint{0.581443in}{0.466951in}}%
\pgfpathlineto{\pgfqpoint{0.587588in}{0.478362in}}%
\pgfpathlineto{\pgfqpoint{0.589922in}{0.483222in}}%
\pgfpathlineto{\pgfqpoint{0.594027in}{0.489772in}}%
\pgfpathlineto{\pgfqpoint{0.600111in}{0.501183in}}%
\pgfpathlineto{\pgfqpoint{0.601654in}{0.504319in}}%
\pgfpathlineto{\pgfqpoint{0.607035in}{0.512594in}}%
\pgfpathlineto{\pgfqpoint{0.612958in}{0.524005in}}%
\pgfpathlineto{\pgfqpoint{0.613386in}{0.524851in}}%
\pgfpathlineto{\pgfqpoint{0.620540in}{0.535416in}}%
\pgfpathlineto{\pgfqpoint{0.625118in}{0.544100in}}%
\pgfpathlineto{\pgfqpoint{0.627143in}{0.546826in}}%
\pgfpathlineto{\pgfqpoint{0.634549in}{0.558237in}}%
\pgfpathlineto{\pgfqpoint{0.636851in}{0.562676in}}%
\pgfpathlineto{\pgfqpoint{0.642205in}{0.569648in}}%
\pgfpathlineto{\pgfqpoint{0.648583in}{0.580789in}}%
\pgfpathlineto{\pgfqpoint{0.648832in}{0.581059in}}%
\pgfpathlineto{\pgfqpoint{0.658029in}{0.592469in}}%
\pgfpathlineto{\pgfqpoint{0.660315in}{0.596082in}}%
\pgfpathlineto{\pgfqpoint{0.668322in}{0.603880in}}%
\pgfpathlineto{\pgfqpoint{0.672047in}{0.608532in}}%
\pgfpathlineto{\pgfqpoint{0.682681in}{0.615291in}}%
\pgfpathlineto{\pgfqpoint{0.683779in}{0.616188in}}%
\pgfpathlineto{\pgfqpoint{0.687139in}{0.615291in}}%
\pgfpathlineto{\pgfqpoint{0.695512in}{0.610706in}}%
\pgfpathlineto{\pgfqpoint{0.699302in}{0.603880in}}%
\pgfpathlineto{\pgfqpoint{0.703602in}{0.592469in}}%
\pgfpathlineto{\pgfqpoint{0.707244in}{0.583185in}}%
\pgfpathlineto{\pgfqpoint{0.711012in}{0.581059in}}%
\pgfpathlineto{\pgfqpoint{0.718976in}{0.576331in}}%
\pgfpathlineto{\pgfqpoint{0.724915in}{0.569648in}}%
\pgfpathlineto{\pgfqpoint{0.730708in}{0.560205in}}%
\pgfpathlineto{\pgfqpoint{0.731485in}{0.558237in}}%
\pgfpathlineto{\pgfqpoint{0.734189in}{0.546826in}}%
\pgfpathlineto{\pgfqpoint{0.736020in}{0.535416in}}%
\pgfpathlineto{\pgfqpoint{0.737599in}{0.524005in}}%
\pgfpathlineto{\pgfqpoint{0.738949in}{0.512594in}}%
\pgfpathlineto{\pgfqpoint{0.740281in}{0.501183in}}%
\pgfpathlineto{\pgfqpoint{0.742025in}{0.489772in}}%
\pgfpathlineto{\pgfqpoint{0.742440in}{0.487053in}}%
\pgfpathlineto{\pgfqpoint{0.743693in}{0.478362in}}%
\pgfpathlineto{\pgfqpoint{0.745209in}{0.466951in}}%
\pgfpathlineto{\pgfqpoint{0.746596in}{0.455540in}}%
\pgfpathlineto{\pgfqpoint{0.747877in}{0.444129in}}%
\pgfpathlineto{\pgfqpoint{0.749057in}{0.432719in}}%
\pgfpathlineto{\pgfqpoint{0.750132in}{0.421308in}}%
\pgfpathlineto{\pgfqpoint{0.751106in}{0.409897in}}%
\pgfpathlineto{\pgfqpoint{0.751990in}{0.398486in}}%
\pgfpathlineto{\pgfqpoint{0.752778in}{0.387075in}}%
\pgfpathlineto{\pgfqpoint{0.754054in}{0.375665in}}%
\pgfpathlineto{\pgfqpoint{0.754173in}{0.374740in}}%
\pgfpathlineto{\pgfqpoint{0.755639in}{0.364254in}}%
\pgfpathlineto{\pgfqpoint{0.757194in}{0.352843in}}%
\pgfpathlineto{\pgfqpoint{0.758722in}{0.341432in}}%
\pgfpathlineto{\pgfqpoint{0.760267in}{0.330022in}}%
\pgfpathlineto{\pgfqpoint{0.761759in}{0.318611in}}%
\pgfpathlineto{\pgfqpoint{0.762945in}{0.307200in}}%
\pgfpathlineto{\pgfqpoint{0.763582in}{0.295789in}}%
\pgfpathlineto{\pgfqpoint{0.764017in}{0.284378in}}%
\pgfpathlineto{\pgfqpoint{0.765905in}{0.284378in}}%
\pgfpathlineto{\pgfqpoint{0.777637in}{0.284378in}}%
\pgfpathlineto{\pgfqpoint{0.789369in}{0.284378in}}%
\pgfpathlineto{\pgfqpoint{0.801101in}{0.284378in}}%
\pgfpathlineto{\pgfqpoint{0.812834in}{0.284378in}}%
\pgfpathlineto{\pgfqpoint{0.824566in}{0.284378in}}%
\pgfpathlineto{\pgfqpoint{0.836298in}{0.284378in}}%
\pgfpathlineto{\pgfqpoint{0.848030in}{0.284378in}}%
\pgfpathlineto{\pgfqpoint{0.859762in}{0.284378in}}%
\pgfpathlineto{\pgfqpoint{0.871495in}{0.284378in}}%
\pgfpathlineto{\pgfqpoint{0.883227in}{0.284378in}}%
\pgfpathlineto{\pgfqpoint{0.894959in}{0.284378in}}%
\pgfpathlineto{\pgfqpoint{0.906691in}{0.284378in}}%
\pgfpathlineto{\pgfqpoint{0.918424in}{0.284378in}}%
\pgfpathlineto{\pgfqpoint{0.930156in}{0.284378in}}%
\pgfpathlineto{\pgfqpoint{0.931434in}{0.284378in}}%
\pgfpathlineto{\pgfqpoint{0.930178in}{0.295789in}}%
\pgfpathlineto{\pgfqpoint{0.930156in}{0.295991in}}%
\pgfpathlineto{\pgfqpoint{0.928339in}{0.307200in}}%
\pgfpathlineto{\pgfqpoint{0.926604in}{0.318611in}}%
\pgfpathlineto{\pgfqpoint{0.924957in}{0.330022in}}%
\pgfpathlineto{\pgfqpoint{0.923384in}{0.341432in}}%
\pgfpathlineto{\pgfqpoint{0.921868in}{0.352843in}}%
\pgfpathlineto{\pgfqpoint{0.920413in}{0.364254in}}%
\pgfpathlineto{\pgfqpoint{0.919003in}{0.375665in}}%
\pgfpathlineto{\pgfqpoint{0.918424in}{0.380291in}}%
\pgfpathlineto{\pgfqpoint{0.917130in}{0.387075in}}%
\pgfpathlineto{\pgfqpoint{0.915020in}{0.398486in}}%
\pgfpathlineto{\pgfqpoint{0.913027in}{0.409897in}}%
\pgfpathlineto{\pgfqpoint{0.911134in}{0.421308in}}%
\pgfpathlineto{\pgfqpoint{0.909326in}{0.432719in}}%
\pgfpathlineto{\pgfqpoint{0.907540in}{0.444129in}}%
\pgfpathlineto{\pgfqpoint{0.906691in}{0.449977in}}%
\pgfpathlineto{\pgfqpoint{0.905351in}{0.455540in}}%
\pgfpathlineto{\pgfqpoint{0.902085in}{0.466951in}}%
\pgfpathlineto{\pgfqpoint{0.899246in}{0.478362in}}%
\pgfpathlineto{\pgfqpoint{0.895562in}{0.489772in}}%
\pgfpathlineto{\pgfqpoint{0.894959in}{0.491544in}}%
\pgfpathlineto{\pgfqpoint{0.890927in}{0.501183in}}%
\pgfpathlineto{\pgfqpoint{0.887597in}{0.512594in}}%
\pgfpathlineto{\pgfqpoint{0.884785in}{0.524005in}}%
\pgfpathlineto{\pgfqpoint{0.883227in}{0.530981in}}%
\pgfpathlineto{\pgfqpoint{0.881684in}{0.535416in}}%
\pgfpathlineto{\pgfqpoint{0.878015in}{0.546826in}}%
\pgfpathlineto{\pgfqpoint{0.874932in}{0.558237in}}%
\pgfpathlineto{\pgfqpoint{0.872310in}{0.569648in}}%
\pgfpathlineto{\pgfqpoint{0.871495in}{0.573399in}}%
\pgfpathlineto{\pgfqpoint{0.868926in}{0.581059in}}%
\pgfpathlineto{\pgfqpoint{0.865522in}{0.592469in}}%
\pgfpathlineto{\pgfqpoint{0.862637in}{0.603880in}}%
\pgfpathlineto{\pgfqpoint{0.860168in}{0.615291in}}%
\pgfpathlineto{\pgfqpoint{0.859762in}{0.617216in}}%
\pgfpathlineto{\pgfqpoint{0.856711in}{0.626702in}}%
\pgfpathlineto{\pgfqpoint{0.853504in}{0.638113in}}%
\pgfpathlineto{\pgfqpoint{0.851033in}{0.649523in}}%
\pgfpathlineto{\pgfqpoint{0.849076in}{0.660934in}}%
\pgfpathlineto{\pgfqpoint{0.848030in}{0.668889in}}%
\pgfpathlineto{\pgfqpoint{0.847375in}{0.672345in}}%
\pgfpathlineto{\pgfqpoint{0.845402in}{0.683756in}}%
\pgfpathlineto{\pgfqpoint{0.843622in}{0.695166in}}%
\pgfpathlineto{\pgfqpoint{0.842450in}{0.706577in}}%
\pgfpathlineto{\pgfqpoint{0.842028in}{0.717988in}}%
\pgfpathlineto{\pgfqpoint{0.842059in}{0.729399in}}%
\pgfpathlineto{\pgfqpoint{0.842192in}{0.740810in}}%
\pgfpathlineto{\pgfqpoint{0.842874in}{0.752220in}}%
\pgfpathlineto{\pgfqpoint{0.844460in}{0.763631in}}%
\pgfpathlineto{\pgfqpoint{0.846581in}{0.775042in}}%
\pgfpathlineto{\pgfqpoint{0.848030in}{0.782595in}}%
\pgfpathlineto{\pgfqpoint{0.848419in}{0.786453in}}%
\pgfpathlineto{\pgfqpoint{0.849544in}{0.797863in}}%
\pgfpathlineto{\pgfqpoint{0.850602in}{0.809274in}}%
\pgfpathlineto{\pgfqpoint{0.851568in}{0.820685in}}%
\pgfpathlineto{\pgfqpoint{0.852240in}{0.832096in}}%
\pgfpathlineto{\pgfqpoint{0.852416in}{0.843507in}}%
\pgfpathlineto{\pgfqpoint{0.852414in}{0.854917in}}%
\pgfpathlineto{\pgfqpoint{0.852603in}{0.866328in}}%
\pgfpathlineto{\pgfqpoint{0.856705in}{0.877739in}}%
\pgfpathlineto{\pgfqpoint{0.859762in}{0.879984in}}%
\pgfpathlineto{\pgfqpoint{0.864034in}{0.877739in}}%
\pgfpathlineto{\pgfqpoint{0.871495in}{0.874882in}}%
\pgfpathlineto{\pgfqpoint{0.883227in}{0.872612in}}%
\pgfpathlineto{\pgfqpoint{0.887272in}{0.877739in}}%
\pgfpathlineto{\pgfqpoint{0.885700in}{0.889150in}}%
\pgfpathlineto{\pgfqpoint{0.883227in}{0.894080in}}%
\pgfpathlineto{\pgfqpoint{0.879695in}{0.900560in}}%
\pgfpathlineto{\pgfqpoint{0.871495in}{0.907978in}}%
\pgfpathlineto{\pgfqpoint{0.861967in}{0.911971in}}%
\pgfpathlineto{\pgfqpoint{0.859762in}{0.912609in}}%
\pgfpathlineto{\pgfqpoint{0.858688in}{0.911971in}}%
\pgfpathlineto{\pgfqpoint{0.851240in}{0.900560in}}%
\pgfpathlineto{\pgfqpoint{0.848030in}{0.893716in}}%
\pgfpathlineto{\pgfqpoint{0.843710in}{0.889150in}}%
\pgfpathlineto{\pgfqpoint{0.836298in}{0.884839in}}%
\pgfpathlineto{\pgfqpoint{0.827900in}{0.877739in}}%
\pgfpathlineto{\pgfqpoint{0.824566in}{0.875011in}}%
\pgfpathlineto{\pgfqpoint{0.813216in}{0.866328in}}%
\pgfpathlineto{\pgfqpoint{0.812834in}{0.865986in}}%
\pgfpathlineto{\pgfqpoint{0.801101in}{0.855821in}}%
\pgfpathlineto{\pgfqpoint{0.799914in}{0.854917in}}%
\pgfpathlineto{\pgfqpoint{0.789369in}{0.844591in}}%
\pgfpathlineto{\pgfqpoint{0.788067in}{0.843507in}}%
\pgfpathlineto{\pgfqpoint{0.777849in}{0.832096in}}%
\pgfpathlineto{\pgfqpoint{0.777637in}{0.831793in}}%
\pgfpathlineto{\pgfqpoint{0.769120in}{0.820685in}}%
\pgfpathlineto{\pgfqpoint{0.765905in}{0.815428in}}%
\pgfpathlineto{\pgfqpoint{0.761905in}{0.809274in}}%
\pgfpathlineto{\pgfqpoint{0.755607in}{0.797863in}}%
\pgfpathlineto{\pgfqpoint{0.754173in}{0.795226in}}%
\pgfpathlineto{\pgfqpoint{0.749379in}{0.786453in}}%
\pgfpathlineto{\pgfqpoint{0.742772in}{0.775042in}}%
\pgfpathlineto{\pgfqpoint{0.742440in}{0.774497in}}%
\pgfpathlineto{\pgfqpoint{0.736113in}{0.763631in}}%
\pgfpathlineto{\pgfqpoint{0.730708in}{0.756023in}}%
\pgfpathlineto{\pgfqpoint{0.728257in}{0.752220in}}%
\pgfpathlineto{\pgfqpoint{0.720281in}{0.740810in}}%
\pgfpathlineto{\pgfqpoint{0.718976in}{0.739182in}}%
\pgfpathlineto{\pgfqpoint{0.712466in}{0.729399in}}%
\pgfpathlineto{\pgfqpoint{0.707244in}{0.723014in}}%
\pgfpathlineto{\pgfqpoint{0.703693in}{0.717988in}}%
\pgfpathlineto{\pgfqpoint{0.695512in}{0.707886in}}%
\pgfpathlineto{\pgfqpoint{0.694556in}{0.706577in}}%
\pgfpathlineto{\pgfqpoint{0.686010in}{0.695166in}}%
\pgfpathlineto{\pgfqpoint{0.683779in}{0.692432in}}%
\pgfpathlineto{\pgfqpoint{0.677556in}{0.683756in}}%
\pgfpathlineto{\pgfqpoint{0.672047in}{0.675949in}}%
\pgfpathlineto{\pgfqpoint{0.669626in}{0.672345in}}%
\pgfpathlineto{\pgfqpoint{0.662628in}{0.660934in}}%
\pgfpathlineto{\pgfqpoint{0.660315in}{0.657429in}}%
\pgfpathlineto{\pgfqpoint{0.655556in}{0.649523in}}%
\pgfpathlineto{\pgfqpoint{0.648583in}{0.638490in}}%
\pgfpathlineto{\pgfqpoint{0.648351in}{0.638113in}}%
\pgfpathlineto{\pgfqpoint{0.641664in}{0.626702in}}%
\pgfpathlineto{\pgfqpoint{0.636851in}{0.618949in}}%
\pgfpathlineto{\pgfqpoint{0.634673in}{0.615291in}}%
\pgfpathlineto{\pgfqpoint{0.628257in}{0.603880in}}%
\pgfpathlineto{\pgfqpoint{0.625118in}{0.598571in}}%
\pgfpathlineto{\pgfqpoint{0.621690in}{0.592469in}}%
\pgfpathlineto{\pgfqpoint{0.615424in}{0.581059in}}%
\pgfpathlineto{\pgfqpoint{0.613386in}{0.577529in}}%
\pgfpathlineto{\pgfqpoint{0.609072in}{0.569648in}}%
\pgfpathlineto{\pgfqpoint{0.602910in}{0.558237in}}%
\pgfpathlineto{\pgfqpoint{0.601654in}{0.556009in}}%
\pgfpathlineto{\pgfqpoint{0.596754in}{0.546826in}}%
\pgfpathlineto{\pgfqpoint{0.590739in}{0.535416in}}%
\pgfpathlineto{\pgfqpoint{0.589922in}{0.533986in}}%
\pgfpathlineto{\pgfqpoint{0.584625in}{0.524005in}}%
\pgfpathlineto{\pgfqpoint{0.578289in}{0.512594in}}%
\pgfpathlineto{\pgfqpoint{0.578190in}{0.512423in}}%
\pgfpathlineto{\pgfqpoint{0.572033in}{0.501183in}}%
\pgfpathlineto{\pgfqpoint{0.566457in}{0.491240in}}%
\pgfpathlineto{\pgfqpoint{0.565625in}{0.489772in}}%
\pgfpathlineto{\pgfqpoint{0.559479in}{0.478362in}}%
\pgfpathlineto{\pgfqpoint{0.554725in}{0.469834in}}%
\pgfpathlineto{\pgfqpoint{0.553114in}{0.466951in}}%
\pgfpathlineto{\pgfqpoint{0.547040in}{0.455540in}}%
\pgfpathlineto{\pgfqpoint{0.542993in}{0.448248in}}%
\pgfpathlineto{\pgfqpoint{0.540723in}{0.444129in}}%
\pgfpathlineto{\pgfqpoint{0.534692in}{0.432719in}}%
\pgfpathlineto{\pgfqpoint{0.531261in}{0.426516in}}%
\pgfpathlineto{\pgfqpoint{0.528426in}{0.421308in}}%
\pgfpathlineto{\pgfqpoint{0.522416in}{0.409897in}}%
\pgfpathlineto{\pgfqpoint{0.519528in}{0.404668in}}%
\pgfpathlineto{\pgfqpoint{0.516199in}{0.398486in}}%
\pgfpathlineto{\pgfqpoint{0.510194in}{0.387075in}}%
\pgfpathlineto{\pgfqpoint{0.507796in}{0.382729in}}%
\pgfpathlineto{\pgfqpoint{0.504027in}{0.375665in}}%
\pgfpathlineto{\pgfqpoint{0.498010in}{0.364254in}}%
\pgfpathlineto{\pgfqpoint{0.496064in}{0.360725in}}%
\pgfpathlineto{\pgfqpoint{0.491893in}{0.352843in}}%
\pgfpathlineto{\pgfqpoint{0.485855in}{0.341432in}}%
\pgfpathlineto{\pgfqpoint{0.484332in}{0.338673in}}%
\pgfpathlineto{\pgfqpoint{0.479787in}{0.330022in}}%
\pgfpathlineto{\pgfqpoint{0.473716in}{0.318611in}}%
\pgfpathlineto{\pgfqpoint{0.472600in}{0.316589in}}%
\pgfpathlineto{\pgfqpoint{0.467698in}{0.307200in}}%
\pgfpathlineto{\pgfqpoint{0.461587in}{0.295789in}}%
\pgfpathlineto{\pgfqpoint{0.460867in}{0.294487in}}%
\pgfpathlineto{\pgfqpoint{0.455621in}{0.284378in}}%
\pgfpathclose%
\pgfpathmoveto{\pgfqpoint{0.794999in}{0.775042in}}%
\pgfpathlineto{\pgfqpoint{0.789369in}{0.777692in}}%
\pgfpathlineto{\pgfqpoint{0.784666in}{0.786453in}}%
\pgfpathlineto{\pgfqpoint{0.783786in}{0.797863in}}%
\pgfpathlineto{\pgfqpoint{0.786192in}{0.809274in}}%
\pgfpathlineto{\pgfqpoint{0.789369in}{0.817761in}}%
\pgfpathlineto{\pgfqpoint{0.790598in}{0.820685in}}%
\pgfpathlineto{\pgfqpoint{0.797490in}{0.832096in}}%
\pgfpathlineto{\pgfqpoint{0.801101in}{0.837000in}}%
\pgfpathlineto{\pgfqpoint{0.806115in}{0.843507in}}%
\pgfpathlineto{\pgfqpoint{0.812834in}{0.850829in}}%
\pgfpathlineto{\pgfqpoint{0.816880in}{0.854917in}}%
\pgfpathlineto{\pgfqpoint{0.824566in}{0.861432in}}%
\pgfpathlineto{\pgfqpoint{0.834515in}{0.866328in}}%
\pgfpathlineto{\pgfqpoint{0.836298in}{0.867417in}}%
\pgfpathlineto{\pgfqpoint{0.837157in}{0.866328in}}%
\pgfpathlineto{\pgfqpoint{0.839749in}{0.854917in}}%
\pgfpathlineto{\pgfqpoint{0.839979in}{0.843507in}}%
\pgfpathlineto{\pgfqpoint{0.839245in}{0.832096in}}%
\pgfpathlineto{\pgfqpoint{0.837807in}{0.820685in}}%
\pgfpathlineto{\pgfqpoint{0.836298in}{0.812398in}}%
\pgfpathlineto{\pgfqpoint{0.834975in}{0.809274in}}%
\pgfpathlineto{\pgfqpoint{0.829619in}{0.797863in}}%
\pgfpathlineto{\pgfqpoint{0.824566in}{0.786707in}}%
\pgfpathlineto{\pgfqpoint{0.824260in}{0.786453in}}%
\pgfpathlineto{\pgfqpoint{0.812834in}{0.775354in}}%
\pgfpathlineto{\pgfqpoint{0.811033in}{0.775042in}}%
\pgfpathlineto{\pgfqpoint{0.801101in}{0.772751in}}%
\pgfpathclose%
\pgfpathmoveto{\pgfqpoint{0.869692in}{0.889150in}}%
\pgfpathlineto{\pgfqpoint{0.868146in}{0.900560in}}%
\pgfpathlineto{\pgfqpoint{0.871495in}{0.900959in}}%
\pgfpathlineto{\pgfqpoint{0.871935in}{0.900560in}}%
\pgfpathlineto{\pgfqpoint{0.872710in}{0.889150in}}%
\pgfpathlineto{\pgfqpoint{0.871495in}{0.888106in}}%
\pgfpathclose%
\pgfusepath{fill}%
\end{pgfscope}%
\begin{pgfscope}%
\pgfpathrectangle{\pgfqpoint{0.211875in}{0.211875in}}{\pgfqpoint{1.313625in}{1.279725in}}%
\pgfusepath{clip}%
\pgfsetbuttcap%
\pgfsetroundjoin%
\definecolor{currentfill}{rgb}{0.312084,0.115053,0.298245}%
\pgfsetfillcolor{currentfill}%
\pgfsetlinewidth{0.000000pt}%
\definecolor{currentstroke}{rgb}{0.000000,0.000000,0.000000}%
\pgfsetstrokecolor{currentstroke}%
\pgfsetdash{}{0pt}%
\pgfpathmoveto{\pgfqpoint{0.286122in}{0.649523in}}%
\pgfpathlineto{\pgfqpoint{0.296617in}{0.654882in}}%
\pgfpathlineto{\pgfqpoint{0.308349in}{0.660689in}}%
\pgfpathlineto{\pgfqpoint{0.308837in}{0.660934in}}%
\pgfpathlineto{\pgfqpoint{0.320081in}{0.666700in}}%
\pgfpathlineto{\pgfqpoint{0.331424in}{0.672345in}}%
\pgfpathlineto{\pgfqpoint{0.331813in}{0.672550in}}%
\pgfpathlineto{\pgfqpoint{0.343545in}{0.678592in}}%
\pgfpathlineto{\pgfqpoint{0.353834in}{0.683756in}}%
\pgfpathlineto{\pgfqpoint{0.355278in}{0.684517in}}%
\pgfpathlineto{\pgfqpoint{0.367010in}{0.690583in}}%
\pgfpathlineto{\pgfqpoint{0.376047in}{0.695166in}}%
\pgfpathlineto{\pgfqpoint{0.378742in}{0.696592in}}%
\pgfpathlineto{\pgfqpoint{0.390474in}{0.702700in}}%
\pgfpathlineto{\pgfqpoint{0.398020in}{0.706577in}}%
\pgfpathlineto{\pgfqpoint{0.402206in}{0.708815in}}%
\pgfpathlineto{\pgfqpoint{0.413939in}{0.714976in}}%
\pgfpathlineto{\pgfqpoint{0.419711in}{0.717988in}}%
\pgfpathlineto{\pgfqpoint{0.425671in}{0.721227in}}%
\pgfpathlineto{\pgfqpoint{0.437403in}{0.727444in}}%
\pgfpathlineto{\pgfqpoint{0.441086in}{0.729399in}}%
\pgfpathlineto{\pgfqpoint{0.449135in}{0.733861in}}%
\pgfpathlineto{\pgfqpoint{0.460867in}{0.740133in}}%
\pgfpathlineto{\pgfqpoint{0.462118in}{0.740810in}}%
\pgfpathlineto{\pgfqpoint{0.472600in}{0.746749in}}%
\pgfpathlineto{\pgfqpoint{0.482677in}{0.752220in}}%
\pgfpathlineto{\pgfqpoint{0.484332in}{0.753223in}}%
\pgfpathlineto{\pgfqpoint{0.496064in}{0.759927in}}%
\pgfpathlineto{\pgfqpoint{0.502740in}{0.763631in}}%
\pgfpathlineto{\pgfqpoint{0.507796in}{0.766762in}}%
\pgfpathlineto{\pgfqpoint{0.519528in}{0.773418in}}%
\pgfpathlineto{\pgfqpoint{0.522468in}{0.775042in}}%
\pgfpathlineto{\pgfqpoint{0.531261in}{0.780507in}}%
\pgfpathlineto{\pgfqpoint{0.542017in}{0.786453in}}%
\pgfpathlineto{\pgfqpoint{0.542993in}{0.787142in}}%
\pgfpathlineto{\pgfqpoint{0.554725in}{0.794364in}}%
\pgfpathlineto{\pgfqpoint{0.560807in}{0.797863in}}%
\pgfpathlineto{\pgfqpoint{0.566457in}{0.802061in}}%
\pgfpathlineto{\pgfqpoint{0.578190in}{0.808907in}}%
\pgfpathlineto{\pgfqpoint{0.578805in}{0.809274in}}%
\pgfpathlineto{\pgfqpoint{0.589922in}{0.817858in}}%
\pgfpathlineto{\pgfqpoint{0.594449in}{0.820685in}}%
\pgfpathlineto{\pgfqpoint{0.601654in}{0.828675in}}%
\pgfpathlineto{\pgfqpoint{0.606428in}{0.832096in}}%
\pgfpathlineto{\pgfqpoint{0.601654in}{0.837160in}}%
\pgfpathlineto{\pgfqpoint{0.589922in}{0.837755in}}%
\pgfpathlineto{\pgfqpoint{0.578190in}{0.836117in}}%
\pgfpathlineto{\pgfqpoint{0.566457in}{0.833963in}}%
\pgfpathlineto{\pgfqpoint{0.557644in}{0.832096in}}%
\pgfpathlineto{\pgfqpoint{0.554725in}{0.831142in}}%
\pgfpathlineto{\pgfqpoint{0.542993in}{0.828193in}}%
\pgfpathlineto{\pgfqpoint{0.531261in}{0.825557in}}%
\pgfpathlineto{\pgfqpoint{0.519528in}{0.822889in}}%
\pgfpathlineto{\pgfqpoint{0.510868in}{0.820685in}}%
\pgfpathlineto{\pgfqpoint{0.507796in}{0.819732in}}%
\pgfpathlineto{\pgfqpoint{0.496064in}{0.816210in}}%
\pgfpathlineto{\pgfqpoint{0.484332in}{0.812956in}}%
\pgfpathlineto{\pgfqpoint{0.472600in}{0.809545in}}%
\pgfpathlineto{\pgfqpoint{0.471756in}{0.809274in}}%
\pgfpathlineto{\pgfqpoint{0.460867in}{0.805683in}}%
\pgfpathlineto{\pgfqpoint{0.449135in}{0.802107in}}%
\pgfpathlineto{\pgfqpoint{0.437403in}{0.798386in}}%
\pgfpathlineto{\pgfqpoint{0.435883in}{0.797863in}}%
\pgfpathlineto{\pgfqpoint{0.425671in}{0.794388in}}%
\pgfpathlineto{\pgfqpoint{0.413939in}{0.790624in}}%
\pgfpathlineto{\pgfqpoint{0.402206in}{0.786669in}}%
\pgfpathlineto{\pgfqpoint{0.401607in}{0.786453in}}%
\pgfpathlineto{\pgfqpoint{0.390474in}{0.782599in}}%
\pgfpathlineto{\pgfqpoint{0.378742in}{0.778710in}}%
\pgfpathlineto{\pgfqpoint{0.368337in}{0.775042in}}%
\pgfpathlineto{\pgfqpoint{0.367010in}{0.774561in}}%
\pgfpathlineto{\pgfqpoint{0.355278in}{0.770494in}}%
\pgfpathlineto{\pgfqpoint{0.343545in}{0.766501in}}%
\pgfpathlineto{\pgfqpoint{0.335651in}{0.763631in}}%
\pgfpathlineto{\pgfqpoint{0.331813in}{0.762238in}}%
\pgfpathlineto{\pgfqpoint{0.320081in}{0.758183in}}%
\pgfpathlineto{\pgfqpoint{0.308349in}{0.754086in}}%
\pgfpathlineto{\pgfqpoint{0.303330in}{0.752220in}}%
\pgfpathlineto{\pgfqpoint{0.296617in}{0.749791in}}%
\pgfpathlineto{\pgfqpoint{0.284884in}{0.745733in}}%
\pgfpathlineto{\pgfqpoint{0.284884in}{0.740810in}}%
\pgfpathlineto{\pgfqpoint{0.284884in}{0.734687in}}%
\pgfpathlineto{\pgfqpoint{0.296617in}{0.738996in}}%
\pgfpathlineto{\pgfqpoint{0.301004in}{0.740810in}}%
\pgfpathlineto{\pgfqpoint{0.308349in}{0.743141in}}%
\pgfpathlineto{\pgfqpoint{0.320081in}{0.746880in}}%
\pgfpathlineto{\pgfqpoint{0.331813in}{0.751350in}}%
\pgfpathlineto{\pgfqpoint{0.333922in}{0.752220in}}%
\pgfpathlineto{\pgfqpoint{0.343545in}{0.755143in}}%
\pgfpathlineto{\pgfqpoint{0.355278in}{0.758849in}}%
\pgfpathlineto{\pgfqpoint{0.367010in}{0.763431in}}%
\pgfpathlineto{\pgfqpoint{0.367500in}{0.763631in}}%
\pgfpathlineto{\pgfqpoint{0.378742in}{0.766844in}}%
\pgfpathlineto{\pgfqpoint{0.390474in}{0.770429in}}%
\pgfpathlineto{\pgfqpoint{0.402206in}{0.774969in}}%
\pgfpathlineto{\pgfqpoint{0.402391in}{0.775042in}}%
\pgfpathlineto{\pgfqpoint{0.413939in}{0.778060in}}%
\pgfpathlineto{\pgfqpoint{0.425671in}{0.781322in}}%
\pgfpathlineto{\pgfqpoint{0.437403in}{0.785426in}}%
\pgfpathlineto{\pgfqpoint{0.440196in}{0.786453in}}%
\pgfpathlineto{\pgfqpoint{0.449135in}{0.788486in}}%
\pgfpathlineto{\pgfqpoint{0.460867in}{0.790984in}}%
\pgfpathlineto{\pgfqpoint{0.472600in}{0.793660in}}%
\pgfpathlineto{\pgfqpoint{0.484332in}{0.796920in}}%
\pgfpathlineto{\pgfqpoint{0.489255in}{0.797863in}}%
\pgfpathlineto{\pgfqpoint{0.496064in}{0.798382in}}%
\pgfpathlineto{\pgfqpoint{0.501944in}{0.797863in}}%
\pgfpathlineto{\pgfqpoint{0.498831in}{0.786453in}}%
\pgfpathlineto{\pgfqpoint{0.496064in}{0.784698in}}%
\pgfpathlineto{\pgfqpoint{0.484332in}{0.775194in}}%
\pgfpathlineto{\pgfqpoint{0.484189in}{0.775042in}}%
\pgfpathlineto{\pgfqpoint{0.472600in}{0.768257in}}%
\pgfpathlineto{\pgfqpoint{0.466262in}{0.763631in}}%
\pgfpathlineto{\pgfqpoint{0.460867in}{0.760678in}}%
\pgfpathlineto{\pgfqpoint{0.449135in}{0.753515in}}%
\pgfpathlineto{\pgfqpoint{0.447206in}{0.752220in}}%
\pgfpathlineto{\pgfqpoint{0.437403in}{0.747059in}}%
\pgfpathlineto{\pgfqpoint{0.427020in}{0.740810in}}%
\pgfpathlineto{\pgfqpoint{0.425671in}{0.740125in}}%
\pgfpathlineto{\pgfqpoint{0.413939in}{0.734034in}}%
\pgfpathlineto{\pgfqpoint{0.405929in}{0.729399in}}%
\pgfpathlineto{\pgfqpoint{0.402206in}{0.727535in}}%
\pgfpathlineto{\pgfqpoint{0.390474in}{0.721449in}}%
\pgfpathlineto{\pgfqpoint{0.384303in}{0.717988in}}%
\pgfpathlineto{\pgfqpoint{0.378742in}{0.715246in}}%
\pgfpathlineto{\pgfqpoint{0.367010in}{0.709190in}}%
\pgfpathlineto{\pgfqpoint{0.362236in}{0.706577in}}%
\pgfpathlineto{\pgfqpoint{0.355278in}{0.703200in}}%
\pgfpathlineto{\pgfqpoint{0.343545in}{0.697171in}}%
\pgfpathlineto{\pgfqpoint{0.339815in}{0.695166in}}%
\pgfpathlineto{\pgfqpoint{0.331813in}{0.691329in}}%
\pgfpathlineto{\pgfqpoint{0.320081in}{0.685324in}}%
\pgfpathlineto{\pgfqpoint{0.317123in}{0.683756in}}%
\pgfpathlineto{\pgfqpoint{0.308349in}{0.679587in}}%
\pgfpathlineto{\pgfqpoint{0.296617in}{0.673599in}}%
\pgfpathlineto{\pgfqpoint{0.294228in}{0.672345in}}%
\pgfpathlineto{\pgfqpoint{0.284884in}{0.667936in}}%
\pgfpathlineto{\pgfqpoint{0.284884in}{0.660934in}}%
\pgfpathlineto{\pgfqpoint{0.284884in}{0.649523in}}%
\pgfpathlineto{\pgfqpoint{0.284884in}{0.648906in}}%
\pgfpathclose%
\pgfusepath{fill}%
\end{pgfscope}%
\begin{pgfscope}%
\pgfpathrectangle{\pgfqpoint{0.211875in}{0.211875in}}{\pgfqpoint{1.313625in}{1.279725in}}%
\pgfusepath{clip}%
\pgfsetbuttcap%
\pgfsetroundjoin%
\definecolor{currentfill}{rgb}{0.312084,0.115053,0.298245}%
\pgfsetfillcolor{currentfill}%
\pgfsetlinewidth{0.000000pt}%
\definecolor{currentstroke}{rgb}{0.000000,0.000000,0.000000}%
\pgfsetstrokecolor{currentstroke}%
\pgfsetdash{}{0pt}%
\pgfpathmoveto{\pgfqpoint{0.812834in}{0.943211in}}%
\pgfpathlineto{\pgfqpoint{0.824566in}{0.942466in}}%
\pgfpathlineto{\pgfqpoint{0.828276in}{0.946204in}}%
\pgfpathlineto{\pgfqpoint{0.828423in}{0.957614in}}%
\pgfpathlineto{\pgfqpoint{0.825922in}{0.969025in}}%
\pgfpathlineto{\pgfqpoint{0.824566in}{0.973233in}}%
\pgfpathlineto{\pgfqpoint{0.822276in}{0.980436in}}%
\pgfpathlineto{\pgfqpoint{0.818601in}{0.991847in}}%
\pgfpathlineto{\pgfqpoint{0.815161in}{1.003257in}}%
\pgfpathlineto{\pgfqpoint{0.812834in}{1.011093in}}%
\pgfpathlineto{\pgfqpoint{0.804005in}{1.014668in}}%
\pgfpathlineto{\pgfqpoint{0.801101in}{1.015344in}}%
\pgfpathlineto{\pgfqpoint{0.789369in}{1.014689in}}%
\pgfpathlineto{\pgfqpoint{0.777637in}{1.017204in}}%
\pgfpathlineto{\pgfqpoint{0.765905in}{1.021910in}}%
\pgfpathlineto{\pgfqpoint{0.756993in}{1.026079in}}%
\pgfpathlineto{\pgfqpoint{0.754173in}{1.028312in}}%
\pgfpathlineto{\pgfqpoint{0.742440in}{1.035704in}}%
\pgfpathlineto{\pgfqpoint{0.738795in}{1.026079in}}%
\pgfpathlineto{\pgfqpoint{0.742440in}{1.021014in}}%
\pgfpathlineto{\pgfqpoint{0.746731in}{1.014668in}}%
\pgfpathlineto{\pgfqpoint{0.753594in}{1.003257in}}%
\pgfpathlineto{\pgfqpoint{0.754173in}{1.002440in}}%
\pgfpathlineto{\pgfqpoint{0.761928in}{0.991847in}}%
\pgfpathlineto{\pgfqpoint{0.765905in}{0.986725in}}%
\pgfpathlineto{\pgfqpoint{0.770553in}{0.980436in}}%
\pgfpathlineto{\pgfqpoint{0.777637in}{0.971653in}}%
\pgfpathlineto{\pgfqpoint{0.779333in}{0.969025in}}%
\pgfpathlineto{\pgfqpoint{0.787837in}{0.957614in}}%
\pgfpathlineto{\pgfqpoint{0.789369in}{0.955775in}}%
\pgfpathlineto{\pgfqpoint{0.801101in}{0.946849in}}%
\pgfpathlineto{\pgfqpoint{0.802889in}{0.946204in}}%
\pgfpathclose%
\pgfpathmoveto{\pgfqpoint{0.799822in}{0.957614in}}%
\pgfpathlineto{\pgfqpoint{0.789885in}{0.969025in}}%
\pgfpathlineto{\pgfqpoint{0.789369in}{0.970017in}}%
\pgfpathlineto{\pgfqpoint{0.782858in}{0.980436in}}%
\pgfpathlineto{\pgfqpoint{0.777637in}{0.990839in}}%
\pgfpathlineto{\pgfqpoint{0.776940in}{0.991847in}}%
\pgfpathlineto{\pgfqpoint{0.777637in}{0.993215in}}%
\pgfpathlineto{\pgfqpoint{0.789369in}{0.996305in}}%
\pgfpathlineto{\pgfqpoint{0.801101in}{1.001528in}}%
\pgfpathlineto{\pgfqpoint{0.810341in}{0.991847in}}%
\pgfpathlineto{\pgfqpoint{0.812834in}{0.986657in}}%
\pgfpathlineto{\pgfqpoint{0.814382in}{0.980436in}}%
\pgfpathlineto{\pgfqpoint{0.817001in}{0.969025in}}%
\pgfpathlineto{\pgfqpoint{0.818593in}{0.957614in}}%
\pgfpathlineto{\pgfqpoint{0.812834in}{0.952770in}}%
\pgfpathlineto{\pgfqpoint{0.801101in}{0.956594in}}%
\pgfpathclose%
\pgfusepath{fill}%
\end{pgfscope}%
\begin{pgfscope}%
\pgfpathrectangle{\pgfqpoint{0.211875in}{0.211875in}}{\pgfqpoint{1.313625in}{1.279725in}}%
\pgfusepath{clip}%
\pgfsetbuttcap%
\pgfsetroundjoin%
\definecolor{currentfill}{rgb}{0.312084,0.115053,0.298245}%
\pgfsetfillcolor{currentfill}%
\pgfsetlinewidth{0.000000pt}%
\definecolor{currentstroke}{rgb}{0.000000,0.000000,0.000000}%
\pgfsetstrokecolor{currentstroke}%
\pgfsetdash{}{0pt}%
\pgfpathmoveto{\pgfqpoint{1.446373in}{1.012295in}}%
\pgfpathlineto{\pgfqpoint{1.446373in}{1.014668in}}%
\pgfpathlineto{\pgfqpoint{1.446373in}{1.026071in}}%
\pgfpathlineto{\pgfqpoint{1.446367in}{1.026079in}}%
\pgfpathlineto{\pgfqpoint{1.446373in}{1.026126in}}%
\pgfpathlineto{\pgfqpoint{1.446373in}{1.037490in}}%
\pgfpathlineto{\pgfqpoint{1.446373in}{1.048901in}}%
\pgfpathlineto{\pgfqpoint{1.446373in}{1.051629in}}%
\pgfpathlineto{\pgfqpoint{1.442075in}{1.048901in}}%
\pgfpathlineto{\pgfqpoint{1.434641in}{1.042739in}}%
\pgfpathlineto{\pgfqpoint{1.432503in}{1.037490in}}%
\pgfpathlineto{\pgfqpoint{1.434641in}{1.029093in}}%
\pgfpathlineto{\pgfqpoint{1.435877in}{1.026079in}}%
\pgfpathlineto{\pgfqpoint{1.443814in}{1.014668in}}%
\pgfpathclose%
\pgfusepath{fill}%
\end{pgfscope}%
\begin{pgfscope}%
\pgfpathrectangle{\pgfqpoint{0.211875in}{0.211875in}}{\pgfqpoint{1.313625in}{1.279725in}}%
\pgfusepath{clip}%
\pgfsetbuttcap%
\pgfsetroundjoin%
\definecolor{currentfill}{rgb}{0.312084,0.115053,0.298245}%
\pgfsetfillcolor{currentfill}%
\pgfsetlinewidth{0.000000pt}%
\definecolor{currentstroke}{rgb}{0.000000,0.000000,0.000000}%
\pgfsetstrokecolor{currentstroke}%
\pgfsetdash{}{0pt}%
\pgfpathmoveto{\pgfqpoint{0.730708in}{1.035957in}}%
\pgfpathlineto{\pgfqpoint{0.738331in}{1.037490in}}%
\pgfpathlineto{\pgfqpoint{0.730708in}{1.040529in}}%
\pgfpathlineto{\pgfqpoint{0.729406in}{1.037490in}}%
\pgfpathclose%
\pgfusepath{fill}%
\end{pgfscope}%
\begin{pgfscope}%
\pgfpathrectangle{\pgfqpoint{0.211875in}{0.211875in}}{\pgfqpoint{1.313625in}{1.279725in}}%
\pgfusepath{clip}%
\pgfsetbuttcap%
\pgfsetroundjoin%
\definecolor{currentfill}{rgb}{0.312084,0.115053,0.298245}%
\pgfsetfillcolor{currentfill}%
\pgfsetlinewidth{0.000000pt}%
\definecolor{currentstroke}{rgb}{0.000000,0.000000,0.000000}%
\pgfsetstrokecolor{currentstroke}%
\pgfsetdash{}{0pt}%
\pgfpathmoveto{\pgfqpoint{1.235193in}{1.042828in}}%
\pgfpathlineto{\pgfqpoint{1.246925in}{1.038851in}}%
\pgfpathlineto{\pgfqpoint{1.258658in}{1.038951in}}%
\pgfpathlineto{\pgfqpoint{1.270390in}{1.039978in}}%
\pgfpathlineto{\pgfqpoint{1.282122in}{1.042247in}}%
\pgfpathlineto{\pgfqpoint{1.293854in}{1.046648in}}%
\pgfpathlineto{\pgfqpoint{1.297808in}{1.048901in}}%
\pgfpathlineto{\pgfqpoint{1.294207in}{1.060311in}}%
\pgfpathlineto{\pgfqpoint{1.293854in}{1.060647in}}%
\pgfpathlineto{\pgfqpoint{1.284900in}{1.071722in}}%
\pgfpathlineto{\pgfqpoint{1.282122in}{1.074316in}}%
\pgfpathlineto{\pgfqpoint{1.270390in}{1.082054in}}%
\pgfpathlineto{\pgfqpoint{1.268058in}{1.083133in}}%
\pgfpathlineto{\pgfqpoint{1.258658in}{1.086696in}}%
\pgfpathlineto{\pgfqpoint{1.246925in}{1.091340in}}%
\pgfpathlineto{\pgfqpoint{1.239185in}{1.094544in}}%
\pgfpathlineto{\pgfqpoint{1.235193in}{1.095909in}}%
\pgfpathlineto{\pgfqpoint{1.223461in}{1.099109in}}%
\pgfpathlineto{\pgfqpoint{1.211729in}{1.101629in}}%
\pgfpathlineto{\pgfqpoint{1.199996in}{1.102536in}}%
\pgfpathlineto{\pgfqpoint{1.188264in}{1.102660in}}%
\pgfpathlineto{\pgfqpoint{1.176532in}{1.098779in}}%
\pgfpathlineto{\pgfqpoint{1.169924in}{1.094544in}}%
\pgfpathlineto{\pgfqpoint{1.166891in}{1.083133in}}%
\pgfpathlineto{\pgfqpoint{1.176532in}{1.073033in}}%
\pgfpathlineto{\pgfqpoint{1.177749in}{1.071722in}}%
\pgfpathlineto{\pgfqpoint{1.188264in}{1.066428in}}%
\pgfpathlineto{\pgfqpoint{1.199828in}{1.060311in}}%
\pgfpathlineto{\pgfqpoint{1.199996in}{1.060194in}}%
\pgfpathlineto{\pgfqpoint{1.211729in}{1.054494in}}%
\pgfpathlineto{\pgfqpoint{1.223461in}{1.049977in}}%
\pgfpathlineto{\pgfqpoint{1.225733in}{1.048901in}}%
\pgfpathclose%
\pgfpathmoveto{\pgfqpoint{1.240275in}{1.048901in}}%
\pgfpathlineto{\pgfqpoint{1.235193in}{1.050438in}}%
\pgfpathlineto{\pgfqpoint{1.223461in}{1.055599in}}%
\pgfpathlineto{\pgfqpoint{1.211729in}{1.059537in}}%
\pgfpathlineto{\pgfqpoint{1.210139in}{1.060311in}}%
\pgfpathlineto{\pgfqpoint{1.199996in}{1.066426in}}%
\pgfpathlineto{\pgfqpoint{1.188264in}{1.070798in}}%
\pgfpathlineto{\pgfqpoint{1.186428in}{1.071722in}}%
\pgfpathlineto{\pgfqpoint{1.176532in}{1.082376in}}%
\pgfpathlineto{\pgfqpoint{1.175810in}{1.083133in}}%
\pgfpathlineto{\pgfqpoint{1.176532in}{1.085290in}}%
\pgfpathlineto{\pgfqpoint{1.185562in}{1.094544in}}%
\pgfpathlineto{\pgfqpoint{1.188264in}{1.095308in}}%
\pgfpathlineto{\pgfqpoint{1.199996in}{1.096201in}}%
\pgfpathlineto{\pgfqpoint{1.211729in}{1.095586in}}%
\pgfpathlineto{\pgfqpoint{1.216969in}{1.094544in}}%
\pgfpathlineto{\pgfqpoint{1.223461in}{1.092636in}}%
\pgfpathlineto{\pgfqpoint{1.235193in}{1.088920in}}%
\pgfpathlineto{\pgfqpoint{1.246925in}{1.084806in}}%
\pgfpathlineto{\pgfqpoint{1.251702in}{1.083133in}}%
\pgfpathlineto{\pgfqpoint{1.258658in}{1.080046in}}%
\pgfpathlineto{\pgfqpoint{1.270390in}{1.075122in}}%
\pgfpathlineto{\pgfqpoint{1.275390in}{1.071722in}}%
\pgfpathlineto{\pgfqpoint{1.282122in}{1.061462in}}%
\pgfpathlineto{\pgfqpoint{1.282957in}{1.060311in}}%
\pgfpathlineto{\pgfqpoint{1.282122in}{1.057045in}}%
\pgfpathlineto{\pgfqpoint{1.276069in}{1.048901in}}%
\pgfpathlineto{\pgfqpoint{1.270390in}{1.047583in}}%
\pgfpathlineto{\pgfqpoint{1.258658in}{1.046491in}}%
\pgfpathlineto{\pgfqpoint{1.246925in}{1.046792in}}%
\pgfpathclose%
\pgfusepath{fill}%
\end{pgfscope}%
\begin{pgfscope}%
\pgfpathrectangle{\pgfqpoint{0.211875in}{0.211875in}}{\pgfqpoint{1.313625in}{1.279725in}}%
\pgfusepath{clip}%
\pgfsetbuttcap%
\pgfsetroundjoin%
\definecolor{currentfill}{rgb}{0.312084,0.115053,0.298245}%
\pgfsetfillcolor{currentfill}%
\pgfsetlinewidth{0.000000pt}%
\definecolor{currentstroke}{rgb}{0.000000,0.000000,0.000000}%
\pgfsetstrokecolor{currentstroke}%
\pgfsetdash{}{0pt}%
\pgfpathmoveto{\pgfqpoint{0.613386in}{1.128427in}}%
\pgfpathlineto{\pgfqpoint{0.614357in}{1.128776in}}%
\pgfpathlineto{\pgfqpoint{0.614566in}{1.140187in}}%
\pgfpathlineto{\pgfqpoint{0.613386in}{1.142084in}}%
\pgfpathlineto{\pgfqpoint{0.606796in}{1.151598in}}%
\pgfpathlineto{\pgfqpoint{0.601654in}{1.158947in}}%
\pgfpathlineto{\pgfqpoint{0.598504in}{1.163008in}}%
\pgfpathlineto{\pgfqpoint{0.589922in}{1.174106in}}%
\pgfpathlineto{\pgfqpoint{0.589650in}{1.174419in}}%
\pgfpathlineto{\pgfqpoint{0.580279in}{1.185830in}}%
\pgfpathlineto{\pgfqpoint{0.578190in}{1.188603in}}%
\pgfpathlineto{\pgfqpoint{0.570713in}{1.197241in}}%
\pgfpathlineto{\pgfqpoint{0.566457in}{1.202726in}}%
\pgfpathlineto{\pgfqpoint{0.561132in}{1.208651in}}%
\pgfpathlineto{\pgfqpoint{0.554725in}{1.217674in}}%
\pgfpathlineto{\pgfqpoint{0.552668in}{1.220062in}}%
\pgfpathlineto{\pgfqpoint{0.546120in}{1.231473in}}%
\pgfpathlineto{\pgfqpoint{0.542993in}{1.238464in}}%
\pgfpathlineto{\pgfqpoint{0.540651in}{1.242884in}}%
\pgfpathlineto{\pgfqpoint{0.536002in}{1.254295in}}%
\pgfpathlineto{\pgfqpoint{0.532387in}{1.265705in}}%
\pgfpathlineto{\pgfqpoint{0.531261in}{1.270275in}}%
\pgfpathlineto{\pgfqpoint{0.529326in}{1.277116in}}%
\pgfpathlineto{\pgfqpoint{0.527399in}{1.288527in}}%
\pgfpathlineto{\pgfqpoint{0.526005in}{1.299938in}}%
\pgfpathlineto{\pgfqpoint{0.527369in}{1.311348in}}%
\pgfpathlineto{\pgfqpoint{0.531261in}{1.319353in}}%
\pgfpathlineto{\pgfqpoint{0.533548in}{1.322759in}}%
\pgfpathlineto{\pgfqpoint{0.542993in}{1.329266in}}%
\pgfpathlineto{\pgfqpoint{0.554725in}{1.332364in}}%
\pgfpathlineto{\pgfqpoint{0.565493in}{1.334170in}}%
\pgfpathlineto{\pgfqpoint{0.566457in}{1.334326in}}%
\pgfpathlineto{\pgfqpoint{0.578190in}{1.337580in}}%
\pgfpathlineto{\pgfqpoint{0.582220in}{1.345581in}}%
\pgfpathlineto{\pgfqpoint{0.581348in}{1.356992in}}%
\pgfpathlineto{\pgfqpoint{0.578190in}{1.365423in}}%
\pgfpathlineto{\pgfqpoint{0.576926in}{1.368402in}}%
\pgfpathlineto{\pgfqpoint{0.568427in}{1.379813in}}%
\pgfpathlineto{\pgfqpoint{0.566457in}{1.382111in}}%
\pgfpathlineto{\pgfqpoint{0.558484in}{1.391224in}}%
\pgfpathlineto{\pgfqpoint{0.554725in}{1.395163in}}%
\pgfpathlineto{\pgfqpoint{0.547477in}{1.402635in}}%
\pgfpathlineto{\pgfqpoint{0.542993in}{1.406939in}}%
\pgfpathlineto{\pgfqpoint{0.533720in}{1.414045in}}%
\pgfpathlineto{\pgfqpoint{0.531261in}{1.414045in}}%
\pgfpathlineto{\pgfqpoint{0.519528in}{1.414045in}}%
\pgfpathlineto{\pgfqpoint{0.507796in}{1.414045in}}%
\pgfpathlineto{\pgfqpoint{0.496064in}{1.414045in}}%
\pgfpathlineto{\pgfqpoint{0.484332in}{1.414045in}}%
\pgfpathlineto{\pgfqpoint{0.472600in}{1.414045in}}%
\pgfpathlineto{\pgfqpoint{0.460867in}{1.414045in}}%
\pgfpathlineto{\pgfqpoint{0.449135in}{1.414045in}}%
\pgfpathlineto{\pgfqpoint{0.437403in}{1.414045in}}%
\pgfpathlineto{\pgfqpoint{0.425671in}{1.414045in}}%
\pgfpathlineto{\pgfqpoint{0.413939in}{1.414045in}}%
\pgfpathlineto{\pgfqpoint{0.402206in}{1.414045in}}%
\pgfpathlineto{\pgfqpoint{0.390474in}{1.414045in}}%
\pgfpathlineto{\pgfqpoint{0.378742in}{1.414045in}}%
\pgfpathlineto{\pgfqpoint{0.367010in}{1.414045in}}%
\pgfpathlineto{\pgfqpoint{0.355278in}{1.414045in}}%
\pgfpathlineto{\pgfqpoint{0.343545in}{1.414045in}}%
\pgfpathlineto{\pgfqpoint{0.331813in}{1.414045in}}%
\pgfpathlineto{\pgfqpoint{0.320081in}{1.414045in}}%
\pgfpathlineto{\pgfqpoint{0.308349in}{1.414045in}}%
\pgfpathlineto{\pgfqpoint{0.296617in}{1.414045in}}%
\pgfpathlineto{\pgfqpoint{0.284884in}{1.414045in}}%
\pgfpathlineto{\pgfqpoint{0.284884in}{1.407488in}}%
\pgfpathlineto{\pgfqpoint{0.290171in}{1.402635in}}%
\pgfpathlineto{\pgfqpoint{0.296617in}{1.397069in}}%
\pgfpathlineto{\pgfqpoint{0.302890in}{1.391224in}}%
\pgfpathlineto{\pgfqpoint{0.308349in}{1.386403in}}%
\pgfpathlineto{\pgfqpoint{0.315631in}{1.379813in}}%
\pgfpathlineto{\pgfqpoint{0.320081in}{1.375815in}}%
\pgfpathlineto{\pgfqpoint{0.328344in}{1.368402in}}%
\pgfpathlineto{\pgfqpoint{0.331813in}{1.365387in}}%
\pgfpathlineto{\pgfqpoint{0.341112in}{1.356992in}}%
\pgfpathlineto{\pgfqpoint{0.343545in}{1.354845in}}%
\pgfpathlineto{\pgfqpoint{0.353961in}{1.345581in}}%
\pgfpathlineto{\pgfqpoint{0.355278in}{1.344398in}}%
\pgfpathlineto{\pgfqpoint{0.366605in}{1.334170in}}%
\pgfpathlineto{\pgfqpoint{0.367010in}{1.333800in}}%
\pgfpathlineto{\pgfqpoint{0.378742in}{1.323219in}}%
\pgfpathlineto{\pgfqpoint{0.379209in}{1.322759in}}%
\pgfpathlineto{\pgfqpoint{0.390474in}{1.312470in}}%
\pgfpathlineto{\pgfqpoint{0.391525in}{1.311348in}}%
\pgfpathlineto{\pgfqpoint{0.402206in}{1.301001in}}%
\pgfpathlineto{\pgfqpoint{0.403186in}{1.299938in}}%
\pgfpathlineto{\pgfqpoint{0.413939in}{1.288615in}}%
\pgfpathlineto{\pgfqpoint{0.414019in}{1.288527in}}%
\pgfpathlineto{\pgfqpoint{0.425671in}{1.277158in}}%
\pgfpathlineto{\pgfqpoint{0.425712in}{1.277116in}}%
\pgfpathlineto{\pgfqpoint{0.437403in}{1.265896in}}%
\pgfpathlineto{\pgfqpoint{0.437601in}{1.265705in}}%
\pgfpathlineto{\pgfqpoint{0.449135in}{1.254512in}}%
\pgfpathlineto{\pgfqpoint{0.449362in}{1.254295in}}%
\pgfpathlineto{\pgfqpoint{0.460867in}{1.243037in}}%
\pgfpathlineto{\pgfqpoint{0.461027in}{1.242884in}}%
\pgfpathlineto{\pgfqpoint{0.472600in}{1.231513in}}%
\pgfpathlineto{\pgfqpoint{0.472642in}{1.231473in}}%
\pgfpathlineto{\pgfqpoint{0.483603in}{1.220062in}}%
\pgfpathlineto{\pgfqpoint{0.484332in}{1.219273in}}%
\pgfpathlineto{\pgfqpoint{0.494399in}{1.208651in}}%
\pgfpathlineto{\pgfqpoint{0.496064in}{1.206877in}}%
\pgfpathlineto{\pgfqpoint{0.505330in}{1.197241in}}%
\pgfpathlineto{\pgfqpoint{0.507796in}{1.194715in}}%
\pgfpathlineto{\pgfqpoint{0.516654in}{1.185830in}}%
\pgfpathlineto{\pgfqpoint{0.519528in}{1.183201in}}%
\pgfpathlineto{\pgfqpoint{0.528902in}{1.174419in}}%
\pgfpathlineto{\pgfqpoint{0.531261in}{1.172658in}}%
\pgfpathlineto{\pgfqpoint{0.542966in}{1.163008in}}%
\pgfpathlineto{\pgfqpoint{0.542993in}{1.162990in}}%
\pgfpathlineto{\pgfqpoint{0.554725in}{1.155727in}}%
\pgfpathlineto{\pgfqpoint{0.560724in}{1.151598in}}%
\pgfpathlineto{\pgfqpoint{0.566457in}{1.148221in}}%
\pgfpathlineto{\pgfqpoint{0.578190in}{1.141317in}}%
\pgfpathlineto{\pgfqpoint{0.580369in}{1.140187in}}%
\pgfpathlineto{\pgfqpoint{0.589922in}{1.136033in}}%
\pgfpathlineto{\pgfqpoint{0.601654in}{1.131331in}}%
\pgfpathlineto{\pgfqpoint{0.611672in}{1.128776in}}%
\pgfpathclose%
\pgfusepath{fill}%
\end{pgfscope}%
\begin{pgfscope}%
\pgfpathrectangle{\pgfqpoint{0.211875in}{0.211875in}}{\pgfqpoint{1.313625in}{1.279725in}}%
\pgfusepath{clip}%
\pgfsetbuttcap%
\pgfsetroundjoin%
\definecolor{currentfill}{rgb}{0.312084,0.115053,0.298245}%
\pgfsetfillcolor{currentfill}%
\pgfsetlinewidth{0.000000pt}%
\definecolor{currentstroke}{rgb}{0.000000,0.000000,0.000000}%
\pgfsetstrokecolor{currentstroke}%
\pgfsetdash{}{0pt}%
\pgfpathmoveto{\pgfqpoint{0.707244in}{1.305061in}}%
\pgfpathlineto{\pgfqpoint{0.708311in}{1.311348in}}%
\pgfpathlineto{\pgfqpoint{0.707244in}{1.312837in}}%
\pgfpathlineto{\pgfqpoint{0.697884in}{1.311348in}}%
\pgfpathclose%
\pgfusepath{fill}%
\end{pgfscope}%
\begin{pgfscope}%
\pgfpathrectangle{\pgfqpoint{0.211875in}{0.211875in}}{\pgfqpoint{1.313625in}{1.279725in}}%
\pgfusepath{clip}%
\pgfsetbuttcap%
\pgfsetroundjoin%
\definecolor{currentfill}{rgb}{0.312084,0.115053,0.298245}%
\pgfsetfillcolor{currentfill}%
\pgfsetlinewidth{0.000000pt}%
\definecolor{currentstroke}{rgb}{0.000000,0.000000,0.000000}%
\pgfsetstrokecolor{currentstroke}%
\pgfsetdash{}{0pt}%
\pgfpathmoveto{\pgfqpoint{0.683779in}{1.322676in}}%
\pgfpathlineto{\pgfqpoint{0.695512in}{1.312049in}}%
\pgfpathlineto{\pgfqpoint{0.699826in}{1.322759in}}%
\pgfpathlineto{\pgfqpoint{0.695512in}{1.328493in}}%
\pgfpathlineto{\pgfqpoint{0.689159in}{1.334170in}}%
\pgfpathlineto{\pgfqpoint{0.683779in}{1.341234in}}%
\pgfpathlineto{\pgfqpoint{0.675706in}{1.345581in}}%
\pgfpathlineto{\pgfqpoint{0.672047in}{1.351772in}}%
\pgfpathlineto{\pgfqpoint{0.669642in}{1.345581in}}%
\pgfpathlineto{\pgfqpoint{0.672047in}{1.341100in}}%
\pgfpathlineto{\pgfqpoint{0.675582in}{1.334170in}}%
\pgfpathlineto{\pgfqpoint{0.683692in}{1.322759in}}%
\pgfpathclose%
\pgfusepath{fill}%
\end{pgfscope}%
\begin{pgfscope}%
\pgfpathrectangle{\pgfqpoint{0.211875in}{0.211875in}}{\pgfqpoint{1.313625in}{1.279725in}}%
\pgfusepath{clip}%
\pgfsetbuttcap%
\pgfsetroundjoin%
\definecolor{currentfill}{rgb}{0.445089,0.121699,0.342662}%
\pgfsetfillcolor{currentfill}%
\pgfsetlinewidth{0.000000pt}%
\definecolor{currentstroke}{rgb}{0.000000,0.000000,0.000000}%
\pgfsetstrokecolor{currentstroke}%
\pgfsetdash{}{0pt}%
\pgfpathmoveto{\pgfqpoint{0.296617in}{0.284378in}}%
\pgfpathlineto{\pgfqpoint{0.308349in}{0.284378in}}%
\pgfpathlineto{\pgfqpoint{0.310030in}{0.284378in}}%
\pgfpathlineto{\pgfqpoint{0.320081in}{0.295761in}}%
\pgfpathlineto{\pgfqpoint{0.320105in}{0.295789in}}%
\pgfpathlineto{\pgfqpoint{0.329924in}{0.307200in}}%
\pgfpathlineto{\pgfqpoint{0.331813in}{0.309364in}}%
\pgfpathlineto{\pgfqpoint{0.339507in}{0.318611in}}%
\pgfpathlineto{\pgfqpoint{0.343545in}{0.323363in}}%
\pgfpathlineto{\pgfqpoint{0.348951in}{0.330022in}}%
\pgfpathlineto{\pgfqpoint{0.355278in}{0.337715in}}%
\pgfpathlineto{\pgfqpoint{0.358221in}{0.341432in}}%
\pgfpathlineto{\pgfqpoint{0.367010in}{0.352398in}}%
\pgfpathlineto{\pgfqpoint{0.367354in}{0.352843in}}%
\pgfpathlineto{\pgfqpoint{0.376316in}{0.364254in}}%
\pgfpathlineto{\pgfqpoint{0.378742in}{0.367316in}}%
\pgfpathlineto{\pgfqpoint{0.385148in}{0.375665in}}%
\pgfpathlineto{\pgfqpoint{0.390474in}{0.382570in}}%
\pgfpathlineto{\pgfqpoint{0.393853in}{0.387075in}}%
\pgfpathlineto{\pgfqpoint{0.402206in}{0.398203in}}%
\pgfpathlineto{\pgfqpoint{0.402414in}{0.398486in}}%
\pgfpathlineto{\pgfqpoint{0.410770in}{0.409897in}}%
\pgfpathlineto{\pgfqpoint{0.413939in}{0.414240in}}%
\pgfpathlineto{\pgfqpoint{0.418991in}{0.421308in}}%
\pgfpathlineto{\pgfqpoint{0.425671in}{0.430502in}}%
\pgfpathlineto{\pgfqpoint{0.427232in}{0.432719in}}%
\pgfpathlineto{\pgfqpoint{0.435798in}{0.444129in}}%
\pgfpathlineto{\pgfqpoint{0.437403in}{0.446263in}}%
\pgfpathlineto{\pgfqpoint{0.444306in}{0.455540in}}%
\pgfpathlineto{\pgfqpoint{0.449135in}{0.462138in}}%
\pgfpathlineto{\pgfqpoint{0.452647in}{0.466951in}}%
\pgfpathlineto{\pgfqpoint{0.460769in}{0.478362in}}%
\pgfpathlineto{\pgfqpoint{0.460867in}{0.478504in}}%
\pgfpathlineto{\pgfqpoint{0.468671in}{0.489772in}}%
\pgfpathlineto{\pgfqpoint{0.472600in}{0.495631in}}%
\pgfpathlineto{\pgfqpoint{0.476350in}{0.501183in}}%
\pgfpathlineto{\pgfqpoint{0.483726in}{0.512594in}}%
\pgfpathlineto{\pgfqpoint{0.484332in}{0.513466in}}%
\pgfpathlineto{\pgfqpoint{0.491197in}{0.524005in}}%
\pgfpathlineto{\pgfqpoint{0.496064in}{0.531047in}}%
\pgfpathlineto{\pgfqpoint{0.499007in}{0.535416in}}%
\pgfpathlineto{\pgfqpoint{0.506627in}{0.546826in}}%
\pgfpathlineto{\pgfqpoint{0.507796in}{0.548761in}}%
\pgfpathlineto{\pgfqpoint{0.513637in}{0.558237in}}%
\pgfpathlineto{\pgfqpoint{0.519528in}{0.568596in}}%
\pgfpathlineto{\pgfqpoint{0.520146in}{0.569648in}}%
\pgfpathlineto{\pgfqpoint{0.525835in}{0.581059in}}%
\pgfpathlineto{\pgfqpoint{0.530744in}{0.592469in}}%
\pgfpathlineto{\pgfqpoint{0.531261in}{0.595015in}}%
\pgfpathlineto{\pgfqpoint{0.533119in}{0.603880in}}%
\pgfpathlineto{\pgfqpoint{0.531261in}{0.607570in}}%
\pgfpathlineto{\pgfqpoint{0.519528in}{0.607307in}}%
\pgfpathlineto{\pgfqpoint{0.511792in}{0.603880in}}%
\pgfpathlineto{\pgfqpoint{0.507796in}{0.601944in}}%
\pgfpathlineto{\pgfqpoint{0.496064in}{0.594245in}}%
\pgfpathlineto{\pgfqpoint{0.493484in}{0.592469in}}%
\pgfpathlineto{\pgfqpoint{0.484332in}{0.585651in}}%
\pgfpathlineto{\pgfqpoint{0.477544in}{0.581059in}}%
\pgfpathlineto{\pgfqpoint{0.472600in}{0.577430in}}%
\pgfpathlineto{\pgfqpoint{0.460900in}{0.569648in}}%
\pgfpathlineto{\pgfqpoint{0.460867in}{0.569624in}}%
\pgfpathlineto{\pgfqpoint{0.449135in}{0.559887in}}%
\pgfpathlineto{\pgfqpoint{0.447013in}{0.558237in}}%
\pgfpathlineto{\pgfqpoint{0.437403in}{0.550266in}}%
\pgfpathlineto{\pgfqpoint{0.432956in}{0.546826in}}%
\pgfpathlineto{\pgfqpoint{0.425671in}{0.540793in}}%
\pgfpathlineto{\pgfqpoint{0.418686in}{0.535416in}}%
\pgfpathlineto{\pgfqpoint{0.413939in}{0.531489in}}%
\pgfpathlineto{\pgfqpoint{0.404178in}{0.524005in}}%
\pgfpathlineto{\pgfqpoint{0.402206in}{0.522374in}}%
\pgfpathlineto{\pgfqpoint{0.390474in}{0.512994in}}%
\pgfpathlineto{\pgfqpoint{0.389993in}{0.512594in}}%
\pgfpathlineto{\pgfqpoint{0.378742in}{0.502769in}}%
\pgfpathlineto{\pgfqpoint{0.376833in}{0.501183in}}%
\pgfpathlineto{\pgfqpoint{0.367010in}{0.492579in}}%
\pgfpathlineto{\pgfqpoint{0.363638in}{0.489772in}}%
\pgfpathlineto{\pgfqpoint{0.355278in}{0.482423in}}%
\pgfpathlineto{\pgfqpoint{0.350407in}{0.478362in}}%
\pgfpathlineto{\pgfqpoint{0.343545in}{0.472303in}}%
\pgfpathlineto{\pgfqpoint{0.336911in}{0.466951in}}%
\pgfpathlineto{\pgfqpoint{0.331813in}{0.462440in}}%
\pgfpathlineto{\pgfqpoint{0.323201in}{0.455540in}}%
\pgfpathlineto{\pgfqpoint{0.320081in}{0.452787in}}%
\pgfpathlineto{\pgfqpoint{0.309274in}{0.444129in}}%
\pgfpathlineto{\pgfqpoint{0.308349in}{0.443315in}}%
\pgfpathlineto{\pgfqpoint{0.296617in}{0.433346in}}%
\pgfpathlineto{\pgfqpoint{0.295863in}{0.432719in}}%
\pgfpathlineto{\pgfqpoint{0.284884in}{0.422985in}}%
\pgfpathlineto{\pgfqpoint{0.284884in}{0.421308in}}%
\pgfpathlineto{\pgfqpoint{0.284884in}{0.409897in}}%
\pgfpathlineto{\pgfqpoint{0.284884in}{0.398486in}}%
\pgfpathlineto{\pgfqpoint{0.284884in}{0.387075in}}%
\pgfpathlineto{\pgfqpoint{0.284884in}{0.375665in}}%
\pgfpathlineto{\pgfqpoint{0.284884in}{0.364254in}}%
\pgfpathlineto{\pgfqpoint{0.284884in}{0.352843in}}%
\pgfpathlineto{\pgfqpoint{0.284884in}{0.341432in}}%
\pgfpathlineto{\pgfqpoint{0.284884in}{0.330022in}}%
\pgfpathlineto{\pgfqpoint{0.284884in}{0.318611in}}%
\pgfpathlineto{\pgfqpoint{0.284884in}{0.307200in}}%
\pgfpathlineto{\pgfqpoint{0.284884in}{0.295789in}}%
\pgfpathlineto{\pgfqpoint{0.284884in}{0.284378in}}%
\pgfpathclose%
\pgfusepath{fill}%
\end{pgfscope}%
\begin{pgfscope}%
\pgfpathrectangle{\pgfqpoint{0.211875in}{0.211875in}}{\pgfqpoint{1.313625in}{1.279725in}}%
\pgfusepath{clip}%
\pgfsetbuttcap%
\pgfsetroundjoin%
\definecolor{currentfill}{rgb}{0.445089,0.121699,0.342662}%
\pgfsetfillcolor{currentfill}%
\pgfsetlinewidth{0.000000pt}%
\definecolor{currentstroke}{rgb}{0.000000,0.000000,0.000000}%
\pgfsetstrokecolor{currentstroke}%
\pgfsetdash{}{0pt}%
\pgfpathmoveto{\pgfqpoint{0.437403in}{0.284378in}}%
\pgfpathlineto{\pgfqpoint{0.449135in}{0.284378in}}%
\pgfpathlineto{\pgfqpoint{0.455621in}{0.284378in}}%
\pgfpathlineto{\pgfqpoint{0.460867in}{0.294487in}}%
\pgfpathlineto{\pgfqpoint{0.461587in}{0.295789in}}%
\pgfpathlineto{\pgfqpoint{0.467698in}{0.307200in}}%
\pgfpathlineto{\pgfqpoint{0.472600in}{0.316589in}}%
\pgfpathlineto{\pgfqpoint{0.473716in}{0.318611in}}%
\pgfpathlineto{\pgfqpoint{0.479787in}{0.330022in}}%
\pgfpathlineto{\pgfqpoint{0.484332in}{0.338673in}}%
\pgfpathlineto{\pgfqpoint{0.485855in}{0.341432in}}%
\pgfpathlineto{\pgfqpoint{0.491893in}{0.352843in}}%
\pgfpathlineto{\pgfqpoint{0.496064in}{0.360725in}}%
\pgfpathlineto{\pgfqpoint{0.498010in}{0.364254in}}%
\pgfpathlineto{\pgfqpoint{0.504027in}{0.375665in}}%
\pgfpathlineto{\pgfqpoint{0.507796in}{0.382729in}}%
\pgfpathlineto{\pgfqpoint{0.510194in}{0.387075in}}%
\pgfpathlineto{\pgfqpoint{0.516199in}{0.398486in}}%
\pgfpathlineto{\pgfqpoint{0.519528in}{0.404668in}}%
\pgfpathlineto{\pgfqpoint{0.522416in}{0.409897in}}%
\pgfpathlineto{\pgfqpoint{0.528426in}{0.421308in}}%
\pgfpathlineto{\pgfqpoint{0.531261in}{0.426516in}}%
\pgfpathlineto{\pgfqpoint{0.534692in}{0.432719in}}%
\pgfpathlineto{\pgfqpoint{0.540723in}{0.444129in}}%
\pgfpathlineto{\pgfqpoint{0.542993in}{0.448248in}}%
\pgfpathlineto{\pgfqpoint{0.547040in}{0.455540in}}%
\pgfpathlineto{\pgfqpoint{0.553114in}{0.466951in}}%
\pgfpathlineto{\pgfqpoint{0.554725in}{0.469834in}}%
\pgfpathlineto{\pgfqpoint{0.559479in}{0.478362in}}%
\pgfpathlineto{\pgfqpoint{0.565625in}{0.489772in}}%
\pgfpathlineto{\pgfqpoint{0.566457in}{0.491240in}}%
\pgfpathlineto{\pgfqpoint{0.572033in}{0.501183in}}%
\pgfpathlineto{\pgfqpoint{0.578190in}{0.512423in}}%
\pgfpathlineto{\pgfqpoint{0.578289in}{0.512594in}}%
\pgfpathlineto{\pgfqpoint{0.584625in}{0.524005in}}%
\pgfpathlineto{\pgfqpoint{0.589922in}{0.533986in}}%
\pgfpathlineto{\pgfqpoint{0.590739in}{0.535416in}}%
\pgfpathlineto{\pgfqpoint{0.596754in}{0.546826in}}%
\pgfpathlineto{\pgfqpoint{0.601654in}{0.556009in}}%
\pgfpathlineto{\pgfqpoint{0.602910in}{0.558237in}}%
\pgfpathlineto{\pgfqpoint{0.609072in}{0.569648in}}%
\pgfpathlineto{\pgfqpoint{0.613386in}{0.577529in}}%
\pgfpathlineto{\pgfqpoint{0.615424in}{0.581059in}}%
\pgfpathlineto{\pgfqpoint{0.621690in}{0.592469in}}%
\pgfpathlineto{\pgfqpoint{0.625118in}{0.598571in}}%
\pgfpathlineto{\pgfqpoint{0.628257in}{0.603880in}}%
\pgfpathlineto{\pgfqpoint{0.634673in}{0.615291in}}%
\pgfpathlineto{\pgfqpoint{0.636851in}{0.618949in}}%
\pgfpathlineto{\pgfqpoint{0.641664in}{0.626702in}}%
\pgfpathlineto{\pgfqpoint{0.648351in}{0.638113in}}%
\pgfpathlineto{\pgfqpoint{0.648583in}{0.638490in}}%
\pgfpathlineto{\pgfqpoint{0.655556in}{0.649523in}}%
\pgfpathlineto{\pgfqpoint{0.660315in}{0.657429in}}%
\pgfpathlineto{\pgfqpoint{0.662628in}{0.660934in}}%
\pgfpathlineto{\pgfqpoint{0.669626in}{0.672345in}}%
\pgfpathlineto{\pgfqpoint{0.672047in}{0.675949in}}%
\pgfpathlineto{\pgfqpoint{0.677556in}{0.683756in}}%
\pgfpathlineto{\pgfqpoint{0.683779in}{0.692432in}}%
\pgfpathlineto{\pgfqpoint{0.686010in}{0.695166in}}%
\pgfpathlineto{\pgfqpoint{0.694556in}{0.706577in}}%
\pgfpathlineto{\pgfqpoint{0.695512in}{0.707886in}}%
\pgfpathlineto{\pgfqpoint{0.703693in}{0.717988in}}%
\pgfpathlineto{\pgfqpoint{0.707244in}{0.723014in}}%
\pgfpathlineto{\pgfqpoint{0.712466in}{0.729399in}}%
\pgfpathlineto{\pgfqpoint{0.718976in}{0.739182in}}%
\pgfpathlineto{\pgfqpoint{0.720281in}{0.740810in}}%
\pgfpathlineto{\pgfqpoint{0.728257in}{0.752220in}}%
\pgfpathlineto{\pgfqpoint{0.730708in}{0.756023in}}%
\pgfpathlineto{\pgfqpoint{0.736113in}{0.763631in}}%
\pgfpathlineto{\pgfqpoint{0.742440in}{0.774497in}}%
\pgfpathlineto{\pgfqpoint{0.742772in}{0.775042in}}%
\pgfpathlineto{\pgfqpoint{0.749379in}{0.786453in}}%
\pgfpathlineto{\pgfqpoint{0.754173in}{0.795226in}}%
\pgfpathlineto{\pgfqpoint{0.755607in}{0.797863in}}%
\pgfpathlineto{\pgfqpoint{0.761905in}{0.809274in}}%
\pgfpathlineto{\pgfqpoint{0.765905in}{0.815428in}}%
\pgfpathlineto{\pgfqpoint{0.769120in}{0.820685in}}%
\pgfpathlineto{\pgfqpoint{0.777637in}{0.831793in}}%
\pgfpathlineto{\pgfqpoint{0.777849in}{0.832096in}}%
\pgfpathlineto{\pgfqpoint{0.788067in}{0.843507in}}%
\pgfpathlineto{\pgfqpoint{0.789369in}{0.844591in}}%
\pgfpathlineto{\pgfqpoint{0.799914in}{0.854917in}}%
\pgfpathlineto{\pgfqpoint{0.801101in}{0.855821in}}%
\pgfpathlineto{\pgfqpoint{0.812834in}{0.865986in}}%
\pgfpathlineto{\pgfqpoint{0.813216in}{0.866328in}}%
\pgfpathlineto{\pgfqpoint{0.824566in}{0.875011in}}%
\pgfpathlineto{\pgfqpoint{0.827900in}{0.877739in}}%
\pgfpathlineto{\pgfqpoint{0.836298in}{0.884839in}}%
\pgfpathlineto{\pgfqpoint{0.843710in}{0.889150in}}%
\pgfpathlineto{\pgfqpoint{0.848030in}{0.893716in}}%
\pgfpathlineto{\pgfqpoint{0.851240in}{0.900560in}}%
\pgfpathlineto{\pgfqpoint{0.858688in}{0.911971in}}%
\pgfpathlineto{\pgfqpoint{0.859762in}{0.912609in}}%
\pgfpathlineto{\pgfqpoint{0.861967in}{0.911971in}}%
\pgfpathlineto{\pgfqpoint{0.871495in}{0.907978in}}%
\pgfpathlineto{\pgfqpoint{0.879695in}{0.900560in}}%
\pgfpathlineto{\pgfqpoint{0.883227in}{0.894080in}}%
\pgfpathlineto{\pgfqpoint{0.885700in}{0.889150in}}%
\pgfpathlineto{\pgfqpoint{0.887272in}{0.877739in}}%
\pgfpathlineto{\pgfqpoint{0.883227in}{0.872612in}}%
\pgfpathlineto{\pgfqpoint{0.871495in}{0.874882in}}%
\pgfpathlineto{\pgfqpoint{0.864034in}{0.877739in}}%
\pgfpathlineto{\pgfqpoint{0.859762in}{0.879984in}}%
\pgfpathlineto{\pgfqpoint{0.856705in}{0.877739in}}%
\pgfpathlineto{\pgfqpoint{0.852603in}{0.866328in}}%
\pgfpathlineto{\pgfqpoint{0.852414in}{0.854917in}}%
\pgfpathlineto{\pgfqpoint{0.852416in}{0.843507in}}%
\pgfpathlineto{\pgfqpoint{0.852240in}{0.832096in}}%
\pgfpathlineto{\pgfqpoint{0.851568in}{0.820685in}}%
\pgfpathlineto{\pgfqpoint{0.850602in}{0.809274in}}%
\pgfpathlineto{\pgfqpoint{0.849544in}{0.797863in}}%
\pgfpathlineto{\pgfqpoint{0.848419in}{0.786453in}}%
\pgfpathlineto{\pgfqpoint{0.848030in}{0.782595in}}%
\pgfpathlineto{\pgfqpoint{0.846581in}{0.775042in}}%
\pgfpathlineto{\pgfqpoint{0.844460in}{0.763631in}}%
\pgfpathlineto{\pgfqpoint{0.842874in}{0.752220in}}%
\pgfpathlineto{\pgfqpoint{0.842192in}{0.740810in}}%
\pgfpathlineto{\pgfqpoint{0.842059in}{0.729399in}}%
\pgfpathlineto{\pgfqpoint{0.842028in}{0.717988in}}%
\pgfpathlineto{\pgfqpoint{0.842450in}{0.706577in}}%
\pgfpathlineto{\pgfqpoint{0.843622in}{0.695166in}}%
\pgfpathlineto{\pgfqpoint{0.845402in}{0.683756in}}%
\pgfpathlineto{\pgfqpoint{0.847375in}{0.672345in}}%
\pgfpathlineto{\pgfqpoint{0.848030in}{0.668889in}}%
\pgfpathlineto{\pgfqpoint{0.849076in}{0.660934in}}%
\pgfpathlineto{\pgfqpoint{0.851033in}{0.649523in}}%
\pgfpathlineto{\pgfqpoint{0.853504in}{0.638113in}}%
\pgfpathlineto{\pgfqpoint{0.856711in}{0.626702in}}%
\pgfpathlineto{\pgfqpoint{0.859762in}{0.617216in}}%
\pgfpathlineto{\pgfqpoint{0.860168in}{0.615291in}}%
\pgfpathlineto{\pgfqpoint{0.862637in}{0.603880in}}%
\pgfpathlineto{\pgfqpoint{0.865522in}{0.592469in}}%
\pgfpathlineto{\pgfqpoint{0.868926in}{0.581059in}}%
\pgfpathlineto{\pgfqpoint{0.871495in}{0.573399in}}%
\pgfpathlineto{\pgfqpoint{0.872310in}{0.569648in}}%
\pgfpathlineto{\pgfqpoint{0.874932in}{0.558237in}}%
\pgfpathlineto{\pgfqpoint{0.878015in}{0.546826in}}%
\pgfpathlineto{\pgfqpoint{0.881684in}{0.535416in}}%
\pgfpathlineto{\pgfqpoint{0.883227in}{0.530981in}}%
\pgfpathlineto{\pgfqpoint{0.884785in}{0.524005in}}%
\pgfpathlineto{\pgfqpoint{0.887597in}{0.512594in}}%
\pgfpathlineto{\pgfqpoint{0.890927in}{0.501183in}}%
\pgfpathlineto{\pgfqpoint{0.894959in}{0.491544in}}%
\pgfpathlineto{\pgfqpoint{0.895562in}{0.489772in}}%
\pgfpathlineto{\pgfqpoint{0.899246in}{0.478362in}}%
\pgfpathlineto{\pgfqpoint{0.902085in}{0.466951in}}%
\pgfpathlineto{\pgfqpoint{0.905351in}{0.455540in}}%
\pgfpathlineto{\pgfqpoint{0.906691in}{0.449977in}}%
\pgfpathlineto{\pgfqpoint{0.907540in}{0.444129in}}%
\pgfpathlineto{\pgfqpoint{0.909326in}{0.432719in}}%
\pgfpathlineto{\pgfqpoint{0.911134in}{0.421308in}}%
\pgfpathlineto{\pgfqpoint{0.913027in}{0.409897in}}%
\pgfpathlineto{\pgfqpoint{0.915020in}{0.398486in}}%
\pgfpathlineto{\pgfqpoint{0.917130in}{0.387075in}}%
\pgfpathlineto{\pgfqpoint{0.918424in}{0.380291in}}%
\pgfpathlineto{\pgfqpoint{0.919003in}{0.375665in}}%
\pgfpathlineto{\pgfqpoint{0.920413in}{0.364254in}}%
\pgfpathlineto{\pgfqpoint{0.921868in}{0.352843in}}%
\pgfpathlineto{\pgfqpoint{0.923384in}{0.341432in}}%
\pgfpathlineto{\pgfqpoint{0.924957in}{0.330022in}}%
\pgfpathlineto{\pgfqpoint{0.926604in}{0.318611in}}%
\pgfpathlineto{\pgfqpoint{0.928339in}{0.307200in}}%
\pgfpathlineto{\pgfqpoint{0.930156in}{0.295991in}}%
\pgfpathlineto{\pgfqpoint{0.930178in}{0.295789in}}%
\pgfpathlineto{\pgfqpoint{0.931434in}{0.284378in}}%
\pgfpathlineto{\pgfqpoint{0.941888in}{0.284378in}}%
\pgfpathlineto{\pgfqpoint{0.953620in}{0.284378in}}%
\pgfpathlineto{\pgfqpoint{0.955630in}{0.284378in}}%
\pgfpathlineto{\pgfqpoint{0.953668in}{0.295789in}}%
\pgfpathlineto{\pgfqpoint{0.953620in}{0.296051in}}%
\pgfpathlineto{\pgfqpoint{0.952221in}{0.307200in}}%
\pgfpathlineto{\pgfqpoint{0.950716in}{0.318611in}}%
\pgfpathlineto{\pgfqpoint{0.949115in}{0.330022in}}%
\pgfpathlineto{\pgfqpoint{0.947406in}{0.341432in}}%
\pgfpathlineto{\pgfqpoint{0.945570in}{0.352843in}}%
\pgfpathlineto{\pgfqpoint{0.943590in}{0.364254in}}%
\pgfpathlineto{\pgfqpoint{0.941888in}{0.373395in}}%
\pgfpathlineto{\pgfqpoint{0.941539in}{0.375665in}}%
\pgfpathlineto{\pgfqpoint{0.939731in}{0.387075in}}%
\pgfpathlineto{\pgfqpoint{0.937819in}{0.398486in}}%
\pgfpathlineto{\pgfqpoint{0.935433in}{0.409897in}}%
\pgfpathlineto{\pgfqpoint{0.933140in}{0.421308in}}%
\pgfpathlineto{\pgfqpoint{0.930339in}{0.432719in}}%
\pgfpathlineto{\pgfqpoint{0.930156in}{0.433426in}}%
\pgfpathlineto{\pgfqpoint{0.927522in}{0.444129in}}%
\pgfpathlineto{\pgfqpoint{0.924777in}{0.455540in}}%
\pgfpathlineto{\pgfqpoint{0.922045in}{0.466951in}}%
\pgfpathlineto{\pgfqpoint{0.919271in}{0.478362in}}%
\pgfpathlineto{\pgfqpoint{0.918424in}{0.481701in}}%
\pgfpathlineto{\pgfqpoint{0.916202in}{0.489772in}}%
\pgfpathlineto{\pgfqpoint{0.912747in}{0.501183in}}%
\pgfpathlineto{\pgfqpoint{0.908475in}{0.512594in}}%
\pgfpathlineto{\pgfqpoint{0.906691in}{0.517201in}}%
\pgfpathlineto{\pgfqpoint{0.904129in}{0.524005in}}%
\pgfpathlineto{\pgfqpoint{0.900896in}{0.535416in}}%
\pgfpathlineto{\pgfqpoint{0.897930in}{0.546826in}}%
\pgfpathlineto{\pgfqpoint{0.895116in}{0.558237in}}%
\pgfpathlineto{\pgfqpoint{0.894959in}{0.558858in}}%
\pgfpathlineto{\pgfqpoint{0.891648in}{0.569648in}}%
\pgfpathlineto{\pgfqpoint{0.888520in}{0.581059in}}%
\pgfpathlineto{\pgfqpoint{0.885662in}{0.592469in}}%
\pgfpathlineto{\pgfqpoint{0.883227in}{0.602709in}}%
\pgfpathlineto{\pgfqpoint{0.882865in}{0.603880in}}%
\pgfpathlineto{\pgfqpoint{0.879446in}{0.615291in}}%
\pgfpathlineto{\pgfqpoint{0.876412in}{0.626702in}}%
\pgfpathlineto{\pgfqpoint{0.873648in}{0.638113in}}%
\pgfpathlineto{\pgfqpoint{0.871495in}{0.647480in}}%
\pgfpathlineto{\pgfqpoint{0.870874in}{0.649523in}}%
\pgfpathlineto{\pgfqpoint{0.867601in}{0.660934in}}%
\pgfpathlineto{\pgfqpoint{0.864899in}{0.672345in}}%
\pgfpathlineto{\pgfqpoint{0.862588in}{0.683756in}}%
\pgfpathlineto{\pgfqpoint{0.860418in}{0.695166in}}%
\pgfpathlineto{\pgfqpoint{0.859762in}{0.699088in}}%
\pgfpathlineto{\pgfqpoint{0.858241in}{0.706577in}}%
\pgfpathlineto{\pgfqpoint{0.856533in}{0.717988in}}%
\pgfpathlineto{\pgfqpoint{0.855383in}{0.729399in}}%
\pgfpathlineto{\pgfqpoint{0.855071in}{0.740810in}}%
\pgfpathlineto{\pgfqpoint{0.854909in}{0.752220in}}%
\pgfpathlineto{\pgfqpoint{0.855452in}{0.763631in}}%
\pgfpathlineto{\pgfqpoint{0.856502in}{0.775042in}}%
\pgfpathlineto{\pgfqpoint{0.857647in}{0.786453in}}%
\pgfpathlineto{\pgfqpoint{0.858679in}{0.797863in}}%
\pgfpathlineto{\pgfqpoint{0.859561in}{0.809274in}}%
\pgfpathlineto{\pgfqpoint{0.859762in}{0.812335in}}%
\pgfpathlineto{\pgfqpoint{0.860206in}{0.820685in}}%
\pgfpathlineto{\pgfqpoint{0.860897in}{0.832096in}}%
\pgfpathlineto{\pgfqpoint{0.861331in}{0.843507in}}%
\pgfpathlineto{\pgfqpoint{0.862457in}{0.854917in}}%
\pgfpathlineto{\pgfqpoint{0.871495in}{0.862499in}}%
\pgfpathlineto{\pgfqpoint{0.883227in}{0.859306in}}%
\pgfpathlineto{\pgfqpoint{0.894959in}{0.855230in}}%
\pgfpathlineto{\pgfqpoint{0.901402in}{0.866328in}}%
\pgfpathlineto{\pgfqpoint{0.897841in}{0.877739in}}%
\pgfpathlineto{\pgfqpoint{0.894959in}{0.884095in}}%
\pgfpathlineto{\pgfqpoint{0.892843in}{0.889150in}}%
\pgfpathlineto{\pgfqpoint{0.887210in}{0.900560in}}%
\pgfpathlineto{\pgfqpoint{0.883227in}{0.905399in}}%
\pgfpathlineto{\pgfqpoint{0.875813in}{0.911971in}}%
\pgfpathlineto{\pgfqpoint{0.871495in}{0.914009in}}%
\pgfpathlineto{\pgfqpoint{0.859762in}{0.917458in}}%
\pgfpathlineto{\pgfqpoint{0.850525in}{0.911971in}}%
\pgfpathlineto{\pgfqpoint{0.848030in}{0.908422in}}%
\pgfpathlineto{\pgfqpoint{0.841745in}{0.900560in}}%
\pgfpathlineto{\pgfqpoint{0.836298in}{0.895059in}}%
\pgfpathlineto{\pgfqpoint{0.830691in}{0.889150in}}%
\pgfpathlineto{\pgfqpoint{0.824566in}{0.883719in}}%
\pgfpathlineto{\pgfqpoint{0.816591in}{0.877739in}}%
\pgfpathlineto{\pgfqpoint{0.812834in}{0.874677in}}%
\pgfpathlineto{\pgfqpoint{0.801180in}{0.866328in}}%
\pgfpathlineto{\pgfqpoint{0.801101in}{0.866260in}}%
\pgfpathlineto{\pgfqpoint{0.789369in}{0.857161in}}%
\pgfpathlineto{\pgfqpoint{0.786092in}{0.854917in}}%
\pgfpathlineto{\pgfqpoint{0.777637in}{0.847753in}}%
\pgfpathlineto{\pgfqpoint{0.771752in}{0.843507in}}%
\pgfpathlineto{\pgfqpoint{0.765905in}{0.838052in}}%
\pgfpathlineto{\pgfqpoint{0.759796in}{0.832096in}}%
\pgfpathlineto{\pgfqpoint{0.754173in}{0.825718in}}%
\pgfpathlineto{\pgfqpoint{0.749164in}{0.820685in}}%
\pgfpathlineto{\pgfqpoint{0.742440in}{0.812391in}}%
\pgfpathlineto{\pgfqpoint{0.739393in}{0.809274in}}%
\pgfpathlineto{\pgfqpoint{0.730708in}{0.798376in}}%
\pgfpathlineto{\pgfqpoint{0.730209in}{0.797863in}}%
\pgfpathlineto{\pgfqpoint{0.722116in}{0.786453in}}%
\pgfpathlineto{\pgfqpoint{0.718976in}{0.780249in}}%
\pgfpathlineto{\pgfqpoint{0.715466in}{0.775042in}}%
\pgfpathlineto{\pgfqpoint{0.710000in}{0.763631in}}%
\pgfpathlineto{\pgfqpoint{0.707244in}{0.757300in}}%
\pgfpathlineto{\pgfqpoint{0.704635in}{0.752220in}}%
\pgfpathlineto{\pgfqpoint{0.698023in}{0.740810in}}%
\pgfpathlineto{\pgfqpoint{0.695512in}{0.736827in}}%
\pgfpathlineto{\pgfqpoint{0.690215in}{0.729399in}}%
\pgfpathlineto{\pgfqpoint{0.683779in}{0.719381in}}%
\pgfpathlineto{\pgfqpoint{0.682732in}{0.717988in}}%
\pgfpathlineto{\pgfqpoint{0.674142in}{0.706577in}}%
\pgfpathlineto{\pgfqpoint{0.672047in}{0.703651in}}%
\pgfpathlineto{\pgfqpoint{0.665131in}{0.695166in}}%
\pgfpathlineto{\pgfqpoint{0.660315in}{0.688342in}}%
\pgfpathlineto{\pgfqpoint{0.656199in}{0.683756in}}%
\pgfpathlineto{\pgfqpoint{0.648583in}{0.673413in}}%
\pgfpathlineto{\pgfqpoint{0.647592in}{0.672345in}}%
\pgfpathlineto{\pgfqpoint{0.638896in}{0.660934in}}%
\pgfpathlineto{\pgfqpoint{0.636851in}{0.657742in}}%
\pgfpathlineto{\pgfqpoint{0.630255in}{0.649523in}}%
\pgfpathlineto{\pgfqpoint{0.625118in}{0.641131in}}%
\pgfpathlineto{\pgfqpoint{0.622599in}{0.638113in}}%
\pgfpathlineto{\pgfqpoint{0.615404in}{0.626702in}}%
\pgfpathlineto{\pgfqpoint{0.613386in}{0.622810in}}%
\pgfpathlineto{\pgfqpoint{0.608129in}{0.615291in}}%
\pgfpathlineto{\pgfqpoint{0.602285in}{0.603880in}}%
\pgfpathlineto{\pgfqpoint{0.601654in}{0.602621in}}%
\pgfpathlineto{\pgfqpoint{0.594906in}{0.592469in}}%
\pgfpathlineto{\pgfqpoint{0.589922in}{0.581874in}}%
\pgfpathlineto{\pgfqpoint{0.589372in}{0.581059in}}%
\pgfpathlineto{\pgfqpoint{0.582241in}{0.569648in}}%
\pgfpathlineto{\pgfqpoint{0.578190in}{0.560882in}}%
\pgfpathlineto{\pgfqpoint{0.576450in}{0.558237in}}%
\pgfpathlineto{\pgfqpoint{0.569931in}{0.546826in}}%
\pgfpathlineto{\pgfqpoint{0.566457in}{0.538957in}}%
\pgfpathlineto{\pgfqpoint{0.564247in}{0.535416in}}%
\pgfpathlineto{\pgfqpoint{0.558143in}{0.524005in}}%
\pgfpathlineto{\pgfqpoint{0.554725in}{0.516067in}}%
\pgfpathlineto{\pgfqpoint{0.552638in}{0.512594in}}%
\pgfpathlineto{\pgfqpoint{0.546532in}{0.501183in}}%
\pgfpathlineto{\pgfqpoint{0.542993in}{0.493228in}}%
\pgfpathlineto{\pgfqpoint{0.540894in}{0.489772in}}%
\pgfpathlineto{\pgfqpoint{0.534256in}{0.478362in}}%
\pgfpathlineto{\pgfqpoint{0.531261in}{0.472295in}}%
\pgfpathlineto{\pgfqpoint{0.527727in}{0.466951in}}%
\pgfpathlineto{\pgfqpoint{0.521569in}{0.455540in}}%
\pgfpathlineto{\pgfqpoint{0.519528in}{0.451326in}}%
\pgfpathlineto{\pgfqpoint{0.514934in}{0.444129in}}%
\pgfpathlineto{\pgfqpoint{0.509132in}{0.432719in}}%
\pgfpathlineto{\pgfqpoint{0.507796in}{0.429948in}}%
\pgfpathlineto{\pgfqpoint{0.502359in}{0.421308in}}%
\pgfpathlineto{\pgfqpoint{0.496805in}{0.409897in}}%
\pgfpathlineto{\pgfqpoint{0.496064in}{0.408358in}}%
\pgfpathlineto{\pgfqpoint{0.489925in}{0.398486in}}%
\pgfpathlineto{\pgfqpoint{0.484556in}{0.387075in}}%
\pgfpathlineto{\pgfqpoint{0.484332in}{0.386610in}}%
\pgfpathlineto{\pgfqpoint{0.477589in}{0.375665in}}%
\pgfpathlineto{\pgfqpoint{0.472600in}{0.364928in}}%
\pgfpathlineto{\pgfqpoint{0.472175in}{0.364254in}}%
\pgfpathlineto{\pgfqpoint{0.465322in}{0.352843in}}%
\pgfpathlineto{\pgfqpoint{0.460867in}{0.343287in}}%
\pgfpathlineto{\pgfqpoint{0.459710in}{0.341432in}}%
\pgfpathlineto{\pgfqpoint{0.453099in}{0.330022in}}%
\pgfpathlineto{\pgfqpoint{0.449135in}{0.321555in}}%
\pgfpathlineto{\pgfqpoint{0.447309in}{0.318611in}}%
\pgfpathlineto{\pgfqpoint{0.440904in}{0.307200in}}%
\pgfpathlineto{\pgfqpoint{0.437403in}{0.299763in}}%
\pgfpathlineto{\pgfqpoint{0.434951in}{0.295789in}}%
\pgfpathlineto{\pgfqpoint{0.428723in}{0.284378in}}%
\pgfpathclose%
\pgfusepath{fill}%
\end{pgfscope}%
\begin{pgfscope}%
\pgfpathrectangle{\pgfqpoint{0.211875in}{0.211875in}}{\pgfqpoint{1.313625in}{1.279725in}}%
\pgfusepath{clip}%
\pgfsetbuttcap%
\pgfsetroundjoin%
\definecolor{currentfill}{rgb}{0.445089,0.121699,0.342662}%
\pgfsetfillcolor{currentfill}%
\pgfsetlinewidth{0.000000pt}%
\definecolor{currentstroke}{rgb}{0.000000,0.000000,0.000000}%
\pgfsetstrokecolor{currentstroke}%
\pgfsetdash{}{0pt}%
\pgfpathmoveto{\pgfqpoint{1.070942in}{0.494587in}}%
\pgfpathlineto{\pgfqpoint{1.074011in}{0.501183in}}%
\pgfpathlineto{\pgfqpoint{1.073792in}{0.512594in}}%
\pgfpathlineto{\pgfqpoint{1.073035in}{0.524005in}}%
\pgfpathlineto{\pgfqpoint{1.071809in}{0.535416in}}%
\pgfpathlineto{\pgfqpoint{1.070942in}{0.541270in}}%
\pgfpathlineto{\pgfqpoint{1.070175in}{0.546826in}}%
\pgfpathlineto{\pgfqpoint{1.068167in}{0.558237in}}%
\pgfpathlineto{\pgfqpoint{1.065775in}{0.569648in}}%
\pgfpathlineto{\pgfqpoint{1.063066in}{0.581059in}}%
\pgfpathlineto{\pgfqpoint{1.060132in}{0.592469in}}%
\pgfpathlineto{\pgfqpoint{1.059210in}{0.595680in}}%
\pgfpathlineto{\pgfqpoint{1.056827in}{0.603880in}}%
\pgfpathlineto{\pgfqpoint{1.053278in}{0.615291in}}%
\pgfpathlineto{\pgfqpoint{1.049798in}{0.626702in}}%
\pgfpathlineto{\pgfqpoint{1.047478in}{0.632938in}}%
\pgfpathlineto{\pgfqpoint{1.045681in}{0.638113in}}%
\pgfpathlineto{\pgfqpoint{1.041549in}{0.649523in}}%
\pgfpathlineto{\pgfqpoint{1.036573in}{0.660934in}}%
\pgfpathlineto{\pgfqpoint{1.035746in}{0.661867in}}%
\pgfpathlineto{\pgfqpoint{1.029103in}{0.672345in}}%
\pgfpathlineto{\pgfqpoint{1.024013in}{0.680531in}}%
\pgfpathlineto{\pgfqpoint{1.019408in}{0.683756in}}%
\pgfpathlineto{\pgfqpoint{1.012281in}{0.693173in}}%
\pgfpathlineto{\pgfqpoint{1.000549in}{0.686718in}}%
\pgfpathlineto{\pgfqpoint{0.999484in}{0.683756in}}%
\pgfpathlineto{\pgfqpoint{0.998527in}{0.672345in}}%
\pgfpathlineto{\pgfqpoint{0.998711in}{0.660934in}}%
\pgfpathlineto{\pgfqpoint{0.999482in}{0.649523in}}%
\pgfpathlineto{\pgfqpoint{1.000549in}{0.644879in}}%
\pgfpathlineto{\pgfqpoint{1.003764in}{0.638113in}}%
\pgfpathlineto{\pgfqpoint{1.006910in}{0.626702in}}%
\pgfpathlineto{\pgfqpoint{1.009182in}{0.615291in}}%
\pgfpathlineto{\pgfqpoint{1.012281in}{0.607099in}}%
\pgfpathlineto{\pgfqpoint{1.014856in}{0.603880in}}%
\pgfpathlineto{\pgfqpoint{1.020217in}{0.592469in}}%
\pgfpathlineto{\pgfqpoint{1.023872in}{0.581059in}}%
\pgfpathlineto{\pgfqpoint{1.024013in}{0.580734in}}%
\pgfpathlineto{\pgfqpoint{1.031478in}{0.569648in}}%
\pgfpathlineto{\pgfqpoint{1.034597in}{0.558237in}}%
\pgfpathlineto{\pgfqpoint{1.035746in}{0.555597in}}%
\pgfpathlineto{\pgfqpoint{1.042929in}{0.546826in}}%
\pgfpathlineto{\pgfqpoint{1.046712in}{0.535416in}}%
\pgfpathlineto{\pgfqpoint{1.047478in}{0.533248in}}%
\pgfpathlineto{\pgfqpoint{1.054532in}{0.524005in}}%
\pgfpathlineto{\pgfqpoint{1.059093in}{0.512594in}}%
\pgfpathlineto{\pgfqpoint{1.059210in}{0.512121in}}%
\pgfpathlineto{\pgfqpoint{1.067643in}{0.501183in}}%
\pgfpathclose%
\pgfusepath{fill}%
\end{pgfscope}%
\begin{pgfscope}%
\pgfpathrectangle{\pgfqpoint{0.211875in}{0.211875in}}{\pgfqpoint{1.313625in}{1.279725in}}%
\pgfusepath{clip}%
\pgfsetbuttcap%
\pgfsetroundjoin%
\definecolor{currentfill}{rgb}{0.445089,0.121699,0.342662}%
\pgfsetfillcolor{currentfill}%
\pgfsetlinewidth{0.000000pt}%
\definecolor{currentstroke}{rgb}{0.000000,0.000000,0.000000}%
\pgfsetstrokecolor{currentstroke}%
\pgfsetdash{}{0pt}%
\pgfpathmoveto{\pgfqpoint{1.375980in}{0.500927in}}%
\pgfpathlineto{\pgfqpoint{1.387712in}{0.493470in}}%
\pgfpathlineto{\pgfqpoint{1.399444in}{0.492321in}}%
\pgfpathlineto{\pgfqpoint{1.411176in}{0.496416in}}%
\pgfpathlineto{\pgfqpoint{1.420668in}{0.501183in}}%
\pgfpathlineto{\pgfqpoint{1.422908in}{0.502594in}}%
\pgfpathlineto{\pgfqpoint{1.434641in}{0.511797in}}%
\pgfpathlineto{\pgfqpoint{1.435338in}{0.512594in}}%
\pgfpathlineto{\pgfqpoint{1.441488in}{0.524005in}}%
\pgfpathlineto{\pgfqpoint{1.443917in}{0.535416in}}%
\pgfpathlineto{\pgfqpoint{1.444120in}{0.546826in}}%
\pgfpathlineto{\pgfqpoint{1.442632in}{0.558237in}}%
\pgfpathlineto{\pgfqpoint{1.440353in}{0.569648in}}%
\pgfpathlineto{\pgfqpoint{1.437559in}{0.581059in}}%
\pgfpathlineto{\pgfqpoint{1.434641in}{0.592200in}}%
\pgfpathlineto{\pgfqpoint{1.434562in}{0.592469in}}%
\pgfpathlineto{\pgfqpoint{1.430766in}{0.603880in}}%
\pgfpathlineto{\pgfqpoint{1.427008in}{0.615291in}}%
\pgfpathlineto{\pgfqpoint{1.423270in}{0.626702in}}%
\pgfpathlineto{\pgfqpoint{1.422908in}{0.627636in}}%
\pgfpathlineto{\pgfqpoint{1.418057in}{0.638113in}}%
\pgfpathlineto{\pgfqpoint{1.412769in}{0.649523in}}%
\pgfpathlineto{\pgfqpoint{1.411176in}{0.652871in}}%
\pgfpathlineto{\pgfqpoint{1.405761in}{0.660934in}}%
\pgfpathlineto{\pgfqpoint{1.399444in}{0.670702in}}%
\pgfpathlineto{\pgfqpoint{1.396698in}{0.672345in}}%
\pgfpathlineto{\pgfqpoint{1.387712in}{0.677807in}}%
\pgfpathlineto{\pgfqpoint{1.381774in}{0.672345in}}%
\pgfpathlineto{\pgfqpoint{1.375980in}{0.667981in}}%
\pgfpathlineto{\pgfqpoint{1.370538in}{0.660934in}}%
\pgfpathlineto{\pgfqpoint{1.364247in}{0.652061in}}%
\pgfpathlineto{\pgfqpoint{1.363182in}{0.649523in}}%
\pgfpathlineto{\pgfqpoint{1.359001in}{0.638113in}}%
\pgfpathlineto{\pgfqpoint{1.355241in}{0.626702in}}%
\pgfpathlineto{\pgfqpoint{1.352515in}{0.617214in}}%
\pgfpathlineto{\pgfqpoint{1.352180in}{0.615291in}}%
\pgfpathlineto{\pgfqpoint{1.351461in}{0.603880in}}%
\pgfpathlineto{\pgfqpoint{1.351252in}{0.592469in}}%
\pgfpathlineto{\pgfqpoint{1.351442in}{0.581059in}}%
\pgfpathlineto{\pgfqpoint{1.352515in}{0.572783in}}%
\pgfpathlineto{\pgfqpoint{1.353457in}{0.569648in}}%
\pgfpathlineto{\pgfqpoint{1.356339in}{0.558237in}}%
\pgfpathlineto{\pgfqpoint{1.358669in}{0.546826in}}%
\pgfpathlineto{\pgfqpoint{1.361258in}{0.535416in}}%
\pgfpathlineto{\pgfqpoint{1.364247in}{0.524743in}}%
\pgfpathlineto{\pgfqpoint{1.364707in}{0.524005in}}%
\pgfpathlineto{\pgfqpoint{1.370794in}{0.512594in}}%
\pgfpathlineto{\pgfqpoint{1.375848in}{0.501183in}}%
\pgfpathclose%
\pgfusepath{fill}%
\end{pgfscope}%
\begin{pgfscope}%
\pgfpathrectangle{\pgfqpoint{0.211875in}{0.211875in}}{\pgfqpoint{1.313625in}{1.279725in}}%
\pgfusepath{clip}%
\pgfsetbuttcap%
\pgfsetroundjoin%
\definecolor{currentfill}{rgb}{0.445089,0.121699,0.342662}%
\pgfsetfillcolor{currentfill}%
\pgfsetlinewidth{0.000000pt}%
\definecolor{currentstroke}{rgb}{0.000000,0.000000,0.000000}%
\pgfsetstrokecolor{currentstroke}%
\pgfsetdash{}{0pt}%
\pgfpathmoveto{\pgfqpoint{0.294891in}{0.638113in}}%
\pgfpathlineto{\pgfqpoint{0.296617in}{0.638983in}}%
\pgfpathlineto{\pgfqpoint{0.308349in}{0.644894in}}%
\pgfpathlineto{\pgfqpoint{0.317363in}{0.649523in}}%
\pgfpathlineto{\pgfqpoint{0.320081in}{0.650902in}}%
\pgfpathlineto{\pgfqpoint{0.331813in}{0.656858in}}%
\pgfpathlineto{\pgfqpoint{0.339685in}{0.660934in}}%
\pgfpathlineto{\pgfqpoint{0.343545in}{0.662907in}}%
\pgfpathlineto{\pgfqpoint{0.355278in}{0.668921in}}%
\pgfpathlineto{\pgfqpoint{0.361822in}{0.672345in}}%
\pgfpathlineto{\pgfqpoint{0.367010in}{0.675022in}}%
\pgfpathlineto{\pgfqpoint{0.378742in}{0.681107in}}%
\pgfpathlineto{\pgfqpoint{0.383744in}{0.683756in}}%
\pgfpathlineto{\pgfqpoint{0.390474in}{0.687271in}}%
\pgfpathlineto{\pgfqpoint{0.402206in}{0.693440in}}%
\pgfpathlineto{\pgfqpoint{0.405422in}{0.695166in}}%
\pgfpathlineto{\pgfqpoint{0.413939in}{0.699678in}}%
\pgfpathlineto{\pgfqpoint{0.425671in}{0.705941in}}%
\pgfpathlineto{\pgfqpoint{0.426838in}{0.706577in}}%
\pgfpathlineto{\pgfqpoint{0.437403in}{0.712262in}}%
\pgfpathlineto{\pgfqpoint{0.447959in}{0.717988in}}%
\pgfpathlineto{\pgfqpoint{0.449135in}{0.718629in}}%
\pgfpathlineto{\pgfqpoint{0.460867in}{0.725037in}}%
\pgfpathlineto{\pgfqpoint{0.468780in}{0.729399in}}%
\pgfpathlineto{\pgfqpoint{0.472600in}{0.731517in}}%
\pgfpathlineto{\pgfqpoint{0.484332in}{0.738010in}}%
\pgfpathlineto{\pgfqpoint{0.489332in}{0.740810in}}%
\pgfpathlineto{\pgfqpoint{0.496064in}{0.744608in}}%
\pgfpathlineto{\pgfqpoint{0.507796in}{0.751174in}}%
\pgfpathlineto{\pgfqpoint{0.509683in}{0.752220in}}%
\pgfpathlineto{\pgfqpoint{0.519528in}{0.757747in}}%
\pgfpathlineto{\pgfqpoint{0.530157in}{0.763631in}}%
\pgfpathlineto{\pgfqpoint{0.531261in}{0.764264in}}%
\pgfpathlineto{\pgfqpoint{0.542993in}{0.770921in}}%
\pgfpathlineto{\pgfqpoint{0.550332in}{0.775042in}}%
\pgfpathlineto{\pgfqpoint{0.554725in}{0.777613in}}%
\pgfpathlineto{\pgfqpoint{0.566457in}{0.784318in}}%
\pgfpathlineto{\pgfqpoint{0.570206in}{0.786453in}}%
\pgfpathlineto{\pgfqpoint{0.578190in}{0.791228in}}%
\pgfpathlineto{\pgfqpoint{0.589677in}{0.797863in}}%
\pgfpathlineto{\pgfqpoint{0.589922in}{0.798018in}}%
\pgfpathlineto{\pgfqpoint{0.601654in}{0.805221in}}%
\pgfpathlineto{\pgfqpoint{0.608490in}{0.809274in}}%
\pgfpathlineto{\pgfqpoint{0.613386in}{0.812499in}}%
\pgfpathlineto{\pgfqpoint{0.625118in}{0.819803in}}%
\pgfpathlineto{\pgfqpoint{0.626551in}{0.820685in}}%
\pgfpathlineto{\pgfqpoint{0.636851in}{0.827884in}}%
\pgfpathlineto{\pgfqpoint{0.643284in}{0.832096in}}%
\pgfpathlineto{\pgfqpoint{0.648583in}{0.836415in}}%
\pgfpathlineto{\pgfqpoint{0.658266in}{0.843507in}}%
\pgfpathlineto{\pgfqpoint{0.660315in}{0.847139in}}%
\pgfpathlineto{\pgfqpoint{0.666646in}{0.854917in}}%
\pgfpathlineto{\pgfqpoint{0.660315in}{0.860714in}}%
\pgfpathlineto{\pgfqpoint{0.651116in}{0.866328in}}%
\pgfpathlineto{\pgfqpoint{0.648583in}{0.866739in}}%
\pgfpathlineto{\pgfqpoint{0.642403in}{0.866328in}}%
\pgfpathlineto{\pgfqpoint{0.636851in}{0.865690in}}%
\pgfpathlineto{\pgfqpoint{0.625118in}{0.862825in}}%
\pgfpathlineto{\pgfqpoint{0.613386in}{0.860169in}}%
\pgfpathlineto{\pgfqpoint{0.601654in}{0.857449in}}%
\pgfpathlineto{\pgfqpoint{0.591927in}{0.854917in}}%
\pgfpathlineto{\pgfqpoint{0.589922in}{0.854303in}}%
\pgfpathlineto{\pgfqpoint{0.578190in}{0.850782in}}%
\pgfpathlineto{\pgfqpoint{0.566457in}{0.847527in}}%
\pgfpathlineto{\pgfqpoint{0.554725in}{0.844569in}}%
\pgfpathlineto{\pgfqpoint{0.550708in}{0.843507in}}%
\pgfpathlineto{\pgfqpoint{0.542993in}{0.841518in}}%
\pgfpathlineto{\pgfqpoint{0.531261in}{0.838558in}}%
\pgfpathlineto{\pgfqpoint{0.519528in}{0.835415in}}%
\pgfpathlineto{\pgfqpoint{0.508653in}{0.832096in}}%
\pgfpathlineto{\pgfqpoint{0.507796in}{0.831854in}}%
\pgfpathlineto{\pgfqpoint{0.496064in}{0.828629in}}%
\pgfpathlineto{\pgfqpoint{0.484332in}{0.825234in}}%
\pgfpathlineto{\pgfqpoint{0.472600in}{0.821335in}}%
\pgfpathlineto{\pgfqpoint{0.470769in}{0.820685in}}%
\pgfpathlineto{\pgfqpoint{0.460867in}{0.817803in}}%
\pgfpathlineto{\pgfqpoint{0.449135in}{0.814197in}}%
\pgfpathlineto{\pgfqpoint{0.437403in}{0.810015in}}%
\pgfpathlineto{\pgfqpoint{0.435462in}{0.809274in}}%
\pgfpathlineto{\pgfqpoint{0.425671in}{0.806306in}}%
\pgfpathlineto{\pgfqpoint{0.413939in}{0.802496in}}%
\pgfpathlineto{\pgfqpoint{0.402206in}{0.798017in}}%
\pgfpathlineto{\pgfqpoint{0.401823in}{0.797863in}}%
\pgfpathlineto{\pgfqpoint{0.390474in}{0.794328in}}%
\pgfpathlineto{\pgfqpoint{0.378742in}{0.790305in}}%
\pgfpathlineto{\pgfqpoint{0.369036in}{0.786453in}}%
\pgfpathlineto{\pgfqpoint{0.367010in}{0.785795in}}%
\pgfpathlineto{\pgfqpoint{0.355278in}{0.782016in}}%
\pgfpathlineto{\pgfqpoint{0.343545in}{0.777768in}}%
\pgfpathlineto{\pgfqpoint{0.336828in}{0.775042in}}%
\pgfpathlineto{\pgfqpoint{0.331813in}{0.773400in}}%
\pgfpathlineto{\pgfqpoint{0.320081in}{0.769475in}}%
\pgfpathlineto{\pgfqpoint{0.308349in}{0.764992in}}%
\pgfpathlineto{\pgfqpoint{0.305041in}{0.763631in}}%
\pgfpathlineto{\pgfqpoint{0.296617in}{0.760863in}}%
\pgfpathlineto{\pgfqpoint{0.284884in}{0.756779in}}%
\pgfpathlineto{\pgfqpoint{0.284884in}{0.752220in}}%
\pgfpathlineto{\pgfqpoint{0.284884in}{0.745733in}}%
\pgfpathlineto{\pgfqpoint{0.296617in}{0.749791in}}%
\pgfpathlineto{\pgfqpoint{0.303330in}{0.752220in}}%
\pgfpathlineto{\pgfqpoint{0.308349in}{0.754086in}}%
\pgfpathlineto{\pgfqpoint{0.320081in}{0.758183in}}%
\pgfpathlineto{\pgfqpoint{0.331813in}{0.762238in}}%
\pgfpathlineto{\pgfqpoint{0.335651in}{0.763631in}}%
\pgfpathlineto{\pgfqpoint{0.343545in}{0.766501in}}%
\pgfpathlineto{\pgfqpoint{0.355278in}{0.770494in}}%
\pgfpathlineto{\pgfqpoint{0.367010in}{0.774561in}}%
\pgfpathlineto{\pgfqpoint{0.368337in}{0.775042in}}%
\pgfpathlineto{\pgfqpoint{0.378742in}{0.778710in}}%
\pgfpathlineto{\pgfqpoint{0.390474in}{0.782599in}}%
\pgfpathlineto{\pgfqpoint{0.401607in}{0.786453in}}%
\pgfpathlineto{\pgfqpoint{0.402206in}{0.786669in}}%
\pgfpathlineto{\pgfqpoint{0.413939in}{0.790624in}}%
\pgfpathlineto{\pgfqpoint{0.425671in}{0.794388in}}%
\pgfpathlineto{\pgfqpoint{0.435883in}{0.797863in}}%
\pgfpathlineto{\pgfqpoint{0.437403in}{0.798386in}}%
\pgfpathlineto{\pgfqpoint{0.449135in}{0.802107in}}%
\pgfpathlineto{\pgfqpoint{0.460867in}{0.805683in}}%
\pgfpathlineto{\pgfqpoint{0.471756in}{0.809274in}}%
\pgfpathlineto{\pgfqpoint{0.472600in}{0.809545in}}%
\pgfpathlineto{\pgfqpoint{0.484332in}{0.812956in}}%
\pgfpathlineto{\pgfqpoint{0.496064in}{0.816210in}}%
\pgfpathlineto{\pgfqpoint{0.507796in}{0.819732in}}%
\pgfpathlineto{\pgfqpoint{0.510868in}{0.820685in}}%
\pgfpathlineto{\pgfqpoint{0.519528in}{0.822889in}}%
\pgfpathlineto{\pgfqpoint{0.531261in}{0.825557in}}%
\pgfpathlineto{\pgfqpoint{0.542993in}{0.828193in}}%
\pgfpathlineto{\pgfqpoint{0.554725in}{0.831142in}}%
\pgfpathlineto{\pgfqpoint{0.557644in}{0.832096in}}%
\pgfpathlineto{\pgfqpoint{0.566457in}{0.833963in}}%
\pgfpathlineto{\pgfqpoint{0.578190in}{0.836117in}}%
\pgfpathlineto{\pgfqpoint{0.589922in}{0.837755in}}%
\pgfpathlineto{\pgfqpoint{0.601654in}{0.837160in}}%
\pgfpathlineto{\pgfqpoint{0.606428in}{0.832096in}}%
\pgfpathlineto{\pgfqpoint{0.601654in}{0.828675in}}%
\pgfpathlineto{\pgfqpoint{0.594449in}{0.820685in}}%
\pgfpathlineto{\pgfqpoint{0.589922in}{0.817858in}}%
\pgfpathlineto{\pgfqpoint{0.578805in}{0.809274in}}%
\pgfpathlineto{\pgfqpoint{0.578190in}{0.808907in}}%
\pgfpathlineto{\pgfqpoint{0.566457in}{0.802061in}}%
\pgfpathlineto{\pgfqpoint{0.560807in}{0.797863in}}%
\pgfpathlineto{\pgfqpoint{0.554725in}{0.794364in}}%
\pgfpathlineto{\pgfqpoint{0.542993in}{0.787142in}}%
\pgfpathlineto{\pgfqpoint{0.542017in}{0.786453in}}%
\pgfpathlineto{\pgfqpoint{0.531261in}{0.780507in}}%
\pgfpathlineto{\pgfqpoint{0.522468in}{0.775042in}}%
\pgfpathlineto{\pgfqpoint{0.519528in}{0.773418in}}%
\pgfpathlineto{\pgfqpoint{0.507796in}{0.766762in}}%
\pgfpathlineto{\pgfqpoint{0.502740in}{0.763631in}}%
\pgfpathlineto{\pgfqpoint{0.496064in}{0.759927in}}%
\pgfpathlineto{\pgfqpoint{0.484332in}{0.753223in}}%
\pgfpathlineto{\pgfqpoint{0.482677in}{0.752220in}}%
\pgfpathlineto{\pgfqpoint{0.472600in}{0.746749in}}%
\pgfpathlineto{\pgfqpoint{0.462118in}{0.740810in}}%
\pgfpathlineto{\pgfqpoint{0.460867in}{0.740133in}}%
\pgfpathlineto{\pgfqpoint{0.449135in}{0.733861in}}%
\pgfpathlineto{\pgfqpoint{0.441086in}{0.729399in}}%
\pgfpathlineto{\pgfqpoint{0.437403in}{0.727444in}}%
\pgfpathlineto{\pgfqpoint{0.425671in}{0.721227in}}%
\pgfpathlineto{\pgfqpoint{0.419711in}{0.717988in}}%
\pgfpathlineto{\pgfqpoint{0.413939in}{0.714976in}}%
\pgfpathlineto{\pgfqpoint{0.402206in}{0.708815in}}%
\pgfpathlineto{\pgfqpoint{0.398020in}{0.706577in}}%
\pgfpathlineto{\pgfqpoint{0.390474in}{0.702700in}}%
\pgfpathlineto{\pgfqpoint{0.378742in}{0.696592in}}%
\pgfpathlineto{\pgfqpoint{0.376047in}{0.695166in}}%
\pgfpathlineto{\pgfqpoint{0.367010in}{0.690583in}}%
\pgfpathlineto{\pgfqpoint{0.355278in}{0.684517in}}%
\pgfpathlineto{\pgfqpoint{0.353834in}{0.683756in}}%
\pgfpathlineto{\pgfqpoint{0.343545in}{0.678592in}}%
\pgfpathlineto{\pgfqpoint{0.331813in}{0.672550in}}%
\pgfpathlineto{\pgfqpoint{0.331424in}{0.672345in}}%
\pgfpathlineto{\pgfqpoint{0.320081in}{0.666700in}}%
\pgfpathlineto{\pgfqpoint{0.308837in}{0.660934in}}%
\pgfpathlineto{\pgfqpoint{0.308349in}{0.660689in}}%
\pgfpathlineto{\pgfqpoint{0.296617in}{0.654882in}}%
\pgfpathlineto{\pgfqpoint{0.286122in}{0.649523in}}%
\pgfpathlineto{\pgfqpoint{0.284884in}{0.648906in}}%
\pgfpathlineto{\pgfqpoint{0.284884in}{0.638113in}}%
\pgfpathlineto{\pgfqpoint{0.284884in}{0.633007in}}%
\pgfpathclose%
\pgfusepath{fill}%
\end{pgfscope}%
\begin{pgfscope}%
\pgfpathrectangle{\pgfqpoint{0.211875in}{0.211875in}}{\pgfqpoint{1.313625in}{1.279725in}}%
\pgfusepath{clip}%
\pgfsetbuttcap%
\pgfsetroundjoin%
\definecolor{currentfill}{rgb}{0.445089,0.121699,0.342662}%
\pgfsetfillcolor{currentfill}%
\pgfsetlinewidth{0.000000pt}%
\definecolor{currentstroke}{rgb}{0.000000,0.000000,0.000000}%
\pgfsetstrokecolor{currentstroke}%
\pgfsetdash{}{0pt}%
\pgfpathmoveto{\pgfqpoint{0.824566in}{0.934415in}}%
\pgfpathlineto{\pgfqpoint{0.826378in}{0.934793in}}%
\pgfpathlineto{\pgfqpoint{0.836256in}{0.946204in}}%
\pgfpathlineto{\pgfqpoint{0.835885in}{0.957614in}}%
\pgfpathlineto{\pgfqpoint{0.833922in}{0.969025in}}%
\pgfpathlineto{\pgfqpoint{0.831381in}{0.980436in}}%
\pgfpathlineto{\pgfqpoint{0.826289in}{0.991847in}}%
\pgfpathlineto{\pgfqpoint{0.824566in}{0.994135in}}%
\pgfpathlineto{\pgfqpoint{0.821275in}{1.003257in}}%
\pgfpathlineto{\pgfqpoint{0.818597in}{1.014668in}}%
\pgfpathlineto{\pgfqpoint{0.813011in}{1.026079in}}%
\pgfpathlineto{\pgfqpoint{0.812834in}{1.026240in}}%
\pgfpathlineto{\pgfqpoint{0.801101in}{1.027108in}}%
\pgfpathlineto{\pgfqpoint{0.789369in}{1.029587in}}%
\pgfpathlineto{\pgfqpoint{0.777637in}{1.036956in}}%
\pgfpathlineto{\pgfqpoint{0.776971in}{1.037490in}}%
\pgfpathlineto{\pgfqpoint{0.765905in}{1.048147in}}%
\pgfpathlineto{\pgfqpoint{0.765061in}{1.048901in}}%
\pgfpathlineto{\pgfqpoint{0.754173in}{1.059042in}}%
\pgfpathlineto{\pgfqpoint{0.752493in}{1.060311in}}%
\pgfpathlineto{\pgfqpoint{0.742440in}{1.068155in}}%
\pgfpathlineto{\pgfqpoint{0.736683in}{1.071722in}}%
\pgfpathlineto{\pgfqpoint{0.730708in}{1.075740in}}%
\pgfpathlineto{\pgfqpoint{0.718976in}{1.081199in}}%
\pgfpathlineto{\pgfqpoint{0.712874in}{1.083133in}}%
\pgfpathlineto{\pgfqpoint{0.707244in}{1.085217in}}%
\pgfpathlineto{\pgfqpoint{0.695512in}{1.092634in}}%
\pgfpathlineto{\pgfqpoint{0.692332in}{1.094544in}}%
\pgfpathlineto{\pgfqpoint{0.683779in}{1.100255in}}%
\pgfpathlineto{\pgfqpoint{0.675245in}{1.105954in}}%
\pgfpathlineto{\pgfqpoint{0.672047in}{1.108089in}}%
\pgfpathlineto{\pgfqpoint{0.660315in}{1.115874in}}%
\pgfpathlineto{\pgfqpoint{0.658623in}{1.117365in}}%
\pgfpathlineto{\pgfqpoint{0.648583in}{1.125863in}}%
\pgfpathlineto{\pgfqpoint{0.647069in}{1.128776in}}%
\pgfpathlineto{\pgfqpoint{0.641250in}{1.140187in}}%
\pgfpathlineto{\pgfqpoint{0.636851in}{1.148812in}}%
\pgfpathlineto{\pgfqpoint{0.635301in}{1.151598in}}%
\pgfpathlineto{\pgfqpoint{0.628997in}{1.163008in}}%
\pgfpathlineto{\pgfqpoint{0.625118in}{1.170406in}}%
\pgfpathlineto{\pgfqpoint{0.622764in}{1.174419in}}%
\pgfpathlineto{\pgfqpoint{0.616634in}{1.185830in}}%
\pgfpathlineto{\pgfqpoint{0.613386in}{1.192625in}}%
\pgfpathlineto{\pgfqpoint{0.610876in}{1.197241in}}%
\pgfpathlineto{\pgfqpoint{0.606137in}{1.208651in}}%
\pgfpathlineto{\pgfqpoint{0.602928in}{1.220062in}}%
\pgfpathlineto{\pgfqpoint{0.601654in}{1.226053in}}%
\pgfpathlineto{\pgfqpoint{0.600341in}{1.231473in}}%
\pgfpathlineto{\pgfqpoint{0.598830in}{1.242884in}}%
\pgfpathlineto{\pgfqpoint{0.600149in}{1.254295in}}%
\pgfpathlineto{\pgfqpoint{0.601654in}{1.259462in}}%
\pgfpathlineto{\pgfqpoint{0.603562in}{1.265705in}}%
\pgfpathlineto{\pgfqpoint{0.608803in}{1.277116in}}%
\pgfpathlineto{\pgfqpoint{0.613386in}{1.282232in}}%
\pgfpathlineto{\pgfqpoint{0.619048in}{1.288527in}}%
\pgfpathlineto{\pgfqpoint{0.618936in}{1.299938in}}%
\pgfpathlineto{\pgfqpoint{0.617050in}{1.311348in}}%
\pgfpathlineto{\pgfqpoint{0.613583in}{1.322759in}}%
\pgfpathlineto{\pgfqpoint{0.613386in}{1.323470in}}%
\pgfpathlineto{\pgfqpoint{0.610179in}{1.334170in}}%
\pgfpathlineto{\pgfqpoint{0.607475in}{1.345581in}}%
\pgfpathlineto{\pgfqpoint{0.606510in}{1.356992in}}%
\pgfpathlineto{\pgfqpoint{0.602870in}{1.368402in}}%
\pgfpathlineto{\pgfqpoint{0.601654in}{1.369391in}}%
\pgfpathlineto{\pgfqpoint{0.594836in}{1.379813in}}%
\pgfpathlineto{\pgfqpoint{0.589922in}{1.383105in}}%
\pgfpathlineto{\pgfqpoint{0.582639in}{1.391224in}}%
\pgfpathlineto{\pgfqpoint{0.578190in}{1.394508in}}%
\pgfpathlineto{\pgfqpoint{0.568609in}{1.402635in}}%
\pgfpathlineto{\pgfqpoint{0.566457in}{1.404234in}}%
\pgfpathlineto{\pgfqpoint{0.554725in}{1.413001in}}%
\pgfpathlineto{\pgfqpoint{0.553643in}{1.414045in}}%
\pgfpathlineto{\pgfqpoint{0.542993in}{1.414045in}}%
\pgfpathlineto{\pgfqpoint{0.533720in}{1.414045in}}%
\pgfpathlineto{\pgfqpoint{0.542993in}{1.406939in}}%
\pgfpathlineto{\pgfqpoint{0.547477in}{1.402635in}}%
\pgfpathlineto{\pgfqpoint{0.554725in}{1.395163in}}%
\pgfpathlineto{\pgfqpoint{0.558484in}{1.391224in}}%
\pgfpathlineto{\pgfqpoint{0.566457in}{1.382111in}}%
\pgfpathlineto{\pgfqpoint{0.568427in}{1.379813in}}%
\pgfpathlineto{\pgfqpoint{0.576926in}{1.368402in}}%
\pgfpathlineto{\pgfqpoint{0.578190in}{1.365423in}}%
\pgfpathlineto{\pgfqpoint{0.581348in}{1.356992in}}%
\pgfpathlineto{\pgfqpoint{0.582220in}{1.345581in}}%
\pgfpathlineto{\pgfqpoint{0.578190in}{1.337580in}}%
\pgfpathlineto{\pgfqpoint{0.566457in}{1.334326in}}%
\pgfpathlineto{\pgfqpoint{0.565493in}{1.334170in}}%
\pgfpathlineto{\pgfqpoint{0.554725in}{1.332364in}}%
\pgfpathlineto{\pgfqpoint{0.542993in}{1.329266in}}%
\pgfpathlineto{\pgfqpoint{0.533548in}{1.322759in}}%
\pgfpathlineto{\pgfqpoint{0.531261in}{1.319353in}}%
\pgfpathlineto{\pgfqpoint{0.527369in}{1.311348in}}%
\pgfpathlineto{\pgfqpoint{0.526005in}{1.299938in}}%
\pgfpathlineto{\pgfqpoint{0.527399in}{1.288527in}}%
\pgfpathlineto{\pgfqpoint{0.529326in}{1.277116in}}%
\pgfpathlineto{\pgfqpoint{0.531261in}{1.270275in}}%
\pgfpathlineto{\pgfqpoint{0.532387in}{1.265705in}}%
\pgfpathlineto{\pgfqpoint{0.536002in}{1.254295in}}%
\pgfpathlineto{\pgfqpoint{0.540651in}{1.242884in}}%
\pgfpathlineto{\pgfqpoint{0.542993in}{1.238464in}}%
\pgfpathlineto{\pgfqpoint{0.546120in}{1.231473in}}%
\pgfpathlineto{\pgfqpoint{0.552668in}{1.220062in}}%
\pgfpathlineto{\pgfqpoint{0.554725in}{1.217674in}}%
\pgfpathlineto{\pgfqpoint{0.561132in}{1.208651in}}%
\pgfpathlineto{\pgfqpoint{0.566457in}{1.202726in}}%
\pgfpathlineto{\pgfqpoint{0.570713in}{1.197241in}}%
\pgfpathlineto{\pgfqpoint{0.578190in}{1.188603in}}%
\pgfpathlineto{\pgfqpoint{0.580279in}{1.185830in}}%
\pgfpathlineto{\pgfqpoint{0.589650in}{1.174419in}}%
\pgfpathlineto{\pgfqpoint{0.589922in}{1.174106in}}%
\pgfpathlineto{\pgfqpoint{0.598504in}{1.163008in}}%
\pgfpathlineto{\pgfqpoint{0.601654in}{1.158947in}}%
\pgfpathlineto{\pgfqpoint{0.606796in}{1.151598in}}%
\pgfpathlineto{\pgfqpoint{0.613386in}{1.142084in}}%
\pgfpathlineto{\pgfqpoint{0.614566in}{1.140187in}}%
\pgfpathlineto{\pgfqpoint{0.614357in}{1.128776in}}%
\pgfpathlineto{\pgfqpoint{0.613386in}{1.128427in}}%
\pgfpathlineto{\pgfqpoint{0.611672in}{1.128776in}}%
\pgfpathlineto{\pgfqpoint{0.601654in}{1.131331in}}%
\pgfpathlineto{\pgfqpoint{0.589922in}{1.136033in}}%
\pgfpathlineto{\pgfqpoint{0.580369in}{1.140187in}}%
\pgfpathlineto{\pgfqpoint{0.578190in}{1.141317in}}%
\pgfpathlineto{\pgfqpoint{0.566457in}{1.148221in}}%
\pgfpathlineto{\pgfqpoint{0.560724in}{1.151598in}}%
\pgfpathlineto{\pgfqpoint{0.554725in}{1.155727in}}%
\pgfpathlineto{\pgfqpoint{0.542993in}{1.162990in}}%
\pgfpathlineto{\pgfqpoint{0.542966in}{1.163008in}}%
\pgfpathlineto{\pgfqpoint{0.531261in}{1.172658in}}%
\pgfpathlineto{\pgfqpoint{0.528902in}{1.174419in}}%
\pgfpathlineto{\pgfqpoint{0.519528in}{1.183201in}}%
\pgfpathlineto{\pgfqpoint{0.516654in}{1.185830in}}%
\pgfpathlineto{\pgfqpoint{0.507796in}{1.194715in}}%
\pgfpathlineto{\pgfqpoint{0.505330in}{1.197241in}}%
\pgfpathlineto{\pgfqpoint{0.496064in}{1.206877in}}%
\pgfpathlineto{\pgfqpoint{0.494399in}{1.208651in}}%
\pgfpathlineto{\pgfqpoint{0.484332in}{1.219273in}}%
\pgfpathlineto{\pgfqpoint{0.483603in}{1.220062in}}%
\pgfpathlineto{\pgfqpoint{0.472642in}{1.231473in}}%
\pgfpathlineto{\pgfqpoint{0.472600in}{1.231513in}}%
\pgfpathlineto{\pgfqpoint{0.461027in}{1.242884in}}%
\pgfpathlineto{\pgfqpoint{0.460867in}{1.243037in}}%
\pgfpathlineto{\pgfqpoint{0.449362in}{1.254295in}}%
\pgfpathlineto{\pgfqpoint{0.449135in}{1.254512in}}%
\pgfpathlineto{\pgfqpoint{0.437601in}{1.265705in}}%
\pgfpathlineto{\pgfqpoint{0.437403in}{1.265896in}}%
\pgfpathlineto{\pgfqpoint{0.425712in}{1.277116in}}%
\pgfpathlineto{\pgfqpoint{0.425671in}{1.277158in}}%
\pgfpathlineto{\pgfqpoint{0.414019in}{1.288527in}}%
\pgfpathlineto{\pgfqpoint{0.413939in}{1.288615in}}%
\pgfpathlineto{\pgfqpoint{0.403186in}{1.299938in}}%
\pgfpathlineto{\pgfqpoint{0.402206in}{1.301001in}}%
\pgfpathlineto{\pgfqpoint{0.391525in}{1.311348in}}%
\pgfpathlineto{\pgfqpoint{0.390474in}{1.312470in}}%
\pgfpathlineto{\pgfqpoint{0.379209in}{1.322759in}}%
\pgfpathlineto{\pgfqpoint{0.378742in}{1.323219in}}%
\pgfpathlineto{\pgfqpoint{0.367010in}{1.333800in}}%
\pgfpathlineto{\pgfqpoint{0.366605in}{1.334170in}}%
\pgfpathlineto{\pgfqpoint{0.355278in}{1.344398in}}%
\pgfpathlineto{\pgfqpoint{0.353961in}{1.345581in}}%
\pgfpathlineto{\pgfqpoint{0.343545in}{1.354845in}}%
\pgfpathlineto{\pgfqpoint{0.341112in}{1.356992in}}%
\pgfpathlineto{\pgfqpoint{0.331813in}{1.365387in}}%
\pgfpathlineto{\pgfqpoint{0.328344in}{1.368402in}}%
\pgfpathlineto{\pgfqpoint{0.320081in}{1.375815in}}%
\pgfpathlineto{\pgfqpoint{0.315631in}{1.379813in}}%
\pgfpathlineto{\pgfqpoint{0.308349in}{1.386403in}}%
\pgfpathlineto{\pgfqpoint{0.302890in}{1.391224in}}%
\pgfpathlineto{\pgfqpoint{0.296617in}{1.397069in}}%
\pgfpathlineto{\pgfqpoint{0.290171in}{1.402635in}}%
\pgfpathlineto{\pgfqpoint{0.284884in}{1.407488in}}%
\pgfpathlineto{\pgfqpoint{0.284884in}{1.402635in}}%
\pgfpathlineto{\pgfqpoint{0.284884in}{1.391224in}}%
\pgfpathlineto{\pgfqpoint{0.284884in}{1.383087in}}%
\pgfpathlineto{\pgfqpoint{0.288515in}{1.379813in}}%
\pgfpathlineto{\pgfqpoint{0.296617in}{1.372814in}}%
\pgfpathlineto{\pgfqpoint{0.301570in}{1.368402in}}%
\pgfpathlineto{\pgfqpoint{0.308349in}{1.362434in}}%
\pgfpathlineto{\pgfqpoint{0.314509in}{1.356992in}}%
\pgfpathlineto{\pgfqpoint{0.320081in}{1.352000in}}%
\pgfpathlineto{\pgfqpoint{0.327242in}{1.345581in}}%
\pgfpathlineto{\pgfqpoint{0.331813in}{1.341454in}}%
\pgfpathlineto{\pgfqpoint{0.339807in}{1.334170in}}%
\pgfpathlineto{\pgfqpoint{0.343545in}{1.330783in}}%
\pgfpathlineto{\pgfqpoint{0.352193in}{1.322759in}}%
\pgfpathlineto{\pgfqpoint{0.355278in}{1.320024in}}%
\pgfpathlineto{\pgfqpoint{0.364597in}{1.311348in}}%
\pgfpathlineto{\pgfqpoint{0.367010in}{1.309231in}}%
\pgfpathlineto{\pgfqpoint{0.377365in}{1.299938in}}%
\pgfpathlineto{\pgfqpoint{0.378742in}{1.298700in}}%
\pgfpathlineto{\pgfqpoint{0.390200in}{1.288527in}}%
\pgfpathlineto{\pgfqpoint{0.390474in}{1.288275in}}%
\pgfpathlineto{\pgfqpoint{0.402206in}{1.277728in}}%
\pgfpathlineto{\pgfqpoint{0.402856in}{1.277116in}}%
\pgfpathlineto{\pgfqpoint{0.413939in}{1.266933in}}%
\pgfpathlineto{\pgfqpoint{0.415198in}{1.265705in}}%
\pgfpathlineto{\pgfqpoint{0.425671in}{1.255146in}}%
\pgfpathlineto{\pgfqpoint{0.426536in}{1.254295in}}%
\pgfpathlineto{\pgfqpoint{0.437403in}{1.243216in}}%
\pgfpathlineto{\pgfqpoint{0.437738in}{1.242884in}}%
\pgfpathlineto{\pgfqpoint{0.448634in}{1.231473in}}%
\pgfpathlineto{\pgfqpoint{0.449135in}{1.230879in}}%
\pgfpathlineto{\pgfqpoint{0.458784in}{1.220062in}}%
\pgfpathlineto{\pgfqpoint{0.460867in}{1.217592in}}%
\pgfpathlineto{\pgfqpoint{0.468815in}{1.208651in}}%
\pgfpathlineto{\pgfqpoint{0.472600in}{1.204289in}}%
\pgfpathlineto{\pgfqpoint{0.478977in}{1.197241in}}%
\pgfpathlineto{\pgfqpoint{0.484332in}{1.191610in}}%
\pgfpathlineto{\pgfqpoint{0.489972in}{1.185830in}}%
\pgfpathlineto{\pgfqpoint{0.496064in}{1.180494in}}%
\pgfpathlineto{\pgfqpoint{0.502848in}{1.174419in}}%
\pgfpathlineto{\pgfqpoint{0.507796in}{1.170412in}}%
\pgfpathlineto{\pgfqpoint{0.517879in}{1.163008in}}%
\pgfpathlineto{\pgfqpoint{0.519528in}{1.161759in}}%
\pgfpathlineto{\pgfqpoint{0.531261in}{1.153420in}}%
\pgfpathlineto{\pgfqpoint{0.533892in}{1.151598in}}%
\pgfpathlineto{\pgfqpoint{0.542993in}{1.145345in}}%
\pgfpathlineto{\pgfqpoint{0.551117in}{1.140187in}}%
\pgfpathlineto{\pgfqpoint{0.554725in}{1.137890in}}%
\pgfpathlineto{\pgfqpoint{0.566457in}{1.130932in}}%
\pgfpathlineto{\pgfqpoint{0.570100in}{1.128776in}}%
\pgfpathlineto{\pgfqpoint{0.578190in}{1.124151in}}%
\pgfpathlineto{\pgfqpoint{0.589922in}{1.117860in}}%
\pgfpathlineto{\pgfqpoint{0.590862in}{1.117365in}}%
\pgfpathlineto{\pgfqpoint{0.601654in}{1.112127in}}%
\pgfpathlineto{\pgfqpoint{0.613386in}{1.106503in}}%
\pgfpathlineto{\pgfqpoint{0.614679in}{1.105954in}}%
\pgfpathlineto{\pgfqpoint{0.625118in}{1.101997in}}%
\pgfpathlineto{\pgfqpoint{0.636851in}{1.097943in}}%
\pgfpathlineto{\pgfqpoint{0.648212in}{1.094544in}}%
\pgfpathlineto{\pgfqpoint{0.648583in}{1.094448in}}%
\pgfpathlineto{\pgfqpoint{0.660315in}{1.087472in}}%
\pgfpathlineto{\pgfqpoint{0.665000in}{1.083133in}}%
\pgfpathlineto{\pgfqpoint{0.672047in}{1.077695in}}%
\pgfpathlineto{\pgfqpoint{0.678656in}{1.071722in}}%
\pgfpathlineto{\pgfqpoint{0.683779in}{1.067782in}}%
\pgfpathlineto{\pgfqpoint{0.692366in}{1.060311in}}%
\pgfpathlineto{\pgfqpoint{0.695512in}{1.057589in}}%
\pgfpathlineto{\pgfqpoint{0.702892in}{1.048901in}}%
\pgfpathlineto{\pgfqpoint{0.707244in}{1.044037in}}%
\pgfpathlineto{\pgfqpoint{0.712861in}{1.037490in}}%
\pgfpathlineto{\pgfqpoint{0.718976in}{1.030747in}}%
\pgfpathlineto{\pgfqpoint{0.723165in}{1.026079in}}%
\pgfpathlineto{\pgfqpoint{0.730708in}{1.017257in}}%
\pgfpathlineto{\pgfqpoint{0.732797in}{1.014668in}}%
\pgfpathlineto{\pgfqpoint{0.740437in}{1.003257in}}%
\pgfpathlineto{\pgfqpoint{0.742440in}{1.000405in}}%
\pgfpathlineto{\pgfqpoint{0.748766in}{0.991847in}}%
\pgfpathlineto{\pgfqpoint{0.754173in}{0.984805in}}%
\pgfpathlineto{\pgfqpoint{0.757665in}{0.980436in}}%
\pgfpathlineto{\pgfqpoint{0.765905in}{0.970573in}}%
\pgfpathlineto{\pgfqpoint{0.767174in}{0.969025in}}%
\pgfpathlineto{\pgfqpoint{0.777089in}{0.957614in}}%
\pgfpathlineto{\pgfqpoint{0.777637in}{0.957018in}}%
\pgfpathlineto{\pgfqpoint{0.786291in}{0.946204in}}%
\pgfpathlineto{\pgfqpoint{0.789369in}{0.942826in}}%
\pgfpathlineto{\pgfqpoint{0.801101in}{0.937498in}}%
\pgfpathlineto{\pgfqpoint{0.812834in}{0.935087in}}%
\pgfpathlineto{\pgfqpoint{0.818057in}{0.934793in}}%
\pgfpathclose%
\pgfpathmoveto{\pgfqpoint{0.802889in}{0.946204in}}%
\pgfpathlineto{\pgfqpoint{0.801101in}{0.946849in}}%
\pgfpathlineto{\pgfqpoint{0.789369in}{0.955775in}}%
\pgfpathlineto{\pgfqpoint{0.787837in}{0.957614in}}%
\pgfpathlineto{\pgfqpoint{0.779333in}{0.969025in}}%
\pgfpathlineto{\pgfqpoint{0.777637in}{0.971653in}}%
\pgfpathlineto{\pgfqpoint{0.770553in}{0.980436in}}%
\pgfpathlineto{\pgfqpoint{0.765905in}{0.986725in}}%
\pgfpathlineto{\pgfqpoint{0.761928in}{0.991847in}}%
\pgfpathlineto{\pgfqpoint{0.754173in}{1.002440in}}%
\pgfpathlineto{\pgfqpoint{0.753594in}{1.003257in}}%
\pgfpathlineto{\pgfqpoint{0.746731in}{1.014668in}}%
\pgfpathlineto{\pgfqpoint{0.742440in}{1.021014in}}%
\pgfpathlineto{\pgfqpoint{0.738795in}{1.026079in}}%
\pgfpathlineto{\pgfqpoint{0.742440in}{1.035704in}}%
\pgfpathlineto{\pgfqpoint{0.754173in}{1.028312in}}%
\pgfpathlineto{\pgfqpoint{0.756993in}{1.026079in}}%
\pgfpathlineto{\pgfqpoint{0.765905in}{1.021910in}}%
\pgfpathlineto{\pgfqpoint{0.777637in}{1.017204in}}%
\pgfpathlineto{\pgfqpoint{0.789369in}{1.014689in}}%
\pgfpathlineto{\pgfqpoint{0.801101in}{1.015344in}}%
\pgfpathlineto{\pgfqpoint{0.804005in}{1.014668in}}%
\pgfpathlineto{\pgfqpoint{0.812834in}{1.011093in}}%
\pgfpathlineto{\pgfqpoint{0.815161in}{1.003257in}}%
\pgfpathlineto{\pgfqpoint{0.818601in}{0.991847in}}%
\pgfpathlineto{\pgfqpoint{0.822276in}{0.980436in}}%
\pgfpathlineto{\pgfqpoint{0.824566in}{0.973233in}}%
\pgfpathlineto{\pgfqpoint{0.825922in}{0.969025in}}%
\pgfpathlineto{\pgfqpoint{0.828423in}{0.957614in}}%
\pgfpathlineto{\pgfqpoint{0.828276in}{0.946204in}}%
\pgfpathlineto{\pgfqpoint{0.824566in}{0.942466in}}%
\pgfpathlineto{\pgfqpoint{0.812834in}{0.943211in}}%
\pgfpathclose%
\pgfpathmoveto{\pgfqpoint{0.729406in}{1.037490in}}%
\pgfpathlineto{\pgfqpoint{0.730708in}{1.040529in}}%
\pgfpathlineto{\pgfqpoint{0.738331in}{1.037490in}}%
\pgfpathlineto{\pgfqpoint{0.730708in}{1.035957in}}%
\pgfpathclose%
\pgfusepath{fill}%
\end{pgfscope}%
\begin{pgfscope}%
\pgfpathrectangle{\pgfqpoint{0.211875in}{0.211875in}}{\pgfqpoint{1.313625in}{1.279725in}}%
\pgfusepath{clip}%
\pgfsetbuttcap%
\pgfsetroundjoin%
\definecolor{currentfill}{rgb}{0.445089,0.121699,0.342662}%
\pgfsetfillcolor{currentfill}%
\pgfsetlinewidth{0.000000pt}%
\definecolor{currentstroke}{rgb}{0.000000,0.000000,0.000000}%
\pgfsetstrokecolor{currentstroke}%
\pgfsetdash{}{0pt}%
\pgfpathmoveto{\pgfqpoint{1.446373in}{1.001071in}}%
\pgfpathlineto{\pgfqpoint{1.446373in}{1.003257in}}%
\pgfpathlineto{\pgfqpoint{1.446373in}{1.012295in}}%
\pgfpathlineto{\pgfqpoint{1.443814in}{1.014668in}}%
\pgfpathlineto{\pgfqpoint{1.435877in}{1.026079in}}%
\pgfpathlineto{\pgfqpoint{1.434641in}{1.029093in}}%
\pgfpathlineto{\pgfqpoint{1.432503in}{1.037490in}}%
\pgfpathlineto{\pgfqpoint{1.434641in}{1.042739in}}%
\pgfpathlineto{\pgfqpoint{1.442075in}{1.048901in}}%
\pgfpathlineto{\pgfqpoint{1.446373in}{1.051629in}}%
\pgfpathlineto{\pgfqpoint{1.446373in}{1.060311in}}%
\pgfpathlineto{\pgfqpoint{1.446373in}{1.062734in}}%
\pgfpathlineto{\pgfqpoint{1.443075in}{1.060311in}}%
\pgfpathlineto{\pgfqpoint{1.434641in}{1.054060in}}%
\pgfpathlineto{\pgfqpoint{1.429212in}{1.048901in}}%
\pgfpathlineto{\pgfqpoint{1.425988in}{1.037490in}}%
\pgfpathlineto{\pgfqpoint{1.427630in}{1.026079in}}%
\pgfpathlineto{\pgfqpoint{1.432192in}{1.014668in}}%
\pgfpathlineto{\pgfqpoint{1.434641in}{1.011206in}}%
\pgfpathlineto{\pgfqpoint{1.443294in}{1.003257in}}%
\pgfpathclose%
\pgfusepath{fill}%
\end{pgfscope}%
\begin{pgfscope}%
\pgfpathrectangle{\pgfqpoint{0.211875in}{0.211875in}}{\pgfqpoint{1.313625in}{1.279725in}}%
\pgfusepath{clip}%
\pgfsetbuttcap%
\pgfsetroundjoin%
\definecolor{currentfill}{rgb}{0.445089,0.121699,0.342662}%
\pgfsetfillcolor{currentfill}%
\pgfsetlinewidth{0.000000pt}%
\definecolor{currentstroke}{rgb}{0.000000,0.000000,0.000000}%
\pgfsetstrokecolor{currentstroke}%
\pgfsetdash{}{0pt}%
\pgfpathmoveto{\pgfqpoint{1.258658in}{1.025231in}}%
\pgfpathlineto{\pgfqpoint{1.270390in}{1.022521in}}%
\pgfpathlineto{\pgfqpoint{1.282122in}{1.021285in}}%
\pgfpathlineto{\pgfqpoint{1.289870in}{1.026079in}}%
\pgfpathlineto{\pgfqpoint{1.293854in}{1.031115in}}%
\pgfpathlineto{\pgfqpoint{1.299887in}{1.037490in}}%
\pgfpathlineto{\pgfqpoint{1.305586in}{1.039226in}}%
\pgfpathlineto{\pgfqpoint{1.313006in}{1.048901in}}%
\pgfpathlineto{\pgfqpoint{1.305586in}{1.059052in}}%
\pgfpathlineto{\pgfqpoint{1.304837in}{1.060311in}}%
\pgfpathlineto{\pgfqpoint{1.293854in}{1.070762in}}%
\pgfpathlineto{\pgfqpoint{1.293077in}{1.071722in}}%
\pgfpathlineto{\pgfqpoint{1.282122in}{1.081952in}}%
\pgfpathlineto{\pgfqpoint{1.280306in}{1.083133in}}%
\pgfpathlineto{\pgfqpoint{1.270390in}{1.087335in}}%
\pgfpathlineto{\pgfqpoint{1.258658in}{1.092378in}}%
\pgfpathlineto{\pgfqpoint{1.253638in}{1.094544in}}%
\pgfpathlineto{\pgfqpoint{1.246925in}{1.097052in}}%
\pgfpathlineto{\pgfqpoint{1.235193in}{1.101307in}}%
\pgfpathlineto{\pgfqpoint{1.223461in}{1.104916in}}%
\pgfpathlineto{\pgfqpoint{1.218931in}{1.105954in}}%
\pgfpathlineto{\pgfqpoint{1.211729in}{1.107239in}}%
\pgfpathlineto{\pgfqpoint{1.199996in}{1.108227in}}%
\pgfpathlineto{\pgfqpoint{1.188264in}{1.108443in}}%
\pgfpathlineto{\pgfqpoint{1.178411in}{1.105954in}}%
\pgfpathlineto{\pgfqpoint{1.176532in}{1.105271in}}%
\pgfpathlineto{\pgfqpoint{1.164800in}{1.097825in}}%
\pgfpathlineto{\pgfqpoint{1.161380in}{1.094544in}}%
\pgfpathlineto{\pgfqpoint{1.159648in}{1.083133in}}%
\pgfpathlineto{\pgfqpoint{1.164800in}{1.077861in}}%
\pgfpathlineto{\pgfqpoint{1.170772in}{1.071722in}}%
\pgfpathlineto{\pgfqpoint{1.176532in}{1.067703in}}%
\pgfpathlineto{\pgfqpoint{1.188264in}{1.062059in}}%
\pgfpathlineto{\pgfqpoint{1.191569in}{1.060311in}}%
\pgfpathlineto{\pgfqpoint{1.199996in}{1.054444in}}%
\pgfpathlineto{\pgfqpoint{1.211729in}{1.049452in}}%
\pgfpathlineto{\pgfqpoint{1.213126in}{1.048901in}}%
\pgfpathlineto{\pgfqpoint{1.223461in}{1.041182in}}%
\pgfpathlineto{\pgfqpoint{1.230075in}{1.037490in}}%
\pgfpathlineto{\pgfqpoint{1.235193in}{1.034833in}}%
\pgfpathlineto{\pgfqpoint{1.246925in}{1.029971in}}%
\pgfpathlineto{\pgfqpoint{1.256617in}{1.026079in}}%
\pgfpathclose%
\pgfpathmoveto{\pgfqpoint{1.225733in}{1.048901in}}%
\pgfpathlineto{\pgfqpoint{1.223461in}{1.049977in}}%
\pgfpathlineto{\pgfqpoint{1.211729in}{1.054494in}}%
\pgfpathlineto{\pgfqpoint{1.199996in}{1.060194in}}%
\pgfpathlineto{\pgfqpoint{1.199828in}{1.060311in}}%
\pgfpathlineto{\pgfqpoint{1.188264in}{1.066428in}}%
\pgfpathlineto{\pgfqpoint{1.177749in}{1.071722in}}%
\pgfpathlineto{\pgfqpoint{1.176532in}{1.073033in}}%
\pgfpathlineto{\pgfqpoint{1.166891in}{1.083133in}}%
\pgfpathlineto{\pgfqpoint{1.169924in}{1.094544in}}%
\pgfpathlineto{\pgfqpoint{1.176532in}{1.098779in}}%
\pgfpathlineto{\pgfqpoint{1.188264in}{1.102660in}}%
\pgfpathlineto{\pgfqpoint{1.199996in}{1.102536in}}%
\pgfpathlineto{\pgfqpoint{1.211729in}{1.101629in}}%
\pgfpathlineto{\pgfqpoint{1.223461in}{1.099109in}}%
\pgfpathlineto{\pgfqpoint{1.235193in}{1.095909in}}%
\pgfpathlineto{\pgfqpoint{1.239185in}{1.094544in}}%
\pgfpathlineto{\pgfqpoint{1.246925in}{1.091340in}}%
\pgfpathlineto{\pgfqpoint{1.258658in}{1.086696in}}%
\pgfpathlineto{\pgfqpoint{1.268058in}{1.083133in}}%
\pgfpathlineto{\pgfqpoint{1.270390in}{1.082054in}}%
\pgfpathlineto{\pgfqpoint{1.282122in}{1.074316in}}%
\pgfpathlineto{\pgfqpoint{1.284900in}{1.071722in}}%
\pgfpathlineto{\pgfqpoint{1.293854in}{1.060647in}}%
\pgfpathlineto{\pgfqpoint{1.294207in}{1.060311in}}%
\pgfpathlineto{\pgfqpoint{1.297808in}{1.048901in}}%
\pgfpathlineto{\pgfqpoint{1.293854in}{1.046648in}}%
\pgfpathlineto{\pgfqpoint{1.282122in}{1.042247in}}%
\pgfpathlineto{\pgfqpoint{1.270390in}{1.039978in}}%
\pgfpathlineto{\pgfqpoint{1.258658in}{1.038951in}}%
\pgfpathlineto{\pgfqpoint{1.246925in}{1.038851in}}%
\pgfpathlineto{\pgfqpoint{1.235193in}{1.042828in}}%
\pgfpathclose%
\pgfusepath{fill}%
\end{pgfscope}%
\begin{pgfscope}%
\pgfpathrectangle{\pgfqpoint{0.211875in}{0.211875in}}{\pgfqpoint{1.313625in}{1.279725in}}%
\pgfusepath{clip}%
\pgfsetbuttcap%
\pgfsetroundjoin%
\definecolor{currentfill}{rgb}{0.445089,0.121699,0.342662}%
\pgfsetfillcolor{currentfill}%
\pgfsetlinewidth{0.000000pt}%
\definecolor{currentstroke}{rgb}{0.000000,0.000000,0.000000}%
\pgfsetstrokecolor{currentstroke}%
\pgfsetdash{}{0pt}%
\pgfpathmoveto{\pgfqpoint{1.000549in}{1.105534in}}%
\pgfpathlineto{\pgfqpoint{1.012281in}{1.105874in}}%
\pgfpathlineto{\pgfqpoint{1.013197in}{1.105954in}}%
\pgfpathlineto{\pgfqpoint{1.020510in}{1.117365in}}%
\pgfpathlineto{\pgfqpoint{1.012281in}{1.120794in}}%
\pgfpathlineto{\pgfqpoint{1.000549in}{1.123845in}}%
\pgfpathlineto{\pgfqpoint{0.988817in}{1.123577in}}%
\pgfpathlineto{\pgfqpoint{0.977085in}{1.122458in}}%
\pgfpathlineto{\pgfqpoint{0.971628in}{1.117365in}}%
\pgfpathlineto{\pgfqpoint{0.977085in}{1.115182in}}%
\pgfpathlineto{\pgfqpoint{0.988817in}{1.110632in}}%
\pgfpathlineto{\pgfqpoint{0.997478in}{1.105954in}}%
\pgfpathclose%
\pgfusepath{fill}%
\end{pgfscope}%
\begin{pgfscope}%
\pgfpathrectangle{\pgfqpoint{0.211875in}{0.211875in}}{\pgfqpoint{1.313625in}{1.279725in}}%
\pgfusepath{clip}%
\pgfsetbuttcap%
\pgfsetroundjoin%
\definecolor{currentfill}{rgb}{0.445089,0.121699,0.342662}%
\pgfsetfillcolor{currentfill}%
\pgfsetlinewidth{0.000000pt}%
\definecolor{currentstroke}{rgb}{0.000000,0.000000,0.000000}%
\pgfsetstrokecolor{currentstroke}%
\pgfsetdash{}{0pt}%
\pgfpathmoveto{\pgfqpoint{0.812834in}{1.136461in}}%
\pgfpathlineto{\pgfqpoint{0.817255in}{1.140187in}}%
\pgfpathlineto{\pgfqpoint{0.824566in}{1.149801in}}%
\pgfpathlineto{\pgfqpoint{0.836298in}{1.148039in}}%
\pgfpathlineto{\pgfqpoint{0.848030in}{1.145988in}}%
\pgfpathlineto{\pgfqpoint{0.859762in}{1.143600in}}%
\pgfpathlineto{\pgfqpoint{0.871495in}{1.140823in}}%
\pgfpathlineto{\pgfqpoint{0.874002in}{1.140187in}}%
\pgfpathlineto{\pgfqpoint{0.883227in}{1.139260in}}%
\pgfpathlineto{\pgfqpoint{0.894959in}{1.137954in}}%
\pgfpathlineto{\pgfqpoint{0.906691in}{1.136539in}}%
\pgfpathlineto{\pgfqpoint{0.918424in}{1.135257in}}%
\pgfpathlineto{\pgfqpoint{0.930156in}{1.134436in}}%
\pgfpathlineto{\pgfqpoint{0.941888in}{1.133375in}}%
\pgfpathlineto{\pgfqpoint{0.953620in}{1.133036in}}%
\pgfpathlineto{\pgfqpoint{0.965352in}{1.134494in}}%
\pgfpathlineto{\pgfqpoint{0.977085in}{1.137742in}}%
\pgfpathlineto{\pgfqpoint{0.984036in}{1.140187in}}%
\pgfpathlineto{\pgfqpoint{0.977085in}{1.143790in}}%
\pgfpathlineto{\pgfqpoint{0.971287in}{1.151598in}}%
\pgfpathlineto{\pgfqpoint{0.965352in}{1.153385in}}%
\pgfpathlineto{\pgfqpoint{0.954104in}{1.163008in}}%
\pgfpathlineto{\pgfqpoint{0.953620in}{1.163250in}}%
\pgfpathlineto{\pgfqpoint{0.941888in}{1.164436in}}%
\pgfpathlineto{\pgfqpoint{0.930156in}{1.168852in}}%
\pgfpathlineto{\pgfqpoint{0.918424in}{1.170793in}}%
\pgfpathlineto{\pgfqpoint{0.906691in}{1.172753in}}%
\pgfpathlineto{\pgfqpoint{0.896801in}{1.174419in}}%
\pgfpathlineto{\pgfqpoint{0.894959in}{1.174692in}}%
\pgfpathlineto{\pgfqpoint{0.883227in}{1.176525in}}%
\pgfpathlineto{\pgfqpoint{0.871495in}{1.178362in}}%
\pgfpathlineto{\pgfqpoint{0.859762in}{1.180148in}}%
\pgfpathlineto{\pgfqpoint{0.848030in}{1.181809in}}%
\pgfpathlineto{\pgfqpoint{0.836298in}{1.183604in}}%
\pgfpathlineto{\pgfqpoint{0.824566in}{1.184580in}}%
\pgfpathlineto{\pgfqpoint{0.812834in}{1.185090in}}%
\pgfpathlineto{\pgfqpoint{0.803446in}{1.185830in}}%
\pgfpathlineto{\pgfqpoint{0.801101in}{1.185946in}}%
\pgfpathlineto{\pgfqpoint{0.799953in}{1.185830in}}%
\pgfpathlineto{\pgfqpoint{0.797452in}{1.174419in}}%
\pgfpathlineto{\pgfqpoint{0.797909in}{1.163008in}}%
\pgfpathlineto{\pgfqpoint{0.801101in}{1.154019in}}%
\pgfpathlineto{\pgfqpoint{0.805270in}{1.151598in}}%
\pgfpathlineto{\pgfqpoint{0.810529in}{1.140187in}}%
\pgfpathclose%
\pgfusepath{fill}%
\end{pgfscope}%
\begin{pgfscope}%
\pgfpathrectangle{\pgfqpoint{0.211875in}{0.211875in}}{\pgfqpoint{1.313625in}{1.279725in}}%
\pgfusepath{clip}%
\pgfsetbuttcap%
\pgfsetroundjoin%
\definecolor{currentfill}{rgb}{0.445089,0.121699,0.342662}%
\pgfsetfillcolor{currentfill}%
\pgfsetlinewidth{0.000000pt}%
\definecolor{currentstroke}{rgb}{0.000000,0.000000,0.000000}%
\pgfsetstrokecolor{currentstroke}%
\pgfsetdash{}{0pt}%
\pgfpathmoveto{\pgfqpoint{0.789369in}{1.187267in}}%
\pgfpathlineto{\pgfqpoint{0.794025in}{1.197241in}}%
\pgfpathlineto{\pgfqpoint{0.789369in}{1.200355in}}%
\pgfpathlineto{\pgfqpoint{0.788078in}{1.197241in}}%
\pgfpathclose%
\pgfusepath{fill}%
\end{pgfscope}%
\begin{pgfscope}%
\pgfpathrectangle{\pgfqpoint{0.211875in}{0.211875in}}{\pgfqpoint{1.313625in}{1.279725in}}%
\pgfusepath{clip}%
\pgfsetbuttcap%
\pgfsetroundjoin%
\definecolor{currentfill}{rgb}{0.445089,0.121699,0.342662}%
\pgfsetfillcolor{currentfill}%
\pgfsetlinewidth{0.000000pt}%
\definecolor{currentstroke}{rgb}{0.000000,0.000000,0.000000}%
\pgfsetstrokecolor{currentstroke}%
\pgfsetdash{}{0pt}%
\pgfpathmoveto{\pgfqpoint{0.730708in}{1.276012in}}%
\pgfpathlineto{\pgfqpoint{0.731864in}{1.277116in}}%
\pgfpathlineto{\pgfqpoint{0.730708in}{1.282257in}}%
\pgfpathlineto{\pgfqpoint{0.729382in}{1.277116in}}%
\pgfpathclose%
\pgfusepath{fill}%
\end{pgfscope}%
\begin{pgfscope}%
\pgfpathrectangle{\pgfqpoint{0.211875in}{0.211875in}}{\pgfqpoint{1.313625in}{1.279725in}}%
\pgfusepath{clip}%
\pgfsetbuttcap%
\pgfsetroundjoin%
\definecolor{currentfill}{rgb}{0.445089,0.121699,0.342662}%
\pgfsetfillcolor{currentfill}%
\pgfsetlinewidth{0.000000pt}%
\definecolor{currentstroke}{rgb}{0.000000,0.000000,0.000000}%
\pgfsetstrokecolor{currentstroke}%
\pgfsetdash{}{0pt}%
\pgfpathmoveto{\pgfqpoint{0.718976in}{1.283498in}}%
\pgfpathlineto{\pgfqpoint{0.728776in}{1.288527in}}%
\pgfpathlineto{\pgfqpoint{0.722898in}{1.299938in}}%
\pgfpathlineto{\pgfqpoint{0.718976in}{1.305512in}}%
\pgfpathlineto{\pgfqpoint{0.714714in}{1.311348in}}%
\pgfpathlineto{\pgfqpoint{0.707244in}{1.321769in}}%
\pgfpathlineto{\pgfqpoint{0.706503in}{1.322759in}}%
\pgfpathlineto{\pgfqpoint{0.697848in}{1.334170in}}%
\pgfpathlineto{\pgfqpoint{0.695512in}{1.337089in}}%
\pgfpathlineto{\pgfqpoint{0.688340in}{1.345581in}}%
\pgfpathlineto{\pgfqpoint{0.683779in}{1.350960in}}%
\pgfpathlineto{\pgfqpoint{0.678366in}{1.356992in}}%
\pgfpathlineto{\pgfqpoint{0.672047in}{1.364620in}}%
\pgfpathlineto{\pgfqpoint{0.667780in}{1.368402in}}%
\pgfpathlineto{\pgfqpoint{0.660315in}{1.378043in}}%
\pgfpathlineto{\pgfqpoint{0.655610in}{1.379813in}}%
\pgfpathlineto{\pgfqpoint{0.649297in}{1.391224in}}%
\pgfpathlineto{\pgfqpoint{0.648583in}{1.392085in}}%
\pgfpathlineto{\pgfqpoint{0.646516in}{1.391224in}}%
\pgfpathlineto{\pgfqpoint{0.644971in}{1.379813in}}%
\pgfpathlineto{\pgfqpoint{0.646753in}{1.368402in}}%
\pgfpathlineto{\pgfqpoint{0.648583in}{1.362244in}}%
\pgfpathlineto{\pgfqpoint{0.650212in}{1.356992in}}%
\pgfpathlineto{\pgfqpoint{0.652924in}{1.345581in}}%
\pgfpathlineto{\pgfqpoint{0.660315in}{1.340688in}}%
\pgfpathlineto{\pgfqpoint{0.665227in}{1.334170in}}%
\pgfpathlineto{\pgfqpoint{0.672047in}{1.327544in}}%
\pgfpathlineto{\pgfqpoint{0.676196in}{1.322759in}}%
\pgfpathlineto{\pgfqpoint{0.683779in}{1.315564in}}%
\pgfpathlineto{\pgfqpoint{0.688264in}{1.311348in}}%
\pgfpathlineto{\pgfqpoint{0.695512in}{1.304633in}}%
\pgfpathlineto{\pgfqpoint{0.700633in}{1.299938in}}%
\pgfpathlineto{\pgfqpoint{0.707244in}{1.293951in}}%
\pgfpathlineto{\pgfqpoint{0.713303in}{1.288527in}}%
\pgfpathclose%
\pgfpathmoveto{\pgfqpoint{0.697884in}{1.311348in}}%
\pgfpathlineto{\pgfqpoint{0.707244in}{1.312837in}}%
\pgfpathlineto{\pgfqpoint{0.708311in}{1.311348in}}%
\pgfpathlineto{\pgfqpoint{0.707244in}{1.305061in}}%
\pgfpathclose%
\pgfpathmoveto{\pgfqpoint{0.683692in}{1.322759in}}%
\pgfpathlineto{\pgfqpoint{0.675582in}{1.334170in}}%
\pgfpathlineto{\pgfqpoint{0.672047in}{1.341100in}}%
\pgfpathlineto{\pgfqpoint{0.669642in}{1.345581in}}%
\pgfpathlineto{\pgfqpoint{0.672047in}{1.351772in}}%
\pgfpathlineto{\pgfqpoint{0.675706in}{1.345581in}}%
\pgfpathlineto{\pgfqpoint{0.683779in}{1.341234in}}%
\pgfpathlineto{\pgfqpoint{0.689159in}{1.334170in}}%
\pgfpathlineto{\pgfqpoint{0.695512in}{1.328493in}}%
\pgfpathlineto{\pgfqpoint{0.699826in}{1.322759in}}%
\pgfpathlineto{\pgfqpoint{0.695512in}{1.312049in}}%
\pgfpathlineto{\pgfqpoint{0.683779in}{1.322676in}}%
\pgfpathclose%
\pgfusepath{fill}%
\end{pgfscope}%
\begin{pgfscope}%
\pgfpathrectangle{\pgfqpoint{0.211875in}{0.211875in}}{\pgfqpoint{1.313625in}{1.279725in}}%
\pgfusepath{clip}%
\pgfsetbuttcap%
\pgfsetroundjoin%
\definecolor{currentfill}{rgb}{0.445089,0.121699,0.342662}%
\pgfsetfillcolor{currentfill}%
\pgfsetlinewidth{0.000000pt}%
\definecolor{currentstroke}{rgb}{0.000000,0.000000,0.000000}%
\pgfsetstrokecolor{currentstroke}%
\pgfsetdash{}{0pt}%
\pgfpathmoveto{\pgfqpoint{0.636851in}{1.398242in}}%
\pgfpathlineto{\pgfqpoint{0.639113in}{1.402635in}}%
\pgfpathlineto{\pgfqpoint{0.636851in}{1.405154in}}%
\pgfpathlineto{\pgfqpoint{0.634516in}{1.402635in}}%
\pgfpathclose%
\pgfusepath{fill}%
\end{pgfscope}%
\begin{pgfscope}%
\pgfpathrectangle{\pgfqpoint{0.211875in}{0.211875in}}{\pgfqpoint{1.313625in}{1.279725in}}%
\pgfusepath{clip}%
\pgfsetbuttcap%
\pgfsetroundjoin%
\definecolor{currentfill}{rgb}{0.584229,0.109227,0.358485}%
\pgfsetfillcolor{currentfill}%
\pgfsetlinewidth{0.000000pt}%
\definecolor{currentstroke}{rgb}{0.000000,0.000000,0.000000}%
\pgfsetstrokecolor{currentstroke}%
\pgfsetdash{}{0pt}%
\pgfpathmoveto{\pgfqpoint{0.320081in}{0.284378in}}%
\pgfpathlineto{\pgfqpoint{0.331813in}{0.284378in}}%
\pgfpathlineto{\pgfqpoint{0.343545in}{0.284378in}}%
\pgfpathlineto{\pgfqpoint{0.355278in}{0.284378in}}%
\pgfpathlineto{\pgfqpoint{0.367010in}{0.284378in}}%
\pgfpathlineto{\pgfqpoint{0.378742in}{0.284378in}}%
\pgfpathlineto{\pgfqpoint{0.390474in}{0.284378in}}%
\pgfpathlineto{\pgfqpoint{0.402206in}{0.284378in}}%
\pgfpathlineto{\pgfqpoint{0.413939in}{0.284378in}}%
\pgfpathlineto{\pgfqpoint{0.425671in}{0.284378in}}%
\pgfpathlineto{\pgfqpoint{0.428723in}{0.284378in}}%
\pgfpathlineto{\pgfqpoint{0.434951in}{0.295789in}}%
\pgfpathlineto{\pgfqpoint{0.437403in}{0.299763in}}%
\pgfpathlineto{\pgfqpoint{0.440904in}{0.307200in}}%
\pgfpathlineto{\pgfqpoint{0.447309in}{0.318611in}}%
\pgfpathlineto{\pgfqpoint{0.449135in}{0.321555in}}%
\pgfpathlineto{\pgfqpoint{0.453099in}{0.330022in}}%
\pgfpathlineto{\pgfqpoint{0.459710in}{0.341432in}}%
\pgfpathlineto{\pgfqpoint{0.460867in}{0.343287in}}%
\pgfpathlineto{\pgfqpoint{0.465322in}{0.352843in}}%
\pgfpathlineto{\pgfqpoint{0.472175in}{0.364254in}}%
\pgfpathlineto{\pgfqpoint{0.472600in}{0.364928in}}%
\pgfpathlineto{\pgfqpoint{0.477589in}{0.375665in}}%
\pgfpathlineto{\pgfqpoint{0.484332in}{0.386610in}}%
\pgfpathlineto{\pgfqpoint{0.484556in}{0.387075in}}%
\pgfpathlineto{\pgfqpoint{0.489925in}{0.398486in}}%
\pgfpathlineto{\pgfqpoint{0.496064in}{0.408358in}}%
\pgfpathlineto{\pgfqpoint{0.496805in}{0.409897in}}%
\pgfpathlineto{\pgfqpoint{0.502359in}{0.421308in}}%
\pgfpathlineto{\pgfqpoint{0.507796in}{0.429948in}}%
\pgfpathlineto{\pgfqpoint{0.509132in}{0.432719in}}%
\pgfpathlineto{\pgfqpoint{0.514934in}{0.444129in}}%
\pgfpathlineto{\pgfqpoint{0.519528in}{0.451326in}}%
\pgfpathlineto{\pgfqpoint{0.521569in}{0.455540in}}%
\pgfpathlineto{\pgfqpoint{0.527727in}{0.466951in}}%
\pgfpathlineto{\pgfqpoint{0.531261in}{0.472295in}}%
\pgfpathlineto{\pgfqpoint{0.534256in}{0.478362in}}%
\pgfpathlineto{\pgfqpoint{0.540894in}{0.489772in}}%
\pgfpathlineto{\pgfqpoint{0.542993in}{0.493228in}}%
\pgfpathlineto{\pgfqpoint{0.546532in}{0.501183in}}%
\pgfpathlineto{\pgfqpoint{0.552638in}{0.512594in}}%
\pgfpathlineto{\pgfqpoint{0.554725in}{0.516067in}}%
\pgfpathlineto{\pgfqpoint{0.558143in}{0.524005in}}%
\pgfpathlineto{\pgfqpoint{0.564247in}{0.535416in}}%
\pgfpathlineto{\pgfqpoint{0.566457in}{0.538957in}}%
\pgfpathlineto{\pgfqpoint{0.569931in}{0.546826in}}%
\pgfpathlineto{\pgfqpoint{0.576450in}{0.558237in}}%
\pgfpathlineto{\pgfqpoint{0.578190in}{0.560882in}}%
\pgfpathlineto{\pgfqpoint{0.582241in}{0.569648in}}%
\pgfpathlineto{\pgfqpoint{0.589372in}{0.581059in}}%
\pgfpathlineto{\pgfqpoint{0.589922in}{0.581874in}}%
\pgfpathlineto{\pgfqpoint{0.594906in}{0.592469in}}%
\pgfpathlineto{\pgfqpoint{0.601654in}{0.602621in}}%
\pgfpathlineto{\pgfqpoint{0.602285in}{0.603880in}}%
\pgfpathlineto{\pgfqpoint{0.608129in}{0.615291in}}%
\pgfpathlineto{\pgfqpoint{0.613386in}{0.622810in}}%
\pgfpathlineto{\pgfqpoint{0.615404in}{0.626702in}}%
\pgfpathlineto{\pgfqpoint{0.622599in}{0.638113in}}%
\pgfpathlineto{\pgfqpoint{0.625118in}{0.641131in}}%
\pgfpathlineto{\pgfqpoint{0.630255in}{0.649523in}}%
\pgfpathlineto{\pgfqpoint{0.636851in}{0.657742in}}%
\pgfpathlineto{\pgfqpoint{0.638896in}{0.660934in}}%
\pgfpathlineto{\pgfqpoint{0.647592in}{0.672345in}}%
\pgfpathlineto{\pgfqpoint{0.648583in}{0.673413in}}%
\pgfpathlineto{\pgfqpoint{0.656199in}{0.683756in}}%
\pgfpathlineto{\pgfqpoint{0.660315in}{0.688342in}}%
\pgfpathlineto{\pgfqpoint{0.665131in}{0.695166in}}%
\pgfpathlineto{\pgfqpoint{0.672047in}{0.703651in}}%
\pgfpathlineto{\pgfqpoint{0.674142in}{0.706577in}}%
\pgfpathlineto{\pgfqpoint{0.682732in}{0.717988in}}%
\pgfpathlineto{\pgfqpoint{0.683779in}{0.719381in}}%
\pgfpathlineto{\pgfqpoint{0.690215in}{0.729399in}}%
\pgfpathlineto{\pgfqpoint{0.695512in}{0.736827in}}%
\pgfpathlineto{\pgfqpoint{0.698023in}{0.740810in}}%
\pgfpathlineto{\pgfqpoint{0.704635in}{0.752220in}}%
\pgfpathlineto{\pgfqpoint{0.707244in}{0.757300in}}%
\pgfpathlineto{\pgfqpoint{0.710000in}{0.763631in}}%
\pgfpathlineto{\pgfqpoint{0.715466in}{0.775042in}}%
\pgfpathlineto{\pgfqpoint{0.718976in}{0.780249in}}%
\pgfpathlineto{\pgfqpoint{0.722116in}{0.786453in}}%
\pgfpathlineto{\pgfqpoint{0.730209in}{0.797863in}}%
\pgfpathlineto{\pgfqpoint{0.730708in}{0.798376in}}%
\pgfpathlineto{\pgfqpoint{0.739393in}{0.809274in}}%
\pgfpathlineto{\pgfqpoint{0.742440in}{0.812391in}}%
\pgfpathlineto{\pgfqpoint{0.749164in}{0.820685in}}%
\pgfpathlineto{\pgfqpoint{0.754173in}{0.825718in}}%
\pgfpathlineto{\pgfqpoint{0.759796in}{0.832096in}}%
\pgfpathlineto{\pgfqpoint{0.765905in}{0.838052in}}%
\pgfpathlineto{\pgfqpoint{0.771752in}{0.843507in}}%
\pgfpathlineto{\pgfqpoint{0.777637in}{0.847753in}}%
\pgfpathlineto{\pgfqpoint{0.786092in}{0.854917in}}%
\pgfpathlineto{\pgfqpoint{0.789369in}{0.857161in}}%
\pgfpathlineto{\pgfqpoint{0.801101in}{0.866260in}}%
\pgfpathlineto{\pgfqpoint{0.801180in}{0.866328in}}%
\pgfpathlineto{\pgfqpoint{0.812834in}{0.874677in}}%
\pgfpathlineto{\pgfqpoint{0.816591in}{0.877739in}}%
\pgfpathlineto{\pgfqpoint{0.824566in}{0.883719in}}%
\pgfpathlineto{\pgfqpoint{0.830691in}{0.889150in}}%
\pgfpathlineto{\pgfqpoint{0.836298in}{0.895059in}}%
\pgfpathlineto{\pgfqpoint{0.841745in}{0.900560in}}%
\pgfpathlineto{\pgfqpoint{0.848030in}{0.908422in}}%
\pgfpathlineto{\pgfqpoint{0.850525in}{0.911971in}}%
\pgfpathlineto{\pgfqpoint{0.859762in}{0.917458in}}%
\pgfpathlineto{\pgfqpoint{0.871495in}{0.914009in}}%
\pgfpathlineto{\pgfqpoint{0.875813in}{0.911971in}}%
\pgfpathlineto{\pgfqpoint{0.883227in}{0.905399in}}%
\pgfpathlineto{\pgfqpoint{0.887210in}{0.900560in}}%
\pgfpathlineto{\pgfqpoint{0.892843in}{0.889150in}}%
\pgfpathlineto{\pgfqpoint{0.894959in}{0.884095in}}%
\pgfpathlineto{\pgfqpoint{0.897841in}{0.877739in}}%
\pgfpathlineto{\pgfqpoint{0.901402in}{0.866328in}}%
\pgfpathlineto{\pgfqpoint{0.894959in}{0.855230in}}%
\pgfpathlineto{\pgfqpoint{0.883227in}{0.859306in}}%
\pgfpathlineto{\pgfqpoint{0.871495in}{0.862499in}}%
\pgfpathlineto{\pgfqpoint{0.862457in}{0.854917in}}%
\pgfpathlineto{\pgfqpoint{0.861331in}{0.843507in}}%
\pgfpathlineto{\pgfqpoint{0.860897in}{0.832096in}}%
\pgfpathlineto{\pgfqpoint{0.860206in}{0.820685in}}%
\pgfpathlineto{\pgfqpoint{0.859762in}{0.812335in}}%
\pgfpathlineto{\pgfqpoint{0.859561in}{0.809274in}}%
\pgfpathlineto{\pgfqpoint{0.858679in}{0.797863in}}%
\pgfpathlineto{\pgfqpoint{0.857647in}{0.786453in}}%
\pgfpathlineto{\pgfqpoint{0.856502in}{0.775042in}}%
\pgfpathlineto{\pgfqpoint{0.855452in}{0.763631in}}%
\pgfpathlineto{\pgfqpoint{0.854909in}{0.752220in}}%
\pgfpathlineto{\pgfqpoint{0.855071in}{0.740810in}}%
\pgfpathlineto{\pgfqpoint{0.855383in}{0.729399in}}%
\pgfpathlineto{\pgfqpoint{0.856533in}{0.717988in}}%
\pgfpathlineto{\pgfqpoint{0.858241in}{0.706577in}}%
\pgfpathlineto{\pgfqpoint{0.859762in}{0.699088in}}%
\pgfpathlineto{\pgfqpoint{0.860418in}{0.695166in}}%
\pgfpathlineto{\pgfqpoint{0.862588in}{0.683756in}}%
\pgfpathlineto{\pgfqpoint{0.864899in}{0.672345in}}%
\pgfpathlineto{\pgfqpoint{0.867601in}{0.660934in}}%
\pgfpathlineto{\pgfqpoint{0.870874in}{0.649523in}}%
\pgfpathlineto{\pgfqpoint{0.871495in}{0.647480in}}%
\pgfpathlineto{\pgfqpoint{0.873648in}{0.638113in}}%
\pgfpathlineto{\pgfqpoint{0.876412in}{0.626702in}}%
\pgfpathlineto{\pgfqpoint{0.879446in}{0.615291in}}%
\pgfpathlineto{\pgfqpoint{0.882865in}{0.603880in}}%
\pgfpathlineto{\pgfqpoint{0.883227in}{0.602709in}}%
\pgfpathlineto{\pgfqpoint{0.885662in}{0.592469in}}%
\pgfpathlineto{\pgfqpoint{0.888520in}{0.581059in}}%
\pgfpathlineto{\pgfqpoint{0.891648in}{0.569648in}}%
\pgfpathlineto{\pgfqpoint{0.894959in}{0.558858in}}%
\pgfpathlineto{\pgfqpoint{0.895116in}{0.558237in}}%
\pgfpathlineto{\pgfqpoint{0.897930in}{0.546826in}}%
\pgfpathlineto{\pgfqpoint{0.900896in}{0.535416in}}%
\pgfpathlineto{\pgfqpoint{0.904129in}{0.524005in}}%
\pgfpathlineto{\pgfqpoint{0.906691in}{0.517201in}}%
\pgfpathlineto{\pgfqpoint{0.908475in}{0.512594in}}%
\pgfpathlineto{\pgfqpoint{0.912747in}{0.501183in}}%
\pgfpathlineto{\pgfqpoint{0.916202in}{0.489772in}}%
\pgfpathlineto{\pgfqpoint{0.918424in}{0.481701in}}%
\pgfpathlineto{\pgfqpoint{0.919271in}{0.478362in}}%
\pgfpathlineto{\pgfqpoint{0.922045in}{0.466951in}}%
\pgfpathlineto{\pgfqpoint{0.924777in}{0.455540in}}%
\pgfpathlineto{\pgfqpoint{0.927522in}{0.444129in}}%
\pgfpathlineto{\pgfqpoint{0.930156in}{0.433426in}}%
\pgfpathlineto{\pgfqpoint{0.930339in}{0.432719in}}%
\pgfpathlineto{\pgfqpoint{0.933140in}{0.421308in}}%
\pgfpathlineto{\pgfqpoint{0.935433in}{0.409897in}}%
\pgfpathlineto{\pgfqpoint{0.937819in}{0.398486in}}%
\pgfpathlineto{\pgfqpoint{0.939731in}{0.387075in}}%
\pgfpathlineto{\pgfqpoint{0.941539in}{0.375665in}}%
\pgfpathlineto{\pgfqpoint{0.941888in}{0.373395in}}%
\pgfpathlineto{\pgfqpoint{0.943590in}{0.364254in}}%
\pgfpathlineto{\pgfqpoint{0.945570in}{0.352843in}}%
\pgfpathlineto{\pgfqpoint{0.947406in}{0.341432in}}%
\pgfpathlineto{\pgfqpoint{0.949115in}{0.330022in}}%
\pgfpathlineto{\pgfqpoint{0.950716in}{0.318611in}}%
\pgfpathlineto{\pgfqpoint{0.952221in}{0.307200in}}%
\pgfpathlineto{\pgfqpoint{0.953620in}{0.296051in}}%
\pgfpathlineto{\pgfqpoint{0.953668in}{0.295789in}}%
\pgfpathlineto{\pgfqpoint{0.955630in}{0.284378in}}%
\pgfpathlineto{\pgfqpoint{0.965352in}{0.284378in}}%
\pgfpathlineto{\pgfqpoint{0.977085in}{0.284378in}}%
\pgfpathlineto{\pgfqpoint{0.988817in}{0.284378in}}%
\pgfpathlineto{\pgfqpoint{1.000549in}{0.284378in}}%
\pgfpathlineto{\pgfqpoint{1.012281in}{0.284378in}}%
\pgfpathlineto{\pgfqpoint{1.015471in}{0.284378in}}%
\pgfpathlineto{\pgfqpoint{1.012281in}{0.292370in}}%
\pgfpathlineto{\pgfqpoint{1.010384in}{0.295789in}}%
\pgfpathlineto{\pgfqpoint{1.004641in}{0.307200in}}%
\pgfpathlineto{\pgfqpoint{1.000549in}{0.316942in}}%
\pgfpathlineto{\pgfqpoint{0.999588in}{0.318611in}}%
\pgfpathlineto{\pgfqpoint{0.993454in}{0.330022in}}%
\pgfpathlineto{\pgfqpoint{0.988817in}{0.340508in}}%
\pgfpathlineto{\pgfqpoint{0.988301in}{0.341432in}}%
\pgfpathlineto{\pgfqpoint{0.982141in}{0.352843in}}%
\pgfpathlineto{\pgfqpoint{0.977085in}{0.363954in}}%
\pgfpathlineto{\pgfqpoint{0.976942in}{0.364254in}}%
\pgfpathlineto{\pgfqpoint{0.971378in}{0.375665in}}%
\pgfpathlineto{\pgfqpoint{0.966269in}{0.387075in}}%
\pgfpathlineto{\pgfqpoint{0.965352in}{0.389176in}}%
\pgfpathlineto{\pgfqpoint{0.961735in}{0.398486in}}%
\pgfpathlineto{\pgfqpoint{0.957214in}{0.409897in}}%
\pgfpathlineto{\pgfqpoint{0.953620in}{0.419048in}}%
\pgfpathlineto{\pgfqpoint{0.952942in}{0.421308in}}%
\pgfpathlineto{\pgfqpoint{0.949260in}{0.432719in}}%
\pgfpathlineto{\pgfqpoint{0.945365in}{0.444129in}}%
\pgfpathlineto{\pgfqpoint{0.941888in}{0.453981in}}%
\pgfpathlineto{\pgfqpoint{0.941463in}{0.455540in}}%
\pgfpathlineto{\pgfqpoint{0.938178in}{0.466951in}}%
\pgfpathlineto{\pgfqpoint{0.934705in}{0.478362in}}%
\pgfpathlineto{\pgfqpoint{0.931061in}{0.489772in}}%
\pgfpathlineto{\pgfqpoint{0.930156in}{0.492492in}}%
\pgfpathlineto{\pgfqpoint{0.927754in}{0.501183in}}%
\pgfpathlineto{\pgfqpoint{0.924133in}{0.512594in}}%
\pgfpathlineto{\pgfqpoint{0.919277in}{0.524005in}}%
\pgfpathlineto{\pgfqpoint{0.918424in}{0.525767in}}%
\pgfpathlineto{\pgfqpoint{0.914628in}{0.535416in}}%
\pgfpathlineto{\pgfqpoint{0.910809in}{0.546826in}}%
\pgfpathlineto{\pgfqpoint{0.907568in}{0.558237in}}%
\pgfpathlineto{\pgfqpoint{0.906691in}{0.561251in}}%
\pgfpathlineto{\pgfqpoint{0.904301in}{0.569648in}}%
\pgfpathlineto{\pgfqpoint{0.901163in}{0.581059in}}%
\pgfpathlineto{\pgfqpoint{0.898075in}{0.592469in}}%
\pgfpathlineto{\pgfqpoint{0.894961in}{0.603880in}}%
\pgfpathlineto{\pgfqpoint{0.894959in}{0.603887in}}%
\pgfpathlineto{\pgfqpoint{0.891749in}{0.615291in}}%
\pgfpathlineto{\pgfqpoint{0.888701in}{0.626702in}}%
\pgfpathlineto{\pgfqpoint{0.885724in}{0.638113in}}%
\pgfpathlineto{\pgfqpoint{0.883227in}{0.647648in}}%
\pgfpathlineto{\pgfqpoint{0.882694in}{0.649523in}}%
\pgfpathlineto{\pgfqpoint{0.879544in}{0.660934in}}%
\pgfpathlineto{\pgfqpoint{0.876623in}{0.672345in}}%
\pgfpathlineto{\pgfqpoint{0.874012in}{0.683756in}}%
\pgfpathlineto{\pgfqpoint{0.871565in}{0.695166in}}%
\pgfpathlineto{\pgfqpoint{0.871495in}{0.695486in}}%
\pgfpathlineto{\pgfqpoint{0.868904in}{0.706577in}}%
\pgfpathlineto{\pgfqpoint{0.866622in}{0.717988in}}%
\pgfpathlineto{\pgfqpoint{0.864855in}{0.729399in}}%
\pgfpathlineto{\pgfqpoint{0.863949in}{0.740810in}}%
\pgfpathlineto{\pgfqpoint{0.863409in}{0.752220in}}%
\pgfpathlineto{\pgfqpoint{0.863497in}{0.763631in}}%
\pgfpathlineto{\pgfqpoint{0.864107in}{0.775042in}}%
\pgfpathlineto{\pgfqpoint{0.864779in}{0.786453in}}%
\pgfpathlineto{\pgfqpoint{0.865379in}{0.797863in}}%
\pgfpathlineto{\pgfqpoint{0.866322in}{0.809274in}}%
\pgfpathlineto{\pgfqpoint{0.867444in}{0.820685in}}%
\pgfpathlineto{\pgfqpoint{0.868971in}{0.832096in}}%
\pgfpathlineto{\pgfqpoint{0.871495in}{0.842242in}}%
\pgfpathlineto{\pgfqpoint{0.872656in}{0.843507in}}%
\pgfpathlineto{\pgfqpoint{0.883227in}{0.847624in}}%
\pgfpathlineto{\pgfqpoint{0.894959in}{0.843845in}}%
\pgfpathlineto{\pgfqpoint{0.895980in}{0.843507in}}%
\pgfpathlineto{\pgfqpoint{0.906691in}{0.839444in}}%
\pgfpathlineto{\pgfqpoint{0.913279in}{0.843507in}}%
\pgfpathlineto{\pgfqpoint{0.913053in}{0.854917in}}%
\pgfpathlineto{\pgfqpoint{0.909686in}{0.866328in}}%
\pgfpathlineto{\pgfqpoint{0.906691in}{0.874811in}}%
\pgfpathlineto{\pgfqpoint{0.905578in}{0.877739in}}%
\pgfpathlineto{\pgfqpoint{0.900260in}{0.889150in}}%
\pgfpathlineto{\pgfqpoint{0.894959in}{0.899661in}}%
\pgfpathlineto{\pgfqpoint{0.894519in}{0.900560in}}%
\pgfpathlineto{\pgfqpoint{0.885598in}{0.911971in}}%
\pgfpathlineto{\pgfqpoint{0.883227in}{0.913416in}}%
\pgfpathlineto{\pgfqpoint{0.871495in}{0.918735in}}%
\pgfpathlineto{\pgfqpoint{0.859762in}{0.922306in}}%
\pgfpathlineto{\pgfqpoint{0.851122in}{0.923382in}}%
\pgfpathlineto{\pgfqpoint{0.848030in}{0.925322in}}%
\pgfpathlineto{\pgfqpoint{0.844041in}{0.934793in}}%
\pgfpathlineto{\pgfqpoint{0.847192in}{0.946204in}}%
\pgfpathlineto{\pgfqpoint{0.848030in}{0.949169in}}%
\pgfpathlineto{\pgfqpoint{0.851050in}{0.957614in}}%
\pgfpathlineto{\pgfqpoint{0.852601in}{0.969025in}}%
\pgfpathlineto{\pgfqpoint{0.848030in}{0.978773in}}%
\pgfpathlineto{\pgfqpoint{0.847138in}{0.980436in}}%
\pgfpathlineto{\pgfqpoint{0.840698in}{0.991847in}}%
\pgfpathlineto{\pgfqpoint{0.836298in}{0.999778in}}%
\pgfpathlineto{\pgfqpoint{0.834300in}{1.003257in}}%
\pgfpathlineto{\pgfqpoint{0.828484in}{1.014668in}}%
\pgfpathlineto{\pgfqpoint{0.824566in}{1.019678in}}%
\pgfpathlineto{\pgfqpoint{0.822767in}{1.026079in}}%
\pgfpathlineto{\pgfqpoint{0.812834in}{1.035104in}}%
\pgfpathlineto{\pgfqpoint{0.803539in}{1.037490in}}%
\pgfpathlineto{\pgfqpoint{0.801101in}{1.038385in}}%
\pgfpathlineto{\pgfqpoint{0.789369in}{1.045022in}}%
\pgfpathlineto{\pgfqpoint{0.784951in}{1.048901in}}%
\pgfpathlineto{\pgfqpoint{0.777637in}{1.055875in}}%
\pgfpathlineto{\pgfqpoint{0.773304in}{1.060311in}}%
\pgfpathlineto{\pgfqpoint{0.765905in}{1.068500in}}%
\pgfpathlineto{\pgfqpoint{0.762634in}{1.071722in}}%
\pgfpathlineto{\pgfqpoint{0.754173in}{1.080518in}}%
\pgfpathlineto{\pgfqpoint{0.751178in}{1.083133in}}%
\pgfpathlineto{\pgfqpoint{0.744966in}{1.094544in}}%
\pgfpathlineto{\pgfqpoint{0.742440in}{1.098216in}}%
\pgfpathlineto{\pgfqpoint{0.730708in}{1.105882in}}%
\pgfpathlineto{\pgfqpoint{0.730551in}{1.105954in}}%
\pgfpathlineto{\pgfqpoint{0.718976in}{1.111489in}}%
\pgfpathlineto{\pgfqpoint{0.711323in}{1.117365in}}%
\pgfpathlineto{\pgfqpoint{0.707244in}{1.120222in}}%
\pgfpathlineto{\pgfqpoint{0.695512in}{1.128633in}}%
\pgfpathlineto{\pgfqpoint{0.695302in}{1.128776in}}%
\pgfpathlineto{\pgfqpoint{0.683779in}{1.136754in}}%
\pgfpathlineto{\pgfqpoint{0.678668in}{1.140187in}}%
\pgfpathlineto{\pgfqpoint{0.672047in}{1.144775in}}%
\pgfpathlineto{\pgfqpoint{0.661900in}{1.151598in}}%
\pgfpathlineto{\pgfqpoint{0.660315in}{1.152798in}}%
\pgfpathlineto{\pgfqpoint{0.654068in}{1.163008in}}%
\pgfpathlineto{\pgfqpoint{0.649318in}{1.174419in}}%
\pgfpathlineto{\pgfqpoint{0.648583in}{1.176697in}}%
\pgfpathlineto{\pgfqpoint{0.645106in}{1.185830in}}%
\pgfpathlineto{\pgfqpoint{0.641949in}{1.197241in}}%
\pgfpathlineto{\pgfqpoint{0.640038in}{1.208651in}}%
\pgfpathlineto{\pgfqpoint{0.639626in}{1.220062in}}%
\pgfpathlineto{\pgfqpoint{0.641597in}{1.231473in}}%
\pgfpathlineto{\pgfqpoint{0.645750in}{1.242884in}}%
\pgfpathlineto{\pgfqpoint{0.648583in}{1.252629in}}%
\pgfpathlineto{\pgfqpoint{0.648769in}{1.254295in}}%
\pgfpathlineto{\pgfqpoint{0.648583in}{1.255331in}}%
\pgfpathlineto{\pgfqpoint{0.645937in}{1.265705in}}%
\pgfpathlineto{\pgfqpoint{0.643308in}{1.277116in}}%
\pgfpathlineto{\pgfqpoint{0.640048in}{1.288527in}}%
\pgfpathlineto{\pgfqpoint{0.636924in}{1.299938in}}%
\pgfpathlineto{\pgfqpoint{0.636851in}{1.300258in}}%
\pgfpathlineto{\pgfqpoint{0.634099in}{1.311348in}}%
\pgfpathlineto{\pgfqpoint{0.632249in}{1.322759in}}%
\pgfpathlineto{\pgfqpoint{0.632980in}{1.334170in}}%
\pgfpathlineto{\pgfqpoint{0.636851in}{1.337441in}}%
\pgfpathlineto{\pgfqpoint{0.648583in}{1.334589in}}%
\pgfpathlineto{\pgfqpoint{0.649293in}{1.334170in}}%
\pgfpathlineto{\pgfqpoint{0.660315in}{1.328444in}}%
\pgfpathlineto{\pgfqpoint{0.666169in}{1.322759in}}%
\pgfpathlineto{\pgfqpoint{0.672047in}{1.318989in}}%
\pgfpathlineto{\pgfqpoint{0.680138in}{1.311348in}}%
\pgfpathlineto{\pgfqpoint{0.683779in}{1.308440in}}%
\pgfpathlineto{\pgfqpoint{0.692964in}{1.299938in}}%
\pgfpathlineto{\pgfqpoint{0.695512in}{1.297627in}}%
\pgfpathlineto{\pgfqpoint{0.705567in}{1.288527in}}%
\pgfpathlineto{\pgfqpoint{0.707244in}{1.287036in}}%
\pgfpathlineto{\pgfqpoint{0.718434in}{1.277116in}}%
\pgfpathlineto{\pgfqpoint{0.718976in}{1.276642in}}%
\pgfpathlineto{\pgfqpoint{0.730708in}{1.266697in}}%
\pgfpathlineto{\pgfqpoint{0.734734in}{1.265705in}}%
\pgfpathlineto{\pgfqpoint{0.741638in}{1.254295in}}%
\pgfpathlineto{\pgfqpoint{0.742440in}{1.253146in}}%
\pgfpathlineto{\pgfqpoint{0.749467in}{1.242884in}}%
\pgfpathlineto{\pgfqpoint{0.754173in}{1.235822in}}%
\pgfpathlineto{\pgfqpoint{0.758923in}{1.231473in}}%
\pgfpathlineto{\pgfqpoint{0.764654in}{1.220062in}}%
\pgfpathlineto{\pgfqpoint{0.765905in}{1.218238in}}%
\pgfpathlineto{\pgfqpoint{0.772373in}{1.208651in}}%
\pgfpathlineto{\pgfqpoint{0.777637in}{1.200844in}}%
\pgfpathlineto{\pgfqpoint{0.780736in}{1.197241in}}%
\pgfpathlineto{\pgfqpoint{0.784606in}{1.185830in}}%
\pgfpathlineto{\pgfqpoint{0.789369in}{1.174818in}}%
\pgfpathlineto{\pgfqpoint{0.789645in}{1.174419in}}%
\pgfpathlineto{\pgfqpoint{0.792871in}{1.163008in}}%
\pgfpathlineto{\pgfqpoint{0.796678in}{1.151598in}}%
\pgfpathlineto{\pgfqpoint{0.801101in}{1.141780in}}%
\pgfpathlineto{\pgfqpoint{0.802220in}{1.140187in}}%
\pgfpathlineto{\pgfqpoint{0.810147in}{1.128776in}}%
\pgfpathlineto{\pgfqpoint{0.812834in}{1.126740in}}%
\pgfpathlineto{\pgfqpoint{0.818094in}{1.128776in}}%
\pgfpathlineto{\pgfqpoint{0.824566in}{1.134401in}}%
\pgfpathlineto{\pgfqpoint{0.836298in}{1.134297in}}%
\pgfpathlineto{\pgfqpoint{0.848030in}{1.132693in}}%
\pgfpathlineto{\pgfqpoint{0.859762in}{1.130860in}}%
\pgfpathlineto{\pgfqpoint{0.871495in}{1.128805in}}%
\pgfpathlineto{\pgfqpoint{0.871659in}{1.128776in}}%
\pgfpathlineto{\pgfqpoint{0.883227in}{1.127777in}}%
\pgfpathlineto{\pgfqpoint{0.894959in}{1.126712in}}%
\pgfpathlineto{\pgfqpoint{0.906691in}{1.125632in}}%
\pgfpathlineto{\pgfqpoint{0.918424in}{1.124358in}}%
\pgfpathlineto{\pgfqpoint{0.930156in}{1.121764in}}%
\pgfpathlineto{\pgfqpoint{0.941784in}{1.117365in}}%
\pgfpathlineto{\pgfqpoint{0.941888in}{1.117325in}}%
\pgfpathlineto{\pgfqpoint{0.953620in}{1.112842in}}%
\pgfpathlineto{\pgfqpoint{0.965352in}{1.107746in}}%
\pgfpathlineto{\pgfqpoint{0.969132in}{1.105954in}}%
\pgfpathlineto{\pgfqpoint{0.977085in}{1.104177in}}%
\pgfpathlineto{\pgfqpoint{0.988817in}{1.101576in}}%
\pgfpathlineto{\pgfqpoint{1.000549in}{1.100155in}}%
\pgfpathlineto{\pgfqpoint{1.012281in}{1.100018in}}%
\pgfpathlineto{\pgfqpoint{1.024013in}{1.100215in}}%
\pgfpathlineto{\pgfqpoint{1.035746in}{1.100488in}}%
\pgfpathlineto{\pgfqpoint{1.047478in}{1.102005in}}%
\pgfpathlineto{\pgfqpoint{1.053937in}{1.105954in}}%
\pgfpathlineto{\pgfqpoint{1.047478in}{1.110799in}}%
\pgfpathlineto{\pgfqpoint{1.040112in}{1.117365in}}%
\pgfpathlineto{\pgfqpoint{1.035746in}{1.120424in}}%
\pgfpathlineto{\pgfqpoint{1.024013in}{1.127324in}}%
\pgfpathlineto{\pgfqpoint{1.021775in}{1.128776in}}%
\pgfpathlineto{\pgfqpoint{1.012281in}{1.137359in}}%
\pgfpathlineto{\pgfqpoint{1.009467in}{1.140187in}}%
\pgfpathlineto{\pgfqpoint{1.000549in}{1.145889in}}%
\pgfpathlineto{\pgfqpoint{0.991120in}{1.151598in}}%
\pgfpathlineto{\pgfqpoint{0.988817in}{1.152869in}}%
\pgfpathlineto{\pgfqpoint{0.977085in}{1.159424in}}%
\pgfpathlineto{\pgfqpoint{0.970796in}{1.163008in}}%
\pgfpathlineto{\pgfqpoint{0.965352in}{1.165968in}}%
\pgfpathlineto{\pgfqpoint{0.953620in}{1.171872in}}%
\pgfpathlineto{\pgfqpoint{0.948469in}{1.174419in}}%
\pgfpathlineto{\pgfqpoint{0.941888in}{1.177487in}}%
\pgfpathlineto{\pgfqpoint{0.930156in}{1.182580in}}%
\pgfpathlineto{\pgfqpoint{0.924089in}{1.185830in}}%
\pgfpathlineto{\pgfqpoint{0.918424in}{1.188460in}}%
\pgfpathlineto{\pgfqpoint{0.906691in}{1.192942in}}%
\pgfpathlineto{\pgfqpoint{0.899547in}{1.197241in}}%
\pgfpathlineto{\pgfqpoint{0.894959in}{1.199370in}}%
\pgfpathlineto{\pgfqpoint{0.883227in}{1.202922in}}%
\pgfpathlineto{\pgfqpoint{0.874944in}{1.208651in}}%
\pgfpathlineto{\pgfqpoint{0.871495in}{1.210373in}}%
\pgfpathlineto{\pgfqpoint{0.859762in}{1.212368in}}%
\pgfpathlineto{\pgfqpoint{0.848030in}{1.219967in}}%
\pgfpathlineto{\pgfqpoint{0.847133in}{1.220062in}}%
\pgfpathlineto{\pgfqpoint{0.836298in}{1.220616in}}%
\pgfpathlineto{\pgfqpoint{0.824566in}{1.222439in}}%
\pgfpathlineto{\pgfqpoint{0.812834in}{1.223633in}}%
\pgfpathlineto{\pgfqpoint{0.801101in}{1.224799in}}%
\pgfpathlineto{\pgfqpoint{0.789369in}{1.225188in}}%
\pgfpathlineto{\pgfqpoint{0.777637in}{1.230546in}}%
\pgfpathlineto{\pgfqpoint{0.776782in}{1.231473in}}%
\pgfpathlineto{\pgfqpoint{0.766594in}{1.242884in}}%
\pgfpathlineto{\pgfqpoint{0.765905in}{1.243446in}}%
\pgfpathlineto{\pgfqpoint{0.756025in}{1.254295in}}%
\pgfpathlineto{\pgfqpoint{0.754173in}{1.256369in}}%
\pgfpathlineto{\pgfqpoint{0.745642in}{1.265705in}}%
\pgfpathlineto{\pgfqpoint{0.742440in}{1.273781in}}%
\pgfpathlineto{\pgfqpoint{0.741619in}{1.277116in}}%
\pgfpathlineto{\pgfqpoint{0.736419in}{1.288527in}}%
\pgfpathlineto{\pgfqpoint{0.730708in}{1.297233in}}%
\pgfpathlineto{\pgfqpoint{0.729068in}{1.299938in}}%
\pgfpathlineto{\pgfqpoint{0.721027in}{1.311348in}}%
\pgfpathlineto{\pgfqpoint{0.718976in}{1.314301in}}%
\pgfpathlineto{\pgfqpoint{0.712890in}{1.322759in}}%
\pgfpathlineto{\pgfqpoint{0.707244in}{1.330342in}}%
\pgfpathlineto{\pgfqpoint{0.704334in}{1.334170in}}%
\pgfpathlineto{\pgfqpoint{0.695512in}{1.345189in}}%
\pgfpathlineto{\pgfqpoint{0.695181in}{1.345581in}}%
\pgfpathlineto{\pgfqpoint{0.685981in}{1.356992in}}%
\pgfpathlineto{\pgfqpoint{0.683779in}{1.360338in}}%
\pgfpathlineto{\pgfqpoint{0.677710in}{1.368402in}}%
\pgfpathlineto{\pgfqpoint{0.672047in}{1.376548in}}%
\pgfpathlineto{\pgfqpoint{0.669596in}{1.379813in}}%
\pgfpathlineto{\pgfqpoint{0.661563in}{1.391224in}}%
\pgfpathlineto{\pgfqpoint{0.660315in}{1.393371in}}%
\pgfpathlineto{\pgfqpoint{0.654580in}{1.402635in}}%
\pgfpathlineto{\pgfqpoint{0.648583in}{1.410927in}}%
\pgfpathlineto{\pgfqpoint{0.646195in}{1.414045in}}%
\pgfpathlineto{\pgfqpoint{0.636851in}{1.414045in}}%
\pgfpathlineto{\pgfqpoint{0.625118in}{1.414045in}}%
\pgfpathlineto{\pgfqpoint{0.615629in}{1.414045in}}%
\pgfpathlineto{\pgfqpoint{0.621993in}{1.402635in}}%
\pgfpathlineto{\pgfqpoint{0.625118in}{1.394285in}}%
\pgfpathlineto{\pgfqpoint{0.626588in}{1.391224in}}%
\pgfpathlineto{\pgfqpoint{0.626768in}{1.379813in}}%
\pgfpathlineto{\pgfqpoint{0.625118in}{1.378092in}}%
\pgfpathlineto{\pgfqpoint{0.621702in}{1.379813in}}%
\pgfpathlineto{\pgfqpoint{0.613386in}{1.383835in}}%
\pgfpathlineto{\pgfqpoint{0.601654in}{1.390187in}}%
\pgfpathlineto{\pgfqpoint{0.600299in}{1.391224in}}%
\pgfpathlineto{\pgfqpoint{0.589922in}{1.398801in}}%
\pgfpathlineto{\pgfqpoint{0.584745in}{1.402635in}}%
\pgfpathlineto{\pgfqpoint{0.578190in}{1.407278in}}%
\pgfpathlineto{\pgfqpoint{0.568860in}{1.414045in}}%
\pgfpathlineto{\pgfqpoint{0.566457in}{1.414045in}}%
\pgfpathlineto{\pgfqpoint{0.554725in}{1.414045in}}%
\pgfpathlineto{\pgfqpoint{0.553643in}{1.414045in}}%
\pgfpathlineto{\pgfqpoint{0.554725in}{1.413001in}}%
\pgfpathlineto{\pgfqpoint{0.566457in}{1.404234in}}%
\pgfpathlineto{\pgfqpoint{0.568609in}{1.402635in}}%
\pgfpathlineto{\pgfqpoint{0.578190in}{1.394508in}}%
\pgfpathlineto{\pgfqpoint{0.582639in}{1.391224in}}%
\pgfpathlineto{\pgfqpoint{0.589922in}{1.383105in}}%
\pgfpathlineto{\pgfqpoint{0.594836in}{1.379813in}}%
\pgfpathlineto{\pgfqpoint{0.601654in}{1.369391in}}%
\pgfpathlineto{\pgfqpoint{0.602870in}{1.368402in}}%
\pgfpathlineto{\pgfqpoint{0.606510in}{1.356992in}}%
\pgfpathlineto{\pgfqpoint{0.607475in}{1.345581in}}%
\pgfpathlineto{\pgfqpoint{0.610179in}{1.334170in}}%
\pgfpathlineto{\pgfqpoint{0.613386in}{1.323470in}}%
\pgfpathlineto{\pgfqpoint{0.613583in}{1.322759in}}%
\pgfpathlineto{\pgfqpoint{0.617050in}{1.311348in}}%
\pgfpathlineto{\pgfqpoint{0.618936in}{1.299938in}}%
\pgfpathlineto{\pgfqpoint{0.619048in}{1.288527in}}%
\pgfpathlineto{\pgfqpoint{0.613386in}{1.282232in}}%
\pgfpathlineto{\pgfqpoint{0.608803in}{1.277116in}}%
\pgfpathlineto{\pgfqpoint{0.603562in}{1.265705in}}%
\pgfpathlineto{\pgfqpoint{0.601654in}{1.259462in}}%
\pgfpathlineto{\pgfqpoint{0.600149in}{1.254295in}}%
\pgfpathlineto{\pgfqpoint{0.598830in}{1.242884in}}%
\pgfpathlineto{\pgfqpoint{0.600341in}{1.231473in}}%
\pgfpathlineto{\pgfqpoint{0.601654in}{1.226053in}}%
\pgfpathlineto{\pgfqpoint{0.602928in}{1.220062in}}%
\pgfpathlineto{\pgfqpoint{0.606137in}{1.208651in}}%
\pgfpathlineto{\pgfqpoint{0.610876in}{1.197241in}}%
\pgfpathlineto{\pgfqpoint{0.613386in}{1.192625in}}%
\pgfpathlineto{\pgfqpoint{0.616634in}{1.185830in}}%
\pgfpathlineto{\pgfqpoint{0.622764in}{1.174419in}}%
\pgfpathlineto{\pgfqpoint{0.625118in}{1.170406in}}%
\pgfpathlineto{\pgfqpoint{0.628997in}{1.163008in}}%
\pgfpathlineto{\pgfqpoint{0.635301in}{1.151598in}}%
\pgfpathlineto{\pgfqpoint{0.636851in}{1.148812in}}%
\pgfpathlineto{\pgfqpoint{0.641250in}{1.140187in}}%
\pgfpathlineto{\pgfqpoint{0.647069in}{1.128776in}}%
\pgfpathlineto{\pgfqpoint{0.648583in}{1.125863in}}%
\pgfpathlineto{\pgfqpoint{0.658623in}{1.117365in}}%
\pgfpathlineto{\pgfqpoint{0.660315in}{1.115874in}}%
\pgfpathlineto{\pgfqpoint{0.672047in}{1.108089in}}%
\pgfpathlineto{\pgfqpoint{0.675245in}{1.105954in}}%
\pgfpathlineto{\pgfqpoint{0.683779in}{1.100255in}}%
\pgfpathlineto{\pgfqpoint{0.692332in}{1.094544in}}%
\pgfpathlineto{\pgfqpoint{0.695512in}{1.092634in}}%
\pgfpathlineto{\pgfqpoint{0.707244in}{1.085217in}}%
\pgfpathlineto{\pgfqpoint{0.712874in}{1.083133in}}%
\pgfpathlineto{\pgfqpoint{0.718976in}{1.081199in}}%
\pgfpathlineto{\pgfqpoint{0.730708in}{1.075740in}}%
\pgfpathlineto{\pgfqpoint{0.736683in}{1.071722in}}%
\pgfpathlineto{\pgfqpoint{0.742440in}{1.068155in}}%
\pgfpathlineto{\pgfqpoint{0.752493in}{1.060311in}}%
\pgfpathlineto{\pgfqpoint{0.754173in}{1.059042in}}%
\pgfpathlineto{\pgfqpoint{0.765061in}{1.048901in}}%
\pgfpathlineto{\pgfqpoint{0.765905in}{1.048147in}}%
\pgfpathlineto{\pgfqpoint{0.776971in}{1.037490in}}%
\pgfpathlineto{\pgfqpoint{0.777637in}{1.036956in}}%
\pgfpathlineto{\pgfqpoint{0.789369in}{1.029587in}}%
\pgfpathlineto{\pgfqpoint{0.801101in}{1.027108in}}%
\pgfpathlineto{\pgfqpoint{0.812834in}{1.026240in}}%
\pgfpathlineto{\pgfqpoint{0.813011in}{1.026079in}}%
\pgfpathlineto{\pgfqpoint{0.818597in}{1.014668in}}%
\pgfpathlineto{\pgfqpoint{0.821275in}{1.003257in}}%
\pgfpathlineto{\pgfqpoint{0.824566in}{0.994135in}}%
\pgfpathlineto{\pgfqpoint{0.826289in}{0.991847in}}%
\pgfpathlineto{\pgfqpoint{0.831381in}{0.980436in}}%
\pgfpathlineto{\pgfqpoint{0.833922in}{0.969025in}}%
\pgfpathlineto{\pgfqpoint{0.835885in}{0.957614in}}%
\pgfpathlineto{\pgfqpoint{0.836256in}{0.946204in}}%
\pgfpathlineto{\pgfqpoint{0.826378in}{0.934793in}}%
\pgfpathlineto{\pgfqpoint{0.824566in}{0.934415in}}%
\pgfpathlineto{\pgfqpoint{0.818057in}{0.934793in}}%
\pgfpathlineto{\pgfqpoint{0.812834in}{0.935087in}}%
\pgfpathlineto{\pgfqpoint{0.801101in}{0.937498in}}%
\pgfpathlineto{\pgfqpoint{0.789369in}{0.942826in}}%
\pgfpathlineto{\pgfqpoint{0.786291in}{0.946204in}}%
\pgfpathlineto{\pgfqpoint{0.777637in}{0.957018in}}%
\pgfpathlineto{\pgfqpoint{0.777089in}{0.957614in}}%
\pgfpathlineto{\pgfqpoint{0.767174in}{0.969025in}}%
\pgfpathlineto{\pgfqpoint{0.765905in}{0.970573in}}%
\pgfpathlineto{\pgfqpoint{0.757665in}{0.980436in}}%
\pgfpathlineto{\pgfqpoint{0.754173in}{0.984805in}}%
\pgfpathlineto{\pgfqpoint{0.748766in}{0.991847in}}%
\pgfpathlineto{\pgfqpoint{0.742440in}{1.000405in}}%
\pgfpathlineto{\pgfqpoint{0.740437in}{1.003257in}}%
\pgfpathlineto{\pgfqpoint{0.732797in}{1.014668in}}%
\pgfpathlineto{\pgfqpoint{0.730708in}{1.017257in}}%
\pgfpathlineto{\pgfqpoint{0.723165in}{1.026079in}}%
\pgfpathlineto{\pgfqpoint{0.718976in}{1.030747in}}%
\pgfpathlineto{\pgfqpoint{0.712861in}{1.037490in}}%
\pgfpathlineto{\pgfqpoint{0.707244in}{1.044037in}}%
\pgfpathlineto{\pgfqpoint{0.702892in}{1.048901in}}%
\pgfpathlineto{\pgfqpoint{0.695512in}{1.057589in}}%
\pgfpathlineto{\pgfqpoint{0.692366in}{1.060311in}}%
\pgfpathlineto{\pgfqpoint{0.683779in}{1.067782in}}%
\pgfpathlineto{\pgfqpoint{0.678656in}{1.071722in}}%
\pgfpathlineto{\pgfqpoint{0.672047in}{1.077695in}}%
\pgfpathlineto{\pgfqpoint{0.665000in}{1.083133in}}%
\pgfpathlineto{\pgfqpoint{0.660315in}{1.087472in}}%
\pgfpathlineto{\pgfqpoint{0.648583in}{1.094448in}}%
\pgfpathlineto{\pgfqpoint{0.648212in}{1.094544in}}%
\pgfpathlineto{\pgfqpoint{0.636851in}{1.097943in}}%
\pgfpathlineto{\pgfqpoint{0.625118in}{1.101997in}}%
\pgfpathlineto{\pgfqpoint{0.614679in}{1.105954in}}%
\pgfpathlineto{\pgfqpoint{0.613386in}{1.106503in}}%
\pgfpathlineto{\pgfqpoint{0.601654in}{1.112127in}}%
\pgfpathlineto{\pgfqpoint{0.590862in}{1.117365in}}%
\pgfpathlineto{\pgfqpoint{0.589922in}{1.117860in}}%
\pgfpathlineto{\pgfqpoint{0.578190in}{1.124151in}}%
\pgfpathlineto{\pgfqpoint{0.570100in}{1.128776in}}%
\pgfpathlineto{\pgfqpoint{0.566457in}{1.130932in}}%
\pgfpathlineto{\pgfqpoint{0.554725in}{1.137890in}}%
\pgfpathlineto{\pgfqpoint{0.551117in}{1.140187in}}%
\pgfpathlineto{\pgfqpoint{0.542993in}{1.145345in}}%
\pgfpathlineto{\pgfqpoint{0.533892in}{1.151598in}}%
\pgfpathlineto{\pgfqpoint{0.531261in}{1.153420in}}%
\pgfpathlineto{\pgfqpoint{0.519528in}{1.161759in}}%
\pgfpathlineto{\pgfqpoint{0.517879in}{1.163008in}}%
\pgfpathlineto{\pgfqpoint{0.507796in}{1.170412in}}%
\pgfpathlineto{\pgfqpoint{0.502848in}{1.174419in}}%
\pgfpathlineto{\pgfqpoint{0.496064in}{1.180494in}}%
\pgfpathlineto{\pgfqpoint{0.489972in}{1.185830in}}%
\pgfpathlineto{\pgfqpoint{0.484332in}{1.191610in}}%
\pgfpathlineto{\pgfqpoint{0.478977in}{1.197241in}}%
\pgfpathlineto{\pgfqpoint{0.472600in}{1.204289in}}%
\pgfpathlineto{\pgfqpoint{0.468815in}{1.208651in}}%
\pgfpathlineto{\pgfqpoint{0.460867in}{1.217592in}}%
\pgfpathlineto{\pgfqpoint{0.458784in}{1.220062in}}%
\pgfpathlineto{\pgfqpoint{0.449135in}{1.230879in}}%
\pgfpathlineto{\pgfqpoint{0.448634in}{1.231473in}}%
\pgfpathlineto{\pgfqpoint{0.437738in}{1.242884in}}%
\pgfpathlineto{\pgfqpoint{0.437403in}{1.243216in}}%
\pgfpathlineto{\pgfqpoint{0.426536in}{1.254295in}}%
\pgfpathlineto{\pgfqpoint{0.425671in}{1.255146in}}%
\pgfpathlineto{\pgfqpoint{0.415198in}{1.265705in}}%
\pgfpathlineto{\pgfqpoint{0.413939in}{1.266933in}}%
\pgfpathlineto{\pgfqpoint{0.402856in}{1.277116in}}%
\pgfpathlineto{\pgfqpoint{0.402206in}{1.277728in}}%
\pgfpathlineto{\pgfqpoint{0.390474in}{1.288275in}}%
\pgfpathlineto{\pgfqpoint{0.390200in}{1.288527in}}%
\pgfpathlineto{\pgfqpoint{0.378742in}{1.298700in}}%
\pgfpathlineto{\pgfqpoint{0.377365in}{1.299938in}}%
\pgfpathlineto{\pgfqpoint{0.367010in}{1.309231in}}%
\pgfpathlineto{\pgfqpoint{0.364597in}{1.311348in}}%
\pgfpathlineto{\pgfqpoint{0.355278in}{1.320024in}}%
\pgfpathlineto{\pgfqpoint{0.352193in}{1.322759in}}%
\pgfpathlineto{\pgfqpoint{0.343545in}{1.330783in}}%
\pgfpathlineto{\pgfqpoint{0.339807in}{1.334170in}}%
\pgfpathlineto{\pgfqpoint{0.331813in}{1.341454in}}%
\pgfpathlineto{\pgfqpoint{0.327242in}{1.345581in}}%
\pgfpathlineto{\pgfqpoint{0.320081in}{1.352000in}}%
\pgfpathlineto{\pgfqpoint{0.314509in}{1.356992in}}%
\pgfpathlineto{\pgfqpoint{0.308349in}{1.362434in}}%
\pgfpathlineto{\pgfqpoint{0.301570in}{1.368402in}}%
\pgfpathlineto{\pgfqpoint{0.296617in}{1.372814in}}%
\pgfpathlineto{\pgfqpoint{0.288515in}{1.379813in}}%
\pgfpathlineto{\pgfqpoint{0.284884in}{1.383087in}}%
\pgfpathlineto{\pgfqpoint{0.284884in}{1.379813in}}%
\pgfpathlineto{\pgfqpoint{0.284884in}{1.368402in}}%
\pgfpathlineto{\pgfqpoint{0.284884in}{1.363849in}}%
\pgfpathlineto{\pgfqpoint{0.292577in}{1.356992in}}%
\pgfpathlineto{\pgfqpoint{0.296617in}{1.353333in}}%
\pgfpathlineto{\pgfqpoint{0.305369in}{1.345581in}}%
\pgfpathlineto{\pgfqpoint{0.308349in}{1.342886in}}%
\pgfpathlineto{\pgfqpoint{0.317985in}{1.334170in}}%
\pgfpathlineto{\pgfqpoint{0.320081in}{1.332309in}}%
\pgfpathlineto{\pgfqpoint{0.330754in}{1.322759in}}%
\pgfpathlineto{\pgfqpoint{0.331813in}{1.321787in}}%
\pgfpathlineto{\pgfqpoint{0.343545in}{1.311508in}}%
\pgfpathlineto{\pgfqpoint{0.343723in}{1.311348in}}%
\pgfpathlineto{\pgfqpoint{0.355278in}{1.300886in}}%
\pgfpathlineto{\pgfqpoint{0.356356in}{1.299938in}}%
\pgfpathlineto{\pgfqpoint{0.367010in}{1.290012in}}%
\pgfpathlineto{\pgfqpoint{0.368667in}{1.288527in}}%
\pgfpathlineto{\pgfqpoint{0.378742in}{1.278831in}}%
\pgfpathlineto{\pgfqpoint{0.380613in}{1.277116in}}%
\pgfpathlineto{\pgfqpoint{0.390474in}{1.267273in}}%
\pgfpathlineto{\pgfqpoint{0.392140in}{1.265705in}}%
\pgfpathlineto{\pgfqpoint{0.402206in}{1.255238in}}%
\pgfpathlineto{\pgfqpoint{0.403179in}{1.254295in}}%
\pgfpathlineto{\pgfqpoint{0.413574in}{1.242884in}}%
\pgfpathlineto{\pgfqpoint{0.413939in}{1.242282in}}%
\pgfpathlineto{\pgfqpoint{0.421663in}{1.231473in}}%
\pgfpathlineto{\pgfqpoint{0.425671in}{1.223296in}}%
\pgfpathlineto{\pgfqpoint{0.427389in}{1.220062in}}%
\pgfpathlineto{\pgfqpoint{0.428071in}{1.208651in}}%
\pgfpathlineto{\pgfqpoint{0.425671in}{1.204879in}}%
\pgfpathlineto{\pgfqpoint{0.413939in}{1.200775in}}%
\pgfpathlineto{\pgfqpoint{0.402206in}{1.200827in}}%
\pgfpathlineto{\pgfqpoint{0.390474in}{1.201969in}}%
\pgfpathlineto{\pgfqpoint{0.378742in}{1.203649in}}%
\pgfpathlineto{\pgfqpoint{0.367010in}{1.205651in}}%
\pgfpathlineto{\pgfqpoint{0.355278in}{1.207888in}}%
\pgfpathlineto{\pgfqpoint{0.351416in}{1.208651in}}%
\pgfpathlineto{\pgfqpoint{0.343545in}{1.209947in}}%
\pgfpathlineto{\pgfqpoint{0.331813in}{1.211890in}}%
\pgfpathlineto{\pgfqpoint{0.320081in}{1.213878in}}%
\pgfpathlineto{\pgfqpoint{0.308349in}{1.215911in}}%
\pgfpathlineto{\pgfqpoint{0.296617in}{1.217982in}}%
\pgfpathlineto{\pgfqpoint{0.284982in}{1.220062in}}%
\pgfpathlineto{\pgfqpoint{0.284884in}{1.220077in}}%
\pgfpathlineto{\pgfqpoint{0.284884in}{1.220062in}}%
\pgfpathlineto{\pgfqpoint{0.284884in}{1.208651in}}%
\pgfpathlineto{\pgfqpoint{0.284884in}{1.197241in}}%
\pgfpathlineto{\pgfqpoint{0.284884in}{1.185830in}}%
\pgfpathlineto{\pgfqpoint{0.284884in}{1.174419in}}%
\pgfpathlineto{\pgfqpoint{0.284884in}{1.163008in}}%
\pgfpathlineto{\pgfqpoint{0.284884in}{1.151598in}}%
\pgfpathlineto{\pgfqpoint{0.284884in}{1.140187in}}%
\pgfpathlineto{\pgfqpoint{0.284884in}{1.128776in}}%
\pgfpathlineto{\pgfqpoint{0.284884in}{1.117365in}}%
\pgfpathlineto{\pgfqpoint{0.284884in}{1.105954in}}%
\pgfpathlineto{\pgfqpoint{0.284884in}{1.094544in}}%
\pgfpathlineto{\pgfqpoint{0.284884in}{1.083133in}}%
\pgfpathlineto{\pgfqpoint{0.284884in}{1.071722in}}%
\pgfpathlineto{\pgfqpoint{0.284884in}{1.060311in}}%
\pgfpathlineto{\pgfqpoint{0.284884in}{1.048901in}}%
\pgfpathlineto{\pgfqpoint{0.284884in}{1.037490in}}%
\pgfpathlineto{\pgfqpoint{0.284884in}{1.037414in}}%
\pgfpathlineto{\pgfqpoint{0.296617in}{1.027728in}}%
\pgfpathlineto{\pgfqpoint{0.298842in}{1.026079in}}%
\pgfpathlineto{\pgfqpoint{0.308349in}{1.018693in}}%
\pgfpathlineto{\pgfqpoint{0.314322in}{1.014668in}}%
\pgfpathlineto{\pgfqpoint{0.320081in}{1.010637in}}%
\pgfpathlineto{\pgfqpoint{0.331813in}{1.003824in}}%
\pgfpathlineto{\pgfqpoint{0.332995in}{1.003257in}}%
\pgfpathlineto{\pgfqpoint{0.343545in}{0.998124in}}%
\pgfpathlineto{\pgfqpoint{0.355278in}{0.993929in}}%
\pgfpathlineto{\pgfqpoint{0.363716in}{0.991847in}}%
\pgfpathlineto{\pgfqpoint{0.367010in}{0.991044in}}%
\pgfpathlineto{\pgfqpoint{0.378742in}{0.989298in}}%
\pgfpathlineto{\pgfqpoint{0.390474in}{0.988652in}}%
\pgfpathlineto{\pgfqpoint{0.402206in}{0.988952in}}%
\pgfpathlineto{\pgfqpoint{0.413939in}{0.990038in}}%
\pgfpathlineto{\pgfqpoint{0.425671in}{0.991800in}}%
\pgfpathlineto{\pgfqpoint{0.425889in}{0.991847in}}%
\pgfpathlineto{\pgfqpoint{0.437403in}{0.994119in}}%
\pgfpathlineto{\pgfqpoint{0.449135in}{0.997007in}}%
\pgfpathlineto{\pgfqpoint{0.460867in}{1.000423in}}%
\pgfpathlineto{\pgfqpoint{0.469260in}{1.003257in}}%
\pgfpathlineto{\pgfqpoint{0.472600in}{1.004332in}}%
\pgfpathlineto{\pgfqpoint{0.484332in}{1.008742in}}%
\pgfpathlineto{\pgfqpoint{0.496064in}{1.013683in}}%
\pgfpathlineto{\pgfqpoint{0.498130in}{1.014668in}}%
\pgfpathlineto{\pgfqpoint{0.507796in}{1.019300in}}%
\pgfpathlineto{\pgfqpoint{0.519528in}{1.025593in}}%
\pgfpathlineto{\pgfqpoint{0.520040in}{1.026079in}}%
\pgfpathlineto{\pgfqpoint{0.526164in}{1.037490in}}%
\pgfpathlineto{\pgfqpoint{0.527169in}{1.048901in}}%
\pgfpathlineto{\pgfqpoint{0.527506in}{1.060311in}}%
\pgfpathlineto{\pgfqpoint{0.527245in}{1.071722in}}%
\pgfpathlineto{\pgfqpoint{0.526423in}{1.083133in}}%
\pgfpathlineto{\pgfqpoint{0.525242in}{1.094544in}}%
\pgfpathlineto{\pgfqpoint{0.524347in}{1.105954in}}%
\pgfpathlineto{\pgfqpoint{0.527043in}{1.117365in}}%
\pgfpathlineto{\pgfqpoint{0.531261in}{1.119343in}}%
\pgfpathlineto{\pgfqpoint{0.542993in}{1.119096in}}%
\pgfpathlineto{\pgfqpoint{0.549906in}{1.117365in}}%
\pgfpathlineto{\pgfqpoint{0.554725in}{1.116011in}}%
\pgfpathlineto{\pgfqpoint{0.566457in}{1.111335in}}%
\pgfpathlineto{\pgfqpoint{0.578190in}{1.107256in}}%
\pgfpathlineto{\pgfqpoint{0.581299in}{1.105954in}}%
\pgfpathlineto{\pgfqpoint{0.589922in}{1.102053in}}%
\pgfpathlineto{\pgfqpoint{0.601654in}{1.097311in}}%
\pgfpathlineto{\pgfqpoint{0.608306in}{1.094544in}}%
\pgfpathlineto{\pgfqpoint{0.613386in}{1.092306in}}%
\pgfpathlineto{\pgfqpoint{0.625118in}{1.087450in}}%
\pgfpathlineto{\pgfqpoint{0.635649in}{1.083133in}}%
\pgfpathlineto{\pgfqpoint{0.636851in}{1.082634in}}%
\pgfpathlineto{\pgfqpoint{0.648583in}{1.078166in}}%
\pgfpathlineto{\pgfqpoint{0.659208in}{1.071722in}}%
\pgfpathlineto{\pgfqpoint{0.660315in}{1.071107in}}%
\pgfpathlineto{\pgfqpoint{0.672047in}{1.062318in}}%
\pgfpathlineto{\pgfqpoint{0.674515in}{1.060311in}}%
\pgfpathlineto{\pgfqpoint{0.683779in}{1.053505in}}%
\pgfpathlineto{\pgfqpoint{0.688917in}{1.048901in}}%
\pgfpathlineto{\pgfqpoint{0.695512in}{1.042437in}}%
\pgfpathlineto{\pgfqpoint{0.700044in}{1.037490in}}%
\pgfpathlineto{\pgfqpoint{0.707244in}{1.029807in}}%
\pgfpathlineto{\pgfqpoint{0.710687in}{1.026079in}}%
\pgfpathlineto{\pgfqpoint{0.718976in}{1.017084in}}%
\pgfpathlineto{\pgfqpoint{0.721110in}{1.014668in}}%
\pgfpathlineto{\pgfqpoint{0.728721in}{1.003257in}}%
\pgfpathlineto{\pgfqpoint{0.730708in}{1.000208in}}%
\pgfpathlineto{\pgfqpoint{0.736301in}{0.991847in}}%
\pgfpathlineto{\pgfqpoint{0.742440in}{0.983047in}}%
\pgfpathlineto{\pgfqpoint{0.744371in}{0.980436in}}%
\pgfpathlineto{\pgfqpoint{0.752816in}{0.969025in}}%
\pgfpathlineto{\pgfqpoint{0.754173in}{0.967158in}}%
\pgfpathlineto{\pgfqpoint{0.761645in}{0.957614in}}%
\pgfpathlineto{\pgfqpoint{0.765905in}{0.952626in}}%
\pgfpathlineto{\pgfqpoint{0.771499in}{0.946204in}}%
\pgfpathlineto{\pgfqpoint{0.777637in}{0.939851in}}%
\pgfpathlineto{\pgfqpoint{0.781973in}{0.934793in}}%
\pgfpathlineto{\pgfqpoint{0.789369in}{0.929430in}}%
\pgfpathlineto{\pgfqpoint{0.801101in}{0.928302in}}%
\pgfpathlineto{\pgfqpoint{0.812834in}{0.927108in}}%
\pgfpathlineto{\pgfqpoint{0.824566in}{0.926054in}}%
\pgfpathlineto{\pgfqpoint{0.836298in}{0.925778in}}%
\pgfpathlineto{\pgfqpoint{0.844696in}{0.923382in}}%
\pgfpathlineto{\pgfqpoint{0.839970in}{0.911971in}}%
\pgfpathlineto{\pgfqpoint{0.836298in}{0.907024in}}%
\pgfpathlineto{\pgfqpoint{0.832738in}{0.900560in}}%
\pgfpathlineto{\pgfqpoint{0.824566in}{0.891970in}}%
\pgfpathlineto{\pgfqpoint{0.821415in}{0.889150in}}%
\pgfpathlineto{\pgfqpoint{0.812834in}{0.882657in}}%
\pgfpathlineto{\pgfqpoint{0.805530in}{0.877739in}}%
\pgfpathlineto{\pgfqpoint{0.801101in}{0.874523in}}%
\pgfpathlineto{\pgfqpoint{0.789369in}{0.866965in}}%
\pgfpathlineto{\pgfqpoint{0.788319in}{0.866328in}}%
\pgfpathlineto{\pgfqpoint{0.777637in}{0.859578in}}%
\pgfpathlineto{\pgfqpoint{0.771031in}{0.854917in}}%
\pgfpathlineto{\pgfqpoint{0.765905in}{0.851197in}}%
\pgfpathlineto{\pgfqpoint{0.757257in}{0.843507in}}%
\pgfpathlineto{\pgfqpoint{0.754173in}{0.840500in}}%
\pgfpathlineto{\pgfqpoint{0.744998in}{0.832096in}}%
\pgfpathlineto{\pgfqpoint{0.742440in}{0.829543in}}%
\pgfpathlineto{\pgfqpoint{0.732968in}{0.820685in}}%
\pgfpathlineto{\pgfqpoint{0.730708in}{0.818410in}}%
\pgfpathlineto{\pgfqpoint{0.720997in}{0.809274in}}%
\pgfpathlineto{\pgfqpoint{0.718976in}{0.807243in}}%
\pgfpathlineto{\pgfqpoint{0.708913in}{0.797863in}}%
\pgfpathlineto{\pgfqpoint{0.707244in}{0.796205in}}%
\pgfpathlineto{\pgfqpoint{0.697177in}{0.786453in}}%
\pgfpathlineto{\pgfqpoint{0.695512in}{0.784604in}}%
\pgfpathlineto{\pgfqpoint{0.687228in}{0.775042in}}%
\pgfpathlineto{\pgfqpoint{0.683779in}{0.771114in}}%
\pgfpathlineto{\pgfqpoint{0.676926in}{0.763631in}}%
\pgfpathlineto{\pgfqpoint{0.672047in}{0.758316in}}%
\pgfpathlineto{\pgfqpoint{0.666003in}{0.752220in}}%
\pgfpathlineto{\pgfqpoint{0.660315in}{0.746381in}}%
\pgfpathlineto{\pgfqpoint{0.653315in}{0.740810in}}%
\pgfpathlineto{\pgfqpoint{0.648583in}{0.736728in}}%
\pgfpathlineto{\pgfqpoint{0.639083in}{0.729399in}}%
\pgfpathlineto{\pgfqpoint{0.636851in}{0.727513in}}%
\pgfpathlineto{\pgfqpoint{0.625118in}{0.718396in}}%
\pgfpathlineto{\pgfqpoint{0.624616in}{0.717988in}}%
\pgfpathlineto{\pgfqpoint{0.613386in}{0.708485in}}%
\pgfpathlineto{\pgfqpoint{0.610985in}{0.706577in}}%
\pgfpathlineto{\pgfqpoint{0.601654in}{0.698735in}}%
\pgfpathlineto{\pgfqpoint{0.597063in}{0.695166in}}%
\pgfpathlineto{\pgfqpoint{0.589922in}{0.689209in}}%
\pgfpathlineto{\pgfqpoint{0.582758in}{0.683756in}}%
\pgfpathlineto{\pgfqpoint{0.578190in}{0.679975in}}%
\pgfpathlineto{\pgfqpoint{0.567951in}{0.672345in}}%
\pgfpathlineto{\pgfqpoint{0.566457in}{0.671120in}}%
\pgfpathlineto{\pgfqpoint{0.554725in}{0.661888in}}%
\pgfpathlineto{\pgfqpoint{0.553522in}{0.660934in}}%
\pgfpathlineto{\pgfqpoint{0.542993in}{0.652198in}}%
\pgfpathlineto{\pgfqpoint{0.539567in}{0.649523in}}%
\pgfpathlineto{\pgfqpoint{0.531261in}{0.642638in}}%
\pgfpathlineto{\pgfqpoint{0.525370in}{0.638113in}}%
\pgfpathlineto{\pgfqpoint{0.519528in}{0.633279in}}%
\pgfpathlineto{\pgfqpoint{0.510726in}{0.626702in}}%
\pgfpathlineto{\pgfqpoint{0.507796in}{0.624289in}}%
\pgfpathlineto{\pgfqpoint{0.496064in}{0.615551in}}%
\pgfpathlineto{\pgfqpoint{0.495736in}{0.615291in}}%
\pgfpathlineto{\pgfqpoint{0.484332in}{0.605974in}}%
\pgfpathlineto{\pgfqpoint{0.481652in}{0.603880in}}%
\pgfpathlineto{\pgfqpoint{0.472600in}{0.596289in}}%
\pgfpathlineto{\pgfqpoint{0.467619in}{0.592469in}}%
\pgfpathlineto{\pgfqpoint{0.460867in}{0.586604in}}%
\pgfpathlineto{\pgfqpoint{0.453950in}{0.581059in}}%
\pgfpathlineto{\pgfqpoint{0.449135in}{0.576896in}}%
\pgfpathlineto{\pgfqpoint{0.440476in}{0.569648in}}%
\pgfpathlineto{\pgfqpoint{0.437403in}{0.566989in}}%
\pgfpathlineto{\pgfqpoint{0.426923in}{0.558237in}}%
\pgfpathlineto{\pgfqpoint{0.425671in}{0.557152in}}%
\pgfpathlineto{\pgfqpoint{0.413939in}{0.547225in}}%
\pgfpathlineto{\pgfqpoint{0.413481in}{0.546826in}}%
\pgfpathlineto{\pgfqpoint{0.402206in}{0.536971in}}%
\pgfpathlineto{\pgfqpoint{0.400419in}{0.535416in}}%
\pgfpathlineto{\pgfqpoint{0.390474in}{0.526695in}}%
\pgfpathlineto{\pgfqpoint{0.387383in}{0.524005in}}%
\pgfpathlineto{\pgfqpoint{0.378742in}{0.516402in}}%
\pgfpathlineto{\pgfqpoint{0.374365in}{0.512594in}}%
\pgfpathlineto{\pgfqpoint{0.367010in}{0.506097in}}%
\pgfpathlineto{\pgfqpoint{0.361305in}{0.501183in}}%
\pgfpathlineto{\pgfqpoint{0.355278in}{0.495841in}}%
\pgfpathlineto{\pgfqpoint{0.348053in}{0.489772in}}%
\pgfpathlineto{\pgfqpoint{0.343545in}{0.485774in}}%
\pgfpathlineto{\pgfqpoint{0.334675in}{0.478362in}}%
\pgfpathlineto{\pgfqpoint{0.331813in}{0.475866in}}%
\pgfpathlineto{\pgfqpoint{0.321123in}{0.466951in}}%
\pgfpathlineto{\pgfqpoint{0.320081in}{0.466041in}}%
\pgfpathlineto{\pgfqpoint{0.308349in}{0.456111in}}%
\pgfpathlineto{\pgfqpoint{0.307687in}{0.455540in}}%
\pgfpathlineto{\pgfqpoint{0.296617in}{0.445935in}}%
\pgfpathlineto{\pgfqpoint{0.294521in}{0.444129in}}%
\pgfpathlineto{\pgfqpoint{0.284884in}{0.435724in}}%
\pgfpathlineto{\pgfqpoint{0.284884in}{0.432719in}}%
\pgfpathlineto{\pgfqpoint{0.284884in}{0.422985in}}%
\pgfpathlineto{\pgfqpoint{0.295863in}{0.432719in}}%
\pgfpathlineto{\pgfqpoint{0.296617in}{0.433346in}}%
\pgfpathlineto{\pgfqpoint{0.308349in}{0.443315in}}%
\pgfpathlineto{\pgfqpoint{0.309274in}{0.444129in}}%
\pgfpathlineto{\pgfqpoint{0.320081in}{0.452787in}}%
\pgfpathlineto{\pgfqpoint{0.323201in}{0.455540in}}%
\pgfpathlineto{\pgfqpoint{0.331813in}{0.462440in}}%
\pgfpathlineto{\pgfqpoint{0.336911in}{0.466951in}}%
\pgfpathlineto{\pgfqpoint{0.343545in}{0.472303in}}%
\pgfpathlineto{\pgfqpoint{0.350407in}{0.478362in}}%
\pgfpathlineto{\pgfqpoint{0.355278in}{0.482423in}}%
\pgfpathlineto{\pgfqpoint{0.363638in}{0.489772in}}%
\pgfpathlineto{\pgfqpoint{0.367010in}{0.492579in}}%
\pgfpathlineto{\pgfqpoint{0.376833in}{0.501183in}}%
\pgfpathlineto{\pgfqpoint{0.378742in}{0.502769in}}%
\pgfpathlineto{\pgfqpoint{0.389993in}{0.512594in}}%
\pgfpathlineto{\pgfqpoint{0.390474in}{0.512994in}}%
\pgfpathlineto{\pgfqpoint{0.402206in}{0.522374in}}%
\pgfpathlineto{\pgfqpoint{0.404178in}{0.524005in}}%
\pgfpathlineto{\pgfqpoint{0.413939in}{0.531489in}}%
\pgfpathlineto{\pgfqpoint{0.418686in}{0.535416in}}%
\pgfpathlineto{\pgfqpoint{0.425671in}{0.540793in}}%
\pgfpathlineto{\pgfqpoint{0.432956in}{0.546826in}}%
\pgfpathlineto{\pgfqpoint{0.437403in}{0.550266in}}%
\pgfpathlineto{\pgfqpoint{0.447013in}{0.558237in}}%
\pgfpathlineto{\pgfqpoint{0.449135in}{0.559887in}}%
\pgfpathlineto{\pgfqpoint{0.460867in}{0.569624in}}%
\pgfpathlineto{\pgfqpoint{0.460900in}{0.569648in}}%
\pgfpathlineto{\pgfqpoint{0.472600in}{0.577430in}}%
\pgfpathlineto{\pgfqpoint{0.477544in}{0.581059in}}%
\pgfpathlineto{\pgfqpoint{0.484332in}{0.585651in}}%
\pgfpathlineto{\pgfqpoint{0.493484in}{0.592469in}}%
\pgfpathlineto{\pgfqpoint{0.496064in}{0.594245in}}%
\pgfpathlineto{\pgfqpoint{0.507796in}{0.601944in}}%
\pgfpathlineto{\pgfqpoint{0.511792in}{0.603880in}}%
\pgfpathlineto{\pgfqpoint{0.519528in}{0.607307in}}%
\pgfpathlineto{\pgfqpoint{0.531261in}{0.607570in}}%
\pgfpathlineto{\pgfqpoint{0.533119in}{0.603880in}}%
\pgfpathlineto{\pgfqpoint{0.531261in}{0.595015in}}%
\pgfpathlineto{\pgfqpoint{0.530744in}{0.592469in}}%
\pgfpathlineto{\pgfqpoint{0.525835in}{0.581059in}}%
\pgfpathlineto{\pgfqpoint{0.520146in}{0.569648in}}%
\pgfpathlineto{\pgfqpoint{0.519528in}{0.568596in}}%
\pgfpathlineto{\pgfqpoint{0.513637in}{0.558237in}}%
\pgfpathlineto{\pgfqpoint{0.507796in}{0.548761in}}%
\pgfpathlineto{\pgfqpoint{0.506627in}{0.546826in}}%
\pgfpathlineto{\pgfqpoint{0.499007in}{0.535416in}}%
\pgfpathlineto{\pgfqpoint{0.496064in}{0.531047in}}%
\pgfpathlineto{\pgfqpoint{0.491197in}{0.524005in}}%
\pgfpathlineto{\pgfqpoint{0.484332in}{0.513466in}}%
\pgfpathlineto{\pgfqpoint{0.483726in}{0.512594in}}%
\pgfpathlineto{\pgfqpoint{0.476350in}{0.501183in}}%
\pgfpathlineto{\pgfqpoint{0.472600in}{0.495631in}}%
\pgfpathlineto{\pgfqpoint{0.468671in}{0.489772in}}%
\pgfpathlineto{\pgfqpoint{0.460867in}{0.478504in}}%
\pgfpathlineto{\pgfqpoint{0.460769in}{0.478362in}}%
\pgfpathlineto{\pgfqpoint{0.452647in}{0.466951in}}%
\pgfpathlineto{\pgfqpoint{0.449135in}{0.462138in}}%
\pgfpathlineto{\pgfqpoint{0.444306in}{0.455540in}}%
\pgfpathlineto{\pgfqpoint{0.437403in}{0.446263in}}%
\pgfpathlineto{\pgfqpoint{0.435798in}{0.444129in}}%
\pgfpathlineto{\pgfqpoint{0.427232in}{0.432719in}}%
\pgfpathlineto{\pgfqpoint{0.425671in}{0.430502in}}%
\pgfpathlineto{\pgfqpoint{0.418991in}{0.421308in}}%
\pgfpathlineto{\pgfqpoint{0.413939in}{0.414240in}}%
\pgfpathlineto{\pgfqpoint{0.410770in}{0.409897in}}%
\pgfpathlineto{\pgfqpoint{0.402414in}{0.398486in}}%
\pgfpathlineto{\pgfqpoint{0.402206in}{0.398203in}}%
\pgfpathlineto{\pgfqpoint{0.393853in}{0.387075in}}%
\pgfpathlineto{\pgfqpoint{0.390474in}{0.382570in}}%
\pgfpathlineto{\pgfqpoint{0.385148in}{0.375665in}}%
\pgfpathlineto{\pgfqpoint{0.378742in}{0.367316in}}%
\pgfpathlineto{\pgfqpoint{0.376316in}{0.364254in}}%
\pgfpathlineto{\pgfqpoint{0.367354in}{0.352843in}}%
\pgfpathlineto{\pgfqpoint{0.367010in}{0.352398in}}%
\pgfpathlineto{\pgfqpoint{0.358221in}{0.341432in}}%
\pgfpathlineto{\pgfqpoint{0.355278in}{0.337715in}}%
\pgfpathlineto{\pgfqpoint{0.348951in}{0.330022in}}%
\pgfpathlineto{\pgfqpoint{0.343545in}{0.323363in}}%
\pgfpathlineto{\pgfqpoint{0.339507in}{0.318611in}}%
\pgfpathlineto{\pgfqpoint{0.331813in}{0.309364in}}%
\pgfpathlineto{\pgfqpoint{0.329924in}{0.307200in}}%
\pgfpathlineto{\pgfqpoint{0.320105in}{0.295789in}}%
\pgfpathlineto{\pgfqpoint{0.320081in}{0.295761in}}%
\pgfpathlineto{\pgfqpoint{0.310030in}{0.284378in}}%
\pgfpathclose%
\pgfpathmoveto{\pgfqpoint{0.997478in}{1.105954in}}%
\pgfpathlineto{\pgfqpoint{0.988817in}{1.110632in}}%
\pgfpathlineto{\pgfqpoint{0.977085in}{1.115182in}}%
\pgfpathlineto{\pgfqpoint{0.971628in}{1.117365in}}%
\pgfpathlineto{\pgfqpoint{0.977085in}{1.122458in}}%
\pgfpathlineto{\pgfqpoint{0.988817in}{1.123577in}}%
\pgfpathlineto{\pgfqpoint{1.000549in}{1.123845in}}%
\pgfpathlineto{\pgfqpoint{1.012281in}{1.120794in}}%
\pgfpathlineto{\pgfqpoint{1.020510in}{1.117365in}}%
\pgfpathlineto{\pgfqpoint{1.013197in}{1.105954in}}%
\pgfpathlineto{\pgfqpoint{1.012281in}{1.105874in}}%
\pgfpathlineto{\pgfqpoint{1.000549in}{1.105534in}}%
\pgfpathclose%
\pgfpathmoveto{\pgfqpoint{0.810529in}{1.140187in}}%
\pgfpathlineto{\pgfqpoint{0.805270in}{1.151598in}}%
\pgfpathlineto{\pgfqpoint{0.801101in}{1.154019in}}%
\pgfpathlineto{\pgfqpoint{0.797909in}{1.163008in}}%
\pgfpathlineto{\pgfqpoint{0.797452in}{1.174419in}}%
\pgfpathlineto{\pgfqpoint{0.799953in}{1.185830in}}%
\pgfpathlineto{\pgfqpoint{0.801101in}{1.185946in}}%
\pgfpathlineto{\pgfqpoint{0.803446in}{1.185830in}}%
\pgfpathlineto{\pgfqpoint{0.812834in}{1.185090in}}%
\pgfpathlineto{\pgfqpoint{0.824566in}{1.184580in}}%
\pgfpathlineto{\pgfqpoint{0.836298in}{1.183604in}}%
\pgfpathlineto{\pgfqpoint{0.848030in}{1.181809in}}%
\pgfpathlineto{\pgfqpoint{0.859762in}{1.180148in}}%
\pgfpathlineto{\pgfqpoint{0.871495in}{1.178362in}}%
\pgfpathlineto{\pgfqpoint{0.883227in}{1.176525in}}%
\pgfpathlineto{\pgfqpoint{0.894959in}{1.174692in}}%
\pgfpathlineto{\pgfqpoint{0.896801in}{1.174419in}}%
\pgfpathlineto{\pgfqpoint{0.906691in}{1.172753in}}%
\pgfpathlineto{\pgfqpoint{0.918424in}{1.170793in}}%
\pgfpathlineto{\pgfqpoint{0.930156in}{1.168852in}}%
\pgfpathlineto{\pgfqpoint{0.941888in}{1.164436in}}%
\pgfpathlineto{\pgfqpoint{0.953620in}{1.163250in}}%
\pgfpathlineto{\pgfqpoint{0.954104in}{1.163008in}}%
\pgfpathlineto{\pgfqpoint{0.965352in}{1.153385in}}%
\pgfpathlineto{\pgfqpoint{0.971287in}{1.151598in}}%
\pgfpathlineto{\pgfqpoint{0.977085in}{1.143790in}}%
\pgfpathlineto{\pgfqpoint{0.984036in}{1.140187in}}%
\pgfpathlineto{\pgfqpoint{0.977085in}{1.137742in}}%
\pgfpathlineto{\pgfqpoint{0.965352in}{1.134494in}}%
\pgfpathlineto{\pgfqpoint{0.953620in}{1.133036in}}%
\pgfpathlineto{\pgfqpoint{0.941888in}{1.133375in}}%
\pgfpathlineto{\pgfqpoint{0.930156in}{1.134436in}}%
\pgfpathlineto{\pgfqpoint{0.918424in}{1.135257in}}%
\pgfpathlineto{\pgfqpoint{0.906691in}{1.136539in}}%
\pgfpathlineto{\pgfqpoint{0.894959in}{1.137954in}}%
\pgfpathlineto{\pgfqpoint{0.883227in}{1.139260in}}%
\pgfpathlineto{\pgfqpoint{0.874002in}{1.140187in}}%
\pgfpathlineto{\pgfqpoint{0.871495in}{1.140823in}}%
\pgfpathlineto{\pgfqpoint{0.859762in}{1.143600in}}%
\pgfpathlineto{\pgfqpoint{0.848030in}{1.145988in}}%
\pgfpathlineto{\pgfqpoint{0.836298in}{1.148039in}}%
\pgfpathlineto{\pgfqpoint{0.824566in}{1.149801in}}%
\pgfpathlineto{\pgfqpoint{0.817255in}{1.140187in}}%
\pgfpathlineto{\pgfqpoint{0.812834in}{1.136461in}}%
\pgfpathclose%
\pgfpathmoveto{\pgfqpoint{0.788078in}{1.197241in}}%
\pgfpathlineto{\pgfqpoint{0.789369in}{1.200355in}}%
\pgfpathlineto{\pgfqpoint{0.794025in}{1.197241in}}%
\pgfpathlineto{\pgfqpoint{0.789369in}{1.187267in}}%
\pgfpathclose%
\pgfpathmoveto{\pgfqpoint{0.729382in}{1.277116in}}%
\pgfpathlineto{\pgfqpoint{0.730708in}{1.282257in}}%
\pgfpathlineto{\pgfqpoint{0.731864in}{1.277116in}}%
\pgfpathlineto{\pgfqpoint{0.730708in}{1.276012in}}%
\pgfpathclose%
\pgfpathmoveto{\pgfqpoint{0.713303in}{1.288527in}}%
\pgfpathlineto{\pgfqpoint{0.707244in}{1.293951in}}%
\pgfpathlineto{\pgfqpoint{0.700633in}{1.299938in}}%
\pgfpathlineto{\pgfqpoint{0.695512in}{1.304633in}}%
\pgfpathlineto{\pgfqpoint{0.688264in}{1.311348in}}%
\pgfpathlineto{\pgfqpoint{0.683779in}{1.315564in}}%
\pgfpathlineto{\pgfqpoint{0.676196in}{1.322759in}}%
\pgfpathlineto{\pgfqpoint{0.672047in}{1.327544in}}%
\pgfpathlineto{\pgfqpoint{0.665227in}{1.334170in}}%
\pgfpathlineto{\pgfqpoint{0.660315in}{1.340688in}}%
\pgfpathlineto{\pgfqpoint{0.652924in}{1.345581in}}%
\pgfpathlineto{\pgfqpoint{0.650212in}{1.356992in}}%
\pgfpathlineto{\pgfqpoint{0.648583in}{1.362244in}}%
\pgfpathlineto{\pgfqpoint{0.646753in}{1.368402in}}%
\pgfpathlineto{\pgfqpoint{0.644971in}{1.379813in}}%
\pgfpathlineto{\pgfqpoint{0.646516in}{1.391224in}}%
\pgfpathlineto{\pgfqpoint{0.648583in}{1.392085in}}%
\pgfpathlineto{\pgfqpoint{0.649297in}{1.391224in}}%
\pgfpathlineto{\pgfqpoint{0.655610in}{1.379813in}}%
\pgfpathlineto{\pgfqpoint{0.660315in}{1.378043in}}%
\pgfpathlineto{\pgfqpoint{0.667780in}{1.368402in}}%
\pgfpathlineto{\pgfqpoint{0.672047in}{1.364620in}}%
\pgfpathlineto{\pgfqpoint{0.678366in}{1.356992in}}%
\pgfpathlineto{\pgfqpoint{0.683779in}{1.350960in}}%
\pgfpathlineto{\pgfqpoint{0.688340in}{1.345581in}}%
\pgfpathlineto{\pgfqpoint{0.695512in}{1.337089in}}%
\pgfpathlineto{\pgfqpoint{0.697848in}{1.334170in}}%
\pgfpathlineto{\pgfqpoint{0.706503in}{1.322759in}}%
\pgfpathlineto{\pgfqpoint{0.707244in}{1.321769in}}%
\pgfpathlineto{\pgfqpoint{0.714714in}{1.311348in}}%
\pgfpathlineto{\pgfqpoint{0.718976in}{1.305512in}}%
\pgfpathlineto{\pgfqpoint{0.722898in}{1.299938in}}%
\pgfpathlineto{\pgfqpoint{0.728776in}{1.288527in}}%
\pgfpathlineto{\pgfqpoint{0.718976in}{1.283498in}}%
\pgfpathclose%
\pgfpathmoveto{\pgfqpoint{0.634516in}{1.402635in}}%
\pgfpathlineto{\pgfqpoint{0.636851in}{1.405154in}}%
\pgfpathlineto{\pgfqpoint{0.639113in}{1.402635in}}%
\pgfpathlineto{\pgfqpoint{0.636851in}{1.398242in}}%
\pgfpathclose%
\pgfusepath{fill}%
\end{pgfscope}%
\begin{pgfscope}%
\pgfpathrectangle{\pgfqpoint{0.211875in}{0.211875in}}{\pgfqpoint{1.313625in}{1.279725in}}%
\pgfusepath{clip}%
\pgfsetbuttcap%
\pgfsetroundjoin%
\definecolor{currentfill}{rgb}{0.584229,0.109227,0.358485}%
\pgfsetfillcolor{currentfill}%
\pgfsetlinewidth{0.000000pt}%
\definecolor{currentstroke}{rgb}{0.000000,0.000000,0.000000}%
\pgfsetstrokecolor{currentstroke}%
\pgfsetdash{}{0pt}%
\pgfpathmoveto{\pgfqpoint{1.399444in}{0.352688in}}%
\pgfpathlineto{\pgfqpoint{1.411176in}{0.351542in}}%
\pgfpathlineto{\pgfqpoint{1.422908in}{0.352064in}}%
\pgfpathlineto{\pgfqpoint{1.434641in}{0.349649in}}%
\pgfpathlineto{\pgfqpoint{1.446373in}{0.344195in}}%
\pgfpathlineto{\pgfqpoint{1.446373in}{0.352843in}}%
\pgfpathlineto{\pgfqpoint{1.446373in}{0.364254in}}%
\pgfpathlineto{\pgfqpoint{1.446373in}{0.375665in}}%
\pgfpathlineto{\pgfqpoint{1.446373in}{0.387075in}}%
\pgfpathlineto{\pgfqpoint{1.446373in}{0.398486in}}%
\pgfpathlineto{\pgfqpoint{1.446373in}{0.409897in}}%
\pgfpathlineto{\pgfqpoint{1.446373in}{0.421308in}}%
\pgfpathlineto{\pgfqpoint{1.446373in}{0.432719in}}%
\pgfpathlineto{\pgfqpoint{1.446373in}{0.444129in}}%
\pgfpathlineto{\pgfqpoint{1.446373in}{0.455540in}}%
\pgfpathlineto{\pgfqpoint{1.446373in}{0.466951in}}%
\pgfpathlineto{\pgfqpoint{1.446373in}{0.478362in}}%
\pgfpathlineto{\pgfqpoint{1.446373in}{0.489772in}}%
\pgfpathlineto{\pgfqpoint{1.446373in}{0.501183in}}%
\pgfpathlineto{\pgfqpoint{1.446373in}{0.512594in}}%
\pgfpathlineto{\pgfqpoint{1.446373in}{0.524005in}}%
\pgfpathlineto{\pgfqpoint{1.446373in}{0.535416in}}%
\pgfpathlineto{\pgfqpoint{1.446373in}{0.546826in}}%
\pgfpathlineto{\pgfqpoint{1.446373in}{0.558237in}}%
\pgfpathlineto{\pgfqpoint{1.446373in}{0.569648in}}%
\pgfpathlineto{\pgfqpoint{1.446373in}{0.581059in}}%
\pgfpathlineto{\pgfqpoint{1.446373in}{0.592469in}}%
\pgfpathlineto{\pgfqpoint{1.446373in}{0.603880in}}%
\pgfpathlineto{\pgfqpoint{1.446373in}{0.615291in}}%
\pgfpathlineto{\pgfqpoint{1.446373in}{0.626702in}}%
\pgfpathlineto{\pgfqpoint{1.446373in}{0.638113in}}%
\pgfpathlineto{\pgfqpoint{1.446373in}{0.649523in}}%
\pgfpathlineto{\pgfqpoint{1.446373in}{0.660934in}}%
\pgfpathlineto{\pgfqpoint{1.446373in}{0.672345in}}%
\pgfpathlineto{\pgfqpoint{1.446373in}{0.683756in}}%
\pgfpathlineto{\pgfqpoint{1.446373in}{0.695166in}}%
\pgfpathlineto{\pgfqpoint{1.446373in}{0.706577in}}%
\pgfpathlineto{\pgfqpoint{1.446373in}{0.717988in}}%
\pgfpathlineto{\pgfqpoint{1.446373in}{0.718654in}}%
\pgfpathlineto{\pgfqpoint{1.441384in}{0.729399in}}%
\pgfpathlineto{\pgfqpoint{1.437283in}{0.740810in}}%
\pgfpathlineto{\pgfqpoint{1.434641in}{0.748236in}}%
\pgfpathlineto{\pgfqpoint{1.431335in}{0.752220in}}%
\pgfpathlineto{\pgfqpoint{1.423889in}{0.763631in}}%
\pgfpathlineto{\pgfqpoint{1.422908in}{0.765982in}}%
\pgfpathlineto{\pgfqpoint{1.414008in}{0.775042in}}%
\pgfpathlineto{\pgfqpoint{1.411176in}{0.778088in}}%
\pgfpathlineto{\pgfqpoint{1.399444in}{0.784511in}}%
\pgfpathlineto{\pgfqpoint{1.387712in}{0.781736in}}%
\pgfpathlineto{\pgfqpoint{1.382980in}{0.775042in}}%
\pgfpathlineto{\pgfqpoint{1.377775in}{0.763631in}}%
\pgfpathlineto{\pgfqpoint{1.376493in}{0.752220in}}%
\pgfpathlineto{\pgfqpoint{1.375980in}{0.747262in}}%
\pgfpathlineto{\pgfqpoint{1.375325in}{0.740810in}}%
\pgfpathlineto{\pgfqpoint{1.364247in}{0.730801in}}%
\pgfpathlineto{\pgfqpoint{1.362788in}{0.729399in}}%
\pgfpathlineto{\pgfqpoint{1.352515in}{0.718777in}}%
\pgfpathlineto{\pgfqpoint{1.352029in}{0.717988in}}%
\pgfpathlineto{\pgfqpoint{1.345038in}{0.706577in}}%
\pgfpathlineto{\pgfqpoint{1.340783in}{0.698154in}}%
\pgfpathlineto{\pgfqpoint{1.335284in}{0.706577in}}%
\pgfpathlineto{\pgfqpoint{1.329051in}{0.716336in}}%
\pgfpathlineto{\pgfqpoint{1.328413in}{0.717988in}}%
\pgfpathlineto{\pgfqpoint{1.322077in}{0.729399in}}%
\pgfpathlineto{\pgfqpoint{1.317319in}{0.736900in}}%
\pgfpathlineto{\pgfqpoint{1.312711in}{0.740810in}}%
\pgfpathlineto{\pgfqpoint{1.305586in}{0.746639in}}%
\pgfpathlineto{\pgfqpoint{1.298852in}{0.752220in}}%
\pgfpathlineto{\pgfqpoint{1.293854in}{0.756236in}}%
\pgfpathlineto{\pgfqpoint{1.282122in}{0.759371in}}%
\pgfpathlineto{\pgfqpoint{1.270390in}{0.758723in}}%
\pgfpathlineto{\pgfqpoint{1.258903in}{0.752220in}}%
\pgfpathlineto{\pgfqpoint{1.258658in}{0.751827in}}%
\pgfpathlineto{\pgfqpoint{1.252258in}{0.740810in}}%
\pgfpathlineto{\pgfqpoint{1.247528in}{0.729399in}}%
\pgfpathlineto{\pgfqpoint{1.246925in}{0.727689in}}%
\pgfpathlineto{\pgfqpoint{1.243486in}{0.717988in}}%
\pgfpathlineto{\pgfqpoint{1.240412in}{0.706577in}}%
\pgfpathlineto{\pgfqpoint{1.239296in}{0.695166in}}%
\pgfpathlineto{\pgfqpoint{1.239334in}{0.683756in}}%
\pgfpathlineto{\pgfqpoint{1.239753in}{0.672345in}}%
\pgfpathlineto{\pgfqpoint{1.240691in}{0.660934in}}%
\pgfpathlineto{\pgfqpoint{1.242163in}{0.649523in}}%
\pgfpathlineto{\pgfqpoint{1.244178in}{0.638113in}}%
\pgfpathlineto{\pgfqpoint{1.246745in}{0.626702in}}%
\pgfpathlineto{\pgfqpoint{1.246925in}{0.626029in}}%
\pgfpathlineto{\pgfqpoint{1.249882in}{0.615291in}}%
\pgfpathlineto{\pgfqpoint{1.253580in}{0.603880in}}%
\pgfpathlineto{\pgfqpoint{1.257839in}{0.592469in}}%
\pgfpathlineto{\pgfqpoint{1.258658in}{0.590498in}}%
\pgfpathlineto{\pgfqpoint{1.262678in}{0.581059in}}%
\pgfpathlineto{\pgfqpoint{1.268082in}{0.569648in}}%
\pgfpathlineto{\pgfqpoint{1.270390in}{0.565188in}}%
\pgfpathlineto{\pgfqpoint{1.274073in}{0.558237in}}%
\pgfpathlineto{\pgfqpoint{1.280904in}{0.546826in}}%
\pgfpathlineto{\pgfqpoint{1.282122in}{0.544922in}}%
\pgfpathlineto{\pgfqpoint{1.288358in}{0.535416in}}%
\pgfpathlineto{\pgfqpoint{1.293854in}{0.527569in}}%
\pgfpathlineto{\pgfqpoint{1.296414in}{0.524005in}}%
\pgfpathlineto{\pgfqpoint{1.305079in}{0.512594in}}%
\pgfpathlineto{\pgfqpoint{1.305586in}{0.511955in}}%
\pgfpathlineto{\pgfqpoint{1.314591in}{0.501183in}}%
\pgfpathlineto{\pgfqpoint{1.317319in}{0.498063in}}%
\pgfpathlineto{\pgfqpoint{1.329051in}{0.500251in}}%
\pgfpathlineto{\pgfqpoint{1.334386in}{0.489772in}}%
\pgfpathlineto{\pgfqpoint{1.340295in}{0.478362in}}%
\pgfpathlineto{\pgfqpoint{1.340783in}{0.477436in}}%
\pgfpathlineto{\pgfqpoint{1.342896in}{0.466951in}}%
\pgfpathlineto{\pgfqpoint{1.345285in}{0.455540in}}%
\pgfpathlineto{\pgfqpoint{1.347762in}{0.444129in}}%
\pgfpathlineto{\pgfqpoint{1.350325in}{0.432719in}}%
\pgfpathlineto{\pgfqpoint{1.352515in}{0.423258in}}%
\pgfpathlineto{\pgfqpoint{1.353190in}{0.421308in}}%
\pgfpathlineto{\pgfqpoint{1.357209in}{0.409897in}}%
\pgfpathlineto{\pgfqpoint{1.361321in}{0.398486in}}%
\pgfpathlineto{\pgfqpoint{1.364247in}{0.390517in}}%
\pgfpathlineto{\pgfqpoint{1.366223in}{0.387075in}}%
\pgfpathlineto{\pgfqpoint{1.372808in}{0.375665in}}%
\pgfpathlineto{\pgfqpoint{1.375980in}{0.370194in}}%
\pgfpathlineto{\pgfqpoint{1.381874in}{0.364254in}}%
\pgfpathlineto{\pgfqpoint{1.387712in}{0.358306in}}%
\pgfpathlineto{\pgfqpoint{1.399126in}{0.352843in}}%
\pgfpathclose%
\pgfpathmoveto{\pgfqpoint{1.375848in}{0.501183in}}%
\pgfpathlineto{\pgfqpoint{1.370794in}{0.512594in}}%
\pgfpathlineto{\pgfqpoint{1.364707in}{0.524005in}}%
\pgfpathlineto{\pgfqpoint{1.364247in}{0.524743in}}%
\pgfpathlineto{\pgfqpoint{1.361258in}{0.535416in}}%
\pgfpathlineto{\pgfqpoint{1.358669in}{0.546826in}}%
\pgfpathlineto{\pgfqpoint{1.356339in}{0.558237in}}%
\pgfpathlineto{\pgfqpoint{1.353457in}{0.569648in}}%
\pgfpathlineto{\pgfqpoint{1.352515in}{0.572783in}}%
\pgfpathlineto{\pgfqpoint{1.351442in}{0.581059in}}%
\pgfpathlineto{\pgfqpoint{1.351252in}{0.592469in}}%
\pgfpathlineto{\pgfqpoint{1.351461in}{0.603880in}}%
\pgfpathlineto{\pgfqpoint{1.352180in}{0.615291in}}%
\pgfpathlineto{\pgfqpoint{1.352515in}{0.617214in}}%
\pgfpathlineto{\pgfqpoint{1.355241in}{0.626702in}}%
\pgfpathlineto{\pgfqpoint{1.359001in}{0.638113in}}%
\pgfpathlineto{\pgfqpoint{1.363182in}{0.649523in}}%
\pgfpathlineto{\pgfqpoint{1.364247in}{0.652061in}}%
\pgfpathlineto{\pgfqpoint{1.370538in}{0.660934in}}%
\pgfpathlineto{\pgfqpoint{1.375980in}{0.667981in}}%
\pgfpathlineto{\pgfqpoint{1.381774in}{0.672345in}}%
\pgfpathlineto{\pgfqpoint{1.387712in}{0.677807in}}%
\pgfpathlineto{\pgfqpoint{1.396698in}{0.672345in}}%
\pgfpathlineto{\pgfqpoint{1.399444in}{0.670702in}}%
\pgfpathlineto{\pgfqpoint{1.405761in}{0.660934in}}%
\pgfpathlineto{\pgfqpoint{1.411176in}{0.652871in}}%
\pgfpathlineto{\pgfqpoint{1.412769in}{0.649523in}}%
\pgfpathlineto{\pgfqpoint{1.418057in}{0.638113in}}%
\pgfpathlineto{\pgfqpoint{1.422908in}{0.627636in}}%
\pgfpathlineto{\pgfqpoint{1.423270in}{0.626702in}}%
\pgfpathlineto{\pgfqpoint{1.427008in}{0.615291in}}%
\pgfpathlineto{\pgfqpoint{1.430766in}{0.603880in}}%
\pgfpathlineto{\pgfqpoint{1.434562in}{0.592469in}}%
\pgfpathlineto{\pgfqpoint{1.434641in}{0.592200in}}%
\pgfpathlineto{\pgfqpoint{1.437559in}{0.581059in}}%
\pgfpathlineto{\pgfqpoint{1.440353in}{0.569648in}}%
\pgfpathlineto{\pgfqpoint{1.442632in}{0.558237in}}%
\pgfpathlineto{\pgfqpoint{1.444120in}{0.546826in}}%
\pgfpathlineto{\pgfqpoint{1.443917in}{0.535416in}}%
\pgfpathlineto{\pgfqpoint{1.441488in}{0.524005in}}%
\pgfpathlineto{\pgfqpoint{1.435338in}{0.512594in}}%
\pgfpathlineto{\pgfqpoint{1.434641in}{0.511797in}}%
\pgfpathlineto{\pgfqpoint{1.422908in}{0.502594in}}%
\pgfpathlineto{\pgfqpoint{1.420668in}{0.501183in}}%
\pgfpathlineto{\pgfqpoint{1.411176in}{0.496416in}}%
\pgfpathlineto{\pgfqpoint{1.399444in}{0.492321in}}%
\pgfpathlineto{\pgfqpoint{1.387712in}{0.493470in}}%
\pgfpathlineto{\pgfqpoint{1.375980in}{0.500927in}}%
\pgfpathclose%
\pgfusepath{fill}%
\end{pgfscope}%
\begin{pgfscope}%
\pgfpathrectangle{\pgfqpoint{0.211875in}{0.211875in}}{\pgfqpoint{1.313625in}{1.279725in}}%
\pgfusepath{clip}%
\pgfsetbuttcap%
\pgfsetroundjoin%
\definecolor{currentfill}{rgb}{0.584229,0.109227,0.358485}%
\pgfsetfillcolor{currentfill}%
\pgfsetlinewidth{0.000000pt}%
\definecolor{currentstroke}{rgb}{0.000000,0.000000,0.000000}%
\pgfsetstrokecolor{currentstroke}%
\pgfsetdash{}{0pt}%
\pgfpathmoveto{\pgfqpoint{1.094407in}{0.418210in}}%
\pgfpathlineto{\pgfqpoint{1.098941in}{0.421308in}}%
\pgfpathlineto{\pgfqpoint{1.101787in}{0.432719in}}%
\pgfpathlineto{\pgfqpoint{1.103733in}{0.444129in}}%
\pgfpathlineto{\pgfqpoint{1.104849in}{0.455540in}}%
\pgfpathlineto{\pgfqpoint{1.105206in}{0.466951in}}%
\pgfpathlineto{\pgfqpoint{1.104876in}{0.478362in}}%
\pgfpathlineto{\pgfqpoint{1.103929in}{0.489772in}}%
\pgfpathlineto{\pgfqpoint{1.102434in}{0.501183in}}%
\pgfpathlineto{\pgfqpoint{1.100460in}{0.512594in}}%
\pgfpathlineto{\pgfqpoint{1.098354in}{0.524005in}}%
\pgfpathlineto{\pgfqpoint{1.095970in}{0.535416in}}%
\pgfpathlineto{\pgfqpoint{1.094407in}{0.538715in}}%
\pgfpathlineto{\pgfqpoint{1.092148in}{0.546826in}}%
\pgfpathlineto{\pgfqpoint{1.091600in}{0.558237in}}%
\pgfpathlineto{\pgfqpoint{1.089276in}{0.569648in}}%
\pgfpathlineto{\pgfqpoint{1.086889in}{0.581059in}}%
\pgfpathlineto{\pgfqpoint{1.085360in}{0.592469in}}%
\pgfpathlineto{\pgfqpoint{1.082935in}{0.603880in}}%
\pgfpathlineto{\pgfqpoint{1.082674in}{0.604767in}}%
\pgfpathlineto{\pgfqpoint{1.080455in}{0.615291in}}%
\pgfpathlineto{\pgfqpoint{1.077645in}{0.626702in}}%
\pgfpathlineto{\pgfqpoint{1.074526in}{0.638113in}}%
\pgfpathlineto{\pgfqpoint{1.071286in}{0.649523in}}%
\pgfpathlineto{\pgfqpoint{1.070942in}{0.650404in}}%
\pgfpathlineto{\pgfqpoint{1.068060in}{0.660934in}}%
\pgfpathlineto{\pgfqpoint{1.068060in}{0.672345in}}%
\pgfpathlineto{\pgfqpoint{1.069485in}{0.683756in}}%
\pgfpathlineto{\pgfqpoint{1.070942in}{0.688016in}}%
\pgfpathlineto{\pgfqpoint{1.072853in}{0.695166in}}%
\pgfpathlineto{\pgfqpoint{1.070942in}{0.697157in}}%
\pgfpathlineto{\pgfqpoint{1.067329in}{0.695166in}}%
\pgfpathlineto{\pgfqpoint{1.059210in}{0.691190in}}%
\pgfpathlineto{\pgfqpoint{1.047478in}{0.690836in}}%
\pgfpathlineto{\pgfqpoint{1.041245in}{0.695166in}}%
\pgfpathlineto{\pgfqpoint{1.035746in}{0.698863in}}%
\pgfpathlineto{\pgfqpoint{1.024494in}{0.706577in}}%
\pgfpathlineto{\pgfqpoint{1.024013in}{0.706897in}}%
\pgfpathlineto{\pgfqpoint{1.016490in}{0.717988in}}%
\pgfpathlineto{\pgfqpoint{1.012281in}{0.721349in}}%
\pgfpathlineto{\pgfqpoint{1.003652in}{0.729399in}}%
\pgfpathlineto{\pgfqpoint{1.000549in}{0.733940in}}%
\pgfpathlineto{\pgfqpoint{0.989388in}{0.740810in}}%
\pgfpathlineto{\pgfqpoint{0.988817in}{0.741188in}}%
\pgfpathlineto{\pgfqpoint{0.977085in}{0.745363in}}%
\pgfpathlineto{\pgfqpoint{0.975914in}{0.740810in}}%
\pgfpathlineto{\pgfqpoint{0.974849in}{0.729399in}}%
\pgfpathlineto{\pgfqpoint{0.976722in}{0.717988in}}%
\pgfpathlineto{\pgfqpoint{0.977085in}{0.716141in}}%
\pgfpathlineto{\pgfqpoint{0.980465in}{0.706577in}}%
\pgfpathlineto{\pgfqpoint{0.982545in}{0.695166in}}%
\pgfpathlineto{\pgfqpoint{0.984619in}{0.683756in}}%
\pgfpathlineto{\pgfqpoint{0.987056in}{0.672345in}}%
\pgfpathlineto{\pgfqpoint{0.988817in}{0.664606in}}%
\pgfpathlineto{\pgfqpoint{0.989919in}{0.660934in}}%
\pgfpathlineto{\pgfqpoint{0.992762in}{0.649523in}}%
\pgfpathlineto{\pgfqpoint{0.995704in}{0.638113in}}%
\pgfpathlineto{\pgfqpoint{0.998687in}{0.626702in}}%
\pgfpathlineto{\pgfqpoint{1.000549in}{0.620204in}}%
\pgfpathlineto{\pgfqpoint{1.002153in}{0.615291in}}%
\pgfpathlineto{\pgfqpoint{1.006558in}{0.603880in}}%
\pgfpathlineto{\pgfqpoint{1.011551in}{0.592469in}}%
\pgfpathlineto{\pgfqpoint{1.012281in}{0.590895in}}%
\pgfpathlineto{\pgfqpoint{1.016865in}{0.581059in}}%
\pgfpathlineto{\pgfqpoint{1.021880in}{0.569648in}}%
\pgfpathlineto{\pgfqpoint{1.024013in}{0.565017in}}%
\pgfpathlineto{\pgfqpoint{1.027314in}{0.558237in}}%
\pgfpathlineto{\pgfqpoint{1.032448in}{0.546826in}}%
\pgfpathlineto{\pgfqpoint{1.035746in}{0.539676in}}%
\pgfpathlineto{\pgfqpoint{1.038168in}{0.535416in}}%
\pgfpathlineto{\pgfqpoint{1.043148in}{0.524005in}}%
\pgfpathlineto{\pgfqpoint{1.047478in}{0.514548in}}%
\pgfpathlineto{\pgfqpoint{1.048793in}{0.512594in}}%
\pgfpathlineto{\pgfqpoint{1.053930in}{0.501183in}}%
\pgfpathlineto{\pgfqpoint{1.059105in}{0.489772in}}%
\pgfpathlineto{\pgfqpoint{1.059210in}{0.489552in}}%
\pgfpathlineto{\pgfqpoint{1.065479in}{0.478362in}}%
\pgfpathlineto{\pgfqpoint{1.069897in}{0.466951in}}%
\pgfpathlineto{\pgfqpoint{1.070942in}{0.464720in}}%
\pgfpathlineto{\pgfqpoint{1.076810in}{0.455540in}}%
\pgfpathlineto{\pgfqpoint{1.080541in}{0.444129in}}%
\pgfpathlineto{\pgfqpoint{1.082674in}{0.439502in}}%
\pgfpathlineto{\pgfqpoint{1.090942in}{0.432719in}}%
\pgfpathlineto{\pgfqpoint{1.093446in}{0.421308in}}%
\pgfpathclose%
\pgfpathmoveto{\pgfqpoint{1.067643in}{0.501183in}}%
\pgfpathlineto{\pgfqpoint{1.059210in}{0.512121in}}%
\pgfpathlineto{\pgfqpoint{1.059093in}{0.512594in}}%
\pgfpathlineto{\pgfqpoint{1.054532in}{0.524005in}}%
\pgfpathlineto{\pgfqpoint{1.047478in}{0.533248in}}%
\pgfpathlineto{\pgfqpoint{1.046712in}{0.535416in}}%
\pgfpathlineto{\pgfqpoint{1.042929in}{0.546826in}}%
\pgfpathlineto{\pgfqpoint{1.035746in}{0.555597in}}%
\pgfpathlineto{\pgfqpoint{1.034597in}{0.558237in}}%
\pgfpathlineto{\pgfqpoint{1.031478in}{0.569648in}}%
\pgfpathlineto{\pgfqpoint{1.024013in}{0.580734in}}%
\pgfpathlineto{\pgfqpoint{1.023872in}{0.581059in}}%
\pgfpathlineto{\pgfqpoint{1.020217in}{0.592469in}}%
\pgfpathlineto{\pgfqpoint{1.014856in}{0.603880in}}%
\pgfpathlineto{\pgfqpoint{1.012281in}{0.607099in}}%
\pgfpathlineto{\pgfqpoint{1.009182in}{0.615291in}}%
\pgfpathlineto{\pgfqpoint{1.006910in}{0.626702in}}%
\pgfpathlineto{\pgfqpoint{1.003764in}{0.638113in}}%
\pgfpathlineto{\pgfqpoint{1.000549in}{0.644879in}}%
\pgfpathlineto{\pgfqpoint{0.999482in}{0.649523in}}%
\pgfpathlineto{\pgfqpoint{0.998711in}{0.660934in}}%
\pgfpathlineto{\pgfqpoint{0.998527in}{0.672345in}}%
\pgfpathlineto{\pgfqpoint{0.999484in}{0.683756in}}%
\pgfpathlineto{\pgfqpoint{1.000549in}{0.686718in}}%
\pgfpathlineto{\pgfqpoint{1.012281in}{0.693173in}}%
\pgfpathlineto{\pgfqpoint{1.019408in}{0.683756in}}%
\pgfpathlineto{\pgfqpoint{1.024013in}{0.680531in}}%
\pgfpathlineto{\pgfqpoint{1.029103in}{0.672345in}}%
\pgfpathlineto{\pgfqpoint{1.035746in}{0.661867in}}%
\pgfpathlineto{\pgfqpoint{1.036573in}{0.660934in}}%
\pgfpathlineto{\pgfqpoint{1.041549in}{0.649523in}}%
\pgfpathlineto{\pgfqpoint{1.045681in}{0.638113in}}%
\pgfpathlineto{\pgfqpoint{1.047478in}{0.632938in}}%
\pgfpathlineto{\pgfqpoint{1.049798in}{0.626702in}}%
\pgfpathlineto{\pgfqpoint{1.053278in}{0.615291in}}%
\pgfpathlineto{\pgfqpoint{1.056827in}{0.603880in}}%
\pgfpathlineto{\pgfqpoint{1.059210in}{0.595680in}}%
\pgfpathlineto{\pgfqpoint{1.060132in}{0.592469in}}%
\pgfpathlineto{\pgfqpoint{1.063066in}{0.581059in}}%
\pgfpathlineto{\pgfqpoint{1.065775in}{0.569648in}}%
\pgfpathlineto{\pgfqpoint{1.068167in}{0.558237in}}%
\pgfpathlineto{\pgfqpoint{1.070175in}{0.546826in}}%
\pgfpathlineto{\pgfqpoint{1.070942in}{0.541270in}}%
\pgfpathlineto{\pgfqpoint{1.071809in}{0.535416in}}%
\pgfpathlineto{\pgfqpoint{1.073035in}{0.524005in}}%
\pgfpathlineto{\pgfqpoint{1.073792in}{0.512594in}}%
\pgfpathlineto{\pgfqpoint{1.074011in}{0.501183in}}%
\pgfpathlineto{\pgfqpoint{1.070942in}{0.494587in}}%
\pgfpathclose%
\pgfusepath{fill}%
\end{pgfscope}%
\begin{pgfscope}%
\pgfpathrectangle{\pgfqpoint{0.211875in}{0.211875in}}{\pgfqpoint{1.313625in}{1.279725in}}%
\pgfusepath{clip}%
\pgfsetbuttcap%
\pgfsetroundjoin%
\definecolor{currentfill}{rgb}{0.584229,0.109227,0.358485}%
\pgfsetfillcolor{currentfill}%
\pgfsetlinewidth{0.000000pt}%
\definecolor{currentstroke}{rgb}{0.000000,0.000000,0.000000}%
\pgfsetstrokecolor{currentstroke}%
\pgfsetdash{}{0pt}%
\pgfpathmoveto{\pgfqpoint{0.296617in}{0.622359in}}%
\pgfpathlineto{\pgfqpoint{0.304801in}{0.626702in}}%
\pgfpathlineto{\pgfqpoint{0.308349in}{0.628477in}}%
\pgfpathlineto{\pgfqpoint{0.320081in}{0.634467in}}%
\pgfpathlineto{\pgfqpoint{0.326880in}{0.638113in}}%
\pgfpathlineto{\pgfqpoint{0.331813in}{0.640605in}}%
\pgfpathlineto{\pgfqpoint{0.343545in}{0.646703in}}%
\pgfpathlineto{\pgfqpoint{0.348740in}{0.649523in}}%
\pgfpathlineto{\pgfqpoint{0.355278in}{0.652867in}}%
\pgfpathlineto{\pgfqpoint{0.367010in}{0.659089in}}%
\pgfpathlineto{\pgfqpoint{0.370359in}{0.660934in}}%
\pgfpathlineto{\pgfqpoint{0.378742in}{0.665281in}}%
\pgfpathlineto{\pgfqpoint{0.390474in}{0.671647in}}%
\pgfpathlineto{\pgfqpoint{0.391723in}{0.672345in}}%
\pgfpathlineto{\pgfqpoint{0.402206in}{0.677864in}}%
\pgfpathlineto{\pgfqpoint{0.412847in}{0.683756in}}%
\pgfpathlineto{\pgfqpoint{0.413939in}{0.684333in}}%
\pgfpathlineto{\pgfqpoint{0.425671in}{0.690625in}}%
\pgfpathlineto{\pgfqpoint{0.433747in}{0.695166in}}%
\pgfpathlineto{\pgfqpoint{0.437403in}{0.697132in}}%
\pgfpathlineto{\pgfqpoint{0.449135in}{0.703567in}}%
\pgfpathlineto{\pgfqpoint{0.454411in}{0.706577in}}%
\pgfpathlineto{\pgfqpoint{0.460867in}{0.710100in}}%
\pgfpathlineto{\pgfqpoint{0.472600in}{0.716686in}}%
\pgfpathlineto{\pgfqpoint{0.474852in}{0.717988in}}%
\pgfpathlineto{\pgfqpoint{0.484332in}{0.723229in}}%
\pgfpathlineto{\pgfqpoint{0.495144in}{0.729399in}}%
\pgfpathlineto{\pgfqpoint{0.496064in}{0.729911in}}%
\pgfpathlineto{\pgfqpoint{0.507796in}{0.736355in}}%
\pgfpathlineto{\pgfqpoint{0.515657in}{0.740810in}}%
\pgfpathlineto{\pgfqpoint{0.519528in}{0.742947in}}%
\pgfpathlineto{\pgfqpoint{0.531261in}{0.749479in}}%
\pgfpathlineto{\pgfqpoint{0.536068in}{0.752220in}}%
\pgfpathlineto{\pgfqpoint{0.542993in}{0.756079in}}%
\pgfpathlineto{\pgfqpoint{0.554725in}{0.762704in}}%
\pgfpathlineto{\pgfqpoint{0.556345in}{0.763631in}}%
\pgfpathlineto{\pgfqpoint{0.566457in}{0.769309in}}%
\pgfpathlineto{\pgfqpoint{0.576501in}{0.775042in}}%
\pgfpathlineto{\pgfqpoint{0.578190in}{0.776007in}}%
\pgfpathlineto{\pgfqpoint{0.589922in}{0.782665in}}%
\pgfpathlineto{\pgfqpoint{0.596503in}{0.786453in}}%
\pgfpathlineto{\pgfqpoint{0.601654in}{0.789441in}}%
\pgfpathlineto{\pgfqpoint{0.613386in}{0.796233in}}%
\pgfpathlineto{\pgfqpoint{0.616178in}{0.797863in}}%
\pgfpathlineto{\pgfqpoint{0.625118in}{0.803170in}}%
\pgfpathlineto{\pgfqpoint{0.635376in}{0.809274in}}%
\pgfpathlineto{\pgfqpoint{0.636851in}{0.810194in}}%
\pgfpathlineto{\pgfqpoint{0.648583in}{0.817388in}}%
\pgfpathlineto{\pgfqpoint{0.653935in}{0.820685in}}%
\pgfpathlineto{\pgfqpoint{0.660315in}{0.824871in}}%
\pgfpathlineto{\pgfqpoint{0.672047in}{0.829300in}}%
\pgfpathlineto{\pgfqpoint{0.676712in}{0.832096in}}%
\pgfpathlineto{\pgfqpoint{0.683779in}{0.837368in}}%
\pgfpathlineto{\pgfqpoint{0.692139in}{0.843507in}}%
\pgfpathlineto{\pgfqpoint{0.695512in}{0.846423in}}%
\pgfpathlineto{\pgfqpoint{0.705697in}{0.854917in}}%
\pgfpathlineto{\pgfqpoint{0.707244in}{0.856596in}}%
\pgfpathlineto{\pgfqpoint{0.716866in}{0.866328in}}%
\pgfpathlineto{\pgfqpoint{0.718976in}{0.869756in}}%
\pgfpathlineto{\pgfqpoint{0.724551in}{0.877739in}}%
\pgfpathlineto{\pgfqpoint{0.723146in}{0.889150in}}%
\pgfpathlineto{\pgfqpoint{0.718976in}{0.890212in}}%
\pgfpathlineto{\pgfqpoint{0.707244in}{0.890055in}}%
\pgfpathlineto{\pgfqpoint{0.703536in}{0.889150in}}%
\pgfpathlineto{\pgfqpoint{0.695512in}{0.887183in}}%
\pgfpathlineto{\pgfqpoint{0.683779in}{0.883966in}}%
\pgfpathlineto{\pgfqpoint{0.672047in}{0.881021in}}%
\pgfpathlineto{\pgfqpoint{0.660315in}{0.878508in}}%
\pgfpathlineto{\pgfqpoint{0.648583in}{0.878558in}}%
\pgfpathlineto{\pgfqpoint{0.641558in}{0.877739in}}%
\pgfpathlineto{\pgfqpoint{0.636851in}{0.877360in}}%
\pgfpathlineto{\pgfqpoint{0.625118in}{0.874520in}}%
\pgfpathlineto{\pgfqpoint{0.613386in}{0.873277in}}%
\pgfpathlineto{\pgfqpoint{0.601654in}{0.870790in}}%
\pgfpathlineto{\pgfqpoint{0.592275in}{0.866328in}}%
\pgfpathlineto{\pgfqpoint{0.589922in}{0.865703in}}%
\pgfpathlineto{\pgfqpoint{0.578190in}{0.862726in}}%
\pgfpathlineto{\pgfqpoint{0.566457in}{0.861146in}}%
\pgfpathlineto{\pgfqpoint{0.554725in}{0.859248in}}%
\pgfpathlineto{\pgfqpoint{0.543378in}{0.854917in}}%
\pgfpathlineto{\pgfqpoint{0.542993in}{0.854837in}}%
\pgfpathlineto{\pgfqpoint{0.531261in}{0.852527in}}%
\pgfpathlineto{\pgfqpoint{0.519528in}{0.850565in}}%
\pgfpathlineto{\pgfqpoint{0.507796in}{0.848706in}}%
\pgfpathlineto{\pgfqpoint{0.496064in}{0.844523in}}%
\pgfpathlineto{\pgfqpoint{0.494244in}{0.843507in}}%
\pgfpathlineto{\pgfqpoint{0.484332in}{0.841412in}}%
\pgfpathlineto{\pgfqpoint{0.472600in}{0.838643in}}%
\pgfpathlineto{\pgfqpoint{0.460867in}{0.836115in}}%
\pgfpathlineto{\pgfqpoint{0.451273in}{0.832096in}}%
\pgfpathlineto{\pgfqpoint{0.449135in}{0.831585in}}%
\pgfpathlineto{\pgfqpoint{0.437403in}{0.828777in}}%
\pgfpathlineto{\pgfqpoint{0.425671in}{0.825236in}}%
\pgfpathlineto{\pgfqpoint{0.414419in}{0.820685in}}%
\pgfpathlineto{\pgfqpoint{0.413939in}{0.820559in}}%
\pgfpathlineto{\pgfqpoint{0.402206in}{0.817520in}}%
\pgfpathlineto{\pgfqpoint{0.390474in}{0.813477in}}%
\pgfpathlineto{\pgfqpoint{0.381463in}{0.809274in}}%
\pgfpathlineto{\pgfqpoint{0.378742in}{0.808510in}}%
\pgfpathlineto{\pgfqpoint{0.367010in}{0.805012in}}%
\pgfpathlineto{\pgfqpoint{0.355278in}{0.800105in}}%
\pgfpathlineto{\pgfqpoint{0.350919in}{0.797863in}}%
\pgfpathlineto{\pgfqpoint{0.343545in}{0.795698in}}%
\pgfpathlineto{\pgfqpoint{0.331813in}{0.791555in}}%
\pgfpathlineto{\pgfqpoint{0.321471in}{0.786453in}}%
\pgfpathlineto{\pgfqpoint{0.320081in}{0.786031in}}%
\pgfpathlineto{\pgfqpoint{0.308349in}{0.782376in}}%
\pgfpathlineto{\pgfqpoint{0.296617in}{0.777479in}}%
\pgfpathlineto{\pgfqpoint{0.291796in}{0.775042in}}%
\pgfpathlineto{\pgfqpoint{0.284884in}{0.772915in}}%
\pgfpathlineto{\pgfqpoint{0.284884in}{0.763631in}}%
\pgfpathlineto{\pgfqpoint{0.284884in}{0.756779in}}%
\pgfpathlineto{\pgfqpoint{0.296617in}{0.760863in}}%
\pgfpathlineto{\pgfqpoint{0.305041in}{0.763631in}}%
\pgfpathlineto{\pgfqpoint{0.308349in}{0.764992in}}%
\pgfpathlineto{\pgfqpoint{0.320081in}{0.769475in}}%
\pgfpathlineto{\pgfqpoint{0.331813in}{0.773400in}}%
\pgfpathlineto{\pgfqpoint{0.336828in}{0.775042in}}%
\pgfpathlineto{\pgfqpoint{0.343545in}{0.777768in}}%
\pgfpathlineto{\pgfqpoint{0.355278in}{0.782016in}}%
\pgfpathlineto{\pgfqpoint{0.367010in}{0.785795in}}%
\pgfpathlineto{\pgfqpoint{0.369036in}{0.786453in}}%
\pgfpathlineto{\pgfqpoint{0.378742in}{0.790305in}}%
\pgfpathlineto{\pgfqpoint{0.390474in}{0.794328in}}%
\pgfpathlineto{\pgfqpoint{0.401823in}{0.797863in}}%
\pgfpathlineto{\pgfqpoint{0.402206in}{0.798017in}}%
\pgfpathlineto{\pgfqpoint{0.413939in}{0.802496in}}%
\pgfpathlineto{\pgfqpoint{0.425671in}{0.806306in}}%
\pgfpathlineto{\pgfqpoint{0.435462in}{0.809274in}}%
\pgfpathlineto{\pgfqpoint{0.437403in}{0.810015in}}%
\pgfpathlineto{\pgfqpoint{0.449135in}{0.814197in}}%
\pgfpathlineto{\pgfqpoint{0.460867in}{0.817803in}}%
\pgfpathlineto{\pgfqpoint{0.470769in}{0.820685in}}%
\pgfpathlineto{\pgfqpoint{0.472600in}{0.821335in}}%
\pgfpathlineto{\pgfqpoint{0.484332in}{0.825234in}}%
\pgfpathlineto{\pgfqpoint{0.496064in}{0.828629in}}%
\pgfpathlineto{\pgfqpoint{0.507796in}{0.831854in}}%
\pgfpathlineto{\pgfqpoint{0.508653in}{0.832096in}}%
\pgfpathlineto{\pgfqpoint{0.519528in}{0.835415in}}%
\pgfpathlineto{\pgfqpoint{0.531261in}{0.838558in}}%
\pgfpathlineto{\pgfqpoint{0.542993in}{0.841518in}}%
\pgfpathlineto{\pgfqpoint{0.550708in}{0.843507in}}%
\pgfpathlineto{\pgfqpoint{0.554725in}{0.844569in}}%
\pgfpathlineto{\pgfqpoint{0.566457in}{0.847527in}}%
\pgfpathlineto{\pgfqpoint{0.578190in}{0.850782in}}%
\pgfpathlineto{\pgfqpoint{0.589922in}{0.854303in}}%
\pgfpathlineto{\pgfqpoint{0.591927in}{0.854917in}}%
\pgfpathlineto{\pgfqpoint{0.601654in}{0.857449in}}%
\pgfpathlineto{\pgfqpoint{0.613386in}{0.860169in}}%
\pgfpathlineto{\pgfqpoint{0.625118in}{0.862825in}}%
\pgfpathlineto{\pgfqpoint{0.636851in}{0.865690in}}%
\pgfpathlineto{\pgfqpoint{0.642403in}{0.866328in}}%
\pgfpathlineto{\pgfqpoint{0.648583in}{0.866739in}}%
\pgfpathlineto{\pgfqpoint{0.651116in}{0.866328in}}%
\pgfpathlineto{\pgfqpoint{0.660315in}{0.860714in}}%
\pgfpathlineto{\pgfqpoint{0.666646in}{0.854917in}}%
\pgfpathlineto{\pgfqpoint{0.660315in}{0.847139in}}%
\pgfpathlineto{\pgfqpoint{0.658266in}{0.843507in}}%
\pgfpathlineto{\pgfqpoint{0.648583in}{0.836415in}}%
\pgfpathlineto{\pgfqpoint{0.643284in}{0.832096in}}%
\pgfpathlineto{\pgfqpoint{0.636851in}{0.827884in}}%
\pgfpathlineto{\pgfqpoint{0.626551in}{0.820685in}}%
\pgfpathlineto{\pgfqpoint{0.625118in}{0.819803in}}%
\pgfpathlineto{\pgfqpoint{0.613386in}{0.812499in}}%
\pgfpathlineto{\pgfqpoint{0.608490in}{0.809274in}}%
\pgfpathlineto{\pgfqpoint{0.601654in}{0.805221in}}%
\pgfpathlineto{\pgfqpoint{0.589922in}{0.798018in}}%
\pgfpathlineto{\pgfqpoint{0.589677in}{0.797863in}}%
\pgfpathlineto{\pgfqpoint{0.578190in}{0.791228in}}%
\pgfpathlineto{\pgfqpoint{0.570206in}{0.786453in}}%
\pgfpathlineto{\pgfqpoint{0.566457in}{0.784318in}}%
\pgfpathlineto{\pgfqpoint{0.554725in}{0.777613in}}%
\pgfpathlineto{\pgfqpoint{0.550332in}{0.775042in}}%
\pgfpathlineto{\pgfqpoint{0.542993in}{0.770921in}}%
\pgfpathlineto{\pgfqpoint{0.531261in}{0.764264in}}%
\pgfpathlineto{\pgfqpoint{0.530157in}{0.763631in}}%
\pgfpathlineto{\pgfqpoint{0.519528in}{0.757747in}}%
\pgfpathlineto{\pgfqpoint{0.509683in}{0.752220in}}%
\pgfpathlineto{\pgfqpoint{0.507796in}{0.751174in}}%
\pgfpathlineto{\pgfqpoint{0.496064in}{0.744608in}}%
\pgfpathlineto{\pgfqpoint{0.489332in}{0.740810in}}%
\pgfpathlineto{\pgfqpoint{0.484332in}{0.738010in}}%
\pgfpathlineto{\pgfqpoint{0.472600in}{0.731517in}}%
\pgfpathlineto{\pgfqpoint{0.468780in}{0.729399in}}%
\pgfpathlineto{\pgfqpoint{0.460867in}{0.725037in}}%
\pgfpathlineto{\pgfqpoint{0.449135in}{0.718629in}}%
\pgfpathlineto{\pgfqpoint{0.447959in}{0.717988in}}%
\pgfpathlineto{\pgfqpoint{0.437403in}{0.712262in}}%
\pgfpathlineto{\pgfqpoint{0.426838in}{0.706577in}}%
\pgfpathlineto{\pgfqpoint{0.425671in}{0.705941in}}%
\pgfpathlineto{\pgfqpoint{0.413939in}{0.699678in}}%
\pgfpathlineto{\pgfqpoint{0.405422in}{0.695166in}}%
\pgfpathlineto{\pgfqpoint{0.402206in}{0.693440in}}%
\pgfpathlineto{\pgfqpoint{0.390474in}{0.687271in}}%
\pgfpathlineto{\pgfqpoint{0.383744in}{0.683756in}}%
\pgfpathlineto{\pgfqpoint{0.378742in}{0.681107in}}%
\pgfpathlineto{\pgfqpoint{0.367010in}{0.675022in}}%
\pgfpathlineto{\pgfqpoint{0.361822in}{0.672345in}}%
\pgfpathlineto{\pgfqpoint{0.355278in}{0.668921in}}%
\pgfpathlineto{\pgfqpoint{0.343545in}{0.662907in}}%
\pgfpathlineto{\pgfqpoint{0.339685in}{0.660934in}}%
\pgfpathlineto{\pgfqpoint{0.331813in}{0.656858in}}%
\pgfpathlineto{\pgfqpoint{0.320081in}{0.650902in}}%
\pgfpathlineto{\pgfqpoint{0.317363in}{0.649523in}}%
\pgfpathlineto{\pgfqpoint{0.308349in}{0.644894in}}%
\pgfpathlineto{\pgfqpoint{0.296617in}{0.638983in}}%
\pgfpathlineto{\pgfqpoint{0.294891in}{0.638113in}}%
\pgfpathlineto{\pgfqpoint{0.284884in}{0.633007in}}%
\pgfpathlineto{\pgfqpoint{0.284884in}{0.626702in}}%
\pgfpathlineto{\pgfqpoint{0.284884in}{0.616460in}}%
\pgfpathclose%
\pgfusepath{fill}%
\end{pgfscope}%
\begin{pgfscope}%
\pgfpathrectangle{\pgfqpoint{0.211875in}{0.211875in}}{\pgfqpoint{1.313625in}{1.279725in}}%
\pgfusepath{clip}%
\pgfsetbuttcap%
\pgfsetroundjoin%
\definecolor{currentfill}{rgb}{0.584229,0.109227,0.358485}%
\pgfsetfillcolor{currentfill}%
\pgfsetlinewidth{0.000000pt}%
\definecolor{currentstroke}{rgb}{0.000000,0.000000,0.000000}%
\pgfsetstrokecolor{currentstroke}%
\pgfsetdash{}{0pt}%
\pgfpathmoveto{\pgfqpoint{1.153068in}{0.739392in}}%
\pgfpathlineto{\pgfqpoint{1.164800in}{0.730662in}}%
\pgfpathlineto{\pgfqpoint{1.176532in}{0.732623in}}%
\pgfpathlineto{\pgfqpoint{1.188264in}{0.736558in}}%
\pgfpathlineto{\pgfqpoint{1.191925in}{0.740810in}}%
\pgfpathlineto{\pgfqpoint{1.188549in}{0.752220in}}%
\pgfpathlineto{\pgfqpoint{1.188264in}{0.752605in}}%
\pgfpathlineto{\pgfqpoint{1.176927in}{0.763631in}}%
\pgfpathlineto{\pgfqpoint{1.176532in}{0.763890in}}%
\pgfpathlineto{\pgfqpoint{1.164800in}{0.767377in}}%
\pgfpathlineto{\pgfqpoint{1.154328in}{0.763631in}}%
\pgfpathlineto{\pgfqpoint{1.153068in}{0.760002in}}%
\pgfpathlineto{\pgfqpoint{1.151047in}{0.752220in}}%
\pgfpathlineto{\pgfqpoint{1.152172in}{0.740810in}}%
\pgfpathclose%
\pgfusepath{fill}%
\end{pgfscope}%
\begin{pgfscope}%
\pgfpathrectangle{\pgfqpoint{0.211875in}{0.211875in}}{\pgfqpoint{1.313625in}{1.279725in}}%
\pgfusepath{clip}%
\pgfsetbuttcap%
\pgfsetroundjoin%
\definecolor{currentfill}{rgb}{0.584229,0.109227,0.358485}%
\pgfsetfillcolor{currentfill}%
\pgfsetlinewidth{0.000000pt}%
\definecolor{currentstroke}{rgb}{0.000000,0.000000,0.000000}%
\pgfsetstrokecolor{currentstroke}%
\pgfsetdash{}{0pt}%
\pgfpathmoveto{\pgfqpoint{1.399444in}{0.853186in}}%
\pgfpathlineto{\pgfqpoint{1.411176in}{0.852626in}}%
\pgfpathlineto{\pgfqpoint{1.412616in}{0.854917in}}%
\pgfpathlineto{\pgfqpoint{1.420059in}{0.866328in}}%
\pgfpathlineto{\pgfqpoint{1.422908in}{0.870588in}}%
\pgfpathlineto{\pgfqpoint{1.426457in}{0.877739in}}%
\pgfpathlineto{\pgfqpoint{1.425790in}{0.889150in}}%
\pgfpathlineto{\pgfqpoint{1.424455in}{0.900560in}}%
\pgfpathlineto{\pgfqpoint{1.422908in}{0.904870in}}%
\pgfpathlineto{\pgfqpoint{1.420119in}{0.911971in}}%
\pgfpathlineto{\pgfqpoint{1.411176in}{0.920287in}}%
\pgfpathlineto{\pgfqpoint{1.407046in}{0.923382in}}%
\pgfpathlineto{\pgfqpoint{1.399444in}{0.926208in}}%
\pgfpathlineto{\pgfqpoint{1.387712in}{0.928541in}}%
\pgfpathlineto{\pgfqpoint{1.375980in}{0.923735in}}%
\pgfpathlineto{\pgfqpoint{1.375442in}{0.923382in}}%
\pgfpathlineto{\pgfqpoint{1.366320in}{0.911971in}}%
\pgfpathlineto{\pgfqpoint{1.364247in}{0.905488in}}%
\pgfpathlineto{\pgfqpoint{1.363087in}{0.900560in}}%
\pgfpathlineto{\pgfqpoint{1.364247in}{0.897075in}}%
\pgfpathlineto{\pgfqpoint{1.367397in}{0.889150in}}%
\pgfpathlineto{\pgfqpoint{1.374160in}{0.877739in}}%
\pgfpathlineto{\pgfqpoint{1.375980in}{0.874805in}}%
\pgfpathlineto{\pgfqpoint{1.383605in}{0.866328in}}%
\pgfpathlineto{\pgfqpoint{1.387712in}{0.861938in}}%
\pgfpathlineto{\pgfqpoint{1.397293in}{0.854917in}}%
\pgfpathclose%
\pgfusepath{fill}%
\end{pgfscope}%
\begin{pgfscope}%
\pgfpathrectangle{\pgfqpoint{0.211875in}{0.211875in}}{\pgfqpoint{1.313625in}{1.279725in}}%
\pgfusepath{clip}%
\pgfsetbuttcap%
\pgfsetroundjoin%
\definecolor{currentfill}{rgb}{0.584229,0.109227,0.358485}%
\pgfsetfillcolor{currentfill}%
\pgfsetlinewidth{0.000000pt}%
\definecolor{currentstroke}{rgb}{0.000000,0.000000,0.000000}%
\pgfsetstrokecolor{currentstroke}%
\pgfsetdash{}{0pt}%
\pgfpathmoveto{\pgfqpoint{1.153068in}{0.888121in}}%
\pgfpathlineto{\pgfqpoint{1.164800in}{0.887984in}}%
\pgfpathlineto{\pgfqpoint{1.176532in}{0.888188in}}%
\pgfpathlineto{\pgfqpoint{1.184216in}{0.889150in}}%
\pgfpathlineto{\pgfqpoint{1.188264in}{0.890016in}}%
\pgfpathlineto{\pgfqpoint{1.199996in}{0.892999in}}%
\pgfpathlineto{\pgfqpoint{1.211729in}{0.896212in}}%
\pgfpathlineto{\pgfqpoint{1.219239in}{0.900560in}}%
\pgfpathlineto{\pgfqpoint{1.223461in}{0.904622in}}%
\pgfpathlineto{\pgfqpoint{1.230067in}{0.911971in}}%
\pgfpathlineto{\pgfqpoint{1.235193in}{0.918182in}}%
\pgfpathlineto{\pgfqpoint{1.246925in}{0.923081in}}%
\pgfpathlineto{\pgfqpoint{1.247879in}{0.923382in}}%
\pgfpathlineto{\pgfqpoint{1.258658in}{0.928777in}}%
\pgfpathlineto{\pgfqpoint{1.264360in}{0.934793in}}%
\pgfpathlineto{\pgfqpoint{1.258658in}{0.938355in}}%
\pgfpathlineto{\pgfqpoint{1.246925in}{0.940708in}}%
\pgfpathlineto{\pgfqpoint{1.235193in}{0.941549in}}%
\pgfpathlineto{\pgfqpoint{1.223461in}{0.943318in}}%
\pgfpathlineto{\pgfqpoint{1.217881in}{0.946204in}}%
\pgfpathlineto{\pgfqpoint{1.211729in}{0.947152in}}%
\pgfpathlineto{\pgfqpoint{1.199996in}{0.948382in}}%
\pgfpathlineto{\pgfqpoint{1.188264in}{0.950126in}}%
\pgfpathlineto{\pgfqpoint{1.176532in}{0.951611in}}%
\pgfpathlineto{\pgfqpoint{1.164800in}{0.950546in}}%
\pgfpathlineto{\pgfqpoint{1.153068in}{0.949052in}}%
\pgfpathlineto{\pgfqpoint{1.141335in}{0.947967in}}%
\pgfpathlineto{\pgfqpoint{1.129603in}{0.946924in}}%
\pgfpathlineto{\pgfqpoint{1.117871in}{0.946325in}}%
\pgfpathlineto{\pgfqpoint{1.116131in}{0.946204in}}%
\pgfpathlineto{\pgfqpoint{1.106139in}{0.944462in}}%
\pgfpathlineto{\pgfqpoint{1.101144in}{0.946204in}}%
\pgfpathlineto{\pgfqpoint{1.094407in}{0.947383in}}%
\pgfpathlineto{\pgfqpoint{1.082674in}{0.950422in}}%
\pgfpathlineto{\pgfqpoint{1.070942in}{0.953909in}}%
\pgfpathlineto{\pgfqpoint{1.059210in}{0.957358in}}%
\pgfpathlineto{\pgfqpoint{1.048700in}{0.957614in}}%
\pgfpathlineto{\pgfqpoint{1.047478in}{0.957626in}}%
\pgfpathlineto{\pgfqpoint{1.046597in}{0.957614in}}%
\pgfpathlineto{\pgfqpoint{1.035746in}{0.957225in}}%
\pgfpathlineto{\pgfqpoint{1.024013in}{0.957223in}}%
\pgfpathlineto{\pgfqpoint{1.022372in}{0.957614in}}%
\pgfpathlineto{\pgfqpoint{1.012281in}{0.958133in}}%
\pgfpathlineto{\pgfqpoint{1.007357in}{0.957614in}}%
\pgfpathlineto{\pgfqpoint{1.000549in}{0.955877in}}%
\pgfpathlineto{\pgfqpoint{0.988817in}{0.951979in}}%
\pgfpathlineto{\pgfqpoint{0.977085in}{0.948664in}}%
\pgfpathlineto{\pgfqpoint{0.966489in}{0.946204in}}%
\pgfpathlineto{\pgfqpoint{0.965352in}{0.943420in}}%
\pgfpathlineto{\pgfqpoint{0.963410in}{0.934793in}}%
\pgfpathlineto{\pgfqpoint{0.965352in}{0.932726in}}%
\pgfpathlineto{\pgfqpoint{0.977085in}{0.926504in}}%
\pgfpathlineto{\pgfqpoint{0.988817in}{0.923978in}}%
\pgfpathlineto{\pgfqpoint{0.992174in}{0.923382in}}%
\pgfpathlineto{\pgfqpoint{1.000549in}{0.922038in}}%
\pgfpathlineto{\pgfqpoint{1.012281in}{0.915903in}}%
\pgfpathlineto{\pgfqpoint{1.016870in}{0.911971in}}%
\pgfpathlineto{\pgfqpoint{1.024013in}{0.908655in}}%
\pgfpathlineto{\pgfqpoint{1.034235in}{0.900560in}}%
\pgfpathlineto{\pgfqpoint{1.035746in}{0.899657in}}%
\pgfpathlineto{\pgfqpoint{1.047478in}{0.896379in}}%
\pgfpathlineto{\pgfqpoint{1.059210in}{0.894012in}}%
\pgfpathlineto{\pgfqpoint{1.070942in}{0.897609in}}%
\pgfpathlineto{\pgfqpoint{1.081345in}{0.900560in}}%
\pgfpathlineto{\pgfqpoint{1.082674in}{0.900998in}}%
\pgfpathlineto{\pgfqpoint{1.094407in}{0.905649in}}%
\pgfpathlineto{\pgfqpoint{1.106139in}{0.905444in}}%
\pgfpathlineto{\pgfqpoint{1.112535in}{0.900560in}}%
\pgfpathlineto{\pgfqpoint{1.117871in}{0.898156in}}%
\pgfpathlineto{\pgfqpoint{1.129603in}{0.893445in}}%
\pgfpathlineto{\pgfqpoint{1.141335in}{0.889820in}}%
\pgfpathlineto{\pgfqpoint{1.143866in}{0.889150in}}%
\pgfpathclose%
\pgfusepath{fill}%
\end{pgfscope}%
\begin{pgfscope}%
\pgfpathrectangle{\pgfqpoint{0.211875in}{0.211875in}}{\pgfqpoint{1.313625in}{1.279725in}}%
\pgfusepath{clip}%
\pgfsetbuttcap%
\pgfsetroundjoin%
\definecolor{currentfill}{rgb}{0.584229,0.109227,0.358485}%
\pgfsetfillcolor{currentfill}%
\pgfsetlinewidth{0.000000pt}%
\definecolor{currentstroke}{rgb}{0.000000,0.000000,0.000000}%
\pgfsetstrokecolor{currentstroke}%
\pgfsetdash{}{0pt}%
\pgfpathmoveto{\pgfqpoint{1.446373in}{0.990905in}}%
\pgfpathlineto{\pgfqpoint{1.446373in}{0.991847in}}%
\pgfpathlineto{\pgfqpoint{1.446373in}{1.001071in}}%
\pgfpathlineto{\pgfqpoint{1.443294in}{1.003257in}}%
\pgfpathlineto{\pgfqpoint{1.434641in}{1.011206in}}%
\pgfpathlineto{\pgfqpoint{1.432192in}{1.014668in}}%
\pgfpathlineto{\pgfqpoint{1.427630in}{1.026079in}}%
\pgfpathlineto{\pgfqpoint{1.425988in}{1.037490in}}%
\pgfpathlineto{\pgfqpoint{1.429212in}{1.048901in}}%
\pgfpathlineto{\pgfqpoint{1.434641in}{1.054060in}}%
\pgfpathlineto{\pgfqpoint{1.443075in}{1.060311in}}%
\pgfpathlineto{\pgfqpoint{1.446373in}{1.062734in}}%
\pgfpathlineto{\pgfqpoint{1.446373in}{1.071052in}}%
\pgfpathlineto{\pgfqpoint{1.434641in}{1.061879in}}%
\pgfpathlineto{\pgfqpoint{1.432779in}{1.060311in}}%
\pgfpathlineto{\pgfqpoint{1.422908in}{1.050892in}}%
\pgfpathlineto{\pgfqpoint{1.421716in}{1.048901in}}%
\pgfpathlineto{\pgfqpoint{1.420300in}{1.037490in}}%
\pgfpathlineto{\pgfqpoint{1.420471in}{1.026079in}}%
\pgfpathlineto{\pgfqpoint{1.422661in}{1.014668in}}%
\pgfpathlineto{\pgfqpoint{1.422908in}{1.013983in}}%
\pgfpathlineto{\pgfqpoint{1.429475in}{1.003257in}}%
\pgfpathlineto{\pgfqpoint{1.434641in}{0.998612in}}%
\pgfpathlineto{\pgfqpoint{1.444411in}{0.991847in}}%
\pgfpathclose%
\pgfusepath{fill}%
\end{pgfscope}%
\begin{pgfscope}%
\pgfpathrectangle{\pgfqpoint{0.211875in}{0.211875in}}{\pgfqpoint{1.313625in}{1.279725in}}%
\pgfusepath{clip}%
\pgfsetbuttcap%
\pgfsetroundjoin%
\definecolor{currentfill}{rgb}{0.584229,0.109227,0.358485}%
\pgfsetfillcolor{currentfill}%
\pgfsetlinewidth{0.000000pt}%
\definecolor{currentstroke}{rgb}{0.000000,0.000000,0.000000}%
\pgfsetstrokecolor{currentstroke}%
\pgfsetdash{}{0pt}%
\pgfpathmoveto{\pgfqpoint{1.293854in}{1.001431in}}%
\pgfpathlineto{\pgfqpoint{1.305586in}{1.001355in}}%
\pgfpathlineto{\pgfqpoint{1.309673in}{1.003257in}}%
\pgfpathlineto{\pgfqpoint{1.317319in}{1.009029in}}%
\pgfpathlineto{\pgfqpoint{1.319405in}{1.014668in}}%
\pgfpathlineto{\pgfqpoint{1.320712in}{1.026079in}}%
\pgfpathlineto{\pgfqpoint{1.322867in}{1.037490in}}%
\pgfpathlineto{\pgfqpoint{1.324760in}{1.048901in}}%
\pgfpathlineto{\pgfqpoint{1.317319in}{1.059334in}}%
\pgfpathlineto{\pgfqpoint{1.316576in}{1.060311in}}%
\pgfpathlineto{\pgfqpoint{1.305586in}{1.068445in}}%
\pgfpathlineto{\pgfqpoint{1.302153in}{1.071722in}}%
\pgfpathlineto{\pgfqpoint{1.293854in}{1.078393in}}%
\pgfpathlineto{\pgfqpoint{1.288785in}{1.083133in}}%
\pgfpathlineto{\pgfqpoint{1.282122in}{1.087808in}}%
\pgfpathlineto{\pgfqpoint{1.270390in}{1.092312in}}%
\pgfpathlineto{\pgfqpoint{1.265461in}{1.094544in}}%
\pgfpathlineto{\pgfqpoint{1.258658in}{1.097722in}}%
\pgfpathlineto{\pgfqpoint{1.246925in}{1.101973in}}%
\pgfpathlineto{\pgfqpoint{1.236915in}{1.105954in}}%
\pgfpathlineto{\pgfqpoint{1.235193in}{1.106692in}}%
\pgfpathlineto{\pgfqpoint{1.223461in}{1.109844in}}%
\pgfpathlineto{\pgfqpoint{1.211729in}{1.111759in}}%
\pgfpathlineto{\pgfqpoint{1.199996in}{1.113164in}}%
\pgfpathlineto{\pgfqpoint{1.188264in}{1.112951in}}%
\pgfpathlineto{\pgfqpoint{1.176532in}{1.109624in}}%
\pgfpathlineto{\pgfqpoint{1.167152in}{1.105954in}}%
\pgfpathlineto{\pgfqpoint{1.164800in}{1.104465in}}%
\pgfpathlineto{\pgfqpoint{1.154459in}{1.094544in}}%
\pgfpathlineto{\pgfqpoint{1.153068in}{1.084263in}}%
\pgfpathlineto{\pgfqpoint{1.152949in}{1.083133in}}%
\pgfpathlineto{\pgfqpoint{1.153068in}{1.082998in}}%
\pgfpathlineto{\pgfqpoint{1.164148in}{1.071722in}}%
\pgfpathlineto{\pgfqpoint{1.164800in}{1.070966in}}%
\pgfpathlineto{\pgfqpoint{1.176532in}{1.063029in}}%
\pgfpathlineto{\pgfqpoint{1.182306in}{1.060311in}}%
\pgfpathlineto{\pgfqpoint{1.188264in}{1.055408in}}%
\pgfpathlineto{\pgfqpoint{1.199489in}{1.048901in}}%
\pgfpathlineto{\pgfqpoint{1.199996in}{1.048518in}}%
\pgfpathlineto{\pgfqpoint{1.211729in}{1.039824in}}%
\pgfpathlineto{\pgfqpoint{1.214920in}{1.037490in}}%
\pgfpathlineto{\pgfqpoint{1.223461in}{1.031631in}}%
\pgfpathlineto{\pgfqpoint{1.234921in}{1.026079in}}%
\pgfpathlineto{\pgfqpoint{1.235193in}{1.025929in}}%
\pgfpathlineto{\pgfqpoint{1.246925in}{1.020657in}}%
\pgfpathlineto{\pgfqpoint{1.258658in}{1.015838in}}%
\pgfpathlineto{\pgfqpoint{1.261583in}{1.014668in}}%
\pgfpathlineto{\pgfqpoint{1.270390in}{1.010971in}}%
\pgfpathlineto{\pgfqpoint{1.282122in}{1.005993in}}%
\pgfpathlineto{\pgfqpoint{1.288801in}{1.003257in}}%
\pgfpathclose%
\pgfpathmoveto{\pgfqpoint{1.256617in}{1.026079in}}%
\pgfpathlineto{\pgfqpoint{1.246925in}{1.029971in}}%
\pgfpathlineto{\pgfqpoint{1.235193in}{1.034833in}}%
\pgfpathlineto{\pgfqpoint{1.230075in}{1.037490in}}%
\pgfpathlineto{\pgfqpoint{1.223461in}{1.041182in}}%
\pgfpathlineto{\pgfqpoint{1.213126in}{1.048901in}}%
\pgfpathlineto{\pgfqpoint{1.211729in}{1.049452in}}%
\pgfpathlineto{\pgfqpoint{1.199996in}{1.054444in}}%
\pgfpathlineto{\pgfqpoint{1.191569in}{1.060311in}}%
\pgfpathlineto{\pgfqpoint{1.188264in}{1.062059in}}%
\pgfpathlineto{\pgfqpoint{1.176532in}{1.067703in}}%
\pgfpathlineto{\pgfqpoint{1.170772in}{1.071722in}}%
\pgfpathlineto{\pgfqpoint{1.164800in}{1.077861in}}%
\pgfpathlineto{\pgfqpoint{1.159648in}{1.083133in}}%
\pgfpathlineto{\pgfqpoint{1.161380in}{1.094544in}}%
\pgfpathlineto{\pgfqpoint{1.164800in}{1.097825in}}%
\pgfpathlineto{\pgfqpoint{1.176532in}{1.105271in}}%
\pgfpathlineto{\pgfqpoint{1.178411in}{1.105954in}}%
\pgfpathlineto{\pgfqpoint{1.188264in}{1.108443in}}%
\pgfpathlineto{\pgfqpoint{1.199996in}{1.108227in}}%
\pgfpathlineto{\pgfqpoint{1.211729in}{1.107239in}}%
\pgfpathlineto{\pgfqpoint{1.218931in}{1.105954in}}%
\pgfpathlineto{\pgfqpoint{1.223461in}{1.104916in}}%
\pgfpathlineto{\pgfqpoint{1.235193in}{1.101307in}}%
\pgfpathlineto{\pgfqpoint{1.246925in}{1.097052in}}%
\pgfpathlineto{\pgfqpoint{1.253638in}{1.094544in}}%
\pgfpathlineto{\pgfqpoint{1.258658in}{1.092378in}}%
\pgfpathlineto{\pgfqpoint{1.270390in}{1.087335in}}%
\pgfpathlineto{\pgfqpoint{1.280306in}{1.083133in}}%
\pgfpathlineto{\pgfqpoint{1.282122in}{1.081952in}}%
\pgfpathlineto{\pgfqpoint{1.293077in}{1.071722in}}%
\pgfpathlineto{\pgfqpoint{1.293854in}{1.070762in}}%
\pgfpathlineto{\pgfqpoint{1.304837in}{1.060311in}}%
\pgfpathlineto{\pgfqpoint{1.305586in}{1.059052in}}%
\pgfpathlineto{\pgfqpoint{1.313006in}{1.048901in}}%
\pgfpathlineto{\pgfqpoint{1.305586in}{1.039226in}}%
\pgfpathlineto{\pgfqpoint{1.299887in}{1.037490in}}%
\pgfpathlineto{\pgfqpoint{1.293854in}{1.031115in}}%
\pgfpathlineto{\pgfqpoint{1.289870in}{1.026079in}}%
\pgfpathlineto{\pgfqpoint{1.282122in}{1.021285in}}%
\pgfpathlineto{\pgfqpoint{1.270390in}{1.022521in}}%
\pgfpathlineto{\pgfqpoint{1.258658in}{1.025231in}}%
\pgfpathclose%
\pgfusepath{fill}%
\end{pgfscope}%
\begin{pgfscope}%
\pgfpathrectangle{\pgfqpoint{0.211875in}{0.211875in}}{\pgfqpoint{1.313625in}{1.279725in}}%
\pgfusepath{clip}%
\pgfsetbuttcap%
\pgfsetroundjoin%
\definecolor{currentfill}{rgb}{0.584229,0.109227,0.358485}%
\pgfsetfillcolor{currentfill}%
\pgfsetlinewidth{0.000000pt}%
\definecolor{currentstroke}{rgb}{0.000000,0.000000,0.000000}%
\pgfsetstrokecolor{currentstroke}%
\pgfsetdash{}{0pt}%
\pgfpathmoveto{\pgfqpoint{1.375980in}{1.078064in}}%
\pgfpathlineto{\pgfqpoint{1.387712in}{1.080719in}}%
\pgfpathlineto{\pgfqpoint{1.389843in}{1.083133in}}%
\pgfpathlineto{\pgfqpoint{1.390349in}{1.094544in}}%
\pgfpathlineto{\pgfqpoint{1.387712in}{1.099667in}}%
\pgfpathlineto{\pgfqpoint{1.380165in}{1.105954in}}%
\pgfpathlineto{\pgfqpoint{1.375980in}{1.108360in}}%
\pgfpathlineto{\pgfqpoint{1.364247in}{1.110731in}}%
\pgfpathlineto{\pgfqpoint{1.360259in}{1.105954in}}%
\pgfpathlineto{\pgfqpoint{1.360232in}{1.094544in}}%
\pgfpathlineto{\pgfqpoint{1.364247in}{1.083973in}}%
\pgfpathlineto{\pgfqpoint{1.364727in}{1.083133in}}%
\pgfpathclose%
\pgfusepath{fill}%
\end{pgfscope}%
\begin{pgfscope}%
\pgfpathrectangle{\pgfqpoint{0.211875in}{0.211875in}}{\pgfqpoint{1.313625in}{1.279725in}}%
\pgfusepath{clip}%
\pgfsetbuttcap%
\pgfsetroundjoin%
\definecolor{currentfill}{rgb}{0.584229,0.109227,0.358485}%
\pgfsetfillcolor{currentfill}%
\pgfsetlinewidth{0.000000pt}%
\definecolor{currentstroke}{rgb}{0.000000,0.000000,0.000000}%
\pgfsetstrokecolor{currentstroke}%
\pgfsetdash{}{0pt}%
\pgfpathmoveto{\pgfqpoint{1.176532in}{1.172362in}}%
\pgfpathlineto{\pgfqpoint{1.179881in}{1.174419in}}%
\pgfpathlineto{\pgfqpoint{1.179278in}{1.185830in}}%
\pgfpathlineto{\pgfqpoint{1.176532in}{1.187833in}}%
\pgfpathlineto{\pgfqpoint{1.164800in}{1.191925in}}%
\pgfpathlineto{\pgfqpoint{1.153968in}{1.185830in}}%
\pgfpathlineto{\pgfqpoint{1.164800in}{1.177515in}}%
\pgfpathlineto{\pgfqpoint{1.168617in}{1.174419in}}%
\pgfpathclose%
\pgfusepath{fill}%
\end{pgfscope}%
\begin{pgfscope}%
\pgfpathrectangle{\pgfqpoint{0.211875in}{0.211875in}}{\pgfqpoint{1.313625in}{1.279725in}}%
\pgfusepath{clip}%
\pgfsetbuttcap%
\pgfsetroundjoin%
\definecolor{currentfill}{rgb}{0.584229,0.109227,0.358485}%
\pgfsetfillcolor{currentfill}%
\pgfsetlinewidth{0.000000pt}%
\definecolor{currentstroke}{rgb}{0.000000,0.000000,0.000000}%
\pgfsetstrokecolor{currentstroke}%
\pgfsetdash{}{0pt}%
\pgfpathmoveto{\pgfqpoint{1.317319in}{1.207789in}}%
\pgfpathlineto{\pgfqpoint{1.318182in}{1.208651in}}%
\pgfpathlineto{\pgfqpoint{1.326740in}{1.220062in}}%
\pgfpathlineto{\pgfqpoint{1.329051in}{1.223535in}}%
\pgfpathlineto{\pgfqpoint{1.334784in}{1.231473in}}%
\pgfpathlineto{\pgfqpoint{1.340783in}{1.240813in}}%
\pgfpathlineto{\pgfqpoint{1.342395in}{1.242884in}}%
\pgfpathlineto{\pgfqpoint{1.346384in}{1.254295in}}%
\pgfpathlineto{\pgfqpoint{1.341787in}{1.265705in}}%
\pgfpathlineto{\pgfqpoint{1.340783in}{1.270709in}}%
\pgfpathlineto{\pgfqpoint{1.339639in}{1.277116in}}%
\pgfpathlineto{\pgfqpoint{1.333893in}{1.288527in}}%
\pgfpathlineto{\pgfqpoint{1.329051in}{1.296218in}}%
\pgfpathlineto{\pgfqpoint{1.327153in}{1.299938in}}%
\pgfpathlineto{\pgfqpoint{1.321547in}{1.311348in}}%
\pgfpathlineto{\pgfqpoint{1.317319in}{1.322246in}}%
\pgfpathlineto{\pgfqpoint{1.316380in}{1.322759in}}%
\pgfpathlineto{\pgfqpoint{1.305586in}{1.328894in}}%
\pgfpathlineto{\pgfqpoint{1.293854in}{1.333805in}}%
\pgfpathlineto{\pgfqpoint{1.282205in}{1.334170in}}%
\pgfpathlineto{\pgfqpoint{1.282122in}{1.334173in}}%
\pgfpathlineto{\pgfqpoint{1.282118in}{1.334170in}}%
\pgfpathlineto{\pgfqpoint{1.270390in}{1.324853in}}%
\pgfpathlineto{\pgfqpoint{1.268022in}{1.322759in}}%
\pgfpathlineto{\pgfqpoint{1.258658in}{1.313845in}}%
\pgfpathlineto{\pgfqpoint{1.256297in}{1.311348in}}%
\pgfpathlineto{\pgfqpoint{1.247108in}{1.299938in}}%
\pgfpathlineto{\pgfqpoint{1.246925in}{1.299265in}}%
\pgfpathlineto{\pgfqpoint{1.244828in}{1.288527in}}%
\pgfpathlineto{\pgfqpoint{1.243503in}{1.277116in}}%
\pgfpathlineto{\pgfqpoint{1.245826in}{1.265705in}}%
\pgfpathlineto{\pgfqpoint{1.246925in}{1.263644in}}%
\pgfpathlineto{\pgfqpoint{1.254598in}{1.254295in}}%
\pgfpathlineto{\pgfqpoint{1.258658in}{1.245693in}}%
\pgfpathlineto{\pgfqpoint{1.261177in}{1.242884in}}%
\pgfpathlineto{\pgfqpoint{1.269846in}{1.231473in}}%
\pgfpathlineto{\pgfqpoint{1.270390in}{1.230820in}}%
\pgfpathlineto{\pgfqpoint{1.282122in}{1.222085in}}%
\pgfpathlineto{\pgfqpoint{1.286997in}{1.220062in}}%
\pgfpathlineto{\pgfqpoint{1.293854in}{1.218393in}}%
\pgfpathlineto{\pgfqpoint{1.305586in}{1.216719in}}%
\pgfpathlineto{\pgfqpoint{1.316666in}{1.208651in}}%
\pgfpathclose%
\pgfusepath{fill}%
\end{pgfscope}%
\begin{pgfscope}%
\pgfpathrectangle{\pgfqpoint{0.211875in}{0.211875in}}{\pgfqpoint{1.313625in}{1.279725in}}%
\pgfusepath{clip}%
\pgfsetbuttcap%
\pgfsetroundjoin%
\definecolor{currentfill}{rgb}{0.584229,0.109227,0.358485}%
\pgfsetfillcolor{currentfill}%
\pgfsetlinewidth{0.000000pt}%
\definecolor{currentstroke}{rgb}{0.000000,0.000000,0.000000}%
\pgfsetstrokecolor{currentstroke}%
\pgfsetdash{}{0pt}%
\pgfpathmoveto{\pgfqpoint{1.446373in}{1.227825in}}%
\pgfpathlineto{\pgfqpoint{1.446373in}{1.231473in}}%
\pgfpathlineto{\pgfqpoint{1.446373in}{1.242884in}}%
\pgfpathlineto{\pgfqpoint{1.446373in}{1.254295in}}%
\pgfpathlineto{\pgfqpoint{1.446373in}{1.265705in}}%
\pgfpathlineto{\pgfqpoint{1.446373in}{1.277116in}}%
\pgfpathlineto{\pgfqpoint{1.446373in}{1.288527in}}%
\pgfpathlineto{\pgfqpoint{1.446373in}{1.299938in}}%
\pgfpathlineto{\pgfqpoint{1.446373in}{1.311348in}}%
\pgfpathlineto{\pgfqpoint{1.446373in}{1.320117in}}%
\pgfpathlineto{\pgfqpoint{1.434641in}{1.315104in}}%
\pgfpathlineto{\pgfqpoint{1.427232in}{1.311348in}}%
\pgfpathlineto{\pgfqpoint{1.422908in}{1.307865in}}%
\pgfpathlineto{\pgfqpoint{1.413728in}{1.299938in}}%
\pgfpathlineto{\pgfqpoint{1.411176in}{1.296370in}}%
\pgfpathlineto{\pgfqpoint{1.402257in}{1.288527in}}%
\pgfpathlineto{\pgfqpoint{1.399444in}{1.281446in}}%
\pgfpathlineto{\pgfqpoint{1.394173in}{1.277116in}}%
\pgfpathlineto{\pgfqpoint{1.399444in}{1.265844in}}%
\pgfpathlineto{\pgfqpoint{1.399492in}{1.265705in}}%
\pgfpathlineto{\pgfqpoint{1.411176in}{1.255202in}}%
\pgfpathlineto{\pgfqpoint{1.412055in}{1.254295in}}%
\pgfpathlineto{\pgfqpoint{1.422908in}{1.245895in}}%
\pgfpathlineto{\pgfqpoint{1.426168in}{1.242884in}}%
\pgfpathlineto{\pgfqpoint{1.434641in}{1.236702in}}%
\pgfpathlineto{\pgfqpoint{1.441140in}{1.231473in}}%
\pgfpathclose%
\pgfusepath{fill}%
\end{pgfscope}%
\begin{pgfscope}%
\pgfpathrectangle{\pgfqpoint{0.211875in}{0.211875in}}{\pgfqpoint{1.313625in}{1.279725in}}%
\pgfusepath{clip}%
\pgfsetbuttcap%
\pgfsetroundjoin%
\definecolor{currentfill}{rgb}{0.584229,0.109227,0.358485}%
\pgfsetfillcolor{currentfill}%
\pgfsetlinewidth{0.000000pt}%
\definecolor{currentstroke}{rgb}{0.000000,0.000000,0.000000}%
\pgfsetstrokecolor{currentstroke}%
\pgfsetdash{}{0pt}%
\pgfpathmoveto{\pgfqpoint{0.824566in}{1.261093in}}%
\pgfpathlineto{\pgfqpoint{0.835613in}{1.265705in}}%
\pgfpathlineto{\pgfqpoint{0.836298in}{1.266269in}}%
\pgfpathlineto{\pgfqpoint{0.847475in}{1.277116in}}%
\pgfpathlineto{\pgfqpoint{0.848030in}{1.278023in}}%
\pgfpathlineto{\pgfqpoint{0.854246in}{1.288527in}}%
\pgfpathlineto{\pgfqpoint{0.859762in}{1.292835in}}%
\pgfpathlineto{\pgfqpoint{0.871495in}{1.299371in}}%
\pgfpathlineto{\pgfqpoint{0.872620in}{1.299938in}}%
\pgfpathlineto{\pgfqpoint{0.883227in}{1.308464in}}%
\pgfpathlineto{\pgfqpoint{0.886645in}{1.311348in}}%
\pgfpathlineto{\pgfqpoint{0.893757in}{1.322759in}}%
\pgfpathlineto{\pgfqpoint{0.894959in}{1.328540in}}%
\pgfpathlineto{\pgfqpoint{0.896091in}{1.334170in}}%
\pgfpathlineto{\pgfqpoint{0.895079in}{1.345581in}}%
\pgfpathlineto{\pgfqpoint{0.894959in}{1.345957in}}%
\pgfpathlineto{\pgfqpoint{0.891565in}{1.356992in}}%
\pgfpathlineto{\pgfqpoint{0.887040in}{1.368402in}}%
\pgfpathlineto{\pgfqpoint{0.883227in}{1.371668in}}%
\pgfpathlineto{\pgfqpoint{0.871495in}{1.373519in}}%
\pgfpathlineto{\pgfqpoint{0.859762in}{1.373694in}}%
\pgfpathlineto{\pgfqpoint{0.848030in}{1.373495in}}%
\pgfpathlineto{\pgfqpoint{0.836298in}{1.372771in}}%
\pgfpathlineto{\pgfqpoint{0.824566in}{1.371271in}}%
\pgfpathlineto{\pgfqpoint{0.812834in}{1.368534in}}%
\pgfpathlineto{\pgfqpoint{0.812465in}{1.368402in}}%
\pgfpathlineto{\pgfqpoint{0.801101in}{1.361564in}}%
\pgfpathlineto{\pgfqpoint{0.797079in}{1.356992in}}%
\pgfpathlineto{\pgfqpoint{0.794689in}{1.345581in}}%
\pgfpathlineto{\pgfqpoint{0.797054in}{1.334170in}}%
\pgfpathlineto{\pgfqpoint{0.801101in}{1.324439in}}%
\pgfpathlineto{\pgfqpoint{0.801748in}{1.322759in}}%
\pgfpathlineto{\pgfqpoint{0.808285in}{1.311348in}}%
\pgfpathlineto{\pgfqpoint{0.812834in}{1.300836in}}%
\pgfpathlineto{\pgfqpoint{0.813289in}{1.299938in}}%
\pgfpathlineto{\pgfqpoint{0.812834in}{1.299050in}}%
\pgfpathlineto{\pgfqpoint{0.809929in}{1.288527in}}%
\pgfpathlineto{\pgfqpoint{0.811032in}{1.277116in}}%
\pgfpathlineto{\pgfqpoint{0.812834in}{1.268648in}}%
\pgfpathlineto{\pgfqpoint{0.814269in}{1.265705in}}%
\pgfpathclose%
\pgfusepath{fill}%
\end{pgfscope}%
\begin{pgfscope}%
\pgfpathrectangle{\pgfqpoint{0.211875in}{0.211875in}}{\pgfqpoint{1.313625in}{1.279725in}}%
\pgfusepath{clip}%
\pgfsetbuttcap%
\pgfsetroundjoin%
\definecolor{currentfill}{rgb}{0.730358,0.086862,0.337485}%
\pgfsetfillcolor{currentfill}%
\pgfsetlinewidth{0.000000pt}%
\definecolor{currentstroke}{rgb}{0.000000,0.000000,0.000000}%
\pgfsetstrokecolor{currentstroke}%
\pgfsetdash{}{0pt}%
\pgfpathmoveto{\pgfqpoint{1.012281in}{0.292370in}}%
\pgfpathlineto{\pgfqpoint{1.015471in}{0.284378in}}%
\pgfpathlineto{\pgfqpoint{1.024013in}{0.284378in}}%
\pgfpathlineto{\pgfqpoint{1.035746in}{0.284378in}}%
\pgfpathlineto{\pgfqpoint{1.036433in}{0.284378in}}%
\pgfpathlineto{\pgfqpoint{1.035746in}{0.285470in}}%
\pgfpathlineto{\pgfqpoint{1.031452in}{0.295789in}}%
\pgfpathlineto{\pgfqpoint{1.027064in}{0.307200in}}%
\pgfpathlineto{\pgfqpoint{1.024013in}{0.314423in}}%
\pgfpathlineto{\pgfqpoint{1.022101in}{0.318611in}}%
\pgfpathlineto{\pgfqpoint{1.017253in}{0.330022in}}%
\pgfpathlineto{\pgfqpoint{1.012883in}{0.341432in}}%
\pgfpathlineto{\pgfqpoint{1.012281in}{0.343009in}}%
\pgfpathlineto{\pgfqpoint{1.007766in}{0.352843in}}%
\pgfpathlineto{\pgfqpoint{1.003108in}{0.364254in}}%
\pgfpathlineto{\pgfqpoint{1.000549in}{0.370859in}}%
\pgfpathlineto{\pgfqpoint{0.998254in}{0.375665in}}%
\pgfpathlineto{\pgfqpoint{0.993228in}{0.387075in}}%
\pgfpathlineto{\pgfqpoint{0.988817in}{0.398245in}}%
\pgfpathlineto{\pgfqpoint{0.988699in}{0.398486in}}%
\pgfpathlineto{\pgfqpoint{0.983285in}{0.409897in}}%
\pgfpathlineto{\pgfqpoint{0.978535in}{0.421308in}}%
\pgfpathlineto{\pgfqpoint{0.977085in}{0.424916in}}%
\pgfpathlineto{\pgfqpoint{0.973366in}{0.432719in}}%
\pgfpathlineto{\pgfqpoint{0.968390in}{0.444129in}}%
\pgfpathlineto{\pgfqpoint{0.965352in}{0.451586in}}%
\pgfpathlineto{\pgfqpoint{0.963563in}{0.455540in}}%
\pgfpathlineto{\pgfqpoint{0.958537in}{0.466951in}}%
\pgfpathlineto{\pgfqpoint{0.953897in}{0.478362in}}%
\pgfpathlineto{\pgfqpoint{0.953620in}{0.479045in}}%
\pgfpathlineto{\pgfqpoint{0.949223in}{0.489772in}}%
\pgfpathlineto{\pgfqpoint{0.944670in}{0.501183in}}%
\pgfpathlineto{\pgfqpoint{0.941888in}{0.508273in}}%
\pgfpathlineto{\pgfqpoint{0.940362in}{0.512594in}}%
\pgfpathlineto{\pgfqpoint{0.936019in}{0.524005in}}%
\pgfpathlineto{\pgfqpoint{0.930462in}{0.535416in}}%
\pgfpathlineto{\pgfqpoint{0.930156in}{0.535973in}}%
\pgfpathlineto{\pgfqpoint{0.924818in}{0.546826in}}%
\pgfpathlineto{\pgfqpoint{0.920133in}{0.558237in}}%
\pgfpathlineto{\pgfqpoint{0.918424in}{0.562989in}}%
\pgfpathlineto{\pgfqpoint{0.916513in}{0.569648in}}%
\pgfpathlineto{\pgfqpoint{0.913184in}{0.581059in}}%
\pgfpathlineto{\pgfqpoint{0.909767in}{0.592469in}}%
\pgfpathlineto{\pgfqpoint{0.906691in}{0.602454in}}%
\pgfpathlineto{\pgfqpoint{0.906295in}{0.603880in}}%
\pgfpathlineto{\pgfqpoint{0.903112in}{0.615291in}}%
\pgfpathlineto{\pgfqpoint{0.899885in}{0.626702in}}%
\pgfpathlineto{\pgfqpoint{0.896575in}{0.638113in}}%
\pgfpathlineto{\pgfqpoint{0.894959in}{0.643530in}}%
\pgfpathlineto{\pgfqpoint{0.893325in}{0.649523in}}%
\pgfpathlineto{\pgfqpoint{0.890230in}{0.660934in}}%
\pgfpathlineto{\pgfqpoint{0.887107in}{0.672345in}}%
\pgfpathlineto{\pgfqpoint{0.883905in}{0.683756in}}%
\pgfpathlineto{\pgfqpoint{0.883227in}{0.686426in}}%
\pgfpathlineto{\pgfqpoint{0.881221in}{0.695166in}}%
\pgfpathlineto{\pgfqpoint{0.878647in}{0.706577in}}%
\pgfpathlineto{\pgfqpoint{0.876070in}{0.717988in}}%
\pgfpathlineto{\pgfqpoint{0.873714in}{0.729399in}}%
\pgfpathlineto{\pgfqpoint{0.871710in}{0.740810in}}%
\pgfpathlineto{\pgfqpoint{0.871495in}{0.742874in}}%
\pgfpathlineto{\pgfqpoint{0.870724in}{0.752220in}}%
\pgfpathlineto{\pgfqpoint{0.870423in}{0.763631in}}%
\pgfpathlineto{\pgfqpoint{0.870823in}{0.775042in}}%
\pgfpathlineto{\pgfqpoint{0.871288in}{0.786453in}}%
\pgfpathlineto{\pgfqpoint{0.871495in}{0.791479in}}%
\pgfpathlineto{\pgfqpoint{0.872024in}{0.797863in}}%
\pgfpathlineto{\pgfqpoint{0.874510in}{0.809274in}}%
\pgfpathlineto{\pgfqpoint{0.877553in}{0.820685in}}%
\pgfpathlineto{\pgfqpoint{0.882919in}{0.832096in}}%
\pgfpathlineto{\pgfqpoint{0.883227in}{0.832414in}}%
\pgfpathlineto{\pgfqpoint{0.894959in}{0.832696in}}%
\pgfpathlineto{\pgfqpoint{0.896262in}{0.832096in}}%
\pgfpathlineto{\pgfqpoint{0.906691in}{0.827420in}}%
\pgfpathlineto{\pgfqpoint{0.918424in}{0.825100in}}%
\pgfpathlineto{\pgfqpoint{0.928860in}{0.832096in}}%
\pgfpathlineto{\pgfqpoint{0.926927in}{0.843507in}}%
\pgfpathlineto{\pgfqpoint{0.922386in}{0.854917in}}%
\pgfpathlineto{\pgfqpoint{0.918424in}{0.861480in}}%
\pgfpathlineto{\pgfqpoint{0.916867in}{0.866328in}}%
\pgfpathlineto{\pgfqpoint{0.913455in}{0.877739in}}%
\pgfpathlineto{\pgfqpoint{0.908249in}{0.889150in}}%
\pgfpathlineto{\pgfqpoint{0.906691in}{0.891619in}}%
\pgfpathlineto{\pgfqpoint{0.902466in}{0.900560in}}%
\pgfpathlineto{\pgfqpoint{0.894959in}{0.911667in}}%
\pgfpathlineto{\pgfqpoint{0.894724in}{0.911971in}}%
\pgfpathlineto{\pgfqpoint{0.883227in}{0.918978in}}%
\pgfpathlineto{\pgfqpoint{0.871740in}{0.923382in}}%
\pgfpathlineto{\pgfqpoint{0.871495in}{0.923467in}}%
\pgfpathlineto{\pgfqpoint{0.859762in}{0.930805in}}%
\pgfpathlineto{\pgfqpoint{0.855002in}{0.934793in}}%
\pgfpathlineto{\pgfqpoint{0.859762in}{0.943814in}}%
\pgfpathlineto{\pgfqpoint{0.860490in}{0.946204in}}%
\pgfpathlineto{\pgfqpoint{0.863675in}{0.957614in}}%
\pgfpathlineto{\pgfqpoint{0.867900in}{0.969025in}}%
\pgfpathlineto{\pgfqpoint{0.871495in}{0.974705in}}%
\pgfpathlineto{\pgfqpoint{0.876211in}{0.980436in}}%
\pgfpathlineto{\pgfqpoint{0.871495in}{0.988142in}}%
\pgfpathlineto{\pgfqpoint{0.868831in}{0.991847in}}%
\pgfpathlineto{\pgfqpoint{0.861050in}{1.003257in}}%
\pgfpathlineto{\pgfqpoint{0.859762in}{1.004598in}}%
\pgfpathlineto{\pgfqpoint{0.852198in}{1.014668in}}%
\pgfpathlineto{\pgfqpoint{0.848030in}{1.018979in}}%
\pgfpathlineto{\pgfqpoint{0.841954in}{1.026079in}}%
\pgfpathlineto{\pgfqpoint{0.836298in}{1.031849in}}%
\pgfpathlineto{\pgfqpoint{0.831520in}{1.037490in}}%
\pgfpathlineto{\pgfqpoint{0.824566in}{1.048010in}}%
\pgfpathlineto{\pgfqpoint{0.815207in}{1.048901in}}%
\pgfpathlineto{\pgfqpoint{0.812834in}{1.049027in}}%
\pgfpathlineto{\pgfqpoint{0.801101in}{1.050831in}}%
\pgfpathlineto{\pgfqpoint{0.789369in}{1.059020in}}%
\pgfpathlineto{\pgfqpoint{0.788021in}{1.060311in}}%
\pgfpathlineto{\pgfqpoint{0.777637in}{1.070996in}}%
\pgfpathlineto{\pgfqpoint{0.776963in}{1.071722in}}%
\pgfpathlineto{\pgfqpoint{0.767048in}{1.083133in}}%
\pgfpathlineto{\pgfqpoint{0.765905in}{1.087430in}}%
\pgfpathlineto{\pgfqpoint{0.763468in}{1.094544in}}%
\pgfpathlineto{\pgfqpoint{0.765905in}{1.103043in}}%
\pgfpathlineto{\pgfqpoint{0.766302in}{1.105954in}}%
\pgfpathlineto{\pgfqpoint{0.765905in}{1.106781in}}%
\pgfpathlineto{\pgfqpoint{0.756162in}{1.117365in}}%
\pgfpathlineto{\pgfqpoint{0.754173in}{1.119298in}}%
\pgfpathlineto{\pgfqpoint{0.742440in}{1.124230in}}%
\pgfpathlineto{\pgfqpoint{0.732307in}{1.128776in}}%
\pgfpathlineto{\pgfqpoint{0.730708in}{1.129605in}}%
\pgfpathlineto{\pgfqpoint{0.718976in}{1.138369in}}%
\pgfpathlineto{\pgfqpoint{0.716544in}{1.140187in}}%
\pgfpathlineto{\pgfqpoint{0.707244in}{1.147495in}}%
\pgfpathlineto{\pgfqpoint{0.702014in}{1.151598in}}%
\pgfpathlineto{\pgfqpoint{0.695512in}{1.157664in}}%
\pgfpathlineto{\pgfqpoint{0.688917in}{1.163008in}}%
\pgfpathlineto{\pgfqpoint{0.683779in}{1.168122in}}%
\pgfpathlineto{\pgfqpoint{0.677033in}{1.174419in}}%
\pgfpathlineto{\pgfqpoint{0.672047in}{1.180259in}}%
\pgfpathlineto{\pgfqpoint{0.669477in}{1.185830in}}%
\pgfpathlineto{\pgfqpoint{0.667567in}{1.197241in}}%
\pgfpathlineto{\pgfqpoint{0.668250in}{1.208651in}}%
\pgfpathlineto{\pgfqpoint{0.670382in}{1.220062in}}%
\pgfpathlineto{\pgfqpoint{0.669752in}{1.231473in}}%
\pgfpathlineto{\pgfqpoint{0.666964in}{1.242884in}}%
\pgfpathlineto{\pgfqpoint{0.663549in}{1.254295in}}%
\pgfpathlineto{\pgfqpoint{0.660334in}{1.265705in}}%
\pgfpathlineto{\pgfqpoint{0.660315in}{1.265780in}}%
\pgfpathlineto{\pgfqpoint{0.657216in}{1.277116in}}%
\pgfpathlineto{\pgfqpoint{0.654615in}{1.288527in}}%
\pgfpathlineto{\pgfqpoint{0.652838in}{1.299938in}}%
\pgfpathlineto{\pgfqpoint{0.657140in}{1.311348in}}%
\pgfpathlineto{\pgfqpoint{0.660315in}{1.312646in}}%
\pgfpathlineto{\pgfqpoint{0.665340in}{1.311348in}}%
\pgfpathlineto{\pgfqpoint{0.672047in}{1.309717in}}%
\pgfpathlineto{\pgfqpoint{0.683779in}{1.301300in}}%
\pgfpathlineto{\pgfqpoint{0.685251in}{1.299938in}}%
\pgfpathlineto{\pgfqpoint{0.695512in}{1.290632in}}%
\pgfpathlineto{\pgfqpoint{0.697838in}{1.288527in}}%
\pgfpathlineto{\pgfqpoint{0.707244in}{1.280164in}}%
\pgfpathlineto{\pgfqpoint{0.710682in}{1.277116in}}%
\pgfpathlineto{\pgfqpoint{0.718976in}{1.269870in}}%
\pgfpathlineto{\pgfqpoint{0.723758in}{1.265705in}}%
\pgfpathlineto{\pgfqpoint{0.730708in}{1.259720in}}%
\pgfpathlineto{\pgfqpoint{0.735599in}{1.254295in}}%
\pgfpathlineto{\pgfqpoint{0.742440in}{1.244502in}}%
\pgfpathlineto{\pgfqpoint{0.743549in}{1.242884in}}%
\pgfpathlineto{\pgfqpoint{0.751097in}{1.231473in}}%
\pgfpathlineto{\pgfqpoint{0.754173in}{1.227005in}}%
\pgfpathlineto{\pgfqpoint{0.758808in}{1.220062in}}%
\pgfpathlineto{\pgfqpoint{0.765905in}{1.209711in}}%
\pgfpathlineto{\pgfqpoint{0.766620in}{1.208651in}}%
\pgfpathlineto{\pgfqpoint{0.774106in}{1.197241in}}%
\pgfpathlineto{\pgfqpoint{0.777637in}{1.190673in}}%
\pgfpathlineto{\pgfqpoint{0.779719in}{1.185830in}}%
\pgfpathlineto{\pgfqpoint{0.784244in}{1.174419in}}%
\pgfpathlineto{\pgfqpoint{0.787622in}{1.163008in}}%
\pgfpathlineto{\pgfqpoint{0.789369in}{1.157854in}}%
\pgfpathlineto{\pgfqpoint{0.791343in}{1.151598in}}%
\pgfpathlineto{\pgfqpoint{0.795969in}{1.140187in}}%
\pgfpathlineto{\pgfqpoint{0.801101in}{1.132621in}}%
\pgfpathlineto{\pgfqpoint{0.803864in}{1.128776in}}%
\pgfpathlineto{\pgfqpoint{0.812834in}{1.121979in}}%
\pgfpathlineto{\pgfqpoint{0.824566in}{1.125589in}}%
\pgfpathlineto{\pgfqpoint{0.836298in}{1.127077in}}%
\pgfpathlineto{\pgfqpoint{0.848030in}{1.125974in}}%
\pgfpathlineto{\pgfqpoint{0.859762in}{1.124789in}}%
\pgfpathlineto{\pgfqpoint{0.871495in}{1.123533in}}%
\pgfpathlineto{\pgfqpoint{0.883227in}{1.122175in}}%
\pgfpathlineto{\pgfqpoint{0.894959in}{1.120729in}}%
\pgfpathlineto{\pgfqpoint{0.906691in}{1.119182in}}%
\pgfpathlineto{\pgfqpoint{0.917586in}{1.117365in}}%
\pgfpathlineto{\pgfqpoint{0.918424in}{1.117189in}}%
\pgfpathlineto{\pgfqpoint{0.930156in}{1.112662in}}%
\pgfpathlineto{\pgfqpoint{0.941888in}{1.107531in}}%
\pgfpathlineto{\pgfqpoint{0.945508in}{1.105954in}}%
\pgfpathlineto{\pgfqpoint{0.953620in}{1.103927in}}%
\pgfpathlineto{\pgfqpoint{0.965352in}{1.101235in}}%
\pgfpathlineto{\pgfqpoint{0.977085in}{1.098609in}}%
\pgfpathlineto{\pgfqpoint{0.988817in}{1.095961in}}%
\pgfpathlineto{\pgfqpoint{1.000549in}{1.094776in}}%
\pgfpathlineto{\pgfqpoint{1.005220in}{1.094544in}}%
\pgfpathlineto{\pgfqpoint{1.012281in}{1.094138in}}%
\pgfpathlineto{\pgfqpoint{1.024013in}{1.093494in}}%
\pgfpathlineto{\pgfqpoint{1.035746in}{1.092794in}}%
\pgfpathlineto{\pgfqpoint{1.047478in}{1.093249in}}%
\pgfpathlineto{\pgfqpoint{1.059210in}{1.093997in}}%
\pgfpathlineto{\pgfqpoint{1.060921in}{1.094544in}}%
\pgfpathlineto{\pgfqpoint{1.068000in}{1.105954in}}%
\pgfpathlineto{\pgfqpoint{1.059210in}{1.112303in}}%
\pgfpathlineto{\pgfqpoint{1.052724in}{1.117365in}}%
\pgfpathlineto{\pgfqpoint{1.047478in}{1.121505in}}%
\pgfpathlineto{\pgfqpoint{1.036632in}{1.128776in}}%
\pgfpathlineto{\pgfqpoint{1.035746in}{1.129451in}}%
\pgfpathlineto{\pgfqpoint{1.024013in}{1.139264in}}%
\pgfpathlineto{\pgfqpoint{1.023008in}{1.140187in}}%
\pgfpathlineto{\pgfqpoint{1.012281in}{1.148193in}}%
\pgfpathlineto{\pgfqpoint{1.006878in}{1.151598in}}%
\pgfpathlineto{\pgfqpoint{1.000549in}{1.155099in}}%
\pgfpathlineto{\pgfqpoint{0.988817in}{1.161612in}}%
\pgfpathlineto{\pgfqpoint{0.986337in}{1.163008in}}%
\pgfpathlineto{\pgfqpoint{0.977085in}{1.168015in}}%
\pgfpathlineto{\pgfqpoint{0.965384in}{1.174419in}}%
\pgfpathlineto{\pgfqpoint{0.965352in}{1.174444in}}%
\pgfpathlineto{\pgfqpoint{0.953620in}{1.180389in}}%
\pgfpathlineto{\pgfqpoint{0.941888in}{1.185614in}}%
\pgfpathlineto{\pgfqpoint{0.941429in}{1.185830in}}%
\pgfpathlineto{\pgfqpoint{0.930156in}{1.191993in}}%
\pgfpathlineto{\pgfqpoint{0.918424in}{1.196542in}}%
\pgfpathlineto{\pgfqpoint{0.916925in}{1.197241in}}%
\pgfpathlineto{\pgfqpoint{0.906691in}{1.203620in}}%
\pgfpathlineto{\pgfqpoint{0.894959in}{1.207469in}}%
\pgfpathlineto{\pgfqpoint{0.892418in}{1.208651in}}%
\pgfpathlineto{\pgfqpoint{0.883227in}{1.215334in}}%
\pgfpathlineto{\pgfqpoint{0.871495in}{1.219130in}}%
\pgfpathlineto{\pgfqpoint{0.869313in}{1.220062in}}%
\pgfpathlineto{\pgfqpoint{0.859762in}{1.227215in}}%
\pgfpathlineto{\pgfqpoint{0.848030in}{1.230855in}}%
\pgfpathlineto{\pgfqpoint{0.846521in}{1.231473in}}%
\pgfpathlineto{\pgfqpoint{0.836298in}{1.239153in}}%
\pgfpathlineto{\pgfqpoint{0.824566in}{1.241497in}}%
\pgfpathlineto{\pgfqpoint{0.821571in}{1.242884in}}%
\pgfpathlineto{\pgfqpoint{0.824566in}{1.244523in}}%
\pgfpathlineto{\pgfqpoint{0.836298in}{1.245815in}}%
\pgfpathlineto{\pgfqpoint{0.848030in}{1.250139in}}%
\pgfpathlineto{\pgfqpoint{0.854713in}{1.254295in}}%
\pgfpathlineto{\pgfqpoint{0.859762in}{1.257970in}}%
\pgfpathlineto{\pgfqpoint{0.870703in}{1.265705in}}%
\pgfpathlineto{\pgfqpoint{0.871495in}{1.266434in}}%
\pgfpathlineto{\pgfqpoint{0.883014in}{1.277116in}}%
\pgfpathlineto{\pgfqpoint{0.883227in}{1.277326in}}%
\pgfpathlineto{\pgfqpoint{0.894959in}{1.286388in}}%
\pgfpathlineto{\pgfqpoint{0.898858in}{1.288527in}}%
\pgfpathlineto{\pgfqpoint{0.906691in}{1.292894in}}%
\pgfpathlineto{\pgfqpoint{0.918424in}{1.299601in}}%
\pgfpathlineto{\pgfqpoint{0.919002in}{1.299938in}}%
\pgfpathlineto{\pgfqpoint{0.930156in}{1.310407in}}%
\pgfpathlineto{\pgfqpoint{0.931880in}{1.311348in}}%
\pgfpathlineto{\pgfqpoint{0.941888in}{1.317632in}}%
\pgfpathlineto{\pgfqpoint{0.953620in}{1.318926in}}%
\pgfpathlineto{\pgfqpoint{0.965352in}{1.316320in}}%
\pgfpathlineto{\pgfqpoint{0.977085in}{1.313820in}}%
\pgfpathlineto{\pgfqpoint{0.988817in}{1.311464in}}%
\pgfpathlineto{\pgfqpoint{0.989456in}{1.311348in}}%
\pgfpathlineto{\pgfqpoint{1.000549in}{1.309910in}}%
\pgfpathlineto{\pgfqpoint{1.012281in}{1.308584in}}%
\pgfpathlineto{\pgfqpoint{1.024013in}{1.308131in}}%
\pgfpathlineto{\pgfqpoint{1.032930in}{1.311348in}}%
\pgfpathlineto{\pgfqpoint{1.035746in}{1.314765in}}%
\pgfpathlineto{\pgfqpoint{1.041095in}{1.322759in}}%
\pgfpathlineto{\pgfqpoint{1.045420in}{1.334170in}}%
\pgfpathlineto{\pgfqpoint{1.046410in}{1.345581in}}%
\pgfpathlineto{\pgfqpoint{1.043801in}{1.356992in}}%
\pgfpathlineto{\pgfqpoint{1.035826in}{1.368402in}}%
\pgfpathlineto{\pgfqpoint{1.035746in}{1.368492in}}%
\pgfpathlineto{\pgfqpoint{1.024573in}{1.379813in}}%
\pgfpathlineto{\pgfqpoint{1.024013in}{1.380315in}}%
\pgfpathlineto{\pgfqpoint{1.012281in}{1.390112in}}%
\pgfpathlineto{\pgfqpoint{1.010865in}{1.391224in}}%
\pgfpathlineto{\pgfqpoint{1.000549in}{1.398848in}}%
\pgfpathlineto{\pgfqpoint{0.995130in}{1.402635in}}%
\pgfpathlineto{\pgfqpoint{0.988817in}{1.406901in}}%
\pgfpathlineto{\pgfqpoint{0.977684in}{1.414045in}}%
\pgfpathlineto{\pgfqpoint{0.977085in}{1.414045in}}%
\pgfpathlineto{\pgfqpoint{0.965352in}{1.414045in}}%
\pgfpathlineto{\pgfqpoint{0.953620in}{1.414045in}}%
\pgfpathlineto{\pgfqpoint{0.941888in}{1.414045in}}%
\pgfpathlineto{\pgfqpoint{0.930156in}{1.414045in}}%
\pgfpathlineto{\pgfqpoint{0.918424in}{1.414045in}}%
\pgfpathlineto{\pgfqpoint{0.906691in}{1.414045in}}%
\pgfpathlineto{\pgfqpoint{0.894959in}{1.414045in}}%
\pgfpathlineto{\pgfqpoint{0.883227in}{1.414045in}}%
\pgfpathlineto{\pgfqpoint{0.871495in}{1.414045in}}%
\pgfpathlineto{\pgfqpoint{0.866303in}{1.414045in}}%
\pgfpathlineto{\pgfqpoint{0.859762in}{1.411990in}}%
\pgfpathlineto{\pgfqpoint{0.848030in}{1.412667in}}%
\pgfpathlineto{\pgfqpoint{0.836298in}{1.413742in}}%
\pgfpathlineto{\pgfqpoint{0.832473in}{1.414045in}}%
\pgfpathlineto{\pgfqpoint{0.824566in}{1.414045in}}%
\pgfpathlineto{\pgfqpoint{0.812834in}{1.414045in}}%
\pgfpathlineto{\pgfqpoint{0.801101in}{1.414045in}}%
\pgfpathlineto{\pgfqpoint{0.789369in}{1.414045in}}%
\pgfpathlineto{\pgfqpoint{0.777637in}{1.414045in}}%
\pgfpathlineto{\pgfqpoint{0.765905in}{1.414045in}}%
\pgfpathlineto{\pgfqpoint{0.755310in}{1.414045in}}%
\pgfpathlineto{\pgfqpoint{0.754173in}{1.413853in}}%
\pgfpathlineto{\pgfqpoint{0.742440in}{1.410087in}}%
\pgfpathlineto{\pgfqpoint{0.731846in}{1.402635in}}%
\pgfpathlineto{\pgfqpoint{0.731766in}{1.391224in}}%
\pgfpathlineto{\pgfqpoint{0.736399in}{1.379813in}}%
\pgfpathlineto{\pgfqpoint{0.742316in}{1.368402in}}%
\pgfpathlineto{\pgfqpoint{0.742440in}{1.368221in}}%
\pgfpathlineto{\pgfqpoint{0.749829in}{1.356992in}}%
\pgfpathlineto{\pgfqpoint{0.754173in}{1.350862in}}%
\pgfpathlineto{\pgfqpoint{0.757758in}{1.345581in}}%
\pgfpathlineto{\pgfqpoint{0.765905in}{1.334333in}}%
\pgfpathlineto{\pgfqpoint{0.766021in}{1.334170in}}%
\pgfpathlineto{\pgfqpoint{0.775020in}{1.322759in}}%
\pgfpathlineto{\pgfqpoint{0.777637in}{1.319468in}}%
\pgfpathlineto{\pgfqpoint{0.784037in}{1.311348in}}%
\pgfpathlineto{\pgfqpoint{0.787581in}{1.299938in}}%
\pgfpathlineto{\pgfqpoint{0.789369in}{1.293440in}}%
\pgfpathlineto{\pgfqpoint{0.790733in}{1.288527in}}%
\pgfpathlineto{\pgfqpoint{0.794323in}{1.277116in}}%
\pgfpathlineto{\pgfqpoint{0.797957in}{1.265705in}}%
\pgfpathlineto{\pgfqpoint{0.801101in}{1.255981in}}%
\pgfpathlineto{\pgfqpoint{0.801717in}{1.254295in}}%
\pgfpathlineto{\pgfqpoint{0.801101in}{1.253497in}}%
\pgfpathlineto{\pgfqpoint{0.798487in}{1.254295in}}%
\pgfpathlineto{\pgfqpoint{0.789369in}{1.256363in}}%
\pgfpathlineto{\pgfqpoint{0.777637in}{1.256392in}}%
\pgfpathlineto{\pgfqpoint{0.765905in}{1.261821in}}%
\pgfpathlineto{\pgfqpoint{0.762536in}{1.265705in}}%
\pgfpathlineto{\pgfqpoint{0.754173in}{1.271501in}}%
\pgfpathlineto{\pgfqpoint{0.750349in}{1.277116in}}%
\pgfpathlineto{\pgfqpoint{0.743042in}{1.288527in}}%
\pgfpathlineto{\pgfqpoint{0.742440in}{1.289364in}}%
\pgfpathlineto{\pgfqpoint{0.735079in}{1.299938in}}%
\pgfpathlineto{\pgfqpoint{0.730708in}{1.306317in}}%
\pgfpathlineto{\pgfqpoint{0.727157in}{1.311348in}}%
\pgfpathlineto{\pgfqpoint{0.719232in}{1.322759in}}%
\pgfpathlineto{\pgfqpoint{0.718976in}{1.323114in}}%
\pgfpathlineto{\pgfqpoint{0.710731in}{1.334170in}}%
\pgfpathlineto{\pgfqpoint{0.707244in}{1.339270in}}%
\pgfpathlineto{\pgfqpoint{0.702345in}{1.345581in}}%
\pgfpathlineto{\pgfqpoint{0.695512in}{1.355980in}}%
\pgfpathlineto{\pgfqpoint{0.694704in}{1.356992in}}%
\pgfpathlineto{\pgfqpoint{0.691491in}{1.368402in}}%
\pgfpathlineto{\pgfqpoint{0.688830in}{1.379813in}}%
\pgfpathlineto{\pgfqpoint{0.685531in}{1.391224in}}%
\pgfpathlineto{\pgfqpoint{0.683779in}{1.395690in}}%
\pgfpathlineto{\pgfqpoint{0.680833in}{1.402635in}}%
\pgfpathlineto{\pgfqpoint{0.674652in}{1.414045in}}%
\pgfpathlineto{\pgfqpoint{0.672047in}{1.414045in}}%
\pgfpathlineto{\pgfqpoint{0.660315in}{1.414045in}}%
\pgfpathlineto{\pgfqpoint{0.648583in}{1.414045in}}%
\pgfpathlineto{\pgfqpoint{0.646195in}{1.414045in}}%
\pgfpathlineto{\pgfqpoint{0.648583in}{1.410927in}}%
\pgfpathlineto{\pgfqpoint{0.654580in}{1.402635in}}%
\pgfpathlineto{\pgfqpoint{0.660315in}{1.393371in}}%
\pgfpathlineto{\pgfqpoint{0.661563in}{1.391224in}}%
\pgfpathlineto{\pgfqpoint{0.669596in}{1.379813in}}%
\pgfpathlineto{\pgfqpoint{0.672047in}{1.376548in}}%
\pgfpathlineto{\pgfqpoint{0.677710in}{1.368402in}}%
\pgfpathlineto{\pgfqpoint{0.683779in}{1.360338in}}%
\pgfpathlineto{\pgfqpoint{0.685981in}{1.356992in}}%
\pgfpathlineto{\pgfqpoint{0.695181in}{1.345581in}}%
\pgfpathlineto{\pgfqpoint{0.695512in}{1.345189in}}%
\pgfpathlineto{\pgfqpoint{0.704334in}{1.334170in}}%
\pgfpathlineto{\pgfqpoint{0.707244in}{1.330342in}}%
\pgfpathlineto{\pgfqpoint{0.712890in}{1.322759in}}%
\pgfpathlineto{\pgfqpoint{0.718976in}{1.314301in}}%
\pgfpathlineto{\pgfqpoint{0.721027in}{1.311348in}}%
\pgfpathlineto{\pgfqpoint{0.729068in}{1.299938in}}%
\pgfpathlineto{\pgfqpoint{0.730708in}{1.297233in}}%
\pgfpathlineto{\pgfqpoint{0.736419in}{1.288527in}}%
\pgfpathlineto{\pgfqpoint{0.741619in}{1.277116in}}%
\pgfpathlineto{\pgfqpoint{0.742440in}{1.273781in}}%
\pgfpathlineto{\pgfqpoint{0.745642in}{1.265705in}}%
\pgfpathlineto{\pgfqpoint{0.754173in}{1.256369in}}%
\pgfpathlineto{\pgfqpoint{0.756025in}{1.254295in}}%
\pgfpathlineto{\pgfqpoint{0.765905in}{1.243446in}}%
\pgfpathlineto{\pgfqpoint{0.766594in}{1.242884in}}%
\pgfpathlineto{\pgfqpoint{0.776782in}{1.231473in}}%
\pgfpathlineto{\pgfqpoint{0.777637in}{1.230546in}}%
\pgfpathlineto{\pgfqpoint{0.789369in}{1.225188in}}%
\pgfpathlineto{\pgfqpoint{0.801101in}{1.224799in}}%
\pgfpathlineto{\pgfqpoint{0.812834in}{1.223633in}}%
\pgfpathlineto{\pgfqpoint{0.824566in}{1.222439in}}%
\pgfpathlineto{\pgfqpoint{0.836298in}{1.220616in}}%
\pgfpathlineto{\pgfqpoint{0.847133in}{1.220062in}}%
\pgfpathlineto{\pgfqpoint{0.848030in}{1.219967in}}%
\pgfpathlineto{\pgfqpoint{0.859762in}{1.212368in}}%
\pgfpathlineto{\pgfqpoint{0.871495in}{1.210373in}}%
\pgfpathlineto{\pgfqpoint{0.874944in}{1.208651in}}%
\pgfpathlineto{\pgfqpoint{0.883227in}{1.202922in}}%
\pgfpathlineto{\pgfqpoint{0.894959in}{1.199370in}}%
\pgfpathlineto{\pgfqpoint{0.899547in}{1.197241in}}%
\pgfpathlineto{\pgfqpoint{0.906691in}{1.192942in}}%
\pgfpathlineto{\pgfqpoint{0.918424in}{1.188460in}}%
\pgfpathlineto{\pgfqpoint{0.924089in}{1.185830in}}%
\pgfpathlineto{\pgfqpoint{0.930156in}{1.182580in}}%
\pgfpathlineto{\pgfqpoint{0.941888in}{1.177487in}}%
\pgfpathlineto{\pgfqpoint{0.948469in}{1.174419in}}%
\pgfpathlineto{\pgfqpoint{0.953620in}{1.171872in}}%
\pgfpathlineto{\pgfqpoint{0.965352in}{1.165968in}}%
\pgfpathlineto{\pgfqpoint{0.970796in}{1.163008in}}%
\pgfpathlineto{\pgfqpoint{0.977085in}{1.159424in}}%
\pgfpathlineto{\pgfqpoint{0.988817in}{1.152869in}}%
\pgfpathlineto{\pgfqpoint{0.991120in}{1.151598in}}%
\pgfpathlineto{\pgfqpoint{1.000549in}{1.145889in}}%
\pgfpathlineto{\pgfqpoint{1.009467in}{1.140187in}}%
\pgfpathlineto{\pgfqpoint{1.012281in}{1.137359in}}%
\pgfpathlineto{\pgfqpoint{1.021775in}{1.128776in}}%
\pgfpathlineto{\pgfqpoint{1.024013in}{1.127324in}}%
\pgfpathlineto{\pgfqpoint{1.035746in}{1.120424in}}%
\pgfpathlineto{\pgfqpoint{1.040112in}{1.117365in}}%
\pgfpathlineto{\pgfqpoint{1.047478in}{1.110799in}}%
\pgfpathlineto{\pgfqpoint{1.053937in}{1.105954in}}%
\pgfpathlineto{\pgfqpoint{1.047478in}{1.102005in}}%
\pgfpathlineto{\pgfqpoint{1.035746in}{1.100488in}}%
\pgfpathlineto{\pgfqpoint{1.024013in}{1.100215in}}%
\pgfpathlineto{\pgfqpoint{1.012281in}{1.100018in}}%
\pgfpathlineto{\pgfqpoint{1.000549in}{1.100155in}}%
\pgfpathlineto{\pgfqpoint{0.988817in}{1.101576in}}%
\pgfpathlineto{\pgfqpoint{0.977085in}{1.104177in}}%
\pgfpathlineto{\pgfqpoint{0.969132in}{1.105954in}}%
\pgfpathlineto{\pgfqpoint{0.965352in}{1.107746in}}%
\pgfpathlineto{\pgfqpoint{0.953620in}{1.112842in}}%
\pgfpathlineto{\pgfqpoint{0.941888in}{1.117325in}}%
\pgfpathlineto{\pgfqpoint{0.941784in}{1.117365in}}%
\pgfpathlineto{\pgfqpoint{0.930156in}{1.121764in}}%
\pgfpathlineto{\pgfqpoint{0.918424in}{1.124358in}}%
\pgfpathlineto{\pgfqpoint{0.906691in}{1.125632in}}%
\pgfpathlineto{\pgfqpoint{0.894959in}{1.126712in}}%
\pgfpathlineto{\pgfqpoint{0.883227in}{1.127777in}}%
\pgfpathlineto{\pgfqpoint{0.871659in}{1.128776in}}%
\pgfpathlineto{\pgfqpoint{0.871495in}{1.128805in}}%
\pgfpathlineto{\pgfqpoint{0.859762in}{1.130860in}}%
\pgfpathlineto{\pgfqpoint{0.848030in}{1.132693in}}%
\pgfpathlineto{\pgfqpoint{0.836298in}{1.134297in}}%
\pgfpathlineto{\pgfqpoint{0.824566in}{1.134401in}}%
\pgfpathlineto{\pgfqpoint{0.818094in}{1.128776in}}%
\pgfpathlineto{\pgfqpoint{0.812834in}{1.126740in}}%
\pgfpathlineto{\pgfqpoint{0.810147in}{1.128776in}}%
\pgfpathlineto{\pgfqpoint{0.802220in}{1.140187in}}%
\pgfpathlineto{\pgfqpoint{0.801101in}{1.141780in}}%
\pgfpathlineto{\pgfqpoint{0.796678in}{1.151598in}}%
\pgfpathlineto{\pgfqpoint{0.792871in}{1.163008in}}%
\pgfpathlineto{\pgfqpoint{0.789645in}{1.174419in}}%
\pgfpathlineto{\pgfqpoint{0.789369in}{1.174818in}}%
\pgfpathlineto{\pgfqpoint{0.784606in}{1.185830in}}%
\pgfpathlineto{\pgfqpoint{0.780736in}{1.197241in}}%
\pgfpathlineto{\pgfqpoint{0.777637in}{1.200844in}}%
\pgfpathlineto{\pgfqpoint{0.772373in}{1.208651in}}%
\pgfpathlineto{\pgfqpoint{0.765905in}{1.218238in}}%
\pgfpathlineto{\pgfqpoint{0.764654in}{1.220062in}}%
\pgfpathlineto{\pgfqpoint{0.758923in}{1.231473in}}%
\pgfpathlineto{\pgfqpoint{0.754173in}{1.235822in}}%
\pgfpathlineto{\pgfqpoint{0.749467in}{1.242884in}}%
\pgfpathlineto{\pgfqpoint{0.742440in}{1.253146in}}%
\pgfpathlineto{\pgfqpoint{0.741638in}{1.254295in}}%
\pgfpathlineto{\pgfqpoint{0.734734in}{1.265705in}}%
\pgfpathlineto{\pgfqpoint{0.730708in}{1.266697in}}%
\pgfpathlineto{\pgfqpoint{0.718976in}{1.276642in}}%
\pgfpathlineto{\pgfqpoint{0.718434in}{1.277116in}}%
\pgfpathlineto{\pgfqpoint{0.707244in}{1.287036in}}%
\pgfpathlineto{\pgfqpoint{0.705567in}{1.288527in}}%
\pgfpathlineto{\pgfqpoint{0.695512in}{1.297627in}}%
\pgfpathlineto{\pgfqpoint{0.692964in}{1.299938in}}%
\pgfpathlineto{\pgfqpoint{0.683779in}{1.308440in}}%
\pgfpathlineto{\pgfqpoint{0.680138in}{1.311348in}}%
\pgfpathlineto{\pgfqpoint{0.672047in}{1.318989in}}%
\pgfpathlineto{\pgfqpoint{0.666169in}{1.322759in}}%
\pgfpathlineto{\pgfqpoint{0.660315in}{1.328444in}}%
\pgfpathlineto{\pgfqpoint{0.649293in}{1.334170in}}%
\pgfpathlineto{\pgfqpoint{0.648583in}{1.334589in}}%
\pgfpathlineto{\pgfqpoint{0.636851in}{1.337441in}}%
\pgfpathlineto{\pgfqpoint{0.632980in}{1.334170in}}%
\pgfpathlineto{\pgfqpoint{0.632249in}{1.322759in}}%
\pgfpathlineto{\pgfqpoint{0.634099in}{1.311348in}}%
\pgfpathlineto{\pgfqpoint{0.636851in}{1.300258in}}%
\pgfpathlineto{\pgfqpoint{0.636924in}{1.299938in}}%
\pgfpathlineto{\pgfqpoint{0.640048in}{1.288527in}}%
\pgfpathlineto{\pgfqpoint{0.643308in}{1.277116in}}%
\pgfpathlineto{\pgfqpoint{0.645937in}{1.265705in}}%
\pgfpathlineto{\pgfqpoint{0.648583in}{1.255331in}}%
\pgfpathlineto{\pgfqpoint{0.648769in}{1.254295in}}%
\pgfpathlineto{\pgfqpoint{0.648583in}{1.252629in}}%
\pgfpathlineto{\pgfqpoint{0.645750in}{1.242884in}}%
\pgfpathlineto{\pgfqpoint{0.641597in}{1.231473in}}%
\pgfpathlineto{\pgfqpoint{0.639626in}{1.220062in}}%
\pgfpathlineto{\pgfqpoint{0.640038in}{1.208651in}}%
\pgfpathlineto{\pgfqpoint{0.641949in}{1.197241in}}%
\pgfpathlineto{\pgfqpoint{0.645106in}{1.185830in}}%
\pgfpathlineto{\pgfqpoint{0.648583in}{1.176697in}}%
\pgfpathlineto{\pgfqpoint{0.649318in}{1.174419in}}%
\pgfpathlineto{\pgfqpoint{0.654068in}{1.163008in}}%
\pgfpathlineto{\pgfqpoint{0.660315in}{1.152798in}}%
\pgfpathlineto{\pgfqpoint{0.661900in}{1.151598in}}%
\pgfpathlineto{\pgfqpoint{0.672047in}{1.144775in}}%
\pgfpathlineto{\pgfqpoint{0.678668in}{1.140187in}}%
\pgfpathlineto{\pgfqpoint{0.683779in}{1.136754in}}%
\pgfpathlineto{\pgfqpoint{0.695302in}{1.128776in}}%
\pgfpathlineto{\pgfqpoint{0.695512in}{1.128633in}}%
\pgfpathlineto{\pgfqpoint{0.707244in}{1.120222in}}%
\pgfpathlineto{\pgfqpoint{0.711323in}{1.117365in}}%
\pgfpathlineto{\pgfqpoint{0.718976in}{1.111489in}}%
\pgfpathlineto{\pgfqpoint{0.730551in}{1.105954in}}%
\pgfpathlineto{\pgfqpoint{0.730708in}{1.105882in}}%
\pgfpathlineto{\pgfqpoint{0.742440in}{1.098216in}}%
\pgfpathlineto{\pgfqpoint{0.744966in}{1.094544in}}%
\pgfpathlineto{\pgfqpoint{0.751178in}{1.083133in}}%
\pgfpathlineto{\pgfqpoint{0.754173in}{1.080518in}}%
\pgfpathlineto{\pgfqpoint{0.762634in}{1.071722in}}%
\pgfpathlineto{\pgfqpoint{0.765905in}{1.068500in}}%
\pgfpathlineto{\pgfqpoint{0.773304in}{1.060311in}}%
\pgfpathlineto{\pgfqpoint{0.777637in}{1.055875in}}%
\pgfpathlineto{\pgfqpoint{0.784951in}{1.048901in}}%
\pgfpathlineto{\pgfqpoint{0.789369in}{1.045022in}}%
\pgfpathlineto{\pgfqpoint{0.801101in}{1.038385in}}%
\pgfpathlineto{\pgfqpoint{0.803539in}{1.037490in}}%
\pgfpathlineto{\pgfqpoint{0.812834in}{1.035104in}}%
\pgfpathlineto{\pgfqpoint{0.822767in}{1.026079in}}%
\pgfpathlineto{\pgfqpoint{0.824566in}{1.019678in}}%
\pgfpathlineto{\pgfqpoint{0.828484in}{1.014668in}}%
\pgfpathlineto{\pgfqpoint{0.834300in}{1.003257in}}%
\pgfpathlineto{\pgfqpoint{0.836298in}{0.999778in}}%
\pgfpathlineto{\pgfqpoint{0.840698in}{0.991847in}}%
\pgfpathlineto{\pgfqpoint{0.847138in}{0.980436in}}%
\pgfpathlineto{\pgfqpoint{0.848030in}{0.978773in}}%
\pgfpathlineto{\pgfqpoint{0.852601in}{0.969025in}}%
\pgfpathlineto{\pgfqpoint{0.851050in}{0.957614in}}%
\pgfpathlineto{\pgfqpoint{0.848030in}{0.949169in}}%
\pgfpathlineto{\pgfqpoint{0.847192in}{0.946204in}}%
\pgfpathlineto{\pgfqpoint{0.844041in}{0.934793in}}%
\pgfpathlineto{\pgfqpoint{0.848030in}{0.925322in}}%
\pgfpathlineto{\pgfqpoint{0.851122in}{0.923382in}}%
\pgfpathlineto{\pgfqpoint{0.859762in}{0.922306in}}%
\pgfpathlineto{\pgfqpoint{0.871495in}{0.918735in}}%
\pgfpathlineto{\pgfqpoint{0.883227in}{0.913416in}}%
\pgfpathlineto{\pgfqpoint{0.885598in}{0.911971in}}%
\pgfpathlineto{\pgfqpoint{0.894519in}{0.900560in}}%
\pgfpathlineto{\pgfqpoint{0.894959in}{0.899661in}}%
\pgfpathlineto{\pgfqpoint{0.900260in}{0.889150in}}%
\pgfpathlineto{\pgfqpoint{0.905578in}{0.877739in}}%
\pgfpathlineto{\pgfqpoint{0.906691in}{0.874811in}}%
\pgfpathlineto{\pgfqpoint{0.909686in}{0.866328in}}%
\pgfpathlineto{\pgfqpoint{0.913053in}{0.854917in}}%
\pgfpathlineto{\pgfqpoint{0.913279in}{0.843507in}}%
\pgfpathlineto{\pgfqpoint{0.906691in}{0.839444in}}%
\pgfpathlineto{\pgfqpoint{0.895980in}{0.843507in}}%
\pgfpathlineto{\pgfqpoint{0.894959in}{0.843845in}}%
\pgfpathlineto{\pgfqpoint{0.883227in}{0.847624in}}%
\pgfpathlineto{\pgfqpoint{0.872656in}{0.843507in}}%
\pgfpathlineto{\pgfqpoint{0.871495in}{0.842242in}}%
\pgfpathlineto{\pgfqpoint{0.868971in}{0.832096in}}%
\pgfpathlineto{\pgfqpoint{0.867444in}{0.820685in}}%
\pgfpathlineto{\pgfqpoint{0.866322in}{0.809274in}}%
\pgfpathlineto{\pgfqpoint{0.865379in}{0.797863in}}%
\pgfpathlineto{\pgfqpoint{0.864779in}{0.786453in}}%
\pgfpathlineto{\pgfqpoint{0.864107in}{0.775042in}}%
\pgfpathlineto{\pgfqpoint{0.863497in}{0.763631in}}%
\pgfpathlineto{\pgfqpoint{0.863409in}{0.752220in}}%
\pgfpathlineto{\pgfqpoint{0.863949in}{0.740810in}}%
\pgfpathlineto{\pgfqpoint{0.864855in}{0.729399in}}%
\pgfpathlineto{\pgfqpoint{0.866622in}{0.717988in}}%
\pgfpathlineto{\pgfqpoint{0.868904in}{0.706577in}}%
\pgfpathlineto{\pgfqpoint{0.871495in}{0.695486in}}%
\pgfpathlineto{\pgfqpoint{0.871565in}{0.695166in}}%
\pgfpathlineto{\pgfqpoint{0.874012in}{0.683756in}}%
\pgfpathlineto{\pgfqpoint{0.876623in}{0.672345in}}%
\pgfpathlineto{\pgfqpoint{0.879544in}{0.660934in}}%
\pgfpathlineto{\pgfqpoint{0.882694in}{0.649523in}}%
\pgfpathlineto{\pgfqpoint{0.883227in}{0.647648in}}%
\pgfpathlineto{\pgfqpoint{0.885724in}{0.638113in}}%
\pgfpathlineto{\pgfqpoint{0.888701in}{0.626702in}}%
\pgfpathlineto{\pgfqpoint{0.891749in}{0.615291in}}%
\pgfpathlineto{\pgfqpoint{0.894959in}{0.603887in}}%
\pgfpathlineto{\pgfqpoint{0.894961in}{0.603880in}}%
\pgfpathlineto{\pgfqpoint{0.898075in}{0.592469in}}%
\pgfpathlineto{\pgfqpoint{0.901163in}{0.581059in}}%
\pgfpathlineto{\pgfqpoint{0.904301in}{0.569648in}}%
\pgfpathlineto{\pgfqpoint{0.906691in}{0.561251in}}%
\pgfpathlineto{\pgfqpoint{0.907568in}{0.558237in}}%
\pgfpathlineto{\pgfqpoint{0.910809in}{0.546826in}}%
\pgfpathlineto{\pgfqpoint{0.914628in}{0.535416in}}%
\pgfpathlineto{\pgfqpoint{0.918424in}{0.525767in}}%
\pgfpathlineto{\pgfqpoint{0.919277in}{0.524005in}}%
\pgfpathlineto{\pgfqpoint{0.924133in}{0.512594in}}%
\pgfpathlineto{\pgfqpoint{0.927754in}{0.501183in}}%
\pgfpathlineto{\pgfqpoint{0.930156in}{0.492492in}}%
\pgfpathlineto{\pgfqpoint{0.931061in}{0.489772in}}%
\pgfpathlineto{\pgfqpoint{0.934705in}{0.478362in}}%
\pgfpathlineto{\pgfqpoint{0.938178in}{0.466951in}}%
\pgfpathlineto{\pgfqpoint{0.941463in}{0.455540in}}%
\pgfpathlineto{\pgfqpoint{0.941888in}{0.453981in}}%
\pgfpathlineto{\pgfqpoint{0.945365in}{0.444129in}}%
\pgfpathlineto{\pgfqpoint{0.949260in}{0.432719in}}%
\pgfpathlineto{\pgfqpoint{0.952942in}{0.421308in}}%
\pgfpathlineto{\pgfqpoint{0.953620in}{0.419048in}}%
\pgfpathlineto{\pgfqpoint{0.957214in}{0.409897in}}%
\pgfpathlineto{\pgfqpoint{0.961735in}{0.398486in}}%
\pgfpathlineto{\pgfqpoint{0.965352in}{0.389176in}}%
\pgfpathlineto{\pgfqpoint{0.966269in}{0.387075in}}%
\pgfpathlineto{\pgfqpoint{0.971378in}{0.375665in}}%
\pgfpathlineto{\pgfqpoint{0.976942in}{0.364254in}}%
\pgfpathlineto{\pgfqpoint{0.977085in}{0.363954in}}%
\pgfpathlineto{\pgfqpoint{0.982141in}{0.352843in}}%
\pgfpathlineto{\pgfqpoint{0.988301in}{0.341432in}}%
\pgfpathlineto{\pgfqpoint{0.988817in}{0.340508in}}%
\pgfpathlineto{\pgfqpoint{0.993454in}{0.330022in}}%
\pgfpathlineto{\pgfqpoint{0.999588in}{0.318611in}}%
\pgfpathlineto{\pgfqpoint{1.000549in}{0.316942in}}%
\pgfpathlineto{\pgfqpoint{1.004641in}{0.307200in}}%
\pgfpathlineto{\pgfqpoint{1.010384in}{0.295789in}}%
\pgfpathclose%
\pgfpathmoveto{\pgfqpoint{0.814269in}{1.265705in}}%
\pgfpathlineto{\pgfqpoint{0.812834in}{1.268648in}}%
\pgfpathlineto{\pgfqpoint{0.811032in}{1.277116in}}%
\pgfpathlineto{\pgfqpoint{0.809929in}{1.288527in}}%
\pgfpathlineto{\pgfqpoint{0.812834in}{1.299050in}}%
\pgfpathlineto{\pgfqpoint{0.813289in}{1.299938in}}%
\pgfpathlineto{\pgfqpoint{0.812834in}{1.300836in}}%
\pgfpathlineto{\pgfqpoint{0.808285in}{1.311348in}}%
\pgfpathlineto{\pgfqpoint{0.801748in}{1.322759in}}%
\pgfpathlineto{\pgfqpoint{0.801101in}{1.324439in}}%
\pgfpathlineto{\pgfqpoint{0.797054in}{1.334170in}}%
\pgfpathlineto{\pgfqpoint{0.794689in}{1.345581in}}%
\pgfpathlineto{\pgfqpoint{0.797079in}{1.356992in}}%
\pgfpathlineto{\pgfqpoint{0.801101in}{1.361564in}}%
\pgfpathlineto{\pgfqpoint{0.812465in}{1.368402in}}%
\pgfpathlineto{\pgfqpoint{0.812834in}{1.368534in}}%
\pgfpathlineto{\pgfqpoint{0.824566in}{1.371271in}}%
\pgfpathlineto{\pgfqpoint{0.836298in}{1.372771in}}%
\pgfpathlineto{\pgfqpoint{0.848030in}{1.373495in}}%
\pgfpathlineto{\pgfqpoint{0.859762in}{1.373694in}}%
\pgfpathlineto{\pgfqpoint{0.871495in}{1.373519in}}%
\pgfpathlineto{\pgfqpoint{0.883227in}{1.371668in}}%
\pgfpathlineto{\pgfqpoint{0.887040in}{1.368402in}}%
\pgfpathlineto{\pgfqpoint{0.891565in}{1.356992in}}%
\pgfpathlineto{\pgfqpoint{0.894959in}{1.345957in}}%
\pgfpathlineto{\pgfqpoint{0.895079in}{1.345581in}}%
\pgfpathlineto{\pgfqpoint{0.896091in}{1.334170in}}%
\pgfpathlineto{\pgfqpoint{0.894959in}{1.328540in}}%
\pgfpathlineto{\pgfqpoint{0.893757in}{1.322759in}}%
\pgfpathlineto{\pgfqpoint{0.886645in}{1.311348in}}%
\pgfpathlineto{\pgfqpoint{0.883227in}{1.308464in}}%
\pgfpathlineto{\pgfqpoint{0.872620in}{1.299938in}}%
\pgfpathlineto{\pgfqpoint{0.871495in}{1.299371in}}%
\pgfpathlineto{\pgfqpoint{0.859762in}{1.292835in}}%
\pgfpathlineto{\pgfqpoint{0.854246in}{1.288527in}}%
\pgfpathlineto{\pgfqpoint{0.848030in}{1.278023in}}%
\pgfpathlineto{\pgfqpoint{0.847475in}{1.277116in}}%
\pgfpathlineto{\pgfqpoint{0.836298in}{1.266269in}}%
\pgfpathlineto{\pgfqpoint{0.835613in}{1.265705in}}%
\pgfpathlineto{\pgfqpoint{0.824566in}{1.261093in}}%
\pgfpathclose%
\pgfusepath{fill}%
\end{pgfscope}%
\begin{pgfscope}%
\pgfpathrectangle{\pgfqpoint{0.211875in}{0.211875in}}{\pgfqpoint{1.313625in}{1.279725in}}%
\pgfusepath{clip}%
\pgfsetbuttcap%
\pgfsetroundjoin%
\definecolor{currentfill}{rgb}{0.730358,0.086862,0.337485}%
\pgfsetfillcolor{currentfill}%
\pgfsetlinewidth{0.000000pt}%
\definecolor{currentstroke}{rgb}{0.000000,0.000000,0.000000}%
\pgfsetstrokecolor{currentstroke}%
\pgfsetdash{}{0pt}%
\pgfpathmoveto{\pgfqpoint{1.352515in}{0.291119in}}%
\pgfpathlineto{\pgfqpoint{1.355086in}{0.284378in}}%
\pgfpathlineto{\pgfqpoint{1.364247in}{0.284378in}}%
\pgfpathlineto{\pgfqpoint{1.375980in}{0.284378in}}%
\pgfpathlineto{\pgfqpoint{1.387712in}{0.284378in}}%
\pgfpathlineto{\pgfqpoint{1.399444in}{0.284378in}}%
\pgfpathlineto{\pgfqpoint{1.411176in}{0.284378in}}%
\pgfpathlineto{\pgfqpoint{1.422908in}{0.284378in}}%
\pgfpathlineto{\pgfqpoint{1.434641in}{0.284378in}}%
\pgfpathlineto{\pgfqpoint{1.446373in}{0.284378in}}%
\pgfpathlineto{\pgfqpoint{1.446373in}{0.295789in}}%
\pgfpathlineto{\pgfqpoint{1.446373in}{0.307200in}}%
\pgfpathlineto{\pgfqpoint{1.446373in}{0.318611in}}%
\pgfpathlineto{\pgfqpoint{1.446373in}{0.330022in}}%
\pgfpathlineto{\pgfqpoint{1.446373in}{0.341432in}}%
\pgfpathlineto{\pgfqpoint{1.446373in}{0.344195in}}%
\pgfpathlineto{\pgfqpoint{1.434641in}{0.349649in}}%
\pgfpathlineto{\pgfqpoint{1.422908in}{0.352064in}}%
\pgfpathlineto{\pgfqpoint{1.411176in}{0.351542in}}%
\pgfpathlineto{\pgfqpoint{1.399444in}{0.352688in}}%
\pgfpathlineto{\pgfqpoint{1.399126in}{0.352843in}}%
\pgfpathlineto{\pgfqpoint{1.387712in}{0.358306in}}%
\pgfpathlineto{\pgfqpoint{1.381874in}{0.364254in}}%
\pgfpathlineto{\pgfqpoint{1.375980in}{0.370194in}}%
\pgfpathlineto{\pgfqpoint{1.372808in}{0.375665in}}%
\pgfpathlineto{\pgfqpoint{1.366223in}{0.387075in}}%
\pgfpathlineto{\pgfqpoint{1.364247in}{0.390517in}}%
\pgfpathlineto{\pgfqpoint{1.361321in}{0.398486in}}%
\pgfpathlineto{\pgfqpoint{1.357209in}{0.409897in}}%
\pgfpathlineto{\pgfqpoint{1.353190in}{0.421308in}}%
\pgfpathlineto{\pgfqpoint{1.352515in}{0.423258in}}%
\pgfpathlineto{\pgfqpoint{1.350325in}{0.432719in}}%
\pgfpathlineto{\pgfqpoint{1.347762in}{0.444129in}}%
\pgfpathlineto{\pgfqpoint{1.345285in}{0.455540in}}%
\pgfpathlineto{\pgfqpoint{1.342896in}{0.466951in}}%
\pgfpathlineto{\pgfqpoint{1.340783in}{0.477436in}}%
\pgfpathlineto{\pgfqpoint{1.340295in}{0.478362in}}%
\pgfpathlineto{\pgfqpoint{1.334386in}{0.489772in}}%
\pgfpathlineto{\pgfqpoint{1.329051in}{0.500251in}}%
\pgfpathlineto{\pgfqpoint{1.317319in}{0.498063in}}%
\pgfpathlineto{\pgfqpoint{1.314591in}{0.501183in}}%
\pgfpathlineto{\pgfqpoint{1.305586in}{0.511955in}}%
\pgfpathlineto{\pgfqpoint{1.305079in}{0.512594in}}%
\pgfpathlineto{\pgfqpoint{1.296414in}{0.524005in}}%
\pgfpathlineto{\pgfqpoint{1.293854in}{0.527569in}}%
\pgfpathlineto{\pgfqpoint{1.288358in}{0.535416in}}%
\pgfpathlineto{\pgfqpoint{1.282122in}{0.544922in}}%
\pgfpathlineto{\pgfqpoint{1.280904in}{0.546826in}}%
\pgfpathlineto{\pgfqpoint{1.274073in}{0.558237in}}%
\pgfpathlineto{\pgfqpoint{1.270390in}{0.565188in}}%
\pgfpathlineto{\pgfqpoint{1.268082in}{0.569648in}}%
\pgfpathlineto{\pgfqpoint{1.262678in}{0.581059in}}%
\pgfpathlineto{\pgfqpoint{1.258658in}{0.590498in}}%
\pgfpathlineto{\pgfqpoint{1.257839in}{0.592469in}}%
\pgfpathlineto{\pgfqpoint{1.253580in}{0.603880in}}%
\pgfpathlineto{\pgfqpoint{1.249882in}{0.615291in}}%
\pgfpathlineto{\pgfqpoint{1.246925in}{0.626029in}}%
\pgfpathlineto{\pgfqpoint{1.246745in}{0.626702in}}%
\pgfpathlineto{\pgfqpoint{1.244178in}{0.638113in}}%
\pgfpathlineto{\pgfqpoint{1.242163in}{0.649523in}}%
\pgfpathlineto{\pgfqpoint{1.240691in}{0.660934in}}%
\pgfpathlineto{\pgfqpoint{1.239753in}{0.672345in}}%
\pgfpathlineto{\pgfqpoint{1.239334in}{0.683756in}}%
\pgfpathlineto{\pgfqpoint{1.239296in}{0.695166in}}%
\pgfpathlineto{\pgfqpoint{1.240412in}{0.706577in}}%
\pgfpathlineto{\pgfqpoint{1.243486in}{0.717988in}}%
\pgfpathlineto{\pgfqpoint{1.246925in}{0.727689in}}%
\pgfpathlineto{\pgfqpoint{1.247528in}{0.729399in}}%
\pgfpathlineto{\pgfqpoint{1.252258in}{0.740810in}}%
\pgfpathlineto{\pgfqpoint{1.258658in}{0.751827in}}%
\pgfpathlineto{\pgfqpoint{1.258903in}{0.752220in}}%
\pgfpathlineto{\pgfqpoint{1.270390in}{0.758723in}}%
\pgfpathlineto{\pgfqpoint{1.282122in}{0.759371in}}%
\pgfpathlineto{\pgfqpoint{1.293854in}{0.756236in}}%
\pgfpathlineto{\pgfqpoint{1.298852in}{0.752220in}}%
\pgfpathlineto{\pgfqpoint{1.305586in}{0.746639in}}%
\pgfpathlineto{\pgfqpoint{1.312711in}{0.740810in}}%
\pgfpathlineto{\pgfqpoint{1.317319in}{0.736900in}}%
\pgfpathlineto{\pgfqpoint{1.322077in}{0.729399in}}%
\pgfpathlineto{\pgfqpoint{1.328413in}{0.717988in}}%
\pgfpathlineto{\pgfqpoint{1.329051in}{0.716336in}}%
\pgfpathlineto{\pgfqpoint{1.335284in}{0.706577in}}%
\pgfpathlineto{\pgfqpoint{1.340783in}{0.698154in}}%
\pgfpathlineto{\pgfqpoint{1.345038in}{0.706577in}}%
\pgfpathlineto{\pgfqpoint{1.352029in}{0.717988in}}%
\pgfpathlineto{\pgfqpoint{1.352515in}{0.718777in}}%
\pgfpathlineto{\pgfqpoint{1.362788in}{0.729399in}}%
\pgfpathlineto{\pgfqpoint{1.364247in}{0.730801in}}%
\pgfpathlineto{\pgfqpoint{1.375325in}{0.740810in}}%
\pgfpathlineto{\pgfqpoint{1.375980in}{0.747262in}}%
\pgfpathlineto{\pgfqpoint{1.376493in}{0.752220in}}%
\pgfpathlineto{\pgfqpoint{1.377775in}{0.763631in}}%
\pgfpathlineto{\pgfqpoint{1.382980in}{0.775042in}}%
\pgfpathlineto{\pgfqpoint{1.387712in}{0.781736in}}%
\pgfpathlineto{\pgfqpoint{1.399444in}{0.784511in}}%
\pgfpathlineto{\pgfqpoint{1.411176in}{0.778088in}}%
\pgfpathlineto{\pgfqpoint{1.414008in}{0.775042in}}%
\pgfpathlineto{\pgfqpoint{1.422908in}{0.765982in}}%
\pgfpathlineto{\pgfqpoint{1.423889in}{0.763631in}}%
\pgfpathlineto{\pgfqpoint{1.431335in}{0.752220in}}%
\pgfpathlineto{\pgfqpoint{1.434641in}{0.748236in}}%
\pgfpathlineto{\pgfqpoint{1.437283in}{0.740810in}}%
\pgfpathlineto{\pgfqpoint{1.441384in}{0.729399in}}%
\pgfpathlineto{\pgfqpoint{1.446373in}{0.718654in}}%
\pgfpathlineto{\pgfqpoint{1.446373in}{0.729399in}}%
\pgfpathlineto{\pgfqpoint{1.446373in}{0.740810in}}%
\pgfpathlineto{\pgfqpoint{1.446373in}{0.752220in}}%
\pgfpathlineto{\pgfqpoint{1.446373in}{0.763631in}}%
\pgfpathlineto{\pgfqpoint{1.446373in}{0.773792in}}%
\pgfpathlineto{\pgfqpoint{1.445878in}{0.775042in}}%
\pgfpathlineto{\pgfqpoint{1.442234in}{0.786453in}}%
\pgfpathlineto{\pgfqpoint{1.438648in}{0.797863in}}%
\pgfpathlineto{\pgfqpoint{1.435642in}{0.809274in}}%
\pgfpathlineto{\pgfqpoint{1.434641in}{0.815576in}}%
\pgfpathlineto{\pgfqpoint{1.433737in}{0.820685in}}%
\pgfpathlineto{\pgfqpoint{1.434641in}{0.827740in}}%
\pgfpathlineto{\pgfqpoint{1.435276in}{0.832096in}}%
\pgfpathlineto{\pgfqpoint{1.440515in}{0.843507in}}%
\pgfpathlineto{\pgfqpoint{1.445960in}{0.854917in}}%
\pgfpathlineto{\pgfqpoint{1.446373in}{0.855736in}}%
\pgfpathlineto{\pgfqpoint{1.446373in}{0.866328in}}%
\pgfpathlineto{\pgfqpoint{1.446373in}{0.877739in}}%
\pgfpathlineto{\pgfqpoint{1.446373in}{0.889150in}}%
\pgfpathlineto{\pgfqpoint{1.446373in}{0.900560in}}%
\pgfpathlineto{\pgfqpoint{1.446373in}{0.911971in}}%
\pgfpathlineto{\pgfqpoint{1.446373in}{0.912627in}}%
\pgfpathlineto{\pgfqpoint{1.436636in}{0.923382in}}%
\pgfpathlineto{\pgfqpoint{1.434641in}{0.924559in}}%
\pgfpathlineto{\pgfqpoint{1.422908in}{0.930681in}}%
\pgfpathlineto{\pgfqpoint{1.413579in}{0.934793in}}%
\pgfpathlineto{\pgfqpoint{1.411176in}{0.935558in}}%
\pgfpathlineto{\pgfqpoint{1.399444in}{0.938353in}}%
\pgfpathlineto{\pgfqpoint{1.387712in}{0.939735in}}%
\pgfpathlineto{\pgfqpoint{1.375980in}{0.936437in}}%
\pgfpathlineto{\pgfqpoint{1.372795in}{0.934793in}}%
\pgfpathlineto{\pgfqpoint{1.364247in}{0.928982in}}%
\pgfpathlineto{\pgfqpoint{1.357984in}{0.923382in}}%
\pgfpathlineto{\pgfqpoint{1.352515in}{0.914641in}}%
\pgfpathlineto{\pgfqpoint{1.351098in}{0.911971in}}%
\pgfpathlineto{\pgfqpoint{1.348572in}{0.900560in}}%
\pgfpathlineto{\pgfqpoint{1.351496in}{0.889150in}}%
\pgfpathlineto{\pgfqpoint{1.352515in}{0.885462in}}%
\pgfpathlineto{\pgfqpoint{1.355486in}{0.877739in}}%
\pgfpathlineto{\pgfqpoint{1.359910in}{0.866328in}}%
\pgfpathlineto{\pgfqpoint{1.363688in}{0.854917in}}%
\pgfpathlineto{\pgfqpoint{1.363969in}{0.843507in}}%
\pgfpathlineto{\pgfqpoint{1.362496in}{0.832096in}}%
\pgfpathlineto{\pgfqpoint{1.360806in}{0.820685in}}%
\pgfpathlineto{\pgfqpoint{1.358924in}{0.809274in}}%
\pgfpathlineto{\pgfqpoint{1.356892in}{0.797863in}}%
\pgfpathlineto{\pgfqpoint{1.354761in}{0.786453in}}%
\pgfpathlineto{\pgfqpoint{1.354059in}{0.775042in}}%
\pgfpathlineto{\pgfqpoint{1.353116in}{0.763631in}}%
\pgfpathlineto{\pgfqpoint{1.352515in}{0.762429in}}%
\pgfpathlineto{\pgfqpoint{1.345425in}{0.752220in}}%
\pgfpathlineto{\pgfqpoint{1.340783in}{0.745285in}}%
\pgfpathlineto{\pgfqpoint{1.329841in}{0.752220in}}%
\pgfpathlineto{\pgfqpoint{1.329051in}{0.752655in}}%
\pgfpathlineto{\pgfqpoint{1.321669in}{0.763631in}}%
\pgfpathlineto{\pgfqpoint{1.317319in}{0.770022in}}%
\pgfpathlineto{\pgfqpoint{1.308487in}{0.775042in}}%
\pgfpathlineto{\pgfqpoint{1.305586in}{0.776645in}}%
\pgfpathlineto{\pgfqpoint{1.293854in}{0.784220in}}%
\pgfpathlineto{\pgfqpoint{1.291004in}{0.786453in}}%
\pgfpathlineto{\pgfqpoint{1.282122in}{0.793205in}}%
\pgfpathlineto{\pgfqpoint{1.273705in}{0.797863in}}%
\pgfpathlineto{\pgfqpoint{1.270390in}{0.800183in}}%
\pgfpathlineto{\pgfqpoint{1.258658in}{0.800799in}}%
\pgfpathlineto{\pgfqpoint{1.246925in}{0.801674in}}%
\pgfpathlineto{\pgfqpoint{1.236077in}{0.809274in}}%
\pgfpathlineto{\pgfqpoint{1.235193in}{0.809880in}}%
\pgfpathlineto{\pgfqpoint{1.223461in}{0.817875in}}%
\pgfpathlineto{\pgfqpoint{1.219145in}{0.820685in}}%
\pgfpathlineto{\pgfqpoint{1.211729in}{0.825447in}}%
\pgfpathlineto{\pgfqpoint{1.201218in}{0.832096in}}%
\pgfpathlineto{\pgfqpoint{1.199996in}{0.832863in}}%
\pgfpathlineto{\pgfqpoint{1.188264in}{0.839903in}}%
\pgfpathlineto{\pgfqpoint{1.176532in}{0.840554in}}%
\pgfpathlineto{\pgfqpoint{1.164800in}{0.839861in}}%
\pgfpathlineto{\pgfqpoint{1.153068in}{0.839150in}}%
\pgfpathlineto{\pgfqpoint{1.141335in}{0.837275in}}%
\pgfpathlineto{\pgfqpoint{1.130941in}{0.832096in}}%
\pgfpathlineto{\pgfqpoint{1.129603in}{0.830910in}}%
\pgfpathlineto{\pgfqpoint{1.118584in}{0.820685in}}%
\pgfpathlineto{\pgfqpoint{1.117871in}{0.820000in}}%
\pgfpathlineto{\pgfqpoint{1.106488in}{0.809274in}}%
\pgfpathlineto{\pgfqpoint{1.106139in}{0.808905in}}%
\pgfpathlineto{\pgfqpoint{1.096392in}{0.797863in}}%
\pgfpathlineto{\pgfqpoint{1.094407in}{0.795560in}}%
\pgfpathlineto{\pgfqpoint{1.082674in}{0.790083in}}%
\pgfpathlineto{\pgfqpoint{1.073410in}{0.786453in}}%
\pgfpathlineto{\pgfqpoint{1.070942in}{0.785446in}}%
\pgfpathlineto{\pgfqpoint{1.059210in}{0.780600in}}%
\pgfpathlineto{\pgfqpoint{1.047478in}{0.775673in}}%
\pgfpathlineto{\pgfqpoint{1.045996in}{0.775042in}}%
\pgfpathlineto{\pgfqpoint{1.035746in}{0.770478in}}%
\pgfpathlineto{\pgfqpoint{1.024013in}{0.765711in}}%
\pgfpathlineto{\pgfqpoint{1.012281in}{0.770140in}}%
\pgfpathlineto{\pgfqpoint{1.003039in}{0.775042in}}%
\pgfpathlineto{\pgfqpoint{1.000549in}{0.776346in}}%
\pgfpathlineto{\pgfqpoint{0.988817in}{0.777177in}}%
\pgfpathlineto{\pgfqpoint{0.977085in}{0.784904in}}%
\pgfpathlineto{\pgfqpoint{0.974744in}{0.786453in}}%
\pgfpathlineto{\pgfqpoint{0.965352in}{0.794415in}}%
\pgfpathlineto{\pgfqpoint{0.961867in}{0.797863in}}%
\pgfpathlineto{\pgfqpoint{0.953620in}{0.804819in}}%
\pgfpathlineto{\pgfqpoint{0.951127in}{0.797863in}}%
\pgfpathlineto{\pgfqpoint{0.953620in}{0.789133in}}%
\pgfpathlineto{\pgfqpoint{0.956427in}{0.786453in}}%
\pgfpathlineto{\pgfqpoint{0.959092in}{0.775042in}}%
\pgfpathlineto{\pgfqpoint{0.959832in}{0.763631in}}%
\pgfpathlineto{\pgfqpoint{0.961683in}{0.752220in}}%
\pgfpathlineto{\pgfqpoint{0.963680in}{0.740810in}}%
\pgfpathlineto{\pgfqpoint{0.965352in}{0.732297in}}%
\pgfpathlineto{\pgfqpoint{0.966022in}{0.729399in}}%
\pgfpathlineto{\pgfqpoint{0.968855in}{0.717988in}}%
\pgfpathlineto{\pgfqpoint{0.971593in}{0.706577in}}%
\pgfpathlineto{\pgfqpoint{0.974315in}{0.695166in}}%
\pgfpathlineto{\pgfqpoint{0.977085in}{0.684426in}}%
\pgfpathlineto{\pgfqpoint{0.977269in}{0.683756in}}%
\pgfpathlineto{\pgfqpoint{0.980395in}{0.672345in}}%
\pgfpathlineto{\pgfqpoint{0.983218in}{0.660934in}}%
\pgfpathlineto{\pgfqpoint{0.986119in}{0.649523in}}%
\pgfpathlineto{\pgfqpoint{0.988817in}{0.640206in}}%
\pgfpathlineto{\pgfqpoint{0.989402in}{0.638113in}}%
\pgfpathlineto{\pgfqpoint{0.992519in}{0.626702in}}%
\pgfpathlineto{\pgfqpoint{0.995620in}{0.615291in}}%
\pgfpathlineto{\pgfqpoint{0.999574in}{0.603880in}}%
\pgfpathlineto{\pgfqpoint{1.000549in}{0.601759in}}%
\pgfpathlineto{\pgfqpoint{1.004580in}{0.592469in}}%
\pgfpathlineto{\pgfqpoint{1.009646in}{0.581059in}}%
\pgfpathlineto{\pgfqpoint{1.012281in}{0.575718in}}%
\pgfpathlineto{\pgfqpoint{1.014990in}{0.569648in}}%
\pgfpathlineto{\pgfqpoint{1.020004in}{0.558237in}}%
\pgfpathlineto{\pgfqpoint{1.024013in}{0.550071in}}%
\pgfpathlineto{\pgfqpoint{1.025518in}{0.546826in}}%
\pgfpathlineto{\pgfqpoint{1.030596in}{0.535416in}}%
\pgfpathlineto{\pgfqpoint{1.035746in}{0.524842in}}%
\pgfpathlineto{\pgfqpoint{1.036143in}{0.524005in}}%
\pgfpathlineto{\pgfqpoint{1.041291in}{0.512594in}}%
\pgfpathlineto{\pgfqpoint{1.046704in}{0.501183in}}%
\pgfpathlineto{\pgfqpoint{1.047478in}{0.499665in}}%
\pgfpathlineto{\pgfqpoint{1.052040in}{0.489772in}}%
\pgfpathlineto{\pgfqpoint{1.057337in}{0.478362in}}%
\pgfpathlineto{\pgfqpoint{1.059210in}{0.474632in}}%
\pgfpathlineto{\pgfqpoint{1.062819in}{0.466951in}}%
\pgfpathlineto{\pgfqpoint{1.068046in}{0.455540in}}%
\pgfpathlineto{\pgfqpoint{1.070942in}{0.449670in}}%
\pgfpathlineto{\pgfqpoint{1.073537in}{0.444129in}}%
\pgfpathlineto{\pgfqpoint{1.078702in}{0.432719in}}%
\pgfpathlineto{\pgfqpoint{1.082674in}{0.424518in}}%
\pgfpathlineto{\pgfqpoint{1.084593in}{0.421308in}}%
\pgfpathlineto{\pgfqpoint{1.089389in}{0.409897in}}%
\pgfpathlineto{\pgfqpoint{1.094407in}{0.399442in}}%
\pgfpathlineto{\pgfqpoint{1.095454in}{0.398486in}}%
\pgfpathlineto{\pgfqpoint{1.101361in}{0.387075in}}%
\pgfpathlineto{\pgfqpoint{1.105954in}{0.375665in}}%
\pgfpathlineto{\pgfqpoint{1.106139in}{0.375281in}}%
\pgfpathlineto{\pgfqpoint{1.116255in}{0.364254in}}%
\pgfpathlineto{\pgfqpoint{1.117871in}{0.360439in}}%
\pgfpathlineto{\pgfqpoint{1.123520in}{0.364254in}}%
\pgfpathlineto{\pgfqpoint{1.128565in}{0.375665in}}%
\pgfpathlineto{\pgfqpoint{1.129603in}{0.378501in}}%
\pgfpathlineto{\pgfqpoint{1.132461in}{0.387075in}}%
\pgfpathlineto{\pgfqpoint{1.135319in}{0.398486in}}%
\pgfpathlineto{\pgfqpoint{1.137228in}{0.409897in}}%
\pgfpathlineto{\pgfqpoint{1.138261in}{0.421308in}}%
\pgfpathlineto{\pgfqpoint{1.138494in}{0.432719in}}%
\pgfpathlineto{\pgfqpoint{1.138010in}{0.444129in}}%
\pgfpathlineto{\pgfqpoint{1.139939in}{0.455540in}}%
\pgfpathlineto{\pgfqpoint{1.141335in}{0.460709in}}%
\pgfpathlineto{\pgfqpoint{1.145106in}{0.466951in}}%
\pgfpathlineto{\pgfqpoint{1.150903in}{0.478362in}}%
\pgfpathlineto{\pgfqpoint{1.153068in}{0.483626in}}%
\pgfpathlineto{\pgfqpoint{1.155441in}{0.489772in}}%
\pgfpathlineto{\pgfqpoint{1.158772in}{0.501183in}}%
\pgfpathlineto{\pgfqpoint{1.161016in}{0.512594in}}%
\pgfpathlineto{\pgfqpoint{1.164800in}{0.518806in}}%
\pgfpathlineto{\pgfqpoint{1.176532in}{0.523869in}}%
\pgfpathlineto{\pgfqpoint{1.182655in}{0.512594in}}%
\pgfpathlineto{\pgfqpoint{1.188264in}{0.502584in}}%
\pgfpathlineto{\pgfqpoint{1.189068in}{0.501183in}}%
\pgfpathlineto{\pgfqpoint{1.196016in}{0.489772in}}%
\pgfpathlineto{\pgfqpoint{1.199996in}{0.483632in}}%
\pgfpathlineto{\pgfqpoint{1.203487in}{0.478362in}}%
\pgfpathlineto{\pgfqpoint{1.211471in}{0.466951in}}%
\pgfpathlineto{\pgfqpoint{1.211729in}{0.466598in}}%
\pgfpathlineto{\pgfqpoint{1.219971in}{0.455540in}}%
\pgfpathlineto{\pgfqpoint{1.223461in}{0.451063in}}%
\pgfpathlineto{\pgfqpoint{1.228998in}{0.444129in}}%
\pgfpathlineto{\pgfqpoint{1.235193in}{0.436670in}}%
\pgfpathlineto{\pgfqpoint{1.238556in}{0.432719in}}%
\pgfpathlineto{\pgfqpoint{1.246925in}{0.423219in}}%
\pgfpathlineto{\pgfqpoint{1.248651in}{0.421308in}}%
\pgfpathlineto{\pgfqpoint{1.258658in}{0.410560in}}%
\pgfpathlineto{\pgfqpoint{1.259290in}{0.409897in}}%
\pgfpathlineto{\pgfqpoint{1.270390in}{0.398576in}}%
\pgfpathlineto{\pgfqpoint{1.270480in}{0.398486in}}%
\pgfpathlineto{\pgfqpoint{1.282122in}{0.387177in}}%
\pgfpathlineto{\pgfqpoint{1.282229in}{0.387075in}}%
\pgfpathlineto{\pgfqpoint{1.293854in}{0.376292in}}%
\pgfpathlineto{\pgfqpoint{1.294547in}{0.375665in}}%
\pgfpathlineto{\pgfqpoint{1.305586in}{0.365864in}}%
\pgfpathlineto{\pgfqpoint{1.307502in}{0.364254in}}%
\pgfpathlineto{\pgfqpoint{1.317319in}{0.356143in}}%
\pgfpathlineto{\pgfqpoint{1.329051in}{0.356738in}}%
\pgfpathlineto{\pgfqpoint{1.331321in}{0.352843in}}%
\pgfpathlineto{\pgfqpoint{1.338010in}{0.341432in}}%
\pgfpathlineto{\pgfqpoint{1.340783in}{0.336728in}}%
\pgfpathlineto{\pgfqpoint{1.342452in}{0.330022in}}%
\pgfpathlineto{\pgfqpoint{1.345337in}{0.318611in}}%
\pgfpathlineto{\pgfqpoint{1.348280in}{0.307200in}}%
\pgfpathlineto{\pgfqpoint{1.351277in}{0.295789in}}%
\pgfpathclose%
\pgfpathmoveto{\pgfqpoint{1.093446in}{0.421308in}}%
\pgfpathlineto{\pgfqpoint{1.090942in}{0.432719in}}%
\pgfpathlineto{\pgfqpoint{1.082674in}{0.439502in}}%
\pgfpathlineto{\pgfqpoint{1.080541in}{0.444129in}}%
\pgfpathlineto{\pgfqpoint{1.076810in}{0.455540in}}%
\pgfpathlineto{\pgfqpoint{1.070942in}{0.464720in}}%
\pgfpathlineto{\pgfqpoint{1.069897in}{0.466951in}}%
\pgfpathlineto{\pgfqpoint{1.065479in}{0.478362in}}%
\pgfpathlineto{\pgfqpoint{1.059210in}{0.489552in}}%
\pgfpathlineto{\pgfqpoint{1.059105in}{0.489772in}}%
\pgfpathlineto{\pgfqpoint{1.053930in}{0.501183in}}%
\pgfpathlineto{\pgfqpoint{1.048793in}{0.512594in}}%
\pgfpathlineto{\pgfqpoint{1.047478in}{0.514548in}}%
\pgfpathlineto{\pgfqpoint{1.043148in}{0.524005in}}%
\pgfpathlineto{\pgfqpoint{1.038168in}{0.535416in}}%
\pgfpathlineto{\pgfqpoint{1.035746in}{0.539676in}}%
\pgfpathlineto{\pgfqpoint{1.032448in}{0.546826in}}%
\pgfpathlineto{\pgfqpoint{1.027314in}{0.558237in}}%
\pgfpathlineto{\pgfqpoint{1.024013in}{0.565017in}}%
\pgfpathlineto{\pgfqpoint{1.021880in}{0.569648in}}%
\pgfpathlineto{\pgfqpoint{1.016865in}{0.581059in}}%
\pgfpathlineto{\pgfqpoint{1.012281in}{0.590895in}}%
\pgfpathlineto{\pgfqpoint{1.011551in}{0.592469in}}%
\pgfpathlineto{\pgfqpoint{1.006558in}{0.603880in}}%
\pgfpathlineto{\pgfqpoint{1.002153in}{0.615291in}}%
\pgfpathlineto{\pgfqpoint{1.000549in}{0.620204in}}%
\pgfpathlineto{\pgfqpoint{0.998687in}{0.626702in}}%
\pgfpathlineto{\pgfqpoint{0.995704in}{0.638113in}}%
\pgfpathlineto{\pgfqpoint{0.992762in}{0.649523in}}%
\pgfpathlineto{\pgfqpoint{0.989919in}{0.660934in}}%
\pgfpathlineto{\pgfqpoint{0.988817in}{0.664606in}}%
\pgfpathlineto{\pgfqpoint{0.987056in}{0.672345in}}%
\pgfpathlineto{\pgfqpoint{0.984619in}{0.683756in}}%
\pgfpathlineto{\pgfqpoint{0.982545in}{0.695166in}}%
\pgfpathlineto{\pgfqpoint{0.980465in}{0.706577in}}%
\pgfpathlineto{\pgfqpoint{0.977085in}{0.716141in}}%
\pgfpathlineto{\pgfqpoint{0.976722in}{0.717988in}}%
\pgfpathlineto{\pgfqpoint{0.974849in}{0.729399in}}%
\pgfpathlineto{\pgfqpoint{0.975914in}{0.740810in}}%
\pgfpathlineto{\pgfqpoint{0.977085in}{0.745363in}}%
\pgfpathlineto{\pgfqpoint{0.988817in}{0.741188in}}%
\pgfpathlineto{\pgfqpoint{0.989388in}{0.740810in}}%
\pgfpathlineto{\pgfqpoint{1.000549in}{0.733940in}}%
\pgfpathlineto{\pgfqpoint{1.003652in}{0.729399in}}%
\pgfpathlineto{\pgfqpoint{1.012281in}{0.721349in}}%
\pgfpathlineto{\pgfqpoint{1.016490in}{0.717988in}}%
\pgfpathlineto{\pgfqpoint{1.024013in}{0.706897in}}%
\pgfpathlineto{\pgfqpoint{1.024494in}{0.706577in}}%
\pgfpathlineto{\pgfqpoint{1.035746in}{0.698863in}}%
\pgfpathlineto{\pgfqpoint{1.041245in}{0.695166in}}%
\pgfpathlineto{\pgfqpoint{1.047478in}{0.690836in}}%
\pgfpathlineto{\pgfqpoint{1.059210in}{0.691190in}}%
\pgfpathlineto{\pgfqpoint{1.067329in}{0.695166in}}%
\pgfpathlineto{\pgfqpoint{1.070942in}{0.697157in}}%
\pgfpathlineto{\pgfqpoint{1.072853in}{0.695166in}}%
\pgfpathlineto{\pgfqpoint{1.070942in}{0.688016in}}%
\pgfpathlineto{\pgfqpoint{1.069485in}{0.683756in}}%
\pgfpathlineto{\pgfqpoint{1.068060in}{0.672345in}}%
\pgfpathlineto{\pgfqpoint{1.068060in}{0.660934in}}%
\pgfpathlineto{\pgfqpoint{1.070942in}{0.650404in}}%
\pgfpathlineto{\pgfqpoint{1.071286in}{0.649523in}}%
\pgfpathlineto{\pgfqpoint{1.074526in}{0.638113in}}%
\pgfpathlineto{\pgfqpoint{1.077645in}{0.626702in}}%
\pgfpathlineto{\pgfqpoint{1.080455in}{0.615291in}}%
\pgfpathlineto{\pgfqpoint{1.082674in}{0.604767in}}%
\pgfpathlineto{\pgfqpoint{1.082935in}{0.603880in}}%
\pgfpathlineto{\pgfqpoint{1.085360in}{0.592469in}}%
\pgfpathlineto{\pgfqpoint{1.086889in}{0.581059in}}%
\pgfpathlineto{\pgfqpoint{1.089276in}{0.569648in}}%
\pgfpathlineto{\pgfqpoint{1.091600in}{0.558237in}}%
\pgfpathlineto{\pgfqpoint{1.092148in}{0.546826in}}%
\pgfpathlineto{\pgfqpoint{1.094407in}{0.538715in}}%
\pgfpathlineto{\pgfqpoint{1.095970in}{0.535416in}}%
\pgfpathlineto{\pgfqpoint{1.098354in}{0.524005in}}%
\pgfpathlineto{\pgfqpoint{1.100460in}{0.512594in}}%
\pgfpathlineto{\pgfqpoint{1.102434in}{0.501183in}}%
\pgfpathlineto{\pgfqpoint{1.103929in}{0.489772in}}%
\pgfpathlineto{\pgfqpoint{1.104876in}{0.478362in}}%
\pgfpathlineto{\pgfqpoint{1.105206in}{0.466951in}}%
\pgfpathlineto{\pgfqpoint{1.104849in}{0.455540in}}%
\pgfpathlineto{\pgfqpoint{1.103733in}{0.444129in}}%
\pgfpathlineto{\pgfqpoint{1.101787in}{0.432719in}}%
\pgfpathlineto{\pgfqpoint{1.098941in}{0.421308in}}%
\pgfpathlineto{\pgfqpoint{1.094407in}{0.418210in}}%
\pgfpathclose%
\pgfpathmoveto{\pgfqpoint{1.152172in}{0.740810in}}%
\pgfpathlineto{\pgfqpoint{1.151047in}{0.752220in}}%
\pgfpathlineto{\pgfqpoint{1.153068in}{0.760002in}}%
\pgfpathlineto{\pgfqpoint{1.154328in}{0.763631in}}%
\pgfpathlineto{\pgfqpoint{1.164800in}{0.767377in}}%
\pgfpathlineto{\pgfqpoint{1.176532in}{0.763890in}}%
\pgfpathlineto{\pgfqpoint{1.176927in}{0.763631in}}%
\pgfpathlineto{\pgfqpoint{1.188264in}{0.752605in}}%
\pgfpathlineto{\pgfqpoint{1.188549in}{0.752220in}}%
\pgfpathlineto{\pgfqpoint{1.191925in}{0.740810in}}%
\pgfpathlineto{\pgfqpoint{1.188264in}{0.736558in}}%
\pgfpathlineto{\pgfqpoint{1.176532in}{0.732623in}}%
\pgfpathlineto{\pgfqpoint{1.164800in}{0.730662in}}%
\pgfpathlineto{\pgfqpoint{1.153068in}{0.739392in}}%
\pgfpathclose%
\pgfpathmoveto{\pgfqpoint{1.397293in}{0.854917in}}%
\pgfpathlineto{\pgfqpoint{1.387712in}{0.861938in}}%
\pgfpathlineto{\pgfqpoint{1.383605in}{0.866328in}}%
\pgfpathlineto{\pgfqpoint{1.375980in}{0.874805in}}%
\pgfpathlineto{\pgfqpoint{1.374160in}{0.877739in}}%
\pgfpathlineto{\pgfqpoint{1.367397in}{0.889150in}}%
\pgfpathlineto{\pgfqpoint{1.364247in}{0.897075in}}%
\pgfpathlineto{\pgfqpoint{1.363087in}{0.900560in}}%
\pgfpathlineto{\pgfqpoint{1.364247in}{0.905488in}}%
\pgfpathlineto{\pgfqpoint{1.366320in}{0.911971in}}%
\pgfpathlineto{\pgfqpoint{1.375442in}{0.923382in}}%
\pgfpathlineto{\pgfqpoint{1.375980in}{0.923735in}}%
\pgfpathlineto{\pgfqpoint{1.387712in}{0.928541in}}%
\pgfpathlineto{\pgfqpoint{1.399444in}{0.926208in}}%
\pgfpathlineto{\pgfqpoint{1.407046in}{0.923382in}}%
\pgfpathlineto{\pgfqpoint{1.411176in}{0.920287in}}%
\pgfpathlineto{\pgfqpoint{1.420119in}{0.911971in}}%
\pgfpathlineto{\pgfqpoint{1.422908in}{0.904870in}}%
\pgfpathlineto{\pgfqpoint{1.424455in}{0.900560in}}%
\pgfpathlineto{\pgfqpoint{1.425790in}{0.889150in}}%
\pgfpathlineto{\pgfqpoint{1.426457in}{0.877739in}}%
\pgfpathlineto{\pgfqpoint{1.422908in}{0.870588in}}%
\pgfpathlineto{\pgfqpoint{1.420059in}{0.866328in}}%
\pgfpathlineto{\pgfqpoint{1.412616in}{0.854917in}}%
\pgfpathlineto{\pgfqpoint{1.411176in}{0.852626in}}%
\pgfpathlineto{\pgfqpoint{1.399444in}{0.853186in}}%
\pgfpathclose%
\pgfusepath{fill}%
\end{pgfscope}%
\begin{pgfscope}%
\pgfpathrectangle{\pgfqpoint{0.211875in}{0.211875in}}{\pgfqpoint{1.313625in}{1.279725in}}%
\pgfusepath{clip}%
\pgfsetbuttcap%
\pgfsetroundjoin%
\definecolor{currentfill}{rgb}{0.730358,0.086862,0.337485}%
\pgfsetfillcolor{currentfill}%
\pgfsetlinewidth{0.000000pt}%
\definecolor{currentstroke}{rgb}{0.000000,0.000000,0.000000}%
\pgfsetstrokecolor{currentstroke}%
\pgfsetdash{}{0pt}%
\pgfpathmoveto{\pgfqpoint{0.294521in}{0.444129in}}%
\pgfpathlineto{\pgfqpoint{0.296617in}{0.445935in}}%
\pgfpathlineto{\pgfqpoint{0.307687in}{0.455540in}}%
\pgfpathlineto{\pgfqpoint{0.308349in}{0.456111in}}%
\pgfpathlineto{\pgfqpoint{0.320081in}{0.466041in}}%
\pgfpathlineto{\pgfqpoint{0.321123in}{0.466951in}}%
\pgfpathlineto{\pgfqpoint{0.331813in}{0.475866in}}%
\pgfpathlineto{\pgfqpoint{0.334675in}{0.478362in}}%
\pgfpathlineto{\pgfqpoint{0.343545in}{0.485774in}}%
\pgfpathlineto{\pgfqpoint{0.348053in}{0.489772in}}%
\pgfpathlineto{\pgfqpoint{0.355278in}{0.495841in}}%
\pgfpathlineto{\pgfqpoint{0.361305in}{0.501183in}}%
\pgfpathlineto{\pgfqpoint{0.367010in}{0.506097in}}%
\pgfpathlineto{\pgfqpoint{0.374365in}{0.512594in}}%
\pgfpathlineto{\pgfqpoint{0.378742in}{0.516402in}}%
\pgfpathlineto{\pgfqpoint{0.387383in}{0.524005in}}%
\pgfpathlineto{\pgfqpoint{0.390474in}{0.526695in}}%
\pgfpathlineto{\pgfqpoint{0.400419in}{0.535416in}}%
\pgfpathlineto{\pgfqpoint{0.402206in}{0.536971in}}%
\pgfpathlineto{\pgfqpoint{0.413481in}{0.546826in}}%
\pgfpathlineto{\pgfqpoint{0.413939in}{0.547225in}}%
\pgfpathlineto{\pgfqpoint{0.425671in}{0.557152in}}%
\pgfpathlineto{\pgfqpoint{0.426923in}{0.558237in}}%
\pgfpathlineto{\pgfqpoint{0.437403in}{0.566989in}}%
\pgfpathlineto{\pgfqpoint{0.440476in}{0.569648in}}%
\pgfpathlineto{\pgfqpoint{0.449135in}{0.576896in}}%
\pgfpathlineto{\pgfqpoint{0.453950in}{0.581059in}}%
\pgfpathlineto{\pgfqpoint{0.460867in}{0.586604in}}%
\pgfpathlineto{\pgfqpoint{0.467619in}{0.592469in}}%
\pgfpathlineto{\pgfqpoint{0.472600in}{0.596289in}}%
\pgfpathlineto{\pgfqpoint{0.481652in}{0.603880in}}%
\pgfpathlineto{\pgfqpoint{0.484332in}{0.605974in}}%
\pgfpathlineto{\pgfqpoint{0.495736in}{0.615291in}}%
\pgfpathlineto{\pgfqpoint{0.496064in}{0.615551in}}%
\pgfpathlineto{\pgfqpoint{0.507796in}{0.624289in}}%
\pgfpathlineto{\pgfqpoint{0.510726in}{0.626702in}}%
\pgfpathlineto{\pgfqpoint{0.519528in}{0.633279in}}%
\pgfpathlineto{\pgfqpoint{0.525370in}{0.638113in}}%
\pgfpathlineto{\pgfqpoint{0.531261in}{0.642638in}}%
\pgfpathlineto{\pgfqpoint{0.539567in}{0.649523in}}%
\pgfpathlineto{\pgfqpoint{0.542993in}{0.652198in}}%
\pgfpathlineto{\pgfqpoint{0.553522in}{0.660934in}}%
\pgfpathlineto{\pgfqpoint{0.554725in}{0.661888in}}%
\pgfpathlineto{\pgfqpoint{0.566457in}{0.671120in}}%
\pgfpathlineto{\pgfqpoint{0.567951in}{0.672345in}}%
\pgfpathlineto{\pgfqpoint{0.578190in}{0.679975in}}%
\pgfpathlineto{\pgfqpoint{0.582758in}{0.683756in}}%
\pgfpathlineto{\pgfqpoint{0.589922in}{0.689209in}}%
\pgfpathlineto{\pgfqpoint{0.597063in}{0.695166in}}%
\pgfpathlineto{\pgfqpoint{0.601654in}{0.698735in}}%
\pgfpathlineto{\pgfqpoint{0.610985in}{0.706577in}}%
\pgfpathlineto{\pgfqpoint{0.613386in}{0.708485in}}%
\pgfpathlineto{\pgfqpoint{0.624616in}{0.717988in}}%
\pgfpathlineto{\pgfqpoint{0.625118in}{0.718396in}}%
\pgfpathlineto{\pgfqpoint{0.636851in}{0.727513in}}%
\pgfpathlineto{\pgfqpoint{0.639083in}{0.729399in}}%
\pgfpathlineto{\pgfqpoint{0.648583in}{0.736728in}}%
\pgfpathlineto{\pgfqpoint{0.653315in}{0.740810in}}%
\pgfpathlineto{\pgfqpoint{0.660315in}{0.746381in}}%
\pgfpathlineto{\pgfqpoint{0.666003in}{0.752220in}}%
\pgfpathlineto{\pgfqpoint{0.672047in}{0.758316in}}%
\pgfpathlineto{\pgfqpoint{0.676926in}{0.763631in}}%
\pgfpathlineto{\pgfqpoint{0.683779in}{0.771114in}}%
\pgfpathlineto{\pgfqpoint{0.687228in}{0.775042in}}%
\pgfpathlineto{\pgfqpoint{0.695512in}{0.784604in}}%
\pgfpathlineto{\pgfqpoint{0.697177in}{0.786453in}}%
\pgfpathlineto{\pgfqpoint{0.707244in}{0.796205in}}%
\pgfpathlineto{\pgfqpoint{0.708913in}{0.797863in}}%
\pgfpathlineto{\pgfqpoint{0.718976in}{0.807243in}}%
\pgfpathlineto{\pgfqpoint{0.720997in}{0.809274in}}%
\pgfpathlineto{\pgfqpoint{0.730708in}{0.818410in}}%
\pgfpathlineto{\pgfqpoint{0.732968in}{0.820685in}}%
\pgfpathlineto{\pgfqpoint{0.742440in}{0.829543in}}%
\pgfpathlineto{\pgfqpoint{0.744998in}{0.832096in}}%
\pgfpathlineto{\pgfqpoint{0.754173in}{0.840500in}}%
\pgfpathlineto{\pgfqpoint{0.757257in}{0.843507in}}%
\pgfpathlineto{\pgfqpoint{0.765905in}{0.851197in}}%
\pgfpathlineto{\pgfqpoint{0.771031in}{0.854917in}}%
\pgfpathlineto{\pgfqpoint{0.777637in}{0.859578in}}%
\pgfpathlineto{\pgfqpoint{0.788319in}{0.866328in}}%
\pgfpathlineto{\pgfqpoint{0.789369in}{0.866965in}}%
\pgfpathlineto{\pgfqpoint{0.801101in}{0.874523in}}%
\pgfpathlineto{\pgfqpoint{0.805530in}{0.877739in}}%
\pgfpathlineto{\pgfqpoint{0.812834in}{0.882657in}}%
\pgfpathlineto{\pgfqpoint{0.821415in}{0.889150in}}%
\pgfpathlineto{\pgfqpoint{0.824566in}{0.891970in}}%
\pgfpathlineto{\pgfqpoint{0.832738in}{0.900560in}}%
\pgfpathlineto{\pgfqpoint{0.836298in}{0.907024in}}%
\pgfpathlineto{\pgfqpoint{0.839970in}{0.911971in}}%
\pgfpathlineto{\pgfqpoint{0.844696in}{0.923382in}}%
\pgfpathlineto{\pgfqpoint{0.836298in}{0.925778in}}%
\pgfpathlineto{\pgfqpoint{0.824566in}{0.926054in}}%
\pgfpathlineto{\pgfqpoint{0.812834in}{0.927108in}}%
\pgfpathlineto{\pgfqpoint{0.801101in}{0.928302in}}%
\pgfpathlineto{\pgfqpoint{0.789369in}{0.929430in}}%
\pgfpathlineto{\pgfqpoint{0.781973in}{0.934793in}}%
\pgfpathlineto{\pgfqpoint{0.777637in}{0.939851in}}%
\pgfpathlineto{\pgfqpoint{0.771499in}{0.946204in}}%
\pgfpathlineto{\pgfqpoint{0.765905in}{0.952626in}}%
\pgfpathlineto{\pgfqpoint{0.761645in}{0.957614in}}%
\pgfpathlineto{\pgfqpoint{0.754173in}{0.967158in}}%
\pgfpathlineto{\pgfqpoint{0.752816in}{0.969025in}}%
\pgfpathlineto{\pgfqpoint{0.744371in}{0.980436in}}%
\pgfpathlineto{\pgfqpoint{0.742440in}{0.983047in}}%
\pgfpathlineto{\pgfqpoint{0.736301in}{0.991847in}}%
\pgfpathlineto{\pgfqpoint{0.730708in}{1.000208in}}%
\pgfpathlineto{\pgfqpoint{0.728721in}{1.003257in}}%
\pgfpathlineto{\pgfqpoint{0.721110in}{1.014668in}}%
\pgfpathlineto{\pgfqpoint{0.718976in}{1.017084in}}%
\pgfpathlineto{\pgfqpoint{0.710687in}{1.026079in}}%
\pgfpathlineto{\pgfqpoint{0.707244in}{1.029807in}}%
\pgfpathlineto{\pgfqpoint{0.700044in}{1.037490in}}%
\pgfpathlineto{\pgfqpoint{0.695512in}{1.042437in}}%
\pgfpathlineto{\pgfqpoint{0.688917in}{1.048901in}}%
\pgfpathlineto{\pgfqpoint{0.683779in}{1.053505in}}%
\pgfpathlineto{\pgfqpoint{0.674515in}{1.060311in}}%
\pgfpathlineto{\pgfqpoint{0.672047in}{1.062318in}}%
\pgfpathlineto{\pgfqpoint{0.660315in}{1.071107in}}%
\pgfpathlineto{\pgfqpoint{0.659208in}{1.071722in}}%
\pgfpathlineto{\pgfqpoint{0.648583in}{1.078166in}}%
\pgfpathlineto{\pgfqpoint{0.636851in}{1.082634in}}%
\pgfpathlineto{\pgfqpoint{0.635649in}{1.083133in}}%
\pgfpathlineto{\pgfqpoint{0.625118in}{1.087450in}}%
\pgfpathlineto{\pgfqpoint{0.613386in}{1.092306in}}%
\pgfpathlineto{\pgfqpoint{0.608306in}{1.094544in}}%
\pgfpathlineto{\pgfqpoint{0.601654in}{1.097311in}}%
\pgfpathlineto{\pgfqpoint{0.589922in}{1.102053in}}%
\pgfpathlineto{\pgfqpoint{0.581299in}{1.105954in}}%
\pgfpathlineto{\pgfqpoint{0.578190in}{1.107256in}}%
\pgfpathlineto{\pgfqpoint{0.566457in}{1.111335in}}%
\pgfpathlineto{\pgfqpoint{0.554725in}{1.116011in}}%
\pgfpathlineto{\pgfqpoint{0.549906in}{1.117365in}}%
\pgfpathlineto{\pgfqpoint{0.542993in}{1.119096in}}%
\pgfpathlineto{\pgfqpoint{0.531261in}{1.119343in}}%
\pgfpathlineto{\pgfqpoint{0.527043in}{1.117365in}}%
\pgfpathlineto{\pgfqpoint{0.524347in}{1.105954in}}%
\pgfpathlineto{\pgfqpoint{0.525242in}{1.094544in}}%
\pgfpathlineto{\pgfqpoint{0.526423in}{1.083133in}}%
\pgfpathlineto{\pgfqpoint{0.527245in}{1.071722in}}%
\pgfpathlineto{\pgfqpoint{0.527506in}{1.060311in}}%
\pgfpathlineto{\pgfqpoint{0.527169in}{1.048901in}}%
\pgfpathlineto{\pgfqpoint{0.526164in}{1.037490in}}%
\pgfpathlineto{\pgfqpoint{0.520040in}{1.026079in}}%
\pgfpathlineto{\pgfqpoint{0.519528in}{1.025593in}}%
\pgfpathlineto{\pgfqpoint{0.507796in}{1.019300in}}%
\pgfpathlineto{\pgfqpoint{0.498130in}{1.014668in}}%
\pgfpathlineto{\pgfqpoint{0.496064in}{1.013683in}}%
\pgfpathlineto{\pgfqpoint{0.484332in}{1.008742in}}%
\pgfpathlineto{\pgfqpoint{0.472600in}{1.004332in}}%
\pgfpathlineto{\pgfqpoint{0.469260in}{1.003257in}}%
\pgfpathlineto{\pgfqpoint{0.460867in}{1.000423in}}%
\pgfpathlineto{\pgfqpoint{0.449135in}{0.997007in}}%
\pgfpathlineto{\pgfqpoint{0.437403in}{0.994119in}}%
\pgfpathlineto{\pgfqpoint{0.425889in}{0.991847in}}%
\pgfpathlineto{\pgfqpoint{0.425671in}{0.991800in}}%
\pgfpathlineto{\pgfqpoint{0.413939in}{0.990038in}}%
\pgfpathlineto{\pgfqpoint{0.402206in}{0.988952in}}%
\pgfpathlineto{\pgfqpoint{0.390474in}{0.988652in}}%
\pgfpathlineto{\pgfqpoint{0.378742in}{0.989298in}}%
\pgfpathlineto{\pgfqpoint{0.367010in}{0.991044in}}%
\pgfpathlineto{\pgfqpoint{0.363716in}{0.991847in}}%
\pgfpathlineto{\pgfqpoint{0.355278in}{0.993929in}}%
\pgfpathlineto{\pgfqpoint{0.343545in}{0.998124in}}%
\pgfpathlineto{\pgfqpoint{0.332995in}{1.003257in}}%
\pgfpathlineto{\pgfqpoint{0.331813in}{1.003824in}}%
\pgfpathlineto{\pgfqpoint{0.320081in}{1.010637in}}%
\pgfpathlineto{\pgfqpoint{0.314322in}{1.014668in}}%
\pgfpathlineto{\pgfqpoint{0.308349in}{1.018693in}}%
\pgfpathlineto{\pgfqpoint{0.298842in}{1.026079in}}%
\pgfpathlineto{\pgfqpoint{0.296617in}{1.027728in}}%
\pgfpathlineto{\pgfqpoint{0.284884in}{1.037414in}}%
\pgfpathlineto{\pgfqpoint{0.284884in}{1.026079in}}%
\pgfpathlineto{\pgfqpoint{0.284884in}{1.014668in}}%
\pgfpathlineto{\pgfqpoint{0.284884in}{1.003257in}}%
\pgfpathlineto{\pgfqpoint{0.284884in}{0.991847in}}%
\pgfpathlineto{\pgfqpoint{0.284884in}{0.980436in}}%
\pgfpathlineto{\pgfqpoint{0.284884in}{0.969025in}}%
\pgfpathlineto{\pgfqpoint{0.284884in}{0.957614in}}%
\pgfpathlineto{\pgfqpoint{0.284884in}{0.946204in}}%
\pgfpathlineto{\pgfqpoint{0.284884in}{0.934793in}}%
\pgfpathlineto{\pgfqpoint{0.284884in}{0.923382in}}%
\pgfpathlineto{\pgfqpoint{0.284884in}{0.911971in}}%
\pgfpathlineto{\pgfqpoint{0.284884in}{0.900560in}}%
\pgfpathlineto{\pgfqpoint{0.284884in}{0.889150in}}%
\pgfpathlineto{\pgfqpoint{0.284884in}{0.877739in}}%
\pgfpathlineto{\pgfqpoint{0.284884in}{0.866328in}}%
\pgfpathlineto{\pgfqpoint{0.284884in}{0.854917in}}%
\pgfpathlineto{\pgfqpoint{0.284884in}{0.843507in}}%
\pgfpathlineto{\pgfqpoint{0.284884in}{0.832096in}}%
\pgfpathlineto{\pgfqpoint{0.284884in}{0.820685in}}%
\pgfpathlineto{\pgfqpoint{0.284884in}{0.809274in}}%
\pgfpathlineto{\pgfqpoint{0.284884in}{0.797863in}}%
\pgfpathlineto{\pgfqpoint{0.284884in}{0.786453in}}%
\pgfpathlineto{\pgfqpoint{0.284884in}{0.775042in}}%
\pgfpathlineto{\pgfqpoint{0.284884in}{0.772915in}}%
\pgfpathlineto{\pgfqpoint{0.291796in}{0.775042in}}%
\pgfpathlineto{\pgfqpoint{0.296617in}{0.777479in}}%
\pgfpathlineto{\pgfqpoint{0.308349in}{0.782376in}}%
\pgfpathlineto{\pgfqpoint{0.320081in}{0.786031in}}%
\pgfpathlineto{\pgfqpoint{0.321471in}{0.786453in}}%
\pgfpathlineto{\pgfqpoint{0.331813in}{0.791555in}}%
\pgfpathlineto{\pgfqpoint{0.343545in}{0.795698in}}%
\pgfpathlineto{\pgfqpoint{0.350919in}{0.797863in}}%
\pgfpathlineto{\pgfqpoint{0.355278in}{0.800105in}}%
\pgfpathlineto{\pgfqpoint{0.367010in}{0.805012in}}%
\pgfpathlineto{\pgfqpoint{0.378742in}{0.808510in}}%
\pgfpathlineto{\pgfqpoint{0.381463in}{0.809274in}}%
\pgfpathlineto{\pgfqpoint{0.390474in}{0.813477in}}%
\pgfpathlineto{\pgfqpoint{0.402206in}{0.817520in}}%
\pgfpathlineto{\pgfqpoint{0.413939in}{0.820559in}}%
\pgfpathlineto{\pgfqpoint{0.414419in}{0.820685in}}%
\pgfpathlineto{\pgfqpoint{0.425671in}{0.825236in}}%
\pgfpathlineto{\pgfqpoint{0.437403in}{0.828777in}}%
\pgfpathlineto{\pgfqpoint{0.449135in}{0.831585in}}%
\pgfpathlineto{\pgfqpoint{0.451273in}{0.832096in}}%
\pgfpathlineto{\pgfqpoint{0.460867in}{0.836115in}}%
\pgfpathlineto{\pgfqpoint{0.472600in}{0.838643in}}%
\pgfpathlineto{\pgfqpoint{0.484332in}{0.841412in}}%
\pgfpathlineto{\pgfqpoint{0.494244in}{0.843507in}}%
\pgfpathlineto{\pgfqpoint{0.496064in}{0.844523in}}%
\pgfpathlineto{\pgfqpoint{0.507796in}{0.848706in}}%
\pgfpathlineto{\pgfqpoint{0.519528in}{0.850565in}}%
\pgfpathlineto{\pgfqpoint{0.531261in}{0.852527in}}%
\pgfpathlineto{\pgfqpoint{0.542993in}{0.854837in}}%
\pgfpathlineto{\pgfqpoint{0.543378in}{0.854917in}}%
\pgfpathlineto{\pgfqpoint{0.554725in}{0.859248in}}%
\pgfpathlineto{\pgfqpoint{0.566457in}{0.861146in}}%
\pgfpathlineto{\pgfqpoint{0.578190in}{0.862726in}}%
\pgfpathlineto{\pgfqpoint{0.589922in}{0.865703in}}%
\pgfpathlineto{\pgfqpoint{0.592275in}{0.866328in}}%
\pgfpathlineto{\pgfqpoint{0.601654in}{0.870790in}}%
\pgfpathlineto{\pgfqpoint{0.613386in}{0.873277in}}%
\pgfpathlineto{\pgfqpoint{0.625118in}{0.874520in}}%
\pgfpathlineto{\pgfqpoint{0.636851in}{0.877360in}}%
\pgfpathlineto{\pgfqpoint{0.641558in}{0.877739in}}%
\pgfpathlineto{\pgfqpoint{0.648583in}{0.878558in}}%
\pgfpathlineto{\pgfqpoint{0.660315in}{0.878508in}}%
\pgfpathlineto{\pgfqpoint{0.672047in}{0.881021in}}%
\pgfpathlineto{\pgfqpoint{0.683779in}{0.883966in}}%
\pgfpathlineto{\pgfqpoint{0.695512in}{0.887183in}}%
\pgfpathlineto{\pgfqpoint{0.703536in}{0.889150in}}%
\pgfpathlineto{\pgfqpoint{0.707244in}{0.890055in}}%
\pgfpathlineto{\pgfqpoint{0.718976in}{0.890212in}}%
\pgfpathlineto{\pgfqpoint{0.723146in}{0.889150in}}%
\pgfpathlineto{\pgfqpoint{0.724551in}{0.877739in}}%
\pgfpathlineto{\pgfqpoint{0.718976in}{0.869756in}}%
\pgfpathlineto{\pgfqpoint{0.716866in}{0.866328in}}%
\pgfpathlineto{\pgfqpoint{0.707244in}{0.856596in}}%
\pgfpathlineto{\pgfqpoint{0.705697in}{0.854917in}}%
\pgfpathlineto{\pgfqpoint{0.695512in}{0.846423in}}%
\pgfpathlineto{\pgfqpoint{0.692139in}{0.843507in}}%
\pgfpathlineto{\pgfqpoint{0.683779in}{0.837368in}}%
\pgfpathlineto{\pgfqpoint{0.676712in}{0.832096in}}%
\pgfpathlineto{\pgfqpoint{0.672047in}{0.829300in}}%
\pgfpathlineto{\pgfqpoint{0.660315in}{0.824871in}}%
\pgfpathlineto{\pgfqpoint{0.653935in}{0.820685in}}%
\pgfpathlineto{\pgfqpoint{0.648583in}{0.817388in}}%
\pgfpathlineto{\pgfqpoint{0.636851in}{0.810194in}}%
\pgfpathlineto{\pgfqpoint{0.635376in}{0.809274in}}%
\pgfpathlineto{\pgfqpoint{0.625118in}{0.803170in}}%
\pgfpathlineto{\pgfqpoint{0.616178in}{0.797863in}}%
\pgfpathlineto{\pgfqpoint{0.613386in}{0.796233in}}%
\pgfpathlineto{\pgfqpoint{0.601654in}{0.789441in}}%
\pgfpathlineto{\pgfqpoint{0.596503in}{0.786453in}}%
\pgfpathlineto{\pgfqpoint{0.589922in}{0.782665in}}%
\pgfpathlineto{\pgfqpoint{0.578190in}{0.776007in}}%
\pgfpathlineto{\pgfqpoint{0.576501in}{0.775042in}}%
\pgfpathlineto{\pgfqpoint{0.566457in}{0.769309in}}%
\pgfpathlineto{\pgfqpoint{0.556345in}{0.763631in}}%
\pgfpathlineto{\pgfqpoint{0.554725in}{0.762704in}}%
\pgfpathlineto{\pgfqpoint{0.542993in}{0.756079in}}%
\pgfpathlineto{\pgfqpoint{0.536068in}{0.752220in}}%
\pgfpathlineto{\pgfqpoint{0.531261in}{0.749479in}}%
\pgfpathlineto{\pgfqpoint{0.519528in}{0.742947in}}%
\pgfpathlineto{\pgfqpoint{0.515657in}{0.740810in}}%
\pgfpathlineto{\pgfqpoint{0.507796in}{0.736355in}}%
\pgfpathlineto{\pgfqpoint{0.496064in}{0.729911in}}%
\pgfpathlineto{\pgfqpoint{0.495144in}{0.729399in}}%
\pgfpathlineto{\pgfqpoint{0.484332in}{0.723229in}}%
\pgfpathlineto{\pgfqpoint{0.474852in}{0.717988in}}%
\pgfpathlineto{\pgfqpoint{0.472600in}{0.716686in}}%
\pgfpathlineto{\pgfqpoint{0.460867in}{0.710100in}}%
\pgfpathlineto{\pgfqpoint{0.454411in}{0.706577in}}%
\pgfpathlineto{\pgfqpoint{0.449135in}{0.703567in}}%
\pgfpathlineto{\pgfqpoint{0.437403in}{0.697132in}}%
\pgfpathlineto{\pgfqpoint{0.433747in}{0.695166in}}%
\pgfpathlineto{\pgfqpoint{0.425671in}{0.690625in}}%
\pgfpathlineto{\pgfqpoint{0.413939in}{0.684333in}}%
\pgfpathlineto{\pgfqpoint{0.412847in}{0.683756in}}%
\pgfpathlineto{\pgfqpoint{0.402206in}{0.677864in}}%
\pgfpathlineto{\pgfqpoint{0.391723in}{0.672345in}}%
\pgfpathlineto{\pgfqpoint{0.390474in}{0.671647in}}%
\pgfpathlineto{\pgfqpoint{0.378742in}{0.665281in}}%
\pgfpathlineto{\pgfqpoint{0.370359in}{0.660934in}}%
\pgfpathlineto{\pgfqpoint{0.367010in}{0.659089in}}%
\pgfpathlineto{\pgfqpoint{0.355278in}{0.652867in}}%
\pgfpathlineto{\pgfqpoint{0.348740in}{0.649523in}}%
\pgfpathlineto{\pgfqpoint{0.343545in}{0.646703in}}%
\pgfpathlineto{\pgfqpoint{0.331813in}{0.640605in}}%
\pgfpathlineto{\pgfqpoint{0.326880in}{0.638113in}}%
\pgfpathlineto{\pgfqpoint{0.320081in}{0.634467in}}%
\pgfpathlineto{\pgfqpoint{0.308349in}{0.628477in}}%
\pgfpathlineto{\pgfqpoint{0.304801in}{0.626702in}}%
\pgfpathlineto{\pgfqpoint{0.296617in}{0.622359in}}%
\pgfpathlineto{\pgfqpoint{0.284884in}{0.616460in}}%
\pgfpathlineto{\pgfqpoint{0.284884in}{0.615291in}}%
\pgfpathlineto{\pgfqpoint{0.284884in}{0.603880in}}%
\pgfpathlineto{\pgfqpoint{0.284884in}{0.596047in}}%
\pgfpathlineto{\pgfqpoint{0.296617in}{0.602358in}}%
\pgfpathlineto{\pgfqpoint{0.299311in}{0.603880in}}%
\pgfpathlineto{\pgfqpoint{0.308349in}{0.608433in}}%
\pgfpathlineto{\pgfqpoint{0.320081in}{0.614965in}}%
\pgfpathlineto{\pgfqpoint{0.320648in}{0.615291in}}%
\pgfpathlineto{\pgfqpoint{0.331813in}{0.621009in}}%
\pgfpathlineto{\pgfqpoint{0.341813in}{0.626702in}}%
\pgfpathlineto{\pgfqpoint{0.343545in}{0.627590in}}%
\pgfpathlineto{\pgfqpoint{0.355278in}{0.633797in}}%
\pgfpathlineto{\pgfqpoint{0.362718in}{0.638113in}}%
\pgfpathlineto{\pgfqpoint{0.367010in}{0.640351in}}%
\pgfpathlineto{\pgfqpoint{0.378742in}{0.646816in}}%
\pgfpathlineto{\pgfqpoint{0.383322in}{0.649523in}}%
\pgfpathlineto{\pgfqpoint{0.390474in}{0.653319in}}%
\pgfpathlineto{\pgfqpoint{0.402206in}{0.660075in}}%
\pgfpathlineto{\pgfqpoint{0.403634in}{0.660934in}}%
\pgfpathlineto{\pgfqpoint{0.413939in}{0.666496in}}%
\pgfpathlineto{\pgfqpoint{0.423806in}{0.672345in}}%
\pgfpathlineto{\pgfqpoint{0.425671in}{0.673351in}}%
\pgfpathlineto{\pgfqpoint{0.437403in}{0.679874in}}%
\pgfpathlineto{\pgfqpoint{0.443855in}{0.683756in}}%
\pgfpathlineto{\pgfqpoint{0.449135in}{0.686644in}}%
\pgfpathlineto{\pgfqpoint{0.460867in}{0.693440in}}%
\pgfpathlineto{\pgfqpoint{0.463703in}{0.695166in}}%
\pgfpathlineto{\pgfqpoint{0.472600in}{0.700089in}}%
\pgfpathlineto{\pgfqpoint{0.483494in}{0.706577in}}%
\pgfpathlineto{\pgfqpoint{0.484332in}{0.707040in}}%
\pgfpathlineto{\pgfqpoint{0.496064in}{0.713540in}}%
\pgfpathlineto{\pgfqpoint{0.503549in}{0.717988in}}%
\pgfpathlineto{\pgfqpoint{0.507796in}{0.720320in}}%
\pgfpathlineto{\pgfqpoint{0.519528in}{0.726992in}}%
\pgfpathlineto{\pgfqpoint{0.523575in}{0.729399in}}%
\pgfpathlineto{\pgfqpoint{0.531261in}{0.733632in}}%
\pgfpathlineto{\pgfqpoint{0.542993in}{0.740465in}}%
\pgfpathlineto{\pgfqpoint{0.543575in}{0.740810in}}%
\pgfpathlineto{\pgfqpoint{0.554725in}{0.746950in}}%
\pgfpathlineto{\pgfqpoint{0.563759in}{0.752220in}}%
\pgfpathlineto{\pgfqpoint{0.566457in}{0.753711in}}%
\pgfpathlineto{\pgfqpoint{0.578190in}{0.760262in}}%
\pgfpathlineto{\pgfqpoint{0.583989in}{0.763631in}}%
\pgfpathlineto{\pgfqpoint{0.589922in}{0.766910in}}%
\pgfpathlineto{\pgfqpoint{0.601654in}{0.773592in}}%
\pgfpathlineto{\pgfqpoint{0.604150in}{0.775042in}}%
\pgfpathlineto{\pgfqpoint{0.613386in}{0.780173in}}%
\pgfpathlineto{\pgfqpoint{0.624215in}{0.786453in}}%
\pgfpathlineto{\pgfqpoint{0.625118in}{0.786965in}}%
\pgfpathlineto{\pgfqpoint{0.636851in}{0.793536in}}%
\pgfpathlineto{\pgfqpoint{0.644257in}{0.797863in}}%
\pgfpathlineto{\pgfqpoint{0.648583in}{0.800353in}}%
\pgfpathlineto{\pgfqpoint{0.660315in}{0.807134in}}%
\pgfpathlineto{\pgfqpoint{0.667878in}{0.809274in}}%
\pgfpathlineto{\pgfqpoint{0.672047in}{0.810411in}}%
\pgfpathlineto{\pgfqpoint{0.683779in}{0.813071in}}%
\pgfpathlineto{\pgfqpoint{0.695512in}{0.819029in}}%
\pgfpathlineto{\pgfqpoint{0.699628in}{0.820685in}}%
\pgfpathlineto{\pgfqpoint{0.707244in}{0.823477in}}%
\pgfpathlineto{\pgfqpoint{0.710024in}{0.820685in}}%
\pgfpathlineto{\pgfqpoint{0.707244in}{0.817770in}}%
\pgfpathlineto{\pgfqpoint{0.701637in}{0.809274in}}%
\pgfpathlineto{\pgfqpoint{0.695512in}{0.803154in}}%
\pgfpathlineto{\pgfqpoint{0.692200in}{0.797863in}}%
\pgfpathlineto{\pgfqpoint{0.683779in}{0.786518in}}%
\pgfpathlineto{\pgfqpoint{0.683741in}{0.786453in}}%
\pgfpathlineto{\pgfqpoint{0.673797in}{0.775042in}}%
\pgfpathlineto{\pgfqpoint{0.672047in}{0.773200in}}%
\pgfpathlineto{\pgfqpoint{0.662990in}{0.763631in}}%
\pgfpathlineto{\pgfqpoint{0.660315in}{0.760968in}}%
\pgfpathlineto{\pgfqpoint{0.650312in}{0.752220in}}%
\pgfpathlineto{\pgfqpoint{0.648583in}{0.750727in}}%
\pgfpathlineto{\pgfqpoint{0.636944in}{0.740810in}}%
\pgfpathlineto{\pgfqpoint{0.636851in}{0.740729in}}%
\pgfpathlineto{\pgfqpoint{0.625118in}{0.730880in}}%
\pgfpathlineto{\pgfqpoint{0.623462in}{0.729399in}}%
\pgfpathlineto{\pgfqpoint{0.613386in}{0.720983in}}%
\pgfpathlineto{\pgfqpoint{0.609947in}{0.717988in}}%
\pgfpathlineto{\pgfqpoint{0.601654in}{0.711083in}}%
\pgfpathlineto{\pgfqpoint{0.596363in}{0.706577in}}%
\pgfpathlineto{\pgfqpoint{0.589922in}{0.701223in}}%
\pgfpathlineto{\pgfqpoint{0.582666in}{0.695166in}}%
\pgfpathlineto{\pgfqpoint{0.578190in}{0.691448in}}%
\pgfpathlineto{\pgfqpoint{0.568805in}{0.683756in}}%
\pgfpathlineto{\pgfqpoint{0.566457in}{0.681806in}}%
\pgfpathlineto{\pgfqpoint{0.554725in}{0.672351in}}%
\pgfpathlineto{\pgfqpoint{0.554718in}{0.672345in}}%
\pgfpathlineto{\pgfqpoint{0.542993in}{0.662870in}}%
\pgfpathlineto{\pgfqpoint{0.540694in}{0.660934in}}%
\pgfpathlineto{\pgfqpoint{0.531261in}{0.653285in}}%
\pgfpathlineto{\pgfqpoint{0.526700in}{0.649523in}}%
\pgfpathlineto{\pgfqpoint{0.519528in}{0.643695in}}%
\pgfpathlineto{\pgfqpoint{0.512659in}{0.638113in}}%
\pgfpathlineto{\pgfqpoint{0.507796in}{0.634176in}}%
\pgfpathlineto{\pgfqpoint{0.498476in}{0.626702in}}%
\pgfpathlineto{\pgfqpoint{0.496064in}{0.624740in}}%
\pgfpathlineto{\pgfqpoint{0.484332in}{0.615461in}}%
\pgfpathlineto{\pgfqpoint{0.484129in}{0.615291in}}%
\pgfpathlineto{\pgfqpoint{0.472600in}{0.606189in}}%
\pgfpathlineto{\pgfqpoint{0.469822in}{0.603880in}}%
\pgfpathlineto{\pgfqpoint{0.460867in}{0.596748in}}%
\pgfpathlineto{\pgfqpoint{0.455670in}{0.592469in}}%
\pgfpathlineto{\pgfqpoint{0.449135in}{0.586974in}}%
\pgfpathlineto{\pgfqpoint{0.441814in}{0.581059in}}%
\pgfpathlineto{\pgfqpoint{0.437403in}{0.577228in}}%
\pgfpathlineto{\pgfqpoint{0.428546in}{0.569648in}}%
\pgfpathlineto{\pgfqpoint{0.425671in}{0.567150in}}%
\pgfpathlineto{\pgfqpoint{0.415446in}{0.558237in}}%
\pgfpathlineto{\pgfqpoint{0.413939in}{0.556924in}}%
\pgfpathlineto{\pgfqpoint{0.402347in}{0.546826in}}%
\pgfpathlineto{\pgfqpoint{0.402206in}{0.546704in}}%
\pgfpathlineto{\pgfqpoint{0.390474in}{0.536561in}}%
\pgfpathlineto{\pgfqpoint{0.389196in}{0.535416in}}%
\pgfpathlineto{\pgfqpoint{0.378742in}{0.526345in}}%
\pgfpathlineto{\pgfqpoint{0.376085in}{0.524005in}}%
\pgfpathlineto{\pgfqpoint{0.367010in}{0.516110in}}%
\pgfpathlineto{\pgfqpoint{0.362975in}{0.512594in}}%
\pgfpathlineto{\pgfqpoint{0.355278in}{0.505946in}}%
\pgfpathlineto{\pgfqpoint{0.349790in}{0.501183in}}%
\pgfpathlineto{\pgfqpoint{0.343545in}{0.495880in}}%
\pgfpathlineto{\pgfqpoint{0.336504in}{0.489772in}}%
\pgfpathlineto{\pgfqpoint{0.331813in}{0.485767in}}%
\pgfpathlineto{\pgfqpoint{0.323266in}{0.478362in}}%
\pgfpathlineto{\pgfqpoint{0.320081in}{0.475627in}}%
\pgfpathlineto{\pgfqpoint{0.310050in}{0.466951in}}%
\pgfpathlineto{\pgfqpoint{0.308349in}{0.465483in}}%
\pgfpathlineto{\pgfqpoint{0.296831in}{0.455540in}}%
\pgfpathlineto{\pgfqpoint{0.296617in}{0.455354in}}%
\pgfpathlineto{\pgfqpoint{0.284884in}{0.445350in}}%
\pgfpathlineto{\pgfqpoint{0.284884in}{0.444129in}}%
\pgfpathlineto{\pgfqpoint{0.284884in}{0.435724in}}%
\pgfpathclose%
\pgfpathmoveto{\pgfqpoint{0.773599in}{0.877739in}}%
\pgfpathlineto{\pgfqpoint{0.775559in}{0.889150in}}%
\pgfpathlineto{\pgfqpoint{0.777637in}{0.892605in}}%
\pgfpathlineto{\pgfqpoint{0.779916in}{0.900560in}}%
\pgfpathlineto{\pgfqpoint{0.787360in}{0.911971in}}%
\pgfpathlineto{\pgfqpoint{0.789369in}{0.912663in}}%
\pgfpathlineto{\pgfqpoint{0.801101in}{0.916664in}}%
\pgfpathlineto{\pgfqpoint{0.812834in}{0.916575in}}%
\pgfpathlineto{\pgfqpoint{0.824566in}{0.912583in}}%
\pgfpathlineto{\pgfqpoint{0.825192in}{0.911971in}}%
\pgfpathlineto{\pgfqpoint{0.824566in}{0.907051in}}%
\pgfpathlineto{\pgfqpoint{0.823601in}{0.900560in}}%
\pgfpathlineto{\pgfqpoint{0.812834in}{0.891554in}}%
\pgfpathlineto{\pgfqpoint{0.809482in}{0.889150in}}%
\pgfpathlineto{\pgfqpoint{0.801101in}{0.884780in}}%
\pgfpathlineto{\pgfqpoint{0.789369in}{0.878434in}}%
\pgfpathlineto{\pgfqpoint{0.784073in}{0.877739in}}%
\pgfpathlineto{\pgfqpoint{0.777637in}{0.876971in}}%
\pgfpathclose%
\pgfpathmoveto{\pgfqpoint{0.657646in}{0.900560in}}%
\pgfpathlineto{\pgfqpoint{0.648583in}{0.910602in}}%
\pgfpathlineto{\pgfqpoint{0.647191in}{0.911971in}}%
\pgfpathlineto{\pgfqpoint{0.636851in}{0.922542in}}%
\pgfpathlineto{\pgfqpoint{0.635957in}{0.923382in}}%
\pgfpathlineto{\pgfqpoint{0.625118in}{0.934452in}}%
\pgfpathlineto{\pgfqpoint{0.624807in}{0.934793in}}%
\pgfpathlineto{\pgfqpoint{0.616510in}{0.946204in}}%
\pgfpathlineto{\pgfqpoint{0.613386in}{0.951947in}}%
\pgfpathlineto{\pgfqpoint{0.610058in}{0.957614in}}%
\pgfpathlineto{\pgfqpoint{0.604051in}{0.969025in}}%
\pgfpathlineto{\pgfqpoint{0.601654in}{0.974226in}}%
\pgfpathlineto{\pgfqpoint{0.598588in}{0.980436in}}%
\pgfpathlineto{\pgfqpoint{0.593729in}{0.991847in}}%
\pgfpathlineto{\pgfqpoint{0.593184in}{1.003257in}}%
\pgfpathlineto{\pgfqpoint{0.593361in}{1.014668in}}%
\pgfpathlineto{\pgfqpoint{0.593650in}{1.026079in}}%
\pgfpathlineto{\pgfqpoint{0.593958in}{1.037490in}}%
\pgfpathlineto{\pgfqpoint{0.594520in}{1.048901in}}%
\pgfpathlineto{\pgfqpoint{0.596317in}{1.060311in}}%
\pgfpathlineto{\pgfqpoint{0.601654in}{1.068840in}}%
\pgfpathlineto{\pgfqpoint{0.607236in}{1.071722in}}%
\pgfpathlineto{\pgfqpoint{0.613386in}{1.072550in}}%
\pgfpathlineto{\pgfqpoint{0.621272in}{1.071722in}}%
\pgfpathlineto{\pgfqpoint{0.625118in}{1.071257in}}%
\pgfpathlineto{\pgfqpoint{0.636851in}{1.067814in}}%
\pgfpathlineto{\pgfqpoint{0.648583in}{1.064325in}}%
\pgfpathlineto{\pgfqpoint{0.656816in}{1.060311in}}%
\pgfpathlineto{\pgfqpoint{0.660315in}{1.058622in}}%
\pgfpathlineto{\pgfqpoint{0.672047in}{1.050171in}}%
\pgfpathlineto{\pgfqpoint{0.673769in}{1.048901in}}%
\pgfpathlineto{\pgfqpoint{0.683779in}{1.041110in}}%
\pgfpathlineto{\pgfqpoint{0.687474in}{1.037490in}}%
\pgfpathlineto{\pgfqpoint{0.695512in}{1.028946in}}%
\pgfpathlineto{\pgfqpoint{0.698185in}{1.026079in}}%
\pgfpathlineto{\pgfqpoint{0.707244in}{1.016250in}}%
\pgfpathlineto{\pgfqpoint{0.708692in}{1.014668in}}%
\pgfpathlineto{\pgfqpoint{0.715539in}{1.003257in}}%
\pgfpathlineto{\pgfqpoint{0.718976in}{0.996766in}}%
\pgfpathlineto{\pgfqpoint{0.721627in}{0.991847in}}%
\pgfpathlineto{\pgfqpoint{0.726581in}{0.980436in}}%
\pgfpathlineto{\pgfqpoint{0.730708in}{0.969098in}}%
\pgfpathlineto{\pgfqpoint{0.730738in}{0.969025in}}%
\pgfpathlineto{\pgfqpoint{0.731860in}{0.957614in}}%
\pgfpathlineto{\pgfqpoint{0.730708in}{0.952599in}}%
\pgfpathlineto{\pgfqpoint{0.727921in}{0.946204in}}%
\pgfpathlineto{\pgfqpoint{0.718976in}{0.938016in}}%
\pgfpathlineto{\pgfqpoint{0.714847in}{0.934793in}}%
\pgfpathlineto{\pgfqpoint{0.707244in}{0.929920in}}%
\pgfpathlineto{\pgfqpoint{0.697750in}{0.923382in}}%
\pgfpathlineto{\pgfqpoint{0.695512in}{0.921950in}}%
\pgfpathlineto{\pgfqpoint{0.683779in}{0.913346in}}%
\pgfpathlineto{\pgfqpoint{0.682057in}{0.911971in}}%
\pgfpathlineto{\pgfqpoint{0.672047in}{0.903862in}}%
\pgfpathlineto{\pgfqpoint{0.665708in}{0.900560in}}%
\pgfpathlineto{\pgfqpoint{0.660315in}{0.897582in}}%
\pgfpathclose%
\pgfusepath{fill}%
\end{pgfscope}%
\begin{pgfscope}%
\pgfpathrectangle{\pgfqpoint{0.211875in}{0.211875in}}{\pgfqpoint{1.313625in}{1.279725in}}%
\pgfusepath{clip}%
\pgfsetbuttcap%
\pgfsetroundjoin%
\definecolor{currentfill}{rgb}{0.730358,0.086862,0.337485}%
\pgfsetfillcolor{currentfill}%
\pgfsetlinewidth{0.000000pt}%
\definecolor{currentstroke}{rgb}{0.000000,0.000000,0.000000}%
\pgfsetstrokecolor{currentstroke}%
\pgfsetdash{}{0pt}%
\pgfpathmoveto{\pgfqpoint{1.047478in}{0.876107in}}%
\pgfpathlineto{\pgfqpoint{1.059210in}{0.873712in}}%
\pgfpathlineto{\pgfqpoint{1.070942in}{0.871410in}}%
\pgfpathlineto{\pgfqpoint{1.082674in}{0.871599in}}%
\pgfpathlineto{\pgfqpoint{1.094407in}{0.873789in}}%
\pgfpathlineto{\pgfqpoint{1.106139in}{0.874748in}}%
\pgfpathlineto{\pgfqpoint{1.117871in}{0.874081in}}%
\pgfpathlineto{\pgfqpoint{1.129603in}{0.874053in}}%
\pgfpathlineto{\pgfqpoint{1.141335in}{0.874755in}}%
\pgfpathlineto{\pgfqpoint{1.153068in}{0.875470in}}%
\pgfpathlineto{\pgfqpoint{1.164800in}{0.876124in}}%
\pgfpathlineto{\pgfqpoint{1.176532in}{0.876775in}}%
\pgfpathlineto{\pgfqpoint{1.188264in}{0.877686in}}%
\pgfpathlineto{\pgfqpoint{1.188659in}{0.877739in}}%
\pgfpathlineto{\pgfqpoint{1.199996in}{0.879392in}}%
\pgfpathlineto{\pgfqpoint{1.211729in}{0.882524in}}%
\pgfpathlineto{\pgfqpoint{1.223461in}{0.887042in}}%
\pgfpathlineto{\pgfqpoint{1.225911in}{0.889150in}}%
\pgfpathlineto{\pgfqpoint{1.235193in}{0.898562in}}%
\pgfpathlineto{\pgfqpoint{1.236961in}{0.900560in}}%
\pgfpathlineto{\pgfqpoint{1.246609in}{0.911971in}}%
\pgfpathlineto{\pgfqpoint{1.246925in}{0.912211in}}%
\pgfpathlineto{\pgfqpoint{1.258658in}{0.917590in}}%
\pgfpathlineto{\pgfqpoint{1.270390in}{0.922769in}}%
\pgfpathlineto{\pgfqpoint{1.271692in}{0.923382in}}%
\pgfpathlineto{\pgfqpoint{1.277617in}{0.934793in}}%
\pgfpathlineto{\pgfqpoint{1.270390in}{0.943963in}}%
\pgfpathlineto{\pgfqpoint{1.265708in}{0.946204in}}%
\pgfpathlineto{\pgfqpoint{1.258658in}{0.947802in}}%
\pgfpathlineto{\pgfqpoint{1.246925in}{0.949923in}}%
\pgfpathlineto{\pgfqpoint{1.235193in}{0.951591in}}%
\pgfpathlineto{\pgfqpoint{1.223461in}{0.952473in}}%
\pgfpathlineto{\pgfqpoint{1.211729in}{0.953571in}}%
\pgfpathlineto{\pgfqpoint{1.199996in}{0.956739in}}%
\pgfpathlineto{\pgfqpoint{1.197254in}{0.957614in}}%
\pgfpathlineto{\pgfqpoint{1.188264in}{0.959529in}}%
\pgfpathlineto{\pgfqpoint{1.176532in}{0.961078in}}%
\pgfpathlineto{\pgfqpoint{1.164800in}{0.961548in}}%
\pgfpathlineto{\pgfqpoint{1.153068in}{0.961957in}}%
\pgfpathlineto{\pgfqpoint{1.141335in}{0.962249in}}%
\pgfpathlineto{\pgfqpoint{1.129603in}{0.961697in}}%
\pgfpathlineto{\pgfqpoint{1.117871in}{0.961636in}}%
\pgfpathlineto{\pgfqpoint{1.106139in}{0.961954in}}%
\pgfpathlineto{\pgfqpoint{1.094407in}{0.964425in}}%
\pgfpathlineto{\pgfqpoint{1.082674in}{0.967383in}}%
\pgfpathlineto{\pgfqpoint{1.074472in}{0.969025in}}%
\pgfpathlineto{\pgfqpoint{1.070942in}{0.969505in}}%
\pgfpathlineto{\pgfqpoint{1.059210in}{0.970770in}}%
\pgfpathlineto{\pgfqpoint{1.047478in}{0.971971in}}%
\pgfpathlineto{\pgfqpoint{1.035746in}{0.972278in}}%
\pgfpathlineto{\pgfqpoint{1.024013in}{0.971468in}}%
\pgfpathlineto{\pgfqpoint{1.012281in}{0.969299in}}%
\pgfpathlineto{\pgfqpoint{1.011199in}{0.969025in}}%
\pgfpathlineto{\pgfqpoint{1.000549in}{0.965684in}}%
\pgfpathlineto{\pgfqpoint{0.988817in}{0.962365in}}%
\pgfpathlineto{\pgfqpoint{0.977085in}{0.959239in}}%
\pgfpathlineto{\pgfqpoint{0.971789in}{0.957614in}}%
\pgfpathlineto{\pgfqpoint{0.965352in}{0.954852in}}%
\pgfpathlineto{\pgfqpoint{0.953620in}{0.949431in}}%
\pgfpathlineto{\pgfqpoint{0.948227in}{0.946204in}}%
\pgfpathlineto{\pgfqpoint{0.941888in}{0.935811in}}%
\pgfpathlineto{\pgfqpoint{0.941363in}{0.934793in}}%
\pgfpathlineto{\pgfqpoint{0.941888in}{0.932666in}}%
\pgfpathlineto{\pgfqpoint{0.946768in}{0.923382in}}%
\pgfpathlineto{\pgfqpoint{0.953620in}{0.918199in}}%
\pgfpathlineto{\pgfqpoint{0.965352in}{0.913596in}}%
\pgfpathlineto{\pgfqpoint{0.971813in}{0.911971in}}%
\pgfpathlineto{\pgfqpoint{0.977085in}{0.910822in}}%
\pgfpathlineto{\pgfqpoint{0.988817in}{0.907706in}}%
\pgfpathlineto{\pgfqpoint{0.999477in}{0.900560in}}%
\pgfpathlineto{\pgfqpoint{1.000549in}{0.899820in}}%
\pgfpathlineto{\pgfqpoint{1.012281in}{0.892627in}}%
\pgfpathlineto{\pgfqpoint{1.017591in}{0.889150in}}%
\pgfpathlineto{\pgfqpoint{1.024013in}{0.885179in}}%
\pgfpathlineto{\pgfqpoint{1.035746in}{0.878993in}}%
\pgfpathlineto{\pgfqpoint{1.039573in}{0.877739in}}%
\pgfpathclose%
\pgfpathmoveto{\pgfqpoint{1.143866in}{0.889150in}}%
\pgfpathlineto{\pgfqpoint{1.141335in}{0.889820in}}%
\pgfpathlineto{\pgfqpoint{1.129603in}{0.893445in}}%
\pgfpathlineto{\pgfqpoint{1.117871in}{0.898156in}}%
\pgfpathlineto{\pgfqpoint{1.112535in}{0.900560in}}%
\pgfpathlineto{\pgfqpoint{1.106139in}{0.905444in}}%
\pgfpathlineto{\pgfqpoint{1.094407in}{0.905649in}}%
\pgfpathlineto{\pgfqpoint{1.082674in}{0.900998in}}%
\pgfpathlineto{\pgfqpoint{1.081345in}{0.900560in}}%
\pgfpathlineto{\pgfqpoint{1.070942in}{0.897609in}}%
\pgfpathlineto{\pgfqpoint{1.059210in}{0.894012in}}%
\pgfpathlineto{\pgfqpoint{1.047478in}{0.896379in}}%
\pgfpathlineto{\pgfqpoint{1.035746in}{0.899657in}}%
\pgfpathlineto{\pgfqpoint{1.034235in}{0.900560in}}%
\pgfpathlineto{\pgfqpoint{1.024013in}{0.908655in}}%
\pgfpathlineto{\pgfqpoint{1.016870in}{0.911971in}}%
\pgfpathlineto{\pgfqpoint{1.012281in}{0.915903in}}%
\pgfpathlineto{\pgfqpoint{1.000549in}{0.922038in}}%
\pgfpathlineto{\pgfqpoint{0.992174in}{0.923382in}}%
\pgfpathlineto{\pgfqpoint{0.988817in}{0.923978in}}%
\pgfpathlineto{\pgfqpoint{0.977085in}{0.926504in}}%
\pgfpathlineto{\pgfqpoint{0.965352in}{0.932726in}}%
\pgfpathlineto{\pgfqpoint{0.963410in}{0.934793in}}%
\pgfpathlineto{\pgfqpoint{0.965352in}{0.943420in}}%
\pgfpathlineto{\pgfqpoint{0.966489in}{0.946204in}}%
\pgfpathlineto{\pgfqpoint{0.977085in}{0.948664in}}%
\pgfpathlineto{\pgfqpoint{0.988817in}{0.951979in}}%
\pgfpathlineto{\pgfqpoint{1.000549in}{0.955877in}}%
\pgfpathlineto{\pgfqpoint{1.007357in}{0.957614in}}%
\pgfpathlineto{\pgfqpoint{1.012281in}{0.958133in}}%
\pgfpathlineto{\pgfqpoint{1.022372in}{0.957614in}}%
\pgfpathlineto{\pgfqpoint{1.024013in}{0.957223in}}%
\pgfpathlineto{\pgfqpoint{1.035746in}{0.957225in}}%
\pgfpathlineto{\pgfqpoint{1.046597in}{0.957614in}}%
\pgfpathlineto{\pgfqpoint{1.047478in}{0.957626in}}%
\pgfpathlineto{\pgfqpoint{1.048700in}{0.957614in}}%
\pgfpathlineto{\pgfqpoint{1.059210in}{0.957358in}}%
\pgfpathlineto{\pgfqpoint{1.070942in}{0.953909in}}%
\pgfpathlineto{\pgfqpoint{1.082674in}{0.950422in}}%
\pgfpathlineto{\pgfqpoint{1.094407in}{0.947383in}}%
\pgfpathlineto{\pgfqpoint{1.101144in}{0.946204in}}%
\pgfpathlineto{\pgfqpoint{1.106139in}{0.944462in}}%
\pgfpathlineto{\pgfqpoint{1.116131in}{0.946204in}}%
\pgfpathlineto{\pgfqpoint{1.117871in}{0.946325in}}%
\pgfpathlineto{\pgfqpoint{1.129603in}{0.946924in}}%
\pgfpathlineto{\pgfqpoint{1.141335in}{0.947967in}}%
\pgfpathlineto{\pgfqpoint{1.153068in}{0.949052in}}%
\pgfpathlineto{\pgfqpoint{1.164800in}{0.950546in}}%
\pgfpathlineto{\pgfqpoint{1.176532in}{0.951611in}}%
\pgfpathlineto{\pgfqpoint{1.188264in}{0.950126in}}%
\pgfpathlineto{\pgfqpoint{1.199996in}{0.948382in}}%
\pgfpathlineto{\pgfqpoint{1.211729in}{0.947152in}}%
\pgfpathlineto{\pgfqpoint{1.217881in}{0.946204in}}%
\pgfpathlineto{\pgfqpoint{1.223461in}{0.943318in}}%
\pgfpathlineto{\pgfqpoint{1.235193in}{0.941549in}}%
\pgfpathlineto{\pgfqpoint{1.246925in}{0.940708in}}%
\pgfpathlineto{\pgfqpoint{1.258658in}{0.938355in}}%
\pgfpathlineto{\pgfqpoint{1.264360in}{0.934793in}}%
\pgfpathlineto{\pgfqpoint{1.258658in}{0.928777in}}%
\pgfpathlineto{\pgfqpoint{1.247879in}{0.923382in}}%
\pgfpathlineto{\pgfqpoint{1.246925in}{0.923081in}}%
\pgfpathlineto{\pgfqpoint{1.235193in}{0.918182in}}%
\pgfpathlineto{\pgfqpoint{1.230067in}{0.911971in}}%
\pgfpathlineto{\pgfqpoint{1.223461in}{0.904622in}}%
\pgfpathlineto{\pgfqpoint{1.219239in}{0.900560in}}%
\pgfpathlineto{\pgfqpoint{1.211729in}{0.896212in}}%
\pgfpathlineto{\pgfqpoint{1.199996in}{0.892999in}}%
\pgfpathlineto{\pgfqpoint{1.188264in}{0.890016in}}%
\pgfpathlineto{\pgfqpoint{1.184216in}{0.889150in}}%
\pgfpathlineto{\pgfqpoint{1.176532in}{0.888188in}}%
\pgfpathlineto{\pgfqpoint{1.164800in}{0.887984in}}%
\pgfpathlineto{\pgfqpoint{1.153068in}{0.888121in}}%
\pgfpathclose%
\pgfusepath{fill}%
\end{pgfscope}%
\begin{pgfscope}%
\pgfpathrectangle{\pgfqpoint{0.211875in}{0.211875in}}{\pgfqpoint{1.313625in}{1.279725in}}%
\pgfusepath{clip}%
\pgfsetbuttcap%
\pgfsetroundjoin%
\definecolor{currentfill}{rgb}{0.730358,0.086862,0.337485}%
\pgfsetfillcolor{currentfill}%
\pgfsetlinewidth{0.000000pt}%
\definecolor{currentstroke}{rgb}{0.000000,0.000000,0.000000}%
\pgfsetstrokecolor{currentstroke}%
\pgfsetdash{}{0pt}%
\pgfpathmoveto{\pgfqpoint{1.293854in}{0.991301in}}%
\pgfpathlineto{\pgfqpoint{1.305586in}{0.987335in}}%
\pgfpathlineto{\pgfqpoint{1.317319in}{0.991243in}}%
\pgfpathlineto{\pgfqpoint{1.318008in}{0.991847in}}%
\pgfpathlineto{\pgfqpoint{1.325372in}{1.003257in}}%
\pgfpathlineto{\pgfqpoint{1.329051in}{1.009921in}}%
\pgfpathlineto{\pgfqpoint{1.331344in}{1.014668in}}%
\pgfpathlineto{\pgfqpoint{1.333353in}{1.026079in}}%
\pgfpathlineto{\pgfqpoint{1.332561in}{1.037490in}}%
\pgfpathlineto{\pgfqpoint{1.338671in}{1.048901in}}%
\pgfpathlineto{\pgfqpoint{1.340783in}{1.051259in}}%
\pgfpathlineto{\pgfqpoint{1.352515in}{1.055940in}}%
\pgfpathlineto{\pgfqpoint{1.357434in}{1.060311in}}%
\pgfpathlineto{\pgfqpoint{1.364247in}{1.065142in}}%
\pgfpathlineto{\pgfqpoint{1.375980in}{1.067570in}}%
\pgfpathlineto{\pgfqpoint{1.387353in}{1.071722in}}%
\pgfpathlineto{\pgfqpoint{1.387712in}{1.071888in}}%
\pgfpathlineto{\pgfqpoint{1.397640in}{1.083133in}}%
\pgfpathlineto{\pgfqpoint{1.399444in}{1.089596in}}%
\pgfpathlineto{\pgfqpoint{1.400631in}{1.094544in}}%
\pgfpathlineto{\pgfqpoint{1.400497in}{1.105954in}}%
\pgfpathlineto{\pgfqpoint{1.399444in}{1.108354in}}%
\pgfpathlineto{\pgfqpoint{1.388253in}{1.117365in}}%
\pgfpathlineto{\pgfqpoint{1.387712in}{1.117567in}}%
\pgfpathlineto{\pgfqpoint{1.375980in}{1.121638in}}%
\pgfpathlineto{\pgfqpoint{1.364247in}{1.125924in}}%
\pgfpathlineto{\pgfqpoint{1.357877in}{1.128776in}}%
\pgfpathlineto{\pgfqpoint{1.352515in}{1.130866in}}%
\pgfpathlineto{\pgfqpoint{1.342515in}{1.128776in}}%
\pgfpathlineto{\pgfqpoint{1.340783in}{1.123349in}}%
\pgfpathlineto{\pgfqpoint{1.340244in}{1.117365in}}%
\pgfpathlineto{\pgfqpoint{1.339873in}{1.105954in}}%
\pgfpathlineto{\pgfqpoint{1.340783in}{1.097648in}}%
\pgfpathlineto{\pgfqpoint{1.341228in}{1.094544in}}%
\pgfpathlineto{\pgfqpoint{1.340783in}{1.093149in}}%
\pgfpathlineto{\pgfqpoint{1.329051in}{1.090479in}}%
\pgfpathlineto{\pgfqpoint{1.317319in}{1.094026in}}%
\pgfpathlineto{\pgfqpoint{1.305586in}{1.089736in}}%
\pgfpathlineto{\pgfqpoint{1.293854in}{1.088766in}}%
\pgfpathlineto{\pgfqpoint{1.282122in}{1.093338in}}%
\pgfpathlineto{\pgfqpoint{1.278799in}{1.094544in}}%
\pgfpathlineto{\pgfqpoint{1.270390in}{1.099407in}}%
\pgfpathlineto{\pgfqpoint{1.258658in}{1.102858in}}%
\pgfpathlineto{\pgfqpoint{1.249749in}{1.105954in}}%
\pgfpathlineto{\pgfqpoint{1.246925in}{1.107591in}}%
\pgfpathlineto{\pgfqpoint{1.235193in}{1.111991in}}%
\pgfpathlineto{\pgfqpoint{1.223461in}{1.114579in}}%
\pgfpathlineto{\pgfqpoint{1.211729in}{1.116278in}}%
\pgfpathlineto{\pgfqpoint{1.204481in}{1.117365in}}%
\pgfpathlineto{\pgfqpoint{1.199996in}{1.118662in}}%
\pgfpathlineto{\pgfqpoint{1.188264in}{1.117574in}}%
\pgfpathlineto{\pgfqpoint{1.187995in}{1.117365in}}%
\pgfpathlineto{\pgfqpoint{1.176532in}{1.113726in}}%
\pgfpathlineto{\pgfqpoint{1.164800in}{1.110342in}}%
\pgfpathlineto{\pgfqpoint{1.158348in}{1.105954in}}%
\pgfpathlineto{\pgfqpoint{1.153068in}{1.100902in}}%
\pgfpathlineto{\pgfqpoint{1.148561in}{1.094544in}}%
\pgfpathlineto{\pgfqpoint{1.147593in}{1.083133in}}%
\pgfpathlineto{\pgfqpoint{1.153068in}{1.076890in}}%
\pgfpathlineto{\pgfqpoint{1.158145in}{1.071722in}}%
\pgfpathlineto{\pgfqpoint{1.164800in}{1.064010in}}%
\pgfpathlineto{\pgfqpoint{1.171364in}{1.060311in}}%
\pgfpathlineto{\pgfqpoint{1.176532in}{1.055518in}}%
\pgfpathlineto{\pgfqpoint{1.185205in}{1.048901in}}%
\pgfpathlineto{\pgfqpoint{1.188264in}{1.046262in}}%
\pgfpathlineto{\pgfqpoint{1.199996in}{1.037891in}}%
\pgfpathlineto{\pgfqpoint{1.200544in}{1.037490in}}%
\pgfpathlineto{\pgfqpoint{1.211729in}{1.028185in}}%
\pgfpathlineto{\pgfqpoint{1.215179in}{1.026079in}}%
\pgfpathlineto{\pgfqpoint{1.223461in}{1.019334in}}%
\pgfpathlineto{\pgfqpoint{1.235193in}{1.015029in}}%
\pgfpathlineto{\pgfqpoint{1.236158in}{1.014668in}}%
\pgfpathlineto{\pgfqpoint{1.246925in}{1.008578in}}%
\pgfpathlineto{\pgfqpoint{1.258658in}{1.003863in}}%
\pgfpathlineto{\pgfqpoint{1.260813in}{1.003257in}}%
\pgfpathlineto{\pgfqpoint{1.270390in}{0.999001in}}%
\pgfpathlineto{\pgfqpoint{1.282122in}{0.993699in}}%
\pgfpathlineto{\pgfqpoint{1.291599in}{0.991847in}}%
\pgfpathclose%
\pgfpathmoveto{\pgfqpoint{1.288801in}{1.003257in}}%
\pgfpathlineto{\pgfqpoint{1.282122in}{1.005993in}}%
\pgfpathlineto{\pgfqpoint{1.270390in}{1.010971in}}%
\pgfpathlineto{\pgfqpoint{1.261583in}{1.014668in}}%
\pgfpathlineto{\pgfqpoint{1.258658in}{1.015838in}}%
\pgfpathlineto{\pgfqpoint{1.246925in}{1.020657in}}%
\pgfpathlineto{\pgfqpoint{1.235193in}{1.025929in}}%
\pgfpathlineto{\pgfqpoint{1.234921in}{1.026079in}}%
\pgfpathlineto{\pgfqpoint{1.223461in}{1.031631in}}%
\pgfpathlineto{\pgfqpoint{1.214920in}{1.037490in}}%
\pgfpathlineto{\pgfqpoint{1.211729in}{1.039824in}}%
\pgfpathlineto{\pgfqpoint{1.199996in}{1.048518in}}%
\pgfpathlineto{\pgfqpoint{1.199489in}{1.048901in}}%
\pgfpathlineto{\pgfqpoint{1.188264in}{1.055408in}}%
\pgfpathlineto{\pgfqpoint{1.182306in}{1.060311in}}%
\pgfpathlineto{\pgfqpoint{1.176532in}{1.063029in}}%
\pgfpathlineto{\pgfqpoint{1.164800in}{1.070966in}}%
\pgfpathlineto{\pgfqpoint{1.164148in}{1.071722in}}%
\pgfpathlineto{\pgfqpoint{1.153068in}{1.082998in}}%
\pgfpathlineto{\pgfqpoint{1.152949in}{1.083133in}}%
\pgfpathlineto{\pgfqpoint{1.153068in}{1.084263in}}%
\pgfpathlineto{\pgfqpoint{1.154459in}{1.094544in}}%
\pgfpathlineto{\pgfqpoint{1.164800in}{1.104465in}}%
\pgfpathlineto{\pgfqpoint{1.167152in}{1.105954in}}%
\pgfpathlineto{\pgfqpoint{1.176532in}{1.109624in}}%
\pgfpathlineto{\pgfqpoint{1.188264in}{1.112951in}}%
\pgfpathlineto{\pgfqpoint{1.199996in}{1.113164in}}%
\pgfpathlineto{\pgfqpoint{1.211729in}{1.111759in}}%
\pgfpathlineto{\pgfqpoint{1.223461in}{1.109844in}}%
\pgfpathlineto{\pgfqpoint{1.235193in}{1.106692in}}%
\pgfpathlineto{\pgfqpoint{1.236915in}{1.105954in}}%
\pgfpathlineto{\pgfqpoint{1.246925in}{1.101973in}}%
\pgfpathlineto{\pgfqpoint{1.258658in}{1.097722in}}%
\pgfpathlineto{\pgfqpoint{1.265461in}{1.094544in}}%
\pgfpathlineto{\pgfqpoint{1.270390in}{1.092312in}}%
\pgfpathlineto{\pgfqpoint{1.282122in}{1.087808in}}%
\pgfpathlineto{\pgfqpoint{1.288785in}{1.083133in}}%
\pgfpathlineto{\pgfqpoint{1.293854in}{1.078393in}}%
\pgfpathlineto{\pgfqpoint{1.302153in}{1.071722in}}%
\pgfpathlineto{\pgfqpoint{1.305586in}{1.068445in}}%
\pgfpathlineto{\pgfqpoint{1.316576in}{1.060311in}}%
\pgfpathlineto{\pgfqpoint{1.317319in}{1.059334in}}%
\pgfpathlineto{\pgfqpoint{1.324760in}{1.048901in}}%
\pgfpathlineto{\pgfqpoint{1.322867in}{1.037490in}}%
\pgfpathlineto{\pgfqpoint{1.320712in}{1.026079in}}%
\pgfpathlineto{\pgfqpoint{1.319405in}{1.014668in}}%
\pgfpathlineto{\pgfqpoint{1.317319in}{1.009029in}}%
\pgfpathlineto{\pgfqpoint{1.309673in}{1.003257in}}%
\pgfpathlineto{\pgfqpoint{1.305586in}{1.001355in}}%
\pgfpathlineto{\pgfqpoint{1.293854in}{1.001431in}}%
\pgfpathclose%
\pgfpathmoveto{\pgfqpoint{1.364727in}{1.083133in}}%
\pgfpathlineto{\pgfqpoint{1.364247in}{1.083973in}}%
\pgfpathlineto{\pgfqpoint{1.360232in}{1.094544in}}%
\pgfpathlineto{\pgfqpoint{1.360259in}{1.105954in}}%
\pgfpathlineto{\pgfqpoint{1.364247in}{1.110731in}}%
\pgfpathlineto{\pgfqpoint{1.375980in}{1.108360in}}%
\pgfpathlineto{\pgfqpoint{1.380165in}{1.105954in}}%
\pgfpathlineto{\pgfqpoint{1.387712in}{1.099667in}}%
\pgfpathlineto{\pgfqpoint{1.390349in}{1.094544in}}%
\pgfpathlineto{\pgfqpoint{1.389843in}{1.083133in}}%
\pgfpathlineto{\pgfqpoint{1.387712in}{1.080719in}}%
\pgfpathlineto{\pgfqpoint{1.375980in}{1.078064in}}%
\pgfpathclose%
\pgfusepath{fill}%
\end{pgfscope}%
\begin{pgfscope}%
\pgfpathrectangle{\pgfqpoint{0.211875in}{0.211875in}}{\pgfqpoint{1.313625in}{1.279725in}}%
\pgfusepath{clip}%
\pgfsetbuttcap%
\pgfsetroundjoin%
\definecolor{currentfill}{rgb}{0.730358,0.086862,0.337485}%
\pgfsetfillcolor{currentfill}%
\pgfsetlinewidth{0.000000pt}%
\definecolor{currentstroke}{rgb}{0.000000,0.000000,0.000000}%
\pgfsetstrokecolor{currentstroke}%
\pgfsetdash{}{0pt}%
\pgfpathmoveto{\pgfqpoint{1.434641in}{0.988688in}}%
\pgfpathlineto{\pgfqpoint{1.446373in}{0.983143in}}%
\pgfpathlineto{\pgfqpoint{1.446373in}{0.990905in}}%
\pgfpathlineto{\pgfqpoint{1.444411in}{0.991847in}}%
\pgfpathlineto{\pgfqpoint{1.434641in}{0.998612in}}%
\pgfpathlineto{\pgfqpoint{1.429475in}{1.003257in}}%
\pgfpathlineto{\pgfqpoint{1.422908in}{1.013983in}}%
\pgfpathlineto{\pgfqpoint{1.422661in}{1.014668in}}%
\pgfpathlineto{\pgfqpoint{1.420471in}{1.026079in}}%
\pgfpathlineto{\pgfqpoint{1.420300in}{1.037490in}}%
\pgfpathlineto{\pgfqpoint{1.421716in}{1.048901in}}%
\pgfpathlineto{\pgfqpoint{1.422908in}{1.050892in}}%
\pgfpathlineto{\pgfqpoint{1.432779in}{1.060311in}}%
\pgfpathlineto{\pgfqpoint{1.434641in}{1.061879in}}%
\pgfpathlineto{\pgfqpoint{1.446373in}{1.071052in}}%
\pgfpathlineto{\pgfqpoint{1.446373in}{1.071722in}}%
\pgfpathlineto{\pgfqpoint{1.446373in}{1.077540in}}%
\pgfpathlineto{\pgfqpoint{1.439284in}{1.071722in}}%
\pgfpathlineto{\pgfqpoint{1.434641in}{1.068023in}}%
\pgfpathlineto{\pgfqpoint{1.425481in}{1.060311in}}%
\pgfpathlineto{\pgfqpoint{1.422908in}{1.057856in}}%
\pgfpathlineto{\pgfqpoint{1.417548in}{1.048901in}}%
\pgfpathlineto{\pgfqpoint{1.415352in}{1.037490in}}%
\pgfpathlineto{\pgfqpoint{1.414466in}{1.026079in}}%
\pgfpathlineto{\pgfqpoint{1.415317in}{1.014668in}}%
\pgfpathlineto{\pgfqpoint{1.418603in}{1.003257in}}%
\pgfpathlineto{\pgfqpoint{1.422908in}{0.997470in}}%
\pgfpathlineto{\pgfqpoint{1.429208in}{0.991847in}}%
\pgfpathclose%
\pgfusepath{fill}%
\end{pgfscope}%
\begin{pgfscope}%
\pgfpathrectangle{\pgfqpoint{0.211875in}{0.211875in}}{\pgfqpoint{1.313625in}{1.279725in}}%
\pgfusepath{clip}%
\pgfsetbuttcap%
\pgfsetroundjoin%
\definecolor{currentfill}{rgb}{0.730358,0.086862,0.337485}%
\pgfsetfillcolor{currentfill}%
\pgfsetlinewidth{0.000000pt}%
\definecolor{currentstroke}{rgb}{0.000000,0.000000,0.000000}%
\pgfsetstrokecolor{currentstroke}%
\pgfsetdash{}{0pt}%
\pgfpathmoveto{\pgfqpoint{0.918424in}{1.003158in}}%
\pgfpathlineto{\pgfqpoint{0.918700in}{1.003257in}}%
\pgfpathlineto{\pgfqpoint{0.918424in}{1.004083in}}%
\pgfpathlineto{\pgfqpoint{0.918095in}{1.003257in}}%
\pgfpathclose%
\pgfusepath{fill}%
\end{pgfscope}%
\begin{pgfscope}%
\pgfpathrectangle{\pgfqpoint{0.211875in}{0.211875in}}{\pgfqpoint{1.313625in}{1.279725in}}%
\pgfusepath{clip}%
\pgfsetbuttcap%
\pgfsetroundjoin%
\definecolor{currentfill}{rgb}{0.730358,0.086862,0.337485}%
\pgfsetfillcolor{currentfill}%
\pgfsetlinewidth{0.000000pt}%
\definecolor{currentstroke}{rgb}{0.000000,0.000000,0.000000}%
\pgfsetstrokecolor{currentstroke}%
\pgfsetdash{}{0pt}%
\pgfpathmoveto{\pgfqpoint{0.930156in}{1.009288in}}%
\pgfpathlineto{\pgfqpoint{0.941888in}{1.012667in}}%
\pgfpathlineto{\pgfqpoint{0.946157in}{1.014668in}}%
\pgfpathlineto{\pgfqpoint{0.953620in}{1.018443in}}%
\pgfpathlineto{\pgfqpoint{0.965352in}{1.022512in}}%
\pgfpathlineto{\pgfqpoint{0.971975in}{1.026079in}}%
\pgfpathlineto{\pgfqpoint{0.977085in}{1.030407in}}%
\pgfpathlineto{\pgfqpoint{0.988817in}{1.037108in}}%
\pgfpathlineto{\pgfqpoint{0.989381in}{1.037490in}}%
\pgfpathlineto{\pgfqpoint{0.988817in}{1.041087in}}%
\pgfpathlineto{\pgfqpoint{0.977085in}{1.045630in}}%
\pgfpathlineto{\pgfqpoint{0.965352in}{1.043512in}}%
\pgfpathlineto{\pgfqpoint{0.953620in}{1.040438in}}%
\pgfpathlineto{\pgfqpoint{0.943032in}{1.037490in}}%
\pgfpathlineto{\pgfqpoint{0.941888in}{1.036952in}}%
\pgfpathlineto{\pgfqpoint{0.930156in}{1.026524in}}%
\pgfpathlineto{\pgfqpoint{0.929827in}{1.026079in}}%
\pgfpathlineto{\pgfqpoint{0.921446in}{1.014668in}}%
\pgfpathclose%
\pgfusepath{fill}%
\end{pgfscope}%
\begin{pgfscope}%
\pgfpathrectangle{\pgfqpoint{0.211875in}{0.211875in}}{\pgfqpoint{1.313625in}{1.279725in}}%
\pgfusepath{clip}%
\pgfsetbuttcap%
\pgfsetroundjoin%
\definecolor{currentfill}{rgb}{0.730358,0.086862,0.337485}%
\pgfsetfillcolor{currentfill}%
\pgfsetlinewidth{0.000000pt}%
\definecolor{currentstroke}{rgb}{0.000000,0.000000,0.000000}%
\pgfsetstrokecolor{currentstroke}%
\pgfsetdash{}{0pt}%
\pgfpathmoveto{\pgfqpoint{1.000549in}{1.047774in}}%
\pgfpathlineto{\pgfqpoint{1.002742in}{1.048901in}}%
\pgfpathlineto{\pgfqpoint{1.000549in}{1.049829in}}%
\pgfpathlineto{\pgfqpoint{0.994692in}{1.048901in}}%
\pgfpathclose%
\pgfusepath{fill}%
\end{pgfscope}%
\begin{pgfscope}%
\pgfpathrectangle{\pgfqpoint{0.211875in}{0.211875in}}{\pgfqpoint{1.313625in}{1.279725in}}%
\pgfusepath{clip}%
\pgfsetbuttcap%
\pgfsetroundjoin%
\definecolor{currentfill}{rgb}{0.730358,0.086862,0.337485}%
\pgfsetfillcolor{currentfill}%
\pgfsetlinewidth{0.000000pt}%
\definecolor{currentstroke}{rgb}{0.000000,0.000000,0.000000}%
\pgfsetstrokecolor{currentstroke}%
\pgfsetdash{}{0pt}%
\pgfpathmoveto{\pgfqpoint{1.188264in}{1.150532in}}%
\pgfpathlineto{\pgfqpoint{1.199996in}{1.151562in}}%
\pgfpathlineto{\pgfqpoint{1.200052in}{1.151598in}}%
\pgfpathlineto{\pgfqpoint{1.211729in}{1.162983in}}%
\pgfpathlineto{\pgfqpoint{1.211750in}{1.163008in}}%
\pgfpathlineto{\pgfqpoint{1.213534in}{1.174419in}}%
\pgfpathlineto{\pgfqpoint{1.211729in}{1.178318in}}%
\pgfpathlineto{\pgfqpoint{1.207889in}{1.185830in}}%
\pgfpathlineto{\pgfqpoint{1.203683in}{1.197241in}}%
\pgfpathlineto{\pgfqpoint{1.199996in}{1.206066in}}%
\pgfpathlineto{\pgfqpoint{1.197607in}{1.208651in}}%
\pgfpathlineto{\pgfqpoint{1.188264in}{1.214293in}}%
\pgfpathlineto{\pgfqpoint{1.177934in}{1.220062in}}%
\pgfpathlineto{\pgfqpoint{1.176532in}{1.220896in}}%
\pgfpathlineto{\pgfqpoint{1.164800in}{1.227137in}}%
\pgfpathlineto{\pgfqpoint{1.156885in}{1.231473in}}%
\pgfpathlineto{\pgfqpoint{1.153068in}{1.233542in}}%
\pgfpathlineto{\pgfqpoint{1.141335in}{1.239904in}}%
\pgfpathlineto{\pgfqpoint{1.135924in}{1.242884in}}%
\pgfpathlineto{\pgfqpoint{1.129603in}{1.246304in}}%
\pgfpathlineto{\pgfqpoint{1.117871in}{1.248851in}}%
\pgfpathlineto{\pgfqpoint{1.106139in}{1.249609in}}%
\pgfpathlineto{\pgfqpoint{1.094407in}{1.248554in}}%
\pgfpathlineto{\pgfqpoint{1.089760in}{1.242884in}}%
\pgfpathlineto{\pgfqpoint{1.091363in}{1.231473in}}%
\pgfpathlineto{\pgfqpoint{1.094407in}{1.221266in}}%
\pgfpathlineto{\pgfqpoint{1.094740in}{1.220062in}}%
\pgfpathlineto{\pgfqpoint{1.099912in}{1.208651in}}%
\pgfpathlineto{\pgfqpoint{1.106139in}{1.201315in}}%
\pgfpathlineto{\pgfqpoint{1.108894in}{1.197241in}}%
\pgfpathlineto{\pgfqpoint{1.117871in}{1.187378in}}%
\pgfpathlineto{\pgfqpoint{1.119479in}{1.185830in}}%
\pgfpathlineto{\pgfqpoint{1.129603in}{1.176366in}}%
\pgfpathlineto{\pgfqpoint{1.131871in}{1.174419in}}%
\pgfpathlineto{\pgfqpoint{1.141335in}{1.166554in}}%
\pgfpathlineto{\pgfqpoint{1.145206in}{1.163008in}}%
\pgfpathlineto{\pgfqpoint{1.153068in}{1.156104in}}%
\pgfpathlineto{\pgfqpoint{1.164800in}{1.154651in}}%
\pgfpathlineto{\pgfqpoint{1.176532in}{1.153539in}}%
\pgfpathlineto{\pgfqpoint{1.183545in}{1.151598in}}%
\pgfpathclose%
\pgfpathmoveto{\pgfqpoint{1.168617in}{1.174419in}}%
\pgfpathlineto{\pgfqpoint{1.164800in}{1.177515in}}%
\pgfpathlineto{\pgfqpoint{1.153968in}{1.185830in}}%
\pgfpathlineto{\pgfqpoint{1.164800in}{1.191925in}}%
\pgfpathlineto{\pgfqpoint{1.176532in}{1.187833in}}%
\pgfpathlineto{\pgfqpoint{1.179278in}{1.185830in}}%
\pgfpathlineto{\pgfqpoint{1.179881in}{1.174419in}}%
\pgfpathlineto{\pgfqpoint{1.176532in}{1.172362in}}%
\pgfpathclose%
\pgfusepath{fill}%
\end{pgfscope}%
\begin{pgfscope}%
\pgfpathrectangle{\pgfqpoint{0.211875in}{0.211875in}}{\pgfqpoint{1.313625in}{1.279725in}}%
\pgfusepath{clip}%
\pgfsetbuttcap%
\pgfsetroundjoin%
\definecolor{currentfill}{rgb}{0.730358,0.086862,0.337485}%
\pgfsetfillcolor{currentfill}%
\pgfsetlinewidth{0.000000pt}%
\definecolor{currentstroke}{rgb}{0.000000,0.000000,0.000000}%
\pgfsetstrokecolor{currentstroke}%
\pgfsetdash{}{0pt}%
\pgfpathmoveto{\pgfqpoint{1.305586in}{1.185820in}}%
\pgfpathlineto{\pgfqpoint{1.317319in}{1.175262in}}%
\pgfpathlineto{\pgfqpoint{1.329051in}{1.180984in}}%
\pgfpathlineto{\pgfqpoint{1.335081in}{1.185830in}}%
\pgfpathlineto{\pgfqpoint{1.340783in}{1.190830in}}%
\pgfpathlineto{\pgfqpoint{1.349711in}{1.197241in}}%
\pgfpathlineto{\pgfqpoint{1.352515in}{1.199471in}}%
\pgfpathlineto{\pgfqpoint{1.364247in}{1.206238in}}%
\pgfpathlineto{\pgfqpoint{1.371283in}{1.208651in}}%
\pgfpathlineto{\pgfqpoint{1.375980in}{1.210460in}}%
\pgfpathlineto{\pgfqpoint{1.387712in}{1.212045in}}%
\pgfpathlineto{\pgfqpoint{1.399444in}{1.211059in}}%
\pgfpathlineto{\pgfqpoint{1.408638in}{1.208651in}}%
\pgfpathlineto{\pgfqpoint{1.411176in}{1.208063in}}%
\pgfpathlineto{\pgfqpoint{1.422908in}{1.203690in}}%
\pgfpathlineto{\pgfqpoint{1.433387in}{1.197241in}}%
\pgfpathlineto{\pgfqpoint{1.434641in}{1.196276in}}%
\pgfpathlineto{\pgfqpoint{1.445477in}{1.185830in}}%
\pgfpathlineto{\pgfqpoint{1.446373in}{1.185045in}}%
\pgfpathlineto{\pgfqpoint{1.446373in}{1.185830in}}%
\pgfpathlineto{\pgfqpoint{1.446373in}{1.197241in}}%
\pgfpathlineto{\pgfqpoint{1.446373in}{1.208651in}}%
\pgfpathlineto{\pgfqpoint{1.446373in}{1.220062in}}%
\pgfpathlineto{\pgfqpoint{1.446373in}{1.227825in}}%
\pgfpathlineto{\pgfqpoint{1.441140in}{1.231473in}}%
\pgfpathlineto{\pgfqpoint{1.434641in}{1.236702in}}%
\pgfpathlineto{\pgfqpoint{1.426168in}{1.242884in}}%
\pgfpathlineto{\pgfqpoint{1.422908in}{1.245895in}}%
\pgfpathlineto{\pgfqpoint{1.412055in}{1.254295in}}%
\pgfpathlineto{\pgfqpoint{1.411176in}{1.255202in}}%
\pgfpathlineto{\pgfqpoint{1.399492in}{1.265705in}}%
\pgfpathlineto{\pgfqpoint{1.399444in}{1.265844in}}%
\pgfpathlineto{\pgfqpoint{1.394173in}{1.277116in}}%
\pgfpathlineto{\pgfqpoint{1.399444in}{1.281446in}}%
\pgfpathlineto{\pgfqpoint{1.402257in}{1.288527in}}%
\pgfpathlineto{\pgfqpoint{1.411176in}{1.296370in}}%
\pgfpathlineto{\pgfqpoint{1.413728in}{1.299938in}}%
\pgfpathlineto{\pgfqpoint{1.422908in}{1.307865in}}%
\pgfpathlineto{\pgfqpoint{1.427232in}{1.311348in}}%
\pgfpathlineto{\pgfqpoint{1.434641in}{1.315104in}}%
\pgfpathlineto{\pgfqpoint{1.446373in}{1.320117in}}%
\pgfpathlineto{\pgfqpoint{1.446373in}{1.322759in}}%
\pgfpathlineto{\pgfqpoint{1.446373in}{1.334083in}}%
\pgfpathlineto{\pgfqpoint{1.434641in}{1.328742in}}%
\pgfpathlineto{\pgfqpoint{1.422908in}{1.323638in}}%
\pgfpathlineto{\pgfqpoint{1.420703in}{1.322759in}}%
\pgfpathlineto{\pgfqpoint{1.411176in}{1.317778in}}%
\pgfpathlineto{\pgfqpoint{1.401461in}{1.311348in}}%
\pgfpathlineto{\pgfqpoint{1.399444in}{1.309789in}}%
\pgfpathlineto{\pgfqpoint{1.387712in}{1.305743in}}%
\pgfpathlineto{\pgfqpoint{1.375980in}{1.306900in}}%
\pgfpathlineto{\pgfqpoint{1.364247in}{1.310983in}}%
\pgfpathlineto{\pgfqpoint{1.363685in}{1.311348in}}%
\pgfpathlineto{\pgfqpoint{1.352515in}{1.319103in}}%
\pgfpathlineto{\pgfqpoint{1.349060in}{1.322759in}}%
\pgfpathlineto{\pgfqpoint{1.340920in}{1.334170in}}%
\pgfpathlineto{\pgfqpoint{1.340783in}{1.334416in}}%
\pgfpathlineto{\pgfqpoint{1.335982in}{1.345581in}}%
\pgfpathlineto{\pgfqpoint{1.331095in}{1.356992in}}%
\pgfpathlineto{\pgfqpoint{1.329051in}{1.360280in}}%
\pgfpathlineto{\pgfqpoint{1.324562in}{1.368402in}}%
\pgfpathlineto{\pgfqpoint{1.317319in}{1.379474in}}%
\pgfpathlineto{\pgfqpoint{1.316513in}{1.379813in}}%
\pgfpathlineto{\pgfqpoint{1.305586in}{1.384524in}}%
\pgfpathlineto{\pgfqpoint{1.293854in}{1.389581in}}%
\pgfpathlineto{\pgfqpoint{1.289909in}{1.391224in}}%
\pgfpathlineto{\pgfqpoint{1.282122in}{1.394659in}}%
\pgfpathlineto{\pgfqpoint{1.270390in}{1.399539in}}%
\pgfpathlineto{\pgfqpoint{1.258658in}{1.400368in}}%
\pgfpathlineto{\pgfqpoint{1.246925in}{1.397965in}}%
\pgfpathlineto{\pgfqpoint{1.240455in}{1.391224in}}%
\pgfpathlineto{\pgfqpoint{1.235193in}{1.384900in}}%
\pgfpathlineto{\pgfqpoint{1.231374in}{1.379813in}}%
\pgfpathlineto{\pgfqpoint{1.223461in}{1.368671in}}%
\pgfpathlineto{\pgfqpoint{1.223281in}{1.368402in}}%
\pgfpathlineto{\pgfqpoint{1.216047in}{1.356992in}}%
\pgfpathlineto{\pgfqpoint{1.211729in}{1.349454in}}%
\pgfpathlineto{\pgfqpoint{1.209657in}{1.345581in}}%
\pgfpathlineto{\pgfqpoint{1.204172in}{1.334170in}}%
\pgfpathlineto{\pgfqpoint{1.200810in}{1.322759in}}%
\pgfpathlineto{\pgfqpoint{1.207135in}{1.311348in}}%
\pgfpathlineto{\pgfqpoint{1.211729in}{1.301805in}}%
\pgfpathlineto{\pgfqpoint{1.212821in}{1.299938in}}%
\pgfpathlineto{\pgfqpoint{1.219049in}{1.288527in}}%
\pgfpathlineto{\pgfqpoint{1.223461in}{1.281322in}}%
\pgfpathlineto{\pgfqpoint{1.225888in}{1.277116in}}%
\pgfpathlineto{\pgfqpoint{1.232792in}{1.265705in}}%
\pgfpathlineto{\pgfqpoint{1.235193in}{1.261975in}}%
\pgfpathlineto{\pgfqpoint{1.240113in}{1.254295in}}%
\pgfpathlineto{\pgfqpoint{1.246925in}{1.244956in}}%
\pgfpathlineto{\pgfqpoint{1.248461in}{1.242884in}}%
\pgfpathlineto{\pgfqpoint{1.258658in}{1.232014in}}%
\pgfpathlineto{\pgfqpoint{1.259156in}{1.231473in}}%
\pgfpathlineto{\pgfqpoint{1.268905in}{1.220062in}}%
\pgfpathlineto{\pgfqpoint{1.270390in}{1.218506in}}%
\pgfpathlineto{\pgfqpoint{1.279599in}{1.208651in}}%
\pgfpathlineto{\pgfqpoint{1.282122in}{1.206511in}}%
\pgfpathlineto{\pgfqpoint{1.293854in}{1.198369in}}%
\pgfpathlineto{\pgfqpoint{1.295568in}{1.197241in}}%
\pgfpathlineto{\pgfqpoint{1.305579in}{1.185830in}}%
\pgfpathclose%
\pgfpathmoveto{\pgfqpoint{1.316666in}{1.208651in}}%
\pgfpathlineto{\pgfqpoint{1.305586in}{1.216719in}}%
\pgfpathlineto{\pgfqpoint{1.293854in}{1.218393in}}%
\pgfpathlineto{\pgfqpoint{1.286997in}{1.220062in}}%
\pgfpathlineto{\pgfqpoint{1.282122in}{1.222085in}}%
\pgfpathlineto{\pgfqpoint{1.270390in}{1.230820in}}%
\pgfpathlineto{\pgfqpoint{1.269846in}{1.231473in}}%
\pgfpathlineto{\pgfqpoint{1.261177in}{1.242884in}}%
\pgfpathlineto{\pgfqpoint{1.258658in}{1.245693in}}%
\pgfpathlineto{\pgfqpoint{1.254598in}{1.254295in}}%
\pgfpathlineto{\pgfqpoint{1.246925in}{1.263644in}}%
\pgfpathlineto{\pgfqpoint{1.245826in}{1.265705in}}%
\pgfpathlineto{\pgfqpoint{1.243503in}{1.277116in}}%
\pgfpathlineto{\pgfqpoint{1.244828in}{1.288527in}}%
\pgfpathlineto{\pgfqpoint{1.246925in}{1.299265in}}%
\pgfpathlineto{\pgfqpoint{1.247108in}{1.299938in}}%
\pgfpathlineto{\pgfqpoint{1.256297in}{1.311348in}}%
\pgfpathlineto{\pgfqpoint{1.258658in}{1.313845in}}%
\pgfpathlineto{\pgfqpoint{1.268022in}{1.322759in}}%
\pgfpathlineto{\pgfqpoint{1.270390in}{1.324853in}}%
\pgfpathlineto{\pgfqpoint{1.282118in}{1.334170in}}%
\pgfpathlineto{\pgfqpoint{1.282122in}{1.334173in}}%
\pgfpathlineto{\pgfqpoint{1.282205in}{1.334170in}}%
\pgfpathlineto{\pgfqpoint{1.293854in}{1.333805in}}%
\pgfpathlineto{\pgfqpoint{1.305586in}{1.328894in}}%
\pgfpathlineto{\pgfqpoint{1.316380in}{1.322759in}}%
\pgfpathlineto{\pgfqpoint{1.317319in}{1.322246in}}%
\pgfpathlineto{\pgfqpoint{1.321547in}{1.311348in}}%
\pgfpathlineto{\pgfqpoint{1.327153in}{1.299938in}}%
\pgfpathlineto{\pgfqpoint{1.329051in}{1.296218in}}%
\pgfpathlineto{\pgfqpoint{1.333893in}{1.288527in}}%
\pgfpathlineto{\pgfqpoint{1.339639in}{1.277116in}}%
\pgfpathlineto{\pgfqpoint{1.340783in}{1.270709in}}%
\pgfpathlineto{\pgfqpoint{1.341787in}{1.265705in}}%
\pgfpathlineto{\pgfqpoint{1.346384in}{1.254295in}}%
\pgfpathlineto{\pgfqpoint{1.342395in}{1.242884in}}%
\pgfpathlineto{\pgfqpoint{1.340783in}{1.240813in}}%
\pgfpathlineto{\pgfqpoint{1.334784in}{1.231473in}}%
\pgfpathlineto{\pgfqpoint{1.329051in}{1.223535in}}%
\pgfpathlineto{\pgfqpoint{1.326740in}{1.220062in}}%
\pgfpathlineto{\pgfqpoint{1.318182in}{1.208651in}}%
\pgfpathlineto{\pgfqpoint{1.317319in}{1.207789in}}%
\pgfpathclose%
\pgfusepath{fill}%
\end{pgfscope}%
\begin{pgfscope}%
\pgfpathrectangle{\pgfqpoint{0.211875in}{0.211875in}}{\pgfqpoint{1.313625in}{1.279725in}}%
\pgfusepath{clip}%
\pgfsetbuttcap%
\pgfsetroundjoin%
\definecolor{currentfill}{rgb}{0.730358,0.086862,0.337485}%
\pgfsetfillcolor{currentfill}%
\pgfsetlinewidth{0.000000pt}%
\definecolor{currentstroke}{rgb}{0.000000,0.000000,0.000000}%
\pgfsetstrokecolor{currentstroke}%
\pgfsetdash{}{0pt}%
\pgfpathmoveto{\pgfqpoint{0.355278in}{1.207888in}}%
\pgfpathlineto{\pgfqpoint{0.367010in}{1.205651in}}%
\pgfpathlineto{\pgfqpoint{0.378742in}{1.203649in}}%
\pgfpathlineto{\pgfqpoint{0.390474in}{1.201969in}}%
\pgfpathlineto{\pgfqpoint{0.402206in}{1.200827in}}%
\pgfpathlineto{\pgfqpoint{0.413939in}{1.200775in}}%
\pgfpathlineto{\pgfqpoint{0.425671in}{1.204879in}}%
\pgfpathlineto{\pgfqpoint{0.428071in}{1.208651in}}%
\pgfpathlineto{\pgfqpoint{0.427389in}{1.220062in}}%
\pgfpathlineto{\pgfqpoint{0.425671in}{1.223296in}}%
\pgfpathlineto{\pgfqpoint{0.421663in}{1.231473in}}%
\pgfpathlineto{\pgfqpoint{0.413939in}{1.242282in}}%
\pgfpathlineto{\pgfqpoint{0.413574in}{1.242884in}}%
\pgfpathlineto{\pgfqpoint{0.403179in}{1.254295in}}%
\pgfpathlineto{\pgfqpoint{0.402206in}{1.255238in}}%
\pgfpathlineto{\pgfqpoint{0.392140in}{1.265705in}}%
\pgfpathlineto{\pgfqpoint{0.390474in}{1.267273in}}%
\pgfpathlineto{\pgfqpoint{0.380613in}{1.277116in}}%
\pgfpathlineto{\pgfqpoint{0.378742in}{1.278831in}}%
\pgfpathlineto{\pgfqpoint{0.368667in}{1.288527in}}%
\pgfpathlineto{\pgfqpoint{0.367010in}{1.290012in}}%
\pgfpathlineto{\pgfqpoint{0.356356in}{1.299938in}}%
\pgfpathlineto{\pgfqpoint{0.355278in}{1.300886in}}%
\pgfpathlineto{\pgfqpoint{0.343723in}{1.311348in}}%
\pgfpathlineto{\pgfqpoint{0.343545in}{1.311508in}}%
\pgfpathlineto{\pgfqpoint{0.331813in}{1.321787in}}%
\pgfpathlineto{\pgfqpoint{0.330754in}{1.322759in}}%
\pgfpathlineto{\pgfqpoint{0.320081in}{1.332309in}}%
\pgfpathlineto{\pgfqpoint{0.317985in}{1.334170in}}%
\pgfpathlineto{\pgfqpoint{0.308349in}{1.342886in}}%
\pgfpathlineto{\pgfqpoint{0.305369in}{1.345581in}}%
\pgfpathlineto{\pgfqpoint{0.296617in}{1.353333in}}%
\pgfpathlineto{\pgfqpoint{0.292577in}{1.356992in}}%
\pgfpathlineto{\pgfqpoint{0.284884in}{1.363849in}}%
\pgfpathlineto{\pgfqpoint{0.284884in}{1.356992in}}%
\pgfpathlineto{\pgfqpoint{0.284884in}{1.345581in}}%
\pgfpathlineto{\pgfqpoint{0.284884in}{1.344111in}}%
\pgfpathlineto{\pgfqpoint{0.296104in}{1.334170in}}%
\pgfpathlineto{\pgfqpoint{0.296617in}{1.333649in}}%
\pgfpathlineto{\pgfqpoint{0.308349in}{1.322787in}}%
\pgfpathlineto{\pgfqpoint{0.308381in}{1.322759in}}%
\pgfpathlineto{\pgfqpoint{0.320037in}{1.311348in}}%
\pgfpathlineto{\pgfqpoint{0.320081in}{1.311296in}}%
\pgfpathlineto{\pgfqpoint{0.330795in}{1.299938in}}%
\pgfpathlineto{\pgfqpoint{0.331813in}{1.298557in}}%
\pgfpathlineto{\pgfqpoint{0.340280in}{1.288527in}}%
\pgfpathlineto{\pgfqpoint{0.343545in}{1.283229in}}%
\pgfpathlineto{\pgfqpoint{0.347929in}{1.277116in}}%
\pgfpathlineto{\pgfqpoint{0.351189in}{1.265705in}}%
\pgfpathlineto{\pgfqpoint{0.343545in}{1.259503in}}%
\pgfpathlineto{\pgfqpoint{0.331813in}{1.258422in}}%
\pgfpathlineto{\pgfqpoint{0.320081in}{1.259675in}}%
\pgfpathlineto{\pgfqpoint{0.308349in}{1.261512in}}%
\pgfpathlineto{\pgfqpoint{0.296617in}{1.263626in}}%
\pgfpathlineto{\pgfqpoint{0.286056in}{1.265705in}}%
\pgfpathlineto{\pgfqpoint{0.284884in}{1.265959in}}%
\pgfpathlineto{\pgfqpoint{0.284884in}{1.265705in}}%
\pgfpathlineto{\pgfqpoint{0.284884in}{1.254295in}}%
\pgfpathlineto{\pgfqpoint{0.284884in}{1.242884in}}%
\pgfpathlineto{\pgfqpoint{0.284884in}{1.231473in}}%
\pgfpathlineto{\pgfqpoint{0.284884in}{1.220077in}}%
\pgfpathlineto{\pgfqpoint{0.284982in}{1.220062in}}%
\pgfpathlineto{\pgfqpoint{0.296617in}{1.217982in}}%
\pgfpathlineto{\pgfqpoint{0.308349in}{1.215911in}}%
\pgfpathlineto{\pgfqpoint{0.320081in}{1.213878in}}%
\pgfpathlineto{\pgfqpoint{0.331813in}{1.211890in}}%
\pgfpathlineto{\pgfqpoint{0.343545in}{1.209947in}}%
\pgfpathlineto{\pgfqpoint{0.351416in}{1.208651in}}%
\pgfpathclose%
\pgfusepath{fill}%
\end{pgfscope}%
\begin{pgfscope}%
\pgfpathrectangle{\pgfqpoint{0.211875in}{0.211875in}}{\pgfqpoint{1.313625in}{1.279725in}}%
\pgfusepath{clip}%
\pgfsetbuttcap%
\pgfsetroundjoin%
\definecolor{currentfill}{rgb}{0.730358,0.086862,0.337485}%
\pgfsetfillcolor{currentfill}%
\pgfsetlinewidth{0.000000pt}%
\definecolor{currentstroke}{rgb}{0.000000,0.000000,0.000000}%
\pgfsetstrokecolor{currentstroke}%
\pgfsetdash{}{0pt}%
\pgfpathmoveto{\pgfqpoint{0.625118in}{1.378092in}}%
\pgfpathlineto{\pgfqpoint{0.626768in}{1.379813in}}%
\pgfpathlineto{\pgfqpoint{0.626588in}{1.391224in}}%
\pgfpathlineto{\pgfqpoint{0.625118in}{1.394285in}}%
\pgfpathlineto{\pgfqpoint{0.621993in}{1.402635in}}%
\pgfpathlineto{\pgfqpoint{0.615629in}{1.414045in}}%
\pgfpathlineto{\pgfqpoint{0.613386in}{1.414045in}}%
\pgfpathlineto{\pgfqpoint{0.604783in}{1.414045in}}%
\pgfpathlineto{\pgfqpoint{0.601654in}{1.408081in}}%
\pgfpathlineto{\pgfqpoint{0.589922in}{1.412720in}}%
\pgfpathlineto{\pgfqpoint{0.587929in}{1.414045in}}%
\pgfpathlineto{\pgfqpoint{0.578190in}{1.414045in}}%
\pgfpathlineto{\pgfqpoint{0.568860in}{1.414045in}}%
\pgfpathlineto{\pgfqpoint{0.578190in}{1.407278in}}%
\pgfpathlineto{\pgfqpoint{0.584745in}{1.402635in}}%
\pgfpathlineto{\pgfqpoint{0.589922in}{1.398801in}}%
\pgfpathlineto{\pgfqpoint{0.600299in}{1.391224in}}%
\pgfpathlineto{\pgfqpoint{0.601654in}{1.390187in}}%
\pgfpathlineto{\pgfqpoint{0.613386in}{1.383835in}}%
\pgfpathlineto{\pgfqpoint{0.621702in}{1.379813in}}%
\pgfpathclose%
\pgfusepath{fill}%
\end{pgfscope}%
\begin{pgfscope}%
\pgfpathrectangle{\pgfqpoint{0.211875in}{0.211875in}}{\pgfqpoint{1.313625in}{1.279725in}}%
\pgfusepath{clip}%
\pgfsetbuttcap%
\pgfsetroundjoin%
\definecolor{currentfill}{rgb}{0.848131,0.150999,0.281943}%
\pgfsetfillcolor{currentfill}%
\pgfsetlinewidth{0.000000pt}%
\definecolor{currentstroke}{rgb}{0.000000,0.000000,0.000000}%
\pgfsetstrokecolor{currentstroke}%
\pgfsetdash{}{0pt}%
\pgfpathmoveto{\pgfqpoint{1.035746in}{0.285470in}}%
\pgfpathlineto{\pgfqpoint{1.036433in}{0.284378in}}%
\pgfpathlineto{\pgfqpoint{1.047478in}{0.284378in}}%
\pgfpathlineto{\pgfqpoint{1.059210in}{0.284378in}}%
\pgfpathlineto{\pgfqpoint{1.062152in}{0.284378in}}%
\pgfpathlineto{\pgfqpoint{1.059210in}{0.289022in}}%
\pgfpathlineto{\pgfqpoint{1.055139in}{0.295789in}}%
\pgfpathlineto{\pgfqpoint{1.048274in}{0.307200in}}%
\pgfpathlineto{\pgfqpoint{1.047478in}{0.308525in}}%
\pgfpathlineto{\pgfqpoint{1.041442in}{0.318611in}}%
\pgfpathlineto{\pgfqpoint{1.035746in}{0.328038in}}%
\pgfpathlineto{\pgfqpoint{1.034903in}{0.330022in}}%
\pgfpathlineto{\pgfqpoint{1.029557in}{0.341432in}}%
\pgfpathlineto{\pgfqpoint{1.024013in}{0.351011in}}%
\pgfpathlineto{\pgfqpoint{1.023266in}{0.352843in}}%
\pgfpathlineto{\pgfqpoint{1.018771in}{0.364254in}}%
\pgfpathlineto{\pgfqpoint{1.014513in}{0.375665in}}%
\pgfpathlineto{\pgfqpoint{1.012281in}{0.381085in}}%
\pgfpathlineto{\pgfqpoint{1.009831in}{0.387075in}}%
\pgfpathlineto{\pgfqpoint{1.005314in}{0.398486in}}%
\pgfpathlineto{\pgfqpoint{1.000806in}{0.409897in}}%
\pgfpathlineto{\pgfqpoint{1.000549in}{0.410553in}}%
\pgfpathlineto{\pgfqpoint{0.996140in}{0.421308in}}%
\pgfpathlineto{\pgfqpoint{0.991554in}{0.432719in}}%
\pgfpathlineto{\pgfqpoint{0.988817in}{0.439502in}}%
\pgfpathlineto{\pgfqpoint{0.986854in}{0.444129in}}%
\pgfpathlineto{\pgfqpoint{0.982169in}{0.455540in}}%
\pgfpathlineto{\pgfqpoint{0.977459in}{0.466951in}}%
\pgfpathlineto{\pgfqpoint{0.977085in}{0.467862in}}%
\pgfpathlineto{\pgfqpoint{0.972663in}{0.478362in}}%
\pgfpathlineto{\pgfqpoint{0.967931in}{0.489772in}}%
\pgfpathlineto{\pgfqpoint{0.965352in}{0.495906in}}%
\pgfpathlineto{\pgfqpoint{0.963116in}{0.501183in}}%
\pgfpathlineto{\pgfqpoint{0.958400in}{0.512594in}}%
\pgfpathlineto{\pgfqpoint{0.953620in}{0.523900in}}%
\pgfpathlineto{\pgfqpoint{0.953577in}{0.524005in}}%
\pgfpathlineto{\pgfqpoint{0.948978in}{0.535416in}}%
\pgfpathlineto{\pgfqpoint{0.943326in}{0.546826in}}%
\pgfpathlineto{\pgfqpoint{0.941888in}{0.549336in}}%
\pgfpathlineto{\pgfqpoint{0.936989in}{0.558237in}}%
\pgfpathlineto{\pgfqpoint{0.930396in}{0.569648in}}%
\pgfpathlineto{\pgfqpoint{0.930156in}{0.570143in}}%
\pgfpathlineto{\pgfqpoint{0.926617in}{0.581059in}}%
\pgfpathlineto{\pgfqpoint{0.922962in}{0.592469in}}%
\pgfpathlineto{\pgfqpoint{0.919236in}{0.603880in}}%
\pgfpathlineto{\pgfqpoint{0.918424in}{0.605865in}}%
\pgfpathlineto{\pgfqpoint{0.915615in}{0.615291in}}%
\pgfpathlineto{\pgfqpoint{0.912129in}{0.626702in}}%
\pgfpathlineto{\pgfqpoint{0.908675in}{0.638113in}}%
\pgfpathlineto{\pgfqpoint{0.906691in}{0.643365in}}%
\pgfpathlineto{\pgfqpoint{0.904946in}{0.649523in}}%
\pgfpathlineto{\pgfqpoint{0.901648in}{0.660934in}}%
\pgfpathlineto{\pgfqpoint{0.898452in}{0.672345in}}%
\pgfpathlineto{\pgfqpoint{0.894959in}{0.682406in}}%
\pgfpathlineto{\pgfqpoint{0.894591in}{0.683756in}}%
\pgfpathlineto{\pgfqpoint{0.891570in}{0.695166in}}%
\pgfpathlineto{\pgfqpoint{0.888963in}{0.706577in}}%
\pgfpathlineto{\pgfqpoint{0.886719in}{0.717988in}}%
\pgfpathlineto{\pgfqpoint{0.883227in}{0.728769in}}%
\pgfpathlineto{\pgfqpoint{0.883090in}{0.729399in}}%
\pgfpathlineto{\pgfqpoint{0.880965in}{0.740810in}}%
\pgfpathlineto{\pgfqpoint{0.879983in}{0.752220in}}%
\pgfpathlineto{\pgfqpoint{0.879752in}{0.763631in}}%
\pgfpathlineto{\pgfqpoint{0.880978in}{0.775042in}}%
\pgfpathlineto{\pgfqpoint{0.882845in}{0.786453in}}%
\pgfpathlineto{\pgfqpoint{0.883227in}{0.788550in}}%
\pgfpathlineto{\pgfqpoint{0.885756in}{0.797863in}}%
\pgfpathlineto{\pgfqpoint{0.889800in}{0.809274in}}%
\pgfpathlineto{\pgfqpoint{0.894959in}{0.815964in}}%
\pgfpathlineto{\pgfqpoint{0.906691in}{0.815606in}}%
\pgfpathlineto{\pgfqpoint{0.918424in}{0.811973in}}%
\pgfpathlineto{\pgfqpoint{0.926892in}{0.809274in}}%
\pgfpathlineto{\pgfqpoint{0.930156in}{0.807906in}}%
\pgfpathlineto{\pgfqpoint{0.938239in}{0.797863in}}%
\pgfpathlineto{\pgfqpoint{0.941888in}{0.790608in}}%
\pgfpathlineto{\pgfqpoint{0.943606in}{0.786453in}}%
\pgfpathlineto{\pgfqpoint{0.948164in}{0.775042in}}%
\pgfpathlineto{\pgfqpoint{0.950853in}{0.763631in}}%
\pgfpathlineto{\pgfqpoint{0.953421in}{0.752220in}}%
\pgfpathlineto{\pgfqpoint{0.953620in}{0.751533in}}%
\pgfpathlineto{\pgfqpoint{0.956172in}{0.740810in}}%
\pgfpathlineto{\pgfqpoint{0.958769in}{0.729399in}}%
\pgfpathlineto{\pgfqpoint{0.961025in}{0.717988in}}%
\pgfpathlineto{\pgfqpoint{0.964089in}{0.706577in}}%
\pgfpathlineto{\pgfqpoint{0.965352in}{0.703047in}}%
\pgfpathlineto{\pgfqpoint{0.967425in}{0.695166in}}%
\pgfpathlineto{\pgfqpoint{0.970302in}{0.683756in}}%
\pgfpathlineto{\pgfqpoint{0.972976in}{0.672345in}}%
\pgfpathlineto{\pgfqpoint{0.976645in}{0.660934in}}%
\pgfpathlineto{\pgfqpoint{0.977085in}{0.659807in}}%
\pgfpathlineto{\pgfqpoint{0.979584in}{0.649523in}}%
\pgfpathlineto{\pgfqpoint{0.982368in}{0.638113in}}%
\pgfpathlineto{\pgfqpoint{0.985213in}{0.626702in}}%
\pgfpathlineto{\pgfqpoint{0.988817in}{0.616894in}}%
\pgfpathlineto{\pgfqpoint{0.989234in}{0.615291in}}%
\pgfpathlineto{\pgfqpoint{0.992191in}{0.603880in}}%
\pgfpathlineto{\pgfqpoint{0.996422in}{0.592469in}}%
\pgfpathlineto{\pgfqpoint{1.000549in}{0.584654in}}%
\pgfpathlineto{\pgfqpoint{1.002027in}{0.581059in}}%
\pgfpathlineto{\pgfqpoint{1.006771in}{0.569648in}}%
\pgfpathlineto{\pgfqpoint{1.012281in}{0.559155in}}%
\pgfpathlineto{\pgfqpoint{1.012671in}{0.558237in}}%
\pgfpathlineto{\pgfqpoint{1.017337in}{0.546826in}}%
\pgfpathlineto{\pgfqpoint{1.023105in}{0.535416in}}%
\pgfpathlineto{\pgfqpoint{1.024013in}{0.533775in}}%
\pgfpathlineto{\pgfqpoint{1.028082in}{0.524005in}}%
\pgfpathlineto{\pgfqpoint{1.033562in}{0.512594in}}%
\pgfpathlineto{\pgfqpoint{1.035746in}{0.508598in}}%
\pgfpathlineto{\pgfqpoint{1.038903in}{0.501183in}}%
\pgfpathlineto{\pgfqpoint{1.044156in}{0.489772in}}%
\pgfpathlineto{\pgfqpoint{1.047478in}{0.483623in}}%
\pgfpathlineto{\pgfqpoint{1.049771in}{0.478362in}}%
\pgfpathlineto{\pgfqpoint{1.054873in}{0.466951in}}%
\pgfpathlineto{\pgfqpoint{1.059210in}{0.458814in}}%
\pgfpathlineto{\pgfqpoint{1.060665in}{0.455540in}}%
\pgfpathlineto{\pgfqpoint{1.065659in}{0.444129in}}%
\pgfpathlineto{\pgfqpoint{1.070942in}{0.434022in}}%
\pgfpathlineto{\pgfqpoint{1.071525in}{0.432719in}}%
\pgfpathlineto{\pgfqpoint{1.076422in}{0.421308in}}%
\pgfpathlineto{\pgfqpoint{1.082178in}{0.409897in}}%
\pgfpathlineto{\pgfqpoint{1.082674in}{0.408977in}}%
\pgfpathlineto{\pgfqpoint{1.087190in}{0.398486in}}%
\pgfpathlineto{\pgfqpoint{1.092705in}{0.387075in}}%
\pgfpathlineto{\pgfqpoint{1.094407in}{0.383896in}}%
\pgfpathlineto{\pgfqpoint{1.098153in}{0.375665in}}%
\pgfpathlineto{\pgfqpoint{1.103331in}{0.364254in}}%
\pgfpathlineto{\pgfqpoint{1.106139in}{0.358954in}}%
\pgfpathlineto{\pgfqpoint{1.109906in}{0.352843in}}%
\pgfpathlineto{\pgfqpoint{1.115366in}{0.341432in}}%
\pgfpathlineto{\pgfqpoint{1.117871in}{0.335728in}}%
\pgfpathlineto{\pgfqpoint{1.121612in}{0.330022in}}%
\pgfpathlineto{\pgfqpoint{1.129146in}{0.318611in}}%
\pgfpathlineto{\pgfqpoint{1.129603in}{0.317775in}}%
\pgfpathlineto{\pgfqpoint{1.138782in}{0.307200in}}%
\pgfpathlineto{\pgfqpoint{1.141335in}{0.302545in}}%
\pgfpathlineto{\pgfqpoint{1.149164in}{0.307200in}}%
\pgfpathlineto{\pgfqpoint{1.153068in}{0.313790in}}%
\pgfpathlineto{\pgfqpoint{1.164800in}{0.309380in}}%
\pgfpathlineto{\pgfqpoint{1.168537in}{0.307200in}}%
\pgfpathlineto{\pgfqpoint{1.176532in}{0.302432in}}%
\pgfpathlineto{\pgfqpoint{1.187668in}{0.295789in}}%
\pgfpathlineto{\pgfqpoint{1.188264in}{0.295424in}}%
\pgfpathlineto{\pgfqpoint{1.199996in}{0.288810in}}%
\pgfpathlineto{\pgfqpoint{1.208148in}{0.284378in}}%
\pgfpathlineto{\pgfqpoint{1.211729in}{0.284378in}}%
\pgfpathlineto{\pgfqpoint{1.223461in}{0.284378in}}%
\pgfpathlineto{\pgfqpoint{1.235193in}{0.284378in}}%
\pgfpathlineto{\pgfqpoint{1.246925in}{0.284378in}}%
\pgfpathlineto{\pgfqpoint{1.258658in}{0.284378in}}%
\pgfpathlineto{\pgfqpoint{1.270390in}{0.284378in}}%
\pgfpathlineto{\pgfqpoint{1.282122in}{0.284378in}}%
\pgfpathlineto{\pgfqpoint{1.293854in}{0.284378in}}%
\pgfpathlineto{\pgfqpoint{1.305586in}{0.284378in}}%
\pgfpathlineto{\pgfqpoint{1.317319in}{0.284378in}}%
\pgfpathlineto{\pgfqpoint{1.329051in}{0.284378in}}%
\pgfpathlineto{\pgfqpoint{1.340783in}{0.284378in}}%
\pgfpathlineto{\pgfqpoint{1.352515in}{0.284378in}}%
\pgfpathlineto{\pgfqpoint{1.355086in}{0.284378in}}%
\pgfpathlineto{\pgfqpoint{1.352515in}{0.291119in}}%
\pgfpathlineto{\pgfqpoint{1.351277in}{0.295789in}}%
\pgfpathlineto{\pgfqpoint{1.348280in}{0.307200in}}%
\pgfpathlineto{\pgfqpoint{1.345337in}{0.318611in}}%
\pgfpathlineto{\pgfqpoint{1.342452in}{0.330022in}}%
\pgfpathlineto{\pgfqpoint{1.340783in}{0.336728in}}%
\pgfpathlineto{\pgfqpoint{1.338010in}{0.341432in}}%
\pgfpathlineto{\pgfqpoint{1.331321in}{0.352843in}}%
\pgfpathlineto{\pgfqpoint{1.329051in}{0.356738in}}%
\pgfpathlineto{\pgfqpoint{1.317319in}{0.356143in}}%
\pgfpathlineto{\pgfqpoint{1.307502in}{0.364254in}}%
\pgfpathlineto{\pgfqpoint{1.305586in}{0.365864in}}%
\pgfpathlineto{\pgfqpoint{1.294547in}{0.375665in}}%
\pgfpathlineto{\pgfqpoint{1.293854in}{0.376292in}}%
\pgfpathlineto{\pgfqpoint{1.282229in}{0.387075in}}%
\pgfpathlineto{\pgfqpoint{1.282122in}{0.387177in}}%
\pgfpathlineto{\pgfqpoint{1.270480in}{0.398486in}}%
\pgfpathlineto{\pgfqpoint{1.270390in}{0.398576in}}%
\pgfpathlineto{\pgfqpoint{1.259290in}{0.409897in}}%
\pgfpathlineto{\pgfqpoint{1.258658in}{0.410560in}}%
\pgfpathlineto{\pgfqpoint{1.248651in}{0.421308in}}%
\pgfpathlineto{\pgfqpoint{1.246925in}{0.423219in}}%
\pgfpathlineto{\pgfqpoint{1.238556in}{0.432719in}}%
\pgfpathlineto{\pgfqpoint{1.235193in}{0.436670in}}%
\pgfpathlineto{\pgfqpoint{1.228998in}{0.444129in}}%
\pgfpathlineto{\pgfqpoint{1.223461in}{0.451063in}}%
\pgfpathlineto{\pgfqpoint{1.219971in}{0.455540in}}%
\pgfpathlineto{\pgfqpoint{1.211729in}{0.466598in}}%
\pgfpathlineto{\pgfqpoint{1.211471in}{0.466951in}}%
\pgfpathlineto{\pgfqpoint{1.203487in}{0.478362in}}%
\pgfpathlineto{\pgfqpoint{1.199996in}{0.483632in}}%
\pgfpathlineto{\pgfqpoint{1.196016in}{0.489772in}}%
\pgfpathlineto{\pgfqpoint{1.189068in}{0.501183in}}%
\pgfpathlineto{\pgfqpoint{1.188264in}{0.502584in}}%
\pgfpathlineto{\pgfqpoint{1.182655in}{0.512594in}}%
\pgfpathlineto{\pgfqpoint{1.176532in}{0.523869in}}%
\pgfpathlineto{\pgfqpoint{1.164800in}{0.518806in}}%
\pgfpathlineto{\pgfqpoint{1.161016in}{0.512594in}}%
\pgfpathlineto{\pgfqpoint{1.158772in}{0.501183in}}%
\pgfpathlineto{\pgfqpoint{1.155441in}{0.489772in}}%
\pgfpathlineto{\pgfqpoint{1.153068in}{0.483626in}}%
\pgfpathlineto{\pgfqpoint{1.150903in}{0.478362in}}%
\pgfpathlineto{\pgfqpoint{1.145106in}{0.466951in}}%
\pgfpathlineto{\pgfqpoint{1.141335in}{0.460709in}}%
\pgfpathlineto{\pgfqpoint{1.139939in}{0.455540in}}%
\pgfpathlineto{\pgfqpoint{1.138010in}{0.444129in}}%
\pgfpathlineto{\pgfqpoint{1.138494in}{0.432719in}}%
\pgfpathlineto{\pgfqpoint{1.138261in}{0.421308in}}%
\pgfpathlineto{\pgfqpoint{1.137228in}{0.409897in}}%
\pgfpathlineto{\pgfqpoint{1.135319in}{0.398486in}}%
\pgfpathlineto{\pgfqpoint{1.132461in}{0.387075in}}%
\pgfpathlineto{\pgfqpoint{1.129603in}{0.378501in}}%
\pgfpathlineto{\pgfqpoint{1.128565in}{0.375665in}}%
\pgfpathlineto{\pgfqpoint{1.123520in}{0.364254in}}%
\pgfpathlineto{\pgfqpoint{1.117871in}{0.360439in}}%
\pgfpathlineto{\pgfqpoint{1.116255in}{0.364254in}}%
\pgfpathlineto{\pgfqpoint{1.106139in}{0.375281in}}%
\pgfpathlineto{\pgfqpoint{1.105954in}{0.375665in}}%
\pgfpathlineto{\pgfqpoint{1.101361in}{0.387075in}}%
\pgfpathlineto{\pgfqpoint{1.095454in}{0.398486in}}%
\pgfpathlineto{\pgfqpoint{1.094407in}{0.399442in}}%
\pgfpathlineto{\pgfqpoint{1.089389in}{0.409897in}}%
\pgfpathlineto{\pgfqpoint{1.084593in}{0.421308in}}%
\pgfpathlineto{\pgfqpoint{1.082674in}{0.424518in}}%
\pgfpathlineto{\pgfqpoint{1.078702in}{0.432719in}}%
\pgfpathlineto{\pgfqpoint{1.073537in}{0.444129in}}%
\pgfpathlineto{\pgfqpoint{1.070942in}{0.449670in}}%
\pgfpathlineto{\pgfqpoint{1.068046in}{0.455540in}}%
\pgfpathlineto{\pgfqpoint{1.062819in}{0.466951in}}%
\pgfpathlineto{\pgfqpoint{1.059210in}{0.474632in}}%
\pgfpathlineto{\pgfqpoint{1.057337in}{0.478362in}}%
\pgfpathlineto{\pgfqpoint{1.052040in}{0.489772in}}%
\pgfpathlineto{\pgfqpoint{1.047478in}{0.499665in}}%
\pgfpathlineto{\pgfqpoint{1.046704in}{0.501183in}}%
\pgfpathlineto{\pgfqpoint{1.041291in}{0.512594in}}%
\pgfpathlineto{\pgfqpoint{1.036143in}{0.524005in}}%
\pgfpathlineto{\pgfqpoint{1.035746in}{0.524842in}}%
\pgfpathlineto{\pgfqpoint{1.030596in}{0.535416in}}%
\pgfpathlineto{\pgfqpoint{1.025518in}{0.546826in}}%
\pgfpathlineto{\pgfqpoint{1.024013in}{0.550071in}}%
\pgfpathlineto{\pgfqpoint{1.020004in}{0.558237in}}%
\pgfpathlineto{\pgfqpoint{1.014990in}{0.569648in}}%
\pgfpathlineto{\pgfqpoint{1.012281in}{0.575718in}}%
\pgfpathlineto{\pgfqpoint{1.009646in}{0.581059in}}%
\pgfpathlineto{\pgfqpoint{1.004580in}{0.592469in}}%
\pgfpathlineto{\pgfqpoint{1.000549in}{0.601759in}}%
\pgfpathlineto{\pgfqpoint{0.999574in}{0.603880in}}%
\pgfpathlineto{\pgfqpoint{0.995620in}{0.615291in}}%
\pgfpathlineto{\pgfqpoint{0.992519in}{0.626702in}}%
\pgfpathlineto{\pgfqpoint{0.989402in}{0.638113in}}%
\pgfpathlineto{\pgfqpoint{0.988817in}{0.640206in}}%
\pgfpathlineto{\pgfqpoint{0.986119in}{0.649523in}}%
\pgfpathlineto{\pgfqpoint{0.983218in}{0.660934in}}%
\pgfpathlineto{\pgfqpoint{0.980395in}{0.672345in}}%
\pgfpathlineto{\pgfqpoint{0.977269in}{0.683756in}}%
\pgfpathlineto{\pgfqpoint{0.977085in}{0.684426in}}%
\pgfpathlineto{\pgfqpoint{0.974315in}{0.695166in}}%
\pgfpathlineto{\pgfqpoint{0.971593in}{0.706577in}}%
\pgfpathlineto{\pgfqpoint{0.968855in}{0.717988in}}%
\pgfpathlineto{\pgfqpoint{0.966022in}{0.729399in}}%
\pgfpathlineto{\pgfqpoint{0.965352in}{0.732297in}}%
\pgfpathlineto{\pgfqpoint{0.963680in}{0.740810in}}%
\pgfpathlineto{\pgfqpoint{0.961683in}{0.752220in}}%
\pgfpathlineto{\pgfqpoint{0.959832in}{0.763631in}}%
\pgfpathlineto{\pgfqpoint{0.959092in}{0.775042in}}%
\pgfpathlineto{\pgfqpoint{0.956427in}{0.786453in}}%
\pgfpathlineto{\pgfqpoint{0.953620in}{0.789133in}}%
\pgfpathlineto{\pgfqpoint{0.951127in}{0.797863in}}%
\pgfpathlineto{\pgfqpoint{0.953620in}{0.804819in}}%
\pgfpathlineto{\pgfqpoint{0.961867in}{0.797863in}}%
\pgfpathlineto{\pgfqpoint{0.965352in}{0.794415in}}%
\pgfpathlineto{\pgfqpoint{0.974744in}{0.786453in}}%
\pgfpathlineto{\pgfqpoint{0.977085in}{0.784904in}}%
\pgfpathlineto{\pgfqpoint{0.988817in}{0.777177in}}%
\pgfpathlineto{\pgfqpoint{1.000549in}{0.776346in}}%
\pgfpathlineto{\pgfqpoint{1.003039in}{0.775042in}}%
\pgfpathlineto{\pgfqpoint{1.012281in}{0.770140in}}%
\pgfpathlineto{\pgfqpoint{1.024013in}{0.765711in}}%
\pgfpathlineto{\pgfqpoint{1.035746in}{0.770478in}}%
\pgfpathlineto{\pgfqpoint{1.045996in}{0.775042in}}%
\pgfpathlineto{\pgfqpoint{1.047478in}{0.775673in}}%
\pgfpathlineto{\pgfqpoint{1.059210in}{0.780600in}}%
\pgfpathlineto{\pgfqpoint{1.070942in}{0.785446in}}%
\pgfpathlineto{\pgfqpoint{1.073410in}{0.786453in}}%
\pgfpathlineto{\pgfqpoint{1.082674in}{0.790083in}}%
\pgfpathlineto{\pgfqpoint{1.094407in}{0.795560in}}%
\pgfpathlineto{\pgfqpoint{1.096392in}{0.797863in}}%
\pgfpathlineto{\pgfqpoint{1.106139in}{0.808905in}}%
\pgfpathlineto{\pgfqpoint{1.106488in}{0.809274in}}%
\pgfpathlineto{\pgfqpoint{1.117871in}{0.820000in}}%
\pgfpathlineto{\pgfqpoint{1.118584in}{0.820685in}}%
\pgfpathlineto{\pgfqpoint{1.129603in}{0.830910in}}%
\pgfpathlineto{\pgfqpoint{1.130941in}{0.832096in}}%
\pgfpathlineto{\pgfqpoint{1.141335in}{0.837275in}}%
\pgfpathlineto{\pgfqpoint{1.153068in}{0.839150in}}%
\pgfpathlineto{\pgfqpoint{1.164800in}{0.839861in}}%
\pgfpathlineto{\pgfqpoint{1.176532in}{0.840554in}}%
\pgfpathlineto{\pgfqpoint{1.188264in}{0.839903in}}%
\pgfpathlineto{\pgfqpoint{1.199996in}{0.832863in}}%
\pgfpathlineto{\pgfqpoint{1.201218in}{0.832096in}}%
\pgfpathlineto{\pgfqpoint{1.211729in}{0.825447in}}%
\pgfpathlineto{\pgfqpoint{1.219145in}{0.820685in}}%
\pgfpathlineto{\pgfqpoint{1.223461in}{0.817875in}}%
\pgfpathlineto{\pgfqpoint{1.235193in}{0.809880in}}%
\pgfpathlineto{\pgfqpoint{1.236077in}{0.809274in}}%
\pgfpathlineto{\pgfqpoint{1.246925in}{0.801674in}}%
\pgfpathlineto{\pgfqpoint{1.258658in}{0.800799in}}%
\pgfpathlineto{\pgfqpoint{1.270390in}{0.800183in}}%
\pgfpathlineto{\pgfqpoint{1.273705in}{0.797863in}}%
\pgfpathlineto{\pgfqpoint{1.282122in}{0.793205in}}%
\pgfpathlineto{\pgfqpoint{1.291004in}{0.786453in}}%
\pgfpathlineto{\pgfqpoint{1.293854in}{0.784220in}}%
\pgfpathlineto{\pgfqpoint{1.305586in}{0.776645in}}%
\pgfpathlineto{\pgfqpoint{1.308487in}{0.775042in}}%
\pgfpathlineto{\pgfqpoint{1.317319in}{0.770022in}}%
\pgfpathlineto{\pgfqpoint{1.321669in}{0.763631in}}%
\pgfpathlineto{\pgfqpoint{1.329051in}{0.752655in}}%
\pgfpathlineto{\pgfqpoint{1.329841in}{0.752220in}}%
\pgfpathlineto{\pgfqpoint{1.340783in}{0.745285in}}%
\pgfpathlineto{\pgfqpoint{1.345425in}{0.752220in}}%
\pgfpathlineto{\pgfqpoint{1.352515in}{0.762429in}}%
\pgfpathlineto{\pgfqpoint{1.353116in}{0.763631in}}%
\pgfpathlineto{\pgfqpoint{1.354059in}{0.775042in}}%
\pgfpathlineto{\pgfqpoint{1.354761in}{0.786453in}}%
\pgfpathlineto{\pgfqpoint{1.356892in}{0.797863in}}%
\pgfpathlineto{\pgfqpoint{1.358924in}{0.809274in}}%
\pgfpathlineto{\pgfqpoint{1.360806in}{0.820685in}}%
\pgfpathlineto{\pgfqpoint{1.362496in}{0.832096in}}%
\pgfpathlineto{\pgfqpoint{1.363969in}{0.843507in}}%
\pgfpathlineto{\pgfqpoint{1.363688in}{0.854917in}}%
\pgfpathlineto{\pgfqpoint{1.359910in}{0.866328in}}%
\pgfpathlineto{\pgfqpoint{1.355486in}{0.877739in}}%
\pgfpathlineto{\pgfqpoint{1.352515in}{0.885462in}}%
\pgfpathlineto{\pgfqpoint{1.351496in}{0.889150in}}%
\pgfpathlineto{\pgfqpoint{1.348572in}{0.900560in}}%
\pgfpathlineto{\pgfqpoint{1.351098in}{0.911971in}}%
\pgfpathlineto{\pgfqpoint{1.352515in}{0.914641in}}%
\pgfpathlineto{\pgfqpoint{1.357984in}{0.923382in}}%
\pgfpathlineto{\pgfqpoint{1.364247in}{0.928982in}}%
\pgfpathlineto{\pgfqpoint{1.372795in}{0.934793in}}%
\pgfpathlineto{\pgfqpoint{1.375980in}{0.936437in}}%
\pgfpathlineto{\pgfqpoint{1.387712in}{0.939735in}}%
\pgfpathlineto{\pgfqpoint{1.399444in}{0.938353in}}%
\pgfpathlineto{\pgfqpoint{1.411176in}{0.935558in}}%
\pgfpathlineto{\pgfqpoint{1.413579in}{0.934793in}}%
\pgfpathlineto{\pgfqpoint{1.422908in}{0.930681in}}%
\pgfpathlineto{\pgfqpoint{1.434641in}{0.924559in}}%
\pgfpathlineto{\pgfqpoint{1.436636in}{0.923382in}}%
\pgfpathlineto{\pgfqpoint{1.446373in}{0.912627in}}%
\pgfpathlineto{\pgfqpoint{1.446373in}{0.923382in}}%
\pgfpathlineto{\pgfqpoint{1.446373in}{0.931743in}}%
\pgfpathlineto{\pgfqpoint{1.441077in}{0.934793in}}%
\pgfpathlineto{\pgfqpoint{1.434641in}{0.937316in}}%
\pgfpathlineto{\pgfqpoint{1.422908in}{0.941954in}}%
\pgfpathlineto{\pgfqpoint{1.411176in}{0.946058in}}%
\pgfpathlineto{\pgfqpoint{1.410445in}{0.946204in}}%
\pgfpathlineto{\pgfqpoint{1.399444in}{0.951876in}}%
\pgfpathlineto{\pgfqpoint{1.387712in}{0.957483in}}%
\pgfpathlineto{\pgfqpoint{1.377231in}{0.946204in}}%
\pgfpathlineto{\pgfqpoint{1.375980in}{0.945866in}}%
\pgfpathlineto{\pgfqpoint{1.364247in}{0.939427in}}%
\pgfpathlineto{\pgfqpoint{1.357521in}{0.934793in}}%
\pgfpathlineto{\pgfqpoint{1.352515in}{0.930284in}}%
\pgfpathlineto{\pgfqpoint{1.346272in}{0.923382in}}%
\pgfpathlineto{\pgfqpoint{1.340783in}{0.914134in}}%
\pgfpathlineto{\pgfqpoint{1.339509in}{0.911971in}}%
\pgfpathlineto{\pgfqpoint{1.337856in}{0.900560in}}%
\pgfpathlineto{\pgfqpoint{1.340181in}{0.889150in}}%
\pgfpathlineto{\pgfqpoint{1.340783in}{0.886748in}}%
\pgfpathlineto{\pgfqpoint{1.343218in}{0.877739in}}%
\pgfpathlineto{\pgfqpoint{1.346400in}{0.866328in}}%
\pgfpathlineto{\pgfqpoint{1.347764in}{0.854917in}}%
\pgfpathlineto{\pgfqpoint{1.346626in}{0.843507in}}%
\pgfpathlineto{\pgfqpoint{1.345372in}{0.832096in}}%
\pgfpathlineto{\pgfqpoint{1.343994in}{0.820685in}}%
\pgfpathlineto{\pgfqpoint{1.342504in}{0.809274in}}%
\pgfpathlineto{\pgfqpoint{1.340925in}{0.797863in}}%
\pgfpathlineto{\pgfqpoint{1.340783in}{0.796160in}}%
\pgfpathlineto{\pgfqpoint{1.329051in}{0.786855in}}%
\pgfpathlineto{\pgfqpoint{1.321349in}{0.797863in}}%
\pgfpathlineto{\pgfqpoint{1.317319in}{0.802798in}}%
\pgfpathlineto{\pgfqpoint{1.306501in}{0.809274in}}%
\pgfpathlineto{\pgfqpoint{1.305586in}{0.809764in}}%
\pgfpathlineto{\pgfqpoint{1.293854in}{0.815989in}}%
\pgfpathlineto{\pgfqpoint{1.285553in}{0.820685in}}%
\pgfpathlineto{\pgfqpoint{1.282122in}{0.822410in}}%
\pgfpathlineto{\pgfqpoint{1.270390in}{0.829194in}}%
\pgfpathlineto{\pgfqpoint{1.265532in}{0.832096in}}%
\pgfpathlineto{\pgfqpoint{1.258658in}{0.835750in}}%
\pgfpathlineto{\pgfqpoint{1.246925in}{0.841000in}}%
\pgfpathlineto{\pgfqpoint{1.243248in}{0.843507in}}%
\pgfpathlineto{\pgfqpoint{1.235193in}{0.849027in}}%
\pgfpathlineto{\pgfqpoint{1.226594in}{0.854917in}}%
\pgfpathlineto{\pgfqpoint{1.223461in}{0.856876in}}%
\pgfpathlineto{\pgfqpoint{1.211729in}{0.864819in}}%
\pgfpathlineto{\pgfqpoint{1.205601in}{0.866328in}}%
\pgfpathlineto{\pgfqpoint{1.211729in}{0.867029in}}%
\pgfpathlineto{\pgfqpoint{1.222286in}{0.877739in}}%
\pgfpathlineto{\pgfqpoint{1.223461in}{0.878305in}}%
\pgfpathlineto{\pgfqpoint{1.235193in}{0.888173in}}%
\pgfpathlineto{\pgfqpoint{1.235905in}{0.889150in}}%
\pgfpathlineto{\pgfqpoint{1.246069in}{0.900560in}}%
\pgfpathlineto{\pgfqpoint{1.246925in}{0.901561in}}%
\pgfpathlineto{\pgfqpoint{1.258658in}{0.909675in}}%
\pgfpathlineto{\pgfqpoint{1.263759in}{0.911971in}}%
\pgfpathlineto{\pgfqpoint{1.270390in}{0.915233in}}%
\pgfpathlineto{\pgfqpoint{1.282122in}{0.920866in}}%
\pgfpathlineto{\pgfqpoint{1.286173in}{0.923382in}}%
\pgfpathlineto{\pgfqpoint{1.288516in}{0.934793in}}%
\pgfpathlineto{\pgfqpoint{1.282661in}{0.946204in}}%
\pgfpathlineto{\pgfqpoint{1.282122in}{0.946634in}}%
\pgfpathlineto{\pgfqpoint{1.270390in}{0.952006in}}%
\pgfpathlineto{\pgfqpoint{1.258658in}{0.954683in}}%
\pgfpathlineto{\pgfqpoint{1.246925in}{0.956944in}}%
\pgfpathlineto{\pgfqpoint{1.242617in}{0.957614in}}%
\pgfpathlineto{\pgfqpoint{1.235193in}{0.960844in}}%
\pgfpathlineto{\pgfqpoint{1.223461in}{0.961525in}}%
\pgfpathlineto{\pgfqpoint{1.211729in}{0.961436in}}%
\pgfpathlineto{\pgfqpoint{1.199996in}{0.963586in}}%
\pgfpathlineto{\pgfqpoint{1.188264in}{0.965109in}}%
\pgfpathlineto{\pgfqpoint{1.176532in}{0.966797in}}%
\pgfpathlineto{\pgfqpoint{1.164800in}{0.968511in}}%
\pgfpathlineto{\pgfqpoint{1.161476in}{0.969025in}}%
\pgfpathlineto{\pgfqpoint{1.153068in}{0.971689in}}%
\pgfpathlineto{\pgfqpoint{1.141335in}{0.972773in}}%
\pgfpathlineto{\pgfqpoint{1.129603in}{0.972233in}}%
\pgfpathlineto{\pgfqpoint{1.117871in}{0.972292in}}%
\pgfpathlineto{\pgfqpoint{1.106139in}{0.973588in}}%
\pgfpathlineto{\pgfqpoint{1.094407in}{0.975966in}}%
\pgfpathlineto{\pgfqpoint{1.082674in}{0.977493in}}%
\pgfpathlineto{\pgfqpoint{1.070942in}{0.978714in}}%
\pgfpathlineto{\pgfqpoint{1.059210in}{0.979893in}}%
\pgfpathlineto{\pgfqpoint{1.053639in}{0.980436in}}%
\pgfpathlineto{\pgfqpoint{1.047478in}{0.981216in}}%
\pgfpathlineto{\pgfqpoint{1.035746in}{0.982162in}}%
\pgfpathlineto{\pgfqpoint{1.030404in}{0.980436in}}%
\pgfpathlineto{\pgfqpoint{1.024013in}{0.979141in}}%
\pgfpathlineto{\pgfqpoint{1.012281in}{0.976261in}}%
\pgfpathlineto{\pgfqpoint{1.000549in}{0.974109in}}%
\pgfpathlineto{\pgfqpoint{0.988817in}{0.970279in}}%
\pgfpathlineto{\pgfqpoint{0.987106in}{0.969025in}}%
\pgfpathlineto{\pgfqpoint{0.977085in}{0.965562in}}%
\pgfpathlineto{\pgfqpoint{0.965352in}{0.962418in}}%
\pgfpathlineto{\pgfqpoint{0.954165in}{0.957614in}}%
\pgfpathlineto{\pgfqpoint{0.953620in}{0.957349in}}%
\pgfpathlineto{\pgfqpoint{0.941888in}{0.950134in}}%
\pgfpathlineto{\pgfqpoint{0.937671in}{0.946204in}}%
\pgfpathlineto{\pgfqpoint{0.930187in}{0.934793in}}%
\pgfpathlineto{\pgfqpoint{0.930333in}{0.923382in}}%
\pgfpathlineto{\pgfqpoint{0.939795in}{0.911971in}}%
\pgfpathlineto{\pgfqpoint{0.941888in}{0.910097in}}%
\pgfpathlineto{\pgfqpoint{0.953620in}{0.905356in}}%
\pgfpathlineto{\pgfqpoint{0.965352in}{0.902146in}}%
\pgfpathlineto{\pgfqpoint{0.970858in}{0.900560in}}%
\pgfpathlineto{\pgfqpoint{0.977085in}{0.898944in}}%
\pgfpathlineto{\pgfqpoint{0.988817in}{0.892651in}}%
\pgfpathlineto{\pgfqpoint{0.993755in}{0.889150in}}%
\pgfpathlineto{\pgfqpoint{1.000549in}{0.885065in}}%
\pgfpathlineto{\pgfqpoint{1.011631in}{0.877739in}}%
\pgfpathlineto{\pgfqpoint{1.012281in}{0.877165in}}%
\pgfpathlineto{\pgfqpoint{1.024013in}{0.871562in}}%
\pgfpathlineto{\pgfqpoint{1.035746in}{0.868089in}}%
\pgfpathlineto{\pgfqpoint{1.043229in}{0.866328in}}%
\pgfpathlineto{\pgfqpoint{1.047478in}{0.862921in}}%
\pgfpathlineto{\pgfqpoint{1.059210in}{0.859118in}}%
\pgfpathlineto{\pgfqpoint{1.070942in}{0.856774in}}%
\pgfpathlineto{\pgfqpoint{1.082674in}{0.855118in}}%
\pgfpathlineto{\pgfqpoint{1.085970in}{0.854917in}}%
\pgfpathlineto{\pgfqpoint{1.082674in}{0.854249in}}%
\pgfpathlineto{\pgfqpoint{1.070942in}{0.849684in}}%
\pgfpathlineto{\pgfqpoint{1.059210in}{0.847199in}}%
\pgfpathlineto{\pgfqpoint{1.047478in}{0.847685in}}%
\pgfpathlineto{\pgfqpoint{1.035746in}{0.846257in}}%
\pgfpathlineto{\pgfqpoint{1.024013in}{0.844788in}}%
\pgfpathlineto{\pgfqpoint{1.014011in}{0.843507in}}%
\pgfpathlineto{\pgfqpoint{1.012281in}{0.843284in}}%
\pgfpathlineto{\pgfqpoint{1.000549in}{0.842660in}}%
\pgfpathlineto{\pgfqpoint{0.988817in}{0.842476in}}%
\pgfpathlineto{\pgfqpoint{0.977085in}{0.842484in}}%
\pgfpathlineto{\pgfqpoint{0.972989in}{0.832096in}}%
\pgfpathlineto{\pgfqpoint{0.965352in}{0.824064in}}%
\pgfpathlineto{\pgfqpoint{0.957385in}{0.832096in}}%
\pgfpathlineto{\pgfqpoint{0.953620in}{0.835100in}}%
\pgfpathlineto{\pgfqpoint{0.946573in}{0.843507in}}%
\pgfpathlineto{\pgfqpoint{0.941888in}{0.850384in}}%
\pgfpathlineto{\pgfqpoint{0.939171in}{0.854917in}}%
\pgfpathlineto{\pgfqpoint{0.932843in}{0.866328in}}%
\pgfpathlineto{\pgfqpoint{0.930156in}{0.871347in}}%
\pgfpathlineto{\pgfqpoint{0.926698in}{0.877739in}}%
\pgfpathlineto{\pgfqpoint{0.919924in}{0.889150in}}%
\pgfpathlineto{\pgfqpoint{0.918424in}{0.890820in}}%
\pgfpathlineto{\pgfqpoint{0.914178in}{0.900560in}}%
\pgfpathlineto{\pgfqpoint{0.906691in}{0.908262in}}%
\pgfpathlineto{\pgfqpoint{0.904475in}{0.911971in}}%
\pgfpathlineto{\pgfqpoint{0.894959in}{0.919402in}}%
\pgfpathlineto{\pgfqpoint{0.886572in}{0.923382in}}%
\pgfpathlineto{\pgfqpoint{0.883227in}{0.924346in}}%
\pgfpathlineto{\pgfqpoint{0.871495in}{0.928508in}}%
\pgfpathlineto{\pgfqpoint{0.863496in}{0.934793in}}%
\pgfpathlineto{\pgfqpoint{0.867059in}{0.946204in}}%
\pgfpathlineto{\pgfqpoint{0.871495in}{0.954450in}}%
\pgfpathlineto{\pgfqpoint{0.873040in}{0.957614in}}%
\pgfpathlineto{\pgfqpoint{0.880262in}{0.969025in}}%
\pgfpathlineto{\pgfqpoint{0.883227in}{0.972396in}}%
\pgfpathlineto{\pgfqpoint{0.888856in}{0.980436in}}%
\pgfpathlineto{\pgfqpoint{0.894959in}{0.987020in}}%
\pgfpathlineto{\pgfqpoint{0.906691in}{0.987231in}}%
\pgfpathlineto{\pgfqpoint{0.918424in}{0.991078in}}%
\pgfpathlineto{\pgfqpoint{0.920557in}{0.991847in}}%
\pgfpathlineto{\pgfqpoint{0.930156in}{0.995370in}}%
\pgfpathlineto{\pgfqpoint{0.941888in}{1.000173in}}%
\pgfpathlineto{\pgfqpoint{0.948764in}{1.003257in}}%
\pgfpathlineto{\pgfqpoint{0.953620in}{1.005464in}}%
\pgfpathlineto{\pgfqpoint{0.965352in}{1.009895in}}%
\pgfpathlineto{\pgfqpoint{0.973576in}{1.014668in}}%
\pgfpathlineto{\pgfqpoint{0.977085in}{1.016414in}}%
\pgfpathlineto{\pgfqpoint{0.988817in}{1.021554in}}%
\pgfpathlineto{\pgfqpoint{0.993430in}{1.026079in}}%
\pgfpathlineto{\pgfqpoint{1.000549in}{1.030728in}}%
\pgfpathlineto{\pgfqpoint{1.009979in}{1.037490in}}%
\pgfpathlineto{\pgfqpoint{1.012281in}{1.038974in}}%
\pgfpathlineto{\pgfqpoint{1.024013in}{1.046428in}}%
\pgfpathlineto{\pgfqpoint{1.027415in}{1.048901in}}%
\pgfpathlineto{\pgfqpoint{1.035746in}{1.053538in}}%
\pgfpathlineto{\pgfqpoint{1.046821in}{1.060311in}}%
\pgfpathlineto{\pgfqpoint{1.047478in}{1.062450in}}%
\pgfpathlineto{\pgfqpoint{1.054657in}{1.071722in}}%
\pgfpathlineto{\pgfqpoint{1.047478in}{1.074275in}}%
\pgfpathlineto{\pgfqpoint{1.035746in}{1.072999in}}%
\pgfpathlineto{\pgfqpoint{1.025176in}{1.071722in}}%
\pgfpathlineto{\pgfqpoint{1.024013in}{1.071560in}}%
\pgfpathlineto{\pgfqpoint{1.012281in}{1.069669in}}%
\pgfpathlineto{\pgfqpoint{1.000549in}{1.067639in}}%
\pgfpathlineto{\pgfqpoint{0.988817in}{1.066097in}}%
\pgfpathlineto{\pgfqpoint{0.977085in}{1.063351in}}%
\pgfpathlineto{\pgfqpoint{0.966225in}{1.060311in}}%
\pgfpathlineto{\pgfqpoint{0.965352in}{1.060119in}}%
\pgfpathlineto{\pgfqpoint{0.953620in}{1.057426in}}%
\pgfpathlineto{\pgfqpoint{0.941888in}{1.054290in}}%
\pgfpathlineto{\pgfqpoint{0.930156in}{1.050469in}}%
\pgfpathlineto{\pgfqpoint{0.926323in}{1.048901in}}%
\pgfpathlineto{\pgfqpoint{0.918424in}{1.045055in}}%
\pgfpathlineto{\pgfqpoint{0.909989in}{1.037490in}}%
\pgfpathlineto{\pgfqpoint{0.906691in}{1.032591in}}%
\pgfpathlineto{\pgfqpoint{0.902705in}{1.026079in}}%
\pgfpathlineto{\pgfqpoint{0.896073in}{1.014668in}}%
\pgfpathlineto{\pgfqpoint{0.894959in}{1.012627in}}%
\pgfpathlineto{\pgfqpoint{0.889694in}{1.003257in}}%
\pgfpathlineto{\pgfqpoint{0.883227in}{0.997766in}}%
\pgfpathlineto{\pgfqpoint{0.878740in}{1.003257in}}%
\pgfpathlineto{\pgfqpoint{0.871495in}{1.009130in}}%
\pgfpathlineto{\pgfqpoint{0.866080in}{1.014668in}}%
\pgfpathlineto{\pgfqpoint{0.859762in}{1.020891in}}%
\pgfpathlineto{\pgfqpoint{0.854718in}{1.026079in}}%
\pgfpathlineto{\pgfqpoint{0.848030in}{1.032746in}}%
\pgfpathlineto{\pgfqpoint{0.843382in}{1.037490in}}%
\pgfpathlineto{\pgfqpoint{0.836298in}{1.048172in}}%
\pgfpathlineto{\pgfqpoint{0.835842in}{1.048901in}}%
\pgfpathlineto{\pgfqpoint{0.830057in}{1.060311in}}%
\pgfpathlineto{\pgfqpoint{0.824847in}{1.071722in}}%
\pgfpathlineto{\pgfqpoint{0.824566in}{1.071979in}}%
\pgfpathlineto{\pgfqpoint{0.824311in}{1.071722in}}%
\pgfpathlineto{\pgfqpoint{0.816681in}{1.060311in}}%
\pgfpathlineto{\pgfqpoint{0.812834in}{1.058251in}}%
\pgfpathlineto{\pgfqpoint{0.805949in}{1.060311in}}%
\pgfpathlineto{\pgfqpoint{0.801101in}{1.061461in}}%
\pgfpathlineto{\pgfqpoint{0.789369in}{1.070520in}}%
\pgfpathlineto{\pgfqpoint{0.788187in}{1.071722in}}%
\pgfpathlineto{\pgfqpoint{0.778231in}{1.083133in}}%
\pgfpathlineto{\pgfqpoint{0.777637in}{1.084642in}}%
\pgfpathlineto{\pgfqpoint{0.773034in}{1.094544in}}%
\pgfpathlineto{\pgfqpoint{0.773367in}{1.105954in}}%
\pgfpathlineto{\pgfqpoint{0.768205in}{1.117365in}}%
\pgfpathlineto{\pgfqpoint{0.765905in}{1.120557in}}%
\pgfpathlineto{\pgfqpoint{0.757445in}{1.128776in}}%
\pgfpathlineto{\pgfqpoint{0.754173in}{1.131865in}}%
\pgfpathlineto{\pgfqpoint{0.744160in}{1.140187in}}%
\pgfpathlineto{\pgfqpoint{0.742440in}{1.141853in}}%
\pgfpathlineto{\pgfqpoint{0.733146in}{1.151598in}}%
\pgfpathlineto{\pgfqpoint{0.730708in}{1.154483in}}%
\pgfpathlineto{\pgfqpoint{0.721696in}{1.163008in}}%
\pgfpathlineto{\pgfqpoint{0.718976in}{1.166537in}}%
\pgfpathlineto{\pgfqpoint{0.711212in}{1.174419in}}%
\pgfpathlineto{\pgfqpoint{0.707244in}{1.180166in}}%
\pgfpathlineto{\pgfqpoint{0.702200in}{1.185830in}}%
\pgfpathlineto{\pgfqpoint{0.695856in}{1.197241in}}%
\pgfpathlineto{\pgfqpoint{0.695512in}{1.197722in}}%
\pgfpathlineto{\pgfqpoint{0.688323in}{1.208651in}}%
\pgfpathlineto{\pgfqpoint{0.685358in}{1.220062in}}%
\pgfpathlineto{\pgfqpoint{0.683779in}{1.225938in}}%
\pgfpathlineto{\pgfqpoint{0.682170in}{1.231473in}}%
\pgfpathlineto{\pgfqpoint{0.678714in}{1.242884in}}%
\pgfpathlineto{\pgfqpoint{0.675604in}{1.254295in}}%
\pgfpathlineto{\pgfqpoint{0.672884in}{1.265705in}}%
\pgfpathlineto{\pgfqpoint{0.672047in}{1.270216in}}%
\pgfpathlineto{\pgfqpoint{0.670738in}{1.277116in}}%
\pgfpathlineto{\pgfqpoint{0.672047in}{1.287935in}}%
\pgfpathlineto{\pgfqpoint{0.672579in}{1.288527in}}%
\pgfpathlineto{\pgfqpoint{0.683779in}{1.289909in}}%
\pgfpathlineto{\pgfqpoint{0.685394in}{1.288527in}}%
\pgfpathlineto{\pgfqpoint{0.695512in}{1.282799in}}%
\pgfpathlineto{\pgfqpoint{0.702021in}{1.277116in}}%
\pgfpathlineto{\pgfqpoint{0.707244in}{1.272890in}}%
\pgfpathlineto{\pgfqpoint{0.715582in}{1.265705in}}%
\pgfpathlineto{\pgfqpoint{0.718976in}{1.262838in}}%
\pgfpathlineto{\pgfqpoint{0.729041in}{1.254295in}}%
\pgfpathlineto{\pgfqpoint{0.730708in}{1.252543in}}%
\pgfpathlineto{\pgfqpoint{0.737316in}{1.242884in}}%
\pgfpathlineto{\pgfqpoint{0.742440in}{1.235459in}}%
\pgfpathlineto{\pgfqpoint{0.745070in}{1.231473in}}%
\pgfpathlineto{\pgfqpoint{0.752774in}{1.220062in}}%
\pgfpathlineto{\pgfqpoint{0.754173in}{1.218024in}}%
\pgfpathlineto{\pgfqpoint{0.760282in}{1.208651in}}%
\pgfpathlineto{\pgfqpoint{0.765905in}{1.200446in}}%
\pgfpathlineto{\pgfqpoint{0.767996in}{1.197241in}}%
\pgfpathlineto{\pgfqpoint{0.774136in}{1.185830in}}%
\pgfpathlineto{\pgfqpoint{0.777637in}{1.177818in}}%
\pgfpathlineto{\pgfqpoint{0.778931in}{1.174419in}}%
\pgfpathlineto{\pgfqpoint{0.781892in}{1.163008in}}%
\pgfpathlineto{\pgfqpoint{0.785382in}{1.151598in}}%
\pgfpathlineto{\pgfqpoint{0.789369in}{1.141918in}}%
\pgfpathlineto{\pgfqpoint{0.790037in}{1.140187in}}%
\pgfpathlineto{\pgfqpoint{0.797121in}{1.128776in}}%
\pgfpathlineto{\pgfqpoint{0.801101in}{1.125406in}}%
\pgfpathlineto{\pgfqpoint{0.812570in}{1.117365in}}%
\pgfpathlineto{\pgfqpoint{0.812834in}{1.117229in}}%
\pgfpathlineto{\pgfqpoint{0.814257in}{1.117365in}}%
\pgfpathlineto{\pgfqpoint{0.824566in}{1.118863in}}%
\pgfpathlineto{\pgfqpoint{0.836298in}{1.122590in}}%
\pgfpathlineto{\pgfqpoint{0.848030in}{1.121260in}}%
\pgfpathlineto{\pgfqpoint{0.859762in}{1.119825in}}%
\pgfpathlineto{\pgfqpoint{0.871495in}{1.118278in}}%
\pgfpathlineto{\pgfqpoint{0.877962in}{1.117365in}}%
\pgfpathlineto{\pgfqpoint{0.883227in}{1.116682in}}%
\pgfpathlineto{\pgfqpoint{0.894959in}{1.114933in}}%
\pgfpathlineto{\pgfqpoint{0.906691in}{1.112591in}}%
\pgfpathlineto{\pgfqpoint{0.918424in}{1.109058in}}%
\pgfpathlineto{\pgfqpoint{0.925720in}{1.105954in}}%
\pgfpathlineto{\pgfqpoint{0.930156in}{1.104632in}}%
\pgfpathlineto{\pgfqpoint{0.941888in}{1.101313in}}%
\pgfpathlineto{\pgfqpoint{0.953620in}{1.098305in}}%
\pgfpathlineto{\pgfqpoint{0.965352in}{1.095671in}}%
\pgfpathlineto{\pgfqpoint{0.970384in}{1.094544in}}%
\pgfpathlineto{\pgfqpoint{0.977085in}{1.092287in}}%
\pgfpathlineto{\pgfqpoint{0.988817in}{1.089206in}}%
\pgfpathlineto{\pgfqpoint{1.000549in}{1.088408in}}%
\pgfpathlineto{\pgfqpoint{1.012281in}{1.087920in}}%
\pgfpathlineto{\pgfqpoint{1.024013in}{1.087402in}}%
\pgfpathlineto{\pgfqpoint{1.035746in}{1.086736in}}%
\pgfpathlineto{\pgfqpoint{1.047478in}{1.086462in}}%
\pgfpathlineto{\pgfqpoint{1.059210in}{1.086171in}}%
\pgfpathlineto{\pgfqpoint{1.070942in}{1.088284in}}%
\pgfpathlineto{\pgfqpoint{1.078877in}{1.094544in}}%
\pgfpathlineto{\pgfqpoint{1.078460in}{1.105954in}}%
\pgfpathlineto{\pgfqpoint{1.070942in}{1.112208in}}%
\pgfpathlineto{\pgfqpoint{1.064224in}{1.117365in}}%
\pgfpathlineto{\pgfqpoint{1.059210in}{1.123413in}}%
\pgfpathlineto{\pgfqpoint{1.051953in}{1.128776in}}%
\pgfpathlineto{\pgfqpoint{1.047478in}{1.131639in}}%
\pgfpathlineto{\pgfqpoint{1.036339in}{1.140187in}}%
\pgfpathlineto{\pgfqpoint{1.035746in}{1.140590in}}%
\pgfpathlineto{\pgfqpoint{1.024013in}{1.149166in}}%
\pgfpathlineto{\pgfqpoint{1.020768in}{1.151598in}}%
\pgfpathlineto{\pgfqpoint{1.012281in}{1.158157in}}%
\pgfpathlineto{\pgfqpoint{1.002310in}{1.163008in}}%
\pgfpathlineto{\pgfqpoint{1.000549in}{1.164570in}}%
\pgfpathlineto{\pgfqpoint{0.988817in}{1.172260in}}%
\pgfpathlineto{\pgfqpoint{0.983509in}{1.174419in}}%
\pgfpathlineto{\pgfqpoint{0.977085in}{1.179927in}}%
\pgfpathlineto{\pgfqpoint{0.968286in}{1.185830in}}%
\pgfpathlineto{\pgfqpoint{0.965352in}{1.188019in}}%
\pgfpathlineto{\pgfqpoint{0.953620in}{1.194907in}}%
\pgfpathlineto{\pgfqpoint{0.949591in}{1.197241in}}%
\pgfpathlineto{\pgfqpoint{0.941888in}{1.201728in}}%
\pgfpathlineto{\pgfqpoint{0.930156in}{1.208547in}}%
\pgfpathlineto{\pgfqpoint{0.929976in}{1.208651in}}%
\pgfpathlineto{\pgfqpoint{0.918424in}{1.215488in}}%
\pgfpathlineto{\pgfqpoint{0.910655in}{1.220062in}}%
\pgfpathlineto{\pgfqpoint{0.906691in}{1.222453in}}%
\pgfpathlineto{\pgfqpoint{0.894959in}{1.229498in}}%
\pgfpathlineto{\pgfqpoint{0.891654in}{1.231473in}}%
\pgfpathlineto{\pgfqpoint{0.883227in}{1.236642in}}%
\pgfpathlineto{\pgfqpoint{0.872983in}{1.242884in}}%
\pgfpathlineto{\pgfqpoint{0.883227in}{1.250604in}}%
\pgfpathlineto{\pgfqpoint{0.886581in}{1.254295in}}%
\pgfpathlineto{\pgfqpoint{0.894959in}{1.260287in}}%
\pgfpathlineto{\pgfqpoint{0.902834in}{1.265705in}}%
\pgfpathlineto{\pgfqpoint{0.906691in}{1.269420in}}%
\pgfpathlineto{\pgfqpoint{0.915196in}{1.277116in}}%
\pgfpathlineto{\pgfqpoint{0.918424in}{1.278781in}}%
\pgfpathlineto{\pgfqpoint{0.930156in}{1.283383in}}%
\pgfpathlineto{\pgfqpoint{0.941888in}{1.288060in}}%
\pgfpathlineto{\pgfqpoint{0.943254in}{1.288527in}}%
\pgfpathlineto{\pgfqpoint{0.953620in}{1.293119in}}%
\pgfpathlineto{\pgfqpoint{0.965352in}{1.296753in}}%
\pgfpathlineto{\pgfqpoint{0.977085in}{1.295694in}}%
\pgfpathlineto{\pgfqpoint{0.988817in}{1.294278in}}%
\pgfpathlineto{\pgfqpoint{1.000549in}{1.293049in}}%
\pgfpathlineto{\pgfqpoint{1.012281in}{1.292017in}}%
\pgfpathlineto{\pgfqpoint{1.024013in}{1.291185in}}%
\pgfpathlineto{\pgfqpoint{1.035746in}{1.290102in}}%
\pgfpathlineto{\pgfqpoint{1.038778in}{1.288527in}}%
\pgfpathlineto{\pgfqpoint{1.044711in}{1.277116in}}%
\pgfpathlineto{\pgfqpoint{1.045530in}{1.265705in}}%
\pgfpathlineto{\pgfqpoint{1.045121in}{1.254295in}}%
\pgfpathlineto{\pgfqpoint{1.047457in}{1.242884in}}%
\pgfpathlineto{\pgfqpoint{1.047478in}{1.242825in}}%
\pgfpathlineto{\pgfqpoint{1.053270in}{1.231473in}}%
\pgfpathlineto{\pgfqpoint{1.059210in}{1.220729in}}%
\pgfpathlineto{\pgfqpoint{1.059559in}{1.220062in}}%
\pgfpathlineto{\pgfqpoint{1.070300in}{1.208651in}}%
\pgfpathlineto{\pgfqpoint{1.070942in}{1.208027in}}%
\pgfpathlineto{\pgfqpoint{1.081764in}{1.197241in}}%
\pgfpathlineto{\pgfqpoint{1.082674in}{1.196336in}}%
\pgfpathlineto{\pgfqpoint{1.093036in}{1.185830in}}%
\pgfpathlineto{\pgfqpoint{1.094407in}{1.184458in}}%
\pgfpathlineto{\pgfqpoint{1.103982in}{1.174419in}}%
\pgfpathlineto{\pgfqpoint{1.106139in}{1.172162in}}%
\pgfpathlineto{\pgfqpoint{1.114721in}{1.163008in}}%
\pgfpathlineto{\pgfqpoint{1.117871in}{1.159845in}}%
\pgfpathlineto{\pgfqpoint{1.126729in}{1.151598in}}%
\pgfpathlineto{\pgfqpoint{1.129603in}{1.148919in}}%
\pgfpathlineto{\pgfqpoint{1.141335in}{1.141101in}}%
\pgfpathlineto{\pgfqpoint{1.153068in}{1.140750in}}%
\pgfpathlineto{\pgfqpoint{1.159240in}{1.140187in}}%
\pgfpathlineto{\pgfqpoint{1.164800in}{1.138488in}}%
\pgfpathlineto{\pgfqpoint{1.176532in}{1.132737in}}%
\pgfpathlineto{\pgfqpoint{1.188264in}{1.131778in}}%
\pgfpathlineto{\pgfqpoint{1.199996in}{1.132363in}}%
\pgfpathlineto{\pgfqpoint{1.211729in}{1.137179in}}%
\pgfpathlineto{\pgfqpoint{1.214358in}{1.140187in}}%
\pgfpathlineto{\pgfqpoint{1.223461in}{1.150279in}}%
\pgfpathlineto{\pgfqpoint{1.224579in}{1.151598in}}%
\pgfpathlineto{\pgfqpoint{1.233800in}{1.163008in}}%
\pgfpathlineto{\pgfqpoint{1.234081in}{1.174419in}}%
\pgfpathlineto{\pgfqpoint{1.231326in}{1.185830in}}%
\pgfpathlineto{\pgfqpoint{1.226265in}{1.197241in}}%
\pgfpathlineto{\pgfqpoint{1.223461in}{1.203197in}}%
\pgfpathlineto{\pgfqpoint{1.221007in}{1.208651in}}%
\pgfpathlineto{\pgfqpoint{1.216960in}{1.220062in}}%
\pgfpathlineto{\pgfqpoint{1.211729in}{1.226037in}}%
\pgfpathlineto{\pgfqpoint{1.203235in}{1.231473in}}%
\pgfpathlineto{\pgfqpoint{1.199996in}{1.233617in}}%
\pgfpathlineto{\pgfqpoint{1.188264in}{1.241375in}}%
\pgfpathlineto{\pgfqpoint{1.185940in}{1.242884in}}%
\pgfpathlineto{\pgfqpoint{1.176532in}{1.249554in}}%
\pgfpathlineto{\pgfqpoint{1.167676in}{1.254295in}}%
\pgfpathlineto{\pgfqpoint{1.164800in}{1.258695in}}%
\pgfpathlineto{\pgfqpoint{1.157458in}{1.265705in}}%
\pgfpathlineto{\pgfqpoint{1.153068in}{1.273822in}}%
\pgfpathlineto{\pgfqpoint{1.151376in}{1.277116in}}%
\pgfpathlineto{\pgfqpoint{1.144141in}{1.288527in}}%
\pgfpathlineto{\pgfqpoint{1.141335in}{1.292280in}}%
\pgfpathlineto{\pgfqpoint{1.135807in}{1.299938in}}%
\pgfpathlineto{\pgfqpoint{1.129603in}{1.307093in}}%
\pgfpathlineto{\pgfqpoint{1.121558in}{1.311348in}}%
\pgfpathlineto{\pgfqpoint{1.117871in}{1.313167in}}%
\pgfpathlineto{\pgfqpoint{1.106139in}{1.319123in}}%
\pgfpathlineto{\pgfqpoint{1.099237in}{1.322759in}}%
\pgfpathlineto{\pgfqpoint{1.094407in}{1.325137in}}%
\pgfpathlineto{\pgfqpoint{1.083457in}{1.334170in}}%
\pgfpathlineto{\pgfqpoint{1.087595in}{1.345581in}}%
\pgfpathlineto{\pgfqpoint{1.090075in}{1.356992in}}%
\pgfpathlineto{\pgfqpoint{1.091381in}{1.368402in}}%
\pgfpathlineto{\pgfqpoint{1.091356in}{1.379813in}}%
\pgfpathlineto{\pgfqpoint{1.089555in}{1.391224in}}%
\pgfpathlineto{\pgfqpoint{1.083734in}{1.402635in}}%
\pgfpathlineto{\pgfqpoint{1.082674in}{1.404050in}}%
\pgfpathlineto{\pgfqpoint{1.073819in}{1.414045in}}%
\pgfpathlineto{\pgfqpoint{1.070942in}{1.414045in}}%
\pgfpathlineto{\pgfqpoint{1.059210in}{1.414045in}}%
\pgfpathlineto{\pgfqpoint{1.047478in}{1.414045in}}%
\pgfpathlineto{\pgfqpoint{1.035746in}{1.414045in}}%
\pgfpathlineto{\pgfqpoint{1.024013in}{1.414045in}}%
\pgfpathlineto{\pgfqpoint{1.012281in}{1.414045in}}%
\pgfpathlineto{\pgfqpoint{1.000549in}{1.414045in}}%
\pgfpathlineto{\pgfqpoint{0.988817in}{1.414045in}}%
\pgfpathlineto{\pgfqpoint{0.977684in}{1.414045in}}%
\pgfpathlineto{\pgfqpoint{0.988817in}{1.406901in}}%
\pgfpathlineto{\pgfqpoint{0.995130in}{1.402635in}}%
\pgfpathlineto{\pgfqpoint{1.000549in}{1.398848in}}%
\pgfpathlineto{\pgfqpoint{1.010865in}{1.391224in}}%
\pgfpathlineto{\pgfqpoint{1.012281in}{1.390112in}}%
\pgfpathlineto{\pgfqpoint{1.024013in}{1.380315in}}%
\pgfpathlineto{\pgfqpoint{1.024573in}{1.379813in}}%
\pgfpathlineto{\pgfqpoint{1.035746in}{1.368492in}}%
\pgfpathlineto{\pgfqpoint{1.035826in}{1.368402in}}%
\pgfpathlineto{\pgfqpoint{1.043801in}{1.356992in}}%
\pgfpathlineto{\pgfqpoint{1.046410in}{1.345581in}}%
\pgfpathlineto{\pgfqpoint{1.045420in}{1.334170in}}%
\pgfpathlineto{\pgfqpoint{1.041095in}{1.322759in}}%
\pgfpathlineto{\pgfqpoint{1.035746in}{1.314765in}}%
\pgfpathlineto{\pgfqpoint{1.032930in}{1.311348in}}%
\pgfpathlineto{\pgfqpoint{1.024013in}{1.308131in}}%
\pgfpathlineto{\pgfqpoint{1.012281in}{1.308584in}}%
\pgfpathlineto{\pgfqpoint{1.000549in}{1.309910in}}%
\pgfpathlineto{\pgfqpoint{0.989456in}{1.311348in}}%
\pgfpathlineto{\pgfqpoint{0.988817in}{1.311464in}}%
\pgfpathlineto{\pgfqpoint{0.977085in}{1.313820in}}%
\pgfpathlineto{\pgfqpoint{0.965352in}{1.316320in}}%
\pgfpathlineto{\pgfqpoint{0.953620in}{1.318926in}}%
\pgfpathlineto{\pgfqpoint{0.941888in}{1.317632in}}%
\pgfpathlineto{\pgfqpoint{0.931880in}{1.311348in}}%
\pgfpathlineto{\pgfqpoint{0.930156in}{1.310407in}}%
\pgfpathlineto{\pgfqpoint{0.919002in}{1.299938in}}%
\pgfpathlineto{\pgfqpoint{0.918424in}{1.299601in}}%
\pgfpathlineto{\pgfqpoint{0.906691in}{1.292894in}}%
\pgfpathlineto{\pgfqpoint{0.898858in}{1.288527in}}%
\pgfpathlineto{\pgfqpoint{0.894959in}{1.286388in}}%
\pgfpathlineto{\pgfqpoint{0.883227in}{1.277326in}}%
\pgfpathlineto{\pgfqpoint{0.883014in}{1.277116in}}%
\pgfpathlineto{\pgfqpoint{0.871495in}{1.266434in}}%
\pgfpathlineto{\pgfqpoint{0.870703in}{1.265705in}}%
\pgfpathlineto{\pgfqpoint{0.859762in}{1.257970in}}%
\pgfpathlineto{\pgfqpoint{0.854713in}{1.254295in}}%
\pgfpathlineto{\pgfqpoint{0.848030in}{1.250139in}}%
\pgfpathlineto{\pgfqpoint{0.836298in}{1.245815in}}%
\pgfpathlineto{\pgfqpoint{0.824566in}{1.244523in}}%
\pgfpathlineto{\pgfqpoint{0.821571in}{1.242884in}}%
\pgfpathlineto{\pgfqpoint{0.824566in}{1.241497in}}%
\pgfpathlineto{\pgfqpoint{0.836298in}{1.239153in}}%
\pgfpathlineto{\pgfqpoint{0.846521in}{1.231473in}}%
\pgfpathlineto{\pgfqpoint{0.848030in}{1.230855in}}%
\pgfpathlineto{\pgfqpoint{0.859762in}{1.227215in}}%
\pgfpathlineto{\pgfqpoint{0.869313in}{1.220062in}}%
\pgfpathlineto{\pgfqpoint{0.871495in}{1.219130in}}%
\pgfpathlineto{\pgfqpoint{0.883227in}{1.215334in}}%
\pgfpathlineto{\pgfqpoint{0.892418in}{1.208651in}}%
\pgfpathlineto{\pgfqpoint{0.894959in}{1.207469in}}%
\pgfpathlineto{\pgfqpoint{0.906691in}{1.203620in}}%
\pgfpathlineto{\pgfqpoint{0.916925in}{1.197241in}}%
\pgfpathlineto{\pgfqpoint{0.918424in}{1.196542in}}%
\pgfpathlineto{\pgfqpoint{0.930156in}{1.191993in}}%
\pgfpathlineto{\pgfqpoint{0.941429in}{1.185830in}}%
\pgfpathlineto{\pgfqpoint{0.941888in}{1.185614in}}%
\pgfpathlineto{\pgfqpoint{0.953620in}{1.180389in}}%
\pgfpathlineto{\pgfqpoint{0.965352in}{1.174444in}}%
\pgfpathlineto{\pgfqpoint{0.965384in}{1.174419in}}%
\pgfpathlineto{\pgfqpoint{0.977085in}{1.168015in}}%
\pgfpathlineto{\pgfqpoint{0.986337in}{1.163008in}}%
\pgfpathlineto{\pgfqpoint{0.988817in}{1.161612in}}%
\pgfpathlineto{\pgfqpoint{1.000549in}{1.155099in}}%
\pgfpathlineto{\pgfqpoint{1.006878in}{1.151598in}}%
\pgfpathlineto{\pgfqpoint{1.012281in}{1.148193in}}%
\pgfpathlineto{\pgfqpoint{1.023008in}{1.140187in}}%
\pgfpathlineto{\pgfqpoint{1.024013in}{1.139264in}}%
\pgfpathlineto{\pgfqpoint{1.035746in}{1.129451in}}%
\pgfpathlineto{\pgfqpoint{1.036632in}{1.128776in}}%
\pgfpathlineto{\pgfqpoint{1.047478in}{1.121505in}}%
\pgfpathlineto{\pgfqpoint{1.052724in}{1.117365in}}%
\pgfpathlineto{\pgfqpoint{1.059210in}{1.112303in}}%
\pgfpathlineto{\pgfqpoint{1.068000in}{1.105954in}}%
\pgfpathlineto{\pgfqpoint{1.060921in}{1.094544in}}%
\pgfpathlineto{\pgfqpoint{1.059210in}{1.093997in}}%
\pgfpathlineto{\pgfqpoint{1.047478in}{1.093249in}}%
\pgfpathlineto{\pgfqpoint{1.035746in}{1.092794in}}%
\pgfpathlineto{\pgfqpoint{1.024013in}{1.093494in}}%
\pgfpathlineto{\pgfqpoint{1.012281in}{1.094138in}}%
\pgfpathlineto{\pgfqpoint{1.005220in}{1.094544in}}%
\pgfpathlineto{\pgfqpoint{1.000549in}{1.094776in}}%
\pgfpathlineto{\pgfqpoint{0.988817in}{1.095961in}}%
\pgfpathlineto{\pgfqpoint{0.977085in}{1.098609in}}%
\pgfpathlineto{\pgfqpoint{0.965352in}{1.101235in}}%
\pgfpathlineto{\pgfqpoint{0.953620in}{1.103927in}}%
\pgfpathlineto{\pgfqpoint{0.945508in}{1.105954in}}%
\pgfpathlineto{\pgfqpoint{0.941888in}{1.107531in}}%
\pgfpathlineto{\pgfqpoint{0.930156in}{1.112662in}}%
\pgfpathlineto{\pgfqpoint{0.918424in}{1.117189in}}%
\pgfpathlineto{\pgfqpoint{0.917586in}{1.117365in}}%
\pgfpathlineto{\pgfqpoint{0.906691in}{1.119182in}}%
\pgfpathlineto{\pgfqpoint{0.894959in}{1.120729in}}%
\pgfpathlineto{\pgfqpoint{0.883227in}{1.122175in}}%
\pgfpathlineto{\pgfqpoint{0.871495in}{1.123533in}}%
\pgfpathlineto{\pgfqpoint{0.859762in}{1.124789in}}%
\pgfpathlineto{\pgfqpoint{0.848030in}{1.125974in}}%
\pgfpathlineto{\pgfqpoint{0.836298in}{1.127077in}}%
\pgfpathlineto{\pgfqpoint{0.824566in}{1.125589in}}%
\pgfpathlineto{\pgfqpoint{0.812834in}{1.121979in}}%
\pgfpathlineto{\pgfqpoint{0.803864in}{1.128776in}}%
\pgfpathlineto{\pgfqpoint{0.801101in}{1.132621in}}%
\pgfpathlineto{\pgfqpoint{0.795969in}{1.140187in}}%
\pgfpathlineto{\pgfqpoint{0.791343in}{1.151598in}}%
\pgfpathlineto{\pgfqpoint{0.789369in}{1.157854in}}%
\pgfpathlineto{\pgfqpoint{0.787622in}{1.163008in}}%
\pgfpathlineto{\pgfqpoint{0.784244in}{1.174419in}}%
\pgfpathlineto{\pgfqpoint{0.779719in}{1.185830in}}%
\pgfpathlineto{\pgfqpoint{0.777637in}{1.190673in}}%
\pgfpathlineto{\pgfqpoint{0.774106in}{1.197241in}}%
\pgfpathlineto{\pgfqpoint{0.766620in}{1.208651in}}%
\pgfpathlineto{\pgfqpoint{0.765905in}{1.209711in}}%
\pgfpathlineto{\pgfqpoint{0.758808in}{1.220062in}}%
\pgfpathlineto{\pgfqpoint{0.754173in}{1.227005in}}%
\pgfpathlineto{\pgfqpoint{0.751097in}{1.231473in}}%
\pgfpathlineto{\pgfqpoint{0.743549in}{1.242884in}}%
\pgfpathlineto{\pgfqpoint{0.742440in}{1.244502in}}%
\pgfpathlineto{\pgfqpoint{0.735599in}{1.254295in}}%
\pgfpathlineto{\pgfqpoint{0.730708in}{1.259720in}}%
\pgfpathlineto{\pgfqpoint{0.723758in}{1.265705in}}%
\pgfpathlineto{\pgfqpoint{0.718976in}{1.269870in}}%
\pgfpathlineto{\pgfqpoint{0.710682in}{1.277116in}}%
\pgfpathlineto{\pgfqpoint{0.707244in}{1.280164in}}%
\pgfpathlineto{\pgfqpoint{0.697838in}{1.288527in}}%
\pgfpathlineto{\pgfqpoint{0.695512in}{1.290632in}}%
\pgfpathlineto{\pgfqpoint{0.685251in}{1.299938in}}%
\pgfpathlineto{\pgfqpoint{0.683779in}{1.301300in}}%
\pgfpathlineto{\pgfqpoint{0.672047in}{1.309717in}}%
\pgfpathlineto{\pgfqpoint{0.665340in}{1.311348in}}%
\pgfpathlineto{\pgfqpoint{0.660315in}{1.312646in}}%
\pgfpathlineto{\pgfqpoint{0.657140in}{1.311348in}}%
\pgfpathlineto{\pgfqpoint{0.652838in}{1.299938in}}%
\pgfpathlineto{\pgfqpoint{0.654615in}{1.288527in}}%
\pgfpathlineto{\pgfqpoint{0.657216in}{1.277116in}}%
\pgfpathlineto{\pgfqpoint{0.660315in}{1.265780in}}%
\pgfpathlineto{\pgfqpoint{0.660334in}{1.265705in}}%
\pgfpathlineto{\pgfqpoint{0.663549in}{1.254295in}}%
\pgfpathlineto{\pgfqpoint{0.666964in}{1.242884in}}%
\pgfpathlineto{\pgfqpoint{0.669752in}{1.231473in}}%
\pgfpathlineto{\pgfqpoint{0.670382in}{1.220062in}}%
\pgfpathlineto{\pgfqpoint{0.668250in}{1.208651in}}%
\pgfpathlineto{\pgfqpoint{0.667567in}{1.197241in}}%
\pgfpathlineto{\pgfqpoint{0.669477in}{1.185830in}}%
\pgfpathlineto{\pgfqpoint{0.672047in}{1.180259in}}%
\pgfpathlineto{\pgfqpoint{0.677033in}{1.174419in}}%
\pgfpathlineto{\pgfqpoint{0.683779in}{1.168122in}}%
\pgfpathlineto{\pgfqpoint{0.688917in}{1.163008in}}%
\pgfpathlineto{\pgfqpoint{0.695512in}{1.157664in}}%
\pgfpathlineto{\pgfqpoint{0.702014in}{1.151598in}}%
\pgfpathlineto{\pgfqpoint{0.707244in}{1.147495in}}%
\pgfpathlineto{\pgfqpoint{0.716544in}{1.140187in}}%
\pgfpathlineto{\pgfqpoint{0.718976in}{1.138369in}}%
\pgfpathlineto{\pgfqpoint{0.730708in}{1.129605in}}%
\pgfpathlineto{\pgfqpoint{0.732307in}{1.128776in}}%
\pgfpathlineto{\pgfqpoint{0.742440in}{1.124230in}}%
\pgfpathlineto{\pgfqpoint{0.754173in}{1.119298in}}%
\pgfpathlineto{\pgfqpoint{0.756162in}{1.117365in}}%
\pgfpathlineto{\pgfqpoint{0.765905in}{1.106781in}}%
\pgfpathlineto{\pgfqpoint{0.766302in}{1.105954in}}%
\pgfpathlineto{\pgfqpoint{0.765905in}{1.103043in}}%
\pgfpathlineto{\pgfqpoint{0.763468in}{1.094544in}}%
\pgfpathlineto{\pgfqpoint{0.765905in}{1.087430in}}%
\pgfpathlineto{\pgfqpoint{0.767048in}{1.083133in}}%
\pgfpathlineto{\pgfqpoint{0.776963in}{1.071722in}}%
\pgfpathlineto{\pgfqpoint{0.777637in}{1.070996in}}%
\pgfpathlineto{\pgfqpoint{0.788021in}{1.060311in}}%
\pgfpathlineto{\pgfqpoint{0.789369in}{1.059020in}}%
\pgfpathlineto{\pgfqpoint{0.801101in}{1.050831in}}%
\pgfpathlineto{\pgfqpoint{0.812834in}{1.049027in}}%
\pgfpathlineto{\pgfqpoint{0.815207in}{1.048901in}}%
\pgfpathlineto{\pgfqpoint{0.824566in}{1.048010in}}%
\pgfpathlineto{\pgfqpoint{0.831520in}{1.037490in}}%
\pgfpathlineto{\pgfqpoint{0.836298in}{1.031849in}}%
\pgfpathlineto{\pgfqpoint{0.841954in}{1.026079in}}%
\pgfpathlineto{\pgfqpoint{0.848030in}{1.018979in}}%
\pgfpathlineto{\pgfqpoint{0.852198in}{1.014668in}}%
\pgfpathlineto{\pgfqpoint{0.859762in}{1.004598in}}%
\pgfpathlineto{\pgfqpoint{0.861050in}{1.003257in}}%
\pgfpathlineto{\pgfqpoint{0.868831in}{0.991847in}}%
\pgfpathlineto{\pgfqpoint{0.871495in}{0.988142in}}%
\pgfpathlineto{\pgfqpoint{0.876211in}{0.980436in}}%
\pgfpathlineto{\pgfqpoint{0.871495in}{0.974705in}}%
\pgfpathlineto{\pgfqpoint{0.867900in}{0.969025in}}%
\pgfpathlineto{\pgfqpoint{0.863675in}{0.957614in}}%
\pgfpathlineto{\pgfqpoint{0.860490in}{0.946204in}}%
\pgfpathlineto{\pgfqpoint{0.859762in}{0.943814in}}%
\pgfpathlineto{\pgfqpoint{0.855002in}{0.934793in}}%
\pgfpathlineto{\pgfqpoint{0.859762in}{0.930805in}}%
\pgfpathlineto{\pgfqpoint{0.871495in}{0.923467in}}%
\pgfpathlineto{\pgfqpoint{0.871740in}{0.923382in}}%
\pgfpathlineto{\pgfqpoint{0.883227in}{0.918978in}}%
\pgfpathlineto{\pgfqpoint{0.894724in}{0.911971in}}%
\pgfpathlineto{\pgfqpoint{0.894959in}{0.911667in}}%
\pgfpathlineto{\pgfqpoint{0.902466in}{0.900560in}}%
\pgfpathlineto{\pgfqpoint{0.906691in}{0.891619in}}%
\pgfpathlineto{\pgfqpoint{0.908249in}{0.889150in}}%
\pgfpathlineto{\pgfqpoint{0.913455in}{0.877739in}}%
\pgfpathlineto{\pgfqpoint{0.916867in}{0.866328in}}%
\pgfpathlineto{\pgfqpoint{0.918424in}{0.861480in}}%
\pgfpathlineto{\pgfqpoint{0.922386in}{0.854917in}}%
\pgfpathlineto{\pgfqpoint{0.926927in}{0.843507in}}%
\pgfpathlineto{\pgfqpoint{0.928860in}{0.832096in}}%
\pgfpathlineto{\pgfqpoint{0.918424in}{0.825100in}}%
\pgfpathlineto{\pgfqpoint{0.906691in}{0.827420in}}%
\pgfpathlineto{\pgfqpoint{0.896262in}{0.832096in}}%
\pgfpathlineto{\pgfqpoint{0.894959in}{0.832696in}}%
\pgfpathlineto{\pgfqpoint{0.883227in}{0.832414in}}%
\pgfpathlineto{\pgfqpoint{0.882919in}{0.832096in}}%
\pgfpathlineto{\pgfqpoint{0.877553in}{0.820685in}}%
\pgfpathlineto{\pgfqpoint{0.874510in}{0.809274in}}%
\pgfpathlineto{\pgfqpoint{0.872024in}{0.797863in}}%
\pgfpathlineto{\pgfqpoint{0.871495in}{0.791479in}}%
\pgfpathlineto{\pgfqpoint{0.871288in}{0.786453in}}%
\pgfpathlineto{\pgfqpoint{0.870823in}{0.775042in}}%
\pgfpathlineto{\pgfqpoint{0.870423in}{0.763631in}}%
\pgfpathlineto{\pgfqpoint{0.870724in}{0.752220in}}%
\pgfpathlineto{\pgfqpoint{0.871495in}{0.742874in}}%
\pgfpathlineto{\pgfqpoint{0.871710in}{0.740810in}}%
\pgfpathlineto{\pgfqpoint{0.873714in}{0.729399in}}%
\pgfpathlineto{\pgfqpoint{0.876070in}{0.717988in}}%
\pgfpathlineto{\pgfqpoint{0.878647in}{0.706577in}}%
\pgfpathlineto{\pgfqpoint{0.881221in}{0.695166in}}%
\pgfpathlineto{\pgfqpoint{0.883227in}{0.686426in}}%
\pgfpathlineto{\pgfqpoint{0.883905in}{0.683756in}}%
\pgfpathlineto{\pgfqpoint{0.887107in}{0.672345in}}%
\pgfpathlineto{\pgfqpoint{0.890230in}{0.660934in}}%
\pgfpathlineto{\pgfqpoint{0.893325in}{0.649523in}}%
\pgfpathlineto{\pgfqpoint{0.894959in}{0.643530in}}%
\pgfpathlineto{\pgfqpoint{0.896575in}{0.638113in}}%
\pgfpathlineto{\pgfqpoint{0.899885in}{0.626702in}}%
\pgfpathlineto{\pgfqpoint{0.903112in}{0.615291in}}%
\pgfpathlineto{\pgfqpoint{0.906295in}{0.603880in}}%
\pgfpathlineto{\pgfqpoint{0.906691in}{0.602454in}}%
\pgfpathlineto{\pgfqpoint{0.909767in}{0.592469in}}%
\pgfpathlineto{\pgfqpoint{0.913184in}{0.581059in}}%
\pgfpathlineto{\pgfqpoint{0.916513in}{0.569648in}}%
\pgfpathlineto{\pgfqpoint{0.918424in}{0.562989in}}%
\pgfpathlineto{\pgfqpoint{0.920133in}{0.558237in}}%
\pgfpathlineto{\pgfqpoint{0.924818in}{0.546826in}}%
\pgfpathlineto{\pgfqpoint{0.930156in}{0.535973in}}%
\pgfpathlineto{\pgfqpoint{0.930462in}{0.535416in}}%
\pgfpathlineto{\pgfqpoint{0.936019in}{0.524005in}}%
\pgfpathlineto{\pgfqpoint{0.940362in}{0.512594in}}%
\pgfpathlineto{\pgfqpoint{0.941888in}{0.508273in}}%
\pgfpathlineto{\pgfqpoint{0.944670in}{0.501183in}}%
\pgfpathlineto{\pgfqpoint{0.949223in}{0.489772in}}%
\pgfpathlineto{\pgfqpoint{0.953620in}{0.479045in}}%
\pgfpathlineto{\pgfqpoint{0.953897in}{0.478362in}}%
\pgfpathlineto{\pgfqpoint{0.958537in}{0.466951in}}%
\pgfpathlineto{\pgfqpoint{0.963563in}{0.455540in}}%
\pgfpathlineto{\pgfqpoint{0.965352in}{0.451586in}}%
\pgfpathlineto{\pgfqpoint{0.968390in}{0.444129in}}%
\pgfpathlineto{\pgfqpoint{0.973366in}{0.432719in}}%
\pgfpathlineto{\pgfqpoint{0.977085in}{0.424916in}}%
\pgfpathlineto{\pgfqpoint{0.978535in}{0.421308in}}%
\pgfpathlineto{\pgfqpoint{0.983285in}{0.409897in}}%
\pgfpathlineto{\pgfqpoint{0.988699in}{0.398486in}}%
\pgfpathlineto{\pgfqpoint{0.988817in}{0.398245in}}%
\pgfpathlineto{\pgfqpoint{0.993228in}{0.387075in}}%
\pgfpathlineto{\pgfqpoint{0.998254in}{0.375665in}}%
\pgfpathlineto{\pgfqpoint{1.000549in}{0.370859in}}%
\pgfpathlineto{\pgfqpoint{1.003108in}{0.364254in}}%
\pgfpathlineto{\pgfqpoint{1.007766in}{0.352843in}}%
\pgfpathlineto{\pgfqpoint{1.012281in}{0.343009in}}%
\pgfpathlineto{\pgfqpoint{1.012883in}{0.341432in}}%
\pgfpathlineto{\pgfqpoint{1.017253in}{0.330022in}}%
\pgfpathlineto{\pgfqpoint{1.022101in}{0.318611in}}%
\pgfpathlineto{\pgfqpoint{1.024013in}{0.314423in}}%
\pgfpathlineto{\pgfqpoint{1.027064in}{0.307200in}}%
\pgfpathlineto{\pgfqpoint{1.031452in}{0.295789in}}%
\pgfpathclose%
\pgfpathmoveto{\pgfqpoint{1.158214in}{0.866328in}}%
\pgfpathlineto{\pgfqpoint{1.164800in}{0.866742in}}%
\pgfpathlineto{\pgfqpoint{1.176532in}{0.866506in}}%
\pgfpathlineto{\pgfqpoint{1.184230in}{0.866328in}}%
\pgfpathlineto{\pgfqpoint{1.176532in}{0.865891in}}%
\pgfpathlineto{\pgfqpoint{1.164800in}{0.865212in}}%
\pgfpathclose%
\pgfpathmoveto{\pgfqpoint{1.039573in}{0.877739in}}%
\pgfpathlineto{\pgfqpoint{1.035746in}{0.878993in}}%
\pgfpathlineto{\pgfqpoint{1.024013in}{0.885179in}}%
\pgfpathlineto{\pgfqpoint{1.017591in}{0.889150in}}%
\pgfpathlineto{\pgfqpoint{1.012281in}{0.892627in}}%
\pgfpathlineto{\pgfqpoint{1.000549in}{0.899820in}}%
\pgfpathlineto{\pgfqpoint{0.999477in}{0.900560in}}%
\pgfpathlineto{\pgfqpoint{0.988817in}{0.907706in}}%
\pgfpathlineto{\pgfqpoint{0.977085in}{0.910822in}}%
\pgfpathlineto{\pgfqpoint{0.971813in}{0.911971in}}%
\pgfpathlineto{\pgfqpoint{0.965352in}{0.913596in}}%
\pgfpathlineto{\pgfqpoint{0.953620in}{0.918199in}}%
\pgfpathlineto{\pgfqpoint{0.946768in}{0.923382in}}%
\pgfpathlineto{\pgfqpoint{0.941888in}{0.932666in}}%
\pgfpathlineto{\pgfqpoint{0.941363in}{0.934793in}}%
\pgfpathlineto{\pgfqpoint{0.941888in}{0.935811in}}%
\pgfpathlineto{\pgfqpoint{0.948227in}{0.946204in}}%
\pgfpathlineto{\pgfqpoint{0.953620in}{0.949431in}}%
\pgfpathlineto{\pgfqpoint{0.965352in}{0.954852in}}%
\pgfpathlineto{\pgfqpoint{0.971789in}{0.957614in}}%
\pgfpathlineto{\pgfqpoint{0.977085in}{0.959239in}}%
\pgfpathlineto{\pgfqpoint{0.988817in}{0.962365in}}%
\pgfpathlineto{\pgfqpoint{1.000549in}{0.965684in}}%
\pgfpathlineto{\pgfqpoint{1.011199in}{0.969025in}}%
\pgfpathlineto{\pgfqpoint{1.012281in}{0.969299in}}%
\pgfpathlineto{\pgfqpoint{1.024013in}{0.971468in}}%
\pgfpathlineto{\pgfqpoint{1.035746in}{0.972278in}}%
\pgfpathlineto{\pgfqpoint{1.047478in}{0.971971in}}%
\pgfpathlineto{\pgfqpoint{1.059210in}{0.970770in}}%
\pgfpathlineto{\pgfqpoint{1.070942in}{0.969505in}}%
\pgfpathlineto{\pgfqpoint{1.074472in}{0.969025in}}%
\pgfpathlineto{\pgfqpoint{1.082674in}{0.967383in}}%
\pgfpathlineto{\pgfqpoint{1.094407in}{0.964425in}}%
\pgfpathlineto{\pgfqpoint{1.106139in}{0.961954in}}%
\pgfpathlineto{\pgfqpoint{1.117871in}{0.961636in}}%
\pgfpathlineto{\pgfqpoint{1.129603in}{0.961697in}}%
\pgfpathlineto{\pgfqpoint{1.141335in}{0.962249in}}%
\pgfpathlineto{\pgfqpoint{1.153068in}{0.961957in}}%
\pgfpathlineto{\pgfqpoint{1.164800in}{0.961548in}}%
\pgfpathlineto{\pgfqpoint{1.176532in}{0.961078in}}%
\pgfpathlineto{\pgfqpoint{1.188264in}{0.959529in}}%
\pgfpathlineto{\pgfqpoint{1.197254in}{0.957614in}}%
\pgfpathlineto{\pgfqpoint{1.199996in}{0.956739in}}%
\pgfpathlineto{\pgfqpoint{1.211729in}{0.953571in}}%
\pgfpathlineto{\pgfqpoint{1.223461in}{0.952473in}}%
\pgfpathlineto{\pgfqpoint{1.235193in}{0.951591in}}%
\pgfpathlineto{\pgfqpoint{1.246925in}{0.949923in}}%
\pgfpathlineto{\pgfqpoint{1.258658in}{0.947802in}}%
\pgfpathlineto{\pgfqpoint{1.265708in}{0.946204in}}%
\pgfpathlineto{\pgfqpoint{1.270390in}{0.943963in}}%
\pgfpathlineto{\pgfqpoint{1.277617in}{0.934793in}}%
\pgfpathlineto{\pgfqpoint{1.271692in}{0.923382in}}%
\pgfpathlineto{\pgfqpoint{1.270390in}{0.922769in}}%
\pgfpathlineto{\pgfqpoint{1.258658in}{0.917590in}}%
\pgfpathlineto{\pgfqpoint{1.246925in}{0.912211in}}%
\pgfpathlineto{\pgfqpoint{1.246609in}{0.911971in}}%
\pgfpathlineto{\pgfqpoint{1.236961in}{0.900560in}}%
\pgfpathlineto{\pgfqpoint{1.235193in}{0.898562in}}%
\pgfpathlineto{\pgfqpoint{1.225911in}{0.889150in}}%
\pgfpathlineto{\pgfqpoint{1.223461in}{0.887042in}}%
\pgfpathlineto{\pgfqpoint{1.211729in}{0.882524in}}%
\pgfpathlineto{\pgfqpoint{1.199996in}{0.879392in}}%
\pgfpathlineto{\pgfqpoint{1.188659in}{0.877739in}}%
\pgfpathlineto{\pgfqpoint{1.188264in}{0.877686in}}%
\pgfpathlineto{\pgfqpoint{1.176532in}{0.876775in}}%
\pgfpathlineto{\pgfqpoint{1.164800in}{0.876124in}}%
\pgfpathlineto{\pgfqpoint{1.153068in}{0.875470in}}%
\pgfpathlineto{\pgfqpoint{1.141335in}{0.874755in}}%
\pgfpathlineto{\pgfqpoint{1.129603in}{0.874053in}}%
\pgfpathlineto{\pgfqpoint{1.117871in}{0.874081in}}%
\pgfpathlineto{\pgfqpoint{1.106139in}{0.874748in}}%
\pgfpathlineto{\pgfqpoint{1.094407in}{0.873789in}}%
\pgfpathlineto{\pgfqpoint{1.082674in}{0.871599in}}%
\pgfpathlineto{\pgfqpoint{1.070942in}{0.871410in}}%
\pgfpathlineto{\pgfqpoint{1.059210in}{0.873712in}}%
\pgfpathlineto{\pgfqpoint{1.047478in}{0.876107in}}%
\pgfpathclose%
\pgfpathmoveto{\pgfqpoint{0.918095in}{1.003257in}}%
\pgfpathlineto{\pgfqpoint{0.918424in}{1.004083in}}%
\pgfpathlineto{\pgfqpoint{0.918700in}{1.003257in}}%
\pgfpathlineto{\pgfqpoint{0.918424in}{1.003158in}}%
\pgfpathclose%
\pgfpathmoveto{\pgfqpoint{0.921446in}{1.014668in}}%
\pgfpathlineto{\pgfqpoint{0.929827in}{1.026079in}}%
\pgfpathlineto{\pgfqpoint{0.930156in}{1.026524in}}%
\pgfpathlineto{\pgfqpoint{0.941888in}{1.036952in}}%
\pgfpathlineto{\pgfqpoint{0.943032in}{1.037490in}}%
\pgfpathlineto{\pgfqpoint{0.953620in}{1.040438in}}%
\pgfpathlineto{\pgfqpoint{0.965352in}{1.043512in}}%
\pgfpathlineto{\pgfqpoint{0.977085in}{1.045630in}}%
\pgfpathlineto{\pgfqpoint{0.988817in}{1.041087in}}%
\pgfpathlineto{\pgfqpoint{0.989381in}{1.037490in}}%
\pgfpathlineto{\pgfqpoint{0.988817in}{1.037108in}}%
\pgfpathlineto{\pgfqpoint{0.977085in}{1.030407in}}%
\pgfpathlineto{\pgfqpoint{0.971975in}{1.026079in}}%
\pgfpathlineto{\pgfqpoint{0.965352in}{1.022512in}}%
\pgfpathlineto{\pgfqpoint{0.953620in}{1.018443in}}%
\pgfpathlineto{\pgfqpoint{0.946157in}{1.014668in}}%
\pgfpathlineto{\pgfqpoint{0.941888in}{1.012667in}}%
\pgfpathlineto{\pgfqpoint{0.930156in}{1.009288in}}%
\pgfpathclose%
\pgfpathmoveto{\pgfqpoint{0.994692in}{1.048901in}}%
\pgfpathlineto{\pgfqpoint{1.000549in}{1.049829in}}%
\pgfpathlineto{\pgfqpoint{1.002742in}{1.048901in}}%
\pgfpathlineto{\pgfqpoint{1.000549in}{1.047774in}}%
\pgfpathclose%
\pgfpathmoveto{\pgfqpoint{1.183545in}{1.151598in}}%
\pgfpathlineto{\pgfqpoint{1.176532in}{1.153539in}}%
\pgfpathlineto{\pgfqpoint{1.164800in}{1.154651in}}%
\pgfpathlineto{\pgfqpoint{1.153068in}{1.156104in}}%
\pgfpathlineto{\pgfqpoint{1.145206in}{1.163008in}}%
\pgfpathlineto{\pgfqpoint{1.141335in}{1.166554in}}%
\pgfpathlineto{\pgfqpoint{1.131871in}{1.174419in}}%
\pgfpathlineto{\pgfqpoint{1.129603in}{1.176366in}}%
\pgfpathlineto{\pgfqpoint{1.119479in}{1.185830in}}%
\pgfpathlineto{\pgfqpoint{1.117871in}{1.187378in}}%
\pgfpathlineto{\pgfqpoint{1.108894in}{1.197241in}}%
\pgfpathlineto{\pgfqpoint{1.106139in}{1.201315in}}%
\pgfpathlineto{\pgfqpoint{1.099912in}{1.208651in}}%
\pgfpathlineto{\pgfqpoint{1.094740in}{1.220062in}}%
\pgfpathlineto{\pgfqpoint{1.094407in}{1.221266in}}%
\pgfpathlineto{\pgfqpoint{1.091363in}{1.231473in}}%
\pgfpathlineto{\pgfqpoint{1.089760in}{1.242884in}}%
\pgfpathlineto{\pgfqpoint{1.094407in}{1.248554in}}%
\pgfpathlineto{\pgfqpoint{1.106139in}{1.249609in}}%
\pgfpathlineto{\pgfqpoint{1.117871in}{1.248851in}}%
\pgfpathlineto{\pgfqpoint{1.129603in}{1.246304in}}%
\pgfpathlineto{\pgfqpoint{1.135924in}{1.242884in}}%
\pgfpathlineto{\pgfqpoint{1.141335in}{1.239904in}}%
\pgfpathlineto{\pgfqpoint{1.153068in}{1.233542in}}%
\pgfpathlineto{\pgfqpoint{1.156885in}{1.231473in}}%
\pgfpathlineto{\pgfqpoint{1.164800in}{1.227137in}}%
\pgfpathlineto{\pgfqpoint{1.176532in}{1.220896in}}%
\pgfpathlineto{\pgfqpoint{1.177934in}{1.220062in}}%
\pgfpathlineto{\pgfqpoint{1.188264in}{1.214293in}}%
\pgfpathlineto{\pgfqpoint{1.197607in}{1.208651in}}%
\pgfpathlineto{\pgfqpoint{1.199996in}{1.206066in}}%
\pgfpathlineto{\pgfqpoint{1.203683in}{1.197241in}}%
\pgfpathlineto{\pgfqpoint{1.207889in}{1.185830in}}%
\pgfpathlineto{\pgfqpoint{1.211729in}{1.178318in}}%
\pgfpathlineto{\pgfqpoint{1.213534in}{1.174419in}}%
\pgfpathlineto{\pgfqpoint{1.211750in}{1.163008in}}%
\pgfpathlineto{\pgfqpoint{1.211729in}{1.162983in}}%
\pgfpathlineto{\pgfqpoint{1.200052in}{1.151598in}}%
\pgfpathlineto{\pgfqpoint{1.199996in}{1.151562in}}%
\pgfpathlineto{\pgfqpoint{1.188264in}{1.150532in}}%
\pgfpathclose%
\pgfusepath{fill}%
\end{pgfscope}%
\begin{pgfscope}%
\pgfpathrectangle{\pgfqpoint{0.211875in}{0.211875in}}{\pgfqpoint{1.313625in}{1.279725in}}%
\pgfusepath{clip}%
\pgfsetbuttcap%
\pgfsetroundjoin%
\definecolor{currentfill}{rgb}{0.848131,0.150999,0.281943}%
\pgfsetfillcolor{currentfill}%
\pgfsetlinewidth{0.000000pt}%
\definecolor{currentstroke}{rgb}{0.000000,0.000000,0.000000}%
\pgfsetstrokecolor{currentstroke}%
\pgfsetdash{}{0pt}%
\pgfpathmoveto{\pgfqpoint{0.296617in}{0.455354in}}%
\pgfpathlineto{\pgfqpoint{0.296831in}{0.455540in}}%
\pgfpathlineto{\pgfqpoint{0.308349in}{0.465483in}}%
\pgfpathlineto{\pgfqpoint{0.310050in}{0.466951in}}%
\pgfpathlineto{\pgfqpoint{0.320081in}{0.475627in}}%
\pgfpathlineto{\pgfqpoint{0.323266in}{0.478362in}}%
\pgfpathlineto{\pgfqpoint{0.331813in}{0.485767in}}%
\pgfpathlineto{\pgfqpoint{0.336504in}{0.489772in}}%
\pgfpathlineto{\pgfqpoint{0.343545in}{0.495880in}}%
\pgfpathlineto{\pgfqpoint{0.349790in}{0.501183in}}%
\pgfpathlineto{\pgfqpoint{0.355278in}{0.505946in}}%
\pgfpathlineto{\pgfqpoint{0.362975in}{0.512594in}}%
\pgfpathlineto{\pgfqpoint{0.367010in}{0.516110in}}%
\pgfpathlineto{\pgfqpoint{0.376085in}{0.524005in}}%
\pgfpathlineto{\pgfqpoint{0.378742in}{0.526345in}}%
\pgfpathlineto{\pgfqpoint{0.389196in}{0.535416in}}%
\pgfpathlineto{\pgfqpoint{0.390474in}{0.536561in}}%
\pgfpathlineto{\pgfqpoint{0.402206in}{0.546704in}}%
\pgfpathlineto{\pgfqpoint{0.402347in}{0.546826in}}%
\pgfpathlineto{\pgfqpoint{0.413939in}{0.556924in}}%
\pgfpathlineto{\pgfqpoint{0.415446in}{0.558237in}}%
\pgfpathlineto{\pgfqpoint{0.425671in}{0.567150in}}%
\pgfpathlineto{\pgfqpoint{0.428546in}{0.569648in}}%
\pgfpathlineto{\pgfqpoint{0.437403in}{0.577228in}}%
\pgfpathlineto{\pgfqpoint{0.441814in}{0.581059in}}%
\pgfpathlineto{\pgfqpoint{0.449135in}{0.586974in}}%
\pgfpathlineto{\pgfqpoint{0.455670in}{0.592469in}}%
\pgfpathlineto{\pgfqpoint{0.460867in}{0.596748in}}%
\pgfpathlineto{\pgfqpoint{0.469822in}{0.603880in}}%
\pgfpathlineto{\pgfqpoint{0.472600in}{0.606189in}}%
\pgfpathlineto{\pgfqpoint{0.484129in}{0.615291in}}%
\pgfpathlineto{\pgfqpoint{0.484332in}{0.615461in}}%
\pgfpathlineto{\pgfqpoint{0.496064in}{0.624740in}}%
\pgfpathlineto{\pgfqpoint{0.498476in}{0.626702in}}%
\pgfpathlineto{\pgfqpoint{0.507796in}{0.634176in}}%
\pgfpathlineto{\pgfqpoint{0.512659in}{0.638113in}}%
\pgfpathlineto{\pgfqpoint{0.519528in}{0.643695in}}%
\pgfpathlineto{\pgfqpoint{0.526700in}{0.649523in}}%
\pgfpathlineto{\pgfqpoint{0.531261in}{0.653285in}}%
\pgfpathlineto{\pgfqpoint{0.540694in}{0.660934in}}%
\pgfpathlineto{\pgfqpoint{0.542993in}{0.662870in}}%
\pgfpathlineto{\pgfqpoint{0.554718in}{0.672345in}}%
\pgfpathlineto{\pgfqpoint{0.554725in}{0.672351in}}%
\pgfpathlineto{\pgfqpoint{0.566457in}{0.681806in}}%
\pgfpathlineto{\pgfqpoint{0.568805in}{0.683756in}}%
\pgfpathlineto{\pgfqpoint{0.578190in}{0.691448in}}%
\pgfpathlineto{\pgfqpoint{0.582666in}{0.695166in}}%
\pgfpathlineto{\pgfqpoint{0.589922in}{0.701223in}}%
\pgfpathlineto{\pgfqpoint{0.596363in}{0.706577in}}%
\pgfpathlineto{\pgfqpoint{0.601654in}{0.711083in}}%
\pgfpathlineto{\pgfqpoint{0.609947in}{0.717988in}}%
\pgfpathlineto{\pgfqpoint{0.613386in}{0.720983in}}%
\pgfpathlineto{\pgfqpoint{0.623462in}{0.729399in}}%
\pgfpathlineto{\pgfqpoint{0.625118in}{0.730880in}}%
\pgfpathlineto{\pgfqpoint{0.636851in}{0.740729in}}%
\pgfpathlineto{\pgfqpoint{0.636944in}{0.740810in}}%
\pgfpathlineto{\pgfqpoint{0.648583in}{0.750727in}}%
\pgfpathlineto{\pgfqpoint{0.650312in}{0.752220in}}%
\pgfpathlineto{\pgfqpoint{0.660315in}{0.760968in}}%
\pgfpathlineto{\pgfqpoint{0.662990in}{0.763631in}}%
\pgfpathlineto{\pgfqpoint{0.672047in}{0.773200in}}%
\pgfpathlineto{\pgfqpoint{0.673797in}{0.775042in}}%
\pgfpathlineto{\pgfqpoint{0.683741in}{0.786453in}}%
\pgfpathlineto{\pgfqpoint{0.683779in}{0.786518in}}%
\pgfpathlineto{\pgfqpoint{0.692200in}{0.797863in}}%
\pgfpathlineto{\pgfqpoint{0.695512in}{0.803154in}}%
\pgfpathlineto{\pgfqpoint{0.701637in}{0.809274in}}%
\pgfpathlineto{\pgfqpoint{0.707244in}{0.817770in}}%
\pgfpathlineto{\pgfqpoint{0.710024in}{0.820685in}}%
\pgfpathlineto{\pgfqpoint{0.707244in}{0.823477in}}%
\pgfpathlineto{\pgfqpoint{0.699628in}{0.820685in}}%
\pgfpathlineto{\pgfqpoint{0.695512in}{0.819029in}}%
\pgfpathlineto{\pgfqpoint{0.683779in}{0.813071in}}%
\pgfpathlineto{\pgfqpoint{0.672047in}{0.810411in}}%
\pgfpathlineto{\pgfqpoint{0.667878in}{0.809274in}}%
\pgfpathlineto{\pgfqpoint{0.660315in}{0.807134in}}%
\pgfpathlineto{\pgfqpoint{0.648583in}{0.800353in}}%
\pgfpathlineto{\pgfqpoint{0.644257in}{0.797863in}}%
\pgfpathlineto{\pgfqpoint{0.636851in}{0.793536in}}%
\pgfpathlineto{\pgfqpoint{0.625118in}{0.786965in}}%
\pgfpathlineto{\pgfqpoint{0.624215in}{0.786453in}}%
\pgfpathlineto{\pgfqpoint{0.613386in}{0.780173in}}%
\pgfpathlineto{\pgfqpoint{0.604150in}{0.775042in}}%
\pgfpathlineto{\pgfqpoint{0.601654in}{0.773592in}}%
\pgfpathlineto{\pgfqpoint{0.589922in}{0.766910in}}%
\pgfpathlineto{\pgfqpoint{0.583989in}{0.763631in}}%
\pgfpathlineto{\pgfqpoint{0.578190in}{0.760262in}}%
\pgfpathlineto{\pgfqpoint{0.566457in}{0.753711in}}%
\pgfpathlineto{\pgfqpoint{0.563759in}{0.752220in}}%
\pgfpathlineto{\pgfqpoint{0.554725in}{0.746950in}}%
\pgfpathlineto{\pgfqpoint{0.543575in}{0.740810in}}%
\pgfpathlineto{\pgfqpoint{0.542993in}{0.740465in}}%
\pgfpathlineto{\pgfqpoint{0.531261in}{0.733632in}}%
\pgfpathlineto{\pgfqpoint{0.523575in}{0.729399in}}%
\pgfpathlineto{\pgfqpoint{0.519528in}{0.726992in}}%
\pgfpathlineto{\pgfqpoint{0.507796in}{0.720320in}}%
\pgfpathlineto{\pgfqpoint{0.503549in}{0.717988in}}%
\pgfpathlineto{\pgfqpoint{0.496064in}{0.713540in}}%
\pgfpathlineto{\pgfqpoint{0.484332in}{0.707040in}}%
\pgfpathlineto{\pgfqpoint{0.483494in}{0.706577in}}%
\pgfpathlineto{\pgfqpoint{0.472600in}{0.700089in}}%
\pgfpathlineto{\pgfqpoint{0.463703in}{0.695166in}}%
\pgfpathlineto{\pgfqpoint{0.460867in}{0.693440in}}%
\pgfpathlineto{\pgfqpoint{0.449135in}{0.686644in}}%
\pgfpathlineto{\pgfqpoint{0.443855in}{0.683756in}}%
\pgfpathlineto{\pgfqpoint{0.437403in}{0.679874in}}%
\pgfpathlineto{\pgfqpoint{0.425671in}{0.673351in}}%
\pgfpathlineto{\pgfqpoint{0.423806in}{0.672345in}}%
\pgfpathlineto{\pgfqpoint{0.413939in}{0.666496in}}%
\pgfpathlineto{\pgfqpoint{0.403634in}{0.660934in}}%
\pgfpathlineto{\pgfqpoint{0.402206in}{0.660075in}}%
\pgfpathlineto{\pgfqpoint{0.390474in}{0.653319in}}%
\pgfpathlineto{\pgfqpoint{0.383322in}{0.649523in}}%
\pgfpathlineto{\pgfqpoint{0.378742in}{0.646816in}}%
\pgfpathlineto{\pgfqpoint{0.367010in}{0.640351in}}%
\pgfpathlineto{\pgfqpoint{0.362718in}{0.638113in}}%
\pgfpathlineto{\pgfqpoint{0.355278in}{0.633797in}}%
\pgfpathlineto{\pgfqpoint{0.343545in}{0.627590in}}%
\pgfpathlineto{\pgfqpoint{0.341813in}{0.626702in}}%
\pgfpathlineto{\pgfqpoint{0.331813in}{0.621009in}}%
\pgfpathlineto{\pgfqpoint{0.320648in}{0.615291in}}%
\pgfpathlineto{\pgfqpoint{0.320081in}{0.614965in}}%
\pgfpathlineto{\pgfqpoint{0.308349in}{0.608433in}}%
\pgfpathlineto{\pgfqpoint{0.299311in}{0.603880in}}%
\pgfpathlineto{\pgfqpoint{0.296617in}{0.602358in}}%
\pgfpathlineto{\pgfqpoint{0.284884in}{0.596047in}}%
\pgfpathlineto{\pgfqpoint{0.284884in}{0.592469in}}%
\pgfpathlineto{\pgfqpoint{0.284884in}{0.581059in}}%
\pgfpathlineto{\pgfqpoint{0.284884in}{0.569648in}}%
\pgfpathlineto{\pgfqpoint{0.284884in}{0.561222in}}%
\pgfpathlineto{\pgfqpoint{0.296617in}{0.568267in}}%
\pgfpathlineto{\pgfqpoint{0.298759in}{0.569648in}}%
\pgfpathlineto{\pgfqpoint{0.308349in}{0.574983in}}%
\pgfpathlineto{\pgfqpoint{0.317853in}{0.581059in}}%
\pgfpathlineto{\pgfqpoint{0.320081in}{0.582287in}}%
\pgfpathlineto{\pgfqpoint{0.331813in}{0.589282in}}%
\pgfpathlineto{\pgfqpoint{0.336631in}{0.592469in}}%
\pgfpathlineto{\pgfqpoint{0.343545in}{0.596420in}}%
\pgfpathlineto{\pgfqpoint{0.354981in}{0.603880in}}%
\pgfpathlineto{\pgfqpoint{0.355278in}{0.604047in}}%
\pgfpathlineto{\pgfqpoint{0.367010in}{0.611025in}}%
\pgfpathlineto{\pgfqpoint{0.373365in}{0.615291in}}%
\pgfpathlineto{\pgfqpoint{0.378742in}{0.618414in}}%
\pgfpathlineto{\pgfqpoint{0.390474in}{0.626063in}}%
\pgfpathlineto{\pgfqpoint{0.391403in}{0.626702in}}%
\pgfpathlineto{\pgfqpoint{0.402206in}{0.633156in}}%
\pgfpathlineto{\pgfqpoint{0.409568in}{0.638113in}}%
\pgfpathlineto{\pgfqpoint{0.413939in}{0.640667in}}%
\pgfpathlineto{\pgfqpoint{0.425671in}{0.648228in}}%
\pgfpathlineto{\pgfqpoint{0.427564in}{0.649523in}}%
\pgfpathlineto{\pgfqpoint{0.437403in}{0.655386in}}%
\pgfpathlineto{\pgfqpoint{0.445704in}{0.660934in}}%
\pgfpathlineto{\pgfqpoint{0.449135in}{0.662934in}}%
\pgfpathlineto{\pgfqpoint{0.460867in}{0.670358in}}%
\pgfpathlineto{\pgfqpoint{0.463789in}{0.672345in}}%
\pgfpathlineto{\pgfqpoint{0.472600in}{0.677575in}}%
\pgfpathlineto{\pgfqpoint{0.481982in}{0.683756in}}%
\pgfpathlineto{\pgfqpoint{0.484332in}{0.685114in}}%
\pgfpathlineto{\pgfqpoint{0.496064in}{0.692277in}}%
\pgfpathlineto{\pgfqpoint{0.500458in}{0.695166in}}%
\pgfpathlineto{\pgfqpoint{0.507796in}{0.699410in}}%
\pgfpathlineto{\pgfqpoint{0.519035in}{0.706577in}}%
\pgfpathlineto{\pgfqpoint{0.519528in}{0.706857in}}%
\pgfpathlineto{\pgfqpoint{0.531261in}{0.713619in}}%
\pgfpathlineto{\pgfqpoint{0.538170in}{0.717988in}}%
\pgfpathlineto{\pgfqpoint{0.542993in}{0.720695in}}%
\pgfpathlineto{\pgfqpoint{0.554725in}{0.727682in}}%
\pgfpathlineto{\pgfqpoint{0.557480in}{0.729399in}}%
\pgfpathlineto{\pgfqpoint{0.566457in}{0.734343in}}%
\pgfpathlineto{\pgfqpoint{0.577104in}{0.740810in}}%
\pgfpathlineto{\pgfqpoint{0.578190in}{0.741402in}}%
\pgfpathlineto{\pgfqpoint{0.589922in}{0.747710in}}%
\pgfpathlineto{\pgfqpoint{0.597492in}{0.752220in}}%
\pgfpathlineto{\pgfqpoint{0.601654in}{0.754428in}}%
\pgfpathlineto{\pgfqpoint{0.613386in}{0.760612in}}%
\pgfpathlineto{\pgfqpoint{0.618683in}{0.763631in}}%
\pgfpathlineto{\pgfqpoint{0.625118in}{0.766794in}}%
\pgfpathlineto{\pgfqpoint{0.636851in}{0.772328in}}%
\pgfpathlineto{\pgfqpoint{0.642455in}{0.775042in}}%
\pgfpathlineto{\pgfqpoint{0.648583in}{0.777324in}}%
\pgfpathlineto{\pgfqpoint{0.653407in}{0.775042in}}%
\pgfpathlineto{\pgfqpoint{0.648583in}{0.769736in}}%
\pgfpathlineto{\pgfqpoint{0.645276in}{0.763631in}}%
\pgfpathlineto{\pgfqpoint{0.636851in}{0.755790in}}%
\pgfpathlineto{\pgfqpoint{0.634128in}{0.752220in}}%
\pgfpathlineto{\pgfqpoint{0.625118in}{0.744364in}}%
\pgfpathlineto{\pgfqpoint{0.621825in}{0.740810in}}%
\pgfpathlineto{\pgfqpoint{0.613386in}{0.733726in}}%
\pgfpathlineto{\pgfqpoint{0.609040in}{0.729399in}}%
\pgfpathlineto{\pgfqpoint{0.601654in}{0.723336in}}%
\pgfpathlineto{\pgfqpoint{0.596009in}{0.717988in}}%
\pgfpathlineto{\pgfqpoint{0.589922in}{0.713053in}}%
\pgfpathlineto{\pgfqpoint{0.582840in}{0.706577in}}%
\pgfpathlineto{\pgfqpoint{0.578190in}{0.702829in}}%
\pgfpathlineto{\pgfqpoint{0.569584in}{0.695166in}}%
\pgfpathlineto{\pgfqpoint{0.566457in}{0.692650in}}%
\pgfpathlineto{\pgfqpoint{0.556258in}{0.683756in}}%
\pgfpathlineto{\pgfqpoint{0.554725in}{0.682521in}}%
\pgfpathlineto{\pgfqpoint{0.542993in}{0.672548in}}%
\pgfpathlineto{\pgfqpoint{0.542767in}{0.672345in}}%
\pgfpathlineto{\pgfqpoint{0.531261in}{0.663488in}}%
\pgfpathlineto{\pgfqpoint{0.528359in}{0.660934in}}%
\pgfpathlineto{\pgfqpoint{0.519528in}{0.654178in}}%
\pgfpathlineto{\pgfqpoint{0.514156in}{0.649523in}}%
\pgfpathlineto{\pgfqpoint{0.507796in}{0.644627in}}%
\pgfpathlineto{\pgfqpoint{0.500184in}{0.638113in}}%
\pgfpathlineto{\pgfqpoint{0.496064in}{0.634916in}}%
\pgfpathlineto{\pgfqpoint{0.486365in}{0.626702in}}%
\pgfpathlineto{\pgfqpoint{0.484332in}{0.625111in}}%
\pgfpathlineto{\pgfqpoint{0.472626in}{0.615291in}}%
\pgfpathlineto{\pgfqpoint{0.472600in}{0.615270in}}%
\pgfpathlineto{\pgfqpoint{0.460867in}{0.606480in}}%
\pgfpathlineto{\pgfqpoint{0.457862in}{0.603880in}}%
\pgfpathlineto{\pgfqpoint{0.449135in}{0.597277in}}%
\pgfpathlineto{\pgfqpoint{0.443544in}{0.592469in}}%
\pgfpathlineto{\pgfqpoint{0.437403in}{0.587770in}}%
\pgfpathlineto{\pgfqpoint{0.429556in}{0.581059in}}%
\pgfpathlineto{\pgfqpoint{0.425671in}{0.577801in}}%
\pgfpathlineto{\pgfqpoint{0.415968in}{0.569648in}}%
\pgfpathlineto{\pgfqpoint{0.413939in}{0.567910in}}%
\pgfpathlineto{\pgfqpoint{0.403119in}{0.558237in}}%
\pgfpathlineto{\pgfqpoint{0.402206in}{0.557452in}}%
\pgfpathlineto{\pgfqpoint{0.390474in}{0.547032in}}%
\pgfpathlineto{\pgfqpoint{0.390244in}{0.546826in}}%
\pgfpathlineto{\pgfqpoint{0.378742in}{0.537119in}}%
\pgfpathlineto{\pgfqpoint{0.376831in}{0.535416in}}%
\pgfpathlineto{\pgfqpoint{0.367010in}{0.527317in}}%
\pgfpathlineto{\pgfqpoint{0.363295in}{0.524005in}}%
\pgfpathlineto{\pgfqpoint{0.355278in}{0.517426in}}%
\pgfpathlineto{\pgfqpoint{0.349863in}{0.512594in}}%
\pgfpathlineto{\pgfqpoint{0.343545in}{0.507377in}}%
\pgfpathlineto{\pgfqpoint{0.336606in}{0.501183in}}%
\pgfpathlineto{\pgfqpoint{0.331813in}{0.497196in}}%
\pgfpathlineto{\pgfqpoint{0.323495in}{0.489772in}}%
\pgfpathlineto{\pgfqpoint{0.320081in}{0.486913in}}%
\pgfpathlineto{\pgfqpoint{0.310509in}{0.478362in}}%
\pgfpathlineto{\pgfqpoint{0.308349in}{0.476542in}}%
\pgfpathlineto{\pgfqpoint{0.297602in}{0.466951in}}%
\pgfpathlineto{\pgfqpoint{0.296617in}{0.466116in}}%
\pgfpathlineto{\pgfqpoint{0.284884in}{0.455742in}}%
\pgfpathlineto{\pgfqpoint{0.284884in}{0.455540in}}%
\pgfpathlineto{\pgfqpoint{0.284884in}{0.445350in}}%
\pgfpathclose%
\pgfusepath{fill}%
\end{pgfscope}%
\begin{pgfscope}%
\pgfpathrectangle{\pgfqpoint{0.211875in}{0.211875in}}{\pgfqpoint{1.313625in}{1.279725in}}%
\pgfusepath{clip}%
\pgfsetbuttcap%
\pgfsetroundjoin%
\definecolor{currentfill}{rgb}{0.848131,0.150999,0.281943}%
\pgfsetfillcolor{currentfill}%
\pgfsetlinewidth{0.000000pt}%
\definecolor{currentstroke}{rgb}{0.000000,0.000000,0.000000}%
\pgfsetstrokecolor{currentstroke}%
\pgfsetdash{}{0pt}%
\pgfpathmoveto{\pgfqpoint{1.446373in}{0.773792in}}%
\pgfpathlineto{\pgfqpoint{1.446373in}{0.775042in}}%
\pgfpathlineto{\pgfqpoint{1.446373in}{0.786453in}}%
\pgfpathlineto{\pgfqpoint{1.446373in}{0.797863in}}%
\pgfpathlineto{\pgfqpoint{1.446373in}{0.809274in}}%
\pgfpathlineto{\pgfqpoint{1.446373in}{0.820685in}}%
\pgfpathlineto{\pgfqpoint{1.446373in}{0.832096in}}%
\pgfpathlineto{\pgfqpoint{1.446373in}{0.843507in}}%
\pgfpathlineto{\pgfqpoint{1.446373in}{0.854917in}}%
\pgfpathlineto{\pgfqpoint{1.446373in}{0.855736in}}%
\pgfpathlineto{\pgfqpoint{1.445960in}{0.854917in}}%
\pgfpathlineto{\pgfqpoint{1.440515in}{0.843507in}}%
\pgfpathlineto{\pgfqpoint{1.435276in}{0.832096in}}%
\pgfpathlineto{\pgfqpoint{1.434641in}{0.827740in}}%
\pgfpathlineto{\pgfqpoint{1.433737in}{0.820685in}}%
\pgfpathlineto{\pgfqpoint{1.434641in}{0.815576in}}%
\pgfpathlineto{\pgfqpoint{1.435642in}{0.809274in}}%
\pgfpathlineto{\pgfqpoint{1.438648in}{0.797863in}}%
\pgfpathlineto{\pgfqpoint{1.442234in}{0.786453in}}%
\pgfpathlineto{\pgfqpoint{1.445878in}{0.775042in}}%
\pgfpathclose%
\pgfusepath{fill}%
\end{pgfscope}%
\begin{pgfscope}%
\pgfpathrectangle{\pgfqpoint{0.211875in}{0.211875in}}{\pgfqpoint{1.313625in}{1.279725in}}%
\pgfusepath{clip}%
\pgfsetbuttcap%
\pgfsetroundjoin%
\definecolor{currentfill}{rgb}{0.848131,0.150999,0.281943}%
\pgfsetfillcolor{currentfill}%
\pgfsetlinewidth{0.000000pt}%
\definecolor{currentstroke}{rgb}{0.000000,0.000000,0.000000}%
\pgfsetstrokecolor{currentstroke}%
\pgfsetdash{}{0pt}%
\pgfpathmoveto{\pgfqpoint{0.777637in}{0.876971in}}%
\pgfpathlineto{\pgfqpoint{0.784073in}{0.877739in}}%
\pgfpathlineto{\pgfqpoint{0.789369in}{0.878434in}}%
\pgfpathlineto{\pgfqpoint{0.801101in}{0.884780in}}%
\pgfpathlineto{\pgfqpoint{0.809482in}{0.889150in}}%
\pgfpathlineto{\pgfqpoint{0.812834in}{0.891554in}}%
\pgfpathlineto{\pgfqpoint{0.823601in}{0.900560in}}%
\pgfpathlineto{\pgfqpoint{0.824566in}{0.907051in}}%
\pgfpathlineto{\pgfqpoint{0.825192in}{0.911971in}}%
\pgfpathlineto{\pgfqpoint{0.824566in}{0.912583in}}%
\pgfpathlineto{\pgfqpoint{0.812834in}{0.916575in}}%
\pgfpathlineto{\pgfqpoint{0.801101in}{0.916664in}}%
\pgfpathlineto{\pgfqpoint{0.789369in}{0.912663in}}%
\pgfpathlineto{\pgfqpoint{0.787360in}{0.911971in}}%
\pgfpathlineto{\pgfqpoint{0.779916in}{0.900560in}}%
\pgfpathlineto{\pgfqpoint{0.777637in}{0.892605in}}%
\pgfpathlineto{\pgfqpoint{0.775559in}{0.889150in}}%
\pgfpathlineto{\pgfqpoint{0.773599in}{0.877739in}}%
\pgfpathclose%
\pgfusepath{fill}%
\end{pgfscope}%
\begin{pgfscope}%
\pgfpathrectangle{\pgfqpoint{0.211875in}{0.211875in}}{\pgfqpoint{1.313625in}{1.279725in}}%
\pgfusepath{clip}%
\pgfsetbuttcap%
\pgfsetroundjoin%
\definecolor{currentfill}{rgb}{0.848131,0.150999,0.281943}%
\pgfsetfillcolor{currentfill}%
\pgfsetlinewidth{0.000000pt}%
\definecolor{currentstroke}{rgb}{0.000000,0.000000,0.000000}%
\pgfsetstrokecolor{currentstroke}%
\pgfsetdash{}{0pt}%
\pgfpathmoveto{\pgfqpoint{0.660315in}{0.897582in}}%
\pgfpathlineto{\pgfqpoint{0.665708in}{0.900560in}}%
\pgfpathlineto{\pgfqpoint{0.672047in}{0.903862in}}%
\pgfpathlineto{\pgfqpoint{0.682057in}{0.911971in}}%
\pgfpathlineto{\pgfqpoint{0.683779in}{0.913346in}}%
\pgfpathlineto{\pgfqpoint{0.695512in}{0.921950in}}%
\pgfpathlineto{\pgfqpoint{0.697750in}{0.923382in}}%
\pgfpathlineto{\pgfqpoint{0.707244in}{0.929920in}}%
\pgfpathlineto{\pgfqpoint{0.714847in}{0.934793in}}%
\pgfpathlineto{\pgfqpoint{0.718976in}{0.938016in}}%
\pgfpathlineto{\pgfqpoint{0.727921in}{0.946204in}}%
\pgfpathlineto{\pgfqpoint{0.730708in}{0.952599in}}%
\pgfpathlineto{\pgfqpoint{0.731860in}{0.957614in}}%
\pgfpathlineto{\pgfqpoint{0.730738in}{0.969025in}}%
\pgfpathlineto{\pgfqpoint{0.730708in}{0.969098in}}%
\pgfpathlineto{\pgfqpoint{0.726581in}{0.980436in}}%
\pgfpathlineto{\pgfqpoint{0.721627in}{0.991847in}}%
\pgfpathlineto{\pgfqpoint{0.718976in}{0.996766in}}%
\pgfpathlineto{\pgfqpoint{0.715539in}{1.003257in}}%
\pgfpathlineto{\pgfqpoint{0.708692in}{1.014668in}}%
\pgfpathlineto{\pgfqpoint{0.707244in}{1.016250in}}%
\pgfpathlineto{\pgfqpoint{0.698185in}{1.026079in}}%
\pgfpathlineto{\pgfqpoint{0.695512in}{1.028946in}}%
\pgfpathlineto{\pgfqpoint{0.687474in}{1.037490in}}%
\pgfpathlineto{\pgfqpoint{0.683779in}{1.041110in}}%
\pgfpathlineto{\pgfqpoint{0.673769in}{1.048901in}}%
\pgfpathlineto{\pgfqpoint{0.672047in}{1.050171in}}%
\pgfpathlineto{\pgfqpoint{0.660315in}{1.058622in}}%
\pgfpathlineto{\pgfqpoint{0.656816in}{1.060311in}}%
\pgfpathlineto{\pgfqpoint{0.648583in}{1.064325in}}%
\pgfpathlineto{\pgfqpoint{0.636851in}{1.067814in}}%
\pgfpathlineto{\pgfqpoint{0.625118in}{1.071257in}}%
\pgfpathlineto{\pgfqpoint{0.621272in}{1.071722in}}%
\pgfpathlineto{\pgfqpoint{0.613386in}{1.072550in}}%
\pgfpathlineto{\pgfqpoint{0.607236in}{1.071722in}}%
\pgfpathlineto{\pgfqpoint{0.601654in}{1.068840in}}%
\pgfpathlineto{\pgfqpoint{0.596317in}{1.060311in}}%
\pgfpathlineto{\pgfqpoint{0.594520in}{1.048901in}}%
\pgfpathlineto{\pgfqpoint{0.593958in}{1.037490in}}%
\pgfpathlineto{\pgfqpoint{0.593650in}{1.026079in}}%
\pgfpathlineto{\pgfqpoint{0.593361in}{1.014668in}}%
\pgfpathlineto{\pgfqpoint{0.593184in}{1.003257in}}%
\pgfpathlineto{\pgfqpoint{0.593729in}{0.991847in}}%
\pgfpathlineto{\pgfqpoint{0.598588in}{0.980436in}}%
\pgfpathlineto{\pgfqpoint{0.601654in}{0.974226in}}%
\pgfpathlineto{\pgfqpoint{0.604051in}{0.969025in}}%
\pgfpathlineto{\pgfqpoint{0.610058in}{0.957614in}}%
\pgfpathlineto{\pgfqpoint{0.613386in}{0.951947in}}%
\pgfpathlineto{\pgfqpoint{0.616510in}{0.946204in}}%
\pgfpathlineto{\pgfqpoint{0.624807in}{0.934793in}}%
\pgfpathlineto{\pgfqpoint{0.625118in}{0.934452in}}%
\pgfpathlineto{\pgfqpoint{0.635957in}{0.923382in}}%
\pgfpathlineto{\pgfqpoint{0.636851in}{0.922542in}}%
\pgfpathlineto{\pgfqpoint{0.647191in}{0.911971in}}%
\pgfpathlineto{\pgfqpoint{0.648583in}{0.910602in}}%
\pgfpathlineto{\pgfqpoint{0.657646in}{0.900560in}}%
\pgfpathclose%
\pgfpathmoveto{\pgfqpoint{0.657410in}{0.957614in}}%
\pgfpathlineto{\pgfqpoint{0.649576in}{0.969025in}}%
\pgfpathlineto{\pgfqpoint{0.648583in}{0.970652in}}%
\pgfpathlineto{\pgfqpoint{0.643136in}{0.980436in}}%
\pgfpathlineto{\pgfqpoint{0.638534in}{0.991847in}}%
\pgfpathlineto{\pgfqpoint{0.636851in}{0.999377in}}%
\pgfpathlineto{\pgfqpoint{0.635958in}{1.003257in}}%
\pgfpathlineto{\pgfqpoint{0.636285in}{1.014668in}}%
\pgfpathlineto{\pgfqpoint{0.636851in}{1.019942in}}%
\pgfpathlineto{\pgfqpoint{0.637547in}{1.026079in}}%
\pgfpathlineto{\pgfqpoint{0.641958in}{1.037490in}}%
\pgfpathlineto{\pgfqpoint{0.648583in}{1.042178in}}%
\pgfpathlineto{\pgfqpoint{0.660315in}{1.042279in}}%
\pgfpathlineto{\pgfqpoint{0.667945in}{1.037490in}}%
\pgfpathlineto{\pgfqpoint{0.672047in}{1.034634in}}%
\pgfpathlineto{\pgfqpoint{0.681622in}{1.026079in}}%
\pgfpathlineto{\pgfqpoint{0.683779in}{1.023524in}}%
\pgfpathlineto{\pgfqpoint{0.690851in}{1.014668in}}%
\pgfpathlineto{\pgfqpoint{0.695512in}{1.005464in}}%
\pgfpathlineto{\pgfqpoint{0.696520in}{1.003257in}}%
\pgfpathlineto{\pgfqpoint{0.696925in}{0.991847in}}%
\pgfpathlineto{\pgfqpoint{0.695512in}{0.986655in}}%
\pgfpathlineto{\pgfqpoint{0.692761in}{0.980436in}}%
\pgfpathlineto{\pgfqpoint{0.683779in}{0.969140in}}%
\pgfpathlineto{\pgfqpoint{0.683666in}{0.969025in}}%
\pgfpathlineto{\pgfqpoint{0.672047in}{0.958567in}}%
\pgfpathlineto{\pgfqpoint{0.669715in}{0.957614in}}%
\pgfpathlineto{\pgfqpoint{0.660315in}{0.953794in}}%
\pgfpathclose%
\pgfusepath{fill}%
\end{pgfscope}%
\begin{pgfscope}%
\pgfpathrectangle{\pgfqpoint{0.211875in}{0.211875in}}{\pgfqpoint{1.313625in}{1.279725in}}%
\pgfusepath{clip}%
\pgfsetbuttcap%
\pgfsetroundjoin%
\definecolor{currentfill}{rgb}{0.848131,0.150999,0.281943}%
\pgfsetfillcolor{currentfill}%
\pgfsetlinewidth{0.000000pt}%
\definecolor{currentstroke}{rgb}{0.000000,0.000000,0.000000}%
\pgfsetstrokecolor{currentstroke}%
\pgfsetdash{}{0pt}%
\pgfpathmoveto{\pgfqpoint{1.364247in}{0.966607in}}%
\pgfpathlineto{\pgfqpoint{1.375980in}{0.961510in}}%
\pgfpathlineto{\pgfqpoint{1.387712in}{0.957774in}}%
\pgfpathlineto{\pgfqpoint{1.394015in}{0.969025in}}%
\pgfpathlineto{\pgfqpoint{1.396115in}{0.980436in}}%
\pgfpathlineto{\pgfqpoint{1.387712in}{0.986839in}}%
\pgfpathlineto{\pgfqpoint{1.375980in}{0.988713in}}%
\pgfpathlineto{\pgfqpoint{1.364247in}{0.988074in}}%
\pgfpathlineto{\pgfqpoint{1.352515in}{0.983291in}}%
\pgfpathlineto{\pgfqpoint{1.346983in}{0.980436in}}%
\pgfpathlineto{\pgfqpoint{1.352515in}{0.973635in}}%
\pgfpathlineto{\pgfqpoint{1.357418in}{0.969025in}}%
\pgfpathclose%
\pgfusepath{fill}%
\end{pgfscope}%
\begin{pgfscope}%
\pgfpathrectangle{\pgfqpoint{0.211875in}{0.211875in}}{\pgfqpoint{1.313625in}{1.279725in}}%
\pgfusepath{clip}%
\pgfsetbuttcap%
\pgfsetroundjoin%
\definecolor{currentfill}{rgb}{0.848131,0.150999,0.281943}%
\pgfsetfillcolor{currentfill}%
\pgfsetlinewidth{0.000000pt}%
\definecolor{currentstroke}{rgb}{0.000000,0.000000,0.000000}%
\pgfsetstrokecolor{currentstroke}%
\pgfsetdash{}{0pt}%
\pgfpathmoveto{\pgfqpoint{1.246925in}{0.980061in}}%
\pgfpathlineto{\pgfqpoint{1.258658in}{0.978077in}}%
\pgfpathlineto{\pgfqpoint{1.270390in}{0.977450in}}%
\pgfpathlineto{\pgfqpoint{1.282122in}{0.976967in}}%
\pgfpathlineto{\pgfqpoint{1.293854in}{0.977060in}}%
\pgfpathlineto{\pgfqpoint{1.305586in}{0.979292in}}%
\pgfpathlineto{\pgfqpoint{1.310117in}{0.980436in}}%
\pgfpathlineto{\pgfqpoint{1.317319in}{0.982266in}}%
\pgfpathlineto{\pgfqpoint{1.328254in}{0.991847in}}%
\pgfpathlineto{\pgfqpoint{1.329051in}{0.993125in}}%
\pgfpathlineto{\pgfqpoint{1.334567in}{1.003257in}}%
\pgfpathlineto{\pgfqpoint{1.339744in}{1.014668in}}%
\pgfpathlineto{\pgfqpoint{1.340783in}{1.019270in}}%
\pgfpathlineto{\pgfqpoint{1.342058in}{1.026079in}}%
\pgfpathlineto{\pgfqpoint{1.341937in}{1.037490in}}%
\pgfpathlineto{\pgfqpoint{1.352515in}{1.045415in}}%
\pgfpathlineto{\pgfqpoint{1.357068in}{1.048901in}}%
\pgfpathlineto{\pgfqpoint{1.364247in}{1.055545in}}%
\pgfpathlineto{\pgfqpoint{1.375188in}{1.060311in}}%
\pgfpathlineto{\pgfqpoint{1.375980in}{1.060605in}}%
\pgfpathlineto{\pgfqpoint{1.387712in}{1.066350in}}%
\pgfpathlineto{\pgfqpoint{1.396788in}{1.071722in}}%
\pgfpathlineto{\pgfqpoint{1.399444in}{1.074825in}}%
\pgfpathlineto{\pgfqpoint{1.403178in}{1.083133in}}%
\pgfpathlineto{\pgfqpoint{1.407332in}{1.094544in}}%
\pgfpathlineto{\pgfqpoint{1.411007in}{1.105954in}}%
\pgfpathlineto{\pgfqpoint{1.411176in}{1.106882in}}%
\pgfpathlineto{\pgfqpoint{1.415378in}{1.117365in}}%
\pgfpathlineto{\pgfqpoint{1.411176in}{1.124431in}}%
\pgfpathlineto{\pgfqpoint{1.399444in}{1.125294in}}%
\pgfpathlineto{\pgfqpoint{1.387712in}{1.128672in}}%
\pgfpathlineto{\pgfqpoint{1.387403in}{1.128776in}}%
\pgfpathlineto{\pgfqpoint{1.375980in}{1.132174in}}%
\pgfpathlineto{\pgfqpoint{1.364247in}{1.136105in}}%
\pgfpathlineto{\pgfqpoint{1.354072in}{1.140187in}}%
\pgfpathlineto{\pgfqpoint{1.352515in}{1.141211in}}%
\pgfpathlineto{\pgfqpoint{1.340783in}{1.150118in}}%
\pgfpathlineto{\pgfqpoint{1.339024in}{1.151598in}}%
\pgfpathlineto{\pgfqpoint{1.340783in}{1.158996in}}%
\pgfpathlineto{\pgfqpoint{1.342991in}{1.163008in}}%
\pgfpathlineto{\pgfqpoint{1.352515in}{1.168974in}}%
\pgfpathlineto{\pgfqpoint{1.364186in}{1.174419in}}%
\pgfpathlineto{\pgfqpoint{1.364247in}{1.174449in}}%
\pgfpathlineto{\pgfqpoint{1.375980in}{1.178083in}}%
\pgfpathlineto{\pgfqpoint{1.387712in}{1.179419in}}%
\pgfpathlineto{\pgfqpoint{1.399444in}{1.178677in}}%
\pgfpathlineto{\pgfqpoint{1.409824in}{1.174419in}}%
\pgfpathlineto{\pgfqpoint{1.411176in}{1.173728in}}%
\pgfpathlineto{\pgfqpoint{1.422908in}{1.165998in}}%
\pgfpathlineto{\pgfqpoint{1.426872in}{1.163008in}}%
\pgfpathlineto{\pgfqpoint{1.434641in}{1.157656in}}%
\pgfpathlineto{\pgfqpoint{1.443000in}{1.151598in}}%
\pgfpathlineto{\pgfqpoint{1.434641in}{1.144371in}}%
\pgfpathlineto{\pgfqpoint{1.429897in}{1.140187in}}%
\pgfpathlineto{\pgfqpoint{1.422908in}{1.131726in}}%
\pgfpathlineto{\pgfqpoint{1.419124in}{1.128776in}}%
\pgfpathlineto{\pgfqpoint{1.422908in}{1.127174in}}%
\pgfpathlineto{\pgfqpoint{1.434641in}{1.126643in}}%
\pgfpathlineto{\pgfqpoint{1.446373in}{1.127618in}}%
\pgfpathlineto{\pgfqpoint{1.446373in}{1.128776in}}%
\pgfpathlineto{\pgfqpoint{1.446373in}{1.140187in}}%
\pgfpathlineto{\pgfqpoint{1.446373in}{1.151598in}}%
\pgfpathlineto{\pgfqpoint{1.446373in}{1.163008in}}%
\pgfpathlineto{\pgfqpoint{1.446373in}{1.174419in}}%
\pgfpathlineto{\pgfqpoint{1.446373in}{1.185045in}}%
\pgfpathlineto{\pgfqpoint{1.445477in}{1.185830in}}%
\pgfpathlineto{\pgfqpoint{1.434641in}{1.196276in}}%
\pgfpathlineto{\pgfqpoint{1.433387in}{1.197241in}}%
\pgfpathlineto{\pgfqpoint{1.422908in}{1.203690in}}%
\pgfpathlineto{\pgfqpoint{1.411176in}{1.208063in}}%
\pgfpathlineto{\pgfqpoint{1.408638in}{1.208651in}}%
\pgfpathlineto{\pgfqpoint{1.399444in}{1.211059in}}%
\pgfpathlineto{\pgfqpoint{1.387712in}{1.212045in}}%
\pgfpathlineto{\pgfqpoint{1.375980in}{1.210460in}}%
\pgfpathlineto{\pgfqpoint{1.371283in}{1.208651in}}%
\pgfpathlineto{\pgfqpoint{1.364247in}{1.206238in}}%
\pgfpathlineto{\pgfqpoint{1.352515in}{1.199471in}}%
\pgfpathlineto{\pgfqpoint{1.349711in}{1.197241in}}%
\pgfpathlineto{\pgfqpoint{1.340783in}{1.190830in}}%
\pgfpathlineto{\pgfqpoint{1.335081in}{1.185830in}}%
\pgfpathlineto{\pgfqpoint{1.329051in}{1.180984in}}%
\pgfpathlineto{\pgfqpoint{1.317319in}{1.175262in}}%
\pgfpathlineto{\pgfqpoint{1.305586in}{1.185820in}}%
\pgfpathlineto{\pgfqpoint{1.305579in}{1.185830in}}%
\pgfpathlineto{\pgfqpoint{1.295568in}{1.197241in}}%
\pgfpathlineto{\pgfqpoint{1.293854in}{1.198369in}}%
\pgfpathlineto{\pgfqpoint{1.282122in}{1.206511in}}%
\pgfpathlineto{\pgfqpoint{1.279599in}{1.208651in}}%
\pgfpathlineto{\pgfqpoint{1.270390in}{1.218506in}}%
\pgfpathlineto{\pgfqpoint{1.268905in}{1.220062in}}%
\pgfpathlineto{\pgfqpoint{1.259156in}{1.231473in}}%
\pgfpathlineto{\pgfqpoint{1.258658in}{1.232014in}}%
\pgfpathlineto{\pgfqpoint{1.248461in}{1.242884in}}%
\pgfpathlineto{\pgfqpoint{1.246925in}{1.244956in}}%
\pgfpathlineto{\pgfqpoint{1.240113in}{1.254295in}}%
\pgfpathlineto{\pgfqpoint{1.235193in}{1.261975in}}%
\pgfpathlineto{\pgfqpoint{1.232792in}{1.265705in}}%
\pgfpathlineto{\pgfqpoint{1.225888in}{1.277116in}}%
\pgfpathlineto{\pgfqpoint{1.223461in}{1.281322in}}%
\pgfpathlineto{\pgfqpoint{1.219049in}{1.288527in}}%
\pgfpathlineto{\pgfqpoint{1.212821in}{1.299938in}}%
\pgfpathlineto{\pgfqpoint{1.211729in}{1.301805in}}%
\pgfpathlineto{\pgfqpoint{1.207135in}{1.311348in}}%
\pgfpathlineto{\pgfqpoint{1.200810in}{1.322759in}}%
\pgfpathlineto{\pgfqpoint{1.204172in}{1.334170in}}%
\pgfpathlineto{\pgfqpoint{1.209657in}{1.345581in}}%
\pgfpathlineto{\pgfqpoint{1.211729in}{1.349454in}}%
\pgfpathlineto{\pgfqpoint{1.216047in}{1.356992in}}%
\pgfpathlineto{\pgfqpoint{1.223281in}{1.368402in}}%
\pgfpathlineto{\pgfqpoint{1.223461in}{1.368671in}}%
\pgfpathlineto{\pgfqpoint{1.231374in}{1.379813in}}%
\pgfpathlineto{\pgfqpoint{1.235193in}{1.384900in}}%
\pgfpathlineto{\pgfqpoint{1.240455in}{1.391224in}}%
\pgfpathlineto{\pgfqpoint{1.246925in}{1.397965in}}%
\pgfpathlineto{\pgfqpoint{1.258658in}{1.400368in}}%
\pgfpathlineto{\pgfqpoint{1.270390in}{1.399539in}}%
\pgfpathlineto{\pgfqpoint{1.282122in}{1.394659in}}%
\pgfpathlineto{\pgfqpoint{1.289909in}{1.391224in}}%
\pgfpathlineto{\pgfqpoint{1.293854in}{1.389581in}}%
\pgfpathlineto{\pgfqpoint{1.305586in}{1.384524in}}%
\pgfpathlineto{\pgfqpoint{1.316513in}{1.379813in}}%
\pgfpathlineto{\pgfqpoint{1.317319in}{1.379474in}}%
\pgfpathlineto{\pgfqpoint{1.324562in}{1.368402in}}%
\pgfpathlineto{\pgfqpoint{1.329051in}{1.360280in}}%
\pgfpathlineto{\pgfqpoint{1.331095in}{1.356992in}}%
\pgfpathlineto{\pgfqpoint{1.335982in}{1.345581in}}%
\pgfpathlineto{\pgfqpoint{1.340783in}{1.334416in}}%
\pgfpathlineto{\pgfqpoint{1.340920in}{1.334170in}}%
\pgfpathlineto{\pgfqpoint{1.349060in}{1.322759in}}%
\pgfpathlineto{\pgfqpoint{1.352515in}{1.319103in}}%
\pgfpathlineto{\pgfqpoint{1.363685in}{1.311348in}}%
\pgfpathlineto{\pgfqpoint{1.364247in}{1.310983in}}%
\pgfpathlineto{\pgfqpoint{1.375980in}{1.306900in}}%
\pgfpathlineto{\pgfqpoint{1.387712in}{1.305743in}}%
\pgfpathlineto{\pgfqpoint{1.399444in}{1.309789in}}%
\pgfpathlineto{\pgfqpoint{1.401461in}{1.311348in}}%
\pgfpathlineto{\pgfqpoint{1.411176in}{1.317778in}}%
\pgfpathlineto{\pgfqpoint{1.420703in}{1.322759in}}%
\pgfpathlineto{\pgfqpoint{1.422908in}{1.323638in}}%
\pgfpathlineto{\pgfqpoint{1.434641in}{1.328742in}}%
\pgfpathlineto{\pgfqpoint{1.446373in}{1.334083in}}%
\pgfpathlineto{\pgfqpoint{1.446373in}{1.334170in}}%
\pgfpathlineto{\pgfqpoint{1.446373in}{1.345581in}}%
\pgfpathlineto{\pgfqpoint{1.446373in}{1.348291in}}%
\pgfpathlineto{\pgfqpoint{1.440558in}{1.345581in}}%
\pgfpathlineto{\pgfqpoint{1.434641in}{1.342887in}}%
\pgfpathlineto{\pgfqpoint{1.422908in}{1.340000in}}%
\pgfpathlineto{\pgfqpoint{1.411176in}{1.339594in}}%
\pgfpathlineto{\pgfqpoint{1.399444in}{1.339767in}}%
\pgfpathlineto{\pgfqpoint{1.387712in}{1.341495in}}%
\pgfpathlineto{\pgfqpoint{1.375980in}{1.345195in}}%
\pgfpathlineto{\pgfqpoint{1.375287in}{1.345581in}}%
\pgfpathlineto{\pgfqpoint{1.364247in}{1.351564in}}%
\pgfpathlineto{\pgfqpoint{1.358152in}{1.356992in}}%
\pgfpathlineto{\pgfqpoint{1.352515in}{1.363085in}}%
\pgfpathlineto{\pgfqpoint{1.348428in}{1.368402in}}%
\pgfpathlineto{\pgfqpoint{1.340783in}{1.379251in}}%
\pgfpathlineto{\pgfqpoint{1.340418in}{1.379813in}}%
\pgfpathlineto{\pgfqpoint{1.333169in}{1.391224in}}%
\pgfpathlineto{\pgfqpoint{1.329051in}{1.398077in}}%
\pgfpathlineto{\pgfqpoint{1.326255in}{1.402635in}}%
\pgfpathlineto{\pgfqpoint{1.319388in}{1.414045in}}%
\pgfpathlineto{\pgfqpoint{1.317319in}{1.414045in}}%
\pgfpathlineto{\pgfqpoint{1.305586in}{1.414045in}}%
\pgfpathlineto{\pgfqpoint{1.293854in}{1.414045in}}%
\pgfpathlineto{\pgfqpoint{1.282122in}{1.414045in}}%
\pgfpathlineto{\pgfqpoint{1.270390in}{1.414045in}}%
\pgfpathlineto{\pgfqpoint{1.258658in}{1.414045in}}%
\pgfpathlineto{\pgfqpoint{1.246925in}{1.414045in}}%
\pgfpathlineto{\pgfqpoint{1.235193in}{1.414045in}}%
\pgfpathlineto{\pgfqpoint{1.223461in}{1.414045in}}%
\pgfpathlineto{\pgfqpoint{1.211729in}{1.414045in}}%
\pgfpathlineto{\pgfqpoint{1.199996in}{1.414045in}}%
\pgfpathlineto{\pgfqpoint{1.188264in}{1.414045in}}%
\pgfpathlineto{\pgfqpoint{1.176532in}{1.414045in}}%
\pgfpathlineto{\pgfqpoint{1.171191in}{1.414045in}}%
\pgfpathlineto{\pgfqpoint{1.164983in}{1.402635in}}%
\pgfpathlineto{\pgfqpoint{1.164800in}{1.402270in}}%
\pgfpathlineto{\pgfqpoint{1.159545in}{1.391224in}}%
\pgfpathlineto{\pgfqpoint{1.160143in}{1.379813in}}%
\pgfpathlineto{\pgfqpoint{1.161896in}{1.368402in}}%
\pgfpathlineto{\pgfqpoint{1.164800in}{1.363561in}}%
\pgfpathlineto{\pgfqpoint{1.168621in}{1.356992in}}%
\pgfpathlineto{\pgfqpoint{1.173393in}{1.345581in}}%
\pgfpathlineto{\pgfqpoint{1.176532in}{1.340966in}}%
\pgfpathlineto{\pgfqpoint{1.180031in}{1.334170in}}%
\pgfpathlineto{\pgfqpoint{1.185368in}{1.322759in}}%
\pgfpathlineto{\pgfqpoint{1.188264in}{1.318932in}}%
\pgfpathlineto{\pgfqpoint{1.192253in}{1.311348in}}%
\pgfpathlineto{\pgfqpoint{1.199739in}{1.299938in}}%
\pgfpathlineto{\pgfqpoint{1.199996in}{1.299633in}}%
\pgfpathlineto{\pgfqpoint{1.206406in}{1.288527in}}%
\pgfpathlineto{\pgfqpoint{1.211729in}{1.281485in}}%
\pgfpathlineto{\pgfqpoint{1.214443in}{1.277116in}}%
\pgfpathlineto{\pgfqpoint{1.221572in}{1.265705in}}%
\pgfpathlineto{\pgfqpoint{1.223461in}{1.263561in}}%
\pgfpathlineto{\pgfqpoint{1.229512in}{1.254295in}}%
\pgfpathlineto{\pgfqpoint{1.235193in}{1.246043in}}%
\pgfpathlineto{\pgfqpoint{1.237472in}{1.242884in}}%
\pgfpathlineto{\pgfqpoint{1.246925in}{1.233136in}}%
\pgfpathlineto{\pgfqpoint{1.248489in}{1.231473in}}%
\pgfpathlineto{\pgfqpoint{1.258658in}{1.221275in}}%
\pgfpathlineto{\pgfqpoint{1.259701in}{1.220062in}}%
\pgfpathlineto{\pgfqpoint{1.270390in}{1.208858in}}%
\pgfpathlineto{\pgfqpoint{1.270583in}{1.208651in}}%
\pgfpathlineto{\pgfqpoint{1.282122in}{1.198863in}}%
\pgfpathlineto{\pgfqpoint{1.284281in}{1.197241in}}%
\pgfpathlineto{\pgfqpoint{1.293854in}{1.187850in}}%
\pgfpathlineto{\pgfqpoint{1.295827in}{1.185830in}}%
\pgfpathlineto{\pgfqpoint{1.304691in}{1.174419in}}%
\pgfpathlineto{\pgfqpoint{1.305586in}{1.173045in}}%
\pgfpathlineto{\pgfqpoint{1.313329in}{1.163008in}}%
\pgfpathlineto{\pgfqpoint{1.317319in}{1.156166in}}%
\pgfpathlineto{\pgfqpoint{1.321506in}{1.151598in}}%
\pgfpathlineto{\pgfqpoint{1.326943in}{1.140187in}}%
\pgfpathlineto{\pgfqpoint{1.326845in}{1.128776in}}%
\pgfpathlineto{\pgfqpoint{1.325800in}{1.117365in}}%
\pgfpathlineto{\pgfqpoint{1.317319in}{1.110839in}}%
\pgfpathlineto{\pgfqpoint{1.305586in}{1.107534in}}%
\pgfpathlineto{\pgfqpoint{1.298611in}{1.105954in}}%
\pgfpathlineto{\pgfqpoint{1.293854in}{1.104835in}}%
\pgfpathlineto{\pgfqpoint{1.289058in}{1.105954in}}%
\pgfpathlineto{\pgfqpoint{1.282122in}{1.107538in}}%
\pgfpathlineto{\pgfqpoint{1.270390in}{1.110132in}}%
\pgfpathlineto{\pgfqpoint{1.258658in}{1.112600in}}%
\pgfpathlineto{\pgfqpoint{1.246925in}{1.116159in}}%
\pgfpathlineto{\pgfqpoint{1.235193in}{1.117290in}}%
\pgfpathlineto{\pgfqpoint{1.234803in}{1.117365in}}%
\pgfpathlineto{\pgfqpoint{1.223461in}{1.122305in}}%
\pgfpathlineto{\pgfqpoint{1.211729in}{1.125411in}}%
\pgfpathlineto{\pgfqpoint{1.199996in}{1.127372in}}%
\pgfpathlineto{\pgfqpoint{1.188264in}{1.127610in}}%
\pgfpathlineto{\pgfqpoint{1.176532in}{1.122558in}}%
\pgfpathlineto{\pgfqpoint{1.172797in}{1.117365in}}%
\pgfpathlineto{\pgfqpoint{1.164800in}{1.115999in}}%
\pgfpathlineto{\pgfqpoint{1.153068in}{1.110528in}}%
\pgfpathlineto{\pgfqpoint{1.150521in}{1.105954in}}%
\pgfpathlineto{\pgfqpoint{1.142920in}{1.094544in}}%
\pgfpathlineto{\pgfqpoint{1.142237in}{1.083133in}}%
\pgfpathlineto{\pgfqpoint{1.151471in}{1.071722in}}%
\pgfpathlineto{\pgfqpoint{1.153068in}{1.068942in}}%
\pgfpathlineto{\pgfqpoint{1.158994in}{1.060311in}}%
\pgfpathlineto{\pgfqpoint{1.164800in}{1.052428in}}%
\pgfpathlineto{\pgfqpoint{1.168692in}{1.048901in}}%
\pgfpathlineto{\pgfqpoint{1.176532in}{1.037899in}}%
\pgfpathlineto{\pgfqpoint{1.177077in}{1.037490in}}%
\pgfpathlineto{\pgfqpoint{1.186958in}{1.026079in}}%
\pgfpathlineto{\pgfqpoint{1.188264in}{1.024520in}}%
\pgfpathlineto{\pgfqpoint{1.196298in}{1.014668in}}%
\pgfpathlineto{\pgfqpoint{1.199996in}{1.010733in}}%
\pgfpathlineto{\pgfqpoint{1.209652in}{1.003257in}}%
\pgfpathlineto{\pgfqpoint{1.211729in}{1.001382in}}%
\pgfpathlineto{\pgfqpoint{1.223461in}{0.992500in}}%
\pgfpathlineto{\pgfqpoint{1.224716in}{0.991847in}}%
\pgfpathlineto{\pgfqpoint{1.235193in}{0.984451in}}%
\pgfpathlineto{\pgfqpoint{1.245535in}{0.980436in}}%
\pgfpathclose%
\pgfpathmoveto{\pgfqpoint{1.291599in}{0.991847in}}%
\pgfpathlineto{\pgfqpoint{1.282122in}{0.993699in}}%
\pgfpathlineto{\pgfqpoint{1.270390in}{0.999001in}}%
\pgfpathlineto{\pgfqpoint{1.260813in}{1.003257in}}%
\pgfpathlineto{\pgfqpoint{1.258658in}{1.003863in}}%
\pgfpathlineto{\pgfqpoint{1.246925in}{1.008578in}}%
\pgfpathlineto{\pgfqpoint{1.236158in}{1.014668in}}%
\pgfpathlineto{\pgfqpoint{1.235193in}{1.015029in}}%
\pgfpathlineto{\pgfqpoint{1.223461in}{1.019334in}}%
\pgfpathlineto{\pgfqpoint{1.215179in}{1.026079in}}%
\pgfpathlineto{\pgfqpoint{1.211729in}{1.028185in}}%
\pgfpathlineto{\pgfqpoint{1.200544in}{1.037490in}}%
\pgfpathlineto{\pgfqpoint{1.199996in}{1.037891in}}%
\pgfpathlineto{\pgfqpoint{1.188264in}{1.046262in}}%
\pgfpathlineto{\pgfqpoint{1.185205in}{1.048901in}}%
\pgfpathlineto{\pgfqpoint{1.176532in}{1.055518in}}%
\pgfpathlineto{\pgfqpoint{1.171364in}{1.060311in}}%
\pgfpathlineto{\pgfqpoint{1.164800in}{1.064010in}}%
\pgfpathlineto{\pgfqpoint{1.158145in}{1.071722in}}%
\pgfpathlineto{\pgfqpoint{1.153068in}{1.076890in}}%
\pgfpathlineto{\pgfqpoint{1.147593in}{1.083133in}}%
\pgfpathlineto{\pgfqpoint{1.148561in}{1.094544in}}%
\pgfpathlineto{\pgfqpoint{1.153068in}{1.100902in}}%
\pgfpathlineto{\pgfqpoint{1.158348in}{1.105954in}}%
\pgfpathlineto{\pgfqpoint{1.164800in}{1.110342in}}%
\pgfpathlineto{\pgfqpoint{1.176532in}{1.113726in}}%
\pgfpathlineto{\pgfqpoint{1.187995in}{1.117365in}}%
\pgfpathlineto{\pgfqpoint{1.188264in}{1.117574in}}%
\pgfpathlineto{\pgfqpoint{1.199996in}{1.118662in}}%
\pgfpathlineto{\pgfqpoint{1.204481in}{1.117365in}}%
\pgfpathlineto{\pgfqpoint{1.211729in}{1.116278in}}%
\pgfpathlineto{\pgfqpoint{1.223461in}{1.114579in}}%
\pgfpathlineto{\pgfqpoint{1.235193in}{1.111991in}}%
\pgfpathlineto{\pgfqpoint{1.246925in}{1.107591in}}%
\pgfpathlineto{\pgfqpoint{1.249749in}{1.105954in}}%
\pgfpathlineto{\pgfqpoint{1.258658in}{1.102858in}}%
\pgfpathlineto{\pgfqpoint{1.270390in}{1.099407in}}%
\pgfpathlineto{\pgfqpoint{1.278799in}{1.094544in}}%
\pgfpathlineto{\pgfqpoint{1.282122in}{1.093338in}}%
\pgfpathlineto{\pgfqpoint{1.293854in}{1.088766in}}%
\pgfpathlineto{\pgfqpoint{1.305586in}{1.089736in}}%
\pgfpathlineto{\pgfqpoint{1.317319in}{1.094026in}}%
\pgfpathlineto{\pgfqpoint{1.329051in}{1.090479in}}%
\pgfpathlineto{\pgfqpoint{1.340783in}{1.093149in}}%
\pgfpathlineto{\pgfqpoint{1.341228in}{1.094544in}}%
\pgfpathlineto{\pgfqpoint{1.340783in}{1.097648in}}%
\pgfpathlineto{\pgfqpoint{1.339873in}{1.105954in}}%
\pgfpathlineto{\pgfqpoint{1.340244in}{1.117365in}}%
\pgfpathlineto{\pgfqpoint{1.340783in}{1.123349in}}%
\pgfpathlineto{\pgfqpoint{1.342515in}{1.128776in}}%
\pgfpathlineto{\pgfqpoint{1.352515in}{1.130866in}}%
\pgfpathlineto{\pgfqpoint{1.357877in}{1.128776in}}%
\pgfpathlineto{\pgfqpoint{1.364247in}{1.125924in}}%
\pgfpathlineto{\pgfqpoint{1.375980in}{1.121638in}}%
\pgfpathlineto{\pgfqpoint{1.387712in}{1.117567in}}%
\pgfpathlineto{\pgfqpoint{1.388253in}{1.117365in}}%
\pgfpathlineto{\pgfqpoint{1.399444in}{1.108354in}}%
\pgfpathlineto{\pgfqpoint{1.400497in}{1.105954in}}%
\pgfpathlineto{\pgfqpoint{1.400631in}{1.094544in}}%
\pgfpathlineto{\pgfqpoint{1.399444in}{1.089596in}}%
\pgfpathlineto{\pgfqpoint{1.397640in}{1.083133in}}%
\pgfpathlineto{\pgfqpoint{1.387712in}{1.071888in}}%
\pgfpathlineto{\pgfqpoint{1.387353in}{1.071722in}}%
\pgfpathlineto{\pgfqpoint{1.375980in}{1.067570in}}%
\pgfpathlineto{\pgfqpoint{1.364247in}{1.065142in}}%
\pgfpathlineto{\pgfqpoint{1.357434in}{1.060311in}}%
\pgfpathlineto{\pgfqpoint{1.352515in}{1.055940in}}%
\pgfpathlineto{\pgfqpoint{1.340783in}{1.051259in}}%
\pgfpathlineto{\pgfqpoint{1.338671in}{1.048901in}}%
\pgfpathlineto{\pgfqpoint{1.332561in}{1.037490in}}%
\pgfpathlineto{\pgfqpoint{1.333353in}{1.026079in}}%
\pgfpathlineto{\pgfqpoint{1.331344in}{1.014668in}}%
\pgfpathlineto{\pgfqpoint{1.329051in}{1.009921in}}%
\pgfpathlineto{\pgfqpoint{1.325372in}{1.003257in}}%
\pgfpathlineto{\pgfqpoint{1.318008in}{0.991847in}}%
\pgfpathlineto{\pgfqpoint{1.317319in}{0.991243in}}%
\pgfpathlineto{\pgfqpoint{1.305586in}{0.987335in}}%
\pgfpathlineto{\pgfqpoint{1.293854in}{0.991301in}}%
\pgfpathclose%
\pgfusepath{fill}%
\end{pgfscope}%
\begin{pgfscope}%
\pgfpathrectangle{\pgfqpoint{0.211875in}{0.211875in}}{\pgfqpoint{1.313625in}{1.279725in}}%
\pgfusepath{clip}%
\pgfsetbuttcap%
\pgfsetroundjoin%
\definecolor{currentfill}{rgb}{0.848131,0.150999,0.281943}%
\pgfsetfillcolor{currentfill}%
\pgfsetlinewidth{0.000000pt}%
\definecolor{currentstroke}{rgb}{0.000000,0.000000,0.000000}%
\pgfsetstrokecolor{currentstroke}%
\pgfsetdash{}{0pt}%
\pgfpathmoveto{\pgfqpoint{1.446373in}{0.974953in}}%
\pgfpathlineto{\pgfqpoint{1.446373in}{0.980436in}}%
\pgfpathlineto{\pgfqpoint{1.446373in}{0.983143in}}%
\pgfpathlineto{\pgfqpoint{1.434641in}{0.988688in}}%
\pgfpathlineto{\pgfqpoint{1.429208in}{0.991847in}}%
\pgfpathlineto{\pgfqpoint{1.422908in}{0.997470in}}%
\pgfpathlineto{\pgfqpoint{1.418603in}{1.003257in}}%
\pgfpathlineto{\pgfqpoint{1.415317in}{1.014668in}}%
\pgfpathlineto{\pgfqpoint{1.414466in}{1.026079in}}%
\pgfpathlineto{\pgfqpoint{1.415352in}{1.037490in}}%
\pgfpathlineto{\pgfqpoint{1.417548in}{1.048901in}}%
\pgfpathlineto{\pgfqpoint{1.422908in}{1.057856in}}%
\pgfpathlineto{\pgfqpoint{1.425481in}{1.060311in}}%
\pgfpathlineto{\pgfqpoint{1.434641in}{1.068023in}}%
\pgfpathlineto{\pgfqpoint{1.439284in}{1.071722in}}%
\pgfpathlineto{\pgfqpoint{1.446373in}{1.077540in}}%
\pgfpathlineto{\pgfqpoint{1.446373in}{1.083133in}}%
\pgfpathlineto{\pgfqpoint{1.446373in}{1.083742in}}%
\pgfpathlineto{\pgfqpoint{1.445678in}{1.083133in}}%
\pgfpathlineto{\pgfqpoint{1.434641in}{1.073745in}}%
\pgfpathlineto{\pgfqpoint{1.432339in}{1.071722in}}%
\pgfpathlineto{\pgfqpoint{1.422908in}{1.063556in}}%
\pgfpathlineto{\pgfqpoint{1.418911in}{1.060311in}}%
\pgfpathlineto{\pgfqpoint{1.413380in}{1.048901in}}%
\pgfpathlineto{\pgfqpoint{1.411176in}{1.040089in}}%
\pgfpathlineto{\pgfqpoint{1.410394in}{1.037490in}}%
\pgfpathlineto{\pgfqpoint{1.409019in}{1.026079in}}%
\pgfpathlineto{\pgfqpoint{1.408722in}{1.014668in}}%
\pgfpathlineto{\pgfqpoint{1.409228in}{1.003257in}}%
\pgfpathlineto{\pgfqpoint{1.411176in}{0.999162in}}%
\pgfpathlineto{\pgfqpoint{1.415618in}{0.991847in}}%
\pgfpathlineto{\pgfqpoint{1.422908in}{0.987365in}}%
\pgfpathlineto{\pgfqpoint{1.434641in}{0.980711in}}%
\pgfpathlineto{\pgfqpoint{1.435231in}{0.980436in}}%
\pgfpathclose%
\pgfusepath{fill}%
\end{pgfscope}%
\begin{pgfscope}%
\pgfpathrectangle{\pgfqpoint{0.211875in}{0.211875in}}{\pgfqpoint{1.313625in}{1.279725in}}%
\pgfusepath{clip}%
\pgfsetbuttcap%
\pgfsetroundjoin%
\definecolor{currentfill}{rgb}{0.848131,0.150999,0.281943}%
\pgfsetfillcolor{currentfill}%
\pgfsetlinewidth{0.000000pt}%
\definecolor{currentstroke}{rgb}{0.000000,0.000000,0.000000}%
\pgfsetstrokecolor{currentstroke}%
\pgfsetdash{}{0pt}%
\pgfpathmoveto{\pgfqpoint{0.801101in}{1.253497in}}%
\pgfpathlineto{\pgfqpoint{0.801717in}{1.254295in}}%
\pgfpathlineto{\pgfqpoint{0.801101in}{1.255981in}}%
\pgfpathlineto{\pgfqpoint{0.797957in}{1.265705in}}%
\pgfpathlineto{\pgfqpoint{0.794323in}{1.277116in}}%
\pgfpathlineto{\pgfqpoint{0.790733in}{1.288527in}}%
\pgfpathlineto{\pgfqpoint{0.789369in}{1.293440in}}%
\pgfpathlineto{\pgfqpoint{0.787581in}{1.299938in}}%
\pgfpathlineto{\pgfqpoint{0.784037in}{1.311348in}}%
\pgfpathlineto{\pgfqpoint{0.777637in}{1.319468in}}%
\pgfpathlineto{\pgfqpoint{0.775020in}{1.322759in}}%
\pgfpathlineto{\pgfqpoint{0.766021in}{1.334170in}}%
\pgfpathlineto{\pgfqpoint{0.765905in}{1.334333in}}%
\pgfpathlineto{\pgfqpoint{0.757758in}{1.345581in}}%
\pgfpathlineto{\pgfqpoint{0.754173in}{1.350862in}}%
\pgfpathlineto{\pgfqpoint{0.749829in}{1.356992in}}%
\pgfpathlineto{\pgfqpoint{0.742440in}{1.368221in}}%
\pgfpathlineto{\pgfqpoint{0.742316in}{1.368402in}}%
\pgfpathlineto{\pgfqpoint{0.736399in}{1.379813in}}%
\pgfpathlineto{\pgfqpoint{0.731766in}{1.391224in}}%
\pgfpathlineto{\pgfqpoint{0.731846in}{1.402635in}}%
\pgfpathlineto{\pgfqpoint{0.742440in}{1.410087in}}%
\pgfpathlineto{\pgfqpoint{0.754173in}{1.413853in}}%
\pgfpathlineto{\pgfqpoint{0.755310in}{1.414045in}}%
\pgfpathlineto{\pgfqpoint{0.754173in}{1.414045in}}%
\pgfpathlineto{\pgfqpoint{0.742440in}{1.414045in}}%
\pgfpathlineto{\pgfqpoint{0.730708in}{1.414045in}}%
\pgfpathlineto{\pgfqpoint{0.718976in}{1.414045in}}%
\pgfpathlineto{\pgfqpoint{0.707244in}{1.414045in}}%
\pgfpathlineto{\pgfqpoint{0.695512in}{1.414045in}}%
\pgfpathlineto{\pgfqpoint{0.683779in}{1.414045in}}%
\pgfpathlineto{\pgfqpoint{0.674652in}{1.414045in}}%
\pgfpathlineto{\pgfqpoint{0.680833in}{1.402635in}}%
\pgfpathlineto{\pgfqpoint{0.683779in}{1.395690in}}%
\pgfpathlineto{\pgfqpoint{0.685531in}{1.391224in}}%
\pgfpathlineto{\pgfqpoint{0.688830in}{1.379813in}}%
\pgfpathlineto{\pgfqpoint{0.691491in}{1.368402in}}%
\pgfpathlineto{\pgfqpoint{0.694704in}{1.356992in}}%
\pgfpathlineto{\pgfqpoint{0.695512in}{1.355980in}}%
\pgfpathlineto{\pgfqpoint{0.702345in}{1.345581in}}%
\pgfpathlineto{\pgfqpoint{0.707244in}{1.339270in}}%
\pgfpathlineto{\pgfqpoint{0.710731in}{1.334170in}}%
\pgfpathlineto{\pgfqpoint{0.718976in}{1.323114in}}%
\pgfpathlineto{\pgfqpoint{0.719232in}{1.322759in}}%
\pgfpathlineto{\pgfqpoint{0.727157in}{1.311348in}}%
\pgfpathlineto{\pgfqpoint{0.730708in}{1.306317in}}%
\pgfpathlineto{\pgfqpoint{0.735079in}{1.299938in}}%
\pgfpathlineto{\pgfqpoint{0.742440in}{1.289364in}}%
\pgfpathlineto{\pgfqpoint{0.743042in}{1.288527in}}%
\pgfpathlineto{\pgfqpoint{0.750349in}{1.277116in}}%
\pgfpathlineto{\pgfqpoint{0.754173in}{1.271501in}}%
\pgfpathlineto{\pgfqpoint{0.762536in}{1.265705in}}%
\pgfpathlineto{\pgfqpoint{0.765905in}{1.261821in}}%
\pgfpathlineto{\pgfqpoint{0.777637in}{1.256392in}}%
\pgfpathlineto{\pgfqpoint{0.789369in}{1.256363in}}%
\pgfpathlineto{\pgfqpoint{0.798487in}{1.254295in}}%
\pgfpathclose%
\pgfpathmoveto{\pgfqpoint{0.761407in}{1.277116in}}%
\pgfpathlineto{\pgfqpoint{0.754173in}{1.281755in}}%
\pgfpathlineto{\pgfqpoint{0.749179in}{1.288527in}}%
\pgfpathlineto{\pgfqpoint{0.742440in}{1.297914in}}%
\pgfpathlineto{\pgfqpoint{0.741032in}{1.299938in}}%
\pgfpathlineto{\pgfqpoint{0.733227in}{1.311348in}}%
\pgfpathlineto{\pgfqpoint{0.730708in}{1.315052in}}%
\pgfpathlineto{\pgfqpoint{0.725349in}{1.322759in}}%
\pgfpathlineto{\pgfqpoint{0.718976in}{1.331595in}}%
\pgfpathlineto{\pgfqpoint{0.717056in}{1.334170in}}%
\pgfpathlineto{\pgfqpoint{0.712716in}{1.345581in}}%
\pgfpathlineto{\pgfqpoint{0.712611in}{1.356992in}}%
\pgfpathlineto{\pgfqpoint{0.715985in}{1.368402in}}%
\pgfpathlineto{\pgfqpoint{0.718976in}{1.370677in}}%
\pgfpathlineto{\pgfqpoint{0.720663in}{1.368402in}}%
\pgfpathlineto{\pgfqpoint{0.729025in}{1.356992in}}%
\pgfpathlineto{\pgfqpoint{0.730708in}{1.356020in}}%
\pgfpathlineto{\pgfqpoint{0.739521in}{1.345581in}}%
\pgfpathlineto{\pgfqpoint{0.742440in}{1.342240in}}%
\pgfpathlineto{\pgfqpoint{0.749300in}{1.334170in}}%
\pgfpathlineto{\pgfqpoint{0.754173in}{1.328591in}}%
\pgfpathlineto{\pgfqpoint{0.759141in}{1.322759in}}%
\pgfpathlineto{\pgfqpoint{0.765905in}{1.312619in}}%
\pgfpathlineto{\pgfqpoint{0.766910in}{1.311348in}}%
\pgfpathlineto{\pgfqpoint{0.771145in}{1.299938in}}%
\pgfpathlineto{\pgfqpoint{0.775342in}{1.288527in}}%
\pgfpathlineto{\pgfqpoint{0.777637in}{1.282388in}}%
\pgfpathlineto{\pgfqpoint{0.779621in}{1.277116in}}%
\pgfpathlineto{\pgfqpoint{0.777637in}{1.273899in}}%
\pgfpathlineto{\pgfqpoint{0.765905in}{1.273715in}}%
\pgfpathclose%
\pgfusepath{fill}%
\end{pgfscope}%
\begin{pgfscope}%
\pgfpathrectangle{\pgfqpoint{0.211875in}{0.211875in}}{\pgfqpoint{1.313625in}{1.279725in}}%
\pgfusepath{clip}%
\pgfsetbuttcap%
\pgfsetroundjoin%
\definecolor{currentfill}{rgb}{0.848131,0.150999,0.281943}%
\pgfsetfillcolor{currentfill}%
\pgfsetlinewidth{0.000000pt}%
\definecolor{currentstroke}{rgb}{0.000000,0.000000,0.000000}%
\pgfsetstrokecolor{currentstroke}%
\pgfsetdash{}{0pt}%
\pgfpathmoveto{\pgfqpoint{0.296617in}{1.263626in}}%
\pgfpathlineto{\pgfqpoint{0.308349in}{1.261512in}}%
\pgfpathlineto{\pgfqpoint{0.320081in}{1.259675in}}%
\pgfpathlineto{\pgfqpoint{0.331813in}{1.258422in}}%
\pgfpathlineto{\pgfqpoint{0.343545in}{1.259503in}}%
\pgfpathlineto{\pgfqpoint{0.351189in}{1.265705in}}%
\pgfpathlineto{\pgfqpoint{0.347929in}{1.277116in}}%
\pgfpathlineto{\pgfqpoint{0.343545in}{1.283229in}}%
\pgfpathlineto{\pgfqpoint{0.340280in}{1.288527in}}%
\pgfpathlineto{\pgfqpoint{0.331813in}{1.298557in}}%
\pgfpathlineto{\pgfqpoint{0.330795in}{1.299938in}}%
\pgfpathlineto{\pgfqpoint{0.320081in}{1.311296in}}%
\pgfpathlineto{\pgfqpoint{0.320037in}{1.311348in}}%
\pgfpathlineto{\pgfqpoint{0.308381in}{1.322759in}}%
\pgfpathlineto{\pgfqpoint{0.308349in}{1.322787in}}%
\pgfpathlineto{\pgfqpoint{0.296617in}{1.333649in}}%
\pgfpathlineto{\pgfqpoint{0.296104in}{1.334170in}}%
\pgfpathlineto{\pgfqpoint{0.284884in}{1.344111in}}%
\pgfpathlineto{\pgfqpoint{0.284884in}{1.334170in}}%
\pgfpathlineto{\pgfqpoint{0.284884in}{1.322759in}}%
\pgfpathlineto{\pgfqpoint{0.284884in}{1.311348in}}%
\pgfpathlineto{\pgfqpoint{0.284884in}{1.299938in}}%
\pgfpathlineto{\pgfqpoint{0.284884in}{1.288527in}}%
\pgfpathlineto{\pgfqpoint{0.284884in}{1.277116in}}%
\pgfpathlineto{\pgfqpoint{0.284884in}{1.265959in}}%
\pgfpathlineto{\pgfqpoint{0.286056in}{1.265705in}}%
\pgfpathclose%
\pgfusepath{fill}%
\end{pgfscope}%
\begin{pgfscope}%
\pgfpathrectangle{\pgfqpoint{0.211875in}{0.211875in}}{\pgfqpoint{1.313625in}{1.279725in}}%
\pgfusepath{clip}%
\pgfsetbuttcap%
\pgfsetroundjoin%
\definecolor{currentfill}{rgb}{0.848131,0.150999,0.281943}%
\pgfsetfillcolor{currentfill}%
\pgfsetlinewidth{0.000000pt}%
\definecolor{currentstroke}{rgb}{0.000000,0.000000,0.000000}%
\pgfsetstrokecolor{currentstroke}%
\pgfsetdash{}{0pt}%
\pgfpathmoveto{\pgfqpoint{0.589922in}{1.412720in}}%
\pgfpathlineto{\pgfqpoint{0.601654in}{1.408081in}}%
\pgfpathlineto{\pgfqpoint{0.604783in}{1.414045in}}%
\pgfpathlineto{\pgfqpoint{0.601654in}{1.414045in}}%
\pgfpathlineto{\pgfqpoint{0.589922in}{1.414045in}}%
\pgfpathlineto{\pgfqpoint{0.587929in}{1.414045in}}%
\pgfpathclose%
\pgfusepath{fill}%
\end{pgfscope}%
\begin{pgfscope}%
\pgfpathrectangle{\pgfqpoint{0.211875in}{0.211875in}}{\pgfqpoint{1.313625in}{1.279725in}}%
\pgfusepath{clip}%
\pgfsetbuttcap%
\pgfsetroundjoin%
\definecolor{currentfill}{rgb}{0.848131,0.150999,0.281943}%
\pgfsetfillcolor{currentfill}%
\pgfsetlinewidth{0.000000pt}%
\definecolor{currentstroke}{rgb}{0.000000,0.000000,0.000000}%
\pgfsetstrokecolor{currentstroke}%
\pgfsetdash{}{0pt}%
\pgfpathmoveto{\pgfqpoint{0.836298in}{1.413742in}}%
\pgfpathlineto{\pgfqpoint{0.848030in}{1.412667in}}%
\pgfpathlineto{\pgfqpoint{0.859762in}{1.411990in}}%
\pgfpathlineto{\pgfqpoint{0.866303in}{1.414045in}}%
\pgfpathlineto{\pgfqpoint{0.859762in}{1.414045in}}%
\pgfpathlineto{\pgfqpoint{0.848030in}{1.414045in}}%
\pgfpathlineto{\pgfqpoint{0.836298in}{1.414045in}}%
\pgfpathlineto{\pgfqpoint{0.832473in}{1.414045in}}%
\pgfpathclose%
\pgfusepath{fill}%
\end{pgfscope}%
\begin{pgfscope}%
\pgfpathrectangle{\pgfqpoint{0.211875in}{0.211875in}}{\pgfqpoint{1.313625in}{1.279725in}}%
\pgfusepath{clip}%
\pgfsetbuttcap%
\pgfsetroundjoin%
\definecolor{currentfill}{rgb}{0.924566,0.290534,0.242426}%
\pgfsetfillcolor{currentfill}%
\pgfsetlinewidth{0.000000pt}%
\definecolor{currentstroke}{rgb}{0.000000,0.000000,0.000000}%
\pgfsetstrokecolor{currentstroke}%
\pgfsetdash{}{0pt}%
\pgfpathmoveto{\pgfqpoint{1.059210in}{0.289022in}}%
\pgfpathlineto{\pgfqpoint{1.062152in}{0.284378in}}%
\pgfpathlineto{\pgfqpoint{1.070942in}{0.284378in}}%
\pgfpathlineto{\pgfqpoint{1.082674in}{0.284378in}}%
\pgfpathlineto{\pgfqpoint{1.091348in}{0.284378in}}%
\pgfpathlineto{\pgfqpoint{1.084110in}{0.295789in}}%
\pgfpathlineto{\pgfqpoint{1.082674in}{0.297939in}}%
\pgfpathlineto{\pgfqpoint{1.077025in}{0.307200in}}%
\pgfpathlineto{\pgfqpoint{1.070942in}{0.316639in}}%
\pgfpathlineto{\pgfqpoint{1.069725in}{0.318611in}}%
\pgfpathlineto{\pgfqpoint{1.062634in}{0.330022in}}%
\pgfpathlineto{\pgfqpoint{1.059210in}{0.335387in}}%
\pgfpathlineto{\pgfqpoint{1.055716in}{0.341432in}}%
\pgfpathlineto{\pgfqpoint{1.049005in}{0.352843in}}%
\pgfpathlineto{\pgfqpoint{1.047478in}{0.355410in}}%
\pgfpathlineto{\pgfqpoint{1.042311in}{0.364254in}}%
\pgfpathlineto{\pgfqpoint{1.035746in}{0.375254in}}%
\pgfpathlineto{\pgfqpoint{1.035500in}{0.375665in}}%
\pgfpathlineto{\pgfqpoint{1.028702in}{0.387075in}}%
\pgfpathlineto{\pgfqpoint{1.024013in}{0.394853in}}%
\pgfpathlineto{\pgfqpoint{1.022653in}{0.398486in}}%
\pgfpathlineto{\pgfqpoint{1.017779in}{0.409897in}}%
\pgfpathlineto{\pgfqpoint{1.012281in}{0.420260in}}%
\pgfpathlineto{\pgfqpoint{1.011878in}{0.421308in}}%
\pgfpathlineto{\pgfqpoint{1.007523in}{0.432719in}}%
\pgfpathlineto{\pgfqpoint{1.003299in}{0.444129in}}%
\pgfpathlineto{\pgfqpoint{1.000549in}{0.451587in}}%
\pgfpathlineto{\pgfqpoint{0.999003in}{0.455540in}}%
\pgfpathlineto{\pgfqpoint{0.994559in}{0.466951in}}%
\pgfpathlineto{\pgfqpoint{0.990275in}{0.478362in}}%
\pgfpathlineto{\pgfqpoint{0.988817in}{0.482282in}}%
\pgfpathlineto{\pgfqpoint{0.985841in}{0.489772in}}%
\pgfpathlineto{\pgfqpoint{0.982360in}{0.501183in}}%
\pgfpathlineto{\pgfqpoint{0.977171in}{0.512594in}}%
\pgfpathlineto{\pgfqpoint{0.977085in}{0.516503in}}%
\pgfpathlineto{\pgfqpoint{0.976946in}{0.524005in}}%
\pgfpathlineto{\pgfqpoint{0.976635in}{0.535416in}}%
\pgfpathlineto{\pgfqpoint{0.976572in}{0.546826in}}%
\pgfpathlineto{\pgfqpoint{0.976843in}{0.558237in}}%
\pgfpathlineto{\pgfqpoint{0.977085in}{0.561262in}}%
\pgfpathlineto{\pgfqpoint{0.978022in}{0.569648in}}%
\pgfpathlineto{\pgfqpoint{0.984544in}{0.581059in}}%
\pgfpathlineto{\pgfqpoint{0.988817in}{0.582254in}}%
\pgfpathlineto{\pgfqpoint{0.989184in}{0.581059in}}%
\pgfpathlineto{\pgfqpoint{0.993105in}{0.569648in}}%
\pgfpathlineto{\pgfqpoint{1.000549in}{0.559888in}}%
\pgfpathlineto{\pgfqpoint{1.001173in}{0.558237in}}%
\pgfpathlineto{\pgfqpoint{1.005671in}{0.546826in}}%
\pgfpathlineto{\pgfqpoint{1.012281in}{0.535481in}}%
\pgfpathlineto{\pgfqpoint{1.012307in}{0.535416in}}%
\pgfpathlineto{\pgfqpoint{1.016663in}{0.524005in}}%
\pgfpathlineto{\pgfqpoint{1.023021in}{0.512594in}}%
\pgfpathlineto{\pgfqpoint{1.024013in}{0.510984in}}%
\pgfpathlineto{\pgfqpoint{1.027796in}{0.501183in}}%
\pgfpathlineto{\pgfqpoint{1.033786in}{0.489772in}}%
\pgfpathlineto{\pgfqpoint{1.035746in}{0.486548in}}%
\pgfpathlineto{\pgfqpoint{1.038946in}{0.478362in}}%
\pgfpathlineto{\pgfqpoint{1.044655in}{0.466951in}}%
\pgfpathlineto{\pgfqpoint{1.047478in}{0.462275in}}%
\pgfpathlineto{\pgfqpoint{1.050150in}{0.455540in}}%
\pgfpathlineto{\pgfqpoint{1.055611in}{0.444129in}}%
\pgfpathlineto{\pgfqpoint{1.059210in}{0.438112in}}%
\pgfpathlineto{\pgfqpoint{1.061372in}{0.432719in}}%
\pgfpathlineto{\pgfqpoint{1.066558in}{0.421308in}}%
\pgfpathlineto{\pgfqpoint{1.070942in}{0.413765in}}%
\pgfpathlineto{\pgfqpoint{1.072498in}{0.409897in}}%
\pgfpathlineto{\pgfqpoint{1.077404in}{0.398486in}}%
\pgfpathlineto{\pgfqpoint{1.082674in}{0.389175in}}%
\pgfpathlineto{\pgfqpoint{1.083532in}{0.387075in}}%
\pgfpathlineto{\pgfqpoint{1.088256in}{0.375665in}}%
\pgfpathlineto{\pgfqpoint{1.094407in}{0.364724in}}%
\pgfpathlineto{\pgfqpoint{1.094602in}{0.364254in}}%
\pgfpathlineto{\pgfqpoint{1.099183in}{0.352843in}}%
\pgfpathlineto{\pgfqpoint{1.105328in}{0.341432in}}%
\pgfpathlineto{\pgfqpoint{1.106139in}{0.340044in}}%
\pgfpathlineto{\pgfqpoint{1.110165in}{0.330022in}}%
\pgfpathlineto{\pgfqpoint{1.115857in}{0.318611in}}%
\pgfpathlineto{\pgfqpoint{1.117871in}{0.315356in}}%
\pgfpathlineto{\pgfqpoint{1.122318in}{0.307200in}}%
\pgfpathlineto{\pgfqpoint{1.127155in}{0.295789in}}%
\pgfpathlineto{\pgfqpoint{1.129603in}{0.291876in}}%
\pgfpathlineto{\pgfqpoint{1.134392in}{0.284378in}}%
\pgfpathlineto{\pgfqpoint{1.141335in}{0.284378in}}%
\pgfpathlineto{\pgfqpoint{1.153068in}{0.284378in}}%
\pgfpathlineto{\pgfqpoint{1.164800in}{0.284378in}}%
\pgfpathlineto{\pgfqpoint{1.176532in}{0.284378in}}%
\pgfpathlineto{\pgfqpoint{1.188264in}{0.284378in}}%
\pgfpathlineto{\pgfqpoint{1.199996in}{0.284378in}}%
\pgfpathlineto{\pgfqpoint{1.208148in}{0.284378in}}%
\pgfpathlineto{\pgfqpoint{1.199996in}{0.288810in}}%
\pgfpathlineto{\pgfqpoint{1.188264in}{0.295424in}}%
\pgfpathlineto{\pgfqpoint{1.187668in}{0.295789in}}%
\pgfpathlineto{\pgfqpoint{1.176532in}{0.302432in}}%
\pgfpathlineto{\pgfqpoint{1.168537in}{0.307200in}}%
\pgfpathlineto{\pgfqpoint{1.164800in}{0.309380in}}%
\pgfpathlineto{\pgfqpoint{1.153068in}{0.313790in}}%
\pgfpathlineto{\pgfqpoint{1.149164in}{0.307200in}}%
\pgfpathlineto{\pgfqpoint{1.141335in}{0.302545in}}%
\pgfpathlineto{\pgfqpoint{1.138782in}{0.307200in}}%
\pgfpathlineto{\pgfqpoint{1.129603in}{0.317775in}}%
\pgfpathlineto{\pgfqpoint{1.129146in}{0.318611in}}%
\pgfpathlineto{\pgfqpoint{1.121612in}{0.330022in}}%
\pgfpathlineto{\pgfqpoint{1.117871in}{0.335728in}}%
\pgfpathlineto{\pgfqpoint{1.115366in}{0.341432in}}%
\pgfpathlineto{\pgfqpoint{1.109906in}{0.352843in}}%
\pgfpathlineto{\pgfqpoint{1.106139in}{0.358954in}}%
\pgfpathlineto{\pgfqpoint{1.103331in}{0.364254in}}%
\pgfpathlineto{\pgfqpoint{1.098153in}{0.375665in}}%
\pgfpathlineto{\pgfqpoint{1.094407in}{0.383896in}}%
\pgfpathlineto{\pgfqpoint{1.092705in}{0.387075in}}%
\pgfpathlineto{\pgfqpoint{1.087190in}{0.398486in}}%
\pgfpathlineto{\pgfqpoint{1.082674in}{0.408977in}}%
\pgfpathlineto{\pgfqpoint{1.082178in}{0.409897in}}%
\pgfpathlineto{\pgfqpoint{1.076422in}{0.421308in}}%
\pgfpathlineto{\pgfqpoint{1.071525in}{0.432719in}}%
\pgfpathlineto{\pgfqpoint{1.070942in}{0.434022in}}%
\pgfpathlineto{\pgfqpoint{1.065659in}{0.444129in}}%
\pgfpathlineto{\pgfqpoint{1.060665in}{0.455540in}}%
\pgfpathlineto{\pgfqpoint{1.059210in}{0.458814in}}%
\pgfpathlineto{\pgfqpoint{1.054873in}{0.466951in}}%
\pgfpathlineto{\pgfqpoint{1.049771in}{0.478362in}}%
\pgfpathlineto{\pgfqpoint{1.047478in}{0.483623in}}%
\pgfpathlineto{\pgfqpoint{1.044156in}{0.489772in}}%
\pgfpathlineto{\pgfqpoint{1.038903in}{0.501183in}}%
\pgfpathlineto{\pgfqpoint{1.035746in}{0.508598in}}%
\pgfpathlineto{\pgfqpoint{1.033562in}{0.512594in}}%
\pgfpathlineto{\pgfqpoint{1.028082in}{0.524005in}}%
\pgfpathlineto{\pgfqpoint{1.024013in}{0.533775in}}%
\pgfpathlineto{\pgfqpoint{1.023105in}{0.535416in}}%
\pgfpathlineto{\pgfqpoint{1.017337in}{0.546826in}}%
\pgfpathlineto{\pgfqpoint{1.012671in}{0.558237in}}%
\pgfpathlineto{\pgfqpoint{1.012281in}{0.559155in}}%
\pgfpathlineto{\pgfqpoint{1.006771in}{0.569648in}}%
\pgfpathlineto{\pgfqpoint{1.002027in}{0.581059in}}%
\pgfpathlineto{\pgfqpoint{1.000549in}{0.584654in}}%
\pgfpathlineto{\pgfqpoint{0.996422in}{0.592469in}}%
\pgfpathlineto{\pgfqpoint{0.992191in}{0.603880in}}%
\pgfpathlineto{\pgfqpoint{0.989234in}{0.615291in}}%
\pgfpathlineto{\pgfqpoint{0.988817in}{0.616894in}}%
\pgfpathlineto{\pgfqpoint{0.985213in}{0.626702in}}%
\pgfpathlineto{\pgfqpoint{0.982368in}{0.638113in}}%
\pgfpathlineto{\pgfqpoint{0.979584in}{0.649523in}}%
\pgfpathlineto{\pgfqpoint{0.977085in}{0.659807in}}%
\pgfpathlineto{\pgfqpoint{0.976645in}{0.660934in}}%
\pgfpathlineto{\pgfqpoint{0.972976in}{0.672345in}}%
\pgfpathlineto{\pgfqpoint{0.970302in}{0.683756in}}%
\pgfpathlineto{\pgfqpoint{0.967425in}{0.695166in}}%
\pgfpathlineto{\pgfqpoint{0.965352in}{0.703047in}}%
\pgfpathlineto{\pgfqpoint{0.964089in}{0.706577in}}%
\pgfpathlineto{\pgfqpoint{0.961025in}{0.717988in}}%
\pgfpathlineto{\pgfqpoint{0.958769in}{0.729399in}}%
\pgfpathlineto{\pgfqpoint{0.956172in}{0.740810in}}%
\pgfpathlineto{\pgfqpoint{0.953620in}{0.751533in}}%
\pgfpathlineto{\pgfqpoint{0.953421in}{0.752220in}}%
\pgfpathlineto{\pgfqpoint{0.950853in}{0.763631in}}%
\pgfpathlineto{\pgfqpoint{0.948164in}{0.775042in}}%
\pgfpathlineto{\pgfqpoint{0.943606in}{0.786453in}}%
\pgfpathlineto{\pgfqpoint{0.941888in}{0.790608in}}%
\pgfpathlineto{\pgfqpoint{0.938239in}{0.797863in}}%
\pgfpathlineto{\pgfqpoint{0.930156in}{0.807906in}}%
\pgfpathlineto{\pgfqpoint{0.926892in}{0.809274in}}%
\pgfpathlineto{\pgfqpoint{0.918424in}{0.811973in}}%
\pgfpathlineto{\pgfqpoint{0.906691in}{0.815606in}}%
\pgfpathlineto{\pgfqpoint{0.894959in}{0.815964in}}%
\pgfpathlineto{\pgfqpoint{0.889800in}{0.809274in}}%
\pgfpathlineto{\pgfqpoint{0.885756in}{0.797863in}}%
\pgfpathlineto{\pgfqpoint{0.883227in}{0.788550in}}%
\pgfpathlineto{\pgfqpoint{0.882845in}{0.786453in}}%
\pgfpathlineto{\pgfqpoint{0.880978in}{0.775042in}}%
\pgfpathlineto{\pgfqpoint{0.879752in}{0.763631in}}%
\pgfpathlineto{\pgfqpoint{0.879983in}{0.752220in}}%
\pgfpathlineto{\pgfqpoint{0.880965in}{0.740810in}}%
\pgfpathlineto{\pgfqpoint{0.883090in}{0.729399in}}%
\pgfpathlineto{\pgfqpoint{0.883227in}{0.728769in}}%
\pgfpathlineto{\pgfqpoint{0.886719in}{0.717988in}}%
\pgfpathlineto{\pgfqpoint{0.888963in}{0.706577in}}%
\pgfpathlineto{\pgfqpoint{0.891570in}{0.695166in}}%
\pgfpathlineto{\pgfqpoint{0.894591in}{0.683756in}}%
\pgfpathlineto{\pgfqpoint{0.894959in}{0.682406in}}%
\pgfpathlineto{\pgfqpoint{0.898452in}{0.672345in}}%
\pgfpathlineto{\pgfqpoint{0.901648in}{0.660934in}}%
\pgfpathlineto{\pgfqpoint{0.904946in}{0.649523in}}%
\pgfpathlineto{\pgfqpoint{0.906691in}{0.643365in}}%
\pgfpathlineto{\pgfqpoint{0.908675in}{0.638113in}}%
\pgfpathlineto{\pgfqpoint{0.912129in}{0.626702in}}%
\pgfpathlineto{\pgfqpoint{0.915615in}{0.615291in}}%
\pgfpathlineto{\pgfqpoint{0.918424in}{0.605865in}}%
\pgfpathlineto{\pgfqpoint{0.919236in}{0.603880in}}%
\pgfpathlineto{\pgfqpoint{0.922962in}{0.592469in}}%
\pgfpathlineto{\pgfqpoint{0.926617in}{0.581059in}}%
\pgfpathlineto{\pgfqpoint{0.930156in}{0.570143in}}%
\pgfpathlineto{\pgfqpoint{0.930396in}{0.569648in}}%
\pgfpathlineto{\pgfqpoint{0.936989in}{0.558237in}}%
\pgfpathlineto{\pgfqpoint{0.941888in}{0.549336in}}%
\pgfpathlineto{\pgfqpoint{0.943326in}{0.546826in}}%
\pgfpathlineto{\pgfqpoint{0.948978in}{0.535416in}}%
\pgfpathlineto{\pgfqpoint{0.953577in}{0.524005in}}%
\pgfpathlineto{\pgfqpoint{0.953620in}{0.523900in}}%
\pgfpathlineto{\pgfqpoint{0.958400in}{0.512594in}}%
\pgfpathlineto{\pgfqpoint{0.963116in}{0.501183in}}%
\pgfpathlineto{\pgfqpoint{0.965352in}{0.495906in}}%
\pgfpathlineto{\pgfqpoint{0.967931in}{0.489772in}}%
\pgfpathlineto{\pgfqpoint{0.972663in}{0.478362in}}%
\pgfpathlineto{\pgfqpoint{0.977085in}{0.467862in}}%
\pgfpathlineto{\pgfqpoint{0.977459in}{0.466951in}}%
\pgfpathlineto{\pgfqpoint{0.982169in}{0.455540in}}%
\pgfpathlineto{\pgfqpoint{0.986854in}{0.444129in}}%
\pgfpathlineto{\pgfqpoint{0.988817in}{0.439502in}}%
\pgfpathlineto{\pgfqpoint{0.991554in}{0.432719in}}%
\pgfpathlineto{\pgfqpoint{0.996140in}{0.421308in}}%
\pgfpathlineto{\pgfqpoint{1.000549in}{0.410553in}}%
\pgfpathlineto{\pgfqpoint{1.000806in}{0.409897in}}%
\pgfpathlineto{\pgfqpoint{1.005314in}{0.398486in}}%
\pgfpathlineto{\pgfqpoint{1.009831in}{0.387075in}}%
\pgfpathlineto{\pgfqpoint{1.012281in}{0.381085in}}%
\pgfpathlineto{\pgfqpoint{1.014513in}{0.375665in}}%
\pgfpathlineto{\pgfqpoint{1.018771in}{0.364254in}}%
\pgfpathlineto{\pgfqpoint{1.023266in}{0.352843in}}%
\pgfpathlineto{\pgfqpoint{1.024013in}{0.351011in}}%
\pgfpathlineto{\pgfqpoint{1.029557in}{0.341432in}}%
\pgfpathlineto{\pgfqpoint{1.034903in}{0.330022in}}%
\pgfpathlineto{\pgfqpoint{1.035746in}{0.328038in}}%
\pgfpathlineto{\pgfqpoint{1.041442in}{0.318611in}}%
\pgfpathlineto{\pgfqpoint{1.047478in}{0.308525in}}%
\pgfpathlineto{\pgfqpoint{1.048274in}{0.307200in}}%
\pgfpathlineto{\pgfqpoint{1.055139in}{0.295789in}}%
\pgfpathclose%
\pgfpathmoveto{\pgfqpoint{0.970181in}{0.603880in}}%
\pgfpathlineto{\pgfqpoint{0.965352in}{0.605177in}}%
\pgfpathlineto{\pgfqpoint{0.958921in}{0.615291in}}%
\pgfpathlineto{\pgfqpoint{0.953620in}{0.623030in}}%
\pgfpathlineto{\pgfqpoint{0.951955in}{0.626702in}}%
\pgfpathlineto{\pgfqpoint{0.947417in}{0.638113in}}%
\pgfpathlineto{\pgfqpoint{0.942646in}{0.649523in}}%
\pgfpathlineto{\pgfqpoint{0.941888in}{0.651140in}}%
\pgfpathlineto{\pgfqpoint{0.937618in}{0.660934in}}%
\pgfpathlineto{\pgfqpoint{0.932752in}{0.672345in}}%
\pgfpathlineto{\pgfqpoint{0.930156in}{0.678145in}}%
\pgfpathlineto{\pgfqpoint{0.927675in}{0.683756in}}%
\pgfpathlineto{\pgfqpoint{0.922692in}{0.695166in}}%
\pgfpathlineto{\pgfqpoint{0.918424in}{0.705120in}}%
\pgfpathlineto{\pgfqpoint{0.917822in}{0.706577in}}%
\pgfpathlineto{\pgfqpoint{0.913138in}{0.717988in}}%
\pgfpathlineto{\pgfqpoint{0.909498in}{0.729399in}}%
\pgfpathlineto{\pgfqpoint{0.906691in}{0.739717in}}%
\pgfpathlineto{\pgfqpoint{0.906424in}{0.740810in}}%
\pgfpathlineto{\pgfqpoint{0.904324in}{0.752220in}}%
\pgfpathlineto{\pgfqpoint{0.902286in}{0.763631in}}%
\pgfpathlineto{\pgfqpoint{0.900351in}{0.775042in}}%
\pgfpathlineto{\pgfqpoint{0.901866in}{0.786453in}}%
\pgfpathlineto{\pgfqpoint{0.906691in}{0.797055in}}%
\pgfpathlineto{\pgfqpoint{0.913802in}{0.797863in}}%
\pgfpathlineto{\pgfqpoint{0.918424in}{0.798015in}}%
\pgfpathlineto{\pgfqpoint{0.918741in}{0.797863in}}%
\pgfpathlineto{\pgfqpoint{0.930156in}{0.788058in}}%
\pgfpathlineto{\pgfqpoint{0.930958in}{0.786453in}}%
\pgfpathlineto{\pgfqpoint{0.936349in}{0.775042in}}%
\pgfpathlineto{\pgfqpoint{0.941888in}{0.765556in}}%
\pgfpathlineto{\pgfqpoint{0.942438in}{0.763631in}}%
\pgfpathlineto{\pgfqpoint{0.944271in}{0.752220in}}%
\pgfpathlineto{\pgfqpoint{0.946112in}{0.740810in}}%
\pgfpathlineto{\pgfqpoint{0.948980in}{0.729399in}}%
\pgfpathlineto{\pgfqpoint{0.952357in}{0.717988in}}%
\pgfpathlineto{\pgfqpoint{0.953620in}{0.714523in}}%
\pgfpathlineto{\pgfqpoint{0.954711in}{0.706577in}}%
\pgfpathlineto{\pgfqpoint{0.956421in}{0.695166in}}%
\pgfpathlineto{\pgfqpoint{0.959181in}{0.683756in}}%
\pgfpathlineto{\pgfqpoint{0.962904in}{0.672345in}}%
\pgfpathlineto{\pgfqpoint{0.965352in}{0.667086in}}%
\pgfpathlineto{\pgfqpoint{0.966170in}{0.660934in}}%
\pgfpathlineto{\pgfqpoint{0.967692in}{0.649523in}}%
\pgfpathlineto{\pgfqpoint{0.969881in}{0.638113in}}%
\pgfpathlineto{\pgfqpoint{0.971871in}{0.626702in}}%
\pgfpathlineto{\pgfqpoint{0.974966in}{0.615291in}}%
\pgfpathlineto{\pgfqpoint{0.977085in}{0.609064in}}%
\pgfpathlineto{\pgfqpoint{0.977330in}{0.603880in}}%
\pgfpathlineto{\pgfqpoint{0.977085in}{0.595334in}}%
\pgfpathclose%
\pgfusepath{fill}%
\end{pgfscope}%
\begin{pgfscope}%
\pgfpathrectangle{\pgfqpoint{0.211875in}{0.211875in}}{\pgfqpoint{1.313625in}{1.279725in}}%
\pgfusepath{clip}%
\pgfsetbuttcap%
\pgfsetroundjoin%
\definecolor{currentfill}{rgb}{0.924566,0.290534,0.242426}%
\pgfsetfillcolor{currentfill}%
\pgfsetlinewidth{0.000000pt}%
\definecolor{currentstroke}{rgb}{0.000000,0.000000,0.000000}%
\pgfsetstrokecolor{currentstroke}%
\pgfsetdash{}{0pt}%
\pgfpathmoveto{\pgfqpoint{0.296617in}{0.466116in}}%
\pgfpathlineto{\pgfqpoint{0.297602in}{0.466951in}}%
\pgfpathlineto{\pgfqpoint{0.308349in}{0.476542in}}%
\pgfpathlineto{\pgfqpoint{0.310509in}{0.478362in}}%
\pgfpathlineto{\pgfqpoint{0.320081in}{0.486913in}}%
\pgfpathlineto{\pgfqpoint{0.323495in}{0.489772in}}%
\pgfpathlineto{\pgfqpoint{0.331813in}{0.497196in}}%
\pgfpathlineto{\pgfqpoint{0.336606in}{0.501183in}}%
\pgfpathlineto{\pgfqpoint{0.343545in}{0.507377in}}%
\pgfpathlineto{\pgfqpoint{0.349863in}{0.512594in}}%
\pgfpathlineto{\pgfqpoint{0.355278in}{0.517426in}}%
\pgfpathlineto{\pgfqpoint{0.363295in}{0.524005in}}%
\pgfpathlineto{\pgfqpoint{0.367010in}{0.527317in}}%
\pgfpathlineto{\pgfqpoint{0.376831in}{0.535416in}}%
\pgfpathlineto{\pgfqpoint{0.378742in}{0.537119in}}%
\pgfpathlineto{\pgfqpoint{0.390244in}{0.546826in}}%
\pgfpathlineto{\pgfqpoint{0.390474in}{0.547032in}}%
\pgfpathlineto{\pgfqpoint{0.402206in}{0.557452in}}%
\pgfpathlineto{\pgfqpoint{0.403119in}{0.558237in}}%
\pgfpathlineto{\pgfqpoint{0.413939in}{0.567910in}}%
\pgfpathlineto{\pgfqpoint{0.415968in}{0.569648in}}%
\pgfpathlineto{\pgfqpoint{0.425671in}{0.577801in}}%
\pgfpathlineto{\pgfqpoint{0.429556in}{0.581059in}}%
\pgfpathlineto{\pgfqpoint{0.437403in}{0.587770in}}%
\pgfpathlineto{\pgfqpoint{0.443544in}{0.592469in}}%
\pgfpathlineto{\pgfqpoint{0.449135in}{0.597277in}}%
\pgfpathlineto{\pgfqpoint{0.457862in}{0.603880in}}%
\pgfpathlineto{\pgfqpoint{0.460867in}{0.606480in}}%
\pgfpathlineto{\pgfqpoint{0.472600in}{0.615270in}}%
\pgfpathlineto{\pgfqpoint{0.472626in}{0.615291in}}%
\pgfpathlineto{\pgfqpoint{0.484332in}{0.625111in}}%
\pgfpathlineto{\pgfqpoint{0.486365in}{0.626702in}}%
\pgfpathlineto{\pgfqpoint{0.496064in}{0.634916in}}%
\pgfpathlineto{\pgfqpoint{0.500184in}{0.638113in}}%
\pgfpathlineto{\pgfqpoint{0.507796in}{0.644627in}}%
\pgfpathlineto{\pgfqpoint{0.514156in}{0.649523in}}%
\pgfpathlineto{\pgfqpoint{0.519528in}{0.654178in}}%
\pgfpathlineto{\pgfqpoint{0.528359in}{0.660934in}}%
\pgfpathlineto{\pgfqpoint{0.531261in}{0.663488in}}%
\pgfpathlineto{\pgfqpoint{0.542767in}{0.672345in}}%
\pgfpathlineto{\pgfqpoint{0.542993in}{0.672548in}}%
\pgfpathlineto{\pgfqpoint{0.554725in}{0.682521in}}%
\pgfpathlineto{\pgfqpoint{0.556258in}{0.683756in}}%
\pgfpathlineto{\pgfqpoint{0.566457in}{0.692650in}}%
\pgfpathlineto{\pgfqpoint{0.569584in}{0.695166in}}%
\pgfpathlineto{\pgfqpoint{0.578190in}{0.702829in}}%
\pgfpathlineto{\pgfqpoint{0.582840in}{0.706577in}}%
\pgfpathlineto{\pgfqpoint{0.589922in}{0.713053in}}%
\pgfpathlineto{\pgfqpoint{0.596009in}{0.717988in}}%
\pgfpathlineto{\pgfqpoint{0.601654in}{0.723336in}}%
\pgfpathlineto{\pgfqpoint{0.609040in}{0.729399in}}%
\pgfpathlineto{\pgfqpoint{0.613386in}{0.733726in}}%
\pgfpathlineto{\pgfqpoint{0.621825in}{0.740810in}}%
\pgfpathlineto{\pgfqpoint{0.625118in}{0.744364in}}%
\pgfpathlineto{\pgfqpoint{0.634128in}{0.752220in}}%
\pgfpathlineto{\pgfqpoint{0.636851in}{0.755790in}}%
\pgfpathlineto{\pgfqpoint{0.645276in}{0.763631in}}%
\pgfpathlineto{\pgfqpoint{0.648583in}{0.769736in}}%
\pgfpathlineto{\pgfqpoint{0.653407in}{0.775042in}}%
\pgfpathlineto{\pgfqpoint{0.648583in}{0.777324in}}%
\pgfpathlineto{\pgfqpoint{0.642455in}{0.775042in}}%
\pgfpathlineto{\pgfqpoint{0.636851in}{0.772328in}}%
\pgfpathlineto{\pgfqpoint{0.625118in}{0.766794in}}%
\pgfpathlineto{\pgfqpoint{0.618683in}{0.763631in}}%
\pgfpathlineto{\pgfqpoint{0.613386in}{0.760612in}}%
\pgfpathlineto{\pgfqpoint{0.601654in}{0.754428in}}%
\pgfpathlineto{\pgfqpoint{0.597492in}{0.752220in}}%
\pgfpathlineto{\pgfqpoint{0.589922in}{0.747710in}}%
\pgfpathlineto{\pgfqpoint{0.578190in}{0.741402in}}%
\pgfpathlineto{\pgfqpoint{0.577104in}{0.740810in}}%
\pgfpathlineto{\pgfqpoint{0.566457in}{0.734343in}}%
\pgfpathlineto{\pgfqpoint{0.557480in}{0.729399in}}%
\pgfpathlineto{\pgfqpoint{0.554725in}{0.727682in}}%
\pgfpathlineto{\pgfqpoint{0.542993in}{0.720695in}}%
\pgfpathlineto{\pgfqpoint{0.538170in}{0.717988in}}%
\pgfpathlineto{\pgfqpoint{0.531261in}{0.713619in}}%
\pgfpathlineto{\pgfqpoint{0.519528in}{0.706857in}}%
\pgfpathlineto{\pgfqpoint{0.519035in}{0.706577in}}%
\pgfpathlineto{\pgfqpoint{0.507796in}{0.699410in}}%
\pgfpathlineto{\pgfqpoint{0.500458in}{0.695166in}}%
\pgfpathlineto{\pgfqpoint{0.496064in}{0.692277in}}%
\pgfpathlineto{\pgfqpoint{0.484332in}{0.685114in}}%
\pgfpathlineto{\pgfqpoint{0.481982in}{0.683756in}}%
\pgfpathlineto{\pgfqpoint{0.472600in}{0.677575in}}%
\pgfpathlineto{\pgfqpoint{0.463789in}{0.672345in}}%
\pgfpathlineto{\pgfqpoint{0.460867in}{0.670358in}}%
\pgfpathlineto{\pgfqpoint{0.449135in}{0.662934in}}%
\pgfpathlineto{\pgfqpoint{0.445704in}{0.660934in}}%
\pgfpathlineto{\pgfqpoint{0.437403in}{0.655386in}}%
\pgfpathlineto{\pgfqpoint{0.427564in}{0.649523in}}%
\pgfpathlineto{\pgfqpoint{0.425671in}{0.648228in}}%
\pgfpathlineto{\pgfqpoint{0.413939in}{0.640667in}}%
\pgfpathlineto{\pgfqpoint{0.409568in}{0.638113in}}%
\pgfpathlineto{\pgfqpoint{0.402206in}{0.633156in}}%
\pgfpathlineto{\pgfqpoint{0.391403in}{0.626702in}}%
\pgfpathlineto{\pgfqpoint{0.390474in}{0.626063in}}%
\pgfpathlineto{\pgfqpoint{0.378742in}{0.618414in}}%
\pgfpathlineto{\pgfqpoint{0.373365in}{0.615291in}}%
\pgfpathlineto{\pgfqpoint{0.367010in}{0.611025in}}%
\pgfpathlineto{\pgfqpoint{0.355278in}{0.604047in}}%
\pgfpathlineto{\pgfqpoint{0.354981in}{0.603880in}}%
\pgfpathlineto{\pgfqpoint{0.343545in}{0.596420in}}%
\pgfpathlineto{\pgfqpoint{0.336631in}{0.592469in}}%
\pgfpathlineto{\pgfqpoint{0.331813in}{0.589282in}}%
\pgfpathlineto{\pgfqpoint{0.320081in}{0.582287in}}%
\pgfpathlineto{\pgfqpoint{0.317853in}{0.581059in}}%
\pgfpathlineto{\pgfqpoint{0.308349in}{0.574983in}}%
\pgfpathlineto{\pgfqpoint{0.298759in}{0.569648in}}%
\pgfpathlineto{\pgfqpoint{0.296617in}{0.568267in}}%
\pgfpathlineto{\pgfqpoint{0.284884in}{0.561222in}}%
\pgfpathlineto{\pgfqpoint{0.284884in}{0.558237in}}%
\pgfpathlineto{\pgfqpoint{0.284884in}{0.546826in}}%
\pgfpathlineto{\pgfqpoint{0.284884in}{0.535416in}}%
\pgfpathlineto{\pgfqpoint{0.284884in}{0.524005in}}%
\pgfpathlineto{\pgfqpoint{0.284884in}{0.512594in}}%
\pgfpathlineto{\pgfqpoint{0.284884in}{0.501183in}}%
\pgfpathlineto{\pgfqpoint{0.284884in}{0.489772in}}%
\pgfpathlineto{\pgfqpoint{0.284884in}{0.478362in}}%
\pgfpathlineto{\pgfqpoint{0.284884in}{0.466951in}}%
\pgfpathlineto{\pgfqpoint{0.284884in}{0.455742in}}%
\pgfpathclose%
\pgfusepath{fill}%
\end{pgfscope}%
\begin{pgfscope}%
\pgfpathrectangle{\pgfqpoint{0.211875in}{0.211875in}}{\pgfqpoint{1.313625in}{1.279725in}}%
\pgfusepath{clip}%
\pgfsetbuttcap%
\pgfsetroundjoin%
\definecolor{currentfill}{rgb}{0.924566,0.290534,0.242426}%
\pgfsetfillcolor{currentfill}%
\pgfsetlinewidth{0.000000pt}%
\definecolor{currentstroke}{rgb}{0.000000,0.000000,0.000000}%
\pgfsetstrokecolor{currentstroke}%
\pgfsetdash{}{0pt}%
\pgfpathmoveto{\pgfqpoint{1.329051in}{0.786855in}}%
\pgfpathlineto{\pgfqpoint{1.340783in}{0.796160in}}%
\pgfpathlineto{\pgfqpoint{1.340925in}{0.797863in}}%
\pgfpathlineto{\pgfqpoint{1.342504in}{0.809274in}}%
\pgfpathlineto{\pgfqpoint{1.343994in}{0.820685in}}%
\pgfpathlineto{\pgfqpoint{1.345372in}{0.832096in}}%
\pgfpathlineto{\pgfqpoint{1.346626in}{0.843507in}}%
\pgfpathlineto{\pgfqpoint{1.347764in}{0.854917in}}%
\pgfpathlineto{\pgfqpoint{1.346400in}{0.866328in}}%
\pgfpathlineto{\pgfqpoint{1.343218in}{0.877739in}}%
\pgfpathlineto{\pgfqpoint{1.340783in}{0.886748in}}%
\pgfpathlineto{\pgfqpoint{1.340181in}{0.889150in}}%
\pgfpathlineto{\pgfqpoint{1.337856in}{0.900560in}}%
\pgfpathlineto{\pgfqpoint{1.339509in}{0.911971in}}%
\pgfpathlineto{\pgfqpoint{1.340783in}{0.914134in}}%
\pgfpathlineto{\pgfqpoint{1.346272in}{0.923382in}}%
\pgfpathlineto{\pgfqpoint{1.352515in}{0.930284in}}%
\pgfpathlineto{\pgfqpoint{1.357521in}{0.934793in}}%
\pgfpathlineto{\pgfqpoint{1.364247in}{0.939427in}}%
\pgfpathlineto{\pgfqpoint{1.375980in}{0.945866in}}%
\pgfpathlineto{\pgfqpoint{1.377231in}{0.946204in}}%
\pgfpathlineto{\pgfqpoint{1.387712in}{0.957483in}}%
\pgfpathlineto{\pgfqpoint{1.399444in}{0.951876in}}%
\pgfpathlineto{\pgfqpoint{1.410445in}{0.946204in}}%
\pgfpathlineto{\pgfqpoint{1.411176in}{0.946058in}}%
\pgfpathlineto{\pgfqpoint{1.422908in}{0.941954in}}%
\pgfpathlineto{\pgfqpoint{1.434641in}{0.937316in}}%
\pgfpathlineto{\pgfqpoint{1.441077in}{0.934793in}}%
\pgfpathlineto{\pgfqpoint{1.446373in}{0.931743in}}%
\pgfpathlineto{\pgfqpoint{1.446373in}{0.934793in}}%
\pgfpathlineto{\pgfqpoint{1.446373in}{0.942726in}}%
\pgfpathlineto{\pgfqpoint{1.437021in}{0.946204in}}%
\pgfpathlineto{\pgfqpoint{1.434641in}{0.947132in}}%
\pgfpathlineto{\pgfqpoint{1.422908in}{0.953178in}}%
\pgfpathlineto{\pgfqpoint{1.415840in}{0.957614in}}%
\pgfpathlineto{\pgfqpoint{1.415999in}{0.969025in}}%
\pgfpathlineto{\pgfqpoint{1.422908in}{0.976770in}}%
\pgfpathlineto{\pgfqpoint{1.434641in}{0.972044in}}%
\pgfpathlineto{\pgfqpoint{1.440974in}{0.969025in}}%
\pgfpathlineto{\pgfqpoint{1.446373in}{0.965595in}}%
\pgfpathlineto{\pgfqpoint{1.446373in}{0.969025in}}%
\pgfpathlineto{\pgfqpoint{1.446373in}{0.974953in}}%
\pgfpathlineto{\pgfqpoint{1.435231in}{0.980436in}}%
\pgfpathlineto{\pgfqpoint{1.434641in}{0.980711in}}%
\pgfpathlineto{\pgfqpoint{1.422908in}{0.987365in}}%
\pgfpathlineto{\pgfqpoint{1.415618in}{0.991847in}}%
\pgfpathlineto{\pgfqpoint{1.411176in}{0.999162in}}%
\pgfpathlineto{\pgfqpoint{1.409228in}{1.003257in}}%
\pgfpathlineto{\pgfqpoint{1.408722in}{1.014668in}}%
\pgfpathlineto{\pgfqpoint{1.409019in}{1.026079in}}%
\pgfpathlineto{\pgfqpoint{1.410394in}{1.037490in}}%
\pgfpathlineto{\pgfqpoint{1.411176in}{1.040089in}}%
\pgfpathlineto{\pgfqpoint{1.413380in}{1.048901in}}%
\pgfpathlineto{\pgfqpoint{1.418911in}{1.060311in}}%
\pgfpathlineto{\pgfqpoint{1.422908in}{1.063556in}}%
\pgfpathlineto{\pgfqpoint{1.432339in}{1.071722in}}%
\pgfpathlineto{\pgfqpoint{1.434641in}{1.073745in}}%
\pgfpathlineto{\pgfqpoint{1.445678in}{1.083133in}}%
\pgfpathlineto{\pgfqpoint{1.446373in}{1.083742in}}%
\pgfpathlineto{\pgfqpoint{1.446373in}{1.088981in}}%
\pgfpathlineto{\pgfqpoint{1.439700in}{1.083133in}}%
\pgfpathlineto{\pgfqpoint{1.434641in}{1.078830in}}%
\pgfpathlineto{\pgfqpoint{1.426552in}{1.071722in}}%
\pgfpathlineto{\pgfqpoint{1.422908in}{1.068567in}}%
\pgfpathlineto{\pgfqpoint{1.412736in}{1.060311in}}%
\pgfpathlineto{\pgfqpoint{1.411176in}{1.056329in}}%
\pgfpathlineto{\pgfqpoint{1.408609in}{1.048901in}}%
\pgfpathlineto{\pgfqpoint{1.405380in}{1.037490in}}%
\pgfpathlineto{\pgfqpoint{1.404247in}{1.026079in}}%
\pgfpathlineto{\pgfqpoint{1.403097in}{1.014668in}}%
\pgfpathlineto{\pgfqpoint{1.399444in}{1.004417in}}%
\pgfpathlineto{\pgfqpoint{1.397757in}{1.003257in}}%
\pgfpathlineto{\pgfqpoint{1.387712in}{0.999070in}}%
\pgfpathlineto{\pgfqpoint{1.375980in}{0.996297in}}%
\pgfpathlineto{\pgfqpoint{1.364247in}{0.996653in}}%
\pgfpathlineto{\pgfqpoint{1.352515in}{0.992011in}}%
\pgfpathlineto{\pgfqpoint{1.351982in}{0.991847in}}%
\pgfpathlineto{\pgfqpoint{1.340783in}{0.988100in}}%
\pgfpathlineto{\pgfqpoint{1.337840in}{0.991847in}}%
\pgfpathlineto{\pgfqpoint{1.340783in}{0.997588in}}%
\pgfpathlineto{\pgfqpoint{1.343517in}{1.003257in}}%
\pgfpathlineto{\pgfqpoint{1.347075in}{1.014668in}}%
\pgfpathlineto{\pgfqpoint{1.349025in}{1.026079in}}%
\pgfpathlineto{\pgfqpoint{1.350949in}{1.037490in}}%
\pgfpathlineto{\pgfqpoint{1.352515in}{1.038663in}}%
\pgfpathlineto{\pgfqpoint{1.364247in}{1.047020in}}%
\pgfpathlineto{\pgfqpoint{1.365959in}{1.048901in}}%
\pgfpathlineto{\pgfqpoint{1.375980in}{1.054969in}}%
\pgfpathlineto{\pgfqpoint{1.386572in}{1.060311in}}%
\pgfpathlineto{\pgfqpoint{1.387712in}{1.060875in}}%
\pgfpathlineto{\pgfqpoint{1.399444in}{1.064817in}}%
\pgfpathlineto{\pgfqpoint{1.405449in}{1.071722in}}%
\pgfpathlineto{\pgfqpoint{1.408037in}{1.083133in}}%
\pgfpathlineto{\pgfqpoint{1.411176in}{1.090007in}}%
\pgfpathlineto{\pgfqpoint{1.413166in}{1.094544in}}%
\pgfpathlineto{\pgfqpoint{1.418166in}{1.105954in}}%
\pgfpathlineto{\pgfqpoint{1.422908in}{1.111628in}}%
\pgfpathlineto{\pgfqpoint{1.434641in}{1.116089in}}%
\pgfpathlineto{\pgfqpoint{1.440084in}{1.117365in}}%
\pgfpathlineto{\pgfqpoint{1.446373in}{1.119915in}}%
\pgfpathlineto{\pgfqpoint{1.446373in}{1.127618in}}%
\pgfpathlineto{\pgfqpoint{1.434641in}{1.126643in}}%
\pgfpathlineto{\pgfqpoint{1.422908in}{1.127174in}}%
\pgfpathlineto{\pgfqpoint{1.419124in}{1.128776in}}%
\pgfpathlineto{\pgfqpoint{1.422908in}{1.131726in}}%
\pgfpathlineto{\pgfqpoint{1.429897in}{1.140187in}}%
\pgfpathlineto{\pgfqpoint{1.434641in}{1.144371in}}%
\pgfpathlineto{\pgfqpoint{1.443000in}{1.151598in}}%
\pgfpathlineto{\pgfqpoint{1.434641in}{1.157656in}}%
\pgfpathlineto{\pgfqpoint{1.426872in}{1.163008in}}%
\pgfpathlineto{\pgfqpoint{1.422908in}{1.165998in}}%
\pgfpathlineto{\pgfqpoint{1.411176in}{1.173728in}}%
\pgfpathlineto{\pgfqpoint{1.409824in}{1.174419in}}%
\pgfpathlineto{\pgfqpoint{1.399444in}{1.178677in}}%
\pgfpathlineto{\pgfqpoint{1.387712in}{1.179419in}}%
\pgfpathlineto{\pgfqpoint{1.375980in}{1.178083in}}%
\pgfpathlineto{\pgfqpoint{1.364247in}{1.174449in}}%
\pgfpathlineto{\pgfqpoint{1.364186in}{1.174419in}}%
\pgfpathlineto{\pgfqpoint{1.352515in}{1.168974in}}%
\pgfpathlineto{\pgfqpoint{1.342991in}{1.163008in}}%
\pgfpathlineto{\pgfqpoint{1.340783in}{1.158996in}}%
\pgfpathlineto{\pgfqpoint{1.339024in}{1.151598in}}%
\pgfpathlineto{\pgfqpoint{1.340783in}{1.150118in}}%
\pgfpathlineto{\pgfqpoint{1.352515in}{1.141211in}}%
\pgfpathlineto{\pgfqpoint{1.354072in}{1.140187in}}%
\pgfpathlineto{\pgfqpoint{1.364247in}{1.136105in}}%
\pgfpathlineto{\pgfqpoint{1.375980in}{1.132174in}}%
\pgfpathlineto{\pgfqpoint{1.387403in}{1.128776in}}%
\pgfpathlineto{\pgfqpoint{1.387712in}{1.128672in}}%
\pgfpathlineto{\pgfqpoint{1.399444in}{1.125294in}}%
\pgfpathlineto{\pgfqpoint{1.411176in}{1.124431in}}%
\pgfpathlineto{\pgfqpoint{1.415378in}{1.117365in}}%
\pgfpathlineto{\pgfqpoint{1.411176in}{1.106882in}}%
\pgfpathlineto{\pgfqpoint{1.411007in}{1.105954in}}%
\pgfpathlineto{\pgfqpoint{1.407332in}{1.094544in}}%
\pgfpathlineto{\pgfqpoint{1.403178in}{1.083133in}}%
\pgfpathlineto{\pgfqpoint{1.399444in}{1.074825in}}%
\pgfpathlineto{\pgfqpoint{1.396788in}{1.071722in}}%
\pgfpathlineto{\pgfqpoint{1.387712in}{1.066350in}}%
\pgfpathlineto{\pgfqpoint{1.375980in}{1.060605in}}%
\pgfpathlineto{\pgfqpoint{1.375188in}{1.060311in}}%
\pgfpathlineto{\pgfqpoint{1.364247in}{1.055545in}}%
\pgfpathlineto{\pgfqpoint{1.357068in}{1.048901in}}%
\pgfpathlineto{\pgfqpoint{1.352515in}{1.045415in}}%
\pgfpathlineto{\pgfqpoint{1.341937in}{1.037490in}}%
\pgfpathlineto{\pgfqpoint{1.342058in}{1.026079in}}%
\pgfpathlineto{\pgfqpoint{1.340783in}{1.019270in}}%
\pgfpathlineto{\pgfqpoint{1.339744in}{1.014668in}}%
\pgfpathlineto{\pgfqpoint{1.334567in}{1.003257in}}%
\pgfpathlineto{\pgfqpoint{1.329051in}{0.993125in}}%
\pgfpathlineto{\pgfqpoint{1.328254in}{0.991847in}}%
\pgfpathlineto{\pgfqpoint{1.317319in}{0.982266in}}%
\pgfpathlineto{\pgfqpoint{1.310117in}{0.980436in}}%
\pgfpathlineto{\pgfqpoint{1.305586in}{0.979292in}}%
\pgfpathlineto{\pgfqpoint{1.293854in}{0.977060in}}%
\pgfpathlineto{\pgfqpoint{1.282122in}{0.976967in}}%
\pgfpathlineto{\pgfqpoint{1.270390in}{0.977450in}}%
\pgfpathlineto{\pgfqpoint{1.258658in}{0.978077in}}%
\pgfpathlineto{\pgfqpoint{1.246925in}{0.980061in}}%
\pgfpathlineto{\pgfqpoint{1.245535in}{0.980436in}}%
\pgfpathlineto{\pgfqpoint{1.235193in}{0.984451in}}%
\pgfpathlineto{\pgfqpoint{1.224716in}{0.991847in}}%
\pgfpathlineto{\pgfqpoint{1.223461in}{0.992500in}}%
\pgfpathlineto{\pgfqpoint{1.211729in}{1.001382in}}%
\pgfpathlineto{\pgfqpoint{1.209652in}{1.003257in}}%
\pgfpathlineto{\pgfqpoint{1.199996in}{1.010733in}}%
\pgfpathlineto{\pgfqpoint{1.196298in}{1.014668in}}%
\pgfpathlineto{\pgfqpoint{1.188264in}{1.024520in}}%
\pgfpathlineto{\pgfqpoint{1.186958in}{1.026079in}}%
\pgfpathlineto{\pgfqpoint{1.177077in}{1.037490in}}%
\pgfpathlineto{\pgfqpoint{1.176532in}{1.037899in}}%
\pgfpathlineto{\pgfqpoint{1.168692in}{1.048901in}}%
\pgfpathlineto{\pgfqpoint{1.164800in}{1.052428in}}%
\pgfpathlineto{\pgfqpoint{1.158994in}{1.060311in}}%
\pgfpathlineto{\pgfqpoint{1.153068in}{1.068942in}}%
\pgfpathlineto{\pgfqpoint{1.151471in}{1.071722in}}%
\pgfpathlineto{\pgfqpoint{1.142237in}{1.083133in}}%
\pgfpathlineto{\pgfqpoint{1.142920in}{1.094544in}}%
\pgfpathlineto{\pgfqpoint{1.150521in}{1.105954in}}%
\pgfpathlineto{\pgfqpoint{1.153068in}{1.110528in}}%
\pgfpathlineto{\pgfqpoint{1.164800in}{1.115999in}}%
\pgfpathlineto{\pgfqpoint{1.172797in}{1.117365in}}%
\pgfpathlineto{\pgfqpoint{1.176532in}{1.122558in}}%
\pgfpathlineto{\pgfqpoint{1.188264in}{1.127610in}}%
\pgfpathlineto{\pgfqpoint{1.199996in}{1.127372in}}%
\pgfpathlineto{\pgfqpoint{1.211729in}{1.125411in}}%
\pgfpathlineto{\pgfqpoint{1.223461in}{1.122305in}}%
\pgfpathlineto{\pgfqpoint{1.234803in}{1.117365in}}%
\pgfpathlineto{\pgfqpoint{1.235193in}{1.117290in}}%
\pgfpathlineto{\pgfqpoint{1.246925in}{1.116159in}}%
\pgfpathlineto{\pgfqpoint{1.258658in}{1.112600in}}%
\pgfpathlineto{\pgfqpoint{1.270390in}{1.110132in}}%
\pgfpathlineto{\pgfqpoint{1.282122in}{1.107538in}}%
\pgfpathlineto{\pgfqpoint{1.289058in}{1.105954in}}%
\pgfpathlineto{\pgfqpoint{1.293854in}{1.104835in}}%
\pgfpathlineto{\pgfqpoint{1.298611in}{1.105954in}}%
\pgfpathlineto{\pgfqpoint{1.305586in}{1.107534in}}%
\pgfpathlineto{\pgfqpoint{1.317319in}{1.110839in}}%
\pgfpathlineto{\pgfqpoint{1.325800in}{1.117365in}}%
\pgfpathlineto{\pgfqpoint{1.326845in}{1.128776in}}%
\pgfpathlineto{\pgfqpoint{1.326943in}{1.140187in}}%
\pgfpathlineto{\pgfqpoint{1.321506in}{1.151598in}}%
\pgfpathlineto{\pgfqpoint{1.317319in}{1.156166in}}%
\pgfpathlineto{\pgfqpoint{1.313329in}{1.163008in}}%
\pgfpathlineto{\pgfqpoint{1.305586in}{1.173045in}}%
\pgfpathlineto{\pgfqpoint{1.304691in}{1.174419in}}%
\pgfpathlineto{\pgfqpoint{1.295827in}{1.185830in}}%
\pgfpathlineto{\pgfqpoint{1.293854in}{1.187850in}}%
\pgfpathlineto{\pgfqpoint{1.284281in}{1.197241in}}%
\pgfpathlineto{\pgfqpoint{1.282122in}{1.198863in}}%
\pgfpathlineto{\pgfqpoint{1.270583in}{1.208651in}}%
\pgfpathlineto{\pgfqpoint{1.270390in}{1.208858in}}%
\pgfpathlineto{\pgfqpoint{1.259701in}{1.220062in}}%
\pgfpathlineto{\pgfqpoint{1.258658in}{1.221275in}}%
\pgfpathlineto{\pgfqpoint{1.248489in}{1.231473in}}%
\pgfpathlineto{\pgfqpoint{1.246925in}{1.233136in}}%
\pgfpathlineto{\pgfqpoint{1.237472in}{1.242884in}}%
\pgfpathlineto{\pgfqpoint{1.235193in}{1.246043in}}%
\pgfpathlineto{\pgfqpoint{1.229512in}{1.254295in}}%
\pgfpathlineto{\pgfqpoint{1.223461in}{1.263561in}}%
\pgfpathlineto{\pgfqpoint{1.221572in}{1.265705in}}%
\pgfpathlineto{\pgfqpoint{1.214443in}{1.277116in}}%
\pgfpathlineto{\pgfqpoint{1.211729in}{1.281485in}}%
\pgfpathlineto{\pgfqpoint{1.206406in}{1.288527in}}%
\pgfpathlineto{\pgfqpoint{1.199996in}{1.299633in}}%
\pgfpathlineto{\pgfqpoint{1.199739in}{1.299938in}}%
\pgfpathlineto{\pgfqpoint{1.192253in}{1.311348in}}%
\pgfpathlineto{\pgfqpoint{1.188264in}{1.318932in}}%
\pgfpathlineto{\pgfqpoint{1.185368in}{1.322759in}}%
\pgfpathlineto{\pgfqpoint{1.180031in}{1.334170in}}%
\pgfpathlineto{\pgfqpoint{1.176532in}{1.340966in}}%
\pgfpathlineto{\pgfqpoint{1.173393in}{1.345581in}}%
\pgfpathlineto{\pgfqpoint{1.168621in}{1.356992in}}%
\pgfpathlineto{\pgfqpoint{1.164800in}{1.363561in}}%
\pgfpathlineto{\pgfqpoint{1.161896in}{1.368402in}}%
\pgfpathlineto{\pgfqpoint{1.160143in}{1.379813in}}%
\pgfpathlineto{\pgfqpoint{1.159545in}{1.391224in}}%
\pgfpathlineto{\pgfqpoint{1.164800in}{1.402270in}}%
\pgfpathlineto{\pgfqpoint{1.164983in}{1.402635in}}%
\pgfpathlineto{\pgfqpoint{1.171191in}{1.414045in}}%
\pgfpathlineto{\pgfqpoint{1.164800in}{1.414045in}}%
\pgfpathlineto{\pgfqpoint{1.153068in}{1.414045in}}%
\pgfpathlineto{\pgfqpoint{1.141335in}{1.414045in}}%
\pgfpathlineto{\pgfqpoint{1.129603in}{1.414045in}}%
\pgfpathlineto{\pgfqpoint{1.117871in}{1.414045in}}%
\pgfpathlineto{\pgfqpoint{1.106139in}{1.414045in}}%
\pgfpathlineto{\pgfqpoint{1.094407in}{1.414045in}}%
\pgfpathlineto{\pgfqpoint{1.082674in}{1.414045in}}%
\pgfpathlineto{\pgfqpoint{1.073819in}{1.414045in}}%
\pgfpathlineto{\pgfqpoint{1.082674in}{1.404050in}}%
\pgfpathlineto{\pgfqpoint{1.083734in}{1.402635in}}%
\pgfpathlineto{\pgfqpoint{1.089555in}{1.391224in}}%
\pgfpathlineto{\pgfqpoint{1.091356in}{1.379813in}}%
\pgfpathlineto{\pgfqpoint{1.091381in}{1.368402in}}%
\pgfpathlineto{\pgfqpoint{1.090075in}{1.356992in}}%
\pgfpathlineto{\pgfqpoint{1.087595in}{1.345581in}}%
\pgfpathlineto{\pgfqpoint{1.083457in}{1.334170in}}%
\pgfpathlineto{\pgfqpoint{1.094407in}{1.325137in}}%
\pgfpathlineto{\pgfqpoint{1.099237in}{1.322759in}}%
\pgfpathlineto{\pgfqpoint{1.106139in}{1.319123in}}%
\pgfpathlineto{\pgfqpoint{1.117871in}{1.313167in}}%
\pgfpathlineto{\pgfqpoint{1.121558in}{1.311348in}}%
\pgfpathlineto{\pgfqpoint{1.129603in}{1.307093in}}%
\pgfpathlineto{\pgfqpoint{1.135807in}{1.299938in}}%
\pgfpathlineto{\pgfqpoint{1.141335in}{1.292280in}}%
\pgfpathlineto{\pgfqpoint{1.144141in}{1.288527in}}%
\pgfpathlineto{\pgfqpoint{1.151376in}{1.277116in}}%
\pgfpathlineto{\pgfqpoint{1.153068in}{1.273822in}}%
\pgfpathlineto{\pgfqpoint{1.157458in}{1.265705in}}%
\pgfpathlineto{\pgfqpoint{1.164800in}{1.258695in}}%
\pgfpathlineto{\pgfqpoint{1.167676in}{1.254295in}}%
\pgfpathlineto{\pgfqpoint{1.176532in}{1.249554in}}%
\pgfpathlineto{\pgfqpoint{1.185940in}{1.242884in}}%
\pgfpathlineto{\pgfqpoint{1.188264in}{1.241375in}}%
\pgfpathlineto{\pgfqpoint{1.199996in}{1.233617in}}%
\pgfpathlineto{\pgfqpoint{1.203235in}{1.231473in}}%
\pgfpathlineto{\pgfqpoint{1.211729in}{1.226037in}}%
\pgfpathlineto{\pgfqpoint{1.216960in}{1.220062in}}%
\pgfpathlineto{\pgfqpoint{1.221007in}{1.208651in}}%
\pgfpathlineto{\pgfqpoint{1.223461in}{1.203197in}}%
\pgfpathlineto{\pgfqpoint{1.226265in}{1.197241in}}%
\pgfpathlineto{\pgfqpoint{1.231326in}{1.185830in}}%
\pgfpathlineto{\pgfqpoint{1.234081in}{1.174419in}}%
\pgfpathlineto{\pgfqpoint{1.233800in}{1.163008in}}%
\pgfpathlineto{\pgfqpoint{1.224579in}{1.151598in}}%
\pgfpathlineto{\pgfqpoint{1.223461in}{1.150279in}}%
\pgfpathlineto{\pgfqpoint{1.214358in}{1.140187in}}%
\pgfpathlineto{\pgfqpoint{1.211729in}{1.137179in}}%
\pgfpathlineto{\pgfqpoint{1.199996in}{1.132363in}}%
\pgfpathlineto{\pgfqpoint{1.188264in}{1.131778in}}%
\pgfpathlineto{\pgfqpoint{1.176532in}{1.132737in}}%
\pgfpathlineto{\pgfqpoint{1.164800in}{1.138488in}}%
\pgfpathlineto{\pgfqpoint{1.159240in}{1.140187in}}%
\pgfpathlineto{\pgfqpoint{1.153068in}{1.140750in}}%
\pgfpathlineto{\pgfqpoint{1.141335in}{1.141101in}}%
\pgfpathlineto{\pgfqpoint{1.129603in}{1.148919in}}%
\pgfpathlineto{\pgfqpoint{1.126729in}{1.151598in}}%
\pgfpathlineto{\pgfqpoint{1.117871in}{1.159845in}}%
\pgfpathlineto{\pgfqpoint{1.114721in}{1.163008in}}%
\pgfpathlineto{\pgfqpoint{1.106139in}{1.172162in}}%
\pgfpathlineto{\pgfqpoint{1.103982in}{1.174419in}}%
\pgfpathlineto{\pgfqpoint{1.094407in}{1.184458in}}%
\pgfpathlineto{\pgfqpoint{1.093036in}{1.185830in}}%
\pgfpathlineto{\pgfqpoint{1.082674in}{1.196336in}}%
\pgfpathlineto{\pgfqpoint{1.081764in}{1.197241in}}%
\pgfpathlineto{\pgfqpoint{1.070942in}{1.208027in}}%
\pgfpathlineto{\pgfqpoint{1.070300in}{1.208651in}}%
\pgfpathlineto{\pgfqpoint{1.059559in}{1.220062in}}%
\pgfpathlineto{\pgfqpoint{1.059210in}{1.220729in}}%
\pgfpathlineto{\pgfqpoint{1.053270in}{1.231473in}}%
\pgfpathlineto{\pgfqpoint{1.047478in}{1.242825in}}%
\pgfpathlineto{\pgfqpoint{1.047457in}{1.242884in}}%
\pgfpathlineto{\pgfqpoint{1.045121in}{1.254295in}}%
\pgfpathlineto{\pgfqpoint{1.045530in}{1.265705in}}%
\pgfpathlineto{\pgfqpoint{1.044711in}{1.277116in}}%
\pgfpathlineto{\pgfqpoint{1.038778in}{1.288527in}}%
\pgfpathlineto{\pgfqpoint{1.035746in}{1.290102in}}%
\pgfpathlineto{\pgfqpoint{1.024013in}{1.291185in}}%
\pgfpathlineto{\pgfqpoint{1.012281in}{1.292017in}}%
\pgfpathlineto{\pgfqpoint{1.000549in}{1.293049in}}%
\pgfpathlineto{\pgfqpoint{0.988817in}{1.294278in}}%
\pgfpathlineto{\pgfqpoint{0.977085in}{1.295694in}}%
\pgfpathlineto{\pgfqpoint{0.965352in}{1.296753in}}%
\pgfpathlineto{\pgfqpoint{0.953620in}{1.293119in}}%
\pgfpathlineto{\pgfqpoint{0.943254in}{1.288527in}}%
\pgfpathlineto{\pgfqpoint{0.941888in}{1.288060in}}%
\pgfpathlineto{\pgfqpoint{0.930156in}{1.283383in}}%
\pgfpathlineto{\pgfqpoint{0.918424in}{1.278781in}}%
\pgfpathlineto{\pgfqpoint{0.915196in}{1.277116in}}%
\pgfpathlineto{\pgfqpoint{0.906691in}{1.269420in}}%
\pgfpathlineto{\pgfqpoint{0.902834in}{1.265705in}}%
\pgfpathlineto{\pgfqpoint{0.894959in}{1.260287in}}%
\pgfpathlineto{\pgfqpoint{0.886581in}{1.254295in}}%
\pgfpathlineto{\pgfqpoint{0.883227in}{1.250604in}}%
\pgfpathlineto{\pgfqpoint{0.872983in}{1.242884in}}%
\pgfpathlineto{\pgfqpoint{0.883227in}{1.236642in}}%
\pgfpathlineto{\pgfqpoint{0.891654in}{1.231473in}}%
\pgfpathlineto{\pgfqpoint{0.894959in}{1.229498in}}%
\pgfpathlineto{\pgfqpoint{0.906691in}{1.222453in}}%
\pgfpathlineto{\pgfqpoint{0.910655in}{1.220062in}}%
\pgfpathlineto{\pgfqpoint{0.918424in}{1.215488in}}%
\pgfpathlineto{\pgfqpoint{0.929976in}{1.208651in}}%
\pgfpathlineto{\pgfqpoint{0.930156in}{1.208547in}}%
\pgfpathlineto{\pgfqpoint{0.941888in}{1.201728in}}%
\pgfpathlineto{\pgfqpoint{0.949591in}{1.197241in}}%
\pgfpathlineto{\pgfqpoint{0.953620in}{1.194907in}}%
\pgfpathlineto{\pgfqpoint{0.965352in}{1.188019in}}%
\pgfpathlineto{\pgfqpoint{0.968286in}{1.185830in}}%
\pgfpathlineto{\pgfqpoint{0.977085in}{1.179927in}}%
\pgfpathlineto{\pgfqpoint{0.983509in}{1.174419in}}%
\pgfpathlineto{\pgfqpoint{0.988817in}{1.172260in}}%
\pgfpathlineto{\pgfqpoint{1.000549in}{1.164570in}}%
\pgfpathlineto{\pgfqpoint{1.002310in}{1.163008in}}%
\pgfpathlineto{\pgfqpoint{1.012281in}{1.158157in}}%
\pgfpathlineto{\pgfqpoint{1.020768in}{1.151598in}}%
\pgfpathlineto{\pgfqpoint{1.024013in}{1.149166in}}%
\pgfpathlineto{\pgfqpoint{1.035746in}{1.140590in}}%
\pgfpathlineto{\pgfqpoint{1.036339in}{1.140187in}}%
\pgfpathlineto{\pgfqpoint{1.047478in}{1.131639in}}%
\pgfpathlineto{\pgfqpoint{1.051953in}{1.128776in}}%
\pgfpathlineto{\pgfqpoint{1.059210in}{1.123413in}}%
\pgfpathlineto{\pgfqpoint{1.064224in}{1.117365in}}%
\pgfpathlineto{\pgfqpoint{1.070942in}{1.112208in}}%
\pgfpathlineto{\pgfqpoint{1.078460in}{1.105954in}}%
\pgfpathlineto{\pgfqpoint{1.078877in}{1.094544in}}%
\pgfpathlineto{\pgfqpoint{1.070942in}{1.088284in}}%
\pgfpathlineto{\pgfqpoint{1.059210in}{1.086171in}}%
\pgfpathlineto{\pgfqpoint{1.047478in}{1.086462in}}%
\pgfpathlineto{\pgfqpoint{1.035746in}{1.086736in}}%
\pgfpathlineto{\pgfqpoint{1.024013in}{1.087402in}}%
\pgfpathlineto{\pgfqpoint{1.012281in}{1.087920in}}%
\pgfpathlineto{\pgfqpoint{1.000549in}{1.088408in}}%
\pgfpathlineto{\pgfqpoint{0.988817in}{1.089206in}}%
\pgfpathlineto{\pgfqpoint{0.977085in}{1.092287in}}%
\pgfpathlineto{\pgfqpoint{0.970384in}{1.094544in}}%
\pgfpathlineto{\pgfqpoint{0.965352in}{1.095671in}}%
\pgfpathlineto{\pgfqpoint{0.953620in}{1.098305in}}%
\pgfpathlineto{\pgfqpoint{0.941888in}{1.101313in}}%
\pgfpathlineto{\pgfqpoint{0.930156in}{1.104632in}}%
\pgfpathlineto{\pgfqpoint{0.925720in}{1.105954in}}%
\pgfpathlineto{\pgfqpoint{0.918424in}{1.109058in}}%
\pgfpathlineto{\pgfqpoint{0.906691in}{1.112591in}}%
\pgfpathlineto{\pgfqpoint{0.894959in}{1.114933in}}%
\pgfpathlineto{\pgfqpoint{0.883227in}{1.116682in}}%
\pgfpathlineto{\pgfqpoint{0.877962in}{1.117365in}}%
\pgfpathlineto{\pgfqpoint{0.871495in}{1.118278in}}%
\pgfpathlineto{\pgfqpoint{0.859762in}{1.119825in}}%
\pgfpathlineto{\pgfqpoint{0.848030in}{1.121260in}}%
\pgfpathlineto{\pgfqpoint{0.836298in}{1.122590in}}%
\pgfpathlineto{\pgfqpoint{0.824566in}{1.118863in}}%
\pgfpathlineto{\pgfqpoint{0.814257in}{1.117365in}}%
\pgfpathlineto{\pgfqpoint{0.812834in}{1.117229in}}%
\pgfpathlineto{\pgfqpoint{0.812570in}{1.117365in}}%
\pgfpathlineto{\pgfqpoint{0.801101in}{1.125406in}}%
\pgfpathlineto{\pgfqpoint{0.797121in}{1.128776in}}%
\pgfpathlineto{\pgfqpoint{0.790037in}{1.140187in}}%
\pgfpathlineto{\pgfqpoint{0.789369in}{1.141918in}}%
\pgfpathlineto{\pgfqpoint{0.785382in}{1.151598in}}%
\pgfpathlineto{\pgfqpoint{0.781892in}{1.163008in}}%
\pgfpathlineto{\pgfqpoint{0.778931in}{1.174419in}}%
\pgfpathlineto{\pgfqpoint{0.777637in}{1.177818in}}%
\pgfpathlineto{\pgfqpoint{0.774136in}{1.185830in}}%
\pgfpathlineto{\pgfqpoint{0.767996in}{1.197241in}}%
\pgfpathlineto{\pgfqpoint{0.765905in}{1.200446in}}%
\pgfpathlineto{\pgfqpoint{0.760282in}{1.208651in}}%
\pgfpathlineto{\pgfqpoint{0.754173in}{1.218024in}}%
\pgfpathlineto{\pgfqpoint{0.752774in}{1.220062in}}%
\pgfpathlineto{\pgfqpoint{0.745070in}{1.231473in}}%
\pgfpathlineto{\pgfqpoint{0.742440in}{1.235459in}}%
\pgfpathlineto{\pgfqpoint{0.737316in}{1.242884in}}%
\pgfpathlineto{\pgfqpoint{0.730708in}{1.252543in}}%
\pgfpathlineto{\pgfqpoint{0.729041in}{1.254295in}}%
\pgfpathlineto{\pgfqpoint{0.718976in}{1.262838in}}%
\pgfpathlineto{\pgfqpoint{0.715582in}{1.265705in}}%
\pgfpathlineto{\pgfqpoint{0.707244in}{1.272890in}}%
\pgfpathlineto{\pgfqpoint{0.702021in}{1.277116in}}%
\pgfpathlineto{\pgfqpoint{0.695512in}{1.282799in}}%
\pgfpathlineto{\pgfqpoint{0.685394in}{1.288527in}}%
\pgfpathlineto{\pgfqpoint{0.683779in}{1.289909in}}%
\pgfpathlineto{\pgfqpoint{0.672579in}{1.288527in}}%
\pgfpathlineto{\pgfqpoint{0.672047in}{1.287935in}}%
\pgfpathlineto{\pgfqpoint{0.670738in}{1.277116in}}%
\pgfpathlineto{\pgfqpoint{0.672047in}{1.270216in}}%
\pgfpathlineto{\pgfqpoint{0.672884in}{1.265705in}}%
\pgfpathlineto{\pgfqpoint{0.675604in}{1.254295in}}%
\pgfpathlineto{\pgfqpoint{0.678714in}{1.242884in}}%
\pgfpathlineto{\pgfqpoint{0.682170in}{1.231473in}}%
\pgfpathlineto{\pgfqpoint{0.683779in}{1.225938in}}%
\pgfpathlineto{\pgfqpoint{0.685358in}{1.220062in}}%
\pgfpathlineto{\pgfqpoint{0.688323in}{1.208651in}}%
\pgfpathlineto{\pgfqpoint{0.695512in}{1.197722in}}%
\pgfpathlineto{\pgfqpoint{0.695856in}{1.197241in}}%
\pgfpathlineto{\pgfqpoint{0.702200in}{1.185830in}}%
\pgfpathlineto{\pgfqpoint{0.707244in}{1.180166in}}%
\pgfpathlineto{\pgfqpoint{0.711212in}{1.174419in}}%
\pgfpathlineto{\pgfqpoint{0.718976in}{1.166537in}}%
\pgfpathlineto{\pgfqpoint{0.721696in}{1.163008in}}%
\pgfpathlineto{\pgfqpoint{0.730708in}{1.154483in}}%
\pgfpathlineto{\pgfqpoint{0.733146in}{1.151598in}}%
\pgfpathlineto{\pgfqpoint{0.742440in}{1.141853in}}%
\pgfpathlineto{\pgfqpoint{0.744160in}{1.140187in}}%
\pgfpathlineto{\pgfqpoint{0.754173in}{1.131865in}}%
\pgfpathlineto{\pgfqpoint{0.757445in}{1.128776in}}%
\pgfpathlineto{\pgfqpoint{0.765905in}{1.120557in}}%
\pgfpathlineto{\pgfqpoint{0.768205in}{1.117365in}}%
\pgfpathlineto{\pgfqpoint{0.773367in}{1.105954in}}%
\pgfpathlineto{\pgfqpoint{0.773034in}{1.094544in}}%
\pgfpathlineto{\pgfqpoint{0.777637in}{1.084642in}}%
\pgfpathlineto{\pgfqpoint{0.778231in}{1.083133in}}%
\pgfpathlineto{\pgfqpoint{0.788187in}{1.071722in}}%
\pgfpathlineto{\pgfqpoint{0.789369in}{1.070520in}}%
\pgfpathlineto{\pgfqpoint{0.801101in}{1.061461in}}%
\pgfpathlineto{\pgfqpoint{0.805949in}{1.060311in}}%
\pgfpathlineto{\pgfqpoint{0.812834in}{1.058251in}}%
\pgfpathlineto{\pgfqpoint{0.816681in}{1.060311in}}%
\pgfpathlineto{\pgfqpoint{0.824311in}{1.071722in}}%
\pgfpathlineto{\pgfqpoint{0.824566in}{1.071979in}}%
\pgfpathlineto{\pgfqpoint{0.824847in}{1.071722in}}%
\pgfpathlineto{\pgfqpoint{0.830057in}{1.060311in}}%
\pgfpathlineto{\pgfqpoint{0.835842in}{1.048901in}}%
\pgfpathlineto{\pgfqpoint{0.836298in}{1.048172in}}%
\pgfpathlineto{\pgfqpoint{0.843382in}{1.037490in}}%
\pgfpathlineto{\pgfqpoint{0.848030in}{1.032746in}}%
\pgfpathlineto{\pgfqpoint{0.854718in}{1.026079in}}%
\pgfpathlineto{\pgfqpoint{0.859762in}{1.020891in}}%
\pgfpathlineto{\pgfqpoint{0.866080in}{1.014668in}}%
\pgfpathlineto{\pgfqpoint{0.871495in}{1.009130in}}%
\pgfpathlineto{\pgfqpoint{0.878740in}{1.003257in}}%
\pgfpathlineto{\pgfqpoint{0.883227in}{0.997766in}}%
\pgfpathlineto{\pgfqpoint{0.889694in}{1.003257in}}%
\pgfpathlineto{\pgfqpoint{0.894959in}{1.012627in}}%
\pgfpathlineto{\pgfqpoint{0.896073in}{1.014668in}}%
\pgfpathlineto{\pgfqpoint{0.902705in}{1.026079in}}%
\pgfpathlineto{\pgfqpoint{0.906691in}{1.032591in}}%
\pgfpathlineto{\pgfqpoint{0.909989in}{1.037490in}}%
\pgfpathlineto{\pgfqpoint{0.918424in}{1.045055in}}%
\pgfpathlineto{\pgfqpoint{0.926323in}{1.048901in}}%
\pgfpathlineto{\pgfqpoint{0.930156in}{1.050469in}}%
\pgfpathlineto{\pgfqpoint{0.941888in}{1.054290in}}%
\pgfpathlineto{\pgfqpoint{0.953620in}{1.057426in}}%
\pgfpathlineto{\pgfqpoint{0.965352in}{1.060119in}}%
\pgfpathlineto{\pgfqpoint{0.966225in}{1.060311in}}%
\pgfpathlineto{\pgfqpoint{0.977085in}{1.063351in}}%
\pgfpathlineto{\pgfqpoint{0.988817in}{1.066097in}}%
\pgfpathlineto{\pgfqpoint{1.000549in}{1.067639in}}%
\pgfpathlineto{\pgfqpoint{1.012281in}{1.069669in}}%
\pgfpathlineto{\pgfqpoint{1.024013in}{1.071560in}}%
\pgfpathlineto{\pgfqpoint{1.025176in}{1.071722in}}%
\pgfpathlineto{\pgfqpoint{1.035746in}{1.072999in}}%
\pgfpathlineto{\pgfqpoint{1.047478in}{1.074275in}}%
\pgfpathlineto{\pgfqpoint{1.054657in}{1.071722in}}%
\pgfpathlineto{\pgfqpoint{1.047478in}{1.062450in}}%
\pgfpathlineto{\pgfqpoint{1.046821in}{1.060311in}}%
\pgfpathlineto{\pgfqpoint{1.035746in}{1.053538in}}%
\pgfpathlineto{\pgfqpoint{1.027415in}{1.048901in}}%
\pgfpathlineto{\pgfqpoint{1.024013in}{1.046428in}}%
\pgfpathlineto{\pgfqpoint{1.012281in}{1.038974in}}%
\pgfpathlineto{\pgfqpoint{1.009979in}{1.037490in}}%
\pgfpathlineto{\pgfqpoint{1.000549in}{1.030728in}}%
\pgfpathlineto{\pgfqpoint{0.993430in}{1.026079in}}%
\pgfpathlineto{\pgfqpoint{0.988817in}{1.021554in}}%
\pgfpathlineto{\pgfqpoint{0.977085in}{1.016414in}}%
\pgfpathlineto{\pgfqpoint{0.973576in}{1.014668in}}%
\pgfpathlineto{\pgfqpoint{0.965352in}{1.009895in}}%
\pgfpathlineto{\pgfqpoint{0.953620in}{1.005464in}}%
\pgfpathlineto{\pgfqpoint{0.948764in}{1.003257in}}%
\pgfpathlineto{\pgfqpoint{0.941888in}{1.000173in}}%
\pgfpathlineto{\pgfqpoint{0.930156in}{0.995370in}}%
\pgfpathlineto{\pgfqpoint{0.920557in}{0.991847in}}%
\pgfpathlineto{\pgfqpoint{0.918424in}{0.991078in}}%
\pgfpathlineto{\pgfqpoint{0.906691in}{0.987231in}}%
\pgfpathlineto{\pgfqpoint{0.894959in}{0.987020in}}%
\pgfpathlineto{\pgfqpoint{0.888856in}{0.980436in}}%
\pgfpathlineto{\pgfqpoint{0.883227in}{0.972396in}}%
\pgfpathlineto{\pgfqpoint{0.880262in}{0.969025in}}%
\pgfpathlineto{\pgfqpoint{0.873040in}{0.957614in}}%
\pgfpathlineto{\pgfqpoint{0.871495in}{0.954450in}}%
\pgfpathlineto{\pgfqpoint{0.867059in}{0.946204in}}%
\pgfpathlineto{\pgfqpoint{0.863496in}{0.934793in}}%
\pgfpathlineto{\pgfqpoint{0.871495in}{0.928508in}}%
\pgfpathlineto{\pgfqpoint{0.883227in}{0.924346in}}%
\pgfpathlineto{\pgfqpoint{0.886572in}{0.923382in}}%
\pgfpathlineto{\pgfqpoint{0.894959in}{0.919402in}}%
\pgfpathlineto{\pgfqpoint{0.904475in}{0.911971in}}%
\pgfpathlineto{\pgfqpoint{0.906691in}{0.908262in}}%
\pgfpathlineto{\pgfqpoint{0.914178in}{0.900560in}}%
\pgfpathlineto{\pgfqpoint{0.918424in}{0.890820in}}%
\pgfpathlineto{\pgfqpoint{0.919924in}{0.889150in}}%
\pgfpathlineto{\pgfqpoint{0.926698in}{0.877739in}}%
\pgfpathlineto{\pgfqpoint{0.930156in}{0.871347in}}%
\pgfpathlineto{\pgfqpoint{0.932843in}{0.866328in}}%
\pgfpathlineto{\pgfqpoint{0.939171in}{0.854917in}}%
\pgfpathlineto{\pgfqpoint{0.941888in}{0.850384in}}%
\pgfpathlineto{\pgfqpoint{0.946573in}{0.843507in}}%
\pgfpathlineto{\pgfqpoint{0.953620in}{0.835100in}}%
\pgfpathlineto{\pgfqpoint{0.957385in}{0.832096in}}%
\pgfpathlineto{\pgfqpoint{0.965352in}{0.824064in}}%
\pgfpathlineto{\pgfqpoint{0.972989in}{0.832096in}}%
\pgfpathlineto{\pgfqpoint{0.977085in}{0.842484in}}%
\pgfpathlineto{\pgfqpoint{0.988817in}{0.842476in}}%
\pgfpathlineto{\pgfqpoint{1.000549in}{0.842660in}}%
\pgfpathlineto{\pgfqpoint{1.012281in}{0.843284in}}%
\pgfpathlineto{\pgfqpoint{1.014011in}{0.843507in}}%
\pgfpathlineto{\pgfqpoint{1.024013in}{0.844788in}}%
\pgfpathlineto{\pgfqpoint{1.035746in}{0.846257in}}%
\pgfpathlineto{\pgfqpoint{1.047478in}{0.847685in}}%
\pgfpathlineto{\pgfqpoint{1.059210in}{0.847199in}}%
\pgfpathlineto{\pgfqpoint{1.070942in}{0.849684in}}%
\pgfpathlineto{\pgfqpoint{1.082674in}{0.854249in}}%
\pgfpathlineto{\pgfqpoint{1.085970in}{0.854917in}}%
\pgfpathlineto{\pgfqpoint{1.082674in}{0.855118in}}%
\pgfpathlineto{\pgfqpoint{1.070942in}{0.856774in}}%
\pgfpathlineto{\pgfqpoint{1.059210in}{0.859118in}}%
\pgfpathlineto{\pgfqpoint{1.047478in}{0.862921in}}%
\pgfpathlineto{\pgfqpoint{1.043229in}{0.866328in}}%
\pgfpathlineto{\pgfqpoint{1.035746in}{0.868089in}}%
\pgfpathlineto{\pgfqpoint{1.024013in}{0.871562in}}%
\pgfpathlineto{\pgfqpoint{1.012281in}{0.877165in}}%
\pgfpathlineto{\pgfqpoint{1.011631in}{0.877739in}}%
\pgfpathlineto{\pgfqpoint{1.000549in}{0.885065in}}%
\pgfpathlineto{\pgfqpoint{0.993755in}{0.889150in}}%
\pgfpathlineto{\pgfqpoint{0.988817in}{0.892651in}}%
\pgfpathlineto{\pgfqpoint{0.977085in}{0.898944in}}%
\pgfpathlineto{\pgfqpoint{0.970858in}{0.900560in}}%
\pgfpathlineto{\pgfqpoint{0.965352in}{0.902146in}}%
\pgfpathlineto{\pgfqpoint{0.953620in}{0.905356in}}%
\pgfpathlineto{\pgfqpoint{0.941888in}{0.910097in}}%
\pgfpathlineto{\pgfqpoint{0.939795in}{0.911971in}}%
\pgfpathlineto{\pgfqpoint{0.930333in}{0.923382in}}%
\pgfpathlineto{\pgfqpoint{0.930187in}{0.934793in}}%
\pgfpathlineto{\pgfqpoint{0.937671in}{0.946204in}}%
\pgfpathlineto{\pgfqpoint{0.941888in}{0.950134in}}%
\pgfpathlineto{\pgfqpoint{0.953620in}{0.957349in}}%
\pgfpathlineto{\pgfqpoint{0.954165in}{0.957614in}}%
\pgfpathlineto{\pgfqpoint{0.965352in}{0.962418in}}%
\pgfpathlineto{\pgfqpoint{0.977085in}{0.965562in}}%
\pgfpathlineto{\pgfqpoint{0.987106in}{0.969025in}}%
\pgfpathlineto{\pgfqpoint{0.988817in}{0.970279in}}%
\pgfpathlineto{\pgfqpoint{1.000549in}{0.974109in}}%
\pgfpathlineto{\pgfqpoint{1.012281in}{0.976261in}}%
\pgfpathlineto{\pgfqpoint{1.024013in}{0.979141in}}%
\pgfpathlineto{\pgfqpoint{1.030404in}{0.980436in}}%
\pgfpathlineto{\pgfqpoint{1.035746in}{0.982162in}}%
\pgfpathlineto{\pgfqpoint{1.047478in}{0.981216in}}%
\pgfpathlineto{\pgfqpoint{1.053639in}{0.980436in}}%
\pgfpathlineto{\pgfqpoint{1.059210in}{0.979893in}}%
\pgfpathlineto{\pgfqpoint{1.070942in}{0.978714in}}%
\pgfpathlineto{\pgfqpoint{1.082674in}{0.977493in}}%
\pgfpathlineto{\pgfqpoint{1.094407in}{0.975966in}}%
\pgfpathlineto{\pgfqpoint{1.106139in}{0.973588in}}%
\pgfpathlineto{\pgfqpoint{1.117871in}{0.972292in}}%
\pgfpathlineto{\pgfqpoint{1.129603in}{0.972233in}}%
\pgfpathlineto{\pgfqpoint{1.141335in}{0.972773in}}%
\pgfpathlineto{\pgfqpoint{1.153068in}{0.971689in}}%
\pgfpathlineto{\pgfqpoint{1.161476in}{0.969025in}}%
\pgfpathlineto{\pgfqpoint{1.164800in}{0.968511in}}%
\pgfpathlineto{\pgfqpoint{1.176532in}{0.966797in}}%
\pgfpathlineto{\pgfqpoint{1.188264in}{0.965109in}}%
\pgfpathlineto{\pgfqpoint{1.199996in}{0.963586in}}%
\pgfpathlineto{\pgfqpoint{1.211729in}{0.961436in}}%
\pgfpathlineto{\pgfqpoint{1.223461in}{0.961525in}}%
\pgfpathlineto{\pgfqpoint{1.235193in}{0.960844in}}%
\pgfpathlineto{\pgfqpoint{1.242617in}{0.957614in}}%
\pgfpathlineto{\pgfqpoint{1.246925in}{0.956944in}}%
\pgfpathlineto{\pgfqpoint{1.258658in}{0.954683in}}%
\pgfpathlineto{\pgfqpoint{1.270390in}{0.952006in}}%
\pgfpathlineto{\pgfqpoint{1.282122in}{0.946634in}}%
\pgfpathlineto{\pgfqpoint{1.282661in}{0.946204in}}%
\pgfpathlineto{\pgfqpoint{1.288516in}{0.934793in}}%
\pgfpathlineto{\pgfqpoint{1.286173in}{0.923382in}}%
\pgfpathlineto{\pgfqpoint{1.282122in}{0.920866in}}%
\pgfpathlineto{\pgfqpoint{1.270390in}{0.915233in}}%
\pgfpathlineto{\pgfqpoint{1.263759in}{0.911971in}}%
\pgfpathlineto{\pgfqpoint{1.258658in}{0.909675in}}%
\pgfpathlineto{\pgfqpoint{1.246925in}{0.901561in}}%
\pgfpathlineto{\pgfqpoint{1.246069in}{0.900560in}}%
\pgfpathlineto{\pgfqpoint{1.235905in}{0.889150in}}%
\pgfpathlineto{\pgfqpoint{1.235193in}{0.888173in}}%
\pgfpathlineto{\pgfqpoint{1.223461in}{0.878305in}}%
\pgfpathlineto{\pgfqpoint{1.222286in}{0.877739in}}%
\pgfpathlineto{\pgfqpoint{1.211729in}{0.867029in}}%
\pgfpathlineto{\pgfqpoint{1.205601in}{0.866328in}}%
\pgfpathlineto{\pgfqpoint{1.211729in}{0.864819in}}%
\pgfpathlineto{\pgfqpoint{1.223461in}{0.856876in}}%
\pgfpathlineto{\pgfqpoint{1.226594in}{0.854917in}}%
\pgfpathlineto{\pgfqpoint{1.235193in}{0.849027in}}%
\pgfpathlineto{\pgfqpoint{1.243248in}{0.843507in}}%
\pgfpathlineto{\pgfqpoint{1.246925in}{0.841000in}}%
\pgfpathlineto{\pgfqpoint{1.258658in}{0.835750in}}%
\pgfpathlineto{\pgfqpoint{1.265532in}{0.832096in}}%
\pgfpathlineto{\pgfqpoint{1.270390in}{0.829194in}}%
\pgfpathlineto{\pgfqpoint{1.282122in}{0.822410in}}%
\pgfpathlineto{\pgfqpoint{1.285553in}{0.820685in}}%
\pgfpathlineto{\pgfqpoint{1.293854in}{0.815989in}}%
\pgfpathlineto{\pgfqpoint{1.305586in}{0.809764in}}%
\pgfpathlineto{\pgfqpoint{1.306501in}{0.809274in}}%
\pgfpathlineto{\pgfqpoint{1.317319in}{0.802798in}}%
\pgfpathlineto{\pgfqpoint{1.321349in}{0.797863in}}%
\pgfpathclose%
\pgfpathmoveto{\pgfqpoint{1.288777in}{0.843507in}}%
\pgfpathlineto{\pgfqpoint{1.282122in}{0.847270in}}%
\pgfpathlineto{\pgfqpoint{1.270390in}{0.853947in}}%
\pgfpathlineto{\pgfqpoint{1.268687in}{0.854917in}}%
\pgfpathlineto{\pgfqpoint{1.258658in}{0.860646in}}%
\pgfpathlineto{\pgfqpoint{1.252115in}{0.866328in}}%
\pgfpathlineto{\pgfqpoint{1.246925in}{0.871653in}}%
\pgfpathlineto{\pgfqpoint{1.238351in}{0.877739in}}%
\pgfpathlineto{\pgfqpoint{1.244181in}{0.889150in}}%
\pgfpathlineto{\pgfqpoint{1.246925in}{0.892207in}}%
\pgfpathlineto{\pgfqpoint{1.255582in}{0.900560in}}%
\pgfpathlineto{\pgfqpoint{1.258658in}{0.902780in}}%
\pgfpathlineto{\pgfqpoint{1.270390in}{0.908315in}}%
\pgfpathlineto{\pgfqpoint{1.278721in}{0.911971in}}%
\pgfpathlineto{\pgfqpoint{1.282122in}{0.913643in}}%
\pgfpathlineto{\pgfqpoint{1.293854in}{0.918640in}}%
\pgfpathlineto{\pgfqpoint{1.297881in}{0.923382in}}%
\pgfpathlineto{\pgfqpoint{1.298518in}{0.934793in}}%
\pgfpathlineto{\pgfqpoint{1.294090in}{0.946204in}}%
\pgfpathlineto{\pgfqpoint{1.293854in}{0.946519in}}%
\pgfpathlineto{\pgfqpoint{1.282122in}{0.955795in}}%
\pgfpathlineto{\pgfqpoint{1.276008in}{0.957614in}}%
\pgfpathlineto{\pgfqpoint{1.282122in}{0.960505in}}%
\pgfpathlineto{\pgfqpoint{1.293854in}{0.966340in}}%
\pgfpathlineto{\pgfqpoint{1.301410in}{0.969025in}}%
\pgfpathlineto{\pgfqpoint{1.305586in}{0.970372in}}%
\pgfpathlineto{\pgfqpoint{1.317319in}{0.974198in}}%
\pgfpathlineto{\pgfqpoint{1.329051in}{0.978095in}}%
\pgfpathlineto{\pgfqpoint{1.340155in}{0.969025in}}%
\pgfpathlineto{\pgfqpoint{1.340783in}{0.968572in}}%
\pgfpathlineto{\pgfqpoint{1.352515in}{0.963698in}}%
\pgfpathlineto{\pgfqpoint{1.364247in}{0.958819in}}%
\pgfpathlineto{\pgfqpoint{1.366287in}{0.957614in}}%
\pgfpathlineto{\pgfqpoint{1.364247in}{0.952929in}}%
\pgfpathlineto{\pgfqpoint{1.361766in}{0.946204in}}%
\pgfpathlineto{\pgfqpoint{1.352515in}{0.939607in}}%
\pgfpathlineto{\pgfqpoint{1.346787in}{0.934793in}}%
\pgfpathlineto{\pgfqpoint{1.340783in}{0.928226in}}%
\pgfpathlineto{\pgfqpoint{1.336521in}{0.923382in}}%
\pgfpathlineto{\pgfqpoint{1.329289in}{0.911971in}}%
\pgfpathlineto{\pgfqpoint{1.329051in}{0.909515in}}%
\pgfpathlineto{\pgfqpoint{1.327687in}{0.911971in}}%
\pgfpathlineto{\pgfqpoint{1.317319in}{0.914532in}}%
\pgfpathlineto{\pgfqpoint{1.310937in}{0.911971in}}%
\pgfpathlineto{\pgfqpoint{1.313388in}{0.900560in}}%
\pgfpathlineto{\pgfqpoint{1.317319in}{0.890885in}}%
\pgfpathlineto{\pgfqpoint{1.329051in}{0.893306in}}%
\pgfpathlineto{\pgfqpoint{1.329568in}{0.889150in}}%
\pgfpathlineto{\pgfqpoint{1.331571in}{0.877739in}}%
\pgfpathlineto{\pgfqpoint{1.333643in}{0.866328in}}%
\pgfpathlineto{\pgfqpoint{1.332027in}{0.854917in}}%
\pgfpathlineto{\pgfqpoint{1.329165in}{0.843507in}}%
\pgfpathlineto{\pgfqpoint{1.329051in}{0.843105in}}%
\pgfpathlineto{\pgfqpoint{1.328663in}{0.843507in}}%
\pgfpathlineto{\pgfqpoint{1.318004in}{0.854917in}}%
\pgfpathlineto{\pgfqpoint{1.317319in}{0.855656in}}%
\pgfpathlineto{\pgfqpoint{1.316634in}{0.854917in}}%
\pgfpathlineto{\pgfqpoint{1.306198in}{0.843507in}}%
\pgfpathlineto{\pgfqpoint{1.305586in}{0.842850in}}%
\pgfpathlineto{\pgfqpoint{1.293854in}{0.840629in}}%
\pgfpathclose%
\pgfpathmoveto{\pgfqpoint{0.905000in}{0.923382in}}%
\pgfpathlineto{\pgfqpoint{0.894959in}{0.925834in}}%
\pgfpathlineto{\pgfqpoint{0.883227in}{0.928977in}}%
\pgfpathlineto{\pgfqpoint{0.871495in}{0.933549in}}%
\pgfpathlineto{\pgfqpoint{0.869912in}{0.934793in}}%
\pgfpathlineto{\pgfqpoint{0.871495in}{0.939718in}}%
\pgfpathlineto{\pgfqpoint{0.873419in}{0.946204in}}%
\pgfpathlineto{\pgfqpoint{0.879006in}{0.957614in}}%
\pgfpathlineto{\pgfqpoint{0.883227in}{0.963130in}}%
\pgfpathlineto{\pgfqpoint{0.888648in}{0.969025in}}%
\pgfpathlineto{\pgfqpoint{0.894959in}{0.977063in}}%
\pgfpathlineto{\pgfqpoint{0.906691in}{0.977703in}}%
\pgfpathlineto{\pgfqpoint{0.918424in}{0.979311in}}%
\pgfpathlineto{\pgfqpoint{0.922525in}{0.980436in}}%
\pgfpathlineto{\pgfqpoint{0.930156in}{0.982693in}}%
\pgfpathlineto{\pgfqpoint{0.939100in}{0.980436in}}%
\pgfpathlineto{\pgfqpoint{0.941888in}{0.976786in}}%
\pgfpathlineto{\pgfqpoint{0.951121in}{0.969025in}}%
\pgfpathlineto{\pgfqpoint{0.941888in}{0.958427in}}%
\pgfpathlineto{\pgfqpoint{0.941786in}{0.957614in}}%
\pgfpathlineto{\pgfqpoint{0.930156in}{0.946964in}}%
\pgfpathlineto{\pgfqpoint{0.929691in}{0.946204in}}%
\pgfpathlineto{\pgfqpoint{0.922399in}{0.934793in}}%
\pgfpathlineto{\pgfqpoint{0.918424in}{0.923574in}}%
\pgfpathlineto{\pgfqpoint{0.917276in}{0.923382in}}%
\pgfpathlineto{\pgfqpoint{0.906691in}{0.922305in}}%
\pgfpathclose%
\pgfpathmoveto{\pgfqpoint{1.357418in}{0.969025in}}%
\pgfpathlineto{\pgfqpoint{1.352515in}{0.973635in}}%
\pgfpathlineto{\pgfqpoint{1.346983in}{0.980436in}}%
\pgfpathlineto{\pgfqpoint{1.352515in}{0.983291in}}%
\pgfpathlineto{\pgfqpoint{1.364247in}{0.988074in}}%
\pgfpathlineto{\pgfqpoint{1.375980in}{0.988713in}}%
\pgfpathlineto{\pgfqpoint{1.387712in}{0.986839in}}%
\pgfpathlineto{\pgfqpoint{1.396115in}{0.980436in}}%
\pgfpathlineto{\pgfqpoint{1.394015in}{0.969025in}}%
\pgfpathlineto{\pgfqpoint{1.387712in}{0.957774in}}%
\pgfpathlineto{\pgfqpoint{1.375980in}{0.961510in}}%
\pgfpathlineto{\pgfqpoint{1.364247in}{0.966607in}}%
\pgfpathclose%
\pgfpathmoveto{\pgfqpoint{1.083247in}{0.991847in}}%
\pgfpathlineto{\pgfqpoint{1.082674in}{0.994040in}}%
\pgfpathlineto{\pgfqpoint{1.080219in}{1.003257in}}%
\pgfpathlineto{\pgfqpoint{1.070942in}{1.011027in}}%
\pgfpathlineto{\pgfqpoint{1.068348in}{1.014668in}}%
\pgfpathlineto{\pgfqpoint{1.059210in}{1.025692in}}%
\pgfpathlineto{\pgfqpoint{1.058893in}{1.026079in}}%
\pgfpathlineto{\pgfqpoint{1.054921in}{1.037490in}}%
\pgfpathlineto{\pgfqpoint{1.059210in}{1.044646in}}%
\pgfpathlineto{\pgfqpoint{1.062719in}{1.048901in}}%
\pgfpathlineto{\pgfqpoint{1.070942in}{1.052483in}}%
\pgfpathlineto{\pgfqpoint{1.075556in}{1.060311in}}%
\pgfpathlineto{\pgfqpoint{1.082674in}{1.065983in}}%
\pgfpathlineto{\pgfqpoint{1.087836in}{1.071722in}}%
\pgfpathlineto{\pgfqpoint{1.082674in}{1.081911in}}%
\pgfpathlineto{\pgfqpoint{1.082106in}{1.083133in}}%
\pgfpathlineto{\pgfqpoint{1.082674in}{1.083535in}}%
\pgfpathlineto{\pgfqpoint{1.087631in}{1.094544in}}%
\pgfpathlineto{\pgfqpoint{1.087753in}{1.105954in}}%
\pgfpathlineto{\pgfqpoint{1.082674in}{1.115388in}}%
\pgfpathlineto{\pgfqpoint{1.079752in}{1.117365in}}%
\pgfpathlineto{\pgfqpoint{1.075358in}{1.128776in}}%
\pgfpathlineto{\pgfqpoint{1.070942in}{1.131152in}}%
\pgfpathlineto{\pgfqpoint{1.059210in}{1.137050in}}%
\pgfpathlineto{\pgfqpoint{1.053297in}{1.140187in}}%
\pgfpathlineto{\pgfqpoint{1.047478in}{1.142772in}}%
\pgfpathlineto{\pgfqpoint{1.035746in}{1.150207in}}%
\pgfpathlineto{\pgfqpoint{1.033849in}{1.151598in}}%
\pgfpathlineto{\pgfqpoint{1.024013in}{1.162402in}}%
\pgfpathlineto{\pgfqpoint{1.023129in}{1.163008in}}%
\pgfpathlineto{\pgfqpoint{1.015626in}{1.174419in}}%
\pgfpathlineto{\pgfqpoint{1.012281in}{1.177957in}}%
\pgfpathlineto{\pgfqpoint{1.002529in}{1.185830in}}%
\pgfpathlineto{\pgfqpoint{1.000549in}{1.187435in}}%
\pgfpathlineto{\pgfqpoint{0.988817in}{1.197047in}}%
\pgfpathlineto{\pgfqpoint{0.988567in}{1.197241in}}%
\pgfpathlineto{\pgfqpoint{0.977085in}{1.205663in}}%
\pgfpathlineto{\pgfqpoint{0.972903in}{1.208651in}}%
\pgfpathlineto{\pgfqpoint{0.965352in}{1.214066in}}%
\pgfpathlineto{\pgfqpoint{0.955111in}{1.220062in}}%
\pgfpathlineto{\pgfqpoint{0.953620in}{1.220957in}}%
\pgfpathlineto{\pgfqpoint{0.941888in}{1.228016in}}%
\pgfpathlineto{\pgfqpoint{0.936126in}{1.231473in}}%
\pgfpathlineto{\pgfqpoint{0.930156in}{1.235148in}}%
\pgfpathlineto{\pgfqpoint{0.918424in}{1.242344in}}%
\pgfpathlineto{\pgfqpoint{0.917543in}{1.242884in}}%
\pgfpathlineto{\pgfqpoint{0.917002in}{1.254295in}}%
\pgfpathlineto{\pgfqpoint{0.918424in}{1.255273in}}%
\pgfpathlineto{\pgfqpoint{0.930156in}{1.263299in}}%
\pgfpathlineto{\pgfqpoint{0.933673in}{1.265705in}}%
\pgfpathlineto{\pgfqpoint{0.941888in}{1.270475in}}%
\pgfpathlineto{\pgfqpoint{0.953620in}{1.275602in}}%
\pgfpathlineto{\pgfqpoint{0.957677in}{1.277116in}}%
\pgfpathlineto{\pgfqpoint{0.965352in}{1.280178in}}%
\pgfpathlineto{\pgfqpoint{0.977085in}{1.282970in}}%
\pgfpathlineto{\pgfqpoint{0.988817in}{1.281939in}}%
\pgfpathlineto{\pgfqpoint{1.000549in}{1.280969in}}%
\pgfpathlineto{\pgfqpoint{1.012281in}{1.280017in}}%
\pgfpathlineto{\pgfqpoint{1.018747in}{1.277116in}}%
\pgfpathlineto{\pgfqpoint{1.024013in}{1.266078in}}%
\pgfpathlineto{\pgfqpoint{1.024163in}{1.265705in}}%
\pgfpathlineto{\pgfqpoint{1.024605in}{1.254295in}}%
\pgfpathlineto{\pgfqpoint{1.025982in}{1.242884in}}%
\pgfpathlineto{\pgfqpoint{1.027898in}{1.231473in}}%
\pgfpathlineto{\pgfqpoint{1.032318in}{1.220062in}}%
\pgfpathlineto{\pgfqpoint{1.035746in}{1.214823in}}%
\pgfpathlineto{\pgfqpoint{1.042404in}{1.208651in}}%
\pgfpathlineto{\pgfqpoint{1.047478in}{1.203934in}}%
\pgfpathlineto{\pgfqpoint{1.054519in}{1.197241in}}%
\pgfpathlineto{\pgfqpoint{1.059210in}{1.192760in}}%
\pgfpathlineto{\pgfqpoint{1.066331in}{1.185830in}}%
\pgfpathlineto{\pgfqpoint{1.070942in}{1.181357in}}%
\pgfpathlineto{\pgfqpoint{1.077987in}{1.174419in}}%
\pgfpathlineto{\pgfqpoint{1.082674in}{1.169943in}}%
\pgfpathlineto{\pgfqpoint{1.089635in}{1.163008in}}%
\pgfpathlineto{\pgfqpoint{1.094407in}{1.158539in}}%
\pgfpathlineto{\pgfqpoint{1.101556in}{1.151598in}}%
\pgfpathlineto{\pgfqpoint{1.106139in}{1.145707in}}%
\pgfpathlineto{\pgfqpoint{1.112026in}{1.140187in}}%
\pgfpathlineto{\pgfqpoint{1.106139in}{1.131784in}}%
\pgfpathlineto{\pgfqpoint{1.103424in}{1.128776in}}%
\pgfpathlineto{\pgfqpoint{1.106139in}{1.126883in}}%
\pgfpathlineto{\pgfqpoint{1.117871in}{1.122174in}}%
\pgfpathlineto{\pgfqpoint{1.129603in}{1.118290in}}%
\pgfpathlineto{\pgfqpoint{1.134334in}{1.117365in}}%
\pgfpathlineto{\pgfqpoint{1.141335in}{1.115116in}}%
\pgfpathlineto{\pgfqpoint{1.143545in}{1.105954in}}%
\pgfpathlineto{\pgfqpoint{1.141335in}{1.102465in}}%
\pgfpathlineto{\pgfqpoint{1.137766in}{1.094544in}}%
\pgfpathlineto{\pgfqpoint{1.137333in}{1.083133in}}%
\pgfpathlineto{\pgfqpoint{1.141138in}{1.071722in}}%
\pgfpathlineto{\pgfqpoint{1.141335in}{1.070513in}}%
\pgfpathlineto{\pgfqpoint{1.144420in}{1.060311in}}%
\pgfpathlineto{\pgfqpoint{1.141335in}{1.054037in}}%
\pgfpathlineto{\pgfqpoint{1.129603in}{1.055152in}}%
\pgfpathlineto{\pgfqpoint{1.117871in}{1.054859in}}%
\pgfpathlineto{\pgfqpoint{1.106139in}{1.051959in}}%
\pgfpathlineto{\pgfqpoint{1.100737in}{1.048901in}}%
\pgfpathlineto{\pgfqpoint{1.095418in}{1.037490in}}%
\pgfpathlineto{\pgfqpoint{1.106139in}{1.031034in}}%
\pgfpathlineto{\pgfqpoint{1.112011in}{1.026079in}}%
\pgfpathlineto{\pgfqpoint{1.117871in}{1.016755in}}%
\pgfpathlineto{\pgfqpoint{1.119307in}{1.014668in}}%
\pgfpathlineto{\pgfqpoint{1.119920in}{1.003257in}}%
\pgfpathlineto{\pgfqpoint{1.120468in}{0.991847in}}%
\pgfpathlineto{\pgfqpoint{1.117871in}{0.989903in}}%
\pgfpathlineto{\pgfqpoint{1.106139in}{0.989206in}}%
\pgfpathlineto{\pgfqpoint{1.094407in}{0.990850in}}%
\pgfpathclose%
\pgfpathmoveto{\pgfqpoint{0.868228in}{1.026079in}}%
\pgfpathlineto{\pgfqpoint{0.859762in}{1.034106in}}%
\pgfpathlineto{\pgfqpoint{0.856370in}{1.037490in}}%
\pgfpathlineto{\pgfqpoint{0.848030in}{1.048753in}}%
\pgfpathlineto{\pgfqpoint{0.847930in}{1.048901in}}%
\pgfpathlineto{\pgfqpoint{0.842394in}{1.060311in}}%
\pgfpathlineto{\pgfqpoint{0.836298in}{1.070854in}}%
\pgfpathlineto{\pgfqpoint{0.835863in}{1.071722in}}%
\pgfpathlineto{\pgfqpoint{0.824566in}{1.082044in}}%
\pgfpathlineto{\pgfqpoint{0.814351in}{1.071722in}}%
\pgfpathlineto{\pgfqpoint{0.812834in}{1.069851in}}%
\pgfpathlineto{\pgfqpoint{0.801101in}{1.068768in}}%
\pgfpathlineto{\pgfqpoint{0.797447in}{1.071722in}}%
\pgfpathlineto{\pgfqpoint{0.789369in}{1.078824in}}%
\pgfpathlineto{\pgfqpoint{0.785454in}{1.083133in}}%
\pgfpathlineto{\pgfqpoint{0.779767in}{1.094544in}}%
\pgfpathlineto{\pgfqpoint{0.780056in}{1.105954in}}%
\pgfpathlineto{\pgfqpoint{0.777637in}{1.114195in}}%
\pgfpathlineto{\pgfqpoint{0.776391in}{1.117365in}}%
\pgfpathlineto{\pgfqpoint{0.770453in}{1.128776in}}%
\pgfpathlineto{\pgfqpoint{0.765905in}{1.132892in}}%
\pgfpathlineto{\pgfqpoint{0.758095in}{1.140187in}}%
\pgfpathlineto{\pgfqpoint{0.754173in}{1.143393in}}%
\pgfpathlineto{\pgfqpoint{0.745704in}{1.151598in}}%
\pgfpathlineto{\pgfqpoint{0.742440in}{1.155354in}}%
\pgfpathlineto{\pgfqpoint{0.736049in}{1.163008in}}%
\pgfpathlineto{\pgfqpoint{0.730708in}{1.169619in}}%
\pgfpathlineto{\pgfqpoint{0.726816in}{1.174419in}}%
\pgfpathlineto{\pgfqpoint{0.718976in}{1.184298in}}%
\pgfpathlineto{\pgfqpoint{0.717747in}{1.185830in}}%
\pgfpathlineto{\pgfqpoint{0.709463in}{1.197241in}}%
\pgfpathlineto{\pgfqpoint{0.707244in}{1.200375in}}%
\pgfpathlineto{\pgfqpoint{0.701270in}{1.208651in}}%
\pgfpathlineto{\pgfqpoint{0.695931in}{1.220062in}}%
\pgfpathlineto{\pgfqpoint{0.695512in}{1.221500in}}%
\pgfpathlineto{\pgfqpoint{0.692809in}{1.231473in}}%
\pgfpathlineto{\pgfqpoint{0.689816in}{1.242884in}}%
\pgfpathlineto{\pgfqpoint{0.688333in}{1.254295in}}%
\pgfpathlineto{\pgfqpoint{0.689481in}{1.265705in}}%
\pgfpathlineto{\pgfqpoint{0.695512in}{1.270016in}}%
\pgfpathlineto{\pgfqpoint{0.704611in}{1.265705in}}%
\pgfpathlineto{\pgfqpoint{0.707244in}{1.265018in}}%
\pgfpathlineto{\pgfqpoint{0.718976in}{1.255393in}}%
\pgfpathlineto{\pgfqpoint{0.720270in}{1.254295in}}%
\pgfpathlineto{\pgfqpoint{0.730708in}{1.243328in}}%
\pgfpathlineto{\pgfqpoint{0.731012in}{1.242884in}}%
\pgfpathlineto{\pgfqpoint{0.738402in}{1.231473in}}%
\pgfpathlineto{\pgfqpoint{0.742440in}{1.225592in}}%
\pgfpathlineto{\pgfqpoint{0.746020in}{1.220062in}}%
\pgfpathlineto{\pgfqpoint{0.753791in}{1.208651in}}%
\pgfpathlineto{\pgfqpoint{0.754173in}{1.208077in}}%
\pgfpathlineto{\pgfqpoint{0.760984in}{1.197241in}}%
\pgfpathlineto{\pgfqpoint{0.765905in}{1.189786in}}%
\pgfpathlineto{\pgfqpoint{0.768035in}{1.185830in}}%
\pgfpathlineto{\pgfqpoint{0.772721in}{1.174419in}}%
\pgfpathlineto{\pgfqpoint{0.775600in}{1.163008in}}%
\pgfpathlineto{\pgfqpoint{0.777637in}{1.156903in}}%
\pgfpathlineto{\pgfqpoint{0.779052in}{1.151598in}}%
\pgfpathlineto{\pgfqpoint{0.782381in}{1.140187in}}%
\pgfpathlineto{\pgfqpoint{0.789369in}{1.129835in}}%
\pgfpathlineto{\pgfqpoint{0.790015in}{1.128776in}}%
\pgfpathlineto{\pgfqpoint{0.801101in}{1.119391in}}%
\pgfpathlineto{\pgfqpoint{0.803990in}{1.117365in}}%
\pgfpathlineto{\pgfqpoint{0.812834in}{1.112807in}}%
\pgfpathlineto{\pgfqpoint{0.824566in}{1.113367in}}%
\pgfpathlineto{\pgfqpoint{0.834251in}{1.117365in}}%
\pgfpathlineto{\pgfqpoint{0.836298in}{1.118103in}}%
\pgfpathlineto{\pgfqpoint{0.841999in}{1.117365in}}%
\pgfpathlineto{\pgfqpoint{0.848030in}{1.116736in}}%
\pgfpathlineto{\pgfqpoint{0.859762in}{1.115388in}}%
\pgfpathlineto{\pgfqpoint{0.871495in}{1.113796in}}%
\pgfpathlineto{\pgfqpoint{0.883227in}{1.111847in}}%
\pgfpathlineto{\pgfqpoint{0.894959in}{1.109377in}}%
\pgfpathlineto{\pgfqpoint{0.906666in}{1.105954in}}%
\pgfpathlineto{\pgfqpoint{0.906691in}{1.105946in}}%
\pgfpathlineto{\pgfqpoint{0.918424in}{1.102288in}}%
\pgfpathlineto{\pgfqpoint{0.930156in}{1.098935in}}%
\pgfpathlineto{\pgfqpoint{0.941888in}{1.095782in}}%
\pgfpathlineto{\pgfqpoint{0.946623in}{1.094544in}}%
\pgfpathlineto{\pgfqpoint{0.953620in}{1.090299in}}%
\pgfpathlineto{\pgfqpoint{0.965352in}{1.086572in}}%
\pgfpathlineto{\pgfqpoint{0.977085in}{1.083923in}}%
\pgfpathlineto{\pgfqpoint{0.981635in}{1.083133in}}%
\pgfpathlineto{\pgfqpoint{0.977085in}{1.081582in}}%
\pgfpathlineto{\pgfqpoint{0.965352in}{1.077623in}}%
\pgfpathlineto{\pgfqpoint{0.953620in}{1.074018in}}%
\pgfpathlineto{\pgfqpoint{0.946469in}{1.071722in}}%
\pgfpathlineto{\pgfqpoint{0.941888in}{1.070372in}}%
\pgfpathlineto{\pgfqpoint{0.930156in}{1.067541in}}%
\pgfpathlineto{\pgfqpoint{0.918424in}{1.065067in}}%
\pgfpathlineto{\pgfqpoint{0.906691in}{1.062658in}}%
\pgfpathlineto{\pgfqpoint{0.896915in}{1.060311in}}%
\pgfpathlineto{\pgfqpoint{0.894959in}{1.058076in}}%
\pgfpathlineto{\pgfqpoint{0.891375in}{1.048901in}}%
\pgfpathlineto{\pgfqpoint{0.886088in}{1.037490in}}%
\pgfpathlineto{\pgfqpoint{0.883227in}{1.032232in}}%
\pgfpathlineto{\pgfqpoint{0.878671in}{1.026079in}}%
\pgfpathlineto{\pgfqpoint{0.871495in}{1.022860in}}%
\pgfpathclose%
\pgfpathmoveto{\pgfqpoint{1.258954in}{1.128776in}}%
\pgfpathlineto{\pgfqpoint{1.258658in}{1.128869in}}%
\pgfpathlineto{\pgfqpoint{1.246925in}{1.133070in}}%
\pgfpathlineto{\pgfqpoint{1.239336in}{1.140187in}}%
\pgfpathlineto{\pgfqpoint{1.245270in}{1.151598in}}%
\pgfpathlineto{\pgfqpoint{1.246925in}{1.156227in}}%
\pgfpathlineto{\pgfqpoint{1.248775in}{1.163008in}}%
\pgfpathlineto{\pgfqpoint{1.248524in}{1.174419in}}%
\pgfpathlineto{\pgfqpoint{1.248373in}{1.185830in}}%
\pgfpathlineto{\pgfqpoint{1.246925in}{1.192420in}}%
\pgfpathlineto{\pgfqpoint{1.245648in}{1.197241in}}%
\pgfpathlineto{\pgfqpoint{1.242214in}{1.208651in}}%
\pgfpathlineto{\pgfqpoint{1.241676in}{1.220062in}}%
\pgfpathlineto{\pgfqpoint{1.246925in}{1.221780in}}%
\pgfpathlineto{\pgfqpoint{1.248635in}{1.220062in}}%
\pgfpathlineto{\pgfqpoint{1.258658in}{1.210192in}}%
\pgfpathlineto{\pgfqpoint{1.260111in}{1.208651in}}%
\pgfpathlineto{\pgfqpoint{1.270390in}{1.200474in}}%
\pgfpathlineto{\pgfqpoint{1.274253in}{1.197241in}}%
\pgfpathlineto{\pgfqpoint{1.282122in}{1.191386in}}%
\pgfpathlineto{\pgfqpoint{1.287798in}{1.185830in}}%
\pgfpathlineto{\pgfqpoint{1.293854in}{1.178256in}}%
\pgfpathlineto{\pgfqpoint{1.296973in}{1.174419in}}%
\pgfpathlineto{\pgfqpoint{1.304117in}{1.163008in}}%
\pgfpathlineto{\pgfqpoint{1.305586in}{1.160515in}}%
\pgfpathlineto{\pgfqpoint{1.310696in}{1.151598in}}%
\pgfpathlineto{\pgfqpoint{1.314776in}{1.140187in}}%
\pgfpathlineto{\pgfqpoint{1.312470in}{1.128776in}}%
\pgfpathlineto{\pgfqpoint{1.305586in}{1.122958in}}%
\pgfpathlineto{\pgfqpoint{1.293854in}{1.120375in}}%
\pgfpathlineto{\pgfqpoint{1.282122in}{1.123267in}}%
\pgfpathlineto{\pgfqpoint{1.270390in}{1.126089in}}%
\pgfpathclose%
\pgfusepath{fill}%
\end{pgfscope}%
\begin{pgfscope}%
\pgfpathrectangle{\pgfqpoint{0.211875in}{0.211875in}}{\pgfqpoint{1.313625in}{1.279725in}}%
\pgfusepath{clip}%
\pgfsetbuttcap%
\pgfsetroundjoin%
\definecolor{currentfill}{rgb}{0.924566,0.290534,0.242426}%
\pgfsetfillcolor{currentfill}%
\pgfsetlinewidth{0.000000pt}%
\definecolor{currentstroke}{rgb}{0.000000,0.000000,0.000000}%
\pgfsetstrokecolor{currentstroke}%
\pgfsetdash{}{0pt}%
\pgfpathmoveto{\pgfqpoint{1.164800in}{0.865212in}}%
\pgfpathlineto{\pgfqpoint{1.176532in}{0.865891in}}%
\pgfpathlineto{\pgfqpoint{1.184230in}{0.866328in}}%
\pgfpathlineto{\pgfqpoint{1.176532in}{0.866506in}}%
\pgfpathlineto{\pgfqpoint{1.164800in}{0.866742in}}%
\pgfpathlineto{\pgfqpoint{1.158214in}{0.866328in}}%
\pgfpathclose%
\pgfusepath{fill}%
\end{pgfscope}%
\begin{pgfscope}%
\pgfpathrectangle{\pgfqpoint{0.211875in}{0.211875in}}{\pgfqpoint{1.313625in}{1.279725in}}%
\pgfusepath{clip}%
\pgfsetbuttcap%
\pgfsetroundjoin%
\definecolor{currentfill}{rgb}{0.924566,0.290534,0.242426}%
\pgfsetfillcolor{currentfill}%
\pgfsetlinewidth{0.000000pt}%
\definecolor{currentstroke}{rgb}{0.000000,0.000000,0.000000}%
\pgfsetstrokecolor{currentstroke}%
\pgfsetdash{}{0pt}%
\pgfpathmoveto{\pgfqpoint{0.660315in}{0.953794in}}%
\pgfpathlineto{\pgfqpoint{0.669715in}{0.957614in}}%
\pgfpathlineto{\pgfqpoint{0.672047in}{0.958567in}}%
\pgfpathlineto{\pgfqpoint{0.683666in}{0.969025in}}%
\pgfpathlineto{\pgfqpoint{0.683779in}{0.969140in}}%
\pgfpathlineto{\pgfqpoint{0.692761in}{0.980436in}}%
\pgfpathlineto{\pgfqpoint{0.695512in}{0.986655in}}%
\pgfpathlineto{\pgfqpoint{0.696925in}{0.991847in}}%
\pgfpathlineto{\pgfqpoint{0.696520in}{1.003257in}}%
\pgfpathlineto{\pgfqpoint{0.695512in}{1.005464in}}%
\pgfpathlineto{\pgfqpoint{0.690851in}{1.014668in}}%
\pgfpathlineto{\pgfqpoint{0.683779in}{1.023524in}}%
\pgfpathlineto{\pgfqpoint{0.681622in}{1.026079in}}%
\pgfpathlineto{\pgfqpoint{0.672047in}{1.034634in}}%
\pgfpathlineto{\pgfqpoint{0.667945in}{1.037490in}}%
\pgfpathlineto{\pgfqpoint{0.660315in}{1.042279in}}%
\pgfpathlineto{\pgfqpoint{0.648583in}{1.042178in}}%
\pgfpathlineto{\pgfqpoint{0.641958in}{1.037490in}}%
\pgfpathlineto{\pgfqpoint{0.637547in}{1.026079in}}%
\pgfpathlineto{\pgfqpoint{0.636851in}{1.019942in}}%
\pgfpathlineto{\pgfqpoint{0.636285in}{1.014668in}}%
\pgfpathlineto{\pgfqpoint{0.635958in}{1.003257in}}%
\pgfpathlineto{\pgfqpoint{0.636851in}{0.999377in}}%
\pgfpathlineto{\pgfqpoint{0.638534in}{0.991847in}}%
\pgfpathlineto{\pgfqpoint{0.643136in}{0.980436in}}%
\pgfpathlineto{\pgfqpoint{0.648583in}{0.970652in}}%
\pgfpathlineto{\pgfqpoint{0.649576in}{0.969025in}}%
\pgfpathlineto{\pgfqpoint{0.657410in}{0.957614in}}%
\pgfpathclose%
\pgfusepath{fill}%
\end{pgfscope}%
\begin{pgfscope}%
\pgfpathrectangle{\pgfqpoint{0.211875in}{0.211875in}}{\pgfqpoint{1.313625in}{1.279725in}}%
\pgfusepath{clip}%
\pgfsetbuttcap%
\pgfsetroundjoin%
\definecolor{currentfill}{rgb}{0.924566,0.290534,0.242426}%
\pgfsetfillcolor{currentfill}%
\pgfsetlinewidth{0.000000pt}%
\definecolor{currentstroke}{rgb}{0.000000,0.000000,0.000000}%
\pgfsetstrokecolor{currentstroke}%
\pgfsetdash{}{0pt}%
\pgfpathmoveto{\pgfqpoint{0.765905in}{1.273715in}}%
\pgfpathlineto{\pgfqpoint{0.777637in}{1.273899in}}%
\pgfpathlineto{\pgfqpoint{0.779621in}{1.277116in}}%
\pgfpathlineto{\pgfqpoint{0.777637in}{1.282388in}}%
\pgfpathlineto{\pgfqpoint{0.775342in}{1.288527in}}%
\pgfpathlineto{\pgfqpoint{0.771145in}{1.299938in}}%
\pgfpathlineto{\pgfqpoint{0.766910in}{1.311348in}}%
\pgfpathlineto{\pgfqpoint{0.765905in}{1.312619in}}%
\pgfpathlineto{\pgfqpoint{0.759141in}{1.322759in}}%
\pgfpathlineto{\pgfqpoint{0.754173in}{1.328591in}}%
\pgfpathlineto{\pgfqpoint{0.749300in}{1.334170in}}%
\pgfpathlineto{\pgfqpoint{0.742440in}{1.342240in}}%
\pgfpathlineto{\pgfqpoint{0.739521in}{1.345581in}}%
\pgfpathlineto{\pgfqpoint{0.730708in}{1.356020in}}%
\pgfpathlineto{\pgfqpoint{0.729025in}{1.356992in}}%
\pgfpathlineto{\pgfqpoint{0.720663in}{1.368402in}}%
\pgfpathlineto{\pgfqpoint{0.718976in}{1.370677in}}%
\pgfpathlineto{\pgfqpoint{0.715985in}{1.368402in}}%
\pgfpathlineto{\pgfqpoint{0.712611in}{1.356992in}}%
\pgfpathlineto{\pgfqpoint{0.712716in}{1.345581in}}%
\pgfpathlineto{\pgfqpoint{0.717056in}{1.334170in}}%
\pgfpathlineto{\pgfqpoint{0.718976in}{1.331595in}}%
\pgfpathlineto{\pgfqpoint{0.725349in}{1.322759in}}%
\pgfpathlineto{\pgfqpoint{0.730708in}{1.315052in}}%
\pgfpathlineto{\pgfqpoint{0.733227in}{1.311348in}}%
\pgfpathlineto{\pgfqpoint{0.741032in}{1.299938in}}%
\pgfpathlineto{\pgfqpoint{0.742440in}{1.297914in}}%
\pgfpathlineto{\pgfqpoint{0.749179in}{1.288527in}}%
\pgfpathlineto{\pgfqpoint{0.754173in}{1.281755in}}%
\pgfpathlineto{\pgfqpoint{0.761407in}{1.277116in}}%
\pgfpathclose%
\pgfpathmoveto{\pgfqpoint{0.752011in}{1.299938in}}%
\pgfpathlineto{\pgfqpoint{0.754173in}{1.305672in}}%
\pgfpathlineto{\pgfqpoint{0.756609in}{1.299938in}}%
\pgfpathlineto{\pgfqpoint{0.754173in}{1.294455in}}%
\pgfpathclose%
\pgfpathmoveto{\pgfqpoint{0.739212in}{1.311348in}}%
\pgfpathlineto{\pgfqpoint{0.734581in}{1.322759in}}%
\pgfpathlineto{\pgfqpoint{0.742440in}{1.326384in}}%
\pgfpathlineto{\pgfqpoint{0.745684in}{1.322759in}}%
\pgfpathlineto{\pgfqpoint{0.751353in}{1.311348in}}%
\pgfpathlineto{\pgfqpoint{0.742440in}{1.306624in}}%
\pgfpathclose%
\pgfpathmoveto{\pgfqpoint{0.729296in}{1.334170in}}%
\pgfpathlineto{\pgfqpoint{0.730708in}{1.336257in}}%
\pgfpathlineto{\pgfqpoint{0.732466in}{1.334170in}}%
\pgfpathlineto{\pgfqpoint{0.730708in}{1.329213in}}%
\pgfpathclose%
\pgfusepath{fill}%
\end{pgfscope}%
\begin{pgfscope}%
\pgfpathrectangle{\pgfqpoint{0.211875in}{0.211875in}}{\pgfqpoint{1.313625in}{1.279725in}}%
\pgfusepath{clip}%
\pgfsetbuttcap%
\pgfsetroundjoin%
\definecolor{currentfill}{rgb}{0.924566,0.290534,0.242426}%
\pgfsetfillcolor{currentfill}%
\pgfsetlinewidth{0.000000pt}%
\definecolor{currentstroke}{rgb}{0.000000,0.000000,0.000000}%
\pgfsetstrokecolor{currentstroke}%
\pgfsetdash{}{0pt}%
\pgfpathmoveto{\pgfqpoint{1.375980in}{1.345195in}}%
\pgfpathlineto{\pgfqpoint{1.387712in}{1.341495in}}%
\pgfpathlineto{\pgfqpoint{1.399444in}{1.339767in}}%
\pgfpathlineto{\pgfqpoint{1.411176in}{1.339594in}}%
\pgfpathlineto{\pgfqpoint{1.422908in}{1.340000in}}%
\pgfpathlineto{\pgfqpoint{1.434641in}{1.342887in}}%
\pgfpathlineto{\pgfqpoint{1.440558in}{1.345581in}}%
\pgfpathlineto{\pgfqpoint{1.446373in}{1.348291in}}%
\pgfpathlineto{\pgfqpoint{1.446373in}{1.356992in}}%
\pgfpathlineto{\pgfqpoint{1.446373in}{1.366181in}}%
\pgfpathlineto{\pgfqpoint{1.434641in}{1.363866in}}%
\pgfpathlineto{\pgfqpoint{1.422908in}{1.362268in}}%
\pgfpathlineto{\pgfqpoint{1.411176in}{1.361776in}}%
\pgfpathlineto{\pgfqpoint{1.399444in}{1.363241in}}%
\pgfpathlineto{\pgfqpoint{1.387712in}{1.366391in}}%
\pgfpathlineto{\pgfqpoint{1.383086in}{1.368402in}}%
\pgfpathlineto{\pgfqpoint{1.375980in}{1.372194in}}%
\pgfpathlineto{\pgfqpoint{1.367722in}{1.379813in}}%
\pgfpathlineto{\pgfqpoint{1.364247in}{1.383347in}}%
\pgfpathlineto{\pgfqpoint{1.358043in}{1.391224in}}%
\pgfpathlineto{\pgfqpoint{1.352515in}{1.398897in}}%
\pgfpathlineto{\pgfqpoint{1.350082in}{1.402635in}}%
\pgfpathlineto{\pgfqpoint{1.342851in}{1.414045in}}%
\pgfpathlineto{\pgfqpoint{1.340783in}{1.414045in}}%
\pgfpathlineto{\pgfqpoint{1.329051in}{1.414045in}}%
\pgfpathlineto{\pgfqpoint{1.319388in}{1.414045in}}%
\pgfpathlineto{\pgfqpoint{1.326255in}{1.402635in}}%
\pgfpathlineto{\pgfqpoint{1.329051in}{1.398077in}}%
\pgfpathlineto{\pgfqpoint{1.333169in}{1.391224in}}%
\pgfpathlineto{\pgfqpoint{1.340418in}{1.379813in}}%
\pgfpathlineto{\pgfqpoint{1.340783in}{1.379251in}}%
\pgfpathlineto{\pgfqpoint{1.348428in}{1.368402in}}%
\pgfpathlineto{\pgfqpoint{1.352515in}{1.363085in}}%
\pgfpathlineto{\pgfqpoint{1.358152in}{1.356992in}}%
\pgfpathlineto{\pgfqpoint{1.364247in}{1.351564in}}%
\pgfpathlineto{\pgfqpoint{1.375287in}{1.345581in}}%
\pgfpathclose%
\pgfusepath{fill}%
\end{pgfscope}%
\begin{pgfscope}%
\pgfpathrectangle{\pgfqpoint{0.211875in}{0.211875in}}{\pgfqpoint{1.313625in}{1.279725in}}%
\pgfusepath{clip}%
\pgfsetbuttcap%
\pgfsetroundjoin%
\definecolor{currentfill}{rgb}{0.953126,0.456614,0.312398}%
\pgfsetfillcolor{currentfill}%
\pgfsetlinewidth{0.000000pt}%
\definecolor{currentstroke}{rgb}{0.000000,0.000000,0.000000}%
\pgfsetstrokecolor{currentstroke}%
\pgfsetdash{}{0pt}%
\pgfpathmoveto{\pgfqpoint{1.094407in}{0.284378in}}%
\pgfpathlineto{\pgfqpoint{1.106139in}{0.284378in}}%
\pgfpathlineto{\pgfqpoint{1.117871in}{0.284378in}}%
\pgfpathlineto{\pgfqpoint{1.129603in}{0.284378in}}%
\pgfpathlineto{\pgfqpoint{1.134392in}{0.284378in}}%
\pgfpathlineto{\pgfqpoint{1.129603in}{0.291876in}}%
\pgfpathlineto{\pgfqpoint{1.127155in}{0.295789in}}%
\pgfpathlineto{\pgfqpoint{1.122318in}{0.307200in}}%
\pgfpathlineto{\pgfqpoint{1.117871in}{0.315356in}}%
\pgfpathlineto{\pgfqpoint{1.115857in}{0.318611in}}%
\pgfpathlineto{\pgfqpoint{1.110165in}{0.330022in}}%
\pgfpathlineto{\pgfqpoint{1.106139in}{0.340044in}}%
\pgfpathlineto{\pgfqpoint{1.105328in}{0.341432in}}%
\pgfpathlineto{\pgfqpoint{1.099183in}{0.352843in}}%
\pgfpathlineto{\pgfqpoint{1.094602in}{0.364254in}}%
\pgfpathlineto{\pgfqpoint{1.094407in}{0.364724in}}%
\pgfpathlineto{\pgfqpoint{1.088256in}{0.375665in}}%
\pgfpathlineto{\pgfqpoint{1.083532in}{0.387075in}}%
\pgfpathlineto{\pgfqpoint{1.082674in}{0.389175in}}%
\pgfpathlineto{\pgfqpoint{1.077404in}{0.398486in}}%
\pgfpathlineto{\pgfqpoint{1.072498in}{0.409897in}}%
\pgfpathlineto{\pgfqpoint{1.070942in}{0.413765in}}%
\pgfpathlineto{\pgfqpoint{1.066558in}{0.421308in}}%
\pgfpathlineto{\pgfqpoint{1.061372in}{0.432719in}}%
\pgfpathlineto{\pgfqpoint{1.059210in}{0.438112in}}%
\pgfpathlineto{\pgfqpoint{1.055611in}{0.444129in}}%
\pgfpathlineto{\pgfqpoint{1.050150in}{0.455540in}}%
\pgfpathlineto{\pgfqpoint{1.047478in}{0.462275in}}%
\pgfpathlineto{\pgfqpoint{1.044655in}{0.466951in}}%
\pgfpathlineto{\pgfqpoint{1.038946in}{0.478362in}}%
\pgfpathlineto{\pgfqpoint{1.035746in}{0.486548in}}%
\pgfpathlineto{\pgfqpoint{1.033786in}{0.489772in}}%
\pgfpathlineto{\pgfqpoint{1.027796in}{0.501183in}}%
\pgfpathlineto{\pgfqpoint{1.024013in}{0.510984in}}%
\pgfpathlineto{\pgfqpoint{1.023021in}{0.512594in}}%
\pgfpathlineto{\pgfqpoint{1.016663in}{0.524005in}}%
\pgfpathlineto{\pgfqpoint{1.012307in}{0.535416in}}%
\pgfpathlineto{\pgfqpoint{1.012281in}{0.535481in}}%
\pgfpathlineto{\pgfqpoint{1.005671in}{0.546826in}}%
\pgfpathlineto{\pgfqpoint{1.001173in}{0.558237in}}%
\pgfpathlineto{\pgfqpoint{1.000549in}{0.559888in}}%
\pgfpathlineto{\pgfqpoint{0.993105in}{0.569648in}}%
\pgfpathlineto{\pgfqpoint{0.989184in}{0.581059in}}%
\pgfpathlineto{\pgfqpoint{0.988817in}{0.582254in}}%
\pgfpathlineto{\pgfqpoint{0.984544in}{0.581059in}}%
\pgfpathlineto{\pgfqpoint{0.978022in}{0.569648in}}%
\pgfpathlineto{\pgfqpoint{0.977085in}{0.561262in}}%
\pgfpathlineto{\pgfqpoint{0.976843in}{0.558237in}}%
\pgfpathlineto{\pgfqpoint{0.976572in}{0.546826in}}%
\pgfpathlineto{\pgfqpoint{0.976635in}{0.535416in}}%
\pgfpathlineto{\pgfqpoint{0.976946in}{0.524005in}}%
\pgfpathlineto{\pgfqpoint{0.977085in}{0.516503in}}%
\pgfpathlineto{\pgfqpoint{0.977171in}{0.512594in}}%
\pgfpathlineto{\pgfqpoint{0.982360in}{0.501183in}}%
\pgfpathlineto{\pgfqpoint{0.985841in}{0.489772in}}%
\pgfpathlineto{\pgfqpoint{0.988817in}{0.482282in}}%
\pgfpathlineto{\pgfqpoint{0.990275in}{0.478362in}}%
\pgfpathlineto{\pgfqpoint{0.994559in}{0.466951in}}%
\pgfpathlineto{\pgfqpoint{0.999003in}{0.455540in}}%
\pgfpathlineto{\pgfqpoint{1.000549in}{0.451587in}}%
\pgfpathlineto{\pgfqpoint{1.003299in}{0.444129in}}%
\pgfpathlineto{\pgfqpoint{1.007523in}{0.432719in}}%
\pgfpathlineto{\pgfqpoint{1.011878in}{0.421308in}}%
\pgfpathlineto{\pgfqpoint{1.012281in}{0.420260in}}%
\pgfpathlineto{\pgfqpoint{1.017779in}{0.409897in}}%
\pgfpathlineto{\pgfqpoint{1.022653in}{0.398486in}}%
\pgfpathlineto{\pgfqpoint{1.024013in}{0.394853in}}%
\pgfpathlineto{\pgfqpoint{1.028702in}{0.387075in}}%
\pgfpathlineto{\pgfqpoint{1.035500in}{0.375665in}}%
\pgfpathlineto{\pgfqpoint{1.035746in}{0.375254in}}%
\pgfpathlineto{\pgfqpoint{1.042311in}{0.364254in}}%
\pgfpathlineto{\pgfqpoint{1.047478in}{0.355410in}}%
\pgfpathlineto{\pgfqpoint{1.049005in}{0.352843in}}%
\pgfpathlineto{\pgfqpoint{1.055716in}{0.341432in}}%
\pgfpathlineto{\pgfqpoint{1.059210in}{0.335387in}}%
\pgfpathlineto{\pgfqpoint{1.062634in}{0.330022in}}%
\pgfpathlineto{\pgfqpoint{1.069725in}{0.318611in}}%
\pgfpathlineto{\pgfqpoint{1.070942in}{0.316639in}}%
\pgfpathlineto{\pgfqpoint{1.077025in}{0.307200in}}%
\pgfpathlineto{\pgfqpoint{1.082674in}{0.297939in}}%
\pgfpathlineto{\pgfqpoint{1.084110in}{0.295789in}}%
\pgfpathlineto{\pgfqpoint{1.091348in}{0.284378in}}%
\pgfpathclose%
\pgfusepath{fill}%
\end{pgfscope}%
\begin{pgfscope}%
\pgfpathrectangle{\pgfqpoint{0.211875in}{0.211875in}}{\pgfqpoint{1.313625in}{1.279725in}}%
\pgfusepath{clip}%
\pgfsetbuttcap%
\pgfsetroundjoin%
\definecolor{currentfill}{rgb}{0.953126,0.456614,0.312398}%
\pgfsetfillcolor{currentfill}%
\pgfsetlinewidth{0.000000pt}%
\definecolor{currentstroke}{rgb}{0.000000,0.000000,0.000000}%
\pgfsetstrokecolor{currentstroke}%
\pgfsetdash{}{0pt}%
\pgfpathmoveto{\pgfqpoint{0.977085in}{0.595334in}}%
\pgfpathlineto{\pgfqpoint{0.977330in}{0.603880in}}%
\pgfpathlineto{\pgfqpoint{0.977085in}{0.609064in}}%
\pgfpathlineto{\pgfqpoint{0.974966in}{0.615291in}}%
\pgfpathlineto{\pgfqpoint{0.971871in}{0.626702in}}%
\pgfpathlineto{\pgfqpoint{0.969881in}{0.638113in}}%
\pgfpathlineto{\pgfqpoint{0.967692in}{0.649523in}}%
\pgfpathlineto{\pgfqpoint{0.966170in}{0.660934in}}%
\pgfpathlineto{\pgfqpoint{0.965352in}{0.667086in}}%
\pgfpathlineto{\pgfqpoint{0.962904in}{0.672345in}}%
\pgfpathlineto{\pgfqpoint{0.959181in}{0.683756in}}%
\pgfpathlineto{\pgfqpoint{0.956421in}{0.695166in}}%
\pgfpathlineto{\pgfqpoint{0.954711in}{0.706577in}}%
\pgfpathlineto{\pgfqpoint{0.953620in}{0.714523in}}%
\pgfpathlineto{\pgfqpoint{0.952357in}{0.717988in}}%
\pgfpathlineto{\pgfqpoint{0.948980in}{0.729399in}}%
\pgfpathlineto{\pgfqpoint{0.946112in}{0.740810in}}%
\pgfpathlineto{\pgfqpoint{0.944271in}{0.752220in}}%
\pgfpathlineto{\pgfqpoint{0.942438in}{0.763631in}}%
\pgfpathlineto{\pgfqpoint{0.941888in}{0.765556in}}%
\pgfpathlineto{\pgfqpoint{0.936349in}{0.775042in}}%
\pgfpathlineto{\pgfqpoint{0.930958in}{0.786453in}}%
\pgfpathlineto{\pgfqpoint{0.930156in}{0.788058in}}%
\pgfpathlineto{\pgfqpoint{0.918741in}{0.797863in}}%
\pgfpathlineto{\pgfqpoint{0.918424in}{0.798015in}}%
\pgfpathlineto{\pgfqpoint{0.913802in}{0.797863in}}%
\pgfpathlineto{\pgfqpoint{0.906691in}{0.797055in}}%
\pgfpathlineto{\pgfqpoint{0.901866in}{0.786453in}}%
\pgfpathlineto{\pgfqpoint{0.900351in}{0.775042in}}%
\pgfpathlineto{\pgfqpoint{0.902286in}{0.763631in}}%
\pgfpathlineto{\pgfqpoint{0.904324in}{0.752220in}}%
\pgfpathlineto{\pgfqpoint{0.906424in}{0.740810in}}%
\pgfpathlineto{\pgfqpoint{0.906691in}{0.739717in}}%
\pgfpathlineto{\pgfqpoint{0.909498in}{0.729399in}}%
\pgfpathlineto{\pgfqpoint{0.913138in}{0.717988in}}%
\pgfpathlineto{\pgfqpoint{0.917822in}{0.706577in}}%
\pgfpathlineto{\pgfqpoint{0.918424in}{0.705120in}}%
\pgfpathlineto{\pgfqpoint{0.922692in}{0.695166in}}%
\pgfpathlineto{\pgfqpoint{0.927675in}{0.683756in}}%
\pgfpathlineto{\pgfqpoint{0.930156in}{0.678145in}}%
\pgfpathlineto{\pgfqpoint{0.932752in}{0.672345in}}%
\pgfpathlineto{\pgfqpoint{0.937618in}{0.660934in}}%
\pgfpathlineto{\pgfqpoint{0.941888in}{0.651140in}}%
\pgfpathlineto{\pgfqpoint{0.942646in}{0.649523in}}%
\pgfpathlineto{\pgfqpoint{0.947417in}{0.638113in}}%
\pgfpathlineto{\pgfqpoint{0.951955in}{0.626702in}}%
\pgfpathlineto{\pgfqpoint{0.953620in}{0.623030in}}%
\pgfpathlineto{\pgfqpoint{0.958921in}{0.615291in}}%
\pgfpathlineto{\pgfqpoint{0.965352in}{0.605177in}}%
\pgfpathlineto{\pgfqpoint{0.970181in}{0.603880in}}%
\pgfpathclose%
\pgfusepath{fill}%
\end{pgfscope}%
\begin{pgfscope}%
\pgfpathrectangle{\pgfqpoint{0.211875in}{0.211875in}}{\pgfqpoint{1.313625in}{1.279725in}}%
\pgfusepath{clip}%
\pgfsetbuttcap%
\pgfsetroundjoin%
\definecolor{currentfill}{rgb}{0.953126,0.456614,0.312398}%
\pgfsetfillcolor{currentfill}%
\pgfsetlinewidth{0.000000pt}%
\definecolor{currentstroke}{rgb}{0.000000,0.000000,0.000000}%
\pgfsetstrokecolor{currentstroke}%
\pgfsetdash{}{0pt}%
\pgfpathmoveto{\pgfqpoint{1.293854in}{0.840629in}}%
\pgfpathlineto{\pgfqpoint{1.305586in}{0.842850in}}%
\pgfpathlineto{\pgfqpoint{1.306198in}{0.843507in}}%
\pgfpathlineto{\pgfqpoint{1.316634in}{0.854917in}}%
\pgfpathlineto{\pgfqpoint{1.317319in}{0.855656in}}%
\pgfpathlineto{\pgfqpoint{1.318004in}{0.854917in}}%
\pgfpathlineto{\pgfqpoint{1.328663in}{0.843507in}}%
\pgfpathlineto{\pgfqpoint{1.329051in}{0.843105in}}%
\pgfpathlineto{\pgfqpoint{1.329165in}{0.843507in}}%
\pgfpathlineto{\pgfqpoint{1.332027in}{0.854917in}}%
\pgfpathlineto{\pgfqpoint{1.333643in}{0.866328in}}%
\pgfpathlineto{\pgfqpoint{1.331571in}{0.877739in}}%
\pgfpathlineto{\pgfqpoint{1.329568in}{0.889150in}}%
\pgfpathlineto{\pgfqpoint{1.329051in}{0.893306in}}%
\pgfpathlineto{\pgfqpoint{1.317319in}{0.890885in}}%
\pgfpathlineto{\pgfqpoint{1.313388in}{0.900560in}}%
\pgfpathlineto{\pgfqpoint{1.310937in}{0.911971in}}%
\pgfpathlineto{\pgfqpoint{1.317319in}{0.914532in}}%
\pgfpathlineto{\pgfqpoint{1.327687in}{0.911971in}}%
\pgfpathlineto{\pgfqpoint{1.329051in}{0.909515in}}%
\pgfpathlineto{\pgfqpoint{1.329289in}{0.911971in}}%
\pgfpathlineto{\pgfqpoint{1.336521in}{0.923382in}}%
\pgfpathlineto{\pgfqpoint{1.340783in}{0.928226in}}%
\pgfpathlineto{\pgfqpoint{1.346787in}{0.934793in}}%
\pgfpathlineto{\pgfqpoint{1.352515in}{0.939607in}}%
\pgfpathlineto{\pgfqpoint{1.361766in}{0.946204in}}%
\pgfpathlineto{\pgfqpoint{1.364247in}{0.952929in}}%
\pgfpathlineto{\pgfqpoint{1.366287in}{0.957614in}}%
\pgfpathlineto{\pgfqpoint{1.364247in}{0.958819in}}%
\pgfpathlineto{\pgfqpoint{1.352515in}{0.963698in}}%
\pgfpathlineto{\pgfqpoint{1.340783in}{0.968572in}}%
\pgfpathlineto{\pgfqpoint{1.340155in}{0.969025in}}%
\pgfpathlineto{\pgfqpoint{1.329051in}{0.978095in}}%
\pgfpathlineto{\pgfqpoint{1.317319in}{0.974198in}}%
\pgfpathlineto{\pgfqpoint{1.305586in}{0.970372in}}%
\pgfpathlineto{\pgfqpoint{1.301410in}{0.969025in}}%
\pgfpathlineto{\pgfqpoint{1.293854in}{0.966340in}}%
\pgfpathlineto{\pgfqpoint{1.282122in}{0.960505in}}%
\pgfpathlineto{\pgfqpoint{1.276008in}{0.957614in}}%
\pgfpathlineto{\pgfqpoint{1.282122in}{0.955795in}}%
\pgfpathlineto{\pgfqpoint{1.293854in}{0.946519in}}%
\pgfpathlineto{\pgfqpoint{1.294090in}{0.946204in}}%
\pgfpathlineto{\pgfqpoint{1.298518in}{0.934793in}}%
\pgfpathlineto{\pgfqpoint{1.297881in}{0.923382in}}%
\pgfpathlineto{\pgfqpoint{1.293854in}{0.918640in}}%
\pgfpathlineto{\pgfqpoint{1.282122in}{0.913643in}}%
\pgfpathlineto{\pgfqpoint{1.278721in}{0.911971in}}%
\pgfpathlineto{\pgfqpoint{1.270390in}{0.908315in}}%
\pgfpathlineto{\pgfqpoint{1.258658in}{0.902780in}}%
\pgfpathlineto{\pgfqpoint{1.255582in}{0.900560in}}%
\pgfpathlineto{\pgfqpoint{1.246925in}{0.892207in}}%
\pgfpathlineto{\pgfqpoint{1.244181in}{0.889150in}}%
\pgfpathlineto{\pgfqpoint{1.238351in}{0.877739in}}%
\pgfpathlineto{\pgfqpoint{1.246925in}{0.871653in}}%
\pgfpathlineto{\pgfqpoint{1.252115in}{0.866328in}}%
\pgfpathlineto{\pgfqpoint{1.258658in}{0.860646in}}%
\pgfpathlineto{\pgfqpoint{1.268687in}{0.854917in}}%
\pgfpathlineto{\pgfqpoint{1.270390in}{0.853947in}}%
\pgfpathlineto{\pgfqpoint{1.282122in}{0.847270in}}%
\pgfpathlineto{\pgfqpoint{1.288777in}{0.843507in}}%
\pgfpathclose%
\pgfpathmoveto{\pgfqpoint{1.269466in}{0.877739in}}%
\pgfpathlineto{\pgfqpoint{1.258658in}{0.884933in}}%
\pgfpathlineto{\pgfqpoint{1.256200in}{0.889150in}}%
\pgfpathlineto{\pgfqpoint{1.258658in}{0.891513in}}%
\pgfpathlineto{\pgfqpoint{1.267688in}{0.900560in}}%
\pgfpathlineto{\pgfqpoint{1.270390in}{0.901868in}}%
\pgfpathlineto{\pgfqpoint{1.277858in}{0.900560in}}%
\pgfpathlineto{\pgfqpoint{1.282122in}{0.890813in}}%
\pgfpathlineto{\pgfqpoint{1.282816in}{0.889150in}}%
\pgfpathlineto{\pgfqpoint{1.282122in}{0.887762in}}%
\pgfpathlineto{\pgfqpoint{1.272187in}{0.877739in}}%
\pgfpathlineto{\pgfqpoint{1.270390in}{0.877141in}}%
\pgfpathclose%
\pgfpathmoveto{\pgfqpoint{1.307931in}{0.934793in}}%
\pgfpathlineto{\pgfqpoint{1.305586in}{0.940754in}}%
\pgfpathlineto{\pgfqpoint{1.303437in}{0.946204in}}%
\pgfpathlineto{\pgfqpoint{1.295794in}{0.957614in}}%
\pgfpathlineto{\pgfqpoint{1.305586in}{0.962181in}}%
\pgfpathlineto{\pgfqpoint{1.317319in}{0.966577in}}%
\pgfpathlineto{\pgfqpoint{1.324183in}{0.969025in}}%
\pgfpathlineto{\pgfqpoint{1.329051in}{0.970787in}}%
\pgfpathlineto{\pgfqpoint{1.331207in}{0.969025in}}%
\pgfpathlineto{\pgfqpoint{1.340783in}{0.962123in}}%
\pgfpathlineto{\pgfqpoint{1.350888in}{0.957614in}}%
\pgfpathlineto{\pgfqpoint{1.351049in}{0.946204in}}%
\pgfpathlineto{\pgfqpoint{1.340783in}{0.937454in}}%
\pgfpathlineto{\pgfqpoint{1.337930in}{0.934793in}}%
\pgfpathlineto{\pgfqpoint{1.329051in}{0.924947in}}%
\pgfpathlineto{\pgfqpoint{1.317319in}{0.926544in}}%
\pgfpathclose%
\pgfusepath{fill}%
\end{pgfscope}%
\begin{pgfscope}%
\pgfpathrectangle{\pgfqpoint{0.211875in}{0.211875in}}{\pgfqpoint{1.313625in}{1.279725in}}%
\pgfusepath{clip}%
\pgfsetbuttcap%
\pgfsetroundjoin%
\definecolor{currentfill}{rgb}{0.953126,0.456614,0.312398}%
\pgfsetfillcolor{currentfill}%
\pgfsetlinewidth{0.000000pt}%
\definecolor{currentstroke}{rgb}{0.000000,0.000000,0.000000}%
\pgfsetstrokecolor{currentstroke}%
\pgfsetdash{}{0pt}%
\pgfpathmoveto{\pgfqpoint{0.906691in}{0.922305in}}%
\pgfpathlineto{\pgfqpoint{0.917276in}{0.923382in}}%
\pgfpathlineto{\pgfqpoint{0.918424in}{0.923574in}}%
\pgfpathlineto{\pgfqpoint{0.922399in}{0.934793in}}%
\pgfpathlineto{\pgfqpoint{0.929691in}{0.946204in}}%
\pgfpathlineto{\pgfqpoint{0.930156in}{0.946964in}}%
\pgfpathlineto{\pgfqpoint{0.941786in}{0.957614in}}%
\pgfpathlineto{\pgfqpoint{0.941888in}{0.958427in}}%
\pgfpathlineto{\pgfqpoint{0.951121in}{0.969025in}}%
\pgfpathlineto{\pgfqpoint{0.941888in}{0.976786in}}%
\pgfpathlineto{\pgfqpoint{0.939100in}{0.980436in}}%
\pgfpathlineto{\pgfqpoint{0.930156in}{0.982693in}}%
\pgfpathlineto{\pgfqpoint{0.922525in}{0.980436in}}%
\pgfpathlineto{\pgfqpoint{0.918424in}{0.979311in}}%
\pgfpathlineto{\pgfqpoint{0.906691in}{0.977703in}}%
\pgfpathlineto{\pgfqpoint{0.894959in}{0.977063in}}%
\pgfpathlineto{\pgfqpoint{0.888648in}{0.969025in}}%
\pgfpathlineto{\pgfqpoint{0.883227in}{0.963130in}}%
\pgfpathlineto{\pgfqpoint{0.879006in}{0.957614in}}%
\pgfpathlineto{\pgfqpoint{0.873419in}{0.946204in}}%
\pgfpathlineto{\pgfqpoint{0.871495in}{0.939718in}}%
\pgfpathlineto{\pgfqpoint{0.869912in}{0.934793in}}%
\pgfpathlineto{\pgfqpoint{0.871495in}{0.933549in}}%
\pgfpathlineto{\pgfqpoint{0.883227in}{0.928977in}}%
\pgfpathlineto{\pgfqpoint{0.894959in}{0.925834in}}%
\pgfpathlineto{\pgfqpoint{0.905000in}{0.923382in}}%
\pgfpathclose%
\pgfpathmoveto{\pgfqpoint{0.880253in}{0.934793in}}%
\pgfpathlineto{\pgfqpoint{0.879343in}{0.946204in}}%
\pgfpathlineto{\pgfqpoint{0.883227in}{0.954093in}}%
\pgfpathlineto{\pgfqpoint{0.885657in}{0.957614in}}%
\pgfpathlineto{\pgfqpoint{0.894959in}{0.967922in}}%
\pgfpathlineto{\pgfqpoint{0.899281in}{0.969025in}}%
\pgfpathlineto{\pgfqpoint{0.906691in}{0.970504in}}%
\pgfpathlineto{\pgfqpoint{0.913565in}{0.969025in}}%
\pgfpathlineto{\pgfqpoint{0.918424in}{0.967120in}}%
\pgfpathlineto{\pgfqpoint{0.930156in}{0.957738in}}%
\pgfpathlineto{\pgfqpoint{0.930237in}{0.957614in}}%
\pgfpathlineto{\pgfqpoint{0.930156in}{0.957539in}}%
\pgfpathlineto{\pgfqpoint{0.923231in}{0.946204in}}%
\pgfpathlineto{\pgfqpoint{0.918424in}{0.939317in}}%
\pgfpathlineto{\pgfqpoint{0.912239in}{0.934793in}}%
\pgfpathlineto{\pgfqpoint{0.906691in}{0.931107in}}%
\pgfpathlineto{\pgfqpoint{0.894959in}{0.930962in}}%
\pgfpathlineto{\pgfqpoint{0.883227in}{0.933608in}}%
\pgfpathclose%
\pgfusepath{fill}%
\end{pgfscope}%
\begin{pgfscope}%
\pgfpathrectangle{\pgfqpoint{0.211875in}{0.211875in}}{\pgfqpoint{1.313625in}{1.279725in}}%
\pgfusepath{clip}%
\pgfsetbuttcap%
\pgfsetroundjoin%
\definecolor{currentfill}{rgb}{0.953126,0.456614,0.312398}%
\pgfsetfillcolor{currentfill}%
\pgfsetlinewidth{0.000000pt}%
\definecolor{currentstroke}{rgb}{0.000000,0.000000,0.000000}%
\pgfsetstrokecolor{currentstroke}%
\pgfsetdash{}{0pt}%
\pgfpathmoveto{\pgfqpoint{1.446373in}{0.942726in}}%
\pgfpathlineto{\pgfqpoint{1.446373in}{0.946204in}}%
\pgfpathlineto{\pgfqpoint{1.446373in}{0.957614in}}%
\pgfpathlineto{\pgfqpoint{1.446373in}{0.965595in}}%
\pgfpathlineto{\pgfqpoint{1.440974in}{0.969025in}}%
\pgfpathlineto{\pgfqpoint{1.434641in}{0.972044in}}%
\pgfpathlineto{\pgfqpoint{1.422908in}{0.976770in}}%
\pgfpathlineto{\pgfqpoint{1.415999in}{0.969025in}}%
\pgfpathlineto{\pgfqpoint{1.415840in}{0.957614in}}%
\pgfpathlineto{\pgfqpoint{1.422908in}{0.953178in}}%
\pgfpathlineto{\pgfqpoint{1.434641in}{0.947132in}}%
\pgfpathlineto{\pgfqpoint{1.437021in}{0.946204in}}%
\pgfpathclose%
\pgfusepath{fill}%
\end{pgfscope}%
\begin{pgfscope}%
\pgfpathrectangle{\pgfqpoint{0.211875in}{0.211875in}}{\pgfqpoint{1.313625in}{1.279725in}}%
\pgfusepath{clip}%
\pgfsetbuttcap%
\pgfsetroundjoin%
\definecolor{currentfill}{rgb}{0.953126,0.456614,0.312398}%
\pgfsetfillcolor{currentfill}%
\pgfsetlinewidth{0.000000pt}%
\definecolor{currentstroke}{rgb}{0.000000,0.000000,0.000000}%
\pgfsetstrokecolor{currentstroke}%
\pgfsetdash{}{0pt}%
\pgfpathmoveto{\pgfqpoint{1.094407in}{0.990850in}}%
\pgfpathlineto{\pgfqpoint{1.106139in}{0.989206in}}%
\pgfpathlineto{\pgfqpoint{1.117871in}{0.989903in}}%
\pgfpathlineto{\pgfqpoint{1.120468in}{0.991847in}}%
\pgfpathlineto{\pgfqpoint{1.119920in}{1.003257in}}%
\pgfpathlineto{\pgfqpoint{1.119307in}{1.014668in}}%
\pgfpathlineto{\pgfqpoint{1.117871in}{1.016755in}}%
\pgfpathlineto{\pgfqpoint{1.112011in}{1.026079in}}%
\pgfpathlineto{\pgfqpoint{1.106139in}{1.031034in}}%
\pgfpathlineto{\pgfqpoint{1.095418in}{1.037490in}}%
\pgfpathlineto{\pgfqpoint{1.100737in}{1.048901in}}%
\pgfpathlineto{\pgfqpoint{1.106139in}{1.051959in}}%
\pgfpathlineto{\pgfqpoint{1.117871in}{1.054859in}}%
\pgfpathlineto{\pgfqpoint{1.129603in}{1.055152in}}%
\pgfpathlineto{\pgfqpoint{1.141335in}{1.054037in}}%
\pgfpathlineto{\pgfqpoint{1.144420in}{1.060311in}}%
\pgfpathlineto{\pgfqpoint{1.141335in}{1.070513in}}%
\pgfpathlineto{\pgfqpoint{1.141138in}{1.071722in}}%
\pgfpathlineto{\pgfqpoint{1.137333in}{1.083133in}}%
\pgfpathlineto{\pgfqpoint{1.137766in}{1.094544in}}%
\pgfpathlineto{\pgfqpoint{1.141335in}{1.102465in}}%
\pgfpathlineto{\pgfqpoint{1.143545in}{1.105954in}}%
\pgfpathlineto{\pgfqpoint{1.141335in}{1.115116in}}%
\pgfpathlineto{\pgfqpoint{1.134334in}{1.117365in}}%
\pgfpathlineto{\pgfqpoint{1.129603in}{1.118290in}}%
\pgfpathlineto{\pgfqpoint{1.117871in}{1.122174in}}%
\pgfpathlineto{\pgfqpoint{1.106139in}{1.126883in}}%
\pgfpathlineto{\pgfqpoint{1.103424in}{1.128776in}}%
\pgfpathlineto{\pgfqpoint{1.106139in}{1.131784in}}%
\pgfpathlineto{\pgfqpoint{1.112026in}{1.140187in}}%
\pgfpathlineto{\pgfqpoint{1.106139in}{1.145707in}}%
\pgfpathlineto{\pgfqpoint{1.101556in}{1.151598in}}%
\pgfpathlineto{\pgfqpoint{1.094407in}{1.158539in}}%
\pgfpathlineto{\pgfqpoint{1.089635in}{1.163008in}}%
\pgfpathlineto{\pgfqpoint{1.082674in}{1.169943in}}%
\pgfpathlineto{\pgfqpoint{1.077987in}{1.174419in}}%
\pgfpathlineto{\pgfqpoint{1.070942in}{1.181357in}}%
\pgfpathlineto{\pgfqpoint{1.066331in}{1.185830in}}%
\pgfpathlineto{\pgfqpoint{1.059210in}{1.192760in}}%
\pgfpathlineto{\pgfqpoint{1.054519in}{1.197241in}}%
\pgfpathlineto{\pgfqpoint{1.047478in}{1.203934in}}%
\pgfpathlineto{\pgfqpoint{1.042404in}{1.208651in}}%
\pgfpathlineto{\pgfqpoint{1.035746in}{1.214823in}}%
\pgfpathlineto{\pgfqpoint{1.032318in}{1.220062in}}%
\pgfpathlineto{\pgfqpoint{1.027898in}{1.231473in}}%
\pgfpathlineto{\pgfqpoint{1.025982in}{1.242884in}}%
\pgfpathlineto{\pgfqpoint{1.024605in}{1.254295in}}%
\pgfpathlineto{\pgfqpoint{1.024163in}{1.265705in}}%
\pgfpathlineto{\pgfqpoint{1.024013in}{1.266078in}}%
\pgfpathlineto{\pgfqpoint{1.018747in}{1.277116in}}%
\pgfpathlineto{\pgfqpoint{1.012281in}{1.280017in}}%
\pgfpathlineto{\pgfqpoint{1.000549in}{1.280969in}}%
\pgfpathlineto{\pgfqpoint{0.988817in}{1.281939in}}%
\pgfpathlineto{\pgfqpoint{0.977085in}{1.282970in}}%
\pgfpathlineto{\pgfqpoint{0.965352in}{1.280178in}}%
\pgfpathlineto{\pgfqpoint{0.957677in}{1.277116in}}%
\pgfpathlineto{\pgfqpoint{0.953620in}{1.275602in}}%
\pgfpathlineto{\pgfqpoint{0.941888in}{1.270475in}}%
\pgfpathlineto{\pgfqpoint{0.933673in}{1.265705in}}%
\pgfpathlineto{\pgfqpoint{0.930156in}{1.263299in}}%
\pgfpathlineto{\pgfqpoint{0.918424in}{1.255273in}}%
\pgfpathlineto{\pgfqpoint{0.917002in}{1.254295in}}%
\pgfpathlineto{\pgfqpoint{0.917543in}{1.242884in}}%
\pgfpathlineto{\pgfqpoint{0.918424in}{1.242344in}}%
\pgfpathlineto{\pgfqpoint{0.930156in}{1.235148in}}%
\pgfpathlineto{\pgfqpoint{0.936126in}{1.231473in}}%
\pgfpathlineto{\pgfqpoint{0.941888in}{1.228016in}}%
\pgfpathlineto{\pgfqpoint{0.953620in}{1.220957in}}%
\pgfpathlineto{\pgfqpoint{0.955111in}{1.220062in}}%
\pgfpathlineto{\pgfqpoint{0.965352in}{1.214066in}}%
\pgfpathlineto{\pgfqpoint{0.972903in}{1.208651in}}%
\pgfpathlineto{\pgfqpoint{0.977085in}{1.205663in}}%
\pgfpathlineto{\pgfqpoint{0.988567in}{1.197241in}}%
\pgfpathlineto{\pgfqpoint{0.988817in}{1.197047in}}%
\pgfpathlineto{\pgfqpoint{1.000549in}{1.187435in}}%
\pgfpathlineto{\pgfqpoint{1.002529in}{1.185830in}}%
\pgfpathlineto{\pgfqpoint{1.012281in}{1.177957in}}%
\pgfpathlineto{\pgfqpoint{1.015626in}{1.174419in}}%
\pgfpathlineto{\pgfqpoint{1.023129in}{1.163008in}}%
\pgfpathlineto{\pgfqpoint{1.024013in}{1.162402in}}%
\pgfpathlineto{\pgfqpoint{1.033849in}{1.151598in}}%
\pgfpathlineto{\pgfqpoint{1.035746in}{1.150207in}}%
\pgfpathlineto{\pgfqpoint{1.047478in}{1.142772in}}%
\pgfpathlineto{\pgfqpoint{1.053297in}{1.140187in}}%
\pgfpathlineto{\pgfqpoint{1.059210in}{1.137050in}}%
\pgfpathlineto{\pgfqpoint{1.070942in}{1.131152in}}%
\pgfpathlineto{\pgfqpoint{1.075358in}{1.128776in}}%
\pgfpathlineto{\pgfqpoint{1.079752in}{1.117365in}}%
\pgfpathlineto{\pgfqpoint{1.082674in}{1.115388in}}%
\pgfpathlineto{\pgfqpoint{1.087753in}{1.105954in}}%
\pgfpathlineto{\pgfqpoint{1.087631in}{1.094544in}}%
\pgfpathlineto{\pgfqpoint{1.082674in}{1.083535in}}%
\pgfpathlineto{\pgfqpoint{1.082106in}{1.083133in}}%
\pgfpathlineto{\pgfqpoint{1.082674in}{1.081911in}}%
\pgfpathlineto{\pgfqpoint{1.087836in}{1.071722in}}%
\pgfpathlineto{\pgfqpoint{1.082674in}{1.065983in}}%
\pgfpathlineto{\pgfqpoint{1.075556in}{1.060311in}}%
\pgfpathlineto{\pgfqpoint{1.070942in}{1.052483in}}%
\pgfpathlineto{\pgfqpoint{1.062719in}{1.048901in}}%
\pgfpathlineto{\pgfqpoint{1.059210in}{1.044646in}}%
\pgfpathlineto{\pgfqpoint{1.054921in}{1.037490in}}%
\pgfpathlineto{\pgfqpoint{1.058893in}{1.026079in}}%
\pgfpathlineto{\pgfqpoint{1.059210in}{1.025692in}}%
\pgfpathlineto{\pgfqpoint{1.068348in}{1.014668in}}%
\pgfpathlineto{\pgfqpoint{1.070942in}{1.011027in}}%
\pgfpathlineto{\pgfqpoint{1.080219in}{1.003257in}}%
\pgfpathlineto{\pgfqpoint{1.082674in}{0.994040in}}%
\pgfpathlineto{\pgfqpoint{1.083247in}{0.991847in}}%
\pgfpathclose%
\pgfpathmoveto{\pgfqpoint{1.102337in}{1.071722in}}%
\pgfpathlineto{\pgfqpoint{1.094407in}{1.082646in}}%
\pgfpathlineto{\pgfqpoint{1.094032in}{1.083133in}}%
\pgfpathlineto{\pgfqpoint{1.094407in}{1.090276in}}%
\pgfpathlineto{\pgfqpoint{1.094536in}{1.094544in}}%
\pgfpathlineto{\pgfqpoint{1.096578in}{1.105954in}}%
\pgfpathlineto{\pgfqpoint{1.106139in}{1.113141in}}%
\pgfpathlineto{\pgfqpoint{1.117871in}{1.113473in}}%
\pgfpathlineto{\pgfqpoint{1.129603in}{1.110782in}}%
\pgfpathlineto{\pgfqpoint{1.135518in}{1.105954in}}%
\pgfpathlineto{\pgfqpoint{1.132802in}{1.094544in}}%
\pgfpathlineto{\pgfqpoint{1.132521in}{1.083133in}}%
\pgfpathlineto{\pgfqpoint{1.132193in}{1.071722in}}%
\pgfpathlineto{\pgfqpoint{1.129603in}{1.068728in}}%
\pgfpathlineto{\pgfqpoint{1.117871in}{1.066112in}}%
\pgfpathlineto{\pgfqpoint{1.106139in}{1.067957in}}%
\pgfpathclose%
\pgfpathmoveto{\pgfqpoint{1.057318in}{1.151598in}}%
\pgfpathlineto{\pgfqpoint{1.057173in}{1.163008in}}%
\pgfpathlineto{\pgfqpoint{1.059210in}{1.164415in}}%
\pgfpathlineto{\pgfqpoint{1.060714in}{1.163008in}}%
\pgfpathlineto{\pgfqpoint{1.070942in}{1.156542in}}%
\pgfpathlineto{\pgfqpoint{1.076429in}{1.151598in}}%
\pgfpathlineto{\pgfqpoint{1.070942in}{1.147631in}}%
\pgfpathlineto{\pgfqpoint{1.059210in}{1.150888in}}%
\pgfpathclose%
\pgfpathmoveto{\pgfqpoint{0.982177in}{1.231473in}}%
\pgfpathlineto{\pgfqpoint{0.977085in}{1.233876in}}%
\pgfpathlineto{\pgfqpoint{0.965352in}{1.242028in}}%
\pgfpathlineto{\pgfqpoint{0.963939in}{1.242884in}}%
\pgfpathlineto{\pgfqpoint{0.953620in}{1.248952in}}%
\pgfpathlineto{\pgfqpoint{0.947795in}{1.254295in}}%
\pgfpathlineto{\pgfqpoint{0.953620in}{1.258311in}}%
\pgfpathlineto{\pgfqpoint{0.964469in}{1.265705in}}%
\pgfpathlineto{\pgfqpoint{0.965352in}{1.266093in}}%
\pgfpathlineto{\pgfqpoint{0.977085in}{1.270984in}}%
\pgfpathlineto{\pgfqpoint{0.988817in}{1.271870in}}%
\pgfpathlineto{\pgfqpoint{1.000549in}{1.267787in}}%
\pgfpathlineto{\pgfqpoint{1.003525in}{1.265705in}}%
\pgfpathlineto{\pgfqpoint{1.005221in}{1.254295in}}%
\pgfpathlineto{\pgfqpoint{1.006485in}{1.242884in}}%
\pgfpathlineto{\pgfqpoint{1.003555in}{1.231473in}}%
\pgfpathlineto{\pgfqpoint{1.000549in}{1.228745in}}%
\pgfpathlineto{\pgfqpoint{0.988817in}{1.227140in}}%
\pgfpathclose%
\pgfusepath{fill}%
\end{pgfscope}%
\begin{pgfscope}%
\pgfpathrectangle{\pgfqpoint{0.211875in}{0.211875in}}{\pgfqpoint{1.313625in}{1.279725in}}%
\pgfusepath{clip}%
\pgfsetbuttcap%
\pgfsetroundjoin%
\definecolor{currentfill}{rgb}{0.953126,0.456614,0.312398}%
\pgfsetfillcolor{currentfill}%
\pgfsetlinewidth{0.000000pt}%
\definecolor{currentstroke}{rgb}{0.000000,0.000000,0.000000}%
\pgfsetstrokecolor{currentstroke}%
\pgfsetdash{}{0pt}%
\pgfpathmoveto{\pgfqpoint{1.340783in}{0.988100in}}%
\pgfpathlineto{\pgfqpoint{1.351982in}{0.991847in}}%
\pgfpathlineto{\pgfqpoint{1.352515in}{0.992011in}}%
\pgfpathlineto{\pgfqpoint{1.364247in}{0.996653in}}%
\pgfpathlineto{\pgfqpoint{1.375980in}{0.996297in}}%
\pgfpathlineto{\pgfqpoint{1.387712in}{0.999070in}}%
\pgfpathlineto{\pgfqpoint{1.397757in}{1.003257in}}%
\pgfpathlineto{\pgfqpoint{1.399444in}{1.004417in}}%
\pgfpathlineto{\pgfqpoint{1.403097in}{1.014668in}}%
\pgfpathlineto{\pgfqpoint{1.404247in}{1.026079in}}%
\pgfpathlineto{\pgfqpoint{1.405380in}{1.037490in}}%
\pgfpathlineto{\pgfqpoint{1.408609in}{1.048901in}}%
\pgfpathlineto{\pgfqpoint{1.411176in}{1.056329in}}%
\pgfpathlineto{\pgfqpoint{1.412736in}{1.060311in}}%
\pgfpathlineto{\pgfqpoint{1.422908in}{1.068567in}}%
\pgfpathlineto{\pgfqpoint{1.426552in}{1.071722in}}%
\pgfpathlineto{\pgfqpoint{1.434641in}{1.078830in}}%
\pgfpathlineto{\pgfqpoint{1.439700in}{1.083133in}}%
\pgfpathlineto{\pgfqpoint{1.446373in}{1.088981in}}%
\pgfpathlineto{\pgfqpoint{1.446373in}{1.094220in}}%
\pgfpathlineto{\pgfqpoint{1.434641in}{1.084084in}}%
\pgfpathlineto{\pgfqpoint{1.433203in}{1.083133in}}%
\pgfpathlineto{\pgfqpoint{1.422908in}{1.074594in}}%
\pgfpathlineto{\pgfqpoint{1.414030in}{1.083133in}}%
\pgfpathlineto{\pgfqpoint{1.417833in}{1.094544in}}%
\pgfpathlineto{\pgfqpoint{1.422908in}{1.103282in}}%
\pgfpathlineto{\pgfqpoint{1.425333in}{1.105954in}}%
\pgfpathlineto{\pgfqpoint{1.434641in}{1.111637in}}%
\pgfpathlineto{\pgfqpoint{1.446373in}{1.114664in}}%
\pgfpathlineto{\pgfqpoint{1.446373in}{1.117365in}}%
\pgfpathlineto{\pgfqpoint{1.446373in}{1.119915in}}%
\pgfpathlineto{\pgfqpoint{1.440084in}{1.117365in}}%
\pgfpathlineto{\pgfqpoint{1.434641in}{1.116089in}}%
\pgfpathlineto{\pgfqpoint{1.422908in}{1.111628in}}%
\pgfpathlineto{\pgfqpoint{1.418166in}{1.105954in}}%
\pgfpathlineto{\pgfqpoint{1.413166in}{1.094544in}}%
\pgfpathlineto{\pgfqpoint{1.411176in}{1.090007in}}%
\pgfpathlineto{\pgfqpoint{1.408037in}{1.083133in}}%
\pgfpathlineto{\pgfqpoint{1.405449in}{1.071722in}}%
\pgfpathlineto{\pgfqpoint{1.399444in}{1.064817in}}%
\pgfpathlineto{\pgfqpoint{1.387712in}{1.060875in}}%
\pgfpathlineto{\pgfqpoint{1.386572in}{1.060311in}}%
\pgfpathlineto{\pgfqpoint{1.375980in}{1.054969in}}%
\pgfpathlineto{\pgfqpoint{1.365959in}{1.048901in}}%
\pgfpathlineto{\pgfqpoint{1.364247in}{1.047020in}}%
\pgfpathlineto{\pgfqpoint{1.352515in}{1.038663in}}%
\pgfpathlineto{\pgfqpoint{1.350949in}{1.037490in}}%
\pgfpathlineto{\pgfqpoint{1.349025in}{1.026079in}}%
\pgfpathlineto{\pgfqpoint{1.347075in}{1.014668in}}%
\pgfpathlineto{\pgfqpoint{1.343517in}{1.003257in}}%
\pgfpathlineto{\pgfqpoint{1.340783in}{0.997588in}}%
\pgfpathlineto{\pgfqpoint{1.337840in}{0.991847in}}%
\pgfpathclose%
\pgfpathmoveto{\pgfqpoint{1.352485in}{1.003257in}}%
\pgfpathlineto{\pgfqpoint{1.352515in}{1.003416in}}%
\pgfpathlineto{\pgfqpoint{1.354181in}{1.014668in}}%
\pgfpathlineto{\pgfqpoint{1.357319in}{1.026079in}}%
\pgfpathlineto{\pgfqpoint{1.364247in}{1.036637in}}%
\pgfpathlineto{\pgfqpoint{1.364768in}{1.037490in}}%
\pgfpathlineto{\pgfqpoint{1.375169in}{1.048901in}}%
\pgfpathlineto{\pgfqpoint{1.375980in}{1.049392in}}%
\pgfpathlineto{\pgfqpoint{1.387712in}{1.052653in}}%
\pgfpathlineto{\pgfqpoint{1.399444in}{1.055296in}}%
\pgfpathlineto{\pgfqpoint{1.403161in}{1.048901in}}%
\pgfpathlineto{\pgfqpoint{1.400366in}{1.037490in}}%
\pgfpathlineto{\pgfqpoint{1.399475in}{1.026079in}}%
\pgfpathlineto{\pgfqpoint{1.399444in}{1.025874in}}%
\pgfpathlineto{\pgfqpoint{1.395338in}{1.014668in}}%
\pgfpathlineto{\pgfqpoint{1.387712in}{1.008547in}}%
\pgfpathlineto{\pgfqpoint{1.377273in}{1.003257in}}%
\pgfpathlineto{\pgfqpoint{1.375980in}{1.002804in}}%
\pgfpathlineto{\pgfqpoint{1.372547in}{1.003257in}}%
\pgfpathlineto{\pgfqpoint{1.364247in}{1.004435in}}%
\pgfpathlineto{\pgfqpoint{1.352748in}{1.003257in}}%
\pgfpathlineto{\pgfqpoint{1.352515in}{1.003219in}}%
\pgfpathclose%
\pgfusepath{fill}%
\end{pgfscope}%
\begin{pgfscope}%
\pgfpathrectangle{\pgfqpoint{0.211875in}{0.211875in}}{\pgfqpoint{1.313625in}{1.279725in}}%
\pgfusepath{clip}%
\pgfsetbuttcap%
\pgfsetroundjoin%
\definecolor{currentfill}{rgb}{0.953126,0.456614,0.312398}%
\pgfsetfillcolor{currentfill}%
\pgfsetlinewidth{0.000000pt}%
\definecolor{currentstroke}{rgb}{0.000000,0.000000,0.000000}%
\pgfsetstrokecolor{currentstroke}%
\pgfsetdash{}{0pt}%
\pgfpathmoveto{\pgfqpoint{0.871495in}{1.022860in}}%
\pgfpathlineto{\pgfqpoint{0.878671in}{1.026079in}}%
\pgfpathlineto{\pgfqpoint{0.883227in}{1.032232in}}%
\pgfpathlineto{\pgfqpoint{0.886088in}{1.037490in}}%
\pgfpathlineto{\pgfqpoint{0.891375in}{1.048901in}}%
\pgfpathlineto{\pgfqpoint{0.894959in}{1.058076in}}%
\pgfpathlineto{\pgfqpoint{0.896915in}{1.060311in}}%
\pgfpathlineto{\pgfqpoint{0.906691in}{1.062658in}}%
\pgfpathlineto{\pgfqpoint{0.918424in}{1.065067in}}%
\pgfpathlineto{\pgfqpoint{0.930156in}{1.067541in}}%
\pgfpathlineto{\pgfqpoint{0.941888in}{1.070372in}}%
\pgfpathlineto{\pgfqpoint{0.946469in}{1.071722in}}%
\pgfpathlineto{\pgfqpoint{0.953620in}{1.074018in}}%
\pgfpathlineto{\pgfqpoint{0.965352in}{1.077623in}}%
\pgfpathlineto{\pgfqpoint{0.977085in}{1.081582in}}%
\pgfpathlineto{\pgfqpoint{0.981635in}{1.083133in}}%
\pgfpathlineto{\pgfqpoint{0.977085in}{1.083923in}}%
\pgfpathlineto{\pgfqpoint{0.965352in}{1.086572in}}%
\pgfpathlineto{\pgfqpoint{0.953620in}{1.090299in}}%
\pgfpathlineto{\pgfqpoint{0.946623in}{1.094544in}}%
\pgfpathlineto{\pgfqpoint{0.941888in}{1.095782in}}%
\pgfpathlineto{\pgfqpoint{0.930156in}{1.098935in}}%
\pgfpathlineto{\pgfqpoint{0.918424in}{1.102288in}}%
\pgfpathlineto{\pgfqpoint{0.906691in}{1.105946in}}%
\pgfpathlineto{\pgfqpoint{0.906666in}{1.105954in}}%
\pgfpathlineto{\pgfqpoint{0.894959in}{1.109377in}}%
\pgfpathlineto{\pgfqpoint{0.883227in}{1.111847in}}%
\pgfpathlineto{\pgfqpoint{0.871495in}{1.113796in}}%
\pgfpathlineto{\pgfqpoint{0.859762in}{1.115388in}}%
\pgfpathlineto{\pgfqpoint{0.848030in}{1.116736in}}%
\pgfpathlineto{\pgfqpoint{0.841999in}{1.117365in}}%
\pgfpathlineto{\pgfqpoint{0.836298in}{1.118103in}}%
\pgfpathlineto{\pgfqpoint{0.834251in}{1.117365in}}%
\pgfpathlineto{\pgfqpoint{0.824566in}{1.113367in}}%
\pgfpathlineto{\pgfqpoint{0.812834in}{1.112807in}}%
\pgfpathlineto{\pgfqpoint{0.803990in}{1.117365in}}%
\pgfpathlineto{\pgfqpoint{0.801101in}{1.119391in}}%
\pgfpathlineto{\pgfqpoint{0.790015in}{1.128776in}}%
\pgfpathlineto{\pgfqpoint{0.789369in}{1.129835in}}%
\pgfpathlineto{\pgfqpoint{0.782381in}{1.140187in}}%
\pgfpathlineto{\pgfqpoint{0.779052in}{1.151598in}}%
\pgfpathlineto{\pgfqpoint{0.777637in}{1.156903in}}%
\pgfpathlineto{\pgfqpoint{0.775600in}{1.163008in}}%
\pgfpathlineto{\pgfqpoint{0.772721in}{1.174419in}}%
\pgfpathlineto{\pgfqpoint{0.768035in}{1.185830in}}%
\pgfpathlineto{\pgfqpoint{0.765905in}{1.189786in}}%
\pgfpathlineto{\pgfqpoint{0.760984in}{1.197241in}}%
\pgfpathlineto{\pgfqpoint{0.754173in}{1.208077in}}%
\pgfpathlineto{\pgfqpoint{0.753791in}{1.208651in}}%
\pgfpathlineto{\pgfqpoint{0.746020in}{1.220062in}}%
\pgfpathlineto{\pgfqpoint{0.742440in}{1.225592in}}%
\pgfpathlineto{\pgfqpoint{0.738402in}{1.231473in}}%
\pgfpathlineto{\pgfqpoint{0.731012in}{1.242884in}}%
\pgfpathlineto{\pgfqpoint{0.730708in}{1.243328in}}%
\pgfpathlineto{\pgfqpoint{0.720270in}{1.254295in}}%
\pgfpathlineto{\pgfqpoint{0.718976in}{1.255393in}}%
\pgfpathlineto{\pgfqpoint{0.707244in}{1.265018in}}%
\pgfpathlineto{\pgfqpoint{0.704611in}{1.265705in}}%
\pgfpathlineto{\pgfqpoint{0.695512in}{1.270016in}}%
\pgfpathlineto{\pgfqpoint{0.689481in}{1.265705in}}%
\pgfpathlineto{\pgfqpoint{0.688333in}{1.254295in}}%
\pgfpathlineto{\pgfqpoint{0.689816in}{1.242884in}}%
\pgfpathlineto{\pgfqpoint{0.692809in}{1.231473in}}%
\pgfpathlineto{\pgfqpoint{0.695512in}{1.221500in}}%
\pgfpathlineto{\pgfqpoint{0.695931in}{1.220062in}}%
\pgfpathlineto{\pgfqpoint{0.701270in}{1.208651in}}%
\pgfpathlineto{\pgfqpoint{0.707244in}{1.200375in}}%
\pgfpathlineto{\pgfqpoint{0.709463in}{1.197241in}}%
\pgfpathlineto{\pgfqpoint{0.717747in}{1.185830in}}%
\pgfpathlineto{\pgfqpoint{0.718976in}{1.184298in}}%
\pgfpathlineto{\pgfqpoint{0.726816in}{1.174419in}}%
\pgfpathlineto{\pgfqpoint{0.730708in}{1.169619in}}%
\pgfpathlineto{\pgfqpoint{0.736049in}{1.163008in}}%
\pgfpathlineto{\pgfqpoint{0.742440in}{1.155354in}}%
\pgfpathlineto{\pgfqpoint{0.745704in}{1.151598in}}%
\pgfpathlineto{\pgfqpoint{0.754173in}{1.143393in}}%
\pgfpathlineto{\pgfqpoint{0.758095in}{1.140187in}}%
\pgfpathlineto{\pgfqpoint{0.765905in}{1.132892in}}%
\pgfpathlineto{\pgfqpoint{0.770453in}{1.128776in}}%
\pgfpathlineto{\pgfqpoint{0.776391in}{1.117365in}}%
\pgfpathlineto{\pgfqpoint{0.777637in}{1.114195in}}%
\pgfpathlineto{\pgfqpoint{0.780056in}{1.105954in}}%
\pgfpathlineto{\pgfqpoint{0.779767in}{1.094544in}}%
\pgfpathlineto{\pgfqpoint{0.785454in}{1.083133in}}%
\pgfpathlineto{\pgfqpoint{0.789369in}{1.078824in}}%
\pgfpathlineto{\pgfqpoint{0.797447in}{1.071722in}}%
\pgfpathlineto{\pgfqpoint{0.801101in}{1.068768in}}%
\pgfpathlineto{\pgfqpoint{0.812834in}{1.069851in}}%
\pgfpathlineto{\pgfqpoint{0.814351in}{1.071722in}}%
\pgfpathlineto{\pgfqpoint{0.824566in}{1.082044in}}%
\pgfpathlineto{\pgfqpoint{0.835863in}{1.071722in}}%
\pgfpathlineto{\pgfqpoint{0.836298in}{1.070854in}}%
\pgfpathlineto{\pgfqpoint{0.842394in}{1.060311in}}%
\pgfpathlineto{\pgfqpoint{0.847930in}{1.048901in}}%
\pgfpathlineto{\pgfqpoint{0.848030in}{1.048753in}}%
\pgfpathlineto{\pgfqpoint{0.856370in}{1.037490in}}%
\pgfpathlineto{\pgfqpoint{0.859762in}{1.034106in}}%
\pgfpathlineto{\pgfqpoint{0.868228in}{1.026079in}}%
\pgfpathclose%
\pgfpathmoveto{\pgfqpoint{0.856867in}{1.060311in}}%
\pgfpathlineto{\pgfqpoint{0.848030in}{1.070300in}}%
\pgfpathlineto{\pgfqpoint{0.847136in}{1.071722in}}%
\pgfpathlineto{\pgfqpoint{0.836298in}{1.081408in}}%
\pgfpathlineto{\pgfqpoint{0.834410in}{1.083133in}}%
\pgfpathlineto{\pgfqpoint{0.824566in}{1.090952in}}%
\pgfpathlineto{\pgfqpoint{0.815222in}{1.083133in}}%
\pgfpathlineto{\pgfqpoint{0.812834in}{1.080715in}}%
\pgfpathlineto{\pgfqpoint{0.801101in}{1.075232in}}%
\pgfpathlineto{\pgfqpoint{0.792356in}{1.083133in}}%
\pgfpathlineto{\pgfqpoint{0.789369in}{1.086543in}}%
\pgfpathlineto{\pgfqpoint{0.784397in}{1.094544in}}%
\pgfpathlineto{\pgfqpoint{0.786170in}{1.105954in}}%
\pgfpathlineto{\pgfqpoint{0.788791in}{1.117365in}}%
\pgfpathlineto{\pgfqpoint{0.789369in}{1.117892in}}%
\pgfpathlineto{\pgfqpoint{0.790099in}{1.117365in}}%
\pgfpathlineto{\pgfqpoint{0.801101in}{1.112622in}}%
\pgfpathlineto{\pgfqpoint{0.812834in}{1.108386in}}%
\pgfpathlineto{\pgfqpoint{0.824566in}{1.108223in}}%
\pgfpathlineto{\pgfqpoint{0.836298in}{1.114113in}}%
\pgfpathlineto{\pgfqpoint{0.848030in}{1.113117in}}%
\pgfpathlineto{\pgfqpoint{0.859762in}{1.111469in}}%
\pgfpathlineto{\pgfqpoint{0.871495in}{1.109477in}}%
\pgfpathlineto{\pgfqpoint{0.883227in}{1.107012in}}%
\pgfpathlineto{\pgfqpoint{0.887482in}{1.105954in}}%
\pgfpathlineto{\pgfqpoint{0.894959in}{1.102990in}}%
\pgfpathlineto{\pgfqpoint{0.906691in}{1.099669in}}%
\pgfpathlineto{\pgfqpoint{0.918424in}{1.096359in}}%
\pgfpathlineto{\pgfqpoint{0.925132in}{1.094544in}}%
\pgfpathlineto{\pgfqpoint{0.926786in}{1.083133in}}%
\pgfpathlineto{\pgfqpoint{0.918424in}{1.080759in}}%
\pgfpathlineto{\pgfqpoint{0.906691in}{1.077407in}}%
\pgfpathlineto{\pgfqpoint{0.894959in}{1.074050in}}%
\pgfpathlineto{\pgfqpoint{0.886440in}{1.071722in}}%
\pgfpathlineto{\pgfqpoint{0.883227in}{1.070839in}}%
\pgfpathlineto{\pgfqpoint{0.871495in}{1.065895in}}%
\pgfpathlineto{\pgfqpoint{0.865296in}{1.060311in}}%
\pgfpathlineto{\pgfqpoint{0.859762in}{1.052273in}}%
\pgfpathclose%
\pgfpathmoveto{\pgfqpoint{0.751725in}{1.163008in}}%
\pgfpathlineto{\pgfqpoint{0.742440in}{1.171817in}}%
\pgfpathlineto{\pgfqpoint{0.740377in}{1.174419in}}%
\pgfpathlineto{\pgfqpoint{0.730739in}{1.185830in}}%
\pgfpathlineto{\pgfqpoint{0.730708in}{1.185865in}}%
\pgfpathlineto{\pgfqpoint{0.723697in}{1.197241in}}%
\pgfpathlineto{\pgfqpoint{0.718976in}{1.202806in}}%
\pgfpathlineto{\pgfqpoint{0.714938in}{1.208651in}}%
\pgfpathlineto{\pgfqpoint{0.707244in}{1.220058in}}%
\pgfpathlineto{\pgfqpoint{0.707242in}{1.220062in}}%
\pgfpathlineto{\pgfqpoint{0.703979in}{1.231473in}}%
\pgfpathlineto{\pgfqpoint{0.704439in}{1.242884in}}%
\pgfpathlineto{\pgfqpoint{0.707244in}{1.248421in}}%
\pgfpathlineto{\pgfqpoint{0.718976in}{1.246896in}}%
\pgfpathlineto{\pgfqpoint{0.722811in}{1.242884in}}%
\pgfpathlineto{\pgfqpoint{0.730708in}{1.232296in}}%
\pgfpathlineto{\pgfqpoint{0.731238in}{1.231473in}}%
\pgfpathlineto{\pgfqpoint{0.737603in}{1.220062in}}%
\pgfpathlineto{\pgfqpoint{0.742440in}{1.214134in}}%
\pgfpathlineto{\pgfqpoint{0.745879in}{1.208651in}}%
\pgfpathlineto{\pgfqpoint{0.752715in}{1.197241in}}%
\pgfpathlineto{\pgfqpoint{0.754173in}{1.195894in}}%
\pgfpathlineto{\pgfqpoint{0.760117in}{1.185830in}}%
\pgfpathlineto{\pgfqpoint{0.765905in}{1.175219in}}%
\pgfpathlineto{\pgfqpoint{0.766224in}{1.174419in}}%
\pgfpathlineto{\pgfqpoint{0.767685in}{1.163008in}}%
\pgfpathlineto{\pgfqpoint{0.765905in}{1.156119in}}%
\pgfpathlineto{\pgfqpoint{0.754173in}{1.159988in}}%
\pgfpathclose%
\pgfusepath{fill}%
\end{pgfscope}%
\begin{pgfscope}%
\pgfpathrectangle{\pgfqpoint{0.211875in}{0.211875in}}{\pgfqpoint{1.313625in}{1.279725in}}%
\pgfusepath{clip}%
\pgfsetbuttcap%
\pgfsetroundjoin%
\definecolor{currentfill}{rgb}{0.953126,0.456614,0.312398}%
\pgfsetfillcolor{currentfill}%
\pgfsetlinewidth{0.000000pt}%
\definecolor{currentstroke}{rgb}{0.000000,0.000000,0.000000}%
\pgfsetstrokecolor{currentstroke}%
\pgfsetdash{}{0pt}%
\pgfpathmoveto{\pgfqpoint{1.270390in}{1.126089in}}%
\pgfpathlineto{\pgfqpoint{1.282122in}{1.123267in}}%
\pgfpathlineto{\pgfqpoint{1.293854in}{1.120375in}}%
\pgfpathlineto{\pgfqpoint{1.305586in}{1.122958in}}%
\pgfpathlineto{\pgfqpoint{1.312470in}{1.128776in}}%
\pgfpathlineto{\pgfqpoint{1.314776in}{1.140187in}}%
\pgfpathlineto{\pgfqpoint{1.310696in}{1.151598in}}%
\pgfpathlineto{\pgfqpoint{1.305586in}{1.160515in}}%
\pgfpathlineto{\pgfqpoint{1.304117in}{1.163008in}}%
\pgfpathlineto{\pgfqpoint{1.296973in}{1.174419in}}%
\pgfpathlineto{\pgfqpoint{1.293854in}{1.178256in}}%
\pgfpathlineto{\pgfqpoint{1.287798in}{1.185830in}}%
\pgfpathlineto{\pgfqpoint{1.282122in}{1.191386in}}%
\pgfpathlineto{\pgfqpoint{1.274253in}{1.197241in}}%
\pgfpathlineto{\pgfqpoint{1.270390in}{1.200474in}}%
\pgfpathlineto{\pgfqpoint{1.260111in}{1.208651in}}%
\pgfpathlineto{\pgfqpoint{1.258658in}{1.210192in}}%
\pgfpathlineto{\pgfqpoint{1.248635in}{1.220062in}}%
\pgfpathlineto{\pgfqpoint{1.246925in}{1.221780in}}%
\pgfpathlineto{\pgfqpoint{1.241676in}{1.220062in}}%
\pgfpathlineto{\pgfqpoint{1.242214in}{1.208651in}}%
\pgfpathlineto{\pgfqpoint{1.245648in}{1.197241in}}%
\pgfpathlineto{\pgfqpoint{1.246925in}{1.192420in}}%
\pgfpathlineto{\pgfqpoint{1.248373in}{1.185830in}}%
\pgfpathlineto{\pgfqpoint{1.248524in}{1.174419in}}%
\pgfpathlineto{\pgfqpoint{1.248775in}{1.163008in}}%
\pgfpathlineto{\pgfqpoint{1.246925in}{1.156227in}}%
\pgfpathlineto{\pgfqpoint{1.245270in}{1.151598in}}%
\pgfpathlineto{\pgfqpoint{1.239336in}{1.140187in}}%
\pgfpathlineto{\pgfqpoint{1.246925in}{1.133070in}}%
\pgfpathlineto{\pgfqpoint{1.258658in}{1.128869in}}%
\pgfpathlineto{\pgfqpoint{1.258954in}{1.128776in}}%
\pgfpathclose%
\pgfpathmoveto{\pgfqpoint{1.276833in}{1.140187in}}%
\pgfpathlineto{\pgfqpoint{1.273424in}{1.151598in}}%
\pgfpathlineto{\pgfqpoint{1.270390in}{1.154900in}}%
\pgfpathlineto{\pgfqpoint{1.265422in}{1.163008in}}%
\pgfpathlineto{\pgfqpoint{1.263169in}{1.174419in}}%
\pgfpathlineto{\pgfqpoint{1.263281in}{1.185830in}}%
\pgfpathlineto{\pgfqpoint{1.270390in}{1.190703in}}%
\pgfpathlineto{\pgfqpoint{1.277949in}{1.185830in}}%
\pgfpathlineto{\pgfqpoint{1.282122in}{1.182495in}}%
\pgfpathlineto{\pgfqpoint{1.288061in}{1.174419in}}%
\pgfpathlineto{\pgfqpoint{1.293854in}{1.164173in}}%
\pgfpathlineto{\pgfqpoint{1.294505in}{1.163008in}}%
\pgfpathlineto{\pgfqpoint{1.300433in}{1.151598in}}%
\pgfpathlineto{\pgfqpoint{1.304821in}{1.140187in}}%
\pgfpathlineto{\pgfqpoint{1.293854in}{1.135340in}}%
\pgfpathlineto{\pgfqpoint{1.282122in}{1.138327in}}%
\pgfpathclose%
\pgfusepath{fill}%
\end{pgfscope}%
\begin{pgfscope}%
\pgfpathrectangle{\pgfqpoint{0.211875in}{0.211875in}}{\pgfqpoint{1.313625in}{1.279725in}}%
\pgfusepath{clip}%
\pgfsetbuttcap%
\pgfsetroundjoin%
\definecolor{currentfill}{rgb}{0.953126,0.456614,0.312398}%
\pgfsetfillcolor{currentfill}%
\pgfsetlinewidth{0.000000pt}%
\definecolor{currentstroke}{rgb}{0.000000,0.000000,0.000000}%
\pgfsetstrokecolor{currentstroke}%
\pgfsetdash{}{0pt}%
\pgfpathmoveto{\pgfqpoint{0.754173in}{1.294455in}}%
\pgfpathlineto{\pgfqpoint{0.756609in}{1.299938in}}%
\pgfpathlineto{\pgfqpoint{0.754173in}{1.305672in}}%
\pgfpathlineto{\pgfqpoint{0.752011in}{1.299938in}}%
\pgfpathclose%
\pgfusepath{fill}%
\end{pgfscope}%
\begin{pgfscope}%
\pgfpathrectangle{\pgfqpoint{0.211875in}{0.211875in}}{\pgfqpoint{1.313625in}{1.279725in}}%
\pgfusepath{clip}%
\pgfsetbuttcap%
\pgfsetroundjoin%
\definecolor{currentfill}{rgb}{0.953126,0.456614,0.312398}%
\pgfsetfillcolor{currentfill}%
\pgfsetlinewidth{0.000000pt}%
\definecolor{currentstroke}{rgb}{0.000000,0.000000,0.000000}%
\pgfsetstrokecolor{currentstroke}%
\pgfsetdash{}{0pt}%
\pgfpathmoveto{\pgfqpoint{0.742440in}{1.306624in}}%
\pgfpathlineto{\pgfqpoint{0.751353in}{1.311348in}}%
\pgfpathlineto{\pgfqpoint{0.745684in}{1.322759in}}%
\pgfpathlineto{\pgfqpoint{0.742440in}{1.326384in}}%
\pgfpathlineto{\pgfqpoint{0.734581in}{1.322759in}}%
\pgfpathlineto{\pgfqpoint{0.739212in}{1.311348in}}%
\pgfpathclose%
\pgfusepath{fill}%
\end{pgfscope}%
\begin{pgfscope}%
\pgfpathrectangle{\pgfqpoint{0.211875in}{0.211875in}}{\pgfqpoint{1.313625in}{1.279725in}}%
\pgfusepath{clip}%
\pgfsetbuttcap%
\pgfsetroundjoin%
\definecolor{currentfill}{rgb}{0.953126,0.456614,0.312398}%
\pgfsetfillcolor{currentfill}%
\pgfsetlinewidth{0.000000pt}%
\definecolor{currentstroke}{rgb}{0.000000,0.000000,0.000000}%
\pgfsetstrokecolor{currentstroke}%
\pgfsetdash{}{0pt}%
\pgfpathmoveto{\pgfqpoint{0.730708in}{1.329213in}}%
\pgfpathlineto{\pgfqpoint{0.732466in}{1.334170in}}%
\pgfpathlineto{\pgfqpoint{0.730708in}{1.336257in}}%
\pgfpathlineto{\pgfqpoint{0.729296in}{1.334170in}}%
\pgfpathclose%
\pgfusepath{fill}%
\end{pgfscope}%
\begin{pgfscope}%
\pgfpathrectangle{\pgfqpoint{0.211875in}{0.211875in}}{\pgfqpoint{1.313625in}{1.279725in}}%
\pgfusepath{clip}%
\pgfsetbuttcap%
\pgfsetroundjoin%
\definecolor{currentfill}{rgb}{0.953126,0.456614,0.312398}%
\pgfsetfillcolor{currentfill}%
\pgfsetlinewidth{0.000000pt}%
\definecolor{currentstroke}{rgb}{0.000000,0.000000,0.000000}%
\pgfsetstrokecolor{currentstroke}%
\pgfsetdash{}{0pt}%
\pgfpathmoveto{\pgfqpoint{1.387712in}{1.366391in}}%
\pgfpathlineto{\pgfqpoint{1.399444in}{1.363241in}}%
\pgfpathlineto{\pgfqpoint{1.411176in}{1.361776in}}%
\pgfpathlineto{\pgfqpoint{1.422908in}{1.362268in}}%
\pgfpathlineto{\pgfqpoint{1.434641in}{1.363866in}}%
\pgfpathlineto{\pgfqpoint{1.446373in}{1.366181in}}%
\pgfpathlineto{\pgfqpoint{1.446373in}{1.368402in}}%
\pgfpathlineto{\pgfqpoint{1.446373in}{1.379813in}}%
\pgfpathlineto{\pgfqpoint{1.446373in}{1.388780in}}%
\pgfpathlineto{\pgfqpoint{1.434641in}{1.386891in}}%
\pgfpathlineto{\pgfqpoint{1.422908in}{1.386476in}}%
\pgfpathlineto{\pgfqpoint{1.411176in}{1.387485in}}%
\pgfpathlineto{\pgfqpoint{1.399444in}{1.390072in}}%
\pgfpathlineto{\pgfqpoint{1.396875in}{1.391224in}}%
\pgfpathlineto{\pgfqpoint{1.387712in}{1.396254in}}%
\pgfpathlineto{\pgfqpoint{1.380212in}{1.402635in}}%
\pgfpathlineto{\pgfqpoint{1.375980in}{1.406566in}}%
\pgfpathlineto{\pgfqpoint{1.369750in}{1.414045in}}%
\pgfpathlineto{\pgfqpoint{1.364247in}{1.414045in}}%
\pgfpathlineto{\pgfqpoint{1.352515in}{1.414045in}}%
\pgfpathlineto{\pgfqpoint{1.342851in}{1.414045in}}%
\pgfpathlineto{\pgfqpoint{1.350082in}{1.402635in}}%
\pgfpathlineto{\pgfqpoint{1.352515in}{1.398897in}}%
\pgfpathlineto{\pgfqpoint{1.358043in}{1.391224in}}%
\pgfpathlineto{\pgfqpoint{1.364247in}{1.383347in}}%
\pgfpathlineto{\pgfqpoint{1.367722in}{1.379813in}}%
\pgfpathlineto{\pgfqpoint{1.375980in}{1.372194in}}%
\pgfpathlineto{\pgfqpoint{1.383086in}{1.368402in}}%
\pgfpathclose%
\pgfusepath{fill}%
\end{pgfscope}%
\begin{pgfscope}%
\pgfpathrectangle{\pgfqpoint{0.211875in}{0.211875in}}{\pgfqpoint{1.313625in}{1.279725in}}%
\pgfusepath{clip}%
\pgfsetbuttcap%
\pgfsetroundjoin%
\definecolor{currentfill}{rgb}{0.962532,0.599594,0.438051}%
\pgfsetfillcolor{currentfill}%
\pgfsetlinewidth{0.000000pt}%
\definecolor{currentstroke}{rgb}{0.000000,0.000000,0.000000}%
\pgfsetstrokecolor{currentstroke}%
\pgfsetdash{}{0pt}%
\pgfpathmoveto{\pgfqpoint{1.270390in}{0.877141in}}%
\pgfpathlineto{\pgfqpoint{1.272187in}{0.877739in}}%
\pgfpathlineto{\pgfqpoint{1.282122in}{0.887762in}}%
\pgfpathlineto{\pgfqpoint{1.282816in}{0.889150in}}%
\pgfpathlineto{\pgfqpoint{1.282122in}{0.890813in}}%
\pgfpathlineto{\pgfqpoint{1.277858in}{0.900560in}}%
\pgfpathlineto{\pgfqpoint{1.270390in}{0.901868in}}%
\pgfpathlineto{\pgfqpoint{1.267688in}{0.900560in}}%
\pgfpathlineto{\pgfqpoint{1.258658in}{0.891513in}}%
\pgfpathlineto{\pgfqpoint{1.256200in}{0.889150in}}%
\pgfpathlineto{\pgfqpoint{1.258658in}{0.884933in}}%
\pgfpathlineto{\pgfqpoint{1.269466in}{0.877739in}}%
\pgfpathclose%
\pgfusepath{fill}%
\end{pgfscope}%
\begin{pgfscope}%
\pgfpathrectangle{\pgfqpoint{0.211875in}{0.211875in}}{\pgfqpoint{1.313625in}{1.279725in}}%
\pgfusepath{clip}%
\pgfsetbuttcap%
\pgfsetroundjoin%
\definecolor{currentfill}{rgb}{0.962532,0.599594,0.438051}%
\pgfsetfillcolor{currentfill}%
\pgfsetlinewidth{0.000000pt}%
\definecolor{currentstroke}{rgb}{0.000000,0.000000,0.000000}%
\pgfsetstrokecolor{currentstroke}%
\pgfsetdash{}{0pt}%
\pgfpathmoveto{\pgfqpoint{0.883227in}{0.933608in}}%
\pgfpathlineto{\pgfqpoint{0.894959in}{0.930962in}}%
\pgfpathlineto{\pgfqpoint{0.906691in}{0.931107in}}%
\pgfpathlineto{\pgfqpoint{0.912239in}{0.934793in}}%
\pgfpathlineto{\pgfqpoint{0.918424in}{0.939317in}}%
\pgfpathlineto{\pgfqpoint{0.923231in}{0.946204in}}%
\pgfpathlineto{\pgfqpoint{0.930156in}{0.957539in}}%
\pgfpathlineto{\pgfqpoint{0.930237in}{0.957614in}}%
\pgfpathlineto{\pgfqpoint{0.930156in}{0.957738in}}%
\pgfpathlineto{\pgfqpoint{0.918424in}{0.967120in}}%
\pgfpathlineto{\pgfqpoint{0.913565in}{0.969025in}}%
\pgfpathlineto{\pgfqpoint{0.906691in}{0.970504in}}%
\pgfpathlineto{\pgfqpoint{0.899281in}{0.969025in}}%
\pgfpathlineto{\pgfqpoint{0.894959in}{0.967922in}}%
\pgfpathlineto{\pgfqpoint{0.885657in}{0.957614in}}%
\pgfpathlineto{\pgfqpoint{0.883227in}{0.954093in}}%
\pgfpathlineto{\pgfqpoint{0.879343in}{0.946204in}}%
\pgfpathlineto{\pgfqpoint{0.880253in}{0.934793in}}%
\pgfpathclose%
\pgfpathmoveto{\pgfqpoint{0.886491in}{0.946204in}}%
\pgfpathlineto{\pgfqpoint{0.893963in}{0.957614in}}%
\pgfpathlineto{\pgfqpoint{0.894959in}{0.958718in}}%
\pgfpathlineto{\pgfqpoint{0.906691in}{0.962342in}}%
\pgfpathlineto{\pgfqpoint{0.914567in}{0.957614in}}%
\pgfpathlineto{\pgfqpoint{0.916524in}{0.946204in}}%
\pgfpathlineto{\pgfqpoint{0.906691in}{0.938189in}}%
\pgfpathlineto{\pgfqpoint{0.894959in}{0.937310in}}%
\pgfpathclose%
\pgfusepath{fill}%
\end{pgfscope}%
\begin{pgfscope}%
\pgfpathrectangle{\pgfqpoint{0.211875in}{0.211875in}}{\pgfqpoint{1.313625in}{1.279725in}}%
\pgfusepath{clip}%
\pgfsetbuttcap%
\pgfsetroundjoin%
\definecolor{currentfill}{rgb}{0.962532,0.599594,0.438051}%
\pgfsetfillcolor{currentfill}%
\pgfsetlinewidth{0.000000pt}%
\definecolor{currentstroke}{rgb}{0.000000,0.000000,0.000000}%
\pgfsetstrokecolor{currentstroke}%
\pgfsetdash{}{0pt}%
\pgfpathmoveto{\pgfqpoint{1.317319in}{0.926544in}}%
\pgfpathlineto{\pgfqpoint{1.329051in}{0.924947in}}%
\pgfpathlineto{\pgfqpoint{1.337930in}{0.934793in}}%
\pgfpathlineto{\pgfqpoint{1.340783in}{0.937454in}}%
\pgfpathlineto{\pgfqpoint{1.351049in}{0.946204in}}%
\pgfpathlineto{\pgfqpoint{1.350888in}{0.957614in}}%
\pgfpathlineto{\pgfqpoint{1.340783in}{0.962123in}}%
\pgfpathlineto{\pgfqpoint{1.331207in}{0.969025in}}%
\pgfpathlineto{\pgfqpoint{1.329051in}{0.970787in}}%
\pgfpathlineto{\pgfqpoint{1.324183in}{0.969025in}}%
\pgfpathlineto{\pgfqpoint{1.317319in}{0.966577in}}%
\pgfpathlineto{\pgfqpoint{1.305586in}{0.962181in}}%
\pgfpathlineto{\pgfqpoint{1.295794in}{0.957614in}}%
\pgfpathlineto{\pgfqpoint{1.303437in}{0.946204in}}%
\pgfpathlineto{\pgfqpoint{1.305586in}{0.940754in}}%
\pgfpathlineto{\pgfqpoint{1.307931in}{0.934793in}}%
\pgfpathclose%
\pgfpathmoveto{\pgfqpoint{1.317252in}{0.934793in}}%
\pgfpathlineto{\pgfqpoint{1.311974in}{0.946204in}}%
\pgfpathlineto{\pgfqpoint{1.313089in}{0.957614in}}%
\pgfpathlineto{\pgfqpoint{1.317319in}{0.959375in}}%
\pgfpathlineto{\pgfqpoint{1.329051in}{0.962979in}}%
\pgfpathlineto{\pgfqpoint{1.337159in}{0.957614in}}%
\pgfpathlineto{\pgfqpoint{1.340783in}{0.950846in}}%
\pgfpathlineto{\pgfqpoint{1.342540in}{0.946204in}}%
\pgfpathlineto{\pgfqpoint{1.340783in}{0.944706in}}%
\pgfpathlineto{\pgfqpoint{1.330154in}{0.934793in}}%
\pgfpathlineto{\pgfqpoint{1.329051in}{0.933569in}}%
\pgfpathlineto{\pgfqpoint{1.317319in}{0.934734in}}%
\pgfpathclose%
\pgfusepath{fill}%
\end{pgfscope}%
\begin{pgfscope}%
\pgfpathrectangle{\pgfqpoint{0.211875in}{0.211875in}}{\pgfqpoint{1.313625in}{1.279725in}}%
\pgfusepath{clip}%
\pgfsetbuttcap%
\pgfsetroundjoin%
\definecolor{currentfill}{rgb}{0.962532,0.599594,0.438051}%
\pgfsetfillcolor{currentfill}%
\pgfsetlinewidth{0.000000pt}%
\definecolor{currentstroke}{rgb}{0.000000,0.000000,0.000000}%
\pgfsetstrokecolor{currentstroke}%
\pgfsetdash{}{0pt}%
\pgfpathmoveto{\pgfqpoint{1.352515in}{1.003219in}}%
\pgfpathlineto{\pgfqpoint{1.352748in}{1.003257in}}%
\pgfpathlineto{\pgfqpoint{1.364247in}{1.004435in}}%
\pgfpathlineto{\pgfqpoint{1.372547in}{1.003257in}}%
\pgfpathlineto{\pgfqpoint{1.375980in}{1.002804in}}%
\pgfpathlineto{\pgfqpoint{1.377273in}{1.003257in}}%
\pgfpathlineto{\pgfqpoint{1.387712in}{1.008547in}}%
\pgfpathlineto{\pgfqpoint{1.395338in}{1.014668in}}%
\pgfpathlineto{\pgfqpoint{1.399444in}{1.025874in}}%
\pgfpathlineto{\pgfqpoint{1.399475in}{1.026079in}}%
\pgfpathlineto{\pgfqpoint{1.400366in}{1.037490in}}%
\pgfpathlineto{\pgfqpoint{1.403161in}{1.048901in}}%
\pgfpathlineto{\pgfqpoint{1.399444in}{1.055296in}}%
\pgfpathlineto{\pgfqpoint{1.387712in}{1.052653in}}%
\pgfpathlineto{\pgfqpoint{1.375980in}{1.049392in}}%
\pgfpathlineto{\pgfqpoint{1.375169in}{1.048901in}}%
\pgfpathlineto{\pgfqpoint{1.364768in}{1.037490in}}%
\pgfpathlineto{\pgfqpoint{1.364247in}{1.036637in}}%
\pgfpathlineto{\pgfqpoint{1.357319in}{1.026079in}}%
\pgfpathlineto{\pgfqpoint{1.354181in}{1.014668in}}%
\pgfpathlineto{\pgfqpoint{1.352515in}{1.003416in}}%
\pgfpathlineto{\pgfqpoint{1.352485in}{1.003257in}}%
\pgfpathclose%
\pgfpathmoveto{\pgfqpoint{1.361060in}{1.014668in}}%
\pgfpathlineto{\pgfqpoint{1.364247in}{1.021778in}}%
\pgfpathlineto{\pgfqpoint{1.366678in}{1.026079in}}%
\pgfpathlineto{\pgfqpoint{1.373733in}{1.037490in}}%
\pgfpathlineto{\pgfqpoint{1.375980in}{1.039949in}}%
\pgfpathlineto{\pgfqpoint{1.387712in}{1.044484in}}%
\pgfpathlineto{\pgfqpoint{1.393825in}{1.037490in}}%
\pgfpathlineto{\pgfqpoint{1.392602in}{1.026079in}}%
\pgfpathlineto{\pgfqpoint{1.387712in}{1.018427in}}%
\pgfpathlineto{\pgfqpoint{1.378811in}{1.014668in}}%
\pgfpathlineto{\pgfqpoint{1.375980in}{1.013452in}}%
\pgfpathlineto{\pgfqpoint{1.364247in}{1.011428in}}%
\pgfpathclose%
\pgfusepath{fill}%
\end{pgfscope}%
\begin{pgfscope}%
\pgfpathrectangle{\pgfqpoint{0.211875in}{0.211875in}}{\pgfqpoint{1.313625in}{1.279725in}}%
\pgfusepath{clip}%
\pgfsetbuttcap%
\pgfsetroundjoin%
\definecolor{currentfill}{rgb}{0.962532,0.599594,0.438051}%
\pgfsetfillcolor{currentfill}%
\pgfsetlinewidth{0.000000pt}%
\definecolor{currentstroke}{rgb}{0.000000,0.000000,0.000000}%
\pgfsetstrokecolor{currentstroke}%
\pgfsetdash{}{0pt}%
\pgfpathmoveto{\pgfqpoint{0.859762in}{1.052273in}}%
\pgfpathlineto{\pgfqpoint{0.865296in}{1.060311in}}%
\pgfpathlineto{\pgfqpoint{0.871495in}{1.065895in}}%
\pgfpathlineto{\pgfqpoint{0.883227in}{1.070839in}}%
\pgfpathlineto{\pgfqpoint{0.886440in}{1.071722in}}%
\pgfpathlineto{\pgfqpoint{0.894959in}{1.074050in}}%
\pgfpathlineto{\pgfqpoint{0.906691in}{1.077407in}}%
\pgfpathlineto{\pgfqpoint{0.918424in}{1.080759in}}%
\pgfpathlineto{\pgfqpoint{0.926786in}{1.083133in}}%
\pgfpathlineto{\pgfqpoint{0.925132in}{1.094544in}}%
\pgfpathlineto{\pgfqpoint{0.918424in}{1.096359in}}%
\pgfpathlineto{\pgfqpoint{0.906691in}{1.099669in}}%
\pgfpathlineto{\pgfqpoint{0.894959in}{1.102990in}}%
\pgfpathlineto{\pgfqpoint{0.887482in}{1.105954in}}%
\pgfpathlineto{\pgfqpoint{0.883227in}{1.107012in}}%
\pgfpathlineto{\pgfqpoint{0.871495in}{1.109477in}}%
\pgfpathlineto{\pgfqpoint{0.859762in}{1.111469in}}%
\pgfpathlineto{\pgfqpoint{0.848030in}{1.113117in}}%
\pgfpathlineto{\pgfqpoint{0.836298in}{1.114113in}}%
\pgfpathlineto{\pgfqpoint{0.824566in}{1.108223in}}%
\pgfpathlineto{\pgfqpoint{0.812834in}{1.108386in}}%
\pgfpathlineto{\pgfqpoint{0.801101in}{1.112622in}}%
\pgfpathlineto{\pgfqpoint{0.790099in}{1.117365in}}%
\pgfpathlineto{\pgfqpoint{0.789369in}{1.117892in}}%
\pgfpathlineto{\pgfqpoint{0.788791in}{1.117365in}}%
\pgfpathlineto{\pgfqpoint{0.786170in}{1.105954in}}%
\pgfpathlineto{\pgfqpoint{0.784397in}{1.094544in}}%
\pgfpathlineto{\pgfqpoint{0.789369in}{1.086543in}}%
\pgfpathlineto{\pgfqpoint{0.792356in}{1.083133in}}%
\pgfpathlineto{\pgfqpoint{0.801101in}{1.075232in}}%
\pgfpathlineto{\pgfqpoint{0.812834in}{1.080715in}}%
\pgfpathlineto{\pgfqpoint{0.815222in}{1.083133in}}%
\pgfpathlineto{\pgfqpoint{0.824566in}{1.090952in}}%
\pgfpathlineto{\pgfqpoint{0.834410in}{1.083133in}}%
\pgfpathlineto{\pgfqpoint{0.836298in}{1.081408in}}%
\pgfpathlineto{\pgfqpoint{0.847136in}{1.071722in}}%
\pgfpathlineto{\pgfqpoint{0.848030in}{1.070300in}}%
\pgfpathlineto{\pgfqpoint{0.856867in}{1.060311in}}%
\pgfpathclose%
\pgfpathmoveto{\pgfqpoint{0.798878in}{1.083133in}}%
\pgfpathlineto{\pgfqpoint{0.789369in}{1.093992in}}%
\pgfpathlineto{\pgfqpoint{0.789027in}{1.094544in}}%
\pgfpathlineto{\pgfqpoint{0.789369in}{1.096077in}}%
\pgfpathlineto{\pgfqpoint{0.801101in}{1.105308in}}%
\pgfpathlineto{\pgfqpoint{0.812834in}{1.100353in}}%
\pgfpathlineto{\pgfqpoint{0.818014in}{1.094544in}}%
\pgfpathlineto{\pgfqpoint{0.812834in}{1.090244in}}%
\pgfpathlineto{\pgfqpoint{0.804695in}{1.083133in}}%
\pgfpathlineto{\pgfqpoint{0.801101in}{1.081124in}}%
\pgfpathclose%
\pgfpathmoveto{\pgfqpoint{0.845752in}{1.083133in}}%
\pgfpathlineto{\pgfqpoint{0.836298in}{1.090460in}}%
\pgfpathlineto{\pgfqpoint{0.831148in}{1.094544in}}%
\pgfpathlineto{\pgfqpoint{0.828527in}{1.105954in}}%
\pgfpathlineto{\pgfqpoint{0.836298in}{1.110221in}}%
\pgfpathlineto{\pgfqpoint{0.848030in}{1.109498in}}%
\pgfpathlineto{\pgfqpoint{0.859762in}{1.107549in}}%
\pgfpathlineto{\pgfqpoint{0.867837in}{1.105954in}}%
\pgfpathlineto{\pgfqpoint{0.871495in}{1.103085in}}%
\pgfpathlineto{\pgfqpoint{0.883227in}{1.097858in}}%
\pgfpathlineto{\pgfqpoint{0.894959in}{1.095271in}}%
\pgfpathlineto{\pgfqpoint{0.898941in}{1.094544in}}%
\pgfpathlineto{\pgfqpoint{0.894959in}{1.092649in}}%
\pgfpathlineto{\pgfqpoint{0.883227in}{1.088916in}}%
\pgfpathlineto{\pgfqpoint{0.871495in}{1.085963in}}%
\pgfpathlineto{\pgfqpoint{0.859762in}{1.083561in}}%
\pgfpathlineto{\pgfqpoint{0.858212in}{1.083133in}}%
\pgfpathlineto{\pgfqpoint{0.848030in}{1.081099in}}%
\pgfpathclose%
\pgfusepath{fill}%
\end{pgfscope}%
\begin{pgfscope}%
\pgfpathrectangle{\pgfqpoint{0.211875in}{0.211875in}}{\pgfqpoint{1.313625in}{1.279725in}}%
\pgfusepath{clip}%
\pgfsetbuttcap%
\pgfsetroundjoin%
\definecolor{currentfill}{rgb}{0.962532,0.599594,0.438051}%
\pgfsetfillcolor{currentfill}%
\pgfsetlinewidth{0.000000pt}%
\definecolor{currentstroke}{rgb}{0.000000,0.000000,0.000000}%
\pgfsetstrokecolor{currentstroke}%
\pgfsetdash{}{0pt}%
\pgfpathmoveto{\pgfqpoint{1.106139in}{1.067957in}}%
\pgfpathlineto{\pgfqpoint{1.117871in}{1.066112in}}%
\pgfpathlineto{\pgfqpoint{1.129603in}{1.068728in}}%
\pgfpathlineto{\pgfqpoint{1.132193in}{1.071722in}}%
\pgfpathlineto{\pgfqpoint{1.132521in}{1.083133in}}%
\pgfpathlineto{\pgfqpoint{1.132802in}{1.094544in}}%
\pgfpathlineto{\pgfqpoint{1.135518in}{1.105954in}}%
\pgfpathlineto{\pgfqpoint{1.129603in}{1.110782in}}%
\pgfpathlineto{\pgfqpoint{1.117871in}{1.113473in}}%
\pgfpathlineto{\pgfqpoint{1.106139in}{1.113141in}}%
\pgfpathlineto{\pgfqpoint{1.096578in}{1.105954in}}%
\pgfpathlineto{\pgfqpoint{1.094536in}{1.094544in}}%
\pgfpathlineto{\pgfqpoint{1.094407in}{1.090276in}}%
\pgfpathlineto{\pgfqpoint{1.094032in}{1.083133in}}%
\pgfpathlineto{\pgfqpoint{1.094407in}{1.082646in}}%
\pgfpathlineto{\pgfqpoint{1.102337in}{1.071722in}}%
\pgfpathclose%
\pgfpathmoveto{\pgfqpoint{1.105120in}{1.083133in}}%
\pgfpathlineto{\pgfqpoint{1.101259in}{1.094544in}}%
\pgfpathlineto{\pgfqpoint{1.104790in}{1.105954in}}%
\pgfpathlineto{\pgfqpoint{1.106139in}{1.106968in}}%
\pgfpathlineto{\pgfqpoint{1.117871in}{1.108279in}}%
\pgfpathlineto{\pgfqpoint{1.124846in}{1.105954in}}%
\pgfpathlineto{\pgfqpoint{1.126778in}{1.094544in}}%
\pgfpathlineto{\pgfqpoint{1.126374in}{1.083133in}}%
\pgfpathlineto{\pgfqpoint{1.117871in}{1.074252in}}%
\pgfpathlineto{\pgfqpoint{1.106139in}{1.081725in}}%
\pgfpathclose%
\pgfusepath{fill}%
\end{pgfscope}%
\begin{pgfscope}%
\pgfpathrectangle{\pgfqpoint{0.211875in}{0.211875in}}{\pgfqpoint{1.313625in}{1.279725in}}%
\pgfusepath{clip}%
\pgfsetbuttcap%
\pgfsetroundjoin%
\definecolor{currentfill}{rgb}{0.962532,0.599594,0.438051}%
\pgfsetfillcolor{currentfill}%
\pgfsetlinewidth{0.000000pt}%
\definecolor{currentstroke}{rgb}{0.000000,0.000000,0.000000}%
\pgfsetstrokecolor{currentstroke}%
\pgfsetdash{}{0pt}%
\pgfpathmoveto{\pgfqpoint{1.422908in}{1.074594in}}%
\pgfpathlineto{\pgfqpoint{1.433203in}{1.083133in}}%
\pgfpathlineto{\pgfqpoint{1.434641in}{1.084084in}}%
\pgfpathlineto{\pgfqpoint{1.446373in}{1.094220in}}%
\pgfpathlineto{\pgfqpoint{1.446373in}{1.094544in}}%
\pgfpathlineto{\pgfqpoint{1.446373in}{1.099653in}}%
\pgfpathlineto{\pgfqpoint{1.439618in}{1.094544in}}%
\pgfpathlineto{\pgfqpoint{1.434641in}{1.090269in}}%
\pgfpathlineto{\pgfqpoint{1.423851in}{1.083133in}}%
\pgfpathlineto{\pgfqpoint{1.422908in}{1.082351in}}%
\pgfpathlineto{\pgfqpoint{1.422095in}{1.083133in}}%
\pgfpathlineto{\pgfqpoint{1.422500in}{1.094544in}}%
\pgfpathlineto{\pgfqpoint{1.422908in}{1.095246in}}%
\pgfpathlineto{\pgfqpoint{1.432625in}{1.105954in}}%
\pgfpathlineto{\pgfqpoint{1.434641in}{1.107185in}}%
\pgfpathlineto{\pgfqpoint{1.446373in}{1.110627in}}%
\pgfpathlineto{\pgfqpoint{1.446373in}{1.114664in}}%
\pgfpathlineto{\pgfqpoint{1.434641in}{1.111637in}}%
\pgfpathlineto{\pgfqpoint{1.425333in}{1.105954in}}%
\pgfpathlineto{\pgfqpoint{1.422908in}{1.103282in}}%
\pgfpathlineto{\pgfqpoint{1.417833in}{1.094544in}}%
\pgfpathlineto{\pgfqpoint{1.414030in}{1.083133in}}%
\pgfpathclose%
\pgfusepath{fill}%
\end{pgfscope}%
\begin{pgfscope}%
\pgfpathrectangle{\pgfqpoint{0.211875in}{0.211875in}}{\pgfqpoint{1.313625in}{1.279725in}}%
\pgfusepath{clip}%
\pgfsetbuttcap%
\pgfsetroundjoin%
\definecolor{currentfill}{rgb}{0.962532,0.599594,0.438051}%
\pgfsetfillcolor{currentfill}%
\pgfsetlinewidth{0.000000pt}%
\definecolor{currentstroke}{rgb}{0.000000,0.000000,0.000000}%
\pgfsetstrokecolor{currentstroke}%
\pgfsetdash{}{0pt}%
\pgfpathmoveto{\pgfqpoint{1.282122in}{1.138327in}}%
\pgfpathlineto{\pgfqpoint{1.293854in}{1.135340in}}%
\pgfpathlineto{\pgfqpoint{1.304821in}{1.140187in}}%
\pgfpathlineto{\pgfqpoint{1.300433in}{1.151598in}}%
\pgfpathlineto{\pgfqpoint{1.294505in}{1.163008in}}%
\pgfpathlineto{\pgfqpoint{1.293854in}{1.164173in}}%
\pgfpathlineto{\pgfqpoint{1.288061in}{1.174419in}}%
\pgfpathlineto{\pgfqpoint{1.282122in}{1.182495in}}%
\pgfpathlineto{\pgfqpoint{1.277949in}{1.185830in}}%
\pgfpathlineto{\pgfqpoint{1.270390in}{1.190703in}}%
\pgfpathlineto{\pgfqpoint{1.263281in}{1.185830in}}%
\pgfpathlineto{\pgfqpoint{1.263169in}{1.174419in}}%
\pgfpathlineto{\pgfqpoint{1.265422in}{1.163008in}}%
\pgfpathlineto{\pgfqpoint{1.270390in}{1.154900in}}%
\pgfpathlineto{\pgfqpoint{1.273424in}{1.151598in}}%
\pgfpathlineto{\pgfqpoint{1.276833in}{1.140187in}}%
\pgfpathclose%
\pgfusepath{fill}%
\end{pgfscope}%
\begin{pgfscope}%
\pgfpathrectangle{\pgfqpoint{0.211875in}{0.211875in}}{\pgfqpoint{1.313625in}{1.279725in}}%
\pgfusepath{clip}%
\pgfsetbuttcap%
\pgfsetroundjoin%
\definecolor{currentfill}{rgb}{0.962532,0.599594,0.438051}%
\pgfsetfillcolor{currentfill}%
\pgfsetlinewidth{0.000000pt}%
\definecolor{currentstroke}{rgb}{0.000000,0.000000,0.000000}%
\pgfsetstrokecolor{currentstroke}%
\pgfsetdash{}{0pt}%
\pgfpathmoveto{\pgfqpoint{1.059210in}{1.150888in}}%
\pgfpathlineto{\pgfqpoint{1.070942in}{1.147631in}}%
\pgfpathlineto{\pgfqpoint{1.076429in}{1.151598in}}%
\pgfpathlineto{\pgfqpoint{1.070942in}{1.156542in}}%
\pgfpathlineto{\pgfqpoint{1.060714in}{1.163008in}}%
\pgfpathlineto{\pgfqpoint{1.059210in}{1.164415in}}%
\pgfpathlineto{\pgfqpoint{1.057173in}{1.163008in}}%
\pgfpathlineto{\pgfqpoint{1.057318in}{1.151598in}}%
\pgfpathclose%
\pgfusepath{fill}%
\end{pgfscope}%
\begin{pgfscope}%
\pgfpathrectangle{\pgfqpoint{0.211875in}{0.211875in}}{\pgfqpoint{1.313625in}{1.279725in}}%
\pgfusepath{clip}%
\pgfsetbuttcap%
\pgfsetroundjoin%
\definecolor{currentfill}{rgb}{0.962532,0.599594,0.438051}%
\pgfsetfillcolor{currentfill}%
\pgfsetlinewidth{0.000000pt}%
\definecolor{currentstroke}{rgb}{0.000000,0.000000,0.000000}%
\pgfsetstrokecolor{currentstroke}%
\pgfsetdash{}{0pt}%
\pgfpathmoveto{\pgfqpoint{0.754173in}{1.159988in}}%
\pgfpathlineto{\pgfqpoint{0.765905in}{1.156119in}}%
\pgfpathlineto{\pgfqpoint{0.767685in}{1.163008in}}%
\pgfpathlineto{\pgfqpoint{0.766224in}{1.174419in}}%
\pgfpathlineto{\pgfqpoint{0.765905in}{1.175219in}}%
\pgfpathlineto{\pgfqpoint{0.760117in}{1.185830in}}%
\pgfpathlineto{\pgfqpoint{0.754173in}{1.195894in}}%
\pgfpathlineto{\pgfqpoint{0.752715in}{1.197241in}}%
\pgfpathlineto{\pgfqpoint{0.745879in}{1.208651in}}%
\pgfpathlineto{\pgfqpoint{0.742440in}{1.214134in}}%
\pgfpathlineto{\pgfqpoint{0.737603in}{1.220062in}}%
\pgfpathlineto{\pgfqpoint{0.731238in}{1.231473in}}%
\pgfpathlineto{\pgfqpoint{0.730708in}{1.232296in}}%
\pgfpathlineto{\pgfqpoint{0.722811in}{1.242884in}}%
\pgfpathlineto{\pgfqpoint{0.718976in}{1.246896in}}%
\pgfpathlineto{\pgfqpoint{0.707244in}{1.248421in}}%
\pgfpathlineto{\pgfqpoint{0.704439in}{1.242884in}}%
\pgfpathlineto{\pgfqpoint{0.703979in}{1.231473in}}%
\pgfpathlineto{\pgfqpoint{0.707242in}{1.220062in}}%
\pgfpathlineto{\pgfqpoint{0.707244in}{1.220058in}}%
\pgfpathlineto{\pgfqpoint{0.714938in}{1.208651in}}%
\pgfpathlineto{\pgfqpoint{0.718976in}{1.202806in}}%
\pgfpathlineto{\pgfqpoint{0.723697in}{1.197241in}}%
\pgfpathlineto{\pgfqpoint{0.730708in}{1.185865in}}%
\pgfpathlineto{\pgfqpoint{0.730739in}{1.185830in}}%
\pgfpathlineto{\pgfqpoint{0.740377in}{1.174419in}}%
\pgfpathlineto{\pgfqpoint{0.742440in}{1.171817in}}%
\pgfpathlineto{\pgfqpoint{0.751725in}{1.163008in}}%
\pgfpathclose%
\pgfusepath{fill}%
\end{pgfscope}%
\begin{pgfscope}%
\pgfpathrectangle{\pgfqpoint{0.211875in}{0.211875in}}{\pgfqpoint{1.313625in}{1.279725in}}%
\pgfusepath{clip}%
\pgfsetbuttcap%
\pgfsetroundjoin%
\definecolor{currentfill}{rgb}{0.962532,0.599594,0.438051}%
\pgfsetfillcolor{currentfill}%
\pgfsetlinewidth{0.000000pt}%
\definecolor{currentstroke}{rgb}{0.000000,0.000000,0.000000}%
\pgfsetstrokecolor{currentstroke}%
\pgfsetdash{}{0pt}%
\pgfpathmoveto{\pgfqpoint{0.988817in}{1.227140in}}%
\pgfpathlineto{\pgfqpoint{1.000549in}{1.228745in}}%
\pgfpathlineto{\pgfqpoint{1.003555in}{1.231473in}}%
\pgfpathlineto{\pgfqpoint{1.006485in}{1.242884in}}%
\pgfpathlineto{\pgfqpoint{1.005221in}{1.254295in}}%
\pgfpathlineto{\pgfqpoint{1.003525in}{1.265705in}}%
\pgfpathlineto{\pgfqpoint{1.000549in}{1.267787in}}%
\pgfpathlineto{\pgfqpoint{0.988817in}{1.271870in}}%
\pgfpathlineto{\pgfqpoint{0.977085in}{1.270984in}}%
\pgfpathlineto{\pgfqpoint{0.965352in}{1.266093in}}%
\pgfpathlineto{\pgfqpoint{0.964469in}{1.265705in}}%
\pgfpathlineto{\pgfqpoint{0.953620in}{1.258311in}}%
\pgfpathlineto{\pgfqpoint{0.947795in}{1.254295in}}%
\pgfpathlineto{\pgfqpoint{0.953620in}{1.248952in}}%
\pgfpathlineto{\pgfqpoint{0.963939in}{1.242884in}}%
\pgfpathlineto{\pgfqpoint{0.965352in}{1.242028in}}%
\pgfpathlineto{\pgfqpoint{0.977085in}{1.233876in}}%
\pgfpathlineto{\pgfqpoint{0.982177in}{1.231473in}}%
\pgfpathclose%
\pgfusepath{fill}%
\end{pgfscope}%
\begin{pgfscope}%
\pgfpathrectangle{\pgfqpoint{0.211875in}{0.211875in}}{\pgfqpoint{1.313625in}{1.279725in}}%
\pgfusepath{clip}%
\pgfsetbuttcap%
\pgfsetroundjoin%
\definecolor{currentfill}{rgb}{0.962532,0.599594,0.438051}%
\pgfsetfillcolor{currentfill}%
\pgfsetlinewidth{0.000000pt}%
\definecolor{currentstroke}{rgb}{0.000000,0.000000,0.000000}%
\pgfsetstrokecolor{currentstroke}%
\pgfsetdash{}{0pt}%
\pgfpathmoveto{\pgfqpoint{1.399444in}{1.390072in}}%
\pgfpathlineto{\pgfqpoint{1.411176in}{1.387485in}}%
\pgfpathlineto{\pgfqpoint{1.422908in}{1.386476in}}%
\pgfpathlineto{\pgfqpoint{1.434641in}{1.386891in}}%
\pgfpathlineto{\pgfqpoint{1.446373in}{1.388780in}}%
\pgfpathlineto{\pgfqpoint{1.446373in}{1.391224in}}%
\pgfpathlineto{\pgfqpoint{1.446373in}{1.402635in}}%
\pgfpathlineto{\pgfqpoint{1.446373in}{1.414045in}}%
\pgfpathlineto{\pgfqpoint{1.434641in}{1.414045in}}%
\pgfpathlineto{\pgfqpoint{1.422908in}{1.414045in}}%
\pgfpathlineto{\pgfqpoint{1.411176in}{1.414045in}}%
\pgfpathlineto{\pgfqpoint{1.399444in}{1.414045in}}%
\pgfpathlineto{\pgfqpoint{1.387712in}{1.414045in}}%
\pgfpathlineto{\pgfqpoint{1.375980in}{1.414045in}}%
\pgfpathlineto{\pgfqpoint{1.369750in}{1.414045in}}%
\pgfpathlineto{\pgfqpoint{1.375980in}{1.406566in}}%
\pgfpathlineto{\pgfqpoint{1.380212in}{1.402635in}}%
\pgfpathlineto{\pgfqpoint{1.387712in}{1.396254in}}%
\pgfpathlineto{\pgfqpoint{1.396875in}{1.391224in}}%
\pgfpathclose%
\pgfusepath{fill}%
\end{pgfscope}%
\begin{pgfscope}%
\pgfpathrectangle{\pgfqpoint{0.211875in}{0.211875in}}{\pgfqpoint{1.313625in}{1.279725in}}%
\pgfusepath{clip}%
\pgfsetbuttcap%
\pgfsetroundjoin%
\definecolor{currentfill}{rgb}{0.965753,0.732351,0.592427}%
\pgfsetfillcolor{currentfill}%
\pgfsetlinewidth{0.000000pt}%
\definecolor{currentstroke}{rgb}{0.000000,0.000000,0.000000}%
\pgfsetstrokecolor{currentstroke}%
\pgfsetdash{}{0pt}%
\pgfpathmoveto{\pgfqpoint{1.317319in}{0.934734in}}%
\pgfpathlineto{\pgfqpoint{1.329051in}{0.933569in}}%
\pgfpathlineto{\pgfqpoint{1.330154in}{0.934793in}}%
\pgfpathlineto{\pgfqpoint{1.340783in}{0.944706in}}%
\pgfpathlineto{\pgfqpoint{1.342540in}{0.946204in}}%
\pgfpathlineto{\pgfqpoint{1.340783in}{0.950846in}}%
\pgfpathlineto{\pgfqpoint{1.337159in}{0.957614in}}%
\pgfpathlineto{\pgfqpoint{1.329051in}{0.962979in}}%
\pgfpathlineto{\pgfqpoint{1.317319in}{0.959375in}}%
\pgfpathlineto{\pgfqpoint{1.313089in}{0.957614in}}%
\pgfpathlineto{\pgfqpoint{1.311974in}{0.946204in}}%
\pgfpathlineto{\pgfqpoint{1.317252in}{0.934793in}}%
\pgfpathclose%
\pgfpathmoveto{\pgfqpoint{1.324146in}{0.946204in}}%
\pgfpathlineto{\pgfqpoint{1.329051in}{0.951212in}}%
\pgfpathlineto{\pgfqpoint{1.331908in}{0.946204in}}%
\pgfpathlineto{\pgfqpoint{1.329051in}{0.943586in}}%
\pgfpathclose%
\pgfusepath{fill}%
\end{pgfscope}%
\begin{pgfscope}%
\pgfpathrectangle{\pgfqpoint{0.211875in}{0.211875in}}{\pgfqpoint{1.313625in}{1.279725in}}%
\pgfusepath{clip}%
\pgfsetbuttcap%
\pgfsetroundjoin%
\definecolor{currentfill}{rgb}{0.965753,0.732351,0.592427}%
\pgfsetfillcolor{currentfill}%
\pgfsetlinewidth{0.000000pt}%
\definecolor{currentstroke}{rgb}{0.000000,0.000000,0.000000}%
\pgfsetstrokecolor{currentstroke}%
\pgfsetdash{}{0pt}%
\pgfpathmoveto{\pgfqpoint{0.894959in}{0.937310in}}%
\pgfpathlineto{\pgfqpoint{0.906691in}{0.938189in}}%
\pgfpathlineto{\pgfqpoint{0.916524in}{0.946204in}}%
\pgfpathlineto{\pgfqpoint{0.914567in}{0.957614in}}%
\pgfpathlineto{\pgfqpoint{0.906691in}{0.962342in}}%
\pgfpathlineto{\pgfqpoint{0.894959in}{0.958718in}}%
\pgfpathlineto{\pgfqpoint{0.893963in}{0.957614in}}%
\pgfpathlineto{\pgfqpoint{0.886491in}{0.946204in}}%
\pgfpathclose%
\pgfpathmoveto{\pgfqpoint{0.897852in}{0.946204in}}%
\pgfpathlineto{\pgfqpoint{0.906691in}{0.951061in}}%
\pgfpathlineto{\pgfqpoint{0.909101in}{0.946204in}}%
\pgfpathlineto{\pgfqpoint{0.906691in}{0.944240in}}%
\pgfpathclose%
\pgfusepath{fill}%
\end{pgfscope}%
\begin{pgfscope}%
\pgfpathrectangle{\pgfqpoint{0.211875in}{0.211875in}}{\pgfqpoint{1.313625in}{1.279725in}}%
\pgfusepath{clip}%
\pgfsetbuttcap%
\pgfsetroundjoin%
\definecolor{currentfill}{rgb}{0.965753,0.732351,0.592427}%
\pgfsetfillcolor{currentfill}%
\pgfsetlinewidth{0.000000pt}%
\definecolor{currentstroke}{rgb}{0.000000,0.000000,0.000000}%
\pgfsetstrokecolor{currentstroke}%
\pgfsetdash{}{0pt}%
\pgfpathmoveto{\pgfqpoint{1.364247in}{1.011428in}}%
\pgfpathlineto{\pgfqpoint{1.375980in}{1.013452in}}%
\pgfpathlineto{\pgfqpoint{1.378811in}{1.014668in}}%
\pgfpathlineto{\pgfqpoint{1.387712in}{1.018427in}}%
\pgfpathlineto{\pgfqpoint{1.392602in}{1.026079in}}%
\pgfpathlineto{\pgfqpoint{1.393825in}{1.037490in}}%
\pgfpathlineto{\pgfqpoint{1.387712in}{1.044484in}}%
\pgfpathlineto{\pgfqpoint{1.375980in}{1.039949in}}%
\pgfpathlineto{\pgfqpoint{1.373733in}{1.037490in}}%
\pgfpathlineto{\pgfqpoint{1.366678in}{1.026079in}}%
\pgfpathlineto{\pgfqpoint{1.364247in}{1.021778in}}%
\pgfpathlineto{\pgfqpoint{1.361060in}{1.014668in}}%
\pgfpathclose%
\pgfpathmoveto{\pgfqpoint{1.375350in}{1.026079in}}%
\pgfpathlineto{\pgfqpoint{1.375980in}{1.027087in}}%
\pgfpathlineto{\pgfqpoint{1.378330in}{1.026079in}}%
\pgfpathlineto{\pgfqpoint{1.375980in}{1.025217in}}%
\pgfpathclose%
\pgfusepath{fill}%
\end{pgfscope}%
\begin{pgfscope}%
\pgfpathrectangle{\pgfqpoint{0.211875in}{0.211875in}}{\pgfqpoint{1.313625in}{1.279725in}}%
\pgfusepath{clip}%
\pgfsetbuttcap%
\pgfsetroundjoin%
\definecolor{currentfill}{rgb}{0.965753,0.732351,0.592427}%
\pgfsetfillcolor{currentfill}%
\pgfsetlinewidth{0.000000pt}%
\definecolor{currentstroke}{rgb}{0.000000,0.000000,0.000000}%
\pgfsetstrokecolor{currentstroke}%
\pgfsetdash{}{0pt}%
\pgfpathmoveto{\pgfqpoint{0.801101in}{1.081124in}}%
\pgfpathlineto{\pgfqpoint{0.804695in}{1.083133in}}%
\pgfpathlineto{\pgfqpoint{0.812834in}{1.090244in}}%
\pgfpathlineto{\pgfqpoint{0.818014in}{1.094544in}}%
\pgfpathlineto{\pgfqpoint{0.812834in}{1.100353in}}%
\pgfpathlineto{\pgfqpoint{0.801101in}{1.105308in}}%
\pgfpathlineto{\pgfqpoint{0.789369in}{1.096077in}}%
\pgfpathlineto{\pgfqpoint{0.789027in}{1.094544in}}%
\pgfpathlineto{\pgfqpoint{0.789369in}{1.093992in}}%
\pgfpathlineto{\pgfqpoint{0.798878in}{1.083133in}}%
\pgfpathclose%
\pgfpathmoveto{\pgfqpoint{0.799671in}{1.094544in}}%
\pgfpathlineto{\pgfqpoint{0.801101in}{1.095770in}}%
\pgfpathlineto{\pgfqpoint{0.803381in}{1.094544in}}%
\pgfpathlineto{\pgfqpoint{0.801101in}{1.092681in}}%
\pgfpathclose%
\pgfusepath{fill}%
\end{pgfscope}%
\begin{pgfscope}%
\pgfpathrectangle{\pgfqpoint{0.211875in}{0.211875in}}{\pgfqpoint{1.313625in}{1.279725in}}%
\pgfusepath{clip}%
\pgfsetbuttcap%
\pgfsetroundjoin%
\definecolor{currentfill}{rgb}{0.965753,0.732351,0.592427}%
\pgfsetfillcolor{currentfill}%
\pgfsetlinewidth{0.000000pt}%
\definecolor{currentstroke}{rgb}{0.000000,0.000000,0.000000}%
\pgfsetstrokecolor{currentstroke}%
\pgfsetdash{}{0pt}%
\pgfpathmoveto{\pgfqpoint{0.848030in}{1.081099in}}%
\pgfpathlineto{\pgfqpoint{0.858212in}{1.083133in}}%
\pgfpathlineto{\pgfqpoint{0.859762in}{1.083561in}}%
\pgfpathlineto{\pgfqpoint{0.871495in}{1.085963in}}%
\pgfpathlineto{\pgfqpoint{0.883227in}{1.088916in}}%
\pgfpathlineto{\pgfqpoint{0.894959in}{1.092649in}}%
\pgfpathlineto{\pgfqpoint{0.898941in}{1.094544in}}%
\pgfpathlineto{\pgfqpoint{0.894959in}{1.095271in}}%
\pgfpathlineto{\pgfqpoint{0.883227in}{1.097858in}}%
\pgfpathlineto{\pgfqpoint{0.871495in}{1.103085in}}%
\pgfpathlineto{\pgfqpoint{0.867837in}{1.105954in}}%
\pgfpathlineto{\pgfqpoint{0.859762in}{1.107549in}}%
\pgfpathlineto{\pgfqpoint{0.848030in}{1.109498in}}%
\pgfpathlineto{\pgfqpoint{0.836298in}{1.110221in}}%
\pgfpathlineto{\pgfqpoint{0.828527in}{1.105954in}}%
\pgfpathlineto{\pgfqpoint{0.831148in}{1.094544in}}%
\pgfpathlineto{\pgfqpoint{0.836298in}{1.090460in}}%
\pgfpathlineto{\pgfqpoint{0.845752in}{1.083133in}}%
\pgfpathclose%
\pgfpathmoveto{\pgfqpoint{0.835615in}{1.105954in}}%
\pgfpathlineto{\pgfqpoint{0.836298in}{1.106329in}}%
\pgfpathlineto{\pgfqpoint{0.845953in}{1.105954in}}%
\pgfpathlineto{\pgfqpoint{0.836298in}{1.104222in}}%
\pgfpathclose%
\pgfusepath{fill}%
\end{pgfscope}%
\begin{pgfscope}%
\pgfpathrectangle{\pgfqpoint{0.211875in}{0.211875in}}{\pgfqpoint{1.313625in}{1.279725in}}%
\pgfusepath{clip}%
\pgfsetbuttcap%
\pgfsetroundjoin%
\definecolor{currentfill}{rgb}{0.965753,0.732351,0.592427}%
\pgfsetfillcolor{currentfill}%
\pgfsetlinewidth{0.000000pt}%
\definecolor{currentstroke}{rgb}{0.000000,0.000000,0.000000}%
\pgfsetstrokecolor{currentstroke}%
\pgfsetdash{}{0pt}%
\pgfpathmoveto{\pgfqpoint{1.106139in}{1.081725in}}%
\pgfpathlineto{\pgfqpoint{1.117871in}{1.074252in}}%
\pgfpathlineto{\pgfqpoint{1.126374in}{1.083133in}}%
\pgfpathlineto{\pgfqpoint{1.126778in}{1.094544in}}%
\pgfpathlineto{\pgfqpoint{1.124846in}{1.105954in}}%
\pgfpathlineto{\pgfqpoint{1.117871in}{1.108279in}}%
\pgfpathlineto{\pgfqpoint{1.106139in}{1.106968in}}%
\pgfpathlineto{\pgfqpoint{1.104790in}{1.105954in}}%
\pgfpathlineto{\pgfqpoint{1.101259in}{1.094544in}}%
\pgfpathlineto{\pgfqpoint{1.105120in}{1.083133in}}%
\pgfpathclose%
\pgfpathmoveto{\pgfqpoint{1.117416in}{1.083133in}}%
\pgfpathlineto{\pgfqpoint{1.114270in}{1.094544in}}%
\pgfpathlineto{\pgfqpoint{1.117871in}{1.096599in}}%
\pgfpathlineto{\pgfqpoint{1.118835in}{1.094544in}}%
\pgfpathlineto{\pgfqpoint{1.118172in}{1.083133in}}%
\pgfpathlineto{\pgfqpoint{1.117871in}{1.082819in}}%
\pgfpathclose%
\pgfusepath{fill}%
\end{pgfscope}%
\begin{pgfscope}%
\pgfpathrectangle{\pgfqpoint{0.211875in}{0.211875in}}{\pgfqpoint{1.313625in}{1.279725in}}%
\pgfusepath{clip}%
\pgfsetbuttcap%
\pgfsetroundjoin%
\definecolor{currentfill}{rgb}{0.965753,0.732351,0.592427}%
\pgfsetfillcolor{currentfill}%
\pgfsetlinewidth{0.000000pt}%
\definecolor{currentstroke}{rgb}{0.000000,0.000000,0.000000}%
\pgfsetstrokecolor{currentstroke}%
\pgfsetdash{}{0pt}%
\pgfpathmoveto{\pgfqpoint{1.422908in}{1.082351in}}%
\pgfpathlineto{\pgfqpoint{1.423851in}{1.083133in}}%
\pgfpathlineto{\pgfqpoint{1.434641in}{1.090269in}}%
\pgfpathlineto{\pgfqpoint{1.439618in}{1.094544in}}%
\pgfpathlineto{\pgfqpoint{1.446373in}{1.099653in}}%
\pgfpathlineto{\pgfqpoint{1.446373in}{1.105099in}}%
\pgfpathlineto{\pgfqpoint{1.444280in}{1.105954in}}%
\pgfpathlineto{\pgfqpoint{1.446373in}{1.106589in}}%
\pgfpathlineto{\pgfqpoint{1.446373in}{1.110627in}}%
\pgfpathlineto{\pgfqpoint{1.434641in}{1.107185in}}%
\pgfpathlineto{\pgfqpoint{1.432625in}{1.105954in}}%
\pgfpathlineto{\pgfqpoint{1.422908in}{1.095246in}}%
\pgfpathlineto{\pgfqpoint{1.422500in}{1.094544in}}%
\pgfpathlineto{\pgfqpoint{1.422095in}{1.083133in}}%
\pgfpathclose%
\pgfusepath{fill}%
\end{pgfscope}%
\begin{pgfscope}%
\pgfpathrectangle{\pgfqpoint{0.211875in}{0.211875in}}{\pgfqpoint{1.313625in}{1.279725in}}%
\pgfusepath{clip}%
\pgfsetbuttcap%
\pgfsetroundjoin%
\definecolor{currentfill}{rgb}{0.974412,0.862387,0.780156}%
\pgfsetfillcolor{currentfill}%
\pgfsetlinewidth{0.000000pt}%
\definecolor{currentstroke}{rgb}{0.000000,0.000000,0.000000}%
\pgfsetstrokecolor{currentstroke}%
\pgfsetdash{}{0pt}%
\pgfpathmoveto{\pgfqpoint{0.906691in}{0.944240in}}%
\pgfpathlineto{\pgfqpoint{0.909101in}{0.946204in}}%
\pgfpathlineto{\pgfqpoint{0.906691in}{0.951061in}}%
\pgfpathlineto{\pgfqpoint{0.897852in}{0.946204in}}%
\pgfpathclose%
\pgfusepath{fill}%
\end{pgfscope}%
\begin{pgfscope}%
\pgfpathrectangle{\pgfqpoint{0.211875in}{0.211875in}}{\pgfqpoint{1.313625in}{1.279725in}}%
\pgfusepath{clip}%
\pgfsetbuttcap%
\pgfsetroundjoin%
\definecolor{currentfill}{rgb}{0.974412,0.862387,0.780156}%
\pgfsetfillcolor{currentfill}%
\pgfsetlinewidth{0.000000pt}%
\definecolor{currentstroke}{rgb}{0.000000,0.000000,0.000000}%
\pgfsetstrokecolor{currentstroke}%
\pgfsetdash{}{0pt}%
\pgfpathmoveto{\pgfqpoint{1.329051in}{0.943586in}}%
\pgfpathlineto{\pgfqpoint{1.331908in}{0.946204in}}%
\pgfpathlineto{\pgfqpoint{1.329051in}{0.951212in}}%
\pgfpathlineto{\pgfqpoint{1.324146in}{0.946204in}}%
\pgfpathclose%
\pgfusepath{fill}%
\end{pgfscope}%
\begin{pgfscope}%
\pgfpathrectangle{\pgfqpoint{0.211875in}{0.211875in}}{\pgfqpoint{1.313625in}{1.279725in}}%
\pgfusepath{clip}%
\pgfsetbuttcap%
\pgfsetroundjoin%
\definecolor{currentfill}{rgb}{0.974412,0.862387,0.780156}%
\pgfsetfillcolor{currentfill}%
\pgfsetlinewidth{0.000000pt}%
\definecolor{currentstroke}{rgb}{0.000000,0.000000,0.000000}%
\pgfsetstrokecolor{currentstroke}%
\pgfsetdash{}{0pt}%
\pgfpathmoveto{\pgfqpoint{1.375980in}{1.025217in}}%
\pgfpathlineto{\pgfqpoint{1.378330in}{1.026079in}}%
\pgfpathlineto{\pgfqpoint{1.375980in}{1.027087in}}%
\pgfpathlineto{\pgfqpoint{1.375350in}{1.026079in}}%
\pgfpathclose%
\pgfusepath{fill}%
\end{pgfscope}%
\begin{pgfscope}%
\pgfpathrectangle{\pgfqpoint{0.211875in}{0.211875in}}{\pgfqpoint{1.313625in}{1.279725in}}%
\pgfusepath{clip}%
\pgfsetbuttcap%
\pgfsetroundjoin%
\definecolor{currentfill}{rgb}{0.974412,0.862387,0.780156}%
\pgfsetfillcolor{currentfill}%
\pgfsetlinewidth{0.000000pt}%
\definecolor{currentstroke}{rgb}{0.000000,0.000000,0.000000}%
\pgfsetstrokecolor{currentstroke}%
\pgfsetdash{}{0pt}%
\pgfpathmoveto{\pgfqpoint{1.117871in}{1.082819in}}%
\pgfpathlineto{\pgfqpoint{1.118172in}{1.083133in}}%
\pgfpathlineto{\pgfqpoint{1.118835in}{1.094544in}}%
\pgfpathlineto{\pgfqpoint{1.117871in}{1.096599in}}%
\pgfpathlineto{\pgfqpoint{1.114270in}{1.094544in}}%
\pgfpathlineto{\pgfqpoint{1.117416in}{1.083133in}}%
\pgfpathclose%
\pgfusepath{fill}%
\end{pgfscope}%
\begin{pgfscope}%
\pgfpathrectangle{\pgfqpoint{0.211875in}{0.211875in}}{\pgfqpoint{1.313625in}{1.279725in}}%
\pgfusepath{clip}%
\pgfsetbuttcap%
\pgfsetroundjoin%
\definecolor{currentfill}{rgb}{0.974412,0.862387,0.780156}%
\pgfsetfillcolor{currentfill}%
\pgfsetlinewidth{0.000000pt}%
\definecolor{currentstroke}{rgb}{0.000000,0.000000,0.000000}%
\pgfsetstrokecolor{currentstroke}%
\pgfsetdash{}{0pt}%
\pgfpathmoveto{\pgfqpoint{0.801101in}{1.092681in}}%
\pgfpathlineto{\pgfqpoint{0.803381in}{1.094544in}}%
\pgfpathlineto{\pgfqpoint{0.801101in}{1.095770in}}%
\pgfpathlineto{\pgfqpoint{0.799671in}{1.094544in}}%
\pgfpathclose%
\pgfusepath{fill}%
\end{pgfscope}%
\begin{pgfscope}%
\pgfpathrectangle{\pgfqpoint{0.211875in}{0.211875in}}{\pgfqpoint{1.313625in}{1.279725in}}%
\pgfusepath{clip}%
\pgfsetbuttcap%
\pgfsetroundjoin%
\definecolor{currentfill}{rgb}{0.974412,0.862387,0.780156}%
\pgfsetfillcolor{currentfill}%
\pgfsetlinewidth{0.000000pt}%
\definecolor{currentstroke}{rgb}{0.000000,0.000000,0.000000}%
\pgfsetstrokecolor{currentstroke}%
\pgfsetdash{}{0pt}%
\pgfpathmoveto{\pgfqpoint{0.836298in}{1.104222in}}%
\pgfpathlineto{\pgfqpoint{0.845953in}{1.105954in}}%
\pgfpathlineto{\pgfqpoint{0.836298in}{1.106329in}}%
\pgfpathlineto{\pgfqpoint{0.835615in}{1.105954in}}%
\pgfpathclose%
\pgfusepath{fill}%
\end{pgfscope}%
\begin{pgfscope}%
\pgfpathrectangle{\pgfqpoint{0.211875in}{0.211875in}}{\pgfqpoint{1.313625in}{1.279725in}}%
\pgfusepath{clip}%
\pgfsetbuttcap%
\pgfsetroundjoin%
\definecolor{currentfill}{rgb}{0.974412,0.862387,0.780156}%
\pgfsetfillcolor{currentfill}%
\pgfsetlinewidth{0.000000pt}%
\definecolor{currentstroke}{rgb}{0.000000,0.000000,0.000000}%
\pgfsetstrokecolor{currentstroke}%
\pgfsetdash{}{0pt}%
\pgfpathmoveto{\pgfqpoint{1.446373in}{1.105099in}}%
\pgfpathlineto{\pgfqpoint{1.446373in}{1.105954in}}%
\pgfpathlineto{\pgfqpoint{1.446373in}{1.106589in}}%
\pgfpathlineto{\pgfqpoint{1.444280in}{1.105954in}}%
\pgfpathclose%
\pgfusepath{fill}%
\end{pgfscope}%
\begin{pgfscope}%
\pgfpathrectangle{\pgfqpoint{0.211875in}{0.211875in}}{\pgfqpoint{1.313625in}{1.279725in}}%
\pgfusepath{clip}%
\pgfsetbuttcap%
\pgfsetroundjoin%
\definecolor{currentfill}{rgb}{0.121569,0.466667,0.705882}%
\pgfsetfillcolor{currentfill}%
\pgfsetlinewidth{1.003750pt}%
\definecolor{currentstroke}{rgb}{0.121569,0.466667,0.705882}%
\pgfsetstrokecolor{currentstroke}%
\pgfsetdash{}{0pt}%
\pgfpathmoveto{\pgfqpoint{0.595496in}{1.128141in}}%
\pgfpathcurveto{\pgfqpoint{0.601320in}{1.128141in}}{\pgfqpoint{0.606906in}{1.130455in}}{\pgfqpoint{0.611024in}{1.134573in}}%
\pgfpathcurveto{\pgfqpoint{0.615142in}{1.138691in}}{\pgfqpoint{0.617456in}{1.144277in}}{\pgfqpoint{0.617456in}{1.150101in}}%
\pgfpathcurveto{\pgfqpoint{0.617456in}{1.155925in}}{\pgfqpoint{0.615142in}{1.161511in}}{\pgfqpoint{0.611024in}{1.165629in}}%
\pgfpathcurveto{\pgfqpoint{0.606906in}{1.169748in}}{\pgfqpoint{0.601320in}{1.172061in}}{\pgfqpoint{0.595496in}{1.172061in}}%
\pgfpathcurveto{\pgfqpoint{0.589672in}{1.172061in}}{\pgfqpoint{0.584086in}{1.169748in}}{\pgfqpoint{0.579968in}{1.165629in}}%
\pgfpathcurveto{\pgfqpoint{0.575850in}{1.161511in}}{\pgfqpoint{0.573536in}{1.155925in}}{\pgfqpoint{0.573536in}{1.150101in}}%
\pgfpathcurveto{\pgfqpoint{0.573536in}{1.144277in}}{\pgfqpoint{0.575850in}{1.138691in}}{\pgfqpoint{0.579968in}{1.134573in}}%
\pgfpathcurveto{\pgfqpoint{0.584086in}{1.130455in}}{\pgfqpoint{0.589672in}{1.128141in}}{\pgfqpoint{0.595496in}{1.128141in}}%
\pgfpathclose%
\pgfusepath{stroke,fill}%
\end{pgfscope}%
\begin{pgfscope}%
\pgfpathrectangle{\pgfqpoint{0.211875in}{0.211875in}}{\pgfqpoint{1.313625in}{1.279725in}}%
\pgfusepath{clip}%
\pgfsetbuttcap%
\pgfsetroundjoin%
\definecolor{currentfill}{rgb}{0.121569,0.466667,0.705882}%
\pgfsetfillcolor{currentfill}%
\pgfsetlinewidth{1.003750pt}%
\definecolor{currentstroke}{rgb}{0.121569,0.466667,0.705882}%
\pgfsetstrokecolor{currentstroke}%
\pgfsetdash{}{0pt}%
\pgfpathmoveto{\pgfqpoint{0.610972in}{0.284645in}}%
\pgfpathcurveto{\pgfqpoint{0.616796in}{0.284645in}}{\pgfqpoint{0.622382in}{0.286959in}}{\pgfqpoint{0.626500in}{0.291077in}}%
\pgfpathcurveto{\pgfqpoint{0.630618in}{0.295195in}}{\pgfqpoint{0.632932in}{0.300781in}}{\pgfqpoint{0.632932in}{0.306605in}}%
\pgfpathcurveto{\pgfqpoint{0.632932in}{0.312429in}}{\pgfqpoint{0.630618in}{0.318015in}}{\pgfqpoint{0.626500in}{0.322133in}}%
\pgfpathcurveto{\pgfqpoint{0.622382in}{0.326252in}}{\pgfqpoint{0.616796in}{0.328565in}}{\pgfqpoint{0.610972in}{0.328565in}}%
\pgfpathcurveto{\pgfqpoint{0.605148in}{0.328565in}}{\pgfqpoint{0.599562in}{0.326252in}}{\pgfqpoint{0.595444in}{0.322133in}}%
\pgfpathcurveto{\pgfqpoint{0.591325in}{0.318015in}}{\pgfqpoint{0.589011in}{0.312429in}}{\pgfqpoint{0.589011in}{0.306605in}}%
\pgfpathcurveto{\pgfqpoint{0.589011in}{0.300781in}}{\pgfqpoint{0.591325in}{0.295195in}}{\pgfqpoint{0.595444in}{0.291077in}}%
\pgfpathcurveto{\pgfqpoint{0.599562in}{0.286959in}}{\pgfqpoint{0.605148in}{0.284645in}}{\pgfqpoint{0.610972in}{0.284645in}}%
\pgfpathclose%
\pgfusepath{stroke,fill}%
\end{pgfscope}%
\begin{pgfscope}%
\pgfpathrectangle{\pgfqpoint{0.211875in}{0.211875in}}{\pgfqpoint{1.313625in}{1.279725in}}%
\pgfusepath{clip}%
\pgfsetbuttcap%
\pgfsetroundjoin%
\definecolor{currentfill}{rgb}{0.121569,0.466667,0.705882}%
\pgfsetfillcolor{currentfill}%
\pgfsetlinewidth{1.003750pt}%
\definecolor{currentstroke}{rgb}{0.121569,0.466667,0.705882}%
\pgfsetstrokecolor{currentstroke}%
\pgfsetdash{}{0pt}%
\pgfpathmoveto{\pgfqpoint{1.202030in}{0.687877in}}%
\pgfpathcurveto{\pgfqpoint{1.207854in}{0.687877in}}{\pgfqpoint{1.213440in}{0.690191in}}{\pgfqpoint{1.217558in}{0.694309in}}%
\pgfpathcurveto{\pgfqpoint{1.221677in}{0.698427in}}{\pgfqpoint{1.223990in}{0.704013in}}{\pgfqpoint{1.223990in}{0.709837in}}%
\pgfpathcurveto{\pgfqpoint{1.223990in}{0.715661in}}{\pgfqpoint{1.221677in}{0.721247in}}{\pgfqpoint{1.217558in}{0.725366in}}%
\pgfpathcurveto{\pgfqpoint{1.213440in}{0.729484in}}{\pgfqpoint{1.207854in}{0.731798in}}{\pgfqpoint{1.202030in}{0.731798in}}%
\pgfpathcurveto{\pgfqpoint{1.196206in}{0.731798in}}{\pgfqpoint{1.190620in}{0.729484in}}{\pgfqpoint{1.186502in}{0.725366in}}%
\pgfpathcurveto{\pgfqpoint{1.182384in}{0.721247in}}{\pgfqpoint{1.180070in}{0.715661in}}{\pgfqpoint{1.180070in}{0.709837in}}%
\pgfpathcurveto{\pgfqpoint{1.180070in}{0.704013in}}{\pgfqpoint{1.182384in}{0.698427in}}{\pgfqpoint{1.186502in}{0.694309in}}%
\pgfpathcurveto{\pgfqpoint{1.190620in}{0.690191in}}{\pgfqpoint{1.196206in}{0.687877in}}{\pgfqpoint{1.202030in}{0.687877in}}%
\pgfpathclose%
\pgfusepath{stroke,fill}%
\end{pgfscope}%
\begin{pgfscope}%
\pgfpathrectangle{\pgfqpoint{0.211875in}{0.211875in}}{\pgfqpoint{1.313625in}{1.279725in}}%
\pgfusepath{clip}%
\pgfsetbuttcap%
\pgfsetroundjoin%
\definecolor{currentfill}{rgb}{0.121569,0.466667,0.705882}%
\pgfsetfillcolor{currentfill}%
\pgfsetlinewidth{1.003750pt}%
\definecolor{currentstroke}{rgb}{0.121569,0.466667,0.705882}%
\pgfsetstrokecolor{currentstroke}%
\pgfsetdash{}{0pt}%
\pgfpathmoveto{\pgfqpoint{0.999507in}{1.309088in}}%
\pgfpathcurveto{\pgfqpoint{1.005331in}{1.309088in}}{\pgfqpoint{1.010917in}{1.311402in}}{\pgfqpoint{1.015035in}{1.315520in}}%
\pgfpathcurveto{\pgfqpoint{1.019154in}{1.319639in}}{\pgfqpoint{1.021467in}{1.325225in}}{\pgfqpoint{1.021467in}{1.331049in}}%
\pgfpathcurveto{\pgfqpoint{1.021467in}{1.336873in}}{\pgfqpoint{1.019154in}{1.342459in}}{\pgfqpoint{1.015035in}{1.346577in}}%
\pgfpathcurveto{\pgfqpoint{1.010917in}{1.350695in}}{\pgfqpoint{1.005331in}{1.353009in}}{\pgfqpoint{0.999507in}{1.353009in}}%
\pgfpathcurveto{\pgfqpoint{0.993683in}{1.353009in}}{\pgfqpoint{0.988097in}{1.350695in}}{\pgfqpoint{0.983979in}{1.346577in}}%
\pgfpathcurveto{\pgfqpoint{0.979861in}{1.342459in}}{\pgfqpoint{0.977547in}{1.336873in}}{\pgfqpoint{0.977547in}{1.331049in}}%
\pgfpathcurveto{\pgfqpoint{0.977547in}{1.325225in}}{\pgfqpoint{0.979861in}{1.319639in}}{\pgfqpoint{0.983979in}{1.315520in}}%
\pgfpathcurveto{\pgfqpoint{0.988097in}{1.311402in}}{\pgfqpoint{0.993683in}{1.309088in}}{\pgfqpoint{0.999507in}{1.309088in}}%
\pgfpathclose%
\pgfusepath{stroke,fill}%
\end{pgfscope}%
\begin{pgfscope}%
\pgfpathrectangle{\pgfqpoint{0.211875in}{0.211875in}}{\pgfqpoint{1.313625in}{1.279725in}}%
\pgfusepath{clip}%
\pgfsetbuttcap%
\pgfsetroundjoin%
\definecolor{currentfill}{rgb}{0.121569,0.466667,0.705882}%
\pgfsetfillcolor{currentfill}%
\pgfsetlinewidth{1.003750pt}%
\definecolor{currentstroke}{rgb}{0.121569,0.466667,0.705882}%
\pgfsetstrokecolor{currentstroke}%
\pgfsetdash{}{0pt}%
\pgfpathmoveto{\pgfqpoint{0.320068in}{0.582274in}}%
\pgfpathcurveto{\pgfqpoint{0.325892in}{0.582274in}}{\pgfqpoint{0.331478in}{0.584588in}}{\pgfqpoint{0.335596in}{0.588706in}}%
\pgfpathcurveto{\pgfqpoint{0.339714in}{0.592824in}}{\pgfqpoint{0.342028in}{0.598410in}}{\pgfqpoint{0.342028in}{0.604234in}}%
\pgfpathcurveto{\pgfqpoint{0.342028in}{0.610058in}}{\pgfqpoint{0.339714in}{0.615644in}}{\pgfqpoint{0.335596in}{0.619762in}}%
\pgfpathcurveto{\pgfqpoint{0.331478in}{0.623881in}}{\pgfqpoint{0.325892in}{0.626194in}}{\pgfqpoint{0.320068in}{0.626194in}}%
\pgfpathcurveto{\pgfqpoint{0.314244in}{0.626194in}}{\pgfqpoint{0.308658in}{0.623881in}}{\pgfqpoint{0.304540in}{0.619762in}}%
\pgfpathcurveto{\pgfqpoint{0.300422in}{0.615644in}}{\pgfqpoint{0.298108in}{0.610058in}}{\pgfqpoint{0.298108in}{0.604234in}}%
\pgfpathcurveto{\pgfqpoint{0.298108in}{0.598410in}}{\pgfqpoint{0.300422in}{0.592824in}}{\pgfqpoint{0.304540in}{0.588706in}}%
\pgfpathcurveto{\pgfqpoint{0.308658in}{0.584588in}}{\pgfqpoint{0.314244in}{0.582274in}}{\pgfqpoint{0.320068in}{0.582274in}}%
\pgfpathclose%
\pgfusepath{stroke,fill}%
\end{pgfscope}%
\begin{pgfscope}%
\pgfpathrectangle{\pgfqpoint{0.211875in}{0.211875in}}{\pgfqpoint{1.313625in}{1.279725in}}%
\pgfusepath{clip}%
\pgfsetbuttcap%
\pgfsetroundjoin%
\definecolor{currentfill}{rgb}{0.121569,0.466667,0.705882}%
\pgfsetfillcolor{currentfill}%
\pgfsetlinewidth{1.003750pt}%
\definecolor{currentstroke}{rgb}{0.121569,0.466667,0.705882}%
\pgfsetstrokecolor{currentstroke}%
\pgfsetdash{}{0pt}%
\pgfpathmoveto{\pgfqpoint{0.853919in}{0.622196in}}%
\pgfpathcurveto{\pgfqpoint{0.859743in}{0.622196in}}{\pgfqpoint{0.865329in}{0.624510in}}{\pgfqpoint{0.869447in}{0.628628in}}%
\pgfpathcurveto{\pgfqpoint{0.873565in}{0.632746in}}{\pgfqpoint{0.875879in}{0.638333in}}{\pgfqpoint{0.875879in}{0.644156in}}%
\pgfpathcurveto{\pgfqpoint{0.875879in}{0.649980in}}{\pgfqpoint{0.873565in}{0.655567in}}{\pgfqpoint{0.869447in}{0.659685in}}%
\pgfpathcurveto{\pgfqpoint{0.865329in}{0.663803in}}{\pgfqpoint{0.859743in}{0.666117in}}{\pgfqpoint{0.853919in}{0.666117in}}%
\pgfpathcurveto{\pgfqpoint{0.848095in}{0.666117in}}{\pgfqpoint{0.842509in}{0.663803in}}{\pgfqpoint{0.838391in}{0.659685in}}%
\pgfpathcurveto{\pgfqpoint{0.834272in}{0.655567in}}{\pgfqpoint{0.831959in}{0.649980in}}{\pgfqpoint{0.831959in}{0.644156in}}%
\pgfpathcurveto{\pgfqpoint{0.831959in}{0.638333in}}{\pgfqpoint{0.834272in}{0.632746in}}{\pgfqpoint{0.838391in}{0.628628in}}%
\pgfpathcurveto{\pgfqpoint{0.842509in}{0.624510in}}{\pgfqpoint{0.848095in}{0.622196in}}{\pgfqpoint{0.853919in}{0.622196in}}%
\pgfpathclose%
\pgfusepath{stroke,fill}%
\end{pgfscope}%
\begin{pgfscope}%
\pgfpathrectangle{\pgfqpoint{0.211875in}{0.211875in}}{\pgfqpoint{1.313625in}{1.279725in}}%
\pgfusepath{clip}%
\pgfsetbuttcap%
\pgfsetroundjoin%
\definecolor{currentfill}{rgb}{0.121569,0.466667,0.705882}%
\pgfsetfillcolor{currentfill}%
\pgfsetlinewidth{1.003750pt}%
\definecolor{currentstroke}{rgb}{0.121569,0.466667,0.705882}%
\pgfsetstrokecolor{currentstroke}%
\pgfsetdash{}{0pt}%
\pgfpathmoveto{\pgfqpoint{1.024141in}{1.321604in}}%
\pgfpathcurveto{\pgfqpoint{1.029965in}{1.321604in}}{\pgfqpoint{1.035551in}{1.323918in}}{\pgfqpoint{1.039669in}{1.328036in}}%
\pgfpathcurveto{\pgfqpoint{1.043787in}{1.332154in}}{\pgfqpoint{1.046101in}{1.337740in}}{\pgfqpoint{1.046101in}{1.343564in}}%
\pgfpathcurveto{\pgfqpoint{1.046101in}{1.349388in}}{\pgfqpoint{1.043787in}{1.354974in}}{\pgfqpoint{1.039669in}{1.359092in}}%
\pgfpathcurveto{\pgfqpoint{1.035551in}{1.363210in}}{\pgfqpoint{1.029965in}{1.365524in}}{\pgfqpoint{1.024141in}{1.365524in}}%
\pgfpathcurveto{\pgfqpoint{1.018317in}{1.365524in}}{\pgfqpoint{1.012731in}{1.363210in}}{\pgfqpoint{1.008613in}{1.359092in}}%
\pgfpathcurveto{\pgfqpoint{1.004494in}{1.354974in}}{\pgfqpoint{1.002181in}{1.349388in}}{\pgfqpoint{1.002181in}{1.343564in}}%
\pgfpathcurveto{\pgfqpoint{1.002181in}{1.337740in}}{\pgfqpoint{1.004494in}{1.332154in}}{\pgfqpoint{1.008613in}{1.328036in}}%
\pgfpathcurveto{\pgfqpoint{1.012731in}{1.323918in}}{\pgfqpoint{1.018317in}{1.321604in}}{\pgfqpoint{1.024141in}{1.321604in}}%
\pgfpathclose%
\pgfusepath{stroke,fill}%
\end{pgfscope}%
\begin{pgfscope}%
\pgfpathrectangle{\pgfqpoint{0.211875in}{0.211875in}}{\pgfqpoint{1.313625in}{1.279725in}}%
\pgfusepath{clip}%
\pgfsetbuttcap%
\pgfsetroundjoin%
\definecolor{currentfill}{rgb}{0.121569,0.466667,0.705882}%
\pgfsetfillcolor{currentfill}%
\pgfsetlinewidth{1.003750pt}%
\definecolor{currentstroke}{rgb}{0.121569,0.466667,0.705882}%
\pgfsetstrokecolor{currentstroke}%
\pgfsetdash{}{0pt}%
\pgfpathmoveto{\pgfqpoint{1.355248in}{1.225447in}}%
\pgfpathcurveto{\pgfqpoint{1.361072in}{1.225447in}}{\pgfqpoint{1.366658in}{1.227760in}}{\pgfqpoint{1.370776in}{1.231879in}}%
\pgfpathcurveto{\pgfqpoint{1.374894in}{1.235997in}}{\pgfqpoint{1.377208in}{1.241583in}}{\pgfqpoint{1.377208in}{1.247407in}}%
\pgfpathcurveto{\pgfqpoint{1.377208in}{1.253231in}}{\pgfqpoint{1.374894in}{1.258817in}}{\pgfqpoint{1.370776in}{1.262935in}}%
\pgfpathcurveto{\pgfqpoint{1.366658in}{1.267053in}}{\pgfqpoint{1.361072in}{1.269367in}}{\pgfqpoint{1.355248in}{1.269367in}}%
\pgfpathcurveto{\pgfqpoint{1.349424in}{1.269367in}}{\pgfqpoint{1.343838in}{1.267053in}}{\pgfqpoint{1.339720in}{1.262935in}}%
\pgfpathcurveto{\pgfqpoint{1.335602in}{1.258817in}}{\pgfqpoint{1.333288in}{1.253231in}}{\pgfqpoint{1.333288in}{1.247407in}}%
\pgfpathcurveto{\pgfqpoint{1.333288in}{1.241583in}}{\pgfqpoint{1.335602in}{1.235997in}}{\pgfqpoint{1.339720in}{1.231879in}}%
\pgfpathcurveto{\pgfqpoint{1.343838in}{1.227760in}}{\pgfqpoint{1.349424in}{1.225447in}}{\pgfqpoint{1.355248in}{1.225447in}}%
\pgfpathclose%
\pgfusepath{stroke,fill}%
\end{pgfscope}%
\begin{pgfscope}%
\pgfpathrectangle{\pgfqpoint{0.211875in}{0.211875in}}{\pgfqpoint{1.313625in}{1.279725in}}%
\pgfusepath{clip}%
\pgfsetbuttcap%
\pgfsetroundjoin%
\definecolor{currentfill}{rgb}{0.121569,0.466667,0.705882}%
\pgfsetfillcolor{currentfill}%
\pgfsetlinewidth{1.003750pt}%
\definecolor{currentstroke}{rgb}{0.121569,0.466667,0.705882}%
\pgfsetstrokecolor{currentstroke}%
\pgfsetdash{}{0pt}%
\pgfpathmoveto{\pgfqpoint{1.085389in}{1.004004in}}%
\pgfpathcurveto{\pgfqpoint{1.091213in}{1.004004in}}{\pgfqpoint{1.096799in}{1.006318in}}{\pgfqpoint{1.100917in}{1.010436in}}%
\pgfpathcurveto{\pgfqpoint{1.105035in}{1.014554in}}{\pgfqpoint{1.107349in}{1.020140in}}{\pgfqpoint{1.107349in}{1.025964in}}%
\pgfpathcurveto{\pgfqpoint{1.107349in}{1.031788in}}{\pgfqpoint{1.105035in}{1.037374in}}{\pgfqpoint{1.100917in}{1.041492in}}%
\pgfpathcurveto{\pgfqpoint{1.096799in}{1.045610in}}{\pgfqpoint{1.091213in}{1.047924in}}{\pgfqpoint{1.085389in}{1.047924in}}%
\pgfpathcurveto{\pgfqpoint{1.079565in}{1.047924in}}{\pgfqpoint{1.073979in}{1.045610in}}{\pgfqpoint{1.069860in}{1.041492in}}%
\pgfpathcurveto{\pgfqpoint{1.065742in}{1.037374in}}{\pgfqpoint{1.063428in}{1.031788in}}{\pgfqpoint{1.063428in}{1.025964in}}%
\pgfpathcurveto{\pgfqpoint{1.063428in}{1.020140in}}{\pgfqpoint{1.065742in}{1.014554in}}{\pgfqpoint{1.069860in}{1.010436in}}%
\pgfpathcurveto{\pgfqpoint{1.073979in}{1.006318in}}{\pgfqpoint{1.079565in}{1.004004in}}{\pgfqpoint{1.085389in}{1.004004in}}%
\pgfpathclose%
\pgfusepath{stroke,fill}%
\end{pgfscope}%
\begin{pgfscope}%
\pgfpathrectangle{\pgfqpoint{0.211875in}{0.211875in}}{\pgfqpoint{1.313625in}{1.279725in}}%
\pgfusepath{clip}%
\pgfsetbuttcap%
\pgfsetroundjoin%
\definecolor{currentfill}{rgb}{0.121569,0.466667,0.705882}%
\pgfsetfillcolor{currentfill}%
\pgfsetlinewidth{1.003750pt}%
\definecolor{currentstroke}{rgb}{0.121569,0.466667,0.705882}%
\pgfsetstrokecolor{currentstroke}%
\pgfsetdash{}{0pt}%
\pgfpathmoveto{\pgfqpoint{1.382821in}{0.810418in}}%
\pgfpathcurveto{\pgfqpoint{1.388645in}{0.810418in}}{\pgfqpoint{1.394231in}{0.812731in}}{\pgfqpoint{1.398349in}{0.816850in}}%
\pgfpathcurveto{\pgfqpoint{1.402467in}{0.820968in}}{\pgfqpoint{1.404781in}{0.826554in}}{\pgfqpoint{1.404781in}{0.832378in}}%
\pgfpathcurveto{\pgfqpoint{1.404781in}{0.838202in}}{\pgfqpoint{1.402467in}{0.843788in}}{\pgfqpoint{1.398349in}{0.847906in}}%
\pgfpathcurveto{\pgfqpoint{1.394231in}{0.852024in}}{\pgfqpoint{1.388645in}{0.854338in}}{\pgfqpoint{1.382821in}{0.854338in}}%
\pgfpathcurveto{\pgfqpoint{1.376997in}{0.854338in}}{\pgfqpoint{1.371411in}{0.852024in}}{\pgfqpoint{1.367293in}{0.847906in}}%
\pgfpathcurveto{\pgfqpoint{1.363175in}{0.843788in}}{\pgfqpoint{1.360861in}{0.838202in}}{\pgfqpoint{1.360861in}{0.832378in}}%
\pgfpathcurveto{\pgfqpoint{1.360861in}{0.826554in}}{\pgfqpoint{1.363175in}{0.820968in}}{\pgfqpoint{1.367293in}{0.816850in}}%
\pgfpathcurveto{\pgfqpoint{1.371411in}{0.812731in}}{\pgfqpoint{1.376997in}{0.810418in}}{\pgfqpoint{1.382821in}{0.810418in}}%
\pgfpathclose%
\pgfusepath{stroke,fill}%
\end{pgfscope}%
\begin{pgfscope}%
\pgfpathrectangle{\pgfqpoint{0.211875in}{0.211875in}}{\pgfqpoint{1.313625in}{1.279725in}}%
\pgfusepath{clip}%
\pgfsetbuttcap%
\pgfsetroundjoin%
\definecolor{currentfill}{rgb}{0.121569,0.466667,0.705882}%
\pgfsetfillcolor{currentfill}%
\pgfsetlinewidth{1.003750pt}%
\definecolor{currentstroke}{rgb}{0.121569,0.466667,0.705882}%
\pgfsetstrokecolor{currentstroke}%
\pgfsetdash{}{0pt}%
\pgfpathmoveto{\pgfqpoint{1.018856in}{0.369413in}}%
\pgfpathcurveto{\pgfqpoint{1.024680in}{0.369413in}}{\pgfqpoint{1.030266in}{0.371727in}}{\pgfqpoint{1.034384in}{0.375845in}}%
\pgfpathcurveto{\pgfqpoint{1.038502in}{0.379963in}}{\pgfqpoint{1.040816in}{0.385549in}}{\pgfqpoint{1.040816in}{0.391373in}}%
\pgfpathcurveto{\pgfqpoint{1.040816in}{0.397197in}}{\pgfqpoint{1.038502in}{0.402783in}}{\pgfqpoint{1.034384in}{0.406902in}}%
\pgfpathcurveto{\pgfqpoint{1.030266in}{0.411020in}}{\pgfqpoint{1.024680in}{0.413334in}}{\pgfqpoint{1.018856in}{0.413334in}}%
\pgfpathcurveto{\pgfqpoint{1.013032in}{0.413334in}}{\pgfqpoint{1.007446in}{0.411020in}}{\pgfqpoint{1.003328in}{0.406902in}}%
\pgfpathcurveto{\pgfqpoint{0.999210in}{0.402783in}}{\pgfqpoint{0.996896in}{0.397197in}}{\pgfqpoint{0.996896in}{0.391373in}}%
\pgfpathcurveto{\pgfqpoint{0.996896in}{0.385549in}}{\pgfqpoint{0.999210in}{0.379963in}}{\pgfqpoint{1.003328in}{0.375845in}}%
\pgfpathcurveto{\pgfqpoint{1.007446in}{0.371727in}}{\pgfqpoint{1.013032in}{0.369413in}}{\pgfqpoint{1.018856in}{0.369413in}}%
\pgfpathclose%
\pgfusepath{stroke,fill}%
\end{pgfscope}%
\begin{pgfscope}%
\pgfpathrectangle{\pgfqpoint{0.211875in}{0.211875in}}{\pgfqpoint{1.313625in}{1.279725in}}%
\pgfusepath{clip}%
\pgfsetbuttcap%
\pgfsetroundjoin%
\definecolor{currentfill}{rgb}{0.121569,0.466667,0.705882}%
\pgfsetfillcolor{currentfill}%
\pgfsetlinewidth{1.003750pt}%
\definecolor{currentstroke}{rgb}{0.121569,0.466667,0.705882}%
\pgfsetstrokecolor{currentstroke}%
\pgfsetdash{}{0pt}%
\pgfpathmoveto{\pgfqpoint{0.324766in}{1.032943in}}%
\pgfpathcurveto{\pgfqpoint{0.330590in}{1.032943in}}{\pgfqpoint{0.336176in}{1.035257in}}{\pgfqpoint{0.340294in}{1.039375in}}%
\pgfpathcurveto{\pgfqpoint{0.344412in}{1.043493in}}{\pgfqpoint{0.346726in}{1.049079in}}{\pgfqpoint{0.346726in}{1.054903in}}%
\pgfpathcurveto{\pgfqpoint{0.346726in}{1.060727in}}{\pgfqpoint{0.344412in}{1.066313in}}{\pgfqpoint{0.340294in}{1.070431in}}%
\pgfpathcurveto{\pgfqpoint{0.336176in}{1.074549in}}{\pgfqpoint{0.330590in}{1.076863in}}{\pgfqpoint{0.324766in}{1.076863in}}%
\pgfpathcurveto{\pgfqpoint{0.318942in}{1.076863in}}{\pgfqpoint{0.313356in}{1.074549in}}{\pgfqpoint{0.309238in}{1.070431in}}%
\pgfpathcurveto{\pgfqpoint{0.305120in}{1.066313in}}{\pgfqpoint{0.302806in}{1.060727in}}{\pgfqpoint{0.302806in}{1.054903in}}%
\pgfpathcurveto{\pgfqpoint{0.302806in}{1.049079in}}{\pgfqpoint{0.305120in}{1.043493in}}{\pgfqpoint{0.309238in}{1.039375in}}%
\pgfpathcurveto{\pgfqpoint{0.313356in}{1.035257in}}{\pgfqpoint{0.318942in}{1.032943in}}{\pgfqpoint{0.324766in}{1.032943in}}%
\pgfpathclose%
\pgfusepath{stroke,fill}%
\end{pgfscope}%
\begin{pgfscope}%
\pgfpathrectangle{\pgfqpoint{0.211875in}{0.211875in}}{\pgfqpoint{1.313625in}{1.279725in}}%
\pgfusepath{clip}%
\pgfsetbuttcap%
\pgfsetroundjoin%
\definecolor{currentfill}{rgb}{0.121569,0.466667,0.705882}%
\pgfsetfillcolor{currentfill}%
\pgfsetlinewidth{1.003750pt}%
\definecolor{currentstroke}{rgb}{0.121569,0.466667,0.705882}%
\pgfsetstrokecolor{currentstroke}%
\pgfsetdash{}{0pt}%
\pgfpathmoveto{\pgfqpoint{0.935729in}{0.876905in}}%
\pgfpathcurveto{\pgfqpoint{0.941552in}{0.876905in}}{\pgfqpoint{0.947139in}{0.879219in}}{\pgfqpoint{0.951257in}{0.883337in}}%
\pgfpathcurveto{\pgfqpoint{0.955375in}{0.887455in}}{\pgfqpoint{0.957689in}{0.893041in}}{\pgfqpoint{0.957689in}{0.898865in}}%
\pgfpathcurveto{\pgfqpoint{0.957689in}{0.904689in}}{\pgfqpoint{0.955375in}{0.910275in}}{\pgfqpoint{0.951257in}{0.914394in}}%
\pgfpathcurveto{\pgfqpoint{0.947139in}{0.918512in}}{\pgfqpoint{0.941552in}{0.920826in}}{\pgfqpoint{0.935729in}{0.920826in}}%
\pgfpathcurveto{\pgfqpoint{0.929905in}{0.920826in}}{\pgfqpoint{0.924318in}{0.918512in}}{\pgfqpoint{0.920200in}{0.914394in}}%
\pgfpathcurveto{\pgfqpoint{0.916082in}{0.910275in}}{\pgfqpoint{0.913768in}{0.904689in}}{\pgfqpoint{0.913768in}{0.898865in}}%
\pgfpathcurveto{\pgfqpoint{0.913768in}{0.893041in}}{\pgfqpoint{0.916082in}{0.887455in}}{\pgfqpoint{0.920200in}{0.883337in}}%
\pgfpathcurveto{\pgfqpoint{0.924318in}{0.879219in}}{\pgfqpoint{0.929905in}{0.876905in}}{\pgfqpoint{0.935729in}{0.876905in}}%
\pgfpathclose%
\pgfusepath{stroke,fill}%
\end{pgfscope}%
\begin{pgfscope}%
\pgfpathrectangle{\pgfqpoint{0.211875in}{0.211875in}}{\pgfqpoint{1.313625in}{1.279725in}}%
\pgfusepath{clip}%
\pgfsetbuttcap%
\pgfsetroundjoin%
\definecolor{currentfill}{rgb}{0.121569,0.466667,0.705882}%
\pgfsetfillcolor{currentfill}%
\pgfsetlinewidth{1.003750pt}%
\definecolor{currentstroke}{rgb}{0.121569,0.466667,0.705882}%
\pgfsetstrokecolor{currentstroke}%
\pgfsetdash{}{0pt}%
\pgfpathmoveto{\pgfqpoint{0.360417in}{1.135756in}}%
\pgfpathcurveto{\pgfqpoint{0.366241in}{1.135756in}}{\pgfqpoint{0.371827in}{1.138070in}}{\pgfqpoint{0.375945in}{1.142188in}}%
\pgfpathcurveto{\pgfqpoint{0.380063in}{1.146306in}}{\pgfqpoint{0.382377in}{1.151892in}}{\pgfqpoint{0.382377in}{1.157716in}}%
\pgfpathcurveto{\pgfqpoint{0.382377in}{1.163540in}}{\pgfqpoint{0.380063in}{1.169126in}}{\pgfqpoint{0.375945in}{1.173244in}}%
\pgfpathcurveto{\pgfqpoint{0.371827in}{1.177362in}}{\pgfqpoint{0.366241in}{1.179676in}}{\pgfqpoint{0.360417in}{1.179676in}}%
\pgfpathcurveto{\pgfqpoint{0.354593in}{1.179676in}}{\pgfqpoint{0.349007in}{1.177362in}}{\pgfqpoint{0.344888in}{1.173244in}}%
\pgfpathcurveto{\pgfqpoint{0.340770in}{1.169126in}}{\pgfqpoint{0.338456in}{1.163540in}}{\pgfqpoint{0.338456in}{1.157716in}}%
\pgfpathcurveto{\pgfqpoint{0.338456in}{1.151892in}}{\pgfqpoint{0.340770in}{1.146306in}}{\pgfqpoint{0.344888in}{1.142188in}}%
\pgfpathcurveto{\pgfqpoint{0.349007in}{1.138070in}}{\pgfqpoint{0.354593in}{1.135756in}}{\pgfqpoint{0.360417in}{1.135756in}}%
\pgfpathclose%
\pgfusepath{stroke,fill}%
\end{pgfscope}%
\begin{pgfscope}%
\pgfpathrectangle{\pgfqpoint{0.211875in}{0.211875in}}{\pgfqpoint{1.313625in}{1.279725in}}%
\pgfusepath{clip}%
\pgfsetbuttcap%
\pgfsetroundjoin%
\definecolor{currentfill}{rgb}{0.121569,0.466667,0.705882}%
\pgfsetfillcolor{currentfill}%
\pgfsetlinewidth{1.003750pt}%
\definecolor{currentstroke}{rgb}{0.121569,0.466667,0.705882}%
\pgfsetstrokecolor{currentstroke}%
\pgfsetdash{}{0pt}%
\pgfpathmoveto{\pgfqpoint{1.351510in}{0.391392in}}%
\pgfpathcurveto{\pgfqpoint{1.357334in}{0.391392in}}{\pgfqpoint{1.362920in}{0.393706in}}{\pgfqpoint{1.367038in}{0.397824in}}%
\pgfpathcurveto{\pgfqpoint{1.371156in}{0.401942in}}{\pgfqpoint{1.373470in}{0.407528in}}{\pgfqpoint{1.373470in}{0.413352in}}%
\pgfpathcurveto{\pgfqpoint{1.373470in}{0.419176in}}{\pgfqpoint{1.371156in}{0.424762in}}{\pgfqpoint{1.367038in}{0.428880in}}%
\pgfpathcurveto{\pgfqpoint{1.362920in}{0.432998in}}{\pgfqpoint{1.357334in}{0.435312in}}{\pgfqpoint{1.351510in}{0.435312in}}%
\pgfpathcurveto{\pgfqpoint{1.345686in}{0.435312in}}{\pgfqpoint{1.340100in}{0.432998in}}{\pgfqpoint{1.335981in}{0.428880in}}%
\pgfpathcurveto{\pgfqpoint{1.331863in}{0.424762in}}{\pgfqpoint{1.329549in}{0.419176in}}{\pgfqpoint{1.329549in}{0.413352in}}%
\pgfpathcurveto{\pgfqpoint{1.329549in}{0.407528in}}{\pgfqpoint{1.331863in}{0.401942in}}{\pgfqpoint{1.335981in}{0.397824in}}%
\pgfpathcurveto{\pgfqpoint{1.340100in}{0.393706in}}{\pgfqpoint{1.345686in}{0.391392in}}{\pgfqpoint{1.351510in}{0.391392in}}%
\pgfpathclose%
\pgfusepath{stroke,fill}%
\end{pgfscope}%
\begin{pgfscope}%
\pgfpathrectangle{\pgfqpoint{0.211875in}{0.211875in}}{\pgfqpoint{1.313625in}{1.279725in}}%
\pgfusepath{clip}%
\pgfsetbuttcap%
\pgfsetroundjoin%
\definecolor{currentfill}{rgb}{0.121569,0.466667,0.705882}%
\pgfsetfillcolor{currentfill}%
\pgfsetlinewidth{1.003750pt}%
\definecolor{currentstroke}{rgb}{0.121569,0.466667,0.705882}%
\pgfsetstrokecolor{currentstroke}%
\pgfsetdash{}{0pt}%
\pgfpathmoveto{\pgfqpoint{0.406369in}{1.174122in}}%
\pgfpathcurveto{\pgfqpoint{0.412193in}{1.174122in}}{\pgfqpoint{0.417779in}{1.176436in}}{\pgfqpoint{0.421898in}{1.180554in}}%
\pgfpathcurveto{\pgfqpoint{0.426016in}{1.184672in}}{\pgfqpoint{0.428330in}{1.190258in}}{\pgfqpoint{0.428330in}{1.196082in}}%
\pgfpathcurveto{\pgfqpoint{0.428330in}{1.201906in}}{\pgfqpoint{0.426016in}{1.207492in}}{\pgfqpoint{0.421898in}{1.211611in}}%
\pgfpathcurveto{\pgfqpoint{0.417779in}{1.215729in}}{\pgfqpoint{0.412193in}{1.218043in}}{\pgfqpoint{0.406369in}{1.218043in}}%
\pgfpathcurveto{\pgfqpoint{0.400545in}{1.218043in}}{\pgfqpoint{0.394959in}{1.215729in}}{\pgfqpoint{0.390841in}{1.211611in}}%
\pgfpathcurveto{\pgfqpoint{0.386723in}{1.207492in}}{\pgfqpoint{0.384409in}{1.201906in}}{\pgfqpoint{0.384409in}{1.196082in}}%
\pgfpathcurveto{\pgfqpoint{0.384409in}{1.190258in}}{\pgfqpoint{0.386723in}{1.184672in}}{\pgfqpoint{0.390841in}{1.180554in}}%
\pgfpathcurveto{\pgfqpoint{0.394959in}{1.176436in}}{\pgfqpoint{0.400545in}{1.174122in}}{\pgfqpoint{0.406369in}{1.174122in}}%
\pgfpathclose%
\pgfusepath{stroke,fill}%
\end{pgfscope}%
\begin{pgfscope}%
\pgfpathrectangle{\pgfqpoint{0.211875in}{0.211875in}}{\pgfqpoint{1.313625in}{1.279725in}}%
\pgfusepath{clip}%
\pgfsetbuttcap%
\pgfsetroundjoin%
\definecolor{currentfill}{rgb}{0.121569,0.466667,0.705882}%
\pgfsetfillcolor{currentfill}%
\pgfsetlinewidth{1.003750pt}%
\definecolor{currentstroke}{rgb}{0.121569,0.466667,0.705882}%
\pgfsetstrokecolor{currentstroke}%
\pgfsetdash{}{0pt}%
\pgfpathmoveto{\pgfqpoint{1.165632in}{1.114099in}}%
\pgfpathcurveto{\pgfqpoint{1.171456in}{1.114099in}}{\pgfqpoint{1.177042in}{1.116413in}}{\pgfqpoint{1.181160in}{1.120531in}}%
\pgfpathcurveto{\pgfqpoint{1.185278in}{1.124649in}}{\pgfqpoint{1.187592in}{1.130235in}}{\pgfqpoint{1.187592in}{1.136059in}}%
\pgfpathcurveto{\pgfqpoint{1.187592in}{1.141883in}}{\pgfqpoint{1.185278in}{1.147469in}}{\pgfqpoint{1.181160in}{1.151588in}}%
\pgfpathcurveto{\pgfqpoint{1.177042in}{1.155706in}}{\pgfqpoint{1.171456in}{1.158020in}}{\pgfqpoint{1.165632in}{1.158020in}}%
\pgfpathcurveto{\pgfqpoint{1.159808in}{1.158020in}}{\pgfqpoint{1.154222in}{1.155706in}}{\pgfqpoint{1.150103in}{1.151588in}}%
\pgfpathcurveto{\pgfqpoint{1.145985in}{1.147469in}}{\pgfqpoint{1.143671in}{1.141883in}}{\pgfqpoint{1.143671in}{1.136059in}}%
\pgfpathcurveto{\pgfqpoint{1.143671in}{1.130235in}}{\pgfqpoint{1.145985in}{1.124649in}}{\pgfqpoint{1.150103in}{1.120531in}}%
\pgfpathcurveto{\pgfqpoint{1.154222in}{1.116413in}}{\pgfqpoint{1.159808in}{1.114099in}}{\pgfqpoint{1.165632in}{1.114099in}}%
\pgfpathclose%
\pgfusepath{stroke,fill}%
\end{pgfscope}%
\begin{pgfscope}%
\pgfpathrectangle{\pgfqpoint{0.211875in}{0.211875in}}{\pgfqpoint{1.313625in}{1.279725in}}%
\pgfusepath{clip}%
\pgfsetbuttcap%
\pgfsetroundjoin%
\definecolor{currentfill}{rgb}{0.121569,0.466667,0.705882}%
\pgfsetfillcolor{currentfill}%
\pgfsetlinewidth{1.003750pt}%
\definecolor{currentstroke}{rgb}{0.121569,0.466667,0.705882}%
\pgfsetstrokecolor{currentstroke}%
\pgfsetdash{}{0pt}%
\pgfpathmoveto{\pgfqpoint{0.635317in}{0.484518in}}%
\pgfpathcurveto{\pgfqpoint{0.641141in}{0.484518in}}{\pgfqpoint{0.646727in}{0.486832in}}{\pgfqpoint{0.650845in}{0.490950in}}%
\pgfpathcurveto{\pgfqpoint{0.654963in}{0.495068in}}{\pgfqpoint{0.657277in}{0.500655in}}{\pgfqpoint{0.657277in}{0.506479in}}%
\pgfpathcurveto{\pgfqpoint{0.657277in}{0.512302in}}{\pgfqpoint{0.654963in}{0.517889in}}{\pgfqpoint{0.650845in}{0.522007in}}%
\pgfpathcurveto{\pgfqpoint{0.646727in}{0.526125in}}{\pgfqpoint{0.641141in}{0.528439in}}{\pgfqpoint{0.635317in}{0.528439in}}%
\pgfpathcurveto{\pgfqpoint{0.629493in}{0.528439in}}{\pgfqpoint{0.623907in}{0.526125in}}{\pgfqpoint{0.619789in}{0.522007in}}%
\pgfpathcurveto{\pgfqpoint{0.615671in}{0.517889in}}{\pgfqpoint{0.613357in}{0.512302in}}{\pgfqpoint{0.613357in}{0.506479in}}%
\pgfpathcurveto{\pgfqpoint{0.613357in}{0.500655in}}{\pgfqpoint{0.615671in}{0.495068in}}{\pgfqpoint{0.619789in}{0.490950in}}%
\pgfpathcurveto{\pgfqpoint{0.623907in}{0.486832in}}{\pgfqpoint{0.629493in}{0.484518in}}{\pgfqpoint{0.635317in}{0.484518in}}%
\pgfpathclose%
\pgfusepath{stroke,fill}%
\end{pgfscope}%
\begin{pgfscope}%
\pgfpathrectangle{\pgfqpoint{0.211875in}{0.211875in}}{\pgfqpoint{1.313625in}{1.279725in}}%
\pgfusepath{clip}%
\pgfsetbuttcap%
\pgfsetroundjoin%
\definecolor{currentfill}{rgb}{0.121569,0.466667,0.705882}%
\pgfsetfillcolor{currentfill}%
\pgfsetlinewidth{1.003750pt}%
\definecolor{currentstroke}{rgb}{0.121569,0.466667,0.705882}%
\pgfsetstrokecolor{currentstroke}%
\pgfsetdash{}{0pt}%
\pgfpathmoveto{\pgfqpoint{0.894060in}{1.133481in}}%
\pgfpathcurveto{\pgfqpoint{0.899884in}{1.133481in}}{\pgfqpoint{0.905471in}{1.135795in}}{\pgfqpoint{0.909589in}{1.139913in}}%
\pgfpathcurveto{\pgfqpoint{0.913707in}{1.144032in}}{\pgfqpoint{0.916021in}{1.149618in}}{\pgfqpoint{0.916021in}{1.155442in}}%
\pgfpathcurveto{\pgfqpoint{0.916021in}{1.161266in}}{\pgfqpoint{0.913707in}{1.166852in}}{\pgfqpoint{0.909589in}{1.170970in}}%
\pgfpathcurveto{\pgfqpoint{0.905471in}{1.175088in}}{\pgfqpoint{0.899884in}{1.177402in}}{\pgfqpoint{0.894060in}{1.177402in}}%
\pgfpathcurveto{\pgfqpoint{0.888237in}{1.177402in}}{\pgfqpoint{0.882650in}{1.175088in}}{\pgfqpoint{0.878532in}{1.170970in}}%
\pgfpathcurveto{\pgfqpoint{0.874414in}{1.166852in}}{\pgfqpoint{0.872100in}{1.161266in}}{\pgfqpoint{0.872100in}{1.155442in}}%
\pgfpathcurveto{\pgfqpoint{0.872100in}{1.149618in}}{\pgfqpoint{0.874414in}{1.144032in}}{\pgfqpoint{0.878532in}{1.139913in}}%
\pgfpathcurveto{\pgfqpoint{0.882650in}{1.135795in}}{\pgfqpoint{0.888237in}{1.133481in}}{\pgfqpoint{0.894060in}{1.133481in}}%
\pgfpathclose%
\pgfusepath{stroke,fill}%
\end{pgfscope}%
\begin{pgfscope}%
\pgfpathrectangle{\pgfqpoint{0.211875in}{0.211875in}}{\pgfqpoint{1.313625in}{1.279725in}}%
\pgfusepath{clip}%
\pgfsetbuttcap%
\pgfsetroundjoin%
\definecolor{currentfill}{rgb}{0.121569,0.466667,0.705882}%
\pgfsetfillcolor{currentfill}%
\pgfsetlinewidth{1.003750pt}%
\definecolor{currentstroke}{rgb}{0.121569,0.466667,0.705882}%
\pgfsetstrokecolor{currentstroke}%
\pgfsetdash{}{0pt}%
\pgfpathmoveto{\pgfqpoint{1.259010in}{0.874732in}}%
\pgfpathcurveto{\pgfqpoint{1.264834in}{0.874732in}}{\pgfqpoint{1.270420in}{0.877046in}}{\pgfqpoint{1.274539in}{0.881164in}}%
\pgfpathcurveto{\pgfqpoint{1.278657in}{0.885282in}}{\pgfqpoint{1.280971in}{0.890868in}}{\pgfqpoint{1.280971in}{0.896692in}}%
\pgfpathcurveto{\pgfqpoint{1.280971in}{0.902516in}}{\pgfqpoint{1.278657in}{0.908103in}}{\pgfqpoint{1.274539in}{0.912221in}}%
\pgfpathcurveto{\pgfqpoint{1.270420in}{0.916339in}}{\pgfqpoint{1.264834in}{0.918653in}}{\pgfqpoint{1.259010in}{0.918653in}}%
\pgfpathcurveto{\pgfqpoint{1.253186in}{0.918653in}}{\pgfqpoint{1.247600in}{0.916339in}}{\pgfqpoint{1.243482in}{0.912221in}}%
\pgfpathcurveto{\pgfqpoint{1.239364in}{0.908103in}}{\pgfqpoint{1.237050in}{0.902516in}}{\pgfqpoint{1.237050in}{0.896692in}}%
\pgfpathcurveto{\pgfqpoint{1.237050in}{0.890868in}}{\pgfqpoint{1.239364in}{0.885282in}}{\pgfqpoint{1.243482in}{0.881164in}}%
\pgfpathcurveto{\pgfqpoint{1.247600in}{0.877046in}}{\pgfqpoint{1.253186in}{0.874732in}}{\pgfqpoint{1.259010in}{0.874732in}}%
\pgfpathclose%
\pgfusepath{stroke,fill}%
\end{pgfscope}%
\begin{pgfscope}%
\pgfpathrectangle{\pgfqpoint{0.211875in}{0.211875in}}{\pgfqpoint{1.313625in}{1.279725in}}%
\pgfusepath{clip}%
\pgfsetbuttcap%
\pgfsetroundjoin%
\definecolor{currentfill}{rgb}{0.121569,0.466667,0.705882}%
\pgfsetfillcolor{currentfill}%
\pgfsetlinewidth{1.003750pt}%
\definecolor{currentstroke}{rgb}{0.121569,0.466667,0.705882}%
\pgfsetstrokecolor{currentstroke}%
\pgfsetdash{}{0pt}%
\pgfpathmoveto{\pgfqpoint{0.347880in}{1.368240in}}%
\pgfpathcurveto{\pgfqpoint{0.353704in}{1.368240in}}{\pgfqpoint{0.359290in}{1.370554in}}{\pgfqpoint{0.363408in}{1.374672in}}%
\pgfpathcurveto{\pgfqpoint{0.367526in}{1.378791in}}{\pgfqpoint{0.369840in}{1.384377in}}{\pgfqpoint{0.369840in}{1.390201in}}%
\pgfpathcurveto{\pgfqpoint{0.369840in}{1.396025in}}{\pgfqpoint{0.367526in}{1.401611in}}{\pgfqpoint{0.363408in}{1.405729in}}%
\pgfpathcurveto{\pgfqpoint{0.359290in}{1.409847in}}{\pgfqpoint{0.353704in}{1.412161in}}{\pgfqpoint{0.347880in}{1.412161in}}%
\pgfpathcurveto{\pgfqpoint{0.342056in}{1.412161in}}{\pgfqpoint{0.336470in}{1.409847in}}{\pgfqpoint{0.332352in}{1.405729in}}%
\pgfpathcurveto{\pgfqpoint{0.328234in}{1.401611in}}{\pgfqpoint{0.325920in}{1.396025in}}{\pgfqpoint{0.325920in}{1.390201in}}%
\pgfpathcurveto{\pgfqpoint{0.325920in}{1.384377in}}{\pgfqpoint{0.328234in}{1.378791in}}{\pgfqpoint{0.332352in}{1.374672in}}%
\pgfpathcurveto{\pgfqpoint{0.336470in}{1.370554in}}{\pgfqpoint{0.342056in}{1.368240in}}{\pgfqpoint{0.347880in}{1.368240in}}%
\pgfpathclose%
\pgfusepath{stroke,fill}%
\end{pgfscope}%
\begin{pgfscope}%
\pgfpathrectangle{\pgfqpoint{0.211875in}{0.211875in}}{\pgfqpoint{1.313625in}{1.279725in}}%
\pgfusepath{clip}%
\pgfsetbuttcap%
\pgfsetroundjoin%
\definecolor{currentfill}{rgb}{0.121569,0.466667,0.705882}%
\pgfsetfillcolor{currentfill}%
\pgfsetlinewidth{1.003750pt}%
\definecolor{currentstroke}{rgb}{0.121569,0.466667,0.705882}%
\pgfsetstrokecolor{currentstroke}%
\pgfsetdash{}{0pt}%
\pgfpathmoveto{\pgfqpoint{1.244544in}{0.932170in}}%
\pgfpathcurveto{\pgfqpoint{1.250368in}{0.932170in}}{\pgfqpoint{1.255954in}{0.934483in}}{\pgfqpoint{1.260073in}{0.938602in}}%
\pgfpathcurveto{\pgfqpoint{1.264191in}{0.942720in}}{\pgfqpoint{1.266505in}{0.948306in}}{\pgfqpoint{1.266505in}{0.954130in}}%
\pgfpathcurveto{\pgfqpoint{1.266505in}{0.959954in}}{\pgfqpoint{1.264191in}{0.965540in}}{\pgfqpoint{1.260073in}{0.969658in}}%
\pgfpathcurveto{\pgfqpoint{1.255954in}{0.973776in}}{\pgfqpoint{1.250368in}{0.976090in}}{\pgfqpoint{1.244544in}{0.976090in}}%
\pgfpathcurveto{\pgfqpoint{1.238720in}{0.976090in}}{\pgfqpoint{1.233134in}{0.973776in}}{\pgfqpoint{1.229016in}{0.969658in}}%
\pgfpathcurveto{\pgfqpoint{1.224898in}{0.965540in}}{\pgfqpoint{1.222584in}{0.959954in}}{\pgfqpoint{1.222584in}{0.954130in}}%
\pgfpathcurveto{\pgfqpoint{1.222584in}{0.948306in}}{\pgfqpoint{1.224898in}{0.942720in}}{\pgfqpoint{1.229016in}{0.938602in}}%
\pgfpathcurveto{\pgfqpoint{1.233134in}{0.934483in}}{\pgfqpoint{1.238720in}{0.932170in}}{\pgfqpoint{1.244544in}{0.932170in}}%
\pgfpathclose%
\pgfusepath{stroke,fill}%
\end{pgfscope}%
\begin{pgfscope}%
\pgfpathrectangle{\pgfqpoint{0.211875in}{0.211875in}}{\pgfqpoint{1.313625in}{1.279725in}}%
\pgfusepath{clip}%
\pgfsetbuttcap%
\pgfsetroundjoin%
\definecolor{currentfill}{rgb}{0.121569,0.466667,0.705882}%
\pgfsetfillcolor{currentfill}%
\pgfsetlinewidth{1.003750pt}%
\definecolor{currentstroke}{rgb}{0.121569,0.466667,0.705882}%
\pgfsetstrokecolor{currentstroke}%
\pgfsetdash{}{0pt}%
\pgfpathmoveto{\pgfqpoint{1.430032in}{0.592525in}}%
\pgfpathcurveto{\pgfqpoint{1.435856in}{0.592525in}}{\pgfqpoint{1.441442in}{0.594839in}}{\pgfqpoint{1.445560in}{0.598957in}}%
\pgfpathcurveto{\pgfqpoint{1.449678in}{0.603075in}}{\pgfqpoint{1.451992in}{0.608661in}}{\pgfqpoint{1.451992in}{0.614485in}}%
\pgfpathcurveto{\pgfqpoint{1.451992in}{0.620309in}}{\pgfqpoint{1.449678in}{0.625895in}}{\pgfqpoint{1.445560in}{0.630013in}}%
\pgfpathcurveto{\pgfqpoint{1.441442in}{0.634131in}}{\pgfqpoint{1.435856in}{0.636445in}}{\pgfqpoint{1.430032in}{0.636445in}}%
\pgfpathcurveto{\pgfqpoint{1.424208in}{0.636445in}}{\pgfqpoint{1.418622in}{0.634131in}}{\pgfqpoint{1.414504in}{0.630013in}}%
\pgfpathcurveto{\pgfqpoint{1.410386in}{0.625895in}}{\pgfqpoint{1.408072in}{0.620309in}}{\pgfqpoint{1.408072in}{0.614485in}}%
\pgfpathcurveto{\pgfqpoint{1.408072in}{0.608661in}}{\pgfqpoint{1.410386in}{0.603075in}}{\pgfqpoint{1.414504in}{0.598957in}}%
\pgfpathcurveto{\pgfqpoint{1.418622in}{0.594839in}}{\pgfqpoint{1.424208in}{0.592525in}}{\pgfqpoint{1.430032in}{0.592525in}}%
\pgfpathclose%
\pgfusepath{stroke,fill}%
\end{pgfscope}%
\begin{pgfscope}%
\pgfpathrectangle{\pgfqpoint{0.211875in}{0.211875in}}{\pgfqpoint{1.313625in}{1.279725in}}%
\pgfusepath{clip}%
\pgfsetbuttcap%
\pgfsetroundjoin%
\definecolor{currentfill}{rgb}{0.121569,0.466667,0.705882}%
\pgfsetfillcolor{currentfill}%
\pgfsetlinewidth{1.003750pt}%
\definecolor{currentstroke}{rgb}{0.121569,0.466667,0.705882}%
\pgfsetstrokecolor{currentstroke}%
\pgfsetdash{}{0pt}%
\pgfpathmoveto{\pgfqpoint{0.910546in}{0.304960in}}%
\pgfpathcurveto{\pgfqpoint{0.916370in}{0.304960in}}{\pgfqpoint{0.921956in}{0.307274in}}{\pgfqpoint{0.926074in}{0.311392in}}%
\pgfpathcurveto{\pgfqpoint{0.930192in}{0.315510in}}{\pgfqpoint{0.932506in}{0.321096in}}{\pgfqpoint{0.932506in}{0.326920in}}%
\pgfpathcurveto{\pgfqpoint{0.932506in}{0.332744in}}{\pgfqpoint{0.930192in}{0.338330in}}{\pgfqpoint{0.926074in}{0.342448in}}%
\pgfpathcurveto{\pgfqpoint{0.921956in}{0.346567in}}{\pgfqpoint{0.916370in}{0.348880in}}{\pgfqpoint{0.910546in}{0.348880in}}%
\pgfpathcurveto{\pgfqpoint{0.904722in}{0.348880in}}{\pgfqpoint{0.899136in}{0.346567in}}{\pgfqpoint{0.895018in}{0.342448in}}%
\pgfpathcurveto{\pgfqpoint{0.890899in}{0.338330in}}{\pgfqpoint{0.888586in}{0.332744in}}{\pgfqpoint{0.888586in}{0.326920in}}%
\pgfpathcurveto{\pgfqpoint{0.888586in}{0.321096in}}{\pgfqpoint{0.890899in}{0.315510in}}{\pgfqpoint{0.895018in}{0.311392in}}%
\pgfpathcurveto{\pgfqpoint{0.899136in}{0.307274in}}{\pgfqpoint{0.904722in}{0.304960in}}{\pgfqpoint{0.910546in}{0.304960in}}%
\pgfpathclose%
\pgfusepath{stroke,fill}%
\end{pgfscope}%
\begin{pgfscope}%
\pgfpathrectangle{\pgfqpoint{0.211875in}{0.211875in}}{\pgfqpoint{1.313625in}{1.279725in}}%
\pgfusepath{clip}%
\pgfsetbuttcap%
\pgfsetroundjoin%
\definecolor{currentfill}{rgb}{0.121569,0.466667,0.705882}%
\pgfsetfillcolor{currentfill}%
\pgfsetlinewidth{1.003750pt}%
\definecolor{currentstroke}{rgb}{0.121569,0.466667,0.705882}%
\pgfsetstrokecolor{currentstroke}%
\pgfsetdash{}{0pt}%
\pgfpathmoveto{\pgfqpoint{1.129620in}{1.386221in}}%
\pgfpathcurveto{\pgfqpoint{1.135444in}{1.386221in}}{\pgfqpoint{1.141030in}{1.388535in}}{\pgfqpoint{1.145148in}{1.392653in}}%
\pgfpathcurveto{\pgfqpoint{1.149267in}{1.396772in}}{\pgfqpoint{1.151580in}{1.402358in}}{\pgfqpoint{1.151580in}{1.408182in}}%
\pgfpathcurveto{\pgfqpoint{1.151580in}{1.414006in}}{\pgfqpoint{1.149267in}{1.419592in}}{\pgfqpoint{1.145148in}{1.423710in}}%
\pgfpathcurveto{\pgfqpoint{1.141030in}{1.427828in}}{\pgfqpoint{1.135444in}{1.430142in}}{\pgfqpoint{1.129620in}{1.430142in}}%
\pgfpathcurveto{\pgfqpoint{1.123796in}{1.430142in}}{\pgfqpoint{1.118210in}{1.427828in}}{\pgfqpoint{1.114092in}{1.423710in}}%
\pgfpathcurveto{\pgfqpoint{1.109974in}{1.419592in}}{\pgfqpoint{1.107660in}{1.414006in}}{\pgfqpoint{1.107660in}{1.408182in}}%
\pgfpathcurveto{\pgfqpoint{1.107660in}{1.402358in}}{\pgfqpoint{1.109974in}{1.396772in}}{\pgfqpoint{1.114092in}{1.392653in}}%
\pgfpathcurveto{\pgfqpoint{1.118210in}{1.388535in}}{\pgfqpoint{1.123796in}{1.386221in}}{\pgfqpoint{1.129620in}{1.386221in}}%
\pgfpathclose%
\pgfusepath{stroke,fill}%
\end{pgfscope}%
\begin{pgfscope}%
\pgfpathrectangle{\pgfqpoint{0.211875in}{0.211875in}}{\pgfqpoint{1.313625in}{1.279725in}}%
\pgfusepath{clip}%
\pgfsetbuttcap%
\pgfsetroundjoin%
\definecolor{currentfill}{rgb}{0.121569,0.466667,0.705882}%
\pgfsetfillcolor{currentfill}%
\pgfsetlinewidth{1.003750pt}%
\definecolor{currentstroke}{rgb}{0.121569,0.466667,0.705882}%
\pgfsetstrokecolor{currentstroke}%
\pgfsetdash{}{0pt}%
\pgfpathmoveto{\pgfqpoint{0.296073in}{0.382209in}}%
\pgfpathcurveto{\pgfqpoint{0.301897in}{0.382209in}}{\pgfqpoint{0.307484in}{0.384523in}}{\pgfqpoint{0.311602in}{0.388641in}}%
\pgfpathcurveto{\pgfqpoint{0.315720in}{0.392759in}}{\pgfqpoint{0.318034in}{0.398346in}}{\pgfqpoint{0.318034in}{0.404169in}}%
\pgfpathcurveto{\pgfqpoint{0.318034in}{0.409993in}}{\pgfqpoint{0.315720in}{0.415580in}}{\pgfqpoint{0.311602in}{0.419698in}}%
\pgfpathcurveto{\pgfqpoint{0.307484in}{0.423816in}}{\pgfqpoint{0.301897in}{0.426130in}}{\pgfqpoint{0.296073in}{0.426130in}}%
\pgfpathcurveto{\pgfqpoint{0.290250in}{0.426130in}}{\pgfqpoint{0.284663in}{0.423816in}}{\pgfqpoint{0.280545in}{0.419698in}}%
\pgfpathcurveto{\pgfqpoint{0.276427in}{0.415580in}}{\pgfqpoint{0.274113in}{0.409993in}}{\pgfqpoint{0.274113in}{0.404169in}}%
\pgfpathcurveto{\pgfqpoint{0.274113in}{0.398346in}}{\pgfqpoint{0.276427in}{0.392759in}}{\pgfqpoint{0.280545in}{0.388641in}}%
\pgfpathcurveto{\pgfqpoint{0.284663in}{0.384523in}}{\pgfqpoint{0.290250in}{0.382209in}}{\pgfqpoint{0.296073in}{0.382209in}}%
\pgfpathclose%
\pgfusepath{stroke,fill}%
\end{pgfscope}%
\begin{pgfscope}%
\pgfpathrectangle{\pgfqpoint{0.211875in}{0.211875in}}{\pgfqpoint{1.313625in}{1.279725in}}%
\pgfusepath{clip}%
\pgfsetbuttcap%
\pgfsetroundjoin%
\definecolor{currentfill}{rgb}{0.121569,0.466667,0.705882}%
\pgfsetfillcolor{currentfill}%
\pgfsetlinewidth{1.003750pt}%
\definecolor{currentstroke}{rgb}{0.121569,0.466667,0.705882}%
\pgfsetstrokecolor{currentstroke}%
\pgfsetdash{}{0pt}%
\pgfpathmoveto{\pgfqpoint{1.160251in}{0.305223in}}%
\pgfpathcurveto{\pgfqpoint{1.166075in}{0.305223in}}{\pgfqpoint{1.171661in}{0.307537in}}{\pgfqpoint{1.175779in}{0.311655in}}%
\pgfpathcurveto{\pgfqpoint{1.179897in}{0.315773in}}{\pgfqpoint{1.182211in}{0.321359in}}{\pgfqpoint{1.182211in}{0.327183in}}%
\pgfpathcurveto{\pgfqpoint{1.182211in}{0.333007in}}{\pgfqpoint{1.179897in}{0.338593in}}{\pgfqpoint{1.175779in}{0.342711in}}%
\pgfpathcurveto{\pgfqpoint{1.171661in}{0.346829in}}{\pgfqpoint{1.166075in}{0.349143in}}{\pgfqpoint{1.160251in}{0.349143in}}%
\pgfpathcurveto{\pgfqpoint{1.154427in}{0.349143in}}{\pgfqpoint{1.148841in}{0.346829in}}{\pgfqpoint{1.144723in}{0.342711in}}%
\pgfpathcurveto{\pgfqpoint{1.140604in}{0.338593in}}{\pgfqpoint{1.138291in}{0.333007in}}{\pgfqpoint{1.138291in}{0.327183in}}%
\pgfpathcurveto{\pgfqpoint{1.138291in}{0.321359in}}{\pgfqpoint{1.140604in}{0.315773in}}{\pgfqpoint{1.144723in}{0.311655in}}%
\pgfpathcurveto{\pgfqpoint{1.148841in}{0.307537in}}{\pgfqpoint{1.154427in}{0.305223in}}{\pgfqpoint{1.160251in}{0.305223in}}%
\pgfpathclose%
\pgfusepath{stroke,fill}%
\end{pgfscope}%
\begin{pgfscope}%
\pgfpathrectangle{\pgfqpoint{0.211875in}{0.211875in}}{\pgfqpoint{1.313625in}{1.279725in}}%
\pgfusepath{clip}%
\pgfsetbuttcap%
\pgfsetroundjoin%
\definecolor{currentfill}{rgb}{0.121569,0.466667,0.705882}%
\pgfsetfillcolor{currentfill}%
\pgfsetlinewidth{1.003750pt}%
\definecolor{currentstroke}{rgb}{0.121569,0.466667,0.705882}%
\pgfsetstrokecolor{currentstroke}%
\pgfsetdash{}{0pt}%
\pgfpathmoveto{\pgfqpoint{1.129907in}{0.869794in}}%
\pgfpathcurveto{\pgfqpoint{1.135731in}{0.869794in}}{\pgfqpoint{1.141318in}{0.872108in}}{\pgfqpoint{1.145436in}{0.876226in}}%
\pgfpathcurveto{\pgfqpoint{1.149554in}{0.880344in}}{\pgfqpoint{1.151868in}{0.885931in}}{\pgfqpoint{1.151868in}{0.891755in}}%
\pgfpathcurveto{\pgfqpoint{1.151868in}{0.897578in}}{\pgfqpoint{1.149554in}{0.903165in}}{\pgfqpoint{1.145436in}{0.907283in}}%
\pgfpathcurveto{\pgfqpoint{1.141318in}{0.911401in}}{\pgfqpoint{1.135731in}{0.913715in}}{\pgfqpoint{1.129907in}{0.913715in}}%
\pgfpathcurveto{\pgfqpoint{1.124083in}{0.913715in}}{\pgfqpoint{1.118497in}{0.911401in}}{\pgfqpoint{1.114379in}{0.907283in}}%
\pgfpathcurveto{\pgfqpoint{1.110261in}{0.903165in}}{\pgfqpoint{1.107947in}{0.897578in}}{\pgfqpoint{1.107947in}{0.891755in}}%
\pgfpathcurveto{\pgfqpoint{1.107947in}{0.885931in}}{\pgfqpoint{1.110261in}{0.880344in}}{\pgfqpoint{1.114379in}{0.876226in}}%
\pgfpathcurveto{\pgfqpoint{1.118497in}{0.872108in}}{\pgfqpoint{1.124083in}{0.869794in}}{\pgfqpoint{1.129907in}{0.869794in}}%
\pgfpathclose%
\pgfusepath{stroke,fill}%
\end{pgfscope}%
\begin{pgfscope}%
\pgfpathrectangle{\pgfqpoint{0.211875in}{0.211875in}}{\pgfqpoint{1.313625in}{1.279725in}}%
\pgfusepath{clip}%
\pgfsetbuttcap%
\pgfsetroundjoin%
\definecolor{currentfill}{rgb}{0.121569,0.466667,0.705882}%
\pgfsetfillcolor{currentfill}%
\pgfsetlinewidth{1.003750pt}%
\definecolor{currentstroke}{rgb}{0.121569,0.466667,0.705882}%
\pgfsetstrokecolor{currentstroke}%
\pgfsetdash{}{0pt}%
\pgfpathmoveto{\pgfqpoint{1.093008in}{1.182390in}}%
\pgfpathcurveto{\pgfqpoint{1.098832in}{1.182390in}}{\pgfqpoint{1.104418in}{1.184704in}}{\pgfqpoint{1.108537in}{1.188822in}}%
\pgfpathcurveto{\pgfqpoint{1.112655in}{1.192941in}}{\pgfqpoint{1.114969in}{1.198527in}}{\pgfqpoint{1.114969in}{1.204351in}}%
\pgfpathcurveto{\pgfqpoint{1.114969in}{1.210175in}}{\pgfqpoint{1.112655in}{1.215761in}}{\pgfqpoint{1.108537in}{1.219879in}}%
\pgfpathcurveto{\pgfqpoint{1.104418in}{1.223997in}}{\pgfqpoint{1.098832in}{1.226311in}}{\pgfqpoint{1.093008in}{1.226311in}}%
\pgfpathcurveto{\pgfqpoint{1.087184in}{1.226311in}}{\pgfqpoint{1.081598in}{1.223997in}}{\pgfqpoint{1.077480in}{1.219879in}}%
\pgfpathcurveto{\pgfqpoint{1.073362in}{1.215761in}}{\pgfqpoint{1.071048in}{1.210175in}}{\pgfqpoint{1.071048in}{1.204351in}}%
\pgfpathcurveto{\pgfqpoint{1.071048in}{1.198527in}}{\pgfqpoint{1.073362in}{1.192941in}}{\pgfqpoint{1.077480in}{1.188822in}}%
\pgfpathcurveto{\pgfqpoint{1.081598in}{1.184704in}}{\pgfqpoint{1.087184in}{1.182390in}}{\pgfqpoint{1.093008in}{1.182390in}}%
\pgfpathclose%
\pgfusepath{stroke,fill}%
\end{pgfscope}%
\begin{pgfscope}%
\pgfpathrectangle{\pgfqpoint{0.211875in}{0.211875in}}{\pgfqpoint{1.313625in}{1.279725in}}%
\pgfusepath{clip}%
\pgfsetbuttcap%
\pgfsetroundjoin%
\definecolor{currentfill}{rgb}{0.121569,0.466667,0.705882}%
\pgfsetfillcolor{currentfill}%
\pgfsetlinewidth{1.003750pt}%
\definecolor{currentstroke}{rgb}{0.121569,0.466667,0.705882}%
\pgfsetstrokecolor{currentstroke}%
\pgfsetdash{}{0pt}%
\pgfpathmoveto{\pgfqpoint{0.607413in}{1.388240in}}%
\pgfpathcurveto{\pgfqpoint{0.613237in}{1.388240in}}{\pgfqpoint{0.618823in}{1.390554in}}{\pgfqpoint{0.622941in}{1.394672in}}%
\pgfpathcurveto{\pgfqpoint{0.627059in}{1.398790in}}{\pgfqpoint{0.629373in}{1.404376in}}{\pgfqpoint{0.629373in}{1.410200in}}%
\pgfpathcurveto{\pgfqpoint{0.629373in}{1.416024in}}{\pgfqpoint{0.627059in}{1.421610in}}{\pgfqpoint{0.622941in}{1.425729in}}%
\pgfpathcurveto{\pgfqpoint{0.618823in}{1.429847in}}{\pgfqpoint{0.613237in}{1.432161in}}{\pgfqpoint{0.607413in}{1.432161in}}%
\pgfpathcurveto{\pgfqpoint{0.601589in}{1.432161in}}{\pgfqpoint{0.596003in}{1.429847in}}{\pgfqpoint{0.591885in}{1.425729in}}%
\pgfpathcurveto{\pgfqpoint{0.587767in}{1.421610in}}{\pgfqpoint{0.585453in}{1.416024in}}{\pgfqpoint{0.585453in}{1.410200in}}%
\pgfpathcurveto{\pgfqpoint{0.585453in}{1.404376in}}{\pgfqpoint{0.587767in}{1.398790in}}{\pgfqpoint{0.591885in}{1.394672in}}%
\pgfpathcurveto{\pgfqpoint{0.596003in}{1.390554in}}{\pgfqpoint{0.601589in}{1.388240in}}{\pgfqpoint{0.607413in}{1.388240in}}%
\pgfpathclose%
\pgfusepath{stroke,fill}%
\end{pgfscope}%
\begin{pgfscope}%
\pgfpathrectangle{\pgfqpoint{0.211875in}{0.211875in}}{\pgfqpoint{1.313625in}{1.279725in}}%
\pgfusepath{clip}%
\pgfsetbuttcap%
\pgfsetroundjoin%
\definecolor{currentfill}{rgb}{0.121569,0.466667,0.705882}%
\pgfsetfillcolor{currentfill}%
\pgfsetlinewidth{1.003750pt}%
\definecolor{currentstroke}{rgb}{0.121569,0.466667,0.705882}%
\pgfsetstrokecolor{currentstroke}%
\pgfsetdash{}{0pt}%
\pgfpathmoveto{\pgfqpoint{1.072070in}{0.288381in}}%
\pgfpathcurveto{\pgfqpoint{1.077894in}{0.288381in}}{\pgfqpoint{1.083480in}{0.290695in}}{\pgfqpoint{1.087599in}{0.294813in}}%
\pgfpathcurveto{\pgfqpoint{1.091717in}{0.298931in}}{\pgfqpoint{1.094031in}{0.304518in}}{\pgfqpoint{1.094031in}{0.310342in}}%
\pgfpathcurveto{\pgfqpoint{1.094031in}{0.316165in}}{\pgfqpoint{1.091717in}{0.321752in}}{\pgfqpoint{1.087599in}{0.325870in}}%
\pgfpathcurveto{\pgfqpoint{1.083480in}{0.329988in}}{\pgfqpoint{1.077894in}{0.332302in}}{\pgfqpoint{1.072070in}{0.332302in}}%
\pgfpathcurveto{\pgfqpoint{1.066246in}{0.332302in}}{\pgfqpoint{1.060660in}{0.329988in}}{\pgfqpoint{1.056542in}{0.325870in}}%
\pgfpathcurveto{\pgfqpoint{1.052424in}{0.321752in}}{\pgfqpoint{1.050110in}{0.316165in}}{\pgfqpoint{1.050110in}{0.310342in}}%
\pgfpathcurveto{\pgfqpoint{1.050110in}{0.304518in}}{\pgfqpoint{1.052424in}{0.298931in}}{\pgfqpoint{1.056542in}{0.294813in}}%
\pgfpathcurveto{\pgfqpoint{1.060660in}{0.290695in}}{\pgfqpoint{1.066246in}{0.288381in}}{\pgfqpoint{1.072070in}{0.288381in}}%
\pgfpathclose%
\pgfusepath{stroke,fill}%
\end{pgfscope}%
\begin{pgfscope}%
\pgfpathrectangle{\pgfqpoint{0.211875in}{0.211875in}}{\pgfqpoint{1.313625in}{1.279725in}}%
\pgfusepath{clip}%
\pgfsetbuttcap%
\pgfsetroundjoin%
\definecolor{currentfill}{rgb}{0.121569,0.466667,0.705882}%
\pgfsetfillcolor{currentfill}%
\pgfsetlinewidth{1.003750pt}%
\definecolor{currentstroke}{rgb}{0.121569,0.466667,0.705882}%
\pgfsetstrokecolor{currentstroke}%
\pgfsetdash{}{0pt}%
\pgfpathmoveto{\pgfqpoint{1.260919in}{0.881293in}}%
\pgfpathcurveto{\pgfqpoint{1.266743in}{0.881293in}}{\pgfqpoint{1.272330in}{0.883607in}}{\pgfqpoint{1.276448in}{0.887725in}}%
\pgfpathcurveto{\pgfqpoint{1.280566in}{0.891843in}}{\pgfqpoint{1.282880in}{0.897429in}}{\pgfqpoint{1.282880in}{0.903253in}}%
\pgfpathcurveto{\pgfqpoint{1.282880in}{0.909077in}}{\pgfqpoint{1.280566in}{0.914663in}}{\pgfqpoint{1.276448in}{0.918781in}}%
\pgfpathcurveto{\pgfqpoint{1.272330in}{0.922899in}}{\pgfqpoint{1.266743in}{0.925213in}}{\pgfqpoint{1.260919in}{0.925213in}}%
\pgfpathcurveto{\pgfqpoint{1.255096in}{0.925213in}}{\pgfqpoint{1.249509in}{0.922899in}}{\pgfqpoint{1.245391in}{0.918781in}}%
\pgfpathcurveto{\pgfqpoint{1.241273in}{0.914663in}}{\pgfqpoint{1.238959in}{0.909077in}}{\pgfqpoint{1.238959in}{0.903253in}}%
\pgfpathcurveto{\pgfqpoint{1.238959in}{0.897429in}}{\pgfqpoint{1.241273in}{0.891843in}}{\pgfqpoint{1.245391in}{0.887725in}}%
\pgfpathcurveto{\pgfqpoint{1.249509in}{0.883607in}}{\pgfqpoint{1.255096in}{0.881293in}}{\pgfqpoint{1.260919in}{0.881293in}}%
\pgfpathclose%
\pgfusepath{stroke,fill}%
\end{pgfscope}%
\begin{pgfscope}%
\pgfpathrectangle{\pgfqpoint{0.211875in}{0.211875in}}{\pgfqpoint{1.313625in}{1.279725in}}%
\pgfusepath{clip}%
\pgfsetbuttcap%
\pgfsetroundjoin%
\definecolor{currentfill}{rgb}{0.121569,0.466667,0.705882}%
\pgfsetfillcolor{currentfill}%
\pgfsetlinewidth{1.003750pt}%
\definecolor{currentstroke}{rgb}{0.121569,0.466667,0.705882}%
\pgfsetstrokecolor{currentstroke}%
\pgfsetdash{}{0pt}%
\pgfpathmoveto{\pgfqpoint{0.664804in}{0.800668in}}%
\pgfpathcurveto{\pgfqpoint{0.670628in}{0.800668in}}{\pgfqpoint{0.676214in}{0.802982in}}{\pgfqpoint{0.680332in}{0.807100in}}%
\pgfpathcurveto{\pgfqpoint{0.684450in}{0.811218in}}{\pgfqpoint{0.686764in}{0.816804in}}{\pgfqpoint{0.686764in}{0.822628in}}%
\pgfpathcurveto{\pgfqpoint{0.686764in}{0.828452in}}{\pgfqpoint{0.684450in}{0.834038in}}{\pgfqpoint{0.680332in}{0.838156in}}%
\pgfpathcurveto{\pgfqpoint{0.676214in}{0.842275in}}{\pgfqpoint{0.670628in}{0.844588in}}{\pgfqpoint{0.664804in}{0.844588in}}%
\pgfpathcurveto{\pgfqpoint{0.658980in}{0.844588in}}{\pgfqpoint{0.653394in}{0.842275in}}{\pgfqpoint{0.649276in}{0.838156in}}%
\pgfpathcurveto{\pgfqpoint{0.645157in}{0.834038in}}{\pgfqpoint{0.642844in}{0.828452in}}{\pgfqpoint{0.642844in}{0.822628in}}%
\pgfpathcurveto{\pgfqpoint{0.642844in}{0.816804in}}{\pgfqpoint{0.645157in}{0.811218in}}{\pgfqpoint{0.649276in}{0.807100in}}%
\pgfpathcurveto{\pgfqpoint{0.653394in}{0.802982in}}{\pgfqpoint{0.658980in}{0.800668in}}{\pgfqpoint{0.664804in}{0.800668in}}%
\pgfpathclose%
\pgfusepath{stroke,fill}%
\end{pgfscope}%
\begin{pgfscope}%
\pgfpathrectangle{\pgfqpoint{0.211875in}{0.211875in}}{\pgfqpoint{1.313625in}{1.279725in}}%
\pgfusepath{clip}%
\pgfsetbuttcap%
\pgfsetroundjoin%
\definecolor{currentfill}{rgb}{0.121569,0.466667,0.705882}%
\pgfsetfillcolor{currentfill}%
\pgfsetlinewidth{1.003750pt}%
\definecolor{currentstroke}{rgb}{0.121569,0.466667,0.705882}%
\pgfsetstrokecolor{currentstroke}%
\pgfsetdash{}{0pt}%
\pgfpathmoveto{\pgfqpoint{0.985022in}{0.311350in}}%
\pgfpathcurveto{\pgfqpoint{0.990846in}{0.311350in}}{\pgfqpoint{0.996432in}{0.313664in}}{\pgfqpoint{1.000551in}{0.317782in}}%
\pgfpathcurveto{\pgfqpoint{1.004669in}{0.321900in}}{\pgfqpoint{1.006983in}{0.327486in}}{\pgfqpoint{1.006983in}{0.333310in}}%
\pgfpathcurveto{\pgfqpoint{1.006983in}{0.339134in}}{\pgfqpoint{1.004669in}{0.344720in}}{\pgfqpoint{1.000551in}{0.348838in}}%
\pgfpathcurveto{\pgfqpoint{0.996432in}{0.352957in}}{\pgfqpoint{0.990846in}{0.355270in}}{\pgfqpoint{0.985022in}{0.355270in}}%
\pgfpathcurveto{\pgfqpoint{0.979198in}{0.355270in}}{\pgfqpoint{0.973612in}{0.352957in}}{\pgfqpoint{0.969494in}{0.348838in}}%
\pgfpathcurveto{\pgfqpoint{0.965376in}{0.344720in}}{\pgfqpoint{0.963062in}{0.339134in}}{\pgfqpoint{0.963062in}{0.333310in}}%
\pgfpathcurveto{\pgfqpoint{0.963062in}{0.327486in}}{\pgfqpoint{0.965376in}{0.321900in}}{\pgfqpoint{0.969494in}{0.317782in}}%
\pgfpathcurveto{\pgfqpoint{0.973612in}{0.313664in}}{\pgfqpoint{0.979198in}{0.311350in}}{\pgfqpoint{0.985022in}{0.311350in}}%
\pgfpathclose%
\pgfusepath{stroke,fill}%
\end{pgfscope}%
\begin{pgfscope}%
\pgfpathrectangle{\pgfqpoint{0.211875in}{0.211875in}}{\pgfqpoint{1.313625in}{1.279725in}}%
\pgfusepath{clip}%
\pgfsetbuttcap%
\pgfsetroundjoin%
\definecolor{currentfill}{rgb}{0.121569,0.466667,0.705882}%
\pgfsetfillcolor{currentfill}%
\pgfsetlinewidth{1.003750pt}%
\definecolor{currentstroke}{rgb}{0.121569,0.466667,0.705882}%
\pgfsetstrokecolor{currentstroke}%
\pgfsetdash{}{0pt}%
\pgfpathmoveto{\pgfqpoint{0.378797in}{0.725393in}}%
\pgfpathcurveto{\pgfqpoint{0.384621in}{0.725393in}}{\pgfqpoint{0.390207in}{0.727707in}}{\pgfqpoint{0.394326in}{0.731825in}}%
\pgfpathcurveto{\pgfqpoint{0.398444in}{0.735943in}}{\pgfqpoint{0.400758in}{0.741529in}}{\pgfqpoint{0.400758in}{0.747353in}}%
\pgfpathcurveto{\pgfqpoint{0.400758in}{0.753177in}}{\pgfqpoint{0.398444in}{0.758763in}}{\pgfqpoint{0.394326in}{0.762882in}}%
\pgfpathcurveto{\pgfqpoint{0.390207in}{0.767000in}}{\pgfqpoint{0.384621in}{0.769314in}}{\pgfqpoint{0.378797in}{0.769314in}}%
\pgfpathcurveto{\pgfqpoint{0.372973in}{0.769314in}}{\pgfqpoint{0.367387in}{0.767000in}}{\pgfqpoint{0.363269in}{0.762882in}}%
\pgfpathcurveto{\pgfqpoint{0.359151in}{0.758763in}}{\pgfqpoint{0.356837in}{0.753177in}}{\pgfqpoint{0.356837in}{0.747353in}}%
\pgfpathcurveto{\pgfqpoint{0.356837in}{0.741529in}}{\pgfqpoint{0.359151in}{0.735943in}}{\pgfqpoint{0.363269in}{0.731825in}}%
\pgfpathcurveto{\pgfqpoint{0.367387in}{0.727707in}}{\pgfqpoint{0.372973in}{0.725393in}}{\pgfqpoint{0.378797in}{0.725393in}}%
\pgfpathclose%
\pgfusepath{stroke,fill}%
\end{pgfscope}%
\begin{pgfscope}%
\pgfpathrectangle{\pgfqpoint{0.211875in}{0.211875in}}{\pgfqpoint{1.313625in}{1.279725in}}%
\pgfusepath{clip}%
\pgfsetbuttcap%
\pgfsetroundjoin%
\definecolor{currentfill}{rgb}{0.121569,0.466667,0.705882}%
\pgfsetfillcolor{currentfill}%
\pgfsetlinewidth{1.003750pt}%
\definecolor{currentstroke}{rgb}{0.121569,0.466667,0.705882}%
\pgfsetstrokecolor{currentstroke}%
\pgfsetdash{}{0pt}%
\pgfpathmoveto{\pgfqpoint{1.427505in}{0.948489in}}%
\pgfpathcurveto{\pgfqpoint{1.433329in}{0.948489in}}{\pgfqpoint{1.438915in}{0.950803in}}{\pgfqpoint{1.443033in}{0.954921in}}%
\pgfpathcurveto{\pgfqpoint{1.447151in}{0.959039in}}{\pgfqpoint{1.449465in}{0.964626in}}{\pgfqpoint{1.449465in}{0.970450in}}%
\pgfpathcurveto{\pgfqpoint{1.449465in}{0.976274in}}{\pgfqpoint{1.447151in}{0.981860in}}{\pgfqpoint{1.443033in}{0.985978in}}%
\pgfpathcurveto{\pgfqpoint{1.438915in}{0.990096in}}{\pgfqpoint{1.433329in}{0.992410in}}{\pgfqpoint{1.427505in}{0.992410in}}%
\pgfpathcurveto{\pgfqpoint{1.421681in}{0.992410in}}{\pgfqpoint{1.416095in}{0.990096in}}{\pgfqpoint{1.411977in}{0.985978in}}%
\pgfpathcurveto{\pgfqpoint{1.407859in}{0.981860in}}{\pgfqpoint{1.405545in}{0.976274in}}{\pgfqpoint{1.405545in}{0.970450in}}%
\pgfpathcurveto{\pgfqpoint{1.405545in}{0.964626in}}{\pgfqpoint{1.407859in}{0.959039in}}{\pgfqpoint{1.411977in}{0.954921in}}%
\pgfpathcurveto{\pgfqpoint{1.416095in}{0.950803in}}{\pgfqpoint{1.421681in}{0.948489in}}{\pgfqpoint{1.427505in}{0.948489in}}%
\pgfpathclose%
\pgfusepath{stroke,fill}%
\end{pgfscope}%
\begin{pgfscope}%
\pgfpathrectangle{\pgfqpoint{0.211875in}{0.211875in}}{\pgfqpoint{1.313625in}{1.279725in}}%
\pgfusepath{clip}%
\pgfsetbuttcap%
\pgfsetroundjoin%
\definecolor{currentfill}{rgb}{0.121569,0.466667,0.705882}%
\pgfsetfillcolor{currentfill}%
\pgfsetlinewidth{1.003750pt}%
\definecolor{currentstroke}{rgb}{0.121569,0.466667,0.705882}%
\pgfsetstrokecolor{currentstroke}%
\pgfsetdash{}{0pt}%
\pgfpathmoveto{\pgfqpoint{1.446373in}{0.926015in}}%
\pgfpathcurveto{\pgfqpoint{1.452197in}{0.926015in}}{\pgfqpoint{1.457783in}{0.928329in}}{\pgfqpoint{1.461901in}{0.932447in}}%
\pgfpathcurveto{\pgfqpoint{1.466019in}{0.936565in}}{\pgfqpoint{1.468333in}{0.942152in}}{\pgfqpoint{1.468333in}{0.947975in}}%
\pgfpathcurveto{\pgfqpoint{1.468333in}{0.953799in}}{\pgfqpoint{1.466019in}{0.959386in}}{\pgfqpoint{1.461901in}{0.963504in}}%
\pgfpathcurveto{\pgfqpoint{1.457783in}{0.967622in}}{\pgfqpoint{1.452197in}{0.969936in}}{\pgfqpoint{1.446373in}{0.969936in}}%
\pgfpathcurveto{\pgfqpoint{1.440549in}{0.969936in}}{\pgfqpoint{1.434963in}{0.967622in}}{\pgfqpoint{1.430845in}{0.963504in}}%
\pgfpathcurveto{\pgfqpoint{1.426726in}{0.959386in}}{\pgfqpoint{1.424413in}{0.953799in}}{\pgfqpoint{1.424413in}{0.947975in}}%
\pgfpathcurveto{\pgfqpoint{1.424413in}{0.942152in}}{\pgfqpoint{1.426726in}{0.936565in}}{\pgfqpoint{1.430845in}{0.932447in}}%
\pgfpathcurveto{\pgfqpoint{1.434963in}{0.928329in}}{\pgfqpoint{1.440549in}{0.926015in}}{\pgfqpoint{1.446373in}{0.926015in}}%
\pgfpathclose%
\pgfusepath{stroke,fill}%
\end{pgfscope}%
\begin{pgfscope}%
\pgfpathrectangle{\pgfqpoint{0.211875in}{0.211875in}}{\pgfqpoint{1.313625in}{1.279725in}}%
\pgfusepath{clip}%
\pgfsetbuttcap%
\pgfsetroundjoin%
\definecolor{currentfill}{rgb}{0.121569,0.466667,0.705882}%
\pgfsetfillcolor{currentfill}%
\pgfsetlinewidth{1.003750pt}%
\definecolor{currentstroke}{rgb}{0.121569,0.466667,0.705882}%
\pgfsetstrokecolor{currentstroke}%
\pgfsetdash{}{0pt}%
\pgfpathmoveto{\pgfqpoint{1.172672in}{1.043029in}}%
\pgfpathcurveto{\pgfqpoint{1.178496in}{1.043029in}}{\pgfqpoint{1.184082in}{1.045343in}}{\pgfqpoint{1.188200in}{1.049461in}}%
\pgfpathcurveto{\pgfqpoint{1.192318in}{1.053579in}}{\pgfqpoint{1.194632in}{1.059165in}}{\pgfqpoint{1.194632in}{1.064989in}}%
\pgfpathcurveto{\pgfqpoint{1.194632in}{1.070813in}}{\pgfqpoint{1.192318in}{1.076399in}}{\pgfqpoint{1.188200in}{1.080517in}}%
\pgfpathcurveto{\pgfqpoint{1.184082in}{1.084635in}}{\pgfqpoint{1.178496in}{1.086949in}}{\pgfqpoint{1.172672in}{1.086949in}}%
\pgfpathcurveto{\pgfqpoint{1.166848in}{1.086949in}}{\pgfqpoint{1.161262in}{1.084635in}}{\pgfqpoint{1.157144in}{1.080517in}}%
\pgfpathcurveto{\pgfqpoint{1.153025in}{1.076399in}}{\pgfqpoint{1.150712in}{1.070813in}}{\pgfqpoint{1.150712in}{1.064989in}}%
\pgfpathcurveto{\pgfqpoint{1.150712in}{1.059165in}}{\pgfqpoint{1.153025in}{1.053579in}}{\pgfqpoint{1.157144in}{1.049461in}}%
\pgfpathcurveto{\pgfqpoint{1.161262in}{1.045343in}}{\pgfqpoint{1.166848in}{1.043029in}}{\pgfqpoint{1.172672in}{1.043029in}}%
\pgfpathclose%
\pgfusepath{stroke,fill}%
\end{pgfscope}%
\begin{pgfscope}%
\pgfpathrectangle{\pgfqpoint{0.211875in}{0.211875in}}{\pgfqpoint{1.313625in}{1.279725in}}%
\pgfusepath{clip}%
\pgfsetbuttcap%
\pgfsetroundjoin%
\definecolor{currentfill}{rgb}{0.121569,0.466667,0.705882}%
\pgfsetfillcolor{currentfill}%
\pgfsetlinewidth{1.003750pt}%
\definecolor{currentstroke}{rgb}{0.121569,0.466667,0.705882}%
\pgfsetstrokecolor{currentstroke}%
\pgfsetdash{}{0pt}%
\pgfpathmoveto{\pgfqpoint{1.427366in}{1.060382in}}%
\pgfpathcurveto{\pgfqpoint{1.433190in}{1.060382in}}{\pgfqpoint{1.438776in}{1.062696in}}{\pgfqpoint{1.442895in}{1.066814in}}%
\pgfpathcurveto{\pgfqpoint{1.447013in}{1.070933in}}{\pgfqpoint{1.449327in}{1.076519in}}{\pgfqpoint{1.449327in}{1.082343in}}%
\pgfpathcurveto{\pgfqpoint{1.449327in}{1.088167in}}{\pgfqpoint{1.447013in}{1.093753in}}{\pgfqpoint{1.442895in}{1.097871in}}%
\pgfpathcurveto{\pgfqpoint{1.438776in}{1.101989in}}{\pgfqpoint{1.433190in}{1.104303in}}{\pgfqpoint{1.427366in}{1.104303in}}%
\pgfpathcurveto{\pgfqpoint{1.421542in}{1.104303in}}{\pgfqpoint{1.415956in}{1.101989in}}{\pgfqpoint{1.411838in}{1.097871in}}%
\pgfpathcurveto{\pgfqpoint{1.407720in}{1.093753in}}{\pgfqpoint{1.405406in}{1.088167in}}{\pgfqpoint{1.405406in}{1.082343in}}%
\pgfpathcurveto{\pgfqpoint{1.405406in}{1.076519in}}{\pgfqpoint{1.407720in}{1.070933in}}{\pgfqpoint{1.411838in}{1.066814in}}%
\pgfpathcurveto{\pgfqpoint{1.415956in}{1.062696in}}{\pgfqpoint{1.421542in}{1.060382in}}{\pgfqpoint{1.427366in}{1.060382in}}%
\pgfpathclose%
\pgfusepath{stroke,fill}%
\end{pgfscope}%
\begin{pgfscope}%
\pgfpathrectangle{\pgfqpoint{0.211875in}{0.211875in}}{\pgfqpoint{1.313625in}{1.279725in}}%
\pgfusepath{clip}%
\pgfsetbuttcap%
\pgfsetroundjoin%
\definecolor{currentfill}{rgb}{0.121569,0.466667,0.705882}%
\pgfsetfillcolor{currentfill}%
\pgfsetlinewidth{1.003750pt}%
\definecolor{currentstroke}{rgb}{0.121569,0.466667,0.705882}%
\pgfsetstrokecolor{currentstroke}%
\pgfsetdash{}{0pt}%
\pgfpathmoveto{\pgfqpoint{0.892839in}{0.758216in}}%
\pgfpathcurveto{\pgfqpoint{0.898663in}{0.758216in}}{\pgfqpoint{0.904249in}{0.760530in}}{\pgfqpoint{0.908368in}{0.764648in}}%
\pgfpathcurveto{\pgfqpoint{0.912486in}{0.768766in}}{\pgfqpoint{0.914800in}{0.774352in}}{\pgfqpoint{0.914800in}{0.780176in}}%
\pgfpathcurveto{\pgfqpoint{0.914800in}{0.786000in}}{\pgfqpoint{0.912486in}{0.791586in}}{\pgfqpoint{0.908368in}{0.795704in}}%
\pgfpathcurveto{\pgfqpoint{0.904249in}{0.799822in}}{\pgfqpoint{0.898663in}{0.802136in}}{\pgfqpoint{0.892839in}{0.802136in}}%
\pgfpathcurveto{\pgfqpoint{0.887015in}{0.802136in}}{\pgfqpoint{0.881429in}{0.799822in}}{\pgfqpoint{0.877311in}{0.795704in}}%
\pgfpathcurveto{\pgfqpoint{0.873193in}{0.791586in}}{\pgfqpoint{0.870879in}{0.786000in}}{\pgfqpoint{0.870879in}{0.780176in}}%
\pgfpathcurveto{\pgfqpoint{0.870879in}{0.774352in}}{\pgfqpoint{0.873193in}{0.768766in}}{\pgfqpoint{0.877311in}{0.764648in}}%
\pgfpathcurveto{\pgfqpoint{0.881429in}{0.760530in}}{\pgfqpoint{0.887015in}{0.758216in}}{\pgfqpoint{0.892839in}{0.758216in}}%
\pgfpathclose%
\pgfusepath{stroke,fill}%
\end{pgfscope}%
\begin{pgfscope}%
\pgfpathrectangle{\pgfqpoint{0.211875in}{0.211875in}}{\pgfqpoint{1.313625in}{1.279725in}}%
\pgfusepath{clip}%
\pgfsetbuttcap%
\pgfsetroundjoin%
\definecolor{currentfill}{rgb}{0.121569,0.466667,0.705882}%
\pgfsetfillcolor{currentfill}%
\pgfsetlinewidth{1.003750pt}%
\definecolor{currentstroke}{rgb}{0.121569,0.466667,0.705882}%
\pgfsetstrokecolor{currentstroke}%
\pgfsetdash{}{0pt}%
\pgfpathmoveto{\pgfqpoint{1.026366in}{0.556154in}}%
\pgfpathcurveto{\pgfqpoint{1.032190in}{0.556154in}}{\pgfqpoint{1.037776in}{0.558468in}}{\pgfqpoint{1.041894in}{0.562586in}}%
\pgfpathcurveto{\pgfqpoint{1.046012in}{0.566704in}}{\pgfqpoint{1.048326in}{0.572291in}}{\pgfqpoint{1.048326in}{0.578114in}}%
\pgfpathcurveto{\pgfqpoint{1.048326in}{0.583938in}}{\pgfqpoint{1.046012in}{0.589525in}}{\pgfqpoint{1.041894in}{0.593643in}}%
\pgfpathcurveto{\pgfqpoint{1.037776in}{0.597761in}}{\pgfqpoint{1.032190in}{0.600075in}}{\pgfqpoint{1.026366in}{0.600075in}}%
\pgfpathcurveto{\pgfqpoint{1.020542in}{0.600075in}}{\pgfqpoint{1.014956in}{0.597761in}}{\pgfqpoint{1.010838in}{0.593643in}}%
\pgfpathcurveto{\pgfqpoint{1.006719in}{0.589525in}}{\pgfqpoint{1.004406in}{0.583938in}}{\pgfqpoint{1.004406in}{0.578114in}}%
\pgfpathcurveto{\pgfqpoint{1.004406in}{0.572291in}}{\pgfqpoint{1.006719in}{0.566704in}}{\pgfqpoint{1.010838in}{0.562586in}}%
\pgfpathcurveto{\pgfqpoint{1.014956in}{0.558468in}}{\pgfqpoint{1.020542in}{0.556154in}}{\pgfqpoint{1.026366in}{0.556154in}}%
\pgfpathclose%
\pgfusepath{stroke,fill}%
\end{pgfscope}%
\begin{pgfscope}%
\pgfpathrectangle{\pgfqpoint{0.211875in}{0.211875in}}{\pgfqpoint{1.313625in}{1.279725in}}%
\pgfusepath{clip}%
\pgfsetbuttcap%
\pgfsetroundjoin%
\definecolor{currentfill}{rgb}{0.121569,0.466667,0.705882}%
\pgfsetfillcolor{currentfill}%
\pgfsetlinewidth{1.003750pt}%
\definecolor{currentstroke}{rgb}{0.121569,0.466667,0.705882}%
\pgfsetstrokecolor{currentstroke}%
\pgfsetdash{}{0pt}%
\pgfpathmoveto{\pgfqpoint{1.406661in}{1.059374in}}%
\pgfpathcurveto{\pgfqpoint{1.412485in}{1.059374in}}{\pgfqpoint{1.418071in}{1.061688in}}{\pgfqpoint{1.422189in}{1.065806in}}%
\pgfpathcurveto{\pgfqpoint{1.426307in}{1.069924in}}{\pgfqpoint{1.428621in}{1.075510in}}{\pgfqpoint{1.428621in}{1.081334in}}%
\pgfpathcurveto{\pgfqpoint{1.428621in}{1.087158in}}{\pgfqpoint{1.426307in}{1.092744in}}{\pgfqpoint{1.422189in}{1.096862in}}%
\pgfpathcurveto{\pgfqpoint{1.418071in}{1.100980in}}{\pgfqpoint{1.412485in}{1.103294in}}{\pgfqpoint{1.406661in}{1.103294in}}%
\pgfpathcurveto{\pgfqpoint{1.400837in}{1.103294in}}{\pgfqpoint{1.395251in}{1.100980in}}{\pgfqpoint{1.391133in}{1.096862in}}%
\pgfpathcurveto{\pgfqpoint{1.387015in}{1.092744in}}{\pgfqpoint{1.384701in}{1.087158in}}{\pgfqpoint{1.384701in}{1.081334in}}%
\pgfpathcurveto{\pgfqpoint{1.384701in}{1.075510in}}{\pgfqpoint{1.387015in}{1.069924in}}{\pgfqpoint{1.391133in}{1.065806in}}%
\pgfpathcurveto{\pgfqpoint{1.395251in}{1.061688in}}{\pgfqpoint{1.400837in}{1.059374in}}{\pgfqpoint{1.406661in}{1.059374in}}%
\pgfpathclose%
\pgfusepath{stroke,fill}%
\end{pgfscope}%
\begin{pgfscope}%
\pgfpathrectangle{\pgfqpoint{0.211875in}{0.211875in}}{\pgfqpoint{1.313625in}{1.279725in}}%
\pgfusepath{clip}%
\pgfsetbuttcap%
\pgfsetroundjoin%
\definecolor{currentfill}{rgb}{0.121569,0.466667,0.705882}%
\pgfsetfillcolor{currentfill}%
\pgfsetlinewidth{1.003750pt}%
\definecolor{currentstroke}{rgb}{0.121569,0.466667,0.705882}%
\pgfsetstrokecolor{currentstroke}%
\pgfsetdash{}{0pt}%
\pgfpathmoveto{\pgfqpoint{0.849941in}{0.770240in}}%
\pgfpathcurveto{\pgfqpoint{0.855765in}{0.770240in}}{\pgfqpoint{0.861351in}{0.772554in}}{\pgfqpoint{0.865469in}{0.776672in}}%
\pgfpathcurveto{\pgfqpoint{0.869587in}{0.780790in}}{\pgfqpoint{0.871901in}{0.786376in}}{\pgfqpoint{0.871901in}{0.792200in}}%
\pgfpathcurveto{\pgfqpoint{0.871901in}{0.798024in}}{\pgfqpoint{0.869587in}{0.803610in}}{\pgfqpoint{0.865469in}{0.807728in}}%
\pgfpathcurveto{\pgfqpoint{0.861351in}{0.811847in}}{\pgfqpoint{0.855765in}{0.814160in}}{\pgfqpoint{0.849941in}{0.814160in}}%
\pgfpathcurveto{\pgfqpoint{0.844117in}{0.814160in}}{\pgfqpoint{0.838531in}{0.811847in}}{\pgfqpoint{0.834413in}{0.807728in}}%
\pgfpathcurveto{\pgfqpoint{0.830295in}{0.803610in}}{\pgfqpoint{0.827981in}{0.798024in}}{\pgfqpoint{0.827981in}{0.792200in}}%
\pgfpathcurveto{\pgfqpoint{0.827981in}{0.786376in}}{\pgfqpoint{0.830295in}{0.780790in}}{\pgfqpoint{0.834413in}{0.776672in}}%
\pgfpathcurveto{\pgfqpoint{0.838531in}{0.772554in}}{\pgfqpoint{0.844117in}{0.770240in}}{\pgfqpoint{0.849941in}{0.770240in}}%
\pgfpathclose%
\pgfusepath{stroke,fill}%
\end{pgfscope}%
\begin{pgfscope}%
\pgfpathrectangle{\pgfqpoint{0.211875in}{0.211875in}}{\pgfqpoint{1.313625in}{1.279725in}}%
\pgfusepath{clip}%
\pgfsetbuttcap%
\pgfsetroundjoin%
\definecolor{currentfill}{rgb}{0.121569,0.466667,0.705882}%
\pgfsetfillcolor{currentfill}%
\pgfsetlinewidth{1.003750pt}%
\definecolor{currentstroke}{rgb}{0.121569,0.466667,0.705882}%
\pgfsetstrokecolor{currentstroke}%
\pgfsetdash{}{0pt}%
\pgfpathmoveto{\pgfqpoint{1.348610in}{1.346200in}}%
\pgfpathcurveto{\pgfqpoint{1.354434in}{1.346200in}}{\pgfqpoint{1.360020in}{1.348514in}}{\pgfqpoint{1.364138in}{1.352632in}}%
\pgfpathcurveto{\pgfqpoint{1.368256in}{1.356750in}}{\pgfqpoint{1.370570in}{1.362336in}}{\pgfqpoint{1.370570in}{1.368160in}}%
\pgfpathcurveto{\pgfqpoint{1.370570in}{1.373984in}}{\pgfqpoint{1.368256in}{1.379570in}}{\pgfqpoint{1.364138in}{1.383689in}}%
\pgfpathcurveto{\pgfqpoint{1.360020in}{1.387807in}}{\pgfqpoint{1.354434in}{1.390121in}}{\pgfqpoint{1.348610in}{1.390121in}}%
\pgfpathcurveto{\pgfqpoint{1.342786in}{1.390121in}}{\pgfqpoint{1.337200in}{1.387807in}}{\pgfqpoint{1.333081in}{1.383689in}}%
\pgfpathcurveto{\pgfqpoint{1.328963in}{1.379570in}}{\pgfqpoint{1.326649in}{1.373984in}}{\pgfqpoint{1.326649in}{1.368160in}}%
\pgfpathcurveto{\pgfqpoint{1.326649in}{1.362336in}}{\pgfqpoint{1.328963in}{1.356750in}}{\pgfqpoint{1.333081in}{1.352632in}}%
\pgfpathcurveto{\pgfqpoint{1.337200in}{1.348514in}}{\pgfqpoint{1.342786in}{1.346200in}}{\pgfqpoint{1.348610in}{1.346200in}}%
\pgfpathclose%
\pgfusepath{stroke,fill}%
\end{pgfscope}%
\begin{pgfscope}%
\pgfpathrectangle{\pgfqpoint{0.211875in}{0.211875in}}{\pgfqpoint{1.313625in}{1.279725in}}%
\pgfusepath{clip}%
\pgfsetbuttcap%
\pgfsetroundjoin%
\definecolor{currentfill}{rgb}{0.121569,0.466667,0.705882}%
\pgfsetfillcolor{currentfill}%
\pgfsetlinewidth{1.003750pt}%
\definecolor{currentstroke}{rgb}{0.121569,0.466667,0.705882}%
\pgfsetstrokecolor{currentstroke}%
\pgfsetdash{}{0pt}%
\pgfpathmoveto{\pgfqpoint{0.426769in}{1.372815in}}%
\pgfpathcurveto{\pgfqpoint{0.432593in}{1.372815in}}{\pgfqpoint{0.438179in}{1.375129in}}{\pgfqpoint{0.442298in}{1.379247in}}%
\pgfpathcurveto{\pgfqpoint{0.446416in}{1.383365in}}{\pgfqpoint{0.448730in}{1.388951in}}{\pgfqpoint{0.448730in}{1.394775in}}%
\pgfpathcurveto{\pgfqpoint{0.448730in}{1.400599in}}{\pgfqpoint{0.446416in}{1.406185in}}{\pgfqpoint{0.442298in}{1.410303in}}%
\pgfpathcurveto{\pgfqpoint{0.438179in}{1.414422in}}{\pgfqpoint{0.432593in}{1.416735in}}{\pgfqpoint{0.426769in}{1.416735in}}%
\pgfpathcurveto{\pgfqpoint{0.420945in}{1.416735in}}{\pgfqpoint{0.415359in}{1.414422in}}{\pgfqpoint{0.411241in}{1.410303in}}%
\pgfpathcurveto{\pgfqpoint{0.407123in}{1.406185in}}{\pgfqpoint{0.404809in}{1.400599in}}{\pgfqpoint{0.404809in}{1.394775in}}%
\pgfpathcurveto{\pgfqpoint{0.404809in}{1.388951in}}{\pgfqpoint{0.407123in}{1.383365in}}{\pgfqpoint{0.411241in}{1.379247in}}%
\pgfpathcurveto{\pgfqpoint{0.415359in}{1.375129in}}{\pgfqpoint{0.420945in}{1.372815in}}{\pgfqpoint{0.426769in}{1.372815in}}%
\pgfpathclose%
\pgfusepath{stroke,fill}%
\end{pgfscope}%
\begin{pgfscope}%
\pgfpathrectangle{\pgfqpoint{0.211875in}{0.211875in}}{\pgfqpoint{1.313625in}{1.279725in}}%
\pgfusepath{clip}%
\pgfsetbuttcap%
\pgfsetroundjoin%
\definecolor{currentfill}{rgb}{0.121569,0.466667,0.705882}%
\pgfsetfillcolor{currentfill}%
\pgfsetlinewidth{1.003750pt}%
\definecolor{currentstroke}{rgb}{0.121569,0.466667,0.705882}%
\pgfsetstrokecolor{currentstroke}%
\pgfsetdash{}{0pt}%
\pgfpathmoveto{\pgfqpoint{1.425713in}{1.019545in}}%
\pgfpathcurveto{\pgfqpoint{1.431537in}{1.019545in}}{\pgfqpoint{1.437123in}{1.021858in}}{\pgfqpoint{1.441242in}{1.025977in}}%
\pgfpathcurveto{\pgfqpoint{1.445360in}{1.030095in}}{\pgfqpoint{1.447674in}{1.035681in}}{\pgfqpoint{1.447674in}{1.041505in}}%
\pgfpathcurveto{\pgfqpoint{1.447674in}{1.047329in}}{\pgfqpoint{1.445360in}{1.052915in}}{\pgfqpoint{1.441242in}{1.057033in}}%
\pgfpathcurveto{\pgfqpoint{1.437123in}{1.061151in}}{\pgfqpoint{1.431537in}{1.063465in}}{\pgfqpoint{1.425713in}{1.063465in}}%
\pgfpathcurveto{\pgfqpoint{1.419889in}{1.063465in}}{\pgfqpoint{1.414303in}{1.061151in}}{\pgfqpoint{1.410185in}{1.057033in}}%
\pgfpathcurveto{\pgfqpoint{1.406067in}{1.052915in}}{\pgfqpoint{1.403753in}{1.047329in}}{\pgfqpoint{1.403753in}{1.041505in}}%
\pgfpathcurveto{\pgfqpoint{1.403753in}{1.035681in}}{\pgfqpoint{1.406067in}{1.030095in}}{\pgfqpoint{1.410185in}{1.025977in}}%
\pgfpathcurveto{\pgfqpoint{1.414303in}{1.021858in}}{\pgfqpoint{1.419889in}{1.019545in}}{\pgfqpoint{1.425713in}{1.019545in}}%
\pgfpathclose%
\pgfusepath{stroke,fill}%
\end{pgfscope}%
\begin{pgfscope}%
\pgfpathrectangle{\pgfqpoint{0.211875in}{0.211875in}}{\pgfqpoint{1.313625in}{1.279725in}}%
\pgfusepath{clip}%
\pgfsetbuttcap%
\pgfsetroundjoin%
\definecolor{currentfill}{rgb}{0.121569,0.466667,0.705882}%
\pgfsetfillcolor{currentfill}%
\pgfsetlinewidth{1.003750pt}%
\definecolor{currentstroke}{rgb}{0.121569,0.466667,0.705882}%
\pgfsetstrokecolor{currentstroke}%
\pgfsetdash{}{0pt}%
\pgfpathmoveto{\pgfqpoint{0.950457in}{0.744584in}}%
\pgfpathcurveto{\pgfqpoint{0.956281in}{0.744584in}}{\pgfqpoint{0.961867in}{0.746898in}}{\pgfqpoint{0.965985in}{0.751016in}}%
\pgfpathcurveto{\pgfqpoint{0.970103in}{0.755134in}}{\pgfqpoint{0.972417in}{0.760721in}}{\pgfqpoint{0.972417in}{0.766545in}}%
\pgfpathcurveto{\pgfqpoint{0.972417in}{0.772368in}}{\pgfqpoint{0.970103in}{0.777955in}}{\pgfqpoint{0.965985in}{0.782073in}}%
\pgfpathcurveto{\pgfqpoint{0.961867in}{0.786191in}}{\pgfqpoint{0.956281in}{0.788505in}}{\pgfqpoint{0.950457in}{0.788505in}}%
\pgfpathcurveto{\pgfqpoint{0.944633in}{0.788505in}}{\pgfqpoint{0.939047in}{0.786191in}}{\pgfqpoint{0.934928in}{0.782073in}}%
\pgfpathcurveto{\pgfqpoint{0.930810in}{0.777955in}}{\pgfqpoint{0.928496in}{0.772368in}}{\pgfqpoint{0.928496in}{0.766545in}}%
\pgfpathcurveto{\pgfqpoint{0.928496in}{0.760721in}}{\pgfqpoint{0.930810in}{0.755134in}}{\pgfqpoint{0.934928in}{0.751016in}}%
\pgfpathcurveto{\pgfqpoint{0.939047in}{0.746898in}}{\pgfqpoint{0.944633in}{0.744584in}}{\pgfqpoint{0.950457in}{0.744584in}}%
\pgfpathclose%
\pgfusepath{stroke,fill}%
\end{pgfscope}%
\begin{pgfscope}%
\pgfpathrectangle{\pgfqpoint{0.211875in}{0.211875in}}{\pgfqpoint{1.313625in}{1.279725in}}%
\pgfusepath{clip}%
\pgfsetbuttcap%
\pgfsetroundjoin%
\definecolor{currentfill}{rgb}{0.121569,0.466667,0.705882}%
\pgfsetfillcolor{currentfill}%
\pgfsetlinewidth{1.003750pt}%
\definecolor{currentstroke}{rgb}{0.121569,0.466667,0.705882}%
\pgfsetstrokecolor{currentstroke}%
\pgfsetdash{}{0pt}%
\pgfpathmoveto{\pgfqpoint{0.995831in}{1.034183in}}%
\pgfpathcurveto{\pgfqpoint{1.001654in}{1.034183in}}{\pgfqpoint{1.007241in}{1.036497in}}{\pgfqpoint{1.011359in}{1.040615in}}%
\pgfpathcurveto{\pgfqpoint{1.015477in}{1.044733in}}{\pgfqpoint{1.017791in}{1.050319in}}{\pgfqpoint{1.017791in}{1.056143in}}%
\pgfpathcurveto{\pgfqpoint{1.017791in}{1.061967in}}{\pgfqpoint{1.015477in}{1.067553in}}{\pgfqpoint{1.011359in}{1.071671in}}%
\pgfpathcurveto{\pgfqpoint{1.007241in}{1.075790in}}{\pgfqpoint{1.001654in}{1.078103in}}{\pgfqpoint{0.995831in}{1.078103in}}%
\pgfpathcurveto{\pgfqpoint{0.990007in}{1.078103in}}{\pgfqpoint{0.984420in}{1.075790in}}{\pgfqpoint{0.980302in}{1.071671in}}%
\pgfpathcurveto{\pgfqpoint{0.976184in}{1.067553in}}{\pgfqpoint{0.973870in}{1.061967in}}{\pgfqpoint{0.973870in}{1.056143in}}%
\pgfpathcurveto{\pgfqpoint{0.973870in}{1.050319in}}{\pgfqpoint{0.976184in}{1.044733in}}{\pgfqpoint{0.980302in}{1.040615in}}%
\pgfpathcurveto{\pgfqpoint{0.984420in}{1.036497in}}{\pgfqpoint{0.990007in}{1.034183in}}{\pgfqpoint{0.995831in}{1.034183in}}%
\pgfpathclose%
\pgfusepath{stroke,fill}%
\end{pgfscope}%
\begin{pgfscope}%
\pgfpathrectangle{\pgfqpoint{0.211875in}{0.211875in}}{\pgfqpoint{1.313625in}{1.279725in}}%
\pgfusepath{clip}%
\pgfsetbuttcap%
\pgfsetroundjoin%
\definecolor{currentfill}{rgb}{0.121569,0.466667,0.705882}%
\pgfsetfillcolor{currentfill}%
\pgfsetlinewidth{1.003750pt}%
\definecolor{currentstroke}{rgb}{0.121569,0.466667,0.705882}%
\pgfsetstrokecolor{currentstroke}%
\pgfsetdash{}{0pt}%
\pgfpathmoveto{\pgfqpoint{1.305996in}{0.926187in}}%
\pgfpathcurveto{\pgfqpoint{1.311820in}{0.926187in}}{\pgfqpoint{1.317406in}{0.928501in}}{\pgfqpoint{1.321524in}{0.932619in}}%
\pgfpathcurveto{\pgfqpoint{1.325642in}{0.936737in}}{\pgfqpoint{1.327956in}{0.942324in}}{\pgfqpoint{1.327956in}{0.948147in}}%
\pgfpathcurveto{\pgfqpoint{1.327956in}{0.953971in}}{\pgfqpoint{1.325642in}{0.959558in}}{\pgfqpoint{1.321524in}{0.963676in}}%
\pgfpathcurveto{\pgfqpoint{1.317406in}{0.967794in}}{\pgfqpoint{1.311820in}{0.970108in}}{\pgfqpoint{1.305996in}{0.970108in}}%
\pgfpathcurveto{\pgfqpoint{1.300172in}{0.970108in}}{\pgfqpoint{1.294586in}{0.967794in}}{\pgfqpoint{1.290468in}{0.963676in}}%
\pgfpathcurveto{\pgfqpoint{1.286349in}{0.959558in}}{\pgfqpoint{1.284036in}{0.953971in}}{\pgfqpoint{1.284036in}{0.948147in}}%
\pgfpathcurveto{\pgfqpoint{1.284036in}{0.942324in}}{\pgfqpoint{1.286349in}{0.936737in}}{\pgfqpoint{1.290468in}{0.932619in}}%
\pgfpathcurveto{\pgfqpoint{1.294586in}{0.928501in}}{\pgfqpoint{1.300172in}{0.926187in}}{\pgfqpoint{1.305996in}{0.926187in}}%
\pgfpathclose%
\pgfusepath{stroke,fill}%
\end{pgfscope}%
\begin{pgfscope}%
\pgfpathrectangle{\pgfqpoint{0.211875in}{0.211875in}}{\pgfqpoint{1.313625in}{1.279725in}}%
\pgfusepath{clip}%
\pgfsetbuttcap%
\pgfsetroundjoin%
\definecolor{currentfill}{rgb}{0.121569,0.466667,0.705882}%
\pgfsetfillcolor{currentfill}%
\pgfsetlinewidth{1.003750pt}%
\definecolor{currentstroke}{rgb}{0.121569,0.466667,0.705882}%
\pgfsetstrokecolor{currentstroke}%
\pgfsetdash{}{0pt}%
\pgfpathmoveto{\pgfqpoint{1.443746in}{1.280120in}}%
\pgfpathcurveto{\pgfqpoint{1.449570in}{1.280120in}}{\pgfqpoint{1.455156in}{1.282434in}}{\pgfqpoint{1.459274in}{1.286552in}}%
\pgfpathcurveto{\pgfqpoint{1.463392in}{1.290670in}}{\pgfqpoint{1.465706in}{1.296256in}}{\pgfqpoint{1.465706in}{1.302080in}}%
\pgfpathcurveto{\pgfqpoint{1.465706in}{1.307904in}}{\pgfqpoint{1.463392in}{1.313491in}}{\pgfqpoint{1.459274in}{1.317609in}}%
\pgfpathcurveto{\pgfqpoint{1.455156in}{1.321727in}}{\pgfqpoint{1.449570in}{1.324041in}}{\pgfqpoint{1.443746in}{1.324041in}}%
\pgfpathcurveto{\pgfqpoint{1.437922in}{1.324041in}}{\pgfqpoint{1.432336in}{1.321727in}}{\pgfqpoint{1.428218in}{1.317609in}}%
\pgfpathcurveto{\pgfqpoint{1.424100in}{1.313491in}}{\pgfqpoint{1.421786in}{1.307904in}}{\pgfqpoint{1.421786in}{1.302080in}}%
\pgfpathcurveto{\pgfqpoint{1.421786in}{1.296256in}}{\pgfqpoint{1.424100in}{1.290670in}}{\pgfqpoint{1.428218in}{1.286552in}}%
\pgfpathcurveto{\pgfqpoint{1.432336in}{1.282434in}}{\pgfqpoint{1.437922in}{1.280120in}}{\pgfqpoint{1.443746in}{1.280120in}}%
\pgfpathclose%
\pgfusepath{stroke,fill}%
\end{pgfscope}%
\begin{pgfscope}%
\pgfpathrectangle{\pgfqpoint{0.211875in}{0.211875in}}{\pgfqpoint{1.313625in}{1.279725in}}%
\pgfusepath{clip}%
\pgfsetbuttcap%
\pgfsetroundjoin%
\definecolor{currentfill}{rgb}{0.121569,0.466667,0.705882}%
\pgfsetfillcolor{currentfill}%
\pgfsetlinewidth{1.003750pt}%
\definecolor{currentstroke}{rgb}{0.121569,0.466667,0.705882}%
\pgfsetstrokecolor{currentstroke}%
\pgfsetdash{}{0pt}%
\pgfpathmoveto{\pgfqpoint{1.305050in}{0.894465in}}%
\pgfpathcurveto{\pgfqpoint{1.310874in}{0.894465in}}{\pgfqpoint{1.316460in}{0.896779in}}{\pgfqpoint{1.320578in}{0.900897in}}%
\pgfpathcurveto{\pgfqpoint{1.324696in}{0.905015in}}{\pgfqpoint{1.327010in}{0.910602in}}{\pgfqpoint{1.327010in}{0.916425in}}%
\pgfpathcurveto{\pgfqpoint{1.327010in}{0.922249in}}{\pgfqpoint{1.324696in}{0.927836in}}{\pgfqpoint{1.320578in}{0.931954in}}%
\pgfpathcurveto{\pgfqpoint{1.316460in}{0.936072in}}{\pgfqpoint{1.310874in}{0.938386in}}{\pgfqpoint{1.305050in}{0.938386in}}%
\pgfpathcurveto{\pgfqpoint{1.299226in}{0.938386in}}{\pgfqpoint{1.293640in}{0.936072in}}{\pgfqpoint{1.289522in}{0.931954in}}%
\pgfpathcurveto{\pgfqpoint{1.285404in}{0.927836in}}{\pgfqpoint{1.283090in}{0.922249in}}{\pgfqpoint{1.283090in}{0.916425in}}%
\pgfpathcurveto{\pgfqpoint{1.283090in}{0.910602in}}{\pgfqpoint{1.285404in}{0.905015in}}{\pgfqpoint{1.289522in}{0.900897in}}%
\pgfpathcurveto{\pgfqpoint{1.293640in}{0.896779in}}{\pgfqpoint{1.299226in}{0.894465in}}{\pgfqpoint{1.305050in}{0.894465in}}%
\pgfpathclose%
\pgfusepath{stroke,fill}%
\end{pgfscope}%
\begin{pgfscope}%
\pgfpathrectangle{\pgfqpoint{0.211875in}{0.211875in}}{\pgfqpoint{1.313625in}{1.279725in}}%
\pgfusepath{clip}%
\pgfsetbuttcap%
\pgfsetroundjoin%
\definecolor{currentfill}{rgb}{0.121569,0.466667,0.705882}%
\pgfsetfillcolor{currentfill}%
\pgfsetlinewidth{1.003750pt}%
\definecolor{currentstroke}{rgb}{0.121569,0.466667,0.705882}%
\pgfsetstrokecolor{currentstroke}%
\pgfsetdash{}{0pt}%
\pgfpathmoveto{\pgfqpoint{1.124319in}{1.070107in}}%
\pgfpathcurveto{\pgfqpoint{1.130143in}{1.070107in}}{\pgfqpoint{1.135729in}{1.072421in}}{\pgfqpoint{1.139847in}{1.076539in}}%
\pgfpathcurveto{\pgfqpoint{1.143966in}{1.080657in}}{\pgfqpoint{1.146279in}{1.086243in}}{\pgfqpoint{1.146279in}{1.092067in}}%
\pgfpathcurveto{\pgfqpoint{1.146279in}{1.097891in}}{\pgfqpoint{1.143966in}{1.103478in}}{\pgfqpoint{1.139847in}{1.107596in}}%
\pgfpathcurveto{\pgfqpoint{1.135729in}{1.111714in}}{\pgfqpoint{1.130143in}{1.114028in}}{\pgfqpoint{1.124319in}{1.114028in}}%
\pgfpathcurveto{\pgfqpoint{1.118495in}{1.114028in}}{\pgfqpoint{1.112909in}{1.111714in}}{\pgfqpoint{1.108791in}{1.107596in}}%
\pgfpathcurveto{\pgfqpoint{1.104673in}{1.103478in}}{\pgfqpoint{1.102359in}{1.097891in}}{\pgfqpoint{1.102359in}{1.092067in}}%
\pgfpathcurveto{\pgfqpoint{1.102359in}{1.086243in}}{\pgfqpoint{1.104673in}{1.080657in}}{\pgfqpoint{1.108791in}{1.076539in}}%
\pgfpathcurveto{\pgfqpoint{1.112909in}{1.072421in}}{\pgfqpoint{1.118495in}{1.070107in}}{\pgfqpoint{1.124319in}{1.070107in}}%
\pgfpathclose%
\pgfusepath{stroke,fill}%
\end{pgfscope}%
\begin{pgfscope}%
\pgfpathrectangle{\pgfqpoint{0.211875in}{0.211875in}}{\pgfqpoint{1.313625in}{1.279725in}}%
\pgfusepath{clip}%
\pgfsetbuttcap%
\pgfsetroundjoin%
\definecolor{currentfill}{rgb}{0.121569,0.466667,0.705882}%
\pgfsetfillcolor{currentfill}%
\pgfsetlinewidth{1.003750pt}%
\definecolor{currentstroke}{rgb}{0.121569,0.466667,0.705882}%
\pgfsetstrokecolor{currentstroke}%
\pgfsetdash{}{0pt}%
\pgfpathmoveto{\pgfqpoint{1.109875in}{1.080735in}}%
\pgfpathcurveto{\pgfqpoint{1.115699in}{1.080735in}}{\pgfqpoint{1.121285in}{1.083049in}}{\pgfqpoint{1.125404in}{1.087167in}}%
\pgfpathcurveto{\pgfqpoint{1.129522in}{1.091285in}}{\pgfqpoint{1.131836in}{1.096871in}}{\pgfqpoint{1.131836in}{1.102695in}}%
\pgfpathcurveto{\pgfqpoint{1.131836in}{1.108519in}}{\pgfqpoint{1.129522in}{1.114105in}}{\pgfqpoint{1.125404in}{1.118224in}}%
\pgfpathcurveto{\pgfqpoint{1.121285in}{1.122342in}}{\pgfqpoint{1.115699in}{1.124656in}}{\pgfqpoint{1.109875in}{1.124656in}}%
\pgfpathcurveto{\pgfqpoint{1.104051in}{1.124656in}}{\pgfqpoint{1.098465in}{1.122342in}}{\pgfqpoint{1.094347in}{1.118224in}}%
\pgfpathcurveto{\pgfqpoint{1.090229in}{1.114105in}}{\pgfqpoint{1.087915in}{1.108519in}}{\pgfqpoint{1.087915in}{1.102695in}}%
\pgfpathcurveto{\pgfqpoint{1.087915in}{1.096871in}}{\pgfqpoint{1.090229in}{1.091285in}}{\pgfqpoint{1.094347in}{1.087167in}}%
\pgfpathcurveto{\pgfqpoint{1.098465in}{1.083049in}}{\pgfqpoint{1.104051in}{1.080735in}}{\pgfqpoint{1.109875in}{1.080735in}}%
\pgfpathclose%
\pgfusepath{stroke,fill}%
\end{pgfscope}%
\begin{pgfscope}%
\pgfpathrectangle{\pgfqpoint{0.211875in}{0.211875in}}{\pgfqpoint{1.313625in}{1.279725in}}%
\pgfusepath{clip}%
\pgfsetbuttcap%
\pgfsetroundjoin%
\definecolor{currentfill}{rgb}{0.121569,0.466667,0.705882}%
\pgfsetfillcolor{currentfill}%
\pgfsetlinewidth{1.003750pt}%
\definecolor{currentstroke}{rgb}{0.121569,0.466667,0.705882}%
\pgfsetstrokecolor{currentstroke}%
\pgfsetdash{}{0pt}%
\pgfpathmoveto{\pgfqpoint{1.139421in}{1.097484in}}%
\pgfpathcurveto{\pgfqpoint{1.145245in}{1.097484in}}{\pgfqpoint{1.150831in}{1.099798in}}{\pgfqpoint{1.154949in}{1.103916in}}%
\pgfpathcurveto{\pgfqpoint{1.159068in}{1.108034in}}{\pgfqpoint{1.161381in}{1.113620in}}{\pgfqpoint{1.161381in}{1.119444in}}%
\pgfpathcurveto{\pgfqpoint{1.161381in}{1.125268in}}{\pgfqpoint{1.159068in}{1.130854in}}{\pgfqpoint{1.154949in}{1.134973in}}%
\pgfpathcurveto{\pgfqpoint{1.150831in}{1.139091in}}{\pgfqpoint{1.145245in}{1.141405in}}{\pgfqpoint{1.139421in}{1.141405in}}%
\pgfpathcurveto{\pgfqpoint{1.133597in}{1.141405in}}{\pgfqpoint{1.128011in}{1.139091in}}{\pgfqpoint{1.123893in}{1.134973in}}%
\pgfpathcurveto{\pgfqpoint{1.119775in}{1.130854in}}{\pgfqpoint{1.117461in}{1.125268in}}{\pgfqpoint{1.117461in}{1.119444in}}%
\pgfpathcurveto{\pgfqpoint{1.117461in}{1.113620in}}{\pgfqpoint{1.119775in}{1.108034in}}{\pgfqpoint{1.123893in}{1.103916in}}%
\pgfpathcurveto{\pgfqpoint{1.128011in}{1.099798in}}{\pgfqpoint{1.133597in}{1.097484in}}{\pgfqpoint{1.139421in}{1.097484in}}%
\pgfpathclose%
\pgfusepath{stroke,fill}%
\end{pgfscope}%
\begin{pgfscope}%
\pgfpathrectangle{\pgfqpoint{0.211875in}{0.211875in}}{\pgfqpoint{1.313625in}{1.279725in}}%
\pgfusepath{clip}%
\pgfsetbuttcap%
\pgfsetroundjoin%
\definecolor{currentfill}{rgb}{0.121569,0.466667,0.705882}%
\pgfsetfillcolor{currentfill}%
\pgfsetlinewidth{1.003750pt}%
\definecolor{currentstroke}{rgb}{0.121569,0.466667,0.705882}%
\pgfsetstrokecolor{currentstroke}%
\pgfsetdash{}{0pt}%
\pgfpathmoveto{\pgfqpoint{1.374930in}{1.119096in}}%
\pgfpathcurveto{\pgfqpoint{1.380754in}{1.119096in}}{\pgfqpoint{1.386340in}{1.121410in}}{\pgfqpoint{1.390458in}{1.125528in}}%
\pgfpathcurveto{\pgfqpoint{1.394576in}{1.129646in}}{\pgfqpoint{1.396890in}{1.135233in}}{\pgfqpoint{1.396890in}{1.141056in}}%
\pgfpathcurveto{\pgfqpoint{1.396890in}{1.146880in}}{\pgfqpoint{1.394576in}{1.152467in}}{\pgfqpoint{1.390458in}{1.156585in}}%
\pgfpathcurveto{\pgfqpoint{1.386340in}{1.160703in}}{\pgfqpoint{1.380754in}{1.163017in}}{\pgfqpoint{1.374930in}{1.163017in}}%
\pgfpathcurveto{\pgfqpoint{1.369106in}{1.163017in}}{\pgfqpoint{1.363520in}{1.160703in}}{\pgfqpoint{1.359401in}{1.156585in}}%
\pgfpathcurveto{\pgfqpoint{1.355283in}{1.152467in}}{\pgfqpoint{1.352969in}{1.146880in}}{\pgfqpoint{1.352969in}{1.141056in}}%
\pgfpathcurveto{\pgfqpoint{1.352969in}{1.135233in}}{\pgfqpoint{1.355283in}{1.129646in}}{\pgfqpoint{1.359401in}{1.125528in}}%
\pgfpathcurveto{\pgfqpoint{1.363520in}{1.121410in}}{\pgfqpoint{1.369106in}{1.119096in}}{\pgfqpoint{1.374930in}{1.119096in}}%
\pgfpathclose%
\pgfusepath{stroke,fill}%
\end{pgfscope}%
\begin{pgfscope}%
\pgfpathrectangle{\pgfqpoint{0.211875in}{0.211875in}}{\pgfqpoint{1.313625in}{1.279725in}}%
\pgfusepath{clip}%
\pgfsetbuttcap%
\pgfsetroundjoin%
\definecolor{currentfill}{rgb}{0.121569,0.466667,0.705882}%
\pgfsetfillcolor{currentfill}%
\pgfsetlinewidth{1.003750pt}%
\definecolor{currentstroke}{rgb}{0.121569,0.466667,0.705882}%
\pgfsetstrokecolor{currentstroke}%
\pgfsetdash{}{0pt}%
\pgfpathmoveto{\pgfqpoint{0.820542in}{0.867639in}}%
\pgfpathcurveto{\pgfqpoint{0.826366in}{0.867639in}}{\pgfqpoint{0.831952in}{0.869953in}}{\pgfqpoint{0.836070in}{0.874071in}}%
\pgfpathcurveto{\pgfqpoint{0.840189in}{0.878190in}}{\pgfqpoint{0.842502in}{0.883776in}}{\pgfqpoint{0.842502in}{0.889600in}}%
\pgfpathcurveto{\pgfqpoint{0.842502in}{0.895424in}}{\pgfqpoint{0.840189in}{0.901010in}}{\pgfqpoint{0.836070in}{0.905128in}}%
\pgfpathcurveto{\pgfqpoint{0.831952in}{0.909246in}}{\pgfqpoint{0.826366in}{0.911560in}}{\pgfqpoint{0.820542in}{0.911560in}}%
\pgfpathcurveto{\pgfqpoint{0.814718in}{0.911560in}}{\pgfqpoint{0.809132in}{0.909246in}}{\pgfqpoint{0.805014in}{0.905128in}}%
\pgfpathcurveto{\pgfqpoint{0.800896in}{0.901010in}}{\pgfqpoint{0.798582in}{0.895424in}}{\pgfqpoint{0.798582in}{0.889600in}}%
\pgfpathcurveto{\pgfqpoint{0.798582in}{0.883776in}}{\pgfqpoint{0.800896in}{0.878190in}}{\pgfqpoint{0.805014in}{0.874071in}}%
\pgfpathcurveto{\pgfqpoint{0.809132in}{0.869953in}}{\pgfqpoint{0.814718in}{0.867639in}}{\pgfqpoint{0.820542in}{0.867639in}}%
\pgfpathclose%
\pgfusepath{stroke,fill}%
\end{pgfscope}%
\begin{pgfscope}%
\pgfpathrectangle{\pgfqpoint{0.211875in}{0.211875in}}{\pgfqpoint{1.313625in}{1.279725in}}%
\pgfusepath{clip}%
\pgfsetbuttcap%
\pgfsetroundjoin%
\definecolor{currentfill}{rgb}{0.121569,0.466667,0.705882}%
\pgfsetfillcolor{currentfill}%
\pgfsetlinewidth{1.003750pt}%
\definecolor{currentstroke}{rgb}{0.121569,0.466667,0.705882}%
\pgfsetstrokecolor{currentstroke}%
\pgfsetdash{}{0pt}%
\pgfpathmoveto{\pgfqpoint{1.045267in}{0.923643in}}%
\pgfpathcurveto{\pgfqpoint{1.051091in}{0.923643in}}{\pgfqpoint{1.056677in}{0.925957in}}{\pgfqpoint{1.060795in}{0.930075in}}%
\pgfpathcurveto{\pgfqpoint{1.064913in}{0.934194in}}{\pgfqpoint{1.067227in}{0.939780in}}{\pgfqpoint{1.067227in}{0.945604in}}%
\pgfpathcurveto{\pgfqpoint{1.067227in}{0.951428in}}{\pgfqpoint{1.064913in}{0.957014in}}{\pgfqpoint{1.060795in}{0.961132in}}%
\pgfpathcurveto{\pgfqpoint{1.056677in}{0.965250in}}{\pgfqpoint{1.051091in}{0.967564in}}{\pgfqpoint{1.045267in}{0.967564in}}%
\pgfpathcurveto{\pgfqpoint{1.039443in}{0.967564in}}{\pgfqpoint{1.033857in}{0.965250in}}{\pgfqpoint{1.029738in}{0.961132in}}%
\pgfpathcurveto{\pgfqpoint{1.025620in}{0.957014in}}{\pgfqpoint{1.023306in}{0.951428in}}{\pgfqpoint{1.023306in}{0.945604in}}%
\pgfpathcurveto{\pgfqpoint{1.023306in}{0.939780in}}{\pgfqpoint{1.025620in}{0.934194in}}{\pgfqpoint{1.029738in}{0.930075in}}%
\pgfpathcurveto{\pgfqpoint{1.033857in}{0.925957in}}{\pgfqpoint{1.039443in}{0.923643in}}{\pgfqpoint{1.045267in}{0.923643in}}%
\pgfpathclose%
\pgfusepath{stroke,fill}%
\end{pgfscope}%
\begin{pgfscope}%
\pgfpathrectangle{\pgfqpoint{0.211875in}{0.211875in}}{\pgfqpoint{1.313625in}{1.279725in}}%
\pgfusepath{clip}%
\pgfsetbuttcap%
\pgfsetroundjoin%
\definecolor{currentfill}{rgb}{0.121569,0.466667,0.705882}%
\pgfsetfillcolor{currentfill}%
\pgfsetlinewidth{1.003750pt}%
\definecolor{currentstroke}{rgb}{0.121569,0.466667,0.705882}%
\pgfsetstrokecolor{currentstroke}%
\pgfsetdash{}{0pt}%
\pgfpathmoveto{\pgfqpoint{1.442343in}{1.169215in}}%
\pgfpathcurveto{\pgfqpoint{1.448167in}{1.169215in}}{\pgfqpoint{1.453753in}{1.171529in}}{\pgfqpoint{1.457871in}{1.175647in}}%
\pgfpathcurveto{\pgfqpoint{1.461989in}{1.179765in}}{\pgfqpoint{1.464303in}{1.185351in}}{\pgfqpoint{1.464303in}{1.191175in}}%
\pgfpathcurveto{\pgfqpoint{1.464303in}{1.196999in}}{\pgfqpoint{1.461989in}{1.202585in}}{\pgfqpoint{1.457871in}{1.206703in}}%
\pgfpathcurveto{\pgfqpoint{1.453753in}{1.210822in}}{\pgfqpoint{1.448167in}{1.213135in}}{\pgfqpoint{1.442343in}{1.213135in}}%
\pgfpathcurveto{\pgfqpoint{1.436519in}{1.213135in}}{\pgfqpoint{1.430933in}{1.210822in}}{\pgfqpoint{1.426815in}{1.206703in}}%
\pgfpathcurveto{\pgfqpoint{1.422697in}{1.202585in}}{\pgfqpoint{1.420383in}{1.196999in}}{\pgfqpoint{1.420383in}{1.191175in}}%
\pgfpathcurveto{\pgfqpoint{1.420383in}{1.185351in}}{\pgfqpoint{1.422697in}{1.179765in}}{\pgfqpoint{1.426815in}{1.175647in}}%
\pgfpathcurveto{\pgfqpoint{1.430933in}{1.171529in}}{\pgfqpoint{1.436519in}{1.169215in}}{\pgfqpoint{1.442343in}{1.169215in}}%
\pgfpathclose%
\pgfusepath{stroke,fill}%
\end{pgfscope}%
\begin{pgfscope}%
\pgfpathrectangle{\pgfqpoint{0.211875in}{0.211875in}}{\pgfqpoint{1.313625in}{1.279725in}}%
\pgfusepath{clip}%
\pgfsetbuttcap%
\pgfsetroundjoin%
\definecolor{currentfill}{rgb}{0.121569,0.466667,0.705882}%
\pgfsetfillcolor{currentfill}%
\pgfsetlinewidth{1.003750pt}%
\definecolor{currentstroke}{rgb}{0.121569,0.466667,0.705882}%
\pgfsetstrokecolor{currentstroke}%
\pgfsetdash{}{0pt}%
\pgfpathmoveto{\pgfqpoint{1.386789in}{1.107176in}}%
\pgfpathcurveto{\pgfqpoint{1.392613in}{1.107176in}}{\pgfqpoint{1.398199in}{1.109490in}}{\pgfqpoint{1.402317in}{1.113608in}}%
\pgfpathcurveto{\pgfqpoint{1.406435in}{1.117726in}}{\pgfqpoint{1.408749in}{1.123312in}}{\pgfqpoint{1.408749in}{1.129136in}}%
\pgfpathcurveto{\pgfqpoint{1.408749in}{1.134960in}}{\pgfqpoint{1.406435in}{1.140546in}}{\pgfqpoint{1.402317in}{1.144664in}}%
\pgfpathcurveto{\pgfqpoint{1.398199in}{1.148782in}}{\pgfqpoint{1.392613in}{1.151096in}}{\pgfqpoint{1.386789in}{1.151096in}}%
\pgfpathcurveto{\pgfqpoint{1.380965in}{1.151096in}}{\pgfqpoint{1.375379in}{1.148782in}}{\pgfqpoint{1.371261in}{1.144664in}}%
\pgfpathcurveto{\pgfqpoint{1.367143in}{1.140546in}}{\pgfqpoint{1.364829in}{1.134960in}}{\pgfqpoint{1.364829in}{1.129136in}}%
\pgfpathcurveto{\pgfqpoint{1.364829in}{1.123312in}}{\pgfqpoint{1.367143in}{1.117726in}}{\pgfqpoint{1.371261in}{1.113608in}}%
\pgfpathcurveto{\pgfqpoint{1.375379in}{1.109490in}}{\pgfqpoint{1.380965in}{1.107176in}}{\pgfqpoint{1.386789in}{1.107176in}}%
\pgfpathclose%
\pgfusepath{stroke,fill}%
\end{pgfscope}%
\begin{pgfscope}%
\pgfpathrectangle{\pgfqpoint{0.211875in}{0.211875in}}{\pgfqpoint{1.313625in}{1.279725in}}%
\pgfusepath{clip}%
\pgfsetbuttcap%
\pgfsetroundjoin%
\definecolor{currentfill}{rgb}{0.121569,0.466667,0.705882}%
\pgfsetfillcolor{currentfill}%
\pgfsetlinewidth{1.003750pt}%
\definecolor{currentstroke}{rgb}{0.121569,0.466667,0.705882}%
\pgfsetstrokecolor{currentstroke}%
\pgfsetdash{}{0pt}%
\pgfpathmoveto{\pgfqpoint{1.045015in}{1.027703in}}%
\pgfpathcurveto{\pgfqpoint{1.050839in}{1.027703in}}{\pgfqpoint{1.056425in}{1.030017in}}{\pgfqpoint{1.060543in}{1.034135in}}%
\pgfpathcurveto{\pgfqpoint{1.064661in}{1.038253in}}{\pgfqpoint{1.066975in}{1.043839in}}{\pgfqpoint{1.066975in}{1.049663in}}%
\pgfpathcurveto{\pgfqpoint{1.066975in}{1.055487in}}{\pgfqpoint{1.064661in}{1.061073in}}{\pgfqpoint{1.060543in}{1.065191in}}%
\pgfpathcurveto{\pgfqpoint{1.056425in}{1.069310in}}{\pgfqpoint{1.050839in}{1.071623in}}{\pgfqpoint{1.045015in}{1.071623in}}%
\pgfpathcurveto{\pgfqpoint{1.039191in}{1.071623in}}{\pgfqpoint{1.033605in}{1.069310in}}{\pgfqpoint{1.029487in}{1.065191in}}%
\pgfpathcurveto{\pgfqpoint{1.025368in}{1.061073in}}{\pgfqpoint{1.023055in}{1.055487in}}{\pgfqpoint{1.023055in}{1.049663in}}%
\pgfpathcurveto{\pgfqpoint{1.023055in}{1.043839in}}{\pgfqpoint{1.025368in}{1.038253in}}{\pgfqpoint{1.029487in}{1.034135in}}%
\pgfpathcurveto{\pgfqpoint{1.033605in}{1.030017in}}{\pgfqpoint{1.039191in}{1.027703in}}{\pgfqpoint{1.045015in}{1.027703in}}%
\pgfpathclose%
\pgfusepath{stroke,fill}%
\end{pgfscope}%
\begin{pgfscope}%
\pgfpathrectangle{\pgfqpoint{0.211875in}{0.211875in}}{\pgfqpoint{1.313625in}{1.279725in}}%
\pgfusepath{clip}%
\pgfsetbuttcap%
\pgfsetroundjoin%
\definecolor{currentfill}{rgb}{0.121569,0.466667,0.705882}%
\pgfsetfillcolor{currentfill}%
\pgfsetlinewidth{1.003750pt}%
\definecolor{currentstroke}{rgb}{0.121569,0.466667,0.705882}%
\pgfsetstrokecolor{currentstroke}%
\pgfsetdash{}{0pt}%
\pgfpathmoveto{\pgfqpoint{1.157886in}{1.183819in}}%
\pgfpathcurveto{\pgfqpoint{1.163710in}{1.183819in}}{\pgfqpoint{1.169296in}{1.186133in}}{\pgfqpoint{1.173414in}{1.190251in}}%
\pgfpathcurveto{\pgfqpoint{1.177532in}{1.194369in}}{\pgfqpoint{1.179846in}{1.199956in}}{\pgfqpoint{1.179846in}{1.205780in}}%
\pgfpathcurveto{\pgfqpoint{1.179846in}{1.211603in}}{\pgfqpoint{1.177532in}{1.217190in}}{\pgfqpoint{1.173414in}{1.221308in}}%
\pgfpathcurveto{\pgfqpoint{1.169296in}{1.225426in}}{\pgfqpoint{1.163710in}{1.227740in}}{\pgfqpoint{1.157886in}{1.227740in}}%
\pgfpathcurveto{\pgfqpoint{1.152062in}{1.227740in}}{\pgfqpoint{1.146476in}{1.225426in}}{\pgfqpoint{1.142358in}{1.221308in}}%
\pgfpathcurveto{\pgfqpoint{1.138240in}{1.217190in}}{\pgfqpoint{1.135926in}{1.211603in}}{\pgfqpoint{1.135926in}{1.205780in}}%
\pgfpathcurveto{\pgfqpoint{1.135926in}{1.199956in}}{\pgfqpoint{1.138240in}{1.194369in}}{\pgfqpoint{1.142358in}{1.190251in}}%
\pgfpathcurveto{\pgfqpoint{1.146476in}{1.186133in}}{\pgfqpoint{1.152062in}{1.183819in}}{\pgfqpoint{1.157886in}{1.183819in}}%
\pgfpathclose%
\pgfusepath{stroke,fill}%
\end{pgfscope}%
\begin{pgfscope}%
\pgfpathrectangle{\pgfqpoint{0.211875in}{0.211875in}}{\pgfqpoint{1.313625in}{1.279725in}}%
\pgfusepath{clip}%
\pgfsetbuttcap%
\pgfsetroundjoin%
\definecolor{currentfill}{rgb}{0.121569,0.466667,0.705882}%
\pgfsetfillcolor{currentfill}%
\pgfsetlinewidth{1.003750pt}%
\definecolor{currentstroke}{rgb}{0.121569,0.466667,0.705882}%
\pgfsetstrokecolor{currentstroke}%
\pgfsetdash{}{0pt}%
\pgfpathmoveto{\pgfqpoint{0.904370in}{0.934244in}}%
\pgfpathcurveto{\pgfqpoint{0.910194in}{0.934244in}}{\pgfqpoint{0.915781in}{0.936558in}}{\pgfqpoint{0.919899in}{0.940676in}}%
\pgfpathcurveto{\pgfqpoint{0.924017in}{0.944794in}}{\pgfqpoint{0.926331in}{0.950380in}}{\pgfqpoint{0.926331in}{0.956204in}}%
\pgfpathcurveto{\pgfqpoint{0.926331in}{0.962028in}}{\pgfqpoint{0.924017in}{0.967614in}}{\pgfqpoint{0.919899in}{0.971732in}}%
\pgfpathcurveto{\pgfqpoint{0.915781in}{0.975850in}}{\pgfqpoint{0.910194in}{0.978164in}}{\pgfqpoint{0.904370in}{0.978164in}}%
\pgfpathcurveto{\pgfqpoint{0.898546in}{0.978164in}}{\pgfqpoint{0.892960in}{0.975850in}}{\pgfqpoint{0.888842in}{0.971732in}}%
\pgfpathcurveto{\pgfqpoint{0.884724in}{0.967614in}}{\pgfqpoint{0.882410in}{0.962028in}}{\pgfqpoint{0.882410in}{0.956204in}}%
\pgfpathcurveto{\pgfqpoint{0.882410in}{0.950380in}}{\pgfqpoint{0.884724in}{0.944794in}}{\pgfqpoint{0.888842in}{0.940676in}}%
\pgfpathcurveto{\pgfqpoint{0.892960in}{0.936558in}}{\pgfqpoint{0.898546in}{0.934244in}}{\pgfqpoint{0.904370in}{0.934244in}}%
\pgfpathclose%
\pgfusepath{stroke,fill}%
\end{pgfscope}%
\begin{pgfscope}%
\pgfpathrectangle{\pgfqpoint{0.211875in}{0.211875in}}{\pgfqpoint{1.313625in}{1.279725in}}%
\pgfusepath{clip}%
\pgfsetbuttcap%
\pgfsetroundjoin%
\definecolor{currentfill}{rgb}{0.121569,0.466667,0.705882}%
\pgfsetfillcolor{currentfill}%
\pgfsetlinewidth{1.003750pt}%
\definecolor{currentstroke}{rgb}{0.121569,0.466667,0.705882}%
\pgfsetstrokecolor{currentstroke}%
\pgfsetdash{}{0pt}%
\pgfpathmoveto{\pgfqpoint{1.264328in}{1.111408in}}%
\pgfpathcurveto{\pgfqpoint{1.270152in}{1.111408in}}{\pgfqpoint{1.275738in}{1.113722in}}{\pgfqpoint{1.279856in}{1.117840in}}%
\pgfpathcurveto{\pgfqpoint{1.283974in}{1.121959in}}{\pgfqpoint{1.286288in}{1.127545in}}{\pgfqpoint{1.286288in}{1.133369in}}%
\pgfpathcurveto{\pgfqpoint{1.286288in}{1.139193in}}{\pgfqpoint{1.283974in}{1.144779in}}{\pgfqpoint{1.279856in}{1.148897in}}%
\pgfpathcurveto{\pgfqpoint{1.275738in}{1.153015in}}{\pgfqpoint{1.270152in}{1.155329in}}{\pgfqpoint{1.264328in}{1.155329in}}%
\pgfpathcurveto{\pgfqpoint{1.258504in}{1.155329in}}{\pgfqpoint{1.252918in}{1.153015in}}{\pgfqpoint{1.248800in}{1.148897in}}%
\pgfpathcurveto{\pgfqpoint{1.244681in}{1.144779in}}{\pgfqpoint{1.242367in}{1.139193in}}{\pgfqpoint{1.242367in}{1.133369in}}%
\pgfpathcurveto{\pgfqpoint{1.242367in}{1.127545in}}{\pgfqpoint{1.244681in}{1.121959in}}{\pgfqpoint{1.248800in}{1.117840in}}%
\pgfpathcurveto{\pgfqpoint{1.252918in}{1.113722in}}{\pgfqpoint{1.258504in}{1.111408in}}{\pgfqpoint{1.264328in}{1.111408in}}%
\pgfpathclose%
\pgfusepath{stroke,fill}%
\end{pgfscope}%
\begin{pgfscope}%
\pgfpathrectangle{\pgfqpoint{0.211875in}{0.211875in}}{\pgfqpoint{1.313625in}{1.279725in}}%
\pgfusepath{clip}%
\pgfsetbuttcap%
\pgfsetroundjoin%
\definecolor{currentfill}{rgb}{0.121569,0.466667,0.705882}%
\pgfsetfillcolor{currentfill}%
\pgfsetlinewidth{1.003750pt}%
\definecolor{currentstroke}{rgb}{0.121569,0.466667,0.705882}%
\pgfsetstrokecolor{currentstroke}%
\pgfsetdash{}{0pt}%
\pgfpathmoveto{\pgfqpoint{1.337467in}{1.119006in}}%
\pgfpathcurveto{\pgfqpoint{1.343290in}{1.119006in}}{\pgfqpoint{1.348877in}{1.121320in}}{\pgfqpoint{1.352995in}{1.125438in}}%
\pgfpathcurveto{\pgfqpoint{1.357113in}{1.129556in}}{\pgfqpoint{1.359427in}{1.135142in}}{\pgfqpoint{1.359427in}{1.140966in}}%
\pgfpathcurveto{\pgfqpoint{1.359427in}{1.146790in}}{\pgfqpoint{1.357113in}{1.152376in}}{\pgfqpoint{1.352995in}{1.156494in}}%
\pgfpathcurveto{\pgfqpoint{1.348877in}{1.160612in}}{\pgfqpoint{1.343290in}{1.162926in}}{\pgfqpoint{1.337467in}{1.162926in}}%
\pgfpathcurveto{\pgfqpoint{1.331643in}{1.162926in}}{\pgfqpoint{1.326056in}{1.160612in}}{\pgfqpoint{1.321938in}{1.156494in}}%
\pgfpathcurveto{\pgfqpoint{1.317820in}{1.152376in}}{\pgfqpoint{1.315506in}{1.146790in}}{\pgfqpoint{1.315506in}{1.140966in}}%
\pgfpathcurveto{\pgfqpoint{1.315506in}{1.135142in}}{\pgfqpoint{1.317820in}{1.129556in}}{\pgfqpoint{1.321938in}{1.125438in}}%
\pgfpathcurveto{\pgfqpoint{1.326056in}{1.121320in}}{\pgfqpoint{1.331643in}{1.119006in}}{\pgfqpoint{1.337467in}{1.119006in}}%
\pgfpathclose%
\pgfusepath{stroke,fill}%
\end{pgfscope}%
\begin{pgfscope}%
\pgfpathrectangle{\pgfqpoint{0.211875in}{0.211875in}}{\pgfqpoint{1.313625in}{1.279725in}}%
\pgfusepath{clip}%
\pgfsetbuttcap%
\pgfsetroundjoin%
\definecolor{currentfill}{rgb}{0.121569,0.466667,0.705882}%
\pgfsetfillcolor{currentfill}%
\pgfsetlinewidth{1.003750pt}%
\definecolor{currentstroke}{rgb}{0.121569,0.466667,0.705882}%
\pgfsetstrokecolor{currentstroke}%
\pgfsetdash{}{0pt}%
\pgfpathmoveto{\pgfqpoint{0.900054in}{0.937462in}}%
\pgfpathcurveto{\pgfqpoint{0.905878in}{0.937462in}}{\pgfqpoint{0.911465in}{0.939776in}}{\pgfqpoint{0.915583in}{0.943894in}}%
\pgfpathcurveto{\pgfqpoint{0.919701in}{0.948013in}}{\pgfqpoint{0.922015in}{0.953599in}}{\pgfqpoint{0.922015in}{0.959423in}}%
\pgfpathcurveto{\pgfqpoint{0.922015in}{0.965247in}}{\pgfqpoint{0.919701in}{0.970833in}}{\pgfqpoint{0.915583in}{0.974951in}}%
\pgfpathcurveto{\pgfqpoint{0.911465in}{0.979069in}}{\pgfqpoint{0.905878in}{0.981383in}}{\pgfqpoint{0.900054in}{0.981383in}}%
\pgfpathcurveto{\pgfqpoint{0.894230in}{0.981383in}}{\pgfqpoint{0.888644in}{0.979069in}}{\pgfqpoint{0.884526in}{0.974951in}}%
\pgfpathcurveto{\pgfqpoint{0.880408in}{0.970833in}}{\pgfqpoint{0.878094in}{0.965247in}}{\pgfqpoint{0.878094in}{0.959423in}}%
\pgfpathcurveto{\pgfqpoint{0.878094in}{0.953599in}}{\pgfqpoint{0.880408in}{0.948013in}}{\pgfqpoint{0.884526in}{0.943894in}}%
\pgfpathcurveto{\pgfqpoint{0.888644in}{0.939776in}}{\pgfqpoint{0.894230in}{0.937462in}}{\pgfqpoint{0.900054in}{0.937462in}}%
\pgfpathclose%
\pgfusepath{stroke,fill}%
\end{pgfscope}%
\begin{pgfscope}%
\pgfpathrectangle{\pgfqpoint{0.211875in}{0.211875in}}{\pgfqpoint{1.313625in}{1.279725in}}%
\pgfusepath{clip}%
\pgfsetbuttcap%
\pgfsetroundjoin%
\definecolor{currentfill}{rgb}{0.121569,0.466667,0.705882}%
\pgfsetfillcolor{currentfill}%
\pgfsetlinewidth{1.003750pt}%
\definecolor{currentstroke}{rgb}{0.121569,0.466667,0.705882}%
\pgfsetstrokecolor{currentstroke}%
\pgfsetdash{}{0pt}%
\pgfpathmoveto{\pgfqpoint{1.094653in}{1.050015in}}%
\pgfpathcurveto{\pgfqpoint{1.100477in}{1.050015in}}{\pgfqpoint{1.106063in}{1.052328in}}{\pgfqpoint{1.110181in}{1.056447in}}%
\pgfpathcurveto{\pgfqpoint{1.114300in}{1.060565in}}{\pgfqpoint{1.116613in}{1.066151in}}{\pgfqpoint{1.116613in}{1.071975in}}%
\pgfpathcurveto{\pgfqpoint{1.116613in}{1.077799in}}{\pgfqpoint{1.114300in}{1.083385in}}{\pgfqpoint{1.110181in}{1.087503in}}%
\pgfpathcurveto{\pgfqpoint{1.106063in}{1.091621in}}{\pgfqpoint{1.100477in}{1.093935in}}{\pgfqpoint{1.094653in}{1.093935in}}%
\pgfpathcurveto{\pgfqpoint{1.088829in}{1.093935in}}{\pgfqpoint{1.083243in}{1.091621in}}{\pgfqpoint{1.079125in}{1.087503in}}%
\pgfpathcurveto{\pgfqpoint{1.075007in}{1.083385in}}{\pgfqpoint{1.072693in}{1.077799in}}{\pgfqpoint{1.072693in}{1.071975in}}%
\pgfpathcurveto{\pgfqpoint{1.072693in}{1.066151in}}{\pgfqpoint{1.075007in}{1.060565in}}{\pgfqpoint{1.079125in}{1.056447in}}%
\pgfpathcurveto{\pgfqpoint{1.083243in}{1.052328in}}{\pgfqpoint{1.088829in}{1.050015in}}{\pgfqpoint{1.094653in}{1.050015in}}%
\pgfpathclose%
\pgfusepath{stroke,fill}%
\end{pgfscope}%
\begin{pgfscope}%
\pgfpathrectangle{\pgfqpoint{0.211875in}{0.211875in}}{\pgfqpoint{1.313625in}{1.279725in}}%
\pgfusepath{clip}%
\pgfsetbuttcap%
\pgfsetroundjoin%
\definecolor{currentfill}{rgb}{0.121569,0.466667,0.705882}%
\pgfsetfillcolor{currentfill}%
\pgfsetlinewidth{1.003750pt}%
\definecolor{currentstroke}{rgb}{0.121569,0.466667,0.705882}%
\pgfsetstrokecolor{currentstroke}%
\pgfsetdash{}{0pt}%
\pgfpathmoveto{\pgfqpoint{0.905852in}{0.894752in}}%
\pgfpathcurveto{\pgfqpoint{0.911676in}{0.894752in}}{\pgfqpoint{0.917262in}{0.897066in}}{\pgfqpoint{0.921380in}{0.901184in}}%
\pgfpathcurveto{\pgfqpoint{0.925498in}{0.905302in}}{\pgfqpoint{0.927812in}{0.910888in}}{\pgfqpoint{0.927812in}{0.916712in}}%
\pgfpathcurveto{\pgfqpoint{0.927812in}{0.922536in}}{\pgfqpoint{0.925498in}{0.928122in}}{\pgfqpoint{0.921380in}{0.932240in}}%
\pgfpathcurveto{\pgfqpoint{0.917262in}{0.936359in}}{\pgfqpoint{0.911676in}{0.938672in}}{\pgfqpoint{0.905852in}{0.938672in}}%
\pgfpathcurveto{\pgfqpoint{0.900028in}{0.938672in}}{\pgfqpoint{0.894442in}{0.936359in}}{\pgfqpoint{0.890324in}{0.932240in}}%
\pgfpathcurveto{\pgfqpoint{0.886205in}{0.928122in}}{\pgfqpoint{0.883892in}{0.922536in}}{\pgfqpoint{0.883892in}{0.916712in}}%
\pgfpathcurveto{\pgfqpoint{0.883892in}{0.910888in}}{\pgfqpoint{0.886205in}{0.905302in}}{\pgfqpoint{0.890324in}{0.901184in}}%
\pgfpathcurveto{\pgfqpoint{0.894442in}{0.897066in}}{\pgfqpoint{0.900028in}{0.894752in}}{\pgfqpoint{0.905852in}{0.894752in}}%
\pgfpathclose%
\pgfusepath{stroke,fill}%
\end{pgfscope}%
\begin{pgfscope}%
\pgfpathrectangle{\pgfqpoint{0.211875in}{0.211875in}}{\pgfqpoint{1.313625in}{1.279725in}}%
\pgfusepath{clip}%
\pgfsetbuttcap%
\pgfsetroundjoin%
\definecolor{currentfill}{rgb}{0.121569,0.466667,0.705882}%
\pgfsetfillcolor{currentfill}%
\pgfsetlinewidth{1.003750pt}%
\definecolor{currentstroke}{rgb}{0.121569,0.466667,0.705882}%
\pgfsetstrokecolor{currentstroke}%
\pgfsetdash{}{0pt}%
\pgfpathmoveto{\pgfqpoint{0.839655in}{0.912430in}}%
\pgfpathcurveto{\pgfqpoint{0.845479in}{0.912430in}}{\pgfqpoint{0.851065in}{0.914744in}}{\pgfqpoint{0.855183in}{0.918862in}}%
\pgfpathcurveto{\pgfqpoint{0.859301in}{0.922980in}}{\pgfqpoint{0.861615in}{0.928566in}}{\pgfqpoint{0.861615in}{0.934390in}}%
\pgfpathcurveto{\pgfqpoint{0.861615in}{0.940214in}}{\pgfqpoint{0.859301in}{0.945800in}}{\pgfqpoint{0.855183in}{0.949918in}}%
\pgfpathcurveto{\pgfqpoint{0.851065in}{0.954036in}}{\pgfqpoint{0.845479in}{0.956350in}}{\pgfqpoint{0.839655in}{0.956350in}}%
\pgfpathcurveto{\pgfqpoint{0.833831in}{0.956350in}}{\pgfqpoint{0.828245in}{0.954036in}}{\pgfqpoint{0.824127in}{0.949918in}}%
\pgfpathcurveto{\pgfqpoint{0.820008in}{0.945800in}}{\pgfqpoint{0.817695in}{0.940214in}}{\pgfqpoint{0.817695in}{0.934390in}}%
\pgfpathcurveto{\pgfqpoint{0.817695in}{0.928566in}}{\pgfqpoint{0.820008in}{0.922980in}}{\pgfqpoint{0.824127in}{0.918862in}}%
\pgfpathcurveto{\pgfqpoint{0.828245in}{0.914744in}}{\pgfqpoint{0.833831in}{0.912430in}}{\pgfqpoint{0.839655in}{0.912430in}}%
\pgfpathclose%
\pgfusepath{stroke,fill}%
\end{pgfscope}%
\begin{pgfscope}%
\pgfpathrectangle{\pgfqpoint{0.211875in}{0.211875in}}{\pgfqpoint{1.313625in}{1.279725in}}%
\pgfusepath{clip}%
\pgfsetbuttcap%
\pgfsetroundjoin%
\definecolor{currentfill}{rgb}{0.121569,0.466667,0.705882}%
\pgfsetfillcolor{currentfill}%
\pgfsetlinewidth{1.003750pt}%
\definecolor{currentstroke}{rgb}{0.121569,0.466667,0.705882}%
\pgfsetstrokecolor{currentstroke}%
\pgfsetdash{}{0pt}%
\pgfpathmoveto{\pgfqpoint{1.445069in}{1.102303in}}%
\pgfpathcurveto{\pgfqpoint{1.450893in}{1.102303in}}{\pgfqpoint{1.456479in}{1.104617in}}{\pgfqpoint{1.460598in}{1.108735in}}%
\pgfpathcurveto{\pgfqpoint{1.464716in}{1.112853in}}{\pgfqpoint{1.467030in}{1.118439in}}{\pgfqpoint{1.467030in}{1.124263in}}%
\pgfpathcurveto{\pgfqpoint{1.467030in}{1.130087in}}{\pgfqpoint{1.464716in}{1.135673in}}{\pgfqpoint{1.460598in}{1.139791in}}%
\pgfpathcurveto{\pgfqpoint{1.456479in}{1.143910in}}{\pgfqpoint{1.450893in}{1.146223in}}{\pgfqpoint{1.445069in}{1.146223in}}%
\pgfpathcurveto{\pgfqpoint{1.439245in}{1.146223in}}{\pgfqpoint{1.433659in}{1.143910in}}{\pgfqpoint{1.429541in}{1.139791in}}%
\pgfpathcurveto{\pgfqpoint{1.425423in}{1.135673in}}{\pgfqpoint{1.423109in}{1.130087in}}{\pgfqpoint{1.423109in}{1.124263in}}%
\pgfpathcurveto{\pgfqpoint{1.423109in}{1.118439in}}{\pgfqpoint{1.425423in}{1.112853in}}{\pgfqpoint{1.429541in}{1.108735in}}%
\pgfpathcurveto{\pgfqpoint{1.433659in}{1.104617in}}{\pgfqpoint{1.439245in}{1.102303in}}{\pgfqpoint{1.445069in}{1.102303in}}%
\pgfpathclose%
\pgfusepath{stroke,fill}%
\end{pgfscope}%
\begin{pgfscope}%
\pgfpathrectangle{\pgfqpoint{0.211875in}{0.211875in}}{\pgfqpoint{1.313625in}{1.279725in}}%
\pgfusepath{clip}%
\pgfsetbuttcap%
\pgfsetroundjoin%
\definecolor{currentfill}{rgb}{0.121569,0.466667,0.705882}%
\pgfsetfillcolor{currentfill}%
\pgfsetlinewidth{1.003750pt}%
\definecolor{currentstroke}{rgb}{0.121569,0.466667,0.705882}%
\pgfsetstrokecolor{currentstroke}%
\pgfsetdash{}{0pt}%
\pgfpathmoveto{\pgfqpoint{1.178532in}{1.087849in}}%
\pgfpathcurveto{\pgfqpoint{1.184356in}{1.087849in}}{\pgfqpoint{1.189942in}{1.090163in}}{\pgfqpoint{1.194060in}{1.094281in}}%
\pgfpathcurveto{\pgfqpoint{1.198178in}{1.098399in}}{\pgfqpoint{1.200492in}{1.103985in}}{\pgfqpoint{1.200492in}{1.109809in}}%
\pgfpathcurveto{\pgfqpoint{1.200492in}{1.115633in}}{\pgfqpoint{1.198178in}{1.121219in}}{\pgfqpoint{1.194060in}{1.125338in}}%
\pgfpathcurveto{\pgfqpoint{1.189942in}{1.129456in}}{\pgfqpoint{1.184356in}{1.131770in}}{\pgfqpoint{1.178532in}{1.131770in}}%
\pgfpathcurveto{\pgfqpoint{1.172708in}{1.131770in}}{\pgfqpoint{1.167122in}{1.129456in}}{\pgfqpoint{1.163003in}{1.125338in}}%
\pgfpathcurveto{\pgfqpoint{1.158885in}{1.121219in}}{\pgfqpoint{1.156571in}{1.115633in}}{\pgfqpoint{1.156571in}{1.109809in}}%
\pgfpathcurveto{\pgfqpoint{1.156571in}{1.103985in}}{\pgfqpoint{1.158885in}{1.098399in}}{\pgfqpoint{1.163003in}{1.094281in}}%
\pgfpathcurveto{\pgfqpoint{1.167122in}{1.090163in}}{\pgfqpoint{1.172708in}{1.087849in}}{\pgfqpoint{1.178532in}{1.087849in}}%
\pgfpathclose%
\pgfusepath{stroke,fill}%
\end{pgfscope}%
\begin{pgfscope}%
\pgfpathrectangle{\pgfqpoint{0.211875in}{0.211875in}}{\pgfqpoint{1.313625in}{1.279725in}}%
\pgfusepath{clip}%
\pgfsetbuttcap%
\pgfsetroundjoin%
\definecolor{currentfill}{rgb}{0.121569,0.466667,0.705882}%
\pgfsetfillcolor{currentfill}%
\pgfsetlinewidth{1.003750pt}%
\definecolor{currentstroke}{rgb}{0.121569,0.466667,0.705882}%
\pgfsetstrokecolor{currentstroke}%
\pgfsetdash{}{0pt}%
\pgfpathmoveto{\pgfqpoint{0.291002in}{0.812713in}}%
\pgfpathcurveto{\pgfqpoint{0.296826in}{0.812713in}}{\pgfqpoint{0.302412in}{0.815027in}}{\pgfqpoint{0.306530in}{0.819145in}}%
\pgfpathcurveto{\pgfqpoint{0.310649in}{0.823263in}}{\pgfqpoint{0.312962in}{0.828849in}}{\pgfqpoint{0.312962in}{0.834673in}}%
\pgfpathcurveto{\pgfqpoint{0.312962in}{0.840497in}}{\pgfqpoint{0.310649in}{0.846083in}}{\pgfqpoint{0.306530in}{0.850201in}}%
\pgfpathcurveto{\pgfqpoint{0.302412in}{0.854320in}}{\pgfqpoint{0.296826in}{0.856633in}}{\pgfqpoint{0.291002in}{0.856633in}}%
\pgfpathcurveto{\pgfqpoint{0.285178in}{0.856633in}}{\pgfqpoint{0.279592in}{0.854320in}}{\pgfqpoint{0.275474in}{0.850201in}}%
\pgfpathcurveto{\pgfqpoint{0.271356in}{0.846083in}}{\pgfqpoint{0.269042in}{0.840497in}}{\pgfqpoint{0.269042in}{0.834673in}}%
\pgfpathcurveto{\pgfqpoint{0.269042in}{0.828849in}}{\pgfqpoint{0.271356in}{0.823263in}}{\pgfqpoint{0.275474in}{0.819145in}}%
\pgfpathcurveto{\pgfqpoint{0.279592in}{0.815027in}}{\pgfqpoint{0.285178in}{0.812713in}}{\pgfqpoint{0.291002in}{0.812713in}}%
\pgfpathclose%
\pgfusepath{stroke,fill}%
\end{pgfscope}%
\begin{pgfscope}%
\pgfpathrectangle{\pgfqpoint{0.211875in}{0.211875in}}{\pgfqpoint{1.313625in}{1.279725in}}%
\pgfusepath{clip}%
\pgfsetbuttcap%
\pgfsetroundjoin%
\definecolor{currentfill}{rgb}{0.121569,0.466667,0.705882}%
\pgfsetfillcolor{currentfill}%
\pgfsetlinewidth{1.003750pt}%
\definecolor{currentstroke}{rgb}{0.121569,0.466667,0.705882}%
\pgfsetstrokecolor{currentstroke}%
\pgfsetdash{}{0pt}%
\pgfpathmoveto{\pgfqpoint{0.713623in}{0.267469in}}%
\pgfpathcurveto{\pgfqpoint{0.719447in}{0.267469in}}{\pgfqpoint{0.725033in}{0.269783in}}{\pgfqpoint{0.729152in}{0.273901in}}%
\pgfpathcurveto{\pgfqpoint{0.733270in}{0.278019in}}{\pgfqpoint{0.735584in}{0.283606in}}{\pgfqpoint{0.735584in}{0.289430in}}%
\pgfpathcurveto{\pgfqpoint{0.735584in}{0.295253in}}{\pgfqpoint{0.733270in}{0.300840in}}{\pgfqpoint{0.729152in}{0.304958in}}%
\pgfpathcurveto{\pgfqpoint{0.725033in}{0.309076in}}{\pgfqpoint{0.719447in}{0.311390in}}{\pgfqpoint{0.713623in}{0.311390in}}%
\pgfpathcurveto{\pgfqpoint{0.707799in}{0.311390in}}{\pgfqpoint{0.702213in}{0.309076in}}{\pgfqpoint{0.698095in}{0.304958in}}%
\pgfpathcurveto{\pgfqpoint{0.693977in}{0.300840in}}{\pgfqpoint{0.691663in}{0.295253in}}{\pgfqpoint{0.691663in}{0.289430in}}%
\pgfpathcurveto{\pgfqpoint{0.691663in}{0.283606in}}{\pgfqpoint{0.693977in}{0.278019in}}{\pgfqpoint{0.698095in}{0.273901in}}%
\pgfpathcurveto{\pgfqpoint{0.702213in}{0.269783in}}{\pgfqpoint{0.707799in}{0.267469in}}{\pgfqpoint{0.713623in}{0.267469in}}%
\pgfpathclose%
\pgfusepath{stroke,fill}%
\end{pgfscope}%
\begin{pgfscope}%
\pgfpathrectangle{\pgfqpoint{0.211875in}{0.211875in}}{\pgfqpoint{1.313625in}{1.279725in}}%
\pgfusepath{clip}%
\pgfsetbuttcap%
\pgfsetroundjoin%
\definecolor{currentfill}{rgb}{0.121569,0.466667,0.705882}%
\pgfsetfillcolor{currentfill}%
\pgfsetlinewidth{1.003750pt}%
\definecolor{currentstroke}{rgb}{0.121569,0.466667,0.705882}%
\pgfsetstrokecolor{currentstroke}%
\pgfsetdash{}{0pt}%
\pgfpathmoveto{\pgfqpoint{1.407079in}{1.044834in}}%
\pgfpathcurveto{\pgfqpoint{1.412903in}{1.044834in}}{\pgfqpoint{1.418489in}{1.047148in}}{\pgfqpoint{1.422607in}{1.051266in}}%
\pgfpathcurveto{\pgfqpoint{1.426726in}{1.055384in}}{\pgfqpoint{1.429039in}{1.060971in}}{\pgfqpoint{1.429039in}{1.066794in}}%
\pgfpathcurveto{\pgfqpoint{1.429039in}{1.072618in}}{\pgfqpoint{1.426726in}{1.078205in}}{\pgfqpoint{1.422607in}{1.082323in}}%
\pgfpathcurveto{\pgfqpoint{1.418489in}{1.086441in}}{\pgfqpoint{1.412903in}{1.088755in}}{\pgfqpoint{1.407079in}{1.088755in}}%
\pgfpathcurveto{\pgfqpoint{1.401255in}{1.088755in}}{\pgfqpoint{1.395669in}{1.086441in}}{\pgfqpoint{1.391551in}{1.082323in}}%
\pgfpathcurveto{\pgfqpoint{1.387433in}{1.078205in}}{\pgfqpoint{1.385119in}{1.072618in}}{\pgfqpoint{1.385119in}{1.066794in}}%
\pgfpathcurveto{\pgfqpoint{1.385119in}{1.060971in}}{\pgfqpoint{1.387433in}{1.055384in}}{\pgfqpoint{1.391551in}{1.051266in}}%
\pgfpathcurveto{\pgfqpoint{1.395669in}{1.047148in}}{\pgfqpoint{1.401255in}{1.044834in}}{\pgfqpoint{1.407079in}{1.044834in}}%
\pgfpathclose%
\pgfusepath{stroke,fill}%
\end{pgfscope}%
\begin{pgfscope}%
\pgfpathrectangle{\pgfqpoint{0.211875in}{0.211875in}}{\pgfqpoint{1.313625in}{1.279725in}}%
\pgfusepath{clip}%
\pgfsetbuttcap%
\pgfsetroundjoin%
\definecolor{currentfill}{rgb}{0.121569,0.466667,0.705882}%
\pgfsetfillcolor{currentfill}%
\pgfsetlinewidth{1.003750pt}%
\definecolor{currentstroke}{rgb}{0.121569,0.466667,0.705882}%
\pgfsetstrokecolor{currentstroke}%
\pgfsetdash{}{0pt}%
\pgfpathmoveto{\pgfqpoint{0.891031in}{0.941585in}}%
\pgfpathcurveto{\pgfqpoint{0.896855in}{0.941585in}}{\pgfqpoint{0.902441in}{0.943899in}}{\pgfqpoint{0.906559in}{0.948017in}}%
\pgfpathcurveto{\pgfqpoint{0.910678in}{0.952135in}}{\pgfqpoint{0.912991in}{0.957722in}}{\pgfqpoint{0.912991in}{0.963545in}}%
\pgfpathcurveto{\pgfqpoint{0.912991in}{0.969369in}}{\pgfqpoint{0.910678in}{0.974956in}}{\pgfqpoint{0.906559in}{0.979074in}}%
\pgfpathcurveto{\pgfqpoint{0.902441in}{0.983192in}}{\pgfqpoint{0.896855in}{0.985506in}}{\pgfqpoint{0.891031in}{0.985506in}}%
\pgfpathcurveto{\pgfqpoint{0.885207in}{0.985506in}}{\pgfqpoint{0.879621in}{0.983192in}}{\pgfqpoint{0.875503in}{0.979074in}}%
\pgfpathcurveto{\pgfqpoint{0.871385in}{0.974956in}}{\pgfqpoint{0.869071in}{0.969369in}}{\pgfqpoint{0.869071in}{0.963545in}}%
\pgfpathcurveto{\pgfqpoint{0.869071in}{0.957722in}}{\pgfqpoint{0.871385in}{0.952135in}}{\pgfqpoint{0.875503in}{0.948017in}}%
\pgfpathcurveto{\pgfqpoint{0.879621in}{0.943899in}}{\pgfqpoint{0.885207in}{0.941585in}}{\pgfqpoint{0.891031in}{0.941585in}}%
\pgfpathclose%
\pgfusepath{stroke,fill}%
\end{pgfscope}%
\begin{pgfscope}%
\pgfpathrectangle{\pgfqpoint{0.211875in}{0.211875in}}{\pgfqpoint{1.313625in}{1.279725in}}%
\pgfusepath{clip}%
\pgfsetbuttcap%
\pgfsetroundjoin%
\definecolor{currentfill}{rgb}{0.121569,0.466667,0.705882}%
\pgfsetfillcolor{currentfill}%
\pgfsetlinewidth{1.003750pt}%
\definecolor{currentstroke}{rgb}{0.121569,0.466667,0.705882}%
\pgfsetstrokecolor{currentstroke}%
\pgfsetdash{}{0pt}%
\pgfpathmoveto{\pgfqpoint{0.943016in}{1.371991in}}%
\pgfpathcurveto{\pgfqpoint{0.948840in}{1.371991in}}{\pgfqpoint{0.954427in}{1.374305in}}{\pgfqpoint{0.958545in}{1.378423in}}%
\pgfpathcurveto{\pgfqpoint{0.962663in}{1.382541in}}{\pgfqpoint{0.964977in}{1.388128in}}{\pgfqpoint{0.964977in}{1.393951in}}%
\pgfpathcurveto{\pgfqpoint{0.964977in}{1.399775in}}{\pgfqpoint{0.962663in}{1.405362in}}{\pgfqpoint{0.958545in}{1.409480in}}%
\pgfpathcurveto{\pgfqpoint{0.954427in}{1.413598in}}{\pgfqpoint{0.948840in}{1.415912in}}{\pgfqpoint{0.943016in}{1.415912in}}%
\pgfpathcurveto{\pgfqpoint{0.937193in}{1.415912in}}{\pgfqpoint{0.931606in}{1.413598in}}{\pgfqpoint{0.927488in}{1.409480in}}%
\pgfpathcurveto{\pgfqpoint{0.923370in}{1.405362in}}{\pgfqpoint{0.921056in}{1.399775in}}{\pgfqpoint{0.921056in}{1.393951in}}%
\pgfpathcurveto{\pgfqpoint{0.921056in}{1.388128in}}{\pgfqpoint{0.923370in}{1.382541in}}{\pgfqpoint{0.927488in}{1.378423in}}%
\pgfpathcurveto{\pgfqpoint{0.931606in}{1.374305in}}{\pgfqpoint{0.937193in}{1.371991in}}{\pgfqpoint{0.943016in}{1.371991in}}%
\pgfpathclose%
\pgfusepath{stroke,fill}%
\end{pgfscope}%
\begin{pgfscope}%
\pgfpathrectangle{\pgfqpoint{0.211875in}{0.211875in}}{\pgfqpoint{1.313625in}{1.279725in}}%
\pgfusepath{clip}%
\pgfsetbuttcap%
\pgfsetroundjoin%
\definecolor{currentfill}{rgb}{0.121569,0.466667,0.705882}%
\pgfsetfillcolor{currentfill}%
\pgfsetlinewidth{1.003750pt}%
\definecolor{currentstroke}{rgb}{0.121569,0.466667,0.705882}%
\pgfsetstrokecolor{currentstroke}%
\pgfsetdash{}{0pt}%
\pgfpathmoveto{\pgfqpoint{0.980971in}{0.965361in}}%
\pgfpathcurveto{\pgfqpoint{0.986795in}{0.965361in}}{\pgfqpoint{0.992381in}{0.967675in}}{\pgfqpoint{0.996499in}{0.971793in}}%
\pgfpathcurveto{\pgfqpoint{1.000617in}{0.975911in}}{\pgfqpoint{1.002931in}{0.981497in}}{\pgfqpoint{1.002931in}{0.987321in}}%
\pgfpathcurveto{\pgfqpoint{1.002931in}{0.993145in}}{\pgfqpoint{1.000617in}{0.998731in}}{\pgfqpoint{0.996499in}{1.002849in}}%
\pgfpathcurveto{\pgfqpoint{0.992381in}{1.006967in}}{\pgfqpoint{0.986795in}{1.009281in}}{\pgfqpoint{0.980971in}{1.009281in}}%
\pgfpathcurveto{\pgfqpoint{0.975147in}{1.009281in}}{\pgfqpoint{0.969560in}{1.006967in}}{\pgfqpoint{0.965442in}{1.002849in}}%
\pgfpathcurveto{\pgfqpoint{0.961324in}{0.998731in}}{\pgfqpoint{0.959010in}{0.993145in}}{\pgfqpoint{0.959010in}{0.987321in}}%
\pgfpathcurveto{\pgfqpoint{0.959010in}{0.981497in}}{\pgfqpoint{0.961324in}{0.975911in}}{\pgfqpoint{0.965442in}{0.971793in}}%
\pgfpathcurveto{\pgfqpoint{0.969560in}{0.967675in}}{\pgfqpoint{0.975147in}{0.965361in}}{\pgfqpoint{0.980971in}{0.965361in}}%
\pgfpathclose%
\pgfusepath{stroke,fill}%
\end{pgfscope}%
\begin{pgfscope}%
\pgfpathrectangle{\pgfqpoint{0.211875in}{0.211875in}}{\pgfqpoint{1.313625in}{1.279725in}}%
\pgfusepath{clip}%
\pgfsetbuttcap%
\pgfsetroundjoin%
\definecolor{currentfill}{rgb}{0.121569,0.466667,0.705882}%
\pgfsetfillcolor{currentfill}%
\pgfsetlinewidth{1.003750pt}%
\definecolor{currentstroke}{rgb}{0.121569,0.466667,0.705882}%
\pgfsetstrokecolor{currentstroke}%
\pgfsetdash{}{0pt}%
\pgfpathmoveto{\pgfqpoint{0.916036in}{0.939561in}}%
\pgfpathcurveto{\pgfqpoint{0.921860in}{0.939561in}}{\pgfqpoint{0.927446in}{0.941875in}}{\pgfqpoint{0.931564in}{0.945993in}}%
\pgfpathcurveto{\pgfqpoint{0.935682in}{0.950112in}}{\pgfqpoint{0.937996in}{0.955698in}}{\pgfqpoint{0.937996in}{0.961522in}}%
\pgfpathcurveto{\pgfqpoint{0.937996in}{0.967346in}}{\pgfqpoint{0.935682in}{0.972932in}}{\pgfqpoint{0.931564in}{0.977050in}}%
\pgfpathcurveto{\pgfqpoint{0.927446in}{0.981168in}}{\pgfqpoint{0.921860in}{0.983482in}}{\pgfqpoint{0.916036in}{0.983482in}}%
\pgfpathcurveto{\pgfqpoint{0.910212in}{0.983482in}}{\pgfqpoint{0.904626in}{0.981168in}}{\pgfqpoint{0.900508in}{0.977050in}}%
\pgfpathcurveto{\pgfqpoint{0.896390in}{0.972932in}}{\pgfqpoint{0.894076in}{0.967346in}}{\pgfqpoint{0.894076in}{0.961522in}}%
\pgfpathcurveto{\pgfqpoint{0.894076in}{0.955698in}}{\pgfqpoint{0.896390in}{0.950112in}}{\pgfqpoint{0.900508in}{0.945993in}}%
\pgfpathcurveto{\pgfqpoint{0.904626in}{0.941875in}}{\pgfqpoint{0.910212in}{0.939561in}}{\pgfqpoint{0.916036in}{0.939561in}}%
\pgfpathclose%
\pgfusepath{stroke,fill}%
\end{pgfscope}%
\begin{pgfscope}%
\pgfpathrectangle{\pgfqpoint{0.211875in}{0.211875in}}{\pgfqpoint{1.313625in}{1.279725in}}%
\pgfusepath{clip}%
\pgfsetbuttcap%
\pgfsetroundjoin%
\definecolor{currentfill}{rgb}{0.121569,0.466667,0.705882}%
\pgfsetfillcolor{currentfill}%
\pgfsetlinewidth{1.003750pt}%
\definecolor{currentstroke}{rgb}{0.121569,0.466667,0.705882}%
\pgfsetstrokecolor{currentstroke}%
\pgfsetdash{}{0pt}%
\pgfpathmoveto{\pgfqpoint{1.310124in}{0.934643in}}%
\pgfpathcurveto{\pgfqpoint{1.315948in}{0.934643in}}{\pgfqpoint{1.321534in}{0.936957in}}{\pgfqpoint{1.325652in}{0.941075in}}%
\pgfpathcurveto{\pgfqpoint{1.329770in}{0.945193in}}{\pgfqpoint{1.332084in}{0.950779in}}{\pgfqpoint{1.332084in}{0.956603in}}%
\pgfpathcurveto{\pgfqpoint{1.332084in}{0.962427in}}{\pgfqpoint{1.329770in}{0.968013in}}{\pgfqpoint{1.325652in}{0.972131in}}%
\pgfpathcurveto{\pgfqpoint{1.321534in}{0.976249in}}{\pgfqpoint{1.315948in}{0.978563in}}{\pgfqpoint{1.310124in}{0.978563in}}%
\pgfpathcurveto{\pgfqpoint{1.304300in}{0.978563in}}{\pgfqpoint{1.298714in}{0.976249in}}{\pgfqpoint{1.294596in}{0.972131in}}%
\pgfpathcurveto{\pgfqpoint{1.290478in}{0.968013in}}{\pgfqpoint{1.288164in}{0.962427in}}{\pgfqpoint{1.288164in}{0.956603in}}%
\pgfpathcurveto{\pgfqpoint{1.288164in}{0.950779in}}{\pgfqpoint{1.290478in}{0.945193in}}{\pgfqpoint{1.294596in}{0.941075in}}%
\pgfpathcurveto{\pgfqpoint{1.298714in}{0.936957in}}{\pgfqpoint{1.304300in}{0.934643in}}{\pgfqpoint{1.310124in}{0.934643in}}%
\pgfpathclose%
\pgfusepath{stroke,fill}%
\end{pgfscope}%
\begin{pgfscope}%
\pgfpathrectangle{\pgfqpoint{0.211875in}{0.211875in}}{\pgfqpoint{1.313625in}{1.279725in}}%
\pgfusepath{clip}%
\pgfsetbuttcap%
\pgfsetroundjoin%
\definecolor{currentfill}{rgb}{0.121569,0.466667,0.705882}%
\pgfsetfillcolor{currentfill}%
\pgfsetlinewidth{1.003750pt}%
\definecolor{currentstroke}{rgb}{0.121569,0.466667,0.705882}%
\pgfsetstrokecolor{currentstroke}%
\pgfsetdash{}{0pt}%
\pgfpathmoveto{\pgfqpoint{0.905540in}{0.939785in}}%
\pgfpathcurveto{\pgfqpoint{0.911364in}{0.939785in}}{\pgfqpoint{0.916950in}{0.942098in}}{\pgfqpoint{0.921068in}{0.946217in}}%
\pgfpathcurveto{\pgfqpoint{0.925186in}{0.950335in}}{\pgfqpoint{0.927500in}{0.955921in}}{\pgfqpoint{0.927500in}{0.961745in}}%
\pgfpathcurveto{\pgfqpoint{0.927500in}{0.967569in}}{\pgfqpoint{0.925186in}{0.973155in}}{\pgfqpoint{0.921068in}{0.977273in}}%
\pgfpathcurveto{\pgfqpoint{0.916950in}{0.981391in}}{\pgfqpoint{0.911364in}{0.983705in}}{\pgfqpoint{0.905540in}{0.983705in}}%
\pgfpathcurveto{\pgfqpoint{0.899716in}{0.983705in}}{\pgfqpoint{0.894129in}{0.981391in}}{\pgfqpoint{0.890011in}{0.977273in}}%
\pgfpathcurveto{\pgfqpoint{0.885893in}{0.973155in}}{\pgfqpoint{0.883579in}{0.967569in}}{\pgfqpoint{0.883579in}{0.961745in}}%
\pgfpathcurveto{\pgfqpoint{0.883579in}{0.955921in}}{\pgfqpoint{0.885893in}{0.950335in}}{\pgfqpoint{0.890011in}{0.946217in}}%
\pgfpathcurveto{\pgfqpoint{0.894129in}{0.942098in}}{\pgfqpoint{0.899716in}{0.939785in}}{\pgfqpoint{0.905540in}{0.939785in}}%
\pgfpathclose%
\pgfusepath{stroke,fill}%
\end{pgfscope}%
\begin{pgfscope}%
\pgfpathrectangle{\pgfqpoint{0.211875in}{0.211875in}}{\pgfqpoint{1.313625in}{1.279725in}}%
\pgfusepath{clip}%
\pgfsetbuttcap%
\pgfsetroundjoin%
\definecolor{currentfill}{rgb}{0.121569,0.466667,0.705882}%
\pgfsetfillcolor{currentfill}%
\pgfsetlinewidth{1.003750pt}%
\definecolor{currentstroke}{rgb}{0.121569,0.466667,0.705882}%
\pgfsetstrokecolor{currentstroke}%
\pgfsetdash{}{0pt}%
\pgfpathmoveto{\pgfqpoint{1.364640in}{0.991044in}}%
\pgfpathcurveto{\pgfqpoint{1.370464in}{0.991044in}}{\pgfqpoint{1.376050in}{0.993358in}}{\pgfqpoint{1.380168in}{0.997476in}}%
\pgfpathcurveto{\pgfqpoint{1.384286in}{1.001594in}}{\pgfqpoint{1.386600in}{1.007180in}}{\pgfqpoint{1.386600in}{1.013004in}}%
\pgfpathcurveto{\pgfqpoint{1.386600in}{1.018828in}}{\pgfqpoint{1.384286in}{1.024414in}}{\pgfqpoint{1.380168in}{1.028532in}}%
\pgfpathcurveto{\pgfqpoint{1.376050in}{1.032650in}}{\pgfqpoint{1.370464in}{1.034964in}}{\pgfqpoint{1.364640in}{1.034964in}}%
\pgfpathcurveto{\pgfqpoint{1.358816in}{1.034964in}}{\pgfqpoint{1.353229in}{1.032650in}}{\pgfqpoint{1.349111in}{1.028532in}}%
\pgfpathcurveto{\pgfqpoint{1.344993in}{1.024414in}}{\pgfqpoint{1.342679in}{1.018828in}}{\pgfqpoint{1.342679in}{1.013004in}}%
\pgfpathcurveto{\pgfqpoint{1.342679in}{1.007180in}}{\pgfqpoint{1.344993in}{1.001594in}}{\pgfqpoint{1.349111in}{0.997476in}}%
\pgfpathcurveto{\pgfqpoint{1.353229in}{0.993358in}}{\pgfqpoint{1.358816in}{0.991044in}}{\pgfqpoint{1.364640in}{0.991044in}}%
\pgfpathclose%
\pgfusepath{stroke,fill}%
\end{pgfscope}%
\begin{pgfscope}%
\pgfpathrectangle{\pgfqpoint{0.211875in}{0.211875in}}{\pgfqpoint{1.313625in}{1.279725in}}%
\pgfusepath{clip}%
\pgfsetbuttcap%
\pgfsetroundjoin%
\definecolor{currentfill}{rgb}{0.121569,0.466667,0.705882}%
\pgfsetfillcolor{currentfill}%
\pgfsetlinewidth{1.003750pt}%
\definecolor{currentstroke}{rgb}{0.121569,0.466667,0.705882}%
\pgfsetstrokecolor{currentstroke}%
\pgfsetdash{}{0pt}%
\pgfpathmoveto{\pgfqpoint{1.307779in}{0.975234in}}%
\pgfpathcurveto{\pgfqpoint{1.313603in}{0.975234in}}{\pgfqpoint{1.319189in}{0.977548in}}{\pgfqpoint{1.323307in}{0.981666in}}%
\pgfpathcurveto{\pgfqpoint{1.327425in}{0.985784in}}{\pgfqpoint{1.329739in}{0.991370in}}{\pgfqpoint{1.329739in}{0.997194in}}%
\pgfpathcurveto{\pgfqpoint{1.329739in}{1.003018in}}{\pgfqpoint{1.327425in}{1.008604in}}{\pgfqpoint{1.323307in}{1.012722in}}%
\pgfpathcurveto{\pgfqpoint{1.319189in}{1.016840in}}{\pgfqpoint{1.313603in}{1.019154in}}{\pgfqpoint{1.307779in}{1.019154in}}%
\pgfpathcurveto{\pgfqpoint{1.301955in}{1.019154in}}{\pgfqpoint{1.296369in}{1.016840in}}{\pgfqpoint{1.292251in}{1.012722in}}%
\pgfpathcurveto{\pgfqpoint{1.288133in}{1.008604in}}{\pgfqpoint{1.285819in}{1.003018in}}{\pgfqpoint{1.285819in}{0.997194in}}%
\pgfpathcurveto{\pgfqpoint{1.285819in}{0.991370in}}{\pgfqpoint{1.288133in}{0.985784in}}{\pgfqpoint{1.292251in}{0.981666in}}%
\pgfpathcurveto{\pgfqpoint{1.296369in}{0.977548in}}{\pgfqpoint{1.301955in}{0.975234in}}{\pgfqpoint{1.307779in}{0.975234in}}%
\pgfpathclose%
\pgfusepath{stroke,fill}%
\end{pgfscope}%
\begin{pgfscope}%
\pgfpathrectangle{\pgfqpoint{0.211875in}{0.211875in}}{\pgfqpoint{1.313625in}{1.279725in}}%
\pgfusepath{clip}%
\pgfsetbuttcap%
\pgfsetroundjoin%
\definecolor{currentfill}{rgb}{0.121569,0.466667,0.705882}%
\pgfsetfillcolor{currentfill}%
\pgfsetlinewidth{1.003750pt}%
\definecolor{currentstroke}{rgb}{0.121569,0.466667,0.705882}%
\pgfsetstrokecolor{currentstroke}%
\pgfsetdash{}{0pt}%
\pgfpathmoveto{\pgfqpoint{1.400632in}{0.933819in}}%
\pgfpathcurveto{\pgfqpoint{1.406456in}{0.933819in}}{\pgfqpoint{1.412042in}{0.936133in}}{\pgfqpoint{1.416161in}{0.940251in}}%
\pgfpathcurveto{\pgfqpoint{1.420279in}{0.944369in}}{\pgfqpoint{1.422593in}{0.949955in}}{\pgfqpoint{1.422593in}{0.955779in}}%
\pgfpathcurveto{\pgfqpoint{1.422593in}{0.961603in}}{\pgfqpoint{1.420279in}{0.967189in}}{\pgfqpoint{1.416161in}{0.971307in}}%
\pgfpathcurveto{\pgfqpoint{1.412042in}{0.975425in}}{\pgfqpoint{1.406456in}{0.977739in}}{\pgfqpoint{1.400632in}{0.977739in}}%
\pgfpathcurveto{\pgfqpoint{1.394808in}{0.977739in}}{\pgfqpoint{1.389222in}{0.975425in}}{\pgfqpoint{1.385104in}{0.971307in}}%
\pgfpathcurveto{\pgfqpoint{1.380986in}{0.967189in}}{\pgfqpoint{1.378672in}{0.961603in}}{\pgfqpoint{1.378672in}{0.955779in}}%
\pgfpathcurveto{\pgfqpoint{1.378672in}{0.949955in}}{\pgfqpoint{1.380986in}{0.944369in}}{\pgfqpoint{1.385104in}{0.940251in}}%
\pgfpathcurveto{\pgfqpoint{1.389222in}{0.936133in}}{\pgfqpoint{1.394808in}{0.933819in}}{\pgfqpoint{1.400632in}{0.933819in}}%
\pgfpathclose%
\pgfusepath{stroke,fill}%
\end{pgfscope}%
\begin{pgfscope}%
\pgfpathrectangle{\pgfqpoint{0.211875in}{0.211875in}}{\pgfqpoint{1.313625in}{1.279725in}}%
\pgfusepath{clip}%
\pgfsetbuttcap%
\pgfsetroundjoin%
\definecolor{currentfill}{rgb}{0.121569,0.466667,0.705882}%
\pgfsetfillcolor{currentfill}%
\pgfsetlinewidth{1.003750pt}%
\definecolor{currentstroke}{rgb}{0.121569,0.466667,0.705882}%
\pgfsetstrokecolor{currentstroke}%
\pgfsetdash{}{0pt}%
\pgfpathmoveto{\pgfqpoint{1.095123in}{1.080224in}}%
\pgfpathcurveto{\pgfqpoint{1.100947in}{1.080224in}}{\pgfqpoint{1.106533in}{1.082538in}}{\pgfqpoint{1.110651in}{1.086656in}}%
\pgfpathcurveto{\pgfqpoint{1.114769in}{1.090775in}}{\pgfqpoint{1.117083in}{1.096361in}}{\pgfqpoint{1.117083in}{1.102185in}}%
\pgfpathcurveto{\pgfqpoint{1.117083in}{1.108009in}}{\pgfqpoint{1.114769in}{1.113595in}}{\pgfqpoint{1.110651in}{1.117713in}}%
\pgfpathcurveto{\pgfqpoint{1.106533in}{1.121831in}}{\pgfqpoint{1.100947in}{1.124145in}}{\pgfqpoint{1.095123in}{1.124145in}}%
\pgfpathcurveto{\pgfqpoint{1.089299in}{1.124145in}}{\pgfqpoint{1.083713in}{1.121831in}}{\pgfqpoint{1.079595in}{1.117713in}}%
\pgfpathcurveto{\pgfqpoint{1.075476in}{1.113595in}}{\pgfqpoint{1.073163in}{1.108009in}}{\pgfqpoint{1.073163in}{1.102185in}}%
\pgfpathcurveto{\pgfqpoint{1.073163in}{1.096361in}}{\pgfqpoint{1.075476in}{1.090775in}}{\pgfqpoint{1.079595in}{1.086656in}}%
\pgfpathcurveto{\pgfqpoint{1.083713in}{1.082538in}}{\pgfqpoint{1.089299in}{1.080224in}}{\pgfqpoint{1.095123in}{1.080224in}}%
\pgfpathclose%
\pgfusepath{stroke,fill}%
\end{pgfscope}%
\begin{pgfscope}%
\pgfpathrectangle{\pgfqpoint{0.211875in}{0.211875in}}{\pgfqpoint{1.313625in}{1.279725in}}%
\pgfusepath{clip}%
\pgfsetbuttcap%
\pgfsetroundjoin%
\definecolor{currentfill}{rgb}{0.121569,0.466667,0.705882}%
\pgfsetfillcolor{currentfill}%
\pgfsetlinewidth{1.003750pt}%
\definecolor{currentstroke}{rgb}{0.121569,0.466667,0.705882}%
\pgfsetstrokecolor{currentstroke}%
\pgfsetdash{}{0pt}%
\pgfpathmoveto{\pgfqpoint{0.949152in}{0.971787in}}%
\pgfpathcurveto{\pgfqpoint{0.954976in}{0.971787in}}{\pgfqpoint{0.960562in}{0.974101in}}{\pgfqpoint{0.964680in}{0.978219in}}%
\pgfpathcurveto{\pgfqpoint{0.968798in}{0.982337in}}{\pgfqpoint{0.971112in}{0.987923in}}{\pgfqpoint{0.971112in}{0.993747in}}%
\pgfpathcurveto{\pgfqpoint{0.971112in}{0.999571in}}{\pgfqpoint{0.968798in}{1.005157in}}{\pgfqpoint{0.964680in}{1.009275in}}%
\pgfpathcurveto{\pgfqpoint{0.960562in}{1.013393in}}{\pgfqpoint{0.954976in}{1.015707in}}{\pgfqpoint{0.949152in}{1.015707in}}%
\pgfpathcurveto{\pgfqpoint{0.943328in}{1.015707in}}{\pgfqpoint{0.937742in}{1.013393in}}{\pgfqpoint{0.933624in}{1.009275in}}%
\pgfpathcurveto{\pgfqpoint{0.929505in}{1.005157in}}{\pgfqpoint{0.927192in}{0.999571in}}{\pgfqpoint{0.927192in}{0.993747in}}%
\pgfpathcurveto{\pgfqpoint{0.927192in}{0.987923in}}{\pgfqpoint{0.929505in}{0.982337in}}{\pgfqpoint{0.933624in}{0.978219in}}%
\pgfpathcurveto{\pgfqpoint{0.937742in}{0.974101in}}{\pgfqpoint{0.943328in}{0.971787in}}{\pgfqpoint{0.949152in}{0.971787in}}%
\pgfpathclose%
\pgfusepath{stroke,fill}%
\end{pgfscope}%
\begin{pgfscope}%
\pgfpathrectangle{\pgfqpoint{0.211875in}{0.211875in}}{\pgfqpoint{1.313625in}{1.279725in}}%
\pgfusepath{clip}%
\pgfsetbuttcap%
\pgfsetroundjoin%
\definecolor{currentfill}{rgb}{0.121569,0.466667,0.705882}%
\pgfsetfillcolor{currentfill}%
\pgfsetlinewidth{1.003750pt}%
\definecolor{currentstroke}{rgb}{0.121569,0.466667,0.705882}%
\pgfsetstrokecolor{currentstroke}%
\pgfsetdash{}{0pt}%
\pgfpathmoveto{\pgfqpoint{1.133892in}{1.148986in}}%
\pgfpathcurveto{\pgfqpoint{1.139716in}{1.148986in}}{\pgfqpoint{1.145302in}{1.151300in}}{\pgfqpoint{1.149420in}{1.155418in}}%
\pgfpathcurveto{\pgfqpoint{1.153538in}{1.159536in}}{\pgfqpoint{1.155852in}{1.165123in}}{\pgfqpoint{1.155852in}{1.170947in}}%
\pgfpathcurveto{\pgfqpoint{1.155852in}{1.176771in}}{\pgfqpoint{1.153538in}{1.182357in}}{\pgfqpoint{1.149420in}{1.186475in}}%
\pgfpathcurveto{\pgfqpoint{1.145302in}{1.190593in}}{\pgfqpoint{1.139716in}{1.192907in}}{\pgfqpoint{1.133892in}{1.192907in}}%
\pgfpathcurveto{\pgfqpoint{1.128068in}{1.192907in}}{\pgfqpoint{1.122482in}{1.190593in}}{\pgfqpoint{1.118364in}{1.186475in}}%
\pgfpathcurveto{\pgfqpoint{1.114246in}{1.182357in}}{\pgfqpoint{1.111932in}{1.176771in}}{\pgfqpoint{1.111932in}{1.170947in}}%
\pgfpathcurveto{\pgfqpoint{1.111932in}{1.165123in}}{\pgfqpoint{1.114246in}{1.159536in}}{\pgfqpoint{1.118364in}{1.155418in}}%
\pgfpathcurveto{\pgfqpoint{1.122482in}{1.151300in}}{\pgfqpoint{1.128068in}{1.148986in}}{\pgfqpoint{1.133892in}{1.148986in}}%
\pgfpathclose%
\pgfusepath{stroke,fill}%
\end{pgfscope}%
\begin{pgfscope}%
\pgfpathrectangle{\pgfqpoint{0.211875in}{0.211875in}}{\pgfqpoint{1.313625in}{1.279725in}}%
\pgfusepath{clip}%
\pgfsetbuttcap%
\pgfsetroundjoin%
\definecolor{currentfill}{rgb}{0.121569,0.466667,0.705882}%
\pgfsetfillcolor{currentfill}%
\pgfsetlinewidth{1.003750pt}%
\definecolor{currentstroke}{rgb}{0.121569,0.466667,0.705882}%
\pgfsetstrokecolor{currentstroke}%
\pgfsetdash{}{0pt}%
\pgfpathmoveto{\pgfqpoint{1.265821in}{1.152990in}}%
\pgfpathcurveto{\pgfqpoint{1.271645in}{1.152990in}}{\pgfqpoint{1.277231in}{1.155303in}}{\pgfqpoint{1.281349in}{1.159422in}}%
\pgfpathcurveto{\pgfqpoint{1.285467in}{1.163540in}}{\pgfqpoint{1.287781in}{1.169126in}}{\pgfqpoint{1.287781in}{1.174950in}}%
\pgfpathcurveto{\pgfqpoint{1.287781in}{1.180774in}}{\pgfqpoint{1.285467in}{1.186360in}}{\pgfqpoint{1.281349in}{1.190478in}}%
\pgfpathcurveto{\pgfqpoint{1.277231in}{1.194596in}}{\pgfqpoint{1.271645in}{1.196910in}}{\pgfqpoint{1.265821in}{1.196910in}}%
\pgfpathcurveto{\pgfqpoint{1.259997in}{1.196910in}}{\pgfqpoint{1.254411in}{1.194596in}}{\pgfqpoint{1.250293in}{1.190478in}}%
\pgfpathcurveto{\pgfqpoint{1.246175in}{1.186360in}}{\pgfqpoint{1.243861in}{1.180774in}}{\pgfqpoint{1.243861in}{1.174950in}}%
\pgfpathcurveto{\pgfqpoint{1.243861in}{1.169126in}}{\pgfqpoint{1.246175in}{1.163540in}}{\pgfqpoint{1.250293in}{1.159422in}}%
\pgfpathcurveto{\pgfqpoint{1.254411in}{1.155303in}}{\pgfqpoint{1.259997in}{1.152990in}}{\pgfqpoint{1.265821in}{1.152990in}}%
\pgfpathclose%
\pgfusepath{stroke,fill}%
\end{pgfscope}%
\begin{pgfscope}%
\pgfpathrectangle{\pgfqpoint{0.211875in}{0.211875in}}{\pgfqpoint{1.313625in}{1.279725in}}%
\pgfusepath{clip}%
\pgfsetbuttcap%
\pgfsetroundjoin%
\definecolor{currentfill}{rgb}{0.121569,0.466667,0.705882}%
\pgfsetfillcolor{currentfill}%
\pgfsetlinewidth{1.003750pt}%
\definecolor{currentstroke}{rgb}{0.121569,0.466667,0.705882}%
\pgfsetstrokecolor{currentstroke}%
\pgfsetdash{}{0pt}%
\pgfpathmoveto{\pgfqpoint{1.103631in}{1.079343in}}%
\pgfpathcurveto{\pgfqpoint{1.109455in}{1.079343in}}{\pgfqpoint{1.115041in}{1.081657in}}{\pgfqpoint{1.119159in}{1.085775in}}%
\pgfpathcurveto{\pgfqpoint{1.123277in}{1.089893in}}{\pgfqpoint{1.125591in}{1.095480in}}{\pgfqpoint{1.125591in}{1.101303in}}%
\pgfpathcurveto{\pgfqpoint{1.125591in}{1.107127in}}{\pgfqpoint{1.123277in}{1.112714in}}{\pgfqpoint{1.119159in}{1.116832in}}%
\pgfpathcurveto{\pgfqpoint{1.115041in}{1.120950in}}{\pgfqpoint{1.109455in}{1.123264in}}{\pgfqpoint{1.103631in}{1.123264in}}%
\pgfpathcurveto{\pgfqpoint{1.097807in}{1.123264in}}{\pgfqpoint{1.092221in}{1.120950in}}{\pgfqpoint{1.088103in}{1.116832in}}%
\pgfpathcurveto{\pgfqpoint{1.083984in}{1.112714in}}{\pgfqpoint{1.081671in}{1.107127in}}{\pgfqpoint{1.081671in}{1.101303in}}%
\pgfpathcurveto{\pgfqpoint{1.081671in}{1.095480in}}{\pgfqpoint{1.083984in}{1.089893in}}{\pgfqpoint{1.088103in}{1.085775in}}%
\pgfpathcurveto{\pgfqpoint{1.092221in}{1.081657in}}{\pgfqpoint{1.097807in}{1.079343in}}{\pgfqpoint{1.103631in}{1.079343in}}%
\pgfpathclose%
\pgfusepath{stroke,fill}%
\end{pgfscope}%
\begin{pgfscope}%
\pgfpathrectangle{\pgfqpoint{0.211875in}{0.211875in}}{\pgfqpoint{1.313625in}{1.279725in}}%
\pgfusepath{clip}%
\pgfsetbuttcap%
\pgfsetroundjoin%
\definecolor{currentfill}{rgb}{0.121569,0.466667,0.705882}%
\pgfsetfillcolor{currentfill}%
\pgfsetlinewidth{1.003750pt}%
\definecolor{currentstroke}{rgb}{0.121569,0.466667,0.705882}%
\pgfsetstrokecolor{currentstroke}%
\pgfsetdash{}{0pt}%
\pgfpathmoveto{\pgfqpoint{1.117677in}{1.087540in}}%
\pgfpathcurveto{\pgfqpoint{1.123501in}{1.087540in}}{\pgfqpoint{1.129087in}{1.089854in}}{\pgfqpoint{1.133205in}{1.093972in}}%
\pgfpathcurveto{\pgfqpoint{1.137324in}{1.098090in}}{\pgfqpoint{1.139637in}{1.103677in}}{\pgfqpoint{1.139637in}{1.109500in}}%
\pgfpathcurveto{\pgfqpoint{1.139637in}{1.115324in}}{\pgfqpoint{1.137324in}{1.120911in}}{\pgfqpoint{1.133205in}{1.125029in}}%
\pgfpathcurveto{\pgfqpoint{1.129087in}{1.129147in}}{\pgfqpoint{1.123501in}{1.131461in}}{\pgfqpoint{1.117677in}{1.131461in}}%
\pgfpathcurveto{\pgfqpoint{1.111853in}{1.131461in}}{\pgfqpoint{1.106267in}{1.129147in}}{\pgfqpoint{1.102149in}{1.125029in}}%
\pgfpathcurveto{\pgfqpoint{1.098031in}{1.120911in}}{\pgfqpoint{1.095717in}{1.115324in}}{\pgfqpoint{1.095717in}{1.109500in}}%
\pgfpathcurveto{\pgfqpoint{1.095717in}{1.103677in}}{\pgfqpoint{1.098031in}{1.098090in}}{\pgfqpoint{1.102149in}{1.093972in}}%
\pgfpathcurveto{\pgfqpoint{1.106267in}{1.089854in}}{\pgfqpoint{1.111853in}{1.087540in}}{\pgfqpoint{1.117677in}{1.087540in}}%
\pgfpathclose%
\pgfusepath{stroke,fill}%
\end{pgfscope}%
\begin{pgfscope}%
\pgfpathrectangle{\pgfqpoint{0.211875in}{0.211875in}}{\pgfqpoint{1.313625in}{1.279725in}}%
\pgfusepath{clip}%
\pgfsetbuttcap%
\pgfsetroundjoin%
\definecolor{currentfill}{rgb}{0.121569,0.466667,0.705882}%
\pgfsetfillcolor{currentfill}%
\pgfsetlinewidth{1.003750pt}%
\definecolor{currentstroke}{rgb}{0.121569,0.466667,0.705882}%
\pgfsetstrokecolor{currentstroke}%
\pgfsetdash{}{0pt}%
\pgfpathmoveto{\pgfqpoint{1.238887in}{1.158415in}}%
\pgfpathcurveto{\pgfqpoint{1.244711in}{1.158415in}}{\pgfqpoint{1.250297in}{1.160729in}}{\pgfqpoint{1.254415in}{1.164847in}}%
\pgfpathcurveto{\pgfqpoint{1.258533in}{1.168966in}}{\pgfqpoint{1.260847in}{1.174552in}}{\pgfqpoint{1.260847in}{1.180376in}}%
\pgfpathcurveto{\pgfqpoint{1.260847in}{1.186200in}}{\pgfqpoint{1.258533in}{1.191786in}}{\pgfqpoint{1.254415in}{1.195904in}}%
\pgfpathcurveto{\pgfqpoint{1.250297in}{1.200022in}}{\pgfqpoint{1.244711in}{1.202336in}}{\pgfqpoint{1.238887in}{1.202336in}}%
\pgfpathcurveto{\pgfqpoint{1.233063in}{1.202336in}}{\pgfqpoint{1.227477in}{1.200022in}}{\pgfqpoint{1.223359in}{1.195904in}}%
\pgfpathcurveto{\pgfqpoint{1.219241in}{1.191786in}}{\pgfqpoint{1.216927in}{1.186200in}}{\pgfqpoint{1.216927in}{1.180376in}}%
\pgfpathcurveto{\pgfqpoint{1.216927in}{1.174552in}}{\pgfqpoint{1.219241in}{1.168966in}}{\pgfqpoint{1.223359in}{1.164847in}}%
\pgfpathcurveto{\pgfqpoint{1.227477in}{1.160729in}}{\pgfqpoint{1.233063in}{1.158415in}}{\pgfqpoint{1.238887in}{1.158415in}}%
\pgfpathclose%
\pgfusepath{stroke,fill}%
\end{pgfscope}%
\begin{pgfscope}%
\pgfpathrectangle{\pgfqpoint{0.211875in}{0.211875in}}{\pgfqpoint{1.313625in}{1.279725in}}%
\pgfusepath{clip}%
\pgfsetbuttcap%
\pgfsetroundjoin%
\definecolor{currentfill}{rgb}{0.121569,0.466667,0.705882}%
\pgfsetfillcolor{currentfill}%
\pgfsetlinewidth{1.003750pt}%
\definecolor{currentstroke}{rgb}{0.121569,0.466667,0.705882}%
\pgfsetstrokecolor{currentstroke}%
\pgfsetdash{}{0pt}%
\pgfpathmoveto{\pgfqpoint{1.145819in}{0.785785in}}%
\pgfpathcurveto{\pgfqpoint{1.151643in}{0.785785in}}{\pgfqpoint{1.157229in}{0.788098in}}{\pgfqpoint{1.161347in}{0.792217in}}%
\pgfpathcurveto{\pgfqpoint{1.165465in}{0.796335in}}{\pgfqpoint{1.167779in}{0.801921in}}{\pgfqpoint{1.167779in}{0.807745in}}%
\pgfpathcurveto{\pgfqpoint{1.167779in}{0.813569in}}{\pgfqpoint{1.165465in}{0.819155in}}{\pgfqpoint{1.161347in}{0.823273in}}%
\pgfpathcurveto{\pgfqpoint{1.157229in}{0.827391in}}{\pgfqpoint{1.151643in}{0.829705in}}{\pgfqpoint{1.145819in}{0.829705in}}%
\pgfpathcurveto{\pgfqpoint{1.139995in}{0.829705in}}{\pgfqpoint{1.134409in}{0.827391in}}{\pgfqpoint{1.130291in}{0.823273in}}%
\pgfpathcurveto{\pgfqpoint{1.126173in}{0.819155in}}{\pgfqpoint{1.123859in}{0.813569in}}{\pgfqpoint{1.123859in}{0.807745in}}%
\pgfpathcurveto{\pgfqpoint{1.123859in}{0.801921in}}{\pgfqpoint{1.126173in}{0.796335in}}{\pgfqpoint{1.130291in}{0.792217in}}%
\pgfpathcurveto{\pgfqpoint{1.134409in}{0.788098in}}{\pgfqpoint{1.139995in}{0.785785in}}{\pgfqpoint{1.145819in}{0.785785in}}%
\pgfpathclose%
\pgfusepath{stroke,fill}%
\end{pgfscope}%
\begin{pgfscope}%
\pgfpathrectangle{\pgfqpoint{0.211875in}{0.211875in}}{\pgfqpoint{1.313625in}{1.279725in}}%
\pgfusepath{clip}%
\pgfsetbuttcap%
\pgfsetroundjoin%
\definecolor{currentfill}{rgb}{0.121569,0.466667,0.705882}%
\pgfsetfillcolor{currentfill}%
\pgfsetlinewidth{1.003750pt}%
\definecolor{currentstroke}{rgb}{0.121569,0.466667,0.705882}%
\pgfsetstrokecolor{currentstroke}%
\pgfsetdash{}{0pt}%
\pgfpathmoveto{\pgfqpoint{1.153390in}{1.247315in}}%
\pgfpathcurveto{\pgfqpoint{1.159214in}{1.247315in}}{\pgfqpoint{1.164800in}{1.249629in}}{\pgfqpoint{1.168918in}{1.253747in}}%
\pgfpathcurveto{\pgfqpoint{1.173036in}{1.257865in}}{\pgfqpoint{1.175350in}{1.263451in}}{\pgfqpoint{1.175350in}{1.269275in}}%
\pgfpathcurveto{\pgfqpoint{1.175350in}{1.275099in}}{\pgfqpoint{1.173036in}{1.280685in}}{\pgfqpoint{1.168918in}{1.284803in}}%
\pgfpathcurveto{\pgfqpoint{1.164800in}{1.288922in}}{\pgfqpoint{1.159214in}{1.291235in}}{\pgfqpoint{1.153390in}{1.291235in}}%
\pgfpathcurveto{\pgfqpoint{1.147566in}{1.291235in}}{\pgfqpoint{1.141980in}{1.288922in}}{\pgfqpoint{1.137862in}{1.284803in}}%
\pgfpathcurveto{\pgfqpoint{1.133744in}{1.280685in}}{\pgfqpoint{1.131430in}{1.275099in}}{\pgfqpoint{1.131430in}{1.269275in}}%
\pgfpathcurveto{\pgfqpoint{1.131430in}{1.263451in}}{\pgfqpoint{1.133744in}{1.257865in}}{\pgfqpoint{1.137862in}{1.253747in}}%
\pgfpathcurveto{\pgfqpoint{1.141980in}{1.249629in}}{\pgfqpoint{1.147566in}{1.247315in}}{\pgfqpoint{1.153390in}{1.247315in}}%
\pgfpathclose%
\pgfusepath{stroke,fill}%
\end{pgfscope}%
\begin{pgfscope}%
\pgfpathrectangle{\pgfqpoint{0.211875in}{0.211875in}}{\pgfqpoint{1.313625in}{1.279725in}}%
\pgfusepath{clip}%
\pgfsetbuttcap%
\pgfsetroundjoin%
\definecolor{currentfill}{rgb}{0.121569,0.466667,0.705882}%
\pgfsetfillcolor{currentfill}%
\pgfsetlinewidth{1.003750pt}%
\definecolor{currentstroke}{rgb}{0.121569,0.466667,0.705882}%
\pgfsetstrokecolor{currentstroke}%
\pgfsetdash{}{0pt}%
\pgfpathmoveto{\pgfqpoint{0.923991in}{0.952014in}}%
\pgfpathcurveto{\pgfqpoint{0.929815in}{0.952014in}}{\pgfqpoint{0.935401in}{0.954328in}}{\pgfqpoint{0.939520in}{0.958446in}}%
\pgfpathcurveto{\pgfqpoint{0.943638in}{0.962564in}}{\pgfqpoint{0.945952in}{0.968150in}}{\pgfqpoint{0.945952in}{0.973974in}}%
\pgfpathcurveto{\pgfqpoint{0.945952in}{0.979798in}}{\pgfqpoint{0.943638in}{0.985384in}}{\pgfqpoint{0.939520in}{0.989502in}}%
\pgfpathcurveto{\pgfqpoint{0.935401in}{0.993620in}}{\pgfqpoint{0.929815in}{0.995934in}}{\pgfqpoint{0.923991in}{0.995934in}}%
\pgfpathcurveto{\pgfqpoint{0.918167in}{0.995934in}}{\pgfqpoint{0.912581in}{0.993620in}}{\pgfqpoint{0.908463in}{0.989502in}}%
\pgfpathcurveto{\pgfqpoint{0.904345in}{0.985384in}}{\pgfqpoint{0.902031in}{0.979798in}}{\pgfqpoint{0.902031in}{0.973974in}}%
\pgfpathcurveto{\pgfqpoint{0.902031in}{0.968150in}}{\pgfqpoint{0.904345in}{0.962564in}}{\pgfqpoint{0.908463in}{0.958446in}}%
\pgfpathcurveto{\pgfqpoint{0.912581in}{0.954328in}}{\pgfqpoint{0.918167in}{0.952014in}}{\pgfqpoint{0.923991in}{0.952014in}}%
\pgfpathclose%
\pgfusepath{stroke,fill}%
\end{pgfscope}%
\begin{pgfscope}%
\pgfpathrectangle{\pgfqpoint{0.211875in}{0.211875in}}{\pgfqpoint{1.313625in}{1.279725in}}%
\pgfusepath{clip}%
\pgfsetbuttcap%
\pgfsetroundjoin%
\definecolor{currentfill}{rgb}{0.121569,0.466667,0.705882}%
\pgfsetfillcolor{currentfill}%
\pgfsetlinewidth{1.003750pt}%
\definecolor{currentstroke}{rgb}{0.121569,0.466667,0.705882}%
\pgfsetstrokecolor{currentstroke}%
\pgfsetdash{}{0pt}%
\pgfpathmoveto{\pgfqpoint{1.058895in}{1.328086in}}%
\pgfpathcurveto{\pgfqpoint{1.064718in}{1.328086in}}{\pgfqpoint{1.070305in}{1.330400in}}{\pgfqpoint{1.074423in}{1.334518in}}%
\pgfpathcurveto{\pgfqpoint{1.078541in}{1.338636in}}{\pgfqpoint{1.080855in}{1.344222in}}{\pgfqpoint{1.080855in}{1.350046in}}%
\pgfpathcurveto{\pgfqpoint{1.080855in}{1.355870in}}{\pgfqpoint{1.078541in}{1.361456in}}{\pgfqpoint{1.074423in}{1.365574in}}%
\pgfpathcurveto{\pgfqpoint{1.070305in}{1.369693in}}{\pgfqpoint{1.064718in}{1.372006in}}{\pgfqpoint{1.058895in}{1.372006in}}%
\pgfpathcurveto{\pgfqpoint{1.053071in}{1.372006in}}{\pgfqpoint{1.047484in}{1.369693in}}{\pgfqpoint{1.043366in}{1.365574in}}%
\pgfpathcurveto{\pgfqpoint{1.039248in}{1.361456in}}{\pgfqpoint{1.036934in}{1.355870in}}{\pgfqpoint{1.036934in}{1.350046in}}%
\pgfpathcurveto{\pgfqpoint{1.036934in}{1.344222in}}{\pgfqpoint{1.039248in}{1.338636in}}{\pgfqpoint{1.043366in}{1.334518in}}%
\pgfpathcurveto{\pgfqpoint{1.047484in}{1.330400in}}{\pgfqpoint{1.053071in}{1.328086in}}{\pgfqpoint{1.058895in}{1.328086in}}%
\pgfpathclose%
\pgfusepath{stroke,fill}%
\end{pgfscope}%
\begin{pgfscope}%
\pgfpathrectangle{\pgfqpoint{0.211875in}{0.211875in}}{\pgfqpoint{1.313625in}{1.279725in}}%
\pgfusepath{clip}%
\pgfsetbuttcap%
\pgfsetroundjoin%
\definecolor{currentfill}{rgb}{0.121569,0.466667,0.705882}%
\pgfsetfillcolor{currentfill}%
\pgfsetlinewidth{1.003750pt}%
\definecolor{currentstroke}{rgb}{0.121569,0.466667,0.705882}%
\pgfsetstrokecolor{currentstroke}%
\pgfsetdash{}{0pt}%
\pgfpathmoveto{\pgfqpoint{1.039685in}{1.157208in}}%
\pgfpathcurveto{\pgfqpoint{1.045509in}{1.157208in}}{\pgfqpoint{1.051095in}{1.159522in}}{\pgfqpoint{1.055213in}{1.163640in}}%
\pgfpathcurveto{\pgfqpoint{1.059331in}{1.167758in}}{\pgfqpoint{1.061645in}{1.173344in}}{\pgfqpoint{1.061645in}{1.179168in}}%
\pgfpathcurveto{\pgfqpoint{1.061645in}{1.184992in}}{\pgfqpoint{1.059331in}{1.190578in}}{\pgfqpoint{1.055213in}{1.194697in}}%
\pgfpathcurveto{\pgfqpoint{1.051095in}{1.198815in}}{\pgfqpoint{1.045509in}{1.201129in}}{\pgfqpoint{1.039685in}{1.201129in}}%
\pgfpathcurveto{\pgfqpoint{1.033861in}{1.201129in}}{\pgfqpoint{1.028275in}{1.198815in}}{\pgfqpoint{1.024156in}{1.194697in}}%
\pgfpathcurveto{\pgfqpoint{1.020038in}{1.190578in}}{\pgfqpoint{1.017724in}{1.184992in}}{\pgfqpoint{1.017724in}{1.179168in}}%
\pgfpathcurveto{\pgfqpoint{1.017724in}{1.173344in}}{\pgfqpoint{1.020038in}{1.167758in}}{\pgfqpoint{1.024156in}{1.163640in}}%
\pgfpathcurveto{\pgfqpoint{1.028275in}{1.159522in}}{\pgfqpoint{1.033861in}{1.157208in}}{\pgfqpoint{1.039685in}{1.157208in}}%
\pgfpathclose%
\pgfusepath{stroke,fill}%
\end{pgfscope}%
\begin{pgfscope}%
\pgfpathrectangle{\pgfqpoint{0.211875in}{0.211875in}}{\pgfqpoint{1.313625in}{1.279725in}}%
\pgfusepath{clip}%
\pgfsetbuttcap%
\pgfsetroundjoin%
\definecolor{currentfill}{rgb}{0.121569,0.466667,0.705882}%
\pgfsetfillcolor{currentfill}%
\pgfsetlinewidth{1.003750pt}%
\definecolor{currentstroke}{rgb}{0.121569,0.466667,0.705882}%
\pgfsetstrokecolor{currentstroke}%
\pgfsetdash{}{0pt}%
\pgfpathmoveto{\pgfqpoint{1.356436in}{1.181815in}}%
\pgfpathcurveto{\pgfqpoint{1.362260in}{1.181815in}}{\pgfqpoint{1.367846in}{1.184129in}}{\pgfqpoint{1.371965in}{1.188247in}}%
\pgfpathcurveto{\pgfqpoint{1.376083in}{1.192365in}}{\pgfqpoint{1.378397in}{1.197952in}}{\pgfqpoint{1.378397in}{1.203776in}}%
\pgfpathcurveto{\pgfqpoint{1.378397in}{1.209600in}}{\pgfqpoint{1.376083in}{1.215186in}}{\pgfqpoint{1.371965in}{1.219304in}}%
\pgfpathcurveto{\pgfqpoint{1.367846in}{1.223422in}}{\pgfqpoint{1.362260in}{1.225736in}}{\pgfqpoint{1.356436in}{1.225736in}}%
\pgfpathcurveto{\pgfqpoint{1.350612in}{1.225736in}}{\pgfqpoint{1.345026in}{1.223422in}}{\pgfqpoint{1.340908in}{1.219304in}}%
\pgfpathcurveto{\pgfqpoint{1.336790in}{1.215186in}}{\pgfqpoint{1.334476in}{1.209600in}}{\pgfqpoint{1.334476in}{1.203776in}}%
\pgfpathcurveto{\pgfqpoint{1.334476in}{1.197952in}}{\pgfqpoint{1.336790in}{1.192365in}}{\pgfqpoint{1.340908in}{1.188247in}}%
\pgfpathcurveto{\pgfqpoint{1.345026in}{1.184129in}}{\pgfqpoint{1.350612in}{1.181815in}}{\pgfqpoint{1.356436in}{1.181815in}}%
\pgfpathclose%
\pgfusepath{stroke,fill}%
\end{pgfscope}%
\begin{pgfscope}%
\pgfpathrectangle{\pgfqpoint{0.211875in}{0.211875in}}{\pgfqpoint{1.313625in}{1.279725in}}%
\pgfusepath{clip}%
\pgfsetbuttcap%
\pgfsetroundjoin%
\definecolor{currentfill}{rgb}{0.121569,0.466667,0.705882}%
\pgfsetfillcolor{currentfill}%
\pgfsetlinewidth{1.003750pt}%
\definecolor{currentstroke}{rgb}{0.121569,0.466667,0.705882}%
\pgfsetstrokecolor{currentstroke}%
\pgfsetdash{}{0pt}%
\pgfpathmoveto{\pgfqpoint{1.113386in}{1.049651in}}%
\pgfpathcurveto{\pgfqpoint{1.119210in}{1.049651in}}{\pgfqpoint{1.124797in}{1.051965in}}{\pgfqpoint{1.128915in}{1.056083in}}%
\pgfpathcurveto{\pgfqpoint{1.133033in}{1.060201in}}{\pgfqpoint{1.135347in}{1.065787in}}{\pgfqpoint{1.135347in}{1.071611in}}%
\pgfpathcurveto{\pgfqpoint{1.135347in}{1.077435in}}{\pgfqpoint{1.133033in}{1.083021in}}{\pgfqpoint{1.128915in}{1.087140in}}%
\pgfpathcurveto{\pgfqpoint{1.124797in}{1.091258in}}{\pgfqpoint{1.119210in}{1.093572in}}{\pgfqpoint{1.113386in}{1.093572in}}%
\pgfpathcurveto{\pgfqpoint{1.107563in}{1.093572in}}{\pgfqpoint{1.101976in}{1.091258in}}{\pgfqpoint{1.097858in}{1.087140in}}%
\pgfpathcurveto{\pgfqpoint{1.093740in}{1.083021in}}{\pgfqpoint{1.091426in}{1.077435in}}{\pgfqpoint{1.091426in}{1.071611in}}%
\pgfpathcurveto{\pgfqpoint{1.091426in}{1.065787in}}{\pgfqpoint{1.093740in}{1.060201in}}{\pgfqpoint{1.097858in}{1.056083in}}%
\pgfpathcurveto{\pgfqpoint{1.101976in}{1.051965in}}{\pgfqpoint{1.107563in}{1.049651in}}{\pgfqpoint{1.113386in}{1.049651in}}%
\pgfpathclose%
\pgfusepath{stroke,fill}%
\end{pgfscope}%
\begin{pgfscope}%
\pgfpathrectangle{\pgfqpoint{0.211875in}{0.211875in}}{\pgfqpoint{1.313625in}{1.279725in}}%
\pgfusepath{clip}%
\pgfsetbuttcap%
\pgfsetroundjoin%
\definecolor{currentfill}{rgb}{0.121569,0.466667,0.705882}%
\pgfsetfillcolor{currentfill}%
\pgfsetlinewidth{1.003750pt}%
\definecolor{currentstroke}{rgb}{0.121569,0.466667,0.705882}%
\pgfsetstrokecolor{currentstroke}%
\pgfsetdash{}{0pt}%
\pgfpathmoveto{\pgfqpoint{0.799388in}{1.323404in}}%
\pgfpathcurveto{\pgfqpoint{0.805212in}{1.323404in}}{\pgfqpoint{0.810799in}{1.325718in}}{\pgfqpoint{0.814917in}{1.329836in}}%
\pgfpathcurveto{\pgfqpoint{0.819035in}{1.333954in}}{\pgfqpoint{0.821349in}{1.339540in}}{\pgfqpoint{0.821349in}{1.345364in}}%
\pgfpathcurveto{\pgfqpoint{0.821349in}{1.351188in}}{\pgfqpoint{0.819035in}{1.356775in}}{\pgfqpoint{0.814917in}{1.360893in}}%
\pgfpathcurveto{\pgfqpoint{0.810799in}{1.365011in}}{\pgfqpoint{0.805212in}{1.367325in}}{\pgfqpoint{0.799388in}{1.367325in}}%
\pgfpathcurveto{\pgfqpoint{0.793564in}{1.367325in}}{\pgfqpoint{0.787978in}{1.365011in}}{\pgfqpoint{0.783860in}{1.360893in}}%
\pgfpathcurveto{\pgfqpoint{0.779742in}{1.356775in}}{\pgfqpoint{0.777428in}{1.351188in}}{\pgfqpoint{0.777428in}{1.345364in}}%
\pgfpathcurveto{\pgfqpoint{0.777428in}{1.339540in}}{\pgfqpoint{0.779742in}{1.333954in}}{\pgfqpoint{0.783860in}{1.329836in}}%
\pgfpathcurveto{\pgfqpoint{0.787978in}{1.325718in}}{\pgfqpoint{0.793564in}{1.323404in}}{\pgfqpoint{0.799388in}{1.323404in}}%
\pgfpathclose%
\pgfusepath{stroke,fill}%
\end{pgfscope}%
\begin{pgfscope}%
\pgfpathrectangle{\pgfqpoint{0.211875in}{0.211875in}}{\pgfqpoint{1.313625in}{1.279725in}}%
\pgfusepath{clip}%
\pgfsetbuttcap%
\pgfsetroundjoin%
\definecolor{currentfill}{rgb}{0.121569,0.466667,0.705882}%
\pgfsetfillcolor{currentfill}%
\pgfsetlinewidth{1.003750pt}%
\definecolor{currentstroke}{rgb}{0.121569,0.466667,0.705882}%
\pgfsetstrokecolor{currentstroke}%
\pgfsetdash{}{0pt}%
\pgfpathmoveto{\pgfqpoint{1.446373in}{0.823862in}}%
\pgfpathcurveto{\pgfqpoint{1.452197in}{0.823862in}}{\pgfqpoint{1.457783in}{0.826176in}}{\pgfqpoint{1.461901in}{0.830294in}}%
\pgfpathcurveto{\pgfqpoint{1.466019in}{0.834412in}}{\pgfqpoint{1.468333in}{0.839998in}}{\pgfqpoint{1.468333in}{0.845822in}}%
\pgfpathcurveto{\pgfqpoint{1.468333in}{0.851646in}}{\pgfqpoint{1.466019in}{0.857232in}}{\pgfqpoint{1.461901in}{0.861350in}}%
\pgfpathcurveto{\pgfqpoint{1.457783in}{0.865469in}}{\pgfqpoint{1.452197in}{0.867782in}}{\pgfqpoint{1.446373in}{0.867782in}}%
\pgfpathcurveto{\pgfqpoint{1.440549in}{0.867782in}}{\pgfqpoint{1.434963in}{0.865469in}}{\pgfqpoint{1.430845in}{0.861350in}}%
\pgfpathcurveto{\pgfqpoint{1.426726in}{0.857232in}}{\pgfqpoint{1.424413in}{0.851646in}}{\pgfqpoint{1.424413in}{0.845822in}}%
\pgfpathcurveto{\pgfqpoint{1.424413in}{0.839998in}}{\pgfqpoint{1.426726in}{0.834412in}}{\pgfqpoint{1.430845in}{0.830294in}}%
\pgfpathcurveto{\pgfqpoint{1.434963in}{0.826176in}}{\pgfqpoint{1.440549in}{0.823862in}}{\pgfqpoint{1.446373in}{0.823862in}}%
\pgfpathclose%
\pgfusepath{stroke,fill}%
\end{pgfscope}%
\begin{pgfscope}%
\pgfpathrectangle{\pgfqpoint{0.211875in}{0.211875in}}{\pgfqpoint{1.313625in}{1.279725in}}%
\pgfusepath{clip}%
\pgfsetbuttcap%
\pgfsetroundjoin%
\definecolor{currentfill}{rgb}{0.121569,0.466667,0.705882}%
\pgfsetfillcolor{currentfill}%
\pgfsetlinewidth{1.003750pt}%
\definecolor{currentstroke}{rgb}{0.121569,0.466667,0.705882}%
\pgfsetstrokecolor{currentstroke}%
\pgfsetdash{}{0pt}%
\pgfpathmoveto{\pgfqpoint{1.282392in}{1.148385in}}%
\pgfpathcurveto{\pgfqpoint{1.288215in}{1.148385in}}{\pgfqpoint{1.293802in}{1.150699in}}{\pgfqpoint{1.297920in}{1.154817in}}%
\pgfpathcurveto{\pgfqpoint{1.302038in}{1.158936in}}{\pgfqpoint{1.304352in}{1.164522in}}{\pgfqpoint{1.304352in}{1.170346in}}%
\pgfpathcurveto{\pgfqpoint{1.304352in}{1.176170in}}{\pgfqpoint{1.302038in}{1.181756in}}{\pgfqpoint{1.297920in}{1.185874in}}%
\pgfpathcurveto{\pgfqpoint{1.293802in}{1.189992in}}{\pgfqpoint{1.288215in}{1.192306in}}{\pgfqpoint{1.282392in}{1.192306in}}%
\pgfpathcurveto{\pgfqpoint{1.276568in}{1.192306in}}{\pgfqpoint{1.270981in}{1.189992in}}{\pgfqpoint{1.266863in}{1.185874in}}%
\pgfpathcurveto{\pgfqpoint{1.262745in}{1.181756in}}{\pgfqpoint{1.260431in}{1.176170in}}{\pgfqpoint{1.260431in}{1.170346in}}%
\pgfpathcurveto{\pgfqpoint{1.260431in}{1.164522in}}{\pgfqpoint{1.262745in}{1.158936in}}{\pgfqpoint{1.266863in}{1.154817in}}%
\pgfpathcurveto{\pgfqpoint{1.270981in}{1.150699in}}{\pgfqpoint{1.276568in}{1.148385in}}{\pgfqpoint{1.282392in}{1.148385in}}%
\pgfpathclose%
\pgfusepath{stroke,fill}%
\end{pgfscope}%
\begin{pgfscope}%
\pgfpathrectangle{\pgfqpoint{0.211875in}{0.211875in}}{\pgfqpoint{1.313625in}{1.279725in}}%
\pgfusepath{clip}%
\pgfsetbuttcap%
\pgfsetroundjoin%
\definecolor{currentfill}{rgb}{0.121569,0.466667,0.705882}%
\pgfsetfillcolor{currentfill}%
\pgfsetlinewidth{1.003750pt}%
\definecolor{currentstroke}{rgb}{0.121569,0.466667,0.705882}%
\pgfsetstrokecolor{currentstroke}%
\pgfsetdash{}{0pt}%
\pgfpathmoveto{\pgfqpoint{1.110835in}{1.064678in}}%
\pgfpathcurveto{\pgfqpoint{1.116659in}{1.064678in}}{\pgfqpoint{1.122245in}{1.066992in}}{\pgfqpoint{1.126363in}{1.071110in}}%
\pgfpathcurveto{\pgfqpoint{1.130481in}{1.075229in}}{\pgfqpoint{1.132795in}{1.080815in}}{\pgfqpoint{1.132795in}{1.086639in}}%
\pgfpathcurveto{\pgfqpoint{1.132795in}{1.092463in}}{\pgfqpoint{1.130481in}{1.098049in}}{\pgfqpoint{1.126363in}{1.102167in}}%
\pgfpathcurveto{\pgfqpoint{1.122245in}{1.106285in}}{\pgfqpoint{1.116659in}{1.108599in}}{\pgfqpoint{1.110835in}{1.108599in}}%
\pgfpathcurveto{\pgfqpoint{1.105011in}{1.108599in}}{\pgfqpoint{1.099425in}{1.106285in}}{\pgfqpoint{1.095307in}{1.102167in}}%
\pgfpathcurveto{\pgfqpoint{1.091189in}{1.098049in}}{\pgfqpoint{1.088875in}{1.092463in}}{\pgfqpoint{1.088875in}{1.086639in}}%
\pgfpathcurveto{\pgfqpoint{1.088875in}{1.080815in}}{\pgfqpoint{1.091189in}{1.075229in}}{\pgfqpoint{1.095307in}{1.071110in}}%
\pgfpathcurveto{\pgfqpoint{1.099425in}{1.066992in}}{\pgfqpoint{1.105011in}{1.064678in}}{\pgfqpoint{1.110835in}{1.064678in}}%
\pgfpathclose%
\pgfusepath{stroke,fill}%
\end{pgfscope}%
\begin{pgfscope}%
\pgfpathrectangle{\pgfqpoint{0.211875in}{0.211875in}}{\pgfqpoint{1.313625in}{1.279725in}}%
\pgfusepath{clip}%
\pgfsetbuttcap%
\pgfsetroundjoin%
\definecolor{currentfill}{rgb}{0.121569,0.466667,0.705882}%
\pgfsetfillcolor{currentfill}%
\pgfsetlinewidth{1.003750pt}%
\definecolor{currentstroke}{rgb}{0.121569,0.466667,0.705882}%
\pgfsetstrokecolor{currentstroke}%
\pgfsetdash{}{0pt}%
\pgfpathmoveto{\pgfqpoint{0.927452in}{1.017551in}}%
\pgfpathcurveto{\pgfqpoint{0.933276in}{1.017551in}}{\pgfqpoint{0.938862in}{1.019865in}}{\pgfqpoint{0.942980in}{1.023983in}}%
\pgfpathcurveto{\pgfqpoint{0.947099in}{1.028101in}}{\pgfqpoint{0.949412in}{1.033688in}}{\pgfqpoint{0.949412in}{1.039511in}}%
\pgfpathcurveto{\pgfqpoint{0.949412in}{1.045335in}}{\pgfqpoint{0.947099in}{1.050922in}}{\pgfqpoint{0.942980in}{1.055040in}}%
\pgfpathcurveto{\pgfqpoint{0.938862in}{1.059158in}}{\pgfqpoint{0.933276in}{1.061472in}}{\pgfqpoint{0.927452in}{1.061472in}}%
\pgfpathcurveto{\pgfqpoint{0.921628in}{1.061472in}}{\pgfqpoint{0.916042in}{1.059158in}}{\pgfqpoint{0.911924in}{1.055040in}}%
\pgfpathcurveto{\pgfqpoint{0.907806in}{1.050922in}}{\pgfqpoint{0.905492in}{1.045335in}}{\pgfqpoint{0.905492in}{1.039511in}}%
\pgfpathcurveto{\pgfqpoint{0.905492in}{1.033688in}}{\pgfqpoint{0.907806in}{1.028101in}}{\pgfqpoint{0.911924in}{1.023983in}}%
\pgfpathcurveto{\pgfqpoint{0.916042in}{1.019865in}}{\pgfqpoint{0.921628in}{1.017551in}}{\pgfqpoint{0.927452in}{1.017551in}}%
\pgfpathclose%
\pgfusepath{stroke,fill}%
\end{pgfscope}%
\begin{pgfscope}%
\pgfpathrectangle{\pgfqpoint{0.211875in}{0.211875in}}{\pgfqpoint{1.313625in}{1.279725in}}%
\pgfusepath{clip}%
\pgfsetbuttcap%
\pgfsetroundjoin%
\definecolor{currentfill}{rgb}{0.121569,0.466667,0.705882}%
\pgfsetfillcolor{currentfill}%
\pgfsetlinewidth{1.003750pt}%
\definecolor{currentstroke}{rgb}{0.121569,0.466667,0.705882}%
\pgfsetstrokecolor{currentstroke}%
\pgfsetdash{}{0pt}%
\pgfpathmoveto{\pgfqpoint{1.098683in}{1.067570in}}%
\pgfpathcurveto{\pgfqpoint{1.104507in}{1.067570in}}{\pgfqpoint{1.110093in}{1.069884in}}{\pgfqpoint{1.114211in}{1.074002in}}%
\pgfpathcurveto{\pgfqpoint{1.118329in}{1.078120in}}{\pgfqpoint{1.120643in}{1.083706in}}{\pgfqpoint{1.120643in}{1.089530in}}%
\pgfpathcurveto{\pgfqpoint{1.120643in}{1.095354in}}{\pgfqpoint{1.118329in}{1.100940in}}{\pgfqpoint{1.114211in}{1.105058in}}%
\pgfpathcurveto{\pgfqpoint{1.110093in}{1.109177in}}{\pgfqpoint{1.104507in}{1.111490in}}{\pgfqpoint{1.098683in}{1.111490in}}%
\pgfpathcurveto{\pgfqpoint{1.092859in}{1.111490in}}{\pgfqpoint{1.087273in}{1.109177in}}{\pgfqpoint{1.083155in}{1.105058in}}%
\pgfpathcurveto{\pgfqpoint{1.079036in}{1.100940in}}{\pgfqpoint{1.076723in}{1.095354in}}{\pgfqpoint{1.076723in}{1.089530in}}%
\pgfpathcurveto{\pgfqpoint{1.076723in}{1.083706in}}{\pgfqpoint{1.079036in}{1.078120in}}{\pgfqpoint{1.083155in}{1.074002in}}%
\pgfpathcurveto{\pgfqpoint{1.087273in}{1.069884in}}{\pgfqpoint{1.092859in}{1.067570in}}{\pgfqpoint{1.098683in}{1.067570in}}%
\pgfpathclose%
\pgfusepath{stroke,fill}%
\end{pgfscope}%
\begin{pgfscope}%
\pgfpathrectangle{\pgfqpoint{0.211875in}{0.211875in}}{\pgfqpoint{1.313625in}{1.279725in}}%
\pgfusepath{clip}%
\pgfsetbuttcap%
\pgfsetroundjoin%
\definecolor{currentfill}{rgb}{0.121569,0.466667,0.705882}%
\pgfsetfillcolor{currentfill}%
\pgfsetlinewidth{1.003750pt}%
\definecolor{currentstroke}{rgb}{0.121569,0.466667,0.705882}%
\pgfsetstrokecolor{currentstroke}%
\pgfsetdash{}{0pt}%
\pgfpathmoveto{\pgfqpoint{1.277091in}{1.160976in}}%
\pgfpathcurveto{\pgfqpoint{1.282915in}{1.160976in}}{\pgfqpoint{1.288501in}{1.163290in}}{\pgfqpoint{1.292619in}{1.167408in}}%
\pgfpathcurveto{\pgfqpoint{1.296737in}{1.171527in}}{\pgfqpoint{1.299051in}{1.177113in}}{\pgfqpoint{1.299051in}{1.182937in}}%
\pgfpathcurveto{\pgfqpoint{1.299051in}{1.188761in}}{\pgfqpoint{1.296737in}{1.194347in}}{\pgfqpoint{1.292619in}{1.198465in}}%
\pgfpathcurveto{\pgfqpoint{1.288501in}{1.202583in}}{\pgfqpoint{1.282915in}{1.204897in}}{\pgfqpoint{1.277091in}{1.204897in}}%
\pgfpathcurveto{\pgfqpoint{1.271267in}{1.204897in}}{\pgfqpoint{1.265681in}{1.202583in}}{\pgfqpoint{1.261563in}{1.198465in}}%
\pgfpathcurveto{\pgfqpoint{1.257445in}{1.194347in}}{\pgfqpoint{1.255131in}{1.188761in}}{\pgfqpoint{1.255131in}{1.182937in}}%
\pgfpathcurveto{\pgfqpoint{1.255131in}{1.177113in}}{\pgfqpoint{1.257445in}{1.171527in}}{\pgfqpoint{1.261563in}{1.167408in}}%
\pgfpathcurveto{\pgfqpoint{1.265681in}{1.163290in}}{\pgfqpoint{1.271267in}{1.160976in}}{\pgfqpoint{1.277091in}{1.160976in}}%
\pgfpathclose%
\pgfusepath{stroke,fill}%
\end{pgfscope}%
\begin{pgfscope}%
\pgfpathrectangle{\pgfqpoint{0.211875in}{0.211875in}}{\pgfqpoint{1.313625in}{1.279725in}}%
\pgfusepath{clip}%
\pgfsetbuttcap%
\pgfsetroundjoin%
\definecolor{currentfill}{rgb}{0.121569,0.466667,0.705882}%
\pgfsetfillcolor{currentfill}%
\pgfsetlinewidth{1.003750pt}%
\definecolor{currentstroke}{rgb}{0.121569,0.466667,0.705882}%
\pgfsetstrokecolor{currentstroke}%
\pgfsetdash{}{0pt}%
\pgfpathmoveto{\pgfqpoint{0.909087in}{0.933962in}}%
\pgfpathcurveto{\pgfqpoint{0.914911in}{0.933962in}}{\pgfqpoint{0.920497in}{0.936275in}}{\pgfqpoint{0.924616in}{0.940394in}}%
\pgfpathcurveto{\pgfqpoint{0.928734in}{0.944512in}}{\pgfqpoint{0.931048in}{0.950098in}}{\pgfqpoint{0.931048in}{0.955922in}}%
\pgfpathcurveto{\pgfqpoint{0.931048in}{0.961746in}}{\pgfqpoint{0.928734in}{0.967332in}}{\pgfqpoint{0.924616in}{0.971450in}}%
\pgfpathcurveto{\pgfqpoint{0.920497in}{0.975568in}}{\pgfqpoint{0.914911in}{0.977882in}}{\pgfqpoint{0.909087in}{0.977882in}}%
\pgfpathcurveto{\pgfqpoint{0.903263in}{0.977882in}}{\pgfqpoint{0.897677in}{0.975568in}}{\pgfqpoint{0.893559in}{0.971450in}}%
\pgfpathcurveto{\pgfqpoint{0.889441in}{0.967332in}}{\pgfqpoint{0.887127in}{0.961746in}}{\pgfqpoint{0.887127in}{0.955922in}}%
\pgfpathcurveto{\pgfqpoint{0.887127in}{0.950098in}}{\pgfqpoint{0.889441in}{0.944512in}}{\pgfqpoint{0.893559in}{0.940394in}}%
\pgfpathcurveto{\pgfqpoint{0.897677in}{0.936275in}}{\pgfqpoint{0.903263in}{0.933962in}}{\pgfqpoint{0.909087in}{0.933962in}}%
\pgfpathclose%
\pgfusepath{stroke,fill}%
\end{pgfscope}%
\begin{pgfscope}%
\pgfpathrectangle{\pgfqpoint{0.211875in}{0.211875in}}{\pgfqpoint{1.313625in}{1.279725in}}%
\pgfusepath{clip}%
\pgfsetbuttcap%
\pgfsetroundjoin%
\definecolor{currentfill}{rgb}{0.121569,0.466667,0.705882}%
\pgfsetfillcolor{currentfill}%
\pgfsetlinewidth{1.003750pt}%
\definecolor{currentstroke}{rgb}{0.121569,0.466667,0.705882}%
\pgfsetstrokecolor{currentstroke}%
\pgfsetdash{}{0pt}%
\pgfpathmoveto{\pgfqpoint{0.906726in}{0.920514in}}%
\pgfpathcurveto{\pgfqpoint{0.912550in}{0.920514in}}{\pgfqpoint{0.918136in}{0.922827in}}{\pgfqpoint{0.922254in}{0.926946in}}%
\pgfpathcurveto{\pgfqpoint{0.926372in}{0.931064in}}{\pgfqpoint{0.928686in}{0.936650in}}{\pgfqpoint{0.928686in}{0.942474in}}%
\pgfpathcurveto{\pgfqpoint{0.928686in}{0.948298in}}{\pgfqpoint{0.926372in}{0.953884in}}{\pgfqpoint{0.922254in}{0.958002in}}%
\pgfpathcurveto{\pgfqpoint{0.918136in}{0.962120in}}{\pgfqpoint{0.912550in}{0.964434in}}{\pgfqpoint{0.906726in}{0.964434in}}%
\pgfpathcurveto{\pgfqpoint{0.900902in}{0.964434in}}{\pgfqpoint{0.895316in}{0.962120in}}{\pgfqpoint{0.891198in}{0.958002in}}%
\pgfpathcurveto{\pgfqpoint{0.887080in}{0.953884in}}{\pgfqpoint{0.884766in}{0.948298in}}{\pgfqpoint{0.884766in}{0.942474in}}%
\pgfpathcurveto{\pgfqpoint{0.884766in}{0.936650in}}{\pgfqpoint{0.887080in}{0.931064in}}{\pgfqpoint{0.891198in}{0.926946in}}%
\pgfpathcurveto{\pgfqpoint{0.895316in}{0.922827in}}{\pgfqpoint{0.900902in}{0.920514in}}{\pgfqpoint{0.906726in}{0.920514in}}%
\pgfpathclose%
\pgfusepath{stroke,fill}%
\end{pgfscope}%
\begin{pgfscope}%
\pgfpathrectangle{\pgfqpoint{0.211875in}{0.211875in}}{\pgfqpoint{1.313625in}{1.279725in}}%
\pgfusepath{clip}%
\pgfsetbuttcap%
\pgfsetroundjoin%
\definecolor{currentfill}{rgb}{0.121569,0.466667,0.705882}%
\pgfsetfillcolor{currentfill}%
\pgfsetlinewidth{1.003750pt}%
\definecolor{currentstroke}{rgb}{0.121569,0.466667,0.705882}%
\pgfsetstrokecolor{currentstroke}%
\pgfsetdash{}{0pt}%
\pgfpathmoveto{\pgfqpoint{1.097760in}{1.078694in}}%
\pgfpathcurveto{\pgfqpoint{1.103584in}{1.078694in}}{\pgfqpoint{1.109171in}{1.081007in}}{\pgfqpoint{1.113289in}{1.085126in}}%
\pgfpathcurveto{\pgfqpoint{1.117407in}{1.089244in}}{\pgfqpoint{1.119721in}{1.094830in}}{\pgfqpoint{1.119721in}{1.100654in}}%
\pgfpathcurveto{\pgfqpoint{1.119721in}{1.106478in}}{\pgfqpoint{1.117407in}{1.112064in}}{\pgfqpoint{1.113289in}{1.116182in}}%
\pgfpathcurveto{\pgfqpoint{1.109171in}{1.120300in}}{\pgfqpoint{1.103584in}{1.122614in}}{\pgfqpoint{1.097760in}{1.122614in}}%
\pgfpathcurveto{\pgfqpoint{1.091936in}{1.122614in}}{\pgfqpoint{1.086350in}{1.120300in}}{\pgfqpoint{1.082232in}{1.116182in}}%
\pgfpathcurveto{\pgfqpoint{1.078114in}{1.112064in}}{\pgfqpoint{1.075800in}{1.106478in}}{\pgfqpoint{1.075800in}{1.100654in}}%
\pgfpathcurveto{\pgfqpoint{1.075800in}{1.094830in}}{\pgfqpoint{1.078114in}{1.089244in}}{\pgfqpoint{1.082232in}{1.085126in}}%
\pgfpathcurveto{\pgfqpoint{1.086350in}{1.081007in}}{\pgfqpoint{1.091936in}{1.078694in}}{\pgfqpoint{1.097760in}{1.078694in}}%
\pgfpathclose%
\pgfusepath{stroke,fill}%
\end{pgfscope}%
\begin{pgfscope}%
\pgfpathrectangle{\pgfqpoint{0.211875in}{0.211875in}}{\pgfqpoint{1.313625in}{1.279725in}}%
\pgfusepath{clip}%
\pgfsetbuttcap%
\pgfsetroundjoin%
\definecolor{currentfill}{rgb}{0.121569,0.466667,0.705882}%
\pgfsetfillcolor{currentfill}%
\pgfsetlinewidth{1.003750pt}%
\definecolor{currentstroke}{rgb}{0.121569,0.466667,0.705882}%
\pgfsetstrokecolor{currentstroke}%
\pgfsetdash{}{0pt}%
\pgfpathmoveto{\pgfqpoint{0.913370in}{0.925396in}}%
\pgfpathcurveto{\pgfqpoint{0.919194in}{0.925396in}}{\pgfqpoint{0.924780in}{0.927710in}}{\pgfqpoint{0.928898in}{0.931828in}}%
\pgfpathcurveto{\pgfqpoint{0.933016in}{0.935946in}}{\pgfqpoint{0.935330in}{0.941532in}}{\pgfqpoint{0.935330in}{0.947356in}}%
\pgfpathcurveto{\pgfqpoint{0.935330in}{0.953180in}}{\pgfqpoint{0.933016in}{0.958766in}}{\pgfqpoint{0.928898in}{0.962885in}}%
\pgfpathcurveto{\pgfqpoint{0.924780in}{0.967003in}}{\pgfqpoint{0.919194in}{0.969317in}}{\pgfqpoint{0.913370in}{0.969317in}}%
\pgfpathcurveto{\pgfqpoint{0.907546in}{0.969317in}}{\pgfqpoint{0.901959in}{0.967003in}}{\pgfqpoint{0.897841in}{0.962885in}}%
\pgfpathcurveto{\pgfqpoint{0.893723in}{0.958766in}}{\pgfqpoint{0.891409in}{0.953180in}}{\pgfqpoint{0.891409in}{0.947356in}}%
\pgfpathcurveto{\pgfqpoint{0.891409in}{0.941532in}}{\pgfqpoint{0.893723in}{0.935946in}}{\pgfqpoint{0.897841in}{0.931828in}}%
\pgfpathcurveto{\pgfqpoint{0.901959in}{0.927710in}}{\pgfqpoint{0.907546in}{0.925396in}}{\pgfqpoint{0.913370in}{0.925396in}}%
\pgfpathclose%
\pgfusepath{stroke,fill}%
\end{pgfscope}%
\begin{pgfscope}%
\pgfpathrectangle{\pgfqpoint{0.211875in}{0.211875in}}{\pgfqpoint{1.313625in}{1.279725in}}%
\pgfusepath{clip}%
\pgfsetbuttcap%
\pgfsetroundjoin%
\definecolor{currentfill}{rgb}{0.121569,0.466667,0.705882}%
\pgfsetfillcolor{currentfill}%
\pgfsetlinewidth{1.003750pt}%
\definecolor{currentstroke}{rgb}{0.121569,0.466667,0.705882}%
\pgfsetstrokecolor{currentstroke}%
\pgfsetdash{}{0pt}%
\pgfpathmoveto{\pgfqpoint{1.427891in}{1.080529in}}%
\pgfpathcurveto{\pgfqpoint{1.433715in}{1.080529in}}{\pgfqpoint{1.439302in}{1.082843in}}{\pgfqpoint{1.443420in}{1.086961in}}%
\pgfpathcurveto{\pgfqpoint{1.447538in}{1.091079in}}{\pgfqpoint{1.449852in}{1.096665in}}{\pgfqpoint{1.449852in}{1.102489in}}%
\pgfpathcurveto{\pgfqpoint{1.449852in}{1.108313in}}{\pgfqpoint{1.447538in}{1.113899in}}{\pgfqpoint{1.443420in}{1.118017in}}%
\pgfpathcurveto{\pgfqpoint{1.439302in}{1.122135in}}{\pgfqpoint{1.433715in}{1.124449in}}{\pgfqpoint{1.427891in}{1.124449in}}%
\pgfpathcurveto{\pgfqpoint{1.422068in}{1.124449in}}{\pgfqpoint{1.416481in}{1.122135in}}{\pgfqpoint{1.412363in}{1.118017in}}%
\pgfpathcurveto{\pgfqpoint{1.408245in}{1.113899in}}{\pgfqpoint{1.405931in}{1.108313in}}{\pgfqpoint{1.405931in}{1.102489in}}%
\pgfpathcurveto{\pgfqpoint{1.405931in}{1.096665in}}{\pgfqpoint{1.408245in}{1.091079in}}{\pgfqpoint{1.412363in}{1.086961in}}%
\pgfpathcurveto{\pgfqpoint{1.416481in}{1.082843in}}{\pgfqpoint{1.422068in}{1.080529in}}{\pgfqpoint{1.427891in}{1.080529in}}%
\pgfpathclose%
\pgfusepath{stroke,fill}%
\end{pgfscope}%
\begin{pgfscope}%
\pgfpathrectangle{\pgfqpoint{0.211875in}{0.211875in}}{\pgfqpoint{1.313625in}{1.279725in}}%
\pgfusepath{clip}%
\pgfsetbuttcap%
\pgfsetroundjoin%
\definecolor{currentfill}{rgb}{0.121569,0.466667,0.705882}%
\pgfsetfillcolor{currentfill}%
\pgfsetlinewidth{1.003750pt}%
\definecolor{currentstroke}{rgb}{0.121569,0.466667,0.705882}%
\pgfsetstrokecolor{currentstroke}%
\pgfsetdash{}{0pt}%
\pgfpathmoveto{\pgfqpoint{1.432035in}{1.088066in}}%
\pgfpathcurveto{\pgfqpoint{1.437859in}{1.088066in}}{\pgfqpoint{1.443445in}{1.090380in}}{\pgfqpoint{1.447563in}{1.094498in}}%
\pgfpathcurveto{\pgfqpoint{1.451681in}{1.098616in}}{\pgfqpoint{1.453995in}{1.104203in}}{\pgfqpoint{1.453995in}{1.110026in}}%
\pgfpathcurveto{\pgfqpoint{1.453995in}{1.115850in}}{\pgfqpoint{1.451681in}{1.121437in}}{\pgfqpoint{1.447563in}{1.125555in}}%
\pgfpathcurveto{\pgfqpoint{1.443445in}{1.129673in}}{\pgfqpoint{1.437859in}{1.131987in}}{\pgfqpoint{1.432035in}{1.131987in}}%
\pgfpathcurveto{\pgfqpoint{1.426211in}{1.131987in}}{\pgfqpoint{1.420625in}{1.129673in}}{\pgfqpoint{1.416507in}{1.125555in}}%
\pgfpathcurveto{\pgfqpoint{1.412389in}{1.121437in}}{\pgfqpoint{1.410075in}{1.115850in}}{\pgfqpoint{1.410075in}{1.110026in}}%
\pgfpathcurveto{\pgfqpoint{1.410075in}{1.104203in}}{\pgfqpoint{1.412389in}{1.098616in}}{\pgfqpoint{1.416507in}{1.094498in}}%
\pgfpathcurveto{\pgfqpoint{1.420625in}{1.090380in}}{\pgfqpoint{1.426211in}{1.088066in}}{\pgfqpoint{1.432035in}{1.088066in}}%
\pgfpathclose%
\pgfusepath{stroke,fill}%
\end{pgfscope}%
\begin{pgfscope}%
\pgfpathrectangle{\pgfqpoint{0.211875in}{0.211875in}}{\pgfqpoint{1.313625in}{1.279725in}}%
\pgfusepath{clip}%
\pgfsetbuttcap%
\pgfsetroundjoin%
\definecolor{currentfill}{rgb}{0.121569,0.466667,0.705882}%
\pgfsetfillcolor{currentfill}%
\pgfsetlinewidth{1.003750pt}%
\definecolor{currentstroke}{rgb}{0.121569,0.466667,0.705882}%
\pgfsetstrokecolor{currentstroke}%
\pgfsetdash{}{0pt}%
\pgfpathmoveto{\pgfqpoint{1.317116in}{1.164888in}}%
\pgfpathcurveto{\pgfqpoint{1.322940in}{1.164888in}}{\pgfqpoint{1.328526in}{1.167201in}}{\pgfqpoint{1.332644in}{1.171320in}}%
\pgfpathcurveto{\pgfqpoint{1.336762in}{1.175438in}}{\pgfqpoint{1.339076in}{1.181024in}}{\pgfqpoint{1.339076in}{1.186848in}}%
\pgfpathcurveto{\pgfqpoint{1.339076in}{1.192672in}}{\pgfqpoint{1.336762in}{1.198258in}}{\pgfqpoint{1.332644in}{1.202376in}}%
\pgfpathcurveto{\pgfqpoint{1.328526in}{1.206494in}}{\pgfqpoint{1.322940in}{1.208808in}}{\pgfqpoint{1.317116in}{1.208808in}}%
\pgfpathcurveto{\pgfqpoint{1.311292in}{1.208808in}}{\pgfqpoint{1.305706in}{1.206494in}}{\pgfqpoint{1.301587in}{1.202376in}}%
\pgfpathcurveto{\pgfqpoint{1.297469in}{1.198258in}}{\pgfqpoint{1.295155in}{1.192672in}}{\pgfqpoint{1.295155in}{1.186848in}}%
\pgfpathcurveto{\pgfqpoint{1.295155in}{1.181024in}}{\pgfqpoint{1.297469in}{1.175438in}}{\pgfqpoint{1.301587in}{1.171320in}}%
\pgfpathcurveto{\pgfqpoint{1.305706in}{1.167201in}}{\pgfqpoint{1.311292in}{1.164888in}}{\pgfqpoint{1.317116in}{1.164888in}}%
\pgfpathclose%
\pgfusepath{stroke,fill}%
\end{pgfscope}%
\begin{pgfscope}%
\pgfpathrectangle{\pgfqpoint{0.211875in}{0.211875in}}{\pgfqpoint{1.313625in}{1.279725in}}%
\pgfusepath{clip}%
\pgfsetbuttcap%
\pgfsetroundjoin%
\definecolor{currentfill}{rgb}{0.121569,0.466667,0.705882}%
\pgfsetfillcolor{currentfill}%
\pgfsetlinewidth{1.003750pt}%
\definecolor{currentstroke}{rgb}{0.121569,0.466667,0.705882}%
\pgfsetstrokecolor{currentstroke}%
\pgfsetdash{}{0pt}%
\pgfpathmoveto{\pgfqpoint{0.799649in}{1.063479in}}%
\pgfpathcurveto{\pgfqpoint{0.805472in}{1.063479in}}{\pgfqpoint{0.811059in}{1.065793in}}{\pgfqpoint{0.815177in}{1.069911in}}%
\pgfpathcurveto{\pgfqpoint{0.819295in}{1.074029in}}{\pgfqpoint{0.821609in}{1.079615in}}{\pgfqpoint{0.821609in}{1.085439in}}%
\pgfpathcurveto{\pgfqpoint{0.821609in}{1.091263in}}{\pgfqpoint{0.819295in}{1.096849in}}{\pgfqpoint{0.815177in}{1.100967in}}%
\pgfpathcurveto{\pgfqpoint{0.811059in}{1.105085in}}{\pgfqpoint{0.805472in}{1.107399in}}{\pgfqpoint{0.799649in}{1.107399in}}%
\pgfpathcurveto{\pgfqpoint{0.793825in}{1.107399in}}{\pgfqpoint{0.788238in}{1.105085in}}{\pgfqpoint{0.784120in}{1.100967in}}%
\pgfpathcurveto{\pgfqpoint{0.780002in}{1.096849in}}{\pgfqpoint{0.777688in}{1.091263in}}{\pgfqpoint{0.777688in}{1.085439in}}%
\pgfpathcurveto{\pgfqpoint{0.777688in}{1.079615in}}{\pgfqpoint{0.780002in}{1.074029in}}{\pgfqpoint{0.784120in}{1.069911in}}%
\pgfpathcurveto{\pgfqpoint{0.788238in}{1.065793in}}{\pgfqpoint{0.793825in}{1.063479in}}{\pgfqpoint{0.799649in}{1.063479in}}%
\pgfpathclose%
\pgfusepath{stroke,fill}%
\end{pgfscope}%
\begin{pgfscope}%
\pgfpathrectangle{\pgfqpoint{0.211875in}{0.211875in}}{\pgfqpoint{1.313625in}{1.279725in}}%
\pgfusepath{clip}%
\pgfsetbuttcap%
\pgfsetroundjoin%
\definecolor{currentfill}{rgb}{0.121569,0.466667,0.705882}%
\pgfsetfillcolor{currentfill}%
\pgfsetlinewidth{1.003750pt}%
\definecolor{currentstroke}{rgb}{0.121569,0.466667,0.705882}%
\pgfsetstrokecolor{currentstroke}%
\pgfsetdash{}{0pt}%
\pgfpathmoveto{\pgfqpoint{1.120394in}{1.057460in}}%
\pgfpathcurveto{\pgfqpoint{1.126218in}{1.057460in}}{\pgfqpoint{1.131804in}{1.059774in}}{\pgfqpoint{1.135923in}{1.063892in}}%
\pgfpathcurveto{\pgfqpoint{1.140041in}{1.068010in}}{\pgfqpoint{1.142355in}{1.073596in}}{\pgfqpoint{1.142355in}{1.079420in}}%
\pgfpathcurveto{\pgfqpoint{1.142355in}{1.085244in}}{\pgfqpoint{1.140041in}{1.090830in}}{\pgfqpoint{1.135923in}{1.094948in}}%
\pgfpathcurveto{\pgfqpoint{1.131804in}{1.099066in}}{\pgfqpoint{1.126218in}{1.101380in}}{\pgfqpoint{1.120394in}{1.101380in}}%
\pgfpathcurveto{\pgfqpoint{1.114570in}{1.101380in}}{\pgfqpoint{1.108984in}{1.099066in}}{\pgfqpoint{1.104866in}{1.094948in}}%
\pgfpathcurveto{\pgfqpoint{1.100748in}{1.090830in}}{\pgfqpoint{1.098434in}{1.085244in}}{\pgfqpoint{1.098434in}{1.079420in}}%
\pgfpathcurveto{\pgfqpoint{1.098434in}{1.073596in}}{\pgfqpoint{1.100748in}{1.068010in}}{\pgfqpoint{1.104866in}{1.063892in}}%
\pgfpathcurveto{\pgfqpoint{1.108984in}{1.059774in}}{\pgfqpoint{1.114570in}{1.057460in}}{\pgfqpoint{1.120394in}{1.057460in}}%
\pgfpathclose%
\pgfusepath{stroke,fill}%
\end{pgfscope}%
\begin{pgfscope}%
\pgfpathrectangle{\pgfqpoint{0.211875in}{0.211875in}}{\pgfqpoint{1.313625in}{1.279725in}}%
\pgfusepath{clip}%
\pgfsetbuttcap%
\pgfsetroundjoin%
\definecolor{currentfill}{rgb}{0.121569,0.466667,0.705882}%
\pgfsetfillcolor{currentfill}%
\pgfsetlinewidth{1.003750pt}%
\definecolor{currentstroke}{rgb}{0.121569,0.466667,0.705882}%
\pgfsetstrokecolor{currentstroke}%
\pgfsetdash{}{0pt}%
\pgfpathmoveto{\pgfqpoint{0.870624in}{1.013397in}}%
\pgfpathcurveto{\pgfqpoint{0.876448in}{1.013397in}}{\pgfqpoint{0.882034in}{1.015711in}}{\pgfqpoint{0.886152in}{1.019829in}}%
\pgfpathcurveto{\pgfqpoint{0.890270in}{1.023947in}}{\pgfqpoint{0.892584in}{1.029533in}}{\pgfqpoint{0.892584in}{1.035357in}}%
\pgfpathcurveto{\pgfqpoint{0.892584in}{1.041181in}}{\pgfqpoint{0.890270in}{1.046767in}}{\pgfqpoint{0.886152in}{1.050885in}}%
\pgfpathcurveto{\pgfqpoint{0.882034in}{1.055004in}}{\pgfqpoint{0.876448in}{1.057317in}}{\pgfqpoint{0.870624in}{1.057317in}}%
\pgfpathcurveto{\pgfqpoint{0.864800in}{1.057317in}}{\pgfqpoint{0.859214in}{1.055004in}}{\pgfqpoint{0.855096in}{1.050885in}}%
\pgfpathcurveto{\pgfqpoint{0.850978in}{1.046767in}}{\pgfqpoint{0.848664in}{1.041181in}}{\pgfqpoint{0.848664in}{1.035357in}}%
\pgfpathcurveto{\pgfqpoint{0.848664in}{1.029533in}}{\pgfqpoint{0.850978in}{1.023947in}}{\pgfqpoint{0.855096in}{1.019829in}}%
\pgfpathcurveto{\pgfqpoint{0.859214in}{1.015711in}}{\pgfqpoint{0.864800in}{1.013397in}}{\pgfqpoint{0.870624in}{1.013397in}}%
\pgfpathclose%
\pgfusepath{stroke,fill}%
\end{pgfscope}%
\begin{pgfscope}%
\pgfpathrectangle{\pgfqpoint{0.211875in}{0.211875in}}{\pgfqpoint{1.313625in}{1.279725in}}%
\pgfusepath{clip}%
\pgfsetbuttcap%
\pgfsetroundjoin%
\definecolor{currentfill}{rgb}{0.121569,0.466667,0.705882}%
\pgfsetfillcolor{currentfill}%
\pgfsetlinewidth{1.003750pt}%
\definecolor{currentstroke}{rgb}{0.121569,0.466667,0.705882}%
\pgfsetstrokecolor{currentstroke}%
\pgfsetdash{}{0pt}%
\pgfpathmoveto{\pgfqpoint{0.674506in}{0.868079in}}%
\pgfpathcurveto{\pgfqpoint{0.680330in}{0.868079in}}{\pgfqpoint{0.685916in}{0.870393in}}{\pgfqpoint{0.690035in}{0.874511in}}%
\pgfpathcurveto{\pgfqpoint{0.694153in}{0.878629in}}{\pgfqpoint{0.696467in}{0.884215in}}{\pgfqpoint{0.696467in}{0.890039in}}%
\pgfpathcurveto{\pgfqpoint{0.696467in}{0.895863in}}{\pgfqpoint{0.694153in}{0.901449in}}{\pgfqpoint{0.690035in}{0.905567in}}%
\pgfpathcurveto{\pgfqpoint{0.685916in}{0.909685in}}{\pgfqpoint{0.680330in}{0.911999in}}{\pgfqpoint{0.674506in}{0.911999in}}%
\pgfpathcurveto{\pgfqpoint{0.668682in}{0.911999in}}{\pgfqpoint{0.663096in}{0.909685in}}{\pgfqpoint{0.658978in}{0.905567in}}%
\pgfpathcurveto{\pgfqpoint{0.654860in}{0.901449in}}{\pgfqpoint{0.652546in}{0.895863in}}{\pgfqpoint{0.652546in}{0.890039in}}%
\pgfpathcurveto{\pgfqpoint{0.652546in}{0.884215in}}{\pgfqpoint{0.654860in}{0.878629in}}{\pgfqpoint{0.658978in}{0.874511in}}%
\pgfpathcurveto{\pgfqpoint{0.663096in}{0.870393in}}{\pgfqpoint{0.668682in}{0.868079in}}{\pgfqpoint{0.674506in}{0.868079in}}%
\pgfpathclose%
\pgfusepath{stroke,fill}%
\end{pgfscope}%
\begin{pgfscope}%
\pgfpathrectangle{\pgfqpoint{0.211875in}{0.211875in}}{\pgfqpoint{1.313625in}{1.279725in}}%
\pgfusepath{clip}%
\pgfsetbuttcap%
\pgfsetroundjoin%
\definecolor{currentfill}{rgb}{0.121569,0.466667,0.705882}%
\pgfsetfillcolor{currentfill}%
\pgfsetlinewidth{1.003750pt}%
\definecolor{currentstroke}{rgb}{0.121569,0.466667,0.705882}%
\pgfsetstrokecolor{currentstroke}%
\pgfsetdash{}{0pt}%
\pgfpathmoveto{\pgfqpoint{1.357869in}{0.955456in}}%
\pgfpathcurveto{\pgfqpoint{1.363693in}{0.955456in}}{\pgfqpoint{1.369279in}{0.957770in}}{\pgfqpoint{1.373397in}{0.961888in}}%
\pgfpathcurveto{\pgfqpoint{1.377515in}{0.966006in}}{\pgfqpoint{1.379829in}{0.971593in}}{\pgfqpoint{1.379829in}{0.977417in}}%
\pgfpathcurveto{\pgfqpoint{1.379829in}{0.983240in}}{\pgfqpoint{1.377515in}{0.988827in}}{\pgfqpoint{1.373397in}{0.992945in}}%
\pgfpathcurveto{\pgfqpoint{1.369279in}{0.997063in}}{\pgfqpoint{1.363693in}{0.999377in}}{\pgfqpoint{1.357869in}{0.999377in}}%
\pgfpathcurveto{\pgfqpoint{1.352045in}{0.999377in}}{\pgfqpoint{1.346459in}{0.997063in}}{\pgfqpoint{1.342341in}{0.992945in}}%
\pgfpathcurveto{\pgfqpoint{1.338223in}{0.988827in}}{\pgfqpoint{1.335909in}{0.983240in}}{\pgfqpoint{1.335909in}{0.977417in}}%
\pgfpathcurveto{\pgfqpoint{1.335909in}{0.971593in}}{\pgfqpoint{1.338223in}{0.966006in}}{\pgfqpoint{1.342341in}{0.961888in}}%
\pgfpathcurveto{\pgfqpoint{1.346459in}{0.957770in}}{\pgfqpoint{1.352045in}{0.955456in}}{\pgfqpoint{1.357869in}{0.955456in}}%
\pgfpathclose%
\pgfusepath{stroke,fill}%
\end{pgfscope}%
\begin{pgfscope}%
\pgfpathrectangle{\pgfqpoint{0.211875in}{0.211875in}}{\pgfqpoint{1.313625in}{1.279725in}}%
\pgfusepath{clip}%
\pgfsetbuttcap%
\pgfsetroundjoin%
\definecolor{currentfill}{rgb}{0.121569,0.466667,0.705882}%
\pgfsetfillcolor{currentfill}%
\pgfsetlinewidth{1.003750pt}%
\definecolor{currentstroke}{rgb}{0.121569,0.466667,0.705882}%
\pgfsetstrokecolor{currentstroke}%
\pgfsetdash{}{0pt}%
\pgfpathmoveto{\pgfqpoint{0.759387in}{1.160588in}}%
\pgfpathcurveto{\pgfqpoint{0.765211in}{1.160588in}}{\pgfqpoint{0.770797in}{1.162902in}}{\pgfqpoint{0.774915in}{1.167020in}}%
\pgfpathcurveto{\pgfqpoint{0.779033in}{1.171138in}}{\pgfqpoint{0.781347in}{1.176724in}}{\pgfqpoint{0.781347in}{1.182548in}}%
\pgfpathcurveto{\pgfqpoint{0.781347in}{1.188372in}}{\pgfqpoint{0.779033in}{1.193958in}}{\pgfqpoint{0.774915in}{1.198076in}}%
\pgfpathcurveto{\pgfqpoint{0.770797in}{1.202194in}}{\pgfqpoint{0.765211in}{1.204508in}}{\pgfqpoint{0.759387in}{1.204508in}}%
\pgfpathcurveto{\pgfqpoint{0.753563in}{1.204508in}}{\pgfqpoint{0.747977in}{1.202194in}}{\pgfqpoint{0.743859in}{1.198076in}}%
\pgfpathcurveto{\pgfqpoint{0.739741in}{1.193958in}}{\pgfqpoint{0.737427in}{1.188372in}}{\pgfqpoint{0.737427in}{1.182548in}}%
\pgfpathcurveto{\pgfqpoint{0.737427in}{1.176724in}}{\pgfqpoint{0.739741in}{1.171138in}}{\pgfqpoint{0.743859in}{1.167020in}}%
\pgfpathcurveto{\pgfqpoint{0.747977in}{1.162902in}}{\pgfqpoint{0.753563in}{1.160588in}}{\pgfqpoint{0.759387in}{1.160588in}}%
\pgfpathclose%
\pgfusepath{stroke,fill}%
\end{pgfscope}%
\begin{pgfscope}%
\pgfpathrectangle{\pgfqpoint{0.211875in}{0.211875in}}{\pgfqpoint{1.313625in}{1.279725in}}%
\pgfusepath{clip}%
\pgfsetbuttcap%
\pgfsetroundjoin%
\definecolor{currentfill}{rgb}{0.121569,0.466667,0.705882}%
\pgfsetfillcolor{currentfill}%
\pgfsetlinewidth{1.003750pt}%
\definecolor{currentstroke}{rgb}{0.121569,0.466667,0.705882}%
\pgfsetstrokecolor{currentstroke}%
\pgfsetdash{}{0pt}%
\pgfpathmoveto{\pgfqpoint{1.249923in}{1.380213in}}%
\pgfpathcurveto{\pgfqpoint{1.255746in}{1.380213in}}{\pgfqpoint{1.261333in}{1.382527in}}{\pgfqpoint{1.265451in}{1.386645in}}%
\pgfpathcurveto{\pgfqpoint{1.269569in}{1.390764in}}{\pgfqpoint{1.271883in}{1.396350in}}{\pgfqpoint{1.271883in}{1.402174in}}%
\pgfpathcurveto{\pgfqpoint{1.271883in}{1.407998in}}{\pgfqpoint{1.269569in}{1.413584in}}{\pgfqpoint{1.265451in}{1.417702in}}%
\pgfpathcurveto{\pgfqpoint{1.261333in}{1.421820in}}{\pgfqpoint{1.255746in}{1.424134in}}{\pgfqpoint{1.249923in}{1.424134in}}%
\pgfpathcurveto{\pgfqpoint{1.244099in}{1.424134in}}{\pgfqpoint{1.238512in}{1.421820in}}{\pgfqpoint{1.234394in}{1.417702in}}%
\pgfpathcurveto{\pgfqpoint{1.230276in}{1.413584in}}{\pgfqpoint{1.227962in}{1.407998in}}{\pgfqpoint{1.227962in}{1.402174in}}%
\pgfpathcurveto{\pgfqpoint{1.227962in}{1.396350in}}{\pgfqpoint{1.230276in}{1.390764in}}{\pgfqpoint{1.234394in}{1.386645in}}%
\pgfpathcurveto{\pgfqpoint{1.238512in}{1.382527in}}{\pgfqpoint{1.244099in}{1.380213in}}{\pgfqpoint{1.249923in}{1.380213in}}%
\pgfpathclose%
\pgfusepath{stroke,fill}%
\end{pgfscope}%
\begin{pgfscope}%
\pgfpathrectangle{\pgfqpoint{0.211875in}{0.211875in}}{\pgfqpoint{1.313625in}{1.279725in}}%
\pgfusepath{clip}%
\pgfsetbuttcap%
\pgfsetroundjoin%
\definecolor{currentfill}{rgb}{0.121569,0.466667,0.705882}%
\pgfsetfillcolor{currentfill}%
\pgfsetlinewidth{1.003750pt}%
\definecolor{currentstroke}{rgb}{0.121569,0.466667,0.705882}%
\pgfsetstrokecolor{currentstroke}%
\pgfsetdash{}{0pt}%
\pgfpathmoveto{\pgfqpoint{1.438021in}{0.972244in}}%
\pgfpathcurveto{\pgfqpoint{1.443845in}{0.972244in}}{\pgfqpoint{1.449431in}{0.974558in}}{\pgfqpoint{1.453550in}{0.978676in}}%
\pgfpathcurveto{\pgfqpoint{1.457668in}{0.982794in}}{\pgfqpoint{1.459982in}{0.988381in}}{\pgfqpoint{1.459982in}{0.994205in}}%
\pgfpathcurveto{\pgfqpoint{1.459982in}{1.000029in}}{\pgfqpoint{1.457668in}{1.005615in}}{\pgfqpoint{1.453550in}{1.009733in}}%
\pgfpathcurveto{\pgfqpoint{1.449431in}{1.013851in}}{\pgfqpoint{1.443845in}{1.016165in}}{\pgfqpoint{1.438021in}{1.016165in}}%
\pgfpathcurveto{\pgfqpoint{1.432197in}{1.016165in}}{\pgfqpoint{1.426611in}{1.013851in}}{\pgfqpoint{1.422493in}{1.009733in}}%
\pgfpathcurveto{\pgfqpoint{1.418375in}{1.005615in}}{\pgfqpoint{1.416061in}{1.000029in}}{\pgfqpoint{1.416061in}{0.994205in}}%
\pgfpathcurveto{\pgfqpoint{1.416061in}{0.988381in}}{\pgfqpoint{1.418375in}{0.982794in}}{\pgfqpoint{1.422493in}{0.978676in}}%
\pgfpathcurveto{\pgfqpoint{1.426611in}{0.974558in}}{\pgfqpoint{1.432197in}{0.972244in}}{\pgfqpoint{1.438021in}{0.972244in}}%
\pgfpathclose%
\pgfusepath{stroke,fill}%
\end{pgfscope}%
\begin{pgfscope}%
\pgfpathrectangle{\pgfqpoint{0.211875in}{0.211875in}}{\pgfqpoint{1.313625in}{1.279725in}}%
\pgfusepath{clip}%
\pgfsetbuttcap%
\pgfsetroundjoin%
\definecolor{currentfill}{rgb}{0.121569,0.466667,0.705882}%
\pgfsetfillcolor{currentfill}%
\pgfsetlinewidth{1.003750pt}%
\definecolor{currentstroke}{rgb}{0.121569,0.466667,0.705882}%
\pgfsetstrokecolor{currentstroke}%
\pgfsetdash{}{0pt}%
\pgfpathmoveto{\pgfqpoint{0.820964in}{1.060957in}}%
\pgfpathcurveto{\pgfqpoint{0.826788in}{1.060957in}}{\pgfqpoint{0.832374in}{1.063271in}}{\pgfqpoint{0.836492in}{1.067389in}}%
\pgfpathcurveto{\pgfqpoint{0.840610in}{1.071507in}}{\pgfqpoint{0.842924in}{1.077093in}}{\pgfqpoint{0.842924in}{1.082917in}}%
\pgfpathcurveto{\pgfqpoint{0.842924in}{1.088741in}}{\pgfqpoint{0.840610in}{1.094327in}}{\pgfqpoint{0.836492in}{1.098445in}}%
\pgfpathcurveto{\pgfqpoint{0.832374in}{1.102563in}}{\pgfqpoint{0.826788in}{1.104877in}}{\pgfqpoint{0.820964in}{1.104877in}}%
\pgfpathcurveto{\pgfqpoint{0.815140in}{1.104877in}}{\pgfqpoint{0.809554in}{1.102563in}}{\pgfqpoint{0.805436in}{1.098445in}}%
\pgfpathcurveto{\pgfqpoint{0.801318in}{1.094327in}}{\pgfqpoint{0.799004in}{1.088741in}}{\pgfqpoint{0.799004in}{1.082917in}}%
\pgfpathcurveto{\pgfqpoint{0.799004in}{1.077093in}}{\pgfqpoint{0.801318in}{1.071507in}}{\pgfqpoint{0.805436in}{1.067389in}}%
\pgfpathcurveto{\pgfqpoint{0.809554in}{1.063271in}}{\pgfqpoint{0.815140in}{1.060957in}}{\pgfqpoint{0.820964in}{1.060957in}}%
\pgfpathclose%
\pgfusepath{stroke,fill}%
\end{pgfscope}%
\begin{pgfscope}%
\pgfpathrectangle{\pgfqpoint{0.211875in}{0.211875in}}{\pgfqpoint{1.313625in}{1.279725in}}%
\pgfusepath{clip}%
\pgfsetbuttcap%
\pgfsetroundjoin%
\definecolor{currentfill}{rgb}{0.121569,0.466667,0.705882}%
\pgfsetfillcolor{currentfill}%
\pgfsetlinewidth{1.003750pt}%
\definecolor{currentstroke}{rgb}{0.121569,0.466667,0.705882}%
\pgfsetstrokecolor{currentstroke}%
\pgfsetdash{}{0pt}%
\pgfpathmoveto{\pgfqpoint{1.121275in}{1.084724in}}%
\pgfpathcurveto{\pgfqpoint{1.127099in}{1.084724in}}{\pgfqpoint{1.132685in}{1.087038in}}{\pgfqpoint{1.136804in}{1.091156in}}%
\pgfpathcurveto{\pgfqpoint{1.140922in}{1.095274in}}{\pgfqpoint{1.143236in}{1.100860in}}{\pgfqpoint{1.143236in}{1.106684in}}%
\pgfpathcurveto{\pgfqpoint{1.143236in}{1.112508in}}{\pgfqpoint{1.140922in}{1.118095in}}{\pgfqpoint{1.136804in}{1.122213in}}%
\pgfpathcurveto{\pgfqpoint{1.132685in}{1.126331in}}{\pgfqpoint{1.127099in}{1.128645in}}{\pgfqpoint{1.121275in}{1.128645in}}%
\pgfpathcurveto{\pgfqpoint{1.115451in}{1.128645in}}{\pgfqpoint{1.109865in}{1.126331in}}{\pgfqpoint{1.105747in}{1.122213in}}%
\pgfpathcurveto{\pgfqpoint{1.101629in}{1.118095in}}{\pgfqpoint{1.099315in}{1.112508in}}{\pgfqpoint{1.099315in}{1.106684in}}%
\pgfpathcurveto{\pgfqpoint{1.099315in}{1.100860in}}{\pgfqpoint{1.101629in}{1.095274in}}{\pgfqpoint{1.105747in}{1.091156in}}%
\pgfpathcurveto{\pgfqpoint{1.109865in}{1.087038in}}{\pgfqpoint{1.115451in}{1.084724in}}{\pgfqpoint{1.121275in}{1.084724in}}%
\pgfpathclose%
\pgfusepath{stroke,fill}%
\end{pgfscope}%
\begin{pgfscope}%
\pgfpathrectangle{\pgfqpoint{0.211875in}{0.211875in}}{\pgfqpoint{1.313625in}{1.279725in}}%
\pgfusepath{clip}%
\pgfsetbuttcap%
\pgfsetroundjoin%
\definecolor{currentfill}{rgb}{0.121569,0.466667,0.705882}%
\pgfsetfillcolor{currentfill}%
\pgfsetlinewidth{1.003750pt}%
\definecolor{currentstroke}{rgb}{0.121569,0.466667,0.705882}%
\pgfsetstrokecolor{currentstroke}%
\pgfsetdash{}{0pt}%
\pgfpathmoveto{\pgfqpoint{0.787211in}{1.084538in}}%
\pgfpathcurveto{\pgfqpoint{0.793035in}{1.084538in}}{\pgfqpoint{0.798621in}{1.086851in}}{\pgfqpoint{0.802739in}{1.090970in}}%
\pgfpathcurveto{\pgfqpoint{0.806857in}{1.095088in}}{\pgfqpoint{0.809171in}{1.100674in}}{\pgfqpoint{0.809171in}{1.106498in}}%
\pgfpathcurveto{\pgfqpoint{0.809171in}{1.112322in}}{\pgfqpoint{0.806857in}{1.117908in}}{\pgfqpoint{0.802739in}{1.122026in}}%
\pgfpathcurveto{\pgfqpoint{0.798621in}{1.126144in}}{\pgfqpoint{0.793035in}{1.128458in}}{\pgfqpoint{0.787211in}{1.128458in}}%
\pgfpathcurveto{\pgfqpoint{0.781387in}{1.128458in}}{\pgfqpoint{0.775800in}{1.126144in}}{\pgfqpoint{0.771682in}{1.122026in}}%
\pgfpathcurveto{\pgfqpoint{0.767564in}{1.117908in}}{\pgfqpoint{0.765250in}{1.112322in}}{\pgfqpoint{0.765250in}{1.106498in}}%
\pgfpathcurveto{\pgfqpoint{0.765250in}{1.100674in}}{\pgfqpoint{0.767564in}{1.095088in}}{\pgfqpoint{0.771682in}{1.090970in}}%
\pgfpathcurveto{\pgfqpoint{0.775800in}{1.086851in}}{\pgfqpoint{0.781387in}{1.084538in}}{\pgfqpoint{0.787211in}{1.084538in}}%
\pgfpathclose%
\pgfusepath{stroke,fill}%
\end{pgfscope}%
\begin{pgfscope}%
\pgfpathrectangle{\pgfqpoint{0.211875in}{0.211875in}}{\pgfqpoint{1.313625in}{1.279725in}}%
\pgfusepath{clip}%
\pgfsetbuttcap%
\pgfsetroundjoin%
\definecolor{currentfill}{rgb}{0.121569,0.466667,0.705882}%
\pgfsetfillcolor{currentfill}%
\pgfsetlinewidth{1.003750pt}%
\definecolor{currentstroke}{rgb}{0.121569,0.466667,0.705882}%
\pgfsetstrokecolor{currentstroke}%
\pgfsetdash{}{0pt}%
\pgfpathmoveto{\pgfqpoint{1.421406in}{1.065667in}}%
\pgfpathcurveto{\pgfqpoint{1.427230in}{1.065667in}}{\pgfqpoint{1.432817in}{1.067981in}}{\pgfqpoint{1.436935in}{1.072099in}}%
\pgfpathcurveto{\pgfqpoint{1.441053in}{1.076217in}}{\pgfqpoint{1.443367in}{1.081803in}}{\pgfqpoint{1.443367in}{1.087627in}}%
\pgfpathcurveto{\pgfqpoint{1.443367in}{1.093451in}}{\pgfqpoint{1.441053in}{1.099037in}}{\pgfqpoint{1.436935in}{1.103155in}}%
\pgfpathcurveto{\pgfqpoint{1.432817in}{1.107273in}}{\pgfqpoint{1.427230in}{1.109587in}}{\pgfqpoint{1.421406in}{1.109587in}}%
\pgfpathcurveto{\pgfqpoint{1.415583in}{1.109587in}}{\pgfqpoint{1.409996in}{1.107273in}}{\pgfqpoint{1.405878in}{1.103155in}}%
\pgfpathcurveto{\pgfqpoint{1.401760in}{1.099037in}}{\pgfqpoint{1.399446in}{1.093451in}}{\pgfqpoint{1.399446in}{1.087627in}}%
\pgfpathcurveto{\pgfqpoint{1.399446in}{1.081803in}}{\pgfqpoint{1.401760in}{1.076217in}}{\pgfqpoint{1.405878in}{1.072099in}}%
\pgfpathcurveto{\pgfqpoint{1.409996in}{1.067981in}}{\pgfqpoint{1.415583in}{1.065667in}}{\pgfqpoint{1.421406in}{1.065667in}}%
\pgfpathclose%
\pgfusepath{stroke,fill}%
\end{pgfscope}%
\begin{pgfscope}%
\pgfpathrectangle{\pgfqpoint{0.211875in}{0.211875in}}{\pgfqpoint{1.313625in}{1.279725in}}%
\pgfusepath{clip}%
\pgfsetbuttcap%
\pgfsetroundjoin%
\definecolor{currentfill}{rgb}{0.121569,0.466667,0.705882}%
\pgfsetfillcolor{currentfill}%
\pgfsetlinewidth{1.003750pt}%
\definecolor{currentstroke}{rgb}{0.121569,0.466667,0.705882}%
\pgfsetstrokecolor{currentstroke}%
\pgfsetdash{}{0pt}%
\pgfpathmoveto{\pgfqpoint{0.740092in}{1.297465in}}%
\pgfpathcurveto{\pgfqpoint{0.745916in}{1.297465in}}{\pgfqpoint{0.751502in}{1.299779in}}{\pgfqpoint{0.755620in}{1.303897in}}%
\pgfpathcurveto{\pgfqpoint{0.759738in}{1.308015in}}{\pgfqpoint{0.762052in}{1.313601in}}{\pgfqpoint{0.762052in}{1.319425in}}%
\pgfpathcurveto{\pgfqpoint{0.762052in}{1.325249in}}{\pgfqpoint{0.759738in}{1.330835in}}{\pgfqpoint{0.755620in}{1.334953in}}%
\pgfpathcurveto{\pgfqpoint{0.751502in}{1.339071in}}{\pgfqpoint{0.745916in}{1.341385in}}{\pgfqpoint{0.740092in}{1.341385in}}%
\pgfpathcurveto{\pgfqpoint{0.734268in}{1.341385in}}{\pgfqpoint{0.728682in}{1.339071in}}{\pgfqpoint{0.724564in}{1.334953in}}%
\pgfpathcurveto{\pgfqpoint{0.720445in}{1.330835in}}{\pgfqpoint{0.718132in}{1.325249in}}{\pgfqpoint{0.718132in}{1.319425in}}%
\pgfpathcurveto{\pgfqpoint{0.718132in}{1.313601in}}{\pgfqpoint{0.720445in}{1.308015in}}{\pgfqpoint{0.724564in}{1.303897in}}%
\pgfpathcurveto{\pgfqpoint{0.728682in}{1.299779in}}{\pgfqpoint{0.734268in}{1.297465in}}{\pgfqpoint{0.740092in}{1.297465in}}%
\pgfpathclose%
\pgfusepath{stroke,fill}%
\end{pgfscope}%
\begin{pgfscope}%
\pgfpathrectangle{\pgfqpoint{0.211875in}{0.211875in}}{\pgfqpoint{1.313625in}{1.279725in}}%
\pgfusepath{clip}%
\pgfsetbuttcap%
\pgfsetroundjoin%
\definecolor{currentfill}{rgb}{0.121569,0.466667,0.705882}%
\pgfsetfillcolor{currentfill}%
\pgfsetlinewidth{1.003750pt}%
\definecolor{currentstroke}{rgb}{0.121569,0.466667,0.705882}%
\pgfsetstrokecolor{currentstroke}%
\pgfsetdash{}{0pt}%
\pgfpathmoveto{\pgfqpoint{1.121732in}{1.080770in}}%
\pgfpathcurveto{\pgfqpoint{1.127556in}{1.080770in}}{\pgfqpoint{1.133142in}{1.083084in}}{\pgfqpoint{1.137260in}{1.087202in}}%
\pgfpathcurveto{\pgfqpoint{1.141378in}{1.091320in}}{\pgfqpoint{1.143692in}{1.096906in}}{\pgfqpoint{1.143692in}{1.102730in}}%
\pgfpathcurveto{\pgfqpoint{1.143692in}{1.108554in}}{\pgfqpoint{1.141378in}{1.114140in}}{\pgfqpoint{1.137260in}{1.118258in}}%
\pgfpathcurveto{\pgfqpoint{1.133142in}{1.122376in}}{\pgfqpoint{1.127556in}{1.124690in}}{\pgfqpoint{1.121732in}{1.124690in}}%
\pgfpathcurveto{\pgfqpoint{1.115908in}{1.124690in}}{\pgfqpoint{1.110322in}{1.122376in}}{\pgfqpoint{1.106203in}{1.118258in}}%
\pgfpathcurveto{\pgfqpoint{1.102085in}{1.114140in}}{\pgfqpoint{1.099771in}{1.108554in}}{\pgfqpoint{1.099771in}{1.102730in}}%
\pgfpathcurveto{\pgfqpoint{1.099771in}{1.096906in}}{\pgfqpoint{1.102085in}{1.091320in}}{\pgfqpoint{1.106203in}{1.087202in}}%
\pgfpathcurveto{\pgfqpoint{1.110322in}{1.083084in}}{\pgfqpoint{1.115908in}{1.080770in}}{\pgfqpoint{1.121732in}{1.080770in}}%
\pgfpathclose%
\pgfusepath{stroke,fill}%
\end{pgfscope}%
\begin{pgfscope}%
\pgfpathrectangle{\pgfqpoint{0.211875in}{0.211875in}}{\pgfqpoint{1.313625in}{1.279725in}}%
\pgfusepath{clip}%
\pgfsetbuttcap%
\pgfsetroundjoin%
\definecolor{currentfill}{rgb}{0.121569,0.466667,0.705882}%
\pgfsetfillcolor{currentfill}%
\pgfsetlinewidth{1.003750pt}%
\definecolor{currentstroke}{rgb}{0.121569,0.466667,0.705882}%
\pgfsetstrokecolor{currentstroke}%
\pgfsetdash{}{0pt}%
\pgfpathmoveto{\pgfqpoint{1.364900in}{0.994976in}}%
\pgfpathcurveto{\pgfqpoint{1.370724in}{0.994976in}}{\pgfqpoint{1.376310in}{0.997289in}}{\pgfqpoint{1.380429in}{1.001408in}}%
\pgfpathcurveto{\pgfqpoint{1.384547in}{1.005526in}}{\pgfqpoint{1.386861in}{1.011112in}}{\pgfqpoint{1.386861in}{1.016936in}}%
\pgfpathcurveto{\pgfqpoint{1.386861in}{1.022760in}}{\pgfqpoint{1.384547in}{1.028346in}}{\pgfqpoint{1.380429in}{1.032464in}}%
\pgfpathcurveto{\pgfqpoint{1.376310in}{1.036582in}}{\pgfqpoint{1.370724in}{1.038896in}}{\pgfqpoint{1.364900in}{1.038896in}}%
\pgfpathcurveto{\pgfqpoint{1.359076in}{1.038896in}}{\pgfqpoint{1.353490in}{1.036582in}}{\pgfqpoint{1.349372in}{1.032464in}}%
\pgfpathcurveto{\pgfqpoint{1.345254in}{1.028346in}}{\pgfqpoint{1.342940in}{1.022760in}}{\pgfqpoint{1.342940in}{1.016936in}}%
\pgfpathcurveto{\pgfqpoint{1.342940in}{1.011112in}}{\pgfqpoint{1.345254in}{1.005526in}}{\pgfqpoint{1.349372in}{1.001408in}}%
\pgfpathcurveto{\pgfqpoint{1.353490in}{0.997289in}}{\pgfqpoint{1.359076in}{0.994976in}}{\pgfqpoint{1.364900in}{0.994976in}}%
\pgfpathclose%
\pgfusepath{stroke,fill}%
\end{pgfscope}%
\begin{pgfscope}%
\pgfpathrectangle{\pgfqpoint{0.211875in}{0.211875in}}{\pgfqpoint{1.313625in}{1.279725in}}%
\pgfusepath{clip}%
\pgfsetbuttcap%
\pgfsetroundjoin%
\definecolor{currentfill}{rgb}{0.121569,0.466667,0.705882}%
\pgfsetfillcolor{currentfill}%
\pgfsetlinewidth{1.003750pt}%
\definecolor{currentstroke}{rgb}{0.121569,0.466667,0.705882}%
\pgfsetstrokecolor{currentstroke}%
\pgfsetdash{}{0pt}%
\pgfpathmoveto{\pgfqpoint{0.829758in}{1.024263in}}%
\pgfpathcurveto{\pgfqpoint{0.835582in}{1.024263in}}{\pgfqpoint{0.841168in}{1.026577in}}{\pgfqpoint{0.845286in}{1.030695in}}%
\pgfpathcurveto{\pgfqpoint{0.849404in}{1.034813in}}{\pgfqpoint{0.851718in}{1.040400in}}{\pgfqpoint{0.851718in}{1.046224in}}%
\pgfpathcurveto{\pgfqpoint{0.851718in}{1.052047in}}{\pgfqpoint{0.849404in}{1.057634in}}{\pgfqpoint{0.845286in}{1.061752in}}%
\pgfpathcurveto{\pgfqpoint{0.841168in}{1.065870in}}{\pgfqpoint{0.835582in}{1.068184in}}{\pgfqpoint{0.829758in}{1.068184in}}%
\pgfpathcurveto{\pgfqpoint{0.823934in}{1.068184in}}{\pgfqpoint{0.818348in}{1.065870in}}{\pgfqpoint{0.814230in}{1.061752in}}%
\pgfpathcurveto{\pgfqpoint{0.810111in}{1.057634in}}{\pgfqpoint{0.807798in}{1.052047in}}{\pgfqpoint{0.807798in}{1.046224in}}%
\pgfpathcurveto{\pgfqpoint{0.807798in}{1.040400in}}{\pgfqpoint{0.810111in}{1.034813in}}{\pgfqpoint{0.814230in}{1.030695in}}%
\pgfpathcurveto{\pgfqpoint{0.818348in}{1.026577in}}{\pgfqpoint{0.823934in}{1.024263in}}{\pgfqpoint{0.829758in}{1.024263in}}%
\pgfpathclose%
\pgfusepath{stroke,fill}%
\end{pgfscope}%
\begin{pgfscope}%
\pgfpathrectangle{\pgfqpoint{0.211875in}{0.211875in}}{\pgfqpoint{1.313625in}{1.279725in}}%
\pgfusepath{clip}%
\pgfsetbuttcap%
\pgfsetroundjoin%
\definecolor{currentfill}{rgb}{0.121569,0.466667,0.705882}%
\pgfsetfillcolor{currentfill}%
\pgfsetlinewidth{1.003750pt}%
\definecolor{currentstroke}{rgb}{0.121569,0.466667,0.705882}%
\pgfsetstrokecolor{currentstroke}%
\pgfsetdash{}{0pt}%
\pgfpathmoveto{\pgfqpoint{0.765967in}{1.198954in}}%
\pgfpathcurveto{\pgfqpoint{0.771791in}{1.198954in}}{\pgfqpoint{0.777377in}{1.201268in}}{\pgfqpoint{0.781495in}{1.205386in}}%
\pgfpathcurveto{\pgfqpoint{0.785614in}{1.209504in}}{\pgfqpoint{0.787927in}{1.215091in}}{\pgfqpoint{0.787927in}{1.220914in}}%
\pgfpathcurveto{\pgfqpoint{0.787927in}{1.226738in}}{\pgfqpoint{0.785614in}{1.232325in}}{\pgfqpoint{0.781495in}{1.236443in}}%
\pgfpathcurveto{\pgfqpoint{0.777377in}{1.240561in}}{\pgfqpoint{0.771791in}{1.242875in}}{\pgfqpoint{0.765967in}{1.242875in}}%
\pgfpathcurveto{\pgfqpoint{0.760143in}{1.242875in}}{\pgfqpoint{0.754557in}{1.240561in}}{\pgfqpoint{0.750439in}{1.236443in}}%
\pgfpathcurveto{\pgfqpoint{0.746321in}{1.232325in}}{\pgfqpoint{0.744007in}{1.226738in}}{\pgfqpoint{0.744007in}{1.220914in}}%
\pgfpathcurveto{\pgfqpoint{0.744007in}{1.215091in}}{\pgfqpoint{0.746321in}{1.209504in}}{\pgfqpoint{0.750439in}{1.205386in}}%
\pgfpathcurveto{\pgfqpoint{0.754557in}{1.201268in}}{\pgfqpoint{0.760143in}{1.198954in}}{\pgfqpoint{0.765967in}{1.198954in}}%
\pgfpathclose%
\pgfusepath{stroke,fill}%
\end{pgfscope}%
\begin{pgfscope}%
\pgfpathrectangle{\pgfqpoint{0.211875in}{0.211875in}}{\pgfqpoint{1.313625in}{1.279725in}}%
\pgfusepath{clip}%
\pgfsetbuttcap%
\pgfsetroundjoin%
\definecolor{currentfill}{rgb}{0.121569,0.466667,0.705882}%
\pgfsetfillcolor{currentfill}%
\pgfsetlinewidth{1.003750pt}%
\definecolor{currentstroke}{rgb}{0.121569,0.466667,0.705882}%
\pgfsetstrokecolor{currentstroke}%
\pgfsetdash{}{0pt}%
\pgfpathmoveto{\pgfqpoint{0.705154in}{1.291846in}}%
\pgfpathcurveto{\pgfqpoint{0.710978in}{1.291846in}}{\pgfqpoint{0.716564in}{1.294159in}}{\pgfqpoint{0.720683in}{1.298278in}}%
\pgfpathcurveto{\pgfqpoint{0.724801in}{1.302396in}}{\pgfqpoint{0.727115in}{1.307982in}}{\pgfqpoint{0.727115in}{1.313806in}}%
\pgfpathcurveto{\pgfqpoint{0.727115in}{1.319630in}}{\pgfqpoint{0.724801in}{1.325216in}}{\pgfqpoint{0.720683in}{1.329334in}}%
\pgfpathcurveto{\pgfqpoint{0.716564in}{1.333452in}}{\pgfqpoint{0.710978in}{1.335766in}}{\pgfqpoint{0.705154in}{1.335766in}}%
\pgfpathcurveto{\pgfqpoint{0.699330in}{1.335766in}}{\pgfqpoint{0.693744in}{1.333452in}}{\pgfqpoint{0.689626in}{1.329334in}}%
\pgfpathcurveto{\pgfqpoint{0.685508in}{1.325216in}}{\pgfqpoint{0.683194in}{1.319630in}}{\pgfqpoint{0.683194in}{1.313806in}}%
\pgfpathcurveto{\pgfqpoint{0.683194in}{1.307982in}}{\pgfqpoint{0.685508in}{1.302396in}}{\pgfqpoint{0.689626in}{1.298278in}}%
\pgfpathcurveto{\pgfqpoint{0.693744in}{1.294159in}}{\pgfqpoint{0.699330in}{1.291846in}}{\pgfqpoint{0.705154in}{1.291846in}}%
\pgfpathclose%
\pgfusepath{stroke,fill}%
\end{pgfscope}%
\begin{pgfscope}%
\pgfpathrectangle{\pgfqpoint{0.211875in}{0.211875in}}{\pgfqpoint{1.313625in}{1.279725in}}%
\pgfusepath{clip}%
\pgfsetbuttcap%
\pgfsetroundjoin%
\definecolor{currentfill}{rgb}{0.121569,0.466667,0.705882}%
\pgfsetfillcolor{currentfill}%
\pgfsetlinewidth{1.003750pt}%
\definecolor{currentstroke}{rgb}{0.121569,0.466667,0.705882}%
\pgfsetstrokecolor{currentstroke}%
\pgfsetdash{}{0pt}%
\pgfpathmoveto{\pgfqpoint{1.437991in}{1.077359in}}%
\pgfpathcurveto{\pgfqpoint{1.443815in}{1.077359in}}{\pgfqpoint{1.449401in}{1.079673in}}{\pgfqpoint{1.453519in}{1.083791in}}%
\pgfpathcurveto{\pgfqpoint{1.457637in}{1.087909in}}{\pgfqpoint{1.459951in}{1.093495in}}{\pgfqpoint{1.459951in}{1.099319in}}%
\pgfpathcurveto{\pgfqpoint{1.459951in}{1.105143in}}{\pgfqpoint{1.457637in}{1.110730in}}{\pgfqpoint{1.453519in}{1.114848in}}%
\pgfpathcurveto{\pgfqpoint{1.449401in}{1.118966in}}{\pgfqpoint{1.443815in}{1.121280in}}{\pgfqpoint{1.437991in}{1.121280in}}%
\pgfpathcurveto{\pgfqpoint{1.432167in}{1.121280in}}{\pgfqpoint{1.426581in}{1.118966in}}{\pgfqpoint{1.422463in}{1.114848in}}%
\pgfpathcurveto{\pgfqpoint{1.418345in}{1.110730in}}{\pgfqpoint{1.416031in}{1.105143in}}{\pgfqpoint{1.416031in}{1.099319in}}%
\pgfpathcurveto{\pgfqpoint{1.416031in}{1.093495in}}{\pgfqpoint{1.418345in}{1.087909in}}{\pgfqpoint{1.422463in}{1.083791in}}%
\pgfpathcurveto{\pgfqpoint{1.426581in}{1.079673in}}{\pgfqpoint{1.432167in}{1.077359in}}{\pgfqpoint{1.437991in}{1.077359in}}%
\pgfpathclose%
\pgfusepath{stroke,fill}%
\end{pgfscope}%
\begin{pgfscope}%
\pgfpathrectangle{\pgfqpoint{0.211875in}{0.211875in}}{\pgfqpoint{1.313625in}{1.279725in}}%
\pgfusepath{clip}%
\pgfsetbuttcap%
\pgfsetroundjoin%
\definecolor{currentfill}{rgb}{0.121569,0.466667,0.705882}%
\pgfsetfillcolor{currentfill}%
\pgfsetlinewidth{1.003750pt}%
\definecolor{currentstroke}{rgb}{0.121569,0.466667,0.705882}%
\pgfsetstrokecolor{currentstroke}%
\pgfsetdash{}{0pt}%
\pgfpathmoveto{\pgfqpoint{0.754799in}{1.349925in}}%
\pgfpathcurveto{\pgfqpoint{0.760623in}{1.349925in}}{\pgfqpoint{0.766210in}{1.352239in}}{\pgfqpoint{0.770328in}{1.356357in}}%
\pgfpathcurveto{\pgfqpoint{0.774446in}{1.360475in}}{\pgfqpoint{0.776760in}{1.366061in}}{\pgfqpoint{0.776760in}{1.371885in}}%
\pgfpathcurveto{\pgfqpoint{0.776760in}{1.377709in}}{\pgfqpoint{0.774446in}{1.383295in}}{\pgfqpoint{0.770328in}{1.387414in}}%
\pgfpathcurveto{\pgfqpoint{0.766210in}{1.391532in}}{\pgfqpoint{0.760623in}{1.393846in}}{\pgfqpoint{0.754799in}{1.393846in}}%
\pgfpathcurveto{\pgfqpoint{0.748976in}{1.393846in}}{\pgfqpoint{0.743389in}{1.391532in}}{\pgfqpoint{0.739271in}{1.387414in}}%
\pgfpathcurveto{\pgfqpoint{0.735153in}{1.383295in}}{\pgfqpoint{0.732839in}{1.377709in}}{\pgfqpoint{0.732839in}{1.371885in}}%
\pgfpathcurveto{\pgfqpoint{0.732839in}{1.366061in}}{\pgfqpoint{0.735153in}{1.360475in}}{\pgfqpoint{0.739271in}{1.356357in}}%
\pgfpathcurveto{\pgfqpoint{0.743389in}{1.352239in}}{\pgfqpoint{0.748976in}{1.349925in}}{\pgfqpoint{0.754799in}{1.349925in}}%
\pgfpathclose%
\pgfusepath{stroke,fill}%
\end{pgfscope}%
\begin{pgfscope}%
\pgfpathrectangle{\pgfqpoint{0.211875in}{0.211875in}}{\pgfqpoint{1.313625in}{1.279725in}}%
\pgfusepath{clip}%
\pgfsetbuttcap%
\pgfsetroundjoin%
\definecolor{currentfill}{rgb}{0.121569,0.466667,0.705882}%
\pgfsetfillcolor{currentfill}%
\pgfsetlinewidth{1.003750pt}%
\definecolor{currentstroke}{rgb}{0.121569,0.466667,0.705882}%
\pgfsetstrokecolor{currentstroke}%
\pgfsetdash{}{0pt}%
\pgfpathmoveto{\pgfqpoint{1.390189in}{1.075470in}}%
\pgfpathcurveto{\pgfqpoint{1.396013in}{1.075470in}}{\pgfqpoint{1.401599in}{1.077784in}}{\pgfqpoint{1.405717in}{1.081902in}}%
\pgfpathcurveto{\pgfqpoint{1.409835in}{1.086020in}}{\pgfqpoint{1.412149in}{1.091606in}}{\pgfqpoint{1.412149in}{1.097430in}}%
\pgfpathcurveto{\pgfqpoint{1.412149in}{1.103254in}}{\pgfqpoint{1.409835in}{1.108840in}}{\pgfqpoint{1.405717in}{1.112958in}}%
\pgfpathcurveto{\pgfqpoint{1.401599in}{1.117076in}}{\pgfqpoint{1.396013in}{1.119390in}}{\pgfqpoint{1.390189in}{1.119390in}}%
\pgfpathcurveto{\pgfqpoint{1.384365in}{1.119390in}}{\pgfqpoint{1.378779in}{1.117076in}}{\pgfqpoint{1.374661in}{1.112958in}}%
\pgfpathcurveto{\pgfqpoint{1.370543in}{1.108840in}}{\pgfqpoint{1.368229in}{1.103254in}}{\pgfqpoint{1.368229in}{1.097430in}}%
\pgfpathcurveto{\pgfqpoint{1.368229in}{1.091606in}}{\pgfqpoint{1.370543in}{1.086020in}}{\pgfqpoint{1.374661in}{1.081902in}}%
\pgfpathcurveto{\pgfqpoint{1.378779in}{1.077784in}}{\pgfqpoint{1.384365in}{1.075470in}}{\pgfqpoint{1.390189in}{1.075470in}}%
\pgfpathclose%
\pgfusepath{stroke,fill}%
\end{pgfscope}%
\begin{pgfscope}%
\pgfpathrectangle{\pgfqpoint{0.211875in}{0.211875in}}{\pgfqpoint{1.313625in}{1.279725in}}%
\pgfusepath{clip}%
\pgfsetbuttcap%
\pgfsetroundjoin%
\definecolor{currentfill}{rgb}{0.121569,0.466667,0.705882}%
\pgfsetfillcolor{currentfill}%
\pgfsetlinewidth{1.003750pt}%
\definecolor{currentstroke}{rgb}{0.121569,0.466667,0.705882}%
\pgfsetstrokecolor{currentstroke}%
\pgfsetdash{}{0pt}%
\pgfpathmoveto{\pgfqpoint{0.761656in}{1.112839in}}%
\pgfpathcurveto{\pgfqpoint{0.767480in}{1.112839in}}{\pgfqpoint{0.773067in}{1.115153in}}{\pgfqpoint{0.777185in}{1.119271in}}%
\pgfpathcurveto{\pgfqpoint{0.781303in}{1.123389in}}{\pgfqpoint{0.783617in}{1.128976in}}{\pgfqpoint{0.783617in}{1.134799in}}%
\pgfpathcurveto{\pgfqpoint{0.783617in}{1.140623in}}{\pgfqpoint{0.781303in}{1.146210in}}{\pgfqpoint{0.777185in}{1.150328in}}%
\pgfpathcurveto{\pgfqpoint{0.773067in}{1.154446in}}{\pgfqpoint{0.767480in}{1.156760in}}{\pgfqpoint{0.761656in}{1.156760in}}%
\pgfpathcurveto{\pgfqpoint{0.755832in}{1.156760in}}{\pgfqpoint{0.750246in}{1.154446in}}{\pgfqpoint{0.746128in}{1.150328in}}%
\pgfpathcurveto{\pgfqpoint{0.742010in}{1.146210in}}{\pgfqpoint{0.739696in}{1.140623in}}{\pgfqpoint{0.739696in}{1.134799in}}%
\pgfpathcurveto{\pgfqpoint{0.739696in}{1.128976in}}{\pgfqpoint{0.742010in}{1.123389in}}{\pgfqpoint{0.746128in}{1.119271in}}%
\pgfpathcurveto{\pgfqpoint{0.750246in}{1.115153in}}{\pgfqpoint{0.755832in}{1.112839in}}{\pgfqpoint{0.761656in}{1.112839in}}%
\pgfpathclose%
\pgfusepath{stroke,fill}%
\end{pgfscope}%
\begin{pgfscope}%
\pgfpathrectangle{\pgfqpoint{0.211875in}{0.211875in}}{\pgfqpoint{1.313625in}{1.279725in}}%
\pgfusepath{clip}%
\pgfsetbuttcap%
\pgfsetroundjoin%
\definecolor{currentfill}{rgb}{0.121569,0.466667,0.705882}%
\pgfsetfillcolor{currentfill}%
\pgfsetlinewidth{1.003750pt}%
\definecolor{currentstroke}{rgb}{0.121569,0.466667,0.705882}%
\pgfsetstrokecolor{currentstroke}%
\pgfsetdash{}{0pt}%
\pgfpathmoveto{\pgfqpoint{0.808251in}{1.259464in}}%
\pgfpathcurveto{\pgfqpoint{0.814075in}{1.259464in}}{\pgfqpoint{0.819661in}{1.261778in}}{\pgfqpoint{0.823780in}{1.265896in}}%
\pgfpathcurveto{\pgfqpoint{0.827898in}{1.270014in}}{\pgfqpoint{0.830212in}{1.275601in}}{\pgfqpoint{0.830212in}{1.281424in}}%
\pgfpathcurveto{\pgfqpoint{0.830212in}{1.287248in}}{\pgfqpoint{0.827898in}{1.292835in}}{\pgfqpoint{0.823780in}{1.296953in}}%
\pgfpathcurveto{\pgfqpoint{0.819661in}{1.301071in}}{\pgfqpoint{0.814075in}{1.303385in}}{\pgfqpoint{0.808251in}{1.303385in}}%
\pgfpathcurveto{\pgfqpoint{0.802427in}{1.303385in}}{\pgfqpoint{0.796841in}{1.301071in}}{\pgfqpoint{0.792723in}{1.296953in}}%
\pgfpathcurveto{\pgfqpoint{0.788605in}{1.292835in}}{\pgfqpoint{0.786291in}{1.287248in}}{\pgfqpoint{0.786291in}{1.281424in}}%
\pgfpathcurveto{\pgfqpoint{0.786291in}{1.275601in}}{\pgfqpoint{0.788605in}{1.270014in}}{\pgfqpoint{0.792723in}{1.265896in}}%
\pgfpathcurveto{\pgfqpoint{0.796841in}{1.261778in}}{\pgfqpoint{0.802427in}{1.259464in}}{\pgfqpoint{0.808251in}{1.259464in}}%
\pgfpathclose%
\pgfusepath{stroke,fill}%
\end{pgfscope}%
\begin{pgfscope}%
\pgfpathrectangle{\pgfqpoint{0.211875in}{0.211875in}}{\pgfqpoint{1.313625in}{1.279725in}}%
\pgfusepath{clip}%
\pgfsetbuttcap%
\pgfsetroundjoin%
\definecolor{currentfill}{rgb}{0.121569,0.466667,0.705882}%
\pgfsetfillcolor{currentfill}%
\pgfsetlinewidth{1.003750pt}%
\definecolor{currentstroke}{rgb}{0.121569,0.466667,0.705882}%
\pgfsetstrokecolor{currentstroke}%
\pgfsetdash{}{0pt}%
\pgfpathmoveto{\pgfqpoint{1.087406in}{1.055068in}}%
\pgfpathcurveto{\pgfqpoint{1.093230in}{1.055068in}}{\pgfqpoint{1.098816in}{1.057382in}}{\pgfqpoint{1.102934in}{1.061500in}}%
\pgfpathcurveto{\pgfqpoint{1.107052in}{1.065618in}}{\pgfqpoint{1.109366in}{1.071204in}}{\pgfqpoint{1.109366in}{1.077028in}}%
\pgfpathcurveto{\pgfqpoint{1.109366in}{1.082852in}}{\pgfqpoint{1.107052in}{1.088438in}}{\pgfqpoint{1.102934in}{1.092557in}}%
\pgfpathcurveto{\pgfqpoint{1.098816in}{1.096675in}}{\pgfqpoint{1.093230in}{1.098989in}}{\pgfqpoint{1.087406in}{1.098989in}}%
\pgfpathcurveto{\pgfqpoint{1.081582in}{1.098989in}}{\pgfqpoint{1.075996in}{1.096675in}}{\pgfqpoint{1.071878in}{1.092557in}}%
\pgfpathcurveto{\pgfqpoint{1.067759in}{1.088438in}}{\pgfqpoint{1.065446in}{1.082852in}}{\pgfqpoint{1.065446in}{1.077028in}}%
\pgfpathcurveto{\pgfqpoint{1.065446in}{1.071204in}}{\pgfqpoint{1.067759in}{1.065618in}}{\pgfqpoint{1.071878in}{1.061500in}}%
\pgfpathcurveto{\pgfqpoint{1.075996in}{1.057382in}}{\pgfqpoint{1.081582in}{1.055068in}}{\pgfqpoint{1.087406in}{1.055068in}}%
\pgfpathclose%
\pgfusepath{stroke,fill}%
\end{pgfscope}%
\begin{pgfscope}%
\pgfpathrectangle{\pgfqpoint{0.211875in}{0.211875in}}{\pgfqpoint{1.313625in}{1.279725in}}%
\pgfusepath{clip}%
\pgfsetbuttcap%
\pgfsetroundjoin%
\definecolor{currentfill}{rgb}{0.121569,0.466667,0.705882}%
\pgfsetfillcolor{currentfill}%
\pgfsetlinewidth{1.003750pt}%
\definecolor{currentstroke}{rgb}{0.121569,0.466667,0.705882}%
\pgfsetstrokecolor{currentstroke}%
\pgfsetdash{}{0pt}%
\pgfpathmoveto{\pgfqpoint{1.332764in}{1.028345in}}%
\pgfpathcurveto{\pgfqpoint{1.338588in}{1.028345in}}{\pgfqpoint{1.344174in}{1.030659in}}{\pgfqpoint{1.348292in}{1.034777in}}%
\pgfpathcurveto{\pgfqpoint{1.352410in}{1.038895in}}{\pgfqpoint{1.354724in}{1.044482in}}{\pgfqpoint{1.354724in}{1.050305in}}%
\pgfpathcurveto{\pgfqpoint{1.354724in}{1.056129in}}{\pgfqpoint{1.352410in}{1.061716in}}{\pgfqpoint{1.348292in}{1.065834in}}%
\pgfpathcurveto{\pgfqpoint{1.344174in}{1.069952in}}{\pgfqpoint{1.338588in}{1.072266in}}{\pgfqpoint{1.332764in}{1.072266in}}%
\pgfpathcurveto{\pgfqpoint{1.326940in}{1.072266in}}{\pgfqpoint{1.321354in}{1.069952in}}{\pgfqpoint{1.317235in}{1.065834in}}%
\pgfpathcurveto{\pgfqpoint{1.313117in}{1.061716in}}{\pgfqpoint{1.310803in}{1.056129in}}{\pgfqpoint{1.310803in}{1.050305in}}%
\pgfpathcurveto{\pgfqpoint{1.310803in}{1.044482in}}{\pgfqpoint{1.313117in}{1.038895in}}{\pgfqpoint{1.317235in}{1.034777in}}%
\pgfpathcurveto{\pgfqpoint{1.321354in}{1.030659in}}{\pgfqpoint{1.326940in}{1.028345in}}{\pgfqpoint{1.332764in}{1.028345in}}%
\pgfpathclose%
\pgfusepath{stroke,fill}%
\end{pgfscope}%
\begin{pgfscope}%
\pgfpathrectangle{\pgfqpoint{0.211875in}{0.211875in}}{\pgfqpoint{1.313625in}{1.279725in}}%
\pgfusepath{clip}%
\pgfsetbuttcap%
\pgfsetroundjoin%
\definecolor{currentfill}{rgb}{0.121569,0.466667,0.705882}%
\pgfsetfillcolor{currentfill}%
\pgfsetlinewidth{1.003750pt}%
\definecolor{currentstroke}{rgb}{0.121569,0.466667,0.705882}%
\pgfsetstrokecolor{currentstroke}%
\pgfsetdash{}{0pt}%
\pgfpathmoveto{\pgfqpoint{1.322202in}{0.947469in}}%
\pgfpathcurveto{\pgfqpoint{1.328026in}{0.947469in}}{\pgfqpoint{1.333612in}{0.949783in}}{\pgfqpoint{1.337730in}{0.953901in}}%
\pgfpathcurveto{\pgfqpoint{1.341849in}{0.958019in}}{\pgfqpoint{1.344162in}{0.963605in}}{\pgfqpoint{1.344162in}{0.969429in}}%
\pgfpathcurveto{\pgfqpoint{1.344162in}{0.975253in}}{\pgfqpoint{1.341849in}{0.980839in}}{\pgfqpoint{1.337730in}{0.984957in}}%
\pgfpathcurveto{\pgfqpoint{1.333612in}{0.989075in}}{\pgfqpoint{1.328026in}{0.991389in}}{\pgfqpoint{1.322202in}{0.991389in}}%
\pgfpathcurveto{\pgfqpoint{1.316378in}{0.991389in}}{\pgfqpoint{1.310792in}{0.989075in}}{\pgfqpoint{1.306674in}{0.984957in}}%
\pgfpathcurveto{\pgfqpoint{1.302556in}{0.980839in}}{\pgfqpoint{1.300242in}{0.975253in}}{\pgfqpoint{1.300242in}{0.969429in}}%
\pgfpathcurveto{\pgfqpoint{1.300242in}{0.963605in}}{\pgfqpoint{1.302556in}{0.958019in}}{\pgfqpoint{1.306674in}{0.953901in}}%
\pgfpathcurveto{\pgfqpoint{1.310792in}{0.949783in}}{\pgfqpoint{1.316378in}{0.947469in}}{\pgfqpoint{1.322202in}{0.947469in}}%
\pgfpathclose%
\pgfusepath{stroke,fill}%
\end{pgfscope}%
\begin{pgfscope}%
\pgfpathrectangle{\pgfqpoint{0.211875in}{0.211875in}}{\pgfqpoint{1.313625in}{1.279725in}}%
\pgfusepath{clip}%
\pgfsetbuttcap%
\pgfsetroundjoin%
\definecolor{currentfill}{rgb}{0.121569,0.466667,0.705882}%
\pgfsetfillcolor{currentfill}%
\pgfsetlinewidth{1.003750pt}%
\definecolor{currentstroke}{rgb}{0.121569,0.466667,0.705882}%
\pgfsetstrokecolor{currentstroke}%
\pgfsetdash{}{0pt}%
\pgfpathmoveto{\pgfqpoint{1.111030in}{1.042072in}}%
\pgfpathcurveto{\pgfqpoint{1.116854in}{1.042072in}}{\pgfqpoint{1.122440in}{1.044386in}}{\pgfqpoint{1.126558in}{1.048504in}}%
\pgfpathcurveto{\pgfqpoint{1.130677in}{1.052622in}}{\pgfqpoint{1.132990in}{1.058208in}}{\pgfqpoint{1.132990in}{1.064032in}}%
\pgfpathcurveto{\pgfqpoint{1.132990in}{1.069856in}}{\pgfqpoint{1.130677in}{1.075442in}}{\pgfqpoint{1.126558in}{1.079560in}}%
\pgfpathcurveto{\pgfqpoint{1.122440in}{1.083678in}}{\pgfqpoint{1.116854in}{1.085992in}}{\pgfqpoint{1.111030in}{1.085992in}}%
\pgfpathcurveto{\pgfqpoint{1.105206in}{1.085992in}}{\pgfqpoint{1.099620in}{1.083678in}}{\pgfqpoint{1.095502in}{1.079560in}}%
\pgfpathcurveto{\pgfqpoint{1.091384in}{1.075442in}}{\pgfqpoint{1.089070in}{1.069856in}}{\pgfqpoint{1.089070in}{1.064032in}}%
\pgfpathcurveto{\pgfqpoint{1.089070in}{1.058208in}}{\pgfqpoint{1.091384in}{1.052622in}}{\pgfqpoint{1.095502in}{1.048504in}}%
\pgfpathcurveto{\pgfqpoint{1.099620in}{1.044386in}}{\pgfqpoint{1.105206in}{1.042072in}}{\pgfqpoint{1.111030in}{1.042072in}}%
\pgfpathclose%
\pgfusepath{stroke,fill}%
\end{pgfscope}%
\begin{pgfscope}%
\pgfpathrectangle{\pgfqpoint{0.211875in}{0.211875in}}{\pgfqpoint{1.313625in}{1.279725in}}%
\pgfusepath{clip}%
\pgfsetbuttcap%
\pgfsetroundjoin%
\definecolor{currentfill}{rgb}{0.121569,0.466667,0.705882}%
\pgfsetfillcolor{currentfill}%
\pgfsetlinewidth{1.003750pt}%
\definecolor{currentstroke}{rgb}{0.121569,0.466667,0.705882}%
\pgfsetstrokecolor{currentstroke}%
\pgfsetdash{}{0pt}%
\pgfpathmoveto{\pgfqpoint{0.804934in}{1.084446in}}%
\pgfpathcurveto{\pgfqpoint{0.810758in}{1.084446in}}{\pgfqpoint{0.816344in}{1.086760in}}{\pgfqpoint{0.820462in}{1.090878in}}%
\pgfpathcurveto{\pgfqpoint{0.824580in}{1.094996in}}{\pgfqpoint{0.826894in}{1.100582in}}{\pgfqpoint{0.826894in}{1.106406in}}%
\pgfpathcurveto{\pgfqpoint{0.826894in}{1.112230in}}{\pgfqpoint{0.824580in}{1.117816in}}{\pgfqpoint{0.820462in}{1.121934in}}%
\pgfpathcurveto{\pgfqpoint{0.816344in}{1.126053in}}{\pgfqpoint{0.810758in}{1.128366in}}{\pgfqpoint{0.804934in}{1.128366in}}%
\pgfpathcurveto{\pgfqpoint{0.799110in}{1.128366in}}{\pgfqpoint{0.793524in}{1.126053in}}{\pgfqpoint{0.789406in}{1.121934in}}%
\pgfpathcurveto{\pgfqpoint{0.785287in}{1.117816in}}{\pgfqpoint{0.782974in}{1.112230in}}{\pgfqpoint{0.782974in}{1.106406in}}%
\pgfpathcurveto{\pgfqpoint{0.782974in}{1.100582in}}{\pgfqpoint{0.785287in}{1.094996in}}{\pgfqpoint{0.789406in}{1.090878in}}%
\pgfpathcurveto{\pgfqpoint{0.793524in}{1.086760in}}{\pgfqpoint{0.799110in}{1.084446in}}{\pgfqpoint{0.804934in}{1.084446in}}%
\pgfpathclose%
\pgfusepath{stroke,fill}%
\end{pgfscope}%
\begin{pgfscope}%
\pgfpathrectangle{\pgfqpoint{0.211875in}{0.211875in}}{\pgfqpoint{1.313625in}{1.279725in}}%
\pgfusepath{clip}%
\pgfsetbuttcap%
\pgfsetroundjoin%
\definecolor{currentfill}{rgb}{0.121569,0.466667,0.705882}%
\pgfsetfillcolor{currentfill}%
\pgfsetlinewidth{1.003750pt}%
\definecolor{currentstroke}{rgb}{0.121569,0.466667,0.705882}%
\pgfsetstrokecolor{currentstroke}%
\pgfsetdash{}{0pt}%
\pgfpathmoveto{\pgfqpoint{0.801898in}{1.104398in}}%
\pgfpathcurveto{\pgfqpoint{0.807722in}{1.104398in}}{\pgfqpoint{0.813308in}{1.106711in}}{\pgfqpoint{0.817426in}{1.110830in}}%
\pgfpathcurveto{\pgfqpoint{0.821545in}{1.114948in}}{\pgfqpoint{0.823858in}{1.120534in}}{\pgfqpoint{0.823858in}{1.126358in}}%
\pgfpathcurveto{\pgfqpoint{0.823858in}{1.132182in}}{\pgfqpoint{0.821545in}{1.137768in}}{\pgfqpoint{0.817426in}{1.141886in}}%
\pgfpathcurveto{\pgfqpoint{0.813308in}{1.146004in}}{\pgfqpoint{0.807722in}{1.148318in}}{\pgfqpoint{0.801898in}{1.148318in}}%
\pgfpathcurveto{\pgfqpoint{0.796074in}{1.148318in}}{\pgfqpoint{0.790488in}{1.146004in}}{\pgfqpoint{0.786370in}{1.141886in}}%
\pgfpathcurveto{\pgfqpoint{0.782252in}{1.137768in}}{\pgfqpoint{0.779938in}{1.132182in}}{\pgfqpoint{0.779938in}{1.126358in}}%
\pgfpathcurveto{\pgfqpoint{0.779938in}{1.120534in}}{\pgfqpoint{0.782252in}{1.114948in}}{\pgfqpoint{0.786370in}{1.110830in}}%
\pgfpathcurveto{\pgfqpoint{0.790488in}{1.106711in}}{\pgfqpoint{0.796074in}{1.104398in}}{\pgfqpoint{0.801898in}{1.104398in}}%
\pgfpathclose%
\pgfusepath{stroke,fill}%
\end{pgfscope}%
\begin{pgfscope}%
\pgfpathrectangle{\pgfqpoint{0.211875in}{0.211875in}}{\pgfqpoint{1.313625in}{1.279725in}}%
\pgfusepath{clip}%
\pgfsetbuttcap%
\pgfsetroundjoin%
\definecolor{currentfill}{rgb}{0.121569,0.466667,0.705882}%
\pgfsetfillcolor{currentfill}%
\pgfsetlinewidth{1.003750pt}%
\definecolor{currentstroke}{rgb}{0.121569,0.466667,0.705882}%
\pgfsetstrokecolor{currentstroke}%
\pgfsetdash{}{0pt}%
\pgfpathmoveto{\pgfqpoint{1.364845in}{0.993493in}}%
\pgfpathcurveto{\pgfqpoint{1.370669in}{0.993493in}}{\pgfqpoint{1.376255in}{0.995807in}}{\pgfqpoint{1.380373in}{0.999925in}}%
\pgfpathcurveto{\pgfqpoint{1.384491in}{1.004043in}}{\pgfqpoint{1.386805in}{1.009630in}}{\pgfqpoint{1.386805in}{1.015453in}}%
\pgfpathcurveto{\pgfqpoint{1.386805in}{1.021277in}}{\pgfqpoint{1.384491in}{1.026864in}}{\pgfqpoint{1.380373in}{1.030982in}}%
\pgfpathcurveto{\pgfqpoint{1.376255in}{1.035100in}}{\pgfqpoint{1.370669in}{1.037414in}}{\pgfqpoint{1.364845in}{1.037414in}}%
\pgfpathcurveto{\pgfqpoint{1.359021in}{1.037414in}}{\pgfqpoint{1.353435in}{1.035100in}}{\pgfqpoint{1.349317in}{1.030982in}}%
\pgfpathcurveto{\pgfqpoint{1.345199in}{1.026864in}}{\pgfqpoint{1.342885in}{1.021277in}}{\pgfqpoint{1.342885in}{1.015453in}}%
\pgfpathcurveto{\pgfqpoint{1.342885in}{1.009630in}}{\pgfqpoint{1.345199in}{1.004043in}}{\pgfqpoint{1.349317in}{0.999925in}}%
\pgfpathcurveto{\pgfqpoint{1.353435in}{0.995807in}}{\pgfqpoint{1.359021in}{0.993493in}}{\pgfqpoint{1.364845in}{0.993493in}}%
\pgfpathclose%
\pgfusepath{stroke,fill}%
\end{pgfscope}%
\begin{pgfscope}%
\pgfpathrectangle{\pgfqpoint{0.211875in}{0.211875in}}{\pgfqpoint{1.313625in}{1.279725in}}%
\pgfusepath{clip}%
\pgfsetbuttcap%
\pgfsetroundjoin%
\definecolor{currentfill}{rgb}{0.121569,0.466667,0.705882}%
\pgfsetfillcolor{currentfill}%
\pgfsetlinewidth{1.003750pt}%
\definecolor{currentstroke}{rgb}{0.121569,0.466667,0.705882}%
\pgfsetstrokecolor{currentstroke}%
\pgfsetdash{}{0pt}%
\pgfpathmoveto{\pgfqpoint{1.070707in}{1.128052in}}%
\pgfpathcurveto{\pgfqpoint{1.076531in}{1.128052in}}{\pgfqpoint{1.082117in}{1.130366in}}{\pgfqpoint{1.086236in}{1.134484in}}%
\pgfpathcurveto{\pgfqpoint{1.090354in}{1.138602in}}{\pgfqpoint{1.092668in}{1.144188in}}{\pgfqpoint{1.092668in}{1.150012in}}%
\pgfpathcurveto{\pgfqpoint{1.092668in}{1.155836in}}{\pgfqpoint{1.090354in}{1.161422in}}{\pgfqpoint{1.086236in}{1.165540in}}%
\pgfpathcurveto{\pgfqpoint{1.082117in}{1.169658in}}{\pgfqpoint{1.076531in}{1.171972in}}{\pgfqpoint{1.070707in}{1.171972in}}%
\pgfpathcurveto{\pgfqpoint{1.064883in}{1.171972in}}{\pgfqpoint{1.059297in}{1.169658in}}{\pgfqpoint{1.055179in}{1.165540in}}%
\pgfpathcurveto{\pgfqpoint{1.051061in}{1.161422in}}{\pgfqpoint{1.048747in}{1.155836in}}{\pgfqpoint{1.048747in}{1.150012in}}%
\pgfpathcurveto{\pgfqpoint{1.048747in}{1.144188in}}{\pgfqpoint{1.051061in}{1.138602in}}{\pgfqpoint{1.055179in}{1.134484in}}%
\pgfpathcurveto{\pgfqpoint{1.059297in}{1.130366in}}{\pgfqpoint{1.064883in}{1.128052in}}{\pgfqpoint{1.070707in}{1.128052in}}%
\pgfpathclose%
\pgfusepath{stroke,fill}%
\end{pgfscope}%
\begin{pgfscope}%
\pgfpathrectangle{\pgfqpoint{0.211875in}{0.211875in}}{\pgfqpoint{1.313625in}{1.279725in}}%
\pgfusepath{clip}%
\pgfsetbuttcap%
\pgfsetroundjoin%
\definecolor{currentfill}{rgb}{0.121569,0.466667,0.705882}%
\pgfsetfillcolor{currentfill}%
\pgfsetlinewidth{1.003750pt}%
\definecolor{currentstroke}{rgb}{0.121569,0.466667,0.705882}%
\pgfsetstrokecolor{currentstroke}%
\pgfsetdash{}{0pt}%
\pgfpathmoveto{\pgfqpoint{1.446373in}{1.355833in}}%
\pgfpathcurveto{\pgfqpoint{1.452197in}{1.355833in}}{\pgfqpoint{1.457783in}{1.358147in}}{\pgfqpoint{1.461901in}{1.362265in}}%
\pgfpathcurveto{\pgfqpoint{1.466019in}{1.366383in}}{\pgfqpoint{1.468333in}{1.371969in}}{\pgfqpoint{1.468333in}{1.377793in}}%
\pgfpathcurveto{\pgfqpoint{1.468333in}{1.383617in}}{\pgfqpoint{1.466019in}{1.389203in}}{\pgfqpoint{1.461901in}{1.393321in}}%
\pgfpathcurveto{\pgfqpoint{1.457783in}{1.397439in}}{\pgfqpoint{1.452197in}{1.399753in}}{\pgfqpoint{1.446373in}{1.399753in}}%
\pgfpathcurveto{\pgfqpoint{1.440549in}{1.399753in}}{\pgfqpoint{1.434963in}{1.397439in}}{\pgfqpoint{1.430845in}{1.393321in}}%
\pgfpathcurveto{\pgfqpoint{1.426726in}{1.389203in}}{\pgfqpoint{1.424413in}{1.383617in}}{\pgfqpoint{1.424413in}{1.377793in}}%
\pgfpathcurveto{\pgfqpoint{1.424413in}{1.371969in}}{\pgfqpoint{1.426726in}{1.366383in}}{\pgfqpoint{1.430845in}{1.362265in}}%
\pgfpathcurveto{\pgfqpoint{1.434963in}{1.358147in}}{\pgfqpoint{1.440549in}{1.355833in}}{\pgfqpoint{1.446373in}{1.355833in}}%
\pgfpathclose%
\pgfusepath{stroke,fill}%
\end{pgfscope}%
\begin{pgfscope}%
\pgfpathrectangle{\pgfqpoint{0.211875in}{0.211875in}}{\pgfqpoint{1.313625in}{1.279725in}}%
\pgfusepath{clip}%
\pgfsetbuttcap%
\pgfsetroundjoin%
\definecolor{currentfill}{rgb}{0.121569,0.466667,0.705882}%
\pgfsetfillcolor{currentfill}%
\pgfsetlinewidth{1.003750pt}%
\definecolor{currentstroke}{rgb}{0.121569,0.466667,0.705882}%
\pgfsetstrokecolor{currentstroke}%
\pgfsetdash{}{0pt}%
\pgfpathmoveto{\pgfqpoint{1.244507in}{1.135749in}}%
\pgfpathcurveto{\pgfqpoint{1.250331in}{1.135749in}}{\pgfqpoint{1.255917in}{1.138063in}}{\pgfqpoint{1.260035in}{1.142181in}}%
\pgfpathcurveto{\pgfqpoint{1.264153in}{1.146299in}}{\pgfqpoint{1.266467in}{1.151885in}}{\pgfqpoint{1.266467in}{1.157709in}}%
\pgfpathcurveto{\pgfqpoint{1.266467in}{1.163533in}}{\pgfqpoint{1.264153in}{1.169119in}}{\pgfqpoint{1.260035in}{1.173237in}}%
\pgfpathcurveto{\pgfqpoint{1.255917in}{1.177355in}}{\pgfqpoint{1.250331in}{1.179669in}}{\pgfqpoint{1.244507in}{1.179669in}}%
\pgfpathcurveto{\pgfqpoint{1.238683in}{1.179669in}}{\pgfqpoint{1.233096in}{1.177355in}}{\pgfqpoint{1.228978in}{1.173237in}}%
\pgfpathcurveto{\pgfqpoint{1.224860in}{1.169119in}}{\pgfqpoint{1.222546in}{1.163533in}}{\pgfqpoint{1.222546in}{1.157709in}}%
\pgfpathcurveto{\pgfqpoint{1.222546in}{1.151885in}}{\pgfqpoint{1.224860in}{1.146299in}}{\pgfqpoint{1.228978in}{1.142181in}}%
\pgfpathcurveto{\pgfqpoint{1.233096in}{1.138063in}}{\pgfqpoint{1.238683in}{1.135749in}}{\pgfqpoint{1.244507in}{1.135749in}}%
\pgfpathclose%
\pgfusepath{stroke,fill}%
\end{pgfscope}%
\begin{pgfscope}%
\pgfpathrectangle{\pgfqpoint{0.211875in}{0.211875in}}{\pgfqpoint{1.313625in}{1.279725in}}%
\pgfusepath{clip}%
\pgfsetbuttcap%
\pgfsetroundjoin%
\definecolor{currentfill}{rgb}{0.121569,0.466667,0.705882}%
\pgfsetfillcolor{currentfill}%
\pgfsetlinewidth{1.003750pt}%
\definecolor{currentstroke}{rgb}{0.121569,0.466667,0.705882}%
\pgfsetstrokecolor{currentstroke}%
\pgfsetdash{}{0pt}%
\pgfpathmoveto{\pgfqpoint{1.098853in}{1.097475in}}%
\pgfpathcurveto{\pgfqpoint{1.104677in}{1.097475in}}{\pgfqpoint{1.110263in}{1.099789in}}{\pgfqpoint{1.114381in}{1.103907in}}%
\pgfpathcurveto{\pgfqpoint{1.118499in}{1.108025in}}{\pgfqpoint{1.120813in}{1.113612in}}{\pgfqpoint{1.120813in}{1.119436in}}%
\pgfpathcurveto{\pgfqpoint{1.120813in}{1.125259in}}{\pgfqpoint{1.118499in}{1.130846in}}{\pgfqpoint{1.114381in}{1.134964in}}%
\pgfpathcurveto{\pgfqpoint{1.110263in}{1.139082in}}{\pgfqpoint{1.104677in}{1.141396in}}{\pgfqpoint{1.098853in}{1.141396in}}%
\pgfpathcurveto{\pgfqpoint{1.093029in}{1.141396in}}{\pgfqpoint{1.087443in}{1.139082in}}{\pgfqpoint{1.083325in}{1.134964in}}%
\pgfpathcurveto{\pgfqpoint{1.079207in}{1.130846in}}{\pgfqpoint{1.076893in}{1.125259in}}{\pgfqpoint{1.076893in}{1.119436in}}%
\pgfpathcurveto{\pgfqpoint{1.076893in}{1.113612in}}{\pgfqpoint{1.079207in}{1.108025in}}{\pgfqpoint{1.083325in}{1.103907in}}%
\pgfpathcurveto{\pgfqpoint{1.087443in}{1.099789in}}{\pgfqpoint{1.093029in}{1.097475in}}{\pgfqpoint{1.098853in}{1.097475in}}%
\pgfpathclose%
\pgfusepath{stroke,fill}%
\end{pgfscope}%
\begin{pgfscope}%
\pgfpathrectangle{\pgfqpoint{0.211875in}{0.211875in}}{\pgfqpoint{1.313625in}{1.279725in}}%
\pgfusepath{clip}%
\pgfsetbuttcap%
\pgfsetroundjoin%
\definecolor{currentfill}{rgb}{0.121569,0.466667,0.705882}%
\pgfsetfillcolor{currentfill}%
\pgfsetlinewidth{1.003750pt}%
\definecolor{currentstroke}{rgb}{0.121569,0.466667,0.705882}%
\pgfsetstrokecolor{currentstroke}%
\pgfsetdash{}{0pt}%
\pgfpathmoveto{\pgfqpoint{1.325887in}{0.921324in}}%
\pgfpathcurveto{\pgfqpoint{1.331711in}{0.921324in}}{\pgfqpoint{1.337297in}{0.923638in}}{\pgfqpoint{1.341416in}{0.927756in}}%
\pgfpathcurveto{\pgfqpoint{1.345534in}{0.931874in}}{\pgfqpoint{1.347848in}{0.937460in}}{\pgfqpoint{1.347848in}{0.943284in}}%
\pgfpathcurveto{\pgfqpoint{1.347848in}{0.949108in}}{\pgfqpoint{1.345534in}{0.954694in}}{\pgfqpoint{1.341416in}{0.958812in}}%
\pgfpathcurveto{\pgfqpoint{1.337297in}{0.962930in}}{\pgfqpoint{1.331711in}{0.965244in}}{\pgfqpoint{1.325887in}{0.965244in}}%
\pgfpathcurveto{\pgfqpoint{1.320063in}{0.965244in}}{\pgfqpoint{1.314477in}{0.962930in}}{\pgfqpoint{1.310359in}{0.958812in}}%
\pgfpathcurveto{\pgfqpoint{1.306241in}{0.954694in}}{\pgfqpoint{1.303927in}{0.949108in}}{\pgfqpoint{1.303927in}{0.943284in}}%
\pgfpathcurveto{\pgfqpoint{1.303927in}{0.937460in}}{\pgfqpoint{1.306241in}{0.931874in}}{\pgfqpoint{1.310359in}{0.927756in}}%
\pgfpathcurveto{\pgfqpoint{1.314477in}{0.923638in}}{\pgfqpoint{1.320063in}{0.921324in}}{\pgfqpoint{1.325887in}{0.921324in}}%
\pgfpathclose%
\pgfusepath{stroke,fill}%
\end{pgfscope}%
\begin{pgfscope}%
\pgfpathrectangle{\pgfqpoint{0.211875in}{0.211875in}}{\pgfqpoint{1.313625in}{1.279725in}}%
\pgfusepath{clip}%
\pgfsetbuttcap%
\pgfsetroundjoin%
\definecolor{currentfill}{rgb}{0.121569,0.466667,0.705882}%
\pgfsetfillcolor{currentfill}%
\pgfsetlinewidth{1.003750pt}%
\definecolor{currentstroke}{rgb}{0.121569,0.466667,0.705882}%
\pgfsetstrokecolor{currentstroke}%
\pgfsetdash{}{0pt}%
\pgfpathmoveto{\pgfqpoint{1.097960in}{1.063923in}}%
\pgfpathcurveto{\pgfqpoint{1.103784in}{1.063923in}}{\pgfqpoint{1.109370in}{1.066237in}}{\pgfqpoint{1.113488in}{1.070355in}}%
\pgfpathcurveto{\pgfqpoint{1.117606in}{1.074473in}}{\pgfqpoint{1.119920in}{1.080060in}}{\pgfqpoint{1.119920in}{1.085884in}}%
\pgfpathcurveto{\pgfqpoint{1.119920in}{1.091708in}}{\pgfqpoint{1.117606in}{1.097294in}}{\pgfqpoint{1.113488in}{1.101412in}}%
\pgfpathcurveto{\pgfqpoint{1.109370in}{1.105530in}}{\pgfqpoint{1.103784in}{1.107844in}}{\pgfqpoint{1.097960in}{1.107844in}}%
\pgfpathcurveto{\pgfqpoint{1.092136in}{1.107844in}}{\pgfqpoint{1.086550in}{1.105530in}}{\pgfqpoint{1.082432in}{1.101412in}}%
\pgfpathcurveto{\pgfqpoint{1.078314in}{1.097294in}}{\pgfqpoint{1.076000in}{1.091708in}}{\pgfqpoint{1.076000in}{1.085884in}}%
\pgfpathcurveto{\pgfqpoint{1.076000in}{1.080060in}}{\pgfqpoint{1.078314in}{1.074473in}}{\pgfqpoint{1.082432in}{1.070355in}}%
\pgfpathcurveto{\pgfqpoint{1.086550in}{1.066237in}}{\pgfqpoint{1.092136in}{1.063923in}}{\pgfqpoint{1.097960in}{1.063923in}}%
\pgfpathclose%
\pgfusepath{stroke,fill}%
\end{pgfscope}%
\begin{pgfscope}%
\pgfpathrectangle{\pgfqpoint{0.211875in}{0.211875in}}{\pgfqpoint{1.313625in}{1.279725in}}%
\pgfusepath{clip}%
\pgfsetbuttcap%
\pgfsetroundjoin%
\definecolor{currentfill}{rgb}{0.121569,0.466667,0.705882}%
\pgfsetfillcolor{currentfill}%
\pgfsetlinewidth{1.003750pt}%
\definecolor{currentstroke}{rgb}{0.121569,0.466667,0.705882}%
\pgfsetstrokecolor{currentstroke}%
\pgfsetdash{}{0pt}%
\pgfpathmoveto{\pgfqpoint{0.901919in}{0.927953in}}%
\pgfpathcurveto{\pgfqpoint{0.907743in}{0.927953in}}{\pgfqpoint{0.913329in}{0.930266in}}{\pgfqpoint{0.917447in}{0.934385in}}%
\pgfpathcurveto{\pgfqpoint{0.921565in}{0.938503in}}{\pgfqpoint{0.923879in}{0.944089in}}{\pgfqpoint{0.923879in}{0.949913in}}%
\pgfpathcurveto{\pgfqpoint{0.923879in}{0.955737in}}{\pgfqpoint{0.921565in}{0.961323in}}{\pgfqpoint{0.917447in}{0.965441in}}%
\pgfpathcurveto{\pgfqpoint{0.913329in}{0.969559in}}{\pgfqpoint{0.907743in}{0.971873in}}{\pgfqpoint{0.901919in}{0.971873in}}%
\pgfpathcurveto{\pgfqpoint{0.896095in}{0.971873in}}{\pgfqpoint{0.890509in}{0.969559in}}{\pgfqpoint{0.886390in}{0.965441in}}%
\pgfpathcurveto{\pgfqpoint{0.882272in}{0.961323in}}{\pgfqpoint{0.879958in}{0.955737in}}{\pgfqpoint{0.879958in}{0.949913in}}%
\pgfpathcurveto{\pgfqpoint{0.879958in}{0.944089in}}{\pgfqpoint{0.882272in}{0.938503in}}{\pgfqpoint{0.886390in}{0.934385in}}%
\pgfpathcurveto{\pgfqpoint{0.890509in}{0.930266in}}{\pgfqpoint{0.896095in}{0.927953in}}{\pgfqpoint{0.901919in}{0.927953in}}%
\pgfpathclose%
\pgfusepath{stroke,fill}%
\end{pgfscope}%
\begin{pgfscope}%
\pgfpathrectangle{\pgfqpoint{0.211875in}{0.211875in}}{\pgfqpoint{1.313625in}{1.279725in}}%
\pgfusepath{clip}%
\pgfsetbuttcap%
\pgfsetroundjoin%
\definecolor{currentfill}{rgb}{0.121569,0.466667,0.705882}%
\pgfsetfillcolor{currentfill}%
\pgfsetlinewidth{1.003750pt}%
\definecolor{currentstroke}{rgb}{0.121569,0.466667,0.705882}%
\pgfsetstrokecolor{currentstroke}%
\pgfsetdash{}{0pt}%
\pgfpathmoveto{\pgfqpoint{0.930059in}{0.918938in}}%
\pgfpathcurveto{\pgfqpoint{0.935882in}{0.918938in}}{\pgfqpoint{0.941469in}{0.921252in}}{\pgfqpoint{0.945587in}{0.925370in}}%
\pgfpathcurveto{\pgfqpoint{0.949705in}{0.929488in}}{\pgfqpoint{0.952019in}{0.935074in}}{\pgfqpoint{0.952019in}{0.940898in}}%
\pgfpathcurveto{\pgfqpoint{0.952019in}{0.946722in}}{\pgfqpoint{0.949705in}{0.952308in}}{\pgfqpoint{0.945587in}{0.956426in}}%
\pgfpathcurveto{\pgfqpoint{0.941469in}{0.960544in}}{\pgfqpoint{0.935882in}{0.962858in}}{\pgfqpoint{0.930059in}{0.962858in}}%
\pgfpathcurveto{\pgfqpoint{0.924235in}{0.962858in}}{\pgfqpoint{0.918648in}{0.960544in}}{\pgfqpoint{0.914530in}{0.956426in}}%
\pgfpathcurveto{\pgfqpoint{0.910412in}{0.952308in}}{\pgfqpoint{0.908098in}{0.946722in}}{\pgfqpoint{0.908098in}{0.940898in}}%
\pgfpathcurveto{\pgfqpoint{0.908098in}{0.935074in}}{\pgfqpoint{0.910412in}{0.929488in}}{\pgfqpoint{0.914530in}{0.925370in}}%
\pgfpathcurveto{\pgfqpoint{0.918648in}{0.921252in}}{\pgfqpoint{0.924235in}{0.918938in}}{\pgfqpoint{0.930059in}{0.918938in}}%
\pgfpathclose%
\pgfusepath{stroke,fill}%
\end{pgfscope}%
\begin{pgfscope}%
\pgfpathrectangle{\pgfqpoint{0.211875in}{0.211875in}}{\pgfqpoint{1.313625in}{1.279725in}}%
\pgfusepath{clip}%
\pgfsetbuttcap%
\pgfsetroundjoin%
\definecolor{currentfill}{rgb}{0.121569,0.466667,0.705882}%
\pgfsetfillcolor{currentfill}%
\pgfsetlinewidth{1.003750pt}%
\definecolor{currentstroke}{rgb}{0.121569,0.466667,0.705882}%
\pgfsetstrokecolor{currentstroke}%
\pgfsetdash{}{0pt}%
\pgfpathmoveto{\pgfqpoint{1.018557in}{1.236493in}}%
\pgfpathcurveto{\pgfqpoint{1.024381in}{1.236493in}}{\pgfqpoint{1.029967in}{1.238807in}}{\pgfqpoint{1.034085in}{1.242925in}}%
\pgfpathcurveto{\pgfqpoint{1.038203in}{1.247043in}}{\pgfqpoint{1.040517in}{1.252629in}}{\pgfqpoint{1.040517in}{1.258453in}}%
\pgfpathcurveto{\pgfqpoint{1.040517in}{1.264277in}}{\pgfqpoint{1.038203in}{1.269863in}}{\pgfqpoint{1.034085in}{1.273981in}}%
\pgfpathcurveto{\pgfqpoint{1.029967in}{1.278100in}}{\pgfqpoint{1.024381in}{1.280413in}}{\pgfqpoint{1.018557in}{1.280413in}}%
\pgfpathcurveto{\pgfqpoint{1.012733in}{1.280413in}}{\pgfqpoint{1.007147in}{1.278100in}}{\pgfqpoint{1.003029in}{1.273981in}}%
\pgfpathcurveto{\pgfqpoint{0.998911in}{1.269863in}}{\pgfqpoint{0.996597in}{1.264277in}}{\pgfqpoint{0.996597in}{1.258453in}}%
\pgfpathcurveto{\pgfqpoint{0.996597in}{1.252629in}}{\pgfqpoint{0.998911in}{1.247043in}}{\pgfqpoint{1.003029in}{1.242925in}}%
\pgfpathcurveto{\pgfqpoint{1.007147in}{1.238807in}}{\pgfqpoint{1.012733in}{1.236493in}}{\pgfqpoint{1.018557in}{1.236493in}}%
\pgfpathclose%
\pgfusepath{stroke,fill}%
\end{pgfscope}%
\begin{pgfscope}%
\pgfpathrectangle{\pgfqpoint{0.211875in}{0.211875in}}{\pgfqpoint{1.313625in}{1.279725in}}%
\pgfusepath{clip}%
\pgfsetbuttcap%
\pgfsetroundjoin%
\definecolor{currentfill}{rgb}{0.121569,0.466667,0.705882}%
\pgfsetfillcolor{currentfill}%
\pgfsetlinewidth{1.003750pt}%
\definecolor{currentstroke}{rgb}{0.121569,0.466667,0.705882}%
\pgfsetstrokecolor{currentstroke}%
\pgfsetdash{}{0pt}%
\pgfpathmoveto{\pgfqpoint{0.908735in}{0.942694in}}%
\pgfpathcurveto{\pgfqpoint{0.914559in}{0.942694in}}{\pgfqpoint{0.920145in}{0.945008in}}{\pgfqpoint{0.924263in}{0.949127in}}%
\pgfpathcurveto{\pgfqpoint{0.928381in}{0.953245in}}{\pgfqpoint{0.930695in}{0.958831in}}{\pgfqpoint{0.930695in}{0.964655in}}%
\pgfpathcurveto{\pgfqpoint{0.930695in}{0.970479in}}{\pgfqpoint{0.928381in}{0.976065in}}{\pgfqpoint{0.924263in}{0.980183in}}%
\pgfpathcurveto{\pgfqpoint{0.920145in}{0.984301in}}{\pgfqpoint{0.914559in}{0.986615in}}{\pgfqpoint{0.908735in}{0.986615in}}%
\pgfpathcurveto{\pgfqpoint{0.902911in}{0.986615in}}{\pgfqpoint{0.897325in}{0.984301in}}{\pgfqpoint{0.893206in}{0.980183in}}%
\pgfpathcurveto{\pgfqpoint{0.889088in}{0.976065in}}{\pgfqpoint{0.886774in}{0.970479in}}{\pgfqpoint{0.886774in}{0.964655in}}%
\pgfpathcurveto{\pgfqpoint{0.886774in}{0.958831in}}{\pgfqpoint{0.889088in}{0.953245in}}{\pgfqpoint{0.893206in}{0.949127in}}%
\pgfpathcurveto{\pgfqpoint{0.897325in}{0.945008in}}{\pgfqpoint{0.902911in}{0.942694in}}{\pgfqpoint{0.908735in}{0.942694in}}%
\pgfpathclose%
\pgfusepath{stroke,fill}%
\end{pgfscope}%
\begin{pgfscope}%
\pgfpathrectangle{\pgfqpoint{0.211875in}{0.211875in}}{\pgfqpoint{1.313625in}{1.279725in}}%
\pgfusepath{clip}%
\pgfsetbuttcap%
\pgfsetroundjoin%
\definecolor{currentfill}{rgb}{0.121569,0.466667,0.705882}%
\pgfsetfillcolor{currentfill}%
\pgfsetlinewidth{1.003750pt}%
\definecolor{currentstroke}{rgb}{0.121569,0.466667,0.705882}%
\pgfsetstrokecolor{currentstroke}%
\pgfsetdash{}{0pt}%
\pgfpathmoveto{\pgfqpoint{0.529853in}{1.343072in}}%
\pgfpathcurveto{\pgfqpoint{0.535677in}{1.343072in}}{\pgfqpoint{0.541263in}{1.345386in}}{\pgfqpoint{0.545382in}{1.349504in}}%
\pgfpathcurveto{\pgfqpoint{0.549500in}{1.353622in}}{\pgfqpoint{0.551814in}{1.359208in}}{\pgfqpoint{0.551814in}{1.365032in}}%
\pgfpathcurveto{\pgfqpoint{0.551814in}{1.370856in}}{\pgfqpoint{0.549500in}{1.376442in}}{\pgfqpoint{0.545382in}{1.380560in}}%
\pgfpathcurveto{\pgfqpoint{0.541263in}{1.384678in}}{\pgfqpoint{0.535677in}{1.386992in}}{\pgfqpoint{0.529853in}{1.386992in}}%
\pgfpathcurveto{\pgfqpoint{0.524029in}{1.386992in}}{\pgfqpoint{0.518443in}{1.384678in}}{\pgfqpoint{0.514325in}{1.380560in}}%
\pgfpathcurveto{\pgfqpoint{0.510207in}{1.376442in}}{\pgfqpoint{0.507893in}{1.370856in}}{\pgfqpoint{0.507893in}{1.365032in}}%
\pgfpathcurveto{\pgfqpoint{0.507893in}{1.359208in}}{\pgfqpoint{0.510207in}{1.353622in}}{\pgfqpoint{0.514325in}{1.349504in}}%
\pgfpathcurveto{\pgfqpoint{0.518443in}{1.345386in}}{\pgfqpoint{0.524029in}{1.343072in}}{\pgfqpoint{0.529853in}{1.343072in}}%
\pgfpathclose%
\pgfusepath{stroke,fill}%
\end{pgfscope}%
\begin{pgfscope}%
\pgfpathrectangle{\pgfqpoint{0.211875in}{0.211875in}}{\pgfqpoint{1.313625in}{1.279725in}}%
\pgfusepath{clip}%
\pgfsetbuttcap%
\pgfsetroundjoin%
\definecolor{currentfill}{rgb}{0.121569,0.466667,0.705882}%
\pgfsetfillcolor{currentfill}%
\pgfsetlinewidth{1.003750pt}%
\definecolor{currentstroke}{rgb}{0.121569,0.466667,0.705882}%
\pgfsetstrokecolor{currentstroke}%
\pgfsetdash{}{0pt}%
\pgfpathmoveto{\pgfqpoint{1.040131in}{1.213000in}}%
\pgfpathcurveto{\pgfqpoint{1.045955in}{1.213000in}}{\pgfqpoint{1.051541in}{1.215314in}}{\pgfqpoint{1.055659in}{1.219432in}}%
\pgfpathcurveto{\pgfqpoint{1.059777in}{1.223550in}}{\pgfqpoint{1.062091in}{1.229136in}}{\pgfqpoint{1.062091in}{1.234960in}}%
\pgfpathcurveto{\pgfqpoint{1.062091in}{1.240784in}}{\pgfqpoint{1.059777in}{1.246370in}}{\pgfqpoint{1.055659in}{1.250488in}}%
\pgfpathcurveto{\pgfqpoint{1.051541in}{1.254607in}}{\pgfqpoint{1.045955in}{1.256920in}}{\pgfqpoint{1.040131in}{1.256920in}}%
\pgfpathcurveto{\pgfqpoint{1.034307in}{1.256920in}}{\pgfqpoint{1.028721in}{1.254607in}}{\pgfqpoint{1.024603in}{1.250488in}}%
\pgfpathcurveto{\pgfqpoint{1.020485in}{1.246370in}}{\pgfqpoint{1.018171in}{1.240784in}}{\pgfqpoint{1.018171in}{1.234960in}}%
\pgfpathcurveto{\pgfqpoint{1.018171in}{1.229136in}}{\pgfqpoint{1.020485in}{1.223550in}}{\pgfqpoint{1.024603in}{1.219432in}}%
\pgfpathcurveto{\pgfqpoint{1.028721in}{1.215314in}}{\pgfqpoint{1.034307in}{1.213000in}}{\pgfqpoint{1.040131in}{1.213000in}}%
\pgfpathclose%
\pgfusepath{stroke,fill}%
\end{pgfscope}%
\begin{pgfscope}%
\pgfpathrectangle{\pgfqpoint{0.211875in}{0.211875in}}{\pgfqpoint{1.313625in}{1.279725in}}%
\pgfusepath{clip}%
\pgfsetbuttcap%
\pgfsetroundjoin%
\definecolor{currentfill}{rgb}{0.121569,0.466667,0.705882}%
\pgfsetfillcolor{currentfill}%
\pgfsetlinewidth{1.003750pt}%
\definecolor{currentstroke}{rgb}{0.121569,0.466667,0.705882}%
\pgfsetstrokecolor{currentstroke}%
\pgfsetdash{}{0pt}%
\pgfpathmoveto{\pgfqpoint{0.738267in}{1.299922in}}%
\pgfpathcurveto{\pgfqpoint{0.744091in}{1.299922in}}{\pgfqpoint{0.749677in}{1.302235in}}{\pgfqpoint{0.753795in}{1.306354in}}%
\pgfpathcurveto{\pgfqpoint{0.757913in}{1.310472in}}{\pgfqpoint{0.760227in}{1.316058in}}{\pgfqpoint{0.760227in}{1.321882in}}%
\pgfpathcurveto{\pgfqpoint{0.760227in}{1.327706in}}{\pgfqpoint{0.757913in}{1.333292in}}{\pgfqpoint{0.753795in}{1.337410in}}%
\pgfpathcurveto{\pgfqpoint{0.749677in}{1.341528in}}{\pgfqpoint{0.744091in}{1.343842in}}{\pgfqpoint{0.738267in}{1.343842in}}%
\pgfpathcurveto{\pgfqpoint{0.732443in}{1.343842in}}{\pgfqpoint{0.726857in}{1.341528in}}{\pgfqpoint{0.722739in}{1.337410in}}%
\pgfpathcurveto{\pgfqpoint{0.718621in}{1.333292in}}{\pgfqpoint{0.716307in}{1.327706in}}{\pgfqpoint{0.716307in}{1.321882in}}%
\pgfpathcurveto{\pgfqpoint{0.716307in}{1.316058in}}{\pgfqpoint{0.718621in}{1.310472in}}{\pgfqpoint{0.722739in}{1.306354in}}%
\pgfpathcurveto{\pgfqpoint{0.726857in}{1.302235in}}{\pgfqpoint{0.732443in}{1.299922in}}{\pgfqpoint{0.738267in}{1.299922in}}%
\pgfpathclose%
\pgfusepath{stroke,fill}%
\end{pgfscope}%
\begin{pgfscope}%
\pgfpathrectangle{\pgfqpoint{0.211875in}{0.211875in}}{\pgfqpoint{1.313625in}{1.279725in}}%
\pgfusepath{clip}%
\pgfsetbuttcap%
\pgfsetroundjoin%
\definecolor{currentfill}{rgb}{0.121569,0.466667,0.705882}%
\pgfsetfillcolor{currentfill}%
\pgfsetlinewidth{1.003750pt}%
\definecolor{currentstroke}{rgb}{0.121569,0.466667,0.705882}%
\pgfsetstrokecolor{currentstroke}%
\pgfsetdash{}{0pt}%
\pgfpathmoveto{\pgfqpoint{1.123956in}{1.051311in}}%
\pgfpathcurveto{\pgfqpoint{1.129779in}{1.051311in}}{\pgfqpoint{1.135366in}{1.053625in}}{\pgfqpoint{1.139484in}{1.057743in}}%
\pgfpathcurveto{\pgfqpoint{1.143602in}{1.061861in}}{\pgfqpoint{1.145916in}{1.067447in}}{\pgfqpoint{1.145916in}{1.073271in}}%
\pgfpathcurveto{\pgfqpoint{1.145916in}{1.079095in}}{\pgfqpoint{1.143602in}{1.084681in}}{\pgfqpoint{1.139484in}{1.088799in}}%
\pgfpathcurveto{\pgfqpoint{1.135366in}{1.092917in}}{\pgfqpoint{1.129779in}{1.095231in}}{\pgfqpoint{1.123956in}{1.095231in}}%
\pgfpathcurveto{\pgfqpoint{1.118132in}{1.095231in}}{\pgfqpoint{1.112545in}{1.092917in}}{\pgfqpoint{1.108427in}{1.088799in}}%
\pgfpathcurveto{\pgfqpoint{1.104309in}{1.084681in}}{\pgfqpoint{1.101995in}{1.079095in}}{\pgfqpoint{1.101995in}{1.073271in}}%
\pgfpathcurveto{\pgfqpoint{1.101995in}{1.067447in}}{\pgfqpoint{1.104309in}{1.061861in}}{\pgfqpoint{1.108427in}{1.057743in}}%
\pgfpathcurveto{\pgfqpoint{1.112545in}{1.053625in}}{\pgfqpoint{1.118132in}{1.051311in}}{\pgfqpoint{1.123956in}{1.051311in}}%
\pgfpathclose%
\pgfusepath{stroke,fill}%
\end{pgfscope}%
\begin{pgfscope}%
\pgfpathrectangle{\pgfqpoint{0.211875in}{0.211875in}}{\pgfqpoint{1.313625in}{1.279725in}}%
\pgfusepath{clip}%
\pgfsetbuttcap%
\pgfsetroundjoin%
\definecolor{currentfill}{rgb}{0.121569,0.466667,0.705882}%
\pgfsetfillcolor{currentfill}%
\pgfsetlinewidth{1.003750pt}%
\definecolor{currentstroke}{rgb}{0.121569,0.466667,0.705882}%
\pgfsetstrokecolor{currentstroke}%
\pgfsetdash{}{0pt}%
\pgfpathmoveto{\pgfqpoint{0.900021in}{0.918686in}}%
\pgfpathcurveto{\pgfqpoint{0.905845in}{0.918686in}}{\pgfqpoint{0.911431in}{0.921000in}}{\pgfqpoint{0.915549in}{0.925118in}}%
\pgfpathcurveto{\pgfqpoint{0.919668in}{0.929236in}}{\pgfqpoint{0.921981in}{0.934822in}}{\pgfqpoint{0.921981in}{0.940646in}}%
\pgfpathcurveto{\pgfqpoint{0.921981in}{0.946470in}}{\pgfqpoint{0.919668in}{0.952056in}}{\pgfqpoint{0.915549in}{0.956175in}}%
\pgfpathcurveto{\pgfqpoint{0.911431in}{0.960293in}}{\pgfqpoint{0.905845in}{0.962607in}}{\pgfqpoint{0.900021in}{0.962607in}}%
\pgfpathcurveto{\pgfqpoint{0.894197in}{0.962607in}}{\pgfqpoint{0.888611in}{0.960293in}}{\pgfqpoint{0.884493in}{0.956175in}}%
\pgfpathcurveto{\pgfqpoint{0.880375in}{0.952056in}}{\pgfqpoint{0.878061in}{0.946470in}}{\pgfqpoint{0.878061in}{0.940646in}}%
\pgfpathcurveto{\pgfqpoint{0.878061in}{0.934822in}}{\pgfqpoint{0.880375in}{0.929236in}}{\pgfqpoint{0.884493in}{0.925118in}}%
\pgfpathcurveto{\pgfqpoint{0.888611in}{0.921000in}}{\pgfqpoint{0.894197in}{0.918686in}}{\pgfqpoint{0.900021in}{0.918686in}}%
\pgfpathclose%
\pgfusepath{stroke,fill}%
\end{pgfscope}%
\begin{pgfscope}%
\pgfpathrectangle{\pgfqpoint{0.211875in}{0.211875in}}{\pgfqpoint{1.313625in}{1.279725in}}%
\pgfusepath{clip}%
\pgfsetbuttcap%
\pgfsetroundjoin%
\definecolor{currentfill}{rgb}{0.121569,0.466667,0.705882}%
\pgfsetfillcolor{currentfill}%
\pgfsetlinewidth{1.003750pt}%
\definecolor{currentstroke}{rgb}{0.121569,0.466667,0.705882}%
\pgfsetstrokecolor{currentstroke}%
\pgfsetdash{}{0pt}%
\pgfpathmoveto{\pgfqpoint{1.126668in}{1.073289in}}%
\pgfpathcurveto{\pgfqpoint{1.132492in}{1.073289in}}{\pgfqpoint{1.138078in}{1.075603in}}{\pgfqpoint{1.142196in}{1.079721in}}%
\pgfpathcurveto{\pgfqpoint{1.146314in}{1.083839in}}{\pgfqpoint{1.148628in}{1.089426in}}{\pgfqpoint{1.148628in}{1.095250in}}%
\pgfpathcurveto{\pgfqpoint{1.148628in}{1.101074in}}{\pgfqpoint{1.146314in}{1.106660in}}{\pgfqpoint{1.142196in}{1.110778in}}%
\pgfpathcurveto{\pgfqpoint{1.138078in}{1.114896in}}{\pgfqpoint{1.132492in}{1.117210in}}{\pgfqpoint{1.126668in}{1.117210in}}%
\pgfpathcurveto{\pgfqpoint{1.120844in}{1.117210in}}{\pgfqpoint{1.115258in}{1.114896in}}{\pgfqpoint{1.111139in}{1.110778in}}%
\pgfpathcurveto{\pgfqpoint{1.107021in}{1.106660in}}{\pgfqpoint{1.104707in}{1.101074in}}{\pgfqpoint{1.104707in}{1.095250in}}%
\pgfpathcurveto{\pgfqpoint{1.104707in}{1.089426in}}{\pgfqpoint{1.107021in}{1.083839in}}{\pgfqpoint{1.111139in}{1.079721in}}%
\pgfpathcurveto{\pgfqpoint{1.115258in}{1.075603in}}{\pgfqpoint{1.120844in}{1.073289in}}{\pgfqpoint{1.126668in}{1.073289in}}%
\pgfpathclose%
\pgfusepath{stroke,fill}%
\end{pgfscope}%
\begin{pgfscope}%
\pgfpathrectangle{\pgfqpoint{0.211875in}{0.211875in}}{\pgfqpoint{1.313625in}{1.279725in}}%
\pgfusepath{clip}%
\pgfsetbuttcap%
\pgfsetroundjoin%
\definecolor{currentfill}{rgb}{0.121569,0.466667,0.705882}%
\pgfsetfillcolor{currentfill}%
\pgfsetlinewidth{1.003750pt}%
\definecolor{currentstroke}{rgb}{0.121569,0.466667,0.705882}%
\pgfsetstrokecolor{currentstroke}%
\pgfsetdash{}{0pt}%
\pgfpathmoveto{\pgfqpoint{0.908565in}{0.830575in}}%
\pgfpathcurveto{\pgfqpoint{0.914389in}{0.830575in}}{\pgfqpoint{0.919975in}{0.832889in}}{\pgfqpoint{0.924093in}{0.837007in}}%
\pgfpathcurveto{\pgfqpoint{0.928211in}{0.841126in}}{\pgfqpoint{0.930525in}{0.846712in}}{\pgfqpoint{0.930525in}{0.852536in}}%
\pgfpathcurveto{\pgfqpoint{0.930525in}{0.858360in}}{\pgfqpoint{0.928211in}{0.863946in}}{\pgfqpoint{0.924093in}{0.868064in}}%
\pgfpathcurveto{\pgfqpoint{0.919975in}{0.872182in}}{\pgfqpoint{0.914389in}{0.874496in}}{\pgfqpoint{0.908565in}{0.874496in}}%
\pgfpathcurveto{\pgfqpoint{0.902741in}{0.874496in}}{\pgfqpoint{0.897155in}{0.872182in}}{\pgfqpoint{0.893037in}{0.868064in}}%
\pgfpathcurveto{\pgfqpoint{0.888919in}{0.863946in}}{\pgfqpoint{0.886605in}{0.858360in}}{\pgfqpoint{0.886605in}{0.852536in}}%
\pgfpathcurveto{\pgfqpoint{0.886605in}{0.846712in}}{\pgfqpoint{0.888919in}{0.841126in}}{\pgfqpoint{0.893037in}{0.837007in}}%
\pgfpathcurveto{\pgfqpoint{0.897155in}{0.832889in}}{\pgfqpoint{0.902741in}{0.830575in}}{\pgfqpoint{0.908565in}{0.830575in}}%
\pgfpathclose%
\pgfusepath{stroke,fill}%
\end{pgfscope}%
\begin{pgfscope}%
\pgfpathrectangle{\pgfqpoint{0.211875in}{0.211875in}}{\pgfqpoint{1.313625in}{1.279725in}}%
\pgfusepath{clip}%
\pgfsetbuttcap%
\pgfsetroundjoin%
\definecolor{currentfill}{rgb}{0.121569,0.466667,0.705882}%
\pgfsetfillcolor{currentfill}%
\pgfsetlinewidth{1.003750pt}%
\definecolor{currentstroke}{rgb}{0.121569,0.466667,0.705882}%
\pgfsetstrokecolor{currentstroke}%
\pgfsetdash{}{0pt}%
\pgfpathmoveto{\pgfqpoint{1.444218in}{1.073112in}}%
\pgfpathcurveto{\pgfqpoint{1.450042in}{1.073112in}}{\pgfqpoint{1.455629in}{1.075426in}}{\pgfqpoint{1.459747in}{1.079544in}}%
\pgfpathcurveto{\pgfqpoint{1.463865in}{1.083662in}}{\pgfqpoint{1.466179in}{1.089248in}}{\pgfqpoint{1.466179in}{1.095072in}}%
\pgfpathcurveto{\pgfqpoint{1.466179in}{1.100896in}}{\pgfqpoint{1.463865in}{1.106482in}}{\pgfqpoint{1.459747in}{1.110601in}}%
\pgfpathcurveto{\pgfqpoint{1.455629in}{1.114719in}}{\pgfqpoint{1.450042in}{1.117033in}}{\pgfqpoint{1.444218in}{1.117033in}}%
\pgfpathcurveto{\pgfqpoint{1.438395in}{1.117033in}}{\pgfqpoint{1.432808in}{1.114719in}}{\pgfqpoint{1.428690in}{1.110601in}}%
\pgfpathcurveto{\pgfqpoint{1.424572in}{1.106482in}}{\pgfqpoint{1.422258in}{1.100896in}}{\pgfqpoint{1.422258in}{1.095072in}}%
\pgfpathcurveto{\pgfqpoint{1.422258in}{1.089248in}}{\pgfqpoint{1.424572in}{1.083662in}}{\pgfqpoint{1.428690in}{1.079544in}}%
\pgfpathcurveto{\pgfqpoint{1.432808in}{1.075426in}}{\pgfqpoint{1.438395in}{1.073112in}}{\pgfqpoint{1.444218in}{1.073112in}}%
\pgfpathclose%
\pgfusepath{stroke,fill}%
\end{pgfscope}%
\begin{pgfscope}%
\pgfpathrectangle{\pgfqpoint{0.211875in}{0.211875in}}{\pgfqpoint{1.313625in}{1.279725in}}%
\pgfusepath{clip}%
\pgfsetbuttcap%
\pgfsetroundjoin%
\definecolor{currentfill}{rgb}{0.121569,0.466667,0.705882}%
\pgfsetfillcolor{currentfill}%
\pgfsetlinewidth{1.003750pt}%
\definecolor{currentstroke}{rgb}{0.121569,0.466667,0.705882}%
\pgfsetstrokecolor{currentstroke}%
\pgfsetdash{}{0pt}%
\pgfpathmoveto{\pgfqpoint{1.118509in}{1.060522in}}%
\pgfpathcurveto{\pgfqpoint{1.124333in}{1.060522in}}{\pgfqpoint{1.129919in}{1.062836in}}{\pgfqpoint{1.134037in}{1.066954in}}%
\pgfpathcurveto{\pgfqpoint{1.138155in}{1.071072in}}{\pgfqpoint{1.140469in}{1.076659in}}{\pgfqpoint{1.140469in}{1.082482in}}%
\pgfpathcurveto{\pgfqpoint{1.140469in}{1.088306in}}{\pgfqpoint{1.138155in}{1.093893in}}{\pgfqpoint{1.134037in}{1.098011in}}%
\pgfpathcurveto{\pgfqpoint{1.129919in}{1.102129in}}{\pgfqpoint{1.124333in}{1.104443in}}{\pgfqpoint{1.118509in}{1.104443in}}%
\pgfpathcurveto{\pgfqpoint{1.112685in}{1.104443in}}{\pgfqpoint{1.107099in}{1.102129in}}{\pgfqpoint{1.102981in}{1.098011in}}%
\pgfpathcurveto{\pgfqpoint{1.098862in}{1.093893in}}{\pgfqpoint{1.096549in}{1.088306in}}{\pgfqpoint{1.096549in}{1.082482in}}%
\pgfpathcurveto{\pgfqpoint{1.096549in}{1.076659in}}{\pgfqpoint{1.098862in}{1.071072in}}{\pgfqpoint{1.102981in}{1.066954in}}%
\pgfpathcurveto{\pgfqpoint{1.107099in}{1.062836in}}{\pgfqpoint{1.112685in}{1.060522in}}{\pgfqpoint{1.118509in}{1.060522in}}%
\pgfpathclose%
\pgfusepath{stroke,fill}%
\end{pgfscope}%
\begin{pgfscope}%
\pgfpathrectangle{\pgfqpoint{0.211875in}{0.211875in}}{\pgfqpoint{1.313625in}{1.279725in}}%
\pgfusepath{clip}%
\pgfsetbuttcap%
\pgfsetroundjoin%
\definecolor{currentfill}{rgb}{0.121569,0.466667,0.705882}%
\pgfsetfillcolor{currentfill}%
\pgfsetlinewidth{1.003750pt}%
\definecolor{currentstroke}{rgb}{0.121569,0.466667,0.705882}%
\pgfsetstrokecolor{currentstroke}%
\pgfsetdash{}{0pt}%
\pgfpathmoveto{\pgfqpoint{0.913893in}{0.922070in}}%
\pgfpathcurveto{\pgfqpoint{0.919717in}{0.922070in}}{\pgfqpoint{0.925303in}{0.924384in}}{\pgfqpoint{0.929421in}{0.928502in}}%
\pgfpathcurveto{\pgfqpoint{0.933539in}{0.932620in}}{\pgfqpoint{0.935853in}{0.938206in}}{\pgfqpoint{0.935853in}{0.944030in}}%
\pgfpathcurveto{\pgfqpoint{0.935853in}{0.949854in}}{\pgfqpoint{0.933539in}{0.955440in}}{\pgfqpoint{0.929421in}{0.959559in}}%
\pgfpathcurveto{\pgfqpoint{0.925303in}{0.963677in}}{\pgfqpoint{0.919717in}{0.965991in}}{\pgfqpoint{0.913893in}{0.965991in}}%
\pgfpathcurveto{\pgfqpoint{0.908069in}{0.965991in}}{\pgfqpoint{0.902483in}{0.963677in}}{\pgfqpoint{0.898365in}{0.959559in}}%
\pgfpathcurveto{\pgfqpoint{0.894247in}{0.955440in}}{\pgfqpoint{0.891933in}{0.949854in}}{\pgfqpoint{0.891933in}{0.944030in}}%
\pgfpathcurveto{\pgfqpoint{0.891933in}{0.938206in}}{\pgfqpoint{0.894247in}{0.932620in}}{\pgfqpoint{0.898365in}{0.928502in}}%
\pgfpathcurveto{\pgfqpoint{0.902483in}{0.924384in}}{\pgfqpoint{0.908069in}{0.922070in}}{\pgfqpoint{0.913893in}{0.922070in}}%
\pgfpathclose%
\pgfusepath{stroke,fill}%
\end{pgfscope}%
\begin{pgfscope}%
\pgfpathrectangle{\pgfqpoint{0.211875in}{0.211875in}}{\pgfqpoint{1.313625in}{1.279725in}}%
\pgfusepath{clip}%
\pgfsetbuttcap%
\pgfsetroundjoin%
\definecolor{currentfill}{rgb}{0.121569,0.466667,0.705882}%
\pgfsetfillcolor{currentfill}%
\pgfsetlinewidth{1.003750pt}%
\definecolor{currentstroke}{rgb}{0.121569,0.466667,0.705882}%
\pgfsetstrokecolor{currentstroke}%
\pgfsetdash{}{0pt}%
\pgfpathmoveto{\pgfqpoint{0.832147in}{1.390690in}}%
\pgfpathcurveto{\pgfqpoint{0.837971in}{1.390690in}}{\pgfqpoint{0.843557in}{1.393004in}}{\pgfqpoint{0.847675in}{1.397122in}}%
\pgfpathcurveto{\pgfqpoint{0.851794in}{1.401240in}}{\pgfqpoint{0.854107in}{1.406827in}}{\pgfqpoint{0.854107in}{1.412651in}}%
\pgfpathcurveto{\pgfqpoint{0.854107in}{1.418474in}}{\pgfqpoint{0.851794in}{1.424061in}}{\pgfqpoint{0.847675in}{1.428179in}}%
\pgfpathcurveto{\pgfqpoint{0.843557in}{1.432297in}}{\pgfqpoint{0.837971in}{1.434611in}}{\pgfqpoint{0.832147in}{1.434611in}}%
\pgfpathcurveto{\pgfqpoint{0.826323in}{1.434611in}}{\pgfqpoint{0.820737in}{1.432297in}}{\pgfqpoint{0.816619in}{1.428179in}}%
\pgfpathcurveto{\pgfqpoint{0.812501in}{1.424061in}}{\pgfqpoint{0.810187in}{1.418474in}}{\pgfqpoint{0.810187in}{1.412651in}}%
\pgfpathcurveto{\pgfqpoint{0.810187in}{1.406827in}}{\pgfqpoint{0.812501in}{1.401240in}}{\pgfqpoint{0.816619in}{1.397122in}}%
\pgfpathcurveto{\pgfqpoint{0.820737in}{1.393004in}}{\pgfqpoint{0.826323in}{1.390690in}}{\pgfqpoint{0.832147in}{1.390690in}}%
\pgfpathclose%
\pgfusepath{stroke,fill}%
\end{pgfscope}%
\begin{pgfscope}%
\pgfpathrectangle{\pgfqpoint{0.211875in}{0.211875in}}{\pgfqpoint{1.313625in}{1.279725in}}%
\pgfusepath{clip}%
\pgfsetbuttcap%
\pgfsetroundjoin%
\definecolor{currentfill}{rgb}{0.121569,0.466667,0.705882}%
\pgfsetfillcolor{currentfill}%
\pgfsetlinewidth{1.003750pt}%
\definecolor{currentstroke}{rgb}{0.121569,0.466667,0.705882}%
\pgfsetstrokecolor{currentstroke}%
\pgfsetdash{}{0pt}%
\pgfpathmoveto{\pgfqpoint{1.366687in}{0.990944in}}%
\pgfpathcurveto{\pgfqpoint{1.372511in}{0.990944in}}{\pgfqpoint{1.378097in}{0.993258in}}{\pgfqpoint{1.382215in}{0.997376in}}%
\pgfpathcurveto{\pgfqpoint{1.386334in}{1.001495in}}{\pgfqpoint{1.388648in}{1.007081in}}{\pgfqpoint{1.388648in}{1.012905in}}%
\pgfpathcurveto{\pgfqpoint{1.388648in}{1.018729in}}{\pgfqpoint{1.386334in}{1.024315in}}{\pgfqpoint{1.382215in}{1.028433in}}%
\pgfpathcurveto{\pgfqpoint{1.378097in}{1.032551in}}{\pgfqpoint{1.372511in}{1.034865in}}{\pgfqpoint{1.366687in}{1.034865in}}%
\pgfpathcurveto{\pgfqpoint{1.360863in}{1.034865in}}{\pgfqpoint{1.355277in}{1.032551in}}{\pgfqpoint{1.351159in}{1.028433in}}%
\pgfpathcurveto{\pgfqpoint{1.347041in}{1.024315in}}{\pgfqpoint{1.344727in}{1.018729in}}{\pgfqpoint{1.344727in}{1.012905in}}%
\pgfpathcurveto{\pgfqpoint{1.344727in}{1.007081in}}{\pgfqpoint{1.347041in}{1.001495in}}{\pgfqpoint{1.351159in}{0.997376in}}%
\pgfpathcurveto{\pgfqpoint{1.355277in}{0.993258in}}{\pgfqpoint{1.360863in}{0.990944in}}{\pgfqpoint{1.366687in}{0.990944in}}%
\pgfpathclose%
\pgfusepath{stroke,fill}%
\end{pgfscope}%
\begin{pgfscope}%
\pgfpathrectangle{\pgfqpoint{0.211875in}{0.211875in}}{\pgfqpoint{1.313625in}{1.279725in}}%
\pgfusepath{clip}%
\pgfsetbuttcap%
\pgfsetroundjoin%
\definecolor{currentfill}{rgb}{0.121569,0.466667,0.705882}%
\pgfsetfillcolor{currentfill}%
\pgfsetlinewidth{1.003750pt}%
\definecolor{currentstroke}{rgb}{0.121569,0.466667,0.705882}%
\pgfsetstrokecolor{currentstroke}%
\pgfsetdash{}{0pt}%
\pgfpathmoveto{\pgfqpoint{0.942963in}{1.255561in}}%
\pgfpathcurveto{\pgfqpoint{0.948787in}{1.255561in}}{\pgfqpoint{0.954373in}{1.257875in}}{\pgfqpoint{0.958491in}{1.261993in}}%
\pgfpathcurveto{\pgfqpoint{0.962609in}{1.266111in}}{\pgfqpoint{0.964923in}{1.271697in}}{\pgfqpoint{0.964923in}{1.277521in}}%
\pgfpathcurveto{\pgfqpoint{0.964923in}{1.283345in}}{\pgfqpoint{0.962609in}{1.288931in}}{\pgfqpoint{0.958491in}{1.293050in}}%
\pgfpathcurveto{\pgfqpoint{0.954373in}{1.297168in}}{\pgfqpoint{0.948787in}{1.299482in}}{\pgfqpoint{0.942963in}{1.299482in}}%
\pgfpathcurveto{\pgfqpoint{0.937139in}{1.299482in}}{\pgfqpoint{0.931553in}{1.297168in}}{\pgfqpoint{0.927435in}{1.293050in}}%
\pgfpathcurveto{\pgfqpoint{0.923317in}{1.288931in}}{\pgfqpoint{0.921003in}{1.283345in}}{\pgfqpoint{0.921003in}{1.277521in}}%
\pgfpathcurveto{\pgfqpoint{0.921003in}{1.271697in}}{\pgfqpoint{0.923317in}{1.266111in}}{\pgfqpoint{0.927435in}{1.261993in}}%
\pgfpathcurveto{\pgfqpoint{0.931553in}{1.257875in}}{\pgfqpoint{0.937139in}{1.255561in}}{\pgfqpoint{0.942963in}{1.255561in}}%
\pgfpathclose%
\pgfusepath{stroke,fill}%
\end{pgfscope}%
\begin{pgfscope}%
\pgfpathrectangle{\pgfqpoint{0.211875in}{0.211875in}}{\pgfqpoint{1.313625in}{1.279725in}}%
\pgfusepath{clip}%
\pgfsetbuttcap%
\pgfsetroundjoin%
\definecolor{currentfill}{rgb}{0.121569,0.466667,0.705882}%
\pgfsetfillcolor{currentfill}%
\pgfsetlinewidth{1.003750pt}%
\definecolor{currentstroke}{rgb}{0.121569,0.466667,0.705882}%
\pgfsetstrokecolor{currentstroke}%
\pgfsetdash{}{0pt}%
\pgfpathmoveto{\pgfqpoint{1.125742in}{1.055341in}}%
\pgfpathcurveto{\pgfqpoint{1.131566in}{1.055341in}}{\pgfqpoint{1.137152in}{1.057654in}}{\pgfqpoint{1.141270in}{1.061773in}}%
\pgfpathcurveto{\pgfqpoint{1.145388in}{1.065891in}}{\pgfqpoint{1.147702in}{1.071477in}}{\pgfqpoint{1.147702in}{1.077301in}}%
\pgfpathcurveto{\pgfqpoint{1.147702in}{1.083125in}}{\pgfqpoint{1.145388in}{1.088711in}}{\pgfqpoint{1.141270in}{1.092829in}}%
\pgfpathcurveto{\pgfqpoint{1.137152in}{1.096947in}}{\pgfqpoint{1.131566in}{1.099261in}}{\pgfqpoint{1.125742in}{1.099261in}}%
\pgfpathcurveto{\pgfqpoint{1.119918in}{1.099261in}}{\pgfqpoint{1.114332in}{1.096947in}}{\pgfqpoint{1.110213in}{1.092829in}}%
\pgfpathcurveto{\pgfqpoint{1.106095in}{1.088711in}}{\pgfqpoint{1.103781in}{1.083125in}}{\pgfqpoint{1.103781in}{1.077301in}}%
\pgfpathcurveto{\pgfqpoint{1.103781in}{1.071477in}}{\pgfqpoint{1.106095in}{1.065891in}}{\pgfqpoint{1.110213in}{1.061773in}}%
\pgfpathcurveto{\pgfqpoint{1.114332in}{1.057654in}}{\pgfqpoint{1.119918in}{1.055341in}}{\pgfqpoint{1.125742in}{1.055341in}}%
\pgfpathclose%
\pgfusepath{stroke,fill}%
\end{pgfscope}%
\begin{pgfscope}%
\pgfpathrectangle{\pgfqpoint{0.211875in}{0.211875in}}{\pgfqpoint{1.313625in}{1.279725in}}%
\pgfusepath{clip}%
\pgfsetbuttcap%
\pgfsetroundjoin%
\definecolor{currentfill}{rgb}{0.121569,0.466667,0.705882}%
\pgfsetfillcolor{currentfill}%
\pgfsetlinewidth{1.003750pt}%
\definecolor{currentstroke}{rgb}{0.121569,0.466667,0.705882}%
\pgfsetstrokecolor{currentstroke}%
\pgfsetdash{}{0pt}%
\pgfpathmoveto{\pgfqpoint{1.111456in}{1.069760in}}%
\pgfpathcurveto{\pgfqpoint{1.117280in}{1.069760in}}{\pgfqpoint{1.122866in}{1.072074in}}{\pgfqpoint{1.126984in}{1.076192in}}%
\pgfpathcurveto{\pgfqpoint{1.131102in}{1.080310in}}{\pgfqpoint{1.133416in}{1.085896in}}{\pgfqpoint{1.133416in}{1.091720in}}%
\pgfpathcurveto{\pgfqpoint{1.133416in}{1.097544in}}{\pgfqpoint{1.131102in}{1.103130in}}{\pgfqpoint{1.126984in}{1.107248in}}%
\pgfpathcurveto{\pgfqpoint{1.122866in}{1.111367in}}{\pgfqpoint{1.117280in}{1.113680in}}{\pgfqpoint{1.111456in}{1.113680in}}%
\pgfpathcurveto{\pgfqpoint{1.105632in}{1.113680in}}{\pgfqpoint{1.100046in}{1.111367in}}{\pgfqpoint{1.095928in}{1.107248in}}%
\pgfpathcurveto{\pgfqpoint{1.091809in}{1.103130in}}{\pgfqpoint{1.089496in}{1.097544in}}{\pgfqpoint{1.089496in}{1.091720in}}%
\pgfpathcurveto{\pgfqpoint{1.089496in}{1.085896in}}{\pgfqpoint{1.091809in}{1.080310in}}{\pgfqpoint{1.095928in}{1.076192in}}%
\pgfpathcurveto{\pgfqpoint{1.100046in}{1.072074in}}{\pgfqpoint{1.105632in}{1.069760in}}{\pgfqpoint{1.111456in}{1.069760in}}%
\pgfpathclose%
\pgfusepath{stroke,fill}%
\end{pgfscope}%
\begin{pgfscope}%
\pgfpathrectangle{\pgfqpoint{0.211875in}{0.211875in}}{\pgfqpoint{1.313625in}{1.279725in}}%
\pgfusepath{clip}%
\pgfsetbuttcap%
\pgfsetroundjoin%
\definecolor{currentfill}{rgb}{0.121569,0.466667,0.705882}%
\pgfsetfillcolor{currentfill}%
\pgfsetlinewidth{1.003750pt}%
\definecolor{currentstroke}{rgb}{0.121569,0.466667,0.705882}%
\pgfsetstrokecolor{currentstroke}%
\pgfsetdash{}{0pt}%
\pgfpathmoveto{\pgfqpoint{1.328286in}{0.938022in}}%
\pgfpathcurveto{\pgfqpoint{1.334110in}{0.938022in}}{\pgfqpoint{1.339697in}{0.940336in}}{\pgfqpoint{1.343815in}{0.944454in}}%
\pgfpathcurveto{\pgfqpoint{1.347933in}{0.948572in}}{\pgfqpoint{1.350247in}{0.954158in}}{\pgfqpoint{1.350247in}{0.959982in}}%
\pgfpathcurveto{\pgfqpoint{1.350247in}{0.965806in}}{\pgfqpoint{1.347933in}{0.971392in}}{\pgfqpoint{1.343815in}{0.975510in}}%
\pgfpathcurveto{\pgfqpoint{1.339697in}{0.979629in}}{\pgfqpoint{1.334110in}{0.981942in}}{\pgfqpoint{1.328286in}{0.981942in}}%
\pgfpathcurveto{\pgfqpoint{1.322463in}{0.981942in}}{\pgfqpoint{1.316876in}{0.979629in}}{\pgfqpoint{1.312758in}{0.975510in}}%
\pgfpathcurveto{\pgfqpoint{1.308640in}{0.971392in}}{\pgfqpoint{1.306326in}{0.965806in}}{\pgfqpoint{1.306326in}{0.959982in}}%
\pgfpathcurveto{\pgfqpoint{1.306326in}{0.954158in}}{\pgfqpoint{1.308640in}{0.948572in}}{\pgfqpoint{1.312758in}{0.944454in}}%
\pgfpathcurveto{\pgfqpoint{1.316876in}{0.940336in}}{\pgfqpoint{1.322463in}{0.938022in}}{\pgfqpoint{1.328286in}{0.938022in}}%
\pgfpathclose%
\pgfusepath{stroke,fill}%
\end{pgfscope}%
\begin{pgfscope}%
\pgfpathrectangle{\pgfqpoint{0.211875in}{0.211875in}}{\pgfqpoint{1.313625in}{1.279725in}}%
\pgfusepath{clip}%
\pgfsetbuttcap%
\pgfsetroundjoin%
\definecolor{currentfill}{rgb}{0.121569,0.466667,0.705882}%
\pgfsetfillcolor{currentfill}%
\pgfsetlinewidth{1.003750pt}%
\definecolor{currentstroke}{rgb}{0.121569,0.466667,0.705882}%
\pgfsetstrokecolor{currentstroke}%
\pgfsetdash{}{0pt}%
\pgfpathmoveto{\pgfqpoint{1.375524in}{1.012512in}}%
\pgfpathcurveto{\pgfqpoint{1.381348in}{1.012512in}}{\pgfqpoint{1.386934in}{1.014826in}}{\pgfqpoint{1.391052in}{1.018944in}}%
\pgfpathcurveto{\pgfqpoint{1.395170in}{1.023063in}}{\pgfqpoint{1.397484in}{1.028649in}}{\pgfqpoint{1.397484in}{1.034473in}}%
\pgfpathcurveto{\pgfqpoint{1.397484in}{1.040297in}}{\pgfqpoint{1.395170in}{1.045883in}}{\pgfqpoint{1.391052in}{1.050001in}}%
\pgfpathcurveto{\pgfqpoint{1.386934in}{1.054119in}}{\pgfqpoint{1.381348in}{1.056433in}}{\pgfqpoint{1.375524in}{1.056433in}}%
\pgfpathcurveto{\pgfqpoint{1.369700in}{1.056433in}}{\pgfqpoint{1.364114in}{1.054119in}}{\pgfqpoint{1.359996in}{1.050001in}}%
\pgfpathcurveto{\pgfqpoint{1.355878in}{1.045883in}}{\pgfqpoint{1.353564in}{1.040297in}}{\pgfqpoint{1.353564in}{1.034473in}}%
\pgfpathcurveto{\pgfqpoint{1.353564in}{1.028649in}}{\pgfqpoint{1.355878in}{1.023063in}}{\pgfqpoint{1.359996in}{1.018944in}}%
\pgfpathcurveto{\pgfqpoint{1.364114in}{1.014826in}}{\pgfqpoint{1.369700in}{1.012512in}}{\pgfqpoint{1.375524in}{1.012512in}}%
\pgfpathclose%
\pgfusepath{stroke,fill}%
\end{pgfscope}%
\begin{pgfscope}%
\pgfpathrectangle{\pgfqpoint{0.211875in}{0.211875in}}{\pgfqpoint{1.313625in}{1.279725in}}%
\pgfusepath{clip}%
\pgfsetbuttcap%
\pgfsetroundjoin%
\definecolor{currentfill}{rgb}{0.121569,0.466667,0.705882}%
\pgfsetfillcolor{currentfill}%
\pgfsetlinewidth{1.003750pt}%
\definecolor{currentstroke}{rgb}{0.121569,0.466667,0.705882}%
\pgfsetstrokecolor{currentstroke}%
\pgfsetdash{}{0pt}%
\pgfpathmoveto{\pgfqpoint{1.138431in}{1.061295in}}%
\pgfpathcurveto{\pgfqpoint{1.144255in}{1.061295in}}{\pgfqpoint{1.149841in}{1.063609in}}{\pgfqpoint{1.153959in}{1.067727in}}%
\pgfpathcurveto{\pgfqpoint{1.158077in}{1.071846in}}{\pgfqpoint{1.160391in}{1.077432in}}{\pgfqpoint{1.160391in}{1.083256in}}%
\pgfpathcurveto{\pgfqpoint{1.160391in}{1.089080in}}{\pgfqpoint{1.158077in}{1.094666in}}{\pgfqpoint{1.153959in}{1.098784in}}%
\pgfpathcurveto{\pgfqpoint{1.149841in}{1.102902in}}{\pgfqpoint{1.144255in}{1.105216in}}{\pgfqpoint{1.138431in}{1.105216in}}%
\pgfpathcurveto{\pgfqpoint{1.132607in}{1.105216in}}{\pgfqpoint{1.127021in}{1.102902in}}{\pgfqpoint{1.122903in}{1.098784in}}%
\pgfpathcurveto{\pgfqpoint{1.118784in}{1.094666in}}{\pgfqpoint{1.116471in}{1.089080in}}{\pgfqpoint{1.116471in}{1.083256in}}%
\pgfpathcurveto{\pgfqpoint{1.116471in}{1.077432in}}{\pgfqpoint{1.118784in}{1.071846in}}{\pgfqpoint{1.122903in}{1.067727in}}%
\pgfpathcurveto{\pgfqpoint{1.127021in}{1.063609in}}{\pgfqpoint{1.132607in}{1.061295in}}{\pgfqpoint{1.138431in}{1.061295in}}%
\pgfpathclose%
\pgfusepath{stroke,fill}%
\end{pgfscope}%
\begin{pgfscope}%
\pgfpathrectangle{\pgfqpoint{0.211875in}{0.211875in}}{\pgfqpoint{1.313625in}{1.279725in}}%
\pgfusepath{clip}%
\pgfsetbuttcap%
\pgfsetroundjoin%
\definecolor{currentfill}{rgb}{0.121569,0.466667,0.705882}%
\pgfsetfillcolor{currentfill}%
\pgfsetlinewidth{1.003750pt}%
\definecolor{currentstroke}{rgb}{0.121569,0.466667,0.705882}%
\pgfsetstrokecolor{currentstroke}%
\pgfsetdash{}{0pt}%
\pgfpathmoveto{\pgfqpoint{1.130235in}{1.059607in}}%
\pgfpathcurveto{\pgfqpoint{1.136059in}{1.059607in}}{\pgfqpoint{1.141645in}{1.061921in}}{\pgfqpoint{1.145763in}{1.066039in}}%
\pgfpathcurveto{\pgfqpoint{1.149881in}{1.070157in}}{\pgfqpoint{1.152195in}{1.075744in}}{\pgfqpoint{1.152195in}{1.081567in}}%
\pgfpathcurveto{\pgfqpoint{1.152195in}{1.087391in}}{\pgfqpoint{1.149881in}{1.092978in}}{\pgfqpoint{1.145763in}{1.097096in}}%
\pgfpathcurveto{\pgfqpoint{1.141645in}{1.101214in}}{\pgfqpoint{1.136059in}{1.103528in}}{\pgfqpoint{1.130235in}{1.103528in}}%
\pgfpathcurveto{\pgfqpoint{1.124411in}{1.103528in}}{\pgfqpoint{1.118825in}{1.101214in}}{\pgfqpoint{1.114707in}{1.097096in}}%
\pgfpathcurveto{\pgfqpoint{1.110589in}{1.092978in}}{\pgfqpoint{1.108275in}{1.087391in}}{\pgfqpoint{1.108275in}{1.081567in}}%
\pgfpathcurveto{\pgfqpoint{1.108275in}{1.075744in}}{\pgfqpoint{1.110589in}{1.070157in}}{\pgfqpoint{1.114707in}{1.066039in}}%
\pgfpathcurveto{\pgfqpoint{1.118825in}{1.061921in}}{\pgfqpoint{1.124411in}{1.059607in}}{\pgfqpoint{1.130235in}{1.059607in}}%
\pgfpathclose%
\pgfusepath{stroke,fill}%
\end{pgfscope}%
\begin{pgfscope}%
\pgfpathrectangle{\pgfqpoint{0.211875in}{0.211875in}}{\pgfqpoint{1.313625in}{1.279725in}}%
\pgfusepath{clip}%
\pgfsetbuttcap%
\pgfsetroundjoin%
\definecolor{currentfill}{rgb}{0.121569,0.466667,0.705882}%
\pgfsetfillcolor{currentfill}%
\pgfsetlinewidth{1.003750pt}%
\definecolor{currentstroke}{rgb}{0.121569,0.466667,0.705882}%
\pgfsetstrokecolor{currentstroke}%
\pgfsetdash{}{0pt}%
\pgfpathmoveto{\pgfqpoint{0.701066in}{1.387647in}}%
\pgfpathcurveto{\pgfqpoint{0.706890in}{1.387647in}}{\pgfqpoint{0.712476in}{1.389960in}}{\pgfqpoint{0.716595in}{1.394079in}}%
\pgfpathcurveto{\pgfqpoint{0.720713in}{1.398197in}}{\pgfqpoint{0.723027in}{1.403783in}}{\pgfqpoint{0.723027in}{1.409607in}}%
\pgfpathcurveto{\pgfqpoint{0.723027in}{1.415431in}}{\pgfqpoint{0.720713in}{1.421017in}}{\pgfqpoint{0.716595in}{1.425135in}}%
\pgfpathcurveto{\pgfqpoint{0.712476in}{1.429253in}}{\pgfqpoint{0.706890in}{1.431567in}}{\pgfqpoint{0.701066in}{1.431567in}}%
\pgfpathcurveto{\pgfqpoint{0.695242in}{1.431567in}}{\pgfqpoint{0.689656in}{1.429253in}}{\pgfqpoint{0.685538in}{1.425135in}}%
\pgfpathcurveto{\pgfqpoint{0.681420in}{1.421017in}}{\pgfqpoint{0.679106in}{1.415431in}}{\pgfqpoint{0.679106in}{1.409607in}}%
\pgfpathcurveto{\pgfqpoint{0.679106in}{1.403783in}}{\pgfqpoint{0.681420in}{1.398197in}}{\pgfqpoint{0.685538in}{1.394079in}}%
\pgfpathcurveto{\pgfqpoint{0.689656in}{1.389960in}}{\pgfqpoint{0.695242in}{1.387647in}}{\pgfqpoint{0.701066in}{1.387647in}}%
\pgfpathclose%
\pgfusepath{stroke,fill}%
\end{pgfscope}%
\begin{pgfscope}%
\pgfpathrectangle{\pgfqpoint{0.211875in}{0.211875in}}{\pgfqpoint{1.313625in}{1.279725in}}%
\pgfusepath{clip}%
\pgfsetbuttcap%
\pgfsetroundjoin%
\definecolor{currentfill}{rgb}{0.121569,0.466667,0.705882}%
\pgfsetfillcolor{currentfill}%
\pgfsetlinewidth{1.003750pt}%
\definecolor{currentstroke}{rgb}{0.121569,0.466667,0.705882}%
\pgfsetstrokecolor{currentstroke}%
\pgfsetdash{}{0pt}%
\pgfpathmoveto{\pgfqpoint{1.052206in}{1.102478in}}%
\pgfpathcurveto{\pgfqpoint{1.058030in}{1.102478in}}{\pgfqpoint{1.063616in}{1.104792in}}{\pgfqpoint{1.067734in}{1.108910in}}%
\pgfpathcurveto{\pgfqpoint{1.071853in}{1.113028in}}{\pgfqpoint{1.074166in}{1.118614in}}{\pgfqpoint{1.074166in}{1.124438in}}%
\pgfpathcurveto{\pgfqpoint{1.074166in}{1.130262in}}{\pgfqpoint{1.071853in}{1.135848in}}{\pgfqpoint{1.067734in}{1.139966in}}%
\pgfpathcurveto{\pgfqpoint{1.063616in}{1.144085in}}{\pgfqpoint{1.058030in}{1.146398in}}{\pgfqpoint{1.052206in}{1.146398in}}%
\pgfpathcurveto{\pgfqpoint{1.046382in}{1.146398in}}{\pgfqpoint{1.040796in}{1.144085in}}{\pgfqpoint{1.036678in}{1.139966in}}%
\pgfpathcurveto{\pgfqpoint{1.032560in}{1.135848in}}{\pgfqpoint{1.030246in}{1.130262in}}{\pgfqpoint{1.030246in}{1.124438in}}%
\pgfpathcurveto{\pgfqpoint{1.030246in}{1.118614in}}{\pgfqpoint{1.032560in}{1.113028in}}{\pgfqpoint{1.036678in}{1.108910in}}%
\pgfpathcurveto{\pgfqpoint{1.040796in}{1.104792in}}{\pgfqpoint{1.046382in}{1.102478in}}{\pgfqpoint{1.052206in}{1.102478in}}%
\pgfpathclose%
\pgfusepath{stroke,fill}%
\end{pgfscope}%
\begin{pgfscope}%
\pgfpathrectangle{\pgfqpoint{0.211875in}{0.211875in}}{\pgfqpoint{1.313625in}{1.279725in}}%
\pgfusepath{clip}%
\pgfsetbuttcap%
\pgfsetroundjoin%
\definecolor{currentfill}{rgb}{0.121569,0.466667,0.705882}%
\pgfsetfillcolor{currentfill}%
\pgfsetlinewidth{1.003750pt}%
\definecolor{currentstroke}{rgb}{0.121569,0.466667,0.705882}%
\pgfsetstrokecolor{currentstroke}%
\pgfsetdash{}{0pt}%
\pgfpathmoveto{\pgfqpoint{0.891012in}{0.940178in}}%
\pgfpathcurveto{\pgfqpoint{0.896836in}{0.940178in}}{\pgfqpoint{0.902422in}{0.942491in}}{\pgfqpoint{0.906540in}{0.946610in}}%
\pgfpathcurveto{\pgfqpoint{0.910658in}{0.950728in}}{\pgfqpoint{0.912972in}{0.956314in}}{\pgfqpoint{0.912972in}{0.962138in}}%
\pgfpathcurveto{\pgfqpoint{0.912972in}{0.967962in}}{\pgfqpoint{0.910658in}{0.973548in}}{\pgfqpoint{0.906540in}{0.977666in}}%
\pgfpathcurveto{\pgfqpoint{0.902422in}{0.981784in}}{\pgfqpoint{0.896836in}{0.984098in}}{\pgfqpoint{0.891012in}{0.984098in}}%
\pgfpathcurveto{\pgfqpoint{0.885188in}{0.984098in}}{\pgfqpoint{0.879602in}{0.981784in}}{\pgfqpoint{0.875484in}{0.977666in}}%
\pgfpathcurveto{\pgfqpoint{0.871365in}{0.973548in}}{\pgfqpoint{0.869052in}{0.967962in}}{\pgfqpoint{0.869052in}{0.962138in}}%
\pgfpathcurveto{\pgfqpoint{0.869052in}{0.956314in}}{\pgfqpoint{0.871365in}{0.950728in}}{\pgfqpoint{0.875484in}{0.946610in}}%
\pgfpathcurveto{\pgfqpoint{0.879602in}{0.942491in}}{\pgfqpoint{0.885188in}{0.940178in}}{\pgfqpoint{0.891012in}{0.940178in}}%
\pgfpathclose%
\pgfusepath{stroke,fill}%
\end{pgfscope}%
\begin{pgfscope}%
\pgfpathrectangle{\pgfqpoint{0.211875in}{0.211875in}}{\pgfqpoint{1.313625in}{1.279725in}}%
\pgfusepath{clip}%
\pgfsetbuttcap%
\pgfsetroundjoin%
\definecolor{currentfill}{rgb}{0.121569,0.466667,0.705882}%
\pgfsetfillcolor{currentfill}%
\pgfsetlinewidth{1.003750pt}%
\definecolor{currentstroke}{rgb}{0.121569,0.466667,0.705882}%
\pgfsetstrokecolor{currentstroke}%
\pgfsetdash{}{0pt}%
\pgfpathmoveto{\pgfqpoint{1.423045in}{1.062275in}}%
\pgfpathcurveto{\pgfqpoint{1.428869in}{1.062275in}}{\pgfqpoint{1.434455in}{1.064589in}}{\pgfqpoint{1.438573in}{1.068707in}}%
\pgfpathcurveto{\pgfqpoint{1.442691in}{1.072826in}}{\pgfqpoint{1.445005in}{1.078412in}}{\pgfqpoint{1.445005in}{1.084236in}}%
\pgfpathcurveto{\pgfqpoint{1.445005in}{1.090060in}}{\pgfqpoint{1.442691in}{1.095646in}}{\pgfqpoint{1.438573in}{1.099764in}}%
\pgfpathcurveto{\pgfqpoint{1.434455in}{1.103882in}}{\pgfqpoint{1.428869in}{1.106196in}}{\pgfqpoint{1.423045in}{1.106196in}}%
\pgfpathcurveto{\pgfqpoint{1.417221in}{1.106196in}}{\pgfqpoint{1.411635in}{1.103882in}}{\pgfqpoint{1.407517in}{1.099764in}}%
\pgfpathcurveto{\pgfqpoint{1.403399in}{1.095646in}}{\pgfqpoint{1.401085in}{1.090060in}}{\pgfqpoint{1.401085in}{1.084236in}}%
\pgfpathcurveto{\pgfqpoint{1.401085in}{1.078412in}}{\pgfqpoint{1.403399in}{1.072826in}}{\pgfqpoint{1.407517in}{1.068707in}}%
\pgfpathcurveto{\pgfqpoint{1.411635in}{1.064589in}}{\pgfqpoint{1.417221in}{1.062275in}}{\pgfqpoint{1.423045in}{1.062275in}}%
\pgfpathclose%
\pgfusepath{stroke,fill}%
\end{pgfscope}%
\begin{pgfscope}%
\pgfpathrectangle{\pgfqpoint{0.211875in}{0.211875in}}{\pgfqpoint{1.313625in}{1.279725in}}%
\pgfusepath{clip}%
\pgfsetbuttcap%
\pgfsetroundjoin%
\definecolor{currentfill}{rgb}{0.121569,0.466667,0.705882}%
\pgfsetfillcolor{currentfill}%
\pgfsetlinewidth{1.003750pt}%
\definecolor{currentstroke}{rgb}{0.121569,0.466667,0.705882}%
\pgfsetstrokecolor{currentstroke}%
\pgfsetdash{}{0pt}%
\pgfpathmoveto{\pgfqpoint{1.124540in}{1.064989in}}%
\pgfpathcurveto{\pgfqpoint{1.130364in}{1.064989in}}{\pgfqpoint{1.135950in}{1.067303in}}{\pgfqpoint{1.140069in}{1.071421in}}%
\pgfpathcurveto{\pgfqpoint{1.144187in}{1.075539in}}{\pgfqpoint{1.146501in}{1.081126in}}{\pgfqpoint{1.146501in}{1.086950in}}%
\pgfpathcurveto{\pgfqpoint{1.146501in}{1.092774in}}{\pgfqpoint{1.144187in}{1.098360in}}{\pgfqpoint{1.140069in}{1.102478in}}%
\pgfpathcurveto{\pgfqpoint{1.135950in}{1.106596in}}{\pgfqpoint{1.130364in}{1.108910in}}{\pgfqpoint{1.124540in}{1.108910in}}%
\pgfpathcurveto{\pgfqpoint{1.118716in}{1.108910in}}{\pgfqpoint{1.113130in}{1.106596in}}{\pgfqpoint{1.109012in}{1.102478in}}%
\pgfpathcurveto{\pgfqpoint{1.104894in}{1.098360in}}{\pgfqpoint{1.102580in}{1.092774in}}{\pgfqpoint{1.102580in}{1.086950in}}%
\pgfpathcurveto{\pgfqpoint{1.102580in}{1.081126in}}{\pgfqpoint{1.104894in}{1.075539in}}{\pgfqpoint{1.109012in}{1.071421in}}%
\pgfpathcurveto{\pgfqpoint{1.113130in}{1.067303in}}{\pgfqpoint{1.118716in}{1.064989in}}{\pgfqpoint{1.124540in}{1.064989in}}%
\pgfpathclose%
\pgfusepath{stroke,fill}%
\end{pgfscope}%
\begin{pgfscope}%
\pgfpathrectangle{\pgfqpoint{0.211875in}{0.211875in}}{\pgfqpoint{1.313625in}{1.279725in}}%
\pgfusepath{clip}%
\pgfsetbuttcap%
\pgfsetroundjoin%
\definecolor{currentfill}{rgb}{0.121569,0.466667,0.705882}%
\pgfsetfillcolor{currentfill}%
\pgfsetlinewidth{1.003750pt}%
\definecolor{currentstroke}{rgb}{0.121569,0.466667,0.705882}%
\pgfsetstrokecolor{currentstroke}%
\pgfsetdash{}{0pt}%
\pgfpathmoveto{\pgfqpoint{1.313081in}{0.935793in}}%
\pgfpathcurveto{\pgfqpoint{1.318905in}{0.935793in}}{\pgfqpoint{1.324491in}{0.938106in}}{\pgfqpoint{1.328610in}{0.942225in}}%
\pgfpathcurveto{\pgfqpoint{1.332728in}{0.946343in}}{\pgfqpoint{1.335042in}{0.951929in}}{\pgfqpoint{1.335042in}{0.957753in}}%
\pgfpathcurveto{\pgfqpoint{1.335042in}{0.963577in}}{\pgfqpoint{1.332728in}{0.969163in}}{\pgfqpoint{1.328610in}{0.973281in}}%
\pgfpathcurveto{\pgfqpoint{1.324491in}{0.977399in}}{\pgfqpoint{1.318905in}{0.979713in}}{\pgfqpoint{1.313081in}{0.979713in}}%
\pgfpathcurveto{\pgfqpoint{1.307257in}{0.979713in}}{\pgfqpoint{1.301671in}{0.977399in}}{\pgfqpoint{1.297553in}{0.973281in}}%
\pgfpathcurveto{\pgfqpoint{1.293435in}{0.969163in}}{\pgfqpoint{1.291121in}{0.963577in}}{\pgfqpoint{1.291121in}{0.957753in}}%
\pgfpathcurveto{\pgfqpoint{1.291121in}{0.951929in}}{\pgfqpoint{1.293435in}{0.946343in}}{\pgfqpoint{1.297553in}{0.942225in}}%
\pgfpathcurveto{\pgfqpoint{1.301671in}{0.938106in}}{\pgfqpoint{1.307257in}{0.935793in}}{\pgfqpoint{1.313081in}{0.935793in}}%
\pgfpathclose%
\pgfusepath{stroke,fill}%
\end{pgfscope}%
\begin{pgfscope}%
\pgfpathrectangle{\pgfqpoint{0.211875in}{0.211875in}}{\pgfqpoint{1.313625in}{1.279725in}}%
\pgfusepath{clip}%
\pgfsetbuttcap%
\pgfsetroundjoin%
\definecolor{currentfill}{rgb}{0.121569,0.466667,0.705882}%
\pgfsetfillcolor{currentfill}%
\pgfsetlinewidth{1.003750pt}%
\definecolor{currentstroke}{rgb}{0.121569,0.466667,0.705882}%
\pgfsetstrokecolor{currentstroke}%
\pgfsetdash{}{0pt}%
\pgfpathmoveto{\pgfqpoint{1.320938in}{0.914128in}}%
\pgfpathcurveto{\pgfqpoint{1.326762in}{0.914128in}}{\pgfqpoint{1.332348in}{0.916442in}}{\pgfqpoint{1.336466in}{0.920560in}}%
\pgfpathcurveto{\pgfqpoint{1.340585in}{0.924678in}}{\pgfqpoint{1.342898in}{0.930264in}}{\pgfqpoint{1.342898in}{0.936088in}}%
\pgfpathcurveto{\pgfqpoint{1.342898in}{0.941912in}}{\pgfqpoint{1.340585in}{0.947498in}}{\pgfqpoint{1.336466in}{0.951616in}}%
\pgfpathcurveto{\pgfqpoint{1.332348in}{0.955734in}}{\pgfqpoint{1.326762in}{0.958048in}}{\pgfqpoint{1.320938in}{0.958048in}}%
\pgfpathcurveto{\pgfqpoint{1.315114in}{0.958048in}}{\pgfqpoint{1.309528in}{0.955734in}}{\pgfqpoint{1.305410in}{0.951616in}}%
\pgfpathcurveto{\pgfqpoint{1.301292in}{0.947498in}}{\pgfqpoint{1.298978in}{0.941912in}}{\pgfqpoint{1.298978in}{0.936088in}}%
\pgfpathcurveto{\pgfqpoint{1.298978in}{0.930264in}}{\pgfqpoint{1.301292in}{0.924678in}}{\pgfqpoint{1.305410in}{0.920560in}}%
\pgfpathcurveto{\pgfqpoint{1.309528in}{0.916442in}}{\pgfqpoint{1.315114in}{0.914128in}}{\pgfqpoint{1.320938in}{0.914128in}}%
\pgfpathclose%
\pgfusepath{stroke,fill}%
\end{pgfscope}%
\begin{pgfscope}%
\pgfpathrectangle{\pgfqpoint{0.211875in}{0.211875in}}{\pgfqpoint{1.313625in}{1.279725in}}%
\pgfusepath{clip}%
\pgfsetbuttcap%
\pgfsetroundjoin%
\definecolor{currentfill}{rgb}{0.121569,0.466667,0.705882}%
\pgfsetfillcolor{currentfill}%
\pgfsetlinewidth{1.003750pt}%
\definecolor{currentstroke}{rgb}{0.121569,0.466667,0.705882}%
\pgfsetstrokecolor{currentstroke}%
\pgfsetdash{}{0pt}%
\pgfpathmoveto{\pgfqpoint{1.130457in}{1.056374in}}%
\pgfpathcurveto{\pgfqpoint{1.136281in}{1.056374in}}{\pgfqpoint{1.141867in}{1.058687in}}{\pgfqpoint{1.145985in}{1.062806in}}%
\pgfpathcurveto{\pgfqpoint{1.150104in}{1.066924in}}{\pgfqpoint{1.152417in}{1.072510in}}{\pgfqpoint{1.152417in}{1.078334in}}%
\pgfpathcurveto{\pgfqpoint{1.152417in}{1.084158in}}{\pgfqpoint{1.150104in}{1.089744in}}{\pgfqpoint{1.145985in}{1.093862in}}%
\pgfpathcurveto{\pgfqpoint{1.141867in}{1.097980in}}{\pgfqpoint{1.136281in}{1.100294in}}{\pgfqpoint{1.130457in}{1.100294in}}%
\pgfpathcurveto{\pgfqpoint{1.124633in}{1.100294in}}{\pgfqpoint{1.119047in}{1.097980in}}{\pgfqpoint{1.114929in}{1.093862in}}%
\pgfpathcurveto{\pgfqpoint{1.110811in}{1.089744in}}{\pgfqpoint{1.108497in}{1.084158in}}{\pgfqpoint{1.108497in}{1.078334in}}%
\pgfpathcurveto{\pgfqpoint{1.108497in}{1.072510in}}{\pgfqpoint{1.110811in}{1.066924in}}{\pgfqpoint{1.114929in}{1.062806in}}%
\pgfpathcurveto{\pgfqpoint{1.119047in}{1.058687in}}{\pgfqpoint{1.124633in}{1.056374in}}{\pgfqpoint{1.130457in}{1.056374in}}%
\pgfpathclose%
\pgfusepath{stroke,fill}%
\end{pgfscope}%
\begin{pgfscope}%
\pgfpathrectangle{\pgfqpoint{0.211875in}{0.211875in}}{\pgfqpoint{1.313625in}{1.279725in}}%
\pgfusepath{clip}%
\pgfsetbuttcap%
\pgfsetroundjoin%
\definecolor{currentfill}{rgb}{0.121569,0.466667,0.705882}%
\pgfsetfillcolor{currentfill}%
\pgfsetlinewidth{1.003750pt}%
\definecolor{currentstroke}{rgb}{0.121569,0.466667,0.705882}%
\pgfsetstrokecolor{currentstroke}%
\pgfsetdash{}{0pt}%
\pgfpathmoveto{\pgfqpoint{0.910049in}{0.940258in}}%
\pgfpathcurveto{\pgfqpoint{0.915873in}{0.940258in}}{\pgfqpoint{0.921460in}{0.942572in}}{\pgfqpoint{0.925578in}{0.946690in}}%
\pgfpathcurveto{\pgfqpoint{0.929696in}{0.950808in}}{\pgfqpoint{0.932010in}{0.956395in}}{\pgfqpoint{0.932010in}{0.962218in}}%
\pgfpathcurveto{\pgfqpoint{0.932010in}{0.968042in}}{\pgfqpoint{0.929696in}{0.973629in}}{\pgfqpoint{0.925578in}{0.977747in}}%
\pgfpathcurveto{\pgfqpoint{0.921460in}{0.981865in}}{\pgfqpoint{0.915873in}{0.984179in}}{\pgfqpoint{0.910049in}{0.984179in}}%
\pgfpathcurveto{\pgfqpoint{0.904225in}{0.984179in}}{\pgfqpoint{0.898639in}{0.981865in}}{\pgfqpoint{0.894521in}{0.977747in}}%
\pgfpathcurveto{\pgfqpoint{0.890403in}{0.973629in}}{\pgfqpoint{0.888089in}{0.968042in}}{\pgfqpoint{0.888089in}{0.962218in}}%
\pgfpathcurveto{\pgfqpoint{0.888089in}{0.956395in}}{\pgfqpoint{0.890403in}{0.950808in}}{\pgfqpoint{0.894521in}{0.946690in}}%
\pgfpathcurveto{\pgfqpoint{0.898639in}{0.942572in}}{\pgfqpoint{0.904225in}{0.940258in}}{\pgfqpoint{0.910049in}{0.940258in}}%
\pgfpathclose%
\pgfusepath{stroke,fill}%
\end{pgfscope}%
\begin{pgfscope}%
\pgfpathrectangle{\pgfqpoint{0.211875in}{0.211875in}}{\pgfqpoint{1.313625in}{1.279725in}}%
\pgfusepath{clip}%
\pgfsetbuttcap%
\pgfsetroundjoin%
\definecolor{currentfill}{rgb}{0.121569,0.466667,0.705882}%
\pgfsetfillcolor{currentfill}%
\pgfsetlinewidth{1.003750pt}%
\definecolor{currentstroke}{rgb}{0.121569,0.466667,0.705882}%
\pgfsetstrokecolor{currentstroke}%
\pgfsetdash{}{0pt}%
\pgfpathmoveto{\pgfqpoint{1.112336in}{1.081391in}}%
\pgfpathcurveto{\pgfqpoint{1.118159in}{1.081391in}}{\pgfqpoint{1.123746in}{1.083705in}}{\pgfqpoint{1.127864in}{1.087823in}}%
\pgfpathcurveto{\pgfqpoint{1.131982in}{1.091941in}}{\pgfqpoint{1.134296in}{1.097527in}}{\pgfqpoint{1.134296in}{1.103351in}}%
\pgfpathcurveto{\pgfqpoint{1.134296in}{1.109175in}}{\pgfqpoint{1.131982in}{1.114761in}}{\pgfqpoint{1.127864in}{1.118879in}}%
\pgfpathcurveto{\pgfqpoint{1.123746in}{1.122998in}}{\pgfqpoint{1.118159in}{1.125311in}}{\pgfqpoint{1.112336in}{1.125311in}}%
\pgfpathcurveto{\pgfqpoint{1.106512in}{1.125311in}}{\pgfqpoint{1.100925in}{1.122998in}}{\pgfqpoint{1.096807in}{1.118879in}}%
\pgfpathcurveto{\pgfqpoint{1.092689in}{1.114761in}}{\pgfqpoint{1.090375in}{1.109175in}}{\pgfqpoint{1.090375in}{1.103351in}}%
\pgfpathcurveto{\pgfqpoint{1.090375in}{1.097527in}}{\pgfqpoint{1.092689in}{1.091941in}}{\pgfqpoint{1.096807in}{1.087823in}}%
\pgfpathcurveto{\pgfqpoint{1.100925in}{1.083705in}}{\pgfqpoint{1.106512in}{1.081391in}}{\pgfqpoint{1.112336in}{1.081391in}}%
\pgfpathclose%
\pgfusepath{stroke,fill}%
\end{pgfscope}%
\begin{pgfscope}%
\pgfpathrectangle{\pgfqpoint{0.211875in}{0.211875in}}{\pgfqpoint{1.313625in}{1.279725in}}%
\pgfusepath{clip}%
\pgfsetbuttcap%
\pgfsetroundjoin%
\definecolor{currentfill}{rgb}{0.121569,0.466667,0.705882}%
\pgfsetfillcolor{currentfill}%
\pgfsetlinewidth{1.003750pt}%
\definecolor{currentstroke}{rgb}{0.121569,0.466667,0.705882}%
\pgfsetstrokecolor{currentstroke}%
\pgfsetdash{}{0pt}%
\pgfpathmoveto{\pgfqpoint{1.359430in}{0.979763in}}%
\pgfpathcurveto{\pgfqpoint{1.365254in}{0.979763in}}{\pgfqpoint{1.370840in}{0.982077in}}{\pgfqpoint{1.374958in}{0.986195in}}%
\pgfpathcurveto{\pgfqpoint{1.379076in}{0.990313in}}{\pgfqpoint{1.381390in}{0.995900in}}{\pgfqpoint{1.381390in}{1.001723in}}%
\pgfpathcurveto{\pgfqpoint{1.381390in}{1.007547in}}{\pgfqpoint{1.379076in}{1.013134in}}{\pgfqpoint{1.374958in}{1.017252in}}%
\pgfpathcurveto{\pgfqpoint{1.370840in}{1.021370in}}{\pgfqpoint{1.365254in}{1.023684in}}{\pgfqpoint{1.359430in}{1.023684in}}%
\pgfpathcurveto{\pgfqpoint{1.353606in}{1.023684in}}{\pgfqpoint{1.348020in}{1.021370in}}{\pgfqpoint{1.343902in}{1.017252in}}%
\pgfpathcurveto{\pgfqpoint{1.339784in}{1.013134in}}{\pgfqpoint{1.337470in}{1.007547in}}{\pgfqpoint{1.337470in}{1.001723in}}%
\pgfpathcurveto{\pgfqpoint{1.337470in}{0.995900in}}{\pgfqpoint{1.339784in}{0.990313in}}{\pgfqpoint{1.343902in}{0.986195in}}%
\pgfpathcurveto{\pgfqpoint{1.348020in}{0.982077in}}{\pgfqpoint{1.353606in}{0.979763in}}{\pgfqpoint{1.359430in}{0.979763in}}%
\pgfpathclose%
\pgfusepath{stroke,fill}%
\end{pgfscope}%
\begin{pgfscope}%
\pgfpathrectangle{\pgfqpoint{0.211875in}{0.211875in}}{\pgfqpoint{1.313625in}{1.279725in}}%
\pgfusepath{clip}%
\pgfsetbuttcap%
\pgfsetroundjoin%
\definecolor{currentfill}{rgb}{0.121569,0.466667,0.705882}%
\pgfsetfillcolor{currentfill}%
\pgfsetlinewidth{1.003750pt}%
\definecolor{currentstroke}{rgb}{0.121569,0.466667,0.705882}%
\pgfsetstrokecolor{currentstroke}%
\pgfsetdash{}{0pt}%
\pgfpathmoveto{\pgfqpoint{0.906206in}{0.921830in}}%
\pgfpathcurveto{\pgfqpoint{0.912030in}{0.921830in}}{\pgfqpoint{0.917616in}{0.924144in}}{\pgfqpoint{0.921734in}{0.928262in}}%
\pgfpathcurveto{\pgfqpoint{0.925852in}{0.932380in}}{\pgfqpoint{0.928166in}{0.937966in}}{\pgfqpoint{0.928166in}{0.943790in}}%
\pgfpathcurveto{\pgfqpoint{0.928166in}{0.949614in}}{\pgfqpoint{0.925852in}{0.955200in}}{\pgfqpoint{0.921734in}{0.959318in}}%
\pgfpathcurveto{\pgfqpoint{0.917616in}{0.963436in}}{\pgfqpoint{0.912030in}{0.965750in}}{\pgfqpoint{0.906206in}{0.965750in}}%
\pgfpathcurveto{\pgfqpoint{0.900382in}{0.965750in}}{\pgfqpoint{0.894796in}{0.963436in}}{\pgfqpoint{0.890677in}{0.959318in}}%
\pgfpathcurveto{\pgfqpoint{0.886559in}{0.955200in}}{\pgfqpoint{0.884245in}{0.949614in}}{\pgfqpoint{0.884245in}{0.943790in}}%
\pgfpathcurveto{\pgfqpoint{0.884245in}{0.937966in}}{\pgfqpoint{0.886559in}{0.932380in}}{\pgfqpoint{0.890677in}{0.928262in}}%
\pgfpathcurveto{\pgfqpoint{0.894796in}{0.924144in}}{\pgfqpoint{0.900382in}{0.921830in}}{\pgfqpoint{0.906206in}{0.921830in}}%
\pgfpathclose%
\pgfusepath{stroke,fill}%
\end{pgfscope}%
\begin{pgfscope}%
\pgfpathrectangle{\pgfqpoint{0.211875in}{0.211875in}}{\pgfqpoint{1.313625in}{1.279725in}}%
\pgfusepath{clip}%
\pgfsetbuttcap%
\pgfsetroundjoin%
\definecolor{currentfill}{rgb}{0.121569,0.466667,0.705882}%
\pgfsetfillcolor{currentfill}%
\pgfsetlinewidth{1.003750pt}%
\definecolor{currentstroke}{rgb}{0.121569,0.466667,0.705882}%
\pgfsetstrokecolor{currentstroke}%
\pgfsetdash{}{0pt}%
\pgfpathmoveto{\pgfqpoint{1.260177in}{0.912534in}}%
\pgfpathcurveto{\pgfqpoint{1.266001in}{0.912534in}}{\pgfqpoint{1.271587in}{0.914848in}}{\pgfqpoint{1.275705in}{0.918966in}}%
\pgfpathcurveto{\pgfqpoint{1.279824in}{0.923084in}}{\pgfqpoint{1.282137in}{0.928670in}}{\pgfqpoint{1.282137in}{0.934494in}}%
\pgfpathcurveto{\pgfqpoint{1.282137in}{0.940318in}}{\pgfqpoint{1.279824in}{0.945905in}}{\pgfqpoint{1.275705in}{0.950023in}}%
\pgfpathcurveto{\pgfqpoint{1.271587in}{0.954141in}}{\pgfqpoint{1.266001in}{0.956455in}}{\pgfqpoint{1.260177in}{0.956455in}}%
\pgfpathcurveto{\pgfqpoint{1.254353in}{0.956455in}}{\pgfqpoint{1.248767in}{0.954141in}}{\pgfqpoint{1.244649in}{0.950023in}}%
\pgfpathcurveto{\pgfqpoint{1.240531in}{0.945905in}}{\pgfqpoint{1.238217in}{0.940318in}}{\pgfqpoint{1.238217in}{0.934494in}}%
\pgfpathcurveto{\pgfqpoint{1.238217in}{0.928670in}}{\pgfqpoint{1.240531in}{0.923084in}}{\pgfqpoint{1.244649in}{0.918966in}}%
\pgfpathcurveto{\pgfqpoint{1.248767in}{0.914848in}}{\pgfqpoint{1.254353in}{0.912534in}}{\pgfqpoint{1.260177in}{0.912534in}}%
\pgfpathclose%
\pgfusepath{stroke,fill}%
\end{pgfscope}%
\begin{pgfscope}%
\pgfpathrectangle{\pgfqpoint{0.211875in}{0.211875in}}{\pgfqpoint{1.313625in}{1.279725in}}%
\pgfusepath{clip}%
\pgfsetbuttcap%
\pgfsetroundjoin%
\definecolor{currentfill}{rgb}{0.121569,0.466667,0.705882}%
\pgfsetfillcolor{currentfill}%
\pgfsetlinewidth{1.003750pt}%
\definecolor{currentstroke}{rgb}{0.121569,0.466667,0.705882}%
\pgfsetstrokecolor{currentstroke}%
\pgfsetdash{}{0pt}%
\pgfpathmoveto{\pgfqpoint{1.061181in}{0.404625in}}%
\pgfpathcurveto{\pgfqpoint{1.067005in}{0.404625in}}{\pgfqpoint{1.072591in}{0.406939in}}{\pgfqpoint{1.076709in}{0.411057in}}%
\pgfpathcurveto{\pgfqpoint{1.080827in}{0.415175in}}{\pgfqpoint{1.083141in}{0.420761in}}{\pgfqpoint{1.083141in}{0.426585in}}%
\pgfpathcurveto{\pgfqpoint{1.083141in}{0.432409in}}{\pgfqpoint{1.080827in}{0.437996in}}{\pgfqpoint{1.076709in}{0.442114in}}%
\pgfpathcurveto{\pgfqpoint{1.072591in}{0.446232in}}{\pgfqpoint{1.067005in}{0.448546in}}{\pgfqpoint{1.061181in}{0.448546in}}%
\pgfpathcurveto{\pgfqpoint{1.055357in}{0.448546in}}{\pgfqpoint{1.049771in}{0.446232in}}{\pgfqpoint{1.045653in}{0.442114in}}%
\pgfpathcurveto{\pgfqpoint{1.041534in}{0.437996in}}{\pgfqpoint{1.039221in}{0.432409in}}{\pgfqpoint{1.039221in}{0.426585in}}%
\pgfpathcurveto{\pgfqpoint{1.039221in}{0.420761in}}{\pgfqpoint{1.041534in}{0.415175in}}{\pgfqpoint{1.045653in}{0.411057in}}%
\pgfpathcurveto{\pgfqpoint{1.049771in}{0.406939in}}{\pgfqpoint{1.055357in}{0.404625in}}{\pgfqpoint{1.061181in}{0.404625in}}%
\pgfpathclose%
\pgfusepath{stroke,fill}%
\end{pgfscope}%
\begin{pgfscope}%
\pgfpathrectangle{\pgfqpoint{0.211875in}{0.211875in}}{\pgfqpoint{1.313625in}{1.279725in}}%
\pgfusepath{clip}%
\pgfsetbuttcap%
\pgfsetroundjoin%
\definecolor{currentfill}{rgb}{0.121569,0.466667,0.705882}%
\pgfsetfillcolor{currentfill}%
\pgfsetlinewidth{1.003750pt}%
\definecolor{currentstroke}{rgb}{0.121569,0.466667,0.705882}%
\pgfsetstrokecolor{currentstroke}%
\pgfsetdash{}{0pt}%
\pgfpathmoveto{\pgfqpoint{1.106369in}{1.067136in}}%
\pgfpathcurveto{\pgfqpoint{1.112192in}{1.067136in}}{\pgfqpoint{1.117779in}{1.069450in}}{\pgfqpoint{1.121897in}{1.073568in}}%
\pgfpathcurveto{\pgfqpoint{1.126015in}{1.077686in}}{\pgfqpoint{1.128329in}{1.083273in}}{\pgfqpoint{1.128329in}{1.089096in}}%
\pgfpathcurveto{\pgfqpoint{1.128329in}{1.094920in}}{\pgfqpoint{1.126015in}{1.100507in}}{\pgfqpoint{1.121897in}{1.104625in}}%
\pgfpathcurveto{\pgfqpoint{1.117779in}{1.108743in}}{\pgfqpoint{1.112192in}{1.111057in}}{\pgfqpoint{1.106369in}{1.111057in}}%
\pgfpathcurveto{\pgfqpoint{1.100545in}{1.111057in}}{\pgfqpoint{1.094958in}{1.108743in}}{\pgfqpoint{1.090840in}{1.104625in}}%
\pgfpathcurveto{\pgfqpoint{1.086722in}{1.100507in}}{\pgfqpoint{1.084408in}{1.094920in}}{\pgfqpoint{1.084408in}{1.089096in}}%
\pgfpathcurveto{\pgfqpoint{1.084408in}{1.083273in}}{\pgfqpoint{1.086722in}{1.077686in}}{\pgfqpoint{1.090840in}{1.073568in}}%
\pgfpathcurveto{\pgfqpoint{1.094958in}{1.069450in}}{\pgfqpoint{1.100545in}{1.067136in}}{\pgfqpoint{1.106369in}{1.067136in}}%
\pgfpathclose%
\pgfusepath{stroke,fill}%
\end{pgfscope}%
\begin{pgfscope}%
\pgfpathrectangle{\pgfqpoint{0.211875in}{0.211875in}}{\pgfqpoint{1.313625in}{1.279725in}}%
\pgfusepath{clip}%
\pgfsetbuttcap%
\pgfsetroundjoin%
\definecolor{currentfill}{rgb}{0.121569,0.466667,0.705882}%
\pgfsetfillcolor{currentfill}%
\pgfsetlinewidth{1.003750pt}%
\definecolor{currentstroke}{rgb}{0.121569,0.466667,0.705882}%
\pgfsetstrokecolor{currentstroke}%
\pgfsetdash{}{0pt}%
\pgfpathmoveto{\pgfqpoint{1.446373in}{0.881083in}}%
\pgfpathcurveto{\pgfqpoint{1.452197in}{0.881083in}}{\pgfqpoint{1.457783in}{0.883397in}}{\pgfqpoint{1.461901in}{0.887515in}}%
\pgfpathcurveto{\pgfqpoint{1.466019in}{0.891633in}}{\pgfqpoint{1.468333in}{0.897219in}}{\pgfqpoint{1.468333in}{0.903043in}}%
\pgfpathcurveto{\pgfqpoint{1.468333in}{0.908867in}}{\pgfqpoint{1.466019in}{0.914453in}}{\pgfqpoint{1.461901in}{0.918571in}}%
\pgfpathcurveto{\pgfqpoint{1.457783in}{0.922689in}}{\pgfqpoint{1.452197in}{0.925003in}}{\pgfqpoint{1.446373in}{0.925003in}}%
\pgfpathcurveto{\pgfqpoint{1.440549in}{0.925003in}}{\pgfqpoint{1.434963in}{0.922689in}}{\pgfqpoint{1.430845in}{0.918571in}}%
\pgfpathcurveto{\pgfqpoint{1.426726in}{0.914453in}}{\pgfqpoint{1.424413in}{0.908867in}}{\pgfqpoint{1.424413in}{0.903043in}}%
\pgfpathcurveto{\pgfqpoint{1.424413in}{0.897219in}}{\pgfqpoint{1.426726in}{0.891633in}}{\pgfqpoint{1.430845in}{0.887515in}}%
\pgfpathcurveto{\pgfqpoint{1.434963in}{0.883397in}}{\pgfqpoint{1.440549in}{0.881083in}}{\pgfqpoint{1.446373in}{0.881083in}}%
\pgfpathclose%
\pgfusepath{stroke,fill}%
\end{pgfscope}%
\begin{pgfscope}%
\pgfpathrectangle{\pgfqpoint{0.211875in}{0.211875in}}{\pgfqpoint{1.313625in}{1.279725in}}%
\pgfusepath{clip}%
\pgfsetbuttcap%
\pgfsetroundjoin%
\definecolor{currentfill}{rgb}{0.121569,0.466667,0.705882}%
\pgfsetfillcolor{currentfill}%
\pgfsetlinewidth{1.003750pt}%
\definecolor{currentstroke}{rgb}{0.121569,0.466667,0.705882}%
\pgfsetstrokecolor{currentstroke}%
\pgfsetdash{}{0pt}%
\pgfpathmoveto{\pgfqpoint{1.110876in}{1.058097in}}%
\pgfpathcurveto{\pgfqpoint{1.116699in}{1.058097in}}{\pgfqpoint{1.122286in}{1.060411in}}{\pgfqpoint{1.126404in}{1.064529in}}%
\pgfpathcurveto{\pgfqpoint{1.130522in}{1.068648in}}{\pgfqpoint{1.132836in}{1.074234in}}{\pgfqpoint{1.132836in}{1.080058in}}%
\pgfpathcurveto{\pgfqpoint{1.132836in}{1.085882in}}{\pgfqpoint{1.130522in}{1.091468in}}{\pgfqpoint{1.126404in}{1.095586in}}%
\pgfpathcurveto{\pgfqpoint{1.122286in}{1.099704in}}{\pgfqpoint{1.116699in}{1.102018in}}{\pgfqpoint{1.110876in}{1.102018in}}%
\pgfpathcurveto{\pgfqpoint{1.105052in}{1.102018in}}{\pgfqpoint{1.099465in}{1.099704in}}{\pgfqpoint{1.095347in}{1.095586in}}%
\pgfpathcurveto{\pgfqpoint{1.091229in}{1.091468in}}{\pgfqpoint{1.088915in}{1.085882in}}{\pgfqpoint{1.088915in}{1.080058in}}%
\pgfpathcurveto{\pgfqpoint{1.088915in}{1.074234in}}{\pgfqpoint{1.091229in}{1.068648in}}{\pgfqpoint{1.095347in}{1.064529in}}%
\pgfpathcurveto{\pgfqpoint{1.099465in}{1.060411in}}{\pgfqpoint{1.105052in}{1.058097in}}{\pgfqpoint{1.110876in}{1.058097in}}%
\pgfpathclose%
\pgfusepath{stroke,fill}%
\end{pgfscope}%
\begin{pgfscope}%
\pgfpathrectangle{\pgfqpoint{0.211875in}{0.211875in}}{\pgfqpoint{1.313625in}{1.279725in}}%
\pgfusepath{clip}%
\pgfsetbuttcap%
\pgfsetroundjoin%
\definecolor{currentfill}{rgb}{0.121569,0.466667,0.705882}%
\pgfsetfillcolor{currentfill}%
\pgfsetlinewidth{1.003750pt}%
\definecolor{currentstroke}{rgb}{0.121569,0.466667,0.705882}%
\pgfsetstrokecolor{currentstroke}%
\pgfsetdash{}{0pt}%
\pgfpathmoveto{\pgfqpoint{1.342071in}{0.957201in}}%
\pgfpathcurveto{\pgfqpoint{1.347895in}{0.957201in}}{\pgfqpoint{1.353481in}{0.959515in}}{\pgfqpoint{1.357599in}{0.963633in}}%
\pgfpathcurveto{\pgfqpoint{1.361717in}{0.967751in}}{\pgfqpoint{1.364031in}{0.973338in}}{\pgfqpoint{1.364031in}{0.979161in}}%
\pgfpathcurveto{\pgfqpoint{1.364031in}{0.984985in}}{\pgfqpoint{1.361717in}{0.990572in}}{\pgfqpoint{1.357599in}{0.994690in}}%
\pgfpathcurveto{\pgfqpoint{1.353481in}{0.998808in}}{\pgfqpoint{1.347895in}{1.001122in}}{\pgfqpoint{1.342071in}{1.001122in}}%
\pgfpathcurveto{\pgfqpoint{1.336247in}{1.001122in}}{\pgfqpoint{1.330661in}{0.998808in}}{\pgfqpoint{1.326543in}{0.994690in}}%
\pgfpathcurveto{\pgfqpoint{1.322424in}{0.990572in}}{\pgfqpoint{1.320111in}{0.984985in}}{\pgfqpoint{1.320111in}{0.979161in}}%
\pgfpathcurveto{\pgfqpoint{1.320111in}{0.973338in}}{\pgfqpoint{1.322424in}{0.967751in}}{\pgfqpoint{1.326543in}{0.963633in}}%
\pgfpathcurveto{\pgfqpoint{1.330661in}{0.959515in}}{\pgfqpoint{1.336247in}{0.957201in}}{\pgfqpoint{1.342071in}{0.957201in}}%
\pgfpathclose%
\pgfusepath{stroke,fill}%
\end{pgfscope}%
\begin{pgfscope}%
\pgfpathrectangle{\pgfqpoint{0.211875in}{0.211875in}}{\pgfqpoint{1.313625in}{1.279725in}}%
\pgfusepath{clip}%
\pgfsetbuttcap%
\pgfsetroundjoin%
\definecolor{currentfill}{rgb}{0.121569,0.466667,0.705882}%
\pgfsetfillcolor{currentfill}%
\pgfsetlinewidth{1.003750pt}%
\definecolor{currentstroke}{rgb}{0.121569,0.466667,0.705882}%
\pgfsetstrokecolor{currentstroke}%
\pgfsetdash{}{0pt}%
\pgfpathmoveto{\pgfqpoint{1.116284in}{1.052924in}}%
\pgfpathcurveto{\pgfqpoint{1.122108in}{1.052924in}}{\pgfqpoint{1.127694in}{1.055238in}}{\pgfqpoint{1.131812in}{1.059356in}}%
\pgfpathcurveto{\pgfqpoint{1.135930in}{1.063474in}}{\pgfqpoint{1.138244in}{1.069061in}}{\pgfqpoint{1.138244in}{1.074884in}}%
\pgfpathcurveto{\pgfqpoint{1.138244in}{1.080708in}}{\pgfqpoint{1.135930in}{1.086295in}}{\pgfqpoint{1.131812in}{1.090413in}}%
\pgfpathcurveto{\pgfqpoint{1.127694in}{1.094531in}}{\pgfqpoint{1.122108in}{1.096845in}}{\pgfqpoint{1.116284in}{1.096845in}}%
\pgfpathcurveto{\pgfqpoint{1.110460in}{1.096845in}}{\pgfqpoint{1.104874in}{1.094531in}}{\pgfqpoint{1.100755in}{1.090413in}}%
\pgfpathcurveto{\pgfqpoint{1.096637in}{1.086295in}}{\pgfqpoint{1.094323in}{1.080708in}}{\pgfqpoint{1.094323in}{1.074884in}}%
\pgfpathcurveto{\pgfqpoint{1.094323in}{1.069061in}}{\pgfqpoint{1.096637in}{1.063474in}}{\pgfqpoint{1.100755in}{1.059356in}}%
\pgfpathcurveto{\pgfqpoint{1.104874in}{1.055238in}}{\pgfqpoint{1.110460in}{1.052924in}}{\pgfqpoint{1.116284in}{1.052924in}}%
\pgfpathclose%
\pgfusepath{stroke,fill}%
\end{pgfscope}%
\begin{pgfscope}%
\pgfpathrectangle{\pgfqpoint{0.211875in}{0.211875in}}{\pgfqpoint{1.313625in}{1.279725in}}%
\pgfusepath{clip}%
\pgfsetbuttcap%
\pgfsetroundjoin%
\definecolor{currentfill}{rgb}{0.121569,0.466667,0.705882}%
\pgfsetfillcolor{currentfill}%
\pgfsetlinewidth{1.003750pt}%
\definecolor{currentstroke}{rgb}{0.121569,0.466667,0.705882}%
\pgfsetstrokecolor{currentstroke}%
\pgfsetdash{}{0pt}%
\pgfpathmoveto{\pgfqpoint{0.922447in}{0.932496in}}%
\pgfpathcurveto{\pgfqpoint{0.928271in}{0.932496in}}{\pgfqpoint{0.933857in}{0.934810in}}{\pgfqpoint{0.937975in}{0.938928in}}%
\pgfpathcurveto{\pgfqpoint{0.942093in}{0.943046in}}{\pgfqpoint{0.944407in}{0.948632in}}{\pgfqpoint{0.944407in}{0.954456in}}%
\pgfpathcurveto{\pgfqpoint{0.944407in}{0.960280in}}{\pgfqpoint{0.942093in}{0.965866in}}{\pgfqpoint{0.937975in}{0.969984in}}%
\pgfpathcurveto{\pgfqpoint{0.933857in}{0.974103in}}{\pgfqpoint{0.928271in}{0.976416in}}{\pgfqpoint{0.922447in}{0.976416in}}%
\pgfpathcurveto{\pgfqpoint{0.916623in}{0.976416in}}{\pgfqpoint{0.911037in}{0.974103in}}{\pgfqpoint{0.906918in}{0.969984in}}%
\pgfpathcurveto{\pgfqpoint{0.902800in}{0.965866in}}{\pgfqpoint{0.900486in}{0.960280in}}{\pgfqpoint{0.900486in}{0.954456in}}%
\pgfpathcurveto{\pgfqpoint{0.900486in}{0.948632in}}{\pgfqpoint{0.902800in}{0.943046in}}{\pgfqpoint{0.906918in}{0.938928in}}%
\pgfpathcurveto{\pgfqpoint{0.911037in}{0.934810in}}{\pgfqpoint{0.916623in}{0.932496in}}{\pgfqpoint{0.922447in}{0.932496in}}%
\pgfpathclose%
\pgfusepath{stroke,fill}%
\end{pgfscope}%
\begin{pgfscope}%
\pgfpathrectangle{\pgfqpoint{0.211875in}{0.211875in}}{\pgfqpoint{1.313625in}{1.279725in}}%
\pgfusepath{clip}%
\pgfsetbuttcap%
\pgfsetroundjoin%
\definecolor{currentfill}{rgb}{0.121569,0.466667,0.705882}%
\pgfsetfillcolor{currentfill}%
\pgfsetlinewidth{1.003750pt}%
\definecolor{currentstroke}{rgb}{0.121569,0.466667,0.705882}%
\pgfsetstrokecolor{currentstroke}%
\pgfsetdash{}{0pt}%
\pgfpathmoveto{\pgfqpoint{1.114865in}{1.073198in}}%
\pgfpathcurveto{\pgfqpoint{1.120689in}{1.073198in}}{\pgfqpoint{1.126275in}{1.075512in}}{\pgfqpoint{1.130394in}{1.079630in}}%
\pgfpathcurveto{\pgfqpoint{1.134512in}{1.083748in}}{\pgfqpoint{1.136826in}{1.089334in}}{\pgfqpoint{1.136826in}{1.095158in}}%
\pgfpathcurveto{\pgfqpoint{1.136826in}{1.100982in}}{\pgfqpoint{1.134512in}{1.106568in}}{\pgfqpoint{1.130394in}{1.110686in}}%
\pgfpathcurveto{\pgfqpoint{1.126275in}{1.114805in}}{\pgfqpoint{1.120689in}{1.117118in}}{\pgfqpoint{1.114865in}{1.117118in}}%
\pgfpathcurveto{\pgfqpoint{1.109041in}{1.117118in}}{\pgfqpoint{1.103455in}{1.114805in}}{\pgfqpoint{1.099337in}{1.110686in}}%
\pgfpathcurveto{\pgfqpoint{1.095219in}{1.106568in}}{\pgfqpoint{1.092905in}{1.100982in}}{\pgfqpoint{1.092905in}{1.095158in}}%
\pgfpathcurveto{\pgfqpoint{1.092905in}{1.089334in}}{\pgfqpoint{1.095219in}{1.083748in}}{\pgfqpoint{1.099337in}{1.079630in}}%
\pgfpathcurveto{\pgfqpoint{1.103455in}{1.075512in}}{\pgfqpoint{1.109041in}{1.073198in}}{\pgfqpoint{1.114865in}{1.073198in}}%
\pgfpathclose%
\pgfusepath{stroke,fill}%
\end{pgfscope}%
\begin{pgfscope}%
\pgfpathrectangle{\pgfqpoint{0.211875in}{0.211875in}}{\pgfqpoint{1.313625in}{1.279725in}}%
\pgfusepath{clip}%
\pgfsetbuttcap%
\pgfsetroundjoin%
\definecolor{currentfill}{rgb}{0.121569,0.466667,0.705882}%
\pgfsetfillcolor{currentfill}%
\pgfsetlinewidth{1.003750pt}%
\definecolor{currentstroke}{rgb}{0.121569,0.466667,0.705882}%
\pgfsetstrokecolor{currentstroke}%
\pgfsetdash{}{0pt}%
\pgfpathmoveto{\pgfqpoint{0.808811in}{1.073188in}}%
\pgfpathcurveto{\pgfqpoint{0.814635in}{1.073188in}}{\pgfqpoint{0.820221in}{1.075502in}}{\pgfqpoint{0.824339in}{1.079620in}}%
\pgfpathcurveto{\pgfqpoint{0.828458in}{1.083738in}}{\pgfqpoint{0.830771in}{1.089324in}}{\pgfqpoint{0.830771in}{1.095148in}}%
\pgfpathcurveto{\pgfqpoint{0.830771in}{1.100972in}}{\pgfqpoint{0.828458in}{1.106558in}}{\pgfqpoint{0.824339in}{1.110676in}}%
\pgfpathcurveto{\pgfqpoint{0.820221in}{1.114795in}}{\pgfqpoint{0.814635in}{1.117109in}}{\pgfqpoint{0.808811in}{1.117109in}}%
\pgfpathcurveto{\pgfqpoint{0.802987in}{1.117109in}}{\pgfqpoint{0.797401in}{1.114795in}}{\pgfqpoint{0.793283in}{1.110676in}}%
\pgfpathcurveto{\pgfqpoint{0.789165in}{1.106558in}}{\pgfqpoint{0.786851in}{1.100972in}}{\pgfqpoint{0.786851in}{1.095148in}}%
\pgfpathcurveto{\pgfqpoint{0.786851in}{1.089324in}}{\pgfqpoint{0.789165in}{1.083738in}}{\pgfqpoint{0.793283in}{1.079620in}}%
\pgfpathcurveto{\pgfqpoint{0.797401in}{1.075502in}}{\pgfqpoint{0.802987in}{1.073188in}}{\pgfqpoint{0.808811in}{1.073188in}}%
\pgfpathclose%
\pgfusepath{stroke,fill}%
\end{pgfscope}%
\begin{pgfscope}%
\pgfpathrectangle{\pgfqpoint{0.211875in}{0.211875in}}{\pgfqpoint{1.313625in}{1.279725in}}%
\pgfusepath{clip}%
\pgfsetbuttcap%
\pgfsetroundjoin%
\definecolor{currentfill}{rgb}{0.121569,0.466667,0.705882}%
\pgfsetfillcolor{currentfill}%
\pgfsetlinewidth{1.003750pt}%
\definecolor{currentstroke}{rgb}{0.121569,0.466667,0.705882}%
\pgfsetstrokecolor{currentstroke}%
\pgfsetdash{}{0pt}%
\pgfpathmoveto{\pgfqpoint{1.246476in}{0.825671in}}%
\pgfpathcurveto{\pgfqpoint{1.252300in}{0.825671in}}{\pgfqpoint{1.257886in}{0.827985in}}{\pgfqpoint{1.262004in}{0.832103in}}%
\pgfpathcurveto{\pgfqpoint{1.266123in}{0.836221in}}{\pgfqpoint{1.268436in}{0.841808in}}{\pgfqpoint{1.268436in}{0.847631in}}%
\pgfpathcurveto{\pgfqpoint{1.268436in}{0.853455in}}{\pgfqpoint{1.266123in}{0.859042in}}{\pgfqpoint{1.262004in}{0.863160in}}%
\pgfpathcurveto{\pgfqpoint{1.257886in}{0.867278in}}{\pgfqpoint{1.252300in}{0.869592in}}{\pgfqpoint{1.246476in}{0.869592in}}%
\pgfpathcurveto{\pgfqpoint{1.240652in}{0.869592in}}{\pgfqpoint{1.235066in}{0.867278in}}{\pgfqpoint{1.230948in}{0.863160in}}%
\pgfpathcurveto{\pgfqpoint{1.226830in}{0.859042in}}{\pgfqpoint{1.224516in}{0.853455in}}{\pgfqpoint{1.224516in}{0.847631in}}%
\pgfpathcurveto{\pgfqpoint{1.224516in}{0.841808in}}{\pgfqpoint{1.226830in}{0.836221in}}{\pgfqpoint{1.230948in}{0.832103in}}%
\pgfpathcurveto{\pgfqpoint{1.235066in}{0.827985in}}{\pgfqpoint{1.240652in}{0.825671in}}{\pgfqpoint{1.246476in}{0.825671in}}%
\pgfpathclose%
\pgfusepath{stroke,fill}%
\end{pgfscope}%
\begin{pgfscope}%
\pgfpathrectangle{\pgfqpoint{0.211875in}{0.211875in}}{\pgfqpoint{1.313625in}{1.279725in}}%
\pgfusepath{clip}%
\pgfsetbuttcap%
\pgfsetroundjoin%
\definecolor{currentfill}{rgb}{0.121569,0.466667,0.705882}%
\pgfsetfillcolor{currentfill}%
\pgfsetlinewidth{1.003750pt}%
\definecolor{currentstroke}{rgb}{0.121569,0.466667,0.705882}%
\pgfsetstrokecolor{currentstroke}%
\pgfsetdash{}{0pt}%
\pgfpathmoveto{\pgfqpoint{1.323916in}{0.924749in}}%
\pgfpathcurveto{\pgfqpoint{1.329740in}{0.924749in}}{\pgfqpoint{1.335326in}{0.927063in}}{\pgfqpoint{1.339444in}{0.931181in}}%
\pgfpathcurveto{\pgfqpoint{1.343562in}{0.935299in}}{\pgfqpoint{1.345876in}{0.940885in}}{\pgfqpoint{1.345876in}{0.946709in}}%
\pgfpathcurveto{\pgfqpoint{1.345876in}{0.952533in}}{\pgfqpoint{1.343562in}{0.958119in}}{\pgfqpoint{1.339444in}{0.962237in}}%
\pgfpathcurveto{\pgfqpoint{1.335326in}{0.966356in}}{\pgfqpoint{1.329740in}{0.968669in}}{\pgfqpoint{1.323916in}{0.968669in}}%
\pgfpathcurveto{\pgfqpoint{1.318092in}{0.968669in}}{\pgfqpoint{1.312506in}{0.966356in}}{\pgfqpoint{1.308388in}{0.962237in}}%
\pgfpathcurveto{\pgfqpoint{1.304270in}{0.958119in}}{\pgfqpoint{1.301956in}{0.952533in}}{\pgfqpoint{1.301956in}{0.946709in}}%
\pgfpathcurveto{\pgfqpoint{1.301956in}{0.940885in}}{\pgfqpoint{1.304270in}{0.935299in}}{\pgfqpoint{1.308388in}{0.931181in}}%
\pgfpathcurveto{\pgfqpoint{1.312506in}{0.927063in}}{\pgfqpoint{1.318092in}{0.924749in}}{\pgfqpoint{1.323916in}{0.924749in}}%
\pgfpathclose%
\pgfusepath{stroke,fill}%
\end{pgfscope}%
\begin{pgfscope}%
\pgfpathrectangle{\pgfqpoint{0.211875in}{0.211875in}}{\pgfqpoint{1.313625in}{1.279725in}}%
\pgfusepath{clip}%
\pgfsetbuttcap%
\pgfsetroundjoin%
\definecolor{currentfill}{rgb}{0.121569,0.466667,0.705882}%
\pgfsetfillcolor{currentfill}%
\pgfsetlinewidth{1.003750pt}%
\definecolor{currentstroke}{rgb}{0.121569,0.466667,0.705882}%
\pgfsetstrokecolor{currentstroke}%
\pgfsetdash{}{0pt}%
\pgfpathmoveto{\pgfqpoint{1.093173in}{0.399076in}}%
\pgfpathcurveto{\pgfqpoint{1.098997in}{0.399076in}}{\pgfqpoint{1.104583in}{0.401390in}}{\pgfqpoint{1.108701in}{0.405508in}}%
\pgfpathcurveto{\pgfqpoint{1.112819in}{0.409626in}}{\pgfqpoint{1.115133in}{0.415212in}}{\pgfqpoint{1.115133in}{0.421036in}}%
\pgfpathcurveto{\pgfqpoint{1.115133in}{0.426860in}}{\pgfqpoint{1.112819in}{0.432446in}}{\pgfqpoint{1.108701in}{0.436564in}}%
\pgfpathcurveto{\pgfqpoint{1.104583in}{0.440683in}}{\pgfqpoint{1.098997in}{0.442996in}}{\pgfqpoint{1.093173in}{0.442996in}}%
\pgfpathcurveto{\pgfqpoint{1.087349in}{0.442996in}}{\pgfqpoint{1.081763in}{0.440683in}}{\pgfqpoint{1.077645in}{0.436564in}}%
\pgfpathcurveto{\pgfqpoint{1.073527in}{0.432446in}}{\pgfqpoint{1.071213in}{0.426860in}}{\pgfqpoint{1.071213in}{0.421036in}}%
\pgfpathcurveto{\pgfqpoint{1.071213in}{0.415212in}}{\pgfqpoint{1.073527in}{0.409626in}}{\pgfqpoint{1.077645in}{0.405508in}}%
\pgfpathcurveto{\pgfqpoint{1.081763in}{0.401390in}}{\pgfqpoint{1.087349in}{0.399076in}}{\pgfqpoint{1.093173in}{0.399076in}}%
\pgfpathclose%
\pgfusepath{stroke,fill}%
\end{pgfscope}%
\begin{pgfscope}%
\pgfpathrectangle{\pgfqpoint{0.211875in}{0.211875in}}{\pgfqpoint{1.313625in}{1.279725in}}%
\pgfusepath{clip}%
\pgfsetbuttcap%
\pgfsetroundjoin%
\definecolor{currentfill}{rgb}{0.121569,0.466667,0.705882}%
\pgfsetfillcolor{currentfill}%
\pgfsetlinewidth{1.003750pt}%
\definecolor{currentstroke}{rgb}{0.121569,0.466667,0.705882}%
\pgfsetstrokecolor{currentstroke}%
\pgfsetdash{}{0pt}%
\pgfpathmoveto{\pgfqpoint{1.319462in}{0.913732in}}%
\pgfpathcurveto{\pgfqpoint{1.325286in}{0.913732in}}{\pgfqpoint{1.330872in}{0.916046in}}{\pgfqpoint{1.334991in}{0.920164in}}%
\pgfpathcurveto{\pgfqpoint{1.339109in}{0.924282in}}{\pgfqpoint{1.341423in}{0.929868in}}{\pgfqpoint{1.341423in}{0.935692in}}%
\pgfpathcurveto{\pgfqpoint{1.341423in}{0.941516in}}{\pgfqpoint{1.339109in}{0.947102in}}{\pgfqpoint{1.334991in}{0.951220in}}%
\pgfpathcurveto{\pgfqpoint{1.330872in}{0.955339in}}{\pgfqpoint{1.325286in}{0.957652in}}{\pgfqpoint{1.319462in}{0.957652in}}%
\pgfpathcurveto{\pgfqpoint{1.313638in}{0.957652in}}{\pgfqpoint{1.308052in}{0.955339in}}{\pgfqpoint{1.303934in}{0.951220in}}%
\pgfpathcurveto{\pgfqpoint{1.299816in}{0.947102in}}{\pgfqpoint{1.297502in}{0.941516in}}{\pgfqpoint{1.297502in}{0.935692in}}%
\pgfpathcurveto{\pgfqpoint{1.297502in}{0.929868in}}{\pgfqpoint{1.299816in}{0.924282in}}{\pgfqpoint{1.303934in}{0.920164in}}%
\pgfpathcurveto{\pgfqpoint{1.308052in}{0.916046in}}{\pgfqpoint{1.313638in}{0.913732in}}{\pgfqpoint{1.319462in}{0.913732in}}%
\pgfpathclose%
\pgfusepath{stroke,fill}%
\end{pgfscope}%
\begin{pgfscope}%
\pgfpathrectangle{\pgfqpoint{0.211875in}{0.211875in}}{\pgfqpoint{1.313625in}{1.279725in}}%
\pgfusepath{clip}%
\pgfsetbuttcap%
\pgfsetroundjoin%
\definecolor{currentfill}{rgb}{0.121569,0.466667,0.705882}%
\pgfsetfillcolor{currentfill}%
\pgfsetlinewidth{1.003750pt}%
\definecolor{currentstroke}{rgb}{0.121569,0.466667,0.705882}%
\pgfsetstrokecolor{currentstroke}%
\pgfsetdash{}{0pt}%
\pgfpathmoveto{\pgfqpoint{1.348029in}{0.901968in}}%
\pgfpathcurveto{\pgfqpoint{1.353853in}{0.901968in}}{\pgfqpoint{1.359439in}{0.904282in}}{\pgfqpoint{1.363557in}{0.908401in}}%
\pgfpathcurveto{\pgfqpoint{1.367676in}{0.912519in}}{\pgfqpoint{1.369989in}{0.918105in}}{\pgfqpoint{1.369989in}{0.923929in}}%
\pgfpathcurveto{\pgfqpoint{1.369989in}{0.929753in}}{\pgfqpoint{1.367676in}{0.935339in}}{\pgfqpoint{1.363557in}{0.939457in}}%
\pgfpathcurveto{\pgfqpoint{1.359439in}{0.943575in}}{\pgfqpoint{1.353853in}{0.945889in}}{\pgfqpoint{1.348029in}{0.945889in}}%
\pgfpathcurveto{\pgfqpoint{1.342205in}{0.945889in}}{\pgfqpoint{1.336619in}{0.943575in}}{\pgfqpoint{1.332501in}{0.939457in}}%
\pgfpathcurveto{\pgfqpoint{1.328383in}{0.935339in}}{\pgfqpoint{1.326069in}{0.929753in}}{\pgfqpoint{1.326069in}{0.923929in}}%
\pgfpathcurveto{\pgfqpoint{1.326069in}{0.918105in}}{\pgfqpoint{1.328383in}{0.912519in}}{\pgfqpoint{1.332501in}{0.908401in}}%
\pgfpathcurveto{\pgfqpoint{1.336619in}{0.904282in}}{\pgfqpoint{1.342205in}{0.901968in}}{\pgfqpoint{1.348029in}{0.901968in}}%
\pgfpathclose%
\pgfusepath{stroke,fill}%
\end{pgfscope}%
\begin{pgfscope}%
\pgfpathrectangle{\pgfqpoint{0.211875in}{0.211875in}}{\pgfqpoint{1.313625in}{1.279725in}}%
\pgfusepath{clip}%
\pgfsetbuttcap%
\pgfsetroundjoin%
\definecolor{currentfill}{rgb}{0.121569,0.466667,0.705882}%
\pgfsetfillcolor{currentfill}%
\pgfsetlinewidth{1.003750pt}%
\definecolor{currentstroke}{rgb}{0.121569,0.466667,0.705882}%
\pgfsetstrokecolor{currentstroke}%
\pgfsetdash{}{0pt}%
\pgfpathmoveto{\pgfqpoint{1.046401in}{0.268950in}}%
\pgfpathcurveto{\pgfqpoint{1.052225in}{0.268950in}}{\pgfqpoint{1.057811in}{0.271264in}}{\pgfqpoint{1.061929in}{0.275382in}}%
\pgfpathcurveto{\pgfqpoint{1.066047in}{0.279500in}}{\pgfqpoint{1.068361in}{0.285086in}}{\pgfqpoint{1.068361in}{0.290910in}}%
\pgfpathcurveto{\pgfqpoint{1.068361in}{0.296734in}}{\pgfqpoint{1.066047in}{0.302320in}}{\pgfqpoint{1.061929in}{0.306438in}}%
\pgfpathcurveto{\pgfqpoint{1.057811in}{0.310556in}}{\pgfqpoint{1.052225in}{0.312870in}}{\pgfqpoint{1.046401in}{0.312870in}}%
\pgfpathcurveto{\pgfqpoint{1.040577in}{0.312870in}}{\pgfqpoint{1.034991in}{0.310556in}}{\pgfqpoint{1.030872in}{0.306438in}}%
\pgfpathcurveto{\pgfqpoint{1.026754in}{0.302320in}}{\pgfqpoint{1.024440in}{0.296734in}}{\pgfqpoint{1.024440in}{0.290910in}}%
\pgfpathcurveto{\pgfqpoint{1.024440in}{0.285086in}}{\pgfqpoint{1.026754in}{0.279500in}}{\pgfqpoint{1.030872in}{0.275382in}}%
\pgfpathcurveto{\pgfqpoint{1.034991in}{0.271264in}}{\pgfqpoint{1.040577in}{0.268950in}}{\pgfqpoint{1.046401in}{0.268950in}}%
\pgfpathclose%
\pgfusepath{stroke,fill}%
\end{pgfscope}%
\begin{pgfscope}%
\pgfpathrectangle{\pgfqpoint{0.211875in}{0.211875in}}{\pgfqpoint{1.313625in}{1.279725in}}%
\pgfusepath{clip}%
\pgfsetbuttcap%
\pgfsetroundjoin%
\definecolor{currentfill}{rgb}{0.121569,0.466667,0.705882}%
\pgfsetfillcolor{currentfill}%
\pgfsetlinewidth{1.003750pt}%
\definecolor{currentstroke}{rgb}{0.121569,0.466667,0.705882}%
\pgfsetstrokecolor{currentstroke}%
\pgfsetdash{}{0pt}%
\pgfpathmoveto{\pgfqpoint{0.920989in}{0.937822in}}%
\pgfpathcurveto{\pgfqpoint{0.926813in}{0.937822in}}{\pgfqpoint{0.932399in}{0.940136in}}{\pgfqpoint{0.936517in}{0.944254in}}%
\pgfpathcurveto{\pgfqpoint{0.940635in}{0.948372in}}{\pgfqpoint{0.942949in}{0.953958in}}{\pgfqpoint{0.942949in}{0.959782in}}%
\pgfpathcurveto{\pgfqpoint{0.942949in}{0.965606in}}{\pgfqpoint{0.940635in}{0.971192in}}{\pgfqpoint{0.936517in}{0.975310in}}%
\pgfpathcurveto{\pgfqpoint{0.932399in}{0.979428in}}{\pgfqpoint{0.926813in}{0.981742in}}{\pgfqpoint{0.920989in}{0.981742in}}%
\pgfpathcurveto{\pgfqpoint{0.915165in}{0.981742in}}{\pgfqpoint{0.909579in}{0.979428in}}{\pgfqpoint{0.905461in}{0.975310in}}%
\pgfpathcurveto{\pgfqpoint{0.901342in}{0.971192in}}{\pgfqpoint{0.899029in}{0.965606in}}{\pgfqpoint{0.899029in}{0.959782in}}%
\pgfpathcurveto{\pgfqpoint{0.899029in}{0.953958in}}{\pgfqpoint{0.901342in}{0.948372in}}{\pgfqpoint{0.905461in}{0.944254in}}%
\pgfpathcurveto{\pgfqpoint{0.909579in}{0.940136in}}{\pgfqpoint{0.915165in}{0.937822in}}{\pgfqpoint{0.920989in}{0.937822in}}%
\pgfpathclose%
\pgfusepath{stroke,fill}%
\end{pgfscope}%
\begin{pgfscope}%
\pgfpathrectangle{\pgfqpoint{0.211875in}{0.211875in}}{\pgfqpoint{1.313625in}{1.279725in}}%
\pgfusepath{clip}%
\pgfsetbuttcap%
\pgfsetroundjoin%
\definecolor{currentfill}{rgb}{0.121569,0.466667,0.705882}%
\pgfsetfillcolor{currentfill}%
\pgfsetlinewidth{1.003750pt}%
\definecolor{currentstroke}{rgb}{0.121569,0.466667,0.705882}%
\pgfsetstrokecolor{currentstroke}%
\pgfsetdash{}{0pt}%
\pgfpathmoveto{\pgfqpoint{1.128655in}{1.063689in}}%
\pgfpathcurveto{\pgfqpoint{1.134479in}{1.063689in}}{\pgfqpoint{1.140065in}{1.066003in}}{\pgfqpoint{1.144183in}{1.070121in}}%
\pgfpathcurveto{\pgfqpoint{1.148301in}{1.074239in}}{\pgfqpoint{1.150615in}{1.079825in}}{\pgfqpoint{1.150615in}{1.085649in}}%
\pgfpathcurveto{\pgfqpoint{1.150615in}{1.091473in}}{\pgfqpoint{1.148301in}{1.097059in}}{\pgfqpoint{1.144183in}{1.101177in}}%
\pgfpathcurveto{\pgfqpoint{1.140065in}{1.105296in}}{\pgfqpoint{1.134479in}{1.107609in}}{\pgfqpoint{1.128655in}{1.107609in}}%
\pgfpathcurveto{\pgfqpoint{1.122831in}{1.107609in}}{\pgfqpoint{1.117245in}{1.105296in}}{\pgfqpoint{1.113126in}{1.101177in}}%
\pgfpathcurveto{\pgfqpoint{1.109008in}{1.097059in}}{\pgfqpoint{1.106694in}{1.091473in}}{\pgfqpoint{1.106694in}{1.085649in}}%
\pgfpathcurveto{\pgfqpoint{1.106694in}{1.079825in}}{\pgfqpoint{1.109008in}{1.074239in}}{\pgfqpoint{1.113126in}{1.070121in}}%
\pgfpathcurveto{\pgfqpoint{1.117245in}{1.066003in}}{\pgfqpoint{1.122831in}{1.063689in}}{\pgfqpoint{1.128655in}{1.063689in}}%
\pgfpathclose%
\pgfusepath{stroke,fill}%
\end{pgfscope}%
\begin{pgfscope}%
\pgfpathrectangle{\pgfqpoint{0.211875in}{0.211875in}}{\pgfqpoint{1.313625in}{1.279725in}}%
\pgfusepath{clip}%
\pgfsetbuttcap%
\pgfsetroundjoin%
\definecolor{currentfill}{rgb}{0.121569,0.466667,0.705882}%
\pgfsetfillcolor{currentfill}%
\pgfsetlinewidth{1.003750pt}%
\definecolor{currentstroke}{rgb}{0.121569,0.466667,0.705882}%
\pgfsetstrokecolor{currentstroke}%
\pgfsetdash{}{0pt}%
\pgfpathmoveto{\pgfqpoint{0.779002in}{1.070197in}}%
\pgfpathcurveto{\pgfqpoint{0.784826in}{1.070197in}}{\pgfqpoint{0.790412in}{1.072511in}}{\pgfqpoint{0.794530in}{1.076629in}}%
\pgfpathcurveto{\pgfqpoint{0.798648in}{1.080747in}}{\pgfqpoint{0.800962in}{1.086333in}}{\pgfqpoint{0.800962in}{1.092157in}}%
\pgfpathcurveto{\pgfqpoint{0.800962in}{1.097981in}}{\pgfqpoint{0.798648in}{1.103567in}}{\pgfqpoint{0.794530in}{1.107685in}}%
\pgfpathcurveto{\pgfqpoint{0.790412in}{1.111803in}}{\pgfqpoint{0.784826in}{1.114117in}}{\pgfqpoint{0.779002in}{1.114117in}}%
\pgfpathcurveto{\pgfqpoint{0.773178in}{1.114117in}}{\pgfqpoint{0.767592in}{1.111803in}}{\pgfqpoint{0.763474in}{1.107685in}}%
\pgfpathcurveto{\pgfqpoint{0.759355in}{1.103567in}}{\pgfqpoint{0.757042in}{1.097981in}}{\pgfqpoint{0.757042in}{1.092157in}}%
\pgfpathcurveto{\pgfqpoint{0.757042in}{1.086333in}}{\pgfqpoint{0.759355in}{1.080747in}}{\pgfqpoint{0.763474in}{1.076629in}}%
\pgfpathcurveto{\pgfqpoint{0.767592in}{1.072511in}}{\pgfqpoint{0.773178in}{1.070197in}}{\pgfqpoint{0.779002in}{1.070197in}}%
\pgfpathclose%
\pgfusepath{stroke,fill}%
\end{pgfscope}%
\begin{pgfscope}%
\pgfpathrectangle{\pgfqpoint{0.211875in}{0.211875in}}{\pgfqpoint{1.313625in}{1.279725in}}%
\pgfusepath{clip}%
\pgfsetbuttcap%
\pgfsetroundjoin%
\definecolor{currentfill}{rgb}{0.121569,0.466667,0.705882}%
\pgfsetfillcolor{currentfill}%
\pgfsetlinewidth{1.003750pt}%
\definecolor{currentstroke}{rgb}{0.121569,0.466667,0.705882}%
\pgfsetstrokecolor{currentstroke}%
\pgfsetdash{}{0pt}%
\pgfpathmoveto{\pgfqpoint{1.268085in}{0.872902in}}%
\pgfpathcurveto{\pgfqpoint{1.273909in}{0.872902in}}{\pgfqpoint{1.279495in}{0.875216in}}{\pgfqpoint{1.283613in}{0.879334in}}%
\pgfpathcurveto{\pgfqpoint{1.287731in}{0.883452in}}{\pgfqpoint{1.290045in}{0.889038in}}{\pgfqpoint{1.290045in}{0.894862in}}%
\pgfpathcurveto{\pgfqpoint{1.290045in}{0.900686in}}{\pgfqpoint{1.287731in}{0.906272in}}{\pgfqpoint{1.283613in}{0.910390in}}%
\pgfpathcurveto{\pgfqpoint{1.279495in}{0.914509in}}{\pgfqpoint{1.273909in}{0.916822in}}{\pgfqpoint{1.268085in}{0.916822in}}%
\pgfpathcurveto{\pgfqpoint{1.262261in}{0.916822in}}{\pgfqpoint{1.256675in}{0.914509in}}{\pgfqpoint{1.252557in}{0.910390in}}%
\pgfpathcurveto{\pgfqpoint{1.248438in}{0.906272in}}{\pgfqpoint{1.246125in}{0.900686in}}{\pgfqpoint{1.246125in}{0.894862in}}%
\pgfpathcurveto{\pgfqpoint{1.246125in}{0.889038in}}{\pgfqpoint{1.248438in}{0.883452in}}{\pgfqpoint{1.252557in}{0.879334in}}%
\pgfpathcurveto{\pgfqpoint{1.256675in}{0.875216in}}{\pgfqpoint{1.262261in}{0.872902in}}{\pgfqpoint{1.268085in}{0.872902in}}%
\pgfpathclose%
\pgfusepath{stroke,fill}%
\end{pgfscope}%
\begin{pgfscope}%
\pgfpathrectangle{\pgfqpoint{0.211875in}{0.211875in}}{\pgfqpoint{1.313625in}{1.279725in}}%
\pgfusepath{clip}%
\pgfsetbuttcap%
\pgfsetroundjoin%
\definecolor{currentfill}{rgb}{0.121569,0.466667,0.705882}%
\pgfsetfillcolor{currentfill}%
\pgfsetlinewidth{1.003750pt}%
\definecolor{currentstroke}{rgb}{0.121569,0.466667,0.705882}%
\pgfsetstrokecolor{currentstroke}%
\pgfsetdash{}{0pt}%
\pgfpathmoveto{\pgfqpoint{0.857409in}{1.021143in}}%
\pgfpathcurveto{\pgfqpoint{0.863233in}{1.021143in}}{\pgfqpoint{0.868819in}{1.023457in}}{\pgfqpoint{0.872938in}{1.027576in}}%
\pgfpathcurveto{\pgfqpoint{0.877056in}{1.031694in}}{\pgfqpoint{0.879370in}{1.037280in}}{\pgfqpoint{0.879370in}{1.043104in}}%
\pgfpathcurveto{\pgfqpoint{0.879370in}{1.048928in}}{\pgfqpoint{0.877056in}{1.054514in}}{\pgfqpoint{0.872938in}{1.058632in}}%
\pgfpathcurveto{\pgfqpoint{0.868819in}{1.062750in}}{\pgfqpoint{0.863233in}{1.065064in}}{\pgfqpoint{0.857409in}{1.065064in}}%
\pgfpathcurveto{\pgfqpoint{0.851585in}{1.065064in}}{\pgfqpoint{0.845999in}{1.062750in}}{\pgfqpoint{0.841881in}{1.058632in}}%
\pgfpathcurveto{\pgfqpoint{0.837763in}{1.054514in}}{\pgfqpoint{0.835449in}{1.048928in}}{\pgfqpoint{0.835449in}{1.043104in}}%
\pgfpathcurveto{\pgfqpoint{0.835449in}{1.037280in}}{\pgfqpoint{0.837763in}{1.031694in}}{\pgfqpoint{0.841881in}{1.027576in}}%
\pgfpathcurveto{\pgfqpoint{0.845999in}{1.023457in}}{\pgfqpoint{0.851585in}{1.021143in}}{\pgfqpoint{0.857409in}{1.021143in}}%
\pgfpathclose%
\pgfusepath{stroke,fill}%
\end{pgfscope}%
\begin{pgfscope}%
\pgfpathrectangle{\pgfqpoint{0.211875in}{0.211875in}}{\pgfqpoint{1.313625in}{1.279725in}}%
\pgfusepath{clip}%
\pgfsetbuttcap%
\pgfsetroundjoin%
\definecolor{currentfill}{rgb}{0.121569,0.466667,0.705882}%
\pgfsetfillcolor{currentfill}%
\pgfsetlinewidth{1.003750pt}%
\definecolor{currentstroke}{rgb}{0.121569,0.466667,0.705882}%
\pgfsetstrokecolor{currentstroke}%
\pgfsetdash{}{0pt}%
\pgfpathmoveto{\pgfqpoint{0.918956in}{0.927512in}}%
\pgfpathcurveto{\pgfqpoint{0.924780in}{0.927512in}}{\pgfqpoint{0.930366in}{0.929825in}}{\pgfqpoint{0.934485in}{0.933944in}}%
\pgfpathcurveto{\pgfqpoint{0.938603in}{0.938062in}}{\pgfqpoint{0.940917in}{0.943648in}}{\pgfqpoint{0.940917in}{0.949472in}}%
\pgfpathcurveto{\pgfqpoint{0.940917in}{0.955296in}}{\pgfqpoint{0.938603in}{0.960882in}}{\pgfqpoint{0.934485in}{0.965000in}}%
\pgfpathcurveto{\pgfqpoint{0.930366in}{0.969118in}}{\pgfqpoint{0.924780in}{0.971432in}}{\pgfqpoint{0.918956in}{0.971432in}}%
\pgfpathcurveto{\pgfqpoint{0.913132in}{0.971432in}}{\pgfqpoint{0.907546in}{0.969118in}}{\pgfqpoint{0.903428in}{0.965000in}}%
\pgfpathcurveto{\pgfqpoint{0.899310in}{0.960882in}}{\pgfqpoint{0.896996in}{0.955296in}}{\pgfqpoint{0.896996in}{0.949472in}}%
\pgfpathcurveto{\pgfqpoint{0.896996in}{0.943648in}}{\pgfqpoint{0.899310in}{0.938062in}}{\pgfqpoint{0.903428in}{0.933944in}}%
\pgfpathcurveto{\pgfqpoint{0.907546in}{0.929825in}}{\pgfqpoint{0.913132in}{0.927512in}}{\pgfqpoint{0.918956in}{0.927512in}}%
\pgfpathclose%
\pgfusepath{stroke,fill}%
\end{pgfscope}%
\begin{pgfscope}%
\pgfpathrectangle{\pgfqpoint{0.211875in}{0.211875in}}{\pgfqpoint{1.313625in}{1.279725in}}%
\pgfusepath{clip}%
\pgfsetbuttcap%
\pgfsetroundjoin%
\definecolor{currentfill}{rgb}{0.121569,0.466667,0.705882}%
\pgfsetfillcolor{currentfill}%
\pgfsetlinewidth{1.003750pt}%
\definecolor{currentstroke}{rgb}{0.121569,0.466667,0.705882}%
\pgfsetstrokecolor{currentstroke}%
\pgfsetdash{}{0pt}%
\pgfpathmoveto{\pgfqpoint{0.892464in}{0.966484in}}%
\pgfpathcurveto{\pgfqpoint{0.898288in}{0.966484in}}{\pgfqpoint{0.903874in}{0.968798in}}{\pgfqpoint{0.907992in}{0.972916in}}%
\pgfpathcurveto{\pgfqpoint{0.912110in}{0.977034in}}{\pgfqpoint{0.914424in}{0.982620in}}{\pgfqpoint{0.914424in}{0.988444in}}%
\pgfpathcurveto{\pgfqpoint{0.914424in}{0.994268in}}{\pgfqpoint{0.912110in}{0.999854in}}{\pgfqpoint{0.907992in}{1.003972in}}%
\pgfpathcurveto{\pgfqpoint{0.903874in}{1.008091in}}{\pgfqpoint{0.898288in}{1.010404in}}{\pgfqpoint{0.892464in}{1.010404in}}%
\pgfpathcurveto{\pgfqpoint{0.886640in}{1.010404in}}{\pgfqpoint{0.881054in}{1.008091in}}{\pgfqpoint{0.876936in}{1.003972in}}%
\pgfpathcurveto{\pgfqpoint{0.872817in}{0.999854in}}{\pgfqpoint{0.870504in}{0.994268in}}{\pgfqpoint{0.870504in}{0.988444in}}%
\pgfpathcurveto{\pgfqpoint{0.870504in}{0.982620in}}{\pgfqpoint{0.872817in}{0.977034in}}{\pgfqpoint{0.876936in}{0.972916in}}%
\pgfpathcurveto{\pgfqpoint{0.881054in}{0.968798in}}{\pgfqpoint{0.886640in}{0.966484in}}{\pgfqpoint{0.892464in}{0.966484in}}%
\pgfpathclose%
\pgfusepath{stroke,fill}%
\end{pgfscope}%
\begin{pgfscope}%
\pgfpathrectangle{\pgfqpoint{0.211875in}{0.211875in}}{\pgfqpoint{1.313625in}{1.279725in}}%
\pgfusepath{clip}%
\pgfsetbuttcap%
\pgfsetroundjoin%
\definecolor{currentfill}{rgb}{0.121569,0.466667,0.705882}%
\pgfsetfillcolor{currentfill}%
\pgfsetlinewidth{1.003750pt}%
\definecolor{currentstroke}{rgb}{0.121569,0.466667,0.705882}%
\pgfsetstrokecolor{currentstroke}%
\pgfsetdash{}{0pt}%
\pgfpathmoveto{\pgfqpoint{1.116209in}{1.078039in}}%
\pgfpathcurveto{\pgfqpoint{1.122033in}{1.078039in}}{\pgfqpoint{1.127619in}{1.080353in}}{\pgfqpoint{1.131737in}{1.084471in}}%
\pgfpathcurveto{\pgfqpoint{1.135855in}{1.088589in}}{\pgfqpoint{1.138169in}{1.094175in}}{\pgfqpoint{1.138169in}{1.099999in}}%
\pgfpathcurveto{\pgfqpoint{1.138169in}{1.105823in}}{\pgfqpoint{1.135855in}{1.111409in}}{\pgfqpoint{1.131737in}{1.115528in}}%
\pgfpathcurveto{\pgfqpoint{1.127619in}{1.119646in}}{\pgfqpoint{1.122033in}{1.121960in}}{\pgfqpoint{1.116209in}{1.121960in}}%
\pgfpathcurveto{\pgfqpoint{1.110385in}{1.121960in}}{\pgfqpoint{1.104799in}{1.119646in}}{\pgfqpoint{1.100680in}{1.115528in}}%
\pgfpathcurveto{\pgfqpoint{1.096562in}{1.111409in}}{\pgfqpoint{1.094248in}{1.105823in}}{\pgfqpoint{1.094248in}{1.099999in}}%
\pgfpathcurveto{\pgfqpoint{1.094248in}{1.094175in}}{\pgfqpoint{1.096562in}{1.088589in}}{\pgfqpoint{1.100680in}{1.084471in}}%
\pgfpathcurveto{\pgfqpoint{1.104799in}{1.080353in}}{\pgfqpoint{1.110385in}{1.078039in}}{\pgfqpoint{1.116209in}{1.078039in}}%
\pgfpathclose%
\pgfusepath{stroke,fill}%
\end{pgfscope}%
\begin{pgfscope}%
\pgfpathrectangle{\pgfqpoint{0.211875in}{0.211875in}}{\pgfqpoint{1.313625in}{1.279725in}}%
\pgfusepath{clip}%
\pgfsetbuttcap%
\pgfsetroundjoin%
\definecolor{currentfill}{rgb}{0.121569,0.466667,0.705882}%
\pgfsetfillcolor{currentfill}%
\pgfsetlinewidth{1.003750pt}%
\definecolor{currentstroke}{rgb}{0.121569,0.466667,0.705882}%
\pgfsetstrokecolor{currentstroke}%
\pgfsetdash{}{0pt}%
\pgfpathmoveto{\pgfqpoint{0.942821in}{0.576683in}}%
\pgfpathcurveto{\pgfqpoint{0.948645in}{0.576683in}}{\pgfqpoint{0.954231in}{0.578997in}}{\pgfqpoint{0.958349in}{0.583115in}}%
\pgfpathcurveto{\pgfqpoint{0.962468in}{0.587233in}}{\pgfqpoint{0.964781in}{0.592819in}}{\pgfqpoint{0.964781in}{0.598643in}}%
\pgfpathcurveto{\pgfqpoint{0.964781in}{0.604467in}}{\pgfqpoint{0.962468in}{0.610053in}}{\pgfqpoint{0.958349in}{0.614171in}}%
\pgfpathcurveto{\pgfqpoint{0.954231in}{0.618289in}}{\pgfqpoint{0.948645in}{0.620603in}}{\pgfqpoint{0.942821in}{0.620603in}}%
\pgfpathcurveto{\pgfqpoint{0.936997in}{0.620603in}}{\pgfqpoint{0.931411in}{0.618289in}}{\pgfqpoint{0.927293in}{0.614171in}}%
\pgfpathcurveto{\pgfqpoint{0.923175in}{0.610053in}}{\pgfqpoint{0.920861in}{0.604467in}}{\pgfqpoint{0.920861in}{0.598643in}}%
\pgfpathcurveto{\pgfqpoint{0.920861in}{0.592819in}}{\pgfqpoint{0.923175in}{0.587233in}}{\pgfqpoint{0.927293in}{0.583115in}}%
\pgfpathcurveto{\pgfqpoint{0.931411in}{0.578997in}}{\pgfqpoint{0.936997in}{0.576683in}}{\pgfqpoint{0.942821in}{0.576683in}}%
\pgfpathclose%
\pgfusepath{stroke,fill}%
\end{pgfscope}%
\begin{pgfscope}%
\pgfpathrectangle{\pgfqpoint{0.211875in}{0.211875in}}{\pgfqpoint{1.313625in}{1.279725in}}%
\pgfusepath{clip}%
\pgfsetbuttcap%
\pgfsetroundjoin%
\definecolor{currentfill}{rgb}{0.121569,0.466667,0.705882}%
\pgfsetfillcolor{currentfill}%
\pgfsetlinewidth{1.003750pt}%
\definecolor{currentstroke}{rgb}{0.121569,0.466667,0.705882}%
\pgfsetstrokecolor{currentstroke}%
\pgfsetdash{}{0pt}%
\pgfpathmoveto{\pgfqpoint{1.314606in}{0.922244in}}%
\pgfpathcurveto{\pgfqpoint{1.320430in}{0.922244in}}{\pgfqpoint{1.326016in}{0.924558in}}{\pgfqpoint{1.330134in}{0.928676in}}%
\pgfpathcurveto{\pgfqpoint{1.334253in}{0.932794in}}{\pgfqpoint{1.336566in}{0.938380in}}{\pgfqpoint{1.336566in}{0.944204in}}%
\pgfpathcurveto{\pgfqpoint{1.336566in}{0.950028in}}{\pgfqpoint{1.334253in}{0.955614in}}{\pgfqpoint{1.330134in}{0.959732in}}%
\pgfpathcurveto{\pgfqpoint{1.326016in}{0.963850in}}{\pgfqpoint{1.320430in}{0.966164in}}{\pgfqpoint{1.314606in}{0.966164in}}%
\pgfpathcurveto{\pgfqpoint{1.308782in}{0.966164in}}{\pgfqpoint{1.303196in}{0.963850in}}{\pgfqpoint{1.299078in}{0.959732in}}%
\pgfpathcurveto{\pgfqpoint{1.294960in}{0.955614in}}{\pgfqpoint{1.292646in}{0.950028in}}{\pgfqpoint{1.292646in}{0.944204in}}%
\pgfpathcurveto{\pgfqpoint{1.292646in}{0.938380in}}{\pgfqpoint{1.294960in}{0.932794in}}{\pgfqpoint{1.299078in}{0.928676in}}%
\pgfpathcurveto{\pgfqpoint{1.303196in}{0.924558in}}{\pgfqpoint{1.308782in}{0.922244in}}{\pgfqpoint{1.314606in}{0.922244in}}%
\pgfpathclose%
\pgfusepath{stroke,fill}%
\end{pgfscope}%
\begin{pgfscope}%
\pgfpathrectangle{\pgfqpoint{0.211875in}{0.211875in}}{\pgfqpoint{1.313625in}{1.279725in}}%
\pgfusepath{clip}%
\pgfsetbuttcap%
\pgfsetroundjoin%
\definecolor{currentfill}{rgb}{0.121569,0.466667,0.705882}%
\pgfsetfillcolor{currentfill}%
\pgfsetlinewidth{1.003750pt}%
\definecolor{currentstroke}{rgb}{0.121569,0.466667,0.705882}%
\pgfsetstrokecolor{currentstroke}%
\pgfsetdash{}{0pt}%
\pgfpathmoveto{\pgfqpoint{0.659553in}{1.392085in}}%
\pgfpathcurveto{\pgfqpoint{0.665377in}{1.392085in}}{\pgfqpoint{0.670964in}{1.394399in}}{\pgfqpoint{0.675082in}{1.398517in}}%
\pgfpathcurveto{\pgfqpoint{0.679200in}{1.402635in}}{\pgfqpoint{0.681514in}{1.408222in}}{\pgfqpoint{0.681514in}{1.414045in}}%
\pgfpathcurveto{\pgfqpoint{0.681514in}{1.419869in}}{\pgfqpoint{0.679200in}{1.425456in}}{\pgfqpoint{0.675082in}{1.429574in}}%
\pgfpathcurveto{\pgfqpoint{0.670964in}{1.433692in}}{\pgfqpoint{0.665377in}{1.436006in}}{\pgfqpoint{0.659553in}{1.436006in}}%
\pgfpathcurveto{\pgfqpoint{0.653730in}{1.436006in}}{\pgfqpoint{0.648143in}{1.433692in}}{\pgfqpoint{0.644025in}{1.429574in}}%
\pgfpathcurveto{\pgfqpoint{0.639907in}{1.425456in}}{\pgfqpoint{0.637593in}{1.419869in}}{\pgfqpoint{0.637593in}{1.414045in}}%
\pgfpathcurveto{\pgfqpoint{0.637593in}{1.408222in}}{\pgfqpoint{0.639907in}{1.402635in}}{\pgfqpoint{0.644025in}{1.398517in}}%
\pgfpathcurveto{\pgfqpoint{0.648143in}{1.394399in}}{\pgfqpoint{0.653730in}{1.392085in}}{\pgfqpoint{0.659553in}{1.392085in}}%
\pgfpathclose%
\pgfusepath{stroke,fill}%
\end{pgfscope}%
\begin{pgfscope}%
\pgfpathrectangle{\pgfqpoint{0.211875in}{0.211875in}}{\pgfqpoint{1.313625in}{1.279725in}}%
\pgfusepath{clip}%
\pgfsetbuttcap%
\pgfsetroundjoin%
\definecolor{currentfill}{rgb}{0.121569,0.466667,0.705882}%
\pgfsetfillcolor{currentfill}%
\pgfsetlinewidth{1.003750pt}%
\definecolor{currentstroke}{rgb}{0.121569,0.466667,0.705882}%
\pgfsetstrokecolor{currentstroke}%
\pgfsetdash{}{0pt}%
\pgfpathmoveto{\pgfqpoint{1.101086in}{1.056748in}}%
\pgfpathcurveto{\pgfqpoint{1.106910in}{1.056748in}}{\pgfqpoint{1.112496in}{1.059062in}}{\pgfqpoint{1.116614in}{1.063180in}}%
\pgfpathcurveto{\pgfqpoint{1.120732in}{1.067298in}}{\pgfqpoint{1.123046in}{1.072884in}}{\pgfqpoint{1.123046in}{1.078708in}}%
\pgfpathcurveto{\pgfqpoint{1.123046in}{1.084532in}}{\pgfqpoint{1.120732in}{1.090118in}}{\pgfqpoint{1.116614in}{1.094236in}}%
\pgfpathcurveto{\pgfqpoint{1.112496in}{1.098355in}}{\pgfqpoint{1.106910in}{1.100668in}}{\pgfqpoint{1.101086in}{1.100668in}}%
\pgfpathcurveto{\pgfqpoint{1.095262in}{1.100668in}}{\pgfqpoint{1.089676in}{1.098355in}}{\pgfqpoint{1.085557in}{1.094236in}}%
\pgfpathcurveto{\pgfqpoint{1.081439in}{1.090118in}}{\pgfqpoint{1.079125in}{1.084532in}}{\pgfqpoint{1.079125in}{1.078708in}}%
\pgfpathcurveto{\pgfqpoint{1.079125in}{1.072884in}}{\pgfqpoint{1.081439in}{1.067298in}}{\pgfqpoint{1.085557in}{1.063180in}}%
\pgfpathcurveto{\pgfqpoint{1.089676in}{1.059062in}}{\pgfqpoint{1.095262in}{1.056748in}}{\pgfqpoint{1.101086in}{1.056748in}}%
\pgfpathclose%
\pgfusepath{stroke,fill}%
\end{pgfscope}%
\begin{pgfscope}%
\pgfpathrectangle{\pgfqpoint{0.211875in}{0.211875in}}{\pgfqpoint{1.313625in}{1.279725in}}%
\pgfusepath{clip}%
\pgfsetbuttcap%
\pgfsetroundjoin%
\definecolor{currentfill}{rgb}{0.121569,0.466667,0.705882}%
\pgfsetfillcolor{currentfill}%
\pgfsetlinewidth{1.003750pt}%
\definecolor{currentstroke}{rgb}{0.121569,0.466667,0.705882}%
\pgfsetstrokecolor{currentstroke}%
\pgfsetdash{}{0pt}%
\pgfpathmoveto{\pgfqpoint{0.928419in}{1.225099in}}%
\pgfpathcurveto{\pgfqpoint{0.934243in}{1.225099in}}{\pgfqpoint{0.939829in}{1.227413in}}{\pgfqpoint{0.943947in}{1.231531in}}%
\pgfpathcurveto{\pgfqpoint{0.948065in}{1.235650in}}{\pgfqpoint{0.950379in}{1.241236in}}{\pgfqpoint{0.950379in}{1.247060in}}%
\pgfpathcurveto{\pgfqpoint{0.950379in}{1.252884in}}{\pgfqpoint{0.948065in}{1.258470in}}{\pgfqpoint{0.943947in}{1.262588in}}%
\pgfpathcurveto{\pgfqpoint{0.939829in}{1.266706in}}{\pgfqpoint{0.934243in}{1.269020in}}{\pgfqpoint{0.928419in}{1.269020in}}%
\pgfpathcurveto{\pgfqpoint{0.922595in}{1.269020in}}{\pgfqpoint{0.917009in}{1.266706in}}{\pgfqpoint{0.912891in}{1.262588in}}%
\pgfpathcurveto{\pgfqpoint{0.908773in}{1.258470in}}{\pgfqpoint{0.906459in}{1.252884in}}{\pgfqpoint{0.906459in}{1.247060in}}%
\pgfpathcurveto{\pgfqpoint{0.906459in}{1.241236in}}{\pgfqpoint{0.908773in}{1.235650in}}{\pgfqpoint{0.912891in}{1.231531in}}%
\pgfpathcurveto{\pgfqpoint{0.917009in}{1.227413in}}{\pgfqpoint{0.922595in}{1.225099in}}{\pgfqpoint{0.928419in}{1.225099in}}%
\pgfpathclose%
\pgfusepath{stroke,fill}%
\end{pgfscope}%
\begin{pgfscope}%
\pgfpathrectangle{\pgfqpoint{0.211875in}{0.211875in}}{\pgfqpoint{1.313625in}{1.279725in}}%
\pgfusepath{clip}%
\pgfsetbuttcap%
\pgfsetroundjoin%
\definecolor{currentfill}{rgb}{0.121569,0.466667,0.705882}%
\pgfsetfillcolor{currentfill}%
\pgfsetlinewidth{1.003750pt}%
\definecolor{currentstroke}{rgb}{0.121569,0.466667,0.705882}%
\pgfsetstrokecolor{currentstroke}%
\pgfsetdash{}{0pt}%
\pgfpathmoveto{\pgfqpoint{0.827026in}{1.048625in}}%
\pgfpathcurveto{\pgfqpoint{0.832850in}{1.048625in}}{\pgfqpoint{0.838437in}{1.050939in}}{\pgfqpoint{0.842555in}{1.055057in}}%
\pgfpathcurveto{\pgfqpoint{0.846673in}{1.059175in}}{\pgfqpoint{0.848987in}{1.064761in}}{\pgfqpoint{0.848987in}{1.070585in}}%
\pgfpathcurveto{\pgfqpoint{0.848987in}{1.076409in}}{\pgfqpoint{0.846673in}{1.081995in}}{\pgfqpoint{0.842555in}{1.086113in}}%
\pgfpathcurveto{\pgfqpoint{0.838437in}{1.090232in}}{\pgfqpoint{0.832850in}{1.092545in}}{\pgfqpoint{0.827026in}{1.092545in}}%
\pgfpathcurveto{\pgfqpoint{0.821203in}{1.092545in}}{\pgfqpoint{0.815616in}{1.090232in}}{\pgfqpoint{0.811498in}{1.086113in}}%
\pgfpathcurveto{\pgfqpoint{0.807380in}{1.081995in}}{\pgfqpoint{0.805066in}{1.076409in}}{\pgfqpoint{0.805066in}{1.070585in}}%
\pgfpathcurveto{\pgfqpoint{0.805066in}{1.064761in}}{\pgfqpoint{0.807380in}{1.059175in}}{\pgfqpoint{0.811498in}{1.055057in}}%
\pgfpathcurveto{\pgfqpoint{0.815616in}{1.050939in}}{\pgfqpoint{0.821203in}{1.048625in}}{\pgfqpoint{0.827026in}{1.048625in}}%
\pgfpathclose%
\pgfusepath{stroke,fill}%
\end{pgfscope}%
\begin{pgfscope}%
\pgfpathrectangle{\pgfqpoint{0.211875in}{0.211875in}}{\pgfqpoint{1.313625in}{1.279725in}}%
\pgfusepath{clip}%
\pgfsetbuttcap%
\pgfsetroundjoin%
\definecolor{currentfill}{rgb}{0.121569,0.466667,0.705882}%
\pgfsetfillcolor{currentfill}%
\pgfsetlinewidth{1.003750pt}%
\definecolor{currentstroke}{rgb}{0.121569,0.466667,0.705882}%
\pgfsetstrokecolor{currentstroke}%
\pgfsetdash{}{0pt}%
\pgfpathmoveto{\pgfqpoint{1.330117in}{0.945198in}}%
\pgfpathcurveto{\pgfqpoint{1.335941in}{0.945198in}}{\pgfqpoint{1.341527in}{0.947512in}}{\pgfqpoint{1.345645in}{0.951630in}}%
\pgfpathcurveto{\pgfqpoint{1.349763in}{0.955748in}}{\pgfqpoint{1.352077in}{0.961334in}}{\pgfqpoint{1.352077in}{0.967158in}}%
\pgfpathcurveto{\pgfqpoint{1.352077in}{0.972982in}}{\pgfqpoint{1.349763in}{0.978568in}}{\pgfqpoint{1.345645in}{0.982686in}}%
\pgfpathcurveto{\pgfqpoint{1.341527in}{0.986804in}}{\pgfqpoint{1.335941in}{0.989118in}}{\pgfqpoint{1.330117in}{0.989118in}}%
\pgfpathcurveto{\pgfqpoint{1.324293in}{0.989118in}}{\pgfqpoint{1.318707in}{0.986804in}}{\pgfqpoint{1.314588in}{0.982686in}}%
\pgfpathcurveto{\pgfqpoint{1.310470in}{0.978568in}}{\pgfqpoint{1.308156in}{0.972982in}}{\pgfqpoint{1.308156in}{0.967158in}}%
\pgfpathcurveto{\pgfqpoint{1.308156in}{0.961334in}}{\pgfqpoint{1.310470in}{0.955748in}}{\pgfqpoint{1.314588in}{0.951630in}}%
\pgfpathcurveto{\pgfqpoint{1.318707in}{0.947512in}}{\pgfqpoint{1.324293in}{0.945198in}}{\pgfqpoint{1.330117in}{0.945198in}}%
\pgfpathclose%
\pgfusepath{stroke,fill}%
\end{pgfscope}%
\begin{pgfscope}%
\pgfpathrectangle{\pgfqpoint{0.211875in}{0.211875in}}{\pgfqpoint{1.313625in}{1.279725in}}%
\pgfusepath{clip}%
\pgfsetbuttcap%
\pgfsetroundjoin%
\definecolor{currentfill}{rgb}{0.121569,0.466667,0.705882}%
\pgfsetfillcolor{currentfill}%
\pgfsetlinewidth{1.003750pt}%
\definecolor{currentstroke}{rgb}{0.121569,0.466667,0.705882}%
\pgfsetstrokecolor{currentstroke}%
\pgfsetdash{}{0pt}%
\pgfpathmoveto{\pgfqpoint{1.307568in}{0.922177in}}%
\pgfpathcurveto{\pgfqpoint{1.313392in}{0.922177in}}{\pgfqpoint{1.318979in}{0.924491in}}{\pgfqpoint{1.323097in}{0.928609in}}%
\pgfpathcurveto{\pgfqpoint{1.327215in}{0.932727in}}{\pgfqpoint{1.329529in}{0.938313in}}{\pgfqpoint{1.329529in}{0.944137in}}%
\pgfpathcurveto{\pgfqpoint{1.329529in}{0.949961in}}{\pgfqpoint{1.327215in}{0.955547in}}{\pgfqpoint{1.323097in}{0.959665in}}%
\pgfpathcurveto{\pgfqpoint{1.318979in}{0.963783in}}{\pgfqpoint{1.313392in}{0.966097in}}{\pgfqpoint{1.307568in}{0.966097in}}%
\pgfpathcurveto{\pgfqpoint{1.301745in}{0.966097in}}{\pgfqpoint{1.296158in}{0.963783in}}{\pgfqpoint{1.292040in}{0.959665in}}%
\pgfpathcurveto{\pgfqpoint{1.287922in}{0.955547in}}{\pgfqpoint{1.285608in}{0.949961in}}{\pgfqpoint{1.285608in}{0.944137in}}%
\pgfpathcurveto{\pgfqpoint{1.285608in}{0.938313in}}{\pgfqpoint{1.287922in}{0.932727in}}{\pgfqpoint{1.292040in}{0.928609in}}%
\pgfpathcurveto{\pgfqpoint{1.296158in}{0.924491in}}{\pgfqpoint{1.301745in}{0.922177in}}{\pgfqpoint{1.307568in}{0.922177in}}%
\pgfpathclose%
\pgfusepath{stroke,fill}%
\end{pgfscope}%
\begin{pgfscope}%
\pgfpathrectangle{\pgfqpoint{0.211875in}{0.211875in}}{\pgfqpoint{1.313625in}{1.279725in}}%
\pgfusepath{clip}%
\pgfsetbuttcap%
\pgfsetroundjoin%
\definecolor{currentfill}{rgb}{0.121569,0.466667,0.705882}%
\pgfsetfillcolor{currentfill}%
\pgfsetlinewidth{1.003750pt}%
\definecolor{currentstroke}{rgb}{0.121569,0.466667,0.705882}%
\pgfsetstrokecolor{currentstroke}%
\pgfsetdash{}{0pt}%
\pgfpathmoveto{\pgfqpoint{0.805949in}{1.075714in}}%
\pgfpathcurveto{\pgfqpoint{0.811772in}{1.075714in}}{\pgfqpoint{0.817359in}{1.078027in}}{\pgfqpoint{0.821477in}{1.082146in}}%
\pgfpathcurveto{\pgfqpoint{0.825595in}{1.086264in}}{\pgfqpoint{0.827909in}{1.091850in}}{\pgfqpoint{0.827909in}{1.097674in}}%
\pgfpathcurveto{\pgfqpoint{0.827909in}{1.103498in}}{\pgfqpoint{0.825595in}{1.109084in}}{\pgfqpoint{0.821477in}{1.113202in}}%
\pgfpathcurveto{\pgfqpoint{0.817359in}{1.117320in}}{\pgfqpoint{0.811772in}{1.119634in}}{\pgfqpoint{0.805949in}{1.119634in}}%
\pgfpathcurveto{\pgfqpoint{0.800125in}{1.119634in}}{\pgfqpoint{0.794538in}{1.117320in}}{\pgfqpoint{0.790420in}{1.113202in}}%
\pgfpathcurveto{\pgfqpoint{0.786302in}{1.109084in}}{\pgfqpoint{0.783988in}{1.103498in}}{\pgfqpoint{0.783988in}{1.097674in}}%
\pgfpathcurveto{\pgfqpoint{0.783988in}{1.091850in}}{\pgfqpoint{0.786302in}{1.086264in}}{\pgfqpoint{0.790420in}{1.082146in}}%
\pgfpathcurveto{\pgfqpoint{0.794538in}{1.078027in}}{\pgfqpoint{0.800125in}{1.075714in}}{\pgfqpoint{0.805949in}{1.075714in}}%
\pgfpathclose%
\pgfusepath{stroke,fill}%
\end{pgfscope}%
\begin{pgfscope}%
\pgfpathrectangle{\pgfqpoint{0.211875in}{0.211875in}}{\pgfqpoint{1.313625in}{1.279725in}}%
\pgfusepath{clip}%
\pgfsetbuttcap%
\pgfsetroundjoin%
\definecolor{currentfill}{rgb}{0.121569,0.466667,0.705882}%
\pgfsetfillcolor{currentfill}%
\pgfsetlinewidth{1.003750pt}%
\definecolor{currentstroke}{rgb}{0.121569,0.466667,0.705882}%
\pgfsetstrokecolor{currentstroke}%
\pgfsetdash{}{0pt}%
\pgfpathmoveto{\pgfqpoint{1.117840in}{1.078433in}}%
\pgfpathcurveto{\pgfqpoint{1.123664in}{1.078433in}}{\pgfqpoint{1.129250in}{1.080747in}}{\pgfqpoint{1.133368in}{1.084865in}}%
\pgfpathcurveto{\pgfqpoint{1.137486in}{1.088983in}}{\pgfqpoint{1.139800in}{1.094569in}}{\pgfqpoint{1.139800in}{1.100393in}}%
\pgfpathcurveto{\pgfqpoint{1.139800in}{1.106217in}}{\pgfqpoint{1.137486in}{1.111803in}}{\pgfqpoint{1.133368in}{1.115921in}}%
\pgfpathcurveto{\pgfqpoint{1.129250in}{1.120040in}}{\pgfqpoint{1.123664in}{1.122353in}}{\pgfqpoint{1.117840in}{1.122353in}}%
\pgfpathcurveto{\pgfqpoint{1.112016in}{1.122353in}}{\pgfqpoint{1.106430in}{1.120040in}}{\pgfqpoint{1.102312in}{1.115921in}}%
\pgfpathcurveto{\pgfqpoint{1.098194in}{1.111803in}}{\pgfqpoint{1.095880in}{1.106217in}}{\pgfqpoint{1.095880in}{1.100393in}}%
\pgfpathcurveto{\pgfqpoint{1.095880in}{1.094569in}}{\pgfqpoint{1.098194in}{1.088983in}}{\pgfqpoint{1.102312in}{1.084865in}}%
\pgfpathcurveto{\pgfqpoint{1.106430in}{1.080747in}}{\pgfqpoint{1.112016in}{1.078433in}}{\pgfqpoint{1.117840in}{1.078433in}}%
\pgfpathclose%
\pgfusepath{stroke,fill}%
\end{pgfscope}%
\begin{pgfscope}%
\pgfpathrectangle{\pgfqpoint{0.211875in}{0.211875in}}{\pgfqpoint{1.313625in}{1.279725in}}%
\pgfusepath{clip}%
\pgfsetbuttcap%
\pgfsetroundjoin%
\definecolor{currentfill}{rgb}{0.121569,0.466667,0.705882}%
\pgfsetfillcolor{currentfill}%
\pgfsetlinewidth{1.003750pt}%
\definecolor{currentstroke}{rgb}{0.121569,0.466667,0.705882}%
\pgfsetstrokecolor{currentstroke}%
\pgfsetdash{}{0pt}%
\pgfpathmoveto{\pgfqpoint{0.943047in}{0.597274in}}%
\pgfpathcurveto{\pgfqpoint{0.948871in}{0.597274in}}{\pgfqpoint{0.954457in}{0.599587in}}{\pgfqpoint{0.958575in}{0.603706in}}%
\pgfpathcurveto{\pgfqpoint{0.962693in}{0.607824in}}{\pgfqpoint{0.965007in}{0.613410in}}{\pgfqpoint{0.965007in}{0.619234in}}%
\pgfpathcurveto{\pgfqpoint{0.965007in}{0.625058in}}{\pgfqpoint{0.962693in}{0.630644in}}{\pgfqpoint{0.958575in}{0.634762in}}%
\pgfpathcurveto{\pgfqpoint{0.954457in}{0.638880in}}{\pgfqpoint{0.948871in}{0.641194in}}{\pgfqpoint{0.943047in}{0.641194in}}%
\pgfpathcurveto{\pgfqpoint{0.937223in}{0.641194in}}{\pgfqpoint{0.931636in}{0.638880in}}{\pgfqpoint{0.927518in}{0.634762in}}%
\pgfpathcurveto{\pgfqpoint{0.923400in}{0.630644in}}{\pgfqpoint{0.921086in}{0.625058in}}{\pgfqpoint{0.921086in}{0.619234in}}%
\pgfpathcurveto{\pgfqpoint{0.921086in}{0.613410in}}{\pgfqpoint{0.923400in}{0.607824in}}{\pgfqpoint{0.927518in}{0.603706in}}%
\pgfpathcurveto{\pgfqpoint{0.931636in}{0.599587in}}{\pgfqpoint{0.937223in}{0.597274in}}{\pgfqpoint{0.943047in}{0.597274in}}%
\pgfpathclose%
\pgfusepath{stroke,fill}%
\end{pgfscope}%
\begin{pgfscope}%
\pgfpathrectangle{\pgfqpoint{0.211875in}{0.211875in}}{\pgfqpoint{1.313625in}{1.279725in}}%
\pgfusepath{clip}%
\pgfsetbuttcap%
\pgfsetroundjoin%
\definecolor{currentfill}{rgb}{0.121569,0.466667,0.705882}%
\pgfsetfillcolor{currentfill}%
\pgfsetlinewidth{1.003750pt}%
\definecolor{currentstroke}{rgb}{0.121569,0.466667,0.705882}%
\pgfsetstrokecolor{currentstroke}%
\pgfsetdash{}{0pt}%
\pgfpathmoveto{\pgfqpoint{0.929779in}{0.781364in}}%
\pgfpathcurveto{\pgfqpoint{0.935603in}{0.781364in}}{\pgfqpoint{0.941189in}{0.783678in}}{\pgfqpoint{0.945307in}{0.787796in}}%
\pgfpathcurveto{\pgfqpoint{0.949426in}{0.791914in}}{\pgfqpoint{0.951739in}{0.797500in}}{\pgfqpoint{0.951739in}{0.803324in}}%
\pgfpathcurveto{\pgfqpoint{0.951739in}{0.809148in}}{\pgfqpoint{0.949426in}{0.814734in}}{\pgfqpoint{0.945307in}{0.818852in}}%
\pgfpathcurveto{\pgfqpoint{0.941189in}{0.822970in}}{\pgfqpoint{0.935603in}{0.825284in}}{\pgfqpoint{0.929779in}{0.825284in}}%
\pgfpathcurveto{\pgfqpoint{0.923955in}{0.825284in}}{\pgfqpoint{0.918369in}{0.822970in}}{\pgfqpoint{0.914251in}{0.818852in}}%
\pgfpathcurveto{\pgfqpoint{0.910133in}{0.814734in}}{\pgfqpoint{0.907819in}{0.809148in}}{\pgfqpoint{0.907819in}{0.803324in}}%
\pgfpathcurveto{\pgfqpoint{0.907819in}{0.797500in}}{\pgfqpoint{0.910133in}{0.791914in}}{\pgfqpoint{0.914251in}{0.787796in}}%
\pgfpathcurveto{\pgfqpoint{0.918369in}{0.783678in}}{\pgfqpoint{0.923955in}{0.781364in}}{\pgfqpoint{0.929779in}{0.781364in}}%
\pgfpathclose%
\pgfusepath{stroke,fill}%
\end{pgfscope}%
\begin{pgfscope}%
\pgfpathrectangle{\pgfqpoint{0.211875in}{0.211875in}}{\pgfqpoint{1.313625in}{1.279725in}}%
\pgfusepath{clip}%
\pgfsetbuttcap%
\pgfsetroundjoin%
\definecolor{currentfill}{rgb}{0.121569,0.466667,0.705882}%
\pgfsetfillcolor{currentfill}%
\pgfsetlinewidth{1.003750pt}%
\definecolor{currentstroke}{rgb}{0.121569,0.466667,0.705882}%
\pgfsetstrokecolor{currentstroke}%
\pgfsetdash{}{0pt}%
\pgfpathmoveto{\pgfqpoint{1.105944in}{1.063627in}}%
\pgfpathcurveto{\pgfqpoint{1.111768in}{1.063627in}}{\pgfqpoint{1.117354in}{1.065941in}}{\pgfqpoint{1.121472in}{1.070059in}}%
\pgfpathcurveto{\pgfqpoint{1.125590in}{1.074177in}}{\pgfqpoint{1.127904in}{1.079763in}}{\pgfqpoint{1.127904in}{1.085587in}}%
\pgfpathcurveto{\pgfqpoint{1.127904in}{1.091411in}}{\pgfqpoint{1.125590in}{1.096997in}}{\pgfqpoint{1.121472in}{1.101115in}}%
\pgfpathcurveto{\pgfqpoint{1.117354in}{1.105233in}}{\pgfqpoint{1.111768in}{1.107547in}}{\pgfqpoint{1.105944in}{1.107547in}}%
\pgfpathcurveto{\pgfqpoint{1.100120in}{1.107547in}}{\pgfqpoint{1.094534in}{1.105233in}}{\pgfqpoint{1.090416in}{1.101115in}}%
\pgfpathcurveto{\pgfqpoint{1.086297in}{1.096997in}}{\pgfqpoint{1.083984in}{1.091411in}}{\pgfqpoint{1.083984in}{1.085587in}}%
\pgfpathcurveto{\pgfqpoint{1.083984in}{1.079763in}}{\pgfqpoint{1.086297in}{1.074177in}}{\pgfqpoint{1.090416in}{1.070059in}}%
\pgfpathcurveto{\pgfqpoint{1.094534in}{1.065941in}}{\pgfqpoint{1.100120in}{1.063627in}}{\pgfqpoint{1.105944in}{1.063627in}}%
\pgfpathclose%
\pgfusepath{stroke,fill}%
\end{pgfscope}%
\begin{pgfscope}%
\pgfpathrectangle{\pgfqpoint{0.211875in}{0.211875in}}{\pgfqpoint{1.313625in}{1.279725in}}%
\pgfusepath{clip}%
\pgfsetbuttcap%
\pgfsetroundjoin%
\definecolor{currentfill}{rgb}{0.121569,0.466667,0.705882}%
\pgfsetfillcolor{currentfill}%
\pgfsetlinewidth{1.003750pt}%
\definecolor{currentstroke}{rgb}{0.121569,0.466667,0.705882}%
\pgfsetstrokecolor{currentstroke}%
\pgfsetdash{}{0pt}%
\pgfpathmoveto{\pgfqpoint{0.939864in}{0.549368in}}%
\pgfpathcurveto{\pgfqpoint{0.945688in}{0.549368in}}{\pgfqpoint{0.951274in}{0.551682in}}{\pgfqpoint{0.955392in}{0.555800in}}%
\pgfpathcurveto{\pgfqpoint{0.959510in}{0.559918in}}{\pgfqpoint{0.961824in}{0.565504in}}{\pgfqpoint{0.961824in}{0.571328in}}%
\pgfpathcurveto{\pgfqpoint{0.961824in}{0.577152in}}{\pgfqpoint{0.959510in}{0.582738in}}{\pgfqpoint{0.955392in}{0.586856in}}%
\pgfpathcurveto{\pgfqpoint{0.951274in}{0.590974in}}{\pgfqpoint{0.945688in}{0.593288in}}{\pgfqpoint{0.939864in}{0.593288in}}%
\pgfpathcurveto{\pgfqpoint{0.934040in}{0.593288in}}{\pgfqpoint{0.928454in}{0.590974in}}{\pgfqpoint{0.924336in}{0.586856in}}%
\pgfpathcurveto{\pgfqpoint{0.920217in}{0.582738in}}{\pgfqpoint{0.917903in}{0.577152in}}{\pgfqpoint{0.917903in}{0.571328in}}%
\pgfpathcurveto{\pgfqpoint{0.917903in}{0.565504in}}{\pgfqpoint{0.920217in}{0.559918in}}{\pgfqpoint{0.924336in}{0.555800in}}%
\pgfpathcurveto{\pgfqpoint{0.928454in}{0.551682in}}{\pgfqpoint{0.934040in}{0.549368in}}{\pgfqpoint{0.939864in}{0.549368in}}%
\pgfpathclose%
\pgfusepath{stroke,fill}%
\end{pgfscope}%
\begin{pgfscope}%
\pgfpathrectangle{\pgfqpoint{0.211875in}{0.211875in}}{\pgfqpoint{1.313625in}{1.279725in}}%
\pgfusepath{clip}%
\pgfsetbuttcap%
\pgfsetroundjoin%
\definecolor{currentfill}{rgb}{0.121569,0.466667,0.705882}%
\pgfsetfillcolor{currentfill}%
\pgfsetlinewidth{1.003750pt}%
\definecolor{currentstroke}{rgb}{0.121569,0.466667,0.705882}%
\pgfsetstrokecolor{currentstroke}%
\pgfsetdash{}{0pt}%
\pgfpathmoveto{\pgfqpoint{1.371740in}{0.999991in}}%
\pgfpathcurveto{\pgfqpoint{1.377564in}{0.999991in}}{\pgfqpoint{1.383151in}{1.002305in}}{\pgfqpoint{1.387269in}{1.006423in}}%
\pgfpathcurveto{\pgfqpoint{1.391387in}{1.010541in}}{\pgfqpoint{1.393701in}{1.016127in}}{\pgfqpoint{1.393701in}{1.021951in}}%
\pgfpathcurveto{\pgfqpoint{1.393701in}{1.027775in}}{\pgfqpoint{1.391387in}{1.033361in}}{\pgfqpoint{1.387269in}{1.037479in}}%
\pgfpathcurveto{\pgfqpoint{1.383151in}{1.041597in}}{\pgfqpoint{1.377564in}{1.043911in}}{\pgfqpoint{1.371740in}{1.043911in}}%
\pgfpathcurveto{\pgfqpoint{1.365917in}{1.043911in}}{\pgfqpoint{1.360330in}{1.041597in}}{\pgfqpoint{1.356212in}{1.037479in}}%
\pgfpathcurveto{\pgfqpoint{1.352094in}{1.033361in}}{\pgfqpoint{1.349780in}{1.027775in}}{\pgfqpoint{1.349780in}{1.021951in}}%
\pgfpathcurveto{\pgfqpoint{1.349780in}{1.016127in}}{\pgfqpoint{1.352094in}{1.010541in}}{\pgfqpoint{1.356212in}{1.006423in}}%
\pgfpathcurveto{\pgfqpoint{1.360330in}{1.002305in}}{\pgfqpoint{1.365917in}{0.999991in}}{\pgfqpoint{1.371740in}{0.999991in}}%
\pgfpathclose%
\pgfusepath{stroke,fill}%
\end{pgfscope}%
\begin{pgfscope}%
\pgfsetrectcap%
\pgfsetmiterjoin%
\pgfsetlinewidth{0.000000pt}%
\definecolor{currentstroke}{rgb}{1.000000,1.000000,1.000000}%
\pgfsetstrokecolor{currentstroke}%
\pgfsetdash{}{0pt}%
\pgfpathmoveto{\pgfqpoint{0.211875in}{0.211875in}}%
\pgfpathlineto{\pgfqpoint{0.211875in}{1.491600in}}%
\pgfusepath{}%
\end{pgfscope}%
\begin{pgfscope}%
\pgfsetrectcap%
\pgfsetmiterjoin%
\pgfsetlinewidth{0.000000pt}%
\definecolor{currentstroke}{rgb}{1.000000,1.000000,1.000000}%
\pgfsetstrokecolor{currentstroke}%
\pgfsetdash{}{0pt}%
\pgfpathmoveto{\pgfqpoint{1.525500in}{0.211875in}}%
\pgfpathlineto{\pgfqpoint{1.525500in}{1.491600in}}%
\pgfusepath{}%
\end{pgfscope}%
\begin{pgfscope}%
\pgfsetrectcap%
\pgfsetmiterjoin%
\pgfsetlinewidth{0.000000pt}%
\definecolor{currentstroke}{rgb}{1.000000,1.000000,1.000000}%
\pgfsetstrokecolor{currentstroke}%
\pgfsetdash{}{0pt}%
\pgfpathmoveto{\pgfqpoint{0.211875in}{0.211875in}}%
\pgfpathlineto{\pgfqpoint{1.525500in}{0.211875in}}%
\pgfusepath{}%
\end{pgfscope}%
\begin{pgfscope}%
\pgfsetrectcap%
\pgfsetmiterjoin%
\pgfsetlinewidth{0.000000pt}%
\definecolor{currentstroke}{rgb}{1.000000,1.000000,1.000000}%
\pgfsetstrokecolor{currentstroke}%
\pgfsetdash{}{0pt}%
\pgfpathmoveto{\pgfqpoint{0.211875in}{1.491600in}}%
\pgfpathlineto{\pgfqpoint{1.525500in}{1.491600in}}%
\pgfusepath{}%
\end{pgfscope}%
\end{pgfpicture}%
\makeatother%
\endgroup%

            \end{minipage}
            \begin{minipage}{0.45\linewidth}
                %% Creator: Matplotlib, PGF backend
%%
%% To include the figure in your LaTeX document, write
%%   \input{<filename>.pgf}
%%
%% Make sure the required packages are loaded in your preamble
%%   \usepackage{pgf}
%%
%% Figures using additional raster images can only be included by \input if
%% they are in the same directory as the main LaTeX file. For loading figures
%% from other directories you can use the `import` package
%%   \usepackage{import}
%% and then include the figures with
%%   \import{<path to file>}{<filename>.pgf}
%%
%% Matplotlib used the following preamble
%%   \usepackage{gensymb}
%%   \usepackage{fontspec}
%%   \setmainfont{DejaVu Serif}
%%   \setsansfont{Arial}
%%   \setmonofont{DejaVu Sans Mono}
%%
\begingroup%
\makeatletter%
\begin{pgfpicture}%
\pgfpathrectangle{\pgfpointorigin}{\pgfqpoint{1.695000in}{1.695000in}}%
\pgfusepath{use as bounding box, clip}%
\begin{pgfscope}%
\pgfsetbuttcap%
\pgfsetmiterjoin%
\definecolor{currentfill}{rgb}{1.000000,1.000000,1.000000}%
\pgfsetfillcolor{currentfill}%
\pgfsetlinewidth{0.000000pt}%
\definecolor{currentstroke}{rgb}{1.000000,1.000000,1.000000}%
\pgfsetstrokecolor{currentstroke}%
\pgfsetdash{}{0pt}%
\pgfpathmoveto{\pgfqpoint{0.000000in}{0.000000in}}%
\pgfpathlineto{\pgfqpoint{1.695000in}{0.000000in}}%
\pgfpathlineto{\pgfqpoint{1.695000in}{1.695000in}}%
\pgfpathlineto{\pgfqpoint{0.000000in}{1.695000in}}%
\pgfpathclose%
\pgfusepath{fill}%
\end{pgfscope}%
\begin{pgfscope}%
\pgfsetbuttcap%
\pgfsetmiterjoin%
\definecolor{currentfill}{rgb}{0.917647,0.917647,0.949020}%
\pgfsetfillcolor{currentfill}%
\pgfsetlinewidth{0.000000pt}%
\definecolor{currentstroke}{rgb}{0.000000,0.000000,0.000000}%
\pgfsetstrokecolor{currentstroke}%
\pgfsetstrokeopacity{0.000000}%
\pgfsetdash{}{0pt}%
\pgfpathmoveto{\pgfqpoint{0.211875in}{0.211875in}}%
\pgfpathlineto{\pgfqpoint{1.525500in}{0.211875in}}%
\pgfpathlineto{\pgfqpoint{1.525500in}{1.491600in}}%
\pgfpathlineto{\pgfqpoint{0.211875in}{1.491600in}}%
\pgfpathclose%
\pgfusepath{fill}%
\end{pgfscope}%
\begin{pgfscope}%
\pgfpathrectangle{\pgfqpoint{0.211875in}{0.211875in}}{\pgfqpoint{1.313625in}{1.279725in}}%
\pgfusepath{clip}%
\pgfsetroundcap%
\pgfsetroundjoin%
\pgfsetlinewidth{0.803000pt}%
\definecolor{currentstroke}{rgb}{1.000000,1.000000,1.000000}%
\pgfsetstrokecolor{currentstroke}%
\pgfsetdash{}{0pt}%
\pgfpathmoveto{\pgfqpoint{0.286849in}{0.211875in}}%
\pgfpathlineto{\pgfqpoint{0.286849in}{1.491600in}}%
\pgfusepath{stroke}%
\end{pgfscope}%
\begin{pgfscope}%
\definecolor{textcolor}{rgb}{0.150000,0.150000,0.150000}%
\pgfsetstrokecolor{textcolor}%
\pgfsetfillcolor{textcolor}%
\pgftext[x=0.286849in,y=0.163264in,,top]{\color{textcolor}\rmfamily\fontsize{8.000000}{9.600000}\selectfont \(\displaystyle -50\)}%
\end{pgfscope}%
\begin{pgfscope}%
\pgfpathrectangle{\pgfqpoint{0.211875in}{0.211875in}}{\pgfqpoint{1.313625in}{1.279725in}}%
\pgfusepath{clip}%
\pgfsetroundcap%
\pgfsetroundjoin%
\pgfsetlinewidth{0.803000pt}%
\definecolor{currentstroke}{rgb}{1.000000,1.000000,1.000000}%
\pgfsetstrokecolor{currentstroke}%
\pgfsetdash{}{0pt}%
\pgfpathmoveto{\pgfqpoint{1.115103in}{0.211875in}}%
\pgfpathlineto{\pgfqpoint{1.115103in}{1.491600in}}%
\pgfusepath{stroke}%
\end{pgfscope}%
\begin{pgfscope}%
\definecolor{textcolor}{rgb}{0.150000,0.150000,0.150000}%
\pgfsetstrokecolor{textcolor}%
\pgfsetfillcolor{textcolor}%
\pgftext[x=1.115103in,y=0.163264in,,top]{\color{textcolor}\rmfamily\fontsize{8.000000}{9.600000}\selectfont \(\displaystyle 0\)}%
\end{pgfscope}%
\begin{pgfscope}%
\pgfpathrectangle{\pgfqpoint{0.211875in}{0.211875in}}{\pgfqpoint{1.313625in}{1.279725in}}%
\pgfusepath{clip}%
\pgfsetroundcap%
\pgfsetroundjoin%
\pgfsetlinewidth{0.803000pt}%
\definecolor{currentstroke}{rgb}{1.000000,1.000000,1.000000}%
\pgfsetstrokecolor{currentstroke}%
\pgfsetdash{}{0pt}%
\pgfpathmoveto{\pgfqpoint{0.211875in}{0.462378in}}%
\pgfpathlineto{\pgfqpoint{1.525500in}{0.462378in}}%
\pgfusepath{stroke}%
\end{pgfscope}%
\begin{pgfscope}%
\definecolor{textcolor}{rgb}{0.150000,0.150000,0.150000}%
\pgfsetstrokecolor{textcolor}%
\pgfsetfillcolor{textcolor}%
\pgftext[x=-0.046616in,y=0.420169in,left,base]{\color{textcolor}\rmfamily\fontsize{8.000000}{9.600000}\selectfont \(\displaystyle -40\)}%
\end{pgfscope}%
\begin{pgfscope}%
\pgfpathrectangle{\pgfqpoint{0.211875in}{0.211875in}}{\pgfqpoint{1.313625in}{1.279725in}}%
\pgfusepath{clip}%
\pgfsetroundcap%
\pgfsetroundjoin%
\pgfsetlinewidth{0.803000pt}%
\definecolor{currentstroke}{rgb}{1.000000,1.000000,1.000000}%
\pgfsetstrokecolor{currentstroke}%
\pgfsetdash{}{0pt}%
\pgfpathmoveto{\pgfqpoint{0.211875in}{0.805452in}}%
\pgfpathlineto{\pgfqpoint{1.525500in}{0.805452in}}%
\pgfusepath{stroke}%
\end{pgfscope}%
\begin{pgfscope}%
\definecolor{textcolor}{rgb}{0.150000,0.150000,0.150000}%
\pgfsetstrokecolor{textcolor}%
\pgfsetfillcolor{textcolor}%
\pgftext[x=-0.046616in,y=0.763243in,left,base]{\color{textcolor}\rmfamily\fontsize{8.000000}{9.600000}\selectfont \(\displaystyle -20\)}%
\end{pgfscope}%
\begin{pgfscope}%
\pgfpathrectangle{\pgfqpoint{0.211875in}{0.211875in}}{\pgfqpoint{1.313625in}{1.279725in}}%
\pgfusepath{clip}%
\pgfsetroundcap%
\pgfsetroundjoin%
\pgfsetlinewidth{0.803000pt}%
\definecolor{currentstroke}{rgb}{1.000000,1.000000,1.000000}%
\pgfsetstrokecolor{currentstroke}%
\pgfsetdash{}{0pt}%
\pgfpathmoveto{\pgfqpoint{0.211875in}{1.148526in}}%
\pgfpathlineto{\pgfqpoint{1.525500in}{1.148526in}}%
\pgfusepath{stroke}%
\end{pgfscope}%
\begin{pgfscope}%
\definecolor{textcolor}{rgb}{0.150000,0.150000,0.150000}%
\pgfsetstrokecolor{textcolor}%
\pgfsetfillcolor{textcolor}%
\pgftext[x=0.104235in,y=1.106317in,left,base]{\color{textcolor}\rmfamily\fontsize{8.000000}{9.600000}\selectfont \(\displaystyle 0\)}%
\end{pgfscope}%
\begin{pgfscope}%
\pgfpathrectangle{\pgfqpoint{0.211875in}{0.211875in}}{\pgfqpoint{1.313625in}{1.279725in}}%
\pgfusepath{clip}%
\pgfsetroundcap%
\pgfsetroundjoin%
\pgfsetlinewidth{0.803000pt}%
\definecolor{currentstroke}{rgb}{1.000000,1.000000,1.000000}%
\pgfsetstrokecolor{currentstroke}%
\pgfsetdash{}{0pt}%
\pgfpathmoveto{\pgfqpoint{0.211875in}{1.491600in}}%
\pgfpathlineto{\pgfqpoint{1.525500in}{1.491600in}}%
\pgfusepath{stroke}%
\end{pgfscope}%
\begin{pgfscope}%
\definecolor{textcolor}{rgb}{0.150000,0.150000,0.150000}%
\pgfsetstrokecolor{textcolor}%
\pgfsetfillcolor{textcolor}%
\pgftext[x=0.045207in,y=1.449391in,left,base]{\color{textcolor}\rmfamily\fontsize{8.000000}{9.600000}\selectfont \(\displaystyle 20\)}%
\end{pgfscope}%
\begin{pgfscope}%
\pgfpathrectangle{\pgfqpoint{0.211875in}{0.211875in}}{\pgfqpoint{1.313625in}{1.279725in}}%
\pgfusepath{clip}%
\pgfsetbuttcap%
\pgfsetroundjoin%
\definecolor{currentfill}{rgb}{0.067555,0.047782,0.142002}%
\pgfsetfillcolor{currentfill}%
\pgfsetlinewidth{0.000000pt}%
\definecolor{currentstroke}{rgb}{0.000000,0.000000,0.000000}%
\pgfsetstrokecolor{currentstroke}%
\pgfsetdash{}{0pt}%
\pgfpathmoveto{\pgfqpoint{0.509390in}{0.298609in}}%
\pgfpathlineto{\pgfqpoint{0.520351in}{0.302970in}}%
\pgfpathlineto{\pgfqpoint{0.521103in}{0.303754in}}%
\pgfpathlineto{\pgfqpoint{0.532815in}{0.313430in}}%
\pgfpathlineto{\pgfqpoint{0.534988in}{0.315099in}}%
\pgfpathlineto{\pgfqpoint{0.532815in}{0.318290in}}%
\pgfpathlineto{\pgfqpoint{0.521103in}{0.317632in}}%
\pgfpathlineto{\pgfqpoint{0.514761in}{0.315099in}}%
\pgfpathlineto{\pgfqpoint{0.509390in}{0.309562in}}%
\pgfpathlineto{\pgfqpoint{0.498719in}{0.302970in}}%
\pgfpathclose%
\pgfusepath{fill}%
\end{pgfscope}%
\begin{pgfscope}%
\pgfpathrectangle{\pgfqpoint{0.211875in}{0.211875in}}{\pgfqpoint{1.313625in}{1.279725in}}%
\pgfusepath{clip}%
\pgfsetbuttcap%
\pgfsetroundjoin%
\definecolor{currentfill}{rgb}{0.198046,0.094652,0.234785}%
\pgfsetfillcolor{currentfill}%
\pgfsetlinewidth{0.000000pt}%
\definecolor{currentstroke}{rgb}{0.000000,0.000000,0.000000}%
\pgfsetstrokecolor{currentstroke}%
\pgfsetdash{}{0pt}%
\pgfpathmoveto{\pgfqpoint{0.462539in}{0.290841in}}%
\pgfpathlineto{\pgfqpoint{0.474252in}{0.290841in}}%
\pgfpathlineto{\pgfqpoint{0.485965in}{0.290841in}}%
\pgfpathlineto{\pgfqpoint{0.497677in}{0.290841in}}%
\pgfpathlineto{\pgfqpoint{0.509390in}{0.290841in}}%
\pgfpathlineto{\pgfqpoint{0.521103in}{0.290841in}}%
\pgfpathlineto{\pgfqpoint{0.532815in}{0.290841in}}%
\pgfpathlineto{\pgfqpoint{0.544528in}{0.290841in}}%
\pgfpathlineto{\pgfqpoint{0.556241in}{0.290841in}}%
\pgfpathlineto{\pgfqpoint{0.567953in}{0.290841in}}%
\pgfpathlineto{\pgfqpoint{0.568273in}{0.290841in}}%
\pgfpathlineto{\pgfqpoint{0.579666in}{0.299061in}}%
\pgfpathlineto{\pgfqpoint{0.584937in}{0.302970in}}%
\pgfpathlineto{\pgfqpoint{0.591379in}{0.308474in}}%
\pgfpathlineto{\pgfqpoint{0.598981in}{0.315099in}}%
\pgfpathlineto{\pgfqpoint{0.603091in}{0.319495in}}%
\pgfpathlineto{\pgfqpoint{0.610208in}{0.327228in}}%
\pgfpathlineto{\pgfqpoint{0.614804in}{0.336658in}}%
\pgfpathlineto{\pgfqpoint{0.616087in}{0.339357in}}%
\pgfpathlineto{\pgfqpoint{0.614804in}{0.343619in}}%
\pgfpathlineto{\pgfqpoint{0.611533in}{0.351486in}}%
\pgfpathlineto{\pgfqpoint{0.603091in}{0.358633in}}%
\pgfpathlineto{\pgfqpoint{0.592771in}{0.363615in}}%
\pgfpathlineto{\pgfqpoint{0.591379in}{0.364048in}}%
\pgfpathlineto{\pgfqpoint{0.584694in}{0.363615in}}%
\pgfpathlineto{\pgfqpoint{0.579666in}{0.363339in}}%
\pgfpathlineto{\pgfqpoint{0.567953in}{0.360952in}}%
\pgfpathlineto{\pgfqpoint{0.556241in}{0.357222in}}%
\pgfpathlineto{\pgfqpoint{0.544528in}{0.352720in}}%
\pgfpathlineto{\pgfqpoint{0.542311in}{0.351486in}}%
\pgfpathlineto{\pgfqpoint{0.532815in}{0.346826in}}%
\pgfpathlineto{\pgfqpoint{0.521103in}{0.340110in}}%
\pgfpathlineto{\pgfqpoint{0.519955in}{0.339357in}}%
\pgfpathlineto{\pgfqpoint{0.509390in}{0.332793in}}%
\pgfpathlineto{\pgfqpoint{0.500401in}{0.327228in}}%
\pgfpathlineto{\pgfqpoint{0.497677in}{0.325302in}}%
\pgfpathlineto{\pgfqpoint{0.485965in}{0.317497in}}%
\pgfpathlineto{\pgfqpoint{0.482302in}{0.315099in}}%
\pgfpathlineto{\pgfqpoint{0.474252in}{0.308757in}}%
\pgfpathlineto{\pgfqpoint{0.465890in}{0.302970in}}%
\pgfpathlineto{\pgfqpoint{0.462539in}{0.299871in}}%
\pgfpathlineto{\pgfqpoint{0.451708in}{0.290841in}}%
\pgfpathclose%
\pgfpathmoveto{\pgfqpoint{0.498719in}{0.302970in}}%
\pgfpathlineto{\pgfqpoint{0.509390in}{0.309562in}}%
\pgfpathlineto{\pgfqpoint{0.514761in}{0.315099in}}%
\pgfpathlineto{\pgfqpoint{0.521103in}{0.317632in}}%
\pgfpathlineto{\pgfqpoint{0.532815in}{0.318290in}}%
\pgfpathlineto{\pgfqpoint{0.534988in}{0.315099in}}%
\pgfpathlineto{\pgfqpoint{0.532815in}{0.313430in}}%
\pgfpathlineto{\pgfqpoint{0.521103in}{0.303754in}}%
\pgfpathlineto{\pgfqpoint{0.520351in}{0.302970in}}%
\pgfpathlineto{\pgfqpoint{0.509390in}{0.298609in}}%
\pgfpathclose%
\pgfusepath{fill}%
\end{pgfscope}%
\begin{pgfscope}%
\pgfpathrectangle{\pgfqpoint{0.211875in}{0.211875in}}{\pgfqpoint{1.313625in}{1.279725in}}%
\pgfusepath{clip}%
\pgfsetbuttcap%
\pgfsetroundjoin%
\definecolor{currentfill}{rgb}{0.198046,0.094652,0.234785}%
\pgfsetfillcolor{currentfill}%
\pgfsetlinewidth{0.000000pt}%
\definecolor{currentstroke}{rgb}{0.000000,0.000000,0.000000}%
\pgfsetstrokecolor{currentstroke}%
\pgfsetdash{}{0pt}%
\pgfpathmoveto{\pgfqpoint{1.001323in}{0.653356in}}%
\pgfpathlineto{\pgfqpoint{1.013036in}{0.653113in}}%
\pgfpathlineto{\pgfqpoint{1.018191in}{0.654708in}}%
\pgfpathlineto{\pgfqpoint{1.013036in}{0.659635in}}%
\pgfpathlineto{\pgfqpoint{1.001323in}{0.664354in}}%
\pgfpathlineto{\pgfqpoint{0.995671in}{0.666836in}}%
\pgfpathlineto{\pgfqpoint{0.989610in}{0.667976in}}%
\pgfpathlineto{\pgfqpoint{0.988239in}{0.666836in}}%
\pgfpathlineto{\pgfqpoint{0.989610in}{0.663336in}}%
\pgfpathlineto{\pgfqpoint{0.994352in}{0.654708in}}%
\pgfpathclose%
\pgfusepath{fill}%
\end{pgfscope}%
\begin{pgfscope}%
\pgfpathrectangle{\pgfqpoint{0.211875in}{0.211875in}}{\pgfqpoint{1.313625in}{1.279725in}}%
\pgfusepath{clip}%
\pgfsetbuttcap%
\pgfsetroundjoin%
\definecolor{currentfill}{rgb}{0.198046,0.094652,0.234785}%
\pgfsetfillcolor{currentfill}%
\pgfsetlinewidth{0.000000pt}%
\definecolor{currentstroke}{rgb}{0.000000,0.000000,0.000000}%
\pgfsetstrokecolor{currentstroke}%
\pgfsetdash{}{0pt}%
\pgfpathmoveto{\pgfqpoint{1.130163in}{1.477078in}}%
\pgfpathlineto{\pgfqpoint{1.141875in}{1.469135in}}%
\pgfpathlineto{\pgfqpoint{1.153588in}{1.468910in}}%
\pgfpathlineto{\pgfqpoint{1.165301in}{1.476758in}}%
\pgfpathlineto{\pgfqpoint{1.168667in}{1.479471in}}%
\pgfpathlineto{\pgfqpoint{1.177013in}{1.486978in}}%
\pgfpathlineto{\pgfqpoint{1.182138in}{1.491600in}}%
\pgfpathlineto{\pgfqpoint{1.177013in}{1.491600in}}%
\pgfpathlineto{\pgfqpoint{1.165301in}{1.491600in}}%
\pgfpathlineto{\pgfqpoint{1.153588in}{1.491600in}}%
\pgfpathlineto{\pgfqpoint{1.141875in}{1.491600in}}%
\pgfpathlineto{\pgfqpoint{1.130163in}{1.491600in}}%
\pgfpathlineto{\pgfqpoint{1.124380in}{1.491600in}}%
\pgfpathlineto{\pgfqpoint{1.128569in}{1.479471in}}%
\pgfpathclose%
\pgfusepath{fill}%
\end{pgfscope}%
\begin{pgfscope}%
\pgfpathrectangle{\pgfqpoint{0.211875in}{0.211875in}}{\pgfqpoint{1.313625in}{1.279725in}}%
\pgfusepath{clip}%
\pgfsetbuttcap%
\pgfsetroundjoin%
\definecolor{currentfill}{rgb}{0.198046,0.094652,0.234785}%
\pgfsetfillcolor{currentfill}%
\pgfsetlinewidth{0.000000pt}%
\definecolor{currentstroke}{rgb}{0.000000,0.000000,0.000000}%
\pgfsetstrokecolor{currentstroke}%
\pgfsetdash{}{0pt}%
\pgfpathmoveto{\pgfqpoint{1.247289in}{1.490580in}}%
\pgfpathlineto{\pgfqpoint{1.259002in}{1.487463in}}%
\pgfpathlineto{\pgfqpoint{1.270715in}{1.485677in}}%
\pgfpathlineto{\pgfqpoint{1.282427in}{1.486440in}}%
\pgfpathlineto{\pgfqpoint{1.286362in}{1.491600in}}%
\pgfpathlineto{\pgfqpoint{1.282427in}{1.491600in}}%
\pgfpathlineto{\pgfqpoint{1.270715in}{1.491600in}}%
\pgfpathlineto{\pgfqpoint{1.259002in}{1.491600in}}%
\pgfpathlineto{\pgfqpoint{1.247289in}{1.491600in}}%
\pgfpathlineto{\pgfqpoint{1.245451in}{1.491600in}}%
\pgfpathclose%
\pgfusepath{fill}%
\end{pgfscope}%
\begin{pgfscope}%
\pgfpathrectangle{\pgfqpoint{0.211875in}{0.211875in}}{\pgfqpoint{1.313625in}{1.279725in}}%
\pgfusepath{clip}%
\pgfsetbuttcap%
\pgfsetroundjoin%
\definecolor{currentfill}{rgb}{0.343142,0.118134,0.311397}%
\pgfsetfillcolor{currentfill}%
\pgfsetlinewidth{0.000000pt}%
\definecolor{currentstroke}{rgb}{0.000000,0.000000,0.000000}%
\pgfsetstrokecolor{currentstroke}%
\pgfsetdash{}{0pt}%
\pgfpathmoveto{\pgfqpoint{0.439114in}{0.290841in}}%
\pgfpathlineto{\pgfqpoint{0.450827in}{0.290841in}}%
\pgfpathlineto{\pgfqpoint{0.451708in}{0.290841in}}%
\pgfpathlineto{\pgfqpoint{0.462539in}{0.299871in}}%
\pgfpathlineto{\pgfqpoint{0.465890in}{0.302970in}}%
\pgfpathlineto{\pgfqpoint{0.474252in}{0.308757in}}%
\pgfpathlineto{\pgfqpoint{0.482302in}{0.315099in}}%
\pgfpathlineto{\pgfqpoint{0.485965in}{0.317497in}}%
\pgfpathlineto{\pgfqpoint{0.497677in}{0.325302in}}%
\pgfpathlineto{\pgfqpoint{0.500401in}{0.327228in}}%
\pgfpathlineto{\pgfqpoint{0.509390in}{0.332793in}}%
\pgfpathlineto{\pgfqpoint{0.519955in}{0.339357in}}%
\pgfpathlineto{\pgfqpoint{0.521103in}{0.340110in}}%
\pgfpathlineto{\pgfqpoint{0.532815in}{0.346826in}}%
\pgfpathlineto{\pgfqpoint{0.542311in}{0.351486in}}%
\pgfpathlineto{\pgfqpoint{0.544528in}{0.352720in}}%
\pgfpathlineto{\pgfqpoint{0.556241in}{0.357222in}}%
\pgfpathlineto{\pgfqpoint{0.567953in}{0.360952in}}%
\pgfpathlineto{\pgfqpoint{0.579666in}{0.363339in}}%
\pgfpathlineto{\pgfqpoint{0.584694in}{0.363615in}}%
\pgfpathlineto{\pgfqpoint{0.591379in}{0.364048in}}%
\pgfpathlineto{\pgfqpoint{0.592771in}{0.363615in}}%
\pgfpathlineto{\pgfqpoint{0.603091in}{0.358633in}}%
\pgfpathlineto{\pgfqpoint{0.611533in}{0.351486in}}%
\pgfpathlineto{\pgfqpoint{0.614804in}{0.343619in}}%
\pgfpathlineto{\pgfqpoint{0.616087in}{0.339357in}}%
\pgfpathlineto{\pgfqpoint{0.614804in}{0.336658in}}%
\pgfpathlineto{\pgfqpoint{0.610208in}{0.327228in}}%
\pgfpathlineto{\pgfqpoint{0.603091in}{0.319495in}}%
\pgfpathlineto{\pgfqpoint{0.598981in}{0.315099in}}%
\pgfpathlineto{\pgfqpoint{0.591379in}{0.308474in}}%
\pgfpathlineto{\pgfqpoint{0.584937in}{0.302970in}}%
\pgfpathlineto{\pgfqpoint{0.579666in}{0.299061in}}%
\pgfpathlineto{\pgfqpoint{0.568273in}{0.290841in}}%
\pgfpathlineto{\pgfqpoint{0.579666in}{0.290841in}}%
\pgfpathlineto{\pgfqpoint{0.591379in}{0.290841in}}%
\pgfpathlineto{\pgfqpoint{0.603091in}{0.290841in}}%
\pgfpathlineto{\pgfqpoint{0.614804in}{0.290841in}}%
\pgfpathlineto{\pgfqpoint{0.625945in}{0.290841in}}%
\pgfpathlineto{\pgfqpoint{0.626517in}{0.291272in}}%
\pgfpathlineto{\pgfqpoint{0.638230in}{0.299558in}}%
\pgfpathlineto{\pgfqpoint{0.642916in}{0.302970in}}%
\pgfpathlineto{\pgfqpoint{0.649942in}{0.309072in}}%
\pgfpathlineto{\pgfqpoint{0.656688in}{0.315099in}}%
\pgfpathlineto{\pgfqpoint{0.661655in}{0.321384in}}%
\pgfpathlineto{\pgfqpoint{0.665906in}{0.327228in}}%
\pgfpathlineto{\pgfqpoint{0.667514in}{0.339357in}}%
\pgfpathlineto{\pgfqpoint{0.662681in}{0.351486in}}%
\pgfpathlineto{\pgfqpoint{0.661655in}{0.353230in}}%
\pgfpathlineto{\pgfqpoint{0.655812in}{0.363615in}}%
\pgfpathlineto{\pgfqpoint{0.650879in}{0.375743in}}%
\pgfpathlineto{\pgfqpoint{0.655715in}{0.387872in}}%
\pgfpathlineto{\pgfqpoint{0.661065in}{0.400001in}}%
\pgfpathlineto{\pgfqpoint{0.660582in}{0.412130in}}%
\pgfpathlineto{\pgfqpoint{0.649942in}{0.417754in}}%
\pgfpathlineto{\pgfqpoint{0.638230in}{0.417960in}}%
\pgfpathlineto{\pgfqpoint{0.626517in}{0.415587in}}%
\pgfpathlineto{\pgfqpoint{0.617233in}{0.412130in}}%
\pgfpathlineto{\pgfqpoint{0.614804in}{0.411154in}}%
\pgfpathlineto{\pgfqpoint{0.603091in}{0.405572in}}%
\pgfpathlineto{\pgfqpoint{0.591379in}{0.400066in}}%
\pgfpathlineto{\pgfqpoint{0.591262in}{0.400001in}}%
\pgfpathlineto{\pgfqpoint{0.579666in}{0.393325in}}%
\pgfpathlineto{\pgfqpoint{0.569899in}{0.387872in}}%
\pgfpathlineto{\pgfqpoint{0.567953in}{0.386668in}}%
\pgfpathlineto{\pgfqpoint{0.556241in}{0.378678in}}%
\pgfpathlineto{\pgfqpoint{0.551944in}{0.375743in}}%
\pgfpathlineto{\pgfqpoint{0.544528in}{0.371509in}}%
\pgfpathlineto{\pgfqpoint{0.533262in}{0.363615in}}%
\pgfpathlineto{\pgfqpoint{0.532815in}{0.363353in}}%
\pgfpathlineto{\pgfqpoint{0.521103in}{0.355140in}}%
\pgfpathlineto{\pgfqpoint{0.516131in}{0.351486in}}%
\pgfpathlineto{\pgfqpoint{0.509390in}{0.347014in}}%
\pgfpathlineto{\pgfqpoint{0.498711in}{0.339357in}}%
\pgfpathlineto{\pgfqpoint{0.497677in}{0.338689in}}%
\pgfpathlineto{\pgfqpoint{0.485965in}{0.330793in}}%
\pgfpathlineto{\pgfqpoint{0.481093in}{0.327228in}}%
\pgfpathlineto{\pgfqpoint{0.474252in}{0.322552in}}%
\pgfpathlineto{\pgfqpoint{0.463470in}{0.315099in}}%
\pgfpathlineto{\pgfqpoint{0.462539in}{0.314401in}}%
\pgfpathlineto{\pgfqpoint{0.450827in}{0.305954in}}%
\pgfpathlineto{\pgfqpoint{0.446663in}{0.302970in}}%
\pgfpathlineto{\pgfqpoint{0.439114in}{0.296676in}}%
\pgfpathlineto{\pgfqpoint{0.431408in}{0.290841in}}%
\pgfpathclose%
\pgfusepath{fill}%
\end{pgfscope}%
\begin{pgfscope}%
\pgfpathrectangle{\pgfqpoint{0.211875in}{0.211875in}}{\pgfqpoint{1.313625in}{1.279725in}}%
\pgfusepath{clip}%
\pgfsetbuttcap%
\pgfsetroundjoin%
\definecolor{currentfill}{rgb}{0.343142,0.118134,0.311397}%
\pgfsetfillcolor{currentfill}%
\pgfsetlinewidth{0.000000pt}%
\definecolor{currentstroke}{rgb}{0.000000,0.000000,0.000000}%
\pgfsetstrokecolor{currentstroke}%
\pgfsetdash{}{0pt}%
\pgfpathmoveto{\pgfqpoint{1.059886in}{0.563698in}}%
\pgfpathlineto{\pgfqpoint{1.071599in}{0.563541in}}%
\pgfpathlineto{\pgfqpoint{1.083312in}{0.562031in}}%
\pgfpathlineto{\pgfqpoint{1.095024in}{0.560578in}}%
\pgfpathlineto{\pgfqpoint{1.106737in}{0.560556in}}%
\pgfpathlineto{\pgfqpoint{1.110805in}{0.569805in}}%
\pgfpathlineto{\pgfqpoint{1.115490in}{0.581934in}}%
\pgfpathlineto{\pgfqpoint{1.112395in}{0.594063in}}%
\pgfpathlineto{\pgfqpoint{1.106737in}{0.605419in}}%
\pgfpathlineto{\pgfqpoint{1.095024in}{0.605221in}}%
\pgfpathlineto{\pgfqpoint{1.083312in}{0.603160in}}%
\pgfpathlineto{\pgfqpoint{1.071599in}{0.606162in}}%
\pgfpathlineto{\pgfqpoint{1.071517in}{0.606192in}}%
\pgfpathlineto{\pgfqpoint{1.059886in}{0.608702in}}%
\pgfpathlineto{\pgfqpoint{1.052413in}{0.606192in}}%
\pgfpathlineto{\pgfqpoint{1.048174in}{0.603704in}}%
\pgfpathlineto{\pgfqpoint{1.042772in}{0.594063in}}%
\pgfpathlineto{\pgfqpoint{1.043895in}{0.581934in}}%
\pgfpathlineto{\pgfqpoint{1.048174in}{0.570334in}}%
\pgfpathlineto{\pgfqpoint{1.049124in}{0.569805in}}%
\pgfpathclose%
\pgfusepath{fill}%
\end{pgfscope}%
\begin{pgfscope}%
\pgfpathrectangle{\pgfqpoint{0.211875in}{0.211875in}}{\pgfqpoint{1.313625in}{1.279725in}}%
\pgfusepath{clip}%
\pgfsetbuttcap%
\pgfsetroundjoin%
\definecolor{currentfill}{rgb}{0.343142,0.118134,0.311397}%
\pgfsetfillcolor{currentfill}%
\pgfsetlinewidth{0.000000pt}%
\definecolor{currentstroke}{rgb}{0.000000,0.000000,0.000000}%
\pgfsetstrokecolor{currentstroke}%
\pgfsetdash{}{0pt}%
\pgfpathmoveto{\pgfqpoint{1.399554in}{0.580419in}}%
\pgfpathlineto{\pgfqpoint{1.407865in}{0.581934in}}%
\pgfpathlineto{\pgfqpoint{1.411267in}{0.582675in}}%
\pgfpathlineto{\pgfqpoint{1.422980in}{0.591569in}}%
\pgfpathlineto{\pgfqpoint{1.424583in}{0.594063in}}%
\pgfpathlineto{\pgfqpoint{1.428365in}{0.606192in}}%
\pgfpathlineto{\pgfqpoint{1.425081in}{0.618321in}}%
\pgfpathlineto{\pgfqpoint{1.422980in}{0.620237in}}%
\pgfpathlineto{\pgfqpoint{1.412189in}{0.630450in}}%
\pgfpathlineto{\pgfqpoint{1.411267in}{0.630941in}}%
\pgfpathlineto{\pgfqpoint{1.400995in}{0.642579in}}%
\pgfpathlineto{\pgfqpoint{1.399554in}{0.643734in}}%
\pgfpathlineto{\pgfqpoint{1.388535in}{0.654708in}}%
\pgfpathlineto{\pgfqpoint{1.387842in}{0.655251in}}%
\pgfpathlineto{\pgfqpoint{1.376129in}{0.665128in}}%
\pgfpathlineto{\pgfqpoint{1.368108in}{0.666836in}}%
\pgfpathlineto{\pgfqpoint{1.364416in}{0.668306in}}%
\pgfpathlineto{\pgfqpoint{1.363161in}{0.666836in}}%
\pgfpathlineto{\pgfqpoint{1.352704in}{0.659729in}}%
\pgfpathlineto{\pgfqpoint{1.346917in}{0.654708in}}%
\pgfpathlineto{\pgfqpoint{1.340991in}{0.648078in}}%
\pgfpathlineto{\pgfqpoint{1.335339in}{0.642579in}}%
\pgfpathlineto{\pgfqpoint{1.333990in}{0.630450in}}%
\pgfpathlineto{\pgfqpoint{1.335842in}{0.618321in}}%
\pgfpathlineto{\pgfqpoint{1.339341in}{0.606192in}}%
\pgfpathlineto{\pgfqpoint{1.340991in}{0.601875in}}%
\pgfpathlineto{\pgfqpoint{1.345868in}{0.594063in}}%
\pgfpathlineto{\pgfqpoint{1.352704in}{0.587867in}}%
\pgfpathlineto{\pgfqpoint{1.364416in}{0.583353in}}%
\pgfpathlineto{\pgfqpoint{1.376129in}{0.582327in}}%
\pgfpathlineto{\pgfqpoint{1.387842in}{0.583250in}}%
\pgfpathlineto{\pgfqpoint{1.393689in}{0.581934in}}%
\pgfpathclose%
\pgfusepath{fill}%
\end{pgfscope}%
\begin{pgfscope}%
\pgfpathrectangle{\pgfqpoint{0.211875in}{0.211875in}}{\pgfqpoint{1.313625in}{1.279725in}}%
\pgfusepath{clip}%
\pgfsetbuttcap%
\pgfsetroundjoin%
\definecolor{currentfill}{rgb}{0.343142,0.118134,0.311397}%
\pgfsetfillcolor{currentfill}%
\pgfsetlinewidth{0.000000pt}%
\definecolor{currentstroke}{rgb}{0.000000,0.000000,0.000000}%
\pgfsetstrokecolor{currentstroke}%
\pgfsetdash{}{0pt}%
\pgfpathmoveto{\pgfqpoint{0.989610in}{0.618212in}}%
\pgfpathlineto{\pgfqpoint{1.001323in}{0.618277in}}%
\pgfpathlineto{\pgfqpoint{1.001553in}{0.618321in}}%
\pgfpathlineto{\pgfqpoint{1.013036in}{0.620563in}}%
\pgfpathlineto{\pgfqpoint{1.024748in}{0.622586in}}%
\pgfpathlineto{\pgfqpoint{1.036461in}{0.624783in}}%
\pgfpathlineto{\pgfqpoint{1.048174in}{0.623642in}}%
\pgfpathlineto{\pgfqpoint{1.059886in}{0.624266in}}%
\pgfpathlineto{\pgfqpoint{1.071599in}{0.628441in}}%
\pgfpathlineto{\pgfqpoint{1.075175in}{0.630450in}}%
\pgfpathlineto{\pgfqpoint{1.083312in}{0.637781in}}%
\pgfpathlineto{\pgfqpoint{1.084156in}{0.642579in}}%
\pgfpathlineto{\pgfqpoint{1.083312in}{0.643934in}}%
\pgfpathlineto{\pgfqpoint{1.078002in}{0.654708in}}%
\pgfpathlineto{\pgfqpoint{1.071692in}{0.666836in}}%
\pgfpathlineto{\pgfqpoint{1.073598in}{0.678965in}}%
\pgfpathlineto{\pgfqpoint{1.074584in}{0.691094in}}%
\pgfpathlineto{\pgfqpoint{1.071599in}{0.696026in}}%
\pgfpathlineto{\pgfqpoint{1.066999in}{0.703223in}}%
\pgfpathlineto{\pgfqpoint{1.059886in}{0.713294in}}%
\pgfpathlineto{\pgfqpoint{1.056786in}{0.715352in}}%
\pgfpathlineto{\pgfqpoint{1.048174in}{0.719600in}}%
\pgfpathlineto{\pgfqpoint{1.036461in}{0.720726in}}%
\pgfpathlineto{\pgfqpoint{1.024748in}{0.721297in}}%
\pgfpathlineto{\pgfqpoint{1.013036in}{0.720677in}}%
\pgfpathlineto{\pgfqpoint{1.001323in}{0.716616in}}%
\pgfpathlineto{\pgfqpoint{0.999332in}{0.715352in}}%
\pgfpathlineto{\pgfqpoint{0.989610in}{0.710299in}}%
\pgfpathlineto{\pgfqpoint{0.977898in}{0.703768in}}%
\pgfpathlineto{\pgfqpoint{0.976766in}{0.703223in}}%
\pgfpathlineto{\pgfqpoint{0.966185in}{0.698933in}}%
\pgfpathlineto{\pgfqpoint{0.954472in}{0.695482in}}%
\pgfpathlineto{\pgfqpoint{0.942759in}{0.693087in}}%
\pgfpathlineto{\pgfqpoint{0.934887in}{0.691094in}}%
\pgfpathlineto{\pgfqpoint{0.931047in}{0.689561in}}%
\pgfpathlineto{\pgfqpoint{0.919334in}{0.685590in}}%
\pgfpathlineto{\pgfqpoint{0.914614in}{0.678965in}}%
\pgfpathlineto{\pgfqpoint{0.918293in}{0.666836in}}%
\pgfpathlineto{\pgfqpoint{0.919334in}{0.664740in}}%
\pgfpathlineto{\pgfqpoint{0.928274in}{0.654708in}}%
\pgfpathlineto{\pgfqpoint{0.931047in}{0.643174in}}%
\pgfpathlineto{\pgfqpoint{0.931313in}{0.642579in}}%
\pgfpathlineto{\pgfqpoint{0.937710in}{0.630450in}}%
\pgfpathlineto{\pgfqpoint{0.942759in}{0.626797in}}%
\pgfpathlineto{\pgfqpoint{0.954472in}{0.623547in}}%
\pgfpathlineto{\pgfqpoint{0.966185in}{0.621448in}}%
\pgfpathlineto{\pgfqpoint{0.977898in}{0.619950in}}%
\pgfpathlineto{\pgfqpoint{0.988588in}{0.618321in}}%
\pgfpathclose%
\pgfpathmoveto{\pgfqpoint{0.994352in}{0.654708in}}%
\pgfpathlineto{\pgfqpoint{0.989610in}{0.663336in}}%
\pgfpathlineto{\pgfqpoint{0.988239in}{0.666836in}}%
\pgfpathlineto{\pgfqpoint{0.989610in}{0.667976in}}%
\pgfpathlineto{\pgfqpoint{0.995671in}{0.666836in}}%
\pgfpathlineto{\pgfqpoint{1.001323in}{0.664354in}}%
\pgfpathlineto{\pgfqpoint{1.013036in}{0.659635in}}%
\pgfpathlineto{\pgfqpoint{1.018191in}{0.654708in}}%
\pgfpathlineto{\pgfqpoint{1.013036in}{0.653113in}}%
\pgfpathlineto{\pgfqpoint{1.001323in}{0.653356in}}%
\pgfpathclose%
\pgfusepath{fill}%
\end{pgfscope}%
\begin{pgfscope}%
\pgfpathrectangle{\pgfqpoint{0.211875in}{0.211875in}}{\pgfqpoint{1.313625in}{1.279725in}}%
\pgfusepath{clip}%
\pgfsetbuttcap%
\pgfsetroundjoin%
\definecolor{currentfill}{rgb}{0.343142,0.118134,0.311397}%
\pgfsetfillcolor{currentfill}%
\pgfsetlinewidth{0.000000pt}%
\definecolor{currentstroke}{rgb}{0.000000,0.000000,0.000000}%
\pgfsetstrokecolor{currentstroke}%
\pgfsetdash{}{0pt}%
\pgfpathmoveto{\pgfqpoint{0.521103in}{0.637056in}}%
\pgfpathlineto{\pgfqpoint{0.532815in}{0.631108in}}%
\pgfpathlineto{\pgfqpoint{0.544528in}{0.631476in}}%
\pgfpathlineto{\pgfqpoint{0.556241in}{0.632773in}}%
\pgfpathlineto{\pgfqpoint{0.567953in}{0.636658in}}%
\pgfpathlineto{\pgfqpoint{0.579666in}{0.640614in}}%
\pgfpathlineto{\pgfqpoint{0.583214in}{0.642579in}}%
\pgfpathlineto{\pgfqpoint{0.591379in}{0.646555in}}%
\pgfpathlineto{\pgfqpoint{0.602950in}{0.654708in}}%
\pgfpathlineto{\pgfqpoint{0.603091in}{0.654807in}}%
\pgfpathlineto{\pgfqpoint{0.614804in}{0.664257in}}%
\pgfpathlineto{\pgfqpoint{0.617473in}{0.666836in}}%
\pgfpathlineto{\pgfqpoint{0.626517in}{0.675891in}}%
\pgfpathlineto{\pgfqpoint{0.628975in}{0.678965in}}%
\pgfpathlineto{\pgfqpoint{0.633689in}{0.691094in}}%
\pgfpathlineto{\pgfqpoint{0.632810in}{0.703223in}}%
\pgfpathlineto{\pgfqpoint{0.626517in}{0.710721in}}%
\pgfpathlineto{\pgfqpoint{0.622658in}{0.715352in}}%
\pgfpathlineto{\pgfqpoint{0.614804in}{0.720714in}}%
\pgfpathlineto{\pgfqpoint{0.604529in}{0.727481in}}%
\pgfpathlineto{\pgfqpoint{0.603091in}{0.728097in}}%
\pgfpathlineto{\pgfqpoint{0.591379in}{0.732284in}}%
\pgfpathlineto{\pgfqpoint{0.579666in}{0.736597in}}%
\pgfpathlineto{\pgfqpoint{0.571219in}{0.739610in}}%
\pgfpathlineto{\pgfqpoint{0.567953in}{0.740583in}}%
\pgfpathlineto{\pgfqpoint{0.556241in}{0.743572in}}%
\pgfpathlineto{\pgfqpoint{0.544528in}{0.746153in}}%
\pgfpathlineto{\pgfqpoint{0.532815in}{0.747640in}}%
\pgfpathlineto{\pgfqpoint{0.521103in}{0.747895in}}%
\pgfpathlineto{\pgfqpoint{0.509390in}{0.746957in}}%
\pgfpathlineto{\pgfqpoint{0.497677in}{0.744789in}}%
\pgfpathlineto{\pgfqpoint{0.485965in}{0.741306in}}%
\pgfpathlineto{\pgfqpoint{0.481436in}{0.739610in}}%
\pgfpathlineto{\pgfqpoint{0.474252in}{0.733482in}}%
\pgfpathlineto{\pgfqpoint{0.468838in}{0.727481in}}%
\pgfpathlineto{\pgfqpoint{0.470515in}{0.715352in}}%
\pgfpathlineto{\pgfqpoint{0.474061in}{0.703223in}}%
\pgfpathlineto{\pgfqpoint{0.473132in}{0.691094in}}%
\pgfpathlineto{\pgfqpoint{0.472783in}{0.678965in}}%
\pgfpathlineto{\pgfqpoint{0.474252in}{0.675949in}}%
\pgfpathlineto{\pgfqpoint{0.480358in}{0.666836in}}%
\pgfpathlineto{\pgfqpoint{0.485965in}{0.660531in}}%
\pgfpathlineto{\pgfqpoint{0.492821in}{0.654708in}}%
\pgfpathlineto{\pgfqpoint{0.497677in}{0.651503in}}%
\pgfpathlineto{\pgfqpoint{0.509390in}{0.645156in}}%
\pgfpathlineto{\pgfqpoint{0.514478in}{0.642579in}}%
\pgfpathclose%
\pgfusepath{fill}%
\end{pgfscope}%
\begin{pgfscope}%
\pgfpathrectangle{\pgfqpoint{0.211875in}{0.211875in}}{\pgfqpoint{1.313625in}{1.279725in}}%
\pgfusepath{clip}%
\pgfsetbuttcap%
\pgfsetroundjoin%
\definecolor{currentfill}{rgb}{0.343142,0.118134,0.311397}%
\pgfsetfillcolor{currentfill}%
\pgfsetlinewidth{0.000000pt}%
\definecolor{currentstroke}{rgb}{0.000000,0.000000,0.000000}%
\pgfsetstrokecolor{currentstroke}%
\pgfsetdash{}{0pt}%
\pgfpathmoveto{\pgfqpoint{1.294140in}{0.787784in}}%
\pgfpathlineto{\pgfqpoint{1.294726in}{0.788125in}}%
\pgfpathlineto{\pgfqpoint{1.305853in}{0.795045in}}%
\pgfpathlineto{\pgfqpoint{1.317566in}{0.798877in}}%
\pgfpathlineto{\pgfqpoint{1.318917in}{0.800254in}}%
\pgfpathlineto{\pgfqpoint{1.329278in}{0.812032in}}%
\pgfpathlineto{\pgfqpoint{1.329467in}{0.812383in}}%
\pgfpathlineto{\pgfqpoint{1.334032in}{0.824512in}}%
\pgfpathlineto{\pgfqpoint{1.340991in}{0.835934in}}%
\pgfpathlineto{\pgfqpoint{1.341436in}{0.836641in}}%
\pgfpathlineto{\pgfqpoint{1.346288in}{0.848770in}}%
\pgfpathlineto{\pgfqpoint{1.352704in}{0.858744in}}%
\pgfpathlineto{\pgfqpoint{1.354242in}{0.860898in}}%
\pgfpathlineto{\pgfqpoint{1.361429in}{0.873027in}}%
\pgfpathlineto{\pgfqpoint{1.364416in}{0.882897in}}%
\pgfpathlineto{\pgfqpoint{1.364986in}{0.885156in}}%
\pgfpathlineto{\pgfqpoint{1.364482in}{0.897285in}}%
\pgfpathlineto{\pgfqpoint{1.364416in}{0.897400in}}%
\pgfpathlineto{\pgfqpoint{1.352704in}{0.904814in}}%
\pgfpathlineto{\pgfqpoint{1.340991in}{0.904445in}}%
\pgfpathlineto{\pgfqpoint{1.329278in}{0.899964in}}%
\pgfpathlineto{\pgfqpoint{1.322662in}{0.897285in}}%
\pgfpathlineto{\pgfqpoint{1.317566in}{0.894614in}}%
\pgfpathlineto{\pgfqpoint{1.306463in}{0.885156in}}%
\pgfpathlineto{\pgfqpoint{1.305853in}{0.883443in}}%
\pgfpathlineto{\pgfqpoint{1.304183in}{0.873027in}}%
\pgfpathlineto{\pgfqpoint{1.303492in}{0.860898in}}%
\pgfpathlineto{\pgfqpoint{1.300591in}{0.848770in}}%
\pgfpathlineto{\pgfqpoint{1.296241in}{0.836641in}}%
\pgfpathlineto{\pgfqpoint{1.294140in}{0.832997in}}%
\pgfpathlineto{\pgfqpoint{1.289885in}{0.824512in}}%
\pgfpathlineto{\pgfqpoint{1.284404in}{0.812383in}}%
\pgfpathlineto{\pgfqpoint{1.282942in}{0.800254in}}%
\pgfpathlineto{\pgfqpoint{1.293118in}{0.788125in}}%
\pgfpathclose%
\pgfusepath{fill}%
\end{pgfscope}%
\begin{pgfscope}%
\pgfpathrectangle{\pgfqpoint{0.211875in}{0.211875in}}{\pgfqpoint{1.313625in}{1.279725in}}%
\pgfusepath{clip}%
\pgfsetbuttcap%
\pgfsetroundjoin%
\definecolor{currentfill}{rgb}{0.343142,0.118134,0.311397}%
\pgfsetfillcolor{currentfill}%
\pgfsetlinewidth{0.000000pt}%
\definecolor{currentstroke}{rgb}{0.000000,0.000000,0.000000}%
\pgfsetstrokecolor{currentstroke}%
\pgfsetdash{}{0pt}%
\pgfpathmoveto{\pgfqpoint{0.778782in}{0.836269in}}%
\pgfpathlineto{\pgfqpoint{0.790495in}{0.834980in}}%
\pgfpathlineto{\pgfqpoint{0.802207in}{0.833480in}}%
\pgfpathlineto{\pgfqpoint{0.813920in}{0.833018in}}%
\pgfpathlineto{\pgfqpoint{0.825633in}{0.833102in}}%
\pgfpathlineto{\pgfqpoint{0.837345in}{0.832945in}}%
\pgfpathlineto{\pgfqpoint{0.844666in}{0.836641in}}%
\pgfpathlineto{\pgfqpoint{0.845559in}{0.848770in}}%
\pgfpathlineto{\pgfqpoint{0.840909in}{0.860898in}}%
\pgfpathlineto{\pgfqpoint{0.837345in}{0.866207in}}%
\pgfpathlineto{\pgfqpoint{0.832157in}{0.873027in}}%
\pgfpathlineto{\pgfqpoint{0.825633in}{0.883647in}}%
\pgfpathlineto{\pgfqpoint{0.824498in}{0.885156in}}%
\pgfpathlineto{\pgfqpoint{0.813920in}{0.894803in}}%
\pgfpathlineto{\pgfqpoint{0.810944in}{0.897285in}}%
\pgfpathlineto{\pgfqpoint{0.802207in}{0.905860in}}%
\pgfpathlineto{\pgfqpoint{0.798925in}{0.909414in}}%
\pgfpathlineto{\pgfqpoint{0.790495in}{0.918105in}}%
\pgfpathlineto{\pgfqpoint{0.786980in}{0.921543in}}%
\pgfpathlineto{\pgfqpoint{0.778782in}{0.929166in}}%
\pgfpathlineto{\pgfqpoint{0.772786in}{0.933672in}}%
\pgfpathlineto{\pgfqpoint{0.767069in}{0.937228in}}%
\pgfpathlineto{\pgfqpoint{0.755356in}{0.942536in}}%
\pgfpathlineto{\pgfqpoint{0.747021in}{0.945801in}}%
\pgfpathlineto{\pgfqpoint{0.743644in}{0.947287in}}%
\pgfpathlineto{\pgfqpoint{0.731931in}{0.952571in}}%
\pgfpathlineto{\pgfqpoint{0.724020in}{0.957929in}}%
\pgfpathlineto{\pgfqpoint{0.724718in}{0.970058in}}%
\pgfpathlineto{\pgfqpoint{0.725348in}{0.982187in}}%
\pgfpathlineto{\pgfqpoint{0.724338in}{0.994316in}}%
\pgfpathlineto{\pgfqpoint{0.722665in}{1.006445in}}%
\pgfpathlineto{\pgfqpoint{0.731130in}{1.018574in}}%
\pgfpathlineto{\pgfqpoint{0.731931in}{1.019835in}}%
\pgfpathlineto{\pgfqpoint{0.737506in}{1.030703in}}%
\pgfpathlineto{\pgfqpoint{0.742157in}{1.042832in}}%
\pgfpathlineto{\pgfqpoint{0.743644in}{1.046372in}}%
\pgfpathlineto{\pgfqpoint{0.747361in}{1.054960in}}%
\pgfpathlineto{\pgfqpoint{0.753871in}{1.067089in}}%
\pgfpathlineto{\pgfqpoint{0.755356in}{1.070494in}}%
\pgfpathlineto{\pgfqpoint{0.759573in}{1.079218in}}%
\pgfpathlineto{\pgfqpoint{0.763949in}{1.091347in}}%
\pgfpathlineto{\pgfqpoint{0.765205in}{1.103476in}}%
\pgfpathlineto{\pgfqpoint{0.755356in}{1.111056in}}%
\pgfpathlineto{\pgfqpoint{0.743644in}{1.108938in}}%
\pgfpathlineto{\pgfqpoint{0.731931in}{1.107109in}}%
\pgfpathlineto{\pgfqpoint{0.720218in}{1.103816in}}%
\pgfpathlineto{\pgfqpoint{0.719359in}{1.103476in}}%
\pgfpathlineto{\pgfqpoint{0.708506in}{1.096759in}}%
\pgfpathlineto{\pgfqpoint{0.700938in}{1.091347in}}%
\pgfpathlineto{\pgfqpoint{0.696793in}{1.089022in}}%
\pgfpathlineto{\pgfqpoint{0.685080in}{1.084033in}}%
\pgfpathlineto{\pgfqpoint{0.673368in}{1.080234in}}%
\pgfpathlineto{\pgfqpoint{0.669506in}{1.079218in}}%
\pgfpathlineto{\pgfqpoint{0.661655in}{1.076953in}}%
\pgfpathlineto{\pgfqpoint{0.649942in}{1.073327in}}%
\pgfpathlineto{\pgfqpoint{0.638230in}{1.069280in}}%
\pgfpathlineto{\pgfqpoint{0.630058in}{1.067089in}}%
\pgfpathlineto{\pgfqpoint{0.626517in}{1.066087in}}%
\pgfpathlineto{\pgfqpoint{0.614804in}{1.062979in}}%
\pgfpathlineto{\pgfqpoint{0.603091in}{1.058151in}}%
\pgfpathlineto{\pgfqpoint{0.597034in}{1.054960in}}%
\pgfpathlineto{\pgfqpoint{0.591379in}{1.051104in}}%
\pgfpathlineto{\pgfqpoint{0.579666in}{1.043430in}}%
\pgfpathlineto{\pgfqpoint{0.578771in}{1.042832in}}%
\pgfpathlineto{\pgfqpoint{0.573448in}{1.030703in}}%
\pgfpathlineto{\pgfqpoint{0.577846in}{1.018574in}}%
\pgfpathlineto{\pgfqpoint{0.579666in}{1.014061in}}%
\pgfpathlineto{\pgfqpoint{0.582656in}{1.006445in}}%
\pgfpathlineto{\pgfqpoint{0.587078in}{0.994316in}}%
\pgfpathlineto{\pgfqpoint{0.591379in}{0.982805in}}%
\pgfpathlineto{\pgfqpoint{0.591627in}{0.982187in}}%
\pgfpathlineto{\pgfqpoint{0.601882in}{0.970058in}}%
\pgfpathlineto{\pgfqpoint{0.603091in}{0.969276in}}%
\pgfpathlineto{\pgfqpoint{0.614804in}{0.963938in}}%
\pgfpathlineto{\pgfqpoint{0.626517in}{0.959251in}}%
\pgfpathlineto{\pgfqpoint{0.629711in}{0.957929in}}%
\pgfpathlineto{\pgfqpoint{0.638230in}{0.952442in}}%
\pgfpathlineto{\pgfqpoint{0.649322in}{0.945801in}}%
\pgfpathlineto{\pgfqpoint{0.649942in}{0.945418in}}%
\pgfpathlineto{\pgfqpoint{0.661655in}{0.938320in}}%
\pgfpathlineto{\pgfqpoint{0.670418in}{0.933672in}}%
\pgfpathlineto{\pgfqpoint{0.673368in}{0.931598in}}%
\pgfpathlineto{\pgfqpoint{0.683866in}{0.921543in}}%
\pgfpathlineto{\pgfqpoint{0.685080in}{0.919554in}}%
\pgfpathlineto{\pgfqpoint{0.690775in}{0.909414in}}%
\pgfpathlineto{\pgfqpoint{0.696793in}{0.899115in}}%
\pgfpathlineto{\pgfqpoint{0.697892in}{0.897285in}}%
\pgfpathlineto{\pgfqpoint{0.705907in}{0.885156in}}%
\pgfpathlineto{\pgfqpoint{0.708506in}{0.881571in}}%
\pgfpathlineto{\pgfqpoint{0.716450in}{0.873027in}}%
\pgfpathlineto{\pgfqpoint{0.720218in}{0.869389in}}%
\pgfpathlineto{\pgfqpoint{0.729927in}{0.860898in}}%
\pgfpathlineto{\pgfqpoint{0.731931in}{0.859780in}}%
\pgfpathlineto{\pgfqpoint{0.743644in}{0.853899in}}%
\pgfpathlineto{\pgfqpoint{0.753644in}{0.848770in}}%
\pgfpathlineto{\pgfqpoint{0.755356in}{0.848009in}}%
\pgfpathlineto{\pgfqpoint{0.767069in}{0.842122in}}%
\pgfpathlineto{\pgfqpoint{0.777585in}{0.836641in}}%
\pgfpathclose%
\pgfusepath{fill}%
\end{pgfscope}%
\begin{pgfscope}%
\pgfpathrectangle{\pgfqpoint{0.211875in}{0.211875in}}{\pgfqpoint{1.313625in}{1.279725in}}%
\pgfusepath{clip}%
\pgfsetbuttcap%
\pgfsetroundjoin%
\definecolor{currentfill}{rgb}{0.343142,0.118134,0.311397}%
\pgfsetfillcolor{currentfill}%
\pgfsetlinewidth{0.000000pt}%
\definecolor{currentstroke}{rgb}{0.000000,0.000000,0.000000}%
\pgfsetstrokecolor{currentstroke}%
\pgfsetdash{}{0pt}%
\pgfpathmoveto{\pgfqpoint{1.411267in}{0.929786in}}%
\pgfpathlineto{\pgfqpoint{1.422980in}{0.928115in}}%
\pgfpathlineto{\pgfqpoint{1.434692in}{0.932063in}}%
\pgfpathlineto{\pgfqpoint{1.437431in}{0.933672in}}%
\pgfpathlineto{\pgfqpoint{1.446405in}{0.939876in}}%
\pgfpathlineto{\pgfqpoint{1.446405in}{0.945801in}}%
\pgfpathlineto{\pgfqpoint{1.446405in}{0.957929in}}%
\pgfpathlineto{\pgfqpoint{1.446405in}{0.970058in}}%
\pgfpathlineto{\pgfqpoint{1.446405in}{0.982187in}}%
\pgfpathlineto{\pgfqpoint{1.446405in}{0.994316in}}%
\pgfpathlineto{\pgfqpoint{1.446405in}{0.999115in}}%
\pgfpathlineto{\pgfqpoint{1.438075in}{0.994316in}}%
\pgfpathlineto{\pgfqpoint{1.434692in}{0.991410in}}%
\pgfpathlineto{\pgfqpoint{1.423273in}{0.982187in}}%
\pgfpathlineto{\pgfqpoint{1.422980in}{0.981752in}}%
\pgfpathlineto{\pgfqpoint{1.414917in}{0.970058in}}%
\pgfpathlineto{\pgfqpoint{1.411267in}{0.964967in}}%
\pgfpathlineto{\pgfqpoint{1.407868in}{0.957929in}}%
\pgfpathlineto{\pgfqpoint{1.401609in}{0.945801in}}%
\pgfpathlineto{\pgfqpoint{1.400823in}{0.933672in}}%
\pgfpathclose%
\pgfusepath{fill}%
\end{pgfscope}%
\begin{pgfscope}%
\pgfpathrectangle{\pgfqpoint{0.211875in}{0.211875in}}{\pgfqpoint{1.313625in}{1.279725in}}%
\pgfusepath{clip}%
\pgfsetbuttcap%
\pgfsetroundjoin%
\definecolor{currentfill}{rgb}{0.343142,0.118134,0.311397}%
\pgfsetfillcolor{currentfill}%
\pgfsetlinewidth{0.000000pt}%
\definecolor{currentstroke}{rgb}{0.000000,0.000000,0.000000}%
\pgfsetstrokecolor{currentstroke}%
\pgfsetdash{}{0pt}%
\pgfpathmoveto{\pgfqpoint{1.212151in}{1.003907in}}%
\pgfpathlineto{\pgfqpoint{1.223864in}{1.004574in}}%
\pgfpathlineto{\pgfqpoint{1.227894in}{1.006445in}}%
\pgfpathlineto{\pgfqpoint{1.235577in}{1.012199in}}%
\pgfpathlineto{\pgfqpoint{1.240019in}{1.018574in}}%
\pgfpathlineto{\pgfqpoint{1.247289in}{1.029712in}}%
\pgfpathlineto{\pgfqpoint{1.247966in}{1.030703in}}%
\pgfpathlineto{\pgfqpoint{1.252316in}{1.042832in}}%
\pgfpathlineto{\pgfqpoint{1.259002in}{1.052927in}}%
\pgfpathlineto{\pgfqpoint{1.260076in}{1.054960in}}%
\pgfpathlineto{\pgfqpoint{1.267752in}{1.067089in}}%
\pgfpathlineto{\pgfqpoint{1.270715in}{1.075134in}}%
\pgfpathlineto{\pgfqpoint{1.271637in}{1.079218in}}%
\pgfpathlineto{\pgfqpoint{1.270715in}{1.082342in}}%
\pgfpathlineto{\pgfqpoint{1.262318in}{1.091347in}}%
\pgfpathlineto{\pgfqpoint{1.259002in}{1.094102in}}%
\pgfpathlineto{\pgfqpoint{1.247289in}{1.097498in}}%
\pgfpathlineto{\pgfqpoint{1.238882in}{1.091347in}}%
\pgfpathlineto{\pgfqpoint{1.235577in}{1.089170in}}%
\pgfpathlineto{\pgfqpoint{1.224464in}{1.079218in}}%
\pgfpathlineto{\pgfqpoint{1.223864in}{1.078814in}}%
\pgfpathlineto{\pgfqpoint{1.212151in}{1.076670in}}%
\pgfpathlineto{\pgfqpoint{1.204889in}{1.079218in}}%
\pgfpathlineto{\pgfqpoint{1.200439in}{1.080325in}}%
\pgfpathlineto{\pgfqpoint{1.197486in}{1.079218in}}%
\pgfpathlineto{\pgfqpoint{1.188726in}{1.074168in}}%
\pgfpathlineto{\pgfqpoint{1.179319in}{1.067089in}}%
\pgfpathlineto{\pgfqpoint{1.177013in}{1.064867in}}%
\pgfpathlineto{\pgfqpoint{1.168887in}{1.054960in}}%
\pgfpathlineto{\pgfqpoint{1.165642in}{1.042832in}}%
\pgfpathlineto{\pgfqpoint{1.168611in}{1.030703in}}%
\pgfpathlineto{\pgfqpoint{1.177013in}{1.023586in}}%
\pgfpathlineto{\pgfqpoint{1.188726in}{1.023420in}}%
\pgfpathlineto{\pgfqpoint{1.200439in}{1.019932in}}%
\pgfpathlineto{\pgfqpoint{1.202231in}{1.018574in}}%
\pgfpathlineto{\pgfqpoint{1.208828in}{1.006445in}}%
\pgfpathclose%
\pgfusepath{fill}%
\end{pgfscope}%
\begin{pgfscope}%
\pgfpathrectangle{\pgfqpoint{0.211875in}{0.211875in}}{\pgfqpoint{1.313625in}{1.279725in}}%
\pgfusepath{clip}%
\pgfsetbuttcap%
\pgfsetroundjoin%
\definecolor{currentfill}{rgb}{0.343142,0.118134,0.311397}%
\pgfsetfillcolor{currentfill}%
\pgfsetlinewidth{0.000000pt}%
\definecolor{currentstroke}{rgb}{0.000000,0.000000,0.000000}%
\pgfsetstrokecolor{currentstroke}%
\pgfsetdash{}{0pt}%
\pgfpathmoveto{\pgfqpoint{1.118450in}{1.438882in}}%
\pgfpathlineto{\pgfqpoint{1.130163in}{1.437997in}}%
\pgfpathlineto{\pgfqpoint{1.141875in}{1.442081in}}%
\pgfpathlineto{\pgfqpoint{1.144731in}{1.443084in}}%
\pgfpathlineto{\pgfqpoint{1.153588in}{1.445754in}}%
\pgfpathlineto{\pgfqpoint{1.165301in}{1.448241in}}%
\pgfpathlineto{\pgfqpoint{1.177013in}{1.450337in}}%
\pgfpathlineto{\pgfqpoint{1.184088in}{1.455213in}}%
\pgfpathlineto{\pgfqpoint{1.188726in}{1.458866in}}%
\pgfpathlineto{\pgfqpoint{1.197198in}{1.467342in}}%
\pgfpathlineto{\pgfqpoint{1.200439in}{1.471140in}}%
\pgfpathlineto{\pgfqpoint{1.209454in}{1.479471in}}%
\pgfpathlineto{\pgfqpoint{1.212151in}{1.481482in}}%
\pgfpathlineto{\pgfqpoint{1.223864in}{1.481866in}}%
\pgfpathlineto{\pgfqpoint{1.226686in}{1.479471in}}%
\pgfpathlineto{\pgfqpoint{1.235577in}{1.475225in}}%
\pgfpathlineto{\pgfqpoint{1.247289in}{1.468879in}}%
\pgfpathlineto{\pgfqpoint{1.251913in}{1.467342in}}%
\pgfpathlineto{\pgfqpoint{1.259002in}{1.464700in}}%
\pgfpathlineto{\pgfqpoint{1.270715in}{1.459991in}}%
\pgfpathlineto{\pgfqpoint{1.282427in}{1.465094in}}%
\pgfpathlineto{\pgfqpoint{1.284238in}{1.467342in}}%
\pgfpathlineto{\pgfqpoint{1.292870in}{1.479471in}}%
\pgfpathlineto{\pgfqpoint{1.294140in}{1.480999in}}%
\pgfpathlineto{\pgfqpoint{1.300415in}{1.491600in}}%
\pgfpathlineto{\pgfqpoint{1.294140in}{1.491600in}}%
\pgfpathlineto{\pgfqpoint{1.286362in}{1.491600in}}%
\pgfpathlineto{\pgfqpoint{1.282427in}{1.486440in}}%
\pgfpathlineto{\pgfqpoint{1.270715in}{1.485677in}}%
\pgfpathlineto{\pgfqpoint{1.259002in}{1.487463in}}%
\pgfpathlineto{\pgfqpoint{1.247289in}{1.490580in}}%
\pgfpathlineto{\pgfqpoint{1.245451in}{1.491600in}}%
\pgfpathlineto{\pgfqpoint{1.235577in}{1.491600in}}%
\pgfpathlineto{\pgfqpoint{1.223864in}{1.491600in}}%
\pgfpathlineto{\pgfqpoint{1.212151in}{1.491600in}}%
\pgfpathlineto{\pgfqpoint{1.200439in}{1.491600in}}%
\pgfpathlineto{\pgfqpoint{1.188726in}{1.491600in}}%
\pgfpathlineto{\pgfqpoint{1.182138in}{1.491600in}}%
\pgfpathlineto{\pgfqpoint{1.177013in}{1.486978in}}%
\pgfpathlineto{\pgfqpoint{1.168667in}{1.479471in}}%
\pgfpathlineto{\pgfqpoint{1.165301in}{1.476758in}}%
\pgfpathlineto{\pgfqpoint{1.153588in}{1.468910in}}%
\pgfpathlineto{\pgfqpoint{1.141875in}{1.469135in}}%
\pgfpathlineto{\pgfqpoint{1.130163in}{1.477078in}}%
\pgfpathlineto{\pgfqpoint{1.128569in}{1.479471in}}%
\pgfpathlineto{\pgfqpoint{1.124380in}{1.491600in}}%
\pgfpathlineto{\pgfqpoint{1.118450in}{1.491600in}}%
\pgfpathlineto{\pgfqpoint{1.106737in}{1.491600in}}%
\pgfpathlineto{\pgfqpoint{1.095024in}{1.491600in}}%
\pgfpathlineto{\pgfqpoint{1.085106in}{1.491600in}}%
\pgfpathlineto{\pgfqpoint{1.088384in}{1.479471in}}%
\pgfpathlineto{\pgfqpoint{1.092985in}{1.467342in}}%
\pgfpathlineto{\pgfqpoint{1.095024in}{1.463477in}}%
\pgfpathlineto{\pgfqpoint{1.100085in}{1.455213in}}%
\pgfpathlineto{\pgfqpoint{1.106737in}{1.447032in}}%
\pgfpathlineto{\pgfqpoint{1.110851in}{1.443084in}}%
\pgfpathclose%
\pgfusepath{fill}%
\end{pgfscope}%
\begin{pgfscope}%
\pgfpathrectangle{\pgfqpoint{0.211875in}{0.211875in}}{\pgfqpoint{1.313625in}{1.279725in}}%
\pgfusepath{clip}%
\pgfsetbuttcap%
\pgfsetroundjoin%
\definecolor{currentfill}{rgb}{0.490838,0.119982,0.351115}%
\pgfsetfillcolor{currentfill}%
\pgfsetlinewidth{0.000000pt}%
\definecolor{currentstroke}{rgb}{0.000000,0.000000,0.000000}%
\pgfsetstrokecolor{currentstroke}%
\pgfsetdash{}{0pt}%
\pgfpathmoveto{\pgfqpoint{0.415688in}{0.290841in}}%
\pgfpathlineto{\pgfqpoint{0.427401in}{0.290841in}}%
\pgfpathlineto{\pgfqpoint{0.431408in}{0.290841in}}%
\pgfpathlineto{\pgfqpoint{0.439114in}{0.296676in}}%
\pgfpathlineto{\pgfqpoint{0.446663in}{0.302970in}}%
\pgfpathlineto{\pgfqpoint{0.450827in}{0.305954in}}%
\pgfpathlineto{\pgfqpoint{0.462539in}{0.314401in}}%
\pgfpathlineto{\pgfqpoint{0.463470in}{0.315099in}}%
\pgfpathlineto{\pgfqpoint{0.474252in}{0.322552in}}%
\pgfpathlineto{\pgfqpoint{0.481093in}{0.327228in}}%
\pgfpathlineto{\pgfqpoint{0.485965in}{0.330793in}}%
\pgfpathlineto{\pgfqpoint{0.497677in}{0.338689in}}%
\pgfpathlineto{\pgfqpoint{0.498711in}{0.339357in}}%
\pgfpathlineto{\pgfqpoint{0.509390in}{0.347014in}}%
\pgfpathlineto{\pgfqpoint{0.516131in}{0.351486in}}%
\pgfpathlineto{\pgfqpoint{0.521103in}{0.355140in}}%
\pgfpathlineto{\pgfqpoint{0.532815in}{0.363353in}}%
\pgfpathlineto{\pgfqpoint{0.533262in}{0.363615in}}%
\pgfpathlineto{\pgfqpoint{0.544528in}{0.371509in}}%
\pgfpathlineto{\pgfqpoint{0.551944in}{0.375743in}}%
\pgfpathlineto{\pgfqpoint{0.556241in}{0.378678in}}%
\pgfpathlineto{\pgfqpoint{0.567953in}{0.386668in}}%
\pgfpathlineto{\pgfqpoint{0.569899in}{0.387872in}}%
\pgfpathlineto{\pgfqpoint{0.579666in}{0.393325in}}%
\pgfpathlineto{\pgfqpoint{0.591262in}{0.400001in}}%
\pgfpathlineto{\pgfqpoint{0.591379in}{0.400066in}}%
\pgfpathlineto{\pgfqpoint{0.603091in}{0.405572in}}%
\pgfpathlineto{\pgfqpoint{0.614804in}{0.411154in}}%
\pgfpathlineto{\pgfqpoint{0.617233in}{0.412130in}}%
\pgfpathlineto{\pgfqpoint{0.626517in}{0.415587in}}%
\pgfpathlineto{\pgfqpoint{0.638230in}{0.417960in}}%
\pgfpathlineto{\pgfqpoint{0.649942in}{0.417754in}}%
\pgfpathlineto{\pgfqpoint{0.660582in}{0.412130in}}%
\pgfpathlineto{\pgfqpoint{0.661065in}{0.400001in}}%
\pgfpathlineto{\pgfqpoint{0.655715in}{0.387872in}}%
\pgfpathlineto{\pgfqpoint{0.650879in}{0.375743in}}%
\pgfpathlineto{\pgfqpoint{0.655812in}{0.363615in}}%
\pgfpathlineto{\pgfqpoint{0.661655in}{0.353230in}}%
\pgfpathlineto{\pgfqpoint{0.662681in}{0.351486in}}%
\pgfpathlineto{\pgfqpoint{0.667514in}{0.339357in}}%
\pgfpathlineto{\pgfqpoint{0.665906in}{0.327228in}}%
\pgfpathlineto{\pgfqpoint{0.661655in}{0.321384in}}%
\pgfpathlineto{\pgfqpoint{0.656688in}{0.315099in}}%
\pgfpathlineto{\pgfqpoint{0.649942in}{0.309072in}}%
\pgfpathlineto{\pgfqpoint{0.642916in}{0.302970in}}%
\pgfpathlineto{\pgfqpoint{0.638230in}{0.299558in}}%
\pgfpathlineto{\pgfqpoint{0.626517in}{0.291272in}}%
\pgfpathlineto{\pgfqpoint{0.625945in}{0.290841in}}%
\pgfpathlineto{\pgfqpoint{0.626517in}{0.290841in}}%
\pgfpathlineto{\pgfqpoint{0.638230in}{0.290841in}}%
\pgfpathlineto{\pgfqpoint{0.649942in}{0.290841in}}%
\pgfpathlineto{\pgfqpoint{0.661655in}{0.290841in}}%
\pgfpathlineto{\pgfqpoint{0.673368in}{0.290841in}}%
\pgfpathlineto{\pgfqpoint{0.685080in}{0.290841in}}%
\pgfpathlineto{\pgfqpoint{0.696793in}{0.290841in}}%
\pgfpathlineto{\pgfqpoint{0.708506in}{0.290841in}}%
\pgfpathlineto{\pgfqpoint{0.720218in}{0.290841in}}%
\pgfpathlineto{\pgfqpoint{0.731931in}{0.290841in}}%
\pgfpathlineto{\pgfqpoint{0.743644in}{0.290841in}}%
\pgfpathlineto{\pgfqpoint{0.755356in}{0.290841in}}%
\pgfpathlineto{\pgfqpoint{0.767069in}{0.290841in}}%
\pgfpathlineto{\pgfqpoint{0.778782in}{0.290841in}}%
\pgfpathlineto{\pgfqpoint{0.790495in}{0.290841in}}%
\pgfpathlineto{\pgfqpoint{0.802207in}{0.290841in}}%
\pgfpathlineto{\pgfqpoint{0.813781in}{0.290841in}}%
\pgfpathlineto{\pgfqpoint{0.802207in}{0.301440in}}%
\pgfpathlineto{\pgfqpoint{0.800654in}{0.302970in}}%
\pgfpathlineto{\pgfqpoint{0.790495in}{0.311319in}}%
\pgfpathlineto{\pgfqpoint{0.783996in}{0.315099in}}%
\pgfpathlineto{\pgfqpoint{0.778782in}{0.319534in}}%
\pgfpathlineto{\pgfqpoint{0.769898in}{0.327228in}}%
\pgfpathlineto{\pgfqpoint{0.767069in}{0.329553in}}%
\pgfpathlineto{\pgfqpoint{0.755356in}{0.337787in}}%
\pgfpathlineto{\pgfqpoint{0.753204in}{0.339357in}}%
\pgfpathlineto{\pgfqpoint{0.743644in}{0.344808in}}%
\pgfpathlineto{\pgfqpoint{0.731931in}{0.350850in}}%
\pgfpathlineto{\pgfqpoint{0.730905in}{0.351486in}}%
\pgfpathlineto{\pgfqpoint{0.720218in}{0.360168in}}%
\pgfpathlineto{\pgfqpoint{0.716828in}{0.363615in}}%
\pgfpathlineto{\pgfqpoint{0.720218in}{0.368448in}}%
\pgfpathlineto{\pgfqpoint{0.729090in}{0.375743in}}%
\pgfpathlineto{\pgfqpoint{0.731931in}{0.377723in}}%
\pgfpathlineto{\pgfqpoint{0.743644in}{0.386042in}}%
\pgfpathlineto{\pgfqpoint{0.745681in}{0.387872in}}%
\pgfpathlineto{\pgfqpoint{0.755009in}{0.400001in}}%
\pgfpathlineto{\pgfqpoint{0.755356in}{0.400725in}}%
\pgfpathlineto{\pgfqpoint{0.759116in}{0.412130in}}%
\pgfpathlineto{\pgfqpoint{0.760155in}{0.424259in}}%
\pgfpathlineto{\pgfqpoint{0.756327in}{0.436388in}}%
\pgfpathlineto{\pgfqpoint{0.755356in}{0.437658in}}%
\pgfpathlineto{\pgfqpoint{0.747050in}{0.448517in}}%
\pgfpathlineto{\pgfqpoint{0.743644in}{0.451912in}}%
\pgfpathlineto{\pgfqpoint{0.731931in}{0.460404in}}%
\pgfpathlineto{\pgfqpoint{0.731314in}{0.460646in}}%
\pgfpathlineto{\pgfqpoint{0.720218in}{0.463411in}}%
\pgfpathlineto{\pgfqpoint{0.708506in}{0.465919in}}%
\pgfpathlineto{\pgfqpoint{0.696793in}{0.466868in}}%
\pgfpathlineto{\pgfqpoint{0.685080in}{0.466894in}}%
\pgfpathlineto{\pgfqpoint{0.673368in}{0.465769in}}%
\pgfpathlineto{\pgfqpoint{0.661655in}{0.462706in}}%
\pgfpathlineto{\pgfqpoint{0.656823in}{0.460646in}}%
\pgfpathlineto{\pgfqpoint{0.649942in}{0.458016in}}%
\pgfpathlineto{\pgfqpoint{0.638230in}{0.452741in}}%
\pgfpathlineto{\pgfqpoint{0.629146in}{0.448517in}}%
\pgfpathlineto{\pgfqpoint{0.626517in}{0.447335in}}%
\pgfpathlineto{\pgfqpoint{0.614804in}{0.440364in}}%
\pgfpathlineto{\pgfqpoint{0.608639in}{0.436388in}}%
\pgfpathlineto{\pgfqpoint{0.603091in}{0.432971in}}%
\pgfpathlineto{\pgfqpoint{0.591379in}{0.424720in}}%
\pgfpathlineto{\pgfqpoint{0.590746in}{0.424259in}}%
\pgfpathlineto{\pgfqpoint{0.579666in}{0.416900in}}%
\pgfpathlineto{\pgfqpoint{0.573439in}{0.412130in}}%
\pgfpathlineto{\pgfqpoint{0.567953in}{0.408944in}}%
\pgfpathlineto{\pgfqpoint{0.556241in}{0.400174in}}%
\pgfpathlineto{\pgfqpoint{0.556045in}{0.400001in}}%
\pgfpathlineto{\pgfqpoint{0.544528in}{0.392558in}}%
\pgfpathlineto{\pgfqpoint{0.539099in}{0.387872in}}%
\pgfpathlineto{\pgfqpoint{0.532815in}{0.382691in}}%
\pgfpathlineto{\pgfqpoint{0.524668in}{0.375743in}}%
\pgfpathlineto{\pgfqpoint{0.521103in}{0.372848in}}%
\pgfpathlineto{\pgfqpoint{0.510469in}{0.363615in}}%
\pgfpathlineto{\pgfqpoint{0.509390in}{0.362829in}}%
\pgfpathlineto{\pgfqpoint{0.497677in}{0.353975in}}%
\pgfpathlineto{\pgfqpoint{0.494703in}{0.351486in}}%
\pgfpathlineto{\pgfqpoint{0.485965in}{0.345032in}}%
\pgfpathlineto{\pgfqpoint{0.478783in}{0.339357in}}%
\pgfpathlineto{\pgfqpoint{0.474252in}{0.336168in}}%
\pgfpathlineto{\pgfqpoint{0.462694in}{0.327228in}}%
\pgfpathlineto{\pgfqpoint{0.462539in}{0.327122in}}%
\pgfpathlineto{\pgfqpoint{0.450827in}{0.318674in}}%
\pgfpathlineto{\pgfqpoint{0.446331in}{0.315099in}}%
\pgfpathlineto{\pgfqpoint{0.439114in}{0.309981in}}%
\pgfpathlineto{\pgfqpoint{0.429539in}{0.302970in}}%
\pgfpathlineto{\pgfqpoint{0.427401in}{0.301312in}}%
\pgfpathlineto{\pgfqpoint{0.415688in}{0.292757in}}%
\pgfpathlineto{\pgfqpoint{0.413031in}{0.290841in}}%
\pgfpathclose%
\pgfusepath{fill}%
\end{pgfscope}%
\begin{pgfscope}%
\pgfpathrectangle{\pgfqpoint{0.211875in}{0.211875in}}{\pgfqpoint{1.313625in}{1.279725in}}%
\pgfusepath{clip}%
\pgfsetbuttcap%
\pgfsetroundjoin%
\definecolor{currentfill}{rgb}{0.490838,0.119982,0.351115}%
\pgfsetfillcolor{currentfill}%
\pgfsetlinewidth{0.000000pt}%
\definecolor{currentstroke}{rgb}{0.000000,0.000000,0.000000}%
\pgfsetstrokecolor{currentstroke}%
\pgfsetdash{}{0pt}%
\pgfpathmoveto{\pgfqpoint{0.837345in}{0.290841in}}%
\pgfpathlineto{\pgfqpoint{0.849058in}{0.290841in}}%
\pgfpathlineto{\pgfqpoint{0.860771in}{0.290841in}}%
\pgfpathlineto{\pgfqpoint{0.872483in}{0.290841in}}%
\pgfpathlineto{\pgfqpoint{0.884196in}{0.290841in}}%
\pgfpathlineto{\pgfqpoint{0.895909in}{0.290841in}}%
\pgfpathlineto{\pgfqpoint{0.907621in}{0.290841in}}%
\pgfpathlineto{\pgfqpoint{0.919334in}{0.290841in}}%
\pgfpathlineto{\pgfqpoint{0.931047in}{0.290841in}}%
\pgfpathlineto{\pgfqpoint{0.942759in}{0.290841in}}%
\pgfpathlineto{\pgfqpoint{0.954472in}{0.290841in}}%
\pgfpathlineto{\pgfqpoint{0.966185in}{0.290841in}}%
\pgfpathlineto{\pgfqpoint{0.977898in}{0.290841in}}%
\pgfpathlineto{\pgfqpoint{0.989610in}{0.290841in}}%
\pgfpathlineto{\pgfqpoint{1.001323in}{0.290841in}}%
\pgfpathlineto{\pgfqpoint{1.005974in}{0.290841in}}%
\pgfpathlineto{\pgfqpoint{1.001323in}{0.296480in}}%
\pgfpathlineto{\pgfqpoint{0.991000in}{0.302970in}}%
\pgfpathlineto{\pgfqpoint{0.989610in}{0.303792in}}%
\pgfpathlineto{\pgfqpoint{0.977898in}{0.305046in}}%
\pgfpathlineto{\pgfqpoint{0.966185in}{0.306373in}}%
\pgfpathlineto{\pgfqpoint{0.954472in}{0.307817in}}%
\pgfpathlineto{\pgfqpoint{0.942759in}{0.308803in}}%
\pgfpathlineto{\pgfqpoint{0.931047in}{0.307775in}}%
\pgfpathlineto{\pgfqpoint{0.919334in}{0.306535in}}%
\pgfpathlineto{\pgfqpoint{0.907621in}{0.305087in}}%
\pgfpathlineto{\pgfqpoint{0.895909in}{0.303513in}}%
\pgfpathlineto{\pgfqpoint{0.892033in}{0.302970in}}%
\pgfpathlineto{\pgfqpoint{0.884196in}{0.301675in}}%
\pgfpathlineto{\pgfqpoint{0.872483in}{0.299351in}}%
\pgfpathlineto{\pgfqpoint{0.860771in}{0.296759in}}%
\pgfpathlineto{\pgfqpoint{0.849058in}{0.294011in}}%
\pgfpathlineto{\pgfqpoint{0.837345in}{0.291060in}}%
\pgfpathlineto{\pgfqpoint{0.836561in}{0.290841in}}%
\pgfpathclose%
\pgfusepath{fill}%
\end{pgfscope}%
\begin{pgfscope}%
\pgfpathrectangle{\pgfqpoint{0.211875in}{0.211875in}}{\pgfqpoint{1.313625in}{1.279725in}}%
\pgfusepath{clip}%
\pgfsetbuttcap%
\pgfsetroundjoin%
\definecolor{currentfill}{rgb}{0.490838,0.119982,0.351115}%
\pgfsetfillcolor{currentfill}%
\pgfsetlinewidth{0.000000pt}%
\definecolor{currentstroke}{rgb}{0.000000,0.000000,0.000000}%
\pgfsetstrokecolor{currentstroke}%
\pgfsetdash{}{0pt}%
\pgfpathmoveto{\pgfqpoint{1.118450in}{0.290841in}}%
\pgfpathlineto{\pgfqpoint{1.130163in}{0.290841in}}%
\pgfpathlineto{\pgfqpoint{1.141875in}{0.290841in}}%
\pgfpathlineto{\pgfqpoint{1.153588in}{0.290841in}}%
\pgfpathlineto{\pgfqpoint{1.165301in}{0.290841in}}%
\pgfpathlineto{\pgfqpoint{1.177013in}{0.290841in}}%
\pgfpathlineto{\pgfqpoint{1.188726in}{0.290841in}}%
\pgfpathlineto{\pgfqpoint{1.200439in}{0.290841in}}%
\pgfpathlineto{\pgfqpoint{1.212151in}{0.290841in}}%
\pgfpathlineto{\pgfqpoint{1.223864in}{0.290841in}}%
\pgfpathlineto{\pgfqpoint{1.235577in}{0.290841in}}%
\pgfpathlineto{\pgfqpoint{1.247289in}{0.290841in}}%
\pgfpathlineto{\pgfqpoint{1.259002in}{0.290841in}}%
\pgfpathlineto{\pgfqpoint{1.270715in}{0.290841in}}%
\pgfpathlineto{\pgfqpoint{1.276969in}{0.290841in}}%
\pgfpathlineto{\pgfqpoint{1.277038in}{0.302970in}}%
\pgfpathlineto{\pgfqpoint{1.270715in}{0.314520in}}%
\pgfpathlineto{\pgfqpoint{1.270125in}{0.315099in}}%
\pgfpathlineto{\pgfqpoint{1.259002in}{0.317962in}}%
\pgfpathlineto{\pgfqpoint{1.247289in}{0.318365in}}%
\pgfpathlineto{\pgfqpoint{1.235577in}{0.317254in}}%
\pgfpathlineto{\pgfqpoint{1.223864in}{0.315453in}}%
\pgfpathlineto{\pgfqpoint{1.222178in}{0.315099in}}%
\pgfpathlineto{\pgfqpoint{1.212151in}{0.311975in}}%
\pgfpathlineto{\pgfqpoint{1.200439in}{0.308327in}}%
\pgfpathlineto{\pgfqpoint{1.188726in}{0.305088in}}%
\pgfpathlineto{\pgfqpoint{1.177013in}{0.303039in}}%
\pgfpathlineto{\pgfqpoint{1.176169in}{0.302970in}}%
\pgfpathlineto{\pgfqpoint{1.165301in}{0.301641in}}%
\pgfpathlineto{\pgfqpoint{1.153588in}{0.300544in}}%
\pgfpathlineto{\pgfqpoint{1.141875in}{0.298919in}}%
\pgfpathlineto{\pgfqpoint{1.130163in}{0.296913in}}%
\pgfpathlineto{\pgfqpoint{1.118450in}{0.293504in}}%
\pgfpathlineto{\pgfqpoint{1.111969in}{0.290841in}}%
\pgfpathclose%
\pgfusepath{fill}%
\end{pgfscope}%
\begin{pgfscope}%
\pgfpathrectangle{\pgfqpoint{0.211875in}{0.211875in}}{\pgfqpoint{1.313625in}{1.279725in}}%
\pgfusepath{clip}%
\pgfsetbuttcap%
\pgfsetroundjoin%
\definecolor{currentfill}{rgb}{0.490838,0.119982,0.351115}%
\pgfsetfillcolor{currentfill}%
\pgfsetlinewidth{0.000000pt}%
\definecolor{currentstroke}{rgb}{0.000000,0.000000,0.000000}%
\pgfsetstrokecolor{currentstroke}%
\pgfsetdash{}{0pt}%
\pgfpathmoveto{\pgfqpoint{0.954472in}{0.495173in}}%
\pgfpathlineto{\pgfqpoint{0.966185in}{0.489190in}}%
\pgfpathlineto{\pgfqpoint{0.977898in}{0.488764in}}%
\pgfpathlineto{\pgfqpoint{0.989610in}{0.491002in}}%
\pgfpathlineto{\pgfqpoint{1.001323in}{0.493053in}}%
\pgfpathlineto{\pgfqpoint{1.013036in}{0.493166in}}%
\pgfpathlineto{\pgfqpoint{1.024748in}{0.492956in}}%
\pgfpathlineto{\pgfqpoint{1.036461in}{0.492616in}}%
\pgfpathlineto{\pgfqpoint{1.048174in}{0.492310in}}%
\pgfpathlineto{\pgfqpoint{1.059886in}{0.492145in}}%
\pgfpathlineto{\pgfqpoint{1.071599in}{0.491578in}}%
\pgfpathlineto{\pgfqpoint{1.083312in}{0.490754in}}%
\pgfpathlineto{\pgfqpoint{1.095024in}{0.490331in}}%
\pgfpathlineto{\pgfqpoint{1.106737in}{0.491372in}}%
\pgfpathlineto{\pgfqpoint{1.118450in}{0.496359in}}%
\pgfpathlineto{\pgfqpoint{1.119249in}{0.497032in}}%
\pgfpathlineto{\pgfqpoint{1.130163in}{0.507275in}}%
\pgfpathlineto{\pgfqpoint{1.131976in}{0.509161in}}%
\pgfpathlineto{\pgfqpoint{1.141875in}{0.520865in}}%
\pgfpathlineto{\pgfqpoint{1.142230in}{0.521290in}}%
\pgfpathlineto{\pgfqpoint{1.150324in}{0.533419in}}%
\pgfpathlineto{\pgfqpoint{1.153588in}{0.540071in}}%
\pgfpathlineto{\pgfqpoint{1.156748in}{0.545548in}}%
\pgfpathlineto{\pgfqpoint{1.159637in}{0.557677in}}%
\pgfpathlineto{\pgfqpoint{1.159896in}{0.569805in}}%
\pgfpathlineto{\pgfqpoint{1.158715in}{0.581934in}}%
\pgfpathlineto{\pgfqpoint{1.154777in}{0.594063in}}%
\pgfpathlineto{\pgfqpoint{1.153588in}{0.595529in}}%
\pgfpathlineto{\pgfqpoint{1.144972in}{0.606192in}}%
\pgfpathlineto{\pgfqpoint{1.143043in}{0.618321in}}%
\pgfpathlineto{\pgfqpoint{1.141875in}{0.623633in}}%
\pgfpathlineto{\pgfqpoint{1.140427in}{0.630450in}}%
\pgfpathlineto{\pgfqpoint{1.131236in}{0.642579in}}%
\pgfpathlineto{\pgfqpoint{1.130163in}{0.643204in}}%
\pgfpathlineto{\pgfqpoint{1.118450in}{0.652839in}}%
\pgfpathlineto{\pgfqpoint{1.116810in}{0.654708in}}%
\pgfpathlineto{\pgfqpoint{1.113739in}{0.666836in}}%
\pgfpathlineto{\pgfqpoint{1.107291in}{0.678965in}}%
\pgfpathlineto{\pgfqpoint{1.106737in}{0.681129in}}%
\pgfpathlineto{\pgfqpoint{1.103871in}{0.691094in}}%
\pgfpathlineto{\pgfqpoint{1.095024in}{0.700521in}}%
\pgfpathlineto{\pgfqpoint{1.092520in}{0.703223in}}%
\pgfpathlineto{\pgfqpoint{1.090093in}{0.715352in}}%
\pgfpathlineto{\pgfqpoint{1.095024in}{0.726107in}}%
\pgfpathlineto{\pgfqpoint{1.095676in}{0.727481in}}%
\pgfpathlineto{\pgfqpoint{1.100776in}{0.739610in}}%
\pgfpathlineto{\pgfqpoint{1.104057in}{0.751739in}}%
\pgfpathlineto{\pgfqpoint{1.095024in}{0.760503in}}%
\pgfpathlineto{\pgfqpoint{1.091650in}{0.763867in}}%
\pgfpathlineto{\pgfqpoint{1.083312in}{0.769214in}}%
\pgfpathlineto{\pgfqpoint{1.071599in}{0.773632in}}%
\pgfpathlineto{\pgfqpoint{1.059886in}{0.772208in}}%
\pgfpathlineto{\pgfqpoint{1.048174in}{0.772499in}}%
\pgfpathlineto{\pgfqpoint{1.036461in}{0.774050in}}%
\pgfpathlineto{\pgfqpoint{1.024748in}{0.772783in}}%
\pgfpathlineto{\pgfqpoint{1.018143in}{0.775996in}}%
\pgfpathlineto{\pgfqpoint{1.013036in}{0.781005in}}%
\pgfpathlineto{\pgfqpoint{1.007957in}{0.788125in}}%
\pgfpathlineto{\pgfqpoint{1.004065in}{0.800254in}}%
\pgfpathlineto{\pgfqpoint{1.006385in}{0.812383in}}%
\pgfpathlineto{\pgfqpoint{1.013036in}{0.821624in}}%
\pgfpathlineto{\pgfqpoint{1.014790in}{0.824512in}}%
\pgfpathlineto{\pgfqpoint{1.022197in}{0.836641in}}%
\pgfpathlineto{\pgfqpoint{1.024748in}{0.842567in}}%
\pgfpathlineto{\pgfqpoint{1.026880in}{0.848770in}}%
\pgfpathlineto{\pgfqpoint{1.026824in}{0.860898in}}%
\pgfpathlineto{\pgfqpoint{1.024748in}{0.868096in}}%
\pgfpathlineto{\pgfqpoint{1.013055in}{0.873027in}}%
\pgfpathlineto{\pgfqpoint{1.013036in}{0.873030in}}%
\pgfpathlineto{\pgfqpoint{1.013020in}{0.873027in}}%
\pgfpathlineto{\pgfqpoint{1.001323in}{0.870445in}}%
\pgfpathlineto{\pgfqpoint{0.989610in}{0.866026in}}%
\pgfpathlineto{\pgfqpoint{0.977898in}{0.861747in}}%
\pgfpathlineto{\pgfqpoint{0.976600in}{0.860898in}}%
\pgfpathlineto{\pgfqpoint{0.966185in}{0.851521in}}%
\pgfpathlineto{\pgfqpoint{0.958820in}{0.848770in}}%
\pgfpathlineto{\pgfqpoint{0.954472in}{0.847009in}}%
\pgfpathlineto{\pgfqpoint{0.942759in}{0.840418in}}%
\pgfpathlineto{\pgfqpoint{0.937415in}{0.836641in}}%
\pgfpathlineto{\pgfqpoint{0.931047in}{0.831585in}}%
\pgfpathlineto{\pgfqpoint{0.919667in}{0.824512in}}%
\pgfpathlineto{\pgfqpoint{0.919334in}{0.824169in}}%
\pgfpathlineto{\pgfqpoint{0.908300in}{0.812383in}}%
\pgfpathlineto{\pgfqpoint{0.907621in}{0.810540in}}%
\pgfpathlineto{\pgfqpoint{0.904796in}{0.800254in}}%
\pgfpathlineto{\pgfqpoint{0.907621in}{0.794460in}}%
\pgfpathlineto{\pgfqpoint{0.912591in}{0.788125in}}%
\pgfpathlineto{\pgfqpoint{0.915175in}{0.775996in}}%
\pgfpathlineto{\pgfqpoint{0.919334in}{0.768828in}}%
\pgfpathlineto{\pgfqpoint{0.924428in}{0.763867in}}%
\pgfpathlineto{\pgfqpoint{0.922092in}{0.751739in}}%
\pgfpathlineto{\pgfqpoint{0.919334in}{0.748462in}}%
\pgfpathlineto{\pgfqpoint{0.915143in}{0.739610in}}%
\pgfpathlineto{\pgfqpoint{0.919334in}{0.735107in}}%
\pgfpathlineto{\pgfqpoint{0.929684in}{0.727481in}}%
\pgfpathlineto{\pgfqpoint{0.931047in}{0.725383in}}%
\pgfpathlineto{\pgfqpoint{0.942759in}{0.721126in}}%
\pgfpathlineto{\pgfqpoint{0.950605in}{0.715352in}}%
\pgfpathlineto{\pgfqpoint{0.942759in}{0.712967in}}%
\pgfpathlineto{\pgfqpoint{0.931047in}{0.713673in}}%
\pgfpathlineto{\pgfqpoint{0.920649in}{0.715352in}}%
\pgfpathlineto{\pgfqpoint{0.919334in}{0.716247in}}%
\pgfpathlineto{\pgfqpoint{0.907621in}{0.717585in}}%
\pgfpathlineto{\pgfqpoint{0.902025in}{0.715352in}}%
\pgfpathlineto{\pgfqpoint{0.895909in}{0.714327in}}%
\pgfpathlineto{\pgfqpoint{0.884196in}{0.712582in}}%
\pgfpathlineto{\pgfqpoint{0.872483in}{0.706939in}}%
\pgfpathlineto{\pgfqpoint{0.866145in}{0.703223in}}%
\pgfpathlineto{\pgfqpoint{0.862240in}{0.691094in}}%
\pgfpathlineto{\pgfqpoint{0.872483in}{0.685324in}}%
\pgfpathlineto{\pgfqpoint{0.880144in}{0.678965in}}%
\pgfpathlineto{\pgfqpoint{0.884196in}{0.676615in}}%
\pgfpathlineto{\pgfqpoint{0.895909in}{0.667515in}}%
\pgfpathlineto{\pgfqpoint{0.896463in}{0.666836in}}%
\pgfpathlineto{\pgfqpoint{0.896756in}{0.654708in}}%
\pgfpathlineto{\pgfqpoint{0.896805in}{0.642579in}}%
\pgfpathlineto{\pgfqpoint{0.898958in}{0.630450in}}%
\pgfpathlineto{\pgfqpoint{0.902837in}{0.618321in}}%
\pgfpathlineto{\pgfqpoint{0.907047in}{0.606192in}}%
\pgfpathlineto{\pgfqpoint{0.907621in}{0.605095in}}%
\pgfpathlineto{\pgfqpoint{0.917223in}{0.594063in}}%
\pgfpathlineto{\pgfqpoint{0.919334in}{0.591600in}}%
\pgfpathlineto{\pgfqpoint{0.927191in}{0.581934in}}%
\pgfpathlineto{\pgfqpoint{0.931047in}{0.578028in}}%
\pgfpathlineto{\pgfqpoint{0.939289in}{0.569805in}}%
\pgfpathlineto{\pgfqpoint{0.939678in}{0.557677in}}%
\pgfpathlineto{\pgfqpoint{0.941272in}{0.545548in}}%
\pgfpathlineto{\pgfqpoint{0.942040in}{0.533419in}}%
\pgfpathlineto{\pgfqpoint{0.939971in}{0.521290in}}%
\pgfpathlineto{\pgfqpoint{0.940740in}{0.509161in}}%
\pgfpathlineto{\pgfqpoint{0.942759in}{0.505875in}}%
\pgfpathlineto{\pgfqpoint{0.951165in}{0.497032in}}%
\pgfpathclose%
\pgfpathmoveto{\pgfqpoint{1.049124in}{0.569805in}}%
\pgfpathlineto{\pgfqpoint{1.048174in}{0.570334in}}%
\pgfpathlineto{\pgfqpoint{1.043895in}{0.581934in}}%
\pgfpathlineto{\pgfqpoint{1.042772in}{0.594063in}}%
\pgfpathlineto{\pgfqpoint{1.048174in}{0.603704in}}%
\pgfpathlineto{\pgfqpoint{1.052413in}{0.606192in}}%
\pgfpathlineto{\pgfqpoint{1.059886in}{0.608702in}}%
\pgfpathlineto{\pgfqpoint{1.071517in}{0.606192in}}%
\pgfpathlineto{\pgfqpoint{1.071599in}{0.606162in}}%
\pgfpathlineto{\pgfqpoint{1.083312in}{0.603160in}}%
\pgfpathlineto{\pgfqpoint{1.095024in}{0.605221in}}%
\pgfpathlineto{\pgfqpoint{1.106737in}{0.605419in}}%
\pgfpathlineto{\pgfqpoint{1.112395in}{0.594063in}}%
\pgfpathlineto{\pgfqpoint{1.115490in}{0.581934in}}%
\pgfpathlineto{\pgfqpoint{1.110805in}{0.569805in}}%
\pgfpathlineto{\pgfqpoint{1.106737in}{0.560556in}}%
\pgfpathlineto{\pgfqpoint{1.095024in}{0.560578in}}%
\pgfpathlineto{\pgfqpoint{1.083312in}{0.562031in}}%
\pgfpathlineto{\pgfqpoint{1.071599in}{0.563541in}}%
\pgfpathlineto{\pgfqpoint{1.059886in}{0.563698in}}%
\pgfpathclose%
\pgfpathmoveto{\pgfqpoint{0.988588in}{0.618321in}}%
\pgfpathlineto{\pgfqpoint{0.977898in}{0.619950in}}%
\pgfpathlineto{\pgfqpoint{0.966185in}{0.621448in}}%
\pgfpathlineto{\pgfqpoint{0.954472in}{0.623547in}}%
\pgfpathlineto{\pgfqpoint{0.942759in}{0.626797in}}%
\pgfpathlineto{\pgfqpoint{0.937710in}{0.630450in}}%
\pgfpathlineto{\pgfqpoint{0.931313in}{0.642579in}}%
\pgfpathlineto{\pgfqpoint{0.931047in}{0.643174in}}%
\pgfpathlineto{\pgfqpoint{0.928274in}{0.654708in}}%
\pgfpathlineto{\pgfqpoint{0.919334in}{0.664740in}}%
\pgfpathlineto{\pgfqpoint{0.918293in}{0.666836in}}%
\pgfpathlineto{\pgfqpoint{0.914614in}{0.678965in}}%
\pgfpathlineto{\pgfqpoint{0.919334in}{0.685590in}}%
\pgfpathlineto{\pgfqpoint{0.931047in}{0.689561in}}%
\pgfpathlineto{\pgfqpoint{0.934887in}{0.691094in}}%
\pgfpathlineto{\pgfqpoint{0.942759in}{0.693087in}}%
\pgfpathlineto{\pgfqpoint{0.954472in}{0.695482in}}%
\pgfpathlineto{\pgfqpoint{0.966185in}{0.698933in}}%
\pgfpathlineto{\pgfqpoint{0.976766in}{0.703223in}}%
\pgfpathlineto{\pgfqpoint{0.977898in}{0.703768in}}%
\pgfpathlineto{\pgfqpoint{0.989610in}{0.710299in}}%
\pgfpathlineto{\pgfqpoint{0.999332in}{0.715352in}}%
\pgfpathlineto{\pgfqpoint{1.001323in}{0.716616in}}%
\pgfpathlineto{\pgfqpoint{1.013036in}{0.720677in}}%
\pgfpathlineto{\pgfqpoint{1.024748in}{0.721297in}}%
\pgfpathlineto{\pgfqpoint{1.036461in}{0.720726in}}%
\pgfpathlineto{\pgfqpoint{1.048174in}{0.719600in}}%
\pgfpathlineto{\pgfqpoint{1.056786in}{0.715352in}}%
\pgfpathlineto{\pgfqpoint{1.059886in}{0.713294in}}%
\pgfpathlineto{\pgfqpoint{1.066999in}{0.703223in}}%
\pgfpathlineto{\pgfqpoint{1.071599in}{0.696026in}}%
\pgfpathlineto{\pgfqpoint{1.074584in}{0.691094in}}%
\pgfpathlineto{\pgfqpoint{1.073598in}{0.678965in}}%
\pgfpathlineto{\pgfqpoint{1.071692in}{0.666836in}}%
\pgfpathlineto{\pgfqpoint{1.078002in}{0.654708in}}%
\pgfpathlineto{\pgfqpoint{1.083312in}{0.643934in}}%
\pgfpathlineto{\pgfqpoint{1.084156in}{0.642579in}}%
\pgfpathlineto{\pgfqpoint{1.083312in}{0.637781in}}%
\pgfpathlineto{\pgfqpoint{1.075175in}{0.630450in}}%
\pgfpathlineto{\pgfqpoint{1.071599in}{0.628441in}}%
\pgfpathlineto{\pgfqpoint{1.059886in}{0.624266in}}%
\pgfpathlineto{\pgfqpoint{1.048174in}{0.623642in}}%
\pgfpathlineto{\pgfqpoint{1.036461in}{0.624783in}}%
\pgfpathlineto{\pgfqpoint{1.024748in}{0.622586in}}%
\pgfpathlineto{\pgfqpoint{1.013036in}{0.620563in}}%
\pgfpathlineto{\pgfqpoint{1.001553in}{0.618321in}}%
\pgfpathlineto{\pgfqpoint{1.001323in}{0.618277in}}%
\pgfpathlineto{\pgfqpoint{0.989610in}{0.618212in}}%
\pgfpathclose%
\pgfusepath{fill}%
\end{pgfscope}%
\begin{pgfscope}%
\pgfpathrectangle{\pgfqpoint{0.211875in}{0.211875in}}{\pgfqpoint{1.313625in}{1.279725in}}%
\pgfusepath{clip}%
\pgfsetbuttcap%
\pgfsetroundjoin%
\definecolor{currentfill}{rgb}{0.490838,0.119982,0.351115}%
\pgfsetfillcolor{currentfill}%
\pgfsetlinewidth{0.000000pt}%
\definecolor{currentstroke}{rgb}{0.000000,0.000000,0.000000}%
\pgfsetstrokecolor{currentstroke}%
\pgfsetdash{}{0pt}%
\pgfpathmoveto{\pgfqpoint{0.392263in}{0.539431in}}%
\pgfpathlineto{\pgfqpoint{0.403976in}{0.536517in}}%
\pgfpathlineto{\pgfqpoint{0.415688in}{0.534803in}}%
\pgfpathlineto{\pgfqpoint{0.427401in}{0.537187in}}%
\pgfpathlineto{\pgfqpoint{0.439114in}{0.539326in}}%
\pgfpathlineto{\pgfqpoint{0.450827in}{0.544762in}}%
\pgfpathlineto{\pgfqpoint{0.453079in}{0.545548in}}%
\pgfpathlineto{\pgfqpoint{0.462539in}{0.548031in}}%
\pgfpathlineto{\pgfqpoint{0.474252in}{0.552672in}}%
\pgfpathlineto{\pgfqpoint{0.485965in}{0.556744in}}%
\pgfpathlineto{\pgfqpoint{0.488508in}{0.557677in}}%
\pgfpathlineto{\pgfqpoint{0.497677in}{0.560976in}}%
\pgfpathlineto{\pgfqpoint{0.509390in}{0.565660in}}%
\pgfpathlineto{\pgfqpoint{0.519676in}{0.569805in}}%
\pgfpathlineto{\pgfqpoint{0.521103in}{0.570384in}}%
\pgfpathlineto{\pgfqpoint{0.532815in}{0.575721in}}%
\pgfpathlineto{\pgfqpoint{0.544528in}{0.581098in}}%
\pgfpathlineto{\pgfqpoint{0.546123in}{0.581934in}}%
\pgfpathlineto{\pgfqpoint{0.556241in}{0.587040in}}%
\pgfpathlineto{\pgfqpoint{0.567953in}{0.593392in}}%
\pgfpathlineto{\pgfqpoint{0.568902in}{0.594063in}}%
\pgfpathlineto{\pgfqpoint{0.579666in}{0.600596in}}%
\pgfpathlineto{\pgfqpoint{0.585235in}{0.606192in}}%
\pgfpathlineto{\pgfqpoint{0.591379in}{0.611749in}}%
\pgfpathlineto{\pgfqpoint{0.597054in}{0.618321in}}%
\pgfpathlineto{\pgfqpoint{0.603091in}{0.624116in}}%
\pgfpathlineto{\pgfqpoint{0.608246in}{0.630450in}}%
\pgfpathlineto{\pgfqpoint{0.614804in}{0.636487in}}%
\pgfpathlineto{\pgfqpoint{0.620230in}{0.642579in}}%
\pgfpathlineto{\pgfqpoint{0.626517in}{0.648786in}}%
\pgfpathlineto{\pgfqpoint{0.631602in}{0.654708in}}%
\pgfpathlineto{\pgfqpoint{0.638230in}{0.662258in}}%
\pgfpathlineto{\pgfqpoint{0.641546in}{0.666836in}}%
\pgfpathlineto{\pgfqpoint{0.648798in}{0.678965in}}%
\pgfpathlineto{\pgfqpoint{0.649942in}{0.682192in}}%
\pgfpathlineto{\pgfqpoint{0.652564in}{0.691094in}}%
\pgfpathlineto{\pgfqpoint{0.653211in}{0.703223in}}%
\pgfpathlineto{\pgfqpoint{0.649942in}{0.708518in}}%
\pgfpathlineto{\pgfqpoint{0.644936in}{0.715352in}}%
\pgfpathlineto{\pgfqpoint{0.638230in}{0.721568in}}%
\pgfpathlineto{\pgfqpoint{0.628235in}{0.727481in}}%
\pgfpathlineto{\pgfqpoint{0.626517in}{0.728373in}}%
\pgfpathlineto{\pgfqpoint{0.614804in}{0.734962in}}%
\pgfpathlineto{\pgfqpoint{0.603091in}{0.738798in}}%
\pgfpathlineto{\pgfqpoint{0.601153in}{0.739610in}}%
\pgfpathlineto{\pgfqpoint{0.591379in}{0.743851in}}%
\pgfpathlineto{\pgfqpoint{0.579666in}{0.747706in}}%
\pgfpathlineto{\pgfqpoint{0.568887in}{0.751739in}}%
\pgfpathlineto{\pgfqpoint{0.567953in}{0.752073in}}%
\pgfpathlineto{\pgfqpoint{0.556241in}{0.755578in}}%
\pgfpathlineto{\pgfqpoint{0.544528in}{0.758621in}}%
\pgfpathlineto{\pgfqpoint{0.532815in}{0.761299in}}%
\pgfpathlineto{\pgfqpoint{0.521103in}{0.763547in}}%
\pgfpathlineto{\pgfqpoint{0.518386in}{0.763867in}}%
\pgfpathlineto{\pgfqpoint{0.509390in}{0.764816in}}%
\pgfpathlineto{\pgfqpoint{0.497677in}{0.765351in}}%
\pgfpathlineto{\pgfqpoint{0.485965in}{0.765035in}}%
\pgfpathlineto{\pgfqpoint{0.474252in}{0.764053in}}%
\pgfpathlineto{\pgfqpoint{0.472907in}{0.763867in}}%
\pgfpathlineto{\pgfqpoint{0.462539in}{0.762019in}}%
\pgfpathlineto{\pgfqpoint{0.450827in}{0.758704in}}%
\pgfpathlineto{\pgfqpoint{0.439114in}{0.753719in}}%
\pgfpathlineto{\pgfqpoint{0.435320in}{0.751739in}}%
\pgfpathlineto{\pgfqpoint{0.427401in}{0.747385in}}%
\pgfpathlineto{\pgfqpoint{0.415688in}{0.739782in}}%
\pgfpathlineto{\pgfqpoint{0.415466in}{0.739610in}}%
\pgfpathlineto{\pgfqpoint{0.412195in}{0.727481in}}%
\pgfpathlineto{\pgfqpoint{0.413465in}{0.715352in}}%
\pgfpathlineto{\pgfqpoint{0.413505in}{0.703223in}}%
\pgfpathlineto{\pgfqpoint{0.410594in}{0.691094in}}%
\pgfpathlineto{\pgfqpoint{0.407190in}{0.678965in}}%
\pgfpathlineto{\pgfqpoint{0.404201in}{0.666836in}}%
\pgfpathlineto{\pgfqpoint{0.403976in}{0.663787in}}%
\pgfpathlineto{\pgfqpoint{0.403289in}{0.654708in}}%
\pgfpathlineto{\pgfqpoint{0.402241in}{0.642579in}}%
\pgfpathlineto{\pgfqpoint{0.403402in}{0.630450in}}%
\pgfpathlineto{\pgfqpoint{0.403523in}{0.618321in}}%
\pgfpathlineto{\pgfqpoint{0.403033in}{0.606192in}}%
\pgfpathlineto{\pgfqpoint{0.400244in}{0.594063in}}%
\pgfpathlineto{\pgfqpoint{0.396326in}{0.581934in}}%
\pgfpathlineto{\pgfqpoint{0.392263in}{0.571496in}}%
\pgfpathlineto{\pgfqpoint{0.391569in}{0.569805in}}%
\pgfpathlineto{\pgfqpoint{0.385859in}{0.557677in}}%
\pgfpathlineto{\pgfqpoint{0.383955in}{0.545548in}}%
\pgfpathclose%
\pgfpathmoveto{\pgfqpoint{0.514478in}{0.642579in}}%
\pgfpathlineto{\pgfqpoint{0.509390in}{0.645156in}}%
\pgfpathlineto{\pgfqpoint{0.497677in}{0.651503in}}%
\pgfpathlineto{\pgfqpoint{0.492821in}{0.654708in}}%
\pgfpathlineto{\pgfqpoint{0.485965in}{0.660531in}}%
\pgfpathlineto{\pgfqpoint{0.480358in}{0.666836in}}%
\pgfpathlineto{\pgfqpoint{0.474252in}{0.675949in}}%
\pgfpathlineto{\pgfqpoint{0.472783in}{0.678965in}}%
\pgfpathlineto{\pgfqpoint{0.473132in}{0.691094in}}%
\pgfpathlineto{\pgfqpoint{0.474061in}{0.703223in}}%
\pgfpathlineto{\pgfqpoint{0.470515in}{0.715352in}}%
\pgfpathlineto{\pgfqpoint{0.468838in}{0.727481in}}%
\pgfpathlineto{\pgfqpoint{0.474252in}{0.733482in}}%
\pgfpathlineto{\pgfqpoint{0.481436in}{0.739610in}}%
\pgfpathlineto{\pgfqpoint{0.485965in}{0.741306in}}%
\pgfpathlineto{\pgfqpoint{0.497677in}{0.744789in}}%
\pgfpathlineto{\pgfqpoint{0.509390in}{0.746957in}}%
\pgfpathlineto{\pgfqpoint{0.521103in}{0.747895in}}%
\pgfpathlineto{\pgfqpoint{0.532815in}{0.747640in}}%
\pgfpathlineto{\pgfqpoint{0.544528in}{0.746153in}}%
\pgfpathlineto{\pgfqpoint{0.556241in}{0.743572in}}%
\pgfpathlineto{\pgfqpoint{0.567953in}{0.740583in}}%
\pgfpathlineto{\pgfqpoint{0.571219in}{0.739610in}}%
\pgfpathlineto{\pgfqpoint{0.579666in}{0.736597in}}%
\pgfpathlineto{\pgfqpoint{0.591379in}{0.732284in}}%
\pgfpathlineto{\pgfqpoint{0.603091in}{0.728097in}}%
\pgfpathlineto{\pgfqpoint{0.604529in}{0.727481in}}%
\pgfpathlineto{\pgfqpoint{0.614804in}{0.720714in}}%
\pgfpathlineto{\pgfqpoint{0.622658in}{0.715352in}}%
\pgfpathlineto{\pgfqpoint{0.626517in}{0.710721in}}%
\pgfpathlineto{\pgfqpoint{0.632810in}{0.703223in}}%
\pgfpathlineto{\pgfqpoint{0.633689in}{0.691094in}}%
\pgfpathlineto{\pgfqpoint{0.628975in}{0.678965in}}%
\pgfpathlineto{\pgfqpoint{0.626517in}{0.675891in}}%
\pgfpathlineto{\pgfqpoint{0.617473in}{0.666836in}}%
\pgfpathlineto{\pgfqpoint{0.614804in}{0.664257in}}%
\pgfpathlineto{\pgfqpoint{0.603091in}{0.654807in}}%
\pgfpathlineto{\pgfqpoint{0.602950in}{0.654708in}}%
\pgfpathlineto{\pgfqpoint{0.591379in}{0.646555in}}%
\pgfpathlineto{\pgfqpoint{0.583214in}{0.642579in}}%
\pgfpathlineto{\pgfqpoint{0.579666in}{0.640614in}}%
\pgfpathlineto{\pgfqpoint{0.567953in}{0.636658in}}%
\pgfpathlineto{\pgfqpoint{0.556241in}{0.632773in}}%
\pgfpathlineto{\pgfqpoint{0.544528in}{0.631476in}}%
\pgfpathlineto{\pgfqpoint{0.532815in}{0.631108in}}%
\pgfpathlineto{\pgfqpoint{0.521103in}{0.637056in}}%
\pgfpathclose%
\pgfusepath{fill}%
\end{pgfscope}%
\begin{pgfscope}%
\pgfpathrectangle{\pgfqpoint{0.211875in}{0.211875in}}{\pgfqpoint{1.313625in}{1.279725in}}%
\pgfusepath{clip}%
\pgfsetbuttcap%
\pgfsetroundjoin%
\definecolor{currentfill}{rgb}{0.490838,0.119982,0.351115}%
\pgfsetfillcolor{currentfill}%
\pgfsetlinewidth{0.000000pt}%
\definecolor{currentstroke}{rgb}{0.000000,0.000000,0.000000}%
\pgfsetstrokecolor{currentstroke}%
\pgfsetdash{}{0pt}%
\pgfpathmoveto{\pgfqpoint{1.376129in}{0.544934in}}%
\pgfpathlineto{\pgfqpoint{1.387842in}{0.540626in}}%
\pgfpathlineto{\pgfqpoint{1.399554in}{0.539058in}}%
\pgfpathlineto{\pgfqpoint{1.411267in}{0.539468in}}%
\pgfpathlineto{\pgfqpoint{1.422980in}{0.540398in}}%
\pgfpathlineto{\pgfqpoint{1.434692in}{0.543438in}}%
\pgfpathlineto{\pgfqpoint{1.439320in}{0.545548in}}%
\pgfpathlineto{\pgfqpoint{1.446405in}{0.550009in}}%
\pgfpathlineto{\pgfqpoint{1.446405in}{0.557677in}}%
\pgfpathlineto{\pgfqpoint{1.446405in}{0.569805in}}%
\pgfpathlineto{\pgfqpoint{1.446405in}{0.581934in}}%
\pgfpathlineto{\pgfqpoint{1.446405in}{0.594063in}}%
\pgfpathlineto{\pgfqpoint{1.446405in}{0.606192in}}%
\pgfpathlineto{\pgfqpoint{1.446405in}{0.618321in}}%
\pgfpathlineto{\pgfqpoint{1.446405in}{0.630450in}}%
\pgfpathlineto{\pgfqpoint{1.446405in}{0.632773in}}%
\pgfpathlineto{\pgfqpoint{1.434692in}{0.638301in}}%
\pgfpathlineto{\pgfqpoint{1.423382in}{0.642579in}}%
\pgfpathlineto{\pgfqpoint{1.422980in}{0.642671in}}%
\pgfpathlineto{\pgfqpoint{1.411267in}{0.647885in}}%
\pgfpathlineto{\pgfqpoint{1.402980in}{0.654708in}}%
\pgfpathlineto{\pgfqpoint{1.399554in}{0.656841in}}%
\pgfpathlineto{\pgfqpoint{1.387842in}{0.666328in}}%
\pgfpathlineto{\pgfqpoint{1.387266in}{0.666836in}}%
\pgfpathlineto{\pgfqpoint{1.376129in}{0.678180in}}%
\pgfpathlineto{\pgfqpoint{1.375086in}{0.678965in}}%
\pgfpathlineto{\pgfqpoint{1.364416in}{0.682924in}}%
\pgfpathlineto{\pgfqpoint{1.356668in}{0.678965in}}%
\pgfpathlineto{\pgfqpoint{1.352704in}{0.674608in}}%
\pgfpathlineto{\pgfqpoint{1.345999in}{0.666836in}}%
\pgfpathlineto{\pgfqpoint{1.340991in}{0.662194in}}%
\pgfpathlineto{\pgfqpoint{1.330947in}{0.654708in}}%
\pgfpathlineto{\pgfqpoint{1.329278in}{0.653049in}}%
\pgfpathlineto{\pgfqpoint{1.317566in}{0.644484in}}%
\pgfpathlineto{\pgfqpoint{1.314566in}{0.642579in}}%
\pgfpathlineto{\pgfqpoint{1.307490in}{0.630450in}}%
\pgfpathlineto{\pgfqpoint{1.305853in}{0.625909in}}%
\pgfpathlineto{\pgfqpoint{1.302749in}{0.618321in}}%
\pgfpathlineto{\pgfqpoint{1.304212in}{0.606192in}}%
\pgfpathlineto{\pgfqpoint{1.305853in}{0.599677in}}%
\pgfpathlineto{\pgfqpoint{1.307341in}{0.594063in}}%
\pgfpathlineto{\pgfqpoint{1.311805in}{0.581934in}}%
\pgfpathlineto{\pgfqpoint{1.316685in}{0.569805in}}%
\pgfpathlineto{\pgfqpoint{1.317566in}{0.567929in}}%
\pgfpathlineto{\pgfqpoint{1.326398in}{0.557677in}}%
\pgfpathlineto{\pgfqpoint{1.329278in}{0.555796in}}%
\pgfpathlineto{\pgfqpoint{1.340991in}{0.552166in}}%
\pgfpathlineto{\pgfqpoint{1.352704in}{0.548815in}}%
\pgfpathlineto{\pgfqpoint{1.364416in}{0.547052in}}%
\pgfpathlineto{\pgfqpoint{1.373205in}{0.545548in}}%
\pgfpathclose%
\pgfpathmoveto{\pgfqpoint{1.393689in}{0.581934in}}%
\pgfpathlineto{\pgfqpoint{1.387842in}{0.583250in}}%
\pgfpathlineto{\pgfqpoint{1.376129in}{0.582327in}}%
\pgfpathlineto{\pgfqpoint{1.364416in}{0.583353in}}%
\pgfpathlineto{\pgfqpoint{1.352704in}{0.587867in}}%
\pgfpathlineto{\pgfqpoint{1.345868in}{0.594063in}}%
\pgfpathlineto{\pgfqpoint{1.340991in}{0.601875in}}%
\pgfpathlineto{\pgfqpoint{1.339341in}{0.606192in}}%
\pgfpathlineto{\pgfqpoint{1.335842in}{0.618321in}}%
\pgfpathlineto{\pgfqpoint{1.333990in}{0.630450in}}%
\pgfpathlineto{\pgfqpoint{1.335339in}{0.642579in}}%
\pgfpathlineto{\pgfqpoint{1.340991in}{0.648078in}}%
\pgfpathlineto{\pgfqpoint{1.346917in}{0.654708in}}%
\pgfpathlineto{\pgfqpoint{1.352704in}{0.659729in}}%
\pgfpathlineto{\pgfqpoint{1.363161in}{0.666836in}}%
\pgfpathlineto{\pgfqpoint{1.364416in}{0.668306in}}%
\pgfpathlineto{\pgfqpoint{1.368108in}{0.666836in}}%
\pgfpathlineto{\pgfqpoint{1.376129in}{0.665128in}}%
\pgfpathlineto{\pgfqpoint{1.387842in}{0.655251in}}%
\pgfpathlineto{\pgfqpoint{1.388535in}{0.654708in}}%
\pgfpathlineto{\pgfqpoint{1.399554in}{0.643734in}}%
\pgfpathlineto{\pgfqpoint{1.400995in}{0.642579in}}%
\pgfpathlineto{\pgfqpoint{1.411267in}{0.630941in}}%
\pgfpathlineto{\pgfqpoint{1.412189in}{0.630450in}}%
\pgfpathlineto{\pgfqpoint{1.422980in}{0.620237in}}%
\pgfpathlineto{\pgfqpoint{1.425081in}{0.618321in}}%
\pgfpathlineto{\pgfqpoint{1.428365in}{0.606192in}}%
\pgfpathlineto{\pgfqpoint{1.424583in}{0.594063in}}%
\pgfpathlineto{\pgfqpoint{1.422980in}{0.591569in}}%
\pgfpathlineto{\pgfqpoint{1.411267in}{0.582675in}}%
\pgfpathlineto{\pgfqpoint{1.407865in}{0.581934in}}%
\pgfpathlineto{\pgfqpoint{1.399554in}{0.580419in}}%
\pgfpathclose%
\pgfusepath{fill}%
\end{pgfscope}%
\begin{pgfscope}%
\pgfpathrectangle{\pgfqpoint{0.211875in}{0.211875in}}{\pgfqpoint{1.313625in}{1.279725in}}%
\pgfusepath{clip}%
\pgfsetbuttcap%
\pgfsetroundjoin%
\definecolor{currentfill}{rgb}{0.490838,0.119982,0.351115}%
\pgfsetfillcolor{currentfill}%
\pgfsetlinewidth{0.000000pt}%
\definecolor{currentstroke}{rgb}{0.000000,0.000000,0.000000}%
\pgfsetstrokecolor{currentstroke}%
\pgfsetdash{}{0pt}%
\pgfpathmoveto{\pgfqpoint{1.270715in}{0.756881in}}%
\pgfpathlineto{\pgfqpoint{1.282427in}{0.753143in}}%
\pgfpathlineto{\pgfqpoint{1.294140in}{0.761560in}}%
\pgfpathlineto{\pgfqpoint{1.297299in}{0.763867in}}%
\pgfpathlineto{\pgfqpoint{1.305853in}{0.770224in}}%
\pgfpathlineto{\pgfqpoint{1.317566in}{0.775153in}}%
\pgfpathlineto{\pgfqpoint{1.319312in}{0.775996in}}%
\pgfpathlineto{\pgfqpoint{1.329278in}{0.783280in}}%
\pgfpathlineto{\pgfqpoint{1.331335in}{0.788125in}}%
\pgfpathlineto{\pgfqpoint{1.337619in}{0.800254in}}%
\pgfpathlineto{\pgfqpoint{1.340991in}{0.806056in}}%
\pgfpathlineto{\pgfqpoint{1.344247in}{0.812383in}}%
\pgfpathlineto{\pgfqpoint{1.350258in}{0.824512in}}%
\pgfpathlineto{\pgfqpoint{1.352704in}{0.828023in}}%
\pgfpathlineto{\pgfqpoint{1.358361in}{0.836641in}}%
\pgfpathlineto{\pgfqpoint{1.364416in}{0.846718in}}%
\pgfpathlineto{\pgfqpoint{1.365676in}{0.848770in}}%
\pgfpathlineto{\pgfqpoint{1.376129in}{0.860843in}}%
\pgfpathlineto{\pgfqpoint{1.376172in}{0.860898in}}%
\pgfpathlineto{\pgfqpoint{1.382510in}{0.873027in}}%
\pgfpathlineto{\pgfqpoint{1.386662in}{0.885156in}}%
\pgfpathlineto{\pgfqpoint{1.387842in}{0.887509in}}%
\pgfpathlineto{\pgfqpoint{1.399554in}{0.895312in}}%
\pgfpathlineto{\pgfqpoint{1.404911in}{0.897285in}}%
\pgfpathlineto{\pgfqpoint{1.411267in}{0.899846in}}%
\pgfpathlineto{\pgfqpoint{1.422980in}{0.904051in}}%
\pgfpathlineto{\pgfqpoint{1.434692in}{0.907710in}}%
\pgfpathlineto{\pgfqpoint{1.438617in}{0.909414in}}%
\pgfpathlineto{\pgfqpoint{1.446405in}{0.913935in}}%
\pgfpathlineto{\pgfqpoint{1.446405in}{0.921543in}}%
\pgfpathlineto{\pgfqpoint{1.446405in}{0.933672in}}%
\pgfpathlineto{\pgfqpoint{1.446405in}{0.939876in}}%
\pgfpathlineto{\pgfqpoint{1.437431in}{0.933672in}}%
\pgfpathlineto{\pgfqpoint{1.434692in}{0.932063in}}%
\pgfpathlineto{\pgfqpoint{1.422980in}{0.928115in}}%
\pgfpathlineto{\pgfqpoint{1.411267in}{0.929786in}}%
\pgfpathlineto{\pgfqpoint{1.400823in}{0.933672in}}%
\pgfpathlineto{\pgfqpoint{1.401609in}{0.945801in}}%
\pgfpathlineto{\pgfqpoint{1.407868in}{0.957929in}}%
\pgfpathlineto{\pgfqpoint{1.411267in}{0.964967in}}%
\pgfpathlineto{\pgfqpoint{1.414917in}{0.970058in}}%
\pgfpathlineto{\pgfqpoint{1.422980in}{0.981752in}}%
\pgfpathlineto{\pgfqpoint{1.423273in}{0.982187in}}%
\pgfpathlineto{\pgfqpoint{1.434692in}{0.991410in}}%
\pgfpathlineto{\pgfqpoint{1.438075in}{0.994316in}}%
\pgfpathlineto{\pgfqpoint{1.446405in}{0.999115in}}%
\pgfpathlineto{\pgfqpoint{1.446405in}{1.006445in}}%
\pgfpathlineto{\pgfqpoint{1.446405in}{1.018574in}}%
\pgfpathlineto{\pgfqpoint{1.446405in}{1.030703in}}%
\pgfpathlineto{\pgfqpoint{1.446405in}{1.042832in}}%
\pgfpathlineto{\pgfqpoint{1.446405in}{1.054073in}}%
\pgfpathlineto{\pgfqpoint{1.434692in}{1.049223in}}%
\pgfpathlineto{\pgfqpoint{1.422980in}{1.044059in}}%
\pgfpathlineto{\pgfqpoint{1.421031in}{1.042832in}}%
\pgfpathlineto{\pgfqpoint{1.411267in}{1.036653in}}%
\pgfpathlineto{\pgfqpoint{1.404469in}{1.030703in}}%
\pgfpathlineto{\pgfqpoint{1.399554in}{1.024752in}}%
\pgfpathlineto{\pgfqpoint{1.396090in}{1.018574in}}%
\pgfpathlineto{\pgfqpoint{1.388922in}{1.006445in}}%
\pgfpathlineto{\pgfqpoint{1.387842in}{1.004785in}}%
\pgfpathlineto{\pgfqpoint{1.378576in}{0.994316in}}%
\pgfpathlineto{\pgfqpoint{1.376129in}{0.991837in}}%
\pgfpathlineto{\pgfqpoint{1.369638in}{0.982187in}}%
\pgfpathlineto{\pgfqpoint{1.364416in}{0.975391in}}%
\pgfpathlineto{\pgfqpoint{1.356596in}{0.970058in}}%
\pgfpathlineto{\pgfqpoint{1.352704in}{0.967068in}}%
\pgfpathlineto{\pgfqpoint{1.341845in}{0.957929in}}%
\pgfpathlineto{\pgfqpoint{1.340991in}{0.956927in}}%
\pgfpathlineto{\pgfqpoint{1.329278in}{0.953089in}}%
\pgfpathlineto{\pgfqpoint{1.317566in}{0.948987in}}%
\pgfpathlineto{\pgfqpoint{1.305853in}{0.947251in}}%
\pgfpathlineto{\pgfqpoint{1.297980in}{0.945801in}}%
\pgfpathlineto{\pgfqpoint{1.294140in}{0.945045in}}%
\pgfpathlineto{\pgfqpoint{1.282427in}{0.937602in}}%
\pgfpathlineto{\pgfqpoint{1.280374in}{0.933672in}}%
\pgfpathlineto{\pgfqpoint{1.276038in}{0.921543in}}%
\pgfpathlineto{\pgfqpoint{1.275227in}{0.909414in}}%
\pgfpathlineto{\pgfqpoint{1.276274in}{0.897285in}}%
\pgfpathlineto{\pgfqpoint{1.278013in}{0.885156in}}%
\pgfpathlineto{\pgfqpoint{1.280910in}{0.873027in}}%
\pgfpathlineto{\pgfqpoint{1.282427in}{0.863671in}}%
\pgfpathlineto{\pgfqpoint{1.282959in}{0.860898in}}%
\pgfpathlineto{\pgfqpoint{1.282427in}{0.859236in}}%
\pgfpathlineto{\pgfqpoint{1.278038in}{0.848770in}}%
\pgfpathlineto{\pgfqpoint{1.270715in}{0.839404in}}%
\pgfpathlineto{\pgfqpoint{1.267703in}{0.836641in}}%
\pgfpathlineto{\pgfqpoint{1.259002in}{0.829541in}}%
\pgfpathlineto{\pgfqpoint{1.251447in}{0.824512in}}%
\pgfpathlineto{\pgfqpoint{1.247289in}{0.821452in}}%
\pgfpathlineto{\pgfqpoint{1.235577in}{0.814257in}}%
\pgfpathlineto{\pgfqpoint{1.231378in}{0.812383in}}%
\pgfpathlineto{\pgfqpoint{1.223864in}{0.807564in}}%
\pgfpathlineto{\pgfqpoint{1.215200in}{0.800254in}}%
\pgfpathlineto{\pgfqpoint{1.212151in}{0.789658in}}%
\pgfpathlineto{\pgfqpoint{1.211845in}{0.788125in}}%
\pgfpathlineto{\pgfqpoint{1.212151in}{0.787293in}}%
\pgfpathlineto{\pgfqpoint{1.220701in}{0.775996in}}%
\pgfpathlineto{\pgfqpoint{1.223864in}{0.774522in}}%
\pgfpathlineto{\pgfqpoint{1.235577in}{0.773365in}}%
\pgfpathlineto{\pgfqpoint{1.247289in}{0.772104in}}%
\pgfpathlineto{\pgfqpoint{1.259002in}{0.766968in}}%
\pgfpathlineto{\pgfqpoint{1.262654in}{0.763867in}}%
\pgfpathclose%
\pgfpathmoveto{\pgfqpoint{1.293118in}{0.788125in}}%
\pgfpathlineto{\pgfqpoint{1.282942in}{0.800254in}}%
\pgfpathlineto{\pgfqpoint{1.284404in}{0.812383in}}%
\pgfpathlineto{\pgfqpoint{1.289885in}{0.824512in}}%
\pgfpathlineto{\pgfqpoint{1.294140in}{0.832997in}}%
\pgfpathlineto{\pgfqpoint{1.296241in}{0.836641in}}%
\pgfpathlineto{\pgfqpoint{1.300591in}{0.848770in}}%
\pgfpathlineto{\pgfqpoint{1.303492in}{0.860898in}}%
\pgfpathlineto{\pgfqpoint{1.304183in}{0.873027in}}%
\pgfpathlineto{\pgfqpoint{1.305853in}{0.883443in}}%
\pgfpathlineto{\pgfqpoint{1.306463in}{0.885156in}}%
\pgfpathlineto{\pgfqpoint{1.317566in}{0.894614in}}%
\pgfpathlineto{\pgfqpoint{1.322662in}{0.897285in}}%
\pgfpathlineto{\pgfqpoint{1.329278in}{0.899964in}}%
\pgfpathlineto{\pgfqpoint{1.340991in}{0.904445in}}%
\pgfpathlineto{\pgfqpoint{1.352704in}{0.904814in}}%
\pgfpathlineto{\pgfqpoint{1.364416in}{0.897400in}}%
\pgfpathlineto{\pgfqpoint{1.364482in}{0.897285in}}%
\pgfpathlineto{\pgfqpoint{1.364986in}{0.885156in}}%
\pgfpathlineto{\pgfqpoint{1.364416in}{0.882897in}}%
\pgfpathlineto{\pgfqpoint{1.361429in}{0.873027in}}%
\pgfpathlineto{\pgfqpoint{1.354242in}{0.860898in}}%
\pgfpathlineto{\pgfqpoint{1.352704in}{0.858744in}}%
\pgfpathlineto{\pgfqpoint{1.346288in}{0.848770in}}%
\pgfpathlineto{\pgfqpoint{1.341436in}{0.836641in}}%
\pgfpathlineto{\pgfqpoint{1.340991in}{0.835934in}}%
\pgfpathlineto{\pgfqpoint{1.334032in}{0.824512in}}%
\pgfpathlineto{\pgfqpoint{1.329467in}{0.812383in}}%
\pgfpathlineto{\pgfqpoint{1.329278in}{0.812032in}}%
\pgfpathlineto{\pgfqpoint{1.318917in}{0.800254in}}%
\pgfpathlineto{\pgfqpoint{1.317566in}{0.798877in}}%
\pgfpathlineto{\pgfqpoint{1.305853in}{0.795045in}}%
\pgfpathlineto{\pgfqpoint{1.294726in}{0.788125in}}%
\pgfpathlineto{\pgfqpoint{1.294140in}{0.787784in}}%
\pgfpathclose%
\pgfusepath{fill}%
\end{pgfscope}%
\begin{pgfscope}%
\pgfpathrectangle{\pgfqpoint{0.211875in}{0.211875in}}{\pgfqpoint{1.313625in}{1.279725in}}%
\pgfusepath{clip}%
\pgfsetbuttcap%
\pgfsetroundjoin%
\definecolor{currentfill}{rgb}{0.490838,0.119982,0.351115}%
\pgfsetfillcolor{currentfill}%
\pgfsetlinewidth{0.000000pt}%
\definecolor{currentstroke}{rgb}{0.000000,0.000000,0.000000}%
\pgfsetstrokecolor{currentstroke}%
\pgfsetdash{}{0pt}%
\pgfpathmoveto{\pgfqpoint{0.813920in}{0.800081in}}%
\pgfpathlineto{\pgfqpoint{0.817208in}{0.800254in}}%
\pgfpathlineto{\pgfqpoint{0.825633in}{0.800986in}}%
\pgfpathlineto{\pgfqpoint{0.837345in}{0.803130in}}%
\pgfpathlineto{\pgfqpoint{0.849058in}{0.805471in}}%
\pgfpathlineto{\pgfqpoint{0.860771in}{0.809081in}}%
\pgfpathlineto{\pgfqpoint{0.868339in}{0.812383in}}%
\pgfpathlineto{\pgfqpoint{0.872483in}{0.815704in}}%
\pgfpathlineto{\pgfqpoint{0.884196in}{0.823786in}}%
\pgfpathlineto{\pgfqpoint{0.884893in}{0.824512in}}%
\pgfpathlineto{\pgfqpoint{0.889853in}{0.836641in}}%
\pgfpathlineto{\pgfqpoint{0.893676in}{0.848770in}}%
\pgfpathlineto{\pgfqpoint{0.895909in}{0.858915in}}%
\pgfpathlineto{\pgfqpoint{0.896536in}{0.860898in}}%
\pgfpathlineto{\pgfqpoint{0.895909in}{0.862108in}}%
\pgfpathlineto{\pgfqpoint{0.888974in}{0.873027in}}%
\pgfpathlineto{\pgfqpoint{0.884196in}{0.882686in}}%
\pgfpathlineto{\pgfqpoint{0.883021in}{0.885156in}}%
\pgfpathlineto{\pgfqpoint{0.875826in}{0.897285in}}%
\pgfpathlineto{\pgfqpoint{0.872483in}{0.902888in}}%
\pgfpathlineto{\pgfqpoint{0.868080in}{0.909414in}}%
\pgfpathlineto{\pgfqpoint{0.860771in}{0.920855in}}%
\pgfpathlineto{\pgfqpoint{0.860403in}{0.921543in}}%
\pgfpathlineto{\pgfqpoint{0.851669in}{0.933672in}}%
\pgfpathlineto{\pgfqpoint{0.849058in}{0.936636in}}%
\pgfpathlineto{\pgfqpoint{0.839375in}{0.945801in}}%
\pgfpathlineto{\pgfqpoint{0.837345in}{0.948213in}}%
\pgfpathlineto{\pgfqpoint{0.828924in}{0.957929in}}%
\pgfpathlineto{\pgfqpoint{0.825633in}{0.962650in}}%
\pgfpathlineto{\pgfqpoint{0.819336in}{0.970058in}}%
\pgfpathlineto{\pgfqpoint{0.813920in}{0.976616in}}%
\pgfpathlineto{\pgfqpoint{0.810306in}{0.982187in}}%
\pgfpathlineto{\pgfqpoint{0.807447in}{0.994316in}}%
\pgfpathlineto{\pgfqpoint{0.808483in}{1.006445in}}%
\pgfpathlineto{\pgfqpoint{0.812292in}{1.018574in}}%
\pgfpathlineto{\pgfqpoint{0.813920in}{1.021235in}}%
\pgfpathlineto{\pgfqpoint{0.819123in}{1.030703in}}%
\pgfpathlineto{\pgfqpoint{0.824461in}{1.042832in}}%
\pgfpathlineto{\pgfqpoint{0.825633in}{1.045032in}}%
\pgfpathlineto{\pgfqpoint{0.831703in}{1.054960in}}%
\pgfpathlineto{\pgfqpoint{0.837345in}{1.061977in}}%
\pgfpathlineto{\pgfqpoint{0.841664in}{1.067089in}}%
\pgfpathlineto{\pgfqpoint{0.849058in}{1.078809in}}%
\pgfpathlineto{\pgfqpoint{0.849285in}{1.079218in}}%
\pgfpathlineto{\pgfqpoint{0.850686in}{1.091347in}}%
\pgfpathlineto{\pgfqpoint{0.850367in}{1.103476in}}%
\pgfpathlineto{\pgfqpoint{0.849058in}{1.112525in}}%
\pgfpathlineto{\pgfqpoint{0.848646in}{1.115605in}}%
\pgfpathlineto{\pgfqpoint{0.846438in}{1.127734in}}%
\pgfpathlineto{\pgfqpoint{0.847676in}{1.139863in}}%
\pgfpathlineto{\pgfqpoint{0.848971in}{1.151991in}}%
\pgfpathlineto{\pgfqpoint{0.839883in}{1.164120in}}%
\pgfpathlineto{\pgfqpoint{0.837345in}{1.165563in}}%
\pgfpathlineto{\pgfqpoint{0.825633in}{1.168297in}}%
\pgfpathlineto{\pgfqpoint{0.813920in}{1.167840in}}%
\pgfpathlineto{\pgfqpoint{0.802207in}{1.166924in}}%
\pgfpathlineto{\pgfqpoint{0.790495in}{1.165914in}}%
\pgfpathlineto{\pgfqpoint{0.778782in}{1.164750in}}%
\pgfpathlineto{\pgfqpoint{0.773024in}{1.164120in}}%
\pgfpathlineto{\pgfqpoint{0.767069in}{1.163400in}}%
\pgfpathlineto{\pgfqpoint{0.755356in}{1.161716in}}%
\pgfpathlineto{\pgfqpoint{0.743644in}{1.159398in}}%
\pgfpathlineto{\pgfqpoint{0.731931in}{1.156481in}}%
\pgfpathlineto{\pgfqpoint{0.720218in}{1.153380in}}%
\pgfpathlineto{\pgfqpoint{0.714818in}{1.151991in}}%
\pgfpathlineto{\pgfqpoint{0.708506in}{1.149519in}}%
\pgfpathlineto{\pgfqpoint{0.696793in}{1.145337in}}%
\pgfpathlineto{\pgfqpoint{0.685080in}{1.140902in}}%
\pgfpathlineto{\pgfqpoint{0.682527in}{1.139863in}}%
\pgfpathlineto{\pgfqpoint{0.673368in}{1.135429in}}%
\pgfpathlineto{\pgfqpoint{0.661655in}{1.129909in}}%
\pgfpathlineto{\pgfqpoint{0.656612in}{1.127734in}}%
\pgfpathlineto{\pgfqpoint{0.649942in}{1.125195in}}%
\pgfpathlineto{\pgfqpoint{0.638230in}{1.122368in}}%
\pgfpathlineto{\pgfqpoint{0.626517in}{1.120874in}}%
\pgfpathlineto{\pgfqpoint{0.614804in}{1.119661in}}%
\pgfpathlineto{\pgfqpoint{0.603091in}{1.118825in}}%
\pgfpathlineto{\pgfqpoint{0.591379in}{1.118359in}}%
\pgfpathlineto{\pgfqpoint{0.579666in}{1.117822in}}%
\pgfpathlineto{\pgfqpoint{0.567953in}{1.117764in}}%
\pgfpathlineto{\pgfqpoint{0.556241in}{1.117717in}}%
\pgfpathlineto{\pgfqpoint{0.544528in}{1.117511in}}%
\pgfpathlineto{\pgfqpoint{0.532815in}{1.117039in}}%
\pgfpathlineto{\pgfqpoint{0.521103in}{1.116262in}}%
\pgfpathlineto{\pgfqpoint{0.513868in}{1.115605in}}%
\pgfpathlineto{\pgfqpoint{0.509390in}{1.115081in}}%
\pgfpathlineto{\pgfqpoint{0.497677in}{1.113433in}}%
\pgfpathlineto{\pgfqpoint{0.485965in}{1.110405in}}%
\pgfpathlineto{\pgfqpoint{0.476887in}{1.103476in}}%
\pgfpathlineto{\pgfqpoint{0.484820in}{1.091347in}}%
\pgfpathlineto{\pgfqpoint{0.485965in}{1.090020in}}%
\pgfpathlineto{\pgfqpoint{0.494793in}{1.079218in}}%
\pgfpathlineto{\pgfqpoint{0.497677in}{1.076029in}}%
\pgfpathlineto{\pgfqpoint{0.504803in}{1.067089in}}%
\pgfpathlineto{\pgfqpoint{0.509390in}{1.062026in}}%
\pgfpathlineto{\pgfqpoint{0.514908in}{1.054960in}}%
\pgfpathlineto{\pgfqpoint{0.521103in}{1.047374in}}%
\pgfpathlineto{\pgfqpoint{0.524301in}{1.042832in}}%
\pgfpathlineto{\pgfqpoint{0.531556in}{1.030703in}}%
\pgfpathlineto{\pgfqpoint{0.532815in}{1.029018in}}%
\pgfpathlineto{\pgfqpoint{0.540267in}{1.018574in}}%
\pgfpathlineto{\pgfqpoint{0.544528in}{1.012910in}}%
\pgfpathlineto{\pgfqpoint{0.549039in}{1.006445in}}%
\pgfpathlineto{\pgfqpoint{0.556241in}{0.994785in}}%
\pgfpathlineto{\pgfqpoint{0.556524in}{0.994316in}}%
\pgfpathlineto{\pgfqpoint{0.563412in}{0.982187in}}%
\pgfpathlineto{\pgfqpoint{0.567953in}{0.974140in}}%
\pgfpathlineto{\pgfqpoint{0.570336in}{0.970058in}}%
\pgfpathlineto{\pgfqpoint{0.578893in}{0.957929in}}%
\pgfpathlineto{\pgfqpoint{0.579666in}{0.957150in}}%
\pgfpathlineto{\pgfqpoint{0.591379in}{0.948004in}}%
\pgfpathlineto{\pgfqpoint{0.594396in}{0.945801in}}%
\pgfpathlineto{\pgfqpoint{0.603091in}{0.939109in}}%
\pgfpathlineto{\pgfqpoint{0.609328in}{0.933672in}}%
\pgfpathlineto{\pgfqpoint{0.614804in}{0.927223in}}%
\pgfpathlineto{\pgfqpoint{0.619953in}{0.921543in}}%
\pgfpathlineto{\pgfqpoint{0.626517in}{0.914427in}}%
\pgfpathlineto{\pgfqpoint{0.632068in}{0.909414in}}%
\pgfpathlineto{\pgfqpoint{0.638230in}{0.903025in}}%
\pgfpathlineto{\pgfqpoint{0.643341in}{0.897285in}}%
\pgfpathlineto{\pgfqpoint{0.649942in}{0.887067in}}%
\pgfpathlineto{\pgfqpoint{0.651286in}{0.885156in}}%
\pgfpathlineto{\pgfqpoint{0.660284in}{0.873027in}}%
\pgfpathlineto{\pgfqpoint{0.661655in}{0.871235in}}%
\pgfpathlineto{\pgfqpoint{0.669617in}{0.860898in}}%
\pgfpathlineto{\pgfqpoint{0.673368in}{0.857376in}}%
\pgfpathlineto{\pgfqpoint{0.683907in}{0.848770in}}%
\pgfpathlineto{\pgfqpoint{0.685080in}{0.848073in}}%
\pgfpathlineto{\pgfqpoint{0.696793in}{0.841506in}}%
\pgfpathlineto{\pgfqpoint{0.704871in}{0.836641in}}%
\pgfpathlineto{\pgfqpoint{0.708506in}{0.834887in}}%
\pgfpathlineto{\pgfqpoint{0.720218in}{0.829465in}}%
\pgfpathlineto{\pgfqpoint{0.731480in}{0.824512in}}%
\pgfpathlineto{\pgfqpoint{0.731931in}{0.824355in}}%
\pgfpathlineto{\pgfqpoint{0.743644in}{0.820227in}}%
\pgfpathlineto{\pgfqpoint{0.755356in}{0.816431in}}%
\pgfpathlineto{\pgfqpoint{0.766124in}{0.812383in}}%
\pgfpathlineto{\pgfqpoint{0.767069in}{0.812106in}}%
\pgfpathlineto{\pgfqpoint{0.778782in}{0.808929in}}%
\pgfpathlineto{\pgfqpoint{0.790495in}{0.806138in}}%
\pgfpathlineto{\pgfqpoint{0.802207in}{0.802999in}}%
\pgfpathlineto{\pgfqpoint{0.812972in}{0.800254in}}%
\pgfpathclose%
\pgfpathmoveto{\pgfqpoint{0.777585in}{0.836641in}}%
\pgfpathlineto{\pgfqpoint{0.767069in}{0.842122in}}%
\pgfpathlineto{\pgfqpoint{0.755356in}{0.848009in}}%
\pgfpathlineto{\pgfqpoint{0.753644in}{0.848770in}}%
\pgfpathlineto{\pgfqpoint{0.743644in}{0.853899in}}%
\pgfpathlineto{\pgfqpoint{0.731931in}{0.859780in}}%
\pgfpathlineto{\pgfqpoint{0.729927in}{0.860898in}}%
\pgfpathlineto{\pgfqpoint{0.720218in}{0.869389in}}%
\pgfpathlineto{\pgfqpoint{0.716450in}{0.873027in}}%
\pgfpathlineto{\pgfqpoint{0.708506in}{0.881571in}}%
\pgfpathlineto{\pgfqpoint{0.705907in}{0.885156in}}%
\pgfpathlineto{\pgfqpoint{0.697892in}{0.897285in}}%
\pgfpathlineto{\pgfqpoint{0.696793in}{0.899115in}}%
\pgfpathlineto{\pgfqpoint{0.690775in}{0.909414in}}%
\pgfpathlineto{\pgfqpoint{0.685080in}{0.919554in}}%
\pgfpathlineto{\pgfqpoint{0.683866in}{0.921543in}}%
\pgfpathlineto{\pgfqpoint{0.673368in}{0.931598in}}%
\pgfpathlineto{\pgfqpoint{0.670418in}{0.933672in}}%
\pgfpathlineto{\pgfqpoint{0.661655in}{0.938320in}}%
\pgfpathlineto{\pgfqpoint{0.649942in}{0.945418in}}%
\pgfpathlineto{\pgfqpoint{0.649322in}{0.945801in}}%
\pgfpathlineto{\pgfqpoint{0.638230in}{0.952442in}}%
\pgfpathlineto{\pgfqpoint{0.629711in}{0.957929in}}%
\pgfpathlineto{\pgfqpoint{0.626517in}{0.959251in}}%
\pgfpathlineto{\pgfqpoint{0.614804in}{0.963938in}}%
\pgfpathlineto{\pgfqpoint{0.603091in}{0.969276in}}%
\pgfpathlineto{\pgfqpoint{0.601882in}{0.970058in}}%
\pgfpathlineto{\pgfqpoint{0.591627in}{0.982187in}}%
\pgfpathlineto{\pgfqpoint{0.591379in}{0.982805in}}%
\pgfpathlineto{\pgfqpoint{0.587078in}{0.994316in}}%
\pgfpathlineto{\pgfqpoint{0.582656in}{1.006445in}}%
\pgfpathlineto{\pgfqpoint{0.579666in}{1.014061in}}%
\pgfpathlineto{\pgfqpoint{0.577846in}{1.018574in}}%
\pgfpathlineto{\pgfqpoint{0.573448in}{1.030703in}}%
\pgfpathlineto{\pgfqpoint{0.578771in}{1.042832in}}%
\pgfpathlineto{\pgfqpoint{0.579666in}{1.043430in}}%
\pgfpathlineto{\pgfqpoint{0.591379in}{1.051104in}}%
\pgfpathlineto{\pgfqpoint{0.597034in}{1.054960in}}%
\pgfpathlineto{\pgfqpoint{0.603091in}{1.058151in}}%
\pgfpathlineto{\pgfqpoint{0.614804in}{1.062979in}}%
\pgfpathlineto{\pgfqpoint{0.626517in}{1.066087in}}%
\pgfpathlineto{\pgfqpoint{0.630058in}{1.067089in}}%
\pgfpathlineto{\pgfqpoint{0.638230in}{1.069280in}}%
\pgfpathlineto{\pgfqpoint{0.649942in}{1.073327in}}%
\pgfpathlineto{\pgfqpoint{0.661655in}{1.076953in}}%
\pgfpathlineto{\pgfqpoint{0.669506in}{1.079218in}}%
\pgfpathlineto{\pgfqpoint{0.673368in}{1.080234in}}%
\pgfpathlineto{\pgfqpoint{0.685080in}{1.084033in}}%
\pgfpathlineto{\pgfqpoint{0.696793in}{1.089022in}}%
\pgfpathlineto{\pgfqpoint{0.700938in}{1.091347in}}%
\pgfpathlineto{\pgfqpoint{0.708506in}{1.096759in}}%
\pgfpathlineto{\pgfqpoint{0.719359in}{1.103476in}}%
\pgfpathlineto{\pgfqpoint{0.720218in}{1.103816in}}%
\pgfpathlineto{\pgfqpoint{0.731931in}{1.107109in}}%
\pgfpathlineto{\pgfqpoint{0.743644in}{1.108938in}}%
\pgfpathlineto{\pgfqpoint{0.755356in}{1.111056in}}%
\pgfpathlineto{\pgfqpoint{0.765205in}{1.103476in}}%
\pgfpathlineto{\pgfqpoint{0.763949in}{1.091347in}}%
\pgfpathlineto{\pgfqpoint{0.759573in}{1.079218in}}%
\pgfpathlineto{\pgfqpoint{0.755356in}{1.070494in}}%
\pgfpathlineto{\pgfqpoint{0.753871in}{1.067089in}}%
\pgfpathlineto{\pgfqpoint{0.747361in}{1.054960in}}%
\pgfpathlineto{\pgfqpoint{0.743644in}{1.046372in}}%
\pgfpathlineto{\pgfqpoint{0.742157in}{1.042832in}}%
\pgfpathlineto{\pgfqpoint{0.737506in}{1.030703in}}%
\pgfpathlineto{\pgfqpoint{0.731931in}{1.019835in}}%
\pgfpathlineto{\pgfqpoint{0.731130in}{1.018574in}}%
\pgfpathlineto{\pgfqpoint{0.722665in}{1.006445in}}%
\pgfpathlineto{\pgfqpoint{0.724338in}{0.994316in}}%
\pgfpathlineto{\pgfqpoint{0.725348in}{0.982187in}}%
\pgfpathlineto{\pgfqpoint{0.724718in}{0.970058in}}%
\pgfpathlineto{\pgfqpoint{0.724020in}{0.957929in}}%
\pgfpathlineto{\pgfqpoint{0.731931in}{0.952571in}}%
\pgfpathlineto{\pgfqpoint{0.743644in}{0.947287in}}%
\pgfpathlineto{\pgfqpoint{0.747021in}{0.945801in}}%
\pgfpathlineto{\pgfqpoint{0.755356in}{0.942536in}}%
\pgfpathlineto{\pgfqpoint{0.767069in}{0.937228in}}%
\pgfpathlineto{\pgfqpoint{0.772786in}{0.933672in}}%
\pgfpathlineto{\pgfqpoint{0.778782in}{0.929166in}}%
\pgfpathlineto{\pgfqpoint{0.786980in}{0.921543in}}%
\pgfpathlineto{\pgfqpoint{0.790495in}{0.918105in}}%
\pgfpathlineto{\pgfqpoint{0.798925in}{0.909414in}}%
\pgfpathlineto{\pgfqpoint{0.802207in}{0.905860in}}%
\pgfpathlineto{\pgfqpoint{0.810944in}{0.897285in}}%
\pgfpathlineto{\pgfqpoint{0.813920in}{0.894803in}}%
\pgfpathlineto{\pgfqpoint{0.824498in}{0.885156in}}%
\pgfpathlineto{\pgfqpoint{0.825633in}{0.883647in}}%
\pgfpathlineto{\pgfqpoint{0.832157in}{0.873027in}}%
\pgfpathlineto{\pgfqpoint{0.837345in}{0.866207in}}%
\pgfpathlineto{\pgfqpoint{0.840909in}{0.860898in}}%
\pgfpathlineto{\pgfqpoint{0.845559in}{0.848770in}}%
\pgfpathlineto{\pgfqpoint{0.844666in}{0.836641in}}%
\pgfpathlineto{\pgfqpoint{0.837345in}{0.832945in}}%
\pgfpathlineto{\pgfqpoint{0.825633in}{0.833102in}}%
\pgfpathlineto{\pgfqpoint{0.813920in}{0.833018in}}%
\pgfpathlineto{\pgfqpoint{0.802207in}{0.833480in}}%
\pgfpathlineto{\pgfqpoint{0.790495in}{0.834980in}}%
\pgfpathlineto{\pgfqpoint{0.778782in}{0.836269in}}%
\pgfpathclose%
\pgfusepath{fill}%
\end{pgfscope}%
\begin{pgfscope}%
\pgfpathrectangle{\pgfqpoint{0.211875in}{0.211875in}}{\pgfqpoint{1.313625in}{1.279725in}}%
\pgfusepath{clip}%
\pgfsetbuttcap%
\pgfsetroundjoin%
\definecolor{currentfill}{rgb}{0.490838,0.119982,0.351115}%
\pgfsetfillcolor{currentfill}%
\pgfsetlinewidth{0.000000pt}%
\definecolor{currentstroke}{rgb}{0.000000,0.000000,0.000000}%
\pgfsetstrokecolor{currentstroke}%
\pgfsetdash{}{0pt}%
\pgfpathmoveto{\pgfqpoint{1.188726in}{0.973638in}}%
\pgfpathlineto{\pgfqpoint{1.200439in}{0.972181in}}%
\pgfpathlineto{\pgfqpoint{1.212151in}{0.973655in}}%
\pgfpathlineto{\pgfqpoint{1.223864in}{0.976337in}}%
\pgfpathlineto{\pgfqpoint{1.235577in}{0.979897in}}%
\pgfpathlineto{\pgfqpoint{1.240892in}{0.982187in}}%
\pgfpathlineto{\pgfqpoint{1.247289in}{0.986053in}}%
\pgfpathlineto{\pgfqpoint{1.254827in}{0.994316in}}%
\pgfpathlineto{\pgfqpoint{1.259002in}{0.999691in}}%
\pgfpathlineto{\pgfqpoint{1.263218in}{1.006445in}}%
\pgfpathlineto{\pgfqpoint{1.269877in}{1.018574in}}%
\pgfpathlineto{\pgfqpoint{1.270715in}{1.020137in}}%
\pgfpathlineto{\pgfqpoint{1.276104in}{1.030703in}}%
\pgfpathlineto{\pgfqpoint{1.281205in}{1.042832in}}%
\pgfpathlineto{\pgfqpoint{1.282427in}{1.046252in}}%
\pgfpathlineto{\pgfqpoint{1.284756in}{1.054960in}}%
\pgfpathlineto{\pgfqpoint{1.289340in}{1.067089in}}%
\pgfpathlineto{\pgfqpoint{1.293782in}{1.079218in}}%
\pgfpathlineto{\pgfqpoint{1.294140in}{1.080498in}}%
\pgfpathlineto{\pgfqpoint{1.297933in}{1.091347in}}%
\pgfpathlineto{\pgfqpoint{1.301319in}{1.103476in}}%
\pgfpathlineto{\pgfqpoint{1.299506in}{1.115605in}}%
\pgfpathlineto{\pgfqpoint{1.294140in}{1.124986in}}%
\pgfpathlineto{\pgfqpoint{1.291345in}{1.127734in}}%
\pgfpathlineto{\pgfqpoint{1.282427in}{1.132216in}}%
\pgfpathlineto{\pgfqpoint{1.270715in}{1.136606in}}%
\pgfpathlineto{\pgfqpoint{1.259002in}{1.139174in}}%
\pgfpathlineto{\pgfqpoint{1.250097in}{1.139863in}}%
\pgfpathlineto{\pgfqpoint{1.247289in}{1.140072in}}%
\pgfpathlineto{\pgfqpoint{1.235577in}{1.141397in}}%
\pgfpathlineto{\pgfqpoint{1.224013in}{1.151991in}}%
\pgfpathlineto{\pgfqpoint{1.223864in}{1.152143in}}%
\pgfpathlineto{\pgfqpoint{1.212151in}{1.155364in}}%
\pgfpathlineto{\pgfqpoint{1.202647in}{1.151991in}}%
\pgfpathlineto{\pgfqpoint{1.200439in}{1.151082in}}%
\pgfpathlineto{\pgfqpoint{1.188726in}{1.147427in}}%
\pgfpathlineto{\pgfqpoint{1.182319in}{1.139863in}}%
\pgfpathlineto{\pgfqpoint{1.177013in}{1.135097in}}%
\pgfpathlineto{\pgfqpoint{1.170781in}{1.127734in}}%
\pgfpathlineto{\pgfqpoint{1.165301in}{1.119173in}}%
\pgfpathlineto{\pgfqpoint{1.163938in}{1.115605in}}%
\pgfpathlineto{\pgfqpoint{1.160424in}{1.103476in}}%
\pgfpathlineto{\pgfqpoint{1.156483in}{1.091347in}}%
\pgfpathlineto{\pgfqpoint{1.153640in}{1.079218in}}%
\pgfpathlineto{\pgfqpoint{1.153588in}{1.079100in}}%
\pgfpathlineto{\pgfqpoint{1.148113in}{1.067089in}}%
\pgfpathlineto{\pgfqpoint{1.141875in}{1.055868in}}%
\pgfpathlineto{\pgfqpoint{1.141390in}{1.054960in}}%
\pgfpathlineto{\pgfqpoint{1.134080in}{1.042832in}}%
\pgfpathlineto{\pgfqpoint{1.130163in}{1.032188in}}%
\pgfpathlineto{\pgfqpoint{1.129677in}{1.030703in}}%
\pgfpathlineto{\pgfqpoint{1.127388in}{1.018574in}}%
\pgfpathlineto{\pgfqpoint{1.130163in}{1.006964in}}%
\pgfpathlineto{\pgfqpoint{1.130334in}{1.006445in}}%
\pgfpathlineto{\pgfqpoint{1.135355in}{0.994316in}}%
\pgfpathlineto{\pgfqpoint{1.141875in}{0.989764in}}%
\pgfpathlineto{\pgfqpoint{1.153588in}{0.993515in}}%
\pgfpathlineto{\pgfqpoint{1.165301in}{0.990900in}}%
\pgfpathlineto{\pgfqpoint{1.177013in}{0.986443in}}%
\pgfpathlineto{\pgfqpoint{1.180001in}{0.982187in}}%
\pgfpathclose%
\pgfpathmoveto{\pgfqpoint{1.208828in}{1.006445in}}%
\pgfpathlineto{\pgfqpoint{1.202231in}{1.018574in}}%
\pgfpathlineto{\pgfqpoint{1.200439in}{1.019932in}}%
\pgfpathlineto{\pgfqpoint{1.188726in}{1.023420in}}%
\pgfpathlineto{\pgfqpoint{1.177013in}{1.023586in}}%
\pgfpathlineto{\pgfqpoint{1.168611in}{1.030703in}}%
\pgfpathlineto{\pgfqpoint{1.165642in}{1.042832in}}%
\pgfpathlineto{\pgfqpoint{1.168887in}{1.054960in}}%
\pgfpathlineto{\pgfqpoint{1.177013in}{1.064867in}}%
\pgfpathlineto{\pgfqpoint{1.179319in}{1.067089in}}%
\pgfpathlineto{\pgfqpoint{1.188726in}{1.074168in}}%
\pgfpathlineto{\pgfqpoint{1.197486in}{1.079218in}}%
\pgfpathlineto{\pgfqpoint{1.200439in}{1.080325in}}%
\pgfpathlineto{\pgfqpoint{1.204889in}{1.079218in}}%
\pgfpathlineto{\pgfqpoint{1.212151in}{1.076670in}}%
\pgfpathlineto{\pgfqpoint{1.223864in}{1.078814in}}%
\pgfpathlineto{\pgfqpoint{1.224464in}{1.079218in}}%
\pgfpathlineto{\pgfqpoint{1.235577in}{1.089170in}}%
\pgfpathlineto{\pgfqpoint{1.238882in}{1.091347in}}%
\pgfpathlineto{\pgfqpoint{1.247289in}{1.097498in}}%
\pgfpathlineto{\pgfqpoint{1.259002in}{1.094102in}}%
\pgfpathlineto{\pgfqpoint{1.262318in}{1.091347in}}%
\pgfpathlineto{\pgfqpoint{1.270715in}{1.082342in}}%
\pgfpathlineto{\pgfqpoint{1.271637in}{1.079218in}}%
\pgfpathlineto{\pgfqpoint{1.270715in}{1.075134in}}%
\pgfpathlineto{\pgfqpoint{1.267752in}{1.067089in}}%
\pgfpathlineto{\pgfqpoint{1.260076in}{1.054960in}}%
\pgfpathlineto{\pgfqpoint{1.259002in}{1.052927in}}%
\pgfpathlineto{\pgfqpoint{1.252316in}{1.042832in}}%
\pgfpathlineto{\pgfqpoint{1.247966in}{1.030703in}}%
\pgfpathlineto{\pgfqpoint{1.247289in}{1.029712in}}%
\pgfpathlineto{\pgfqpoint{1.240019in}{1.018574in}}%
\pgfpathlineto{\pgfqpoint{1.235577in}{1.012199in}}%
\pgfpathlineto{\pgfqpoint{1.227894in}{1.006445in}}%
\pgfpathlineto{\pgfqpoint{1.223864in}{1.004574in}}%
\pgfpathlineto{\pgfqpoint{1.212151in}{1.003907in}}%
\pgfpathclose%
\pgfusepath{fill}%
\end{pgfscope}%
\begin{pgfscope}%
\pgfpathrectangle{\pgfqpoint{0.211875in}{0.211875in}}{\pgfqpoint{1.313625in}{1.279725in}}%
\pgfusepath{clip}%
\pgfsetbuttcap%
\pgfsetroundjoin%
\definecolor{currentfill}{rgb}{0.490838,0.119982,0.351115}%
\pgfsetfillcolor{currentfill}%
\pgfsetlinewidth{0.000000pt}%
\definecolor{currentstroke}{rgb}{0.000000,0.000000,0.000000}%
\pgfsetstrokecolor{currentstroke}%
\pgfsetdash{}{0pt}%
\pgfpathmoveto{\pgfqpoint{1.071599in}{1.378601in}}%
\pgfpathlineto{\pgfqpoint{1.083312in}{1.377891in}}%
\pgfpathlineto{\pgfqpoint{1.095024in}{1.380915in}}%
\pgfpathlineto{\pgfqpoint{1.097047in}{1.382440in}}%
\pgfpathlineto{\pgfqpoint{1.106737in}{1.391469in}}%
\pgfpathlineto{\pgfqpoint{1.109428in}{1.394569in}}%
\pgfpathlineto{\pgfqpoint{1.118450in}{1.401936in}}%
\pgfpathlineto{\pgfqpoint{1.123627in}{1.406698in}}%
\pgfpathlineto{\pgfqpoint{1.130163in}{1.411179in}}%
\pgfpathlineto{\pgfqpoint{1.141875in}{1.418317in}}%
\pgfpathlineto{\pgfqpoint{1.142768in}{1.418827in}}%
\pgfpathlineto{\pgfqpoint{1.153588in}{1.425006in}}%
\pgfpathlineto{\pgfqpoint{1.165301in}{1.427129in}}%
\pgfpathlineto{\pgfqpoint{1.177013in}{1.427385in}}%
\pgfpathlineto{\pgfqpoint{1.188726in}{1.429376in}}%
\pgfpathlineto{\pgfqpoint{1.191131in}{1.430956in}}%
\pgfpathlineto{\pgfqpoint{1.200439in}{1.440602in}}%
\pgfpathlineto{\pgfqpoint{1.204299in}{1.443084in}}%
\pgfpathlineto{\pgfqpoint{1.212151in}{1.451836in}}%
\pgfpathlineto{\pgfqpoint{1.215533in}{1.455213in}}%
\pgfpathlineto{\pgfqpoint{1.223864in}{1.458523in}}%
\pgfpathlineto{\pgfqpoint{1.235577in}{1.457938in}}%
\pgfpathlineto{\pgfqpoint{1.242483in}{1.455213in}}%
\pgfpathlineto{\pgfqpoint{1.247289in}{1.453738in}}%
\pgfpathlineto{\pgfqpoint{1.259002in}{1.446912in}}%
\pgfpathlineto{\pgfqpoint{1.269936in}{1.443084in}}%
\pgfpathlineto{\pgfqpoint{1.270715in}{1.442908in}}%
\pgfpathlineto{\pgfqpoint{1.270973in}{1.443084in}}%
\pgfpathlineto{\pgfqpoint{1.282427in}{1.452258in}}%
\pgfpathlineto{\pgfqpoint{1.285155in}{1.455213in}}%
\pgfpathlineto{\pgfqpoint{1.294140in}{1.466269in}}%
\pgfpathlineto{\pgfqpoint{1.295157in}{1.467342in}}%
\pgfpathlineto{\pgfqpoint{1.303111in}{1.479471in}}%
\pgfpathlineto{\pgfqpoint{1.305853in}{1.484079in}}%
\pgfpathlineto{\pgfqpoint{1.312146in}{1.491600in}}%
\pgfpathlineto{\pgfqpoint{1.305853in}{1.491600in}}%
\pgfpathlineto{\pgfqpoint{1.300415in}{1.491600in}}%
\pgfpathlineto{\pgfqpoint{1.294140in}{1.480999in}}%
\pgfpathlineto{\pgfqpoint{1.292870in}{1.479471in}}%
\pgfpathlineto{\pgfqpoint{1.284238in}{1.467342in}}%
\pgfpathlineto{\pgfqpoint{1.282427in}{1.465094in}}%
\pgfpathlineto{\pgfqpoint{1.270715in}{1.459991in}}%
\pgfpathlineto{\pgfqpoint{1.259002in}{1.464700in}}%
\pgfpathlineto{\pgfqpoint{1.251913in}{1.467342in}}%
\pgfpathlineto{\pgfqpoint{1.247289in}{1.468879in}}%
\pgfpathlineto{\pgfqpoint{1.235577in}{1.475225in}}%
\pgfpathlineto{\pgfqpoint{1.226686in}{1.479471in}}%
\pgfpathlineto{\pgfqpoint{1.223864in}{1.481866in}}%
\pgfpathlineto{\pgfqpoint{1.212151in}{1.481482in}}%
\pgfpathlineto{\pgfqpoint{1.209454in}{1.479471in}}%
\pgfpathlineto{\pgfqpoint{1.200439in}{1.471140in}}%
\pgfpathlineto{\pgfqpoint{1.197198in}{1.467342in}}%
\pgfpathlineto{\pgfqpoint{1.188726in}{1.458866in}}%
\pgfpathlineto{\pgfqpoint{1.184088in}{1.455213in}}%
\pgfpathlineto{\pgfqpoint{1.177013in}{1.450337in}}%
\pgfpathlineto{\pgfqpoint{1.165301in}{1.448241in}}%
\pgfpathlineto{\pgfqpoint{1.153588in}{1.445754in}}%
\pgfpathlineto{\pgfqpoint{1.144731in}{1.443084in}}%
\pgfpathlineto{\pgfqpoint{1.141875in}{1.442081in}}%
\pgfpathlineto{\pgfqpoint{1.130163in}{1.437997in}}%
\pgfpathlineto{\pgfqpoint{1.118450in}{1.438882in}}%
\pgfpathlineto{\pgfqpoint{1.110851in}{1.443084in}}%
\pgfpathlineto{\pgfqpoint{1.106737in}{1.447032in}}%
\pgfpathlineto{\pgfqpoint{1.100085in}{1.455213in}}%
\pgfpathlineto{\pgfqpoint{1.095024in}{1.463477in}}%
\pgfpathlineto{\pgfqpoint{1.092985in}{1.467342in}}%
\pgfpathlineto{\pgfqpoint{1.088384in}{1.479471in}}%
\pgfpathlineto{\pgfqpoint{1.085106in}{1.491600in}}%
\pgfpathlineto{\pgfqpoint{1.083312in}{1.491600in}}%
\pgfpathlineto{\pgfqpoint{1.071599in}{1.491600in}}%
\pgfpathlineto{\pgfqpoint{1.059886in}{1.491600in}}%
\pgfpathlineto{\pgfqpoint{1.056045in}{1.491600in}}%
\pgfpathlineto{\pgfqpoint{1.056957in}{1.479471in}}%
\pgfpathlineto{\pgfqpoint{1.057915in}{1.467342in}}%
\pgfpathlineto{\pgfqpoint{1.058945in}{1.455213in}}%
\pgfpathlineto{\pgfqpoint{1.059886in}{1.447638in}}%
\pgfpathlineto{\pgfqpoint{1.060452in}{1.443084in}}%
\pgfpathlineto{\pgfqpoint{1.063777in}{1.430956in}}%
\pgfpathlineto{\pgfqpoint{1.068429in}{1.418827in}}%
\pgfpathlineto{\pgfqpoint{1.065133in}{1.406698in}}%
\pgfpathlineto{\pgfqpoint{1.063634in}{1.394569in}}%
\pgfpathlineto{\pgfqpoint{1.066457in}{1.382440in}}%
\pgfpathclose%
\pgfusepath{fill}%
\end{pgfscope}%
\begin{pgfscope}%
\pgfpathrectangle{\pgfqpoint{0.211875in}{0.211875in}}{\pgfqpoint{1.313625in}{1.279725in}}%
\pgfusepath{clip}%
\pgfsetbuttcap%
\pgfsetroundjoin%
\definecolor{currentfill}{rgb}{0.644838,0.098089,0.355336}%
\pgfsetfillcolor{currentfill}%
\pgfsetlinewidth{0.000000pt}%
\definecolor{currentstroke}{rgb}{0.000000,0.000000,0.000000}%
\pgfsetstrokecolor{currentstroke}%
\pgfsetdash{}{0pt}%
\pgfpathmoveto{\pgfqpoint{0.403976in}{0.290841in}}%
\pgfpathlineto{\pgfqpoint{0.413031in}{0.290841in}}%
\pgfpathlineto{\pgfqpoint{0.415688in}{0.292757in}}%
\pgfpathlineto{\pgfqpoint{0.427401in}{0.301312in}}%
\pgfpathlineto{\pgfqpoint{0.429539in}{0.302970in}}%
\pgfpathlineto{\pgfqpoint{0.439114in}{0.309981in}}%
\pgfpathlineto{\pgfqpoint{0.446331in}{0.315099in}}%
\pgfpathlineto{\pgfqpoint{0.450827in}{0.318674in}}%
\pgfpathlineto{\pgfqpoint{0.462539in}{0.327122in}}%
\pgfpathlineto{\pgfqpoint{0.462694in}{0.327228in}}%
\pgfpathlineto{\pgfqpoint{0.474252in}{0.336168in}}%
\pgfpathlineto{\pgfqpoint{0.478783in}{0.339357in}}%
\pgfpathlineto{\pgfqpoint{0.485965in}{0.345032in}}%
\pgfpathlineto{\pgfqpoint{0.494703in}{0.351486in}}%
\pgfpathlineto{\pgfqpoint{0.497677in}{0.353975in}}%
\pgfpathlineto{\pgfqpoint{0.509390in}{0.362829in}}%
\pgfpathlineto{\pgfqpoint{0.510469in}{0.363615in}}%
\pgfpathlineto{\pgfqpoint{0.521103in}{0.372848in}}%
\pgfpathlineto{\pgfqpoint{0.524668in}{0.375743in}}%
\pgfpathlineto{\pgfqpoint{0.532815in}{0.382691in}}%
\pgfpathlineto{\pgfqpoint{0.539099in}{0.387872in}}%
\pgfpathlineto{\pgfqpoint{0.544528in}{0.392558in}}%
\pgfpathlineto{\pgfqpoint{0.556045in}{0.400001in}}%
\pgfpathlineto{\pgfqpoint{0.556241in}{0.400174in}}%
\pgfpathlineto{\pgfqpoint{0.567953in}{0.408944in}}%
\pgfpathlineto{\pgfqpoint{0.573439in}{0.412130in}}%
\pgfpathlineto{\pgfqpoint{0.579666in}{0.416900in}}%
\pgfpathlineto{\pgfqpoint{0.590746in}{0.424259in}}%
\pgfpathlineto{\pgfqpoint{0.591379in}{0.424720in}}%
\pgfpathlineto{\pgfqpoint{0.603091in}{0.432971in}}%
\pgfpathlineto{\pgfqpoint{0.608639in}{0.436388in}}%
\pgfpathlineto{\pgfqpoint{0.614804in}{0.440364in}}%
\pgfpathlineto{\pgfqpoint{0.626517in}{0.447335in}}%
\pgfpathlineto{\pgfqpoint{0.629146in}{0.448517in}}%
\pgfpathlineto{\pgfqpoint{0.638230in}{0.452741in}}%
\pgfpathlineto{\pgfqpoint{0.649942in}{0.458016in}}%
\pgfpathlineto{\pgfqpoint{0.656823in}{0.460646in}}%
\pgfpathlineto{\pgfqpoint{0.661655in}{0.462706in}}%
\pgfpathlineto{\pgfqpoint{0.673368in}{0.465769in}}%
\pgfpathlineto{\pgfqpoint{0.685080in}{0.466894in}}%
\pgfpathlineto{\pgfqpoint{0.696793in}{0.466868in}}%
\pgfpathlineto{\pgfqpoint{0.708506in}{0.465919in}}%
\pgfpathlineto{\pgfqpoint{0.720218in}{0.463411in}}%
\pgfpathlineto{\pgfqpoint{0.731314in}{0.460646in}}%
\pgfpathlineto{\pgfqpoint{0.731931in}{0.460404in}}%
\pgfpathlineto{\pgfqpoint{0.743644in}{0.451912in}}%
\pgfpathlineto{\pgfqpoint{0.747050in}{0.448517in}}%
\pgfpathlineto{\pgfqpoint{0.755356in}{0.437658in}}%
\pgfpathlineto{\pgfqpoint{0.756327in}{0.436388in}}%
\pgfpathlineto{\pgfqpoint{0.760155in}{0.424259in}}%
\pgfpathlineto{\pgfqpoint{0.759116in}{0.412130in}}%
\pgfpathlineto{\pgfqpoint{0.755356in}{0.400725in}}%
\pgfpathlineto{\pgfqpoint{0.755009in}{0.400001in}}%
\pgfpathlineto{\pgfqpoint{0.745681in}{0.387872in}}%
\pgfpathlineto{\pgfqpoint{0.743644in}{0.386042in}}%
\pgfpathlineto{\pgfqpoint{0.731931in}{0.377723in}}%
\pgfpathlineto{\pgfqpoint{0.729090in}{0.375743in}}%
\pgfpathlineto{\pgfqpoint{0.720218in}{0.368448in}}%
\pgfpathlineto{\pgfqpoint{0.716828in}{0.363615in}}%
\pgfpathlineto{\pgfqpoint{0.720218in}{0.360168in}}%
\pgfpathlineto{\pgfqpoint{0.730905in}{0.351486in}}%
\pgfpathlineto{\pgfqpoint{0.731931in}{0.350850in}}%
\pgfpathlineto{\pgfqpoint{0.743644in}{0.344808in}}%
\pgfpathlineto{\pgfqpoint{0.753204in}{0.339357in}}%
\pgfpathlineto{\pgfqpoint{0.755356in}{0.337787in}}%
\pgfpathlineto{\pgfqpoint{0.767069in}{0.329553in}}%
\pgfpathlineto{\pgfqpoint{0.769898in}{0.327228in}}%
\pgfpathlineto{\pgfqpoint{0.778782in}{0.319534in}}%
\pgfpathlineto{\pgfqpoint{0.783996in}{0.315099in}}%
\pgfpathlineto{\pgfqpoint{0.790495in}{0.311319in}}%
\pgfpathlineto{\pgfqpoint{0.800654in}{0.302970in}}%
\pgfpathlineto{\pgfqpoint{0.802207in}{0.301440in}}%
\pgfpathlineto{\pgfqpoint{0.813781in}{0.290841in}}%
\pgfpathlineto{\pgfqpoint{0.813920in}{0.290841in}}%
\pgfpathlineto{\pgfqpoint{0.825633in}{0.290841in}}%
\pgfpathlineto{\pgfqpoint{0.836561in}{0.290841in}}%
\pgfpathlineto{\pgfqpoint{0.837345in}{0.291060in}}%
\pgfpathlineto{\pgfqpoint{0.849058in}{0.294011in}}%
\pgfpathlineto{\pgfqpoint{0.860771in}{0.296759in}}%
\pgfpathlineto{\pgfqpoint{0.872483in}{0.299351in}}%
\pgfpathlineto{\pgfqpoint{0.884196in}{0.301675in}}%
\pgfpathlineto{\pgfqpoint{0.892033in}{0.302970in}}%
\pgfpathlineto{\pgfqpoint{0.895909in}{0.303513in}}%
\pgfpathlineto{\pgfqpoint{0.907621in}{0.305087in}}%
\pgfpathlineto{\pgfqpoint{0.919334in}{0.306535in}}%
\pgfpathlineto{\pgfqpoint{0.931047in}{0.307775in}}%
\pgfpathlineto{\pgfqpoint{0.942759in}{0.308803in}}%
\pgfpathlineto{\pgfqpoint{0.954472in}{0.307817in}}%
\pgfpathlineto{\pgfqpoint{0.966185in}{0.306373in}}%
\pgfpathlineto{\pgfqpoint{0.977898in}{0.305046in}}%
\pgfpathlineto{\pgfqpoint{0.989610in}{0.303792in}}%
\pgfpathlineto{\pgfqpoint{0.991000in}{0.302970in}}%
\pgfpathlineto{\pgfqpoint{1.001323in}{0.296480in}}%
\pgfpathlineto{\pgfqpoint{1.005974in}{0.290841in}}%
\pgfpathlineto{\pgfqpoint{1.013036in}{0.290841in}}%
\pgfpathlineto{\pgfqpoint{1.024748in}{0.290841in}}%
\pgfpathlineto{\pgfqpoint{1.028569in}{0.290841in}}%
\pgfpathlineto{\pgfqpoint{1.024748in}{0.293361in}}%
\pgfpathlineto{\pgfqpoint{1.014240in}{0.302970in}}%
\pgfpathlineto{\pgfqpoint{1.013036in}{0.304425in}}%
\pgfpathlineto{\pgfqpoint{1.003381in}{0.315099in}}%
\pgfpathlineto{\pgfqpoint{1.001323in}{0.317461in}}%
\pgfpathlineto{\pgfqpoint{0.989610in}{0.323753in}}%
\pgfpathlineto{\pgfqpoint{0.977898in}{0.326955in}}%
\pgfpathlineto{\pgfqpoint{0.976862in}{0.327228in}}%
\pgfpathlineto{\pgfqpoint{0.966185in}{0.332397in}}%
\pgfpathlineto{\pgfqpoint{0.954472in}{0.335738in}}%
\pgfpathlineto{\pgfqpoint{0.942759in}{0.335556in}}%
\pgfpathlineto{\pgfqpoint{0.931047in}{0.334681in}}%
\pgfpathlineto{\pgfqpoint{0.919334in}{0.333573in}}%
\pgfpathlineto{\pgfqpoint{0.907621in}{0.332428in}}%
\pgfpathlineto{\pgfqpoint{0.895909in}{0.331614in}}%
\pgfpathlineto{\pgfqpoint{0.884196in}{0.330827in}}%
\pgfpathlineto{\pgfqpoint{0.872483in}{0.330968in}}%
\pgfpathlineto{\pgfqpoint{0.860771in}{0.332006in}}%
\pgfpathlineto{\pgfqpoint{0.849058in}{0.332757in}}%
\pgfpathlineto{\pgfqpoint{0.837345in}{0.333392in}}%
\pgfpathlineto{\pgfqpoint{0.825633in}{0.333862in}}%
\pgfpathlineto{\pgfqpoint{0.813920in}{0.336442in}}%
\pgfpathlineto{\pgfqpoint{0.811251in}{0.339357in}}%
\pgfpathlineto{\pgfqpoint{0.809249in}{0.351486in}}%
\pgfpathlineto{\pgfqpoint{0.813920in}{0.359195in}}%
\pgfpathlineto{\pgfqpoint{0.825633in}{0.359239in}}%
\pgfpathlineto{\pgfqpoint{0.837345in}{0.358305in}}%
\pgfpathlineto{\pgfqpoint{0.849058in}{0.359034in}}%
\pgfpathlineto{\pgfqpoint{0.860771in}{0.361191in}}%
\pgfpathlineto{\pgfqpoint{0.871406in}{0.363615in}}%
\pgfpathlineto{\pgfqpoint{0.872483in}{0.363840in}}%
\pgfpathlineto{\pgfqpoint{0.884196in}{0.365768in}}%
\pgfpathlineto{\pgfqpoint{0.895909in}{0.367919in}}%
\pgfpathlineto{\pgfqpoint{0.907621in}{0.371995in}}%
\pgfpathlineto{\pgfqpoint{0.917129in}{0.375743in}}%
\pgfpathlineto{\pgfqpoint{0.919334in}{0.376690in}}%
\pgfpathlineto{\pgfqpoint{0.931047in}{0.383328in}}%
\pgfpathlineto{\pgfqpoint{0.937129in}{0.387872in}}%
\pgfpathlineto{\pgfqpoint{0.942759in}{0.392323in}}%
\pgfpathlineto{\pgfqpoint{0.951612in}{0.400001in}}%
\pgfpathlineto{\pgfqpoint{0.954472in}{0.402698in}}%
\pgfpathlineto{\pgfqpoint{0.966185in}{0.411368in}}%
\pgfpathlineto{\pgfqpoint{0.967475in}{0.412130in}}%
\pgfpathlineto{\pgfqpoint{0.977898in}{0.420383in}}%
\pgfpathlineto{\pgfqpoint{0.980381in}{0.424259in}}%
\pgfpathlineto{\pgfqpoint{0.986037in}{0.436388in}}%
\pgfpathlineto{\pgfqpoint{0.989610in}{0.439575in}}%
\pgfpathlineto{\pgfqpoint{1.001323in}{0.441615in}}%
\pgfpathlineto{\pgfqpoint{1.013036in}{0.442499in}}%
\pgfpathlineto{\pgfqpoint{1.024748in}{0.443240in}}%
\pgfpathlineto{\pgfqpoint{1.036461in}{0.441445in}}%
\pgfpathlineto{\pgfqpoint{1.048174in}{0.439260in}}%
\pgfpathlineto{\pgfqpoint{1.059886in}{0.437623in}}%
\pgfpathlineto{\pgfqpoint{1.071599in}{0.437164in}}%
\pgfpathlineto{\pgfqpoint{1.083312in}{0.440382in}}%
\pgfpathlineto{\pgfqpoint{1.095024in}{0.442976in}}%
\pgfpathlineto{\pgfqpoint{1.106737in}{0.444212in}}%
\pgfpathlineto{\pgfqpoint{1.118450in}{0.448195in}}%
\pgfpathlineto{\pgfqpoint{1.118785in}{0.448517in}}%
\pgfpathlineto{\pgfqpoint{1.130163in}{0.459314in}}%
\pgfpathlineto{\pgfqpoint{1.131519in}{0.460646in}}%
\pgfpathlineto{\pgfqpoint{1.141875in}{0.470501in}}%
\pgfpathlineto{\pgfqpoint{1.144119in}{0.472774in}}%
\pgfpathlineto{\pgfqpoint{1.153588in}{0.481485in}}%
\pgfpathlineto{\pgfqpoint{1.157044in}{0.484903in}}%
\pgfpathlineto{\pgfqpoint{1.165301in}{0.493142in}}%
\pgfpathlineto{\pgfqpoint{1.169447in}{0.497032in}}%
\pgfpathlineto{\pgfqpoint{1.177013in}{0.504262in}}%
\pgfpathlineto{\pgfqpoint{1.183615in}{0.509161in}}%
\pgfpathlineto{\pgfqpoint{1.188726in}{0.513313in}}%
\pgfpathlineto{\pgfqpoint{1.200439in}{0.519784in}}%
\pgfpathlineto{\pgfqpoint{1.203869in}{0.521290in}}%
\pgfpathlineto{\pgfqpoint{1.212151in}{0.525346in}}%
\pgfpathlineto{\pgfqpoint{1.223864in}{0.529876in}}%
\pgfpathlineto{\pgfqpoint{1.235577in}{0.527911in}}%
\pgfpathlineto{\pgfqpoint{1.242409in}{0.521290in}}%
\pgfpathlineto{\pgfqpoint{1.247289in}{0.515438in}}%
\pgfpathlineto{\pgfqpoint{1.253259in}{0.509161in}}%
\pgfpathlineto{\pgfqpoint{1.259002in}{0.502520in}}%
\pgfpathlineto{\pgfqpoint{1.266028in}{0.497032in}}%
\pgfpathlineto{\pgfqpoint{1.270715in}{0.494475in}}%
\pgfpathlineto{\pgfqpoint{1.282427in}{0.489708in}}%
\pgfpathlineto{\pgfqpoint{1.294032in}{0.484903in}}%
\pgfpathlineto{\pgfqpoint{1.294140in}{0.484859in}}%
\pgfpathlineto{\pgfqpoint{1.305853in}{0.483722in}}%
\pgfpathlineto{\pgfqpoint{1.310716in}{0.484903in}}%
\pgfpathlineto{\pgfqpoint{1.317566in}{0.486970in}}%
\pgfpathlineto{\pgfqpoint{1.329278in}{0.494378in}}%
\pgfpathlineto{\pgfqpoint{1.332464in}{0.497032in}}%
\pgfpathlineto{\pgfqpoint{1.340991in}{0.504168in}}%
\pgfpathlineto{\pgfqpoint{1.352704in}{0.505310in}}%
\pgfpathlineto{\pgfqpoint{1.364416in}{0.498915in}}%
\pgfpathlineto{\pgfqpoint{1.367866in}{0.497032in}}%
\pgfpathlineto{\pgfqpoint{1.376129in}{0.492234in}}%
\pgfpathlineto{\pgfqpoint{1.387842in}{0.486517in}}%
\pgfpathlineto{\pgfqpoint{1.391103in}{0.484903in}}%
\pgfpathlineto{\pgfqpoint{1.399554in}{0.479896in}}%
\pgfpathlineto{\pgfqpoint{1.407721in}{0.472774in}}%
\pgfpathlineto{\pgfqpoint{1.411267in}{0.468232in}}%
\pgfpathlineto{\pgfqpoint{1.416492in}{0.460646in}}%
\pgfpathlineto{\pgfqpoint{1.422980in}{0.449034in}}%
\pgfpathlineto{\pgfqpoint{1.423299in}{0.448517in}}%
\pgfpathlineto{\pgfqpoint{1.429353in}{0.436388in}}%
\pgfpathlineto{\pgfqpoint{1.434621in}{0.424259in}}%
\pgfpathlineto{\pgfqpoint{1.434602in}{0.412130in}}%
\pgfpathlineto{\pgfqpoint{1.431994in}{0.400001in}}%
\pgfpathlineto{\pgfqpoint{1.434692in}{0.392725in}}%
\pgfpathlineto{\pgfqpoint{1.436958in}{0.387872in}}%
\pgfpathlineto{\pgfqpoint{1.446197in}{0.375743in}}%
\pgfpathlineto{\pgfqpoint{1.446405in}{0.375534in}}%
\pgfpathlineto{\pgfqpoint{1.446405in}{0.375743in}}%
\pgfpathlineto{\pgfqpoint{1.446405in}{0.387872in}}%
\pgfpathlineto{\pgfqpoint{1.446405in}{0.400001in}}%
\pgfpathlineto{\pgfqpoint{1.446405in}{0.412130in}}%
\pgfpathlineto{\pgfqpoint{1.446405in}{0.424259in}}%
\pgfpathlineto{\pgfqpoint{1.446405in}{0.436388in}}%
\pgfpathlineto{\pgfqpoint{1.446405in}{0.448517in}}%
\pgfpathlineto{\pgfqpoint{1.446405in}{0.460646in}}%
\pgfpathlineto{\pgfqpoint{1.446405in}{0.472774in}}%
\pgfpathlineto{\pgfqpoint{1.446405in}{0.484903in}}%
\pgfpathlineto{\pgfqpoint{1.446405in}{0.497032in}}%
\pgfpathlineto{\pgfqpoint{1.446405in}{0.509161in}}%
\pgfpathlineto{\pgfqpoint{1.446405in}{0.521290in}}%
\pgfpathlineto{\pgfqpoint{1.446405in}{0.533419in}}%
\pgfpathlineto{\pgfqpoint{1.446405in}{0.545548in}}%
\pgfpathlineto{\pgfqpoint{1.446405in}{0.550009in}}%
\pgfpathlineto{\pgfqpoint{1.439320in}{0.545548in}}%
\pgfpathlineto{\pgfqpoint{1.434692in}{0.543438in}}%
\pgfpathlineto{\pgfqpoint{1.422980in}{0.540398in}}%
\pgfpathlineto{\pgfqpoint{1.411267in}{0.539468in}}%
\pgfpathlineto{\pgfqpoint{1.399554in}{0.539058in}}%
\pgfpathlineto{\pgfqpoint{1.387842in}{0.540626in}}%
\pgfpathlineto{\pgfqpoint{1.376129in}{0.544934in}}%
\pgfpathlineto{\pgfqpoint{1.373205in}{0.545548in}}%
\pgfpathlineto{\pgfqpoint{1.364416in}{0.547052in}}%
\pgfpathlineto{\pgfqpoint{1.352704in}{0.548815in}}%
\pgfpathlineto{\pgfqpoint{1.340991in}{0.552166in}}%
\pgfpathlineto{\pgfqpoint{1.329278in}{0.555796in}}%
\pgfpathlineto{\pgfqpoint{1.326398in}{0.557677in}}%
\pgfpathlineto{\pgfqpoint{1.317566in}{0.567929in}}%
\pgfpathlineto{\pgfqpoint{1.316685in}{0.569805in}}%
\pgfpathlineto{\pgfqpoint{1.311805in}{0.581934in}}%
\pgfpathlineto{\pgfqpoint{1.307341in}{0.594063in}}%
\pgfpathlineto{\pgfqpoint{1.305853in}{0.599677in}}%
\pgfpathlineto{\pgfqpoint{1.304212in}{0.606192in}}%
\pgfpathlineto{\pgfqpoint{1.302749in}{0.618321in}}%
\pgfpathlineto{\pgfqpoint{1.305853in}{0.625909in}}%
\pgfpathlineto{\pgfqpoint{1.307490in}{0.630450in}}%
\pgfpathlineto{\pgfqpoint{1.314566in}{0.642579in}}%
\pgfpathlineto{\pgfqpoint{1.317566in}{0.644484in}}%
\pgfpathlineto{\pgfqpoint{1.329278in}{0.653049in}}%
\pgfpathlineto{\pgfqpoint{1.330947in}{0.654708in}}%
\pgfpathlineto{\pgfqpoint{1.340991in}{0.662194in}}%
\pgfpathlineto{\pgfqpoint{1.345999in}{0.666836in}}%
\pgfpathlineto{\pgfqpoint{1.352704in}{0.674608in}}%
\pgfpathlineto{\pgfqpoint{1.356668in}{0.678965in}}%
\pgfpathlineto{\pgfqpoint{1.364416in}{0.682924in}}%
\pgfpathlineto{\pgfqpoint{1.375086in}{0.678965in}}%
\pgfpathlineto{\pgfqpoint{1.376129in}{0.678180in}}%
\pgfpathlineto{\pgfqpoint{1.387266in}{0.666836in}}%
\pgfpathlineto{\pgfqpoint{1.387842in}{0.666328in}}%
\pgfpathlineto{\pgfqpoint{1.399554in}{0.656841in}}%
\pgfpathlineto{\pgfqpoint{1.402980in}{0.654708in}}%
\pgfpathlineto{\pgfqpoint{1.411267in}{0.647885in}}%
\pgfpathlineto{\pgfqpoint{1.422980in}{0.642671in}}%
\pgfpathlineto{\pgfqpoint{1.423382in}{0.642579in}}%
\pgfpathlineto{\pgfqpoint{1.434692in}{0.638301in}}%
\pgfpathlineto{\pgfqpoint{1.446405in}{0.632773in}}%
\pgfpathlineto{\pgfqpoint{1.446405in}{0.642579in}}%
\pgfpathlineto{\pgfqpoint{1.446405in}{0.649189in}}%
\pgfpathlineto{\pgfqpoint{1.434692in}{0.651744in}}%
\pgfpathlineto{\pgfqpoint{1.422980in}{0.653661in}}%
\pgfpathlineto{\pgfqpoint{1.420995in}{0.654708in}}%
\pgfpathlineto{\pgfqpoint{1.411267in}{0.658772in}}%
\pgfpathlineto{\pgfqpoint{1.399554in}{0.666524in}}%
\pgfpathlineto{\pgfqpoint{1.399171in}{0.666836in}}%
\pgfpathlineto{\pgfqpoint{1.387842in}{0.677010in}}%
\pgfpathlineto{\pgfqpoint{1.385927in}{0.678965in}}%
\pgfpathlineto{\pgfqpoint{1.376129in}{0.685834in}}%
\pgfpathlineto{\pgfqpoint{1.364416in}{0.689212in}}%
\pgfpathlineto{\pgfqpoint{1.352704in}{0.682876in}}%
\pgfpathlineto{\pgfqpoint{1.347230in}{0.678965in}}%
\pgfpathlineto{\pgfqpoint{1.340991in}{0.671990in}}%
\pgfpathlineto{\pgfqpoint{1.335154in}{0.666836in}}%
\pgfpathlineto{\pgfqpoint{1.329278in}{0.662061in}}%
\pgfpathlineto{\pgfqpoint{1.317566in}{0.655687in}}%
\pgfpathlineto{\pgfqpoint{1.315401in}{0.654708in}}%
\pgfpathlineto{\pgfqpoint{1.305853in}{0.648343in}}%
\pgfpathlineto{\pgfqpoint{1.297766in}{0.642579in}}%
\pgfpathlineto{\pgfqpoint{1.294140in}{0.637668in}}%
\pgfpathlineto{\pgfqpoint{1.287962in}{0.630450in}}%
\pgfpathlineto{\pgfqpoint{1.282427in}{0.622450in}}%
\pgfpathlineto{\pgfqpoint{1.277886in}{0.618321in}}%
\pgfpathlineto{\pgfqpoint{1.270715in}{0.609674in}}%
\pgfpathlineto{\pgfqpoint{1.259002in}{0.609768in}}%
\pgfpathlineto{\pgfqpoint{1.247289in}{0.610741in}}%
\pgfpathlineto{\pgfqpoint{1.235577in}{0.610411in}}%
\pgfpathlineto{\pgfqpoint{1.223864in}{0.609609in}}%
\pgfpathlineto{\pgfqpoint{1.213643in}{0.606192in}}%
\pgfpathlineto{\pgfqpoint{1.212151in}{0.605665in}}%
\pgfpathlineto{\pgfqpoint{1.200439in}{0.602155in}}%
\pgfpathlineto{\pgfqpoint{1.188726in}{0.602056in}}%
\pgfpathlineto{\pgfqpoint{1.178370in}{0.606192in}}%
\pgfpathlineto{\pgfqpoint{1.177013in}{0.608011in}}%
\pgfpathlineto{\pgfqpoint{1.173486in}{0.618321in}}%
\pgfpathlineto{\pgfqpoint{1.177013in}{0.625965in}}%
\pgfpathlineto{\pgfqpoint{1.179306in}{0.630450in}}%
\pgfpathlineto{\pgfqpoint{1.177013in}{0.638496in}}%
\pgfpathlineto{\pgfqpoint{1.175763in}{0.642579in}}%
\pgfpathlineto{\pgfqpoint{1.165301in}{0.647080in}}%
\pgfpathlineto{\pgfqpoint{1.153588in}{0.651814in}}%
\pgfpathlineto{\pgfqpoint{1.147418in}{0.654708in}}%
\pgfpathlineto{\pgfqpoint{1.141875in}{0.660702in}}%
\pgfpathlineto{\pgfqpoint{1.138265in}{0.666836in}}%
\pgfpathlineto{\pgfqpoint{1.130163in}{0.677464in}}%
\pgfpathlineto{\pgfqpoint{1.129108in}{0.678965in}}%
\pgfpathlineto{\pgfqpoint{1.122774in}{0.691094in}}%
\pgfpathlineto{\pgfqpoint{1.118450in}{0.699260in}}%
\pgfpathlineto{\pgfqpoint{1.115651in}{0.703223in}}%
\pgfpathlineto{\pgfqpoint{1.118450in}{0.708808in}}%
\pgfpathlineto{\pgfqpoint{1.122766in}{0.715352in}}%
\pgfpathlineto{\pgfqpoint{1.129713in}{0.727481in}}%
\pgfpathlineto{\pgfqpoint{1.130163in}{0.727914in}}%
\pgfpathlineto{\pgfqpoint{1.141057in}{0.739610in}}%
\pgfpathlineto{\pgfqpoint{1.137982in}{0.751739in}}%
\pgfpathlineto{\pgfqpoint{1.130163in}{0.761694in}}%
\pgfpathlineto{\pgfqpoint{1.128339in}{0.763867in}}%
\pgfpathlineto{\pgfqpoint{1.118450in}{0.772521in}}%
\pgfpathlineto{\pgfqpoint{1.114050in}{0.775996in}}%
\pgfpathlineto{\pgfqpoint{1.106737in}{0.780748in}}%
\pgfpathlineto{\pgfqpoint{1.095267in}{0.788125in}}%
\pgfpathlineto{\pgfqpoint{1.095024in}{0.788256in}}%
\pgfpathlineto{\pgfqpoint{1.083312in}{0.794881in}}%
\pgfpathlineto{\pgfqpoint{1.071599in}{0.799571in}}%
\pgfpathlineto{\pgfqpoint{1.068988in}{0.800254in}}%
\pgfpathlineto{\pgfqpoint{1.059886in}{0.804914in}}%
\pgfpathlineto{\pgfqpoint{1.055325in}{0.812383in}}%
\pgfpathlineto{\pgfqpoint{1.054320in}{0.824512in}}%
\pgfpathlineto{\pgfqpoint{1.057162in}{0.836641in}}%
\pgfpathlineto{\pgfqpoint{1.057000in}{0.848770in}}%
\pgfpathlineto{\pgfqpoint{1.056464in}{0.860898in}}%
\pgfpathlineto{\pgfqpoint{1.055303in}{0.873027in}}%
\pgfpathlineto{\pgfqpoint{1.053108in}{0.885156in}}%
\pgfpathlineto{\pgfqpoint{1.049496in}{0.897285in}}%
\pgfpathlineto{\pgfqpoint{1.048174in}{0.899995in}}%
\pgfpathlineto{\pgfqpoint{1.039860in}{0.909414in}}%
\pgfpathlineto{\pgfqpoint{1.036461in}{0.911039in}}%
\pgfpathlineto{\pgfqpoint{1.024748in}{0.909757in}}%
\pgfpathlineto{\pgfqpoint{1.022742in}{0.909414in}}%
\pgfpathlineto{\pgfqpoint{1.013036in}{0.907728in}}%
\pgfpathlineto{\pgfqpoint{1.001323in}{0.905150in}}%
\pgfpathlineto{\pgfqpoint{0.989610in}{0.901020in}}%
\pgfpathlineto{\pgfqpoint{0.980811in}{0.897285in}}%
\pgfpathlineto{\pgfqpoint{0.977898in}{0.895993in}}%
\pgfpathlineto{\pgfqpoint{0.966185in}{0.891415in}}%
\pgfpathlineto{\pgfqpoint{0.954472in}{0.889958in}}%
\pgfpathlineto{\pgfqpoint{0.942759in}{0.890605in}}%
\pgfpathlineto{\pgfqpoint{0.931047in}{0.892795in}}%
\pgfpathlineto{\pgfqpoint{0.926029in}{0.897285in}}%
\pgfpathlineto{\pgfqpoint{0.919334in}{0.907977in}}%
\pgfpathlineto{\pgfqpoint{0.918406in}{0.909414in}}%
\pgfpathlineto{\pgfqpoint{0.909518in}{0.921543in}}%
\pgfpathlineto{\pgfqpoint{0.907621in}{0.923942in}}%
\pgfpathlineto{\pgfqpoint{0.900010in}{0.933672in}}%
\pgfpathlineto{\pgfqpoint{0.895909in}{0.938472in}}%
\pgfpathlineto{\pgfqpoint{0.889451in}{0.945801in}}%
\pgfpathlineto{\pgfqpoint{0.884196in}{0.954336in}}%
\pgfpathlineto{\pgfqpoint{0.882148in}{0.957929in}}%
\pgfpathlineto{\pgfqpoint{0.876278in}{0.970058in}}%
\pgfpathlineto{\pgfqpoint{0.872483in}{0.977825in}}%
\pgfpathlineto{\pgfqpoint{0.870104in}{0.982187in}}%
\pgfpathlineto{\pgfqpoint{0.863771in}{0.994316in}}%
\pgfpathlineto{\pgfqpoint{0.864192in}{1.006445in}}%
\pgfpathlineto{\pgfqpoint{0.867639in}{1.018574in}}%
\pgfpathlineto{\pgfqpoint{0.872049in}{1.030703in}}%
\pgfpathlineto{\pgfqpoint{0.872483in}{1.031333in}}%
\pgfpathlineto{\pgfqpoint{0.880237in}{1.042832in}}%
\pgfpathlineto{\pgfqpoint{0.884196in}{1.048069in}}%
\pgfpathlineto{\pgfqpoint{0.889183in}{1.054960in}}%
\pgfpathlineto{\pgfqpoint{0.895414in}{1.067089in}}%
\pgfpathlineto{\pgfqpoint{0.895909in}{1.068364in}}%
\pgfpathlineto{\pgfqpoint{0.899784in}{1.079218in}}%
\pgfpathlineto{\pgfqpoint{0.903165in}{1.091347in}}%
\pgfpathlineto{\pgfqpoint{0.903951in}{1.103476in}}%
\pgfpathlineto{\pgfqpoint{0.903469in}{1.115605in}}%
\pgfpathlineto{\pgfqpoint{0.902649in}{1.127734in}}%
\pgfpathlineto{\pgfqpoint{0.905243in}{1.139863in}}%
\pgfpathlineto{\pgfqpoint{0.907621in}{1.147192in}}%
\pgfpathlineto{\pgfqpoint{0.909698in}{1.151991in}}%
\pgfpathlineto{\pgfqpoint{0.907621in}{1.155439in}}%
\pgfpathlineto{\pgfqpoint{0.901683in}{1.164120in}}%
\pgfpathlineto{\pgfqpoint{0.895909in}{1.170049in}}%
\pgfpathlineto{\pgfqpoint{0.891134in}{1.176249in}}%
\pgfpathlineto{\pgfqpoint{0.884196in}{1.184658in}}%
\pgfpathlineto{\pgfqpoint{0.879426in}{1.188378in}}%
\pgfpathlineto{\pgfqpoint{0.872483in}{1.191135in}}%
\pgfpathlineto{\pgfqpoint{0.860771in}{1.192850in}}%
\pgfpathlineto{\pgfqpoint{0.849058in}{1.193863in}}%
\pgfpathlineto{\pgfqpoint{0.837345in}{1.194095in}}%
\pgfpathlineto{\pgfqpoint{0.825633in}{1.194192in}}%
\pgfpathlineto{\pgfqpoint{0.813920in}{1.193731in}}%
\pgfpathlineto{\pgfqpoint{0.802207in}{1.192698in}}%
\pgfpathlineto{\pgfqpoint{0.790495in}{1.191464in}}%
\pgfpathlineto{\pgfqpoint{0.778782in}{1.190168in}}%
\pgfpathlineto{\pgfqpoint{0.767069in}{1.188705in}}%
\pgfpathlineto{\pgfqpoint{0.764831in}{1.188378in}}%
\pgfpathlineto{\pgfqpoint{0.755356in}{1.187044in}}%
\pgfpathlineto{\pgfqpoint{0.743644in}{1.185452in}}%
\pgfpathlineto{\pgfqpoint{0.731931in}{1.183895in}}%
\pgfpathlineto{\pgfqpoint{0.720218in}{1.182118in}}%
\pgfpathlineto{\pgfqpoint{0.708506in}{1.179946in}}%
\pgfpathlineto{\pgfqpoint{0.696793in}{1.177668in}}%
\pgfpathlineto{\pgfqpoint{0.689369in}{1.176249in}}%
\pgfpathlineto{\pgfqpoint{0.685080in}{1.175421in}}%
\pgfpathlineto{\pgfqpoint{0.673368in}{1.172967in}}%
\pgfpathlineto{\pgfqpoint{0.661655in}{1.170284in}}%
\pgfpathlineto{\pgfqpoint{0.649942in}{1.167845in}}%
\pgfpathlineto{\pgfqpoint{0.638230in}{1.165416in}}%
\pgfpathlineto{\pgfqpoint{0.632782in}{1.164120in}}%
\pgfpathlineto{\pgfqpoint{0.626517in}{1.162576in}}%
\pgfpathlineto{\pgfqpoint{0.614804in}{1.160325in}}%
\pgfpathlineto{\pgfqpoint{0.603091in}{1.158769in}}%
\pgfpathlineto{\pgfqpoint{0.591379in}{1.157836in}}%
\pgfpathlineto{\pgfqpoint{0.579666in}{1.157242in}}%
\pgfpathlineto{\pgfqpoint{0.567953in}{1.157374in}}%
\pgfpathlineto{\pgfqpoint{0.556241in}{1.157624in}}%
\pgfpathlineto{\pgfqpoint{0.544528in}{1.158164in}}%
\pgfpathlineto{\pgfqpoint{0.532815in}{1.158830in}}%
\pgfpathlineto{\pgfqpoint{0.521103in}{1.159663in}}%
\pgfpathlineto{\pgfqpoint{0.509390in}{1.160690in}}%
\pgfpathlineto{\pgfqpoint{0.497677in}{1.161824in}}%
\pgfpathlineto{\pgfqpoint{0.485965in}{1.163059in}}%
\pgfpathlineto{\pgfqpoint{0.477277in}{1.164120in}}%
\pgfpathlineto{\pgfqpoint{0.474252in}{1.164509in}}%
\pgfpathlineto{\pgfqpoint{0.462539in}{1.166144in}}%
\pgfpathlineto{\pgfqpoint{0.450827in}{1.168552in}}%
\pgfpathlineto{\pgfqpoint{0.439114in}{1.171266in}}%
\pgfpathlineto{\pgfqpoint{0.427401in}{1.174057in}}%
\pgfpathlineto{\pgfqpoint{0.418322in}{1.176249in}}%
\pgfpathlineto{\pgfqpoint{0.415688in}{1.176911in}}%
\pgfpathlineto{\pgfqpoint{0.403976in}{1.179759in}}%
\pgfpathlineto{\pgfqpoint{0.392263in}{1.182560in}}%
\pgfpathlineto{\pgfqpoint{0.380550in}{1.184437in}}%
\pgfpathlineto{\pgfqpoint{0.368838in}{1.179851in}}%
\pgfpathlineto{\pgfqpoint{0.367513in}{1.176249in}}%
\pgfpathlineto{\pgfqpoint{0.368838in}{1.174185in}}%
\pgfpathlineto{\pgfqpoint{0.374667in}{1.164120in}}%
\pgfpathlineto{\pgfqpoint{0.380550in}{1.155671in}}%
\pgfpathlineto{\pgfqpoint{0.382769in}{1.151991in}}%
\pgfpathlineto{\pgfqpoint{0.392263in}{1.141120in}}%
\pgfpathlineto{\pgfqpoint{0.393252in}{1.139863in}}%
\pgfpathlineto{\pgfqpoint{0.403976in}{1.130505in}}%
\pgfpathlineto{\pgfqpoint{0.406956in}{1.127734in}}%
\pgfpathlineto{\pgfqpoint{0.415688in}{1.120372in}}%
\pgfpathlineto{\pgfqpoint{0.421013in}{1.115605in}}%
\pgfpathlineto{\pgfqpoint{0.427401in}{1.110177in}}%
\pgfpathlineto{\pgfqpoint{0.434717in}{1.103476in}}%
\pgfpathlineto{\pgfqpoint{0.439114in}{1.099365in}}%
\pgfpathlineto{\pgfqpoint{0.447218in}{1.091347in}}%
\pgfpathlineto{\pgfqpoint{0.450827in}{1.087912in}}%
\pgfpathlineto{\pgfqpoint{0.459507in}{1.079218in}}%
\pgfpathlineto{\pgfqpoint{0.462539in}{1.076264in}}%
\pgfpathlineto{\pgfqpoint{0.471591in}{1.067089in}}%
\pgfpathlineto{\pgfqpoint{0.474252in}{1.064461in}}%
\pgfpathlineto{\pgfqpoint{0.483593in}{1.054960in}}%
\pgfpathlineto{\pgfqpoint{0.485965in}{1.052077in}}%
\pgfpathlineto{\pgfqpoint{0.493702in}{1.042832in}}%
\pgfpathlineto{\pgfqpoint{0.497677in}{1.037337in}}%
\pgfpathlineto{\pgfqpoint{0.502336in}{1.030703in}}%
\pgfpathlineto{\pgfqpoint{0.509390in}{1.020501in}}%
\pgfpathlineto{\pgfqpoint{0.510765in}{1.018574in}}%
\pgfpathlineto{\pgfqpoint{0.521103in}{1.006462in}}%
\pgfpathlineto{\pgfqpoint{0.521117in}{1.006445in}}%
\pgfpathlineto{\pgfqpoint{0.530108in}{0.994316in}}%
\pgfpathlineto{\pgfqpoint{0.532815in}{0.990315in}}%
\pgfpathlineto{\pgfqpoint{0.538081in}{0.982187in}}%
\pgfpathlineto{\pgfqpoint{0.544528in}{0.971823in}}%
\pgfpathlineto{\pgfqpoint{0.545607in}{0.970058in}}%
\pgfpathlineto{\pgfqpoint{0.552037in}{0.957929in}}%
\pgfpathlineto{\pgfqpoint{0.556241in}{0.947992in}}%
\pgfpathlineto{\pgfqpoint{0.557635in}{0.945801in}}%
\pgfpathlineto{\pgfqpoint{0.567953in}{0.936999in}}%
\pgfpathlineto{\pgfqpoint{0.572155in}{0.933672in}}%
\pgfpathlineto{\pgfqpoint{0.579666in}{0.927455in}}%
\pgfpathlineto{\pgfqpoint{0.586223in}{0.921543in}}%
\pgfpathlineto{\pgfqpoint{0.591379in}{0.916045in}}%
\pgfpathlineto{\pgfqpoint{0.596811in}{0.909414in}}%
\pgfpathlineto{\pgfqpoint{0.603091in}{0.902212in}}%
\pgfpathlineto{\pgfqpoint{0.607457in}{0.897285in}}%
\pgfpathlineto{\pgfqpoint{0.614804in}{0.887049in}}%
\pgfpathlineto{\pgfqpoint{0.616165in}{0.885156in}}%
\pgfpathlineto{\pgfqpoint{0.624560in}{0.873027in}}%
\pgfpathlineto{\pgfqpoint{0.626517in}{0.870221in}}%
\pgfpathlineto{\pgfqpoint{0.633168in}{0.860898in}}%
\pgfpathlineto{\pgfqpoint{0.638230in}{0.854304in}}%
\pgfpathlineto{\pgfqpoint{0.642416in}{0.848770in}}%
\pgfpathlineto{\pgfqpoint{0.649942in}{0.840704in}}%
\pgfpathlineto{\pgfqpoint{0.655404in}{0.836641in}}%
\pgfpathlineto{\pgfqpoint{0.661655in}{0.833459in}}%
\pgfpathlineto{\pgfqpoint{0.673368in}{0.827855in}}%
\pgfpathlineto{\pgfqpoint{0.680231in}{0.824512in}}%
\pgfpathlineto{\pgfqpoint{0.685080in}{0.822739in}}%
\pgfpathlineto{\pgfqpoint{0.696793in}{0.818457in}}%
\pgfpathlineto{\pgfqpoint{0.708506in}{0.814129in}}%
\pgfpathlineto{\pgfqpoint{0.713652in}{0.812383in}}%
\pgfpathlineto{\pgfqpoint{0.720218in}{0.810589in}}%
\pgfpathlineto{\pgfqpoint{0.731931in}{0.807392in}}%
\pgfpathlineto{\pgfqpoint{0.743644in}{0.804165in}}%
\pgfpathlineto{\pgfqpoint{0.755356in}{0.800950in}}%
\pgfpathlineto{\pgfqpoint{0.757887in}{0.800254in}}%
\pgfpathlineto{\pgfqpoint{0.767069in}{0.797960in}}%
\pgfpathlineto{\pgfqpoint{0.778782in}{0.794706in}}%
\pgfpathlineto{\pgfqpoint{0.790495in}{0.791819in}}%
\pgfpathlineto{\pgfqpoint{0.802207in}{0.789194in}}%
\pgfpathlineto{\pgfqpoint{0.808423in}{0.788125in}}%
\pgfpathlineto{\pgfqpoint{0.813920in}{0.787268in}}%
\pgfpathlineto{\pgfqpoint{0.825633in}{0.786115in}}%
\pgfpathlineto{\pgfqpoint{0.837345in}{0.786553in}}%
\pgfpathlineto{\pgfqpoint{0.849058in}{0.786187in}}%
\pgfpathlineto{\pgfqpoint{0.860412in}{0.775996in}}%
\pgfpathlineto{\pgfqpoint{0.860771in}{0.775473in}}%
\pgfpathlineto{\pgfqpoint{0.870568in}{0.763867in}}%
\pgfpathlineto{\pgfqpoint{0.869799in}{0.751739in}}%
\pgfpathlineto{\pgfqpoint{0.866902in}{0.739610in}}%
\pgfpathlineto{\pgfqpoint{0.865957in}{0.727481in}}%
\pgfpathlineto{\pgfqpoint{0.860771in}{0.725066in}}%
\pgfpathlineto{\pgfqpoint{0.849058in}{0.720393in}}%
\pgfpathlineto{\pgfqpoint{0.837345in}{0.715920in}}%
\pgfpathlineto{\pgfqpoint{0.835760in}{0.715352in}}%
\pgfpathlineto{\pgfqpoint{0.825633in}{0.707627in}}%
\pgfpathlineto{\pgfqpoint{0.820238in}{0.703223in}}%
\pgfpathlineto{\pgfqpoint{0.817917in}{0.691094in}}%
\pgfpathlineto{\pgfqpoint{0.825633in}{0.686150in}}%
\pgfpathlineto{\pgfqpoint{0.837345in}{0.679404in}}%
\pgfpathlineto{\pgfqpoint{0.838064in}{0.678965in}}%
\pgfpathlineto{\pgfqpoint{0.849058in}{0.671887in}}%
\pgfpathlineto{\pgfqpoint{0.858284in}{0.666836in}}%
\pgfpathlineto{\pgfqpoint{0.860771in}{0.665086in}}%
\pgfpathlineto{\pgfqpoint{0.869423in}{0.654708in}}%
\pgfpathlineto{\pgfqpoint{0.868983in}{0.642579in}}%
\pgfpathlineto{\pgfqpoint{0.868947in}{0.630450in}}%
\pgfpathlineto{\pgfqpoint{0.868455in}{0.618321in}}%
\pgfpathlineto{\pgfqpoint{0.870202in}{0.606192in}}%
\pgfpathlineto{\pgfqpoint{0.871800in}{0.594063in}}%
\pgfpathlineto{\pgfqpoint{0.872483in}{0.585323in}}%
\pgfpathlineto{\pgfqpoint{0.872789in}{0.581934in}}%
\pgfpathlineto{\pgfqpoint{0.877118in}{0.569805in}}%
\pgfpathlineto{\pgfqpoint{0.883844in}{0.557677in}}%
\pgfpathlineto{\pgfqpoint{0.884196in}{0.556016in}}%
\pgfpathlineto{\pgfqpoint{0.885871in}{0.545548in}}%
\pgfpathlineto{\pgfqpoint{0.884196in}{0.540501in}}%
\pgfpathlineto{\pgfqpoint{0.881598in}{0.533419in}}%
\pgfpathlineto{\pgfqpoint{0.880283in}{0.521290in}}%
\pgfpathlineto{\pgfqpoint{0.872483in}{0.510385in}}%
\pgfpathlineto{\pgfqpoint{0.871130in}{0.509161in}}%
\pgfpathlineto{\pgfqpoint{0.860771in}{0.502799in}}%
\pgfpathlineto{\pgfqpoint{0.849058in}{0.497263in}}%
\pgfpathlineto{\pgfqpoint{0.848526in}{0.497032in}}%
\pgfpathlineto{\pgfqpoint{0.837345in}{0.491225in}}%
\pgfpathlineto{\pgfqpoint{0.825633in}{0.485251in}}%
\pgfpathlineto{\pgfqpoint{0.824919in}{0.484903in}}%
\pgfpathlineto{\pgfqpoint{0.813920in}{0.477977in}}%
\pgfpathlineto{\pgfqpoint{0.804971in}{0.472774in}}%
\pgfpathlineto{\pgfqpoint{0.802207in}{0.470857in}}%
\pgfpathlineto{\pgfqpoint{0.790495in}{0.465564in}}%
\pgfpathlineto{\pgfqpoint{0.782689in}{0.472774in}}%
\pgfpathlineto{\pgfqpoint{0.778782in}{0.475495in}}%
\pgfpathlineto{\pgfqpoint{0.767069in}{0.482728in}}%
\pgfpathlineto{\pgfqpoint{0.761969in}{0.484903in}}%
\pgfpathlineto{\pgfqpoint{0.755356in}{0.487970in}}%
\pgfpathlineto{\pgfqpoint{0.743644in}{0.494903in}}%
\pgfpathlineto{\pgfqpoint{0.739179in}{0.497032in}}%
\pgfpathlineto{\pgfqpoint{0.731931in}{0.500278in}}%
\pgfpathlineto{\pgfqpoint{0.720218in}{0.503891in}}%
\pgfpathlineto{\pgfqpoint{0.708506in}{0.506131in}}%
\pgfpathlineto{\pgfqpoint{0.696793in}{0.506841in}}%
\pgfpathlineto{\pgfqpoint{0.685080in}{0.504398in}}%
\pgfpathlineto{\pgfqpoint{0.673368in}{0.500452in}}%
\pgfpathlineto{\pgfqpoint{0.664704in}{0.497032in}}%
\pgfpathlineto{\pgfqpoint{0.661655in}{0.495757in}}%
\pgfpathlineto{\pgfqpoint{0.649942in}{0.490381in}}%
\pgfpathlineto{\pgfqpoint{0.639218in}{0.484903in}}%
\pgfpathlineto{\pgfqpoint{0.638230in}{0.484387in}}%
\pgfpathlineto{\pgfqpoint{0.626517in}{0.478407in}}%
\pgfpathlineto{\pgfqpoint{0.615693in}{0.472774in}}%
\pgfpathlineto{\pgfqpoint{0.614804in}{0.472299in}}%
\pgfpathlineto{\pgfqpoint{0.603091in}{0.466782in}}%
\pgfpathlineto{\pgfqpoint{0.592134in}{0.460646in}}%
\pgfpathlineto{\pgfqpoint{0.591379in}{0.460245in}}%
\pgfpathlineto{\pgfqpoint{0.579666in}{0.455388in}}%
\pgfpathlineto{\pgfqpoint{0.569316in}{0.448517in}}%
\pgfpathlineto{\pgfqpoint{0.567953in}{0.447629in}}%
\pgfpathlineto{\pgfqpoint{0.556241in}{0.437537in}}%
\pgfpathlineto{\pgfqpoint{0.555430in}{0.436388in}}%
\pgfpathlineto{\pgfqpoint{0.544528in}{0.425220in}}%
\pgfpathlineto{\pgfqpoint{0.543745in}{0.424259in}}%
\pgfpathlineto{\pgfqpoint{0.532815in}{0.412745in}}%
\pgfpathlineto{\pgfqpoint{0.532305in}{0.412130in}}%
\pgfpathlineto{\pgfqpoint{0.521103in}{0.400063in}}%
\pgfpathlineto{\pgfqpoint{0.521047in}{0.400001in}}%
\pgfpathlineto{\pgfqpoint{0.509843in}{0.387872in}}%
\pgfpathlineto{\pgfqpoint{0.509390in}{0.387486in}}%
\pgfpathlineto{\pgfqpoint{0.498427in}{0.375743in}}%
\pgfpathlineto{\pgfqpoint{0.497677in}{0.375091in}}%
\pgfpathlineto{\pgfqpoint{0.485965in}{0.364724in}}%
\pgfpathlineto{\pgfqpoint{0.484897in}{0.363615in}}%
\pgfpathlineto{\pgfqpoint{0.474252in}{0.354405in}}%
\pgfpathlineto{\pgfqpoint{0.471280in}{0.351486in}}%
\pgfpathlineto{\pgfqpoint{0.462539in}{0.344733in}}%
\pgfpathlineto{\pgfqpoint{0.456500in}{0.339357in}}%
\pgfpathlineto{\pgfqpoint{0.450827in}{0.334808in}}%
\pgfpathlineto{\pgfqpoint{0.442052in}{0.327228in}}%
\pgfpathlineto{\pgfqpoint{0.439114in}{0.325026in}}%
\pgfpathlineto{\pgfqpoint{0.427401in}{0.315304in}}%
\pgfpathlineto{\pgfqpoint{0.427168in}{0.315099in}}%
\pgfpathlineto{\pgfqpoint{0.415688in}{0.306942in}}%
\pgfpathlineto{\pgfqpoint{0.410749in}{0.302970in}}%
\pgfpathlineto{\pgfqpoint{0.403976in}{0.298260in}}%
\pgfpathlineto{\pgfqpoint{0.394097in}{0.290841in}}%
\pgfpathclose%
\pgfpathmoveto{\pgfqpoint{0.951165in}{0.497032in}}%
\pgfpathlineto{\pgfqpoint{0.942759in}{0.505875in}}%
\pgfpathlineto{\pgfqpoint{0.940740in}{0.509161in}}%
\pgfpathlineto{\pgfqpoint{0.939971in}{0.521290in}}%
\pgfpathlineto{\pgfqpoint{0.942040in}{0.533419in}}%
\pgfpathlineto{\pgfqpoint{0.941272in}{0.545548in}}%
\pgfpathlineto{\pgfqpoint{0.939678in}{0.557677in}}%
\pgfpathlineto{\pgfqpoint{0.939289in}{0.569805in}}%
\pgfpathlineto{\pgfqpoint{0.931047in}{0.578028in}}%
\pgfpathlineto{\pgfqpoint{0.927191in}{0.581934in}}%
\pgfpathlineto{\pgfqpoint{0.919334in}{0.591600in}}%
\pgfpathlineto{\pgfqpoint{0.917223in}{0.594063in}}%
\pgfpathlineto{\pgfqpoint{0.907621in}{0.605095in}}%
\pgfpathlineto{\pgfqpoint{0.907047in}{0.606192in}}%
\pgfpathlineto{\pgfqpoint{0.902837in}{0.618321in}}%
\pgfpathlineto{\pgfqpoint{0.898958in}{0.630450in}}%
\pgfpathlineto{\pgfqpoint{0.896805in}{0.642579in}}%
\pgfpathlineto{\pgfqpoint{0.896756in}{0.654708in}}%
\pgfpathlineto{\pgfqpoint{0.896463in}{0.666836in}}%
\pgfpathlineto{\pgfqpoint{0.895909in}{0.667515in}}%
\pgfpathlineto{\pgfqpoint{0.884196in}{0.676615in}}%
\pgfpathlineto{\pgfqpoint{0.880144in}{0.678965in}}%
\pgfpathlineto{\pgfqpoint{0.872483in}{0.685324in}}%
\pgfpathlineto{\pgfqpoint{0.862240in}{0.691094in}}%
\pgfpathlineto{\pgfqpoint{0.866145in}{0.703223in}}%
\pgfpathlineto{\pgfqpoint{0.872483in}{0.706939in}}%
\pgfpathlineto{\pgfqpoint{0.884196in}{0.712582in}}%
\pgfpathlineto{\pgfqpoint{0.895909in}{0.714327in}}%
\pgfpathlineto{\pgfqpoint{0.902025in}{0.715352in}}%
\pgfpathlineto{\pgfqpoint{0.907621in}{0.717585in}}%
\pgfpathlineto{\pgfqpoint{0.919334in}{0.716247in}}%
\pgfpathlineto{\pgfqpoint{0.920649in}{0.715352in}}%
\pgfpathlineto{\pgfqpoint{0.931047in}{0.713673in}}%
\pgfpathlineto{\pgfqpoint{0.942759in}{0.712967in}}%
\pgfpathlineto{\pgfqpoint{0.950605in}{0.715352in}}%
\pgfpathlineto{\pgfqpoint{0.942759in}{0.721126in}}%
\pgfpathlineto{\pgfqpoint{0.931047in}{0.725383in}}%
\pgfpathlineto{\pgfqpoint{0.929684in}{0.727481in}}%
\pgfpathlineto{\pgfqpoint{0.919334in}{0.735107in}}%
\pgfpathlineto{\pgfqpoint{0.915143in}{0.739610in}}%
\pgfpathlineto{\pgfqpoint{0.919334in}{0.748462in}}%
\pgfpathlineto{\pgfqpoint{0.922092in}{0.751739in}}%
\pgfpathlineto{\pgfqpoint{0.924428in}{0.763867in}}%
\pgfpathlineto{\pgfqpoint{0.919334in}{0.768828in}}%
\pgfpathlineto{\pgfqpoint{0.915175in}{0.775996in}}%
\pgfpathlineto{\pgfqpoint{0.912591in}{0.788125in}}%
\pgfpathlineto{\pgfqpoint{0.907621in}{0.794460in}}%
\pgfpathlineto{\pgfqpoint{0.904796in}{0.800254in}}%
\pgfpathlineto{\pgfqpoint{0.907621in}{0.810540in}}%
\pgfpathlineto{\pgfqpoint{0.908300in}{0.812383in}}%
\pgfpathlineto{\pgfqpoint{0.919334in}{0.824169in}}%
\pgfpathlineto{\pgfqpoint{0.919667in}{0.824512in}}%
\pgfpathlineto{\pgfqpoint{0.931047in}{0.831585in}}%
\pgfpathlineto{\pgfqpoint{0.937415in}{0.836641in}}%
\pgfpathlineto{\pgfqpoint{0.942759in}{0.840418in}}%
\pgfpathlineto{\pgfqpoint{0.954472in}{0.847009in}}%
\pgfpathlineto{\pgfqpoint{0.958820in}{0.848770in}}%
\pgfpathlineto{\pgfqpoint{0.966185in}{0.851521in}}%
\pgfpathlineto{\pgfqpoint{0.976600in}{0.860898in}}%
\pgfpathlineto{\pgfqpoint{0.977898in}{0.861747in}}%
\pgfpathlineto{\pgfqpoint{0.989610in}{0.866026in}}%
\pgfpathlineto{\pgfqpoint{1.001323in}{0.870445in}}%
\pgfpathlineto{\pgfqpoint{1.013020in}{0.873027in}}%
\pgfpathlineto{\pgfqpoint{1.013036in}{0.873030in}}%
\pgfpathlineto{\pgfqpoint{1.013055in}{0.873027in}}%
\pgfpathlineto{\pgfqpoint{1.024748in}{0.868096in}}%
\pgfpathlineto{\pgfqpoint{1.026824in}{0.860898in}}%
\pgfpathlineto{\pgfqpoint{1.026880in}{0.848770in}}%
\pgfpathlineto{\pgfqpoint{1.024748in}{0.842567in}}%
\pgfpathlineto{\pgfqpoint{1.022197in}{0.836641in}}%
\pgfpathlineto{\pgfqpoint{1.014790in}{0.824512in}}%
\pgfpathlineto{\pgfqpoint{1.013036in}{0.821624in}}%
\pgfpathlineto{\pgfqpoint{1.006385in}{0.812383in}}%
\pgfpathlineto{\pgfqpoint{1.004065in}{0.800254in}}%
\pgfpathlineto{\pgfqpoint{1.007957in}{0.788125in}}%
\pgfpathlineto{\pgfqpoint{1.013036in}{0.781005in}}%
\pgfpathlineto{\pgfqpoint{1.018143in}{0.775996in}}%
\pgfpathlineto{\pgfqpoint{1.024748in}{0.772783in}}%
\pgfpathlineto{\pgfqpoint{1.036461in}{0.774050in}}%
\pgfpathlineto{\pgfqpoint{1.048174in}{0.772499in}}%
\pgfpathlineto{\pgfqpoint{1.059886in}{0.772208in}}%
\pgfpathlineto{\pgfqpoint{1.071599in}{0.773632in}}%
\pgfpathlineto{\pgfqpoint{1.083312in}{0.769214in}}%
\pgfpathlineto{\pgfqpoint{1.091650in}{0.763867in}}%
\pgfpathlineto{\pgfqpoint{1.095024in}{0.760503in}}%
\pgfpathlineto{\pgfqpoint{1.104057in}{0.751739in}}%
\pgfpathlineto{\pgfqpoint{1.100776in}{0.739610in}}%
\pgfpathlineto{\pgfqpoint{1.095676in}{0.727481in}}%
\pgfpathlineto{\pgfqpoint{1.095024in}{0.726107in}}%
\pgfpathlineto{\pgfqpoint{1.090093in}{0.715352in}}%
\pgfpathlineto{\pgfqpoint{1.092520in}{0.703223in}}%
\pgfpathlineto{\pgfqpoint{1.095024in}{0.700521in}}%
\pgfpathlineto{\pgfqpoint{1.103871in}{0.691094in}}%
\pgfpathlineto{\pgfqpoint{1.106737in}{0.681129in}}%
\pgfpathlineto{\pgfqpoint{1.107291in}{0.678965in}}%
\pgfpathlineto{\pgfqpoint{1.113739in}{0.666836in}}%
\pgfpathlineto{\pgfqpoint{1.116810in}{0.654708in}}%
\pgfpathlineto{\pgfqpoint{1.118450in}{0.652839in}}%
\pgfpathlineto{\pgfqpoint{1.130163in}{0.643204in}}%
\pgfpathlineto{\pgfqpoint{1.131236in}{0.642579in}}%
\pgfpathlineto{\pgfqpoint{1.140427in}{0.630450in}}%
\pgfpathlineto{\pgfqpoint{1.141875in}{0.623633in}}%
\pgfpathlineto{\pgfqpoint{1.143043in}{0.618321in}}%
\pgfpathlineto{\pgfqpoint{1.144972in}{0.606192in}}%
\pgfpathlineto{\pgfqpoint{1.153588in}{0.595529in}}%
\pgfpathlineto{\pgfqpoint{1.154777in}{0.594063in}}%
\pgfpathlineto{\pgfqpoint{1.158715in}{0.581934in}}%
\pgfpathlineto{\pgfqpoint{1.159896in}{0.569805in}}%
\pgfpathlineto{\pgfqpoint{1.159637in}{0.557677in}}%
\pgfpathlineto{\pgfqpoint{1.156748in}{0.545548in}}%
\pgfpathlineto{\pgfqpoint{1.153588in}{0.540071in}}%
\pgfpathlineto{\pgfqpoint{1.150324in}{0.533419in}}%
\pgfpathlineto{\pgfqpoint{1.142230in}{0.521290in}}%
\pgfpathlineto{\pgfqpoint{1.141875in}{0.520865in}}%
\pgfpathlineto{\pgfqpoint{1.131976in}{0.509161in}}%
\pgfpathlineto{\pgfqpoint{1.130163in}{0.507275in}}%
\pgfpathlineto{\pgfqpoint{1.119249in}{0.497032in}}%
\pgfpathlineto{\pgfqpoint{1.118450in}{0.496359in}}%
\pgfpathlineto{\pgfqpoint{1.106737in}{0.491372in}}%
\pgfpathlineto{\pgfqpoint{1.095024in}{0.490331in}}%
\pgfpathlineto{\pgfqpoint{1.083312in}{0.490754in}}%
\pgfpathlineto{\pgfqpoint{1.071599in}{0.491578in}}%
\pgfpathlineto{\pgfqpoint{1.059886in}{0.492145in}}%
\pgfpathlineto{\pgfqpoint{1.048174in}{0.492310in}}%
\pgfpathlineto{\pgfqpoint{1.036461in}{0.492616in}}%
\pgfpathlineto{\pgfqpoint{1.024748in}{0.492956in}}%
\pgfpathlineto{\pgfqpoint{1.013036in}{0.493166in}}%
\pgfpathlineto{\pgfqpoint{1.001323in}{0.493053in}}%
\pgfpathlineto{\pgfqpoint{0.989610in}{0.491002in}}%
\pgfpathlineto{\pgfqpoint{0.977898in}{0.488764in}}%
\pgfpathlineto{\pgfqpoint{0.966185in}{0.489190in}}%
\pgfpathlineto{\pgfqpoint{0.954472in}{0.495173in}}%
\pgfpathclose%
\pgfpathmoveto{\pgfqpoint{0.812972in}{0.800254in}}%
\pgfpathlineto{\pgfqpoint{0.802207in}{0.802999in}}%
\pgfpathlineto{\pgfqpoint{0.790495in}{0.806138in}}%
\pgfpathlineto{\pgfqpoint{0.778782in}{0.808929in}}%
\pgfpathlineto{\pgfqpoint{0.767069in}{0.812106in}}%
\pgfpathlineto{\pgfqpoint{0.766124in}{0.812383in}}%
\pgfpathlineto{\pgfqpoint{0.755356in}{0.816431in}}%
\pgfpathlineto{\pgfqpoint{0.743644in}{0.820227in}}%
\pgfpathlineto{\pgfqpoint{0.731931in}{0.824355in}}%
\pgfpathlineto{\pgfqpoint{0.731480in}{0.824512in}}%
\pgfpathlineto{\pgfqpoint{0.720218in}{0.829465in}}%
\pgfpathlineto{\pgfqpoint{0.708506in}{0.834887in}}%
\pgfpathlineto{\pgfqpoint{0.704871in}{0.836641in}}%
\pgfpathlineto{\pgfqpoint{0.696793in}{0.841506in}}%
\pgfpathlineto{\pgfqpoint{0.685080in}{0.848073in}}%
\pgfpathlineto{\pgfqpoint{0.683907in}{0.848770in}}%
\pgfpathlineto{\pgfqpoint{0.673368in}{0.857376in}}%
\pgfpathlineto{\pgfqpoint{0.669617in}{0.860898in}}%
\pgfpathlineto{\pgfqpoint{0.661655in}{0.871235in}}%
\pgfpathlineto{\pgfqpoint{0.660284in}{0.873027in}}%
\pgfpathlineto{\pgfqpoint{0.651286in}{0.885156in}}%
\pgfpathlineto{\pgfqpoint{0.649942in}{0.887067in}}%
\pgfpathlineto{\pgfqpoint{0.643341in}{0.897285in}}%
\pgfpathlineto{\pgfqpoint{0.638230in}{0.903025in}}%
\pgfpathlineto{\pgfqpoint{0.632068in}{0.909414in}}%
\pgfpathlineto{\pgfqpoint{0.626517in}{0.914427in}}%
\pgfpathlineto{\pgfqpoint{0.619953in}{0.921543in}}%
\pgfpathlineto{\pgfqpoint{0.614804in}{0.927223in}}%
\pgfpathlineto{\pgfqpoint{0.609328in}{0.933672in}}%
\pgfpathlineto{\pgfqpoint{0.603091in}{0.939109in}}%
\pgfpathlineto{\pgfqpoint{0.594396in}{0.945801in}}%
\pgfpathlineto{\pgfqpoint{0.591379in}{0.948004in}}%
\pgfpathlineto{\pgfqpoint{0.579666in}{0.957150in}}%
\pgfpathlineto{\pgfqpoint{0.578893in}{0.957929in}}%
\pgfpathlineto{\pgfqpoint{0.570336in}{0.970058in}}%
\pgfpathlineto{\pgfqpoint{0.567953in}{0.974140in}}%
\pgfpathlineto{\pgfqpoint{0.563412in}{0.982187in}}%
\pgfpathlineto{\pgfqpoint{0.556524in}{0.994316in}}%
\pgfpathlineto{\pgfqpoint{0.556241in}{0.994785in}}%
\pgfpathlineto{\pgfqpoint{0.549039in}{1.006445in}}%
\pgfpathlineto{\pgfqpoint{0.544528in}{1.012910in}}%
\pgfpathlineto{\pgfqpoint{0.540267in}{1.018574in}}%
\pgfpathlineto{\pgfqpoint{0.532815in}{1.029018in}}%
\pgfpathlineto{\pgfqpoint{0.531556in}{1.030703in}}%
\pgfpathlineto{\pgfqpoint{0.524301in}{1.042832in}}%
\pgfpathlineto{\pgfqpoint{0.521103in}{1.047374in}}%
\pgfpathlineto{\pgfqpoint{0.514908in}{1.054960in}}%
\pgfpathlineto{\pgfqpoint{0.509390in}{1.062026in}}%
\pgfpathlineto{\pgfqpoint{0.504803in}{1.067089in}}%
\pgfpathlineto{\pgfqpoint{0.497677in}{1.076029in}}%
\pgfpathlineto{\pgfqpoint{0.494793in}{1.079218in}}%
\pgfpathlineto{\pgfqpoint{0.485965in}{1.090020in}}%
\pgfpathlineto{\pgfqpoint{0.484820in}{1.091347in}}%
\pgfpathlineto{\pgfqpoint{0.476887in}{1.103476in}}%
\pgfpathlineto{\pgfqpoint{0.485965in}{1.110405in}}%
\pgfpathlineto{\pgfqpoint{0.497677in}{1.113433in}}%
\pgfpathlineto{\pgfqpoint{0.509390in}{1.115081in}}%
\pgfpathlineto{\pgfqpoint{0.513868in}{1.115605in}}%
\pgfpathlineto{\pgfqpoint{0.521103in}{1.116262in}}%
\pgfpathlineto{\pgfqpoint{0.532815in}{1.117039in}}%
\pgfpathlineto{\pgfqpoint{0.544528in}{1.117511in}}%
\pgfpathlineto{\pgfqpoint{0.556241in}{1.117717in}}%
\pgfpathlineto{\pgfqpoint{0.567953in}{1.117764in}}%
\pgfpathlineto{\pgfqpoint{0.579666in}{1.117822in}}%
\pgfpathlineto{\pgfqpoint{0.591379in}{1.118359in}}%
\pgfpathlineto{\pgfqpoint{0.603091in}{1.118825in}}%
\pgfpathlineto{\pgfqpoint{0.614804in}{1.119661in}}%
\pgfpathlineto{\pgfqpoint{0.626517in}{1.120874in}}%
\pgfpathlineto{\pgfqpoint{0.638230in}{1.122368in}}%
\pgfpathlineto{\pgfqpoint{0.649942in}{1.125195in}}%
\pgfpathlineto{\pgfqpoint{0.656612in}{1.127734in}}%
\pgfpathlineto{\pgfqpoint{0.661655in}{1.129909in}}%
\pgfpathlineto{\pgfqpoint{0.673368in}{1.135429in}}%
\pgfpathlineto{\pgfqpoint{0.682527in}{1.139863in}}%
\pgfpathlineto{\pgfqpoint{0.685080in}{1.140902in}}%
\pgfpathlineto{\pgfqpoint{0.696793in}{1.145337in}}%
\pgfpathlineto{\pgfqpoint{0.708506in}{1.149519in}}%
\pgfpathlineto{\pgfqpoint{0.714818in}{1.151991in}}%
\pgfpathlineto{\pgfqpoint{0.720218in}{1.153380in}}%
\pgfpathlineto{\pgfqpoint{0.731931in}{1.156481in}}%
\pgfpathlineto{\pgfqpoint{0.743644in}{1.159398in}}%
\pgfpathlineto{\pgfqpoint{0.755356in}{1.161716in}}%
\pgfpathlineto{\pgfqpoint{0.767069in}{1.163400in}}%
\pgfpathlineto{\pgfqpoint{0.773024in}{1.164120in}}%
\pgfpathlineto{\pgfqpoint{0.778782in}{1.164750in}}%
\pgfpathlineto{\pgfqpoint{0.790495in}{1.165914in}}%
\pgfpathlineto{\pgfqpoint{0.802207in}{1.166924in}}%
\pgfpathlineto{\pgfqpoint{0.813920in}{1.167840in}}%
\pgfpathlineto{\pgfqpoint{0.825633in}{1.168297in}}%
\pgfpathlineto{\pgfqpoint{0.837345in}{1.165563in}}%
\pgfpathlineto{\pgfqpoint{0.839883in}{1.164120in}}%
\pgfpathlineto{\pgfqpoint{0.848971in}{1.151991in}}%
\pgfpathlineto{\pgfqpoint{0.847676in}{1.139863in}}%
\pgfpathlineto{\pgfqpoint{0.846438in}{1.127734in}}%
\pgfpathlineto{\pgfqpoint{0.848646in}{1.115605in}}%
\pgfpathlineto{\pgfqpoint{0.849058in}{1.112525in}}%
\pgfpathlineto{\pgfqpoint{0.850367in}{1.103476in}}%
\pgfpathlineto{\pgfqpoint{0.850686in}{1.091347in}}%
\pgfpathlineto{\pgfqpoint{0.849285in}{1.079218in}}%
\pgfpathlineto{\pgfqpoint{0.849058in}{1.078809in}}%
\pgfpathlineto{\pgfqpoint{0.841664in}{1.067089in}}%
\pgfpathlineto{\pgfqpoint{0.837345in}{1.061977in}}%
\pgfpathlineto{\pgfqpoint{0.831703in}{1.054960in}}%
\pgfpathlineto{\pgfqpoint{0.825633in}{1.045032in}}%
\pgfpathlineto{\pgfqpoint{0.824461in}{1.042832in}}%
\pgfpathlineto{\pgfqpoint{0.819123in}{1.030703in}}%
\pgfpathlineto{\pgfqpoint{0.813920in}{1.021235in}}%
\pgfpathlineto{\pgfqpoint{0.812292in}{1.018574in}}%
\pgfpathlineto{\pgfqpoint{0.808483in}{1.006445in}}%
\pgfpathlineto{\pgfqpoint{0.807447in}{0.994316in}}%
\pgfpathlineto{\pgfqpoint{0.810306in}{0.982187in}}%
\pgfpathlineto{\pgfqpoint{0.813920in}{0.976616in}}%
\pgfpathlineto{\pgfqpoint{0.819336in}{0.970058in}}%
\pgfpathlineto{\pgfqpoint{0.825633in}{0.962650in}}%
\pgfpathlineto{\pgfqpoint{0.828924in}{0.957929in}}%
\pgfpathlineto{\pgfqpoint{0.837345in}{0.948213in}}%
\pgfpathlineto{\pgfqpoint{0.839375in}{0.945801in}}%
\pgfpathlineto{\pgfqpoint{0.849058in}{0.936636in}}%
\pgfpathlineto{\pgfqpoint{0.851669in}{0.933672in}}%
\pgfpathlineto{\pgfqpoint{0.860403in}{0.921543in}}%
\pgfpathlineto{\pgfqpoint{0.860771in}{0.920855in}}%
\pgfpathlineto{\pgfqpoint{0.868080in}{0.909414in}}%
\pgfpathlineto{\pgfqpoint{0.872483in}{0.902888in}}%
\pgfpathlineto{\pgfqpoint{0.875826in}{0.897285in}}%
\pgfpathlineto{\pgfqpoint{0.883021in}{0.885156in}}%
\pgfpathlineto{\pgfqpoint{0.884196in}{0.882686in}}%
\pgfpathlineto{\pgfqpoint{0.888974in}{0.873027in}}%
\pgfpathlineto{\pgfqpoint{0.895909in}{0.862108in}}%
\pgfpathlineto{\pgfqpoint{0.896536in}{0.860898in}}%
\pgfpathlineto{\pgfqpoint{0.895909in}{0.858915in}}%
\pgfpathlineto{\pgfqpoint{0.893676in}{0.848770in}}%
\pgfpathlineto{\pgfqpoint{0.889853in}{0.836641in}}%
\pgfpathlineto{\pgfqpoint{0.884893in}{0.824512in}}%
\pgfpathlineto{\pgfqpoint{0.884196in}{0.823786in}}%
\pgfpathlineto{\pgfqpoint{0.872483in}{0.815704in}}%
\pgfpathlineto{\pgfqpoint{0.868339in}{0.812383in}}%
\pgfpathlineto{\pgfqpoint{0.860771in}{0.809081in}}%
\pgfpathlineto{\pgfqpoint{0.849058in}{0.805471in}}%
\pgfpathlineto{\pgfqpoint{0.837345in}{0.803130in}}%
\pgfpathlineto{\pgfqpoint{0.825633in}{0.800986in}}%
\pgfpathlineto{\pgfqpoint{0.817208in}{0.800254in}}%
\pgfpathlineto{\pgfqpoint{0.813920in}{0.800081in}}%
\pgfpathclose%
\pgfusepath{fill}%
\end{pgfscope}%
\begin{pgfscope}%
\pgfpathrectangle{\pgfqpoint{0.211875in}{0.211875in}}{\pgfqpoint{1.313625in}{1.279725in}}%
\pgfusepath{clip}%
\pgfsetbuttcap%
\pgfsetroundjoin%
\definecolor{currentfill}{rgb}{0.644838,0.098089,0.355336}%
\pgfsetfillcolor{currentfill}%
\pgfsetlinewidth{0.000000pt}%
\definecolor{currentstroke}{rgb}{0.000000,0.000000,0.000000}%
\pgfsetstrokecolor{currentstroke}%
\pgfsetdash{}{0pt}%
\pgfpathmoveto{\pgfqpoint{1.095024in}{0.290841in}}%
\pgfpathlineto{\pgfqpoint{1.106737in}{0.290841in}}%
\pgfpathlineto{\pgfqpoint{1.111969in}{0.290841in}}%
\pgfpathlineto{\pgfqpoint{1.118450in}{0.293504in}}%
\pgfpathlineto{\pgfqpoint{1.130163in}{0.296913in}}%
\pgfpathlineto{\pgfqpoint{1.141875in}{0.298919in}}%
\pgfpathlineto{\pgfqpoint{1.153588in}{0.300544in}}%
\pgfpathlineto{\pgfqpoint{1.165301in}{0.301641in}}%
\pgfpathlineto{\pgfqpoint{1.176169in}{0.302970in}}%
\pgfpathlineto{\pgfqpoint{1.177013in}{0.303039in}}%
\pgfpathlineto{\pgfqpoint{1.188726in}{0.305088in}}%
\pgfpathlineto{\pgfqpoint{1.200439in}{0.308327in}}%
\pgfpathlineto{\pgfqpoint{1.212151in}{0.311975in}}%
\pgfpathlineto{\pgfqpoint{1.222178in}{0.315099in}}%
\pgfpathlineto{\pgfqpoint{1.223864in}{0.315453in}}%
\pgfpathlineto{\pgfqpoint{1.235577in}{0.317254in}}%
\pgfpathlineto{\pgfqpoint{1.247289in}{0.318365in}}%
\pgfpathlineto{\pgfqpoint{1.259002in}{0.317962in}}%
\pgfpathlineto{\pgfqpoint{1.270125in}{0.315099in}}%
\pgfpathlineto{\pgfqpoint{1.270715in}{0.314520in}}%
\pgfpathlineto{\pgfqpoint{1.277038in}{0.302970in}}%
\pgfpathlineto{\pgfqpoint{1.276969in}{0.290841in}}%
\pgfpathlineto{\pgfqpoint{1.282427in}{0.290841in}}%
\pgfpathlineto{\pgfqpoint{1.294140in}{0.290841in}}%
\pgfpathlineto{\pgfqpoint{1.305853in}{0.290841in}}%
\pgfpathlineto{\pgfqpoint{1.317566in}{0.290841in}}%
\pgfpathlineto{\pgfqpoint{1.329278in}{0.290841in}}%
\pgfpathlineto{\pgfqpoint{1.340991in}{0.290841in}}%
\pgfpathlineto{\pgfqpoint{1.352704in}{0.290841in}}%
\pgfpathlineto{\pgfqpoint{1.364416in}{0.290841in}}%
\pgfpathlineto{\pgfqpoint{1.376129in}{0.290841in}}%
\pgfpathlineto{\pgfqpoint{1.387842in}{0.290841in}}%
\pgfpathlineto{\pgfqpoint{1.399554in}{0.290841in}}%
\pgfpathlineto{\pgfqpoint{1.411267in}{0.290841in}}%
\pgfpathlineto{\pgfqpoint{1.422980in}{0.290841in}}%
\pgfpathlineto{\pgfqpoint{1.434692in}{0.290841in}}%
\pgfpathlineto{\pgfqpoint{1.446405in}{0.290841in}}%
\pgfpathlineto{\pgfqpoint{1.446405in}{0.302970in}}%
\pgfpathlineto{\pgfqpoint{1.446405in}{0.315099in}}%
\pgfpathlineto{\pgfqpoint{1.446405in}{0.327228in}}%
\pgfpathlineto{\pgfqpoint{1.446405in}{0.339357in}}%
\pgfpathlineto{\pgfqpoint{1.446405in}{0.351486in}}%
\pgfpathlineto{\pgfqpoint{1.446405in}{0.359358in}}%
\pgfpathlineto{\pgfqpoint{1.434692in}{0.361059in}}%
\pgfpathlineto{\pgfqpoint{1.422980in}{0.362240in}}%
\pgfpathlineto{\pgfqpoint{1.411267in}{0.361791in}}%
\pgfpathlineto{\pgfqpoint{1.399554in}{0.359658in}}%
\pgfpathlineto{\pgfqpoint{1.387842in}{0.356413in}}%
\pgfpathlineto{\pgfqpoint{1.376129in}{0.358238in}}%
\pgfpathlineto{\pgfqpoint{1.364416in}{0.360909in}}%
\pgfpathlineto{\pgfqpoint{1.353576in}{0.363615in}}%
\pgfpathlineto{\pgfqpoint{1.352704in}{0.364020in}}%
\pgfpathlineto{\pgfqpoint{1.340991in}{0.366500in}}%
\pgfpathlineto{\pgfqpoint{1.329278in}{0.365987in}}%
\pgfpathlineto{\pgfqpoint{1.317566in}{0.363700in}}%
\pgfpathlineto{\pgfqpoint{1.317265in}{0.363615in}}%
\pgfpathlineto{\pgfqpoint{1.305853in}{0.360775in}}%
\pgfpathlineto{\pgfqpoint{1.294140in}{0.357795in}}%
\pgfpathlineto{\pgfqpoint{1.282427in}{0.354196in}}%
\pgfpathlineto{\pgfqpoint{1.275466in}{0.351486in}}%
\pgfpathlineto{\pgfqpoint{1.270715in}{0.349188in}}%
\pgfpathlineto{\pgfqpoint{1.259002in}{0.344065in}}%
\pgfpathlineto{\pgfqpoint{1.247289in}{0.340211in}}%
\pgfpathlineto{\pgfqpoint{1.245097in}{0.339357in}}%
\pgfpathlineto{\pgfqpoint{1.235577in}{0.335023in}}%
\pgfpathlineto{\pgfqpoint{1.223864in}{0.330823in}}%
\pgfpathlineto{\pgfqpoint{1.213323in}{0.327228in}}%
\pgfpathlineto{\pgfqpoint{1.212151in}{0.326727in}}%
\pgfpathlineto{\pgfqpoint{1.200439in}{0.321918in}}%
\pgfpathlineto{\pgfqpoint{1.188726in}{0.317762in}}%
\pgfpathlineto{\pgfqpoint{1.181244in}{0.315099in}}%
\pgfpathlineto{\pgfqpoint{1.177013in}{0.313399in}}%
\pgfpathlineto{\pgfqpoint{1.165301in}{0.311312in}}%
\pgfpathlineto{\pgfqpoint{1.153588in}{0.309698in}}%
\pgfpathlineto{\pgfqpoint{1.141875in}{0.307887in}}%
\pgfpathlineto{\pgfqpoint{1.130163in}{0.305662in}}%
\pgfpathlineto{\pgfqpoint{1.122039in}{0.302970in}}%
\pgfpathlineto{\pgfqpoint{1.118450in}{0.301743in}}%
\pgfpathlineto{\pgfqpoint{1.106737in}{0.296602in}}%
\pgfpathlineto{\pgfqpoint{1.095024in}{0.291142in}}%
\pgfpathlineto{\pgfqpoint{1.094417in}{0.290841in}}%
\pgfpathclose%
\pgfusepath{fill}%
\end{pgfscope}%
\begin{pgfscope}%
\pgfpathrectangle{\pgfqpoint{0.211875in}{0.211875in}}{\pgfqpoint{1.313625in}{1.279725in}}%
\pgfusepath{clip}%
\pgfsetbuttcap%
\pgfsetroundjoin%
\definecolor{currentfill}{rgb}{0.644838,0.098089,0.355336}%
\pgfsetfillcolor{currentfill}%
\pgfsetlinewidth{0.000000pt}%
\definecolor{currentstroke}{rgb}{0.000000,0.000000,0.000000}%
\pgfsetstrokecolor{currentstroke}%
\pgfsetdash{}{0pt}%
\pgfpathmoveto{\pgfqpoint{0.287698in}{0.400001in}}%
\pgfpathlineto{\pgfqpoint{0.298562in}{0.403374in}}%
\pgfpathlineto{\pgfqpoint{0.310274in}{0.406460in}}%
\pgfpathlineto{\pgfqpoint{0.321987in}{0.409980in}}%
\pgfpathlineto{\pgfqpoint{0.329419in}{0.412130in}}%
\pgfpathlineto{\pgfqpoint{0.333700in}{0.413572in}}%
\pgfpathlineto{\pgfqpoint{0.345412in}{0.417679in}}%
\pgfpathlineto{\pgfqpoint{0.357125in}{0.421673in}}%
\pgfpathlineto{\pgfqpoint{0.364563in}{0.424259in}}%
\pgfpathlineto{\pgfqpoint{0.368838in}{0.425679in}}%
\pgfpathlineto{\pgfqpoint{0.380550in}{0.429145in}}%
\pgfpathlineto{\pgfqpoint{0.392263in}{0.432253in}}%
\pgfpathlineto{\pgfqpoint{0.403976in}{0.435340in}}%
\pgfpathlineto{\pgfqpoint{0.407824in}{0.436388in}}%
\pgfpathlineto{\pgfqpoint{0.415688in}{0.438621in}}%
\pgfpathlineto{\pgfqpoint{0.427401in}{0.441873in}}%
\pgfpathlineto{\pgfqpoint{0.439114in}{0.445122in}}%
\pgfpathlineto{\pgfqpoint{0.450827in}{0.448320in}}%
\pgfpathlineto{\pgfqpoint{0.451730in}{0.448517in}}%
\pgfpathlineto{\pgfqpoint{0.462539in}{0.451450in}}%
\pgfpathlineto{\pgfqpoint{0.474252in}{0.453795in}}%
\pgfpathlineto{\pgfqpoint{0.485965in}{0.455544in}}%
\pgfpathlineto{\pgfqpoint{0.497677in}{0.457822in}}%
\pgfpathlineto{\pgfqpoint{0.507175in}{0.460646in}}%
\pgfpathlineto{\pgfqpoint{0.509390in}{0.461336in}}%
\pgfpathlineto{\pgfqpoint{0.521103in}{0.463873in}}%
\pgfpathlineto{\pgfqpoint{0.532815in}{0.466990in}}%
\pgfpathlineto{\pgfqpoint{0.544528in}{0.470210in}}%
\pgfpathlineto{\pgfqpoint{0.554297in}{0.472774in}}%
\pgfpathlineto{\pgfqpoint{0.556241in}{0.473353in}}%
\pgfpathlineto{\pgfqpoint{0.567953in}{0.477685in}}%
\pgfpathlineto{\pgfqpoint{0.574923in}{0.484903in}}%
\pgfpathlineto{\pgfqpoint{0.579666in}{0.490673in}}%
\pgfpathlineto{\pgfqpoint{0.584387in}{0.497032in}}%
\pgfpathlineto{\pgfqpoint{0.591379in}{0.508555in}}%
\pgfpathlineto{\pgfqpoint{0.591743in}{0.509161in}}%
\pgfpathlineto{\pgfqpoint{0.597877in}{0.521290in}}%
\pgfpathlineto{\pgfqpoint{0.603091in}{0.531874in}}%
\pgfpathlineto{\pgfqpoint{0.603771in}{0.533419in}}%
\pgfpathlineto{\pgfqpoint{0.609177in}{0.545548in}}%
\pgfpathlineto{\pgfqpoint{0.614271in}{0.557677in}}%
\pgfpathlineto{\pgfqpoint{0.614804in}{0.559045in}}%
\pgfpathlineto{\pgfqpoint{0.618302in}{0.569805in}}%
\pgfpathlineto{\pgfqpoint{0.622819in}{0.581934in}}%
\pgfpathlineto{\pgfqpoint{0.626517in}{0.593481in}}%
\pgfpathlineto{\pgfqpoint{0.626674in}{0.594063in}}%
\pgfpathlineto{\pgfqpoint{0.630977in}{0.606192in}}%
\pgfpathlineto{\pgfqpoint{0.634938in}{0.618321in}}%
\pgfpathlineto{\pgfqpoint{0.638230in}{0.629789in}}%
\pgfpathlineto{\pgfqpoint{0.638389in}{0.630450in}}%
\pgfpathlineto{\pgfqpoint{0.644171in}{0.642579in}}%
\pgfpathlineto{\pgfqpoint{0.649942in}{0.651623in}}%
\pgfpathlineto{\pgfqpoint{0.651493in}{0.654708in}}%
\pgfpathlineto{\pgfqpoint{0.658653in}{0.666836in}}%
\pgfpathlineto{\pgfqpoint{0.661655in}{0.674372in}}%
\pgfpathlineto{\pgfqpoint{0.663232in}{0.678965in}}%
\pgfpathlineto{\pgfqpoint{0.666992in}{0.691094in}}%
\pgfpathlineto{\pgfqpoint{0.669946in}{0.703223in}}%
\pgfpathlineto{\pgfqpoint{0.673368in}{0.709947in}}%
\pgfpathlineto{\pgfqpoint{0.680509in}{0.715352in}}%
\pgfpathlineto{\pgfqpoint{0.673403in}{0.727481in}}%
\pgfpathlineto{\pgfqpoint{0.673368in}{0.727497in}}%
\pgfpathlineto{\pgfqpoint{0.661655in}{0.732296in}}%
\pgfpathlineto{\pgfqpoint{0.649942in}{0.736934in}}%
\pgfpathlineto{\pgfqpoint{0.642850in}{0.739610in}}%
\pgfpathlineto{\pgfqpoint{0.638230in}{0.740964in}}%
\pgfpathlineto{\pgfqpoint{0.626517in}{0.745463in}}%
\pgfpathlineto{\pgfqpoint{0.614804in}{0.749839in}}%
\pgfpathlineto{\pgfqpoint{0.608501in}{0.751739in}}%
\pgfpathlineto{\pgfqpoint{0.603091in}{0.753144in}}%
\pgfpathlineto{\pgfqpoint{0.591379in}{0.755818in}}%
\pgfpathlineto{\pgfqpoint{0.579666in}{0.759153in}}%
\pgfpathlineto{\pgfqpoint{0.567953in}{0.761982in}}%
\pgfpathlineto{\pgfqpoint{0.561369in}{0.763867in}}%
\pgfpathlineto{\pgfqpoint{0.556241in}{0.765220in}}%
\pgfpathlineto{\pgfqpoint{0.544528in}{0.768239in}}%
\pgfpathlineto{\pgfqpoint{0.532815in}{0.771296in}}%
\pgfpathlineto{\pgfqpoint{0.521103in}{0.774564in}}%
\pgfpathlineto{\pgfqpoint{0.515469in}{0.775996in}}%
\pgfpathlineto{\pgfqpoint{0.509390in}{0.777786in}}%
\pgfpathlineto{\pgfqpoint{0.497677in}{0.780821in}}%
\pgfpathlineto{\pgfqpoint{0.485965in}{0.783644in}}%
\pgfpathlineto{\pgfqpoint{0.474252in}{0.786075in}}%
\pgfpathlineto{\pgfqpoint{0.462539in}{0.787975in}}%
\pgfpathlineto{\pgfqpoint{0.457060in}{0.788125in}}%
\pgfpathlineto{\pgfqpoint{0.450827in}{0.788337in}}%
\pgfpathlineto{\pgfqpoint{0.449403in}{0.788125in}}%
\pgfpathlineto{\pgfqpoint{0.439114in}{0.786579in}}%
\pgfpathlineto{\pgfqpoint{0.427401in}{0.782486in}}%
\pgfpathlineto{\pgfqpoint{0.415688in}{0.776274in}}%
\pgfpathlineto{\pgfqpoint{0.415260in}{0.775996in}}%
\pgfpathlineto{\pgfqpoint{0.403976in}{0.771187in}}%
\pgfpathlineto{\pgfqpoint{0.392263in}{0.765923in}}%
\pgfpathlineto{\pgfqpoint{0.388883in}{0.763867in}}%
\pgfpathlineto{\pgfqpoint{0.380550in}{0.757892in}}%
\pgfpathlineto{\pgfqpoint{0.373049in}{0.751739in}}%
\pgfpathlineto{\pgfqpoint{0.368838in}{0.747307in}}%
\pgfpathlineto{\pgfqpoint{0.361805in}{0.739610in}}%
\pgfpathlineto{\pgfqpoint{0.358135in}{0.727481in}}%
\pgfpathlineto{\pgfqpoint{0.357125in}{0.718342in}}%
\pgfpathlineto{\pgfqpoint{0.356752in}{0.715352in}}%
\pgfpathlineto{\pgfqpoint{0.354770in}{0.703223in}}%
\pgfpathlineto{\pgfqpoint{0.351783in}{0.691094in}}%
\pgfpathlineto{\pgfqpoint{0.348613in}{0.678965in}}%
\pgfpathlineto{\pgfqpoint{0.345412in}{0.667440in}}%
\pgfpathlineto{\pgfqpoint{0.345245in}{0.666836in}}%
\pgfpathlineto{\pgfqpoint{0.342558in}{0.654708in}}%
\pgfpathlineto{\pgfqpoint{0.339033in}{0.642579in}}%
\pgfpathlineto{\pgfqpoint{0.334564in}{0.630450in}}%
\pgfpathlineto{\pgfqpoint{0.333700in}{0.628601in}}%
\pgfpathlineto{\pgfqpoint{0.328755in}{0.618321in}}%
\pgfpathlineto{\pgfqpoint{0.322029in}{0.606192in}}%
\pgfpathlineto{\pgfqpoint{0.321987in}{0.606126in}}%
\pgfpathlineto{\pgfqpoint{0.313927in}{0.594063in}}%
\pgfpathlineto{\pgfqpoint{0.310274in}{0.589459in}}%
\pgfpathlineto{\pgfqpoint{0.304092in}{0.581934in}}%
\pgfpathlineto{\pgfqpoint{0.298562in}{0.575830in}}%
\pgfpathlineto{\pgfqpoint{0.292809in}{0.569805in}}%
\pgfpathlineto{\pgfqpoint{0.286849in}{0.563686in}}%
\pgfpathlineto{\pgfqpoint{0.286849in}{0.557677in}}%
\pgfpathlineto{\pgfqpoint{0.286849in}{0.545548in}}%
\pgfpathlineto{\pgfqpoint{0.286849in}{0.533419in}}%
\pgfpathlineto{\pgfqpoint{0.286849in}{0.521290in}}%
\pgfpathlineto{\pgfqpoint{0.286849in}{0.509161in}}%
\pgfpathlineto{\pgfqpoint{0.286849in}{0.497032in}}%
\pgfpathlineto{\pgfqpoint{0.286849in}{0.484903in}}%
\pgfpathlineto{\pgfqpoint{0.286849in}{0.472774in}}%
\pgfpathlineto{\pgfqpoint{0.286849in}{0.460646in}}%
\pgfpathlineto{\pgfqpoint{0.286849in}{0.448517in}}%
\pgfpathlineto{\pgfqpoint{0.286849in}{0.436388in}}%
\pgfpathlineto{\pgfqpoint{0.286849in}{0.424259in}}%
\pgfpathlineto{\pgfqpoint{0.286849in}{0.412130in}}%
\pgfpathlineto{\pgfqpoint{0.286849in}{0.400001in}}%
\pgfpathlineto{\pgfqpoint{0.286849in}{0.399743in}}%
\pgfpathclose%
\pgfpathmoveto{\pgfqpoint{0.383955in}{0.545548in}}%
\pgfpathlineto{\pgfqpoint{0.385859in}{0.557677in}}%
\pgfpathlineto{\pgfqpoint{0.391569in}{0.569805in}}%
\pgfpathlineto{\pgfqpoint{0.392263in}{0.571496in}}%
\pgfpathlineto{\pgfqpoint{0.396326in}{0.581934in}}%
\pgfpathlineto{\pgfqpoint{0.400244in}{0.594063in}}%
\pgfpathlineto{\pgfqpoint{0.403033in}{0.606192in}}%
\pgfpathlineto{\pgfqpoint{0.403523in}{0.618321in}}%
\pgfpathlineto{\pgfqpoint{0.403402in}{0.630450in}}%
\pgfpathlineto{\pgfqpoint{0.402241in}{0.642579in}}%
\pgfpathlineto{\pgfqpoint{0.403289in}{0.654708in}}%
\pgfpathlineto{\pgfqpoint{0.403976in}{0.663787in}}%
\pgfpathlineto{\pgfqpoint{0.404201in}{0.666836in}}%
\pgfpathlineto{\pgfqpoint{0.407190in}{0.678965in}}%
\pgfpathlineto{\pgfqpoint{0.410594in}{0.691094in}}%
\pgfpathlineto{\pgfqpoint{0.413505in}{0.703223in}}%
\pgfpathlineto{\pgfqpoint{0.413465in}{0.715352in}}%
\pgfpathlineto{\pgfqpoint{0.412195in}{0.727481in}}%
\pgfpathlineto{\pgfqpoint{0.415466in}{0.739610in}}%
\pgfpathlineto{\pgfqpoint{0.415688in}{0.739782in}}%
\pgfpathlineto{\pgfqpoint{0.427401in}{0.747385in}}%
\pgfpathlineto{\pgfqpoint{0.435320in}{0.751739in}}%
\pgfpathlineto{\pgfqpoint{0.439114in}{0.753719in}}%
\pgfpathlineto{\pgfqpoint{0.450827in}{0.758704in}}%
\pgfpathlineto{\pgfqpoint{0.462539in}{0.762019in}}%
\pgfpathlineto{\pgfqpoint{0.472907in}{0.763867in}}%
\pgfpathlineto{\pgfqpoint{0.474252in}{0.764053in}}%
\pgfpathlineto{\pgfqpoint{0.485965in}{0.765035in}}%
\pgfpathlineto{\pgfqpoint{0.497677in}{0.765351in}}%
\pgfpathlineto{\pgfqpoint{0.509390in}{0.764816in}}%
\pgfpathlineto{\pgfqpoint{0.518386in}{0.763867in}}%
\pgfpathlineto{\pgfqpoint{0.521103in}{0.763547in}}%
\pgfpathlineto{\pgfqpoint{0.532815in}{0.761299in}}%
\pgfpathlineto{\pgfqpoint{0.544528in}{0.758621in}}%
\pgfpathlineto{\pgfqpoint{0.556241in}{0.755578in}}%
\pgfpathlineto{\pgfqpoint{0.567953in}{0.752073in}}%
\pgfpathlineto{\pgfqpoint{0.568887in}{0.751739in}}%
\pgfpathlineto{\pgfqpoint{0.579666in}{0.747706in}}%
\pgfpathlineto{\pgfqpoint{0.591379in}{0.743851in}}%
\pgfpathlineto{\pgfqpoint{0.601153in}{0.739610in}}%
\pgfpathlineto{\pgfqpoint{0.603091in}{0.738798in}}%
\pgfpathlineto{\pgfqpoint{0.614804in}{0.734962in}}%
\pgfpathlineto{\pgfqpoint{0.626517in}{0.728373in}}%
\pgfpathlineto{\pgfqpoint{0.628235in}{0.727481in}}%
\pgfpathlineto{\pgfqpoint{0.638230in}{0.721568in}}%
\pgfpathlineto{\pgfqpoint{0.644936in}{0.715352in}}%
\pgfpathlineto{\pgfqpoint{0.649942in}{0.708518in}}%
\pgfpathlineto{\pgfqpoint{0.653211in}{0.703223in}}%
\pgfpathlineto{\pgfqpoint{0.652564in}{0.691094in}}%
\pgfpathlineto{\pgfqpoint{0.649942in}{0.682192in}}%
\pgfpathlineto{\pgfqpoint{0.648798in}{0.678965in}}%
\pgfpathlineto{\pgfqpoint{0.641546in}{0.666836in}}%
\pgfpathlineto{\pgfqpoint{0.638230in}{0.662258in}}%
\pgfpathlineto{\pgfqpoint{0.631602in}{0.654708in}}%
\pgfpathlineto{\pgfqpoint{0.626517in}{0.648786in}}%
\pgfpathlineto{\pgfqpoint{0.620230in}{0.642579in}}%
\pgfpathlineto{\pgfqpoint{0.614804in}{0.636487in}}%
\pgfpathlineto{\pgfqpoint{0.608246in}{0.630450in}}%
\pgfpathlineto{\pgfqpoint{0.603091in}{0.624116in}}%
\pgfpathlineto{\pgfqpoint{0.597054in}{0.618321in}}%
\pgfpathlineto{\pgfqpoint{0.591379in}{0.611749in}}%
\pgfpathlineto{\pgfqpoint{0.585235in}{0.606192in}}%
\pgfpathlineto{\pgfqpoint{0.579666in}{0.600596in}}%
\pgfpathlineto{\pgfqpoint{0.568902in}{0.594063in}}%
\pgfpathlineto{\pgfqpoint{0.567953in}{0.593392in}}%
\pgfpathlineto{\pgfqpoint{0.556241in}{0.587040in}}%
\pgfpathlineto{\pgfqpoint{0.546123in}{0.581934in}}%
\pgfpathlineto{\pgfqpoint{0.544528in}{0.581098in}}%
\pgfpathlineto{\pgfqpoint{0.532815in}{0.575721in}}%
\pgfpathlineto{\pgfqpoint{0.521103in}{0.570384in}}%
\pgfpathlineto{\pgfqpoint{0.519676in}{0.569805in}}%
\pgfpathlineto{\pgfqpoint{0.509390in}{0.565660in}}%
\pgfpathlineto{\pgfqpoint{0.497677in}{0.560976in}}%
\pgfpathlineto{\pgfqpoint{0.488508in}{0.557677in}}%
\pgfpathlineto{\pgfqpoint{0.485965in}{0.556744in}}%
\pgfpathlineto{\pgfqpoint{0.474252in}{0.552672in}}%
\pgfpathlineto{\pgfqpoint{0.462539in}{0.548031in}}%
\pgfpathlineto{\pgfqpoint{0.453079in}{0.545548in}}%
\pgfpathlineto{\pgfqpoint{0.450827in}{0.544762in}}%
\pgfpathlineto{\pgfqpoint{0.439114in}{0.539326in}}%
\pgfpathlineto{\pgfqpoint{0.427401in}{0.537187in}}%
\pgfpathlineto{\pgfqpoint{0.415688in}{0.534803in}}%
\pgfpathlineto{\pgfqpoint{0.403976in}{0.536517in}}%
\pgfpathlineto{\pgfqpoint{0.392263in}{0.539431in}}%
\pgfpathclose%
\pgfusepath{fill}%
\end{pgfscope}%
\begin{pgfscope}%
\pgfpathrectangle{\pgfqpoint{0.211875in}{0.211875in}}{\pgfqpoint{1.313625in}{1.279725in}}%
\pgfusepath{clip}%
\pgfsetbuttcap%
\pgfsetroundjoin%
\definecolor{currentfill}{rgb}{0.644838,0.098089,0.355336}%
\pgfsetfillcolor{currentfill}%
\pgfsetlinewidth{0.000000pt}%
\definecolor{currentstroke}{rgb}{0.000000,0.000000,0.000000}%
\pgfsetstrokecolor{currentstroke}%
\pgfsetdash{}{0pt}%
\pgfpathmoveto{\pgfqpoint{1.259002in}{0.733065in}}%
\pgfpathlineto{\pgfqpoint{1.270715in}{0.727948in}}%
\pgfpathlineto{\pgfqpoint{1.282427in}{0.732141in}}%
\pgfpathlineto{\pgfqpoint{1.293299in}{0.739610in}}%
\pgfpathlineto{\pgfqpoint{1.294140in}{0.740201in}}%
\pgfpathlineto{\pgfqpoint{1.305853in}{0.749674in}}%
\pgfpathlineto{\pgfqpoint{1.309926in}{0.751739in}}%
\pgfpathlineto{\pgfqpoint{1.317566in}{0.755488in}}%
\pgfpathlineto{\pgfqpoint{1.329278in}{0.758530in}}%
\pgfpathlineto{\pgfqpoint{1.333142in}{0.763867in}}%
\pgfpathlineto{\pgfqpoint{1.338916in}{0.775996in}}%
\pgfpathlineto{\pgfqpoint{1.340991in}{0.780746in}}%
\pgfpathlineto{\pgfqpoint{1.344248in}{0.788125in}}%
\pgfpathlineto{\pgfqpoint{1.350649in}{0.800254in}}%
\pgfpathlineto{\pgfqpoint{1.352704in}{0.804343in}}%
\pgfpathlineto{\pgfqpoint{1.356366in}{0.812383in}}%
\pgfpathlineto{\pgfqpoint{1.363825in}{0.824512in}}%
\pgfpathlineto{\pgfqpoint{1.364416in}{0.825387in}}%
\pgfpathlineto{\pgfqpoint{1.371279in}{0.836641in}}%
\pgfpathlineto{\pgfqpoint{1.376129in}{0.843834in}}%
\pgfpathlineto{\pgfqpoint{1.379670in}{0.848770in}}%
\pgfpathlineto{\pgfqpoint{1.387842in}{0.859099in}}%
\pgfpathlineto{\pgfqpoint{1.389198in}{0.860898in}}%
\pgfpathlineto{\pgfqpoint{1.397801in}{0.873027in}}%
\pgfpathlineto{\pgfqpoint{1.399554in}{0.875371in}}%
\pgfpathlineto{\pgfqpoint{1.411267in}{0.882736in}}%
\pgfpathlineto{\pgfqpoint{1.415999in}{0.885156in}}%
\pgfpathlineto{\pgfqpoint{1.422980in}{0.888777in}}%
\pgfpathlineto{\pgfqpoint{1.434692in}{0.892476in}}%
\pgfpathlineto{\pgfqpoint{1.446405in}{0.896872in}}%
\pgfpathlineto{\pgfqpoint{1.446405in}{0.897285in}}%
\pgfpathlineto{\pgfqpoint{1.446405in}{0.909414in}}%
\pgfpathlineto{\pgfqpoint{1.446405in}{0.913935in}}%
\pgfpathlineto{\pgfqpoint{1.438617in}{0.909414in}}%
\pgfpathlineto{\pgfqpoint{1.434692in}{0.907710in}}%
\pgfpathlineto{\pgfqpoint{1.422980in}{0.904051in}}%
\pgfpathlineto{\pgfqpoint{1.411267in}{0.899846in}}%
\pgfpathlineto{\pgfqpoint{1.404911in}{0.897285in}}%
\pgfpathlineto{\pgfqpoint{1.399554in}{0.895312in}}%
\pgfpathlineto{\pgfqpoint{1.387842in}{0.887509in}}%
\pgfpathlineto{\pgfqpoint{1.386662in}{0.885156in}}%
\pgfpathlineto{\pgfqpoint{1.382510in}{0.873027in}}%
\pgfpathlineto{\pgfqpoint{1.376172in}{0.860898in}}%
\pgfpathlineto{\pgfqpoint{1.376129in}{0.860843in}}%
\pgfpathlineto{\pgfqpoint{1.365676in}{0.848770in}}%
\pgfpathlineto{\pgfqpoint{1.364416in}{0.846718in}}%
\pgfpathlineto{\pgfqpoint{1.358361in}{0.836641in}}%
\pgfpathlineto{\pgfqpoint{1.352704in}{0.828023in}}%
\pgfpathlineto{\pgfqpoint{1.350258in}{0.824512in}}%
\pgfpathlineto{\pgfqpoint{1.344247in}{0.812383in}}%
\pgfpathlineto{\pgfqpoint{1.340991in}{0.806056in}}%
\pgfpathlineto{\pgfqpoint{1.337619in}{0.800254in}}%
\pgfpathlineto{\pgfqpoint{1.331335in}{0.788125in}}%
\pgfpathlineto{\pgfqpoint{1.329278in}{0.783280in}}%
\pgfpathlineto{\pgfqpoint{1.319312in}{0.775996in}}%
\pgfpathlineto{\pgfqpoint{1.317566in}{0.775153in}}%
\pgfpathlineto{\pgfqpoint{1.305853in}{0.770224in}}%
\pgfpathlineto{\pgfqpoint{1.297299in}{0.763867in}}%
\pgfpathlineto{\pgfqpoint{1.294140in}{0.761560in}}%
\pgfpathlineto{\pgfqpoint{1.282427in}{0.753143in}}%
\pgfpathlineto{\pgfqpoint{1.270715in}{0.756881in}}%
\pgfpathlineto{\pgfqpoint{1.262654in}{0.763867in}}%
\pgfpathlineto{\pgfqpoint{1.259002in}{0.766968in}}%
\pgfpathlineto{\pgfqpoint{1.247289in}{0.772104in}}%
\pgfpathlineto{\pgfqpoint{1.235577in}{0.773365in}}%
\pgfpathlineto{\pgfqpoint{1.223864in}{0.774522in}}%
\pgfpathlineto{\pgfqpoint{1.220701in}{0.775996in}}%
\pgfpathlineto{\pgfqpoint{1.212151in}{0.787293in}}%
\pgfpathlineto{\pgfqpoint{1.211845in}{0.788125in}}%
\pgfpathlineto{\pgfqpoint{1.212151in}{0.789658in}}%
\pgfpathlineto{\pgfqpoint{1.215200in}{0.800254in}}%
\pgfpathlineto{\pgfqpoint{1.223864in}{0.807564in}}%
\pgfpathlineto{\pgfqpoint{1.231378in}{0.812383in}}%
\pgfpathlineto{\pgfqpoint{1.235577in}{0.814257in}}%
\pgfpathlineto{\pgfqpoint{1.247289in}{0.821452in}}%
\pgfpathlineto{\pgfqpoint{1.251447in}{0.824512in}}%
\pgfpathlineto{\pgfqpoint{1.259002in}{0.829541in}}%
\pgfpathlineto{\pgfqpoint{1.267703in}{0.836641in}}%
\pgfpathlineto{\pgfqpoint{1.270715in}{0.839404in}}%
\pgfpathlineto{\pgfqpoint{1.278038in}{0.848770in}}%
\pgfpathlineto{\pgfqpoint{1.282427in}{0.859236in}}%
\pgfpathlineto{\pgfqpoint{1.282959in}{0.860898in}}%
\pgfpathlineto{\pgfqpoint{1.282427in}{0.863671in}}%
\pgfpathlineto{\pgfqpoint{1.280910in}{0.873027in}}%
\pgfpathlineto{\pgfqpoint{1.278013in}{0.885156in}}%
\pgfpathlineto{\pgfqpoint{1.276274in}{0.897285in}}%
\pgfpathlineto{\pgfqpoint{1.275227in}{0.909414in}}%
\pgfpathlineto{\pgfqpoint{1.276038in}{0.921543in}}%
\pgfpathlineto{\pgfqpoint{1.280374in}{0.933672in}}%
\pgfpathlineto{\pgfqpoint{1.282427in}{0.937602in}}%
\pgfpathlineto{\pgfqpoint{1.294140in}{0.945045in}}%
\pgfpathlineto{\pgfqpoint{1.297980in}{0.945801in}}%
\pgfpathlineto{\pgfqpoint{1.305853in}{0.947251in}}%
\pgfpathlineto{\pgfqpoint{1.317566in}{0.948987in}}%
\pgfpathlineto{\pgfqpoint{1.329278in}{0.953089in}}%
\pgfpathlineto{\pgfqpoint{1.340991in}{0.956927in}}%
\pgfpathlineto{\pgfqpoint{1.341845in}{0.957929in}}%
\pgfpathlineto{\pgfqpoint{1.352704in}{0.967068in}}%
\pgfpathlineto{\pgfqpoint{1.356596in}{0.970058in}}%
\pgfpathlineto{\pgfqpoint{1.364416in}{0.975391in}}%
\pgfpathlineto{\pgfqpoint{1.369638in}{0.982187in}}%
\pgfpathlineto{\pgfqpoint{1.376129in}{0.991837in}}%
\pgfpathlineto{\pgfqpoint{1.378576in}{0.994316in}}%
\pgfpathlineto{\pgfqpoint{1.387842in}{1.004785in}}%
\pgfpathlineto{\pgfqpoint{1.388922in}{1.006445in}}%
\pgfpathlineto{\pgfqpoint{1.396090in}{1.018574in}}%
\pgfpathlineto{\pgfqpoint{1.399554in}{1.024752in}}%
\pgfpathlineto{\pgfqpoint{1.404469in}{1.030703in}}%
\pgfpathlineto{\pgfqpoint{1.411267in}{1.036653in}}%
\pgfpathlineto{\pgfqpoint{1.421031in}{1.042832in}}%
\pgfpathlineto{\pgfqpoint{1.422980in}{1.044059in}}%
\pgfpathlineto{\pgfqpoint{1.434692in}{1.049223in}}%
\pgfpathlineto{\pgfqpoint{1.446405in}{1.054073in}}%
\pgfpathlineto{\pgfqpoint{1.446405in}{1.054960in}}%
\pgfpathlineto{\pgfqpoint{1.446405in}{1.067089in}}%
\pgfpathlineto{\pgfqpoint{1.446405in}{1.079218in}}%
\pgfpathlineto{\pgfqpoint{1.446405in}{1.082126in}}%
\pgfpathlineto{\pgfqpoint{1.434692in}{1.079941in}}%
\pgfpathlineto{\pgfqpoint{1.432842in}{1.079218in}}%
\pgfpathlineto{\pgfqpoint{1.422980in}{1.074477in}}%
\pgfpathlineto{\pgfqpoint{1.411267in}{1.067814in}}%
\pgfpathlineto{\pgfqpoint{1.410138in}{1.067089in}}%
\pgfpathlineto{\pgfqpoint{1.399554in}{1.058145in}}%
\pgfpathlineto{\pgfqpoint{1.396567in}{1.054960in}}%
\pgfpathlineto{\pgfqpoint{1.387842in}{1.045623in}}%
\pgfpathlineto{\pgfqpoint{1.384554in}{1.042832in}}%
\pgfpathlineto{\pgfqpoint{1.376129in}{1.033871in}}%
\pgfpathlineto{\pgfqpoint{1.371472in}{1.030703in}}%
\pgfpathlineto{\pgfqpoint{1.364416in}{1.023966in}}%
\pgfpathlineto{\pgfqpoint{1.354799in}{1.018574in}}%
\pgfpathlineto{\pgfqpoint{1.352704in}{1.017509in}}%
\pgfpathlineto{\pgfqpoint{1.340991in}{1.011816in}}%
\pgfpathlineto{\pgfqpoint{1.335599in}{1.006445in}}%
\pgfpathlineto{\pgfqpoint{1.329278in}{0.999969in}}%
\pgfpathlineto{\pgfqpoint{1.323402in}{0.994316in}}%
\pgfpathlineto{\pgfqpoint{1.317566in}{0.989792in}}%
\pgfpathlineto{\pgfqpoint{1.305853in}{0.992529in}}%
\pgfpathlineto{\pgfqpoint{1.301479in}{0.994316in}}%
\pgfpathlineto{\pgfqpoint{1.294140in}{0.997378in}}%
\pgfpathlineto{\pgfqpoint{1.289601in}{1.006445in}}%
\pgfpathlineto{\pgfqpoint{1.289795in}{1.018574in}}%
\pgfpathlineto{\pgfqpoint{1.293321in}{1.030703in}}%
\pgfpathlineto{\pgfqpoint{1.294140in}{1.032971in}}%
\pgfpathlineto{\pgfqpoint{1.298368in}{1.042832in}}%
\pgfpathlineto{\pgfqpoint{1.303071in}{1.054960in}}%
\pgfpathlineto{\pgfqpoint{1.305853in}{1.062656in}}%
\pgfpathlineto{\pgfqpoint{1.307464in}{1.067089in}}%
\pgfpathlineto{\pgfqpoint{1.312338in}{1.079218in}}%
\pgfpathlineto{\pgfqpoint{1.317278in}{1.091347in}}%
\pgfpathlineto{\pgfqpoint{1.317566in}{1.092299in}}%
\pgfpathlineto{\pgfqpoint{1.320693in}{1.103476in}}%
\pgfpathlineto{\pgfqpoint{1.324150in}{1.115605in}}%
\pgfpathlineto{\pgfqpoint{1.324338in}{1.127734in}}%
\pgfpathlineto{\pgfqpoint{1.320906in}{1.139863in}}%
\pgfpathlineto{\pgfqpoint{1.317566in}{1.147034in}}%
\pgfpathlineto{\pgfqpoint{1.313453in}{1.151991in}}%
\pgfpathlineto{\pgfqpoint{1.305853in}{1.157741in}}%
\pgfpathlineto{\pgfqpoint{1.294140in}{1.160009in}}%
\pgfpathlineto{\pgfqpoint{1.282427in}{1.160878in}}%
\pgfpathlineto{\pgfqpoint{1.271836in}{1.164120in}}%
\pgfpathlineto{\pgfqpoint{1.270715in}{1.164645in}}%
\pgfpathlineto{\pgfqpoint{1.259002in}{1.168810in}}%
\pgfpathlineto{\pgfqpoint{1.247289in}{1.174117in}}%
\pgfpathlineto{\pgfqpoint{1.246286in}{1.176249in}}%
\pgfpathlineto{\pgfqpoint{1.242796in}{1.188378in}}%
\pgfpathlineto{\pgfqpoint{1.238096in}{1.200507in}}%
\pgfpathlineto{\pgfqpoint{1.235577in}{1.204878in}}%
\pgfpathlineto{\pgfqpoint{1.228540in}{1.212636in}}%
\pgfpathlineto{\pgfqpoint{1.223864in}{1.216384in}}%
\pgfpathlineto{\pgfqpoint{1.212151in}{1.222603in}}%
\pgfpathlineto{\pgfqpoint{1.200439in}{1.222597in}}%
\pgfpathlineto{\pgfqpoint{1.192926in}{1.212636in}}%
\pgfpathlineto{\pgfqpoint{1.188726in}{1.204020in}}%
\pgfpathlineto{\pgfqpoint{1.187375in}{1.200507in}}%
\pgfpathlineto{\pgfqpoint{1.183037in}{1.188378in}}%
\pgfpathlineto{\pgfqpoint{1.177631in}{1.176249in}}%
\pgfpathlineto{\pgfqpoint{1.177013in}{1.175396in}}%
\pgfpathlineto{\pgfqpoint{1.169356in}{1.164120in}}%
\pgfpathlineto{\pgfqpoint{1.165301in}{1.158817in}}%
\pgfpathlineto{\pgfqpoint{1.160949in}{1.151991in}}%
\pgfpathlineto{\pgfqpoint{1.153588in}{1.142156in}}%
\pgfpathlineto{\pgfqpoint{1.152195in}{1.139863in}}%
\pgfpathlineto{\pgfqpoint{1.145144in}{1.127734in}}%
\pgfpathlineto{\pgfqpoint{1.141875in}{1.118031in}}%
\pgfpathlineto{\pgfqpoint{1.140888in}{1.115605in}}%
\pgfpathlineto{\pgfqpoint{1.133262in}{1.103476in}}%
\pgfpathlineto{\pgfqpoint{1.130163in}{1.098032in}}%
\pgfpathlineto{\pgfqpoint{1.126457in}{1.091347in}}%
\pgfpathlineto{\pgfqpoint{1.120719in}{1.079218in}}%
\pgfpathlineto{\pgfqpoint{1.118450in}{1.069176in}}%
\pgfpathlineto{\pgfqpoint{1.117636in}{1.067089in}}%
\pgfpathlineto{\pgfqpoint{1.111764in}{1.054960in}}%
\pgfpathlineto{\pgfqpoint{1.106896in}{1.042832in}}%
\pgfpathlineto{\pgfqpoint{1.106737in}{1.042545in}}%
\pgfpathlineto{\pgfqpoint{1.097810in}{1.030703in}}%
\pgfpathlineto{\pgfqpoint{1.095024in}{1.027932in}}%
\pgfpathlineto{\pgfqpoint{1.087022in}{1.018574in}}%
\pgfpathlineto{\pgfqpoint{1.083312in}{1.015166in}}%
\pgfpathlineto{\pgfqpoint{1.077325in}{1.006445in}}%
\pgfpathlineto{\pgfqpoint{1.071599in}{0.996858in}}%
\pgfpathlineto{\pgfqpoint{1.070144in}{0.994316in}}%
\pgfpathlineto{\pgfqpoint{1.065070in}{0.982187in}}%
\pgfpathlineto{\pgfqpoint{1.062520in}{0.970058in}}%
\pgfpathlineto{\pgfqpoint{1.064060in}{0.957929in}}%
\pgfpathlineto{\pgfqpoint{1.071599in}{0.951214in}}%
\pgfpathlineto{\pgfqpoint{1.083312in}{0.949151in}}%
\pgfpathlineto{\pgfqpoint{1.092493in}{0.945801in}}%
\pgfpathlineto{\pgfqpoint{1.095024in}{0.936867in}}%
\pgfpathlineto{\pgfqpoint{1.106737in}{0.941858in}}%
\pgfpathlineto{\pgfqpoint{1.110818in}{0.945801in}}%
\pgfpathlineto{\pgfqpoint{1.118450in}{0.948822in}}%
\pgfpathlineto{\pgfqpoint{1.130163in}{0.954790in}}%
\pgfpathlineto{\pgfqpoint{1.136313in}{0.957929in}}%
\pgfpathlineto{\pgfqpoint{1.141875in}{0.960674in}}%
\pgfpathlineto{\pgfqpoint{1.153588in}{0.959341in}}%
\pgfpathlineto{\pgfqpoint{1.158547in}{0.957929in}}%
\pgfpathlineto{\pgfqpoint{1.165301in}{0.955986in}}%
\pgfpathlineto{\pgfqpoint{1.177013in}{0.950591in}}%
\pgfpathlineto{\pgfqpoint{1.188726in}{0.951123in}}%
\pgfpathlineto{\pgfqpoint{1.200439in}{0.953716in}}%
\pgfpathlineto{\pgfqpoint{1.212151in}{0.955315in}}%
\pgfpathlineto{\pgfqpoint{1.223864in}{0.956224in}}%
\pgfpathlineto{\pgfqpoint{1.235577in}{0.952803in}}%
\pgfpathlineto{\pgfqpoint{1.243245in}{0.945801in}}%
\pgfpathlineto{\pgfqpoint{1.246553in}{0.933672in}}%
\pgfpathlineto{\pgfqpoint{1.247289in}{0.930987in}}%
\pgfpathlineto{\pgfqpoint{1.249510in}{0.921543in}}%
\pgfpathlineto{\pgfqpoint{1.254410in}{0.909414in}}%
\pgfpathlineto{\pgfqpoint{1.257732in}{0.897285in}}%
\pgfpathlineto{\pgfqpoint{1.259002in}{0.890492in}}%
\pgfpathlineto{\pgfqpoint{1.259978in}{0.885156in}}%
\pgfpathlineto{\pgfqpoint{1.262367in}{0.873027in}}%
\pgfpathlineto{\pgfqpoint{1.259002in}{0.861082in}}%
\pgfpathlineto{\pgfqpoint{1.258902in}{0.860898in}}%
\pgfpathlineto{\pgfqpoint{1.247289in}{0.852299in}}%
\pgfpathlineto{\pgfqpoint{1.241723in}{0.848770in}}%
\pgfpathlineto{\pgfqpoint{1.235577in}{0.844115in}}%
\pgfpathlineto{\pgfqpoint{1.223864in}{0.838613in}}%
\pgfpathlineto{\pgfqpoint{1.220297in}{0.836641in}}%
\pgfpathlineto{\pgfqpoint{1.212151in}{0.831805in}}%
\pgfpathlineto{\pgfqpoint{1.202042in}{0.824512in}}%
\pgfpathlineto{\pgfqpoint{1.200439in}{0.823307in}}%
\pgfpathlineto{\pgfqpoint{1.188726in}{0.812553in}}%
\pgfpathlineto{\pgfqpoint{1.188591in}{0.812383in}}%
\pgfpathlineto{\pgfqpoint{1.183274in}{0.800254in}}%
\pgfpathlineto{\pgfqpoint{1.183992in}{0.788125in}}%
\pgfpathlineto{\pgfqpoint{1.186598in}{0.775996in}}%
\pgfpathlineto{\pgfqpoint{1.188726in}{0.771975in}}%
\pgfpathlineto{\pgfqpoint{1.194937in}{0.763867in}}%
\pgfpathlineto{\pgfqpoint{1.200439in}{0.760134in}}%
\pgfpathlineto{\pgfqpoint{1.212151in}{0.753136in}}%
\pgfpathlineto{\pgfqpoint{1.216436in}{0.751739in}}%
\pgfpathlineto{\pgfqpoint{1.223864in}{0.750099in}}%
\pgfpathlineto{\pgfqpoint{1.235577in}{0.747102in}}%
\pgfpathlineto{\pgfqpoint{1.247289in}{0.740858in}}%
\pgfpathlineto{\pgfqpoint{1.249214in}{0.739610in}}%
\pgfpathclose%
\pgfpathmoveto{\pgfqpoint{1.180001in}{0.982187in}}%
\pgfpathlineto{\pgfqpoint{1.177013in}{0.986443in}}%
\pgfpathlineto{\pgfqpoint{1.165301in}{0.990900in}}%
\pgfpathlineto{\pgfqpoint{1.153588in}{0.993515in}}%
\pgfpathlineto{\pgfqpoint{1.141875in}{0.989764in}}%
\pgfpathlineto{\pgfqpoint{1.135355in}{0.994316in}}%
\pgfpathlineto{\pgfqpoint{1.130334in}{1.006445in}}%
\pgfpathlineto{\pgfqpoint{1.130163in}{1.006964in}}%
\pgfpathlineto{\pgfqpoint{1.127388in}{1.018574in}}%
\pgfpathlineto{\pgfqpoint{1.129677in}{1.030703in}}%
\pgfpathlineto{\pgfqpoint{1.130163in}{1.032188in}}%
\pgfpathlineto{\pgfqpoint{1.134080in}{1.042832in}}%
\pgfpathlineto{\pgfqpoint{1.141390in}{1.054960in}}%
\pgfpathlineto{\pgfqpoint{1.141875in}{1.055868in}}%
\pgfpathlineto{\pgfqpoint{1.148113in}{1.067089in}}%
\pgfpathlineto{\pgfqpoint{1.153588in}{1.079100in}}%
\pgfpathlineto{\pgfqpoint{1.153640in}{1.079218in}}%
\pgfpathlineto{\pgfqpoint{1.156483in}{1.091347in}}%
\pgfpathlineto{\pgfqpoint{1.160424in}{1.103476in}}%
\pgfpathlineto{\pgfqpoint{1.163938in}{1.115605in}}%
\pgfpathlineto{\pgfqpoint{1.165301in}{1.119173in}}%
\pgfpathlineto{\pgfqpoint{1.170781in}{1.127734in}}%
\pgfpathlineto{\pgfqpoint{1.177013in}{1.135097in}}%
\pgfpathlineto{\pgfqpoint{1.182319in}{1.139863in}}%
\pgfpathlineto{\pgfqpoint{1.188726in}{1.147427in}}%
\pgfpathlineto{\pgfqpoint{1.200439in}{1.151082in}}%
\pgfpathlineto{\pgfqpoint{1.202647in}{1.151991in}}%
\pgfpathlineto{\pgfqpoint{1.212151in}{1.155364in}}%
\pgfpathlineto{\pgfqpoint{1.223864in}{1.152143in}}%
\pgfpathlineto{\pgfqpoint{1.224013in}{1.151991in}}%
\pgfpathlineto{\pgfqpoint{1.235577in}{1.141397in}}%
\pgfpathlineto{\pgfqpoint{1.247289in}{1.140072in}}%
\pgfpathlineto{\pgfqpoint{1.250097in}{1.139863in}}%
\pgfpathlineto{\pgfqpoint{1.259002in}{1.139174in}}%
\pgfpathlineto{\pgfqpoint{1.270715in}{1.136606in}}%
\pgfpathlineto{\pgfqpoint{1.282427in}{1.132216in}}%
\pgfpathlineto{\pgfqpoint{1.291345in}{1.127734in}}%
\pgfpathlineto{\pgfqpoint{1.294140in}{1.124986in}}%
\pgfpathlineto{\pgfqpoint{1.299506in}{1.115605in}}%
\pgfpathlineto{\pgfqpoint{1.301319in}{1.103476in}}%
\pgfpathlineto{\pgfqpoint{1.297933in}{1.091347in}}%
\pgfpathlineto{\pgfqpoint{1.294140in}{1.080498in}}%
\pgfpathlineto{\pgfqpoint{1.293782in}{1.079218in}}%
\pgfpathlineto{\pgfqpoint{1.289340in}{1.067089in}}%
\pgfpathlineto{\pgfqpoint{1.284756in}{1.054960in}}%
\pgfpathlineto{\pgfqpoint{1.282427in}{1.046252in}}%
\pgfpathlineto{\pgfqpoint{1.281205in}{1.042832in}}%
\pgfpathlineto{\pgfqpoint{1.276104in}{1.030703in}}%
\pgfpathlineto{\pgfqpoint{1.270715in}{1.020137in}}%
\pgfpathlineto{\pgfqpoint{1.269877in}{1.018574in}}%
\pgfpathlineto{\pgfqpoint{1.263218in}{1.006445in}}%
\pgfpathlineto{\pgfqpoint{1.259002in}{0.999691in}}%
\pgfpathlineto{\pgfqpoint{1.254827in}{0.994316in}}%
\pgfpathlineto{\pgfqpoint{1.247289in}{0.986053in}}%
\pgfpathlineto{\pgfqpoint{1.240892in}{0.982187in}}%
\pgfpathlineto{\pgfqpoint{1.235577in}{0.979897in}}%
\pgfpathlineto{\pgfqpoint{1.223864in}{0.976337in}}%
\pgfpathlineto{\pgfqpoint{1.212151in}{0.973655in}}%
\pgfpathlineto{\pgfqpoint{1.200439in}{0.972181in}}%
\pgfpathlineto{\pgfqpoint{1.188726in}{0.973638in}}%
\pgfpathclose%
\pgfusepath{fill}%
\end{pgfscope}%
\begin{pgfscope}%
\pgfpathrectangle{\pgfqpoint{0.211875in}{0.211875in}}{\pgfqpoint{1.313625in}{1.279725in}}%
\pgfusepath{clip}%
\pgfsetbuttcap%
\pgfsetroundjoin%
\definecolor{currentfill}{rgb}{0.644838,0.098089,0.355336}%
\pgfsetfillcolor{currentfill}%
\pgfsetlinewidth{0.000000pt}%
\definecolor{currentstroke}{rgb}{0.000000,0.000000,0.000000}%
\pgfsetstrokecolor{currentstroke}%
\pgfsetdash{}{0pt}%
\pgfpathmoveto{\pgfqpoint{1.446405in}{0.749397in}}%
\pgfpathlineto{\pgfqpoint{1.446405in}{0.751739in}}%
\pgfpathlineto{\pgfqpoint{1.446405in}{0.763867in}}%
\pgfpathlineto{\pgfqpoint{1.446405in}{0.775996in}}%
\pgfpathlineto{\pgfqpoint{1.446405in}{0.788125in}}%
\pgfpathlineto{\pgfqpoint{1.446405in}{0.794651in}}%
\pgfpathlineto{\pgfqpoint{1.440623in}{0.788125in}}%
\pgfpathlineto{\pgfqpoint{1.434692in}{0.778112in}}%
\pgfpathlineto{\pgfqpoint{1.433402in}{0.775996in}}%
\pgfpathlineto{\pgfqpoint{1.432661in}{0.763867in}}%
\pgfpathlineto{\pgfqpoint{1.434692in}{0.758317in}}%
\pgfpathlineto{\pgfqpoint{1.440054in}{0.751739in}}%
\pgfpathclose%
\pgfusepath{fill}%
\end{pgfscope}%
\begin{pgfscope}%
\pgfpathrectangle{\pgfqpoint{0.211875in}{0.211875in}}{\pgfqpoint{1.313625in}{1.279725in}}%
\pgfusepath{clip}%
\pgfsetbuttcap%
\pgfsetroundjoin%
\definecolor{currentfill}{rgb}{0.644838,0.098089,0.355336}%
\pgfsetfillcolor{currentfill}%
\pgfsetlinewidth{0.000000pt}%
\definecolor{currentstroke}{rgb}{0.000000,0.000000,0.000000}%
\pgfsetstrokecolor{currentstroke}%
\pgfsetdash{}{0pt}%
\pgfpathmoveto{\pgfqpoint{1.446405in}{1.220637in}}%
\pgfpathlineto{\pgfqpoint{1.446405in}{1.224765in}}%
\pgfpathlineto{\pgfqpoint{1.446405in}{1.236894in}}%
\pgfpathlineto{\pgfqpoint{1.446405in}{1.249022in}}%
\pgfpathlineto{\pgfqpoint{1.446405in}{1.261151in}}%
\pgfpathlineto{\pgfqpoint{1.446405in}{1.273280in}}%
\pgfpathlineto{\pgfqpoint{1.446405in}{1.285409in}}%
\pgfpathlineto{\pgfqpoint{1.446405in}{1.296824in}}%
\pgfpathlineto{\pgfqpoint{1.437110in}{1.297538in}}%
\pgfpathlineto{\pgfqpoint{1.434692in}{1.297720in}}%
\pgfpathlineto{\pgfqpoint{1.433571in}{1.297538in}}%
\pgfpathlineto{\pgfqpoint{1.422980in}{1.295551in}}%
\pgfpathlineto{\pgfqpoint{1.411267in}{1.286494in}}%
\pgfpathlineto{\pgfqpoint{1.410596in}{1.285409in}}%
\pgfpathlineto{\pgfqpoint{1.408668in}{1.273280in}}%
\pgfpathlineto{\pgfqpoint{1.410619in}{1.261151in}}%
\pgfpathlineto{\pgfqpoint{1.411267in}{1.260272in}}%
\pgfpathlineto{\pgfqpoint{1.417365in}{1.249022in}}%
\pgfpathlineto{\pgfqpoint{1.422980in}{1.239556in}}%
\pgfpathlineto{\pgfqpoint{1.424723in}{1.236894in}}%
\pgfpathlineto{\pgfqpoint{1.434692in}{1.225276in}}%
\pgfpathlineto{\pgfqpoint{1.435365in}{1.224765in}}%
\pgfpathclose%
\pgfusepath{fill}%
\end{pgfscope}%
\begin{pgfscope}%
\pgfpathrectangle{\pgfqpoint{0.211875in}{0.211875in}}{\pgfqpoint{1.313625in}{1.279725in}}%
\pgfusepath{clip}%
\pgfsetbuttcap%
\pgfsetroundjoin%
\definecolor{currentfill}{rgb}{0.644838,0.098089,0.355336}%
\pgfsetfillcolor{currentfill}%
\pgfsetlinewidth{0.000000pt}%
\definecolor{currentstroke}{rgb}{0.000000,0.000000,0.000000}%
\pgfsetstrokecolor{currentstroke}%
\pgfsetdash{}{0pt}%
\pgfpathmoveto{\pgfqpoint{0.298562in}{1.244800in}}%
\pgfpathlineto{\pgfqpoint{0.310216in}{1.249022in}}%
\pgfpathlineto{\pgfqpoint{0.310274in}{1.249061in}}%
\pgfpathlineto{\pgfqpoint{0.320321in}{1.261151in}}%
\pgfpathlineto{\pgfqpoint{0.321987in}{1.266469in}}%
\pgfpathlineto{\pgfqpoint{0.323262in}{1.273280in}}%
\pgfpathlineto{\pgfqpoint{0.324393in}{1.285409in}}%
\pgfpathlineto{\pgfqpoint{0.325776in}{1.297538in}}%
\pgfpathlineto{\pgfqpoint{0.327766in}{1.309667in}}%
\pgfpathlineto{\pgfqpoint{0.331921in}{1.321796in}}%
\pgfpathlineto{\pgfqpoint{0.333700in}{1.326952in}}%
\pgfpathlineto{\pgfqpoint{0.336227in}{1.333925in}}%
\pgfpathlineto{\pgfqpoint{0.341528in}{1.346053in}}%
\pgfpathlineto{\pgfqpoint{0.345412in}{1.355774in}}%
\pgfpathlineto{\pgfqpoint{0.346329in}{1.358182in}}%
\pgfpathlineto{\pgfqpoint{0.349843in}{1.370311in}}%
\pgfpathlineto{\pgfqpoint{0.351972in}{1.382440in}}%
\pgfpathlineto{\pgfqpoint{0.354707in}{1.394569in}}%
\pgfpathlineto{\pgfqpoint{0.357125in}{1.401913in}}%
\pgfpathlineto{\pgfqpoint{0.359276in}{1.406698in}}%
\pgfpathlineto{\pgfqpoint{0.365391in}{1.418827in}}%
\pgfpathlineto{\pgfqpoint{0.368838in}{1.426774in}}%
\pgfpathlineto{\pgfqpoint{0.370522in}{1.430956in}}%
\pgfpathlineto{\pgfqpoint{0.373388in}{1.443084in}}%
\pgfpathlineto{\pgfqpoint{0.374285in}{1.455213in}}%
\pgfpathlineto{\pgfqpoint{0.372446in}{1.467342in}}%
\pgfpathlineto{\pgfqpoint{0.369102in}{1.479471in}}%
\pgfpathlineto{\pgfqpoint{0.368838in}{1.480136in}}%
\pgfpathlineto{\pgfqpoint{0.364259in}{1.491600in}}%
\pgfpathlineto{\pgfqpoint{0.357125in}{1.491600in}}%
\pgfpathlineto{\pgfqpoint{0.345412in}{1.491600in}}%
\pgfpathlineto{\pgfqpoint{0.333700in}{1.491600in}}%
\pgfpathlineto{\pgfqpoint{0.321987in}{1.491600in}}%
\pgfpathlineto{\pgfqpoint{0.310274in}{1.491600in}}%
\pgfpathlineto{\pgfqpoint{0.298562in}{1.491600in}}%
\pgfpathlineto{\pgfqpoint{0.286849in}{1.491600in}}%
\pgfpathlineto{\pgfqpoint{0.286849in}{1.479471in}}%
\pgfpathlineto{\pgfqpoint{0.286849in}{1.467342in}}%
\pgfpathlineto{\pgfqpoint{0.286849in}{1.455213in}}%
\pgfpathlineto{\pgfqpoint{0.286849in}{1.443084in}}%
\pgfpathlineto{\pgfqpoint{0.286849in}{1.430956in}}%
\pgfpathlineto{\pgfqpoint{0.286849in}{1.418827in}}%
\pgfpathlineto{\pgfqpoint{0.286849in}{1.406698in}}%
\pgfpathlineto{\pgfqpoint{0.286849in}{1.394569in}}%
\pgfpathlineto{\pgfqpoint{0.286849in}{1.382440in}}%
\pgfpathlineto{\pgfqpoint{0.286849in}{1.370311in}}%
\pgfpathlineto{\pgfqpoint{0.286849in}{1.358182in}}%
\pgfpathlineto{\pgfqpoint{0.286849in}{1.346053in}}%
\pgfpathlineto{\pgfqpoint{0.286849in}{1.333925in}}%
\pgfpathlineto{\pgfqpoint{0.286849in}{1.321796in}}%
\pgfpathlineto{\pgfqpoint{0.286849in}{1.309667in}}%
\pgfpathlineto{\pgfqpoint{0.286849in}{1.297538in}}%
\pgfpathlineto{\pgfqpoint{0.286849in}{1.285409in}}%
\pgfpathlineto{\pgfqpoint{0.286849in}{1.273280in}}%
\pgfpathlineto{\pgfqpoint{0.286849in}{1.261151in}}%
\pgfpathlineto{\pgfqpoint{0.286849in}{1.249022in}}%
\pgfpathlineto{\pgfqpoint{0.286849in}{1.244257in}}%
\pgfpathclose%
\pgfusepath{fill}%
\end{pgfscope}%
\begin{pgfscope}%
\pgfpathrectangle{\pgfqpoint{0.211875in}{0.211875in}}{\pgfqpoint{1.313625in}{1.279725in}}%
\pgfusepath{clip}%
\pgfsetbuttcap%
\pgfsetroundjoin%
\definecolor{currentfill}{rgb}{0.644838,0.098089,0.355336}%
\pgfsetfillcolor{currentfill}%
\pgfsetlinewidth{0.000000pt}%
\definecolor{currentstroke}{rgb}{0.000000,0.000000,0.000000}%
\pgfsetstrokecolor{currentstroke}%
\pgfsetdash{}{0pt}%
\pgfpathmoveto{\pgfqpoint{1.001323in}{1.320460in}}%
\pgfpathlineto{\pgfqpoint{1.013036in}{1.317745in}}%
\pgfpathlineto{\pgfqpoint{1.024748in}{1.316835in}}%
\pgfpathlineto{\pgfqpoint{1.036461in}{1.316748in}}%
\pgfpathlineto{\pgfqpoint{1.048174in}{1.319274in}}%
\pgfpathlineto{\pgfqpoint{1.057112in}{1.321796in}}%
\pgfpathlineto{\pgfqpoint{1.059886in}{1.322489in}}%
\pgfpathlineto{\pgfqpoint{1.071599in}{1.325753in}}%
\pgfpathlineto{\pgfqpoint{1.083312in}{1.327872in}}%
\pgfpathlineto{\pgfqpoint{1.095024in}{1.330710in}}%
\pgfpathlineto{\pgfqpoint{1.099298in}{1.333925in}}%
\pgfpathlineto{\pgfqpoint{1.106737in}{1.340251in}}%
\pgfpathlineto{\pgfqpoint{1.109846in}{1.346053in}}%
\pgfpathlineto{\pgfqpoint{1.118450in}{1.356924in}}%
\pgfpathlineto{\pgfqpoint{1.119319in}{1.358182in}}%
\pgfpathlineto{\pgfqpoint{1.127951in}{1.370311in}}%
\pgfpathlineto{\pgfqpoint{1.130163in}{1.374095in}}%
\pgfpathlineto{\pgfqpoint{1.133554in}{1.382440in}}%
\pgfpathlineto{\pgfqpoint{1.141875in}{1.393412in}}%
\pgfpathlineto{\pgfqpoint{1.142870in}{1.394569in}}%
\pgfpathlineto{\pgfqpoint{1.153588in}{1.403594in}}%
\pgfpathlineto{\pgfqpoint{1.160129in}{1.406698in}}%
\pgfpathlineto{\pgfqpoint{1.165301in}{1.408921in}}%
\pgfpathlineto{\pgfqpoint{1.177013in}{1.409190in}}%
\pgfpathlineto{\pgfqpoint{1.188726in}{1.410039in}}%
\pgfpathlineto{\pgfqpoint{1.200439in}{1.408565in}}%
\pgfpathlineto{\pgfqpoint{1.212151in}{1.411802in}}%
\pgfpathlineto{\pgfqpoint{1.219994in}{1.418827in}}%
\pgfpathlineto{\pgfqpoint{1.223864in}{1.422812in}}%
\pgfpathlineto{\pgfqpoint{1.227913in}{1.430956in}}%
\pgfpathlineto{\pgfqpoint{1.235577in}{1.438996in}}%
\pgfpathlineto{\pgfqpoint{1.247289in}{1.439342in}}%
\pgfpathlineto{\pgfqpoint{1.259002in}{1.433868in}}%
\pgfpathlineto{\pgfqpoint{1.270715in}{1.433427in}}%
\pgfpathlineto{\pgfqpoint{1.282427in}{1.441369in}}%
\pgfpathlineto{\pgfqpoint{1.285072in}{1.443084in}}%
\pgfpathlineto{\pgfqpoint{1.294140in}{1.453111in}}%
\pgfpathlineto{\pgfqpoint{1.296500in}{1.455213in}}%
\pgfpathlineto{\pgfqpoint{1.305853in}{1.465193in}}%
\pgfpathlineto{\pgfqpoint{1.307795in}{1.467342in}}%
\pgfpathlineto{\pgfqpoint{1.317566in}{1.478800in}}%
\pgfpathlineto{\pgfqpoint{1.318163in}{1.479471in}}%
\pgfpathlineto{\pgfqpoint{1.327980in}{1.491600in}}%
\pgfpathlineto{\pgfqpoint{1.317566in}{1.491600in}}%
\pgfpathlineto{\pgfqpoint{1.312146in}{1.491600in}}%
\pgfpathlineto{\pgfqpoint{1.305853in}{1.484079in}}%
\pgfpathlineto{\pgfqpoint{1.303111in}{1.479471in}}%
\pgfpathlineto{\pgfqpoint{1.295157in}{1.467342in}}%
\pgfpathlineto{\pgfqpoint{1.294140in}{1.466269in}}%
\pgfpathlineto{\pgfqpoint{1.285155in}{1.455213in}}%
\pgfpathlineto{\pgfqpoint{1.282427in}{1.452258in}}%
\pgfpathlineto{\pgfqpoint{1.270973in}{1.443084in}}%
\pgfpathlineto{\pgfqpoint{1.270715in}{1.442908in}}%
\pgfpathlineto{\pgfqpoint{1.269936in}{1.443084in}}%
\pgfpathlineto{\pgfqpoint{1.259002in}{1.446912in}}%
\pgfpathlineto{\pgfqpoint{1.247289in}{1.453738in}}%
\pgfpathlineto{\pgfqpoint{1.242483in}{1.455213in}}%
\pgfpathlineto{\pgfqpoint{1.235577in}{1.457938in}}%
\pgfpathlineto{\pgfqpoint{1.223864in}{1.458523in}}%
\pgfpathlineto{\pgfqpoint{1.215533in}{1.455213in}}%
\pgfpathlineto{\pgfqpoint{1.212151in}{1.451836in}}%
\pgfpathlineto{\pgfqpoint{1.204299in}{1.443084in}}%
\pgfpathlineto{\pgfqpoint{1.200439in}{1.440602in}}%
\pgfpathlineto{\pgfqpoint{1.191131in}{1.430956in}}%
\pgfpathlineto{\pgfqpoint{1.188726in}{1.429376in}}%
\pgfpathlineto{\pgfqpoint{1.177013in}{1.427385in}}%
\pgfpathlineto{\pgfqpoint{1.165301in}{1.427129in}}%
\pgfpathlineto{\pgfqpoint{1.153588in}{1.425006in}}%
\pgfpathlineto{\pgfqpoint{1.142768in}{1.418827in}}%
\pgfpathlineto{\pgfqpoint{1.141875in}{1.418317in}}%
\pgfpathlineto{\pgfqpoint{1.130163in}{1.411179in}}%
\pgfpathlineto{\pgfqpoint{1.123627in}{1.406698in}}%
\pgfpathlineto{\pgfqpoint{1.118450in}{1.401936in}}%
\pgfpathlineto{\pgfqpoint{1.109428in}{1.394569in}}%
\pgfpathlineto{\pgfqpoint{1.106737in}{1.391469in}}%
\pgfpathlineto{\pgfqpoint{1.097047in}{1.382440in}}%
\pgfpathlineto{\pgfqpoint{1.095024in}{1.380915in}}%
\pgfpathlineto{\pgfqpoint{1.083312in}{1.377891in}}%
\pgfpathlineto{\pgfqpoint{1.071599in}{1.378601in}}%
\pgfpathlineto{\pgfqpoint{1.066457in}{1.382440in}}%
\pgfpathlineto{\pgfqpoint{1.063634in}{1.394569in}}%
\pgfpathlineto{\pgfqpoint{1.065133in}{1.406698in}}%
\pgfpathlineto{\pgfqpoint{1.068429in}{1.418827in}}%
\pgfpathlineto{\pgfqpoint{1.063777in}{1.430956in}}%
\pgfpathlineto{\pgfqpoint{1.060452in}{1.443084in}}%
\pgfpathlineto{\pgfqpoint{1.059886in}{1.447638in}}%
\pgfpathlineto{\pgfqpoint{1.058945in}{1.455213in}}%
\pgfpathlineto{\pgfqpoint{1.057915in}{1.467342in}}%
\pgfpathlineto{\pgfqpoint{1.056957in}{1.479471in}}%
\pgfpathlineto{\pgfqpoint{1.056045in}{1.491600in}}%
\pgfpathlineto{\pgfqpoint{1.048174in}{1.491600in}}%
\pgfpathlineto{\pgfqpoint{1.036461in}{1.491600in}}%
\pgfpathlineto{\pgfqpoint{1.024748in}{1.491600in}}%
\pgfpathlineto{\pgfqpoint{1.019842in}{1.491600in}}%
\pgfpathlineto{\pgfqpoint{1.019825in}{1.479471in}}%
\pgfpathlineto{\pgfqpoint{1.020635in}{1.467342in}}%
\pgfpathlineto{\pgfqpoint{1.022195in}{1.455213in}}%
\pgfpathlineto{\pgfqpoint{1.023668in}{1.443084in}}%
\pgfpathlineto{\pgfqpoint{1.024491in}{1.430956in}}%
\pgfpathlineto{\pgfqpoint{1.023973in}{1.418827in}}%
\pgfpathlineto{\pgfqpoint{1.024748in}{1.409555in}}%
\pgfpathlineto{\pgfqpoint{1.024961in}{1.406698in}}%
\pgfpathlineto{\pgfqpoint{1.024748in}{1.405539in}}%
\pgfpathlineto{\pgfqpoint{1.022410in}{1.394569in}}%
\pgfpathlineto{\pgfqpoint{1.017264in}{1.382440in}}%
\pgfpathlineto{\pgfqpoint{1.013036in}{1.372781in}}%
\pgfpathlineto{\pgfqpoint{1.011795in}{1.370311in}}%
\pgfpathlineto{\pgfqpoint{1.008036in}{1.358182in}}%
\pgfpathlineto{\pgfqpoint{1.004942in}{1.346053in}}%
\pgfpathlineto{\pgfqpoint{1.002158in}{1.333925in}}%
\pgfpathlineto{\pgfqpoint{1.001323in}{1.327848in}}%
\pgfpathlineto{\pgfqpoint{1.000331in}{1.321796in}}%
\pgfpathclose%
\pgfusepath{fill}%
\end{pgfscope}%
\begin{pgfscope}%
\pgfpathrectangle{\pgfqpoint{0.211875in}{0.211875in}}{\pgfqpoint{1.313625in}{1.279725in}}%
\pgfusepath{clip}%
\pgfsetbuttcap%
\pgfsetroundjoin%
\definecolor{currentfill}{rgb}{0.644838,0.098089,0.355336}%
\pgfsetfillcolor{currentfill}%
\pgfsetlinewidth{0.000000pt}%
\definecolor{currentstroke}{rgb}{0.000000,0.000000,0.000000}%
\pgfsetstrokecolor{currentstroke}%
\pgfsetdash{}{0pt}%
\pgfpathmoveto{\pgfqpoint{1.317566in}{1.368321in}}%
\pgfpathlineto{\pgfqpoint{1.329278in}{1.366391in}}%
\pgfpathlineto{\pgfqpoint{1.335273in}{1.370311in}}%
\pgfpathlineto{\pgfqpoint{1.340991in}{1.380210in}}%
\pgfpathlineto{\pgfqpoint{1.341639in}{1.382440in}}%
\pgfpathlineto{\pgfqpoint{1.340991in}{1.386076in}}%
\pgfpathlineto{\pgfqpoint{1.329278in}{1.386230in}}%
\pgfpathlineto{\pgfqpoint{1.326183in}{1.382440in}}%
\pgfpathlineto{\pgfqpoint{1.317566in}{1.371611in}}%
\pgfpathlineto{\pgfqpoint{1.317110in}{1.370311in}}%
\pgfpathclose%
\pgfusepath{fill}%
\end{pgfscope}%
\begin{pgfscope}%
\pgfpathrectangle{\pgfqpoint{0.211875in}{0.211875in}}{\pgfqpoint{1.313625in}{1.279725in}}%
\pgfusepath{clip}%
\pgfsetbuttcap%
\pgfsetroundjoin%
\definecolor{currentfill}{rgb}{0.796501,0.105066,0.310630}%
\pgfsetfillcolor{currentfill}%
\pgfsetlinewidth{0.000000pt}%
\definecolor{currentstroke}{rgb}{0.000000,0.000000,0.000000}%
\pgfsetstrokecolor{currentstroke}%
\pgfsetdash{}{0pt}%
\pgfpathmoveto{\pgfqpoint{0.380550in}{0.290841in}}%
\pgfpathlineto{\pgfqpoint{0.392263in}{0.290841in}}%
\pgfpathlineto{\pgfqpoint{0.394097in}{0.290841in}}%
\pgfpathlineto{\pgfqpoint{0.403976in}{0.298260in}}%
\pgfpathlineto{\pgfqpoint{0.410749in}{0.302970in}}%
\pgfpathlineto{\pgfqpoint{0.415688in}{0.306942in}}%
\pgfpathlineto{\pgfqpoint{0.427168in}{0.315099in}}%
\pgfpathlineto{\pgfqpoint{0.427401in}{0.315304in}}%
\pgfpathlineto{\pgfqpoint{0.439114in}{0.325026in}}%
\pgfpathlineto{\pgfqpoint{0.442052in}{0.327228in}}%
\pgfpathlineto{\pgfqpoint{0.450827in}{0.334808in}}%
\pgfpathlineto{\pgfqpoint{0.456500in}{0.339357in}}%
\pgfpathlineto{\pgfqpoint{0.462539in}{0.344733in}}%
\pgfpathlineto{\pgfqpoint{0.471280in}{0.351486in}}%
\pgfpathlineto{\pgfqpoint{0.474252in}{0.354405in}}%
\pgfpathlineto{\pgfqpoint{0.484897in}{0.363615in}}%
\pgfpathlineto{\pgfqpoint{0.485965in}{0.364724in}}%
\pgfpathlineto{\pgfqpoint{0.497677in}{0.375091in}}%
\pgfpathlineto{\pgfqpoint{0.498427in}{0.375743in}}%
\pgfpathlineto{\pgfqpoint{0.509390in}{0.387486in}}%
\pgfpathlineto{\pgfqpoint{0.509843in}{0.387872in}}%
\pgfpathlineto{\pgfqpoint{0.521047in}{0.400001in}}%
\pgfpathlineto{\pgfqpoint{0.521103in}{0.400063in}}%
\pgfpathlineto{\pgfqpoint{0.532305in}{0.412130in}}%
\pgfpathlineto{\pgfqpoint{0.532815in}{0.412745in}}%
\pgfpathlineto{\pgfqpoint{0.543745in}{0.424259in}}%
\pgfpathlineto{\pgfqpoint{0.544528in}{0.425220in}}%
\pgfpathlineto{\pgfqpoint{0.555430in}{0.436388in}}%
\pgfpathlineto{\pgfqpoint{0.556241in}{0.437537in}}%
\pgfpathlineto{\pgfqpoint{0.567953in}{0.447629in}}%
\pgfpathlineto{\pgfqpoint{0.569316in}{0.448517in}}%
\pgfpathlineto{\pgfqpoint{0.579666in}{0.455388in}}%
\pgfpathlineto{\pgfqpoint{0.591379in}{0.460245in}}%
\pgfpathlineto{\pgfqpoint{0.592134in}{0.460646in}}%
\pgfpathlineto{\pgfqpoint{0.603091in}{0.466782in}}%
\pgfpathlineto{\pgfqpoint{0.614804in}{0.472299in}}%
\pgfpathlineto{\pgfqpoint{0.615693in}{0.472774in}}%
\pgfpathlineto{\pgfqpoint{0.626517in}{0.478407in}}%
\pgfpathlineto{\pgfqpoint{0.638230in}{0.484387in}}%
\pgfpathlineto{\pgfqpoint{0.639218in}{0.484903in}}%
\pgfpathlineto{\pgfqpoint{0.649942in}{0.490381in}}%
\pgfpathlineto{\pgfqpoint{0.661655in}{0.495757in}}%
\pgfpathlineto{\pgfqpoint{0.664704in}{0.497032in}}%
\pgfpathlineto{\pgfqpoint{0.673368in}{0.500452in}}%
\pgfpathlineto{\pgfqpoint{0.685080in}{0.504398in}}%
\pgfpathlineto{\pgfqpoint{0.696793in}{0.506841in}}%
\pgfpathlineto{\pgfqpoint{0.708506in}{0.506131in}}%
\pgfpathlineto{\pgfqpoint{0.720218in}{0.503891in}}%
\pgfpathlineto{\pgfqpoint{0.731931in}{0.500278in}}%
\pgfpathlineto{\pgfqpoint{0.739179in}{0.497032in}}%
\pgfpathlineto{\pgfqpoint{0.743644in}{0.494903in}}%
\pgfpathlineto{\pgfqpoint{0.755356in}{0.487970in}}%
\pgfpathlineto{\pgfqpoint{0.761969in}{0.484903in}}%
\pgfpathlineto{\pgfqpoint{0.767069in}{0.482728in}}%
\pgfpathlineto{\pgfqpoint{0.778782in}{0.475495in}}%
\pgfpathlineto{\pgfqpoint{0.782689in}{0.472774in}}%
\pgfpathlineto{\pgfqpoint{0.790495in}{0.465564in}}%
\pgfpathlineto{\pgfqpoint{0.802207in}{0.470857in}}%
\pgfpathlineto{\pgfqpoint{0.804971in}{0.472774in}}%
\pgfpathlineto{\pgfqpoint{0.813920in}{0.477977in}}%
\pgfpathlineto{\pgfqpoint{0.824919in}{0.484903in}}%
\pgfpathlineto{\pgfqpoint{0.825633in}{0.485251in}}%
\pgfpathlineto{\pgfqpoint{0.837345in}{0.491225in}}%
\pgfpathlineto{\pgfqpoint{0.848526in}{0.497032in}}%
\pgfpathlineto{\pgfqpoint{0.849058in}{0.497263in}}%
\pgfpathlineto{\pgfqpoint{0.860771in}{0.502799in}}%
\pgfpathlineto{\pgfqpoint{0.871130in}{0.509161in}}%
\pgfpathlineto{\pgfqpoint{0.872483in}{0.510385in}}%
\pgfpathlineto{\pgfqpoint{0.880283in}{0.521290in}}%
\pgfpathlineto{\pgfqpoint{0.881598in}{0.533419in}}%
\pgfpathlineto{\pgfqpoint{0.884196in}{0.540501in}}%
\pgfpathlineto{\pgfqpoint{0.885871in}{0.545548in}}%
\pgfpathlineto{\pgfqpoint{0.884196in}{0.556016in}}%
\pgfpathlineto{\pgfqpoint{0.883844in}{0.557677in}}%
\pgfpathlineto{\pgfqpoint{0.877118in}{0.569805in}}%
\pgfpathlineto{\pgfqpoint{0.872789in}{0.581934in}}%
\pgfpathlineto{\pgfqpoint{0.872483in}{0.585323in}}%
\pgfpathlineto{\pgfqpoint{0.871800in}{0.594063in}}%
\pgfpathlineto{\pgfqpoint{0.870202in}{0.606192in}}%
\pgfpathlineto{\pgfqpoint{0.868455in}{0.618321in}}%
\pgfpathlineto{\pgfqpoint{0.868947in}{0.630450in}}%
\pgfpathlineto{\pgfqpoint{0.868983in}{0.642579in}}%
\pgfpathlineto{\pgfqpoint{0.869423in}{0.654708in}}%
\pgfpathlineto{\pgfqpoint{0.860771in}{0.665086in}}%
\pgfpathlineto{\pgfqpoint{0.858284in}{0.666836in}}%
\pgfpathlineto{\pgfqpoint{0.849058in}{0.671887in}}%
\pgfpathlineto{\pgfqpoint{0.838064in}{0.678965in}}%
\pgfpathlineto{\pgfqpoint{0.837345in}{0.679404in}}%
\pgfpathlineto{\pgfqpoint{0.825633in}{0.686150in}}%
\pgfpathlineto{\pgfqpoint{0.817917in}{0.691094in}}%
\pgfpathlineto{\pgfqpoint{0.820238in}{0.703223in}}%
\pgfpathlineto{\pgfqpoint{0.825633in}{0.707627in}}%
\pgfpathlineto{\pgfqpoint{0.835760in}{0.715352in}}%
\pgfpathlineto{\pgfqpoint{0.837345in}{0.715920in}}%
\pgfpathlineto{\pgfqpoint{0.849058in}{0.720393in}}%
\pgfpathlineto{\pgfqpoint{0.860771in}{0.725066in}}%
\pgfpathlineto{\pgfqpoint{0.865957in}{0.727481in}}%
\pgfpathlineto{\pgfqpoint{0.866902in}{0.739610in}}%
\pgfpathlineto{\pgfqpoint{0.869799in}{0.751739in}}%
\pgfpathlineto{\pgfqpoint{0.870568in}{0.763867in}}%
\pgfpathlineto{\pgfqpoint{0.860771in}{0.775473in}}%
\pgfpathlineto{\pgfqpoint{0.860412in}{0.775996in}}%
\pgfpathlineto{\pgfqpoint{0.849058in}{0.786187in}}%
\pgfpathlineto{\pgfqpoint{0.837345in}{0.786553in}}%
\pgfpathlineto{\pgfqpoint{0.825633in}{0.786115in}}%
\pgfpathlineto{\pgfqpoint{0.813920in}{0.787268in}}%
\pgfpathlineto{\pgfqpoint{0.808423in}{0.788125in}}%
\pgfpathlineto{\pgfqpoint{0.802207in}{0.789194in}}%
\pgfpathlineto{\pgfqpoint{0.790495in}{0.791819in}}%
\pgfpathlineto{\pgfqpoint{0.778782in}{0.794706in}}%
\pgfpathlineto{\pgfqpoint{0.767069in}{0.797960in}}%
\pgfpathlineto{\pgfqpoint{0.757887in}{0.800254in}}%
\pgfpathlineto{\pgfqpoint{0.755356in}{0.800950in}}%
\pgfpathlineto{\pgfqpoint{0.743644in}{0.804165in}}%
\pgfpathlineto{\pgfqpoint{0.731931in}{0.807392in}}%
\pgfpathlineto{\pgfqpoint{0.720218in}{0.810589in}}%
\pgfpathlineto{\pgfqpoint{0.713652in}{0.812383in}}%
\pgfpathlineto{\pgfqpoint{0.708506in}{0.814129in}}%
\pgfpathlineto{\pgfqpoint{0.696793in}{0.818457in}}%
\pgfpathlineto{\pgfqpoint{0.685080in}{0.822739in}}%
\pgfpathlineto{\pgfqpoint{0.680231in}{0.824512in}}%
\pgfpathlineto{\pgfqpoint{0.673368in}{0.827855in}}%
\pgfpathlineto{\pgfqpoint{0.661655in}{0.833459in}}%
\pgfpathlineto{\pgfqpoint{0.655404in}{0.836641in}}%
\pgfpathlineto{\pgfqpoint{0.649942in}{0.840704in}}%
\pgfpathlineto{\pgfqpoint{0.642416in}{0.848770in}}%
\pgfpathlineto{\pgfqpoint{0.638230in}{0.854304in}}%
\pgfpathlineto{\pgfqpoint{0.633168in}{0.860898in}}%
\pgfpathlineto{\pgfqpoint{0.626517in}{0.870221in}}%
\pgfpathlineto{\pgfqpoint{0.624560in}{0.873027in}}%
\pgfpathlineto{\pgfqpoint{0.616165in}{0.885156in}}%
\pgfpathlineto{\pgfqpoint{0.614804in}{0.887049in}}%
\pgfpathlineto{\pgfqpoint{0.607457in}{0.897285in}}%
\pgfpathlineto{\pgfqpoint{0.603091in}{0.902212in}}%
\pgfpathlineto{\pgfqpoint{0.596811in}{0.909414in}}%
\pgfpathlineto{\pgfqpoint{0.591379in}{0.916045in}}%
\pgfpathlineto{\pgfqpoint{0.586223in}{0.921543in}}%
\pgfpathlineto{\pgfqpoint{0.579666in}{0.927455in}}%
\pgfpathlineto{\pgfqpoint{0.572155in}{0.933672in}}%
\pgfpathlineto{\pgfqpoint{0.567953in}{0.936999in}}%
\pgfpathlineto{\pgfqpoint{0.557635in}{0.945801in}}%
\pgfpathlineto{\pgfqpoint{0.556241in}{0.947992in}}%
\pgfpathlineto{\pgfqpoint{0.552037in}{0.957929in}}%
\pgfpathlineto{\pgfqpoint{0.545607in}{0.970058in}}%
\pgfpathlineto{\pgfqpoint{0.544528in}{0.971823in}}%
\pgfpathlineto{\pgfqpoint{0.538081in}{0.982187in}}%
\pgfpathlineto{\pgfqpoint{0.532815in}{0.990315in}}%
\pgfpathlineto{\pgfqpoint{0.530108in}{0.994316in}}%
\pgfpathlineto{\pgfqpoint{0.521117in}{1.006445in}}%
\pgfpathlineto{\pgfqpoint{0.521103in}{1.006462in}}%
\pgfpathlineto{\pgfqpoint{0.510765in}{1.018574in}}%
\pgfpathlineto{\pgfqpoint{0.509390in}{1.020501in}}%
\pgfpathlineto{\pgfqpoint{0.502336in}{1.030703in}}%
\pgfpathlineto{\pgfqpoint{0.497677in}{1.037337in}}%
\pgfpathlineto{\pgfqpoint{0.493702in}{1.042832in}}%
\pgfpathlineto{\pgfqpoint{0.485965in}{1.052077in}}%
\pgfpathlineto{\pgfqpoint{0.483593in}{1.054960in}}%
\pgfpathlineto{\pgfqpoint{0.474252in}{1.064461in}}%
\pgfpathlineto{\pgfqpoint{0.471591in}{1.067089in}}%
\pgfpathlineto{\pgfqpoint{0.462539in}{1.076264in}}%
\pgfpathlineto{\pgfqpoint{0.459507in}{1.079218in}}%
\pgfpathlineto{\pgfqpoint{0.450827in}{1.087912in}}%
\pgfpathlineto{\pgfqpoint{0.447218in}{1.091347in}}%
\pgfpathlineto{\pgfqpoint{0.439114in}{1.099365in}}%
\pgfpathlineto{\pgfqpoint{0.434717in}{1.103476in}}%
\pgfpathlineto{\pgfqpoint{0.427401in}{1.110177in}}%
\pgfpathlineto{\pgfqpoint{0.421013in}{1.115605in}}%
\pgfpathlineto{\pgfqpoint{0.415688in}{1.120372in}}%
\pgfpathlineto{\pgfqpoint{0.406956in}{1.127734in}}%
\pgfpathlineto{\pgfqpoint{0.403976in}{1.130505in}}%
\pgfpathlineto{\pgfqpoint{0.393252in}{1.139863in}}%
\pgfpathlineto{\pgfqpoint{0.392263in}{1.141120in}}%
\pgfpathlineto{\pgfqpoint{0.382769in}{1.151991in}}%
\pgfpathlineto{\pgfqpoint{0.380550in}{1.155671in}}%
\pgfpathlineto{\pgfqpoint{0.374667in}{1.164120in}}%
\pgfpathlineto{\pgfqpoint{0.368838in}{1.174185in}}%
\pgfpathlineto{\pgfqpoint{0.367513in}{1.176249in}}%
\pgfpathlineto{\pgfqpoint{0.368838in}{1.179851in}}%
\pgfpathlineto{\pgfqpoint{0.380550in}{1.184437in}}%
\pgfpathlineto{\pgfqpoint{0.392263in}{1.182560in}}%
\pgfpathlineto{\pgfqpoint{0.403976in}{1.179759in}}%
\pgfpathlineto{\pgfqpoint{0.415688in}{1.176911in}}%
\pgfpathlineto{\pgfqpoint{0.418322in}{1.176249in}}%
\pgfpathlineto{\pgfqpoint{0.427401in}{1.174057in}}%
\pgfpathlineto{\pgfqpoint{0.439114in}{1.171266in}}%
\pgfpathlineto{\pgfqpoint{0.450827in}{1.168552in}}%
\pgfpathlineto{\pgfqpoint{0.462539in}{1.166144in}}%
\pgfpathlineto{\pgfqpoint{0.474252in}{1.164509in}}%
\pgfpathlineto{\pgfqpoint{0.477277in}{1.164120in}}%
\pgfpathlineto{\pgfqpoint{0.485965in}{1.163059in}}%
\pgfpathlineto{\pgfqpoint{0.497677in}{1.161824in}}%
\pgfpathlineto{\pgfqpoint{0.509390in}{1.160690in}}%
\pgfpathlineto{\pgfqpoint{0.521103in}{1.159663in}}%
\pgfpathlineto{\pgfqpoint{0.532815in}{1.158830in}}%
\pgfpathlineto{\pgfqpoint{0.544528in}{1.158164in}}%
\pgfpathlineto{\pgfqpoint{0.556241in}{1.157624in}}%
\pgfpathlineto{\pgfqpoint{0.567953in}{1.157374in}}%
\pgfpathlineto{\pgfqpoint{0.579666in}{1.157242in}}%
\pgfpathlineto{\pgfqpoint{0.591379in}{1.157836in}}%
\pgfpathlineto{\pgfqpoint{0.603091in}{1.158769in}}%
\pgfpathlineto{\pgfqpoint{0.614804in}{1.160325in}}%
\pgfpathlineto{\pgfqpoint{0.626517in}{1.162576in}}%
\pgfpathlineto{\pgfqpoint{0.632782in}{1.164120in}}%
\pgfpathlineto{\pgfqpoint{0.638230in}{1.165416in}}%
\pgfpathlineto{\pgfqpoint{0.649942in}{1.167845in}}%
\pgfpathlineto{\pgfqpoint{0.661655in}{1.170284in}}%
\pgfpathlineto{\pgfqpoint{0.673368in}{1.172967in}}%
\pgfpathlineto{\pgfqpoint{0.685080in}{1.175421in}}%
\pgfpathlineto{\pgfqpoint{0.689369in}{1.176249in}}%
\pgfpathlineto{\pgfqpoint{0.696793in}{1.177668in}}%
\pgfpathlineto{\pgfqpoint{0.708506in}{1.179946in}}%
\pgfpathlineto{\pgfqpoint{0.720218in}{1.182118in}}%
\pgfpathlineto{\pgfqpoint{0.731931in}{1.183895in}}%
\pgfpathlineto{\pgfqpoint{0.743644in}{1.185452in}}%
\pgfpathlineto{\pgfqpoint{0.755356in}{1.187044in}}%
\pgfpathlineto{\pgfqpoint{0.764831in}{1.188378in}}%
\pgfpathlineto{\pgfqpoint{0.767069in}{1.188705in}}%
\pgfpathlineto{\pgfqpoint{0.778782in}{1.190168in}}%
\pgfpathlineto{\pgfqpoint{0.790495in}{1.191464in}}%
\pgfpathlineto{\pgfqpoint{0.802207in}{1.192698in}}%
\pgfpathlineto{\pgfqpoint{0.813920in}{1.193731in}}%
\pgfpathlineto{\pgfqpoint{0.825633in}{1.194192in}}%
\pgfpathlineto{\pgfqpoint{0.837345in}{1.194095in}}%
\pgfpathlineto{\pgfqpoint{0.849058in}{1.193863in}}%
\pgfpathlineto{\pgfqpoint{0.860771in}{1.192850in}}%
\pgfpathlineto{\pgfqpoint{0.872483in}{1.191135in}}%
\pgfpathlineto{\pgfqpoint{0.879426in}{1.188378in}}%
\pgfpathlineto{\pgfqpoint{0.884196in}{1.184658in}}%
\pgfpathlineto{\pgfqpoint{0.891134in}{1.176249in}}%
\pgfpathlineto{\pgfqpoint{0.895909in}{1.170049in}}%
\pgfpathlineto{\pgfqpoint{0.901683in}{1.164120in}}%
\pgfpathlineto{\pgfqpoint{0.907621in}{1.155439in}}%
\pgfpathlineto{\pgfqpoint{0.909698in}{1.151991in}}%
\pgfpathlineto{\pgfqpoint{0.907621in}{1.147192in}}%
\pgfpathlineto{\pgfqpoint{0.905243in}{1.139863in}}%
\pgfpathlineto{\pgfqpoint{0.902649in}{1.127734in}}%
\pgfpathlineto{\pgfqpoint{0.903469in}{1.115605in}}%
\pgfpathlineto{\pgfqpoint{0.903951in}{1.103476in}}%
\pgfpathlineto{\pgfqpoint{0.903165in}{1.091347in}}%
\pgfpathlineto{\pgfqpoint{0.899784in}{1.079218in}}%
\pgfpathlineto{\pgfqpoint{0.895909in}{1.068364in}}%
\pgfpathlineto{\pgfqpoint{0.895414in}{1.067089in}}%
\pgfpathlineto{\pgfqpoint{0.889183in}{1.054960in}}%
\pgfpathlineto{\pgfqpoint{0.884196in}{1.048069in}}%
\pgfpathlineto{\pgfqpoint{0.880237in}{1.042832in}}%
\pgfpathlineto{\pgfqpoint{0.872483in}{1.031333in}}%
\pgfpathlineto{\pgfqpoint{0.872049in}{1.030703in}}%
\pgfpathlineto{\pgfqpoint{0.867639in}{1.018574in}}%
\pgfpathlineto{\pgfqpoint{0.864192in}{1.006445in}}%
\pgfpathlineto{\pgfqpoint{0.863771in}{0.994316in}}%
\pgfpathlineto{\pgfqpoint{0.870104in}{0.982187in}}%
\pgfpathlineto{\pgfqpoint{0.872483in}{0.977825in}}%
\pgfpathlineto{\pgfqpoint{0.876278in}{0.970058in}}%
\pgfpathlineto{\pgfqpoint{0.882148in}{0.957929in}}%
\pgfpathlineto{\pgfqpoint{0.884196in}{0.954336in}}%
\pgfpathlineto{\pgfqpoint{0.889451in}{0.945801in}}%
\pgfpathlineto{\pgfqpoint{0.895909in}{0.938472in}}%
\pgfpathlineto{\pgfqpoint{0.900010in}{0.933672in}}%
\pgfpathlineto{\pgfqpoint{0.907621in}{0.923942in}}%
\pgfpathlineto{\pgfqpoint{0.909518in}{0.921543in}}%
\pgfpathlineto{\pgfqpoint{0.918406in}{0.909414in}}%
\pgfpathlineto{\pgfqpoint{0.919334in}{0.907977in}}%
\pgfpathlineto{\pgfqpoint{0.926029in}{0.897285in}}%
\pgfpathlineto{\pgfqpoint{0.931047in}{0.892795in}}%
\pgfpathlineto{\pgfqpoint{0.942759in}{0.890605in}}%
\pgfpathlineto{\pgfqpoint{0.954472in}{0.889958in}}%
\pgfpathlineto{\pgfqpoint{0.966185in}{0.891415in}}%
\pgfpathlineto{\pgfqpoint{0.977898in}{0.895993in}}%
\pgfpathlineto{\pgfqpoint{0.980811in}{0.897285in}}%
\pgfpathlineto{\pgfqpoint{0.989610in}{0.901020in}}%
\pgfpathlineto{\pgfqpoint{1.001323in}{0.905150in}}%
\pgfpathlineto{\pgfqpoint{1.013036in}{0.907728in}}%
\pgfpathlineto{\pgfqpoint{1.022742in}{0.909414in}}%
\pgfpathlineto{\pgfqpoint{1.024748in}{0.909757in}}%
\pgfpathlineto{\pgfqpoint{1.036461in}{0.911039in}}%
\pgfpathlineto{\pgfqpoint{1.039860in}{0.909414in}}%
\pgfpathlineto{\pgfqpoint{1.048174in}{0.899995in}}%
\pgfpathlineto{\pgfqpoint{1.049496in}{0.897285in}}%
\pgfpathlineto{\pgfqpoint{1.053108in}{0.885156in}}%
\pgfpathlineto{\pgfqpoint{1.055303in}{0.873027in}}%
\pgfpathlineto{\pgfqpoint{1.056464in}{0.860898in}}%
\pgfpathlineto{\pgfqpoint{1.057000in}{0.848770in}}%
\pgfpathlineto{\pgfqpoint{1.057162in}{0.836641in}}%
\pgfpathlineto{\pgfqpoint{1.054320in}{0.824512in}}%
\pgfpathlineto{\pgfqpoint{1.055325in}{0.812383in}}%
\pgfpathlineto{\pgfqpoint{1.059886in}{0.804914in}}%
\pgfpathlineto{\pgfqpoint{1.068988in}{0.800254in}}%
\pgfpathlineto{\pgfqpoint{1.071599in}{0.799571in}}%
\pgfpathlineto{\pgfqpoint{1.083312in}{0.794881in}}%
\pgfpathlineto{\pgfqpoint{1.095024in}{0.788256in}}%
\pgfpathlineto{\pgfqpoint{1.095267in}{0.788125in}}%
\pgfpathlineto{\pgfqpoint{1.106737in}{0.780748in}}%
\pgfpathlineto{\pgfqpoint{1.114050in}{0.775996in}}%
\pgfpathlineto{\pgfqpoint{1.118450in}{0.772521in}}%
\pgfpathlineto{\pgfqpoint{1.128339in}{0.763867in}}%
\pgfpathlineto{\pgfqpoint{1.130163in}{0.761694in}}%
\pgfpathlineto{\pgfqpoint{1.137982in}{0.751739in}}%
\pgfpathlineto{\pgfqpoint{1.141057in}{0.739610in}}%
\pgfpathlineto{\pgfqpoint{1.130163in}{0.727914in}}%
\pgfpathlineto{\pgfqpoint{1.129713in}{0.727481in}}%
\pgfpathlineto{\pgfqpoint{1.122766in}{0.715352in}}%
\pgfpathlineto{\pgfqpoint{1.118450in}{0.708808in}}%
\pgfpathlineto{\pgfqpoint{1.115651in}{0.703223in}}%
\pgfpathlineto{\pgfqpoint{1.118450in}{0.699260in}}%
\pgfpathlineto{\pgfqpoint{1.122774in}{0.691094in}}%
\pgfpathlineto{\pgfqpoint{1.129108in}{0.678965in}}%
\pgfpathlineto{\pgfqpoint{1.130163in}{0.677464in}}%
\pgfpathlineto{\pgfqpoint{1.138265in}{0.666836in}}%
\pgfpathlineto{\pgfqpoint{1.141875in}{0.660702in}}%
\pgfpathlineto{\pgfqpoint{1.147418in}{0.654708in}}%
\pgfpathlineto{\pgfqpoint{1.153588in}{0.651814in}}%
\pgfpathlineto{\pgfqpoint{1.165301in}{0.647080in}}%
\pgfpathlineto{\pgfqpoint{1.175763in}{0.642579in}}%
\pgfpathlineto{\pgfqpoint{1.177013in}{0.638496in}}%
\pgfpathlineto{\pgfqpoint{1.179306in}{0.630450in}}%
\pgfpathlineto{\pgfqpoint{1.177013in}{0.625965in}}%
\pgfpathlineto{\pgfqpoint{1.173486in}{0.618321in}}%
\pgfpathlineto{\pgfqpoint{1.177013in}{0.608011in}}%
\pgfpathlineto{\pgfqpoint{1.178370in}{0.606192in}}%
\pgfpathlineto{\pgfqpoint{1.188726in}{0.602056in}}%
\pgfpathlineto{\pgfqpoint{1.200439in}{0.602155in}}%
\pgfpathlineto{\pgfqpoint{1.212151in}{0.605665in}}%
\pgfpathlineto{\pgfqpoint{1.213643in}{0.606192in}}%
\pgfpathlineto{\pgfqpoint{1.223864in}{0.609609in}}%
\pgfpathlineto{\pgfqpoint{1.235577in}{0.610411in}}%
\pgfpathlineto{\pgfqpoint{1.247289in}{0.610741in}}%
\pgfpathlineto{\pgfqpoint{1.259002in}{0.609768in}}%
\pgfpathlineto{\pgfqpoint{1.270715in}{0.609674in}}%
\pgfpathlineto{\pgfqpoint{1.277886in}{0.618321in}}%
\pgfpathlineto{\pgfqpoint{1.282427in}{0.622450in}}%
\pgfpathlineto{\pgfqpoint{1.287962in}{0.630450in}}%
\pgfpathlineto{\pgfqpoint{1.294140in}{0.637668in}}%
\pgfpathlineto{\pgfqpoint{1.297766in}{0.642579in}}%
\pgfpathlineto{\pgfqpoint{1.305853in}{0.648343in}}%
\pgfpathlineto{\pgfqpoint{1.315401in}{0.654708in}}%
\pgfpathlineto{\pgfqpoint{1.317566in}{0.655687in}}%
\pgfpathlineto{\pgfqpoint{1.329278in}{0.662061in}}%
\pgfpathlineto{\pgfqpoint{1.335154in}{0.666836in}}%
\pgfpathlineto{\pgfqpoint{1.340991in}{0.671990in}}%
\pgfpathlineto{\pgfqpoint{1.347230in}{0.678965in}}%
\pgfpathlineto{\pgfqpoint{1.352704in}{0.682876in}}%
\pgfpathlineto{\pgfqpoint{1.364416in}{0.689212in}}%
\pgfpathlineto{\pgfqpoint{1.376129in}{0.685834in}}%
\pgfpathlineto{\pgfqpoint{1.385927in}{0.678965in}}%
\pgfpathlineto{\pgfqpoint{1.387842in}{0.677010in}}%
\pgfpathlineto{\pgfqpoint{1.399171in}{0.666836in}}%
\pgfpathlineto{\pgfqpoint{1.399554in}{0.666524in}}%
\pgfpathlineto{\pgfqpoint{1.411267in}{0.658772in}}%
\pgfpathlineto{\pgfqpoint{1.420995in}{0.654708in}}%
\pgfpathlineto{\pgfqpoint{1.422980in}{0.653661in}}%
\pgfpathlineto{\pgfqpoint{1.434692in}{0.651744in}}%
\pgfpathlineto{\pgfqpoint{1.446405in}{0.649189in}}%
\pgfpathlineto{\pgfqpoint{1.446405in}{0.654708in}}%
\pgfpathlineto{\pgfqpoint{1.446405in}{0.661765in}}%
\pgfpathlineto{\pgfqpoint{1.434692in}{0.662006in}}%
\pgfpathlineto{\pgfqpoint{1.422980in}{0.662441in}}%
\pgfpathlineto{\pgfqpoint{1.412730in}{0.666836in}}%
\pgfpathlineto{\pgfqpoint{1.411267in}{0.667480in}}%
\pgfpathlineto{\pgfqpoint{1.399554in}{0.676220in}}%
\pgfpathlineto{\pgfqpoint{1.396539in}{0.678965in}}%
\pgfpathlineto{\pgfqpoint{1.387842in}{0.685631in}}%
\pgfpathlineto{\pgfqpoint{1.379595in}{0.691094in}}%
\pgfpathlineto{\pgfqpoint{1.376129in}{0.695391in}}%
\pgfpathlineto{\pgfqpoint{1.364416in}{0.699443in}}%
\pgfpathlineto{\pgfqpoint{1.357123in}{0.691094in}}%
\pgfpathlineto{\pgfqpoint{1.352704in}{0.688645in}}%
\pgfpathlineto{\pgfqpoint{1.340991in}{0.680208in}}%
\pgfpathlineto{\pgfqpoint{1.339235in}{0.678965in}}%
\pgfpathlineto{\pgfqpoint{1.329278in}{0.669845in}}%
\pgfpathlineto{\pgfqpoint{1.323965in}{0.666836in}}%
\pgfpathlineto{\pgfqpoint{1.317566in}{0.663060in}}%
\pgfpathlineto{\pgfqpoint{1.305853in}{0.657851in}}%
\pgfpathlineto{\pgfqpoint{1.298382in}{0.654708in}}%
\pgfpathlineto{\pgfqpoint{1.294140in}{0.651847in}}%
\pgfpathlineto{\pgfqpoint{1.282427in}{0.645461in}}%
\pgfpathlineto{\pgfqpoint{1.278544in}{0.642579in}}%
\pgfpathlineto{\pgfqpoint{1.270715in}{0.636612in}}%
\pgfpathlineto{\pgfqpoint{1.264007in}{0.630450in}}%
\pgfpathlineto{\pgfqpoint{1.259002in}{0.627648in}}%
\pgfpathlineto{\pgfqpoint{1.247289in}{0.630156in}}%
\pgfpathlineto{\pgfqpoint{1.245451in}{0.630450in}}%
\pgfpathlineto{\pgfqpoint{1.235577in}{0.634999in}}%
\pgfpathlineto{\pgfqpoint{1.230271in}{0.642579in}}%
\pgfpathlineto{\pgfqpoint{1.223864in}{0.644874in}}%
\pgfpathlineto{\pgfqpoint{1.212151in}{0.646222in}}%
\pgfpathlineto{\pgfqpoint{1.200439in}{0.648419in}}%
\pgfpathlineto{\pgfqpoint{1.188726in}{0.653071in}}%
\pgfpathlineto{\pgfqpoint{1.185073in}{0.654708in}}%
\pgfpathlineto{\pgfqpoint{1.177013in}{0.658497in}}%
\pgfpathlineto{\pgfqpoint{1.165301in}{0.666515in}}%
\pgfpathlineto{\pgfqpoint{1.164902in}{0.666836in}}%
\pgfpathlineto{\pgfqpoint{1.154702in}{0.678965in}}%
\pgfpathlineto{\pgfqpoint{1.153588in}{0.680190in}}%
\pgfpathlineto{\pgfqpoint{1.142630in}{0.691094in}}%
\pgfpathlineto{\pgfqpoint{1.146068in}{0.703223in}}%
\pgfpathlineto{\pgfqpoint{1.153588in}{0.713870in}}%
\pgfpathlineto{\pgfqpoint{1.154499in}{0.715352in}}%
\pgfpathlineto{\pgfqpoint{1.163419in}{0.727481in}}%
\pgfpathlineto{\pgfqpoint{1.165301in}{0.736250in}}%
\pgfpathlineto{\pgfqpoint{1.165820in}{0.739610in}}%
\pgfpathlineto{\pgfqpoint{1.165301in}{0.742071in}}%
\pgfpathlineto{\pgfqpoint{1.162599in}{0.751739in}}%
\pgfpathlineto{\pgfqpoint{1.153588in}{0.762918in}}%
\pgfpathlineto{\pgfqpoint{1.152764in}{0.763867in}}%
\pgfpathlineto{\pgfqpoint{1.141875in}{0.773205in}}%
\pgfpathlineto{\pgfqpoint{1.138189in}{0.775996in}}%
\pgfpathlineto{\pgfqpoint{1.130163in}{0.780702in}}%
\pgfpathlineto{\pgfqpoint{1.119374in}{0.788125in}}%
\pgfpathlineto{\pgfqpoint{1.118450in}{0.788539in}}%
\pgfpathlineto{\pgfqpoint{1.106737in}{0.794160in}}%
\pgfpathlineto{\pgfqpoint{1.096756in}{0.800254in}}%
\pgfpathlineto{\pgfqpoint{1.095024in}{0.801459in}}%
\pgfpathlineto{\pgfqpoint{1.084872in}{0.812383in}}%
\pgfpathlineto{\pgfqpoint{1.083312in}{0.814922in}}%
\pgfpathlineto{\pgfqpoint{1.075203in}{0.824512in}}%
\pgfpathlineto{\pgfqpoint{1.075010in}{0.836641in}}%
\pgfpathlineto{\pgfqpoint{1.076283in}{0.848770in}}%
\pgfpathlineto{\pgfqpoint{1.076817in}{0.860898in}}%
\pgfpathlineto{\pgfqpoint{1.077332in}{0.873027in}}%
\pgfpathlineto{\pgfqpoint{1.081658in}{0.885156in}}%
\pgfpathlineto{\pgfqpoint{1.083312in}{0.886814in}}%
\pgfpathlineto{\pgfqpoint{1.095024in}{0.895202in}}%
\pgfpathlineto{\pgfqpoint{1.098123in}{0.897285in}}%
\pgfpathlineto{\pgfqpoint{1.106737in}{0.904130in}}%
\pgfpathlineto{\pgfqpoint{1.110316in}{0.909414in}}%
\pgfpathlineto{\pgfqpoint{1.117999in}{0.921543in}}%
\pgfpathlineto{\pgfqpoint{1.118450in}{0.922229in}}%
\pgfpathlineto{\pgfqpoint{1.128009in}{0.933672in}}%
\pgfpathlineto{\pgfqpoint{1.130163in}{0.935380in}}%
\pgfpathlineto{\pgfqpoint{1.141875in}{0.940006in}}%
\pgfpathlineto{\pgfqpoint{1.153588in}{0.938053in}}%
\pgfpathlineto{\pgfqpoint{1.165301in}{0.934067in}}%
\pgfpathlineto{\pgfqpoint{1.170138in}{0.933672in}}%
\pgfpathlineto{\pgfqpoint{1.177013in}{0.933224in}}%
\pgfpathlineto{\pgfqpoint{1.180079in}{0.933672in}}%
\pgfpathlineto{\pgfqpoint{1.188726in}{0.934896in}}%
\pgfpathlineto{\pgfqpoint{1.200439in}{0.935774in}}%
\pgfpathlineto{\pgfqpoint{1.212151in}{0.934758in}}%
\pgfpathlineto{\pgfqpoint{1.214232in}{0.933672in}}%
\pgfpathlineto{\pgfqpoint{1.223864in}{0.924606in}}%
\pgfpathlineto{\pgfqpoint{1.225663in}{0.921543in}}%
\pgfpathlineto{\pgfqpoint{1.231823in}{0.909414in}}%
\pgfpathlineto{\pgfqpoint{1.235577in}{0.903037in}}%
\pgfpathlineto{\pgfqpoint{1.238744in}{0.897285in}}%
\pgfpathlineto{\pgfqpoint{1.242130in}{0.885156in}}%
\pgfpathlineto{\pgfqpoint{1.237410in}{0.873027in}}%
\pgfpathlineto{\pgfqpoint{1.235577in}{0.871749in}}%
\pgfpathlineto{\pgfqpoint{1.223864in}{0.866683in}}%
\pgfpathlineto{\pgfqpoint{1.212151in}{0.862950in}}%
\pgfpathlineto{\pgfqpoint{1.207022in}{0.860898in}}%
\pgfpathlineto{\pgfqpoint{1.200439in}{0.857046in}}%
\pgfpathlineto{\pgfqpoint{1.188726in}{0.849269in}}%
\pgfpathlineto{\pgfqpoint{1.188123in}{0.848770in}}%
\pgfpathlineto{\pgfqpoint{1.177684in}{0.836641in}}%
\pgfpathlineto{\pgfqpoint{1.177013in}{0.835929in}}%
\pgfpathlineto{\pgfqpoint{1.170312in}{0.824512in}}%
\pgfpathlineto{\pgfqpoint{1.165301in}{0.812768in}}%
\pgfpathlineto{\pgfqpoint{1.165168in}{0.812383in}}%
\pgfpathlineto{\pgfqpoint{1.162100in}{0.800254in}}%
\pgfpathlineto{\pgfqpoint{1.161737in}{0.788125in}}%
\pgfpathlineto{\pgfqpoint{1.165301in}{0.776166in}}%
\pgfpathlineto{\pgfqpoint{1.165356in}{0.775996in}}%
\pgfpathlineto{\pgfqpoint{1.169802in}{0.763867in}}%
\pgfpathlineto{\pgfqpoint{1.177013in}{0.756804in}}%
\pgfpathlineto{\pgfqpoint{1.182855in}{0.751739in}}%
\pgfpathlineto{\pgfqpoint{1.188726in}{0.748364in}}%
\pgfpathlineto{\pgfqpoint{1.200439in}{0.742604in}}%
\pgfpathlineto{\pgfqpoint{1.207799in}{0.739610in}}%
\pgfpathlineto{\pgfqpoint{1.212151in}{0.738030in}}%
\pgfpathlineto{\pgfqpoint{1.223864in}{0.735121in}}%
\pgfpathlineto{\pgfqpoint{1.235577in}{0.729221in}}%
\pgfpathlineto{\pgfqpoint{1.237643in}{0.727481in}}%
\pgfpathlineto{\pgfqpoint{1.247289in}{0.719662in}}%
\pgfpathlineto{\pgfqpoint{1.253464in}{0.715352in}}%
\pgfpathlineto{\pgfqpoint{1.259002in}{0.711553in}}%
\pgfpathlineto{\pgfqpoint{1.270715in}{0.709653in}}%
\pgfpathlineto{\pgfqpoint{1.282427in}{0.714197in}}%
\pgfpathlineto{\pgfqpoint{1.284378in}{0.715352in}}%
\pgfpathlineto{\pgfqpoint{1.294140in}{0.721424in}}%
\pgfpathlineto{\pgfqpoint{1.301968in}{0.727481in}}%
\pgfpathlineto{\pgfqpoint{1.305853in}{0.730744in}}%
\pgfpathlineto{\pgfqpoint{1.317566in}{0.737290in}}%
\pgfpathlineto{\pgfqpoint{1.328589in}{0.739610in}}%
\pgfpathlineto{\pgfqpoint{1.329278in}{0.739760in}}%
\pgfpathlineto{\pgfqpoint{1.340991in}{0.746374in}}%
\pgfpathlineto{\pgfqpoint{1.342806in}{0.751739in}}%
\pgfpathlineto{\pgfqpoint{1.345181in}{0.763867in}}%
\pgfpathlineto{\pgfqpoint{1.349894in}{0.775996in}}%
\pgfpathlineto{\pgfqpoint{1.352704in}{0.781306in}}%
\pgfpathlineto{\pgfqpoint{1.356667in}{0.788125in}}%
\pgfpathlineto{\pgfqpoint{1.363039in}{0.800254in}}%
\pgfpathlineto{\pgfqpoint{1.364416in}{0.803528in}}%
\pgfpathlineto{\pgfqpoint{1.368002in}{0.812383in}}%
\pgfpathlineto{\pgfqpoint{1.375721in}{0.824512in}}%
\pgfpathlineto{\pgfqpoint{1.376129in}{0.825176in}}%
\pgfpathlineto{\pgfqpoint{1.382678in}{0.836641in}}%
\pgfpathlineto{\pgfqpoint{1.387842in}{0.843510in}}%
\pgfpathlineto{\pgfqpoint{1.391745in}{0.848770in}}%
\pgfpathlineto{\pgfqpoint{1.399554in}{0.858834in}}%
\pgfpathlineto{\pgfqpoint{1.401563in}{0.860898in}}%
\pgfpathlineto{\pgfqpoint{1.411267in}{0.869401in}}%
\pgfpathlineto{\pgfqpoint{1.418761in}{0.873027in}}%
\pgfpathlineto{\pgfqpoint{1.422980in}{0.875295in}}%
\pgfpathlineto{\pgfqpoint{1.434692in}{0.879896in}}%
\pgfpathlineto{\pgfqpoint{1.446405in}{0.884395in}}%
\pgfpathlineto{\pgfqpoint{1.446405in}{0.885156in}}%
\pgfpathlineto{\pgfqpoint{1.446405in}{0.896872in}}%
\pgfpathlineto{\pgfqpoint{1.434692in}{0.892476in}}%
\pgfpathlineto{\pgfqpoint{1.422980in}{0.888777in}}%
\pgfpathlineto{\pgfqpoint{1.415999in}{0.885156in}}%
\pgfpathlineto{\pgfqpoint{1.411267in}{0.882736in}}%
\pgfpathlineto{\pgfqpoint{1.399554in}{0.875371in}}%
\pgfpathlineto{\pgfqpoint{1.397801in}{0.873027in}}%
\pgfpathlineto{\pgfqpoint{1.389198in}{0.860898in}}%
\pgfpathlineto{\pgfqpoint{1.387842in}{0.859099in}}%
\pgfpathlineto{\pgfqpoint{1.379670in}{0.848770in}}%
\pgfpathlineto{\pgfqpoint{1.376129in}{0.843834in}}%
\pgfpathlineto{\pgfqpoint{1.371279in}{0.836641in}}%
\pgfpathlineto{\pgfqpoint{1.364416in}{0.825387in}}%
\pgfpathlineto{\pgfqpoint{1.363825in}{0.824512in}}%
\pgfpathlineto{\pgfqpoint{1.356366in}{0.812383in}}%
\pgfpathlineto{\pgfqpoint{1.352704in}{0.804343in}}%
\pgfpathlineto{\pgfqpoint{1.350649in}{0.800254in}}%
\pgfpathlineto{\pgfqpoint{1.344248in}{0.788125in}}%
\pgfpathlineto{\pgfqpoint{1.340991in}{0.780746in}}%
\pgfpathlineto{\pgfqpoint{1.338916in}{0.775996in}}%
\pgfpathlineto{\pgfqpoint{1.333142in}{0.763867in}}%
\pgfpathlineto{\pgfqpoint{1.329278in}{0.758530in}}%
\pgfpathlineto{\pgfqpoint{1.317566in}{0.755488in}}%
\pgfpathlineto{\pgfqpoint{1.309926in}{0.751739in}}%
\pgfpathlineto{\pgfqpoint{1.305853in}{0.749674in}}%
\pgfpathlineto{\pgfqpoint{1.294140in}{0.740201in}}%
\pgfpathlineto{\pgfqpoint{1.293299in}{0.739610in}}%
\pgfpathlineto{\pgfqpoint{1.282427in}{0.732141in}}%
\pgfpathlineto{\pgfqpoint{1.270715in}{0.727948in}}%
\pgfpathlineto{\pgfqpoint{1.259002in}{0.733065in}}%
\pgfpathlineto{\pgfqpoint{1.249214in}{0.739610in}}%
\pgfpathlineto{\pgfqpoint{1.247289in}{0.740858in}}%
\pgfpathlineto{\pgfqpoint{1.235577in}{0.747102in}}%
\pgfpathlineto{\pgfqpoint{1.223864in}{0.750099in}}%
\pgfpathlineto{\pgfqpoint{1.216436in}{0.751739in}}%
\pgfpathlineto{\pgfqpoint{1.212151in}{0.753136in}}%
\pgfpathlineto{\pgfqpoint{1.200439in}{0.760134in}}%
\pgfpathlineto{\pgfqpoint{1.194937in}{0.763867in}}%
\pgfpathlineto{\pgfqpoint{1.188726in}{0.771975in}}%
\pgfpathlineto{\pgfqpoint{1.186598in}{0.775996in}}%
\pgfpathlineto{\pgfqpoint{1.183992in}{0.788125in}}%
\pgfpathlineto{\pgfqpoint{1.183274in}{0.800254in}}%
\pgfpathlineto{\pgfqpoint{1.188591in}{0.812383in}}%
\pgfpathlineto{\pgfqpoint{1.188726in}{0.812553in}}%
\pgfpathlineto{\pgfqpoint{1.200439in}{0.823307in}}%
\pgfpathlineto{\pgfqpoint{1.202042in}{0.824512in}}%
\pgfpathlineto{\pgfqpoint{1.212151in}{0.831805in}}%
\pgfpathlineto{\pgfqpoint{1.220297in}{0.836641in}}%
\pgfpathlineto{\pgfqpoint{1.223864in}{0.838613in}}%
\pgfpathlineto{\pgfqpoint{1.235577in}{0.844115in}}%
\pgfpathlineto{\pgfqpoint{1.241723in}{0.848770in}}%
\pgfpathlineto{\pgfqpoint{1.247289in}{0.852299in}}%
\pgfpathlineto{\pgfqpoint{1.258902in}{0.860898in}}%
\pgfpathlineto{\pgfqpoint{1.259002in}{0.861082in}}%
\pgfpathlineto{\pgfqpoint{1.262367in}{0.873027in}}%
\pgfpathlineto{\pgfqpoint{1.259978in}{0.885156in}}%
\pgfpathlineto{\pgfqpoint{1.259002in}{0.890492in}}%
\pgfpathlineto{\pgfqpoint{1.257732in}{0.897285in}}%
\pgfpathlineto{\pgfqpoint{1.254410in}{0.909414in}}%
\pgfpathlineto{\pgfqpoint{1.249510in}{0.921543in}}%
\pgfpathlineto{\pgfqpoint{1.247289in}{0.930987in}}%
\pgfpathlineto{\pgfqpoint{1.246553in}{0.933672in}}%
\pgfpathlineto{\pgfqpoint{1.243245in}{0.945801in}}%
\pgfpathlineto{\pgfqpoint{1.235577in}{0.952803in}}%
\pgfpathlineto{\pgfqpoint{1.223864in}{0.956224in}}%
\pgfpathlineto{\pgfqpoint{1.212151in}{0.955315in}}%
\pgfpathlineto{\pgfqpoint{1.200439in}{0.953716in}}%
\pgfpathlineto{\pgfqpoint{1.188726in}{0.951123in}}%
\pgfpathlineto{\pgfqpoint{1.177013in}{0.950591in}}%
\pgfpathlineto{\pgfqpoint{1.165301in}{0.955986in}}%
\pgfpathlineto{\pgfqpoint{1.158547in}{0.957929in}}%
\pgfpathlineto{\pgfqpoint{1.153588in}{0.959341in}}%
\pgfpathlineto{\pgfqpoint{1.141875in}{0.960674in}}%
\pgfpathlineto{\pgfqpoint{1.136313in}{0.957929in}}%
\pgfpathlineto{\pgfqpoint{1.130163in}{0.954790in}}%
\pgfpathlineto{\pgfqpoint{1.118450in}{0.948822in}}%
\pgfpathlineto{\pgfqpoint{1.110818in}{0.945801in}}%
\pgfpathlineto{\pgfqpoint{1.106737in}{0.941858in}}%
\pgfpathlineto{\pgfqpoint{1.095024in}{0.936867in}}%
\pgfpathlineto{\pgfqpoint{1.092493in}{0.945801in}}%
\pgfpathlineto{\pgfqpoint{1.083312in}{0.949151in}}%
\pgfpathlineto{\pgfqpoint{1.071599in}{0.951214in}}%
\pgfpathlineto{\pgfqpoint{1.064060in}{0.957929in}}%
\pgfpathlineto{\pgfqpoint{1.062520in}{0.970058in}}%
\pgfpathlineto{\pgfqpoint{1.065070in}{0.982187in}}%
\pgfpathlineto{\pgfqpoint{1.070144in}{0.994316in}}%
\pgfpathlineto{\pgfqpoint{1.071599in}{0.996858in}}%
\pgfpathlineto{\pgfqpoint{1.077325in}{1.006445in}}%
\pgfpathlineto{\pgfqpoint{1.083312in}{1.015166in}}%
\pgfpathlineto{\pgfqpoint{1.087022in}{1.018574in}}%
\pgfpathlineto{\pgfqpoint{1.095024in}{1.027932in}}%
\pgfpathlineto{\pgfqpoint{1.097810in}{1.030703in}}%
\pgfpathlineto{\pgfqpoint{1.106737in}{1.042545in}}%
\pgfpathlineto{\pgfqpoint{1.106896in}{1.042832in}}%
\pgfpathlineto{\pgfqpoint{1.111764in}{1.054960in}}%
\pgfpathlineto{\pgfqpoint{1.117636in}{1.067089in}}%
\pgfpathlineto{\pgfqpoint{1.118450in}{1.069176in}}%
\pgfpathlineto{\pgfqpoint{1.120719in}{1.079218in}}%
\pgfpathlineto{\pgfqpoint{1.126457in}{1.091347in}}%
\pgfpathlineto{\pgfqpoint{1.130163in}{1.098032in}}%
\pgfpathlineto{\pgfqpoint{1.133262in}{1.103476in}}%
\pgfpathlineto{\pgfqpoint{1.140888in}{1.115605in}}%
\pgfpathlineto{\pgfqpoint{1.141875in}{1.118031in}}%
\pgfpathlineto{\pgfqpoint{1.145144in}{1.127734in}}%
\pgfpathlineto{\pgfqpoint{1.152195in}{1.139863in}}%
\pgfpathlineto{\pgfqpoint{1.153588in}{1.142156in}}%
\pgfpathlineto{\pgfqpoint{1.160949in}{1.151991in}}%
\pgfpathlineto{\pgfqpoint{1.165301in}{1.158817in}}%
\pgfpathlineto{\pgfqpoint{1.169356in}{1.164120in}}%
\pgfpathlineto{\pgfqpoint{1.177013in}{1.175396in}}%
\pgfpathlineto{\pgfqpoint{1.177631in}{1.176249in}}%
\pgfpathlineto{\pgfqpoint{1.183037in}{1.188378in}}%
\pgfpathlineto{\pgfqpoint{1.187375in}{1.200507in}}%
\pgfpathlineto{\pgfqpoint{1.188726in}{1.204020in}}%
\pgfpathlineto{\pgfqpoint{1.192926in}{1.212636in}}%
\pgfpathlineto{\pgfqpoint{1.200439in}{1.222597in}}%
\pgfpathlineto{\pgfqpoint{1.212151in}{1.222603in}}%
\pgfpathlineto{\pgfqpoint{1.223864in}{1.216384in}}%
\pgfpathlineto{\pgfqpoint{1.228540in}{1.212636in}}%
\pgfpathlineto{\pgfqpoint{1.235577in}{1.204878in}}%
\pgfpathlineto{\pgfqpoint{1.238096in}{1.200507in}}%
\pgfpathlineto{\pgfqpoint{1.242796in}{1.188378in}}%
\pgfpathlineto{\pgfqpoint{1.246286in}{1.176249in}}%
\pgfpathlineto{\pgfqpoint{1.247289in}{1.174117in}}%
\pgfpathlineto{\pgfqpoint{1.259002in}{1.168810in}}%
\pgfpathlineto{\pgfqpoint{1.270715in}{1.164645in}}%
\pgfpathlineto{\pgfqpoint{1.271836in}{1.164120in}}%
\pgfpathlineto{\pgfqpoint{1.282427in}{1.160878in}}%
\pgfpathlineto{\pgfqpoint{1.294140in}{1.160009in}}%
\pgfpathlineto{\pgfqpoint{1.305853in}{1.157741in}}%
\pgfpathlineto{\pgfqpoint{1.313453in}{1.151991in}}%
\pgfpathlineto{\pgfqpoint{1.317566in}{1.147034in}}%
\pgfpathlineto{\pgfqpoint{1.320906in}{1.139863in}}%
\pgfpathlineto{\pgfqpoint{1.324338in}{1.127734in}}%
\pgfpathlineto{\pgfqpoint{1.324150in}{1.115605in}}%
\pgfpathlineto{\pgfqpoint{1.320693in}{1.103476in}}%
\pgfpathlineto{\pgfqpoint{1.317566in}{1.092299in}}%
\pgfpathlineto{\pgfqpoint{1.317278in}{1.091347in}}%
\pgfpathlineto{\pgfqpoint{1.312338in}{1.079218in}}%
\pgfpathlineto{\pgfqpoint{1.307464in}{1.067089in}}%
\pgfpathlineto{\pgfqpoint{1.305853in}{1.062656in}}%
\pgfpathlineto{\pgfqpoint{1.303071in}{1.054960in}}%
\pgfpathlineto{\pgfqpoint{1.298368in}{1.042832in}}%
\pgfpathlineto{\pgfqpoint{1.294140in}{1.032971in}}%
\pgfpathlineto{\pgfqpoint{1.293321in}{1.030703in}}%
\pgfpathlineto{\pgfqpoint{1.289795in}{1.018574in}}%
\pgfpathlineto{\pgfqpoint{1.289601in}{1.006445in}}%
\pgfpathlineto{\pgfqpoint{1.294140in}{0.997378in}}%
\pgfpathlineto{\pgfqpoint{1.301479in}{0.994316in}}%
\pgfpathlineto{\pgfqpoint{1.305853in}{0.992529in}}%
\pgfpathlineto{\pgfqpoint{1.317566in}{0.989792in}}%
\pgfpathlineto{\pgfqpoint{1.323402in}{0.994316in}}%
\pgfpathlineto{\pgfqpoint{1.329278in}{0.999969in}}%
\pgfpathlineto{\pgfqpoint{1.335599in}{1.006445in}}%
\pgfpathlineto{\pgfqpoint{1.340991in}{1.011816in}}%
\pgfpathlineto{\pgfqpoint{1.352704in}{1.017509in}}%
\pgfpathlineto{\pgfqpoint{1.354799in}{1.018574in}}%
\pgfpathlineto{\pgfqpoint{1.364416in}{1.023966in}}%
\pgfpathlineto{\pgfqpoint{1.371472in}{1.030703in}}%
\pgfpathlineto{\pgfqpoint{1.376129in}{1.033871in}}%
\pgfpathlineto{\pgfqpoint{1.384554in}{1.042832in}}%
\pgfpathlineto{\pgfqpoint{1.387842in}{1.045623in}}%
\pgfpathlineto{\pgfqpoint{1.396567in}{1.054960in}}%
\pgfpathlineto{\pgfqpoint{1.399554in}{1.058145in}}%
\pgfpathlineto{\pgfqpoint{1.410138in}{1.067089in}}%
\pgfpathlineto{\pgfqpoint{1.411267in}{1.067814in}}%
\pgfpathlineto{\pgfqpoint{1.422980in}{1.074477in}}%
\pgfpathlineto{\pgfqpoint{1.432842in}{1.079218in}}%
\pgfpathlineto{\pgfqpoint{1.434692in}{1.079941in}}%
\pgfpathlineto{\pgfqpoint{1.446405in}{1.082126in}}%
\pgfpathlineto{\pgfqpoint{1.446405in}{1.091347in}}%
\pgfpathlineto{\pgfqpoint{1.446405in}{1.100178in}}%
\pgfpathlineto{\pgfqpoint{1.434692in}{1.097594in}}%
\pgfpathlineto{\pgfqpoint{1.422980in}{1.093423in}}%
\pgfpathlineto{\pgfqpoint{1.417717in}{1.091347in}}%
\pgfpathlineto{\pgfqpoint{1.411267in}{1.088342in}}%
\pgfpathlineto{\pgfqpoint{1.399554in}{1.084173in}}%
\pgfpathlineto{\pgfqpoint{1.388549in}{1.079218in}}%
\pgfpathlineto{\pgfqpoint{1.387842in}{1.078686in}}%
\pgfpathlineto{\pgfqpoint{1.376129in}{1.068978in}}%
\pgfpathlineto{\pgfqpoint{1.365387in}{1.067089in}}%
\pgfpathlineto{\pgfqpoint{1.364416in}{1.066899in}}%
\pgfpathlineto{\pgfqpoint{1.352704in}{1.066636in}}%
\pgfpathlineto{\pgfqpoint{1.340991in}{1.066954in}}%
\pgfpathlineto{\pgfqpoint{1.340881in}{1.067089in}}%
\pgfpathlineto{\pgfqpoint{1.340991in}{1.067578in}}%
\pgfpathlineto{\pgfqpoint{1.343344in}{1.079218in}}%
\pgfpathlineto{\pgfqpoint{1.340991in}{1.089401in}}%
\pgfpathlineto{\pgfqpoint{1.340664in}{1.091347in}}%
\pgfpathlineto{\pgfqpoint{1.340404in}{1.103476in}}%
\pgfpathlineto{\pgfqpoint{1.340991in}{1.108892in}}%
\pgfpathlineto{\pgfqpoint{1.341777in}{1.115605in}}%
\pgfpathlineto{\pgfqpoint{1.344708in}{1.127734in}}%
\pgfpathlineto{\pgfqpoint{1.341636in}{1.139863in}}%
\pgfpathlineto{\pgfqpoint{1.340991in}{1.140744in}}%
\pgfpathlineto{\pgfqpoint{1.336230in}{1.151991in}}%
\pgfpathlineto{\pgfqpoint{1.330763in}{1.164120in}}%
\pgfpathlineto{\pgfqpoint{1.329278in}{1.166928in}}%
\pgfpathlineto{\pgfqpoint{1.323687in}{1.176249in}}%
\pgfpathlineto{\pgfqpoint{1.317566in}{1.186115in}}%
\pgfpathlineto{\pgfqpoint{1.314807in}{1.188378in}}%
\pgfpathlineto{\pgfqpoint{1.305853in}{1.195878in}}%
\pgfpathlineto{\pgfqpoint{1.297942in}{1.200507in}}%
\pgfpathlineto{\pgfqpoint{1.294140in}{1.202760in}}%
\pgfpathlineto{\pgfqpoint{1.282427in}{1.206565in}}%
\pgfpathlineto{\pgfqpoint{1.270715in}{1.208836in}}%
\pgfpathlineto{\pgfqpoint{1.266463in}{1.212636in}}%
\pgfpathlineto{\pgfqpoint{1.259002in}{1.220098in}}%
\pgfpathlineto{\pgfqpoint{1.255162in}{1.224765in}}%
\pgfpathlineto{\pgfqpoint{1.247289in}{1.233751in}}%
\pgfpathlineto{\pgfqpoint{1.242532in}{1.236894in}}%
\pgfpathlineto{\pgfqpoint{1.235577in}{1.239367in}}%
\pgfpathlineto{\pgfqpoint{1.223864in}{1.246040in}}%
\pgfpathlineto{\pgfqpoint{1.217819in}{1.249022in}}%
\pgfpathlineto{\pgfqpoint{1.212151in}{1.251741in}}%
\pgfpathlineto{\pgfqpoint{1.200439in}{1.257740in}}%
\pgfpathlineto{\pgfqpoint{1.198128in}{1.261151in}}%
\pgfpathlineto{\pgfqpoint{1.193438in}{1.273280in}}%
\pgfpathlineto{\pgfqpoint{1.188726in}{1.278843in}}%
\pgfpathlineto{\pgfqpoint{1.177013in}{1.281137in}}%
\pgfpathlineto{\pgfqpoint{1.168582in}{1.273280in}}%
\pgfpathlineto{\pgfqpoint{1.165301in}{1.267680in}}%
\pgfpathlineto{\pgfqpoint{1.162303in}{1.261151in}}%
\pgfpathlineto{\pgfqpoint{1.161982in}{1.249022in}}%
\pgfpathlineto{\pgfqpoint{1.164958in}{1.236894in}}%
\pgfpathlineto{\pgfqpoint{1.165301in}{1.231945in}}%
\pgfpathlineto{\pgfqpoint{1.165735in}{1.224765in}}%
\pgfpathlineto{\pgfqpoint{1.165785in}{1.212636in}}%
\pgfpathlineto{\pgfqpoint{1.165301in}{1.208906in}}%
\pgfpathlineto{\pgfqpoint{1.163788in}{1.200507in}}%
\pgfpathlineto{\pgfqpoint{1.159213in}{1.188378in}}%
\pgfpathlineto{\pgfqpoint{1.154932in}{1.176249in}}%
\pgfpathlineto{\pgfqpoint{1.153588in}{1.173973in}}%
\pgfpathlineto{\pgfqpoint{1.147548in}{1.164120in}}%
\pgfpathlineto{\pgfqpoint{1.141875in}{1.154539in}}%
\pgfpathlineto{\pgfqpoint{1.139955in}{1.151991in}}%
\pgfpathlineto{\pgfqpoint{1.130483in}{1.139863in}}%
\pgfpathlineto{\pgfqpoint{1.130163in}{1.139416in}}%
\pgfpathlineto{\pgfqpoint{1.122160in}{1.127734in}}%
\pgfpathlineto{\pgfqpoint{1.118450in}{1.120437in}}%
\pgfpathlineto{\pgfqpoint{1.115024in}{1.115605in}}%
\pgfpathlineto{\pgfqpoint{1.106737in}{1.103948in}}%
\pgfpathlineto{\pgfqpoint{1.106390in}{1.103476in}}%
\pgfpathlineto{\pgfqpoint{1.098962in}{1.091347in}}%
\pgfpathlineto{\pgfqpoint{1.095024in}{1.084587in}}%
\pgfpathlineto{\pgfqpoint{1.091710in}{1.079218in}}%
\pgfpathlineto{\pgfqpoint{1.086629in}{1.067089in}}%
\pgfpathlineto{\pgfqpoint{1.083312in}{1.059357in}}%
\pgfpathlineto{\pgfqpoint{1.081391in}{1.054960in}}%
\pgfpathlineto{\pgfqpoint{1.076522in}{1.042832in}}%
\pgfpathlineto{\pgfqpoint{1.071990in}{1.030703in}}%
\pgfpathlineto{\pgfqpoint{1.071599in}{1.030008in}}%
\pgfpathlineto{\pgfqpoint{1.065049in}{1.018574in}}%
\pgfpathlineto{\pgfqpoint{1.059886in}{1.010312in}}%
\pgfpathlineto{\pgfqpoint{1.057627in}{1.006445in}}%
\pgfpathlineto{\pgfqpoint{1.051055in}{0.994316in}}%
\pgfpathlineto{\pgfqpoint{1.048174in}{0.989051in}}%
\pgfpathlineto{\pgfqpoint{1.044506in}{0.982187in}}%
\pgfpathlineto{\pgfqpoint{1.040189in}{0.970058in}}%
\pgfpathlineto{\pgfqpoint{1.036461in}{0.958368in}}%
\pgfpathlineto{\pgfqpoint{1.036295in}{0.957929in}}%
\pgfpathlineto{\pgfqpoint{1.030498in}{0.945801in}}%
\pgfpathlineto{\pgfqpoint{1.024748in}{0.942131in}}%
\pgfpathlineto{\pgfqpoint{1.013036in}{0.936406in}}%
\pgfpathlineto{\pgfqpoint{1.007624in}{0.933672in}}%
\pgfpathlineto{\pgfqpoint{1.001323in}{0.930251in}}%
\pgfpathlineto{\pgfqpoint{0.989610in}{0.924560in}}%
\pgfpathlineto{\pgfqpoint{0.980394in}{0.921543in}}%
\pgfpathlineto{\pgfqpoint{0.977898in}{0.920634in}}%
\pgfpathlineto{\pgfqpoint{0.966185in}{0.920102in}}%
\pgfpathlineto{\pgfqpoint{0.961676in}{0.921543in}}%
\pgfpathlineto{\pgfqpoint{0.954472in}{0.924644in}}%
\pgfpathlineto{\pgfqpoint{0.946417in}{0.933672in}}%
\pgfpathlineto{\pgfqpoint{0.942759in}{0.938318in}}%
\pgfpathlineto{\pgfqpoint{0.936843in}{0.945801in}}%
\pgfpathlineto{\pgfqpoint{0.931047in}{0.953325in}}%
\pgfpathlineto{\pgfqpoint{0.927413in}{0.957929in}}%
\pgfpathlineto{\pgfqpoint{0.920349in}{0.970058in}}%
\pgfpathlineto{\pgfqpoint{0.919334in}{0.972709in}}%
\pgfpathlineto{\pgfqpoint{0.916075in}{0.982187in}}%
\pgfpathlineto{\pgfqpoint{0.914037in}{0.994316in}}%
\pgfpathlineto{\pgfqpoint{0.912148in}{1.006445in}}%
\pgfpathlineto{\pgfqpoint{0.912227in}{1.018574in}}%
\pgfpathlineto{\pgfqpoint{0.919128in}{1.030703in}}%
\pgfpathlineto{\pgfqpoint{0.919334in}{1.031014in}}%
\pgfpathlineto{\pgfqpoint{0.926663in}{1.042832in}}%
\pgfpathlineto{\pgfqpoint{0.931047in}{1.053304in}}%
\pgfpathlineto{\pgfqpoint{0.931693in}{1.054960in}}%
\pgfpathlineto{\pgfqpoint{0.935402in}{1.067089in}}%
\pgfpathlineto{\pgfqpoint{0.939619in}{1.079218in}}%
\pgfpathlineto{\pgfqpoint{0.942759in}{1.084274in}}%
\pgfpathlineto{\pgfqpoint{0.949660in}{1.091347in}}%
\pgfpathlineto{\pgfqpoint{0.954472in}{1.103396in}}%
\pgfpathlineto{\pgfqpoint{0.954499in}{1.103476in}}%
\pgfpathlineto{\pgfqpoint{0.960312in}{1.115605in}}%
\pgfpathlineto{\pgfqpoint{0.966185in}{1.125288in}}%
\pgfpathlineto{\pgfqpoint{0.967628in}{1.127734in}}%
\pgfpathlineto{\pgfqpoint{0.976655in}{1.139863in}}%
\pgfpathlineto{\pgfqpoint{0.977898in}{1.141620in}}%
\pgfpathlineto{\pgfqpoint{0.984600in}{1.151991in}}%
\pgfpathlineto{\pgfqpoint{0.986863in}{1.164120in}}%
\pgfpathlineto{\pgfqpoint{0.989023in}{1.176249in}}%
\pgfpathlineto{\pgfqpoint{0.989610in}{1.181513in}}%
\pgfpathlineto{\pgfqpoint{0.990495in}{1.188378in}}%
\pgfpathlineto{\pgfqpoint{0.991983in}{1.200507in}}%
\pgfpathlineto{\pgfqpoint{0.996073in}{1.212636in}}%
\pgfpathlineto{\pgfqpoint{1.001323in}{1.218567in}}%
\pgfpathlineto{\pgfqpoint{1.006215in}{1.224765in}}%
\pgfpathlineto{\pgfqpoint{1.013036in}{1.231918in}}%
\pgfpathlineto{\pgfqpoint{1.018241in}{1.236894in}}%
\pgfpathlineto{\pgfqpoint{1.024748in}{1.243486in}}%
\pgfpathlineto{\pgfqpoint{1.034715in}{1.249022in}}%
\pgfpathlineto{\pgfqpoint{1.036461in}{1.249598in}}%
\pgfpathlineto{\pgfqpoint{1.048174in}{1.253599in}}%
\pgfpathlineto{\pgfqpoint{1.059886in}{1.257452in}}%
\pgfpathlineto{\pgfqpoint{1.068091in}{1.261151in}}%
\pgfpathlineto{\pgfqpoint{1.071599in}{1.262957in}}%
\pgfpathlineto{\pgfqpoint{1.083312in}{1.271335in}}%
\pgfpathlineto{\pgfqpoint{1.085545in}{1.273280in}}%
\pgfpathlineto{\pgfqpoint{1.095024in}{1.281470in}}%
\pgfpathlineto{\pgfqpoint{1.100065in}{1.285409in}}%
\pgfpathlineto{\pgfqpoint{1.106737in}{1.290845in}}%
\pgfpathlineto{\pgfqpoint{1.113754in}{1.297538in}}%
\pgfpathlineto{\pgfqpoint{1.118450in}{1.302205in}}%
\pgfpathlineto{\pgfqpoint{1.124443in}{1.309667in}}%
\pgfpathlineto{\pgfqpoint{1.128846in}{1.321796in}}%
\pgfpathlineto{\pgfqpoint{1.130163in}{1.325243in}}%
\pgfpathlineto{\pgfqpoint{1.133549in}{1.333925in}}%
\pgfpathlineto{\pgfqpoint{1.139178in}{1.346053in}}%
\pgfpathlineto{\pgfqpoint{1.141875in}{1.354474in}}%
\pgfpathlineto{\pgfqpoint{1.142658in}{1.358182in}}%
\pgfpathlineto{\pgfqpoint{1.146828in}{1.370311in}}%
\pgfpathlineto{\pgfqpoint{1.153588in}{1.381536in}}%
\pgfpathlineto{\pgfqpoint{1.154382in}{1.382440in}}%
\pgfpathlineto{\pgfqpoint{1.165301in}{1.390621in}}%
\pgfpathlineto{\pgfqpoint{1.177013in}{1.392249in}}%
\pgfpathlineto{\pgfqpoint{1.188726in}{1.391438in}}%
\pgfpathlineto{\pgfqpoint{1.200439in}{1.388566in}}%
\pgfpathlineto{\pgfqpoint{1.211426in}{1.382440in}}%
\pgfpathlineto{\pgfqpoint{1.212151in}{1.381690in}}%
\pgfpathlineto{\pgfqpoint{1.223232in}{1.370311in}}%
\pgfpathlineto{\pgfqpoint{1.223864in}{1.369558in}}%
\pgfpathlineto{\pgfqpoint{1.224844in}{1.370311in}}%
\pgfpathlineto{\pgfqpoint{1.229440in}{1.382440in}}%
\pgfpathlineto{\pgfqpoint{1.234043in}{1.394569in}}%
\pgfpathlineto{\pgfqpoint{1.235577in}{1.398299in}}%
\pgfpathlineto{\pgfqpoint{1.237248in}{1.406698in}}%
\pgfpathlineto{\pgfqpoint{1.241774in}{1.418827in}}%
\pgfpathlineto{\pgfqpoint{1.247289in}{1.423977in}}%
\pgfpathlineto{\pgfqpoint{1.259002in}{1.422505in}}%
\pgfpathlineto{\pgfqpoint{1.270715in}{1.424872in}}%
\pgfpathlineto{\pgfqpoint{1.281276in}{1.430956in}}%
\pgfpathlineto{\pgfqpoint{1.282427in}{1.431733in}}%
\pgfpathlineto{\pgfqpoint{1.294140in}{1.436259in}}%
\pgfpathlineto{\pgfqpoint{1.305177in}{1.443084in}}%
\pgfpathlineto{\pgfqpoint{1.305853in}{1.443645in}}%
\pgfpathlineto{\pgfqpoint{1.316853in}{1.455213in}}%
\pgfpathlineto{\pgfqpoint{1.317566in}{1.456016in}}%
\pgfpathlineto{\pgfqpoint{1.327503in}{1.467342in}}%
\pgfpathlineto{\pgfqpoint{1.329278in}{1.469372in}}%
\pgfpathlineto{\pgfqpoint{1.338287in}{1.479471in}}%
\pgfpathlineto{\pgfqpoint{1.340991in}{1.482734in}}%
\pgfpathlineto{\pgfqpoint{1.348654in}{1.491600in}}%
\pgfpathlineto{\pgfqpoint{1.340991in}{1.491600in}}%
\pgfpathlineto{\pgfqpoint{1.329278in}{1.491600in}}%
\pgfpathlineto{\pgfqpoint{1.327980in}{1.491600in}}%
\pgfpathlineto{\pgfqpoint{1.318163in}{1.479471in}}%
\pgfpathlineto{\pgfqpoint{1.317566in}{1.478800in}}%
\pgfpathlineto{\pgfqpoint{1.307795in}{1.467342in}}%
\pgfpathlineto{\pgfqpoint{1.305853in}{1.465193in}}%
\pgfpathlineto{\pgfqpoint{1.296500in}{1.455213in}}%
\pgfpathlineto{\pgfqpoint{1.294140in}{1.453111in}}%
\pgfpathlineto{\pgfqpoint{1.285072in}{1.443084in}}%
\pgfpathlineto{\pgfqpoint{1.282427in}{1.441369in}}%
\pgfpathlineto{\pgfqpoint{1.270715in}{1.433427in}}%
\pgfpathlineto{\pgfqpoint{1.259002in}{1.433868in}}%
\pgfpathlineto{\pgfqpoint{1.247289in}{1.439342in}}%
\pgfpathlineto{\pgfqpoint{1.235577in}{1.438996in}}%
\pgfpathlineto{\pgfqpoint{1.227913in}{1.430956in}}%
\pgfpathlineto{\pgfqpoint{1.223864in}{1.422812in}}%
\pgfpathlineto{\pgfqpoint{1.219994in}{1.418827in}}%
\pgfpathlineto{\pgfqpoint{1.212151in}{1.411802in}}%
\pgfpathlineto{\pgfqpoint{1.200439in}{1.408565in}}%
\pgfpathlineto{\pgfqpoint{1.188726in}{1.410039in}}%
\pgfpathlineto{\pgfqpoint{1.177013in}{1.409190in}}%
\pgfpathlineto{\pgfqpoint{1.165301in}{1.408921in}}%
\pgfpathlineto{\pgfqpoint{1.160129in}{1.406698in}}%
\pgfpathlineto{\pgfqpoint{1.153588in}{1.403594in}}%
\pgfpathlineto{\pgfqpoint{1.142870in}{1.394569in}}%
\pgfpathlineto{\pgfqpoint{1.141875in}{1.393412in}}%
\pgfpathlineto{\pgfqpoint{1.133554in}{1.382440in}}%
\pgfpathlineto{\pgfqpoint{1.130163in}{1.374095in}}%
\pgfpathlineto{\pgfqpoint{1.127951in}{1.370311in}}%
\pgfpathlineto{\pgfqpoint{1.119319in}{1.358182in}}%
\pgfpathlineto{\pgfqpoint{1.118450in}{1.356924in}}%
\pgfpathlineto{\pgfqpoint{1.109846in}{1.346053in}}%
\pgfpathlineto{\pgfqpoint{1.106737in}{1.340251in}}%
\pgfpathlineto{\pgfqpoint{1.099298in}{1.333925in}}%
\pgfpathlineto{\pgfqpoint{1.095024in}{1.330710in}}%
\pgfpathlineto{\pgfqpoint{1.083312in}{1.327872in}}%
\pgfpathlineto{\pgfqpoint{1.071599in}{1.325753in}}%
\pgfpathlineto{\pgfqpoint{1.059886in}{1.322489in}}%
\pgfpathlineto{\pgfqpoint{1.057112in}{1.321796in}}%
\pgfpathlineto{\pgfqpoint{1.048174in}{1.319274in}}%
\pgfpathlineto{\pgfqpoint{1.036461in}{1.316748in}}%
\pgfpathlineto{\pgfqpoint{1.024748in}{1.316835in}}%
\pgfpathlineto{\pgfqpoint{1.013036in}{1.317745in}}%
\pgfpathlineto{\pgfqpoint{1.001323in}{1.320460in}}%
\pgfpathlineto{\pgfqpoint{1.000331in}{1.321796in}}%
\pgfpathlineto{\pgfqpoint{1.001323in}{1.327848in}}%
\pgfpathlineto{\pgfqpoint{1.002158in}{1.333925in}}%
\pgfpathlineto{\pgfqpoint{1.004942in}{1.346053in}}%
\pgfpathlineto{\pgfqpoint{1.008036in}{1.358182in}}%
\pgfpathlineto{\pgfqpoint{1.011795in}{1.370311in}}%
\pgfpathlineto{\pgfqpoint{1.013036in}{1.372781in}}%
\pgfpathlineto{\pgfqpoint{1.017264in}{1.382440in}}%
\pgfpathlineto{\pgfqpoint{1.022410in}{1.394569in}}%
\pgfpathlineto{\pgfqpoint{1.024748in}{1.405539in}}%
\pgfpathlineto{\pgfqpoint{1.024961in}{1.406698in}}%
\pgfpathlineto{\pgfqpoint{1.024748in}{1.409555in}}%
\pgfpathlineto{\pgfqpoint{1.023973in}{1.418827in}}%
\pgfpathlineto{\pgfqpoint{1.024491in}{1.430956in}}%
\pgfpathlineto{\pgfqpoint{1.023668in}{1.443084in}}%
\pgfpathlineto{\pgfqpoint{1.022195in}{1.455213in}}%
\pgfpathlineto{\pgfqpoint{1.020635in}{1.467342in}}%
\pgfpathlineto{\pgfqpoint{1.019825in}{1.479471in}}%
\pgfpathlineto{\pgfqpoint{1.019842in}{1.491600in}}%
\pgfpathlineto{\pgfqpoint{1.013036in}{1.491600in}}%
\pgfpathlineto{\pgfqpoint{1.001323in}{1.491600in}}%
\pgfpathlineto{\pgfqpoint{0.989610in}{1.491600in}}%
\pgfpathlineto{\pgfqpoint{0.977898in}{1.491600in}}%
\pgfpathlineto{\pgfqpoint{0.976321in}{1.491600in}}%
\pgfpathlineto{\pgfqpoint{0.974933in}{1.479471in}}%
\pgfpathlineto{\pgfqpoint{0.974174in}{1.467342in}}%
\pgfpathlineto{\pgfqpoint{0.974227in}{1.455213in}}%
\pgfpathlineto{\pgfqpoint{0.971574in}{1.443084in}}%
\pgfpathlineto{\pgfqpoint{0.967601in}{1.430956in}}%
\pgfpathlineto{\pgfqpoint{0.966185in}{1.428010in}}%
\pgfpathlineto{\pgfqpoint{0.962105in}{1.418827in}}%
\pgfpathlineto{\pgfqpoint{0.957673in}{1.406698in}}%
\pgfpathlineto{\pgfqpoint{0.954472in}{1.397555in}}%
\pgfpathlineto{\pgfqpoint{0.953475in}{1.394569in}}%
\pgfpathlineto{\pgfqpoint{0.950277in}{1.382440in}}%
\pgfpathlineto{\pgfqpoint{0.944127in}{1.370311in}}%
\pgfpathlineto{\pgfqpoint{0.942759in}{1.367508in}}%
\pgfpathlineto{\pgfqpoint{0.938515in}{1.358182in}}%
\pgfpathlineto{\pgfqpoint{0.932967in}{1.346053in}}%
\pgfpathlineto{\pgfqpoint{0.931047in}{1.342665in}}%
\pgfpathlineto{\pgfqpoint{0.925830in}{1.333925in}}%
\pgfpathlineto{\pgfqpoint{0.919334in}{1.322724in}}%
\pgfpathlineto{\pgfqpoint{0.918742in}{1.321796in}}%
\pgfpathlineto{\pgfqpoint{0.910895in}{1.309667in}}%
\pgfpathlineto{\pgfqpoint{0.907621in}{1.304389in}}%
\pgfpathlineto{\pgfqpoint{0.902989in}{1.297538in}}%
\pgfpathlineto{\pgfqpoint{0.896983in}{1.285409in}}%
\pgfpathlineto{\pgfqpoint{0.895909in}{1.280044in}}%
\pgfpathlineto{\pgfqpoint{0.894758in}{1.273280in}}%
\pgfpathlineto{\pgfqpoint{0.891017in}{1.261151in}}%
\pgfpathlineto{\pgfqpoint{0.884586in}{1.249022in}}%
\pgfpathlineto{\pgfqpoint{0.884196in}{1.248539in}}%
\pgfpathlineto{\pgfqpoint{0.874705in}{1.236894in}}%
\pgfpathlineto{\pgfqpoint{0.872483in}{1.235106in}}%
\pgfpathlineto{\pgfqpoint{0.860771in}{1.228099in}}%
\pgfpathlineto{\pgfqpoint{0.852817in}{1.224765in}}%
\pgfpathlineto{\pgfqpoint{0.849058in}{1.223375in}}%
\pgfpathlineto{\pgfqpoint{0.837345in}{1.220850in}}%
\pgfpathlineto{\pgfqpoint{0.825633in}{1.220510in}}%
\pgfpathlineto{\pgfqpoint{0.813920in}{1.220709in}}%
\pgfpathlineto{\pgfqpoint{0.802207in}{1.220687in}}%
\pgfpathlineto{\pgfqpoint{0.790495in}{1.220072in}}%
\pgfpathlineto{\pgfqpoint{0.778782in}{1.218367in}}%
\pgfpathlineto{\pgfqpoint{0.767069in}{1.216361in}}%
\pgfpathlineto{\pgfqpoint{0.755356in}{1.214428in}}%
\pgfpathlineto{\pgfqpoint{0.743644in}{1.212796in}}%
\pgfpathlineto{\pgfqpoint{0.742353in}{1.212636in}}%
\pgfpathlineto{\pgfqpoint{0.731931in}{1.211653in}}%
\pgfpathlineto{\pgfqpoint{0.720218in}{1.210629in}}%
\pgfpathlineto{\pgfqpoint{0.708506in}{1.209611in}}%
\pgfpathlineto{\pgfqpoint{0.696793in}{1.208417in}}%
\pgfpathlineto{\pgfqpoint{0.685080in}{1.207123in}}%
\pgfpathlineto{\pgfqpoint{0.673368in}{1.205767in}}%
\pgfpathlineto{\pgfqpoint{0.661655in}{1.204169in}}%
\pgfpathlineto{\pgfqpoint{0.649942in}{1.201930in}}%
\pgfpathlineto{\pgfqpoint{0.642702in}{1.200507in}}%
\pgfpathlineto{\pgfqpoint{0.638230in}{1.199676in}}%
\pgfpathlineto{\pgfqpoint{0.626517in}{1.197597in}}%
\pgfpathlineto{\pgfqpoint{0.614804in}{1.195815in}}%
\pgfpathlineto{\pgfqpoint{0.603091in}{1.193932in}}%
\pgfpathlineto{\pgfqpoint{0.591379in}{1.192346in}}%
\pgfpathlineto{\pgfqpoint{0.579666in}{1.191408in}}%
\pgfpathlineto{\pgfqpoint{0.567953in}{1.190857in}}%
\pgfpathlineto{\pgfqpoint{0.556241in}{1.190492in}}%
\pgfpathlineto{\pgfqpoint{0.544528in}{1.190611in}}%
\pgfpathlineto{\pgfqpoint{0.532815in}{1.191188in}}%
\pgfpathlineto{\pgfqpoint{0.521103in}{1.192147in}}%
\pgfpathlineto{\pgfqpoint{0.509390in}{1.193754in}}%
\pgfpathlineto{\pgfqpoint{0.497677in}{1.195551in}}%
\pgfpathlineto{\pgfqpoint{0.485965in}{1.197899in}}%
\pgfpathlineto{\pgfqpoint{0.474870in}{1.200507in}}%
\pgfpathlineto{\pgfqpoint{0.474252in}{1.200661in}}%
\pgfpathlineto{\pgfqpoint{0.462539in}{1.203693in}}%
\pgfpathlineto{\pgfqpoint{0.450827in}{1.206972in}}%
\pgfpathlineto{\pgfqpoint{0.439114in}{1.210885in}}%
\pgfpathlineto{\pgfqpoint{0.434715in}{1.212636in}}%
\pgfpathlineto{\pgfqpoint{0.427401in}{1.215928in}}%
\pgfpathlineto{\pgfqpoint{0.415688in}{1.221322in}}%
\pgfpathlineto{\pgfqpoint{0.408860in}{1.224765in}}%
\pgfpathlineto{\pgfqpoint{0.403976in}{1.228308in}}%
\pgfpathlineto{\pgfqpoint{0.395661in}{1.236894in}}%
\pgfpathlineto{\pgfqpoint{0.395783in}{1.249022in}}%
\pgfpathlineto{\pgfqpoint{0.398871in}{1.261151in}}%
\pgfpathlineto{\pgfqpoint{0.403976in}{1.270907in}}%
\pgfpathlineto{\pgfqpoint{0.405453in}{1.273280in}}%
\pgfpathlineto{\pgfqpoint{0.414731in}{1.285409in}}%
\pgfpathlineto{\pgfqpoint{0.415688in}{1.286584in}}%
\pgfpathlineto{\pgfqpoint{0.427401in}{1.297500in}}%
\pgfpathlineto{\pgfqpoint{0.427446in}{1.297538in}}%
\pgfpathlineto{\pgfqpoint{0.439114in}{1.308135in}}%
\pgfpathlineto{\pgfqpoint{0.440925in}{1.309667in}}%
\pgfpathlineto{\pgfqpoint{0.450827in}{1.317513in}}%
\pgfpathlineto{\pgfqpoint{0.457058in}{1.321796in}}%
\pgfpathlineto{\pgfqpoint{0.462539in}{1.325549in}}%
\pgfpathlineto{\pgfqpoint{0.474252in}{1.333233in}}%
\pgfpathlineto{\pgfqpoint{0.475278in}{1.333925in}}%
\pgfpathlineto{\pgfqpoint{0.485965in}{1.341949in}}%
\pgfpathlineto{\pgfqpoint{0.491334in}{1.346053in}}%
\pgfpathlineto{\pgfqpoint{0.497677in}{1.351551in}}%
\pgfpathlineto{\pgfqpoint{0.505162in}{1.358182in}}%
\pgfpathlineto{\pgfqpoint{0.509390in}{1.363792in}}%
\pgfpathlineto{\pgfqpoint{0.513886in}{1.370311in}}%
\pgfpathlineto{\pgfqpoint{0.519665in}{1.382440in}}%
\pgfpathlineto{\pgfqpoint{0.521103in}{1.386015in}}%
\pgfpathlineto{\pgfqpoint{0.524769in}{1.394569in}}%
\pgfpathlineto{\pgfqpoint{0.530018in}{1.406698in}}%
\pgfpathlineto{\pgfqpoint{0.532026in}{1.418827in}}%
\pgfpathlineto{\pgfqpoint{0.527018in}{1.430956in}}%
\pgfpathlineto{\pgfqpoint{0.521103in}{1.441874in}}%
\pgfpathlineto{\pgfqpoint{0.520454in}{1.443084in}}%
\pgfpathlineto{\pgfqpoint{0.512762in}{1.455213in}}%
\pgfpathlineto{\pgfqpoint{0.509390in}{1.459949in}}%
\pgfpathlineto{\pgfqpoint{0.504128in}{1.467342in}}%
\pgfpathlineto{\pgfqpoint{0.497677in}{1.475596in}}%
\pgfpathlineto{\pgfqpoint{0.494664in}{1.479471in}}%
\pgfpathlineto{\pgfqpoint{0.485965in}{1.489920in}}%
\pgfpathlineto{\pgfqpoint{0.484559in}{1.491600in}}%
\pgfpathlineto{\pgfqpoint{0.474252in}{1.491600in}}%
\pgfpathlineto{\pgfqpoint{0.462539in}{1.491600in}}%
\pgfpathlineto{\pgfqpoint{0.450827in}{1.491600in}}%
\pgfpathlineto{\pgfqpoint{0.439114in}{1.491600in}}%
\pgfpathlineto{\pgfqpoint{0.427401in}{1.491600in}}%
\pgfpathlineto{\pgfqpoint{0.415688in}{1.491600in}}%
\pgfpathlineto{\pgfqpoint{0.403976in}{1.491600in}}%
\pgfpathlineto{\pgfqpoint{0.392263in}{1.491600in}}%
\pgfpathlineto{\pgfqpoint{0.380550in}{1.491600in}}%
\pgfpathlineto{\pgfqpoint{0.368838in}{1.491600in}}%
\pgfpathlineto{\pgfqpoint{0.364259in}{1.491600in}}%
\pgfpathlineto{\pgfqpoint{0.368838in}{1.480136in}}%
\pgfpathlineto{\pgfqpoint{0.369102in}{1.479471in}}%
\pgfpathlineto{\pgfqpoint{0.372446in}{1.467342in}}%
\pgfpathlineto{\pgfqpoint{0.374285in}{1.455213in}}%
\pgfpathlineto{\pgfqpoint{0.373388in}{1.443084in}}%
\pgfpathlineto{\pgfqpoint{0.370522in}{1.430956in}}%
\pgfpathlineto{\pgfqpoint{0.368838in}{1.426774in}}%
\pgfpathlineto{\pgfqpoint{0.365391in}{1.418827in}}%
\pgfpathlineto{\pgfqpoint{0.359276in}{1.406698in}}%
\pgfpathlineto{\pgfqpoint{0.357125in}{1.401913in}}%
\pgfpathlineto{\pgfqpoint{0.354707in}{1.394569in}}%
\pgfpathlineto{\pgfqpoint{0.351972in}{1.382440in}}%
\pgfpathlineto{\pgfqpoint{0.349843in}{1.370311in}}%
\pgfpathlineto{\pgfqpoint{0.346329in}{1.358182in}}%
\pgfpathlineto{\pgfqpoint{0.345412in}{1.355774in}}%
\pgfpathlineto{\pgfqpoint{0.341528in}{1.346053in}}%
\pgfpathlineto{\pgfqpoint{0.336227in}{1.333925in}}%
\pgfpathlineto{\pgfqpoint{0.333700in}{1.326952in}}%
\pgfpathlineto{\pgfqpoint{0.331921in}{1.321796in}}%
\pgfpathlineto{\pgfqpoint{0.327766in}{1.309667in}}%
\pgfpathlineto{\pgfqpoint{0.325776in}{1.297538in}}%
\pgfpathlineto{\pgfqpoint{0.324393in}{1.285409in}}%
\pgfpathlineto{\pgfqpoint{0.323262in}{1.273280in}}%
\pgfpathlineto{\pgfqpoint{0.321987in}{1.266469in}}%
\pgfpathlineto{\pgfqpoint{0.320321in}{1.261151in}}%
\pgfpathlineto{\pgfqpoint{0.310274in}{1.249061in}}%
\pgfpathlineto{\pgfqpoint{0.310216in}{1.249022in}}%
\pgfpathlineto{\pgfqpoint{0.298562in}{1.244800in}}%
\pgfpathlineto{\pgfqpoint{0.286849in}{1.244257in}}%
\pgfpathlineto{\pgfqpoint{0.286849in}{1.236894in}}%
\pgfpathlineto{\pgfqpoint{0.286849in}{1.224765in}}%
\pgfpathlineto{\pgfqpoint{0.286849in}{1.212636in}}%
\pgfpathlineto{\pgfqpoint{0.286849in}{1.200507in}}%
\pgfpathlineto{\pgfqpoint{0.286849in}{1.200270in}}%
\pgfpathlineto{\pgfqpoint{0.298508in}{1.188378in}}%
\pgfpathlineto{\pgfqpoint{0.298562in}{1.188332in}}%
\pgfpathlineto{\pgfqpoint{0.310274in}{1.177966in}}%
\pgfpathlineto{\pgfqpoint{0.312042in}{1.176249in}}%
\pgfpathlineto{\pgfqpoint{0.321987in}{1.167244in}}%
\pgfpathlineto{\pgfqpoint{0.325158in}{1.164120in}}%
\pgfpathlineto{\pgfqpoint{0.333700in}{1.156517in}}%
\pgfpathlineto{\pgfqpoint{0.338477in}{1.151991in}}%
\pgfpathlineto{\pgfqpoint{0.345412in}{1.146386in}}%
\pgfpathlineto{\pgfqpoint{0.353191in}{1.139863in}}%
\pgfpathlineto{\pgfqpoint{0.357125in}{1.137094in}}%
\pgfpathlineto{\pgfqpoint{0.368838in}{1.128645in}}%
\pgfpathlineto{\pgfqpoint{0.369971in}{1.127734in}}%
\pgfpathlineto{\pgfqpoint{0.380550in}{1.119805in}}%
\pgfpathlineto{\pgfqpoint{0.385889in}{1.115605in}}%
\pgfpathlineto{\pgfqpoint{0.392263in}{1.110832in}}%
\pgfpathlineto{\pgfqpoint{0.401743in}{1.103476in}}%
\pgfpathlineto{\pgfqpoint{0.403976in}{1.101729in}}%
\pgfpathlineto{\pgfqpoint{0.415688in}{1.091956in}}%
\pgfpathlineto{\pgfqpoint{0.416373in}{1.091347in}}%
\pgfpathlineto{\pgfqpoint{0.427401in}{1.081616in}}%
\pgfpathlineto{\pgfqpoint{0.430024in}{1.079218in}}%
\pgfpathlineto{\pgfqpoint{0.439114in}{1.071039in}}%
\pgfpathlineto{\pgfqpoint{0.443349in}{1.067089in}}%
\pgfpathlineto{\pgfqpoint{0.450827in}{1.059880in}}%
\pgfpathlineto{\pgfqpoint{0.455919in}{1.054960in}}%
\pgfpathlineto{\pgfqpoint{0.462539in}{1.046637in}}%
\pgfpathlineto{\pgfqpoint{0.465561in}{1.042832in}}%
\pgfpathlineto{\pgfqpoint{0.474252in}{1.031386in}}%
\pgfpathlineto{\pgfqpoint{0.474757in}{1.030703in}}%
\pgfpathlineto{\pgfqpoint{0.483436in}{1.018574in}}%
\pgfpathlineto{\pgfqpoint{0.485965in}{1.015197in}}%
\pgfpathlineto{\pgfqpoint{0.492614in}{1.006445in}}%
\pgfpathlineto{\pgfqpoint{0.497677in}{0.999941in}}%
\pgfpathlineto{\pgfqpoint{0.502232in}{0.994316in}}%
\pgfpathlineto{\pgfqpoint{0.509390in}{0.984303in}}%
\pgfpathlineto{\pgfqpoint{0.510852in}{0.982187in}}%
\pgfpathlineto{\pgfqpoint{0.517858in}{0.970058in}}%
\pgfpathlineto{\pgfqpoint{0.521103in}{0.961907in}}%
\pgfpathlineto{\pgfqpoint{0.522654in}{0.957929in}}%
\pgfpathlineto{\pgfqpoint{0.525323in}{0.945801in}}%
\pgfpathlineto{\pgfqpoint{0.527105in}{0.933672in}}%
\pgfpathlineto{\pgfqpoint{0.532815in}{0.925694in}}%
\pgfpathlineto{\pgfqpoint{0.537573in}{0.921543in}}%
\pgfpathlineto{\pgfqpoint{0.544528in}{0.916439in}}%
\pgfpathlineto{\pgfqpoint{0.554312in}{0.909414in}}%
\pgfpathlineto{\pgfqpoint{0.556241in}{0.908135in}}%
\pgfpathlineto{\pgfqpoint{0.567953in}{0.897359in}}%
\pgfpathlineto{\pgfqpoint{0.568017in}{0.897285in}}%
\pgfpathlineto{\pgfqpoint{0.578277in}{0.885156in}}%
\pgfpathlineto{\pgfqpoint{0.579666in}{0.883222in}}%
\pgfpathlineto{\pgfqpoint{0.586441in}{0.873027in}}%
\pgfpathlineto{\pgfqpoint{0.591379in}{0.866644in}}%
\pgfpathlineto{\pgfqpoint{0.595959in}{0.860898in}}%
\pgfpathlineto{\pgfqpoint{0.603091in}{0.851720in}}%
\pgfpathlineto{\pgfqpoint{0.605331in}{0.848770in}}%
\pgfpathlineto{\pgfqpoint{0.614804in}{0.837881in}}%
\pgfpathlineto{\pgfqpoint{0.615889in}{0.836641in}}%
\pgfpathlineto{\pgfqpoint{0.626517in}{0.826383in}}%
\pgfpathlineto{\pgfqpoint{0.628579in}{0.824512in}}%
\pgfpathlineto{\pgfqpoint{0.638230in}{0.819608in}}%
\pgfpathlineto{\pgfqpoint{0.649942in}{0.816159in}}%
\pgfpathlineto{\pgfqpoint{0.660598in}{0.812383in}}%
\pgfpathlineto{\pgfqpoint{0.661655in}{0.812093in}}%
\pgfpathlineto{\pgfqpoint{0.673368in}{0.808914in}}%
\pgfpathlineto{\pgfqpoint{0.685080in}{0.805532in}}%
\pgfpathlineto{\pgfqpoint{0.696793in}{0.801709in}}%
\pgfpathlineto{\pgfqpoint{0.702187in}{0.800254in}}%
\pgfpathlineto{\pgfqpoint{0.708506in}{0.798865in}}%
\pgfpathlineto{\pgfqpoint{0.720218in}{0.796194in}}%
\pgfpathlineto{\pgfqpoint{0.731931in}{0.793378in}}%
\pgfpathlineto{\pgfqpoint{0.743644in}{0.790346in}}%
\pgfpathlineto{\pgfqpoint{0.752448in}{0.788125in}}%
\pgfpathlineto{\pgfqpoint{0.755356in}{0.787191in}}%
\pgfpathlineto{\pgfqpoint{0.767069in}{0.783430in}}%
\pgfpathlineto{\pgfqpoint{0.778782in}{0.779845in}}%
\pgfpathlineto{\pgfqpoint{0.790495in}{0.777741in}}%
\pgfpathlineto{\pgfqpoint{0.802207in}{0.776111in}}%
\pgfpathlineto{\pgfqpoint{0.804062in}{0.775996in}}%
\pgfpathlineto{\pgfqpoint{0.813920in}{0.775376in}}%
\pgfpathlineto{\pgfqpoint{0.825633in}{0.775145in}}%
\pgfpathlineto{\pgfqpoint{0.837345in}{0.771066in}}%
\pgfpathlineto{\pgfqpoint{0.841305in}{0.763867in}}%
\pgfpathlineto{\pgfqpoint{0.841596in}{0.751739in}}%
\pgfpathlineto{\pgfqpoint{0.841597in}{0.739610in}}%
\pgfpathlineto{\pgfqpoint{0.837345in}{0.735195in}}%
\pgfpathlineto{\pgfqpoint{0.825633in}{0.728658in}}%
\pgfpathlineto{\pgfqpoint{0.822590in}{0.727481in}}%
\pgfpathlineto{\pgfqpoint{0.813920in}{0.723769in}}%
\pgfpathlineto{\pgfqpoint{0.802207in}{0.718806in}}%
\pgfpathlineto{\pgfqpoint{0.794394in}{0.715352in}}%
\pgfpathlineto{\pgfqpoint{0.790495in}{0.712843in}}%
\pgfpathlineto{\pgfqpoint{0.778782in}{0.705598in}}%
\pgfpathlineto{\pgfqpoint{0.775399in}{0.703223in}}%
\pgfpathlineto{\pgfqpoint{0.778782in}{0.699623in}}%
\pgfpathlineto{\pgfqpoint{0.784149in}{0.691094in}}%
\pgfpathlineto{\pgfqpoint{0.790495in}{0.684527in}}%
\pgfpathlineto{\pgfqpoint{0.796120in}{0.678965in}}%
\pgfpathlineto{\pgfqpoint{0.802207in}{0.672548in}}%
\pgfpathlineto{\pgfqpoint{0.809171in}{0.666836in}}%
\pgfpathlineto{\pgfqpoint{0.813920in}{0.662478in}}%
\pgfpathlineto{\pgfqpoint{0.825051in}{0.654708in}}%
\pgfpathlineto{\pgfqpoint{0.825633in}{0.654257in}}%
\pgfpathlineto{\pgfqpoint{0.837345in}{0.645368in}}%
\pgfpathlineto{\pgfqpoint{0.839967in}{0.642579in}}%
\pgfpathlineto{\pgfqpoint{0.841433in}{0.630450in}}%
\pgfpathlineto{\pgfqpoint{0.839870in}{0.618321in}}%
\pgfpathlineto{\pgfqpoint{0.839126in}{0.606192in}}%
\pgfpathlineto{\pgfqpoint{0.838722in}{0.594063in}}%
\pgfpathlineto{\pgfqpoint{0.838072in}{0.581934in}}%
\pgfpathlineto{\pgfqpoint{0.838465in}{0.569805in}}%
\pgfpathlineto{\pgfqpoint{0.838898in}{0.557677in}}%
\pgfpathlineto{\pgfqpoint{0.839609in}{0.545548in}}%
\pgfpathlineto{\pgfqpoint{0.841712in}{0.533419in}}%
\pgfpathlineto{\pgfqpoint{0.837345in}{0.528908in}}%
\pgfpathlineto{\pgfqpoint{0.827234in}{0.521290in}}%
\pgfpathlineto{\pgfqpoint{0.825633in}{0.520632in}}%
\pgfpathlineto{\pgfqpoint{0.813920in}{0.518401in}}%
\pgfpathlineto{\pgfqpoint{0.802207in}{0.513781in}}%
\pgfpathlineto{\pgfqpoint{0.790495in}{0.510891in}}%
\pgfpathlineto{\pgfqpoint{0.778782in}{0.510281in}}%
\pgfpathlineto{\pgfqpoint{0.767069in}{0.516670in}}%
\pgfpathlineto{\pgfqpoint{0.757800in}{0.521290in}}%
\pgfpathlineto{\pgfqpoint{0.755356in}{0.522379in}}%
\pgfpathlineto{\pgfqpoint{0.743644in}{0.526761in}}%
\pgfpathlineto{\pgfqpoint{0.731931in}{0.532522in}}%
\pgfpathlineto{\pgfqpoint{0.728371in}{0.533419in}}%
\pgfpathlineto{\pgfqpoint{0.720218in}{0.535216in}}%
\pgfpathlineto{\pgfqpoint{0.708506in}{0.534993in}}%
\pgfpathlineto{\pgfqpoint{0.696793in}{0.533695in}}%
\pgfpathlineto{\pgfqpoint{0.685080in}{0.534991in}}%
\pgfpathlineto{\pgfqpoint{0.675352in}{0.545548in}}%
\pgfpathlineto{\pgfqpoint{0.673368in}{0.549924in}}%
\pgfpathlineto{\pgfqpoint{0.671155in}{0.557677in}}%
\pgfpathlineto{\pgfqpoint{0.670806in}{0.569805in}}%
\pgfpathlineto{\pgfqpoint{0.670406in}{0.581934in}}%
\pgfpathlineto{\pgfqpoint{0.670465in}{0.594063in}}%
\pgfpathlineto{\pgfqpoint{0.671684in}{0.606192in}}%
\pgfpathlineto{\pgfqpoint{0.673042in}{0.618321in}}%
\pgfpathlineto{\pgfqpoint{0.673368in}{0.627288in}}%
\pgfpathlineto{\pgfqpoint{0.673466in}{0.630450in}}%
\pgfpathlineto{\pgfqpoint{0.673368in}{0.631501in}}%
\pgfpathlineto{\pgfqpoint{0.672388in}{0.642579in}}%
\pgfpathlineto{\pgfqpoint{0.671797in}{0.654708in}}%
\pgfpathlineto{\pgfqpoint{0.673368in}{0.662110in}}%
\pgfpathlineto{\pgfqpoint{0.674123in}{0.666836in}}%
\pgfpathlineto{\pgfqpoint{0.677379in}{0.678965in}}%
\pgfpathlineto{\pgfqpoint{0.680641in}{0.691094in}}%
\pgfpathlineto{\pgfqpoint{0.685080in}{0.696993in}}%
\pgfpathlineto{\pgfqpoint{0.696353in}{0.703223in}}%
\pgfpathlineto{\pgfqpoint{0.696793in}{0.703499in}}%
\pgfpathlineto{\pgfqpoint{0.708506in}{0.707773in}}%
\pgfpathlineto{\pgfqpoint{0.720218in}{0.711083in}}%
\pgfpathlineto{\pgfqpoint{0.731465in}{0.715352in}}%
\pgfpathlineto{\pgfqpoint{0.725277in}{0.727481in}}%
\pgfpathlineto{\pgfqpoint{0.720218in}{0.730151in}}%
\pgfpathlineto{\pgfqpoint{0.708506in}{0.737137in}}%
\pgfpathlineto{\pgfqpoint{0.703524in}{0.739610in}}%
\pgfpathlineto{\pgfqpoint{0.696793in}{0.741174in}}%
\pgfpathlineto{\pgfqpoint{0.685080in}{0.743369in}}%
\pgfpathlineto{\pgfqpoint{0.673368in}{0.746401in}}%
\pgfpathlineto{\pgfqpoint{0.661655in}{0.750470in}}%
\pgfpathlineto{\pgfqpoint{0.658118in}{0.751739in}}%
\pgfpathlineto{\pgfqpoint{0.649942in}{0.753985in}}%
\pgfpathlineto{\pgfqpoint{0.638230in}{0.757412in}}%
\pgfpathlineto{\pgfqpoint{0.626517in}{0.761398in}}%
\pgfpathlineto{\pgfqpoint{0.618340in}{0.763867in}}%
\pgfpathlineto{\pgfqpoint{0.614804in}{0.764662in}}%
\pgfpathlineto{\pgfqpoint{0.603091in}{0.766569in}}%
\pgfpathlineto{\pgfqpoint{0.591379in}{0.768897in}}%
\pgfpathlineto{\pgfqpoint{0.579666in}{0.771239in}}%
\pgfpathlineto{\pgfqpoint{0.567953in}{0.772796in}}%
\pgfpathlineto{\pgfqpoint{0.556241in}{0.774349in}}%
\pgfpathlineto{\pgfqpoint{0.548447in}{0.775996in}}%
\pgfpathlineto{\pgfqpoint{0.544528in}{0.776986in}}%
\pgfpathlineto{\pgfqpoint{0.532815in}{0.780896in}}%
\pgfpathlineto{\pgfqpoint{0.521103in}{0.785438in}}%
\pgfpathlineto{\pgfqpoint{0.514383in}{0.788125in}}%
\pgfpathlineto{\pgfqpoint{0.509390in}{0.790452in}}%
\pgfpathlineto{\pgfqpoint{0.497677in}{0.795900in}}%
\pgfpathlineto{\pgfqpoint{0.489023in}{0.800254in}}%
\pgfpathlineto{\pgfqpoint{0.485965in}{0.801925in}}%
\pgfpathlineto{\pgfqpoint{0.474252in}{0.807653in}}%
\pgfpathlineto{\pgfqpoint{0.463437in}{0.812383in}}%
\pgfpathlineto{\pgfqpoint{0.462539in}{0.812795in}}%
\pgfpathlineto{\pgfqpoint{0.450827in}{0.818299in}}%
\pgfpathlineto{\pgfqpoint{0.443056in}{0.824512in}}%
\pgfpathlineto{\pgfqpoint{0.439151in}{0.836641in}}%
\pgfpathlineto{\pgfqpoint{0.443997in}{0.848770in}}%
\pgfpathlineto{\pgfqpoint{0.446141in}{0.860898in}}%
\pgfpathlineto{\pgfqpoint{0.445946in}{0.873027in}}%
\pgfpathlineto{\pgfqpoint{0.439530in}{0.885156in}}%
\pgfpathlineto{\pgfqpoint{0.439114in}{0.885676in}}%
\pgfpathlineto{\pgfqpoint{0.428164in}{0.897285in}}%
\pgfpathlineto{\pgfqpoint{0.427401in}{0.898138in}}%
\pgfpathlineto{\pgfqpoint{0.415976in}{0.909414in}}%
\pgfpathlineto{\pgfqpoint{0.415688in}{0.909718in}}%
\pgfpathlineto{\pgfqpoint{0.403976in}{0.920475in}}%
\pgfpathlineto{\pgfqpoint{0.402673in}{0.921543in}}%
\pgfpathlineto{\pgfqpoint{0.392263in}{0.930528in}}%
\pgfpathlineto{\pgfqpoint{0.388058in}{0.933672in}}%
\pgfpathlineto{\pgfqpoint{0.380550in}{0.938908in}}%
\pgfpathlineto{\pgfqpoint{0.369523in}{0.945801in}}%
\pgfpathlineto{\pgfqpoint{0.368838in}{0.946292in}}%
\pgfpathlineto{\pgfqpoint{0.357125in}{0.953703in}}%
\pgfpathlineto{\pgfqpoint{0.350224in}{0.957929in}}%
\pgfpathlineto{\pgfqpoint{0.345412in}{0.961062in}}%
\pgfpathlineto{\pgfqpoint{0.333700in}{0.968334in}}%
\pgfpathlineto{\pgfqpoint{0.330773in}{0.970058in}}%
\pgfpathlineto{\pgfqpoint{0.321987in}{0.976194in}}%
\pgfpathlineto{\pgfqpoint{0.312393in}{0.982187in}}%
\pgfpathlineto{\pgfqpoint{0.310274in}{0.983561in}}%
\pgfpathlineto{\pgfqpoint{0.298562in}{0.990408in}}%
\pgfpathlineto{\pgfqpoint{0.291327in}{0.994316in}}%
\pgfpathlineto{\pgfqpoint{0.286849in}{0.996732in}}%
\pgfpathlineto{\pgfqpoint{0.286849in}{0.994316in}}%
\pgfpathlineto{\pgfqpoint{0.286849in}{0.982187in}}%
\pgfpathlineto{\pgfqpoint{0.286849in}{0.970058in}}%
\pgfpathlineto{\pgfqpoint{0.286849in}{0.957929in}}%
\pgfpathlineto{\pgfqpoint{0.286849in}{0.945801in}}%
\pgfpathlineto{\pgfqpoint{0.286849in}{0.933672in}}%
\pgfpathlineto{\pgfqpoint{0.286849in}{0.921543in}}%
\pgfpathlineto{\pgfqpoint{0.286849in}{0.909414in}}%
\pgfpathlineto{\pgfqpoint{0.286849in}{0.897285in}}%
\pgfpathlineto{\pgfqpoint{0.286849in}{0.885156in}}%
\pgfpathlineto{\pgfqpoint{0.286849in}{0.873027in}}%
\pgfpathlineto{\pgfqpoint{0.286849in}{0.860898in}}%
\pgfpathlineto{\pgfqpoint{0.286849in}{0.851253in}}%
\pgfpathlineto{\pgfqpoint{0.289433in}{0.848770in}}%
\pgfpathlineto{\pgfqpoint{0.298562in}{0.841393in}}%
\pgfpathlineto{\pgfqpoint{0.304228in}{0.836641in}}%
\pgfpathlineto{\pgfqpoint{0.310274in}{0.831730in}}%
\pgfpathlineto{\pgfqpoint{0.319176in}{0.824512in}}%
\pgfpathlineto{\pgfqpoint{0.321987in}{0.821606in}}%
\pgfpathlineto{\pgfqpoint{0.330784in}{0.812383in}}%
\pgfpathlineto{\pgfqpoint{0.333700in}{0.807560in}}%
\pgfpathlineto{\pgfqpoint{0.339374in}{0.800254in}}%
\pgfpathlineto{\pgfqpoint{0.333700in}{0.790548in}}%
\pgfpathlineto{\pgfqpoint{0.332359in}{0.788125in}}%
\pgfpathlineto{\pgfqpoint{0.322171in}{0.775996in}}%
\pgfpathlineto{\pgfqpoint{0.321987in}{0.775754in}}%
\pgfpathlineto{\pgfqpoint{0.314239in}{0.763867in}}%
\pgfpathlineto{\pgfqpoint{0.310274in}{0.756647in}}%
\pgfpathlineto{\pgfqpoint{0.307879in}{0.751739in}}%
\pgfpathlineto{\pgfqpoint{0.304575in}{0.739610in}}%
\pgfpathlineto{\pgfqpoint{0.309477in}{0.727481in}}%
\pgfpathlineto{\pgfqpoint{0.310274in}{0.723164in}}%
\pgfpathlineto{\pgfqpoint{0.311286in}{0.715352in}}%
\pgfpathlineto{\pgfqpoint{0.310274in}{0.713502in}}%
\pgfpathlineto{\pgfqpoint{0.305359in}{0.703223in}}%
\pgfpathlineto{\pgfqpoint{0.298562in}{0.691974in}}%
\pgfpathlineto{\pgfqpoint{0.298020in}{0.691094in}}%
\pgfpathlineto{\pgfqpoint{0.292488in}{0.678965in}}%
\pgfpathlineto{\pgfqpoint{0.288840in}{0.666836in}}%
\pgfpathlineto{\pgfqpoint{0.286849in}{0.660751in}}%
\pgfpathlineto{\pgfqpoint{0.286849in}{0.654708in}}%
\pgfpathlineto{\pgfqpoint{0.286849in}{0.642579in}}%
\pgfpathlineto{\pgfqpoint{0.286849in}{0.630450in}}%
\pgfpathlineto{\pgfqpoint{0.286849in}{0.618321in}}%
\pgfpathlineto{\pgfqpoint{0.286849in}{0.606192in}}%
\pgfpathlineto{\pgfqpoint{0.286849in}{0.594063in}}%
\pgfpathlineto{\pgfqpoint{0.286849in}{0.581934in}}%
\pgfpathlineto{\pgfqpoint{0.286849in}{0.569805in}}%
\pgfpathlineto{\pgfqpoint{0.286849in}{0.563686in}}%
\pgfpathlineto{\pgfqpoint{0.292809in}{0.569805in}}%
\pgfpathlineto{\pgfqpoint{0.298562in}{0.575830in}}%
\pgfpathlineto{\pgfqpoint{0.304092in}{0.581934in}}%
\pgfpathlineto{\pgfqpoint{0.310274in}{0.589459in}}%
\pgfpathlineto{\pgfqpoint{0.313927in}{0.594063in}}%
\pgfpathlineto{\pgfqpoint{0.321987in}{0.606126in}}%
\pgfpathlineto{\pgfqpoint{0.322029in}{0.606192in}}%
\pgfpathlineto{\pgfqpoint{0.328755in}{0.618321in}}%
\pgfpathlineto{\pgfqpoint{0.333700in}{0.628601in}}%
\pgfpathlineto{\pgfqpoint{0.334564in}{0.630450in}}%
\pgfpathlineto{\pgfqpoint{0.339033in}{0.642579in}}%
\pgfpathlineto{\pgfqpoint{0.342558in}{0.654708in}}%
\pgfpathlineto{\pgfqpoint{0.345245in}{0.666836in}}%
\pgfpathlineto{\pgfqpoint{0.345412in}{0.667440in}}%
\pgfpathlineto{\pgfqpoint{0.348613in}{0.678965in}}%
\pgfpathlineto{\pgfqpoint{0.351783in}{0.691094in}}%
\pgfpathlineto{\pgfqpoint{0.354770in}{0.703223in}}%
\pgfpathlineto{\pgfqpoint{0.356752in}{0.715352in}}%
\pgfpathlineto{\pgfqpoint{0.357125in}{0.718342in}}%
\pgfpathlineto{\pgfqpoint{0.358135in}{0.727481in}}%
\pgfpathlineto{\pgfqpoint{0.361805in}{0.739610in}}%
\pgfpathlineto{\pgfqpoint{0.368838in}{0.747307in}}%
\pgfpathlineto{\pgfqpoint{0.373049in}{0.751739in}}%
\pgfpathlineto{\pgfqpoint{0.380550in}{0.757892in}}%
\pgfpathlineto{\pgfqpoint{0.388883in}{0.763867in}}%
\pgfpathlineto{\pgfqpoint{0.392263in}{0.765923in}}%
\pgfpathlineto{\pgfqpoint{0.403976in}{0.771187in}}%
\pgfpathlineto{\pgfqpoint{0.415260in}{0.775996in}}%
\pgfpathlineto{\pgfqpoint{0.415688in}{0.776274in}}%
\pgfpathlineto{\pgfqpoint{0.427401in}{0.782486in}}%
\pgfpathlineto{\pgfqpoint{0.439114in}{0.786579in}}%
\pgfpathlineto{\pgfqpoint{0.449403in}{0.788125in}}%
\pgfpathlineto{\pgfqpoint{0.450827in}{0.788337in}}%
\pgfpathlineto{\pgfqpoint{0.457060in}{0.788125in}}%
\pgfpathlineto{\pgfqpoint{0.462539in}{0.787975in}}%
\pgfpathlineto{\pgfqpoint{0.474252in}{0.786075in}}%
\pgfpathlineto{\pgfqpoint{0.485965in}{0.783644in}}%
\pgfpathlineto{\pgfqpoint{0.497677in}{0.780821in}}%
\pgfpathlineto{\pgfqpoint{0.509390in}{0.777786in}}%
\pgfpathlineto{\pgfqpoint{0.515469in}{0.775996in}}%
\pgfpathlineto{\pgfqpoint{0.521103in}{0.774564in}}%
\pgfpathlineto{\pgfqpoint{0.532815in}{0.771296in}}%
\pgfpathlineto{\pgfqpoint{0.544528in}{0.768239in}}%
\pgfpathlineto{\pgfqpoint{0.556241in}{0.765220in}}%
\pgfpathlineto{\pgfqpoint{0.561369in}{0.763867in}}%
\pgfpathlineto{\pgfqpoint{0.567953in}{0.761982in}}%
\pgfpathlineto{\pgfqpoint{0.579666in}{0.759153in}}%
\pgfpathlineto{\pgfqpoint{0.591379in}{0.755818in}}%
\pgfpathlineto{\pgfqpoint{0.603091in}{0.753144in}}%
\pgfpathlineto{\pgfqpoint{0.608501in}{0.751739in}}%
\pgfpathlineto{\pgfqpoint{0.614804in}{0.749839in}}%
\pgfpathlineto{\pgfqpoint{0.626517in}{0.745463in}}%
\pgfpathlineto{\pgfqpoint{0.638230in}{0.740964in}}%
\pgfpathlineto{\pgfqpoint{0.642850in}{0.739610in}}%
\pgfpathlineto{\pgfqpoint{0.649942in}{0.736934in}}%
\pgfpathlineto{\pgfqpoint{0.661655in}{0.732296in}}%
\pgfpathlineto{\pgfqpoint{0.673368in}{0.727497in}}%
\pgfpathlineto{\pgfqpoint{0.673403in}{0.727481in}}%
\pgfpathlineto{\pgfqpoint{0.680509in}{0.715352in}}%
\pgfpathlineto{\pgfqpoint{0.673368in}{0.709947in}}%
\pgfpathlineto{\pgfqpoint{0.669946in}{0.703223in}}%
\pgfpathlineto{\pgfqpoint{0.666992in}{0.691094in}}%
\pgfpathlineto{\pgfqpoint{0.663232in}{0.678965in}}%
\pgfpathlineto{\pgfqpoint{0.661655in}{0.674372in}}%
\pgfpathlineto{\pgfqpoint{0.658653in}{0.666836in}}%
\pgfpathlineto{\pgfqpoint{0.651493in}{0.654708in}}%
\pgfpathlineto{\pgfqpoint{0.649942in}{0.651623in}}%
\pgfpathlineto{\pgfqpoint{0.644171in}{0.642579in}}%
\pgfpathlineto{\pgfqpoint{0.638389in}{0.630450in}}%
\pgfpathlineto{\pgfqpoint{0.638230in}{0.629789in}}%
\pgfpathlineto{\pgfqpoint{0.634938in}{0.618321in}}%
\pgfpathlineto{\pgfqpoint{0.630977in}{0.606192in}}%
\pgfpathlineto{\pgfqpoint{0.626674in}{0.594063in}}%
\pgfpathlineto{\pgfqpoint{0.626517in}{0.593481in}}%
\pgfpathlineto{\pgfqpoint{0.622819in}{0.581934in}}%
\pgfpathlineto{\pgfqpoint{0.618302in}{0.569805in}}%
\pgfpathlineto{\pgfqpoint{0.614804in}{0.559045in}}%
\pgfpathlineto{\pgfqpoint{0.614271in}{0.557677in}}%
\pgfpathlineto{\pgfqpoint{0.609177in}{0.545548in}}%
\pgfpathlineto{\pgfqpoint{0.603771in}{0.533419in}}%
\pgfpathlineto{\pgfqpoint{0.603091in}{0.531874in}}%
\pgfpathlineto{\pgfqpoint{0.597877in}{0.521290in}}%
\pgfpathlineto{\pgfqpoint{0.591743in}{0.509161in}}%
\pgfpathlineto{\pgfqpoint{0.591379in}{0.508555in}}%
\pgfpathlineto{\pgfqpoint{0.584387in}{0.497032in}}%
\pgfpathlineto{\pgfqpoint{0.579666in}{0.490673in}}%
\pgfpathlineto{\pgfqpoint{0.574923in}{0.484903in}}%
\pgfpathlineto{\pgfqpoint{0.567953in}{0.477685in}}%
\pgfpathlineto{\pgfqpoint{0.556241in}{0.473353in}}%
\pgfpathlineto{\pgfqpoint{0.554297in}{0.472774in}}%
\pgfpathlineto{\pgfqpoint{0.544528in}{0.470210in}}%
\pgfpathlineto{\pgfqpoint{0.532815in}{0.466990in}}%
\pgfpathlineto{\pgfqpoint{0.521103in}{0.463873in}}%
\pgfpathlineto{\pgfqpoint{0.509390in}{0.461336in}}%
\pgfpathlineto{\pgfqpoint{0.507175in}{0.460646in}}%
\pgfpathlineto{\pgfqpoint{0.497677in}{0.457822in}}%
\pgfpathlineto{\pgfqpoint{0.485965in}{0.455544in}}%
\pgfpathlineto{\pgfqpoint{0.474252in}{0.453795in}}%
\pgfpathlineto{\pgfqpoint{0.462539in}{0.451450in}}%
\pgfpathlineto{\pgfqpoint{0.451730in}{0.448517in}}%
\pgfpathlineto{\pgfqpoint{0.450827in}{0.448320in}}%
\pgfpathlineto{\pgfqpoint{0.439114in}{0.445122in}}%
\pgfpathlineto{\pgfqpoint{0.427401in}{0.441873in}}%
\pgfpathlineto{\pgfqpoint{0.415688in}{0.438621in}}%
\pgfpathlineto{\pgfqpoint{0.407824in}{0.436388in}}%
\pgfpathlineto{\pgfqpoint{0.403976in}{0.435340in}}%
\pgfpathlineto{\pgfqpoint{0.392263in}{0.432253in}}%
\pgfpathlineto{\pgfqpoint{0.380550in}{0.429145in}}%
\pgfpathlineto{\pgfqpoint{0.368838in}{0.425679in}}%
\pgfpathlineto{\pgfqpoint{0.364563in}{0.424259in}}%
\pgfpathlineto{\pgfqpoint{0.357125in}{0.421673in}}%
\pgfpathlineto{\pgfqpoint{0.345412in}{0.417679in}}%
\pgfpathlineto{\pgfqpoint{0.333700in}{0.413572in}}%
\pgfpathlineto{\pgfqpoint{0.329419in}{0.412130in}}%
\pgfpathlineto{\pgfqpoint{0.321987in}{0.409980in}}%
\pgfpathlineto{\pgfqpoint{0.310274in}{0.406460in}}%
\pgfpathlineto{\pgfqpoint{0.298562in}{0.403374in}}%
\pgfpathlineto{\pgfqpoint{0.287698in}{0.400001in}}%
\pgfpathlineto{\pgfqpoint{0.286849in}{0.399743in}}%
\pgfpathlineto{\pgfqpoint{0.286849in}{0.387872in}}%
\pgfpathlineto{\pgfqpoint{0.286849in}{0.375743in}}%
\pgfpathlineto{\pgfqpoint{0.286849in}{0.363615in}}%
\pgfpathlineto{\pgfqpoint{0.286849in}{0.351486in}}%
\pgfpathlineto{\pgfqpoint{0.286849in}{0.347572in}}%
\pgfpathlineto{\pgfqpoint{0.298562in}{0.348900in}}%
\pgfpathlineto{\pgfqpoint{0.310274in}{0.350412in}}%
\pgfpathlineto{\pgfqpoint{0.317122in}{0.351486in}}%
\pgfpathlineto{\pgfqpoint{0.321987in}{0.352299in}}%
\pgfpathlineto{\pgfqpoint{0.333700in}{0.354473in}}%
\pgfpathlineto{\pgfqpoint{0.345412in}{0.356778in}}%
\pgfpathlineto{\pgfqpoint{0.357125in}{0.359279in}}%
\pgfpathlineto{\pgfqpoint{0.368838in}{0.361852in}}%
\pgfpathlineto{\pgfqpoint{0.377355in}{0.363615in}}%
\pgfpathlineto{\pgfqpoint{0.380550in}{0.364291in}}%
\pgfpathlineto{\pgfqpoint{0.392263in}{0.366155in}}%
\pgfpathlineto{\pgfqpoint{0.403976in}{0.367910in}}%
\pgfpathlineto{\pgfqpoint{0.415688in}{0.369524in}}%
\pgfpathlineto{\pgfqpoint{0.427401in}{0.369693in}}%
\pgfpathlineto{\pgfqpoint{0.436866in}{0.363615in}}%
\pgfpathlineto{\pgfqpoint{0.436470in}{0.351486in}}%
\pgfpathlineto{\pgfqpoint{0.427401in}{0.340544in}}%
\pgfpathlineto{\pgfqpoint{0.426794in}{0.339357in}}%
\pgfpathlineto{\pgfqpoint{0.415688in}{0.329589in}}%
\pgfpathlineto{\pgfqpoint{0.413544in}{0.327228in}}%
\pgfpathlineto{\pgfqpoint{0.403976in}{0.318627in}}%
\pgfpathlineto{\pgfqpoint{0.400370in}{0.315099in}}%
\pgfpathlineto{\pgfqpoint{0.392263in}{0.308726in}}%
\pgfpathlineto{\pgfqpoint{0.385967in}{0.302970in}}%
\pgfpathlineto{\pgfqpoint{0.380550in}{0.299172in}}%
\pgfpathlineto{\pgfqpoint{0.370739in}{0.290841in}}%
\pgfpathclose%
\pgfusepath{fill}%
\end{pgfscope}%
\begin{pgfscope}%
\pgfpathrectangle{\pgfqpoint{0.211875in}{0.211875in}}{\pgfqpoint{1.313625in}{1.279725in}}%
\pgfusepath{clip}%
\pgfsetbuttcap%
\pgfsetroundjoin%
\definecolor{currentfill}{rgb}{0.796501,0.105066,0.310630}%
\pgfsetfillcolor{currentfill}%
\pgfsetlinewidth{0.000000pt}%
\definecolor{currentstroke}{rgb}{0.000000,0.000000,0.000000}%
\pgfsetstrokecolor{currentstroke}%
\pgfsetdash{}{0pt}%
\pgfpathmoveto{\pgfqpoint{1.024748in}{0.293361in}}%
\pgfpathlineto{\pgfqpoint{1.028569in}{0.290841in}}%
\pgfpathlineto{\pgfqpoint{1.036461in}{0.290841in}}%
\pgfpathlineto{\pgfqpoint{1.048174in}{0.290841in}}%
\pgfpathlineto{\pgfqpoint{1.057695in}{0.290841in}}%
\pgfpathlineto{\pgfqpoint{1.048174in}{0.295073in}}%
\pgfpathlineto{\pgfqpoint{1.036461in}{0.301683in}}%
\pgfpathlineto{\pgfqpoint{1.034624in}{0.302970in}}%
\pgfpathlineto{\pgfqpoint{1.024748in}{0.314396in}}%
\pgfpathlineto{\pgfqpoint{1.024176in}{0.315099in}}%
\pgfpathlineto{\pgfqpoint{1.019216in}{0.327228in}}%
\pgfpathlineto{\pgfqpoint{1.017404in}{0.339357in}}%
\pgfpathlineto{\pgfqpoint{1.018481in}{0.351486in}}%
\pgfpathlineto{\pgfqpoint{1.024748in}{0.362151in}}%
\pgfpathlineto{\pgfqpoint{1.025711in}{0.363615in}}%
\pgfpathlineto{\pgfqpoint{1.034622in}{0.375743in}}%
\pgfpathlineto{\pgfqpoint{1.036461in}{0.378278in}}%
\pgfpathlineto{\pgfqpoint{1.048174in}{0.387004in}}%
\pgfpathlineto{\pgfqpoint{1.052318in}{0.387872in}}%
\pgfpathlineto{\pgfqpoint{1.059886in}{0.389097in}}%
\pgfpathlineto{\pgfqpoint{1.071599in}{0.391413in}}%
\pgfpathlineto{\pgfqpoint{1.083312in}{0.394418in}}%
\pgfpathlineto{\pgfqpoint{1.095024in}{0.397660in}}%
\pgfpathlineto{\pgfqpoint{1.101213in}{0.400001in}}%
\pgfpathlineto{\pgfqpoint{1.106737in}{0.402078in}}%
\pgfpathlineto{\pgfqpoint{1.118450in}{0.406749in}}%
\pgfpathlineto{\pgfqpoint{1.123972in}{0.412130in}}%
\pgfpathlineto{\pgfqpoint{1.130163in}{0.419029in}}%
\pgfpathlineto{\pgfqpoint{1.134865in}{0.424259in}}%
\pgfpathlineto{\pgfqpoint{1.141875in}{0.432027in}}%
\pgfpathlineto{\pgfqpoint{1.146007in}{0.436388in}}%
\pgfpathlineto{\pgfqpoint{1.153588in}{0.444328in}}%
\pgfpathlineto{\pgfqpoint{1.158511in}{0.448517in}}%
\pgfpathlineto{\pgfqpoint{1.165301in}{0.455043in}}%
\pgfpathlineto{\pgfqpoint{1.172382in}{0.460646in}}%
\pgfpathlineto{\pgfqpoint{1.177013in}{0.464643in}}%
\pgfpathlineto{\pgfqpoint{1.188726in}{0.468329in}}%
\pgfpathlineto{\pgfqpoint{1.200439in}{0.466381in}}%
\pgfpathlineto{\pgfqpoint{1.205750in}{0.460646in}}%
\pgfpathlineto{\pgfqpoint{1.211497in}{0.448517in}}%
\pgfpathlineto{\pgfqpoint{1.212151in}{0.447338in}}%
\pgfpathlineto{\pgfqpoint{1.219008in}{0.436388in}}%
\pgfpathlineto{\pgfqpoint{1.223864in}{0.430015in}}%
\pgfpathlineto{\pgfqpoint{1.230489in}{0.424259in}}%
\pgfpathlineto{\pgfqpoint{1.235577in}{0.420063in}}%
\pgfpathlineto{\pgfqpoint{1.247289in}{0.412646in}}%
\pgfpathlineto{\pgfqpoint{1.248151in}{0.412130in}}%
\pgfpathlineto{\pgfqpoint{1.259002in}{0.403616in}}%
\pgfpathlineto{\pgfqpoint{1.262867in}{0.400001in}}%
\pgfpathlineto{\pgfqpoint{1.270715in}{0.390722in}}%
\pgfpathlineto{\pgfqpoint{1.272315in}{0.387872in}}%
\pgfpathlineto{\pgfqpoint{1.274997in}{0.375743in}}%
\pgfpathlineto{\pgfqpoint{1.270715in}{0.366828in}}%
\pgfpathlineto{\pgfqpoint{1.269830in}{0.363615in}}%
\pgfpathlineto{\pgfqpoint{1.259002in}{0.360163in}}%
\pgfpathlineto{\pgfqpoint{1.247762in}{0.351486in}}%
\pgfpathlineto{\pgfqpoint{1.247289in}{0.351279in}}%
\pgfpathlineto{\pgfqpoint{1.235577in}{0.347305in}}%
\pgfpathlineto{\pgfqpoint{1.223864in}{0.342396in}}%
\pgfpathlineto{\pgfqpoint{1.219391in}{0.339357in}}%
\pgfpathlineto{\pgfqpoint{1.212151in}{0.336451in}}%
\pgfpathlineto{\pgfqpoint{1.200439in}{0.332682in}}%
\pgfpathlineto{\pgfqpoint{1.189426in}{0.327228in}}%
\pgfpathlineto{\pgfqpoint{1.188726in}{0.326939in}}%
\pgfpathlineto{\pgfqpoint{1.177013in}{0.322423in}}%
\pgfpathlineto{\pgfqpoint{1.165301in}{0.320430in}}%
\pgfpathlineto{\pgfqpoint{1.153588in}{0.318811in}}%
\pgfpathlineto{\pgfqpoint{1.141875in}{0.316298in}}%
\pgfpathlineto{\pgfqpoint{1.137961in}{0.315099in}}%
\pgfpathlineto{\pgfqpoint{1.130163in}{0.313647in}}%
\pgfpathlineto{\pgfqpoint{1.118450in}{0.310263in}}%
\pgfpathlineto{\pgfqpoint{1.106737in}{0.304771in}}%
\pgfpathlineto{\pgfqpoint{1.103424in}{0.302970in}}%
\pgfpathlineto{\pgfqpoint{1.095024in}{0.299221in}}%
\pgfpathlineto{\pgfqpoint{1.083312in}{0.293717in}}%
\pgfpathlineto{\pgfqpoint{1.075775in}{0.290841in}}%
\pgfpathlineto{\pgfqpoint{1.083312in}{0.290841in}}%
\pgfpathlineto{\pgfqpoint{1.094417in}{0.290841in}}%
\pgfpathlineto{\pgfqpoint{1.095024in}{0.291142in}}%
\pgfpathlineto{\pgfqpoint{1.106737in}{0.296602in}}%
\pgfpathlineto{\pgfqpoint{1.118450in}{0.301743in}}%
\pgfpathlineto{\pgfqpoint{1.122039in}{0.302970in}}%
\pgfpathlineto{\pgfqpoint{1.130163in}{0.305662in}}%
\pgfpathlineto{\pgfqpoint{1.141875in}{0.307887in}}%
\pgfpathlineto{\pgfqpoint{1.153588in}{0.309698in}}%
\pgfpathlineto{\pgfqpoint{1.165301in}{0.311312in}}%
\pgfpathlineto{\pgfqpoint{1.177013in}{0.313399in}}%
\pgfpathlineto{\pgfqpoint{1.181244in}{0.315099in}}%
\pgfpathlineto{\pgfqpoint{1.188726in}{0.317762in}}%
\pgfpathlineto{\pgfqpoint{1.200439in}{0.321918in}}%
\pgfpathlineto{\pgfqpoint{1.212151in}{0.326727in}}%
\pgfpathlineto{\pgfqpoint{1.213323in}{0.327228in}}%
\pgfpathlineto{\pgfqpoint{1.223864in}{0.330823in}}%
\pgfpathlineto{\pgfqpoint{1.235577in}{0.335023in}}%
\pgfpathlineto{\pgfqpoint{1.245097in}{0.339357in}}%
\pgfpathlineto{\pgfqpoint{1.247289in}{0.340211in}}%
\pgfpathlineto{\pgfqpoint{1.259002in}{0.344065in}}%
\pgfpathlineto{\pgfqpoint{1.270715in}{0.349188in}}%
\pgfpathlineto{\pgfqpoint{1.275466in}{0.351486in}}%
\pgfpathlineto{\pgfqpoint{1.282427in}{0.354196in}}%
\pgfpathlineto{\pgfqpoint{1.294140in}{0.357795in}}%
\pgfpathlineto{\pgfqpoint{1.305853in}{0.360775in}}%
\pgfpathlineto{\pgfqpoint{1.317265in}{0.363615in}}%
\pgfpathlineto{\pgfqpoint{1.317566in}{0.363700in}}%
\pgfpathlineto{\pgfqpoint{1.329278in}{0.365987in}}%
\pgfpathlineto{\pgfqpoint{1.340991in}{0.366500in}}%
\pgfpathlineto{\pgfqpoint{1.352704in}{0.364020in}}%
\pgfpathlineto{\pgfqpoint{1.353576in}{0.363615in}}%
\pgfpathlineto{\pgfqpoint{1.364416in}{0.360909in}}%
\pgfpathlineto{\pgfqpoint{1.376129in}{0.358238in}}%
\pgfpathlineto{\pgfqpoint{1.387842in}{0.356413in}}%
\pgfpathlineto{\pgfqpoint{1.399554in}{0.359658in}}%
\pgfpathlineto{\pgfqpoint{1.411267in}{0.361791in}}%
\pgfpathlineto{\pgfqpoint{1.422980in}{0.362240in}}%
\pgfpathlineto{\pgfqpoint{1.434692in}{0.361059in}}%
\pgfpathlineto{\pgfqpoint{1.446405in}{0.359358in}}%
\pgfpathlineto{\pgfqpoint{1.446405in}{0.363615in}}%
\pgfpathlineto{\pgfqpoint{1.446405in}{0.375534in}}%
\pgfpathlineto{\pgfqpoint{1.446197in}{0.375743in}}%
\pgfpathlineto{\pgfqpoint{1.436958in}{0.387872in}}%
\pgfpathlineto{\pgfqpoint{1.434692in}{0.392725in}}%
\pgfpathlineto{\pgfqpoint{1.431994in}{0.400001in}}%
\pgfpathlineto{\pgfqpoint{1.434602in}{0.412130in}}%
\pgfpathlineto{\pgfqpoint{1.434621in}{0.424259in}}%
\pgfpathlineto{\pgfqpoint{1.429353in}{0.436388in}}%
\pgfpathlineto{\pgfqpoint{1.423299in}{0.448517in}}%
\pgfpathlineto{\pgfqpoint{1.422980in}{0.449034in}}%
\pgfpathlineto{\pgfqpoint{1.416492in}{0.460646in}}%
\pgfpathlineto{\pgfqpoint{1.411267in}{0.468232in}}%
\pgfpathlineto{\pgfqpoint{1.407721in}{0.472774in}}%
\pgfpathlineto{\pgfqpoint{1.399554in}{0.479896in}}%
\pgfpathlineto{\pgfqpoint{1.391103in}{0.484903in}}%
\pgfpathlineto{\pgfqpoint{1.387842in}{0.486517in}}%
\pgfpathlineto{\pgfqpoint{1.376129in}{0.492234in}}%
\pgfpathlineto{\pgfqpoint{1.367866in}{0.497032in}}%
\pgfpathlineto{\pgfqpoint{1.364416in}{0.498915in}}%
\pgfpathlineto{\pgfqpoint{1.352704in}{0.505310in}}%
\pgfpathlineto{\pgfqpoint{1.340991in}{0.504168in}}%
\pgfpathlineto{\pgfqpoint{1.332464in}{0.497032in}}%
\pgfpathlineto{\pgfqpoint{1.329278in}{0.494378in}}%
\pgfpathlineto{\pgfqpoint{1.317566in}{0.486970in}}%
\pgfpathlineto{\pgfqpoint{1.310716in}{0.484903in}}%
\pgfpathlineto{\pgfqpoint{1.305853in}{0.483722in}}%
\pgfpathlineto{\pgfqpoint{1.294140in}{0.484859in}}%
\pgfpathlineto{\pgfqpoint{1.294032in}{0.484903in}}%
\pgfpathlineto{\pgfqpoint{1.282427in}{0.489708in}}%
\pgfpathlineto{\pgfqpoint{1.270715in}{0.494475in}}%
\pgfpathlineto{\pgfqpoint{1.266028in}{0.497032in}}%
\pgfpathlineto{\pgfqpoint{1.259002in}{0.502520in}}%
\pgfpathlineto{\pgfqpoint{1.253259in}{0.509161in}}%
\pgfpathlineto{\pgfqpoint{1.247289in}{0.515438in}}%
\pgfpathlineto{\pgfqpoint{1.242409in}{0.521290in}}%
\pgfpathlineto{\pgfqpoint{1.235577in}{0.527911in}}%
\pgfpathlineto{\pgfqpoint{1.223864in}{0.529876in}}%
\pgfpathlineto{\pgfqpoint{1.212151in}{0.525346in}}%
\pgfpathlineto{\pgfqpoint{1.203869in}{0.521290in}}%
\pgfpathlineto{\pgfqpoint{1.200439in}{0.519784in}}%
\pgfpathlineto{\pgfqpoint{1.188726in}{0.513313in}}%
\pgfpathlineto{\pgfqpoint{1.183615in}{0.509161in}}%
\pgfpathlineto{\pgfqpoint{1.177013in}{0.504262in}}%
\pgfpathlineto{\pgfqpoint{1.169447in}{0.497032in}}%
\pgfpathlineto{\pgfqpoint{1.165301in}{0.493142in}}%
\pgfpathlineto{\pgfqpoint{1.157044in}{0.484903in}}%
\pgfpathlineto{\pgfqpoint{1.153588in}{0.481485in}}%
\pgfpathlineto{\pgfqpoint{1.144119in}{0.472774in}}%
\pgfpathlineto{\pgfqpoint{1.141875in}{0.470501in}}%
\pgfpathlineto{\pgfqpoint{1.131519in}{0.460646in}}%
\pgfpathlineto{\pgfqpoint{1.130163in}{0.459314in}}%
\pgfpathlineto{\pgfqpoint{1.118785in}{0.448517in}}%
\pgfpathlineto{\pgfqpoint{1.118450in}{0.448195in}}%
\pgfpathlineto{\pgfqpoint{1.106737in}{0.444212in}}%
\pgfpathlineto{\pgfqpoint{1.095024in}{0.442976in}}%
\pgfpathlineto{\pgfqpoint{1.083312in}{0.440382in}}%
\pgfpathlineto{\pgfqpoint{1.071599in}{0.437164in}}%
\pgfpathlineto{\pgfqpoint{1.059886in}{0.437623in}}%
\pgfpathlineto{\pgfqpoint{1.048174in}{0.439260in}}%
\pgfpathlineto{\pgfqpoint{1.036461in}{0.441445in}}%
\pgfpathlineto{\pgfqpoint{1.024748in}{0.443240in}}%
\pgfpathlineto{\pgfqpoint{1.013036in}{0.442499in}}%
\pgfpathlineto{\pgfqpoint{1.001323in}{0.441615in}}%
\pgfpathlineto{\pgfqpoint{0.989610in}{0.439575in}}%
\pgfpathlineto{\pgfqpoint{0.986037in}{0.436388in}}%
\pgfpathlineto{\pgfqpoint{0.980381in}{0.424259in}}%
\pgfpathlineto{\pgfqpoint{0.977898in}{0.420383in}}%
\pgfpathlineto{\pgfqpoint{0.967475in}{0.412130in}}%
\pgfpathlineto{\pgfqpoint{0.966185in}{0.411368in}}%
\pgfpathlineto{\pgfqpoint{0.954472in}{0.402698in}}%
\pgfpathlineto{\pgfqpoint{0.951612in}{0.400001in}}%
\pgfpathlineto{\pgfqpoint{0.942759in}{0.392323in}}%
\pgfpathlineto{\pgfqpoint{0.937129in}{0.387872in}}%
\pgfpathlineto{\pgfqpoint{0.931047in}{0.383328in}}%
\pgfpathlineto{\pgfqpoint{0.919334in}{0.376690in}}%
\pgfpathlineto{\pgfqpoint{0.917129in}{0.375743in}}%
\pgfpathlineto{\pgfqpoint{0.907621in}{0.371995in}}%
\pgfpathlineto{\pgfqpoint{0.895909in}{0.367919in}}%
\pgfpathlineto{\pgfqpoint{0.884196in}{0.365768in}}%
\pgfpathlineto{\pgfqpoint{0.872483in}{0.363840in}}%
\pgfpathlineto{\pgfqpoint{0.871406in}{0.363615in}}%
\pgfpathlineto{\pgfqpoint{0.860771in}{0.361191in}}%
\pgfpathlineto{\pgfqpoint{0.849058in}{0.359034in}}%
\pgfpathlineto{\pgfqpoint{0.837345in}{0.358305in}}%
\pgfpathlineto{\pgfqpoint{0.825633in}{0.359239in}}%
\pgfpathlineto{\pgfqpoint{0.813920in}{0.359195in}}%
\pgfpathlineto{\pgfqpoint{0.809249in}{0.351486in}}%
\pgfpathlineto{\pgfqpoint{0.811251in}{0.339357in}}%
\pgfpathlineto{\pgfqpoint{0.813920in}{0.336442in}}%
\pgfpathlineto{\pgfqpoint{0.825633in}{0.333862in}}%
\pgfpathlineto{\pgfqpoint{0.837345in}{0.333392in}}%
\pgfpathlineto{\pgfqpoint{0.849058in}{0.332757in}}%
\pgfpathlineto{\pgfqpoint{0.860771in}{0.332006in}}%
\pgfpathlineto{\pgfqpoint{0.872483in}{0.330968in}}%
\pgfpathlineto{\pgfqpoint{0.884196in}{0.330827in}}%
\pgfpathlineto{\pgfqpoint{0.895909in}{0.331614in}}%
\pgfpathlineto{\pgfqpoint{0.907621in}{0.332428in}}%
\pgfpathlineto{\pgfqpoint{0.919334in}{0.333573in}}%
\pgfpathlineto{\pgfqpoint{0.931047in}{0.334681in}}%
\pgfpathlineto{\pgfqpoint{0.942759in}{0.335556in}}%
\pgfpathlineto{\pgfqpoint{0.954472in}{0.335738in}}%
\pgfpathlineto{\pgfqpoint{0.966185in}{0.332397in}}%
\pgfpathlineto{\pgfqpoint{0.976862in}{0.327228in}}%
\pgfpathlineto{\pgfqpoint{0.977898in}{0.326955in}}%
\pgfpathlineto{\pgfqpoint{0.989610in}{0.323753in}}%
\pgfpathlineto{\pgfqpoint{1.001323in}{0.317461in}}%
\pgfpathlineto{\pgfqpoint{1.003381in}{0.315099in}}%
\pgfpathlineto{\pgfqpoint{1.013036in}{0.304425in}}%
\pgfpathlineto{\pgfqpoint{1.014240in}{0.302970in}}%
\pgfpathclose%
\pgfusepath{fill}%
\end{pgfscope}%
\begin{pgfscope}%
\pgfpathrectangle{\pgfqpoint{0.211875in}{0.211875in}}{\pgfqpoint{1.313625in}{1.279725in}}%
\pgfusepath{clip}%
\pgfsetbuttcap%
\pgfsetroundjoin%
\definecolor{currentfill}{rgb}{0.796501,0.105066,0.310630}%
\pgfsetfillcolor{currentfill}%
\pgfsetlinewidth{0.000000pt}%
\definecolor{currentstroke}{rgb}{0.000000,0.000000,0.000000}%
\pgfsetstrokecolor{currentstroke}%
\pgfsetdash{}{0pt}%
\pgfpathmoveto{\pgfqpoint{1.446405in}{0.737506in}}%
\pgfpathlineto{\pgfqpoint{1.446405in}{0.739610in}}%
\pgfpathlineto{\pgfqpoint{1.446405in}{0.749397in}}%
\pgfpathlineto{\pgfqpoint{1.440054in}{0.751739in}}%
\pgfpathlineto{\pgfqpoint{1.434692in}{0.758317in}}%
\pgfpathlineto{\pgfqpoint{1.432661in}{0.763867in}}%
\pgfpathlineto{\pgfqpoint{1.433402in}{0.775996in}}%
\pgfpathlineto{\pgfqpoint{1.434692in}{0.778112in}}%
\pgfpathlineto{\pgfqpoint{1.440623in}{0.788125in}}%
\pgfpathlineto{\pgfqpoint{1.446405in}{0.794651in}}%
\pgfpathlineto{\pgfqpoint{1.446405in}{0.800254in}}%
\pgfpathlineto{\pgfqpoint{1.446405in}{0.808813in}}%
\pgfpathlineto{\pgfqpoint{1.436388in}{0.800254in}}%
\pgfpathlineto{\pgfqpoint{1.434692in}{0.798341in}}%
\pgfpathlineto{\pgfqpoint{1.424586in}{0.788125in}}%
\pgfpathlineto{\pgfqpoint{1.422980in}{0.785362in}}%
\pgfpathlineto{\pgfqpoint{1.418746in}{0.775996in}}%
\pgfpathlineto{\pgfqpoint{1.418008in}{0.763867in}}%
\pgfpathlineto{\pgfqpoint{1.420703in}{0.751739in}}%
\pgfpathlineto{\pgfqpoint{1.422980in}{0.749040in}}%
\pgfpathlineto{\pgfqpoint{1.434692in}{0.741429in}}%
\pgfpathlineto{\pgfqpoint{1.439959in}{0.739610in}}%
\pgfpathclose%
\pgfusepath{fill}%
\end{pgfscope}%
\begin{pgfscope}%
\pgfpathrectangle{\pgfqpoint{0.211875in}{0.211875in}}{\pgfqpoint{1.313625in}{1.279725in}}%
\pgfusepath{clip}%
\pgfsetbuttcap%
\pgfsetroundjoin%
\definecolor{currentfill}{rgb}{0.796501,0.105066,0.310630}%
\pgfsetfillcolor{currentfill}%
\pgfsetlinewidth{0.000000pt}%
\definecolor{currentstroke}{rgb}{0.000000,0.000000,0.000000}%
\pgfsetstrokecolor{currentstroke}%
\pgfsetdash{}{0pt}%
\pgfpathmoveto{\pgfqpoint{1.422980in}{1.212449in}}%
\pgfpathlineto{\pgfqpoint{1.434692in}{1.205793in}}%
\pgfpathlineto{\pgfqpoint{1.446405in}{1.201742in}}%
\pgfpathlineto{\pgfqpoint{1.446405in}{1.212636in}}%
\pgfpathlineto{\pgfqpoint{1.446405in}{1.220637in}}%
\pgfpathlineto{\pgfqpoint{1.435365in}{1.224765in}}%
\pgfpathlineto{\pgfqpoint{1.434692in}{1.225276in}}%
\pgfpathlineto{\pgfqpoint{1.424723in}{1.236894in}}%
\pgfpathlineto{\pgfqpoint{1.422980in}{1.239556in}}%
\pgfpathlineto{\pgfqpoint{1.417365in}{1.249022in}}%
\pgfpathlineto{\pgfqpoint{1.411267in}{1.260272in}}%
\pgfpathlineto{\pgfqpoint{1.410619in}{1.261151in}}%
\pgfpathlineto{\pgfqpoint{1.408668in}{1.273280in}}%
\pgfpathlineto{\pgfqpoint{1.410596in}{1.285409in}}%
\pgfpathlineto{\pgfqpoint{1.411267in}{1.286494in}}%
\pgfpathlineto{\pgfqpoint{1.422980in}{1.295551in}}%
\pgfpathlineto{\pgfqpoint{1.433571in}{1.297538in}}%
\pgfpathlineto{\pgfqpoint{1.434692in}{1.297720in}}%
\pgfpathlineto{\pgfqpoint{1.437110in}{1.297538in}}%
\pgfpathlineto{\pgfqpoint{1.446405in}{1.296824in}}%
\pgfpathlineto{\pgfqpoint{1.446405in}{1.297538in}}%
\pgfpathlineto{\pgfqpoint{1.446405in}{1.309667in}}%
\pgfpathlineto{\pgfqpoint{1.446405in}{1.321796in}}%
\pgfpathlineto{\pgfqpoint{1.446405in}{1.333925in}}%
\pgfpathlineto{\pgfqpoint{1.446405in}{1.346053in}}%
\pgfpathlineto{\pgfqpoint{1.446405in}{1.358182in}}%
\pgfpathlineto{\pgfqpoint{1.446405in}{1.370311in}}%
\pgfpathlineto{\pgfqpoint{1.446405in}{1.382440in}}%
\pgfpathlineto{\pgfqpoint{1.446405in}{1.394569in}}%
\pgfpathlineto{\pgfqpoint{1.446405in}{1.399918in}}%
\pgfpathlineto{\pgfqpoint{1.434692in}{1.399671in}}%
\pgfpathlineto{\pgfqpoint{1.422980in}{1.404165in}}%
\pgfpathlineto{\pgfqpoint{1.421358in}{1.406698in}}%
\pgfpathlineto{\pgfqpoint{1.421554in}{1.418827in}}%
\pgfpathlineto{\pgfqpoint{1.421240in}{1.430956in}}%
\pgfpathlineto{\pgfqpoint{1.420118in}{1.443084in}}%
\pgfpathlineto{\pgfqpoint{1.420500in}{1.455213in}}%
\pgfpathlineto{\pgfqpoint{1.416760in}{1.467342in}}%
\pgfpathlineto{\pgfqpoint{1.411267in}{1.475814in}}%
\pgfpathlineto{\pgfqpoint{1.399554in}{1.471791in}}%
\pgfpathlineto{\pgfqpoint{1.393812in}{1.467342in}}%
\pgfpathlineto{\pgfqpoint{1.387842in}{1.462946in}}%
\pgfpathlineto{\pgfqpoint{1.378362in}{1.455213in}}%
\pgfpathlineto{\pgfqpoint{1.376129in}{1.453558in}}%
\pgfpathlineto{\pgfqpoint{1.364416in}{1.443852in}}%
\pgfpathlineto{\pgfqpoint{1.363122in}{1.443084in}}%
\pgfpathlineto{\pgfqpoint{1.352704in}{1.437200in}}%
\pgfpathlineto{\pgfqpoint{1.344311in}{1.430956in}}%
\pgfpathlineto{\pgfqpoint{1.340991in}{1.428820in}}%
\pgfpathlineto{\pgfqpoint{1.331493in}{1.418827in}}%
\pgfpathlineto{\pgfqpoint{1.329278in}{1.417007in}}%
\pgfpathlineto{\pgfqpoint{1.320444in}{1.406698in}}%
\pgfpathlineto{\pgfqpoint{1.317566in}{1.403288in}}%
\pgfpathlineto{\pgfqpoint{1.311877in}{1.394569in}}%
\pgfpathlineto{\pgfqpoint{1.305865in}{1.382440in}}%
\pgfpathlineto{\pgfqpoint{1.305853in}{1.382405in}}%
\pgfpathlineto{\pgfqpoint{1.303990in}{1.370311in}}%
\pgfpathlineto{\pgfqpoint{1.303975in}{1.358182in}}%
\pgfpathlineto{\pgfqpoint{1.305853in}{1.352871in}}%
\pgfpathlineto{\pgfqpoint{1.311080in}{1.346053in}}%
\pgfpathlineto{\pgfqpoint{1.317566in}{1.337278in}}%
\pgfpathlineto{\pgfqpoint{1.320821in}{1.333925in}}%
\pgfpathlineto{\pgfqpoint{1.329278in}{1.323176in}}%
\pgfpathlineto{\pgfqpoint{1.330419in}{1.321796in}}%
\pgfpathlineto{\pgfqpoint{1.340368in}{1.309667in}}%
\pgfpathlineto{\pgfqpoint{1.340991in}{1.308936in}}%
\pgfpathlineto{\pgfqpoint{1.350145in}{1.297538in}}%
\pgfpathlineto{\pgfqpoint{1.352704in}{1.294379in}}%
\pgfpathlineto{\pgfqpoint{1.359822in}{1.285409in}}%
\pgfpathlineto{\pgfqpoint{1.364416in}{1.279971in}}%
\pgfpathlineto{\pgfqpoint{1.370044in}{1.273280in}}%
\pgfpathlineto{\pgfqpoint{1.376129in}{1.266254in}}%
\pgfpathlineto{\pgfqpoint{1.380707in}{1.261151in}}%
\pgfpathlineto{\pgfqpoint{1.387842in}{1.253120in}}%
\pgfpathlineto{\pgfqpoint{1.391610in}{1.249022in}}%
\pgfpathlineto{\pgfqpoint{1.399554in}{1.240307in}}%
\pgfpathlineto{\pgfqpoint{1.402460in}{1.236894in}}%
\pgfpathlineto{\pgfqpoint{1.411267in}{1.226086in}}%
\pgfpathlineto{\pgfqpoint{1.412307in}{1.224765in}}%
\pgfpathlineto{\pgfqpoint{1.422757in}{1.212636in}}%
\pgfpathclose%
\pgfpathmoveto{\pgfqpoint{1.317110in}{1.370311in}}%
\pgfpathlineto{\pgfqpoint{1.317566in}{1.371611in}}%
\pgfpathlineto{\pgfqpoint{1.326183in}{1.382440in}}%
\pgfpathlineto{\pgfqpoint{1.329278in}{1.386230in}}%
\pgfpathlineto{\pgfqpoint{1.340991in}{1.386076in}}%
\pgfpathlineto{\pgfqpoint{1.341639in}{1.382440in}}%
\pgfpathlineto{\pgfqpoint{1.340991in}{1.380210in}}%
\pgfpathlineto{\pgfqpoint{1.335273in}{1.370311in}}%
\pgfpathlineto{\pgfqpoint{1.329278in}{1.366391in}}%
\pgfpathlineto{\pgfqpoint{1.317566in}{1.368321in}}%
\pgfpathclose%
\pgfusepath{fill}%
\end{pgfscope}%
\begin{pgfscope}%
\pgfpathrectangle{\pgfqpoint{0.211875in}{0.211875in}}{\pgfqpoint{1.313625in}{1.279725in}}%
\pgfusepath{clip}%
\pgfsetbuttcap%
\pgfsetroundjoin%
\definecolor{currentfill}{rgb}{0.901975,0.231521,0.249182}%
\pgfsetfillcolor{currentfill}%
\pgfsetlinewidth{0.000000pt}%
\definecolor{currentstroke}{rgb}{0.000000,0.000000,0.000000}%
\pgfsetstrokecolor{currentstroke}%
\pgfsetdash{}{0pt}%
\pgfpathmoveto{\pgfqpoint{0.298562in}{0.294686in}}%
\pgfpathlineto{\pgfqpoint{0.310274in}{0.293003in}}%
\pgfpathlineto{\pgfqpoint{0.312321in}{0.290841in}}%
\pgfpathlineto{\pgfqpoint{0.321987in}{0.290841in}}%
\pgfpathlineto{\pgfqpoint{0.333700in}{0.290841in}}%
\pgfpathlineto{\pgfqpoint{0.345412in}{0.290841in}}%
\pgfpathlineto{\pgfqpoint{0.357125in}{0.290841in}}%
\pgfpathlineto{\pgfqpoint{0.368838in}{0.290841in}}%
\pgfpathlineto{\pgfqpoint{0.370739in}{0.290841in}}%
\pgfpathlineto{\pgfqpoint{0.380550in}{0.299172in}}%
\pgfpathlineto{\pgfqpoint{0.385967in}{0.302970in}}%
\pgfpathlineto{\pgfqpoint{0.392263in}{0.308726in}}%
\pgfpathlineto{\pgfqpoint{0.400370in}{0.315099in}}%
\pgfpathlineto{\pgfqpoint{0.403976in}{0.318627in}}%
\pgfpathlineto{\pgfqpoint{0.413544in}{0.327228in}}%
\pgfpathlineto{\pgfqpoint{0.415688in}{0.329589in}}%
\pgfpathlineto{\pgfqpoint{0.426794in}{0.339357in}}%
\pgfpathlineto{\pgfqpoint{0.427401in}{0.340544in}}%
\pgfpathlineto{\pgfqpoint{0.436470in}{0.351486in}}%
\pgfpathlineto{\pgfqpoint{0.436866in}{0.363615in}}%
\pgfpathlineto{\pgfqpoint{0.427401in}{0.369693in}}%
\pgfpathlineto{\pgfqpoint{0.415688in}{0.369524in}}%
\pgfpathlineto{\pgfqpoint{0.403976in}{0.367910in}}%
\pgfpathlineto{\pgfqpoint{0.392263in}{0.366155in}}%
\pgfpathlineto{\pgfqpoint{0.380550in}{0.364291in}}%
\pgfpathlineto{\pgfqpoint{0.377355in}{0.363615in}}%
\pgfpathlineto{\pgfqpoint{0.368838in}{0.361852in}}%
\pgfpathlineto{\pgfqpoint{0.357125in}{0.359279in}}%
\pgfpathlineto{\pgfqpoint{0.345412in}{0.356778in}}%
\pgfpathlineto{\pgfqpoint{0.333700in}{0.354473in}}%
\pgfpathlineto{\pgfqpoint{0.321987in}{0.352299in}}%
\pgfpathlineto{\pgfqpoint{0.317122in}{0.351486in}}%
\pgfpathlineto{\pgfqpoint{0.310274in}{0.350412in}}%
\pgfpathlineto{\pgfqpoint{0.298562in}{0.348900in}}%
\pgfpathlineto{\pgfqpoint{0.286849in}{0.347572in}}%
\pgfpathlineto{\pgfqpoint{0.286849in}{0.339357in}}%
\pgfpathlineto{\pgfqpoint{0.286849in}{0.327228in}}%
\pgfpathlineto{\pgfqpoint{0.286849in}{0.315099in}}%
\pgfpathlineto{\pgfqpoint{0.286849in}{0.302970in}}%
\pgfpathlineto{\pgfqpoint{0.286849in}{0.293728in}}%
\pgfpathclose%
\pgfusepath{fill}%
\end{pgfscope}%
\begin{pgfscope}%
\pgfpathrectangle{\pgfqpoint{0.211875in}{0.211875in}}{\pgfqpoint{1.313625in}{1.279725in}}%
\pgfusepath{clip}%
\pgfsetbuttcap%
\pgfsetroundjoin%
\definecolor{currentfill}{rgb}{0.901975,0.231521,0.249182}%
\pgfsetfillcolor{currentfill}%
\pgfsetlinewidth{0.000000pt}%
\definecolor{currentstroke}{rgb}{0.000000,0.000000,0.000000}%
\pgfsetstrokecolor{currentstroke}%
\pgfsetdash{}{0pt}%
\pgfpathmoveto{\pgfqpoint{1.036461in}{0.301683in}}%
\pgfpathlineto{\pgfqpoint{1.048174in}{0.295073in}}%
\pgfpathlineto{\pgfqpoint{1.057695in}{0.290841in}}%
\pgfpathlineto{\pgfqpoint{1.059886in}{0.290841in}}%
\pgfpathlineto{\pgfqpoint{1.071599in}{0.290841in}}%
\pgfpathlineto{\pgfqpoint{1.075775in}{0.290841in}}%
\pgfpathlineto{\pgfqpoint{1.083312in}{0.293717in}}%
\pgfpathlineto{\pgfqpoint{1.095024in}{0.299221in}}%
\pgfpathlineto{\pgfqpoint{1.103424in}{0.302970in}}%
\pgfpathlineto{\pgfqpoint{1.106737in}{0.304771in}}%
\pgfpathlineto{\pgfqpoint{1.118450in}{0.310263in}}%
\pgfpathlineto{\pgfqpoint{1.130163in}{0.313647in}}%
\pgfpathlineto{\pgfqpoint{1.137961in}{0.315099in}}%
\pgfpathlineto{\pgfqpoint{1.141875in}{0.316298in}}%
\pgfpathlineto{\pgfqpoint{1.153588in}{0.318811in}}%
\pgfpathlineto{\pgfqpoint{1.165301in}{0.320430in}}%
\pgfpathlineto{\pgfqpoint{1.177013in}{0.322423in}}%
\pgfpathlineto{\pgfqpoint{1.188726in}{0.326939in}}%
\pgfpathlineto{\pgfqpoint{1.189426in}{0.327228in}}%
\pgfpathlineto{\pgfqpoint{1.200439in}{0.332682in}}%
\pgfpathlineto{\pgfqpoint{1.212151in}{0.336451in}}%
\pgfpathlineto{\pgfqpoint{1.219391in}{0.339357in}}%
\pgfpathlineto{\pgfqpoint{1.223864in}{0.342396in}}%
\pgfpathlineto{\pgfqpoint{1.235577in}{0.347305in}}%
\pgfpathlineto{\pgfqpoint{1.247289in}{0.351279in}}%
\pgfpathlineto{\pgfqpoint{1.247762in}{0.351486in}}%
\pgfpathlineto{\pgfqpoint{1.259002in}{0.360163in}}%
\pgfpathlineto{\pgfqpoint{1.269830in}{0.363615in}}%
\pgfpathlineto{\pgfqpoint{1.270715in}{0.366828in}}%
\pgfpathlineto{\pgfqpoint{1.274997in}{0.375743in}}%
\pgfpathlineto{\pgfqpoint{1.272315in}{0.387872in}}%
\pgfpathlineto{\pgfqpoint{1.270715in}{0.390722in}}%
\pgfpathlineto{\pgfqpoint{1.262867in}{0.400001in}}%
\pgfpathlineto{\pgfqpoint{1.259002in}{0.403616in}}%
\pgfpathlineto{\pgfqpoint{1.248151in}{0.412130in}}%
\pgfpathlineto{\pgfqpoint{1.247289in}{0.412646in}}%
\pgfpathlineto{\pgfqpoint{1.235577in}{0.420063in}}%
\pgfpathlineto{\pgfqpoint{1.230489in}{0.424259in}}%
\pgfpathlineto{\pgfqpoint{1.223864in}{0.430015in}}%
\pgfpathlineto{\pgfqpoint{1.219008in}{0.436388in}}%
\pgfpathlineto{\pgfqpoint{1.212151in}{0.447338in}}%
\pgfpathlineto{\pgfqpoint{1.211497in}{0.448517in}}%
\pgfpathlineto{\pgfqpoint{1.205750in}{0.460646in}}%
\pgfpathlineto{\pgfqpoint{1.200439in}{0.466381in}}%
\pgfpathlineto{\pgfqpoint{1.188726in}{0.468329in}}%
\pgfpathlineto{\pgfqpoint{1.177013in}{0.464643in}}%
\pgfpathlineto{\pgfqpoint{1.172382in}{0.460646in}}%
\pgfpathlineto{\pgfqpoint{1.165301in}{0.455043in}}%
\pgfpathlineto{\pgfqpoint{1.158511in}{0.448517in}}%
\pgfpathlineto{\pgfqpoint{1.153588in}{0.444328in}}%
\pgfpathlineto{\pgfqpoint{1.146007in}{0.436388in}}%
\pgfpathlineto{\pgfqpoint{1.141875in}{0.432027in}}%
\pgfpathlineto{\pgfqpoint{1.134865in}{0.424259in}}%
\pgfpathlineto{\pgfqpoint{1.130163in}{0.419029in}}%
\pgfpathlineto{\pgfqpoint{1.123972in}{0.412130in}}%
\pgfpathlineto{\pgfqpoint{1.118450in}{0.406749in}}%
\pgfpathlineto{\pgfqpoint{1.106737in}{0.402078in}}%
\pgfpathlineto{\pgfqpoint{1.101213in}{0.400001in}}%
\pgfpathlineto{\pgfqpoint{1.095024in}{0.397660in}}%
\pgfpathlineto{\pgfqpoint{1.083312in}{0.394418in}}%
\pgfpathlineto{\pgfqpoint{1.071599in}{0.391413in}}%
\pgfpathlineto{\pgfqpoint{1.059886in}{0.389097in}}%
\pgfpathlineto{\pgfqpoint{1.052318in}{0.387872in}}%
\pgfpathlineto{\pgfqpoint{1.048174in}{0.387004in}}%
\pgfpathlineto{\pgfqpoint{1.036461in}{0.378278in}}%
\pgfpathlineto{\pgfqpoint{1.034622in}{0.375743in}}%
\pgfpathlineto{\pgfqpoint{1.025711in}{0.363615in}}%
\pgfpathlineto{\pgfqpoint{1.024748in}{0.362151in}}%
\pgfpathlineto{\pgfqpoint{1.018481in}{0.351486in}}%
\pgfpathlineto{\pgfqpoint{1.017404in}{0.339357in}}%
\pgfpathlineto{\pgfqpoint{1.019216in}{0.327228in}}%
\pgfpathlineto{\pgfqpoint{1.024176in}{0.315099in}}%
\pgfpathlineto{\pgfqpoint{1.024748in}{0.314396in}}%
\pgfpathlineto{\pgfqpoint{1.034624in}{0.302970in}}%
\pgfpathclose%
\pgfpathmoveto{\pgfqpoint{1.058892in}{0.302970in}}%
\pgfpathlineto{\pgfqpoint{1.048174in}{0.312146in}}%
\pgfpathlineto{\pgfqpoint{1.045420in}{0.315099in}}%
\pgfpathlineto{\pgfqpoint{1.046783in}{0.327228in}}%
\pgfpathlineto{\pgfqpoint{1.048174in}{0.330098in}}%
\pgfpathlineto{\pgfqpoint{1.053279in}{0.339357in}}%
\pgfpathlineto{\pgfqpoint{1.059886in}{0.348292in}}%
\pgfpathlineto{\pgfqpoint{1.067962in}{0.351486in}}%
\pgfpathlineto{\pgfqpoint{1.071599in}{0.352825in}}%
\pgfpathlineto{\pgfqpoint{1.083312in}{0.356632in}}%
\pgfpathlineto{\pgfqpoint{1.095024in}{0.359763in}}%
\pgfpathlineto{\pgfqpoint{1.106607in}{0.363615in}}%
\pgfpathlineto{\pgfqpoint{1.106737in}{0.363657in}}%
\pgfpathlineto{\pgfqpoint{1.106862in}{0.363615in}}%
\pgfpathlineto{\pgfqpoint{1.116212in}{0.351486in}}%
\pgfpathlineto{\pgfqpoint{1.118386in}{0.339357in}}%
\pgfpathlineto{\pgfqpoint{1.118109in}{0.327228in}}%
\pgfpathlineto{\pgfqpoint{1.107939in}{0.315099in}}%
\pgfpathlineto{\pgfqpoint{1.106737in}{0.314609in}}%
\pgfpathlineto{\pgfqpoint{1.095024in}{0.309767in}}%
\pgfpathlineto{\pgfqpoint{1.083805in}{0.302970in}}%
\pgfpathlineto{\pgfqpoint{1.083312in}{0.302757in}}%
\pgfpathlineto{\pgfqpoint{1.071599in}{0.299445in}}%
\pgfpathlineto{\pgfqpoint{1.059886in}{0.302500in}}%
\pgfpathclose%
\pgfusepath{fill}%
\end{pgfscope}%
\begin{pgfscope}%
\pgfpathrectangle{\pgfqpoint{0.211875in}{0.211875in}}{\pgfqpoint{1.313625in}{1.279725in}}%
\pgfusepath{clip}%
\pgfsetbuttcap%
\pgfsetroundjoin%
\definecolor{currentfill}{rgb}{0.901975,0.231521,0.249182}%
\pgfsetfillcolor{currentfill}%
\pgfsetlinewidth{0.000000pt}%
\definecolor{currentstroke}{rgb}{0.000000,0.000000,0.000000}%
\pgfsetstrokecolor{currentstroke}%
\pgfsetdash{}{0pt}%
\pgfpathmoveto{\pgfqpoint{0.767069in}{0.516670in}}%
\pgfpathlineto{\pgfqpoint{0.778782in}{0.510281in}}%
\pgfpathlineto{\pgfqpoint{0.790495in}{0.510891in}}%
\pgfpathlineto{\pgfqpoint{0.802207in}{0.513781in}}%
\pgfpathlineto{\pgfqpoint{0.813920in}{0.518401in}}%
\pgfpathlineto{\pgfqpoint{0.825633in}{0.520632in}}%
\pgfpathlineto{\pgfqpoint{0.827234in}{0.521290in}}%
\pgfpathlineto{\pgfqpoint{0.837345in}{0.528908in}}%
\pgfpathlineto{\pgfqpoint{0.841712in}{0.533419in}}%
\pgfpathlineto{\pgfqpoint{0.839609in}{0.545548in}}%
\pgfpathlineto{\pgfqpoint{0.838898in}{0.557677in}}%
\pgfpathlineto{\pgfqpoint{0.838465in}{0.569805in}}%
\pgfpathlineto{\pgfqpoint{0.838072in}{0.581934in}}%
\pgfpathlineto{\pgfqpoint{0.838722in}{0.594063in}}%
\pgfpathlineto{\pgfqpoint{0.839126in}{0.606192in}}%
\pgfpathlineto{\pgfqpoint{0.839870in}{0.618321in}}%
\pgfpathlineto{\pgfqpoint{0.841433in}{0.630450in}}%
\pgfpathlineto{\pgfqpoint{0.839967in}{0.642579in}}%
\pgfpathlineto{\pgfqpoint{0.837345in}{0.645368in}}%
\pgfpathlineto{\pgfqpoint{0.825633in}{0.654257in}}%
\pgfpathlineto{\pgfqpoint{0.825051in}{0.654708in}}%
\pgfpathlineto{\pgfqpoint{0.813920in}{0.662478in}}%
\pgfpathlineto{\pgfqpoint{0.809171in}{0.666836in}}%
\pgfpathlineto{\pgfqpoint{0.802207in}{0.672548in}}%
\pgfpathlineto{\pgfqpoint{0.796120in}{0.678965in}}%
\pgfpathlineto{\pgfqpoint{0.790495in}{0.684527in}}%
\pgfpathlineto{\pgfqpoint{0.784149in}{0.691094in}}%
\pgfpathlineto{\pgfqpoint{0.778782in}{0.699623in}}%
\pgfpathlineto{\pgfqpoint{0.775399in}{0.703223in}}%
\pgfpathlineto{\pgfqpoint{0.778782in}{0.705598in}}%
\pgfpathlineto{\pgfqpoint{0.790495in}{0.712843in}}%
\pgfpathlineto{\pgfqpoint{0.794394in}{0.715352in}}%
\pgfpathlineto{\pgfqpoint{0.802207in}{0.718806in}}%
\pgfpathlineto{\pgfqpoint{0.813920in}{0.723769in}}%
\pgfpathlineto{\pgfqpoint{0.822590in}{0.727481in}}%
\pgfpathlineto{\pgfqpoint{0.825633in}{0.728658in}}%
\pgfpathlineto{\pgfqpoint{0.837345in}{0.735195in}}%
\pgfpathlineto{\pgfqpoint{0.841597in}{0.739610in}}%
\pgfpathlineto{\pgfqpoint{0.841596in}{0.751739in}}%
\pgfpathlineto{\pgfqpoint{0.841305in}{0.763867in}}%
\pgfpathlineto{\pgfqpoint{0.837345in}{0.771066in}}%
\pgfpathlineto{\pgfqpoint{0.825633in}{0.775145in}}%
\pgfpathlineto{\pgfqpoint{0.813920in}{0.775376in}}%
\pgfpathlineto{\pgfqpoint{0.804062in}{0.775996in}}%
\pgfpathlineto{\pgfqpoint{0.802207in}{0.776111in}}%
\pgfpathlineto{\pgfqpoint{0.790495in}{0.777741in}}%
\pgfpathlineto{\pgfqpoint{0.778782in}{0.779845in}}%
\pgfpathlineto{\pgfqpoint{0.767069in}{0.783430in}}%
\pgfpathlineto{\pgfqpoint{0.755356in}{0.787191in}}%
\pgfpathlineto{\pgfqpoint{0.752448in}{0.788125in}}%
\pgfpathlineto{\pgfqpoint{0.743644in}{0.790346in}}%
\pgfpathlineto{\pgfqpoint{0.731931in}{0.793378in}}%
\pgfpathlineto{\pgfqpoint{0.720218in}{0.796194in}}%
\pgfpathlineto{\pgfqpoint{0.708506in}{0.798865in}}%
\pgfpathlineto{\pgfqpoint{0.702187in}{0.800254in}}%
\pgfpathlineto{\pgfqpoint{0.696793in}{0.801709in}}%
\pgfpathlineto{\pgfqpoint{0.685080in}{0.805532in}}%
\pgfpathlineto{\pgfqpoint{0.673368in}{0.808914in}}%
\pgfpathlineto{\pgfqpoint{0.661655in}{0.812093in}}%
\pgfpathlineto{\pgfqpoint{0.660598in}{0.812383in}}%
\pgfpathlineto{\pgfqpoint{0.649942in}{0.816159in}}%
\pgfpathlineto{\pgfqpoint{0.638230in}{0.819608in}}%
\pgfpathlineto{\pgfqpoint{0.628579in}{0.824512in}}%
\pgfpathlineto{\pgfqpoint{0.626517in}{0.826383in}}%
\pgfpathlineto{\pgfqpoint{0.615889in}{0.836641in}}%
\pgfpathlineto{\pgfqpoint{0.614804in}{0.837881in}}%
\pgfpathlineto{\pgfqpoint{0.605331in}{0.848770in}}%
\pgfpathlineto{\pgfqpoint{0.603091in}{0.851720in}}%
\pgfpathlineto{\pgfqpoint{0.595959in}{0.860898in}}%
\pgfpathlineto{\pgfqpoint{0.591379in}{0.866644in}}%
\pgfpathlineto{\pgfqpoint{0.586441in}{0.873027in}}%
\pgfpathlineto{\pgfqpoint{0.579666in}{0.883222in}}%
\pgfpathlineto{\pgfqpoint{0.578277in}{0.885156in}}%
\pgfpathlineto{\pgfqpoint{0.568017in}{0.897285in}}%
\pgfpathlineto{\pgfqpoint{0.567953in}{0.897359in}}%
\pgfpathlineto{\pgfqpoint{0.556241in}{0.908135in}}%
\pgfpathlineto{\pgfqpoint{0.554312in}{0.909414in}}%
\pgfpathlineto{\pgfqpoint{0.544528in}{0.916439in}}%
\pgfpathlineto{\pgfqpoint{0.537573in}{0.921543in}}%
\pgfpathlineto{\pgfqpoint{0.532815in}{0.925694in}}%
\pgfpathlineto{\pgfqpoint{0.527105in}{0.933672in}}%
\pgfpathlineto{\pgfqpoint{0.525323in}{0.945801in}}%
\pgfpathlineto{\pgfqpoint{0.522654in}{0.957929in}}%
\pgfpathlineto{\pgfqpoint{0.521103in}{0.961907in}}%
\pgfpathlineto{\pgfqpoint{0.517858in}{0.970058in}}%
\pgfpathlineto{\pgfqpoint{0.510852in}{0.982187in}}%
\pgfpathlineto{\pgfqpoint{0.509390in}{0.984303in}}%
\pgfpathlineto{\pgfqpoint{0.502232in}{0.994316in}}%
\pgfpathlineto{\pgfqpoint{0.497677in}{0.999941in}}%
\pgfpathlineto{\pgfqpoint{0.492614in}{1.006445in}}%
\pgfpathlineto{\pgfqpoint{0.485965in}{1.015197in}}%
\pgfpathlineto{\pgfqpoint{0.483436in}{1.018574in}}%
\pgfpathlineto{\pgfqpoint{0.474757in}{1.030703in}}%
\pgfpathlineto{\pgfqpoint{0.474252in}{1.031386in}}%
\pgfpathlineto{\pgfqpoint{0.465561in}{1.042832in}}%
\pgfpathlineto{\pgfqpoint{0.462539in}{1.046637in}}%
\pgfpathlineto{\pgfqpoint{0.455919in}{1.054960in}}%
\pgfpathlineto{\pgfqpoint{0.450827in}{1.059880in}}%
\pgfpathlineto{\pgfqpoint{0.443349in}{1.067089in}}%
\pgfpathlineto{\pgfqpoint{0.439114in}{1.071039in}}%
\pgfpathlineto{\pgfqpoint{0.430024in}{1.079218in}}%
\pgfpathlineto{\pgfqpoint{0.427401in}{1.081616in}}%
\pgfpathlineto{\pgfqpoint{0.416373in}{1.091347in}}%
\pgfpathlineto{\pgfqpoint{0.415688in}{1.091956in}}%
\pgfpathlineto{\pgfqpoint{0.403976in}{1.101729in}}%
\pgfpathlineto{\pgfqpoint{0.401743in}{1.103476in}}%
\pgfpathlineto{\pgfqpoint{0.392263in}{1.110832in}}%
\pgfpathlineto{\pgfqpoint{0.385889in}{1.115605in}}%
\pgfpathlineto{\pgfqpoint{0.380550in}{1.119805in}}%
\pgfpathlineto{\pgfqpoint{0.369971in}{1.127734in}}%
\pgfpathlineto{\pgfqpoint{0.368838in}{1.128645in}}%
\pgfpathlineto{\pgfqpoint{0.357125in}{1.137094in}}%
\pgfpathlineto{\pgfqpoint{0.353191in}{1.139863in}}%
\pgfpathlineto{\pgfqpoint{0.345412in}{1.146386in}}%
\pgfpathlineto{\pgfqpoint{0.338477in}{1.151991in}}%
\pgfpathlineto{\pgfqpoint{0.333700in}{1.156517in}}%
\pgfpathlineto{\pgfqpoint{0.325158in}{1.164120in}}%
\pgfpathlineto{\pgfqpoint{0.321987in}{1.167244in}}%
\pgfpathlineto{\pgfqpoint{0.312042in}{1.176249in}}%
\pgfpathlineto{\pgfqpoint{0.310274in}{1.177966in}}%
\pgfpathlineto{\pgfqpoint{0.298562in}{1.188332in}}%
\pgfpathlineto{\pgfqpoint{0.298508in}{1.188378in}}%
\pgfpathlineto{\pgfqpoint{0.286849in}{1.200270in}}%
\pgfpathlineto{\pgfqpoint{0.286849in}{1.188378in}}%
\pgfpathlineto{\pgfqpoint{0.286849in}{1.176249in}}%
\pgfpathlineto{\pgfqpoint{0.286849in}{1.164120in}}%
\pgfpathlineto{\pgfqpoint{0.286849in}{1.151991in}}%
\pgfpathlineto{\pgfqpoint{0.286849in}{1.139863in}}%
\pgfpathlineto{\pgfqpoint{0.286849in}{1.128207in}}%
\pgfpathlineto{\pgfqpoint{0.293531in}{1.139863in}}%
\pgfpathlineto{\pgfqpoint{0.298562in}{1.141935in}}%
\pgfpathlineto{\pgfqpoint{0.301435in}{1.139863in}}%
\pgfpathlineto{\pgfqpoint{0.310274in}{1.134404in}}%
\pgfpathlineto{\pgfqpoint{0.320665in}{1.127734in}}%
\pgfpathlineto{\pgfqpoint{0.321987in}{1.126945in}}%
\pgfpathlineto{\pgfqpoint{0.333700in}{1.119841in}}%
\pgfpathlineto{\pgfqpoint{0.340628in}{1.115605in}}%
\pgfpathlineto{\pgfqpoint{0.345412in}{1.112774in}}%
\pgfpathlineto{\pgfqpoint{0.357125in}{1.105753in}}%
\pgfpathlineto{\pgfqpoint{0.360993in}{1.103476in}}%
\pgfpathlineto{\pgfqpoint{0.368838in}{1.098806in}}%
\pgfpathlineto{\pgfqpoint{0.380550in}{1.091688in}}%
\pgfpathlineto{\pgfqpoint{0.381024in}{1.091347in}}%
\pgfpathlineto{\pgfqpoint{0.392263in}{1.082590in}}%
\pgfpathlineto{\pgfqpoint{0.396389in}{1.079218in}}%
\pgfpathlineto{\pgfqpoint{0.403976in}{1.072805in}}%
\pgfpathlineto{\pgfqpoint{0.410890in}{1.067089in}}%
\pgfpathlineto{\pgfqpoint{0.415688in}{1.061632in}}%
\pgfpathlineto{\pgfqpoint{0.422144in}{1.054960in}}%
\pgfpathlineto{\pgfqpoint{0.427401in}{1.047995in}}%
\pgfpathlineto{\pgfqpoint{0.431232in}{1.042832in}}%
\pgfpathlineto{\pgfqpoint{0.439114in}{1.031870in}}%
\pgfpathlineto{\pgfqpoint{0.439932in}{1.030703in}}%
\pgfpathlineto{\pgfqpoint{0.447891in}{1.018574in}}%
\pgfpathlineto{\pgfqpoint{0.450827in}{1.013938in}}%
\pgfpathlineto{\pgfqpoint{0.455549in}{1.006445in}}%
\pgfpathlineto{\pgfqpoint{0.462539in}{0.996046in}}%
\pgfpathlineto{\pgfqpoint{0.463654in}{0.994316in}}%
\pgfpathlineto{\pgfqpoint{0.471769in}{0.982187in}}%
\pgfpathlineto{\pgfqpoint{0.474252in}{0.978892in}}%
\pgfpathlineto{\pgfqpoint{0.480342in}{0.970058in}}%
\pgfpathlineto{\pgfqpoint{0.485420in}{0.957929in}}%
\pgfpathlineto{\pgfqpoint{0.485965in}{0.955647in}}%
\pgfpathlineto{\pgfqpoint{0.488059in}{0.945801in}}%
\pgfpathlineto{\pgfqpoint{0.488455in}{0.933672in}}%
\pgfpathlineto{\pgfqpoint{0.487990in}{0.921543in}}%
\pgfpathlineto{\pgfqpoint{0.487213in}{0.909414in}}%
\pgfpathlineto{\pgfqpoint{0.497677in}{0.899532in}}%
\pgfpathlineto{\pgfqpoint{0.502622in}{0.897285in}}%
\pgfpathlineto{\pgfqpoint{0.509390in}{0.894828in}}%
\pgfpathlineto{\pgfqpoint{0.521103in}{0.889832in}}%
\pgfpathlineto{\pgfqpoint{0.531153in}{0.885156in}}%
\pgfpathlineto{\pgfqpoint{0.532815in}{0.884309in}}%
\pgfpathlineto{\pgfqpoint{0.544528in}{0.874184in}}%
\pgfpathlineto{\pgfqpoint{0.545559in}{0.873027in}}%
\pgfpathlineto{\pgfqpoint{0.553611in}{0.860898in}}%
\pgfpathlineto{\pgfqpoint{0.556241in}{0.857339in}}%
\pgfpathlineto{\pgfqpoint{0.563115in}{0.848770in}}%
\pgfpathlineto{\pgfqpoint{0.567953in}{0.844036in}}%
\pgfpathlineto{\pgfqpoint{0.575056in}{0.836641in}}%
\pgfpathlineto{\pgfqpoint{0.579666in}{0.832455in}}%
\pgfpathlineto{\pgfqpoint{0.588085in}{0.824512in}}%
\pgfpathlineto{\pgfqpoint{0.591379in}{0.821891in}}%
\pgfpathlineto{\pgfqpoint{0.603091in}{0.813227in}}%
\pgfpathlineto{\pgfqpoint{0.604214in}{0.812383in}}%
\pgfpathlineto{\pgfqpoint{0.614804in}{0.807214in}}%
\pgfpathlineto{\pgfqpoint{0.626517in}{0.803152in}}%
\pgfpathlineto{\pgfqpoint{0.638230in}{0.800998in}}%
\pgfpathlineto{\pgfqpoint{0.643216in}{0.800254in}}%
\pgfpathlineto{\pgfqpoint{0.649942in}{0.799430in}}%
\pgfpathlineto{\pgfqpoint{0.661655in}{0.797123in}}%
\pgfpathlineto{\pgfqpoint{0.673368in}{0.794423in}}%
\pgfpathlineto{\pgfqpoint{0.685080in}{0.791449in}}%
\pgfpathlineto{\pgfqpoint{0.696793in}{0.788585in}}%
\pgfpathlineto{\pgfqpoint{0.698874in}{0.788125in}}%
\pgfpathlineto{\pgfqpoint{0.708506in}{0.784703in}}%
\pgfpathlineto{\pgfqpoint{0.720218in}{0.780564in}}%
\pgfpathlineto{\pgfqpoint{0.731931in}{0.776534in}}%
\pgfpathlineto{\pgfqpoint{0.733537in}{0.775996in}}%
\pgfpathlineto{\pgfqpoint{0.743644in}{0.771529in}}%
\pgfpathlineto{\pgfqpoint{0.755356in}{0.767590in}}%
\pgfpathlineto{\pgfqpoint{0.767069in}{0.764408in}}%
\pgfpathlineto{\pgfqpoint{0.769580in}{0.763867in}}%
\pgfpathlineto{\pgfqpoint{0.778782in}{0.761497in}}%
\pgfpathlineto{\pgfqpoint{0.790495in}{0.760077in}}%
\pgfpathlineto{\pgfqpoint{0.802207in}{0.760247in}}%
\pgfpathlineto{\pgfqpoint{0.813920in}{0.758717in}}%
\pgfpathlineto{\pgfqpoint{0.818715in}{0.751739in}}%
\pgfpathlineto{\pgfqpoint{0.818997in}{0.739610in}}%
\pgfpathlineto{\pgfqpoint{0.813920in}{0.736892in}}%
\pgfpathlineto{\pgfqpoint{0.802207in}{0.732992in}}%
\pgfpathlineto{\pgfqpoint{0.790495in}{0.730610in}}%
\pgfpathlineto{\pgfqpoint{0.778782in}{0.730083in}}%
\pgfpathlineto{\pgfqpoint{0.767069in}{0.732927in}}%
\pgfpathlineto{\pgfqpoint{0.755513in}{0.739610in}}%
\pgfpathlineto{\pgfqpoint{0.755356in}{0.739669in}}%
\pgfpathlineto{\pgfqpoint{0.743644in}{0.743420in}}%
\pgfpathlineto{\pgfqpoint{0.731931in}{0.746614in}}%
\pgfpathlineto{\pgfqpoint{0.720218in}{0.751213in}}%
\pgfpathlineto{\pgfqpoint{0.717855in}{0.751739in}}%
\pgfpathlineto{\pgfqpoint{0.708506in}{0.753433in}}%
\pgfpathlineto{\pgfqpoint{0.696793in}{0.754635in}}%
\pgfpathlineto{\pgfqpoint{0.685080in}{0.756574in}}%
\pgfpathlineto{\pgfqpoint{0.673368in}{0.758757in}}%
\pgfpathlineto{\pgfqpoint{0.661655in}{0.762034in}}%
\pgfpathlineto{\pgfqpoint{0.656202in}{0.763867in}}%
\pgfpathlineto{\pgfqpoint{0.649942in}{0.765607in}}%
\pgfpathlineto{\pgfqpoint{0.638230in}{0.768879in}}%
\pgfpathlineto{\pgfqpoint{0.626517in}{0.772770in}}%
\pgfpathlineto{\pgfqpoint{0.614804in}{0.775810in}}%
\pgfpathlineto{\pgfqpoint{0.613916in}{0.775996in}}%
\pgfpathlineto{\pgfqpoint{0.603091in}{0.778232in}}%
\pgfpathlineto{\pgfqpoint{0.591379in}{0.780002in}}%
\pgfpathlineto{\pgfqpoint{0.579666in}{0.781863in}}%
\pgfpathlineto{\pgfqpoint{0.567953in}{0.783481in}}%
\pgfpathlineto{\pgfqpoint{0.556241in}{0.785498in}}%
\pgfpathlineto{\pgfqpoint{0.544528in}{0.787895in}}%
\pgfpathlineto{\pgfqpoint{0.543665in}{0.788125in}}%
\pgfpathlineto{\pgfqpoint{0.532815in}{0.792097in}}%
\pgfpathlineto{\pgfqpoint{0.521103in}{0.796904in}}%
\pgfpathlineto{\pgfqpoint{0.514542in}{0.800254in}}%
\pgfpathlineto{\pgfqpoint{0.509390in}{0.803545in}}%
\pgfpathlineto{\pgfqpoint{0.497677in}{0.810677in}}%
\pgfpathlineto{\pgfqpoint{0.494789in}{0.812383in}}%
\pgfpathlineto{\pgfqpoint{0.485965in}{0.818825in}}%
\pgfpathlineto{\pgfqpoint{0.475858in}{0.824512in}}%
\pgfpathlineto{\pgfqpoint{0.474252in}{0.825854in}}%
\pgfpathlineto{\pgfqpoint{0.466348in}{0.836641in}}%
\pgfpathlineto{\pgfqpoint{0.465560in}{0.848770in}}%
\pgfpathlineto{\pgfqpoint{0.466267in}{0.860898in}}%
\pgfpathlineto{\pgfqpoint{0.465931in}{0.873027in}}%
\pgfpathlineto{\pgfqpoint{0.465767in}{0.885156in}}%
\pgfpathlineto{\pgfqpoint{0.462539in}{0.890634in}}%
\pgfpathlineto{\pgfqpoint{0.457674in}{0.897285in}}%
\pgfpathlineto{\pgfqpoint{0.450827in}{0.902187in}}%
\pgfpathlineto{\pgfqpoint{0.445149in}{0.909414in}}%
\pgfpathlineto{\pgfqpoint{0.439114in}{0.915456in}}%
\pgfpathlineto{\pgfqpoint{0.433863in}{0.921543in}}%
\pgfpathlineto{\pgfqpoint{0.427401in}{0.929498in}}%
\pgfpathlineto{\pgfqpoint{0.423682in}{0.933672in}}%
\pgfpathlineto{\pgfqpoint{0.415688in}{0.943555in}}%
\pgfpathlineto{\pgfqpoint{0.413612in}{0.945801in}}%
\pgfpathlineto{\pgfqpoint{0.403976in}{0.957169in}}%
\pgfpathlineto{\pgfqpoint{0.403201in}{0.957929in}}%
\pgfpathlineto{\pgfqpoint{0.392263in}{0.969485in}}%
\pgfpathlineto{\pgfqpoint{0.391661in}{0.970058in}}%
\pgfpathlineto{\pgfqpoint{0.381240in}{0.982187in}}%
\pgfpathlineto{\pgfqpoint{0.380550in}{0.983075in}}%
\pgfpathlineto{\pgfqpoint{0.370335in}{0.994316in}}%
\pgfpathlineto{\pgfqpoint{0.368838in}{0.995989in}}%
\pgfpathlineto{\pgfqpoint{0.358183in}{1.006445in}}%
\pgfpathlineto{\pgfqpoint{0.357125in}{1.007567in}}%
\pgfpathlineto{\pgfqpoint{0.345412in}{1.018537in}}%
\pgfpathlineto{\pgfqpoint{0.345371in}{1.018574in}}%
\pgfpathlineto{\pgfqpoint{0.333700in}{1.029015in}}%
\pgfpathlineto{\pgfqpoint{0.331739in}{1.030703in}}%
\pgfpathlineto{\pgfqpoint{0.321987in}{1.038939in}}%
\pgfpathlineto{\pgfqpoint{0.317251in}{1.042832in}}%
\pgfpathlineto{\pgfqpoint{0.310274in}{1.048654in}}%
\pgfpathlineto{\pgfqpoint{0.302421in}{1.054960in}}%
\pgfpathlineto{\pgfqpoint{0.298562in}{1.058349in}}%
\pgfpathlineto{\pgfqpoint{0.288850in}{1.067089in}}%
\pgfpathlineto{\pgfqpoint{0.286849in}{1.069147in}}%
\pgfpathlineto{\pgfqpoint{0.286849in}{1.067089in}}%
\pgfpathlineto{\pgfqpoint{0.286849in}{1.054960in}}%
\pgfpathlineto{\pgfqpoint{0.286849in}{1.042832in}}%
\pgfpathlineto{\pgfqpoint{0.286849in}{1.030703in}}%
\pgfpathlineto{\pgfqpoint{0.286849in}{1.018574in}}%
\pgfpathlineto{\pgfqpoint{0.286849in}{1.006445in}}%
\pgfpathlineto{\pgfqpoint{0.286849in}{0.996732in}}%
\pgfpathlineto{\pgfqpoint{0.291327in}{0.994316in}}%
\pgfpathlineto{\pgfqpoint{0.298562in}{0.990408in}}%
\pgfpathlineto{\pgfqpoint{0.310274in}{0.983561in}}%
\pgfpathlineto{\pgfqpoint{0.312393in}{0.982187in}}%
\pgfpathlineto{\pgfqpoint{0.321987in}{0.976194in}}%
\pgfpathlineto{\pgfqpoint{0.330773in}{0.970058in}}%
\pgfpathlineto{\pgfqpoint{0.333700in}{0.968334in}}%
\pgfpathlineto{\pgfqpoint{0.345412in}{0.961062in}}%
\pgfpathlineto{\pgfqpoint{0.350224in}{0.957929in}}%
\pgfpathlineto{\pgfqpoint{0.357125in}{0.953703in}}%
\pgfpathlineto{\pgfqpoint{0.368838in}{0.946292in}}%
\pgfpathlineto{\pgfqpoint{0.369523in}{0.945801in}}%
\pgfpathlineto{\pgfqpoint{0.380550in}{0.938908in}}%
\pgfpathlineto{\pgfqpoint{0.388058in}{0.933672in}}%
\pgfpathlineto{\pgfqpoint{0.392263in}{0.930528in}}%
\pgfpathlineto{\pgfqpoint{0.402673in}{0.921543in}}%
\pgfpathlineto{\pgfqpoint{0.403976in}{0.920475in}}%
\pgfpathlineto{\pgfqpoint{0.415688in}{0.909718in}}%
\pgfpathlineto{\pgfqpoint{0.415976in}{0.909414in}}%
\pgfpathlineto{\pgfqpoint{0.427401in}{0.898138in}}%
\pgfpathlineto{\pgfqpoint{0.428164in}{0.897285in}}%
\pgfpathlineto{\pgfqpoint{0.439114in}{0.885676in}}%
\pgfpathlineto{\pgfqpoint{0.439530in}{0.885156in}}%
\pgfpathlineto{\pgfqpoint{0.445946in}{0.873027in}}%
\pgfpathlineto{\pgfqpoint{0.446141in}{0.860898in}}%
\pgfpathlineto{\pgfqpoint{0.443997in}{0.848770in}}%
\pgfpathlineto{\pgfqpoint{0.439151in}{0.836641in}}%
\pgfpathlineto{\pgfqpoint{0.443056in}{0.824512in}}%
\pgfpathlineto{\pgfqpoint{0.450827in}{0.818299in}}%
\pgfpathlineto{\pgfqpoint{0.462539in}{0.812795in}}%
\pgfpathlineto{\pgfqpoint{0.463437in}{0.812383in}}%
\pgfpathlineto{\pgfqpoint{0.474252in}{0.807653in}}%
\pgfpathlineto{\pgfqpoint{0.485965in}{0.801925in}}%
\pgfpathlineto{\pgfqpoint{0.489023in}{0.800254in}}%
\pgfpathlineto{\pgfqpoint{0.497677in}{0.795900in}}%
\pgfpathlineto{\pgfqpoint{0.509390in}{0.790452in}}%
\pgfpathlineto{\pgfqpoint{0.514383in}{0.788125in}}%
\pgfpathlineto{\pgfqpoint{0.521103in}{0.785438in}}%
\pgfpathlineto{\pgfqpoint{0.532815in}{0.780896in}}%
\pgfpathlineto{\pgfqpoint{0.544528in}{0.776986in}}%
\pgfpathlineto{\pgfqpoint{0.548447in}{0.775996in}}%
\pgfpathlineto{\pgfqpoint{0.556241in}{0.774349in}}%
\pgfpathlineto{\pgfqpoint{0.567953in}{0.772796in}}%
\pgfpathlineto{\pgfqpoint{0.579666in}{0.771239in}}%
\pgfpathlineto{\pgfqpoint{0.591379in}{0.768897in}}%
\pgfpathlineto{\pgfqpoint{0.603091in}{0.766569in}}%
\pgfpathlineto{\pgfqpoint{0.614804in}{0.764662in}}%
\pgfpathlineto{\pgfqpoint{0.618340in}{0.763867in}}%
\pgfpathlineto{\pgfqpoint{0.626517in}{0.761398in}}%
\pgfpathlineto{\pgfqpoint{0.638230in}{0.757412in}}%
\pgfpathlineto{\pgfqpoint{0.649942in}{0.753985in}}%
\pgfpathlineto{\pgfqpoint{0.658118in}{0.751739in}}%
\pgfpathlineto{\pgfqpoint{0.661655in}{0.750470in}}%
\pgfpathlineto{\pgfqpoint{0.673368in}{0.746401in}}%
\pgfpathlineto{\pgfqpoint{0.685080in}{0.743369in}}%
\pgfpathlineto{\pgfqpoint{0.696793in}{0.741174in}}%
\pgfpathlineto{\pgfqpoint{0.703524in}{0.739610in}}%
\pgfpathlineto{\pgfqpoint{0.708506in}{0.737137in}}%
\pgfpathlineto{\pgfqpoint{0.720218in}{0.730151in}}%
\pgfpathlineto{\pgfqpoint{0.725277in}{0.727481in}}%
\pgfpathlineto{\pgfqpoint{0.731465in}{0.715352in}}%
\pgfpathlineto{\pgfqpoint{0.720218in}{0.711083in}}%
\pgfpathlineto{\pgfqpoint{0.708506in}{0.707773in}}%
\pgfpathlineto{\pgfqpoint{0.696793in}{0.703499in}}%
\pgfpathlineto{\pgfqpoint{0.696353in}{0.703223in}}%
\pgfpathlineto{\pgfqpoint{0.685080in}{0.696993in}}%
\pgfpathlineto{\pgfqpoint{0.680641in}{0.691094in}}%
\pgfpathlineto{\pgfqpoint{0.677379in}{0.678965in}}%
\pgfpathlineto{\pgfqpoint{0.674123in}{0.666836in}}%
\pgfpathlineto{\pgfqpoint{0.673368in}{0.662110in}}%
\pgfpathlineto{\pgfqpoint{0.671797in}{0.654708in}}%
\pgfpathlineto{\pgfqpoint{0.672388in}{0.642579in}}%
\pgfpathlineto{\pgfqpoint{0.673368in}{0.631501in}}%
\pgfpathlineto{\pgfqpoint{0.673466in}{0.630450in}}%
\pgfpathlineto{\pgfqpoint{0.673368in}{0.627288in}}%
\pgfpathlineto{\pgfqpoint{0.673042in}{0.618321in}}%
\pgfpathlineto{\pgfqpoint{0.671684in}{0.606192in}}%
\pgfpathlineto{\pgfqpoint{0.670465in}{0.594063in}}%
\pgfpathlineto{\pgfqpoint{0.670406in}{0.581934in}}%
\pgfpathlineto{\pgfqpoint{0.670806in}{0.569805in}}%
\pgfpathlineto{\pgfqpoint{0.671155in}{0.557677in}}%
\pgfpathlineto{\pgfqpoint{0.673368in}{0.549924in}}%
\pgfpathlineto{\pgfqpoint{0.675352in}{0.545548in}}%
\pgfpathlineto{\pgfqpoint{0.685080in}{0.534991in}}%
\pgfpathlineto{\pgfqpoint{0.696793in}{0.533695in}}%
\pgfpathlineto{\pgfqpoint{0.708506in}{0.534993in}}%
\pgfpathlineto{\pgfqpoint{0.720218in}{0.535216in}}%
\pgfpathlineto{\pgfqpoint{0.728371in}{0.533419in}}%
\pgfpathlineto{\pgfqpoint{0.731931in}{0.532522in}}%
\pgfpathlineto{\pgfqpoint{0.743644in}{0.526761in}}%
\pgfpathlineto{\pgfqpoint{0.755356in}{0.522379in}}%
\pgfpathlineto{\pgfqpoint{0.757800in}{0.521290in}}%
\pgfpathclose%
\pgfpathmoveto{\pgfqpoint{0.765624in}{0.545548in}}%
\pgfpathlineto{\pgfqpoint{0.755356in}{0.550072in}}%
\pgfpathlineto{\pgfqpoint{0.743644in}{0.554528in}}%
\pgfpathlineto{\pgfqpoint{0.735300in}{0.557677in}}%
\pgfpathlineto{\pgfqpoint{0.731931in}{0.559209in}}%
\pgfpathlineto{\pgfqpoint{0.721055in}{0.569805in}}%
\pgfpathlineto{\pgfqpoint{0.720218in}{0.571339in}}%
\pgfpathlineto{\pgfqpoint{0.712893in}{0.581934in}}%
\pgfpathlineto{\pgfqpoint{0.708506in}{0.591872in}}%
\pgfpathlineto{\pgfqpoint{0.707662in}{0.594063in}}%
\pgfpathlineto{\pgfqpoint{0.704578in}{0.606192in}}%
\pgfpathlineto{\pgfqpoint{0.702677in}{0.618321in}}%
\pgfpathlineto{\pgfqpoint{0.700748in}{0.630450in}}%
\pgfpathlineto{\pgfqpoint{0.697792in}{0.642579in}}%
\pgfpathlineto{\pgfqpoint{0.696793in}{0.645947in}}%
\pgfpathlineto{\pgfqpoint{0.694092in}{0.654708in}}%
\pgfpathlineto{\pgfqpoint{0.690948in}{0.666836in}}%
\pgfpathlineto{\pgfqpoint{0.691597in}{0.678965in}}%
\pgfpathlineto{\pgfqpoint{0.696793in}{0.688672in}}%
\pgfpathlineto{\pgfqpoint{0.700429in}{0.691094in}}%
\pgfpathlineto{\pgfqpoint{0.708506in}{0.693686in}}%
\pgfpathlineto{\pgfqpoint{0.720218in}{0.694951in}}%
\pgfpathlineto{\pgfqpoint{0.731931in}{0.693655in}}%
\pgfpathlineto{\pgfqpoint{0.740361in}{0.691094in}}%
\pgfpathlineto{\pgfqpoint{0.743644in}{0.689453in}}%
\pgfpathlineto{\pgfqpoint{0.755356in}{0.682915in}}%
\pgfpathlineto{\pgfqpoint{0.760678in}{0.678965in}}%
\pgfpathlineto{\pgfqpoint{0.767069in}{0.669304in}}%
\pgfpathlineto{\pgfqpoint{0.768665in}{0.666836in}}%
\pgfpathlineto{\pgfqpoint{0.777965in}{0.654708in}}%
\pgfpathlineto{\pgfqpoint{0.778782in}{0.653190in}}%
\pgfpathlineto{\pgfqpoint{0.785588in}{0.642579in}}%
\pgfpathlineto{\pgfqpoint{0.790495in}{0.636562in}}%
\pgfpathlineto{\pgfqpoint{0.796576in}{0.630450in}}%
\pgfpathlineto{\pgfqpoint{0.802207in}{0.625320in}}%
\pgfpathlineto{\pgfqpoint{0.808657in}{0.618321in}}%
\pgfpathlineto{\pgfqpoint{0.807837in}{0.606192in}}%
\pgfpathlineto{\pgfqpoint{0.805216in}{0.594063in}}%
\pgfpathlineto{\pgfqpoint{0.803863in}{0.581934in}}%
\pgfpathlineto{\pgfqpoint{0.803241in}{0.569805in}}%
\pgfpathlineto{\pgfqpoint{0.802657in}{0.557677in}}%
\pgfpathlineto{\pgfqpoint{0.802207in}{0.556979in}}%
\pgfpathlineto{\pgfqpoint{0.790495in}{0.547843in}}%
\pgfpathlineto{\pgfqpoint{0.785170in}{0.545548in}}%
\pgfpathlineto{\pgfqpoint{0.778782in}{0.543294in}}%
\pgfpathlineto{\pgfqpoint{0.767069in}{0.544915in}}%
\pgfpathclose%
\pgfusepath{fill}%
\end{pgfscope}%
\begin{pgfscope}%
\pgfpathrectangle{\pgfqpoint{0.211875in}{0.211875in}}{\pgfqpoint{1.313625in}{1.279725in}}%
\pgfusepath{clip}%
\pgfsetbuttcap%
\pgfsetroundjoin%
\definecolor{currentfill}{rgb}{0.901975,0.231521,0.249182}%
\pgfsetfillcolor{currentfill}%
\pgfsetlinewidth{0.000000pt}%
\definecolor{currentstroke}{rgb}{0.000000,0.000000,0.000000}%
\pgfsetstrokecolor{currentstroke}%
\pgfsetdash{}{0pt}%
\pgfpathmoveto{\pgfqpoint{1.247289in}{0.630156in}}%
\pgfpathlineto{\pgfqpoint{1.259002in}{0.627648in}}%
\pgfpathlineto{\pgfqpoint{1.264007in}{0.630450in}}%
\pgfpathlineto{\pgfqpoint{1.270715in}{0.636612in}}%
\pgfpathlineto{\pgfqpoint{1.278544in}{0.642579in}}%
\pgfpathlineto{\pgfqpoint{1.282427in}{0.645461in}}%
\pgfpathlineto{\pgfqpoint{1.294140in}{0.651847in}}%
\pgfpathlineto{\pgfqpoint{1.298382in}{0.654708in}}%
\pgfpathlineto{\pgfqpoint{1.305853in}{0.657851in}}%
\pgfpathlineto{\pgfqpoint{1.317566in}{0.663060in}}%
\pgfpathlineto{\pgfqpoint{1.323965in}{0.666836in}}%
\pgfpathlineto{\pgfqpoint{1.329278in}{0.669845in}}%
\pgfpathlineto{\pgfqpoint{1.339235in}{0.678965in}}%
\pgfpathlineto{\pgfqpoint{1.340991in}{0.680208in}}%
\pgfpathlineto{\pgfqpoint{1.352704in}{0.688645in}}%
\pgfpathlineto{\pgfqpoint{1.357123in}{0.691094in}}%
\pgfpathlineto{\pgfqpoint{1.364416in}{0.699443in}}%
\pgfpathlineto{\pgfqpoint{1.376129in}{0.695391in}}%
\pgfpathlineto{\pgfqpoint{1.379595in}{0.691094in}}%
\pgfpathlineto{\pgfqpoint{1.387842in}{0.685631in}}%
\pgfpathlineto{\pgfqpoint{1.396539in}{0.678965in}}%
\pgfpathlineto{\pgfqpoint{1.399554in}{0.676220in}}%
\pgfpathlineto{\pgfqpoint{1.411267in}{0.667480in}}%
\pgfpathlineto{\pgfqpoint{1.412730in}{0.666836in}}%
\pgfpathlineto{\pgfqpoint{1.422980in}{0.662441in}}%
\pgfpathlineto{\pgfqpoint{1.434692in}{0.662006in}}%
\pgfpathlineto{\pgfqpoint{1.446405in}{0.661765in}}%
\pgfpathlineto{\pgfqpoint{1.446405in}{0.666836in}}%
\pgfpathlineto{\pgfqpoint{1.446405in}{0.673149in}}%
\pgfpathlineto{\pgfqpoint{1.434692in}{0.671220in}}%
\pgfpathlineto{\pgfqpoint{1.422980in}{0.670737in}}%
\pgfpathlineto{\pgfqpoint{1.411267in}{0.676250in}}%
\pgfpathlineto{\pgfqpoint{1.407734in}{0.678965in}}%
\pgfpathlineto{\pgfqpoint{1.399554in}{0.685267in}}%
\pgfpathlineto{\pgfqpoint{1.391740in}{0.691094in}}%
\pgfpathlineto{\pgfqpoint{1.388242in}{0.703223in}}%
\pgfpathlineto{\pgfqpoint{1.388246in}{0.715352in}}%
\pgfpathlineto{\pgfqpoint{1.387842in}{0.716604in}}%
\pgfpathlineto{\pgfqpoint{1.376129in}{0.725381in}}%
\pgfpathlineto{\pgfqpoint{1.364416in}{0.720108in}}%
\pgfpathlineto{\pgfqpoint{1.352704in}{0.717369in}}%
\pgfpathlineto{\pgfqpoint{1.351915in}{0.715352in}}%
\pgfpathlineto{\pgfqpoint{1.350823in}{0.703223in}}%
\pgfpathlineto{\pgfqpoint{1.348403in}{0.691094in}}%
\pgfpathlineto{\pgfqpoint{1.340991in}{0.685673in}}%
\pgfpathlineto{\pgfqpoint{1.331510in}{0.678965in}}%
\pgfpathlineto{\pgfqpoint{1.329278in}{0.676922in}}%
\pgfpathlineto{\pgfqpoint{1.317566in}{0.669793in}}%
\pgfpathlineto{\pgfqpoint{1.309058in}{0.666836in}}%
\pgfpathlineto{\pgfqpoint{1.305853in}{0.665441in}}%
\pgfpathlineto{\pgfqpoint{1.294140in}{0.664125in}}%
\pgfpathlineto{\pgfqpoint{1.284889in}{0.666836in}}%
\pgfpathlineto{\pgfqpoint{1.282427in}{0.668093in}}%
\pgfpathlineto{\pgfqpoint{1.270715in}{0.670295in}}%
\pgfpathlineto{\pgfqpoint{1.259002in}{0.668995in}}%
\pgfpathlineto{\pgfqpoint{1.255657in}{0.666836in}}%
\pgfpathlineto{\pgfqpoint{1.247289in}{0.661235in}}%
\pgfpathlineto{\pgfqpoint{1.235577in}{0.657280in}}%
\pgfpathlineto{\pgfqpoint{1.223864in}{0.657348in}}%
\pgfpathlineto{\pgfqpoint{1.212151in}{0.657837in}}%
\pgfpathlineto{\pgfqpoint{1.200439in}{0.659496in}}%
\pgfpathlineto{\pgfqpoint{1.188726in}{0.665012in}}%
\pgfpathlineto{\pgfqpoint{1.185845in}{0.666836in}}%
\pgfpathlineto{\pgfqpoint{1.177013in}{0.677744in}}%
\pgfpathlineto{\pgfqpoint{1.175909in}{0.678965in}}%
\pgfpathlineto{\pgfqpoint{1.166676in}{0.691094in}}%
\pgfpathlineto{\pgfqpoint{1.172778in}{0.703223in}}%
\pgfpathlineto{\pgfqpoint{1.177013in}{0.712175in}}%
\pgfpathlineto{\pgfqpoint{1.178036in}{0.715352in}}%
\pgfpathlineto{\pgfqpoint{1.182914in}{0.727481in}}%
\pgfpathlineto{\pgfqpoint{1.188726in}{0.732176in}}%
\pgfpathlineto{\pgfqpoint{1.200439in}{0.728805in}}%
\pgfpathlineto{\pgfqpoint{1.204398in}{0.727481in}}%
\pgfpathlineto{\pgfqpoint{1.212151in}{0.725230in}}%
\pgfpathlineto{\pgfqpoint{1.223864in}{0.721580in}}%
\pgfpathlineto{\pgfqpoint{1.233509in}{0.715352in}}%
\pgfpathlineto{\pgfqpoint{1.235577in}{0.713860in}}%
\pgfpathlineto{\pgfqpoint{1.247289in}{0.703232in}}%
\pgfpathlineto{\pgfqpoint{1.247302in}{0.703223in}}%
\pgfpathlineto{\pgfqpoint{1.259002in}{0.695792in}}%
\pgfpathlineto{\pgfqpoint{1.270715in}{0.694207in}}%
\pgfpathlineto{\pgfqpoint{1.282427in}{0.698540in}}%
\pgfpathlineto{\pgfqpoint{1.290261in}{0.703223in}}%
\pgfpathlineto{\pgfqpoint{1.294140in}{0.705586in}}%
\pgfpathlineto{\pgfqpoint{1.305853in}{0.714250in}}%
\pgfpathlineto{\pgfqpoint{1.307482in}{0.715352in}}%
\pgfpathlineto{\pgfqpoint{1.317566in}{0.722016in}}%
\pgfpathlineto{\pgfqpoint{1.329278in}{0.724583in}}%
\pgfpathlineto{\pgfqpoint{1.340991in}{0.727455in}}%
\pgfpathlineto{\pgfqpoint{1.341060in}{0.727481in}}%
\pgfpathlineto{\pgfqpoint{1.347824in}{0.739610in}}%
\pgfpathlineto{\pgfqpoint{1.350934in}{0.751739in}}%
\pgfpathlineto{\pgfqpoint{1.352704in}{0.759539in}}%
\pgfpathlineto{\pgfqpoint{1.353784in}{0.763867in}}%
\pgfpathlineto{\pgfqpoint{1.360313in}{0.775996in}}%
\pgfpathlineto{\pgfqpoint{1.364416in}{0.782551in}}%
\pgfpathlineto{\pgfqpoint{1.367834in}{0.788125in}}%
\pgfpathlineto{\pgfqpoint{1.374202in}{0.800254in}}%
\pgfpathlineto{\pgfqpoint{1.376129in}{0.804583in}}%
\pgfpathlineto{\pgfqpoint{1.379901in}{0.812383in}}%
\pgfpathlineto{\pgfqpoint{1.386824in}{0.824512in}}%
\pgfpathlineto{\pgfqpoint{1.387842in}{0.826302in}}%
\pgfpathlineto{\pgfqpoint{1.393365in}{0.836641in}}%
\pgfpathlineto{\pgfqpoint{1.399554in}{0.844879in}}%
\pgfpathlineto{\pgfqpoint{1.403065in}{0.848770in}}%
\pgfpathlineto{\pgfqpoint{1.411267in}{0.857203in}}%
\pgfpathlineto{\pgfqpoint{1.416652in}{0.860898in}}%
\pgfpathlineto{\pgfqpoint{1.422980in}{0.864409in}}%
\pgfpathlineto{\pgfqpoint{1.434692in}{0.868980in}}%
\pgfpathlineto{\pgfqpoint{1.446365in}{0.873027in}}%
\pgfpathlineto{\pgfqpoint{1.446405in}{0.873043in}}%
\pgfpathlineto{\pgfqpoint{1.446405in}{0.884395in}}%
\pgfpathlineto{\pgfqpoint{1.434692in}{0.879896in}}%
\pgfpathlineto{\pgfqpoint{1.422980in}{0.875295in}}%
\pgfpathlineto{\pgfqpoint{1.418761in}{0.873027in}}%
\pgfpathlineto{\pgfqpoint{1.411267in}{0.869401in}}%
\pgfpathlineto{\pgfqpoint{1.401563in}{0.860898in}}%
\pgfpathlineto{\pgfqpoint{1.399554in}{0.858834in}}%
\pgfpathlineto{\pgfqpoint{1.391745in}{0.848770in}}%
\pgfpathlineto{\pgfqpoint{1.387842in}{0.843510in}}%
\pgfpathlineto{\pgfqpoint{1.382678in}{0.836641in}}%
\pgfpathlineto{\pgfqpoint{1.376129in}{0.825176in}}%
\pgfpathlineto{\pgfqpoint{1.375721in}{0.824512in}}%
\pgfpathlineto{\pgfqpoint{1.368002in}{0.812383in}}%
\pgfpathlineto{\pgfqpoint{1.364416in}{0.803528in}}%
\pgfpathlineto{\pgfqpoint{1.363039in}{0.800254in}}%
\pgfpathlineto{\pgfqpoint{1.356667in}{0.788125in}}%
\pgfpathlineto{\pgfqpoint{1.352704in}{0.781306in}}%
\pgfpathlineto{\pgfqpoint{1.349894in}{0.775996in}}%
\pgfpathlineto{\pgfqpoint{1.345181in}{0.763867in}}%
\pgfpathlineto{\pgfqpoint{1.342806in}{0.751739in}}%
\pgfpathlineto{\pgfqpoint{1.340991in}{0.746374in}}%
\pgfpathlineto{\pgfqpoint{1.329278in}{0.739760in}}%
\pgfpathlineto{\pgfqpoint{1.328589in}{0.739610in}}%
\pgfpathlineto{\pgfqpoint{1.317566in}{0.737290in}}%
\pgfpathlineto{\pgfqpoint{1.305853in}{0.730744in}}%
\pgfpathlineto{\pgfqpoint{1.301968in}{0.727481in}}%
\pgfpathlineto{\pgfqpoint{1.294140in}{0.721424in}}%
\pgfpathlineto{\pgfqpoint{1.284378in}{0.715352in}}%
\pgfpathlineto{\pgfqpoint{1.282427in}{0.714197in}}%
\pgfpathlineto{\pgfqpoint{1.270715in}{0.709653in}}%
\pgfpathlineto{\pgfqpoint{1.259002in}{0.711553in}}%
\pgfpathlineto{\pgfqpoint{1.253464in}{0.715352in}}%
\pgfpathlineto{\pgfqpoint{1.247289in}{0.719662in}}%
\pgfpathlineto{\pgfqpoint{1.237643in}{0.727481in}}%
\pgfpathlineto{\pgfqpoint{1.235577in}{0.729221in}}%
\pgfpathlineto{\pgfqpoint{1.223864in}{0.735121in}}%
\pgfpathlineto{\pgfqpoint{1.212151in}{0.738030in}}%
\pgfpathlineto{\pgfqpoint{1.207799in}{0.739610in}}%
\pgfpathlineto{\pgfqpoint{1.200439in}{0.742604in}}%
\pgfpathlineto{\pgfqpoint{1.188726in}{0.748364in}}%
\pgfpathlineto{\pgfqpoint{1.182855in}{0.751739in}}%
\pgfpathlineto{\pgfqpoint{1.177013in}{0.756804in}}%
\pgfpathlineto{\pgfqpoint{1.169802in}{0.763867in}}%
\pgfpathlineto{\pgfqpoint{1.165356in}{0.775996in}}%
\pgfpathlineto{\pgfqpoint{1.165301in}{0.776166in}}%
\pgfpathlineto{\pgfqpoint{1.161737in}{0.788125in}}%
\pgfpathlineto{\pgfqpoint{1.162100in}{0.800254in}}%
\pgfpathlineto{\pgfqpoint{1.165168in}{0.812383in}}%
\pgfpathlineto{\pgfqpoint{1.165301in}{0.812768in}}%
\pgfpathlineto{\pgfqpoint{1.170312in}{0.824512in}}%
\pgfpathlineto{\pgfqpoint{1.177013in}{0.835929in}}%
\pgfpathlineto{\pgfqpoint{1.177684in}{0.836641in}}%
\pgfpathlineto{\pgfqpoint{1.188123in}{0.848770in}}%
\pgfpathlineto{\pgfqpoint{1.188726in}{0.849269in}}%
\pgfpathlineto{\pgfqpoint{1.200439in}{0.857046in}}%
\pgfpathlineto{\pgfqpoint{1.207022in}{0.860898in}}%
\pgfpathlineto{\pgfqpoint{1.212151in}{0.862950in}}%
\pgfpathlineto{\pgfqpoint{1.223864in}{0.866683in}}%
\pgfpathlineto{\pgfqpoint{1.235577in}{0.871749in}}%
\pgfpathlineto{\pgfqpoint{1.237410in}{0.873027in}}%
\pgfpathlineto{\pgfqpoint{1.242130in}{0.885156in}}%
\pgfpathlineto{\pgfqpoint{1.238744in}{0.897285in}}%
\pgfpathlineto{\pgfqpoint{1.235577in}{0.903037in}}%
\pgfpathlineto{\pgfqpoint{1.231823in}{0.909414in}}%
\pgfpathlineto{\pgfqpoint{1.225663in}{0.921543in}}%
\pgfpathlineto{\pgfqpoint{1.223864in}{0.924606in}}%
\pgfpathlineto{\pgfqpoint{1.214232in}{0.933672in}}%
\pgfpathlineto{\pgfqpoint{1.212151in}{0.934758in}}%
\pgfpathlineto{\pgfqpoint{1.200439in}{0.935774in}}%
\pgfpathlineto{\pgfqpoint{1.188726in}{0.934896in}}%
\pgfpathlineto{\pgfqpoint{1.180079in}{0.933672in}}%
\pgfpathlineto{\pgfqpoint{1.177013in}{0.933224in}}%
\pgfpathlineto{\pgfqpoint{1.170138in}{0.933672in}}%
\pgfpathlineto{\pgfqpoint{1.165301in}{0.934067in}}%
\pgfpathlineto{\pgfqpoint{1.153588in}{0.938053in}}%
\pgfpathlineto{\pgfqpoint{1.141875in}{0.940006in}}%
\pgfpathlineto{\pgfqpoint{1.130163in}{0.935380in}}%
\pgfpathlineto{\pgfqpoint{1.128009in}{0.933672in}}%
\pgfpathlineto{\pgfqpoint{1.118450in}{0.922229in}}%
\pgfpathlineto{\pgfqpoint{1.117999in}{0.921543in}}%
\pgfpathlineto{\pgfqpoint{1.110316in}{0.909414in}}%
\pgfpathlineto{\pgfqpoint{1.106737in}{0.904130in}}%
\pgfpathlineto{\pgfqpoint{1.098123in}{0.897285in}}%
\pgfpathlineto{\pgfqpoint{1.095024in}{0.895202in}}%
\pgfpathlineto{\pgfqpoint{1.083312in}{0.886814in}}%
\pgfpathlineto{\pgfqpoint{1.081658in}{0.885156in}}%
\pgfpathlineto{\pgfqpoint{1.077332in}{0.873027in}}%
\pgfpathlineto{\pgfqpoint{1.076817in}{0.860898in}}%
\pgfpathlineto{\pgfqpoint{1.076283in}{0.848770in}}%
\pgfpathlineto{\pgfqpoint{1.075010in}{0.836641in}}%
\pgfpathlineto{\pgfqpoint{1.075203in}{0.824512in}}%
\pgfpathlineto{\pgfqpoint{1.083312in}{0.814922in}}%
\pgfpathlineto{\pgfqpoint{1.084872in}{0.812383in}}%
\pgfpathlineto{\pgfqpoint{1.095024in}{0.801459in}}%
\pgfpathlineto{\pgfqpoint{1.096756in}{0.800254in}}%
\pgfpathlineto{\pgfqpoint{1.106737in}{0.794160in}}%
\pgfpathlineto{\pgfqpoint{1.118450in}{0.788539in}}%
\pgfpathlineto{\pgfqpoint{1.119374in}{0.788125in}}%
\pgfpathlineto{\pgfqpoint{1.130163in}{0.780702in}}%
\pgfpathlineto{\pgfqpoint{1.138189in}{0.775996in}}%
\pgfpathlineto{\pgfqpoint{1.141875in}{0.773205in}}%
\pgfpathlineto{\pgfqpoint{1.152764in}{0.763867in}}%
\pgfpathlineto{\pgfqpoint{1.153588in}{0.762918in}}%
\pgfpathlineto{\pgfqpoint{1.162599in}{0.751739in}}%
\pgfpathlineto{\pgfqpoint{1.165301in}{0.742071in}}%
\pgfpathlineto{\pgfqpoint{1.165820in}{0.739610in}}%
\pgfpathlineto{\pgfqpoint{1.165301in}{0.736250in}}%
\pgfpathlineto{\pgfqpoint{1.163419in}{0.727481in}}%
\pgfpathlineto{\pgfqpoint{1.154499in}{0.715352in}}%
\pgfpathlineto{\pgfqpoint{1.153588in}{0.713870in}}%
\pgfpathlineto{\pgfqpoint{1.146068in}{0.703223in}}%
\pgfpathlineto{\pgfqpoint{1.142630in}{0.691094in}}%
\pgfpathlineto{\pgfqpoint{1.153588in}{0.680190in}}%
\pgfpathlineto{\pgfqpoint{1.154702in}{0.678965in}}%
\pgfpathlineto{\pgfqpoint{1.164902in}{0.666836in}}%
\pgfpathlineto{\pgfqpoint{1.165301in}{0.666515in}}%
\pgfpathlineto{\pgfqpoint{1.177013in}{0.658497in}}%
\pgfpathlineto{\pgfqpoint{1.185073in}{0.654708in}}%
\pgfpathlineto{\pgfqpoint{1.188726in}{0.653071in}}%
\pgfpathlineto{\pgfqpoint{1.200439in}{0.648419in}}%
\pgfpathlineto{\pgfqpoint{1.212151in}{0.646222in}}%
\pgfpathlineto{\pgfqpoint{1.223864in}{0.644874in}}%
\pgfpathlineto{\pgfqpoint{1.230271in}{0.642579in}}%
\pgfpathlineto{\pgfqpoint{1.235577in}{0.634999in}}%
\pgfpathlineto{\pgfqpoint{1.245451in}{0.630450in}}%
\pgfpathclose%
\pgfpathmoveto{\pgfqpoint{1.117327in}{0.800254in}}%
\pgfpathlineto{\pgfqpoint{1.106737in}{0.805083in}}%
\pgfpathlineto{\pgfqpoint{1.097160in}{0.812383in}}%
\pgfpathlineto{\pgfqpoint{1.095024in}{0.814535in}}%
\pgfpathlineto{\pgfqpoint{1.088152in}{0.824512in}}%
\pgfpathlineto{\pgfqpoint{1.085979in}{0.836641in}}%
\pgfpathlineto{\pgfqpoint{1.086364in}{0.848770in}}%
\pgfpathlineto{\pgfqpoint{1.087591in}{0.860898in}}%
\pgfpathlineto{\pgfqpoint{1.092381in}{0.873027in}}%
\pgfpathlineto{\pgfqpoint{1.095024in}{0.875996in}}%
\pgfpathlineto{\pgfqpoint{1.106500in}{0.885156in}}%
\pgfpathlineto{\pgfqpoint{1.106737in}{0.885338in}}%
\pgfpathlineto{\pgfqpoint{1.118450in}{0.894775in}}%
\pgfpathlineto{\pgfqpoint{1.122646in}{0.897285in}}%
\pgfpathlineto{\pgfqpoint{1.129196in}{0.909414in}}%
\pgfpathlineto{\pgfqpoint{1.130163in}{0.910955in}}%
\pgfpathlineto{\pgfqpoint{1.141875in}{0.913102in}}%
\pgfpathlineto{\pgfqpoint{1.153588in}{0.913088in}}%
\pgfpathlineto{\pgfqpoint{1.165301in}{0.914246in}}%
\pgfpathlineto{\pgfqpoint{1.177013in}{0.916349in}}%
\pgfpathlineto{\pgfqpoint{1.188726in}{0.915993in}}%
\pgfpathlineto{\pgfqpoint{1.200439in}{0.910277in}}%
\pgfpathlineto{\pgfqpoint{1.201348in}{0.909414in}}%
\pgfpathlineto{\pgfqpoint{1.207142in}{0.897285in}}%
\pgfpathlineto{\pgfqpoint{1.200439in}{0.887245in}}%
\pgfpathlineto{\pgfqpoint{1.197705in}{0.885156in}}%
\pgfpathlineto{\pgfqpoint{1.188726in}{0.881912in}}%
\pgfpathlineto{\pgfqpoint{1.177013in}{0.876226in}}%
\pgfpathlineto{\pgfqpoint{1.172037in}{0.873027in}}%
\pgfpathlineto{\pgfqpoint{1.165301in}{0.866702in}}%
\pgfpathlineto{\pgfqpoint{1.159939in}{0.860898in}}%
\pgfpathlineto{\pgfqpoint{1.156022in}{0.848770in}}%
\pgfpathlineto{\pgfqpoint{1.155167in}{0.836641in}}%
\pgfpathlineto{\pgfqpoint{1.153588in}{0.830824in}}%
\pgfpathlineto{\pgfqpoint{1.152284in}{0.824512in}}%
\pgfpathlineto{\pgfqpoint{1.148592in}{0.812383in}}%
\pgfpathlineto{\pgfqpoint{1.145078in}{0.800254in}}%
\pgfpathlineto{\pgfqpoint{1.141875in}{0.794638in}}%
\pgfpathlineto{\pgfqpoint{1.130163in}{0.795585in}}%
\pgfpathlineto{\pgfqpoint{1.118450in}{0.799705in}}%
\pgfpathclose%
\pgfusepath{fill}%
\end{pgfscope}%
\begin{pgfscope}%
\pgfpathrectangle{\pgfqpoint{0.211875in}{0.211875in}}{\pgfqpoint{1.313625in}{1.279725in}}%
\pgfusepath{clip}%
\pgfsetbuttcap%
\pgfsetroundjoin%
\definecolor{currentfill}{rgb}{0.901975,0.231521,0.249182}%
\pgfsetfillcolor{currentfill}%
\pgfsetlinewidth{0.000000pt}%
\definecolor{currentstroke}{rgb}{0.000000,0.000000,0.000000}%
\pgfsetstrokecolor{currentstroke}%
\pgfsetdash{}{0pt}%
\pgfpathmoveto{\pgfqpoint{0.288840in}{0.666836in}}%
\pgfpathlineto{\pgfqpoint{0.292488in}{0.678965in}}%
\pgfpathlineto{\pgfqpoint{0.298020in}{0.691094in}}%
\pgfpathlineto{\pgfqpoint{0.298562in}{0.691974in}}%
\pgfpathlineto{\pgfqpoint{0.305359in}{0.703223in}}%
\pgfpathlineto{\pgfqpoint{0.310274in}{0.713502in}}%
\pgfpathlineto{\pgfqpoint{0.311286in}{0.715352in}}%
\pgfpathlineto{\pgfqpoint{0.310274in}{0.723164in}}%
\pgfpathlineto{\pgfqpoint{0.309477in}{0.727481in}}%
\pgfpathlineto{\pgfqpoint{0.304575in}{0.739610in}}%
\pgfpathlineto{\pgfqpoint{0.307879in}{0.751739in}}%
\pgfpathlineto{\pgfqpoint{0.310274in}{0.756647in}}%
\pgfpathlineto{\pgfqpoint{0.314239in}{0.763867in}}%
\pgfpathlineto{\pgfqpoint{0.321987in}{0.775754in}}%
\pgfpathlineto{\pgfqpoint{0.322171in}{0.775996in}}%
\pgfpathlineto{\pgfqpoint{0.332359in}{0.788125in}}%
\pgfpathlineto{\pgfqpoint{0.333700in}{0.790548in}}%
\pgfpathlineto{\pgfqpoint{0.339374in}{0.800254in}}%
\pgfpathlineto{\pgfqpoint{0.333700in}{0.807560in}}%
\pgfpathlineto{\pgfqpoint{0.330784in}{0.812383in}}%
\pgfpathlineto{\pgfqpoint{0.321987in}{0.821606in}}%
\pgfpathlineto{\pgfqpoint{0.319176in}{0.824512in}}%
\pgfpathlineto{\pgfqpoint{0.310274in}{0.831730in}}%
\pgfpathlineto{\pgfqpoint{0.304228in}{0.836641in}}%
\pgfpathlineto{\pgfqpoint{0.298562in}{0.841393in}}%
\pgfpathlineto{\pgfqpoint{0.289433in}{0.848770in}}%
\pgfpathlineto{\pgfqpoint{0.286849in}{0.851253in}}%
\pgfpathlineto{\pgfqpoint{0.286849in}{0.848770in}}%
\pgfpathlineto{\pgfqpoint{0.286849in}{0.836641in}}%
\pgfpathlineto{\pgfqpoint{0.286849in}{0.824512in}}%
\pgfpathlineto{\pgfqpoint{0.286849in}{0.812383in}}%
\pgfpathlineto{\pgfqpoint{0.286849in}{0.800254in}}%
\pgfpathlineto{\pgfqpoint{0.286849in}{0.788125in}}%
\pgfpathlineto{\pgfqpoint{0.286849in}{0.775996in}}%
\pgfpathlineto{\pgfqpoint{0.286849in}{0.763867in}}%
\pgfpathlineto{\pgfqpoint{0.286849in}{0.751739in}}%
\pgfpathlineto{\pgfqpoint{0.286849in}{0.739610in}}%
\pgfpathlineto{\pgfqpoint{0.286849in}{0.727481in}}%
\pgfpathlineto{\pgfqpoint{0.286849in}{0.715352in}}%
\pgfpathlineto{\pgfqpoint{0.286849in}{0.703223in}}%
\pgfpathlineto{\pgfqpoint{0.286849in}{0.691094in}}%
\pgfpathlineto{\pgfqpoint{0.286849in}{0.678965in}}%
\pgfpathlineto{\pgfqpoint{0.286849in}{0.666836in}}%
\pgfpathlineto{\pgfqpoint{0.286849in}{0.660751in}}%
\pgfpathclose%
\pgfusepath{fill}%
\end{pgfscope}%
\begin{pgfscope}%
\pgfpathrectangle{\pgfqpoint{0.211875in}{0.211875in}}{\pgfqpoint{1.313625in}{1.279725in}}%
\pgfusepath{clip}%
\pgfsetbuttcap%
\pgfsetroundjoin%
\definecolor{currentfill}{rgb}{0.901975,0.231521,0.249182}%
\pgfsetfillcolor{currentfill}%
\pgfsetlinewidth{0.000000pt}%
\definecolor{currentstroke}{rgb}{0.000000,0.000000,0.000000}%
\pgfsetstrokecolor{currentstroke}%
\pgfsetdash{}{0pt}%
\pgfpathmoveto{\pgfqpoint{1.446405in}{0.725727in}}%
\pgfpathlineto{\pgfqpoint{1.446405in}{0.727481in}}%
\pgfpathlineto{\pgfqpoint{1.446405in}{0.737506in}}%
\pgfpathlineto{\pgfqpoint{1.439959in}{0.739610in}}%
\pgfpathlineto{\pgfqpoint{1.434692in}{0.741429in}}%
\pgfpathlineto{\pgfqpoint{1.422980in}{0.749040in}}%
\pgfpathlineto{\pgfqpoint{1.420703in}{0.751739in}}%
\pgfpathlineto{\pgfqpoint{1.418008in}{0.763867in}}%
\pgfpathlineto{\pgfqpoint{1.418746in}{0.775996in}}%
\pgfpathlineto{\pgfqpoint{1.422980in}{0.785362in}}%
\pgfpathlineto{\pgfqpoint{1.424586in}{0.788125in}}%
\pgfpathlineto{\pgfqpoint{1.434692in}{0.798341in}}%
\pgfpathlineto{\pgfqpoint{1.436388in}{0.800254in}}%
\pgfpathlineto{\pgfqpoint{1.446405in}{0.808813in}}%
\pgfpathlineto{\pgfqpoint{1.446405in}{0.812383in}}%
\pgfpathlineto{\pgfqpoint{1.446405in}{0.820969in}}%
\pgfpathlineto{\pgfqpoint{1.436783in}{0.812383in}}%
\pgfpathlineto{\pgfqpoint{1.434692in}{0.810563in}}%
\pgfpathlineto{\pgfqpoint{1.422980in}{0.802236in}}%
\pgfpathlineto{\pgfqpoint{1.420137in}{0.800254in}}%
\pgfpathlineto{\pgfqpoint{1.411267in}{0.788831in}}%
\pgfpathlineto{\pgfqpoint{1.410711in}{0.788125in}}%
\pgfpathlineto{\pgfqpoint{1.404334in}{0.775996in}}%
\pgfpathlineto{\pgfqpoint{1.400671in}{0.763867in}}%
\pgfpathlineto{\pgfqpoint{1.399625in}{0.751739in}}%
\pgfpathlineto{\pgfqpoint{1.401855in}{0.739610in}}%
\pgfpathlineto{\pgfqpoint{1.411267in}{0.732637in}}%
\pgfpathlineto{\pgfqpoint{1.422980in}{0.733187in}}%
\pgfpathlineto{\pgfqpoint{1.434692in}{0.730055in}}%
\pgfpathlineto{\pgfqpoint{1.441956in}{0.727481in}}%
\pgfpathclose%
\pgfusepath{fill}%
\end{pgfscope}%
\begin{pgfscope}%
\pgfpathrectangle{\pgfqpoint{0.211875in}{0.211875in}}{\pgfqpoint{1.313625in}{1.279725in}}%
\pgfusepath{clip}%
\pgfsetbuttcap%
\pgfsetroundjoin%
\definecolor{currentfill}{rgb}{0.901975,0.231521,0.249182}%
\pgfsetfillcolor{currentfill}%
\pgfsetlinewidth{0.000000pt}%
\definecolor{currentstroke}{rgb}{0.000000,0.000000,0.000000}%
\pgfsetstrokecolor{currentstroke}%
\pgfsetdash{}{0pt}%
\pgfpathmoveto{\pgfqpoint{0.966185in}{0.920102in}}%
\pgfpathlineto{\pgfqpoint{0.977898in}{0.920634in}}%
\pgfpathlineto{\pgfqpoint{0.980394in}{0.921543in}}%
\pgfpathlineto{\pgfqpoint{0.989610in}{0.924560in}}%
\pgfpathlineto{\pgfqpoint{1.001323in}{0.930251in}}%
\pgfpathlineto{\pgfqpoint{1.007624in}{0.933672in}}%
\pgfpathlineto{\pgfqpoint{1.013036in}{0.936406in}}%
\pgfpathlineto{\pgfqpoint{1.024748in}{0.942131in}}%
\pgfpathlineto{\pgfqpoint{1.030498in}{0.945801in}}%
\pgfpathlineto{\pgfqpoint{1.036295in}{0.957929in}}%
\pgfpathlineto{\pgfqpoint{1.036461in}{0.958368in}}%
\pgfpathlineto{\pgfqpoint{1.040189in}{0.970058in}}%
\pgfpathlineto{\pgfqpoint{1.044506in}{0.982187in}}%
\pgfpathlineto{\pgfqpoint{1.048174in}{0.989051in}}%
\pgfpathlineto{\pgfqpoint{1.051055in}{0.994316in}}%
\pgfpathlineto{\pgfqpoint{1.057627in}{1.006445in}}%
\pgfpathlineto{\pgfqpoint{1.059886in}{1.010312in}}%
\pgfpathlineto{\pgfqpoint{1.065049in}{1.018574in}}%
\pgfpathlineto{\pgfqpoint{1.071599in}{1.030008in}}%
\pgfpathlineto{\pgfqpoint{1.071990in}{1.030703in}}%
\pgfpathlineto{\pgfqpoint{1.076522in}{1.042832in}}%
\pgfpathlineto{\pgfqpoint{1.081391in}{1.054960in}}%
\pgfpathlineto{\pgfqpoint{1.083312in}{1.059357in}}%
\pgfpathlineto{\pgfqpoint{1.086629in}{1.067089in}}%
\pgfpathlineto{\pgfqpoint{1.091710in}{1.079218in}}%
\pgfpathlineto{\pgfqpoint{1.095024in}{1.084587in}}%
\pgfpathlineto{\pgfqpoint{1.098962in}{1.091347in}}%
\pgfpathlineto{\pgfqpoint{1.106390in}{1.103476in}}%
\pgfpathlineto{\pgfqpoint{1.106737in}{1.103948in}}%
\pgfpathlineto{\pgfqpoint{1.115024in}{1.115605in}}%
\pgfpathlineto{\pgfqpoint{1.118450in}{1.120437in}}%
\pgfpathlineto{\pgfqpoint{1.122160in}{1.127734in}}%
\pgfpathlineto{\pgfqpoint{1.130163in}{1.139416in}}%
\pgfpathlineto{\pgfqpoint{1.130483in}{1.139863in}}%
\pgfpathlineto{\pgfqpoint{1.139955in}{1.151991in}}%
\pgfpathlineto{\pgfqpoint{1.141875in}{1.154539in}}%
\pgfpathlineto{\pgfqpoint{1.147548in}{1.164120in}}%
\pgfpathlineto{\pgfqpoint{1.153588in}{1.173973in}}%
\pgfpathlineto{\pgfqpoint{1.154932in}{1.176249in}}%
\pgfpathlineto{\pgfqpoint{1.159213in}{1.188378in}}%
\pgfpathlineto{\pgfqpoint{1.163788in}{1.200507in}}%
\pgfpathlineto{\pgfqpoint{1.165301in}{1.208906in}}%
\pgfpathlineto{\pgfqpoint{1.165785in}{1.212636in}}%
\pgfpathlineto{\pgfqpoint{1.165735in}{1.224765in}}%
\pgfpathlineto{\pgfqpoint{1.165301in}{1.231945in}}%
\pgfpathlineto{\pgfqpoint{1.164958in}{1.236894in}}%
\pgfpathlineto{\pgfqpoint{1.161982in}{1.249022in}}%
\pgfpathlineto{\pgfqpoint{1.162303in}{1.261151in}}%
\pgfpathlineto{\pgfqpoint{1.165301in}{1.267680in}}%
\pgfpathlineto{\pgfqpoint{1.168582in}{1.273280in}}%
\pgfpathlineto{\pgfqpoint{1.177013in}{1.281137in}}%
\pgfpathlineto{\pgfqpoint{1.188726in}{1.278843in}}%
\pgfpathlineto{\pgfqpoint{1.193438in}{1.273280in}}%
\pgfpathlineto{\pgfqpoint{1.198128in}{1.261151in}}%
\pgfpathlineto{\pgfqpoint{1.200439in}{1.257740in}}%
\pgfpathlineto{\pgfqpoint{1.212151in}{1.251741in}}%
\pgfpathlineto{\pgfqpoint{1.217819in}{1.249022in}}%
\pgfpathlineto{\pgfqpoint{1.223864in}{1.246040in}}%
\pgfpathlineto{\pgfqpoint{1.235577in}{1.239367in}}%
\pgfpathlineto{\pgfqpoint{1.242532in}{1.236894in}}%
\pgfpathlineto{\pgfqpoint{1.247289in}{1.233751in}}%
\pgfpathlineto{\pgfqpoint{1.255162in}{1.224765in}}%
\pgfpathlineto{\pgfqpoint{1.259002in}{1.220098in}}%
\pgfpathlineto{\pgfqpoint{1.266463in}{1.212636in}}%
\pgfpathlineto{\pgfqpoint{1.270715in}{1.208836in}}%
\pgfpathlineto{\pgfqpoint{1.282427in}{1.206565in}}%
\pgfpathlineto{\pgfqpoint{1.294140in}{1.202760in}}%
\pgfpathlineto{\pgfqpoint{1.297942in}{1.200507in}}%
\pgfpathlineto{\pgfqpoint{1.305853in}{1.195878in}}%
\pgfpathlineto{\pgfqpoint{1.314807in}{1.188378in}}%
\pgfpathlineto{\pgfqpoint{1.317566in}{1.186115in}}%
\pgfpathlineto{\pgfqpoint{1.323687in}{1.176249in}}%
\pgfpathlineto{\pgfqpoint{1.329278in}{1.166928in}}%
\pgfpathlineto{\pgfqpoint{1.330763in}{1.164120in}}%
\pgfpathlineto{\pgfqpoint{1.336230in}{1.151991in}}%
\pgfpathlineto{\pgfqpoint{1.340991in}{1.140744in}}%
\pgfpathlineto{\pgfqpoint{1.341636in}{1.139863in}}%
\pgfpathlineto{\pgfqpoint{1.344708in}{1.127734in}}%
\pgfpathlineto{\pgfqpoint{1.341777in}{1.115605in}}%
\pgfpathlineto{\pgfqpoint{1.340991in}{1.108892in}}%
\pgfpathlineto{\pgfqpoint{1.340404in}{1.103476in}}%
\pgfpathlineto{\pgfqpoint{1.340664in}{1.091347in}}%
\pgfpathlineto{\pgfqpoint{1.340991in}{1.089401in}}%
\pgfpathlineto{\pgfqpoint{1.343344in}{1.079218in}}%
\pgfpathlineto{\pgfqpoint{1.340991in}{1.067578in}}%
\pgfpathlineto{\pgfqpoint{1.340881in}{1.067089in}}%
\pgfpathlineto{\pgfqpoint{1.340991in}{1.066954in}}%
\pgfpathlineto{\pgfqpoint{1.352704in}{1.066636in}}%
\pgfpathlineto{\pgfqpoint{1.364416in}{1.066899in}}%
\pgfpathlineto{\pgfqpoint{1.365387in}{1.067089in}}%
\pgfpathlineto{\pgfqpoint{1.376129in}{1.068978in}}%
\pgfpathlineto{\pgfqpoint{1.387842in}{1.078686in}}%
\pgfpathlineto{\pgfqpoint{1.388549in}{1.079218in}}%
\pgfpathlineto{\pgfqpoint{1.399554in}{1.084173in}}%
\pgfpathlineto{\pgfqpoint{1.411267in}{1.088342in}}%
\pgfpathlineto{\pgfqpoint{1.417717in}{1.091347in}}%
\pgfpathlineto{\pgfqpoint{1.422980in}{1.093423in}}%
\pgfpathlineto{\pgfqpoint{1.434692in}{1.097594in}}%
\pgfpathlineto{\pgfqpoint{1.446405in}{1.100178in}}%
\pgfpathlineto{\pgfqpoint{1.446405in}{1.103476in}}%
\pgfpathlineto{\pgfqpoint{1.446405in}{1.113737in}}%
\pgfpathlineto{\pgfqpoint{1.434692in}{1.112170in}}%
\pgfpathlineto{\pgfqpoint{1.422980in}{1.108816in}}%
\pgfpathlineto{\pgfqpoint{1.411267in}{1.105030in}}%
\pgfpathlineto{\pgfqpoint{1.399554in}{1.104881in}}%
\pgfpathlineto{\pgfqpoint{1.387842in}{1.104217in}}%
\pgfpathlineto{\pgfqpoint{1.381677in}{1.103476in}}%
\pgfpathlineto{\pgfqpoint{1.376129in}{1.102731in}}%
\pgfpathlineto{\pgfqpoint{1.374845in}{1.103476in}}%
\pgfpathlineto{\pgfqpoint{1.368766in}{1.115605in}}%
\pgfpathlineto{\pgfqpoint{1.369141in}{1.127734in}}%
\pgfpathlineto{\pgfqpoint{1.371814in}{1.139863in}}%
\pgfpathlineto{\pgfqpoint{1.370251in}{1.151991in}}%
\pgfpathlineto{\pgfqpoint{1.364416in}{1.159378in}}%
\pgfpathlineto{\pgfqpoint{1.361341in}{1.164120in}}%
\pgfpathlineto{\pgfqpoint{1.352704in}{1.173339in}}%
\pgfpathlineto{\pgfqpoint{1.349275in}{1.176249in}}%
\pgfpathlineto{\pgfqpoint{1.340991in}{1.181754in}}%
\pgfpathlineto{\pgfqpoint{1.337082in}{1.188378in}}%
\pgfpathlineto{\pgfqpoint{1.329278in}{1.198504in}}%
\pgfpathlineto{\pgfqpoint{1.327712in}{1.200507in}}%
\pgfpathlineto{\pgfqpoint{1.317717in}{1.212636in}}%
\pgfpathlineto{\pgfqpoint{1.317566in}{1.213020in}}%
\pgfpathlineto{\pgfqpoint{1.308687in}{1.224765in}}%
\pgfpathlineto{\pgfqpoint{1.305853in}{1.234683in}}%
\pgfpathlineto{\pgfqpoint{1.304972in}{1.236894in}}%
\pgfpathlineto{\pgfqpoint{1.300290in}{1.249022in}}%
\pgfpathlineto{\pgfqpoint{1.297669in}{1.261151in}}%
\pgfpathlineto{\pgfqpoint{1.295027in}{1.273280in}}%
\pgfpathlineto{\pgfqpoint{1.294140in}{1.276201in}}%
\pgfpathlineto{\pgfqpoint{1.292303in}{1.285409in}}%
\pgfpathlineto{\pgfqpoint{1.291252in}{1.297538in}}%
\pgfpathlineto{\pgfqpoint{1.289915in}{1.309667in}}%
\pgfpathlineto{\pgfqpoint{1.285974in}{1.321796in}}%
\pgfpathlineto{\pgfqpoint{1.282427in}{1.325146in}}%
\pgfpathlineto{\pgfqpoint{1.270715in}{1.324840in}}%
\pgfpathlineto{\pgfqpoint{1.264995in}{1.321796in}}%
\pgfpathlineto{\pgfqpoint{1.259002in}{1.317524in}}%
\pgfpathlineto{\pgfqpoint{1.254074in}{1.309667in}}%
\pgfpathlineto{\pgfqpoint{1.251617in}{1.297538in}}%
\pgfpathlineto{\pgfqpoint{1.249849in}{1.285409in}}%
\pgfpathlineto{\pgfqpoint{1.247289in}{1.275655in}}%
\pgfpathlineto{\pgfqpoint{1.246495in}{1.273280in}}%
\pgfpathlineto{\pgfqpoint{1.235577in}{1.262065in}}%
\pgfpathlineto{\pgfqpoint{1.223864in}{1.267722in}}%
\pgfpathlineto{\pgfqpoint{1.212151in}{1.272730in}}%
\pgfpathlineto{\pgfqpoint{1.211693in}{1.273280in}}%
\pgfpathlineto{\pgfqpoint{1.203750in}{1.285409in}}%
\pgfpathlineto{\pgfqpoint{1.200439in}{1.293348in}}%
\pgfpathlineto{\pgfqpoint{1.196241in}{1.297538in}}%
\pgfpathlineto{\pgfqpoint{1.188726in}{1.303881in}}%
\pgfpathlineto{\pgfqpoint{1.177013in}{1.306394in}}%
\pgfpathlineto{\pgfqpoint{1.168760in}{1.309667in}}%
\pgfpathlineto{\pgfqpoint{1.165301in}{1.312512in}}%
\pgfpathlineto{\pgfqpoint{1.157617in}{1.321796in}}%
\pgfpathlineto{\pgfqpoint{1.155312in}{1.333925in}}%
\pgfpathlineto{\pgfqpoint{1.156815in}{1.346053in}}%
\pgfpathlineto{\pgfqpoint{1.159115in}{1.358182in}}%
\pgfpathlineto{\pgfqpoint{1.165087in}{1.370311in}}%
\pgfpathlineto{\pgfqpoint{1.165301in}{1.370548in}}%
\pgfpathlineto{\pgfqpoint{1.177013in}{1.375430in}}%
\pgfpathlineto{\pgfqpoint{1.188726in}{1.374493in}}%
\pgfpathlineto{\pgfqpoint{1.196447in}{1.370311in}}%
\pgfpathlineto{\pgfqpoint{1.200439in}{1.365310in}}%
\pgfpathlineto{\pgfqpoint{1.204690in}{1.358182in}}%
\pgfpathlineto{\pgfqpoint{1.212151in}{1.347470in}}%
\pgfpathlineto{\pgfqpoint{1.213917in}{1.346053in}}%
\pgfpathlineto{\pgfqpoint{1.223864in}{1.341622in}}%
\pgfpathlineto{\pgfqpoint{1.229661in}{1.346053in}}%
\pgfpathlineto{\pgfqpoint{1.235577in}{1.351375in}}%
\pgfpathlineto{\pgfqpoint{1.238430in}{1.358182in}}%
\pgfpathlineto{\pgfqpoint{1.241720in}{1.370311in}}%
\pgfpathlineto{\pgfqpoint{1.244456in}{1.382440in}}%
\pgfpathlineto{\pgfqpoint{1.247289in}{1.391986in}}%
\pgfpathlineto{\pgfqpoint{1.247783in}{1.394569in}}%
\pgfpathlineto{\pgfqpoint{1.251239in}{1.406698in}}%
\pgfpathlineto{\pgfqpoint{1.259002in}{1.412504in}}%
\pgfpathlineto{\pgfqpoint{1.270715in}{1.416699in}}%
\pgfpathlineto{\pgfqpoint{1.278144in}{1.418827in}}%
\pgfpathlineto{\pgfqpoint{1.282427in}{1.420558in}}%
\pgfpathlineto{\pgfqpoint{1.287723in}{1.418827in}}%
\pgfpathlineto{\pgfqpoint{1.294140in}{1.414961in}}%
\pgfpathlineto{\pgfqpoint{1.300533in}{1.406698in}}%
\pgfpathlineto{\pgfqpoint{1.299510in}{1.394569in}}%
\pgfpathlineto{\pgfqpoint{1.298186in}{1.382440in}}%
\pgfpathlineto{\pgfqpoint{1.297015in}{1.370311in}}%
\pgfpathlineto{\pgfqpoint{1.296493in}{1.358182in}}%
\pgfpathlineto{\pgfqpoint{1.296738in}{1.346053in}}%
\pgfpathlineto{\pgfqpoint{1.301704in}{1.333925in}}%
\pgfpathlineto{\pgfqpoint{1.305853in}{1.328420in}}%
\pgfpathlineto{\pgfqpoint{1.309157in}{1.321796in}}%
\pgfpathlineto{\pgfqpoint{1.314834in}{1.309667in}}%
\pgfpathlineto{\pgfqpoint{1.317566in}{1.303171in}}%
\pgfpathlineto{\pgfqpoint{1.321718in}{1.297538in}}%
\pgfpathlineto{\pgfqpoint{1.329278in}{1.289243in}}%
\pgfpathlineto{\pgfqpoint{1.333256in}{1.285409in}}%
\pgfpathlineto{\pgfqpoint{1.340991in}{1.279960in}}%
\pgfpathlineto{\pgfqpoint{1.346490in}{1.273280in}}%
\pgfpathlineto{\pgfqpoint{1.352704in}{1.267493in}}%
\pgfpathlineto{\pgfqpoint{1.357905in}{1.261151in}}%
\pgfpathlineto{\pgfqpoint{1.364416in}{1.254719in}}%
\pgfpathlineto{\pgfqpoint{1.369339in}{1.249022in}}%
\pgfpathlineto{\pgfqpoint{1.376129in}{1.241776in}}%
\pgfpathlineto{\pgfqpoint{1.380509in}{1.236894in}}%
\pgfpathlineto{\pgfqpoint{1.387842in}{1.229283in}}%
\pgfpathlineto{\pgfqpoint{1.391940in}{1.224765in}}%
\pgfpathlineto{\pgfqpoint{1.399554in}{1.216274in}}%
\pgfpathlineto{\pgfqpoint{1.402505in}{1.212636in}}%
\pgfpathlineto{\pgfqpoint{1.411267in}{1.202195in}}%
\pgfpathlineto{\pgfqpoint{1.413413in}{1.200507in}}%
\pgfpathlineto{\pgfqpoint{1.422980in}{1.194645in}}%
\pgfpathlineto{\pgfqpoint{1.434692in}{1.189363in}}%
\pgfpathlineto{\pgfqpoint{1.438857in}{1.188378in}}%
\pgfpathlineto{\pgfqpoint{1.446405in}{1.186862in}}%
\pgfpathlineto{\pgfqpoint{1.446405in}{1.188378in}}%
\pgfpathlineto{\pgfqpoint{1.446405in}{1.200507in}}%
\pgfpathlineto{\pgfqpoint{1.446405in}{1.201742in}}%
\pgfpathlineto{\pgfqpoint{1.434692in}{1.205793in}}%
\pgfpathlineto{\pgfqpoint{1.422980in}{1.212449in}}%
\pgfpathlineto{\pgfqpoint{1.422757in}{1.212636in}}%
\pgfpathlineto{\pgfqpoint{1.412307in}{1.224765in}}%
\pgfpathlineto{\pgfqpoint{1.411267in}{1.226086in}}%
\pgfpathlineto{\pgfqpoint{1.402460in}{1.236894in}}%
\pgfpathlineto{\pgfqpoint{1.399554in}{1.240307in}}%
\pgfpathlineto{\pgfqpoint{1.391610in}{1.249022in}}%
\pgfpathlineto{\pgfqpoint{1.387842in}{1.253120in}}%
\pgfpathlineto{\pgfqpoint{1.380707in}{1.261151in}}%
\pgfpathlineto{\pgfqpoint{1.376129in}{1.266254in}}%
\pgfpathlineto{\pgfqpoint{1.370044in}{1.273280in}}%
\pgfpathlineto{\pgfqpoint{1.364416in}{1.279971in}}%
\pgfpathlineto{\pgfqpoint{1.359822in}{1.285409in}}%
\pgfpathlineto{\pgfqpoint{1.352704in}{1.294379in}}%
\pgfpathlineto{\pgfqpoint{1.350145in}{1.297538in}}%
\pgfpathlineto{\pgfqpoint{1.340991in}{1.308936in}}%
\pgfpathlineto{\pgfqpoint{1.340368in}{1.309667in}}%
\pgfpathlineto{\pgfqpoint{1.330419in}{1.321796in}}%
\pgfpathlineto{\pgfqpoint{1.329278in}{1.323176in}}%
\pgfpathlineto{\pgfqpoint{1.320821in}{1.333925in}}%
\pgfpathlineto{\pgfqpoint{1.317566in}{1.337278in}}%
\pgfpathlineto{\pgfqpoint{1.311080in}{1.346053in}}%
\pgfpathlineto{\pgfqpoint{1.305853in}{1.352871in}}%
\pgfpathlineto{\pgfqpoint{1.303975in}{1.358182in}}%
\pgfpathlineto{\pgfqpoint{1.303990in}{1.370311in}}%
\pgfpathlineto{\pgfqpoint{1.305853in}{1.382405in}}%
\pgfpathlineto{\pgfqpoint{1.305865in}{1.382440in}}%
\pgfpathlineto{\pgfqpoint{1.311877in}{1.394569in}}%
\pgfpathlineto{\pgfqpoint{1.317566in}{1.403288in}}%
\pgfpathlineto{\pgfqpoint{1.320444in}{1.406698in}}%
\pgfpathlineto{\pgfqpoint{1.329278in}{1.417007in}}%
\pgfpathlineto{\pgfqpoint{1.331493in}{1.418827in}}%
\pgfpathlineto{\pgfqpoint{1.340991in}{1.428820in}}%
\pgfpathlineto{\pgfqpoint{1.344311in}{1.430956in}}%
\pgfpathlineto{\pgfqpoint{1.352704in}{1.437200in}}%
\pgfpathlineto{\pgfqpoint{1.363122in}{1.443084in}}%
\pgfpathlineto{\pgfqpoint{1.364416in}{1.443852in}}%
\pgfpathlineto{\pgfqpoint{1.376129in}{1.453558in}}%
\pgfpathlineto{\pgfqpoint{1.378362in}{1.455213in}}%
\pgfpathlineto{\pgfqpoint{1.387842in}{1.462946in}}%
\pgfpathlineto{\pgfqpoint{1.393812in}{1.467342in}}%
\pgfpathlineto{\pgfqpoint{1.399554in}{1.471791in}}%
\pgfpathlineto{\pgfqpoint{1.411267in}{1.475814in}}%
\pgfpathlineto{\pgfqpoint{1.416760in}{1.467342in}}%
\pgfpathlineto{\pgfqpoint{1.420500in}{1.455213in}}%
\pgfpathlineto{\pgfqpoint{1.420118in}{1.443084in}}%
\pgfpathlineto{\pgfqpoint{1.421240in}{1.430956in}}%
\pgfpathlineto{\pgfqpoint{1.421554in}{1.418827in}}%
\pgfpathlineto{\pgfqpoint{1.421358in}{1.406698in}}%
\pgfpathlineto{\pgfqpoint{1.422980in}{1.404165in}}%
\pgfpathlineto{\pgfqpoint{1.434692in}{1.399671in}}%
\pgfpathlineto{\pgfqpoint{1.446405in}{1.399918in}}%
\pgfpathlineto{\pgfqpoint{1.446405in}{1.406698in}}%
\pgfpathlineto{\pgfqpoint{1.446405in}{1.418827in}}%
\pgfpathlineto{\pgfqpoint{1.446405in}{1.430956in}}%
\pgfpathlineto{\pgfqpoint{1.446405in}{1.443084in}}%
\pgfpathlineto{\pgfqpoint{1.446405in}{1.455213in}}%
\pgfpathlineto{\pgfqpoint{1.446405in}{1.467342in}}%
\pgfpathlineto{\pgfqpoint{1.446405in}{1.479471in}}%
\pgfpathlineto{\pgfqpoint{1.446405in}{1.491600in}}%
\pgfpathlineto{\pgfqpoint{1.434692in}{1.491600in}}%
\pgfpathlineto{\pgfqpoint{1.422980in}{1.491600in}}%
\pgfpathlineto{\pgfqpoint{1.411267in}{1.491600in}}%
\pgfpathlineto{\pgfqpoint{1.399554in}{1.491600in}}%
\pgfpathlineto{\pgfqpoint{1.387842in}{1.491600in}}%
\pgfpathlineto{\pgfqpoint{1.376129in}{1.491600in}}%
\pgfpathlineto{\pgfqpoint{1.364416in}{1.491600in}}%
\pgfpathlineto{\pgfqpoint{1.352704in}{1.491600in}}%
\pgfpathlineto{\pgfqpoint{1.348654in}{1.491600in}}%
\pgfpathlineto{\pgfqpoint{1.340991in}{1.482734in}}%
\pgfpathlineto{\pgfqpoint{1.338287in}{1.479471in}}%
\pgfpathlineto{\pgfqpoint{1.329278in}{1.469372in}}%
\pgfpathlineto{\pgfqpoint{1.327503in}{1.467342in}}%
\pgfpathlineto{\pgfqpoint{1.317566in}{1.456016in}}%
\pgfpathlineto{\pgfqpoint{1.316853in}{1.455213in}}%
\pgfpathlineto{\pgfqpoint{1.305853in}{1.443645in}}%
\pgfpathlineto{\pgfqpoint{1.305177in}{1.443084in}}%
\pgfpathlineto{\pgfqpoint{1.294140in}{1.436259in}}%
\pgfpathlineto{\pgfqpoint{1.282427in}{1.431733in}}%
\pgfpathlineto{\pgfqpoint{1.281276in}{1.430956in}}%
\pgfpathlineto{\pgfqpoint{1.270715in}{1.424872in}}%
\pgfpathlineto{\pgfqpoint{1.259002in}{1.422505in}}%
\pgfpathlineto{\pgfqpoint{1.247289in}{1.423977in}}%
\pgfpathlineto{\pgfqpoint{1.241774in}{1.418827in}}%
\pgfpathlineto{\pgfqpoint{1.237248in}{1.406698in}}%
\pgfpathlineto{\pgfqpoint{1.235577in}{1.398299in}}%
\pgfpathlineto{\pgfqpoint{1.234043in}{1.394569in}}%
\pgfpathlineto{\pgfqpoint{1.229440in}{1.382440in}}%
\pgfpathlineto{\pgfqpoint{1.224844in}{1.370311in}}%
\pgfpathlineto{\pgfqpoint{1.223864in}{1.369558in}}%
\pgfpathlineto{\pgfqpoint{1.223232in}{1.370311in}}%
\pgfpathlineto{\pgfqpoint{1.212151in}{1.381690in}}%
\pgfpathlineto{\pgfqpoint{1.211426in}{1.382440in}}%
\pgfpathlineto{\pgfqpoint{1.200439in}{1.388566in}}%
\pgfpathlineto{\pgfqpoint{1.188726in}{1.391438in}}%
\pgfpathlineto{\pgfqpoint{1.177013in}{1.392249in}}%
\pgfpathlineto{\pgfqpoint{1.165301in}{1.390621in}}%
\pgfpathlineto{\pgfqpoint{1.154382in}{1.382440in}}%
\pgfpathlineto{\pgfqpoint{1.153588in}{1.381536in}}%
\pgfpathlineto{\pgfqpoint{1.146828in}{1.370311in}}%
\pgfpathlineto{\pgfqpoint{1.142658in}{1.358182in}}%
\pgfpathlineto{\pgfqpoint{1.141875in}{1.354474in}}%
\pgfpathlineto{\pgfqpoint{1.139178in}{1.346053in}}%
\pgfpathlineto{\pgfqpoint{1.133549in}{1.333925in}}%
\pgfpathlineto{\pgfqpoint{1.130163in}{1.325243in}}%
\pgfpathlineto{\pgfqpoint{1.128846in}{1.321796in}}%
\pgfpathlineto{\pgfqpoint{1.124443in}{1.309667in}}%
\pgfpathlineto{\pgfqpoint{1.118450in}{1.302205in}}%
\pgfpathlineto{\pgfqpoint{1.113754in}{1.297538in}}%
\pgfpathlineto{\pgfqpoint{1.106737in}{1.290845in}}%
\pgfpathlineto{\pgfqpoint{1.100065in}{1.285409in}}%
\pgfpathlineto{\pgfqpoint{1.095024in}{1.281470in}}%
\pgfpathlineto{\pgfqpoint{1.085545in}{1.273280in}}%
\pgfpathlineto{\pgfqpoint{1.083312in}{1.271335in}}%
\pgfpathlineto{\pgfqpoint{1.071599in}{1.262957in}}%
\pgfpathlineto{\pgfqpoint{1.068091in}{1.261151in}}%
\pgfpathlineto{\pgfqpoint{1.059886in}{1.257452in}}%
\pgfpathlineto{\pgfqpoint{1.048174in}{1.253599in}}%
\pgfpathlineto{\pgfqpoint{1.036461in}{1.249598in}}%
\pgfpathlineto{\pgfqpoint{1.034715in}{1.249022in}}%
\pgfpathlineto{\pgfqpoint{1.024748in}{1.243486in}}%
\pgfpathlineto{\pgfqpoint{1.018241in}{1.236894in}}%
\pgfpathlineto{\pgfqpoint{1.013036in}{1.231918in}}%
\pgfpathlineto{\pgfqpoint{1.006215in}{1.224765in}}%
\pgfpathlineto{\pgfqpoint{1.001323in}{1.218567in}}%
\pgfpathlineto{\pgfqpoint{0.996073in}{1.212636in}}%
\pgfpathlineto{\pgfqpoint{0.991983in}{1.200507in}}%
\pgfpathlineto{\pgfqpoint{0.990495in}{1.188378in}}%
\pgfpathlineto{\pgfqpoint{0.989610in}{1.181513in}}%
\pgfpathlineto{\pgfqpoint{0.989023in}{1.176249in}}%
\pgfpathlineto{\pgfqpoint{0.986863in}{1.164120in}}%
\pgfpathlineto{\pgfqpoint{0.984600in}{1.151991in}}%
\pgfpathlineto{\pgfqpoint{0.977898in}{1.141620in}}%
\pgfpathlineto{\pgfqpoint{0.976655in}{1.139863in}}%
\pgfpathlineto{\pgfqpoint{0.967628in}{1.127734in}}%
\pgfpathlineto{\pgfqpoint{0.966185in}{1.125288in}}%
\pgfpathlineto{\pgfqpoint{0.960312in}{1.115605in}}%
\pgfpathlineto{\pgfqpoint{0.954499in}{1.103476in}}%
\pgfpathlineto{\pgfqpoint{0.954472in}{1.103396in}}%
\pgfpathlineto{\pgfqpoint{0.949660in}{1.091347in}}%
\pgfpathlineto{\pgfqpoint{0.942759in}{1.084274in}}%
\pgfpathlineto{\pgfqpoint{0.939619in}{1.079218in}}%
\pgfpathlineto{\pgfqpoint{0.935402in}{1.067089in}}%
\pgfpathlineto{\pgfqpoint{0.931693in}{1.054960in}}%
\pgfpathlineto{\pgfqpoint{0.931047in}{1.053304in}}%
\pgfpathlineto{\pgfqpoint{0.926663in}{1.042832in}}%
\pgfpathlineto{\pgfqpoint{0.919334in}{1.031014in}}%
\pgfpathlineto{\pgfqpoint{0.919128in}{1.030703in}}%
\pgfpathlineto{\pgfqpoint{0.912227in}{1.018574in}}%
\pgfpathlineto{\pgfqpoint{0.912148in}{1.006445in}}%
\pgfpathlineto{\pgfqpoint{0.914037in}{0.994316in}}%
\pgfpathlineto{\pgfqpoint{0.916075in}{0.982187in}}%
\pgfpathlineto{\pgfqpoint{0.919334in}{0.972709in}}%
\pgfpathlineto{\pgfqpoint{0.920349in}{0.970058in}}%
\pgfpathlineto{\pgfqpoint{0.927413in}{0.957929in}}%
\pgfpathlineto{\pgfqpoint{0.931047in}{0.953325in}}%
\pgfpathlineto{\pgfqpoint{0.936843in}{0.945801in}}%
\pgfpathlineto{\pgfqpoint{0.942759in}{0.938318in}}%
\pgfpathlineto{\pgfqpoint{0.946417in}{0.933672in}}%
\pgfpathlineto{\pgfqpoint{0.954472in}{0.924644in}}%
\pgfpathlineto{\pgfqpoint{0.961676in}{0.921543in}}%
\pgfpathclose%
\pgfpathmoveto{\pgfqpoint{0.974602in}{0.957929in}}%
\pgfpathlineto{\pgfqpoint{0.971568in}{0.970058in}}%
\pgfpathlineto{\pgfqpoint{0.969620in}{0.982187in}}%
\pgfpathlineto{\pgfqpoint{0.967241in}{0.994316in}}%
\pgfpathlineto{\pgfqpoint{0.966603in}{1.006445in}}%
\pgfpathlineto{\pgfqpoint{0.966185in}{1.007738in}}%
\pgfpathlineto{\pgfqpoint{0.962521in}{1.018574in}}%
\pgfpathlineto{\pgfqpoint{0.959709in}{1.030703in}}%
\pgfpathlineto{\pgfqpoint{0.963964in}{1.042832in}}%
\pgfpathlineto{\pgfqpoint{0.966185in}{1.045765in}}%
\pgfpathlineto{\pgfqpoint{0.977898in}{1.054360in}}%
\pgfpathlineto{\pgfqpoint{0.978897in}{1.054960in}}%
\pgfpathlineto{\pgfqpoint{0.989610in}{1.063555in}}%
\pgfpathlineto{\pgfqpoint{0.995226in}{1.067089in}}%
\pgfpathlineto{\pgfqpoint{1.001323in}{1.071149in}}%
\pgfpathlineto{\pgfqpoint{1.013036in}{1.078608in}}%
\pgfpathlineto{\pgfqpoint{1.020489in}{1.079218in}}%
\pgfpathlineto{\pgfqpoint{1.024748in}{1.082772in}}%
\pgfpathlineto{\pgfqpoint{1.025392in}{1.079218in}}%
\pgfpathlineto{\pgfqpoint{1.034921in}{1.067089in}}%
\pgfpathlineto{\pgfqpoint{1.035121in}{1.054960in}}%
\pgfpathlineto{\pgfqpoint{1.036461in}{1.051801in}}%
\pgfpathlineto{\pgfqpoint{1.038004in}{1.042832in}}%
\pgfpathlineto{\pgfqpoint{1.036461in}{1.037610in}}%
\pgfpathlineto{\pgfqpoint{1.033346in}{1.030703in}}%
\pgfpathlineto{\pgfqpoint{1.026691in}{1.018574in}}%
\pgfpathlineto{\pgfqpoint{1.025358in}{1.006445in}}%
\pgfpathlineto{\pgfqpoint{1.024748in}{1.002490in}}%
\pgfpathlineto{\pgfqpoint{1.023486in}{0.994316in}}%
\pgfpathlineto{\pgfqpoint{1.023837in}{0.982187in}}%
\pgfpathlineto{\pgfqpoint{1.018512in}{0.970058in}}%
\pgfpathlineto{\pgfqpoint{1.013036in}{0.962616in}}%
\pgfpathlineto{\pgfqpoint{1.004727in}{0.957929in}}%
\pgfpathlineto{\pgfqpoint{1.001323in}{0.956419in}}%
\pgfpathlineto{\pgfqpoint{0.989610in}{0.955315in}}%
\pgfpathlineto{\pgfqpoint{0.977898in}{0.955537in}}%
\pgfpathclose%
\pgfpathmoveto{\pgfqpoint{1.023714in}{1.103476in}}%
\pgfpathlineto{\pgfqpoint{1.023164in}{1.115605in}}%
\pgfpathlineto{\pgfqpoint{1.024344in}{1.127734in}}%
\pgfpathlineto{\pgfqpoint{1.024748in}{1.128592in}}%
\pgfpathlineto{\pgfqpoint{1.031788in}{1.139863in}}%
\pgfpathlineto{\pgfqpoint{1.036461in}{1.150814in}}%
\pgfpathlineto{\pgfqpoint{1.037146in}{1.151991in}}%
\pgfpathlineto{\pgfqpoint{1.038151in}{1.164120in}}%
\pgfpathlineto{\pgfqpoint{1.041643in}{1.176249in}}%
\pgfpathlineto{\pgfqpoint{1.045894in}{1.188378in}}%
\pgfpathlineto{\pgfqpoint{1.048174in}{1.192851in}}%
\pgfpathlineto{\pgfqpoint{1.052534in}{1.200507in}}%
\pgfpathlineto{\pgfqpoint{1.059886in}{1.211161in}}%
\pgfpathlineto{\pgfqpoint{1.061039in}{1.212636in}}%
\pgfpathlineto{\pgfqpoint{1.071599in}{1.222034in}}%
\pgfpathlineto{\pgfqpoint{1.074568in}{1.224765in}}%
\pgfpathlineto{\pgfqpoint{1.083312in}{1.232852in}}%
\pgfpathlineto{\pgfqpoint{1.087962in}{1.236894in}}%
\pgfpathlineto{\pgfqpoint{1.095024in}{1.241631in}}%
\pgfpathlineto{\pgfqpoint{1.105033in}{1.249022in}}%
\pgfpathlineto{\pgfqpoint{1.106737in}{1.250394in}}%
\pgfpathlineto{\pgfqpoint{1.118450in}{1.251468in}}%
\pgfpathlineto{\pgfqpoint{1.119950in}{1.249022in}}%
\pgfpathlineto{\pgfqpoint{1.123058in}{1.236894in}}%
\pgfpathlineto{\pgfqpoint{1.124772in}{1.224765in}}%
\pgfpathlineto{\pgfqpoint{1.127297in}{1.212636in}}%
\pgfpathlineto{\pgfqpoint{1.128417in}{1.200507in}}%
\pgfpathlineto{\pgfqpoint{1.126836in}{1.188378in}}%
\pgfpathlineto{\pgfqpoint{1.123349in}{1.176249in}}%
\pgfpathlineto{\pgfqpoint{1.118450in}{1.168980in}}%
\pgfpathlineto{\pgfqpoint{1.112840in}{1.164120in}}%
\pgfpathlineto{\pgfqpoint{1.106737in}{1.159619in}}%
\pgfpathlineto{\pgfqpoint{1.095503in}{1.151991in}}%
\pgfpathlineto{\pgfqpoint{1.095024in}{1.151667in}}%
\pgfpathlineto{\pgfqpoint{1.083312in}{1.148085in}}%
\pgfpathlineto{\pgfqpoint{1.071599in}{1.145157in}}%
\pgfpathlineto{\pgfqpoint{1.064162in}{1.139863in}}%
\pgfpathlineto{\pgfqpoint{1.062479in}{1.127734in}}%
\pgfpathlineto{\pgfqpoint{1.059886in}{1.122823in}}%
\pgfpathlineto{\pgfqpoint{1.054766in}{1.115605in}}%
\pgfpathlineto{\pgfqpoint{1.048476in}{1.103476in}}%
\pgfpathlineto{\pgfqpoint{1.048174in}{1.102919in}}%
\pgfpathlineto{\pgfqpoint{1.036461in}{1.095857in}}%
\pgfpathlineto{\pgfqpoint{1.024748in}{1.096125in}}%
\pgfpathclose%
\pgfusepath{fill}%
\end{pgfscope}%
\begin{pgfscope}%
\pgfpathrectangle{\pgfqpoint{0.211875in}{0.211875in}}{\pgfqpoint{1.313625in}{1.279725in}}%
\pgfusepath{clip}%
\pgfsetbuttcap%
\pgfsetroundjoin%
\definecolor{currentfill}{rgb}{0.901975,0.231521,0.249182}%
\pgfsetfillcolor{currentfill}%
\pgfsetlinewidth{0.000000pt}%
\definecolor{currentstroke}{rgb}{0.000000,0.000000,0.000000}%
\pgfsetstrokecolor{currentstroke}%
\pgfsetdash{}{0pt}%
\pgfpathmoveto{\pgfqpoint{0.485965in}{1.197899in}}%
\pgfpathlineto{\pgfqpoint{0.497677in}{1.195551in}}%
\pgfpathlineto{\pgfqpoint{0.509390in}{1.193754in}}%
\pgfpathlineto{\pgfqpoint{0.521103in}{1.192147in}}%
\pgfpathlineto{\pgfqpoint{0.532815in}{1.191188in}}%
\pgfpathlineto{\pgfqpoint{0.544528in}{1.190611in}}%
\pgfpathlineto{\pgfqpoint{0.556241in}{1.190492in}}%
\pgfpathlineto{\pgfqpoint{0.567953in}{1.190857in}}%
\pgfpathlineto{\pgfqpoint{0.579666in}{1.191408in}}%
\pgfpathlineto{\pgfqpoint{0.591379in}{1.192346in}}%
\pgfpathlineto{\pgfqpoint{0.603091in}{1.193932in}}%
\pgfpathlineto{\pgfqpoint{0.614804in}{1.195815in}}%
\pgfpathlineto{\pgfqpoint{0.626517in}{1.197597in}}%
\pgfpathlineto{\pgfqpoint{0.638230in}{1.199676in}}%
\pgfpathlineto{\pgfqpoint{0.642702in}{1.200507in}}%
\pgfpathlineto{\pgfqpoint{0.649942in}{1.201930in}}%
\pgfpathlineto{\pgfqpoint{0.661655in}{1.204169in}}%
\pgfpathlineto{\pgfqpoint{0.673368in}{1.205767in}}%
\pgfpathlineto{\pgfqpoint{0.685080in}{1.207123in}}%
\pgfpathlineto{\pgfqpoint{0.696793in}{1.208417in}}%
\pgfpathlineto{\pgfqpoint{0.708506in}{1.209611in}}%
\pgfpathlineto{\pgfqpoint{0.720218in}{1.210629in}}%
\pgfpathlineto{\pgfqpoint{0.731931in}{1.211653in}}%
\pgfpathlineto{\pgfqpoint{0.742353in}{1.212636in}}%
\pgfpathlineto{\pgfqpoint{0.743644in}{1.212796in}}%
\pgfpathlineto{\pgfqpoint{0.755356in}{1.214428in}}%
\pgfpathlineto{\pgfqpoint{0.767069in}{1.216361in}}%
\pgfpathlineto{\pgfqpoint{0.778782in}{1.218367in}}%
\pgfpathlineto{\pgfqpoint{0.790495in}{1.220072in}}%
\pgfpathlineto{\pgfqpoint{0.802207in}{1.220687in}}%
\pgfpathlineto{\pgfqpoint{0.813920in}{1.220709in}}%
\pgfpathlineto{\pgfqpoint{0.825633in}{1.220510in}}%
\pgfpathlineto{\pgfqpoint{0.837345in}{1.220850in}}%
\pgfpathlineto{\pgfqpoint{0.849058in}{1.223375in}}%
\pgfpathlineto{\pgfqpoint{0.852817in}{1.224765in}}%
\pgfpathlineto{\pgfqpoint{0.860771in}{1.228099in}}%
\pgfpathlineto{\pgfqpoint{0.872483in}{1.235106in}}%
\pgfpathlineto{\pgfqpoint{0.874705in}{1.236894in}}%
\pgfpathlineto{\pgfqpoint{0.884196in}{1.248539in}}%
\pgfpathlineto{\pgfqpoint{0.884586in}{1.249022in}}%
\pgfpathlineto{\pgfqpoint{0.891017in}{1.261151in}}%
\pgfpathlineto{\pgfqpoint{0.894758in}{1.273280in}}%
\pgfpathlineto{\pgfqpoint{0.895909in}{1.280044in}}%
\pgfpathlineto{\pgfqpoint{0.896983in}{1.285409in}}%
\pgfpathlineto{\pgfqpoint{0.902989in}{1.297538in}}%
\pgfpathlineto{\pgfqpoint{0.907621in}{1.304389in}}%
\pgfpathlineto{\pgfqpoint{0.910895in}{1.309667in}}%
\pgfpathlineto{\pgfqpoint{0.918742in}{1.321796in}}%
\pgfpathlineto{\pgfqpoint{0.919334in}{1.322724in}}%
\pgfpathlineto{\pgfqpoint{0.925830in}{1.333925in}}%
\pgfpathlineto{\pgfqpoint{0.931047in}{1.342665in}}%
\pgfpathlineto{\pgfqpoint{0.932967in}{1.346053in}}%
\pgfpathlineto{\pgfqpoint{0.938515in}{1.358182in}}%
\pgfpathlineto{\pgfqpoint{0.942759in}{1.367508in}}%
\pgfpathlineto{\pgfqpoint{0.944127in}{1.370311in}}%
\pgfpathlineto{\pgfqpoint{0.950277in}{1.382440in}}%
\pgfpathlineto{\pgfqpoint{0.953475in}{1.394569in}}%
\pgfpathlineto{\pgfqpoint{0.954472in}{1.397555in}}%
\pgfpathlineto{\pgfqpoint{0.957673in}{1.406698in}}%
\pgfpathlineto{\pgfqpoint{0.962105in}{1.418827in}}%
\pgfpathlineto{\pgfqpoint{0.966185in}{1.428010in}}%
\pgfpathlineto{\pgfqpoint{0.967601in}{1.430956in}}%
\pgfpathlineto{\pgfqpoint{0.971574in}{1.443084in}}%
\pgfpathlineto{\pgfqpoint{0.974227in}{1.455213in}}%
\pgfpathlineto{\pgfqpoint{0.974174in}{1.467342in}}%
\pgfpathlineto{\pgfqpoint{0.974933in}{1.479471in}}%
\pgfpathlineto{\pgfqpoint{0.976321in}{1.491600in}}%
\pgfpathlineto{\pgfqpoint{0.966185in}{1.491600in}}%
\pgfpathlineto{\pgfqpoint{0.954472in}{1.491600in}}%
\pgfpathlineto{\pgfqpoint{0.942759in}{1.491600in}}%
\pgfpathlineto{\pgfqpoint{0.931047in}{1.491600in}}%
\pgfpathlineto{\pgfqpoint{0.919334in}{1.491600in}}%
\pgfpathlineto{\pgfqpoint{0.907621in}{1.491600in}}%
\pgfpathlineto{\pgfqpoint{0.906148in}{1.491600in}}%
\pgfpathlineto{\pgfqpoint{0.907621in}{1.483461in}}%
\pgfpathlineto{\pgfqpoint{0.908307in}{1.479471in}}%
\pgfpathlineto{\pgfqpoint{0.907621in}{1.473592in}}%
\pgfpathlineto{\pgfqpoint{0.906616in}{1.467342in}}%
\pgfpathlineto{\pgfqpoint{0.901653in}{1.455213in}}%
\pgfpathlineto{\pgfqpoint{0.896849in}{1.443084in}}%
\pgfpathlineto{\pgfqpoint{0.895909in}{1.440436in}}%
\pgfpathlineto{\pgfqpoint{0.892129in}{1.430956in}}%
\pgfpathlineto{\pgfqpoint{0.886959in}{1.418827in}}%
\pgfpathlineto{\pgfqpoint{0.884196in}{1.411139in}}%
\pgfpathlineto{\pgfqpoint{0.882490in}{1.406698in}}%
\pgfpathlineto{\pgfqpoint{0.877798in}{1.394569in}}%
\pgfpathlineto{\pgfqpoint{0.874446in}{1.382440in}}%
\pgfpathlineto{\pgfqpoint{0.872483in}{1.374384in}}%
\pgfpathlineto{\pgfqpoint{0.871315in}{1.370311in}}%
\pgfpathlineto{\pgfqpoint{0.866702in}{1.358182in}}%
\pgfpathlineto{\pgfqpoint{0.860771in}{1.350996in}}%
\pgfpathlineto{\pgfqpoint{0.856577in}{1.346053in}}%
\pgfpathlineto{\pgfqpoint{0.849058in}{1.338059in}}%
\pgfpathlineto{\pgfqpoint{0.846043in}{1.333925in}}%
\pgfpathlineto{\pgfqpoint{0.839305in}{1.321796in}}%
\pgfpathlineto{\pgfqpoint{0.837345in}{1.317548in}}%
\pgfpathlineto{\pgfqpoint{0.833495in}{1.309667in}}%
\pgfpathlineto{\pgfqpoint{0.828778in}{1.297538in}}%
\pgfpathlineto{\pgfqpoint{0.828845in}{1.285409in}}%
\pgfpathlineto{\pgfqpoint{0.828219in}{1.273280in}}%
\pgfpathlineto{\pgfqpoint{0.825633in}{1.269354in}}%
\pgfpathlineto{\pgfqpoint{0.815397in}{1.261151in}}%
\pgfpathlineto{\pgfqpoint{0.813920in}{1.260517in}}%
\pgfpathlineto{\pgfqpoint{0.802207in}{1.259468in}}%
\pgfpathlineto{\pgfqpoint{0.790495in}{1.260643in}}%
\pgfpathlineto{\pgfqpoint{0.778782in}{1.260742in}}%
\pgfpathlineto{\pgfqpoint{0.767069in}{1.260701in}}%
\pgfpathlineto{\pgfqpoint{0.755356in}{1.260708in}}%
\pgfpathlineto{\pgfqpoint{0.743644in}{1.260323in}}%
\pgfpathlineto{\pgfqpoint{0.731931in}{1.259199in}}%
\pgfpathlineto{\pgfqpoint{0.720218in}{1.258281in}}%
\pgfpathlineto{\pgfqpoint{0.708506in}{1.257465in}}%
\pgfpathlineto{\pgfqpoint{0.696793in}{1.256796in}}%
\pgfpathlineto{\pgfqpoint{0.685080in}{1.256315in}}%
\pgfpathlineto{\pgfqpoint{0.673368in}{1.256071in}}%
\pgfpathlineto{\pgfqpoint{0.661655in}{1.255903in}}%
\pgfpathlineto{\pgfqpoint{0.649942in}{1.255783in}}%
\pgfpathlineto{\pgfqpoint{0.638230in}{1.255335in}}%
\pgfpathlineto{\pgfqpoint{0.626517in}{1.254034in}}%
\pgfpathlineto{\pgfqpoint{0.614804in}{1.252061in}}%
\pgfpathlineto{\pgfqpoint{0.603091in}{1.250240in}}%
\pgfpathlineto{\pgfqpoint{0.596587in}{1.249022in}}%
\pgfpathlineto{\pgfqpoint{0.591379in}{1.248103in}}%
\pgfpathlineto{\pgfqpoint{0.579666in}{1.245296in}}%
\pgfpathlineto{\pgfqpoint{0.567953in}{1.244075in}}%
\pgfpathlineto{\pgfqpoint{0.556241in}{1.245105in}}%
\pgfpathlineto{\pgfqpoint{0.544528in}{1.247195in}}%
\pgfpathlineto{\pgfqpoint{0.536100in}{1.249022in}}%
\pgfpathlineto{\pgfqpoint{0.532815in}{1.250188in}}%
\pgfpathlineto{\pgfqpoint{0.521103in}{1.256039in}}%
\pgfpathlineto{\pgfqpoint{0.514632in}{1.261151in}}%
\pgfpathlineto{\pgfqpoint{0.521103in}{1.264258in}}%
\pgfpathlineto{\pgfqpoint{0.531580in}{1.273280in}}%
\pgfpathlineto{\pgfqpoint{0.532815in}{1.273876in}}%
\pgfpathlineto{\pgfqpoint{0.544528in}{1.279071in}}%
\pgfpathlineto{\pgfqpoint{0.556241in}{1.283455in}}%
\pgfpathlineto{\pgfqpoint{0.560469in}{1.285409in}}%
\pgfpathlineto{\pgfqpoint{0.567953in}{1.288564in}}%
\pgfpathlineto{\pgfqpoint{0.579666in}{1.292387in}}%
\pgfpathlineto{\pgfqpoint{0.591379in}{1.296894in}}%
\pgfpathlineto{\pgfqpoint{0.592934in}{1.297538in}}%
\pgfpathlineto{\pgfqpoint{0.603091in}{1.302694in}}%
\pgfpathlineto{\pgfqpoint{0.614718in}{1.309667in}}%
\pgfpathlineto{\pgfqpoint{0.614804in}{1.309729in}}%
\pgfpathlineto{\pgfqpoint{0.626517in}{1.319000in}}%
\pgfpathlineto{\pgfqpoint{0.629972in}{1.321796in}}%
\pgfpathlineto{\pgfqpoint{0.638230in}{1.329738in}}%
\pgfpathlineto{\pgfqpoint{0.642631in}{1.333925in}}%
\pgfpathlineto{\pgfqpoint{0.649942in}{1.340448in}}%
\pgfpathlineto{\pgfqpoint{0.655920in}{1.346053in}}%
\pgfpathlineto{\pgfqpoint{0.661655in}{1.355115in}}%
\pgfpathlineto{\pgfqpoint{0.662716in}{1.358182in}}%
\pgfpathlineto{\pgfqpoint{0.661655in}{1.360810in}}%
\pgfpathlineto{\pgfqpoint{0.658017in}{1.370311in}}%
\pgfpathlineto{\pgfqpoint{0.651936in}{1.382440in}}%
\pgfpathlineto{\pgfqpoint{0.649942in}{1.385827in}}%
\pgfpathlineto{\pgfqpoint{0.645044in}{1.394569in}}%
\pgfpathlineto{\pgfqpoint{0.638230in}{1.405662in}}%
\pgfpathlineto{\pgfqpoint{0.637616in}{1.406698in}}%
\pgfpathlineto{\pgfqpoint{0.629656in}{1.418827in}}%
\pgfpathlineto{\pgfqpoint{0.626517in}{1.423415in}}%
\pgfpathlineto{\pgfqpoint{0.621477in}{1.430956in}}%
\pgfpathlineto{\pgfqpoint{0.614804in}{1.440639in}}%
\pgfpathlineto{\pgfqpoint{0.613088in}{1.443084in}}%
\pgfpathlineto{\pgfqpoint{0.606149in}{1.455213in}}%
\pgfpathlineto{\pgfqpoint{0.603091in}{1.460762in}}%
\pgfpathlineto{\pgfqpoint{0.599492in}{1.467342in}}%
\pgfpathlineto{\pgfqpoint{0.591937in}{1.479471in}}%
\pgfpathlineto{\pgfqpoint{0.591379in}{1.480297in}}%
\pgfpathlineto{\pgfqpoint{0.583557in}{1.491600in}}%
\pgfpathlineto{\pgfqpoint{0.579666in}{1.491600in}}%
\pgfpathlineto{\pgfqpoint{0.567953in}{1.491600in}}%
\pgfpathlineto{\pgfqpoint{0.556241in}{1.491600in}}%
\pgfpathlineto{\pgfqpoint{0.544528in}{1.491600in}}%
\pgfpathlineto{\pgfqpoint{0.532815in}{1.491600in}}%
\pgfpathlineto{\pgfqpoint{0.521103in}{1.491600in}}%
\pgfpathlineto{\pgfqpoint{0.509390in}{1.491600in}}%
\pgfpathlineto{\pgfqpoint{0.497677in}{1.491600in}}%
\pgfpathlineto{\pgfqpoint{0.485965in}{1.491600in}}%
\pgfpathlineto{\pgfqpoint{0.484559in}{1.491600in}}%
\pgfpathlineto{\pgfqpoint{0.485965in}{1.489920in}}%
\pgfpathlineto{\pgfqpoint{0.494664in}{1.479471in}}%
\pgfpathlineto{\pgfqpoint{0.497677in}{1.475596in}}%
\pgfpathlineto{\pgfqpoint{0.504128in}{1.467342in}}%
\pgfpathlineto{\pgfqpoint{0.509390in}{1.459949in}}%
\pgfpathlineto{\pgfqpoint{0.512762in}{1.455213in}}%
\pgfpathlineto{\pgfqpoint{0.520454in}{1.443084in}}%
\pgfpathlineto{\pgfqpoint{0.521103in}{1.441874in}}%
\pgfpathlineto{\pgfqpoint{0.527018in}{1.430956in}}%
\pgfpathlineto{\pgfqpoint{0.532026in}{1.418827in}}%
\pgfpathlineto{\pgfqpoint{0.530018in}{1.406698in}}%
\pgfpathlineto{\pgfqpoint{0.524769in}{1.394569in}}%
\pgfpathlineto{\pgfqpoint{0.521103in}{1.386015in}}%
\pgfpathlineto{\pgfqpoint{0.519665in}{1.382440in}}%
\pgfpathlineto{\pgfqpoint{0.513886in}{1.370311in}}%
\pgfpathlineto{\pgfqpoint{0.509390in}{1.363792in}}%
\pgfpathlineto{\pgfqpoint{0.505162in}{1.358182in}}%
\pgfpathlineto{\pgfqpoint{0.497677in}{1.351551in}}%
\pgfpathlineto{\pgfqpoint{0.491334in}{1.346053in}}%
\pgfpathlineto{\pgfqpoint{0.485965in}{1.341949in}}%
\pgfpathlineto{\pgfqpoint{0.475278in}{1.333925in}}%
\pgfpathlineto{\pgfqpoint{0.474252in}{1.333233in}}%
\pgfpathlineto{\pgfqpoint{0.462539in}{1.325549in}}%
\pgfpathlineto{\pgfqpoint{0.457058in}{1.321796in}}%
\pgfpathlineto{\pgfqpoint{0.450827in}{1.317513in}}%
\pgfpathlineto{\pgfqpoint{0.440925in}{1.309667in}}%
\pgfpathlineto{\pgfqpoint{0.439114in}{1.308135in}}%
\pgfpathlineto{\pgfqpoint{0.427446in}{1.297538in}}%
\pgfpathlineto{\pgfqpoint{0.427401in}{1.297500in}}%
\pgfpathlineto{\pgfqpoint{0.415688in}{1.286584in}}%
\pgfpathlineto{\pgfqpoint{0.414731in}{1.285409in}}%
\pgfpathlineto{\pgfqpoint{0.405453in}{1.273280in}}%
\pgfpathlineto{\pgfqpoint{0.403976in}{1.270907in}}%
\pgfpathlineto{\pgfqpoint{0.398871in}{1.261151in}}%
\pgfpathlineto{\pgfqpoint{0.395783in}{1.249022in}}%
\pgfpathlineto{\pgfqpoint{0.395661in}{1.236894in}}%
\pgfpathlineto{\pgfqpoint{0.403976in}{1.228308in}}%
\pgfpathlineto{\pgfqpoint{0.408860in}{1.224765in}}%
\pgfpathlineto{\pgfqpoint{0.415688in}{1.221322in}}%
\pgfpathlineto{\pgfqpoint{0.427401in}{1.215928in}}%
\pgfpathlineto{\pgfqpoint{0.434715in}{1.212636in}}%
\pgfpathlineto{\pgfqpoint{0.439114in}{1.210885in}}%
\pgfpathlineto{\pgfqpoint{0.450827in}{1.206972in}}%
\pgfpathlineto{\pgfqpoint{0.462539in}{1.203693in}}%
\pgfpathlineto{\pgfqpoint{0.474252in}{1.200661in}}%
\pgfpathlineto{\pgfqpoint{0.474870in}{1.200507in}}%
\pgfpathclose%
\pgfusepath{fill}%
\end{pgfscope}%
\begin{pgfscope}%
\pgfpathrectangle{\pgfqpoint{0.211875in}{0.211875in}}{\pgfqpoint{1.313625in}{1.279725in}}%
\pgfusepath{clip}%
\pgfsetbuttcap%
\pgfsetroundjoin%
\definecolor{currentfill}{rgb}{0.947270,0.405591,0.279023}%
\pgfsetfillcolor{currentfill}%
\pgfsetlinewidth{0.000000pt}%
\definecolor{currentstroke}{rgb}{0.000000,0.000000,0.000000}%
\pgfsetstrokecolor{currentstroke}%
\pgfsetdash{}{0pt}%
\pgfpathmoveto{\pgfqpoint{0.298562in}{0.290841in}}%
\pgfpathlineto{\pgfqpoint{0.310274in}{0.290841in}}%
\pgfpathlineto{\pgfqpoint{0.312321in}{0.290841in}}%
\pgfpathlineto{\pgfqpoint{0.310274in}{0.293003in}}%
\pgfpathlineto{\pgfqpoint{0.298562in}{0.294686in}}%
\pgfpathlineto{\pgfqpoint{0.286849in}{0.293728in}}%
\pgfpathlineto{\pgfqpoint{0.286849in}{0.290841in}}%
\pgfpathclose%
\pgfusepath{fill}%
\end{pgfscope}%
\begin{pgfscope}%
\pgfpathrectangle{\pgfqpoint{0.211875in}{0.211875in}}{\pgfqpoint{1.313625in}{1.279725in}}%
\pgfusepath{clip}%
\pgfsetbuttcap%
\pgfsetroundjoin%
\definecolor{currentfill}{rgb}{0.947270,0.405591,0.279023}%
\pgfsetfillcolor{currentfill}%
\pgfsetlinewidth{0.000000pt}%
\definecolor{currentstroke}{rgb}{0.000000,0.000000,0.000000}%
\pgfsetstrokecolor{currentstroke}%
\pgfsetdash{}{0pt}%
\pgfpathmoveto{\pgfqpoint{1.059886in}{0.302500in}}%
\pgfpathlineto{\pgfqpoint{1.071599in}{0.299445in}}%
\pgfpathlineto{\pgfqpoint{1.083312in}{0.302757in}}%
\pgfpathlineto{\pgfqpoint{1.083805in}{0.302970in}}%
\pgfpathlineto{\pgfqpoint{1.095024in}{0.309767in}}%
\pgfpathlineto{\pgfqpoint{1.106737in}{0.314609in}}%
\pgfpathlineto{\pgfqpoint{1.107939in}{0.315099in}}%
\pgfpathlineto{\pgfqpoint{1.118109in}{0.327228in}}%
\pgfpathlineto{\pgfqpoint{1.118386in}{0.339357in}}%
\pgfpathlineto{\pgfqpoint{1.116212in}{0.351486in}}%
\pgfpathlineto{\pgfqpoint{1.106862in}{0.363615in}}%
\pgfpathlineto{\pgfqpoint{1.106737in}{0.363657in}}%
\pgfpathlineto{\pgfqpoint{1.106607in}{0.363615in}}%
\pgfpathlineto{\pgfqpoint{1.095024in}{0.359763in}}%
\pgfpathlineto{\pgfqpoint{1.083312in}{0.356632in}}%
\pgfpathlineto{\pgfqpoint{1.071599in}{0.352825in}}%
\pgfpathlineto{\pgfqpoint{1.067962in}{0.351486in}}%
\pgfpathlineto{\pgfqpoint{1.059886in}{0.348292in}}%
\pgfpathlineto{\pgfqpoint{1.053279in}{0.339357in}}%
\pgfpathlineto{\pgfqpoint{1.048174in}{0.330098in}}%
\pgfpathlineto{\pgfqpoint{1.046783in}{0.327228in}}%
\pgfpathlineto{\pgfqpoint{1.045420in}{0.315099in}}%
\pgfpathlineto{\pgfqpoint{1.048174in}{0.312146in}}%
\pgfpathlineto{\pgfqpoint{1.058892in}{0.302970in}}%
\pgfpathclose%
\pgfusepath{fill}%
\end{pgfscope}%
\begin{pgfscope}%
\pgfpathrectangle{\pgfqpoint{0.211875in}{0.211875in}}{\pgfqpoint{1.313625in}{1.279725in}}%
\pgfusepath{clip}%
\pgfsetbuttcap%
\pgfsetroundjoin%
\definecolor{currentfill}{rgb}{0.947270,0.405591,0.279023}%
\pgfsetfillcolor{currentfill}%
\pgfsetlinewidth{0.000000pt}%
\definecolor{currentstroke}{rgb}{0.000000,0.000000,0.000000}%
\pgfsetstrokecolor{currentstroke}%
\pgfsetdash{}{0pt}%
\pgfpathmoveto{\pgfqpoint{0.767069in}{0.544915in}}%
\pgfpathlineto{\pgfqpoint{0.778782in}{0.543294in}}%
\pgfpathlineto{\pgfqpoint{0.785170in}{0.545548in}}%
\pgfpathlineto{\pgfqpoint{0.790495in}{0.547843in}}%
\pgfpathlineto{\pgfqpoint{0.802207in}{0.556979in}}%
\pgfpathlineto{\pgfqpoint{0.802657in}{0.557677in}}%
\pgfpathlineto{\pgfqpoint{0.803241in}{0.569805in}}%
\pgfpathlineto{\pgfqpoint{0.803863in}{0.581934in}}%
\pgfpathlineto{\pgfqpoint{0.805216in}{0.594063in}}%
\pgfpathlineto{\pgfqpoint{0.807837in}{0.606192in}}%
\pgfpathlineto{\pgfqpoint{0.808657in}{0.618321in}}%
\pgfpathlineto{\pgfqpoint{0.802207in}{0.625320in}}%
\pgfpathlineto{\pgfqpoint{0.796576in}{0.630450in}}%
\pgfpathlineto{\pgfqpoint{0.790495in}{0.636562in}}%
\pgfpathlineto{\pgfqpoint{0.785588in}{0.642579in}}%
\pgfpathlineto{\pgfqpoint{0.778782in}{0.653190in}}%
\pgfpathlineto{\pgfqpoint{0.777965in}{0.654708in}}%
\pgfpathlineto{\pgfqpoint{0.768665in}{0.666836in}}%
\pgfpathlineto{\pgfqpoint{0.767069in}{0.669304in}}%
\pgfpathlineto{\pgfqpoint{0.760678in}{0.678965in}}%
\pgfpathlineto{\pgfqpoint{0.755356in}{0.682915in}}%
\pgfpathlineto{\pgfqpoint{0.743644in}{0.689453in}}%
\pgfpathlineto{\pgfqpoint{0.740361in}{0.691094in}}%
\pgfpathlineto{\pgfqpoint{0.731931in}{0.693655in}}%
\pgfpathlineto{\pgfqpoint{0.720218in}{0.694951in}}%
\pgfpathlineto{\pgfqpoint{0.708506in}{0.693686in}}%
\pgfpathlineto{\pgfqpoint{0.700429in}{0.691094in}}%
\pgfpathlineto{\pgfqpoint{0.696793in}{0.688672in}}%
\pgfpathlineto{\pgfqpoint{0.691597in}{0.678965in}}%
\pgfpathlineto{\pgfqpoint{0.690948in}{0.666836in}}%
\pgfpathlineto{\pgfqpoint{0.694092in}{0.654708in}}%
\pgfpathlineto{\pgfqpoint{0.696793in}{0.645947in}}%
\pgfpathlineto{\pgfqpoint{0.697792in}{0.642579in}}%
\pgfpathlineto{\pgfqpoint{0.700748in}{0.630450in}}%
\pgfpathlineto{\pgfqpoint{0.702677in}{0.618321in}}%
\pgfpathlineto{\pgfqpoint{0.704578in}{0.606192in}}%
\pgfpathlineto{\pgfqpoint{0.707662in}{0.594063in}}%
\pgfpathlineto{\pgfqpoint{0.708506in}{0.591872in}}%
\pgfpathlineto{\pgfqpoint{0.712893in}{0.581934in}}%
\pgfpathlineto{\pgfqpoint{0.720218in}{0.571339in}}%
\pgfpathlineto{\pgfqpoint{0.721055in}{0.569805in}}%
\pgfpathlineto{\pgfqpoint{0.731931in}{0.559209in}}%
\pgfpathlineto{\pgfqpoint{0.735300in}{0.557677in}}%
\pgfpathlineto{\pgfqpoint{0.743644in}{0.554528in}}%
\pgfpathlineto{\pgfqpoint{0.755356in}{0.550072in}}%
\pgfpathlineto{\pgfqpoint{0.765624in}{0.545548in}}%
\pgfpathclose%
\pgfpathmoveto{\pgfqpoint{0.747141in}{0.581934in}}%
\pgfpathlineto{\pgfqpoint{0.743644in}{0.587401in}}%
\pgfpathlineto{\pgfqpoint{0.740083in}{0.594063in}}%
\pgfpathlineto{\pgfqpoint{0.732949in}{0.606192in}}%
\pgfpathlineto{\pgfqpoint{0.731931in}{0.607942in}}%
\pgfpathlineto{\pgfqpoint{0.727218in}{0.618321in}}%
\pgfpathlineto{\pgfqpoint{0.722671in}{0.630450in}}%
\pgfpathlineto{\pgfqpoint{0.720236in}{0.642579in}}%
\pgfpathlineto{\pgfqpoint{0.720218in}{0.642623in}}%
\pgfpathlineto{\pgfqpoint{0.717735in}{0.654708in}}%
\pgfpathlineto{\pgfqpoint{0.719117in}{0.666836in}}%
\pgfpathlineto{\pgfqpoint{0.720218in}{0.667639in}}%
\pgfpathlineto{\pgfqpoint{0.721257in}{0.666836in}}%
\pgfpathlineto{\pgfqpoint{0.731931in}{0.658058in}}%
\pgfpathlineto{\pgfqpoint{0.735462in}{0.654708in}}%
\pgfpathlineto{\pgfqpoint{0.743644in}{0.645619in}}%
\pgfpathlineto{\pgfqpoint{0.745838in}{0.642579in}}%
\pgfpathlineto{\pgfqpoint{0.752433in}{0.630450in}}%
\pgfpathlineto{\pgfqpoint{0.755356in}{0.624771in}}%
\pgfpathlineto{\pgfqpoint{0.759046in}{0.618321in}}%
\pgfpathlineto{\pgfqpoint{0.767069in}{0.606473in}}%
\pgfpathlineto{\pgfqpoint{0.767291in}{0.606192in}}%
\pgfpathlineto{\pgfqpoint{0.774530in}{0.594063in}}%
\pgfpathlineto{\pgfqpoint{0.771025in}{0.581934in}}%
\pgfpathlineto{\pgfqpoint{0.767069in}{0.577579in}}%
\pgfpathlineto{\pgfqpoint{0.755356in}{0.576834in}}%
\pgfpathclose%
\pgfusepath{fill}%
\end{pgfscope}%
\begin{pgfscope}%
\pgfpathrectangle{\pgfqpoint{0.211875in}{0.211875in}}{\pgfqpoint{1.313625in}{1.279725in}}%
\pgfusepath{clip}%
\pgfsetbuttcap%
\pgfsetroundjoin%
\definecolor{currentfill}{rgb}{0.947270,0.405591,0.279023}%
\pgfsetfillcolor{currentfill}%
\pgfsetlinewidth{0.000000pt}%
\definecolor{currentstroke}{rgb}{0.000000,0.000000,0.000000}%
\pgfsetstrokecolor{currentstroke}%
\pgfsetdash{}{0pt}%
\pgfpathmoveto{\pgfqpoint{1.188726in}{0.665012in}}%
\pgfpathlineto{\pgfqpoint{1.200439in}{0.659496in}}%
\pgfpathlineto{\pgfqpoint{1.212151in}{0.657837in}}%
\pgfpathlineto{\pgfqpoint{1.223864in}{0.657348in}}%
\pgfpathlineto{\pgfqpoint{1.235577in}{0.657280in}}%
\pgfpathlineto{\pgfqpoint{1.247289in}{0.661235in}}%
\pgfpathlineto{\pgfqpoint{1.255657in}{0.666836in}}%
\pgfpathlineto{\pgfqpoint{1.259002in}{0.668995in}}%
\pgfpathlineto{\pgfqpoint{1.270715in}{0.670295in}}%
\pgfpathlineto{\pgfqpoint{1.282427in}{0.668093in}}%
\pgfpathlineto{\pgfqpoint{1.284889in}{0.666836in}}%
\pgfpathlineto{\pgfqpoint{1.294140in}{0.664125in}}%
\pgfpathlineto{\pgfqpoint{1.305853in}{0.665441in}}%
\pgfpathlineto{\pgfqpoint{1.309058in}{0.666836in}}%
\pgfpathlineto{\pgfqpoint{1.317566in}{0.669793in}}%
\pgfpathlineto{\pgfqpoint{1.329278in}{0.676922in}}%
\pgfpathlineto{\pgfqpoint{1.331510in}{0.678965in}}%
\pgfpathlineto{\pgfqpoint{1.340991in}{0.685673in}}%
\pgfpathlineto{\pgfqpoint{1.348403in}{0.691094in}}%
\pgfpathlineto{\pgfqpoint{1.350823in}{0.703223in}}%
\pgfpathlineto{\pgfqpoint{1.351915in}{0.715352in}}%
\pgfpathlineto{\pgfqpoint{1.352704in}{0.717369in}}%
\pgfpathlineto{\pgfqpoint{1.364416in}{0.720108in}}%
\pgfpathlineto{\pgfqpoint{1.376129in}{0.725381in}}%
\pgfpathlineto{\pgfqpoint{1.387842in}{0.716604in}}%
\pgfpathlineto{\pgfqpoint{1.388246in}{0.715352in}}%
\pgfpathlineto{\pgfqpoint{1.388242in}{0.703223in}}%
\pgfpathlineto{\pgfqpoint{1.391740in}{0.691094in}}%
\pgfpathlineto{\pgfqpoint{1.399554in}{0.685267in}}%
\pgfpathlineto{\pgfqpoint{1.407734in}{0.678965in}}%
\pgfpathlineto{\pgfqpoint{1.411267in}{0.676250in}}%
\pgfpathlineto{\pgfqpoint{1.422980in}{0.670737in}}%
\pgfpathlineto{\pgfqpoint{1.434692in}{0.671220in}}%
\pgfpathlineto{\pgfqpoint{1.446405in}{0.673149in}}%
\pgfpathlineto{\pgfqpoint{1.446405in}{0.678965in}}%
\pgfpathlineto{\pgfqpoint{1.446405in}{0.685971in}}%
\pgfpathlineto{\pgfqpoint{1.434692in}{0.680410in}}%
\pgfpathlineto{\pgfqpoint{1.425087in}{0.678965in}}%
\pgfpathlineto{\pgfqpoint{1.422980in}{0.678764in}}%
\pgfpathlineto{\pgfqpoint{1.422570in}{0.678965in}}%
\pgfpathlineto{\pgfqpoint{1.411267in}{0.685494in}}%
\pgfpathlineto{\pgfqpoint{1.403804in}{0.691094in}}%
\pgfpathlineto{\pgfqpoint{1.401686in}{0.703223in}}%
\pgfpathlineto{\pgfqpoint{1.411267in}{0.715261in}}%
\pgfpathlineto{\pgfqpoint{1.411418in}{0.715352in}}%
\pgfpathlineto{\pgfqpoint{1.422980in}{0.720698in}}%
\pgfpathlineto{\pgfqpoint{1.434692in}{0.717338in}}%
\pgfpathlineto{\pgfqpoint{1.440248in}{0.715352in}}%
\pgfpathlineto{\pgfqpoint{1.446405in}{0.711897in}}%
\pgfpathlineto{\pgfqpoint{1.446405in}{0.715352in}}%
\pgfpathlineto{\pgfqpoint{1.446405in}{0.725727in}}%
\pgfpathlineto{\pgfqpoint{1.441956in}{0.727481in}}%
\pgfpathlineto{\pgfqpoint{1.434692in}{0.730055in}}%
\pgfpathlineto{\pgfqpoint{1.422980in}{0.733187in}}%
\pgfpathlineto{\pgfqpoint{1.411267in}{0.732637in}}%
\pgfpathlineto{\pgfqpoint{1.401855in}{0.739610in}}%
\pgfpathlineto{\pgfqpoint{1.399625in}{0.751739in}}%
\pgfpathlineto{\pgfqpoint{1.400671in}{0.763867in}}%
\pgfpathlineto{\pgfqpoint{1.404334in}{0.775996in}}%
\pgfpathlineto{\pgfqpoint{1.410711in}{0.788125in}}%
\pgfpathlineto{\pgfqpoint{1.411267in}{0.788831in}}%
\pgfpathlineto{\pgfqpoint{1.420137in}{0.800254in}}%
\pgfpathlineto{\pgfqpoint{1.422980in}{0.802236in}}%
\pgfpathlineto{\pgfqpoint{1.434692in}{0.810563in}}%
\pgfpathlineto{\pgfqpoint{1.436783in}{0.812383in}}%
\pgfpathlineto{\pgfqpoint{1.446405in}{0.820969in}}%
\pgfpathlineto{\pgfqpoint{1.446405in}{0.824512in}}%
\pgfpathlineto{\pgfqpoint{1.446405in}{0.832833in}}%
\pgfpathlineto{\pgfqpoint{1.439050in}{0.824512in}}%
\pgfpathlineto{\pgfqpoint{1.434692in}{0.820504in}}%
\pgfpathlineto{\pgfqpoint{1.422980in}{0.813484in}}%
\pgfpathlineto{\pgfqpoint{1.415811in}{0.812383in}}%
\pgfpathlineto{\pgfqpoint{1.411267in}{0.810774in}}%
\pgfpathlineto{\pgfqpoint{1.403747in}{0.812383in}}%
\pgfpathlineto{\pgfqpoint{1.399554in}{0.815861in}}%
\pgfpathlineto{\pgfqpoint{1.398423in}{0.824512in}}%
\pgfpathlineto{\pgfqpoint{1.399554in}{0.826855in}}%
\pgfpathlineto{\pgfqpoint{1.404422in}{0.836641in}}%
\pgfpathlineto{\pgfqpoint{1.411267in}{0.844204in}}%
\pgfpathlineto{\pgfqpoint{1.415546in}{0.848770in}}%
\pgfpathlineto{\pgfqpoint{1.422980in}{0.854525in}}%
\pgfpathlineto{\pgfqpoint{1.434692in}{0.859229in}}%
\pgfpathlineto{\pgfqpoint{1.440798in}{0.860898in}}%
\pgfpathlineto{\pgfqpoint{1.446405in}{0.862641in}}%
\pgfpathlineto{\pgfqpoint{1.446405in}{0.873027in}}%
\pgfpathlineto{\pgfqpoint{1.446405in}{0.873043in}}%
\pgfpathlineto{\pgfqpoint{1.446365in}{0.873027in}}%
\pgfpathlineto{\pgfqpoint{1.434692in}{0.868980in}}%
\pgfpathlineto{\pgfqpoint{1.422980in}{0.864409in}}%
\pgfpathlineto{\pgfqpoint{1.416652in}{0.860898in}}%
\pgfpathlineto{\pgfqpoint{1.411267in}{0.857203in}}%
\pgfpathlineto{\pgfqpoint{1.403065in}{0.848770in}}%
\pgfpathlineto{\pgfqpoint{1.399554in}{0.844879in}}%
\pgfpathlineto{\pgfqpoint{1.393365in}{0.836641in}}%
\pgfpathlineto{\pgfqpoint{1.387842in}{0.826302in}}%
\pgfpathlineto{\pgfqpoint{1.386824in}{0.824512in}}%
\pgfpathlineto{\pgfqpoint{1.379901in}{0.812383in}}%
\pgfpathlineto{\pgfqpoint{1.376129in}{0.804583in}}%
\pgfpathlineto{\pgfqpoint{1.374202in}{0.800254in}}%
\pgfpathlineto{\pgfqpoint{1.367834in}{0.788125in}}%
\pgfpathlineto{\pgfqpoint{1.364416in}{0.782551in}}%
\pgfpathlineto{\pgfqpoint{1.360313in}{0.775996in}}%
\pgfpathlineto{\pgfqpoint{1.353784in}{0.763867in}}%
\pgfpathlineto{\pgfqpoint{1.352704in}{0.759539in}}%
\pgfpathlineto{\pgfqpoint{1.350934in}{0.751739in}}%
\pgfpathlineto{\pgfqpoint{1.347824in}{0.739610in}}%
\pgfpathlineto{\pgfqpoint{1.341060in}{0.727481in}}%
\pgfpathlineto{\pgfqpoint{1.340991in}{0.727455in}}%
\pgfpathlineto{\pgfqpoint{1.329278in}{0.724583in}}%
\pgfpathlineto{\pgfqpoint{1.317566in}{0.722016in}}%
\pgfpathlineto{\pgfqpoint{1.307482in}{0.715352in}}%
\pgfpathlineto{\pgfqpoint{1.305853in}{0.714250in}}%
\pgfpathlineto{\pgfqpoint{1.294140in}{0.705586in}}%
\pgfpathlineto{\pgfqpoint{1.290261in}{0.703223in}}%
\pgfpathlineto{\pgfqpoint{1.282427in}{0.698540in}}%
\pgfpathlineto{\pgfqpoint{1.270715in}{0.694207in}}%
\pgfpathlineto{\pgfqpoint{1.259002in}{0.695792in}}%
\pgfpathlineto{\pgfqpoint{1.247302in}{0.703223in}}%
\pgfpathlineto{\pgfqpoint{1.247289in}{0.703232in}}%
\pgfpathlineto{\pgfqpoint{1.235577in}{0.713860in}}%
\pgfpathlineto{\pgfqpoint{1.233509in}{0.715352in}}%
\pgfpathlineto{\pgfqpoint{1.223864in}{0.721580in}}%
\pgfpathlineto{\pgfqpoint{1.212151in}{0.725230in}}%
\pgfpathlineto{\pgfqpoint{1.204398in}{0.727481in}}%
\pgfpathlineto{\pgfqpoint{1.200439in}{0.728805in}}%
\pgfpathlineto{\pgfqpoint{1.188726in}{0.732176in}}%
\pgfpathlineto{\pgfqpoint{1.182914in}{0.727481in}}%
\pgfpathlineto{\pgfqpoint{1.178036in}{0.715352in}}%
\pgfpathlineto{\pgfqpoint{1.177013in}{0.712175in}}%
\pgfpathlineto{\pgfqpoint{1.172778in}{0.703223in}}%
\pgfpathlineto{\pgfqpoint{1.166676in}{0.691094in}}%
\pgfpathlineto{\pgfqpoint{1.175909in}{0.678965in}}%
\pgfpathlineto{\pgfqpoint{1.177013in}{0.677744in}}%
\pgfpathlineto{\pgfqpoint{1.185845in}{0.666836in}}%
\pgfpathclose%
\pgfpathmoveto{\pgfqpoint{1.210918in}{0.666836in}}%
\pgfpathlineto{\pgfqpoint{1.200439in}{0.671124in}}%
\pgfpathlineto{\pgfqpoint{1.194820in}{0.678965in}}%
\pgfpathlineto{\pgfqpoint{1.189767in}{0.691094in}}%
\pgfpathlineto{\pgfqpoint{1.193141in}{0.703223in}}%
\pgfpathlineto{\pgfqpoint{1.198135in}{0.715352in}}%
\pgfpathlineto{\pgfqpoint{1.200439in}{0.716460in}}%
\pgfpathlineto{\pgfqpoint{1.204638in}{0.715352in}}%
\pgfpathlineto{\pgfqpoint{1.212151in}{0.713433in}}%
\pgfpathlineto{\pgfqpoint{1.223864in}{0.708846in}}%
\pgfpathlineto{\pgfqpoint{1.231231in}{0.703223in}}%
\pgfpathlineto{\pgfqpoint{1.235577in}{0.698437in}}%
\pgfpathlineto{\pgfqpoint{1.242276in}{0.691094in}}%
\pgfpathlineto{\pgfqpoint{1.247289in}{0.680271in}}%
\pgfpathlineto{\pgfqpoint{1.248274in}{0.678965in}}%
\pgfpathlineto{\pgfqpoint{1.247289in}{0.678341in}}%
\pgfpathlineto{\pgfqpoint{1.235577in}{0.669171in}}%
\pgfpathlineto{\pgfqpoint{1.225374in}{0.666836in}}%
\pgfpathlineto{\pgfqpoint{1.223864in}{0.666623in}}%
\pgfpathlineto{\pgfqpoint{1.212151in}{0.666657in}}%
\pgfpathclose%
\pgfpathmoveto{\pgfqpoint{1.295939in}{0.678965in}}%
\pgfpathlineto{\pgfqpoint{1.294795in}{0.691094in}}%
\pgfpathlineto{\pgfqpoint{1.305853in}{0.700859in}}%
\pgfpathlineto{\pgfqpoint{1.308634in}{0.703223in}}%
\pgfpathlineto{\pgfqpoint{1.317566in}{0.709616in}}%
\pgfpathlineto{\pgfqpoint{1.329278in}{0.712172in}}%
\pgfpathlineto{\pgfqpoint{1.340991in}{0.708663in}}%
\pgfpathlineto{\pgfqpoint{1.342577in}{0.703223in}}%
\pgfpathlineto{\pgfqpoint{1.340991in}{0.691587in}}%
\pgfpathlineto{\pgfqpoint{1.340923in}{0.691094in}}%
\pgfpathlineto{\pgfqpoint{1.329278in}{0.683073in}}%
\pgfpathlineto{\pgfqpoint{1.322475in}{0.678965in}}%
\pgfpathlineto{\pgfqpoint{1.317566in}{0.675855in}}%
\pgfpathlineto{\pgfqpoint{1.305853in}{0.674964in}}%
\pgfpathclose%
\pgfpathmoveto{\pgfqpoint{1.360092in}{0.751739in}}%
\pgfpathlineto{\pgfqpoint{1.362732in}{0.763867in}}%
\pgfpathlineto{\pgfqpoint{1.364416in}{0.766753in}}%
\pgfpathlineto{\pgfqpoint{1.374346in}{0.775996in}}%
\pgfpathlineto{\pgfqpoint{1.376129in}{0.780545in}}%
\pgfpathlineto{\pgfqpoint{1.385443in}{0.775996in}}%
\pgfpathlineto{\pgfqpoint{1.383178in}{0.763867in}}%
\pgfpathlineto{\pgfqpoint{1.381236in}{0.751739in}}%
\pgfpathlineto{\pgfqpoint{1.376129in}{0.745341in}}%
\pgfpathlineto{\pgfqpoint{1.364416in}{0.740225in}}%
\pgfpathclose%
\pgfusepath{fill}%
\end{pgfscope}%
\begin{pgfscope}%
\pgfpathrectangle{\pgfqpoint{0.211875in}{0.211875in}}{\pgfqpoint{1.313625in}{1.279725in}}%
\pgfusepath{clip}%
\pgfsetbuttcap%
\pgfsetroundjoin%
\definecolor{currentfill}{rgb}{0.947270,0.405591,0.279023}%
\pgfsetfillcolor{currentfill}%
\pgfsetlinewidth{0.000000pt}%
\definecolor{currentstroke}{rgb}{0.000000,0.000000,0.000000}%
\pgfsetstrokecolor{currentstroke}%
\pgfsetdash{}{0pt}%
\pgfpathmoveto{\pgfqpoint{0.767069in}{0.732927in}}%
\pgfpathlineto{\pgfqpoint{0.778782in}{0.730083in}}%
\pgfpathlineto{\pgfqpoint{0.790495in}{0.730610in}}%
\pgfpathlineto{\pgfqpoint{0.802207in}{0.732992in}}%
\pgfpathlineto{\pgfqpoint{0.813920in}{0.736892in}}%
\pgfpathlineto{\pgfqpoint{0.818997in}{0.739610in}}%
\pgfpathlineto{\pgfqpoint{0.818715in}{0.751739in}}%
\pgfpathlineto{\pgfqpoint{0.813920in}{0.758717in}}%
\pgfpathlineto{\pgfqpoint{0.802207in}{0.760247in}}%
\pgfpathlineto{\pgfqpoint{0.790495in}{0.760077in}}%
\pgfpathlineto{\pgfqpoint{0.778782in}{0.761497in}}%
\pgfpathlineto{\pgfqpoint{0.769580in}{0.763867in}}%
\pgfpathlineto{\pgfqpoint{0.767069in}{0.764408in}}%
\pgfpathlineto{\pgfqpoint{0.755356in}{0.767590in}}%
\pgfpathlineto{\pgfqpoint{0.743644in}{0.771529in}}%
\pgfpathlineto{\pgfqpoint{0.733537in}{0.775996in}}%
\pgfpathlineto{\pgfqpoint{0.731931in}{0.776534in}}%
\pgfpathlineto{\pgfqpoint{0.720218in}{0.780564in}}%
\pgfpathlineto{\pgfqpoint{0.708506in}{0.784703in}}%
\pgfpathlineto{\pgfqpoint{0.698874in}{0.788125in}}%
\pgfpathlineto{\pgfqpoint{0.696793in}{0.788585in}}%
\pgfpathlineto{\pgfqpoint{0.685080in}{0.791449in}}%
\pgfpathlineto{\pgfqpoint{0.673368in}{0.794423in}}%
\pgfpathlineto{\pgfqpoint{0.661655in}{0.797123in}}%
\pgfpathlineto{\pgfqpoint{0.649942in}{0.799430in}}%
\pgfpathlineto{\pgfqpoint{0.643216in}{0.800254in}}%
\pgfpathlineto{\pgfqpoint{0.638230in}{0.800998in}}%
\pgfpathlineto{\pgfqpoint{0.626517in}{0.803152in}}%
\pgfpathlineto{\pgfqpoint{0.614804in}{0.807214in}}%
\pgfpathlineto{\pgfqpoint{0.604214in}{0.812383in}}%
\pgfpathlineto{\pgfqpoint{0.603091in}{0.813227in}}%
\pgfpathlineto{\pgfqpoint{0.591379in}{0.821891in}}%
\pgfpathlineto{\pgfqpoint{0.588085in}{0.824512in}}%
\pgfpathlineto{\pgfqpoint{0.579666in}{0.832455in}}%
\pgfpathlineto{\pgfqpoint{0.575056in}{0.836641in}}%
\pgfpathlineto{\pgfqpoint{0.567953in}{0.844036in}}%
\pgfpathlineto{\pgfqpoint{0.563115in}{0.848770in}}%
\pgfpathlineto{\pgfqpoint{0.556241in}{0.857339in}}%
\pgfpathlineto{\pgfqpoint{0.553611in}{0.860898in}}%
\pgfpathlineto{\pgfqpoint{0.545559in}{0.873027in}}%
\pgfpathlineto{\pgfqpoint{0.544528in}{0.874184in}}%
\pgfpathlineto{\pgfqpoint{0.532815in}{0.884309in}}%
\pgfpathlineto{\pgfqpoint{0.531153in}{0.885156in}}%
\pgfpathlineto{\pgfqpoint{0.521103in}{0.889832in}}%
\pgfpathlineto{\pgfqpoint{0.509390in}{0.894828in}}%
\pgfpathlineto{\pgfqpoint{0.502622in}{0.897285in}}%
\pgfpathlineto{\pgfqpoint{0.497677in}{0.899532in}}%
\pgfpathlineto{\pgfqpoint{0.487213in}{0.909414in}}%
\pgfpathlineto{\pgfqpoint{0.487990in}{0.921543in}}%
\pgfpathlineto{\pgfqpoint{0.488455in}{0.933672in}}%
\pgfpathlineto{\pgfqpoint{0.488059in}{0.945801in}}%
\pgfpathlineto{\pgfqpoint{0.485965in}{0.955647in}}%
\pgfpathlineto{\pgfqpoint{0.485420in}{0.957929in}}%
\pgfpathlineto{\pgfqpoint{0.480342in}{0.970058in}}%
\pgfpathlineto{\pgfqpoint{0.474252in}{0.978892in}}%
\pgfpathlineto{\pgfqpoint{0.471769in}{0.982187in}}%
\pgfpathlineto{\pgfqpoint{0.463654in}{0.994316in}}%
\pgfpathlineto{\pgfqpoint{0.462539in}{0.996046in}}%
\pgfpathlineto{\pgfqpoint{0.455549in}{1.006445in}}%
\pgfpathlineto{\pgfqpoint{0.450827in}{1.013938in}}%
\pgfpathlineto{\pgfqpoint{0.447891in}{1.018574in}}%
\pgfpathlineto{\pgfqpoint{0.439932in}{1.030703in}}%
\pgfpathlineto{\pgfqpoint{0.439114in}{1.031870in}}%
\pgfpathlineto{\pgfqpoint{0.431232in}{1.042832in}}%
\pgfpathlineto{\pgfqpoint{0.427401in}{1.047995in}}%
\pgfpathlineto{\pgfqpoint{0.422144in}{1.054960in}}%
\pgfpathlineto{\pgfqpoint{0.415688in}{1.061632in}}%
\pgfpathlineto{\pgfqpoint{0.410890in}{1.067089in}}%
\pgfpathlineto{\pgfqpoint{0.403976in}{1.072805in}}%
\pgfpathlineto{\pgfqpoint{0.396389in}{1.079218in}}%
\pgfpathlineto{\pgfqpoint{0.392263in}{1.082590in}}%
\pgfpathlineto{\pgfqpoint{0.381024in}{1.091347in}}%
\pgfpathlineto{\pgfqpoint{0.380550in}{1.091688in}}%
\pgfpathlineto{\pgfqpoint{0.368838in}{1.098806in}}%
\pgfpathlineto{\pgfqpoint{0.360993in}{1.103476in}}%
\pgfpathlineto{\pgfqpoint{0.357125in}{1.105753in}}%
\pgfpathlineto{\pgfqpoint{0.345412in}{1.112774in}}%
\pgfpathlineto{\pgfqpoint{0.340628in}{1.115605in}}%
\pgfpathlineto{\pgfqpoint{0.333700in}{1.119841in}}%
\pgfpathlineto{\pgfqpoint{0.321987in}{1.126945in}}%
\pgfpathlineto{\pgfqpoint{0.320665in}{1.127734in}}%
\pgfpathlineto{\pgfqpoint{0.310274in}{1.134404in}}%
\pgfpathlineto{\pgfqpoint{0.301435in}{1.139863in}}%
\pgfpathlineto{\pgfqpoint{0.298562in}{1.141935in}}%
\pgfpathlineto{\pgfqpoint{0.293531in}{1.139863in}}%
\pgfpathlineto{\pgfqpoint{0.286849in}{1.128207in}}%
\pgfpathlineto{\pgfqpoint{0.286849in}{1.127734in}}%
\pgfpathlineto{\pgfqpoint{0.286849in}{1.115605in}}%
\pgfpathlineto{\pgfqpoint{0.286849in}{1.103476in}}%
\pgfpathlineto{\pgfqpoint{0.286849in}{1.091347in}}%
\pgfpathlineto{\pgfqpoint{0.286849in}{1.079218in}}%
\pgfpathlineto{\pgfqpoint{0.286849in}{1.069147in}}%
\pgfpathlineto{\pgfqpoint{0.288850in}{1.067089in}}%
\pgfpathlineto{\pgfqpoint{0.298562in}{1.058349in}}%
\pgfpathlineto{\pgfqpoint{0.302421in}{1.054960in}}%
\pgfpathlineto{\pgfqpoint{0.310274in}{1.048654in}}%
\pgfpathlineto{\pgfqpoint{0.317251in}{1.042832in}}%
\pgfpathlineto{\pgfqpoint{0.321987in}{1.038939in}}%
\pgfpathlineto{\pgfqpoint{0.331739in}{1.030703in}}%
\pgfpathlineto{\pgfqpoint{0.333700in}{1.029015in}}%
\pgfpathlineto{\pgfqpoint{0.345371in}{1.018574in}}%
\pgfpathlineto{\pgfqpoint{0.345412in}{1.018537in}}%
\pgfpathlineto{\pgfqpoint{0.357125in}{1.007567in}}%
\pgfpathlineto{\pgfqpoint{0.358183in}{1.006445in}}%
\pgfpathlineto{\pgfqpoint{0.368838in}{0.995989in}}%
\pgfpathlineto{\pgfqpoint{0.370335in}{0.994316in}}%
\pgfpathlineto{\pgfqpoint{0.380550in}{0.983075in}}%
\pgfpathlineto{\pgfqpoint{0.381240in}{0.982187in}}%
\pgfpathlineto{\pgfqpoint{0.391661in}{0.970058in}}%
\pgfpathlineto{\pgfqpoint{0.392263in}{0.969485in}}%
\pgfpathlineto{\pgfqpoint{0.403201in}{0.957929in}}%
\pgfpathlineto{\pgfqpoint{0.403976in}{0.957169in}}%
\pgfpathlineto{\pgfqpoint{0.413612in}{0.945801in}}%
\pgfpathlineto{\pgfqpoint{0.415688in}{0.943555in}}%
\pgfpathlineto{\pgfqpoint{0.423682in}{0.933672in}}%
\pgfpathlineto{\pgfqpoint{0.427401in}{0.929498in}}%
\pgfpathlineto{\pgfqpoint{0.433863in}{0.921543in}}%
\pgfpathlineto{\pgfqpoint{0.439114in}{0.915456in}}%
\pgfpathlineto{\pgfqpoint{0.445149in}{0.909414in}}%
\pgfpathlineto{\pgfqpoint{0.450827in}{0.902187in}}%
\pgfpathlineto{\pgfqpoint{0.457674in}{0.897285in}}%
\pgfpathlineto{\pgfqpoint{0.462539in}{0.890634in}}%
\pgfpathlineto{\pgfqpoint{0.465767in}{0.885156in}}%
\pgfpathlineto{\pgfqpoint{0.465931in}{0.873027in}}%
\pgfpathlineto{\pgfqpoint{0.466267in}{0.860898in}}%
\pgfpathlineto{\pgfqpoint{0.465560in}{0.848770in}}%
\pgfpathlineto{\pgfqpoint{0.466348in}{0.836641in}}%
\pgfpathlineto{\pgfqpoint{0.474252in}{0.825854in}}%
\pgfpathlineto{\pgfqpoint{0.475858in}{0.824512in}}%
\pgfpathlineto{\pgfqpoint{0.485965in}{0.818825in}}%
\pgfpathlineto{\pgfqpoint{0.494789in}{0.812383in}}%
\pgfpathlineto{\pgfqpoint{0.497677in}{0.810677in}}%
\pgfpathlineto{\pgfqpoint{0.509390in}{0.803545in}}%
\pgfpathlineto{\pgfqpoint{0.514542in}{0.800254in}}%
\pgfpathlineto{\pgfqpoint{0.521103in}{0.796904in}}%
\pgfpathlineto{\pgfqpoint{0.532815in}{0.792097in}}%
\pgfpathlineto{\pgfqpoint{0.543665in}{0.788125in}}%
\pgfpathlineto{\pgfqpoint{0.544528in}{0.787895in}}%
\pgfpathlineto{\pgfqpoint{0.556241in}{0.785498in}}%
\pgfpathlineto{\pgfqpoint{0.567953in}{0.783481in}}%
\pgfpathlineto{\pgfqpoint{0.579666in}{0.781863in}}%
\pgfpathlineto{\pgfqpoint{0.591379in}{0.780002in}}%
\pgfpathlineto{\pgfqpoint{0.603091in}{0.778232in}}%
\pgfpathlineto{\pgfqpoint{0.613916in}{0.775996in}}%
\pgfpathlineto{\pgfqpoint{0.614804in}{0.775810in}}%
\pgfpathlineto{\pgfqpoint{0.626517in}{0.772770in}}%
\pgfpathlineto{\pgfqpoint{0.638230in}{0.768879in}}%
\pgfpathlineto{\pgfqpoint{0.649942in}{0.765607in}}%
\pgfpathlineto{\pgfqpoint{0.656202in}{0.763867in}}%
\pgfpathlineto{\pgfqpoint{0.661655in}{0.762034in}}%
\pgfpathlineto{\pgfqpoint{0.673368in}{0.758757in}}%
\pgfpathlineto{\pgfqpoint{0.685080in}{0.756574in}}%
\pgfpathlineto{\pgfqpoint{0.696793in}{0.754635in}}%
\pgfpathlineto{\pgfqpoint{0.708506in}{0.753433in}}%
\pgfpathlineto{\pgfqpoint{0.717855in}{0.751739in}}%
\pgfpathlineto{\pgfqpoint{0.720218in}{0.751213in}}%
\pgfpathlineto{\pgfqpoint{0.731931in}{0.746614in}}%
\pgfpathlineto{\pgfqpoint{0.743644in}{0.743420in}}%
\pgfpathlineto{\pgfqpoint{0.755356in}{0.739669in}}%
\pgfpathlineto{\pgfqpoint{0.755513in}{0.739610in}}%
\pgfpathclose%
\pgfpathmoveto{\pgfqpoint{0.646274in}{0.775996in}}%
\pgfpathlineto{\pgfqpoint{0.649942in}{0.780709in}}%
\pgfpathlineto{\pgfqpoint{0.661655in}{0.778804in}}%
\pgfpathlineto{\pgfqpoint{0.668541in}{0.775996in}}%
\pgfpathlineto{\pgfqpoint{0.661655in}{0.774753in}}%
\pgfpathlineto{\pgfqpoint{0.649942in}{0.774851in}}%
\pgfpathclose%
\pgfpathmoveto{\pgfqpoint{0.496838in}{0.848770in}}%
\pgfpathlineto{\pgfqpoint{0.497677in}{0.849344in}}%
\pgfpathlineto{\pgfqpoint{0.498494in}{0.848770in}}%
\pgfpathlineto{\pgfqpoint{0.497677in}{0.847326in}}%
\pgfpathclose%
\pgfpathmoveto{\pgfqpoint{0.356410in}{1.054960in}}%
\pgfpathlineto{\pgfqpoint{0.357125in}{1.061149in}}%
\pgfpathlineto{\pgfqpoint{0.361464in}{1.054960in}}%
\pgfpathlineto{\pgfqpoint{0.357125in}{1.054092in}}%
\pgfpathclose%
\pgfusepath{fill}%
\end{pgfscope}%
\begin{pgfscope}%
\pgfpathrectangle{\pgfqpoint{0.211875in}{0.211875in}}{\pgfqpoint{1.313625in}{1.279725in}}%
\pgfusepath{clip}%
\pgfsetbuttcap%
\pgfsetroundjoin%
\definecolor{currentfill}{rgb}{0.947270,0.405591,0.279023}%
\pgfsetfillcolor{currentfill}%
\pgfsetlinewidth{0.000000pt}%
\definecolor{currentstroke}{rgb}{0.000000,0.000000,0.000000}%
\pgfsetstrokecolor{currentstroke}%
\pgfsetdash{}{0pt}%
\pgfpathmoveto{\pgfqpoint{1.118450in}{0.799705in}}%
\pgfpathlineto{\pgfqpoint{1.130163in}{0.795585in}}%
\pgfpathlineto{\pgfqpoint{1.141875in}{0.794638in}}%
\pgfpathlineto{\pgfqpoint{1.145078in}{0.800254in}}%
\pgfpathlineto{\pgfqpoint{1.148592in}{0.812383in}}%
\pgfpathlineto{\pgfqpoint{1.152284in}{0.824512in}}%
\pgfpathlineto{\pgfqpoint{1.153588in}{0.830824in}}%
\pgfpathlineto{\pgfqpoint{1.155167in}{0.836641in}}%
\pgfpathlineto{\pgfqpoint{1.156022in}{0.848770in}}%
\pgfpathlineto{\pgfqpoint{1.159939in}{0.860898in}}%
\pgfpathlineto{\pgfqpoint{1.165301in}{0.866702in}}%
\pgfpathlineto{\pgfqpoint{1.172037in}{0.873027in}}%
\pgfpathlineto{\pgfqpoint{1.177013in}{0.876226in}}%
\pgfpathlineto{\pgfqpoint{1.188726in}{0.881912in}}%
\pgfpathlineto{\pgfqpoint{1.197705in}{0.885156in}}%
\pgfpathlineto{\pgfqpoint{1.200439in}{0.887245in}}%
\pgfpathlineto{\pgfqpoint{1.207142in}{0.897285in}}%
\pgfpathlineto{\pgfqpoint{1.201348in}{0.909414in}}%
\pgfpathlineto{\pgfqpoint{1.200439in}{0.910277in}}%
\pgfpathlineto{\pgfqpoint{1.188726in}{0.915993in}}%
\pgfpathlineto{\pgfqpoint{1.177013in}{0.916349in}}%
\pgfpathlineto{\pgfqpoint{1.165301in}{0.914246in}}%
\pgfpathlineto{\pgfqpoint{1.153588in}{0.913088in}}%
\pgfpathlineto{\pgfqpoint{1.141875in}{0.913102in}}%
\pgfpathlineto{\pgfqpoint{1.130163in}{0.910955in}}%
\pgfpathlineto{\pgfqpoint{1.129196in}{0.909414in}}%
\pgfpathlineto{\pgfqpoint{1.122646in}{0.897285in}}%
\pgfpathlineto{\pgfqpoint{1.118450in}{0.894775in}}%
\pgfpathlineto{\pgfqpoint{1.106737in}{0.885338in}}%
\pgfpathlineto{\pgfqpoint{1.106500in}{0.885156in}}%
\pgfpathlineto{\pgfqpoint{1.095024in}{0.875996in}}%
\pgfpathlineto{\pgfqpoint{1.092381in}{0.873027in}}%
\pgfpathlineto{\pgfqpoint{1.087591in}{0.860898in}}%
\pgfpathlineto{\pgfqpoint{1.086364in}{0.848770in}}%
\pgfpathlineto{\pgfqpoint{1.085979in}{0.836641in}}%
\pgfpathlineto{\pgfqpoint{1.088152in}{0.824512in}}%
\pgfpathlineto{\pgfqpoint{1.095024in}{0.814535in}}%
\pgfpathlineto{\pgfqpoint{1.097160in}{0.812383in}}%
\pgfpathlineto{\pgfqpoint{1.106737in}{0.805083in}}%
\pgfpathlineto{\pgfqpoint{1.117327in}{0.800254in}}%
\pgfpathclose%
\pgfpathmoveto{\pgfqpoint{1.113224in}{0.812383in}}%
\pgfpathlineto{\pgfqpoint{1.106737in}{0.815633in}}%
\pgfpathlineto{\pgfqpoint{1.097977in}{0.824512in}}%
\pgfpathlineto{\pgfqpoint{1.095024in}{0.831733in}}%
\pgfpathlineto{\pgfqpoint{1.093597in}{0.836641in}}%
\pgfpathlineto{\pgfqpoint{1.093521in}{0.848770in}}%
\pgfpathlineto{\pgfqpoint{1.095024in}{0.856883in}}%
\pgfpathlineto{\pgfqpoint{1.096018in}{0.860898in}}%
\pgfpathlineto{\pgfqpoint{1.106400in}{0.873027in}}%
\pgfpathlineto{\pgfqpoint{1.106737in}{0.873320in}}%
\pgfpathlineto{\pgfqpoint{1.118450in}{0.879620in}}%
\pgfpathlineto{\pgfqpoint{1.130163in}{0.873410in}}%
\pgfpathlineto{\pgfqpoint{1.130639in}{0.873027in}}%
\pgfpathlineto{\pgfqpoint{1.137937in}{0.860898in}}%
\pgfpathlineto{\pgfqpoint{1.141246in}{0.848770in}}%
\pgfpathlineto{\pgfqpoint{1.141557in}{0.836641in}}%
\pgfpathlineto{\pgfqpoint{1.139664in}{0.824512in}}%
\pgfpathlineto{\pgfqpoint{1.133980in}{0.812383in}}%
\pgfpathlineto{\pgfqpoint{1.130163in}{0.808893in}}%
\pgfpathlineto{\pgfqpoint{1.118450in}{0.809930in}}%
\pgfpathclose%
\pgfusepath{fill}%
\end{pgfscope}%
\begin{pgfscope}%
\pgfpathrectangle{\pgfqpoint{0.211875in}{0.211875in}}{\pgfqpoint{1.313625in}{1.279725in}}%
\pgfusepath{clip}%
\pgfsetbuttcap%
\pgfsetroundjoin%
\definecolor{currentfill}{rgb}{0.947270,0.405591,0.279023}%
\pgfsetfillcolor{currentfill}%
\pgfsetlinewidth{0.000000pt}%
\definecolor{currentstroke}{rgb}{0.000000,0.000000,0.000000}%
\pgfsetstrokecolor{currentstroke}%
\pgfsetdash{}{0pt}%
\pgfpathmoveto{\pgfqpoint{0.977898in}{0.955537in}}%
\pgfpathlineto{\pgfqpoint{0.989610in}{0.955315in}}%
\pgfpathlineto{\pgfqpoint{1.001323in}{0.956419in}}%
\pgfpathlineto{\pgfqpoint{1.004727in}{0.957929in}}%
\pgfpathlineto{\pgfqpoint{1.013036in}{0.962616in}}%
\pgfpathlineto{\pgfqpoint{1.018512in}{0.970058in}}%
\pgfpathlineto{\pgfqpoint{1.023837in}{0.982187in}}%
\pgfpathlineto{\pgfqpoint{1.023486in}{0.994316in}}%
\pgfpathlineto{\pgfqpoint{1.024748in}{1.002490in}}%
\pgfpathlineto{\pgfqpoint{1.025358in}{1.006445in}}%
\pgfpathlineto{\pgfqpoint{1.026691in}{1.018574in}}%
\pgfpathlineto{\pgfqpoint{1.033346in}{1.030703in}}%
\pgfpathlineto{\pgfqpoint{1.036461in}{1.037610in}}%
\pgfpathlineto{\pgfqpoint{1.038004in}{1.042832in}}%
\pgfpathlineto{\pgfqpoint{1.036461in}{1.051801in}}%
\pgfpathlineto{\pgfqpoint{1.035121in}{1.054960in}}%
\pgfpathlineto{\pgfqpoint{1.034921in}{1.067089in}}%
\pgfpathlineto{\pgfqpoint{1.025392in}{1.079218in}}%
\pgfpathlineto{\pgfqpoint{1.024748in}{1.082772in}}%
\pgfpathlineto{\pgfqpoint{1.020489in}{1.079218in}}%
\pgfpathlineto{\pgfqpoint{1.013036in}{1.078608in}}%
\pgfpathlineto{\pgfqpoint{1.001323in}{1.071149in}}%
\pgfpathlineto{\pgfqpoint{0.995226in}{1.067089in}}%
\pgfpathlineto{\pgfqpoint{0.989610in}{1.063555in}}%
\pgfpathlineto{\pgfqpoint{0.978897in}{1.054960in}}%
\pgfpathlineto{\pgfqpoint{0.977898in}{1.054360in}}%
\pgfpathlineto{\pgfqpoint{0.966185in}{1.045765in}}%
\pgfpathlineto{\pgfqpoint{0.963964in}{1.042832in}}%
\pgfpathlineto{\pgfqpoint{0.959709in}{1.030703in}}%
\pgfpathlineto{\pgfqpoint{0.962521in}{1.018574in}}%
\pgfpathlineto{\pgfqpoint{0.966185in}{1.007738in}}%
\pgfpathlineto{\pgfqpoint{0.966603in}{1.006445in}}%
\pgfpathlineto{\pgfqpoint{0.967241in}{0.994316in}}%
\pgfpathlineto{\pgfqpoint{0.969620in}{0.982187in}}%
\pgfpathlineto{\pgfqpoint{0.971568in}{0.970058in}}%
\pgfpathlineto{\pgfqpoint{0.974602in}{0.957929in}}%
\pgfpathclose%
\pgfusepath{fill}%
\end{pgfscope}%
\begin{pgfscope}%
\pgfpathrectangle{\pgfqpoint{0.211875in}{0.211875in}}{\pgfqpoint{1.313625in}{1.279725in}}%
\pgfusepath{clip}%
\pgfsetbuttcap%
\pgfsetroundjoin%
\definecolor{currentfill}{rgb}{0.947270,0.405591,0.279023}%
\pgfsetfillcolor{currentfill}%
\pgfsetlinewidth{0.000000pt}%
\definecolor{currentstroke}{rgb}{0.000000,0.000000,0.000000}%
\pgfsetstrokecolor{currentstroke}%
\pgfsetdash{}{0pt}%
\pgfpathmoveto{\pgfqpoint{1.024748in}{1.096125in}}%
\pgfpathlineto{\pgfqpoint{1.036461in}{1.095857in}}%
\pgfpathlineto{\pgfqpoint{1.048174in}{1.102919in}}%
\pgfpathlineto{\pgfqpoint{1.048476in}{1.103476in}}%
\pgfpathlineto{\pgfqpoint{1.054766in}{1.115605in}}%
\pgfpathlineto{\pgfqpoint{1.059886in}{1.122823in}}%
\pgfpathlineto{\pgfqpoint{1.062479in}{1.127734in}}%
\pgfpathlineto{\pgfqpoint{1.064162in}{1.139863in}}%
\pgfpathlineto{\pgfqpoint{1.071599in}{1.145157in}}%
\pgfpathlineto{\pgfqpoint{1.083312in}{1.148085in}}%
\pgfpathlineto{\pgfqpoint{1.095024in}{1.151667in}}%
\pgfpathlineto{\pgfqpoint{1.095503in}{1.151991in}}%
\pgfpathlineto{\pgfqpoint{1.106737in}{1.159619in}}%
\pgfpathlineto{\pgfqpoint{1.112840in}{1.164120in}}%
\pgfpathlineto{\pgfqpoint{1.118450in}{1.168980in}}%
\pgfpathlineto{\pgfqpoint{1.123349in}{1.176249in}}%
\pgfpathlineto{\pgfqpoint{1.126836in}{1.188378in}}%
\pgfpathlineto{\pgfqpoint{1.128417in}{1.200507in}}%
\pgfpathlineto{\pgfqpoint{1.127297in}{1.212636in}}%
\pgfpathlineto{\pgfqpoint{1.124772in}{1.224765in}}%
\pgfpathlineto{\pgfqpoint{1.123058in}{1.236894in}}%
\pgfpathlineto{\pgfqpoint{1.119950in}{1.249022in}}%
\pgfpathlineto{\pgfqpoint{1.118450in}{1.251468in}}%
\pgfpathlineto{\pgfqpoint{1.106737in}{1.250394in}}%
\pgfpathlineto{\pgfqpoint{1.105033in}{1.249022in}}%
\pgfpathlineto{\pgfqpoint{1.095024in}{1.241631in}}%
\pgfpathlineto{\pgfqpoint{1.087962in}{1.236894in}}%
\pgfpathlineto{\pgfqpoint{1.083312in}{1.232852in}}%
\pgfpathlineto{\pgfqpoint{1.074568in}{1.224765in}}%
\pgfpathlineto{\pgfqpoint{1.071599in}{1.222034in}}%
\pgfpathlineto{\pgfqpoint{1.061039in}{1.212636in}}%
\pgfpathlineto{\pgfqpoint{1.059886in}{1.211161in}}%
\pgfpathlineto{\pgfqpoint{1.052534in}{1.200507in}}%
\pgfpathlineto{\pgfqpoint{1.048174in}{1.192851in}}%
\pgfpathlineto{\pgfqpoint{1.045894in}{1.188378in}}%
\pgfpathlineto{\pgfqpoint{1.041643in}{1.176249in}}%
\pgfpathlineto{\pgfqpoint{1.038151in}{1.164120in}}%
\pgfpathlineto{\pgfqpoint{1.037146in}{1.151991in}}%
\pgfpathlineto{\pgfqpoint{1.036461in}{1.150814in}}%
\pgfpathlineto{\pgfqpoint{1.031788in}{1.139863in}}%
\pgfpathlineto{\pgfqpoint{1.024748in}{1.128592in}}%
\pgfpathlineto{\pgfqpoint{1.024344in}{1.127734in}}%
\pgfpathlineto{\pgfqpoint{1.023164in}{1.115605in}}%
\pgfpathlineto{\pgfqpoint{1.023714in}{1.103476in}}%
\pgfpathclose%
\pgfusepath{fill}%
\end{pgfscope}%
\begin{pgfscope}%
\pgfpathrectangle{\pgfqpoint{0.211875in}{0.211875in}}{\pgfqpoint{1.313625in}{1.279725in}}%
\pgfusepath{clip}%
\pgfsetbuttcap%
\pgfsetroundjoin%
\definecolor{currentfill}{rgb}{0.947270,0.405591,0.279023}%
\pgfsetfillcolor{currentfill}%
\pgfsetlinewidth{0.000000pt}%
\definecolor{currentstroke}{rgb}{0.000000,0.000000,0.000000}%
\pgfsetstrokecolor{currentstroke}%
\pgfsetdash{}{0pt}%
\pgfpathmoveto{\pgfqpoint{1.376129in}{1.102731in}}%
\pgfpathlineto{\pgfqpoint{1.381677in}{1.103476in}}%
\pgfpathlineto{\pgfqpoint{1.387842in}{1.104217in}}%
\pgfpathlineto{\pgfqpoint{1.399554in}{1.104881in}}%
\pgfpathlineto{\pgfqpoint{1.411267in}{1.105030in}}%
\pgfpathlineto{\pgfqpoint{1.422980in}{1.108816in}}%
\pgfpathlineto{\pgfqpoint{1.434692in}{1.112170in}}%
\pgfpathlineto{\pgfqpoint{1.446405in}{1.113737in}}%
\pgfpathlineto{\pgfqpoint{1.446405in}{1.115605in}}%
\pgfpathlineto{\pgfqpoint{1.446405in}{1.125266in}}%
\pgfpathlineto{\pgfqpoint{1.434692in}{1.124274in}}%
\pgfpathlineto{\pgfqpoint{1.422980in}{1.123052in}}%
\pgfpathlineto{\pgfqpoint{1.411267in}{1.123257in}}%
\pgfpathlineto{\pgfqpoint{1.400483in}{1.127734in}}%
\pgfpathlineto{\pgfqpoint{1.399777in}{1.139863in}}%
\pgfpathlineto{\pgfqpoint{1.399554in}{1.143480in}}%
\pgfpathlineto{\pgfqpoint{1.398818in}{1.151991in}}%
\pgfpathlineto{\pgfqpoint{1.396241in}{1.164120in}}%
\pgfpathlineto{\pgfqpoint{1.399554in}{1.175446in}}%
\pgfpathlineto{\pgfqpoint{1.400163in}{1.176249in}}%
\pgfpathlineto{\pgfqpoint{1.411267in}{1.179671in}}%
\pgfpathlineto{\pgfqpoint{1.422980in}{1.178210in}}%
\pgfpathlineto{\pgfqpoint{1.434692in}{1.176290in}}%
\pgfpathlineto{\pgfqpoint{1.434864in}{1.176249in}}%
\pgfpathlineto{\pgfqpoint{1.446405in}{1.173045in}}%
\pgfpathlineto{\pgfqpoint{1.446405in}{1.176249in}}%
\pgfpathlineto{\pgfqpoint{1.446405in}{1.186862in}}%
\pgfpathlineto{\pgfqpoint{1.438857in}{1.188378in}}%
\pgfpathlineto{\pgfqpoint{1.434692in}{1.189363in}}%
\pgfpathlineto{\pgfqpoint{1.422980in}{1.194645in}}%
\pgfpathlineto{\pgfqpoint{1.413413in}{1.200507in}}%
\pgfpathlineto{\pgfqpoint{1.411267in}{1.202195in}}%
\pgfpathlineto{\pgfqpoint{1.402505in}{1.212636in}}%
\pgfpathlineto{\pgfqpoint{1.399554in}{1.216274in}}%
\pgfpathlineto{\pgfqpoint{1.391940in}{1.224765in}}%
\pgfpathlineto{\pgfqpoint{1.387842in}{1.229283in}}%
\pgfpathlineto{\pgfqpoint{1.380509in}{1.236894in}}%
\pgfpathlineto{\pgfqpoint{1.376129in}{1.241776in}}%
\pgfpathlineto{\pgfqpoint{1.369339in}{1.249022in}}%
\pgfpathlineto{\pgfqpoint{1.364416in}{1.254719in}}%
\pgfpathlineto{\pgfqpoint{1.357905in}{1.261151in}}%
\pgfpathlineto{\pgfqpoint{1.352704in}{1.267493in}}%
\pgfpathlineto{\pgfqpoint{1.346490in}{1.273280in}}%
\pgfpathlineto{\pgfqpoint{1.340991in}{1.279960in}}%
\pgfpathlineto{\pgfqpoint{1.333256in}{1.285409in}}%
\pgfpathlineto{\pgfqpoint{1.329278in}{1.289243in}}%
\pgfpathlineto{\pgfqpoint{1.321718in}{1.297538in}}%
\pgfpathlineto{\pgfqpoint{1.317566in}{1.303171in}}%
\pgfpathlineto{\pgfqpoint{1.314834in}{1.309667in}}%
\pgfpathlineto{\pgfqpoint{1.309157in}{1.321796in}}%
\pgfpathlineto{\pgfqpoint{1.305853in}{1.328420in}}%
\pgfpathlineto{\pgfqpoint{1.301704in}{1.333925in}}%
\pgfpathlineto{\pgfqpoint{1.296738in}{1.346053in}}%
\pgfpathlineto{\pgfqpoint{1.296493in}{1.358182in}}%
\pgfpathlineto{\pgfqpoint{1.297015in}{1.370311in}}%
\pgfpathlineto{\pgfqpoint{1.298186in}{1.382440in}}%
\pgfpathlineto{\pgfqpoint{1.299510in}{1.394569in}}%
\pgfpathlineto{\pgfqpoint{1.300533in}{1.406698in}}%
\pgfpathlineto{\pgfqpoint{1.294140in}{1.414961in}}%
\pgfpathlineto{\pgfqpoint{1.287723in}{1.418827in}}%
\pgfpathlineto{\pgfqpoint{1.282427in}{1.420558in}}%
\pgfpathlineto{\pgfqpoint{1.278144in}{1.418827in}}%
\pgfpathlineto{\pgfqpoint{1.270715in}{1.416699in}}%
\pgfpathlineto{\pgfqpoint{1.259002in}{1.412504in}}%
\pgfpathlineto{\pgfqpoint{1.251239in}{1.406698in}}%
\pgfpathlineto{\pgfqpoint{1.247783in}{1.394569in}}%
\pgfpathlineto{\pgfqpoint{1.247289in}{1.391986in}}%
\pgfpathlineto{\pgfqpoint{1.244456in}{1.382440in}}%
\pgfpathlineto{\pgfqpoint{1.241720in}{1.370311in}}%
\pgfpathlineto{\pgfqpoint{1.238430in}{1.358182in}}%
\pgfpathlineto{\pgfqpoint{1.235577in}{1.351375in}}%
\pgfpathlineto{\pgfqpoint{1.229661in}{1.346053in}}%
\pgfpathlineto{\pgfqpoint{1.223864in}{1.341622in}}%
\pgfpathlineto{\pgfqpoint{1.213917in}{1.346053in}}%
\pgfpathlineto{\pgfqpoint{1.212151in}{1.347470in}}%
\pgfpathlineto{\pgfqpoint{1.204690in}{1.358182in}}%
\pgfpathlineto{\pgfqpoint{1.200439in}{1.365310in}}%
\pgfpathlineto{\pgfqpoint{1.196447in}{1.370311in}}%
\pgfpathlineto{\pgfqpoint{1.188726in}{1.374493in}}%
\pgfpathlineto{\pgfqpoint{1.177013in}{1.375430in}}%
\pgfpathlineto{\pgfqpoint{1.165301in}{1.370548in}}%
\pgfpathlineto{\pgfqpoint{1.165087in}{1.370311in}}%
\pgfpathlineto{\pgfqpoint{1.159115in}{1.358182in}}%
\pgfpathlineto{\pgfqpoint{1.156815in}{1.346053in}}%
\pgfpathlineto{\pgfqpoint{1.155312in}{1.333925in}}%
\pgfpathlineto{\pgfqpoint{1.157617in}{1.321796in}}%
\pgfpathlineto{\pgfqpoint{1.165301in}{1.312512in}}%
\pgfpathlineto{\pgfqpoint{1.168760in}{1.309667in}}%
\pgfpathlineto{\pgfqpoint{1.177013in}{1.306394in}}%
\pgfpathlineto{\pgfqpoint{1.188726in}{1.303881in}}%
\pgfpathlineto{\pgfqpoint{1.196241in}{1.297538in}}%
\pgfpathlineto{\pgfqpoint{1.200439in}{1.293348in}}%
\pgfpathlineto{\pgfqpoint{1.203750in}{1.285409in}}%
\pgfpathlineto{\pgfqpoint{1.211693in}{1.273280in}}%
\pgfpathlineto{\pgfqpoint{1.212151in}{1.272730in}}%
\pgfpathlineto{\pgfqpoint{1.223864in}{1.267722in}}%
\pgfpathlineto{\pgfqpoint{1.235577in}{1.262065in}}%
\pgfpathlineto{\pgfqpoint{1.246495in}{1.273280in}}%
\pgfpathlineto{\pgfqpoint{1.247289in}{1.275655in}}%
\pgfpathlineto{\pgfqpoint{1.249849in}{1.285409in}}%
\pgfpathlineto{\pgfqpoint{1.251617in}{1.297538in}}%
\pgfpathlineto{\pgfqpoint{1.254074in}{1.309667in}}%
\pgfpathlineto{\pgfqpoint{1.259002in}{1.317524in}}%
\pgfpathlineto{\pgfqpoint{1.264995in}{1.321796in}}%
\pgfpathlineto{\pgfqpoint{1.270715in}{1.324840in}}%
\pgfpathlineto{\pgfqpoint{1.282427in}{1.325146in}}%
\pgfpathlineto{\pgfqpoint{1.285974in}{1.321796in}}%
\pgfpathlineto{\pgfqpoint{1.289915in}{1.309667in}}%
\pgfpathlineto{\pgfqpoint{1.291252in}{1.297538in}}%
\pgfpathlineto{\pgfqpoint{1.292303in}{1.285409in}}%
\pgfpathlineto{\pgfqpoint{1.294140in}{1.276201in}}%
\pgfpathlineto{\pgfqpoint{1.295027in}{1.273280in}}%
\pgfpathlineto{\pgfqpoint{1.297669in}{1.261151in}}%
\pgfpathlineto{\pgfqpoint{1.300290in}{1.249022in}}%
\pgfpathlineto{\pgfqpoint{1.304972in}{1.236894in}}%
\pgfpathlineto{\pgfqpoint{1.305853in}{1.234683in}}%
\pgfpathlineto{\pgfqpoint{1.308687in}{1.224765in}}%
\pgfpathlineto{\pgfqpoint{1.317566in}{1.213020in}}%
\pgfpathlineto{\pgfqpoint{1.317717in}{1.212636in}}%
\pgfpathlineto{\pgfqpoint{1.327712in}{1.200507in}}%
\pgfpathlineto{\pgfqpoint{1.329278in}{1.198504in}}%
\pgfpathlineto{\pgfqpoint{1.337082in}{1.188378in}}%
\pgfpathlineto{\pgfqpoint{1.340991in}{1.181754in}}%
\pgfpathlineto{\pgfqpoint{1.349275in}{1.176249in}}%
\pgfpathlineto{\pgfqpoint{1.352704in}{1.173339in}}%
\pgfpathlineto{\pgfqpoint{1.361341in}{1.164120in}}%
\pgfpathlineto{\pgfqpoint{1.364416in}{1.159378in}}%
\pgfpathlineto{\pgfqpoint{1.370251in}{1.151991in}}%
\pgfpathlineto{\pgfqpoint{1.371814in}{1.139863in}}%
\pgfpathlineto{\pgfqpoint{1.369141in}{1.127734in}}%
\pgfpathlineto{\pgfqpoint{1.368766in}{1.115605in}}%
\pgfpathlineto{\pgfqpoint{1.374845in}{1.103476in}}%
\pgfpathclose%
\pgfpathmoveto{\pgfqpoint{1.217876in}{1.297538in}}%
\pgfpathlineto{\pgfqpoint{1.212826in}{1.309667in}}%
\pgfpathlineto{\pgfqpoint{1.221709in}{1.321796in}}%
\pgfpathlineto{\pgfqpoint{1.223864in}{1.323034in}}%
\pgfpathlineto{\pgfqpoint{1.232339in}{1.321796in}}%
\pgfpathlineto{\pgfqpoint{1.235577in}{1.314171in}}%
\pgfpathlineto{\pgfqpoint{1.235841in}{1.309667in}}%
\pgfpathlineto{\pgfqpoint{1.235577in}{1.307432in}}%
\pgfpathlineto{\pgfqpoint{1.233200in}{1.297538in}}%
\pgfpathlineto{\pgfqpoint{1.223864in}{1.289588in}}%
\pgfpathclose%
\pgfpathmoveto{\pgfqpoint{1.248210in}{1.346053in}}%
\pgfpathlineto{\pgfqpoint{1.247319in}{1.358182in}}%
\pgfpathlineto{\pgfqpoint{1.249375in}{1.370311in}}%
\pgfpathlineto{\pgfqpoint{1.251618in}{1.382440in}}%
\pgfpathlineto{\pgfqpoint{1.254496in}{1.394569in}}%
\pgfpathlineto{\pgfqpoint{1.259002in}{1.402254in}}%
\pgfpathlineto{\pgfqpoint{1.266157in}{1.406698in}}%
\pgfpathlineto{\pgfqpoint{1.270715in}{1.408681in}}%
\pgfpathlineto{\pgfqpoint{1.282427in}{1.408001in}}%
\pgfpathlineto{\pgfqpoint{1.284119in}{1.406698in}}%
\pgfpathlineto{\pgfqpoint{1.290023in}{1.394569in}}%
\pgfpathlineto{\pgfqpoint{1.290515in}{1.382440in}}%
\pgfpathlineto{\pgfqpoint{1.290088in}{1.370311in}}%
\pgfpathlineto{\pgfqpoint{1.288657in}{1.358182in}}%
\pgfpathlineto{\pgfqpoint{1.282704in}{1.346053in}}%
\pgfpathlineto{\pgfqpoint{1.282427in}{1.345730in}}%
\pgfpathlineto{\pgfqpoint{1.270715in}{1.342544in}}%
\pgfpathlineto{\pgfqpoint{1.259002in}{1.339420in}}%
\pgfpathclose%
\pgfusepath{fill}%
\end{pgfscope}%
\begin{pgfscope}%
\pgfpathrectangle{\pgfqpoint{0.211875in}{0.211875in}}{\pgfqpoint{1.313625in}{1.279725in}}%
\pgfusepath{clip}%
\pgfsetbuttcap%
\pgfsetroundjoin%
\definecolor{currentfill}{rgb}{0.947270,0.405591,0.279023}%
\pgfsetfillcolor{currentfill}%
\pgfsetlinewidth{0.000000pt}%
\definecolor{currentstroke}{rgb}{0.000000,0.000000,0.000000}%
\pgfsetstrokecolor{currentstroke}%
\pgfsetdash{}{0pt}%
\pgfpathmoveto{\pgfqpoint{0.544528in}{1.247195in}}%
\pgfpathlineto{\pgfqpoint{0.556241in}{1.245105in}}%
\pgfpathlineto{\pgfqpoint{0.567953in}{1.244075in}}%
\pgfpathlineto{\pgfqpoint{0.579666in}{1.245296in}}%
\pgfpathlineto{\pgfqpoint{0.591379in}{1.248103in}}%
\pgfpathlineto{\pgfqpoint{0.596587in}{1.249022in}}%
\pgfpathlineto{\pgfqpoint{0.603091in}{1.250240in}}%
\pgfpathlineto{\pgfqpoint{0.614804in}{1.252061in}}%
\pgfpathlineto{\pgfqpoint{0.626517in}{1.254034in}}%
\pgfpathlineto{\pgfqpoint{0.638230in}{1.255335in}}%
\pgfpathlineto{\pgfqpoint{0.649942in}{1.255783in}}%
\pgfpathlineto{\pgfqpoint{0.661655in}{1.255903in}}%
\pgfpathlineto{\pgfqpoint{0.673368in}{1.256071in}}%
\pgfpathlineto{\pgfqpoint{0.685080in}{1.256315in}}%
\pgfpathlineto{\pgfqpoint{0.696793in}{1.256796in}}%
\pgfpathlineto{\pgfqpoint{0.708506in}{1.257465in}}%
\pgfpathlineto{\pgfqpoint{0.720218in}{1.258281in}}%
\pgfpathlineto{\pgfqpoint{0.731931in}{1.259199in}}%
\pgfpathlineto{\pgfqpoint{0.743644in}{1.260323in}}%
\pgfpathlineto{\pgfqpoint{0.755356in}{1.260708in}}%
\pgfpathlineto{\pgfqpoint{0.767069in}{1.260701in}}%
\pgfpathlineto{\pgfqpoint{0.778782in}{1.260742in}}%
\pgfpathlineto{\pgfqpoint{0.790495in}{1.260643in}}%
\pgfpathlineto{\pgfqpoint{0.802207in}{1.259468in}}%
\pgfpathlineto{\pgfqpoint{0.813920in}{1.260517in}}%
\pgfpathlineto{\pgfqpoint{0.815397in}{1.261151in}}%
\pgfpathlineto{\pgfqpoint{0.825633in}{1.269354in}}%
\pgfpathlineto{\pgfqpoint{0.828219in}{1.273280in}}%
\pgfpathlineto{\pgfqpoint{0.828845in}{1.285409in}}%
\pgfpathlineto{\pgfqpoint{0.828778in}{1.297538in}}%
\pgfpathlineto{\pgfqpoint{0.833495in}{1.309667in}}%
\pgfpathlineto{\pgfqpoint{0.837345in}{1.317548in}}%
\pgfpathlineto{\pgfqpoint{0.839305in}{1.321796in}}%
\pgfpathlineto{\pgfqpoint{0.846043in}{1.333925in}}%
\pgfpathlineto{\pgfqpoint{0.849058in}{1.338059in}}%
\pgfpathlineto{\pgfqpoint{0.856577in}{1.346053in}}%
\pgfpathlineto{\pgfqpoint{0.860771in}{1.350996in}}%
\pgfpathlineto{\pgfqpoint{0.866702in}{1.358182in}}%
\pgfpathlineto{\pgfqpoint{0.871315in}{1.370311in}}%
\pgfpathlineto{\pgfqpoint{0.872483in}{1.374384in}}%
\pgfpathlineto{\pgfqpoint{0.874446in}{1.382440in}}%
\pgfpathlineto{\pgfqpoint{0.877798in}{1.394569in}}%
\pgfpathlineto{\pgfqpoint{0.882490in}{1.406698in}}%
\pgfpathlineto{\pgfqpoint{0.884196in}{1.411139in}}%
\pgfpathlineto{\pgfqpoint{0.886959in}{1.418827in}}%
\pgfpathlineto{\pgfqpoint{0.892129in}{1.430956in}}%
\pgfpathlineto{\pgfqpoint{0.895909in}{1.440436in}}%
\pgfpathlineto{\pgfqpoint{0.896849in}{1.443084in}}%
\pgfpathlineto{\pgfqpoint{0.901653in}{1.455213in}}%
\pgfpathlineto{\pgfqpoint{0.906616in}{1.467342in}}%
\pgfpathlineto{\pgfqpoint{0.907621in}{1.473592in}}%
\pgfpathlineto{\pgfqpoint{0.908307in}{1.479471in}}%
\pgfpathlineto{\pgfqpoint{0.907621in}{1.483461in}}%
\pgfpathlineto{\pgfqpoint{0.906148in}{1.491600in}}%
\pgfpathlineto{\pgfqpoint{0.895909in}{1.491600in}}%
\pgfpathlineto{\pgfqpoint{0.884196in}{1.491600in}}%
\pgfpathlineto{\pgfqpoint{0.872483in}{1.491600in}}%
\pgfpathlineto{\pgfqpoint{0.860771in}{1.491600in}}%
\pgfpathlineto{\pgfqpoint{0.849058in}{1.491600in}}%
\pgfpathlineto{\pgfqpoint{0.837345in}{1.491600in}}%
\pgfpathlineto{\pgfqpoint{0.825633in}{1.491600in}}%
\pgfpathlineto{\pgfqpoint{0.813920in}{1.491600in}}%
\pgfpathlineto{\pgfqpoint{0.802207in}{1.491600in}}%
\pgfpathlineto{\pgfqpoint{0.790495in}{1.491600in}}%
\pgfpathlineto{\pgfqpoint{0.778782in}{1.491600in}}%
\pgfpathlineto{\pgfqpoint{0.767069in}{1.491600in}}%
\pgfpathlineto{\pgfqpoint{0.755356in}{1.491600in}}%
\pgfpathlineto{\pgfqpoint{0.743644in}{1.491600in}}%
\pgfpathlineto{\pgfqpoint{0.731931in}{1.491600in}}%
\pgfpathlineto{\pgfqpoint{0.720218in}{1.491600in}}%
\pgfpathlineto{\pgfqpoint{0.708506in}{1.491600in}}%
\pgfpathlineto{\pgfqpoint{0.696793in}{1.491600in}}%
\pgfpathlineto{\pgfqpoint{0.685080in}{1.491600in}}%
\pgfpathlineto{\pgfqpoint{0.673368in}{1.491600in}}%
\pgfpathlineto{\pgfqpoint{0.661655in}{1.491600in}}%
\pgfpathlineto{\pgfqpoint{0.649942in}{1.491600in}}%
\pgfpathlineto{\pgfqpoint{0.638230in}{1.491600in}}%
\pgfpathlineto{\pgfqpoint{0.626517in}{1.491600in}}%
\pgfpathlineto{\pgfqpoint{0.614804in}{1.491600in}}%
\pgfpathlineto{\pgfqpoint{0.603091in}{1.491600in}}%
\pgfpathlineto{\pgfqpoint{0.591379in}{1.491600in}}%
\pgfpathlineto{\pgfqpoint{0.583557in}{1.491600in}}%
\pgfpathlineto{\pgfqpoint{0.591379in}{1.480297in}}%
\pgfpathlineto{\pgfqpoint{0.591937in}{1.479471in}}%
\pgfpathlineto{\pgfqpoint{0.599492in}{1.467342in}}%
\pgfpathlineto{\pgfqpoint{0.603091in}{1.460762in}}%
\pgfpathlineto{\pgfqpoint{0.606149in}{1.455213in}}%
\pgfpathlineto{\pgfqpoint{0.613088in}{1.443084in}}%
\pgfpathlineto{\pgfqpoint{0.614804in}{1.440639in}}%
\pgfpathlineto{\pgfqpoint{0.621477in}{1.430956in}}%
\pgfpathlineto{\pgfqpoint{0.626517in}{1.423415in}}%
\pgfpathlineto{\pgfqpoint{0.629656in}{1.418827in}}%
\pgfpathlineto{\pgfqpoint{0.637616in}{1.406698in}}%
\pgfpathlineto{\pgfqpoint{0.638230in}{1.405662in}}%
\pgfpathlineto{\pgfqpoint{0.645044in}{1.394569in}}%
\pgfpathlineto{\pgfqpoint{0.649942in}{1.385827in}}%
\pgfpathlineto{\pgfqpoint{0.651936in}{1.382440in}}%
\pgfpathlineto{\pgfqpoint{0.658017in}{1.370311in}}%
\pgfpathlineto{\pgfqpoint{0.661655in}{1.360810in}}%
\pgfpathlineto{\pgfqpoint{0.662716in}{1.358182in}}%
\pgfpathlineto{\pgfqpoint{0.661655in}{1.355115in}}%
\pgfpathlineto{\pgfqpoint{0.655920in}{1.346053in}}%
\pgfpathlineto{\pgfqpoint{0.649942in}{1.340448in}}%
\pgfpathlineto{\pgfqpoint{0.642631in}{1.333925in}}%
\pgfpathlineto{\pgfqpoint{0.638230in}{1.329738in}}%
\pgfpathlineto{\pgfqpoint{0.629972in}{1.321796in}}%
\pgfpathlineto{\pgfqpoint{0.626517in}{1.319000in}}%
\pgfpathlineto{\pgfqpoint{0.614804in}{1.309729in}}%
\pgfpathlineto{\pgfqpoint{0.614718in}{1.309667in}}%
\pgfpathlineto{\pgfqpoint{0.603091in}{1.302694in}}%
\pgfpathlineto{\pgfqpoint{0.592934in}{1.297538in}}%
\pgfpathlineto{\pgfqpoint{0.591379in}{1.296894in}}%
\pgfpathlineto{\pgfqpoint{0.579666in}{1.292387in}}%
\pgfpathlineto{\pgfqpoint{0.567953in}{1.288564in}}%
\pgfpathlineto{\pgfqpoint{0.560469in}{1.285409in}}%
\pgfpathlineto{\pgfqpoint{0.556241in}{1.283455in}}%
\pgfpathlineto{\pgfqpoint{0.544528in}{1.279071in}}%
\pgfpathlineto{\pgfqpoint{0.532815in}{1.273876in}}%
\pgfpathlineto{\pgfqpoint{0.531580in}{1.273280in}}%
\pgfpathlineto{\pgfqpoint{0.521103in}{1.264258in}}%
\pgfpathlineto{\pgfqpoint{0.514632in}{1.261151in}}%
\pgfpathlineto{\pgfqpoint{0.521103in}{1.256039in}}%
\pgfpathlineto{\pgfqpoint{0.532815in}{1.250188in}}%
\pgfpathlineto{\pgfqpoint{0.536100in}{1.249022in}}%
\pgfpathclose%
\pgfusepath{fill}%
\end{pgfscope}%
\begin{pgfscope}%
\pgfpathrectangle{\pgfqpoint{0.211875in}{0.211875in}}{\pgfqpoint{1.313625in}{1.279725in}}%
\pgfusepath{clip}%
\pgfsetbuttcap%
\pgfsetroundjoin%
\definecolor{currentfill}{rgb}{0.961115,0.566634,0.405693}%
\pgfsetfillcolor{currentfill}%
\pgfsetlinewidth{0.000000pt}%
\definecolor{currentstroke}{rgb}{0.000000,0.000000,0.000000}%
\pgfsetstrokecolor{currentstroke}%
\pgfsetdash{}{0pt}%
\pgfpathmoveto{\pgfqpoint{0.755356in}{0.576834in}}%
\pgfpathlineto{\pgfqpoint{0.767069in}{0.577579in}}%
\pgfpathlineto{\pgfqpoint{0.771025in}{0.581934in}}%
\pgfpathlineto{\pgfqpoint{0.774530in}{0.594063in}}%
\pgfpathlineto{\pgfqpoint{0.767291in}{0.606192in}}%
\pgfpathlineto{\pgfqpoint{0.767069in}{0.606473in}}%
\pgfpathlineto{\pgfqpoint{0.759046in}{0.618321in}}%
\pgfpathlineto{\pgfqpoint{0.755356in}{0.624771in}}%
\pgfpathlineto{\pgfqpoint{0.752433in}{0.630450in}}%
\pgfpathlineto{\pgfqpoint{0.745838in}{0.642579in}}%
\pgfpathlineto{\pgfqpoint{0.743644in}{0.645619in}}%
\pgfpathlineto{\pgfqpoint{0.735462in}{0.654708in}}%
\pgfpathlineto{\pgfqpoint{0.731931in}{0.658058in}}%
\pgfpathlineto{\pgfqpoint{0.721257in}{0.666836in}}%
\pgfpathlineto{\pgfqpoint{0.720218in}{0.667639in}}%
\pgfpathlineto{\pgfqpoint{0.719117in}{0.666836in}}%
\pgfpathlineto{\pgfqpoint{0.717735in}{0.654708in}}%
\pgfpathlineto{\pgfqpoint{0.720218in}{0.642623in}}%
\pgfpathlineto{\pgfqpoint{0.720236in}{0.642579in}}%
\pgfpathlineto{\pgfqpoint{0.722671in}{0.630450in}}%
\pgfpathlineto{\pgfqpoint{0.727218in}{0.618321in}}%
\pgfpathlineto{\pgfqpoint{0.731931in}{0.607942in}}%
\pgfpathlineto{\pgfqpoint{0.732949in}{0.606192in}}%
\pgfpathlineto{\pgfqpoint{0.740083in}{0.594063in}}%
\pgfpathlineto{\pgfqpoint{0.743644in}{0.587401in}}%
\pgfpathlineto{\pgfqpoint{0.747141in}{0.581934in}}%
\pgfpathclose%
\pgfusepath{fill}%
\end{pgfscope}%
\begin{pgfscope}%
\pgfpathrectangle{\pgfqpoint{0.211875in}{0.211875in}}{\pgfqpoint{1.313625in}{1.279725in}}%
\pgfusepath{clip}%
\pgfsetbuttcap%
\pgfsetroundjoin%
\definecolor{currentfill}{rgb}{0.961115,0.566634,0.405693}%
\pgfsetfillcolor{currentfill}%
\pgfsetlinewidth{0.000000pt}%
\definecolor{currentstroke}{rgb}{0.000000,0.000000,0.000000}%
\pgfsetstrokecolor{currentstroke}%
\pgfsetdash{}{0pt}%
\pgfpathmoveto{\pgfqpoint{1.212151in}{0.666657in}}%
\pgfpathlineto{\pgfqpoint{1.223864in}{0.666623in}}%
\pgfpathlineto{\pgfqpoint{1.225374in}{0.666836in}}%
\pgfpathlineto{\pgfqpoint{1.235577in}{0.669171in}}%
\pgfpathlineto{\pgfqpoint{1.247289in}{0.678341in}}%
\pgfpathlineto{\pgfqpoint{1.248274in}{0.678965in}}%
\pgfpathlineto{\pgfqpoint{1.247289in}{0.680271in}}%
\pgfpathlineto{\pgfqpoint{1.242276in}{0.691094in}}%
\pgfpathlineto{\pgfqpoint{1.235577in}{0.698437in}}%
\pgfpathlineto{\pgfqpoint{1.231231in}{0.703223in}}%
\pgfpathlineto{\pgfqpoint{1.223864in}{0.708846in}}%
\pgfpathlineto{\pgfqpoint{1.212151in}{0.713433in}}%
\pgfpathlineto{\pgfqpoint{1.204638in}{0.715352in}}%
\pgfpathlineto{\pgfqpoint{1.200439in}{0.716460in}}%
\pgfpathlineto{\pgfqpoint{1.198135in}{0.715352in}}%
\pgfpathlineto{\pgfqpoint{1.193141in}{0.703223in}}%
\pgfpathlineto{\pgfqpoint{1.189767in}{0.691094in}}%
\pgfpathlineto{\pgfqpoint{1.194820in}{0.678965in}}%
\pgfpathlineto{\pgfqpoint{1.200439in}{0.671124in}}%
\pgfpathlineto{\pgfqpoint{1.210918in}{0.666836in}}%
\pgfpathclose%
\pgfpathmoveto{\pgfqpoint{1.213835in}{0.678965in}}%
\pgfpathlineto{\pgfqpoint{1.212151in}{0.679852in}}%
\pgfpathlineto{\pgfqpoint{1.207247in}{0.691094in}}%
\pgfpathlineto{\pgfqpoint{1.212151in}{0.699529in}}%
\pgfpathlineto{\pgfqpoint{1.223864in}{0.693150in}}%
\pgfpathlineto{\pgfqpoint{1.225730in}{0.691094in}}%
\pgfpathlineto{\pgfqpoint{1.227337in}{0.678965in}}%
\pgfpathlineto{\pgfqpoint{1.223864in}{0.677386in}}%
\pgfpathclose%
\pgfusepath{fill}%
\end{pgfscope}%
\begin{pgfscope}%
\pgfpathrectangle{\pgfqpoint{0.211875in}{0.211875in}}{\pgfqpoint{1.313625in}{1.279725in}}%
\pgfusepath{clip}%
\pgfsetbuttcap%
\pgfsetroundjoin%
\definecolor{currentfill}{rgb}{0.961115,0.566634,0.405693}%
\pgfsetfillcolor{currentfill}%
\pgfsetlinewidth{0.000000pt}%
\definecolor{currentstroke}{rgb}{0.000000,0.000000,0.000000}%
\pgfsetstrokecolor{currentstroke}%
\pgfsetdash{}{0pt}%
\pgfpathmoveto{\pgfqpoint{1.305853in}{0.674964in}}%
\pgfpathlineto{\pgfqpoint{1.317566in}{0.675855in}}%
\pgfpathlineto{\pgfqpoint{1.322475in}{0.678965in}}%
\pgfpathlineto{\pgfqpoint{1.329278in}{0.683073in}}%
\pgfpathlineto{\pgfqpoint{1.340923in}{0.691094in}}%
\pgfpathlineto{\pgfqpoint{1.340991in}{0.691587in}}%
\pgfpathlineto{\pgfqpoint{1.342577in}{0.703223in}}%
\pgfpathlineto{\pgfqpoint{1.340991in}{0.708663in}}%
\pgfpathlineto{\pgfqpoint{1.329278in}{0.712172in}}%
\pgfpathlineto{\pgfqpoint{1.317566in}{0.709616in}}%
\pgfpathlineto{\pgfqpoint{1.308634in}{0.703223in}}%
\pgfpathlineto{\pgfqpoint{1.305853in}{0.700859in}}%
\pgfpathlineto{\pgfqpoint{1.294795in}{0.691094in}}%
\pgfpathlineto{\pgfqpoint{1.295939in}{0.678965in}}%
\pgfpathclose%
\pgfpathmoveto{\pgfqpoint{1.310579in}{0.691094in}}%
\pgfpathlineto{\pgfqpoint{1.317566in}{0.697091in}}%
\pgfpathlineto{\pgfqpoint{1.329278in}{0.699610in}}%
\pgfpathlineto{\pgfqpoint{1.332539in}{0.691094in}}%
\pgfpathlineto{\pgfqpoint{1.329278in}{0.688848in}}%
\pgfpathlineto{\pgfqpoint{1.317566in}{0.685281in}}%
\pgfpathclose%
\pgfusepath{fill}%
\end{pgfscope}%
\begin{pgfscope}%
\pgfpathrectangle{\pgfqpoint{0.211875in}{0.211875in}}{\pgfqpoint{1.313625in}{1.279725in}}%
\pgfusepath{clip}%
\pgfsetbuttcap%
\pgfsetroundjoin%
\definecolor{currentfill}{rgb}{0.961115,0.566634,0.405693}%
\pgfsetfillcolor{currentfill}%
\pgfsetlinewidth{0.000000pt}%
\definecolor{currentstroke}{rgb}{0.000000,0.000000,0.000000}%
\pgfsetstrokecolor{currentstroke}%
\pgfsetdash{}{0pt}%
\pgfpathmoveto{\pgfqpoint{1.422980in}{0.678764in}}%
\pgfpathlineto{\pgfqpoint{1.425087in}{0.678965in}}%
\pgfpathlineto{\pgfqpoint{1.434692in}{0.680410in}}%
\pgfpathlineto{\pgfqpoint{1.446405in}{0.685971in}}%
\pgfpathlineto{\pgfqpoint{1.446405in}{0.691094in}}%
\pgfpathlineto{\pgfqpoint{1.446405in}{0.703223in}}%
\pgfpathlineto{\pgfqpoint{1.446405in}{0.711897in}}%
\pgfpathlineto{\pgfqpoint{1.440248in}{0.715352in}}%
\pgfpathlineto{\pgfqpoint{1.434692in}{0.717338in}}%
\pgfpathlineto{\pgfqpoint{1.422980in}{0.720698in}}%
\pgfpathlineto{\pgfqpoint{1.411418in}{0.715352in}}%
\pgfpathlineto{\pgfqpoint{1.411267in}{0.715261in}}%
\pgfpathlineto{\pgfqpoint{1.401686in}{0.703223in}}%
\pgfpathlineto{\pgfqpoint{1.403804in}{0.691094in}}%
\pgfpathlineto{\pgfqpoint{1.411267in}{0.685494in}}%
\pgfpathlineto{\pgfqpoint{1.422570in}{0.678965in}}%
\pgfpathclose%
\pgfpathmoveto{\pgfqpoint{1.419085in}{0.691094in}}%
\pgfpathlineto{\pgfqpoint{1.419083in}{0.703223in}}%
\pgfpathlineto{\pgfqpoint{1.422980in}{0.705767in}}%
\pgfpathlineto{\pgfqpoint{1.429492in}{0.703223in}}%
\pgfpathlineto{\pgfqpoint{1.429596in}{0.691094in}}%
\pgfpathlineto{\pgfqpoint{1.422980in}{0.689002in}}%
\pgfpathclose%
\pgfusepath{fill}%
\end{pgfscope}%
\begin{pgfscope}%
\pgfpathrectangle{\pgfqpoint{0.211875in}{0.211875in}}{\pgfqpoint{1.313625in}{1.279725in}}%
\pgfusepath{clip}%
\pgfsetbuttcap%
\pgfsetroundjoin%
\definecolor{currentfill}{rgb}{0.961115,0.566634,0.405693}%
\pgfsetfillcolor{currentfill}%
\pgfsetlinewidth{0.000000pt}%
\definecolor{currentstroke}{rgb}{0.000000,0.000000,0.000000}%
\pgfsetstrokecolor{currentstroke}%
\pgfsetdash{}{0pt}%
\pgfpathmoveto{\pgfqpoint{1.364416in}{0.740225in}}%
\pgfpathlineto{\pgfqpoint{1.376129in}{0.745341in}}%
\pgfpathlineto{\pgfqpoint{1.381236in}{0.751739in}}%
\pgfpathlineto{\pgfqpoint{1.383178in}{0.763867in}}%
\pgfpathlineto{\pgfqpoint{1.385443in}{0.775996in}}%
\pgfpathlineto{\pgfqpoint{1.376129in}{0.780545in}}%
\pgfpathlineto{\pgfqpoint{1.374346in}{0.775996in}}%
\pgfpathlineto{\pgfqpoint{1.364416in}{0.766753in}}%
\pgfpathlineto{\pgfqpoint{1.362732in}{0.763867in}}%
\pgfpathlineto{\pgfqpoint{1.360092in}{0.751739in}}%
\pgfpathclose%
\pgfusepath{fill}%
\end{pgfscope}%
\begin{pgfscope}%
\pgfpathrectangle{\pgfqpoint{0.211875in}{0.211875in}}{\pgfqpoint{1.313625in}{1.279725in}}%
\pgfusepath{clip}%
\pgfsetbuttcap%
\pgfsetroundjoin%
\definecolor{currentfill}{rgb}{0.961115,0.566634,0.405693}%
\pgfsetfillcolor{currentfill}%
\pgfsetlinewidth{0.000000pt}%
\definecolor{currentstroke}{rgb}{0.000000,0.000000,0.000000}%
\pgfsetstrokecolor{currentstroke}%
\pgfsetdash{}{0pt}%
\pgfpathmoveto{\pgfqpoint{0.649942in}{0.774851in}}%
\pgfpathlineto{\pgfqpoint{0.661655in}{0.774753in}}%
\pgfpathlineto{\pgfqpoint{0.668541in}{0.775996in}}%
\pgfpathlineto{\pgfqpoint{0.661655in}{0.778804in}}%
\pgfpathlineto{\pgfqpoint{0.649942in}{0.780709in}}%
\pgfpathlineto{\pgfqpoint{0.646274in}{0.775996in}}%
\pgfpathclose%
\pgfusepath{fill}%
\end{pgfscope}%
\begin{pgfscope}%
\pgfpathrectangle{\pgfqpoint{0.211875in}{0.211875in}}{\pgfqpoint{1.313625in}{1.279725in}}%
\pgfusepath{clip}%
\pgfsetbuttcap%
\pgfsetroundjoin%
\definecolor{currentfill}{rgb}{0.961115,0.566634,0.405693}%
\pgfsetfillcolor{currentfill}%
\pgfsetlinewidth{0.000000pt}%
\definecolor{currentstroke}{rgb}{0.000000,0.000000,0.000000}%
\pgfsetstrokecolor{currentstroke}%
\pgfsetdash{}{0pt}%
\pgfpathmoveto{\pgfqpoint{1.118450in}{0.809930in}}%
\pgfpathlineto{\pgfqpoint{1.130163in}{0.808893in}}%
\pgfpathlineto{\pgfqpoint{1.133980in}{0.812383in}}%
\pgfpathlineto{\pgfqpoint{1.139664in}{0.824512in}}%
\pgfpathlineto{\pgfqpoint{1.141557in}{0.836641in}}%
\pgfpathlineto{\pgfqpoint{1.141246in}{0.848770in}}%
\pgfpathlineto{\pgfqpoint{1.137937in}{0.860898in}}%
\pgfpathlineto{\pgfqpoint{1.130639in}{0.873027in}}%
\pgfpathlineto{\pgfqpoint{1.130163in}{0.873410in}}%
\pgfpathlineto{\pgfqpoint{1.118450in}{0.879620in}}%
\pgfpathlineto{\pgfqpoint{1.106737in}{0.873320in}}%
\pgfpathlineto{\pgfqpoint{1.106400in}{0.873027in}}%
\pgfpathlineto{\pgfqpoint{1.096018in}{0.860898in}}%
\pgfpathlineto{\pgfqpoint{1.095024in}{0.856883in}}%
\pgfpathlineto{\pgfqpoint{1.093521in}{0.848770in}}%
\pgfpathlineto{\pgfqpoint{1.093597in}{0.836641in}}%
\pgfpathlineto{\pgfqpoint{1.095024in}{0.831733in}}%
\pgfpathlineto{\pgfqpoint{1.097977in}{0.824512in}}%
\pgfpathlineto{\pgfqpoint{1.106737in}{0.815633in}}%
\pgfpathlineto{\pgfqpoint{1.113224in}{0.812383in}}%
\pgfpathclose%
\pgfpathmoveto{\pgfqpoint{1.111498in}{0.824512in}}%
\pgfpathlineto{\pgfqpoint{1.106737in}{0.826951in}}%
\pgfpathlineto{\pgfqpoint{1.101218in}{0.836641in}}%
\pgfpathlineto{\pgfqpoint{1.100467in}{0.848770in}}%
\pgfpathlineto{\pgfqpoint{1.105578in}{0.860898in}}%
\pgfpathlineto{\pgfqpoint{1.106737in}{0.862239in}}%
\pgfpathlineto{\pgfqpoint{1.118450in}{0.863039in}}%
\pgfpathlineto{\pgfqpoint{1.120804in}{0.860898in}}%
\pgfpathlineto{\pgfqpoint{1.127854in}{0.848770in}}%
\pgfpathlineto{\pgfqpoint{1.128396in}{0.836641in}}%
\pgfpathlineto{\pgfqpoint{1.124200in}{0.824512in}}%
\pgfpathlineto{\pgfqpoint{1.118450in}{0.821362in}}%
\pgfpathclose%
\pgfusepath{fill}%
\end{pgfscope}%
\begin{pgfscope}%
\pgfpathrectangle{\pgfqpoint{0.211875in}{0.211875in}}{\pgfqpoint{1.313625in}{1.279725in}}%
\pgfusepath{clip}%
\pgfsetbuttcap%
\pgfsetroundjoin%
\definecolor{currentfill}{rgb}{0.961115,0.566634,0.405693}%
\pgfsetfillcolor{currentfill}%
\pgfsetlinewidth{0.000000pt}%
\definecolor{currentstroke}{rgb}{0.000000,0.000000,0.000000}%
\pgfsetstrokecolor{currentstroke}%
\pgfsetdash{}{0pt}%
\pgfpathmoveto{\pgfqpoint{1.411267in}{0.810774in}}%
\pgfpathlineto{\pgfqpoint{1.415811in}{0.812383in}}%
\pgfpathlineto{\pgfqpoint{1.422980in}{0.813484in}}%
\pgfpathlineto{\pgfqpoint{1.434692in}{0.820504in}}%
\pgfpathlineto{\pgfqpoint{1.439050in}{0.824512in}}%
\pgfpathlineto{\pgfqpoint{1.446405in}{0.832833in}}%
\pgfpathlineto{\pgfqpoint{1.446405in}{0.836641in}}%
\pgfpathlineto{\pgfqpoint{1.446405in}{0.848770in}}%
\pgfpathlineto{\pgfqpoint{1.446405in}{0.860898in}}%
\pgfpathlineto{\pgfqpoint{1.446405in}{0.862641in}}%
\pgfpathlineto{\pgfqpoint{1.440798in}{0.860898in}}%
\pgfpathlineto{\pgfqpoint{1.434692in}{0.859229in}}%
\pgfpathlineto{\pgfqpoint{1.422980in}{0.854525in}}%
\pgfpathlineto{\pgfqpoint{1.415546in}{0.848770in}}%
\pgfpathlineto{\pgfqpoint{1.411267in}{0.844204in}}%
\pgfpathlineto{\pgfqpoint{1.404422in}{0.836641in}}%
\pgfpathlineto{\pgfqpoint{1.399554in}{0.826855in}}%
\pgfpathlineto{\pgfqpoint{1.398423in}{0.824512in}}%
\pgfpathlineto{\pgfqpoint{1.399554in}{0.815861in}}%
\pgfpathlineto{\pgfqpoint{1.403747in}{0.812383in}}%
\pgfpathclose%
\pgfpathmoveto{\pgfqpoint{1.419943in}{0.824512in}}%
\pgfpathlineto{\pgfqpoint{1.416879in}{0.836641in}}%
\pgfpathlineto{\pgfqpoint{1.422980in}{0.843159in}}%
\pgfpathlineto{\pgfqpoint{1.431612in}{0.848770in}}%
\pgfpathlineto{\pgfqpoint{1.434692in}{0.850081in}}%
\pgfpathlineto{\pgfqpoint{1.440558in}{0.848770in}}%
\pgfpathlineto{\pgfqpoint{1.439217in}{0.836641in}}%
\pgfpathlineto{\pgfqpoint{1.434692in}{0.831516in}}%
\pgfpathlineto{\pgfqpoint{1.424788in}{0.824512in}}%
\pgfpathlineto{\pgfqpoint{1.422980in}{0.823450in}}%
\pgfpathclose%
\pgfusepath{fill}%
\end{pgfscope}%
\begin{pgfscope}%
\pgfpathrectangle{\pgfqpoint{0.211875in}{0.211875in}}{\pgfqpoint{1.313625in}{1.279725in}}%
\pgfusepath{clip}%
\pgfsetbuttcap%
\pgfsetroundjoin%
\definecolor{currentfill}{rgb}{0.961115,0.566634,0.405693}%
\pgfsetfillcolor{currentfill}%
\pgfsetlinewidth{0.000000pt}%
\definecolor{currentstroke}{rgb}{0.000000,0.000000,0.000000}%
\pgfsetstrokecolor{currentstroke}%
\pgfsetdash{}{0pt}%
\pgfpathmoveto{\pgfqpoint{0.497677in}{0.847326in}}%
\pgfpathlineto{\pgfqpoint{0.498494in}{0.848770in}}%
\pgfpathlineto{\pgfqpoint{0.497677in}{0.849344in}}%
\pgfpathlineto{\pgfqpoint{0.496838in}{0.848770in}}%
\pgfpathclose%
\pgfusepath{fill}%
\end{pgfscope}%
\begin{pgfscope}%
\pgfpathrectangle{\pgfqpoint{0.211875in}{0.211875in}}{\pgfqpoint{1.313625in}{1.279725in}}%
\pgfusepath{clip}%
\pgfsetbuttcap%
\pgfsetroundjoin%
\definecolor{currentfill}{rgb}{0.961115,0.566634,0.405693}%
\pgfsetfillcolor{currentfill}%
\pgfsetlinewidth{0.000000pt}%
\definecolor{currentstroke}{rgb}{0.000000,0.000000,0.000000}%
\pgfsetstrokecolor{currentstroke}%
\pgfsetdash{}{0pt}%
\pgfpathmoveto{\pgfqpoint{0.357125in}{1.054092in}}%
\pgfpathlineto{\pgfqpoint{0.361464in}{1.054960in}}%
\pgfpathlineto{\pgfqpoint{0.357125in}{1.061149in}}%
\pgfpathlineto{\pgfqpoint{0.356410in}{1.054960in}}%
\pgfpathclose%
\pgfusepath{fill}%
\end{pgfscope}%
\begin{pgfscope}%
\pgfpathrectangle{\pgfqpoint{0.211875in}{0.211875in}}{\pgfqpoint{1.313625in}{1.279725in}}%
\pgfusepath{clip}%
\pgfsetbuttcap%
\pgfsetroundjoin%
\definecolor{currentfill}{rgb}{0.961115,0.566634,0.405693}%
\pgfsetfillcolor{currentfill}%
\pgfsetlinewidth{0.000000pt}%
\definecolor{currentstroke}{rgb}{0.000000,0.000000,0.000000}%
\pgfsetstrokecolor{currentstroke}%
\pgfsetdash{}{0pt}%
\pgfpathmoveto{\pgfqpoint{1.411267in}{1.123257in}}%
\pgfpathlineto{\pgfqpoint{1.422980in}{1.123052in}}%
\pgfpathlineto{\pgfqpoint{1.434692in}{1.124274in}}%
\pgfpathlineto{\pgfqpoint{1.446405in}{1.125266in}}%
\pgfpathlineto{\pgfqpoint{1.446405in}{1.127734in}}%
\pgfpathlineto{\pgfqpoint{1.446405in}{1.139863in}}%
\pgfpathlineto{\pgfqpoint{1.446405in}{1.151991in}}%
\pgfpathlineto{\pgfqpoint{1.446405in}{1.164120in}}%
\pgfpathlineto{\pgfqpoint{1.446405in}{1.173045in}}%
\pgfpathlineto{\pgfqpoint{1.434864in}{1.176249in}}%
\pgfpathlineto{\pgfqpoint{1.434692in}{1.176290in}}%
\pgfpathlineto{\pgfqpoint{1.422980in}{1.178210in}}%
\pgfpathlineto{\pgfqpoint{1.411267in}{1.179671in}}%
\pgfpathlineto{\pgfqpoint{1.400163in}{1.176249in}}%
\pgfpathlineto{\pgfqpoint{1.399554in}{1.175446in}}%
\pgfpathlineto{\pgfqpoint{1.396241in}{1.164120in}}%
\pgfpathlineto{\pgfqpoint{1.398818in}{1.151991in}}%
\pgfpathlineto{\pgfqpoint{1.399554in}{1.143480in}}%
\pgfpathlineto{\pgfqpoint{1.399777in}{1.139863in}}%
\pgfpathlineto{\pgfqpoint{1.400483in}{1.127734in}}%
\pgfpathclose%
\pgfpathmoveto{\pgfqpoint{1.415995in}{1.139863in}}%
\pgfpathlineto{\pgfqpoint{1.412827in}{1.151991in}}%
\pgfpathlineto{\pgfqpoint{1.421511in}{1.164120in}}%
\pgfpathlineto{\pgfqpoint{1.422980in}{1.164948in}}%
\pgfpathlineto{\pgfqpoint{1.434692in}{1.165227in}}%
\pgfpathlineto{\pgfqpoint{1.436974in}{1.164120in}}%
\pgfpathlineto{\pgfqpoint{1.446302in}{1.151991in}}%
\pgfpathlineto{\pgfqpoint{1.444242in}{1.139863in}}%
\pgfpathlineto{\pgfqpoint{1.434692in}{1.135419in}}%
\pgfpathlineto{\pgfqpoint{1.422980in}{1.135408in}}%
\pgfpathclose%
\pgfusepath{fill}%
\end{pgfscope}%
\begin{pgfscope}%
\pgfpathrectangle{\pgfqpoint{0.211875in}{0.211875in}}{\pgfqpoint{1.313625in}{1.279725in}}%
\pgfusepath{clip}%
\pgfsetbuttcap%
\pgfsetroundjoin%
\definecolor{currentfill}{rgb}{0.961115,0.566634,0.405693}%
\pgfsetfillcolor{currentfill}%
\pgfsetlinewidth{0.000000pt}%
\definecolor{currentstroke}{rgb}{0.000000,0.000000,0.000000}%
\pgfsetstrokecolor{currentstroke}%
\pgfsetdash{}{0pt}%
\pgfpathmoveto{\pgfqpoint{1.223864in}{1.289588in}}%
\pgfpathlineto{\pgfqpoint{1.233200in}{1.297538in}}%
\pgfpathlineto{\pgfqpoint{1.235577in}{1.307432in}}%
\pgfpathlineto{\pgfqpoint{1.235841in}{1.309667in}}%
\pgfpathlineto{\pgfqpoint{1.235577in}{1.314171in}}%
\pgfpathlineto{\pgfqpoint{1.232339in}{1.321796in}}%
\pgfpathlineto{\pgfqpoint{1.223864in}{1.323034in}}%
\pgfpathlineto{\pgfqpoint{1.221709in}{1.321796in}}%
\pgfpathlineto{\pgfqpoint{1.212826in}{1.309667in}}%
\pgfpathlineto{\pgfqpoint{1.217876in}{1.297538in}}%
\pgfpathclose%
\pgfusepath{fill}%
\end{pgfscope}%
\begin{pgfscope}%
\pgfpathrectangle{\pgfqpoint{0.211875in}{0.211875in}}{\pgfqpoint{1.313625in}{1.279725in}}%
\pgfusepath{clip}%
\pgfsetbuttcap%
\pgfsetroundjoin%
\definecolor{currentfill}{rgb}{0.961115,0.566634,0.405693}%
\pgfsetfillcolor{currentfill}%
\pgfsetlinewidth{0.000000pt}%
\definecolor{currentstroke}{rgb}{0.000000,0.000000,0.000000}%
\pgfsetstrokecolor{currentstroke}%
\pgfsetdash{}{0pt}%
\pgfpathmoveto{\pgfqpoint{1.259002in}{1.339420in}}%
\pgfpathlineto{\pgfqpoint{1.270715in}{1.342544in}}%
\pgfpathlineto{\pgfqpoint{1.282427in}{1.345730in}}%
\pgfpathlineto{\pgfqpoint{1.282704in}{1.346053in}}%
\pgfpathlineto{\pgfqpoint{1.288657in}{1.358182in}}%
\pgfpathlineto{\pgfqpoint{1.290088in}{1.370311in}}%
\pgfpathlineto{\pgfqpoint{1.290515in}{1.382440in}}%
\pgfpathlineto{\pgfqpoint{1.290023in}{1.394569in}}%
\pgfpathlineto{\pgfqpoint{1.284119in}{1.406698in}}%
\pgfpathlineto{\pgfqpoint{1.282427in}{1.408001in}}%
\pgfpathlineto{\pgfqpoint{1.270715in}{1.408681in}}%
\pgfpathlineto{\pgfqpoint{1.266157in}{1.406698in}}%
\pgfpathlineto{\pgfqpoint{1.259002in}{1.402254in}}%
\pgfpathlineto{\pgfqpoint{1.254496in}{1.394569in}}%
\pgfpathlineto{\pgfqpoint{1.251618in}{1.382440in}}%
\pgfpathlineto{\pgfqpoint{1.249375in}{1.370311in}}%
\pgfpathlineto{\pgfqpoint{1.247319in}{1.358182in}}%
\pgfpathlineto{\pgfqpoint{1.248210in}{1.346053in}}%
\pgfpathclose%
\pgfpathmoveto{\pgfqpoint{1.264704in}{1.358182in}}%
\pgfpathlineto{\pgfqpoint{1.259002in}{1.361910in}}%
\pgfpathlineto{\pgfqpoint{1.256752in}{1.370311in}}%
\pgfpathlineto{\pgfqpoint{1.257559in}{1.382440in}}%
\pgfpathlineto{\pgfqpoint{1.259002in}{1.387594in}}%
\pgfpathlineto{\pgfqpoint{1.263902in}{1.394569in}}%
\pgfpathlineto{\pgfqpoint{1.270715in}{1.399153in}}%
\pgfpathlineto{\pgfqpoint{1.279207in}{1.394569in}}%
\pgfpathlineto{\pgfqpoint{1.282427in}{1.385346in}}%
\pgfpathlineto{\pgfqpoint{1.282846in}{1.382440in}}%
\pgfpathlineto{\pgfqpoint{1.283194in}{1.370311in}}%
\pgfpathlineto{\pgfqpoint{1.282427in}{1.366252in}}%
\pgfpathlineto{\pgfqpoint{1.275308in}{1.358182in}}%
\pgfpathlineto{\pgfqpoint{1.270715in}{1.356266in}}%
\pgfpathclose%
\pgfusepath{fill}%
\end{pgfscope}%
\begin{pgfscope}%
\pgfpathrectangle{\pgfqpoint{0.211875in}{0.211875in}}{\pgfqpoint{1.313625in}{1.279725in}}%
\pgfusepath{clip}%
\pgfsetbuttcap%
\pgfsetroundjoin%
\definecolor{currentfill}{rgb}{0.965440,0.720101,0.576404}%
\pgfsetfillcolor{currentfill}%
\pgfsetlinewidth{0.000000pt}%
\definecolor{currentstroke}{rgb}{0.000000,0.000000,0.000000}%
\pgfsetstrokecolor{currentstroke}%
\pgfsetdash{}{0pt}%
\pgfpathmoveto{\pgfqpoint{1.223864in}{0.677386in}}%
\pgfpathlineto{\pgfqpoint{1.227337in}{0.678965in}}%
\pgfpathlineto{\pgfqpoint{1.225730in}{0.691094in}}%
\pgfpathlineto{\pgfqpoint{1.223864in}{0.693150in}}%
\pgfpathlineto{\pgfqpoint{1.212151in}{0.699529in}}%
\pgfpathlineto{\pgfqpoint{1.207247in}{0.691094in}}%
\pgfpathlineto{\pgfqpoint{1.212151in}{0.679852in}}%
\pgfpathlineto{\pgfqpoint{1.213835in}{0.678965in}}%
\pgfpathclose%
\pgfusepath{fill}%
\end{pgfscope}%
\begin{pgfscope}%
\pgfpathrectangle{\pgfqpoint{0.211875in}{0.211875in}}{\pgfqpoint{1.313625in}{1.279725in}}%
\pgfusepath{clip}%
\pgfsetbuttcap%
\pgfsetroundjoin%
\definecolor{currentfill}{rgb}{0.965440,0.720101,0.576404}%
\pgfsetfillcolor{currentfill}%
\pgfsetlinewidth{0.000000pt}%
\definecolor{currentstroke}{rgb}{0.000000,0.000000,0.000000}%
\pgfsetstrokecolor{currentstroke}%
\pgfsetdash{}{0pt}%
\pgfpathmoveto{\pgfqpoint{1.317566in}{0.685281in}}%
\pgfpathlineto{\pgfqpoint{1.329278in}{0.688848in}}%
\pgfpathlineto{\pgfqpoint{1.332539in}{0.691094in}}%
\pgfpathlineto{\pgfqpoint{1.329278in}{0.699610in}}%
\pgfpathlineto{\pgfqpoint{1.317566in}{0.697091in}}%
\pgfpathlineto{\pgfqpoint{1.310579in}{0.691094in}}%
\pgfpathclose%
\pgfusepath{fill}%
\end{pgfscope}%
\begin{pgfscope}%
\pgfpathrectangle{\pgfqpoint{0.211875in}{0.211875in}}{\pgfqpoint{1.313625in}{1.279725in}}%
\pgfusepath{clip}%
\pgfsetbuttcap%
\pgfsetroundjoin%
\definecolor{currentfill}{rgb}{0.965440,0.720101,0.576404}%
\pgfsetfillcolor{currentfill}%
\pgfsetlinewidth{0.000000pt}%
\definecolor{currentstroke}{rgb}{0.000000,0.000000,0.000000}%
\pgfsetstrokecolor{currentstroke}%
\pgfsetdash{}{0pt}%
\pgfpathmoveto{\pgfqpoint{1.422980in}{0.689002in}}%
\pgfpathlineto{\pgfqpoint{1.429596in}{0.691094in}}%
\pgfpathlineto{\pgfqpoint{1.429492in}{0.703223in}}%
\pgfpathlineto{\pgfqpoint{1.422980in}{0.705767in}}%
\pgfpathlineto{\pgfqpoint{1.419083in}{0.703223in}}%
\pgfpathlineto{\pgfqpoint{1.419085in}{0.691094in}}%
\pgfpathclose%
\pgfusepath{fill}%
\end{pgfscope}%
\begin{pgfscope}%
\pgfpathrectangle{\pgfqpoint{0.211875in}{0.211875in}}{\pgfqpoint{1.313625in}{1.279725in}}%
\pgfusepath{clip}%
\pgfsetbuttcap%
\pgfsetroundjoin%
\definecolor{currentfill}{rgb}{0.965440,0.720101,0.576404}%
\pgfsetfillcolor{currentfill}%
\pgfsetlinewidth{0.000000pt}%
\definecolor{currentstroke}{rgb}{0.000000,0.000000,0.000000}%
\pgfsetstrokecolor{currentstroke}%
\pgfsetdash{}{0pt}%
\pgfpathmoveto{\pgfqpoint{1.118450in}{0.821362in}}%
\pgfpathlineto{\pgfqpoint{1.124200in}{0.824512in}}%
\pgfpathlineto{\pgfqpoint{1.128396in}{0.836641in}}%
\pgfpathlineto{\pgfqpoint{1.127854in}{0.848770in}}%
\pgfpathlineto{\pgfqpoint{1.120804in}{0.860898in}}%
\pgfpathlineto{\pgfqpoint{1.118450in}{0.863039in}}%
\pgfpathlineto{\pgfqpoint{1.106737in}{0.862239in}}%
\pgfpathlineto{\pgfqpoint{1.105578in}{0.860898in}}%
\pgfpathlineto{\pgfqpoint{1.100467in}{0.848770in}}%
\pgfpathlineto{\pgfqpoint{1.101218in}{0.836641in}}%
\pgfpathlineto{\pgfqpoint{1.106737in}{0.826951in}}%
\pgfpathlineto{\pgfqpoint{1.111498in}{0.824512in}}%
\pgfpathclose%
\pgfusepath{fill}%
\end{pgfscope}%
\begin{pgfscope}%
\pgfpathrectangle{\pgfqpoint{0.211875in}{0.211875in}}{\pgfqpoint{1.313625in}{1.279725in}}%
\pgfusepath{clip}%
\pgfsetbuttcap%
\pgfsetroundjoin%
\definecolor{currentfill}{rgb}{0.965440,0.720101,0.576404}%
\pgfsetfillcolor{currentfill}%
\pgfsetlinewidth{0.000000pt}%
\definecolor{currentstroke}{rgb}{0.000000,0.000000,0.000000}%
\pgfsetstrokecolor{currentstroke}%
\pgfsetdash{}{0pt}%
\pgfpathmoveto{\pgfqpoint{1.422980in}{0.823450in}}%
\pgfpathlineto{\pgfqpoint{1.424788in}{0.824512in}}%
\pgfpathlineto{\pgfqpoint{1.434692in}{0.831516in}}%
\pgfpathlineto{\pgfqpoint{1.439217in}{0.836641in}}%
\pgfpathlineto{\pgfqpoint{1.440558in}{0.848770in}}%
\pgfpathlineto{\pgfqpoint{1.434692in}{0.850081in}}%
\pgfpathlineto{\pgfqpoint{1.431612in}{0.848770in}}%
\pgfpathlineto{\pgfqpoint{1.422980in}{0.843159in}}%
\pgfpathlineto{\pgfqpoint{1.416879in}{0.836641in}}%
\pgfpathlineto{\pgfqpoint{1.419943in}{0.824512in}}%
\pgfpathclose%
\pgfusepath{fill}%
\end{pgfscope}%
\begin{pgfscope}%
\pgfpathrectangle{\pgfqpoint{0.211875in}{0.211875in}}{\pgfqpoint{1.313625in}{1.279725in}}%
\pgfusepath{clip}%
\pgfsetbuttcap%
\pgfsetroundjoin%
\definecolor{currentfill}{rgb}{0.965440,0.720101,0.576404}%
\pgfsetfillcolor{currentfill}%
\pgfsetlinewidth{0.000000pt}%
\definecolor{currentstroke}{rgb}{0.000000,0.000000,0.000000}%
\pgfsetstrokecolor{currentstroke}%
\pgfsetdash{}{0pt}%
\pgfpathmoveto{\pgfqpoint{1.422980in}{1.135408in}}%
\pgfpathlineto{\pgfqpoint{1.434692in}{1.135419in}}%
\pgfpathlineto{\pgfqpoint{1.444242in}{1.139863in}}%
\pgfpathlineto{\pgfqpoint{1.446302in}{1.151991in}}%
\pgfpathlineto{\pgfqpoint{1.436974in}{1.164120in}}%
\pgfpathlineto{\pgfqpoint{1.434692in}{1.165227in}}%
\pgfpathlineto{\pgfqpoint{1.422980in}{1.164948in}}%
\pgfpathlineto{\pgfqpoint{1.421511in}{1.164120in}}%
\pgfpathlineto{\pgfqpoint{1.412827in}{1.151991in}}%
\pgfpathlineto{\pgfqpoint{1.415995in}{1.139863in}}%
\pgfpathclose%
\pgfusepath{fill}%
\end{pgfscope}%
\begin{pgfscope}%
\pgfpathrectangle{\pgfqpoint{0.211875in}{0.211875in}}{\pgfqpoint{1.313625in}{1.279725in}}%
\pgfusepath{clip}%
\pgfsetbuttcap%
\pgfsetroundjoin%
\definecolor{currentfill}{rgb}{0.965440,0.720101,0.576404}%
\pgfsetfillcolor{currentfill}%
\pgfsetlinewidth{0.000000pt}%
\definecolor{currentstroke}{rgb}{0.000000,0.000000,0.000000}%
\pgfsetstrokecolor{currentstroke}%
\pgfsetdash{}{0pt}%
\pgfpathmoveto{\pgfqpoint{1.270715in}{1.356266in}}%
\pgfpathlineto{\pgfqpoint{1.275308in}{1.358182in}}%
\pgfpathlineto{\pgfqpoint{1.282427in}{1.366252in}}%
\pgfpathlineto{\pgfqpoint{1.283194in}{1.370311in}}%
\pgfpathlineto{\pgfqpoint{1.282846in}{1.382440in}}%
\pgfpathlineto{\pgfqpoint{1.282427in}{1.385346in}}%
\pgfpathlineto{\pgfqpoint{1.279207in}{1.394569in}}%
\pgfpathlineto{\pgfqpoint{1.270715in}{1.399153in}}%
\pgfpathlineto{\pgfqpoint{1.263902in}{1.394569in}}%
\pgfpathlineto{\pgfqpoint{1.259002in}{1.387594in}}%
\pgfpathlineto{\pgfqpoint{1.257559in}{1.382440in}}%
\pgfpathlineto{\pgfqpoint{1.256752in}{1.370311in}}%
\pgfpathlineto{\pgfqpoint{1.259002in}{1.361910in}}%
\pgfpathlineto{\pgfqpoint{1.264704in}{1.358182in}}%
\pgfpathclose%
\pgfpathmoveto{\pgfqpoint{1.267850in}{1.382440in}}%
\pgfpathlineto{\pgfqpoint{1.270715in}{1.386214in}}%
\pgfpathlineto{\pgfqpoint{1.273126in}{1.382440in}}%
\pgfpathlineto{\pgfqpoint{1.270715in}{1.373973in}}%
\pgfpathclose%
\pgfusepath{fill}%
\end{pgfscope}%
\begin{pgfscope}%
\pgfpathrectangle{\pgfqpoint{0.211875in}{0.211875in}}{\pgfqpoint{1.313625in}{1.279725in}}%
\pgfusepath{clip}%
\pgfsetbuttcap%
\pgfsetroundjoin%
\definecolor{currentfill}{rgb}{0.973832,0.856556,0.771584}%
\pgfsetfillcolor{currentfill}%
\pgfsetlinewidth{0.000000pt}%
\definecolor{currentstroke}{rgb}{0.000000,0.000000,0.000000}%
\pgfsetstrokecolor{currentstroke}%
\pgfsetdash{}{0pt}%
\pgfpathmoveto{\pgfqpoint{1.270715in}{1.373973in}}%
\pgfpathlineto{\pgfqpoint{1.273126in}{1.382440in}}%
\pgfpathlineto{\pgfqpoint{1.270715in}{1.386214in}}%
\pgfpathlineto{\pgfqpoint{1.267850in}{1.382440in}}%
\pgfpathclose%
\pgfusepath{fill}%
\end{pgfscope}%
\begin{pgfscope}%
\pgfpathrectangle{\pgfqpoint{0.211875in}{0.211875in}}{\pgfqpoint{1.313625in}{1.279725in}}%
\pgfusepath{clip}%
\pgfsetbuttcap%
\pgfsetroundjoin%
\definecolor{currentfill}{rgb}{0.121569,0.466667,0.705882}%
\pgfsetfillcolor{currentfill}%
\pgfsetlinewidth{1.003750pt}%
\definecolor{currentstroke}{rgb}{0.121569,0.466667,0.705882}%
\pgfsetstrokecolor{currentstroke}%
\pgfsetdash{}{0pt}%
\pgfpathmoveto{\pgfqpoint{1.182698in}{1.330695in}}%
\pgfpathcurveto{\pgfqpoint{1.188521in}{1.330695in}}{\pgfqpoint{1.194108in}{1.333008in}}{\pgfqpoint{1.198226in}{1.337127in}}%
\pgfpathcurveto{\pgfqpoint{1.202344in}{1.341245in}}{\pgfqpoint{1.204658in}{1.346831in}}{\pgfqpoint{1.204658in}{1.352655in}}%
\pgfpathcurveto{\pgfqpoint{1.204658in}{1.358479in}}{\pgfqpoint{1.202344in}{1.364065in}}{\pgfqpoint{1.198226in}{1.368183in}}%
\pgfpathcurveto{\pgfqpoint{1.194108in}{1.372301in}}{\pgfqpoint{1.188521in}{1.374615in}}{\pgfqpoint{1.182698in}{1.374615in}}%
\pgfpathcurveto{\pgfqpoint{1.176874in}{1.374615in}}{\pgfqpoint{1.171287in}{1.372301in}}{\pgfqpoint{1.167169in}{1.368183in}}%
\pgfpathcurveto{\pgfqpoint{1.163051in}{1.364065in}}{\pgfqpoint{1.160737in}{1.358479in}}{\pgfqpoint{1.160737in}{1.352655in}}%
\pgfpathcurveto{\pgfqpoint{1.160737in}{1.346831in}}{\pgfqpoint{1.163051in}{1.341245in}}{\pgfqpoint{1.167169in}{1.337127in}}%
\pgfpathcurveto{\pgfqpoint{1.171287in}{1.333008in}}{\pgfqpoint{1.176874in}{1.330695in}}{\pgfqpoint{1.182698in}{1.330695in}}%
\pgfpathclose%
\pgfusepath{stroke,fill}%
\end{pgfscope}%
\begin{pgfscope}%
\pgfpathrectangle{\pgfqpoint{0.211875in}{0.211875in}}{\pgfqpoint{1.313625in}{1.279725in}}%
\pgfusepath{clip}%
\pgfsetbuttcap%
\pgfsetroundjoin%
\definecolor{currentfill}{rgb}{0.121569,0.466667,0.705882}%
\pgfsetfillcolor{currentfill}%
\pgfsetlinewidth{1.003750pt}%
\definecolor{currentstroke}{rgb}{0.121569,0.466667,0.705882}%
\pgfsetstrokecolor{currentstroke}%
\pgfsetdash{}{0pt}%
\pgfpathmoveto{\pgfqpoint{0.913334in}{0.879500in}}%
\pgfpathcurveto{\pgfqpoint{0.919158in}{0.879500in}}{\pgfqpoint{0.924744in}{0.881814in}}{\pgfqpoint{0.928862in}{0.885932in}}%
\pgfpathcurveto{\pgfqpoint{0.932980in}{0.890050in}}{\pgfqpoint{0.935294in}{0.895636in}}{\pgfqpoint{0.935294in}{0.901460in}}%
\pgfpathcurveto{\pgfqpoint{0.935294in}{0.907284in}}{\pgfqpoint{0.932980in}{0.912870in}}{\pgfqpoint{0.928862in}{0.916988in}}%
\pgfpathcurveto{\pgfqpoint{0.924744in}{0.921106in}}{\pgfqpoint{0.919158in}{0.923420in}}{\pgfqpoint{0.913334in}{0.923420in}}%
\pgfpathcurveto{\pgfqpoint{0.907510in}{0.923420in}}{\pgfqpoint{0.901924in}{0.921106in}}{\pgfqpoint{0.897805in}{0.916988in}}%
\pgfpathcurveto{\pgfqpoint{0.893687in}{0.912870in}}{\pgfqpoint{0.891373in}{0.907284in}}{\pgfqpoint{0.891373in}{0.901460in}}%
\pgfpathcurveto{\pgfqpoint{0.891373in}{0.895636in}}{\pgfqpoint{0.893687in}{0.890050in}}{\pgfqpoint{0.897805in}{0.885932in}}%
\pgfpathcurveto{\pgfqpoint{0.901924in}{0.881814in}}{\pgfqpoint{0.907510in}{0.879500in}}{\pgfqpoint{0.913334in}{0.879500in}}%
\pgfpathclose%
\pgfusepath{stroke,fill}%
\end{pgfscope}%
\begin{pgfscope}%
\pgfpathrectangle{\pgfqpoint{0.211875in}{0.211875in}}{\pgfqpoint{1.313625in}{1.279725in}}%
\pgfusepath{clip}%
\pgfsetbuttcap%
\pgfsetroundjoin%
\definecolor{currentfill}{rgb}{0.121569,0.466667,0.705882}%
\pgfsetfillcolor{currentfill}%
\pgfsetlinewidth{1.003750pt}%
\definecolor{currentstroke}{rgb}{0.121569,0.466667,0.705882}%
\pgfsetstrokecolor{currentstroke}%
\pgfsetdash{}{0pt}%
\pgfpathmoveto{\pgfqpoint{0.985237in}{0.934502in}}%
\pgfpathcurveto{\pgfqpoint{0.991060in}{0.934502in}}{\pgfqpoint{0.996647in}{0.936816in}}{\pgfqpoint{1.000765in}{0.940934in}}%
\pgfpathcurveto{\pgfqpoint{1.004883in}{0.945052in}}{\pgfqpoint{1.007197in}{0.950639in}}{\pgfqpoint{1.007197in}{0.956462in}}%
\pgfpathcurveto{\pgfqpoint{1.007197in}{0.962286in}}{\pgfqpoint{1.004883in}{0.967873in}}{\pgfqpoint{1.000765in}{0.971991in}}%
\pgfpathcurveto{\pgfqpoint{0.996647in}{0.976109in}}{\pgfqpoint{0.991060in}{0.978423in}}{\pgfqpoint{0.985237in}{0.978423in}}%
\pgfpathcurveto{\pgfqpoint{0.979413in}{0.978423in}}{\pgfqpoint{0.973826in}{0.976109in}}{\pgfqpoint{0.969708in}{0.971991in}}%
\pgfpathcurveto{\pgfqpoint{0.965590in}{0.967873in}}{\pgfqpoint{0.963276in}{0.962286in}}{\pgfqpoint{0.963276in}{0.956462in}}%
\pgfpathcurveto{\pgfqpoint{0.963276in}{0.950639in}}{\pgfqpoint{0.965590in}{0.945052in}}{\pgfqpoint{0.969708in}{0.940934in}}%
\pgfpathcurveto{\pgfqpoint{0.973826in}{0.936816in}}{\pgfqpoint{0.979413in}{0.934502in}}{\pgfqpoint{0.985237in}{0.934502in}}%
\pgfpathclose%
\pgfusepath{stroke,fill}%
\end{pgfscope}%
\begin{pgfscope}%
\pgfpathrectangle{\pgfqpoint{0.211875in}{0.211875in}}{\pgfqpoint{1.313625in}{1.279725in}}%
\pgfusepath{clip}%
\pgfsetbuttcap%
\pgfsetroundjoin%
\definecolor{currentfill}{rgb}{0.121569,0.466667,0.705882}%
\pgfsetfillcolor{currentfill}%
\pgfsetlinewidth{1.003750pt}%
\definecolor{currentstroke}{rgb}{0.121569,0.466667,0.705882}%
\pgfsetstrokecolor{currentstroke}%
\pgfsetdash{}{0pt}%
\pgfpathmoveto{\pgfqpoint{1.075619in}{0.974429in}}%
\pgfpathcurveto{\pgfqpoint{1.081443in}{0.974429in}}{\pgfqpoint{1.087029in}{0.976743in}}{\pgfqpoint{1.091148in}{0.980861in}}%
\pgfpathcurveto{\pgfqpoint{1.095266in}{0.984980in}}{\pgfqpoint{1.097580in}{0.990566in}}{\pgfqpoint{1.097580in}{0.996390in}}%
\pgfpathcurveto{\pgfqpoint{1.097580in}{1.002214in}}{\pgfqpoint{1.095266in}{1.007800in}}{\pgfqpoint{1.091148in}{1.011918in}}%
\pgfpathcurveto{\pgfqpoint{1.087029in}{1.016036in}}{\pgfqpoint{1.081443in}{1.018350in}}{\pgfqpoint{1.075619in}{1.018350in}}%
\pgfpathcurveto{\pgfqpoint{1.069795in}{1.018350in}}{\pgfqpoint{1.064209in}{1.016036in}}{\pgfqpoint{1.060091in}{1.011918in}}%
\pgfpathcurveto{\pgfqpoint{1.055973in}{1.007800in}}{\pgfqpoint{1.053659in}{1.002214in}}{\pgfqpoint{1.053659in}{0.996390in}}%
\pgfpathcurveto{\pgfqpoint{1.053659in}{0.990566in}}{\pgfqpoint{1.055973in}{0.984980in}}{\pgfqpoint{1.060091in}{0.980861in}}%
\pgfpathcurveto{\pgfqpoint{1.064209in}{0.976743in}}{\pgfqpoint{1.069795in}{0.974429in}}{\pgfqpoint{1.075619in}{0.974429in}}%
\pgfpathclose%
\pgfusepath{stroke,fill}%
\end{pgfscope}%
\begin{pgfscope}%
\pgfpathrectangle{\pgfqpoint{0.211875in}{0.211875in}}{\pgfqpoint{1.313625in}{1.279725in}}%
\pgfusepath{clip}%
\pgfsetbuttcap%
\pgfsetroundjoin%
\definecolor{currentfill}{rgb}{0.121569,0.466667,0.705882}%
\pgfsetfillcolor{currentfill}%
\pgfsetlinewidth{1.003750pt}%
\definecolor{currentstroke}{rgb}{0.121569,0.466667,0.705882}%
\pgfsetstrokecolor{currentstroke}%
\pgfsetdash{}{0pt}%
\pgfpathmoveto{\pgfqpoint{0.433180in}{1.222930in}}%
\pgfpathcurveto{\pgfqpoint{0.439004in}{1.222930in}}{\pgfqpoint{0.444590in}{1.225244in}}{\pgfqpoint{0.448709in}{1.229362in}}%
\pgfpathcurveto{\pgfqpoint{0.452827in}{1.233481in}}{\pgfqpoint{0.455141in}{1.239067in}}{\pgfqpoint{0.455141in}{1.244891in}}%
\pgfpathcurveto{\pgfqpoint{0.455141in}{1.250715in}}{\pgfqpoint{0.452827in}{1.256301in}}{\pgfqpoint{0.448709in}{1.260419in}}%
\pgfpathcurveto{\pgfqpoint{0.444590in}{1.264537in}}{\pgfqpoint{0.439004in}{1.266851in}}{\pgfqpoint{0.433180in}{1.266851in}}%
\pgfpathcurveto{\pgfqpoint{0.427356in}{1.266851in}}{\pgfqpoint{0.421770in}{1.264537in}}{\pgfqpoint{0.417652in}{1.260419in}}%
\pgfpathcurveto{\pgfqpoint{0.413534in}{1.256301in}}{\pgfqpoint{0.411220in}{1.250715in}}{\pgfqpoint{0.411220in}{1.244891in}}%
\pgfpathcurveto{\pgfqpoint{0.411220in}{1.239067in}}{\pgfqpoint{0.413534in}{1.233481in}}{\pgfqpoint{0.417652in}{1.229362in}}%
\pgfpathcurveto{\pgfqpoint{0.421770in}{1.225244in}}{\pgfqpoint{0.427356in}{1.222930in}}{\pgfqpoint{0.433180in}{1.222930in}}%
\pgfpathclose%
\pgfusepath{stroke,fill}%
\end{pgfscope}%
\begin{pgfscope}%
\pgfpathrectangle{\pgfqpoint{0.211875in}{0.211875in}}{\pgfqpoint{1.313625in}{1.279725in}}%
\pgfusepath{clip}%
\pgfsetbuttcap%
\pgfsetroundjoin%
\definecolor{currentfill}{rgb}{0.121569,0.466667,0.705882}%
\pgfsetfillcolor{currentfill}%
\pgfsetlinewidth{1.003750pt}%
\definecolor{currentstroke}{rgb}{0.121569,0.466667,0.705882}%
\pgfsetstrokecolor{currentstroke}%
\pgfsetdash{}{0pt}%
\pgfpathmoveto{\pgfqpoint{1.172901in}{0.966911in}}%
\pgfpathcurveto{\pgfqpoint{1.178724in}{0.966911in}}{\pgfqpoint{1.184311in}{0.969225in}}{\pgfqpoint{1.188429in}{0.973343in}}%
\pgfpathcurveto{\pgfqpoint{1.192547in}{0.977461in}}{\pgfqpoint{1.194861in}{0.983047in}}{\pgfqpoint{1.194861in}{0.988871in}}%
\pgfpathcurveto{\pgfqpoint{1.194861in}{0.994695in}}{\pgfqpoint{1.192547in}{1.000281in}}{\pgfqpoint{1.188429in}{1.004399in}}%
\pgfpathcurveto{\pgfqpoint{1.184311in}{1.008518in}}{\pgfqpoint{1.178724in}{1.010831in}}{\pgfqpoint{1.172901in}{1.010831in}}%
\pgfpathcurveto{\pgfqpoint{1.167077in}{1.010831in}}{\pgfqpoint{1.161490in}{1.008518in}}{\pgfqpoint{1.157372in}{1.004399in}}%
\pgfpathcurveto{\pgfqpoint{1.153254in}{1.000281in}}{\pgfqpoint{1.150940in}{0.994695in}}{\pgfqpoint{1.150940in}{0.988871in}}%
\pgfpathcurveto{\pgfqpoint{1.150940in}{0.983047in}}{\pgfqpoint{1.153254in}{0.977461in}}{\pgfqpoint{1.157372in}{0.973343in}}%
\pgfpathcurveto{\pgfqpoint{1.161490in}{0.969225in}}{\pgfqpoint{1.167077in}{0.966911in}}{\pgfqpoint{1.172901in}{0.966911in}}%
\pgfpathclose%
\pgfusepath{stroke,fill}%
\end{pgfscope}%
\begin{pgfscope}%
\pgfpathrectangle{\pgfqpoint{0.211875in}{0.211875in}}{\pgfqpoint{1.313625in}{1.279725in}}%
\pgfusepath{clip}%
\pgfsetbuttcap%
\pgfsetroundjoin%
\definecolor{currentfill}{rgb}{0.121569,0.466667,0.705882}%
\pgfsetfillcolor{currentfill}%
\pgfsetlinewidth{1.003750pt}%
\definecolor{currentstroke}{rgb}{0.121569,0.466667,0.705882}%
\pgfsetstrokecolor{currentstroke}%
\pgfsetdash{}{0pt}%
\pgfpathmoveto{\pgfqpoint{1.227640in}{0.684841in}}%
\pgfpathcurveto{\pgfqpoint{1.233464in}{0.684841in}}{\pgfqpoint{1.239050in}{0.687155in}}{\pgfqpoint{1.243168in}{0.691273in}}%
\pgfpathcurveto{\pgfqpoint{1.247286in}{0.695391in}}{\pgfqpoint{1.249600in}{0.700977in}}{\pgfqpoint{1.249600in}{0.706801in}}%
\pgfpathcurveto{\pgfqpoint{1.249600in}{0.712625in}}{\pgfqpoint{1.247286in}{0.718211in}}{\pgfqpoint{1.243168in}{0.722330in}}%
\pgfpathcurveto{\pgfqpoint{1.239050in}{0.726448in}}{\pgfqpoint{1.233464in}{0.728762in}}{\pgfqpoint{1.227640in}{0.728762in}}%
\pgfpathcurveto{\pgfqpoint{1.221816in}{0.728762in}}{\pgfqpoint{1.216230in}{0.726448in}}{\pgfqpoint{1.212112in}{0.722330in}}%
\pgfpathcurveto{\pgfqpoint{1.207993in}{0.718211in}}{\pgfqpoint{1.205679in}{0.712625in}}{\pgfqpoint{1.205679in}{0.706801in}}%
\pgfpathcurveto{\pgfqpoint{1.205679in}{0.700977in}}{\pgfqpoint{1.207993in}{0.695391in}}{\pgfqpoint{1.212112in}{0.691273in}}%
\pgfpathcurveto{\pgfqpoint{1.216230in}{0.687155in}}{\pgfqpoint{1.221816in}{0.684841in}}{\pgfqpoint{1.227640in}{0.684841in}}%
\pgfpathclose%
\pgfusepath{stroke,fill}%
\end{pgfscope}%
\begin{pgfscope}%
\pgfpathrectangle{\pgfqpoint{0.211875in}{0.211875in}}{\pgfqpoint{1.313625in}{1.279725in}}%
\pgfusepath{clip}%
\pgfsetbuttcap%
\pgfsetroundjoin%
\definecolor{currentfill}{rgb}{0.121569,0.466667,0.705882}%
\pgfsetfillcolor{currentfill}%
\pgfsetlinewidth{1.003750pt}%
\definecolor{currentstroke}{rgb}{0.121569,0.466667,0.705882}%
\pgfsetstrokecolor{currentstroke}%
\pgfsetdash{}{0pt}%
\pgfpathmoveto{\pgfqpoint{1.279960in}{0.489344in}}%
\pgfpathcurveto{\pgfqpoint{1.285784in}{0.489344in}}{\pgfqpoint{1.291370in}{0.491658in}}{\pgfqpoint{1.295488in}{0.495776in}}%
\pgfpathcurveto{\pgfqpoint{1.299606in}{0.499894in}}{\pgfqpoint{1.301920in}{0.505480in}}{\pgfqpoint{1.301920in}{0.511304in}}%
\pgfpathcurveto{\pgfqpoint{1.301920in}{0.517128in}}{\pgfqpoint{1.299606in}{0.522715in}}{\pgfqpoint{1.295488in}{0.526833in}}%
\pgfpathcurveto{\pgfqpoint{1.291370in}{0.530951in}}{\pgfqpoint{1.285784in}{0.533265in}}{\pgfqpoint{1.279960in}{0.533265in}}%
\pgfpathcurveto{\pgfqpoint{1.274136in}{0.533265in}}{\pgfqpoint{1.268550in}{0.530951in}}{\pgfqpoint{1.264432in}{0.526833in}}%
\pgfpathcurveto{\pgfqpoint{1.260313in}{0.522715in}}{\pgfqpoint{1.258000in}{0.517128in}}{\pgfqpoint{1.258000in}{0.511304in}}%
\pgfpathcurveto{\pgfqpoint{1.258000in}{0.505480in}}{\pgfqpoint{1.260313in}{0.499894in}}{\pgfqpoint{1.264432in}{0.495776in}}%
\pgfpathcurveto{\pgfqpoint{1.268550in}{0.491658in}}{\pgfqpoint{1.274136in}{0.489344in}}{\pgfqpoint{1.279960in}{0.489344in}}%
\pgfpathclose%
\pgfusepath{stroke,fill}%
\end{pgfscope}%
\begin{pgfscope}%
\pgfpathrectangle{\pgfqpoint{0.211875in}{0.211875in}}{\pgfqpoint{1.313625in}{1.279725in}}%
\pgfusepath{clip}%
\pgfsetbuttcap%
\pgfsetroundjoin%
\definecolor{currentfill}{rgb}{0.121569,0.466667,0.705882}%
\pgfsetfillcolor{currentfill}%
\pgfsetlinewidth{1.003750pt}%
\definecolor{currentstroke}{rgb}{0.121569,0.466667,0.705882}%
\pgfsetstrokecolor{currentstroke}%
\pgfsetdash{}{0pt}%
\pgfpathmoveto{\pgfqpoint{1.181835in}{0.869895in}}%
\pgfpathcurveto{\pgfqpoint{1.187659in}{0.869895in}}{\pgfqpoint{1.193245in}{0.872209in}}{\pgfqpoint{1.197363in}{0.876327in}}%
\pgfpathcurveto{\pgfqpoint{1.201481in}{0.880445in}}{\pgfqpoint{1.203795in}{0.886031in}}{\pgfqpoint{1.203795in}{0.891855in}}%
\pgfpathcurveto{\pgfqpoint{1.203795in}{0.897679in}}{\pgfqpoint{1.201481in}{0.903265in}}{\pgfqpoint{1.197363in}{0.907383in}}%
\pgfpathcurveto{\pgfqpoint{1.193245in}{0.911501in}}{\pgfqpoint{1.187659in}{0.913815in}}{\pgfqpoint{1.181835in}{0.913815in}}%
\pgfpathcurveto{\pgfqpoint{1.176011in}{0.913815in}}{\pgfqpoint{1.170425in}{0.911501in}}{\pgfqpoint{1.166307in}{0.907383in}}%
\pgfpathcurveto{\pgfqpoint{1.162188in}{0.903265in}}{\pgfqpoint{1.159875in}{0.897679in}}{\pgfqpoint{1.159875in}{0.891855in}}%
\pgfpathcurveto{\pgfqpoint{1.159875in}{0.886031in}}{\pgfqpoint{1.162188in}{0.880445in}}{\pgfqpoint{1.166307in}{0.876327in}}%
\pgfpathcurveto{\pgfqpoint{1.170425in}{0.872209in}}{\pgfqpoint{1.176011in}{0.869895in}}{\pgfqpoint{1.181835in}{0.869895in}}%
\pgfpathclose%
\pgfusepath{stroke,fill}%
\end{pgfscope}%
\begin{pgfscope}%
\pgfpathrectangle{\pgfqpoint{0.211875in}{0.211875in}}{\pgfqpoint{1.313625in}{1.279725in}}%
\pgfusepath{clip}%
\pgfsetbuttcap%
\pgfsetroundjoin%
\definecolor{currentfill}{rgb}{0.121569,0.466667,0.705882}%
\pgfsetfillcolor{currentfill}%
\pgfsetlinewidth{1.003750pt}%
\definecolor{currentstroke}{rgb}{0.121569,0.466667,0.705882}%
\pgfsetstrokecolor{currentstroke}%
\pgfsetdash{}{0pt}%
\pgfpathmoveto{\pgfqpoint{0.421679in}{1.118284in}}%
\pgfpathcurveto{\pgfqpoint{0.427503in}{1.118284in}}{\pgfqpoint{0.433089in}{1.120598in}}{\pgfqpoint{0.437208in}{1.124716in}}%
\pgfpathcurveto{\pgfqpoint{0.441326in}{1.128834in}}{\pgfqpoint{0.443640in}{1.134421in}}{\pgfqpoint{0.443640in}{1.140245in}}%
\pgfpathcurveto{\pgfqpoint{0.443640in}{1.146069in}}{\pgfqpoint{0.441326in}{1.151655in}}{\pgfqpoint{0.437208in}{1.155773in}}%
\pgfpathcurveto{\pgfqpoint{0.433089in}{1.159891in}}{\pgfqpoint{0.427503in}{1.162205in}}{\pgfqpoint{0.421679in}{1.162205in}}%
\pgfpathcurveto{\pgfqpoint{0.415855in}{1.162205in}}{\pgfqpoint{0.410269in}{1.159891in}}{\pgfqpoint{0.406151in}{1.155773in}}%
\pgfpathcurveto{\pgfqpoint{0.402033in}{1.151655in}}{\pgfqpoint{0.399719in}{1.146069in}}{\pgfqpoint{0.399719in}{1.140245in}}%
\pgfpathcurveto{\pgfqpoint{0.399719in}{1.134421in}}{\pgfqpoint{0.402033in}{1.128834in}}{\pgfqpoint{0.406151in}{1.124716in}}%
\pgfpathcurveto{\pgfqpoint{0.410269in}{1.120598in}}{\pgfqpoint{0.415855in}{1.118284in}}{\pgfqpoint{0.421679in}{1.118284in}}%
\pgfpathclose%
\pgfusepath{stroke,fill}%
\end{pgfscope}%
\begin{pgfscope}%
\pgfpathrectangle{\pgfqpoint{0.211875in}{0.211875in}}{\pgfqpoint{1.313625in}{1.279725in}}%
\pgfusepath{clip}%
\pgfsetbuttcap%
\pgfsetroundjoin%
\definecolor{currentfill}{rgb}{0.121569,0.466667,0.705882}%
\pgfsetfillcolor{currentfill}%
\pgfsetlinewidth{1.003750pt}%
\definecolor{currentstroke}{rgb}{0.121569,0.466667,0.705882}%
\pgfsetstrokecolor{currentstroke}%
\pgfsetdash{}{0pt}%
\pgfpathmoveto{\pgfqpoint{0.830319in}{0.606339in}}%
\pgfpathcurveto{\pgfqpoint{0.836143in}{0.606339in}}{\pgfqpoint{0.841729in}{0.608653in}}{\pgfqpoint{0.845848in}{0.612771in}}%
\pgfpathcurveto{\pgfqpoint{0.849966in}{0.616889in}}{\pgfqpoint{0.852280in}{0.622476in}}{\pgfqpoint{0.852280in}{0.628300in}}%
\pgfpathcurveto{\pgfqpoint{0.852280in}{0.634123in}}{\pgfqpoint{0.849966in}{0.639710in}}{\pgfqpoint{0.845848in}{0.643828in}}%
\pgfpathcurveto{\pgfqpoint{0.841729in}{0.647946in}}{\pgfqpoint{0.836143in}{0.650260in}}{\pgfqpoint{0.830319in}{0.650260in}}%
\pgfpathcurveto{\pgfqpoint{0.824495in}{0.650260in}}{\pgfqpoint{0.818909in}{0.647946in}}{\pgfqpoint{0.814791in}{0.643828in}}%
\pgfpathcurveto{\pgfqpoint{0.810673in}{0.639710in}}{\pgfqpoint{0.808359in}{0.634123in}}{\pgfqpoint{0.808359in}{0.628300in}}%
\pgfpathcurveto{\pgfqpoint{0.808359in}{0.622476in}}{\pgfqpoint{0.810673in}{0.616889in}}{\pgfqpoint{0.814791in}{0.612771in}}%
\pgfpathcurveto{\pgfqpoint{0.818909in}{0.608653in}}{\pgfqpoint{0.824495in}{0.606339in}}{\pgfqpoint{0.830319in}{0.606339in}}%
\pgfpathclose%
\pgfusepath{stroke,fill}%
\end{pgfscope}%
\begin{pgfscope}%
\pgfpathrectangle{\pgfqpoint{0.211875in}{0.211875in}}{\pgfqpoint{1.313625in}{1.279725in}}%
\pgfusepath{clip}%
\pgfsetbuttcap%
\pgfsetroundjoin%
\definecolor{currentfill}{rgb}{0.121569,0.466667,0.705882}%
\pgfsetfillcolor{currentfill}%
\pgfsetlinewidth{1.003750pt}%
\definecolor{currentstroke}{rgb}{0.121569,0.466667,0.705882}%
\pgfsetstrokecolor{currentstroke}%
\pgfsetdash{}{0pt}%
\pgfpathmoveto{\pgfqpoint{0.824478in}{0.273368in}}%
\pgfpathcurveto{\pgfqpoint{0.830302in}{0.273368in}}{\pgfqpoint{0.835888in}{0.275682in}}{\pgfqpoint{0.840006in}{0.279800in}}%
\pgfpathcurveto{\pgfqpoint{0.844124in}{0.283918in}}{\pgfqpoint{0.846438in}{0.289504in}}{\pgfqpoint{0.846438in}{0.295328in}}%
\pgfpathcurveto{\pgfqpoint{0.846438in}{0.301152in}}{\pgfqpoint{0.844124in}{0.306738in}}{\pgfqpoint{0.840006in}{0.310857in}}%
\pgfpathcurveto{\pgfqpoint{0.835888in}{0.314975in}}{\pgfqpoint{0.830302in}{0.317289in}}{\pgfqpoint{0.824478in}{0.317289in}}%
\pgfpathcurveto{\pgfqpoint{0.818654in}{0.317289in}}{\pgfqpoint{0.813068in}{0.314975in}}{\pgfqpoint{0.808950in}{0.310857in}}%
\pgfpathcurveto{\pgfqpoint{0.804831in}{0.306738in}}{\pgfqpoint{0.802518in}{0.301152in}}{\pgfqpoint{0.802518in}{0.295328in}}%
\pgfpathcurveto{\pgfqpoint{0.802518in}{0.289504in}}{\pgfqpoint{0.804831in}{0.283918in}}{\pgfqpoint{0.808950in}{0.279800in}}%
\pgfpathcurveto{\pgfqpoint{0.813068in}{0.275682in}}{\pgfqpoint{0.818654in}{0.273368in}}{\pgfqpoint{0.824478in}{0.273368in}}%
\pgfpathclose%
\pgfusepath{stroke,fill}%
\end{pgfscope}%
\begin{pgfscope}%
\pgfpathrectangle{\pgfqpoint{0.211875in}{0.211875in}}{\pgfqpoint{1.313625in}{1.279725in}}%
\pgfusepath{clip}%
\pgfsetbuttcap%
\pgfsetroundjoin%
\definecolor{currentfill}{rgb}{0.121569,0.466667,0.705882}%
\pgfsetfillcolor{currentfill}%
\pgfsetlinewidth{1.003750pt}%
\definecolor{currentstroke}{rgb}{0.121569,0.466667,0.705882}%
\pgfsetstrokecolor{currentstroke}%
\pgfsetdash{}{0pt}%
\pgfpathmoveto{\pgfqpoint{0.709347in}{0.662573in}}%
\pgfpathcurveto{\pgfqpoint{0.715171in}{0.662573in}}{\pgfqpoint{0.720757in}{0.664887in}}{\pgfqpoint{0.724875in}{0.669005in}}%
\pgfpathcurveto{\pgfqpoint{0.728993in}{0.673123in}}{\pgfqpoint{0.731307in}{0.678709in}}{\pgfqpoint{0.731307in}{0.684533in}}%
\pgfpathcurveto{\pgfqpoint{0.731307in}{0.690357in}}{\pgfqpoint{0.728993in}{0.695943in}}{\pgfqpoint{0.724875in}{0.700061in}}%
\pgfpathcurveto{\pgfqpoint{0.720757in}{0.704180in}}{\pgfqpoint{0.715171in}{0.706493in}}{\pgfqpoint{0.709347in}{0.706493in}}%
\pgfpathcurveto{\pgfqpoint{0.703523in}{0.706493in}}{\pgfqpoint{0.697937in}{0.704180in}}{\pgfqpoint{0.693818in}{0.700061in}}%
\pgfpathcurveto{\pgfqpoint{0.689700in}{0.695943in}}{\pgfqpoint{0.687386in}{0.690357in}}{\pgfqpoint{0.687386in}{0.684533in}}%
\pgfpathcurveto{\pgfqpoint{0.687386in}{0.678709in}}{\pgfqpoint{0.689700in}{0.673123in}}{\pgfqpoint{0.693818in}{0.669005in}}%
\pgfpathcurveto{\pgfqpoint{0.697937in}{0.664887in}}{\pgfqpoint{0.703523in}{0.662573in}}{\pgfqpoint{0.709347in}{0.662573in}}%
\pgfpathclose%
\pgfusepath{stroke,fill}%
\end{pgfscope}%
\begin{pgfscope}%
\pgfpathrectangle{\pgfqpoint{0.211875in}{0.211875in}}{\pgfqpoint{1.313625in}{1.279725in}}%
\pgfusepath{clip}%
\pgfsetbuttcap%
\pgfsetroundjoin%
\definecolor{currentfill}{rgb}{0.121569,0.466667,0.705882}%
\pgfsetfillcolor{currentfill}%
\pgfsetlinewidth{1.003750pt}%
\definecolor{currentstroke}{rgb}{0.121569,0.466667,0.705882}%
\pgfsetstrokecolor{currentstroke}%
\pgfsetdash{}{0pt}%
\pgfpathmoveto{\pgfqpoint{1.183581in}{0.848748in}}%
\pgfpathcurveto{\pgfqpoint{1.189405in}{0.848748in}}{\pgfqpoint{1.194991in}{0.851061in}}{\pgfqpoint{1.199109in}{0.855180in}}%
\pgfpathcurveto{\pgfqpoint{1.203227in}{0.859298in}}{\pgfqpoint{1.205541in}{0.864884in}}{\pgfqpoint{1.205541in}{0.870708in}}%
\pgfpathcurveto{\pgfqpoint{1.205541in}{0.876532in}}{\pgfqpoint{1.203227in}{0.882118in}}{\pgfqpoint{1.199109in}{0.886236in}}%
\pgfpathcurveto{\pgfqpoint{1.194991in}{0.890354in}}{\pgfqpoint{1.189405in}{0.892668in}}{\pgfqpoint{1.183581in}{0.892668in}}%
\pgfpathcurveto{\pgfqpoint{1.177757in}{0.892668in}}{\pgfqpoint{1.172171in}{0.890354in}}{\pgfqpoint{1.168052in}{0.886236in}}%
\pgfpathcurveto{\pgfqpoint{1.163934in}{0.882118in}}{\pgfqpoint{1.161620in}{0.876532in}}{\pgfqpoint{1.161620in}{0.870708in}}%
\pgfpathcurveto{\pgfqpoint{1.161620in}{0.864884in}}{\pgfqpoint{1.163934in}{0.859298in}}{\pgfqpoint{1.168052in}{0.855180in}}%
\pgfpathcurveto{\pgfqpoint{1.172171in}{0.851061in}}{\pgfqpoint{1.177757in}{0.848748in}}{\pgfqpoint{1.183581in}{0.848748in}}%
\pgfpathclose%
\pgfusepath{stroke,fill}%
\end{pgfscope}%
\begin{pgfscope}%
\pgfpathrectangle{\pgfqpoint{0.211875in}{0.211875in}}{\pgfqpoint{1.313625in}{1.279725in}}%
\pgfusepath{clip}%
\pgfsetbuttcap%
\pgfsetroundjoin%
\definecolor{currentfill}{rgb}{0.121569,0.466667,0.705882}%
\pgfsetfillcolor{currentfill}%
\pgfsetlinewidth{1.003750pt}%
\definecolor{currentstroke}{rgb}{0.121569,0.466667,0.705882}%
\pgfsetstrokecolor{currentstroke}%
\pgfsetdash{}{0pt}%
\pgfpathmoveto{\pgfqpoint{1.094411in}{1.162082in}}%
\pgfpathcurveto{\pgfqpoint{1.100235in}{1.162082in}}{\pgfqpoint{1.105821in}{1.164396in}}{\pgfqpoint{1.109939in}{1.168514in}}%
\pgfpathcurveto{\pgfqpoint{1.114057in}{1.172632in}}{\pgfqpoint{1.116371in}{1.178218in}}{\pgfqpoint{1.116371in}{1.184042in}}%
\pgfpathcurveto{\pgfqpoint{1.116371in}{1.189866in}}{\pgfqpoint{1.114057in}{1.195452in}}{\pgfqpoint{1.109939in}{1.199570in}}%
\pgfpathcurveto{\pgfqpoint{1.105821in}{1.203689in}}{\pgfqpoint{1.100235in}{1.206002in}}{\pgfqpoint{1.094411in}{1.206002in}}%
\pgfpathcurveto{\pgfqpoint{1.088587in}{1.206002in}}{\pgfqpoint{1.083001in}{1.203689in}}{\pgfqpoint{1.078882in}{1.199570in}}%
\pgfpathcurveto{\pgfqpoint{1.074764in}{1.195452in}}{\pgfqpoint{1.072450in}{1.189866in}}{\pgfqpoint{1.072450in}{1.184042in}}%
\pgfpathcurveto{\pgfqpoint{1.072450in}{1.178218in}}{\pgfqpoint{1.074764in}{1.172632in}}{\pgfqpoint{1.078882in}{1.168514in}}%
\pgfpathcurveto{\pgfqpoint{1.083001in}{1.164396in}}{\pgfqpoint{1.088587in}{1.162082in}}{\pgfqpoint{1.094411in}{1.162082in}}%
\pgfpathclose%
\pgfusepath{stroke,fill}%
\end{pgfscope}%
\begin{pgfscope}%
\pgfpathrectangle{\pgfqpoint{0.211875in}{0.211875in}}{\pgfqpoint{1.313625in}{1.279725in}}%
\pgfusepath{clip}%
\pgfsetbuttcap%
\pgfsetroundjoin%
\definecolor{currentfill}{rgb}{0.121569,0.466667,0.705882}%
\pgfsetfillcolor{currentfill}%
\pgfsetlinewidth{1.003750pt}%
\definecolor{currentstroke}{rgb}{0.121569,0.466667,0.705882}%
\pgfsetstrokecolor{currentstroke}%
\pgfsetdash{}{0pt}%
\pgfpathmoveto{\pgfqpoint{0.343896in}{1.053697in}}%
\pgfpathcurveto{\pgfqpoint{0.349720in}{1.053697in}}{\pgfqpoint{0.355306in}{1.056011in}}{\pgfqpoint{0.359425in}{1.060129in}}%
\pgfpathcurveto{\pgfqpoint{0.363543in}{1.064247in}}{\pgfqpoint{0.365857in}{1.069833in}}{\pgfqpoint{0.365857in}{1.075657in}}%
\pgfpathcurveto{\pgfqpoint{0.365857in}{1.081481in}}{\pgfqpoint{0.363543in}{1.087067in}}{\pgfqpoint{0.359425in}{1.091185in}}%
\pgfpathcurveto{\pgfqpoint{0.355306in}{1.095303in}}{\pgfqpoint{0.349720in}{1.097617in}}{\pgfqpoint{0.343896in}{1.097617in}}%
\pgfpathcurveto{\pgfqpoint{0.338072in}{1.097617in}}{\pgfqpoint{0.332486in}{1.095303in}}{\pgfqpoint{0.328368in}{1.091185in}}%
\pgfpathcurveto{\pgfqpoint{0.324250in}{1.087067in}}{\pgfqpoint{0.321936in}{1.081481in}}{\pgfqpoint{0.321936in}{1.075657in}}%
\pgfpathcurveto{\pgfqpoint{0.321936in}{1.069833in}}{\pgfqpoint{0.324250in}{1.064247in}}{\pgfqpoint{0.328368in}{1.060129in}}%
\pgfpathcurveto{\pgfqpoint{0.332486in}{1.056011in}}{\pgfqpoint{0.338072in}{1.053697in}}{\pgfqpoint{0.343896in}{1.053697in}}%
\pgfpathclose%
\pgfusepath{stroke,fill}%
\end{pgfscope}%
\begin{pgfscope}%
\pgfpathrectangle{\pgfqpoint{0.211875in}{0.211875in}}{\pgfqpoint{1.313625in}{1.279725in}}%
\pgfusepath{clip}%
\pgfsetbuttcap%
\pgfsetroundjoin%
\definecolor{currentfill}{rgb}{0.121569,0.466667,0.705882}%
\pgfsetfillcolor{currentfill}%
\pgfsetlinewidth{1.003750pt}%
\definecolor{currentstroke}{rgb}{0.121569,0.466667,0.705882}%
\pgfsetstrokecolor{currentstroke}%
\pgfsetdash{}{0pt}%
\pgfpathmoveto{\pgfqpoint{0.510949in}{0.472343in}}%
\pgfpathcurveto{\pgfqpoint{0.516773in}{0.472343in}}{\pgfqpoint{0.522359in}{0.474657in}}{\pgfqpoint{0.526478in}{0.478775in}}%
\pgfpathcurveto{\pgfqpoint{0.530596in}{0.482893in}}{\pgfqpoint{0.532910in}{0.488479in}}{\pgfqpoint{0.532910in}{0.494303in}}%
\pgfpathcurveto{\pgfqpoint{0.532910in}{0.500127in}}{\pgfqpoint{0.530596in}{0.505713in}}{\pgfqpoint{0.526478in}{0.509832in}}%
\pgfpathcurveto{\pgfqpoint{0.522359in}{0.513950in}}{\pgfqpoint{0.516773in}{0.516264in}}{\pgfqpoint{0.510949in}{0.516264in}}%
\pgfpathcurveto{\pgfqpoint{0.505125in}{0.516264in}}{\pgfqpoint{0.499539in}{0.513950in}}{\pgfqpoint{0.495421in}{0.509832in}}%
\pgfpathcurveto{\pgfqpoint{0.491303in}{0.505713in}}{\pgfqpoint{0.488989in}{0.500127in}}{\pgfqpoint{0.488989in}{0.494303in}}%
\pgfpathcurveto{\pgfqpoint{0.488989in}{0.488479in}}{\pgfqpoint{0.491303in}{0.482893in}}{\pgfqpoint{0.495421in}{0.478775in}}%
\pgfpathcurveto{\pgfqpoint{0.499539in}{0.474657in}}{\pgfqpoint{0.505125in}{0.472343in}}{\pgfqpoint{0.510949in}{0.472343in}}%
\pgfpathclose%
\pgfusepath{stroke,fill}%
\end{pgfscope}%
\begin{pgfscope}%
\pgfpathrectangle{\pgfqpoint{0.211875in}{0.211875in}}{\pgfqpoint{1.313625in}{1.279725in}}%
\pgfusepath{clip}%
\pgfsetbuttcap%
\pgfsetroundjoin%
\definecolor{currentfill}{rgb}{0.121569,0.466667,0.705882}%
\pgfsetfillcolor{currentfill}%
\pgfsetlinewidth{1.003750pt}%
\definecolor{currentstroke}{rgb}{0.121569,0.466667,0.705882}%
\pgfsetstrokecolor{currentstroke}%
\pgfsetdash{}{0pt}%
\pgfpathmoveto{\pgfqpoint{1.182030in}{0.406332in}}%
\pgfpathcurveto{\pgfqpoint{1.187854in}{0.406332in}}{\pgfqpoint{1.193440in}{0.408646in}}{\pgfqpoint{1.197558in}{0.412764in}}%
\pgfpathcurveto{\pgfqpoint{1.201676in}{0.416882in}}{\pgfqpoint{1.203990in}{0.422468in}}{\pgfqpoint{1.203990in}{0.428292in}}%
\pgfpathcurveto{\pgfqpoint{1.203990in}{0.434116in}}{\pgfqpoint{1.201676in}{0.439702in}}{\pgfqpoint{1.197558in}{0.443820in}}%
\pgfpathcurveto{\pgfqpoint{1.193440in}{0.447939in}}{\pgfqpoint{1.187854in}{0.450252in}}{\pgfqpoint{1.182030in}{0.450252in}}%
\pgfpathcurveto{\pgfqpoint{1.176206in}{0.450252in}}{\pgfqpoint{1.170620in}{0.447939in}}{\pgfqpoint{1.166502in}{0.443820in}}%
\pgfpathcurveto{\pgfqpoint{1.162383in}{0.439702in}}{\pgfqpoint{1.160070in}{0.434116in}}{\pgfqpoint{1.160070in}{0.428292in}}%
\pgfpathcurveto{\pgfqpoint{1.160070in}{0.422468in}}{\pgfqpoint{1.162383in}{0.416882in}}{\pgfqpoint{1.166502in}{0.412764in}}%
\pgfpathcurveto{\pgfqpoint{1.170620in}{0.408646in}}{\pgfqpoint{1.176206in}{0.406332in}}{\pgfqpoint{1.182030in}{0.406332in}}%
\pgfpathclose%
\pgfusepath{stroke,fill}%
\end{pgfscope}%
\begin{pgfscope}%
\pgfpathrectangle{\pgfqpoint{0.211875in}{0.211875in}}{\pgfqpoint{1.313625in}{1.279725in}}%
\pgfusepath{clip}%
\pgfsetbuttcap%
\pgfsetroundjoin%
\definecolor{currentfill}{rgb}{0.121569,0.466667,0.705882}%
\pgfsetfillcolor{currentfill}%
\pgfsetlinewidth{1.003750pt}%
\definecolor{currentstroke}{rgb}{0.121569,0.466667,0.705882}%
\pgfsetstrokecolor{currentstroke}%
\pgfsetdash{}{0pt}%
\pgfpathmoveto{\pgfqpoint{1.118771in}{1.367600in}}%
\pgfpathcurveto{\pgfqpoint{1.124595in}{1.367600in}}{\pgfqpoint{1.130181in}{1.369914in}}{\pgfqpoint{1.134300in}{1.374032in}}%
\pgfpathcurveto{\pgfqpoint{1.138418in}{1.378150in}}{\pgfqpoint{1.140732in}{1.383737in}}{\pgfqpoint{1.140732in}{1.389561in}}%
\pgfpathcurveto{\pgfqpoint{1.140732in}{1.395384in}}{\pgfqpoint{1.138418in}{1.400971in}}{\pgfqpoint{1.134300in}{1.405089in}}%
\pgfpathcurveto{\pgfqpoint{1.130181in}{1.409207in}}{\pgfqpoint{1.124595in}{1.411521in}}{\pgfqpoint{1.118771in}{1.411521in}}%
\pgfpathcurveto{\pgfqpoint{1.112947in}{1.411521in}}{\pgfqpoint{1.107361in}{1.409207in}}{\pgfqpoint{1.103243in}{1.405089in}}%
\pgfpathcurveto{\pgfqpoint{1.099125in}{1.400971in}}{\pgfqpoint{1.096811in}{1.395384in}}{\pgfqpoint{1.096811in}{1.389561in}}%
\pgfpathcurveto{\pgfqpoint{1.096811in}{1.383737in}}{\pgfqpoint{1.099125in}{1.378150in}}{\pgfqpoint{1.103243in}{1.374032in}}%
\pgfpathcurveto{\pgfqpoint{1.107361in}{1.369914in}}{\pgfqpoint{1.112947in}{1.367600in}}{\pgfqpoint{1.118771in}{1.367600in}}%
\pgfpathclose%
\pgfusepath{stroke,fill}%
\end{pgfscope}%
\begin{pgfscope}%
\pgfpathrectangle{\pgfqpoint{0.211875in}{0.211875in}}{\pgfqpoint{1.313625in}{1.279725in}}%
\pgfusepath{clip}%
\pgfsetbuttcap%
\pgfsetroundjoin%
\definecolor{currentfill}{rgb}{0.121569,0.466667,0.705882}%
\pgfsetfillcolor{currentfill}%
\pgfsetlinewidth{1.003750pt}%
\definecolor{currentstroke}{rgb}{0.121569,0.466667,0.705882}%
\pgfsetstrokecolor{currentstroke}%
\pgfsetdash{}{0pt}%
\pgfpathmoveto{\pgfqpoint{1.227903in}{0.861323in}}%
\pgfpathcurveto{\pgfqpoint{1.233727in}{0.861323in}}{\pgfqpoint{1.239313in}{0.863637in}}{\pgfqpoint{1.243431in}{0.867756in}}%
\pgfpathcurveto{\pgfqpoint{1.247549in}{0.871874in}}{\pgfqpoint{1.249863in}{0.877460in}}{\pgfqpoint{1.249863in}{0.883284in}}%
\pgfpathcurveto{\pgfqpoint{1.249863in}{0.889108in}}{\pgfqpoint{1.247549in}{0.894694in}}{\pgfqpoint{1.243431in}{0.898812in}}%
\pgfpathcurveto{\pgfqpoint{1.239313in}{0.902930in}}{\pgfqpoint{1.233727in}{0.905244in}}{\pgfqpoint{1.227903in}{0.905244in}}%
\pgfpathcurveto{\pgfqpoint{1.222079in}{0.905244in}}{\pgfqpoint{1.216493in}{0.902930in}}{\pgfqpoint{1.212374in}{0.898812in}}%
\pgfpathcurveto{\pgfqpoint{1.208256in}{0.894694in}}{\pgfqpoint{1.205942in}{0.889108in}}{\pgfqpoint{1.205942in}{0.883284in}}%
\pgfpathcurveto{\pgfqpoint{1.205942in}{0.877460in}}{\pgfqpoint{1.208256in}{0.871874in}}{\pgfqpoint{1.212374in}{0.867756in}}%
\pgfpathcurveto{\pgfqpoint{1.216493in}{0.863637in}}{\pgfqpoint{1.222079in}{0.861323in}}{\pgfqpoint{1.227903in}{0.861323in}}%
\pgfpathclose%
\pgfusepath{stroke,fill}%
\end{pgfscope}%
\begin{pgfscope}%
\pgfpathrectangle{\pgfqpoint{0.211875in}{0.211875in}}{\pgfqpoint{1.313625in}{1.279725in}}%
\pgfusepath{clip}%
\pgfsetbuttcap%
\pgfsetroundjoin%
\definecolor{currentfill}{rgb}{0.121569,0.466667,0.705882}%
\pgfsetfillcolor{currentfill}%
\pgfsetlinewidth{1.003750pt}%
\definecolor{currentstroke}{rgb}{0.121569,0.466667,0.705882}%
\pgfsetstrokecolor{currentstroke}%
\pgfsetdash{}{0pt}%
\pgfpathmoveto{\pgfqpoint{0.357132in}{1.403525in}}%
\pgfpathcurveto{\pgfqpoint{0.362956in}{1.403525in}}{\pgfqpoint{0.368542in}{1.405839in}}{\pgfqpoint{0.372660in}{1.409957in}}%
\pgfpathcurveto{\pgfqpoint{0.376778in}{1.414075in}}{\pgfqpoint{0.379092in}{1.419661in}}{\pgfqpoint{0.379092in}{1.425485in}}%
\pgfpathcurveto{\pgfqpoint{0.379092in}{1.431309in}}{\pgfqpoint{0.376778in}{1.436896in}}{\pgfqpoint{0.372660in}{1.441014in}}%
\pgfpathcurveto{\pgfqpoint{0.368542in}{1.445132in}}{\pgfqpoint{0.362956in}{1.447446in}}{\pgfqpoint{0.357132in}{1.447446in}}%
\pgfpathcurveto{\pgfqpoint{0.351308in}{1.447446in}}{\pgfqpoint{0.345722in}{1.445132in}}{\pgfqpoint{0.341604in}{1.441014in}}%
\pgfpathcurveto{\pgfqpoint{0.337486in}{1.436896in}}{\pgfqpoint{0.335172in}{1.431309in}}{\pgfqpoint{0.335172in}{1.425485in}}%
\pgfpathcurveto{\pgfqpoint{0.335172in}{1.419661in}}{\pgfqpoint{0.337486in}{1.414075in}}{\pgfqpoint{0.341604in}{1.409957in}}%
\pgfpathcurveto{\pgfqpoint{0.345722in}{1.405839in}}{\pgfqpoint{0.351308in}{1.403525in}}{\pgfqpoint{0.357132in}{1.403525in}}%
\pgfpathclose%
\pgfusepath{stroke,fill}%
\end{pgfscope}%
\begin{pgfscope}%
\pgfpathrectangle{\pgfqpoint{0.211875in}{0.211875in}}{\pgfqpoint{1.313625in}{1.279725in}}%
\pgfusepath{clip}%
\pgfsetbuttcap%
\pgfsetroundjoin%
\definecolor{currentfill}{rgb}{0.121569,0.466667,0.705882}%
\pgfsetfillcolor{currentfill}%
\pgfsetlinewidth{1.003750pt}%
\definecolor{currentstroke}{rgb}{0.121569,0.466667,0.705882}%
\pgfsetstrokecolor{currentstroke}%
\pgfsetdash{}{0pt}%
\pgfpathmoveto{\pgfqpoint{1.435455in}{1.388115in}}%
\pgfpathcurveto{\pgfqpoint{1.441279in}{1.388115in}}{\pgfqpoint{1.446865in}{1.390429in}}{\pgfqpoint{1.450983in}{1.394547in}}%
\pgfpathcurveto{\pgfqpoint{1.455101in}{1.398666in}}{\pgfqpoint{1.457415in}{1.404252in}}{\pgfqpoint{1.457415in}{1.410076in}}%
\pgfpathcurveto{\pgfqpoint{1.457415in}{1.415900in}}{\pgfqpoint{1.455101in}{1.421486in}}{\pgfqpoint{1.450983in}{1.425604in}}%
\pgfpathcurveto{\pgfqpoint{1.446865in}{1.429722in}}{\pgfqpoint{1.441279in}{1.432036in}}{\pgfqpoint{1.435455in}{1.432036in}}%
\pgfpathcurveto{\pgfqpoint{1.429631in}{1.432036in}}{\pgfqpoint{1.424045in}{1.429722in}}{\pgfqpoint{1.419927in}{1.425604in}}%
\pgfpathcurveto{\pgfqpoint{1.415809in}{1.421486in}}{\pgfqpoint{1.413495in}{1.415900in}}{\pgfqpoint{1.413495in}{1.410076in}}%
\pgfpathcurveto{\pgfqpoint{1.413495in}{1.404252in}}{\pgfqpoint{1.415809in}{1.398666in}}{\pgfqpoint{1.419927in}{1.394547in}}%
\pgfpathcurveto{\pgfqpoint{1.424045in}{1.390429in}}{\pgfqpoint{1.429631in}{1.388115in}}{\pgfqpoint{1.435455in}{1.388115in}}%
\pgfpathclose%
\pgfusepath{stroke,fill}%
\end{pgfscope}%
\begin{pgfscope}%
\pgfpathrectangle{\pgfqpoint{0.211875in}{0.211875in}}{\pgfqpoint{1.313625in}{1.279725in}}%
\pgfusepath{clip}%
\pgfsetbuttcap%
\pgfsetroundjoin%
\definecolor{currentfill}{rgb}{0.121569,0.466667,0.705882}%
\pgfsetfillcolor{currentfill}%
\pgfsetlinewidth{1.003750pt}%
\definecolor{currentstroke}{rgb}{0.121569,0.466667,0.705882}%
\pgfsetstrokecolor{currentstroke}%
\pgfsetdash{}{0pt}%
\pgfpathmoveto{\pgfqpoint{0.304365in}{0.845252in}}%
\pgfpathcurveto{\pgfqpoint{0.310189in}{0.845252in}}{\pgfqpoint{0.315776in}{0.847565in}}{\pgfqpoint{0.319894in}{0.851684in}}%
\pgfpathcurveto{\pgfqpoint{0.324012in}{0.855802in}}{\pgfqpoint{0.326326in}{0.861388in}}{\pgfqpoint{0.326326in}{0.867212in}}%
\pgfpathcurveto{\pgfqpoint{0.326326in}{0.873036in}}{\pgfqpoint{0.324012in}{0.878622in}}{\pgfqpoint{0.319894in}{0.882740in}}%
\pgfpathcurveto{\pgfqpoint{0.315776in}{0.886858in}}{\pgfqpoint{0.310189in}{0.889172in}}{\pgfqpoint{0.304365in}{0.889172in}}%
\pgfpathcurveto{\pgfqpoint{0.298542in}{0.889172in}}{\pgfqpoint{0.292955in}{0.886858in}}{\pgfqpoint{0.288837in}{0.882740in}}%
\pgfpathcurveto{\pgfqpoint{0.284719in}{0.878622in}}{\pgfqpoint{0.282405in}{0.873036in}}{\pgfqpoint{0.282405in}{0.867212in}}%
\pgfpathcurveto{\pgfqpoint{0.282405in}{0.861388in}}{\pgfqpoint{0.284719in}{0.855802in}}{\pgfqpoint{0.288837in}{0.851684in}}%
\pgfpathcurveto{\pgfqpoint{0.292955in}{0.847565in}}{\pgfqpoint{0.298542in}{0.845252in}}{\pgfqpoint{0.304365in}{0.845252in}}%
\pgfpathclose%
\pgfusepath{stroke,fill}%
\end{pgfscope}%
\begin{pgfscope}%
\pgfpathrectangle{\pgfqpoint{0.211875in}{0.211875in}}{\pgfqpoint{1.313625in}{1.279725in}}%
\pgfusepath{clip}%
\pgfsetbuttcap%
\pgfsetroundjoin%
\definecolor{currentfill}{rgb}{0.121569,0.466667,0.705882}%
\pgfsetfillcolor{currentfill}%
\pgfsetlinewidth{1.003750pt}%
\definecolor{currentstroke}{rgb}{0.121569,0.466667,0.705882}%
\pgfsetstrokecolor{currentstroke}%
\pgfsetdash{}{0pt}%
\pgfpathmoveto{\pgfqpoint{0.333314in}{0.287031in}}%
\pgfpathcurveto{\pgfqpoint{0.339138in}{0.287031in}}{\pgfqpoint{0.344724in}{0.289345in}}{\pgfqpoint{0.348842in}{0.293463in}}%
\pgfpathcurveto{\pgfqpoint{0.352960in}{0.297581in}}{\pgfqpoint{0.355274in}{0.303167in}}{\pgfqpoint{0.355274in}{0.308991in}}%
\pgfpathcurveto{\pgfqpoint{0.355274in}{0.314815in}}{\pgfqpoint{0.352960in}{0.320401in}}{\pgfqpoint{0.348842in}{0.324519in}}%
\pgfpathcurveto{\pgfqpoint{0.344724in}{0.328637in}}{\pgfqpoint{0.339138in}{0.330951in}}{\pgfqpoint{0.333314in}{0.330951in}}%
\pgfpathcurveto{\pgfqpoint{0.327490in}{0.330951in}}{\pgfqpoint{0.321904in}{0.328637in}}{\pgfqpoint{0.317786in}{0.324519in}}%
\pgfpathcurveto{\pgfqpoint{0.313668in}{0.320401in}}{\pgfqpoint{0.311354in}{0.314815in}}{\pgfqpoint{0.311354in}{0.308991in}}%
\pgfpathcurveto{\pgfqpoint{0.311354in}{0.303167in}}{\pgfqpoint{0.313668in}{0.297581in}}{\pgfqpoint{0.317786in}{0.293463in}}%
\pgfpathcurveto{\pgfqpoint{0.321904in}{0.289345in}}{\pgfqpoint{0.327490in}{0.287031in}}{\pgfqpoint{0.333314in}{0.287031in}}%
\pgfpathclose%
\pgfusepath{stroke,fill}%
\end{pgfscope}%
\begin{pgfscope}%
\pgfpathrectangle{\pgfqpoint{0.211875in}{0.211875in}}{\pgfqpoint{1.313625in}{1.279725in}}%
\pgfusepath{clip}%
\pgfsetbuttcap%
\pgfsetroundjoin%
\definecolor{currentfill}{rgb}{0.121569,0.466667,0.705882}%
\pgfsetfillcolor{currentfill}%
\pgfsetlinewidth{1.003750pt}%
\definecolor{currentstroke}{rgb}{0.121569,0.466667,0.705882}%
\pgfsetstrokecolor{currentstroke}%
\pgfsetdash{}{0pt}%
\pgfpathmoveto{\pgfqpoint{1.434583in}{0.321927in}}%
\pgfpathcurveto{\pgfqpoint{1.440406in}{0.321927in}}{\pgfqpoint{1.445993in}{0.324241in}}{\pgfqpoint{1.450111in}{0.328359in}}%
\pgfpathcurveto{\pgfqpoint{1.454229in}{0.332477in}}{\pgfqpoint{1.456543in}{0.338064in}}{\pgfqpoint{1.456543in}{0.343887in}}%
\pgfpathcurveto{\pgfqpoint{1.456543in}{0.349711in}}{\pgfqpoint{1.454229in}{0.355298in}}{\pgfqpoint{1.450111in}{0.359416in}}%
\pgfpathcurveto{\pgfqpoint{1.445993in}{0.363534in}}{\pgfqpoint{1.440406in}{0.365848in}}{\pgfqpoint{1.434583in}{0.365848in}}%
\pgfpathcurveto{\pgfqpoint{1.428759in}{0.365848in}}{\pgfqpoint{1.423172in}{0.363534in}}{\pgfqpoint{1.419054in}{0.359416in}}%
\pgfpathcurveto{\pgfqpoint{1.414936in}{0.355298in}}{\pgfqpoint{1.412622in}{0.349711in}}{\pgfqpoint{1.412622in}{0.343887in}}%
\pgfpathcurveto{\pgfqpoint{1.412622in}{0.338064in}}{\pgfqpoint{1.414936in}{0.332477in}}{\pgfqpoint{1.419054in}{0.328359in}}%
\pgfpathcurveto{\pgfqpoint{1.423172in}{0.324241in}}{\pgfqpoint{1.428759in}{0.321927in}}{\pgfqpoint{1.434583in}{0.321927in}}%
\pgfpathclose%
\pgfusepath{stroke,fill}%
\end{pgfscope}%
\begin{pgfscope}%
\pgfpathrectangle{\pgfqpoint{0.211875in}{0.211875in}}{\pgfqpoint{1.313625in}{1.279725in}}%
\pgfusepath{clip}%
\pgfsetbuttcap%
\pgfsetroundjoin%
\definecolor{currentfill}{rgb}{0.121569,0.466667,0.705882}%
\pgfsetfillcolor{currentfill}%
\pgfsetlinewidth{1.003750pt}%
\definecolor{currentstroke}{rgb}{0.121569,0.466667,0.705882}%
\pgfsetstrokecolor{currentstroke}%
\pgfsetdash{}{0pt}%
\pgfpathmoveto{\pgfqpoint{0.297494in}{0.389704in}}%
\pgfpathcurveto{\pgfqpoint{0.303318in}{0.389704in}}{\pgfqpoint{0.308904in}{0.392018in}}{\pgfqpoint{0.313022in}{0.396136in}}%
\pgfpathcurveto{\pgfqpoint{0.317140in}{0.400254in}}{\pgfqpoint{0.319454in}{0.405841in}}{\pgfqpoint{0.319454in}{0.411665in}}%
\pgfpathcurveto{\pgfqpoint{0.319454in}{0.417488in}}{\pgfqpoint{0.317140in}{0.423075in}}{\pgfqpoint{0.313022in}{0.427193in}}%
\pgfpathcurveto{\pgfqpoint{0.308904in}{0.431311in}}{\pgfqpoint{0.303318in}{0.433625in}}{\pgfqpoint{0.297494in}{0.433625in}}%
\pgfpathcurveto{\pgfqpoint{0.291670in}{0.433625in}}{\pgfqpoint{0.286084in}{0.431311in}}{\pgfqpoint{0.281966in}{0.427193in}}%
\pgfpathcurveto{\pgfqpoint{0.277848in}{0.423075in}}{\pgfqpoint{0.275534in}{0.417488in}}{\pgfqpoint{0.275534in}{0.411665in}}%
\pgfpathcurveto{\pgfqpoint{0.275534in}{0.405841in}}{\pgfqpoint{0.277848in}{0.400254in}}{\pgfqpoint{0.281966in}{0.396136in}}%
\pgfpathcurveto{\pgfqpoint{0.286084in}{0.392018in}}{\pgfqpoint{0.291670in}{0.389704in}}{\pgfqpoint{0.297494in}{0.389704in}}%
\pgfpathclose%
\pgfusepath{stroke,fill}%
\end{pgfscope}%
\begin{pgfscope}%
\pgfpathrectangle{\pgfqpoint{0.211875in}{0.211875in}}{\pgfqpoint{1.313625in}{1.279725in}}%
\pgfusepath{clip}%
\pgfsetbuttcap%
\pgfsetroundjoin%
\definecolor{currentfill}{rgb}{0.121569,0.466667,0.705882}%
\pgfsetfillcolor{currentfill}%
\pgfsetlinewidth{1.003750pt}%
\definecolor{currentstroke}{rgb}{0.121569,0.466667,0.705882}%
\pgfsetstrokecolor{currentstroke}%
\pgfsetdash{}{0pt}%
\pgfpathmoveto{\pgfqpoint{1.328311in}{0.668311in}}%
\pgfpathcurveto{\pgfqpoint{1.334135in}{0.668311in}}{\pgfqpoint{1.339721in}{0.670624in}}{\pgfqpoint{1.343839in}{0.674743in}}%
\pgfpathcurveto{\pgfqpoint{1.347957in}{0.678861in}}{\pgfqpoint{1.350271in}{0.684447in}}{\pgfqpoint{1.350271in}{0.690271in}}%
\pgfpathcurveto{\pgfqpoint{1.350271in}{0.696095in}}{\pgfqpoint{1.347957in}{0.701681in}}{\pgfqpoint{1.343839in}{0.705799in}}%
\pgfpathcurveto{\pgfqpoint{1.339721in}{0.709917in}}{\pgfqpoint{1.334135in}{0.712231in}}{\pgfqpoint{1.328311in}{0.712231in}}%
\pgfpathcurveto{\pgfqpoint{1.322487in}{0.712231in}}{\pgfqpoint{1.316901in}{0.709917in}}{\pgfqpoint{1.312783in}{0.705799in}}%
\pgfpathcurveto{\pgfqpoint{1.308664in}{0.701681in}}{\pgfqpoint{1.306350in}{0.696095in}}{\pgfqpoint{1.306350in}{0.690271in}}%
\pgfpathcurveto{\pgfqpoint{1.306350in}{0.684447in}}{\pgfqpoint{1.308664in}{0.678861in}}{\pgfqpoint{1.312783in}{0.674743in}}%
\pgfpathcurveto{\pgfqpoint{1.316901in}{0.670624in}}{\pgfqpoint{1.322487in}{0.668311in}}{\pgfqpoint{1.328311in}{0.668311in}}%
\pgfpathclose%
\pgfusepath{stroke,fill}%
\end{pgfscope}%
\begin{pgfscope}%
\pgfpathrectangle{\pgfqpoint{0.211875in}{0.211875in}}{\pgfqpoint{1.313625in}{1.279725in}}%
\pgfusepath{clip}%
\pgfsetbuttcap%
\pgfsetroundjoin%
\definecolor{currentfill}{rgb}{0.121569,0.466667,0.705882}%
\pgfsetfillcolor{currentfill}%
\pgfsetlinewidth{1.003750pt}%
\definecolor{currentstroke}{rgb}{0.121569,0.466667,0.705882}%
\pgfsetstrokecolor{currentstroke}%
\pgfsetdash{}{0pt}%
\pgfpathmoveto{\pgfqpoint{0.365269in}{1.222950in}}%
\pgfpathcurveto{\pgfqpoint{0.371093in}{1.222950in}}{\pgfqpoint{0.376680in}{1.225264in}}{\pgfqpoint{0.380798in}{1.229382in}}%
\pgfpathcurveto{\pgfqpoint{0.384916in}{1.233500in}}{\pgfqpoint{0.387230in}{1.239086in}}{\pgfqpoint{0.387230in}{1.244910in}}%
\pgfpathcurveto{\pgfqpoint{0.387230in}{1.250734in}}{\pgfqpoint{0.384916in}{1.256320in}}{\pgfqpoint{0.380798in}{1.260439in}}%
\pgfpathcurveto{\pgfqpoint{0.376680in}{1.264557in}}{\pgfqpoint{0.371093in}{1.266871in}}{\pgfqpoint{0.365269in}{1.266871in}}%
\pgfpathcurveto{\pgfqpoint{0.359445in}{1.266871in}}{\pgfqpoint{0.353859in}{1.264557in}}{\pgfqpoint{0.349741in}{1.260439in}}%
\pgfpathcurveto{\pgfqpoint{0.345623in}{1.256320in}}{\pgfqpoint{0.343309in}{1.250734in}}{\pgfqpoint{0.343309in}{1.244910in}}%
\pgfpathcurveto{\pgfqpoint{0.343309in}{1.239086in}}{\pgfqpoint{0.345623in}{1.233500in}}{\pgfqpoint{0.349741in}{1.229382in}}%
\pgfpathcurveto{\pgfqpoint{0.353859in}{1.225264in}}{\pgfqpoint{0.359445in}{1.222950in}}{\pgfqpoint{0.365269in}{1.222950in}}%
\pgfpathclose%
\pgfusepath{stroke,fill}%
\end{pgfscope}%
\begin{pgfscope}%
\pgfpathrectangle{\pgfqpoint{0.211875in}{0.211875in}}{\pgfqpoint{1.313625in}{1.279725in}}%
\pgfusepath{clip}%
\pgfsetbuttcap%
\pgfsetroundjoin%
\definecolor{currentfill}{rgb}{0.121569,0.466667,0.705882}%
\pgfsetfillcolor{currentfill}%
\pgfsetlinewidth{1.003750pt}%
\definecolor{currentstroke}{rgb}{0.121569,0.466667,0.705882}%
\pgfsetstrokecolor{currentstroke}%
\pgfsetdash{}{0pt}%
\pgfpathmoveto{\pgfqpoint{1.427789in}{0.811675in}}%
\pgfpathcurveto{\pgfqpoint{1.433613in}{0.811675in}}{\pgfqpoint{1.439199in}{0.813988in}}{\pgfqpoint{1.443318in}{0.818107in}}%
\pgfpathcurveto{\pgfqpoint{1.447436in}{0.822225in}}{\pgfqpoint{1.449750in}{0.827811in}}{\pgfqpoint{1.449750in}{0.833635in}}%
\pgfpathcurveto{\pgfqpoint{1.449750in}{0.839459in}}{\pgfqpoint{1.447436in}{0.845045in}}{\pgfqpoint{1.443318in}{0.849163in}}%
\pgfpathcurveto{\pgfqpoint{1.439199in}{0.853281in}}{\pgfqpoint{1.433613in}{0.855595in}}{\pgfqpoint{1.427789in}{0.855595in}}%
\pgfpathcurveto{\pgfqpoint{1.421965in}{0.855595in}}{\pgfqpoint{1.416379in}{0.853281in}}{\pgfqpoint{1.412261in}{0.849163in}}%
\pgfpathcurveto{\pgfqpoint{1.408143in}{0.845045in}}{\pgfqpoint{1.405829in}{0.839459in}}{\pgfqpoint{1.405829in}{0.833635in}}%
\pgfpathcurveto{\pgfqpoint{1.405829in}{0.827811in}}{\pgfqpoint{1.408143in}{0.822225in}}{\pgfqpoint{1.412261in}{0.818107in}}%
\pgfpathcurveto{\pgfqpoint{1.416379in}{0.813988in}}{\pgfqpoint{1.421965in}{0.811675in}}{\pgfqpoint{1.427789in}{0.811675in}}%
\pgfpathclose%
\pgfusepath{stroke,fill}%
\end{pgfscope}%
\begin{pgfscope}%
\pgfpathrectangle{\pgfqpoint{0.211875in}{0.211875in}}{\pgfqpoint{1.313625in}{1.279725in}}%
\pgfusepath{clip}%
\pgfsetbuttcap%
\pgfsetroundjoin%
\definecolor{currentfill}{rgb}{0.121569,0.466667,0.705882}%
\pgfsetfillcolor{currentfill}%
\pgfsetlinewidth{1.003750pt}%
\definecolor{currentstroke}{rgb}{0.121569,0.466667,0.705882}%
\pgfsetstrokecolor{currentstroke}%
\pgfsetdash{}{0pt}%
\pgfpathmoveto{\pgfqpoint{1.425365in}{0.593197in}}%
\pgfpathcurveto{\pgfqpoint{1.431189in}{0.593197in}}{\pgfqpoint{1.436775in}{0.595511in}}{\pgfqpoint{1.440893in}{0.599629in}}%
\pgfpathcurveto{\pgfqpoint{1.445011in}{0.603747in}}{\pgfqpoint{1.447325in}{0.609333in}}{\pgfqpoint{1.447325in}{0.615157in}}%
\pgfpathcurveto{\pgfqpoint{1.447325in}{0.620981in}}{\pgfqpoint{1.445011in}{0.626567in}}{\pgfqpoint{1.440893in}{0.630685in}}%
\pgfpathcurveto{\pgfqpoint{1.436775in}{0.634804in}}{\pgfqpoint{1.431189in}{0.637118in}}{\pgfqpoint{1.425365in}{0.637118in}}%
\pgfpathcurveto{\pgfqpoint{1.419541in}{0.637118in}}{\pgfqpoint{1.413955in}{0.634804in}}{\pgfqpoint{1.409836in}{0.630685in}}%
\pgfpathcurveto{\pgfqpoint{1.405718in}{0.626567in}}{\pgfqpoint{1.403404in}{0.620981in}}{\pgfqpoint{1.403404in}{0.615157in}}%
\pgfpathcurveto{\pgfqpoint{1.403404in}{0.609333in}}{\pgfqpoint{1.405718in}{0.603747in}}{\pgfqpoint{1.409836in}{0.599629in}}%
\pgfpathcurveto{\pgfqpoint{1.413955in}{0.595511in}}{\pgfqpoint{1.419541in}{0.593197in}}{\pgfqpoint{1.425365in}{0.593197in}}%
\pgfpathclose%
\pgfusepath{stroke,fill}%
\end{pgfscope}%
\begin{pgfscope}%
\pgfpathrectangle{\pgfqpoint{0.211875in}{0.211875in}}{\pgfqpoint{1.313625in}{1.279725in}}%
\pgfusepath{clip}%
\pgfsetbuttcap%
\pgfsetroundjoin%
\definecolor{currentfill}{rgb}{0.121569,0.466667,0.705882}%
\pgfsetfillcolor{currentfill}%
\pgfsetlinewidth{1.003750pt}%
\definecolor{currentstroke}{rgb}{0.121569,0.466667,0.705882}%
\pgfsetstrokecolor{currentstroke}%
\pgfsetdash{}{0pt}%
\pgfpathmoveto{\pgfqpoint{1.006300in}{0.584664in}}%
\pgfpathcurveto{\pgfqpoint{1.012124in}{0.584664in}}{\pgfqpoint{1.017710in}{0.586978in}}{\pgfqpoint{1.021829in}{0.591096in}}%
\pgfpathcurveto{\pgfqpoint{1.025947in}{0.595214in}}{\pgfqpoint{1.028261in}{0.600800in}}{\pgfqpoint{1.028261in}{0.606624in}}%
\pgfpathcurveto{\pgfqpoint{1.028261in}{0.612448in}}{\pgfqpoint{1.025947in}{0.618034in}}{\pgfqpoint{1.021829in}{0.622152in}}%
\pgfpathcurveto{\pgfqpoint{1.017710in}{0.626271in}}{\pgfqpoint{1.012124in}{0.628584in}}{\pgfqpoint{1.006300in}{0.628584in}}%
\pgfpathcurveto{\pgfqpoint{1.000476in}{0.628584in}}{\pgfqpoint{0.994890in}{0.626271in}}{\pgfqpoint{0.990772in}{0.622152in}}%
\pgfpathcurveto{\pgfqpoint{0.986654in}{0.618034in}}{\pgfqpoint{0.984340in}{0.612448in}}{\pgfqpoint{0.984340in}{0.606624in}}%
\pgfpathcurveto{\pgfqpoint{0.984340in}{0.600800in}}{\pgfqpoint{0.986654in}{0.595214in}}{\pgfqpoint{0.990772in}{0.591096in}}%
\pgfpathcurveto{\pgfqpoint{0.994890in}{0.586978in}}{\pgfqpoint{1.000476in}{0.584664in}}{\pgfqpoint{1.006300in}{0.584664in}}%
\pgfpathclose%
\pgfusepath{stroke,fill}%
\end{pgfscope}%
\begin{pgfscope}%
\pgfpathrectangle{\pgfqpoint{0.211875in}{0.211875in}}{\pgfqpoint{1.313625in}{1.279725in}}%
\pgfusepath{clip}%
\pgfsetbuttcap%
\pgfsetroundjoin%
\definecolor{currentfill}{rgb}{0.121569,0.466667,0.705882}%
\pgfsetfillcolor{currentfill}%
\pgfsetlinewidth{1.003750pt}%
\definecolor{currentstroke}{rgb}{0.121569,0.466667,0.705882}%
\pgfsetstrokecolor{currentstroke}%
\pgfsetdash{}{0pt}%
\pgfpathmoveto{\pgfqpoint{0.751176in}{1.122282in}}%
\pgfpathcurveto{\pgfqpoint{0.757000in}{1.122282in}}{\pgfqpoint{0.762586in}{1.124596in}}{\pgfqpoint{0.766705in}{1.128714in}}%
\pgfpathcurveto{\pgfqpoint{0.770823in}{1.132832in}}{\pgfqpoint{0.773137in}{1.138419in}}{\pgfqpoint{0.773137in}{1.144243in}}%
\pgfpathcurveto{\pgfqpoint{0.773137in}{1.150067in}}{\pgfqpoint{0.770823in}{1.155653in}}{\pgfqpoint{0.766705in}{1.159771in}}%
\pgfpathcurveto{\pgfqpoint{0.762586in}{1.163889in}}{\pgfqpoint{0.757000in}{1.166203in}}{\pgfqpoint{0.751176in}{1.166203in}}%
\pgfpathcurveto{\pgfqpoint{0.745352in}{1.166203in}}{\pgfqpoint{0.739766in}{1.163889in}}{\pgfqpoint{0.735648in}{1.159771in}}%
\pgfpathcurveto{\pgfqpoint{0.731530in}{1.155653in}}{\pgfqpoint{0.729216in}{1.150067in}}{\pgfqpoint{0.729216in}{1.144243in}}%
\pgfpathcurveto{\pgfqpoint{0.729216in}{1.138419in}}{\pgfqpoint{0.731530in}{1.132832in}}{\pgfqpoint{0.735648in}{1.128714in}}%
\pgfpathcurveto{\pgfqpoint{0.739766in}{1.124596in}}{\pgfqpoint{0.745352in}{1.122282in}}{\pgfqpoint{0.751176in}{1.122282in}}%
\pgfpathclose%
\pgfusepath{stroke,fill}%
\end{pgfscope}%
\begin{pgfscope}%
\pgfpathrectangle{\pgfqpoint{0.211875in}{0.211875in}}{\pgfqpoint{1.313625in}{1.279725in}}%
\pgfusepath{clip}%
\pgfsetbuttcap%
\pgfsetroundjoin%
\definecolor{currentfill}{rgb}{0.121569,0.466667,0.705882}%
\pgfsetfillcolor{currentfill}%
\pgfsetlinewidth{1.003750pt}%
\definecolor{currentstroke}{rgb}{0.121569,0.466667,0.705882}%
\pgfsetstrokecolor{currentstroke}%
\pgfsetdash{}{0pt}%
\pgfpathmoveto{\pgfqpoint{1.427891in}{0.898602in}}%
\pgfpathcurveto{\pgfqpoint{1.433715in}{0.898602in}}{\pgfqpoint{1.439301in}{0.900916in}}{\pgfqpoint{1.443419in}{0.905034in}}%
\pgfpathcurveto{\pgfqpoint{1.447537in}{0.909152in}}{\pgfqpoint{1.449851in}{0.914738in}}{\pgfqpoint{1.449851in}{0.920562in}}%
\pgfpathcurveto{\pgfqpoint{1.449851in}{0.926386in}}{\pgfqpoint{1.447537in}{0.931972in}}{\pgfqpoint{1.443419in}{0.936090in}}%
\pgfpathcurveto{\pgfqpoint{1.439301in}{0.940209in}}{\pgfqpoint{1.433715in}{0.942522in}}{\pgfqpoint{1.427891in}{0.942522in}}%
\pgfpathcurveto{\pgfqpoint{1.422067in}{0.942522in}}{\pgfqpoint{1.416481in}{0.940209in}}{\pgfqpoint{1.412363in}{0.936090in}}%
\pgfpathcurveto{\pgfqpoint{1.408245in}{0.931972in}}{\pgfqpoint{1.405931in}{0.926386in}}{\pgfqpoint{1.405931in}{0.920562in}}%
\pgfpathcurveto{\pgfqpoint{1.405931in}{0.914738in}}{\pgfqpoint{1.408245in}{0.909152in}}{\pgfqpoint{1.412363in}{0.905034in}}%
\pgfpathcurveto{\pgfqpoint{1.416481in}{0.900916in}}{\pgfqpoint{1.422067in}{0.898602in}}{\pgfqpoint{1.427891in}{0.898602in}}%
\pgfpathclose%
\pgfusepath{stroke,fill}%
\end{pgfscope}%
\begin{pgfscope}%
\pgfpathrectangle{\pgfqpoint{0.211875in}{0.211875in}}{\pgfqpoint{1.313625in}{1.279725in}}%
\pgfusepath{clip}%
\pgfsetbuttcap%
\pgfsetroundjoin%
\definecolor{currentfill}{rgb}{0.121569,0.466667,0.705882}%
\pgfsetfillcolor{currentfill}%
\pgfsetlinewidth{1.003750pt}%
\definecolor{currentstroke}{rgb}{0.121569,0.466667,0.705882}%
\pgfsetstrokecolor{currentstroke}%
\pgfsetdash{}{0pt}%
\pgfpathmoveto{\pgfqpoint{1.366304in}{0.732754in}}%
\pgfpathcurveto{\pgfqpoint{1.372128in}{0.732754in}}{\pgfqpoint{1.377714in}{0.735068in}}{\pgfqpoint{1.381833in}{0.739186in}}%
\pgfpathcurveto{\pgfqpoint{1.385951in}{0.743304in}}{\pgfqpoint{1.388265in}{0.748890in}}{\pgfqpoint{1.388265in}{0.754714in}}%
\pgfpathcurveto{\pgfqpoint{1.388265in}{0.760538in}}{\pgfqpoint{1.385951in}{0.766124in}}{\pgfqpoint{1.381833in}{0.770242in}}%
\pgfpathcurveto{\pgfqpoint{1.377714in}{0.774360in}}{\pgfqpoint{1.372128in}{0.776674in}}{\pgfqpoint{1.366304in}{0.776674in}}%
\pgfpathcurveto{\pgfqpoint{1.360480in}{0.776674in}}{\pgfqpoint{1.354894in}{0.774360in}}{\pgfqpoint{1.350776in}{0.770242in}}%
\pgfpathcurveto{\pgfqpoint{1.346658in}{0.766124in}}{\pgfqpoint{1.344344in}{0.760538in}}{\pgfqpoint{1.344344in}{0.754714in}}%
\pgfpathcurveto{\pgfqpoint{1.344344in}{0.748890in}}{\pgfqpoint{1.346658in}{0.743304in}}{\pgfqpoint{1.350776in}{0.739186in}}%
\pgfpathcurveto{\pgfqpoint{1.354894in}{0.735068in}}{\pgfqpoint{1.360480in}{0.732754in}}{\pgfqpoint{1.366304in}{0.732754in}}%
\pgfpathclose%
\pgfusepath{stroke,fill}%
\end{pgfscope}%
\begin{pgfscope}%
\pgfpathrectangle{\pgfqpoint{0.211875in}{0.211875in}}{\pgfqpoint{1.313625in}{1.279725in}}%
\pgfusepath{clip}%
\pgfsetbuttcap%
\pgfsetroundjoin%
\definecolor{currentfill}{rgb}{0.121569,0.466667,0.705882}%
\pgfsetfillcolor{currentfill}%
\pgfsetlinewidth{1.003750pt}%
\definecolor{currentstroke}{rgb}{0.121569,0.466667,0.705882}%
\pgfsetstrokecolor{currentstroke}%
\pgfsetdash{}{0pt}%
\pgfpathmoveto{\pgfqpoint{0.652597in}{0.753174in}}%
\pgfpathcurveto{\pgfqpoint{0.658421in}{0.753174in}}{\pgfqpoint{0.664007in}{0.755488in}}{\pgfqpoint{0.668125in}{0.759606in}}%
\pgfpathcurveto{\pgfqpoint{0.672243in}{0.763724in}}{\pgfqpoint{0.674557in}{0.769311in}}{\pgfqpoint{0.674557in}{0.775135in}}%
\pgfpathcurveto{\pgfqpoint{0.674557in}{0.780959in}}{\pgfqpoint{0.672243in}{0.786545in}}{\pgfqpoint{0.668125in}{0.790663in}}%
\pgfpathcurveto{\pgfqpoint{0.664007in}{0.794781in}}{\pgfqpoint{0.658421in}{0.797095in}}{\pgfqpoint{0.652597in}{0.797095in}}%
\pgfpathcurveto{\pgfqpoint{0.646773in}{0.797095in}}{\pgfqpoint{0.641187in}{0.794781in}}{\pgfqpoint{0.637068in}{0.790663in}}%
\pgfpathcurveto{\pgfqpoint{0.632950in}{0.786545in}}{\pgfqpoint{0.630636in}{0.780959in}}{\pgfqpoint{0.630636in}{0.775135in}}%
\pgfpathcurveto{\pgfqpoint{0.630636in}{0.769311in}}{\pgfqpoint{0.632950in}{0.763724in}}{\pgfqpoint{0.637068in}{0.759606in}}%
\pgfpathcurveto{\pgfqpoint{0.641187in}{0.755488in}}{\pgfqpoint{0.646773in}{0.753174in}}{\pgfqpoint{0.652597in}{0.753174in}}%
\pgfpathclose%
\pgfusepath{stroke,fill}%
\end{pgfscope}%
\begin{pgfscope}%
\pgfpathrectangle{\pgfqpoint{0.211875in}{0.211875in}}{\pgfqpoint{1.313625in}{1.279725in}}%
\pgfusepath{clip}%
\pgfsetbuttcap%
\pgfsetroundjoin%
\definecolor{currentfill}{rgb}{0.121569,0.466667,0.705882}%
\pgfsetfillcolor{currentfill}%
\pgfsetlinewidth{1.003750pt}%
\definecolor{currentstroke}{rgb}{0.121569,0.466667,0.705882}%
\pgfsetstrokecolor{currentstroke}%
\pgfsetdash{}{0pt}%
\pgfpathmoveto{\pgfqpoint{0.728164in}{0.714893in}}%
\pgfpathcurveto{\pgfqpoint{0.733988in}{0.714893in}}{\pgfqpoint{0.739574in}{0.717207in}}{\pgfqpoint{0.743692in}{0.721325in}}%
\pgfpathcurveto{\pgfqpoint{0.747810in}{0.725443in}}{\pgfqpoint{0.750124in}{0.731029in}}{\pgfqpoint{0.750124in}{0.736853in}}%
\pgfpathcurveto{\pgfqpoint{0.750124in}{0.742677in}}{\pgfqpoint{0.747810in}{0.748263in}}{\pgfqpoint{0.743692in}{0.752381in}}%
\pgfpathcurveto{\pgfqpoint{0.739574in}{0.756499in}}{\pgfqpoint{0.733988in}{0.758813in}}{\pgfqpoint{0.728164in}{0.758813in}}%
\pgfpathcurveto{\pgfqpoint{0.722340in}{0.758813in}}{\pgfqpoint{0.716754in}{0.756499in}}{\pgfqpoint{0.712635in}{0.752381in}}%
\pgfpathcurveto{\pgfqpoint{0.708517in}{0.748263in}}{\pgfqpoint{0.706203in}{0.742677in}}{\pgfqpoint{0.706203in}{0.736853in}}%
\pgfpathcurveto{\pgfqpoint{0.706203in}{0.731029in}}{\pgfqpoint{0.708517in}{0.725443in}}{\pgfqpoint{0.712635in}{0.721325in}}%
\pgfpathcurveto{\pgfqpoint{0.716754in}{0.717207in}}{\pgfqpoint{0.722340in}{0.714893in}}{\pgfqpoint{0.728164in}{0.714893in}}%
\pgfpathclose%
\pgfusepath{stroke,fill}%
\end{pgfscope}%
\begin{pgfscope}%
\pgfpathrectangle{\pgfqpoint{0.211875in}{0.211875in}}{\pgfqpoint{1.313625in}{1.279725in}}%
\pgfusepath{clip}%
\pgfsetbuttcap%
\pgfsetroundjoin%
\definecolor{currentfill}{rgb}{0.121569,0.466667,0.705882}%
\pgfsetfillcolor{currentfill}%
\pgfsetlinewidth{1.003750pt}%
\definecolor{currentstroke}{rgb}{0.121569,0.466667,0.705882}%
\pgfsetstrokecolor{currentstroke}%
\pgfsetdash{}{0pt}%
\pgfpathmoveto{\pgfqpoint{0.435278in}{0.917094in}}%
\pgfpathcurveto{\pgfqpoint{0.441102in}{0.917094in}}{\pgfqpoint{0.446688in}{0.919408in}}{\pgfqpoint{0.450806in}{0.923526in}}%
\pgfpathcurveto{\pgfqpoint{0.454924in}{0.927644in}}{\pgfqpoint{0.457238in}{0.933230in}}{\pgfqpoint{0.457238in}{0.939054in}}%
\pgfpathcurveto{\pgfqpoint{0.457238in}{0.944878in}}{\pgfqpoint{0.454924in}{0.950464in}}{\pgfqpoint{0.450806in}{0.954582in}}%
\pgfpathcurveto{\pgfqpoint{0.446688in}{0.958701in}}{\pgfqpoint{0.441102in}{0.961014in}}{\pgfqpoint{0.435278in}{0.961014in}}%
\pgfpathcurveto{\pgfqpoint{0.429454in}{0.961014in}}{\pgfqpoint{0.423868in}{0.958701in}}{\pgfqpoint{0.419750in}{0.954582in}}%
\pgfpathcurveto{\pgfqpoint{0.415632in}{0.950464in}}{\pgfqpoint{0.413318in}{0.944878in}}{\pgfqpoint{0.413318in}{0.939054in}}%
\pgfpathcurveto{\pgfqpoint{0.413318in}{0.933230in}}{\pgfqpoint{0.415632in}{0.927644in}}{\pgfqpoint{0.419750in}{0.923526in}}%
\pgfpathcurveto{\pgfqpoint{0.423868in}{0.919408in}}{\pgfqpoint{0.429454in}{0.917094in}}{\pgfqpoint{0.435278in}{0.917094in}}%
\pgfpathclose%
\pgfusepath{stroke,fill}%
\end{pgfscope}%
\begin{pgfscope}%
\pgfpathrectangle{\pgfqpoint{0.211875in}{0.211875in}}{\pgfqpoint{1.313625in}{1.279725in}}%
\pgfusepath{clip}%
\pgfsetbuttcap%
\pgfsetroundjoin%
\definecolor{currentfill}{rgb}{0.121569,0.466667,0.705882}%
\pgfsetfillcolor{currentfill}%
\pgfsetlinewidth{1.003750pt}%
\definecolor{currentstroke}{rgb}{0.121569,0.466667,0.705882}%
\pgfsetstrokecolor{currentstroke}%
\pgfsetdash{}{0pt}%
\pgfpathmoveto{\pgfqpoint{1.429971in}{0.679904in}}%
\pgfpathcurveto{\pgfqpoint{1.435794in}{0.679904in}}{\pgfqpoint{1.441381in}{0.682217in}}{\pgfqpoint{1.445499in}{0.686336in}}%
\pgfpathcurveto{\pgfqpoint{1.449617in}{0.690454in}}{\pgfqpoint{1.451931in}{0.696040in}}{\pgfqpoint{1.451931in}{0.701864in}}%
\pgfpathcurveto{\pgfqpoint{1.451931in}{0.707688in}}{\pgfqpoint{1.449617in}{0.713274in}}{\pgfqpoint{1.445499in}{0.717392in}}%
\pgfpathcurveto{\pgfqpoint{1.441381in}{0.721510in}}{\pgfqpoint{1.435794in}{0.723824in}}{\pgfqpoint{1.429971in}{0.723824in}}%
\pgfpathcurveto{\pgfqpoint{1.424147in}{0.723824in}}{\pgfqpoint{1.418560in}{0.721510in}}{\pgfqpoint{1.414442in}{0.717392in}}%
\pgfpathcurveto{\pgfqpoint{1.410324in}{0.713274in}}{\pgfqpoint{1.408010in}{0.707688in}}{\pgfqpoint{1.408010in}{0.701864in}}%
\pgfpathcurveto{\pgfqpoint{1.408010in}{0.696040in}}{\pgfqpoint{1.410324in}{0.690454in}}{\pgfqpoint{1.414442in}{0.686336in}}%
\pgfpathcurveto{\pgfqpoint{1.418560in}{0.682217in}}{\pgfqpoint{1.424147in}{0.679904in}}{\pgfqpoint{1.429971in}{0.679904in}}%
\pgfpathclose%
\pgfusepath{stroke,fill}%
\end{pgfscope}%
\begin{pgfscope}%
\pgfpathrectangle{\pgfqpoint{0.211875in}{0.211875in}}{\pgfqpoint{1.313625in}{1.279725in}}%
\pgfusepath{clip}%
\pgfsetbuttcap%
\pgfsetroundjoin%
\definecolor{currentfill}{rgb}{0.121569,0.466667,0.705882}%
\pgfsetfillcolor{currentfill}%
\pgfsetlinewidth{1.003750pt}%
\definecolor{currentstroke}{rgb}{0.121569,0.466667,0.705882}%
\pgfsetstrokecolor{currentstroke}%
\pgfsetdash{}{0pt}%
\pgfpathmoveto{\pgfqpoint{1.442621in}{0.733623in}}%
\pgfpathcurveto{\pgfqpoint{1.448445in}{0.733623in}}{\pgfqpoint{1.454031in}{0.735937in}}{\pgfqpoint{1.458149in}{0.740055in}}%
\pgfpathcurveto{\pgfqpoint{1.462267in}{0.744173in}}{\pgfqpoint{1.464581in}{0.749759in}}{\pgfqpoint{1.464581in}{0.755583in}}%
\pgfpathcurveto{\pgfqpoint{1.464581in}{0.761407in}}{\pgfqpoint{1.462267in}{0.766993in}}{\pgfqpoint{1.458149in}{0.771111in}}%
\pgfpathcurveto{\pgfqpoint{1.454031in}{0.775230in}}{\pgfqpoint{1.448445in}{0.777543in}}{\pgfqpoint{1.442621in}{0.777543in}}%
\pgfpathcurveto{\pgfqpoint{1.436797in}{0.777543in}}{\pgfqpoint{1.431211in}{0.775230in}}{\pgfqpoint{1.427093in}{0.771111in}}%
\pgfpathcurveto{\pgfqpoint{1.422974in}{0.766993in}}{\pgfqpoint{1.420661in}{0.761407in}}{\pgfqpoint{1.420661in}{0.755583in}}%
\pgfpathcurveto{\pgfqpoint{1.420661in}{0.749759in}}{\pgfqpoint{1.422974in}{0.744173in}}{\pgfqpoint{1.427093in}{0.740055in}}%
\pgfpathcurveto{\pgfqpoint{1.431211in}{0.735937in}}{\pgfqpoint{1.436797in}{0.733623in}}{\pgfqpoint{1.442621in}{0.733623in}}%
\pgfpathclose%
\pgfusepath{stroke,fill}%
\end{pgfscope}%
\begin{pgfscope}%
\pgfpathrectangle{\pgfqpoint{0.211875in}{0.211875in}}{\pgfqpoint{1.313625in}{1.279725in}}%
\pgfusepath{clip}%
\pgfsetbuttcap%
\pgfsetroundjoin%
\definecolor{currentfill}{rgb}{0.121569,0.466667,0.705882}%
\pgfsetfillcolor{currentfill}%
\pgfsetlinewidth{1.003750pt}%
\definecolor{currentstroke}{rgb}{0.121569,0.466667,0.705882}%
\pgfsetstrokecolor{currentstroke}%
\pgfsetdash{}{0pt}%
\pgfpathmoveto{\pgfqpoint{1.365741in}{0.661160in}}%
\pgfpathcurveto{\pgfqpoint{1.371565in}{0.661160in}}{\pgfqpoint{1.377151in}{0.663474in}}{\pgfqpoint{1.381269in}{0.667592in}}%
\pgfpathcurveto{\pgfqpoint{1.385387in}{0.671710in}}{\pgfqpoint{1.387701in}{0.677296in}}{\pgfqpoint{1.387701in}{0.683120in}}%
\pgfpathcurveto{\pgfqpoint{1.387701in}{0.688944in}}{\pgfqpoint{1.385387in}{0.694530in}}{\pgfqpoint{1.381269in}{0.698648in}}%
\pgfpathcurveto{\pgfqpoint{1.377151in}{0.702767in}}{\pgfqpoint{1.371565in}{0.705080in}}{\pgfqpoint{1.365741in}{0.705080in}}%
\pgfpathcurveto{\pgfqpoint{1.359917in}{0.705080in}}{\pgfqpoint{1.354331in}{0.702767in}}{\pgfqpoint{1.350213in}{0.698648in}}%
\pgfpathcurveto{\pgfqpoint{1.346094in}{0.694530in}}{\pgfqpoint{1.343781in}{0.688944in}}{\pgfqpoint{1.343781in}{0.683120in}}%
\pgfpathcurveto{\pgfqpoint{1.343781in}{0.677296in}}{\pgfqpoint{1.346094in}{0.671710in}}{\pgfqpoint{1.350213in}{0.667592in}}%
\pgfpathcurveto{\pgfqpoint{1.354331in}{0.663474in}}{\pgfqpoint{1.359917in}{0.661160in}}{\pgfqpoint{1.365741in}{0.661160in}}%
\pgfpathclose%
\pgfusepath{stroke,fill}%
\end{pgfscope}%
\begin{pgfscope}%
\pgfpathrectangle{\pgfqpoint{0.211875in}{0.211875in}}{\pgfqpoint{1.313625in}{1.279725in}}%
\pgfusepath{clip}%
\pgfsetbuttcap%
\pgfsetroundjoin%
\definecolor{currentfill}{rgb}{0.121569,0.466667,0.705882}%
\pgfsetfillcolor{currentfill}%
\pgfsetlinewidth{1.003750pt}%
\definecolor{currentstroke}{rgb}{0.121569,0.466667,0.705882}%
\pgfsetstrokecolor{currentstroke}%
\pgfsetdash{}{0pt}%
\pgfpathmoveto{\pgfqpoint{1.327489in}{0.759003in}}%
\pgfpathcurveto{\pgfqpoint{1.333313in}{0.759003in}}{\pgfqpoint{1.338899in}{0.761317in}}{\pgfqpoint{1.343017in}{0.765435in}}%
\pgfpathcurveto{\pgfqpoint{1.347135in}{0.769553in}}{\pgfqpoint{1.349449in}{0.775140in}}{\pgfqpoint{1.349449in}{0.780963in}}%
\pgfpathcurveto{\pgfqpoint{1.349449in}{0.786787in}}{\pgfqpoint{1.347135in}{0.792374in}}{\pgfqpoint{1.343017in}{0.796492in}}%
\pgfpathcurveto{\pgfqpoint{1.338899in}{0.800610in}}{\pgfqpoint{1.333313in}{0.802924in}}{\pgfqpoint{1.327489in}{0.802924in}}%
\pgfpathcurveto{\pgfqpoint{1.321665in}{0.802924in}}{\pgfqpoint{1.316079in}{0.800610in}}{\pgfqpoint{1.311961in}{0.796492in}}%
\pgfpathcurveto{\pgfqpoint{1.307843in}{0.792374in}}{\pgfqpoint{1.305529in}{0.786787in}}{\pgfqpoint{1.305529in}{0.780963in}}%
\pgfpathcurveto{\pgfqpoint{1.305529in}{0.775140in}}{\pgfqpoint{1.307843in}{0.769553in}}{\pgfqpoint{1.311961in}{0.765435in}}%
\pgfpathcurveto{\pgfqpoint{1.316079in}{0.761317in}}{\pgfqpoint{1.321665in}{0.759003in}}{\pgfqpoint{1.327489in}{0.759003in}}%
\pgfpathclose%
\pgfusepath{stroke,fill}%
\end{pgfscope}%
\begin{pgfscope}%
\pgfpathrectangle{\pgfqpoint{0.211875in}{0.211875in}}{\pgfqpoint{1.313625in}{1.279725in}}%
\pgfusepath{clip}%
\pgfsetbuttcap%
\pgfsetroundjoin%
\definecolor{currentfill}{rgb}{0.121569,0.466667,0.705882}%
\pgfsetfillcolor{currentfill}%
\pgfsetlinewidth{1.003750pt}%
\definecolor{currentstroke}{rgb}{0.121569,0.466667,0.705882}%
\pgfsetstrokecolor{currentstroke}%
\pgfsetdash{}{0pt}%
\pgfpathmoveto{\pgfqpoint{1.371773in}{0.718684in}}%
\pgfpathcurveto{\pgfqpoint{1.377597in}{0.718684in}}{\pgfqpoint{1.383183in}{0.720998in}}{\pgfqpoint{1.387301in}{0.725116in}}%
\pgfpathcurveto{\pgfqpoint{1.391419in}{0.729234in}}{\pgfqpoint{1.393733in}{0.734820in}}{\pgfqpoint{1.393733in}{0.740644in}}%
\pgfpathcurveto{\pgfqpoint{1.393733in}{0.746468in}}{\pgfqpoint{1.391419in}{0.752054in}}{\pgfqpoint{1.387301in}{0.756172in}}%
\pgfpathcurveto{\pgfqpoint{1.383183in}{0.760290in}}{\pgfqpoint{1.377597in}{0.762604in}}{\pgfqpoint{1.371773in}{0.762604in}}%
\pgfpathcurveto{\pgfqpoint{1.365949in}{0.762604in}}{\pgfqpoint{1.360363in}{0.760290in}}{\pgfqpoint{1.356245in}{0.756172in}}%
\pgfpathcurveto{\pgfqpoint{1.352127in}{0.752054in}}{\pgfqpoint{1.349813in}{0.746468in}}{\pgfqpoint{1.349813in}{0.740644in}}%
\pgfpathcurveto{\pgfqpoint{1.349813in}{0.734820in}}{\pgfqpoint{1.352127in}{0.729234in}}{\pgfqpoint{1.356245in}{0.725116in}}%
\pgfpathcurveto{\pgfqpoint{1.360363in}{0.720998in}}{\pgfqpoint{1.365949in}{0.718684in}}{\pgfqpoint{1.371773in}{0.718684in}}%
\pgfpathclose%
\pgfusepath{stroke,fill}%
\end{pgfscope}%
\begin{pgfscope}%
\pgfpathrectangle{\pgfqpoint{0.211875in}{0.211875in}}{\pgfqpoint{1.313625in}{1.279725in}}%
\pgfusepath{clip}%
\pgfsetbuttcap%
\pgfsetroundjoin%
\definecolor{currentfill}{rgb}{0.121569,0.466667,0.705882}%
\pgfsetfillcolor{currentfill}%
\pgfsetlinewidth{1.003750pt}%
\definecolor{currentstroke}{rgb}{0.121569,0.466667,0.705882}%
\pgfsetstrokecolor{currentstroke}%
\pgfsetdash{}{0pt}%
\pgfpathmoveto{\pgfqpoint{0.573733in}{0.904240in}}%
\pgfpathcurveto{\pgfqpoint{0.579557in}{0.904240in}}{\pgfqpoint{0.585143in}{0.906554in}}{\pgfqpoint{0.589261in}{0.910672in}}%
\pgfpathcurveto{\pgfqpoint{0.593379in}{0.914790in}}{\pgfqpoint{0.595693in}{0.920376in}}{\pgfqpoint{0.595693in}{0.926200in}}%
\pgfpathcurveto{\pgfqpoint{0.595693in}{0.932024in}}{\pgfqpoint{0.593379in}{0.937610in}}{\pgfqpoint{0.589261in}{0.941729in}}%
\pgfpathcurveto{\pgfqpoint{0.585143in}{0.945847in}}{\pgfqpoint{0.579557in}{0.948161in}}{\pgfqpoint{0.573733in}{0.948161in}}%
\pgfpathcurveto{\pgfqpoint{0.567909in}{0.948161in}}{\pgfqpoint{0.562323in}{0.945847in}}{\pgfqpoint{0.558205in}{0.941729in}}%
\pgfpathcurveto{\pgfqpoint{0.554086in}{0.937610in}}{\pgfqpoint{0.551773in}{0.932024in}}{\pgfqpoint{0.551773in}{0.926200in}}%
\pgfpathcurveto{\pgfqpoint{0.551773in}{0.920376in}}{\pgfqpoint{0.554086in}{0.914790in}}{\pgfqpoint{0.558205in}{0.910672in}}%
\pgfpathcurveto{\pgfqpoint{0.562323in}{0.906554in}}{\pgfqpoint{0.567909in}{0.904240in}}{\pgfqpoint{0.573733in}{0.904240in}}%
\pgfpathclose%
\pgfusepath{stroke,fill}%
\end{pgfscope}%
\begin{pgfscope}%
\pgfpathrectangle{\pgfqpoint{0.211875in}{0.211875in}}{\pgfqpoint{1.313625in}{1.279725in}}%
\pgfusepath{clip}%
\pgfsetbuttcap%
\pgfsetroundjoin%
\definecolor{currentfill}{rgb}{0.121569,0.466667,0.705882}%
\pgfsetfillcolor{currentfill}%
\pgfsetlinewidth{1.003750pt}%
\definecolor{currentstroke}{rgb}{0.121569,0.466667,0.705882}%
\pgfsetstrokecolor{currentstroke}%
\pgfsetdash{}{0pt}%
\pgfpathmoveto{\pgfqpoint{1.446306in}{0.806132in}}%
\pgfpathcurveto{\pgfqpoint{1.452130in}{0.806132in}}{\pgfqpoint{1.457716in}{0.808446in}}{\pgfqpoint{1.461834in}{0.812564in}}%
\pgfpathcurveto{\pgfqpoint{1.465952in}{0.816682in}}{\pgfqpoint{1.468266in}{0.822269in}}{\pgfqpoint{1.468266in}{0.828093in}}%
\pgfpathcurveto{\pgfqpoint{1.468266in}{0.833916in}}{\pgfqpoint{1.465952in}{0.839503in}}{\pgfqpoint{1.461834in}{0.843621in}}%
\pgfpathcurveto{\pgfqpoint{1.457716in}{0.847739in}}{\pgfqpoint{1.452130in}{0.850053in}}{\pgfqpoint{1.446306in}{0.850053in}}%
\pgfpathcurveto{\pgfqpoint{1.440482in}{0.850053in}}{\pgfqpoint{1.434895in}{0.847739in}}{\pgfqpoint{1.430777in}{0.843621in}}%
\pgfpathcurveto{\pgfqpoint{1.426659in}{0.839503in}}{\pgfqpoint{1.424345in}{0.833916in}}{\pgfqpoint{1.424345in}{0.828093in}}%
\pgfpathcurveto{\pgfqpoint{1.424345in}{0.822269in}}{\pgfqpoint{1.426659in}{0.816682in}}{\pgfqpoint{1.430777in}{0.812564in}}%
\pgfpathcurveto{\pgfqpoint{1.434895in}{0.808446in}}{\pgfqpoint{1.440482in}{0.806132in}}{\pgfqpoint{1.446306in}{0.806132in}}%
\pgfpathclose%
\pgfusepath{stroke,fill}%
\end{pgfscope}%
\begin{pgfscope}%
\pgfpathrectangle{\pgfqpoint{0.211875in}{0.211875in}}{\pgfqpoint{1.313625in}{1.279725in}}%
\pgfusepath{clip}%
\pgfsetbuttcap%
\pgfsetroundjoin%
\definecolor{currentfill}{rgb}{0.121569,0.466667,0.705882}%
\pgfsetfillcolor{currentfill}%
\pgfsetlinewidth{1.003750pt}%
\definecolor{currentstroke}{rgb}{0.121569,0.466667,0.705882}%
\pgfsetstrokecolor{currentstroke}%
\pgfsetdash{}{0pt}%
\pgfpathmoveto{\pgfqpoint{0.475343in}{0.603046in}}%
\pgfpathcurveto{\pgfqpoint{0.481167in}{0.603046in}}{\pgfqpoint{0.486754in}{0.605359in}}{\pgfqpoint{0.490872in}{0.609478in}}%
\pgfpathcurveto{\pgfqpoint{0.494990in}{0.613596in}}{\pgfqpoint{0.497304in}{0.619182in}}{\pgfqpoint{0.497304in}{0.625006in}}%
\pgfpathcurveto{\pgfqpoint{0.497304in}{0.630830in}}{\pgfqpoint{0.494990in}{0.636416in}}{\pgfqpoint{0.490872in}{0.640534in}}%
\pgfpathcurveto{\pgfqpoint{0.486754in}{0.644652in}}{\pgfqpoint{0.481167in}{0.646966in}}{\pgfqpoint{0.475343in}{0.646966in}}%
\pgfpathcurveto{\pgfqpoint{0.469519in}{0.646966in}}{\pgfqpoint{0.463933in}{0.644652in}}{\pgfqpoint{0.459815in}{0.640534in}}%
\pgfpathcurveto{\pgfqpoint{0.455697in}{0.636416in}}{\pgfqpoint{0.453383in}{0.630830in}}{\pgfqpoint{0.453383in}{0.625006in}}%
\pgfpathcurveto{\pgfqpoint{0.453383in}{0.619182in}}{\pgfqpoint{0.455697in}{0.613596in}}{\pgfqpoint{0.459815in}{0.609478in}}%
\pgfpathcurveto{\pgfqpoint{0.463933in}{0.605359in}}{\pgfqpoint{0.469519in}{0.603046in}}{\pgfqpoint{0.475343in}{0.603046in}}%
\pgfpathclose%
\pgfusepath{stroke,fill}%
\end{pgfscope}%
\begin{pgfscope}%
\pgfpathrectangle{\pgfqpoint{0.211875in}{0.211875in}}{\pgfqpoint{1.313625in}{1.279725in}}%
\pgfusepath{clip}%
\pgfsetbuttcap%
\pgfsetroundjoin%
\definecolor{currentfill}{rgb}{0.121569,0.466667,0.705882}%
\pgfsetfillcolor{currentfill}%
\pgfsetlinewidth{1.003750pt}%
\definecolor{currentstroke}{rgb}{0.121569,0.466667,0.705882}%
\pgfsetstrokecolor{currentstroke}%
\pgfsetdash{}{0pt}%
\pgfpathmoveto{\pgfqpoint{1.338611in}{0.683088in}}%
\pgfpathcurveto{\pgfqpoint{1.344435in}{0.683088in}}{\pgfqpoint{1.350022in}{0.685402in}}{\pgfqpoint{1.354140in}{0.689520in}}%
\pgfpathcurveto{\pgfqpoint{1.358258in}{0.693639in}}{\pgfqpoint{1.360572in}{0.699225in}}{\pgfqpoint{1.360572in}{0.705049in}}%
\pgfpathcurveto{\pgfqpoint{1.360572in}{0.710873in}}{\pgfqpoint{1.358258in}{0.716459in}}{\pgfqpoint{1.354140in}{0.720577in}}%
\pgfpathcurveto{\pgfqpoint{1.350022in}{0.724695in}}{\pgfqpoint{1.344435in}{0.727009in}}{\pgfqpoint{1.338611in}{0.727009in}}%
\pgfpathcurveto{\pgfqpoint{1.332788in}{0.727009in}}{\pgfqpoint{1.327201in}{0.724695in}}{\pgfqpoint{1.323083in}{0.720577in}}%
\pgfpathcurveto{\pgfqpoint{1.318965in}{0.716459in}}{\pgfqpoint{1.316651in}{0.710873in}}{\pgfqpoint{1.316651in}{0.705049in}}%
\pgfpathcurveto{\pgfqpoint{1.316651in}{0.699225in}}{\pgfqpoint{1.318965in}{0.693639in}}{\pgfqpoint{1.323083in}{0.689520in}}%
\pgfpathcurveto{\pgfqpoint{1.327201in}{0.685402in}}{\pgfqpoint{1.332788in}{0.683088in}}{\pgfqpoint{1.338611in}{0.683088in}}%
\pgfpathclose%
\pgfusepath{stroke,fill}%
\end{pgfscope}%
\begin{pgfscope}%
\pgfpathrectangle{\pgfqpoint{0.211875in}{0.211875in}}{\pgfqpoint{1.313625in}{1.279725in}}%
\pgfusepath{clip}%
\pgfsetbuttcap%
\pgfsetroundjoin%
\definecolor{currentfill}{rgb}{0.121569,0.466667,0.705882}%
\pgfsetfillcolor{currentfill}%
\pgfsetlinewidth{1.003750pt}%
\definecolor{currentstroke}{rgb}{0.121569,0.466667,0.705882}%
\pgfsetstrokecolor{currentstroke}%
\pgfsetdash{}{0pt}%
\pgfpathmoveto{\pgfqpoint{0.704757in}{0.640149in}}%
\pgfpathcurveto{\pgfqpoint{0.710581in}{0.640149in}}{\pgfqpoint{0.716167in}{0.642463in}}{\pgfqpoint{0.720285in}{0.646581in}}%
\pgfpathcurveto{\pgfqpoint{0.724403in}{0.650699in}}{\pgfqpoint{0.726717in}{0.656285in}}{\pgfqpoint{0.726717in}{0.662109in}}%
\pgfpathcurveto{\pgfqpoint{0.726717in}{0.667933in}}{\pgfqpoint{0.724403in}{0.673519in}}{\pgfqpoint{0.720285in}{0.677637in}}%
\pgfpathcurveto{\pgfqpoint{0.716167in}{0.681755in}}{\pgfqpoint{0.710581in}{0.684069in}}{\pgfqpoint{0.704757in}{0.684069in}}%
\pgfpathcurveto{\pgfqpoint{0.698933in}{0.684069in}}{\pgfqpoint{0.693347in}{0.681755in}}{\pgfqpoint{0.689229in}{0.677637in}}%
\pgfpathcurveto{\pgfqpoint{0.685111in}{0.673519in}}{\pgfqpoint{0.682797in}{0.667933in}}{\pgfqpoint{0.682797in}{0.662109in}}%
\pgfpathcurveto{\pgfqpoint{0.682797in}{0.656285in}}{\pgfqpoint{0.685111in}{0.650699in}}{\pgfqpoint{0.689229in}{0.646581in}}%
\pgfpathcurveto{\pgfqpoint{0.693347in}{0.642463in}}{\pgfqpoint{0.698933in}{0.640149in}}{\pgfqpoint{0.704757in}{0.640149in}}%
\pgfpathclose%
\pgfusepath{stroke,fill}%
\end{pgfscope}%
\begin{pgfscope}%
\pgfpathrectangle{\pgfqpoint{0.211875in}{0.211875in}}{\pgfqpoint{1.313625in}{1.279725in}}%
\pgfusepath{clip}%
\pgfsetbuttcap%
\pgfsetroundjoin%
\definecolor{currentfill}{rgb}{0.121569,0.466667,0.705882}%
\pgfsetfillcolor{currentfill}%
\pgfsetlinewidth{1.003750pt}%
\definecolor{currentstroke}{rgb}{0.121569,0.466667,0.705882}%
\pgfsetstrokecolor{currentstroke}%
\pgfsetdash{}{0pt}%
\pgfpathmoveto{\pgfqpoint{1.297289in}{1.205269in}}%
\pgfpathcurveto{\pgfqpoint{1.303113in}{1.205269in}}{\pgfqpoint{1.308699in}{1.207582in}}{\pgfqpoint{1.312817in}{1.211701in}}%
\pgfpathcurveto{\pgfqpoint{1.316935in}{1.215819in}}{\pgfqpoint{1.319249in}{1.221405in}}{\pgfqpoint{1.319249in}{1.227229in}}%
\pgfpathcurveto{\pgfqpoint{1.319249in}{1.233053in}}{\pgfqpoint{1.316935in}{1.238639in}}{\pgfqpoint{1.312817in}{1.242757in}}%
\pgfpathcurveto{\pgfqpoint{1.308699in}{1.246875in}}{\pgfqpoint{1.303113in}{1.249189in}}{\pgfqpoint{1.297289in}{1.249189in}}%
\pgfpathcurveto{\pgfqpoint{1.291465in}{1.249189in}}{\pgfqpoint{1.285879in}{1.246875in}}{\pgfqpoint{1.281761in}{1.242757in}}%
\pgfpathcurveto{\pgfqpoint{1.277643in}{1.238639in}}{\pgfqpoint{1.275329in}{1.233053in}}{\pgfqpoint{1.275329in}{1.227229in}}%
\pgfpathcurveto{\pgfqpoint{1.275329in}{1.221405in}}{\pgfqpoint{1.277643in}{1.215819in}}{\pgfqpoint{1.281761in}{1.211701in}}%
\pgfpathcurveto{\pgfqpoint{1.285879in}{1.207582in}}{\pgfqpoint{1.291465in}{1.205269in}}{\pgfqpoint{1.297289in}{1.205269in}}%
\pgfpathclose%
\pgfusepath{stroke,fill}%
\end{pgfscope}%
\begin{pgfscope}%
\pgfpathrectangle{\pgfqpoint{0.211875in}{0.211875in}}{\pgfqpoint{1.313625in}{1.279725in}}%
\pgfusepath{clip}%
\pgfsetbuttcap%
\pgfsetroundjoin%
\definecolor{currentfill}{rgb}{0.121569,0.466667,0.705882}%
\pgfsetfillcolor{currentfill}%
\pgfsetlinewidth{1.003750pt}%
\definecolor{currentstroke}{rgb}{0.121569,0.466667,0.705882}%
\pgfsetstrokecolor{currentstroke}%
\pgfsetdash{}{0pt}%
\pgfpathmoveto{\pgfqpoint{1.145786in}{0.655436in}}%
\pgfpathcurveto{\pgfqpoint{1.151610in}{0.655436in}}{\pgfqpoint{1.157197in}{0.657750in}}{\pgfqpoint{1.161315in}{0.661868in}}%
\pgfpathcurveto{\pgfqpoint{1.165433in}{0.665986in}}{\pgfqpoint{1.167747in}{0.671572in}}{\pgfqpoint{1.167747in}{0.677396in}}%
\pgfpathcurveto{\pgfqpoint{1.167747in}{0.683220in}}{\pgfqpoint{1.165433in}{0.688806in}}{\pgfqpoint{1.161315in}{0.692924in}}%
\pgfpathcurveto{\pgfqpoint{1.157197in}{0.697043in}}{\pgfqpoint{1.151610in}{0.699356in}}{\pgfqpoint{1.145786in}{0.699356in}}%
\pgfpathcurveto{\pgfqpoint{1.139963in}{0.699356in}}{\pgfqpoint{1.134376in}{0.697043in}}{\pgfqpoint{1.130258in}{0.692924in}}%
\pgfpathcurveto{\pgfqpoint{1.126140in}{0.688806in}}{\pgfqpoint{1.123826in}{0.683220in}}{\pgfqpoint{1.123826in}{0.677396in}}%
\pgfpathcurveto{\pgfqpoint{1.123826in}{0.671572in}}{\pgfqpoint{1.126140in}{0.665986in}}{\pgfqpoint{1.130258in}{0.661868in}}%
\pgfpathcurveto{\pgfqpoint{1.134376in}{0.657750in}}{\pgfqpoint{1.139963in}{0.655436in}}{\pgfqpoint{1.145786in}{0.655436in}}%
\pgfpathclose%
\pgfusepath{stroke,fill}%
\end{pgfscope}%
\begin{pgfscope}%
\pgfpathrectangle{\pgfqpoint{0.211875in}{0.211875in}}{\pgfqpoint{1.313625in}{1.279725in}}%
\pgfusepath{clip}%
\pgfsetbuttcap%
\pgfsetroundjoin%
\definecolor{currentfill}{rgb}{0.121569,0.466667,0.705882}%
\pgfsetfillcolor{currentfill}%
\pgfsetlinewidth{1.003750pt}%
\definecolor{currentstroke}{rgb}{0.121569,0.466667,0.705882}%
\pgfsetstrokecolor{currentstroke}%
\pgfsetdash{}{0pt}%
\pgfpathmoveto{\pgfqpoint{0.559098in}{0.761432in}}%
\pgfpathcurveto{\pgfqpoint{0.564922in}{0.761432in}}{\pgfqpoint{0.570508in}{0.763745in}}{\pgfqpoint{0.574626in}{0.767864in}}%
\pgfpathcurveto{\pgfqpoint{0.578745in}{0.771982in}}{\pgfqpoint{0.581059in}{0.777568in}}{\pgfqpoint{0.581059in}{0.783392in}}%
\pgfpathcurveto{\pgfqpoint{0.581059in}{0.789216in}}{\pgfqpoint{0.578745in}{0.794802in}}{\pgfqpoint{0.574626in}{0.798920in}}%
\pgfpathcurveto{\pgfqpoint{0.570508in}{0.803038in}}{\pgfqpoint{0.564922in}{0.805352in}}{\pgfqpoint{0.559098in}{0.805352in}}%
\pgfpathcurveto{\pgfqpoint{0.553274in}{0.805352in}}{\pgfqpoint{0.547688in}{0.803038in}}{\pgfqpoint{0.543570in}{0.798920in}}%
\pgfpathcurveto{\pgfqpoint{0.539452in}{0.794802in}}{\pgfqpoint{0.537138in}{0.789216in}}{\pgfqpoint{0.537138in}{0.783392in}}%
\pgfpathcurveto{\pgfqpoint{0.537138in}{0.777568in}}{\pgfqpoint{0.539452in}{0.771982in}}{\pgfqpoint{0.543570in}{0.767864in}}%
\pgfpathcurveto{\pgfqpoint{0.547688in}{0.763745in}}{\pgfqpoint{0.553274in}{0.761432in}}{\pgfqpoint{0.559098in}{0.761432in}}%
\pgfpathclose%
\pgfusepath{stroke,fill}%
\end{pgfscope}%
\begin{pgfscope}%
\pgfpathrectangle{\pgfqpoint{0.211875in}{0.211875in}}{\pgfqpoint{1.313625in}{1.279725in}}%
\pgfusepath{clip}%
\pgfsetbuttcap%
\pgfsetroundjoin%
\definecolor{currentfill}{rgb}{0.121569,0.466667,0.705882}%
\pgfsetfillcolor{currentfill}%
\pgfsetlinewidth{1.003750pt}%
\definecolor{currentstroke}{rgb}{0.121569,0.466667,0.705882}%
\pgfsetstrokecolor{currentstroke}%
\pgfsetdash{}{0pt}%
\pgfpathmoveto{\pgfqpoint{1.348816in}{0.697588in}}%
\pgfpathcurveto{\pgfqpoint{1.354640in}{0.697588in}}{\pgfqpoint{1.360227in}{0.699902in}}{\pgfqpoint{1.364345in}{0.704020in}}%
\pgfpathcurveto{\pgfqpoint{1.368463in}{0.708138in}}{\pgfqpoint{1.370777in}{0.713725in}}{\pgfqpoint{1.370777in}{0.719549in}}%
\pgfpathcurveto{\pgfqpoint{1.370777in}{0.725373in}}{\pgfqpoint{1.368463in}{0.730959in}}{\pgfqpoint{1.364345in}{0.735077in}}%
\pgfpathcurveto{\pgfqpoint{1.360227in}{0.739195in}}{\pgfqpoint{1.354640in}{0.741509in}}{\pgfqpoint{1.348816in}{0.741509in}}%
\pgfpathcurveto{\pgfqpoint{1.342993in}{0.741509in}}{\pgfqpoint{1.337406in}{0.739195in}}{\pgfqpoint{1.333288in}{0.735077in}}%
\pgfpathcurveto{\pgfqpoint{1.329170in}{0.730959in}}{\pgfqpoint{1.326856in}{0.725373in}}{\pgfqpoint{1.326856in}{0.719549in}}%
\pgfpathcurveto{\pgfqpoint{1.326856in}{0.713725in}}{\pgfqpoint{1.329170in}{0.708138in}}{\pgfqpoint{1.333288in}{0.704020in}}%
\pgfpathcurveto{\pgfqpoint{1.337406in}{0.699902in}}{\pgfqpoint{1.342993in}{0.697588in}}{\pgfqpoint{1.348816in}{0.697588in}}%
\pgfpathclose%
\pgfusepath{stroke,fill}%
\end{pgfscope}%
\begin{pgfscope}%
\pgfpathrectangle{\pgfqpoint{0.211875in}{0.211875in}}{\pgfqpoint{1.313625in}{1.279725in}}%
\pgfusepath{clip}%
\pgfsetbuttcap%
\pgfsetroundjoin%
\definecolor{currentfill}{rgb}{0.121569,0.466667,0.705882}%
\pgfsetfillcolor{currentfill}%
\pgfsetlinewidth{1.003750pt}%
\definecolor{currentstroke}{rgb}{0.121569,0.466667,0.705882}%
\pgfsetstrokecolor{currentstroke}%
\pgfsetdash{}{0pt}%
\pgfpathmoveto{\pgfqpoint{1.205275in}{1.208958in}}%
\pgfpathcurveto{\pgfqpoint{1.211099in}{1.208958in}}{\pgfqpoint{1.216686in}{1.211271in}}{\pgfqpoint{1.220804in}{1.215390in}}%
\pgfpathcurveto{\pgfqpoint{1.224922in}{1.219508in}}{\pgfqpoint{1.227236in}{1.225094in}}{\pgfqpoint{1.227236in}{1.230918in}}%
\pgfpathcurveto{\pgfqpoint{1.227236in}{1.236742in}}{\pgfqpoint{1.224922in}{1.242328in}}{\pgfqpoint{1.220804in}{1.246446in}}%
\pgfpathcurveto{\pgfqpoint{1.216686in}{1.250564in}}{\pgfqpoint{1.211099in}{1.252878in}}{\pgfqpoint{1.205275in}{1.252878in}}%
\pgfpathcurveto{\pgfqpoint{1.199452in}{1.252878in}}{\pgfqpoint{1.193865in}{1.250564in}}{\pgfqpoint{1.189747in}{1.246446in}}%
\pgfpathcurveto{\pgfqpoint{1.185629in}{1.242328in}}{\pgfqpoint{1.183315in}{1.236742in}}{\pgfqpoint{1.183315in}{1.230918in}}%
\pgfpathcurveto{\pgfqpoint{1.183315in}{1.225094in}}{\pgfqpoint{1.185629in}{1.219508in}}{\pgfqpoint{1.189747in}{1.215390in}}%
\pgfpathcurveto{\pgfqpoint{1.193865in}{1.211271in}}{\pgfqpoint{1.199452in}{1.208958in}}{\pgfqpoint{1.205275in}{1.208958in}}%
\pgfpathclose%
\pgfusepath{stroke,fill}%
\end{pgfscope}%
\begin{pgfscope}%
\pgfpathrectangle{\pgfqpoint{0.211875in}{0.211875in}}{\pgfqpoint{1.313625in}{1.279725in}}%
\pgfusepath{clip}%
\pgfsetbuttcap%
\pgfsetroundjoin%
\definecolor{currentfill}{rgb}{0.121569,0.466667,0.705882}%
\pgfsetfillcolor{currentfill}%
\pgfsetlinewidth{1.003750pt}%
\definecolor{currentstroke}{rgb}{0.121569,0.466667,0.705882}%
\pgfsetstrokecolor{currentstroke}%
\pgfsetdash{}{0pt}%
\pgfpathmoveto{\pgfqpoint{1.219949in}{0.681565in}}%
\pgfpathcurveto{\pgfqpoint{1.225773in}{0.681565in}}{\pgfqpoint{1.231359in}{0.683879in}}{\pgfqpoint{1.235478in}{0.687997in}}%
\pgfpathcurveto{\pgfqpoint{1.239596in}{0.692116in}}{\pgfqpoint{1.241910in}{0.697702in}}{\pgfqpoint{1.241910in}{0.703526in}}%
\pgfpathcurveto{\pgfqpoint{1.241910in}{0.709350in}}{\pgfqpoint{1.239596in}{0.714936in}}{\pgfqpoint{1.235478in}{0.719054in}}%
\pgfpathcurveto{\pgfqpoint{1.231359in}{0.723172in}}{\pgfqpoint{1.225773in}{0.725486in}}{\pgfqpoint{1.219949in}{0.725486in}}%
\pgfpathcurveto{\pgfqpoint{1.214125in}{0.725486in}}{\pgfqpoint{1.208539in}{0.723172in}}{\pgfqpoint{1.204421in}{0.719054in}}%
\pgfpathcurveto{\pgfqpoint{1.200303in}{0.714936in}}{\pgfqpoint{1.197989in}{0.709350in}}{\pgfqpoint{1.197989in}{0.703526in}}%
\pgfpathcurveto{\pgfqpoint{1.197989in}{0.697702in}}{\pgfqpoint{1.200303in}{0.692116in}}{\pgfqpoint{1.204421in}{0.687997in}}%
\pgfpathcurveto{\pgfqpoint{1.208539in}{0.683879in}}{\pgfqpoint{1.214125in}{0.681565in}}{\pgfqpoint{1.219949in}{0.681565in}}%
\pgfpathclose%
\pgfusepath{stroke,fill}%
\end{pgfscope}%
\begin{pgfscope}%
\pgfpathrectangle{\pgfqpoint{0.211875in}{0.211875in}}{\pgfqpoint{1.313625in}{1.279725in}}%
\pgfusepath{clip}%
\pgfsetbuttcap%
\pgfsetroundjoin%
\definecolor{currentfill}{rgb}{0.121569,0.466667,0.705882}%
\pgfsetfillcolor{currentfill}%
\pgfsetlinewidth{1.003750pt}%
\definecolor{currentstroke}{rgb}{0.121569,0.466667,0.705882}%
\pgfsetstrokecolor{currentstroke}%
\pgfsetdash{}{0pt}%
\pgfpathmoveto{\pgfqpoint{1.311968in}{0.669110in}}%
\pgfpathcurveto{\pgfqpoint{1.317792in}{0.669110in}}{\pgfqpoint{1.323378in}{0.671424in}}{\pgfqpoint{1.327496in}{0.675542in}}%
\pgfpathcurveto{\pgfqpoint{1.331614in}{0.679660in}}{\pgfqpoint{1.333928in}{0.685246in}}{\pgfqpoint{1.333928in}{0.691070in}}%
\pgfpathcurveto{\pgfqpoint{1.333928in}{0.696894in}}{\pgfqpoint{1.331614in}{0.702481in}}{\pgfqpoint{1.327496in}{0.706599in}}%
\pgfpathcurveto{\pgfqpoint{1.323378in}{0.710717in}}{\pgfqpoint{1.317792in}{0.713031in}}{\pgfqpoint{1.311968in}{0.713031in}}%
\pgfpathcurveto{\pgfqpoint{1.306144in}{0.713031in}}{\pgfqpoint{1.300558in}{0.710717in}}{\pgfqpoint{1.296440in}{0.706599in}}%
\pgfpathcurveto{\pgfqpoint{1.292322in}{0.702481in}}{\pgfqpoint{1.290008in}{0.696894in}}{\pgfqpoint{1.290008in}{0.691070in}}%
\pgfpathcurveto{\pgfqpoint{1.290008in}{0.685246in}}{\pgfqpoint{1.292322in}{0.679660in}}{\pgfqpoint{1.296440in}{0.675542in}}%
\pgfpathcurveto{\pgfqpoint{1.300558in}{0.671424in}}{\pgfqpoint{1.306144in}{0.669110in}}{\pgfqpoint{1.311968in}{0.669110in}}%
\pgfpathclose%
\pgfusepath{stroke,fill}%
\end{pgfscope}%
\begin{pgfscope}%
\pgfpathrectangle{\pgfqpoint{0.211875in}{0.211875in}}{\pgfqpoint{1.313625in}{1.279725in}}%
\pgfusepath{clip}%
\pgfsetbuttcap%
\pgfsetroundjoin%
\definecolor{currentfill}{rgb}{0.121569,0.466667,0.705882}%
\pgfsetfillcolor{currentfill}%
\pgfsetlinewidth{1.003750pt}%
\definecolor{currentstroke}{rgb}{0.121569,0.466667,0.705882}%
\pgfsetstrokecolor{currentstroke}%
\pgfsetdash{}{0pt}%
\pgfpathmoveto{\pgfqpoint{0.290970in}{1.002290in}}%
\pgfpathcurveto{\pgfqpoint{0.296794in}{1.002290in}}{\pgfqpoint{0.302380in}{1.004603in}}{\pgfqpoint{0.306498in}{1.008722in}}%
\pgfpathcurveto{\pgfqpoint{0.310616in}{1.012840in}}{\pgfqpoint{0.312930in}{1.018426in}}{\pgfqpoint{0.312930in}{1.024250in}}%
\pgfpathcurveto{\pgfqpoint{0.312930in}{1.030074in}}{\pgfqpoint{0.310616in}{1.035660in}}{\pgfqpoint{0.306498in}{1.039778in}}%
\pgfpathcurveto{\pgfqpoint{0.302380in}{1.043896in}}{\pgfqpoint{0.296794in}{1.046210in}}{\pgfqpoint{0.290970in}{1.046210in}}%
\pgfpathcurveto{\pgfqpoint{0.285146in}{1.046210in}}{\pgfqpoint{0.279560in}{1.043896in}}{\pgfqpoint{0.275442in}{1.039778in}}%
\pgfpathcurveto{\pgfqpoint{0.271323in}{1.035660in}}{\pgfqpoint{0.269010in}{1.030074in}}{\pgfqpoint{0.269010in}{1.024250in}}%
\pgfpathcurveto{\pgfqpoint{0.269010in}{1.018426in}}{\pgfqpoint{0.271323in}{1.012840in}}{\pgfqpoint{0.275442in}{1.008722in}}%
\pgfpathcurveto{\pgfqpoint{0.279560in}{1.004603in}}{\pgfqpoint{0.285146in}{1.002290in}}{\pgfqpoint{0.290970in}{1.002290in}}%
\pgfpathclose%
\pgfusepath{stroke,fill}%
\end{pgfscope}%
\begin{pgfscope}%
\pgfpathrectangle{\pgfqpoint{0.211875in}{0.211875in}}{\pgfqpoint{1.313625in}{1.279725in}}%
\pgfusepath{clip}%
\pgfsetbuttcap%
\pgfsetroundjoin%
\definecolor{currentfill}{rgb}{0.121569,0.466667,0.705882}%
\pgfsetfillcolor{currentfill}%
\pgfsetlinewidth{1.003750pt}%
\definecolor{currentstroke}{rgb}{0.121569,0.466667,0.705882}%
\pgfsetstrokecolor{currentstroke}%
\pgfsetdash{}{0pt}%
\pgfpathmoveto{\pgfqpoint{1.375908in}{0.838505in}}%
\pgfpathcurveto{\pgfqpoint{1.381732in}{0.838505in}}{\pgfqpoint{1.387319in}{0.840819in}}{\pgfqpoint{1.391437in}{0.844937in}}%
\pgfpathcurveto{\pgfqpoint{1.395555in}{0.849055in}}{\pgfqpoint{1.397869in}{0.854641in}}{\pgfqpoint{1.397869in}{0.860465in}}%
\pgfpathcurveto{\pgfqpoint{1.397869in}{0.866289in}}{\pgfqpoint{1.395555in}{0.871875in}}{\pgfqpoint{1.391437in}{0.875994in}}%
\pgfpathcurveto{\pgfqpoint{1.387319in}{0.880112in}}{\pgfqpoint{1.381732in}{0.882426in}}{\pgfqpoint{1.375908in}{0.882426in}}%
\pgfpathcurveto{\pgfqpoint{1.370085in}{0.882426in}}{\pgfqpoint{1.364498in}{0.880112in}}{\pgfqpoint{1.360380in}{0.875994in}}%
\pgfpathcurveto{\pgfqpoint{1.356262in}{0.871875in}}{\pgfqpoint{1.353948in}{0.866289in}}{\pgfqpoint{1.353948in}{0.860465in}}%
\pgfpathcurveto{\pgfqpoint{1.353948in}{0.854641in}}{\pgfqpoint{1.356262in}{0.849055in}}{\pgfqpoint{1.360380in}{0.844937in}}%
\pgfpathcurveto{\pgfqpoint{1.364498in}{0.840819in}}{\pgfqpoint{1.370085in}{0.838505in}}{\pgfqpoint{1.375908in}{0.838505in}}%
\pgfpathclose%
\pgfusepath{stroke,fill}%
\end{pgfscope}%
\begin{pgfscope}%
\pgfpathrectangle{\pgfqpoint{0.211875in}{0.211875in}}{\pgfqpoint{1.313625in}{1.279725in}}%
\pgfusepath{clip}%
\pgfsetbuttcap%
\pgfsetroundjoin%
\definecolor{currentfill}{rgb}{0.121569,0.466667,0.705882}%
\pgfsetfillcolor{currentfill}%
\pgfsetlinewidth{1.003750pt}%
\definecolor{currentstroke}{rgb}{0.121569,0.466667,0.705882}%
\pgfsetstrokecolor{currentstroke}%
\pgfsetdash{}{0pt}%
\pgfpathmoveto{\pgfqpoint{1.231458in}{0.680977in}}%
\pgfpathcurveto{\pgfqpoint{1.237282in}{0.680977in}}{\pgfqpoint{1.242869in}{0.683291in}}{\pgfqpoint{1.246987in}{0.687409in}}%
\pgfpathcurveto{\pgfqpoint{1.251105in}{0.691527in}}{\pgfqpoint{1.253419in}{0.697113in}}{\pgfqpoint{1.253419in}{0.702937in}}%
\pgfpathcurveto{\pgfqpoint{1.253419in}{0.708761in}}{\pgfqpoint{1.251105in}{0.714348in}}{\pgfqpoint{1.246987in}{0.718466in}}%
\pgfpathcurveto{\pgfqpoint{1.242869in}{0.722584in}}{\pgfqpoint{1.237282in}{0.724898in}}{\pgfqpoint{1.231458in}{0.724898in}}%
\pgfpathcurveto{\pgfqpoint{1.225635in}{0.724898in}}{\pgfqpoint{1.220048in}{0.722584in}}{\pgfqpoint{1.215930in}{0.718466in}}%
\pgfpathcurveto{\pgfqpoint{1.211812in}{0.714348in}}{\pgfqpoint{1.209498in}{0.708761in}}{\pgfqpoint{1.209498in}{0.702937in}}%
\pgfpathcurveto{\pgfqpoint{1.209498in}{0.697113in}}{\pgfqpoint{1.211812in}{0.691527in}}{\pgfqpoint{1.215930in}{0.687409in}}%
\pgfpathcurveto{\pgfqpoint{1.220048in}{0.683291in}}{\pgfqpoint{1.225635in}{0.680977in}}{\pgfqpoint{1.231458in}{0.680977in}}%
\pgfpathclose%
\pgfusepath{stroke,fill}%
\end{pgfscope}%
\begin{pgfscope}%
\pgfpathrectangle{\pgfqpoint{0.211875in}{0.211875in}}{\pgfqpoint{1.313625in}{1.279725in}}%
\pgfusepath{clip}%
\pgfsetbuttcap%
\pgfsetroundjoin%
\definecolor{currentfill}{rgb}{0.121569,0.466667,0.705882}%
\pgfsetfillcolor{currentfill}%
\pgfsetlinewidth{1.003750pt}%
\definecolor{currentstroke}{rgb}{0.121569,0.466667,0.705882}%
\pgfsetstrokecolor{currentstroke}%
\pgfsetdash{}{0pt}%
\pgfpathmoveto{\pgfqpoint{1.429039in}{0.810089in}}%
\pgfpathcurveto{\pgfqpoint{1.434863in}{0.810089in}}{\pgfqpoint{1.440449in}{0.812402in}}{\pgfqpoint{1.444567in}{0.816521in}}%
\pgfpathcurveto{\pgfqpoint{1.448685in}{0.820639in}}{\pgfqpoint{1.450999in}{0.826225in}}{\pgfqpoint{1.450999in}{0.832049in}}%
\pgfpathcurveto{\pgfqpoint{1.450999in}{0.837873in}}{\pgfqpoint{1.448685in}{0.843459in}}{\pgfqpoint{1.444567in}{0.847577in}}%
\pgfpathcurveto{\pgfqpoint{1.440449in}{0.851695in}}{\pgfqpoint{1.434863in}{0.854009in}}{\pgfqpoint{1.429039in}{0.854009in}}%
\pgfpathcurveto{\pgfqpoint{1.423215in}{0.854009in}}{\pgfqpoint{1.417629in}{0.851695in}}{\pgfqpoint{1.413511in}{0.847577in}}%
\pgfpathcurveto{\pgfqpoint{1.409392in}{0.843459in}}{\pgfqpoint{1.407079in}{0.837873in}}{\pgfqpoint{1.407079in}{0.832049in}}%
\pgfpathcurveto{\pgfqpoint{1.407079in}{0.826225in}}{\pgfqpoint{1.409392in}{0.820639in}}{\pgfqpoint{1.413511in}{0.816521in}}%
\pgfpathcurveto{\pgfqpoint{1.417629in}{0.812402in}}{\pgfqpoint{1.423215in}{0.810089in}}{\pgfqpoint{1.429039in}{0.810089in}}%
\pgfpathclose%
\pgfusepath{stroke,fill}%
\end{pgfscope}%
\begin{pgfscope}%
\pgfpathrectangle{\pgfqpoint{0.211875in}{0.211875in}}{\pgfqpoint{1.313625in}{1.279725in}}%
\pgfusepath{clip}%
\pgfsetbuttcap%
\pgfsetroundjoin%
\definecolor{currentfill}{rgb}{0.121569,0.466667,0.705882}%
\pgfsetfillcolor{currentfill}%
\pgfsetlinewidth{1.003750pt}%
\definecolor{currentstroke}{rgb}{0.121569,0.466667,0.705882}%
\pgfsetstrokecolor{currentstroke}%
\pgfsetdash{}{0pt}%
\pgfpathmoveto{\pgfqpoint{1.197776in}{0.688859in}}%
\pgfpathcurveto{\pgfqpoint{1.203600in}{0.688859in}}{\pgfqpoint{1.209186in}{0.691173in}}{\pgfqpoint{1.213304in}{0.695291in}}%
\pgfpathcurveto{\pgfqpoint{1.217422in}{0.699409in}}{\pgfqpoint{1.219736in}{0.704995in}}{\pgfqpoint{1.219736in}{0.710819in}}%
\pgfpathcurveto{\pgfqpoint{1.219736in}{0.716643in}}{\pgfqpoint{1.217422in}{0.722229in}}{\pgfqpoint{1.213304in}{0.726347in}}%
\pgfpathcurveto{\pgfqpoint{1.209186in}{0.730466in}}{\pgfqpoint{1.203600in}{0.732779in}}{\pgfqpoint{1.197776in}{0.732779in}}%
\pgfpathcurveto{\pgfqpoint{1.191952in}{0.732779in}}{\pgfqpoint{1.186366in}{0.730466in}}{\pgfqpoint{1.182248in}{0.726347in}}%
\pgfpathcurveto{\pgfqpoint{1.178130in}{0.722229in}}{\pgfqpoint{1.175816in}{0.716643in}}{\pgfqpoint{1.175816in}{0.710819in}}%
\pgfpathcurveto{\pgfqpoint{1.175816in}{0.704995in}}{\pgfqpoint{1.178130in}{0.699409in}}{\pgfqpoint{1.182248in}{0.695291in}}%
\pgfpathcurveto{\pgfqpoint{1.186366in}{0.691173in}}{\pgfqpoint{1.191952in}{0.688859in}}{\pgfqpoint{1.197776in}{0.688859in}}%
\pgfpathclose%
\pgfusepath{stroke,fill}%
\end{pgfscope}%
\begin{pgfscope}%
\pgfpathrectangle{\pgfqpoint{0.211875in}{0.211875in}}{\pgfqpoint{1.313625in}{1.279725in}}%
\pgfusepath{clip}%
\pgfsetbuttcap%
\pgfsetroundjoin%
\definecolor{currentfill}{rgb}{0.121569,0.466667,0.705882}%
\pgfsetfillcolor{currentfill}%
\pgfsetlinewidth{1.003750pt}%
\definecolor{currentstroke}{rgb}{0.121569,0.466667,0.705882}%
\pgfsetstrokecolor{currentstroke}%
\pgfsetdash{}{0pt}%
\pgfpathmoveto{\pgfqpoint{1.216280in}{0.671347in}}%
\pgfpathcurveto{\pgfqpoint{1.222104in}{0.671347in}}{\pgfqpoint{1.227690in}{0.673661in}}{\pgfqpoint{1.231808in}{0.677779in}}%
\pgfpathcurveto{\pgfqpoint{1.235926in}{0.681897in}}{\pgfqpoint{1.238240in}{0.687484in}}{\pgfqpoint{1.238240in}{0.693308in}}%
\pgfpathcurveto{\pgfqpoint{1.238240in}{0.699131in}}{\pgfqpoint{1.235926in}{0.704718in}}{\pgfqpoint{1.231808in}{0.708836in}}%
\pgfpathcurveto{\pgfqpoint{1.227690in}{0.712954in}}{\pgfqpoint{1.222104in}{0.715268in}}{\pgfqpoint{1.216280in}{0.715268in}}%
\pgfpathcurveto{\pgfqpoint{1.210456in}{0.715268in}}{\pgfqpoint{1.204870in}{0.712954in}}{\pgfqpoint{1.200752in}{0.708836in}}%
\pgfpathcurveto{\pgfqpoint{1.196633in}{0.704718in}}{\pgfqpoint{1.194320in}{0.699131in}}{\pgfqpoint{1.194320in}{0.693308in}}%
\pgfpathcurveto{\pgfqpoint{1.194320in}{0.687484in}}{\pgfqpoint{1.196633in}{0.681897in}}{\pgfqpoint{1.200752in}{0.677779in}}%
\pgfpathcurveto{\pgfqpoint{1.204870in}{0.673661in}}{\pgfqpoint{1.210456in}{0.671347in}}{\pgfqpoint{1.216280in}{0.671347in}}%
\pgfpathclose%
\pgfusepath{stroke,fill}%
\end{pgfscope}%
\begin{pgfscope}%
\pgfpathrectangle{\pgfqpoint{0.211875in}{0.211875in}}{\pgfqpoint{1.313625in}{1.279725in}}%
\pgfusepath{clip}%
\pgfsetbuttcap%
\pgfsetroundjoin%
\definecolor{currentfill}{rgb}{0.121569,0.466667,0.705882}%
\pgfsetfillcolor{currentfill}%
\pgfsetlinewidth{1.003750pt}%
\definecolor{currentstroke}{rgb}{0.121569,0.466667,0.705882}%
\pgfsetstrokecolor{currentstroke}%
\pgfsetdash{}{0pt}%
\pgfpathmoveto{\pgfqpoint{1.269417in}{1.375915in}}%
\pgfpathcurveto{\pgfqpoint{1.275241in}{1.375915in}}{\pgfqpoint{1.280827in}{1.378229in}}{\pgfqpoint{1.284945in}{1.382347in}}%
\pgfpathcurveto{\pgfqpoint{1.289063in}{1.386465in}}{\pgfqpoint{1.291377in}{1.392051in}}{\pgfqpoint{1.291377in}{1.397875in}}%
\pgfpathcurveto{\pgfqpoint{1.291377in}{1.403699in}}{\pgfqpoint{1.289063in}{1.409285in}}{\pgfqpoint{1.284945in}{1.413403in}}%
\pgfpathcurveto{\pgfqpoint{1.280827in}{1.417522in}}{\pgfqpoint{1.275241in}{1.419835in}}{\pgfqpoint{1.269417in}{1.419835in}}%
\pgfpathcurveto{\pgfqpoint{1.263593in}{1.419835in}}{\pgfqpoint{1.258007in}{1.417522in}}{\pgfqpoint{1.253889in}{1.413403in}}%
\pgfpathcurveto{\pgfqpoint{1.249771in}{1.409285in}}{\pgfqpoint{1.247457in}{1.403699in}}{\pgfqpoint{1.247457in}{1.397875in}}%
\pgfpathcurveto{\pgfqpoint{1.247457in}{1.392051in}}{\pgfqpoint{1.249771in}{1.386465in}}{\pgfqpoint{1.253889in}{1.382347in}}%
\pgfpathcurveto{\pgfqpoint{1.258007in}{1.378229in}}{\pgfqpoint{1.263593in}{1.375915in}}{\pgfqpoint{1.269417in}{1.375915in}}%
\pgfpathclose%
\pgfusepath{stroke,fill}%
\end{pgfscope}%
\begin{pgfscope}%
\pgfpathrectangle{\pgfqpoint{0.211875in}{0.211875in}}{\pgfqpoint{1.313625in}{1.279725in}}%
\pgfusepath{clip}%
\pgfsetbuttcap%
\pgfsetroundjoin%
\definecolor{currentfill}{rgb}{0.121569,0.466667,0.705882}%
\pgfsetfillcolor{currentfill}%
\pgfsetlinewidth{1.003750pt}%
\definecolor{currentstroke}{rgb}{0.121569,0.466667,0.705882}%
\pgfsetstrokecolor{currentstroke}%
\pgfsetdash{}{0pt}%
\pgfpathmoveto{\pgfqpoint{1.273517in}{1.425853in}}%
\pgfpathcurveto{\pgfqpoint{1.279340in}{1.425853in}}{\pgfqpoint{1.284927in}{1.428167in}}{\pgfqpoint{1.289045in}{1.432285in}}%
\pgfpathcurveto{\pgfqpoint{1.293163in}{1.436403in}}{\pgfqpoint{1.295477in}{1.441989in}}{\pgfqpoint{1.295477in}{1.447813in}}%
\pgfpathcurveto{\pgfqpoint{1.295477in}{1.453637in}}{\pgfqpoint{1.293163in}{1.459223in}}{\pgfqpoint{1.289045in}{1.463341in}}%
\pgfpathcurveto{\pgfqpoint{1.284927in}{1.467460in}}{\pgfqpoint{1.279340in}{1.469773in}}{\pgfqpoint{1.273517in}{1.469773in}}%
\pgfpathcurveto{\pgfqpoint{1.267693in}{1.469773in}}{\pgfqpoint{1.262106in}{1.467460in}}{\pgfqpoint{1.257988in}{1.463341in}}%
\pgfpathcurveto{\pgfqpoint{1.253870in}{1.459223in}}{\pgfqpoint{1.251556in}{1.453637in}}{\pgfqpoint{1.251556in}{1.447813in}}%
\pgfpathcurveto{\pgfqpoint{1.251556in}{1.441989in}}{\pgfqpoint{1.253870in}{1.436403in}}{\pgfqpoint{1.257988in}{1.432285in}}%
\pgfpathcurveto{\pgfqpoint{1.262106in}{1.428167in}}{\pgfqpoint{1.267693in}{1.425853in}}{\pgfqpoint{1.273517in}{1.425853in}}%
\pgfpathclose%
\pgfusepath{stroke,fill}%
\end{pgfscope}%
\begin{pgfscope}%
\pgfpathrectangle{\pgfqpoint{0.211875in}{0.211875in}}{\pgfqpoint{1.313625in}{1.279725in}}%
\pgfusepath{clip}%
\pgfsetbuttcap%
\pgfsetroundjoin%
\definecolor{currentfill}{rgb}{0.121569,0.466667,0.705882}%
\pgfsetfillcolor{currentfill}%
\pgfsetlinewidth{1.003750pt}%
\definecolor{currentstroke}{rgb}{0.121569,0.466667,0.705882}%
\pgfsetstrokecolor{currentstroke}%
\pgfsetdash{}{0pt}%
\pgfpathmoveto{\pgfqpoint{0.620854in}{0.752749in}}%
\pgfpathcurveto{\pgfqpoint{0.626678in}{0.752749in}}{\pgfqpoint{0.632264in}{0.755063in}}{\pgfqpoint{0.636382in}{0.759181in}}%
\pgfpathcurveto{\pgfqpoint{0.640500in}{0.763299in}}{\pgfqpoint{0.642814in}{0.768885in}}{\pgfqpoint{0.642814in}{0.774709in}}%
\pgfpathcurveto{\pgfqpoint{0.642814in}{0.780533in}}{\pgfqpoint{0.640500in}{0.786120in}}{\pgfqpoint{0.636382in}{0.790238in}}%
\pgfpathcurveto{\pgfqpoint{0.632264in}{0.794356in}}{\pgfqpoint{0.626678in}{0.796670in}}{\pgfqpoint{0.620854in}{0.796670in}}%
\pgfpathcurveto{\pgfqpoint{0.615030in}{0.796670in}}{\pgfqpoint{0.609444in}{0.794356in}}{\pgfqpoint{0.605326in}{0.790238in}}%
\pgfpathcurveto{\pgfqpoint{0.601208in}{0.786120in}}{\pgfqpoint{0.598894in}{0.780533in}}{\pgfqpoint{0.598894in}{0.774709in}}%
\pgfpathcurveto{\pgfqpoint{0.598894in}{0.768885in}}{\pgfqpoint{0.601208in}{0.763299in}}{\pgfqpoint{0.605326in}{0.759181in}}%
\pgfpathcurveto{\pgfqpoint{0.609444in}{0.755063in}}{\pgfqpoint{0.615030in}{0.752749in}}{\pgfqpoint{0.620854in}{0.752749in}}%
\pgfpathclose%
\pgfusepath{stroke,fill}%
\end{pgfscope}%
\begin{pgfscope}%
\pgfpathrectangle{\pgfqpoint{0.211875in}{0.211875in}}{\pgfqpoint{1.313625in}{1.279725in}}%
\pgfusepath{clip}%
\pgfsetbuttcap%
\pgfsetroundjoin%
\definecolor{currentfill}{rgb}{0.121569,0.466667,0.705882}%
\pgfsetfillcolor{currentfill}%
\pgfsetlinewidth{1.003750pt}%
\definecolor{currentstroke}{rgb}{0.121569,0.466667,0.705882}%
\pgfsetstrokecolor{currentstroke}%
\pgfsetdash{}{0pt}%
\pgfpathmoveto{\pgfqpoint{0.492045in}{0.278706in}}%
\pgfpathcurveto{\pgfqpoint{0.497869in}{0.278706in}}{\pgfqpoint{0.503455in}{0.281020in}}{\pgfqpoint{0.507574in}{0.285138in}}%
\pgfpathcurveto{\pgfqpoint{0.511692in}{0.289256in}}{\pgfqpoint{0.514006in}{0.294842in}}{\pgfqpoint{0.514006in}{0.300666in}}%
\pgfpathcurveto{\pgfqpoint{0.514006in}{0.306490in}}{\pgfqpoint{0.511692in}{0.312076in}}{\pgfqpoint{0.507574in}{0.316194in}}%
\pgfpathcurveto{\pgfqpoint{0.503455in}{0.320312in}}{\pgfqpoint{0.497869in}{0.322626in}}{\pgfqpoint{0.492045in}{0.322626in}}%
\pgfpathcurveto{\pgfqpoint{0.486221in}{0.322626in}}{\pgfqpoint{0.480635in}{0.320312in}}{\pgfqpoint{0.476517in}{0.316194in}}%
\pgfpathcurveto{\pgfqpoint{0.472399in}{0.312076in}}{\pgfqpoint{0.470085in}{0.306490in}}{\pgfqpoint{0.470085in}{0.300666in}}%
\pgfpathcurveto{\pgfqpoint{0.470085in}{0.294842in}}{\pgfqpoint{0.472399in}{0.289256in}}{\pgfqpoint{0.476517in}{0.285138in}}%
\pgfpathcurveto{\pgfqpoint{0.480635in}{0.281020in}}{\pgfqpoint{0.486221in}{0.278706in}}{\pgfqpoint{0.492045in}{0.278706in}}%
\pgfpathclose%
\pgfusepath{stroke,fill}%
\end{pgfscope}%
\begin{pgfscope}%
\pgfpathrectangle{\pgfqpoint{0.211875in}{0.211875in}}{\pgfqpoint{1.313625in}{1.279725in}}%
\pgfusepath{clip}%
\pgfsetbuttcap%
\pgfsetroundjoin%
\definecolor{currentfill}{rgb}{0.121569,0.466667,0.705882}%
\pgfsetfillcolor{currentfill}%
\pgfsetlinewidth{1.003750pt}%
\definecolor{currentstroke}{rgb}{0.121569,0.466667,0.705882}%
\pgfsetstrokecolor{currentstroke}%
\pgfsetdash{}{0pt}%
\pgfpathmoveto{\pgfqpoint{0.551038in}{1.317552in}}%
\pgfpathcurveto{\pgfqpoint{0.556862in}{1.317552in}}{\pgfqpoint{0.562448in}{1.319866in}}{\pgfqpoint{0.566566in}{1.323984in}}%
\pgfpathcurveto{\pgfqpoint{0.570684in}{1.328102in}}{\pgfqpoint{0.572998in}{1.333688in}}{\pgfqpoint{0.572998in}{1.339512in}}%
\pgfpathcurveto{\pgfqpoint{0.572998in}{1.345336in}}{\pgfqpoint{0.570684in}{1.350922in}}{\pgfqpoint{0.566566in}{1.355040in}}%
\pgfpathcurveto{\pgfqpoint{0.562448in}{1.359159in}}{\pgfqpoint{0.556862in}{1.361472in}}{\pgfqpoint{0.551038in}{1.361472in}}%
\pgfpathcurveto{\pgfqpoint{0.545214in}{1.361472in}}{\pgfqpoint{0.539628in}{1.359159in}}{\pgfqpoint{0.535510in}{1.355040in}}%
\pgfpathcurveto{\pgfqpoint{0.531392in}{1.350922in}}{\pgfqpoint{0.529078in}{1.345336in}}{\pgfqpoint{0.529078in}{1.339512in}}%
\pgfpathcurveto{\pgfqpoint{0.529078in}{1.333688in}}{\pgfqpoint{0.531392in}{1.328102in}}{\pgfqpoint{0.535510in}{1.323984in}}%
\pgfpathcurveto{\pgfqpoint{0.539628in}{1.319866in}}{\pgfqpoint{0.545214in}{1.317552in}}{\pgfqpoint{0.551038in}{1.317552in}}%
\pgfpathclose%
\pgfusepath{stroke,fill}%
\end{pgfscope}%
\begin{pgfscope}%
\pgfpathrectangle{\pgfqpoint{0.211875in}{0.211875in}}{\pgfqpoint{1.313625in}{1.279725in}}%
\pgfusepath{clip}%
\pgfsetbuttcap%
\pgfsetroundjoin%
\definecolor{currentfill}{rgb}{0.121569,0.466667,0.705882}%
\pgfsetfillcolor{currentfill}%
\pgfsetlinewidth{1.003750pt}%
\definecolor{currentstroke}{rgb}{0.121569,0.466667,0.705882}%
\pgfsetstrokecolor{currentstroke}%
\pgfsetdash{}{0pt}%
\pgfpathmoveto{\pgfqpoint{1.258579in}{1.354995in}}%
\pgfpathcurveto{\pgfqpoint{1.264403in}{1.354995in}}{\pgfqpoint{1.269989in}{1.357309in}}{\pgfqpoint{1.274108in}{1.361427in}}%
\pgfpathcurveto{\pgfqpoint{1.278226in}{1.365545in}}{\pgfqpoint{1.280540in}{1.371131in}}{\pgfqpoint{1.280540in}{1.376955in}}%
\pgfpathcurveto{\pgfqpoint{1.280540in}{1.382779in}}{\pgfqpoint{1.278226in}{1.388365in}}{\pgfqpoint{1.274108in}{1.392483in}}%
\pgfpathcurveto{\pgfqpoint{1.269989in}{1.396601in}}{\pgfqpoint{1.264403in}{1.398915in}}{\pgfqpoint{1.258579in}{1.398915in}}%
\pgfpathcurveto{\pgfqpoint{1.252755in}{1.398915in}}{\pgfqpoint{1.247169in}{1.396601in}}{\pgfqpoint{1.243051in}{1.392483in}}%
\pgfpathcurveto{\pgfqpoint{1.238933in}{1.388365in}}{\pgfqpoint{1.236619in}{1.382779in}}{\pgfqpoint{1.236619in}{1.376955in}}%
\pgfpathcurveto{\pgfqpoint{1.236619in}{1.371131in}}{\pgfqpoint{1.238933in}{1.365545in}}{\pgfqpoint{1.243051in}{1.361427in}}%
\pgfpathcurveto{\pgfqpoint{1.247169in}{1.357309in}}{\pgfqpoint{1.252755in}{1.354995in}}{\pgfqpoint{1.258579in}{1.354995in}}%
\pgfpathclose%
\pgfusepath{stroke,fill}%
\end{pgfscope}%
\begin{pgfscope}%
\pgfpathrectangle{\pgfqpoint{0.211875in}{0.211875in}}{\pgfqpoint{1.313625in}{1.279725in}}%
\pgfusepath{clip}%
\pgfsetbuttcap%
\pgfsetroundjoin%
\definecolor{currentfill}{rgb}{0.121569,0.466667,0.705882}%
\pgfsetfillcolor{currentfill}%
\pgfsetlinewidth{1.003750pt}%
\definecolor{currentstroke}{rgb}{0.121569,0.466667,0.705882}%
\pgfsetstrokecolor{currentstroke}%
\pgfsetdash{}{0pt}%
\pgfpathmoveto{\pgfqpoint{1.306303in}{0.673060in}}%
\pgfpathcurveto{\pgfqpoint{1.312127in}{0.673060in}}{\pgfqpoint{1.317713in}{0.675374in}}{\pgfqpoint{1.321831in}{0.679492in}}%
\pgfpathcurveto{\pgfqpoint{1.325949in}{0.683610in}}{\pgfqpoint{1.328263in}{0.689196in}}{\pgfqpoint{1.328263in}{0.695020in}}%
\pgfpathcurveto{\pgfqpoint{1.328263in}{0.700844in}}{\pgfqpoint{1.325949in}{0.706430in}}{\pgfqpoint{1.321831in}{0.710548in}}%
\pgfpathcurveto{\pgfqpoint{1.317713in}{0.714666in}}{\pgfqpoint{1.312127in}{0.716980in}}{\pgfqpoint{1.306303in}{0.716980in}}%
\pgfpathcurveto{\pgfqpoint{1.300479in}{0.716980in}}{\pgfqpoint{1.294893in}{0.714666in}}{\pgfqpoint{1.290774in}{0.710548in}}%
\pgfpathcurveto{\pgfqpoint{1.286656in}{0.706430in}}{\pgfqpoint{1.284342in}{0.700844in}}{\pgfqpoint{1.284342in}{0.695020in}}%
\pgfpathcurveto{\pgfqpoint{1.284342in}{0.689196in}}{\pgfqpoint{1.286656in}{0.683610in}}{\pgfqpoint{1.290774in}{0.679492in}}%
\pgfpathcurveto{\pgfqpoint{1.294893in}{0.675374in}}{\pgfqpoint{1.300479in}{0.673060in}}{\pgfqpoint{1.306303in}{0.673060in}}%
\pgfpathclose%
\pgfusepath{stroke,fill}%
\end{pgfscope}%
\begin{pgfscope}%
\pgfpathrectangle{\pgfqpoint{0.211875in}{0.211875in}}{\pgfqpoint{1.313625in}{1.279725in}}%
\pgfusepath{clip}%
\pgfsetbuttcap%
\pgfsetroundjoin%
\definecolor{currentfill}{rgb}{0.121569,0.466667,0.705882}%
\pgfsetfillcolor{currentfill}%
\pgfsetlinewidth{1.003750pt}%
\definecolor{currentstroke}{rgb}{0.121569,0.466667,0.705882}%
\pgfsetstrokecolor{currentstroke}%
\pgfsetdash{}{0pt}%
\pgfpathmoveto{\pgfqpoint{1.018627in}{1.171494in}}%
\pgfpathcurveto{\pgfqpoint{1.024451in}{1.171494in}}{\pgfqpoint{1.030037in}{1.173808in}}{\pgfqpoint{1.034155in}{1.177926in}}%
\pgfpathcurveto{\pgfqpoint{1.038273in}{1.182045in}}{\pgfqpoint{1.040587in}{1.187631in}}{\pgfqpoint{1.040587in}{1.193455in}}%
\pgfpathcurveto{\pgfqpoint{1.040587in}{1.199279in}}{\pgfqpoint{1.038273in}{1.204865in}}{\pgfqpoint{1.034155in}{1.208983in}}%
\pgfpathcurveto{\pgfqpoint{1.030037in}{1.213101in}}{\pgfqpoint{1.024451in}{1.215415in}}{\pgfqpoint{1.018627in}{1.215415in}}%
\pgfpathcurveto{\pgfqpoint{1.012803in}{1.215415in}}{\pgfqpoint{1.007217in}{1.213101in}}{\pgfqpoint{1.003099in}{1.208983in}}%
\pgfpathcurveto{\pgfqpoint{0.998980in}{1.204865in}}{\pgfqpoint{0.996667in}{1.199279in}}{\pgfqpoint{0.996667in}{1.193455in}}%
\pgfpathcurveto{\pgfqpoint{0.996667in}{1.187631in}}{\pgfqpoint{0.998980in}{1.182045in}}{\pgfqpoint{1.003099in}{1.177926in}}%
\pgfpathcurveto{\pgfqpoint{1.007217in}{1.173808in}}{\pgfqpoint{1.012803in}{1.171494in}}{\pgfqpoint{1.018627in}{1.171494in}}%
\pgfpathclose%
\pgfusepath{stroke,fill}%
\end{pgfscope}%
\begin{pgfscope}%
\pgfpathrectangle{\pgfqpoint{0.211875in}{0.211875in}}{\pgfqpoint{1.313625in}{1.279725in}}%
\pgfusepath{clip}%
\pgfsetbuttcap%
\pgfsetroundjoin%
\definecolor{currentfill}{rgb}{0.121569,0.466667,0.705882}%
\pgfsetfillcolor{currentfill}%
\pgfsetlinewidth{1.003750pt}%
\definecolor{currentstroke}{rgb}{0.121569,0.466667,0.705882}%
\pgfsetstrokecolor{currentstroke}%
\pgfsetdash{}{0pt}%
\pgfpathmoveto{\pgfqpoint{1.210668in}{0.662651in}}%
\pgfpathcurveto{\pgfqpoint{1.216492in}{0.662651in}}{\pgfqpoint{1.222078in}{0.664965in}}{\pgfqpoint{1.226196in}{0.669083in}}%
\pgfpathcurveto{\pgfqpoint{1.230314in}{0.673201in}}{\pgfqpoint{1.232628in}{0.678788in}}{\pgfqpoint{1.232628in}{0.684612in}}%
\pgfpathcurveto{\pgfqpoint{1.232628in}{0.690435in}}{\pgfqpoint{1.230314in}{0.696022in}}{\pgfqpoint{1.226196in}{0.700140in}}%
\pgfpathcurveto{\pgfqpoint{1.222078in}{0.704258in}}{\pgfqpoint{1.216492in}{0.706572in}}{\pgfqpoint{1.210668in}{0.706572in}}%
\pgfpathcurveto{\pgfqpoint{1.204844in}{0.706572in}}{\pgfqpoint{1.199258in}{0.704258in}}{\pgfqpoint{1.195140in}{0.700140in}}%
\pgfpathcurveto{\pgfqpoint{1.191022in}{0.696022in}}{\pgfqpoint{1.188708in}{0.690435in}}{\pgfqpoint{1.188708in}{0.684612in}}%
\pgfpathcurveto{\pgfqpoint{1.188708in}{0.678788in}}{\pgfqpoint{1.191022in}{0.673201in}}{\pgfqpoint{1.195140in}{0.669083in}}%
\pgfpathcurveto{\pgfqpoint{1.199258in}{0.664965in}}{\pgfqpoint{1.204844in}{0.662651in}}{\pgfqpoint{1.210668in}{0.662651in}}%
\pgfpathclose%
\pgfusepath{stroke,fill}%
\end{pgfscope}%
\begin{pgfscope}%
\pgfpathrectangle{\pgfqpoint{0.211875in}{0.211875in}}{\pgfqpoint{1.313625in}{1.279725in}}%
\pgfusepath{clip}%
\pgfsetbuttcap%
\pgfsetroundjoin%
\definecolor{currentfill}{rgb}{0.121569,0.466667,0.705882}%
\pgfsetfillcolor{currentfill}%
\pgfsetlinewidth{1.003750pt}%
\definecolor{currentstroke}{rgb}{0.121569,0.466667,0.705882}%
\pgfsetstrokecolor{currentstroke}%
\pgfsetdash{}{0pt}%
\pgfpathmoveto{\pgfqpoint{1.240226in}{1.346017in}}%
\pgfpathcurveto{\pgfqpoint{1.246050in}{1.346017in}}{\pgfqpoint{1.251636in}{1.348331in}}{\pgfqpoint{1.255754in}{1.352449in}}%
\pgfpathcurveto{\pgfqpoint{1.259872in}{1.356567in}}{\pgfqpoint{1.262186in}{1.362153in}}{\pgfqpoint{1.262186in}{1.367977in}}%
\pgfpathcurveto{\pgfqpoint{1.262186in}{1.373801in}}{\pgfqpoint{1.259872in}{1.379387in}}{\pgfqpoint{1.255754in}{1.383505in}}%
\pgfpathcurveto{\pgfqpoint{1.251636in}{1.387623in}}{\pgfqpoint{1.246050in}{1.389937in}}{\pgfqpoint{1.240226in}{1.389937in}}%
\pgfpathcurveto{\pgfqpoint{1.234402in}{1.389937in}}{\pgfqpoint{1.228816in}{1.387623in}}{\pgfqpoint{1.224698in}{1.383505in}}%
\pgfpathcurveto{\pgfqpoint{1.220580in}{1.379387in}}{\pgfqpoint{1.218266in}{1.373801in}}{\pgfqpoint{1.218266in}{1.367977in}}%
\pgfpathcurveto{\pgfqpoint{1.218266in}{1.362153in}}{\pgfqpoint{1.220580in}{1.356567in}}{\pgfqpoint{1.224698in}{1.352449in}}%
\pgfpathcurveto{\pgfqpoint{1.228816in}{1.348331in}}{\pgfqpoint{1.234402in}{1.346017in}}{\pgfqpoint{1.240226in}{1.346017in}}%
\pgfpathclose%
\pgfusepath{stroke,fill}%
\end{pgfscope}%
\begin{pgfscope}%
\pgfpathrectangle{\pgfqpoint{0.211875in}{0.211875in}}{\pgfqpoint{1.313625in}{1.279725in}}%
\pgfusepath{clip}%
\pgfsetbuttcap%
\pgfsetroundjoin%
\definecolor{currentfill}{rgb}{0.121569,0.466667,0.705882}%
\pgfsetfillcolor{currentfill}%
\pgfsetlinewidth{1.003750pt}%
\definecolor{currentstroke}{rgb}{0.121569,0.466667,0.705882}%
\pgfsetstrokecolor{currentstroke}%
\pgfsetdash{}{0pt}%
\pgfpathmoveto{\pgfqpoint{1.409632in}{0.792770in}}%
\pgfpathcurveto{\pgfqpoint{1.415456in}{0.792770in}}{\pgfqpoint{1.421042in}{0.795084in}}{\pgfqpoint{1.425161in}{0.799202in}}%
\pgfpathcurveto{\pgfqpoint{1.429279in}{0.803320in}}{\pgfqpoint{1.431593in}{0.808907in}}{\pgfqpoint{1.431593in}{0.814730in}}%
\pgfpathcurveto{\pgfqpoint{1.431593in}{0.820554in}}{\pgfqpoint{1.429279in}{0.826141in}}{\pgfqpoint{1.425161in}{0.830259in}}%
\pgfpathcurveto{\pgfqpoint{1.421042in}{0.834377in}}{\pgfqpoint{1.415456in}{0.836691in}}{\pgfqpoint{1.409632in}{0.836691in}}%
\pgfpathcurveto{\pgfqpoint{1.403808in}{0.836691in}}{\pgfqpoint{1.398222in}{0.834377in}}{\pgfqpoint{1.394104in}{0.830259in}}%
\pgfpathcurveto{\pgfqpoint{1.389986in}{0.826141in}}{\pgfqpoint{1.387672in}{0.820554in}}{\pgfqpoint{1.387672in}{0.814730in}}%
\pgfpathcurveto{\pgfqpoint{1.387672in}{0.808907in}}{\pgfqpoint{1.389986in}{0.803320in}}{\pgfqpoint{1.394104in}{0.799202in}}%
\pgfpathcurveto{\pgfqpoint{1.398222in}{0.795084in}}{\pgfqpoint{1.403808in}{0.792770in}}{\pgfqpoint{1.409632in}{0.792770in}}%
\pgfpathclose%
\pgfusepath{stroke,fill}%
\end{pgfscope}%
\begin{pgfscope}%
\pgfpathrectangle{\pgfqpoint{0.211875in}{0.211875in}}{\pgfqpoint{1.313625in}{1.279725in}}%
\pgfusepath{clip}%
\pgfsetbuttcap%
\pgfsetroundjoin%
\definecolor{currentfill}{rgb}{0.121569,0.466667,0.705882}%
\pgfsetfillcolor{currentfill}%
\pgfsetlinewidth{1.003750pt}%
\definecolor{currentstroke}{rgb}{0.121569,0.466667,0.705882}%
\pgfsetstrokecolor{currentstroke}%
\pgfsetdash{}{0pt}%
\pgfpathmoveto{\pgfqpoint{1.233335in}{0.649951in}}%
\pgfpathcurveto{\pgfqpoint{1.239159in}{0.649951in}}{\pgfqpoint{1.244746in}{0.652265in}}{\pgfqpoint{1.248864in}{0.656383in}}%
\pgfpathcurveto{\pgfqpoint{1.252982in}{0.660502in}}{\pgfqpoint{1.255296in}{0.666088in}}{\pgfqpoint{1.255296in}{0.671912in}}%
\pgfpathcurveto{\pgfqpoint{1.255296in}{0.677736in}}{\pgfqpoint{1.252982in}{0.683322in}}{\pgfqpoint{1.248864in}{0.687440in}}%
\pgfpathcurveto{\pgfqpoint{1.244746in}{0.691558in}}{\pgfqpoint{1.239159in}{0.693872in}}{\pgfqpoint{1.233335in}{0.693872in}}%
\pgfpathcurveto{\pgfqpoint{1.227512in}{0.693872in}}{\pgfqpoint{1.221925in}{0.691558in}}{\pgfqpoint{1.217807in}{0.687440in}}%
\pgfpathcurveto{\pgfqpoint{1.213689in}{0.683322in}}{\pgfqpoint{1.211375in}{0.677736in}}{\pgfqpoint{1.211375in}{0.671912in}}%
\pgfpathcurveto{\pgfqpoint{1.211375in}{0.666088in}}{\pgfqpoint{1.213689in}{0.660502in}}{\pgfqpoint{1.217807in}{0.656383in}}%
\pgfpathcurveto{\pgfqpoint{1.221925in}{0.652265in}}{\pgfqpoint{1.227512in}{0.649951in}}{\pgfqpoint{1.233335in}{0.649951in}}%
\pgfpathclose%
\pgfusepath{stroke,fill}%
\end{pgfscope}%
\begin{pgfscope}%
\pgfpathrectangle{\pgfqpoint{0.211875in}{0.211875in}}{\pgfqpoint{1.313625in}{1.279725in}}%
\pgfusepath{clip}%
\pgfsetbuttcap%
\pgfsetroundjoin%
\definecolor{currentfill}{rgb}{0.121569,0.466667,0.705882}%
\pgfsetfillcolor{currentfill}%
\pgfsetlinewidth{1.003750pt}%
\definecolor{currentstroke}{rgb}{0.121569,0.466667,0.705882}%
\pgfsetstrokecolor{currentstroke}%
\pgfsetdash{}{0pt}%
\pgfpathmoveto{\pgfqpoint{1.237221in}{0.653381in}}%
\pgfpathcurveto{\pgfqpoint{1.243045in}{0.653381in}}{\pgfqpoint{1.248631in}{0.655695in}}{\pgfqpoint{1.252749in}{0.659813in}}%
\pgfpathcurveto{\pgfqpoint{1.256868in}{0.663931in}}{\pgfqpoint{1.259181in}{0.669517in}}{\pgfqpoint{1.259181in}{0.675341in}}%
\pgfpathcurveto{\pgfqpoint{1.259181in}{0.681165in}}{\pgfqpoint{1.256868in}{0.686751in}}{\pgfqpoint{1.252749in}{0.690869in}}%
\pgfpathcurveto{\pgfqpoint{1.248631in}{0.694987in}}{\pgfqpoint{1.243045in}{0.697301in}}{\pgfqpoint{1.237221in}{0.697301in}}%
\pgfpathcurveto{\pgfqpoint{1.231397in}{0.697301in}}{\pgfqpoint{1.225811in}{0.694987in}}{\pgfqpoint{1.221693in}{0.690869in}}%
\pgfpathcurveto{\pgfqpoint{1.217575in}{0.686751in}}{\pgfqpoint{1.215261in}{0.681165in}}{\pgfqpoint{1.215261in}{0.675341in}}%
\pgfpathcurveto{\pgfqpoint{1.215261in}{0.669517in}}{\pgfqpoint{1.217575in}{0.663931in}}{\pgfqpoint{1.221693in}{0.659813in}}%
\pgfpathcurveto{\pgfqpoint{1.225811in}{0.655695in}}{\pgfqpoint{1.231397in}{0.653381in}}{\pgfqpoint{1.237221in}{0.653381in}}%
\pgfpathclose%
\pgfusepath{stroke,fill}%
\end{pgfscope}%
\begin{pgfscope}%
\pgfpathrectangle{\pgfqpoint{0.211875in}{0.211875in}}{\pgfqpoint{1.313625in}{1.279725in}}%
\pgfusepath{clip}%
\pgfsetbuttcap%
\pgfsetroundjoin%
\definecolor{currentfill}{rgb}{0.121569,0.466667,0.705882}%
\pgfsetfillcolor{currentfill}%
\pgfsetlinewidth{1.003750pt}%
\definecolor{currentstroke}{rgb}{0.121569,0.466667,0.705882}%
\pgfsetstrokecolor{currentstroke}%
\pgfsetdash{}{0pt}%
\pgfpathmoveto{\pgfqpoint{1.273884in}{1.350789in}}%
\pgfpathcurveto{\pgfqpoint{1.279708in}{1.350789in}}{\pgfqpoint{1.285294in}{1.353103in}}{\pgfqpoint{1.289413in}{1.357221in}}%
\pgfpathcurveto{\pgfqpoint{1.293531in}{1.361339in}}{\pgfqpoint{1.295845in}{1.366925in}}{\pgfqpoint{1.295845in}{1.372749in}}%
\pgfpathcurveto{\pgfqpoint{1.295845in}{1.378573in}}{\pgfqpoint{1.293531in}{1.384160in}}{\pgfqpoint{1.289413in}{1.388278in}}%
\pgfpathcurveto{\pgfqpoint{1.285294in}{1.392396in}}{\pgfqpoint{1.279708in}{1.394710in}}{\pgfqpoint{1.273884in}{1.394710in}}%
\pgfpathcurveto{\pgfqpoint{1.268060in}{1.394710in}}{\pgfqpoint{1.262474in}{1.392396in}}{\pgfqpoint{1.258356in}{1.388278in}}%
\pgfpathcurveto{\pgfqpoint{1.254238in}{1.384160in}}{\pgfqpoint{1.251924in}{1.378573in}}{\pgfqpoint{1.251924in}{1.372749in}}%
\pgfpathcurveto{\pgfqpoint{1.251924in}{1.366925in}}{\pgfqpoint{1.254238in}{1.361339in}}{\pgfqpoint{1.258356in}{1.357221in}}%
\pgfpathcurveto{\pgfqpoint{1.262474in}{1.353103in}}{\pgfqpoint{1.268060in}{1.350789in}}{\pgfqpoint{1.273884in}{1.350789in}}%
\pgfpathclose%
\pgfusepath{stroke,fill}%
\end{pgfscope}%
\begin{pgfscope}%
\pgfpathrectangle{\pgfqpoint{0.211875in}{0.211875in}}{\pgfqpoint{1.313625in}{1.279725in}}%
\pgfusepath{clip}%
\pgfsetbuttcap%
\pgfsetroundjoin%
\definecolor{currentfill}{rgb}{0.121569,0.466667,0.705882}%
\pgfsetfillcolor{currentfill}%
\pgfsetlinewidth{1.003750pt}%
\definecolor{currentstroke}{rgb}{0.121569,0.466667,0.705882}%
\pgfsetstrokecolor{currentstroke}%
\pgfsetdash{}{0pt}%
\pgfpathmoveto{\pgfqpoint{1.396123in}{1.190310in}}%
\pgfpathcurveto{\pgfqpoint{1.401947in}{1.190310in}}{\pgfqpoint{1.407533in}{1.192624in}}{\pgfqpoint{1.411651in}{1.196742in}}%
\pgfpathcurveto{\pgfqpoint{1.415769in}{1.200860in}}{\pgfqpoint{1.418083in}{1.206446in}}{\pgfqpoint{1.418083in}{1.212270in}}%
\pgfpathcurveto{\pgfqpoint{1.418083in}{1.218094in}}{\pgfqpoint{1.415769in}{1.223680in}}{\pgfqpoint{1.411651in}{1.227798in}}%
\pgfpathcurveto{\pgfqpoint{1.407533in}{1.231916in}}{\pgfqpoint{1.401947in}{1.234230in}}{\pgfqpoint{1.396123in}{1.234230in}}%
\pgfpathcurveto{\pgfqpoint{1.390299in}{1.234230in}}{\pgfqpoint{1.384713in}{1.231916in}}{\pgfqpoint{1.380595in}{1.227798in}}%
\pgfpathcurveto{\pgfqpoint{1.376477in}{1.223680in}}{\pgfqpoint{1.374163in}{1.218094in}}{\pgfqpoint{1.374163in}{1.212270in}}%
\pgfpathcurveto{\pgfqpoint{1.374163in}{1.206446in}}{\pgfqpoint{1.376477in}{1.200860in}}{\pgfqpoint{1.380595in}{1.196742in}}%
\pgfpathcurveto{\pgfqpoint{1.384713in}{1.192624in}}{\pgfqpoint{1.390299in}{1.190310in}}{\pgfqpoint{1.396123in}{1.190310in}}%
\pgfpathclose%
\pgfusepath{stroke,fill}%
\end{pgfscope}%
\begin{pgfscope}%
\pgfpathrectangle{\pgfqpoint{0.211875in}{0.211875in}}{\pgfqpoint{1.313625in}{1.279725in}}%
\pgfusepath{clip}%
\pgfsetbuttcap%
\pgfsetroundjoin%
\definecolor{currentfill}{rgb}{0.121569,0.466667,0.705882}%
\pgfsetfillcolor{currentfill}%
\pgfsetlinewidth{1.003750pt}%
\definecolor{currentstroke}{rgb}{0.121569,0.466667,0.705882}%
\pgfsetstrokecolor{currentstroke}%
\pgfsetdash{}{0pt}%
\pgfpathmoveto{\pgfqpoint{1.309105in}{1.340323in}}%
\pgfpathcurveto{\pgfqpoint{1.314929in}{1.340323in}}{\pgfqpoint{1.320515in}{1.342637in}}{\pgfqpoint{1.324633in}{1.346755in}}%
\pgfpathcurveto{\pgfqpoint{1.328751in}{1.350873in}}{\pgfqpoint{1.331065in}{1.356460in}}{\pgfqpoint{1.331065in}{1.362283in}}%
\pgfpathcurveto{\pgfqpoint{1.331065in}{1.368107in}}{\pgfqpoint{1.328751in}{1.373694in}}{\pgfqpoint{1.324633in}{1.377812in}}%
\pgfpathcurveto{\pgfqpoint{1.320515in}{1.381930in}}{\pgfqpoint{1.314929in}{1.384244in}}{\pgfqpoint{1.309105in}{1.384244in}}%
\pgfpathcurveto{\pgfqpoint{1.303281in}{1.384244in}}{\pgfqpoint{1.297695in}{1.381930in}}{\pgfqpoint{1.293577in}{1.377812in}}%
\pgfpathcurveto{\pgfqpoint{1.289458in}{1.373694in}}{\pgfqpoint{1.287145in}{1.368107in}}{\pgfqpoint{1.287145in}{1.362283in}}%
\pgfpathcurveto{\pgfqpoint{1.287145in}{1.356460in}}{\pgfqpoint{1.289458in}{1.350873in}}{\pgfqpoint{1.293577in}{1.346755in}}%
\pgfpathcurveto{\pgfqpoint{1.297695in}{1.342637in}}{\pgfqpoint{1.303281in}{1.340323in}}{\pgfqpoint{1.309105in}{1.340323in}}%
\pgfpathclose%
\pgfusepath{stroke,fill}%
\end{pgfscope}%
\begin{pgfscope}%
\pgfpathrectangle{\pgfqpoint{0.211875in}{0.211875in}}{\pgfqpoint{1.313625in}{1.279725in}}%
\pgfusepath{clip}%
\pgfsetbuttcap%
\pgfsetroundjoin%
\definecolor{currentfill}{rgb}{0.121569,0.466667,0.705882}%
\pgfsetfillcolor{currentfill}%
\pgfsetlinewidth{1.003750pt}%
\definecolor{currentstroke}{rgb}{0.121569,0.466667,0.705882}%
\pgfsetstrokecolor{currentstroke}%
\pgfsetdash{}{0pt}%
\pgfpathmoveto{\pgfqpoint{1.175618in}{0.302274in}}%
\pgfpathcurveto{\pgfqpoint{1.181442in}{0.302274in}}{\pgfqpoint{1.187028in}{0.304588in}}{\pgfqpoint{1.191146in}{0.308706in}}%
\pgfpathcurveto{\pgfqpoint{1.195264in}{0.312825in}}{\pgfqpoint{1.197578in}{0.318411in}}{\pgfqpoint{1.197578in}{0.324235in}}%
\pgfpathcurveto{\pgfqpoint{1.197578in}{0.330059in}}{\pgfqpoint{1.195264in}{0.335645in}}{\pgfqpoint{1.191146in}{0.339763in}}%
\pgfpathcurveto{\pgfqpoint{1.187028in}{0.343881in}}{\pgfqpoint{1.181442in}{0.346195in}}{\pgfqpoint{1.175618in}{0.346195in}}%
\pgfpathcurveto{\pgfqpoint{1.169794in}{0.346195in}}{\pgfqpoint{1.164208in}{0.343881in}}{\pgfqpoint{1.160090in}{0.339763in}}%
\pgfpathcurveto{\pgfqpoint{1.155971in}{0.335645in}}{\pgfqpoint{1.153658in}{0.330059in}}{\pgfqpoint{1.153658in}{0.324235in}}%
\pgfpathcurveto{\pgfqpoint{1.153658in}{0.318411in}}{\pgfqpoint{1.155971in}{0.312825in}}{\pgfqpoint{1.160090in}{0.308706in}}%
\pgfpathcurveto{\pgfqpoint{1.164208in}{0.304588in}}{\pgfqpoint{1.169794in}{0.302274in}}{\pgfqpoint{1.175618in}{0.302274in}}%
\pgfpathclose%
\pgfusepath{stroke,fill}%
\end{pgfscope}%
\begin{pgfscope}%
\pgfpathrectangle{\pgfqpoint{0.211875in}{0.211875in}}{\pgfqpoint{1.313625in}{1.279725in}}%
\pgfusepath{clip}%
\pgfsetbuttcap%
\pgfsetroundjoin%
\definecolor{currentfill}{rgb}{0.121569,0.466667,0.705882}%
\pgfsetfillcolor{currentfill}%
\pgfsetlinewidth{1.003750pt}%
\definecolor{currentstroke}{rgb}{0.121569,0.466667,0.705882}%
\pgfsetstrokecolor{currentstroke}%
\pgfsetdash{}{0pt}%
\pgfpathmoveto{\pgfqpoint{1.262211in}{1.337753in}}%
\pgfpathcurveto{\pgfqpoint{1.268035in}{1.337753in}}{\pgfqpoint{1.273621in}{1.340067in}}{\pgfqpoint{1.277739in}{1.344185in}}%
\pgfpathcurveto{\pgfqpoint{1.281857in}{1.348303in}}{\pgfqpoint{1.284171in}{1.353889in}}{\pgfqpoint{1.284171in}{1.359713in}}%
\pgfpathcurveto{\pgfqpoint{1.284171in}{1.365537in}}{\pgfqpoint{1.281857in}{1.371123in}}{\pgfqpoint{1.277739in}{1.375241in}}%
\pgfpathcurveto{\pgfqpoint{1.273621in}{1.379360in}}{\pgfqpoint{1.268035in}{1.381673in}}{\pgfqpoint{1.262211in}{1.381673in}}%
\pgfpathcurveto{\pgfqpoint{1.256387in}{1.381673in}}{\pgfqpoint{1.250801in}{1.379360in}}{\pgfqpoint{1.246683in}{1.375241in}}%
\pgfpathcurveto{\pgfqpoint{1.242564in}{1.371123in}}{\pgfqpoint{1.240251in}{1.365537in}}{\pgfqpoint{1.240251in}{1.359713in}}%
\pgfpathcurveto{\pgfqpoint{1.240251in}{1.353889in}}{\pgfqpoint{1.242564in}{1.348303in}}{\pgfqpoint{1.246683in}{1.344185in}}%
\pgfpathcurveto{\pgfqpoint{1.250801in}{1.340067in}}{\pgfqpoint{1.256387in}{1.337753in}}{\pgfqpoint{1.262211in}{1.337753in}}%
\pgfpathclose%
\pgfusepath{stroke,fill}%
\end{pgfscope}%
\begin{pgfscope}%
\pgfpathrectangle{\pgfqpoint{0.211875in}{0.211875in}}{\pgfqpoint{1.313625in}{1.279725in}}%
\pgfusepath{clip}%
\pgfsetbuttcap%
\pgfsetroundjoin%
\definecolor{currentfill}{rgb}{0.121569,0.466667,0.705882}%
\pgfsetfillcolor{currentfill}%
\pgfsetlinewidth{1.003750pt}%
\definecolor{currentstroke}{rgb}{0.121569,0.466667,0.705882}%
\pgfsetstrokecolor{currentstroke}%
\pgfsetdash{}{0pt}%
\pgfpathmoveto{\pgfqpoint{0.374580in}{0.994004in}}%
\pgfpathcurveto{\pgfqpoint{0.380404in}{0.994004in}}{\pgfqpoint{0.385991in}{0.996318in}}{\pgfqpoint{0.390109in}{1.000436in}}%
\pgfpathcurveto{\pgfqpoint{0.394227in}{1.004554in}}{\pgfqpoint{0.396541in}{1.010140in}}{\pgfqpoint{0.396541in}{1.015964in}}%
\pgfpathcurveto{\pgfqpoint{0.396541in}{1.021788in}}{\pgfqpoint{0.394227in}{1.027374in}}{\pgfqpoint{0.390109in}{1.031493in}}%
\pgfpathcurveto{\pgfqpoint{0.385991in}{1.035611in}}{\pgfqpoint{0.380404in}{1.037925in}}{\pgfqpoint{0.374580in}{1.037925in}}%
\pgfpathcurveto{\pgfqpoint{0.368757in}{1.037925in}}{\pgfqpoint{0.363170in}{1.035611in}}{\pgfqpoint{0.359052in}{1.031493in}}%
\pgfpathcurveto{\pgfqpoint{0.354934in}{1.027374in}}{\pgfqpoint{0.352620in}{1.021788in}}{\pgfqpoint{0.352620in}{1.015964in}}%
\pgfpathcurveto{\pgfqpoint{0.352620in}{1.010140in}}{\pgfqpoint{0.354934in}{1.004554in}}{\pgfqpoint{0.359052in}{1.000436in}}%
\pgfpathcurveto{\pgfqpoint{0.363170in}{0.996318in}}{\pgfqpoint{0.368757in}{0.994004in}}{\pgfqpoint{0.374580in}{0.994004in}}%
\pgfpathclose%
\pgfusepath{stroke,fill}%
\end{pgfscope}%
\begin{pgfscope}%
\pgfpathrectangle{\pgfqpoint{0.211875in}{0.211875in}}{\pgfqpoint{1.313625in}{1.279725in}}%
\pgfusepath{clip}%
\pgfsetbuttcap%
\pgfsetroundjoin%
\definecolor{currentfill}{rgb}{0.121569,0.466667,0.705882}%
\pgfsetfillcolor{currentfill}%
\pgfsetlinewidth{1.003750pt}%
\definecolor{currentstroke}{rgb}{0.121569,0.466667,0.705882}%
\pgfsetstrokecolor{currentstroke}%
\pgfsetdash{}{0pt}%
\pgfpathmoveto{\pgfqpoint{1.320525in}{0.664398in}}%
\pgfpathcurveto{\pgfqpoint{1.326349in}{0.664398in}}{\pgfqpoint{1.331935in}{0.666712in}}{\pgfqpoint{1.336053in}{0.670830in}}%
\pgfpathcurveto{\pgfqpoint{1.340171in}{0.674948in}}{\pgfqpoint{1.342485in}{0.680534in}}{\pgfqpoint{1.342485in}{0.686358in}}%
\pgfpathcurveto{\pgfqpoint{1.342485in}{0.692182in}}{\pgfqpoint{1.340171in}{0.697768in}}{\pgfqpoint{1.336053in}{0.701886in}}%
\pgfpathcurveto{\pgfqpoint{1.331935in}{0.706005in}}{\pgfqpoint{1.326349in}{0.708318in}}{\pgfqpoint{1.320525in}{0.708318in}}%
\pgfpathcurveto{\pgfqpoint{1.314701in}{0.708318in}}{\pgfqpoint{1.309115in}{0.706005in}}{\pgfqpoint{1.304997in}{0.701886in}}%
\pgfpathcurveto{\pgfqpoint{1.300878in}{0.697768in}}{\pgfqpoint{1.298565in}{0.692182in}}{\pgfqpoint{1.298565in}{0.686358in}}%
\pgfpathcurveto{\pgfqpoint{1.298565in}{0.680534in}}{\pgfqpoint{1.300878in}{0.674948in}}{\pgfqpoint{1.304997in}{0.670830in}}%
\pgfpathcurveto{\pgfqpoint{1.309115in}{0.666712in}}{\pgfqpoint{1.314701in}{0.664398in}}{\pgfqpoint{1.320525in}{0.664398in}}%
\pgfpathclose%
\pgfusepath{stroke,fill}%
\end{pgfscope}%
\begin{pgfscope}%
\pgfpathrectangle{\pgfqpoint{0.211875in}{0.211875in}}{\pgfqpoint{1.313625in}{1.279725in}}%
\pgfusepath{clip}%
\pgfsetbuttcap%
\pgfsetroundjoin%
\definecolor{currentfill}{rgb}{0.121569,0.466667,0.705882}%
\pgfsetfillcolor{currentfill}%
\pgfsetlinewidth{1.003750pt}%
\definecolor{currentstroke}{rgb}{0.121569,0.466667,0.705882}%
\pgfsetstrokecolor{currentstroke}%
\pgfsetdash{}{0pt}%
\pgfpathmoveto{\pgfqpoint{1.069605in}{0.297124in}}%
\pgfpathcurveto{\pgfqpoint{1.075429in}{0.297124in}}{\pgfqpoint{1.081015in}{0.299438in}}{\pgfqpoint{1.085133in}{0.303556in}}%
\pgfpathcurveto{\pgfqpoint{1.089251in}{0.307674in}}{\pgfqpoint{1.091565in}{0.313260in}}{\pgfqpoint{1.091565in}{0.319084in}}%
\pgfpathcurveto{\pgfqpoint{1.091565in}{0.324908in}}{\pgfqpoint{1.089251in}{0.330494in}}{\pgfqpoint{1.085133in}{0.334612in}}%
\pgfpathcurveto{\pgfqpoint{1.081015in}{0.338730in}}{\pgfqpoint{1.075429in}{0.341044in}}{\pgfqpoint{1.069605in}{0.341044in}}%
\pgfpathcurveto{\pgfqpoint{1.063781in}{0.341044in}}{\pgfqpoint{1.058195in}{0.338730in}}{\pgfqpoint{1.054077in}{0.334612in}}%
\pgfpathcurveto{\pgfqpoint{1.049959in}{0.330494in}}{\pgfqpoint{1.047645in}{0.324908in}}{\pgfqpoint{1.047645in}{0.319084in}}%
\pgfpathcurveto{\pgfqpoint{1.047645in}{0.313260in}}{\pgfqpoint{1.049959in}{0.307674in}}{\pgfqpoint{1.054077in}{0.303556in}}%
\pgfpathcurveto{\pgfqpoint{1.058195in}{0.299438in}}{\pgfqpoint{1.063781in}{0.297124in}}{\pgfqpoint{1.069605in}{0.297124in}}%
\pgfpathclose%
\pgfusepath{stroke,fill}%
\end{pgfscope}%
\begin{pgfscope}%
\pgfpathrectangle{\pgfqpoint{0.211875in}{0.211875in}}{\pgfqpoint{1.313625in}{1.279725in}}%
\pgfusepath{clip}%
\pgfsetbuttcap%
\pgfsetroundjoin%
\definecolor{currentfill}{rgb}{0.121569,0.466667,0.705882}%
\pgfsetfillcolor{currentfill}%
\pgfsetlinewidth{1.003750pt}%
\definecolor{currentstroke}{rgb}{0.121569,0.466667,0.705882}%
\pgfsetstrokecolor{currentstroke}%
\pgfsetdash{}{0pt}%
\pgfpathmoveto{\pgfqpoint{0.992871in}{0.281049in}}%
\pgfpathcurveto{\pgfqpoint{0.998695in}{0.281049in}}{\pgfqpoint{1.004281in}{0.283363in}}{\pgfqpoint{1.008399in}{0.287481in}}%
\pgfpathcurveto{\pgfqpoint{1.012517in}{0.291599in}}{\pgfqpoint{1.014831in}{0.297185in}}{\pgfqpoint{1.014831in}{0.303009in}}%
\pgfpathcurveto{\pgfqpoint{1.014831in}{0.308833in}}{\pgfqpoint{1.012517in}{0.314419in}}{\pgfqpoint{1.008399in}{0.318537in}}%
\pgfpathcurveto{\pgfqpoint{1.004281in}{0.322655in}}{\pgfqpoint{0.998695in}{0.324969in}}{\pgfqpoint{0.992871in}{0.324969in}}%
\pgfpathcurveto{\pgfqpoint{0.987047in}{0.324969in}}{\pgfqpoint{0.981461in}{0.322655in}}{\pgfqpoint{0.977342in}{0.318537in}}%
\pgfpathcurveto{\pgfqpoint{0.973224in}{0.314419in}}{\pgfqpoint{0.970910in}{0.308833in}}{\pgfqpoint{0.970910in}{0.303009in}}%
\pgfpathcurveto{\pgfqpoint{0.970910in}{0.297185in}}{\pgfqpoint{0.973224in}{0.291599in}}{\pgfqpoint{0.977342in}{0.287481in}}%
\pgfpathcurveto{\pgfqpoint{0.981461in}{0.283363in}}{\pgfqpoint{0.987047in}{0.281049in}}{\pgfqpoint{0.992871in}{0.281049in}}%
\pgfpathclose%
\pgfusepath{stroke,fill}%
\end{pgfscope}%
\begin{pgfscope}%
\pgfpathrectangle{\pgfqpoint{0.211875in}{0.211875in}}{\pgfqpoint{1.313625in}{1.279725in}}%
\pgfusepath{clip}%
\pgfsetbuttcap%
\pgfsetroundjoin%
\definecolor{currentfill}{rgb}{0.121569,0.466667,0.705882}%
\pgfsetfillcolor{currentfill}%
\pgfsetlinewidth{1.003750pt}%
\definecolor{currentstroke}{rgb}{0.121569,0.466667,0.705882}%
\pgfsetstrokecolor{currentstroke}%
\pgfsetdash{}{0pt}%
\pgfpathmoveto{\pgfqpoint{1.077215in}{0.280228in}}%
\pgfpathcurveto{\pgfqpoint{1.083039in}{0.280228in}}{\pgfqpoint{1.088625in}{0.282542in}}{\pgfqpoint{1.092743in}{0.286660in}}%
\pgfpathcurveto{\pgfqpoint{1.096861in}{0.290778in}}{\pgfqpoint{1.099175in}{0.296364in}}{\pgfqpoint{1.099175in}{0.302188in}}%
\pgfpathcurveto{\pgfqpoint{1.099175in}{0.308012in}}{\pgfqpoint{1.096861in}{0.313598in}}{\pgfqpoint{1.092743in}{0.317716in}}%
\pgfpathcurveto{\pgfqpoint{1.088625in}{0.321834in}}{\pgfqpoint{1.083039in}{0.324148in}}{\pgfqpoint{1.077215in}{0.324148in}}%
\pgfpathcurveto{\pgfqpoint{1.071391in}{0.324148in}}{\pgfqpoint{1.065805in}{0.321834in}}{\pgfqpoint{1.061687in}{0.317716in}}%
\pgfpathcurveto{\pgfqpoint{1.057569in}{0.313598in}}{\pgfqpoint{1.055255in}{0.308012in}}{\pgfqpoint{1.055255in}{0.302188in}}%
\pgfpathcurveto{\pgfqpoint{1.055255in}{0.296364in}}{\pgfqpoint{1.057569in}{0.290778in}}{\pgfqpoint{1.061687in}{0.286660in}}%
\pgfpathcurveto{\pgfqpoint{1.065805in}{0.282542in}}{\pgfqpoint{1.071391in}{0.280228in}}{\pgfqpoint{1.077215in}{0.280228in}}%
\pgfpathclose%
\pgfusepath{stroke,fill}%
\end{pgfscope}%
\begin{pgfscope}%
\pgfpathrectangle{\pgfqpoint{0.211875in}{0.211875in}}{\pgfqpoint{1.313625in}{1.279725in}}%
\pgfusepath{clip}%
\pgfsetbuttcap%
\pgfsetroundjoin%
\definecolor{currentfill}{rgb}{0.121569,0.466667,0.705882}%
\pgfsetfillcolor{currentfill}%
\pgfsetlinewidth{1.003750pt}%
\definecolor{currentstroke}{rgb}{0.121569,0.466667,0.705882}%
\pgfsetstrokecolor{currentstroke}%
\pgfsetdash{}{0pt}%
\pgfpathmoveto{\pgfqpoint{1.228242in}{1.286101in}}%
\pgfpathcurveto{\pgfqpoint{1.234066in}{1.286101in}}{\pgfqpoint{1.239652in}{1.288415in}}{\pgfqpoint{1.243770in}{1.292533in}}%
\pgfpathcurveto{\pgfqpoint{1.247888in}{1.296651in}}{\pgfqpoint{1.250202in}{1.302237in}}{\pgfqpoint{1.250202in}{1.308061in}}%
\pgfpathcurveto{\pgfqpoint{1.250202in}{1.313885in}}{\pgfqpoint{1.247888in}{1.319471in}}{\pgfqpoint{1.243770in}{1.323590in}}%
\pgfpathcurveto{\pgfqpoint{1.239652in}{1.327708in}}{\pgfqpoint{1.234066in}{1.330022in}}{\pgfqpoint{1.228242in}{1.330022in}}%
\pgfpathcurveto{\pgfqpoint{1.222418in}{1.330022in}}{\pgfqpoint{1.216832in}{1.327708in}}{\pgfqpoint{1.212714in}{1.323590in}}%
\pgfpathcurveto{\pgfqpoint{1.208596in}{1.319471in}}{\pgfqpoint{1.206282in}{1.313885in}}{\pgfqpoint{1.206282in}{1.308061in}}%
\pgfpathcurveto{\pgfqpoint{1.206282in}{1.302237in}}{\pgfqpoint{1.208596in}{1.296651in}}{\pgfqpoint{1.212714in}{1.292533in}}%
\pgfpathcurveto{\pgfqpoint{1.216832in}{1.288415in}}{\pgfqpoint{1.222418in}{1.286101in}}{\pgfqpoint{1.228242in}{1.286101in}}%
\pgfpathclose%
\pgfusepath{stroke,fill}%
\end{pgfscope}%
\begin{pgfscope}%
\pgfpathrectangle{\pgfqpoint{0.211875in}{0.211875in}}{\pgfqpoint{1.313625in}{1.279725in}}%
\pgfusepath{clip}%
\pgfsetbuttcap%
\pgfsetroundjoin%
\definecolor{currentfill}{rgb}{0.121569,0.466667,0.705882}%
\pgfsetfillcolor{currentfill}%
\pgfsetlinewidth{1.003750pt}%
\definecolor{currentstroke}{rgb}{0.121569,0.466667,0.705882}%
\pgfsetstrokecolor{currentstroke}%
\pgfsetdash{}{0pt}%
\pgfpathmoveto{\pgfqpoint{1.182723in}{1.260820in}}%
\pgfpathcurveto{\pgfqpoint{1.188547in}{1.260820in}}{\pgfqpoint{1.194133in}{1.263134in}}{\pgfqpoint{1.198252in}{1.267252in}}%
\pgfpathcurveto{\pgfqpoint{1.202370in}{1.271370in}}{\pgfqpoint{1.204684in}{1.276956in}}{\pgfqpoint{1.204684in}{1.282780in}}%
\pgfpathcurveto{\pgfqpoint{1.204684in}{1.288604in}}{\pgfqpoint{1.202370in}{1.294190in}}{\pgfqpoint{1.198252in}{1.298308in}}%
\pgfpathcurveto{\pgfqpoint{1.194133in}{1.302426in}}{\pgfqpoint{1.188547in}{1.304740in}}{\pgfqpoint{1.182723in}{1.304740in}}%
\pgfpathcurveto{\pgfqpoint{1.176899in}{1.304740in}}{\pgfqpoint{1.171313in}{1.302426in}}{\pgfqpoint{1.167195in}{1.298308in}}%
\pgfpathcurveto{\pgfqpoint{1.163077in}{1.294190in}}{\pgfqpoint{1.160763in}{1.288604in}}{\pgfqpoint{1.160763in}{1.282780in}}%
\pgfpathcurveto{\pgfqpoint{1.160763in}{1.276956in}}{\pgfqpoint{1.163077in}{1.271370in}}{\pgfqpoint{1.167195in}{1.267252in}}%
\pgfpathcurveto{\pgfqpoint{1.171313in}{1.263134in}}{\pgfqpoint{1.176899in}{1.260820in}}{\pgfqpoint{1.182723in}{1.260820in}}%
\pgfpathclose%
\pgfusepath{stroke,fill}%
\end{pgfscope}%
\begin{pgfscope}%
\pgfpathrectangle{\pgfqpoint{0.211875in}{0.211875in}}{\pgfqpoint{1.313625in}{1.279725in}}%
\pgfusepath{clip}%
\pgfsetbuttcap%
\pgfsetroundjoin%
\definecolor{currentfill}{rgb}{0.121569,0.466667,0.705882}%
\pgfsetfillcolor{currentfill}%
\pgfsetlinewidth{1.003750pt}%
\definecolor{currentstroke}{rgb}{0.121569,0.466667,0.705882}%
\pgfsetstrokecolor{currentstroke}%
\pgfsetdash{}{0pt}%
\pgfpathmoveto{\pgfqpoint{1.223778in}{1.317161in}}%
\pgfpathcurveto{\pgfqpoint{1.229602in}{1.317161in}}{\pgfqpoint{1.235188in}{1.319474in}}{\pgfqpoint{1.239307in}{1.323593in}}%
\pgfpathcurveto{\pgfqpoint{1.243425in}{1.327711in}}{\pgfqpoint{1.245739in}{1.333297in}}{\pgfqpoint{1.245739in}{1.339121in}}%
\pgfpathcurveto{\pgfqpoint{1.245739in}{1.344945in}}{\pgfqpoint{1.243425in}{1.350531in}}{\pgfqpoint{1.239307in}{1.354649in}}%
\pgfpathcurveto{\pgfqpoint{1.235188in}{1.358767in}}{\pgfqpoint{1.229602in}{1.361081in}}{\pgfqpoint{1.223778in}{1.361081in}}%
\pgfpathcurveto{\pgfqpoint{1.217954in}{1.361081in}}{\pgfqpoint{1.212368in}{1.358767in}}{\pgfqpoint{1.208250in}{1.354649in}}%
\pgfpathcurveto{\pgfqpoint{1.204132in}{1.350531in}}{\pgfqpoint{1.201818in}{1.344945in}}{\pgfqpoint{1.201818in}{1.339121in}}%
\pgfpathcurveto{\pgfqpoint{1.201818in}{1.333297in}}{\pgfqpoint{1.204132in}{1.327711in}}{\pgfqpoint{1.208250in}{1.323593in}}%
\pgfpathcurveto{\pgfqpoint{1.212368in}{1.319474in}}{\pgfqpoint{1.217954in}{1.317161in}}{\pgfqpoint{1.223778in}{1.317161in}}%
\pgfpathclose%
\pgfusepath{stroke,fill}%
\end{pgfscope}%
\begin{pgfscope}%
\pgfpathrectangle{\pgfqpoint{0.211875in}{0.211875in}}{\pgfqpoint{1.313625in}{1.279725in}}%
\pgfusepath{clip}%
\pgfsetbuttcap%
\pgfsetroundjoin%
\definecolor{currentfill}{rgb}{0.121569,0.466667,0.705882}%
\pgfsetfillcolor{currentfill}%
\pgfsetlinewidth{1.003750pt}%
\definecolor{currentstroke}{rgb}{0.121569,0.466667,0.705882}%
\pgfsetstrokecolor{currentstroke}%
\pgfsetdash{}{0pt}%
\pgfpathmoveto{\pgfqpoint{1.245301in}{1.308993in}}%
\pgfpathcurveto{\pgfqpoint{1.251125in}{1.308993in}}{\pgfqpoint{1.256711in}{1.311306in}}{\pgfqpoint{1.260829in}{1.315425in}}%
\pgfpathcurveto{\pgfqpoint{1.264947in}{1.319543in}}{\pgfqpoint{1.267261in}{1.325129in}}{\pgfqpoint{1.267261in}{1.330953in}}%
\pgfpathcurveto{\pgfqpoint{1.267261in}{1.336777in}}{\pgfqpoint{1.264947in}{1.342363in}}{\pgfqpoint{1.260829in}{1.346481in}}%
\pgfpathcurveto{\pgfqpoint{1.256711in}{1.350599in}}{\pgfqpoint{1.251125in}{1.352913in}}{\pgfqpoint{1.245301in}{1.352913in}}%
\pgfpathcurveto{\pgfqpoint{1.239477in}{1.352913in}}{\pgfqpoint{1.233891in}{1.350599in}}{\pgfqpoint{1.229773in}{1.346481in}}%
\pgfpathcurveto{\pgfqpoint{1.225655in}{1.342363in}}{\pgfqpoint{1.223341in}{1.336777in}}{\pgfqpoint{1.223341in}{1.330953in}}%
\pgfpathcurveto{\pgfqpoint{1.223341in}{1.325129in}}{\pgfqpoint{1.225655in}{1.319543in}}{\pgfqpoint{1.229773in}{1.315425in}}%
\pgfpathcurveto{\pgfqpoint{1.233891in}{1.311306in}}{\pgfqpoint{1.239477in}{1.308993in}}{\pgfqpoint{1.245301in}{1.308993in}}%
\pgfpathclose%
\pgfusepath{stroke,fill}%
\end{pgfscope}%
\begin{pgfscope}%
\pgfpathrectangle{\pgfqpoint{0.211875in}{0.211875in}}{\pgfqpoint{1.313625in}{1.279725in}}%
\pgfusepath{clip}%
\pgfsetbuttcap%
\pgfsetroundjoin%
\definecolor{currentfill}{rgb}{0.121569,0.466667,0.705882}%
\pgfsetfillcolor{currentfill}%
\pgfsetlinewidth{1.003750pt}%
\definecolor{currentstroke}{rgb}{0.121569,0.466667,0.705882}%
\pgfsetstrokecolor{currentstroke}%
\pgfsetdash{}{0pt}%
\pgfpathmoveto{\pgfqpoint{1.274732in}{1.370396in}}%
\pgfpathcurveto{\pgfqpoint{1.280556in}{1.370396in}}{\pgfqpoint{1.286142in}{1.372710in}}{\pgfqpoint{1.290260in}{1.376828in}}%
\pgfpathcurveto{\pgfqpoint{1.294378in}{1.380947in}}{\pgfqpoint{1.296692in}{1.386533in}}{\pgfqpoint{1.296692in}{1.392357in}}%
\pgfpathcurveto{\pgfqpoint{1.296692in}{1.398181in}}{\pgfqpoint{1.294378in}{1.403767in}}{\pgfqpoint{1.290260in}{1.407885in}}%
\pgfpathcurveto{\pgfqpoint{1.286142in}{1.412003in}}{\pgfqpoint{1.280556in}{1.414317in}}{\pgfqpoint{1.274732in}{1.414317in}}%
\pgfpathcurveto{\pgfqpoint{1.268908in}{1.414317in}}{\pgfqpoint{1.263322in}{1.412003in}}{\pgfqpoint{1.259204in}{1.407885in}}%
\pgfpathcurveto{\pgfqpoint{1.255085in}{1.403767in}}{\pgfqpoint{1.252772in}{1.398181in}}{\pgfqpoint{1.252772in}{1.392357in}}%
\pgfpathcurveto{\pgfqpoint{1.252772in}{1.386533in}}{\pgfqpoint{1.255085in}{1.380947in}}{\pgfqpoint{1.259204in}{1.376828in}}%
\pgfpathcurveto{\pgfqpoint{1.263322in}{1.372710in}}{\pgfqpoint{1.268908in}{1.370396in}}{\pgfqpoint{1.274732in}{1.370396in}}%
\pgfpathclose%
\pgfusepath{stroke,fill}%
\end{pgfscope}%
\begin{pgfscope}%
\pgfpathrectangle{\pgfqpoint{0.211875in}{0.211875in}}{\pgfqpoint{1.313625in}{1.279725in}}%
\pgfusepath{clip}%
\pgfsetbuttcap%
\pgfsetroundjoin%
\definecolor{currentfill}{rgb}{0.121569,0.466667,0.705882}%
\pgfsetfillcolor{currentfill}%
\pgfsetlinewidth{1.003750pt}%
\definecolor{currentstroke}{rgb}{0.121569,0.466667,0.705882}%
\pgfsetstrokecolor{currentstroke}%
\pgfsetdash{}{0pt}%
\pgfpathmoveto{\pgfqpoint{1.252643in}{0.678182in}}%
\pgfpathcurveto{\pgfqpoint{1.258467in}{0.678182in}}{\pgfqpoint{1.264053in}{0.680495in}}{\pgfqpoint{1.268171in}{0.684614in}}%
\pgfpathcurveto{\pgfqpoint{1.272289in}{0.688732in}}{\pgfqpoint{1.274603in}{0.694318in}}{\pgfqpoint{1.274603in}{0.700142in}}%
\pgfpathcurveto{\pgfqpoint{1.274603in}{0.705966in}}{\pgfqpoint{1.272289in}{0.711552in}}{\pgfqpoint{1.268171in}{0.715670in}}%
\pgfpathcurveto{\pgfqpoint{1.264053in}{0.719788in}}{\pgfqpoint{1.258467in}{0.722102in}}{\pgfqpoint{1.252643in}{0.722102in}}%
\pgfpathcurveto{\pgfqpoint{1.246819in}{0.722102in}}{\pgfqpoint{1.241233in}{0.719788in}}{\pgfqpoint{1.237115in}{0.715670in}}%
\pgfpathcurveto{\pgfqpoint{1.232997in}{0.711552in}}{\pgfqpoint{1.230683in}{0.705966in}}{\pgfqpoint{1.230683in}{0.700142in}}%
\pgfpathcurveto{\pgfqpoint{1.230683in}{0.694318in}}{\pgfqpoint{1.232997in}{0.688732in}}{\pgfqpoint{1.237115in}{0.684614in}}%
\pgfpathcurveto{\pgfqpoint{1.241233in}{0.680495in}}{\pgfqpoint{1.246819in}{0.678182in}}{\pgfqpoint{1.252643in}{0.678182in}}%
\pgfpathclose%
\pgfusepath{stroke,fill}%
\end{pgfscope}%
\begin{pgfscope}%
\pgfpathrectangle{\pgfqpoint{0.211875in}{0.211875in}}{\pgfqpoint{1.313625in}{1.279725in}}%
\pgfusepath{clip}%
\pgfsetbuttcap%
\pgfsetroundjoin%
\definecolor{currentfill}{rgb}{0.121569,0.466667,0.705882}%
\pgfsetfillcolor{currentfill}%
\pgfsetlinewidth{1.003750pt}%
\definecolor{currentstroke}{rgb}{0.121569,0.466667,0.705882}%
\pgfsetstrokecolor{currentstroke}%
\pgfsetdash{}{0pt}%
\pgfpathmoveto{\pgfqpoint{1.274484in}{1.339350in}}%
\pgfpathcurveto{\pgfqpoint{1.280308in}{1.339350in}}{\pgfqpoint{1.285895in}{1.341664in}}{\pgfqpoint{1.290013in}{1.345782in}}%
\pgfpathcurveto{\pgfqpoint{1.294131in}{1.349900in}}{\pgfqpoint{1.296445in}{1.355486in}}{\pgfqpoint{1.296445in}{1.361310in}}%
\pgfpathcurveto{\pgfqpoint{1.296445in}{1.367134in}}{\pgfqpoint{1.294131in}{1.372720in}}{\pgfqpoint{1.290013in}{1.376839in}}%
\pgfpathcurveto{\pgfqpoint{1.285895in}{1.380957in}}{\pgfqpoint{1.280308in}{1.383271in}}{\pgfqpoint{1.274484in}{1.383271in}}%
\pgfpathcurveto{\pgfqpoint{1.268661in}{1.383271in}}{\pgfqpoint{1.263074in}{1.380957in}}{\pgfqpoint{1.258956in}{1.376839in}}%
\pgfpathcurveto{\pgfqpoint{1.254838in}{1.372720in}}{\pgfqpoint{1.252524in}{1.367134in}}{\pgfqpoint{1.252524in}{1.361310in}}%
\pgfpathcurveto{\pgfqpoint{1.252524in}{1.355486in}}{\pgfqpoint{1.254838in}{1.349900in}}{\pgfqpoint{1.258956in}{1.345782in}}%
\pgfpathcurveto{\pgfqpoint{1.263074in}{1.341664in}}{\pgfqpoint{1.268661in}{1.339350in}}{\pgfqpoint{1.274484in}{1.339350in}}%
\pgfpathclose%
\pgfusepath{stroke,fill}%
\end{pgfscope}%
\begin{pgfscope}%
\pgfpathrectangle{\pgfqpoint{0.211875in}{0.211875in}}{\pgfqpoint{1.313625in}{1.279725in}}%
\pgfusepath{clip}%
\pgfsetbuttcap%
\pgfsetroundjoin%
\definecolor{currentfill}{rgb}{0.121569,0.466667,0.705882}%
\pgfsetfillcolor{currentfill}%
\pgfsetlinewidth{1.003750pt}%
\definecolor{currentstroke}{rgb}{0.121569,0.466667,0.705882}%
\pgfsetstrokecolor{currentstroke}%
\pgfsetdash{}{0pt}%
\pgfpathmoveto{\pgfqpoint{1.320474in}{0.681516in}}%
\pgfpathcurveto{\pgfqpoint{1.326298in}{0.681516in}}{\pgfqpoint{1.331884in}{0.683830in}}{\pgfqpoint{1.336002in}{0.687948in}}%
\pgfpathcurveto{\pgfqpoint{1.340120in}{0.692066in}}{\pgfqpoint{1.342434in}{0.697653in}}{\pgfqpoint{1.342434in}{0.703476in}}%
\pgfpathcurveto{\pgfqpoint{1.342434in}{0.709300in}}{\pgfqpoint{1.340120in}{0.714887in}}{\pgfqpoint{1.336002in}{0.719005in}}%
\pgfpathcurveto{\pgfqpoint{1.331884in}{0.723123in}}{\pgfqpoint{1.326298in}{0.725437in}}{\pgfqpoint{1.320474in}{0.725437in}}%
\pgfpathcurveto{\pgfqpoint{1.314650in}{0.725437in}}{\pgfqpoint{1.309064in}{0.723123in}}{\pgfqpoint{1.304945in}{0.719005in}}%
\pgfpathcurveto{\pgfqpoint{1.300827in}{0.714887in}}{\pgfqpoint{1.298513in}{0.709300in}}{\pgfqpoint{1.298513in}{0.703476in}}%
\pgfpathcurveto{\pgfqpoint{1.298513in}{0.697653in}}{\pgfqpoint{1.300827in}{0.692066in}}{\pgfqpoint{1.304945in}{0.687948in}}%
\pgfpathcurveto{\pgfqpoint{1.309064in}{0.683830in}}{\pgfqpoint{1.314650in}{0.681516in}}{\pgfqpoint{1.320474in}{0.681516in}}%
\pgfpathclose%
\pgfusepath{stroke,fill}%
\end{pgfscope}%
\begin{pgfscope}%
\pgfpathrectangle{\pgfqpoint{0.211875in}{0.211875in}}{\pgfqpoint{1.313625in}{1.279725in}}%
\pgfusepath{clip}%
\pgfsetbuttcap%
\pgfsetroundjoin%
\definecolor{currentfill}{rgb}{0.121569,0.466667,0.705882}%
\pgfsetfillcolor{currentfill}%
\pgfsetlinewidth{1.003750pt}%
\definecolor{currentstroke}{rgb}{0.121569,0.466667,0.705882}%
\pgfsetstrokecolor{currentstroke}%
\pgfsetdash{}{0pt}%
\pgfpathmoveto{\pgfqpoint{1.123954in}{0.271459in}}%
\pgfpathcurveto{\pgfqpoint{1.129778in}{0.271459in}}{\pgfqpoint{1.135364in}{0.273773in}}{\pgfqpoint{1.139482in}{0.277891in}}%
\pgfpathcurveto{\pgfqpoint{1.143600in}{0.282009in}}{\pgfqpoint{1.145914in}{0.287596in}}{\pgfqpoint{1.145914in}{0.293419in}}%
\pgfpathcurveto{\pgfqpoint{1.145914in}{0.299243in}}{\pgfqpoint{1.143600in}{0.304830in}}{\pgfqpoint{1.139482in}{0.308948in}}%
\pgfpathcurveto{\pgfqpoint{1.135364in}{0.313066in}}{\pgfqpoint{1.129778in}{0.315380in}}{\pgfqpoint{1.123954in}{0.315380in}}%
\pgfpathcurveto{\pgfqpoint{1.118130in}{0.315380in}}{\pgfqpoint{1.112544in}{0.313066in}}{\pgfqpoint{1.108426in}{0.308948in}}%
\pgfpathcurveto{\pgfqpoint{1.104308in}{0.304830in}}{\pgfqpoint{1.101994in}{0.299243in}}{\pgfqpoint{1.101994in}{0.293419in}}%
\pgfpathcurveto{\pgfqpoint{1.101994in}{0.287596in}}{\pgfqpoint{1.104308in}{0.282009in}}{\pgfqpoint{1.108426in}{0.277891in}}%
\pgfpathcurveto{\pgfqpoint{1.112544in}{0.273773in}}{\pgfqpoint{1.118130in}{0.271459in}}{\pgfqpoint{1.123954in}{0.271459in}}%
\pgfpathclose%
\pgfusepath{stroke,fill}%
\end{pgfscope}%
\begin{pgfscope}%
\pgfpathrectangle{\pgfqpoint{0.211875in}{0.211875in}}{\pgfqpoint{1.313625in}{1.279725in}}%
\pgfusepath{clip}%
\pgfsetbuttcap%
\pgfsetroundjoin%
\definecolor{currentfill}{rgb}{0.121569,0.466667,0.705882}%
\pgfsetfillcolor{currentfill}%
\pgfsetlinewidth{1.003750pt}%
\definecolor{currentstroke}{rgb}{0.121569,0.466667,0.705882}%
\pgfsetstrokecolor{currentstroke}%
\pgfsetdash{}{0pt}%
\pgfpathmoveto{\pgfqpoint{1.084144in}{0.391891in}}%
\pgfpathcurveto{\pgfqpoint{1.089968in}{0.391891in}}{\pgfqpoint{1.095555in}{0.394205in}}{\pgfqpoint{1.099673in}{0.398323in}}%
\pgfpathcurveto{\pgfqpoint{1.103791in}{0.402441in}}{\pgfqpoint{1.106105in}{0.408028in}}{\pgfqpoint{1.106105in}{0.413851in}}%
\pgfpathcurveto{\pgfqpoint{1.106105in}{0.419675in}}{\pgfqpoint{1.103791in}{0.425262in}}{\pgfqpoint{1.099673in}{0.429380in}}%
\pgfpathcurveto{\pgfqpoint{1.095555in}{0.433498in}}{\pgfqpoint{1.089968in}{0.435812in}}{\pgfqpoint{1.084144in}{0.435812in}}%
\pgfpathcurveto{\pgfqpoint{1.078321in}{0.435812in}}{\pgfqpoint{1.072734in}{0.433498in}}{\pgfqpoint{1.068616in}{0.429380in}}%
\pgfpathcurveto{\pgfqpoint{1.064498in}{0.425262in}}{\pgfqpoint{1.062184in}{0.419675in}}{\pgfqpoint{1.062184in}{0.413851in}}%
\pgfpathcurveto{\pgfqpoint{1.062184in}{0.408028in}}{\pgfqpoint{1.064498in}{0.402441in}}{\pgfqpoint{1.068616in}{0.398323in}}%
\pgfpathcurveto{\pgfqpoint{1.072734in}{0.394205in}}{\pgfqpoint{1.078321in}{0.391891in}}{\pgfqpoint{1.084144in}{0.391891in}}%
\pgfpathclose%
\pgfusepath{stroke,fill}%
\end{pgfscope}%
\begin{pgfscope}%
\pgfpathrectangle{\pgfqpoint{0.211875in}{0.211875in}}{\pgfqpoint{1.313625in}{1.279725in}}%
\pgfusepath{clip}%
\pgfsetbuttcap%
\pgfsetroundjoin%
\definecolor{currentfill}{rgb}{0.121569,0.466667,0.705882}%
\pgfsetfillcolor{currentfill}%
\pgfsetlinewidth{1.003750pt}%
\definecolor{currentstroke}{rgb}{0.121569,0.466667,0.705882}%
\pgfsetstrokecolor{currentstroke}%
\pgfsetdash{}{0pt}%
\pgfpathmoveto{\pgfqpoint{0.640228in}{0.682578in}}%
\pgfpathcurveto{\pgfqpoint{0.646052in}{0.682578in}}{\pgfqpoint{0.651638in}{0.684892in}}{\pgfqpoint{0.655756in}{0.689010in}}%
\pgfpathcurveto{\pgfqpoint{0.659874in}{0.693128in}}{\pgfqpoint{0.662188in}{0.698714in}}{\pgfqpoint{0.662188in}{0.704538in}}%
\pgfpathcurveto{\pgfqpoint{0.662188in}{0.710362in}}{\pgfqpoint{0.659874in}{0.715948in}}{\pgfqpoint{0.655756in}{0.720066in}}%
\pgfpathcurveto{\pgfqpoint{0.651638in}{0.724185in}}{\pgfqpoint{0.646052in}{0.726498in}}{\pgfqpoint{0.640228in}{0.726498in}}%
\pgfpathcurveto{\pgfqpoint{0.634404in}{0.726498in}}{\pgfqpoint{0.628818in}{0.724185in}}{\pgfqpoint{0.624700in}{0.720066in}}%
\pgfpathcurveto{\pgfqpoint{0.620582in}{0.715948in}}{\pgfqpoint{0.618268in}{0.710362in}}{\pgfqpoint{0.618268in}{0.704538in}}%
\pgfpathcurveto{\pgfqpoint{0.618268in}{0.698714in}}{\pgfqpoint{0.620582in}{0.693128in}}{\pgfqpoint{0.624700in}{0.689010in}}%
\pgfpathcurveto{\pgfqpoint{0.628818in}{0.684892in}}{\pgfqpoint{0.634404in}{0.682578in}}{\pgfqpoint{0.640228in}{0.682578in}}%
\pgfpathclose%
\pgfusepath{stroke,fill}%
\end{pgfscope}%
\begin{pgfscope}%
\pgfpathrectangle{\pgfqpoint{0.211875in}{0.211875in}}{\pgfqpoint{1.313625in}{1.279725in}}%
\pgfusepath{clip}%
\pgfsetbuttcap%
\pgfsetroundjoin%
\definecolor{currentfill}{rgb}{0.121569,0.466667,0.705882}%
\pgfsetfillcolor{currentfill}%
\pgfsetlinewidth{1.003750pt}%
\definecolor{currentstroke}{rgb}{0.121569,0.466667,0.705882}%
\pgfsetstrokecolor{currentstroke}%
\pgfsetdash{}{0pt}%
\pgfpathmoveto{\pgfqpoint{0.653154in}{0.765852in}}%
\pgfpathcurveto{\pgfqpoint{0.658978in}{0.765852in}}{\pgfqpoint{0.664564in}{0.768166in}}{\pgfqpoint{0.668682in}{0.772284in}}%
\pgfpathcurveto{\pgfqpoint{0.672801in}{0.776402in}}{\pgfqpoint{0.675115in}{0.781988in}}{\pgfqpoint{0.675115in}{0.787812in}}%
\pgfpathcurveto{\pgfqpoint{0.675115in}{0.793636in}}{\pgfqpoint{0.672801in}{0.799222in}}{\pgfqpoint{0.668682in}{0.803340in}}%
\pgfpathcurveto{\pgfqpoint{0.664564in}{0.807459in}}{\pgfqpoint{0.658978in}{0.809772in}}{\pgfqpoint{0.653154in}{0.809772in}}%
\pgfpathcurveto{\pgfqpoint{0.647330in}{0.809772in}}{\pgfqpoint{0.641744in}{0.807459in}}{\pgfqpoint{0.637626in}{0.803340in}}%
\pgfpathcurveto{\pgfqpoint{0.633508in}{0.799222in}}{\pgfqpoint{0.631194in}{0.793636in}}{\pgfqpoint{0.631194in}{0.787812in}}%
\pgfpathcurveto{\pgfqpoint{0.631194in}{0.781988in}}{\pgfqpoint{0.633508in}{0.776402in}}{\pgfqpoint{0.637626in}{0.772284in}}%
\pgfpathcurveto{\pgfqpoint{0.641744in}{0.768166in}}{\pgfqpoint{0.647330in}{0.765852in}}{\pgfqpoint{0.653154in}{0.765852in}}%
\pgfpathclose%
\pgfusepath{stroke,fill}%
\end{pgfscope}%
\begin{pgfscope}%
\pgfpathrectangle{\pgfqpoint{0.211875in}{0.211875in}}{\pgfqpoint{1.313625in}{1.279725in}}%
\pgfusepath{clip}%
\pgfsetbuttcap%
\pgfsetroundjoin%
\definecolor{currentfill}{rgb}{0.121569,0.466667,0.705882}%
\pgfsetfillcolor{currentfill}%
\pgfsetlinewidth{1.003750pt}%
\definecolor{currentstroke}{rgb}{0.121569,0.466667,0.705882}%
\pgfsetstrokecolor{currentstroke}%
\pgfsetdash{}{0pt}%
\pgfpathmoveto{\pgfqpoint{1.222417in}{0.646175in}}%
\pgfpathcurveto{\pgfqpoint{1.228241in}{0.646175in}}{\pgfqpoint{1.233827in}{0.648488in}}{\pgfqpoint{1.237945in}{0.652607in}}%
\pgfpathcurveto{\pgfqpoint{1.242063in}{0.656725in}}{\pgfqpoint{1.244377in}{0.662311in}}{\pgfqpoint{1.244377in}{0.668135in}}%
\pgfpathcurveto{\pgfqpoint{1.244377in}{0.673959in}}{\pgfqpoint{1.242063in}{0.679545in}}{\pgfqpoint{1.237945in}{0.683663in}}%
\pgfpathcurveto{\pgfqpoint{1.233827in}{0.687781in}}{\pgfqpoint{1.228241in}{0.690095in}}{\pgfqpoint{1.222417in}{0.690095in}}%
\pgfpathcurveto{\pgfqpoint{1.216593in}{0.690095in}}{\pgfqpoint{1.211007in}{0.687781in}}{\pgfqpoint{1.206889in}{0.683663in}}%
\pgfpathcurveto{\pgfqpoint{1.202771in}{0.679545in}}{\pgfqpoint{1.200457in}{0.673959in}}{\pgfqpoint{1.200457in}{0.668135in}}%
\pgfpathcurveto{\pgfqpoint{1.200457in}{0.662311in}}{\pgfqpoint{1.202771in}{0.656725in}}{\pgfqpoint{1.206889in}{0.652607in}}%
\pgfpathcurveto{\pgfqpoint{1.211007in}{0.648488in}}{\pgfqpoint{1.216593in}{0.646175in}}{\pgfqpoint{1.222417in}{0.646175in}}%
\pgfpathclose%
\pgfusepath{stroke,fill}%
\end{pgfscope}%
\begin{pgfscope}%
\pgfpathrectangle{\pgfqpoint{0.211875in}{0.211875in}}{\pgfqpoint{1.313625in}{1.279725in}}%
\pgfusepath{clip}%
\pgfsetbuttcap%
\pgfsetroundjoin%
\definecolor{currentfill}{rgb}{0.121569,0.466667,0.705882}%
\pgfsetfillcolor{currentfill}%
\pgfsetlinewidth{1.003750pt}%
\definecolor{currentstroke}{rgb}{0.121569,0.466667,0.705882}%
\pgfsetstrokecolor{currentstroke}%
\pgfsetdash{}{0pt}%
\pgfpathmoveto{\pgfqpoint{1.410366in}{1.144997in}}%
\pgfpathcurveto{\pgfqpoint{1.416190in}{1.144997in}}{\pgfqpoint{1.421776in}{1.147311in}}{\pgfqpoint{1.425894in}{1.151429in}}%
\pgfpathcurveto{\pgfqpoint{1.430012in}{1.155547in}}{\pgfqpoint{1.432326in}{1.161133in}}{\pgfqpoint{1.432326in}{1.166957in}}%
\pgfpathcurveto{\pgfqpoint{1.432326in}{1.172781in}}{\pgfqpoint{1.430012in}{1.178367in}}{\pgfqpoint{1.425894in}{1.182485in}}%
\pgfpathcurveto{\pgfqpoint{1.421776in}{1.186604in}}{\pgfqpoint{1.416190in}{1.188917in}}{\pgfqpoint{1.410366in}{1.188917in}}%
\pgfpathcurveto{\pgfqpoint{1.404542in}{1.188917in}}{\pgfqpoint{1.398956in}{1.186604in}}{\pgfqpoint{1.394837in}{1.182485in}}%
\pgfpathcurveto{\pgfqpoint{1.390719in}{1.178367in}}{\pgfqpoint{1.388405in}{1.172781in}}{\pgfqpoint{1.388405in}{1.166957in}}%
\pgfpathcurveto{\pgfqpoint{1.388405in}{1.161133in}}{\pgfqpoint{1.390719in}{1.155547in}}{\pgfqpoint{1.394837in}{1.151429in}}%
\pgfpathcurveto{\pgfqpoint{1.398956in}{1.147311in}}{\pgfqpoint{1.404542in}{1.144997in}}{\pgfqpoint{1.410366in}{1.144997in}}%
\pgfpathclose%
\pgfusepath{stroke,fill}%
\end{pgfscope}%
\begin{pgfscope}%
\pgfpathrectangle{\pgfqpoint{0.211875in}{0.211875in}}{\pgfqpoint{1.313625in}{1.279725in}}%
\pgfusepath{clip}%
\pgfsetbuttcap%
\pgfsetroundjoin%
\definecolor{currentfill}{rgb}{0.121569,0.466667,0.705882}%
\pgfsetfillcolor{currentfill}%
\pgfsetlinewidth{1.003750pt}%
\definecolor{currentstroke}{rgb}{0.121569,0.466667,0.705882}%
\pgfsetstrokecolor{currentstroke}%
\pgfsetdash{}{0pt}%
\pgfpathmoveto{\pgfqpoint{1.198169in}{0.303845in}}%
\pgfpathcurveto{\pgfqpoint{1.203993in}{0.303845in}}{\pgfqpoint{1.209579in}{0.306159in}}{\pgfqpoint{1.213697in}{0.310277in}}%
\pgfpathcurveto{\pgfqpoint{1.217815in}{0.314395in}}{\pgfqpoint{1.220129in}{0.319981in}}{\pgfqpoint{1.220129in}{0.325805in}}%
\pgfpathcurveto{\pgfqpoint{1.220129in}{0.331629in}}{\pgfqpoint{1.217815in}{0.337215in}}{\pgfqpoint{1.213697in}{0.341333in}}%
\pgfpathcurveto{\pgfqpoint{1.209579in}{0.345451in}}{\pgfqpoint{1.203993in}{0.347765in}}{\pgfqpoint{1.198169in}{0.347765in}}%
\pgfpathcurveto{\pgfqpoint{1.192345in}{0.347765in}}{\pgfqpoint{1.186759in}{0.345451in}}{\pgfqpoint{1.182641in}{0.341333in}}%
\pgfpathcurveto{\pgfqpoint{1.178523in}{0.337215in}}{\pgfqpoint{1.176209in}{0.331629in}}{\pgfqpoint{1.176209in}{0.325805in}}%
\pgfpathcurveto{\pgfqpoint{1.176209in}{0.319981in}}{\pgfqpoint{1.178523in}{0.314395in}}{\pgfqpoint{1.182641in}{0.310277in}}%
\pgfpathcurveto{\pgfqpoint{1.186759in}{0.306159in}}{\pgfqpoint{1.192345in}{0.303845in}}{\pgfqpoint{1.198169in}{0.303845in}}%
\pgfpathclose%
\pgfusepath{stroke,fill}%
\end{pgfscope}%
\begin{pgfscope}%
\pgfpathrectangle{\pgfqpoint{0.211875in}{0.211875in}}{\pgfqpoint{1.313625in}{1.279725in}}%
\pgfusepath{clip}%
\pgfsetbuttcap%
\pgfsetroundjoin%
\definecolor{currentfill}{rgb}{0.121569,0.466667,0.705882}%
\pgfsetfillcolor{currentfill}%
\pgfsetlinewidth{1.003750pt}%
\definecolor{currentstroke}{rgb}{0.121569,0.466667,0.705882}%
\pgfsetstrokecolor{currentstroke}%
\pgfsetdash{}{0pt}%
\pgfpathmoveto{\pgfqpoint{1.222322in}{0.662352in}}%
\pgfpathcurveto{\pgfqpoint{1.228146in}{0.662352in}}{\pgfqpoint{1.233732in}{0.664666in}}{\pgfqpoint{1.237850in}{0.668784in}}%
\pgfpathcurveto{\pgfqpoint{1.241968in}{0.672902in}}{\pgfqpoint{1.244282in}{0.678488in}}{\pgfqpoint{1.244282in}{0.684312in}}%
\pgfpathcurveto{\pgfqpoint{1.244282in}{0.690136in}}{\pgfqpoint{1.241968in}{0.695722in}}{\pgfqpoint{1.237850in}{0.699840in}}%
\pgfpathcurveto{\pgfqpoint{1.233732in}{0.703958in}}{\pgfqpoint{1.228146in}{0.706272in}}{\pgfqpoint{1.222322in}{0.706272in}}%
\pgfpathcurveto{\pgfqpoint{1.216498in}{0.706272in}}{\pgfqpoint{1.210912in}{0.703958in}}{\pgfqpoint{1.206794in}{0.699840in}}%
\pgfpathcurveto{\pgfqpoint{1.202675in}{0.695722in}}{\pgfqpoint{1.200361in}{0.690136in}}{\pgfqpoint{1.200361in}{0.684312in}}%
\pgfpathcurveto{\pgfqpoint{1.200361in}{0.678488in}}{\pgfqpoint{1.202675in}{0.672902in}}{\pgfqpoint{1.206794in}{0.668784in}}%
\pgfpathcurveto{\pgfqpoint{1.210912in}{0.664666in}}{\pgfqpoint{1.216498in}{0.662352in}}{\pgfqpoint{1.222322in}{0.662352in}}%
\pgfpathclose%
\pgfusepath{stroke,fill}%
\end{pgfscope}%
\begin{pgfscope}%
\pgfpathrectangle{\pgfqpoint{0.211875in}{0.211875in}}{\pgfqpoint{1.313625in}{1.279725in}}%
\pgfusepath{clip}%
\pgfsetbuttcap%
\pgfsetroundjoin%
\definecolor{currentfill}{rgb}{0.121569,0.466667,0.705882}%
\pgfsetfillcolor{currentfill}%
\pgfsetlinewidth{1.003750pt}%
\definecolor{currentstroke}{rgb}{0.121569,0.466667,0.705882}%
\pgfsetstrokecolor{currentstroke}%
\pgfsetdash{}{0pt}%
\pgfpathmoveto{\pgfqpoint{1.312482in}{0.662283in}}%
\pgfpathcurveto{\pgfqpoint{1.318306in}{0.662283in}}{\pgfqpoint{1.323892in}{0.664597in}}{\pgfqpoint{1.328010in}{0.668715in}}%
\pgfpathcurveto{\pgfqpoint{1.332128in}{0.672833in}}{\pgfqpoint{1.334442in}{0.678419in}}{\pgfqpoint{1.334442in}{0.684243in}}%
\pgfpathcurveto{\pgfqpoint{1.334442in}{0.690067in}}{\pgfqpoint{1.332128in}{0.695653in}}{\pgfqpoint{1.328010in}{0.699771in}}%
\pgfpathcurveto{\pgfqpoint{1.323892in}{0.703889in}}{\pgfqpoint{1.318306in}{0.706203in}}{\pgfqpoint{1.312482in}{0.706203in}}%
\pgfpathcurveto{\pgfqpoint{1.306658in}{0.706203in}}{\pgfqpoint{1.301072in}{0.703889in}}{\pgfqpoint{1.296953in}{0.699771in}}%
\pgfpathcurveto{\pgfqpoint{1.292835in}{0.695653in}}{\pgfqpoint{1.290521in}{0.690067in}}{\pgfqpoint{1.290521in}{0.684243in}}%
\pgfpathcurveto{\pgfqpoint{1.290521in}{0.678419in}}{\pgfqpoint{1.292835in}{0.672833in}}{\pgfqpoint{1.296953in}{0.668715in}}%
\pgfpathcurveto{\pgfqpoint{1.301072in}{0.664597in}}{\pgfqpoint{1.306658in}{0.662283in}}{\pgfqpoint{1.312482in}{0.662283in}}%
\pgfpathclose%
\pgfusepath{stroke,fill}%
\end{pgfscope}%
\begin{pgfscope}%
\pgfpathrectangle{\pgfqpoint{0.211875in}{0.211875in}}{\pgfqpoint{1.313625in}{1.279725in}}%
\pgfusepath{clip}%
\pgfsetbuttcap%
\pgfsetroundjoin%
\definecolor{currentfill}{rgb}{0.121569,0.466667,0.705882}%
\pgfsetfillcolor{currentfill}%
\pgfsetlinewidth{1.003750pt}%
\definecolor{currentstroke}{rgb}{0.121569,0.466667,0.705882}%
\pgfsetstrokecolor{currentstroke}%
\pgfsetdash{}{0pt}%
\pgfpathmoveto{\pgfqpoint{1.406787in}{1.143755in}}%
\pgfpathcurveto{\pgfqpoint{1.412610in}{1.143755in}}{\pgfqpoint{1.418197in}{1.146069in}}{\pgfqpoint{1.422315in}{1.150187in}}%
\pgfpathcurveto{\pgfqpoint{1.426433in}{1.154305in}}{\pgfqpoint{1.428747in}{1.159891in}}{\pgfqpoint{1.428747in}{1.165715in}}%
\pgfpathcurveto{\pgfqpoint{1.428747in}{1.171539in}}{\pgfqpoint{1.426433in}{1.177125in}}{\pgfqpoint{1.422315in}{1.181243in}}%
\pgfpathcurveto{\pgfqpoint{1.418197in}{1.185361in}}{\pgfqpoint{1.412610in}{1.187675in}}{\pgfqpoint{1.406787in}{1.187675in}}%
\pgfpathcurveto{\pgfqpoint{1.400963in}{1.187675in}}{\pgfqpoint{1.395376in}{1.185361in}}{\pgfqpoint{1.391258in}{1.181243in}}%
\pgfpathcurveto{\pgfqpoint{1.387140in}{1.177125in}}{\pgfqpoint{1.384826in}{1.171539in}}{\pgfqpoint{1.384826in}{1.165715in}}%
\pgfpathcurveto{\pgfqpoint{1.384826in}{1.159891in}}{\pgfqpoint{1.387140in}{1.154305in}}{\pgfqpoint{1.391258in}{1.150187in}}%
\pgfpathcurveto{\pgfqpoint{1.395376in}{1.146069in}}{\pgfqpoint{1.400963in}{1.143755in}}{\pgfqpoint{1.406787in}{1.143755in}}%
\pgfpathclose%
\pgfusepath{stroke,fill}%
\end{pgfscope}%
\begin{pgfscope}%
\pgfpathrectangle{\pgfqpoint{0.211875in}{0.211875in}}{\pgfqpoint{1.313625in}{1.279725in}}%
\pgfusepath{clip}%
\pgfsetbuttcap%
\pgfsetroundjoin%
\definecolor{currentfill}{rgb}{0.121569,0.466667,0.705882}%
\pgfsetfillcolor{currentfill}%
\pgfsetlinewidth{1.003750pt}%
\definecolor{currentstroke}{rgb}{0.121569,0.466667,0.705882}%
\pgfsetstrokecolor{currentstroke}%
\pgfsetdash{}{0pt}%
\pgfpathmoveto{\pgfqpoint{0.623433in}{0.791989in}}%
\pgfpathcurveto{\pgfqpoint{0.629257in}{0.791989in}}{\pgfqpoint{0.634843in}{0.794303in}}{\pgfqpoint{0.638961in}{0.798421in}}%
\pgfpathcurveto{\pgfqpoint{0.643079in}{0.802539in}}{\pgfqpoint{0.645393in}{0.808125in}}{\pgfqpoint{0.645393in}{0.813949in}}%
\pgfpathcurveto{\pgfqpoint{0.645393in}{0.819773in}}{\pgfqpoint{0.643079in}{0.825359in}}{\pgfqpoint{0.638961in}{0.829477in}}%
\pgfpathcurveto{\pgfqpoint{0.634843in}{0.833596in}}{\pgfqpoint{0.629257in}{0.835909in}}{\pgfqpoint{0.623433in}{0.835909in}}%
\pgfpathcurveto{\pgfqpoint{0.617609in}{0.835909in}}{\pgfqpoint{0.612022in}{0.833596in}}{\pgfqpoint{0.607904in}{0.829477in}}%
\pgfpathcurveto{\pgfqpoint{0.603786in}{0.825359in}}{\pgfqpoint{0.601472in}{0.819773in}}{\pgfqpoint{0.601472in}{0.813949in}}%
\pgfpathcurveto{\pgfqpoint{0.601472in}{0.808125in}}{\pgfqpoint{0.603786in}{0.802539in}}{\pgfqpoint{0.607904in}{0.798421in}}%
\pgfpathcurveto{\pgfqpoint{0.612022in}{0.794303in}}{\pgfqpoint{0.617609in}{0.791989in}}{\pgfqpoint{0.623433in}{0.791989in}}%
\pgfpathclose%
\pgfusepath{stroke,fill}%
\end{pgfscope}%
\begin{pgfscope}%
\pgfpathrectangle{\pgfqpoint{0.211875in}{0.211875in}}{\pgfqpoint{1.313625in}{1.279725in}}%
\pgfusepath{clip}%
\pgfsetbuttcap%
\pgfsetroundjoin%
\definecolor{currentfill}{rgb}{0.121569,0.466667,0.705882}%
\pgfsetfillcolor{currentfill}%
\pgfsetlinewidth{1.003750pt}%
\definecolor{currentstroke}{rgb}{0.121569,0.466667,0.705882}%
\pgfsetstrokecolor{currentstroke}%
\pgfsetdash{}{0pt}%
\pgfpathmoveto{\pgfqpoint{1.429793in}{0.812467in}}%
\pgfpathcurveto{\pgfqpoint{1.435617in}{0.812467in}}{\pgfqpoint{1.441203in}{0.814780in}}{\pgfqpoint{1.445321in}{0.818899in}}%
\pgfpathcurveto{\pgfqpoint{1.449439in}{0.823017in}}{\pgfqpoint{1.451753in}{0.828603in}}{\pgfqpoint{1.451753in}{0.834427in}}%
\pgfpathcurveto{\pgfqpoint{1.451753in}{0.840251in}}{\pgfqpoint{1.449439in}{0.845837in}}{\pgfqpoint{1.445321in}{0.849955in}}%
\pgfpathcurveto{\pgfqpoint{1.441203in}{0.854073in}}{\pgfqpoint{1.435617in}{0.856387in}}{\pgfqpoint{1.429793in}{0.856387in}}%
\pgfpathcurveto{\pgfqpoint{1.423969in}{0.856387in}}{\pgfqpoint{1.418382in}{0.854073in}}{\pgfqpoint{1.414264in}{0.849955in}}%
\pgfpathcurveto{\pgfqpoint{1.410146in}{0.845837in}}{\pgfqpoint{1.407832in}{0.840251in}}{\pgfqpoint{1.407832in}{0.834427in}}%
\pgfpathcurveto{\pgfqpoint{1.407832in}{0.828603in}}{\pgfqpoint{1.410146in}{0.823017in}}{\pgfqpoint{1.414264in}{0.818899in}}%
\pgfpathcurveto{\pgfqpoint{1.418382in}{0.814780in}}{\pgfqpoint{1.423969in}{0.812467in}}{\pgfqpoint{1.429793in}{0.812467in}}%
\pgfpathclose%
\pgfusepath{stroke,fill}%
\end{pgfscope}%
\begin{pgfscope}%
\pgfpathrectangle{\pgfqpoint{0.211875in}{0.211875in}}{\pgfqpoint{1.313625in}{1.279725in}}%
\pgfusepath{clip}%
\pgfsetbuttcap%
\pgfsetroundjoin%
\definecolor{currentfill}{rgb}{0.121569,0.466667,0.705882}%
\pgfsetfillcolor{currentfill}%
\pgfsetlinewidth{1.003750pt}%
\definecolor{currentstroke}{rgb}{0.121569,0.466667,0.705882}%
\pgfsetstrokecolor{currentstroke}%
\pgfsetdash{}{0pt}%
\pgfpathmoveto{\pgfqpoint{1.413840in}{0.687048in}}%
\pgfpathcurveto{\pgfqpoint{1.419664in}{0.687048in}}{\pgfqpoint{1.425251in}{0.689362in}}{\pgfqpoint{1.429369in}{0.693480in}}%
\pgfpathcurveto{\pgfqpoint{1.433487in}{0.697598in}}{\pgfqpoint{1.435801in}{0.703184in}}{\pgfqpoint{1.435801in}{0.709008in}}%
\pgfpathcurveto{\pgfqpoint{1.435801in}{0.714832in}}{\pgfqpoint{1.433487in}{0.720418in}}{\pgfqpoint{1.429369in}{0.724536in}}%
\pgfpathcurveto{\pgfqpoint{1.425251in}{0.728655in}}{\pgfqpoint{1.419664in}{0.730968in}}{\pgfqpoint{1.413840in}{0.730968in}}%
\pgfpathcurveto{\pgfqpoint{1.408016in}{0.730968in}}{\pgfqpoint{1.402430in}{0.728655in}}{\pgfqpoint{1.398312in}{0.724536in}}%
\pgfpathcurveto{\pgfqpoint{1.394194in}{0.720418in}}{\pgfqpoint{1.391880in}{0.714832in}}{\pgfqpoint{1.391880in}{0.709008in}}%
\pgfpathcurveto{\pgfqpoint{1.391880in}{0.703184in}}{\pgfqpoint{1.394194in}{0.697598in}}{\pgfqpoint{1.398312in}{0.693480in}}%
\pgfpathcurveto{\pgfqpoint{1.402430in}{0.689362in}}{\pgfqpoint{1.408016in}{0.687048in}}{\pgfqpoint{1.413840in}{0.687048in}}%
\pgfpathclose%
\pgfusepath{stroke,fill}%
\end{pgfscope}%
\begin{pgfscope}%
\pgfpathrectangle{\pgfqpoint{0.211875in}{0.211875in}}{\pgfqpoint{1.313625in}{1.279725in}}%
\pgfusepath{clip}%
\pgfsetbuttcap%
\pgfsetroundjoin%
\definecolor{currentfill}{rgb}{0.121569,0.466667,0.705882}%
\pgfsetfillcolor{currentfill}%
\pgfsetlinewidth{1.003750pt}%
\definecolor{currentstroke}{rgb}{0.121569,0.466667,0.705882}%
\pgfsetstrokecolor{currentstroke}%
\pgfsetdash{}{0pt}%
\pgfpathmoveto{\pgfqpoint{1.443582in}{0.837166in}}%
\pgfpathcurveto{\pgfqpoint{1.449406in}{0.837166in}}{\pgfqpoint{1.454992in}{0.839480in}}{\pgfqpoint{1.459110in}{0.843598in}}%
\pgfpathcurveto{\pgfqpoint{1.463228in}{0.847716in}}{\pgfqpoint{1.465542in}{0.853303in}}{\pgfqpoint{1.465542in}{0.859126in}}%
\pgfpathcurveto{\pgfqpoint{1.465542in}{0.864950in}}{\pgfqpoint{1.463228in}{0.870537in}}{\pgfqpoint{1.459110in}{0.874655in}}%
\pgfpathcurveto{\pgfqpoint{1.454992in}{0.878773in}}{\pgfqpoint{1.449406in}{0.881087in}}{\pgfqpoint{1.443582in}{0.881087in}}%
\pgfpathcurveto{\pgfqpoint{1.437758in}{0.881087in}}{\pgfqpoint{1.432172in}{0.878773in}}{\pgfqpoint{1.428054in}{0.874655in}}%
\pgfpathcurveto{\pgfqpoint{1.423936in}{0.870537in}}{\pgfqpoint{1.421622in}{0.864950in}}{\pgfqpoint{1.421622in}{0.859126in}}%
\pgfpathcurveto{\pgfqpoint{1.421622in}{0.853303in}}{\pgfqpoint{1.423936in}{0.847716in}}{\pgfqpoint{1.428054in}{0.843598in}}%
\pgfpathcurveto{\pgfqpoint{1.432172in}{0.839480in}}{\pgfqpoint{1.437758in}{0.837166in}}{\pgfqpoint{1.443582in}{0.837166in}}%
\pgfpathclose%
\pgfusepath{stroke,fill}%
\end{pgfscope}%
\begin{pgfscope}%
\pgfpathrectangle{\pgfqpoint{0.211875in}{0.211875in}}{\pgfqpoint{1.313625in}{1.279725in}}%
\pgfusepath{clip}%
\pgfsetbuttcap%
\pgfsetroundjoin%
\definecolor{currentfill}{rgb}{0.121569,0.466667,0.705882}%
\pgfsetfillcolor{currentfill}%
\pgfsetlinewidth{1.003750pt}%
\definecolor{currentstroke}{rgb}{0.121569,0.466667,0.705882}%
\pgfsetstrokecolor{currentstroke}%
\pgfsetdash{}{0pt}%
\pgfpathmoveto{\pgfqpoint{1.316076in}{0.668947in}}%
\pgfpathcurveto{\pgfqpoint{1.321900in}{0.668947in}}{\pgfqpoint{1.327486in}{0.671261in}}{\pgfqpoint{1.331604in}{0.675379in}}%
\pgfpathcurveto{\pgfqpoint{1.335722in}{0.679497in}}{\pgfqpoint{1.338036in}{0.685083in}}{\pgfqpoint{1.338036in}{0.690907in}}%
\pgfpathcurveto{\pgfqpoint{1.338036in}{0.696731in}}{\pgfqpoint{1.335722in}{0.702317in}}{\pgfqpoint{1.331604in}{0.706435in}}%
\pgfpathcurveto{\pgfqpoint{1.327486in}{0.710553in}}{\pgfqpoint{1.321900in}{0.712867in}}{\pgfqpoint{1.316076in}{0.712867in}}%
\pgfpathcurveto{\pgfqpoint{1.310252in}{0.712867in}}{\pgfqpoint{1.304666in}{0.710553in}}{\pgfqpoint{1.300547in}{0.706435in}}%
\pgfpathcurveto{\pgfqpoint{1.296429in}{0.702317in}}{\pgfqpoint{1.294115in}{0.696731in}}{\pgfqpoint{1.294115in}{0.690907in}}%
\pgfpathcurveto{\pgfqpoint{1.294115in}{0.685083in}}{\pgfqpoint{1.296429in}{0.679497in}}{\pgfqpoint{1.300547in}{0.675379in}}%
\pgfpathcurveto{\pgfqpoint{1.304666in}{0.671261in}}{\pgfqpoint{1.310252in}{0.668947in}}{\pgfqpoint{1.316076in}{0.668947in}}%
\pgfpathclose%
\pgfusepath{stroke,fill}%
\end{pgfscope}%
\begin{pgfscope}%
\pgfpathrectangle{\pgfqpoint{0.211875in}{0.211875in}}{\pgfqpoint{1.313625in}{1.279725in}}%
\pgfusepath{clip}%
\pgfsetbuttcap%
\pgfsetroundjoin%
\definecolor{currentfill}{rgb}{0.121569,0.466667,0.705882}%
\pgfsetfillcolor{currentfill}%
\pgfsetlinewidth{1.003750pt}%
\definecolor{currentstroke}{rgb}{0.121569,0.466667,0.705882}%
\pgfsetstrokecolor{currentstroke}%
\pgfsetdash{}{0pt}%
\pgfpathmoveto{\pgfqpoint{0.726153in}{0.677145in}}%
\pgfpathcurveto{\pgfqpoint{0.731977in}{0.677145in}}{\pgfqpoint{0.737563in}{0.679459in}}{\pgfqpoint{0.741681in}{0.683577in}}%
\pgfpathcurveto{\pgfqpoint{0.745799in}{0.687696in}}{\pgfqpoint{0.748113in}{0.693282in}}{\pgfqpoint{0.748113in}{0.699106in}}%
\pgfpathcurveto{\pgfqpoint{0.748113in}{0.704930in}}{\pgfqpoint{0.745799in}{0.710516in}}{\pgfqpoint{0.741681in}{0.714634in}}%
\pgfpathcurveto{\pgfqpoint{0.737563in}{0.718752in}}{\pgfqpoint{0.731977in}{0.721066in}}{\pgfqpoint{0.726153in}{0.721066in}}%
\pgfpathcurveto{\pgfqpoint{0.720329in}{0.721066in}}{\pgfqpoint{0.714743in}{0.718752in}}{\pgfqpoint{0.710625in}{0.714634in}}%
\pgfpathcurveto{\pgfqpoint{0.706506in}{0.710516in}}{\pgfqpoint{0.704193in}{0.704930in}}{\pgfqpoint{0.704193in}{0.699106in}}%
\pgfpathcurveto{\pgfqpoint{0.704193in}{0.693282in}}{\pgfqpoint{0.706506in}{0.687696in}}{\pgfqpoint{0.710625in}{0.683577in}}%
\pgfpathcurveto{\pgfqpoint{0.714743in}{0.679459in}}{\pgfqpoint{0.720329in}{0.677145in}}{\pgfqpoint{0.726153in}{0.677145in}}%
\pgfpathclose%
\pgfusepath{stroke,fill}%
\end{pgfscope}%
\begin{pgfscope}%
\pgfpathrectangle{\pgfqpoint{0.211875in}{0.211875in}}{\pgfqpoint{1.313625in}{1.279725in}}%
\pgfusepath{clip}%
\pgfsetbuttcap%
\pgfsetroundjoin%
\definecolor{currentfill}{rgb}{0.121569,0.466667,0.705882}%
\pgfsetfillcolor{currentfill}%
\pgfsetlinewidth{1.003750pt}%
\definecolor{currentstroke}{rgb}{0.121569,0.466667,0.705882}%
\pgfsetstrokecolor{currentstroke}%
\pgfsetdash{}{0pt}%
\pgfpathmoveto{\pgfqpoint{1.434180in}{0.823555in}}%
\pgfpathcurveto{\pgfqpoint{1.440004in}{0.823555in}}{\pgfqpoint{1.445590in}{0.825869in}}{\pgfqpoint{1.449708in}{0.829987in}}%
\pgfpathcurveto{\pgfqpoint{1.453827in}{0.834105in}}{\pgfqpoint{1.456140in}{0.839691in}}{\pgfqpoint{1.456140in}{0.845515in}}%
\pgfpathcurveto{\pgfqpoint{1.456140in}{0.851339in}}{\pgfqpoint{1.453827in}{0.856925in}}{\pgfqpoint{1.449708in}{0.861043in}}%
\pgfpathcurveto{\pgfqpoint{1.445590in}{0.865162in}}{\pgfqpoint{1.440004in}{0.867475in}}{\pgfqpoint{1.434180in}{0.867475in}}%
\pgfpathcurveto{\pgfqpoint{1.428356in}{0.867475in}}{\pgfqpoint{1.422770in}{0.865162in}}{\pgfqpoint{1.418652in}{0.861043in}}%
\pgfpathcurveto{\pgfqpoint{1.414534in}{0.856925in}}{\pgfqpoint{1.412220in}{0.851339in}}{\pgfqpoint{1.412220in}{0.845515in}}%
\pgfpathcurveto{\pgfqpoint{1.412220in}{0.839691in}}{\pgfqpoint{1.414534in}{0.834105in}}{\pgfqpoint{1.418652in}{0.829987in}}%
\pgfpathcurveto{\pgfqpoint{1.422770in}{0.825869in}}{\pgfqpoint{1.428356in}{0.823555in}}{\pgfqpoint{1.434180in}{0.823555in}}%
\pgfpathclose%
\pgfusepath{stroke,fill}%
\end{pgfscope}%
\begin{pgfscope}%
\pgfpathrectangle{\pgfqpoint{0.211875in}{0.211875in}}{\pgfqpoint{1.313625in}{1.279725in}}%
\pgfusepath{clip}%
\pgfsetbuttcap%
\pgfsetroundjoin%
\definecolor{currentfill}{rgb}{0.121569,0.466667,0.705882}%
\pgfsetfillcolor{currentfill}%
\pgfsetlinewidth{1.003750pt}%
\definecolor{currentstroke}{rgb}{0.121569,0.466667,0.705882}%
\pgfsetstrokecolor{currentstroke}%
\pgfsetdash{}{0pt}%
\pgfpathmoveto{\pgfqpoint{0.752920in}{0.577645in}}%
\pgfpathcurveto{\pgfqpoint{0.758743in}{0.577645in}}{\pgfqpoint{0.764330in}{0.579959in}}{\pgfqpoint{0.768448in}{0.584077in}}%
\pgfpathcurveto{\pgfqpoint{0.772566in}{0.588195in}}{\pgfqpoint{0.774880in}{0.593781in}}{\pgfqpoint{0.774880in}{0.599605in}}%
\pgfpathcurveto{\pgfqpoint{0.774880in}{0.605429in}}{\pgfqpoint{0.772566in}{0.611015in}}{\pgfqpoint{0.768448in}{0.615134in}}%
\pgfpathcurveto{\pgfqpoint{0.764330in}{0.619252in}}{\pgfqpoint{0.758743in}{0.621566in}}{\pgfqpoint{0.752920in}{0.621566in}}%
\pgfpathcurveto{\pgfqpoint{0.747096in}{0.621566in}}{\pgfqpoint{0.741509in}{0.619252in}}{\pgfqpoint{0.737391in}{0.615134in}}%
\pgfpathcurveto{\pgfqpoint{0.733273in}{0.611015in}}{\pgfqpoint{0.730959in}{0.605429in}}{\pgfqpoint{0.730959in}{0.599605in}}%
\pgfpathcurveto{\pgfqpoint{0.730959in}{0.593781in}}{\pgfqpoint{0.733273in}{0.588195in}}{\pgfqpoint{0.737391in}{0.584077in}}%
\pgfpathcurveto{\pgfqpoint{0.741509in}{0.579959in}}{\pgfqpoint{0.747096in}{0.577645in}}{\pgfqpoint{0.752920in}{0.577645in}}%
\pgfpathclose%
\pgfusepath{stroke,fill}%
\end{pgfscope}%
\begin{pgfscope}%
\pgfpathrectangle{\pgfqpoint{0.211875in}{0.211875in}}{\pgfqpoint{1.313625in}{1.279725in}}%
\pgfusepath{clip}%
\pgfsetbuttcap%
\pgfsetroundjoin%
\definecolor{currentfill}{rgb}{0.121569,0.466667,0.705882}%
\pgfsetfillcolor{currentfill}%
\pgfsetlinewidth{1.003750pt}%
\definecolor{currentstroke}{rgb}{0.121569,0.466667,0.705882}%
\pgfsetstrokecolor{currentstroke}%
\pgfsetdash{}{0pt}%
\pgfpathmoveto{\pgfqpoint{1.208967in}{0.668055in}}%
\pgfpathcurveto{\pgfqpoint{1.214791in}{0.668055in}}{\pgfqpoint{1.220377in}{0.670369in}}{\pgfqpoint{1.224495in}{0.674487in}}%
\pgfpathcurveto{\pgfqpoint{1.228613in}{0.678605in}}{\pgfqpoint{1.230927in}{0.684191in}}{\pgfqpoint{1.230927in}{0.690015in}}%
\pgfpathcurveto{\pgfqpoint{1.230927in}{0.695839in}}{\pgfqpoint{1.228613in}{0.701425in}}{\pgfqpoint{1.224495in}{0.705543in}}%
\pgfpathcurveto{\pgfqpoint{1.220377in}{0.709661in}}{\pgfqpoint{1.214791in}{0.711975in}}{\pgfqpoint{1.208967in}{0.711975in}}%
\pgfpathcurveto{\pgfqpoint{1.203143in}{0.711975in}}{\pgfqpoint{1.197557in}{0.709661in}}{\pgfqpoint{1.193439in}{0.705543in}}%
\pgfpathcurveto{\pgfqpoint{1.189320in}{0.701425in}}{\pgfqpoint{1.187007in}{0.695839in}}{\pgfqpoint{1.187007in}{0.690015in}}%
\pgfpathcurveto{\pgfqpoint{1.187007in}{0.684191in}}{\pgfqpoint{1.189320in}{0.678605in}}{\pgfqpoint{1.193439in}{0.674487in}}%
\pgfpathcurveto{\pgfqpoint{1.197557in}{0.670369in}}{\pgfqpoint{1.203143in}{0.668055in}}{\pgfqpoint{1.208967in}{0.668055in}}%
\pgfpathclose%
\pgfusepath{stroke,fill}%
\end{pgfscope}%
\begin{pgfscope}%
\pgfpathrectangle{\pgfqpoint{0.211875in}{0.211875in}}{\pgfqpoint{1.313625in}{1.279725in}}%
\pgfusepath{clip}%
\pgfsetbuttcap%
\pgfsetroundjoin%
\definecolor{currentfill}{rgb}{0.121569,0.466667,0.705882}%
\pgfsetfillcolor{currentfill}%
\pgfsetlinewidth{1.003750pt}%
\definecolor{currentstroke}{rgb}{0.121569,0.466667,0.705882}%
\pgfsetstrokecolor{currentstroke}%
\pgfsetdash{}{0pt}%
\pgfpathmoveto{\pgfqpoint{0.737597in}{0.573552in}}%
\pgfpathcurveto{\pgfqpoint{0.743421in}{0.573552in}}{\pgfqpoint{0.749008in}{0.575866in}}{\pgfqpoint{0.753126in}{0.579984in}}%
\pgfpathcurveto{\pgfqpoint{0.757244in}{0.584102in}}{\pgfqpoint{0.759558in}{0.589689in}}{\pgfqpoint{0.759558in}{0.595513in}}%
\pgfpathcurveto{\pgfqpoint{0.759558in}{0.601337in}}{\pgfqpoint{0.757244in}{0.606923in}}{\pgfqpoint{0.753126in}{0.611041in}}%
\pgfpathcurveto{\pgfqpoint{0.749008in}{0.615159in}}{\pgfqpoint{0.743421in}{0.617473in}}{\pgfqpoint{0.737597in}{0.617473in}}%
\pgfpathcurveto{\pgfqpoint{0.731774in}{0.617473in}}{\pgfqpoint{0.726187in}{0.615159in}}{\pgfqpoint{0.722069in}{0.611041in}}%
\pgfpathcurveto{\pgfqpoint{0.717951in}{0.606923in}}{\pgfqpoint{0.715637in}{0.601337in}}{\pgfqpoint{0.715637in}{0.595513in}}%
\pgfpathcurveto{\pgfqpoint{0.715637in}{0.589689in}}{\pgfqpoint{0.717951in}{0.584102in}}{\pgfqpoint{0.722069in}{0.579984in}}%
\pgfpathcurveto{\pgfqpoint{0.726187in}{0.575866in}}{\pgfqpoint{0.731774in}{0.573552in}}{\pgfqpoint{0.737597in}{0.573552in}}%
\pgfpathclose%
\pgfusepath{stroke,fill}%
\end{pgfscope}%
\begin{pgfscope}%
\pgfpathrectangle{\pgfqpoint{0.211875in}{0.211875in}}{\pgfqpoint{1.313625in}{1.279725in}}%
\pgfusepath{clip}%
\pgfsetbuttcap%
\pgfsetroundjoin%
\definecolor{currentfill}{rgb}{0.121569,0.466667,0.705882}%
\pgfsetfillcolor{currentfill}%
\pgfsetlinewidth{1.003750pt}%
\definecolor{currentstroke}{rgb}{0.121569,0.466667,0.705882}%
\pgfsetstrokecolor{currentstroke}%
\pgfsetdash{}{0pt}%
\pgfpathmoveto{\pgfqpoint{1.442288in}{1.133059in}}%
\pgfpathcurveto{\pgfqpoint{1.448112in}{1.133059in}}{\pgfqpoint{1.453698in}{1.135373in}}{\pgfqpoint{1.457817in}{1.139492in}}%
\pgfpathcurveto{\pgfqpoint{1.461935in}{1.143610in}}{\pgfqpoint{1.464249in}{1.149196in}}{\pgfqpoint{1.464249in}{1.155020in}}%
\pgfpathcurveto{\pgfqpoint{1.464249in}{1.160844in}}{\pgfqpoint{1.461935in}{1.166430in}}{\pgfqpoint{1.457817in}{1.170548in}}%
\pgfpathcurveto{\pgfqpoint{1.453698in}{1.174666in}}{\pgfqpoint{1.448112in}{1.176980in}}{\pgfqpoint{1.442288in}{1.176980in}}%
\pgfpathcurveto{\pgfqpoint{1.436464in}{1.176980in}}{\pgfqpoint{1.430878in}{1.174666in}}{\pgfqpoint{1.426760in}{1.170548in}}%
\pgfpathcurveto{\pgfqpoint{1.422642in}{1.166430in}}{\pgfqpoint{1.420328in}{1.160844in}}{\pgfqpoint{1.420328in}{1.155020in}}%
\pgfpathcurveto{\pgfqpoint{1.420328in}{1.149196in}}{\pgfqpoint{1.422642in}{1.143610in}}{\pgfqpoint{1.426760in}{1.139492in}}%
\pgfpathcurveto{\pgfqpoint{1.430878in}{1.135373in}}{\pgfqpoint{1.436464in}{1.133059in}}{\pgfqpoint{1.442288in}{1.133059in}}%
\pgfpathclose%
\pgfusepath{stroke,fill}%
\end{pgfscope}%
\begin{pgfscope}%
\pgfpathrectangle{\pgfqpoint{0.211875in}{0.211875in}}{\pgfqpoint{1.313625in}{1.279725in}}%
\pgfusepath{clip}%
\pgfsetbuttcap%
\pgfsetroundjoin%
\definecolor{currentfill}{rgb}{0.121569,0.466667,0.705882}%
\pgfsetfillcolor{currentfill}%
\pgfsetlinewidth{1.003750pt}%
\definecolor{currentstroke}{rgb}{0.121569,0.466667,0.705882}%
\pgfsetstrokecolor{currentstroke}%
\pgfsetdash{}{0pt}%
\pgfpathmoveto{\pgfqpoint{1.277912in}{1.320646in}}%
\pgfpathcurveto{\pgfqpoint{1.283736in}{1.320646in}}{\pgfqpoint{1.289322in}{1.322960in}}{\pgfqpoint{1.293440in}{1.327078in}}%
\pgfpathcurveto{\pgfqpoint{1.297559in}{1.331196in}}{\pgfqpoint{1.299872in}{1.336782in}}{\pgfqpoint{1.299872in}{1.342606in}}%
\pgfpathcurveto{\pgfqpoint{1.299872in}{1.348430in}}{\pgfqpoint{1.297559in}{1.354016in}}{\pgfqpoint{1.293440in}{1.358134in}}%
\pgfpathcurveto{\pgfqpoint{1.289322in}{1.362253in}}{\pgfqpoint{1.283736in}{1.364566in}}{\pgfqpoint{1.277912in}{1.364566in}}%
\pgfpathcurveto{\pgfqpoint{1.272088in}{1.364566in}}{\pgfqpoint{1.266502in}{1.362253in}}{\pgfqpoint{1.262384in}{1.358134in}}%
\pgfpathcurveto{\pgfqpoint{1.258266in}{1.354016in}}{\pgfqpoint{1.255952in}{1.348430in}}{\pgfqpoint{1.255952in}{1.342606in}}%
\pgfpathcurveto{\pgfqpoint{1.255952in}{1.336782in}}{\pgfqpoint{1.258266in}{1.331196in}}{\pgfqpoint{1.262384in}{1.327078in}}%
\pgfpathcurveto{\pgfqpoint{1.266502in}{1.322960in}}{\pgfqpoint{1.272088in}{1.320646in}}{\pgfqpoint{1.277912in}{1.320646in}}%
\pgfpathclose%
\pgfusepath{stroke,fill}%
\end{pgfscope}%
\begin{pgfscope}%
\pgfpathrectangle{\pgfqpoint{0.211875in}{0.211875in}}{\pgfqpoint{1.313625in}{1.279725in}}%
\pgfusepath{clip}%
\pgfsetbuttcap%
\pgfsetroundjoin%
\definecolor{currentfill}{rgb}{0.121569,0.466667,0.705882}%
\pgfsetfillcolor{currentfill}%
\pgfsetlinewidth{1.003750pt}%
\definecolor{currentstroke}{rgb}{0.121569,0.466667,0.705882}%
\pgfsetstrokecolor{currentstroke}%
\pgfsetdash{}{0pt}%
\pgfpathmoveto{\pgfqpoint{0.750933in}{0.615330in}}%
\pgfpathcurveto{\pgfqpoint{0.756756in}{0.615330in}}{\pgfqpoint{0.762343in}{0.617644in}}{\pgfqpoint{0.766461in}{0.621762in}}%
\pgfpathcurveto{\pgfqpoint{0.770579in}{0.625881in}}{\pgfqpoint{0.772893in}{0.631467in}}{\pgfqpoint{0.772893in}{0.637291in}}%
\pgfpathcurveto{\pgfqpoint{0.772893in}{0.643115in}}{\pgfqpoint{0.770579in}{0.648701in}}{\pgfqpoint{0.766461in}{0.652819in}}%
\pgfpathcurveto{\pgfqpoint{0.762343in}{0.656937in}}{\pgfqpoint{0.756756in}{0.659251in}}{\pgfqpoint{0.750933in}{0.659251in}}%
\pgfpathcurveto{\pgfqpoint{0.745109in}{0.659251in}}{\pgfqpoint{0.739522in}{0.656937in}}{\pgfqpoint{0.735404in}{0.652819in}}%
\pgfpathcurveto{\pgfqpoint{0.731286in}{0.648701in}}{\pgfqpoint{0.728972in}{0.643115in}}{\pgfqpoint{0.728972in}{0.637291in}}%
\pgfpathcurveto{\pgfqpoint{0.728972in}{0.631467in}}{\pgfqpoint{0.731286in}{0.625881in}}{\pgfqpoint{0.735404in}{0.621762in}}%
\pgfpathcurveto{\pgfqpoint{0.739522in}{0.617644in}}{\pgfqpoint{0.745109in}{0.615330in}}{\pgfqpoint{0.750933in}{0.615330in}}%
\pgfpathclose%
\pgfusepath{stroke,fill}%
\end{pgfscope}%
\begin{pgfscope}%
\pgfpathrectangle{\pgfqpoint{0.211875in}{0.211875in}}{\pgfqpoint{1.313625in}{1.279725in}}%
\pgfusepath{clip}%
\pgfsetbuttcap%
\pgfsetroundjoin%
\definecolor{currentfill}{rgb}{0.121569,0.466667,0.705882}%
\pgfsetfillcolor{currentfill}%
\pgfsetlinewidth{1.003750pt}%
\definecolor{currentstroke}{rgb}{0.121569,0.466667,0.705882}%
\pgfsetstrokecolor{currentstroke}%
\pgfsetdash{}{0pt}%
\pgfpathmoveto{\pgfqpoint{1.424457in}{0.670831in}}%
\pgfpathcurveto{\pgfqpoint{1.430281in}{0.670831in}}{\pgfqpoint{1.435867in}{0.673145in}}{\pgfqpoint{1.439985in}{0.677263in}}%
\pgfpathcurveto{\pgfqpoint{1.444103in}{0.681381in}}{\pgfqpoint{1.446417in}{0.686967in}}{\pgfqpoint{1.446417in}{0.692791in}}%
\pgfpathcurveto{\pgfqpoint{1.446417in}{0.698615in}}{\pgfqpoint{1.444103in}{0.704201in}}{\pgfqpoint{1.439985in}{0.708319in}}%
\pgfpathcurveto{\pgfqpoint{1.435867in}{0.712437in}}{\pgfqpoint{1.430281in}{0.714751in}}{\pgfqpoint{1.424457in}{0.714751in}}%
\pgfpathcurveto{\pgfqpoint{1.418633in}{0.714751in}}{\pgfqpoint{1.413047in}{0.712437in}}{\pgfqpoint{1.408928in}{0.708319in}}%
\pgfpathcurveto{\pgfqpoint{1.404810in}{0.704201in}}{\pgfqpoint{1.402496in}{0.698615in}}{\pgfqpoint{1.402496in}{0.692791in}}%
\pgfpathcurveto{\pgfqpoint{1.402496in}{0.686967in}}{\pgfqpoint{1.404810in}{0.681381in}}{\pgfqpoint{1.408928in}{0.677263in}}%
\pgfpathcurveto{\pgfqpoint{1.413047in}{0.673145in}}{\pgfqpoint{1.418633in}{0.670831in}}{\pgfqpoint{1.424457in}{0.670831in}}%
\pgfpathclose%
\pgfusepath{stroke,fill}%
\end{pgfscope}%
\begin{pgfscope}%
\pgfpathrectangle{\pgfqpoint{0.211875in}{0.211875in}}{\pgfqpoint{1.313625in}{1.279725in}}%
\pgfusepath{clip}%
\pgfsetbuttcap%
\pgfsetroundjoin%
\definecolor{currentfill}{rgb}{0.121569,0.466667,0.705882}%
\pgfsetfillcolor{currentfill}%
\pgfsetlinewidth{1.003750pt}%
\definecolor{currentstroke}{rgb}{0.121569,0.466667,0.705882}%
\pgfsetstrokecolor{currentstroke}%
\pgfsetdash{}{0pt}%
\pgfpathmoveto{\pgfqpoint{0.898508in}{0.464012in}}%
\pgfpathcurveto{\pgfqpoint{0.904332in}{0.464012in}}{\pgfqpoint{0.909918in}{0.466326in}}{\pgfqpoint{0.914036in}{0.470444in}}%
\pgfpathcurveto{\pgfqpoint{0.918154in}{0.474562in}}{\pgfqpoint{0.920468in}{0.480148in}}{\pgfqpoint{0.920468in}{0.485972in}}%
\pgfpathcurveto{\pgfqpoint{0.920468in}{0.491796in}}{\pgfqpoint{0.918154in}{0.497382in}}{\pgfqpoint{0.914036in}{0.501500in}}%
\pgfpathcurveto{\pgfqpoint{0.909918in}{0.505618in}}{\pgfqpoint{0.904332in}{0.507932in}}{\pgfqpoint{0.898508in}{0.507932in}}%
\pgfpathcurveto{\pgfqpoint{0.892684in}{0.507932in}}{\pgfqpoint{0.887098in}{0.505618in}}{\pgfqpoint{0.882979in}{0.501500in}}%
\pgfpathcurveto{\pgfqpoint{0.878861in}{0.497382in}}{\pgfqpoint{0.876547in}{0.491796in}}{\pgfqpoint{0.876547in}{0.485972in}}%
\pgfpathcurveto{\pgfqpoint{0.876547in}{0.480148in}}{\pgfqpoint{0.878861in}{0.474562in}}{\pgfqpoint{0.882979in}{0.470444in}}%
\pgfpathcurveto{\pgfqpoint{0.887098in}{0.466326in}}{\pgfqpoint{0.892684in}{0.464012in}}{\pgfqpoint{0.898508in}{0.464012in}}%
\pgfpathclose%
\pgfusepath{stroke,fill}%
\end{pgfscope}%
\begin{pgfscope}%
\pgfpathrectangle{\pgfqpoint{0.211875in}{0.211875in}}{\pgfqpoint{1.313625in}{1.279725in}}%
\pgfusepath{clip}%
\pgfsetbuttcap%
\pgfsetroundjoin%
\definecolor{currentfill}{rgb}{0.121569,0.466667,0.705882}%
\pgfsetfillcolor{currentfill}%
\pgfsetlinewidth{1.003750pt}%
\definecolor{currentstroke}{rgb}{0.121569,0.466667,0.705882}%
\pgfsetstrokecolor{currentstroke}%
\pgfsetdash{}{0pt}%
\pgfpathmoveto{\pgfqpoint{1.440057in}{1.207367in}}%
\pgfpathcurveto{\pgfqpoint{1.445881in}{1.207367in}}{\pgfqpoint{1.451467in}{1.209681in}}{\pgfqpoint{1.455586in}{1.213799in}}%
\pgfpathcurveto{\pgfqpoint{1.459704in}{1.217917in}}{\pgfqpoint{1.462018in}{1.223503in}}{\pgfqpoint{1.462018in}{1.229327in}}%
\pgfpathcurveto{\pgfqpoint{1.462018in}{1.235151in}}{\pgfqpoint{1.459704in}{1.240737in}}{\pgfqpoint{1.455586in}{1.244855in}}%
\pgfpathcurveto{\pgfqpoint{1.451467in}{1.248974in}}{\pgfqpoint{1.445881in}{1.251287in}}{\pgfqpoint{1.440057in}{1.251287in}}%
\pgfpathcurveto{\pgfqpoint{1.434233in}{1.251287in}}{\pgfqpoint{1.428647in}{1.248974in}}{\pgfqpoint{1.424529in}{1.244855in}}%
\pgfpathcurveto{\pgfqpoint{1.420411in}{1.240737in}}{\pgfqpoint{1.418097in}{1.235151in}}{\pgfqpoint{1.418097in}{1.229327in}}%
\pgfpathcurveto{\pgfqpoint{1.418097in}{1.223503in}}{\pgfqpoint{1.420411in}{1.217917in}}{\pgfqpoint{1.424529in}{1.213799in}}%
\pgfpathcurveto{\pgfqpoint{1.428647in}{1.209681in}}{\pgfqpoint{1.434233in}{1.207367in}}{\pgfqpoint{1.440057in}{1.207367in}}%
\pgfpathclose%
\pgfusepath{stroke,fill}%
\end{pgfscope}%
\begin{pgfscope}%
\pgfpathrectangle{\pgfqpoint{0.211875in}{0.211875in}}{\pgfqpoint{1.313625in}{1.279725in}}%
\pgfusepath{clip}%
\pgfsetbuttcap%
\pgfsetroundjoin%
\definecolor{currentfill}{rgb}{0.121569,0.466667,0.705882}%
\pgfsetfillcolor{currentfill}%
\pgfsetlinewidth{1.003750pt}%
\definecolor{currentstroke}{rgb}{0.121569,0.466667,0.705882}%
\pgfsetstrokecolor{currentstroke}%
\pgfsetdash{}{0pt}%
\pgfpathmoveto{\pgfqpoint{1.441397in}{0.671469in}}%
\pgfpathcurveto{\pgfqpoint{1.447221in}{0.671469in}}{\pgfqpoint{1.452807in}{0.673783in}}{\pgfqpoint{1.456925in}{0.677901in}}%
\pgfpathcurveto{\pgfqpoint{1.461043in}{0.682019in}}{\pgfqpoint{1.463357in}{0.687605in}}{\pgfqpoint{1.463357in}{0.693429in}}%
\pgfpathcurveto{\pgfqpoint{1.463357in}{0.699253in}}{\pgfqpoint{1.461043in}{0.704839in}}{\pgfqpoint{1.456925in}{0.708957in}}%
\pgfpathcurveto{\pgfqpoint{1.452807in}{0.713075in}}{\pgfqpoint{1.447221in}{0.715389in}}{\pgfqpoint{1.441397in}{0.715389in}}%
\pgfpathcurveto{\pgfqpoint{1.435573in}{0.715389in}}{\pgfqpoint{1.429986in}{0.713075in}}{\pgfqpoint{1.425868in}{0.708957in}}%
\pgfpathcurveto{\pgfqpoint{1.421750in}{0.704839in}}{\pgfqpoint{1.419436in}{0.699253in}}{\pgfqpoint{1.419436in}{0.693429in}}%
\pgfpathcurveto{\pgfqpoint{1.419436in}{0.687605in}}{\pgfqpoint{1.421750in}{0.682019in}}{\pgfqpoint{1.425868in}{0.677901in}}%
\pgfpathcurveto{\pgfqpoint{1.429986in}{0.673783in}}{\pgfqpoint{1.435573in}{0.671469in}}{\pgfqpoint{1.441397in}{0.671469in}}%
\pgfpathclose%
\pgfusepath{stroke,fill}%
\end{pgfscope}%
\begin{pgfscope}%
\pgfpathrectangle{\pgfqpoint{0.211875in}{0.211875in}}{\pgfqpoint{1.313625in}{1.279725in}}%
\pgfusepath{clip}%
\pgfsetbuttcap%
\pgfsetroundjoin%
\definecolor{currentfill}{rgb}{0.121569,0.466667,0.705882}%
\pgfsetfillcolor{currentfill}%
\pgfsetlinewidth{1.003750pt}%
\definecolor{currentstroke}{rgb}{0.121569,0.466667,0.705882}%
\pgfsetstrokecolor{currentstroke}%
\pgfsetdash{}{0pt}%
\pgfpathmoveto{\pgfqpoint{1.437289in}{1.116444in}}%
\pgfpathcurveto{\pgfqpoint{1.443113in}{1.116444in}}{\pgfqpoint{1.448699in}{1.118758in}}{\pgfqpoint{1.452817in}{1.122876in}}%
\pgfpathcurveto{\pgfqpoint{1.456935in}{1.126994in}}{\pgfqpoint{1.459249in}{1.132580in}}{\pgfqpoint{1.459249in}{1.138404in}}%
\pgfpathcurveto{\pgfqpoint{1.459249in}{1.144228in}}{\pgfqpoint{1.456935in}{1.149814in}}{\pgfqpoint{1.452817in}{1.153932in}}%
\pgfpathcurveto{\pgfqpoint{1.448699in}{1.158051in}}{\pgfqpoint{1.443113in}{1.160364in}}{\pgfqpoint{1.437289in}{1.160364in}}%
\pgfpathcurveto{\pgfqpoint{1.431465in}{1.160364in}}{\pgfqpoint{1.425879in}{1.158051in}}{\pgfqpoint{1.421761in}{1.153932in}}%
\pgfpathcurveto{\pgfqpoint{1.417643in}{1.149814in}}{\pgfqpoint{1.415329in}{1.144228in}}{\pgfqpoint{1.415329in}{1.138404in}}%
\pgfpathcurveto{\pgfqpoint{1.415329in}{1.132580in}}{\pgfqpoint{1.417643in}{1.126994in}}{\pgfqpoint{1.421761in}{1.122876in}}%
\pgfpathcurveto{\pgfqpoint{1.425879in}{1.118758in}}{\pgfqpoint{1.431465in}{1.116444in}}{\pgfqpoint{1.437289in}{1.116444in}}%
\pgfpathclose%
\pgfusepath{stroke,fill}%
\end{pgfscope}%
\begin{pgfscope}%
\pgfpathrectangle{\pgfqpoint{0.211875in}{0.211875in}}{\pgfqpoint{1.313625in}{1.279725in}}%
\pgfusepath{clip}%
\pgfsetbuttcap%
\pgfsetroundjoin%
\definecolor{currentfill}{rgb}{0.121569,0.466667,0.705882}%
\pgfsetfillcolor{currentfill}%
\pgfsetlinewidth{1.003750pt}%
\definecolor{currentstroke}{rgb}{0.121569,0.466667,0.705882}%
\pgfsetstrokecolor{currentstroke}%
\pgfsetdash{}{0pt}%
\pgfpathmoveto{\pgfqpoint{1.411785in}{1.102232in}}%
\pgfpathcurveto{\pgfqpoint{1.417609in}{1.102232in}}{\pgfqpoint{1.423195in}{1.104546in}}{\pgfqpoint{1.427313in}{1.108664in}}%
\pgfpathcurveto{\pgfqpoint{1.431431in}{1.112782in}}{\pgfqpoint{1.433745in}{1.118369in}}{\pgfqpoint{1.433745in}{1.124193in}}%
\pgfpathcurveto{\pgfqpoint{1.433745in}{1.130017in}}{\pgfqpoint{1.431431in}{1.135603in}}{\pgfqpoint{1.427313in}{1.139721in}}%
\pgfpathcurveto{\pgfqpoint{1.423195in}{1.143839in}}{\pgfqpoint{1.417609in}{1.146153in}}{\pgfqpoint{1.411785in}{1.146153in}}%
\pgfpathcurveto{\pgfqpoint{1.405961in}{1.146153in}}{\pgfqpoint{1.400375in}{1.143839in}}{\pgfqpoint{1.396257in}{1.139721in}}%
\pgfpathcurveto{\pgfqpoint{1.392138in}{1.135603in}}{\pgfqpoint{1.389825in}{1.130017in}}{\pgfqpoint{1.389825in}{1.124193in}}%
\pgfpathcurveto{\pgfqpoint{1.389825in}{1.118369in}}{\pgfqpoint{1.392138in}{1.112782in}}{\pgfqpoint{1.396257in}{1.108664in}}%
\pgfpathcurveto{\pgfqpoint{1.400375in}{1.104546in}}{\pgfqpoint{1.405961in}{1.102232in}}{\pgfqpoint{1.411785in}{1.102232in}}%
\pgfpathclose%
\pgfusepath{stroke,fill}%
\end{pgfscope}%
\begin{pgfscope}%
\pgfpathrectangle{\pgfqpoint{0.211875in}{0.211875in}}{\pgfqpoint{1.313625in}{1.279725in}}%
\pgfusepath{clip}%
\pgfsetbuttcap%
\pgfsetroundjoin%
\definecolor{currentfill}{rgb}{0.121569,0.466667,0.705882}%
\pgfsetfillcolor{currentfill}%
\pgfsetlinewidth{1.003750pt}%
\definecolor{currentstroke}{rgb}{0.121569,0.466667,0.705882}%
\pgfsetstrokecolor{currentstroke}%
\pgfsetdash{}{0pt}%
\pgfpathmoveto{\pgfqpoint{1.216447in}{0.661717in}}%
\pgfpathcurveto{\pgfqpoint{1.222271in}{0.661717in}}{\pgfqpoint{1.227858in}{0.664030in}}{\pgfqpoint{1.231976in}{0.668149in}}%
\pgfpathcurveto{\pgfqpoint{1.236094in}{0.672267in}}{\pgfqpoint{1.238408in}{0.677853in}}{\pgfqpoint{1.238408in}{0.683677in}}%
\pgfpathcurveto{\pgfqpoint{1.238408in}{0.689501in}}{\pgfqpoint{1.236094in}{0.695087in}}{\pgfqpoint{1.231976in}{0.699205in}}%
\pgfpathcurveto{\pgfqpoint{1.227858in}{0.703323in}}{\pgfqpoint{1.222271in}{0.705637in}}{\pgfqpoint{1.216447in}{0.705637in}}%
\pgfpathcurveto{\pgfqpoint{1.210624in}{0.705637in}}{\pgfqpoint{1.205037in}{0.703323in}}{\pgfqpoint{1.200919in}{0.699205in}}%
\pgfpathcurveto{\pgfqpoint{1.196801in}{0.695087in}}{\pgfqpoint{1.194487in}{0.689501in}}{\pgfqpoint{1.194487in}{0.683677in}}%
\pgfpathcurveto{\pgfqpoint{1.194487in}{0.677853in}}{\pgfqpoint{1.196801in}{0.672267in}}{\pgfqpoint{1.200919in}{0.668149in}}%
\pgfpathcurveto{\pgfqpoint{1.205037in}{0.664030in}}{\pgfqpoint{1.210624in}{0.661717in}}{\pgfqpoint{1.216447in}{0.661717in}}%
\pgfpathclose%
\pgfusepath{stroke,fill}%
\end{pgfscope}%
\begin{pgfscope}%
\pgfpathrectangle{\pgfqpoint{0.211875in}{0.211875in}}{\pgfqpoint{1.313625in}{1.279725in}}%
\pgfusepath{clip}%
\pgfsetbuttcap%
\pgfsetroundjoin%
\definecolor{currentfill}{rgb}{0.121569,0.466667,0.705882}%
\pgfsetfillcolor{currentfill}%
\pgfsetlinewidth{1.003750pt}%
\definecolor{currentstroke}{rgb}{0.121569,0.466667,0.705882}%
\pgfsetstrokecolor{currentstroke}%
\pgfsetdash{}{0pt}%
\pgfpathmoveto{\pgfqpoint{1.439973in}{0.827710in}}%
\pgfpathcurveto{\pgfqpoint{1.445797in}{0.827710in}}{\pgfqpoint{1.451383in}{0.830024in}}{\pgfqpoint{1.455501in}{0.834142in}}%
\pgfpathcurveto{\pgfqpoint{1.459620in}{0.838260in}}{\pgfqpoint{1.461933in}{0.843846in}}{\pgfqpoint{1.461933in}{0.849670in}}%
\pgfpathcurveto{\pgfqpoint{1.461933in}{0.855494in}}{\pgfqpoint{1.459620in}{0.861080in}}{\pgfqpoint{1.455501in}{0.865199in}}%
\pgfpathcurveto{\pgfqpoint{1.451383in}{0.869317in}}{\pgfqpoint{1.445797in}{0.871631in}}{\pgfqpoint{1.439973in}{0.871631in}}%
\pgfpathcurveto{\pgfqpoint{1.434149in}{0.871631in}}{\pgfqpoint{1.428563in}{0.869317in}}{\pgfqpoint{1.424445in}{0.865199in}}%
\pgfpathcurveto{\pgfqpoint{1.420327in}{0.861080in}}{\pgfqpoint{1.418013in}{0.855494in}}{\pgfqpoint{1.418013in}{0.849670in}}%
\pgfpathcurveto{\pgfqpoint{1.418013in}{0.843846in}}{\pgfqpoint{1.420327in}{0.838260in}}{\pgfqpoint{1.424445in}{0.834142in}}%
\pgfpathcurveto{\pgfqpoint{1.428563in}{0.830024in}}{\pgfqpoint{1.434149in}{0.827710in}}{\pgfqpoint{1.439973in}{0.827710in}}%
\pgfpathclose%
\pgfusepath{stroke,fill}%
\end{pgfscope}%
\begin{pgfscope}%
\pgfpathrectangle{\pgfqpoint{0.211875in}{0.211875in}}{\pgfqpoint{1.313625in}{1.279725in}}%
\pgfusepath{clip}%
\pgfsetbuttcap%
\pgfsetroundjoin%
\definecolor{currentfill}{rgb}{0.121569,0.466667,0.705882}%
\pgfsetfillcolor{currentfill}%
\pgfsetlinewidth{1.003750pt}%
\definecolor{currentstroke}{rgb}{0.121569,0.466667,0.705882}%
\pgfsetstrokecolor{currentstroke}%
\pgfsetdash{}{0pt}%
\pgfpathmoveto{\pgfqpoint{1.321279in}{0.680651in}}%
\pgfpathcurveto{\pgfqpoint{1.327103in}{0.680651in}}{\pgfqpoint{1.332689in}{0.682965in}}{\pgfqpoint{1.336807in}{0.687083in}}%
\pgfpathcurveto{\pgfqpoint{1.340925in}{0.691201in}}{\pgfqpoint{1.343239in}{0.696787in}}{\pgfqpoint{1.343239in}{0.702611in}}%
\pgfpathcurveto{\pgfqpoint{1.343239in}{0.708435in}}{\pgfqpoint{1.340925in}{0.714021in}}{\pgfqpoint{1.336807in}{0.718140in}}%
\pgfpathcurveto{\pgfqpoint{1.332689in}{0.722258in}}{\pgfqpoint{1.327103in}{0.724572in}}{\pgfqpoint{1.321279in}{0.724572in}}%
\pgfpathcurveto{\pgfqpoint{1.315455in}{0.724572in}}{\pgfqpoint{1.309869in}{0.722258in}}{\pgfqpoint{1.305751in}{0.718140in}}%
\pgfpathcurveto{\pgfqpoint{1.301633in}{0.714021in}}{\pgfqpoint{1.299319in}{0.708435in}}{\pgfqpoint{1.299319in}{0.702611in}}%
\pgfpathcurveto{\pgfqpoint{1.299319in}{0.696787in}}{\pgfqpoint{1.301633in}{0.691201in}}{\pgfqpoint{1.305751in}{0.687083in}}%
\pgfpathcurveto{\pgfqpoint{1.309869in}{0.682965in}}{\pgfqpoint{1.315455in}{0.680651in}}{\pgfqpoint{1.321279in}{0.680651in}}%
\pgfpathclose%
\pgfusepath{stroke,fill}%
\end{pgfscope}%
\begin{pgfscope}%
\pgfpathrectangle{\pgfqpoint{0.211875in}{0.211875in}}{\pgfqpoint{1.313625in}{1.279725in}}%
\pgfusepath{clip}%
\pgfsetbuttcap%
\pgfsetroundjoin%
\definecolor{currentfill}{rgb}{0.121569,0.466667,0.705882}%
\pgfsetfillcolor{currentfill}%
\pgfsetlinewidth{1.003750pt}%
\definecolor{currentstroke}{rgb}{0.121569,0.466667,0.705882}%
\pgfsetstrokecolor{currentstroke}%
\pgfsetdash{}{0pt}%
\pgfpathmoveto{\pgfqpoint{1.212143in}{0.667672in}}%
\pgfpathcurveto{\pgfqpoint{1.217967in}{0.667672in}}{\pgfqpoint{1.223553in}{0.669986in}}{\pgfqpoint{1.227671in}{0.674104in}}%
\pgfpathcurveto{\pgfqpoint{1.231789in}{0.678223in}}{\pgfqpoint{1.234103in}{0.683809in}}{\pgfqpoint{1.234103in}{0.689633in}}%
\pgfpathcurveto{\pgfqpoint{1.234103in}{0.695457in}}{\pgfqpoint{1.231789in}{0.701043in}}{\pgfqpoint{1.227671in}{0.705161in}}%
\pgfpathcurveto{\pgfqpoint{1.223553in}{0.709279in}}{\pgfqpoint{1.217967in}{0.711593in}}{\pgfqpoint{1.212143in}{0.711593in}}%
\pgfpathcurveto{\pgfqpoint{1.206319in}{0.711593in}}{\pgfqpoint{1.200733in}{0.709279in}}{\pgfqpoint{1.196615in}{0.705161in}}%
\pgfpathcurveto{\pgfqpoint{1.192496in}{0.701043in}}{\pgfqpoint{1.190183in}{0.695457in}}{\pgfqpoint{1.190183in}{0.689633in}}%
\pgfpathcurveto{\pgfqpoint{1.190183in}{0.683809in}}{\pgfqpoint{1.192496in}{0.678223in}}{\pgfqpoint{1.196615in}{0.674104in}}%
\pgfpathcurveto{\pgfqpoint{1.200733in}{0.669986in}}{\pgfqpoint{1.206319in}{0.667672in}}{\pgfqpoint{1.212143in}{0.667672in}}%
\pgfpathclose%
\pgfusepath{stroke,fill}%
\end{pgfscope}%
\begin{pgfscope}%
\pgfpathrectangle{\pgfqpoint{0.211875in}{0.211875in}}{\pgfqpoint{1.313625in}{1.279725in}}%
\pgfusepath{clip}%
\pgfsetbuttcap%
\pgfsetroundjoin%
\definecolor{currentfill}{rgb}{0.121569,0.466667,0.705882}%
\pgfsetfillcolor{currentfill}%
\pgfsetlinewidth{1.003750pt}%
\definecolor{currentstroke}{rgb}{0.121569,0.466667,0.705882}%
\pgfsetstrokecolor{currentstroke}%
\pgfsetdash{}{0pt}%
\pgfpathmoveto{\pgfqpoint{1.429730in}{1.144329in}}%
\pgfpathcurveto{\pgfqpoint{1.435554in}{1.144329in}}{\pgfqpoint{1.441140in}{1.146643in}}{\pgfqpoint{1.445258in}{1.150761in}}%
\pgfpathcurveto{\pgfqpoint{1.449376in}{1.154879in}}{\pgfqpoint{1.451690in}{1.160465in}}{\pgfqpoint{1.451690in}{1.166289in}}%
\pgfpathcurveto{\pgfqpoint{1.451690in}{1.172113in}}{\pgfqpoint{1.449376in}{1.177699in}}{\pgfqpoint{1.445258in}{1.181817in}}%
\pgfpathcurveto{\pgfqpoint{1.441140in}{1.185935in}}{\pgfqpoint{1.435554in}{1.188249in}}{\pgfqpoint{1.429730in}{1.188249in}}%
\pgfpathcurveto{\pgfqpoint{1.423906in}{1.188249in}}{\pgfqpoint{1.418320in}{1.185935in}}{\pgfqpoint{1.414201in}{1.181817in}}%
\pgfpathcurveto{\pgfqpoint{1.410083in}{1.177699in}}{\pgfqpoint{1.407769in}{1.172113in}}{\pgfqpoint{1.407769in}{1.166289in}}%
\pgfpathcurveto{\pgfqpoint{1.407769in}{1.160465in}}{\pgfqpoint{1.410083in}{1.154879in}}{\pgfqpoint{1.414201in}{1.150761in}}%
\pgfpathcurveto{\pgfqpoint{1.418320in}{1.146643in}}{\pgfqpoint{1.423906in}{1.144329in}}{\pgfqpoint{1.429730in}{1.144329in}}%
\pgfpathclose%
\pgfusepath{stroke,fill}%
\end{pgfscope}%
\begin{pgfscope}%
\pgfpathrectangle{\pgfqpoint{0.211875in}{0.211875in}}{\pgfqpoint{1.313625in}{1.279725in}}%
\pgfusepath{clip}%
\pgfsetbuttcap%
\pgfsetroundjoin%
\definecolor{currentfill}{rgb}{0.121569,0.466667,0.705882}%
\pgfsetfillcolor{currentfill}%
\pgfsetlinewidth{1.003750pt}%
\definecolor{currentstroke}{rgb}{0.121569,0.466667,0.705882}%
\pgfsetstrokecolor{currentstroke}%
\pgfsetdash{}{0pt}%
\pgfpathmoveto{\pgfqpoint{1.254981in}{1.330197in}}%
\pgfpathcurveto{\pgfqpoint{1.260805in}{1.330197in}}{\pgfqpoint{1.266391in}{1.332511in}}{\pgfqpoint{1.270509in}{1.336629in}}%
\pgfpathcurveto{\pgfqpoint{1.274627in}{1.340747in}}{\pgfqpoint{1.276941in}{1.346333in}}{\pgfqpoint{1.276941in}{1.352157in}}%
\pgfpathcurveto{\pgfqpoint{1.276941in}{1.357981in}}{\pgfqpoint{1.274627in}{1.363567in}}{\pgfqpoint{1.270509in}{1.367686in}}%
\pgfpathcurveto{\pgfqpoint{1.266391in}{1.371804in}}{\pgfqpoint{1.260805in}{1.374118in}}{\pgfqpoint{1.254981in}{1.374118in}}%
\pgfpathcurveto{\pgfqpoint{1.249157in}{1.374118in}}{\pgfqpoint{1.243571in}{1.371804in}}{\pgfqpoint{1.239453in}{1.367686in}}%
\pgfpathcurveto{\pgfqpoint{1.235334in}{1.363567in}}{\pgfqpoint{1.233021in}{1.357981in}}{\pgfqpoint{1.233021in}{1.352157in}}%
\pgfpathcurveto{\pgfqpoint{1.233021in}{1.346333in}}{\pgfqpoint{1.235334in}{1.340747in}}{\pgfqpoint{1.239453in}{1.336629in}}%
\pgfpathcurveto{\pgfqpoint{1.243571in}{1.332511in}}{\pgfqpoint{1.249157in}{1.330197in}}{\pgfqpoint{1.254981in}{1.330197in}}%
\pgfpathclose%
\pgfusepath{stroke,fill}%
\end{pgfscope}%
\begin{pgfscope}%
\pgfpathrectangle{\pgfqpoint{0.211875in}{0.211875in}}{\pgfqpoint{1.313625in}{1.279725in}}%
\pgfusepath{clip}%
\pgfsetbuttcap%
\pgfsetroundjoin%
\definecolor{currentfill}{rgb}{0.121569,0.466667,0.705882}%
\pgfsetfillcolor{currentfill}%
\pgfsetlinewidth{1.003750pt}%
\definecolor{currentstroke}{rgb}{0.121569,0.466667,0.705882}%
\pgfsetstrokecolor{currentstroke}%
\pgfsetdash{}{0pt}%
\pgfpathmoveto{\pgfqpoint{1.416404in}{0.670739in}}%
\pgfpathcurveto{\pgfqpoint{1.422228in}{0.670739in}}{\pgfqpoint{1.427814in}{0.673053in}}{\pgfqpoint{1.431933in}{0.677171in}}%
\pgfpathcurveto{\pgfqpoint{1.436051in}{0.681289in}}{\pgfqpoint{1.438365in}{0.686876in}}{\pgfqpoint{1.438365in}{0.692699in}}%
\pgfpathcurveto{\pgfqpoint{1.438365in}{0.698523in}}{\pgfqpoint{1.436051in}{0.704110in}}{\pgfqpoint{1.431933in}{0.708228in}}%
\pgfpathcurveto{\pgfqpoint{1.427814in}{0.712346in}}{\pgfqpoint{1.422228in}{0.714660in}}{\pgfqpoint{1.416404in}{0.714660in}}%
\pgfpathcurveto{\pgfqpoint{1.410580in}{0.714660in}}{\pgfqpoint{1.404994in}{0.712346in}}{\pgfqpoint{1.400876in}{0.708228in}}%
\pgfpathcurveto{\pgfqpoint{1.396758in}{0.704110in}}{\pgfqpoint{1.394444in}{0.698523in}}{\pgfqpoint{1.394444in}{0.692699in}}%
\pgfpathcurveto{\pgfqpoint{1.394444in}{0.686876in}}{\pgfqpoint{1.396758in}{0.681289in}}{\pgfqpoint{1.400876in}{0.677171in}}%
\pgfpathcurveto{\pgfqpoint{1.404994in}{0.673053in}}{\pgfqpoint{1.410580in}{0.670739in}}{\pgfqpoint{1.416404in}{0.670739in}}%
\pgfpathclose%
\pgfusepath{stroke,fill}%
\end{pgfscope}%
\begin{pgfscope}%
\pgfpathrectangle{\pgfqpoint{0.211875in}{0.211875in}}{\pgfqpoint{1.313625in}{1.279725in}}%
\pgfusepath{clip}%
\pgfsetbuttcap%
\pgfsetroundjoin%
\definecolor{currentfill}{rgb}{0.121569,0.466667,0.705882}%
\pgfsetfillcolor{currentfill}%
\pgfsetlinewidth{1.003750pt}%
\definecolor{currentstroke}{rgb}{0.121569,0.466667,0.705882}%
\pgfsetstrokecolor{currentstroke}%
\pgfsetdash{}{0pt}%
\pgfpathmoveto{\pgfqpoint{1.444227in}{1.104473in}}%
\pgfpathcurveto{\pgfqpoint{1.450051in}{1.104473in}}{\pgfqpoint{1.455637in}{1.106786in}}{\pgfqpoint{1.459755in}{1.110905in}}%
\pgfpathcurveto{\pgfqpoint{1.463873in}{1.115023in}}{\pgfqpoint{1.466187in}{1.120609in}}{\pgfqpoint{1.466187in}{1.126433in}}%
\pgfpathcurveto{\pgfqpoint{1.466187in}{1.132257in}}{\pgfqpoint{1.463873in}{1.137843in}}{\pgfqpoint{1.459755in}{1.141961in}}%
\pgfpathcurveto{\pgfqpoint{1.455637in}{1.146079in}}{\pgfqpoint{1.450051in}{1.148393in}}{\pgfqpoint{1.444227in}{1.148393in}}%
\pgfpathcurveto{\pgfqpoint{1.438403in}{1.148393in}}{\pgfqpoint{1.432817in}{1.146079in}}{\pgfqpoint{1.428699in}{1.141961in}}%
\pgfpathcurveto{\pgfqpoint{1.424581in}{1.137843in}}{\pgfqpoint{1.422267in}{1.132257in}}{\pgfqpoint{1.422267in}{1.126433in}}%
\pgfpathcurveto{\pgfqpoint{1.422267in}{1.120609in}}{\pgfqpoint{1.424581in}{1.115023in}}{\pgfqpoint{1.428699in}{1.110905in}}%
\pgfpathcurveto{\pgfqpoint{1.432817in}{1.106786in}}{\pgfqpoint{1.438403in}{1.104473in}}{\pgfqpoint{1.444227in}{1.104473in}}%
\pgfpathclose%
\pgfusepath{stroke,fill}%
\end{pgfscope}%
\begin{pgfscope}%
\pgfpathrectangle{\pgfqpoint{0.211875in}{0.211875in}}{\pgfqpoint{1.313625in}{1.279725in}}%
\pgfusepath{clip}%
\pgfsetbuttcap%
\pgfsetroundjoin%
\definecolor{currentfill}{rgb}{0.121569,0.466667,0.705882}%
\pgfsetfillcolor{currentfill}%
\pgfsetlinewidth{1.003750pt}%
\definecolor{currentstroke}{rgb}{0.121569,0.466667,0.705882}%
\pgfsetstrokecolor{currentstroke}%
\pgfsetdash{}{0pt}%
\pgfpathmoveto{\pgfqpoint{1.203949in}{0.658338in}}%
\pgfpathcurveto{\pgfqpoint{1.209773in}{0.658338in}}{\pgfqpoint{1.215359in}{0.660651in}}{\pgfqpoint{1.219477in}{0.664770in}}%
\pgfpathcurveto{\pgfqpoint{1.223596in}{0.668888in}}{\pgfqpoint{1.225909in}{0.674474in}}{\pgfqpoint{1.225909in}{0.680298in}}%
\pgfpathcurveto{\pgfqpoint{1.225909in}{0.686122in}}{\pgfqpoint{1.223596in}{0.691708in}}{\pgfqpoint{1.219477in}{0.695826in}}%
\pgfpathcurveto{\pgfqpoint{1.215359in}{0.699944in}}{\pgfqpoint{1.209773in}{0.702258in}}{\pgfqpoint{1.203949in}{0.702258in}}%
\pgfpathcurveto{\pgfqpoint{1.198125in}{0.702258in}}{\pgfqpoint{1.192539in}{0.699944in}}{\pgfqpoint{1.188421in}{0.695826in}}%
\pgfpathcurveto{\pgfqpoint{1.184303in}{0.691708in}}{\pgfqpoint{1.181989in}{0.686122in}}{\pgfqpoint{1.181989in}{0.680298in}}%
\pgfpathcurveto{\pgfqpoint{1.181989in}{0.674474in}}{\pgfqpoint{1.184303in}{0.668888in}}{\pgfqpoint{1.188421in}{0.664770in}}%
\pgfpathcurveto{\pgfqpoint{1.192539in}{0.660651in}}{\pgfqpoint{1.198125in}{0.658338in}}{\pgfqpoint{1.203949in}{0.658338in}}%
\pgfpathclose%
\pgfusepath{stroke,fill}%
\end{pgfscope}%
\begin{pgfscope}%
\pgfpathrectangle{\pgfqpoint{0.211875in}{0.211875in}}{\pgfqpoint{1.313625in}{1.279725in}}%
\pgfusepath{clip}%
\pgfsetbuttcap%
\pgfsetroundjoin%
\definecolor{currentfill}{rgb}{0.121569,0.466667,0.705882}%
\pgfsetfillcolor{currentfill}%
\pgfsetlinewidth{1.003750pt}%
\definecolor{currentstroke}{rgb}{0.121569,0.466667,0.705882}%
\pgfsetstrokecolor{currentstroke}%
\pgfsetdash{}{0pt}%
\pgfpathmoveto{\pgfqpoint{1.414461in}{0.814983in}}%
\pgfpathcurveto{\pgfqpoint{1.420285in}{0.814983in}}{\pgfqpoint{1.425871in}{0.817297in}}{\pgfqpoint{1.429989in}{0.821415in}}%
\pgfpathcurveto{\pgfqpoint{1.434107in}{0.825533in}}{\pgfqpoint{1.436421in}{0.831119in}}{\pgfqpoint{1.436421in}{0.836943in}}%
\pgfpathcurveto{\pgfqpoint{1.436421in}{0.842767in}}{\pgfqpoint{1.434107in}{0.848353in}}{\pgfqpoint{1.429989in}{0.852472in}}%
\pgfpathcurveto{\pgfqpoint{1.425871in}{0.856590in}}{\pgfqpoint{1.420285in}{0.858904in}}{\pgfqpoint{1.414461in}{0.858904in}}%
\pgfpathcurveto{\pgfqpoint{1.408637in}{0.858904in}}{\pgfqpoint{1.403051in}{0.856590in}}{\pgfqpoint{1.398932in}{0.852472in}}%
\pgfpathcurveto{\pgfqpoint{1.394814in}{0.848353in}}{\pgfqpoint{1.392500in}{0.842767in}}{\pgfqpoint{1.392500in}{0.836943in}}%
\pgfpathcurveto{\pgfqpoint{1.392500in}{0.831119in}}{\pgfqpoint{1.394814in}{0.825533in}}{\pgfqpoint{1.398932in}{0.821415in}}%
\pgfpathcurveto{\pgfqpoint{1.403051in}{0.817297in}}{\pgfqpoint{1.408637in}{0.814983in}}{\pgfqpoint{1.414461in}{0.814983in}}%
\pgfpathclose%
\pgfusepath{stroke,fill}%
\end{pgfscope}%
\begin{pgfscope}%
\pgfpathrectangle{\pgfqpoint{0.211875in}{0.211875in}}{\pgfqpoint{1.313625in}{1.279725in}}%
\pgfusepath{clip}%
\pgfsetbuttcap%
\pgfsetroundjoin%
\definecolor{currentfill}{rgb}{0.121569,0.466667,0.705882}%
\pgfsetfillcolor{currentfill}%
\pgfsetlinewidth{1.003750pt}%
\definecolor{currentstroke}{rgb}{0.121569,0.466667,0.705882}%
\pgfsetstrokecolor{currentstroke}%
\pgfsetdash{}{0pt}%
\pgfpathmoveto{\pgfqpoint{1.403753in}{1.124100in}}%
\pgfpathcurveto{\pgfqpoint{1.409577in}{1.124100in}}{\pgfqpoint{1.415163in}{1.126414in}}{\pgfqpoint{1.419281in}{1.130532in}}%
\pgfpathcurveto{\pgfqpoint{1.423399in}{1.134650in}}{\pgfqpoint{1.425713in}{1.140236in}}{\pgfqpoint{1.425713in}{1.146060in}}%
\pgfpathcurveto{\pgfqpoint{1.425713in}{1.151884in}}{\pgfqpoint{1.423399in}{1.157470in}}{\pgfqpoint{1.419281in}{1.161588in}}%
\pgfpathcurveto{\pgfqpoint{1.415163in}{1.165706in}}{\pgfqpoint{1.409577in}{1.168020in}}{\pgfqpoint{1.403753in}{1.168020in}}%
\pgfpathcurveto{\pgfqpoint{1.397929in}{1.168020in}}{\pgfqpoint{1.392342in}{1.165706in}}{\pgfqpoint{1.388224in}{1.161588in}}%
\pgfpathcurveto{\pgfqpoint{1.384106in}{1.157470in}}{\pgfqpoint{1.381792in}{1.151884in}}{\pgfqpoint{1.381792in}{1.146060in}}%
\pgfpathcurveto{\pgfqpoint{1.381792in}{1.140236in}}{\pgfqpoint{1.384106in}{1.134650in}}{\pgfqpoint{1.388224in}{1.130532in}}%
\pgfpathcurveto{\pgfqpoint{1.392342in}{1.126414in}}{\pgfqpoint{1.397929in}{1.124100in}}{\pgfqpoint{1.403753in}{1.124100in}}%
\pgfpathclose%
\pgfusepath{stroke,fill}%
\end{pgfscope}%
\begin{pgfscope}%
\pgfpathrectangle{\pgfqpoint{0.211875in}{0.211875in}}{\pgfqpoint{1.313625in}{1.279725in}}%
\pgfusepath{clip}%
\pgfsetbuttcap%
\pgfsetroundjoin%
\definecolor{currentfill}{rgb}{0.121569,0.466667,0.705882}%
\pgfsetfillcolor{currentfill}%
\pgfsetlinewidth{1.003750pt}%
\definecolor{currentstroke}{rgb}{0.121569,0.466667,0.705882}%
\pgfsetstrokecolor{currentstroke}%
\pgfsetdash{}{0pt}%
\pgfpathmoveto{\pgfqpoint{0.770504in}{0.613521in}}%
\pgfpathcurveto{\pgfqpoint{0.776328in}{0.613521in}}{\pgfqpoint{0.781914in}{0.615835in}}{\pgfqpoint{0.786032in}{0.619953in}}%
\pgfpathcurveto{\pgfqpoint{0.790150in}{0.624071in}}{\pgfqpoint{0.792464in}{0.629658in}}{\pgfqpoint{0.792464in}{0.635482in}}%
\pgfpathcurveto{\pgfqpoint{0.792464in}{0.641306in}}{\pgfqpoint{0.790150in}{0.646892in}}{\pgfqpoint{0.786032in}{0.651010in}}%
\pgfpathcurveto{\pgfqpoint{0.781914in}{0.655128in}}{\pgfqpoint{0.776328in}{0.657442in}}{\pgfqpoint{0.770504in}{0.657442in}}%
\pgfpathcurveto{\pgfqpoint{0.764680in}{0.657442in}}{\pgfqpoint{0.759094in}{0.655128in}}{\pgfqpoint{0.754976in}{0.651010in}}%
\pgfpathcurveto{\pgfqpoint{0.750858in}{0.646892in}}{\pgfqpoint{0.748544in}{0.641306in}}{\pgfqpoint{0.748544in}{0.635482in}}%
\pgfpathcurveto{\pgfqpoint{0.748544in}{0.629658in}}{\pgfqpoint{0.750858in}{0.624071in}}{\pgfqpoint{0.754976in}{0.619953in}}%
\pgfpathcurveto{\pgfqpoint{0.759094in}{0.615835in}}{\pgfqpoint{0.764680in}{0.613521in}}{\pgfqpoint{0.770504in}{0.613521in}}%
\pgfpathclose%
\pgfusepath{stroke,fill}%
\end{pgfscope}%
\begin{pgfscope}%
\pgfpathrectangle{\pgfqpoint{0.211875in}{0.211875in}}{\pgfqpoint{1.313625in}{1.279725in}}%
\pgfusepath{clip}%
\pgfsetbuttcap%
\pgfsetroundjoin%
\definecolor{currentfill}{rgb}{0.121569,0.466667,0.705882}%
\pgfsetfillcolor{currentfill}%
\pgfsetlinewidth{1.003750pt}%
\definecolor{currentstroke}{rgb}{0.121569,0.466667,0.705882}%
\pgfsetstrokecolor{currentstroke}%
\pgfsetdash{}{0pt}%
\pgfpathmoveto{\pgfqpoint{1.439026in}{1.135378in}}%
\pgfpathcurveto{\pgfqpoint{1.444850in}{1.135378in}}{\pgfqpoint{1.450436in}{1.137691in}}{\pgfqpoint{1.454554in}{1.141810in}}%
\pgfpathcurveto{\pgfqpoint{1.458672in}{1.145928in}}{\pgfqpoint{1.460986in}{1.151514in}}{\pgfqpoint{1.460986in}{1.157338in}}%
\pgfpathcurveto{\pgfqpoint{1.460986in}{1.163162in}}{\pgfqpoint{1.458672in}{1.168748in}}{\pgfqpoint{1.454554in}{1.172866in}}%
\pgfpathcurveto{\pgfqpoint{1.450436in}{1.176984in}}{\pgfqpoint{1.444850in}{1.179298in}}{\pgfqpoint{1.439026in}{1.179298in}}%
\pgfpathcurveto{\pgfqpoint{1.433202in}{1.179298in}}{\pgfqpoint{1.427616in}{1.176984in}}{\pgfqpoint{1.423498in}{1.172866in}}%
\pgfpathcurveto{\pgfqpoint{1.419380in}{1.168748in}}{\pgfqpoint{1.417066in}{1.163162in}}{\pgfqpoint{1.417066in}{1.157338in}}%
\pgfpathcurveto{\pgfqpoint{1.417066in}{1.151514in}}{\pgfqpoint{1.419380in}{1.145928in}}{\pgfqpoint{1.423498in}{1.141810in}}%
\pgfpathcurveto{\pgfqpoint{1.427616in}{1.137691in}}{\pgfqpoint{1.433202in}{1.135378in}}{\pgfqpoint{1.439026in}{1.135378in}}%
\pgfpathclose%
\pgfusepath{stroke,fill}%
\end{pgfscope}%
\begin{pgfscope}%
\pgfpathrectangle{\pgfqpoint{0.211875in}{0.211875in}}{\pgfqpoint{1.313625in}{1.279725in}}%
\pgfusepath{clip}%
\pgfsetbuttcap%
\pgfsetroundjoin%
\definecolor{currentfill}{rgb}{0.121569,0.466667,0.705882}%
\pgfsetfillcolor{currentfill}%
\pgfsetlinewidth{1.003750pt}%
\definecolor{currentstroke}{rgb}{0.121569,0.466667,0.705882}%
\pgfsetstrokecolor{currentstroke}%
\pgfsetdash{}{0pt}%
\pgfpathmoveto{\pgfqpoint{1.439732in}{1.144792in}}%
\pgfpathcurveto{\pgfqpoint{1.445556in}{1.144792in}}{\pgfqpoint{1.451142in}{1.147106in}}{\pgfqpoint{1.455260in}{1.151224in}}%
\pgfpathcurveto{\pgfqpoint{1.459378in}{1.155342in}}{\pgfqpoint{1.461692in}{1.160928in}}{\pgfqpoint{1.461692in}{1.166752in}}%
\pgfpathcurveto{\pgfqpoint{1.461692in}{1.172576in}}{\pgfqpoint{1.459378in}{1.178162in}}{\pgfqpoint{1.455260in}{1.182280in}}%
\pgfpathcurveto{\pgfqpoint{1.451142in}{1.186398in}}{\pgfqpoint{1.445556in}{1.188712in}}{\pgfqpoint{1.439732in}{1.188712in}}%
\pgfpathcurveto{\pgfqpoint{1.433908in}{1.188712in}}{\pgfqpoint{1.428322in}{1.186398in}}{\pgfqpoint{1.424204in}{1.182280in}}%
\pgfpathcurveto{\pgfqpoint{1.420085in}{1.178162in}}{\pgfqpoint{1.417772in}{1.172576in}}{\pgfqpoint{1.417772in}{1.166752in}}%
\pgfpathcurveto{\pgfqpoint{1.417772in}{1.160928in}}{\pgfqpoint{1.420085in}{1.155342in}}{\pgfqpoint{1.424204in}{1.151224in}}%
\pgfpathcurveto{\pgfqpoint{1.428322in}{1.147106in}}{\pgfqpoint{1.433908in}{1.144792in}}{\pgfqpoint{1.439732in}{1.144792in}}%
\pgfpathclose%
\pgfusepath{stroke,fill}%
\end{pgfscope}%
\begin{pgfscope}%
\pgfpathrectangle{\pgfqpoint{0.211875in}{0.211875in}}{\pgfqpoint{1.313625in}{1.279725in}}%
\pgfusepath{clip}%
\pgfsetbuttcap%
\pgfsetroundjoin%
\definecolor{currentfill}{rgb}{0.121569,0.466667,0.705882}%
\pgfsetfillcolor{currentfill}%
\pgfsetlinewidth{1.003750pt}%
\definecolor{currentstroke}{rgb}{0.121569,0.466667,0.705882}%
\pgfsetstrokecolor{currentstroke}%
\pgfsetdash{}{0pt}%
\pgfpathmoveto{\pgfqpoint{1.443989in}{0.667277in}}%
\pgfpathcurveto{\pgfqpoint{1.449813in}{0.667277in}}{\pgfqpoint{1.455399in}{0.669591in}}{\pgfqpoint{1.459517in}{0.673709in}}%
\pgfpathcurveto{\pgfqpoint{1.463636in}{0.677827in}}{\pgfqpoint{1.465949in}{0.683413in}}{\pgfqpoint{1.465949in}{0.689237in}}%
\pgfpathcurveto{\pgfqpoint{1.465949in}{0.695061in}}{\pgfqpoint{1.463636in}{0.700647in}}{\pgfqpoint{1.459517in}{0.704765in}}%
\pgfpathcurveto{\pgfqpoint{1.455399in}{0.708883in}}{\pgfqpoint{1.449813in}{0.711197in}}{\pgfqpoint{1.443989in}{0.711197in}}%
\pgfpathcurveto{\pgfqpoint{1.438165in}{0.711197in}}{\pgfqpoint{1.432579in}{0.708883in}}{\pgfqpoint{1.428461in}{0.704765in}}%
\pgfpathcurveto{\pgfqpoint{1.424343in}{0.700647in}}{\pgfqpoint{1.422029in}{0.695061in}}{\pgfqpoint{1.422029in}{0.689237in}}%
\pgfpathcurveto{\pgfqpoint{1.422029in}{0.683413in}}{\pgfqpoint{1.424343in}{0.677827in}}{\pgfqpoint{1.428461in}{0.673709in}}%
\pgfpathcurveto{\pgfqpoint{1.432579in}{0.669591in}}{\pgfqpoint{1.438165in}{0.667277in}}{\pgfqpoint{1.443989in}{0.667277in}}%
\pgfpathclose%
\pgfusepath{stroke,fill}%
\end{pgfscope}%
\begin{pgfscope}%
\pgfpathrectangle{\pgfqpoint{0.211875in}{0.211875in}}{\pgfqpoint{1.313625in}{1.279725in}}%
\pgfusepath{clip}%
\pgfsetbuttcap%
\pgfsetroundjoin%
\definecolor{currentfill}{rgb}{0.121569,0.466667,0.705882}%
\pgfsetfillcolor{currentfill}%
\pgfsetlinewidth{1.003750pt}%
\definecolor{currentstroke}{rgb}{0.121569,0.466667,0.705882}%
\pgfsetstrokecolor{currentstroke}%
\pgfsetdash{}{0pt}%
\pgfpathmoveto{\pgfqpoint{1.266953in}{1.360280in}}%
\pgfpathcurveto{\pgfqpoint{1.272777in}{1.360280in}}{\pgfqpoint{1.278363in}{1.362594in}}{\pgfqpoint{1.282481in}{1.366712in}}%
\pgfpathcurveto{\pgfqpoint{1.286599in}{1.370830in}}{\pgfqpoint{1.288913in}{1.376417in}}{\pgfqpoint{1.288913in}{1.382241in}}%
\pgfpathcurveto{\pgfqpoint{1.288913in}{1.388064in}}{\pgfqpoint{1.286599in}{1.393651in}}{\pgfqpoint{1.282481in}{1.397769in}}%
\pgfpathcurveto{\pgfqpoint{1.278363in}{1.401887in}}{\pgfqpoint{1.272777in}{1.404201in}}{\pgfqpoint{1.266953in}{1.404201in}}%
\pgfpathcurveto{\pgfqpoint{1.261129in}{1.404201in}}{\pgfqpoint{1.255543in}{1.401887in}}{\pgfqpoint{1.251425in}{1.397769in}}%
\pgfpathcurveto{\pgfqpoint{1.247306in}{1.393651in}}{\pgfqpoint{1.244993in}{1.388064in}}{\pgfqpoint{1.244993in}{1.382241in}}%
\pgfpathcurveto{\pgfqpoint{1.244993in}{1.376417in}}{\pgfqpoint{1.247306in}{1.370830in}}{\pgfqpoint{1.251425in}{1.366712in}}%
\pgfpathcurveto{\pgfqpoint{1.255543in}{1.362594in}}{\pgfqpoint{1.261129in}{1.360280in}}{\pgfqpoint{1.266953in}{1.360280in}}%
\pgfpathclose%
\pgfusepath{stroke,fill}%
\end{pgfscope}%
\begin{pgfscope}%
\pgfpathrectangle{\pgfqpoint{0.211875in}{0.211875in}}{\pgfqpoint{1.313625in}{1.279725in}}%
\pgfusepath{clip}%
\pgfsetbuttcap%
\pgfsetroundjoin%
\definecolor{currentfill}{rgb}{0.121569,0.466667,0.705882}%
\pgfsetfillcolor{currentfill}%
\pgfsetlinewidth{1.003750pt}%
\definecolor{currentstroke}{rgb}{0.121569,0.466667,0.705882}%
\pgfsetstrokecolor{currentstroke}%
\pgfsetdash{}{0pt}%
\pgfpathmoveto{\pgfqpoint{1.301375in}{0.668151in}}%
\pgfpathcurveto{\pgfqpoint{1.307199in}{0.668151in}}{\pgfqpoint{1.312785in}{0.670465in}}{\pgfqpoint{1.316903in}{0.674583in}}%
\pgfpathcurveto{\pgfqpoint{1.321021in}{0.678701in}}{\pgfqpoint{1.323335in}{0.684288in}}{\pgfqpoint{1.323335in}{0.690111in}}%
\pgfpathcurveto{\pgfqpoint{1.323335in}{0.695935in}}{\pgfqpoint{1.321021in}{0.701522in}}{\pgfqpoint{1.316903in}{0.705640in}}%
\pgfpathcurveto{\pgfqpoint{1.312785in}{0.709758in}}{\pgfqpoint{1.307199in}{0.712072in}}{\pgfqpoint{1.301375in}{0.712072in}}%
\pgfpathcurveto{\pgfqpoint{1.295551in}{0.712072in}}{\pgfqpoint{1.289964in}{0.709758in}}{\pgfqpoint{1.285846in}{0.705640in}}%
\pgfpathcurveto{\pgfqpoint{1.281728in}{0.701522in}}{\pgfqpoint{1.279414in}{0.695935in}}{\pgfqpoint{1.279414in}{0.690111in}}%
\pgfpathcurveto{\pgfqpoint{1.279414in}{0.684288in}}{\pgfqpoint{1.281728in}{0.678701in}}{\pgfqpoint{1.285846in}{0.674583in}}%
\pgfpathcurveto{\pgfqpoint{1.289964in}{0.670465in}}{\pgfqpoint{1.295551in}{0.668151in}}{\pgfqpoint{1.301375in}{0.668151in}}%
\pgfpathclose%
\pgfusepath{stroke,fill}%
\end{pgfscope}%
\begin{pgfscope}%
\pgfpathrectangle{\pgfqpoint{0.211875in}{0.211875in}}{\pgfqpoint{1.313625in}{1.279725in}}%
\pgfusepath{clip}%
\pgfsetbuttcap%
\pgfsetroundjoin%
\definecolor{currentfill}{rgb}{0.121569,0.466667,0.705882}%
\pgfsetfillcolor{currentfill}%
\pgfsetlinewidth{1.003750pt}%
\definecolor{currentstroke}{rgb}{0.121569,0.466667,0.705882}%
\pgfsetstrokecolor{currentstroke}%
\pgfsetdash{}{0pt}%
\pgfpathmoveto{\pgfqpoint{0.425721in}{1.240663in}}%
\pgfpathcurveto{\pgfqpoint{0.431545in}{1.240663in}}{\pgfqpoint{0.437131in}{1.242977in}}{\pgfqpoint{0.441249in}{1.247095in}}%
\pgfpathcurveto{\pgfqpoint{0.445367in}{1.251214in}}{\pgfqpoint{0.447681in}{1.256800in}}{\pgfqpoint{0.447681in}{1.262624in}}%
\pgfpathcurveto{\pgfqpoint{0.447681in}{1.268448in}}{\pgfqpoint{0.445367in}{1.274034in}}{\pgfqpoint{0.441249in}{1.278152in}}%
\pgfpathcurveto{\pgfqpoint{0.437131in}{1.282270in}}{\pgfqpoint{0.431545in}{1.284584in}}{\pgfqpoint{0.425721in}{1.284584in}}%
\pgfpathcurveto{\pgfqpoint{0.419897in}{1.284584in}}{\pgfqpoint{0.414311in}{1.282270in}}{\pgfqpoint{0.410192in}{1.278152in}}%
\pgfpathcurveto{\pgfqpoint{0.406074in}{1.274034in}}{\pgfqpoint{0.403760in}{1.268448in}}{\pgfqpoint{0.403760in}{1.262624in}}%
\pgfpathcurveto{\pgfqpoint{0.403760in}{1.256800in}}{\pgfqpoint{0.406074in}{1.251214in}}{\pgfqpoint{0.410192in}{1.247095in}}%
\pgfpathcurveto{\pgfqpoint{0.414311in}{1.242977in}}{\pgfqpoint{0.419897in}{1.240663in}}{\pgfqpoint{0.425721in}{1.240663in}}%
\pgfpathclose%
\pgfusepath{stroke,fill}%
\end{pgfscope}%
\begin{pgfscope}%
\pgfpathrectangle{\pgfqpoint{0.211875in}{0.211875in}}{\pgfqpoint{1.313625in}{1.279725in}}%
\pgfusepath{clip}%
\pgfsetbuttcap%
\pgfsetroundjoin%
\definecolor{currentfill}{rgb}{0.121569,0.466667,0.705882}%
\pgfsetfillcolor{currentfill}%
\pgfsetlinewidth{1.003750pt}%
\definecolor{currentstroke}{rgb}{0.121569,0.466667,0.705882}%
\pgfsetstrokecolor{currentstroke}%
\pgfsetdash{}{0pt}%
\pgfpathmoveto{\pgfqpoint{1.443012in}{1.127966in}}%
\pgfpathcurveto{\pgfqpoint{1.448836in}{1.127966in}}{\pgfqpoint{1.454422in}{1.130280in}}{\pgfqpoint{1.458540in}{1.134398in}}%
\pgfpathcurveto{\pgfqpoint{1.462658in}{1.138516in}}{\pgfqpoint{1.464972in}{1.144102in}}{\pgfqpoint{1.464972in}{1.149926in}}%
\pgfpathcurveto{\pgfqpoint{1.464972in}{1.155750in}}{\pgfqpoint{1.462658in}{1.161337in}}{\pgfqpoint{1.458540in}{1.165455in}}%
\pgfpathcurveto{\pgfqpoint{1.454422in}{1.169573in}}{\pgfqpoint{1.448836in}{1.171887in}}{\pgfqpoint{1.443012in}{1.171887in}}%
\pgfpathcurveto{\pgfqpoint{1.437188in}{1.171887in}}{\pgfqpoint{1.431602in}{1.169573in}}{\pgfqpoint{1.427484in}{1.165455in}}%
\pgfpathcurveto{\pgfqpoint{1.423366in}{1.161337in}}{\pgfqpoint{1.421052in}{1.155750in}}{\pgfqpoint{1.421052in}{1.149926in}}%
\pgfpathcurveto{\pgfqpoint{1.421052in}{1.144102in}}{\pgfqpoint{1.423366in}{1.138516in}}{\pgfqpoint{1.427484in}{1.134398in}}%
\pgfpathcurveto{\pgfqpoint{1.431602in}{1.130280in}}{\pgfqpoint{1.437188in}{1.127966in}}{\pgfqpoint{1.443012in}{1.127966in}}%
\pgfpathclose%
\pgfusepath{stroke,fill}%
\end{pgfscope}%
\begin{pgfscope}%
\pgfpathrectangle{\pgfqpoint{0.211875in}{0.211875in}}{\pgfqpoint{1.313625in}{1.279725in}}%
\pgfusepath{clip}%
\pgfsetbuttcap%
\pgfsetroundjoin%
\definecolor{currentfill}{rgb}{0.121569,0.466667,0.705882}%
\pgfsetfillcolor{currentfill}%
\pgfsetlinewidth{1.003750pt}%
\definecolor{currentstroke}{rgb}{0.121569,0.466667,0.705882}%
\pgfsetstrokecolor{currentstroke}%
\pgfsetdash{}{0pt}%
\pgfpathmoveto{\pgfqpoint{1.232041in}{0.664241in}}%
\pgfpathcurveto{\pgfqpoint{1.237865in}{0.664241in}}{\pgfqpoint{1.243451in}{0.666555in}}{\pgfqpoint{1.247569in}{0.670673in}}%
\pgfpathcurveto{\pgfqpoint{1.251687in}{0.674791in}}{\pgfqpoint{1.254001in}{0.680377in}}{\pgfqpoint{1.254001in}{0.686201in}}%
\pgfpathcurveto{\pgfqpoint{1.254001in}{0.692025in}}{\pgfqpoint{1.251687in}{0.697611in}}{\pgfqpoint{1.247569in}{0.701729in}}%
\pgfpathcurveto{\pgfqpoint{1.243451in}{0.705848in}}{\pgfqpoint{1.237865in}{0.708161in}}{\pgfqpoint{1.232041in}{0.708161in}}%
\pgfpathcurveto{\pgfqpoint{1.226217in}{0.708161in}}{\pgfqpoint{1.220631in}{0.705848in}}{\pgfqpoint{1.216513in}{0.701729in}}%
\pgfpathcurveto{\pgfqpoint{1.212395in}{0.697611in}}{\pgfqpoint{1.210081in}{0.692025in}}{\pgfqpoint{1.210081in}{0.686201in}}%
\pgfpathcurveto{\pgfqpoint{1.210081in}{0.680377in}}{\pgfqpoint{1.212395in}{0.674791in}}{\pgfqpoint{1.216513in}{0.670673in}}%
\pgfpathcurveto{\pgfqpoint{1.220631in}{0.666555in}}{\pgfqpoint{1.226217in}{0.664241in}}{\pgfqpoint{1.232041in}{0.664241in}}%
\pgfpathclose%
\pgfusepath{stroke,fill}%
\end{pgfscope}%
\begin{pgfscope}%
\pgfpathrectangle{\pgfqpoint{0.211875in}{0.211875in}}{\pgfqpoint{1.313625in}{1.279725in}}%
\pgfusepath{clip}%
\pgfsetbuttcap%
\pgfsetroundjoin%
\definecolor{currentfill}{rgb}{0.121569,0.466667,0.705882}%
\pgfsetfillcolor{currentfill}%
\pgfsetlinewidth{1.003750pt}%
\definecolor{currentstroke}{rgb}{0.121569,0.466667,0.705882}%
\pgfsetstrokecolor{currentstroke}%
\pgfsetdash{}{0pt}%
\pgfpathmoveto{\pgfqpoint{1.283710in}{1.342033in}}%
\pgfpathcurveto{\pgfqpoint{1.289534in}{1.342033in}}{\pgfqpoint{1.295120in}{1.344347in}}{\pgfqpoint{1.299239in}{1.348465in}}%
\pgfpathcurveto{\pgfqpoint{1.303357in}{1.352584in}}{\pgfqpoint{1.305671in}{1.358170in}}{\pgfqpoint{1.305671in}{1.363994in}}%
\pgfpathcurveto{\pgfqpoint{1.305671in}{1.369818in}}{\pgfqpoint{1.303357in}{1.375404in}}{\pgfqpoint{1.299239in}{1.379522in}}%
\pgfpathcurveto{\pgfqpoint{1.295120in}{1.383640in}}{\pgfqpoint{1.289534in}{1.385954in}}{\pgfqpoint{1.283710in}{1.385954in}}%
\pgfpathcurveto{\pgfqpoint{1.277886in}{1.385954in}}{\pgfqpoint{1.272300in}{1.383640in}}{\pgfqpoint{1.268182in}{1.379522in}}%
\pgfpathcurveto{\pgfqpoint{1.264064in}{1.375404in}}{\pgfqpoint{1.261750in}{1.369818in}}{\pgfqpoint{1.261750in}{1.363994in}}%
\pgfpathcurveto{\pgfqpoint{1.261750in}{1.358170in}}{\pgfqpoint{1.264064in}{1.352584in}}{\pgfqpoint{1.268182in}{1.348465in}}%
\pgfpathcurveto{\pgfqpoint{1.272300in}{1.344347in}}{\pgfqpoint{1.277886in}{1.342033in}}{\pgfqpoint{1.283710in}{1.342033in}}%
\pgfpathclose%
\pgfusepath{stroke,fill}%
\end{pgfscope}%
\begin{pgfscope}%
\pgfpathrectangle{\pgfqpoint{0.211875in}{0.211875in}}{\pgfqpoint{1.313625in}{1.279725in}}%
\pgfusepath{clip}%
\pgfsetbuttcap%
\pgfsetroundjoin%
\definecolor{currentfill}{rgb}{0.121569,0.466667,0.705882}%
\pgfsetfillcolor{currentfill}%
\pgfsetlinewidth{1.003750pt}%
\definecolor{currentstroke}{rgb}{0.121569,0.466667,0.705882}%
\pgfsetstrokecolor{currentstroke}%
\pgfsetdash{}{0pt}%
\pgfpathmoveto{\pgfqpoint{1.264491in}{1.320434in}}%
\pgfpathcurveto{\pgfqpoint{1.270315in}{1.320434in}}{\pgfqpoint{1.275901in}{1.322748in}}{\pgfqpoint{1.280019in}{1.326866in}}%
\pgfpathcurveto{\pgfqpoint{1.284138in}{1.330984in}}{\pgfqpoint{1.286451in}{1.336570in}}{\pgfqpoint{1.286451in}{1.342394in}}%
\pgfpathcurveto{\pgfqpoint{1.286451in}{1.348218in}}{\pgfqpoint{1.284138in}{1.353804in}}{\pgfqpoint{1.280019in}{1.357923in}}%
\pgfpathcurveto{\pgfqpoint{1.275901in}{1.362041in}}{\pgfqpoint{1.270315in}{1.364355in}}{\pgfqpoint{1.264491in}{1.364355in}}%
\pgfpathcurveto{\pgfqpoint{1.258667in}{1.364355in}}{\pgfqpoint{1.253081in}{1.362041in}}{\pgfqpoint{1.248963in}{1.357923in}}%
\pgfpathcurveto{\pgfqpoint{1.244845in}{1.353804in}}{\pgfqpoint{1.242531in}{1.348218in}}{\pgfqpoint{1.242531in}{1.342394in}}%
\pgfpathcurveto{\pgfqpoint{1.242531in}{1.336570in}}{\pgfqpoint{1.244845in}{1.330984in}}{\pgfqpoint{1.248963in}{1.326866in}}%
\pgfpathcurveto{\pgfqpoint{1.253081in}{1.322748in}}{\pgfqpoint{1.258667in}{1.320434in}}{\pgfqpoint{1.264491in}{1.320434in}}%
\pgfpathclose%
\pgfusepath{stroke,fill}%
\end{pgfscope}%
\begin{pgfscope}%
\pgfpathrectangle{\pgfqpoint{0.211875in}{0.211875in}}{\pgfqpoint{1.313625in}{1.279725in}}%
\pgfusepath{clip}%
\pgfsetbuttcap%
\pgfsetroundjoin%
\definecolor{currentfill}{rgb}{0.121569,0.466667,0.705882}%
\pgfsetfillcolor{currentfill}%
\pgfsetlinewidth{1.003750pt}%
\definecolor{currentstroke}{rgb}{0.121569,0.466667,0.705882}%
\pgfsetstrokecolor{currentstroke}%
\pgfsetdash{}{0pt}%
\pgfpathmoveto{\pgfqpoint{1.322162in}{0.668454in}}%
\pgfpathcurveto{\pgfqpoint{1.327985in}{0.668454in}}{\pgfqpoint{1.333572in}{0.670768in}}{\pgfqpoint{1.337690in}{0.674886in}}%
\pgfpathcurveto{\pgfqpoint{1.341808in}{0.679004in}}{\pgfqpoint{1.344122in}{0.684590in}}{\pgfqpoint{1.344122in}{0.690414in}}%
\pgfpathcurveto{\pgfqpoint{1.344122in}{0.696238in}}{\pgfqpoint{1.341808in}{0.701824in}}{\pgfqpoint{1.337690in}{0.705943in}}%
\pgfpathcurveto{\pgfqpoint{1.333572in}{0.710061in}}{\pgfqpoint{1.327985in}{0.712375in}}{\pgfqpoint{1.322162in}{0.712375in}}%
\pgfpathcurveto{\pgfqpoint{1.316338in}{0.712375in}}{\pgfqpoint{1.310751in}{0.710061in}}{\pgfqpoint{1.306633in}{0.705943in}}%
\pgfpathcurveto{\pgfqpoint{1.302515in}{0.701824in}}{\pgfqpoint{1.300201in}{0.696238in}}{\pgfqpoint{1.300201in}{0.690414in}}%
\pgfpathcurveto{\pgfqpoint{1.300201in}{0.684590in}}{\pgfqpoint{1.302515in}{0.679004in}}{\pgfqpoint{1.306633in}{0.674886in}}%
\pgfpathcurveto{\pgfqpoint{1.310751in}{0.670768in}}{\pgfqpoint{1.316338in}{0.668454in}}{\pgfqpoint{1.322162in}{0.668454in}}%
\pgfpathclose%
\pgfusepath{stroke,fill}%
\end{pgfscope}%
\begin{pgfscope}%
\pgfpathrectangle{\pgfqpoint{0.211875in}{0.211875in}}{\pgfqpoint{1.313625in}{1.279725in}}%
\pgfusepath{clip}%
\pgfsetbuttcap%
\pgfsetroundjoin%
\definecolor{currentfill}{rgb}{0.121569,0.466667,0.705882}%
\pgfsetfillcolor{currentfill}%
\pgfsetlinewidth{1.003750pt}%
\definecolor{currentstroke}{rgb}{0.121569,0.466667,0.705882}%
\pgfsetstrokecolor{currentstroke}%
\pgfsetdash{}{0pt}%
\pgfpathmoveto{\pgfqpoint{1.422977in}{1.139507in}}%
\pgfpathcurveto{\pgfqpoint{1.428801in}{1.139507in}}{\pgfqpoint{1.434387in}{1.141821in}}{\pgfqpoint{1.438505in}{1.145939in}}%
\pgfpathcurveto{\pgfqpoint{1.442623in}{1.150057in}}{\pgfqpoint{1.444937in}{1.155644in}}{\pgfqpoint{1.444937in}{1.161467in}}%
\pgfpathcurveto{\pgfqpoint{1.444937in}{1.167291in}}{\pgfqpoint{1.442623in}{1.172878in}}{\pgfqpoint{1.438505in}{1.176996in}}%
\pgfpathcurveto{\pgfqpoint{1.434387in}{1.181114in}}{\pgfqpoint{1.428801in}{1.183428in}}{\pgfqpoint{1.422977in}{1.183428in}}%
\pgfpathcurveto{\pgfqpoint{1.417153in}{1.183428in}}{\pgfqpoint{1.411567in}{1.181114in}}{\pgfqpoint{1.407448in}{1.176996in}}%
\pgfpathcurveto{\pgfqpoint{1.403330in}{1.172878in}}{\pgfqpoint{1.401016in}{1.167291in}}{\pgfqpoint{1.401016in}{1.161467in}}%
\pgfpathcurveto{\pgfqpoint{1.401016in}{1.155644in}}{\pgfqpoint{1.403330in}{1.150057in}}{\pgfqpoint{1.407448in}{1.145939in}}%
\pgfpathcurveto{\pgfqpoint{1.411567in}{1.141821in}}{\pgfqpoint{1.417153in}{1.139507in}}{\pgfqpoint{1.422977in}{1.139507in}}%
\pgfpathclose%
\pgfusepath{stroke,fill}%
\end{pgfscope}%
\begin{pgfscope}%
\pgfpathrectangle{\pgfqpoint{0.211875in}{0.211875in}}{\pgfqpoint{1.313625in}{1.279725in}}%
\pgfusepath{clip}%
\pgfsetbuttcap%
\pgfsetroundjoin%
\definecolor{currentfill}{rgb}{0.121569,0.466667,0.705882}%
\pgfsetfillcolor{currentfill}%
\pgfsetlinewidth{1.003750pt}%
\definecolor{currentstroke}{rgb}{0.121569,0.466667,0.705882}%
\pgfsetstrokecolor{currentstroke}%
\pgfsetdash{}{0pt}%
\pgfpathmoveto{\pgfqpoint{1.417122in}{1.148751in}}%
\pgfpathcurveto{\pgfqpoint{1.422946in}{1.148751in}}{\pgfqpoint{1.428532in}{1.151065in}}{\pgfqpoint{1.432650in}{1.155183in}}%
\pgfpathcurveto{\pgfqpoint{1.436768in}{1.159301in}}{\pgfqpoint{1.439082in}{1.164887in}}{\pgfqpoint{1.439082in}{1.170711in}}%
\pgfpathcurveto{\pgfqpoint{1.439082in}{1.176535in}}{\pgfqpoint{1.436768in}{1.182121in}}{\pgfqpoint{1.432650in}{1.186239in}}%
\pgfpathcurveto{\pgfqpoint{1.428532in}{1.190357in}}{\pgfqpoint{1.422946in}{1.192671in}}{\pgfqpoint{1.417122in}{1.192671in}}%
\pgfpathcurveto{\pgfqpoint{1.411298in}{1.192671in}}{\pgfqpoint{1.405712in}{1.190357in}}{\pgfqpoint{1.401593in}{1.186239in}}%
\pgfpathcurveto{\pgfqpoint{1.397475in}{1.182121in}}{\pgfqpoint{1.395161in}{1.176535in}}{\pgfqpoint{1.395161in}{1.170711in}}%
\pgfpathcurveto{\pgfqpoint{1.395161in}{1.164887in}}{\pgfqpoint{1.397475in}{1.159301in}}{\pgfqpoint{1.401593in}{1.155183in}}%
\pgfpathcurveto{\pgfqpoint{1.405712in}{1.151065in}}{\pgfqpoint{1.411298in}{1.148751in}}{\pgfqpoint{1.417122in}{1.148751in}}%
\pgfpathclose%
\pgfusepath{stroke,fill}%
\end{pgfscope}%
\begin{pgfscope}%
\pgfpathrectangle{\pgfqpoint{0.211875in}{0.211875in}}{\pgfqpoint{1.313625in}{1.279725in}}%
\pgfusepath{clip}%
\pgfsetbuttcap%
\pgfsetroundjoin%
\definecolor{currentfill}{rgb}{0.121569,0.466667,0.705882}%
\pgfsetfillcolor{currentfill}%
\pgfsetlinewidth{1.003750pt}%
\definecolor{currentstroke}{rgb}{0.121569,0.466667,0.705882}%
\pgfsetstrokecolor{currentstroke}%
\pgfsetdash{}{0pt}%
\pgfpathmoveto{\pgfqpoint{0.741122in}{0.594223in}}%
\pgfpathcurveto{\pgfqpoint{0.746946in}{0.594223in}}{\pgfqpoint{0.752532in}{0.596537in}}{\pgfqpoint{0.756650in}{0.600655in}}%
\pgfpathcurveto{\pgfqpoint{0.760768in}{0.604773in}}{\pgfqpoint{0.763082in}{0.610359in}}{\pgfqpoint{0.763082in}{0.616183in}}%
\pgfpathcurveto{\pgfqpoint{0.763082in}{0.622007in}}{\pgfqpoint{0.760768in}{0.627593in}}{\pgfqpoint{0.756650in}{0.631712in}}%
\pgfpathcurveto{\pgfqpoint{0.752532in}{0.635830in}}{\pgfqpoint{0.746946in}{0.638144in}}{\pgfqpoint{0.741122in}{0.638144in}}%
\pgfpathcurveto{\pgfqpoint{0.735298in}{0.638144in}}{\pgfqpoint{0.729712in}{0.635830in}}{\pgfqpoint{0.725594in}{0.631712in}}%
\pgfpathcurveto{\pgfqpoint{0.721476in}{0.627593in}}{\pgfqpoint{0.719162in}{0.622007in}}{\pgfqpoint{0.719162in}{0.616183in}}%
\pgfpathcurveto{\pgfqpoint{0.719162in}{0.610359in}}{\pgfqpoint{0.721476in}{0.604773in}}{\pgfqpoint{0.725594in}{0.600655in}}%
\pgfpathcurveto{\pgfqpoint{0.729712in}{0.596537in}}{\pgfqpoint{0.735298in}{0.594223in}}{\pgfqpoint{0.741122in}{0.594223in}}%
\pgfpathclose%
\pgfusepath{stroke,fill}%
\end{pgfscope}%
\begin{pgfscope}%
\pgfpathrectangle{\pgfqpoint{0.211875in}{0.211875in}}{\pgfqpoint{1.313625in}{1.279725in}}%
\pgfusepath{clip}%
\pgfsetbuttcap%
\pgfsetroundjoin%
\definecolor{currentfill}{rgb}{0.121569,0.466667,0.705882}%
\pgfsetfillcolor{currentfill}%
\pgfsetlinewidth{1.003750pt}%
\definecolor{currentstroke}{rgb}{0.121569,0.466667,0.705882}%
\pgfsetstrokecolor{currentstroke}%
\pgfsetdash{}{0pt}%
\pgfpathmoveto{\pgfqpoint{1.272494in}{1.356087in}}%
\pgfpathcurveto{\pgfqpoint{1.278318in}{1.356087in}}{\pgfqpoint{1.283905in}{1.358401in}}{\pgfqpoint{1.288023in}{1.362519in}}%
\pgfpathcurveto{\pgfqpoint{1.292141in}{1.366637in}}{\pgfqpoint{1.294455in}{1.372224in}}{\pgfqpoint{1.294455in}{1.378047in}}%
\pgfpathcurveto{\pgfqpoint{1.294455in}{1.383871in}}{\pgfqpoint{1.292141in}{1.389458in}}{\pgfqpoint{1.288023in}{1.393576in}}%
\pgfpathcurveto{\pgfqpoint{1.283905in}{1.397694in}}{\pgfqpoint{1.278318in}{1.400008in}}{\pgfqpoint{1.272494in}{1.400008in}}%
\pgfpathcurveto{\pgfqpoint{1.266670in}{1.400008in}}{\pgfqpoint{1.261084in}{1.397694in}}{\pgfqpoint{1.256966in}{1.393576in}}%
\pgfpathcurveto{\pgfqpoint{1.252848in}{1.389458in}}{\pgfqpoint{1.250534in}{1.383871in}}{\pgfqpoint{1.250534in}{1.378047in}}%
\pgfpathcurveto{\pgfqpoint{1.250534in}{1.372224in}}{\pgfqpoint{1.252848in}{1.366637in}}{\pgfqpoint{1.256966in}{1.362519in}}%
\pgfpathcurveto{\pgfqpoint{1.261084in}{1.358401in}}{\pgfqpoint{1.266670in}{1.356087in}}{\pgfqpoint{1.272494in}{1.356087in}}%
\pgfpathclose%
\pgfusepath{stroke,fill}%
\end{pgfscope}%
\begin{pgfscope}%
\pgfpathrectangle{\pgfqpoint{0.211875in}{0.211875in}}{\pgfqpoint{1.313625in}{1.279725in}}%
\pgfusepath{clip}%
\pgfsetbuttcap%
\pgfsetroundjoin%
\definecolor{currentfill}{rgb}{0.121569,0.466667,0.705882}%
\pgfsetfillcolor{currentfill}%
\pgfsetlinewidth{1.003750pt}%
\definecolor{currentstroke}{rgb}{0.121569,0.466667,0.705882}%
\pgfsetstrokecolor{currentstroke}%
\pgfsetdash{}{0pt}%
\pgfpathmoveto{\pgfqpoint{1.196068in}{0.675042in}}%
\pgfpathcurveto{\pgfqpoint{1.201892in}{0.675042in}}{\pgfqpoint{1.207478in}{0.677356in}}{\pgfqpoint{1.211596in}{0.681474in}}%
\pgfpathcurveto{\pgfqpoint{1.215714in}{0.685592in}}{\pgfqpoint{1.218028in}{0.691178in}}{\pgfqpoint{1.218028in}{0.697002in}}%
\pgfpathcurveto{\pgfqpoint{1.218028in}{0.702826in}}{\pgfqpoint{1.215714in}{0.708412in}}{\pgfqpoint{1.211596in}{0.712530in}}%
\pgfpathcurveto{\pgfqpoint{1.207478in}{0.716648in}}{\pgfqpoint{1.201892in}{0.718962in}}{\pgfqpoint{1.196068in}{0.718962in}}%
\pgfpathcurveto{\pgfqpoint{1.190244in}{0.718962in}}{\pgfqpoint{1.184658in}{0.716648in}}{\pgfqpoint{1.180539in}{0.712530in}}%
\pgfpathcurveto{\pgfqpoint{1.176421in}{0.708412in}}{\pgfqpoint{1.174107in}{0.702826in}}{\pgfqpoint{1.174107in}{0.697002in}}%
\pgfpathcurveto{\pgfqpoint{1.174107in}{0.691178in}}{\pgfqpoint{1.176421in}{0.685592in}}{\pgfqpoint{1.180539in}{0.681474in}}%
\pgfpathcurveto{\pgfqpoint{1.184658in}{0.677356in}}{\pgfqpoint{1.190244in}{0.675042in}}{\pgfqpoint{1.196068in}{0.675042in}}%
\pgfpathclose%
\pgfusepath{stroke,fill}%
\end{pgfscope}%
\begin{pgfscope}%
\pgfpathrectangle{\pgfqpoint{0.211875in}{0.211875in}}{\pgfqpoint{1.313625in}{1.279725in}}%
\pgfusepath{clip}%
\pgfsetbuttcap%
\pgfsetroundjoin%
\definecolor{currentfill}{rgb}{0.121569,0.466667,0.705882}%
\pgfsetfillcolor{currentfill}%
\pgfsetlinewidth{1.003750pt}%
\definecolor{currentstroke}{rgb}{0.121569,0.466667,0.705882}%
\pgfsetstrokecolor{currentstroke}%
\pgfsetdash{}{0pt}%
\pgfpathmoveto{\pgfqpoint{1.272183in}{1.330940in}}%
\pgfpathcurveto{\pgfqpoint{1.278007in}{1.330940in}}{\pgfqpoint{1.283593in}{1.333254in}}{\pgfqpoint{1.287711in}{1.337372in}}%
\pgfpathcurveto{\pgfqpoint{1.291829in}{1.341490in}}{\pgfqpoint{1.294143in}{1.347076in}}{\pgfqpoint{1.294143in}{1.352900in}}%
\pgfpathcurveto{\pgfqpoint{1.294143in}{1.358724in}}{\pgfqpoint{1.291829in}{1.364310in}}{\pgfqpoint{1.287711in}{1.368429in}}%
\pgfpathcurveto{\pgfqpoint{1.283593in}{1.372547in}}{\pgfqpoint{1.278007in}{1.374861in}}{\pgfqpoint{1.272183in}{1.374861in}}%
\pgfpathcurveto{\pgfqpoint{1.266359in}{1.374861in}}{\pgfqpoint{1.260773in}{1.372547in}}{\pgfqpoint{1.256655in}{1.368429in}}%
\pgfpathcurveto{\pgfqpoint{1.252536in}{1.364310in}}{\pgfqpoint{1.250223in}{1.358724in}}{\pgfqpoint{1.250223in}{1.352900in}}%
\pgfpathcurveto{\pgfqpoint{1.250223in}{1.347076in}}{\pgfqpoint{1.252536in}{1.341490in}}{\pgfqpoint{1.256655in}{1.337372in}}%
\pgfpathcurveto{\pgfqpoint{1.260773in}{1.333254in}}{\pgfqpoint{1.266359in}{1.330940in}}{\pgfqpoint{1.272183in}{1.330940in}}%
\pgfpathclose%
\pgfusepath{stroke,fill}%
\end{pgfscope}%
\begin{pgfscope}%
\pgfpathrectangle{\pgfqpoint{0.211875in}{0.211875in}}{\pgfqpoint{1.313625in}{1.279725in}}%
\pgfusepath{clip}%
\pgfsetbuttcap%
\pgfsetroundjoin%
\definecolor{currentfill}{rgb}{0.121569,0.466667,0.705882}%
\pgfsetfillcolor{currentfill}%
\pgfsetlinewidth{1.003750pt}%
\definecolor{currentstroke}{rgb}{0.121569,0.466667,0.705882}%
\pgfsetstrokecolor{currentstroke}%
\pgfsetdash{}{0pt}%
\pgfpathmoveto{\pgfqpoint{1.426029in}{1.119726in}}%
\pgfpathcurveto{\pgfqpoint{1.431853in}{1.119726in}}{\pgfqpoint{1.437439in}{1.122040in}}{\pgfqpoint{1.441557in}{1.126159in}}%
\pgfpathcurveto{\pgfqpoint{1.445675in}{1.130277in}}{\pgfqpoint{1.447989in}{1.135863in}}{\pgfqpoint{1.447989in}{1.141687in}}%
\pgfpathcurveto{\pgfqpoint{1.447989in}{1.147511in}}{\pgfqpoint{1.445675in}{1.153097in}}{\pgfqpoint{1.441557in}{1.157215in}}%
\pgfpathcurveto{\pgfqpoint{1.437439in}{1.161333in}}{\pgfqpoint{1.431853in}{1.163647in}}{\pgfqpoint{1.426029in}{1.163647in}}%
\pgfpathcurveto{\pgfqpoint{1.420205in}{1.163647in}}{\pgfqpoint{1.414619in}{1.161333in}}{\pgfqpoint{1.410501in}{1.157215in}}%
\pgfpathcurveto{\pgfqpoint{1.406383in}{1.153097in}}{\pgfqpoint{1.404069in}{1.147511in}}{\pgfqpoint{1.404069in}{1.141687in}}%
\pgfpathcurveto{\pgfqpoint{1.404069in}{1.135863in}}{\pgfqpoint{1.406383in}{1.130277in}}{\pgfqpoint{1.410501in}{1.126159in}}%
\pgfpathcurveto{\pgfqpoint{1.414619in}{1.122040in}}{\pgfqpoint{1.420205in}{1.119726in}}{\pgfqpoint{1.426029in}{1.119726in}}%
\pgfpathclose%
\pgfusepath{stroke,fill}%
\end{pgfscope}%
\begin{pgfscope}%
\pgfpathrectangle{\pgfqpoint{0.211875in}{0.211875in}}{\pgfqpoint{1.313625in}{1.279725in}}%
\pgfusepath{clip}%
\pgfsetbuttcap%
\pgfsetroundjoin%
\definecolor{currentfill}{rgb}{0.121569,0.466667,0.705882}%
\pgfsetfillcolor{currentfill}%
\pgfsetlinewidth{1.003750pt}%
\definecolor{currentstroke}{rgb}{0.121569,0.466667,0.705882}%
\pgfsetstrokecolor{currentstroke}%
\pgfsetdash{}{0pt}%
\pgfpathmoveto{\pgfqpoint{1.271209in}{1.325575in}}%
\pgfpathcurveto{\pgfqpoint{1.277033in}{1.325575in}}{\pgfqpoint{1.282620in}{1.327889in}}{\pgfqpoint{1.286738in}{1.332007in}}%
\pgfpathcurveto{\pgfqpoint{1.290856in}{1.336125in}}{\pgfqpoint{1.293170in}{1.341711in}}{\pgfqpoint{1.293170in}{1.347535in}}%
\pgfpathcurveto{\pgfqpoint{1.293170in}{1.353359in}}{\pgfqpoint{1.290856in}{1.358945in}}{\pgfqpoint{1.286738in}{1.363063in}}%
\pgfpathcurveto{\pgfqpoint{1.282620in}{1.367181in}}{\pgfqpoint{1.277033in}{1.369495in}}{\pgfqpoint{1.271209in}{1.369495in}}%
\pgfpathcurveto{\pgfqpoint{1.265385in}{1.369495in}}{\pgfqpoint{1.259799in}{1.367181in}}{\pgfqpoint{1.255681in}{1.363063in}}%
\pgfpathcurveto{\pgfqpoint{1.251563in}{1.358945in}}{\pgfqpoint{1.249249in}{1.353359in}}{\pgfqpoint{1.249249in}{1.347535in}}%
\pgfpathcurveto{\pgfqpoint{1.249249in}{1.341711in}}{\pgfqpoint{1.251563in}{1.336125in}}{\pgfqpoint{1.255681in}{1.332007in}}%
\pgfpathcurveto{\pgfqpoint{1.259799in}{1.327889in}}{\pgfqpoint{1.265385in}{1.325575in}}{\pgfqpoint{1.271209in}{1.325575in}}%
\pgfpathclose%
\pgfusepath{stroke,fill}%
\end{pgfscope}%
\begin{pgfscope}%
\pgfpathrectangle{\pgfqpoint{0.211875in}{0.211875in}}{\pgfqpoint{1.313625in}{1.279725in}}%
\pgfusepath{clip}%
\pgfsetbuttcap%
\pgfsetroundjoin%
\definecolor{currentfill}{rgb}{0.121569,0.466667,0.705882}%
\pgfsetfillcolor{currentfill}%
\pgfsetlinewidth{1.003750pt}%
\definecolor{currentstroke}{rgb}{0.121569,0.466667,0.705882}%
\pgfsetstrokecolor{currentstroke}%
\pgfsetdash{}{0pt}%
\pgfpathmoveto{\pgfqpoint{0.451226in}{0.913875in}}%
\pgfpathcurveto{\pgfqpoint{0.457050in}{0.913875in}}{\pgfqpoint{0.462637in}{0.916189in}}{\pgfqpoint{0.466755in}{0.920307in}}%
\pgfpathcurveto{\pgfqpoint{0.470873in}{0.924425in}}{\pgfqpoint{0.473187in}{0.930011in}}{\pgfqpoint{0.473187in}{0.935835in}}%
\pgfpathcurveto{\pgfqpoint{0.473187in}{0.941659in}}{\pgfqpoint{0.470873in}{0.947245in}}{\pgfqpoint{0.466755in}{0.951364in}}%
\pgfpathcurveto{\pgfqpoint{0.462637in}{0.955482in}}{\pgfqpoint{0.457050in}{0.957796in}}{\pgfqpoint{0.451226in}{0.957796in}}%
\pgfpathcurveto{\pgfqpoint{0.445403in}{0.957796in}}{\pgfqpoint{0.439816in}{0.955482in}}{\pgfqpoint{0.435698in}{0.951364in}}%
\pgfpathcurveto{\pgfqpoint{0.431580in}{0.947245in}}{\pgfqpoint{0.429266in}{0.941659in}}{\pgfqpoint{0.429266in}{0.935835in}}%
\pgfpathcurveto{\pgfqpoint{0.429266in}{0.930011in}}{\pgfqpoint{0.431580in}{0.924425in}}{\pgfqpoint{0.435698in}{0.920307in}}%
\pgfpathcurveto{\pgfqpoint{0.439816in}{0.916189in}}{\pgfqpoint{0.445403in}{0.913875in}}{\pgfqpoint{0.451226in}{0.913875in}}%
\pgfpathclose%
\pgfusepath{stroke,fill}%
\end{pgfscope}%
\begin{pgfscope}%
\pgfpathrectangle{\pgfqpoint{0.211875in}{0.211875in}}{\pgfqpoint{1.313625in}{1.279725in}}%
\pgfusepath{clip}%
\pgfsetbuttcap%
\pgfsetroundjoin%
\definecolor{currentfill}{rgb}{0.121569,0.466667,0.705882}%
\pgfsetfillcolor{currentfill}%
\pgfsetlinewidth{1.003750pt}%
\definecolor{currentstroke}{rgb}{0.121569,0.466667,0.705882}%
\pgfsetstrokecolor{currentstroke}%
\pgfsetdash{}{0pt}%
\pgfpathmoveto{\pgfqpoint{1.434627in}{1.134387in}}%
\pgfpathcurveto{\pgfqpoint{1.440451in}{1.134387in}}{\pgfqpoint{1.446037in}{1.136701in}}{\pgfqpoint{1.450155in}{1.140819in}}%
\pgfpathcurveto{\pgfqpoint{1.454273in}{1.144937in}}{\pgfqpoint{1.456587in}{1.150523in}}{\pgfqpoint{1.456587in}{1.156347in}}%
\pgfpathcurveto{\pgfqpoint{1.456587in}{1.162171in}}{\pgfqpoint{1.454273in}{1.167757in}}{\pgfqpoint{1.450155in}{1.171875in}}%
\pgfpathcurveto{\pgfqpoint{1.446037in}{1.175994in}}{\pgfqpoint{1.440451in}{1.178307in}}{\pgfqpoint{1.434627in}{1.178307in}}%
\pgfpathcurveto{\pgfqpoint{1.428803in}{1.178307in}}{\pgfqpoint{1.423217in}{1.175994in}}{\pgfqpoint{1.419098in}{1.171875in}}%
\pgfpathcurveto{\pgfqpoint{1.414980in}{1.167757in}}{\pgfqpoint{1.412666in}{1.162171in}}{\pgfqpoint{1.412666in}{1.156347in}}%
\pgfpathcurveto{\pgfqpoint{1.412666in}{1.150523in}}{\pgfqpoint{1.414980in}{1.144937in}}{\pgfqpoint{1.419098in}{1.140819in}}%
\pgfpathcurveto{\pgfqpoint{1.423217in}{1.136701in}}{\pgfqpoint{1.428803in}{1.134387in}}{\pgfqpoint{1.434627in}{1.134387in}}%
\pgfpathclose%
\pgfusepath{stroke,fill}%
\end{pgfscope}%
\begin{pgfscope}%
\pgfpathrectangle{\pgfqpoint{0.211875in}{0.211875in}}{\pgfqpoint{1.313625in}{1.279725in}}%
\pgfusepath{clip}%
\pgfsetbuttcap%
\pgfsetroundjoin%
\definecolor{currentfill}{rgb}{0.121569,0.466667,0.705882}%
\pgfsetfillcolor{currentfill}%
\pgfsetlinewidth{1.003750pt}%
\definecolor{currentstroke}{rgb}{0.121569,0.466667,0.705882}%
\pgfsetstrokecolor{currentstroke}%
\pgfsetdash{}{0pt}%
\pgfpathmoveto{\pgfqpoint{1.266304in}{1.369719in}}%
\pgfpathcurveto{\pgfqpoint{1.272128in}{1.369719in}}{\pgfqpoint{1.277714in}{1.372033in}}{\pgfqpoint{1.281832in}{1.376151in}}%
\pgfpathcurveto{\pgfqpoint{1.285950in}{1.380270in}}{\pgfqpoint{1.288264in}{1.385856in}}{\pgfqpoint{1.288264in}{1.391680in}}%
\pgfpathcurveto{\pgfqpoint{1.288264in}{1.397504in}}{\pgfqpoint{1.285950in}{1.403090in}}{\pgfqpoint{1.281832in}{1.407208in}}%
\pgfpathcurveto{\pgfqpoint{1.277714in}{1.411326in}}{\pgfqpoint{1.272128in}{1.413640in}}{\pgfqpoint{1.266304in}{1.413640in}}%
\pgfpathcurveto{\pgfqpoint{1.260480in}{1.413640in}}{\pgfqpoint{1.254894in}{1.411326in}}{\pgfqpoint{1.250776in}{1.407208in}}%
\pgfpathcurveto{\pgfqpoint{1.246658in}{1.403090in}}{\pgfqpoint{1.244344in}{1.397504in}}{\pgfqpoint{1.244344in}{1.391680in}}%
\pgfpathcurveto{\pgfqpoint{1.244344in}{1.385856in}}{\pgfqpoint{1.246658in}{1.380270in}}{\pgfqpoint{1.250776in}{1.376151in}}%
\pgfpathcurveto{\pgfqpoint{1.254894in}{1.372033in}}{\pgfqpoint{1.260480in}{1.369719in}}{\pgfqpoint{1.266304in}{1.369719in}}%
\pgfpathclose%
\pgfusepath{stroke,fill}%
\end{pgfscope}%
\begin{pgfscope}%
\pgfpathrectangle{\pgfqpoint{0.211875in}{0.211875in}}{\pgfqpoint{1.313625in}{1.279725in}}%
\pgfusepath{clip}%
\pgfsetbuttcap%
\pgfsetroundjoin%
\definecolor{currentfill}{rgb}{0.121569,0.466667,0.705882}%
\pgfsetfillcolor{currentfill}%
\pgfsetlinewidth{1.003750pt}%
\definecolor{currentstroke}{rgb}{0.121569,0.466667,0.705882}%
\pgfsetstrokecolor{currentstroke}%
\pgfsetdash{}{0pt}%
\pgfpathmoveto{\pgfqpoint{1.427583in}{1.127777in}}%
\pgfpathcurveto{\pgfqpoint{1.433407in}{1.127777in}}{\pgfqpoint{1.438993in}{1.130091in}}{\pgfqpoint{1.443111in}{1.134209in}}%
\pgfpathcurveto{\pgfqpoint{1.447229in}{1.138327in}}{\pgfqpoint{1.449543in}{1.143914in}}{\pgfqpoint{1.449543in}{1.149737in}}%
\pgfpathcurveto{\pgfqpoint{1.449543in}{1.155561in}}{\pgfqpoint{1.447229in}{1.161148in}}{\pgfqpoint{1.443111in}{1.165266in}}%
\pgfpathcurveto{\pgfqpoint{1.438993in}{1.169384in}}{\pgfqpoint{1.433407in}{1.171698in}}{\pgfqpoint{1.427583in}{1.171698in}}%
\pgfpathcurveto{\pgfqpoint{1.421759in}{1.171698in}}{\pgfqpoint{1.416173in}{1.169384in}}{\pgfqpoint{1.412055in}{1.165266in}}%
\pgfpathcurveto{\pgfqpoint{1.407936in}{1.161148in}}{\pgfqpoint{1.405623in}{1.155561in}}{\pgfqpoint{1.405623in}{1.149737in}}%
\pgfpathcurveto{\pgfqpoint{1.405623in}{1.143914in}}{\pgfqpoint{1.407936in}{1.138327in}}{\pgfqpoint{1.412055in}{1.134209in}}%
\pgfpathcurveto{\pgfqpoint{1.416173in}{1.130091in}}{\pgfqpoint{1.421759in}{1.127777in}}{\pgfqpoint{1.427583in}{1.127777in}}%
\pgfpathclose%
\pgfusepath{stroke,fill}%
\end{pgfscope}%
\begin{pgfscope}%
\pgfpathrectangle{\pgfqpoint{0.211875in}{0.211875in}}{\pgfqpoint{1.313625in}{1.279725in}}%
\pgfusepath{clip}%
\pgfsetbuttcap%
\pgfsetroundjoin%
\definecolor{currentfill}{rgb}{0.121569,0.466667,0.705882}%
\pgfsetfillcolor{currentfill}%
\pgfsetlinewidth{1.003750pt}%
\definecolor{currentstroke}{rgb}{0.121569,0.466667,0.705882}%
\pgfsetstrokecolor{currentstroke}%
\pgfsetdash{}{0pt}%
\pgfpathmoveto{\pgfqpoint{1.418004in}{1.145957in}}%
\pgfpathcurveto{\pgfqpoint{1.423827in}{1.145957in}}{\pgfqpoint{1.429414in}{1.148271in}}{\pgfqpoint{1.433532in}{1.152389in}}%
\pgfpathcurveto{\pgfqpoint{1.437650in}{1.156507in}}{\pgfqpoint{1.439964in}{1.162093in}}{\pgfqpoint{1.439964in}{1.167917in}}%
\pgfpathcurveto{\pgfqpoint{1.439964in}{1.173741in}}{\pgfqpoint{1.437650in}{1.179327in}}{\pgfqpoint{1.433532in}{1.183446in}}%
\pgfpathcurveto{\pgfqpoint{1.429414in}{1.187564in}}{\pgfqpoint{1.423827in}{1.189878in}}{\pgfqpoint{1.418004in}{1.189878in}}%
\pgfpathcurveto{\pgfqpoint{1.412180in}{1.189878in}}{\pgfqpoint{1.406593in}{1.187564in}}{\pgfqpoint{1.402475in}{1.183446in}}%
\pgfpathcurveto{\pgfqpoint{1.398357in}{1.179327in}}{\pgfqpoint{1.396043in}{1.173741in}}{\pgfqpoint{1.396043in}{1.167917in}}%
\pgfpathcurveto{\pgfqpoint{1.396043in}{1.162093in}}{\pgfqpoint{1.398357in}{1.156507in}}{\pgfqpoint{1.402475in}{1.152389in}}%
\pgfpathcurveto{\pgfqpoint{1.406593in}{1.148271in}}{\pgfqpoint{1.412180in}{1.145957in}}{\pgfqpoint{1.418004in}{1.145957in}}%
\pgfpathclose%
\pgfusepath{stroke,fill}%
\end{pgfscope}%
\begin{pgfscope}%
\pgfpathrectangle{\pgfqpoint{0.211875in}{0.211875in}}{\pgfqpoint{1.313625in}{1.279725in}}%
\pgfusepath{clip}%
\pgfsetbuttcap%
\pgfsetroundjoin%
\definecolor{currentfill}{rgb}{0.121569,0.466667,0.705882}%
\pgfsetfillcolor{currentfill}%
\pgfsetlinewidth{1.003750pt}%
\definecolor{currentstroke}{rgb}{0.121569,0.466667,0.705882}%
\pgfsetstrokecolor{currentstroke}%
\pgfsetdash{}{0pt}%
\pgfpathmoveto{\pgfqpoint{1.108005in}{0.813831in}}%
\pgfpathcurveto{\pgfqpoint{1.113829in}{0.813831in}}{\pgfqpoint{1.119415in}{0.816144in}}{\pgfqpoint{1.123533in}{0.820263in}}%
\pgfpathcurveto{\pgfqpoint{1.127652in}{0.824381in}}{\pgfqpoint{1.129965in}{0.829967in}}{\pgfqpoint{1.129965in}{0.835791in}}%
\pgfpathcurveto{\pgfqpoint{1.129965in}{0.841615in}}{\pgfqpoint{1.127652in}{0.847201in}}{\pgfqpoint{1.123533in}{0.851319in}}%
\pgfpathcurveto{\pgfqpoint{1.119415in}{0.855437in}}{\pgfqpoint{1.113829in}{0.857751in}}{\pgfqpoint{1.108005in}{0.857751in}}%
\pgfpathcurveto{\pgfqpoint{1.102181in}{0.857751in}}{\pgfqpoint{1.096595in}{0.855437in}}{\pgfqpoint{1.092477in}{0.851319in}}%
\pgfpathcurveto{\pgfqpoint{1.088359in}{0.847201in}}{\pgfqpoint{1.086045in}{0.841615in}}{\pgfqpoint{1.086045in}{0.835791in}}%
\pgfpathcurveto{\pgfqpoint{1.086045in}{0.829967in}}{\pgfqpoint{1.088359in}{0.824381in}}{\pgfqpoint{1.092477in}{0.820263in}}%
\pgfpathcurveto{\pgfqpoint{1.096595in}{0.816144in}}{\pgfqpoint{1.102181in}{0.813831in}}{\pgfqpoint{1.108005in}{0.813831in}}%
\pgfpathclose%
\pgfusepath{stroke,fill}%
\end{pgfscope}%
\begin{pgfscope}%
\pgfpathrectangle{\pgfqpoint{0.211875in}{0.211875in}}{\pgfqpoint{1.313625in}{1.279725in}}%
\pgfusepath{clip}%
\pgfsetbuttcap%
\pgfsetroundjoin%
\definecolor{currentfill}{rgb}{0.121569,0.466667,0.705882}%
\pgfsetfillcolor{currentfill}%
\pgfsetlinewidth{1.003750pt}%
\definecolor{currentstroke}{rgb}{0.121569,0.466667,0.705882}%
\pgfsetstrokecolor{currentstroke}%
\pgfsetdash{}{0pt}%
\pgfpathmoveto{\pgfqpoint{1.272293in}{1.355379in}}%
\pgfpathcurveto{\pgfqpoint{1.278117in}{1.355379in}}{\pgfqpoint{1.283703in}{1.357692in}}{\pgfqpoint{1.287821in}{1.361811in}}%
\pgfpathcurveto{\pgfqpoint{1.291940in}{1.365929in}}{\pgfqpoint{1.294253in}{1.371515in}}{\pgfqpoint{1.294253in}{1.377339in}}%
\pgfpathcurveto{\pgfqpoint{1.294253in}{1.383163in}}{\pgfqpoint{1.291940in}{1.388749in}}{\pgfqpoint{1.287821in}{1.392867in}}%
\pgfpathcurveto{\pgfqpoint{1.283703in}{1.396985in}}{\pgfqpoint{1.278117in}{1.399299in}}{\pgfqpoint{1.272293in}{1.399299in}}%
\pgfpathcurveto{\pgfqpoint{1.266469in}{1.399299in}}{\pgfqpoint{1.260883in}{1.396985in}}{\pgfqpoint{1.256765in}{1.392867in}}%
\pgfpathcurveto{\pgfqpoint{1.252647in}{1.388749in}}{\pgfqpoint{1.250333in}{1.383163in}}{\pgfqpoint{1.250333in}{1.377339in}}%
\pgfpathcurveto{\pgfqpoint{1.250333in}{1.371515in}}{\pgfqpoint{1.252647in}{1.365929in}}{\pgfqpoint{1.256765in}{1.361811in}}%
\pgfpathcurveto{\pgfqpoint{1.260883in}{1.357692in}}{\pgfqpoint{1.266469in}{1.355379in}}{\pgfqpoint{1.272293in}{1.355379in}}%
\pgfpathclose%
\pgfusepath{stroke,fill}%
\end{pgfscope}%
\begin{pgfscope}%
\pgfpathrectangle{\pgfqpoint{0.211875in}{0.211875in}}{\pgfqpoint{1.313625in}{1.279725in}}%
\pgfusepath{clip}%
\pgfsetbuttcap%
\pgfsetroundjoin%
\definecolor{currentfill}{rgb}{0.121569,0.466667,0.705882}%
\pgfsetfillcolor{currentfill}%
\pgfsetlinewidth{1.003750pt}%
\definecolor{currentstroke}{rgb}{0.121569,0.466667,0.705882}%
\pgfsetstrokecolor{currentstroke}%
\pgfsetdash{}{0pt}%
\pgfpathmoveto{\pgfqpoint{1.125287in}{0.825616in}}%
\pgfpathcurveto{\pgfqpoint{1.131111in}{0.825616in}}{\pgfqpoint{1.136698in}{0.827930in}}{\pgfqpoint{1.140816in}{0.832048in}}%
\pgfpathcurveto{\pgfqpoint{1.144934in}{0.836166in}}{\pgfqpoint{1.147248in}{0.841753in}}{\pgfqpoint{1.147248in}{0.847576in}}%
\pgfpathcurveto{\pgfqpoint{1.147248in}{0.853400in}}{\pgfqpoint{1.144934in}{0.858987in}}{\pgfqpoint{1.140816in}{0.863105in}}%
\pgfpathcurveto{\pgfqpoint{1.136698in}{0.867223in}}{\pgfqpoint{1.131111in}{0.869537in}}{\pgfqpoint{1.125287in}{0.869537in}}%
\pgfpathcurveto{\pgfqpoint{1.119463in}{0.869537in}}{\pgfqpoint{1.113877in}{0.867223in}}{\pgfqpoint{1.109759in}{0.863105in}}%
\pgfpathcurveto{\pgfqpoint{1.105641in}{0.858987in}}{\pgfqpoint{1.103327in}{0.853400in}}{\pgfqpoint{1.103327in}{0.847576in}}%
\pgfpathcurveto{\pgfqpoint{1.103327in}{0.841753in}}{\pgfqpoint{1.105641in}{0.836166in}}{\pgfqpoint{1.109759in}{0.832048in}}%
\pgfpathcurveto{\pgfqpoint{1.113877in}{0.827930in}}{\pgfqpoint{1.119463in}{0.825616in}}{\pgfqpoint{1.125287in}{0.825616in}}%
\pgfpathclose%
\pgfusepath{stroke,fill}%
\end{pgfscope}%
\begin{pgfscope}%
\pgfpathrectangle{\pgfqpoint{0.211875in}{0.211875in}}{\pgfqpoint{1.313625in}{1.279725in}}%
\pgfusepath{clip}%
\pgfsetbuttcap%
\pgfsetroundjoin%
\definecolor{currentfill}{rgb}{0.121569,0.466667,0.705882}%
\pgfsetfillcolor{currentfill}%
\pgfsetlinewidth{1.003750pt}%
\definecolor{currentstroke}{rgb}{0.121569,0.466667,0.705882}%
\pgfsetstrokecolor{currentstroke}%
\pgfsetdash{}{0pt}%
\pgfpathmoveto{\pgfqpoint{1.048116in}{0.831566in}}%
\pgfpathcurveto{\pgfqpoint{1.053940in}{0.831566in}}{\pgfqpoint{1.059526in}{0.833880in}}{\pgfqpoint{1.063644in}{0.837998in}}%
\pgfpathcurveto{\pgfqpoint{1.067763in}{0.842116in}}{\pgfqpoint{1.070077in}{0.847703in}}{\pgfqpoint{1.070077in}{0.853527in}}%
\pgfpathcurveto{\pgfqpoint{1.070077in}{0.859351in}}{\pgfqpoint{1.067763in}{0.864937in}}{\pgfqpoint{1.063644in}{0.869055in}}%
\pgfpathcurveto{\pgfqpoint{1.059526in}{0.873173in}}{\pgfqpoint{1.053940in}{0.875487in}}{\pgfqpoint{1.048116in}{0.875487in}}%
\pgfpathcurveto{\pgfqpoint{1.042292in}{0.875487in}}{\pgfqpoint{1.036706in}{0.873173in}}{\pgfqpoint{1.032588in}{0.869055in}}%
\pgfpathcurveto{\pgfqpoint{1.028470in}{0.864937in}}{\pgfqpoint{1.026156in}{0.859351in}}{\pgfqpoint{1.026156in}{0.853527in}}%
\pgfpathcurveto{\pgfqpoint{1.026156in}{0.847703in}}{\pgfqpoint{1.028470in}{0.842116in}}{\pgfqpoint{1.032588in}{0.837998in}}%
\pgfpathcurveto{\pgfqpoint{1.036706in}{0.833880in}}{\pgfqpoint{1.042292in}{0.831566in}}{\pgfqpoint{1.048116in}{0.831566in}}%
\pgfpathclose%
\pgfusepath{stroke,fill}%
\end{pgfscope}%
\begin{pgfscope}%
\pgfpathrectangle{\pgfqpoint{0.211875in}{0.211875in}}{\pgfqpoint{1.313625in}{1.279725in}}%
\pgfusepath{clip}%
\pgfsetbuttcap%
\pgfsetroundjoin%
\definecolor{currentfill}{rgb}{0.121569,0.466667,0.705882}%
\pgfsetfillcolor{currentfill}%
\pgfsetlinewidth{1.003750pt}%
\definecolor{currentstroke}{rgb}{0.121569,0.466667,0.705882}%
\pgfsetstrokecolor{currentstroke}%
\pgfsetdash{}{0pt}%
\pgfpathmoveto{\pgfqpoint{1.159657in}{0.788310in}}%
\pgfpathcurveto{\pgfqpoint{1.165481in}{0.788310in}}{\pgfqpoint{1.171067in}{0.790624in}}{\pgfqpoint{1.175185in}{0.794742in}}%
\pgfpathcurveto{\pgfqpoint{1.179303in}{0.798860in}}{\pgfqpoint{1.181617in}{0.804446in}}{\pgfqpoint{1.181617in}{0.810270in}}%
\pgfpathcurveto{\pgfqpoint{1.181617in}{0.816094in}}{\pgfqpoint{1.179303in}{0.821680in}}{\pgfqpoint{1.175185in}{0.825798in}}%
\pgfpathcurveto{\pgfqpoint{1.171067in}{0.829916in}}{\pgfqpoint{1.165481in}{0.832230in}}{\pgfqpoint{1.159657in}{0.832230in}}%
\pgfpathcurveto{\pgfqpoint{1.153833in}{0.832230in}}{\pgfqpoint{1.148247in}{0.829916in}}{\pgfqpoint{1.144129in}{0.825798in}}%
\pgfpathcurveto{\pgfqpoint{1.140011in}{0.821680in}}{\pgfqpoint{1.137697in}{0.816094in}}{\pgfqpoint{1.137697in}{0.810270in}}%
\pgfpathcurveto{\pgfqpoint{1.137697in}{0.804446in}}{\pgfqpoint{1.140011in}{0.798860in}}{\pgfqpoint{1.144129in}{0.794742in}}%
\pgfpathcurveto{\pgfqpoint{1.148247in}{0.790624in}}{\pgfqpoint{1.153833in}{0.788310in}}{\pgfqpoint{1.159657in}{0.788310in}}%
\pgfpathclose%
\pgfusepath{stroke,fill}%
\end{pgfscope}%
\begin{pgfscope}%
\pgfpathrectangle{\pgfqpoint{0.211875in}{0.211875in}}{\pgfqpoint{1.313625in}{1.279725in}}%
\pgfusepath{clip}%
\pgfsetbuttcap%
\pgfsetroundjoin%
\definecolor{currentfill}{rgb}{0.121569,0.466667,0.705882}%
\pgfsetfillcolor{currentfill}%
\pgfsetlinewidth{1.003750pt}%
\definecolor{currentstroke}{rgb}{0.121569,0.466667,0.705882}%
\pgfsetstrokecolor{currentstroke}%
\pgfsetdash{}{0pt}%
\pgfpathmoveto{\pgfqpoint{1.287040in}{1.379555in}}%
\pgfpathcurveto{\pgfqpoint{1.292864in}{1.379555in}}{\pgfqpoint{1.298450in}{1.381868in}}{\pgfqpoint{1.302569in}{1.385987in}}%
\pgfpathcurveto{\pgfqpoint{1.306687in}{1.390105in}}{\pgfqpoint{1.309001in}{1.395691in}}{\pgfqpoint{1.309001in}{1.401515in}}%
\pgfpathcurveto{\pgfqpoint{1.309001in}{1.407339in}}{\pgfqpoint{1.306687in}{1.412925in}}{\pgfqpoint{1.302569in}{1.417043in}}%
\pgfpathcurveto{\pgfqpoint{1.298450in}{1.421161in}}{\pgfqpoint{1.292864in}{1.423475in}}{\pgfqpoint{1.287040in}{1.423475in}}%
\pgfpathcurveto{\pgfqpoint{1.281216in}{1.423475in}}{\pgfqpoint{1.275630in}{1.421161in}}{\pgfqpoint{1.271512in}{1.417043in}}%
\pgfpathcurveto{\pgfqpoint{1.267394in}{1.412925in}}{\pgfqpoint{1.265080in}{1.407339in}}{\pgfqpoint{1.265080in}{1.401515in}}%
\pgfpathcurveto{\pgfqpoint{1.265080in}{1.395691in}}{\pgfqpoint{1.267394in}{1.390105in}}{\pgfqpoint{1.271512in}{1.385987in}}%
\pgfpathcurveto{\pgfqpoint{1.275630in}{1.381868in}}{\pgfqpoint{1.281216in}{1.379555in}}{\pgfqpoint{1.287040in}{1.379555in}}%
\pgfpathclose%
\pgfusepath{stroke,fill}%
\end{pgfscope}%
\begin{pgfscope}%
\pgfpathrectangle{\pgfqpoint{0.211875in}{0.211875in}}{\pgfqpoint{1.313625in}{1.279725in}}%
\pgfusepath{clip}%
\pgfsetbuttcap%
\pgfsetroundjoin%
\definecolor{currentfill}{rgb}{0.121569,0.466667,0.705882}%
\pgfsetfillcolor{currentfill}%
\pgfsetlinewidth{1.003750pt}%
\definecolor{currentstroke}{rgb}{0.121569,0.466667,0.705882}%
\pgfsetstrokecolor{currentstroke}%
\pgfsetdash{}{0pt}%
\pgfpathmoveto{\pgfqpoint{0.494523in}{0.827187in}}%
\pgfpathcurveto{\pgfqpoint{0.500347in}{0.827187in}}{\pgfqpoint{0.505933in}{0.829501in}}{\pgfqpoint{0.510051in}{0.833619in}}%
\pgfpathcurveto{\pgfqpoint{0.514169in}{0.837737in}}{\pgfqpoint{0.516483in}{0.843323in}}{\pgfqpoint{0.516483in}{0.849147in}}%
\pgfpathcurveto{\pgfqpoint{0.516483in}{0.854971in}}{\pgfqpoint{0.514169in}{0.860558in}}{\pgfqpoint{0.510051in}{0.864676in}}%
\pgfpathcurveto{\pgfqpoint{0.505933in}{0.868794in}}{\pgfqpoint{0.500347in}{0.871108in}}{\pgfqpoint{0.494523in}{0.871108in}}%
\pgfpathcurveto{\pgfqpoint{0.488699in}{0.871108in}}{\pgfqpoint{0.483113in}{0.868794in}}{\pgfqpoint{0.478995in}{0.864676in}}%
\pgfpathcurveto{\pgfqpoint{0.474877in}{0.860558in}}{\pgfqpoint{0.472563in}{0.854971in}}{\pgfqpoint{0.472563in}{0.849147in}}%
\pgfpathcurveto{\pgfqpoint{0.472563in}{0.843323in}}{\pgfqpoint{0.474877in}{0.837737in}}{\pgfqpoint{0.478995in}{0.833619in}}%
\pgfpathcurveto{\pgfqpoint{0.483113in}{0.829501in}}{\pgfqpoint{0.488699in}{0.827187in}}{\pgfqpoint{0.494523in}{0.827187in}}%
\pgfpathclose%
\pgfusepath{stroke,fill}%
\end{pgfscope}%
\begin{pgfscope}%
\pgfpathrectangle{\pgfqpoint{0.211875in}{0.211875in}}{\pgfqpoint{1.313625in}{1.279725in}}%
\pgfusepath{clip}%
\pgfsetbuttcap%
\pgfsetroundjoin%
\definecolor{currentfill}{rgb}{0.121569,0.466667,0.705882}%
\pgfsetfillcolor{currentfill}%
\pgfsetlinewidth{1.003750pt}%
\definecolor{currentstroke}{rgb}{0.121569,0.466667,0.705882}%
\pgfsetstrokecolor{currentstroke}%
\pgfsetdash{}{0pt}%
\pgfpathmoveto{\pgfqpoint{1.266873in}{1.343715in}}%
\pgfpathcurveto{\pgfqpoint{1.272697in}{1.343715in}}{\pgfqpoint{1.278284in}{1.346029in}}{\pgfqpoint{1.282402in}{1.350147in}}%
\pgfpathcurveto{\pgfqpoint{1.286520in}{1.354265in}}{\pgfqpoint{1.288834in}{1.359851in}}{\pgfqpoint{1.288834in}{1.365675in}}%
\pgfpathcurveto{\pgfqpoint{1.288834in}{1.371499in}}{\pgfqpoint{1.286520in}{1.377085in}}{\pgfqpoint{1.282402in}{1.381204in}}%
\pgfpathcurveto{\pgfqpoint{1.278284in}{1.385322in}}{\pgfqpoint{1.272697in}{1.387636in}}{\pgfqpoint{1.266873in}{1.387636in}}%
\pgfpathcurveto{\pgfqpoint{1.261050in}{1.387636in}}{\pgfqpoint{1.255463in}{1.385322in}}{\pgfqpoint{1.251345in}{1.381204in}}%
\pgfpathcurveto{\pgfqpoint{1.247227in}{1.377085in}}{\pgfqpoint{1.244913in}{1.371499in}}{\pgfqpoint{1.244913in}{1.365675in}}%
\pgfpathcurveto{\pgfqpoint{1.244913in}{1.359851in}}{\pgfqpoint{1.247227in}{1.354265in}}{\pgfqpoint{1.251345in}{1.350147in}}%
\pgfpathcurveto{\pgfqpoint{1.255463in}{1.346029in}}{\pgfqpoint{1.261050in}{1.343715in}}{\pgfqpoint{1.266873in}{1.343715in}}%
\pgfpathclose%
\pgfusepath{stroke,fill}%
\end{pgfscope}%
\begin{pgfscope}%
\pgfpathrectangle{\pgfqpoint{0.211875in}{0.211875in}}{\pgfqpoint{1.313625in}{1.279725in}}%
\pgfusepath{clip}%
\pgfsetbuttcap%
\pgfsetroundjoin%
\definecolor{currentfill}{rgb}{0.121569,0.466667,0.705882}%
\pgfsetfillcolor{currentfill}%
\pgfsetlinewidth{1.003750pt}%
\definecolor{currentstroke}{rgb}{0.121569,0.466667,0.705882}%
\pgfsetstrokecolor{currentstroke}%
\pgfsetdash{}{0pt}%
\pgfpathmoveto{\pgfqpoint{1.132125in}{0.829839in}}%
\pgfpathcurveto{\pgfqpoint{1.137949in}{0.829839in}}{\pgfqpoint{1.143535in}{0.832152in}}{\pgfqpoint{1.147653in}{0.836271in}}%
\pgfpathcurveto{\pgfqpoint{1.151771in}{0.840389in}}{\pgfqpoint{1.154085in}{0.845975in}}{\pgfqpoint{1.154085in}{0.851799in}}%
\pgfpathcurveto{\pgfqpoint{1.154085in}{0.857623in}}{\pgfqpoint{1.151771in}{0.863209in}}{\pgfqpoint{1.147653in}{0.867327in}}%
\pgfpathcurveto{\pgfqpoint{1.143535in}{0.871445in}}{\pgfqpoint{1.137949in}{0.873759in}}{\pgfqpoint{1.132125in}{0.873759in}}%
\pgfpathcurveto{\pgfqpoint{1.126301in}{0.873759in}}{\pgfqpoint{1.120715in}{0.871445in}}{\pgfqpoint{1.116597in}{0.867327in}}%
\pgfpathcurveto{\pgfqpoint{1.112478in}{0.863209in}}{\pgfqpoint{1.110165in}{0.857623in}}{\pgfqpoint{1.110165in}{0.851799in}}%
\pgfpathcurveto{\pgfqpoint{1.110165in}{0.845975in}}{\pgfqpoint{1.112478in}{0.840389in}}{\pgfqpoint{1.116597in}{0.836271in}}%
\pgfpathcurveto{\pgfqpoint{1.120715in}{0.832152in}}{\pgfqpoint{1.126301in}{0.829839in}}{\pgfqpoint{1.132125in}{0.829839in}}%
\pgfpathclose%
\pgfusepath{stroke,fill}%
\end{pgfscope}%
\begin{pgfscope}%
\pgfpathrectangle{\pgfqpoint{0.211875in}{0.211875in}}{\pgfqpoint{1.313625in}{1.279725in}}%
\pgfusepath{clip}%
\pgfsetbuttcap%
\pgfsetroundjoin%
\definecolor{currentfill}{rgb}{0.121569,0.466667,0.705882}%
\pgfsetfillcolor{currentfill}%
\pgfsetlinewidth{1.003750pt}%
\definecolor{currentstroke}{rgb}{0.121569,0.466667,0.705882}%
\pgfsetstrokecolor{currentstroke}%
\pgfsetdash{}{0pt}%
\pgfpathmoveto{\pgfqpoint{0.476421in}{0.814402in}}%
\pgfpathcurveto{\pgfqpoint{0.482245in}{0.814402in}}{\pgfqpoint{0.487831in}{0.816715in}}{\pgfqpoint{0.491949in}{0.820834in}}%
\pgfpathcurveto{\pgfqpoint{0.496067in}{0.824952in}}{\pgfqpoint{0.498381in}{0.830538in}}{\pgfqpoint{0.498381in}{0.836362in}}%
\pgfpathcurveto{\pgfqpoint{0.498381in}{0.842186in}}{\pgfqpoint{0.496067in}{0.847772in}}{\pgfqpoint{0.491949in}{0.851890in}}%
\pgfpathcurveto{\pgfqpoint{0.487831in}{0.856008in}}{\pgfqpoint{0.482245in}{0.858322in}}{\pgfqpoint{0.476421in}{0.858322in}}%
\pgfpathcurveto{\pgfqpoint{0.470597in}{0.858322in}}{\pgfqpoint{0.465011in}{0.856008in}}{\pgfqpoint{0.460893in}{0.851890in}}%
\pgfpathcurveto{\pgfqpoint{0.456774in}{0.847772in}}{\pgfqpoint{0.454461in}{0.842186in}}{\pgfqpoint{0.454461in}{0.836362in}}%
\pgfpathcurveto{\pgfqpoint{0.454461in}{0.830538in}}{\pgfqpoint{0.456774in}{0.824952in}}{\pgfqpoint{0.460893in}{0.820834in}}%
\pgfpathcurveto{\pgfqpoint{0.465011in}{0.816715in}}{\pgfqpoint{0.470597in}{0.814402in}}{\pgfqpoint{0.476421in}{0.814402in}}%
\pgfpathclose%
\pgfusepath{stroke,fill}%
\end{pgfscope}%
\begin{pgfscope}%
\pgfpathrectangle{\pgfqpoint{0.211875in}{0.211875in}}{\pgfqpoint{1.313625in}{1.279725in}}%
\pgfusepath{clip}%
\pgfsetbuttcap%
\pgfsetroundjoin%
\definecolor{currentfill}{rgb}{0.121569,0.466667,0.705882}%
\pgfsetfillcolor{currentfill}%
\pgfsetlinewidth{1.003750pt}%
\definecolor{currentstroke}{rgb}{0.121569,0.466667,0.705882}%
\pgfsetstrokecolor{currentstroke}%
\pgfsetdash{}{0pt}%
\pgfpathmoveto{\pgfqpoint{0.479500in}{0.828235in}}%
\pgfpathcurveto{\pgfqpoint{0.485324in}{0.828235in}}{\pgfqpoint{0.490911in}{0.830549in}}{\pgfqpoint{0.495029in}{0.834667in}}%
\pgfpathcurveto{\pgfqpoint{0.499147in}{0.838785in}}{\pgfqpoint{0.501461in}{0.844371in}}{\pgfqpoint{0.501461in}{0.850195in}}%
\pgfpathcurveto{\pgfqpoint{0.501461in}{0.856019in}}{\pgfqpoint{0.499147in}{0.861605in}}{\pgfqpoint{0.495029in}{0.865724in}}%
\pgfpathcurveto{\pgfqpoint{0.490911in}{0.869842in}}{\pgfqpoint{0.485324in}{0.872156in}}{\pgfqpoint{0.479500in}{0.872156in}}%
\pgfpathcurveto{\pgfqpoint{0.473677in}{0.872156in}}{\pgfqpoint{0.468090in}{0.869842in}}{\pgfqpoint{0.463972in}{0.865724in}}%
\pgfpathcurveto{\pgfqpoint{0.459854in}{0.861605in}}{\pgfqpoint{0.457540in}{0.856019in}}{\pgfqpoint{0.457540in}{0.850195in}}%
\pgfpathcurveto{\pgfqpoint{0.457540in}{0.844371in}}{\pgfqpoint{0.459854in}{0.838785in}}{\pgfqpoint{0.463972in}{0.834667in}}%
\pgfpathcurveto{\pgfqpoint{0.468090in}{0.830549in}}{\pgfqpoint{0.473677in}{0.828235in}}{\pgfqpoint{0.479500in}{0.828235in}}%
\pgfpathclose%
\pgfusepath{stroke,fill}%
\end{pgfscope}%
\begin{pgfscope}%
\pgfpathrectangle{\pgfqpoint{0.211875in}{0.211875in}}{\pgfqpoint{1.313625in}{1.279725in}}%
\pgfusepath{clip}%
\pgfsetbuttcap%
\pgfsetroundjoin%
\definecolor{currentfill}{rgb}{0.121569,0.466667,0.705882}%
\pgfsetfillcolor{currentfill}%
\pgfsetlinewidth{1.003750pt}%
\definecolor{currentstroke}{rgb}{0.121569,0.466667,0.705882}%
\pgfsetstrokecolor{currentstroke}%
\pgfsetdash{}{0pt}%
\pgfpathmoveto{\pgfqpoint{1.264550in}{1.359945in}}%
\pgfpathcurveto{\pgfqpoint{1.270374in}{1.359945in}}{\pgfqpoint{1.275960in}{1.362259in}}{\pgfqpoint{1.280078in}{1.366377in}}%
\pgfpathcurveto{\pgfqpoint{1.284196in}{1.370495in}}{\pgfqpoint{1.286510in}{1.376081in}}{\pgfqpoint{1.286510in}{1.381905in}}%
\pgfpathcurveto{\pgfqpoint{1.286510in}{1.387729in}}{\pgfqpoint{1.284196in}{1.393315in}}{\pgfqpoint{1.280078in}{1.397433in}}%
\pgfpathcurveto{\pgfqpoint{1.275960in}{1.401551in}}{\pgfqpoint{1.270374in}{1.403865in}}{\pgfqpoint{1.264550in}{1.403865in}}%
\pgfpathcurveto{\pgfqpoint{1.258726in}{1.403865in}}{\pgfqpoint{1.253140in}{1.401551in}}{\pgfqpoint{1.249022in}{1.397433in}}%
\pgfpathcurveto{\pgfqpoint{1.244904in}{1.393315in}}{\pgfqpoint{1.242590in}{1.387729in}}{\pgfqpoint{1.242590in}{1.381905in}}%
\pgfpathcurveto{\pgfqpoint{1.242590in}{1.376081in}}{\pgfqpoint{1.244904in}{1.370495in}}{\pgfqpoint{1.249022in}{1.366377in}}%
\pgfpathcurveto{\pgfqpoint{1.253140in}{1.362259in}}{\pgfqpoint{1.258726in}{1.359945in}}{\pgfqpoint{1.264550in}{1.359945in}}%
\pgfpathclose%
\pgfusepath{stroke,fill}%
\end{pgfscope}%
\begin{pgfscope}%
\pgfpathrectangle{\pgfqpoint{0.211875in}{0.211875in}}{\pgfqpoint{1.313625in}{1.279725in}}%
\pgfusepath{clip}%
\pgfsetbuttcap%
\pgfsetroundjoin%
\definecolor{currentfill}{rgb}{0.121569,0.466667,0.705882}%
\pgfsetfillcolor{currentfill}%
\pgfsetlinewidth{1.003750pt}%
\definecolor{currentstroke}{rgb}{0.121569,0.466667,0.705882}%
\pgfsetstrokecolor{currentstroke}%
\pgfsetdash{}{0pt}%
\pgfpathmoveto{\pgfqpoint{0.665161in}{0.753200in}}%
\pgfpathcurveto{\pgfqpoint{0.670985in}{0.753200in}}{\pgfqpoint{0.676571in}{0.755513in}}{\pgfqpoint{0.680689in}{0.759632in}}%
\pgfpathcurveto{\pgfqpoint{0.684807in}{0.763750in}}{\pgfqpoint{0.687121in}{0.769336in}}{\pgfqpoint{0.687121in}{0.775160in}}%
\pgfpathcurveto{\pgfqpoint{0.687121in}{0.780984in}}{\pgfqpoint{0.684807in}{0.786570in}}{\pgfqpoint{0.680689in}{0.790688in}}%
\pgfpathcurveto{\pgfqpoint{0.676571in}{0.794806in}}{\pgfqpoint{0.670985in}{0.797120in}}{\pgfqpoint{0.665161in}{0.797120in}}%
\pgfpathcurveto{\pgfqpoint{0.659337in}{0.797120in}}{\pgfqpoint{0.653751in}{0.794806in}}{\pgfqpoint{0.649633in}{0.790688in}}%
\pgfpathcurveto{\pgfqpoint{0.645515in}{0.786570in}}{\pgfqpoint{0.643201in}{0.780984in}}{\pgfqpoint{0.643201in}{0.775160in}}%
\pgfpathcurveto{\pgfqpoint{0.643201in}{0.769336in}}{\pgfqpoint{0.645515in}{0.763750in}}{\pgfqpoint{0.649633in}{0.759632in}}%
\pgfpathcurveto{\pgfqpoint{0.653751in}{0.755513in}}{\pgfqpoint{0.659337in}{0.753200in}}{\pgfqpoint{0.665161in}{0.753200in}}%
\pgfpathclose%
\pgfusepath{stroke,fill}%
\end{pgfscope}%
\begin{pgfscope}%
\pgfpathrectangle{\pgfqpoint{0.211875in}{0.211875in}}{\pgfqpoint{1.313625in}{1.279725in}}%
\pgfusepath{clip}%
\pgfsetbuttcap%
\pgfsetroundjoin%
\definecolor{currentfill}{rgb}{0.121569,0.466667,0.705882}%
\pgfsetfillcolor{currentfill}%
\pgfsetlinewidth{1.003750pt}%
\definecolor{currentstroke}{rgb}{0.121569,0.466667,0.705882}%
\pgfsetstrokecolor{currentstroke}%
\pgfsetdash{}{0pt}%
\pgfpathmoveto{\pgfqpoint{0.505173in}{0.823386in}}%
\pgfpathcurveto{\pgfqpoint{0.510997in}{0.823386in}}{\pgfqpoint{0.516583in}{0.825700in}}{\pgfqpoint{0.520701in}{0.829818in}}%
\pgfpathcurveto{\pgfqpoint{0.524819in}{0.833936in}}{\pgfqpoint{0.527133in}{0.839522in}}{\pgfqpoint{0.527133in}{0.845346in}}%
\pgfpathcurveto{\pgfqpoint{0.527133in}{0.851170in}}{\pgfqpoint{0.524819in}{0.856756in}}{\pgfqpoint{0.520701in}{0.860875in}}%
\pgfpathcurveto{\pgfqpoint{0.516583in}{0.864993in}}{\pgfqpoint{0.510997in}{0.867307in}}{\pgfqpoint{0.505173in}{0.867307in}}%
\pgfpathcurveto{\pgfqpoint{0.499349in}{0.867307in}}{\pgfqpoint{0.493763in}{0.864993in}}{\pgfqpoint{0.489644in}{0.860875in}}%
\pgfpathcurveto{\pgfqpoint{0.485526in}{0.856756in}}{\pgfqpoint{0.483212in}{0.851170in}}{\pgfqpoint{0.483212in}{0.845346in}}%
\pgfpathcurveto{\pgfqpoint{0.483212in}{0.839522in}}{\pgfqpoint{0.485526in}{0.833936in}}{\pgfqpoint{0.489644in}{0.829818in}}%
\pgfpathcurveto{\pgfqpoint{0.493763in}{0.825700in}}{\pgfqpoint{0.499349in}{0.823386in}}{\pgfqpoint{0.505173in}{0.823386in}}%
\pgfpathclose%
\pgfusepath{stroke,fill}%
\end{pgfscope}%
\begin{pgfscope}%
\pgfpathrectangle{\pgfqpoint{0.211875in}{0.211875in}}{\pgfqpoint{1.313625in}{1.279725in}}%
\pgfusepath{clip}%
\pgfsetbuttcap%
\pgfsetroundjoin%
\definecolor{currentfill}{rgb}{0.121569,0.466667,0.705882}%
\pgfsetfillcolor{currentfill}%
\pgfsetlinewidth{1.003750pt}%
\definecolor{currentstroke}{rgb}{0.121569,0.466667,0.705882}%
\pgfsetstrokecolor{currentstroke}%
\pgfsetdash{}{0pt}%
\pgfpathmoveto{\pgfqpoint{1.265560in}{1.362033in}}%
\pgfpathcurveto{\pgfqpoint{1.271384in}{1.362033in}}{\pgfqpoint{1.276970in}{1.364347in}}{\pgfqpoint{1.281089in}{1.368465in}}%
\pgfpathcurveto{\pgfqpoint{1.285207in}{1.372583in}}{\pgfqpoint{1.287521in}{1.378169in}}{\pgfqpoint{1.287521in}{1.383993in}}%
\pgfpathcurveto{\pgfqpoint{1.287521in}{1.389817in}}{\pgfqpoint{1.285207in}{1.395403in}}{\pgfqpoint{1.281089in}{1.399521in}}%
\pgfpathcurveto{\pgfqpoint{1.276970in}{1.403639in}}{\pgfqpoint{1.271384in}{1.405953in}}{\pgfqpoint{1.265560in}{1.405953in}}%
\pgfpathcurveto{\pgfqpoint{1.259736in}{1.405953in}}{\pgfqpoint{1.254150in}{1.403639in}}{\pgfqpoint{1.250032in}{1.399521in}}%
\pgfpathcurveto{\pgfqpoint{1.245914in}{1.395403in}}{\pgfqpoint{1.243600in}{1.389817in}}{\pgfqpoint{1.243600in}{1.383993in}}%
\pgfpathcurveto{\pgfqpoint{1.243600in}{1.378169in}}{\pgfqpoint{1.245914in}{1.372583in}}{\pgfqpoint{1.250032in}{1.368465in}}%
\pgfpathcurveto{\pgfqpoint{1.254150in}{1.364347in}}{\pgfqpoint{1.259736in}{1.362033in}}{\pgfqpoint{1.265560in}{1.362033in}}%
\pgfpathclose%
\pgfusepath{stroke,fill}%
\end{pgfscope}%
\begin{pgfscope}%
\pgfpathrectangle{\pgfqpoint{0.211875in}{0.211875in}}{\pgfqpoint{1.313625in}{1.279725in}}%
\pgfusepath{clip}%
\pgfsetbuttcap%
\pgfsetroundjoin%
\definecolor{currentfill}{rgb}{0.121569,0.466667,0.705882}%
\pgfsetfillcolor{currentfill}%
\pgfsetlinewidth{1.003750pt}%
\definecolor{currentstroke}{rgb}{0.121569,0.466667,0.705882}%
\pgfsetstrokecolor{currentstroke}%
\pgfsetdash{}{0pt}%
\pgfpathmoveto{\pgfqpoint{0.779296in}{1.279203in}}%
\pgfpathcurveto{\pgfqpoint{0.785120in}{1.279203in}}{\pgfqpoint{0.790706in}{1.281517in}}{\pgfqpoint{0.794824in}{1.285635in}}%
\pgfpathcurveto{\pgfqpoint{0.798942in}{1.289753in}}{\pgfqpoint{0.801256in}{1.295339in}}{\pgfqpoint{0.801256in}{1.301163in}}%
\pgfpathcurveto{\pgfqpoint{0.801256in}{1.306987in}}{\pgfqpoint{0.798942in}{1.312573in}}{\pgfqpoint{0.794824in}{1.316691in}}%
\pgfpathcurveto{\pgfqpoint{0.790706in}{1.320810in}}{\pgfqpoint{0.785120in}{1.323123in}}{\pgfqpoint{0.779296in}{1.323123in}}%
\pgfpathcurveto{\pgfqpoint{0.773472in}{1.323123in}}{\pgfqpoint{0.767886in}{1.320810in}}{\pgfqpoint{0.763768in}{1.316691in}}%
\pgfpathcurveto{\pgfqpoint{0.759649in}{1.312573in}}{\pgfqpoint{0.757336in}{1.306987in}}{\pgfqpoint{0.757336in}{1.301163in}}%
\pgfpathcurveto{\pgfqpoint{0.757336in}{1.295339in}}{\pgfqpoint{0.759649in}{1.289753in}}{\pgfqpoint{0.763768in}{1.285635in}}%
\pgfpathcurveto{\pgfqpoint{0.767886in}{1.281517in}}{\pgfqpoint{0.773472in}{1.279203in}}{\pgfqpoint{0.779296in}{1.279203in}}%
\pgfpathclose%
\pgfusepath{stroke,fill}%
\end{pgfscope}%
\begin{pgfscope}%
\pgfpathrectangle{\pgfqpoint{0.211875in}{0.211875in}}{\pgfqpoint{1.313625in}{1.279725in}}%
\pgfusepath{clip}%
\pgfsetbuttcap%
\pgfsetroundjoin%
\definecolor{currentfill}{rgb}{0.121569,0.466667,0.705882}%
\pgfsetfillcolor{currentfill}%
\pgfsetlinewidth{1.003750pt}%
\definecolor{currentstroke}{rgb}{0.121569,0.466667,0.705882}%
\pgfsetstrokecolor{currentstroke}%
\pgfsetdash{}{0pt}%
\pgfpathmoveto{\pgfqpoint{1.123516in}{0.815729in}}%
\pgfpathcurveto{\pgfqpoint{1.129340in}{0.815729in}}{\pgfqpoint{1.134927in}{0.818043in}}{\pgfqpoint{1.139045in}{0.822161in}}%
\pgfpathcurveto{\pgfqpoint{1.143163in}{0.826279in}}{\pgfqpoint{1.145477in}{0.831865in}}{\pgfqpoint{1.145477in}{0.837689in}}%
\pgfpathcurveto{\pgfqpoint{1.145477in}{0.843513in}}{\pgfqpoint{1.143163in}{0.849099in}}{\pgfqpoint{1.139045in}{0.853217in}}%
\pgfpathcurveto{\pgfqpoint{1.134927in}{0.857335in}}{\pgfqpoint{1.129340in}{0.859649in}}{\pgfqpoint{1.123516in}{0.859649in}}%
\pgfpathcurveto{\pgfqpoint{1.117692in}{0.859649in}}{\pgfqpoint{1.112106in}{0.857335in}}{\pgfqpoint{1.107988in}{0.853217in}}%
\pgfpathcurveto{\pgfqpoint{1.103870in}{0.849099in}}{\pgfqpoint{1.101556in}{0.843513in}}{\pgfqpoint{1.101556in}{0.837689in}}%
\pgfpathcurveto{\pgfqpoint{1.101556in}{0.831865in}}{\pgfqpoint{1.103870in}{0.826279in}}{\pgfqpoint{1.107988in}{0.822161in}}%
\pgfpathcurveto{\pgfqpoint{1.112106in}{0.818043in}}{\pgfqpoint{1.117692in}{0.815729in}}{\pgfqpoint{1.123516in}{0.815729in}}%
\pgfpathclose%
\pgfusepath{stroke,fill}%
\end{pgfscope}%
\begin{pgfscope}%
\pgfpathrectangle{\pgfqpoint{0.211875in}{0.211875in}}{\pgfqpoint{1.313625in}{1.279725in}}%
\pgfusepath{clip}%
\pgfsetbuttcap%
\pgfsetroundjoin%
\definecolor{currentfill}{rgb}{0.121569,0.466667,0.705882}%
\pgfsetfillcolor{currentfill}%
\pgfsetlinewidth{1.003750pt}%
\definecolor{currentstroke}{rgb}{0.121569,0.466667,0.705882}%
\pgfsetstrokecolor{currentstroke}%
\pgfsetdash{}{0pt}%
\pgfpathmoveto{\pgfqpoint{1.105886in}{0.831215in}}%
\pgfpathcurveto{\pgfqpoint{1.111709in}{0.831215in}}{\pgfqpoint{1.117296in}{0.833528in}}{\pgfqpoint{1.121414in}{0.837647in}}%
\pgfpathcurveto{\pgfqpoint{1.125532in}{0.841765in}}{\pgfqpoint{1.127846in}{0.847351in}}{\pgfqpoint{1.127846in}{0.853175in}}%
\pgfpathcurveto{\pgfqpoint{1.127846in}{0.858999in}}{\pgfqpoint{1.125532in}{0.864585in}}{\pgfqpoint{1.121414in}{0.868703in}}%
\pgfpathcurveto{\pgfqpoint{1.117296in}{0.872821in}}{\pgfqpoint{1.111709in}{0.875135in}}{\pgfqpoint{1.105886in}{0.875135in}}%
\pgfpathcurveto{\pgfqpoint{1.100062in}{0.875135in}}{\pgfqpoint{1.094475in}{0.872821in}}{\pgfqpoint{1.090357in}{0.868703in}}%
\pgfpathcurveto{\pgfqpoint{1.086239in}{0.864585in}}{\pgfqpoint{1.083925in}{0.858999in}}{\pgfqpoint{1.083925in}{0.853175in}}%
\pgfpathcurveto{\pgfqpoint{1.083925in}{0.847351in}}{\pgfqpoint{1.086239in}{0.841765in}}{\pgfqpoint{1.090357in}{0.837647in}}%
\pgfpathcurveto{\pgfqpoint{1.094475in}{0.833528in}}{\pgfqpoint{1.100062in}{0.831215in}}{\pgfqpoint{1.105886in}{0.831215in}}%
\pgfpathclose%
\pgfusepath{stroke,fill}%
\end{pgfscope}%
\begin{pgfscope}%
\pgfpathrectangle{\pgfqpoint{0.211875in}{0.211875in}}{\pgfqpoint{1.313625in}{1.279725in}}%
\pgfusepath{clip}%
\pgfsetbuttcap%
\pgfsetroundjoin%
\definecolor{currentfill}{rgb}{0.121569,0.466667,0.705882}%
\pgfsetfillcolor{currentfill}%
\pgfsetlinewidth{1.003750pt}%
\definecolor{currentstroke}{rgb}{0.121569,0.466667,0.705882}%
\pgfsetstrokecolor{currentstroke}%
\pgfsetdash{}{0pt}%
\pgfpathmoveto{\pgfqpoint{1.264054in}{1.360896in}}%
\pgfpathcurveto{\pgfqpoint{1.269878in}{1.360896in}}{\pgfqpoint{1.275465in}{1.363210in}}{\pgfqpoint{1.279583in}{1.367328in}}%
\pgfpathcurveto{\pgfqpoint{1.283701in}{1.371446in}}{\pgfqpoint{1.286015in}{1.377032in}}{\pgfqpoint{1.286015in}{1.382856in}}%
\pgfpathcurveto{\pgfqpoint{1.286015in}{1.388680in}}{\pgfqpoint{1.283701in}{1.394266in}}{\pgfqpoint{1.279583in}{1.398385in}}%
\pgfpathcurveto{\pgfqpoint{1.275465in}{1.402503in}}{\pgfqpoint{1.269878in}{1.404817in}}{\pgfqpoint{1.264054in}{1.404817in}}%
\pgfpathcurveto{\pgfqpoint{1.258231in}{1.404817in}}{\pgfqpoint{1.252644in}{1.402503in}}{\pgfqpoint{1.248526in}{1.398385in}}%
\pgfpathcurveto{\pgfqpoint{1.244408in}{1.394266in}}{\pgfqpoint{1.242094in}{1.388680in}}{\pgfqpoint{1.242094in}{1.382856in}}%
\pgfpathcurveto{\pgfqpoint{1.242094in}{1.377032in}}{\pgfqpoint{1.244408in}{1.371446in}}{\pgfqpoint{1.248526in}{1.367328in}}%
\pgfpathcurveto{\pgfqpoint{1.252644in}{1.363210in}}{\pgfqpoint{1.258231in}{1.360896in}}{\pgfqpoint{1.264054in}{1.360896in}}%
\pgfpathclose%
\pgfusepath{stroke,fill}%
\end{pgfscope}%
\begin{pgfscope}%
\pgfpathrectangle{\pgfqpoint{0.211875in}{0.211875in}}{\pgfqpoint{1.313625in}{1.279725in}}%
\pgfusepath{clip}%
\pgfsetbuttcap%
\pgfsetroundjoin%
\definecolor{currentfill}{rgb}{0.121569,0.466667,0.705882}%
\pgfsetfillcolor{currentfill}%
\pgfsetlinewidth{1.003750pt}%
\definecolor{currentstroke}{rgb}{0.121569,0.466667,0.705882}%
\pgfsetstrokecolor{currentstroke}%
\pgfsetdash{}{0pt}%
\pgfpathmoveto{\pgfqpoint{1.124800in}{0.808847in}}%
\pgfpathcurveto{\pgfqpoint{1.130624in}{0.808847in}}{\pgfqpoint{1.136210in}{0.811161in}}{\pgfqpoint{1.140328in}{0.815279in}}%
\pgfpathcurveto{\pgfqpoint{1.144446in}{0.819397in}}{\pgfqpoint{1.146760in}{0.824983in}}{\pgfqpoint{1.146760in}{0.830807in}}%
\pgfpathcurveto{\pgfqpoint{1.146760in}{0.836631in}}{\pgfqpoint{1.144446in}{0.842217in}}{\pgfqpoint{1.140328in}{0.846335in}}%
\pgfpathcurveto{\pgfqpoint{1.136210in}{0.850454in}}{\pgfqpoint{1.130624in}{0.852768in}}{\pgfqpoint{1.124800in}{0.852768in}}%
\pgfpathcurveto{\pgfqpoint{1.118976in}{0.852768in}}{\pgfqpoint{1.113390in}{0.850454in}}{\pgfqpoint{1.109272in}{0.846335in}}%
\pgfpathcurveto{\pgfqpoint{1.105154in}{0.842217in}}{\pgfqpoint{1.102840in}{0.836631in}}{\pgfqpoint{1.102840in}{0.830807in}}%
\pgfpathcurveto{\pgfqpoint{1.102840in}{0.824983in}}{\pgfqpoint{1.105154in}{0.819397in}}{\pgfqpoint{1.109272in}{0.815279in}}%
\pgfpathcurveto{\pgfqpoint{1.113390in}{0.811161in}}{\pgfqpoint{1.118976in}{0.808847in}}{\pgfqpoint{1.124800in}{0.808847in}}%
\pgfpathclose%
\pgfusepath{stroke,fill}%
\end{pgfscope}%
\begin{pgfscope}%
\pgfpathrectangle{\pgfqpoint{0.211875in}{0.211875in}}{\pgfqpoint{1.313625in}{1.279725in}}%
\pgfusepath{clip}%
\pgfsetbuttcap%
\pgfsetroundjoin%
\definecolor{currentfill}{rgb}{0.121569,0.466667,0.705882}%
\pgfsetfillcolor{currentfill}%
\pgfsetlinewidth{1.003750pt}%
\definecolor{currentstroke}{rgb}{0.121569,0.466667,0.705882}%
\pgfsetstrokecolor{currentstroke}%
\pgfsetdash{}{0pt}%
\pgfpathmoveto{\pgfqpoint{0.441141in}{0.853537in}}%
\pgfpathcurveto{\pgfqpoint{0.446965in}{0.853537in}}{\pgfqpoint{0.452551in}{0.855851in}}{\pgfqpoint{0.456669in}{0.859969in}}%
\pgfpathcurveto{\pgfqpoint{0.460787in}{0.864088in}}{\pgfqpoint{0.463101in}{0.869674in}}{\pgfqpoint{0.463101in}{0.875498in}}%
\pgfpathcurveto{\pgfqpoint{0.463101in}{0.881322in}}{\pgfqpoint{0.460787in}{0.886908in}}{\pgfqpoint{0.456669in}{0.891026in}}%
\pgfpathcurveto{\pgfqpoint{0.452551in}{0.895144in}}{\pgfqpoint{0.446965in}{0.897458in}}{\pgfqpoint{0.441141in}{0.897458in}}%
\pgfpathcurveto{\pgfqpoint{0.435317in}{0.897458in}}{\pgfqpoint{0.429731in}{0.895144in}}{\pgfqpoint{0.425613in}{0.891026in}}%
\pgfpathcurveto{\pgfqpoint{0.421494in}{0.886908in}}{\pgfqpoint{0.419181in}{0.881322in}}{\pgfqpoint{0.419181in}{0.875498in}}%
\pgfpathcurveto{\pgfqpoint{0.419181in}{0.869674in}}{\pgfqpoint{0.421494in}{0.864088in}}{\pgfqpoint{0.425613in}{0.859969in}}%
\pgfpathcurveto{\pgfqpoint{0.429731in}{0.855851in}}{\pgfqpoint{0.435317in}{0.853537in}}{\pgfqpoint{0.441141in}{0.853537in}}%
\pgfpathclose%
\pgfusepath{stroke,fill}%
\end{pgfscope}%
\begin{pgfscope}%
\pgfpathrectangle{\pgfqpoint{0.211875in}{0.211875in}}{\pgfqpoint{1.313625in}{1.279725in}}%
\pgfusepath{clip}%
\pgfsetbuttcap%
\pgfsetroundjoin%
\definecolor{currentfill}{rgb}{0.121569,0.466667,0.705882}%
\pgfsetfillcolor{currentfill}%
\pgfsetlinewidth{1.003750pt}%
\definecolor{currentstroke}{rgb}{0.121569,0.466667,0.705882}%
\pgfsetstrokecolor{currentstroke}%
\pgfsetdash{}{0pt}%
\pgfpathmoveto{\pgfqpoint{1.144039in}{0.872064in}}%
\pgfpathcurveto{\pgfqpoint{1.149862in}{0.872064in}}{\pgfqpoint{1.155449in}{0.874378in}}{\pgfqpoint{1.159567in}{0.878496in}}%
\pgfpathcurveto{\pgfqpoint{1.163685in}{0.882615in}}{\pgfqpoint{1.165999in}{0.888201in}}{\pgfqpoint{1.165999in}{0.894025in}}%
\pgfpathcurveto{\pgfqpoint{1.165999in}{0.899849in}}{\pgfqpoint{1.163685in}{0.905435in}}{\pgfqpoint{1.159567in}{0.909553in}}%
\pgfpathcurveto{\pgfqpoint{1.155449in}{0.913671in}}{\pgfqpoint{1.149862in}{0.915985in}}{\pgfqpoint{1.144039in}{0.915985in}}%
\pgfpathcurveto{\pgfqpoint{1.138215in}{0.915985in}}{\pgfqpoint{1.132628in}{0.913671in}}{\pgfqpoint{1.128510in}{0.909553in}}%
\pgfpathcurveto{\pgfqpoint{1.124392in}{0.905435in}}{\pgfqpoint{1.122078in}{0.899849in}}{\pgfqpoint{1.122078in}{0.894025in}}%
\pgfpathcurveto{\pgfqpoint{1.122078in}{0.888201in}}{\pgfqpoint{1.124392in}{0.882615in}}{\pgfqpoint{1.128510in}{0.878496in}}%
\pgfpathcurveto{\pgfqpoint{1.132628in}{0.874378in}}{\pgfqpoint{1.138215in}{0.872064in}}{\pgfqpoint{1.144039in}{0.872064in}}%
\pgfpathclose%
\pgfusepath{stroke,fill}%
\end{pgfscope}%
\begin{pgfscope}%
\pgfpathrectangle{\pgfqpoint{0.211875in}{0.211875in}}{\pgfqpoint{1.313625in}{1.279725in}}%
\pgfusepath{clip}%
\pgfsetbuttcap%
\pgfsetroundjoin%
\definecolor{currentfill}{rgb}{0.121569,0.466667,0.705882}%
\pgfsetfillcolor{currentfill}%
\pgfsetlinewidth{1.003750pt}%
\definecolor{currentstroke}{rgb}{0.121569,0.466667,0.705882}%
\pgfsetstrokecolor{currentstroke}%
\pgfsetdash{}{0pt}%
\pgfpathmoveto{\pgfqpoint{1.132416in}{0.823731in}}%
\pgfpathcurveto{\pgfqpoint{1.138240in}{0.823731in}}{\pgfqpoint{1.143826in}{0.826045in}}{\pgfqpoint{1.147944in}{0.830163in}}%
\pgfpathcurveto{\pgfqpoint{1.152062in}{0.834282in}}{\pgfqpoint{1.154376in}{0.839868in}}{\pgfqpoint{1.154376in}{0.845692in}}%
\pgfpathcurveto{\pgfqpoint{1.154376in}{0.851516in}}{\pgfqpoint{1.152062in}{0.857102in}}{\pgfqpoint{1.147944in}{0.861220in}}%
\pgfpathcurveto{\pgfqpoint{1.143826in}{0.865338in}}{\pgfqpoint{1.138240in}{0.867652in}}{\pgfqpoint{1.132416in}{0.867652in}}%
\pgfpathcurveto{\pgfqpoint{1.126592in}{0.867652in}}{\pgfqpoint{1.121006in}{0.865338in}}{\pgfqpoint{1.116888in}{0.861220in}}%
\pgfpathcurveto{\pgfqpoint{1.112769in}{0.857102in}}{\pgfqpoint{1.110456in}{0.851516in}}{\pgfqpoint{1.110456in}{0.845692in}}%
\pgfpathcurveto{\pgfqpoint{1.110456in}{0.839868in}}{\pgfqpoint{1.112769in}{0.834282in}}{\pgfqpoint{1.116888in}{0.830163in}}%
\pgfpathcurveto{\pgfqpoint{1.121006in}{0.826045in}}{\pgfqpoint{1.126592in}{0.823731in}}{\pgfqpoint{1.132416in}{0.823731in}}%
\pgfpathclose%
\pgfusepath{stroke,fill}%
\end{pgfscope}%
\begin{pgfscope}%
\pgfpathrectangle{\pgfqpoint{0.211875in}{0.211875in}}{\pgfqpoint{1.313625in}{1.279725in}}%
\pgfusepath{clip}%
\pgfsetbuttcap%
\pgfsetroundjoin%
\definecolor{currentfill}{rgb}{0.121569,0.466667,0.705882}%
\pgfsetfillcolor{currentfill}%
\pgfsetlinewidth{1.003750pt}%
\definecolor{currentstroke}{rgb}{0.121569,0.466667,0.705882}%
\pgfsetstrokecolor{currentstroke}%
\pgfsetdash{}{0pt}%
\pgfpathmoveto{\pgfqpoint{1.132138in}{0.830354in}}%
\pgfpathcurveto{\pgfqpoint{1.137962in}{0.830354in}}{\pgfqpoint{1.143548in}{0.832668in}}{\pgfqpoint{1.147666in}{0.836786in}}%
\pgfpathcurveto{\pgfqpoint{1.151784in}{0.840904in}}{\pgfqpoint{1.154098in}{0.846491in}}{\pgfqpoint{1.154098in}{0.852315in}}%
\pgfpathcurveto{\pgfqpoint{1.154098in}{0.858138in}}{\pgfqpoint{1.151784in}{0.863725in}}{\pgfqpoint{1.147666in}{0.867843in}}%
\pgfpathcurveto{\pgfqpoint{1.143548in}{0.871961in}}{\pgfqpoint{1.137962in}{0.874275in}}{\pgfqpoint{1.132138in}{0.874275in}}%
\pgfpathcurveto{\pgfqpoint{1.126314in}{0.874275in}}{\pgfqpoint{1.120728in}{0.871961in}}{\pgfqpoint{1.116610in}{0.867843in}}%
\pgfpathcurveto{\pgfqpoint{1.112491in}{0.863725in}}{\pgfqpoint{1.110178in}{0.858138in}}{\pgfqpoint{1.110178in}{0.852315in}}%
\pgfpathcurveto{\pgfqpoint{1.110178in}{0.846491in}}{\pgfqpoint{1.112491in}{0.840904in}}{\pgfqpoint{1.116610in}{0.836786in}}%
\pgfpathcurveto{\pgfqpoint{1.120728in}{0.832668in}}{\pgfqpoint{1.126314in}{0.830354in}}{\pgfqpoint{1.132138in}{0.830354in}}%
\pgfpathclose%
\pgfusepath{stroke,fill}%
\end{pgfscope}%
\begin{pgfscope}%
\pgfpathrectangle{\pgfqpoint{0.211875in}{0.211875in}}{\pgfqpoint{1.313625in}{1.279725in}}%
\pgfusepath{clip}%
\pgfsetbuttcap%
\pgfsetroundjoin%
\definecolor{currentfill}{rgb}{0.121569,0.466667,0.705882}%
\pgfsetfillcolor{currentfill}%
\pgfsetlinewidth{1.003750pt}%
\definecolor{currentstroke}{rgb}{0.121569,0.466667,0.705882}%
\pgfsetstrokecolor{currentstroke}%
\pgfsetdash{}{0pt}%
\pgfpathmoveto{\pgfqpoint{1.446405in}{1.093437in}}%
\pgfpathcurveto{\pgfqpoint{1.452229in}{1.093437in}}{\pgfqpoint{1.457815in}{1.095751in}}{\pgfqpoint{1.461933in}{1.099869in}}%
\pgfpathcurveto{\pgfqpoint{1.466052in}{1.103987in}}{\pgfqpoint{1.468365in}{1.109573in}}{\pgfqpoint{1.468365in}{1.115397in}}%
\pgfpathcurveto{\pgfqpoint{1.468365in}{1.121221in}}{\pgfqpoint{1.466052in}{1.126807in}}{\pgfqpoint{1.461933in}{1.130926in}}%
\pgfpathcurveto{\pgfqpoint{1.457815in}{1.135044in}}{\pgfqpoint{1.452229in}{1.137358in}}{\pgfqpoint{1.446405in}{1.137358in}}%
\pgfpathcurveto{\pgfqpoint{1.440581in}{1.137358in}}{\pgfqpoint{1.434995in}{1.135044in}}{\pgfqpoint{1.430877in}{1.130926in}}%
\pgfpathcurveto{\pgfqpoint{1.426759in}{1.126807in}}{\pgfqpoint{1.424445in}{1.121221in}}{\pgfqpoint{1.424445in}{1.115397in}}%
\pgfpathcurveto{\pgfqpoint{1.424445in}{1.109573in}}{\pgfqpoint{1.426759in}{1.103987in}}{\pgfqpoint{1.430877in}{1.099869in}}%
\pgfpathcurveto{\pgfqpoint{1.434995in}{1.095751in}}{\pgfqpoint{1.440581in}{1.093437in}}{\pgfqpoint{1.446405in}{1.093437in}}%
\pgfpathclose%
\pgfusepath{stroke,fill}%
\end{pgfscope}%
\begin{pgfscope}%
\pgfpathrectangle{\pgfqpoint{0.211875in}{0.211875in}}{\pgfqpoint{1.313625in}{1.279725in}}%
\pgfusepath{clip}%
\pgfsetbuttcap%
\pgfsetroundjoin%
\definecolor{currentfill}{rgb}{0.121569,0.466667,0.705882}%
\pgfsetfillcolor{currentfill}%
\pgfsetlinewidth{1.003750pt}%
\definecolor{currentstroke}{rgb}{0.121569,0.466667,0.705882}%
\pgfsetstrokecolor{currentstroke}%
\pgfsetdash{}{0pt}%
\pgfpathmoveto{\pgfqpoint{1.276093in}{1.361398in}}%
\pgfpathcurveto{\pgfqpoint{1.281917in}{1.361398in}}{\pgfqpoint{1.287503in}{1.363711in}}{\pgfqpoint{1.291622in}{1.367830in}}%
\pgfpathcurveto{\pgfqpoint{1.295740in}{1.371948in}}{\pgfqpoint{1.298054in}{1.377534in}}{\pgfqpoint{1.298054in}{1.383358in}}%
\pgfpathcurveto{\pgfqpoint{1.298054in}{1.389182in}}{\pgfqpoint{1.295740in}{1.394768in}}{\pgfqpoint{1.291622in}{1.398886in}}%
\pgfpathcurveto{\pgfqpoint{1.287503in}{1.403004in}}{\pgfqpoint{1.281917in}{1.405318in}}{\pgfqpoint{1.276093in}{1.405318in}}%
\pgfpathcurveto{\pgfqpoint{1.270269in}{1.405318in}}{\pgfqpoint{1.264683in}{1.403004in}}{\pgfqpoint{1.260565in}{1.398886in}}%
\pgfpathcurveto{\pgfqpoint{1.256447in}{1.394768in}}{\pgfqpoint{1.254133in}{1.389182in}}{\pgfqpoint{1.254133in}{1.383358in}}%
\pgfpathcurveto{\pgfqpoint{1.254133in}{1.377534in}}{\pgfqpoint{1.256447in}{1.371948in}}{\pgfqpoint{1.260565in}{1.367830in}}%
\pgfpathcurveto{\pgfqpoint{1.264683in}{1.363711in}}{\pgfqpoint{1.270269in}{1.361398in}}{\pgfqpoint{1.276093in}{1.361398in}}%
\pgfpathclose%
\pgfusepath{stroke,fill}%
\end{pgfscope}%
\begin{pgfscope}%
\pgfpathrectangle{\pgfqpoint{0.211875in}{0.211875in}}{\pgfqpoint{1.313625in}{1.279725in}}%
\pgfusepath{clip}%
\pgfsetbuttcap%
\pgfsetroundjoin%
\definecolor{currentfill}{rgb}{0.121569,0.466667,0.705882}%
\pgfsetfillcolor{currentfill}%
\pgfsetlinewidth{1.003750pt}%
\definecolor{currentstroke}{rgb}{0.121569,0.466667,0.705882}%
\pgfsetstrokecolor{currentstroke}%
\pgfsetdash{}{0pt}%
\pgfpathmoveto{\pgfqpoint{1.432063in}{0.821153in}}%
\pgfpathcurveto{\pgfqpoint{1.437887in}{0.821153in}}{\pgfqpoint{1.443473in}{0.823467in}}{\pgfqpoint{1.447591in}{0.827585in}}%
\pgfpathcurveto{\pgfqpoint{1.451709in}{0.831703in}}{\pgfqpoint{1.454023in}{0.837289in}}{\pgfqpoint{1.454023in}{0.843113in}}%
\pgfpathcurveto{\pgfqpoint{1.454023in}{0.848937in}}{\pgfqpoint{1.451709in}{0.854523in}}{\pgfqpoint{1.447591in}{0.858641in}}%
\pgfpathcurveto{\pgfqpoint{1.443473in}{0.862759in}}{\pgfqpoint{1.437887in}{0.865073in}}{\pgfqpoint{1.432063in}{0.865073in}}%
\pgfpathcurveto{\pgfqpoint{1.426239in}{0.865073in}}{\pgfqpoint{1.420653in}{0.862759in}}{\pgfqpoint{1.416535in}{0.858641in}}%
\pgfpathcurveto{\pgfqpoint{1.412417in}{0.854523in}}{\pgfqpoint{1.410103in}{0.848937in}}{\pgfqpoint{1.410103in}{0.843113in}}%
\pgfpathcurveto{\pgfqpoint{1.410103in}{0.837289in}}{\pgfqpoint{1.412417in}{0.831703in}}{\pgfqpoint{1.416535in}{0.827585in}}%
\pgfpathcurveto{\pgfqpoint{1.420653in}{0.823467in}}{\pgfqpoint{1.426239in}{0.821153in}}{\pgfqpoint{1.432063in}{0.821153in}}%
\pgfpathclose%
\pgfusepath{stroke,fill}%
\end{pgfscope}%
\begin{pgfscope}%
\pgfpathrectangle{\pgfqpoint{0.211875in}{0.211875in}}{\pgfqpoint{1.313625in}{1.279725in}}%
\pgfusepath{clip}%
\pgfsetbuttcap%
\pgfsetroundjoin%
\definecolor{currentfill}{rgb}{0.121569,0.466667,0.705882}%
\pgfsetfillcolor{currentfill}%
\pgfsetlinewidth{1.003750pt}%
\definecolor{currentstroke}{rgb}{0.121569,0.466667,0.705882}%
\pgfsetstrokecolor{currentstroke}%
\pgfsetdash{}{0pt}%
\pgfpathmoveto{\pgfqpoint{1.110115in}{0.804412in}}%
\pgfpathcurveto{\pgfqpoint{1.115939in}{0.804412in}}{\pgfqpoint{1.121525in}{0.806726in}}{\pgfqpoint{1.125643in}{0.810844in}}%
\pgfpathcurveto{\pgfqpoint{1.129761in}{0.814962in}}{\pgfqpoint{1.132075in}{0.820548in}}{\pgfqpoint{1.132075in}{0.826372in}}%
\pgfpathcurveto{\pgfqpoint{1.132075in}{0.832196in}}{\pgfqpoint{1.129761in}{0.837782in}}{\pgfqpoint{1.125643in}{0.841900in}}%
\pgfpathcurveto{\pgfqpoint{1.121525in}{0.846018in}}{\pgfqpoint{1.115939in}{0.848332in}}{\pgfqpoint{1.110115in}{0.848332in}}%
\pgfpathcurveto{\pgfqpoint{1.104291in}{0.848332in}}{\pgfqpoint{1.098705in}{0.846018in}}{\pgfqpoint{1.094587in}{0.841900in}}%
\pgfpathcurveto{\pgfqpoint{1.090469in}{0.837782in}}{\pgfqpoint{1.088155in}{0.832196in}}{\pgfqpoint{1.088155in}{0.826372in}}%
\pgfpathcurveto{\pgfqpoint{1.088155in}{0.820548in}}{\pgfqpoint{1.090469in}{0.814962in}}{\pgfqpoint{1.094587in}{0.810844in}}%
\pgfpathcurveto{\pgfqpoint{1.098705in}{0.806726in}}{\pgfqpoint{1.104291in}{0.804412in}}{\pgfqpoint{1.110115in}{0.804412in}}%
\pgfpathclose%
\pgfusepath{stroke,fill}%
\end{pgfscope}%
\begin{pgfscope}%
\pgfpathrectangle{\pgfqpoint{0.211875in}{0.211875in}}{\pgfqpoint{1.313625in}{1.279725in}}%
\pgfusepath{clip}%
\pgfsetbuttcap%
\pgfsetroundjoin%
\definecolor{currentfill}{rgb}{0.121569,0.466667,0.705882}%
\pgfsetfillcolor{currentfill}%
\pgfsetlinewidth{1.003750pt}%
\definecolor{currentstroke}{rgb}{0.121569,0.466667,0.705882}%
\pgfsetstrokecolor{currentstroke}%
\pgfsetdash{}{0pt}%
\pgfpathmoveto{\pgfqpoint{1.115285in}{0.800222in}}%
\pgfpathcurveto{\pgfqpoint{1.121109in}{0.800222in}}{\pgfqpoint{1.126695in}{0.802536in}}{\pgfqpoint{1.130814in}{0.806654in}}%
\pgfpathcurveto{\pgfqpoint{1.134932in}{0.810772in}}{\pgfqpoint{1.137246in}{0.816358in}}{\pgfqpoint{1.137246in}{0.822182in}}%
\pgfpathcurveto{\pgfqpoint{1.137246in}{0.828006in}}{\pgfqpoint{1.134932in}{0.833592in}}{\pgfqpoint{1.130814in}{0.837711in}}%
\pgfpathcurveto{\pgfqpoint{1.126695in}{0.841829in}}{\pgfqpoint{1.121109in}{0.844143in}}{\pgfqpoint{1.115285in}{0.844143in}}%
\pgfpathcurveto{\pgfqpoint{1.109461in}{0.844143in}}{\pgfqpoint{1.103875in}{0.841829in}}{\pgfqpoint{1.099757in}{0.837711in}}%
\pgfpathcurveto{\pgfqpoint{1.095639in}{0.833592in}}{\pgfqpoint{1.093325in}{0.828006in}}{\pgfqpoint{1.093325in}{0.822182in}}%
\pgfpathcurveto{\pgfqpoint{1.093325in}{0.816358in}}{\pgfqpoint{1.095639in}{0.810772in}}{\pgfqpoint{1.099757in}{0.806654in}}%
\pgfpathcurveto{\pgfqpoint{1.103875in}{0.802536in}}{\pgfqpoint{1.109461in}{0.800222in}}{\pgfqpoint{1.115285in}{0.800222in}}%
\pgfpathclose%
\pgfusepath{stroke,fill}%
\end{pgfscope}%
\begin{pgfscope}%
\pgfpathrectangle{\pgfqpoint{0.211875in}{0.211875in}}{\pgfqpoint{1.313625in}{1.279725in}}%
\pgfusepath{clip}%
\pgfsetbuttcap%
\pgfsetroundjoin%
\definecolor{currentfill}{rgb}{0.121569,0.466667,0.705882}%
\pgfsetfillcolor{currentfill}%
\pgfsetlinewidth{1.003750pt}%
\definecolor{currentstroke}{rgb}{0.121569,0.466667,0.705882}%
\pgfsetstrokecolor{currentstroke}%
\pgfsetdash{}{0pt}%
\pgfpathmoveto{\pgfqpoint{1.415842in}{0.815938in}}%
\pgfpathcurveto{\pgfqpoint{1.421666in}{0.815938in}}{\pgfqpoint{1.427252in}{0.818252in}}{\pgfqpoint{1.431370in}{0.822370in}}%
\pgfpathcurveto{\pgfqpoint{1.435488in}{0.826488in}}{\pgfqpoint{1.437802in}{0.832074in}}{\pgfqpoint{1.437802in}{0.837898in}}%
\pgfpathcurveto{\pgfqpoint{1.437802in}{0.843722in}}{\pgfqpoint{1.435488in}{0.849308in}}{\pgfqpoint{1.431370in}{0.853426in}}%
\pgfpathcurveto{\pgfqpoint{1.427252in}{0.857544in}}{\pgfqpoint{1.421666in}{0.859858in}}{\pgfqpoint{1.415842in}{0.859858in}}%
\pgfpathcurveto{\pgfqpoint{1.410018in}{0.859858in}}{\pgfqpoint{1.404432in}{0.857544in}}{\pgfqpoint{1.400313in}{0.853426in}}%
\pgfpathcurveto{\pgfqpoint{1.396195in}{0.849308in}}{\pgfqpoint{1.393881in}{0.843722in}}{\pgfqpoint{1.393881in}{0.837898in}}%
\pgfpathcurveto{\pgfqpoint{1.393881in}{0.832074in}}{\pgfqpoint{1.396195in}{0.826488in}}{\pgfqpoint{1.400313in}{0.822370in}}%
\pgfpathcurveto{\pgfqpoint{1.404432in}{0.818252in}}{\pgfqpoint{1.410018in}{0.815938in}}{\pgfqpoint{1.415842in}{0.815938in}}%
\pgfpathclose%
\pgfusepath{stroke,fill}%
\end{pgfscope}%
\begin{pgfscope}%
\pgfpathrectangle{\pgfqpoint{0.211875in}{0.211875in}}{\pgfqpoint{1.313625in}{1.279725in}}%
\pgfusepath{clip}%
\pgfsetbuttcap%
\pgfsetroundjoin%
\definecolor{currentfill}{rgb}{0.121569,0.466667,0.705882}%
\pgfsetfillcolor{currentfill}%
\pgfsetlinewidth{1.003750pt}%
\definecolor{currentstroke}{rgb}{0.121569,0.466667,0.705882}%
\pgfsetstrokecolor{currentstroke}%
\pgfsetdash{}{0pt}%
\pgfpathmoveto{\pgfqpoint{1.430265in}{0.817404in}}%
\pgfpathcurveto{\pgfqpoint{1.436089in}{0.817404in}}{\pgfqpoint{1.441675in}{0.819718in}}{\pgfqpoint{1.445794in}{0.823836in}}%
\pgfpathcurveto{\pgfqpoint{1.449912in}{0.827954in}}{\pgfqpoint{1.452226in}{0.833540in}}{\pgfqpoint{1.452226in}{0.839364in}}%
\pgfpathcurveto{\pgfqpoint{1.452226in}{0.845188in}}{\pgfqpoint{1.449912in}{0.850774in}}{\pgfqpoint{1.445794in}{0.854892in}}%
\pgfpathcurveto{\pgfqpoint{1.441675in}{0.859011in}}{\pgfqpoint{1.436089in}{0.861324in}}{\pgfqpoint{1.430265in}{0.861324in}}%
\pgfpathcurveto{\pgfqpoint{1.424441in}{0.861324in}}{\pgfqpoint{1.418855in}{0.859011in}}{\pgfqpoint{1.414737in}{0.854892in}}%
\pgfpathcurveto{\pgfqpoint{1.410619in}{0.850774in}}{\pgfqpoint{1.408305in}{0.845188in}}{\pgfqpoint{1.408305in}{0.839364in}}%
\pgfpathcurveto{\pgfqpoint{1.408305in}{0.833540in}}{\pgfqpoint{1.410619in}{0.827954in}}{\pgfqpoint{1.414737in}{0.823836in}}%
\pgfpathcurveto{\pgfqpoint{1.418855in}{0.819718in}}{\pgfqpoint{1.424441in}{0.817404in}}{\pgfqpoint{1.430265in}{0.817404in}}%
\pgfpathclose%
\pgfusepath{stroke,fill}%
\end{pgfscope}%
\begin{pgfscope}%
\pgfpathrectangle{\pgfqpoint{0.211875in}{0.211875in}}{\pgfqpoint{1.313625in}{1.279725in}}%
\pgfusepath{clip}%
\pgfsetbuttcap%
\pgfsetroundjoin%
\definecolor{currentfill}{rgb}{0.121569,0.466667,0.705882}%
\pgfsetfillcolor{currentfill}%
\pgfsetlinewidth{1.003750pt}%
\definecolor{currentstroke}{rgb}{0.121569,0.466667,0.705882}%
\pgfsetstrokecolor{currentstroke}%
\pgfsetdash{}{0pt}%
\pgfpathmoveto{\pgfqpoint{1.435710in}{0.676029in}}%
\pgfpathcurveto{\pgfqpoint{1.441534in}{0.676029in}}{\pgfqpoint{1.447120in}{0.678343in}}{\pgfqpoint{1.451239in}{0.682461in}}%
\pgfpathcurveto{\pgfqpoint{1.455357in}{0.686579in}}{\pgfqpoint{1.457671in}{0.692165in}}{\pgfqpoint{1.457671in}{0.697989in}}%
\pgfpathcurveto{\pgfqpoint{1.457671in}{0.703813in}}{\pgfqpoint{1.455357in}{0.709400in}}{\pgfqpoint{1.451239in}{0.713518in}}%
\pgfpathcurveto{\pgfqpoint{1.447120in}{0.717636in}}{\pgfqpoint{1.441534in}{0.719950in}}{\pgfqpoint{1.435710in}{0.719950in}}%
\pgfpathcurveto{\pgfqpoint{1.429886in}{0.719950in}}{\pgfqpoint{1.424300in}{0.717636in}}{\pgfqpoint{1.420182in}{0.713518in}}%
\pgfpathcurveto{\pgfqpoint{1.416064in}{0.709400in}}{\pgfqpoint{1.413750in}{0.703813in}}{\pgfqpoint{1.413750in}{0.697989in}}%
\pgfpathcurveto{\pgfqpoint{1.413750in}{0.692165in}}{\pgfqpoint{1.416064in}{0.686579in}}{\pgfqpoint{1.420182in}{0.682461in}}%
\pgfpathcurveto{\pgfqpoint{1.424300in}{0.678343in}}{\pgfqpoint{1.429886in}{0.676029in}}{\pgfqpoint{1.435710in}{0.676029in}}%
\pgfpathclose%
\pgfusepath{stroke,fill}%
\end{pgfscope}%
\begin{pgfscope}%
\pgfpathrectangle{\pgfqpoint{0.211875in}{0.211875in}}{\pgfqpoint{1.313625in}{1.279725in}}%
\pgfusepath{clip}%
\pgfsetbuttcap%
\pgfsetroundjoin%
\definecolor{currentfill}{rgb}{0.121569,0.466667,0.705882}%
\pgfsetfillcolor{currentfill}%
\pgfsetlinewidth{1.003750pt}%
\definecolor{currentstroke}{rgb}{0.121569,0.466667,0.705882}%
\pgfsetstrokecolor{currentstroke}%
\pgfsetdash{}{0pt}%
\pgfpathmoveto{\pgfqpoint{1.093285in}{0.815222in}}%
\pgfpathcurveto{\pgfqpoint{1.099109in}{0.815222in}}{\pgfqpoint{1.104695in}{0.817536in}}{\pgfqpoint{1.108813in}{0.821654in}}%
\pgfpathcurveto{\pgfqpoint{1.112931in}{0.825772in}}{\pgfqpoint{1.115245in}{0.831358in}}{\pgfqpoint{1.115245in}{0.837182in}}%
\pgfpathcurveto{\pgfqpoint{1.115245in}{0.843006in}}{\pgfqpoint{1.112931in}{0.848592in}}{\pgfqpoint{1.108813in}{0.852710in}}%
\pgfpathcurveto{\pgfqpoint{1.104695in}{0.856829in}}{\pgfqpoint{1.099109in}{0.859142in}}{\pgfqpoint{1.093285in}{0.859142in}}%
\pgfpathcurveto{\pgfqpoint{1.087461in}{0.859142in}}{\pgfqpoint{1.081875in}{0.856829in}}{\pgfqpoint{1.077757in}{0.852710in}}%
\pgfpathcurveto{\pgfqpoint{1.073639in}{0.848592in}}{\pgfqpoint{1.071325in}{0.843006in}}{\pgfqpoint{1.071325in}{0.837182in}}%
\pgfpathcurveto{\pgfqpoint{1.071325in}{0.831358in}}{\pgfqpoint{1.073639in}{0.825772in}}{\pgfqpoint{1.077757in}{0.821654in}}%
\pgfpathcurveto{\pgfqpoint{1.081875in}{0.817536in}}{\pgfqpoint{1.087461in}{0.815222in}}{\pgfqpoint{1.093285in}{0.815222in}}%
\pgfpathclose%
\pgfusepath{stroke,fill}%
\end{pgfscope}%
\begin{pgfscope}%
\pgfpathrectangle{\pgfqpoint{0.211875in}{0.211875in}}{\pgfqpoint{1.313625in}{1.279725in}}%
\pgfusepath{clip}%
\pgfsetbuttcap%
\pgfsetroundjoin%
\definecolor{currentfill}{rgb}{0.121569,0.466667,0.705882}%
\pgfsetfillcolor{currentfill}%
\pgfsetlinewidth{1.003750pt}%
\definecolor{currentstroke}{rgb}{0.121569,0.466667,0.705882}%
\pgfsetstrokecolor{currentstroke}%
\pgfsetdash{}{0pt}%
\pgfpathmoveto{\pgfqpoint{1.431453in}{1.129875in}}%
\pgfpathcurveto{\pgfqpoint{1.437277in}{1.129875in}}{\pgfqpoint{1.442863in}{1.132189in}}{\pgfqpoint{1.446981in}{1.136307in}}%
\pgfpathcurveto{\pgfqpoint{1.451099in}{1.140425in}}{\pgfqpoint{1.453413in}{1.146011in}}{\pgfqpoint{1.453413in}{1.151835in}}%
\pgfpathcurveto{\pgfqpoint{1.453413in}{1.157659in}}{\pgfqpoint{1.451099in}{1.163245in}}{\pgfqpoint{1.446981in}{1.167363in}}%
\pgfpathcurveto{\pgfqpoint{1.442863in}{1.171481in}}{\pgfqpoint{1.437277in}{1.173795in}}{\pgfqpoint{1.431453in}{1.173795in}}%
\pgfpathcurveto{\pgfqpoint{1.425629in}{1.173795in}}{\pgfqpoint{1.420043in}{1.171481in}}{\pgfqpoint{1.415925in}{1.167363in}}%
\pgfpathcurveto{\pgfqpoint{1.411807in}{1.163245in}}{\pgfqpoint{1.409493in}{1.157659in}}{\pgfqpoint{1.409493in}{1.151835in}}%
\pgfpathcurveto{\pgfqpoint{1.409493in}{1.146011in}}{\pgfqpoint{1.411807in}{1.140425in}}{\pgfqpoint{1.415925in}{1.136307in}}%
\pgfpathcurveto{\pgfqpoint{1.420043in}{1.132189in}}{\pgfqpoint{1.425629in}{1.129875in}}{\pgfqpoint{1.431453in}{1.129875in}}%
\pgfpathclose%
\pgfusepath{stroke,fill}%
\end{pgfscope}%
\begin{pgfscope}%
\pgfpathrectangle{\pgfqpoint{0.211875in}{0.211875in}}{\pgfqpoint{1.313625in}{1.279725in}}%
\pgfusepath{clip}%
\pgfsetbuttcap%
\pgfsetroundjoin%
\definecolor{currentfill}{rgb}{0.121569,0.466667,0.705882}%
\pgfsetfillcolor{currentfill}%
\pgfsetlinewidth{1.003750pt}%
\definecolor{currentstroke}{rgb}{0.121569,0.466667,0.705882}%
\pgfsetstrokecolor{currentstroke}%
\pgfsetdash{}{0pt}%
\pgfpathmoveto{\pgfqpoint{1.253347in}{1.359842in}}%
\pgfpathcurveto{\pgfqpoint{1.259171in}{1.359842in}}{\pgfqpoint{1.264757in}{1.362156in}}{\pgfqpoint{1.268875in}{1.366274in}}%
\pgfpathcurveto{\pgfqpoint{1.272993in}{1.370392in}}{\pgfqpoint{1.275307in}{1.375978in}}{\pgfqpoint{1.275307in}{1.381802in}}%
\pgfpathcurveto{\pgfqpoint{1.275307in}{1.387626in}}{\pgfqpoint{1.272993in}{1.393212in}}{\pgfqpoint{1.268875in}{1.397330in}}%
\pgfpathcurveto{\pgfqpoint{1.264757in}{1.401448in}}{\pgfqpoint{1.259171in}{1.403762in}}{\pgfqpoint{1.253347in}{1.403762in}}%
\pgfpathcurveto{\pgfqpoint{1.247523in}{1.403762in}}{\pgfqpoint{1.241937in}{1.401448in}}{\pgfqpoint{1.237818in}{1.397330in}}%
\pgfpathcurveto{\pgfqpoint{1.233700in}{1.393212in}}{\pgfqpoint{1.231386in}{1.387626in}}{\pgfqpoint{1.231386in}{1.381802in}}%
\pgfpathcurveto{\pgfqpoint{1.231386in}{1.375978in}}{\pgfqpoint{1.233700in}{1.370392in}}{\pgfqpoint{1.237818in}{1.366274in}}%
\pgfpathcurveto{\pgfqpoint{1.241937in}{1.362156in}}{\pgfqpoint{1.247523in}{1.359842in}}{\pgfqpoint{1.253347in}{1.359842in}}%
\pgfpathclose%
\pgfusepath{stroke,fill}%
\end{pgfscope}%
\begin{pgfscope}%
\pgfpathrectangle{\pgfqpoint{0.211875in}{0.211875in}}{\pgfqpoint{1.313625in}{1.279725in}}%
\pgfusepath{clip}%
\pgfsetbuttcap%
\pgfsetroundjoin%
\definecolor{currentfill}{rgb}{0.121569,0.466667,0.705882}%
\pgfsetfillcolor{currentfill}%
\pgfsetlinewidth{1.003750pt}%
\definecolor{currentstroke}{rgb}{0.121569,0.466667,0.705882}%
\pgfsetstrokecolor{currentstroke}%
\pgfsetdash{}{0pt}%
\pgfpathmoveto{\pgfqpoint{1.263322in}{1.356703in}}%
\pgfpathcurveto{\pgfqpoint{1.269146in}{1.356703in}}{\pgfqpoint{1.274732in}{1.359017in}}{\pgfqpoint{1.278850in}{1.363135in}}%
\pgfpathcurveto{\pgfqpoint{1.282968in}{1.367253in}}{\pgfqpoint{1.285282in}{1.372839in}}{\pgfqpoint{1.285282in}{1.378663in}}%
\pgfpathcurveto{\pgfqpoint{1.285282in}{1.384487in}}{\pgfqpoint{1.282968in}{1.390073in}}{\pgfqpoint{1.278850in}{1.394191in}}%
\pgfpathcurveto{\pgfqpoint{1.274732in}{1.398309in}}{\pgfqpoint{1.269146in}{1.400623in}}{\pgfqpoint{1.263322in}{1.400623in}}%
\pgfpathcurveto{\pgfqpoint{1.257498in}{1.400623in}}{\pgfqpoint{1.251912in}{1.398309in}}{\pgfqpoint{1.247794in}{1.394191in}}%
\pgfpathcurveto{\pgfqpoint{1.243675in}{1.390073in}}{\pgfqpoint{1.241362in}{1.384487in}}{\pgfqpoint{1.241362in}{1.378663in}}%
\pgfpathcurveto{\pgfqpoint{1.241362in}{1.372839in}}{\pgfqpoint{1.243675in}{1.367253in}}{\pgfqpoint{1.247794in}{1.363135in}}%
\pgfpathcurveto{\pgfqpoint{1.251912in}{1.359017in}}{\pgfqpoint{1.257498in}{1.356703in}}{\pgfqpoint{1.263322in}{1.356703in}}%
\pgfpathclose%
\pgfusepath{stroke,fill}%
\end{pgfscope}%
\begin{pgfscope}%
\pgfpathrectangle{\pgfqpoint{0.211875in}{0.211875in}}{\pgfqpoint{1.313625in}{1.279725in}}%
\pgfusepath{clip}%
\pgfsetbuttcap%
\pgfsetroundjoin%
\definecolor{currentfill}{rgb}{0.121569,0.466667,0.705882}%
\pgfsetfillcolor{currentfill}%
\pgfsetlinewidth{1.003750pt}%
\definecolor{currentstroke}{rgb}{0.121569,0.466667,0.705882}%
\pgfsetstrokecolor{currentstroke}%
\pgfsetdash{}{0pt}%
\pgfpathmoveto{\pgfqpoint{1.283875in}{1.341847in}}%
\pgfpathcurveto{\pgfqpoint{1.289699in}{1.341847in}}{\pgfqpoint{1.295285in}{1.344161in}}{\pgfqpoint{1.299404in}{1.348279in}}%
\pgfpathcurveto{\pgfqpoint{1.303522in}{1.352397in}}{\pgfqpoint{1.305836in}{1.357983in}}{\pgfqpoint{1.305836in}{1.363807in}}%
\pgfpathcurveto{\pgfqpoint{1.305836in}{1.369631in}}{\pgfqpoint{1.303522in}{1.375217in}}{\pgfqpoint{1.299404in}{1.379335in}}%
\pgfpathcurveto{\pgfqpoint{1.295285in}{1.383454in}}{\pgfqpoint{1.289699in}{1.385767in}}{\pgfqpoint{1.283875in}{1.385767in}}%
\pgfpathcurveto{\pgfqpoint{1.278051in}{1.385767in}}{\pgfqpoint{1.272465in}{1.383454in}}{\pgfqpoint{1.268347in}{1.379335in}}%
\pgfpathcurveto{\pgfqpoint{1.264229in}{1.375217in}}{\pgfqpoint{1.261915in}{1.369631in}}{\pgfqpoint{1.261915in}{1.363807in}}%
\pgfpathcurveto{\pgfqpoint{1.261915in}{1.357983in}}{\pgfqpoint{1.264229in}{1.352397in}}{\pgfqpoint{1.268347in}{1.348279in}}%
\pgfpathcurveto{\pgfqpoint{1.272465in}{1.344161in}}{\pgfqpoint{1.278051in}{1.341847in}}{\pgfqpoint{1.283875in}{1.341847in}}%
\pgfpathclose%
\pgfusepath{stroke,fill}%
\end{pgfscope}%
\begin{pgfscope}%
\pgfpathrectangle{\pgfqpoint{0.211875in}{0.211875in}}{\pgfqpoint{1.313625in}{1.279725in}}%
\pgfusepath{clip}%
\pgfsetbuttcap%
\pgfsetroundjoin%
\definecolor{currentfill}{rgb}{0.121569,0.466667,0.705882}%
\pgfsetfillcolor{currentfill}%
\pgfsetlinewidth{1.003750pt}%
\definecolor{currentstroke}{rgb}{0.121569,0.466667,0.705882}%
\pgfsetstrokecolor{currentstroke}%
\pgfsetdash{}{0pt}%
\pgfpathmoveto{\pgfqpoint{1.264128in}{1.336675in}}%
\pgfpathcurveto{\pgfqpoint{1.269952in}{1.336675in}}{\pgfqpoint{1.275538in}{1.338989in}}{\pgfqpoint{1.279656in}{1.343107in}}%
\pgfpathcurveto{\pgfqpoint{1.283774in}{1.347225in}}{\pgfqpoint{1.286088in}{1.352811in}}{\pgfqpoint{1.286088in}{1.358635in}}%
\pgfpathcurveto{\pgfqpoint{1.286088in}{1.364459in}}{\pgfqpoint{1.283774in}{1.370045in}}{\pgfqpoint{1.279656in}{1.374163in}}%
\pgfpathcurveto{\pgfqpoint{1.275538in}{1.378281in}}{\pgfqpoint{1.269952in}{1.380595in}}{\pgfqpoint{1.264128in}{1.380595in}}%
\pgfpathcurveto{\pgfqpoint{1.258304in}{1.380595in}}{\pgfqpoint{1.252717in}{1.378281in}}{\pgfqpoint{1.248599in}{1.374163in}}%
\pgfpathcurveto{\pgfqpoint{1.244481in}{1.370045in}}{\pgfqpoint{1.242167in}{1.364459in}}{\pgfqpoint{1.242167in}{1.358635in}}%
\pgfpathcurveto{\pgfqpoint{1.242167in}{1.352811in}}{\pgfqpoint{1.244481in}{1.347225in}}{\pgfqpoint{1.248599in}{1.343107in}}%
\pgfpathcurveto{\pgfqpoint{1.252717in}{1.338989in}}{\pgfqpoint{1.258304in}{1.336675in}}{\pgfqpoint{1.264128in}{1.336675in}}%
\pgfpathclose%
\pgfusepath{stroke,fill}%
\end{pgfscope}%
\begin{pgfscope}%
\pgfpathrectangle{\pgfqpoint{0.211875in}{0.211875in}}{\pgfqpoint{1.313625in}{1.279725in}}%
\pgfusepath{clip}%
\pgfsetbuttcap%
\pgfsetroundjoin%
\definecolor{currentfill}{rgb}{0.121569,0.466667,0.705882}%
\pgfsetfillcolor{currentfill}%
\pgfsetlinewidth{1.003750pt}%
\definecolor{currentstroke}{rgb}{0.121569,0.466667,0.705882}%
\pgfsetstrokecolor{currentstroke}%
\pgfsetdash{}{0pt}%
\pgfpathmoveto{\pgfqpoint{1.110376in}{0.830981in}}%
\pgfpathcurveto{\pgfqpoint{1.116200in}{0.830981in}}{\pgfqpoint{1.121786in}{0.833295in}}{\pgfqpoint{1.125904in}{0.837413in}}%
\pgfpathcurveto{\pgfqpoint{1.130022in}{0.841531in}}{\pgfqpoint{1.132336in}{0.847117in}}{\pgfqpoint{1.132336in}{0.852941in}}%
\pgfpathcurveto{\pgfqpoint{1.132336in}{0.858765in}}{\pgfqpoint{1.130022in}{0.864351in}}{\pgfqpoint{1.125904in}{0.868469in}}%
\pgfpathcurveto{\pgfqpoint{1.121786in}{0.872587in}}{\pgfqpoint{1.116200in}{0.874901in}}{\pgfqpoint{1.110376in}{0.874901in}}%
\pgfpathcurveto{\pgfqpoint{1.104552in}{0.874901in}}{\pgfqpoint{1.098966in}{0.872587in}}{\pgfqpoint{1.094848in}{0.868469in}}%
\pgfpathcurveto{\pgfqpoint{1.090730in}{0.864351in}}{\pgfqpoint{1.088416in}{0.858765in}}{\pgfqpoint{1.088416in}{0.852941in}}%
\pgfpathcurveto{\pgfqpoint{1.088416in}{0.847117in}}{\pgfqpoint{1.090730in}{0.841531in}}{\pgfqpoint{1.094848in}{0.837413in}}%
\pgfpathcurveto{\pgfqpoint{1.098966in}{0.833295in}}{\pgfqpoint{1.104552in}{0.830981in}}{\pgfqpoint{1.110376in}{0.830981in}}%
\pgfpathclose%
\pgfusepath{stroke,fill}%
\end{pgfscope}%
\begin{pgfscope}%
\pgfpathrectangle{\pgfqpoint{0.211875in}{0.211875in}}{\pgfqpoint{1.313625in}{1.279725in}}%
\pgfusepath{clip}%
\pgfsetbuttcap%
\pgfsetroundjoin%
\definecolor{currentfill}{rgb}{0.121569,0.466667,0.705882}%
\pgfsetfillcolor{currentfill}%
\pgfsetlinewidth{1.003750pt}%
\definecolor{currentstroke}{rgb}{0.121569,0.466667,0.705882}%
\pgfsetstrokecolor{currentstroke}%
\pgfsetdash{}{0pt}%
\pgfpathmoveto{\pgfqpoint{1.219761in}{0.656706in}}%
\pgfpathcurveto{\pgfqpoint{1.225585in}{0.656706in}}{\pgfqpoint{1.231171in}{0.659020in}}{\pgfqpoint{1.235289in}{0.663138in}}%
\pgfpathcurveto{\pgfqpoint{1.239407in}{0.667256in}}{\pgfqpoint{1.241721in}{0.672842in}}{\pgfqpoint{1.241721in}{0.678666in}}%
\pgfpathcurveto{\pgfqpoint{1.241721in}{0.684490in}}{\pgfqpoint{1.239407in}{0.690076in}}{\pgfqpoint{1.235289in}{0.694194in}}%
\pgfpathcurveto{\pgfqpoint{1.231171in}{0.698313in}}{\pgfqpoint{1.225585in}{0.700626in}}{\pgfqpoint{1.219761in}{0.700626in}}%
\pgfpathcurveto{\pgfqpoint{1.213937in}{0.700626in}}{\pgfqpoint{1.208351in}{0.698313in}}{\pgfqpoint{1.204233in}{0.694194in}}%
\pgfpathcurveto{\pgfqpoint{1.200114in}{0.690076in}}{\pgfqpoint{1.197801in}{0.684490in}}{\pgfqpoint{1.197801in}{0.678666in}}%
\pgfpathcurveto{\pgfqpoint{1.197801in}{0.672842in}}{\pgfqpoint{1.200114in}{0.667256in}}{\pgfqpoint{1.204233in}{0.663138in}}%
\pgfpathcurveto{\pgfqpoint{1.208351in}{0.659020in}}{\pgfqpoint{1.213937in}{0.656706in}}{\pgfqpoint{1.219761in}{0.656706in}}%
\pgfpathclose%
\pgfusepath{stroke,fill}%
\end{pgfscope}%
\begin{pgfscope}%
\pgfpathrectangle{\pgfqpoint{0.211875in}{0.211875in}}{\pgfqpoint{1.313625in}{1.279725in}}%
\pgfusepath{clip}%
\pgfsetbuttcap%
\pgfsetroundjoin%
\definecolor{currentfill}{rgb}{0.121569,0.466667,0.705882}%
\pgfsetfillcolor{currentfill}%
\pgfsetlinewidth{1.003750pt}%
\definecolor{currentstroke}{rgb}{0.121569,0.466667,0.705882}%
\pgfsetstrokecolor{currentstroke}%
\pgfsetdash{}{0pt}%
\pgfpathmoveto{\pgfqpoint{1.109836in}{0.820646in}}%
\pgfpathcurveto{\pgfqpoint{1.115660in}{0.820646in}}{\pgfqpoint{1.121246in}{0.822960in}}{\pgfqpoint{1.125364in}{0.827078in}}%
\pgfpathcurveto{\pgfqpoint{1.129482in}{0.831196in}}{\pgfqpoint{1.131796in}{0.836782in}}{\pgfqpoint{1.131796in}{0.842606in}}%
\pgfpathcurveto{\pgfqpoint{1.131796in}{0.848430in}}{\pgfqpoint{1.129482in}{0.854016in}}{\pgfqpoint{1.125364in}{0.858134in}}%
\pgfpathcurveto{\pgfqpoint{1.121246in}{0.862253in}}{\pgfqpoint{1.115660in}{0.864566in}}{\pgfqpoint{1.109836in}{0.864566in}}%
\pgfpathcurveto{\pgfqpoint{1.104012in}{0.864566in}}{\pgfqpoint{1.098426in}{0.862253in}}{\pgfqpoint{1.094308in}{0.858134in}}%
\pgfpathcurveto{\pgfqpoint{1.090189in}{0.854016in}}{\pgfqpoint{1.087876in}{0.848430in}}{\pgfqpoint{1.087876in}{0.842606in}}%
\pgfpathcurveto{\pgfqpoint{1.087876in}{0.836782in}}{\pgfqpoint{1.090189in}{0.831196in}}{\pgfqpoint{1.094308in}{0.827078in}}%
\pgfpathcurveto{\pgfqpoint{1.098426in}{0.822960in}}{\pgfqpoint{1.104012in}{0.820646in}}{\pgfqpoint{1.109836in}{0.820646in}}%
\pgfpathclose%
\pgfusepath{stroke,fill}%
\end{pgfscope}%
\begin{pgfscope}%
\pgfpathrectangle{\pgfqpoint{0.211875in}{0.211875in}}{\pgfqpoint{1.313625in}{1.279725in}}%
\pgfusepath{clip}%
\pgfsetbuttcap%
\pgfsetroundjoin%
\definecolor{currentfill}{rgb}{0.121569,0.466667,0.705882}%
\pgfsetfillcolor{currentfill}%
\pgfsetlinewidth{1.003750pt}%
\definecolor{currentstroke}{rgb}{0.121569,0.466667,0.705882}%
\pgfsetstrokecolor{currentstroke}%
\pgfsetdash{}{0pt}%
\pgfpathmoveto{\pgfqpoint{1.103529in}{0.796678in}}%
\pgfpathcurveto{\pgfqpoint{1.109353in}{0.796678in}}{\pgfqpoint{1.114939in}{0.798991in}}{\pgfqpoint{1.119057in}{0.803110in}}%
\pgfpathcurveto{\pgfqpoint{1.123175in}{0.807228in}}{\pgfqpoint{1.125489in}{0.812814in}}{\pgfqpoint{1.125489in}{0.818638in}}%
\pgfpathcurveto{\pgfqpoint{1.125489in}{0.824462in}}{\pgfqpoint{1.123175in}{0.830048in}}{\pgfqpoint{1.119057in}{0.834166in}}%
\pgfpathcurveto{\pgfqpoint{1.114939in}{0.838284in}}{\pgfqpoint{1.109353in}{0.840598in}}{\pgfqpoint{1.103529in}{0.840598in}}%
\pgfpathcurveto{\pgfqpoint{1.097705in}{0.840598in}}{\pgfqpoint{1.092119in}{0.838284in}}{\pgfqpoint{1.088000in}{0.834166in}}%
\pgfpathcurveto{\pgfqpoint{1.083882in}{0.830048in}}{\pgfqpoint{1.081568in}{0.824462in}}{\pgfqpoint{1.081568in}{0.818638in}}%
\pgfpathcurveto{\pgfqpoint{1.081568in}{0.812814in}}{\pgfqpoint{1.083882in}{0.807228in}}{\pgfqpoint{1.088000in}{0.803110in}}%
\pgfpathcurveto{\pgfqpoint{1.092119in}{0.798991in}}{\pgfqpoint{1.097705in}{0.796678in}}{\pgfqpoint{1.103529in}{0.796678in}}%
\pgfpathclose%
\pgfusepath{stroke,fill}%
\end{pgfscope}%
\begin{pgfscope}%
\pgfpathrectangle{\pgfqpoint{0.211875in}{0.211875in}}{\pgfqpoint{1.313625in}{1.279725in}}%
\pgfusepath{clip}%
\pgfsetbuttcap%
\pgfsetroundjoin%
\definecolor{currentfill}{rgb}{0.121569,0.466667,0.705882}%
\pgfsetfillcolor{currentfill}%
\pgfsetlinewidth{1.003750pt}%
\definecolor{currentstroke}{rgb}{0.121569,0.466667,0.705882}%
\pgfsetstrokecolor{currentstroke}%
\pgfsetdash{}{0pt}%
\pgfpathmoveto{\pgfqpoint{1.108339in}{0.809202in}}%
\pgfpathcurveto{\pgfqpoint{1.114163in}{0.809202in}}{\pgfqpoint{1.119749in}{0.811516in}}{\pgfqpoint{1.123867in}{0.815634in}}%
\pgfpathcurveto{\pgfqpoint{1.127985in}{0.819752in}}{\pgfqpoint{1.130299in}{0.825338in}}{\pgfqpoint{1.130299in}{0.831162in}}%
\pgfpathcurveto{\pgfqpoint{1.130299in}{0.836986in}}{\pgfqpoint{1.127985in}{0.842572in}}{\pgfqpoint{1.123867in}{0.846690in}}%
\pgfpathcurveto{\pgfqpoint{1.119749in}{0.850808in}}{\pgfqpoint{1.114163in}{0.853122in}}{\pgfqpoint{1.108339in}{0.853122in}}%
\pgfpathcurveto{\pgfqpoint{1.102515in}{0.853122in}}{\pgfqpoint{1.096929in}{0.850808in}}{\pgfqpoint{1.092811in}{0.846690in}}%
\pgfpathcurveto{\pgfqpoint{1.088693in}{0.842572in}}{\pgfqpoint{1.086379in}{0.836986in}}{\pgfqpoint{1.086379in}{0.831162in}}%
\pgfpathcurveto{\pgfqpoint{1.086379in}{0.825338in}}{\pgfqpoint{1.088693in}{0.819752in}}{\pgfqpoint{1.092811in}{0.815634in}}%
\pgfpathcurveto{\pgfqpoint{1.096929in}{0.811516in}}{\pgfqpoint{1.102515in}{0.809202in}}{\pgfqpoint{1.108339in}{0.809202in}}%
\pgfpathclose%
\pgfusepath{stroke,fill}%
\end{pgfscope}%
\begin{pgfscope}%
\pgfpathrectangle{\pgfqpoint{0.211875in}{0.211875in}}{\pgfqpoint{1.313625in}{1.279725in}}%
\pgfusepath{clip}%
\pgfsetbuttcap%
\pgfsetroundjoin%
\definecolor{currentfill}{rgb}{0.121569,0.466667,0.705882}%
\pgfsetfillcolor{currentfill}%
\pgfsetlinewidth{1.003750pt}%
\definecolor{currentstroke}{rgb}{0.121569,0.466667,0.705882}%
\pgfsetstrokecolor{currentstroke}%
\pgfsetdash{}{0pt}%
\pgfpathmoveto{\pgfqpoint{1.427407in}{0.801644in}}%
\pgfpathcurveto{\pgfqpoint{1.433231in}{0.801644in}}{\pgfqpoint{1.438817in}{0.803958in}}{\pgfqpoint{1.442935in}{0.808076in}}%
\pgfpathcurveto{\pgfqpoint{1.447054in}{0.812194in}}{\pgfqpoint{1.449367in}{0.817781in}}{\pgfqpoint{1.449367in}{0.823604in}}%
\pgfpathcurveto{\pgfqpoint{1.449367in}{0.829428in}}{\pgfqpoint{1.447054in}{0.835015in}}{\pgfqpoint{1.442935in}{0.839133in}}%
\pgfpathcurveto{\pgfqpoint{1.438817in}{0.843251in}}{\pgfqpoint{1.433231in}{0.845565in}}{\pgfqpoint{1.427407in}{0.845565in}}%
\pgfpathcurveto{\pgfqpoint{1.421583in}{0.845565in}}{\pgfqpoint{1.415997in}{0.843251in}}{\pgfqpoint{1.411879in}{0.839133in}}%
\pgfpathcurveto{\pgfqpoint{1.407761in}{0.835015in}}{\pgfqpoint{1.405447in}{0.829428in}}{\pgfqpoint{1.405447in}{0.823604in}}%
\pgfpathcurveto{\pgfqpoint{1.405447in}{0.817781in}}{\pgfqpoint{1.407761in}{0.812194in}}{\pgfqpoint{1.411879in}{0.808076in}}%
\pgfpathcurveto{\pgfqpoint{1.415997in}{0.803958in}}{\pgfqpoint{1.421583in}{0.801644in}}{\pgfqpoint{1.427407in}{0.801644in}}%
\pgfpathclose%
\pgfusepath{stroke,fill}%
\end{pgfscope}%
\begin{pgfscope}%
\pgfpathrectangle{\pgfqpoint{0.211875in}{0.211875in}}{\pgfqpoint{1.313625in}{1.279725in}}%
\pgfusepath{clip}%
\pgfsetbuttcap%
\pgfsetroundjoin%
\definecolor{currentfill}{rgb}{0.121569,0.466667,0.705882}%
\pgfsetfillcolor{currentfill}%
\pgfsetlinewidth{1.003750pt}%
\definecolor{currentstroke}{rgb}{0.121569,0.466667,0.705882}%
\pgfsetstrokecolor{currentstroke}%
\pgfsetdash{}{0pt}%
\pgfpathmoveto{\pgfqpoint{1.254988in}{1.362016in}}%
\pgfpathcurveto{\pgfqpoint{1.260812in}{1.362016in}}{\pgfqpoint{1.266398in}{1.364330in}}{\pgfqpoint{1.270516in}{1.368448in}}%
\pgfpathcurveto{\pgfqpoint{1.274634in}{1.372567in}}{\pgfqpoint{1.276948in}{1.378153in}}{\pgfqpoint{1.276948in}{1.383977in}}%
\pgfpathcurveto{\pgfqpoint{1.276948in}{1.389801in}}{\pgfqpoint{1.274634in}{1.395387in}}{\pgfqpoint{1.270516in}{1.399505in}}%
\pgfpathcurveto{\pgfqpoint{1.266398in}{1.403623in}}{\pgfqpoint{1.260812in}{1.405937in}}{\pgfqpoint{1.254988in}{1.405937in}}%
\pgfpathcurveto{\pgfqpoint{1.249164in}{1.405937in}}{\pgfqpoint{1.243578in}{1.403623in}}{\pgfqpoint{1.239460in}{1.399505in}}%
\pgfpathcurveto{\pgfqpoint{1.235342in}{1.395387in}}{\pgfqpoint{1.233028in}{1.389801in}}{\pgfqpoint{1.233028in}{1.383977in}}%
\pgfpathcurveto{\pgfqpoint{1.233028in}{1.378153in}}{\pgfqpoint{1.235342in}{1.372567in}}{\pgfqpoint{1.239460in}{1.368448in}}%
\pgfpathcurveto{\pgfqpoint{1.243578in}{1.364330in}}{\pgfqpoint{1.249164in}{1.362016in}}{\pgfqpoint{1.254988in}{1.362016in}}%
\pgfpathclose%
\pgfusepath{stroke,fill}%
\end{pgfscope}%
\begin{pgfscope}%
\pgfpathrectangle{\pgfqpoint{0.211875in}{0.211875in}}{\pgfqpoint{1.313625in}{1.279725in}}%
\pgfusepath{clip}%
\pgfsetbuttcap%
\pgfsetroundjoin%
\definecolor{currentfill}{rgb}{0.121569,0.466667,0.705882}%
\pgfsetfillcolor{currentfill}%
\pgfsetlinewidth{1.003750pt}%
\definecolor{currentstroke}{rgb}{0.121569,0.466667,0.705882}%
\pgfsetstrokecolor{currentstroke}%
\pgfsetdash{}{0pt}%
\pgfpathmoveto{\pgfqpoint{1.264312in}{1.329533in}}%
\pgfpathcurveto{\pgfqpoint{1.270136in}{1.329533in}}{\pgfqpoint{1.275722in}{1.331847in}}{\pgfqpoint{1.279840in}{1.335965in}}%
\pgfpathcurveto{\pgfqpoint{1.283958in}{1.340083in}}{\pgfqpoint{1.286272in}{1.345669in}}{\pgfqpoint{1.286272in}{1.351493in}}%
\pgfpathcurveto{\pgfqpoint{1.286272in}{1.357317in}}{\pgfqpoint{1.283958in}{1.362903in}}{\pgfqpoint{1.279840in}{1.367021in}}%
\pgfpathcurveto{\pgfqpoint{1.275722in}{1.371139in}}{\pgfqpoint{1.270136in}{1.373453in}}{\pgfqpoint{1.264312in}{1.373453in}}%
\pgfpathcurveto{\pgfqpoint{1.258488in}{1.373453in}}{\pgfqpoint{1.252902in}{1.371139in}}{\pgfqpoint{1.248784in}{1.367021in}}%
\pgfpathcurveto{\pgfqpoint{1.244665in}{1.362903in}}{\pgfqpoint{1.242352in}{1.357317in}}{\pgfqpoint{1.242352in}{1.351493in}}%
\pgfpathcurveto{\pgfqpoint{1.242352in}{1.345669in}}{\pgfqpoint{1.244665in}{1.340083in}}{\pgfqpoint{1.248784in}{1.335965in}}%
\pgfpathcurveto{\pgfqpoint{1.252902in}{1.331847in}}{\pgfqpoint{1.258488in}{1.329533in}}{\pgfqpoint{1.264312in}{1.329533in}}%
\pgfpathclose%
\pgfusepath{stroke,fill}%
\end{pgfscope}%
\begin{pgfscope}%
\pgfpathrectangle{\pgfqpoint{0.211875in}{0.211875in}}{\pgfqpoint{1.313625in}{1.279725in}}%
\pgfusepath{clip}%
\pgfsetbuttcap%
\pgfsetroundjoin%
\definecolor{currentfill}{rgb}{0.121569,0.466667,0.705882}%
\pgfsetfillcolor{currentfill}%
\pgfsetlinewidth{1.003750pt}%
\definecolor{currentstroke}{rgb}{0.121569,0.466667,0.705882}%
\pgfsetstrokecolor{currentstroke}%
\pgfsetdash{}{0pt}%
\pgfpathmoveto{\pgfqpoint{1.211570in}{0.677396in}}%
\pgfpathcurveto{\pgfqpoint{1.217394in}{0.677396in}}{\pgfqpoint{1.222980in}{0.679710in}}{\pgfqpoint{1.227098in}{0.683828in}}%
\pgfpathcurveto{\pgfqpoint{1.231216in}{0.687946in}}{\pgfqpoint{1.233530in}{0.693533in}}{\pgfqpoint{1.233530in}{0.699356in}}%
\pgfpathcurveto{\pgfqpoint{1.233530in}{0.705180in}}{\pgfqpoint{1.231216in}{0.710767in}}{\pgfqpoint{1.227098in}{0.714885in}}%
\pgfpathcurveto{\pgfqpoint{1.222980in}{0.719003in}}{\pgfqpoint{1.217394in}{0.721317in}}{\pgfqpoint{1.211570in}{0.721317in}}%
\pgfpathcurveto{\pgfqpoint{1.205746in}{0.721317in}}{\pgfqpoint{1.200160in}{0.719003in}}{\pgfqpoint{1.196042in}{0.714885in}}%
\pgfpathcurveto{\pgfqpoint{1.191923in}{0.710767in}}{\pgfqpoint{1.189610in}{0.705180in}}{\pgfqpoint{1.189610in}{0.699356in}}%
\pgfpathcurveto{\pgfqpoint{1.189610in}{0.693533in}}{\pgfqpoint{1.191923in}{0.687946in}}{\pgfqpoint{1.196042in}{0.683828in}}%
\pgfpathcurveto{\pgfqpoint{1.200160in}{0.679710in}}{\pgfqpoint{1.205746in}{0.677396in}}{\pgfqpoint{1.211570in}{0.677396in}}%
\pgfpathclose%
\pgfusepath{stroke,fill}%
\end{pgfscope}%
\begin{pgfscope}%
\pgfpathrectangle{\pgfqpoint{0.211875in}{0.211875in}}{\pgfqpoint{1.313625in}{1.279725in}}%
\pgfusepath{clip}%
\pgfsetbuttcap%
\pgfsetroundjoin%
\definecolor{currentfill}{rgb}{0.121569,0.466667,0.705882}%
\pgfsetfillcolor{currentfill}%
\pgfsetlinewidth{1.003750pt}%
\definecolor{currentstroke}{rgb}{0.121569,0.466667,0.705882}%
\pgfsetstrokecolor{currentstroke}%
\pgfsetdash{}{0pt}%
\pgfpathmoveto{\pgfqpoint{1.105604in}{0.824146in}}%
\pgfpathcurveto{\pgfqpoint{1.111427in}{0.824146in}}{\pgfqpoint{1.117014in}{0.826460in}}{\pgfqpoint{1.121132in}{0.830578in}}%
\pgfpathcurveto{\pgfqpoint{1.125250in}{0.834697in}}{\pgfqpoint{1.127564in}{0.840283in}}{\pgfqpoint{1.127564in}{0.846107in}}%
\pgfpathcurveto{\pgfqpoint{1.127564in}{0.851931in}}{\pgfqpoint{1.125250in}{0.857517in}}{\pgfqpoint{1.121132in}{0.861635in}}%
\pgfpathcurveto{\pgfqpoint{1.117014in}{0.865753in}}{\pgfqpoint{1.111427in}{0.868067in}}{\pgfqpoint{1.105604in}{0.868067in}}%
\pgfpathcurveto{\pgfqpoint{1.099780in}{0.868067in}}{\pgfqpoint{1.094193in}{0.865753in}}{\pgfqpoint{1.090075in}{0.861635in}}%
\pgfpathcurveto{\pgfqpoint{1.085957in}{0.857517in}}{\pgfqpoint{1.083643in}{0.851931in}}{\pgfqpoint{1.083643in}{0.846107in}}%
\pgfpathcurveto{\pgfqpoint{1.083643in}{0.840283in}}{\pgfqpoint{1.085957in}{0.834697in}}{\pgfqpoint{1.090075in}{0.830578in}}%
\pgfpathcurveto{\pgfqpoint{1.094193in}{0.826460in}}{\pgfqpoint{1.099780in}{0.824146in}}{\pgfqpoint{1.105604in}{0.824146in}}%
\pgfpathclose%
\pgfusepath{stroke,fill}%
\end{pgfscope}%
\begin{pgfscope}%
\pgfpathrectangle{\pgfqpoint{0.211875in}{0.211875in}}{\pgfqpoint{1.313625in}{1.279725in}}%
\pgfusepath{clip}%
\pgfsetbuttcap%
\pgfsetroundjoin%
\definecolor{currentfill}{rgb}{0.121569,0.466667,0.705882}%
\pgfsetfillcolor{currentfill}%
\pgfsetlinewidth{1.003750pt}%
\definecolor{currentstroke}{rgb}{0.121569,0.466667,0.705882}%
\pgfsetstrokecolor{currentstroke}%
\pgfsetdash{}{0pt}%
\pgfpathmoveto{\pgfqpoint{1.274454in}{1.349720in}}%
\pgfpathcurveto{\pgfqpoint{1.280278in}{1.349720in}}{\pgfqpoint{1.285864in}{1.352033in}}{\pgfqpoint{1.289982in}{1.356152in}}%
\pgfpathcurveto{\pgfqpoint{1.294100in}{1.360270in}}{\pgfqpoint{1.296414in}{1.365856in}}{\pgfqpoint{1.296414in}{1.371680in}}%
\pgfpathcurveto{\pgfqpoint{1.296414in}{1.377504in}}{\pgfqpoint{1.294100in}{1.383090in}}{\pgfqpoint{1.289982in}{1.387208in}}%
\pgfpathcurveto{\pgfqpoint{1.285864in}{1.391326in}}{\pgfqpoint{1.280278in}{1.393640in}}{\pgfqpoint{1.274454in}{1.393640in}}%
\pgfpathcurveto{\pgfqpoint{1.268630in}{1.393640in}}{\pgfqpoint{1.263044in}{1.391326in}}{\pgfqpoint{1.258926in}{1.387208in}}%
\pgfpathcurveto{\pgfqpoint{1.254807in}{1.383090in}}{\pgfqpoint{1.252494in}{1.377504in}}{\pgfqpoint{1.252494in}{1.371680in}}%
\pgfpathcurveto{\pgfqpoint{1.252494in}{1.365856in}}{\pgfqpoint{1.254807in}{1.360270in}}{\pgfqpoint{1.258926in}{1.356152in}}%
\pgfpathcurveto{\pgfqpoint{1.263044in}{1.352033in}}{\pgfqpoint{1.268630in}{1.349720in}}{\pgfqpoint{1.274454in}{1.349720in}}%
\pgfpathclose%
\pgfusepath{stroke,fill}%
\end{pgfscope}%
\begin{pgfscope}%
\pgfpathrectangle{\pgfqpoint{0.211875in}{0.211875in}}{\pgfqpoint{1.313625in}{1.279725in}}%
\pgfusepath{clip}%
\pgfsetbuttcap%
\pgfsetroundjoin%
\definecolor{currentfill}{rgb}{0.121569,0.466667,0.705882}%
\pgfsetfillcolor{currentfill}%
\pgfsetlinewidth{1.003750pt}%
\definecolor{currentstroke}{rgb}{0.121569,0.466667,0.705882}%
\pgfsetstrokecolor{currentstroke}%
\pgfsetdash{}{0pt}%
\pgfpathmoveto{\pgfqpoint{1.427963in}{0.676622in}}%
\pgfpathcurveto{\pgfqpoint{1.433787in}{0.676622in}}{\pgfqpoint{1.439373in}{0.678936in}}{\pgfqpoint{1.443491in}{0.683054in}}%
\pgfpathcurveto{\pgfqpoint{1.447609in}{0.687172in}}{\pgfqpoint{1.449923in}{0.692759in}}{\pgfqpoint{1.449923in}{0.698583in}}%
\pgfpathcurveto{\pgfqpoint{1.449923in}{0.704407in}}{\pgfqpoint{1.447609in}{0.709993in}}{\pgfqpoint{1.443491in}{0.714111in}}%
\pgfpathcurveto{\pgfqpoint{1.439373in}{0.718229in}}{\pgfqpoint{1.433787in}{0.720543in}}{\pgfqpoint{1.427963in}{0.720543in}}%
\pgfpathcurveto{\pgfqpoint{1.422139in}{0.720543in}}{\pgfqpoint{1.416553in}{0.718229in}}{\pgfqpoint{1.412435in}{0.714111in}}%
\pgfpathcurveto{\pgfqpoint{1.408316in}{0.709993in}}{\pgfqpoint{1.406003in}{0.704407in}}{\pgfqpoint{1.406003in}{0.698583in}}%
\pgfpathcurveto{\pgfqpoint{1.406003in}{0.692759in}}{\pgfqpoint{1.408316in}{0.687172in}}{\pgfqpoint{1.412435in}{0.683054in}}%
\pgfpathcurveto{\pgfqpoint{1.416553in}{0.678936in}}{\pgfqpoint{1.422139in}{0.676622in}}{\pgfqpoint{1.427963in}{0.676622in}}%
\pgfpathclose%
\pgfusepath{stroke,fill}%
\end{pgfscope}%
\begin{pgfscope}%
\pgfpathrectangle{\pgfqpoint{0.211875in}{0.211875in}}{\pgfqpoint{1.313625in}{1.279725in}}%
\pgfusepath{clip}%
\pgfsetbuttcap%
\pgfsetroundjoin%
\definecolor{currentfill}{rgb}{0.121569,0.466667,0.705882}%
\pgfsetfillcolor{currentfill}%
\pgfsetlinewidth{1.003750pt}%
\definecolor{currentstroke}{rgb}{0.121569,0.466667,0.705882}%
\pgfsetstrokecolor{currentstroke}%
\pgfsetdash{}{0pt}%
\pgfpathmoveto{\pgfqpoint{1.140262in}{0.843624in}}%
\pgfpathcurveto{\pgfqpoint{1.146086in}{0.843624in}}{\pgfqpoint{1.151672in}{0.845938in}}{\pgfqpoint{1.155790in}{0.850056in}}%
\pgfpathcurveto{\pgfqpoint{1.159909in}{0.854174in}}{\pgfqpoint{1.162222in}{0.859761in}}{\pgfqpoint{1.162222in}{0.865585in}}%
\pgfpathcurveto{\pgfqpoint{1.162222in}{0.871409in}}{\pgfqpoint{1.159909in}{0.876995in}}{\pgfqpoint{1.155790in}{0.881113in}}%
\pgfpathcurveto{\pgfqpoint{1.151672in}{0.885231in}}{\pgfqpoint{1.146086in}{0.887545in}}{\pgfqpoint{1.140262in}{0.887545in}}%
\pgfpathcurveto{\pgfqpoint{1.134438in}{0.887545in}}{\pgfqpoint{1.128852in}{0.885231in}}{\pgfqpoint{1.124734in}{0.881113in}}%
\pgfpathcurveto{\pgfqpoint{1.120616in}{0.876995in}}{\pgfqpoint{1.118302in}{0.871409in}}{\pgfqpoint{1.118302in}{0.865585in}}%
\pgfpathcurveto{\pgfqpoint{1.118302in}{0.859761in}}{\pgfqpoint{1.120616in}{0.854174in}}{\pgfqpoint{1.124734in}{0.850056in}}%
\pgfpathcurveto{\pgfqpoint{1.128852in}{0.845938in}}{\pgfqpoint{1.134438in}{0.843624in}}{\pgfqpoint{1.140262in}{0.843624in}}%
\pgfpathclose%
\pgfusepath{stroke,fill}%
\end{pgfscope}%
\begin{pgfscope}%
\pgfpathrectangle{\pgfqpoint{0.211875in}{0.211875in}}{\pgfqpoint{1.313625in}{1.279725in}}%
\pgfusepath{clip}%
\pgfsetbuttcap%
\pgfsetroundjoin%
\definecolor{currentfill}{rgb}{0.121569,0.466667,0.705882}%
\pgfsetfillcolor{currentfill}%
\pgfsetlinewidth{1.003750pt}%
\definecolor{currentstroke}{rgb}{0.121569,0.466667,0.705882}%
\pgfsetstrokecolor{currentstroke}%
\pgfsetdash{}{0pt}%
\pgfpathmoveto{\pgfqpoint{1.430404in}{1.133454in}}%
\pgfpathcurveto{\pgfqpoint{1.436228in}{1.133454in}}{\pgfqpoint{1.441814in}{1.135768in}}{\pgfqpoint{1.445933in}{1.139886in}}%
\pgfpathcurveto{\pgfqpoint{1.450051in}{1.144004in}}{\pgfqpoint{1.452365in}{1.149590in}}{\pgfqpoint{1.452365in}{1.155414in}}%
\pgfpathcurveto{\pgfqpoint{1.452365in}{1.161238in}}{\pgfqpoint{1.450051in}{1.166824in}}{\pgfqpoint{1.445933in}{1.170943in}}%
\pgfpathcurveto{\pgfqpoint{1.441814in}{1.175061in}}{\pgfqpoint{1.436228in}{1.177375in}}{\pgfqpoint{1.430404in}{1.177375in}}%
\pgfpathcurveto{\pgfqpoint{1.424580in}{1.177375in}}{\pgfqpoint{1.418994in}{1.175061in}}{\pgfqpoint{1.414876in}{1.170943in}}%
\pgfpathcurveto{\pgfqpoint{1.410758in}{1.166824in}}{\pgfqpoint{1.408444in}{1.161238in}}{\pgfqpoint{1.408444in}{1.155414in}}%
\pgfpathcurveto{\pgfqpoint{1.408444in}{1.149590in}}{\pgfqpoint{1.410758in}{1.144004in}}{\pgfqpoint{1.414876in}{1.139886in}}%
\pgfpathcurveto{\pgfqpoint{1.418994in}{1.135768in}}{\pgfqpoint{1.424580in}{1.133454in}}{\pgfqpoint{1.430404in}{1.133454in}}%
\pgfpathclose%
\pgfusepath{stroke,fill}%
\end{pgfscope}%
\begin{pgfscope}%
\pgfpathrectangle{\pgfqpoint{0.211875in}{0.211875in}}{\pgfqpoint{1.313625in}{1.279725in}}%
\pgfusepath{clip}%
\pgfsetbuttcap%
\pgfsetroundjoin%
\definecolor{currentfill}{rgb}{0.121569,0.466667,0.705882}%
\pgfsetfillcolor{currentfill}%
\pgfsetlinewidth{1.003750pt}%
\definecolor{currentstroke}{rgb}{0.121569,0.466667,0.705882}%
\pgfsetstrokecolor{currentstroke}%
\pgfsetdash{}{0pt}%
\pgfpathmoveto{\pgfqpoint{1.205298in}{0.668189in}}%
\pgfpathcurveto{\pgfqpoint{1.211122in}{0.668189in}}{\pgfqpoint{1.216708in}{0.670502in}}{\pgfqpoint{1.220826in}{0.674621in}}%
\pgfpathcurveto{\pgfqpoint{1.224944in}{0.678739in}}{\pgfqpoint{1.227258in}{0.684325in}}{\pgfqpoint{1.227258in}{0.690149in}}%
\pgfpathcurveto{\pgfqpoint{1.227258in}{0.695973in}}{\pgfqpoint{1.224944in}{0.701559in}}{\pgfqpoint{1.220826in}{0.705677in}}%
\pgfpathcurveto{\pgfqpoint{1.216708in}{0.709795in}}{\pgfqpoint{1.211122in}{0.712109in}}{\pgfqpoint{1.205298in}{0.712109in}}%
\pgfpathcurveto{\pgfqpoint{1.199474in}{0.712109in}}{\pgfqpoint{1.193888in}{0.709795in}}{\pgfqpoint{1.189770in}{0.705677in}}%
\pgfpathcurveto{\pgfqpoint{1.185652in}{0.701559in}}{\pgfqpoint{1.183338in}{0.695973in}}{\pgfqpoint{1.183338in}{0.690149in}}%
\pgfpathcurveto{\pgfqpoint{1.183338in}{0.684325in}}{\pgfqpoint{1.185652in}{0.678739in}}{\pgfqpoint{1.189770in}{0.674621in}}%
\pgfpathcurveto{\pgfqpoint{1.193888in}{0.670502in}}{\pgfqpoint{1.199474in}{0.668189in}}{\pgfqpoint{1.205298in}{0.668189in}}%
\pgfpathclose%
\pgfusepath{stroke,fill}%
\end{pgfscope}%
\begin{pgfscope}%
\pgfpathrectangle{\pgfqpoint{0.211875in}{0.211875in}}{\pgfqpoint{1.313625in}{1.279725in}}%
\pgfusepath{clip}%
\pgfsetbuttcap%
\pgfsetroundjoin%
\definecolor{currentfill}{rgb}{0.121569,0.466667,0.705882}%
\pgfsetfillcolor{currentfill}%
\pgfsetlinewidth{1.003750pt}%
\definecolor{currentstroke}{rgb}{0.121569,0.466667,0.705882}%
\pgfsetstrokecolor{currentstroke}%
\pgfsetdash{}{0pt}%
\pgfpathmoveto{\pgfqpoint{1.268322in}{1.346873in}}%
\pgfpathcurveto{\pgfqpoint{1.274146in}{1.346873in}}{\pgfqpoint{1.279732in}{1.349187in}}{\pgfqpoint{1.283850in}{1.353305in}}%
\pgfpathcurveto{\pgfqpoint{1.287968in}{1.357423in}}{\pgfqpoint{1.290282in}{1.363009in}}{\pgfqpoint{1.290282in}{1.368833in}}%
\pgfpathcurveto{\pgfqpoint{1.290282in}{1.374657in}}{\pgfqpoint{1.287968in}{1.380243in}}{\pgfqpoint{1.283850in}{1.384361in}}%
\pgfpathcurveto{\pgfqpoint{1.279732in}{1.388479in}}{\pgfqpoint{1.274146in}{1.390793in}}{\pgfqpoint{1.268322in}{1.390793in}}%
\pgfpathcurveto{\pgfqpoint{1.262498in}{1.390793in}}{\pgfqpoint{1.256912in}{1.388479in}}{\pgfqpoint{1.252793in}{1.384361in}}%
\pgfpathcurveto{\pgfqpoint{1.248675in}{1.380243in}}{\pgfqpoint{1.246361in}{1.374657in}}{\pgfqpoint{1.246361in}{1.368833in}}%
\pgfpathcurveto{\pgfqpoint{1.246361in}{1.363009in}}{\pgfqpoint{1.248675in}{1.357423in}}{\pgfqpoint{1.252793in}{1.353305in}}%
\pgfpathcurveto{\pgfqpoint{1.256912in}{1.349187in}}{\pgfqpoint{1.262498in}{1.346873in}}{\pgfqpoint{1.268322in}{1.346873in}}%
\pgfpathclose%
\pgfusepath{stroke,fill}%
\end{pgfscope}%
\begin{pgfscope}%
\pgfpathrectangle{\pgfqpoint{0.211875in}{0.211875in}}{\pgfqpoint{1.313625in}{1.279725in}}%
\pgfusepath{clip}%
\pgfsetbuttcap%
\pgfsetroundjoin%
\definecolor{currentfill}{rgb}{0.121569,0.466667,0.705882}%
\pgfsetfillcolor{currentfill}%
\pgfsetlinewidth{1.003750pt}%
\definecolor{currentstroke}{rgb}{0.121569,0.466667,0.705882}%
\pgfsetstrokecolor{currentstroke}%
\pgfsetdash{}{0pt}%
\pgfpathmoveto{\pgfqpoint{1.314205in}{0.675090in}}%
\pgfpathcurveto{\pgfqpoint{1.320029in}{0.675090in}}{\pgfqpoint{1.325616in}{0.677404in}}{\pgfqpoint{1.329734in}{0.681522in}}%
\pgfpathcurveto{\pgfqpoint{1.333852in}{0.685641in}}{\pgfqpoint{1.336166in}{0.691227in}}{\pgfqpoint{1.336166in}{0.697051in}}%
\pgfpathcurveto{\pgfqpoint{1.336166in}{0.702875in}}{\pgfqpoint{1.333852in}{0.708461in}}{\pgfqpoint{1.329734in}{0.712579in}}%
\pgfpathcurveto{\pgfqpoint{1.325616in}{0.716697in}}{\pgfqpoint{1.320029in}{0.719011in}}{\pgfqpoint{1.314205in}{0.719011in}}%
\pgfpathcurveto{\pgfqpoint{1.308381in}{0.719011in}}{\pgfqpoint{1.302795in}{0.716697in}}{\pgfqpoint{1.298677in}{0.712579in}}%
\pgfpathcurveto{\pgfqpoint{1.294559in}{0.708461in}}{\pgfqpoint{1.292245in}{0.702875in}}{\pgfqpoint{1.292245in}{0.697051in}}%
\pgfpathcurveto{\pgfqpoint{1.292245in}{0.691227in}}{\pgfqpoint{1.294559in}{0.685641in}}{\pgfqpoint{1.298677in}{0.681522in}}%
\pgfpathcurveto{\pgfqpoint{1.302795in}{0.677404in}}{\pgfqpoint{1.308381in}{0.675090in}}{\pgfqpoint{1.314205in}{0.675090in}}%
\pgfpathclose%
\pgfusepath{stroke,fill}%
\end{pgfscope}%
\begin{pgfscope}%
\pgfpathrectangle{\pgfqpoint{0.211875in}{0.211875in}}{\pgfqpoint{1.313625in}{1.279725in}}%
\pgfusepath{clip}%
\pgfsetbuttcap%
\pgfsetroundjoin%
\definecolor{currentfill}{rgb}{0.121569,0.466667,0.705882}%
\pgfsetfillcolor{currentfill}%
\pgfsetlinewidth{1.003750pt}%
\definecolor{currentstroke}{rgb}{0.121569,0.466667,0.705882}%
\pgfsetstrokecolor{currentstroke}%
\pgfsetdash{}{0pt}%
\pgfpathmoveto{\pgfqpoint{1.417177in}{1.116680in}}%
\pgfpathcurveto{\pgfqpoint{1.423000in}{1.116680in}}{\pgfqpoint{1.428587in}{1.118994in}}{\pgfqpoint{1.432705in}{1.123112in}}%
\pgfpathcurveto{\pgfqpoint{1.436823in}{1.127230in}}{\pgfqpoint{1.439137in}{1.132816in}}{\pgfqpoint{1.439137in}{1.138640in}}%
\pgfpathcurveto{\pgfqpoint{1.439137in}{1.144464in}}{\pgfqpoint{1.436823in}{1.150050in}}{\pgfqpoint{1.432705in}{1.154168in}}%
\pgfpathcurveto{\pgfqpoint{1.428587in}{1.158287in}}{\pgfqpoint{1.423000in}{1.160600in}}{\pgfqpoint{1.417177in}{1.160600in}}%
\pgfpathcurveto{\pgfqpoint{1.411353in}{1.160600in}}{\pgfqpoint{1.405766in}{1.158287in}}{\pgfqpoint{1.401648in}{1.154168in}}%
\pgfpathcurveto{\pgfqpoint{1.397530in}{1.150050in}}{\pgfqpoint{1.395216in}{1.144464in}}{\pgfqpoint{1.395216in}{1.138640in}}%
\pgfpathcurveto{\pgfqpoint{1.395216in}{1.132816in}}{\pgfqpoint{1.397530in}{1.127230in}}{\pgfqpoint{1.401648in}{1.123112in}}%
\pgfpathcurveto{\pgfqpoint{1.405766in}{1.118994in}}{\pgfqpoint{1.411353in}{1.116680in}}{\pgfqpoint{1.417177in}{1.116680in}}%
\pgfpathclose%
\pgfusepath{stroke,fill}%
\end{pgfscope}%
\begin{pgfscope}%
\pgfpathrectangle{\pgfqpoint{0.211875in}{0.211875in}}{\pgfqpoint{1.313625in}{1.279725in}}%
\pgfusepath{clip}%
\pgfsetbuttcap%
\pgfsetroundjoin%
\definecolor{currentfill}{rgb}{0.121569,0.466667,0.705882}%
\pgfsetfillcolor{currentfill}%
\pgfsetlinewidth{1.003750pt}%
\definecolor{currentstroke}{rgb}{0.121569,0.466667,0.705882}%
\pgfsetstrokecolor{currentstroke}%
\pgfsetdash{}{0pt}%
\pgfpathmoveto{\pgfqpoint{1.108478in}{0.811091in}}%
\pgfpathcurveto{\pgfqpoint{1.114302in}{0.811091in}}{\pgfqpoint{1.119889in}{0.813405in}}{\pgfqpoint{1.124007in}{0.817523in}}%
\pgfpathcurveto{\pgfqpoint{1.128125in}{0.821642in}}{\pgfqpoint{1.130439in}{0.827228in}}{\pgfqpoint{1.130439in}{0.833052in}}%
\pgfpathcurveto{\pgfqpoint{1.130439in}{0.838876in}}{\pgfqpoint{1.128125in}{0.844462in}}{\pgfqpoint{1.124007in}{0.848580in}}%
\pgfpathcurveto{\pgfqpoint{1.119889in}{0.852698in}}{\pgfqpoint{1.114302in}{0.855012in}}{\pgfqpoint{1.108478in}{0.855012in}}%
\pgfpathcurveto{\pgfqpoint{1.102655in}{0.855012in}}{\pgfqpoint{1.097068in}{0.852698in}}{\pgfqpoint{1.092950in}{0.848580in}}%
\pgfpathcurveto{\pgfqpoint{1.088832in}{0.844462in}}{\pgfqpoint{1.086518in}{0.838876in}}{\pgfqpoint{1.086518in}{0.833052in}}%
\pgfpathcurveto{\pgfqpoint{1.086518in}{0.827228in}}{\pgfqpoint{1.088832in}{0.821642in}}{\pgfqpoint{1.092950in}{0.817523in}}%
\pgfpathcurveto{\pgfqpoint{1.097068in}{0.813405in}}{\pgfqpoint{1.102655in}{0.811091in}}{\pgfqpoint{1.108478in}{0.811091in}}%
\pgfpathclose%
\pgfusepath{stroke,fill}%
\end{pgfscope}%
\begin{pgfscope}%
\pgfpathrectangle{\pgfqpoint{0.211875in}{0.211875in}}{\pgfqpoint{1.313625in}{1.279725in}}%
\pgfusepath{clip}%
\pgfsetbuttcap%
\pgfsetroundjoin%
\definecolor{currentfill}{rgb}{0.121569,0.466667,0.705882}%
\pgfsetfillcolor{currentfill}%
\pgfsetlinewidth{1.003750pt}%
\definecolor{currentstroke}{rgb}{0.121569,0.466667,0.705882}%
\pgfsetstrokecolor{currentstroke}%
\pgfsetdash{}{0pt}%
\pgfpathmoveto{\pgfqpoint{1.222661in}{0.661765in}}%
\pgfpathcurveto{\pgfqpoint{1.228485in}{0.661765in}}{\pgfqpoint{1.234071in}{0.664078in}}{\pgfqpoint{1.238189in}{0.668197in}}%
\pgfpathcurveto{\pgfqpoint{1.242307in}{0.672315in}}{\pgfqpoint{1.244621in}{0.677901in}}{\pgfqpoint{1.244621in}{0.683725in}}%
\pgfpathcurveto{\pgfqpoint{1.244621in}{0.689549in}}{\pgfqpoint{1.242307in}{0.695135in}}{\pgfqpoint{1.238189in}{0.699253in}}%
\pgfpathcurveto{\pgfqpoint{1.234071in}{0.703371in}}{\pgfqpoint{1.228485in}{0.705685in}}{\pgfqpoint{1.222661in}{0.705685in}}%
\pgfpathcurveto{\pgfqpoint{1.216837in}{0.705685in}}{\pgfqpoint{1.211251in}{0.703371in}}{\pgfqpoint{1.207133in}{0.699253in}}%
\pgfpathcurveto{\pgfqpoint{1.203014in}{0.695135in}}{\pgfqpoint{1.200701in}{0.689549in}}{\pgfqpoint{1.200701in}{0.683725in}}%
\pgfpathcurveto{\pgfqpoint{1.200701in}{0.677901in}}{\pgfqpoint{1.203014in}{0.672315in}}{\pgfqpoint{1.207133in}{0.668197in}}%
\pgfpathcurveto{\pgfqpoint{1.211251in}{0.664078in}}{\pgfqpoint{1.216837in}{0.661765in}}{\pgfqpoint{1.222661in}{0.661765in}}%
\pgfpathclose%
\pgfusepath{stroke,fill}%
\end{pgfscope}%
\begin{pgfscope}%
\pgfpathrectangle{\pgfqpoint{0.211875in}{0.211875in}}{\pgfqpoint{1.313625in}{1.279725in}}%
\pgfusepath{clip}%
\pgfsetbuttcap%
\pgfsetroundjoin%
\definecolor{currentfill}{rgb}{0.121569,0.466667,0.705882}%
\pgfsetfillcolor{currentfill}%
\pgfsetlinewidth{1.003750pt}%
\definecolor{currentstroke}{rgb}{0.121569,0.466667,0.705882}%
\pgfsetstrokecolor{currentstroke}%
\pgfsetdash{}{0pt}%
\pgfpathmoveto{\pgfqpoint{1.103685in}{0.824149in}}%
\pgfpathcurveto{\pgfqpoint{1.109509in}{0.824149in}}{\pgfqpoint{1.115095in}{0.826463in}}{\pgfqpoint{1.119213in}{0.830581in}}%
\pgfpathcurveto{\pgfqpoint{1.123331in}{0.834699in}}{\pgfqpoint{1.125645in}{0.840285in}}{\pgfqpoint{1.125645in}{0.846109in}}%
\pgfpathcurveto{\pgfqpoint{1.125645in}{0.851933in}}{\pgfqpoint{1.123331in}{0.857519in}}{\pgfqpoint{1.119213in}{0.861637in}}%
\pgfpathcurveto{\pgfqpoint{1.115095in}{0.865755in}}{\pgfqpoint{1.109509in}{0.868069in}}{\pgfqpoint{1.103685in}{0.868069in}}%
\pgfpathcurveto{\pgfqpoint{1.097861in}{0.868069in}}{\pgfqpoint{1.092275in}{0.865755in}}{\pgfqpoint{1.088157in}{0.861637in}}%
\pgfpathcurveto{\pgfqpoint{1.084039in}{0.857519in}}{\pgfqpoint{1.081725in}{0.851933in}}{\pgfqpoint{1.081725in}{0.846109in}}%
\pgfpathcurveto{\pgfqpoint{1.081725in}{0.840285in}}{\pgfqpoint{1.084039in}{0.834699in}}{\pgfqpoint{1.088157in}{0.830581in}}%
\pgfpathcurveto{\pgfqpoint{1.092275in}{0.826463in}}{\pgfqpoint{1.097861in}{0.824149in}}{\pgfqpoint{1.103685in}{0.824149in}}%
\pgfpathclose%
\pgfusepath{stroke,fill}%
\end{pgfscope}%
\begin{pgfscope}%
\pgfpathrectangle{\pgfqpoint{0.211875in}{0.211875in}}{\pgfqpoint{1.313625in}{1.279725in}}%
\pgfusepath{clip}%
\pgfsetbuttcap%
\pgfsetroundjoin%
\definecolor{currentfill}{rgb}{0.121569,0.466667,0.705882}%
\pgfsetfillcolor{currentfill}%
\pgfsetlinewidth{1.003750pt}%
\definecolor{currentstroke}{rgb}{0.121569,0.466667,0.705882}%
\pgfsetstrokecolor{currentstroke}%
\pgfsetdash{}{0pt}%
\pgfpathmoveto{\pgfqpoint{1.225596in}{0.660490in}}%
\pgfpathcurveto{\pgfqpoint{1.231420in}{0.660490in}}{\pgfqpoint{1.237006in}{0.662804in}}{\pgfqpoint{1.241124in}{0.666922in}}%
\pgfpathcurveto{\pgfqpoint{1.245242in}{0.671040in}}{\pgfqpoint{1.247556in}{0.676626in}}{\pgfqpoint{1.247556in}{0.682450in}}%
\pgfpathcurveto{\pgfqpoint{1.247556in}{0.688274in}}{\pgfqpoint{1.245242in}{0.693860in}}{\pgfqpoint{1.241124in}{0.697978in}}%
\pgfpathcurveto{\pgfqpoint{1.237006in}{0.702096in}}{\pgfqpoint{1.231420in}{0.704410in}}{\pgfqpoint{1.225596in}{0.704410in}}%
\pgfpathcurveto{\pgfqpoint{1.219772in}{0.704410in}}{\pgfqpoint{1.214186in}{0.702096in}}{\pgfqpoint{1.210068in}{0.697978in}}%
\pgfpathcurveto{\pgfqpoint{1.205949in}{0.693860in}}{\pgfqpoint{1.203636in}{0.688274in}}{\pgfqpoint{1.203636in}{0.682450in}}%
\pgfpathcurveto{\pgfqpoint{1.203636in}{0.676626in}}{\pgfqpoint{1.205949in}{0.671040in}}{\pgfqpoint{1.210068in}{0.666922in}}%
\pgfpathcurveto{\pgfqpoint{1.214186in}{0.662804in}}{\pgfqpoint{1.219772in}{0.660490in}}{\pgfqpoint{1.225596in}{0.660490in}}%
\pgfpathclose%
\pgfusepath{stroke,fill}%
\end{pgfscope}%
\begin{pgfscope}%
\pgfpathrectangle{\pgfqpoint{0.211875in}{0.211875in}}{\pgfqpoint{1.313625in}{1.279725in}}%
\pgfusepath{clip}%
\pgfsetbuttcap%
\pgfsetroundjoin%
\definecolor{currentfill}{rgb}{0.121569,0.466667,0.705882}%
\pgfsetfillcolor{currentfill}%
\pgfsetlinewidth{1.003750pt}%
\definecolor{currentstroke}{rgb}{0.121569,0.466667,0.705882}%
\pgfsetstrokecolor{currentstroke}%
\pgfsetdash{}{0pt}%
\pgfpathmoveto{\pgfqpoint{1.265383in}{1.363424in}}%
\pgfpathcurveto{\pgfqpoint{1.271207in}{1.363424in}}{\pgfqpoint{1.276793in}{1.365738in}}{\pgfqpoint{1.280911in}{1.369856in}}%
\pgfpathcurveto{\pgfqpoint{1.285030in}{1.373974in}}{\pgfqpoint{1.287343in}{1.379560in}}{\pgfqpoint{1.287343in}{1.385384in}}%
\pgfpathcurveto{\pgfqpoint{1.287343in}{1.391208in}}{\pgfqpoint{1.285030in}{1.396794in}}{\pgfqpoint{1.280911in}{1.400912in}}%
\pgfpathcurveto{\pgfqpoint{1.276793in}{1.405030in}}{\pgfqpoint{1.271207in}{1.407344in}}{\pgfqpoint{1.265383in}{1.407344in}}%
\pgfpathcurveto{\pgfqpoint{1.259559in}{1.407344in}}{\pgfqpoint{1.253973in}{1.405030in}}{\pgfqpoint{1.249855in}{1.400912in}}%
\pgfpathcurveto{\pgfqpoint{1.245737in}{1.396794in}}{\pgfqpoint{1.243423in}{1.391208in}}{\pgfqpoint{1.243423in}{1.385384in}}%
\pgfpathcurveto{\pgfqpoint{1.243423in}{1.379560in}}{\pgfqpoint{1.245737in}{1.373974in}}{\pgfqpoint{1.249855in}{1.369856in}}%
\pgfpathcurveto{\pgfqpoint{1.253973in}{1.365738in}}{\pgfqpoint{1.259559in}{1.363424in}}{\pgfqpoint{1.265383in}{1.363424in}}%
\pgfpathclose%
\pgfusepath{stroke,fill}%
\end{pgfscope}%
\begin{pgfscope}%
\pgfpathrectangle{\pgfqpoint{0.211875in}{0.211875in}}{\pgfqpoint{1.313625in}{1.279725in}}%
\pgfusepath{clip}%
\pgfsetbuttcap%
\pgfsetroundjoin%
\definecolor{currentfill}{rgb}{0.121569,0.466667,0.705882}%
\pgfsetfillcolor{currentfill}%
\pgfsetlinewidth{1.003750pt}%
\definecolor{currentstroke}{rgb}{0.121569,0.466667,0.705882}%
\pgfsetstrokecolor{currentstroke}%
\pgfsetdash{}{0pt}%
\pgfpathmoveto{\pgfqpoint{1.319873in}{0.670779in}}%
\pgfpathcurveto{\pgfqpoint{1.325697in}{0.670779in}}{\pgfqpoint{1.331283in}{0.673093in}}{\pgfqpoint{1.335402in}{0.677211in}}%
\pgfpathcurveto{\pgfqpoint{1.339520in}{0.681329in}}{\pgfqpoint{1.341834in}{0.686915in}}{\pgfqpoint{1.341834in}{0.692739in}}%
\pgfpathcurveto{\pgfqpoint{1.341834in}{0.698563in}}{\pgfqpoint{1.339520in}{0.704149in}}{\pgfqpoint{1.335402in}{0.708267in}}%
\pgfpathcurveto{\pgfqpoint{1.331283in}{0.712386in}}{\pgfqpoint{1.325697in}{0.714699in}}{\pgfqpoint{1.319873in}{0.714699in}}%
\pgfpathcurveto{\pgfqpoint{1.314049in}{0.714699in}}{\pgfqpoint{1.308463in}{0.712386in}}{\pgfqpoint{1.304345in}{0.708267in}}%
\pgfpathcurveto{\pgfqpoint{1.300227in}{0.704149in}}{\pgfqpoint{1.297913in}{0.698563in}}{\pgfqpoint{1.297913in}{0.692739in}}%
\pgfpathcurveto{\pgfqpoint{1.297913in}{0.686915in}}{\pgfqpoint{1.300227in}{0.681329in}}{\pgfqpoint{1.304345in}{0.677211in}}%
\pgfpathcurveto{\pgfqpoint{1.308463in}{0.673093in}}{\pgfqpoint{1.314049in}{0.670779in}}{\pgfqpoint{1.319873in}{0.670779in}}%
\pgfpathclose%
\pgfusepath{stroke,fill}%
\end{pgfscope}%
\begin{pgfscope}%
\pgfpathrectangle{\pgfqpoint{0.211875in}{0.211875in}}{\pgfqpoint{1.313625in}{1.279725in}}%
\pgfusepath{clip}%
\pgfsetbuttcap%
\pgfsetroundjoin%
\definecolor{currentfill}{rgb}{0.121569,0.466667,0.705882}%
\pgfsetfillcolor{currentfill}%
\pgfsetlinewidth{1.003750pt}%
\definecolor{currentstroke}{rgb}{0.121569,0.466667,0.705882}%
\pgfsetstrokecolor{currentstroke}%
\pgfsetdash{}{0pt}%
\pgfpathmoveto{\pgfqpoint{1.103016in}{0.836207in}}%
\pgfpathcurveto{\pgfqpoint{1.108840in}{0.836207in}}{\pgfqpoint{1.114427in}{0.838521in}}{\pgfqpoint{1.118545in}{0.842639in}}%
\pgfpathcurveto{\pgfqpoint{1.122663in}{0.846757in}}{\pgfqpoint{1.124977in}{0.852343in}}{\pgfqpoint{1.124977in}{0.858167in}}%
\pgfpathcurveto{\pgfqpoint{1.124977in}{0.863991in}}{\pgfqpoint{1.122663in}{0.869577in}}{\pgfqpoint{1.118545in}{0.873695in}}%
\pgfpathcurveto{\pgfqpoint{1.114427in}{0.877813in}}{\pgfqpoint{1.108840in}{0.880127in}}{\pgfqpoint{1.103016in}{0.880127in}}%
\pgfpathcurveto{\pgfqpoint{1.097192in}{0.880127in}}{\pgfqpoint{1.091606in}{0.877813in}}{\pgfqpoint{1.087488in}{0.873695in}}%
\pgfpathcurveto{\pgfqpoint{1.083370in}{0.869577in}}{\pgfqpoint{1.081056in}{0.863991in}}{\pgfqpoint{1.081056in}{0.858167in}}%
\pgfpathcurveto{\pgfqpoint{1.081056in}{0.852343in}}{\pgfqpoint{1.083370in}{0.846757in}}{\pgfqpoint{1.087488in}{0.842639in}}%
\pgfpathcurveto{\pgfqpoint{1.091606in}{0.838521in}}{\pgfqpoint{1.097192in}{0.836207in}}{\pgfqpoint{1.103016in}{0.836207in}}%
\pgfpathclose%
\pgfusepath{stroke,fill}%
\end{pgfscope}%
\begin{pgfscope}%
\pgfpathrectangle{\pgfqpoint{0.211875in}{0.211875in}}{\pgfqpoint{1.313625in}{1.279725in}}%
\pgfusepath{clip}%
\pgfsetbuttcap%
\pgfsetroundjoin%
\definecolor{currentfill}{rgb}{0.121569,0.466667,0.705882}%
\pgfsetfillcolor{currentfill}%
\pgfsetlinewidth{1.003750pt}%
\definecolor{currentstroke}{rgb}{0.121569,0.466667,0.705882}%
\pgfsetstrokecolor{currentstroke}%
\pgfsetdash{}{0pt}%
\pgfpathmoveto{\pgfqpoint{0.511380in}{0.821354in}}%
\pgfpathcurveto{\pgfqpoint{0.517204in}{0.821354in}}{\pgfqpoint{0.522790in}{0.823668in}}{\pgfqpoint{0.526908in}{0.827786in}}%
\pgfpathcurveto{\pgfqpoint{0.531026in}{0.831904in}}{\pgfqpoint{0.533340in}{0.837490in}}{\pgfqpoint{0.533340in}{0.843314in}}%
\pgfpathcurveto{\pgfqpoint{0.533340in}{0.849138in}}{\pgfqpoint{0.531026in}{0.854724in}}{\pgfqpoint{0.526908in}{0.858842in}}%
\pgfpathcurveto{\pgfqpoint{0.522790in}{0.862961in}}{\pgfqpoint{0.517204in}{0.865274in}}{\pgfqpoint{0.511380in}{0.865274in}}%
\pgfpathcurveto{\pgfqpoint{0.505556in}{0.865274in}}{\pgfqpoint{0.499970in}{0.862961in}}{\pgfqpoint{0.495852in}{0.858842in}}%
\pgfpathcurveto{\pgfqpoint{0.491733in}{0.854724in}}{\pgfqpoint{0.489420in}{0.849138in}}{\pgfqpoint{0.489420in}{0.843314in}}%
\pgfpathcurveto{\pgfqpoint{0.489420in}{0.837490in}}{\pgfqpoint{0.491733in}{0.831904in}}{\pgfqpoint{0.495852in}{0.827786in}}%
\pgfpathcurveto{\pgfqpoint{0.499970in}{0.823668in}}{\pgfqpoint{0.505556in}{0.821354in}}{\pgfqpoint{0.511380in}{0.821354in}}%
\pgfpathclose%
\pgfusepath{stroke,fill}%
\end{pgfscope}%
\begin{pgfscope}%
\pgfsetrectcap%
\pgfsetmiterjoin%
\pgfsetlinewidth{0.000000pt}%
\definecolor{currentstroke}{rgb}{1.000000,1.000000,1.000000}%
\pgfsetstrokecolor{currentstroke}%
\pgfsetdash{}{0pt}%
\pgfpathmoveto{\pgfqpoint{0.211875in}{0.211875in}}%
\pgfpathlineto{\pgfqpoint{0.211875in}{1.491600in}}%
\pgfusepath{}%
\end{pgfscope}%
\begin{pgfscope}%
\pgfsetrectcap%
\pgfsetmiterjoin%
\pgfsetlinewidth{0.000000pt}%
\definecolor{currentstroke}{rgb}{1.000000,1.000000,1.000000}%
\pgfsetstrokecolor{currentstroke}%
\pgfsetdash{}{0pt}%
\pgfpathmoveto{\pgfqpoint{1.525500in}{0.211875in}}%
\pgfpathlineto{\pgfqpoint{1.525500in}{1.491600in}}%
\pgfusepath{}%
\end{pgfscope}%
\begin{pgfscope}%
\pgfsetrectcap%
\pgfsetmiterjoin%
\pgfsetlinewidth{0.000000pt}%
\definecolor{currentstroke}{rgb}{1.000000,1.000000,1.000000}%
\pgfsetstrokecolor{currentstroke}%
\pgfsetdash{}{0pt}%
\pgfpathmoveto{\pgfqpoint{0.211875in}{0.211875in}}%
\pgfpathlineto{\pgfqpoint{1.525500in}{0.211875in}}%
\pgfusepath{}%
\end{pgfscope}%
\begin{pgfscope}%
\pgfsetrectcap%
\pgfsetmiterjoin%
\pgfsetlinewidth{0.000000pt}%
\definecolor{currentstroke}{rgb}{1.000000,1.000000,1.000000}%
\pgfsetstrokecolor{currentstroke}%
\pgfsetdash{}{0pt}%
\pgfpathmoveto{\pgfqpoint{0.211875in}{1.491600in}}%
\pgfpathlineto{\pgfqpoint{1.525500in}{1.491600in}}%
\pgfusepath{}%
\end{pgfscope}%
\end{pgfpicture}%
\makeatother%
\endgroup%

            \end{minipage}
            %     \input{fig/griewank-exploration.pgf}
        \end{minipage}
        \captionof{figure}{
            These plots shows 200 observations of the Griekwank function using three different models each plotting the associated acquisition function.
            The left plot shows the Simple Regret.
            The right plot have four parts.
            \emph{Top left}: the Griekwank function. 
            \emph{Top right}: GP.
            \emph{Bottom left}: DNGO retrain-reset.
            \emph{Bottom right}: 5 $\times$ DNGO retrain-reset.}
        \vfill
        \newpage
        \begin{minipage}{\linewidth}
            \centering
            %% Creator: Matplotlib, PGF backend
%%
%% To include the figure in your LaTeX document, write
%%   \input{<filename>.pgf}
%%
%% Make sure the required packages are loaded in your preamble
%%   \usepackage{pgf}
%%
%% Figures using additional raster images can only be included by \input if
%% they are in the same directory as the main LaTeX file. For loading figures
%% from other directories you can use the `import` package
%%   \usepackage{import}
%% and then include the figures with
%%   \import{<path to file>}{<filename>.pgf}
%%
%% Matplotlib used the following preamble
%%   \usepackage{gensymb}
%%   \usepackage{fontspec}
%%   \setmainfont{DejaVu Serif}
%%   \setsansfont{Arial}
%%   \setmonofont{DejaVu Sans Mono}
%%
\begingroup%
\makeatletter%
\begin{pgfpicture}%
\pgfpathrectangle{\pgfpointorigin}{\pgfqpoint{6.900000in}{3.000000in}}%
\pgfusepath{use as bounding box, clip}%
\begin{pgfscope}%
\pgfsetbuttcap%
\pgfsetmiterjoin%
\definecolor{currentfill}{rgb}{1.000000,1.000000,1.000000}%
\pgfsetfillcolor{currentfill}%
\pgfsetlinewidth{0.000000pt}%
\definecolor{currentstroke}{rgb}{1.000000,1.000000,1.000000}%
\pgfsetstrokecolor{currentstroke}%
\pgfsetdash{}{0pt}%
\pgfpathmoveto{\pgfqpoint{0.000000in}{0.000000in}}%
\pgfpathlineto{\pgfqpoint{6.900000in}{0.000000in}}%
\pgfpathlineto{\pgfqpoint{6.900000in}{3.000000in}}%
\pgfpathlineto{\pgfqpoint{0.000000in}{3.000000in}}%
\pgfpathclose%
\pgfusepath{fill}%
\end{pgfscope}%
\begin{pgfscope}%
\pgfsetbuttcap%
\pgfsetmiterjoin%
\definecolor{currentfill}{rgb}{0.917647,0.917647,0.949020}%
\pgfsetfillcolor{currentfill}%
\pgfsetlinewidth{0.000000pt}%
\definecolor{currentstroke}{rgb}{0.000000,0.000000,0.000000}%
\pgfsetstrokecolor{currentstroke}%
\pgfsetstrokeopacity{0.000000}%
\pgfsetdash{}{0pt}%
\pgfpathmoveto{\pgfqpoint{0.862500in}{0.375000in}}%
\pgfpathlineto{\pgfqpoint{6.210000in}{0.375000in}}%
\pgfpathlineto{\pgfqpoint{6.210000in}{2.640000in}}%
\pgfpathlineto{\pgfqpoint{0.862500in}{2.640000in}}%
\pgfpathclose%
\pgfusepath{fill}%
\end{pgfscope}%
\begin{pgfscope}%
\pgfpathrectangle{\pgfqpoint{0.862500in}{0.375000in}}{\pgfqpoint{5.347500in}{2.265000in}}%
\pgfusepath{clip}%
\pgfsetroundcap%
\pgfsetroundjoin%
\pgfsetlinewidth{0.803000pt}%
\definecolor{currentstroke}{rgb}{1.000000,1.000000,1.000000}%
\pgfsetstrokecolor{currentstroke}%
\pgfsetdash{}{0pt}%
\pgfpathmoveto{\pgfqpoint{1.105568in}{0.375000in}}%
\pgfpathlineto{\pgfqpoint{1.105568in}{2.640000in}}%
\pgfusepath{stroke}%
\end{pgfscope}%
\begin{pgfscope}%
\definecolor{textcolor}{rgb}{0.150000,0.150000,0.150000}%
\pgfsetstrokecolor{textcolor}%
\pgfsetfillcolor{textcolor}%
\pgftext[x=1.105568in,y=0.326389in,,top]{\color{textcolor}\rmfamily\fontsize{8.000000}{9.600000}\selectfont \(\displaystyle 0\)}%
\end{pgfscope}%
\begin{pgfscope}%
\pgfpathrectangle{\pgfqpoint{0.862500in}{0.375000in}}{\pgfqpoint{5.347500in}{2.265000in}}%
\pgfusepath{clip}%
\pgfsetroundcap%
\pgfsetroundjoin%
\pgfsetlinewidth{0.803000pt}%
\definecolor{currentstroke}{rgb}{1.000000,1.000000,1.000000}%
\pgfsetstrokecolor{currentstroke}%
\pgfsetdash{}{0pt}%
\pgfpathmoveto{\pgfqpoint{2.215469in}{0.375000in}}%
\pgfpathlineto{\pgfqpoint{2.215469in}{2.640000in}}%
\pgfusepath{stroke}%
\end{pgfscope}%
\begin{pgfscope}%
\definecolor{textcolor}{rgb}{0.150000,0.150000,0.150000}%
\pgfsetstrokecolor{textcolor}%
\pgfsetfillcolor{textcolor}%
\pgftext[x=2.215469in,y=0.326389in,,top]{\color{textcolor}\rmfamily\fontsize{8.000000}{9.600000}\selectfont \(\displaystyle 50\)}%
\end{pgfscope}%
\begin{pgfscope}%
\pgfpathrectangle{\pgfqpoint{0.862500in}{0.375000in}}{\pgfqpoint{5.347500in}{2.265000in}}%
\pgfusepath{clip}%
\pgfsetroundcap%
\pgfsetroundjoin%
\pgfsetlinewidth{0.803000pt}%
\definecolor{currentstroke}{rgb}{1.000000,1.000000,1.000000}%
\pgfsetstrokecolor{currentstroke}%
\pgfsetdash{}{0pt}%
\pgfpathmoveto{\pgfqpoint{3.325369in}{0.375000in}}%
\pgfpathlineto{\pgfqpoint{3.325369in}{2.640000in}}%
\pgfusepath{stroke}%
\end{pgfscope}%
\begin{pgfscope}%
\definecolor{textcolor}{rgb}{0.150000,0.150000,0.150000}%
\pgfsetstrokecolor{textcolor}%
\pgfsetfillcolor{textcolor}%
\pgftext[x=3.325369in,y=0.326389in,,top]{\color{textcolor}\rmfamily\fontsize{8.000000}{9.600000}\selectfont \(\displaystyle 100\)}%
\end{pgfscope}%
\begin{pgfscope}%
\pgfpathrectangle{\pgfqpoint{0.862500in}{0.375000in}}{\pgfqpoint{5.347500in}{2.265000in}}%
\pgfusepath{clip}%
\pgfsetroundcap%
\pgfsetroundjoin%
\pgfsetlinewidth{0.803000pt}%
\definecolor{currentstroke}{rgb}{1.000000,1.000000,1.000000}%
\pgfsetstrokecolor{currentstroke}%
\pgfsetdash{}{0pt}%
\pgfpathmoveto{\pgfqpoint{4.435269in}{0.375000in}}%
\pgfpathlineto{\pgfqpoint{4.435269in}{2.640000in}}%
\pgfusepath{stroke}%
\end{pgfscope}%
\begin{pgfscope}%
\definecolor{textcolor}{rgb}{0.150000,0.150000,0.150000}%
\pgfsetstrokecolor{textcolor}%
\pgfsetfillcolor{textcolor}%
\pgftext[x=4.435269in,y=0.326389in,,top]{\color{textcolor}\rmfamily\fontsize{8.000000}{9.600000}\selectfont \(\displaystyle 150\)}%
\end{pgfscope}%
\begin{pgfscope}%
\pgfpathrectangle{\pgfqpoint{0.862500in}{0.375000in}}{\pgfqpoint{5.347500in}{2.265000in}}%
\pgfusepath{clip}%
\pgfsetroundcap%
\pgfsetroundjoin%
\pgfsetlinewidth{0.803000pt}%
\definecolor{currentstroke}{rgb}{1.000000,1.000000,1.000000}%
\pgfsetstrokecolor{currentstroke}%
\pgfsetdash{}{0pt}%
\pgfpathmoveto{\pgfqpoint{5.545170in}{0.375000in}}%
\pgfpathlineto{\pgfqpoint{5.545170in}{2.640000in}}%
\pgfusepath{stroke}%
\end{pgfscope}%
\begin{pgfscope}%
\definecolor{textcolor}{rgb}{0.150000,0.150000,0.150000}%
\pgfsetstrokecolor{textcolor}%
\pgfsetfillcolor{textcolor}%
\pgftext[x=5.545170in,y=0.326389in,,top]{\color{textcolor}\rmfamily\fontsize{8.000000}{9.600000}\selectfont \(\displaystyle 200\)}%
\end{pgfscope}%
\begin{pgfscope}%
\definecolor{textcolor}{rgb}{0.150000,0.150000,0.150000}%
\pgfsetstrokecolor{textcolor}%
\pgfsetfillcolor{textcolor}%
\pgftext[x=3.536250in,y=0.163303in,,top]{\color{textcolor}\rmfamily\fontsize{8.000000}{9.600000}\selectfont Step}%
\end{pgfscope}%
\begin{pgfscope}%
\pgfpathrectangle{\pgfqpoint{0.862500in}{0.375000in}}{\pgfqpoint{5.347500in}{2.265000in}}%
\pgfusepath{clip}%
\pgfsetroundcap%
\pgfsetroundjoin%
\pgfsetlinewidth{0.803000pt}%
\definecolor{currentstroke}{rgb}{1.000000,1.000000,1.000000}%
\pgfsetstrokecolor{currentstroke}%
\pgfsetdash{}{0pt}%
\pgfpathmoveto{\pgfqpoint{0.862500in}{0.479609in}}%
\pgfpathlineto{\pgfqpoint{6.210000in}{0.479609in}}%
\pgfusepath{stroke}%
\end{pgfscope}%
\begin{pgfscope}%
\definecolor{textcolor}{rgb}{0.150000,0.150000,0.150000}%
\pgfsetstrokecolor{textcolor}%
\pgfsetfillcolor{textcolor}%
\pgftext[x=0.557716in,y=0.437400in,left,base]{\color{textcolor}\rmfamily\fontsize{8.000000}{9.600000}\selectfont \(\displaystyle 10^{-3}\)}%
\end{pgfscope}%
\begin{pgfscope}%
\pgfpathrectangle{\pgfqpoint{0.862500in}{0.375000in}}{\pgfqpoint{5.347500in}{2.265000in}}%
\pgfusepath{clip}%
\pgfsetroundcap%
\pgfsetroundjoin%
\pgfsetlinewidth{0.803000pt}%
\definecolor{currentstroke}{rgb}{1.000000,1.000000,1.000000}%
\pgfsetstrokecolor{currentstroke}%
\pgfsetdash{}{0pt}%
\pgfpathmoveto{\pgfqpoint{0.862500in}{0.995854in}}%
\pgfpathlineto{\pgfqpoint{6.210000in}{0.995854in}}%
\pgfusepath{stroke}%
\end{pgfscope}%
\begin{pgfscope}%
\definecolor{textcolor}{rgb}{0.150000,0.150000,0.150000}%
\pgfsetstrokecolor{textcolor}%
\pgfsetfillcolor{textcolor}%
\pgftext[x=0.557716in,y=0.953645in,left,base]{\color{textcolor}\rmfamily\fontsize{8.000000}{9.600000}\selectfont \(\displaystyle 10^{-2}\)}%
\end{pgfscope}%
\begin{pgfscope}%
\pgfpathrectangle{\pgfqpoint{0.862500in}{0.375000in}}{\pgfqpoint{5.347500in}{2.265000in}}%
\pgfusepath{clip}%
\pgfsetroundcap%
\pgfsetroundjoin%
\pgfsetlinewidth{0.803000pt}%
\definecolor{currentstroke}{rgb}{1.000000,1.000000,1.000000}%
\pgfsetstrokecolor{currentstroke}%
\pgfsetdash{}{0pt}%
\pgfpathmoveto{\pgfqpoint{0.862500in}{1.512099in}}%
\pgfpathlineto{\pgfqpoint{6.210000in}{1.512099in}}%
\pgfusepath{stroke}%
\end{pgfscope}%
\begin{pgfscope}%
\definecolor{textcolor}{rgb}{0.150000,0.150000,0.150000}%
\pgfsetstrokecolor{textcolor}%
\pgfsetfillcolor{textcolor}%
\pgftext[x=0.557716in,y=1.469890in,left,base]{\color{textcolor}\rmfamily\fontsize{8.000000}{9.600000}\selectfont \(\displaystyle 10^{-1}\)}%
\end{pgfscope}%
\begin{pgfscope}%
\pgfpathrectangle{\pgfqpoint{0.862500in}{0.375000in}}{\pgfqpoint{5.347500in}{2.265000in}}%
\pgfusepath{clip}%
\pgfsetroundcap%
\pgfsetroundjoin%
\pgfsetlinewidth{0.803000pt}%
\definecolor{currentstroke}{rgb}{1.000000,1.000000,1.000000}%
\pgfsetstrokecolor{currentstroke}%
\pgfsetdash{}{0pt}%
\pgfpathmoveto{\pgfqpoint{0.862500in}{2.028344in}}%
\pgfpathlineto{\pgfqpoint{6.210000in}{2.028344in}}%
\pgfusepath{stroke}%
\end{pgfscope}%
\begin{pgfscope}%
\definecolor{textcolor}{rgb}{0.150000,0.150000,0.150000}%
\pgfsetstrokecolor{textcolor}%
\pgfsetfillcolor{textcolor}%
\pgftext[x=0.637962in,y=1.986135in,left,base]{\color{textcolor}\rmfamily\fontsize{8.000000}{9.600000}\selectfont \(\displaystyle 10^{0}\)}%
\end{pgfscope}%
\begin{pgfscope}%
\pgfpathrectangle{\pgfqpoint{0.862500in}{0.375000in}}{\pgfqpoint{5.347500in}{2.265000in}}%
\pgfusepath{clip}%
\pgfsetroundcap%
\pgfsetroundjoin%
\pgfsetlinewidth{0.803000pt}%
\definecolor{currentstroke}{rgb}{1.000000,1.000000,1.000000}%
\pgfsetstrokecolor{currentstroke}%
\pgfsetdash{}{0pt}%
\pgfpathmoveto{\pgfqpoint{0.862500in}{2.544590in}}%
\pgfpathlineto{\pgfqpoint{6.210000in}{2.544590in}}%
\pgfusepath{stroke}%
\end{pgfscope}%
\begin{pgfscope}%
\definecolor{textcolor}{rgb}{0.150000,0.150000,0.150000}%
\pgfsetstrokecolor{textcolor}%
\pgfsetfillcolor{textcolor}%
\pgftext[x=0.637962in,y=2.502380in,left,base]{\color{textcolor}\rmfamily\fontsize{8.000000}{9.600000}\selectfont \(\displaystyle 10^{1}\)}%
\end{pgfscope}%
\begin{pgfscope}%
\definecolor{textcolor}{rgb}{0.150000,0.150000,0.150000}%
\pgfsetstrokecolor{textcolor}%
\pgfsetfillcolor{textcolor}%
\pgftext[x=0.502160in,y=1.507500in,,bottom,rotate=90.000000]{\color{textcolor}\rmfamily\fontsize{8.000000}{9.600000}\selectfont Simple Regret}%
\end{pgfscope}%
\begin{pgfscope}%
\pgfpathrectangle{\pgfqpoint{0.862500in}{0.375000in}}{\pgfqpoint{5.347500in}{2.265000in}}%
\pgfusepath{clip}%
\pgfsetbuttcap%
\pgfsetroundjoin%
\definecolor{currentfill}{rgb}{0.121569,0.466667,0.705882}%
\pgfsetfillcolor{currentfill}%
\pgfsetfillopacity{0.200000}%
\pgfsetlinewidth{0.000000pt}%
\definecolor{currentstroke}{rgb}{0.000000,0.000000,0.000000}%
\pgfsetstrokecolor{currentstroke}%
\pgfsetdash{}{0pt}%
\pgfpathmoveto{\pgfqpoint{1.105568in}{2.414983in}}%
\pgfpathlineto{\pgfqpoint{1.105568in}{2.537045in}}%
\pgfpathlineto{\pgfqpoint{1.127766in}{2.361379in}}%
\pgfpathlineto{\pgfqpoint{1.149964in}{2.276125in}}%
\pgfpathlineto{\pgfqpoint{1.172162in}{2.240465in}}%
\pgfpathlineto{\pgfqpoint{1.194360in}{2.224697in}}%
\pgfpathlineto{\pgfqpoint{1.216558in}{2.224697in}}%
\pgfpathlineto{\pgfqpoint{1.238756in}{2.206199in}}%
\pgfpathlineto{\pgfqpoint{1.260954in}{2.199008in}}%
\pgfpathlineto{\pgfqpoint{1.283152in}{2.184851in}}%
\pgfpathlineto{\pgfqpoint{1.305350in}{2.172915in}}%
\pgfpathlineto{\pgfqpoint{1.327548in}{2.147533in}}%
\pgfpathlineto{\pgfqpoint{1.349746in}{2.147533in}}%
\pgfpathlineto{\pgfqpoint{1.371944in}{2.134855in}}%
\pgfpathlineto{\pgfqpoint{1.394142in}{2.133123in}}%
\pgfpathlineto{\pgfqpoint{1.416340in}{2.074553in}}%
\pgfpathlineto{\pgfqpoint{1.438538in}{2.057502in}}%
\pgfpathlineto{\pgfqpoint{1.460736in}{2.057502in}}%
\pgfpathlineto{\pgfqpoint{1.482934in}{2.046926in}}%
\pgfpathlineto{\pgfqpoint{1.505132in}{2.046926in}}%
\pgfpathlineto{\pgfqpoint{1.527330in}{2.046926in}}%
\pgfpathlineto{\pgfqpoint{1.549528in}{2.043316in}}%
\pgfpathlineto{\pgfqpoint{1.571726in}{2.043316in}}%
\pgfpathlineto{\pgfqpoint{1.593924in}{2.043316in}}%
\pgfpathlineto{\pgfqpoint{1.616122in}{2.043316in}}%
\pgfpathlineto{\pgfqpoint{1.638320in}{2.043316in}}%
\pgfpathlineto{\pgfqpoint{1.660518in}{2.043316in}}%
\pgfpathlineto{\pgfqpoint{1.682716in}{2.043316in}}%
\pgfpathlineto{\pgfqpoint{1.704914in}{2.043316in}}%
\pgfpathlineto{\pgfqpoint{1.727112in}{2.043316in}}%
\pgfpathlineto{\pgfqpoint{1.749310in}{2.043316in}}%
\pgfpathlineto{\pgfqpoint{1.771508in}{2.001140in}}%
\pgfpathlineto{\pgfqpoint{1.793706in}{1.986660in}}%
\pgfpathlineto{\pgfqpoint{1.815904in}{1.986660in}}%
\pgfpathlineto{\pgfqpoint{1.838102in}{1.965004in}}%
\pgfpathlineto{\pgfqpoint{1.860300in}{1.960659in}}%
\pgfpathlineto{\pgfqpoint{1.882498in}{1.960659in}}%
\pgfpathlineto{\pgfqpoint{1.904696in}{1.960659in}}%
\pgfpathlineto{\pgfqpoint{1.926894in}{1.960659in}}%
\pgfpathlineto{\pgfqpoint{1.949092in}{1.960659in}}%
\pgfpathlineto{\pgfqpoint{1.971290in}{1.960659in}}%
\pgfpathlineto{\pgfqpoint{1.993488in}{1.960659in}}%
\pgfpathlineto{\pgfqpoint{2.015686in}{1.960659in}}%
\pgfpathlineto{\pgfqpoint{2.037884in}{1.943039in}}%
\pgfpathlineto{\pgfqpoint{2.060083in}{1.943039in}}%
\pgfpathlineto{\pgfqpoint{2.082281in}{1.943039in}}%
\pgfpathlineto{\pgfqpoint{2.104479in}{1.943039in}}%
\pgfpathlineto{\pgfqpoint{2.126677in}{1.943039in}}%
\pgfpathlineto{\pgfqpoint{2.148875in}{1.943039in}}%
\pgfpathlineto{\pgfqpoint{2.171073in}{1.943039in}}%
\pgfpathlineto{\pgfqpoint{2.193271in}{1.943039in}}%
\pgfpathlineto{\pgfqpoint{2.215469in}{1.934715in}}%
\pgfpathlineto{\pgfqpoint{2.237667in}{1.934715in}}%
\pgfpathlineto{\pgfqpoint{2.259865in}{1.934715in}}%
\pgfpathlineto{\pgfqpoint{2.282063in}{1.934715in}}%
\pgfpathlineto{\pgfqpoint{2.304261in}{1.934715in}}%
\pgfpathlineto{\pgfqpoint{2.326459in}{1.934715in}}%
\pgfpathlineto{\pgfqpoint{2.348657in}{1.934715in}}%
\pgfpathlineto{\pgfqpoint{2.370855in}{1.934715in}}%
\pgfpathlineto{\pgfqpoint{2.393053in}{1.934715in}}%
\pgfpathlineto{\pgfqpoint{2.415251in}{1.934715in}}%
\pgfpathlineto{\pgfqpoint{2.437449in}{1.934715in}}%
\pgfpathlineto{\pgfqpoint{2.459647in}{1.934715in}}%
\pgfpathlineto{\pgfqpoint{2.481845in}{1.934715in}}%
\pgfpathlineto{\pgfqpoint{2.504043in}{1.881188in}}%
\pgfpathlineto{\pgfqpoint{2.526241in}{1.881188in}}%
\pgfpathlineto{\pgfqpoint{2.548439in}{1.855669in}}%
\pgfpathlineto{\pgfqpoint{2.570637in}{1.855669in}}%
\pgfpathlineto{\pgfqpoint{2.592835in}{1.855669in}}%
\pgfpathlineto{\pgfqpoint{2.615033in}{1.855669in}}%
\pgfpathlineto{\pgfqpoint{2.637231in}{1.855669in}}%
\pgfpathlineto{\pgfqpoint{2.659429in}{1.855669in}}%
\pgfpathlineto{\pgfqpoint{2.681627in}{1.855669in}}%
\pgfpathlineto{\pgfqpoint{2.703825in}{1.855137in}}%
\pgfpathlineto{\pgfqpoint{2.726023in}{1.855137in}}%
\pgfpathlineto{\pgfqpoint{2.748221in}{1.855137in}}%
\pgfpathlineto{\pgfqpoint{2.770419in}{1.834159in}}%
\pgfpathlineto{\pgfqpoint{2.792617in}{1.834159in}}%
\pgfpathlineto{\pgfqpoint{2.814815in}{1.834159in}}%
\pgfpathlineto{\pgfqpoint{2.837013in}{1.834159in}}%
\pgfpathlineto{\pgfqpoint{2.859211in}{1.834159in}}%
\pgfpathlineto{\pgfqpoint{2.881409in}{1.834159in}}%
\pgfpathlineto{\pgfqpoint{2.903607in}{1.834159in}}%
\pgfpathlineto{\pgfqpoint{2.925805in}{1.834159in}}%
\pgfpathlineto{\pgfqpoint{2.948003in}{1.834159in}}%
\pgfpathlineto{\pgfqpoint{2.970201in}{1.826524in}}%
\pgfpathlineto{\pgfqpoint{2.992399in}{1.826524in}}%
\pgfpathlineto{\pgfqpoint{3.014597in}{1.826524in}}%
\pgfpathlineto{\pgfqpoint{3.036795in}{1.826524in}}%
\pgfpathlineto{\pgfqpoint{3.058993in}{1.826524in}}%
\pgfpathlineto{\pgfqpoint{3.081191in}{1.826524in}}%
\pgfpathlineto{\pgfqpoint{3.103389in}{1.826524in}}%
\pgfpathlineto{\pgfqpoint{3.125587in}{1.826524in}}%
\pgfpathlineto{\pgfqpoint{3.147785in}{1.826524in}}%
\pgfpathlineto{\pgfqpoint{3.169983in}{1.826524in}}%
\pgfpathlineto{\pgfqpoint{3.192181in}{1.823095in}}%
\pgfpathlineto{\pgfqpoint{3.214379in}{1.823095in}}%
\pgfpathlineto{\pgfqpoint{3.236577in}{1.823095in}}%
\pgfpathlineto{\pgfqpoint{3.258775in}{1.816056in}}%
\pgfpathlineto{\pgfqpoint{3.280973in}{1.816056in}}%
\pgfpathlineto{\pgfqpoint{3.303171in}{1.816056in}}%
\pgfpathlineto{\pgfqpoint{3.325369in}{1.816056in}}%
\pgfpathlineto{\pgfqpoint{3.347567in}{1.816056in}}%
\pgfpathlineto{\pgfqpoint{3.369765in}{1.816056in}}%
\pgfpathlineto{\pgfqpoint{3.391963in}{1.816056in}}%
\pgfpathlineto{\pgfqpoint{3.414161in}{1.814428in}}%
\pgfpathlineto{\pgfqpoint{3.436359in}{1.814428in}}%
\pgfpathlineto{\pgfqpoint{3.458557in}{1.814428in}}%
\pgfpathlineto{\pgfqpoint{3.480755in}{1.814428in}}%
\pgfpathlineto{\pgfqpoint{3.502953in}{1.814428in}}%
\pgfpathlineto{\pgfqpoint{3.525151in}{1.814428in}}%
\pgfpathlineto{\pgfqpoint{3.547349in}{1.814428in}}%
\pgfpathlineto{\pgfqpoint{3.569547in}{1.814428in}}%
\pgfpathlineto{\pgfqpoint{3.591745in}{1.814428in}}%
\pgfpathlineto{\pgfqpoint{3.613943in}{1.814428in}}%
\pgfpathlineto{\pgfqpoint{3.636141in}{1.814428in}}%
\pgfpathlineto{\pgfqpoint{3.658339in}{1.814428in}}%
\pgfpathlineto{\pgfqpoint{3.680537in}{1.814428in}}%
\pgfpathlineto{\pgfqpoint{3.702735in}{1.814428in}}%
\pgfpathlineto{\pgfqpoint{3.724933in}{1.814428in}}%
\pgfpathlineto{\pgfqpoint{3.747131in}{1.814428in}}%
\pgfpathlineto{\pgfqpoint{3.769329in}{1.814428in}}%
\pgfpathlineto{\pgfqpoint{3.791527in}{1.814428in}}%
\pgfpathlineto{\pgfqpoint{3.813725in}{1.739535in}}%
\pgfpathlineto{\pgfqpoint{3.835923in}{1.739535in}}%
\pgfpathlineto{\pgfqpoint{3.858121in}{1.739535in}}%
\pgfpathlineto{\pgfqpoint{3.880319in}{1.739535in}}%
\pgfpathlineto{\pgfqpoint{3.902517in}{1.739535in}}%
\pgfpathlineto{\pgfqpoint{3.924715in}{1.739535in}}%
\pgfpathlineto{\pgfqpoint{3.946913in}{1.739535in}}%
\pgfpathlineto{\pgfqpoint{3.969111in}{1.739535in}}%
\pgfpathlineto{\pgfqpoint{3.991309in}{1.739535in}}%
\pgfpathlineto{\pgfqpoint{4.013507in}{1.739535in}}%
\pgfpathlineto{\pgfqpoint{4.035705in}{1.739535in}}%
\pgfpathlineto{\pgfqpoint{4.057903in}{1.739535in}}%
\pgfpathlineto{\pgfqpoint{4.080101in}{1.739535in}}%
\pgfpathlineto{\pgfqpoint{4.102299in}{1.739535in}}%
\pgfpathlineto{\pgfqpoint{4.124497in}{1.739535in}}%
\pgfpathlineto{\pgfqpoint{4.146695in}{1.739535in}}%
\pgfpathlineto{\pgfqpoint{4.168893in}{1.739535in}}%
\pgfpathlineto{\pgfqpoint{4.191091in}{1.739535in}}%
\pgfpathlineto{\pgfqpoint{4.213289in}{1.739535in}}%
\pgfpathlineto{\pgfqpoint{4.235487in}{1.716083in}}%
\pgfpathlineto{\pgfqpoint{4.257685in}{1.716083in}}%
\pgfpathlineto{\pgfqpoint{4.279883in}{1.716083in}}%
\pgfpathlineto{\pgfqpoint{4.302081in}{1.716083in}}%
\pgfpathlineto{\pgfqpoint{4.324279in}{1.716083in}}%
\pgfpathlineto{\pgfqpoint{4.346477in}{1.716083in}}%
\pgfpathlineto{\pgfqpoint{4.368675in}{1.716083in}}%
\pgfpathlineto{\pgfqpoint{4.390873in}{1.716083in}}%
\pgfpathlineto{\pgfqpoint{4.413071in}{1.716083in}}%
\pgfpathlineto{\pgfqpoint{4.435269in}{1.716083in}}%
\pgfpathlineto{\pgfqpoint{4.457467in}{1.716083in}}%
\pgfpathlineto{\pgfqpoint{4.479665in}{1.716083in}}%
\pgfpathlineto{\pgfqpoint{4.501863in}{1.716083in}}%
\pgfpathlineto{\pgfqpoint{4.524061in}{1.716083in}}%
\pgfpathlineto{\pgfqpoint{4.546259in}{1.716083in}}%
\pgfpathlineto{\pgfqpoint{4.568457in}{1.716083in}}%
\pgfpathlineto{\pgfqpoint{4.590655in}{1.716083in}}%
\pgfpathlineto{\pgfqpoint{4.612853in}{1.716083in}}%
\pgfpathlineto{\pgfqpoint{4.635051in}{1.716083in}}%
\pgfpathlineto{\pgfqpoint{4.657249in}{1.716083in}}%
\pgfpathlineto{\pgfqpoint{4.679447in}{1.716083in}}%
\pgfpathlineto{\pgfqpoint{4.701645in}{1.716083in}}%
\pgfpathlineto{\pgfqpoint{4.723843in}{1.716083in}}%
\pgfpathlineto{\pgfqpoint{4.746041in}{1.716083in}}%
\pgfpathlineto{\pgfqpoint{4.768239in}{1.716083in}}%
\pgfpathlineto{\pgfqpoint{4.790437in}{1.716083in}}%
\pgfpathlineto{\pgfqpoint{4.812635in}{1.716083in}}%
\pgfpathlineto{\pgfqpoint{4.834833in}{1.716083in}}%
\pgfpathlineto{\pgfqpoint{4.857031in}{1.716083in}}%
\pgfpathlineto{\pgfqpoint{4.879229in}{1.716083in}}%
\pgfpathlineto{\pgfqpoint{4.901427in}{1.716083in}}%
\pgfpathlineto{\pgfqpoint{4.923625in}{1.716083in}}%
\pgfpathlineto{\pgfqpoint{4.945823in}{1.716083in}}%
\pgfpathlineto{\pgfqpoint{4.968021in}{1.716083in}}%
\pgfpathlineto{\pgfqpoint{4.990219in}{1.716083in}}%
\pgfpathlineto{\pgfqpoint{5.012417in}{1.716083in}}%
\pgfpathlineto{\pgfqpoint{5.034616in}{1.716083in}}%
\pgfpathlineto{\pgfqpoint{5.056814in}{1.716083in}}%
\pgfpathlineto{\pgfqpoint{5.079012in}{1.716083in}}%
\pgfpathlineto{\pgfqpoint{5.101210in}{1.697848in}}%
\pgfpathlineto{\pgfqpoint{5.123408in}{1.697848in}}%
\pgfpathlineto{\pgfqpoint{5.145606in}{1.697848in}}%
\pgfpathlineto{\pgfqpoint{5.167804in}{1.697848in}}%
\pgfpathlineto{\pgfqpoint{5.190002in}{1.697848in}}%
\pgfpathlineto{\pgfqpoint{5.212200in}{1.697848in}}%
\pgfpathlineto{\pgfqpoint{5.234398in}{1.697848in}}%
\pgfpathlineto{\pgfqpoint{5.256596in}{1.697848in}}%
\pgfpathlineto{\pgfqpoint{5.278794in}{1.697848in}}%
\pgfpathlineto{\pgfqpoint{5.300992in}{1.697848in}}%
\pgfpathlineto{\pgfqpoint{5.323190in}{1.697848in}}%
\pgfpathlineto{\pgfqpoint{5.345388in}{1.697848in}}%
\pgfpathlineto{\pgfqpoint{5.367586in}{1.694847in}}%
\pgfpathlineto{\pgfqpoint{5.389784in}{1.677215in}}%
\pgfpathlineto{\pgfqpoint{5.411982in}{1.677215in}}%
\pgfpathlineto{\pgfqpoint{5.434180in}{1.677215in}}%
\pgfpathlineto{\pgfqpoint{5.456378in}{1.677215in}}%
\pgfpathlineto{\pgfqpoint{5.478576in}{1.677215in}}%
\pgfpathlineto{\pgfqpoint{5.500774in}{1.677215in}}%
\pgfpathlineto{\pgfqpoint{5.522972in}{1.677215in}}%
\pgfpathlineto{\pgfqpoint{5.545170in}{1.677215in}}%
\pgfpathlineto{\pgfqpoint{5.567368in}{1.677215in}}%
\pgfpathlineto{\pgfqpoint{5.589566in}{1.677215in}}%
\pgfpathlineto{\pgfqpoint{5.611764in}{1.677215in}}%
\pgfpathlineto{\pgfqpoint{5.633962in}{1.677215in}}%
\pgfpathlineto{\pgfqpoint{5.656160in}{1.677215in}}%
\pgfpathlineto{\pgfqpoint{5.678358in}{1.677215in}}%
\pgfpathlineto{\pgfqpoint{5.700556in}{1.677215in}}%
\pgfpathlineto{\pgfqpoint{5.722754in}{1.677215in}}%
\pgfpathlineto{\pgfqpoint{5.744952in}{1.677215in}}%
\pgfpathlineto{\pgfqpoint{5.767150in}{1.677215in}}%
\pgfpathlineto{\pgfqpoint{5.789348in}{1.677215in}}%
\pgfpathlineto{\pgfqpoint{5.811546in}{1.677215in}}%
\pgfpathlineto{\pgfqpoint{5.833744in}{1.677215in}}%
\pgfpathlineto{\pgfqpoint{5.855942in}{1.677215in}}%
\pgfpathlineto{\pgfqpoint{5.878140in}{1.677215in}}%
\pgfpathlineto{\pgfqpoint{5.900338in}{1.677215in}}%
\pgfpathlineto{\pgfqpoint{5.922536in}{1.677215in}}%
\pgfpathlineto{\pgfqpoint{5.944734in}{1.677215in}}%
\pgfpathlineto{\pgfqpoint{5.966932in}{1.677215in}}%
\pgfpathlineto{\pgfqpoint{5.966932in}{1.459332in}}%
\pgfpathlineto{\pgfqpoint{5.966932in}{1.459332in}}%
\pgfpathlineto{\pgfqpoint{5.944734in}{1.459332in}}%
\pgfpathlineto{\pgfqpoint{5.922536in}{1.459332in}}%
\pgfpathlineto{\pgfqpoint{5.900338in}{1.459332in}}%
\pgfpathlineto{\pgfqpoint{5.878140in}{1.459332in}}%
\pgfpathlineto{\pgfqpoint{5.855942in}{1.459332in}}%
\pgfpathlineto{\pgfqpoint{5.833744in}{1.459332in}}%
\pgfpathlineto{\pgfqpoint{5.811546in}{1.459332in}}%
\pgfpathlineto{\pgfqpoint{5.789348in}{1.459332in}}%
\pgfpathlineto{\pgfqpoint{5.767150in}{1.459332in}}%
\pgfpathlineto{\pgfqpoint{5.744952in}{1.459332in}}%
\pgfpathlineto{\pgfqpoint{5.722754in}{1.459332in}}%
\pgfpathlineto{\pgfqpoint{5.700556in}{1.459332in}}%
\pgfpathlineto{\pgfqpoint{5.678358in}{1.459332in}}%
\pgfpathlineto{\pgfqpoint{5.656160in}{1.459332in}}%
\pgfpathlineto{\pgfqpoint{5.633962in}{1.459332in}}%
\pgfpathlineto{\pgfqpoint{5.611764in}{1.459332in}}%
\pgfpathlineto{\pgfqpoint{5.589566in}{1.459332in}}%
\pgfpathlineto{\pgfqpoint{5.567368in}{1.459332in}}%
\pgfpathlineto{\pgfqpoint{5.545170in}{1.459332in}}%
\pgfpathlineto{\pgfqpoint{5.522972in}{1.459332in}}%
\pgfpathlineto{\pgfqpoint{5.500774in}{1.459332in}}%
\pgfpathlineto{\pgfqpoint{5.478576in}{1.459332in}}%
\pgfpathlineto{\pgfqpoint{5.456378in}{1.459332in}}%
\pgfpathlineto{\pgfqpoint{5.434180in}{1.459332in}}%
\pgfpathlineto{\pgfqpoint{5.411982in}{1.459332in}}%
\pgfpathlineto{\pgfqpoint{5.389784in}{1.459332in}}%
\pgfpathlineto{\pgfqpoint{5.367586in}{1.479959in}}%
\pgfpathlineto{\pgfqpoint{5.345388in}{1.488805in}}%
\pgfpathlineto{\pgfqpoint{5.323190in}{1.488805in}}%
\pgfpathlineto{\pgfqpoint{5.300992in}{1.488805in}}%
\pgfpathlineto{\pgfqpoint{5.278794in}{1.488805in}}%
\pgfpathlineto{\pgfqpoint{5.256596in}{1.488805in}}%
\pgfpathlineto{\pgfqpoint{5.234398in}{1.488805in}}%
\pgfpathlineto{\pgfqpoint{5.212200in}{1.488805in}}%
\pgfpathlineto{\pgfqpoint{5.190002in}{1.488805in}}%
\pgfpathlineto{\pgfqpoint{5.167804in}{1.488805in}}%
\pgfpathlineto{\pgfqpoint{5.145606in}{1.488805in}}%
\pgfpathlineto{\pgfqpoint{5.123408in}{1.488805in}}%
\pgfpathlineto{\pgfqpoint{5.101210in}{1.488805in}}%
\pgfpathlineto{\pgfqpoint{5.079012in}{1.497040in}}%
\pgfpathlineto{\pgfqpoint{5.056814in}{1.497040in}}%
\pgfpathlineto{\pgfqpoint{5.034616in}{1.497040in}}%
\pgfpathlineto{\pgfqpoint{5.012417in}{1.497040in}}%
\pgfpathlineto{\pgfqpoint{4.990219in}{1.497040in}}%
\pgfpathlineto{\pgfqpoint{4.968021in}{1.497040in}}%
\pgfpathlineto{\pgfqpoint{4.945823in}{1.497040in}}%
\pgfpathlineto{\pgfqpoint{4.923625in}{1.497040in}}%
\pgfpathlineto{\pgfqpoint{4.901427in}{1.497040in}}%
\pgfpathlineto{\pgfqpoint{4.879229in}{1.497040in}}%
\pgfpathlineto{\pgfqpoint{4.857031in}{1.497040in}}%
\pgfpathlineto{\pgfqpoint{4.834833in}{1.497040in}}%
\pgfpathlineto{\pgfqpoint{4.812635in}{1.497040in}}%
\pgfpathlineto{\pgfqpoint{4.790437in}{1.497040in}}%
\pgfpathlineto{\pgfqpoint{4.768239in}{1.497040in}}%
\pgfpathlineto{\pgfqpoint{4.746041in}{1.497040in}}%
\pgfpathlineto{\pgfqpoint{4.723843in}{1.497040in}}%
\pgfpathlineto{\pgfqpoint{4.701645in}{1.497040in}}%
\pgfpathlineto{\pgfqpoint{4.679447in}{1.497040in}}%
\pgfpathlineto{\pgfqpoint{4.657249in}{1.497040in}}%
\pgfpathlineto{\pgfqpoint{4.635051in}{1.497040in}}%
\pgfpathlineto{\pgfqpoint{4.612853in}{1.497040in}}%
\pgfpathlineto{\pgfqpoint{4.590655in}{1.497040in}}%
\pgfpathlineto{\pgfqpoint{4.568457in}{1.497040in}}%
\pgfpathlineto{\pgfqpoint{4.546259in}{1.497040in}}%
\pgfpathlineto{\pgfqpoint{4.524061in}{1.497040in}}%
\pgfpathlineto{\pgfqpoint{4.501863in}{1.497040in}}%
\pgfpathlineto{\pgfqpoint{4.479665in}{1.497040in}}%
\pgfpathlineto{\pgfqpoint{4.457467in}{1.497040in}}%
\pgfpathlineto{\pgfqpoint{4.435269in}{1.497040in}}%
\pgfpathlineto{\pgfqpoint{4.413071in}{1.497040in}}%
\pgfpathlineto{\pgfqpoint{4.390873in}{1.497040in}}%
\pgfpathlineto{\pgfqpoint{4.368675in}{1.497040in}}%
\pgfpathlineto{\pgfqpoint{4.346477in}{1.497040in}}%
\pgfpathlineto{\pgfqpoint{4.324279in}{1.497040in}}%
\pgfpathlineto{\pgfqpoint{4.302081in}{1.497040in}}%
\pgfpathlineto{\pgfqpoint{4.279883in}{1.497040in}}%
\pgfpathlineto{\pgfqpoint{4.257685in}{1.497040in}}%
\pgfpathlineto{\pgfqpoint{4.235487in}{1.497040in}}%
\pgfpathlineto{\pgfqpoint{4.213289in}{1.558992in}}%
\pgfpathlineto{\pgfqpoint{4.191091in}{1.558992in}}%
\pgfpathlineto{\pgfqpoint{4.168893in}{1.558992in}}%
\pgfpathlineto{\pgfqpoint{4.146695in}{1.558992in}}%
\pgfpathlineto{\pgfqpoint{4.124497in}{1.558992in}}%
\pgfpathlineto{\pgfqpoint{4.102299in}{1.558992in}}%
\pgfpathlineto{\pgfqpoint{4.080101in}{1.558992in}}%
\pgfpathlineto{\pgfqpoint{4.057903in}{1.558992in}}%
\pgfpathlineto{\pgfqpoint{4.035705in}{1.558992in}}%
\pgfpathlineto{\pgfqpoint{4.013507in}{1.558992in}}%
\pgfpathlineto{\pgfqpoint{3.991309in}{1.558992in}}%
\pgfpathlineto{\pgfqpoint{3.969111in}{1.558992in}}%
\pgfpathlineto{\pgfqpoint{3.946913in}{1.558992in}}%
\pgfpathlineto{\pgfqpoint{3.924715in}{1.558992in}}%
\pgfpathlineto{\pgfqpoint{3.902517in}{1.558992in}}%
\pgfpathlineto{\pgfqpoint{3.880319in}{1.558992in}}%
\pgfpathlineto{\pgfqpoint{3.858121in}{1.558992in}}%
\pgfpathlineto{\pgfqpoint{3.835923in}{1.558992in}}%
\pgfpathlineto{\pgfqpoint{3.813725in}{1.558992in}}%
\pgfpathlineto{\pgfqpoint{3.791527in}{1.624388in}}%
\pgfpathlineto{\pgfqpoint{3.769329in}{1.624388in}}%
\pgfpathlineto{\pgfqpoint{3.747131in}{1.624388in}}%
\pgfpathlineto{\pgfqpoint{3.724933in}{1.624388in}}%
\pgfpathlineto{\pgfqpoint{3.702735in}{1.624388in}}%
\pgfpathlineto{\pgfqpoint{3.680537in}{1.624388in}}%
\pgfpathlineto{\pgfqpoint{3.658339in}{1.624388in}}%
\pgfpathlineto{\pgfqpoint{3.636141in}{1.624388in}}%
\pgfpathlineto{\pgfqpoint{3.613943in}{1.624388in}}%
\pgfpathlineto{\pgfqpoint{3.591745in}{1.624388in}}%
\pgfpathlineto{\pgfqpoint{3.569547in}{1.624388in}}%
\pgfpathlineto{\pgfqpoint{3.547349in}{1.624388in}}%
\pgfpathlineto{\pgfqpoint{3.525151in}{1.624388in}}%
\pgfpathlineto{\pgfqpoint{3.502953in}{1.624388in}}%
\pgfpathlineto{\pgfqpoint{3.480755in}{1.624388in}}%
\pgfpathlineto{\pgfqpoint{3.458557in}{1.624388in}}%
\pgfpathlineto{\pgfqpoint{3.436359in}{1.624388in}}%
\pgfpathlineto{\pgfqpoint{3.414161in}{1.624388in}}%
\pgfpathlineto{\pgfqpoint{3.391963in}{1.627382in}}%
\pgfpathlineto{\pgfqpoint{3.369765in}{1.627382in}}%
\pgfpathlineto{\pgfqpoint{3.347567in}{1.627382in}}%
\pgfpathlineto{\pgfqpoint{3.325369in}{1.627382in}}%
\pgfpathlineto{\pgfqpoint{3.303171in}{1.627382in}}%
\pgfpathlineto{\pgfqpoint{3.280973in}{1.627382in}}%
\pgfpathlineto{\pgfqpoint{3.258775in}{1.627382in}}%
\pgfpathlineto{\pgfqpoint{3.236577in}{1.626112in}}%
\pgfpathlineto{\pgfqpoint{3.214379in}{1.626112in}}%
\pgfpathlineto{\pgfqpoint{3.192181in}{1.626112in}}%
\pgfpathlineto{\pgfqpoint{3.169983in}{1.631778in}}%
\pgfpathlineto{\pgfqpoint{3.147785in}{1.631778in}}%
\pgfpathlineto{\pgfqpoint{3.125587in}{1.631778in}}%
\pgfpathlineto{\pgfqpoint{3.103389in}{1.631778in}}%
\pgfpathlineto{\pgfqpoint{3.081191in}{1.631778in}}%
\pgfpathlineto{\pgfqpoint{3.058993in}{1.631778in}}%
\pgfpathlineto{\pgfqpoint{3.036795in}{1.631778in}}%
\pgfpathlineto{\pgfqpoint{3.014597in}{1.631778in}}%
\pgfpathlineto{\pgfqpoint{2.992399in}{1.631778in}}%
\pgfpathlineto{\pgfqpoint{2.970201in}{1.631778in}}%
\pgfpathlineto{\pgfqpoint{2.948003in}{1.654576in}}%
\pgfpathlineto{\pgfqpoint{2.925805in}{1.654576in}}%
\pgfpathlineto{\pgfqpoint{2.903607in}{1.654576in}}%
\pgfpathlineto{\pgfqpoint{2.881409in}{1.654576in}}%
\pgfpathlineto{\pgfqpoint{2.859211in}{1.654576in}}%
\pgfpathlineto{\pgfqpoint{2.837013in}{1.654576in}}%
\pgfpathlineto{\pgfqpoint{2.814815in}{1.654576in}}%
\pgfpathlineto{\pgfqpoint{2.792617in}{1.654576in}}%
\pgfpathlineto{\pgfqpoint{2.770419in}{1.654576in}}%
\pgfpathlineto{\pgfqpoint{2.748221in}{1.722987in}}%
\pgfpathlineto{\pgfqpoint{2.726023in}{1.722987in}}%
\pgfpathlineto{\pgfqpoint{2.703825in}{1.722987in}}%
\pgfpathlineto{\pgfqpoint{2.681627in}{1.723704in}}%
\pgfpathlineto{\pgfqpoint{2.659429in}{1.723704in}}%
\pgfpathlineto{\pgfqpoint{2.637231in}{1.723704in}}%
\pgfpathlineto{\pgfqpoint{2.615033in}{1.723704in}}%
\pgfpathlineto{\pgfqpoint{2.592835in}{1.723704in}}%
\pgfpathlineto{\pgfqpoint{2.570637in}{1.723704in}}%
\pgfpathlineto{\pgfqpoint{2.548439in}{1.723704in}}%
\pgfpathlineto{\pgfqpoint{2.526241in}{1.735231in}}%
\pgfpathlineto{\pgfqpoint{2.504043in}{1.735231in}}%
\pgfpathlineto{\pgfqpoint{2.481845in}{1.787078in}}%
\pgfpathlineto{\pgfqpoint{2.459647in}{1.787078in}}%
\pgfpathlineto{\pgfqpoint{2.437449in}{1.787078in}}%
\pgfpathlineto{\pgfqpoint{2.415251in}{1.787078in}}%
\pgfpathlineto{\pgfqpoint{2.393053in}{1.787078in}}%
\pgfpathlineto{\pgfqpoint{2.370855in}{1.787078in}}%
\pgfpathlineto{\pgfqpoint{2.348657in}{1.787078in}}%
\pgfpathlineto{\pgfqpoint{2.326459in}{1.787078in}}%
\pgfpathlineto{\pgfqpoint{2.304261in}{1.787078in}}%
\pgfpathlineto{\pgfqpoint{2.282063in}{1.787078in}}%
\pgfpathlineto{\pgfqpoint{2.259865in}{1.787078in}}%
\pgfpathlineto{\pgfqpoint{2.237667in}{1.787078in}}%
\pgfpathlineto{\pgfqpoint{2.215469in}{1.787078in}}%
\pgfpathlineto{\pgfqpoint{2.193271in}{1.801478in}}%
\pgfpathlineto{\pgfqpoint{2.171073in}{1.801478in}}%
\pgfpathlineto{\pgfqpoint{2.148875in}{1.801478in}}%
\pgfpathlineto{\pgfqpoint{2.126677in}{1.801478in}}%
\pgfpathlineto{\pgfqpoint{2.104479in}{1.801478in}}%
\pgfpathlineto{\pgfqpoint{2.082281in}{1.801478in}}%
\pgfpathlineto{\pgfqpoint{2.060083in}{1.801478in}}%
\pgfpathlineto{\pgfqpoint{2.037884in}{1.801478in}}%
\pgfpathlineto{\pgfqpoint{2.015686in}{1.800079in}}%
\pgfpathlineto{\pgfqpoint{1.993488in}{1.800079in}}%
\pgfpathlineto{\pgfqpoint{1.971290in}{1.800079in}}%
\pgfpathlineto{\pgfqpoint{1.949092in}{1.800079in}}%
\pgfpathlineto{\pgfqpoint{1.926894in}{1.800079in}}%
\pgfpathlineto{\pgfqpoint{1.904696in}{1.800079in}}%
\pgfpathlineto{\pgfqpoint{1.882498in}{1.800079in}}%
\pgfpathlineto{\pgfqpoint{1.860300in}{1.800079in}}%
\pgfpathlineto{\pgfqpoint{1.838102in}{1.803699in}}%
\pgfpathlineto{\pgfqpoint{1.815904in}{1.833096in}}%
\pgfpathlineto{\pgfqpoint{1.793706in}{1.833096in}}%
\pgfpathlineto{\pgfqpoint{1.771508in}{1.859853in}}%
\pgfpathlineto{\pgfqpoint{1.749310in}{1.876406in}}%
\pgfpathlineto{\pgfqpoint{1.727112in}{1.876406in}}%
\pgfpathlineto{\pgfqpoint{1.704914in}{1.876406in}}%
\pgfpathlineto{\pgfqpoint{1.682716in}{1.876406in}}%
\pgfpathlineto{\pgfqpoint{1.660518in}{1.876406in}}%
\pgfpathlineto{\pgfqpoint{1.638320in}{1.876406in}}%
\pgfpathlineto{\pgfqpoint{1.616122in}{1.876406in}}%
\pgfpathlineto{\pgfqpoint{1.593924in}{1.876406in}}%
\pgfpathlineto{\pgfqpoint{1.571726in}{1.876406in}}%
\pgfpathlineto{\pgfqpoint{1.549528in}{1.876406in}}%
\pgfpathlineto{\pgfqpoint{1.527330in}{1.881532in}}%
\pgfpathlineto{\pgfqpoint{1.505132in}{1.881532in}}%
\pgfpathlineto{\pgfqpoint{1.482934in}{1.881532in}}%
\pgfpathlineto{\pgfqpoint{1.460736in}{1.891269in}}%
\pgfpathlineto{\pgfqpoint{1.438538in}{1.891269in}}%
\pgfpathlineto{\pgfqpoint{1.416340in}{1.898181in}}%
\pgfpathlineto{\pgfqpoint{1.394142in}{1.955010in}}%
\pgfpathlineto{\pgfqpoint{1.371944in}{1.956413in}}%
\pgfpathlineto{\pgfqpoint{1.349746in}{1.963595in}}%
\pgfpathlineto{\pgfqpoint{1.327548in}{1.963595in}}%
\pgfpathlineto{\pgfqpoint{1.305350in}{1.964531in}}%
\pgfpathlineto{\pgfqpoint{1.283152in}{1.962646in}}%
\pgfpathlineto{\pgfqpoint{1.260954in}{2.023924in}}%
\pgfpathlineto{\pgfqpoint{1.238756in}{2.022383in}}%
\pgfpathlineto{\pgfqpoint{1.216558in}{2.049124in}}%
\pgfpathlineto{\pgfqpoint{1.194360in}{2.049124in}}%
\pgfpathlineto{\pgfqpoint{1.172162in}{2.098781in}}%
\pgfpathlineto{\pgfqpoint{1.149964in}{2.127110in}}%
\pgfpathlineto{\pgfqpoint{1.127766in}{2.206975in}}%
\pgfpathlineto{\pgfqpoint{1.105568in}{2.414983in}}%
\pgfpathclose%
\pgfusepath{fill}%
\end{pgfscope}%
\begin{pgfscope}%
\pgfpathrectangle{\pgfqpoint{0.862500in}{0.375000in}}{\pgfqpoint{5.347500in}{2.265000in}}%
\pgfusepath{clip}%
\pgfsetbuttcap%
\pgfsetroundjoin%
\definecolor{currentfill}{rgb}{1.000000,0.498039,0.054902}%
\pgfsetfillcolor{currentfill}%
\pgfsetfillopacity{0.200000}%
\pgfsetlinewidth{0.000000pt}%
\definecolor{currentstroke}{rgb}{0.000000,0.000000,0.000000}%
\pgfsetstrokecolor{currentstroke}%
\pgfsetdash{}{0pt}%
\pgfpathmoveto{\pgfqpoint{1.105568in}{2.420424in}}%
\pgfpathlineto{\pgfqpoint{1.105568in}{2.490801in}}%
\pgfpathlineto{\pgfqpoint{1.127766in}{2.464394in}}%
\pgfpathlineto{\pgfqpoint{1.149964in}{2.386214in}}%
\pgfpathlineto{\pgfqpoint{1.172162in}{2.325418in}}%
\pgfpathlineto{\pgfqpoint{1.194360in}{2.287788in}}%
\pgfpathlineto{\pgfqpoint{1.216558in}{2.256228in}}%
\pgfpathlineto{\pgfqpoint{1.238756in}{2.249378in}}%
\pgfpathlineto{\pgfqpoint{1.260954in}{2.213566in}}%
\pgfpathlineto{\pgfqpoint{1.283152in}{2.213566in}}%
\pgfpathlineto{\pgfqpoint{1.305350in}{2.213566in}}%
\pgfpathlineto{\pgfqpoint{1.327548in}{2.213566in}}%
\pgfpathlineto{\pgfqpoint{1.349746in}{2.198217in}}%
\pgfpathlineto{\pgfqpoint{1.371944in}{2.198217in}}%
\pgfpathlineto{\pgfqpoint{1.394142in}{2.198217in}}%
\pgfpathlineto{\pgfqpoint{1.416340in}{2.198217in}}%
\pgfpathlineto{\pgfqpoint{1.438538in}{2.198217in}}%
\pgfpathlineto{\pgfqpoint{1.460736in}{2.057478in}}%
\pgfpathlineto{\pgfqpoint{1.482934in}{2.057478in}}%
\pgfpathlineto{\pgfqpoint{1.505132in}{2.057478in}}%
\pgfpathlineto{\pgfqpoint{1.527330in}{2.057478in}}%
\pgfpathlineto{\pgfqpoint{1.549528in}{2.057478in}}%
\pgfpathlineto{\pgfqpoint{1.571726in}{2.057478in}}%
\pgfpathlineto{\pgfqpoint{1.593924in}{2.057478in}}%
\pgfpathlineto{\pgfqpoint{1.616122in}{2.057478in}}%
\pgfpathlineto{\pgfqpoint{1.638320in}{2.057478in}}%
\pgfpathlineto{\pgfqpoint{1.660518in}{2.005691in}}%
\pgfpathlineto{\pgfqpoint{1.682716in}{2.005691in}}%
\pgfpathlineto{\pgfqpoint{1.704914in}{1.997236in}}%
\pgfpathlineto{\pgfqpoint{1.727112in}{1.997236in}}%
\pgfpathlineto{\pgfqpoint{1.749310in}{1.925670in}}%
\pgfpathlineto{\pgfqpoint{1.771508in}{1.908429in}}%
\pgfpathlineto{\pgfqpoint{1.793706in}{1.908429in}}%
\pgfpathlineto{\pgfqpoint{1.815904in}{1.860876in}}%
\pgfpathlineto{\pgfqpoint{1.838102in}{1.860876in}}%
\pgfpathlineto{\pgfqpoint{1.860300in}{1.860876in}}%
\pgfpathlineto{\pgfqpoint{1.882498in}{1.824343in}}%
\pgfpathlineto{\pgfqpoint{1.904696in}{1.797932in}}%
\pgfpathlineto{\pgfqpoint{1.926894in}{1.436302in}}%
\pgfpathlineto{\pgfqpoint{1.949092in}{1.436302in}}%
\pgfpathlineto{\pgfqpoint{1.971290in}{1.436302in}}%
\pgfpathlineto{\pgfqpoint{1.993488in}{1.417864in}}%
\pgfpathlineto{\pgfqpoint{2.015686in}{1.415377in}}%
\pgfpathlineto{\pgfqpoint{2.037884in}{1.415377in}}%
\pgfpathlineto{\pgfqpoint{2.060083in}{1.415377in}}%
\pgfpathlineto{\pgfqpoint{2.082281in}{1.415377in}}%
\pgfpathlineto{\pgfqpoint{2.104479in}{1.400824in}}%
\pgfpathlineto{\pgfqpoint{2.126677in}{1.400824in}}%
\pgfpathlineto{\pgfqpoint{2.148875in}{1.400824in}}%
\pgfpathlineto{\pgfqpoint{2.171073in}{1.400824in}}%
\pgfpathlineto{\pgfqpoint{2.193271in}{1.400824in}}%
\pgfpathlineto{\pgfqpoint{2.215469in}{1.400824in}}%
\pgfpathlineto{\pgfqpoint{2.237667in}{1.393664in}}%
\pgfpathlineto{\pgfqpoint{2.259865in}{1.393664in}}%
\pgfpathlineto{\pgfqpoint{2.282063in}{1.393664in}}%
\pgfpathlineto{\pgfqpoint{2.304261in}{1.393664in}}%
\pgfpathlineto{\pgfqpoint{2.326459in}{1.393664in}}%
\pgfpathlineto{\pgfqpoint{2.348657in}{1.377571in}}%
\pgfpathlineto{\pgfqpoint{2.370855in}{1.377571in}}%
\pgfpathlineto{\pgfqpoint{2.393053in}{1.300725in}}%
\pgfpathlineto{\pgfqpoint{2.415251in}{1.300410in}}%
\pgfpathlineto{\pgfqpoint{2.437449in}{1.300410in}}%
\pgfpathlineto{\pgfqpoint{2.459647in}{1.291495in}}%
\pgfpathlineto{\pgfqpoint{2.481845in}{1.202103in}}%
\pgfpathlineto{\pgfqpoint{2.504043in}{1.202103in}}%
\pgfpathlineto{\pgfqpoint{2.526241in}{1.202103in}}%
\pgfpathlineto{\pgfqpoint{2.548439in}{1.202103in}}%
\pgfpathlineto{\pgfqpoint{2.570637in}{1.199208in}}%
\pgfpathlineto{\pgfqpoint{2.592835in}{1.199192in}}%
\pgfpathlineto{\pgfqpoint{2.615033in}{1.199192in}}%
\pgfpathlineto{\pgfqpoint{2.637231in}{1.183121in}}%
\pgfpathlineto{\pgfqpoint{2.659429in}{1.183121in}}%
\pgfpathlineto{\pgfqpoint{2.681627in}{1.183121in}}%
\pgfpathlineto{\pgfqpoint{2.703825in}{1.183121in}}%
\pgfpathlineto{\pgfqpoint{2.726023in}{1.183121in}}%
\pgfpathlineto{\pgfqpoint{2.748221in}{1.183121in}}%
\pgfpathlineto{\pgfqpoint{2.770419in}{1.098817in}}%
\pgfpathlineto{\pgfqpoint{2.792617in}{1.092318in}}%
\pgfpathlineto{\pgfqpoint{2.814815in}{1.092288in}}%
\pgfpathlineto{\pgfqpoint{2.837013in}{1.092288in}}%
\pgfpathlineto{\pgfqpoint{2.859211in}{1.000888in}}%
\pgfpathlineto{\pgfqpoint{2.881409in}{1.000888in}}%
\pgfpathlineto{\pgfqpoint{2.903607in}{1.000888in}}%
\pgfpathlineto{\pgfqpoint{2.925805in}{1.000888in}}%
\pgfpathlineto{\pgfqpoint{2.948003in}{0.893712in}}%
\pgfpathlineto{\pgfqpoint{2.970201in}{0.893712in}}%
\pgfpathlineto{\pgfqpoint{2.992399in}{0.893712in}}%
\pgfpathlineto{\pgfqpoint{3.014597in}{0.893712in}}%
\pgfpathlineto{\pgfqpoint{3.036795in}{0.893712in}}%
\pgfpathlineto{\pgfqpoint{3.058993in}{0.893712in}}%
\pgfpathlineto{\pgfqpoint{3.081191in}{0.893712in}}%
\pgfpathlineto{\pgfqpoint{3.103389in}{0.893712in}}%
\pgfpathlineto{\pgfqpoint{3.125587in}{0.893712in}}%
\pgfpathlineto{\pgfqpoint{3.147785in}{0.893712in}}%
\pgfpathlineto{\pgfqpoint{3.169983in}{0.893712in}}%
\pgfpathlineto{\pgfqpoint{3.192181in}{0.893712in}}%
\pgfpathlineto{\pgfqpoint{3.214379in}{0.893712in}}%
\pgfpathlineto{\pgfqpoint{3.236577in}{0.893712in}}%
\pgfpathlineto{\pgfqpoint{3.258775in}{0.893712in}}%
\pgfpathlineto{\pgfqpoint{3.280973in}{0.893712in}}%
\pgfpathlineto{\pgfqpoint{3.303171in}{0.806391in}}%
\pgfpathlineto{\pgfqpoint{3.325369in}{0.806391in}}%
\pgfpathlineto{\pgfqpoint{3.347567in}{0.806391in}}%
\pgfpathlineto{\pgfqpoint{3.369765in}{0.806391in}}%
\pgfpathlineto{\pgfqpoint{3.391963in}{0.806391in}}%
\pgfpathlineto{\pgfqpoint{3.414161in}{0.806391in}}%
\pgfpathlineto{\pgfqpoint{3.436359in}{0.806391in}}%
\pgfpathlineto{\pgfqpoint{3.458557in}{0.806391in}}%
\pgfpathlineto{\pgfqpoint{3.480755in}{0.806391in}}%
\pgfpathlineto{\pgfqpoint{3.502953in}{0.806391in}}%
\pgfpathlineto{\pgfqpoint{3.525151in}{0.788674in}}%
\pgfpathlineto{\pgfqpoint{3.547349in}{0.788674in}}%
\pgfpathlineto{\pgfqpoint{3.569547in}{0.757175in}}%
\pgfpathlineto{\pgfqpoint{3.591745in}{0.757175in}}%
\pgfpathlineto{\pgfqpoint{3.613943in}{0.757175in}}%
\pgfpathlineto{\pgfqpoint{3.636141in}{0.757175in}}%
\pgfpathlineto{\pgfqpoint{3.658339in}{0.757175in}}%
\pgfpathlineto{\pgfqpoint{3.680537in}{0.757175in}}%
\pgfpathlineto{\pgfqpoint{3.702735in}{0.757175in}}%
\pgfpathlineto{\pgfqpoint{3.724933in}{0.757175in}}%
\pgfpathlineto{\pgfqpoint{3.747131in}{0.757175in}}%
\pgfpathlineto{\pgfqpoint{3.769329in}{0.757175in}}%
\pgfpathlineto{\pgfqpoint{3.791527in}{0.757175in}}%
\pgfpathlineto{\pgfqpoint{3.813725in}{0.757175in}}%
\pgfpathlineto{\pgfqpoint{3.835923in}{0.757175in}}%
\pgfpathlineto{\pgfqpoint{3.858121in}{0.757175in}}%
\pgfpathlineto{\pgfqpoint{3.880319in}{0.757175in}}%
\pgfpathlineto{\pgfqpoint{3.902517in}{0.757175in}}%
\pgfpathlineto{\pgfqpoint{3.924715in}{0.757175in}}%
\pgfpathlineto{\pgfqpoint{3.946913in}{0.757175in}}%
\pgfpathlineto{\pgfqpoint{3.969111in}{0.757175in}}%
\pgfpathlineto{\pgfqpoint{3.991309in}{0.757175in}}%
\pgfpathlineto{\pgfqpoint{4.013507in}{0.757175in}}%
\pgfpathlineto{\pgfqpoint{4.035705in}{0.757175in}}%
\pgfpathlineto{\pgfqpoint{4.057903in}{0.757175in}}%
\pgfpathlineto{\pgfqpoint{4.080101in}{0.757175in}}%
\pgfpathlineto{\pgfqpoint{4.102299in}{0.757175in}}%
\pgfpathlineto{\pgfqpoint{4.124497in}{0.757175in}}%
\pgfpathlineto{\pgfqpoint{4.146695in}{0.757175in}}%
\pgfpathlineto{\pgfqpoint{4.168893in}{0.757175in}}%
\pgfpathlineto{\pgfqpoint{4.191091in}{0.757175in}}%
\pgfpathlineto{\pgfqpoint{4.213289in}{0.757175in}}%
\pgfpathlineto{\pgfqpoint{4.235487in}{0.757175in}}%
\pgfpathlineto{\pgfqpoint{4.257685in}{0.757175in}}%
\pgfpathlineto{\pgfqpoint{4.279883in}{0.757175in}}%
\pgfpathlineto{\pgfqpoint{4.302081in}{0.757175in}}%
\pgfpathlineto{\pgfqpoint{4.324279in}{0.757175in}}%
\pgfpathlineto{\pgfqpoint{4.346477in}{0.757175in}}%
\pgfpathlineto{\pgfqpoint{4.368675in}{0.757175in}}%
\pgfpathlineto{\pgfqpoint{4.390873in}{0.757175in}}%
\pgfpathlineto{\pgfqpoint{4.413071in}{0.757175in}}%
\pgfpathlineto{\pgfqpoint{4.435269in}{0.757175in}}%
\pgfpathlineto{\pgfqpoint{4.457467in}{0.757175in}}%
\pgfpathlineto{\pgfqpoint{4.479665in}{0.757175in}}%
\pgfpathlineto{\pgfqpoint{4.501863in}{0.757175in}}%
\pgfpathlineto{\pgfqpoint{4.524061in}{0.757175in}}%
\pgfpathlineto{\pgfqpoint{4.546259in}{0.757175in}}%
\pgfpathlineto{\pgfqpoint{4.568457in}{0.757175in}}%
\pgfpathlineto{\pgfqpoint{4.590655in}{0.757175in}}%
\pgfpathlineto{\pgfqpoint{4.612853in}{0.701852in}}%
\pgfpathlineto{\pgfqpoint{4.635051in}{0.701852in}}%
\pgfpathlineto{\pgfqpoint{4.657249in}{0.701852in}}%
\pgfpathlineto{\pgfqpoint{4.679447in}{0.701852in}}%
\pgfpathlineto{\pgfqpoint{4.701645in}{0.701852in}}%
\pgfpathlineto{\pgfqpoint{4.723843in}{0.701852in}}%
\pgfpathlineto{\pgfqpoint{4.746041in}{0.701852in}}%
\pgfpathlineto{\pgfqpoint{4.768239in}{0.701852in}}%
\pgfpathlineto{\pgfqpoint{4.790437in}{0.673223in}}%
\pgfpathlineto{\pgfqpoint{4.812635in}{0.673223in}}%
\pgfpathlineto{\pgfqpoint{4.834833in}{0.673223in}}%
\pgfpathlineto{\pgfqpoint{4.857031in}{0.673223in}}%
\pgfpathlineto{\pgfqpoint{4.879229in}{0.673223in}}%
\pgfpathlineto{\pgfqpoint{4.901427in}{0.673223in}}%
\pgfpathlineto{\pgfqpoint{4.923625in}{0.673223in}}%
\pgfpathlineto{\pgfqpoint{4.945823in}{0.673223in}}%
\pgfpathlineto{\pgfqpoint{4.968021in}{0.673223in}}%
\pgfpathlineto{\pgfqpoint{4.990219in}{0.673223in}}%
\pgfpathlineto{\pgfqpoint{5.012417in}{0.673223in}}%
\pgfpathlineto{\pgfqpoint{5.034616in}{0.673223in}}%
\pgfpathlineto{\pgfqpoint{5.056814in}{0.673223in}}%
\pgfpathlineto{\pgfqpoint{5.079012in}{0.673223in}}%
\pgfpathlineto{\pgfqpoint{5.101210in}{0.673223in}}%
\pgfpathlineto{\pgfqpoint{5.123408in}{0.673223in}}%
\pgfpathlineto{\pgfqpoint{5.145606in}{0.673223in}}%
\pgfpathlineto{\pgfqpoint{5.167804in}{0.673223in}}%
\pgfpathlineto{\pgfqpoint{5.190002in}{0.673223in}}%
\pgfpathlineto{\pgfqpoint{5.212200in}{0.673223in}}%
\pgfpathlineto{\pgfqpoint{5.234398in}{0.639581in}}%
\pgfpathlineto{\pgfqpoint{5.256596in}{0.639581in}}%
\pgfpathlineto{\pgfqpoint{5.278794in}{0.639581in}}%
\pgfpathlineto{\pgfqpoint{5.300992in}{0.639581in}}%
\pgfpathlineto{\pgfqpoint{5.323190in}{0.639581in}}%
\pgfpathlineto{\pgfqpoint{5.345388in}{0.639581in}}%
\pgfpathlineto{\pgfqpoint{5.367586in}{0.639581in}}%
\pgfpathlineto{\pgfqpoint{5.389784in}{0.639581in}}%
\pgfpathlineto{\pgfqpoint{5.411982in}{0.639581in}}%
\pgfpathlineto{\pgfqpoint{5.434180in}{0.639581in}}%
\pgfpathlineto{\pgfqpoint{5.456378in}{0.639581in}}%
\pgfpathlineto{\pgfqpoint{5.478576in}{0.639581in}}%
\pgfpathlineto{\pgfqpoint{5.500774in}{0.639581in}}%
\pgfpathlineto{\pgfqpoint{5.522972in}{0.639581in}}%
\pgfpathlineto{\pgfqpoint{5.545170in}{0.639581in}}%
\pgfpathlineto{\pgfqpoint{5.567368in}{0.639581in}}%
\pgfpathlineto{\pgfqpoint{5.589566in}{0.639581in}}%
\pgfpathlineto{\pgfqpoint{5.611764in}{0.639581in}}%
\pgfpathlineto{\pgfqpoint{5.633962in}{0.639581in}}%
\pgfpathlineto{\pgfqpoint{5.656160in}{0.639581in}}%
\pgfpathlineto{\pgfqpoint{5.678358in}{0.639581in}}%
\pgfpathlineto{\pgfqpoint{5.700556in}{0.639581in}}%
\pgfpathlineto{\pgfqpoint{5.722754in}{0.626981in}}%
\pgfpathlineto{\pgfqpoint{5.744952in}{0.626981in}}%
\pgfpathlineto{\pgfqpoint{5.767150in}{0.626981in}}%
\pgfpathlineto{\pgfqpoint{5.789348in}{0.626981in}}%
\pgfpathlineto{\pgfqpoint{5.811546in}{0.626981in}}%
\pgfpathlineto{\pgfqpoint{5.833744in}{0.626981in}}%
\pgfpathlineto{\pgfqpoint{5.855942in}{0.626981in}}%
\pgfpathlineto{\pgfqpoint{5.878140in}{0.626981in}}%
\pgfpathlineto{\pgfqpoint{5.900338in}{0.626981in}}%
\pgfpathlineto{\pgfqpoint{5.922536in}{0.586791in}}%
\pgfpathlineto{\pgfqpoint{5.944734in}{0.586791in}}%
\pgfpathlineto{\pgfqpoint{5.966932in}{0.586791in}}%
\pgfpathlineto{\pgfqpoint{5.966932in}{0.477955in}}%
\pgfpathlineto{\pgfqpoint{5.966932in}{0.477955in}}%
\pgfpathlineto{\pgfqpoint{5.944734in}{0.477955in}}%
\pgfpathlineto{\pgfqpoint{5.922536in}{0.477955in}}%
\pgfpathlineto{\pgfqpoint{5.900338in}{0.501008in}}%
\pgfpathlineto{\pgfqpoint{5.878140in}{0.501008in}}%
\pgfpathlineto{\pgfqpoint{5.855942in}{0.501008in}}%
\pgfpathlineto{\pgfqpoint{5.833744in}{0.501008in}}%
\pgfpathlineto{\pgfqpoint{5.811546in}{0.501008in}}%
\pgfpathlineto{\pgfqpoint{5.789348in}{0.501008in}}%
\pgfpathlineto{\pgfqpoint{5.767150in}{0.501008in}}%
\pgfpathlineto{\pgfqpoint{5.744952in}{0.501008in}}%
\pgfpathlineto{\pgfqpoint{5.722754in}{0.501008in}}%
\pgfpathlineto{\pgfqpoint{5.700556in}{0.514804in}}%
\pgfpathlineto{\pgfqpoint{5.678358in}{0.514804in}}%
\pgfpathlineto{\pgfqpoint{5.656160in}{0.514804in}}%
\pgfpathlineto{\pgfqpoint{5.633962in}{0.514804in}}%
\pgfpathlineto{\pgfqpoint{5.611764in}{0.514804in}}%
\pgfpathlineto{\pgfqpoint{5.589566in}{0.514804in}}%
\pgfpathlineto{\pgfqpoint{5.567368in}{0.514804in}}%
\pgfpathlineto{\pgfqpoint{5.545170in}{0.514804in}}%
\pgfpathlineto{\pgfqpoint{5.522972in}{0.514804in}}%
\pgfpathlineto{\pgfqpoint{5.500774in}{0.514804in}}%
\pgfpathlineto{\pgfqpoint{5.478576in}{0.514804in}}%
\pgfpathlineto{\pgfqpoint{5.456378in}{0.514804in}}%
\pgfpathlineto{\pgfqpoint{5.434180in}{0.514804in}}%
\pgfpathlineto{\pgfqpoint{5.411982in}{0.514804in}}%
\pgfpathlineto{\pgfqpoint{5.389784in}{0.514804in}}%
\pgfpathlineto{\pgfqpoint{5.367586in}{0.514804in}}%
\pgfpathlineto{\pgfqpoint{5.345388in}{0.514804in}}%
\pgfpathlineto{\pgfqpoint{5.323190in}{0.514804in}}%
\pgfpathlineto{\pgfqpoint{5.300992in}{0.514804in}}%
\pgfpathlineto{\pgfqpoint{5.278794in}{0.514804in}}%
\pgfpathlineto{\pgfqpoint{5.256596in}{0.514804in}}%
\pgfpathlineto{\pgfqpoint{5.234398in}{0.514804in}}%
\pgfpathlineto{\pgfqpoint{5.212200in}{0.571125in}}%
\pgfpathlineto{\pgfqpoint{5.190002in}{0.571125in}}%
\pgfpathlineto{\pgfqpoint{5.167804in}{0.571125in}}%
\pgfpathlineto{\pgfqpoint{5.145606in}{0.571125in}}%
\pgfpathlineto{\pgfqpoint{5.123408in}{0.571125in}}%
\pgfpathlineto{\pgfqpoint{5.101210in}{0.571125in}}%
\pgfpathlineto{\pgfqpoint{5.079012in}{0.571125in}}%
\pgfpathlineto{\pgfqpoint{5.056814in}{0.571125in}}%
\pgfpathlineto{\pgfqpoint{5.034616in}{0.571125in}}%
\pgfpathlineto{\pgfqpoint{5.012417in}{0.571125in}}%
\pgfpathlineto{\pgfqpoint{4.990219in}{0.571125in}}%
\pgfpathlineto{\pgfqpoint{4.968021in}{0.571125in}}%
\pgfpathlineto{\pgfqpoint{4.945823in}{0.571125in}}%
\pgfpathlineto{\pgfqpoint{4.923625in}{0.571125in}}%
\pgfpathlineto{\pgfqpoint{4.901427in}{0.571125in}}%
\pgfpathlineto{\pgfqpoint{4.879229in}{0.571125in}}%
\pgfpathlineto{\pgfqpoint{4.857031in}{0.571125in}}%
\pgfpathlineto{\pgfqpoint{4.834833in}{0.571125in}}%
\pgfpathlineto{\pgfqpoint{4.812635in}{0.571125in}}%
\pgfpathlineto{\pgfqpoint{4.790437in}{0.571125in}}%
\pgfpathlineto{\pgfqpoint{4.768239in}{0.596814in}}%
\pgfpathlineto{\pgfqpoint{4.746041in}{0.596814in}}%
\pgfpathlineto{\pgfqpoint{4.723843in}{0.596814in}}%
\pgfpathlineto{\pgfqpoint{4.701645in}{0.596814in}}%
\pgfpathlineto{\pgfqpoint{4.679447in}{0.596814in}}%
\pgfpathlineto{\pgfqpoint{4.657249in}{0.596814in}}%
\pgfpathlineto{\pgfqpoint{4.635051in}{0.596814in}}%
\pgfpathlineto{\pgfqpoint{4.612853in}{0.596814in}}%
\pgfpathlineto{\pgfqpoint{4.590655in}{0.630357in}}%
\pgfpathlineto{\pgfqpoint{4.568457in}{0.630357in}}%
\pgfpathlineto{\pgfqpoint{4.546259in}{0.630357in}}%
\pgfpathlineto{\pgfqpoint{4.524061in}{0.630357in}}%
\pgfpathlineto{\pgfqpoint{4.501863in}{0.630357in}}%
\pgfpathlineto{\pgfqpoint{4.479665in}{0.630357in}}%
\pgfpathlineto{\pgfqpoint{4.457467in}{0.630357in}}%
\pgfpathlineto{\pgfqpoint{4.435269in}{0.630357in}}%
\pgfpathlineto{\pgfqpoint{4.413071in}{0.630357in}}%
\pgfpathlineto{\pgfqpoint{4.390873in}{0.630357in}}%
\pgfpathlineto{\pgfqpoint{4.368675in}{0.630357in}}%
\pgfpathlineto{\pgfqpoint{4.346477in}{0.630357in}}%
\pgfpathlineto{\pgfqpoint{4.324279in}{0.630357in}}%
\pgfpathlineto{\pgfqpoint{4.302081in}{0.630357in}}%
\pgfpathlineto{\pgfqpoint{4.279883in}{0.630357in}}%
\pgfpathlineto{\pgfqpoint{4.257685in}{0.630357in}}%
\pgfpathlineto{\pgfqpoint{4.235487in}{0.630357in}}%
\pgfpathlineto{\pgfqpoint{4.213289in}{0.630357in}}%
\pgfpathlineto{\pgfqpoint{4.191091in}{0.630357in}}%
\pgfpathlineto{\pgfqpoint{4.168893in}{0.630357in}}%
\pgfpathlineto{\pgfqpoint{4.146695in}{0.630357in}}%
\pgfpathlineto{\pgfqpoint{4.124497in}{0.630357in}}%
\pgfpathlineto{\pgfqpoint{4.102299in}{0.630357in}}%
\pgfpathlineto{\pgfqpoint{4.080101in}{0.630357in}}%
\pgfpathlineto{\pgfqpoint{4.057903in}{0.630357in}}%
\pgfpathlineto{\pgfqpoint{4.035705in}{0.630357in}}%
\pgfpathlineto{\pgfqpoint{4.013507in}{0.630357in}}%
\pgfpathlineto{\pgfqpoint{3.991309in}{0.630357in}}%
\pgfpathlineto{\pgfqpoint{3.969111in}{0.630357in}}%
\pgfpathlineto{\pgfqpoint{3.946913in}{0.630357in}}%
\pgfpathlineto{\pgfqpoint{3.924715in}{0.630357in}}%
\pgfpathlineto{\pgfqpoint{3.902517in}{0.630357in}}%
\pgfpathlineto{\pgfqpoint{3.880319in}{0.630357in}}%
\pgfpathlineto{\pgfqpoint{3.858121in}{0.630357in}}%
\pgfpathlineto{\pgfqpoint{3.835923in}{0.630357in}}%
\pgfpathlineto{\pgfqpoint{3.813725in}{0.630357in}}%
\pgfpathlineto{\pgfqpoint{3.791527in}{0.630357in}}%
\pgfpathlineto{\pgfqpoint{3.769329in}{0.630357in}}%
\pgfpathlineto{\pgfqpoint{3.747131in}{0.630357in}}%
\pgfpathlineto{\pgfqpoint{3.724933in}{0.630357in}}%
\pgfpathlineto{\pgfqpoint{3.702735in}{0.630357in}}%
\pgfpathlineto{\pgfqpoint{3.680537in}{0.630357in}}%
\pgfpathlineto{\pgfqpoint{3.658339in}{0.630357in}}%
\pgfpathlineto{\pgfqpoint{3.636141in}{0.630357in}}%
\pgfpathlineto{\pgfqpoint{3.613943in}{0.630357in}}%
\pgfpathlineto{\pgfqpoint{3.591745in}{0.630357in}}%
\pgfpathlineto{\pgfqpoint{3.569547in}{0.630357in}}%
\pgfpathlineto{\pgfqpoint{3.547349in}{0.676274in}}%
\pgfpathlineto{\pgfqpoint{3.525151in}{0.676274in}}%
\pgfpathlineto{\pgfqpoint{3.502953in}{0.682377in}}%
\pgfpathlineto{\pgfqpoint{3.480755in}{0.682377in}}%
\pgfpathlineto{\pgfqpoint{3.458557in}{0.682377in}}%
\pgfpathlineto{\pgfqpoint{3.436359in}{0.682377in}}%
\pgfpathlineto{\pgfqpoint{3.414161in}{0.682377in}}%
\pgfpathlineto{\pgfqpoint{3.391963in}{0.682377in}}%
\pgfpathlineto{\pgfqpoint{3.369765in}{0.682377in}}%
\pgfpathlineto{\pgfqpoint{3.347567in}{0.682377in}}%
\pgfpathlineto{\pgfqpoint{3.325369in}{0.682377in}}%
\pgfpathlineto{\pgfqpoint{3.303171in}{0.682377in}}%
\pgfpathlineto{\pgfqpoint{3.280973in}{0.694397in}}%
\pgfpathlineto{\pgfqpoint{3.258775in}{0.694397in}}%
\pgfpathlineto{\pgfqpoint{3.236577in}{0.694397in}}%
\pgfpathlineto{\pgfqpoint{3.214379in}{0.694397in}}%
\pgfpathlineto{\pgfqpoint{3.192181in}{0.694397in}}%
\pgfpathlineto{\pgfqpoint{3.169983in}{0.694397in}}%
\pgfpathlineto{\pgfqpoint{3.147785in}{0.694397in}}%
\pgfpathlineto{\pgfqpoint{3.125587in}{0.694397in}}%
\pgfpathlineto{\pgfqpoint{3.103389in}{0.694397in}}%
\pgfpathlineto{\pgfqpoint{3.081191in}{0.694397in}}%
\pgfpathlineto{\pgfqpoint{3.058993in}{0.694397in}}%
\pgfpathlineto{\pgfqpoint{3.036795in}{0.694397in}}%
\pgfpathlineto{\pgfqpoint{3.014597in}{0.694397in}}%
\pgfpathlineto{\pgfqpoint{2.992399in}{0.694397in}}%
\pgfpathlineto{\pgfqpoint{2.970201in}{0.694397in}}%
\pgfpathlineto{\pgfqpoint{2.948003in}{0.694397in}}%
\pgfpathlineto{\pgfqpoint{2.925805in}{0.775383in}}%
\pgfpathlineto{\pgfqpoint{2.903607in}{0.775383in}}%
\pgfpathlineto{\pgfqpoint{2.881409in}{0.775383in}}%
\pgfpathlineto{\pgfqpoint{2.859211in}{0.775383in}}%
\pgfpathlineto{\pgfqpoint{2.837013in}{0.789234in}}%
\pgfpathlineto{\pgfqpoint{2.814815in}{0.789234in}}%
\pgfpathlineto{\pgfqpoint{2.792617in}{0.789421in}}%
\pgfpathlineto{\pgfqpoint{2.770419in}{0.833429in}}%
\pgfpathlineto{\pgfqpoint{2.748221in}{0.928811in}}%
\pgfpathlineto{\pgfqpoint{2.726023in}{0.928811in}}%
\pgfpathlineto{\pgfqpoint{2.703825in}{0.928811in}}%
\pgfpathlineto{\pgfqpoint{2.681627in}{0.928811in}}%
\pgfpathlineto{\pgfqpoint{2.659429in}{0.928811in}}%
\pgfpathlineto{\pgfqpoint{2.637231in}{0.928811in}}%
\pgfpathlineto{\pgfqpoint{2.615033in}{0.993553in}}%
\pgfpathlineto{\pgfqpoint{2.592835in}{0.993553in}}%
\pgfpathlineto{\pgfqpoint{2.570637in}{0.993656in}}%
\pgfpathlineto{\pgfqpoint{2.548439in}{1.008703in}}%
\pgfpathlineto{\pgfqpoint{2.526241in}{1.008703in}}%
\pgfpathlineto{\pgfqpoint{2.504043in}{1.008703in}}%
\pgfpathlineto{\pgfqpoint{2.481845in}{1.008703in}}%
\pgfpathlineto{\pgfqpoint{2.459647in}{1.091499in}}%
\pgfpathlineto{\pgfqpoint{2.437449in}{1.101775in}}%
\pgfpathlineto{\pgfqpoint{2.415251in}{1.101775in}}%
\pgfpathlineto{\pgfqpoint{2.393053in}{1.102654in}}%
\pgfpathlineto{\pgfqpoint{2.370855in}{1.206182in}}%
\pgfpathlineto{\pgfqpoint{2.348657in}{1.206182in}}%
\pgfpathlineto{\pgfqpoint{2.326459in}{1.220230in}}%
\pgfpathlineto{\pgfqpoint{2.304261in}{1.220230in}}%
\pgfpathlineto{\pgfqpoint{2.282063in}{1.220230in}}%
\pgfpathlineto{\pgfqpoint{2.259865in}{1.220230in}}%
\pgfpathlineto{\pgfqpoint{2.237667in}{1.220230in}}%
\pgfpathlineto{\pgfqpoint{2.215469in}{1.254659in}}%
\pgfpathlineto{\pgfqpoint{2.193271in}{1.254659in}}%
\pgfpathlineto{\pgfqpoint{2.171073in}{1.254659in}}%
\pgfpathlineto{\pgfqpoint{2.148875in}{1.254659in}}%
\pgfpathlineto{\pgfqpoint{2.126677in}{1.254659in}}%
\pgfpathlineto{\pgfqpoint{2.104479in}{1.254659in}}%
\pgfpathlineto{\pgfqpoint{2.082281in}{1.294599in}}%
\pgfpathlineto{\pgfqpoint{2.060083in}{1.294599in}}%
\pgfpathlineto{\pgfqpoint{2.037884in}{1.294599in}}%
\pgfpathlineto{\pgfqpoint{2.015686in}{1.294599in}}%
\pgfpathlineto{\pgfqpoint{1.993488in}{1.320022in}}%
\pgfpathlineto{\pgfqpoint{1.971290in}{1.367544in}}%
\pgfpathlineto{\pgfqpoint{1.949092in}{1.367544in}}%
\pgfpathlineto{\pgfqpoint{1.926894in}{1.367544in}}%
\pgfpathlineto{\pgfqpoint{1.904696in}{1.493108in}}%
\pgfpathlineto{\pgfqpoint{1.882498in}{1.554747in}}%
\pgfpathlineto{\pgfqpoint{1.860300in}{1.596178in}}%
\pgfpathlineto{\pgfqpoint{1.838102in}{1.596178in}}%
\pgfpathlineto{\pgfqpoint{1.815904in}{1.596178in}}%
\pgfpathlineto{\pgfqpoint{1.793706in}{1.708786in}}%
\pgfpathlineto{\pgfqpoint{1.771508in}{1.708786in}}%
\pgfpathlineto{\pgfqpoint{1.749310in}{1.742287in}}%
\pgfpathlineto{\pgfqpoint{1.727112in}{1.829633in}}%
\pgfpathlineto{\pgfqpoint{1.704914in}{1.829633in}}%
\pgfpathlineto{\pgfqpoint{1.682716in}{1.832262in}}%
\pgfpathlineto{\pgfqpoint{1.660518in}{1.832262in}}%
\pgfpathlineto{\pgfqpoint{1.638320in}{1.879701in}}%
\pgfpathlineto{\pgfqpoint{1.616122in}{1.879701in}}%
\pgfpathlineto{\pgfqpoint{1.593924in}{1.879701in}}%
\pgfpathlineto{\pgfqpoint{1.571726in}{1.879701in}}%
\pgfpathlineto{\pgfqpoint{1.549528in}{1.879701in}}%
\pgfpathlineto{\pgfqpoint{1.527330in}{1.879701in}}%
\pgfpathlineto{\pgfqpoint{1.505132in}{1.879701in}}%
\pgfpathlineto{\pgfqpoint{1.482934in}{1.879701in}}%
\pgfpathlineto{\pgfqpoint{1.460736in}{1.879701in}}%
\pgfpathlineto{\pgfqpoint{1.438538in}{1.948144in}}%
\pgfpathlineto{\pgfqpoint{1.416340in}{1.948144in}}%
\pgfpathlineto{\pgfqpoint{1.394142in}{1.948144in}}%
\pgfpathlineto{\pgfqpoint{1.371944in}{1.948144in}}%
\pgfpathlineto{\pgfqpoint{1.349746in}{1.948144in}}%
\pgfpathlineto{\pgfqpoint{1.327548in}{1.978117in}}%
\pgfpathlineto{\pgfqpoint{1.305350in}{1.978117in}}%
\pgfpathlineto{\pgfqpoint{1.283152in}{1.978117in}}%
\pgfpathlineto{\pgfqpoint{1.260954in}{1.978117in}}%
\pgfpathlineto{\pgfqpoint{1.238756in}{2.069191in}}%
\pgfpathlineto{\pgfqpoint{1.216558in}{2.086042in}}%
\pgfpathlineto{\pgfqpoint{1.194360in}{2.126097in}}%
\pgfpathlineto{\pgfqpoint{1.172162in}{2.208187in}}%
\pgfpathlineto{\pgfqpoint{1.149964in}{2.343946in}}%
\pgfpathlineto{\pgfqpoint{1.127766in}{2.409215in}}%
\pgfpathlineto{\pgfqpoint{1.105568in}{2.420424in}}%
\pgfpathclose%
\pgfusepath{fill}%
\end{pgfscope}%
\begin{pgfscope}%
\pgfpathrectangle{\pgfqpoint{0.862500in}{0.375000in}}{\pgfqpoint{5.347500in}{2.265000in}}%
\pgfusepath{clip}%
\pgfsetbuttcap%
\pgfsetroundjoin%
\definecolor{currentfill}{rgb}{0.172549,0.627451,0.172549}%
\pgfsetfillcolor{currentfill}%
\pgfsetfillopacity{0.200000}%
\pgfsetlinewidth{0.000000pt}%
\definecolor{currentstroke}{rgb}{0.000000,0.000000,0.000000}%
\pgfsetstrokecolor{currentstroke}%
\pgfsetdash{}{0pt}%
\pgfpathmoveto{\pgfqpoint{1.105568in}{2.286990in}}%
\pgfpathlineto{\pgfqpoint{1.105568in}{2.410185in}}%
\pgfpathlineto{\pgfqpoint{1.127766in}{2.311821in}}%
\pgfpathlineto{\pgfqpoint{1.149964in}{2.287665in}}%
\pgfpathlineto{\pgfqpoint{1.172162in}{2.221393in}}%
\pgfpathlineto{\pgfqpoint{1.194360in}{2.200397in}}%
\pgfpathlineto{\pgfqpoint{1.216558in}{2.193823in}}%
\pgfpathlineto{\pgfqpoint{1.238756in}{2.146017in}}%
\pgfpathlineto{\pgfqpoint{1.260954in}{2.111192in}}%
\pgfpathlineto{\pgfqpoint{1.283152in}{2.111192in}}%
\pgfpathlineto{\pgfqpoint{1.305350in}{2.082363in}}%
\pgfpathlineto{\pgfqpoint{1.327548in}{2.082363in}}%
\pgfpathlineto{\pgfqpoint{1.349746in}{2.082363in}}%
\pgfpathlineto{\pgfqpoint{1.371944in}{2.047725in}}%
\pgfpathlineto{\pgfqpoint{1.394142in}{2.047725in}}%
\pgfpathlineto{\pgfqpoint{1.416340in}{2.047725in}}%
\pgfpathlineto{\pgfqpoint{1.438538in}{2.038053in}}%
\pgfpathlineto{\pgfqpoint{1.460736in}{2.038053in}}%
\pgfpathlineto{\pgfqpoint{1.482934in}{2.038053in}}%
\pgfpathlineto{\pgfqpoint{1.505132in}{2.038053in}}%
\pgfpathlineto{\pgfqpoint{1.527330in}{2.038053in}}%
\pgfpathlineto{\pgfqpoint{1.549528in}{2.012366in}}%
\pgfpathlineto{\pgfqpoint{1.571726in}{2.012366in}}%
\pgfpathlineto{\pgfqpoint{1.593924in}{2.012366in}}%
\pgfpathlineto{\pgfqpoint{1.616122in}{1.990562in}}%
\pgfpathlineto{\pgfqpoint{1.638320in}{1.990562in}}%
\pgfpathlineto{\pgfqpoint{1.660518in}{1.960120in}}%
\pgfpathlineto{\pgfqpoint{1.682716in}{1.960120in}}%
\pgfpathlineto{\pgfqpoint{1.704914in}{1.954892in}}%
\pgfpathlineto{\pgfqpoint{1.727112in}{1.954892in}}%
\pgfpathlineto{\pgfqpoint{1.749310in}{1.950098in}}%
\pgfpathlineto{\pgfqpoint{1.771508in}{1.950098in}}%
\pgfpathlineto{\pgfqpoint{1.793706in}{1.950098in}}%
\pgfpathlineto{\pgfqpoint{1.815904in}{1.950098in}}%
\pgfpathlineto{\pgfqpoint{1.838102in}{1.944069in}}%
\pgfpathlineto{\pgfqpoint{1.860300in}{1.892148in}}%
\pgfpathlineto{\pgfqpoint{1.882498in}{1.891623in}}%
\pgfpathlineto{\pgfqpoint{1.904696in}{1.886564in}}%
\pgfpathlineto{\pgfqpoint{1.926894in}{1.769729in}}%
\pgfpathlineto{\pgfqpoint{1.949092in}{1.769729in}}%
\pgfpathlineto{\pgfqpoint{1.971290in}{1.769729in}}%
\pgfpathlineto{\pgfqpoint{1.993488in}{1.769729in}}%
\pgfpathlineto{\pgfqpoint{2.015686in}{1.769729in}}%
\pgfpathlineto{\pgfqpoint{2.037884in}{1.769729in}}%
\pgfpathlineto{\pgfqpoint{2.060083in}{1.769729in}}%
\pgfpathlineto{\pgfqpoint{2.082281in}{1.769729in}}%
\pgfpathlineto{\pgfqpoint{2.104479in}{1.769729in}}%
\pgfpathlineto{\pgfqpoint{2.126677in}{1.769729in}}%
\pgfpathlineto{\pgfqpoint{2.148875in}{1.710880in}}%
\pgfpathlineto{\pgfqpoint{2.171073in}{1.704287in}}%
\pgfpathlineto{\pgfqpoint{2.193271in}{1.704287in}}%
\pgfpathlineto{\pgfqpoint{2.215469in}{1.704287in}}%
\pgfpathlineto{\pgfqpoint{2.237667in}{1.704287in}}%
\pgfpathlineto{\pgfqpoint{2.259865in}{1.703442in}}%
\pgfpathlineto{\pgfqpoint{2.282063in}{1.606471in}}%
\pgfpathlineto{\pgfqpoint{2.304261in}{1.606471in}}%
\pgfpathlineto{\pgfqpoint{2.326459in}{1.512798in}}%
\pgfpathlineto{\pgfqpoint{2.348657in}{1.445040in}}%
\pgfpathlineto{\pgfqpoint{2.370855in}{1.445040in}}%
\pgfpathlineto{\pgfqpoint{2.393053in}{1.436845in}}%
\pgfpathlineto{\pgfqpoint{2.415251in}{1.436845in}}%
\pgfpathlineto{\pgfqpoint{2.437449in}{1.082573in}}%
\pgfpathlineto{\pgfqpoint{2.459647in}{1.082573in}}%
\pgfpathlineto{\pgfqpoint{2.481845in}{1.082573in}}%
\pgfpathlineto{\pgfqpoint{2.504043in}{1.082573in}}%
\pgfpathlineto{\pgfqpoint{2.526241in}{1.078205in}}%
\pgfpathlineto{\pgfqpoint{2.548439in}{0.929015in}}%
\pgfpathlineto{\pgfqpoint{2.570637in}{0.929015in}}%
\pgfpathlineto{\pgfqpoint{2.592835in}{0.929015in}}%
\pgfpathlineto{\pgfqpoint{2.615033in}{0.929015in}}%
\pgfpathlineto{\pgfqpoint{2.637231in}{0.929015in}}%
\pgfpathlineto{\pgfqpoint{2.659429in}{0.929015in}}%
\pgfpathlineto{\pgfqpoint{2.681627in}{0.929015in}}%
\pgfpathlineto{\pgfqpoint{2.703825in}{0.929015in}}%
\pgfpathlineto{\pgfqpoint{2.726023in}{0.929015in}}%
\pgfpathlineto{\pgfqpoint{2.748221in}{0.929015in}}%
\pgfpathlineto{\pgfqpoint{2.770419in}{0.929015in}}%
\pgfpathlineto{\pgfqpoint{2.792617in}{0.929015in}}%
\pgfpathlineto{\pgfqpoint{2.814815in}{0.929015in}}%
\pgfpathlineto{\pgfqpoint{2.837013in}{0.929015in}}%
\pgfpathlineto{\pgfqpoint{2.859211in}{0.929015in}}%
\pgfpathlineto{\pgfqpoint{2.881409in}{0.929015in}}%
\pgfpathlineto{\pgfqpoint{2.903607in}{0.929015in}}%
\pgfpathlineto{\pgfqpoint{2.925805in}{0.929015in}}%
\pgfpathlineto{\pgfqpoint{2.948003in}{0.929015in}}%
\pgfpathlineto{\pgfqpoint{2.970201in}{0.929015in}}%
\pgfpathlineto{\pgfqpoint{2.992399in}{0.929015in}}%
\pgfpathlineto{\pgfqpoint{3.014597in}{0.929015in}}%
\pgfpathlineto{\pgfqpoint{3.036795in}{0.929015in}}%
\pgfpathlineto{\pgfqpoint{3.058993in}{0.929015in}}%
\pgfpathlineto{\pgfqpoint{3.081191in}{0.929015in}}%
\pgfpathlineto{\pgfqpoint{3.103389in}{0.929015in}}%
\pgfpathlineto{\pgfqpoint{3.125587in}{0.929015in}}%
\pgfpathlineto{\pgfqpoint{3.147785in}{0.929015in}}%
\pgfpathlineto{\pgfqpoint{3.169983in}{0.929015in}}%
\pgfpathlineto{\pgfqpoint{3.192181in}{0.929015in}}%
\pgfpathlineto{\pgfqpoint{3.214379in}{0.929015in}}%
\pgfpathlineto{\pgfqpoint{3.236577in}{0.929015in}}%
\pgfpathlineto{\pgfqpoint{3.258775in}{0.929015in}}%
\pgfpathlineto{\pgfqpoint{3.280973in}{0.925872in}}%
\pgfpathlineto{\pgfqpoint{3.303171in}{0.925872in}}%
\pgfpathlineto{\pgfqpoint{3.325369in}{0.925872in}}%
\pgfpathlineto{\pgfqpoint{3.347567in}{0.925872in}}%
\pgfpathlineto{\pgfqpoint{3.369765in}{0.925872in}}%
\pgfpathlineto{\pgfqpoint{3.391963in}{0.925872in}}%
\pgfpathlineto{\pgfqpoint{3.414161in}{0.925872in}}%
\pgfpathlineto{\pgfqpoint{3.436359in}{0.925872in}}%
\pgfpathlineto{\pgfqpoint{3.458557in}{0.912003in}}%
\pgfpathlineto{\pgfqpoint{3.480755in}{0.912003in}}%
\pgfpathlineto{\pgfqpoint{3.502953in}{0.912003in}}%
\pgfpathlineto{\pgfqpoint{3.525151in}{0.912003in}}%
\pgfpathlineto{\pgfqpoint{3.547349in}{0.912003in}}%
\pgfpathlineto{\pgfqpoint{3.569547in}{0.912003in}}%
\pgfpathlineto{\pgfqpoint{3.591745in}{0.912003in}}%
\pgfpathlineto{\pgfqpoint{3.613943in}{0.912003in}}%
\pgfpathlineto{\pgfqpoint{3.636141in}{0.912003in}}%
\pgfpathlineto{\pgfqpoint{3.658339in}{0.912003in}}%
\pgfpathlineto{\pgfqpoint{3.680537in}{0.912003in}}%
\pgfpathlineto{\pgfqpoint{3.702735in}{0.912003in}}%
\pgfpathlineto{\pgfqpoint{3.724933in}{0.881892in}}%
\pgfpathlineto{\pgfqpoint{3.747131in}{0.830834in}}%
\pgfpathlineto{\pgfqpoint{3.769329in}{0.830834in}}%
\pgfpathlineto{\pgfqpoint{3.791527in}{0.830834in}}%
\pgfpathlineto{\pgfqpoint{3.813725in}{0.830834in}}%
\pgfpathlineto{\pgfqpoint{3.835923in}{0.830834in}}%
\pgfpathlineto{\pgfqpoint{3.858121in}{0.830834in}}%
\pgfpathlineto{\pgfqpoint{3.880319in}{0.830834in}}%
\pgfpathlineto{\pgfqpoint{3.902517in}{0.830834in}}%
\pgfpathlineto{\pgfqpoint{3.924715in}{0.830834in}}%
\pgfpathlineto{\pgfqpoint{3.946913in}{0.696435in}}%
\pgfpathlineto{\pgfqpoint{3.969111in}{0.696435in}}%
\pgfpathlineto{\pgfqpoint{3.991309in}{0.696435in}}%
\pgfpathlineto{\pgfqpoint{4.013507in}{0.696435in}}%
\pgfpathlineto{\pgfqpoint{4.035705in}{0.683109in}}%
\pgfpathlineto{\pgfqpoint{4.057903in}{0.683109in}}%
\pgfpathlineto{\pgfqpoint{4.080101in}{0.683109in}}%
\pgfpathlineto{\pgfqpoint{4.102299in}{0.642778in}}%
\pgfpathlineto{\pgfqpoint{4.124497in}{0.642778in}}%
\pgfpathlineto{\pgfqpoint{4.146695in}{0.642778in}}%
\pgfpathlineto{\pgfqpoint{4.168893in}{0.642778in}}%
\pgfpathlineto{\pgfqpoint{4.191091in}{0.642778in}}%
\pgfpathlineto{\pgfqpoint{4.213289in}{0.642778in}}%
\pgfpathlineto{\pgfqpoint{4.235487in}{0.642778in}}%
\pgfpathlineto{\pgfqpoint{4.257685in}{0.642778in}}%
\pgfpathlineto{\pgfqpoint{4.279883in}{0.642778in}}%
\pgfpathlineto{\pgfqpoint{4.302081in}{0.642778in}}%
\pgfpathlineto{\pgfqpoint{4.324279in}{0.642778in}}%
\pgfpathlineto{\pgfqpoint{4.346477in}{0.642778in}}%
\pgfpathlineto{\pgfqpoint{4.368675in}{0.642778in}}%
\pgfpathlineto{\pgfqpoint{4.390873in}{0.642778in}}%
\pgfpathlineto{\pgfqpoint{4.413071in}{0.642778in}}%
\pgfpathlineto{\pgfqpoint{4.435269in}{0.642778in}}%
\pgfpathlineto{\pgfqpoint{4.457467in}{0.642778in}}%
\pgfpathlineto{\pgfqpoint{4.479665in}{0.642778in}}%
\pgfpathlineto{\pgfqpoint{4.501863in}{0.642778in}}%
\pgfpathlineto{\pgfqpoint{4.524061in}{0.642778in}}%
\pgfpathlineto{\pgfqpoint{4.546259in}{0.642778in}}%
\pgfpathlineto{\pgfqpoint{4.568457in}{0.642778in}}%
\pgfpathlineto{\pgfqpoint{4.590655in}{0.642778in}}%
\pgfpathlineto{\pgfqpoint{4.612853in}{0.642778in}}%
\pgfpathlineto{\pgfqpoint{4.635051in}{0.642778in}}%
\pgfpathlineto{\pgfqpoint{4.657249in}{0.642778in}}%
\pgfpathlineto{\pgfqpoint{4.679447in}{0.642778in}}%
\pgfpathlineto{\pgfqpoint{4.701645in}{0.642778in}}%
\pgfpathlineto{\pgfqpoint{4.723843in}{0.642778in}}%
\pgfpathlineto{\pgfqpoint{4.746041in}{0.642778in}}%
\pgfpathlineto{\pgfqpoint{4.768239in}{0.642778in}}%
\pgfpathlineto{\pgfqpoint{4.790437in}{0.642778in}}%
\pgfpathlineto{\pgfqpoint{4.812635in}{0.642778in}}%
\pgfpathlineto{\pgfqpoint{4.834833in}{0.642778in}}%
\pgfpathlineto{\pgfqpoint{4.857031in}{0.642778in}}%
\pgfpathlineto{\pgfqpoint{4.879229in}{0.642778in}}%
\pgfpathlineto{\pgfqpoint{4.901427in}{0.642778in}}%
\pgfpathlineto{\pgfqpoint{4.923625in}{0.642778in}}%
\pgfpathlineto{\pgfqpoint{4.945823in}{0.642778in}}%
\pgfpathlineto{\pgfqpoint{4.968021in}{0.642778in}}%
\pgfpathlineto{\pgfqpoint{4.990219in}{0.642778in}}%
\pgfpathlineto{\pgfqpoint{5.012417in}{0.642778in}}%
\pgfpathlineto{\pgfqpoint{5.034616in}{0.642778in}}%
\pgfpathlineto{\pgfqpoint{5.056814in}{0.642778in}}%
\pgfpathlineto{\pgfqpoint{5.079012in}{0.642778in}}%
\pgfpathlineto{\pgfqpoint{5.101210in}{0.642778in}}%
\pgfpathlineto{\pgfqpoint{5.123408in}{0.642778in}}%
\pgfpathlineto{\pgfqpoint{5.145606in}{0.642778in}}%
\pgfpathlineto{\pgfqpoint{5.167804in}{0.642778in}}%
\pgfpathlineto{\pgfqpoint{5.190002in}{0.642778in}}%
\pgfpathlineto{\pgfqpoint{5.212200in}{0.642778in}}%
\pgfpathlineto{\pgfqpoint{5.234398in}{0.642778in}}%
\pgfpathlineto{\pgfqpoint{5.256596in}{0.642778in}}%
\pgfpathlineto{\pgfqpoint{5.278794in}{0.642778in}}%
\pgfpathlineto{\pgfqpoint{5.300992in}{0.642778in}}%
\pgfpathlineto{\pgfqpoint{5.323190in}{0.642778in}}%
\pgfpathlineto{\pgfqpoint{5.345388in}{0.642778in}}%
\pgfpathlineto{\pgfqpoint{5.367586in}{0.642778in}}%
\pgfpathlineto{\pgfqpoint{5.389784in}{0.641189in}}%
\pgfpathlineto{\pgfqpoint{5.411982in}{0.641189in}}%
\pgfpathlineto{\pgfqpoint{5.434180in}{0.641189in}}%
\pgfpathlineto{\pgfqpoint{5.456378in}{0.641189in}}%
\pgfpathlineto{\pgfqpoint{5.478576in}{0.641189in}}%
\pgfpathlineto{\pgfqpoint{5.500774in}{0.641189in}}%
\pgfpathlineto{\pgfqpoint{5.522972in}{0.641189in}}%
\pgfpathlineto{\pgfqpoint{5.545170in}{0.641189in}}%
\pgfpathlineto{\pgfqpoint{5.567368in}{0.641189in}}%
\pgfpathlineto{\pgfqpoint{5.589566in}{0.641189in}}%
\pgfpathlineto{\pgfqpoint{5.611764in}{0.641189in}}%
\pgfpathlineto{\pgfqpoint{5.633962in}{0.641189in}}%
\pgfpathlineto{\pgfqpoint{5.656160in}{0.641189in}}%
\pgfpathlineto{\pgfqpoint{5.678358in}{0.641189in}}%
\pgfpathlineto{\pgfqpoint{5.700556in}{0.641189in}}%
\pgfpathlineto{\pgfqpoint{5.722754in}{0.641189in}}%
\pgfpathlineto{\pgfqpoint{5.744952in}{0.641189in}}%
\pgfpathlineto{\pgfqpoint{5.767150in}{0.641189in}}%
\pgfpathlineto{\pgfqpoint{5.789348in}{0.641189in}}%
\pgfpathlineto{\pgfqpoint{5.811546in}{0.641189in}}%
\pgfpathlineto{\pgfqpoint{5.833744in}{0.641189in}}%
\pgfpathlineto{\pgfqpoint{5.855942in}{0.641189in}}%
\pgfpathlineto{\pgfqpoint{5.878140in}{0.641189in}}%
\pgfpathlineto{\pgfqpoint{5.900338in}{0.641189in}}%
\pgfpathlineto{\pgfqpoint{5.922536in}{0.641189in}}%
\pgfpathlineto{\pgfqpoint{5.944734in}{0.641189in}}%
\pgfpathlineto{\pgfqpoint{5.966932in}{0.641189in}}%
\pgfpathlineto{\pgfqpoint{5.966932in}{0.524822in}}%
\pgfpathlineto{\pgfqpoint{5.966932in}{0.524822in}}%
\pgfpathlineto{\pgfqpoint{5.944734in}{0.524822in}}%
\pgfpathlineto{\pgfqpoint{5.922536in}{0.524822in}}%
\pgfpathlineto{\pgfqpoint{5.900338in}{0.524822in}}%
\pgfpathlineto{\pgfqpoint{5.878140in}{0.524822in}}%
\pgfpathlineto{\pgfqpoint{5.855942in}{0.524822in}}%
\pgfpathlineto{\pgfqpoint{5.833744in}{0.524822in}}%
\pgfpathlineto{\pgfqpoint{5.811546in}{0.524822in}}%
\pgfpathlineto{\pgfqpoint{5.789348in}{0.524822in}}%
\pgfpathlineto{\pgfqpoint{5.767150in}{0.524822in}}%
\pgfpathlineto{\pgfqpoint{5.744952in}{0.524822in}}%
\pgfpathlineto{\pgfqpoint{5.722754in}{0.524822in}}%
\pgfpathlineto{\pgfqpoint{5.700556in}{0.524822in}}%
\pgfpathlineto{\pgfqpoint{5.678358in}{0.524822in}}%
\pgfpathlineto{\pgfqpoint{5.656160in}{0.524822in}}%
\pgfpathlineto{\pgfqpoint{5.633962in}{0.524822in}}%
\pgfpathlineto{\pgfqpoint{5.611764in}{0.524822in}}%
\pgfpathlineto{\pgfqpoint{5.589566in}{0.524822in}}%
\pgfpathlineto{\pgfqpoint{5.567368in}{0.524822in}}%
\pgfpathlineto{\pgfqpoint{5.545170in}{0.524822in}}%
\pgfpathlineto{\pgfqpoint{5.522972in}{0.524822in}}%
\pgfpathlineto{\pgfqpoint{5.500774in}{0.524822in}}%
\pgfpathlineto{\pgfqpoint{5.478576in}{0.524822in}}%
\pgfpathlineto{\pgfqpoint{5.456378in}{0.524822in}}%
\pgfpathlineto{\pgfqpoint{5.434180in}{0.524822in}}%
\pgfpathlineto{\pgfqpoint{5.411982in}{0.524822in}}%
\pgfpathlineto{\pgfqpoint{5.389784in}{0.524822in}}%
\pgfpathlineto{\pgfqpoint{5.367586in}{0.527087in}}%
\pgfpathlineto{\pgfqpoint{5.345388in}{0.527087in}}%
\pgfpathlineto{\pgfqpoint{5.323190in}{0.527087in}}%
\pgfpathlineto{\pgfqpoint{5.300992in}{0.527087in}}%
\pgfpathlineto{\pgfqpoint{5.278794in}{0.527087in}}%
\pgfpathlineto{\pgfqpoint{5.256596in}{0.527087in}}%
\pgfpathlineto{\pgfqpoint{5.234398in}{0.527087in}}%
\pgfpathlineto{\pgfqpoint{5.212200in}{0.527087in}}%
\pgfpathlineto{\pgfqpoint{5.190002in}{0.527087in}}%
\pgfpathlineto{\pgfqpoint{5.167804in}{0.527087in}}%
\pgfpathlineto{\pgfqpoint{5.145606in}{0.527087in}}%
\pgfpathlineto{\pgfqpoint{5.123408in}{0.527087in}}%
\pgfpathlineto{\pgfqpoint{5.101210in}{0.527087in}}%
\pgfpathlineto{\pgfqpoint{5.079012in}{0.527087in}}%
\pgfpathlineto{\pgfqpoint{5.056814in}{0.527087in}}%
\pgfpathlineto{\pgfqpoint{5.034616in}{0.527087in}}%
\pgfpathlineto{\pgfqpoint{5.012417in}{0.527087in}}%
\pgfpathlineto{\pgfqpoint{4.990219in}{0.527087in}}%
\pgfpathlineto{\pgfqpoint{4.968021in}{0.527087in}}%
\pgfpathlineto{\pgfqpoint{4.945823in}{0.527087in}}%
\pgfpathlineto{\pgfqpoint{4.923625in}{0.527087in}}%
\pgfpathlineto{\pgfqpoint{4.901427in}{0.527087in}}%
\pgfpathlineto{\pgfqpoint{4.879229in}{0.527087in}}%
\pgfpathlineto{\pgfqpoint{4.857031in}{0.527087in}}%
\pgfpathlineto{\pgfqpoint{4.834833in}{0.527087in}}%
\pgfpathlineto{\pgfqpoint{4.812635in}{0.527087in}}%
\pgfpathlineto{\pgfqpoint{4.790437in}{0.527087in}}%
\pgfpathlineto{\pgfqpoint{4.768239in}{0.527087in}}%
\pgfpathlineto{\pgfqpoint{4.746041in}{0.527087in}}%
\pgfpathlineto{\pgfqpoint{4.723843in}{0.527087in}}%
\pgfpathlineto{\pgfqpoint{4.701645in}{0.527087in}}%
\pgfpathlineto{\pgfqpoint{4.679447in}{0.527087in}}%
\pgfpathlineto{\pgfqpoint{4.657249in}{0.527087in}}%
\pgfpathlineto{\pgfqpoint{4.635051in}{0.527087in}}%
\pgfpathlineto{\pgfqpoint{4.612853in}{0.527087in}}%
\pgfpathlineto{\pgfqpoint{4.590655in}{0.527087in}}%
\pgfpathlineto{\pgfqpoint{4.568457in}{0.527087in}}%
\pgfpathlineto{\pgfqpoint{4.546259in}{0.527087in}}%
\pgfpathlineto{\pgfqpoint{4.524061in}{0.527087in}}%
\pgfpathlineto{\pgfqpoint{4.501863in}{0.527087in}}%
\pgfpathlineto{\pgfqpoint{4.479665in}{0.527087in}}%
\pgfpathlineto{\pgfqpoint{4.457467in}{0.527087in}}%
\pgfpathlineto{\pgfqpoint{4.435269in}{0.527087in}}%
\pgfpathlineto{\pgfqpoint{4.413071in}{0.527087in}}%
\pgfpathlineto{\pgfqpoint{4.390873in}{0.527087in}}%
\pgfpathlineto{\pgfqpoint{4.368675in}{0.527087in}}%
\pgfpathlineto{\pgfqpoint{4.346477in}{0.527087in}}%
\pgfpathlineto{\pgfqpoint{4.324279in}{0.527087in}}%
\pgfpathlineto{\pgfqpoint{4.302081in}{0.527087in}}%
\pgfpathlineto{\pgfqpoint{4.279883in}{0.527087in}}%
\pgfpathlineto{\pgfqpoint{4.257685in}{0.527087in}}%
\pgfpathlineto{\pgfqpoint{4.235487in}{0.527087in}}%
\pgfpathlineto{\pgfqpoint{4.213289in}{0.527087in}}%
\pgfpathlineto{\pgfqpoint{4.191091in}{0.527087in}}%
\pgfpathlineto{\pgfqpoint{4.168893in}{0.527087in}}%
\pgfpathlineto{\pgfqpoint{4.146695in}{0.527087in}}%
\pgfpathlineto{\pgfqpoint{4.124497in}{0.527087in}}%
\pgfpathlineto{\pgfqpoint{4.102299in}{0.527087in}}%
\pgfpathlineto{\pgfqpoint{4.080101in}{0.543725in}}%
\pgfpathlineto{\pgfqpoint{4.057903in}{0.543725in}}%
\pgfpathlineto{\pgfqpoint{4.035705in}{0.543725in}}%
\pgfpathlineto{\pgfqpoint{4.013507in}{0.580759in}}%
\pgfpathlineto{\pgfqpoint{3.991309in}{0.580759in}}%
\pgfpathlineto{\pgfqpoint{3.969111in}{0.580759in}}%
\pgfpathlineto{\pgfqpoint{3.946913in}{0.580759in}}%
\pgfpathlineto{\pgfqpoint{3.924715in}{0.610252in}}%
\pgfpathlineto{\pgfqpoint{3.902517in}{0.610252in}}%
\pgfpathlineto{\pgfqpoint{3.880319in}{0.610252in}}%
\pgfpathlineto{\pgfqpoint{3.858121in}{0.610252in}}%
\pgfpathlineto{\pgfqpoint{3.835923in}{0.610252in}}%
\pgfpathlineto{\pgfqpoint{3.813725in}{0.610252in}}%
\pgfpathlineto{\pgfqpoint{3.791527in}{0.610252in}}%
\pgfpathlineto{\pgfqpoint{3.769329in}{0.610252in}}%
\pgfpathlineto{\pgfqpoint{3.747131in}{0.610252in}}%
\pgfpathlineto{\pgfqpoint{3.724933in}{0.612644in}}%
\pgfpathlineto{\pgfqpoint{3.702735in}{0.675334in}}%
\pgfpathlineto{\pgfqpoint{3.680537in}{0.675334in}}%
\pgfpathlineto{\pgfqpoint{3.658339in}{0.675334in}}%
\pgfpathlineto{\pgfqpoint{3.636141in}{0.675334in}}%
\pgfpathlineto{\pgfqpoint{3.613943in}{0.675334in}}%
\pgfpathlineto{\pgfqpoint{3.591745in}{0.675334in}}%
\pgfpathlineto{\pgfqpoint{3.569547in}{0.675334in}}%
\pgfpathlineto{\pgfqpoint{3.547349in}{0.675334in}}%
\pgfpathlineto{\pgfqpoint{3.525151in}{0.675334in}}%
\pgfpathlineto{\pgfqpoint{3.502953in}{0.675334in}}%
\pgfpathlineto{\pgfqpoint{3.480755in}{0.675334in}}%
\pgfpathlineto{\pgfqpoint{3.458557in}{0.675334in}}%
\pgfpathlineto{\pgfqpoint{3.436359in}{0.691396in}}%
\pgfpathlineto{\pgfqpoint{3.414161in}{0.691396in}}%
\pgfpathlineto{\pgfqpoint{3.391963in}{0.691396in}}%
\pgfpathlineto{\pgfqpoint{3.369765in}{0.691396in}}%
\pgfpathlineto{\pgfqpoint{3.347567in}{0.691396in}}%
\pgfpathlineto{\pgfqpoint{3.325369in}{0.691396in}}%
\pgfpathlineto{\pgfqpoint{3.303171in}{0.691396in}}%
\pgfpathlineto{\pgfqpoint{3.280973in}{0.691396in}}%
\pgfpathlineto{\pgfqpoint{3.258775in}{0.709430in}}%
\pgfpathlineto{\pgfqpoint{3.236577in}{0.709430in}}%
\pgfpathlineto{\pgfqpoint{3.214379in}{0.709430in}}%
\pgfpathlineto{\pgfqpoint{3.192181in}{0.709430in}}%
\pgfpathlineto{\pgfqpoint{3.169983in}{0.709430in}}%
\pgfpathlineto{\pgfqpoint{3.147785in}{0.709430in}}%
\pgfpathlineto{\pgfqpoint{3.125587in}{0.709430in}}%
\pgfpathlineto{\pgfqpoint{3.103389in}{0.709430in}}%
\pgfpathlineto{\pgfqpoint{3.081191in}{0.709430in}}%
\pgfpathlineto{\pgfqpoint{3.058993in}{0.709430in}}%
\pgfpathlineto{\pgfqpoint{3.036795in}{0.709430in}}%
\pgfpathlineto{\pgfqpoint{3.014597in}{0.709430in}}%
\pgfpathlineto{\pgfqpoint{2.992399in}{0.709430in}}%
\pgfpathlineto{\pgfqpoint{2.970201in}{0.709430in}}%
\pgfpathlineto{\pgfqpoint{2.948003in}{0.709430in}}%
\pgfpathlineto{\pgfqpoint{2.925805in}{0.709430in}}%
\pgfpathlineto{\pgfqpoint{2.903607in}{0.709430in}}%
\pgfpathlineto{\pgfqpoint{2.881409in}{0.709430in}}%
\pgfpathlineto{\pgfqpoint{2.859211in}{0.709430in}}%
\pgfpathlineto{\pgfqpoint{2.837013in}{0.709430in}}%
\pgfpathlineto{\pgfqpoint{2.814815in}{0.709430in}}%
\pgfpathlineto{\pgfqpoint{2.792617in}{0.709430in}}%
\pgfpathlineto{\pgfqpoint{2.770419in}{0.709430in}}%
\pgfpathlineto{\pgfqpoint{2.748221in}{0.709430in}}%
\pgfpathlineto{\pgfqpoint{2.726023in}{0.709430in}}%
\pgfpathlineto{\pgfqpoint{2.703825in}{0.709430in}}%
\pgfpathlineto{\pgfqpoint{2.681627in}{0.709430in}}%
\pgfpathlineto{\pgfqpoint{2.659429in}{0.709430in}}%
\pgfpathlineto{\pgfqpoint{2.637231in}{0.709430in}}%
\pgfpathlineto{\pgfqpoint{2.615033in}{0.709430in}}%
\pgfpathlineto{\pgfqpoint{2.592835in}{0.709430in}}%
\pgfpathlineto{\pgfqpoint{2.570637in}{0.709430in}}%
\pgfpathlineto{\pgfqpoint{2.548439in}{0.709430in}}%
\pgfpathlineto{\pgfqpoint{2.526241in}{0.836507in}}%
\pgfpathlineto{\pgfqpoint{2.504043in}{0.862864in}}%
\pgfpathlineto{\pgfqpoint{2.481845in}{0.862864in}}%
\pgfpathlineto{\pgfqpoint{2.459647in}{0.862864in}}%
\pgfpathlineto{\pgfqpoint{2.437449in}{0.862864in}}%
\pgfpathlineto{\pgfqpoint{2.415251in}{0.969152in}}%
\pgfpathlineto{\pgfqpoint{2.393053in}{0.969152in}}%
\pgfpathlineto{\pgfqpoint{2.370855in}{1.051808in}}%
\pgfpathlineto{\pgfqpoint{2.348657in}{1.051808in}}%
\pgfpathlineto{\pgfqpoint{2.326459in}{1.111119in}}%
\pgfpathlineto{\pgfqpoint{2.304261in}{1.324733in}}%
\pgfpathlineto{\pgfqpoint{2.282063in}{1.324733in}}%
\pgfpathlineto{\pgfqpoint{2.259865in}{1.369916in}}%
\pgfpathlineto{\pgfqpoint{2.237667in}{1.377140in}}%
\pgfpathlineto{\pgfqpoint{2.215469in}{1.377140in}}%
\pgfpathlineto{\pgfqpoint{2.193271in}{1.377140in}}%
\pgfpathlineto{\pgfqpoint{2.171073in}{1.377140in}}%
\pgfpathlineto{\pgfqpoint{2.148875in}{1.424050in}}%
\pgfpathlineto{\pgfqpoint{2.126677in}{1.492269in}}%
\pgfpathlineto{\pgfqpoint{2.104479in}{1.492269in}}%
\pgfpathlineto{\pgfqpoint{2.082281in}{1.492269in}}%
\pgfpathlineto{\pgfqpoint{2.060083in}{1.492269in}}%
\pgfpathlineto{\pgfqpoint{2.037884in}{1.492269in}}%
\pgfpathlineto{\pgfqpoint{2.015686in}{1.492269in}}%
\pgfpathlineto{\pgfqpoint{1.993488in}{1.492269in}}%
\pgfpathlineto{\pgfqpoint{1.971290in}{1.492269in}}%
\pgfpathlineto{\pgfqpoint{1.949092in}{1.492269in}}%
\pgfpathlineto{\pgfqpoint{1.926894in}{1.492269in}}%
\pgfpathlineto{\pgfqpoint{1.904696in}{1.651667in}}%
\pgfpathlineto{\pgfqpoint{1.882498in}{1.680078in}}%
\pgfpathlineto{\pgfqpoint{1.860300in}{1.680929in}}%
\pgfpathlineto{\pgfqpoint{1.838102in}{1.774091in}}%
\pgfpathlineto{\pgfqpoint{1.815904in}{1.784348in}}%
\pgfpathlineto{\pgfqpoint{1.793706in}{1.784348in}}%
\pgfpathlineto{\pgfqpoint{1.771508in}{1.784348in}}%
\pgfpathlineto{\pgfqpoint{1.749310in}{1.784348in}}%
\pgfpathlineto{\pgfqpoint{1.727112in}{1.817555in}}%
\pgfpathlineto{\pgfqpoint{1.704914in}{1.817555in}}%
\pgfpathlineto{\pgfqpoint{1.682716in}{1.819959in}}%
\pgfpathlineto{\pgfqpoint{1.660518in}{1.819959in}}%
\pgfpathlineto{\pgfqpoint{1.638320in}{1.892298in}}%
\pgfpathlineto{\pgfqpoint{1.616122in}{1.892298in}}%
\pgfpathlineto{\pgfqpoint{1.593924in}{1.915201in}}%
\pgfpathlineto{\pgfqpoint{1.571726in}{1.915201in}}%
\pgfpathlineto{\pgfqpoint{1.549528in}{1.915201in}}%
\pgfpathlineto{\pgfqpoint{1.527330in}{1.988534in}}%
\pgfpathlineto{\pgfqpoint{1.505132in}{1.988534in}}%
\pgfpathlineto{\pgfqpoint{1.482934in}{1.988534in}}%
\pgfpathlineto{\pgfqpoint{1.460736in}{1.988534in}}%
\pgfpathlineto{\pgfqpoint{1.438538in}{1.988534in}}%
\pgfpathlineto{\pgfqpoint{1.416340in}{1.993016in}}%
\pgfpathlineto{\pgfqpoint{1.394142in}{1.993016in}}%
\pgfpathlineto{\pgfqpoint{1.371944in}{1.993016in}}%
\pgfpathlineto{\pgfqpoint{1.349746in}{1.999272in}}%
\pgfpathlineto{\pgfqpoint{1.327548in}{1.999272in}}%
\pgfpathlineto{\pgfqpoint{1.305350in}{1.999272in}}%
\pgfpathlineto{\pgfqpoint{1.283152in}{2.017213in}}%
\pgfpathlineto{\pgfqpoint{1.260954in}{2.017213in}}%
\pgfpathlineto{\pgfqpoint{1.238756in}{2.071661in}}%
\pgfpathlineto{\pgfqpoint{1.216558in}{2.084468in}}%
\pgfpathlineto{\pgfqpoint{1.194360in}{2.096842in}}%
\pgfpathlineto{\pgfqpoint{1.172162in}{2.152474in}}%
\pgfpathlineto{\pgfqpoint{1.149964in}{2.215210in}}%
\pgfpathlineto{\pgfqpoint{1.127766in}{2.240099in}}%
\pgfpathlineto{\pgfqpoint{1.105568in}{2.286990in}}%
\pgfpathclose%
\pgfusepath{fill}%
\end{pgfscope}%
\begin{pgfscope}%
\pgfpathrectangle{\pgfqpoint{0.862500in}{0.375000in}}{\pgfqpoint{5.347500in}{2.265000in}}%
\pgfusepath{clip}%
\pgfsetbuttcap%
\pgfsetroundjoin%
\definecolor{currentfill}{rgb}{0.839216,0.152941,0.156863}%
\pgfsetfillcolor{currentfill}%
\pgfsetfillopacity{0.200000}%
\pgfsetlinewidth{0.000000pt}%
\definecolor{currentstroke}{rgb}{0.000000,0.000000,0.000000}%
\pgfsetstrokecolor{currentstroke}%
\pgfsetdash{}{0pt}%
\pgfpathmoveto{\pgfqpoint{1.105568in}{2.321169in}}%
\pgfpathlineto{\pgfqpoint{1.105568in}{2.454955in}}%
\pgfpathlineto{\pgfqpoint{1.127766in}{2.321361in}}%
\pgfpathlineto{\pgfqpoint{1.149964in}{2.278065in}}%
\pgfpathlineto{\pgfqpoint{1.172162in}{2.271931in}}%
\pgfpathlineto{\pgfqpoint{1.194360in}{2.271931in}}%
\pgfpathlineto{\pgfqpoint{1.216558in}{2.115699in}}%
\pgfpathlineto{\pgfqpoint{1.238756in}{2.115699in}}%
\pgfpathlineto{\pgfqpoint{1.260954in}{2.115699in}}%
\pgfpathlineto{\pgfqpoint{1.283152in}{2.115699in}}%
\pgfpathlineto{\pgfqpoint{1.305350in}{2.115699in}}%
\pgfpathlineto{\pgfqpoint{1.327548in}{2.115699in}}%
\pgfpathlineto{\pgfqpoint{1.349746in}{2.115699in}}%
\pgfpathlineto{\pgfqpoint{1.371944in}{2.115699in}}%
\pgfpathlineto{\pgfqpoint{1.394142in}{2.115699in}}%
\pgfpathlineto{\pgfqpoint{1.416340in}{2.115699in}}%
\pgfpathlineto{\pgfqpoint{1.438538in}{2.115699in}}%
\pgfpathlineto{\pgfqpoint{1.460736in}{2.108994in}}%
\pgfpathlineto{\pgfqpoint{1.482934in}{2.102053in}}%
\pgfpathlineto{\pgfqpoint{1.505132in}{2.102053in}}%
\pgfpathlineto{\pgfqpoint{1.527330in}{2.102053in}}%
\pgfpathlineto{\pgfqpoint{1.549528in}{1.956630in}}%
\pgfpathlineto{\pgfqpoint{1.571726in}{1.920988in}}%
\pgfpathlineto{\pgfqpoint{1.593924in}{1.920710in}}%
\pgfpathlineto{\pgfqpoint{1.616122in}{1.790681in}}%
\pgfpathlineto{\pgfqpoint{1.638320in}{1.790627in}}%
\pgfpathlineto{\pgfqpoint{1.660518in}{1.786625in}}%
\pgfpathlineto{\pgfqpoint{1.682716in}{1.782537in}}%
\pgfpathlineto{\pgfqpoint{1.704914in}{1.780038in}}%
\pgfpathlineto{\pgfqpoint{1.727112in}{1.780038in}}%
\pgfpathlineto{\pgfqpoint{1.749310in}{1.780038in}}%
\pgfpathlineto{\pgfqpoint{1.771508in}{1.779184in}}%
\pgfpathlineto{\pgfqpoint{1.793706in}{1.709067in}}%
\pgfpathlineto{\pgfqpoint{1.815904in}{1.708595in}}%
\pgfpathlineto{\pgfqpoint{1.838102in}{1.708595in}}%
\pgfpathlineto{\pgfqpoint{1.860300in}{1.708595in}}%
\pgfpathlineto{\pgfqpoint{1.882498in}{1.700509in}}%
\pgfpathlineto{\pgfqpoint{1.904696in}{1.670330in}}%
\pgfpathlineto{\pgfqpoint{1.926894in}{1.566794in}}%
\pgfpathlineto{\pgfqpoint{1.949092in}{1.566794in}}%
\pgfpathlineto{\pgfqpoint{1.971290in}{1.492845in}}%
\pgfpathlineto{\pgfqpoint{1.993488in}{1.492845in}}%
\pgfpathlineto{\pgfqpoint{2.015686in}{1.492845in}}%
\pgfpathlineto{\pgfqpoint{2.037884in}{1.492845in}}%
\pgfpathlineto{\pgfqpoint{2.060083in}{1.492845in}}%
\pgfpathlineto{\pgfqpoint{2.082281in}{1.450162in}}%
\pgfpathlineto{\pgfqpoint{2.104479in}{1.449805in}}%
\pgfpathlineto{\pgfqpoint{2.126677in}{1.449805in}}%
\pgfpathlineto{\pgfqpoint{2.148875in}{1.449805in}}%
\pgfpathlineto{\pgfqpoint{2.171073in}{1.363791in}}%
\pgfpathlineto{\pgfqpoint{2.193271in}{1.363791in}}%
\pgfpathlineto{\pgfqpoint{2.215469in}{1.363791in}}%
\pgfpathlineto{\pgfqpoint{2.237667in}{1.363791in}}%
\pgfpathlineto{\pgfqpoint{2.259865in}{1.363791in}}%
\pgfpathlineto{\pgfqpoint{2.282063in}{1.363791in}}%
\pgfpathlineto{\pgfqpoint{2.304261in}{1.363791in}}%
\pgfpathlineto{\pgfqpoint{2.326459in}{1.363791in}}%
\pgfpathlineto{\pgfqpoint{2.348657in}{1.363791in}}%
\pgfpathlineto{\pgfqpoint{2.370855in}{1.363791in}}%
\pgfpathlineto{\pgfqpoint{2.393053in}{1.363791in}}%
\pgfpathlineto{\pgfqpoint{2.415251in}{1.363791in}}%
\pgfpathlineto{\pgfqpoint{2.437449in}{1.363791in}}%
\pgfpathlineto{\pgfqpoint{2.459647in}{1.363791in}}%
\pgfpathlineto{\pgfqpoint{2.481845in}{1.363791in}}%
\pgfpathlineto{\pgfqpoint{2.504043in}{1.363791in}}%
\pgfpathlineto{\pgfqpoint{2.526241in}{1.363791in}}%
\pgfpathlineto{\pgfqpoint{2.548439in}{1.363791in}}%
\pgfpathlineto{\pgfqpoint{2.570637in}{1.363791in}}%
\pgfpathlineto{\pgfqpoint{2.592835in}{1.363791in}}%
\pgfpathlineto{\pgfqpoint{2.615033in}{1.363791in}}%
\pgfpathlineto{\pgfqpoint{2.637231in}{1.363791in}}%
\pgfpathlineto{\pgfqpoint{2.659429in}{1.363791in}}%
\pgfpathlineto{\pgfqpoint{2.681627in}{1.363791in}}%
\pgfpathlineto{\pgfqpoint{2.703825in}{1.363791in}}%
\pgfpathlineto{\pgfqpoint{2.726023in}{1.363791in}}%
\pgfpathlineto{\pgfqpoint{2.748221in}{1.363791in}}%
\pgfpathlineto{\pgfqpoint{2.770419in}{1.363791in}}%
\pgfpathlineto{\pgfqpoint{2.792617in}{1.363791in}}%
\pgfpathlineto{\pgfqpoint{2.814815in}{1.363791in}}%
\pgfpathlineto{\pgfqpoint{2.837013in}{1.363791in}}%
\pgfpathlineto{\pgfqpoint{2.859211in}{1.363791in}}%
\pgfpathlineto{\pgfqpoint{2.881409in}{1.363791in}}%
\pgfpathlineto{\pgfqpoint{2.903607in}{1.363791in}}%
\pgfpathlineto{\pgfqpoint{2.925805in}{1.363791in}}%
\pgfpathlineto{\pgfqpoint{2.948003in}{1.363791in}}%
\pgfpathlineto{\pgfqpoint{2.970201in}{1.363791in}}%
\pgfpathlineto{\pgfqpoint{2.992399in}{1.363791in}}%
\pgfpathlineto{\pgfqpoint{3.014597in}{1.363791in}}%
\pgfpathlineto{\pgfqpoint{3.036795in}{1.363464in}}%
\pgfpathlineto{\pgfqpoint{3.058993in}{1.363464in}}%
\pgfpathlineto{\pgfqpoint{3.081191in}{1.342195in}}%
\pgfpathlineto{\pgfqpoint{3.103389in}{1.342195in}}%
\pgfpathlineto{\pgfqpoint{3.125587in}{1.342195in}}%
\pgfpathlineto{\pgfqpoint{3.147785in}{1.342195in}}%
\pgfpathlineto{\pgfqpoint{3.169983in}{1.342195in}}%
\pgfpathlineto{\pgfqpoint{3.192181in}{1.342195in}}%
\pgfpathlineto{\pgfqpoint{3.214379in}{1.342195in}}%
\pgfpathlineto{\pgfqpoint{3.236577in}{1.342195in}}%
\pgfpathlineto{\pgfqpoint{3.258775in}{1.342195in}}%
\pgfpathlineto{\pgfqpoint{3.280973in}{1.342195in}}%
\pgfpathlineto{\pgfqpoint{3.303171in}{1.342195in}}%
\pgfpathlineto{\pgfqpoint{3.325369in}{1.342195in}}%
\pgfpathlineto{\pgfqpoint{3.347567in}{1.342195in}}%
\pgfpathlineto{\pgfqpoint{3.369765in}{1.342195in}}%
\pgfpathlineto{\pgfqpoint{3.391963in}{1.342195in}}%
\pgfpathlineto{\pgfqpoint{3.414161in}{1.342195in}}%
\pgfpathlineto{\pgfqpoint{3.436359in}{1.342195in}}%
\pgfpathlineto{\pgfqpoint{3.458557in}{1.342195in}}%
\pgfpathlineto{\pgfqpoint{3.480755in}{1.342195in}}%
\pgfpathlineto{\pgfqpoint{3.502953in}{1.342195in}}%
\pgfpathlineto{\pgfqpoint{3.525151in}{1.342195in}}%
\pgfpathlineto{\pgfqpoint{3.547349in}{1.342195in}}%
\pgfpathlineto{\pgfqpoint{3.569547in}{1.342195in}}%
\pgfpathlineto{\pgfqpoint{3.591745in}{1.342195in}}%
\pgfpathlineto{\pgfqpoint{3.613943in}{1.342195in}}%
\pgfpathlineto{\pgfqpoint{3.636141in}{1.342195in}}%
\pgfpathlineto{\pgfqpoint{3.658339in}{1.342195in}}%
\pgfpathlineto{\pgfqpoint{3.680537in}{1.342195in}}%
\pgfpathlineto{\pgfqpoint{3.702735in}{1.342195in}}%
\pgfpathlineto{\pgfqpoint{3.724933in}{1.280545in}}%
\pgfpathlineto{\pgfqpoint{3.747131in}{1.280545in}}%
\pgfpathlineto{\pgfqpoint{3.769329in}{1.280545in}}%
\pgfpathlineto{\pgfqpoint{3.791527in}{1.280545in}}%
\pgfpathlineto{\pgfqpoint{3.813725in}{1.280545in}}%
\pgfpathlineto{\pgfqpoint{3.835923in}{1.280545in}}%
\pgfpathlineto{\pgfqpoint{3.858121in}{1.280545in}}%
\pgfpathlineto{\pgfqpoint{3.880319in}{1.280545in}}%
\pgfpathlineto{\pgfqpoint{3.902517in}{1.280545in}}%
\pgfpathlineto{\pgfqpoint{3.924715in}{1.280545in}}%
\pgfpathlineto{\pgfqpoint{3.946913in}{1.280545in}}%
\pgfpathlineto{\pgfqpoint{3.969111in}{1.280545in}}%
\pgfpathlineto{\pgfqpoint{3.991309in}{1.280545in}}%
\pgfpathlineto{\pgfqpoint{4.013507in}{1.280545in}}%
\pgfpathlineto{\pgfqpoint{4.035705in}{1.280545in}}%
\pgfpathlineto{\pgfqpoint{4.057903in}{1.280545in}}%
\pgfpathlineto{\pgfqpoint{4.080101in}{1.280545in}}%
\pgfpathlineto{\pgfqpoint{4.102299in}{1.280545in}}%
\pgfpathlineto{\pgfqpoint{4.124497in}{1.280545in}}%
\pgfpathlineto{\pgfqpoint{4.146695in}{1.280545in}}%
\pgfpathlineto{\pgfqpoint{4.168893in}{1.280545in}}%
\pgfpathlineto{\pgfqpoint{4.191091in}{1.280545in}}%
\pgfpathlineto{\pgfqpoint{4.213289in}{1.280545in}}%
\pgfpathlineto{\pgfqpoint{4.235487in}{1.280545in}}%
\pgfpathlineto{\pgfqpoint{4.257685in}{1.280545in}}%
\pgfpathlineto{\pgfqpoint{4.279883in}{1.280545in}}%
\pgfpathlineto{\pgfqpoint{4.302081in}{1.280545in}}%
\pgfpathlineto{\pgfqpoint{4.324279in}{1.280545in}}%
\pgfpathlineto{\pgfqpoint{4.346477in}{1.280545in}}%
\pgfpathlineto{\pgfqpoint{4.368675in}{1.280545in}}%
\pgfpathlineto{\pgfqpoint{4.390873in}{1.280545in}}%
\pgfpathlineto{\pgfqpoint{4.413071in}{1.280545in}}%
\pgfpathlineto{\pgfqpoint{4.435269in}{1.280545in}}%
\pgfpathlineto{\pgfqpoint{4.457467in}{1.280545in}}%
\pgfpathlineto{\pgfqpoint{4.479665in}{1.280545in}}%
\pgfpathlineto{\pgfqpoint{4.501863in}{1.280545in}}%
\pgfpathlineto{\pgfqpoint{4.524061in}{1.280545in}}%
\pgfpathlineto{\pgfqpoint{4.546259in}{1.280545in}}%
\pgfpathlineto{\pgfqpoint{4.568457in}{1.280545in}}%
\pgfpathlineto{\pgfqpoint{4.590655in}{1.280545in}}%
\pgfpathlineto{\pgfqpoint{4.612853in}{1.280545in}}%
\pgfpathlineto{\pgfqpoint{4.635051in}{1.280545in}}%
\pgfpathlineto{\pgfqpoint{4.657249in}{1.280545in}}%
\pgfpathlineto{\pgfqpoint{4.679447in}{1.280545in}}%
\pgfpathlineto{\pgfqpoint{4.701645in}{1.280545in}}%
\pgfpathlineto{\pgfqpoint{4.723843in}{1.280545in}}%
\pgfpathlineto{\pgfqpoint{4.746041in}{1.280545in}}%
\pgfpathlineto{\pgfqpoint{4.768239in}{1.280545in}}%
\pgfpathlineto{\pgfqpoint{4.790437in}{1.280545in}}%
\pgfpathlineto{\pgfqpoint{4.812635in}{1.280545in}}%
\pgfpathlineto{\pgfqpoint{4.834833in}{1.280545in}}%
\pgfpathlineto{\pgfqpoint{4.857031in}{1.280545in}}%
\pgfpathlineto{\pgfqpoint{4.879229in}{1.280545in}}%
\pgfpathlineto{\pgfqpoint{4.901427in}{1.280545in}}%
\pgfpathlineto{\pgfqpoint{4.923625in}{1.280545in}}%
\pgfpathlineto{\pgfqpoint{4.945823in}{1.280545in}}%
\pgfpathlineto{\pgfqpoint{4.968021in}{1.280545in}}%
\pgfpathlineto{\pgfqpoint{4.990219in}{1.280545in}}%
\pgfpathlineto{\pgfqpoint{5.012417in}{1.280545in}}%
\pgfpathlineto{\pgfqpoint{5.034616in}{1.280545in}}%
\pgfpathlineto{\pgfqpoint{5.056814in}{1.280545in}}%
\pgfpathlineto{\pgfqpoint{5.079012in}{1.280545in}}%
\pgfpathlineto{\pgfqpoint{5.101210in}{1.280545in}}%
\pgfpathlineto{\pgfqpoint{5.123408in}{1.048592in}}%
\pgfpathlineto{\pgfqpoint{5.145606in}{1.048592in}}%
\pgfpathlineto{\pgfqpoint{5.167804in}{1.048592in}}%
\pgfpathlineto{\pgfqpoint{5.190002in}{1.048592in}}%
\pgfpathlineto{\pgfqpoint{5.212200in}{1.048592in}}%
\pgfpathlineto{\pgfqpoint{5.234398in}{1.048592in}}%
\pgfpathlineto{\pgfqpoint{5.256596in}{1.048592in}}%
\pgfpathlineto{\pgfqpoint{5.278794in}{1.048592in}}%
\pgfpathlineto{\pgfqpoint{5.300992in}{1.048592in}}%
\pgfpathlineto{\pgfqpoint{5.323190in}{1.048592in}}%
\pgfpathlineto{\pgfqpoint{5.345388in}{1.048592in}}%
\pgfpathlineto{\pgfqpoint{5.367586in}{1.048592in}}%
\pgfpathlineto{\pgfqpoint{5.389784in}{1.048592in}}%
\pgfpathlineto{\pgfqpoint{5.411982in}{1.048592in}}%
\pgfpathlineto{\pgfqpoint{5.434180in}{1.048592in}}%
\pgfpathlineto{\pgfqpoint{5.456378in}{1.048592in}}%
\pgfpathlineto{\pgfqpoint{5.478576in}{1.048592in}}%
\pgfpathlineto{\pgfqpoint{5.500774in}{1.048592in}}%
\pgfpathlineto{\pgfqpoint{5.522972in}{1.048592in}}%
\pgfpathlineto{\pgfqpoint{5.545170in}{1.048592in}}%
\pgfpathlineto{\pgfqpoint{5.567368in}{1.048592in}}%
\pgfpathlineto{\pgfqpoint{5.589566in}{1.048592in}}%
\pgfpathlineto{\pgfqpoint{5.611764in}{1.048592in}}%
\pgfpathlineto{\pgfqpoint{5.633962in}{1.048592in}}%
\pgfpathlineto{\pgfqpoint{5.656160in}{1.048592in}}%
\pgfpathlineto{\pgfqpoint{5.678358in}{1.048592in}}%
\pgfpathlineto{\pgfqpoint{5.700556in}{1.048592in}}%
\pgfpathlineto{\pgfqpoint{5.722754in}{1.048592in}}%
\pgfpathlineto{\pgfqpoint{5.744952in}{1.048592in}}%
\pgfpathlineto{\pgfqpoint{5.767150in}{1.048592in}}%
\pgfpathlineto{\pgfqpoint{5.789348in}{1.048592in}}%
\pgfpathlineto{\pgfqpoint{5.811546in}{1.048592in}}%
\pgfpathlineto{\pgfqpoint{5.833744in}{1.048592in}}%
\pgfpathlineto{\pgfqpoint{5.855942in}{0.900676in}}%
\pgfpathlineto{\pgfqpoint{5.878140in}{0.900676in}}%
\pgfpathlineto{\pgfqpoint{5.900338in}{0.900676in}}%
\pgfpathlineto{\pgfqpoint{5.922536in}{0.900676in}}%
\pgfpathlineto{\pgfqpoint{5.944734in}{0.900676in}}%
\pgfpathlineto{\pgfqpoint{5.966932in}{0.900676in}}%
\pgfpathlineto{\pgfqpoint{5.966932in}{0.778669in}}%
\pgfpathlineto{\pgfqpoint{5.966932in}{0.778669in}}%
\pgfpathlineto{\pgfqpoint{5.944734in}{0.778669in}}%
\pgfpathlineto{\pgfqpoint{5.922536in}{0.778669in}}%
\pgfpathlineto{\pgfqpoint{5.900338in}{0.778669in}}%
\pgfpathlineto{\pgfqpoint{5.878140in}{0.778669in}}%
\pgfpathlineto{\pgfqpoint{5.855942in}{0.778669in}}%
\pgfpathlineto{\pgfqpoint{5.833744in}{0.854280in}}%
\pgfpathlineto{\pgfqpoint{5.811546in}{0.854280in}}%
\pgfpathlineto{\pgfqpoint{5.789348in}{0.854280in}}%
\pgfpathlineto{\pgfqpoint{5.767150in}{0.854280in}}%
\pgfpathlineto{\pgfqpoint{5.744952in}{0.854280in}}%
\pgfpathlineto{\pgfqpoint{5.722754in}{0.854280in}}%
\pgfpathlineto{\pgfqpoint{5.700556in}{0.854280in}}%
\pgfpathlineto{\pgfqpoint{5.678358in}{0.854280in}}%
\pgfpathlineto{\pgfqpoint{5.656160in}{0.854280in}}%
\pgfpathlineto{\pgfqpoint{5.633962in}{0.854280in}}%
\pgfpathlineto{\pgfqpoint{5.611764in}{0.854280in}}%
\pgfpathlineto{\pgfqpoint{5.589566in}{0.854280in}}%
\pgfpathlineto{\pgfqpoint{5.567368in}{0.854280in}}%
\pgfpathlineto{\pgfqpoint{5.545170in}{0.854280in}}%
\pgfpathlineto{\pgfqpoint{5.522972in}{0.854280in}}%
\pgfpathlineto{\pgfqpoint{5.500774in}{0.854280in}}%
\pgfpathlineto{\pgfqpoint{5.478576in}{0.854280in}}%
\pgfpathlineto{\pgfqpoint{5.456378in}{0.854280in}}%
\pgfpathlineto{\pgfqpoint{5.434180in}{0.854280in}}%
\pgfpathlineto{\pgfqpoint{5.411982in}{0.854280in}}%
\pgfpathlineto{\pgfqpoint{5.389784in}{0.854280in}}%
\pgfpathlineto{\pgfqpoint{5.367586in}{0.854280in}}%
\pgfpathlineto{\pgfqpoint{5.345388in}{0.854280in}}%
\pgfpathlineto{\pgfqpoint{5.323190in}{0.854280in}}%
\pgfpathlineto{\pgfqpoint{5.300992in}{0.854280in}}%
\pgfpathlineto{\pgfqpoint{5.278794in}{0.854280in}}%
\pgfpathlineto{\pgfqpoint{5.256596in}{0.854280in}}%
\pgfpathlineto{\pgfqpoint{5.234398in}{0.854280in}}%
\pgfpathlineto{\pgfqpoint{5.212200in}{0.854280in}}%
\pgfpathlineto{\pgfqpoint{5.190002in}{0.854280in}}%
\pgfpathlineto{\pgfqpoint{5.167804in}{0.854280in}}%
\pgfpathlineto{\pgfqpoint{5.145606in}{0.854280in}}%
\pgfpathlineto{\pgfqpoint{5.123408in}{0.854280in}}%
\pgfpathlineto{\pgfqpoint{5.101210in}{1.009734in}}%
\pgfpathlineto{\pgfqpoint{5.079012in}{1.009734in}}%
\pgfpathlineto{\pgfqpoint{5.056814in}{1.009734in}}%
\pgfpathlineto{\pgfqpoint{5.034616in}{1.009734in}}%
\pgfpathlineto{\pgfqpoint{5.012417in}{1.009734in}}%
\pgfpathlineto{\pgfqpoint{4.990219in}{1.009734in}}%
\pgfpathlineto{\pgfqpoint{4.968021in}{1.009734in}}%
\pgfpathlineto{\pgfqpoint{4.945823in}{1.009734in}}%
\pgfpathlineto{\pgfqpoint{4.923625in}{1.009734in}}%
\pgfpathlineto{\pgfqpoint{4.901427in}{1.009734in}}%
\pgfpathlineto{\pgfqpoint{4.879229in}{1.009734in}}%
\pgfpathlineto{\pgfqpoint{4.857031in}{1.009734in}}%
\pgfpathlineto{\pgfqpoint{4.834833in}{1.009734in}}%
\pgfpathlineto{\pgfqpoint{4.812635in}{1.009734in}}%
\pgfpathlineto{\pgfqpoint{4.790437in}{1.009734in}}%
\pgfpathlineto{\pgfqpoint{4.768239in}{1.009734in}}%
\pgfpathlineto{\pgfqpoint{4.746041in}{1.009734in}}%
\pgfpathlineto{\pgfqpoint{4.723843in}{1.009734in}}%
\pgfpathlineto{\pgfqpoint{4.701645in}{1.009734in}}%
\pgfpathlineto{\pgfqpoint{4.679447in}{1.009734in}}%
\pgfpathlineto{\pgfqpoint{4.657249in}{1.009734in}}%
\pgfpathlineto{\pgfqpoint{4.635051in}{1.009734in}}%
\pgfpathlineto{\pgfqpoint{4.612853in}{1.009734in}}%
\pgfpathlineto{\pgfqpoint{4.590655in}{1.009734in}}%
\pgfpathlineto{\pgfqpoint{4.568457in}{1.009734in}}%
\pgfpathlineto{\pgfqpoint{4.546259in}{1.009734in}}%
\pgfpathlineto{\pgfqpoint{4.524061in}{1.009734in}}%
\pgfpathlineto{\pgfqpoint{4.501863in}{1.009734in}}%
\pgfpathlineto{\pgfqpoint{4.479665in}{1.009734in}}%
\pgfpathlineto{\pgfqpoint{4.457467in}{1.009734in}}%
\pgfpathlineto{\pgfqpoint{4.435269in}{1.009734in}}%
\pgfpathlineto{\pgfqpoint{4.413071in}{1.009734in}}%
\pgfpathlineto{\pgfqpoint{4.390873in}{1.009734in}}%
\pgfpathlineto{\pgfqpoint{4.368675in}{1.009734in}}%
\pgfpathlineto{\pgfqpoint{4.346477in}{1.009734in}}%
\pgfpathlineto{\pgfqpoint{4.324279in}{1.009734in}}%
\pgfpathlineto{\pgfqpoint{4.302081in}{1.009734in}}%
\pgfpathlineto{\pgfqpoint{4.279883in}{1.009734in}}%
\pgfpathlineto{\pgfqpoint{4.257685in}{1.009734in}}%
\pgfpathlineto{\pgfqpoint{4.235487in}{1.009734in}}%
\pgfpathlineto{\pgfqpoint{4.213289in}{1.009734in}}%
\pgfpathlineto{\pgfqpoint{4.191091in}{1.009734in}}%
\pgfpathlineto{\pgfqpoint{4.168893in}{1.009734in}}%
\pgfpathlineto{\pgfqpoint{4.146695in}{1.009734in}}%
\pgfpathlineto{\pgfqpoint{4.124497in}{1.009734in}}%
\pgfpathlineto{\pgfqpoint{4.102299in}{1.009734in}}%
\pgfpathlineto{\pgfqpoint{4.080101in}{1.009734in}}%
\pgfpathlineto{\pgfqpoint{4.057903in}{1.009734in}}%
\pgfpathlineto{\pgfqpoint{4.035705in}{1.009734in}}%
\pgfpathlineto{\pgfqpoint{4.013507in}{1.009734in}}%
\pgfpathlineto{\pgfqpoint{3.991309in}{1.009734in}}%
\pgfpathlineto{\pgfqpoint{3.969111in}{1.009734in}}%
\pgfpathlineto{\pgfqpoint{3.946913in}{1.009734in}}%
\pgfpathlineto{\pgfqpoint{3.924715in}{1.009734in}}%
\pgfpathlineto{\pgfqpoint{3.902517in}{1.009734in}}%
\pgfpathlineto{\pgfqpoint{3.880319in}{1.009734in}}%
\pgfpathlineto{\pgfqpoint{3.858121in}{1.009734in}}%
\pgfpathlineto{\pgfqpoint{3.835923in}{1.009734in}}%
\pgfpathlineto{\pgfqpoint{3.813725in}{1.009734in}}%
\pgfpathlineto{\pgfqpoint{3.791527in}{1.009734in}}%
\pgfpathlineto{\pgfqpoint{3.769329in}{1.009734in}}%
\pgfpathlineto{\pgfqpoint{3.747131in}{1.009734in}}%
\pgfpathlineto{\pgfqpoint{3.724933in}{1.009734in}}%
\pgfpathlineto{\pgfqpoint{3.702735in}{1.024376in}}%
\pgfpathlineto{\pgfqpoint{3.680537in}{1.024376in}}%
\pgfpathlineto{\pgfqpoint{3.658339in}{1.024376in}}%
\pgfpathlineto{\pgfqpoint{3.636141in}{1.024376in}}%
\pgfpathlineto{\pgfqpoint{3.613943in}{1.024376in}}%
\pgfpathlineto{\pgfqpoint{3.591745in}{1.024376in}}%
\pgfpathlineto{\pgfqpoint{3.569547in}{1.024376in}}%
\pgfpathlineto{\pgfqpoint{3.547349in}{1.024376in}}%
\pgfpathlineto{\pgfqpoint{3.525151in}{1.024376in}}%
\pgfpathlineto{\pgfqpoint{3.502953in}{1.024376in}}%
\pgfpathlineto{\pgfqpoint{3.480755in}{1.024376in}}%
\pgfpathlineto{\pgfqpoint{3.458557in}{1.024376in}}%
\pgfpathlineto{\pgfqpoint{3.436359in}{1.024376in}}%
\pgfpathlineto{\pgfqpoint{3.414161in}{1.024376in}}%
\pgfpathlineto{\pgfqpoint{3.391963in}{1.024376in}}%
\pgfpathlineto{\pgfqpoint{3.369765in}{1.024376in}}%
\pgfpathlineto{\pgfqpoint{3.347567in}{1.024376in}}%
\pgfpathlineto{\pgfqpoint{3.325369in}{1.024376in}}%
\pgfpathlineto{\pgfqpoint{3.303171in}{1.024376in}}%
\pgfpathlineto{\pgfqpoint{3.280973in}{1.024376in}}%
\pgfpathlineto{\pgfqpoint{3.258775in}{1.024376in}}%
\pgfpathlineto{\pgfqpoint{3.236577in}{1.024376in}}%
\pgfpathlineto{\pgfqpoint{3.214379in}{1.024376in}}%
\pgfpathlineto{\pgfqpoint{3.192181in}{1.024376in}}%
\pgfpathlineto{\pgfqpoint{3.169983in}{1.024376in}}%
\pgfpathlineto{\pgfqpoint{3.147785in}{1.024376in}}%
\pgfpathlineto{\pgfqpoint{3.125587in}{1.024376in}}%
\pgfpathlineto{\pgfqpoint{3.103389in}{1.024376in}}%
\pgfpathlineto{\pgfqpoint{3.081191in}{1.024376in}}%
\pgfpathlineto{\pgfqpoint{3.058993in}{1.095448in}}%
\pgfpathlineto{\pgfqpoint{3.036795in}{1.095448in}}%
\pgfpathlineto{\pgfqpoint{3.014597in}{1.097932in}}%
\pgfpathlineto{\pgfqpoint{2.992399in}{1.097932in}}%
\pgfpathlineto{\pgfqpoint{2.970201in}{1.097932in}}%
\pgfpathlineto{\pgfqpoint{2.948003in}{1.097932in}}%
\pgfpathlineto{\pgfqpoint{2.925805in}{1.097932in}}%
\pgfpathlineto{\pgfqpoint{2.903607in}{1.097932in}}%
\pgfpathlineto{\pgfqpoint{2.881409in}{1.097932in}}%
\pgfpathlineto{\pgfqpoint{2.859211in}{1.097932in}}%
\pgfpathlineto{\pgfqpoint{2.837013in}{1.097932in}}%
\pgfpathlineto{\pgfqpoint{2.814815in}{1.097932in}}%
\pgfpathlineto{\pgfqpoint{2.792617in}{1.097932in}}%
\pgfpathlineto{\pgfqpoint{2.770419in}{1.097932in}}%
\pgfpathlineto{\pgfqpoint{2.748221in}{1.097932in}}%
\pgfpathlineto{\pgfqpoint{2.726023in}{1.097932in}}%
\pgfpathlineto{\pgfqpoint{2.703825in}{1.097932in}}%
\pgfpathlineto{\pgfqpoint{2.681627in}{1.097932in}}%
\pgfpathlineto{\pgfqpoint{2.659429in}{1.097932in}}%
\pgfpathlineto{\pgfqpoint{2.637231in}{1.097932in}}%
\pgfpathlineto{\pgfqpoint{2.615033in}{1.097932in}}%
\pgfpathlineto{\pgfqpoint{2.592835in}{1.097932in}}%
\pgfpathlineto{\pgfqpoint{2.570637in}{1.097932in}}%
\pgfpathlineto{\pgfqpoint{2.548439in}{1.097932in}}%
\pgfpathlineto{\pgfqpoint{2.526241in}{1.097932in}}%
\pgfpathlineto{\pgfqpoint{2.504043in}{1.097932in}}%
\pgfpathlineto{\pgfqpoint{2.481845in}{1.097932in}}%
\pgfpathlineto{\pgfqpoint{2.459647in}{1.097932in}}%
\pgfpathlineto{\pgfqpoint{2.437449in}{1.097932in}}%
\pgfpathlineto{\pgfqpoint{2.415251in}{1.097932in}}%
\pgfpathlineto{\pgfqpoint{2.393053in}{1.097932in}}%
\pgfpathlineto{\pgfqpoint{2.370855in}{1.097932in}}%
\pgfpathlineto{\pgfqpoint{2.348657in}{1.097932in}}%
\pgfpathlineto{\pgfqpoint{2.326459in}{1.097932in}}%
\pgfpathlineto{\pgfqpoint{2.304261in}{1.097932in}}%
\pgfpathlineto{\pgfqpoint{2.282063in}{1.097932in}}%
\pgfpathlineto{\pgfqpoint{2.259865in}{1.097932in}}%
\pgfpathlineto{\pgfqpoint{2.237667in}{1.097932in}}%
\pgfpathlineto{\pgfqpoint{2.215469in}{1.097932in}}%
\pgfpathlineto{\pgfqpoint{2.193271in}{1.097932in}}%
\pgfpathlineto{\pgfqpoint{2.171073in}{1.097932in}}%
\pgfpathlineto{\pgfqpoint{2.148875in}{1.199358in}}%
\pgfpathlineto{\pgfqpoint{2.126677in}{1.199358in}}%
\pgfpathlineto{\pgfqpoint{2.104479in}{1.199358in}}%
\pgfpathlineto{\pgfqpoint{2.082281in}{1.202014in}}%
\pgfpathlineto{\pgfqpoint{2.060083in}{1.330095in}}%
\pgfpathlineto{\pgfqpoint{2.037884in}{1.330095in}}%
\pgfpathlineto{\pgfqpoint{2.015686in}{1.330095in}}%
\pgfpathlineto{\pgfqpoint{1.993488in}{1.330095in}}%
\pgfpathlineto{\pgfqpoint{1.971290in}{1.330095in}}%
\pgfpathlineto{\pgfqpoint{1.949092in}{1.374710in}}%
\pgfpathlineto{\pgfqpoint{1.926894in}{1.374710in}}%
\pgfpathlineto{\pgfqpoint{1.904696in}{1.404232in}}%
\pgfpathlineto{\pgfqpoint{1.882498in}{1.411597in}}%
\pgfpathlineto{\pgfqpoint{1.860300in}{1.439659in}}%
\pgfpathlineto{\pgfqpoint{1.838102in}{1.439659in}}%
\pgfpathlineto{\pgfqpoint{1.815904in}{1.439659in}}%
\pgfpathlineto{\pgfqpoint{1.793706in}{1.441080in}}%
\pgfpathlineto{\pgfqpoint{1.771508in}{1.545346in}}%
\pgfpathlineto{\pgfqpoint{1.749310in}{1.551698in}}%
\pgfpathlineto{\pgfqpoint{1.727112in}{1.551698in}}%
\pgfpathlineto{\pgfqpoint{1.704914in}{1.551698in}}%
\pgfpathlineto{\pgfqpoint{1.682716in}{1.563146in}}%
\pgfpathlineto{\pgfqpoint{1.660518in}{1.579108in}}%
\pgfpathlineto{\pgfqpoint{1.638320in}{1.592161in}}%
\pgfpathlineto{\pgfqpoint{1.616122in}{1.592558in}}%
\pgfpathlineto{\pgfqpoint{1.593924in}{1.740812in}}%
\pgfpathlineto{\pgfqpoint{1.571726in}{1.743039in}}%
\pgfpathlineto{\pgfqpoint{1.549528in}{1.798212in}}%
\pgfpathlineto{\pgfqpoint{1.527330in}{1.966425in}}%
\pgfpathlineto{\pgfqpoint{1.505132in}{1.966425in}}%
\pgfpathlineto{\pgfqpoint{1.482934in}{1.966425in}}%
\pgfpathlineto{\pgfqpoint{1.460736in}{2.005676in}}%
\pgfpathlineto{\pgfqpoint{1.438538in}{2.014551in}}%
\pgfpathlineto{\pgfqpoint{1.416340in}{2.014551in}}%
\pgfpathlineto{\pgfqpoint{1.394142in}{2.014551in}}%
\pgfpathlineto{\pgfqpoint{1.371944in}{2.014551in}}%
\pgfpathlineto{\pgfqpoint{1.349746in}{2.014551in}}%
\pgfpathlineto{\pgfqpoint{1.327548in}{2.014551in}}%
\pgfpathlineto{\pgfqpoint{1.305350in}{2.014551in}}%
\pgfpathlineto{\pgfqpoint{1.283152in}{2.014551in}}%
\pgfpathlineto{\pgfqpoint{1.260954in}{2.014551in}}%
\pgfpathlineto{\pgfqpoint{1.238756in}{2.014551in}}%
\pgfpathlineto{\pgfqpoint{1.216558in}{2.014551in}}%
\pgfpathlineto{\pgfqpoint{1.194360in}{2.069356in}}%
\pgfpathlineto{\pgfqpoint{1.172162in}{2.069356in}}%
\pgfpathlineto{\pgfqpoint{1.149964in}{2.088150in}}%
\pgfpathlineto{\pgfqpoint{1.127766in}{2.142169in}}%
\pgfpathlineto{\pgfqpoint{1.105568in}{2.321169in}}%
\pgfpathclose%
\pgfusepath{fill}%
\end{pgfscope}%
\begin{pgfscope}%
\pgfpathrectangle{\pgfqpoint{0.862500in}{0.375000in}}{\pgfqpoint{5.347500in}{2.265000in}}%
\pgfusepath{clip}%
\pgfsetroundcap%
\pgfsetroundjoin%
\pgfsetlinewidth{1.505625pt}%
\definecolor{currentstroke}{rgb}{0.121569,0.466667,0.705882}%
\pgfsetstrokecolor{currentstroke}%
\pgfsetdash{}{0pt}%
\pgfpathmoveto{\pgfqpoint{1.105568in}{2.484221in}}%
\pgfpathlineto{\pgfqpoint{1.127766in}{2.297214in}}%
\pgfpathlineto{\pgfqpoint{1.149964in}{2.213776in}}%
\pgfpathlineto{\pgfqpoint{1.172162in}{2.180634in}}%
\pgfpathlineto{\pgfqpoint{1.194360in}{2.153675in}}%
\pgfpathlineto{\pgfqpoint{1.216558in}{2.153675in}}%
\pgfpathlineto{\pgfqpoint{1.238756in}{2.132624in}}%
\pgfpathlineto{\pgfqpoint{1.260954in}{2.128139in}}%
\pgfpathlineto{\pgfqpoint{1.283152in}{2.100218in}}%
\pgfpathlineto{\pgfqpoint{1.305350in}{2.092109in}}%
\pgfpathlineto{\pgfqpoint{1.327548in}{2.073920in}}%
\pgfpathlineto{\pgfqpoint{1.349746in}{2.073920in}}%
\pgfpathlineto{\pgfqpoint{1.371944in}{2.062937in}}%
\pgfpathlineto{\pgfqpoint{1.394142in}{2.061307in}}%
\pgfpathlineto{\pgfqpoint{1.416340in}{2.003280in}}%
\pgfpathlineto{\pgfqpoint{1.438538in}{1.989451in}}%
\pgfpathlineto{\pgfqpoint{1.460736in}{1.989451in}}%
\pgfpathlineto{\pgfqpoint{1.482934in}{1.979147in}}%
\pgfpathlineto{\pgfqpoint{1.527330in}{1.979147in}}%
\pgfpathlineto{\pgfqpoint{1.549528in}{1.975047in}}%
\pgfpathlineto{\pgfqpoint{1.749310in}{1.975047in}}%
\pgfpathlineto{\pgfqpoint{1.771508in}{1.941447in}}%
\pgfpathlineto{\pgfqpoint{1.793706in}{1.922776in}}%
\pgfpathlineto{\pgfqpoint{1.815904in}{1.922776in}}%
\pgfpathlineto{\pgfqpoint{1.838102in}{1.898556in}}%
\pgfpathlineto{\pgfqpoint{1.860300in}{1.894448in}}%
\pgfpathlineto{\pgfqpoint{2.015686in}{1.894448in}}%
\pgfpathlineto{\pgfqpoint{2.037884in}{1.883250in}}%
\pgfpathlineto{\pgfqpoint{2.193271in}{1.883250in}}%
\pgfpathlineto{\pgfqpoint{2.215469in}{1.872835in}}%
\pgfpathlineto{\pgfqpoint{2.481845in}{1.872835in}}%
\pgfpathlineto{\pgfqpoint{2.504043in}{1.819883in}}%
\pgfpathlineto{\pgfqpoint{2.526241in}{1.819883in}}%
\pgfpathlineto{\pgfqpoint{2.548439in}{1.799259in}}%
\pgfpathlineto{\pgfqpoint{2.748221in}{1.798661in}}%
\pgfpathlineto{\pgfqpoint{2.770419in}{1.761887in}}%
\pgfpathlineto{\pgfqpoint{2.948003in}{1.761887in}}%
\pgfpathlineto{\pgfqpoint{2.970201in}{1.749663in}}%
\pgfpathlineto{\pgfqpoint{3.169983in}{1.749663in}}%
\pgfpathlineto{\pgfqpoint{3.192181in}{1.745575in}}%
\pgfpathlineto{\pgfqpoint{3.236577in}{1.745575in}}%
\pgfpathlineto{\pgfqpoint{3.258775in}{1.741007in}}%
\pgfpathlineto{\pgfqpoint{3.391963in}{1.741007in}}%
\pgfpathlineto{\pgfqpoint{3.414161in}{1.738968in}}%
\pgfpathlineto{\pgfqpoint{3.791527in}{1.738968in}}%
\pgfpathlineto{\pgfqpoint{3.813725in}{1.666966in}}%
\pgfpathlineto{\pgfqpoint{4.213289in}{1.666966in}}%
\pgfpathlineto{\pgfqpoint{4.235487in}{1.632311in}}%
\pgfpathlineto{\pgfqpoint{5.079012in}{1.632311in}}%
\pgfpathlineto{\pgfqpoint{5.101210in}{1.616856in}}%
\pgfpathlineto{\pgfqpoint{5.345388in}{1.616856in}}%
\pgfpathlineto{\pgfqpoint{5.367586in}{1.612219in}}%
\pgfpathlineto{\pgfqpoint{5.389784in}{1.593761in}}%
\pgfpathlineto{\pgfqpoint{5.966932in}{1.593761in}}%
\pgfpathlineto{\pgfqpoint{5.966932in}{1.593761in}}%
\pgfusepath{stroke}%
\end{pgfscope}%
\begin{pgfscope}%
\pgfpathrectangle{\pgfqpoint{0.862500in}{0.375000in}}{\pgfqpoint{5.347500in}{2.265000in}}%
\pgfusepath{clip}%
\pgfsetroundcap%
\pgfsetroundjoin%
\pgfsetlinewidth{1.505625pt}%
\definecolor{currentstroke}{rgb}{1.000000,0.498039,0.054902}%
\pgfsetstrokecolor{currentstroke}%
\pgfsetdash{}{0pt}%
\pgfpathmoveto{\pgfqpoint{1.105568in}{2.456518in}}%
\pgfpathlineto{\pgfqpoint{1.127766in}{2.437048in}}%
\pgfpathlineto{\pgfqpoint{1.149964in}{2.364963in}}%
\pgfpathlineto{\pgfqpoint{1.172162in}{2.271342in}}%
\pgfpathlineto{\pgfqpoint{1.194360in}{2.217096in}}%
\pgfpathlineto{\pgfqpoint{1.216558in}{2.182593in}}%
\pgfpathlineto{\pgfqpoint{1.238756in}{2.172371in}}%
\pgfpathlineto{\pgfqpoint{1.260954in}{2.119669in}}%
\pgfpathlineto{\pgfqpoint{1.327548in}{2.119669in}}%
\pgfpathlineto{\pgfqpoint{1.349746in}{2.100321in}}%
\pgfpathlineto{\pgfqpoint{1.438538in}{2.100321in}}%
\pgfpathlineto{\pgfqpoint{1.460736in}{1.981275in}}%
\pgfpathlineto{\pgfqpoint{1.638320in}{1.981275in}}%
\pgfpathlineto{\pgfqpoint{1.660518in}{1.930952in}}%
\pgfpathlineto{\pgfqpoint{1.682716in}{1.930952in}}%
\pgfpathlineto{\pgfqpoint{1.704914in}{1.924488in}}%
\pgfpathlineto{\pgfqpoint{1.727112in}{1.924488in}}%
\pgfpathlineto{\pgfqpoint{1.749310in}{1.847606in}}%
\pgfpathlineto{\pgfqpoint{1.771508in}{1.825144in}}%
\pgfpathlineto{\pgfqpoint{1.793706in}{1.825144in}}%
\pgfpathlineto{\pgfqpoint{1.815904in}{1.759168in}}%
\pgfpathlineto{\pgfqpoint{1.860300in}{1.759168in}}%
\pgfpathlineto{\pgfqpoint{1.882498in}{1.721399in}}%
\pgfpathlineto{\pgfqpoint{1.904696in}{1.686677in}}%
\pgfpathlineto{\pgfqpoint{1.926894in}{1.402746in}}%
\pgfpathlineto{\pgfqpoint{1.971290in}{1.402746in}}%
\pgfpathlineto{\pgfqpoint{1.993488in}{1.371689in}}%
\pgfpathlineto{\pgfqpoint{2.015686in}{1.359899in}}%
\pgfpathlineto{\pgfqpoint{2.082281in}{1.359899in}}%
\pgfpathlineto{\pgfqpoint{2.104479in}{1.335700in}}%
\pgfpathlineto{\pgfqpoint{2.215469in}{1.335700in}}%
\pgfpathlineto{\pgfqpoint{2.237667in}{1.318923in}}%
\pgfpathlineto{\pgfqpoint{2.326459in}{1.318923in}}%
\pgfpathlineto{\pgfqpoint{2.348657in}{1.303525in}}%
\pgfpathlineto{\pgfqpoint{2.370855in}{1.303525in}}%
\pgfpathlineto{\pgfqpoint{2.393053in}{1.217934in}}%
\pgfpathlineto{\pgfqpoint{2.437449in}{1.217442in}}%
\pgfpathlineto{\pgfqpoint{2.459647in}{1.208100in}}%
\pgfpathlineto{\pgfqpoint{2.481845in}{1.120793in}}%
\pgfpathlineto{\pgfqpoint{2.548439in}{1.120793in}}%
\pgfpathlineto{\pgfqpoint{2.570637in}{1.114090in}}%
\pgfpathlineto{\pgfqpoint{2.615033in}{1.114047in}}%
\pgfpathlineto{\pgfqpoint{2.637231in}{1.084102in}}%
\pgfpathlineto{\pgfqpoint{2.748221in}{1.084102in}}%
\pgfpathlineto{\pgfqpoint{2.770419in}{0.996934in}}%
\pgfpathlineto{\pgfqpoint{2.792617in}{0.981492in}}%
\pgfpathlineto{\pgfqpoint{2.837013in}{0.981427in}}%
\pgfpathlineto{\pgfqpoint{2.859211in}{0.909819in}}%
\pgfpathlineto{\pgfqpoint{2.925805in}{0.909819in}}%
\pgfpathlineto{\pgfqpoint{2.948003in}{0.810531in}}%
\pgfpathlineto{\pgfqpoint{3.280973in}{0.810531in}}%
\pgfpathlineto{\pgfqpoint{3.303171in}{0.749646in}}%
\pgfpathlineto{\pgfqpoint{3.502953in}{0.749646in}}%
\pgfpathlineto{\pgfqpoint{3.525151in}{0.736529in}}%
\pgfpathlineto{\pgfqpoint{3.547349in}{0.736529in}}%
\pgfpathlineto{\pgfqpoint{3.569547in}{0.699340in}}%
\pgfpathlineto{\pgfqpoint{4.590655in}{0.699340in}}%
\pgfpathlineto{\pgfqpoint{4.612853in}{0.652697in}}%
\pgfpathlineto{\pgfqpoint{4.768239in}{0.652697in}}%
\pgfpathlineto{\pgfqpoint{4.790437in}{0.625279in}}%
\pgfpathlineto{\pgfqpoint{5.212200in}{0.625279in}}%
\pgfpathlineto{\pgfqpoint{5.234398in}{0.582539in}}%
\pgfpathlineto{\pgfqpoint{5.700556in}{0.582539in}}%
\pgfpathlineto{\pgfqpoint{5.722754in}{0.569474in}}%
\pgfpathlineto{\pgfqpoint{5.900338in}{0.569474in}}%
\pgfpathlineto{\pgfqpoint{5.922536in}{0.536086in}}%
\pgfpathlineto{\pgfqpoint{5.966932in}{0.536086in}}%
\pgfpathlineto{\pgfqpoint{5.966932in}{0.536086in}}%
\pgfusepath{stroke}%
\end{pgfscope}%
\begin{pgfscope}%
\pgfpathrectangle{\pgfqpoint{0.862500in}{0.375000in}}{\pgfqpoint{5.347500in}{2.265000in}}%
\pgfusepath{clip}%
\pgfsetroundcap%
\pgfsetroundjoin%
\pgfsetlinewidth{1.505625pt}%
\definecolor{currentstroke}{rgb}{0.172549,0.627451,0.172549}%
\pgfsetstrokecolor{currentstroke}%
\pgfsetdash{}{0pt}%
\pgfpathmoveto{\pgfqpoint{1.105568in}{2.353759in}}%
\pgfpathlineto{\pgfqpoint{1.127766in}{2.276936in}}%
\pgfpathlineto{\pgfqpoint{1.149964in}{2.252453in}}%
\pgfpathlineto{\pgfqpoint{1.172162in}{2.187765in}}%
\pgfpathlineto{\pgfqpoint{1.194360in}{2.151852in}}%
\pgfpathlineto{\pgfqpoint{1.216558in}{2.142908in}}%
\pgfpathlineto{\pgfqpoint{1.238756in}{2.109960in}}%
\pgfpathlineto{\pgfqpoint{1.260954in}{2.066640in}}%
\pgfpathlineto{\pgfqpoint{1.283152in}{2.066640in}}%
\pgfpathlineto{\pgfqpoint{1.305350in}{2.042472in}}%
\pgfpathlineto{\pgfqpoint{1.349746in}{2.042472in}}%
\pgfpathlineto{\pgfqpoint{1.371944in}{2.020598in}}%
\pgfpathlineto{\pgfqpoint{1.416340in}{2.020598in}}%
\pgfpathlineto{\pgfqpoint{1.438538in}{2.013356in}}%
\pgfpathlineto{\pgfqpoint{1.527330in}{2.013356in}}%
\pgfpathlineto{\pgfqpoint{1.549528in}{1.966474in}}%
\pgfpathlineto{\pgfqpoint{1.593924in}{1.966474in}}%
\pgfpathlineto{\pgfqpoint{1.616122in}{1.944211in}}%
\pgfpathlineto{\pgfqpoint{1.638320in}{1.944211in}}%
\pgfpathlineto{\pgfqpoint{1.660518in}{1.897216in}}%
\pgfpathlineto{\pgfqpoint{1.682716in}{1.897216in}}%
\pgfpathlineto{\pgfqpoint{1.704914in}{1.893046in}}%
\pgfpathlineto{\pgfqpoint{1.727112in}{1.893046in}}%
\pgfpathlineto{\pgfqpoint{1.749310in}{1.877991in}}%
\pgfpathlineto{\pgfqpoint{1.815904in}{1.877991in}}%
\pgfpathlineto{\pgfqpoint{1.838102in}{1.870505in}}%
\pgfpathlineto{\pgfqpoint{1.860300in}{1.805302in}}%
\pgfpathlineto{\pgfqpoint{1.882498in}{1.804678in}}%
\pgfpathlineto{\pgfqpoint{1.904696in}{1.792822in}}%
\pgfpathlineto{\pgfqpoint{1.926894in}{1.664843in}}%
\pgfpathlineto{\pgfqpoint{2.126677in}{1.664843in}}%
\pgfpathlineto{\pgfqpoint{2.148875in}{1.603746in}}%
\pgfpathlineto{\pgfqpoint{2.171073in}{1.588265in}}%
\pgfpathlineto{\pgfqpoint{2.237667in}{1.588265in}}%
\pgfpathlineto{\pgfqpoint{2.259865in}{1.586126in}}%
\pgfpathlineto{\pgfqpoint{2.282063in}{1.500550in}}%
\pgfpathlineto{\pgfqpoint{2.304261in}{1.500550in}}%
\pgfpathlineto{\pgfqpoint{2.326459in}{1.383368in}}%
\pgfpathlineto{\pgfqpoint{2.348657in}{1.316952in}}%
\pgfpathlineto{\pgfqpoint{2.370855in}{1.316952in}}%
\pgfpathlineto{\pgfqpoint{2.393053in}{1.298296in}}%
\pgfpathlineto{\pgfqpoint{2.415251in}{1.298296in}}%
\pgfpathlineto{\pgfqpoint{2.437449in}{0.993194in}}%
\pgfpathlineto{\pgfqpoint{2.504043in}{0.993194in}}%
\pgfpathlineto{\pgfqpoint{2.526241in}{0.982576in}}%
\pgfpathlineto{\pgfqpoint{2.548439in}{0.839673in}}%
\pgfpathlineto{\pgfqpoint{3.258775in}{0.839673in}}%
\pgfpathlineto{\pgfqpoint{3.280973in}{0.832248in}}%
\pgfpathlineto{\pgfqpoint{3.436359in}{0.832248in}}%
\pgfpathlineto{\pgfqpoint{3.458557in}{0.817765in}}%
\pgfpathlineto{\pgfqpoint{3.702735in}{0.817765in}}%
\pgfpathlineto{\pgfqpoint{3.747131in}{0.741199in}}%
\pgfpathlineto{\pgfqpoint{3.924715in}{0.741199in}}%
\pgfpathlineto{\pgfqpoint{3.946913in}{0.642978in}}%
\pgfpathlineto{\pgfqpoint{4.013507in}{0.642978in}}%
\pgfpathlineto{\pgfqpoint{4.035705in}{0.620496in}}%
\pgfpathlineto{\pgfqpoint{4.080101in}{0.620496in}}%
\pgfpathlineto{\pgfqpoint{4.102299in}{0.589315in}}%
\pgfpathlineto{\pgfqpoint{5.367586in}{0.589315in}}%
\pgfpathlineto{\pgfqpoint{5.389784in}{0.587456in}}%
\pgfpathlineto{\pgfqpoint{5.966932in}{0.587456in}}%
\pgfpathlineto{\pgfqpoint{5.966932in}{0.587456in}}%
\pgfusepath{stroke}%
\end{pgfscope}%
\begin{pgfscope}%
\pgfpathrectangle{\pgfqpoint{0.862500in}{0.375000in}}{\pgfqpoint{5.347500in}{2.265000in}}%
\pgfusepath{clip}%
\pgfsetroundcap%
\pgfsetroundjoin%
\pgfsetlinewidth{1.505625pt}%
\definecolor{currentstroke}{rgb}{0.839216,0.152941,0.156863}%
\pgfsetstrokecolor{currentstroke}%
\pgfsetdash{}{0pt}%
\pgfpathmoveto{\pgfqpoint{1.105568in}{2.394450in}}%
\pgfpathlineto{\pgfqpoint{1.127766in}{2.244685in}}%
\pgfpathlineto{\pgfqpoint{1.149964in}{2.197873in}}%
\pgfpathlineto{\pgfqpoint{1.172162in}{2.187732in}}%
\pgfpathlineto{\pgfqpoint{1.194360in}{2.187732in}}%
\pgfpathlineto{\pgfqpoint{1.216558in}{2.068148in}}%
\pgfpathlineto{\pgfqpoint{1.438538in}{2.068148in}}%
\pgfpathlineto{\pgfqpoint{1.460736in}{2.060547in}}%
\pgfpathlineto{\pgfqpoint{1.482934in}{2.040851in}}%
\pgfpathlineto{\pgfqpoint{1.527330in}{2.040851in}}%
\pgfpathlineto{\pgfqpoint{1.549528in}{1.887091in}}%
\pgfpathlineto{\pgfqpoint{1.571726in}{1.844727in}}%
\pgfpathlineto{\pgfqpoint{1.593924in}{1.843799in}}%
\pgfpathlineto{\pgfqpoint{1.616122in}{1.707874in}}%
\pgfpathlineto{\pgfqpoint{1.638320in}{1.707712in}}%
\pgfpathlineto{\pgfqpoint{1.704914in}{1.688154in}}%
\pgfpathlineto{\pgfqpoint{1.749310in}{1.688154in}}%
\pgfpathlineto{\pgfqpoint{1.771508in}{1.685739in}}%
\pgfpathlineto{\pgfqpoint{1.793706in}{1.606527in}}%
\pgfpathlineto{\pgfqpoint{1.860300in}{1.605816in}}%
\pgfpathlineto{\pgfqpoint{1.882498in}{1.592884in}}%
\pgfpathlineto{\pgfqpoint{1.904696in}{1.568267in}}%
\pgfpathlineto{\pgfqpoint{1.926894in}{1.485905in}}%
\pgfpathlineto{\pgfqpoint{1.949092in}{1.485905in}}%
\pgfpathlineto{\pgfqpoint{1.971290in}{1.421782in}}%
\pgfpathlineto{\pgfqpoint{2.060083in}{1.421782in}}%
\pgfpathlineto{\pgfqpoint{2.082281in}{1.352782in}}%
\pgfpathlineto{\pgfqpoint{2.126677in}{1.351809in}}%
\pgfpathlineto{\pgfqpoint{2.148875in}{1.351809in}}%
\pgfpathlineto{\pgfqpoint{2.171073in}{1.261788in}}%
\pgfpathlineto{\pgfqpoint{3.058993in}{1.260916in}}%
\pgfpathlineto{\pgfqpoint{3.081191in}{1.228120in}}%
\pgfpathlineto{\pgfqpoint{3.702735in}{1.228120in}}%
\pgfpathlineto{\pgfqpoint{3.724933in}{1.177298in}}%
\pgfpathlineto{\pgfqpoint{5.101210in}{1.177298in}}%
\pgfpathlineto{\pgfqpoint{5.123408in}{0.966991in}}%
\pgfpathlineto{\pgfqpoint{5.833744in}{0.966991in}}%
\pgfpathlineto{\pgfqpoint{5.855942in}{0.844715in}}%
\pgfpathlineto{\pgfqpoint{5.966932in}{0.844715in}}%
\pgfpathlineto{\pgfqpoint{5.966932in}{0.844715in}}%
\pgfusepath{stroke}%
\end{pgfscope}%
\begin{pgfscope}%
\pgfsetrectcap%
\pgfsetmiterjoin%
\pgfsetlinewidth{0.000000pt}%
\definecolor{currentstroke}{rgb}{1.000000,1.000000,1.000000}%
\pgfsetstrokecolor{currentstroke}%
\pgfsetdash{}{0pt}%
\pgfpathmoveto{\pgfqpoint{0.862500in}{0.375000in}}%
\pgfpathlineto{\pgfqpoint{0.862500in}{2.640000in}}%
\pgfusepath{}%
\end{pgfscope}%
\begin{pgfscope}%
\pgfsetrectcap%
\pgfsetmiterjoin%
\pgfsetlinewidth{0.000000pt}%
\definecolor{currentstroke}{rgb}{1.000000,1.000000,1.000000}%
\pgfsetstrokecolor{currentstroke}%
\pgfsetdash{}{0pt}%
\pgfpathmoveto{\pgfqpoint{6.210000in}{0.375000in}}%
\pgfpathlineto{\pgfqpoint{6.210000in}{2.640000in}}%
\pgfusepath{}%
\end{pgfscope}%
\begin{pgfscope}%
\pgfsetrectcap%
\pgfsetmiterjoin%
\pgfsetlinewidth{0.000000pt}%
\definecolor{currentstroke}{rgb}{1.000000,1.000000,1.000000}%
\pgfsetstrokecolor{currentstroke}%
\pgfsetdash{}{0pt}%
\pgfpathmoveto{\pgfqpoint{0.862500in}{0.375000in}}%
\pgfpathlineto{\pgfqpoint{6.210000in}{0.375000in}}%
\pgfusepath{}%
\end{pgfscope}%
\begin{pgfscope}%
\pgfsetrectcap%
\pgfsetmiterjoin%
\pgfsetlinewidth{0.000000pt}%
\definecolor{currentstroke}{rgb}{1.000000,1.000000,1.000000}%
\pgfsetstrokecolor{currentstroke}%
\pgfsetdash{}{0pt}%
\pgfpathmoveto{\pgfqpoint{0.862500in}{2.640000in}}%
\pgfpathlineto{\pgfqpoint{6.210000in}{2.640000in}}%
\pgfusepath{}%
\end{pgfscope}%
\begin{pgfscope}%
\definecolor{textcolor}{rgb}{0.150000,0.150000,0.150000}%
\pgfsetstrokecolor{textcolor}%
\pgfsetfillcolor{textcolor}%
\pgftext[x=3.536250in,y=2.723333in,,base]{\color{textcolor}\rmfamily\fontsize{8.000000}{9.600000}\selectfont Alpine01}%
\end{pgfscope}%
\begin{pgfscope}%
\pgfsetroundcap%
\pgfsetroundjoin%
\pgfsetlinewidth{1.505625pt}%
\definecolor{currentstroke}{rgb}{0.121569,0.466667,0.705882}%
\pgfsetstrokecolor{currentstroke}%
\pgfsetdash{}{0pt}%
\pgfpathmoveto{\pgfqpoint{4.479607in}{2.494470in}}%
\pgfpathlineto{\pgfqpoint{4.701829in}{2.494470in}}%
\pgfusepath{stroke}%
\end{pgfscope}%
\begin{pgfscope}%
\definecolor{textcolor}{rgb}{0.150000,0.150000,0.150000}%
\pgfsetstrokecolor{textcolor}%
\pgfsetfillcolor{textcolor}%
\pgftext[x=4.790718in,y=2.455582in,left,base]{\color{textcolor}\rmfamily\fontsize{8.000000}{9.600000}\selectfont random}%
\end{pgfscope}%
\begin{pgfscope}%
\pgfsetroundcap%
\pgfsetroundjoin%
\pgfsetlinewidth{1.505625pt}%
\definecolor{currentstroke}{rgb}{1.000000,0.498039,0.054902}%
\pgfsetstrokecolor{currentstroke}%
\pgfsetdash{}{0pt}%
\pgfpathmoveto{\pgfqpoint{4.479607in}{2.331385in}}%
\pgfpathlineto{\pgfqpoint{4.701829in}{2.331385in}}%
\pgfusepath{stroke}%
\end{pgfscope}%
\begin{pgfscope}%
\definecolor{textcolor}{rgb}{0.150000,0.150000,0.150000}%
\pgfsetstrokecolor{textcolor}%
\pgfsetfillcolor{textcolor}%
\pgftext[x=4.790718in,y=2.292496in,left,base]{\color{textcolor}\rmfamily\fontsize{8.000000}{9.600000}\selectfont 5 x DNGO retrain-reset}%
\end{pgfscope}%
\begin{pgfscope}%
\pgfsetroundcap%
\pgfsetroundjoin%
\pgfsetlinewidth{1.505625pt}%
\definecolor{currentstroke}{rgb}{0.172549,0.627451,0.172549}%
\pgfsetstrokecolor{currentstroke}%
\pgfsetdash{}{0pt}%
\pgfpathmoveto{\pgfqpoint{4.479607in}{2.168299in}}%
\pgfpathlineto{\pgfqpoint{4.701829in}{2.168299in}}%
\pgfusepath{stroke}%
\end{pgfscope}%
\begin{pgfscope}%
\definecolor{textcolor}{rgb}{0.150000,0.150000,0.150000}%
\pgfsetstrokecolor{textcolor}%
\pgfsetfillcolor{textcolor}%
\pgftext[x=4.790718in,y=2.129410in,left,base]{\color{textcolor}\rmfamily\fontsize{8.000000}{9.600000}\selectfont DNGO retrain-reset}%
\end{pgfscope}%
\begin{pgfscope}%
\pgfsetroundcap%
\pgfsetroundjoin%
\pgfsetlinewidth{1.505625pt}%
\definecolor{currentstroke}{rgb}{0.839216,0.152941,0.156863}%
\pgfsetstrokecolor{currentstroke}%
\pgfsetdash{}{0pt}%
\pgfpathmoveto{\pgfqpoint{4.479607in}{2.005213in}}%
\pgfpathlineto{\pgfqpoint{4.701829in}{2.005213in}}%
\pgfusepath{stroke}%
\end{pgfscope}%
\begin{pgfscope}%
\definecolor{textcolor}{rgb}{0.150000,0.150000,0.150000}%
\pgfsetstrokecolor{textcolor}%
\pgfsetfillcolor{textcolor}%
\pgftext[x=4.790718in,y=1.966324in,left,base]{\color{textcolor}\rmfamily\fontsize{8.000000}{9.600000}\selectfont GP}%
\end{pgfscope}%
\end{pgfpicture}%
\makeatother%
\endgroup%

            %% Creator: Matplotlib, PGF backend
%%
%% To include the figure in your LaTeX document, write
%%   \input{<filename>.pgf}
%%
%% Make sure the required packages are loaded in your preamble
%%   \usepackage{pgf}
%%
%% Figures using additional raster images can only be included by \input if
%% they are in the same directory as the main LaTeX file. For loading figures
%% from other directories you can use the `import` package
%%   \usepackage{import}
%% and then include the figures with
%%   \import{<path to file>}{<filename>.pgf}
%%
%% Matplotlib used the following preamble
%%   \usepackage{gensymb}
%%   \usepackage{fontspec}
%%   \setmainfont{DejaVu Serif}
%%   \setsansfont{Arial}
%%   \setmonofont{DejaVu Sans Mono}
%%
\begingroup%
\makeatletter%
\begin{pgfpicture}%
\pgfpathrectangle{\pgfpointorigin}{\pgfqpoint{6.900000in}{3.000000in}}%
\pgfusepath{use as bounding box, clip}%
\begin{pgfscope}%
\pgfsetbuttcap%
\pgfsetmiterjoin%
\definecolor{currentfill}{rgb}{1.000000,1.000000,1.000000}%
\pgfsetfillcolor{currentfill}%
\pgfsetlinewidth{0.000000pt}%
\definecolor{currentstroke}{rgb}{1.000000,1.000000,1.000000}%
\pgfsetstrokecolor{currentstroke}%
\pgfsetdash{}{0pt}%
\pgfpathmoveto{\pgfqpoint{0.000000in}{0.000000in}}%
\pgfpathlineto{\pgfqpoint{6.900000in}{0.000000in}}%
\pgfpathlineto{\pgfqpoint{6.900000in}{3.000000in}}%
\pgfpathlineto{\pgfqpoint{0.000000in}{3.000000in}}%
\pgfpathclose%
\pgfusepath{fill}%
\end{pgfscope}%
\begin{pgfscope}%
\pgfsetbuttcap%
\pgfsetmiterjoin%
\definecolor{currentfill}{rgb}{0.917647,0.917647,0.949020}%
\pgfsetfillcolor{currentfill}%
\pgfsetlinewidth{0.000000pt}%
\definecolor{currentstroke}{rgb}{0.000000,0.000000,0.000000}%
\pgfsetstrokecolor{currentstroke}%
\pgfsetstrokeopacity{0.000000}%
\pgfsetdash{}{0pt}%
\pgfpathmoveto{\pgfqpoint{0.862500in}{0.375000in}}%
\pgfpathlineto{\pgfqpoint{6.210000in}{0.375000in}}%
\pgfpathlineto{\pgfqpoint{6.210000in}{2.640000in}}%
\pgfpathlineto{\pgfqpoint{0.862500in}{2.640000in}}%
\pgfpathclose%
\pgfusepath{fill}%
\end{pgfscope}%
\begin{pgfscope}%
\pgfpathrectangle{\pgfqpoint{0.862500in}{0.375000in}}{\pgfqpoint{5.347500in}{2.265000in}}%
\pgfusepath{clip}%
\pgfsetroundcap%
\pgfsetroundjoin%
\pgfsetlinewidth{0.803000pt}%
\definecolor{currentstroke}{rgb}{1.000000,1.000000,1.000000}%
\pgfsetstrokecolor{currentstroke}%
\pgfsetdash{}{0pt}%
\pgfpathmoveto{\pgfqpoint{1.105568in}{0.375000in}}%
\pgfpathlineto{\pgfqpoint{1.105568in}{2.640000in}}%
\pgfusepath{stroke}%
\end{pgfscope}%
\begin{pgfscope}%
\definecolor{textcolor}{rgb}{0.150000,0.150000,0.150000}%
\pgfsetstrokecolor{textcolor}%
\pgfsetfillcolor{textcolor}%
\pgftext[x=1.105568in,y=0.326389in,,top]{\color{textcolor}\rmfamily\fontsize{8.000000}{9.600000}\selectfont \(\displaystyle 0\)}%
\end{pgfscope}%
\begin{pgfscope}%
\pgfpathrectangle{\pgfqpoint{0.862500in}{0.375000in}}{\pgfqpoint{5.347500in}{2.265000in}}%
\pgfusepath{clip}%
\pgfsetroundcap%
\pgfsetroundjoin%
\pgfsetlinewidth{0.803000pt}%
\definecolor{currentstroke}{rgb}{1.000000,1.000000,1.000000}%
\pgfsetstrokecolor{currentstroke}%
\pgfsetdash{}{0pt}%
\pgfpathmoveto{\pgfqpoint{1.824705in}{0.375000in}}%
\pgfpathlineto{\pgfqpoint{1.824705in}{2.640000in}}%
\pgfusepath{stroke}%
\end{pgfscope}%
\begin{pgfscope}%
\definecolor{textcolor}{rgb}{0.150000,0.150000,0.150000}%
\pgfsetstrokecolor{textcolor}%
\pgfsetfillcolor{textcolor}%
\pgftext[x=1.824705in,y=0.326389in,,top]{\color{textcolor}\rmfamily\fontsize{8.000000}{9.600000}\selectfont \(\displaystyle 25\)}%
\end{pgfscope}%
\begin{pgfscope}%
\pgfpathrectangle{\pgfqpoint{0.862500in}{0.375000in}}{\pgfqpoint{5.347500in}{2.265000in}}%
\pgfusepath{clip}%
\pgfsetroundcap%
\pgfsetroundjoin%
\pgfsetlinewidth{0.803000pt}%
\definecolor{currentstroke}{rgb}{1.000000,1.000000,1.000000}%
\pgfsetstrokecolor{currentstroke}%
\pgfsetdash{}{0pt}%
\pgfpathmoveto{\pgfqpoint{2.543841in}{0.375000in}}%
\pgfpathlineto{\pgfqpoint{2.543841in}{2.640000in}}%
\pgfusepath{stroke}%
\end{pgfscope}%
\begin{pgfscope}%
\definecolor{textcolor}{rgb}{0.150000,0.150000,0.150000}%
\pgfsetstrokecolor{textcolor}%
\pgfsetfillcolor{textcolor}%
\pgftext[x=2.543841in,y=0.326389in,,top]{\color{textcolor}\rmfamily\fontsize{8.000000}{9.600000}\selectfont \(\displaystyle 50\)}%
\end{pgfscope}%
\begin{pgfscope}%
\pgfpathrectangle{\pgfqpoint{0.862500in}{0.375000in}}{\pgfqpoint{5.347500in}{2.265000in}}%
\pgfusepath{clip}%
\pgfsetroundcap%
\pgfsetroundjoin%
\pgfsetlinewidth{0.803000pt}%
\definecolor{currentstroke}{rgb}{1.000000,1.000000,1.000000}%
\pgfsetstrokecolor{currentstroke}%
\pgfsetdash{}{0pt}%
\pgfpathmoveto{\pgfqpoint{3.262978in}{0.375000in}}%
\pgfpathlineto{\pgfqpoint{3.262978in}{2.640000in}}%
\pgfusepath{stroke}%
\end{pgfscope}%
\begin{pgfscope}%
\definecolor{textcolor}{rgb}{0.150000,0.150000,0.150000}%
\pgfsetstrokecolor{textcolor}%
\pgfsetfillcolor{textcolor}%
\pgftext[x=3.262978in,y=0.326389in,,top]{\color{textcolor}\rmfamily\fontsize{8.000000}{9.600000}\selectfont \(\displaystyle 75\)}%
\end{pgfscope}%
\begin{pgfscope}%
\pgfpathrectangle{\pgfqpoint{0.862500in}{0.375000in}}{\pgfqpoint{5.347500in}{2.265000in}}%
\pgfusepath{clip}%
\pgfsetroundcap%
\pgfsetroundjoin%
\pgfsetlinewidth{0.803000pt}%
\definecolor{currentstroke}{rgb}{1.000000,1.000000,1.000000}%
\pgfsetstrokecolor{currentstroke}%
\pgfsetdash{}{0pt}%
\pgfpathmoveto{\pgfqpoint{3.982115in}{0.375000in}}%
\pgfpathlineto{\pgfqpoint{3.982115in}{2.640000in}}%
\pgfusepath{stroke}%
\end{pgfscope}%
\begin{pgfscope}%
\definecolor{textcolor}{rgb}{0.150000,0.150000,0.150000}%
\pgfsetstrokecolor{textcolor}%
\pgfsetfillcolor{textcolor}%
\pgftext[x=3.982115in,y=0.326389in,,top]{\color{textcolor}\rmfamily\fontsize{8.000000}{9.600000}\selectfont \(\displaystyle 100\)}%
\end{pgfscope}%
\begin{pgfscope}%
\pgfpathrectangle{\pgfqpoint{0.862500in}{0.375000in}}{\pgfqpoint{5.347500in}{2.265000in}}%
\pgfusepath{clip}%
\pgfsetroundcap%
\pgfsetroundjoin%
\pgfsetlinewidth{0.803000pt}%
\definecolor{currentstroke}{rgb}{1.000000,1.000000,1.000000}%
\pgfsetstrokecolor{currentstroke}%
\pgfsetdash{}{0pt}%
\pgfpathmoveto{\pgfqpoint{4.701251in}{0.375000in}}%
\pgfpathlineto{\pgfqpoint{4.701251in}{2.640000in}}%
\pgfusepath{stroke}%
\end{pgfscope}%
\begin{pgfscope}%
\definecolor{textcolor}{rgb}{0.150000,0.150000,0.150000}%
\pgfsetstrokecolor{textcolor}%
\pgfsetfillcolor{textcolor}%
\pgftext[x=4.701251in,y=0.326389in,,top]{\color{textcolor}\rmfamily\fontsize{8.000000}{9.600000}\selectfont \(\displaystyle 125\)}%
\end{pgfscope}%
\begin{pgfscope}%
\pgfpathrectangle{\pgfqpoint{0.862500in}{0.375000in}}{\pgfqpoint{5.347500in}{2.265000in}}%
\pgfusepath{clip}%
\pgfsetroundcap%
\pgfsetroundjoin%
\pgfsetlinewidth{0.803000pt}%
\definecolor{currentstroke}{rgb}{1.000000,1.000000,1.000000}%
\pgfsetstrokecolor{currentstroke}%
\pgfsetdash{}{0pt}%
\pgfpathmoveto{\pgfqpoint{5.420388in}{0.375000in}}%
\pgfpathlineto{\pgfqpoint{5.420388in}{2.640000in}}%
\pgfusepath{stroke}%
\end{pgfscope}%
\begin{pgfscope}%
\definecolor{textcolor}{rgb}{0.150000,0.150000,0.150000}%
\pgfsetstrokecolor{textcolor}%
\pgfsetfillcolor{textcolor}%
\pgftext[x=5.420388in,y=0.326389in,,top]{\color{textcolor}\rmfamily\fontsize{8.000000}{9.600000}\selectfont \(\displaystyle 150\)}%
\end{pgfscope}%
\begin{pgfscope}%
\pgfpathrectangle{\pgfqpoint{0.862500in}{0.375000in}}{\pgfqpoint{5.347500in}{2.265000in}}%
\pgfusepath{clip}%
\pgfsetroundcap%
\pgfsetroundjoin%
\pgfsetlinewidth{0.803000pt}%
\definecolor{currentstroke}{rgb}{1.000000,1.000000,1.000000}%
\pgfsetstrokecolor{currentstroke}%
\pgfsetdash{}{0pt}%
\pgfpathmoveto{\pgfqpoint{6.139525in}{0.375000in}}%
\pgfpathlineto{\pgfqpoint{6.139525in}{2.640000in}}%
\pgfusepath{stroke}%
\end{pgfscope}%
\begin{pgfscope}%
\definecolor{textcolor}{rgb}{0.150000,0.150000,0.150000}%
\pgfsetstrokecolor{textcolor}%
\pgfsetfillcolor{textcolor}%
\pgftext[x=6.139525in,y=0.326389in,,top]{\color{textcolor}\rmfamily\fontsize{8.000000}{9.600000}\selectfont \(\displaystyle 175\)}%
\end{pgfscope}%
\begin{pgfscope}%
\definecolor{textcolor}{rgb}{0.150000,0.150000,0.150000}%
\pgfsetstrokecolor{textcolor}%
\pgfsetfillcolor{textcolor}%
\pgftext[x=3.536250in,y=0.163303in,,top]{\color{textcolor}\rmfamily\fontsize{8.000000}{9.600000}\selectfont Step}%
\end{pgfscope}%
\begin{pgfscope}%
\pgfpathrectangle{\pgfqpoint{0.862500in}{0.375000in}}{\pgfqpoint{5.347500in}{2.265000in}}%
\pgfusepath{clip}%
\pgfsetroundcap%
\pgfsetroundjoin%
\pgfsetlinewidth{0.803000pt}%
\definecolor{currentstroke}{rgb}{1.000000,1.000000,1.000000}%
\pgfsetstrokecolor{currentstroke}%
\pgfsetdash{}{0pt}%
\pgfpathmoveto{\pgfqpoint{0.862500in}{1.050847in}}%
\pgfpathlineto{\pgfqpoint{6.210000in}{1.050847in}}%
\pgfusepath{stroke}%
\end{pgfscope}%
\begin{pgfscope}%
\definecolor{textcolor}{rgb}{0.150000,0.150000,0.150000}%
\pgfsetstrokecolor{textcolor}%
\pgfsetfillcolor{textcolor}%
\pgftext[x=0.557716in,y=1.008638in,left,base]{\color{textcolor}\rmfamily\fontsize{8.000000}{9.600000}\selectfont \(\displaystyle 10^{-1}\)}%
\end{pgfscope}%
\begin{pgfscope}%
\pgfpathrectangle{\pgfqpoint{0.862500in}{0.375000in}}{\pgfqpoint{5.347500in}{2.265000in}}%
\pgfusepath{clip}%
\pgfsetroundcap%
\pgfsetroundjoin%
\pgfsetlinewidth{0.803000pt}%
\definecolor{currentstroke}{rgb}{1.000000,1.000000,1.000000}%
\pgfsetstrokecolor{currentstroke}%
\pgfsetdash{}{0pt}%
\pgfpathmoveto{\pgfqpoint{0.862500in}{2.030870in}}%
\pgfpathlineto{\pgfqpoint{6.210000in}{2.030870in}}%
\pgfusepath{stroke}%
\end{pgfscope}%
\begin{pgfscope}%
\definecolor{textcolor}{rgb}{0.150000,0.150000,0.150000}%
\pgfsetstrokecolor{textcolor}%
\pgfsetfillcolor{textcolor}%
\pgftext[x=0.637962in,y=1.988661in,left,base]{\color{textcolor}\rmfamily\fontsize{8.000000}{9.600000}\selectfont \(\displaystyle 10^{0}\)}%
\end{pgfscope}%
\begin{pgfscope}%
\definecolor{textcolor}{rgb}{0.150000,0.150000,0.150000}%
\pgfsetstrokecolor{textcolor}%
\pgfsetfillcolor{textcolor}%
\pgftext[x=0.502160in,y=1.507500in,,bottom,rotate=90.000000]{\color{textcolor}\rmfamily\fontsize{8.000000}{9.600000}\selectfont Simple Regret}%
\end{pgfscope}%
\begin{pgfscope}%
\pgfpathrectangle{\pgfqpoint{0.862500in}{0.375000in}}{\pgfqpoint{5.347500in}{2.265000in}}%
\pgfusepath{clip}%
\pgfsetbuttcap%
\pgfsetroundjoin%
\definecolor{currentfill}{rgb}{0.121569,0.466667,0.705882}%
\pgfsetfillcolor{currentfill}%
\pgfsetfillopacity{0.200000}%
\pgfsetlinewidth{0.000000pt}%
\definecolor{currentstroke}{rgb}{0.000000,0.000000,0.000000}%
\pgfsetstrokecolor{currentstroke}%
\pgfsetdash{}{0pt}%
\pgfpathmoveto{\pgfqpoint{1.105568in}{2.482680in}}%
\pgfpathlineto{\pgfqpoint{1.105568in}{2.524333in}}%
\pgfpathlineto{\pgfqpoint{1.134334in}{2.524052in}}%
\pgfpathlineto{\pgfqpoint{1.163099in}{2.506836in}}%
\pgfpathlineto{\pgfqpoint{1.191865in}{2.506836in}}%
\pgfpathlineto{\pgfqpoint{1.220630in}{2.488307in}}%
\pgfpathlineto{\pgfqpoint{1.249396in}{2.476475in}}%
\pgfpathlineto{\pgfqpoint{1.278161in}{2.469388in}}%
\pgfpathlineto{\pgfqpoint{1.306926in}{2.464437in}}%
\pgfpathlineto{\pgfqpoint{1.335692in}{2.464437in}}%
\pgfpathlineto{\pgfqpoint{1.364457in}{2.464007in}}%
\pgfpathlineto{\pgfqpoint{1.393223in}{2.439649in}}%
\pgfpathlineto{\pgfqpoint{1.421988in}{2.429740in}}%
\pgfpathlineto{\pgfqpoint{1.450754in}{2.426876in}}%
\pgfpathlineto{\pgfqpoint{1.479519in}{2.423154in}}%
\pgfpathlineto{\pgfqpoint{1.508285in}{2.396747in}}%
\pgfpathlineto{\pgfqpoint{1.537050in}{2.388954in}}%
\pgfpathlineto{\pgfqpoint{1.565816in}{2.388954in}}%
\pgfpathlineto{\pgfqpoint{1.594581in}{2.388954in}}%
\pgfpathlineto{\pgfqpoint{1.623347in}{2.359944in}}%
\pgfpathlineto{\pgfqpoint{1.652112in}{2.359944in}}%
\pgfpathlineto{\pgfqpoint{1.680877in}{2.359944in}}%
\pgfpathlineto{\pgfqpoint{1.709643in}{2.319423in}}%
\pgfpathlineto{\pgfqpoint{1.738408in}{2.319423in}}%
\pgfpathlineto{\pgfqpoint{1.767174in}{2.319423in}}%
\pgfpathlineto{\pgfqpoint{1.795939in}{2.319423in}}%
\pgfpathlineto{\pgfqpoint{1.824705in}{2.319423in}}%
\pgfpathlineto{\pgfqpoint{1.853470in}{2.319423in}}%
\pgfpathlineto{\pgfqpoint{1.882236in}{2.319423in}}%
\pgfpathlineto{\pgfqpoint{1.911001in}{2.319423in}}%
\pgfpathlineto{\pgfqpoint{1.939767in}{2.313289in}}%
\pgfpathlineto{\pgfqpoint{1.968532in}{2.300406in}}%
\pgfpathlineto{\pgfqpoint{1.997298in}{2.300406in}}%
\pgfpathlineto{\pgfqpoint{2.026063in}{2.300406in}}%
\pgfpathlineto{\pgfqpoint{2.054829in}{2.300406in}}%
\pgfpathlineto{\pgfqpoint{2.083594in}{2.255705in}}%
\pgfpathlineto{\pgfqpoint{2.112359in}{2.255705in}}%
\pgfpathlineto{\pgfqpoint{2.141125in}{2.255705in}}%
\pgfpathlineto{\pgfqpoint{2.169890in}{2.255705in}}%
\pgfpathlineto{\pgfqpoint{2.198656in}{2.252984in}}%
\pgfpathlineto{\pgfqpoint{2.227421in}{2.252984in}}%
\pgfpathlineto{\pgfqpoint{2.256187in}{2.252984in}}%
\pgfpathlineto{\pgfqpoint{2.284952in}{2.252984in}}%
\pgfpathlineto{\pgfqpoint{2.313718in}{2.252984in}}%
\pgfpathlineto{\pgfqpoint{2.342483in}{2.252984in}}%
\pgfpathlineto{\pgfqpoint{2.371249in}{2.252984in}}%
\pgfpathlineto{\pgfqpoint{2.400014in}{2.252984in}}%
\pgfpathlineto{\pgfqpoint{2.428780in}{2.252984in}}%
\pgfpathlineto{\pgfqpoint{2.457545in}{2.225638in}}%
\pgfpathlineto{\pgfqpoint{2.486311in}{2.225638in}}%
\pgfpathlineto{\pgfqpoint{2.515076in}{2.225638in}}%
\pgfpathlineto{\pgfqpoint{2.543841in}{2.225638in}}%
\pgfpathlineto{\pgfqpoint{2.572607in}{2.225638in}}%
\pgfpathlineto{\pgfqpoint{2.601372in}{2.225638in}}%
\pgfpathlineto{\pgfqpoint{2.630138in}{2.225638in}}%
\pgfpathlineto{\pgfqpoint{2.658903in}{2.225638in}}%
\pgfpathlineto{\pgfqpoint{2.687669in}{2.225638in}}%
\pgfpathlineto{\pgfqpoint{2.716434in}{2.192793in}}%
\pgfpathlineto{\pgfqpoint{2.745200in}{2.192793in}}%
\pgfpathlineto{\pgfqpoint{2.773965in}{2.192793in}}%
\pgfpathlineto{\pgfqpoint{2.802731in}{2.192793in}}%
\pgfpathlineto{\pgfqpoint{2.831496in}{2.192793in}}%
\pgfpathlineto{\pgfqpoint{2.860262in}{2.192793in}}%
\pgfpathlineto{\pgfqpoint{2.889027in}{2.192793in}}%
\pgfpathlineto{\pgfqpoint{2.917792in}{2.192793in}}%
\pgfpathlineto{\pgfqpoint{2.946558in}{2.172723in}}%
\pgfpathlineto{\pgfqpoint{2.975323in}{2.172723in}}%
\pgfpathlineto{\pgfqpoint{3.004089in}{2.172723in}}%
\pgfpathlineto{\pgfqpoint{3.032854in}{2.172723in}}%
\pgfpathlineto{\pgfqpoint{3.061620in}{2.172723in}}%
\pgfpathlineto{\pgfqpoint{3.090385in}{2.172723in}}%
\pgfpathlineto{\pgfqpoint{3.119151in}{2.172723in}}%
\pgfpathlineto{\pgfqpoint{3.147916in}{2.172723in}}%
\pgfpathlineto{\pgfqpoint{3.176682in}{2.172723in}}%
\pgfpathlineto{\pgfqpoint{3.205447in}{2.172723in}}%
\pgfpathlineto{\pgfqpoint{3.234213in}{2.172723in}}%
\pgfpathlineto{\pgfqpoint{3.262978in}{2.172723in}}%
\pgfpathlineto{\pgfqpoint{3.291744in}{2.172723in}}%
\pgfpathlineto{\pgfqpoint{3.320509in}{2.172723in}}%
\pgfpathlineto{\pgfqpoint{3.349274in}{2.172723in}}%
\pgfpathlineto{\pgfqpoint{3.378040in}{2.172723in}}%
\pgfpathlineto{\pgfqpoint{3.406805in}{2.155038in}}%
\pgfpathlineto{\pgfqpoint{3.435571in}{2.155038in}}%
\pgfpathlineto{\pgfqpoint{3.464336in}{2.147809in}}%
\pgfpathlineto{\pgfqpoint{3.493102in}{2.147809in}}%
\pgfpathlineto{\pgfqpoint{3.521867in}{2.147809in}}%
\pgfpathlineto{\pgfqpoint{3.550633in}{2.147809in}}%
\pgfpathlineto{\pgfqpoint{3.579398in}{2.147809in}}%
\pgfpathlineto{\pgfqpoint{3.608164in}{2.147809in}}%
\pgfpathlineto{\pgfqpoint{3.636929in}{2.147809in}}%
\pgfpathlineto{\pgfqpoint{3.665695in}{2.147809in}}%
\pgfpathlineto{\pgfqpoint{3.694460in}{2.147809in}}%
\pgfpathlineto{\pgfqpoint{3.723226in}{2.147809in}}%
\pgfpathlineto{\pgfqpoint{3.751991in}{2.147809in}}%
\pgfpathlineto{\pgfqpoint{3.780756in}{2.147809in}}%
\pgfpathlineto{\pgfqpoint{3.809522in}{2.147809in}}%
\pgfpathlineto{\pgfqpoint{3.838287in}{2.147809in}}%
\pgfpathlineto{\pgfqpoint{3.867053in}{2.147809in}}%
\pgfpathlineto{\pgfqpoint{3.895818in}{2.147809in}}%
\pgfpathlineto{\pgfqpoint{3.924584in}{2.147809in}}%
\pgfpathlineto{\pgfqpoint{3.953349in}{2.147809in}}%
\pgfpathlineto{\pgfqpoint{3.982115in}{2.147809in}}%
\pgfpathlineto{\pgfqpoint{4.010880in}{2.147809in}}%
\pgfpathlineto{\pgfqpoint{4.039646in}{2.147809in}}%
\pgfpathlineto{\pgfqpoint{4.068411in}{2.147809in}}%
\pgfpathlineto{\pgfqpoint{4.097177in}{2.147809in}}%
\pgfpathlineto{\pgfqpoint{4.125942in}{2.147809in}}%
\pgfpathlineto{\pgfqpoint{4.154708in}{2.147809in}}%
\pgfpathlineto{\pgfqpoint{4.183473in}{2.147809in}}%
\pgfpathlineto{\pgfqpoint{4.212238in}{2.147809in}}%
\pgfpathlineto{\pgfqpoint{4.241004in}{2.147809in}}%
\pgfpathlineto{\pgfqpoint{4.269769in}{2.147809in}}%
\pgfpathlineto{\pgfqpoint{4.298535in}{2.147809in}}%
\pgfpathlineto{\pgfqpoint{4.327300in}{2.147809in}}%
\pgfpathlineto{\pgfqpoint{4.356066in}{2.147809in}}%
\pgfpathlineto{\pgfqpoint{4.384831in}{2.147809in}}%
\pgfpathlineto{\pgfqpoint{4.413597in}{2.147809in}}%
\pgfpathlineto{\pgfqpoint{4.442362in}{2.147809in}}%
\pgfpathlineto{\pgfqpoint{4.471128in}{2.147809in}}%
\pgfpathlineto{\pgfqpoint{4.499893in}{2.147809in}}%
\pgfpathlineto{\pgfqpoint{4.528659in}{2.147809in}}%
\pgfpathlineto{\pgfqpoint{4.557424in}{2.147809in}}%
\pgfpathlineto{\pgfqpoint{4.586189in}{2.147809in}}%
\pgfpathlineto{\pgfqpoint{4.614955in}{2.104719in}}%
\pgfpathlineto{\pgfqpoint{4.643720in}{2.104719in}}%
\pgfpathlineto{\pgfqpoint{4.672486in}{2.104719in}}%
\pgfpathlineto{\pgfqpoint{4.701251in}{2.104719in}}%
\pgfpathlineto{\pgfqpoint{4.730017in}{2.104719in}}%
\pgfpathlineto{\pgfqpoint{4.758782in}{2.104719in}}%
\pgfpathlineto{\pgfqpoint{4.787548in}{2.104719in}}%
\pgfpathlineto{\pgfqpoint{4.816313in}{2.104719in}}%
\pgfpathlineto{\pgfqpoint{4.845079in}{2.104719in}}%
\pgfpathlineto{\pgfqpoint{4.873844in}{2.104719in}}%
\pgfpathlineto{\pgfqpoint{4.902610in}{2.104719in}}%
\pgfpathlineto{\pgfqpoint{4.931375in}{2.104719in}}%
\pgfpathlineto{\pgfqpoint{4.960141in}{2.104719in}}%
\pgfpathlineto{\pgfqpoint{4.988906in}{2.104719in}}%
\pgfpathlineto{\pgfqpoint{5.017671in}{2.104719in}}%
\pgfpathlineto{\pgfqpoint{5.046437in}{2.104719in}}%
\pgfpathlineto{\pgfqpoint{5.075202in}{2.104719in}}%
\pgfpathlineto{\pgfqpoint{5.103968in}{2.104719in}}%
\pgfpathlineto{\pgfqpoint{5.132733in}{2.104719in}}%
\pgfpathlineto{\pgfqpoint{5.161499in}{2.104719in}}%
\pgfpathlineto{\pgfqpoint{5.190264in}{2.104719in}}%
\pgfpathlineto{\pgfqpoint{5.219030in}{2.101274in}}%
\pgfpathlineto{\pgfqpoint{5.247795in}{2.101274in}}%
\pgfpathlineto{\pgfqpoint{5.276561in}{2.101274in}}%
\pgfpathlineto{\pgfqpoint{5.305326in}{2.097546in}}%
\pgfpathlineto{\pgfqpoint{5.334092in}{2.097546in}}%
\pgfpathlineto{\pgfqpoint{5.362857in}{2.097546in}}%
\pgfpathlineto{\pgfqpoint{5.391623in}{2.097546in}}%
\pgfpathlineto{\pgfqpoint{5.420388in}{2.097546in}}%
\pgfpathlineto{\pgfqpoint{5.449153in}{2.097546in}}%
\pgfpathlineto{\pgfqpoint{5.477919in}{2.097546in}}%
\pgfpathlineto{\pgfqpoint{5.506684in}{2.097546in}}%
\pgfpathlineto{\pgfqpoint{5.535450in}{2.097546in}}%
\pgfpathlineto{\pgfqpoint{5.564215in}{2.097546in}}%
\pgfpathlineto{\pgfqpoint{5.592981in}{2.097546in}}%
\pgfpathlineto{\pgfqpoint{5.621746in}{2.097546in}}%
\pgfpathlineto{\pgfqpoint{5.650512in}{2.097546in}}%
\pgfpathlineto{\pgfqpoint{5.679277in}{2.097546in}}%
\pgfpathlineto{\pgfqpoint{5.708043in}{2.097546in}}%
\pgfpathlineto{\pgfqpoint{5.736808in}{2.097546in}}%
\pgfpathlineto{\pgfqpoint{5.765574in}{2.097546in}}%
\pgfpathlineto{\pgfqpoint{5.794339in}{2.097546in}}%
\pgfpathlineto{\pgfqpoint{5.823104in}{2.097546in}}%
\pgfpathlineto{\pgfqpoint{5.851870in}{2.097546in}}%
\pgfpathlineto{\pgfqpoint{5.880635in}{2.097546in}}%
\pgfpathlineto{\pgfqpoint{5.909401in}{2.097546in}}%
\pgfpathlineto{\pgfqpoint{5.938166in}{2.097546in}}%
\pgfpathlineto{\pgfqpoint{5.966932in}{2.097546in}}%
\pgfpathlineto{\pgfqpoint{5.966932in}{1.978886in}}%
\pgfpathlineto{\pgfqpoint{5.966932in}{1.978886in}}%
\pgfpathlineto{\pgfqpoint{5.938166in}{1.978886in}}%
\pgfpathlineto{\pgfqpoint{5.909401in}{1.978886in}}%
\pgfpathlineto{\pgfqpoint{5.880635in}{1.978886in}}%
\pgfpathlineto{\pgfqpoint{5.851870in}{1.978886in}}%
\pgfpathlineto{\pgfqpoint{5.823104in}{1.978886in}}%
\pgfpathlineto{\pgfqpoint{5.794339in}{1.978886in}}%
\pgfpathlineto{\pgfqpoint{5.765574in}{1.978886in}}%
\pgfpathlineto{\pgfqpoint{5.736808in}{1.978886in}}%
\pgfpathlineto{\pgfqpoint{5.708043in}{1.978886in}}%
\pgfpathlineto{\pgfqpoint{5.679277in}{1.978886in}}%
\pgfpathlineto{\pgfqpoint{5.650512in}{1.978886in}}%
\pgfpathlineto{\pgfqpoint{5.621746in}{1.978886in}}%
\pgfpathlineto{\pgfqpoint{5.592981in}{1.978886in}}%
\pgfpathlineto{\pgfqpoint{5.564215in}{1.978886in}}%
\pgfpathlineto{\pgfqpoint{5.535450in}{1.978886in}}%
\pgfpathlineto{\pgfqpoint{5.506684in}{1.978886in}}%
\pgfpathlineto{\pgfqpoint{5.477919in}{1.978886in}}%
\pgfpathlineto{\pgfqpoint{5.449153in}{1.978886in}}%
\pgfpathlineto{\pgfqpoint{5.420388in}{1.978886in}}%
\pgfpathlineto{\pgfqpoint{5.391623in}{1.978886in}}%
\pgfpathlineto{\pgfqpoint{5.362857in}{1.978886in}}%
\pgfpathlineto{\pgfqpoint{5.334092in}{1.978886in}}%
\pgfpathlineto{\pgfqpoint{5.305326in}{1.978886in}}%
\pgfpathlineto{\pgfqpoint{5.276561in}{1.980880in}}%
\pgfpathlineto{\pgfqpoint{5.247795in}{1.980880in}}%
\pgfpathlineto{\pgfqpoint{5.219030in}{1.980880in}}%
\pgfpathlineto{\pgfqpoint{5.190264in}{2.002841in}}%
\pgfpathlineto{\pgfqpoint{5.161499in}{2.002841in}}%
\pgfpathlineto{\pgfqpoint{5.132733in}{2.002841in}}%
\pgfpathlineto{\pgfqpoint{5.103968in}{2.002841in}}%
\pgfpathlineto{\pgfqpoint{5.075202in}{2.002841in}}%
\pgfpathlineto{\pgfqpoint{5.046437in}{2.002841in}}%
\pgfpathlineto{\pgfqpoint{5.017671in}{2.002841in}}%
\pgfpathlineto{\pgfqpoint{4.988906in}{2.002841in}}%
\pgfpathlineto{\pgfqpoint{4.960141in}{2.002841in}}%
\pgfpathlineto{\pgfqpoint{4.931375in}{2.002841in}}%
\pgfpathlineto{\pgfqpoint{4.902610in}{2.002841in}}%
\pgfpathlineto{\pgfqpoint{4.873844in}{2.002841in}}%
\pgfpathlineto{\pgfqpoint{4.845079in}{2.002841in}}%
\pgfpathlineto{\pgfqpoint{4.816313in}{2.002841in}}%
\pgfpathlineto{\pgfqpoint{4.787548in}{2.002841in}}%
\pgfpathlineto{\pgfqpoint{4.758782in}{2.002841in}}%
\pgfpathlineto{\pgfqpoint{4.730017in}{2.002841in}}%
\pgfpathlineto{\pgfqpoint{4.701251in}{2.002841in}}%
\pgfpathlineto{\pgfqpoint{4.672486in}{2.002841in}}%
\pgfpathlineto{\pgfqpoint{4.643720in}{2.002841in}}%
\pgfpathlineto{\pgfqpoint{4.614955in}{2.002841in}}%
\pgfpathlineto{\pgfqpoint{4.586189in}{2.013646in}}%
\pgfpathlineto{\pgfqpoint{4.557424in}{2.013646in}}%
\pgfpathlineto{\pgfqpoint{4.528659in}{2.013646in}}%
\pgfpathlineto{\pgfqpoint{4.499893in}{2.013646in}}%
\pgfpathlineto{\pgfqpoint{4.471128in}{2.013646in}}%
\pgfpathlineto{\pgfqpoint{4.442362in}{2.013646in}}%
\pgfpathlineto{\pgfqpoint{4.413597in}{2.013646in}}%
\pgfpathlineto{\pgfqpoint{4.384831in}{2.013646in}}%
\pgfpathlineto{\pgfqpoint{4.356066in}{2.013646in}}%
\pgfpathlineto{\pgfqpoint{4.327300in}{2.013646in}}%
\pgfpathlineto{\pgfqpoint{4.298535in}{2.013646in}}%
\pgfpathlineto{\pgfqpoint{4.269769in}{2.013646in}}%
\pgfpathlineto{\pgfqpoint{4.241004in}{2.013646in}}%
\pgfpathlineto{\pgfqpoint{4.212238in}{2.013646in}}%
\pgfpathlineto{\pgfqpoint{4.183473in}{2.013646in}}%
\pgfpathlineto{\pgfqpoint{4.154708in}{2.013646in}}%
\pgfpathlineto{\pgfqpoint{4.125942in}{2.013646in}}%
\pgfpathlineto{\pgfqpoint{4.097177in}{2.013646in}}%
\pgfpathlineto{\pgfqpoint{4.068411in}{2.013646in}}%
\pgfpathlineto{\pgfqpoint{4.039646in}{2.013646in}}%
\pgfpathlineto{\pgfqpoint{4.010880in}{2.013646in}}%
\pgfpathlineto{\pgfqpoint{3.982115in}{2.013646in}}%
\pgfpathlineto{\pgfqpoint{3.953349in}{2.013646in}}%
\pgfpathlineto{\pgfqpoint{3.924584in}{2.013646in}}%
\pgfpathlineto{\pgfqpoint{3.895818in}{2.013646in}}%
\pgfpathlineto{\pgfqpoint{3.867053in}{2.013646in}}%
\pgfpathlineto{\pgfqpoint{3.838287in}{2.013646in}}%
\pgfpathlineto{\pgfqpoint{3.809522in}{2.013646in}}%
\pgfpathlineto{\pgfqpoint{3.780756in}{2.013646in}}%
\pgfpathlineto{\pgfqpoint{3.751991in}{2.013646in}}%
\pgfpathlineto{\pgfqpoint{3.723226in}{2.013646in}}%
\pgfpathlineto{\pgfqpoint{3.694460in}{2.013646in}}%
\pgfpathlineto{\pgfqpoint{3.665695in}{2.013646in}}%
\pgfpathlineto{\pgfqpoint{3.636929in}{2.013646in}}%
\pgfpathlineto{\pgfqpoint{3.608164in}{2.013646in}}%
\pgfpathlineto{\pgfqpoint{3.579398in}{2.013646in}}%
\pgfpathlineto{\pgfqpoint{3.550633in}{2.013646in}}%
\pgfpathlineto{\pgfqpoint{3.521867in}{2.013646in}}%
\pgfpathlineto{\pgfqpoint{3.493102in}{2.013646in}}%
\pgfpathlineto{\pgfqpoint{3.464336in}{2.013646in}}%
\pgfpathlineto{\pgfqpoint{3.435571in}{2.013209in}}%
\pgfpathlineto{\pgfqpoint{3.406805in}{2.013209in}}%
\pgfpathlineto{\pgfqpoint{3.378040in}{2.019093in}}%
\pgfpathlineto{\pgfqpoint{3.349274in}{2.019093in}}%
\pgfpathlineto{\pgfqpoint{3.320509in}{2.019093in}}%
\pgfpathlineto{\pgfqpoint{3.291744in}{2.019093in}}%
\pgfpathlineto{\pgfqpoint{3.262978in}{2.019093in}}%
\pgfpathlineto{\pgfqpoint{3.234213in}{2.019093in}}%
\pgfpathlineto{\pgfqpoint{3.205447in}{2.019093in}}%
\pgfpathlineto{\pgfqpoint{3.176682in}{2.019093in}}%
\pgfpathlineto{\pgfqpoint{3.147916in}{2.019093in}}%
\pgfpathlineto{\pgfqpoint{3.119151in}{2.019093in}}%
\pgfpathlineto{\pgfqpoint{3.090385in}{2.019093in}}%
\pgfpathlineto{\pgfqpoint{3.061620in}{2.019093in}}%
\pgfpathlineto{\pgfqpoint{3.032854in}{2.019093in}}%
\pgfpathlineto{\pgfqpoint{3.004089in}{2.019093in}}%
\pgfpathlineto{\pgfqpoint{2.975323in}{2.019093in}}%
\pgfpathlineto{\pgfqpoint{2.946558in}{2.019093in}}%
\pgfpathlineto{\pgfqpoint{2.917792in}{2.032349in}}%
\pgfpathlineto{\pgfqpoint{2.889027in}{2.032349in}}%
\pgfpathlineto{\pgfqpoint{2.860262in}{2.032349in}}%
\pgfpathlineto{\pgfqpoint{2.831496in}{2.032349in}}%
\pgfpathlineto{\pgfqpoint{2.802731in}{2.032349in}}%
\pgfpathlineto{\pgfqpoint{2.773965in}{2.032349in}}%
\pgfpathlineto{\pgfqpoint{2.745200in}{2.032349in}}%
\pgfpathlineto{\pgfqpoint{2.716434in}{2.032349in}}%
\pgfpathlineto{\pgfqpoint{2.687669in}{2.072143in}}%
\pgfpathlineto{\pgfqpoint{2.658903in}{2.072143in}}%
\pgfpathlineto{\pgfqpoint{2.630138in}{2.072143in}}%
\pgfpathlineto{\pgfqpoint{2.601372in}{2.072143in}}%
\pgfpathlineto{\pgfqpoint{2.572607in}{2.072143in}}%
\pgfpathlineto{\pgfqpoint{2.543841in}{2.072143in}}%
\pgfpathlineto{\pgfqpoint{2.515076in}{2.072143in}}%
\pgfpathlineto{\pgfqpoint{2.486311in}{2.072143in}}%
\pgfpathlineto{\pgfqpoint{2.457545in}{2.072143in}}%
\pgfpathlineto{\pgfqpoint{2.428780in}{2.092341in}}%
\pgfpathlineto{\pgfqpoint{2.400014in}{2.092341in}}%
\pgfpathlineto{\pgfqpoint{2.371249in}{2.092341in}}%
\pgfpathlineto{\pgfqpoint{2.342483in}{2.092341in}}%
\pgfpathlineto{\pgfqpoint{2.313718in}{2.092341in}}%
\pgfpathlineto{\pgfqpoint{2.284952in}{2.092341in}}%
\pgfpathlineto{\pgfqpoint{2.256187in}{2.092341in}}%
\pgfpathlineto{\pgfqpoint{2.227421in}{2.092341in}}%
\pgfpathlineto{\pgfqpoint{2.198656in}{2.092341in}}%
\pgfpathlineto{\pgfqpoint{2.169890in}{2.094152in}}%
\pgfpathlineto{\pgfqpoint{2.141125in}{2.094152in}}%
\pgfpathlineto{\pgfqpoint{2.112359in}{2.094152in}}%
\pgfpathlineto{\pgfqpoint{2.083594in}{2.094152in}}%
\pgfpathlineto{\pgfqpoint{2.054829in}{2.134139in}}%
\pgfpathlineto{\pgfqpoint{2.026063in}{2.134139in}}%
\pgfpathlineto{\pgfqpoint{1.997298in}{2.134139in}}%
\pgfpathlineto{\pgfqpoint{1.968532in}{2.134139in}}%
\pgfpathlineto{\pgfqpoint{1.939767in}{2.142789in}}%
\pgfpathlineto{\pgfqpoint{1.911001in}{2.146369in}}%
\pgfpathlineto{\pgfqpoint{1.882236in}{2.146369in}}%
\pgfpathlineto{\pgfqpoint{1.853470in}{2.146369in}}%
\pgfpathlineto{\pgfqpoint{1.824705in}{2.146369in}}%
\pgfpathlineto{\pgfqpoint{1.795939in}{2.146369in}}%
\pgfpathlineto{\pgfqpoint{1.767174in}{2.146369in}}%
\pgfpathlineto{\pgfqpoint{1.738408in}{2.146369in}}%
\pgfpathlineto{\pgfqpoint{1.709643in}{2.146369in}}%
\pgfpathlineto{\pgfqpoint{1.680877in}{2.165357in}}%
\pgfpathlineto{\pgfqpoint{1.652112in}{2.165357in}}%
\pgfpathlineto{\pgfqpoint{1.623347in}{2.165357in}}%
\pgfpathlineto{\pgfqpoint{1.594581in}{2.220177in}}%
\pgfpathlineto{\pgfqpoint{1.565816in}{2.220177in}}%
\pgfpathlineto{\pgfqpoint{1.537050in}{2.220177in}}%
\pgfpathlineto{\pgfqpoint{1.508285in}{2.226192in}}%
\pgfpathlineto{\pgfqpoint{1.479519in}{2.274959in}}%
\pgfpathlineto{\pgfqpoint{1.450754in}{2.277678in}}%
\pgfpathlineto{\pgfqpoint{1.421988in}{2.321177in}}%
\pgfpathlineto{\pgfqpoint{1.393223in}{2.341315in}}%
\pgfpathlineto{\pgfqpoint{1.364457in}{2.366697in}}%
\pgfpathlineto{\pgfqpoint{1.335692in}{2.366838in}}%
\pgfpathlineto{\pgfqpoint{1.306926in}{2.366838in}}%
\pgfpathlineto{\pgfqpoint{1.278161in}{2.377364in}}%
\pgfpathlineto{\pgfqpoint{1.249396in}{2.430550in}}%
\pgfpathlineto{\pgfqpoint{1.220630in}{2.440809in}}%
\pgfpathlineto{\pgfqpoint{1.191865in}{2.457453in}}%
\pgfpathlineto{\pgfqpoint{1.163099in}{2.457453in}}%
\pgfpathlineto{\pgfqpoint{1.134334in}{2.482565in}}%
\pgfpathlineto{\pgfqpoint{1.105568in}{2.482680in}}%
\pgfpathclose%
\pgfusepath{fill}%
\end{pgfscope}%
\begin{pgfscope}%
\pgfpathrectangle{\pgfqpoint{0.862500in}{0.375000in}}{\pgfqpoint{5.347500in}{2.265000in}}%
\pgfusepath{clip}%
\pgfsetbuttcap%
\pgfsetroundjoin%
\definecolor{currentfill}{rgb}{1.000000,0.498039,0.054902}%
\pgfsetfillcolor{currentfill}%
\pgfsetfillopacity{0.200000}%
\pgfsetlinewidth{0.000000pt}%
\definecolor{currentstroke}{rgb}{0.000000,0.000000,0.000000}%
\pgfsetstrokecolor{currentstroke}%
\pgfsetdash{}{0pt}%
\pgfpathmoveto{\pgfqpoint{1.105568in}{2.488000in}}%
\pgfpathlineto{\pgfqpoint{1.105568in}{2.518485in}}%
\pgfpathlineto{\pgfqpoint{1.134334in}{2.478756in}}%
\pgfpathlineto{\pgfqpoint{1.163099in}{2.473895in}}%
\pgfpathlineto{\pgfqpoint{1.191865in}{2.471364in}}%
\pgfpathlineto{\pgfqpoint{1.220630in}{2.471364in}}%
\pgfpathlineto{\pgfqpoint{1.249396in}{2.469698in}}%
\pgfpathlineto{\pgfqpoint{1.278161in}{2.452167in}}%
\pgfpathlineto{\pgfqpoint{1.306926in}{2.431714in}}%
\pgfpathlineto{\pgfqpoint{1.335692in}{2.431714in}}%
\pgfpathlineto{\pgfqpoint{1.364457in}{2.431714in}}%
\pgfpathlineto{\pgfqpoint{1.393223in}{2.431714in}}%
\pgfpathlineto{\pgfqpoint{1.421988in}{2.430680in}}%
\pgfpathlineto{\pgfqpoint{1.450754in}{2.423568in}}%
\pgfpathlineto{\pgfqpoint{1.479519in}{2.423568in}}%
\pgfpathlineto{\pgfqpoint{1.508285in}{2.406908in}}%
\pgfpathlineto{\pgfqpoint{1.537050in}{2.356988in}}%
\pgfpathlineto{\pgfqpoint{1.565816in}{2.356988in}}%
\pgfpathlineto{\pgfqpoint{1.594581in}{2.356988in}}%
\pgfpathlineto{\pgfqpoint{1.623347in}{2.356988in}}%
\pgfpathlineto{\pgfqpoint{1.652112in}{2.356988in}}%
\pgfpathlineto{\pgfqpoint{1.680877in}{2.356988in}}%
\pgfpathlineto{\pgfqpoint{1.709643in}{2.356988in}}%
\pgfpathlineto{\pgfqpoint{1.738408in}{2.356988in}}%
\pgfpathlineto{\pgfqpoint{1.767174in}{2.335743in}}%
\pgfpathlineto{\pgfqpoint{1.795939in}{2.269672in}}%
\pgfpathlineto{\pgfqpoint{1.824705in}{2.256664in}}%
\pgfpathlineto{\pgfqpoint{1.853470in}{2.233074in}}%
\pgfpathlineto{\pgfqpoint{1.882236in}{2.225561in}}%
\pgfpathlineto{\pgfqpoint{1.911001in}{2.199108in}}%
\pgfpathlineto{\pgfqpoint{1.939767in}{2.198213in}}%
\pgfpathlineto{\pgfqpoint{1.968532in}{2.193439in}}%
\pgfpathlineto{\pgfqpoint{1.997298in}{2.193439in}}%
\pgfpathlineto{\pgfqpoint{2.026063in}{2.146519in}}%
\pgfpathlineto{\pgfqpoint{2.054829in}{2.094501in}}%
\pgfpathlineto{\pgfqpoint{2.083594in}{2.066697in}}%
\pgfpathlineto{\pgfqpoint{2.112359in}{2.055546in}}%
\pgfpathlineto{\pgfqpoint{2.141125in}{2.006562in}}%
\pgfpathlineto{\pgfqpoint{2.169890in}{1.953639in}}%
\pgfpathlineto{\pgfqpoint{2.198656in}{1.874344in}}%
\pgfpathlineto{\pgfqpoint{2.227421in}{1.874344in}}%
\pgfpathlineto{\pgfqpoint{2.256187in}{1.873614in}}%
\pgfpathlineto{\pgfqpoint{2.284952in}{1.873614in}}%
\pgfpathlineto{\pgfqpoint{2.313718in}{1.873071in}}%
\pgfpathlineto{\pgfqpoint{2.342483in}{1.651146in}}%
\pgfpathlineto{\pgfqpoint{2.371249in}{1.651146in}}%
\pgfpathlineto{\pgfqpoint{2.400014in}{1.651146in}}%
\pgfpathlineto{\pgfqpoint{2.428780in}{1.628121in}}%
\pgfpathlineto{\pgfqpoint{2.457545in}{1.622838in}}%
\pgfpathlineto{\pgfqpoint{2.486311in}{1.622838in}}%
\pgfpathlineto{\pgfqpoint{2.515076in}{1.622838in}}%
\pgfpathlineto{\pgfqpoint{2.543841in}{1.554882in}}%
\pgfpathlineto{\pgfqpoint{2.572607in}{1.549648in}}%
\pgfpathlineto{\pgfqpoint{2.601372in}{1.549648in}}%
\pgfpathlineto{\pgfqpoint{2.630138in}{1.527649in}}%
\pgfpathlineto{\pgfqpoint{2.658903in}{1.527649in}}%
\pgfpathlineto{\pgfqpoint{2.687669in}{1.527649in}}%
\pgfpathlineto{\pgfqpoint{2.716434in}{1.514900in}}%
\pgfpathlineto{\pgfqpoint{2.745200in}{1.514900in}}%
\pgfpathlineto{\pgfqpoint{2.773965in}{1.514900in}}%
\pgfpathlineto{\pgfqpoint{2.802731in}{1.478061in}}%
\pgfpathlineto{\pgfqpoint{2.831496in}{1.478061in}}%
\pgfpathlineto{\pgfqpoint{2.860262in}{1.473141in}}%
\pgfpathlineto{\pgfqpoint{2.889027in}{1.468549in}}%
\pgfpathlineto{\pgfqpoint{2.917792in}{1.418884in}}%
\pgfpathlineto{\pgfqpoint{2.946558in}{1.418884in}}%
\pgfpathlineto{\pgfqpoint{2.975323in}{1.418884in}}%
\pgfpathlineto{\pgfqpoint{3.004089in}{1.418884in}}%
\pgfpathlineto{\pgfqpoint{3.032854in}{1.418369in}}%
\pgfpathlineto{\pgfqpoint{3.061620in}{1.365271in}}%
\pgfpathlineto{\pgfqpoint{3.090385in}{1.333327in}}%
\pgfpathlineto{\pgfqpoint{3.119151in}{1.333327in}}%
\pgfpathlineto{\pgfqpoint{3.147916in}{1.313717in}}%
\pgfpathlineto{\pgfqpoint{3.176682in}{1.313717in}}%
\pgfpathlineto{\pgfqpoint{3.205447in}{1.313717in}}%
\pgfpathlineto{\pgfqpoint{3.234213in}{1.313717in}}%
\pgfpathlineto{\pgfqpoint{3.262978in}{1.313717in}}%
\pgfpathlineto{\pgfqpoint{3.291744in}{1.313717in}}%
\pgfpathlineto{\pgfqpoint{3.320509in}{1.313717in}}%
\pgfpathlineto{\pgfqpoint{3.349274in}{1.313717in}}%
\pgfpathlineto{\pgfqpoint{3.378040in}{1.313717in}}%
\pgfpathlineto{\pgfqpoint{3.406805in}{1.303120in}}%
\pgfpathlineto{\pgfqpoint{3.435571in}{1.303120in}}%
\pgfpathlineto{\pgfqpoint{3.464336in}{1.300231in}}%
\pgfpathlineto{\pgfqpoint{3.493102in}{1.300231in}}%
\pgfpathlineto{\pgfqpoint{3.521867in}{1.299998in}}%
\pgfpathlineto{\pgfqpoint{3.550633in}{1.278637in}}%
\pgfpathlineto{\pgfqpoint{3.579398in}{1.273186in}}%
\pgfpathlineto{\pgfqpoint{3.608164in}{1.273186in}}%
\pgfpathlineto{\pgfqpoint{3.636929in}{1.273186in}}%
\pgfpathlineto{\pgfqpoint{3.665695in}{1.273186in}}%
\pgfpathlineto{\pgfqpoint{3.694460in}{1.273186in}}%
\pgfpathlineto{\pgfqpoint{3.723226in}{1.273186in}}%
\pgfpathlineto{\pgfqpoint{3.751991in}{1.273186in}}%
\pgfpathlineto{\pgfqpoint{3.780756in}{1.273186in}}%
\pgfpathlineto{\pgfqpoint{3.809522in}{1.273186in}}%
\pgfpathlineto{\pgfqpoint{3.838287in}{1.223853in}}%
\pgfpathlineto{\pgfqpoint{3.867053in}{1.223853in}}%
\pgfpathlineto{\pgfqpoint{3.895818in}{1.223853in}}%
\pgfpathlineto{\pgfqpoint{3.924584in}{1.223853in}}%
\pgfpathlineto{\pgfqpoint{3.953349in}{1.223853in}}%
\pgfpathlineto{\pgfqpoint{3.982115in}{1.223853in}}%
\pgfpathlineto{\pgfqpoint{4.010880in}{1.223853in}}%
\pgfpathlineto{\pgfqpoint{4.039646in}{1.223853in}}%
\pgfpathlineto{\pgfqpoint{4.068411in}{1.223853in}}%
\pgfpathlineto{\pgfqpoint{4.097177in}{1.223853in}}%
\pgfpathlineto{\pgfqpoint{4.125942in}{1.149371in}}%
\pgfpathlineto{\pgfqpoint{4.154708in}{1.147633in}}%
\pgfpathlineto{\pgfqpoint{4.183473in}{1.147633in}}%
\pgfpathlineto{\pgfqpoint{4.212238in}{1.147633in}}%
\pgfpathlineto{\pgfqpoint{4.241004in}{1.147633in}}%
\pgfpathlineto{\pgfqpoint{4.269769in}{1.147633in}}%
\pgfpathlineto{\pgfqpoint{4.298535in}{1.147633in}}%
\pgfpathlineto{\pgfqpoint{4.327300in}{1.147633in}}%
\pgfpathlineto{\pgfqpoint{4.356066in}{1.147633in}}%
\pgfpathlineto{\pgfqpoint{4.384831in}{1.147633in}}%
\pgfpathlineto{\pgfqpoint{4.413597in}{1.147633in}}%
\pgfpathlineto{\pgfqpoint{4.442362in}{1.147633in}}%
\pgfpathlineto{\pgfqpoint{4.471128in}{1.147633in}}%
\pgfpathlineto{\pgfqpoint{4.499893in}{1.147633in}}%
\pgfpathlineto{\pgfqpoint{4.528659in}{1.147633in}}%
\pgfpathlineto{\pgfqpoint{4.557424in}{1.147633in}}%
\pgfpathlineto{\pgfqpoint{4.586189in}{1.147633in}}%
\pgfpathlineto{\pgfqpoint{4.614955in}{1.147633in}}%
\pgfpathlineto{\pgfqpoint{4.643720in}{1.142960in}}%
\pgfpathlineto{\pgfqpoint{4.672486in}{1.142960in}}%
\pgfpathlineto{\pgfqpoint{4.701251in}{1.142960in}}%
\pgfpathlineto{\pgfqpoint{4.730017in}{1.142960in}}%
\pgfpathlineto{\pgfqpoint{4.758782in}{1.142960in}}%
\pgfpathlineto{\pgfqpoint{4.787548in}{1.142960in}}%
\pgfpathlineto{\pgfqpoint{4.816313in}{1.142960in}}%
\pgfpathlineto{\pgfqpoint{4.845079in}{1.142960in}}%
\pgfpathlineto{\pgfqpoint{4.873844in}{1.142960in}}%
\pgfpathlineto{\pgfqpoint{4.902610in}{1.142960in}}%
\pgfpathlineto{\pgfqpoint{4.931375in}{1.142960in}}%
\pgfpathlineto{\pgfqpoint{4.960141in}{1.142960in}}%
\pgfpathlineto{\pgfqpoint{4.988906in}{1.142960in}}%
\pgfpathlineto{\pgfqpoint{5.017671in}{1.142960in}}%
\pgfpathlineto{\pgfqpoint{5.046437in}{1.142960in}}%
\pgfpathlineto{\pgfqpoint{5.075202in}{1.142960in}}%
\pgfpathlineto{\pgfqpoint{5.103968in}{1.142960in}}%
\pgfpathlineto{\pgfqpoint{5.132733in}{1.142960in}}%
\pgfpathlineto{\pgfqpoint{5.161499in}{1.142960in}}%
\pgfpathlineto{\pgfqpoint{5.190264in}{1.142960in}}%
\pgfpathlineto{\pgfqpoint{5.219030in}{1.142960in}}%
\pgfpathlineto{\pgfqpoint{5.247795in}{1.142960in}}%
\pgfpathlineto{\pgfqpoint{5.276561in}{1.142960in}}%
\pgfpathlineto{\pgfqpoint{5.305326in}{1.142960in}}%
\pgfpathlineto{\pgfqpoint{5.334092in}{1.142960in}}%
\pgfpathlineto{\pgfqpoint{5.362857in}{1.142960in}}%
\pgfpathlineto{\pgfqpoint{5.391623in}{1.142960in}}%
\pgfpathlineto{\pgfqpoint{5.420388in}{1.142960in}}%
\pgfpathlineto{\pgfqpoint{5.449153in}{1.142960in}}%
\pgfpathlineto{\pgfqpoint{5.477919in}{1.142960in}}%
\pgfpathlineto{\pgfqpoint{5.506684in}{1.142960in}}%
\pgfpathlineto{\pgfqpoint{5.535450in}{1.142960in}}%
\pgfpathlineto{\pgfqpoint{5.564215in}{1.142960in}}%
\pgfpathlineto{\pgfqpoint{5.592981in}{1.142960in}}%
\pgfpathlineto{\pgfqpoint{5.621746in}{1.142960in}}%
\pgfpathlineto{\pgfqpoint{5.650512in}{1.142960in}}%
\pgfpathlineto{\pgfqpoint{5.679277in}{1.142960in}}%
\pgfpathlineto{\pgfqpoint{5.708043in}{1.142960in}}%
\pgfpathlineto{\pgfqpoint{5.736808in}{1.142960in}}%
\pgfpathlineto{\pgfqpoint{5.765574in}{1.111026in}}%
\pgfpathlineto{\pgfqpoint{5.794339in}{1.111026in}}%
\pgfpathlineto{\pgfqpoint{5.823104in}{1.111026in}}%
\pgfpathlineto{\pgfqpoint{5.851870in}{1.111026in}}%
\pgfpathlineto{\pgfqpoint{5.880635in}{1.111026in}}%
\pgfpathlineto{\pgfqpoint{5.909401in}{1.111026in}}%
\pgfpathlineto{\pgfqpoint{5.938166in}{1.111026in}}%
\pgfpathlineto{\pgfqpoint{5.966932in}{1.111026in}}%
\pgfpathlineto{\pgfqpoint{5.966932in}{0.827135in}}%
\pgfpathlineto{\pgfqpoint{5.966932in}{0.827135in}}%
\pgfpathlineto{\pgfqpoint{5.938166in}{0.827135in}}%
\pgfpathlineto{\pgfqpoint{5.909401in}{0.827135in}}%
\pgfpathlineto{\pgfqpoint{5.880635in}{0.827135in}}%
\pgfpathlineto{\pgfqpoint{5.851870in}{0.827135in}}%
\pgfpathlineto{\pgfqpoint{5.823104in}{0.827135in}}%
\pgfpathlineto{\pgfqpoint{5.794339in}{0.827135in}}%
\pgfpathlineto{\pgfqpoint{5.765574in}{0.827135in}}%
\pgfpathlineto{\pgfqpoint{5.736808in}{0.878862in}}%
\pgfpathlineto{\pgfqpoint{5.708043in}{0.878862in}}%
\pgfpathlineto{\pgfqpoint{5.679277in}{0.878862in}}%
\pgfpathlineto{\pgfqpoint{5.650512in}{0.878862in}}%
\pgfpathlineto{\pgfqpoint{5.621746in}{0.878862in}}%
\pgfpathlineto{\pgfqpoint{5.592981in}{0.878862in}}%
\pgfpathlineto{\pgfqpoint{5.564215in}{0.878862in}}%
\pgfpathlineto{\pgfqpoint{5.535450in}{0.878862in}}%
\pgfpathlineto{\pgfqpoint{5.506684in}{0.878862in}}%
\pgfpathlineto{\pgfqpoint{5.477919in}{0.878862in}}%
\pgfpathlineto{\pgfqpoint{5.449153in}{0.878862in}}%
\pgfpathlineto{\pgfqpoint{5.420388in}{0.878862in}}%
\pgfpathlineto{\pgfqpoint{5.391623in}{0.878862in}}%
\pgfpathlineto{\pgfqpoint{5.362857in}{0.878862in}}%
\pgfpathlineto{\pgfqpoint{5.334092in}{0.878862in}}%
\pgfpathlineto{\pgfqpoint{5.305326in}{0.878862in}}%
\pgfpathlineto{\pgfqpoint{5.276561in}{0.878862in}}%
\pgfpathlineto{\pgfqpoint{5.247795in}{0.878862in}}%
\pgfpathlineto{\pgfqpoint{5.219030in}{0.878862in}}%
\pgfpathlineto{\pgfqpoint{5.190264in}{0.878862in}}%
\pgfpathlineto{\pgfqpoint{5.161499in}{0.878862in}}%
\pgfpathlineto{\pgfqpoint{5.132733in}{0.878862in}}%
\pgfpathlineto{\pgfqpoint{5.103968in}{0.878862in}}%
\pgfpathlineto{\pgfqpoint{5.075202in}{0.878862in}}%
\pgfpathlineto{\pgfqpoint{5.046437in}{0.878862in}}%
\pgfpathlineto{\pgfqpoint{5.017671in}{0.878862in}}%
\pgfpathlineto{\pgfqpoint{4.988906in}{0.878862in}}%
\pgfpathlineto{\pgfqpoint{4.960141in}{0.878862in}}%
\pgfpathlineto{\pgfqpoint{4.931375in}{0.878862in}}%
\pgfpathlineto{\pgfqpoint{4.902610in}{0.878862in}}%
\pgfpathlineto{\pgfqpoint{4.873844in}{0.878862in}}%
\pgfpathlineto{\pgfqpoint{4.845079in}{0.878862in}}%
\pgfpathlineto{\pgfqpoint{4.816313in}{0.878862in}}%
\pgfpathlineto{\pgfqpoint{4.787548in}{0.878862in}}%
\pgfpathlineto{\pgfqpoint{4.758782in}{0.878862in}}%
\pgfpathlineto{\pgfqpoint{4.730017in}{0.878862in}}%
\pgfpathlineto{\pgfqpoint{4.701251in}{0.878862in}}%
\pgfpathlineto{\pgfqpoint{4.672486in}{0.878862in}}%
\pgfpathlineto{\pgfqpoint{4.643720in}{0.878862in}}%
\pgfpathlineto{\pgfqpoint{4.614955in}{0.913508in}}%
\pgfpathlineto{\pgfqpoint{4.586189in}{0.913508in}}%
\pgfpathlineto{\pgfqpoint{4.557424in}{0.913508in}}%
\pgfpathlineto{\pgfqpoint{4.528659in}{0.913508in}}%
\pgfpathlineto{\pgfqpoint{4.499893in}{0.913508in}}%
\pgfpathlineto{\pgfqpoint{4.471128in}{0.913508in}}%
\pgfpathlineto{\pgfqpoint{4.442362in}{0.913508in}}%
\pgfpathlineto{\pgfqpoint{4.413597in}{0.913508in}}%
\pgfpathlineto{\pgfqpoint{4.384831in}{0.913508in}}%
\pgfpathlineto{\pgfqpoint{4.356066in}{0.913508in}}%
\pgfpathlineto{\pgfqpoint{4.327300in}{0.913508in}}%
\pgfpathlineto{\pgfqpoint{4.298535in}{0.913508in}}%
\pgfpathlineto{\pgfqpoint{4.269769in}{0.913508in}}%
\pgfpathlineto{\pgfqpoint{4.241004in}{0.913508in}}%
\pgfpathlineto{\pgfqpoint{4.212238in}{0.913508in}}%
\pgfpathlineto{\pgfqpoint{4.183473in}{0.913508in}}%
\pgfpathlineto{\pgfqpoint{4.154708in}{0.913508in}}%
\pgfpathlineto{\pgfqpoint{4.125942in}{0.915514in}}%
\pgfpathlineto{\pgfqpoint{4.097177in}{0.940927in}}%
\pgfpathlineto{\pgfqpoint{4.068411in}{0.940927in}}%
\pgfpathlineto{\pgfqpoint{4.039646in}{0.940927in}}%
\pgfpathlineto{\pgfqpoint{4.010880in}{0.940927in}}%
\pgfpathlineto{\pgfqpoint{3.982115in}{0.940927in}}%
\pgfpathlineto{\pgfqpoint{3.953349in}{0.940927in}}%
\pgfpathlineto{\pgfqpoint{3.924584in}{0.940927in}}%
\pgfpathlineto{\pgfqpoint{3.895818in}{0.940927in}}%
\pgfpathlineto{\pgfqpoint{3.867053in}{0.940927in}}%
\pgfpathlineto{\pgfqpoint{3.838287in}{0.940927in}}%
\pgfpathlineto{\pgfqpoint{3.809522in}{0.999556in}}%
\pgfpathlineto{\pgfqpoint{3.780756in}{0.999556in}}%
\pgfpathlineto{\pgfqpoint{3.751991in}{0.999556in}}%
\pgfpathlineto{\pgfqpoint{3.723226in}{0.999556in}}%
\pgfpathlineto{\pgfqpoint{3.694460in}{0.999556in}}%
\pgfpathlineto{\pgfqpoint{3.665695in}{0.999556in}}%
\pgfpathlineto{\pgfqpoint{3.636929in}{0.999556in}}%
\pgfpathlineto{\pgfqpoint{3.608164in}{0.999556in}}%
\pgfpathlineto{\pgfqpoint{3.579398in}{0.999556in}}%
\pgfpathlineto{\pgfqpoint{3.550633in}{1.001985in}}%
\pgfpathlineto{\pgfqpoint{3.521867in}{1.089802in}}%
\pgfpathlineto{\pgfqpoint{3.493102in}{1.092367in}}%
\pgfpathlineto{\pgfqpoint{3.464336in}{1.092367in}}%
\pgfpathlineto{\pgfqpoint{3.435571in}{1.119313in}}%
\pgfpathlineto{\pgfqpoint{3.406805in}{1.119313in}}%
\pgfpathlineto{\pgfqpoint{3.378040in}{1.175446in}}%
\pgfpathlineto{\pgfqpoint{3.349274in}{1.175446in}}%
\pgfpathlineto{\pgfqpoint{3.320509in}{1.175446in}}%
\pgfpathlineto{\pgfqpoint{3.291744in}{1.175446in}}%
\pgfpathlineto{\pgfqpoint{3.262978in}{1.175446in}}%
\pgfpathlineto{\pgfqpoint{3.234213in}{1.175446in}}%
\pgfpathlineto{\pgfqpoint{3.205447in}{1.175446in}}%
\pgfpathlineto{\pgfqpoint{3.176682in}{1.175446in}}%
\pgfpathlineto{\pgfqpoint{3.147916in}{1.175446in}}%
\pgfpathlineto{\pgfqpoint{3.119151in}{1.205636in}}%
\pgfpathlineto{\pgfqpoint{3.090385in}{1.205636in}}%
\pgfpathlineto{\pgfqpoint{3.061620in}{1.224276in}}%
\pgfpathlineto{\pgfqpoint{3.032854in}{1.264211in}}%
\pgfpathlineto{\pgfqpoint{3.004089in}{1.265150in}}%
\pgfpathlineto{\pgfqpoint{2.975323in}{1.265150in}}%
\pgfpathlineto{\pgfqpoint{2.946558in}{1.265150in}}%
\pgfpathlineto{\pgfqpoint{2.917792in}{1.265150in}}%
\pgfpathlineto{\pgfqpoint{2.889027in}{1.274391in}}%
\pgfpathlineto{\pgfqpoint{2.860262in}{1.280713in}}%
\pgfpathlineto{\pgfqpoint{2.831496in}{1.290842in}}%
\pgfpathlineto{\pgfqpoint{2.802731in}{1.290842in}}%
\pgfpathlineto{\pgfqpoint{2.773965in}{1.348080in}}%
\pgfpathlineto{\pgfqpoint{2.745200in}{1.348080in}}%
\pgfpathlineto{\pgfqpoint{2.716434in}{1.348080in}}%
\pgfpathlineto{\pgfqpoint{2.687669in}{1.376804in}}%
\pgfpathlineto{\pgfqpoint{2.658903in}{1.376804in}}%
\pgfpathlineto{\pgfqpoint{2.630138in}{1.376804in}}%
\pgfpathlineto{\pgfqpoint{2.601372in}{1.388077in}}%
\pgfpathlineto{\pgfqpoint{2.572607in}{1.388077in}}%
\pgfpathlineto{\pgfqpoint{2.543841in}{1.398175in}}%
\pgfpathlineto{\pgfqpoint{2.515076in}{1.409802in}}%
\pgfpathlineto{\pgfqpoint{2.486311in}{1.409802in}}%
\pgfpathlineto{\pgfqpoint{2.457545in}{1.409802in}}%
\pgfpathlineto{\pgfqpoint{2.428780in}{1.410323in}}%
\pgfpathlineto{\pgfqpoint{2.400014in}{1.450746in}}%
\pgfpathlineto{\pgfqpoint{2.371249in}{1.450746in}}%
\pgfpathlineto{\pgfqpoint{2.342483in}{1.450746in}}%
\pgfpathlineto{\pgfqpoint{2.313718in}{1.491352in}}%
\pgfpathlineto{\pgfqpoint{2.284952in}{1.491426in}}%
\pgfpathlineto{\pgfqpoint{2.256187in}{1.491426in}}%
\pgfpathlineto{\pgfqpoint{2.227421in}{1.493719in}}%
\pgfpathlineto{\pgfqpoint{2.198656in}{1.493719in}}%
\pgfpathlineto{\pgfqpoint{2.169890in}{1.607468in}}%
\pgfpathlineto{\pgfqpoint{2.141125in}{1.756479in}}%
\pgfpathlineto{\pgfqpoint{2.112359in}{1.828120in}}%
\pgfpathlineto{\pgfqpoint{2.083594in}{1.842791in}}%
\pgfpathlineto{\pgfqpoint{2.054829in}{1.886039in}}%
\pgfpathlineto{\pgfqpoint{2.026063in}{1.925914in}}%
\pgfpathlineto{\pgfqpoint{1.997298in}{1.978014in}}%
\pgfpathlineto{\pgfqpoint{1.968532in}{1.978014in}}%
\pgfpathlineto{\pgfqpoint{1.939767in}{1.991130in}}%
\pgfpathlineto{\pgfqpoint{1.911001in}{1.996693in}}%
\pgfpathlineto{\pgfqpoint{1.882236in}{2.035842in}}%
\pgfpathlineto{\pgfqpoint{1.853470in}{2.043465in}}%
\pgfpathlineto{\pgfqpoint{1.824705in}{2.130685in}}%
\pgfpathlineto{\pgfqpoint{1.795939in}{2.139094in}}%
\pgfpathlineto{\pgfqpoint{1.767174in}{2.287878in}}%
\pgfpathlineto{\pgfqpoint{1.738408in}{2.312004in}}%
\pgfpathlineto{\pgfqpoint{1.709643in}{2.312004in}}%
\pgfpathlineto{\pgfqpoint{1.680877in}{2.312004in}}%
\pgfpathlineto{\pgfqpoint{1.652112in}{2.312004in}}%
\pgfpathlineto{\pgfqpoint{1.623347in}{2.312004in}}%
\pgfpathlineto{\pgfqpoint{1.594581in}{2.312004in}}%
\pgfpathlineto{\pgfqpoint{1.565816in}{2.312004in}}%
\pgfpathlineto{\pgfqpoint{1.537050in}{2.312004in}}%
\pgfpathlineto{\pgfqpoint{1.508285in}{2.344033in}}%
\pgfpathlineto{\pgfqpoint{1.479519in}{2.351767in}}%
\pgfpathlineto{\pgfqpoint{1.450754in}{2.351767in}}%
\pgfpathlineto{\pgfqpoint{1.421988in}{2.356709in}}%
\pgfpathlineto{\pgfqpoint{1.393223in}{2.364210in}}%
\pgfpathlineto{\pgfqpoint{1.364457in}{2.364210in}}%
\pgfpathlineto{\pgfqpoint{1.335692in}{2.364210in}}%
\pgfpathlineto{\pgfqpoint{1.306926in}{2.364210in}}%
\pgfpathlineto{\pgfqpoint{1.278161in}{2.405833in}}%
\pgfpathlineto{\pgfqpoint{1.249396in}{2.413570in}}%
\pgfpathlineto{\pgfqpoint{1.220630in}{2.418676in}}%
\pgfpathlineto{\pgfqpoint{1.191865in}{2.418676in}}%
\pgfpathlineto{\pgfqpoint{1.163099in}{2.419381in}}%
\pgfpathlineto{\pgfqpoint{1.134334in}{2.429445in}}%
\pgfpathlineto{\pgfqpoint{1.105568in}{2.488000in}}%
\pgfpathclose%
\pgfusepath{fill}%
\end{pgfscope}%
\begin{pgfscope}%
\pgfpathrectangle{\pgfqpoint{0.862500in}{0.375000in}}{\pgfqpoint{5.347500in}{2.265000in}}%
\pgfusepath{clip}%
\pgfsetbuttcap%
\pgfsetroundjoin%
\definecolor{currentfill}{rgb}{0.172549,0.627451,0.172549}%
\pgfsetfillcolor{currentfill}%
\pgfsetfillopacity{0.200000}%
\pgfsetlinewidth{0.000000pt}%
\definecolor{currentstroke}{rgb}{0.000000,0.000000,0.000000}%
\pgfsetstrokecolor{currentstroke}%
\pgfsetdash{}{0pt}%
\pgfpathmoveto{\pgfqpoint{1.105568in}{2.517446in}}%
\pgfpathlineto{\pgfqpoint{1.105568in}{2.532462in}}%
\pgfpathlineto{\pgfqpoint{1.134334in}{2.532462in}}%
\pgfpathlineto{\pgfqpoint{1.163099in}{2.508823in}}%
\pgfpathlineto{\pgfqpoint{1.191865in}{2.502501in}}%
\pgfpathlineto{\pgfqpoint{1.220630in}{2.460658in}}%
\pgfpathlineto{\pgfqpoint{1.249396in}{2.425670in}}%
\pgfpathlineto{\pgfqpoint{1.278161in}{2.425670in}}%
\pgfpathlineto{\pgfqpoint{1.306926in}{2.425670in}}%
\pgfpathlineto{\pgfqpoint{1.335692in}{2.420938in}}%
\pgfpathlineto{\pgfqpoint{1.364457in}{2.420938in}}%
\pgfpathlineto{\pgfqpoint{1.393223in}{2.420938in}}%
\pgfpathlineto{\pgfqpoint{1.421988in}{2.420938in}}%
\pgfpathlineto{\pgfqpoint{1.450754in}{2.420938in}}%
\pgfpathlineto{\pgfqpoint{1.479519in}{2.410770in}}%
\pgfpathlineto{\pgfqpoint{1.508285in}{2.379164in}}%
\pgfpathlineto{\pgfqpoint{1.537050in}{2.379164in}}%
\pgfpathlineto{\pgfqpoint{1.565816in}{2.314712in}}%
\pgfpathlineto{\pgfqpoint{1.594581in}{2.312082in}}%
\pgfpathlineto{\pgfqpoint{1.623347in}{2.312082in}}%
\pgfpathlineto{\pgfqpoint{1.652112in}{2.312082in}}%
\pgfpathlineto{\pgfqpoint{1.680877in}{2.312082in}}%
\pgfpathlineto{\pgfqpoint{1.709643in}{2.312082in}}%
\pgfpathlineto{\pgfqpoint{1.738408in}{2.307858in}}%
\pgfpathlineto{\pgfqpoint{1.767174in}{2.307858in}}%
\pgfpathlineto{\pgfqpoint{1.795939in}{2.307858in}}%
\pgfpathlineto{\pgfqpoint{1.824705in}{2.307858in}}%
\pgfpathlineto{\pgfqpoint{1.853470in}{2.241132in}}%
\pgfpathlineto{\pgfqpoint{1.882236in}{2.240947in}}%
\pgfpathlineto{\pgfqpoint{1.911001in}{2.240947in}}%
\pgfpathlineto{\pgfqpoint{1.939767in}{2.240947in}}%
\pgfpathlineto{\pgfqpoint{1.968532in}{2.192645in}}%
\pgfpathlineto{\pgfqpoint{1.997298in}{2.192645in}}%
\pgfpathlineto{\pgfqpoint{2.026063in}{2.192645in}}%
\pgfpathlineto{\pgfqpoint{2.054829in}{2.189559in}}%
\pgfpathlineto{\pgfqpoint{2.083594in}{2.158859in}}%
\pgfpathlineto{\pgfqpoint{2.112359in}{2.094560in}}%
\pgfpathlineto{\pgfqpoint{2.141125in}{2.078112in}}%
\pgfpathlineto{\pgfqpoint{2.169890in}{2.025109in}}%
\pgfpathlineto{\pgfqpoint{2.198656in}{2.025109in}}%
\pgfpathlineto{\pgfqpoint{2.227421in}{1.954045in}}%
\pgfpathlineto{\pgfqpoint{2.256187in}{1.954045in}}%
\pgfpathlineto{\pgfqpoint{2.284952in}{1.922075in}}%
\pgfpathlineto{\pgfqpoint{2.313718in}{1.922075in}}%
\pgfpathlineto{\pgfqpoint{2.342483in}{1.903636in}}%
\pgfpathlineto{\pgfqpoint{2.371249in}{1.844663in}}%
\pgfpathlineto{\pgfqpoint{2.400014in}{1.760449in}}%
\pgfpathlineto{\pgfqpoint{2.428780in}{1.760449in}}%
\pgfpathlineto{\pgfqpoint{2.457545in}{1.757145in}}%
\pgfpathlineto{\pgfqpoint{2.486311in}{1.757145in}}%
\pgfpathlineto{\pgfqpoint{2.515076in}{1.757145in}}%
\pgfpathlineto{\pgfqpoint{2.543841in}{1.754936in}}%
\pgfpathlineto{\pgfqpoint{2.572607in}{1.754936in}}%
\pgfpathlineto{\pgfqpoint{2.601372in}{1.724421in}}%
\pgfpathlineto{\pgfqpoint{2.630138in}{1.720524in}}%
\pgfpathlineto{\pgfqpoint{2.658903in}{1.718097in}}%
\pgfpathlineto{\pgfqpoint{2.687669in}{1.718097in}}%
\pgfpathlineto{\pgfqpoint{2.716434in}{1.709389in}}%
\pgfpathlineto{\pgfqpoint{2.745200in}{1.622425in}}%
\pgfpathlineto{\pgfqpoint{2.773965in}{1.560680in}}%
\pgfpathlineto{\pgfqpoint{2.802731in}{1.560680in}}%
\pgfpathlineto{\pgfqpoint{2.831496in}{1.557614in}}%
\pgfpathlineto{\pgfqpoint{2.860262in}{1.502039in}}%
\pgfpathlineto{\pgfqpoint{2.889027in}{1.415946in}}%
\pgfpathlineto{\pgfqpoint{2.917792in}{1.411418in}}%
\pgfpathlineto{\pgfqpoint{2.946558in}{1.411418in}}%
\pgfpathlineto{\pgfqpoint{2.975323in}{1.372008in}}%
\pgfpathlineto{\pgfqpoint{3.004089in}{1.372008in}}%
\pgfpathlineto{\pgfqpoint{3.032854in}{1.372008in}}%
\pgfpathlineto{\pgfqpoint{3.061620in}{1.338023in}}%
\pgfpathlineto{\pgfqpoint{3.090385in}{1.331418in}}%
\pgfpathlineto{\pgfqpoint{3.119151in}{1.331418in}}%
\pgfpathlineto{\pgfqpoint{3.147916in}{1.309813in}}%
\pgfpathlineto{\pgfqpoint{3.176682in}{1.309813in}}%
\pgfpathlineto{\pgfqpoint{3.205447in}{1.309813in}}%
\pgfpathlineto{\pgfqpoint{3.234213in}{1.309813in}}%
\pgfpathlineto{\pgfqpoint{3.262978in}{1.281704in}}%
\pgfpathlineto{\pgfqpoint{3.291744in}{1.281704in}}%
\pgfpathlineto{\pgfqpoint{3.320509in}{1.281704in}}%
\pgfpathlineto{\pgfqpoint{3.349274in}{1.281704in}}%
\pgfpathlineto{\pgfqpoint{3.378040in}{1.272875in}}%
\pgfpathlineto{\pgfqpoint{3.406805in}{1.268676in}}%
\pgfpathlineto{\pgfqpoint{3.435571in}{1.268676in}}%
\pgfpathlineto{\pgfqpoint{3.464336in}{1.267321in}}%
\pgfpathlineto{\pgfqpoint{3.493102in}{1.267321in}}%
\pgfpathlineto{\pgfqpoint{3.521867in}{1.210708in}}%
\pgfpathlineto{\pgfqpoint{3.550633in}{1.210708in}}%
\pgfpathlineto{\pgfqpoint{3.579398in}{1.210708in}}%
\pgfpathlineto{\pgfqpoint{3.608164in}{1.210708in}}%
\pgfpathlineto{\pgfqpoint{3.636929in}{1.210708in}}%
\pgfpathlineto{\pgfqpoint{3.665695in}{1.195285in}}%
\pgfpathlineto{\pgfqpoint{3.694460in}{1.195285in}}%
\pgfpathlineto{\pgfqpoint{3.723226in}{1.190481in}}%
\pgfpathlineto{\pgfqpoint{3.751991in}{1.167513in}}%
\pgfpathlineto{\pgfqpoint{3.780756in}{1.156597in}}%
\pgfpathlineto{\pgfqpoint{3.809522in}{1.156597in}}%
\pgfpathlineto{\pgfqpoint{3.838287in}{1.156597in}}%
\pgfpathlineto{\pgfqpoint{3.867053in}{1.156597in}}%
\pgfpathlineto{\pgfqpoint{3.895818in}{1.151017in}}%
\pgfpathlineto{\pgfqpoint{3.924584in}{1.151017in}}%
\pgfpathlineto{\pgfqpoint{3.953349in}{1.151017in}}%
\pgfpathlineto{\pgfqpoint{3.982115in}{1.151017in}}%
\pgfpathlineto{\pgfqpoint{4.010880in}{1.151017in}}%
\pgfpathlineto{\pgfqpoint{4.039646in}{1.151017in}}%
\pgfpathlineto{\pgfqpoint{4.068411in}{1.151017in}}%
\pgfpathlineto{\pgfqpoint{4.097177in}{1.151017in}}%
\pgfpathlineto{\pgfqpoint{4.125942in}{1.149120in}}%
\pgfpathlineto{\pgfqpoint{4.154708in}{1.132306in}}%
\pgfpathlineto{\pgfqpoint{4.183473in}{1.132306in}}%
\pgfpathlineto{\pgfqpoint{4.212238in}{1.131743in}}%
\pgfpathlineto{\pgfqpoint{4.241004in}{1.131743in}}%
\pgfpathlineto{\pgfqpoint{4.269769in}{1.131743in}}%
\pgfpathlineto{\pgfqpoint{4.298535in}{1.131743in}}%
\pgfpathlineto{\pgfqpoint{4.327300in}{1.131743in}}%
\pgfpathlineto{\pgfqpoint{4.356066in}{1.131743in}}%
\pgfpathlineto{\pgfqpoint{4.384831in}{1.090809in}}%
\pgfpathlineto{\pgfqpoint{4.413597in}{1.090809in}}%
\pgfpathlineto{\pgfqpoint{4.442362in}{1.090809in}}%
\pgfpathlineto{\pgfqpoint{4.471128in}{1.088169in}}%
\pgfpathlineto{\pgfqpoint{4.499893in}{1.088169in}}%
\pgfpathlineto{\pgfqpoint{4.528659in}{1.079845in}}%
\pgfpathlineto{\pgfqpoint{4.557424in}{1.079845in}}%
\pgfpathlineto{\pgfqpoint{4.586189in}{1.079845in}}%
\pgfpathlineto{\pgfqpoint{4.614955in}{1.079845in}}%
\pgfpathlineto{\pgfqpoint{4.643720in}{1.079845in}}%
\pgfpathlineto{\pgfqpoint{4.672486in}{1.079845in}}%
\pgfpathlineto{\pgfqpoint{4.701251in}{1.079845in}}%
\pgfpathlineto{\pgfqpoint{4.730017in}{1.079845in}}%
\pgfpathlineto{\pgfqpoint{4.758782in}{1.079845in}}%
\pgfpathlineto{\pgfqpoint{4.787548in}{1.078413in}}%
\pgfpathlineto{\pgfqpoint{4.816313in}{1.078413in}}%
\pgfpathlineto{\pgfqpoint{4.845079in}{1.078413in}}%
\pgfpathlineto{\pgfqpoint{4.873844in}{1.078413in}}%
\pgfpathlineto{\pgfqpoint{4.902610in}{1.078413in}}%
\pgfpathlineto{\pgfqpoint{4.931375in}{1.078413in}}%
\pgfpathlineto{\pgfqpoint{4.960141in}{1.078413in}}%
\pgfpathlineto{\pgfqpoint{4.988906in}{1.078406in}}%
\pgfpathlineto{\pgfqpoint{5.017671in}{1.078406in}}%
\pgfpathlineto{\pgfqpoint{5.046437in}{1.078406in}}%
\pgfpathlineto{\pgfqpoint{5.075202in}{1.066992in}}%
\pgfpathlineto{\pgfqpoint{5.103968in}{1.066992in}}%
\pgfpathlineto{\pgfqpoint{5.132733in}{1.066992in}}%
\pgfpathlineto{\pgfqpoint{5.161499in}{1.066992in}}%
\pgfpathlineto{\pgfqpoint{5.190264in}{1.066992in}}%
\pgfpathlineto{\pgfqpoint{5.219030in}{1.066992in}}%
\pgfpathlineto{\pgfqpoint{5.247795in}{1.066992in}}%
\pgfpathlineto{\pgfqpoint{5.276561in}{1.066992in}}%
\pgfpathlineto{\pgfqpoint{5.305326in}{1.044576in}}%
\pgfpathlineto{\pgfqpoint{5.334092in}{1.044576in}}%
\pgfpathlineto{\pgfqpoint{5.362857in}{1.038206in}}%
\pgfpathlineto{\pgfqpoint{5.391623in}{1.038206in}}%
\pgfpathlineto{\pgfqpoint{5.420388in}{1.038206in}}%
\pgfpathlineto{\pgfqpoint{5.449153in}{1.038206in}}%
\pgfpathlineto{\pgfqpoint{5.477919in}{1.038001in}}%
\pgfpathlineto{\pgfqpoint{5.506684in}{1.038001in}}%
\pgfpathlineto{\pgfqpoint{5.535450in}{1.032849in}}%
\pgfpathlineto{\pgfqpoint{5.564215in}{1.032849in}}%
\pgfpathlineto{\pgfqpoint{5.592981in}{1.032849in}}%
\pgfpathlineto{\pgfqpoint{5.621746in}{1.032849in}}%
\pgfpathlineto{\pgfqpoint{5.650512in}{1.032849in}}%
\pgfpathlineto{\pgfqpoint{5.679277in}{0.984480in}}%
\pgfpathlineto{\pgfqpoint{5.708043in}{0.984480in}}%
\pgfpathlineto{\pgfqpoint{5.736808in}{0.984480in}}%
\pgfpathlineto{\pgfqpoint{5.765574in}{0.984480in}}%
\pgfpathlineto{\pgfqpoint{5.794339in}{0.984480in}}%
\pgfpathlineto{\pgfqpoint{5.823104in}{0.984480in}}%
\pgfpathlineto{\pgfqpoint{5.851870in}{0.984480in}}%
\pgfpathlineto{\pgfqpoint{5.880635in}{0.984480in}}%
\pgfpathlineto{\pgfqpoint{5.909401in}{0.984480in}}%
\pgfpathlineto{\pgfqpoint{5.938166in}{0.983528in}}%
\pgfpathlineto{\pgfqpoint{5.966932in}{0.983528in}}%
\pgfpathlineto{\pgfqpoint{5.966932in}{0.477955in}}%
\pgfpathlineto{\pgfqpoint{5.966932in}{0.477955in}}%
\pgfpathlineto{\pgfqpoint{5.938166in}{0.477955in}}%
\pgfpathlineto{\pgfqpoint{5.909401in}{0.485086in}}%
\pgfpathlineto{\pgfqpoint{5.880635in}{0.485086in}}%
\pgfpathlineto{\pgfqpoint{5.851870in}{0.485086in}}%
\pgfpathlineto{\pgfqpoint{5.823104in}{0.485086in}}%
\pgfpathlineto{\pgfqpoint{5.794339in}{0.485086in}}%
\pgfpathlineto{\pgfqpoint{5.765574in}{0.485086in}}%
\pgfpathlineto{\pgfqpoint{5.736808in}{0.485086in}}%
\pgfpathlineto{\pgfqpoint{5.708043in}{0.485086in}}%
\pgfpathlineto{\pgfqpoint{5.679277in}{0.485086in}}%
\pgfpathlineto{\pgfqpoint{5.650512in}{0.522867in}}%
\pgfpathlineto{\pgfqpoint{5.621746in}{0.522867in}}%
\pgfpathlineto{\pgfqpoint{5.592981in}{0.522867in}}%
\pgfpathlineto{\pgfqpoint{5.564215in}{0.522867in}}%
\pgfpathlineto{\pgfqpoint{5.535450in}{0.522867in}}%
\pgfpathlineto{\pgfqpoint{5.506684in}{0.559686in}}%
\pgfpathlineto{\pgfqpoint{5.477919in}{0.559686in}}%
\pgfpathlineto{\pgfqpoint{5.449153in}{0.561022in}}%
\pgfpathlineto{\pgfqpoint{5.420388in}{0.561022in}}%
\pgfpathlineto{\pgfqpoint{5.391623in}{0.561022in}}%
\pgfpathlineto{\pgfqpoint{5.362857in}{0.561022in}}%
\pgfpathlineto{\pgfqpoint{5.334092in}{0.602551in}}%
\pgfpathlineto{\pgfqpoint{5.305326in}{0.602551in}}%
\pgfpathlineto{\pgfqpoint{5.276561in}{0.614783in}}%
\pgfpathlineto{\pgfqpoint{5.247795in}{0.614783in}}%
\pgfpathlineto{\pgfqpoint{5.219030in}{0.614783in}}%
\pgfpathlineto{\pgfqpoint{5.190264in}{0.614783in}}%
\pgfpathlineto{\pgfqpoint{5.161499in}{0.614783in}}%
\pgfpathlineto{\pgfqpoint{5.132733in}{0.614783in}}%
\pgfpathlineto{\pgfqpoint{5.103968in}{0.614783in}}%
\pgfpathlineto{\pgfqpoint{5.075202in}{0.614783in}}%
\pgfpathlineto{\pgfqpoint{5.046437in}{0.628096in}}%
\pgfpathlineto{\pgfqpoint{5.017671in}{0.628096in}}%
\pgfpathlineto{\pgfqpoint{4.988906in}{0.628096in}}%
\pgfpathlineto{\pgfqpoint{4.960141in}{0.628140in}}%
\pgfpathlineto{\pgfqpoint{4.931375in}{0.628140in}}%
\pgfpathlineto{\pgfqpoint{4.902610in}{0.628140in}}%
\pgfpathlineto{\pgfqpoint{4.873844in}{0.628140in}}%
\pgfpathlineto{\pgfqpoint{4.845079in}{0.628140in}}%
\pgfpathlineto{\pgfqpoint{4.816313in}{0.628140in}}%
\pgfpathlineto{\pgfqpoint{4.787548in}{0.628140in}}%
\pgfpathlineto{\pgfqpoint{4.758782in}{0.628957in}}%
\pgfpathlineto{\pgfqpoint{4.730017in}{0.628957in}}%
\pgfpathlineto{\pgfqpoint{4.701251in}{0.628957in}}%
\pgfpathlineto{\pgfqpoint{4.672486in}{0.628957in}}%
\pgfpathlineto{\pgfqpoint{4.643720in}{0.628957in}}%
\pgfpathlineto{\pgfqpoint{4.614955in}{0.628957in}}%
\pgfpathlineto{\pgfqpoint{4.586189in}{0.628957in}}%
\pgfpathlineto{\pgfqpoint{4.557424in}{0.628957in}}%
\pgfpathlineto{\pgfqpoint{4.528659in}{0.628957in}}%
\pgfpathlineto{\pgfqpoint{4.499893in}{0.637973in}}%
\pgfpathlineto{\pgfqpoint{4.471128in}{0.637973in}}%
\pgfpathlineto{\pgfqpoint{4.442362in}{0.656353in}}%
\pgfpathlineto{\pgfqpoint{4.413597in}{0.656353in}}%
\pgfpathlineto{\pgfqpoint{4.384831in}{0.656353in}}%
\pgfpathlineto{\pgfqpoint{4.356066in}{0.678040in}}%
\pgfpathlineto{\pgfqpoint{4.327300in}{0.678040in}}%
\pgfpathlineto{\pgfqpoint{4.298535in}{0.678040in}}%
\pgfpathlineto{\pgfqpoint{4.269769in}{0.678040in}}%
\pgfpathlineto{\pgfqpoint{4.241004in}{0.678040in}}%
\pgfpathlineto{\pgfqpoint{4.212238in}{0.678040in}}%
\pgfpathlineto{\pgfqpoint{4.183473in}{0.681494in}}%
\pgfpathlineto{\pgfqpoint{4.154708in}{0.681494in}}%
\pgfpathlineto{\pgfqpoint{4.125942in}{0.689283in}}%
\pgfpathlineto{\pgfqpoint{4.097177in}{0.691970in}}%
\pgfpathlineto{\pgfqpoint{4.068411in}{0.691970in}}%
\pgfpathlineto{\pgfqpoint{4.039646in}{0.691970in}}%
\pgfpathlineto{\pgfqpoint{4.010880in}{0.691970in}}%
\pgfpathlineto{\pgfqpoint{3.982115in}{0.691970in}}%
\pgfpathlineto{\pgfqpoint{3.953349in}{0.691970in}}%
\pgfpathlineto{\pgfqpoint{3.924584in}{0.691970in}}%
\pgfpathlineto{\pgfqpoint{3.895818in}{0.691970in}}%
\pgfpathlineto{\pgfqpoint{3.867053in}{0.726045in}}%
\pgfpathlineto{\pgfqpoint{3.838287in}{0.726045in}}%
\pgfpathlineto{\pgfqpoint{3.809522in}{0.726045in}}%
\pgfpathlineto{\pgfqpoint{3.780756in}{0.726045in}}%
\pgfpathlineto{\pgfqpoint{3.751991in}{0.787491in}}%
\pgfpathlineto{\pgfqpoint{3.723226in}{0.878474in}}%
\pgfpathlineto{\pgfqpoint{3.694460in}{0.907683in}}%
\pgfpathlineto{\pgfqpoint{3.665695in}{0.907683in}}%
\pgfpathlineto{\pgfqpoint{3.636929in}{0.911877in}}%
\pgfpathlineto{\pgfqpoint{3.608164in}{0.911877in}}%
\pgfpathlineto{\pgfqpoint{3.579398in}{0.911877in}}%
\pgfpathlineto{\pgfqpoint{3.550633in}{0.911877in}}%
\pgfpathlineto{\pgfqpoint{3.521867in}{0.911877in}}%
\pgfpathlineto{\pgfqpoint{3.493102in}{0.961557in}}%
\pgfpathlineto{\pgfqpoint{3.464336in}{0.961557in}}%
\pgfpathlineto{\pgfqpoint{3.435571in}{0.966031in}}%
\pgfpathlineto{\pgfqpoint{3.406805in}{0.966031in}}%
\pgfpathlineto{\pgfqpoint{3.378040in}{0.988192in}}%
\pgfpathlineto{\pgfqpoint{3.349274in}{1.030361in}}%
\pgfpathlineto{\pgfqpoint{3.320509in}{1.030361in}}%
\pgfpathlineto{\pgfqpoint{3.291744in}{1.030361in}}%
\pgfpathlineto{\pgfqpoint{3.262978in}{1.030361in}}%
\pgfpathlineto{\pgfqpoint{3.234213in}{1.095621in}}%
\pgfpathlineto{\pgfqpoint{3.205447in}{1.095621in}}%
\pgfpathlineto{\pgfqpoint{3.176682in}{1.095621in}}%
\pgfpathlineto{\pgfqpoint{3.147916in}{1.095621in}}%
\pgfpathlineto{\pgfqpoint{3.119151in}{1.179377in}}%
\pgfpathlineto{\pgfqpoint{3.090385in}{1.179377in}}%
\pgfpathlineto{\pgfqpoint{3.061620in}{1.224849in}}%
\pgfpathlineto{\pgfqpoint{3.032854in}{1.270928in}}%
\pgfpathlineto{\pgfqpoint{3.004089in}{1.270928in}}%
\pgfpathlineto{\pgfqpoint{2.975323in}{1.270928in}}%
\pgfpathlineto{\pgfqpoint{2.946558in}{1.283541in}}%
\pgfpathlineto{\pgfqpoint{2.917792in}{1.283541in}}%
\pgfpathlineto{\pgfqpoint{2.889027in}{1.295998in}}%
\pgfpathlineto{\pgfqpoint{2.860262in}{1.337306in}}%
\pgfpathlineto{\pgfqpoint{2.831496in}{1.420702in}}%
\pgfpathlineto{\pgfqpoint{2.802731in}{1.421934in}}%
\pgfpathlineto{\pgfqpoint{2.773965in}{1.421934in}}%
\pgfpathlineto{\pgfqpoint{2.745200in}{1.461720in}}%
\pgfpathlineto{\pgfqpoint{2.716434in}{1.483807in}}%
\pgfpathlineto{\pgfqpoint{2.687669in}{1.530038in}}%
\pgfpathlineto{\pgfqpoint{2.658903in}{1.530038in}}%
\pgfpathlineto{\pgfqpoint{2.630138in}{1.537937in}}%
\pgfpathlineto{\pgfqpoint{2.601372in}{1.552853in}}%
\pgfpathlineto{\pgfqpoint{2.572607in}{1.587507in}}%
\pgfpathlineto{\pgfqpoint{2.543841in}{1.587507in}}%
\pgfpathlineto{\pgfqpoint{2.515076in}{1.592514in}}%
\pgfpathlineto{\pgfqpoint{2.486311in}{1.592514in}}%
\pgfpathlineto{\pgfqpoint{2.457545in}{1.592514in}}%
\pgfpathlineto{\pgfqpoint{2.428780in}{1.599264in}}%
\pgfpathlineto{\pgfqpoint{2.400014in}{1.599264in}}%
\pgfpathlineto{\pgfqpoint{2.371249in}{1.607349in}}%
\pgfpathlineto{\pgfqpoint{2.342483in}{1.613328in}}%
\pgfpathlineto{\pgfqpoint{2.313718in}{1.665409in}}%
\pgfpathlineto{\pgfqpoint{2.284952in}{1.665409in}}%
\pgfpathlineto{\pgfqpoint{2.256187in}{1.714593in}}%
\pgfpathlineto{\pgfqpoint{2.227421in}{1.714593in}}%
\pgfpathlineto{\pgfqpoint{2.198656in}{1.726722in}}%
\pgfpathlineto{\pgfqpoint{2.169890in}{1.726722in}}%
\pgfpathlineto{\pgfqpoint{2.141125in}{1.734911in}}%
\pgfpathlineto{\pgfqpoint{2.112359in}{1.789518in}}%
\pgfpathlineto{\pgfqpoint{2.083594in}{1.875667in}}%
\pgfpathlineto{\pgfqpoint{2.054829in}{1.970377in}}%
\pgfpathlineto{\pgfqpoint{2.026063in}{1.975996in}}%
\pgfpathlineto{\pgfqpoint{1.997298in}{1.975996in}}%
\pgfpathlineto{\pgfqpoint{1.968532in}{1.975996in}}%
\pgfpathlineto{\pgfqpoint{1.939767in}{2.060628in}}%
\pgfpathlineto{\pgfqpoint{1.911001in}{2.060628in}}%
\pgfpathlineto{\pgfqpoint{1.882236in}{2.060628in}}%
\pgfpathlineto{\pgfqpoint{1.853470in}{2.063700in}}%
\pgfpathlineto{\pgfqpoint{1.824705in}{2.099800in}}%
\pgfpathlineto{\pgfqpoint{1.795939in}{2.099800in}}%
\pgfpathlineto{\pgfqpoint{1.767174in}{2.099800in}}%
\pgfpathlineto{\pgfqpoint{1.738408in}{2.099800in}}%
\pgfpathlineto{\pgfqpoint{1.709643in}{2.101603in}}%
\pgfpathlineto{\pgfqpoint{1.680877in}{2.101603in}}%
\pgfpathlineto{\pgfqpoint{1.652112in}{2.101603in}}%
\pgfpathlineto{\pgfqpoint{1.623347in}{2.101603in}}%
\pgfpathlineto{\pgfqpoint{1.594581in}{2.101603in}}%
\pgfpathlineto{\pgfqpoint{1.565816in}{2.103325in}}%
\pgfpathlineto{\pgfqpoint{1.537050in}{2.244916in}}%
\pgfpathlineto{\pgfqpoint{1.508285in}{2.244916in}}%
\pgfpathlineto{\pgfqpoint{1.479519in}{2.298871in}}%
\pgfpathlineto{\pgfqpoint{1.450754in}{2.353880in}}%
\pgfpathlineto{\pgfqpoint{1.421988in}{2.353880in}}%
\pgfpathlineto{\pgfqpoint{1.393223in}{2.353880in}}%
\pgfpathlineto{\pgfqpoint{1.364457in}{2.353880in}}%
\pgfpathlineto{\pgfqpoint{1.335692in}{2.353880in}}%
\pgfpathlineto{\pgfqpoint{1.306926in}{2.357952in}}%
\pgfpathlineto{\pgfqpoint{1.278161in}{2.357952in}}%
\pgfpathlineto{\pgfqpoint{1.249396in}{2.357952in}}%
\pgfpathlineto{\pgfqpoint{1.220630in}{2.372685in}}%
\pgfpathlineto{\pgfqpoint{1.191865in}{2.476988in}}%
\pgfpathlineto{\pgfqpoint{1.163099in}{2.482881in}}%
\pgfpathlineto{\pgfqpoint{1.134334in}{2.517446in}}%
\pgfpathlineto{\pgfqpoint{1.105568in}{2.517446in}}%
\pgfpathclose%
\pgfusepath{fill}%
\end{pgfscope}%
\begin{pgfscope}%
\pgfpathrectangle{\pgfqpoint{0.862500in}{0.375000in}}{\pgfqpoint{5.347500in}{2.265000in}}%
\pgfusepath{clip}%
\pgfsetbuttcap%
\pgfsetroundjoin%
\definecolor{currentfill}{rgb}{0.839216,0.152941,0.156863}%
\pgfsetfillcolor{currentfill}%
\pgfsetfillopacity{0.200000}%
\pgfsetlinewidth{0.000000pt}%
\definecolor{currentstroke}{rgb}{0.000000,0.000000,0.000000}%
\pgfsetstrokecolor{currentstroke}%
\pgfsetdash{}{0pt}%
\pgfpathmoveto{\pgfqpoint{1.105568in}{2.474870in}}%
\pgfpathlineto{\pgfqpoint{1.105568in}{2.537045in}}%
\pgfpathlineto{\pgfqpoint{1.134334in}{2.535796in}}%
\pgfpathlineto{\pgfqpoint{1.163099in}{2.497300in}}%
\pgfpathlineto{\pgfqpoint{1.191865in}{2.454149in}}%
\pgfpathlineto{\pgfqpoint{1.220630in}{2.417582in}}%
\pgfpathlineto{\pgfqpoint{1.249396in}{2.417582in}}%
\pgfpathlineto{\pgfqpoint{1.278161in}{2.417582in}}%
\pgfpathlineto{\pgfqpoint{1.306926in}{2.370196in}}%
\pgfpathlineto{\pgfqpoint{1.335692in}{2.370196in}}%
\pgfpathlineto{\pgfqpoint{1.364457in}{2.370196in}}%
\pgfpathlineto{\pgfqpoint{1.393223in}{2.358545in}}%
\pgfpathlineto{\pgfqpoint{1.421988in}{2.358545in}}%
\pgfpathlineto{\pgfqpoint{1.450754in}{2.351918in}}%
\pgfpathlineto{\pgfqpoint{1.479519in}{2.351918in}}%
\pgfpathlineto{\pgfqpoint{1.508285in}{2.351918in}}%
\pgfpathlineto{\pgfqpoint{1.537050in}{2.316012in}}%
\pgfpathlineto{\pgfqpoint{1.565816in}{2.316012in}}%
\pgfpathlineto{\pgfqpoint{1.594581in}{2.316012in}}%
\pgfpathlineto{\pgfqpoint{1.623347in}{2.316012in}}%
\pgfpathlineto{\pgfqpoint{1.652112in}{2.316012in}}%
\pgfpathlineto{\pgfqpoint{1.680877in}{2.316012in}}%
\pgfpathlineto{\pgfqpoint{1.709643in}{2.316012in}}%
\pgfpathlineto{\pgfqpoint{1.738408in}{2.316012in}}%
\pgfpathlineto{\pgfqpoint{1.767174in}{2.316012in}}%
\pgfpathlineto{\pgfqpoint{1.795939in}{2.269686in}}%
\pgfpathlineto{\pgfqpoint{1.824705in}{2.259512in}}%
\pgfpathlineto{\pgfqpoint{1.853470in}{2.258351in}}%
\pgfpathlineto{\pgfqpoint{1.882236in}{2.239499in}}%
\pgfpathlineto{\pgfqpoint{1.911001in}{2.239499in}}%
\pgfpathlineto{\pgfqpoint{1.939767in}{2.239499in}}%
\pgfpathlineto{\pgfqpoint{1.968532in}{2.146461in}}%
\pgfpathlineto{\pgfqpoint{1.997298in}{2.097524in}}%
\pgfpathlineto{\pgfqpoint{2.026063in}{2.097524in}}%
\pgfpathlineto{\pgfqpoint{2.054829in}{2.097524in}}%
\pgfpathlineto{\pgfqpoint{2.083594in}{2.097524in}}%
\pgfpathlineto{\pgfqpoint{2.112359in}{2.094463in}}%
\pgfpathlineto{\pgfqpoint{2.141125in}{2.090391in}}%
\pgfpathlineto{\pgfqpoint{2.169890in}{2.090391in}}%
\pgfpathlineto{\pgfqpoint{2.198656in}{2.082167in}}%
\pgfpathlineto{\pgfqpoint{2.227421in}{2.081287in}}%
\pgfpathlineto{\pgfqpoint{2.256187in}{2.005962in}}%
\pgfpathlineto{\pgfqpoint{2.284952in}{2.005962in}}%
\pgfpathlineto{\pgfqpoint{2.313718in}{2.005962in}}%
\pgfpathlineto{\pgfqpoint{2.342483in}{1.990363in}}%
\pgfpathlineto{\pgfqpoint{2.371249in}{1.990363in}}%
\pgfpathlineto{\pgfqpoint{2.400014in}{1.971032in}}%
\pgfpathlineto{\pgfqpoint{2.428780in}{1.914724in}}%
\pgfpathlineto{\pgfqpoint{2.457545in}{1.914724in}}%
\pgfpathlineto{\pgfqpoint{2.486311in}{1.911551in}}%
\pgfpathlineto{\pgfqpoint{2.515076in}{1.910977in}}%
\pgfpathlineto{\pgfqpoint{2.543841in}{1.910977in}}%
\pgfpathlineto{\pgfqpoint{2.572607in}{1.881994in}}%
\pgfpathlineto{\pgfqpoint{2.601372in}{1.828238in}}%
\pgfpathlineto{\pgfqpoint{2.630138in}{1.776155in}}%
\pgfpathlineto{\pgfqpoint{2.658903in}{1.776155in}}%
\pgfpathlineto{\pgfqpoint{2.687669in}{1.776155in}}%
\pgfpathlineto{\pgfqpoint{2.716434in}{1.776155in}}%
\pgfpathlineto{\pgfqpoint{2.745200in}{1.776155in}}%
\pgfpathlineto{\pgfqpoint{2.773965in}{1.747567in}}%
\pgfpathlineto{\pgfqpoint{2.802731in}{1.747567in}}%
\pgfpathlineto{\pgfqpoint{2.831496in}{1.747567in}}%
\pgfpathlineto{\pgfqpoint{2.860262in}{1.747567in}}%
\pgfpathlineto{\pgfqpoint{2.889027in}{1.658966in}}%
\pgfpathlineto{\pgfqpoint{2.917792in}{1.628832in}}%
\pgfpathlineto{\pgfqpoint{2.946558in}{1.603128in}}%
\pgfpathlineto{\pgfqpoint{2.975323in}{1.603128in}}%
\pgfpathlineto{\pgfqpoint{3.004089in}{1.603128in}}%
\pgfpathlineto{\pgfqpoint{3.032854in}{1.603128in}}%
\pgfpathlineto{\pgfqpoint{3.061620in}{1.593117in}}%
\pgfpathlineto{\pgfqpoint{3.090385in}{1.574808in}}%
\pgfpathlineto{\pgfqpoint{3.119151in}{1.574808in}}%
\pgfpathlineto{\pgfqpoint{3.147916in}{1.574808in}}%
\pgfpathlineto{\pgfqpoint{3.176682in}{1.574808in}}%
\pgfpathlineto{\pgfqpoint{3.205447in}{1.574808in}}%
\pgfpathlineto{\pgfqpoint{3.234213in}{1.574808in}}%
\pgfpathlineto{\pgfqpoint{3.262978in}{1.574808in}}%
\pgfpathlineto{\pgfqpoint{3.291744in}{1.574808in}}%
\pgfpathlineto{\pgfqpoint{3.320509in}{1.574808in}}%
\pgfpathlineto{\pgfqpoint{3.349274in}{1.574808in}}%
\pgfpathlineto{\pgfqpoint{3.378040in}{1.573930in}}%
\pgfpathlineto{\pgfqpoint{3.406805in}{1.598026in}}%
\pgfpathlineto{\pgfqpoint{3.435571in}{1.592489in}}%
\pgfpathlineto{\pgfqpoint{3.464336in}{1.529669in}}%
\pgfpathlineto{\pgfqpoint{3.493102in}{1.529669in}}%
\pgfpathlineto{\pgfqpoint{3.521867in}{1.521898in}}%
\pgfpathlineto{\pgfqpoint{3.550633in}{1.521898in}}%
\pgfpathlineto{\pgfqpoint{3.579398in}{1.464726in}}%
\pgfpathlineto{\pgfqpoint{3.608164in}{1.464726in}}%
\pgfpathlineto{\pgfqpoint{3.636929in}{1.463142in}}%
\pgfpathlineto{\pgfqpoint{3.665695in}{1.463142in}}%
\pgfpathlineto{\pgfqpoint{3.694460in}{1.463142in}}%
\pgfpathlineto{\pgfqpoint{3.723226in}{1.463142in}}%
\pgfpathlineto{\pgfqpoint{3.751991in}{1.458182in}}%
\pgfpathlineto{\pgfqpoint{3.780756in}{1.458182in}}%
\pgfpathlineto{\pgfqpoint{3.809522in}{1.458182in}}%
\pgfpathlineto{\pgfqpoint{3.838287in}{1.458182in}}%
\pgfpathlineto{\pgfqpoint{3.867053in}{1.458182in}}%
\pgfpathlineto{\pgfqpoint{3.895818in}{1.458182in}}%
\pgfpathlineto{\pgfqpoint{3.924584in}{1.458182in}}%
\pgfpathlineto{\pgfqpoint{3.953349in}{1.458182in}}%
\pgfpathlineto{\pgfqpoint{3.982115in}{1.419883in}}%
\pgfpathlineto{\pgfqpoint{4.010880in}{1.419883in}}%
\pgfpathlineto{\pgfqpoint{4.039646in}{1.419883in}}%
\pgfpathlineto{\pgfqpoint{4.068411in}{1.419883in}}%
\pgfpathlineto{\pgfqpoint{4.097177in}{1.419883in}}%
\pgfpathlineto{\pgfqpoint{4.125942in}{1.419883in}}%
\pgfpathlineto{\pgfqpoint{4.154708in}{1.419883in}}%
\pgfpathlineto{\pgfqpoint{4.183473in}{1.417022in}}%
\pgfpathlineto{\pgfqpoint{4.212238in}{1.417022in}}%
\pgfpathlineto{\pgfqpoint{4.241004in}{1.377746in}}%
\pgfpathlineto{\pgfqpoint{4.269769in}{1.377746in}}%
\pgfpathlineto{\pgfqpoint{4.298535in}{1.315887in}}%
\pgfpathlineto{\pgfqpoint{4.327300in}{1.315887in}}%
\pgfpathlineto{\pgfqpoint{4.356066in}{1.286191in}}%
\pgfpathlineto{\pgfqpoint{4.384831in}{1.286191in}}%
\pgfpathlineto{\pgfqpoint{4.413597in}{1.286191in}}%
\pgfpathlineto{\pgfqpoint{4.442362in}{1.286191in}}%
\pgfpathlineto{\pgfqpoint{4.471128in}{1.286191in}}%
\pgfpathlineto{\pgfqpoint{4.499893in}{1.286191in}}%
\pgfpathlineto{\pgfqpoint{4.528659in}{1.286191in}}%
\pgfpathlineto{\pgfqpoint{4.557424in}{1.286191in}}%
\pgfpathlineto{\pgfqpoint{4.586189in}{1.286191in}}%
\pgfpathlineto{\pgfqpoint{4.614955in}{1.286191in}}%
\pgfpathlineto{\pgfqpoint{4.643720in}{1.286191in}}%
\pgfpathlineto{\pgfqpoint{4.672486in}{1.286191in}}%
\pgfpathlineto{\pgfqpoint{4.701251in}{1.286191in}}%
\pgfpathlineto{\pgfqpoint{4.730017in}{1.286191in}}%
\pgfpathlineto{\pgfqpoint{4.758782in}{1.286191in}}%
\pgfpathlineto{\pgfqpoint{4.787548in}{1.286191in}}%
\pgfpathlineto{\pgfqpoint{4.816313in}{1.286191in}}%
\pgfpathlineto{\pgfqpoint{4.845079in}{1.286191in}}%
\pgfpathlineto{\pgfqpoint{4.873844in}{1.286191in}}%
\pgfpathlineto{\pgfqpoint{4.902610in}{1.286191in}}%
\pgfpathlineto{\pgfqpoint{4.931375in}{1.286191in}}%
\pgfpathlineto{\pgfqpoint{4.960141in}{1.286191in}}%
\pgfpathlineto{\pgfqpoint{4.988906in}{1.286191in}}%
\pgfpathlineto{\pgfqpoint{5.017671in}{1.286191in}}%
\pgfpathlineto{\pgfqpoint{5.046437in}{1.286191in}}%
\pgfpathlineto{\pgfqpoint{5.075202in}{1.283031in}}%
\pgfpathlineto{\pgfqpoint{5.103968in}{1.283031in}}%
\pgfpathlineto{\pgfqpoint{5.132733in}{1.283031in}}%
\pgfpathlineto{\pgfqpoint{5.161499in}{1.283031in}}%
\pgfpathlineto{\pgfqpoint{5.190264in}{1.283031in}}%
\pgfpathlineto{\pgfqpoint{5.219030in}{1.283031in}}%
\pgfpathlineto{\pgfqpoint{5.247795in}{1.283031in}}%
\pgfpathlineto{\pgfqpoint{5.276561in}{1.283031in}}%
\pgfpathlineto{\pgfqpoint{5.305326in}{1.283031in}}%
\pgfpathlineto{\pgfqpoint{5.334092in}{1.283031in}}%
\pgfpathlineto{\pgfqpoint{5.362857in}{1.283031in}}%
\pgfpathlineto{\pgfqpoint{5.391623in}{1.255005in}}%
\pgfpathlineto{\pgfqpoint{5.420388in}{1.255005in}}%
\pgfpathlineto{\pgfqpoint{5.449153in}{1.255005in}}%
\pgfpathlineto{\pgfqpoint{5.477919in}{1.255005in}}%
\pgfpathlineto{\pgfqpoint{5.506684in}{1.255005in}}%
\pgfpathlineto{\pgfqpoint{5.535450in}{1.255005in}}%
\pgfpathlineto{\pgfqpoint{5.564215in}{1.255005in}}%
\pgfpathlineto{\pgfqpoint{5.592981in}{1.255005in}}%
\pgfpathlineto{\pgfqpoint{5.621746in}{1.255005in}}%
\pgfpathlineto{\pgfqpoint{5.650512in}{1.255005in}}%
\pgfpathlineto{\pgfqpoint{5.679277in}{1.255005in}}%
\pgfpathlineto{\pgfqpoint{5.708043in}{1.255005in}}%
\pgfpathlineto{\pgfqpoint{5.736808in}{1.195912in}}%
\pgfpathlineto{\pgfqpoint{5.765574in}{1.195912in}}%
\pgfpathlineto{\pgfqpoint{5.794339in}{1.195912in}}%
\pgfpathlineto{\pgfqpoint{5.823104in}{1.180690in}}%
\pgfpathlineto{\pgfqpoint{5.851870in}{1.180690in}}%
\pgfpathlineto{\pgfqpoint{5.880635in}{1.180690in}}%
\pgfpathlineto{\pgfqpoint{5.909401in}{1.180690in}}%
\pgfpathlineto{\pgfqpoint{5.938166in}{1.180690in}}%
\pgfpathlineto{\pgfqpoint{5.966932in}{1.180690in}}%
\pgfpathlineto{\pgfqpoint{5.966932in}{1.062071in}}%
\pgfpathlineto{\pgfqpoint{5.966932in}{1.062071in}}%
\pgfpathlineto{\pgfqpoint{5.938166in}{1.062071in}}%
\pgfpathlineto{\pgfqpoint{5.909401in}{1.062071in}}%
\pgfpathlineto{\pgfqpoint{5.880635in}{1.062071in}}%
\pgfpathlineto{\pgfqpoint{5.851870in}{1.062071in}}%
\pgfpathlineto{\pgfqpoint{5.823104in}{1.062071in}}%
\pgfpathlineto{\pgfqpoint{5.794339in}{1.092517in}}%
\pgfpathlineto{\pgfqpoint{5.765574in}{1.092517in}}%
\pgfpathlineto{\pgfqpoint{5.736808in}{1.092517in}}%
\pgfpathlineto{\pgfqpoint{5.708043in}{1.138297in}}%
\pgfpathlineto{\pgfqpoint{5.679277in}{1.138297in}}%
\pgfpathlineto{\pgfqpoint{5.650512in}{1.138297in}}%
\pgfpathlineto{\pgfqpoint{5.621746in}{1.138297in}}%
\pgfpathlineto{\pgfqpoint{5.592981in}{1.138297in}}%
\pgfpathlineto{\pgfqpoint{5.564215in}{1.138297in}}%
\pgfpathlineto{\pgfqpoint{5.535450in}{1.138297in}}%
\pgfpathlineto{\pgfqpoint{5.506684in}{1.138297in}}%
\pgfpathlineto{\pgfqpoint{5.477919in}{1.138297in}}%
\pgfpathlineto{\pgfqpoint{5.449153in}{1.138297in}}%
\pgfpathlineto{\pgfqpoint{5.420388in}{1.138297in}}%
\pgfpathlineto{\pgfqpoint{5.391623in}{1.138297in}}%
\pgfpathlineto{\pgfqpoint{5.362857in}{1.151005in}}%
\pgfpathlineto{\pgfqpoint{5.334092in}{1.151005in}}%
\pgfpathlineto{\pgfqpoint{5.305326in}{1.151005in}}%
\pgfpathlineto{\pgfqpoint{5.276561in}{1.151005in}}%
\pgfpathlineto{\pgfqpoint{5.247795in}{1.151005in}}%
\pgfpathlineto{\pgfqpoint{5.219030in}{1.151005in}}%
\pgfpathlineto{\pgfqpoint{5.190264in}{1.151005in}}%
\pgfpathlineto{\pgfqpoint{5.161499in}{1.151005in}}%
\pgfpathlineto{\pgfqpoint{5.132733in}{1.151005in}}%
\pgfpathlineto{\pgfqpoint{5.103968in}{1.151005in}}%
\pgfpathlineto{\pgfqpoint{5.075202in}{1.151005in}}%
\pgfpathlineto{\pgfqpoint{5.046437in}{1.161239in}}%
\pgfpathlineto{\pgfqpoint{5.017671in}{1.161239in}}%
\pgfpathlineto{\pgfqpoint{4.988906in}{1.161239in}}%
\pgfpathlineto{\pgfqpoint{4.960141in}{1.161239in}}%
\pgfpathlineto{\pgfqpoint{4.931375in}{1.161239in}}%
\pgfpathlineto{\pgfqpoint{4.902610in}{1.161239in}}%
\pgfpathlineto{\pgfqpoint{4.873844in}{1.161239in}}%
\pgfpathlineto{\pgfqpoint{4.845079in}{1.161239in}}%
\pgfpathlineto{\pgfqpoint{4.816313in}{1.161239in}}%
\pgfpathlineto{\pgfqpoint{4.787548in}{1.161239in}}%
\pgfpathlineto{\pgfqpoint{4.758782in}{1.161239in}}%
\pgfpathlineto{\pgfqpoint{4.730017in}{1.161239in}}%
\pgfpathlineto{\pgfqpoint{4.701251in}{1.161239in}}%
\pgfpathlineto{\pgfqpoint{4.672486in}{1.161239in}}%
\pgfpathlineto{\pgfqpoint{4.643720in}{1.161239in}}%
\pgfpathlineto{\pgfqpoint{4.614955in}{1.161239in}}%
\pgfpathlineto{\pgfqpoint{4.586189in}{1.161239in}}%
\pgfpathlineto{\pgfqpoint{4.557424in}{1.161239in}}%
\pgfpathlineto{\pgfqpoint{4.528659in}{1.161239in}}%
\pgfpathlineto{\pgfqpoint{4.499893in}{1.161239in}}%
\pgfpathlineto{\pgfqpoint{4.471128in}{1.161239in}}%
\pgfpathlineto{\pgfqpoint{4.442362in}{1.161239in}}%
\pgfpathlineto{\pgfqpoint{4.413597in}{1.161239in}}%
\pgfpathlineto{\pgfqpoint{4.384831in}{1.161239in}}%
\pgfpathlineto{\pgfqpoint{4.356066in}{1.161239in}}%
\pgfpathlineto{\pgfqpoint{4.327300in}{1.177080in}}%
\pgfpathlineto{\pgfqpoint{4.298535in}{1.177080in}}%
\pgfpathlineto{\pgfqpoint{4.269769in}{1.262167in}}%
\pgfpathlineto{\pgfqpoint{4.241004in}{1.262167in}}%
\pgfpathlineto{\pgfqpoint{4.212238in}{1.288801in}}%
\pgfpathlineto{\pgfqpoint{4.183473in}{1.288801in}}%
\pgfpathlineto{\pgfqpoint{4.154708in}{1.324970in}}%
\pgfpathlineto{\pgfqpoint{4.125942in}{1.324970in}}%
\pgfpathlineto{\pgfqpoint{4.097177in}{1.324970in}}%
\pgfpathlineto{\pgfqpoint{4.068411in}{1.324970in}}%
\pgfpathlineto{\pgfqpoint{4.039646in}{1.324970in}}%
\pgfpathlineto{\pgfqpoint{4.010880in}{1.324970in}}%
\pgfpathlineto{\pgfqpoint{3.982115in}{1.324970in}}%
\pgfpathlineto{\pgfqpoint{3.953349in}{1.419989in}}%
\pgfpathlineto{\pgfqpoint{3.924584in}{1.419989in}}%
\pgfpathlineto{\pgfqpoint{3.895818in}{1.419989in}}%
\pgfpathlineto{\pgfqpoint{3.867053in}{1.419989in}}%
\pgfpathlineto{\pgfqpoint{3.838287in}{1.419989in}}%
\pgfpathlineto{\pgfqpoint{3.809522in}{1.419989in}}%
\pgfpathlineto{\pgfqpoint{3.780756in}{1.419989in}}%
\pgfpathlineto{\pgfqpoint{3.751991in}{1.419989in}}%
\pgfpathlineto{\pgfqpoint{3.723226in}{1.428691in}}%
\pgfpathlineto{\pgfqpoint{3.694460in}{1.428691in}}%
\pgfpathlineto{\pgfqpoint{3.665695in}{1.428691in}}%
\pgfpathlineto{\pgfqpoint{3.636929in}{1.428691in}}%
\pgfpathlineto{\pgfqpoint{3.608164in}{1.439140in}}%
\pgfpathlineto{\pgfqpoint{3.579398in}{1.439140in}}%
\pgfpathlineto{\pgfqpoint{3.550633in}{1.452594in}}%
\pgfpathlineto{\pgfqpoint{3.521867in}{1.452594in}}%
\pgfpathlineto{\pgfqpoint{3.493102in}{1.453875in}}%
\pgfpathlineto{\pgfqpoint{3.464336in}{1.453875in}}%
\pgfpathlineto{\pgfqpoint{3.435571in}{1.492366in}}%
\pgfpathlineto{\pgfqpoint{3.406805in}{1.510911in}}%
\pgfpathlineto{\pgfqpoint{3.378040in}{1.466692in}}%
\pgfpathlineto{\pgfqpoint{3.349274in}{1.471069in}}%
\pgfpathlineto{\pgfqpoint{3.320509in}{1.471069in}}%
\pgfpathlineto{\pgfqpoint{3.291744in}{1.471069in}}%
\pgfpathlineto{\pgfqpoint{3.262978in}{1.471069in}}%
\pgfpathlineto{\pgfqpoint{3.234213in}{1.471069in}}%
\pgfpathlineto{\pgfqpoint{3.205447in}{1.471069in}}%
\pgfpathlineto{\pgfqpoint{3.176682in}{1.471069in}}%
\pgfpathlineto{\pgfqpoint{3.147916in}{1.471069in}}%
\pgfpathlineto{\pgfqpoint{3.119151in}{1.471069in}}%
\pgfpathlineto{\pgfqpoint{3.090385in}{1.471069in}}%
\pgfpathlineto{\pgfqpoint{3.061620in}{1.475012in}}%
\pgfpathlineto{\pgfqpoint{3.032854in}{1.495740in}}%
\pgfpathlineto{\pgfqpoint{3.004089in}{1.495740in}}%
\pgfpathlineto{\pgfqpoint{2.975323in}{1.495740in}}%
\pgfpathlineto{\pgfqpoint{2.946558in}{1.495740in}}%
\pgfpathlineto{\pgfqpoint{2.917792in}{1.553836in}}%
\pgfpathlineto{\pgfqpoint{2.889027in}{1.575303in}}%
\pgfpathlineto{\pgfqpoint{2.860262in}{1.634484in}}%
\pgfpathlineto{\pgfqpoint{2.831496in}{1.634484in}}%
\pgfpathlineto{\pgfqpoint{2.802731in}{1.634484in}}%
\pgfpathlineto{\pgfqpoint{2.773965in}{1.634484in}}%
\pgfpathlineto{\pgfqpoint{2.745200in}{1.669747in}}%
\pgfpathlineto{\pgfqpoint{2.716434in}{1.669747in}}%
\pgfpathlineto{\pgfqpoint{2.687669in}{1.669747in}}%
\pgfpathlineto{\pgfqpoint{2.658903in}{1.669747in}}%
\pgfpathlineto{\pgfqpoint{2.630138in}{1.669747in}}%
\pgfpathlineto{\pgfqpoint{2.601372in}{1.684531in}}%
\pgfpathlineto{\pgfqpoint{2.572607in}{1.706395in}}%
\pgfpathlineto{\pgfqpoint{2.543841in}{1.781103in}}%
\pgfpathlineto{\pgfqpoint{2.515076in}{1.781103in}}%
\pgfpathlineto{\pgfqpoint{2.486311in}{1.782576in}}%
\pgfpathlineto{\pgfqpoint{2.457545in}{1.788690in}}%
\pgfpathlineto{\pgfqpoint{2.428780in}{1.788690in}}%
\pgfpathlineto{\pgfqpoint{2.400014in}{1.795853in}}%
\pgfpathlineto{\pgfqpoint{2.371249in}{1.830413in}}%
\pgfpathlineto{\pgfqpoint{2.342483in}{1.830413in}}%
\pgfpathlineto{\pgfqpoint{2.313718in}{1.846993in}}%
\pgfpathlineto{\pgfqpoint{2.284952in}{1.846993in}}%
\pgfpathlineto{\pgfqpoint{2.256187in}{1.846993in}}%
\pgfpathlineto{\pgfqpoint{2.227421in}{1.996676in}}%
\pgfpathlineto{\pgfqpoint{2.198656in}{1.998538in}}%
\pgfpathlineto{\pgfqpoint{2.169890in}{2.000813in}}%
\pgfpathlineto{\pgfqpoint{2.141125in}{2.000813in}}%
\pgfpathlineto{\pgfqpoint{2.112359in}{2.002907in}}%
\pgfpathlineto{\pgfqpoint{2.083594in}{2.008074in}}%
\pgfpathlineto{\pgfqpoint{2.054829in}{2.008074in}}%
\pgfpathlineto{\pgfqpoint{2.026063in}{2.008074in}}%
\pgfpathlineto{\pgfqpoint{1.997298in}{2.008074in}}%
\pgfpathlineto{\pgfqpoint{1.968532in}{2.041870in}}%
\pgfpathlineto{\pgfqpoint{1.939767in}{2.099564in}}%
\pgfpathlineto{\pgfqpoint{1.911001in}{2.099564in}}%
\pgfpathlineto{\pgfqpoint{1.882236in}{2.099564in}}%
\pgfpathlineto{\pgfqpoint{1.853470in}{2.124633in}}%
\pgfpathlineto{\pgfqpoint{1.824705in}{2.125801in}}%
\pgfpathlineto{\pgfqpoint{1.795939in}{2.128449in}}%
\pgfpathlineto{\pgfqpoint{1.767174in}{2.189209in}}%
\pgfpathlineto{\pgfqpoint{1.738408in}{2.189209in}}%
\pgfpathlineto{\pgfqpoint{1.709643in}{2.189209in}}%
\pgfpathlineto{\pgfqpoint{1.680877in}{2.189209in}}%
\pgfpathlineto{\pgfqpoint{1.652112in}{2.189209in}}%
\pgfpathlineto{\pgfqpoint{1.623347in}{2.189209in}}%
\pgfpathlineto{\pgfqpoint{1.594581in}{2.189209in}}%
\pgfpathlineto{\pgfqpoint{1.565816in}{2.189209in}}%
\pgfpathlineto{\pgfqpoint{1.537050in}{2.189209in}}%
\pgfpathlineto{\pgfqpoint{1.508285in}{2.200935in}}%
\pgfpathlineto{\pgfqpoint{1.479519in}{2.200935in}}%
\pgfpathlineto{\pgfqpoint{1.450754in}{2.200935in}}%
\pgfpathlineto{\pgfqpoint{1.421988in}{2.202591in}}%
\pgfpathlineto{\pgfqpoint{1.393223in}{2.202591in}}%
\pgfpathlineto{\pgfqpoint{1.364457in}{2.209119in}}%
\pgfpathlineto{\pgfqpoint{1.335692in}{2.209119in}}%
\pgfpathlineto{\pgfqpoint{1.306926in}{2.209119in}}%
\pgfpathlineto{\pgfqpoint{1.278161in}{2.325007in}}%
\pgfpathlineto{\pgfqpoint{1.249396in}{2.325007in}}%
\pgfpathlineto{\pgfqpoint{1.220630in}{2.325007in}}%
\pgfpathlineto{\pgfqpoint{1.191865in}{2.340699in}}%
\pgfpathlineto{\pgfqpoint{1.163099in}{2.396452in}}%
\pgfpathlineto{\pgfqpoint{1.134334in}{2.473891in}}%
\pgfpathlineto{\pgfqpoint{1.105568in}{2.474870in}}%
\pgfpathclose%
\pgfusepath{fill}%
\end{pgfscope}%
\begin{pgfscope}%
\pgfpathrectangle{\pgfqpoint{0.862500in}{0.375000in}}{\pgfqpoint{5.347500in}{2.265000in}}%
\pgfusepath{clip}%
\pgfsetbuttcap%
\pgfsetroundjoin%
\definecolor{currentfill}{rgb}{0.580392,0.403922,0.741176}%
\pgfsetfillcolor{currentfill}%
\pgfsetfillopacity{0.200000}%
\pgfsetlinewidth{0.000000pt}%
\definecolor{currentstroke}{rgb}{0.000000,0.000000,0.000000}%
\pgfsetstrokecolor{currentstroke}%
\pgfsetdash{}{0pt}%
\pgfpathmoveto{\pgfqpoint{1.105568in}{2.513232in}}%
\pgfpathlineto{\pgfqpoint{1.105568in}{2.530522in}}%
\pgfpathlineto{\pgfqpoint{1.134334in}{2.505717in}}%
\pgfpathlineto{\pgfqpoint{1.163099in}{2.505717in}}%
\pgfpathlineto{\pgfqpoint{1.191865in}{2.490775in}}%
\pgfpathlineto{\pgfqpoint{1.220630in}{2.449327in}}%
\pgfpathlineto{\pgfqpoint{1.249396in}{2.443622in}}%
\pgfpathlineto{\pgfqpoint{1.278161in}{2.443622in}}%
\pgfpathlineto{\pgfqpoint{1.306926in}{2.443622in}}%
\pgfpathlineto{\pgfqpoint{1.335692in}{2.443622in}}%
\pgfpathlineto{\pgfqpoint{1.364457in}{2.443622in}}%
\pgfpathlineto{\pgfqpoint{1.393223in}{2.428312in}}%
\pgfpathlineto{\pgfqpoint{1.421988in}{2.428312in}}%
\pgfpathlineto{\pgfqpoint{1.450754in}{2.428312in}}%
\pgfpathlineto{\pgfqpoint{1.479519in}{2.428312in}}%
\pgfpathlineto{\pgfqpoint{1.508285in}{2.428312in}}%
\pgfpathlineto{\pgfqpoint{1.537050in}{2.391305in}}%
\pgfpathlineto{\pgfqpoint{1.565816in}{2.391305in}}%
\pgfpathlineto{\pgfqpoint{1.594581in}{2.391305in}}%
\pgfpathlineto{\pgfqpoint{1.623347in}{2.391305in}}%
\pgfpathlineto{\pgfqpoint{1.652112in}{2.391305in}}%
\pgfpathlineto{\pgfqpoint{1.680877in}{2.391305in}}%
\pgfpathlineto{\pgfqpoint{1.709643in}{2.391305in}}%
\pgfpathlineto{\pgfqpoint{1.738408in}{2.386852in}}%
\pgfpathlineto{\pgfqpoint{1.767174in}{2.386852in}}%
\pgfpathlineto{\pgfqpoint{1.795939in}{2.386852in}}%
\pgfpathlineto{\pgfqpoint{1.824705in}{2.363083in}}%
\pgfpathlineto{\pgfqpoint{1.853470in}{2.363083in}}%
\pgfpathlineto{\pgfqpoint{1.882236in}{2.363083in}}%
\pgfpathlineto{\pgfqpoint{1.911001in}{2.354923in}}%
\pgfpathlineto{\pgfqpoint{1.939767in}{2.288020in}}%
\pgfpathlineto{\pgfqpoint{1.968532in}{2.288020in}}%
\pgfpathlineto{\pgfqpoint{1.997298in}{2.278459in}}%
\pgfpathlineto{\pgfqpoint{2.026063in}{2.278459in}}%
\pgfpathlineto{\pgfqpoint{2.054829in}{2.243202in}}%
\pgfpathlineto{\pgfqpoint{2.083594in}{2.243202in}}%
\pgfpathlineto{\pgfqpoint{2.112359in}{2.207864in}}%
\pgfpathlineto{\pgfqpoint{2.141125in}{2.145165in}}%
\pgfpathlineto{\pgfqpoint{2.169890in}{2.145165in}}%
\pgfpathlineto{\pgfqpoint{2.198656in}{2.145165in}}%
\pgfpathlineto{\pgfqpoint{2.227421in}{2.131999in}}%
\pgfpathlineto{\pgfqpoint{2.256187in}{2.019992in}}%
\pgfpathlineto{\pgfqpoint{2.284952in}{2.019992in}}%
\pgfpathlineto{\pgfqpoint{2.313718in}{2.019992in}}%
\pgfpathlineto{\pgfqpoint{2.342483in}{2.012128in}}%
\pgfpathlineto{\pgfqpoint{2.371249in}{1.943691in}}%
\pgfpathlineto{\pgfqpoint{2.400014in}{1.943691in}}%
\pgfpathlineto{\pgfqpoint{2.428780in}{1.943691in}}%
\pgfpathlineto{\pgfqpoint{2.457545in}{1.943691in}}%
\pgfpathlineto{\pgfqpoint{2.486311in}{1.943691in}}%
\pgfpathlineto{\pgfqpoint{2.515076in}{1.811732in}}%
\pgfpathlineto{\pgfqpoint{2.543841in}{1.757597in}}%
\pgfpathlineto{\pgfqpoint{2.572607in}{1.743146in}}%
\pgfpathlineto{\pgfqpoint{2.601372in}{1.621171in}}%
\pgfpathlineto{\pgfqpoint{2.630138in}{1.589312in}}%
\pgfpathlineto{\pgfqpoint{2.658903in}{1.546790in}}%
\pgfpathlineto{\pgfqpoint{2.687669in}{1.546790in}}%
\pgfpathlineto{\pgfqpoint{2.716434in}{1.546790in}}%
\pgfpathlineto{\pgfqpoint{2.745200in}{1.546790in}}%
\pgfpathlineto{\pgfqpoint{2.773965in}{1.546790in}}%
\pgfpathlineto{\pgfqpoint{2.802731in}{1.536071in}}%
\pgfpathlineto{\pgfqpoint{2.831496in}{1.524533in}}%
\pgfpathlineto{\pgfqpoint{2.860262in}{1.524533in}}%
\pgfpathlineto{\pgfqpoint{2.889027in}{1.524533in}}%
\pgfpathlineto{\pgfqpoint{2.917792in}{1.524533in}}%
\pgfpathlineto{\pgfqpoint{2.946558in}{1.439264in}}%
\pgfpathlineto{\pgfqpoint{2.975323in}{1.439264in}}%
\pgfpathlineto{\pgfqpoint{3.004089in}{1.385086in}}%
\pgfpathlineto{\pgfqpoint{3.032854in}{1.385086in}}%
\pgfpathlineto{\pgfqpoint{3.061620in}{1.385086in}}%
\pgfpathlineto{\pgfqpoint{3.090385in}{1.385086in}}%
\pgfpathlineto{\pgfqpoint{3.119151in}{1.385086in}}%
\pgfpathlineto{\pgfqpoint{3.147916in}{1.385086in}}%
\pgfpathlineto{\pgfqpoint{3.176682in}{1.342771in}}%
\pgfpathlineto{\pgfqpoint{3.205447in}{1.342771in}}%
\pgfpathlineto{\pgfqpoint{3.234213in}{1.342771in}}%
\pgfpathlineto{\pgfqpoint{3.262978in}{1.342771in}}%
\pgfpathlineto{\pgfqpoint{3.291744in}{1.342771in}}%
\pgfpathlineto{\pgfqpoint{3.320509in}{1.315047in}}%
\pgfpathlineto{\pgfqpoint{3.349274in}{1.315047in}}%
\pgfpathlineto{\pgfqpoint{3.378040in}{1.315047in}}%
\pgfpathlineto{\pgfqpoint{3.406805in}{1.290944in}}%
\pgfpathlineto{\pgfqpoint{3.435571in}{1.290944in}}%
\pgfpathlineto{\pgfqpoint{3.464336in}{1.290944in}}%
\pgfpathlineto{\pgfqpoint{3.493102in}{1.290944in}}%
\pgfpathlineto{\pgfqpoint{3.521867in}{1.290944in}}%
\pgfpathlineto{\pgfqpoint{3.550633in}{1.289022in}}%
\pgfpathlineto{\pgfqpoint{3.579398in}{1.289022in}}%
\pgfpathlineto{\pgfqpoint{3.608164in}{1.289022in}}%
\pgfpathlineto{\pgfqpoint{3.636929in}{1.289022in}}%
\pgfpathlineto{\pgfqpoint{3.665695in}{1.232052in}}%
\pgfpathlineto{\pgfqpoint{3.694460in}{1.232052in}}%
\pgfpathlineto{\pgfqpoint{3.723226in}{1.232052in}}%
\pgfpathlineto{\pgfqpoint{3.751991in}{1.232052in}}%
\pgfpathlineto{\pgfqpoint{3.780756in}{1.223142in}}%
\pgfpathlineto{\pgfqpoint{3.809522in}{1.223142in}}%
\pgfpathlineto{\pgfqpoint{3.838287in}{1.223142in}}%
\pgfpathlineto{\pgfqpoint{3.867053in}{1.223142in}}%
\pgfpathlineto{\pgfqpoint{3.895818in}{1.223142in}}%
\pgfpathlineto{\pgfqpoint{3.924584in}{1.207512in}}%
\pgfpathlineto{\pgfqpoint{3.953349in}{1.207512in}}%
\pgfpathlineto{\pgfqpoint{3.982115in}{1.207512in}}%
\pgfpathlineto{\pgfqpoint{4.010880in}{1.207512in}}%
\pgfpathlineto{\pgfqpoint{4.039646in}{1.207512in}}%
\pgfpathlineto{\pgfqpoint{4.068411in}{1.207512in}}%
\pgfpathlineto{\pgfqpoint{4.097177in}{1.200859in}}%
\pgfpathlineto{\pgfqpoint{4.125942in}{1.200859in}}%
\pgfpathlineto{\pgfqpoint{4.154708in}{1.200859in}}%
\pgfpathlineto{\pgfqpoint{4.183473in}{1.200859in}}%
\pgfpathlineto{\pgfqpoint{4.212238in}{1.200859in}}%
\pgfpathlineto{\pgfqpoint{4.241004in}{1.200859in}}%
\pgfpathlineto{\pgfqpoint{4.269769in}{1.200859in}}%
\pgfpathlineto{\pgfqpoint{4.298535in}{1.200859in}}%
\pgfpathlineto{\pgfqpoint{4.327300in}{1.200859in}}%
\pgfpathlineto{\pgfqpoint{4.356066in}{1.197066in}}%
\pgfpathlineto{\pgfqpoint{4.384831in}{1.197066in}}%
\pgfpathlineto{\pgfqpoint{4.413597in}{1.197066in}}%
\pgfpathlineto{\pgfqpoint{4.442362in}{1.192095in}}%
\pgfpathlineto{\pgfqpoint{4.471128in}{1.192095in}}%
\pgfpathlineto{\pgfqpoint{4.499893in}{1.192095in}}%
\pgfpathlineto{\pgfqpoint{4.528659in}{1.188666in}}%
\pgfpathlineto{\pgfqpoint{4.557424in}{1.174263in}}%
\pgfpathlineto{\pgfqpoint{4.586189in}{1.174263in}}%
\pgfpathlineto{\pgfqpoint{4.614955in}{1.174263in}}%
\pgfpathlineto{\pgfqpoint{4.643720in}{1.174263in}}%
\pgfpathlineto{\pgfqpoint{4.672486in}{1.174263in}}%
\pgfpathlineto{\pgfqpoint{4.701251in}{1.174263in}}%
\pgfpathlineto{\pgfqpoint{4.730017in}{1.166482in}}%
\pgfpathlineto{\pgfqpoint{4.758782in}{1.155977in}}%
\pgfpathlineto{\pgfqpoint{4.787548in}{1.155977in}}%
\pgfpathlineto{\pgfqpoint{4.816313in}{1.155977in}}%
\pgfpathlineto{\pgfqpoint{4.845079in}{1.147136in}}%
\pgfpathlineto{\pgfqpoint{4.873844in}{1.133806in}}%
\pgfpathlineto{\pgfqpoint{4.902610in}{1.133806in}}%
\pgfpathlineto{\pgfqpoint{4.931375in}{1.133806in}}%
\pgfpathlineto{\pgfqpoint{4.960141in}{1.133806in}}%
\pgfpathlineto{\pgfqpoint{4.988906in}{1.133806in}}%
\pgfpathlineto{\pgfqpoint{5.017671in}{1.130155in}}%
\pgfpathlineto{\pgfqpoint{5.046437in}{1.130155in}}%
\pgfpathlineto{\pgfqpoint{5.075202in}{1.130155in}}%
\pgfpathlineto{\pgfqpoint{5.103968in}{1.130155in}}%
\pgfpathlineto{\pgfqpoint{5.132733in}{1.130155in}}%
\pgfpathlineto{\pgfqpoint{5.161499in}{1.130155in}}%
\pgfpathlineto{\pgfqpoint{5.190264in}{1.130155in}}%
\pgfpathlineto{\pgfqpoint{5.219030in}{1.130155in}}%
\pgfpathlineto{\pgfqpoint{5.247795in}{1.125527in}}%
\pgfpathlineto{\pgfqpoint{5.276561in}{1.125527in}}%
\pgfpathlineto{\pgfqpoint{5.305326in}{1.091530in}}%
\pgfpathlineto{\pgfqpoint{5.334092in}{1.091530in}}%
\pgfpathlineto{\pgfqpoint{5.362857in}{1.086510in}}%
\pgfpathlineto{\pgfqpoint{5.391623in}{1.086510in}}%
\pgfpathlineto{\pgfqpoint{5.420388in}{1.086510in}}%
\pgfpathlineto{\pgfqpoint{5.449153in}{1.086510in}}%
\pgfpathlineto{\pgfqpoint{5.477919in}{1.084058in}}%
\pgfpathlineto{\pgfqpoint{5.506684in}{1.084058in}}%
\pgfpathlineto{\pgfqpoint{5.535450in}{1.084058in}}%
\pgfpathlineto{\pgfqpoint{5.564215in}{1.084058in}}%
\pgfpathlineto{\pgfqpoint{5.592981in}{1.084058in}}%
\pgfpathlineto{\pgfqpoint{5.621746in}{1.078210in}}%
\pgfpathlineto{\pgfqpoint{5.650512in}{1.078210in}}%
\pgfpathlineto{\pgfqpoint{5.679277in}{1.078210in}}%
\pgfpathlineto{\pgfqpoint{5.708043in}{1.077690in}}%
\pgfpathlineto{\pgfqpoint{5.736808in}{1.077690in}}%
\pgfpathlineto{\pgfqpoint{5.765574in}{1.077690in}}%
\pgfpathlineto{\pgfqpoint{5.794339in}{1.077690in}}%
\pgfpathlineto{\pgfqpoint{5.823104in}{1.077690in}}%
\pgfpathlineto{\pgfqpoint{5.851870in}{1.077690in}}%
\pgfpathlineto{\pgfqpoint{5.880635in}{1.077690in}}%
\pgfpathlineto{\pgfqpoint{5.909401in}{1.077690in}}%
\pgfpathlineto{\pgfqpoint{5.938166in}{1.077690in}}%
\pgfpathlineto{\pgfqpoint{5.966932in}{1.077690in}}%
\pgfpathlineto{\pgfqpoint{5.966932in}{0.705689in}}%
\pgfpathlineto{\pgfqpoint{5.966932in}{0.705689in}}%
\pgfpathlineto{\pgfqpoint{5.938166in}{0.705689in}}%
\pgfpathlineto{\pgfqpoint{5.909401in}{0.705689in}}%
\pgfpathlineto{\pgfqpoint{5.880635in}{0.705689in}}%
\pgfpathlineto{\pgfqpoint{5.851870in}{0.705689in}}%
\pgfpathlineto{\pgfqpoint{5.823104in}{0.705689in}}%
\pgfpathlineto{\pgfqpoint{5.794339in}{0.705689in}}%
\pgfpathlineto{\pgfqpoint{5.765574in}{0.705689in}}%
\pgfpathlineto{\pgfqpoint{5.736808in}{0.705689in}}%
\pgfpathlineto{\pgfqpoint{5.708043in}{0.705689in}}%
\pgfpathlineto{\pgfqpoint{5.679277in}{0.707627in}}%
\pgfpathlineto{\pgfqpoint{5.650512in}{0.707627in}}%
\pgfpathlineto{\pgfqpoint{5.621746in}{0.707627in}}%
\pgfpathlineto{\pgfqpoint{5.592981in}{0.744411in}}%
\pgfpathlineto{\pgfqpoint{5.564215in}{0.744411in}}%
\pgfpathlineto{\pgfqpoint{5.535450in}{0.744411in}}%
\pgfpathlineto{\pgfqpoint{5.506684in}{0.744411in}}%
\pgfpathlineto{\pgfqpoint{5.477919in}{0.744411in}}%
\pgfpathlineto{\pgfqpoint{5.449153in}{0.757427in}}%
\pgfpathlineto{\pgfqpoint{5.420388in}{0.757427in}}%
\pgfpathlineto{\pgfqpoint{5.391623in}{0.757427in}}%
\pgfpathlineto{\pgfqpoint{5.362857in}{0.757427in}}%
\pgfpathlineto{\pgfqpoint{5.334092in}{0.787464in}}%
\pgfpathlineto{\pgfqpoint{5.305326in}{0.787464in}}%
\pgfpathlineto{\pgfqpoint{5.276561in}{0.835116in}}%
\pgfpathlineto{\pgfqpoint{5.247795in}{0.835116in}}%
\pgfpathlineto{\pgfqpoint{5.219030in}{0.859117in}}%
\pgfpathlineto{\pgfqpoint{5.190264in}{0.859117in}}%
\pgfpathlineto{\pgfqpoint{5.161499in}{0.859117in}}%
\pgfpathlineto{\pgfqpoint{5.132733in}{0.859117in}}%
\pgfpathlineto{\pgfqpoint{5.103968in}{0.859117in}}%
\pgfpathlineto{\pgfqpoint{5.075202in}{0.859117in}}%
\pgfpathlineto{\pgfqpoint{5.046437in}{0.859117in}}%
\pgfpathlineto{\pgfqpoint{5.017671in}{0.859117in}}%
\pgfpathlineto{\pgfqpoint{4.988906in}{0.869403in}}%
\pgfpathlineto{\pgfqpoint{4.960141in}{0.869403in}}%
\pgfpathlineto{\pgfqpoint{4.931375in}{0.869403in}}%
\pgfpathlineto{\pgfqpoint{4.902610in}{0.869403in}}%
\pgfpathlineto{\pgfqpoint{4.873844in}{0.869403in}}%
\pgfpathlineto{\pgfqpoint{4.845079in}{0.875746in}}%
\pgfpathlineto{\pgfqpoint{4.816313in}{0.919729in}}%
\pgfpathlineto{\pgfqpoint{4.787548in}{0.919729in}}%
\pgfpathlineto{\pgfqpoint{4.758782in}{0.919729in}}%
\pgfpathlineto{\pgfqpoint{4.730017in}{0.945791in}}%
\pgfpathlineto{\pgfqpoint{4.701251in}{0.950406in}}%
\pgfpathlineto{\pgfqpoint{4.672486in}{0.950406in}}%
\pgfpathlineto{\pgfqpoint{4.643720in}{0.950406in}}%
\pgfpathlineto{\pgfqpoint{4.614955in}{0.950406in}}%
\pgfpathlineto{\pgfqpoint{4.586189in}{0.950406in}}%
\pgfpathlineto{\pgfqpoint{4.557424in}{0.950406in}}%
\pgfpathlineto{\pgfqpoint{4.528659in}{0.956446in}}%
\pgfpathlineto{\pgfqpoint{4.499893in}{0.970560in}}%
\pgfpathlineto{\pgfqpoint{4.471128in}{0.970560in}}%
\pgfpathlineto{\pgfqpoint{4.442362in}{0.970560in}}%
\pgfpathlineto{\pgfqpoint{4.413597in}{0.998065in}}%
\pgfpathlineto{\pgfqpoint{4.384831in}{0.998065in}}%
\pgfpathlineto{\pgfqpoint{4.356066in}{0.998065in}}%
\pgfpathlineto{\pgfqpoint{4.327300in}{1.012623in}}%
\pgfpathlineto{\pgfqpoint{4.298535in}{1.012623in}}%
\pgfpathlineto{\pgfqpoint{4.269769in}{1.012623in}}%
\pgfpathlineto{\pgfqpoint{4.241004in}{1.012623in}}%
\pgfpathlineto{\pgfqpoint{4.212238in}{1.012623in}}%
\pgfpathlineto{\pgfqpoint{4.183473in}{1.012623in}}%
\pgfpathlineto{\pgfqpoint{4.154708in}{1.012623in}}%
\pgfpathlineto{\pgfqpoint{4.125942in}{1.012623in}}%
\pgfpathlineto{\pgfqpoint{4.097177in}{1.012623in}}%
\pgfpathlineto{\pgfqpoint{4.068411in}{1.027540in}}%
\pgfpathlineto{\pgfqpoint{4.039646in}{1.027540in}}%
\pgfpathlineto{\pgfqpoint{4.010880in}{1.027540in}}%
\pgfpathlineto{\pgfqpoint{3.982115in}{1.027540in}}%
\pgfpathlineto{\pgfqpoint{3.953349in}{1.027540in}}%
\pgfpathlineto{\pgfqpoint{3.924584in}{1.027540in}}%
\pgfpathlineto{\pgfqpoint{3.895818in}{1.069284in}}%
\pgfpathlineto{\pgfqpoint{3.867053in}{1.069284in}}%
\pgfpathlineto{\pgfqpoint{3.838287in}{1.069284in}}%
\pgfpathlineto{\pgfqpoint{3.809522in}{1.069284in}}%
\pgfpathlineto{\pgfqpoint{3.780756in}{1.069284in}}%
\pgfpathlineto{\pgfqpoint{3.751991in}{1.074417in}}%
\pgfpathlineto{\pgfqpoint{3.723226in}{1.074417in}}%
\pgfpathlineto{\pgfqpoint{3.694460in}{1.074417in}}%
\pgfpathlineto{\pgfqpoint{3.665695in}{1.074417in}}%
\pgfpathlineto{\pgfqpoint{3.636929in}{1.096054in}}%
\pgfpathlineto{\pgfqpoint{3.608164in}{1.096054in}}%
\pgfpathlineto{\pgfqpoint{3.579398in}{1.096054in}}%
\pgfpathlineto{\pgfqpoint{3.550633in}{1.096054in}}%
\pgfpathlineto{\pgfqpoint{3.521867in}{1.096585in}}%
\pgfpathlineto{\pgfqpoint{3.493102in}{1.096585in}}%
\pgfpathlineto{\pgfqpoint{3.464336in}{1.096585in}}%
\pgfpathlineto{\pgfqpoint{3.435571in}{1.096585in}}%
\pgfpathlineto{\pgfqpoint{3.406805in}{1.096585in}}%
\pgfpathlineto{\pgfqpoint{3.378040in}{1.112410in}}%
\pgfpathlineto{\pgfqpoint{3.349274in}{1.112410in}}%
\pgfpathlineto{\pgfqpoint{3.320509in}{1.112410in}}%
\pgfpathlineto{\pgfqpoint{3.291744in}{1.166687in}}%
\pgfpathlineto{\pgfqpoint{3.262978in}{1.166687in}}%
\pgfpathlineto{\pgfqpoint{3.234213in}{1.166687in}}%
\pgfpathlineto{\pgfqpoint{3.205447in}{1.166687in}}%
\pgfpathlineto{\pgfqpoint{3.176682in}{1.166687in}}%
\pgfpathlineto{\pgfqpoint{3.147916in}{1.186865in}}%
\pgfpathlineto{\pgfqpoint{3.119151in}{1.186865in}}%
\pgfpathlineto{\pgfqpoint{3.090385in}{1.186865in}}%
\pgfpathlineto{\pgfqpoint{3.061620in}{1.186865in}}%
\pgfpathlineto{\pgfqpoint{3.032854in}{1.186865in}}%
\pgfpathlineto{\pgfqpoint{3.004089in}{1.186865in}}%
\pgfpathlineto{\pgfqpoint{2.975323in}{1.271187in}}%
\pgfpathlineto{\pgfqpoint{2.946558in}{1.271187in}}%
\pgfpathlineto{\pgfqpoint{2.917792in}{1.423391in}}%
\pgfpathlineto{\pgfqpoint{2.889027in}{1.423391in}}%
\pgfpathlineto{\pgfqpoint{2.860262in}{1.423391in}}%
\pgfpathlineto{\pgfqpoint{2.831496in}{1.423391in}}%
\pgfpathlineto{\pgfqpoint{2.802731in}{1.438890in}}%
\pgfpathlineto{\pgfqpoint{2.773965in}{1.455470in}}%
\pgfpathlineto{\pgfqpoint{2.745200in}{1.455470in}}%
\pgfpathlineto{\pgfqpoint{2.716434in}{1.455470in}}%
\pgfpathlineto{\pgfqpoint{2.687669in}{1.455470in}}%
\pgfpathlineto{\pgfqpoint{2.658903in}{1.455470in}}%
\pgfpathlineto{\pgfqpoint{2.630138in}{1.482342in}}%
\pgfpathlineto{\pgfqpoint{2.601372in}{1.565306in}}%
\pgfpathlineto{\pgfqpoint{2.572607in}{1.587123in}}%
\pgfpathlineto{\pgfqpoint{2.543841in}{1.605434in}}%
\pgfpathlineto{\pgfqpoint{2.515076in}{1.683181in}}%
\pgfpathlineto{\pgfqpoint{2.486311in}{1.730023in}}%
\pgfpathlineto{\pgfqpoint{2.457545in}{1.730023in}}%
\pgfpathlineto{\pgfqpoint{2.428780in}{1.730023in}}%
\pgfpathlineto{\pgfqpoint{2.400014in}{1.730023in}}%
\pgfpathlineto{\pgfqpoint{2.371249in}{1.730023in}}%
\pgfpathlineto{\pgfqpoint{2.342483in}{1.842308in}}%
\pgfpathlineto{\pgfqpoint{2.313718in}{1.891886in}}%
\pgfpathlineto{\pgfqpoint{2.284952in}{1.891886in}}%
\pgfpathlineto{\pgfqpoint{2.256187in}{1.891886in}}%
\pgfpathlineto{\pgfqpoint{2.227421in}{1.910229in}}%
\pgfpathlineto{\pgfqpoint{2.198656in}{1.932370in}}%
\pgfpathlineto{\pgfqpoint{2.169890in}{1.932370in}}%
\pgfpathlineto{\pgfqpoint{2.141125in}{1.932370in}}%
\pgfpathlineto{\pgfqpoint{2.112359in}{2.044884in}}%
\pgfpathlineto{\pgfqpoint{2.083594in}{2.052163in}}%
\pgfpathlineto{\pgfqpoint{2.054829in}{2.052163in}}%
\pgfpathlineto{\pgfqpoint{2.026063in}{2.105073in}}%
\pgfpathlineto{\pgfqpoint{1.997298in}{2.105073in}}%
\pgfpathlineto{\pgfqpoint{1.968532in}{2.107490in}}%
\pgfpathlineto{\pgfqpoint{1.939767in}{2.107490in}}%
\pgfpathlineto{\pgfqpoint{1.911001in}{2.186075in}}%
\pgfpathlineto{\pgfqpoint{1.882236in}{2.199228in}}%
\pgfpathlineto{\pgfqpoint{1.853470in}{2.199228in}}%
\pgfpathlineto{\pgfqpoint{1.824705in}{2.199228in}}%
\pgfpathlineto{\pgfqpoint{1.795939in}{2.279181in}}%
\pgfpathlineto{\pgfqpoint{1.767174in}{2.279181in}}%
\pgfpathlineto{\pgfqpoint{1.738408in}{2.279181in}}%
\pgfpathlineto{\pgfqpoint{1.709643in}{2.282090in}}%
\pgfpathlineto{\pgfqpoint{1.680877in}{2.282090in}}%
\pgfpathlineto{\pgfqpoint{1.652112in}{2.282090in}}%
\pgfpathlineto{\pgfqpoint{1.623347in}{2.282090in}}%
\pgfpathlineto{\pgfqpoint{1.594581in}{2.282090in}}%
\pgfpathlineto{\pgfqpoint{1.565816in}{2.282090in}}%
\pgfpathlineto{\pgfqpoint{1.537050in}{2.282090in}}%
\pgfpathlineto{\pgfqpoint{1.508285in}{2.372917in}}%
\pgfpathlineto{\pgfqpoint{1.479519in}{2.372917in}}%
\pgfpathlineto{\pgfqpoint{1.450754in}{2.372917in}}%
\pgfpathlineto{\pgfqpoint{1.421988in}{2.372917in}}%
\pgfpathlineto{\pgfqpoint{1.393223in}{2.372917in}}%
\pgfpathlineto{\pgfqpoint{1.364457in}{2.377603in}}%
\pgfpathlineto{\pgfqpoint{1.335692in}{2.377603in}}%
\pgfpathlineto{\pgfqpoint{1.306926in}{2.377603in}}%
\pgfpathlineto{\pgfqpoint{1.278161in}{2.377603in}}%
\pgfpathlineto{\pgfqpoint{1.249396in}{2.377603in}}%
\pgfpathlineto{\pgfqpoint{1.220630in}{2.378889in}}%
\pgfpathlineto{\pgfqpoint{1.191865in}{2.432344in}}%
\pgfpathlineto{\pgfqpoint{1.163099in}{2.445893in}}%
\pgfpathlineto{\pgfqpoint{1.134334in}{2.445893in}}%
\pgfpathlineto{\pgfqpoint{1.105568in}{2.513232in}}%
\pgfpathclose%
\pgfusepath{fill}%
\end{pgfscope}%
\begin{pgfscope}%
\pgfpathrectangle{\pgfqpoint{0.862500in}{0.375000in}}{\pgfqpoint{5.347500in}{2.265000in}}%
\pgfusepath{clip}%
\pgfsetroundcap%
\pgfsetroundjoin%
\pgfsetlinewidth{1.505625pt}%
\definecolor{currentstroke}{rgb}{0.121569,0.466667,0.705882}%
\pgfsetstrokecolor{currentstroke}%
\pgfsetdash{}{0pt}%
\pgfpathmoveto{\pgfqpoint{1.105568in}{2.504016in}}%
\pgfpathlineto{\pgfqpoint{1.134334in}{2.503814in}}%
\pgfpathlineto{\pgfqpoint{1.163099in}{2.482861in}}%
\pgfpathlineto{\pgfqpoint{1.191865in}{2.482861in}}%
\pgfpathlineto{\pgfqpoint{1.220630in}{2.465220in}}%
\pgfpathlineto{\pgfqpoint{1.249396in}{2.454132in}}%
\pgfpathlineto{\pgfqpoint{1.278161in}{2.425858in}}%
\pgfpathlineto{\pgfqpoint{1.306926in}{2.418429in}}%
\pgfpathlineto{\pgfqpoint{1.364457in}{2.418127in}}%
\pgfpathlineto{\pgfqpoint{1.393223in}{2.393315in}}%
\pgfpathlineto{\pgfqpoint{1.421988in}{2.378911in}}%
\pgfpathlineto{\pgfqpoint{1.450754in}{2.358781in}}%
\pgfpathlineto{\pgfqpoint{1.479519in}{2.355474in}}%
\pgfpathlineto{\pgfqpoint{1.508285in}{2.319956in}}%
\pgfpathlineto{\pgfqpoint{1.537050in}{2.312877in}}%
\pgfpathlineto{\pgfqpoint{1.594581in}{2.312877in}}%
\pgfpathlineto{\pgfqpoint{1.623347in}{2.273675in}}%
\pgfpathlineto{\pgfqpoint{1.680877in}{2.273675in}}%
\pgfpathlineto{\pgfqpoint{1.709643in}{2.241631in}}%
\pgfpathlineto{\pgfqpoint{1.911001in}{2.241631in}}%
\pgfpathlineto{\pgfqpoint{1.939767in}{2.236520in}}%
\pgfpathlineto{\pgfqpoint{1.968532in}{2.225341in}}%
\pgfpathlineto{\pgfqpoint{2.054829in}{2.225341in}}%
\pgfpathlineto{\pgfqpoint{2.083594in}{2.182548in}}%
\pgfpathlineto{\pgfqpoint{2.169890in}{2.182548in}}%
\pgfpathlineto{\pgfqpoint{2.198656in}{2.180197in}}%
\pgfpathlineto{\pgfqpoint{2.428780in}{2.180197in}}%
\pgfpathlineto{\pgfqpoint{2.457545in}{2.155773in}}%
\pgfpathlineto{\pgfqpoint{2.687669in}{2.155773in}}%
\pgfpathlineto{\pgfqpoint{2.716434in}{2.120087in}}%
\pgfpathlineto{\pgfqpoint{2.917792in}{2.120087in}}%
\pgfpathlineto{\pgfqpoint{2.946558in}{2.102802in}}%
\pgfpathlineto{\pgfqpoint{3.378040in}{2.102802in}}%
\pgfpathlineto{\pgfqpoint{3.406805in}{2.090004in}}%
\pgfpathlineto{\pgfqpoint{3.435571in}{2.090004in}}%
\pgfpathlineto{\pgfqpoint{3.464336in}{2.085992in}}%
\pgfpathlineto{\pgfqpoint{4.586189in}{2.085992in}}%
\pgfpathlineto{\pgfqpoint{4.614955in}{2.056821in}}%
\pgfpathlineto{\pgfqpoint{5.190264in}{2.056821in}}%
\pgfpathlineto{\pgfqpoint{5.219030in}{2.045320in}}%
\pgfpathlineto{\pgfqpoint{5.276561in}{2.045320in}}%
\pgfpathlineto{\pgfqpoint{5.305326in}{2.042338in}}%
\pgfpathlineto{\pgfqpoint{5.966932in}{2.042338in}}%
\pgfpathlineto{\pgfqpoint{5.966932in}{2.042338in}}%
\pgfusepath{stroke}%
\end{pgfscope}%
\begin{pgfscope}%
\pgfpathrectangle{\pgfqpoint{0.862500in}{0.375000in}}{\pgfqpoint{5.347500in}{2.265000in}}%
\pgfusepath{clip}%
\pgfsetroundcap%
\pgfsetroundjoin%
\pgfsetlinewidth{1.505625pt}%
\definecolor{currentstroke}{rgb}{1.000000,0.498039,0.054902}%
\pgfsetstrokecolor{currentstroke}%
\pgfsetdash{}{0pt}%
\pgfpathmoveto{\pgfqpoint{1.105568in}{2.502713in}}%
\pgfpathlineto{\pgfqpoint{1.134334in}{2.453516in}}%
\pgfpathlineto{\pgfqpoint{1.163099in}{2.446075in}}%
\pgfpathlineto{\pgfqpoint{1.191865in}{2.444448in}}%
\pgfpathlineto{\pgfqpoint{1.220630in}{2.444448in}}%
\pgfpathlineto{\pgfqpoint{1.249396in}{2.441082in}}%
\pgfpathlineto{\pgfqpoint{1.278161in}{2.428411in}}%
\pgfpathlineto{\pgfqpoint{1.306926in}{2.397523in}}%
\pgfpathlineto{\pgfqpoint{1.393223in}{2.397523in}}%
\pgfpathlineto{\pgfqpoint{1.421988in}{2.393353in}}%
\pgfpathlineto{\pgfqpoint{1.450754in}{2.387290in}}%
\pgfpathlineto{\pgfqpoint{1.479519in}{2.387290in}}%
\pgfpathlineto{\pgfqpoint{1.508285in}{2.374976in}}%
\pgfpathlineto{\pgfqpoint{1.537050in}{2.333906in}}%
\pgfpathlineto{\pgfqpoint{1.738408in}{2.333906in}}%
\pgfpathlineto{\pgfqpoint{1.767174in}{2.311223in}}%
\pgfpathlineto{\pgfqpoint{1.795939in}{2.205947in}}%
\pgfpathlineto{\pgfqpoint{1.824705in}{2.195014in}}%
\pgfpathlineto{\pgfqpoint{1.853470in}{2.143805in}}%
\pgfpathlineto{\pgfqpoint{1.882236in}{2.136246in}}%
\pgfpathlineto{\pgfqpoint{1.911001in}{2.104561in}}%
\pgfpathlineto{\pgfqpoint{1.939767in}{2.101765in}}%
\pgfpathlineto{\pgfqpoint{1.968532in}{2.093625in}}%
\pgfpathlineto{\pgfqpoint{1.997298in}{2.093625in}}%
\pgfpathlineto{\pgfqpoint{2.054829in}{1.997494in}}%
\pgfpathlineto{\pgfqpoint{2.083594in}{1.963502in}}%
\pgfpathlineto{\pgfqpoint{2.112359in}{1.950959in}}%
\pgfpathlineto{\pgfqpoint{2.141125in}{1.893183in}}%
\pgfpathlineto{\pgfqpoint{2.198656in}{1.715730in}}%
\pgfpathlineto{\pgfqpoint{2.284952in}{1.714512in}}%
\pgfpathlineto{\pgfqpoint{2.313718in}{1.714115in}}%
\pgfpathlineto{\pgfqpoint{2.342483in}{1.557423in}}%
\pgfpathlineto{\pgfqpoint{2.400014in}{1.557423in}}%
\pgfpathlineto{\pgfqpoint{2.428780in}{1.527357in}}%
\pgfpathlineto{\pgfqpoint{2.457545in}{1.523984in}}%
\pgfpathlineto{\pgfqpoint{2.515076in}{1.523984in}}%
\pgfpathlineto{\pgfqpoint{2.543841in}{1.479603in}}%
\pgfpathlineto{\pgfqpoint{2.572607in}{1.472262in}}%
\pgfpathlineto{\pgfqpoint{2.601372in}{1.472262in}}%
\pgfpathlineto{\pgfqpoint{2.630138in}{1.454928in}}%
\pgfpathlineto{\pgfqpoint{2.687669in}{1.454928in}}%
\pgfpathlineto{\pgfqpoint{2.716434in}{1.435254in}}%
\pgfpathlineto{\pgfqpoint{2.773965in}{1.435254in}}%
\pgfpathlineto{\pgfqpoint{2.802731in}{1.389787in}}%
\pgfpathlineto{\pgfqpoint{2.831496in}{1.389787in}}%
\pgfpathlineto{\pgfqpoint{2.860262in}{1.382701in}}%
\pgfpathlineto{\pgfqpoint{2.889027in}{1.377394in}}%
\pgfpathlineto{\pgfqpoint{2.917792in}{1.344900in}}%
\pgfpathlineto{\pgfqpoint{3.032854in}{1.344200in}}%
\pgfpathlineto{\pgfqpoint{3.061620in}{1.296893in}}%
\pgfpathlineto{\pgfqpoint{3.090385in}{1.270904in}}%
\pgfpathlineto{\pgfqpoint{3.119151in}{1.270904in}}%
\pgfpathlineto{\pgfqpoint{3.147916in}{1.246550in}}%
\pgfpathlineto{\pgfqpoint{3.378040in}{1.246550in}}%
\pgfpathlineto{\pgfqpoint{3.406805in}{1.216273in}}%
\pgfpathlineto{\pgfqpoint{3.435571in}{1.216273in}}%
\pgfpathlineto{\pgfqpoint{3.464336in}{1.203467in}}%
\pgfpathlineto{\pgfqpoint{3.521867in}{1.202290in}}%
\pgfpathlineto{\pgfqpoint{3.550633in}{1.155311in}}%
\pgfpathlineto{\pgfqpoint{3.579398in}{1.150972in}}%
\pgfpathlineto{\pgfqpoint{3.809522in}{1.150972in}}%
\pgfpathlineto{\pgfqpoint{3.838287in}{1.098234in}}%
\pgfpathlineto{\pgfqpoint{4.097177in}{1.098234in}}%
\pgfpathlineto{\pgfqpoint{4.125942in}{1.042260in}}%
\pgfpathlineto{\pgfqpoint{4.154708in}{1.040417in}}%
\pgfpathlineto{\pgfqpoint{4.614955in}{1.040417in}}%
\pgfpathlineto{\pgfqpoint{4.643720in}{1.024286in}}%
\pgfpathlineto{\pgfqpoint{5.736808in}{1.024286in}}%
\pgfpathlineto{\pgfqpoint{5.765574in}{0.985057in}}%
\pgfpathlineto{\pgfqpoint{5.966932in}{0.985057in}}%
\pgfpathlineto{\pgfqpoint{5.966932in}{0.985057in}}%
\pgfusepath{stroke}%
\end{pgfscope}%
\begin{pgfscope}%
\pgfpathrectangle{\pgfqpoint{0.862500in}{0.375000in}}{\pgfqpoint{5.347500in}{2.265000in}}%
\pgfusepath{clip}%
\pgfsetroundcap%
\pgfsetroundjoin%
\pgfsetlinewidth{1.505625pt}%
\definecolor{currentstroke}{rgb}{0.172549,0.627451,0.172549}%
\pgfsetstrokecolor{currentstroke}%
\pgfsetdash{}{0pt}%
\pgfpathmoveto{\pgfqpoint{1.105568in}{2.524625in}}%
\pgfpathlineto{\pgfqpoint{1.134334in}{2.524625in}}%
\pgfpathlineto{\pgfqpoint{1.163099in}{2.495367in}}%
\pgfpathlineto{\pgfqpoint{1.191865in}{2.489264in}}%
\pgfpathlineto{\pgfqpoint{1.220630in}{2.416627in}}%
\pgfpathlineto{\pgfqpoint{1.249396in}{2.391375in}}%
\pgfpathlineto{\pgfqpoint{1.306926in}{2.391375in}}%
\pgfpathlineto{\pgfqpoint{1.335692in}{2.386963in}}%
\pgfpathlineto{\pgfqpoint{1.450754in}{2.386963in}}%
\pgfpathlineto{\pgfqpoint{1.479519in}{2.355549in}}%
\pgfpathlineto{\pgfqpoint{1.508285in}{2.313793in}}%
\pgfpathlineto{\pgfqpoint{1.537050in}{2.313793in}}%
\pgfpathlineto{\pgfqpoint{1.565816in}{2.216523in}}%
\pgfpathlineto{\pgfqpoint{1.594581in}{2.214259in}}%
\pgfpathlineto{\pgfqpoint{1.709643in}{2.214259in}}%
\pgfpathlineto{\pgfqpoint{1.738408in}{2.211015in}}%
\pgfpathlineto{\pgfqpoint{1.824705in}{2.211015in}}%
\pgfpathlineto{\pgfqpoint{1.853470in}{2.156968in}}%
\pgfpathlineto{\pgfqpoint{1.911001in}{2.155565in}}%
\pgfpathlineto{\pgfqpoint{1.939767in}{2.155565in}}%
\pgfpathlineto{\pgfqpoint{1.968532in}{2.092341in}}%
\pgfpathlineto{\pgfqpoint{2.026063in}{2.092341in}}%
\pgfpathlineto{\pgfqpoint{2.054829in}{2.088242in}}%
\pgfpathlineto{\pgfqpoint{2.083594in}{2.033144in}}%
\pgfpathlineto{\pgfqpoint{2.112359in}{1.961030in}}%
\pgfpathlineto{\pgfqpoint{2.141125in}{1.931546in}}%
\pgfpathlineto{\pgfqpoint{2.169890in}{1.893932in}}%
\pgfpathlineto{\pgfqpoint{2.198656in}{1.893932in}}%
\pgfpathlineto{\pgfqpoint{2.227421in}{1.844756in}}%
\pgfpathlineto{\pgfqpoint{2.256187in}{1.844756in}}%
\pgfpathlineto{\pgfqpoint{2.284952in}{1.806196in}}%
\pgfpathlineto{\pgfqpoint{2.313718in}{1.806196in}}%
\pgfpathlineto{\pgfqpoint{2.342483in}{1.775347in}}%
\pgfpathlineto{\pgfqpoint{2.371249in}{1.736204in}}%
\pgfpathlineto{\pgfqpoint{2.400014in}{1.683229in}}%
\pgfpathlineto{\pgfqpoint{2.428780in}{1.683229in}}%
\pgfpathlineto{\pgfqpoint{2.457545in}{1.678439in}}%
\pgfpathlineto{\pgfqpoint{2.515076in}{1.678439in}}%
\pgfpathlineto{\pgfqpoint{2.543841in}{1.675029in}}%
\pgfpathlineto{\pgfqpoint{2.572607in}{1.675029in}}%
\pgfpathlineto{\pgfqpoint{2.601372in}{1.642746in}}%
\pgfpathlineto{\pgfqpoint{2.630138in}{1.634188in}}%
\pgfpathlineto{\pgfqpoint{2.658903in}{1.629473in}}%
\pgfpathlineto{\pgfqpoint{2.687669in}{1.629473in}}%
\pgfpathlineto{\pgfqpoint{2.716434in}{1.605530in}}%
\pgfpathlineto{\pgfqpoint{2.745200in}{1.545413in}}%
\pgfpathlineto{\pgfqpoint{2.773965in}{1.493301in}}%
\pgfpathlineto{\pgfqpoint{2.802731in}{1.493301in}}%
\pgfpathlineto{\pgfqpoint{2.831496in}{1.491053in}}%
\pgfpathlineto{\pgfqpoint{2.889027in}{1.357036in}}%
\pgfpathlineto{\pgfqpoint{2.917792in}{1.348911in}}%
\pgfpathlineto{\pgfqpoint{2.946558in}{1.348911in}}%
\pgfpathlineto{\pgfqpoint{2.975323in}{1.321806in}}%
\pgfpathlineto{\pgfqpoint{3.032854in}{1.321806in}}%
\pgfpathlineto{\pgfqpoint{3.061620in}{1.282216in}}%
\pgfpathlineto{\pgfqpoint{3.090385in}{1.258173in}}%
\pgfpathlineto{\pgfqpoint{3.119151in}{1.258173in}}%
\pgfpathlineto{\pgfqpoint{3.147916in}{1.210494in}}%
\pgfpathlineto{\pgfqpoint{3.234213in}{1.210494in}}%
\pgfpathlineto{\pgfqpoint{3.262978in}{1.167845in}}%
\pgfpathlineto{\pgfqpoint{3.349274in}{1.167845in}}%
\pgfpathlineto{\pgfqpoint{3.378040in}{1.146618in}}%
\pgfpathlineto{\pgfqpoint{3.406805in}{1.135991in}}%
\pgfpathlineto{\pgfqpoint{3.435571in}{1.135991in}}%
\pgfpathlineto{\pgfqpoint{3.464336in}{1.133537in}}%
\pgfpathlineto{\pgfqpoint{3.493102in}{1.133537in}}%
\pgfpathlineto{\pgfqpoint{3.521867in}{1.079373in}}%
\pgfpathlineto{\pgfqpoint{3.636929in}{1.079373in}}%
\pgfpathlineto{\pgfqpoint{3.665695in}{1.067972in}}%
\pgfpathlineto{\pgfqpoint{3.694460in}{1.067972in}}%
\pgfpathlineto{\pgfqpoint{3.723226in}{1.054514in}}%
\pgfpathlineto{\pgfqpoint{3.751991in}{1.009087in}}%
\pgfpathlineto{\pgfqpoint{3.780756in}{0.983005in}}%
\pgfpathlineto{\pgfqpoint{3.867053in}{0.983005in}}%
\pgfpathlineto{\pgfqpoint{3.895818in}{0.969419in}}%
\pgfpathlineto{\pgfqpoint{4.097177in}{0.969419in}}%
\pgfpathlineto{\pgfqpoint{4.125942in}{0.967306in}}%
\pgfpathlineto{\pgfqpoint{4.154708in}{0.952982in}}%
\pgfpathlineto{\pgfqpoint{4.356066in}{0.951617in}}%
\pgfpathlineto{\pgfqpoint{4.384831in}{0.916098in}}%
\pgfpathlineto{\pgfqpoint{4.442362in}{0.916098in}}%
\pgfpathlineto{\pgfqpoint{4.471128in}{0.909017in}}%
\pgfpathlineto{\pgfqpoint{4.499893in}{0.909017in}}%
\pgfpathlineto{\pgfqpoint{4.528659in}{0.900500in}}%
\pgfpathlineto{\pgfqpoint{5.046437in}{0.899222in}}%
\pgfpathlineto{\pgfqpoint{5.075202in}{0.887280in}}%
\pgfpathlineto{\pgfqpoint{5.276561in}{0.887280in}}%
\pgfpathlineto{\pgfqpoint{5.305326in}{0.867714in}}%
\pgfpathlineto{\pgfqpoint{5.334092in}{0.867714in}}%
\pgfpathlineto{\pgfqpoint{5.362857in}{0.851710in}}%
\pgfpathlineto{\pgfqpoint{5.506684in}{0.851205in}}%
\pgfpathlineto{\pgfqpoint{5.535450in}{0.837878in}}%
\pgfpathlineto{\pgfqpoint{5.650512in}{0.837878in}}%
\pgfpathlineto{\pgfqpoint{5.679277in}{0.792192in}}%
\pgfpathlineto{\pgfqpoint{5.909401in}{0.792192in}}%
\pgfpathlineto{\pgfqpoint{5.938166in}{0.789668in}}%
\pgfpathlineto{\pgfqpoint{5.966932in}{0.789668in}}%
\pgfpathlineto{\pgfqpoint{5.966932in}{0.789668in}}%
\pgfusepath{stroke}%
\end{pgfscope}%
\begin{pgfscope}%
\pgfpathrectangle{\pgfqpoint{0.862500in}{0.375000in}}{\pgfqpoint{5.347500in}{2.265000in}}%
\pgfusepath{clip}%
\pgfsetroundcap%
\pgfsetroundjoin%
\pgfsetlinewidth{1.505625pt}%
\definecolor{currentstroke}{rgb}{0.839216,0.152941,0.156863}%
\pgfsetstrokecolor{currentstroke}%
\pgfsetdash{}{0pt}%
\pgfpathmoveto{\pgfqpoint{1.105568in}{2.505456in}}%
\pgfpathlineto{\pgfqpoint{1.134334in}{2.504339in}}%
\pgfpathlineto{\pgfqpoint{1.163099in}{2.447206in}}%
\pgfpathlineto{\pgfqpoint{1.191865in}{2.398215in}}%
\pgfpathlineto{\pgfqpoint{1.220630in}{2.371373in}}%
\pgfpathlineto{\pgfqpoint{1.278161in}{2.371373in}}%
\pgfpathlineto{\pgfqpoint{1.306926in}{2.293023in}}%
\pgfpathlineto{\pgfqpoint{1.364457in}{2.293023in}}%
\pgfpathlineto{\pgfqpoint{1.393223in}{2.283593in}}%
\pgfpathlineto{\pgfqpoint{1.421988in}{2.283593in}}%
\pgfpathlineto{\pgfqpoint{1.450754in}{2.279136in}}%
\pgfpathlineto{\pgfqpoint{1.508285in}{2.279136in}}%
\pgfpathlineto{\pgfqpoint{1.537050in}{2.253990in}}%
\pgfpathlineto{\pgfqpoint{1.767174in}{2.253990in}}%
\pgfpathlineto{\pgfqpoint{1.795939in}{2.201200in}}%
\pgfpathlineto{\pgfqpoint{1.824705in}{2.194382in}}%
\pgfpathlineto{\pgfqpoint{1.853470in}{2.193217in}}%
\pgfpathlineto{\pgfqpoint{1.882236in}{2.171591in}}%
\pgfpathlineto{\pgfqpoint{1.939767in}{2.171591in}}%
\pgfpathlineto{\pgfqpoint{1.968532in}{2.094623in}}%
\pgfpathlineto{\pgfqpoint{1.997298in}{2.052793in}}%
\pgfpathlineto{\pgfqpoint{2.083594in}{2.052793in}}%
\pgfpathlineto{\pgfqpoint{2.141125in}{2.045599in}}%
\pgfpathlineto{\pgfqpoint{2.169890in}{2.045599in}}%
\pgfpathlineto{\pgfqpoint{2.198656in}{2.040204in}}%
\pgfpathlineto{\pgfqpoint{2.227421in}{2.038856in}}%
\pgfpathlineto{\pgfqpoint{2.256187in}{1.929701in}}%
\pgfpathlineto{\pgfqpoint{2.313718in}{1.929701in}}%
\pgfpathlineto{\pgfqpoint{2.342483in}{1.913677in}}%
\pgfpathlineto{\pgfqpoint{2.371249in}{1.913677in}}%
\pgfpathlineto{\pgfqpoint{2.400014in}{1.887822in}}%
\pgfpathlineto{\pgfqpoint{2.428780in}{1.853050in}}%
\pgfpathlineto{\pgfqpoint{2.457545in}{1.853050in}}%
\pgfpathlineto{\pgfqpoint{2.486311in}{1.848549in}}%
\pgfpathlineto{\pgfqpoint{2.543841in}{1.847569in}}%
\pgfpathlineto{\pgfqpoint{2.572607in}{1.798606in}}%
\pgfpathlineto{\pgfqpoint{2.601372in}{1.758658in}}%
\pgfpathlineto{\pgfqpoint{2.630138in}{1.723473in}}%
\pgfpathlineto{\pgfqpoint{2.745200in}{1.723473in}}%
\pgfpathlineto{\pgfqpoint{2.773965in}{1.691802in}}%
\pgfpathlineto{\pgfqpoint{2.860262in}{1.691802in}}%
\pgfpathlineto{\pgfqpoint{2.889027in}{1.616987in}}%
\pgfpathlineto{\pgfqpoint{2.917792in}{1.591010in}}%
\pgfpathlineto{\pgfqpoint{2.946558in}{1.549992in}}%
\pgfpathlineto{\pgfqpoint{3.032854in}{1.549992in}}%
\pgfpathlineto{\pgfqpoint{3.061620in}{1.535049in}}%
\pgfpathlineto{\pgfqpoint{3.090385in}{1.523366in}}%
\pgfpathlineto{\pgfqpoint{3.349274in}{1.523366in}}%
\pgfpathlineto{\pgfqpoint{3.378040in}{1.520863in}}%
\pgfpathlineto{\pgfqpoint{3.406805in}{1.554403in}}%
\pgfpathlineto{\pgfqpoint{3.435571in}{1.542734in}}%
\pgfpathlineto{\pgfqpoint{3.464336in}{1.491463in}}%
\pgfpathlineto{\pgfqpoint{3.493102in}{1.491463in}}%
\pgfpathlineto{\pgfqpoint{3.521867in}{1.486831in}}%
\pgfpathlineto{\pgfqpoint{3.550633in}{1.486831in}}%
\pgfpathlineto{\pgfqpoint{3.579398in}{1.451451in}}%
\pgfpathlineto{\pgfqpoint{3.608164in}{1.451451in}}%
\pgfpathlineto{\pgfqpoint{3.636929in}{1.445358in}}%
\pgfpathlineto{\pgfqpoint{3.723226in}{1.445358in}}%
\pgfpathlineto{\pgfqpoint{3.751991in}{1.438508in}}%
\pgfpathlineto{\pgfqpoint{3.953349in}{1.438508in}}%
\pgfpathlineto{\pgfqpoint{3.982115in}{1.372572in}}%
\pgfpathlineto{\pgfqpoint{4.154708in}{1.372572in}}%
\pgfpathlineto{\pgfqpoint{4.183473in}{1.354360in}}%
\pgfpathlineto{\pgfqpoint{4.212238in}{1.354360in}}%
\pgfpathlineto{\pgfqpoint{4.241004in}{1.320834in}}%
\pgfpathlineto{\pgfqpoint{4.269769in}{1.320834in}}%
\pgfpathlineto{\pgfqpoint{4.298535in}{1.248481in}}%
\pgfpathlineto{\pgfqpoint{4.327300in}{1.248481in}}%
\pgfpathlineto{\pgfqpoint{4.356066in}{1.225006in}}%
\pgfpathlineto{\pgfqpoint{5.046437in}{1.225006in}}%
\pgfpathlineto{\pgfqpoint{5.075202in}{1.218656in}}%
\pgfpathlineto{\pgfqpoint{5.362857in}{1.218656in}}%
\pgfpathlineto{\pgfqpoint{5.391623in}{1.197576in}}%
\pgfpathlineto{\pgfqpoint{5.708043in}{1.197576in}}%
\pgfpathlineto{\pgfqpoint{5.736808in}{1.144630in}}%
\pgfpathlineto{\pgfqpoint{5.794339in}{1.144630in}}%
\pgfpathlineto{\pgfqpoint{5.823104in}{1.122387in}}%
\pgfpathlineto{\pgfqpoint{5.966932in}{1.122387in}}%
\pgfpathlineto{\pgfqpoint{5.966932in}{1.122387in}}%
\pgfusepath{stroke}%
\end{pgfscope}%
\begin{pgfscope}%
\pgfpathrectangle{\pgfqpoint{0.862500in}{0.375000in}}{\pgfqpoint{5.347500in}{2.265000in}}%
\pgfusepath{clip}%
\pgfsetroundcap%
\pgfsetroundjoin%
\pgfsetlinewidth{1.505625pt}%
\definecolor{currentstroke}{rgb}{0.580392,0.403922,0.741176}%
\pgfsetstrokecolor{currentstroke}%
\pgfsetdash{}{0pt}%
\pgfpathmoveto{\pgfqpoint{1.105568in}{2.521510in}}%
\pgfpathlineto{\pgfqpoint{1.134334in}{2.475281in}}%
\pgfpathlineto{\pgfqpoint{1.163099in}{2.475281in}}%
\pgfpathlineto{\pgfqpoint{1.191865in}{2.461023in}}%
\pgfpathlineto{\pgfqpoint{1.220630in}{2.413710in}}%
\pgfpathlineto{\pgfqpoint{1.249396in}{2.410154in}}%
\pgfpathlineto{\pgfqpoint{1.364457in}{2.410154in}}%
\pgfpathlineto{\pgfqpoint{1.393223in}{2.400057in}}%
\pgfpathlineto{\pgfqpoint{1.508285in}{2.400057in}}%
\pgfpathlineto{\pgfqpoint{1.537050in}{2.337323in}}%
\pgfpathlineto{\pgfqpoint{1.709643in}{2.337323in}}%
\pgfpathlineto{\pgfqpoint{1.738408in}{2.333584in}}%
\pgfpathlineto{\pgfqpoint{1.795939in}{2.333584in}}%
\pgfpathlineto{\pgfqpoint{1.824705in}{2.284711in}}%
\pgfpathlineto{\pgfqpoint{1.882236in}{2.284711in}}%
\pgfpathlineto{\pgfqpoint{1.911001in}{2.274409in}}%
\pgfpathlineto{\pgfqpoint{1.939767in}{2.202549in}}%
\pgfpathlineto{\pgfqpoint{1.968532in}{2.202549in}}%
\pgfpathlineto{\pgfqpoint{1.997298in}{2.196010in}}%
\pgfpathlineto{\pgfqpoint{2.026063in}{2.196010in}}%
\pgfpathlineto{\pgfqpoint{2.054829in}{2.153339in}}%
\pgfpathlineto{\pgfqpoint{2.083594in}{2.153339in}}%
\pgfpathlineto{\pgfqpoint{2.112359in}{2.129870in}}%
\pgfpathlineto{\pgfqpoint{2.141125in}{2.046408in}}%
\pgfpathlineto{\pgfqpoint{2.198656in}{2.046408in}}%
\pgfpathlineto{\pgfqpoint{2.227421in}{2.029651in}}%
\pgfpathlineto{\pgfqpoint{2.256187in}{1.957381in}}%
\pgfpathlineto{\pgfqpoint{2.313718in}{1.957381in}}%
\pgfpathlineto{\pgfqpoint{2.342483in}{1.931199in}}%
\pgfpathlineto{\pgfqpoint{2.371249in}{1.844583in}}%
\pgfpathlineto{\pgfqpoint{2.486311in}{1.844583in}}%
\pgfpathlineto{\pgfqpoint{2.515076in}{1.748921in}}%
\pgfpathlineto{\pgfqpoint{2.543841in}{1.684299in}}%
\pgfpathlineto{\pgfqpoint{2.572607in}{1.668165in}}%
\pgfpathlineto{\pgfqpoint{2.601372in}{1.592684in}}%
\pgfpathlineto{\pgfqpoint{2.630138in}{1.536369in}}%
\pgfpathlineto{\pgfqpoint{2.658903in}{1.501174in}}%
\pgfpathlineto{\pgfqpoint{2.773965in}{1.501174in}}%
\pgfpathlineto{\pgfqpoint{2.831496in}{1.474302in}}%
\pgfpathlineto{\pgfqpoint{2.917792in}{1.474302in}}%
\pgfpathlineto{\pgfqpoint{2.946558in}{1.359080in}}%
\pgfpathlineto{\pgfqpoint{2.975323in}{1.359080in}}%
\pgfpathlineto{\pgfqpoint{3.004089in}{1.292257in}}%
\pgfpathlineto{\pgfqpoint{3.147916in}{1.292257in}}%
\pgfpathlineto{\pgfqpoint{3.176682in}{1.259178in}}%
\pgfpathlineto{\pgfqpoint{3.291744in}{1.259178in}}%
\pgfpathlineto{\pgfqpoint{3.320509in}{1.220409in}}%
\pgfpathlineto{\pgfqpoint{3.378040in}{1.220409in}}%
\pgfpathlineto{\pgfqpoint{3.406805in}{1.199706in}}%
\pgfpathlineto{\pgfqpoint{3.636929in}{1.198359in}}%
\pgfpathlineto{\pgfqpoint{3.665695in}{1.156370in}}%
\pgfpathlineto{\pgfqpoint{3.751991in}{1.156370in}}%
\pgfpathlineto{\pgfqpoint{3.780756in}{1.149103in}}%
\pgfpathlineto{\pgfqpoint{3.895818in}{1.149103in}}%
\pgfpathlineto{\pgfqpoint{3.924584in}{1.122276in}}%
\pgfpathlineto{\pgfqpoint{4.068411in}{1.122276in}}%
\pgfpathlineto{\pgfqpoint{4.097177in}{1.112161in}}%
\pgfpathlineto{\pgfqpoint{4.327300in}{1.112161in}}%
\pgfpathlineto{\pgfqpoint{4.356066in}{1.103917in}}%
\pgfpathlineto{\pgfqpoint{4.413597in}{1.103917in}}%
\pgfpathlineto{\pgfqpoint{4.442362in}{1.089841in}}%
\pgfpathlineto{\pgfqpoint{4.499893in}{1.089841in}}%
\pgfpathlineto{\pgfqpoint{4.528659in}{1.082195in}}%
\pgfpathlineto{\pgfqpoint{4.557424in}{1.071087in}}%
\pgfpathlineto{\pgfqpoint{4.701251in}{1.071087in}}%
\pgfpathlineto{\pgfqpoint{4.730017in}{1.064564in}}%
\pgfpathlineto{\pgfqpoint{4.758782in}{1.047933in}}%
\pgfpathlineto{\pgfqpoint{4.816313in}{1.047933in}}%
\pgfpathlineto{\pgfqpoint{4.845079in}{1.025749in}}%
\pgfpathlineto{\pgfqpoint{4.873844in}{1.015018in}}%
\pgfpathlineto{\pgfqpoint{4.988906in}{1.015018in}}%
\pgfpathlineto{\pgfqpoint{5.017671in}{1.008899in}}%
\pgfpathlineto{\pgfqpoint{5.219030in}{1.008899in}}%
\pgfpathlineto{\pgfqpoint{5.247795in}{0.997201in}}%
\pgfpathlineto{\pgfqpoint{5.276561in}{0.997201in}}%
\pgfpathlineto{\pgfqpoint{5.305326in}{0.958344in}}%
\pgfpathlineto{\pgfqpoint{5.334092in}{0.958344in}}%
\pgfpathlineto{\pgfqpoint{5.362857in}{0.944678in}}%
\pgfpathlineto{\pgfqpoint{5.449153in}{0.944678in}}%
\pgfpathlineto{\pgfqpoint{5.477919in}{0.938674in}}%
\pgfpathlineto{\pgfqpoint{5.592981in}{0.938674in}}%
\pgfpathlineto{\pgfqpoint{5.621746in}{0.922759in}}%
\pgfpathlineto{\pgfqpoint{5.966932in}{0.921789in}}%
\pgfpathlineto{\pgfqpoint{5.966932in}{0.921789in}}%
\pgfusepath{stroke}%
\end{pgfscope}%
\begin{pgfscope}%
\pgfsetrectcap%
\pgfsetmiterjoin%
\pgfsetlinewidth{0.000000pt}%
\definecolor{currentstroke}{rgb}{1.000000,1.000000,1.000000}%
\pgfsetstrokecolor{currentstroke}%
\pgfsetdash{}{0pt}%
\pgfpathmoveto{\pgfqpoint{0.862500in}{0.375000in}}%
\pgfpathlineto{\pgfqpoint{0.862500in}{2.640000in}}%
\pgfusepath{}%
\end{pgfscope}%
\begin{pgfscope}%
\pgfsetrectcap%
\pgfsetmiterjoin%
\pgfsetlinewidth{0.000000pt}%
\definecolor{currentstroke}{rgb}{1.000000,1.000000,1.000000}%
\pgfsetstrokecolor{currentstroke}%
\pgfsetdash{}{0pt}%
\pgfpathmoveto{\pgfqpoint{6.210000in}{0.375000in}}%
\pgfpathlineto{\pgfqpoint{6.210000in}{2.640000in}}%
\pgfusepath{}%
\end{pgfscope}%
\begin{pgfscope}%
\pgfsetrectcap%
\pgfsetmiterjoin%
\pgfsetlinewidth{0.000000pt}%
\definecolor{currentstroke}{rgb}{1.000000,1.000000,1.000000}%
\pgfsetstrokecolor{currentstroke}%
\pgfsetdash{}{0pt}%
\pgfpathmoveto{\pgfqpoint{0.862500in}{0.375000in}}%
\pgfpathlineto{\pgfqpoint{6.210000in}{0.375000in}}%
\pgfusepath{}%
\end{pgfscope}%
\begin{pgfscope}%
\pgfsetrectcap%
\pgfsetmiterjoin%
\pgfsetlinewidth{0.000000pt}%
\definecolor{currentstroke}{rgb}{1.000000,1.000000,1.000000}%
\pgfsetstrokecolor{currentstroke}%
\pgfsetdash{}{0pt}%
\pgfpathmoveto{\pgfqpoint{0.862500in}{2.640000in}}%
\pgfpathlineto{\pgfqpoint{6.210000in}{2.640000in}}%
\pgfusepath{}%
\end{pgfscope}%
\begin{pgfscope}%
\definecolor{textcolor}{rgb}{0.150000,0.150000,0.150000}%
\pgfsetstrokecolor{textcolor}%
\pgfsetfillcolor{textcolor}%
\pgftext[x=3.536250in,y=2.723333in,,base]{\color{textcolor}\rmfamily\fontsize{8.000000}{9.600000}\selectfont Hartmann6}%
\end{pgfscope}%
\begin{pgfscope}%
\pgfsetroundcap%
\pgfsetroundjoin%
\pgfsetlinewidth{1.505625pt}%
\definecolor{currentstroke}{rgb}{0.121569,0.466667,0.705882}%
\pgfsetstrokecolor{currentstroke}%
\pgfsetdash{}{0pt}%
\pgfpathmoveto{\pgfqpoint{0.962500in}{1.189344in}}%
\pgfpathlineto{\pgfqpoint{1.184722in}{1.189344in}}%
\pgfusepath{stroke}%
\end{pgfscope}%
\begin{pgfscope}%
\definecolor{textcolor}{rgb}{0.150000,0.150000,0.150000}%
\pgfsetstrokecolor{textcolor}%
\pgfsetfillcolor{textcolor}%
\pgftext[x=1.273611in,y=1.150455in,left,base]{\color{textcolor}\rmfamily\fontsize{8.000000}{9.600000}\selectfont random}%
\end{pgfscope}%
\begin{pgfscope}%
\pgfsetroundcap%
\pgfsetroundjoin%
\pgfsetlinewidth{1.505625pt}%
\definecolor{currentstroke}{rgb}{1.000000,0.498039,0.054902}%
\pgfsetstrokecolor{currentstroke}%
\pgfsetdash{}{0pt}%
\pgfpathmoveto{\pgfqpoint{0.962500in}{1.026258in}}%
\pgfpathlineto{\pgfqpoint{1.184722in}{1.026258in}}%
\pgfusepath{stroke}%
\end{pgfscope}%
\begin{pgfscope}%
\definecolor{textcolor}{rgb}{0.150000,0.150000,0.150000}%
\pgfsetstrokecolor{textcolor}%
\pgfsetfillcolor{textcolor}%
\pgftext[x=1.273611in,y=0.987369in,left,base]{\color{textcolor}\rmfamily\fontsize{8.000000}{9.600000}\selectfont 5 x DNGO retrain}%
\end{pgfscope}%
\begin{pgfscope}%
\pgfsetroundcap%
\pgfsetroundjoin%
\pgfsetlinewidth{1.505625pt}%
\definecolor{currentstroke}{rgb}{0.172549,0.627451,0.172549}%
\pgfsetstrokecolor{currentstroke}%
\pgfsetdash{}{0pt}%
\pgfpathmoveto{\pgfqpoint{0.962500in}{0.863172in}}%
\pgfpathlineto{\pgfqpoint{1.184722in}{0.863172in}}%
\pgfusepath{stroke}%
\end{pgfscope}%
\begin{pgfscope}%
\definecolor{textcolor}{rgb}{0.150000,0.150000,0.150000}%
\pgfsetstrokecolor{textcolor}%
\pgfsetfillcolor{textcolor}%
\pgftext[x=1.273611in,y=0.824283in,left,base]{\color{textcolor}\rmfamily\fontsize{8.000000}{9.600000}\selectfont 5 x DNGO retrain-reset}%
\end{pgfscope}%
\begin{pgfscope}%
\pgfsetroundcap%
\pgfsetroundjoin%
\pgfsetlinewidth{1.505625pt}%
\definecolor{currentstroke}{rgb}{0.839216,0.152941,0.156863}%
\pgfsetstrokecolor{currentstroke}%
\pgfsetdash{}{0pt}%
\pgfpathmoveto{\pgfqpoint{0.962500in}{0.700087in}}%
\pgfpathlineto{\pgfqpoint{1.184722in}{0.700087in}}%
\pgfusepath{stroke}%
\end{pgfscope}%
\begin{pgfscope}%
\definecolor{textcolor}{rgb}{0.150000,0.150000,0.150000}%
\pgfsetstrokecolor{textcolor}%
\pgfsetfillcolor{textcolor}%
\pgftext[x=1.273611in,y=0.661198in,left,base]{\color{textcolor}\rmfamily\fontsize{8.000000}{9.600000}\selectfont DNGO retrain}%
\end{pgfscope}%
\begin{pgfscope}%
\pgfsetroundcap%
\pgfsetroundjoin%
\pgfsetlinewidth{1.505625pt}%
\definecolor{currentstroke}{rgb}{0.580392,0.403922,0.741176}%
\pgfsetstrokecolor{currentstroke}%
\pgfsetdash{}{0pt}%
\pgfpathmoveto{\pgfqpoint{0.962500in}{0.537001in}}%
\pgfpathlineto{\pgfqpoint{1.184722in}{0.537001in}}%
\pgfusepath{stroke}%
\end{pgfscope}%
\begin{pgfscope}%
\definecolor{textcolor}{rgb}{0.150000,0.150000,0.150000}%
\pgfsetstrokecolor{textcolor}%
\pgfsetfillcolor{textcolor}%
\pgftext[x=1.273611in,y=0.498112in,left,base]{\color{textcolor}\rmfamily\fontsize{8.000000}{9.600000}\selectfont DNGO retrain-reset}%
\end{pgfscope}%
\end{pgfpicture}%
\makeatother%
\endgroup%

            %% Creator: Matplotlib, PGF backend
%%
%% To include the figure in your LaTeX document, write
%%   \input{<filename>.pgf}
%%
%% Make sure the required packages are loaded in your preamble
%%   \usepackage{pgf}
%%
%% Figures using additional raster images can only be included by \input if
%% they are in the same directory as the main LaTeX file. For loading figures
%% from other directories you can use the `import` package
%%   \usepackage{import}
%% and then include the figures with
%%   \import{<path to file>}{<filename>.pgf}
%%
%% Matplotlib used the following preamble
%%   \usepackage{gensymb}
%%   \usepackage{fontspec}
%%   \setmainfont{DejaVu Serif}
%%   \setsansfont{Arial}
%%   \setmonofont{DejaVu Sans Mono}
%%
\begingroup%
\makeatletter%
\begin{pgfpicture}%
\pgfpathrectangle{\pgfpointorigin}{\pgfqpoint{6.900000in}{3.000000in}}%
\pgfusepath{use as bounding box, clip}%
\begin{pgfscope}%
\pgfsetbuttcap%
\pgfsetmiterjoin%
\definecolor{currentfill}{rgb}{1.000000,1.000000,1.000000}%
\pgfsetfillcolor{currentfill}%
\pgfsetlinewidth{0.000000pt}%
\definecolor{currentstroke}{rgb}{1.000000,1.000000,1.000000}%
\pgfsetstrokecolor{currentstroke}%
\pgfsetdash{}{0pt}%
\pgfpathmoveto{\pgfqpoint{0.000000in}{0.000000in}}%
\pgfpathlineto{\pgfqpoint{6.900000in}{0.000000in}}%
\pgfpathlineto{\pgfqpoint{6.900000in}{3.000000in}}%
\pgfpathlineto{\pgfqpoint{0.000000in}{3.000000in}}%
\pgfpathclose%
\pgfusepath{fill}%
\end{pgfscope}%
\begin{pgfscope}%
\pgfsetbuttcap%
\pgfsetmiterjoin%
\definecolor{currentfill}{rgb}{0.917647,0.917647,0.949020}%
\pgfsetfillcolor{currentfill}%
\pgfsetlinewidth{0.000000pt}%
\definecolor{currentstroke}{rgb}{0.000000,0.000000,0.000000}%
\pgfsetstrokecolor{currentstroke}%
\pgfsetstrokeopacity{0.000000}%
\pgfsetdash{}{0pt}%
\pgfpathmoveto{\pgfqpoint{0.862500in}{0.375000in}}%
\pgfpathlineto{\pgfqpoint{6.210000in}{0.375000in}}%
\pgfpathlineto{\pgfqpoint{6.210000in}{2.640000in}}%
\pgfpathlineto{\pgfqpoint{0.862500in}{2.640000in}}%
\pgfpathclose%
\pgfusepath{fill}%
\end{pgfscope}%
\begin{pgfscope}%
\pgfpathrectangle{\pgfqpoint{0.862500in}{0.375000in}}{\pgfqpoint{5.347500in}{2.265000in}}%
\pgfusepath{clip}%
\pgfsetroundcap%
\pgfsetroundjoin%
\pgfsetlinewidth{0.803000pt}%
\definecolor{currentstroke}{rgb}{1.000000,1.000000,1.000000}%
\pgfsetstrokecolor{currentstroke}%
\pgfsetdash{}{0pt}%
\pgfpathmoveto{\pgfqpoint{0.862500in}{0.375000in}}%
\pgfpathlineto{\pgfqpoint{0.862500in}{2.640000in}}%
\pgfusepath{stroke}%
\end{pgfscope}%
\begin{pgfscope}%
\definecolor{textcolor}{rgb}{0.150000,0.150000,0.150000}%
\pgfsetstrokecolor{textcolor}%
\pgfsetfillcolor{textcolor}%
\pgftext[x=0.862500in,y=0.326389in,,top]{\color{textcolor}\rmfamily\fontsize{8.000000}{9.600000}\selectfont \(\displaystyle 100\)}%
\end{pgfscope}%
\begin{pgfscope}%
\pgfpathrectangle{\pgfqpoint{0.862500in}{0.375000in}}{\pgfqpoint{5.347500in}{2.265000in}}%
\pgfusepath{clip}%
\pgfsetroundcap%
\pgfsetroundjoin%
\pgfsetlinewidth{0.803000pt}%
\definecolor{currentstroke}{rgb}{1.000000,1.000000,1.000000}%
\pgfsetstrokecolor{currentstroke}%
\pgfsetdash{}{0pt}%
\pgfpathmoveto{\pgfqpoint{1.575500in}{0.375000in}}%
\pgfpathlineto{\pgfqpoint{1.575500in}{2.640000in}}%
\pgfusepath{stroke}%
\end{pgfscope}%
\begin{pgfscope}%
\definecolor{textcolor}{rgb}{0.150000,0.150000,0.150000}%
\pgfsetstrokecolor{textcolor}%
\pgfsetfillcolor{textcolor}%
\pgftext[x=1.575500in,y=0.326389in,,top]{\color{textcolor}\rmfamily\fontsize{8.000000}{9.600000}\selectfont \(\displaystyle 120\)}%
\end{pgfscope}%
\begin{pgfscope}%
\pgfpathrectangle{\pgfqpoint{0.862500in}{0.375000in}}{\pgfqpoint{5.347500in}{2.265000in}}%
\pgfusepath{clip}%
\pgfsetroundcap%
\pgfsetroundjoin%
\pgfsetlinewidth{0.803000pt}%
\definecolor{currentstroke}{rgb}{1.000000,1.000000,1.000000}%
\pgfsetstrokecolor{currentstroke}%
\pgfsetdash{}{0pt}%
\pgfpathmoveto{\pgfqpoint{2.288500in}{0.375000in}}%
\pgfpathlineto{\pgfqpoint{2.288500in}{2.640000in}}%
\pgfusepath{stroke}%
\end{pgfscope}%
\begin{pgfscope}%
\definecolor{textcolor}{rgb}{0.150000,0.150000,0.150000}%
\pgfsetstrokecolor{textcolor}%
\pgfsetfillcolor{textcolor}%
\pgftext[x=2.288500in,y=0.326389in,,top]{\color{textcolor}\rmfamily\fontsize{8.000000}{9.600000}\selectfont \(\displaystyle 140\)}%
\end{pgfscope}%
\begin{pgfscope}%
\pgfpathrectangle{\pgfqpoint{0.862500in}{0.375000in}}{\pgfqpoint{5.347500in}{2.265000in}}%
\pgfusepath{clip}%
\pgfsetroundcap%
\pgfsetroundjoin%
\pgfsetlinewidth{0.803000pt}%
\definecolor{currentstroke}{rgb}{1.000000,1.000000,1.000000}%
\pgfsetstrokecolor{currentstroke}%
\pgfsetdash{}{0pt}%
\pgfpathmoveto{\pgfqpoint{3.001500in}{0.375000in}}%
\pgfpathlineto{\pgfqpoint{3.001500in}{2.640000in}}%
\pgfusepath{stroke}%
\end{pgfscope}%
\begin{pgfscope}%
\definecolor{textcolor}{rgb}{0.150000,0.150000,0.150000}%
\pgfsetstrokecolor{textcolor}%
\pgfsetfillcolor{textcolor}%
\pgftext[x=3.001500in,y=0.326389in,,top]{\color{textcolor}\rmfamily\fontsize{8.000000}{9.600000}\selectfont \(\displaystyle 160\)}%
\end{pgfscope}%
\begin{pgfscope}%
\pgfpathrectangle{\pgfqpoint{0.862500in}{0.375000in}}{\pgfqpoint{5.347500in}{2.265000in}}%
\pgfusepath{clip}%
\pgfsetroundcap%
\pgfsetroundjoin%
\pgfsetlinewidth{0.803000pt}%
\definecolor{currentstroke}{rgb}{1.000000,1.000000,1.000000}%
\pgfsetstrokecolor{currentstroke}%
\pgfsetdash{}{0pt}%
\pgfpathmoveto{\pgfqpoint{3.714500in}{0.375000in}}%
\pgfpathlineto{\pgfqpoint{3.714500in}{2.640000in}}%
\pgfusepath{stroke}%
\end{pgfscope}%
\begin{pgfscope}%
\definecolor{textcolor}{rgb}{0.150000,0.150000,0.150000}%
\pgfsetstrokecolor{textcolor}%
\pgfsetfillcolor{textcolor}%
\pgftext[x=3.714500in,y=0.326389in,,top]{\color{textcolor}\rmfamily\fontsize{8.000000}{9.600000}\selectfont \(\displaystyle 180\)}%
\end{pgfscope}%
\begin{pgfscope}%
\pgfpathrectangle{\pgfqpoint{0.862500in}{0.375000in}}{\pgfqpoint{5.347500in}{2.265000in}}%
\pgfusepath{clip}%
\pgfsetroundcap%
\pgfsetroundjoin%
\pgfsetlinewidth{0.803000pt}%
\definecolor{currentstroke}{rgb}{1.000000,1.000000,1.000000}%
\pgfsetstrokecolor{currentstroke}%
\pgfsetdash{}{0pt}%
\pgfpathmoveto{\pgfqpoint{4.427500in}{0.375000in}}%
\pgfpathlineto{\pgfqpoint{4.427500in}{2.640000in}}%
\pgfusepath{stroke}%
\end{pgfscope}%
\begin{pgfscope}%
\definecolor{textcolor}{rgb}{0.150000,0.150000,0.150000}%
\pgfsetstrokecolor{textcolor}%
\pgfsetfillcolor{textcolor}%
\pgftext[x=4.427500in,y=0.326389in,,top]{\color{textcolor}\rmfamily\fontsize{8.000000}{9.600000}\selectfont \(\displaystyle 200\)}%
\end{pgfscope}%
\begin{pgfscope}%
\pgfpathrectangle{\pgfqpoint{0.862500in}{0.375000in}}{\pgfqpoint{5.347500in}{2.265000in}}%
\pgfusepath{clip}%
\pgfsetroundcap%
\pgfsetroundjoin%
\pgfsetlinewidth{0.803000pt}%
\definecolor{currentstroke}{rgb}{1.000000,1.000000,1.000000}%
\pgfsetstrokecolor{currentstroke}%
\pgfsetdash{}{0pt}%
\pgfpathmoveto{\pgfqpoint{5.140500in}{0.375000in}}%
\pgfpathlineto{\pgfqpoint{5.140500in}{2.640000in}}%
\pgfusepath{stroke}%
\end{pgfscope}%
\begin{pgfscope}%
\definecolor{textcolor}{rgb}{0.150000,0.150000,0.150000}%
\pgfsetstrokecolor{textcolor}%
\pgfsetfillcolor{textcolor}%
\pgftext[x=5.140500in,y=0.326389in,,top]{\color{textcolor}\rmfamily\fontsize{8.000000}{9.600000}\selectfont \(\displaystyle 220\)}%
\end{pgfscope}%
\begin{pgfscope}%
\pgfpathrectangle{\pgfqpoint{0.862500in}{0.375000in}}{\pgfqpoint{5.347500in}{2.265000in}}%
\pgfusepath{clip}%
\pgfsetroundcap%
\pgfsetroundjoin%
\pgfsetlinewidth{0.803000pt}%
\definecolor{currentstroke}{rgb}{1.000000,1.000000,1.000000}%
\pgfsetstrokecolor{currentstroke}%
\pgfsetdash{}{0pt}%
\pgfpathmoveto{\pgfqpoint{5.853500in}{0.375000in}}%
\pgfpathlineto{\pgfqpoint{5.853500in}{2.640000in}}%
\pgfusepath{stroke}%
\end{pgfscope}%
\begin{pgfscope}%
\definecolor{textcolor}{rgb}{0.150000,0.150000,0.150000}%
\pgfsetstrokecolor{textcolor}%
\pgfsetfillcolor{textcolor}%
\pgftext[x=5.853500in,y=0.326389in,,top]{\color{textcolor}\rmfamily\fontsize{8.000000}{9.600000}\selectfont \(\displaystyle 240\)}%
\end{pgfscope}%
\begin{pgfscope}%
\definecolor{textcolor}{rgb}{0.150000,0.150000,0.150000}%
\pgfsetstrokecolor{textcolor}%
\pgfsetfillcolor{textcolor}%
\pgftext[x=3.536250in,y=0.163303in,,top]{\color{textcolor}\rmfamily\fontsize{8.000000}{9.600000}\selectfont Step}%
\end{pgfscope}%
\begin{pgfscope}%
\pgfpathrectangle{\pgfqpoint{0.862500in}{0.375000in}}{\pgfqpoint{5.347500in}{2.265000in}}%
\pgfusepath{clip}%
\pgfsetroundcap%
\pgfsetroundjoin%
\pgfsetlinewidth{0.803000pt}%
\definecolor{currentstroke}{rgb}{1.000000,1.000000,1.000000}%
\pgfsetstrokecolor{currentstroke}%
\pgfsetdash{}{0pt}%
\pgfpathmoveto{\pgfqpoint{0.862500in}{0.899072in}}%
\pgfpathlineto{\pgfqpoint{6.210000in}{0.899072in}}%
\pgfusepath{stroke}%
\end{pgfscope}%
\begin{pgfscope}%
\definecolor{textcolor}{rgb}{0.150000,0.150000,0.150000}%
\pgfsetstrokecolor{textcolor}%
\pgfsetfillcolor{textcolor}%
\pgftext[x=0.637962in,y=0.856862in,left,base]{\color{textcolor}\rmfamily\fontsize{8.000000}{9.600000}\selectfont \(\displaystyle 10^{3}\)}%
\end{pgfscope}%
\begin{pgfscope}%
\pgfpathrectangle{\pgfqpoint{0.862500in}{0.375000in}}{\pgfqpoint{5.347500in}{2.265000in}}%
\pgfusepath{clip}%
\pgfsetroundcap%
\pgfsetroundjoin%
\pgfsetlinewidth{0.803000pt}%
\definecolor{currentstroke}{rgb}{1.000000,1.000000,1.000000}%
\pgfsetstrokecolor{currentstroke}%
\pgfsetdash{}{0pt}%
\pgfpathmoveto{\pgfqpoint{0.862500in}{2.640000in}}%
\pgfpathlineto{\pgfqpoint{6.210000in}{2.640000in}}%
\pgfusepath{stroke}%
\end{pgfscope}%
\begin{pgfscope}%
\definecolor{textcolor}{rgb}{0.150000,0.150000,0.150000}%
\pgfsetstrokecolor{textcolor}%
\pgfsetfillcolor{textcolor}%
\pgftext[x=0.637962in,y=2.597791in,left,base]{\color{textcolor}\rmfamily\fontsize{8.000000}{9.600000}\selectfont \(\displaystyle 10^{4}\)}%
\end{pgfscope}%
\begin{pgfscope}%
\definecolor{textcolor}{rgb}{0.150000,0.150000,0.150000}%
\pgfsetstrokecolor{textcolor}%
\pgfsetfillcolor{textcolor}%
\pgftext[x=0.582407in,y=1.507500in,,bottom,rotate=90.000000]{\color{textcolor}\rmfamily\fontsize{8.000000}{9.600000}\selectfont Simple Regret}%
\end{pgfscope}%
\begin{pgfscope}%
\pgfpathrectangle{\pgfqpoint{0.862500in}{0.375000in}}{\pgfqpoint{5.347500in}{2.265000in}}%
\pgfusepath{clip}%
\pgfsetbuttcap%
\pgfsetroundjoin%
\definecolor{currentfill}{rgb}{0.121569,0.466667,0.705882}%
\pgfsetfillcolor{currentfill}%
\pgfsetfillopacity{0.200000}%
\pgfsetlinewidth{0.000000pt}%
\definecolor{currentstroke}{rgb}{0.000000,0.000000,0.000000}%
\pgfsetstrokecolor{currentstroke}%
\pgfsetdash{}{0pt}%
\pgfpathmoveto{\pgfqpoint{-2.702500in}{5.383122in}}%
\pgfpathlineto{\pgfqpoint{-2.702500in}{6.004118in}}%
\pgfpathlineto{\pgfqpoint{-2.666850in}{5.482781in}}%
\pgfpathlineto{\pgfqpoint{-2.631200in}{4.647644in}}%
\pgfpathlineto{\pgfqpoint{-2.595550in}{4.607661in}}%
\pgfpathlineto{\pgfqpoint{-2.559900in}{4.586163in}}%
\pgfpathlineto{\pgfqpoint{-2.524250in}{4.361660in}}%
\pgfpathlineto{\pgfqpoint{-2.488600in}{4.361660in}}%
\pgfpathlineto{\pgfqpoint{-2.452950in}{4.339921in}}%
\pgfpathlineto{\pgfqpoint{-2.417300in}{4.339921in}}%
\pgfpathlineto{\pgfqpoint{-2.381650in}{4.339921in}}%
\pgfpathlineto{\pgfqpoint{-2.346000in}{3.919607in}}%
\pgfpathlineto{\pgfqpoint{-2.310350in}{3.919607in}}%
\pgfpathlineto{\pgfqpoint{-2.274700in}{3.900709in}}%
\pgfpathlineto{\pgfqpoint{-2.239050in}{3.874987in}}%
\pgfpathlineto{\pgfqpoint{-2.203400in}{3.874987in}}%
\pgfpathlineto{\pgfqpoint{-2.167750in}{3.857448in}}%
\pgfpathlineto{\pgfqpoint{-2.132100in}{3.716925in}}%
\pgfpathlineto{\pgfqpoint{-2.096450in}{3.716925in}}%
\pgfpathlineto{\pgfqpoint{-2.060800in}{3.693328in}}%
\pgfpathlineto{\pgfqpoint{-2.025150in}{3.693328in}}%
\pgfpathlineto{\pgfqpoint{-1.989500in}{3.670805in}}%
\pgfpathlineto{\pgfqpoint{-1.953850in}{3.384468in}}%
\pgfpathlineto{\pgfqpoint{-1.918200in}{3.384468in}}%
\pgfpathlineto{\pgfqpoint{-1.882550in}{3.345204in}}%
\pgfpathlineto{\pgfqpoint{-1.846900in}{3.331003in}}%
\pgfpathlineto{\pgfqpoint{-1.811250in}{3.296578in}}%
\pgfpathlineto{\pgfqpoint{-1.775600in}{3.069575in}}%
\pgfpathlineto{\pgfqpoint{-1.739950in}{3.069575in}}%
\pgfpathlineto{\pgfqpoint{-1.704300in}{2.997689in}}%
\pgfpathlineto{\pgfqpoint{-1.668650in}{2.997689in}}%
\pgfpathlineto{\pgfqpoint{-1.633000in}{2.997689in}}%
\pgfpathlineto{\pgfqpoint{-1.597350in}{2.997689in}}%
\pgfpathlineto{\pgfqpoint{-1.561700in}{2.997689in}}%
\pgfpathlineto{\pgfqpoint{-1.526050in}{2.920059in}}%
\pgfpathlineto{\pgfqpoint{-1.490400in}{2.861780in}}%
\pgfpathlineto{\pgfqpoint{-1.454750in}{2.842628in}}%
\pgfpathlineto{\pgfqpoint{-1.419100in}{2.842628in}}%
\pgfpathlineto{\pgfqpoint{-1.383450in}{2.842628in}}%
\pgfpathlineto{\pgfqpoint{-1.347800in}{2.842628in}}%
\pgfpathlineto{\pgfqpoint{-1.312150in}{2.842628in}}%
\pgfpathlineto{\pgfqpoint{-1.276500in}{2.842628in}}%
\pgfpathlineto{\pgfqpoint{-1.240850in}{2.842628in}}%
\pgfpathlineto{\pgfqpoint{-1.205200in}{2.842628in}}%
\pgfpathlineto{\pgfqpoint{-1.169550in}{2.804576in}}%
\pgfpathlineto{\pgfqpoint{-1.133900in}{2.570987in}}%
\pgfpathlineto{\pgfqpoint{-1.098250in}{2.570987in}}%
\pgfpathlineto{\pgfqpoint{-1.062600in}{2.570987in}}%
\pgfpathlineto{\pgfqpoint{-1.026950in}{2.570987in}}%
\pgfpathlineto{\pgfqpoint{-0.991300in}{2.432629in}}%
\pgfpathlineto{\pgfqpoint{-0.955650in}{2.432629in}}%
\pgfpathlineto{\pgfqpoint{-0.920000in}{2.363795in}}%
\pgfpathlineto{\pgfqpoint{-0.884350in}{2.363795in}}%
\pgfpathlineto{\pgfqpoint{-0.848700in}{2.363795in}}%
\pgfpathlineto{\pgfqpoint{-0.813050in}{2.363795in}}%
\pgfpathlineto{\pgfqpoint{-0.777400in}{2.363795in}}%
\pgfpathlineto{\pgfqpoint{-0.741750in}{2.363795in}}%
\pgfpathlineto{\pgfqpoint{-0.706100in}{2.363795in}}%
\pgfpathlineto{\pgfqpoint{-0.670450in}{2.363795in}}%
\pgfpathlineto{\pgfqpoint{-0.634800in}{2.363795in}}%
\pgfpathlineto{\pgfqpoint{-0.599150in}{2.363795in}}%
\pgfpathlineto{\pgfqpoint{-0.563500in}{2.363795in}}%
\pgfpathlineto{\pgfqpoint{-0.527850in}{2.363795in}}%
\pgfpathlineto{\pgfqpoint{-0.492200in}{2.363795in}}%
\pgfpathlineto{\pgfqpoint{-0.456550in}{2.363795in}}%
\pgfpathlineto{\pgfqpoint{-0.420900in}{2.363795in}}%
\pgfpathlineto{\pgfqpoint{-0.385250in}{2.363795in}}%
\pgfpathlineto{\pgfqpoint{-0.349600in}{2.363795in}}%
\pgfpathlineto{\pgfqpoint{-0.313950in}{2.363795in}}%
\pgfpathlineto{\pgfqpoint{-0.278300in}{2.363795in}}%
\pgfpathlineto{\pgfqpoint{-0.242650in}{2.363795in}}%
\pgfpathlineto{\pgfqpoint{-0.207000in}{2.363795in}}%
\pgfpathlineto{\pgfqpoint{-0.171350in}{2.363795in}}%
\pgfpathlineto{\pgfqpoint{-0.135700in}{2.363795in}}%
\pgfpathlineto{\pgfqpoint{-0.100050in}{2.363795in}}%
\pgfpathlineto{\pgfqpoint{-0.064400in}{2.363795in}}%
\pgfpathlineto{\pgfqpoint{-0.028750in}{2.363795in}}%
\pgfpathlineto{\pgfqpoint{0.006900in}{2.363795in}}%
\pgfpathlineto{\pgfqpoint{0.042550in}{2.350593in}}%
\pgfpathlineto{\pgfqpoint{0.078200in}{2.350593in}}%
\pgfpathlineto{\pgfqpoint{0.113850in}{2.350593in}}%
\pgfpathlineto{\pgfqpoint{0.149500in}{2.350593in}}%
\pgfpathlineto{\pgfqpoint{0.185150in}{2.350593in}}%
\pgfpathlineto{\pgfqpoint{0.220800in}{2.350593in}}%
\pgfpathlineto{\pgfqpoint{0.256450in}{2.350593in}}%
\pgfpathlineto{\pgfqpoint{0.292100in}{2.350593in}}%
\pgfpathlineto{\pgfqpoint{0.327750in}{2.350593in}}%
\pgfpathlineto{\pgfqpoint{0.363400in}{2.350593in}}%
\pgfpathlineto{\pgfqpoint{0.399050in}{2.350593in}}%
\pgfpathlineto{\pgfqpoint{0.434700in}{2.350593in}}%
\pgfpathlineto{\pgfqpoint{0.470350in}{2.350593in}}%
\pgfpathlineto{\pgfqpoint{0.506000in}{2.350593in}}%
\pgfpathlineto{\pgfqpoint{0.541650in}{2.350593in}}%
\pgfpathlineto{\pgfqpoint{0.577300in}{2.350593in}}%
\pgfpathlineto{\pgfqpoint{0.612950in}{2.350593in}}%
\pgfpathlineto{\pgfqpoint{0.648600in}{2.350593in}}%
\pgfpathlineto{\pgfqpoint{0.684250in}{2.350593in}}%
\pgfpathlineto{\pgfqpoint{0.719900in}{2.350593in}}%
\pgfpathlineto{\pgfqpoint{0.755550in}{2.350593in}}%
\pgfpathlineto{\pgfqpoint{0.791200in}{2.350593in}}%
\pgfpathlineto{\pgfqpoint{0.826850in}{2.335272in}}%
\pgfpathlineto{\pgfqpoint{0.862500in}{2.335272in}}%
\pgfpathlineto{\pgfqpoint{0.898150in}{2.335272in}}%
\pgfpathlineto{\pgfqpoint{0.933800in}{2.335272in}}%
\pgfpathlineto{\pgfqpoint{0.969450in}{2.335272in}}%
\pgfpathlineto{\pgfqpoint{1.005100in}{2.335272in}}%
\pgfpathlineto{\pgfqpoint{1.040750in}{2.335272in}}%
\pgfpathlineto{\pgfqpoint{1.076400in}{2.332171in}}%
\pgfpathlineto{\pgfqpoint{1.112050in}{2.332171in}}%
\pgfpathlineto{\pgfqpoint{1.147700in}{2.328337in}}%
\pgfpathlineto{\pgfqpoint{1.183350in}{2.314104in}}%
\pgfpathlineto{\pgfqpoint{1.219000in}{2.235940in}}%
\pgfpathlineto{\pgfqpoint{1.254650in}{2.235940in}}%
\pgfpathlineto{\pgfqpoint{1.290300in}{2.235940in}}%
\pgfpathlineto{\pgfqpoint{1.325950in}{2.235940in}}%
\pgfpathlineto{\pgfqpoint{1.361600in}{2.235940in}}%
\pgfpathlineto{\pgfqpoint{1.397250in}{2.235940in}}%
\pgfpathlineto{\pgfqpoint{1.432900in}{2.235940in}}%
\pgfpathlineto{\pgfqpoint{1.468550in}{2.235940in}}%
\pgfpathlineto{\pgfqpoint{1.504200in}{2.235940in}}%
\pgfpathlineto{\pgfqpoint{1.539850in}{2.235940in}}%
\pgfpathlineto{\pgfqpoint{1.575500in}{2.235940in}}%
\pgfpathlineto{\pgfqpoint{1.611150in}{2.235940in}}%
\pgfpathlineto{\pgfqpoint{1.646800in}{2.235940in}}%
\pgfpathlineto{\pgfqpoint{1.682450in}{2.235940in}}%
\pgfpathlineto{\pgfqpoint{1.718100in}{2.235940in}}%
\pgfpathlineto{\pgfqpoint{1.753750in}{2.235940in}}%
\pgfpathlineto{\pgfqpoint{1.789400in}{2.235940in}}%
\pgfpathlineto{\pgfqpoint{1.825050in}{2.235940in}}%
\pgfpathlineto{\pgfqpoint{1.860700in}{2.235940in}}%
\pgfpathlineto{\pgfqpoint{1.896350in}{2.235940in}}%
\pgfpathlineto{\pgfqpoint{1.932000in}{2.235940in}}%
\pgfpathlineto{\pgfqpoint{1.967650in}{2.235940in}}%
\pgfpathlineto{\pgfqpoint{2.003300in}{2.235940in}}%
\pgfpathlineto{\pgfqpoint{2.038950in}{2.235940in}}%
\pgfpathlineto{\pgfqpoint{2.074600in}{2.235940in}}%
\pgfpathlineto{\pgfqpoint{2.110250in}{2.235940in}}%
\pgfpathlineto{\pgfqpoint{2.145900in}{2.231831in}}%
\pgfpathlineto{\pgfqpoint{2.181550in}{2.231831in}}%
\pgfpathlineto{\pgfqpoint{2.217200in}{2.231831in}}%
\pgfpathlineto{\pgfqpoint{2.252850in}{2.229131in}}%
\pgfpathlineto{\pgfqpoint{2.288500in}{2.229131in}}%
\pgfpathlineto{\pgfqpoint{2.324150in}{2.229131in}}%
\pgfpathlineto{\pgfqpoint{2.359800in}{2.229131in}}%
\pgfpathlineto{\pgfqpoint{2.395450in}{2.229131in}}%
\pgfpathlineto{\pgfqpoint{2.431100in}{2.229131in}}%
\pgfpathlineto{\pgfqpoint{2.466750in}{2.229131in}}%
\pgfpathlineto{\pgfqpoint{2.502400in}{2.229131in}}%
\pgfpathlineto{\pgfqpoint{2.538050in}{2.229131in}}%
\pgfpathlineto{\pgfqpoint{2.573700in}{2.229131in}}%
\pgfpathlineto{\pgfqpoint{2.609350in}{2.229131in}}%
\pgfpathlineto{\pgfqpoint{2.645000in}{2.229131in}}%
\pgfpathlineto{\pgfqpoint{2.680650in}{2.229131in}}%
\pgfpathlineto{\pgfqpoint{2.716300in}{2.229131in}}%
\pgfpathlineto{\pgfqpoint{2.751950in}{2.229131in}}%
\pgfpathlineto{\pgfqpoint{2.787600in}{2.229131in}}%
\pgfpathlineto{\pgfqpoint{2.823250in}{2.229131in}}%
\pgfpathlineto{\pgfqpoint{2.858900in}{2.229131in}}%
\pgfpathlineto{\pgfqpoint{2.894550in}{2.229131in}}%
\pgfpathlineto{\pgfqpoint{2.930200in}{2.229131in}}%
\pgfpathlineto{\pgfqpoint{2.965850in}{2.229131in}}%
\pgfpathlineto{\pgfqpoint{3.001500in}{2.229131in}}%
\pgfpathlineto{\pgfqpoint{3.037150in}{2.229131in}}%
\pgfpathlineto{\pgfqpoint{3.072800in}{2.229131in}}%
\pgfpathlineto{\pgfqpoint{3.108450in}{2.229131in}}%
\pgfpathlineto{\pgfqpoint{3.144100in}{2.229131in}}%
\pgfpathlineto{\pgfqpoint{3.179750in}{2.229131in}}%
\pgfpathlineto{\pgfqpoint{3.215400in}{2.229131in}}%
\pgfpathlineto{\pgfqpoint{3.251050in}{2.229131in}}%
\pgfpathlineto{\pgfqpoint{3.286700in}{2.225427in}}%
\pgfpathlineto{\pgfqpoint{3.322350in}{2.225427in}}%
\pgfpathlineto{\pgfqpoint{3.358000in}{2.174890in}}%
\pgfpathlineto{\pgfqpoint{3.393650in}{2.174890in}}%
\pgfpathlineto{\pgfqpoint{3.429300in}{2.174890in}}%
\pgfpathlineto{\pgfqpoint{3.464950in}{2.174890in}}%
\pgfpathlineto{\pgfqpoint{3.500600in}{2.174890in}}%
\pgfpathlineto{\pgfqpoint{3.536250in}{2.174890in}}%
\pgfpathlineto{\pgfqpoint{3.571900in}{2.174890in}}%
\pgfpathlineto{\pgfqpoint{3.607550in}{2.174890in}}%
\pgfpathlineto{\pgfqpoint{3.643200in}{2.174890in}}%
\pgfpathlineto{\pgfqpoint{3.678850in}{2.174890in}}%
\pgfpathlineto{\pgfqpoint{3.714500in}{2.174890in}}%
\pgfpathlineto{\pgfqpoint{3.750150in}{2.174890in}}%
\pgfpathlineto{\pgfqpoint{3.785800in}{2.174890in}}%
\pgfpathlineto{\pgfqpoint{3.821450in}{2.174890in}}%
\pgfpathlineto{\pgfqpoint{3.857100in}{2.174890in}}%
\pgfpathlineto{\pgfqpoint{3.892750in}{2.174890in}}%
\pgfpathlineto{\pgfqpoint{3.928400in}{2.174890in}}%
\pgfpathlineto{\pgfqpoint{3.964050in}{2.174890in}}%
\pgfpathlineto{\pgfqpoint{3.999700in}{2.174890in}}%
\pgfpathlineto{\pgfqpoint{4.035350in}{2.085473in}}%
\pgfpathlineto{\pgfqpoint{4.071000in}{2.085473in}}%
\pgfpathlineto{\pgfqpoint{4.106650in}{2.085473in}}%
\pgfpathlineto{\pgfqpoint{4.142300in}{2.085473in}}%
\pgfpathlineto{\pgfqpoint{4.177950in}{2.085473in}}%
\pgfpathlineto{\pgfqpoint{4.213600in}{2.085473in}}%
\pgfpathlineto{\pgfqpoint{4.249250in}{2.085473in}}%
\pgfpathlineto{\pgfqpoint{4.284900in}{2.085473in}}%
\pgfpathlineto{\pgfqpoint{4.320550in}{2.085473in}}%
\pgfpathlineto{\pgfqpoint{4.356200in}{2.085473in}}%
\pgfpathlineto{\pgfqpoint{4.391850in}{2.085473in}}%
\pgfpathlineto{\pgfqpoint{4.427500in}{2.085473in}}%
\pgfpathlineto{\pgfqpoint{4.463150in}{2.085473in}}%
\pgfpathlineto{\pgfqpoint{4.498800in}{2.085473in}}%
\pgfpathlineto{\pgfqpoint{4.534450in}{2.085473in}}%
\pgfpathlineto{\pgfqpoint{4.570100in}{2.085473in}}%
\pgfpathlineto{\pgfqpoint{4.605750in}{2.085473in}}%
\pgfpathlineto{\pgfqpoint{4.641400in}{2.085473in}}%
\pgfpathlineto{\pgfqpoint{4.677050in}{2.085473in}}%
\pgfpathlineto{\pgfqpoint{4.712700in}{2.085473in}}%
\pgfpathlineto{\pgfqpoint{4.748350in}{2.085473in}}%
\pgfpathlineto{\pgfqpoint{4.784000in}{2.085473in}}%
\pgfpathlineto{\pgfqpoint{4.819650in}{2.085473in}}%
\pgfpathlineto{\pgfqpoint{4.855300in}{2.085473in}}%
\pgfpathlineto{\pgfqpoint{4.890950in}{2.085473in}}%
\pgfpathlineto{\pgfqpoint{4.926600in}{2.085473in}}%
\pgfpathlineto{\pgfqpoint{4.962250in}{2.085473in}}%
\pgfpathlineto{\pgfqpoint{4.997900in}{2.085473in}}%
\pgfpathlineto{\pgfqpoint{5.033550in}{2.085473in}}%
\pgfpathlineto{\pgfqpoint{5.069200in}{2.085473in}}%
\pgfpathlineto{\pgfqpoint{5.104850in}{2.085473in}}%
\pgfpathlineto{\pgfqpoint{5.140500in}{2.085473in}}%
\pgfpathlineto{\pgfqpoint{5.176150in}{2.085473in}}%
\pgfpathlineto{\pgfqpoint{5.211800in}{2.085473in}}%
\pgfpathlineto{\pgfqpoint{5.247450in}{2.085473in}}%
\pgfpathlineto{\pgfqpoint{5.283100in}{2.085473in}}%
\pgfpathlineto{\pgfqpoint{5.318750in}{2.085473in}}%
\pgfpathlineto{\pgfqpoint{5.354400in}{2.085473in}}%
\pgfpathlineto{\pgfqpoint{5.390050in}{2.085473in}}%
\pgfpathlineto{\pgfqpoint{5.425700in}{2.085473in}}%
\pgfpathlineto{\pgfqpoint{5.461350in}{2.085473in}}%
\pgfpathlineto{\pgfqpoint{5.497000in}{2.085473in}}%
\pgfpathlineto{\pgfqpoint{5.532650in}{2.085473in}}%
\pgfpathlineto{\pgfqpoint{5.568300in}{2.085473in}}%
\pgfpathlineto{\pgfqpoint{5.603950in}{2.085473in}}%
\pgfpathlineto{\pgfqpoint{5.639600in}{2.085473in}}%
\pgfpathlineto{\pgfqpoint{5.675250in}{2.085473in}}%
\pgfpathlineto{\pgfqpoint{5.710900in}{2.085473in}}%
\pgfpathlineto{\pgfqpoint{5.746550in}{2.085473in}}%
\pgfpathlineto{\pgfqpoint{5.782200in}{2.077549in}}%
\pgfpathlineto{\pgfqpoint{5.817850in}{2.077549in}}%
\pgfpathlineto{\pgfqpoint{5.853500in}{2.077549in}}%
\pgfpathlineto{\pgfqpoint{5.889150in}{2.077549in}}%
\pgfpathlineto{\pgfqpoint{5.924800in}{2.077549in}}%
\pgfpathlineto{\pgfqpoint{5.960450in}{2.077549in}}%
\pgfpathlineto{\pgfqpoint{5.996100in}{2.077549in}}%
\pgfpathlineto{\pgfqpoint{6.031750in}{2.077549in}}%
\pgfpathlineto{\pgfqpoint{6.067400in}{2.077549in}}%
\pgfpathlineto{\pgfqpoint{6.103050in}{2.077549in}}%
\pgfpathlineto{\pgfqpoint{6.138700in}{2.077549in}}%
\pgfpathlineto{\pgfqpoint{6.174350in}{2.077549in}}%
\pgfpathlineto{\pgfqpoint{6.210000in}{2.077549in}}%
\pgfpathlineto{\pgfqpoint{6.245650in}{2.077549in}}%
\pgfpathlineto{\pgfqpoint{6.281300in}{2.077549in}}%
\pgfpathlineto{\pgfqpoint{6.316950in}{2.032807in}}%
\pgfpathlineto{\pgfqpoint{6.352600in}{2.032807in}}%
\pgfpathlineto{\pgfqpoint{6.388250in}{2.032807in}}%
\pgfpathlineto{\pgfqpoint{6.423900in}{2.032807in}}%
\pgfpathlineto{\pgfqpoint{6.459550in}{2.032807in}}%
\pgfpathlineto{\pgfqpoint{6.495200in}{2.032807in}}%
\pgfpathlineto{\pgfqpoint{6.530850in}{2.032807in}}%
\pgfpathlineto{\pgfqpoint{6.566500in}{2.020580in}}%
\pgfpathlineto{\pgfqpoint{6.602150in}{2.020580in}}%
\pgfpathlineto{\pgfqpoint{6.637800in}{2.020580in}}%
\pgfpathlineto{\pgfqpoint{6.673450in}{2.020580in}}%
\pgfpathlineto{\pgfqpoint{6.709100in}{2.020580in}}%
\pgfpathlineto{\pgfqpoint{6.744750in}{2.020580in}}%
\pgfpathlineto{\pgfqpoint{6.780400in}{2.020580in}}%
\pgfpathlineto{\pgfqpoint{6.816050in}{2.020061in}}%
\pgfpathlineto{\pgfqpoint{6.851700in}{2.020061in}}%
\pgfpathlineto{\pgfqpoint{6.887350in}{2.020061in}}%
\pgfpathlineto{\pgfqpoint{6.923000in}{2.020061in}}%
\pgfpathlineto{\pgfqpoint{6.958650in}{2.020061in}}%
\pgfpathlineto{\pgfqpoint{6.994300in}{2.020061in}}%
\pgfpathlineto{\pgfqpoint{7.029950in}{2.020061in}}%
\pgfpathlineto{\pgfqpoint{7.065600in}{2.020061in}}%
\pgfpathlineto{\pgfqpoint{7.101250in}{2.020061in}}%
\pgfpathlineto{\pgfqpoint{7.136900in}{2.020061in}}%
\pgfpathlineto{\pgfqpoint{7.172550in}{2.020061in}}%
\pgfpathlineto{\pgfqpoint{7.208200in}{2.020061in}}%
\pgfpathlineto{\pgfqpoint{7.243850in}{1.998415in}}%
\pgfpathlineto{\pgfqpoint{7.279500in}{1.962969in}}%
\pgfpathlineto{\pgfqpoint{7.315150in}{1.962969in}}%
\pgfpathlineto{\pgfqpoint{7.350800in}{1.962969in}}%
\pgfpathlineto{\pgfqpoint{7.386450in}{1.962969in}}%
\pgfpathlineto{\pgfqpoint{7.422100in}{1.962969in}}%
\pgfpathlineto{\pgfqpoint{7.457750in}{1.962969in}}%
\pgfpathlineto{\pgfqpoint{7.493400in}{1.962969in}}%
\pgfpathlineto{\pgfqpoint{7.529050in}{1.962969in}}%
\pgfpathlineto{\pgfqpoint{7.564700in}{1.962969in}}%
\pgfpathlineto{\pgfqpoint{7.600350in}{1.962969in}}%
\pgfpathlineto{\pgfqpoint{7.636000in}{1.962969in}}%
\pgfpathlineto{\pgfqpoint{7.671650in}{1.962969in}}%
\pgfpathlineto{\pgfqpoint{7.707300in}{1.962969in}}%
\pgfpathlineto{\pgfqpoint{7.742950in}{1.962969in}}%
\pgfpathlineto{\pgfqpoint{7.778600in}{1.962969in}}%
\pgfpathlineto{\pgfqpoint{7.814250in}{1.962969in}}%
\pgfpathlineto{\pgfqpoint{7.849900in}{1.962969in}}%
\pgfpathlineto{\pgfqpoint{7.885550in}{1.962969in}}%
\pgfpathlineto{\pgfqpoint{7.921200in}{1.962969in}}%
\pgfpathlineto{\pgfqpoint{7.956850in}{1.962969in}}%
\pgfpathlineto{\pgfqpoint{7.956850in}{1.645365in}}%
\pgfpathlineto{\pgfqpoint{7.956850in}{1.645365in}}%
\pgfpathlineto{\pgfqpoint{7.921200in}{1.645365in}}%
\pgfpathlineto{\pgfqpoint{7.885550in}{1.645365in}}%
\pgfpathlineto{\pgfqpoint{7.849900in}{1.645365in}}%
\pgfpathlineto{\pgfqpoint{7.814250in}{1.645365in}}%
\pgfpathlineto{\pgfqpoint{7.778600in}{1.645365in}}%
\pgfpathlineto{\pgfqpoint{7.742950in}{1.645365in}}%
\pgfpathlineto{\pgfqpoint{7.707300in}{1.645365in}}%
\pgfpathlineto{\pgfqpoint{7.671650in}{1.645365in}}%
\pgfpathlineto{\pgfqpoint{7.636000in}{1.645365in}}%
\pgfpathlineto{\pgfqpoint{7.600350in}{1.645365in}}%
\pgfpathlineto{\pgfqpoint{7.564700in}{1.645365in}}%
\pgfpathlineto{\pgfqpoint{7.529050in}{1.645365in}}%
\pgfpathlineto{\pgfqpoint{7.493400in}{1.645365in}}%
\pgfpathlineto{\pgfqpoint{7.457750in}{1.645365in}}%
\pgfpathlineto{\pgfqpoint{7.422100in}{1.645365in}}%
\pgfpathlineto{\pgfqpoint{7.386450in}{1.645365in}}%
\pgfpathlineto{\pgfqpoint{7.350800in}{1.645365in}}%
\pgfpathlineto{\pgfqpoint{7.315150in}{1.645365in}}%
\pgfpathlineto{\pgfqpoint{7.279500in}{1.645365in}}%
\pgfpathlineto{\pgfqpoint{7.243850in}{1.714312in}}%
\pgfpathlineto{\pgfqpoint{7.208200in}{1.738984in}}%
\pgfpathlineto{\pgfqpoint{7.172550in}{1.738984in}}%
\pgfpathlineto{\pgfqpoint{7.136900in}{1.738984in}}%
\pgfpathlineto{\pgfqpoint{7.101250in}{1.738984in}}%
\pgfpathlineto{\pgfqpoint{7.065600in}{1.738984in}}%
\pgfpathlineto{\pgfqpoint{7.029950in}{1.738984in}}%
\pgfpathlineto{\pgfqpoint{6.994300in}{1.738984in}}%
\pgfpathlineto{\pgfqpoint{6.958650in}{1.738984in}}%
\pgfpathlineto{\pgfqpoint{6.923000in}{1.738984in}}%
\pgfpathlineto{\pgfqpoint{6.887350in}{1.738984in}}%
\pgfpathlineto{\pgfqpoint{6.851700in}{1.738984in}}%
\pgfpathlineto{\pgfqpoint{6.816050in}{1.738984in}}%
\pgfpathlineto{\pgfqpoint{6.780400in}{1.740043in}}%
\pgfpathlineto{\pgfqpoint{6.744750in}{1.740043in}}%
\pgfpathlineto{\pgfqpoint{6.709100in}{1.740043in}}%
\pgfpathlineto{\pgfqpoint{6.673450in}{1.740043in}}%
\pgfpathlineto{\pgfqpoint{6.637800in}{1.740043in}}%
\pgfpathlineto{\pgfqpoint{6.602150in}{1.740043in}}%
\pgfpathlineto{\pgfqpoint{6.566500in}{1.740043in}}%
\pgfpathlineto{\pgfqpoint{6.530850in}{1.776730in}}%
\pgfpathlineto{\pgfqpoint{6.495200in}{1.776730in}}%
\pgfpathlineto{\pgfqpoint{6.459550in}{1.776730in}}%
\pgfpathlineto{\pgfqpoint{6.423900in}{1.776730in}}%
\pgfpathlineto{\pgfqpoint{6.388250in}{1.776730in}}%
\pgfpathlineto{\pgfqpoint{6.352600in}{1.776730in}}%
\pgfpathlineto{\pgfqpoint{6.316950in}{1.776730in}}%
\pgfpathlineto{\pgfqpoint{6.281300in}{1.827473in}}%
\pgfpathlineto{\pgfqpoint{6.245650in}{1.827473in}}%
\pgfpathlineto{\pgfqpoint{6.210000in}{1.827473in}}%
\pgfpathlineto{\pgfqpoint{6.174350in}{1.827473in}}%
\pgfpathlineto{\pgfqpoint{6.138700in}{1.827473in}}%
\pgfpathlineto{\pgfqpoint{6.103050in}{1.827473in}}%
\pgfpathlineto{\pgfqpoint{6.067400in}{1.827473in}}%
\pgfpathlineto{\pgfqpoint{6.031750in}{1.827473in}}%
\pgfpathlineto{\pgfqpoint{5.996100in}{1.827473in}}%
\pgfpathlineto{\pgfqpoint{5.960450in}{1.827473in}}%
\pgfpathlineto{\pgfqpoint{5.924800in}{1.827473in}}%
\pgfpathlineto{\pgfqpoint{5.889150in}{1.827473in}}%
\pgfpathlineto{\pgfqpoint{5.853500in}{1.827473in}}%
\pgfpathlineto{\pgfqpoint{5.817850in}{1.827473in}}%
\pgfpathlineto{\pgfqpoint{5.782200in}{1.827473in}}%
\pgfpathlineto{\pgfqpoint{5.746550in}{1.835788in}}%
\pgfpathlineto{\pgfqpoint{5.710900in}{1.835788in}}%
\pgfpathlineto{\pgfqpoint{5.675250in}{1.835788in}}%
\pgfpathlineto{\pgfqpoint{5.639600in}{1.835788in}}%
\pgfpathlineto{\pgfqpoint{5.603950in}{1.835788in}}%
\pgfpathlineto{\pgfqpoint{5.568300in}{1.835788in}}%
\pgfpathlineto{\pgfqpoint{5.532650in}{1.835788in}}%
\pgfpathlineto{\pgfqpoint{5.497000in}{1.835788in}}%
\pgfpathlineto{\pgfqpoint{5.461350in}{1.835788in}}%
\pgfpathlineto{\pgfqpoint{5.425700in}{1.835788in}}%
\pgfpathlineto{\pgfqpoint{5.390050in}{1.835788in}}%
\pgfpathlineto{\pgfqpoint{5.354400in}{1.835788in}}%
\pgfpathlineto{\pgfqpoint{5.318750in}{1.835788in}}%
\pgfpathlineto{\pgfqpoint{5.283100in}{1.835788in}}%
\pgfpathlineto{\pgfqpoint{5.247450in}{1.835788in}}%
\pgfpathlineto{\pgfqpoint{5.211800in}{1.835788in}}%
\pgfpathlineto{\pgfqpoint{5.176150in}{1.835788in}}%
\pgfpathlineto{\pgfqpoint{5.140500in}{1.835788in}}%
\pgfpathlineto{\pgfqpoint{5.104850in}{1.835788in}}%
\pgfpathlineto{\pgfqpoint{5.069200in}{1.835788in}}%
\pgfpathlineto{\pgfqpoint{5.033550in}{1.835788in}}%
\pgfpathlineto{\pgfqpoint{4.997900in}{1.835788in}}%
\pgfpathlineto{\pgfqpoint{4.962250in}{1.835788in}}%
\pgfpathlineto{\pgfqpoint{4.926600in}{1.835788in}}%
\pgfpathlineto{\pgfqpoint{4.890950in}{1.835788in}}%
\pgfpathlineto{\pgfqpoint{4.855300in}{1.835788in}}%
\pgfpathlineto{\pgfqpoint{4.819650in}{1.835788in}}%
\pgfpathlineto{\pgfqpoint{4.784000in}{1.835788in}}%
\pgfpathlineto{\pgfqpoint{4.748350in}{1.835788in}}%
\pgfpathlineto{\pgfqpoint{4.712700in}{1.835788in}}%
\pgfpathlineto{\pgfqpoint{4.677050in}{1.835788in}}%
\pgfpathlineto{\pgfqpoint{4.641400in}{1.835788in}}%
\pgfpathlineto{\pgfqpoint{4.605750in}{1.835788in}}%
\pgfpathlineto{\pgfqpoint{4.570100in}{1.835788in}}%
\pgfpathlineto{\pgfqpoint{4.534450in}{1.835788in}}%
\pgfpathlineto{\pgfqpoint{4.498800in}{1.835788in}}%
\pgfpathlineto{\pgfqpoint{4.463150in}{1.835788in}}%
\pgfpathlineto{\pgfqpoint{4.427500in}{1.835788in}}%
\pgfpathlineto{\pgfqpoint{4.391850in}{1.835788in}}%
\pgfpathlineto{\pgfqpoint{4.356200in}{1.835788in}}%
\pgfpathlineto{\pgfqpoint{4.320550in}{1.835788in}}%
\pgfpathlineto{\pgfqpoint{4.284900in}{1.835788in}}%
\pgfpathlineto{\pgfqpoint{4.249250in}{1.835788in}}%
\pgfpathlineto{\pgfqpoint{4.213600in}{1.835788in}}%
\pgfpathlineto{\pgfqpoint{4.177950in}{1.835788in}}%
\pgfpathlineto{\pgfqpoint{4.142300in}{1.835788in}}%
\pgfpathlineto{\pgfqpoint{4.106650in}{1.835788in}}%
\pgfpathlineto{\pgfqpoint{4.071000in}{1.835788in}}%
\pgfpathlineto{\pgfqpoint{4.035350in}{1.835788in}}%
\pgfpathlineto{\pgfqpoint{3.999700in}{1.835489in}}%
\pgfpathlineto{\pgfqpoint{3.964050in}{1.835489in}}%
\pgfpathlineto{\pgfqpoint{3.928400in}{1.835489in}}%
\pgfpathlineto{\pgfqpoint{3.892750in}{1.835489in}}%
\pgfpathlineto{\pgfqpoint{3.857100in}{1.835489in}}%
\pgfpathlineto{\pgfqpoint{3.821450in}{1.835489in}}%
\pgfpathlineto{\pgfqpoint{3.785800in}{1.835489in}}%
\pgfpathlineto{\pgfqpoint{3.750150in}{1.835489in}}%
\pgfpathlineto{\pgfqpoint{3.714500in}{1.835489in}}%
\pgfpathlineto{\pgfqpoint{3.678850in}{1.835489in}}%
\pgfpathlineto{\pgfqpoint{3.643200in}{1.835489in}}%
\pgfpathlineto{\pgfqpoint{3.607550in}{1.835489in}}%
\pgfpathlineto{\pgfqpoint{3.571900in}{1.835489in}}%
\pgfpathlineto{\pgfqpoint{3.536250in}{1.835489in}}%
\pgfpathlineto{\pgfqpoint{3.500600in}{1.835489in}}%
\pgfpathlineto{\pgfqpoint{3.464950in}{1.835489in}}%
\pgfpathlineto{\pgfqpoint{3.429300in}{1.835489in}}%
\pgfpathlineto{\pgfqpoint{3.393650in}{1.835489in}}%
\pgfpathlineto{\pgfqpoint{3.358000in}{1.835489in}}%
\pgfpathlineto{\pgfqpoint{3.322350in}{1.858552in}}%
\pgfpathlineto{\pgfqpoint{3.286700in}{1.858552in}}%
\pgfpathlineto{\pgfqpoint{3.251050in}{1.869562in}}%
\pgfpathlineto{\pgfqpoint{3.215400in}{1.869562in}}%
\pgfpathlineto{\pgfqpoint{3.179750in}{1.869562in}}%
\pgfpathlineto{\pgfqpoint{3.144100in}{1.869562in}}%
\pgfpathlineto{\pgfqpoint{3.108450in}{1.869562in}}%
\pgfpathlineto{\pgfqpoint{3.072800in}{1.869562in}}%
\pgfpathlineto{\pgfqpoint{3.037150in}{1.869562in}}%
\pgfpathlineto{\pgfqpoint{3.001500in}{1.869562in}}%
\pgfpathlineto{\pgfqpoint{2.965850in}{1.869562in}}%
\pgfpathlineto{\pgfqpoint{2.930200in}{1.869562in}}%
\pgfpathlineto{\pgfqpoint{2.894550in}{1.869562in}}%
\pgfpathlineto{\pgfqpoint{2.858900in}{1.869562in}}%
\pgfpathlineto{\pgfqpoint{2.823250in}{1.869562in}}%
\pgfpathlineto{\pgfqpoint{2.787600in}{1.869562in}}%
\pgfpathlineto{\pgfqpoint{2.751950in}{1.869562in}}%
\pgfpathlineto{\pgfqpoint{2.716300in}{1.869562in}}%
\pgfpathlineto{\pgfqpoint{2.680650in}{1.869562in}}%
\pgfpathlineto{\pgfqpoint{2.645000in}{1.869562in}}%
\pgfpathlineto{\pgfqpoint{2.609350in}{1.869562in}}%
\pgfpathlineto{\pgfqpoint{2.573700in}{1.869562in}}%
\pgfpathlineto{\pgfqpoint{2.538050in}{1.869562in}}%
\pgfpathlineto{\pgfqpoint{2.502400in}{1.869562in}}%
\pgfpathlineto{\pgfqpoint{2.466750in}{1.869562in}}%
\pgfpathlineto{\pgfqpoint{2.431100in}{1.869562in}}%
\pgfpathlineto{\pgfqpoint{2.395450in}{1.869562in}}%
\pgfpathlineto{\pgfqpoint{2.359800in}{1.869562in}}%
\pgfpathlineto{\pgfqpoint{2.324150in}{1.869562in}}%
\pgfpathlineto{\pgfqpoint{2.288500in}{1.869562in}}%
\pgfpathlineto{\pgfqpoint{2.252850in}{1.869562in}}%
\pgfpathlineto{\pgfqpoint{2.217200in}{1.877534in}}%
\pgfpathlineto{\pgfqpoint{2.181550in}{1.877534in}}%
\pgfpathlineto{\pgfqpoint{2.145900in}{1.877534in}}%
\pgfpathlineto{\pgfqpoint{2.110250in}{1.884229in}}%
\pgfpathlineto{\pgfqpoint{2.074600in}{1.884229in}}%
\pgfpathlineto{\pgfqpoint{2.038950in}{1.884229in}}%
\pgfpathlineto{\pgfqpoint{2.003300in}{1.884229in}}%
\pgfpathlineto{\pgfqpoint{1.967650in}{1.884229in}}%
\pgfpathlineto{\pgfqpoint{1.932000in}{1.884229in}}%
\pgfpathlineto{\pgfqpoint{1.896350in}{1.884229in}}%
\pgfpathlineto{\pgfqpoint{1.860700in}{1.884229in}}%
\pgfpathlineto{\pgfqpoint{1.825050in}{1.884229in}}%
\pgfpathlineto{\pgfqpoint{1.789400in}{1.884229in}}%
\pgfpathlineto{\pgfqpoint{1.753750in}{1.884229in}}%
\pgfpathlineto{\pgfqpoint{1.718100in}{1.884229in}}%
\pgfpathlineto{\pgfqpoint{1.682450in}{1.884229in}}%
\pgfpathlineto{\pgfqpoint{1.646800in}{1.884229in}}%
\pgfpathlineto{\pgfqpoint{1.611150in}{1.884229in}}%
\pgfpathlineto{\pgfqpoint{1.575500in}{1.884229in}}%
\pgfpathlineto{\pgfqpoint{1.539850in}{1.884229in}}%
\pgfpathlineto{\pgfqpoint{1.504200in}{1.884229in}}%
\pgfpathlineto{\pgfqpoint{1.468550in}{1.884229in}}%
\pgfpathlineto{\pgfqpoint{1.432900in}{1.884229in}}%
\pgfpathlineto{\pgfqpoint{1.397250in}{1.884229in}}%
\pgfpathlineto{\pgfqpoint{1.361600in}{1.884229in}}%
\pgfpathlineto{\pgfqpoint{1.325950in}{1.884229in}}%
\pgfpathlineto{\pgfqpoint{1.290300in}{1.884229in}}%
\pgfpathlineto{\pgfqpoint{1.254650in}{1.884229in}}%
\pgfpathlineto{\pgfqpoint{1.219000in}{1.884229in}}%
\pgfpathlineto{\pgfqpoint{1.183350in}{1.960263in}}%
\pgfpathlineto{\pgfqpoint{1.147700in}{1.981740in}}%
\pgfpathlineto{\pgfqpoint{1.112050in}{1.994245in}}%
\pgfpathlineto{\pgfqpoint{1.076400in}{1.994245in}}%
\pgfpathlineto{\pgfqpoint{1.040750in}{1.994874in}}%
\pgfpathlineto{\pgfqpoint{1.005100in}{1.994874in}}%
\pgfpathlineto{\pgfqpoint{0.969450in}{1.994874in}}%
\pgfpathlineto{\pgfqpoint{0.933800in}{1.994874in}}%
\pgfpathlineto{\pgfqpoint{0.898150in}{1.994874in}}%
\pgfpathlineto{\pgfqpoint{0.862500in}{1.994874in}}%
\pgfpathlineto{\pgfqpoint{0.826850in}{1.994874in}}%
\pgfpathlineto{\pgfqpoint{0.791200in}{2.027347in}}%
\pgfpathlineto{\pgfqpoint{0.755550in}{2.027347in}}%
\pgfpathlineto{\pgfqpoint{0.719900in}{2.027347in}}%
\pgfpathlineto{\pgfqpoint{0.684250in}{2.027347in}}%
\pgfpathlineto{\pgfqpoint{0.648600in}{2.027347in}}%
\pgfpathlineto{\pgfqpoint{0.612950in}{2.027347in}}%
\pgfpathlineto{\pgfqpoint{0.577300in}{2.027347in}}%
\pgfpathlineto{\pgfqpoint{0.541650in}{2.027347in}}%
\pgfpathlineto{\pgfqpoint{0.506000in}{2.027347in}}%
\pgfpathlineto{\pgfqpoint{0.470350in}{2.027347in}}%
\pgfpathlineto{\pgfqpoint{0.434700in}{2.027347in}}%
\pgfpathlineto{\pgfqpoint{0.399050in}{2.027347in}}%
\pgfpathlineto{\pgfqpoint{0.363400in}{2.027347in}}%
\pgfpathlineto{\pgfqpoint{0.327750in}{2.027347in}}%
\pgfpathlineto{\pgfqpoint{0.292100in}{2.027347in}}%
\pgfpathlineto{\pgfqpoint{0.256450in}{2.027347in}}%
\pgfpathlineto{\pgfqpoint{0.220800in}{2.027347in}}%
\pgfpathlineto{\pgfqpoint{0.185150in}{2.027347in}}%
\pgfpathlineto{\pgfqpoint{0.149500in}{2.027347in}}%
\pgfpathlineto{\pgfqpoint{0.113850in}{2.027347in}}%
\pgfpathlineto{\pgfqpoint{0.078200in}{2.027347in}}%
\pgfpathlineto{\pgfqpoint{0.042550in}{2.027347in}}%
\pgfpathlineto{\pgfqpoint{0.006900in}{2.111770in}}%
\pgfpathlineto{\pgfqpoint{-0.028750in}{2.111770in}}%
\pgfpathlineto{\pgfqpoint{-0.064400in}{2.111770in}}%
\pgfpathlineto{\pgfqpoint{-0.100050in}{2.111770in}}%
\pgfpathlineto{\pgfqpoint{-0.135700in}{2.111770in}}%
\pgfpathlineto{\pgfqpoint{-0.171350in}{2.111770in}}%
\pgfpathlineto{\pgfqpoint{-0.207000in}{2.111770in}}%
\pgfpathlineto{\pgfqpoint{-0.242650in}{2.111770in}}%
\pgfpathlineto{\pgfqpoint{-0.278300in}{2.111770in}}%
\pgfpathlineto{\pgfqpoint{-0.313950in}{2.111770in}}%
\pgfpathlineto{\pgfqpoint{-0.349600in}{2.111770in}}%
\pgfpathlineto{\pgfqpoint{-0.385250in}{2.111770in}}%
\pgfpathlineto{\pgfqpoint{-0.420900in}{2.111770in}}%
\pgfpathlineto{\pgfqpoint{-0.456550in}{2.111770in}}%
\pgfpathlineto{\pgfqpoint{-0.492200in}{2.111770in}}%
\pgfpathlineto{\pgfqpoint{-0.527850in}{2.111770in}}%
\pgfpathlineto{\pgfqpoint{-0.563500in}{2.111770in}}%
\pgfpathlineto{\pgfqpoint{-0.599150in}{2.111770in}}%
\pgfpathlineto{\pgfqpoint{-0.634800in}{2.111770in}}%
\pgfpathlineto{\pgfqpoint{-0.670450in}{2.111770in}}%
\pgfpathlineto{\pgfqpoint{-0.706100in}{2.111770in}}%
\pgfpathlineto{\pgfqpoint{-0.741750in}{2.111770in}}%
\pgfpathlineto{\pgfqpoint{-0.777400in}{2.111770in}}%
\pgfpathlineto{\pgfqpoint{-0.813050in}{2.111770in}}%
\pgfpathlineto{\pgfqpoint{-0.848700in}{2.111770in}}%
\pgfpathlineto{\pgfqpoint{-0.884350in}{2.111770in}}%
\pgfpathlineto{\pgfqpoint{-0.920000in}{2.111770in}}%
\pgfpathlineto{\pgfqpoint{-0.955650in}{2.168382in}}%
\pgfpathlineto{\pgfqpoint{-0.991300in}{2.168382in}}%
\pgfpathlineto{\pgfqpoint{-1.026950in}{2.219443in}}%
\pgfpathlineto{\pgfqpoint{-1.062600in}{2.219443in}}%
\pgfpathlineto{\pgfqpoint{-1.098250in}{2.219443in}}%
\pgfpathlineto{\pgfqpoint{-1.133900in}{2.219443in}}%
\pgfpathlineto{\pgfqpoint{-1.169550in}{2.338370in}}%
\pgfpathlineto{\pgfqpoint{-1.205200in}{2.413867in}}%
\pgfpathlineto{\pgfqpoint{-1.240850in}{2.413867in}}%
\pgfpathlineto{\pgfqpoint{-1.276500in}{2.413867in}}%
\pgfpathlineto{\pgfqpoint{-1.312150in}{2.413867in}}%
\pgfpathlineto{\pgfqpoint{-1.347800in}{2.413867in}}%
\pgfpathlineto{\pgfqpoint{-1.383450in}{2.413867in}}%
\pgfpathlineto{\pgfqpoint{-1.419100in}{2.413867in}}%
\pgfpathlineto{\pgfqpoint{-1.454750in}{2.413867in}}%
\pgfpathlineto{\pgfqpoint{-1.490400in}{2.484772in}}%
\pgfpathlineto{\pgfqpoint{-1.526050in}{2.566520in}}%
\pgfpathlineto{\pgfqpoint{-1.561700in}{2.583928in}}%
\pgfpathlineto{\pgfqpoint{-1.597350in}{2.583928in}}%
\pgfpathlineto{\pgfqpoint{-1.633000in}{2.583928in}}%
\pgfpathlineto{\pgfqpoint{-1.668650in}{2.583928in}}%
\pgfpathlineto{\pgfqpoint{-1.704300in}{2.583928in}}%
\pgfpathlineto{\pgfqpoint{-1.739950in}{2.641862in}}%
\pgfpathlineto{\pgfqpoint{-1.775600in}{2.641862in}}%
\pgfpathlineto{\pgfqpoint{-1.811250in}{2.609288in}}%
\pgfpathlineto{\pgfqpoint{-1.846900in}{2.695792in}}%
\pgfpathlineto{\pgfqpoint{-1.882550in}{2.748998in}}%
\pgfpathlineto{\pgfqpoint{-1.918200in}{2.790830in}}%
\pgfpathlineto{\pgfqpoint{-1.953850in}{2.790830in}}%
\pgfpathlineto{\pgfqpoint{-1.989500in}{2.909454in}}%
\pgfpathlineto{\pgfqpoint{-2.025150in}{3.021144in}}%
\pgfpathlineto{\pgfqpoint{-2.060800in}{3.021144in}}%
\pgfpathlineto{\pgfqpoint{-2.096450in}{3.058074in}}%
\pgfpathlineto{\pgfqpoint{-2.132100in}{3.058074in}}%
\pgfpathlineto{\pgfqpoint{-2.167750in}{3.264485in}}%
\pgfpathlineto{\pgfqpoint{-2.203400in}{3.311708in}}%
\pgfpathlineto{\pgfqpoint{-2.239050in}{3.311708in}}%
\pgfpathlineto{\pgfqpoint{-2.274700in}{3.344767in}}%
\pgfpathlineto{\pgfqpoint{-2.310350in}{3.402187in}}%
\pgfpathlineto{\pgfqpoint{-2.346000in}{3.402187in}}%
\pgfpathlineto{\pgfqpoint{-2.381650in}{3.321016in}}%
\pgfpathlineto{\pgfqpoint{-2.417300in}{3.321016in}}%
\pgfpathlineto{\pgfqpoint{-2.452950in}{3.321016in}}%
\pgfpathlineto{\pgfqpoint{-2.488600in}{3.368912in}}%
\pgfpathlineto{\pgfqpoint{-2.524250in}{3.368912in}}%
\pgfpathlineto{\pgfqpoint{-2.559900in}{3.743102in}}%
\pgfpathlineto{\pgfqpoint{-2.595550in}{3.863959in}}%
\pgfpathlineto{\pgfqpoint{-2.631200in}{4.004253in}}%
\pgfpathlineto{\pgfqpoint{-2.666850in}{4.402043in}}%
\pgfpathlineto{\pgfqpoint{-2.702500in}{5.383122in}}%
\pgfpathclose%
\pgfusepath{fill}%
\end{pgfscope}%
\begin{pgfscope}%
\pgfpathrectangle{\pgfqpoint{0.862500in}{0.375000in}}{\pgfqpoint{5.347500in}{2.265000in}}%
\pgfusepath{clip}%
\pgfsetbuttcap%
\pgfsetroundjoin%
\definecolor{currentfill}{rgb}{1.000000,0.498039,0.054902}%
\pgfsetfillcolor{currentfill}%
\pgfsetfillopacity{0.200000}%
\pgfsetlinewidth{0.000000pt}%
\definecolor{currentstroke}{rgb}{0.000000,0.000000,0.000000}%
\pgfsetstrokecolor{currentstroke}%
\pgfsetdash{}{0pt}%
\pgfpathmoveto{\pgfqpoint{-2.702500in}{5.135953in}}%
\pgfpathlineto{\pgfqpoint{-2.702500in}{5.667332in}}%
\pgfpathlineto{\pgfqpoint{-2.666850in}{5.375277in}}%
\pgfpathlineto{\pgfqpoint{-2.631200in}{5.375277in}}%
\pgfpathlineto{\pgfqpoint{-2.595550in}{4.657746in}}%
\pgfpathlineto{\pgfqpoint{-2.559900in}{4.611256in}}%
\pgfpathlineto{\pgfqpoint{-2.524250in}{4.292828in}}%
\pgfpathlineto{\pgfqpoint{-2.488600in}{4.292828in}}%
\pgfpathlineto{\pgfqpoint{-2.452950in}{4.292828in}}%
\pgfpathlineto{\pgfqpoint{-2.417300in}{4.292828in}}%
\pgfpathlineto{\pgfqpoint{-2.381650in}{4.285858in}}%
\pgfpathlineto{\pgfqpoint{-2.346000in}{3.962525in}}%
\pgfpathlineto{\pgfqpoint{-2.310350in}{3.962525in}}%
\pgfpathlineto{\pgfqpoint{-2.274700in}{3.962525in}}%
\pgfpathlineto{\pgfqpoint{-2.239050in}{3.962525in}}%
\pgfpathlineto{\pgfqpoint{-2.203400in}{3.911223in}}%
\pgfpathlineto{\pgfqpoint{-2.167750in}{3.911223in}}%
\pgfpathlineto{\pgfqpoint{-2.132100in}{3.515113in}}%
\pgfpathlineto{\pgfqpoint{-2.096450in}{3.515113in}}%
\pgfpathlineto{\pgfqpoint{-2.060800in}{3.205033in}}%
\pgfpathlineto{\pgfqpoint{-2.025150in}{3.195086in}}%
\pgfpathlineto{\pgfqpoint{-1.989500in}{3.180085in}}%
\pgfpathlineto{\pgfqpoint{-1.953850in}{3.180085in}}%
\pgfpathlineto{\pgfqpoint{-1.918200in}{3.180085in}}%
\pgfpathlineto{\pgfqpoint{-1.882550in}{3.180085in}}%
\pgfpathlineto{\pgfqpoint{-1.846900in}{3.180085in}}%
\pgfpathlineto{\pgfqpoint{-1.811250in}{3.180085in}}%
\pgfpathlineto{\pgfqpoint{-1.775600in}{3.092908in}}%
\pgfpathlineto{\pgfqpoint{-1.739950in}{3.092908in}}%
\pgfpathlineto{\pgfqpoint{-1.704300in}{3.092908in}}%
\pgfpathlineto{\pgfqpoint{-1.668650in}{3.092908in}}%
\pgfpathlineto{\pgfqpoint{-1.633000in}{3.092908in}}%
\pgfpathlineto{\pgfqpoint{-1.597350in}{3.092908in}}%
\pgfpathlineto{\pgfqpoint{-1.561700in}{3.043501in}}%
\pgfpathlineto{\pgfqpoint{-1.526050in}{3.043501in}}%
\pgfpathlineto{\pgfqpoint{-1.490400in}{3.043501in}}%
\pgfpathlineto{\pgfqpoint{-1.454750in}{3.015146in}}%
\pgfpathlineto{\pgfqpoint{-1.419100in}{2.297718in}}%
\pgfpathlineto{\pgfqpoint{-1.383450in}{2.297718in}}%
\pgfpathlineto{\pgfqpoint{-1.347800in}{2.297718in}}%
\pgfpathlineto{\pgfqpoint{-1.312150in}{2.297718in}}%
\pgfpathlineto{\pgfqpoint{-1.276500in}{2.297718in}}%
\pgfpathlineto{\pgfqpoint{-1.240850in}{2.297718in}}%
\pgfpathlineto{\pgfqpoint{-1.205200in}{2.297718in}}%
\pgfpathlineto{\pgfqpoint{-1.169550in}{2.297718in}}%
\pgfpathlineto{\pgfqpoint{-1.133900in}{2.297718in}}%
\pgfpathlineto{\pgfqpoint{-1.098250in}{2.297718in}}%
\pgfpathlineto{\pgfqpoint{-1.062600in}{2.297718in}}%
\pgfpathlineto{\pgfqpoint{-1.026950in}{2.297718in}}%
\pgfpathlineto{\pgfqpoint{-0.991300in}{2.297718in}}%
\pgfpathlineto{\pgfqpoint{-0.955650in}{2.297718in}}%
\pgfpathlineto{\pgfqpoint{-0.920000in}{2.297718in}}%
\pgfpathlineto{\pgfqpoint{-0.884350in}{2.297718in}}%
\pgfpathlineto{\pgfqpoint{-0.848700in}{2.297718in}}%
\pgfpathlineto{\pgfqpoint{-0.813050in}{2.297718in}}%
\pgfpathlineto{\pgfqpoint{-0.777400in}{2.297718in}}%
\pgfpathlineto{\pgfqpoint{-0.741750in}{2.297718in}}%
\pgfpathlineto{\pgfqpoint{-0.706100in}{2.297718in}}%
\pgfpathlineto{\pgfqpoint{-0.670450in}{2.297718in}}%
\pgfpathlineto{\pgfqpoint{-0.634800in}{2.297718in}}%
\pgfpathlineto{\pgfqpoint{-0.599150in}{2.297718in}}%
\pgfpathlineto{\pgfqpoint{-0.563500in}{2.297718in}}%
\pgfpathlineto{\pgfqpoint{-0.527850in}{2.258811in}}%
\pgfpathlineto{\pgfqpoint{-0.492200in}{2.258811in}}%
\pgfpathlineto{\pgfqpoint{-0.456550in}{2.258811in}}%
\pgfpathlineto{\pgfqpoint{-0.420900in}{2.258811in}}%
\pgfpathlineto{\pgfqpoint{-0.385250in}{2.258811in}}%
\pgfpathlineto{\pgfqpoint{-0.349600in}{2.258811in}}%
\pgfpathlineto{\pgfqpoint{-0.313950in}{2.258811in}}%
\pgfpathlineto{\pgfqpoint{-0.278300in}{2.258811in}}%
\pgfpathlineto{\pgfqpoint{-0.242650in}{2.258811in}}%
\pgfpathlineto{\pgfqpoint{-0.207000in}{2.258811in}}%
\pgfpathlineto{\pgfqpoint{-0.171350in}{2.258811in}}%
\pgfpathlineto{\pgfqpoint{-0.135700in}{2.258811in}}%
\pgfpathlineto{\pgfqpoint{-0.100050in}{2.258811in}}%
\pgfpathlineto{\pgfqpoint{-0.064400in}{2.258811in}}%
\pgfpathlineto{\pgfqpoint{-0.028750in}{2.258811in}}%
\pgfpathlineto{\pgfqpoint{0.006900in}{2.258811in}}%
\pgfpathlineto{\pgfqpoint{0.042550in}{2.258811in}}%
\pgfpathlineto{\pgfqpoint{0.078200in}{2.258811in}}%
\pgfpathlineto{\pgfqpoint{0.113850in}{2.258811in}}%
\pgfpathlineto{\pgfqpoint{0.149500in}{2.258811in}}%
\pgfpathlineto{\pgfqpoint{0.185150in}{2.258811in}}%
\pgfpathlineto{\pgfqpoint{0.220800in}{2.258811in}}%
\pgfpathlineto{\pgfqpoint{0.256450in}{2.258811in}}%
\pgfpathlineto{\pgfqpoint{0.292100in}{2.258811in}}%
\pgfpathlineto{\pgfqpoint{0.327750in}{2.258811in}}%
\pgfpathlineto{\pgfqpoint{0.363400in}{2.258811in}}%
\pgfpathlineto{\pgfqpoint{0.399050in}{2.258811in}}%
\pgfpathlineto{\pgfqpoint{0.434700in}{2.258811in}}%
\pgfpathlineto{\pgfqpoint{0.470350in}{2.258811in}}%
\pgfpathlineto{\pgfqpoint{0.506000in}{2.258811in}}%
\pgfpathlineto{\pgfqpoint{0.541650in}{2.258811in}}%
\pgfpathlineto{\pgfqpoint{0.577300in}{2.258811in}}%
\pgfpathlineto{\pgfqpoint{0.612950in}{2.258811in}}%
\pgfpathlineto{\pgfqpoint{0.648600in}{2.258811in}}%
\pgfpathlineto{\pgfqpoint{0.684250in}{2.258811in}}%
\pgfpathlineto{\pgfqpoint{0.719900in}{2.258811in}}%
\pgfpathlineto{\pgfqpoint{0.755550in}{2.258811in}}%
\pgfpathlineto{\pgfqpoint{0.791200in}{2.258811in}}%
\pgfpathlineto{\pgfqpoint{0.826850in}{2.258811in}}%
\pgfpathlineto{\pgfqpoint{0.862500in}{2.258811in}}%
\pgfpathlineto{\pgfqpoint{0.898150in}{2.258811in}}%
\pgfpathlineto{\pgfqpoint{0.933800in}{2.258811in}}%
\pgfpathlineto{\pgfqpoint{0.969450in}{2.083949in}}%
\pgfpathlineto{\pgfqpoint{1.005100in}{2.083949in}}%
\pgfpathlineto{\pgfqpoint{1.040750in}{2.083949in}}%
\pgfpathlineto{\pgfqpoint{1.076400in}{2.031326in}}%
\pgfpathlineto{\pgfqpoint{1.112050in}{2.031326in}}%
\pgfpathlineto{\pgfqpoint{1.147700in}{2.031326in}}%
\pgfpathlineto{\pgfqpoint{1.183350in}{2.031326in}}%
\pgfpathlineto{\pgfqpoint{1.219000in}{2.031326in}}%
\pgfpathlineto{\pgfqpoint{1.254650in}{2.031326in}}%
\pgfpathlineto{\pgfqpoint{1.290300in}{2.031326in}}%
\pgfpathlineto{\pgfqpoint{1.325950in}{2.031326in}}%
\pgfpathlineto{\pgfqpoint{1.361600in}{2.031326in}}%
\pgfpathlineto{\pgfqpoint{1.397250in}{1.842163in}}%
\pgfpathlineto{\pgfqpoint{1.432900in}{1.842163in}}%
\pgfpathlineto{\pgfqpoint{1.468550in}{1.842163in}}%
\pgfpathlineto{\pgfqpoint{1.504200in}{1.742881in}}%
\pgfpathlineto{\pgfqpoint{1.539850in}{1.742881in}}%
\pgfpathlineto{\pgfqpoint{1.575500in}{1.742881in}}%
\pgfpathlineto{\pgfqpoint{1.611150in}{1.742881in}}%
\pgfpathlineto{\pgfqpoint{1.646800in}{1.742881in}}%
\pgfpathlineto{\pgfqpoint{1.682450in}{1.742881in}}%
\pgfpathlineto{\pgfqpoint{1.718100in}{1.716030in}}%
\pgfpathlineto{\pgfqpoint{1.753750in}{1.671399in}}%
\pgfpathlineto{\pgfqpoint{1.789400in}{1.671399in}}%
\pgfpathlineto{\pgfqpoint{1.825050in}{1.634644in}}%
\pgfpathlineto{\pgfqpoint{1.860700in}{1.634644in}}%
\pgfpathlineto{\pgfqpoint{1.896350in}{1.634644in}}%
\pgfpathlineto{\pgfqpoint{1.932000in}{1.634644in}}%
\pgfpathlineto{\pgfqpoint{1.967650in}{1.634644in}}%
\pgfpathlineto{\pgfqpoint{2.003300in}{1.634644in}}%
\pgfpathlineto{\pgfqpoint{2.038950in}{1.634644in}}%
\pgfpathlineto{\pgfqpoint{2.074600in}{1.634644in}}%
\pgfpathlineto{\pgfqpoint{2.110250in}{1.634644in}}%
\pgfpathlineto{\pgfqpoint{2.145900in}{1.634644in}}%
\pgfpathlineto{\pgfqpoint{2.181550in}{1.634644in}}%
\pgfpathlineto{\pgfqpoint{2.217200in}{1.634201in}}%
\pgfpathlineto{\pgfqpoint{2.252850in}{1.634201in}}%
\pgfpathlineto{\pgfqpoint{2.288500in}{1.634201in}}%
\pgfpathlineto{\pgfqpoint{2.324150in}{1.634201in}}%
\pgfpathlineto{\pgfqpoint{2.359800in}{1.605627in}}%
\pgfpathlineto{\pgfqpoint{2.395450in}{1.605627in}}%
\pgfpathlineto{\pgfqpoint{2.431100in}{1.585581in}}%
\pgfpathlineto{\pgfqpoint{2.466750in}{1.498159in}}%
\pgfpathlineto{\pgfqpoint{2.502400in}{1.498159in}}%
\pgfpathlineto{\pgfqpoint{2.538050in}{1.498159in}}%
\pgfpathlineto{\pgfqpoint{2.573700in}{1.477403in}}%
\pgfpathlineto{\pgfqpoint{2.609350in}{1.477403in}}%
\pgfpathlineto{\pgfqpoint{2.645000in}{1.477403in}}%
\pgfpathlineto{\pgfqpoint{2.680650in}{1.281273in}}%
\pgfpathlineto{\pgfqpoint{2.716300in}{1.281273in}}%
\pgfpathlineto{\pgfqpoint{2.751950in}{1.281273in}}%
\pgfpathlineto{\pgfqpoint{2.787600in}{1.281273in}}%
\pgfpathlineto{\pgfqpoint{2.823250in}{1.281273in}}%
\pgfpathlineto{\pgfqpoint{2.858900in}{1.270543in}}%
\pgfpathlineto{\pgfqpoint{2.894550in}{1.270543in}}%
\pgfpathlineto{\pgfqpoint{2.930200in}{1.270543in}}%
\pgfpathlineto{\pgfqpoint{2.965850in}{1.270543in}}%
\pgfpathlineto{\pgfqpoint{3.001500in}{1.270543in}}%
\pgfpathlineto{\pgfqpoint{3.037150in}{1.270543in}}%
\pgfpathlineto{\pgfqpoint{3.072800in}{1.188517in}}%
\pgfpathlineto{\pgfqpoint{3.108450in}{1.188517in}}%
\pgfpathlineto{\pgfqpoint{3.144100in}{1.188517in}}%
\pgfpathlineto{\pgfqpoint{3.179750in}{1.188517in}}%
\pgfpathlineto{\pgfqpoint{3.215400in}{1.145078in}}%
\pgfpathlineto{\pgfqpoint{3.251050in}{1.145078in}}%
\pgfpathlineto{\pgfqpoint{3.286700in}{1.145078in}}%
\pgfpathlineto{\pgfqpoint{3.322350in}{1.145078in}}%
\pgfpathlineto{\pgfqpoint{3.358000in}{1.145078in}}%
\pgfpathlineto{\pgfqpoint{3.393650in}{1.145078in}}%
\pgfpathlineto{\pgfqpoint{3.429300in}{1.004794in}}%
\pgfpathlineto{\pgfqpoint{3.464950in}{1.004794in}}%
\pgfpathlineto{\pgfqpoint{3.500600in}{1.004794in}}%
\pgfpathlineto{\pgfqpoint{3.536250in}{1.004794in}}%
\pgfpathlineto{\pgfqpoint{3.571900in}{1.004794in}}%
\pgfpathlineto{\pgfqpoint{3.607550in}{1.004794in}}%
\pgfpathlineto{\pgfqpoint{3.643200in}{1.004794in}}%
\pgfpathlineto{\pgfqpoint{3.678850in}{1.004794in}}%
\pgfpathlineto{\pgfqpoint{3.714500in}{0.993791in}}%
\pgfpathlineto{\pgfqpoint{3.750150in}{0.993791in}}%
\pgfpathlineto{\pgfqpoint{3.785800in}{0.993791in}}%
\pgfpathlineto{\pgfqpoint{3.821450in}{0.927149in}}%
\pgfpathlineto{\pgfqpoint{3.857100in}{0.927149in}}%
\pgfpathlineto{\pgfqpoint{3.892750in}{0.927149in}}%
\pgfpathlineto{\pgfqpoint{3.928400in}{0.927149in}}%
\pgfpathlineto{\pgfqpoint{3.964050in}{0.927149in}}%
\pgfpathlineto{\pgfqpoint{3.999700in}{0.927149in}}%
\pgfpathlineto{\pgfqpoint{4.035350in}{0.927149in}}%
\pgfpathlineto{\pgfqpoint{4.071000in}{0.927149in}}%
\pgfpathlineto{\pgfqpoint{4.106650in}{0.927149in}}%
\pgfpathlineto{\pgfqpoint{4.142300in}{0.927149in}}%
\pgfpathlineto{\pgfqpoint{4.177950in}{0.927149in}}%
\pgfpathlineto{\pgfqpoint{4.213600in}{0.927149in}}%
\pgfpathlineto{\pgfqpoint{4.249250in}{0.927149in}}%
\pgfpathlineto{\pgfqpoint{4.284900in}{0.927149in}}%
\pgfpathlineto{\pgfqpoint{4.320550in}{0.927149in}}%
\pgfpathlineto{\pgfqpoint{4.356200in}{0.927149in}}%
\pgfpathlineto{\pgfqpoint{4.391850in}{0.927149in}}%
\pgfpathlineto{\pgfqpoint{4.427500in}{0.927149in}}%
\pgfpathlineto{\pgfqpoint{4.463150in}{0.927149in}}%
\pgfpathlineto{\pgfqpoint{4.498800in}{0.927149in}}%
\pgfpathlineto{\pgfqpoint{4.534450in}{0.927149in}}%
\pgfpathlineto{\pgfqpoint{4.570100in}{0.927149in}}%
\pgfpathlineto{\pgfqpoint{4.605750in}{0.927149in}}%
\pgfpathlineto{\pgfqpoint{4.641400in}{0.927149in}}%
\pgfpathlineto{\pgfqpoint{4.677050in}{0.927149in}}%
\pgfpathlineto{\pgfqpoint{4.712700in}{0.927149in}}%
\pgfpathlineto{\pgfqpoint{4.748350in}{0.927149in}}%
\pgfpathlineto{\pgfqpoint{4.784000in}{0.927149in}}%
\pgfpathlineto{\pgfqpoint{4.819650in}{0.927149in}}%
\pgfpathlineto{\pgfqpoint{4.855300in}{0.927149in}}%
\pgfpathlineto{\pgfqpoint{4.890950in}{0.927149in}}%
\pgfpathlineto{\pgfqpoint{4.926600in}{0.927149in}}%
\pgfpathlineto{\pgfqpoint{4.962250in}{0.927149in}}%
\pgfpathlineto{\pgfqpoint{4.997900in}{0.927149in}}%
\pgfpathlineto{\pgfqpoint{5.033550in}{0.927149in}}%
\pgfpathlineto{\pgfqpoint{5.069200in}{0.927149in}}%
\pgfpathlineto{\pgfqpoint{5.104850in}{0.927149in}}%
\pgfpathlineto{\pgfqpoint{5.140500in}{0.927149in}}%
\pgfpathlineto{\pgfqpoint{5.176150in}{0.927149in}}%
\pgfpathlineto{\pgfqpoint{5.211800in}{0.927149in}}%
\pgfpathlineto{\pgfqpoint{5.247450in}{0.927149in}}%
\pgfpathlineto{\pgfqpoint{5.283100in}{0.927149in}}%
\pgfpathlineto{\pgfqpoint{5.318750in}{0.927149in}}%
\pgfpathlineto{\pgfqpoint{5.354400in}{0.927149in}}%
\pgfpathlineto{\pgfqpoint{5.390050in}{0.927149in}}%
\pgfpathlineto{\pgfqpoint{5.425700in}{0.927149in}}%
\pgfpathlineto{\pgfqpoint{5.461350in}{0.927149in}}%
\pgfpathlineto{\pgfqpoint{5.497000in}{0.927149in}}%
\pgfpathlineto{\pgfqpoint{5.532650in}{0.883583in}}%
\pgfpathlineto{\pgfqpoint{5.568300in}{0.883583in}}%
\pgfpathlineto{\pgfqpoint{5.603950in}{0.883583in}}%
\pgfpathlineto{\pgfqpoint{5.639600in}{0.883583in}}%
\pgfpathlineto{\pgfqpoint{5.675250in}{0.883583in}}%
\pgfpathlineto{\pgfqpoint{5.710900in}{0.883583in}}%
\pgfpathlineto{\pgfqpoint{5.746550in}{0.883583in}}%
\pgfpathlineto{\pgfqpoint{5.782200in}{0.883583in}}%
\pgfpathlineto{\pgfqpoint{5.817850in}{0.883583in}}%
\pgfpathlineto{\pgfqpoint{5.853500in}{0.883583in}}%
\pgfpathlineto{\pgfqpoint{5.889150in}{0.883583in}}%
\pgfpathlineto{\pgfqpoint{5.924800in}{0.883583in}}%
\pgfpathlineto{\pgfqpoint{5.960450in}{0.883583in}}%
\pgfpathlineto{\pgfqpoint{5.996100in}{0.883583in}}%
\pgfpathlineto{\pgfqpoint{6.031750in}{0.883583in}}%
\pgfpathlineto{\pgfqpoint{6.067400in}{0.883583in}}%
\pgfpathlineto{\pgfqpoint{6.103050in}{0.701037in}}%
\pgfpathlineto{\pgfqpoint{6.138700in}{0.701037in}}%
\pgfpathlineto{\pgfqpoint{6.174350in}{0.701037in}}%
\pgfpathlineto{\pgfqpoint{6.210000in}{0.701037in}}%
\pgfpathlineto{\pgfqpoint{6.245650in}{0.701037in}}%
\pgfpathlineto{\pgfqpoint{6.281300in}{0.701037in}}%
\pgfpathlineto{\pgfqpoint{6.316950in}{0.701037in}}%
\pgfpathlineto{\pgfqpoint{6.352600in}{0.701037in}}%
\pgfpathlineto{\pgfqpoint{6.388250in}{0.701037in}}%
\pgfpathlineto{\pgfqpoint{6.423900in}{0.701037in}}%
\pgfpathlineto{\pgfqpoint{6.459550in}{0.701037in}}%
\pgfpathlineto{\pgfqpoint{6.495200in}{0.701037in}}%
\pgfpathlineto{\pgfqpoint{6.530850in}{0.701037in}}%
\pgfpathlineto{\pgfqpoint{6.566500in}{0.701037in}}%
\pgfpathlineto{\pgfqpoint{6.602150in}{0.701037in}}%
\pgfpathlineto{\pgfqpoint{6.637800in}{0.701037in}}%
\pgfpathlineto{\pgfqpoint{6.673450in}{0.701037in}}%
\pgfpathlineto{\pgfqpoint{6.709100in}{0.701037in}}%
\pgfpathlineto{\pgfqpoint{6.744750in}{0.701037in}}%
\pgfpathlineto{\pgfqpoint{6.780400in}{0.701037in}}%
\pgfpathlineto{\pgfqpoint{6.816050in}{0.701037in}}%
\pgfpathlineto{\pgfqpoint{6.851700in}{0.701037in}}%
\pgfpathlineto{\pgfqpoint{6.887350in}{0.701037in}}%
\pgfpathlineto{\pgfqpoint{6.923000in}{0.701037in}}%
\pgfpathlineto{\pgfqpoint{6.958650in}{0.701037in}}%
\pgfpathlineto{\pgfqpoint{6.994300in}{0.701037in}}%
\pgfpathlineto{\pgfqpoint{7.029950in}{0.701037in}}%
\pgfpathlineto{\pgfqpoint{7.065600in}{0.701037in}}%
\pgfpathlineto{\pgfqpoint{7.101250in}{0.701037in}}%
\pgfpathlineto{\pgfqpoint{7.136900in}{0.701037in}}%
\pgfpathlineto{\pgfqpoint{7.172550in}{0.701037in}}%
\pgfpathlineto{\pgfqpoint{7.208200in}{0.621060in}}%
\pgfpathlineto{\pgfqpoint{7.243850in}{0.621060in}}%
\pgfpathlineto{\pgfqpoint{7.279500in}{0.621060in}}%
\pgfpathlineto{\pgfqpoint{7.315150in}{0.621060in}}%
\pgfpathlineto{\pgfqpoint{7.350800in}{0.621060in}}%
\pgfpathlineto{\pgfqpoint{7.386450in}{0.621060in}}%
\pgfpathlineto{\pgfqpoint{7.422100in}{0.621060in}}%
\pgfpathlineto{\pgfqpoint{7.457750in}{0.621060in}}%
\pgfpathlineto{\pgfqpoint{7.493400in}{0.621060in}}%
\pgfpathlineto{\pgfqpoint{7.529050in}{0.621060in}}%
\pgfpathlineto{\pgfqpoint{7.564700in}{0.621060in}}%
\pgfpathlineto{\pgfqpoint{7.600350in}{0.621060in}}%
\pgfpathlineto{\pgfqpoint{7.636000in}{0.621060in}}%
\pgfpathlineto{\pgfqpoint{7.671650in}{0.621060in}}%
\pgfpathlineto{\pgfqpoint{7.707300in}{0.621060in}}%
\pgfpathlineto{\pgfqpoint{7.742950in}{0.621060in}}%
\pgfpathlineto{\pgfqpoint{7.778600in}{0.621060in}}%
\pgfpathlineto{\pgfqpoint{7.814250in}{0.621060in}}%
\pgfpathlineto{\pgfqpoint{7.849900in}{0.621060in}}%
\pgfpathlineto{\pgfqpoint{7.885550in}{0.621060in}}%
\pgfpathlineto{\pgfqpoint{7.921200in}{0.621060in}}%
\pgfpathlineto{\pgfqpoint{7.956850in}{0.621060in}}%
\pgfpathlineto{\pgfqpoint{7.956850in}{0.323213in}}%
\pgfpathlineto{\pgfqpoint{7.956850in}{0.323213in}}%
\pgfpathlineto{\pgfqpoint{7.921200in}{0.323213in}}%
\pgfpathlineto{\pgfqpoint{7.885550in}{0.323213in}}%
\pgfpathlineto{\pgfqpoint{7.849900in}{0.323213in}}%
\pgfpathlineto{\pgfqpoint{7.814250in}{0.323213in}}%
\pgfpathlineto{\pgfqpoint{7.778600in}{0.323213in}}%
\pgfpathlineto{\pgfqpoint{7.742950in}{0.323213in}}%
\pgfpathlineto{\pgfqpoint{7.707300in}{0.323213in}}%
\pgfpathlineto{\pgfqpoint{7.671650in}{0.323213in}}%
\pgfpathlineto{\pgfqpoint{7.636000in}{0.323213in}}%
\pgfpathlineto{\pgfqpoint{7.600350in}{0.323213in}}%
\pgfpathlineto{\pgfqpoint{7.564700in}{0.323213in}}%
\pgfpathlineto{\pgfqpoint{7.529050in}{0.323213in}}%
\pgfpathlineto{\pgfqpoint{7.493400in}{0.323213in}}%
\pgfpathlineto{\pgfqpoint{7.457750in}{0.323213in}}%
\pgfpathlineto{\pgfqpoint{7.422100in}{0.323213in}}%
\pgfpathlineto{\pgfqpoint{7.386450in}{0.323213in}}%
\pgfpathlineto{\pgfqpoint{7.350800in}{0.323213in}}%
\pgfpathlineto{\pgfqpoint{7.315150in}{0.323213in}}%
\pgfpathlineto{\pgfqpoint{7.279500in}{0.323213in}}%
\pgfpathlineto{\pgfqpoint{7.243850in}{0.323213in}}%
\pgfpathlineto{\pgfqpoint{7.208200in}{0.323213in}}%
\pgfpathlineto{\pgfqpoint{7.172550in}{0.394464in}}%
\pgfpathlineto{\pgfqpoint{7.136900in}{0.394464in}}%
\pgfpathlineto{\pgfqpoint{7.101250in}{0.394464in}}%
\pgfpathlineto{\pgfqpoint{7.065600in}{0.394464in}}%
\pgfpathlineto{\pgfqpoint{7.029950in}{0.394464in}}%
\pgfpathlineto{\pgfqpoint{6.994300in}{0.394464in}}%
\pgfpathlineto{\pgfqpoint{6.958650in}{0.394464in}}%
\pgfpathlineto{\pgfqpoint{6.923000in}{0.394464in}}%
\pgfpathlineto{\pgfqpoint{6.887350in}{0.394464in}}%
\pgfpathlineto{\pgfqpoint{6.851700in}{0.394464in}}%
\pgfpathlineto{\pgfqpoint{6.816050in}{0.394464in}}%
\pgfpathlineto{\pgfqpoint{6.780400in}{0.394464in}}%
\pgfpathlineto{\pgfqpoint{6.744750in}{0.394464in}}%
\pgfpathlineto{\pgfqpoint{6.709100in}{0.394464in}}%
\pgfpathlineto{\pgfqpoint{6.673450in}{0.394464in}}%
\pgfpathlineto{\pgfqpoint{6.637800in}{0.394464in}}%
\pgfpathlineto{\pgfqpoint{6.602150in}{0.394464in}}%
\pgfpathlineto{\pgfqpoint{6.566500in}{0.394464in}}%
\pgfpathlineto{\pgfqpoint{6.530850in}{0.394464in}}%
\pgfpathlineto{\pgfqpoint{6.495200in}{0.394464in}}%
\pgfpathlineto{\pgfqpoint{6.459550in}{0.394464in}}%
\pgfpathlineto{\pgfqpoint{6.423900in}{0.394464in}}%
\pgfpathlineto{\pgfqpoint{6.388250in}{0.394464in}}%
\pgfpathlineto{\pgfqpoint{6.352600in}{0.394464in}}%
\pgfpathlineto{\pgfqpoint{6.316950in}{0.394464in}}%
\pgfpathlineto{\pgfqpoint{6.281300in}{0.394464in}}%
\pgfpathlineto{\pgfqpoint{6.245650in}{0.394464in}}%
\pgfpathlineto{\pgfqpoint{6.210000in}{0.394464in}}%
\pgfpathlineto{\pgfqpoint{6.174350in}{0.394464in}}%
\pgfpathlineto{\pgfqpoint{6.138700in}{0.394464in}}%
\pgfpathlineto{\pgfqpoint{6.103050in}{0.394464in}}%
\pgfpathlineto{\pgfqpoint{6.067400in}{0.457823in}}%
\pgfpathlineto{\pgfqpoint{6.031750in}{0.457823in}}%
\pgfpathlineto{\pgfqpoint{5.996100in}{0.457823in}}%
\pgfpathlineto{\pgfqpoint{5.960450in}{0.457823in}}%
\pgfpathlineto{\pgfqpoint{5.924800in}{0.457823in}}%
\pgfpathlineto{\pgfqpoint{5.889150in}{0.457823in}}%
\pgfpathlineto{\pgfqpoint{5.853500in}{0.457823in}}%
\pgfpathlineto{\pgfqpoint{5.817850in}{0.457823in}}%
\pgfpathlineto{\pgfqpoint{5.782200in}{0.457823in}}%
\pgfpathlineto{\pgfqpoint{5.746550in}{0.457823in}}%
\pgfpathlineto{\pgfqpoint{5.710900in}{0.457823in}}%
\pgfpathlineto{\pgfqpoint{5.675250in}{0.457823in}}%
\pgfpathlineto{\pgfqpoint{5.639600in}{0.457823in}}%
\pgfpathlineto{\pgfqpoint{5.603950in}{0.457823in}}%
\pgfpathlineto{\pgfqpoint{5.568300in}{0.457823in}}%
\pgfpathlineto{\pgfqpoint{5.532650in}{0.457823in}}%
\pgfpathlineto{\pgfqpoint{5.497000in}{0.467191in}}%
\pgfpathlineto{\pgfqpoint{5.461350in}{0.467191in}}%
\pgfpathlineto{\pgfqpoint{5.425700in}{0.467191in}}%
\pgfpathlineto{\pgfqpoint{5.390050in}{0.467191in}}%
\pgfpathlineto{\pgfqpoint{5.354400in}{0.467191in}}%
\pgfpathlineto{\pgfqpoint{5.318750in}{0.467191in}}%
\pgfpathlineto{\pgfqpoint{5.283100in}{0.467191in}}%
\pgfpathlineto{\pgfqpoint{5.247450in}{0.467191in}}%
\pgfpathlineto{\pgfqpoint{5.211800in}{0.467191in}}%
\pgfpathlineto{\pgfqpoint{5.176150in}{0.467191in}}%
\pgfpathlineto{\pgfqpoint{5.140500in}{0.467191in}}%
\pgfpathlineto{\pgfqpoint{5.104850in}{0.467191in}}%
\pgfpathlineto{\pgfqpoint{5.069200in}{0.467191in}}%
\pgfpathlineto{\pgfqpoint{5.033550in}{0.467191in}}%
\pgfpathlineto{\pgfqpoint{4.997900in}{0.467191in}}%
\pgfpathlineto{\pgfqpoint{4.962250in}{0.467191in}}%
\pgfpathlineto{\pgfqpoint{4.926600in}{0.467191in}}%
\pgfpathlineto{\pgfqpoint{4.890950in}{0.467191in}}%
\pgfpathlineto{\pgfqpoint{4.855300in}{0.467191in}}%
\pgfpathlineto{\pgfqpoint{4.819650in}{0.467191in}}%
\pgfpathlineto{\pgfqpoint{4.784000in}{0.467191in}}%
\pgfpathlineto{\pgfqpoint{4.748350in}{0.467191in}}%
\pgfpathlineto{\pgfqpoint{4.712700in}{0.467191in}}%
\pgfpathlineto{\pgfqpoint{4.677050in}{0.467191in}}%
\pgfpathlineto{\pgfqpoint{4.641400in}{0.467191in}}%
\pgfpathlineto{\pgfqpoint{4.605750in}{0.467191in}}%
\pgfpathlineto{\pgfqpoint{4.570100in}{0.467191in}}%
\pgfpathlineto{\pgfqpoint{4.534450in}{0.467191in}}%
\pgfpathlineto{\pgfqpoint{4.498800in}{0.467191in}}%
\pgfpathlineto{\pgfqpoint{4.463150in}{0.467191in}}%
\pgfpathlineto{\pgfqpoint{4.427500in}{0.467191in}}%
\pgfpathlineto{\pgfqpoint{4.391850in}{0.467191in}}%
\pgfpathlineto{\pgfqpoint{4.356200in}{0.467191in}}%
\pgfpathlineto{\pgfqpoint{4.320550in}{0.467191in}}%
\pgfpathlineto{\pgfqpoint{4.284900in}{0.467191in}}%
\pgfpathlineto{\pgfqpoint{4.249250in}{0.467191in}}%
\pgfpathlineto{\pgfqpoint{4.213600in}{0.467191in}}%
\pgfpathlineto{\pgfqpoint{4.177950in}{0.467191in}}%
\pgfpathlineto{\pgfqpoint{4.142300in}{0.467191in}}%
\pgfpathlineto{\pgfqpoint{4.106650in}{0.467191in}}%
\pgfpathlineto{\pgfqpoint{4.071000in}{0.467191in}}%
\pgfpathlineto{\pgfqpoint{4.035350in}{0.467191in}}%
\pgfpathlineto{\pgfqpoint{3.999700in}{0.467191in}}%
\pgfpathlineto{\pgfqpoint{3.964050in}{0.467191in}}%
\pgfpathlineto{\pgfqpoint{3.928400in}{0.467191in}}%
\pgfpathlineto{\pgfqpoint{3.892750in}{0.467191in}}%
\pgfpathlineto{\pgfqpoint{3.857100in}{0.467191in}}%
\pgfpathlineto{\pgfqpoint{3.821450in}{0.467191in}}%
\pgfpathlineto{\pgfqpoint{3.785800in}{0.583131in}}%
\pgfpathlineto{\pgfqpoint{3.750150in}{0.583131in}}%
\pgfpathlineto{\pgfqpoint{3.714500in}{0.583131in}}%
\pgfpathlineto{\pgfqpoint{3.678850in}{0.595427in}}%
\pgfpathlineto{\pgfqpoint{3.643200in}{0.595427in}}%
\pgfpathlineto{\pgfqpoint{3.607550in}{0.595427in}}%
\pgfpathlineto{\pgfqpoint{3.571900in}{0.595427in}}%
\pgfpathlineto{\pgfqpoint{3.536250in}{0.595427in}}%
\pgfpathlineto{\pgfqpoint{3.500600in}{0.595427in}}%
\pgfpathlineto{\pgfqpoint{3.464950in}{0.595427in}}%
\pgfpathlineto{\pgfqpoint{3.429300in}{0.595427in}}%
\pgfpathlineto{\pgfqpoint{3.393650in}{0.630519in}}%
\pgfpathlineto{\pgfqpoint{3.358000in}{0.630519in}}%
\pgfpathlineto{\pgfqpoint{3.322350in}{0.630519in}}%
\pgfpathlineto{\pgfqpoint{3.286700in}{0.630519in}}%
\pgfpathlineto{\pgfqpoint{3.251050in}{0.630519in}}%
\pgfpathlineto{\pgfqpoint{3.215400in}{0.630519in}}%
\pgfpathlineto{\pgfqpoint{3.179750in}{0.797781in}}%
\pgfpathlineto{\pgfqpoint{3.144100in}{0.797781in}}%
\pgfpathlineto{\pgfqpoint{3.108450in}{0.797781in}}%
\pgfpathlineto{\pgfqpoint{3.072800in}{0.797781in}}%
\pgfpathlineto{\pgfqpoint{3.037150in}{0.873465in}}%
\pgfpathlineto{\pgfqpoint{3.001500in}{0.873465in}}%
\pgfpathlineto{\pgfqpoint{2.965850in}{0.873465in}}%
\pgfpathlineto{\pgfqpoint{2.930200in}{0.873465in}}%
\pgfpathlineto{\pgfqpoint{2.894550in}{0.873465in}}%
\pgfpathlineto{\pgfqpoint{2.858900in}{0.873465in}}%
\pgfpathlineto{\pgfqpoint{2.823250in}{0.903626in}}%
\pgfpathlineto{\pgfqpoint{2.787600in}{0.903626in}}%
\pgfpathlineto{\pgfqpoint{2.751950in}{0.903626in}}%
\pgfpathlineto{\pgfqpoint{2.716300in}{0.903626in}}%
\pgfpathlineto{\pgfqpoint{2.680650in}{0.903626in}}%
\pgfpathlineto{\pgfqpoint{2.645000in}{1.140881in}}%
\pgfpathlineto{\pgfqpoint{2.609350in}{1.140881in}}%
\pgfpathlineto{\pgfqpoint{2.573700in}{1.140881in}}%
\pgfpathlineto{\pgfqpoint{2.538050in}{1.204694in}}%
\pgfpathlineto{\pgfqpoint{2.502400in}{1.204694in}}%
\pgfpathlineto{\pgfqpoint{2.466750in}{1.204694in}}%
\pgfpathlineto{\pgfqpoint{2.431100in}{1.274490in}}%
\pgfpathlineto{\pgfqpoint{2.395450in}{1.413047in}}%
\pgfpathlineto{\pgfqpoint{2.359800in}{1.413047in}}%
\pgfpathlineto{\pgfqpoint{2.324150in}{1.475953in}}%
\pgfpathlineto{\pgfqpoint{2.288500in}{1.475953in}}%
\pgfpathlineto{\pgfqpoint{2.252850in}{1.475953in}}%
\pgfpathlineto{\pgfqpoint{2.217200in}{1.475953in}}%
\pgfpathlineto{\pgfqpoint{2.181550in}{1.482855in}}%
\pgfpathlineto{\pgfqpoint{2.145900in}{1.482855in}}%
\pgfpathlineto{\pgfqpoint{2.110250in}{1.482855in}}%
\pgfpathlineto{\pgfqpoint{2.074600in}{1.482855in}}%
\pgfpathlineto{\pgfqpoint{2.038950in}{1.482855in}}%
\pgfpathlineto{\pgfqpoint{2.003300in}{1.482855in}}%
\pgfpathlineto{\pgfqpoint{1.967650in}{1.482855in}}%
\pgfpathlineto{\pgfqpoint{1.932000in}{1.482855in}}%
\pgfpathlineto{\pgfqpoint{1.896350in}{1.482855in}}%
\pgfpathlineto{\pgfqpoint{1.860700in}{1.482855in}}%
\pgfpathlineto{\pgfqpoint{1.825050in}{1.482855in}}%
\pgfpathlineto{\pgfqpoint{1.789400in}{1.498025in}}%
\pgfpathlineto{\pgfqpoint{1.753750in}{1.498025in}}%
\pgfpathlineto{\pgfqpoint{1.718100in}{1.543868in}}%
\pgfpathlineto{\pgfqpoint{1.682450in}{1.554649in}}%
\pgfpathlineto{\pgfqpoint{1.646800in}{1.554649in}}%
\pgfpathlineto{\pgfqpoint{1.611150in}{1.554649in}}%
\pgfpathlineto{\pgfqpoint{1.575500in}{1.554649in}}%
\pgfpathlineto{\pgfqpoint{1.539850in}{1.554649in}}%
\pgfpathlineto{\pgfqpoint{1.504200in}{1.554649in}}%
\pgfpathlineto{\pgfqpoint{1.468550in}{1.578038in}}%
\pgfpathlineto{\pgfqpoint{1.432900in}{1.578038in}}%
\pgfpathlineto{\pgfqpoint{1.397250in}{1.578038in}}%
\pgfpathlineto{\pgfqpoint{1.361600in}{1.782862in}}%
\pgfpathlineto{\pgfqpoint{1.325950in}{1.782862in}}%
\pgfpathlineto{\pgfqpoint{1.290300in}{1.782862in}}%
\pgfpathlineto{\pgfqpoint{1.254650in}{1.782862in}}%
\pgfpathlineto{\pgfqpoint{1.219000in}{1.782862in}}%
\pgfpathlineto{\pgfqpoint{1.183350in}{1.782862in}}%
\pgfpathlineto{\pgfqpoint{1.147700in}{1.782862in}}%
\pgfpathlineto{\pgfqpoint{1.112050in}{1.782862in}}%
\pgfpathlineto{\pgfqpoint{1.076400in}{1.782862in}}%
\pgfpathlineto{\pgfqpoint{1.040750in}{1.898102in}}%
\pgfpathlineto{\pgfqpoint{1.005100in}{1.898102in}}%
\pgfpathlineto{\pgfqpoint{0.969450in}{1.898102in}}%
\pgfpathlineto{\pgfqpoint{0.933800in}{1.958867in}}%
\pgfpathlineto{\pgfqpoint{0.898150in}{1.958867in}}%
\pgfpathlineto{\pgfqpoint{0.862500in}{1.958867in}}%
\pgfpathlineto{\pgfqpoint{0.826850in}{1.958867in}}%
\pgfpathlineto{\pgfqpoint{0.791200in}{1.958867in}}%
\pgfpathlineto{\pgfqpoint{0.755550in}{1.958867in}}%
\pgfpathlineto{\pgfqpoint{0.719900in}{1.958867in}}%
\pgfpathlineto{\pgfqpoint{0.684250in}{1.958867in}}%
\pgfpathlineto{\pgfqpoint{0.648600in}{1.958867in}}%
\pgfpathlineto{\pgfqpoint{0.612950in}{1.958867in}}%
\pgfpathlineto{\pgfqpoint{0.577300in}{1.958867in}}%
\pgfpathlineto{\pgfqpoint{0.541650in}{1.958867in}}%
\pgfpathlineto{\pgfqpoint{0.506000in}{1.958867in}}%
\pgfpathlineto{\pgfqpoint{0.470350in}{1.958867in}}%
\pgfpathlineto{\pgfqpoint{0.434700in}{1.958867in}}%
\pgfpathlineto{\pgfqpoint{0.399050in}{1.958867in}}%
\pgfpathlineto{\pgfqpoint{0.363400in}{1.958867in}}%
\pgfpathlineto{\pgfqpoint{0.327750in}{1.958867in}}%
\pgfpathlineto{\pgfqpoint{0.292100in}{1.958867in}}%
\pgfpathlineto{\pgfqpoint{0.256450in}{1.958867in}}%
\pgfpathlineto{\pgfqpoint{0.220800in}{1.958867in}}%
\pgfpathlineto{\pgfqpoint{0.185150in}{1.958867in}}%
\pgfpathlineto{\pgfqpoint{0.149500in}{1.958867in}}%
\pgfpathlineto{\pgfqpoint{0.113850in}{1.958867in}}%
\pgfpathlineto{\pgfqpoint{0.078200in}{1.958867in}}%
\pgfpathlineto{\pgfqpoint{0.042550in}{1.958867in}}%
\pgfpathlineto{\pgfqpoint{0.006900in}{1.958867in}}%
\pgfpathlineto{\pgfqpoint{-0.028750in}{1.958867in}}%
\pgfpathlineto{\pgfqpoint{-0.064400in}{1.958867in}}%
\pgfpathlineto{\pgfqpoint{-0.100050in}{1.958867in}}%
\pgfpathlineto{\pgfqpoint{-0.135700in}{1.958867in}}%
\pgfpathlineto{\pgfqpoint{-0.171350in}{1.958867in}}%
\pgfpathlineto{\pgfqpoint{-0.207000in}{1.958867in}}%
\pgfpathlineto{\pgfqpoint{-0.242650in}{1.958867in}}%
\pgfpathlineto{\pgfqpoint{-0.278300in}{1.958867in}}%
\pgfpathlineto{\pgfqpoint{-0.313950in}{1.958867in}}%
\pgfpathlineto{\pgfqpoint{-0.349600in}{1.958867in}}%
\pgfpathlineto{\pgfqpoint{-0.385250in}{1.958867in}}%
\pgfpathlineto{\pgfqpoint{-0.420900in}{1.958867in}}%
\pgfpathlineto{\pgfqpoint{-0.456550in}{1.958867in}}%
\pgfpathlineto{\pgfqpoint{-0.492200in}{1.958867in}}%
\pgfpathlineto{\pgfqpoint{-0.527850in}{1.958867in}}%
\pgfpathlineto{\pgfqpoint{-0.563500in}{2.057288in}}%
\pgfpathlineto{\pgfqpoint{-0.599150in}{2.057288in}}%
\pgfpathlineto{\pgfqpoint{-0.634800in}{2.057288in}}%
\pgfpathlineto{\pgfqpoint{-0.670450in}{2.057288in}}%
\pgfpathlineto{\pgfqpoint{-0.706100in}{2.057288in}}%
\pgfpathlineto{\pgfqpoint{-0.741750in}{2.057288in}}%
\pgfpathlineto{\pgfqpoint{-0.777400in}{2.057288in}}%
\pgfpathlineto{\pgfqpoint{-0.813050in}{2.057288in}}%
\pgfpathlineto{\pgfqpoint{-0.848700in}{2.057288in}}%
\pgfpathlineto{\pgfqpoint{-0.884350in}{2.057288in}}%
\pgfpathlineto{\pgfqpoint{-0.920000in}{2.057288in}}%
\pgfpathlineto{\pgfqpoint{-0.955650in}{2.057288in}}%
\pgfpathlineto{\pgfqpoint{-0.991300in}{2.057288in}}%
\pgfpathlineto{\pgfqpoint{-1.026950in}{2.057288in}}%
\pgfpathlineto{\pgfqpoint{-1.062600in}{2.057288in}}%
\pgfpathlineto{\pgfqpoint{-1.098250in}{2.057288in}}%
\pgfpathlineto{\pgfqpoint{-1.133900in}{2.057288in}}%
\pgfpathlineto{\pgfqpoint{-1.169550in}{2.057288in}}%
\pgfpathlineto{\pgfqpoint{-1.205200in}{2.057288in}}%
\pgfpathlineto{\pgfqpoint{-1.240850in}{2.057288in}}%
\pgfpathlineto{\pgfqpoint{-1.276500in}{2.057288in}}%
\pgfpathlineto{\pgfqpoint{-1.312150in}{2.057288in}}%
\pgfpathlineto{\pgfqpoint{-1.347800in}{2.057288in}}%
\pgfpathlineto{\pgfqpoint{-1.383450in}{2.057288in}}%
\pgfpathlineto{\pgfqpoint{-1.419100in}{2.057288in}}%
\pgfpathlineto{\pgfqpoint{-1.454750in}{2.498722in}}%
\pgfpathlineto{\pgfqpoint{-1.490400in}{2.597731in}}%
\pgfpathlineto{\pgfqpoint{-1.526050in}{2.597731in}}%
\pgfpathlineto{\pgfqpoint{-1.561700in}{2.597731in}}%
\pgfpathlineto{\pgfqpoint{-1.597350in}{2.625604in}}%
\pgfpathlineto{\pgfqpoint{-1.633000in}{2.625604in}}%
\pgfpathlineto{\pgfqpoint{-1.668650in}{2.625604in}}%
\pgfpathlineto{\pgfqpoint{-1.704300in}{2.625604in}}%
\pgfpathlineto{\pgfqpoint{-1.739950in}{2.625604in}}%
\pgfpathlineto{\pgfqpoint{-1.775600in}{2.625604in}}%
\pgfpathlineto{\pgfqpoint{-1.811250in}{2.661714in}}%
\pgfpathlineto{\pgfqpoint{-1.846900in}{2.661714in}}%
\pgfpathlineto{\pgfqpoint{-1.882550in}{2.661714in}}%
\pgfpathlineto{\pgfqpoint{-1.918200in}{2.661714in}}%
\pgfpathlineto{\pgfqpoint{-1.953850in}{2.661714in}}%
\pgfpathlineto{\pgfqpoint{-1.989500in}{2.661714in}}%
\pgfpathlineto{\pgfqpoint{-2.025150in}{2.733426in}}%
\pgfpathlineto{\pgfqpoint{-2.060800in}{2.769482in}}%
\pgfpathlineto{\pgfqpoint{-2.096450in}{3.047117in}}%
\pgfpathlineto{\pgfqpoint{-2.132100in}{3.047117in}}%
\pgfpathlineto{\pgfqpoint{-2.167750in}{3.394337in}}%
\pgfpathlineto{\pgfqpoint{-2.203400in}{3.394337in}}%
\pgfpathlineto{\pgfqpoint{-2.239050in}{3.494932in}}%
\pgfpathlineto{\pgfqpoint{-2.274700in}{3.494932in}}%
\pgfpathlineto{\pgfqpoint{-2.310350in}{3.494932in}}%
\pgfpathlineto{\pgfqpoint{-2.346000in}{3.494932in}}%
\pgfpathlineto{\pgfqpoint{-2.381650in}{3.691764in}}%
\pgfpathlineto{\pgfqpoint{-2.417300in}{3.739772in}}%
\pgfpathlineto{\pgfqpoint{-2.452950in}{3.739772in}}%
\pgfpathlineto{\pgfqpoint{-2.488600in}{3.739772in}}%
\pgfpathlineto{\pgfqpoint{-2.524250in}{3.739772in}}%
\pgfpathlineto{\pgfqpoint{-2.559900in}{4.022082in}}%
\pgfpathlineto{\pgfqpoint{-2.595550in}{4.129574in}}%
\pgfpathlineto{\pgfqpoint{-2.631200in}{4.297516in}}%
\pgfpathlineto{\pgfqpoint{-2.666850in}{4.297516in}}%
\pgfpathlineto{\pgfqpoint{-2.702500in}{5.135953in}}%
\pgfpathclose%
\pgfusepath{fill}%
\end{pgfscope}%
\begin{pgfscope}%
\pgfpathrectangle{\pgfqpoint{0.862500in}{0.375000in}}{\pgfqpoint{5.347500in}{2.265000in}}%
\pgfusepath{clip}%
\pgfsetbuttcap%
\pgfsetroundjoin%
\definecolor{currentfill}{rgb}{0.172549,0.627451,0.172549}%
\pgfsetfillcolor{currentfill}%
\pgfsetfillopacity{0.200000}%
\pgfsetlinewidth{0.000000pt}%
\definecolor{currentstroke}{rgb}{0.000000,0.000000,0.000000}%
\pgfsetstrokecolor{currentstroke}%
\pgfsetdash{}{0pt}%
\pgfpathmoveto{\pgfqpoint{-2.702500in}{5.744105in}}%
\pgfpathlineto{\pgfqpoint{-2.702500in}{6.073485in}}%
\pgfpathlineto{\pgfqpoint{-2.666850in}{5.613399in}}%
\pgfpathlineto{\pgfqpoint{-2.631200in}{5.586055in}}%
\pgfpathlineto{\pgfqpoint{-2.595550in}{5.238285in}}%
\pgfpathlineto{\pgfqpoint{-2.559900in}{5.154068in}}%
\pgfpathlineto{\pgfqpoint{-2.524250in}{4.874476in}}%
\pgfpathlineto{\pgfqpoint{-2.488600in}{4.432227in}}%
\pgfpathlineto{\pgfqpoint{-2.452950in}{4.432227in}}%
\pgfpathlineto{\pgfqpoint{-2.417300in}{4.432227in}}%
\pgfpathlineto{\pgfqpoint{-2.381650in}{4.108857in}}%
\pgfpathlineto{\pgfqpoint{-2.346000in}{4.016541in}}%
\pgfpathlineto{\pgfqpoint{-2.310350in}{4.016541in}}%
\pgfpathlineto{\pgfqpoint{-2.274700in}{4.016541in}}%
\pgfpathlineto{\pgfqpoint{-2.239050in}{4.016541in}}%
\pgfpathlineto{\pgfqpoint{-2.203400in}{3.736172in}}%
\pgfpathlineto{\pgfqpoint{-2.167750in}{3.736172in}}%
\pgfpathlineto{\pgfqpoint{-2.132100in}{3.736172in}}%
\pgfpathlineto{\pgfqpoint{-2.096450in}{3.736172in}}%
\pgfpathlineto{\pgfqpoint{-2.060800in}{3.736172in}}%
\pgfpathlineto{\pgfqpoint{-2.025150in}{3.396977in}}%
\pgfpathlineto{\pgfqpoint{-1.989500in}{3.396977in}}%
\pgfpathlineto{\pgfqpoint{-1.953850in}{3.396977in}}%
\pgfpathlineto{\pgfqpoint{-1.918200in}{3.396977in}}%
\pgfpathlineto{\pgfqpoint{-1.882550in}{3.396977in}}%
\pgfpathlineto{\pgfqpoint{-1.846900in}{3.396977in}}%
\pgfpathlineto{\pgfqpoint{-1.811250in}{3.396977in}}%
\pgfpathlineto{\pgfqpoint{-1.775600in}{3.396977in}}%
\pgfpathlineto{\pgfqpoint{-1.739950in}{3.396977in}}%
\pgfpathlineto{\pgfqpoint{-1.704300in}{3.315311in}}%
\pgfpathlineto{\pgfqpoint{-1.668650in}{3.315311in}}%
\pgfpathlineto{\pgfqpoint{-1.633000in}{3.315311in}}%
\pgfpathlineto{\pgfqpoint{-1.597350in}{3.259605in}}%
\pgfpathlineto{\pgfqpoint{-1.561700in}{3.255116in}}%
\pgfpathlineto{\pgfqpoint{-1.526050in}{3.255116in}}%
\pgfpathlineto{\pgfqpoint{-1.490400in}{3.064866in}}%
\pgfpathlineto{\pgfqpoint{-1.454750in}{3.064866in}}%
\pgfpathlineto{\pgfqpoint{-1.419100in}{3.064866in}}%
\pgfpathlineto{\pgfqpoint{-1.383450in}{3.064866in}}%
\pgfpathlineto{\pgfqpoint{-1.347800in}{3.064054in}}%
\pgfpathlineto{\pgfqpoint{-1.312150in}{3.026250in}}%
\pgfpathlineto{\pgfqpoint{-1.276500in}{3.026250in}}%
\pgfpathlineto{\pgfqpoint{-1.240850in}{3.026250in}}%
\pgfpathlineto{\pgfqpoint{-1.205200in}{3.026250in}}%
\pgfpathlineto{\pgfqpoint{-1.169550in}{3.026250in}}%
\pgfpathlineto{\pgfqpoint{-1.133900in}{3.026250in}}%
\pgfpathlineto{\pgfqpoint{-1.098250in}{2.982676in}}%
\pgfpathlineto{\pgfqpoint{-1.062600in}{2.982676in}}%
\pgfpathlineto{\pgfqpoint{-1.026950in}{2.982676in}}%
\pgfpathlineto{\pgfqpoint{-0.991300in}{2.982676in}}%
\pgfpathlineto{\pgfqpoint{-0.955650in}{2.982676in}}%
\pgfpathlineto{\pgfqpoint{-0.920000in}{2.982676in}}%
\pgfpathlineto{\pgfqpoint{-0.884350in}{2.982676in}}%
\pgfpathlineto{\pgfqpoint{-0.848700in}{2.967138in}}%
\pgfpathlineto{\pgfqpoint{-0.813050in}{2.967138in}}%
\pgfpathlineto{\pgfqpoint{-0.777400in}{2.839331in}}%
\pgfpathlineto{\pgfqpoint{-0.741750in}{2.839331in}}%
\pgfpathlineto{\pgfqpoint{-0.706100in}{2.839331in}}%
\pgfpathlineto{\pgfqpoint{-0.670450in}{2.839331in}}%
\pgfpathlineto{\pgfqpoint{-0.634800in}{2.839331in}}%
\pgfpathlineto{\pgfqpoint{-0.599150in}{2.839331in}}%
\pgfpathlineto{\pgfqpoint{-0.563500in}{2.808287in}}%
\pgfpathlineto{\pgfqpoint{-0.527850in}{2.808287in}}%
\pgfpathlineto{\pgfqpoint{-0.492200in}{2.808287in}}%
\pgfpathlineto{\pgfqpoint{-0.456550in}{2.808287in}}%
\pgfpathlineto{\pgfqpoint{-0.420900in}{2.808287in}}%
\pgfpathlineto{\pgfqpoint{-0.385250in}{2.808287in}}%
\pgfpathlineto{\pgfqpoint{-0.349600in}{2.808287in}}%
\pgfpathlineto{\pgfqpoint{-0.313950in}{2.808287in}}%
\pgfpathlineto{\pgfqpoint{-0.278300in}{2.808287in}}%
\pgfpathlineto{\pgfqpoint{-0.242650in}{2.808287in}}%
\pgfpathlineto{\pgfqpoint{-0.207000in}{2.808287in}}%
\pgfpathlineto{\pgfqpoint{-0.171350in}{2.808287in}}%
\pgfpathlineto{\pgfqpoint{-0.135700in}{2.808287in}}%
\pgfpathlineto{\pgfqpoint{-0.100050in}{2.808287in}}%
\pgfpathlineto{\pgfqpoint{-0.064400in}{2.808287in}}%
\pgfpathlineto{\pgfqpoint{-0.028750in}{2.808287in}}%
\pgfpathlineto{\pgfqpoint{0.006900in}{2.808287in}}%
\pgfpathlineto{\pgfqpoint{0.042550in}{2.808287in}}%
\pgfpathlineto{\pgfqpoint{0.078200in}{2.808287in}}%
\pgfpathlineto{\pgfqpoint{0.113850in}{2.808287in}}%
\pgfpathlineto{\pgfqpoint{0.149500in}{2.808287in}}%
\pgfpathlineto{\pgfqpoint{0.185150in}{2.808287in}}%
\pgfpathlineto{\pgfqpoint{0.220800in}{2.808287in}}%
\pgfpathlineto{\pgfqpoint{0.256450in}{2.808287in}}%
\pgfpathlineto{\pgfqpoint{0.292100in}{2.808287in}}%
\pgfpathlineto{\pgfqpoint{0.327750in}{2.808287in}}%
\pgfpathlineto{\pgfqpoint{0.363400in}{2.761236in}}%
\pgfpathlineto{\pgfqpoint{0.399050in}{2.761236in}}%
\pgfpathlineto{\pgfqpoint{0.434700in}{2.761236in}}%
\pgfpathlineto{\pgfqpoint{0.470350in}{2.761236in}}%
\pgfpathlineto{\pgfqpoint{0.506000in}{2.761236in}}%
\pgfpathlineto{\pgfqpoint{0.541650in}{2.761236in}}%
\pgfpathlineto{\pgfqpoint{0.577300in}{2.761236in}}%
\pgfpathlineto{\pgfqpoint{0.612950in}{2.761236in}}%
\pgfpathlineto{\pgfqpoint{0.648600in}{2.761236in}}%
\pgfpathlineto{\pgfqpoint{0.684250in}{2.761236in}}%
\pgfpathlineto{\pgfqpoint{0.719900in}{2.761236in}}%
\pgfpathlineto{\pgfqpoint{0.755550in}{2.761236in}}%
\pgfpathlineto{\pgfqpoint{0.791200in}{2.697570in}}%
\pgfpathlineto{\pgfqpoint{0.826850in}{2.697570in}}%
\pgfpathlineto{\pgfqpoint{0.862500in}{2.697570in}}%
\pgfpathlineto{\pgfqpoint{0.898150in}{2.689031in}}%
\pgfpathlineto{\pgfqpoint{0.933800in}{2.689031in}}%
\pgfpathlineto{\pgfqpoint{0.969450in}{2.486185in}}%
\pgfpathlineto{\pgfqpoint{1.005100in}{2.354705in}}%
\pgfpathlineto{\pgfqpoint{1.040750in}{2.330833in}}%
\pgfpathlineto{\pgfqpoint{1.076400in}{2.330833in}}%
\pgfpathlineto{\pgfqpoint{1.112050in}{2.026460in}}%
\pgfpathlineto{\pgfqpoint{1.147700in}{2.026460in}}%
\pgfpathlineto{\pgfqpoint{1.183350in}{2.026460in}}%
\pgfpathlineto{\pgfqpoint{1.219000in}{2.026460in}}%
\pgfpathlineto{\pgfqpoint{1.254650in}{2.026460in}}%
\pgfpathlineto{\pgfqpoint{1.290300in}{2.001678in}}%
\pgfpathlineto{\pgfqpoint{1.325950in}{2.001678in}}%
\pgfpathlineto{\pgfqpoint{1.361600in}{2.001678in}}%
\pgfpathlineto{\pgfqpoint{1.397250in}{2.001678in}}%
\pgfpathlineto{\pgfqpoint{1.432900in}{1.955518in}}%
\pgfpathlineto{\pgfqpoint{1.468550in}{1.955518in}}%
\pgfpathlineto{\pgfqpoint{1.504200in}{1.955518in}}%
\pgfpathlineto{\pgfqpoint{1.539850in}{1.955518in}}%
\pgfpathlineto{\pgfqpoint{1.575500in}{1.955518in}}%
\pgfpathlineto{\pgfqpoint{1.611150in}{1.915222in}}%
\pgfpathlineto{\pgfqpoint{1.646800in}{1.805995in}}%
\pgfpathlineto{\pgfqpoint{1.682450in}{1.721620in}}%
\pgfpathlineto{\pgfqpoint{1.718100in}{1.721620in}}%
\pgfpathlineto{\pgfqpoint{1.753750in}{1.721620in}}%
\pgfpathlineto{\pgfqpoint{1.789400in}{1.721620in}}%
\pgfpathlineto{\pgfqpoint{1.825050in}{1.721620in}}%
\pgfpathlineto{\pgfqpoint{1.860700in}{1.721302in}}%
\pgfpathlineto{\pgfqpoint{1.896350in}{1.721302in}}%
\pgfpathlineto{\pgfqpoint{1.932000in}{1.721302in}}%
\pgfpathlineto{\pgfqpoint{1.967650in}{1.721302in}}%
\pgfpathlineto{\pgfqpoint{2.003300in}{1.721302in}}%
\pgfpathlineto{\pgfqpoint{2.038950in}{1.721302in}}%
\pgfpathlineto{\pgfqpoint{2.074600in}{1.721302in}}%
\pgfpathlineto{\pgfqpoint{2.110250in}{1.721302in}}%
\pgfpathlineto{\pgfqpoint{2.145900in}{1.721302in}}%
\pgfpathlineto{\pgfqpoint{2.181550in}{1.656453in}}%
\pgfpathlineto{\pgfqpoint{2.217200in}{1.656453in}}%
\pgfpathlineto{\pgfqpoint{2.252850in}{1.656453in}}%
\pgfpathlineto{\pgfqpoint{2.288500in}{1.656453in}}%
\pgfpathlineto{\pgfqpoint{2.324150in}{1.656453in}}%
\pgfpathlineto{\pgfqpoint{2.359800in}{1.656453in}}%
\pgfpathlineto{\pgfqpoint{2.395450in}{1.656453in}}%
\pgfpathlineto{\pgfqpoint{2.431100in}{1.649518in}}%
\pgfpathlineto{\pgfqpoint{2.466750in}{1.649518in}}%
\pgfpathlineto{\pgfqpoint{2.502400in}{1.649518in}}%
\pgfpathlineto{\pgfqpoint{2.538050in}{1.649518in}}%
\pgfpathlineto{\pgfqpoint{2.573700in}{1.649518in}}%
\pgfpathlineto{\pgfqpoint{2.609350in}{1.649518in}}%
\pgfpathlineto{\pgfqpoint{2.645000in}{1.649518in}}%
\pgfpathlineto{\pgfqpoint{2.680650in}{1.535404in}}%
\pgfpathlineto{\pgfqpoint{2.716300in}{1.523692in}}%
\pgfpathlineto{\pgfqpoint{2.751950in}{1.523692in}}%
\pgfpathlineto{\pgfqpoint{2.787600in}{1.493922in}}%
\pgfpathlineto{\pgfqpoint{2.823250in}{1.493922in}}%
\pgfpathlineto{\pgfqpoint{2.858900in}{1.493922in}}%
\pgfpathlineto{\pgfqpoint{2.894550in}{1.493922in}}%
\pgfpathlineto{\pgfqpoint{2.930200in}{1.493922in}}%
\pgfpathlineto{\pgfqpoint{2.965850in}{1.493922in}}%
\pgfpathlineto{\pgfqpoint{3.001500in}{1.493922in}}%
\pgfpathlineto{\pgfqpoint{3.037150in}{1.493922in}}%
\pgfpathlineto{\pgfqpoint{3.072800in}{1.493922in}}%
\pgfpathlineto{\pgfqpoint{3.108450in}{1.493922in}}%
\pgfpathlineto{\pgfqpoint{3.144100in}{1.493922in}}%
\pgfpathlineto{\pgfqpoint{3.179750in}{1.493922in}}%
\pgfpathlineto{\pgfqpoint{3.215400in}{1.493922in}}%
\pgfpathlineto{\pgfqpoint{3.251050in}{1.493922in}}%
\pgfpathlineto{\pgfqpoint{3.286700in}{1.493922in}}%
\pgfpathlineto{\pgfqpoint{3.322350in}{1.493922in}}%
\pgfpathlineto{\pgfqpoint{3.358000in}{1.493922in}}%
\pgfpathlineto{\pgfqpoint{3.393650in}{1.493922in}}%
\pgfpathlineto{\pgfqpoint{3.429300in}{1.493922in}}%
\pgfpathlineto{\pgfqpoint{3.464950in}{1.483893in}}%
\pgfpathlineto{\pgfqpoint{3.500600in}{1.483893in}}%
\pgfpathlineto{\pgfqpoint{3.536250in}{1.250767in}}%
\pgfpathlineto{\pgfqpoint{3.571900in}{1.250767in}}%
\pgfpathlineto{\pgfqpoint{3.607550in}{1.250767in}}%
\pgfpathlineto{\pgfqpoint{3.643200in}{1.250767in}}%
\pgfpathlineto{\pgfqpoint{3.678850in}{1.250767in}}%
\pgfpathlineto{\pgfqpoint{3.714500in}{1.250767in}}%
\pgfpathlineto{\pgfqpoint{3.750150in}{1.250767in}}%
\pgfpathlineto{\pgfqpoint{3.785800in}{1.250767in}}%
\pgfpathlineto{\pgfqpoint{3.821450in}{1.250767in}}%
\pgfpathlineto{\pgfqpoint{3.857100in}{1.250767in}}%
\pgfpathlineto{\pgfqpoint{3.892750in}{1.250767in}}%
\pgfpathlineto{\pgfqpoint{3.928400in}{1.250767in}}%
\pgfpathlineto{\pgfqpoint{3.964050in}{1.250767in}}%
\pgfpathlineto{\pgfqpoint{3.999700in}{1.250767in}}%
\pgfpathlineto{\pgfqpoint{4.035350in}{1.250767in}}%
\pgfpathlineto{\pgfqpoint{4.071000in}{1.250767in}}%
\pgfpathlineto{\pgfqpoint{4.106650in}{1.250767in}}%
\pgfpathlineto{\pgfqpoint{4.142300in}{1.250767in}}%
\pgfpathlineto{\pgfqpoint{4.177950in}{1.250767in}}%
\pgfpathlineto{\pgfqpoint{4.213600in}{1.250767in}}%
\pgfpathlineto{\pgfqpoint{4.249250in}{1.250767in}}%
\pgfpathlineto{\pgfqpoint{4.284900in}{1.250767in}}%
\pgfpathlineto{\pgfqpoint{4.320550in}{1.250767in}}%
\pgfpathlineto{\pgfqpoint{4.356200in}{1.222122in}}%
\pgfpathlineto{\pgfqpoint{4.391850in}{1.222122in}}%
\pgfpathlineto{\pgfqpoint{4.427500in}{1.222122in}}%
\pgfpathlineto{\pgfqpoint{4.463150in}{1.222122in}}%
\pgfpathlineto{\pgfqpoint{4.498800in}{1.222122in}}%
\pgfpathlineto{\pgfqpoint{4.534450in}{1.222122in}}%
\pgfpathlineto{\pgfqpoint{4.570100in}{1.222122in}}%
\pgfpathlineto{\pgfqpoint{4.605750in}{1.222122in}}%
\pgfpathlineto{\pgfqpoint{4.641400in}{1.222122in}}%
\pgfpathlineto{\pgfqpoint{4.677050in}{1.214796in}}%
\pgfpathlineto{\pgfqpoint{4.712700in}{1.214796in}}%
\pgfpathlineto{\pgfqpoint{4.748350in}{1.214796in}}%
\pgfpathlineto{\pgfqpoint{4.784000in}{1.214796in}}%
\pgfpathlineto{\pgfqpoint{4.819650in}{1.138118in}}%
\pgfpathlineto{\pgfqpoint{4.855300in}{1.138118in}}%
\pgfpathlineto{\pgfqpoint{4.890950in}{1.138118in}}%
\pgfpathlineto{\pgfqpoint{4.926600in}{1.030430in}}%
\pgfpathlineto{\pgfqpoint{4.962250in}{1.030430in}}%
\pgfpathlineto{\pgfqpoint{4.997900in}{1.030430in}}%
\pgfpathlineto{\pgfqpoint{5.033550in}{1.030430in}}%
\pgfpathlineto{\pgfqpoint{5.069200in}{1.030430in}}%
\pgfpathlineto{\pgfqpoint{5.104850in}{1.030430in}}%
\pgfpathlineto{\pgfqpoint{5.140500in}{1.030430in}}%
\pgfpathlineto{\pgfqpoint{5.176150in}{1.030430in}}%
\pgfpathlineto{\pgfqpoint{5.211800in}{1.030430in}}%
\pgfpathlineto{\pgfqpoint{5.247450in}{1.030430in}}%
\pgfpathlineto{\pgfqpoint{5.283100in}{1.030430in}}%
\pgfpathlineto{\pgfqpoint{5.318750in}{1.017337in}}%
\pgfpathlineto{\pgfqpoint{5.354400in}{1.017337in}}%
\pgfpathlineto{\pgfqpoint{5.390050in}{1.017337in}}%
\pgfpathlineto{\pgfqpoint{5.425700in}{1.017337in}}%
\pgfpathlineto{\pgfqpoint{5.461350in}{1.017337in}}%
\pgfpathlineto{\pgfqpoint{5.497000in}{1.017337in}}%
\pgfpathlineto{\pgfqpoint{5.532650in}{1.017337in}}%
\pgfpathlineto{\pgfqpoint{5.568300in}{1.017337in}}%
\pgfpathlineto{\pgfqpoint{5.603950in}{1.009622in}}%
\pgfpathlineto{\pgfqpoint{5.639600in}{1.009622in}}%
\pgfpathlineto{\pgfqpoint{5.675250in}{1.009622in}}%
\pgfpathlineto{\pgfqpoint{5.710900in}{1.009622in}}%
\pgfpathlineto{\pgfqpoint{5.746550in}{1.009622in}}%
\pgfpathlineto{\pgfqpoint{5.782200in}{1.009622in}}%
\pgfpathlineto{\pgfqpoint{5.817850in}{1.009622in}}%
\pgfpathlineto{\pgfqpoint{5.853500in}{1.009622in}}%
\pgfpathlineto{\pgfqpoint{5.889150in}{1.009622in}}%
\pgfpathlineto{\pgfqpoint{5.924800in}{1.009622in}}%
\pgfpathlineto{\pgfqpoint{5.960450in}{1.009622in}}%
\pgfpathlineto{\pgfqpoint{5.996100in}{1.009622in}}%
\pgfpathlineto{\pgfqpoint{6.031750in}{1.009622in}}%
\pgfpathlineto{\pgfqpoint{6.067400in}{1.009622in}}%
\pgfpathlineto{\pgfqpoint{6.103050in}{1.009622in}}%
\pgfpathlineto{\pgfqpoint{6.138700in}{1.009622in}}%
\pgfpathlineto{\pgfqpoint{6.174350in}{1.009622in}}%
\pgfpathlineto{\pgfqpoint{6.210000in}{1.009622in}}%
\pgfpathlineto{\pgfqpoint{6.245650in}{1.009622in}}%
\pgfpathlineto{\pgfqpoint{6.281300in}{1.009622in}}%
\pgfpathlineto{\pgfqpoint{6.316950in}{1.009622in}}%
\pgfpathlineto{\pgfqpoint{6.352600in}{1.009622in}}%
\pgfpathlineto{\pgfqpoint{6.388250in}{1.009622in}}%
\pgfpathlineto{\pgfqpoint{6.423900in}{1.009622in}}%
\pgfpathlineto{\pgfqpoint{6.459550in}{1.009622in}}%
\pgfpathlineto{\pgfqpoint{6.495200in}{1.009622in}}%
\pgfpathlineto{\pgfqpoint{6.530850in}{1.009622in}}%
\pgfpathlineto{\pgfqpoint{6.566500in}{0.865636in}}%
\pgfpathlineto{\pgfqpoint{6.602150in}{0.769050in}}%
\pgfpathlineto{\pgfqpoint{6.637800in}{0.769050in}}%
\pgfpathlineto{\pgfqpoint{6.673450in}{0.769050in}}%
\pgfpathlineto{\pgfqpoint{6.709100in}{0.769050in}}%
\pgfpathlineto{\pgfqpoint{6.744750in}{0.769050in}}%
\pgfpathlineto{\pgfqpoint{6.780400in}{0.769050in}}%
\pgfpathlineto{\pgfqpoint{6.816050in}{0.769050in}}%
\pgfpathlineto{\pgfqpoint{6.851700in}{0.769050in}}%
\pgfpathlineto{\pgfqpoint{6.887350in}{0.769050in}}%
\pgfpathlineto{\pgfqpoint{6.923000in}{0.769050in}}%
\pgfpathlineto{\pgfqpoint{6.958650in}{0.769050in}}%
\pgfpathlineto{\pgfqpoint{6.994300in}{0.769050in}}%
\pgfpathlineto{\pgfqpoint{7.029950in}{0.769050in}}%
\pgfpathlineto{\pgfqpoint{7.065600in}{0.769050in}}%
\pgfpathlineto{\pgfqpoint{7.101250in}{0.769050in}}%
\pgfpathlineto{\pgfqpoint{7.136900in}{0.769050in}}%
\pgfpathlineto{\pgfqpoint{7.172550in}{0.769050in}}%
\pgfpathlineto{\pgfqpoint{7.208200in}{0.769050in}}%
\pgfpathlineto{\pgfqpoint{7.243850in}{0.769050in}}%
\pgfpathlineto{\pgfqpoint{7.279500in}{0.769050in}}%
\pgfpathlineto{\pgfqpoint{7.315150in}{0.769050in}}%
\pgfpathlineto{\pgfqpoint{7.350800in}{0.769050in}}%
\pgfpathlineto{\pgfqpoint{7.386450in}{0.769050in}}%
\pgfpathlineto{\pgfqpoint{7.422100in}{0.769050in}}%
\pgfpathlineto{\pgfqpoint{7.457750in}{0.769050in}}%
\pgfpathlineto{\pgfqpoint{7.493400in}{0.769050in}}%
\pgfpathlineto{\pgfqpoint{7.529050in}{0.769050in}}%
\pgfpathlineto{\pgfqpoint{7.564700in}{0.769050in}}%
\pgfpathlineto{\pgfqpoint{7.600350in}{0.769050in}}%
\pgfpathlineto{\pgfqpoint{7.636000in}{0.769050in}}%
\pgfpathlineto{\pgfqpoint{7.671650in}{0.769050in}}%
\pgfpathlineto{\pgfqpoint{7.707300in}{0.769050in}}%
\pgfpathlineto{\pgfqpoint{7.742950in}{0.631514in}}%
\pgfpathlineto{\pgfqpoint{7.778600in}{0.631514in}}%
\pgfpathlineto{\pgfqpoint{7.814250in}{0.631514in}}%
\pgfpathlineto{\pgfqpoint{7.849900in}{0.631514in}}%
\pgfpathlineto{\pgfqpoint{7.885550in}{0.631514in}}%
\pgfpathlineto{\pgfqpoint{7.921200in}{0.631514in}}%
\pgfpathlineto{\pgfqpoint{7.956850in}{0.631514in}}%
\pgfpathlineto{\pgfqpoint{7.956850in}{0.400543in}}%
\pgfpathlineto{\pgfqpoint{7.956850in}{0.400543in}}%
\pgfpathlineto{\pgfqpoint{7.921200in}{0.400543in}}%
\pgfpathlineto{\pgfqpoint{7.885550in}{0.400543in}}%
\pgfpathlineto{\pgfqpoint{7.849900in}{0.400543in}}%
\pgfpathlineto{\pgfqpoint{7.814250in}{0.400543in}}%
\pgfpathlineto{\pgfqpoint{7.778600in}{0.400543in}}%
\pgfpathlineto{\pgfqpoint{7.742950in}{0.400543in}}%
\pgfpathlineto{\pgfqpoint{7.707300in}{0.532304in}}%
\pgfpathlineto{\pgfqpoint{7.671650in}{0.532304in}}%
\pgfpathlineto{\pgfqpoint{7.636000in}{0.532304in}}%
\pgfpathlineto{\pgfqpoint{7.600350in}{0.532304in}}%
\pgfpathlineto{\pgfqpoint{7.564700in}{0.532304in}}%
\pgfpathlineto{\pgfqpoint{7.529050in}{0.532304in}}%
\pgfpathlineto{\pgfqpoint{7.493400in}{0.532304in}}%
\pgfpathlineto{\pgfqpoint{7.457750in}{0.532304in}}%
\pgfpathlineto{\pgfqpoint{7.422100in}{0.532304in}}%
\pgfpathlineto{\pgfqpoint{7.386450in}{0.532304in}}%
\pgfpathlineto{\pgfqpoint{7.350800in}{0.532304in}}%
\pgfpathlineto{\pgfqpoint{7.315150in}{0.532304in}}%
\pgfpathlineto{\pgfqpoint{7.279500in}{0.532304in}}%
\pgfpathlineto{\pgfqpoint{7.243850in}{0.532304in}}%
\pgfpathlineto{\pgfqpoint{7.208200in}{0.532304in}}%
\pgfpathlineto{\pgfqpoint{7.172550in}{0.532304in}}%
\pgfpathlineto{\pgfqpoint{7.136900in}{0.532304in}}%
\pgfpathlineto{\pgfqpoint{7.101250in}{0.532304in}}%
\pgfpathlineto{\pgfqpoint{7.065600in}{0.532304in}}%
\pgfpathlineto{\pgfqpoint{7.029950in}{0.532304in}}%
\pgfpathlineto{\pgfqpoint{6.994300in}{0.532304in}}%
\pgfpathlineto{\pgfqpoint{6.958650in}{0.532304in}}%
\pgfpathlineto{\pgfqpoint{6.923000in}{0.532304in}}%
\pgfpathlineto{\pgfqpoint{6.887350in}{0.532304in}}%
\pgfpathlineto{\pgfqpoint{6.851700in}{0.532304in}}%
\pgfpathlineto{\pgfqpoint{6.816050in}{0.532304in}}%
\pgfpathlineto{\pgfqpoint{6.780400in}{0.532304in}}%
\pgfpathlineto{\pgfqpoint{6.744750in}{0.532304in}}%
\pgfpathlineto{\pgfqpoint{6.709100in}{0.532304in}}%
\pgfpathlineto{\pgfqpoint{6.673450in}{0.532304in}}%
\pgfpathlineto{\pgfqpoint{6.637800in}{0.532304in}}%
\pgfpathlineto{\pgfqpoint{6.602150in}{0.532304in}}%
\pgfpathlineto{\pgfqpoint{6.566500in}{0.703577in}}%
\pgfpathlineto{\pgfqpoint{6.530850in}{0.737353in}}%
\pgfpathlineto{\pgfqpoint{6.495200in}{0.737353in}}%
\pgfpathlineto{\pgfqpoint{6.459550in}{0.737353in}}%
\pgfpathlineto{\pgfqpoint{6.423900in}{0.737353in}}%
\pgfpathlineto{\pgfqpoint{6.388250in}{0.737353in}}%
\pgfpathlineto{\pgfqpoint{6.352600in}{0.737353in}}%
\pgfpathlineto{\pgfqpoint{6.316950in}{0.737353in}}%
\pgfpathlineto{\pgfqpoint{6.281300in}{0.737353in}}%
\pgfpathlineto{\pgfqpoint{6.245650in}{0.737353in}}%
\pgfpathlineto{\pgfqpoint{6.210000in}{0.737353in}}%
\pgfpathlineto{\pgfqpoint{6.174350in}{0.737353in}}%
\pgfpathlineto{\pgfqpoint{6.138700in}{0.737353in}}%
\pgfpathlineto{\pgfqpoint{6.103050in}{0.737353in}}%
\pgfpathlineto{\pgfqpoint{6.067400in}{0.737353in}}%
\pgfpathlineto{\pgfqpoint{6.031750in}{0.737353in}}%
\pgfpathlineto{\pgfqpoint{5.996100in}{0.737353in}}%
\pgfpathlineto{\pgfqpoint{5.960450in}{0.737353in}}%
\pgfpathlineto{\pgfqpoint{5.924800in}{0.737353in}}%
\pgfpathlineto{\pgfqpoint{5.889150in}{0.737353in}}%
\pgfpathlineto{\pgfqpoint{5.853500in}{0.737353in}}%
\pgfpathlineto{\pgfqpoint{5.817850in}{0.737353in}}%
\pgfpathlineto{\pgfqpoint{5.782200in}{0.737353in}}%
\pgfpathlineto{\pgfqpoint{5.746550in}{0.737353in}}%
\pgfpathlineto{\pgfqpoint{5.710900in}{0.737353in}}%
\pgfpathlineto{\pgfqpoint{5.675250in}{0.737353in}}%
\pgfpathlineto{\pgfqpoint{5.639600in}{0.737353in}}%
\pgfpathlineto{\pgfqpoint{5.603950in}{0.737353in}}%
\pgfpathlineto{\pgfqpoint{5.568300in}{0.746186in}}%
\pgfpathlineto{\pgfqpoint{5.532650in}{0.746186in}}%
\pgfpathlineto{\pgfqpoint{5.497000in}{0.746186in}}%
\pgfpathlineto{\pgfqpoint{5.461350in}{0.746186in}}%
\pgfpathlineto{\pgfqpoint{5.425700in}{0.746186in}}%
\pgfpathlineto{\pgfqpoint{5.390050in}{0.746186in}}%
\pgfpathlineto{\pgfqpoint{5.354400in}{0.746186in}}%
\pgfpathlineto{\pgfqpoint{5.318750in}{0.746186in}}%
\pgfpathlineto{\pgfqpoint{5.283100in}{0.803601in}}%
\pgfpathlineto{\pgfqpoint{5.247450in}{0.803601in}}%
\pgfpathlineto{\pgfqpoint{5.211800in}{0.803601in}}%
\pgfpathlineto{\pgfqpoint{5.176150in}{0.803601in}}%
\pgfpathlineto{\pgfqpoint{5.140500in}{0.803601in}}%
\pgfpathlineto{\pgfqpoint{5.104850in}{0.803601in}}%
\pgfpathlineto{\pgfqpoint{5.069200in}{0.803601in}}%
\pgfpathlineto{\pgfqpoint{5.033550in}{0.803601in}}%
\pgfpathlineto{\pgfqpoint{4.997900in}{0.803601in}}%
\pgfpathlineto{\pgfqpoint{4.962250in}{0.803601in}}%
\pgfpathlineto{\pgfqpoint{4.926600in}{0.803601in}}%
\pgfpathlineto{\pgfqpoint{4.890950in}{0.902304in}}%
\pgfpathlineto{\pgfqpoint{4.855300in}{0.902304in}}%
\pgfpathlineto{\pgfqpoint{4.819650in}{0.902304in}}%
\pgfpathlineto{\pgfqpoint{4.784000in}{0.925962in}}%
\pgfpathlineto{\pgfqpoint{4.748350in}{0.925962in}}%
\pgfpathlineto{\pgfqpoint{4.712700in}{0.925962in}}%
\pgfpathlineto{\pgfqpoint{4.677050in}{0.925962in}}%
\pgfpathlineto{\pgfqpoint{4.641400in}{0.939860in}}%
\pgfpathlineto{\pgfqpoint{4.605750in}{0.939860in}}%
\pgfpathlineto{\pgfqpoint{4.570100in}{0.939860in}}%
\pgfpathlineto{\pgfqpoint{4.534450in}{0.939860in}}%
\pgfpathlineto{\pgfqpoint{4.498800in}{0.939860in}}%
\pgfpathlineto{\pgfqpoint{4.463150in}{0.939860in}}%
\pgfpathlineto{\pgfqpoint{4.427500in}{0.939860in}}%
\pgfpathlineto{\pgfqpoint{4.391850in}{0.939860in}}%
\pgfpathlineto{\pgfqpoint{4.356200in}{0.939860in}}%
\pgfpathlineto{\pgfqpoint{4.320550in}{0.946427in}}%
\pgfpathlineto{\pgfqpoint{4.284900in}{0.946427in}}%
\pgfpathlineto{\pgfqpoint{4.249250in}{0.946427in}}%
\pgfpathlineto{\pgfqpoint{4.213600in}{0.946427in}}%
\pgfpathlineto{\pgfqpoint{4.177950in}{0.946427in}}%
\pgfpathlineto{\pgfqpoint{4.142300in}{0.946427in}}%
\pgfpathlineto{\pgfqpoint{4.106650in}{0.946427in}}%
\pgfpathlineto{\pgfqpoint{4.071000in}{0.946427in}}%
\pgfpathlineto{\pgfqpoint{4.035350in}{0.946427in}}%
\pgfpathlineto{\pgfqpoint{3.999700in}{0.946427in}}%
\pgfpathlineto{\pgfqpoint{3.964050in}{0.946427in}}%
\pgfpathlineto{\pgfqpoint{3.928400in}{0.946427in}}%
\pgfpathlineto{\pgfqpoint{3.892750in}{0.946427in}}%
\pgfpathlineto{\pgfqpoint{3.857100in}{0.946427in}}%
\pgfpathlineto{\pgfqpoint{3.821450in}{0.946427in}}%
\pgfpathlineto{\pgfqpoint{3.785800in}{0.946427in}}%
\pgfpathlineto{\pgfqpoint{3.750150in}{0.946427in}}%
\pgfpathlineto{\pgfqpoint{3.714500in}{0.946427in}}%
\pgfpathlineto{\pgfqpoint{3.678850in}{0.946427in}}%
\pgfpathlineto{\pgfqpoint{3.643200in}{0.946427in}}%
\pgfpathlineto{\pgfqpoint{3.607550in}{0.946427in}}%
\pgfpathlineto{\pgfqpoint{3.571900in}{0.946427in}}%
\pgfpathlineto{\pgfqpoint{3.536250in}{0.946427in}}%
\pgfpathlineto{\pgfqpoint{3.500600in}{1.143928in}}%
\pgfpathlineto{\pgfqpoint{3.464950in}{1.143928in}}%
\pgfpathlineto{\pgfqpoint{3.429300in}{1.154269in}}%
\pgfpathlineto{\pgfqpoint{3.393650in}{1.154269in}}%
\pgfpathlineto{\pgfqpoint{3.358000in}{1.154269in}}%
\pgfpathlineto{\pgfqpoint{3.322350in}{1.154269in}}%
\pgfpathlineto{\pgfqpoint{3.286700in}{1.154269in}}%
\pgfpathlineto{\pgfqpoint{3.251050in}{1.154269in}}%
\pgfpathlineto{\pgfqpoint{3.215400in}{1.154269in}}%
\pgfpathlineto{\pgfqpoint{3.179750in}{1.154269in}}%
\pgfpathlineto{\pgfqpoint{3.144100in}{1.154269in}}%
\pgfpathlineto{\pgfqpoint{3.108450in}{1.154269in}}%
\pgfpathlineto{\pgfqpoint{3.072800in}{1.154269in}}%
\pgfpathlineto{\pgfqpoint{3.037150in}{1.154269in}}%
\pgfpathlineto{\pgfqpoint{3.001500in}{1.154269in}}%
\pgfpathlineto{\pgfqpoint{2.965850in}{1.154269in}}%
\pgfpathlineto{\pgfqpoint{2.930200in}{1.154269in}}%
\pgfpathlineto{\pgfqpoint{2.894550in}{1.154269in}}%
\pgfpathlineto{\pgfqpoint{2.858900in}{1.154269in}}%
\pgfpathlineto{\pgfqpoint{2.823250in}{1.154269in}}%
\pgfpathlineto{\pgfqpoint{2.787600in}{1.154269in}}%
\pgfpathlineto{\pgfqpoint{2.751950in}{1.225261in}}%
\pgfpathlineto{\pgfqpoint{2.716300in}{1.225261in}}%
\pgfpathlineto{\pgfqpoint{2.680650in}{1.227881in}}%
\pgfpathlineto{\pgfqpoint{2.645000in}{1.248495in}}%
\pgfpathlineto{\pgfqpoint{2.609350in}{1.248495in}}%
\pgfpathlineto{\pgfqpoint{2.573700in}{1.248495in}}%
\pgfpathlineto{\pgfqpoint{2.538050in}{1.248495in}}%
\pgfpathlineto{\pgfqpoint{2.502400in}{1.248495in}}%
\pgfpathlineto{\pgfqpoint{2.466750in}{1.248495in}}%
\pgfpathlineto{\pgfqpoint{2.431100in}{1.248495in}}%
\pgfpathlineto{\pgfqpoint{2.395450in}{1.287095in}}%
\pgfpathlineto{\pgfqpoint{2.359800in}{1.287095in}}%
\pgfpathlineto{\pgfqpoint{2.324150in}{1.287095in}}%
\pgfpathlineto{\pgfqpoint{2.288500in}{1.287095in}}%
\pgfpathlineto{\pgfqpoint{2.252850in}{1.287095in}}%
\pgfpathlineto{\pgfqpoint{2.217200in}{1.287095in}}%
\pgfpathlineto{\pgfqpoint{2.181550in}{1.287095in}}%
\pgfpathlineto{\pgfqpoint{2.145900in}{1.393845in}}%
\pgfpathlineto{\pgfqpoint{2.110250in}{1.393845in}}%
\pgfpathlineto{\pgfqpoint{2.074600in}{1.393845in}}%
\pgfpathlineto{\pgfqpoint{2.038950in}{1.393845in}}%
\pgfpathlineto{\pgfqpoint{2.003300in}{1.393845in}}%
\pgfpathlineto{\pgfqpoint{1.967650in}{1.393845in}}%
\pgfpathlineto{\pgfqpoint{1.932000in}{1.393845in}}%
\pgfpathlineto{\pgfqpoint{1.896350in}{1.393845in}}%
\pgfpathlineto{\pgfqpoint{1.860700in}{1.393845in}}%
\pgfpathlineto{\pgfqpoint{1.825050in}{1.394443in}}%
\pgfpathlineto{\pgfqpoint{1.789400in}{1.394443in}}%
\pgfpathlineto{\pgfqpoint{1.753750in}{1.394443in}}%
\pgfpathlineto{\pgfqpoint{1.718100in}{1.394443in}}%
\pgfpathlineto{\pgfqpoint{1.682450in}{1.394443in}}%
\pgfpathlineto{\pgfqpoint{1.646800in}{1.488783in}}%
\pgfpathlineto{\pgfqpoint{1.611150in}{1.644389in}}%
\pgfpathlineto{\pgfqpoint{1.575500in}{1.699089in}}%
\pgfpathlineto{\pgfqpoint{1.539850in}{1.699089in}}%
\pgfpathlineto{\pgfqpoint{1.504200in}{1.699089in}}%
\pgfpathlineto{\pgfqpoint{1.468550in}{1.699089in}}%
\pgfpathlineto{\pgfqpoint{1.432900in}{1.699089in}}%
\pgfpathlineto{\pgfqpoint{1.397250in}{1.725998in}}%
\pgfpathlineto{\pgfqpoint{1.361600in}{1.725998in}}%
\pgfpathlineto{\pgfqpoint{1.325950in}{1.725998in}}%
\pgfpathlineto{\pgfqpoint{1.290300in}{1.725998in}}%
\pgfpathlineto{\pgfqpoint{1.254650in}{1.750906in}}%
\pgfpathlineto{\pgfqpoint{1.219000in}{1.750906in}}%
\pgfpathlineto{\pgfqpoint{1.183350in}{1.750906in}}%
\pgfpathlineto{\pgfqpoint{1.147700in}{1.750906in}}%
\pgfpathlineto{\pgfqpoint{1.112050in}{1.750906in}}%
\pgfpathlineto{\pgfqpoint{1.076400in}{1.891055in}}%
\pgfpathlineto{\pgfqpoint{1.040750in}{1.891055in}}%
\pgfpathlineto{\pgfqpoint{1.005100in}{1.935372in}}%
\pgfpathlineto{\pgfqpoint{0.969450in}{2.236561in}}%
\pgfpathlineto{\pgfqpoint{0.933800in}{2.272063in}}%
\pgfpathlineto{\pgfqpoint{0.898150in}{2.272063in}}%
\pgfpathlineto{\pgfqpoint{0.862500in}{2.284278in}}%
\pgfpathlineto{\pgfqpoint{0.826850in}{2.284278in}}%
\pgfpathlineto{\pgfqpoint{0.791200in}{2.284278in}}%
\pgfpathlineto{\pgfqpoint{0.755550in}{2.293436in}}%
\pgfpathlineto{\pgfqpoint{0.719900in}{2.293436in}}%
\pgfpathlineto{\pgfqpoint{0.684250in}{2.293436in}}%
\pgfpathlineto{\pgfqpoint{0.648600in}{2.293436in}}%
\pgfpathlineto{\pgfqpoint{0.612950in}{2.293436in}}%
\pgfpathlineto{\pgfqpoint{0.577300in}{2.293436in}}%
\pgfpathlineto{\pgfqpoint{0.541650in}{2.293436in}}%
\pgfpathlineto{\pgfqpoint{0.506000in}{2.293436in}}%
\pgfpathlineto{\pgfqpoint{0.470350in}{2.293436in}}%
\pgfpathlineto{\pgfqpoint{0.434700in}{2.293436in}}%
\pgfpathlineto{\pgfqpoint{0.399050in}{2.293436in}}%
\pgfpathlineto{\pgfqpoint{0.363400in}{2.293436in}}%
\pgfpathlineto{\pgfqpoint{0.327750in}{2.402765in}}%
\pgfpathlineto{\pgfqpoint{0.292100in}{2.402765in}}%
\pgfpathlineto{\pgfqpoint{0.256450in}{2.402765in}}%
\pgfpathlineto{\pgfqpoint{0.220800in}{2.402765in}}%
\pgfpathlineto{\pgfqpoint{0.185150in}{2.402765in}}%
\pgfpathlineto{\pgfqpoint{0.149500in}{2.402765in}}%
\pgfpathlineto{\pgfqpoint{0.113850in}{2.402765in}}%
\pgfpathlineto{\pgfqpoint{0.078200in}{2.402765in}}%
\pgfpathlineto{\pgfqpoint{0.042550in}{2.402765in}}%
\pgfpathlineto{\pgfqpoint{0.006900in}{2.402765in}}%
\pgfpathlineto{\pgfqpoint{-0.028750in}{2.402765in}}%
\pgfpathlineto{\pgfqpoint{-0.064400in}{2.402765in}}%
\pgfpathlineto{\pgfqpoint{-0.100050in}{2.402765in}}%
\pgfpathlineto{\pgfqpoint{-0.135700in}{2.402765in}}%
\pgfpathlineto{\pgfqpoint{-0.171350in}{2.402765in}}%
\pgfpathlineto{\pgfqpoint{-0.207000in}{2.402765in}}%
\pgfpathlineto{\pgfqpoint{-0.242650in}{2.402765in}}%
\pgfpathlineto{\pgfqpoint{-0.278300in}{2.402765in}}%
\pgfpathlineto{\pgfqpoint{-0.313950in}{2.402765in}}%
\pgfpathlineto{\pgfqpoint{-0.349600in}{2.402765in}}%
\pgfpathlineto{\pgfqpoint{-0.385250in}{2.402765in}}%
\pgfpathlineto{\pgfqpoint{-0.420900in}{2.402765in}}%
\pgfpathlineto{\pgfqpoint{-0.456550in}{2.402765in}}%
\pgfpathlineto{\pgfqpoint{-0.492200in}{2.402765in}}%
\pgfpathlineto{\pgfqpoint{-0.527850in}{2.402765in}}%
\pgfpathlineto{\pgfqpoint{-0.563500in}{2.402765in}}%
\pgfpathlineto{\pgfqpoint{-0.599150in}{2.509611in}}%
\pgfpathlineto{\pgfqpoint{-0.634800in}{2.509611in}}%
\pgfpathlineto{\pgfqpoint{-0.670450in}{2.509611in}}%
\pgfpathlineto{\pgfqpoint{-0.706100in}{2.509611in}}%
\pgfpathlineto{\pgfqpoint{-0.741750in}{2.509611in}}%
\pgfpathlineto{\pgfqpoint{-0.777400in}{2.509611in}}%
\pgfpathlineto{\pgfqpoint{-0.813050in}{2.524395in}}%
\pgfpathlineto{\pgfqpoint{-0.848700in}{2.524395in}}%
\pgfpathlineto{\pgfqpoint{-0.884350in}{2.556585in}}%
\pgfpathlineto{\pgfqpoint{-0.920000in}{2.556585in}}%
\pgfpathlineto{\pgfqpoint{-0.955650in}{2.556585in}}%
\pgfpathlineto{\pgfqpoint{-0.991300in}{2.556585in}}%
\pgfpathlineto{\pgfqpoint{-1.026950in}{2.556585in}}%
\pgfpathlineto{\pgfqpoint{-1.062600in}{2.556585in}}%
\pgfpathlineto{\pgfqpoint{-1.098250in}{2.556585in}}%
\pgfpathlineto{\pgfqpoint{-1.133900in}{2.637840in}}%
\pgfpathlineto{\pgfqpoint{-1.169550in}{2.637840in}}%
\pgfpathlineto{\pgfqpoint{-1.205200in}{2.637840in}}%
\pgfpathlineto{\pgfqpoint{-1.240850in}{2.637840in}}%
\pgfpathlineto{\pgfqpoint{-1.276500in}{2.637840in}}%
\pgfpathlineto{\pgfqpoint{-1.312150in}{2.637840in}}%
\pgfpathlineto{\pgfqpoint{-1.347800in}{2.769048in}}%
\pgfpathlineto{\pgfqpoint{-1.383450in}{2.772587in}}%
\pgfpathlineto{\pgfqpoint{-1.419100in}{2.772587in}}%
\pgfpathlineto{\pgfqpoint{-1.454750in}{2.772587in}}%
\pgfpathlineto{\pgfqpoint{-1.490400in}{2.772587in}}%
\pgfpathlineto{\pgfqpoint{-1.526050in}{2.892242in}}%
\pgfpathlineto{\pgfqpoint{-1.561700in}{2.892242in}}%
\pgfpathlineto{\pgfqpoint{-1.597350in}{2.905293in}}%
\pgfpathlineto{\pgfqpoint{-1.633000in}{3.018030in}}%
\pgfpathlineto{\pgfqpoint{-1.668650in}{3.018030in}}%
\pgfpathlineto{\pgfqpoint{-1.704300in}{3.018030in}}%
\pgfpathlineto{\pgfqpoint{-1.739950in}{3.083883in}}%
\pgfpathlineto{\pgfqpoint{-1.775600in}{3.083883in}}%
\pgfpathlineto{\pgfqpoint{-1.811250in}{3.083883in}}%
\pgfpathlineto{\pgfqpoint{-1.846900in}{3.083883in}}%
\pgfpathlineto{\pgfqpoint{-1.882550in}{3.083883in}}%
\pgfpathlineto{\pgfqpoint{-1.918200in}{3.083883in}}%
\pgfpathlineto{\pgfqpoint{-1.953850in}{3.083883in}}%
\pgfpathlineto{\pgfqpoint{-1.989500in}{3.083883in}}%
\pgfpathlineto{\pgfqpoint{-2.025150in}{3.083883in}}%
\pgfpathlineto{\pgfqpoint{-2.060800in}{3.241881in}}%
\pgfpathlineto{\pgfqpoint{-2.096450in}{3.241881in}}%
\pgfpathlineto{\pgfqpoint{-2.132100in}{3.241881in}}%
\pgfpathlineto{\pgfqpoint{-2.167750in}{3.241881in}}%
\pgfpathlineto{\pgfqpoint{-2.203400in}{3.241881in}}%
\pgfpathlineto{\pgfqpoint{-2.239050in}{3.577260in}}%
\pgfpathlineto{\pgfqpoint{-2.274700in}{3.577260in}}%
\pgfpathlineto{\pgfqpoint{-2.310350in}{3.577260in}}%
\pgfpathlineto{\pgfqpoint{-2.346000in}{3.577260in}}%
\pgfpathlineto{\pgfqpoint{-2.381650in}{3.757069in}}%
\pgfpathlineto{\pgfqpoint{-2.417300in}{3.974466in}}%
\pgfpathlineto{\pgfqpoint{-2.452950in}{3.974466in}}%
\pgfpathlineto{\pgfqpoint{-2.488600in}{3.974466in}}%
\pgfpathlineto{\pgfqpoint{-2.524250in}{4.222330in}}%
\pgfpathlineto{\pgfqpoint{-2.559900in}{4.643071in}}%
\pgfpathlineto{\pgfqpoint{-2.595550in}{4.802772in}}%
\pgfpathlineto{\pgfqpoint{-2.631200in}{5.232115in}}%
\pgfpathlineto{\pgfqpoint{-2.666850in}{5.384080in}}%
\pgfpathlineto{\pgfqpoint{-2.702500in}{5.744105in}}%
\pgfpathclose%
\pgfusepath{fill}%
\end{pgfscope}%
\begin{pgfscope}%
\pgfpathrectangle{\pgfqpoint{0.862500in}{0.375000in}}{\pgfqpoint{5.347500in}{2.265000in}}%
\pgfusepath{clip}%
\pgfsetbuttcap%
\pgfsetroundjoin%
\definecolor{currentfill}{rgb}{0.839216,0.152941,0.156863}%
\pgfsetfillcolor{currentfill}%
\pgfsetfillopacity{0.200000}%
\pgfsetlinewidth{0.000000pt}%
\definecolor{currentstroke}{rgb}{0.000000,0.000000,0.000000}%
\pgfsetstrokecolor{currentstroke}%
\pgfsetdash{}{0pt}%
\pgfpathmoveto{\pgfqpoint{-2.702500in}{5.645340in}}%
\pgfpathlineto{\pgfqpoint{-2.702500in}{5.978678in}}%
\pgfpathlineto{\pgfqpoint{-2.666850in}{5.734678in}}%
\pgfpathlineto{\pgfqpoint{-2.631200in}{5.234646in}}%
\pgfpathlineto{\pgfqpoint{-2.595550in}{5.158507in}}%
\pgfpathlineto{\pgfqpoint{-2.559900in}{5.045589in}}%
\pgfpathlineto{\pgfqpoint{-2.524250in}{4.867168in}}%
\pgfpathlineto{\pgfqpoint{-2.488600in}{4.867168in}}%
\pgfpathlineto{\pgfqpoint{-2.452950in}{4.766975in}}%
\pgfpathlineto{\pgfqpoint{-2.417300in}{4.766975in}}%
\pgfpathlineto{\pgfqpoint{-2.381650in}{4.647966in}}%
\pgfpathlineto{\pgfqpoint{-2.346000in}{4.147502in}}%
\pgfpathlineto{\pgfqpoint{-2.310350in}{4.110716in}}%
\pgfpathlineto{\pgfqpoint{-2.274700in}{3.934047in}}%
\pgfpathlineto{\pgfqpoint{-2.239050in}{3.735507in}}%
\pgfpathlineto{\pgfqpoint{-2.203400in}{3.609425in}}%
\pgfpathlineto{\pgfqpoint{-2.167750in}{3.528211in}}%
\pgfpathlineto{\pgfqpoint{-2.132100in}{3.528211in}}%
\pgfpathlineto{\pgfqpoint{-2.096450in}{3.528211in}}%
\pgfpathlineto{\pgfqpoint{-2.060800in}{3.528211in}}%
\pgfpathlineto{\pgfqpoint{-2.025150in}{3.528211in}}%
\pgfpathlineto{\pgfqpoint{-1.989500in}{3.528211in}}%
\pgfpathlineto{\pgfqpoint{-1.953850in}{3.528211in}}%
\pgfpathlineto{\pgfqpoint{-1.918200in}{3.528211in}}%
\pgfpathlineto{\pgfqpoint{-1.882550in}{3.528211in}}%
\pgfpathlineto{\pgfqpoint{-1.846900in}{3.528211in}}%
\pgfpathlineto{\pgfqpoint{-1.811250in}{3.449617in}}%
\pgfpathlineto{\pgfqpoint{-1.775600in}{3.328303in}}%
\pgfpathlineto{\pgfqpoint{-1.739950in}{3.328303in}}%
\pgfpathlineto{\pgfqpoint{-1.704300in}{3.257764in}}%
\pgfpathlineto{\pgfqpoint{-1.668650in}{3.169538in}}%
\pgfpathlineto{\pgfqpoint{-1.633000in}{3.169538in}}%
\pgfpathlineto{\pgfqpoint{-1.597350in}{3.169538in}}%
\pgfpathlineto{\pgfqpoint{-1.561700in}{3.169538in}}%
\pgfpathlineto{\pgfqpoint{-1.526050in}{3.169538in}}%
\pgfpathlineto{\pgfqpoint{-1.490400in}{3.169538in}}%
\pgfpathlineto{\pgfqpoint{-1.454750in}{3.169538in}}%
\pgfpathlineto{\pgfqpoint{-1.419100in}{3.169538in}}%
\pgfpathlineto{\pgfqpoint{-1.383450in}{3.169538in}}%
\pgfpathlineto{\pgfqpoint{-1.347800in}{3.169538in}}%
\pgfpathlineto{\pgfqpoint{-1.312150in}{3.169538in}}%
\pgfpathlineto{\pgfqpoint{-1.276500in}{3.169538in}}%
\pgfpathlineto{\pgfqpoint{-1.240850in}{3.169538in}}%
\pgfpathlineto{\pgfqpoint{-1.205200in}{3.169538in}}%
\pgfpathlineto{\pgfqpoint{-1.169550in}{3.169538in}}%
\pgfpathlineto{\pgfqpoint{-1.133900in}{3.169538in}}%
\pgfpathlineto{\pgfqpoint{-1.098250in}{3.169538in}}%
\pgfpathlineto{\pgfqpoint{-1.062600in}{2.964147in}}%
\pgfpathlineto{\pgfqpoint{-1.026950in}{2.964147in}}%
\pgfpathlineto{\pgfqpoint{-0.991300in}{2.964147in}}%
\pgfpathlineto{\pgfqpoint{-0.955650in}{2.964147in}}%
\pgfpathlineto{\pgfqpoint{-0.920000in}{2.695355in}}%
\pgfpathlineto{\pgfqpoint{-0.884350in}{2.695355in}}%
\pgfpathlineto{\pgfqpoint{-0.848700in}{2.695355in}}%
\pgfpathlineto{\pgfqpoint{-0.813050in}{2.695355in}}%
\pgfpathlineto{\pgfqpoint{-0.777400in}{2.695355in}}%
\pgfpathlineto{\pgfqpoint{-0.741750in}{2.695355in}}%
\pgfpathlineto{\pgfqpoint{-0.706100in}{2.695355in}}%
\pgfpathlineto{\pgfqpoint{-0.670450in}{2.695355in}}%
\pgfpathlineto{\pgfqpoint{-0.634800in}{2.695355in}}%
\pgfpathlineto{\pgfqpoint{-0.599150in}{2.695355in}}%
\pgfpathlineto{\pgfqpoint{-0.563500in}{2.695355in}}%
\pgfpathlineto{\pgfqpoint{-0.527850in}{2.695355in}}%
\pgfpathlineto{\pgfqpoint{-0.492200in}{2.695355in}}%
\pgfpathlineto{\pgfqpoint{-0.456550in}{2.695355in}}%
\pgfpathlineto{\pgfqpoint{-0.420900in}{2.695355in}}%
\pgfpathlineto{\pgfqpoint{-0.385250in}{2.695355in}}%
\pgfpathlineto{\pgfqpoint{-0.349600in}{2.695355in}}%
\pgfpathlineto{\pgfqpoint{-0.313950in}{2.695355in}}%
\pgfpathlineto{\pgfqpoint{-0.278300in}{2.695355in}}%
\pgfpathlineto{\pgfqpoint{-0.242650in}{2.695355in}}%
\pgfpathlineto{\pgfqpoint{-0.207000in}{2.695355in}}%
\pgfpathlineto{\pgfqpoint{-0.171350in}{2.695355in}}%
\pgfpathlineto{\pgfqpoint{-0.135700in}{2.695355in}}%
\pgfpathlineto{\pgfqpoint{-0.100050in}{2.695355in}}%
\pgfpathlineto{\pgfqpoint{-0.064400in}{2.695355in}}%
\pgfpathlineto{\pgfqpoint{-0.028750in}{2.695355in}}%
\pgfpathlineto{\pgfqpoint{0.006900in}{2.695355in}}%
\pgfpathlineto{\pgfqpoint{0.042550in}{2.695355in}}%
\pgfpathlineto{\pgfqpoint{0.078200in}{2.695355in}}%
\pgfpathlineto{\pgfqpoint{0.113850in}{2.695355in}}%
\pgfpathlineto{\pgfqpoint{0.149500in}{2.695355in}}%
\pgfpathlineto{\pgfqpoint{0.185150in}{2.695355in}}%
\pgfpathlineto{\pgfqpoint{0.220800in}{2.695355in}}%
\pgfpathlineto{\pgfqpoint{0.256450in}{2.695355in}}%
\pgfpathlineto{\pgfqpoint{0.292100in}{2.695355in}}%
\pgfpathlineto{\pgfqpoint{0.327750in}{2.695355in}}%
\pgfpathlineto{\pgfqpoint{0.363400in}{2.695355in}}%
\pgfpathlineto{\pgfqpoint{0.399050in}{2.695355in}}%
\pgfpathlineto{\pgfqpoint{0.434700in}{2.611835in}}%
\pgfpathlineto{\pgfqpoint{0.470350in}{2.611835in}}%
\pgfpathlineto{\pgfqpoint{0.506000in}{2.611835in}}%
\pgfpathlineto{\pgfqpoint{0.541650in}{2.611835in}}%
\pgfpathlineto{\pgfqpoint{0.577300in}{2.611835in}}%
\pgfpathlineto{\pgfqpoint{0.612950in}{2.611835in}}%
\pgfpathlineto{\pgfqpoint{0.648600in}{2.611835in}}%
\pgfpathlineto{\pgfqpoint{0.684250in}{2.611835in}}%
\pgfpathlineto{\pgfqpoint{0.719900in}{2.611835in}}%
\pgfpathlineto{\pgfqpoint{0.755550in}{2.611835in}}%
\pgfpathlineto{\pgfqpoint{0.791200in}{2.611835in}}%
\pgfpathlineto{\pgfqpoint{0.826850in}{2.611835in}}%
\pgfpathlineto{\pgfqpoint{0.862500in}{2.611835in}}%
\pgfpathlineto{\pgfqpoint{0.898150in}{2.611835in}}%
\pgfpathlineto{\pgfqpoint{0.933800in}{2.611835in}}%
\pgfpathlineto{\pgfqpoint{0.969450in}{2.611835in}}%
\pgfpathlineto{\pgfqpoint{1.005100in}{2.611835in}}%
\pgfpathlineto{\pgfqpoint{1.040750in}{2.611835in}}%
\pgfpathlineto{\pgfqpoint{1.076400in}{2.571044in}}%
\pgfpathlineto{\pgfqpoint{1.112050in}{1.985378in}}%
\pgfpathlineto{\pgfqpoint{1.147700in}{1.985378in}}%
\pgfpathlineto{\pgfqpoint{1.183350in}{1.985378in}}%
\pgfpathlineto{\pgfqpoint{1.219000in}{1.985378in}}%
\pgfpathlineto{\pgfqpoint{1.254650in}{1.985378in}}%
\pgfpathlineto{\pgfqpoint{1.290300in}{1.985378in}}%
\pgfpathlineto{\pgfqpoint{1.325950in}{1.985378in}}%
\pgfpathlineto{\pgfqpoint{1.361600in}{1.985378in}}%
\pgfpathlineto{\pgfqpoint{1.397250in}{1.914705in}}%
\pgfpathlineto{\pgfqpoint{1.432900in}{1.877080in}}%
\pgfpathlineto{\pgfqpoint{1.468550in}{1.877080in}}%
\pgfpathlineto{\pgfqpoint{1.504200in}{1.861023in}}%
\pgfpathlineto{\pgfqpoint{1.539850in}{1.861023in}}%
\pgfpathlineto{\pgfqpoint{1.575500in}{1.861023in}}%
\pgfpathlineto{\pgfqpoint{1.611150in}{1.856537in}}%
\pgfpathlineto{\pgfqpoint{1.646800in}{1.856537in}}%
\pgfpathlineto{\pgfqpoint{1.682450in}{1.830965in}}%
\pgfpathlineto{\pgfqpoint{1.718100in}{1.830965in}}%
\pgfpathlineto{\pgfqpoint{1.753750in}{1.830965in}}%
\pgfpathlineto{\pgfqpoint{1.789400in}{1.830965in}}%
\pgfpathlineto{\pgfqpoint{1.825050in}{1.830965in}}%
\pgfpathlineto{\pgfqpoint{1.860700in}{1.830965in}}%
\pgfpathlineto{\pgfqpoint{1.896350in}{1.830965in}}%
\pgfpathlineto{\pgfqpoint{1.932000in}{1.830965in}}%
\pgfpathlineto{\pgfqpoint{1.967650in}{1.811848in}}%
\pgfpathlineto{\pgfqpoint{2.003300in}{1.811848in}}%
\pgfpathlineto{\pgfqpoint{2.038950in}{1.811848in}}%
\pgfpathlineto{\pgfqpoint{2.074600in}{1.804099in}}%
\pgfpathlineto{\pgfqpoint{2.110250in}{1.804099in}}%
\pgfpathlineto{\pgfqpoint{2.145900in}{1.804099in}}%
\pgfpathlineto{\pgfqpoint{2.181550in}{1.804099in}}%
\pgfpathlineto{\pgfqpoint{2.217200in}{1.802354in}}%
\pgfpathlineto{\pgfqpoint{2.252850in}{1.802354in}}%
\pgfpathlineto{\pgfqpoint{2.288500in}{1.802354in}}%
\pgfpathlineto{\pgfqpoint{2.324150in}{1.802354in}}%
\pgfpathlineto{\pgfqpoint{2.359800in}{1.679123in}}%
\pgfpathlineto{\pgfqpoint{2.395450in}{1.679123in}}%
\pgfpathlineto{\pgfqpoint{2.431100in}{1.679123in}}%
\pgfpathlineto{\pgfqpoint{2.466750in}{1.679123in}}%
\pgfpathlineto{\pgfqpoint{2.502400in}{1.671666in}}%
\pgfpathlineto{\pgfqpoint{2.538050in}{1.671666in}}%
\pgfpathlineto{\pgfqpoint{2.573700in}{1.671666in}}%
\pgfpathlineto{\pgfqpoint{2.609350in}{1.671666in}}%
\pgfpathlineto{\pgfqpoint{2.645000in}{1.671666in}}%
\pgfpathlineto{\pgfqpoint{2.680650in}{1.671666in}}%
\pgfpathlineto{\pgfqpoint{2.716300in}{1.671666in}}%
\pgfpathlineto{\pgfqpoint{2.751950in}{1.671666in}}%
\pgfpathlineto{\pgfqpoint{2.787600in}{1.671666in}}%
\pgfpathlineto{\pgfqpoint{2.823250in}{1.671666in}}%
\pgfpathlineto{\pgfqpoint{2.858900in}{1.671666in}}%
\pgfpathlineto{\pgfqpoint{2.894550in}{1.662086in}}%
\pgfpathlineto{\pgfqpoint{2.930200in}{1.662086in}}%
\pgfpathlineto{\pgfqpoint{2.965850in}{1.643623in}}%
\pgfpathlineto{\pgfqpoint{3.001500in}{1.677951in}}%
\pgfpathlineto{\pgfqpoint{3.037150in}{1.677951in}}%
\pgfpathlineto{\pgfqpoint{3.072800in}{1.677951in}}%
\pgfpathlineto{\pgfqpoint{3.108450in}{1.677951in}}%
\pgfpathlineto{\pgfqpoint{3.144100in}{1.677951in}}%
\pgfpathlineto{\pgfqpoint{3.179750in}{1.677951in}}%
\pgfpathlineto{\pgfqpoint{3.215400in}{1.677951in}}%
\pgfpathlineto{\pgfqpoint{3.251050in}{1.677951in}}%
\pgfpathlineto{\pgfqpoint{3.286700in}{1.677951in}}%
\pgfpathlineto{\pgfqpoint{3.322350in}{1.677951in}}%
\pgfpathlineto{\pgfqpoint{3.358000in}{1.647032in}}%
\pgfpathlineto{\pgfqpoint{3.393650in}{1.647032in}}%
\pgfpathlineto{\pgfqpoint{3.429300in}{1.647032in}}%
\pgfpathlineto{\pgfqpoint{3.464950in}{1.647032in}}%
\pgfpathlineto{\pgfqpoint{3.500600in}{1.647032in}}%
\pgfpathlineto{\pgfqpoint{3.536250in}{1.647032in}}%
\pgfpathlineto{\pgfqpoint{3.571900in}{1.647032in}}%
\pgfpathlineto{\pgfqpoint{3.607550in}{1.647032in}}%
\pgfpathlineto{\pgfqpoint{3.643200in}{1.647032in}}%
\pgfpathlineto{\pgfqpoint{3.678850in}{1.647032in}}%
\pgfpathlineto{\pgfqpoint{3.714500in}{1.647032in}}%
\pgfpathlineto{\pgfqpoint{3.750150in}{1.647032in}}%
\pgfpathlineto{\pgfqpoint{3.785800in}{1.627207in}}%
\pgfpathlineto{\pgfqpoint{3.821450in}{1.627207in}}%
\pgfpathlineto{\pgfqpoint{3.857100in}{1.590760in}}%
\pgfpathlineto{\pgfqpoint{3.892750in}{1.590760in}}%
\pgfpathlineto{\pgfqpoint{3.928400in}{1.590760in}}%
\pgfpathlineto{\pgfqpoint{3.964050in}{1.590760in}}%
\pgfpathlineto{\pgfqpoint{3.999700in}{1.590760in}}%
\pgfpathlineto{\pgfqpoint{4.035350in}{1.590760in}}%
\pgfpathlineto{\pgfqpoint{4.071000in}{1.590760in}}%
\pgfpathlineto{\pgfqpoint{4.106650in}{1.590760in}}%
\pgfpathlineto{\pgfqpoint{4.142300in}{1.590760in}}%
\pgfpathlineto{\pgfqpoint{4.177950in}{1.590760in}}%
\pgfpathlineto{\pgfqpoint{4.213600in}{1.590760in}}%
\pgfpathlineto{\pgfqpoint{4.249250in}{1.590760in}}%
\pgfpathlineto{\pgfqpoint{4.284900in}{1.590760in}}%
\pgfpathlineto{\pgfqpoint{4.320550in}{1.590760in}}%
\pgfpathlineto{\pgfqpoint{4.356200in}{1.590760in}}%
\pgfpathlineto{\pgfqpoint{4.391850in}{1.590760in}}%
\pgfpathlineto{\pgfqpoint{4.427500in}{1.590760in}}%
\pgfpathlineto{\pgfqpoint{4.463150in}{1.590760in}}%
\pgfpathlineto{\pgfqpoint{4.498800in}{1.590760in}}%
\pgfpathlineto{\pgfqpoint{4.534450in}{1.590760in}}%
\pgfpathlineto{\pgfqpoint{4.570100in}{1.590760in}}%
\pgfpathlineto{\pgfqpoint{4.605750in}{1.590760in}}%
\pgfpathlineto{\pgfqpoint{4.641400in}{1.590760in}}%
\pgfpathlineto{\pgfqpoint{4.677050in}{1.590760in}}%
\pgfpathlineto{\pgfqpoint{4.712700in}{1.590760in}}%
\pgfpathlineto{\pgfqpoint{4.748350in}{1.590760in}}%
\pgfpathlineto{\pgfqpoint{4.784000in}{1.590760in}}%
\pgfpathlineto{\pgfqpoint{4.819650in}{1.590760in}}%
\pgfpathlineto{\pgfqpoint{4.855300in}{1.590760in}}%
\pgfpathlineto{\pgfqpoint{4.890950in}{1.590760in}}%
\pgfpathlineto{\pgfqpoint{4.926600in}{1.590760in}}%
\pgfpathlineto{\pgfqpoint{4.962250in}{1.590760in}}%
\pgfpathlineto{\pgfqpoint{4.997900in}{1.590760in}}%
\pgfpathlineto{\pgfqpoint{5.033550in}{1.590760in}}%
\pgfpathlineto{\pgfqpoint{5.069200in}{1.590760in}}%
\pgfpathlineto{\pgfqpoint{5.104850in}{1.590760in}}%
\pgfpathlineto{\pgfqpoint{5.140500in}{1.590760in}}%
\pgfpathlineto{\pgfqpoint{5.176150in}{1.471031in}}%
\pgfpathlineto{\pgfqpoint{5.211800in}{1.471031in}}%
\pgfpathlineto{\pgfqpoint{5.247450in}{1.471031in}}%
\pgfpathlineto{\pgfqpoint{5.283100in}{1.471031in}}%
\pgfpathlineto{\pgfqpoint{5.318750in}{1.471031in}}%
\pgfpathlineto{\pgfqpoint{5.354400in}{1.471031in}}%
\pgfpathlineto{\pgfqpoint{5.390050in}{1.471031in}}%
\pgfpathlineto{\pgfqpoint{5.425700in}{1.471031in}}%
\pgfpathlineto{\pgfqpoint{5.461350in}{1.471031in}}%
\pgfpathlineto{\pgfqpoint{5.497000in}{1.471031in}}%
\pgfpathlineto{\pgfqpoint{5.532650in}{1.471031in}}%
\pgfpathlineto{\pgfqpoint{5.568300in}{1.471031in}}%
\pgfpathlineto{\pgfqpoint{5.603950in}{1.471031in}}%
\pgfpathlineto{\pgfqpoint{5.639600in}{1.432086in}}%
\pgfpathlineto{\pgfqpoint{5.675250in}{1.432086in}}%
\pgfpathlineto{\pgfqpoint{5.710900in}{1.432086in}}%
\pgfpathlineto{\pgfqpoint{5.746550in}{1.432086in}}%
\pgfpathlineto{\pgfqpoint{5.782200in}{1.432086in}}%
\pgfpathlineto{\pgfqpoint{5.817850in}{1.432086in}}%
\pgfpathlineto{\pgfqpoint{5.853500in}{1.432086in}}%
\pgfpathlineto{\pgfqpoint{5.889150in}{1.432086in}}%
\pgfpathlineto{\pgfqpoint{5.924800in}{1.432086in}}%
\pgfpathlineto{\pgfqpoint{5.960450in}{1.432086in}}%
\pgfpathlineto{\pgfqpoint{5.996100in}{1.432086in}}%
\pgfpathlineto{\pgfqpoint{6.031750in}{1.432086in}}%
\pgfpathlineto{\pgfqpoint{6.067400in}{1.432086in}}%
\pgfpathlineto{\pgfqpoint{6.103050in}{1.432086in}}%
\pgfpathlineto{\pgfqpoint{6.138700in}{1.432086in}}%
\pgfpathlineto{\pgfqpoint{6.174350in}{1.432086in}}%
\pgfpathlineto{\pgfqpoint{6.210000in}{1.432086in}}%
\pgfpathlineto{\pgfqpoint{6.245650in}{1.432086in}}%
\pgfpathlineto{\pgfqpoint{6.281300in}{1.432086in}}%
\pgfpathlineto{\pgfqpoint{6.316950in}{1.432086in}}%
\pgfpathlineto{\pgfqpoint{6.352600in}{1.432086in}}%
\pgfpathlineto{\pgfqpoint{6.388250in}{1.432086in}}%
\pgfpathlineto{\pgfqpoint{6.423900in}{1.432086in}}%
\pgfpathlineto{\pgfqpoint{6.459550in}{1.432086in}}%
\pgfpathlineto{\pgfqpoint{6.495200in}{1.432086in}}%
\pgfpathlineto{\pgfqpoint{6.530850in}{1.414483in}}%
\pgfpathlineto{\pgfqpoint{6.566500in}{1.414483in}}%
\pgfpathlineto{\pgfqpoint{6.602150in}{1.414483in}}%
\pgfpathlineto{\pgfqpoint{6.637800in}{1.414483in}}%
\pgfpathlineto{\pgfqpoint{6.673450in}{1.414483in}}%
\pgfpathlineto{\pgfqpoint{6.709100in}{1.414483in}}%
\pgfpathlineto{\pgfqpoint{6.744750in}{1.414483in}}%
\pgfpathlineto{\pgfqpoint{6.780400in}{1.047844in}}%
\pgfpathlineto{\pgfqpoint{6.816050in}{1.047844in}}%
\pgfpathlineto{\pgfqpoint{6.851700in}{1.047844in}}%
\pgfpathlineto{\pgfqpoint{6.887350in}{1.047844in}}%
\pgfpathlineto{\pgfqpoint{6.923000in}{1.047844in}}%
\pgfpathlineto{\pgfqpoint{6.958650in}{1.047844in}}%
\pgfpathlineto{\pgfqpoint{6.994300in}{1.047844in}}%
\pgfpathlineto{\pgfqpoint{7.029950in}{1.047844in}}%
\pgfpathlineto{\pgfqpoint{7.065600in}{1.047844in}}%
\pgfpathlineto{\pgfqpoint{7.101250in}{1.047844in}}%
\pgfpathlineto{\pgfqpoint{7.136900in}{1.047844in}}%
\pgfpathlineto{\pgfqpoint{7.172550in}{1.047844in}}%
\pgfpathlineto{\pgfqpoint{7.208200in}{1.047844in}}%
\pgfpathlineto{\pgfqpoint{7.243850in}{1.047844in}}%
\pgfpathlineto{\pgfqpoint{7.279500in}{1.047844in}}%
\pgfpathlineto{\pgfqpoint{7.315150in}{0.983088in}}%
\pgfpathlineto{\pgfqpoint{7.350800in}{0.983088in}}%
\pgfpathlineto{\pgfqpoint{7.386450in}{0.983088in}}%
\pgfpathlineto{\pgfqpoint{7.422100in}{0.849497in}}%
\pgfpathlineto{\pgfqpoint{7.457750in}{0.849497in}}%
\pgfpathlineto{\pgfqpoint{7.493400in}{0.849497in}}%
\pgfpathlineto{\pgfqpoint{7.529050in}{0.849497in}}%
\pgfpathlineto{\pgfqpoint{7.564700in}{0.849497in}}%
\pgfpathlineto{\pgfqpoint{7.600350in}{0.849497in}}%
\pgfpathlineto{\pgfqpoint{7.636000in}{0.842629in}}%
\pgfpathlineto{\pgfqpoint{7.671650in}{0.842629in}}%
\pgfpathlineto{\pgfqpoint{7.707300in}{0.842629in}}%
\pgfpathlineto{\pgfqpoint{7.742950in}{0.842629in}}%
\pgfpathlineto{\pgfqpoint{7.778600in}{0.813123in}}%
\pgfpathlineto{\pgfqpoint{7.814250in}{0.813123in}}%
\pgfpathlineto{\pgfqpoint{7.849900in}{0.813123in}}%
\pgfpathlineto{\pgfqpoint{7.885550in}{0.813123in}}%
\pgfpathlineto{\pgfqpoint{7.921200in}{0.813123in}}%
\pgfpathlineto{\pgfqpoint{7.956850in}{0.813123in}}%
\pgfpathlineto{\pgfqpoint{7.956850in}{0.508742in}}%
\pgfpathlineto{\pgfqpoint{7.956850in}{0.508742in}}%
\pgfpathlineto{\pgfqpoint{7.921200in}{0.508742in}}%
\pgfpathlineto{\pgfqpoint{7.885550in}{0.508742in}}%
\pgfpathlineto{\pgfqpoint{7.849900in}{0.508742in}}%
\pgfpathlineto{\pgfqpoint{7.814250in}{0.508742in}}%
\pgfpathlineto{\pgfqpoint{7.778600in}{0.508742in}}%
\pgfpathlineto{\pgfqpoint{7.742950in}{0.598812in}}%
\pgfpathlineto{\pgfqpoint{7.707300in}{0.598812in}}%
\pgfpathlineto{\pgfqpoint{7.671650in}{0.598812in}}%
\pgfpathlineto{\pgfqpoint{7.636000in}{0.598812in}}%
\pgfpathlineto{\pgfqpoint{7.600350in}{0.600856in}}%
\pgfpathlineto{\pgfqpoint{7.564700in}{0.600856in}}%
\pgfpathlineto{\pgfqpoint{7.529050in}{0.600856in}}%
\pgfpathlineto{\pgfqpoint{7.493400in}{0.600856in}}%
\pgfpathlineto{\pgfqpoint{7.457750in}{0.600856in}}%
\pgfpathlineto{\pgfqpoint{7.422100in}{0.600856in}}%
\pgfpathlineto{\pgfqpoint{7.386450in}{0.773964in}}%
\pgfpathlineto{\pgfqpoint{7.350800in}{0.773964in}}%
\pgfpathlineto{\pgfqpoint{7.315150in}{0.773964in}}%
\pgfpathlineto{\pgfqpoint{7.279500in}{0.809974in}}%
\pgfpathlineto{\pgfqpoint{7.243850in}{0.809974in}}%
\pgfpathlineto{\pgfqpoint{7.208200in}{0.809974in}}%
\pgfpathlineto{\pgfqpoint{7.172550in}{0.809974in}}%
\pgfpathlineto{\pgfqpoint{7.136900in}{0.809974in}}%
\pgfpathlineto{\pgfqpoint{7.101250in}{0.809974in}}%
\pgfpathlineto{\pgfqpoint{7.065600in}{0.809974in}}%
\pgfpathlineto{\pgfqpoint{7.029950in}{0.809974in}}%
\pgfpathlineto{\pgfqpoint{6.994300in}{0.809974in}}%
\pgfpathlineto{\pgfqpoint{6.958650in}{0.809974in}}%
\pgfpathlineto{\pgfqpoint{6.923000in}{0.809974in}}%
\pgfpathlineto{\pgfqpoint{6.887350in}{0.809974in}}%
\pgfpathlineto{\pgfqpoint{6.851700in}{0.809974in}}%
\pgfpathlineto{\pgfqpoint{6.816050in}{0.809974in}}%
\pgfpathlineto{\pgfqpoint{6.780400in}{0.809974in}}%
\pgfpathlineto{\pgfqpoint{6.744750in}{1.166438in}}%
\pgfpathlineto{\pgfqpoint{6.709100in}{1.166438in}}%
\pgfpathlineto{\pgfqpoint{6.673450in}{1.166438in}}%
\pgfpathlineto{\pgfqpoint{6.637800in}{1.166438in}}%
\pgfpathlineto{\pgfqpoint{6.602150in}{1.166438in}}%
\pgfpathlineto{\pgfqpoint{6.566500in}{1.166438in}}%
\pgfpathlineto{\pgfqpoint{6.530850in}{1.166438in}}%
\pgfpathlineto{\pgfqpoint{6.495200in}{1.202348in}}%
\pgfpathlineto{\pgfqpoint{6.459550in}{1.202348in}}%
\pgfpathlineto{\pgfqpoint{6.423900in}{1.202348in}}%
\pgfpathlineto{\pgfqpoint{6.388250in}{1.202348in}}%
\pgfpathlineto{\pgfqpoint{6.352600in}{1.202348in}}%
\pgfpathlineto{\pgfqpoint{6.316950in}{1.202348in}}%
\pgfpathlineto{\pgfqpoint{6.281300in}{1.202348in}}%
\pgfpathlineto{\pgfqpoint{6.245650in}{1.202348in}}%
\pgfpathlineto{\pgfqpoint{6.210000in}{1.202348in}}%
\pgfpathlineto{\pgfqpoint{6.174350in}{1.202348in}}%
\pgfpathlineto{\pgfqpoint{6.138700in}{1.202348in}}%
\pgfpathlineto{\pgfqpoint{6.103050in}{1.202348in}}%
\pgfpathlineto{\pgfqpoint{6.067400in}{1.202348in}}%
\pgfpathlineto{\pgfqpoint{6.031750in}{1.202348in}}%
\pgfpathlineto{\pgfqpoint{5.996100in}{1.202348in}}%
\pgfpathlineto{\pgfqpoint{5.960450in}{1.202348in}}%
\pgfpathlineto{\pgfqpoint{5.924800in}{1.202348in}}%
\pgfpathlineto{\pgfqpoint{5.889150in}{1.202348in}}%
\pgfpathlineto{\pgfqpoint{5.853500in}{1.202348in}}%
\pgfpathlineto{\pgfqpoint{5.817850in}{1.202348in}}%
\pgfpathlineto{\pgfqpoint{5.782200in}{1.202348in}}%
\pgfpathlineto{\pgfqpoint{5.746550in}{1.202348in}}%
\pgfpathlineto{\pgfqpoint{5.710900in}{1.202348in}}%
\pgfpathlineto{\pgfqpoint{5.675250in}{1.202348in}}%
\pgfpathlineto{\pgfqpoint{5.639600in}{1.202348in}}%
\pgfpathlineto{\pgfqpoint{5.603950in}{1.260252in}}%
\pgfpathlineto{\pgfqpoint{5.568300in}{1.260252in}}%
\pgfpathlineto{\pgfqpoint{5.532650in}{1.260252in}}%
\pgfpathlineto{\pgfqpoint{5.497000in}{1.260252in}}%
\pgfpathlineto{\pgfqpoint{5.461350in}{1.260252in}}%
\pgfpathlineto{\pgfqpoint{5.425700in}{1.260252in}}%
\pgfpathlineto{\pgfqpoint{5.390050in}{1.260252in}}%
\pgfpathlineto{\pgfqpoint{5.354400in}{1.260252in}}%
\pgfpathlineto{\pgfqpoint{5.318750in}{1.260252in}}%
\pgfpathlineto{\pgfqpoint{5.283100in}{1.260252in}}%
\pgfpathlineto{\pgfqpoint{5.247450in}{1.260252in}}%
\pgfpathlineto{\pgfqpoint{5.211800in}{1.260252in}}%
\pgfpathlineto{\pgfqpoint{5.176150in}{1.260252in}}%
\pgfpathlineto{\pgfqpoint{5.140500in}{1.355719in}}%
\pgfpathlineto{\pgfqpoint{5.104850in}{1.355719in}}%
\pgfpathlineto{\pgfqpoint{5.069200in}{1.355719in}}%
\pgfpathlineto{\pgfqpoint{5.033550in}{1.355719in}}%
\pgfpathlineto{\pgfqpoint{4.997900in}{1.355719in}}%
\pgfpathlineto{\pgfqpoint{4.962250in}{1.355719in}}%
\pgfpathlineto{\pgfqpoint{4.926600in}{1.355719in}}%
\pgfpathlineto{\pgfqpoint{4.890950in}{1.355719in}}%
\pgfpathlineto{\pgfqpoint{4.855300in}{1.355719in}}%
\pgfpathlineto{\pgfqpoint{4.819650in}{1.355719in}}%
\pgfpathlineto{\pgfqpoint{4.784000in}{1.355719in}}%
\pgfpathlineto{\pgfqpoint{4.748350in}{1.355719in}}%
\pgfpathlineto{\pgfqpoint{4.712700in}{1.355719in}}%
\pgfpathlineto{\pgfqpoint{4.677050in}{1.355719in}}%
\pgfpathlineto{\pgfqpoint{4.641400in}{1.355719in}}%
\pgfpathlineto{\pgfqpoint{4.605750in}{1.355719in}}%
\pgfpathlineto{\pgfqpoint{4.570100in}{1.355719in}}%
\pgfpathlineto{\pgfqpoint{4.534450in}{1.355719in}}%
\pgfpathlineto{\pgfqpoint{4.498800in}{1.355719in}}%
\pgfpathlineto{\pgfqpoint{4.463150in}{1.355719in}}%
\pgfpathlineto{\pgfqpoint{4.427500in}{1.355719in}}%
\pgfpathlineto{\pgfqpoint{4.391850in}{1.355719in}}%
\pgfpathlineto{\pgfqpoint{4.356200in}{1.355719in}}%
\pgfpathlineto{\pgfqpoint{4.320550in}{1.355719in}}%
\pgfpathlineto{\pgfqpoint{4.284900in}{1.355719in}}%
\pgfpathlineto{\pgfqpoint{4.249250in}{1.355719in}}%
\pgfpathlineto{\pgfqpoint{4.213600in}{1.355719in}}%
\pgfpathlineto{\pgfqpoint{4.177950in}{1.355719in}}%
\pgfpathlineto{\pgfqpoint{4.142300in}{1.355719in}}%
\pgfpathlineto{\pgfqpoint{4.106650in}{1.355719in}}%
\pgfpathlineto{\pgfqpoint{4.071000in}{1.355719in}}%
\pgfpathlineto{\pgfqpoint{4.035350in}{1.355719in}}%
\pgfpathlineto{\pgfqpoint{3.999700in}{1.355719in}}%
\pgfpathlineto{\pgfqpoint{3.964050in}{1.355719in}}%
\pgfpathlineto{\pgfqpoint{3.928400in}{1.355719in}}%
\pgfpathlineto{\pgfqpoint{3.892750in}{1.355719in}}%
\pgfpathlineto{\pgfqpoint{3.857100in}{1.355719in}}%
\pgfpathlineto{\pgfqpoint{3.821450in}{1.383462in}}%
\pgfpathlineto{\pgfqpoint{3.785800in}{1.383462in}}%
\pgfpathlineto{\pgfqpoint{3.750150in}{1.524387in}}%
\pgfpathlineto{\pgfqpoint{3.714500in}{1.524387in}}%
\pgfpathlineto{\pgfqpoint{3.678850in}{1.524387in}}%
\pgfpathlineto{\pgfqpoint{3.643200in}{1.524387in}}%
\pgfpathlineto{\pgfqpoint{3.607550in}{1.524387in}}%
\pgfpathlineto{\pgfqpoint{3.571900in}{1.524387in}}%
\pgfpathlineto{\pgfqpoint{3.536250in}{1.524387in}}%
\pgfpathlineto{\pgfqpoint{3.500600in}{1.524387in}}%
\pgfpathlineto{\pgfqpoint{3.464950in}{1.524387in}}%
\pgfpathlineto{\pgfqpoint{3.429300in}{1.524387in}}%
\pgfpathlineto{\pgfqpoint{3.393650in}{1.524387in}}%
\pgfpathlineto{\pgfqpoint{3.358000in}{1.524387in}}%
\pgfpathlineto{\pgfqpoint{3.322350in}{1.562422in}}%
\pgfpathlineto{\pgfqpoint{3.286700in}{1.562422in}}%
\pgfpathlineto{\pgfqpoint{3.251050in}{1.562422in}}%
\pgfpathlineto{\pgfqpoint{3.215400in}{1.562422in}}%
\pgfpathlineto{\pgfqpoint{3.179750in}{1.562422in}}%
\pgfpathlineto{\pgfqpoint{3.144100in}{1.562422in}}%
\pgfpathlineto{\pgfqpoint{3.108450in}{1.562422in}}%
\pgfpathlineto{\pgfqpoint{3.072800in}{1.562422in}}%
\pgfpathlineto{\pgfqpoint{3.037150in}{1.562422in}}%
\pgfpathlineto{\pgfqpoint{3.001500in}{1.562422in}}%
\pgfpathlineto{\pgfqpoint{2.965850in}{1.496773in}}%
\pgfpathlineto{\pgfqpoint{2.930200in}{1.506422in}}%
\pgfpathlineto{\pgfqpoint{2.894550in}{1.506422in}}%
\pgfpathlineto{\pgfqpoint{2.858900in}{1.512770in}}%
\pgfpathlineto{\pgfqpoint{2.823250in}{1.512770in}}%
\pgfpathlineto{\pgfqpoint{2.787600in}{1.512770in}}%
\pgfpathlineto{\pgfqpoint{2.751950in}{1.512770in}}%
\pgfpathlineto{\pgfqpoint{2.716300in}{1.512770in}}%
\pgfpathlineto{\pgfqpoint{2.680650in}{1.512770in}}%
\pgfpathlineto{\pgfqpoint{2.645000in}{1.512770in}}%
\pgfpathlineto{\pgfqpoint{2.609350in}{1.512770in}}%
\pgfpathlineto{\pgfqpoint{2.573700in}{1.512770in}}%
\pgfpathlineto{\pgfqpoint{2.538050in}{1.512770in}}%
\pgfpathlineto{\pgfqpoint{2.502400in}{1.512770in}}%
\pgfpathlineto{\pgfqpoint{2.466750in}{1.516253in}}%
\pgfpathlineto{\pgfqpoint{2.431100in}{1.516253in}}%
\pgfpathlineto{\pgfqpoint{2.395450in}{1.516253in}}%
\pgfpathlineto{\pgfqpoint{2.359800in}{1.516253in}}%
\pgfpathlineto{\pgfqpoint{2.324150in}{1.628899in}}%
\pgfpathlineto{\pgfqpoint{2.288500in}{1.628899in}}%
\pgfpathlineto{\pgfqpoint{2.252850in}{1.628899in}}%
\pgfpathlineto{\pgfqpoint{2.217200in}{1.628899in}}%
\pgfpathlineto{\pgfqpoint{2.181550in}{1.630643in}}%
\pgfpathlineto{\pgfqpoint{2.145900in}{1.630643in}}%
\pgfpathlineto{\pgfqpoint{2.110250in}{1.630643in}}%
\pgfpathlineto{\pgfqpoint{2.074600in}{1.630643in}}%
\pgfpathlineto{\pgfqpoint{2.038950in}{1.632447in}}%
\pgfpathlineto{\pgfqpoint{2.003300in}{1.632447in}}%
\pgfpathlineto{\pgfqpoint{1.967650in}{1.632447in}}%
\pgfpathlineto{\pgfqpoint{1.932000in}{1.636430in}}%
\pgfpathlineto{\pgfqpoint{1.896350in}{1.636430in}}%
\pgfpathlineto{\pgfqpoint{1.860700in}{1.636430in}}%
\pgfpathlineto{\pgfqpoint{1.825050in}{1.636430in}}%
\pgfpathlineto{\pgfqpoint{1.789400in}{1.636430in}}%
\pgfpathlineto{\pgfqpoint{1.753750in}{1.636430in}}%
\pgfpathlineto{\pgfqpoint{1.718100in}{1.636430in}}%
\pgfpathlineto{\pgfqpoint{1.682450in}{1.636430in}}%
\pgfpathlineto{\pgfqpoint{1.646800in}{1.640990in}}%
\pgfpathlineto{\pgfqpoint{1.611150in}{1.640990in}}%
\pgfpathlineto{\pgfqpoint{1.575500in}{1.669564in}}%
\pgfpathlineto{\pgfqpoint{1.539850in}{1.669564in}}%
\pgfpathlineto{\pgfqpoint{1.504200in}{1.669564in}}%
\pgfpathlineto{\pgfqpoint{1.468550in}{1.687366in}}%
\pgfpathlineto{\pgfqpoint{1.432900in}{1.687366in}}%
\pgfpathlineto{\pgfqpoint{1.397250in}{1.784385in}}%
\pgfpathlineto{\pgfqpoint{1.361600in}{1.826780in}}%
\pgfpathlineto{\pgfqpoint{1.325950in}{1.826780in}}%
\pgfpathlineto{\pgfqpoint{1.290300in}{1.826780in}}%
\pgfpathlineto{\pgfqpoint{1.254650in}{1.826780in}}%
\pgfpathlineto{\pgfqpoint{1.219000in}{1.826780in}}%
\pgfpathlineto{\pgfqpoint{1.183350in}{1.826780in}}%
\pgfpathlineto{\pgfqpoint{1.147700in}{1.826780in}}%
\pgfpathlineto{\pgfqpoint{1.112050in}{1.826780in}}%
\pgfpathlineto{\pgfqpoint{1.076400in}{2.062497in}}%
\pgfpathlineto{\pgfqpoint{1.040750in}{2.150673in}}%
\pgfpathlineto{\pgfqpoint{1.005100in}{2.150673in}}%
\pgfpathlineto{\pgfqpoint{0.969450in}{2.150673in}}%
\pgfpathlineto{\pgfqpoint{0.933800in}{2.150673in}}%
\pgfpathlineto{\pgfqpoint{0.898150in}{2.150673in}}%
\pgfpathlineto{\pgfqpoint{0.862500in}{2.150673in}}%
\pgfpathlineto{\pgfqpoint{0.826850in}{2.150673in}}%
\pgfpathlineto{\pgfqpoint{0.791200in}{2.150673in}}%
\pgfpathlineto{\pgfqpoint{0.755550in}{2.150673in}}%
\pgfpathlineto{\pgfqpoint{0.719900in}{2.150673in}}%
\pgfpathlineto{\pgfqpoint{0.684250in}{2.150673in}}%
\pgfpathlineto{\pgfqpoint{0.648600in}{2.150673in}}%
\pgfpathlineto{\pgfqpoint{0.612950in}{2.150673in}}%
\pgfpathlineto{\pgfqpoint{0.577300in}{2.150673in}}%
\pgfpathlineto{\pgfqpoint{0.541650in}{2.150673in}}%
\pgfpathlineto{\pgfqpoint{0.506000in}{2.150673in}}%
\pgfpathlineto{\pgfqpoint{0.470350in}{2.150673in}}%
\pgfpathlineto{\pgfqpoint{0.434700in}{2.150673in}}%
\pgfpathlineto{\pgfqpoint{0.399050in}{2.354529in}}%
\pgfpathlineto{\pgfqpoint{0.363400in}{2.354529in}}%
\pgfpathlineto{\pgfqpoint{0.327750in}{2.354529in}}%
\pgfpathlineto{\pgfqpoint{0.292100in}{2.354529in}}%
\pgfpathlineto{\pgfqpoint{0.256450in}{2.354529in}}%
\pgfpathlineto{\pgfqpoint{0.220800in}{2.354529in}}%
\pgfpathlineto{\pgfqpoint{0.185150in}{2.354529in}}%
\pgfpathlineto{\pgfqpoint{0.149500in}{2.354529in}}%
\pgfpathlineto{\pgfqpoint{0.113850in}{2.354529in}}%
\pgfpathlineto{\pgfqpoint{0.078200in}{2.354529in}}%
\pgfpathlineto{\pgfqpoint{0.042550in}{2.354529in}}%
\pgfpathlineto{\pgfqpoint{0.006900in}{2.354529in}}%
\pgfpathlineto{\pgfqpoint{-0.028750in}{2.354529in}}%
\pgfpathlineto{\pgfqpoint{-0.064400in}{2.354529in}}%
\pgfpathlineto{\pgfqpoint{-0.100050in}{2.354529in}}%
\pgfpathlineto{\pgfqpoint{-0.135700in}{2.354529in}}%
\pgfpathlineto{\pgfqpoint{-0.171350in}{2.354529in}}%
\pgfpathlineto{\pgfqpoint{-0.207000in}{2.354529in}}%
\pgfpathlineto{\pgfqpoint{-0.242650in}{2.354529in}}%
\pgfpathlineto{\pgfqpoint{-0.278300in}{2.354529in}}%
\pgfpathlineto{\pgfqpoint{-0.313950in}{2.354529in}}%
\pgfpathlineto{\pgfqpoint{-0.349600in}{2.354529in}}%
\pgfpathlineto{\pgfqpoint{-0.385250in}{2.354529in}}%
\pgfpathlineto{\pgfqpoint{-0.420900in}{2.354529in}}%
\pgfpathlineto{\pgfqpoint{-0.456550in}{2.354529in}}%
\pgfpathlineto{\pgfqpoint{-0.492200in}{2.354529in}}%
\pgfpathlineto{\pgfqpoint{-0.527850in}{2.354529in}}%
\pgfpathlineto{\pgfqpoint{-0.563500in}{2.354529in}}%
\pgfpathlineto{\pgfqpoint{-0.599150in}{2.354529in}}%
\pgfpathlineto{\pgfqpoint{-0.634800in}{2.354529in}}%
\pgfpathlineto{\pgfqpoint{-0.670450in}{2.354529in}}%
\pgfpathlineto{\pgfqpoint{-0.706100in}{2.354529in}}%
\pgfpathlineto{\pgfqpoint{-0.741750in}{2.354529in}}%
\pgfpathlineto{\pgfqpoint{-0.777400in}{2.354529in}}%
\pgfpathlineto{\pgfqpoint{-0.813050in}{2.354529in}}%
\pgfpathlineto{\pgfqpoint{-0.848700in}{2.354529in}}%
\pgfpathlineto{\pgfqpoint{-0.884350in}{2.354529in}}%
\pgfpathlineto{\pgfqpoint{-0.920000in}{2.354529in}}%
\pgfpathlineto{\pgfqpoint{-0.955650in}{2.490061in}}%
\pgfpathlineto{\pgfqpoint{-0.991300in}{2.490061in}}%
\pgfpathlineto{\pgfqpoint{-1.026950in}{2.490061in}}%
\pgfpathlineto{\pgfqpoint{-1.062600in}{2.490061in}}%
\pgfpathlineto{\pgfqpoint{-1.098250in}{2.738642in}}%
\pgfpathlineto{\pgfqpoint{-1.133900in}{2.738642in}}%
\pgfpathlineto{\pgfqpoint{-1.169550in}{2.738642in}}%
\pgfpathlineto{\pgfqpoint{-1.205200in}{2.738642in}}%
\pgfpathlineto{\pgfqpoint{-1.240850in}{2.738642in}}%
\pgfpathlineto{\pgfqpoint{-1.276500in}{2.738642in}}%
\pgfpathlineto{\pgfqpoint{-1.312150in}{2.738642in}}%
\pgfpathlineto{\pgfqpoint{-1.347800in}{2.738642in}}%
\pgfpathlineto{\pgfqpoint{-1.383450in}{2.738642in}}%
\pgfpathlineto{\pgfqpoint{-1.419100in}{2.738642in}}%
\pgfpathlineto{\pgfqpoint{-1.454750in}{2.738642in}}%
\pgfpathlineto{\pgfqpoint{-1.490400in}{2.738642in}}%
\pgfpathlineto{\pgfqpoint{-1.526050in}{2.738642in}}%
\pgfpathlineto{\pgfqpoint{-1.561700in}{2.738642in}}%
\pgfpathlineto{\pgfqpoint{-1.597350in}{2.738642in}}%
\pgfpathlineto{\pgfqpoint{-1.633000in}{2.738642in}}%
\pgfpathlineto{\pgfqpoint{-1.668650in}{2.738642in}}%
\pgfpathlineto{\pgfqpoint{-1.704300in}{2.841268in}}%
\pgfpathlineto{\pgfqpoint{-1.739950in}{3.094972in}}%
\pgfpathlineto{\pgfqpoint{-1.775600in}{3.094972in}}%
\pgfpathlineto{\pgfqpoint{-1.811250in}{3.147112in}}%
\pgfpathlineto{\pgfqpoint{-1.846900in}{3.233855in}}%
\pgfpathlineto{\pgfqpoint{-1.882550in}{3.233855in}}%
\pgfpathlineto{\pgfqpoint{-1.918200in}{3.233855in}}%
\pgfpathlineto{\pgfqpoint{-1.953850in}{3.233855in}}%
\pgfpathlineto{\pgfqpoint{-1.989500in}{3.233855in}}%
\pgfpathlineto{\pgfqpoint{-2.025150in}{3.233855in}}%
\pgfpathlineto{\pgfqpoint{-2.060800in}{3.233855in}}%
\pgfpathlineto{\pgfqpoint{-2.096450in}{3.233855in}}%
\pgfpathlineto{\pgfqpoint{-2.132100in}{3.233855in}}%
\pgfpathlineto{\pgfqpoint{-2.167750in}{3.233855in}}%
\pgfpathlineto{\pgfqpoint{-2.203400in}{3.279087in}}%
\pgfpathlineto{\pgfqpoint{-2.239050in}{3.355332in}}%
\pgfpathlineto{\pgfqpoint{-2.274700in}{3.538291in}}%
\pgfpathlineto{\pgfqpoint{-2.310350in}{3.855619in}}%
\pgfpathlineto{\pgfqpoint{-2.346000in}{4.001832in}}%
\pgfpathlineto{\pgfqpoint{-2.381650in}{4.118795in}}%
\pgfpathlineto{\pgfqpoint{-2.417300in}{4.355124in}}%
\pgfpathlineto{\pgfqpoint{-2.452950in}{4.355124in}}%
\pgfpathlineto{\pgfqpoint{-2.488600in}{4.375632in}}%
\pgfpathlineto{\pgfqpoint{-2.524250in}{4.375632in}}%
\pgfpathlineto{\pgfqpoint{-2.559900in}{4.732734in}}%
\pgfpathlineto{\pgfqpoint{-2.595550in}{4.846577in}}%
\pgfpathlineto{\pgfqpoint{-2.631200in}{4.910715in}}%
\pgfpathlineto{\pgfqpoint{-2.666850in}{5.064790in}}%
\pgfpathlineto{\pgfqpoint{-2.702500in}{5.645340in}}%
\pgfpathclose%
\pgfusepath{fill}%
\end{pgfscope}%
\begin{pgfscope}%
\pgfpathrectangle{\pgfqpoint{0.862500in}{0.375000in}}{\pgfqpoint{5.347500in}{2.265000in}}%
\pgfusepath{clip}%
\pgfsetroundcap%
\pgfsetroundjoin%
\pgfsetlinewidth{1.505625pt}%
\definecolor{currentstroke}{rgb}{0.121569,0.466667,0.705882}%
\pgfsetstrokecolor{currentstroke}%
\pgfsetdash{}{0pt}%
\pgfpathmoveto{\pgfqpoint{0.848611in}{2.184070in}}%
\pgfpathlineto{\pgfqpoint{1.040750in}{2.184070in}}%
\pgfpathlineto{\pgfqpoint{1.076400in}{2.181932in}}%
\pgfpathlineto{\pgfqpoint{1.112050in}{2.181932in}}%
\pgfpathlineto{\pgfqpoint{1.147700in}{2.174728in}}%
\pgfpathlineto{\pgfqpoint{1.183350in}{2.157697in}}%
\pgfpathlineto{\pgfqpoint{1.219000in}{2.080354in}}%
\pgfpathlineto{\pgfqpoint{2.110250in}{2.080354in}}%
\pgfpathlineto{\pgfqpoint{2.145900in}{2.075249in}}%
\pgfpathlineto{\pgfqpoint{2.217200in}{2.075249in}}%
\pgfpathlineto{\pgfqpoint{2.252850in}{2.070523in}}%
\pgfpathlineto{\pgfqpoint{3.251050in}{2.070523in}}%
\pgfpathlineto{\pgfqpoint{3.286700in}{2.064027in}}%
\pgfpathlineto{\pgfqpoint{3.322350in}{2.064027in}}%
\pgfpathlineto{\pgfqpoint{3.358000in}{2.024076in}}%
\pgfpathlineto{\pgfqpoint{3.999700in}{2.024076in}}%
\pgfpathlineto{\pgfqpoint{4.035350in}{1.970891in}}%
\pgfpathlineto{\pgfqpoint{5.746550in}{1.970891in}}%
\pgfpathlineto{\pgfqpoint{5.782200in}{1.962804in}}%
\pgfpathlineto{\pgfqpoint{6.223889in}{1.962804in}}%
\pgfpathlineto{\pgfqpoint{6.223889in}{1.962804in}}%
\pgfusepath{stroke}%
\end{pgfscope}%
\begin{pgfscope}%
\pgfpathrectangle{\pgfqpoint{0.862500in}{0.375000in}}{\pgfqpoint{5.347500in}{2.265000in}}%
\pgfusepath{clip}%
\pgfsetroundcap%
\pgfsetroundjoin%
\pgfsetlinewidth{1.505625pt}%
\definecolor{currentstroke}{rgb}{1.000000,0.498039,0.054902}%
\pgfsetstrokecolor{currentstroke}%
\pgfsetdash{}{0pt}%
\pgfpathmoveto{\pgfqpoint{0.848611in}{2.115785in}}%
\pgfpathlineto{\pgfqpoint{0.933800in}{2.115785in}}%
\pgfpathlineto{\pgfqpoint{0.969450in}{1.991839in}}%
\pgfpathlineto{\pgfqpoint{1.040750in}{1.991839in}}%
\pgfpathlineto{\pgfqpoint{1.076400in}{1.910747in}}%
\pgfpathlineto{\pgfqpoint{1.361600in}{1.910747in}}%
\pgfpathlineto{\pgfqpoint{1.397250in}{1.714664in}}%
\pgfpathlineto{\pgfqpoint{1.468550in}{1.714664in}}%
\pgfpathlineto{\pgfqpoint{1.504200in}{1.649663in}}%
\pgfpathlineto{\pgfqpoint{1.682450in}{1.649663in}}%
\pgfpathlineto{\pgfqpoint{1.718100in}{1.630314in}}%
\pgfpathlineto{\pgfqpoint{1.753750in}{1.585114in}}%
\pgfpathlineto{\pgfqpoint{1.789400in}{1.585114in}}%
\pgfpathlineto{\pgfqpoint{1.825050in}{1.558561in}}%
\pgfpathlineto{\pgfqpoint{2.181550in}{1.558561in}}%
\pgfpathlineto{\pgfqpoint{2.217200in}{1.555049in}}%
\pgfpathlineto{\pgfqpoint{2.324150in}{1.555049in}}%
\pgfpathlineto{\pgfqpoint{2.359800in}{1.510394in}}%
\pgfpathlineto{\pgfqpoint{2.395450in}{1.510394in}}%
\pgfpathlineto{\pgfqpoint{2.431100in}{1.437807in}}%
\pgfpathlineto{\pgfqpoint{2.466750in}{1.357911in}}%
\pgfpathlineto{\pgfqpoint{2.538050in}{1.357911in}}%
\pgfpathlineto{\pgfqpoint{2.573700in}{1.318949in}}%
\pgfpathlineto{\pgfqpoint{2.645000in}{1.318949in}}%
\pgfpathlineto{\pgfqpoint{2.680650in}{1.105987in}}%
\pgfpathlineto{\pgfqpoint{2.823250in}{1.105987in}}%
\pgfpathlineto{\pgfqpoint{2.858900in}{1.087492in}}%
\pgfpathlineto{\pgfqpoint{3.037150in}{1.087492in}}%
\pgfpathlineto{\pgfqpoint{3.072800in}{1.007987in}}%
\pgfpathlineto{\pgfqpoint{3.179750in}{1.007987in}}%
\pgfpathlineto{\pgfqpoint{3.215400in}{0.917597in}}%
\pgfpathlineto{\pgfqpoint{3.393650in}{0.917597in}}%
\pgfpathlineto{\pgfqpoint{3.429300in}{0.816894in}}%
\pgfpathlineto{\pgfqpoint{3.678850in}{0.816894in}}%
\pgfpathlineto{\pgfqpoint{3.714500in}{0.805383in}}%
\pgfpathlineto{\pgfqpoint{3.785800in}{0.805383in}}%
\pgfpathlineto{\pgfqpoint{3.821450in}{0.719784in}}%
\pgfpathlineto{\pgfqpoint{5.497000in}{0.719784in}}%
\pgfpathlineto{\pgfqpoint{5.532650in}{0.689287in}}%
\pgfpathlineto{\pgfqpoint{6.067400in}{0.689287in}}%
\pgfpathlineto{\pgfqpoint{6.103050in}{0.555183in}}%
\pgfpathlineto{\pgfqpoint{6.223889in}{0.555183in}}%
\pgfpathlineto{\pgfqpoint{6.223889in}{0.555183in}}%
\pgfusepath{stroke}%
\end{pgfscope}%
\begin{pgfscope}%
\pgfpathrectangle{\pgfqpoint{0.862500in}{0.375000in}}{\pgfqpoint{5.347500in}{2.265000in}}%
\pgfusepath{clip}%
\pgfsetroundcap%
\pgfsetroundjoin%
\pgfsetlinewidth{1.505625pt}%
\definecolor{currentstroke}{rgb}{0.172549,0.627451,0.172549}%
\pgfsetstrokecolor{currentstroke}%
\pgfsetdash{}{0pt}%
\pgfpathmoveto{\pgfqpoint{0.848611in}{2.508131in}}%
\pgfpathlineto{\pgfqpoint{0.862500in}{2.508131in}}%
\pgfpathlineto{\pgfqpoint{0.898150in}{2.498155in}}%
\pgfpathlineto{\pgfqpoint{0.933800in}{2.498155in}}%
\pgfpathlineto{\pgfqpoint{0.969450in}{2.365090in}}%
\pgfpathlineto{\pgfqpoint{1.005100in}{2.162907in}}%
\pgfpathlineto{\pgfqpoint{1.040750in}{2.131136in}}%
\pgfpathlineto{\pgfqpoint{1.076400in}{2.131136in}}%
\pgfpathlineto{\pgfqpoint{1.112050in}{1.893961in}}%
\pgfpathlineto{\pgfqpoint{1.254650in}{1.893961in}}%
\pgfpathlineto{\pgfqpoint{1.290300in}{1.869124in}}%
\pgfpathlineto{\pgfqpoint{1.397250in}{1.869124in}}%
\pgfpathlineto{\pgfqpoint{1.432900in}{1.831409in}}%
\pgfpathlineto{\pgfqpoint{1.575500in}{1.831409in}}%
\pgfpathlineto{\pgfqpoint{1.611150in}{1.784783in}}%
\pgfpathlineto{\pgfqpoint{1.646800in}{1.655632in}}%
\pgfpathlineto{\pgfqpoint{1.682450in}{1.567067in}}%
\pgfpathlineto{\pgfqpoint{2.145900in}{1.566631in}}%
\pgfpathlineto{\pgfqpoint{2.181550in}{1.484517in}}%
\pgfpathlineto{\pgfqpoint{2.395450in}{1.484517in}}%
\pgfpathlineto{\pgfqpoint{2.431100in}{1.464905in}}%
\pgfpathlineto{\pgfqpoint{2.645000in}{1.464905in}}%
\pgfpathlineto{\pgfqpoint{2.680650in}{1.389145in}}%
\pgfpathlineto{\pgfqpoint{2.716300in}{1.381313in}}%
\pgfpathlineto{\pgfqpoint{2.751950in}{1.381313in}}%
\pgfpathlineto{\pgfqpoint{2.787600in}{1.334168in}}%
\pgfpathlineto{\pgfqpoint{3.429300in}{1.334168in}}%
\pgfpathlineto{\pgfqpoint{3.464950in}{1.324009in}}%
\pgfpathlineto{\pgfqpoint{3.500600in}{1.324009in}}%
\pgfpathlineto{\pgfqpoint{3.536250in}{1.105864in}}%
\pgfpathlineto{\pgfqpoint{4.320550in}{1.105864in}}%
\pgfpathlineto{\pgfqpoint{4.356200in}{1.086708in}}%
\pgfpathlineto{\pgfqpoint{4.641400in}{1.086708in}}%
\pgfpathlineto{\pgfqpoint{4.677050in}{1.076541in}}%
\pgfpathlineto{\pgfqpoint{4.784000in}{1.076541in}}%
\pgfpathlineto{\pgfqpoint{4.819650in}{1.023187in}}%
\pgfpathlineto{\pgfqpoint{4.890950in}{1.023187in}}%
\pgfpathlineto{\pgfqpoint{4.926600in}{0.919542in}}%
\pgfpathlineto{\pgfqpoint{5.283100in}{0.919542in}}%
\pgfpathlineto{\pgfqpoint{5.318750in}{0.886759in}}%
\pgfpathlineto{\pgfqpoint{5.568300in}{0.886759in}}%
\pgfpathlineto{\pgfqpoint{5.603950in}{0.878556in}}%
\pgfpathlineto{\pgfqpoint{6.223889in}{0.878556in}}%
\pgfpathlineto{\pgfqpoint{6.223889in}{0.878556in}}%
\pgfusepath{stroke}%
\end{pgfscope}%
\begin{pgfscope}%
\pgfpathrectangle{\pgfqpoint{0.862500in}{0.375000in}}{\pgfqpoint{5.347500in}{2.265000in}}%
\pgfusepath{clip}%
\pgfsetroundcap%
\pgfsetroundjoin%
\pgfsetlinewidth{1.505625pt}%
\definecolor{currentstroke}{rgb}{0.839216,0.152941,0.156863}%
\pgfsetstrokecolor{currentstroke}%
\pgfsetdash{}{0pt}%
\pgfpathmoveto{\pgfqpoint{0.848611in}{2.404017in}}%
\pgfpathlineto{\pgfqpoint{1.040750in}{2.404017in}}%
\pgfpathlineto{\pgfqpoint{1.076400in}{2.345734in}}%
\pgfpathlineto{\pgfqpoint{1.112050in}{1.906060in}}%
\pgfpathlineto{\pgfqpoint{1.361600in}{1.906060in}}%
\pgfpathlineto{\pgfqpoint{1.397250in}{1.848921in}}%
\pgfpathlineto{\pgfqpoint{1.432900in}{1.783175in}}%
\pgfpathlineto{\pgfqpoint{1.468550in}{1.783175in}}%
\pgfpathlineto{\pgfqpoint{1.504200in}{1.766309in}}%
\pgfpathlineto{\pgfqpoint{1.575500in}{1.766309in}}%
\pgfpathlineto{\pgfqpoint{1.611150in}{1.750764in}}%
\pgfpathlineto{\pgfqpoint{1.646800in}{1.750764in}}%
\pgfpathlineto{\pgfqpoint{1.682450in}{1.734828in}}%
\pgfpathlineto{\pgfqpoint{1.932000in}{1.734828in}}%
\pgfpathlineto{\pgfqpoint{1.967650in}{1.722742in}}%
\pgfpathlineto{\pgfqpoint{2.038950in}{1.722742in}}%
\pgfpathlineto{\pgfqpoint{2.074600in}{1.717776in}}%
\pgfpathlineto{\pgfqpoint{2.181550in}{1.717776in}}%
\pgfpathlineto{\pgfqpoint{2.217200in}{1.716031in}}%
\pgfpathlineto{\pgfqpoint{2.324150in}{1.716031in}}%
\pgfpathlineto{\pgfqpoint{2.359800in}{1.597784in}}%
\pgfpathlineto{\pgfqpoint{2.466750in}{1.597784in}}%
\pgfpathlineto{\pgfqpoint{2.502400in}{1.592207in}}%
\pgfpathlineto{\pgfqpoint{2.858900in}{1.592207in}}%
\pgfpathlineto{\pgfqpoint{2.894550in}{1.584160in}}%
\pgfpathlineto{\pgfqpoint{2.930200in}{1.584160in}}%
\pgfpathlineto{\pgfqpoint{2.965850in}{1.569895in}}%
\pgfpathlineto{\pgfqpoint{3.001500in}{1.619350in}}%
\pgfpathlineto{\pgfqpoint{3.322350in}{1.619350in}}%
\pgfpathlineto{\pgfqpoint{3.358000in}{1.584966in}}%
\pgfpathlineto{\pgfqpoint{3.750150in}{1.584966in}}%
\pgfpathlineto{\pgfqpoint{3.785800in}{1.508728in}}%
\pgfpathlineto{\pgfqpoint{3.821450in}{1.508728in}}%
\pgfpathlineto{\pgfqpoint{3.857100in}{1.476175in}}%
\pgfpathlineto{\pgfqpoint{5.140500in}{1.476175in}}%
\pgfpathlineto{\pgfqpoint{5.176150in}{1.367432in}}%
\pgfpathlineto{\pgfqpoint{5.603950in}{1.367432in}}%
\pgfpathlineto{\pgfqpoint{5.639600in}{1.319886in}}%
\pgfpathlineto{\pgfqpoint{6.223889in}{1.319886in}}%
\pgfpathlineto{\pgfqpoint{6.223889in}{1.319886in}}%
\pgfusepath{stroke}%
\end{pgfscope}%
\begin{pgfscope}%
\pgfsetrectcap%
\pgfsetmiterjoin%
\pgfsetlinewidth{0.000000pt}%
\definecolor{currentstroke}{rgb}{1.000000,1.000000,1.000000}%
\pgfsetstrokecolor{currentstroke}%
\pgfsetdash{}{0pt}%
\pgfpathmoveto{\pgfqpoint{0.862500in}{0.375000in}}%
\pgfpathlineto{\pgfqpoint{0.862500in}{2.640000in}}%
\pgfusepath{}%
\end{pgfscope}%
\begin{pgfscope}%
\pgfsetrectcap%
\pgfsetmiterjoin%
\pgfsetlinewidth{0.000000pt}%
\definecolor{currentstroke}{rgb}{1.000000,1.000000,1.000000}%
\pgfsetstrokecolor{currentstroke}%
\pgfsetdash{}{0pt}%
\pgfpathmoveto{\pgfqpoint{6.210000in}{0.375000in}}%
\pgfpathlineto{\pgfqpoint{6.210000in}{2.640000in}}%
\pgfusepath{}%
\end{pgfscope}%
\begin{pgfscope}%
\pgfsetrectcap%
\pgfsetmiterjoin%
\pgfsetlinewidth{0.000000pt}%
\definecolor{currentstroke}{rgb}{1.000000,1.000000,1.000000}%
\pgfsetstrokecolor{currentstroke}%
\pgfsetdash{}{0pt}%
\pgfpathmoveto{\pgfqpoint{0.862500in}{0.375000in}}%
\pgfpathlineto{\pgfqpoint{6.210000in}{0.375000in}}%
\pgfusepath{}%
\end{pgfscope}%
\begin{pgfscope}%
\pgfsetrectcap%
\pgfsetmiterjoin%
\pgfsetlinewidth{0.000000pt}%
\definecolor{currentstroke}{rgb}{1.000000,1.000000,1.000000}%
\pgfsetstrokecolor{currentstroke}%
\pgfsetdash{}{0pt}%
\pgfpathmoveto{\pgfqpoint{0.862500in}{2.640000in}}%
\pgfpathlineto{\pgfqpoint{6.210000in}{2.640000in}}%
\pgfusepath{}%
\end{pgfscope}%
\begin{pgfscope}%
\definecolor{textcolor}{rgb}{0.150000,0.150000,0.150000}%
\pgfsetstrokecolor{textcolor}%
\pgfsetfillcolor{textcolor}%
\pgftext[x=3.536250in,y=2.723333in,,base]{\color{textcolor}\rmfamily\fontsize{8.000000}{9.600000}\selectfont Rosenbrock8D}%
\end{pgfscope}%
\begin{pgfscope}%
\pgfsetroundcap%
\pgfsetroundjoin%
\pgfsetlinewidth{1.505625pt}%
\definecolor{currentstroke}{rgb}{0.121569,0.466667,0.705882}%
\pgfsetstrokecolor{currentstroke}%
\pgfsetdash{}{0pt}%
\pgfpathmoveto{\pgfqpoint{0.962500in}{1.026258in}}%
\pgfpathlineto{\pgfqpoint{1.184722in}{1.026258in}}%
\pgfusepath{stroke}%
\end{pgfscope}%
\begin{pgfscope}%
\definecolor{textcolor}{rgb}{0.150000,0.150000,0.150000}%
\pgfsetstrokecolor{textcolor}%
\pgfsetfillcolor{textcolor}%
\pgftext[x=1.273611in,y=0.987369in,left,base]{\color{textcolor}\rmfamily\fontsize{8.000000}{9.600000}\selectfont random}%
\end{pgfscope}%
\begin{pgfscope}%
\pgfsetroundcap%
\pgfsetroundjoin%
\pgfsetlinewidth{1.505625pt}%
\definecolor{currentstroke}{rgb}{1.000000,0.498039,0.054902}%
\pgfsetstrokecolor{currentstroke}%
\pgfsetdash{}{0pt}%
\pgfpathmoveto{\pgfqpoint{0.962500in}{0.863172in}}%
\pgfpathlineto{\pgfqpoint{1.184722in}{0.863172in}}%
\pgfusepath{stroke}%
\end{pgfscope}%
\begin{pgfscope}%
\definecolor{textcolor}{rgb}{0.150000,0.150000,0.150000}%
\pgfsetstrokecolor{textcolor}%
\pgfsetfillcolor{textcolor}%
\pgftext[x=1.273611in,y=0.824283in,left,base]{\color{textcolor}\rmfamily\fontsize{8.000000}{9.600000}\selectfont 5 x DNGO retrain-reset}%
\end{pgfscope}%
\begin{pgfscope}%
\pgfsetroundcap%
\pgfsetroundjoin%
\pgfsetlinewidth{1.505625pt}%
\definecolor{currentstroke}{rgb}{0.172549,0.627451,0.172549}%
\pgfsetstrokecolor{currentstroke}%
\pgfsetdash{}{0pt}%
\pgfpathmoveto{\pgfqpoint{0.962500in}{0.700087in}}%
\pgfpathlineto{\pgfqpoint{1.184722in}{0.700087in}}%
\pgfusepath{stroke}%
\end{pgfscope}%
\begin{pgfscope}%
\definecolor{textcolor}{rgb}{0.150000,0.150000,0.150000}%
\pgfsetstrokecolor{textcolor}%
\pgfsetfillcolor{textcolor}%
\pgftext[x=1.273611in,y=0.661198in,left,base]{\color{textcolor}\rmfamily\fontsize{8.000000}{9.600000}\selectfont DNGO retrain-reset}%
\end{pgfscope}%
\begin{pgfscope}%
\pgfsetroundcap%
\pgfsetroundjoin%
\pgfsetlinewidth{1.505625pt}%
\definecolor{currentstroke}{rgb}{0.839216,0.152941,0.156863}%
\pgfsetstrokecolor{currentstroke}%
\pgfsetdash{}{0pt}%
\pgfpathmoveto{\pgfqpoint{0.962500in}{0.537001in}}%
\pgfpathlineto{\pgfqpoint{1.184722in}{0.537001in}}%
\pgfusepath{stroke}%
\end{pgfscope}%
\begin{pgfscope}%
\definecolor{textcolor}{rgb}{0.150000,0.150000,0.150000}%
\pgfsetstrokecolor{textcolor}%
\pgfsetfillcolor{textcolor}%
\pgftext[x=1.273611in,y=0.498112in,left,base]{\color{textcolor}\rmfamily\fontsize{8.000000}{9.600000}\selectfont GP}%
\end{pgfscope}%
\end{pgfpicture}%
\makeatother%
\endgroup%

            \captionof{figure}{Shows the nonsmooth Alpine01 from \parencite{dewancker_stratified_2016} and Hartmann6 and Rosenbrock8D from \parencite{eggensperger_towards_2013}.}
        \end{minipage}

    \subsection{Machine Learning Hyperparameter Optimization Tasks}\label{sec:appml}
    %These results have no confidence interval as each configuration was only run once.

    \begin{minipage}{\linewidth}
        \centering
        \begin{tabular}{|l||c|c|c|}  
            \hline
            \textbf{Method} & \textbf{Logistic regression} & \textbf{Fully Connected Network} & \textbf{CNN} \\
            \hline
            5 $\times$ DNGO fixed    & 0.069  &        --        & -- \\
            DNGO fixed               & 0.072  &        --        & -- \\
            DNGO retrain-reset       & 0.069  & 0.0125           & 0.897  \\
            DNGO retrain-reset MCMC  & 0.069  &        --        & -- \\
            GP                       & 0.068  & 0.0119           & 0.897  \\
            GP MCMC                  & 0.067  &        --        & -- \\
            Random                   & 0.098  &        --        & 0.897  \\
            \hline
        \end{tabular}
        % 5 x DNGO fixed           & -0.0689243168112 4.68117472733e-07 &        --        & -- \\
        % DNGO fixed               & -0.0718365924113 6.12426049571e-06 &        --        & -- \\
        % DNGO retrain-reset       & -0.0690516213337 7.34755313694e-08 & -0.0124999922514 & -0.89710643372 1.03975230477e-09 \\
        % DNGO retrain-reset MCMC  & -0.0694848831195 8.69431609179e-07 &        --        & -- \\
        % GP                       & -0.0678089527119 4.34300910663e-07 & -0.011869047662  & -0.896966526844 0.0 \\
        % GP MCMC                  & -0.0674540907964 2.37421881001e-08 &        --        & -- \\
        % Rand                     & -0.0976534200291 5.94025833014e-05 &        --        & -0.897074078582 1.15673764801e-08 \\
        \captionof{table}{
        Logistic Regression indicates average over 5 runs while the rest was run once. 
        Some cells are empty because it was decided not to waste computational resources on a fruitless endevour.}
    \end{minipage}
    \begin{minipage}{\linewidth}
        \centering
        %% Creator: Matplotlib, PGF backend
%%
%% To include the figure in your LaTeX document, write
%%   \input{<filename>.pgf}
%%
%% Make sure the required packages are loaded in your preamble
%%   \usepackage{pgf}
%%
%% Figures using additional raster images can only be included by \input if
%% they are in the same directory as the main LaTeX file. For loading figures
%% from other directories you can use the `import` package
%%   \usepackage{import}
%% and then include the figures with
%%   \import{<path to file>}{<filename>.pgf}
%%
%% Matplotlib used the following preamble
%%   \usepackage{gensymb}
%%   \usepackage{fontspec}
%%   \setmainfont{DejaVu Serif}
%%   \setsansfont{Arial}
%%   \setmonofont{DejaVu Sans Mono}
%%
\begingroup%
\makeatletter%
\begin{pgfpicture}%
\pgfpathrectangle{\pgfpointorigin}{\pgfqpoint{6.900000in}{3.000000in}}%
\pgfusepath{use as bounding box, clip}%
\begin{pgfscope}%
\pgfsetbuttcap%
\pgfsetmiterjoin%
\definecolor{currentfill}{rgb}{1.000000,1.000000,1.000000}%
\pgfsetfillcolor{currentfill}%
\pgfsetlinewidth{0.000000pt}%
\definecolor{currentstroke}{rgb}{1.000000,1.000000,1.000000}%
\pgfsetstrokecolor{currentstroke}%
\pgfsetdash{}{0pt}%
\pgfpathmoveto{\pgfqpoint{0.000000in}{0.000000in}}%
\pgfpathlineto{\pgfqpoint{6.900000in}{0.000000in}}%
\pgfpathlineto{\pgfqpoint{6.900000in}{3.000000in}}%
\pgfpathlineto{\pgfqpoint{0.000000in}{3.000000in}}%
\pgfpathclose%
\pgfusepath{fill}%
\end{pgfscope}%
\begin{pgfscope}%
\pgfsetbuttcap%
\pgfsetmiterjoin%
\definecolor{currentfill}{rgb}{0.917647,0.917647,0.949020}%
\pgfsetfillcolor{currentfill}%
\pgfsetlinewidth{0.000000pt}%
\definecolor{currentstroke}{rgb}{0.000000,0.000000,0.000000}%
\pgfsetstrokecolor{currentstroke}%
\pgfsetstrokeopacity{0.000000}%
\pgfsetdash{}{0pt}%
\pgfpathmoveto{\pgfqpoint{0.862500in}{0.375000in}}%
\pgfpathlineto{\pgfqpoint{6.210000in}{0.375000in}}%
\pgfpathlineto{\pgfqpoint{6.210000in}{2.640000in}}%
\pgfpathlineto{\pgfqpoint{0.862500in}{2.640000in}}%
\pgfpathclose%
\pgfusepath{fill}%
\end{pgfscope}%
\begin{pgfscope}%
\pgfpathrectangle{\pgfqpoint{0.862500in}{0.375000in}}{\pgfqpoint{5.347500in}{2.265000in}}%
\pgfusepath{clip}%
\pgfsetroundcap%
\pgfsetroundjoin%
\pgfsetlinewidth{0.803000pt}%
\definecolor{currentstroke}{rgb}{1.000000,1.000000,1.000000}%
\pgfsetstrokecolor{currentstroke}%
\pgfsetdash{}{0pt}%
\pgfpathmoveto{\pgfqpoint{0.862500in}{0.375000in}}%
\pgfpathlineto{\pgfqpoint{0.862500in}{2.640000in}}%
\pgfusepath{stroke}%
\end{pgfscope}%
\begin{pgfscope}%
\definecolor{textcolor}{rgb}{0.150000,0.150000,0.150000}%
\pgfsetstrokecolor{textcolor}%
\pgfsetfillcolor{textcolor}%
\pgftext[x=0.862500in,y=0.326389in,,top]{\color{textcolor}\rmfamily\fontsize{8.000000}{9.600000}\selectfont \(\displaystyle 0\)}%
\end{pgfscope}%
\begin{pgfscope}%
\pgfpathrectangle{\pgfqpoint{0.862500in}{0.375000in}}{\pgfqpoint{5.347500in}{2.265000in}}%
\pgfusepath{clip}%
\pgfsetroundcap%
\pgfsetroundjoin%
\pgfsetlinewidth{0.803000pt}%
\definecolor{currentstroke}{rgb}{1.000000,1.000000,1.000000}%
\pgfsetstrokecolor{currentstroke}%
\pgfsetdash{}{0pt}%
\pgfpathmoveto{\pgfqpoint{1.932000in}{0.375000in}}%
\pgfpathlineto{\pgfqpoint{1.932000in}{2.640000in}}%
\pgfusepath{stroke}%
\end{pgfscope}%
\begin{pgfscope}%
\definecolor{textcolor}{rgb}{0.150000,0.150000,0.150000}%
\pgfsetstrokecolor{textcolor}%
\pgfsetfillcolor{textcolor}%
\pgftext[x=1.932000in,y=0.326389in,,top]{\color{textcolor}\rmfamily\fontsize{8.000000}{9.600000}\selectfont \(\displaystyle 20\)}%
\end{pgfscope}%
\begin{pgfscope}%
\pgfpathrectangle{\pgfqpoint{0.862500in}{0.375000in}}{\pgfqpoint{5.347500in}{2.265000in}}%
\pgfusepath{clip}%
\pgfsetroundcap%
\pgfsetroundjoin%
\pgfsetlinewidth{0.803000pt}%
\definecolor{currentstroke}{rgb}{1.000000,1.000000,1.000000}%
\pgfsetstrokecolor{currentstroke}%
\pgfsetdash{}{0pt}%
\pgfpathmoveto{\pgfqpoint{3.001500in}{0.375000in}}%
\pgfpathlineto{\pgfqpoint{3.001500in}{2.640000in}}%
\pgfusepath{stroke}%
\end{pgfscope}%
\begin{pgfscope}%
\definecolor{textcolor}{rgb}{0.150000,0.150000,0.150000}%
\pgfsetstrokecolor{textcolor}%
\pgfsetfillcolor{textcolor}%
\pgftext[x=3.001500in,y=0.326389in,,top]{\color{textcolor}\rmfamily\fontsize{8.000000}{9.600000}\selectfont \(\displaystyle 40\)}%
\end{pgfscope}%
\begin{pgfscope}%
\pgfpathrectangle{\pgfqpoint{0.862500in}{0.375000in}}{\pgfqpoint{5.347500in}{2.265000in}}%
\pgfusepath{clip}%
\pgfsetroundcap%
\pgfsetroundjoin%
\pgfsetlinewidth{0.803000pt}%
\definecolor{currentstroke}{rgb}{1.000000,1.000000,1.000000}%
\pgfsetstrokecolor{currentstroke}%
\pgfsetdash{}{0pt}%
\pgfpathmoveto{\pgfqpoint{4.071000in}{0.375000in}}%
\pgfpathlineto{\pgfqpoint{4.071000in}{2.640000in}}%
\pgfusepath{stroke}%
\end{pgfscope}%
\begin{pgfscope}%
\definecolor{textcolor}{rgb}{0.150000,0.150000,0.150000}%
\pgfsetstrokecolor{textcolor}%
\pgfsetfillcolor{textcolor}%
\pgftext[x=4.071000in,y=0.326389in,,top]{\color{textcolor}\rmfamily\fontsize{8.000000}{9.600000}\selectfont \(\displaystyle 60\)}%
\end{pgfscope}%
\begin{pgfscope}%
\pgfpathrectangle{\pgfqpoint{0.862500in}{0.375000in}}{\pgfqpoint{5.347500in}{2.265000in}}%
\pgfusepath{clip}%
\pgfsetroundcap%
\pgfsetroundjoin%
\pgfsetlinewidth{0.803000pt}%
\definecolor{currentstroke}{rgb}{1.000000,1.000000,1.000000}%
\pgfsetstrokecolor{currentstroke}%
\pgfsetdash{}{0pt}%
\pgfpathmoveto{\pgfqpoint{5.140500in}{0.375000in}}%
\pgfpathlineto{\pgfqpoint{5.140500in}{2.640000in}}%
\pgfusepath{stroke}%
\end{pgfscope}%
\begin{pgfscope}%
\definecolor{textcolor}{rgb}{0.150000,0.150000,0.150000}%
\pgfsetstrokecolor{textcolor}%
\pgfsetfillcolor{textcolor}%
\pgftext[x=5.140500in,y=0.326389in,,top]{\color{textcolor}\rmfamily\fontsize{8.000000}{9.600000}\selectfont \(\displaystyle 80\)}%
\end{pgfscope}%
\begin{pgfscope}%
\pgfpathrectangle{\pgfqpoint{0.862500in}{0.375000in}}{\pgfqpoint{5.347500in}{2.265000in}}%
\pgfusepath{clip}%
\pgfsetroundcap%
\pgfsetroundjoin%
\pgfsetlinewidth{0.803000pt}%
\definecolor{currentstroke}{rgb}{1.000000,1.000000,1.000000}%
\pgfsetstrokecolor{currentstroke}%
\pgfsetdash{}{0pt}%
\pgfpathmoveto{\pgfqpoint{6.210000in}{0.375000in}}%
\pgfpathlineto{\pgfqpoint{6.210000in}{2.640000in}}%
\pgfusepath{stroke}%
\end{pgfscope}%
\begin{pgfscope}%
\definecolor{textcolor}{rgb}{0.150000,0.150000,0.150000}%
\pgfsetstrokecolor{textcolor}%
\pgfsetfillcolor{textcolor}%
\pgftext[x=6.210000in,y=0.326389in,,top]{\color{textcolor}\rmfamily\fontsize{8.000000}{9.600000}\selectfont \(\displaystyle 100\)}%
\end{pgfscope}%
\begin{pgfscope}%
\definecolor{textcolor}{rgb}{0.150000,0.150000,0.150000}%
\pgfsetstrokecolor{textcolor}%
\pgfsetfillcolor{textcolor}%
\pgftext[x=3.536250in,y=0.163303in,,top]{\color{textcolor}\rmfamily\fontsize{8.000000}{9.600000}\selectfont Step}%
\end{pgfscope}%
\begin{pgfscope}%
\pgfpathrectangle{\pgfqpoint{0.862500in}{0.375000in}}{\pgfqpoint{5.347500in}{2.265000in}}%
\pgfusepath{clip}%
\pgfsetroundcap%
\pgfsetroundjoin%
\pgfsetlinewidth{0.803000pt}%
\definecolor{currentstroke}{rgb}{1.000000,1.000000,1.000000}%
\pgfsetstrokecolor{currentstroke}%
\pgfsetdash{}{0pt}%
\pgfpathmoveto{\pgfqpoint{0.862500in}{0.809977in}}%
\pgfpathlineto{\pgfqpoint{6.210000in}{0.809977in}}%
\pgfusepath{stroke}%
\end{pgfscope}%
\begin{pgfscope}%
\definecolor{textcolor}{rgb}{0.150000,0.150000,0.150000}%
\pgfsetstrokecolor{textcolor}%
\pgfsetfillcolor{textcolor}%
\pgftext[x=0.557716in,y=0.767767in,left,base]{\color{textcolor}\rmfamily\fontsize{8.000000}{9.600000}\selectfont \(\displaystyle 10^{-1}\)}%
\end{pgfscope}%
\begin{pgfscope}%
\definecolor{textcolor}{rgb}{0.150000,0.150000,0.150000}%
\pgfsetstrokecolor{textcolor}%
\pgfsetfillcolor{textcolor}%
\pgftext[x=0.502160in,y=1.507500in,,bottom,rotate=90.000000]{\color{textcolor}\rmfamily\fontsize{8.000000}{9.600000}\selectfont Simple Regret}%
\end{pgfscope}%
\begin{pgfscope}%
\pgfpathrectangle{\pgfqpoint{0.862500in}{0.375000in}}{\pgfqpoint{5.347500in}{2.265000in}}%
\pgfusepath{clip}%
\pgfsetbuttcap%
\pgfsetroundjoin%
\definecolor{currentfill}{rgb}{0.121569,0.466667,0.705882}%
\pgfsetfillcolor{currentfill}%
\pgfsetfillopacity{0.200000}%
\pgfsetlinewidth{0.000000pt}%
\definecolor{currentstroke}{rgb}{0.000000,0.000000,0.000000}%
\pgfsetstrokecolor{currentstroke}%
\pgfsetdash{}{0pt}%
\pgfpathmoveto{\pgfqpoint{0.862500in}{2.138516in}}%
\pgfpathlineto{\pgfqpoint{0.862500in}{2.537045in}}%
\pgfpathlineto{\pgfqpoint{0.915975in}{2.334845in}}%
\pgfpathlineto{\pgfqpoint{0.969450in}{1.444798in}}%
\pgfpathlineto{\pgfqpoint{1.022925in}{1.431042in}}%
\pgfpathlineto{\pgfqpoint{1.076400in}{1.431042in}}%
\pgfpathlineto{\pgfqpoint{1.129875in}{1.431042in}}%
\pgfpathlineto{\pgfqpoint{1.183350in}{1.210921in}}%
\pgfpathlineto{\pgfqpoint{1.236825in}{1.115136in}}%
\pgfpathlineto{\pgfqpoint{1.290300in}{1.115136in}}%
\pgfpathlineto{\pgfqpoint{1.343775in}{1.115136in}}%
\pgfpathlineto{\pgfqpoint{1.397250in}{1.078183in}}%
\pgfpathlineto{\pgfqpoint{1.450725in}{1.078183in}}%
\pgfpathlineto{\pgfqpoint{1.504200in}{1.078183in}}%
\pgfpathlineto{\pgfqpoint{1.557675in}{1.078183in}}%
\pgfpathlineto{\pgfqpoint{1.611150in}{1.078183in}}%
\pgfpathlineto{\pgfqpoint{1.664625in}{1.078183in}}%
\pgfpathlineto{\pgfqpoint{1.718100in}{1.078183in}}%
\pgfpathlineto{\pgfqpoint{1.771575in}{1.078183in}}%
\pgfpathlineto{\pgfqpoint{1.825050in}{1.078183in}}%
\pgfpathlineto{\pgfqpoint{1.878525in}{1.078183in}}%
\pgfpathlineto{\pgfqpoint{1.932000in}{0.605108in}}%
\pgfpathlineto{\pgfqpoint{1.985475in}{0.605108in}}%
\pgfpathlineto{\pgfqpoint{2.038950in}{0.605108in}}%
\pgfpathlineto{\pgfqpoint{2.092425in}{0.599757in}}%
\pgfpathlineto{\pgfqpoint{2.145900in}{0.599757in}}%
\pgfpathlineto{\pgfqpoint{2.199375in}{0.599757in}}%
\pgfpathlineto{\pgfqpoint{2.252850in}{0.599757in}}%
\pgfpathlineto{\pgfqpoint{2.306325in}{0.599757in}}%
\pgfpathlineto{\pgfqpoint{2.359800in}{0.554173in}}%
\pgfpathlineto{\pgfqpoint{2.413275in}{0.554173in}}%
\pgfpathlineto{\pgfqpoint{2.466750in}{0.554173in}}%
\pgfpathlineto{\pgfqpoint{2.520225in}{0.554173in}}%
\pgfpathlineto{\pgfqpoint{2.573700in}{0.554173in}}%
\pgfpathlineto{\pgfqpoint{2.627175in}{0.554173in}}%
\pgfpathlineto{\pgfqpoint{2.680650in}{0.554173in}}%
\pgfpathlineto{\pgfqpoint{2.734125in}{0.554173in}}%
\pgfpathlineto{\pgfqpoint{2.787600in}{0.552917in}}%
\pgfpathlineto{\pgfqpoint{2.841075in}{0.552917in}}%
\pgfpathlineto{\pgfqpoint{2.894550in}{0.552917in}}%
\pgfpathlineto{\pgfqpoint{2.948025in}{0.552917in}}%
\pgfpathlineto{\pgfqpoint{3.001500in}{0.552917in}}%
\pgfpathlineto{\pgfqpoint{3.054975in}{0.552917in}}%
\pgfpathlineto{\pgfqpoint{3.108450in}{0.552917in}}%
\pgfpathlineto{\pgfqpoint{3.161925in}{0.552917in}}%
\pgfpathlineto{\pgfqpoint{3.215400in}{0.552917in}}%
\pgfpathlineto{\pgfqpoint{3.268875in}{0.550997in}}%
\pgfpathlineto{\pgfqpoint{3.322350in}{0.550997in}}%
\pgfpathlineto{\pgfqpoint{3.375825in}{0.550997in}}%
\pgfpathlineto{\pgfqpoint{3.429300in}{0.537994in}}%
\pgfpathlineto{\pgfqpoint{3.482775in}{0.535904in}}%
\pgfpathlineto{\pgfqpoint{3.536250in}{0.511782in}}%
\pgfpathlineto{\pgfqpoint{3.589725in}{0.511782in}}%
\pgfpathlineto{\pgfqpoint{3.643200in}{0.511782in}}%
\pgfpathlineto{\pgfqpoint{3.696675in}{0.511782in}}%
\pgfpathlineto{\pgfqpoint{3.750150in}{0.511782in}}%
\pgfpathlineto{\pgfqpoint{3.803625in}{0.509387in}}%
\pgfpathlineto{\pgfqpoint{3.857100in}{0.505368in}}%
\pgfpathlineto{\pgfqpoint{3.910575in}{0.505368in}}%
\pgfpathlineto{\pgfqpoint{3.964050in}{0.505368in}}%
\pgfpathlineto{\pgfqpoint{4.017525in}{0.505368in}}%
\pgfpathlineto{\pgfqpoint{4.071000in}{0.505368in}}%
\pgfpathlineto{\pgfqpoint{4.124475in}{0.505368in}}%
\pgfpathlineto{\pgfqpoint{4.177950in}{0.505368in}}%
\pgfpathlineto{\pgfqpoint{4.231425in}{0.505368in}}%
\pgfpathlineto{\pgfqpoint{4.284900in}{0.505368in}}%
\pgfpathlineto{\pgfqpoint{4.338375in}{0.505368in}}%
\pgfpathlineto{\pgfqpoint{4.391850in}{0.505368in}}%
\pgfpathlineto{\pgfqpoint{4.445325in}{0.505368in}}%
\pgfpathlineto{\pgfqpoint{4.498800in}{0.505368in}}%
\pgfpathlineto{\pgfqpoint{4.552275in}{0.505368in}}%
\pgfpathlineto{\pgfqpoint{4.605750in}{0.505368in}}%
\pgfpathlineto{\pgfqpoint{4.659225in}{0.504839in}}%
\pgfpathlineto{\pgfqpoint{4.712700in}{0.504839in}}%
\pgfpathlineto{\pgfqpoint{4.766175in}{0.504839in}}%
\pgfpathlineto{\pgfqpoint{4.819650in}{0.504839in}}%
\pgfpathlineto{\pgfqpoint{4.873125in}{0.504839in}}%
\pgfpathlineto{\pgfqpoint{4.926600in}{0.504839in}}%
\pgfpathlineto{\pgfqpoint{4.980075in}{0.504839in}}%
\pgfpathlineto{\pgfqpoint{5.033550in}{0.504839in}}%
\pgfpathlineto{\pgfqpoint{5.087025in}{0.504839in}}%
\pgfpathlineto{\pgfqpoint{5.140500in}{0.504839in}}%
\pgfpathlineto{\pgfqpoint{5.193975in}{0.504839in}}%
\pgfpathlineto{\pgfqpoint{5.247450in}{0.504839in}}%
\pgfpathlineto{\pgfqpoint{5.300925in}{0.504839in}}%
\pgfpathlineto{\pgfqpoint{5.354400in}{0.504839in}}%
\pgfpathlineto{\pgfqpoint{5.407875in}{0.504839in}}%
\pgfpathlineto{\pgfqpoint{5.461350in}{0.504839in}}%
\pgfpathlineto{\pgfqpoint{5.514825in}{0.504839in}}%
\pgfpathlineto{\pgfqpoint{5.568300in}{0.504839in}}%
\pgfpathlineto{\pgfqpoint{5.621775in}{0.504839in}}%
\pgfpathlineto{\pgfqpoint{5.675250in}{0.504839in}}%
\pgfpathlineto{\pgfqpoint{5.728725in}{0.504839in}}%
\pgfpathlineto{\pgfqpoint{5.782200in}{0.504839in}}%
\pgfpathlineto{\pgfqpoint{5.835675in}{0.504839in}}%
\pgfpathlineto{\pgfqpoint{5.889150in}{0.504839in}}%
\pgfpathlineto{\pgfqpoint{5.942625in}{0.504839in}}%
\pgfpathlineto{\pgfqpoint{5.996100in}{0.504839in}}%
\pgfpathlineto{\pgfqpoint{6.049575in}{0.504839in}}%
\pgfpathlineto{\pgfqpoint{6.103050in}{0.504839in}}%
\pgfpathlineto{\pgfqpoint{6.156525in}{0.504839in}}%
\pgfpathlineto{\pgfqpoint{6.210000in}{0.504839in}}%
\pgfpathlineto{\pgfqpoint{6.263475in}{0.504839in}}%
\pgfpathlineto{\pgfqpoint{6.316950in}{0.501200in}}%
\pgfpathlineto{\pgfqpoint{6.370425in}{0.501200in}}%
\pgfpathlineto{\pgfqpoint{6.423900in}{0.501200in}}%
\pgfpathlineto{\pgfqpoint{6.477375in}{0.501119in}}%
\pgfpathlineto{\pgfqpoint{6.530850in}{0.501119in}}%
\pgfpathlineto{\pgfqpoint{6.584325in}{0.501119in}}%
\pgfpathlineto{\pgfqpoint{6.637800in}{0.501119in}}%
\pgfpathlineto{\pgfqpoint{6.691275in}{0.501119in}}%
\pgfpathlineto{\pgfqpoint{6.744750in}{0.501119in}}%
\pgfpathlineto{\pgfqpoint{6.798225in}{0.501119in}}%
\pgfpathlineto{\pgfqpoint{6.851700in}{0.501119in}}%
\pgfpathlineto{\pgfqpoint{6.905175in}{0.501119in}}%
\pgfpathlineto{\pgfqpoint{6.958650in}{0.501119in}}%
\pgfpathlineto{\pgfqpoint{7.012125in}{0.501119in}}%
\pgfpathlineto{\pgfqpoint{7.065600in}{0.501119in}}%
\pgfpathlineto{\pgfqpoint{7.119075in}{0.501119in}}%
\pgfpathlineto{\pgfqpoint{7.172550in}{0.501119in}}%
\pgfpathlineto{\pgfqpoint{7.226025in}{0.501119in}}%
\pgfpathlineto{\pgfqpoint{7.226025in}{0.493189in}}%
\pgfpathlineto{\pgfqpoint{7.226025in}{0.493189in}}%
\pgfpathlineto{\pgfqpoint{7.172550in}{0.493189in}}%
\pgfpathlineto{\pgfqpoint{7.119075in}{0.493189in}}%
\pgfpathlineto{\pgfqpoint{7.065600in}{0.493189in}}%
\pgfpathlineto{\pgfqpoint{7.012125in}{0.493189in}}%
\pgfpathlineto{\pgfqpoint{6.958650in}{0.493189in}}%
\pgfpathlineto{\pgfqpoint{6.905175in}{0.493189in}}%
\pgfpathlineto{\pgfqpoint{6.851700in}{0.493189in}}%
\pgfpathlineto{\pgfqpoint{6.798225in}{0.493189in}}%
\pgfpathlineto{\pgfqpoint{6.744750in}{0.493189in}}%
\pgfpathlineto{\pgfqpoint{6.691275in}{0.493189in}}%
\pgfpathlineto{\pgfqpoint{6.637800in}{0.493189in}}%
\pgfpathlineto{\pgfqpoint{6.584325in}{0.493189in}}%
\pgfpathlineto{\pgfqpoint{6.530850in}{0.493189in}}%
\pgfpathlineto{\pgfqpoint{6.477375in}{0.493189in}}%
\pgfpathlineto{\pgfqpoint{6.423900in}{0.493217in}}%
\pgfpathlineto{\pgfqpoint{6.370425in}{0.493217in}}%
\pgfpathlineto{\pgfqpoint{6.316950in}{0.493217in}}%
\pgfpathlineto{\pgfqpoint{6.263475in}{0.494331in}}%
\pgfpathlineto{\pgfqpoint{6.210000in}{0.494331in}}%
\pgfpathlineto{\pgfqpoint{6.156525in}{0.494331in}}%
\pgfpathlineto{\pgfqpoint{6.103050in}{0.494331in}}%
\pgfpathlineto{\pgfqpoint{6.049575in}{0.494331in}}%
\pgfpathlineto{\pgfqpoint{5.996100in}{0.494331in}}%
\pgfpathlineto{\pgfqpoint{5.942625in}{0.494331in}}%
\pgfpathlineto{\pgfqpoint{5.889150in}{0.494331in}}%
\pgfpathlineto{\pgfqpoint{5.835675in}{0.494331in}}%
\pgfpathlineto{\pgfqpoint{5.782200in}{0.494331in}}%
\pgfpathlineto{\pgfqpoint{5.728725in}{0.494331in}}%
\pgfpathlineto{\pgfqpoint{5.675250in}{0.494331in}}%
\pgfpathlineto{\pgfqpoint{5.621775in}{0.494331in}}%
\pgfpathlineto{\pgfqpoint{5.568300in}{0.494331in}}%
\pgfpathlineto{\pgfqpoint{5.514825in}{0.494331in}}%
\pgfpathlineto{\pgfqpoint{5.461350in}{0.494331in}}%
\pgfpathlineto{\pgfqpoint{5.407875in}{0.494331in}}%
\pgfpathlineto{\pgfqpoint{5.354400in}{0.494331in}}%
\pgfpathlineto{\pgfqpoint{5.300925in}{0.494331in}}%
\pgfpathlineto{\pgfqpoint{5.247450in}{0.494331in}}%
\pgfpathlineto{\pgfqpoint{5.193975in}{0.494331in}}%
\pgfpathlineto{\pgfqpoint{5.140500in}{0.494331in}}%
\pgfpathlineto{\pgfqpoint{5.087025in}{0.494331in}}%
\pgfpathlineto{\pgfqpoint{5.033550in}{0.494331in}}%
\pgfpathlineto{\pgfqpoint{4.980075in}{0.494331in}}%
\pgfpathlineto{\pgfqpoint{4.926600in}{0.494331in}}%
\pgfpathlineto{\pgfqpoint{4.873125in}{0.494331in}}%
\pgfpathlineto{\pgfqpoint{4.819650in}{0.494331in}}%
\pgfpathlineto{\pgfqpoint{4.766175in}{0.494331in}}%
\pgfpathlineto{\pgfqpoint{4.712700in}{0.494331in}}%
\pgfpathlineto{\pgfqpoint{4.659225in}{0.494331in}}%
\pgfpathlineto{\pgfqpoint{4.605750in}{0.494479in}}%
\pgfpathlineto{\pgfqpoint{4.552275in}{0.494479in}}%
\pgfpathlineto{\pgfqpoint{4.498800in}{0.494479in}}%
\pgfpathlineto{\pgfqpoint{4.445325in}{0.494479in}}%
\pgfpathlineto{\pgfqpoint{4.391850in}{0.494479in}}%
\pgfpathlineto{\pgfqpoint{4.338375in}{0.494479in}}%
\pgfpathlineto{\pgfqpoint{4.284900in}{0.494479in}}%
\pgfpathlineto{\pgfqpoint{4.231425in}{0.494479in}}%
\pgfpathlineto{\pgfqpoint{4.177950in}{0.494479in}}%
\pgfpathlineto{\pgfqpoint{4.124475in}{0.494479in}}%
\pgfpathlineto{\pgfqpoint{4.071000in}{0.494479in}}%
\pgfpathlineto{\pgfqpoint{4.017525in}{0.494479in}}%
\pgfpathlineto{\pgfqpoint{3.964050in}{0.494479in}}%
\pgfpathlineto{\pgfqpoint{3.910575in}{0.494479in}}%
\pgfpathlineto{\pgfqpoint{3.857100in}{0.494479in}}%
\pgfpathlineto{\pgfqpoint{3.803625in}{0.497642in}}%
\pgfpathlineto{\pgfqpoint{3.750150in}{0.498603in}}%
\pgfpathlineto{\pgfqpoint{3.696675in}{0.498603in}}%
\pgfpathlineto{\pgfqpoint{3.643200in}{0.498603in}}%
\pgfpathlineto{\pgfqpoint{3.589725in}{0.498603in}}%
\pgfpathlineto{\pgfqpoint{3.536250in}{0.498603in}}%
\pgfpathlineto{\pgfqpoint{3.482775in}{0.518753in}}%
\pgfpathlineto{\pgfqpoint{3.429300in}{0.523570in}}%
\pgfpathlineto{\pgfqpoint{3.375825in}{0.526410in}}%
\pgfpathlineto{\pgfqpoint{3.322350in}{0.526410in}}%
\pgfpathlineto{\pgfqpoint{3.268875in}{0.526410in}}%
\pgfpathlineto{\pgfqpoint{3.215400in}{0.529919in}}%
\pgfpathlineto{\pgfqpoint{3.161925in}{0.529919in}}%
\pgfpathlineto{\pgfqpoint{3.108450in}{0.529919in}}%
\pgfpathlineto{\pgfqpoint{3.054975in}{0.529919in}}%
\pgfpathlineto{\pgfqpoint{3.001500in}{0.529919in}}%
\pgfpathlineto{\pgfqpoint{2.948025in}{0.529919in}}%
\pgfpathlineto{\pgfqpoint{2.894550in}{0.529919in}}%
\pgfpathlineto{\pgfqpoint{2.841075in}{0.529919in}}%
\pgfpathlineto{\pgfqpoint{2.787600in}{0.529919in}}%
\pgfpathlineto{\pgfqpoint{2.734125in}{0.533013in}}%
\pgfpathlineto{\pgfqpoint{2.680650in}{0.533013in}}%
\pgfpathlineto{\pgfqpoint{2.627175in}{0.533013in}}%
\pgfpathlineto{\pgfqpoint{2.573700in}{0.533013in}}%
\pgfpathlineto{\pgfqpoint{2.520225in}{0.533013in}}%
\pgfpathlineto{\pgfqpoint{2.466750in}{0.533013in}}%
\pgfpathlineto{\pgfqpoint{2.413275in}{0.533013in}}%
\pgfpathlineto{\pgfqpoint{2.359800in}{0.533013in}}%
\pgfpathlineto{\pgfqpoint{2.306325in}{0.558368in}}%
\pgfpathlineto{\pgfqpoint{2.252850in}{0.558368in}}%
\pgfpathlineto{\pgfqpoint{2.199375in}{0.558368in}}%
\pgfpathlineto{\pgfqpoint{2.145900in}{0.558368in}}%
\pgfpathlineto{\pgfqpoint{2.092425in}{0.558368in}}%
\pgfpathlineto{\pgfqpoint{2.038950in}{0.575781in}}%
\pgfpathlineto{\pgfqpoint{1.985475in}{0.575781in}}%
\pgfpathlineto{\pgfqpoint{1.932000in}{0.575781in}}%
\pgfpathlineto{\pgfqpoint{1.878525in}{1.031344in}}%
\pgfpathlineto{\pgfqpoint{1.825050in}{1.031344in}}%
\pgfpathlineto{\pgfqpoint{1.771575in}{1.031344in}}%
\pgfpathlineto{\pgfqpoint{1.718100in}{1.031344in}}%
\pgfpathlineto{\pgfqpoint{1.664625in}{1.031344in}}%
\pgfpathlineto{\pgfqpoint{1.611150in}{1.031344in}}%
\pgfpathlineto{\pgfqpoint{1.557675in}{1.031344in}}%
\pgfpathlineto{\pgfqpoint{1.504200in}{1.031344in}}%
\pgfpathlineto{\pgfqpoint{1.450725in}{1.031344in}}%
\pgfpathlineto{\pgfqpoint{1.397250in}{1.031344in}}%
\pgfpathlineto{\pgfqpoint{1.343775in}{1.073448in}}%
\pgfpathlineto{\pgfqpoint{1.290300in}{1.073448in}}%
\pgfpathlineto{\pgfqpoint{1.236825in}{1.073448in}}%
\pgfpathlineto{\pgfqpoint{1.183350in}{1.142542in}}%
\pgfpathlineto{\pgfqpoint{1.129875in}{1.267907in}}%
\pgfpathlineto{\pgfqpoint{1.076400in}{1.267907in}}%
\pgfpathlineto{\pgfqpoint{1.022925in}{1.267907in}}%
\pgfpathlineto{\pgfqpoint{0.969450in}{1.323684in}}%
\pgfpathlineto{\pgfqpoint{0.915975in}{1.919551in}}%
\pgfpathlineto{\pgfqpoint{0.862500in}{2.138516in}}%
\pgfpathclose%
\pgfusepath{fill}%
\end{pgfscope}%
\begin{pgfscope}%
\pgfpathrectangle{\pgfqpoint{0.862500in}{0.375000in}}{\pgfqpoint{5.347500in}{2.265000in}}%
\pgfusepath{clip}%
\pgfsetbuttcap%
\pgfsetroundjoin%
\definecolor{currentfill}{rgb}{1.000000,0.498039,0.054902}%
\pgfsetfillcolor{currentfill}%
\pgfsetfillopacity{0.200000}%
\pgfsetlinewidth{0.000000pt}%
\definecolor{currentstroke}{rgb}{0.000000,0.000000,0.000000}%
\pgfsetstrokecolor{currentstroke}%
\pgfsetdash{}{0pt}%
\pgfpathmoveto{\pgfqpoint{0.862500in}{1.883605in}}%
\pgfpathlineto{\pgfqpoint{0.862500in}{2.393966in}}%
\pgfpathlineto{\pgfqpoint{0.915975in}{1.863186in}}%
\pgfpathlineto{\pgfqpoint{0.969450in}{1.270343in}}%
\pgfpathlineto{\pgfqpoint{1.022925in}{1.270343in}}%
\pgfpathlineto{\pgfqpoint{1.076400in}{1.250829in}}%
\pgfpathlineto{\pgfqpoint{1.129875in}{1.250829in}}%
\pgfpathlineto{\pgfqpoint{1.183350in}{1.202013in}}%
\pgfpathlineto{\pgfqpoint{1.236825in}{1.202013in}}%
\pgfpathlineto{\pgfqpoint{1.290300in}{1.202013in}}%
\pgfpathlineto{\pgfqpoint{1.343775in}{1.202013in}}%
\pgfpathlineto{\pgfqpoint{1.397250in}{1.202013in}}%
\pgfpathlineto{\pgfqpoint{1.450725in}{1.202013in}}%
\pgfpathlineto{\pgfqpoint{1.504200in}{1.202013in}}%
\pgfpathlineto{\pgfqpoint{1.557675in}{1.202013in}}%
\pgfpathlineto{\pgfqpoint{1.611150in}{1.010227in}}%
\pgfpathlineto{\pgfqpoint{1.664625in}{1.010227in}}%
\pgfpathlineto{\pgfqpoint{1.718100in}{1.010227in}}%
\pgfpathlineto{\pgfqpoint{1.771575in}{1.010227in}}%
\pgfpathlineto{\pgfqpoint{1.825050in}{1.010227in}}%
\pgfpathlineto{\pgfqpoint{1.878525in}{1.010227in}}%
\pgfpathlineto{\pgfqpoint{1.932000in}{0.976586in}}%
\pgfpathlineto{\pgfqpoint{1.985475in}{0.802793in}}%
\pgfpathlineto{\pgfqpoint{2.038950in}{0.802793in}}%
\pgfpathlineto{\pgfqpoint{2.092425in}{0.782151in}}%
\pgfpathlineto{\pgfqpoint{2.145900in}{0.782151in}}%
\pgfpathlineto{\pgfqpoint{2.199375in}{0.782151in}}%
\pgfpathlineto{\pgfqpoint{2.252850in}{0.782151in}}%
\pgfpathlineto{\pgfqpoint{2.306325in}{0.782151in}}%
\pgfpathlineto{\pgfqpoint{2.359800in}{0.782151in}}%
\pgfpathlineto{\pgfqpoint{2.413275in}{0.782151in}}%
\pgfpathlineto{\pgfqpoint{2.466750in}{0.782151in}}%
\pgfpathlineto{\pgfqpoint{2.520225in}{0.782151in}}%
\pgfpathlineto{\pgfqpoint{2.573700in}{0.782151in}}%
\pgfpathlineto{\pgfqpoint{2.627175in}{0.615452in}}%
\pgfpathlineto{\pgfqpoint{2.680650in}{0.615452in}}%
\pgfpathlineto{\pgfqpoint{2.734125in}{0.615452in}}%
\pgfpathlineto{\pgfqpoint{2.787600in}{0.615452in}}%
\pgfpathlineto{\pgfqpoint{2.841075in}{0.615452in}}%
\pgfpathlineto{\pgfqpoint{2.894550in}{0.615452in}}%
\pgfpathlineto{\pgfqpoint{2.948025in}{0.609570in}}%
\pgfpathlineto{\pgfqpoint{3.001500in}{0.609570in}}%
\pgfpathlineto{\pgfqpoint{3.054975in}{0.609570in}}%
\pgfpathlineto{\pgfqpoint{3.108450in}{0.609570in}}%
\pgfpathlineto{\pgfqpoint{3.161925in}{0.609570in}}%
\pgfpathlineto{\pgfqpoint{3.215400in}{0.601015in}}%
\pgfpathlineto{\pgfqpoint{3.268875in}{0.601015in}}%
\pgfpathlineto{\pgfqpoint{3.322350in}{0.601015in}}%
\pgfpathlineto{\pgfqpoint{3.375825in}{0.601015in}}%
\pgfpathlineto{\pgfqpoint{3.429300in}{0.601015in}}%
\pgfpathlineto{\pgfqpoint{3.482775in}{0.601015in}}%
\pgfpathlineto{\pgfqpoint{3.536250in}{0.593318in}}%
\pgfpathlineto{\pgfqpoint{3.589725in}{0.593318in}}%
\pgfpathlineto{\pgfqpoint{3.643200in}{0.593318in}}%
\pgfpathlineto{\pgfqpoint{3.696675in}{0.593318in}}%
\pgfpathlineto{\pgfqpoint{3.750150in}{0.593318in}}%
\pgfpathlineto{\pgfqpoint{3.803625in}{0.593318in}}%
\pgfpathlineto{\pgfqpoint{3.857100in}{0.593318in}}%
\pgfpathlineto{\pgfqpoint{3.910575in}{0.593318in}}%
\pgfpathlineto{\pgfqpoint{3.964050in}{0.593318in}}%
\pgfpathlineto{\pgfqpoint{4.017525in}{0.593318in}}%
\pgfpathlineto{\pgfqpoint{4.071000in}{0.593318in}}%
\pgfpathlineto{\pgfqpoint{4.124475in}{0.593318in}}%
\pgfpathlineto{\pgfqpoint{4.177950in}{0.593318in}}%
\pgfpathlineto{\pgfqpoint{4.231425in}{0.591618in}}%
\pgfpathlineto{\pgfqpoint{4.284900in}{0.591618in}}%
\pgfpathlineto{\pgfqpoint{4.338375in}{0.567512in}}%
\pgfpathlineto{\pgfqpoint{4.391850in}{0.567512in}}%
\pgfpathlineto{\pgfqpoint{4.445325in}{0.567512in}}%
\pgfpathlineto{\pgfqpoint{4.498800in}{0.567512in}}%
\pgfpathlineto{\pgfqpoint{4.552275in}{0.566446in}}%
\pgfpathlineto{\pgfqpoint{4.605750in}{0.566446in}}%
\pgfpathlineto{\pgfqpoint{4.659225in}{0.566446in}}%
\pgfpathlineto{\pgfqpoint{4.712700in}{0.566446in}}%
\pgfpathlineto{\pgfqpoint{4.766175in}{0.562224in}}%
\pgfpathlineto{\pgfqpoint{4.819650in}{0.562224in}}%
\pgfpathlineto{\pgfqpoint{4.873125in}{0.562224in}}%
\pgfpathlineto{\pgfqpoint{4.926600in}{0.562224in}}%
\pgfpathlineto{\pgfqpoint{4.980075in}{0.562224in}}%
\pgfpathlineto{\pgfqpoint{5.033550in}{0.562224in}}%
\pgfpathlineto{\pgfqpoint{5.087025in}{0.562224in}}%
\pgfpathlineto{\pgfqpoint{5.140500in}{0.562224in}}%
\pgfpathlineto{\pgfqpoint{5.193975in}{0.560406in}}%
\pgfpathlineto{\pgfqpoint{5.247450in}{0.560406in}}%
\pgfpathlineto{\pgfqpoint{5.300925in}{0.560406in}}%
\pgfpathlineto{\pgfqpoint{5.354400in}{0.560406in}}%
\pgfpathlineto{\pgfqpoint{5.407875in}{0.560406in}}%
\pgfpathlineto{\pgfqpoint{5.461350in}{0.560406in}}%
\pgfpathlineto{\pgfqpoint{5.514825in}{0.560406in}}%
\pgfpathlineto{\pgfqpoint{5.568300in}{0.560406in}}%
\pgfpathlineto{\pgfqpoint{5.621775in}{0.560406in}}%
\pgfpathlineto{\pgfqpoint{5.675250in}{0.560406in}}%
\pgfpathlineto{\pgfqpoint{5.728725in}{0.560406in}}%
\pgfpathlineto{\pgfqpoint{5.782200in}{0.560406in}}%
\pgfpathlineto{\pgfqpoint{5.835675in}{0.549201in}}%
\pgfpathlineto{\pgfqpoint{5.889150in}{0.549201in}}%
\pgfpathlineto{\pgfqpoint{5.942625in}{0.548849in}}%
\pgfpathlineto{\pgfqpoint{5.996100in}{0.548849in}}%
\pgfpathlineto{\pgfqpoint{6.049575in}{0.548849in}}%
\pgfpathlineto{\pgfqpoint{6.103050in}{0.548849in}}%
\pgfpathlineto{\pgfqpoint{6.156525in}{0.548849in}}%
\pgfpathlineto{\pgfqpoint{6.210000in}{0.548849in}}%
\pgfpathlineto{\pgfqpoint{6.263475in}{0.548849in}}%
\pgfpathlineto{\pgfqpoint{6.316950in}{0.548849in}}%
\pgfpathlineto{\pgfqpoint{6.370425in}{0.548849in}}%
\pgfpathlineto{\pgfqpoint{6.423900in}{0.548512in}}%
\pgfpathlineto{\pgfqpoint{6.477375in}{0.548512in}}%
\pgfpathlineto{\pgfqpoint{6.530850in}{0.548512in}}%
\pgfpathlineto{\pgfqpoint{6.584325in}{0.548512in}}%
\pgfpathlineto{\pgfqpoint{6.637800in}{0.548512in}}%
\pgfpathlineto{\pgfqpoint{6.691275in}{0.548512in}}%
\pgfpathlineto{\pgfqpoint{6.744750in}{0.548512in}}%
\pgfpathlineto{\pgfqpoint{6.798225in}{0.546128in}}%
\pgfpathlineto{\pgfqpoint{6.851700in}{0.546128in}}%
\pgfpathlineto{\pgfqpoint{6.905175in}{0.546128in}}%
\pgfpathlineto{\pgfqpoint{6.958650in}{0.546128in}}%
\pgfpathlineto{\pgfqpoint{7.012125in}{0.546128in}}%
\pgfpathlineto{\pgfqpoint{7.065600in}{0.546128in}}%
\pgfpathlineto{\pgfqpoint{7.119075in}{0.546128in}}%
\pgfpathlineto{\pgfqpoint{7.172550in}{0.546128in}}%
\pgfpathlineto{\pgfqpoint{7.226025in}{0.546128in}}%
\pgfpathlineto{\pgfqpoint{7.226025in}{0.518623in}}%
\pgfpathlineto{\pgfqpoint{7.226025in}{0.518623in}}%
\pgfpathlineto{\pgfqpoint{7.172550in}{0.518623in}}%
\pgfpathlineto{\pgfqpoint{7.119075in}{0.518623in}}%
\pgfpathlineto{\pgfqpoint{7.065600in}{0.518623in}}%
\pgfpathlineto{\pgfqpoint{7.012125in}{0.518623in}}%
\pgfpathlineto{\pgfqpoint{6.958650in}{0.518623in}}%
\pgfpathlineto{\pgfqpoint{6.905175in}{0.518623in}}%
\pgfpathlineto{\pgfqpoint{6.851700in}{0.518623in}}%
\pgfpathlineto{\pgfqpoint{6.798225in}{0.518623in}}%
\pgfpathlineto{\pgfqpoint{6.744750in}{0.522106in}}%
\pgfpathlineto{\pgfqpoint{6.691275in}{0.522106in}}%
\pgfpathlineto{\pgfqpoint{6.637800in}{0.522106in}}%
\pgfpathlineto{\pgfqpoint{6.584325in}{0.522106in}}%
\pgfpathlineto{\pgfqpoint{6.530850in}{0.522106in}}%
\pgfpathlineto{\pgfqpoint{6.477375in}{0.522106in}}%
\pgfpathlineto{\pgfqpoint{6.423900in}{0.522106in}}%
\pgfpathlineto{\pgfqpoint{6.370425in}{0.522546in}}%
\pgfpathlineto{\pgfqpoint{6.316950in}{0.522546in}}%
\pgfpathlineto{\pgfqpoint{6.263475in}{0.522546in}}%
\pgfpathlineto{\pgfqpoint{6.210000in}{0.522546in}}%
\pgfpathlineto{\pgfqpoint{6.156525in}{0.522546in}}%
\pgfpathlineto{\pgfqpoint{6.103050in}{0.522546in}}%
\pgfpathlineto{\pgfqpoint{6.049575in}{0.522546in}}%
\pgfpathlineto{\pgfqpoint{5.996100in}{0.522546in}}%
\pgfpathlineto{\pgfqpoint{5.942625in}{0.522546in}}%
\pgfpathlineto{\pgfqpoint{5.889150in}{0.523719in}}%
\pgfpathlineto{\pgfqpoint{5.835675in}{0.523719in}}%
\pgfpathlineto{\pgfqpoint{5.782200in}{0.526412in}}%
\pgfpathlineto{\pgfqpoint{5.728725in}{0.526412in}}%
\pgfpathlineto{\pgfqpoint{5.675250in}{0.526412in}}%
\pgfpathlineto{\pgfqpoint{5.621775in}{0.526412in}}%
\pgfpathlineto{\pgfqpoint{5.568300in}{0.526412in}}%
\pgfpathlineto{\pgfqpoint{5.514825in}{0.526412in}}%
\pgfpathlineto{\pgfqpoint{5.461350in}{0.526412in}}%
\pgfpathlineto{\pgfqpoint{5.407875in}{0.526412in}}%
\pgfpathlineto{\pgfqpoint{5.354400in}{0.526412in}}%
\pgfpathlineto{\pgfqpoint{5.300925in}{0.526412in}}%
\pgfpathlineto{\pgfqpoint{5.247450in}{0.526412in}}%
\pgfpathlineto{\pgfqpoint{5.193975in}{0.526412in}}%
\pgfpathlineto{\pgfqpoint{5.140500in}{0.531411in}}%
\pgfpathlineto{\pgfqpoint{5.087025in}{0.531411in}}%
\pgfpathlineto{\pgfqpoint{5.033550in}{0.531411in}}%
\pgfpathlineto{\pgfqpoint{4.980075in}{0.531411in}}%
\pgfpathlineto{\pgfqpoint{4.926600in}{0.531411in}}%
\pgfpathlineto{\pgfqpoint{4.873125in}{0.531411in}}%
\pgfpathlineto{\pgfqpoint{4.819650in}{0.531411in}}%
\pgfpathlineto{\pgfqpoint{4.766175in}{0.531411in}}%
\pgfpathlineto{\pgfqpoint{4.712700in}{0.537741in}}%
\pgfpathlineto{\pgfqpoint{4.659225in}{0.537741in}}%
\pgfpathlineto{\pgfqpoint{4.605750in}{0.537741in}}%
\pgfpathlineto{\pgfqpoint{4.552275in}{0.537741in}}%
\pgfpathlineto{\pgfqpoint{4.498800in}{0.538010in}}%
\pgfpathlineto{\pgfqpoint{4.445325in}{0.538010in}}%
\pgfpathlineto{\pgfqpoint{4.391850in}{0.538010in}}%
\pgfpathlineto{\pgfqpoint{4.338375in}{0.538010in}}%
\pgfpathlineto{\pgfqpoint{4.284900in}{0.543836in}}%
\pgfpathlineto{\pgfqpoint{4.231425in}{0.543836in}}%
\pgfpathlineto{\pgfqpoint{4.177950in}{0.544238in}}%
\pgfpathlineto{\pgfqpoint{4.124475in}{0.544238in}}%
\pgfpathlineto{\pgfqpoint{4.071000in}{0.544238in}}%
\pgfpathlineto{\pgfqpoint{4.017525in}{0.544238in}}%
\pgfpathlineto{\pgfqpoint{3.964050in}{0.544238in}}%
\pgfpathlineto{\pgfqpoint{3.910575in}{0.544238in}}%
\pgfpathlineto{\pgfqpoint{3.857100in}{0.544238in}}%
\pgfpathlineto{\pgfqpoint{3.803625in}{0.544238in}}%
\pgfpathlineto{\pgfqpoint{3.750150in}{0.544238in}}%
\pgfpathlineto{\pgfqpoint{3.696675in}{0.544238in}}%
\pgfpathlineto{\pgfqpoint{3.643200in}{0.544238in}}%
\pgfpathlineto{\pgfqpoint{3.589725in}{0.544238in}}%
\pgfpathlineto{\pgfqpoint{3.536250in}{0.544238in}}%
\pgfpathlineto{\pgfqpoint{3.482775in}{0.546055in}}%
\pgfpathlineto{\pgfqpoint{3.429300in}{0.546055in}}%
\pgfpathlineto{\pgfqpoint{3.375825in}{0.546055in}}%
\pgfpathlineto{\pgfqpoint{3.322350in}{0.546055in}}%
\pgfpathlineto{\pgfqpoint{3.268875in}{0.546055in}}%
\pgfpathlineto{\pgfqpoint{3.215400in}{0.546055in}}%
\pgfpathlineto{\pgfqpoint{3.161925in}{0.548072in}}%
\pgfpathlineto{\pgfqpoint{3.108450in}{0.548072in}}%
\pgfpathlineto{\pgfqpoint{3.054975in}{0.548072in}}%
\pgfpathlineto{\pgfqpoint{3.001500in}{0.548072in}}%
\pgfpathlineto{\pgfqpoint{2.948025in}{0.548072in}}%
\pgfpathlineto{\pgfqpoint{2.894550in}{0.562175in}}%
\pgfpathlineto{\pgfqpoint{2.841075in}{0.562175in}}%
\pgfpathlineto{\pgfqpoint{2.787600in}{0.562175in}}%
\pgfpathlineto{\pgfqpoint{2.734125in}{0.562175in}}%
\pgfpathlineto{\pgfqpoint{2.680650in}{0.562175in}}%
\pgfpathlineto{\pgfqpoint{2.627175in}{0.562175in}}%
\pgfpathlineto{\pgfqpoint{2.573700in}{0.603016in}}%
\pgfpathlineto{\pgfqpoint{2.520225in}{0.603016in}}%
\pgfpathlineto{\pgfqpoint{2.466750in}{0.603016in}}%
\pgfpathlineto{\pgfqpoint{2.413275in}{0.603016in}}%
\pgfpathlineto{\pgfqpoint{2.359800in}{0.603016in}}%
\pgfpathlineto{\pgfqpoint{2.306325in}{0.603016in}}%
\pgfpathlineto{\pgfqpoint{2.252850in}{0.603016in}}%
\pgfpathlineto{\pgfqpoint{2.199375in}{0.603016in}}%
\pgfpathlineto{\pgfqpoint{2.145900in}{0.603016in}}%
\pgfpathlineto{\pgfqpoint{2.092425in}{0.603016in}}%
\pgfpathlineto{\pgfqpoint{2.038950in}{0.646485in}}%
\pgfpathlineto{\pgfqpoint{1.985475in}{0.646485in}}%
\pgfpathlineto{\pgfqpoint{1.932000in}{0.833541in}}%
\pgfpathlineto{\pgfqpoint{1.878525in}{0.998716in}}%
\pgfpathlineto{\pgfqpoint{1.825050in}{0.998716in}}%
\pgfpathlineto{\pgfqpoint{1.771575in}{0.998716in}}%
\pgfpathlineto{\pgfqpoint{1.718100in}{0.998716in}}%
\pgfpathlineto{\pgfqpoint{1.664625in}{0.998716in}}%
\pgfpathlineto{\pgfqpoint{1.611150in}{0.998716in}}%
\pgfpathlineto{\pgfqpoint{1.557675in}{1.050345in}}%
\pgfpathlineto{\pgfqpoint{1.504200in}{1.050345in}}%
\pgfpathlineto{\pgfqpoint{1.450725in}{1.050345in}}%
\pgfpathlineto{\pgfqpoint{1.397250in}{1.050345in}}%
\pgfpathlineto{\pgfqpoint{1.343775in}{1.050345in}}%
\pgfpathlineto{\pgfqpoint{1.290300in}{1.050345in}}%
\pgfpathlineto{\pgfqpoint{1.236825in}{1.050345in}}%
\pgfpathlineto{\pgfqpoint{1.183350in}{1.050345in}}%
\pgfpathlineto{\pgfqpoint{1.129875in}{1.123248in}}%
\pgfpathlineto{\pgfqpoint{1.076400in}{1.123248in}}%
\pgfpathlineto{\pgfqpoint{1.022925in}{1.140248in}}%
\pgfpathlineto{\pgfqpoint{0.969450in}{1.140248in}}%
\pgfpathlineto{\pgfqpoint{0.915975in}{1.451514in}}%
\pgfpathlineto{\pgfqpoint{0.862500in}{1.883605in}}%
\pgfpathclose%
\pgfusepath{fill}%
\end{pgfscope}%
\begin{pgfscope}%
\pgfpathrectangle{\pgfqpoint{0.862500in}{0.375000in}}{\pgfqpoint{5.347500in}{2.265000in}}%
\pgfusepath{clip}%
\pgfsetbuttcap%
\pgfsetroundjoin%
\definecolor{currentfill}{rgb}{0.172549,0.627451,0.172549}%
\pgfsetfillcolor{currentfill}%
\pgfsetfillopacity{0.200000}%
\pgfsetlinewidth{0.000000pt}%
\definecolor{currentstroke}{rgb}{0.000000,0.000000,0.000000}%
\pgfsetstrokecolor{currentstroke}%
\pgfsetdash{}{0pt}%
\pgfpathmoveto{\pgfqpoint{0.862500in}{1.682101in}}%
\pgfpathlineto{\pgfqpoint{0.862500in}{2.000578in}}%
\pgfpathlineto{\pgfqpoint{0.915975in}{1.566120in}}%
\pgfpathlineto{\pgfqpoint{0.969450in}{1.447805in}}%
\pgfpathlineto{\pgfqpoint{1.022925in}{1.440763in}}%
\pgfpathlineto{\pgfqpoint{1.076400in}{1.440763in}}%
\pgfpathlineto{\pgfqpoint{1.129875in}{1.273772in}}%
\pgfpathlineto{\pgfqpoint{1.183350in}{0.848953in}}%
\pgfpathlineto{\pgfqpoint{1.236825in}{0.523583in}}%
\pgfpathlineto{\pgfqpoint{1.290300in}{0.523583in}}%
\pgfpathlineto{\pgfqpoint{1.343775in}{0.523583in}}%
\pgfpathlineto{\pgfqpoint{1.397250in}{0.523583in}}%
\pgfpathlineto{\pgfqpoint{1.450725in}{0.523583in}}%
\pgfpathlineto{\pgfqpoint{1.504200in}{0.523583in}}%
\pgfpathlineto{\pgfqpoint{1.557675in}{0.523583in}}%
\pgfpathlineto{\pgfqpoint{1.611150in}{0.523583in}}%
\pgfpathlineto{\pgfqpoint{1.664625in}{0.523583in}}%
\pgfpathlineto{\pgfqpoint{1.718100in}{0.523583in}}%
\pgfpathlineto{\pgfqpoint{1.771575in}{0.523583in}}%
\pgfpathlineto{\pgfqpoint{1.825050in}{0.523583in}}%
\pgfpathlineto{\pgfqpoint{1.878525in}{0.523583in}}%
\pgfpathlineto{\pgfqpoint{1.932000in}{0.523583in}}%
\pgfpathlineto{\pgfqpoint{1.985475in}{0.523583in}}%
\pgfpathlineto{\pgfqpoint{2.038950in}{0.523583in}}%
\pgfpathlineto{\pgfqpoint{2.092425in}{0.523583in}}%
\pgfpathlineto{\pgfqpoint{2.145900in}{0.523583in}}%
\pgfpathlineto{\pgfqpoint{2.199375in}{0.523583in}}%
\pgfpathlineto{\pgfqpoint{2.252850in}{0.519837in}}%
\pgfpathlineto{\pgfqpoint{2.306325in}{0.519837in}}%
\pgfpathlineto{\pgfqpoint{2.359800in}{0.519837in}}%
\pgfpathlineto{\pgfqpoint{2.413275in}{0.519837in}}%
\pgfpathlineto{\pgfqpoint{2.466750in}{0.519837in}}%
\pgfpathlineto{\pgfqpoint{2.520225in}{0.519837in}}%
\pgfpathlineto{\pgfqpoint{2.573700in}{0.519837in}}%
\pgfpathlineto{\pgfqpoint{2.627175in}{0.519837in}}%
\pgfpathlineto{\pgfqpoint{2.680650in}{0.519837in}}%
\pgfpathlineto{\pgfqpoint{2.734125in}{0.519837in}}%
\pgfpathlineto{\pgfqpoint{2.787600in}{0.519837in}}%
\pgfpathlineto{\pgfqpoint{2.841075in}{0.519837in}}%
\pgfpathlineto{\pgfqpoint{2.894550in}{0.519837in}}%
\pgfpathlineto{\pgfqpoint{2.948025in}{0.519837in}}%
\pgfpathlineto{\pgfqpoint{3.001500in}{0.519837in}}%
\pgfpathlineto{\pgfqpoint{3.054975in}{0.519837in}}%
\pgfpathlineto{\pgfqpoint{3.108450in}{0.518515in}}%
\pgfpathlineto{\pgfqpoint{3.161925in}{0.517932in}}%
\pgfpathlineto{\pgfqpoint{3.215400in}{0.517932in}}%
\pgfpathlineto{\pgfqpoint{3.268875in}{0.510518in}}%
\pgfpathlineto{\pgfqpoint{3.322350in}{0.510518in}}%
\pgfpathlineto{\pgfqpoint{3.375825in}{0.510518in}}%
\pgfpathlineto{\pgfqpoint{3.429300in}{0.510518in}}%
\pgfpathlineto{\pgfqpoint{3.482775in}{0.509208in}}%
\pgfpathlineto{\pgfqpoint{3.536250in}{0.509134in}}%
\pgfpathlineto{\pgfqpoint{3.589725in}{0.509134in}}%
\pgfpathlineto{\pgfqpoint{3.643200in}{0.504527in}}%
\pgfpathlineto{\pgfqpoint{3.696675in}{0.504527in}}%
\pgfpathlineto{\pgfqpoint{3.750150in}{0.504527in}}%
\pgfpathlineto{\pgfqpoint{3.803625in}{0.504527in}}%
\pgfpathlineto{\pgfqpoint{3.857100in}{0.502187in}}%
\pgfpathlineto{\pgfqpoint{3.910575in}{0.502187in}}%
\pgfpathlineto{\pgfqpoint{3.964050in}{0.502187in}}%
\pgfpathlineto{\pgfqpoint{4.017525in}{0.502187in}}%
\pgfpathlineto{\pgfqpoint{4.071000in}{0.502187in}}%
\pgfpathlineto{\pgfqpoint{4.124475in}{0.502187in}}%
\pgfpathlineto{\pgfqpoint{4.177950in}{0.502187in}}%
\pgfpathlineto{\pgfqpoint{4.231425in}{0.502187in}}%
\pgfpathlineto{\pgfqpoint{4.284900in}{0.502187in}}%
\pgfpathlineto{\pgfqpoint{4.338375in}{0.502187in}}%
\pgfpathlineto{\pgfqpoint{4.391850in}{0.502187in}}%
\pgfpathlineto{\pgfqpoint{4.445325in}{0.502187in}}%
\pgfpathlineto{\pgfqpoint{4.498800in}{0.502187in}}%
\pgfpathlineto{\pgfqpoint{4.552275in}{0.502187in}}%
\pgfpathlineto{\pgfqpoint{4.605750in}{0.502187in}}%
\pgfpathlineto{\pgfqpoint{4.659225in}{0.502187in}}%
\pgfpathlineto{\pgfqpoint{4.712700in}{0.502187in}}%
\pgfpathlineto{\pgfqpoint{4.766175in}{0.502187in}}%
\pgfpathlineto{\pgfqpoint{4.819650in}{0.502187in}}%
\pgfpathlineto{\pgfqpoint{4.873125in}{0.502187in}}%
\pgfpathlineto{\pgfqpoint{4.926600in}{0.502187in}}%
\pgfpathlineto{\pgfqpoint{4.980075in}{0.502187in}}%
\pgfpathlineto{\pgfqpoint{5.033550in}{0.502187in}}%
\pgfpathlineto{\pgfqpoint{5.087025in}{0.502187in}}%
\pgfpathlineto{\pgfqpoint{5.140500in}{0.502187in}}%
\pgfpathlineto{\pgfqpoint{5.193975in}{0.502187in}}%
\pgfpathlineto{\pgfqpoint{5.247450in}{0.502187in}}%
\pgfpathlineto{\pgfqpoint{5.300925in}{0.502187in}}%
\pgfpathlineto{\pgfqpoint{5.354400in}{0.502187in}}%
\pgfpathlineto{\pgfqpoint{5.407875in}{0.502075in}}%
\pgfpathlineto{\pgfqpoint{5.461350in}{0.502075in}}%
\pgfpathlineto{\pgfqpoint{5.514825in}{0.502075in}}%
\pgfpathlineto{\pgfqpoint{5.568300in}{0.502075in}}%
\pgfpathlineto{\pgfqpoint{5.621775in}{0.502075in}}%
\pgfpathlineto{\pgfqpoint{5.675250in}{0.502075in}}%
\pgfpathlineto{\pgfqpoint{5.728725in}{0.500156in}}%
\pgfpathlineto{\pgfqpoint{5.782200in}{0.500156in}}%
\pgfpathlineto{\pgfqpoint{5.835675in}{0.500156in}}%
\pgfpathlineto{\pgfqpoint{5.889150in}{0.500156in}}%
\pgfpathlineto{\pgfqpoint{5.942625in}{0.500156in}}%
\pgfpathlineto{\pgfqpoint{5.996100in}{0.500156in}}%
\pgfpathlineto{\pgfqpoint{6.049575in}{0.500156in}}%
\pgfpathlineto{\pgfqpoint{6.103050in}{0.500156in}}%
\pgfpathlineto{\pgfqpoint{6.156525in}{0.500156in}}%
\pgfpathlineto{\pgfqpoint{6.210000in}{0.500156in}}%
\pgfpathlineto{\pgfqpoint{6.263475in}{0.500156in}}%
\pgfpathlineto{\pgfqpoint{6.263475in}{0.497020in}}%
\pgfpathlineto{\pgfqpoint{6.263475in}{0.497020in}}%
\pgfpathlineto{\pgfqpoint{6.210000in}{0.497020in}}%
\pgfpathlineto{\pgfqpoint{6.156525in}{0.497020in}}%
\pgfpathlineto{\pgfqpoint{6.103050in}{0.497020in}}%
\pgfpathlineto{\pgfqpoint{6.049575in}{0.497020in}}%
\pgfpathlineto{\pgfqpoint{5.996100in}{0.497020in}}%
\pgfpathlineto{\pgfqpoint{5.942625in}{0.497020in}}%
\pgfpathlineto{\pgfqpoint{5.889150in}{0.497020in}}%
\pgfpathlineto{\pgfqpoint{5.835675in}{0.497020in}}%
\pgfpathlineto{\pgfqpoint{5.782200in}{0.497020in}}%
\pgfpathlineto{\pgfqpoint{5.728725in}{0.497020in}}%
\pgfpathlineto{\pgfqpoint{5.675250in}{0.497442in}}%
\pgfpathlineto{\pgfqpoint{5.621775in}{0.497442in}}%
\pgfpathlineto{\pgfqpoint{5.568300in}{0.497442in}}%
\pgfpathlineto{\pgfqpoint{5.514825in}{0.497442in}}%
\pgfpathlineto{\pgfqpoint{5.461350in}{0.497442in}}%
\pgfpathlineto{\pgfqpoint{5.407875in}{0.497442in}}%
\pgfpathlineto{\pgfqpoint{5.354400in}{0.497466in}}%
\pgfpathlineto{\pgfqpoint{5.300925in}{0.497466in}}%
\pgfpathlineto{\pgfqpoint{5.247450in}{0.497466in}}%
\pgfpathlineto{\pgfqpoint{5.193975in}{0.497466in}}%
\pgfpathlineto{\pgfqpoint{5.140500in}{0.497466in}}%
\pgfpathlineto{\pgfqpoint{5.087025in}{0.497466in}}%
\pgfpathlineto{\pgfqpoint{5.033550in}{0.497466in}}%
\pgfpathlineto{\pgfqpoint{4.980075in}{0.497466in}}%
\pgfpathlineto{\pgfqpoint{4.926600in}{0.497466in}}%
\pgfpathlineto{\pgfqpoint{4.873125in}{0.497466in}}%
\pgfpathlineto{\pgfqpoint{4.819650in}{0.497466in}}%
\pgfpathlineto{\pgfqpoint{4.766175in}{0.497466in}}%
\pgfpathlineto{\pgfqpoint{4.712700in}{0.497466in}}%
\pgfpathlineto{\pgfqpoint{4.659225in}{0.497466in}}%
\pgfpathlineto{\pgfqpoint{4.605750in}{0.497466in}}%
\pgfpathlineto{\pgfqpoint{4.552275in}{0.497466in}}%
\pgfpathlineto{\pgfqpoint{4.498800in}{0.497466in}}%
\pgfpathlineto{\pgfqpoint{4.445325in}{0.497466in}}%
\pgfpathlineto{\pgfqpoint{4.391850in}{0.497466in}}%
\pgfpathlineto{\pgfqpoint{4.338375in}{0.497466in}}%
\pgfpathlineto{\pgfqpoint{4.284900in}{0.497466in}}%
\pgfpathlineto{\pgfqpoint{4.231425in}{0.497466in}}%
\pgfpathlineto{\pgfqpoint{4.177950in}{0.497466in}}%
\pgfpathlineto{\pgfqpoint{4.124475in}{0.497466in}}%
\pgfpathlineto{\pgfqpoint{4.071000in}{0.497466in}}%
\pgfpathlineto{\pgfqpoint{4.017525in}{0.497466in}}%
\pgfpathlineto{\pgfqpoint{3.964050in}{0.497466in}}%
\pgfpathlineto{\pgfqpoint{3.910575in}{0.497466in}}%
\pgfpathlineto{\pgfqpoint{3.857100in}{0.497466in}}%
\pgfpathlineto{\pgfqpoint{3.803625in}{0.499954in}}%
\pgfpathlineto{\pgfqpoint{3.750150in}{0.499954in}}%
\pgfpathlineto{\pgfqpoint{3.696675in}{0.499954in}}%
\pgfpathlineto{\pgfqpoint{3.643200in}{0.499954in}}%
\pgfpathlineto{\pgfqpoint{3.589725in}{0.501376in}}%
\pgfpathlineto{\pgfqpoint{3.536250in}{0.501376in}}%
\pgfpathlineto{\pgfqpoint{3.482775in}{0.501396in}}%
\pgfpathlineto{\pgfqpoint{3.429300in}{0.501733in}}%
\pgfpathlineto{\pgfqpoint{3.375825in}{0.501733in}}%
\pgfpathlineto{\pgfqpoint{3.322350in}{0.501733in}}%
\pgfpathlineto{\pgfqpoint{3.268875in}{0.501733in}}%
\pgfpathlineto{\pgfqpoint{3.215400in}{0.503499in}}%
\pgfpathlineto{\pgfqpoint{3.161925in}{0.503499in}}%
\pgfpathlineto{\pgfqpoint{3.108450in}{0.504935in}}%
\pgfpathlineto{\pgfqpoint{3.054975in}{0.506897in}}%
\pgfpathlineto{\pgfqpoint{3.001500in}{0.506897in}}%
\pgfpathlineto{\pgfqpoint{2.948025in}{0.506897in}}%
\pgfpathlineto{\pgfqpoint{2.894550in}{0.506897in}}%
\pgfpathlineto{\pgfqpoint{2.841075in}{0.506897in}}%
\pgfpathlineto{\pgfqpoint{2.787600in}{0.506897in}}%
\pgfpathlineto{\pgfqpoint{2.734125in}{0.506897in}}%
\pgfpathlineto{\pgfqpoint{2.680650in}{0.506897in}}%
\pgfpathlineto{\pgfqpoint{2.627175in}{0.506897in}}%
\pgfpathlineto{\pgfqpoint{2.573700in}{0.506897in}}%
\pgfpathlineto{\pgfqpoint{2.520225in}{0.506897in}}%
\pgfpathlineto{\pgfqpoint{2.466750in}{0.506897in}}%
\pgfpathlineto{\pgfqpoint{2.413275in}{0.506897in}}%
\pgfpathlineto{\pgfqpoint{2.359800in}{0.506897in}}%
\pgfpathlineto{\pgfqpoint{2.306325in}{0.506897in}}%
\pgfpathlineto{\pgfqpoint{2.252850in}{0.506897in}}%
\pgfpathlineto{\pgfqpoint{2.199375in}{0.510702in}}%
\pgfpathlineto{\pgfqpoint{2.145900in}{0.510702in}}%
\pgfpathlineto{\pgfqpoint{2.092425in}{0.510702in}}%
\pgfpathlineto{\pgfqpoint{2.038950in}{0.510702in}}%
\pgfpathlineto{\pgfqpoint{1.985475in}{0.510702in}}%
\pgfpathlineto{\pgfqpoint{1.932000in}{0.510702in}}%
\pgfpathlineto{\pgfqpoint{1.878525in}{0.510702in}}%
\pgfpathlineto{\pgfqpoint{1.825050in}{0.510702in}}%
\pgfpathlineto{\pgfqpoint{1.771575in}{0.510702in}}%
\pgfpathlineto{\pgfqpoint{1.718100in}{0.510702in}}%
\pgfpathlineto{\pgfqpoint{1.664625in}{0.510702in}}%
\pgfpathlineto{\pgfqpoint{1.611150in}{0.510702in}}%
\pgfpathlineto{\pgfqpoint{1.557675in}{0.510702in}}%
\pgfpathlineto{\pgfqpoint{1.504200in}{0.510702in}}%
\pgfpathlineto{\pgfqpoint{1.450725in}{0.510702in}}%
\pgfpathlineto{\pgfqpoint{1.397250in}{0.510702in}}%
\pgfpathlineto{\pgfqpoint{1.343775in}{0.510702in}}%
\pgfpathlineto{\pgfqpoint{1.290300in}{0.510702in}}%
\pgfpathlineto{\pgfqpoint{1.236825in}{0.510702in}}%
\pgfpathlineto{\pgfqpoint{1.183350in}{0.597908in}}%
\pgfpathlineto{\pgfqpoint{1.129875in}{0.743332in}}%
\pgfpathlineto{\pgfqpoint{1.076400in}{1.093030in}}%
\pgfpathlineto{\pgfqpoint{1.022925in}{1.093030in}}%
\pgfpathlineto{\pgfqpoint{0.969450in}{1.144100in}}%
\pgfpathlineto{\pgfqpoint{0.915975in}{1.536470in}}%
\pgfpathlineto{\pgfqpoint{0.862500in}{1.682101in}}%
\pgfpathclose%
\pgfusepath{fill}%
\end{pgfscope}%
\begin{pgfscope}%
\pgfpathrectangle{\pgfqpoint{0.862500in}{0.375000in}}{\pgfqpoint{5.347500in}{2.265000in}}%
\pgfusepath{clip}%
\pgfsetbuttcap%
\pgfsetroundjoin%
\definecolor{currentfill}{rgb}{0.839216,0.152941,0.156863}%
\pgfsetfillcolor{currentfill}%
\pgfsetfillopacity{0.200000}%
\pgfsetlinewidth{0.000000pt}%
\definecolor{currentstroke}{rgb}{0.000000,0.000000,0.000000}%
\pgfsetstrokecolor{currentstroke}%
\pgfsetdash{}{0pt}%
\pgfpathmoveto{\pgfqpoint{0.862500in}{1.914711in}}%
\pgfpathlineto{\pgfqpoint{0.862500in}{2.151990in}}%
\pgfpathlineto{\pgfqpoint{0.915975in}{1.779904in}}%
\pgfpathlineto{\pgfqpoint{0.969450in}{1.779904in}}%
\pgfpathlineto{\pgfqpoint{1.022925in}{1.779904in}}%
\pgfpathlineto{\pgfqpoint{1.076400in}{1.779904in}}%
\pgfpathlineto{\pgfqpoint{1.129875in}{1.582030in}}%
\pgfpathlineto{\pgfqpoint{1.183350in}{1.582030in}}%
\pgfpathlineto{\pgfqpoint{1.236825in}{1.397641in}}%
\pgfpathlineto{\pgfqpoint{1.290300in}{1.249729in}}%
\pgfpathlineto{\pgfqpoint{1.343775in}{1.249729in}}%
\pgfpathlineto{\pgfqpoint{1.397250in}{1.249729in}}%
\pgfpathlineto{\pgfqpoint{1.450725in}{1.249729in}}%
\pgfpathlineto{\pgfqpoint{1.504200in}{0.744298in}}%
\pgfpathlineto{\pgfqpoint{1.557675in}{0.744298in}}%
\pgfpathlineto{\pgfqpoint{1.611150in}{0.736551in}}%
\pgfpathlineto{\pgfqpoint{1.664625in}{0.721877in}}%
\pgfpathlineto{\pgfqpoint{1.718100in}{0.721877in}}%
\pgfpathlineto{\pgfqpoint{1.771575in}{0.721877in}}%
\pgfpathlineto{\pgfqpoint{1.825050in}{0.721877in}}%
\pgfpathlineto{\pgfqpoint{1.878525in}{0.642806in}}%
\pgfpathlineto{\pgfqpoint{1.932000in}{0.642806in}}%
\pgfpathlineto{\pgfqpoint{1.985475in}{0.632386in}}%
\pgfpathlineto{\pgfqpoint{2.038950in}{0.632386in}}%
\pgfpathlineto{\pgfqpoint{2.092425in}{0.632386in}}%
\pgfpathlineto{\pgfqpoint{2.145900in}{0.632386in}}%
\pgfpathlineto{\pgfqpoint{2.199375in}{0.632386in}}%
\pgfpathlineto{\pgfqpoint{2.252850in}{0.632386in}}%
\pgfpathlineto{\pgfqpoint{2.306325in}{0.632386in}}%
\pgfpathlineto{\pgfqpoint{2.359800in}{0.632386in}}%
\pgfpathlineto{\pgfqpoint{2.413275in}{0.631441in}}%
\pgfpathlineto{\pgfqpoint{2.466750in}{0.631403in}}%
\pgfpathlineto{\pgfqpoint{2.520225in}{0.631403in}}%
\pgfpathlineto{\pgfqpoint{2.573700in}{0.631403in}}%
\pgfpathlineto{\pgfqpoint{2.627175in}{0.631403in}}%
\pgfpathlineto{\pgfqpoint{2.680650in}{0.627752in}}%
\pgfpathlineto{\pgfqpoint{2.734125in}{0.627752in}}%
\pgfpathlineto{\pgfqpoint{2.787600in}{0.598410in}}%
\pgfpathlineto{\pgfqpoint{2.841075in}{0.598410in}}%
\pgfpathlineto{\pgfqpoint{2.894550in}{0.598410in}}%
\pgfpathlineto{\pgfqpoint{2.948025in}{0.597850in}}%
\pgfpathlineto{\pgfqpoint{3.001500in}{0.597850in}}%
\pgfpathlineto{\pgfqpoint{3.054975in}{0.597850in}}%
\pgfpathlineto{\pgfqpoint{3.108450in}{0.535173in}}%
\pgfpathlineto{\pgfqpoint{3.161925in}{0.535173in}}%
\pgfpathlineto{\pgfqpoint{3.215400in}{0.535173in}}%
\pgfpathlineto{\pgfqpoint{3.268875in}{0.535173in}}%
\pgfpathlineto{\pgfqpoint{3.322350in}{0.535173in}}%
\pgfpathlineto{\pgfqpoint{3.375825in}{0.535173in}}%
\pgfpathlineto{\pgfqpoint{3.429300in}{0.535173in}}%
\pgfpathlineto{\pgfqpoint{3.482775in}{0.535173in}}%
\pgfpathlineto{\pgfqpoint{3.536250in}{0.535173in}}%
\pgfpathlineto{\pgfqpoint{3.589725in}{0.535173in}}%
\pgfpathlineto{\pgfqpoint{3.643200in}{0.535173in}}%
\pgfpathlineto{\pgfqpoint{3.696675in}{0.535173in}}%
\pgfpathlineto{\pgfqpoint{3.750150in}{0.535173in}}%
\pgfpathlineto{\pgfqpoint{3.803625in}{0.534354in}}%
\pgfpathlineto{\pgfqpoint{3.857100in}{0.534354in}}%
\pgfpathlineto{\pgfqpoint{3.910575in}{0.534354in}}%
\pgfpathlineto{\pgfqpoint{3.964050in}{0.534354in}}%
\pgfpathlineto{\pgfqpoint{4.017525in}{0.534354in}}%
\pgfpathlineto{\pgfqpoint{4.071000in}{0.534354in}}%
\pgfpathlineto{\pgfqpoint{4.124475in}{0.534059in}}%
\pgfpathlineto{\pgfqpoint{4.177950in}{0.534059in}}%
\pgfpathlineto{\pgfqpoint{4.231425in}{0.534059in}}%
\pgfpathlineto{\pgfqpoint{4.284900in}{0.534059in}}%
\pgfpathlineto{\pgfqpoint{4.338375in}{0.534059in}}%
\pgfpathlineto{\pgfqpoint{4.391850in}{0.534059in}}%
\pgfpathlineto{\pgfqpoint{4.445325in}{0.534059in}}%
\pgfpathlineto{\pgfqpoint{4.498800in}{0.534059in}}%
\pgfpathlineto{\pgfqpoint{4.552275in}{0.534059in}}%
\pgfpathlineto{\pgfqpoint{4.605750in}{0.534059in}}%
\pgfpathlineto{\pgfqpoint{4.659225in}{0.534059in}}%
\pgfpathlineto{\pgfqpoint{4.712700in}{0.534059in}}%
\pgfpathlineto{\pgfqpoint{4.766175in}{0.534059in}}%
\pgfpathlineto{\pgfqpoint{4.819650in}{0.509393in}}%
\pgfpathlineto{\pgfqpoint{4.873125in}{0.509393in}}%
\pgfpathlineto{\pgfqpoint{4.926600in}{0.509393in}}%
\pgfpathlineto{\pgfqpoint{4.980075in}{0.509393in}}%
\pgfpathlineto{\pgfqpoint{5.033550in}{0.509393in}}%
\pgfpathlineto{\pgfqpoint{5.087025in}{0.509393in}}%
\pgfpathlineto{\pgfqpoint{5.140500in}{0.509393in}}%
\pgfpathlineto{\pgfqpoint{5.193975in}{0.509393in}}%
\pgfpathlineto{\pgfqpoint{5.247450in}{0.509393in}}%
\pgfpathlineto{\pgfqpoint{5.300925in}{0.509393in}}%
\pgfpathlineto{\pgfqpoint{5.354400in}{0.509393in}}%
\pgfpathlineto{\pgfqpoint{5.407875in}{0.509393in}}%
\pgfpathlineto{\pgfqpoint{5.461350in}{0.509393in}}%
\pgfpathlineto{\pgfqpoint{5.514825in}{0.509393in}}%
\pgfpathlineto{\pgfqpoint{5.568300in}{0.509393in}}%
\pgfpathlineto{\pgfqpoint{5.621775in}{0.509393in}}%
\pgfpathlineto{\pgfqpoint{5.675250in}{0.509393in}}%
\pgfpathlineto{\pgfqpoint{5.728725in}{0.509393in}}%
\pgfpathlineto{\pgfqpoint{5.782200in}{0.509393in}}%
\pgfpathlineto{\pgfqpoint{5.835675in}{0.509393in}}%
\pgfpathlineto{\pgfqpoint{5.889150in}{0.509393in}}%
\pgfpathlineto{\pgfqpoint{5.942625in}{0.509393in}}%
\pgfpathlineto{\pgfqpoint{5.996100in}{0.509393in}}%
\pgfpathlineto{\pgfqpoint{6.049575in}{0.509393in}}%
\pgfpathlineto{\pgfqpoint{6.103050in}{0.509393in}}%
\pgfpathlineto{\pgfqpoint{6.156525in}{0.509393in}}%
\pgfpathlineto{\pgfqpoint{6.210000in}{0.509393in}}%
\pgfpathlineto{\pgfqpoint{6.263475in}{0.509393in}}%
\pgfpathlineto{\pgfqpoint{6.263475in}{0.498674in}}%
\pgfpathlineto{\pgfqpoint{6.263475in}{0.498674in}}%
\pgfpathlineto{\pgfqpoint{6.210000in}{0.498674in}}%
\pgfpathlineto{\pgfqpoint{6.156525in}{0.498674in}}%
\pgfpathlineto{\pgfqpoint{6.103050in}{0.498674in}}%
\pgfpathlineto{\pgfqpoint{6.049575in}{0.498674in}}%
\pgfpathlineto{\pgfqpoint{5.996100in}{0.498674in}}%
\pgfpathlineto{\pgfqpoint{5.942625in}{0.498674in}}%
\pgfpathlineto{\pgfqpoint{5.889150in}{0.498674in}}%
\pgfpathlineto{\pgfqpoint{5.835675in}{0.498674in}}%
\pgfpathlineto{\pgfqpoint{5.782200in}{0.498674in}}%
\pgfpathlineto{\pgfqpoint{5.728725in}{0.498674in}}%
\pgfpathlineto{\pgfqpoint{5.675250in}{0.498674in}}%
\pgfpathlineto{\pgfqpoint{5.621775in}{0.498674in}}%
\pgfpathlineto{\pgfqpoint{5.568300in}{0.498674in}}%
\pgfpathlineto{\pgfqpoint{5.514825in}{0.498674in}}%
\pgfpathlineto{\pgfqpoint{5.461350in}{0.498674in}}%
\pgfpathlineto{\pgfqpoint{5.407875in}{0.498674in}}%
\pgfpathlineto{\pgfqpoint{5.354400in}{0.498674in}}%
\pgfpathlineto{\pgfqpoint{5.300925in}{0.498674in}}%
\pgfpathlineto{\pgfqpoint{5.247450in}{0.498674in}}%
\pgfpathlineto{\pgfqpoint{5.193975in}{0.498674in}}%
\pgfpathlineto{\pgfqpoint{5.140500in}{0.498674in}}%
\pgfpathlineto{\pgfqpoint{5.087025in}{0.498674in}}%
\pgfpathlineto{\pgfqpoint{5.033550in}{0.498674in}}%
\pgfpathlineto{\pgfqpoint{4.980075in}{0.498674in}}%
\pgfpathlineto{\pgfqpoint{4.926600in}{0.498674in}}%
\pgfpathlineto{\pgfqpoint{4.873125in}{0.498674in}}%
\pgfpathlineto{\pgfqpoint{4.819650in}{0.498674in}}%
\pgfpathlineto{\pgfqpoint{4.766175in}{0.506710in}}%
\pgfpathlineto{\pgfqpoint{4.712700in}{0.506710in}}%
\pgfpathlineto{\pgfqpoint{4.659225in}{0.506710in}}%
\pgfpathlineto{\pgfqpoint{4.605750in}{0.506710in}}%
\pgfpathlineto{\pgfqpoint{4.552275in}{0.506710in}}%
\pgfpathlineto{\pgfqpoint{4.498800in}{0.506710in}}%
\pgfpathlineto{\pgfqpoint{4.445325in}{0.506710in}}%
\pgfpathlineto{\pgfqpoint{4.391850in}{0.506710in}}%
\pgfpathlineto{\pgfqpoint{4.338375in}{0.506710in}}%
\pgfpathlineto{\pgfqpoint{4.284900in}{0.506710in}}%
\pgfpathlineto{\pgfqpoint{4.231425in}{0.506710in}}%
\pgfpathlineto{\pgfqpoint{4.177950in}{0.506710in}}%
\pgfpathlineto{\pgfqpoint{4.124475in}{0.506710in}}%
\pgfpathlineto{\pgfqpoint{4.071000in}{0.507088in}}%
\pgfpathlineto{\pgfqpoint{4.017525in}{0.507088in}}%
\pgfpathlineto{\pgfqpoint{3.964050in}{0.507088in}}%
\pgfpathlineto{\pgfqpoint{3.910575in}{0.507088in}}%
\pgfpathlineto{\pgfqpoint{3.857100in}{0.507088in}}%
\pgfpathlineto{\pgfqpoint{3.803625in}{0.507088in}}%
\pgfpathlineto{\pgfqpoint{3.750150in}{0.508091in}}%
\pgfpathlineto{\pgfqpoint{3.696675in}{0.508091in}}%
\pgfpathlineto{\pgfqpoint{3.643200in}{0.508091in}}%
\pgfpathlineto{\pgfqpoint{3.589725in}{0.508091in}}%
\pgfpathlineto{\pgfqpoint{3.536250in}{0.508091in}}%
\pgfpathlineto{\pgfqpoint{3.482775in}{0.508091in}}%
\pgfpathlineto{\pgfqpoint{3.429300in}{0.508091in}}%
\pgfpathlineto{\pgfqpoint{3.375825in}{0.508091in}}%
\pgfpathlineto{\pgfqpoint{3.322350in}{0.508091in}}%
\pgfpathlineto{\pgfqpoint{3.268875in}{0.508091in}}%
\pgfpathlineto{\pgfqpoint{3.215400in}{0.508091in}}%
\pgfpathlineto{\pgfqpoint{3.161925in}{0.508091in}}%
\pgfpathlineto{\pgfqpoint{3.108450in}{0.508091in}}%
\pgfpathlineto{\pgfqpoint{3.054975in}{0.523199in}}%
\pgfpathlineto{\pgfqpoint{3.001500in}{0.523199in}}%
\pgfpathlineto{\pgfqpoint{2.948025in}{0.523199in}}%
\pgfpathlineto{\pgfqpoint{2.894550in}{0.524218in}}%
\pgfpathlineto{\pgfqpoint{2.841075in}{0.524218in}}%
\pgfpathlineto{\pgfqpoint{2.787600in}{0.524218in}}%
\pgfpathlineto{\pgfqpoint{2.734125in}{0.531270in}}%
\pgfpathlineto{\pgfqpoint{2.680650in}{0.531270in}}%
\pgfpathlineto{\pgfqpoint{2.627175in}{0.540533in}}%
\pgfpathlineto{\pgfqpoint{2.573700in}{0.540533in}}%
\pgfpathlineto{\pgfqpoint{2.520225in}{0.540533in}}%
\pgfpathlineto{\pgfqpoint{2.466750in}{0.540533in}}%
\pgfpathlineto{\pgfqpoint{2.413275in}{0.540625in}}%
\pgfpathlineto{\pgfqpoint{2.359800in}{0.542630in}}%
\pgfpathlineto{\pgfqpoint{2.306325in}{0.542630in}}%
\pgfpathlineto{\pgfqpoint{2.252850in}{0.542630in}}%
\pgfpathlineto{\pgfqpoint{2.199375in}{0.542630in}}%
\pgfpathlineto{\pgfqpoint{2.145900in}{0.542630in}}%
\pgfpathlineto{\pgfqpoint{2.092425in}{0.542630in}}%
\pgfpathlineto{\pgfqpoint{2.038950in}{0.542630in}}%
\pgfpathlineto{\pgfqpoint{1.985475in}{0.542630in}}%
\pgfpathlineto{\pgfqpoint{1.932000in}{0.545119in}}%
\pgfpathlineto{\pgfqpoint{1.878525in}{0.545119in}}%
\pgfpathlineto{\pgfqpoint{1.825050in}{0.564758in}}%
\pgfpathlineto{\pgfqpoint{1.771575in}{0.564758in}}%
\pgfpathlineto{\pgfqpoint{1.718100in}{0.564758in}}%
\pgfpathlineto{\pgfqpoint{1.664625in}{0.564758in}}%
\pgfpathlineto{\pgfqpoint{1.611150in}{0.568553in}}%
\pgfpathlineto{\pgfqpoint{1.557675in}{0.570577in}}%
\pgfpathlineto{\pgfqpoint{1.504200in}{0.570577in}}%
\pgfpathlineto{\pgfqpoint{1.450725in}{0.754404in}}%
\pgfpathlineto{\pgfqpoint{1.397250in}{0.754404in}}%
\pgfpathlineto{\pgfqpoint{1.343775in}{0.754404in}}%
\pgfpathlineto{\pgfqpoint{1.290300in}{0.754404in}}%
\pgfpathlineto{\pgfqpoint{1.236825in}{0.814312in}}%
\pgfpathlineto{\pgfqpoint{1.183350in}{0.897893in}}%
\pgfpathlineto{\pgfqpoint{1.129875in}{0.897893in}}%
\pgfpathlineto{\pgfqpoint{1.076400in}{1.561915in}}%
\pgfpathlineto{\pgfqpoint{1.022925in}{1.561915in}}%
\pgfpathlineto{\pgfqpoint{0.969450in}{1.561915in}}%
\pgfpathlineto{\pgfqpoint{0.915975in}{1.561915in}}%
\pgfpathlineto{\pgfqpoint{0.862500in}{1.914711in}}%
\pgfpathclose%
\pgfusepath{fill}%
\end{pgfscope}%
\begin{pgfscope}%
\pgfpathrectangle{\pgfqpoint{0.862500in}{0.375000in}}{\pgfqpoint{5.347500in}{2.265000in}}%
\pgfusepath{clip}%
\pgfsetbuttcap%
\pgfsetroundjoin%
\definecolor{currentfill}{rgb}{0.580392,0.403922,0.741176}%
\pgfsetfillcolor{currentfill}%
\pgfsetfillopacity{0.200000}%
\pgfsetlinewidth{0.000000pt}%
\definecolor{currentstroke}{rgb}{0.000000,0.000000,0.000000}%
\pgfsetstrokecolor{currentstroke}%
\pgfsetdash{}{0pt}%
\pgfpathmoveto{\pgfqpoint{0.862500in}{1.344808in}}%
\pgfpathlineto{\pgfqpoint{0.862500in}{1.536286in}}%
\pgfpathlineto{\pgfqpoint{0.915975in}{1.438102in}}%
\pgfpathlineto{\pgfqpoint{0.969450in}{1.405567in}}%
\pgfpathlineto{\pgfqpoint{1.022925in}{1.303048in}}%
\pgfpathlineto{\pgfqpoint{1.076400in}{1.302697in}}%
\pgfpathlineto{\pgfqpoint{1.129875in}{1.302697in}}%
\pgfpathlineto{\pgfqpoint{1.183350in}{1.301564in}}%
\pgfpathlineto{\pgfqpoint{1.236825in}{1.289625in}}%
\pgfpathlineto{\pgfqpoint{1.290300in}{1.224656in}}%
\pgfpathlineto{\pgfqpoint{1.343775in}{1.224656in}}%
\pgfpathlineto{\pgfqpoint{1.397250in}{1.224656in}}%
\pgfpathlineto{\pgfqpoint{1.450725in}{1.222844in}}%
\pgfpathlineto{\pgfqpoint{1.504200in}{1.222207in}}%
\pgfpathlineto{\pgfqpoint{1.557675in}{0.969312in}}%
\pgfpathlineto{\pgfqpoint{1.611150in}{0.754283in}}%
\pgfpathlineto{\pgfqpoint{1.664625in}{0.681632in}}%
\pgfpathlineto{\pgfqpoint{1.718100in}{0.596707in}}%
\pgfpathlineto{\pgfqpoint{1.771575in}{0.595423in}}%
\pgfpathlineto{\pgfqpoint{1.825050in}{0.594516in}}%
\pgfpathlineto{\pgfqpoint{1.878525in}{0.528049in}}%
\pgfpathlineto{\pgfqpoint{1.932000in}{0.528049in}}%
\pgfpathlineto{\pgfqpoint{1.985475in}{0.528049in}}%
\pgfpathlineto{\pgfqpoint{2.038950in}{0.528049in}}%
\pgfpathlineto{\pgfqpoint{2.092425in}{0.528049in}}%
\pgfpathlineto{\pgfqpoint{2.145900in}{0.528049in}}%
\pgfpathlineto{\pgfqpoint{2.199375in}{0.528049in}}%
\pgfpathlineto{\pgfqpoint{2.252850in}{0.516288in}}%
\pgfpathlineto{\pgfqpoint{2.306325in}{0.516288in}}%
\pgfpathlineto{\pgfqpoint{2.359800in}{0.516288in}}%
\pgfpathlineto{\pgfqpoint{2.413275in}{0.512154in}}%
\pgfpathlineto{\pgfqpoint{2.466750in}{0.512154in}}%
\pgfpathlineto{\pgfqpoint{2.520225in}{0.512154in}}%
\pgfpathlineto{\pgfqpoint{2.573700in}{0.512154in}}%
\pgfpathlineto{\pgfqpoint{2.627175in}{0.512154in}}%
\pgfpathlineto{\pgfqpoint{2.680650in}{0.512154in}}%
\pgfpathlineto{\pgfqpoint{2.734125in}{0.512154in}}%
\pgfpathlineto{\pgfqpoint{2.787600in}{0.512154in}}%
\pgfpathlineto{\pgfqpoint{2.841075in}{0.512154in}}%
\pgfpathlineto{\pgfqpoint{2.894550in}{0.512154in}}%
\pgfpathlineto{\pgfqpoint{2.948025in}{0.512154in}}%
\pgfpathlineto{\pgfqpoint{3.001500in}{0.512154in}}%
\pgfpathlineto{\pgfqpoint{3.054975in}{0.512154in}}%
\pgfpathlineto{\pgfqpoint{3.108450in}{0.512154in}}%
\pgfpathlineto{\pgfqpoint{3.161925in}{0.512154in}}%
\pgfpathlineto{\pgfqpoint{3.215400in}{0.512154in}}%
\pgfpathlineto{\pgfqpoint{3.268875in}{0.512154in}}%
\pgfpathlineto{\pgfqpoint{3.322350in}{0.512154in}}%
\pgfpathlineto{\pgfqpoint{3.375825in}{0.509305in}}%
\pgfpathlineto{\pgfqpoint{3.429300in}{0.509305in}}%
\pgfpathlineto{\pgfqpoint{3.482775in}{0.509305in}}%
\pgfpathlineto{\pgfqpoint{3.536250in}{0.509305in}}%
\pgfpathlineto{\pgfqpoint{3.589725in}{0.509305in}}%
\pgfpathlineto{\pgfqpoint{3.643200in}{0.509305in}}%
\pgfpathlineto{\pgfqpoint{3.696675in}{0.509305in}}%
\pgfpathlineto{\pgfqpoint{3.750150in}{0.509305in}}%
\pgfpathlineto{\pgfqpoint{3.803625in}{0.509305in}}%
\pgfpathlineto{\pgfqpoint{3.857100in}{0.509305in}}%
\pgfpathlineto{\pgfqpoint{3.910575in}{0.509305in}}%
\pgfpathlineto{\pgfqpoint{3.964050in}{0.508483in}}%
\pgfpathlineto{\pgfqpoint{4.017525in}{0.508483in}}%
\pgfpathlineto{\pgfqpoint{4.071000in}{0.508483in}}%
\pgfpathlineto{\pgfqpoint{4.124475in}{0.508483in}}%
\pgfpathlineto{\pgfqpoint{4.177950in}{0.508483in}}%
\pgfpathlineto{\pgfqpoint{4.231425in}{0.508483in}}%
\pgfpathlineto{\pgfqpoint{4.284900in}{0.508365in}}%
\pgfpathlineto{\pgfqpoint{4.338375in}{0.508365in}}%
\pgfpathlineto{\pgfqpoint{4.391850in}{0.508365in}}%
\pgfpathlineto{\pgfqpoint{4.445325in}{0.508365in}}%
\pgfpathlineto{\pgfqpoint{4.498800in}{0.508365in}}%
\pgfpathlineto{\pgfqpoint{4.552275in}{0.508365in}}%
\pgfpathlineto{\pgfqpoint{4.605750in}{0.508365in}}%
\pgfpathlineto{\pgfqpoint{4.659225in}{0.508365in}}%
\pgfpathlineto{\pgfqpoint{4.712700in}{0.507655in}}%
\pgfpathlineto{\pgfqpoint{4.766175in}{0.507655in}}%
\pgfpathlineto{\pgfqpoint{4.819650in}{0.507655in}}%
\pgfpathlineto{\pgfqpoint{4.873125in}{0.506408in}}%
\pgfpathlineto{\pgfqpoint{4.926600in}{0.504935in}}%
\pgfpathlineto{\pgfqpoint{4.980075in}{0.504935in}}%
\pgfpathlineto{\pgfqpoint{5.033550in}{0.504935in}}%
\pgfpathlineto{\pgfqpoint{5.087025in}{0.504835in}}%
\pgfpathlineto{\pgfqpoint{5.140500in}{0.497319in}}%
\pgfpathlineto{\pgfqpoint{5.193975in}{0.497319in}}%
\pgfpathlineto{\pgfqpoint{5.247450in}{0.497319in}}%
\pgfpathlineto{\pgfqpoint{5.300925in}{0.497319in}}%
\pgfpathlineto{\pgfqpoint{5.354400in}{0.497319in}}%
\pgfpathlineto{\pgfqpoint{5.407875in}{0.488469in}}%
\pgfpathlineto{\pgfqpoint{5.461350in}{0.488469in}}%
\pgfpathlineto{\pgfqpoint{5.514825in}{0.488469in}}%
\pgfpathlineto{\pgfqpoint{5.568300in}{0.488469in}}%
\pgfpathlineto{\pgfqpoint{5.621775in}{0.488469in}}%
\pgfpathlineto{\pgfqpoint{5.675250in}{0.488469in}}%
\pgfpathlineto{\pgfqpoint{5.728725in}{0.488469in}}%
\pgfpathlineto{\pgfqpoint{5.782200in}{0.488469in}}%
\pgfpathlineto{\pgfqpoint{5.835675in}{0.488469in}}%
\pgfpathlineto{\pgfqpoint{5.889150in}{0.488469in}}%
\pgfpathlineto{\pgfqpoint{5.942625in}{0.488469in}}%
\pgfpathlineto{\pgfqpoint{5.996100in}{0.488469in}}%
\pgfpathlineto{\pgfqpoint{6.049575in}{0.487386in}}%
\pgfpathlineto{\pgfqpoint{6.103050in}{0.487386in}}%
\pgfpathlineto{\pgfqpoint{6.156525in}{0.487310in}}%
\pgfpathlineto{\pgfqpoint{6.210000in}{0.487310in}}%
\pgfpathlineto{\pgfqpoint{6.263475in}{0.487310in}}%
\pgfpathlineto{\pgfqpoint{6.263475in}{0.479546in}}%
\pgfpathlineto{\pgfqpoint{6.263475in}{0.479546in}}%
\pgfpathlineto{\pgfqpoint{6.210000in}{0.479546in}}%
\pgfpathlineto{\pgfqpoint{6.156525in}{0.479546in}}%
\pgfpathlineto{\pgfqpoint{6.103050in}{0.479649in}}%
\pgfpathlineto{\pgfqpoint{6.049575in}{0.479649in}}%
\pgfpathlineto{\pgfqpoint{5.996100in}{0.482457in}}%
\pgfpathlineto{\pgfqpoint{5.942625in}{0.482457in}}%
\pgfpathlineto{\pgfqpoint{5.889150in}{0.482457in}}%
\pgfpathlineto{\pgfqpoint{5.835675in}{0.482457in}}%
\pgfpathlineto{\pgfqpoint{5.782200in}{0.482457in}}%
\pgfpathlineto{\pgfqpoint{5.728725in}{0.482457in}}%
\pgfpathlineto{\pgfqpoint{5.675250in}{0.482457in}}%
\pgfpathlineto{\pgfqpoint{5.621775in}{0.482457in}}%
\pgfpathlineto{\pgfqpoint{5.568300in}{0.482457in}}%
\pgfpathlineto{\pgfqpoint{5.514825in}{0.482457in}}%
\pgfpathlineto{\pgfqpoint{5.461350in}{0.482457in}}%
\pgfpathlineto{\pgfqpoint{5.407875in}{0.482457in}}%
\pgfpathlineto{\pgfqpoint{5.354400in}{0.488677in}}%
\pgfpathlineto{\pgfqpoint{5.300925in}{0.488677in}}%
\pgfpathlineto{\pgfqpoint{5.247450in}{0.488677in}}%
\pgfpathlineto{\pgfqpoint{5.193975in}{0.488677in}}%
\pgfpathlineto{\pgfqpoint{5.140500in}{0.488677in}}%
\pgfpathlineto{\pgfqpoint{5.087025in}{0.492730in}}%
\pgfpathlineto{\pgfqpoint{5.033550in}{0.493246in}}%
\pgfpathlineto{\pgfqpoint{4.980075in}{0.493246in}}%
\pgfpathlineto{\pgfqpoint{4.926600in}{0.493246in}}%
\pgfpathlineto{\pgfqpoint{4.873125in}{0.494230in}}%
\pgfpathlineto{\pgfqpoint{4.819650in}{0.494951in}}%
\pgfpathlineto{\pgfqpoint{4.766175in}{0.494951in}}%
\pgfpathlineto{\pgfqpoint{4.712700in}{0.494951in}}%
\pgfpathlineto{\pgfqpoint{4.659225in}{0.495325in}}%
\pgfpathlineto{\pgfqpoint{4.605750in}{0.495325in}}%
\pgfpathlineto{\pgfqpoint{4.552275in}{0.495325in}}%
\pgfpathlineto{\pgfqpoint{4.498800in}{0.495325in}}%
\pgfpathlineto{\pgfqpoint{4.445325in}{0.495325in}}%
\pgfpathlineto{\pgfqpoint{4.391850in}{0.495325in}}%
\pgfpathlineto{\pgfqpoint{4.338375in}{0.495325in}}%
\pgfpathlineto{\pgfqpoint{4.284900in}{0.495325in}}%
\pgfpathlineto{\pgfqpoint{4.231425in}{0.495992in}}%
\pgfpathlineto{\pgfqpoint{4.177950in}{0.495992in}}%
\pgfpathlineto{\pgfqpoint{4.124475in}{0.495992in}}%
\pgfpathlineto{\pgfqpoint{4.071000in}{0.495992in}}%
\pgfpathlineto{\pgfqpoint{4.017525in}{0.495992in}}%
\pgfpathlineto{\pgfqpoint{3.964050in}{0.495992in}}%
\pgfpathlineto{\pgfqpoint{3.910575in}{0.500644in}}%
\pgfpathlineto{\pgfqpoint{3.857100in}{0.500644in}}%
\pgfpathlineto{\pgfqpoint{3.803625in}{0.500644in}}%
\pgfpathlineto{\pgfqpoint{3.750150in}{0.500644in}}%
\pgfpathlineto{\pgfqpoint{3.696675in}{0.500644in}}%
\pgfpathlineto{\pgfqpoint{3.643200in}{0.500644in}}%
\pgfpathlineto{\pgfqpoint{3.589725in}{0.500644in}}%
\pgfpathlineto{\pgfqpoint{3.536250in}{0.500644in}}%
\pgfpathlineto{\pgfqpoint{3.482775in}{0.500644in}}%
\pgfpathlineto{\pgfqpoint{3.429300in}{0.500644in}}%
\pgfpathlineto{\pgfqpoint{3.375825in}{0.500644in}}%
\pgfpathlineto{\pgfqpoint{3.322350in}{0.501895in}}%
\pgfpathlineto{\pgfqpoint{3.268875in}{0.501895in}}%
\pgfpathlineto{\pgfqpoint{3.215400in}{0.501895in}}%
\pgfpathlineto{\pgfqpoint{3.161925in}{0.501895in}}%
\pgfpathlineto{\pgfqpoint{3.108450in}{0.501895in}}%
\pgfpathlineto{\pgfqpoint{3.054975in}{0.501895in}}%
\pgfpathlineto{\pgfqpoint{3.001500in}{0.501895in}}%
\pgfpathlineto{\pgfqpoint{2.948025in}{0.501895in}}%
\pgfpathlineto{\pgfqpoint{2.894550in}{0.501895in}}%
\pgfpathlineto{\pgfqpoint{2.841075in}{0.501895in}}%
\pgfpathlineto{\pgfqpoint{2.787600in}{0.501895in}}%
\pgfpathlineto{\pgfqpoint{2.734125in}{0.501895in}}%
\pgfpathlineto{\pgfqpoint{2.680650in}{0.501895in}}%
\pgfpathlineto{\pgfqpoint{2.627175in}{0.501895in}}%
\pgfpathlineto{\pgfqpoint{2.573700in}{0.501895in}}%
\pgfpathlineto{\pgfqpoint{2.520225in}{0.501895in}}%
\pgfpathlineto{\pgfqpoint{2.466750in}{0.501895in}}%
\pgfpathlineto{\pgfqpoint{2.413275in}{0.501895in}}%
\pgfpathlineto{\pgfqpoint{2.359800in}{0.503304in}}%
\pgfpathlineto{\pgfqpoint{2.306325in}{0.503304in}}%
\pgfpathlineto{\pgfqpoint{2.252850in}{0.503304in}}%
\pgfpathlineto{\pgfqpoint{2.199375in}{0.506472in}}%
\pgfpathlineto{\pgfqpoint{2.145900in}{0.506472in}}%
\pgfpathlineto{\pgfqpoint{2.092425in}{0.506472in}}%
\pgfpathlineto{\pgfqpoint{2.038950in}{0.506472in}}%
\pgfpathlineto{\pgfqpoint{1.985475in}{0.506472in}}%
\pgfpathlineto{\pgfqpoint{1.932000in}{0.506472in}}%
\pgfpathlineto{\pgfqpoint{1.878525in}{0.506472in}}%
\pgfpathlineto{\pgfqpoint{1.825050in}{0.522004in}}%
\pgfpathlineto{\pgfqpoint{1.771575in}{0.524265in}}%
\pgfpathlineto{\pgfqpoint{1.718100in}{0.527336in}}%
\pgfpathlineto{\pgfqpoint{1.664625in}{0.547771in}}%
\pgfpathlineto{\pgfqpoint{1.611150in}{0.641660in}}%
\pgfpathlineto{\pgfqpoint{1.557675in}{0.717093in}}%
\pgfpathlineto{\pgfqpoint{1.504200in}{0.813785in}}%
\pgfpathlineto{\pgfqpoint{1.450725in}{0.816451in}}%
\pgfpathlineto{\pgfqpoint{1.397250in}{0.823924in}}%
\pgfpathlineto{\pgfqpoint{1.343775in}{0.823924in}}%
\pgfpathlineto{\pgfqpoint{1.290300in}{0.823924in}}%
\pgfpathlineto{\pgfqpoint{1.236825in}{0.847317in}}%
\pgfpathlineto{\pgfqpoint{1.183350in}{0.890906in}}%
\pgfpathlineto{\pgfqpoint{1.129875in}{0.895250in}}%
\pgfpathlineto{\pgfqpoint{1.076400in}{0.895250in}}%
\pgfpathlineto{\pgfqpoint{1.022925in}{0.896586in}}%
\pgfpathlineto{\pgfqpoint{0.969450in}{1.074535in}}%
\pgfpathlineto{\pgfqpoint{0.915975in}{1.181097in}}%
\pgfpathlineto{\pgfqpoint{0.862500in}{1.344808in}}%
\pgfpathclose%
\pgfusepath{fill}%
\end{pgfscope}%
\begin{pgfscope}%
\pgfpathrectangle{\pgfqpoint{0.862500in}{0.375000in}}{\pgfqpoint{5.347500in}{2.265000in}}%
\pgfusepath{clip}%
\pgfsetbuttcap%
\pgfsetroundjoin%
\definecolor{currentfill}{rgb}{0.549020,0.337255,0.294118}%
\pgfsetfillcolor{currentfill}%
\pgfsetfillopacity{0.200000}%
\pgfsetlinewidth{0.000000pt}%
\definecolor{currentstroke}{rgb}{0.000000,0.000000,0.000000}%
\pgfsetstrokecolor{currentstroke}%
\pgfsetdash{}{0pt}%
\pgfpathmoveto{\pgfqpoint{0.862500in}{1.550104in}}%
\pgfpathlineto{\pgfqpoint{0.862500in}{1.585315in}}%
\pgfpathlineto{\pgfqpoint{0.915975in}{1.570327in}}%
\pgfpathlineto{\pgfqpoint{0.969450in}{1.527579in}}%
\pgfpathlineto{\pgfqpoint{1.022925in}{1.499624in}}%
\pgfpathlineto{\pgfqpoint{1.076400in}{1.439174in}}%
\pgfpathlineto{\pgfqpoint{1.129875in}{1.439174in}}%
\pgfpathlineto{\pgfqpoint{1.183350in}{1.439174in}}%
\pgfpathlineto{\pgfqpoint{1.236825in}{1.357836in}}%
\pgfpathlineto{\pgfqpoint{1.290300in}{1.199410in}}%
\pgfpathlineto{\pgfqpoint{1.343775in}{1.199410in}}%
\pgfpathlineto{\pgfqpoint{1.397250in}{0.802457in}}%
\pgfpathlineto{\pgfqpoint{1.450725in}{0.718455in}}%
\pgfpathlineto{\pgfqpoint{1.504200in}{0.718455in}}%
\pgfpathlineto{\pgfqpoint{1.557675in}{0.564697in}}%
\pgfpathlineto{\pgfqpoint{1.611150in}{0.558743in}}%
\pgfpathlineto{\pgfqpoint{1.664625in}{0.558743in}}%
\pgfpathlineto{\pgfqpoint{1.718100in}{0.507550in}}%
\pgfpathlineto{\pgfqpoint{1.771575in}{0.507550in}}%
\pgfpathlineto{\pgfqpoint{1.825050in}{0.507550in}}%
\pgfpathlineto{\pgfqpoint{1.878525in}{0.507550in}}%
\pgfpathlineto{\pgfqpoint{1.932000in}{0.507550in}}%
\pgfpathlineto{\pgfqpoint{1.985475in}{0.507550in}}%
\pgfpathlineto{\pgfqpoint{2.038950in}{0.507550in}}%
\pgfpathlineto{\pgfqpoint{2.092425in}{0.507550in}}%
\pgfpathlineto{\pgfqpoint{2.145900in}{0.507550in}}%
\pgfpathlineto{\pgfqpoint{2.199375in}{0.507550in}}%
\pgfpathlineto{\pgfqpoint{2.252850in}{0.507550in}}%
\pgfpathlineto{\pgfqpoint{2.306325in}{0.507550in}}%
\pgfpathlineto{\pgfqpoint{2.359800in}{0.507550in}}%
\pgfpathlineto{\pgfqpoint{2.413275in}{0.507550in}}%
\pgfpathlineto{\pgfqpoint{2.466750in}{0.507550in}}%
\pgfpathlineto{\pgfqpoint{2.520225in}{0.506669in}}%
\pgfpathlineto{\pgfqpoint{2.573700in}{0.506669in}}%
\pgfpathlineto{\pgfqpoint{2.627175in}{0.506669in}}%
\pgfpathlineto{\pgfqpoint{2.680650in}{0.506669in}}%
\pgfpathlineto{\pgfqpoint{2.734125in}{0.506669in}}%
\pgfpathlineto{\pgfqpoint{2.787600in}{0.506669in}}%
\pgfpathlineto{\pgfqpoint{2.841075in}{0.506669in}}%
\pgfpathlineto{\pgfqpoint{2.894550in}{0.506669in}}%
\pgfpathlineto{\pgfqpoint{2.948025in}{0.506669in}}%
\pgfpathlineto{\pgfqpoint{3.001500in}{0.506669in}}%
\pgfpathlineto{\pgfqpoint{3.054975in}{0.506669in}}%
\pgfpathlineto{\pgfqpoint{3.108450in}{0.506669in}}%
\pgfpathlineto{\pgfqpoint{3.161925in}{0.499268in}}%
\pgfpathlineto{\pgfqpoint{3.215400in}{0.499268in}}%
\pgfpathlineto{\pgfqpoint{3.268875in}{0.492682in}}%
\pgfpathlineto{\pgfqpoint{3.322350in}{0.492682in}}%
\pgfpathlineto{\pgfqpoint{3.375825in}{0.492682in}}%
\pgfpathlineto{\pgfqpoint{3.429300in}{0.492682in}}%
\pgfpathlineto{\pgfqpoint{3.482775in}{0.492682in}}%
\pgfpathlineto{\pgfqpoint{3.536250in}{0.492682in}}%
\pgfpathlineto{\pgfqpoint{3.589725in}{0.492682in}}%
\pgfpathlineto{\pgfqpoint{3.643200in}{0.492682in}}%
\pgfpathlineto{\pgfqpoint{3.696675in}{0.492682in}}%
\pgfpathlineto{\pgfqpoint{3.750150in}{0.492682in}}%
\pgfpathlineto{\pgfqpoint{3.803625in}{0.492682in}}%
\pgfpathlineto{\pgfqpoint{3.857100in}{0.492682in}}%
\pgfpathlineto{\pgfqpoint{3.910575in}{0.492682in}}%
\pgfpathlineto{\pgfqpoint{3.964050in}{0.492682in}}%
\pgfpathlineto{\pgfqpoint{4.017525in}{0.492682in}}%
\pgfpathlineto{\pgfqpoint{4.071000in}{0.492682in}}%
\pgfpathlineto{\pgfqpoint{4.124475in}{0.492682in}}%
\pgfpathlineto{\pgfqpoint{4.177950in}{0.492682in}}%
\pgfpathlineto{\pgfqpoint{4.231425in}{0.492682in}}%
\pgfpathlineto{\pgfqpoint{4.284900in}{0.492682in}}%
\pgfpathlineto{\pgfqpoint{4.338375in}{0.488920in}}%
\pgfpathlineto{\pgfqpoint{4.391850in}{0.488920in}}%
\pgfpathlineto{\pgfqpoint{4.445325in}{0.488920in}}%
\pgfpathlineto{\pgfqpoint{4.498800in}{0.488920in}}%
\pgfpathlineto{\pgfqpoint{4.552275in}{0.488920in}}%
\pgfpathlineto{\pgfqpoint{4.605750in}{0.488920in}}%
\pgfpathlineto{\pgfqpoint{4.659225in}{0.488920in}}%
\pgfpathlineto{\pgfqpoint{4.712700in}{0.488920in}}%
\pgfpathlineto{\pgfqpoint{4.766175in}{0.488920in}}%
\pgfpathlineto{\pgfqpoint{4.819650in}{0.488920in}}%
\pgfpathlineto{\pgfqpoint{4.873125in}{0.488920in}}%
\pgfpathlineto{\pgfqpoint{4.926600in}{0.487144in}}%
\pgfpathlineto{\pgfqpoint{4.980075in}{0.487144in}}%
\pgfpathlineto{\pgfqpoint{5.033550in}{0.486430in}}%
\pgfpathlineto{\pgfqpoint{5.087025in}{0.480637in}}%
\pgfpathlineto{\pgfqpoint{5.140500in}{0.480637in}}%
\pgfpathlineto{\pgfqpoint{5.193975in}{0.480637in}}%
\pgfpathlineto{\pgfqpoint{5.247450in}{0.480637in}}%
\pgfpathlineto{\pgfqpoint{5.300925in}{0.480637in}}%
\pgfpathlineto{\pgfqpoint{5.354400in}{0.480637in}}%
\pgfpathlineto{\pgfqpoint{5.407875in}{0.480637in}}%
\pgfpathlineto{\pgfqpoint{5.461350in}{0.479780in}}%
\pgfpathlineto{\pgfqpoint{5.514825in}{0.479780in}}%
\pgfpathlineto{\pgfqpoint{5.568300in}{0.479780in}}%
\pgfpathlineto{\pgfqpoint{5.621775in}{0.479780in}}%
\pgfpathlineto{\pgfqpoint{5.675250in}{0.479780in}}%
\pgfpathlineto{\pgfqpoint{5.728725in}{0.479780in}}%
\pgfpathlineto{\pgfqpoint{5.782200in}{0.479780in}}%
\pgfpathlineto{\pgfqpoint{5.835675in}{0.479780in}}%
\pgfpathlineto{\pgfqpoint{5.889150in}{0.479780in}}%
\pgfpathlineto{\pgfqpoint{5.942625in}{0.479780in}}%
\pgfpathlineto{\pgfqpoint{5.996100in}{0.479780in}}%
\pgfpathlineto{\pgfqpoint{6.049575in}{0.479780in}}%
\pgfpathlineto{\pgfqpoint{6.103050in}{0.479780in}}%
\pgfpathlineto{\pgfqpoint{6.156525in}{0.479780in}}%
\pgfpathlineto{\pgfqpoint{6.210000in}{0.479780in}}%
\pgfpathlineto{\pgfqpoint{6.263475in}{0.479780in}}%
\pgfpathlineto{\pgfqpoint{6.263475in}{0.477955in}}%
\pgfpathlineto{\pgfqpoint{6.263475in}{0.477955in}}%
\pgfpathlineto{\pgfqpoint{6.210000in}{0.477955in}}%
\pgfpathlineto{\pgfqpoint{6.156525in}{0.477955in}}%
\pgfpathlineto{\pgfqpoint{6.103050in}{0.477955in}}%
\pgfpathlineto{\pgfqpoint{6.049575in}{0.477955in}}%
\pgfpathlineto{\pgfqpoint{5.996100in}{0.477955in}}%
\pgfpathlineto{\pgfqpoint{5.942625in}{0.477955in}}%
\pgfpathlineto{\pgfqpoint{5.889150in}{0.477955in}}%
\pgfpathlineto{\pgfqpoint{5.835675in}{0.477955in}}%
\pgfpathlineto{\pgfqpoint{5.782200in}{0.477955in}}%
\pgfpathlineto{\pgfqpoint{5.728725in}{0.477955in}}%
\pgfpathlineto{\pgfqpoint{5.675250in}{0.477955in}}%
\pgfpathlineto{\pgfqpoint{5.621775in}{0.477955in}}%
\pgfpathlineto{\pgfqpoint{5.568300in}{0.477955in}}%
\pgfpathlineto{\pgfqpoint{5.514825in}{0.477955in}}%
\pgfpathlineto{\pgfqpoint{5.461350in}{0.477955in}}%
\pgfpathlineto{\pgfqpoint{5.407875in}{0.478270in}}%
\pgfpathlineto{\pgfqpoint{5.354400in}{0.478270in}}%
\pgfpathlineto{\pgfqpoint{5.300925in}{0.478270in}}%
\pgfpathlineto{\pgfqpoint{5.247450in}{0.478270in}}%
\pgfpathlineto{\pgfqpoint{5.193975in}{0.478270in}}%
\pgfpathlineto{\pgfqpoint{5.140500in}{0.478270in}}%
\pgfpathlineto{\pgfqpoint{5.087025in}{0.478270in}}%
\pgfpathlineto{\pgfqpoint{5.033550in}{0.480405in}}%
\pgfpathlineto{\pgfqpoint{4.980075in}{0.480668in}}%
\pgfpathlineto{\pgfqpoint{4.926600in}{0.480668in}}%
\pgfpathlineto{\pgfqpoint{4.873125in}{0.481325in}}%
\pgfpathlineto{\pgfqpoint{4.819650in}{0.481325in}}%
\pgfpathlineto{\pgfqpoint{4.766175in}{0.481325in}}%
\pgfpathlineto{\pgfqpoint{4.712700in}{0.481325in}}%
\pgfpathlineto{\pgfqpoint{4.659225in}{0.481325in}}%
\pgfpathlineto{\pgfqpoint{4.605750in}{0.481325in}}%
\pgfpathlineto{\pgfqpoint{4.552275in}{0.481325in}}%
\pgfpathlineto{\pgfqpoint{4.498800in}{0.481325in}}%
\pgfpathlineto{\pgfqpoint{4.445325in}{0.481325in}}%
\pgfpathlineto{\pgfqpoint{4.391850in}{0.481325in}}%
\pgfpathlineto{\pgfqpoint{4.338375in}{0.481325in}}%
\pgfpathlineto{\pgfqpoint{4.284900in}{0.482719in}}%
\pgfpathlineto{\pgfqpoint{4.231425in}{0.482719in}}%
\pgfpathlineto{\pgfqpoint{4.177950in}{0.482719in}}%
\pgfpathlineto{\pgfqpoint{4.124475in}{0.482719in}}%
\pgfpathlineto{\pgfqpoint{4.071000in}{0.482719in}}%
\pgfpathlineto{\pgfqpoint{4.017525in}{0.482719in}}%
\pgfpathlineto{\pgfqpoint{3.964050in}{0.482719in}}%
\pgfpathlineto{\pgfqpoint{3.910575in}{0.482719in}}%
\pgfpathlineto{\pgfqpoint{3.857100in}{0.482719in}}%
\pgfpathlineto{\pgfqpoint{3.803625in}{0.482719in}}%
\pgfpathlineto{\pgfqpoint{3.750150in}{0.482719in}}%
\pgfpathlineto{\pgfqpoint{3.696675in}{0.482719in}}%
\pgfpathlineto{\pgfqpoint{3.643200in}{0.482719in}}%
\pgfpathlineto{\pgfqpoint{3.589725in}{0.482719in}}%
\pgfpathlineto{\pgfqpoint{3.536250in}{0.482719in}}%
\pgfpathlineto{\pgfqpoint{3.482775in}{0.482719in}}%
\pgfpathlineto{\pgfqpoint{3.429300in}{0.482719in}}%
\pgfpathlineto{\pgfqpoint{3.375825in}{0.482719in}}%
\pgfpathlineto{\pgfqpoint{3.322350in}{0.482719in}}%
\pgfpathlineto{\pgfqpoint{3.268875in}{0.482719in}}%
\pgfpathlineto{\pgfqpoint{3.215400in}{0.498388in}}%
\pgfpathlineto{\pgfqpoint{3.161925in}{0.498388in}}%
\pgfpathlineto{\pgfqpoint{3.108450in}{0.502649in}}%
\pgfpathlineto{\pgfqpoint{3.054975in}{0.502649in}}%
\pgfpathlineto{\pgfqpoint{3.001500in}{0.502649in}}%
\pgfpathlineto{\pgfqpoint{2.948025in}{0.502649in}}%
\pgfpathlineto{\pgfqpoint{2.894550in}{0.502649in}}%
\pgfpathlineto{\pgfqpoint{2.841075in}{0.502649in}}%
\pgfpathlineto{\pgfqpoint{2.787600in}{0.502649in}}%
\pgfpathlineto{\pgfqpoint{2.734125in}{0.502649in}}%
\pgfpathlineto{\pgfqpoint{2.680650in}{0.502649in}}%
\pgfpathlineto{\pgfqpoint{2.627175in}{0.502649in}}%
\pgfpathlineto{\pgfqpoint{2.573700in}{0.502649in}}%
\pgfpathlineto{\pgfqpoint{2.520225in}{0.502649in}}%
\pgfpathlineto{\pgfqpoint{2.466750in}{0.505213in}}%
\pgfpathlineto{\pgfqpoint{2.413275in}{0.505213in}}%
\pgfpathlineto{\pgfqpoint{2.359800in}{0.505213in}}%
\pgfpathlineto{\pgfqpoint{2.306325in}{0.505213in}}%
\pgfpathlineto{\pgfqpoint{2.252850in}{0.505213in}}%
\pgfpathlineto{\pgfqpoint{2.199375in}{0.505213in}}%
\pgfpathlineto{\pgfqpoint{2.145900in}{0.505213in}}%
\pgfpathlineto{\pgfqpoint{2.092425in}{0.505213in}}%
\pgfpathlineto{\pgfqpoint{2.038950in}{0.505213in}}%
\pgfpathlineto{\pgfqpoint{1.985475in}{0.505213in}}%
\pgfpathlineto{\pgfqpoint{1.932000in}{0.505213in}}%
\pgfpathlineto{\pgfqpoint{1.878525in}{0.505213in}}%
\pgfpathlineto{\pgfqpoint{1.825050in}{0.505213in}}%
\pgfpathlineto{\pgfqpoint{1.771575in}{0.505213in}}%
\pgfpathlineto{\pgfqpoint{1.718100in}{0.505213in}}%
\pgfpathlineto{\pgfqpoint{1.664625in}{0.524400in}}%
\pgfpathlineto{\pgfqpoint{1.611150in}{0.524400in}}%
\pgfpathlineto{\pgfqpoint{1.557675in}{0.542259in}}%
\pgfpathlineto{\pgfqpoint{1.504200in}{0.631235in}}%
\pgfpathlineto{\pgfqpoint{1.450725in}{0.631235in}}%
\pgfpathlineto{\pgfqpoint{1.397250in}{0.776419in}}%
\pgfpathlineto{\pgfqpoint{1.343775in}{0.949234in}}%
\pgfpathlineto{\pgfqpoint{1.290300in}{0.949234in}}%
\pgfpathlineto{\pgfqpoint{1.236825in}{1.061201in}}%
\pgfpathlineto{\pgfqpoint{1.183350in}{1.355526in}}%
\pgfpathlineto{\pgfqpoint{1.129875in}{1.355526in}}%
\pgfpathlineto{\pgfqpoint{1.076400in}{1.355526in}}%
\pgfpathlineto{\pgfqpoint{1.022925in}{1.487293in}}%
\pgfpathlineto{\pgfqpoint{0.969450in}{1.514003in}}%
\pgfpathlineto{\pgfqpoint{0.915975in}{1.544405in}}%
\pgfpathlineto{\pgfqpoint{0.862500in}{1.550104in}}%
\pgfpathclose%
\pgfusepath{fill}%
\end{pgfscope}%
\begin{pgfscope}%
\pgfpathrectangle{\pgfqpoint{0.862500in}{0.375000in}}{\pgfqpoint{5.347500in}{2.265000in}}%
\pgfusepath{clip}%
\pgfsetbuttcap%
\pgfsetroundjoin%
\definecolor{currentfill}{rgb}{0.890196,0.466667,0.760784}%
\pgfsetfillcolor{currentfill}%
\pgfsetfillopacity{0.200000}%
\pgfsetlinewidth{0.000000pt}%
\definecolor{currentstroke}{rgb}{0.000000,0.000000,0.000000}%
\pgfsetstrokecolor{currentstroke}%
\pgfsetdash{}{0pt}%
\pgfpathmoveto{\pgfqpoint{0.862500in}{1.546069in}}%
\pgfpathlineto{\pgfqpoint{0.862500in}{1.991709in}}%
\pgfpathlineto{\pgfqpoint{0.915975in}{1.723452in}}%
\pgfpathlineto{\pgfqpoint{0.969450in}{1.397917in}}%
\pgfpathlineto{\pgfqpoint{1.022925in}{1.390744in}}%
\pgfpathlineto{\pgfqpoint{1.076400in}{1.390744in}}%
\pgfpathlineto{\pgfqpoint{1.129875in}{1.296346in}}%
\pgfpathlineto{\pgfqpoint{1.183350in}{1.296346in}}%
\pgfpathlineto{\pgfqpoint{1.236825in}{1.296346in}}%
\pgfpathlineto{\pgfqpoint{1.290300in}{1.296346in}}%
\pgfpathlineto{\pgfqpoint{1.343775in}{1.296346in}}%
\pgfpathlineto{\pgfqpoint{1.397250in}{1.296346in}}%
\pgfpathlineto{\pgfqpoint{1.450725in}{1.296346in}}%
\pgfpathlineto{\pgfqpoint{1.504200in}{1.255021in}}%
\pgfpathlineto{\pgfqpoint{1.557675in}{1.191448in}}%
\pgfpathlineto{\pgfqpoint{1.611150in}{1.163246in}}%
\pgfpathlineto{\pgfqpoint{1.664625in}{1.163246in}}%
\pgfpathlineto{\pgfqpoint{1.718100in}{1.163246in}}%
\pgfpathlineto{\pgfqpoint{1.771575in}{1.163246in}}%
\pgfpathlineto{\pgfqpoint{1.825050in}{1.163246in}}%
\pgfpathlineto{\pgfqpoint{1.878525in}{1.163246in}}%
\pgfpathlineto{\pgfqpoint{1.932000in}{1.163246in}}%
\pgfpathlineto{\pgfqpoint{1.985475in}{1.163246in}}%
\pgfpathlineto{\pgfqpoint{2.038950in}{1.163246in}}%
\pgfpathlineto{\pgfqpoint{2.092425in}{1.163246in}}%
\pgfpathlineto{\pgfqpoint{2.145900in}{1.163246in}}%
\pgfpathlineto{\pgfqpoint{2.199375in}{1.163246in}}%
\pgfpathlineto{\pgfqpoint{2.252850in}{1.163246in}}%
\pgfpathlineto{\pgfqpoint{2.306325in}{1.163246in}}%
\pgfpathlineto{\pgfqpoint{2.359800in}{1.163246in}}%
\pgfpathlineto{\pgfqpoint{2.413275in}{1.163246in}}%
\pgfpathlineto{\pgfqpoint{2.466750in}{1.163246in}}%
\pgfpathlineto{\pgfqpoint{2.520225in}{1.122342in}}%
\pgfpathlineto{\pgfqpoint{2.573700in}{1.122342in}}%
\pgfpathlineto{\pgfqpoint{2.627175in}{1.122342in}}%
\pgfpathlineto{\pgfqpoint{2.680650in}{1.122342in}}%
\pgfpathlineto{\pgfqpoint{2.734125in}{1.122342in}}%
\pgfpathlineto{\pgfqpoint{2.787600in}{1.033403in}}%
\pgfpathlineto{\pgfqpoint{2.841075in}{1.033403in}}%
\pgfpathlineto{\pgfqpoint{2.894550in}{1.033403in}}%
\pgfpathlineto{\pgfqpoint{2.948025in}{1.033403in}}%
\pgfpathlineto{\pgfqpoint{3.001500in}{1.033403in}}%
\pgfpathlineto{\pgfqpoint{3.054975in}{0.974912in}}%
\pgfpathlineto{\pgfqpoint{3.108450in}{0.974912in}}%
\pgfpathlineto{\pgfqpoint{3.161925in}{0.974912in}}%
\pgfpathlineto{\pgfqpoint{3.215400in}{0.974912in}}%
\pgfpathlineto{\pgfqpoint{3.268875in}{0.974912in}}%
\pgfpathlineto{\pgfqpoint{3.322350in}{0.974912in}}%
\pgfpathlineto{\pgfqpoint{3.375825in}{0.974912in}}%
\pgfpathlineto{\pgfqpoint{3.429300in}{0.974912in}}%
\pgfpathlineto{\pgfqpoint{3.482775in}{0.974912in}}%
\pgfpathlineto{\pgfqpoint{3.536250in}{0.974912in}}%
\pgfpathlineto{\pgfqpoint{3.589725in}{0.974912in}}%
\pgfpathlineto{\pgfqpoint{3.643200in}{0.974912in}}%
\pgfpathlineto{\pgfqpoint{3.696675in}{0.974912in}}%
\pgfpathlineto{\pgfqpoint{3.750150in}{0.974912in}}%
\pgfpathlineto{\pgfqpoint{3.803625in}{0.974912in}}%
\pgfpathlineto{\pgfqpoint{3.857100in}{0.974912in}}%
\pgfpathlineto{\pgfqpoint{3.910575in}{0.974912in}}%
\pgfpathlineto{\pgfqpoint{3.964050in}{0.974912in}}%
\pgfpathlineto{\pgfqpoint{4.017525in}{0.822558in}}%
\pgfpathlineto{\pgfqpoint{4.071000in}{0.822558in}}%
\pgfpathlineto{\pgfqpoint{4.124475in}{0.822558in}}%
\pgfpathlineto{\pgfqpoint{4.177950in}{0.822558in}}%
\pgfpathlineto{\pgfqpoint{4.231425in}{0.822558in}}%
\pgfpathlineto{\pgfqpoint{4.284900in}{0.822558in}}%
\pgfpathlineto{\pgfqpoint{4.338375in}{0.822558in}}%
\pgfpathlineto{\pgfqpoint{4.391850in}{0.822558in}}%
\pgfpathlineto{\pgfqpoint{4.445325in}{0.822558in}}%
\pgfpathlineto{\pgfqpoint{4.498800in}{0.822558in}}%
\pgfpathlineto{\pgfqpoint{4.552275in}{0.822558in}}%
\pgfpathlineto{\pgfqpoint{4.605750in}{0.822558in}}%
\pgfpathlineto{\pgfqpoint{4.659225in}{0.822558in}}%
\pgfpathlineto{\pgfqpoint{4.712700in}{0.822558in}}%
\pgfpathlineto{\pgfqpoint{4.766175in}{0.822558in}}%
\pgfpathlineto{\pgfqpoint{4.819650in}{0.822558in}}%
\pgfpathlineto{\pgfqpoint{4.873125in}{0.822558in}}%
\pgfpathlineto{\pgfqpoint{4.926600in}{0.822558in}}%
\pgfpathlineto{\pgfqpoint{4.980075in}{0.822558in}}%
\pgfpathlineto{\pgfqpoint{5.033550in}{0.822558in}}%
\pgfpathlineto{\pgfqpoint{5.087025in}{0.822558in}}%
\pgfpathlineto{\pgfqpoint{5.140500in}{0.822558in}}%
\pgfpathlineto{\pgfqpoint{5.193975in}{0.822558in}}%
\pgfpathlineto{\pgfqpoint{5.247450in}{0.822558in}}%
\pgfpathlineto{\pgfqpoint{5.300925in}{0.822558in}}%
\pgfpathlineto{\pgfqpoint{5.354400in}{0.822558in}}%
\pgfpathlineto{\pgfqpoint{5.407875in}{0.822558in}}%
\pgfpathlineto{\pgfqpoint{5.461350in}{0.822558in}}%
\pgfpathlineto{\pgfqpoint{5.514825in}{0.822558in}}%
\pgfpathlineto{\pgfqpoint{5.568300in}{0.822558in}}%
\pgfpathlineto{\pgfqpoint{5.621775in}{0.822558in}}%
\pgfpathlineto{\pgfqpoint{5.675250in}{0.822558in}}%
\pgfpathlineto{\pgfqpoint{5.728725in}{0.822558in}}%
\pgfpathlineto{\pgfqpoint{5.782200in}{0.822558in}}%
\pgfpathlineto{\pgfqpoint{5.835675in}{0.822558in}}%
\pgfpathlineto{\pgfqpoint{5.889150in}{0.822558in}}%
\pgfpathlineto{\pgfqpoint{5.942625in}{0.822558in}}%
\pgfpathlineto{\pgfqpoint{5.996100in}{0.822558in}}%
\pgfpathlineto{\pgfqpoint{6.049575in}{0.822558in}}%
\pgfpathlineto{\pgfqpoint{6.103050in}{0.822558in}}%
\pgfpathlineto{\pgfqpoint{6.156525in}{0.822558in}}%
\pgfpathlineto{\pgfqpoint{6.210000in}{0.822558in}}%
\pgfpathlineto{\pgfqpoint{6.263475in}{0.822558in}}%
\pgfpathlineto{\pgfqpoint{6.316950in}{0.822558in}}%
\pgfpathlineto{\pgfqpoint{6.370425in}{0.822558in}}%
\pgfpathlineto{\pgfqpoint{6.423900in}{0.822558in}}%
\pgfpathlineto{\pgfqpoint{6.477375in}{0.822558in}}%
\pgfpathlineto{\pgfqpoint{6.530850in}{0.822558in}}%
\pgfpathlineto{\pgfqpoint{6.584325in}{0.822558in}}%
\pgfpathlineto{\pgfqpoint{6.637800in}{0.822558in}}%
\pgfpathlineto{\pgfqpoint{6.691275in}{0.822558in}}%
\pgfpathlineto{\pgfqpoint{6.744750in}{0.822558in}}%
\pgfpathlineto{\pgfqpoint{6.798225in}{0.822558in}}%
\pgfpathlineto{\pgfqpoint{6.851700in}{0.822558in}}%
\pgfpathlineto{\pgfqpoint{6.905175in}{0.822558in}}%
\pgfpathlineto{\pgfqpoint{6.958650in}{0.822558in}}%
\pgfpathlineto{\pgfqpoint{7.012125in}{0.822558in}}%
\pgfpathlineto{\pgfqpoint{7.065600in}{0.822558in}}%
\pgfpathlineto{\pgfqpoint{7.119075in}{0.822558in}}%
\pgfpathlineto{\pgfqpoint{7.172550in}{0.822558in}}%
\pgfpathlineto{\pgfqpoint{7.226025in}{0.822558in}}%
\pgfpathlineto{\pgfqpoint{7.226025in}{0.759589in}}%
\pgfpathlineto{\pgfqpoint{7.226025in}{0.759589in}}%
\pgfpathlineto{\pgfqpoint{7.172550in}{0.759589in}}%
\pgfpathlineto{\pgfqpoint{7.119075in}{0.759589in}}%
\pgfpathlineto{\pgfqpoint{7.065600in}{0.759589in}}%
\pgfpathlineto{\pgfqpoint{7.012125in}{0.759589in}}%
\pgfpathlineto{\pgfqpoint{6.958650in}{0.759589in}}%
\pgfpathlineto{\pgfqpoint{6.905175in}{0.759589in}}%
\pgfpathlineto{\pgfqpoint{6.851700in}{0.759589in}}%
\pgfpathlineto{\pgfqpoint{6.798225in}{0.759589in}}%
\pgfpathlineto{\pgfqpoint{6.744750in}{0.759589in}}%
\pgfpathlineto{\pgfqpoint{6.691275in}{0.759589in}}%
\pgfpathlineto{\pgfqpoint{6.637800in}{0.759589in}}%
\pgfpathlineto{\pgfqpoint{6.584325in}{0.759589in}}%
\pgfpathlineto{\pgfqpoint{6.530850in}{0.759589in}}%
\pgfpathlineto{\pgfqpoint{6.477375in}{0.759589in}}%
\pgfpathlineto{\pgfqpoint{6.423900in}{0.759589in}}%
\pgfpathlineto{\pgfqpoint{6.370425in}{0.759589in}}%
\pgfpathlineto{\pgfqpoint{6.316950in}{0.759589in}}%
\pgfpathlineto{\pgfqpoint{6.263475in}{0.759589in}}%
\pgfpathlineto{\pgfqpoint{6.210000in}{0.759589in}}%
\pgfpathlineto{\pgfqpoint{6.156525in}{0.759589in}}%
\pgfpathlineto{\pgfqpoint{6.103050in}{0.759589in}}%
\pgfpathlineto{\pgfqpoint{6.049575in}{0.759589in}}%
\pgfpathlineto{\pgfqpoint{5.996100in}{0.759589in}}%
\pgfpathlineto{\pgfqpoint{5.942625in}{0.759589in}}%
\pgfpathlineto{\pgfqpoint{5.889150in}{0.759589in}}%
\pgfpathlineto{\pgfqpoint{5.835675in}{0.759589in}}%
\pgfpathlineto{\pgfqpoint{5.782200in}{0.759589in}}%
\pgfpathlineto{\pgfqpoint{5.728725in}{0.759589in}}%
\pgfpathlineto{\pgfqpoint{5.675250in}{0.759589in}}%
\pgfpathlineto{\pgfqpoint{5.621775in}{0.759589in}}%
\pgfpathlineto{\pgfqpoint{5.568300in}{0.759589in}}%
\pgfpathlineto{\pgfqpoint{5.514825in}{0.759589in}}%
\pgfpathlineto{\pgfqpoint{5.461350in}{0.759589in}}%
\pgfpathlineto{\pgfqpoint{5.407875in}{0.759589in}}%
\pgfpathlineto{\pgfqpoint{5.354400in}{0.759589in}}%
\pgfpathlineto{\pgfqpoint{5.300925in}{0.759589in}}%
\pgfpathlineto{\pgfqpoint{5.247450in}{0.759589in}}%
\pgfpathlineto{\pgfqpoint{5.193975in}{0.759589in}}%
\pgfpathlineto{\pgfqpoint{5.140500in}{0.759589in}}%
\pgfpathlineto{\pgfqpoint{5.087025in}{0.759589in}}%
\pgfpathlineto{\pgfqpoint{5.033550in}{0.759589in}}%
\pgfpathlineto{\pgfqpoint{4.980075in}{0.759589in}}%
\pgfpathlineto{\pgfqpoint{4.926600in}{0.759589in}}%
\pgfpathlineto{\pgfqpoint{4.873125in}{0.759589in}}%
\pgfpathlineto{\pgfqpoint{4.819650in}{0.759589in}}%
\pgfpathlineto{\pgfqpoint{4.766175in}{0.759589in}}%
\pgfpathlineto{\pgfqpoint{4.712700in}{0.759589in}}%
\pgfpathlineto{\pgfqpoint{4.659225in}{0.759589in}}%
\pgfpathlineto{\pgfqpoint{4.605750in}{0.759589in}}%
\pgfpathlineto{\pgfqpoint{4.552275in}{0.759589in}}%
\pgfpathlineto{\pgfqpoint{4.498800in}{0.759589in}}%
\pgfpathlineto{\pgfqpoint{4.445325in}{0.759589in}}%
\pgfpathlineto{\pgfqpoint{4.391850in}{0.759589in}}%
\pgfpathlineto{\pgfqpoint{4.338375in}{0.759589in}}%
\pgfpathlineto{\pgfqpoint{4.284900in}{0.759589in}}%
\pgfpathlineto{\pgfqpoint{4.231425in}{0.759589in}}%
\pgfpathlineto{\pgfqpoint{4.177950in}{0.759589in}}%
\pgfpathlineto{\pgfqpoint{4.124475in}{0.759589in}}%
\pgfpathlineto{\pgfqpoint{4.071000in}{0.759589in}}%
\pgfpathlineto{\pgfqpoint{4.017525in}{0.759589in}}%
\pgfpathlineto{\pgfqpoint{3.964050in}{0.864038in}}%
\pgfpathlineto{\pgfqpoint{3.910575in}{0.864038in}}%
\pgfpathlineto{\pgfqpoint{3.857100in}{0.864038in}}%
\pgfpathlineto{\pgfqpoint{3.803625in}{0.864038in}}%
\pgfpathlineto{\pgfqpoint{3.750150in}{0.864038in}}%
\pgfpathlineto{\pgfqpoint{3.696675in}{0.864038in}}%
\pgfpathlineto{\pgfqpoint{3.643200in}{0.864038in}}%
\pgfpathlineto{\pgfqpoint{3.589725in}{0.864038in}}%
\pgfpathlineto{\pgfqpoint{3.536250in}{0.864038in}}%
\pgfpathlineto{\pgfqpoint{3.482775in}{0.864038in}}%
\pgfpathlineto{\pgfqpoint{3.429300in}{0.864038in}}%
\pgfpathlineto{\pgfqpoint{3.375825in}{0.864038in}}%
\pgfpathlineto{\pgfqpoint{3.322350in}{0.864038in}}%
\pgfpathlineto{\pgfqpoint{3.268875in}{0.864038in}}%
\pgfpathlineto{\pgfqpoint{3.215400in}{0.864038in}}%
\pgfpathlineto{\pgfqpoint{3.161925in}{0.864038in}}%
\pgfpathlineto{\pgfqpoint{3.108450in}{0.864038in}}%
\pgfpathlineto{\pgfqpoint{3.054975in}{0.864038in}}%
\pgfpathlineto{\pgfqpoint{3.001500in}{0.891659in}}%
\pgfpathlineto{\pgfqpoint{2.948025in}{0.891659in}}%
\pgfpathlineto{\pgfqpoint{2.894550in}{0.891659in}}%
\pgfpathlineto{\pgfqpoint{2.841075in}{0.891659in}}%
\pgfpathlineto{\pgfqpoint{2.787600in}{0.891659in}}%
\pgfpathlineto{\pgfqpoint{2.734125in}{0.923129in}}%
\pgfpathlineto{\pgfqpoint{2.680650in}{0.923129in}}%
\pgfpathlineto{\pgfqpoint{2.627175in}{0.923129in}}%
\pgfpathlineto{\pgfqpoint{2.573700in}{0.923129in}}%
\pgfpathlineto{\pgfqpoint{2.520225in}{0.923129in}}%
\pgfpathlineto{\pgfqpoint{2.466750in}{0.970073in}}%
\pgfpathlineto{\pgfqpoint{2.413275in}{0.970073in}}%
\pgfpathlineto{\pgfqpoint{2.359800in}{0.970073in}}%
\pgfpathlineto{\pgfqpoint{2.306325in}{0.970073in}}%
\pgfpathlineto{\pgfqpoint{2.252850in}{0.970073in}}%
\pgfpathlineto{\pgfqpoint{2.199375in}{0.970073in}}%
\pgfpathlineto{\pgfqpoint{2.145900in}{0.970073in}}%
\pgfpathlineto{\pgfqpoint{2.092425in}{0.970073in}}%
\pgfpathlineto{\pgfqpoint{2.038950in}{0.970073in}}%
\pgfpathlineto{\pgfqpoint{1.985475in}{0.970073in}}%
\pgfpathlineto{\pgfqpoint{1.932000in}{0.970073in}}%
\pgfpathlineto{\pgfqpoint{1.878525in}{0.970073in}}%
\pgfpathlineto{\pgfqpoint{1.825050in}{0.970073in}}%
\pgfpathlineto{\pgfqpoint{1.771575in}{0.970073in}}%
\pgfpathlineto{\pgfqpoint{1.718100in}{0.970073in}}%
\pgfpathlineto{\pgfqpoint{1.664625in}{0.970073in}}%
\pgfpathlineto{\pgfqpoint{1.611150in}{0.970073in}}%
\pgfpathlineto{\pgfqpoint{1.557675in}{1.111095in}}%
\pgfpathlineto{\pgfqpoint{1.504200in}{1.193762in}}%
\pgfpathlineto{\pgfqpoint{1.450725in}{1.213640in}}%
\pgfpathlineto{\pgfqpoint{1.397250in}{1.213640in}}%
\pgfpathlineto{\pgfqpoint{1.343775in}{1.213640in}}%
\pgfpathlineto{\pgfqpoint{1.290300in}{1.213640in}}%
\pgfpathlineto{\pgfqpoint{1.236825in}{1.213640in}}%
\pgfpathlineto{\pgfqpoint{1.183350in}{1.213640in}}%
\pgfpathlineto{\pgfqpoint{1.129875in}{1.213640in}}%
\pgfpathlineto{\pgfqpoint{1.076400in}{1.260731in}}%
\pgfpathlineto{\pgfqpoint{1.022925in}{1.260731in}}%
\pgfpathlineto{\pgfqpoint{0.969450in}{1.263237in}}%
\pgfpathlineto{\pgfqpoint{0.915975in}{1.407835in}}%
\pgfpathlineto{\pgfqpoint{0.862500in}{1.546069in}}%
\pgfpathclose%
\pgfusepath{fill}%
\end{pgfscope}%
\begin{pgfscope}%
\pgfpathrectangle{\pgfqpoint{0.862500in}{0.375000in}}{\pgfqpoint{5.347500in}{2.265000in}}%
\pgfusepath{clip}%
\pgfsetroundcap%
\pgfsetroundjoin%
\pgfsetlinewidth{1.505625pt}%
\definecolor{currentstroke}{rgb}{0.121569,0.466667,0.705882}%
\pgfsetstrokecolor{currentstroke}%
\pgfsetdash{}{0pt}%
\pgfpathmoveto{\pgfqpoint{0.862500in}{2.350808in}}%
\pgfpathlineto{\pgfqpoint{0.915975in}{2.141793in}}%
\pgfpathlineto{\pgfqpoint{0.969450in}{1.383232in}}%
\pgfpathlineto{\pgfqpoint{1.022925in}{1.349133in}}%
\pgfpathlineto{\pgfqpoint{1.076400in}{1.349133in}}%
\pgfpathlineto{\pgfqpoint{1.129875in}{1.349133in}}%
\pgfpathlineto{\pgfqpoint{1.183350in}{1.175626in}}%
\pgfpathlineto{\pgfqpoint{1.236825in}{1.093453in}}%
\pgfpathlineto{\pgfqpoint{1.290300in}{1.093453in}}%
\pgfpathlineto{\pgfqpoint{1.343775in}{1.093453in}}%
\pgfpathlineto{\pgfqpoint{1.397250in}{1.053857in}}%
\pgfpathlineto{\pgfqpoint{1.450725in}{1.053857in}}%
\pgfpathlineto{\pgfqpoint{1.504200in}{1.053857in}}%
\pgfpathlineto{\pgfqpoint{1.557675in}{1.053857in}}%
\pgfpathlineto{\pgfqpoint{1.611150in}{1.053857in}}%
\pgfpathlineto{\pgfqpoint{1.664625in}{1.053857in}}%
\pgfpathlineto{\pgfqpoint{1.718100in}{1.053857in}}%
\pgfpathlineto{\pgfqpoint{1.771575in}{1.053857in}}%
\pgfpathlineto{\pgfqpoint{1.825050in}{1.053857in}}%
\pgfpathlineto{\pgfqpoint{1.878525in}{1.053857in}}%
\pgfpathlineto{\pgfqpoint{1.932000in}{0.589800in}}%
\pgfpathlineto{\pgfqpoint{1.985475in}{0.589800in}}%
\pgfpathlineto{\pgfqpoint{2.038950in}{0.589800in}}%
\pgfpathlineto{\pgfqpoint{2.092425in}{0.578228in}}%
\pgfpathlineto{\pgfqpoint{2.145900in}{0.578228in}}%
\pgfpathlineto{\pgfqpoint{2.199375in}{0.578228in}}%
\pgfpathlineto{\pgfqpoint{2.252850in}{0.578228in}}%
\pgfpathlineto{\pgfqpoint{2.306325in}{0.578228in}}%
\pgfpathlineto{\pgfqpoint{2.359800in}{0.543102in}}%
\pgfpathlineto{\pgfqpoint{2.413275in}{0.543102in}}%
\pgfpathlineto{\pgfqpoint{2.466750in}{0.543102in}}%
\pgfpathlineto{\pgfqpoint{2.520225in}{0.543102in}}%
\pgfpathlineto{\pgfqpoint{2.573700in}{0.543102in}}%
\pgfpathlineto{\pgfqpoint{2.627175in}{0.543102in}}%
\pgfpathlineto{\pgfqpoint{2.680650in}{0.543102in}}%
\pgfpathlineto{\pgfqpoint{2.734125in}{0.543102in}}%
\pgfpathlineto{\pgfqpoint{2.787600in}{0.540891in}}%
\pgfpathlineto{\pgfqpoint{2.841075in}{0.540891in}}%
\pgfpathlineto{\pgfqpoint{2.894550in}{0.540891in}}%
\pgfpathlineto{\pgfqpoint{2.948025in}{0.540891in}}%
\pgfpathlineto{\pgfqpoint{3.001500in}{0.540891in}}%
\pgfpathlineto{\pgfqpoint{3.054975in}{0.540891in}}%
\pgfpathlineto{\pgfqpoint{3.108450in}{0.540891in}}%
\pgfpathlineto{\pgfqpoint{3.161925in}{0.540891in}}%
\pgfpathlineto{\pgfqpoint{3.215400in}{0.540891in}}%
\pgfpathlineto{\pgfqpoint{3.268875in}{0.538146in}}%
\pgfpathlineto{\pgfqpoint{3.322350in}{0.538146in}}%
\pgfpathlineto{\pgfqpoint{3.375825in}{0.538146in}}%
\pgfpathlineto{\pgfqpoint{3.429300in}{0.530433in}}%
\pgfpathlineto{\pgfqpoint{3.482775in}{0.526921in}}%
\pgfpathlineto{\pgfqpoint{3.536250in}{0.504872in}}%
\pgfpathlineto{\pgfqpoint{3.589725in}{0.504872in}}%
\pgfpathlineto{\pgfqpoint{3.643200in}{0.504872in}}%
\pgfpathlineto{\pgfqpoint{3.696675in}{0.504872in}}%
\pgfpathlineto{\pgfqpoint{3.750150in}{0.504872in}}%
\pgfpathlineto{\pgfqpoint{3.803625in}{0.503226in}}%
\pgfpathlineto{\pgfqpoint{3.857100in}{0.499654in}}%
\pgfpathlineto{\pgfqpoint{3.910575in}{0.499654in}}%
\pgfpathlineto{\pgfqpoint{3.964050in}{0.499654in}}%
\pgfpathlineto{\pgfqpoint{4.017525in}{0.499654in}}%
\pgfpathlineto{\pgfqpoint{4.071000in}{0.499654in}}%
\pgfpathlineto{\pgfqpoint{4.124475in}{0.499654in}}%
\pgfpathlineto{\pgfqpoint{4.177950in}{0.499654in}}%
\pgfpathlineto{\pgfqpoint{4.231425in}{0.499654in}}%
\pgfpathlineto{\pgfqpoint{4.284900in}{0.499654in}}%
\pgfpathlineto{\pgfqpoint{4.338375in}{0.499654in}}%
\pgfpathlineto{\pgfqpoint{4.391850in}{0.499654in}}%
\pgfpathlineto{\pgfqpoint{4.445325in}{0.499654in}}%
\pgfpathlineto{\pgfqpoint{4.498800in}{0.499654in}}%
\pgfpathlineto{\pgfqpoint{4.552275in}{0.499654in}}%
\pgfpathlineto{\pgfqpoint{4.605750in}{0.499654in}}%
\pgfpathlineto{\pgfqpoint{4.659225in}{0.499325in}}%
\pgfpathlineto{\pgfqpoint{4.712700in}{0.499325in}}%
\pgfpathlineto{\pgfqpoint{4.766175in}{0.499325in}}%
\pgfpathlineto{\pgfqpoint{4.819650in}{0.499325in}}%
\pgfpathlineto{\pgfqpoint{4.873125in}{0.499325in}}%
\pgfpathlineto{\pgfqpoint{4.926600in}{0.499325in}}%
\pgfpathlineto{\pgfqpoint{4.980075in}{0.499325in}}%
\pgfpathlineto{\pgfqpoint{5.033550in}{0.499325in}}%
\pgfpathlineto{\pgfqpoint{5.087025in}{0.499325in}}%
\pgfpathlineto{\pgfqpoint{5.140500in}{0.499325in}}%
\pgfpathlineto{\pgfqpoint{5.193975in}{0.499325in}}%
\pgfpathlineto{\pgfqpoint{5.247450in}{0.499325in}}%
\pgfpathlineto{\pgfqpoint{5.300925in}{0.499325in}}%
\pgfpathlineto{\pgfqpoint{5.354400in}{0.499325in}}%
\pgfpathlineto{\pgfqpoint{5.407875in}{0.499325in}}%
\pgfpathlineto{\pgfqpoint{5.461350in}{0.499325in}}%
\pgfpathlineto{\pgfqpoint{5.514825in}{0.499325in}}%
\pgfpathlineto{\pgfqpoint{5.568300in}{0.499325in}}%
\pgfpathlineto{\pgfqpoint{5.621775in}{0.499325in}}%
\pgfpathlineto{\pgfqpoint{5.675250in}{0.499325in}}%
\pgfpathlineto{\pgfqpoint{5.728725in}{0.499325in}}%
\pgfpathlineto{\pgfqpoint{5.782200in}{0.499325in}}%
\pgfpathlineto{\pgfqpoint{5.835675in}{0.499325in}}%
\pgfpathlineto{\pgfqpoint{5.889150in}{0.499325in}}%
\pgfpathlineto{\pgfqpoint{5.942625in}{0.499325in}}%
\pgfpathlineto{\pgfqpoint{5.996100in}{0.499325in}}%
\pgfpathlineto{\pgfqpoint{6.049575in}{0.499325in}}%
\pgfpathlineto{\pgfqpoint{6.103050in}{0.499325in}}%
\pgfpathlineto{\pgfqpoint{6.156525in}{0.499325in}}%
\pgfpathlineto{\pgfqpoint{6.210000in}{0.499325in}}%
\pgfpathlineto{\pgfqpoint{6.223889in}{0.499325in}}%
\pgfusepath{stroke}%
\end{pgfscope}%
\begin{pgfscope}%
\pgfpathrectangle{\pgfqpoint{0.862500in}{0.375000in}}{\pgfqpoint{5.347500in}{2.265000in}}%
\pgfusepath{clip}%
\pgfsetroundcap%
\pgfsetroundjoin%
\pgfsetlinewidth{1.505625pt}%
\definecolor{currentstroke}{rgb}{1.000000,0.498039,0.054902}%
\pgfsetstrokecolor{currentstroke}%
\pgfsetdash{}{0pt}%
\pgfpathmoveto{\pgfqpoint{0.862500in}{2.163780in}}%
\pgfpathlineto{\pgfqpoint{0.915975in}{1.671599in}}%
\pgfpathlineto{\pgfqpoint{0.969450in}{1.204385in}}%
\pgfpathlineto{\pgfqpoint{1.022925in}{1.204385in}}%
\pgfpathlineto{\pgfqpoint{1.076400in}{1.186098in}}%
\pgfpathlineto{\pgfqpoint{1.129875in}{1.186098in}}%
\pgfpathlineto{\pgfqpoint{1.183350in}{1.125603in}}%
\pgfpathlineto{\pgfqpoint{1.236825in}{1.125603in}}%
\pgfpathlineto{\pgfqpoint{1.290300in}{1.125603in}}%
\pgfpathlineto{\pgfqpoint{1.343775in}{1.125603in}}%
\pgfpathlineto{\pgfqpoint{1.397250in}{1.125603in}}%
\pgfpathlineto{\pgfqpoint{1.450725in}{1.125603in}}%
\pgfpathlineto{\pgfqpoint{1.504200in}{1.125603in}}%
\pgfpathlineto{\pgfqpoint{1.557675in}{1.125603in}}%
\pgfpathlineto{\pgfqpoint{1.611150in}{1.004188in}}%
\pgfpathlineto{\pgfqpoint{1.664625in}{1.004188in}}%
\pgfpathlineto{\pgfqpoint{1.718100in}{1.004188in}}%
\pgfpathlineto{\pgfqpoint{1.771575in}{1.004188in}}%
\pgfpathlineto{\pgfqpoint{1.825050in}{1.004188in}}%
\pgfpathlineto{\pgfqpoint{1.878525in}{1.004188in}}%
\pgfpathlineto{\pgfqpoint{1.932000in}{0.904337in}}%
\pgfpathlineto{\pgfqpoint{1.985475in}{0.724153in}}%
\pgfpathlineto{\pgfqpoint{2.038950in}{0.724153in}}%
\pgfpathlineto{\pgfqpoint{2.092425in}{0.692634in}}%
\pgfpathlineto{\pgfqpoint{2.145900in}{0.692634in}}%
\pgfpathlineto{\pgfqpoint{2.199375in}{0.692634in}}%
\pgfpathlineto{\pgfqpoint{2.252850in}{0.692634in}}%
\pgfpathlineto{\pgfqpoint{2.306325in}{0.692634in}}%
\pgfpathlineto{\pgfqpoint{2.359800in}{0.692634in}}%
\pgfpathlineto{\pgfqpoint{2.413275in}{0.692634in}}%
\pgfpathlineto{\pgfqpoint{2.466750in}{0.692634in}}%
\pgfpathlineto{\pgfqpoint{2.520225in}{0.692634in}}%
\pgfpathlineto{\pgfqpoint{2.573700in}{0.692634in}}%
\pgfpathlineto{\pgfqpoint{2.627175in}{0.587833in}}%
\pgfpathlineto{\pgfqpoint{2.680650in}{0.587833in}}%
\pgfpathlineto{\pgfqpoint{2.734125in}{0.587833in}}%
\pgfpathlineto{\pgfqpoint{2.787600in}{0.587833in}}%
\pgfpathlineto{\pgfqpoint{2.841075in}{0.587833in}}%
\pgfpathlineto{\pgfqpoint{2.894550in}{0.587833in}}%
\pgfpathlineto{\pgfqpoint{2.948025in}{0.577764in}}%
\pgfpathlineto{\pgfqpoint{3.001500in}{0.577764in}}%
\pgfpathlineto{\pgfqpoint{3.054975in}{0.577764in}}%
\pgfpathlineto{\pgfqpoint{3.108450in}{0.577764in}}%
\pgfpathlineto{\pgfqpoint{3.161925in}{0.577764in}}%
\pgfpathlineto{\pgfqpoint{3.215400in}{0.572537in}}%
\pgfpathlineto{\pgfqpoint{3.268875in}{0.572537in}}%
\pgfpathlineto{\pgfqpoint{3.322350in}{0.572537in}}%
\pgfpathlineto{\pgfqpoint{3.375825in}{0.572537in}}%
\pgfpathlineto{\pgfqpoint{3.429300in}{0.572537in}}%
\pgfpathlineto{\pgfqpoint{3.482775in}{0.572537in}}%
\pgfpathlineto{\pgfqpoint{3.536250in}{0.567844in}}%
\pgfpathlineto{\pgfqpoint{3.589725in}{0.567844in}}%
\pgfpathlineto{\pgfqpoint{3.643200in}{0.567844in}}%
\pgfpathlineto{\pgfqpoint{3.696675in}{0.567844in}}%
\pgfpathlineto{\pgfqpoint{3.750150in}{0.567844in}}%
\pgfpathlineto{\pgfqpoint{3.803625in}{0.567844in}}%
\pgfpathlineto{\pgfqpoint{3.857100in}{0.567844in}}%
\pgfpathlineto{\pgfqpoint{3.910575in}{0.567844in}}%
\pgfpathlineto{\pgfqpoint{3.964050in}{0.567844in}}%
\pgfpathlineto{\pgfqpoint{4.017525in}{0.567844in}}%
\pgfpathlineto{\pgfqpoint{4.071000in}{0.567844in}}%
\pgfpathlineto{\pgfqpoint{4.124475in}{0.567844in}}%
\pgfpathlineto{\pgfqpoint{4.177950in}{0.567844in}}%
\pgfpathlineto{\pgfqpoint{4.231425in}{0.566809in}}%
\pgfpathlineto{\pgfqpoint{4.284900in}{0.566809in}}%
\pgfpathlineto{\pgfqpoint{4.338375in}{0.552114in}}%
\pgfpathlineto{\pgfqpoint{4.391850in}{0.552114in}}%
\pgfpathlineto{\pgfqpoint{4.445325in}{0.552114in}}%
\pgfpathlineto{\pgfqpoint{4.498800in}{0.552114in}}%
\pgfpathlineto{\pgfqpoint{4.552275in}{0.551460in}}%
\pgfpathlineto{\pgfqpoint{4.605750in}{0.551460in}}%
\pgfpathlineto{\pgfqpoint{4.659225in}{0.551460in}}%
\pgfpathlineto{\pgfqpoint{4.712700in}{0.551460in}}%
\pgfpathlineto{\pgfqpoint{4.766175in}{0.546147in}}%
\pgfpathlineto{\pgfqpoint{4.819650in}{0.546147in}}%
\pgfpathlineto{\pgfqpoint{4.873125in}{0.546147in}}%
\pgfpathlineto{\pgfqpoint{4.926600in}{0.546147in}}%
\pgfpathlineto{\pgfqpoint{4.980075in}{0.546147in}}%
\pgfpathlineto{\pgfqpoint{5.033550in}{0.546147in}}%
\pgfpathlineto{\pgfqpoint{5.087025in}{0.546147in}}%
\pgfpathlineto{\pgfqpoint{5.140500in}{0.546147in}}%
\pgfpathlineto{\pgfqpoint{5.193975in}{0.542686in}}%
\pgfpathlineto{\pgfqpoint{5.247450in}{0.542686in}}%
\pgfpathlineto{\pgfqpoint{5.300925in}{0.542686in}}%
\pgfpathlineto{\pgfqpoint{5.354400in}{0.542686in}}%
\pgfpathlineto{\pgfqpoint{5.407875in}{0.542686in}}%
\pgfpathlineto{\pgfqpoint{5.461350in}{0.542686in}}%
\pgfpathlineto{\pgfqpoint{5.514825in}{0.542686in}}%
\pgfpathlineto{\pgfqpoint{5.568300in}{0.542686in}}%
\pgfpathlineto{\pgfqpoint{5.621775in}{0.542686in}}%
\pgfpathlineto{\pgfqpoint{5.675250in}{0.542686in}}%
\pgfpathlineto{\pgfqpoint{5.728725in}{0.542686in}}%
\pgfpathlineto{\pgfqpoint{5.782200in}{0.542686in}}%
\pgfpathlineto{\pgfqpoint{5.835675in}{0.535885in}}%
\pgfpathlineto{\pgfqpoint{5.889150in}{0.535885in}}%
\pgfpathlineto{\pgfqpoint{5.942625in}{0.535108in}}%
\pgfpathlineto{\pgfqpoint{5.996100in}{0.535108in}}%
\pgfpathlineto{\pgfqpoint{6.049575in}{0.535108in}}%
\pgfpathlineto{\pgfqpoint{6.103050in}{0.535108in}}%
\pgfpathlineto{\pgfqpoint{6.156525in}{0.535108in}}%
\pgfpathlineto{\pgfqpoint{6.210000in}{0.535108in}}%
\pgfpathlineto{\pgfqpoint{6.223889in}{0.535108in}}%
\pgfusepath{stroke}%
\end{pgfscope}%
\begin{pgfscope}%
\pgfpathrectangle{\pgfqpoint{0.862500in}{0.375000in}}{\pgfqpoint{5.347500in}{2.265000in}}%
\pgfusepath{clip}%
\pgfsetroundcap%
\pgfsetroundjoin%
\pgfsetlinewidth{1.505625pt}%
\definecolor{currentstroke}{rgb}{0.172549,0.627451,0.172549}%
\pgfsetstrokecolor{currentstroke}%
\pgfsetdash{}{0pt}%
\pgfpathmoveto{\pgfqpoint{0.862500in}{1.848001in}}%
\pgfpathlineto{\pgfqpoint{0.915975in}{1.550645in}}%
\pgfpathlineto{\pgfqpoint{0.969450in}{1.301643in}}%
\pgfpathlineto{\pgfqpoint{1.022925in}{1.275669in}}%
\pgfpathlineto{\pgfqpoint{1.076400in}{1.275669in}}%
\pgfpathlineto{\pgfqpoint{1.129875in}{1.036069in}}%
\pgfpathlineto{\pgfqpoint{1.183350in}{0.726179in}}%
\pgfpathlineto{\pgfqpoint{1.236825in}{0.516828in}}%
\pgfpathlineto{\pgfqpoint{1.290300in}{0.516828in}}%
\pgfpathlineto{\pgfqpoint{1.343775in}{0.516828in}}%
\pgfpathlineto{\pgfqpoint{1.397250in}{0.516828in}}%
\pgfpathlineto{\pgfqpoint{1.450725in}{0.516828in}}%
\pgfpathlineto{\pgfqpoint{1.504200in}{0.516828in}}%
\pgfpathlineto{\pgfqpoint{1.557675in}{0.516828in}}%
\pgfpathlineto{\pgfqpoint{1.611150in}{0.516828in}}%
\pgfpathlineto{\pgfqpoint{1.664625in}{0.516828in}}%
\pgfpathlineto{\pgfqpoint{1.718100in}{0.516828in}}%
\pgfpathlineto{\pgfqpoint{1.771575in}{0.516828in}}%
\pgfpathlineto{\pgfqpoint{1.825050in}{0.516828in}}%
\pgfpathlineto{\pgfqpoint{1.878525in}{0.516828in}}%
\pgfpathlineto{\pgfqpoint{1.932000in}{0.516828in}}%
\pgfpathlineto{\pgfqpoint{1.985475in}{0.516828in}}%
\pgfpathlineto{\pgfqpoint{2.038950in}{0.516828in}}%
\pgfpathlineto{\pgfqpoint{2.092425in}{0.516828in}}%
\pgfpathlineto{\pgfqpoint{2.145900in}{0.516828in}}%
\pgfpathlineto{\pgfqpoint{2.199375in}{0.516828in}}%
\pgfpathlineto{\pgfqpoint{2.252850in}{0.513051in}}%
\pgfpathlineto{\pgfqpoint{2.306325in}{0.513051in}}%
\pgfpathlineto{\pgfqpoint{2.359800in}{0.513051in}}%
\pgfpathlineto{\pgfqpoint{2.413275in}{0.513051in}}%
\pgfpathlineto{\pgfqpoint{2.466750in}{0.513051in}}%
\pgfpathlineto{\pgfqpoint{2.520225in}{0.513051in}}%
\pgfpathlineto{\pgfqpoint{2.573700in}{0.513051in}}%
\pgfpathlineto{\pgfqpoint{2.627175in}{0.513051in}}%
\pgfpathlineto{\pgfqpoint{2.680650in}{0.513051in}}%
\pgfpathlineto{\pgfqpoint{2.734125in}{0.513051in}}%
\pgfpathlineto{\pgfqpoint{2.787600in}{0.513051in}}%
\pgfpathlineto{\pgfqpoint{2.841075in}{0.513051in}}%
\pgfpathlineto{\pgfqpoint{2.894550in}{0.513051in}}%
\pgfpathlineto{\pgfqpoint{2.948025in}{0.513051in}}%
\pgfpathlineto{\pgfqpoint{3.001500in}{0.513051in}}%
\pgfpathlineto{\pgfqpoint{3.054975in}{0.513051in}}%
\pgfpathlineto{\pgfqpoint{3.108450in}{0.511395in}}%
\pgfpathlineto{\pgfqpoint{3.161925in}{0.510367in}}%
\pgfpathlineto{\pgfqpoint{3.215400in}{0.510367in}}%
\pgfpathlineto{\pgfqpoint{3.268875in}{0.505906in}}%
\pgfpathlineto{\pgfqpoint{3.322350in}{0.505906in}}%
\pgfpathlineto{\pgfqpoint{3.375825in}{0.505906in}}%
\pgfpathlineto{\pgfqpoint{3.429300in}{0.505906in}}%
\pgfpathlineto{\pgfqpoint{3.482775in}{0.505105in}}%
\pgfpathlineto{\pgfqpoint{3.536250in}{0.505060in}}%
\pgfpathlineto{\pgfqpoint{3.589725in}{0.505060in}}%
\pgfpathlineto{\pgfqpoint{3.643200in}{0.502123in}}%
\pgfpathlineto{\pgfqpoint{3.696675in}{0.502123in}}%
\pgfpathlineto{\pgfqpoint{3.750150in}{0.502123in}}%
\pgfpathlineto{\pgfqpoint{3.803625in}{0.502123in}}%
\pgfpathlineto{\pgfqpoint{3.857100in}{0.499705in}}%
\pgfpathlineto{\pgfqpoint{3.910575in}{0.499705in}}%
\pgfpathlineto{\pgfqpoint{3.964050in}{0.499705in}}%
\pgfpathlineto{\pgfqpoint{4.017525in}{0.499705in}}%
\pgfpathlineto{\pgfqpoint{4.071000in}{0.499705in}}%
\pgfpathlineto{\pgfqpoint{4.124475in}{0.499705in}}%
\pgfpathlineto{\pgfqpoint{4.177950in}{0.499705in}}%
\pgfpathlineto{\pgfqpoint{4.231425in}{0.499705in}}%
\pgfpathlineto{\pgfqpoint{4.284900in}{0.499705in}}%
\pgfpathlineto{\pgfqpoint{4.338375in}{0.499705in}}%
\pgfpathlineto{\pgfqpoint{4.391850in}{0.499705in}}%
\pgfpathlineto{\pgfqpoint{4.445325in}{0.499705in}}%
\pgfpathlineto{\pgfqpoint{4.498800in}{0.499705in}}%
\pgfpathlineto{\pgfqpoint{4.552275in}{0.499705in}}%
\pgfpathlineto{\pgfqpoint{4.605750in}{0.499705in}}%
\pgfpathlineto{\pgfqpoint{4.659225in}{0.499705in}}%
\pgfpathlineto{\pgfqpoint{4.712700in}{0.499705in}}%
\pgfpathlineto{\pgfqpoint{4.766175in}{0.499705in}}%
\pgfpathlineto{\pgfqpoint{4.819650in}{0.499705in}}%
\pgfpathlineto{\pgfqpoint{4.873125in}{0.499705in}}%
\pgfpathlineto{\pgfqpoint{4.926600in}{0.499705in}}%
\pgfpathlineto{\pgfqpoint{4.980075in}{0.499705in}}%
\pgfpathlineto{\pgfqpoint{5.033550in}{0.499705in}}%
\pgfpathlineto{\pgfqpoint{5.087025in}{0.499705in}}%
\pgfpathlineto{\pgfqpoint{5.140500in}{0.499705in}}%
\pgfpathlineto{\pgfqpoint{5.193975in}{0.499705in}}%
\pgfpathlineto{\pgfqpoint{5.247450in}{0.499705in}}%
\pgfpathlineto{\pgfqpoint{5.300925in}{0.499705in}}%
\pgfpathlineto{\pgfqpoint{5.354400in}{0.499705in}}%
\pgfpathlineto{\pgfqpoint{5.407875in}{0.499640in}}%
\pgfpathlineto{\pgfqpoint{5.461350in}{0.499640in}}%
\pgfpathlineto{\pgfqpoint{5.514825in}{0.499640in}}%
\pgfpathlineto{\pgfqpoint{5.568300in}{0.499640in}}%
\pgfpathlineto{\pgfqpoint{5.621775in}{0.499640in}}%
\pgfpathlineto{\pgfqpoint{5.675250in}{0.499640in}}%
\pgfpathlineto{\pgfqpoint{5.728725in}{0.498507in}}%
\pgfpathlineto{\pgfqpoint{5.782200in}{0.498507in}}%
\pgfpathlineto{\pgfqpoint{5.835675in}{0.498507in}}%
\pgfpathlineto{\pgfqpoint{5.889150in}{0.498507in}}%
\pgfpathlineto{\pgfqpoint{5.942625in}{0.498507in}}%
\pgfpathlineto{\pgfqpoint{5.996100in}{0.498507in}}%
\pgfpathlineto{\pgfqpoint{6.049575in}{0.498507in}}%
\pgfpathlineto{\pgfqpoint{6.103050in}{0.498507in}}%
\pgfpathlineto{\pgfqpoint{6.156525in}{0.498507in}}%
\pgfpathlineto{\pgfqpoint{6.210000in}{0.498507in}}%
\pgfpathlineto{\pgfqpoint{6.223889in}{0.498507in}}%
\pgfusepath{stroke}%
\end{pgfscope}%
\begin{pgfscope}%
\pgfpathrectangle{\pgfqpoint{0.862500in}{0.375000in}}{\pgfqpoint{5.347500in}{2.265000in}}%
\pgfusepath{clip}%
\pgfsetroundcap%
\pgfsetroundjoin%
\pgfsetlinewidth{1.505625pt}%
\definecolor{currentstroke}{rgb}{0.839216,0.152941,0.156863}%
\pgfsetstrokecolor{currentstroke}%
\pgfsetdash{}{0pt}%
\pgfpathmoveto{\pgfqpoint{0.862500in}{2.035464in}}%
\pgfpathlineto{\pgfqpoint{0.915975in}{1.672228in}}%
\pgfpathlineto{\pgfqpoint{0.969450in}{1.672228in}}%
\pgfpathlineto{\pgfqpoint{1.022925in}{1.672228in}}%
\pgfpathlineto{\pgfqpoint{1.076400in}{1.672228in}}%
\pgfpathlineto{\pgfqpoint{1.129875in}{1.290435in}}%
\pgfpathlineto{\pgfqpoint{1.183350in}{1.290435in}}%
\pgfpathlineto{\pgfqpoint{1.236825in}{1.140670in}}%
\pgfpathlineto{\pgfqpoint{1.290300in}{1.025246in}}%
\pgfpathlineto{\pgfqpoint{1.343775in}{1.025246in}}%
\pgfpathlineto{\pgfqpoint{1.397250in}{1.025246in}}%
\pgfpathlineto{\pgfqpoint{1.450725in}{1.025246in}}%
\pgfpathlineto{\pgfqpoint{1.504200in}{0.657347in}}%
\pgfpathlineto{\pgfqpoint{1.557675in}{0.657347in}}%
\pgfpathlineto{\pgfqpoint{1.611150in}{0.652322in}}%
\pgfpathlineto{\pgfqpoint{1.664625in}{0.642848in}}%
\pgfpathlineto{\pgfqpoint{1.718100in}{0.642848in}}%
\pgfpathlineto{\pgfqpoint{1.771575in}{0.642848in}}%
\pgfpathlineto{\pgfqpoint{1.825050in}{0.642848in}}%
\pgfpathlineto{\pgfqpoint{1.878525in}{0.592808in}}%
\pgfpathlineto{\pgfqpoint{1.932000in}{0.592808in}}%
\pgfpathlineto{\pgfqpoint{1.985475in}{0.586341in}}%
\pgfpathlineto{\pgfqpoint{2.038950in}{0.586341in}}%
\pgfpathlineto{\pgfqpoint{2.092425in}{0.586341in}}%
\pgfpathlineto{\pgfqpoint{2.145900in}{0.586341in}}%
\pgfpathlineto{\pgfqpoint{2.199375in}{0.586341in}}%
\pgfpathlineto{\pgfqpoint{2.252850in}{0.586341in}}%
\pgfpathlineto{\pgfqpoint{2.306325in}{0.586341in}}%
\pgfpathlineto{\pgfqpoint{2.359800in}{0.586341in}}%
\pgfpathlineto{\pgfqpoint{2.413275in}{0.584867in}}%
\pgfpathlineto{\pgfqpoint{2.466750in}{0.584802in}}%
\pgfpathlineto{\pgfqpoint{2.520225in}{0.584802in}}%
\pgfpathlineto{\pgfqpoint{2.573700in}{0.584802in}}%
\pgfpathlineto{\pgfqpoint{2.627175in}{0.584802in}}%
\pgfpathlineto{\pgfqpoint{2.680650in}{0.578353in}}%
\pgfpathlineto{\pgfqpoint{2.734125in}{0.578353in}}%
\pgfpathlineto{\pgfqpoint{2.787600in}{0.560178in}}%
\pgfpathlineto{\pgfqpoint{2.841075in}{0.560178in}}%
\pgfpathlineto{\pgfqpoint{2.894550in}{0.560178in}}%
\pgfpathlineto{\pgfqpoint{2.948025in}{0.559387in}}%
\pgfpathlineto{\pgfqpoint{3.001500in}{0.559387in}}%
\pgfpathlineto{\pgfqpoint{3.054975in}{0.559387in}}%
\pgfpathlineto{\pgfqpoint{3.108450in}{0.521028in}}%
\pgfpathlineto{\pgfqpoint{3.161925in}{0.521028in}}%
\pgfpathlineto{\pgfqpoint{3.215400in}{0.521028in}}%
\pgfpathlineto{\pgfqpoint{3.268875in}{0.521028in}}%
\pgfpathlineto{\pgfqpoint{3.322350in}{0.521028in}}%
\pgfpathlineto{\pgfqpoint{3.375825in}{0.521028in}}%
\pgfpathlineto{\pgfqpoint{3.429300in}{0.521028in}}%
\pgfpathlineto{\pgfqpoint{3.482775in}{0.521028in}}%
\pgfpathlineto{\pgfqpoint{3.536250in}{0.521028in}}%
\pgfpathlineto{\pgfqpoint{3.589725in}{0.521028in}}%
\pgfpathlineto{\pgfqpoint{3.643200in}{0.521028in}}%
\pgfpathlineto{\pgfqpoint{3.696675in}{0.521028in}}%
\pgfpathlineto{\pgfqpoint{3.750150in}{0.521028in}}%
\pgfpathlineto{\pgfqpoint{3.803625in}{0.520114in}}%
\pgfpathlineto{\pgfqpoint{3.857100in}{0.520114in}}%
\pgfpathlineto{\pgfqpoint{3.910575in}{0.520114in}}%
\pgfpathlineto{\pgfqpoint{3.964050in}{0.520114in}}%
\pgfpathlineto{\pgfqpoint{4.017525in}{0.520114in}}%
\pgfpathlineto{\pgfqpoint{4.071000in}{0.520114in}}%
\pgfpathlineto{\pgfqpoint{4.124475in}{0.519776in}}%
\pgfpathlineto{\pgfqpoint{4.177950in}{0.519776in}}%
\pgfpathlineto{\pgfqpoint{4.231425in}{0.519776in}}%
\pgfpathlineto{\pgfqpoint{4.284900in}{0.519776in}}%
\pgfpathlineto{\pgfqpoint{4.338375in}{0.519776in}}%
\pgfpathlineto{\pgfqpoint{4.391850in}{0.519776in}}%
\pgfpathlineto{\pgfqpoint{4.445325in}{0.519776in}}%
\pgfpathlineto{\pgfqpoint{4.498800in}{0.519776in}}%
\pgfpathlineto{\pgfqpoint{4.552275in}{0.519776in}}%
\pgfpathlineto{\pgfqpoint{4.605750in}{0.519776in}}%
\pgfpathlineto{\pgfqpoint{4.659225in}{0.519776in}}%
\pgfpathlineto{\pgfqpoint{4.712700in}{0.519776in}}%
\pgfpathlineto{\pgfqpoint{4.766175in}{0.519776in}}%
\pgfpathlineto{\pgfqpoint{4.819650in}{0.503768in}}%
\pgfpathlineto{\pgfqpoint{4.873125in}{0.503768in}}%
\pgfpathlineto{\pgfqpoint{4.926600in}{0.503768in}}%
\pgfpathlineto{\pgfqpoint{4.980075in}{0.503768in}}%
\pgfpathlineto{\pgfqpoint{5.033550in}{0.503768in}}%
\pgfpathlineto{\pgfqpoint{5.087025in}{0.503768in}}%
\pgfpathlineto{\pgfqpoint{5.140500in}{0.503768in}}%
\pgfpathlineto{\pgfqpoint{5.193975in}{0.503768in}}%
\pgfpathlineto{\pgfqpoint{5.247450in}{0.503768in}}%
\pgfpathlineto{\pgfqpoint{5.300925in}{0.503768in}}%
\pgfpathlineto{\pgfqpoint{5.354400in}{0.503768in}}%
\pgfpathlineto{\pgfqpoint{5.407875in}{0.503768in}}%
\pgfpathlineto{\pgfqpoint{5.461350in}{0.503768in}}%
\pgfpathlineto{\pgfqpoint{5.514825in}{0.503768in}}%
\pgfpathlineto{\pgfqpoint{5.568300in}{0.503768in}}%
\pgfpathlineto{\pgfqpoint{5.621775in}{0.503768in}}%
\pgfpathlineto{\pgfqpoint{5.675250in}{0.503768in}}%
\pgfpathlineto{\pgfqpoint{5.728725in}{0.503768in}}%
\pgfpathlineto{\pgfqpoint{5.782200in}{0.503768in}}%
\pgfpathlineto{\pgfqpoint{5.835675in}{0.503768in}}%
\pgfpathlineto{\pgfqpoint{5.889150in}{0.503768in}}%
\pgfpathlineto{\pgfqpoint{5.942625in}{0.503768in}}%
\pgfpathlineto{\pgfqpoint{5.996100in}{0.503768in}}%
\pgfpathlineto{\pgfqpoint{6.049575in}{0.503768in}}%
\pgfpathlineto{\pgfqpoint{6.103050in}{0.503768in}}%
\pgfpathlineto{\pgfqpoint{6.156525in}{0.503768in}}%
\pgfpathlineto{\pgfqpoint{6.210000in}{0.503768in}}%
\pgfpathlineto{\pgfqpoint{6.223889in}{0.503768in}}%
\pgfusepath{stroke}%
\end{pgfscope}%
\begin{pgfscope}%
\pgfpathrectangle{\pgfqpoint{0.862500in}{0.375000in}}{\pgfqpoint{5.347500in}{2.265000in}}%
\pgfusepath{clip}%
\pgfsetroundcap%
\pgfsetroundjoin%
\pgfsetlinewidth{1.505625pt}%
\definecolor{currentstroke}{rgb}{0.580392,0.403922,0.741176}%
\pgfsetstrokecolor{currentstroke}%
\pgfsetdash{}{0pt}%
\pgfpathmoveto{\pgfqpoint{0.862500in}{1.440952in}}%
\pgfpathlineto{\pgfqpoint{0.915975in}{1.312640in}}%
\pgfpathlineto{\pgfqpoint{0.969450in}{1.247588in}}%
\pgfpathlineto{\pgfqpoint{1.022925in}{1.113576in}}%
\pgfpathlineto{\pgfqpoint{1.076400in}{1.112824in}}%
\pgfpathlineto{\pgfqpoint{1.129875in}{1.112824in}}%
\pgfpathlineto{\pgfqpoint{1.183350in}{1.110388in}}%
\pgfpathlineto{\pgfqpoint{1.236825in}{1.085760in}}%
\pgfpathlineto{\pgfqpoint{1.290300in}{1.037519in}}%
\pgfpathlineto{\pgfqpoint{1.343775in}{1.037519in}}%
\pgfpathlineto{\pgfqpoint{1.397250in}{1.037519in}}%
\pgfpathlineto{\pgfqpoint{1.450725in}{1.033400in}}%
\pgfpathlineto{\pgfqpoint{1.504200in}{1.031939in}}%
\pgfpathlineto{\pgfqpoint{1.557675in}{0.846007in}}%
\pgfpathlineto{\pgfqpoint{1.611150in}{0.696890in}}%
\pgfpathlineto{\pgfqpoint{1.664625in}{0.613839in}}%
\pgfpathlineto{\pgfqpoint{1.718100in}{0.560910in}}%
\pgfpathlineto{\pgfqpoint{1.771575in}{0.558723in}}%
\pgfpathlineto{\pgfqpoint{1.825050in}{0.557132in}}%
\pgfpathlineto{\pgfqpoint{1.878525in}{0.516762in}}%
\pgfpathlineto{\pgfqpoint{1.932000in}{0.516762in}}%
\pgfpathlineto{\pgfqpoint{1.985475in}{0.516762in}}%
\pgfpathlineto{\pgfqpoint{2.038950in}{0.516762in}}%
\pgfpathlineto{\pgfqpoint{2.092425in}{0.516762in}}%
\pgfpathlineto{\pgfqpoint{2.145900in}{0.516762in}}%
\pgfpathlineto{\pgfqpoint{2.199375in}{0.516762in}}%
\pgfpathlineto{\pgfqpoint{2.252850in}{0.509479in}}%
\pgfpathlineto{\pgfqpoint{2.306325in}{0.509479in}}%
\pgfpathlineto{\pgfqpoint{2.359800in}{0.509479in}}%
\pgfpathlineto{\pgfqpoint{2.413275in}{0.506770in}}%
\pgfpathlineto{\pgfqpoint{2.466750in}{0.506770in}}%
\pgfpathlineto{\pgfqpoint{2.520225in}{0.506770in}}%
\pgfpathlineto{\pgfqpoint{2.573700in}{0.506770in}}%
\pgfpathlineto{\pgfqpoint{2.627175in}{0.506770in}}%
\pgfpathlineto{\pgfqpoint{2.680650in}{0.506770in}}%
\pgfpathlineto{\pgfqpoint{2.734125in}{0.506770in}}%
\pgfpathlineto{\pgfqpoint{2.787600in}{0.506770in}}%
\pgfpathlineto{\pgfqpoint{2.841075in}{0.506770in}}%
\pgfpathlineto{\pgfqpoint{2.894550in}{0.506770in}}%
\pgfpathlineto{\pgfqpoint{2.948025in}{0.506770in}}%
\pgfpathlineto{\pgfqpoint{3.001500in}{0.506770in}}%
\pgfpathlineto{\pgfqpoint{3.054975in}{0.506770in}}%
\pgfpathlineto{\pgfqpoint{3.108450in}{0.506770in}}%
\pgfpathlineto{\pgfqpoint{3.161925in}{0.506770in}}%
\pgfpathlineto{\pgfqpoint{3.215400in}{0.506770in}}%
\pgfpathlineto{\pgfqpoint{3.268875in}{0.506770in}}%
\pgfpathlineto{\pgfqpoint{3.322350in}{0.506770in}}%
\pgfpathlineto{\pgfqpoint{3.375825in}{0.504757in}}%
\pgfpathlineto{\pgfqpoint{3.429300in}{0.504757in}}%
\pgfpathlineto{\pgfqpoint{3.482775in}{0.504757in}}%
\pgfpathlineto{\pgfqpoint{3.536250in}{0.504757in}}%
\pgfpathlineto{\pgfqpoint{3.589725in}{0.504757in}}%
\pgfpathlineto{\pgfqpoint{3.643200in}{0.504757in}}%
\pgfpathlineto{\pgfqpoint{3.696675in}{0.504757in}}%
\pgfpathlineto{\pgfqpoint{3.750150in}{0.504757in}}%
\pgfpathlineto{\pgfqpoint{3.803625in}{0.504757in}}%
\pgfpathlineto{\pgfqpoint{3.857100in}{0.504757in}}%
\pgfpathlineto{\pgfqpoint{3.910575in}{0.504757in}}%
\pgfpathlineto{\pgfqpoint{3.964050in}{0.501932in}}%
\pgfpathlineto{\pgfqpoint{4.017525in}{0.501932in}}%
\pgfpathlineto{\pgfqpoint{4.071000in}{0.501932in}}%
\pgfpathlineto{\pgfqpoint{4.124475in}{0.501932in}}%
\pgfpathlineto{\pgfqpoint{4.177950in}{0.501932in}}%
\pgfpathlineto{\pgfqpoint{4.231425in}{0.501932in}}%
\pgfpathlineto{\pgfqpoint{4.284900in}{0.501527in}}%
\pgfpathlineto{\pgfqpoint{4.338375in}{0.501527in}}%
\pgfpathlineto{\pgfqpoint{4.391850in}{0.501527in}}%
\pgfpathlineto{\pgfqpoint{4.445325in}{0.501527in}}%
\pgfpathlineto{\pgfqpoint{4.498800in}{0.501527in}}%
\pgfpathlineto{\pgfqpoint{4.552275in}{0.501527in}}%
\pgfpathlineto{\pgfqpoint{4.605750in}{0.501527in}}%
\pgfpathlineto{\pgfqpoint{4.659225in}{0.501527in}}%
\pgfpathlineto{\pgfqpoint{4.712700in}{0.500993in}}%
\pgfpathlineto{\pgfqpoint{4.766175in}{0.500993in}}%
\pgfpathlineto{\pgfqpoint{4.819650in}{0.500993in}}%
\pgfpathlineto{\pgfqpoint{4.873125in}{0.500020in}}%
\pgfpathlineto{\pgfqpoint{4.926600in}{0.498803in}}%
\pgfpathlineto{\pgfqpoint{4.980075in}{0.498803in}}%
\pgfpathlineto{\pgfqpoint{5.033550in}{0.498803in}}%
\pgfpathlineto{\pgfqpoint{5.087025in}{0.498485in}}%
\pgfpathlineto{\pgfqpoint{5.140500in}{0.492782in}}%
\pgfpathlineto{\pgfqpoint{5.193975in}{0.492782in}}%
\pgfpathlineto{\pgfqpoint{5.247450in}{0.492782in}}%
\pgfpathlineto{\pgfqpoint{5.300925in}{0.492782in}}%
\pgfpathlineto{\pgfqpoint{5.354400in}{0.492782in}}%
\pgfpathlineto{\pgfqpoint{5.407875in}{0.485310in}}%
\pgfpathlineto{\pgfqpoint{5.461350in}{0.485310in}}%
\pgfpathlineto{\pgfqpoint{5.514825in}{0.485310in}}%
\pgfpathlineto{\pgfqpoint{5.568300in}{0.485310in}}%
\pgfpathlineto{\pgfqpoint{5.621775in}{0.485310in}}%
\pgfpathlineto{\pgfqpoint{5.675250in}{0.485310in}}%
\pgfpathlineto{\pgfqpoint{5.728725in}{0.485310in}}%
\pgfpathlineto{\pgfqpoint{5.782200in}{0.485310in}}%
\pgfpathlineto{\pgfqpoint{5.835675in}{0.485310in}}%
\pgfpathlineto{\pgfqpoint{5.889150in}{0.485310in}}%
\pgfpathlineto{\pgfqpoint{5.942625in}{0.485310in}}%
\pgfpathlineto{\pgfqpoint{5.996100in}{0.485310in}}%
\pgfpathlineto{\pgfqpoint{6.049575in}{0.483323in}}%
\pgfpathlineto{\pgfqpoint{6.103050in}{0.483323in}}%
\pgfpathlineto{\pgfqpoint{6.156525in}{0.483233in}}%
\pgfpathlineto{\pgfqpoint{6.210000in}{0.483233in}}%
\pgfpathlineto{\pgfqpoint{6.223889in}{0.483233in}}%
\pgfusepath{stroke}%
\end{pgfscope}%
\begin{pgfscope}%
\pgfpathrectangle{\pgfqpoint{0.862500in}{0.375000in}}{\pgfqpoint{5.347500in}{2.265000in}}%
\pgfusepath{clip}%
\pgfsetroundcap%
\pgfsetroundjoin%
\pgfsetlinewidth{1.505625pt}%
\definecolor{currentstroke}{rgb}{0.549020,0.337255,0.294118}%
\pgfsetstrokecolor{currentstroke}%
\pgfsetdash{}{0pt}%
\pgfpathmoveto{\pgfqpoint{0.862500in}{1.566967in}}%
\pgfpathlineto{\pgfqpoint{0.915975in}{1.556784in}}%
\pgfpathlineto{\pgfqpoint{0.969450in}{1.520461in}}%
\pgfpathlineto{\pgfqpoint{1.022925in}{1.493156in}}%
\pgfpathlineto{\pgfqpoint{1.076400in}{1.396187in}}%
\pgfpathlineto{\pgfqpoint{1.129875in}{1.396187in}}%
\pgfpathlineto{\pgfqpoint{1.183350in}{1.396187in}}%
\pgfpathlineto{\pgfqpoint{1.236825in}{1.214766in}}%
\pgfpathlineto{\pgfqpoint{1.290300in}{1.077029in}}%
\pgfpathlineto{\pgfqpoint{1.343775in}{1.077029in}}%
\pgfpathlineto{\pgfqpoint{1.397250in}{0.788853in}}%
\pgfpathlineto{\pgfqpoint{1.450725in}{0.673678in}}%
\pgfpathlineto{\pgfqpoint{1.504200in}{0.673678in}}%
\pgfpathlineto{\pgfqpoint{1.557675in}{0.552962in}}%
\pgfpathlineto{\pgfqpoint{1.611150in}{0.540843in}}%
\pgfpathlineto{\pgfqpoint{1.664625in}{0.540843in}}%
\pgfpathlineto{\pgfqpoint{1.718100in}{0.506321in}}%
\pgfpathlineto{\pgfqpoint{1.771575in}{0.506321in}}%
\pgfpathlineto{\pgfqpoint{1.825050in}{0.506321in}}%
\pgfpathlineto{\pgfqpoint{1.878525in}{0.506321in}}%
\pgfpathlineto{\pgfqpoint{1.932000in}{0.506321in}}%
\pgfpathlineto{\pgfqpoint{1.985475in}{0.506321in}}%
\pgfpathlineto{\pgfqpoint{2.038950in}{0.506321in}}%
\pgfpathlineto{\pgfqpoint{2.092425in}{0.506321in}}%
\pgfpathlineto{\pgfqpoint{2.145900in}{0.506321in}}%
\pgfpathlineto{\pgfqpoint{2.199375in}{0.506321in}}%
\pgfpathlineto{\pgfqpoint{2.252850in}{0.506321in}}%
\pgfpathlineto{\pgfqpoint{2.306325in}{0.506321in}}%
\pgfpathlineto{\pgfqpoint{2.359800in}{0.506321in}}%
\pgfpathlineto{\pgfqpoint{2.413275in}{0.506321in}}%
\pgfpathlineto{\pgfqpoint{2.466750in}{0.506321in}}%
\pgfpathlineto{\pgfqpoint{2.520225in}{0.504556in}}%
\pgfpathlineto{\pgfqpoint{2.573700in}{0.504556in}}%
\pgfpathlineto{\pgfqpoint{2.627175in}{0.504556in}}%
\pgfpathlineto{\pgfqpoint{2.680650in}{0.504556in}}%
\pgfpathlineto{\pgfqpoint{2.734125in}{0.504556in}}%
\pgfpathlineto{\pgfqpoint{2.787600in}{0.504556in}}%
\pgfpathlineto{\pgfqpoint{2.841075in}{0.504556in}}%
\pgfpathlineto{\pgfqpoint{2.894550in}{0.504556in}}%
\pgfpathlineto{\pgfqpoint{2.948025in}{0.504556in}}%
\pgfpathlineto{\pgfqpoint{3.001500in}{0.504556in}}%
\pgfpathlineto{\pgfqpoint{3.054975in}{0.504556in}}%
\pgfpathlineto{\pgfqpoint{3.108450in}{0.504556in}}%
\pgfpathlineto{\pgfqpoint{3.161925in}{0.498805in}}%
\pgfpathlineto{\pgfqpoint{3.215400in}{0.498805in}}%
\pgfpathlineto{\pgfqpoint{3.268875in}{0.487453in}}%
\pgfpathlineto{\pgfqpoint{3.322350in}{0.487453in}}%
\pgfpathlineto{\pgfqpoint{3.375825in}{0.487453in}}%
\pgfpathlineto{\pgfqpoint{3.429300in}{0.487453in}}%
\pgfpathlineto{\pgfqpoint{3.482775in}{0.487453in}}%
\pgfpathlineto{\pgfqpoint{3.536250in}{0.487453in}}%
\pgfpathlineto{\pgfqpoint{3.589725in}{0.487453in}}%
\pgfpathlineto{\pgfqpoint{3.643200in}{0.487453in}}%
\pgfpathlineto{\pgfqpoint{3.696675in}{0.487453in}}%
\pgfpathlineto{\pgfqpoint{3.750150in}{0.487453in}}%
\pgfpathlineto{\pgfqpoint{3.803625in}{0.487453in}}%
\pgfpathlineto{\pgfqpoint{3.857100in}{0.487453in}}%
\pgfpathlineto{\pgfqpoint{3.910575in}{0.487453in}}%
\pgfpathlineto{\pgfqpoint{3.964050in}{0.487453in}}%
\pgfpathlineto{\pgfqpoint{4.017525in}{0.487453in}}%
\pgfpathlineto{\pgfqpoint{4.071000in}{0.487453in}}%
\pgfpathlineto{\pgfqpoint{4.124475in}{0.487453in}}%
\pgfpathlineto{\pgfqpoint{4.177950in}{0.487453in}}%
\pgfpathlineto{\pgfqpoint{4.231425in}{0.487453in}}%
\pgfpathlineto{\pgfqpoint{4.284900in}{0.487453in}}%
\pgfpathlineto{\pgfqpoint{4.338375in}{0.484931in}}%
\pgfpathlineto{\pgfqpoint{4.391850in}{0.484931in}}%
\pgfpathlineto{\pgfqpoint{4.445325in}{0.484931in}}%
\pgfpathlineto{\pgfqpoint{4.498800in}{0.484931in}}%
\pgfpathlineto{\pgfqpoint{4.552275in}{0.484931in}}%
\pgfpathlineto{\pgfqpoint{4.605750in}{0.484931in}}%
\pgfpathlineto{\pgfqpoint{4.659225in}{0.484931in}}%
\pgfpathlineto{\pgfqpoint{4.712700in}{0.484931in}}%
\pgfpathlineto{\pgfqpoint{4.766175in}{0.484931in}}%
\pgfpathlineto{\pgfqpoint{4.819650in}{0.484931in}}%
\pgfpathlineto{\pgfqpoint{4.873125in}{0.484931in}}%
\pgfpathlineto{\pgfqpoint{4.926600in}{0.483742in}}%
\pgfpathlineto{\pgfqpoint{4.980075in}{0.483742in}}%
\pgfpathlineto{\pgfqpoint{5.033550in}{0.483264in}}%
\pgfpathlineto{\pgfqpoint{5.087025in}{0.479392in}}%
\pgfpathlineto{\pgfqpoint{5.140500in}{0.479392in}}%
\pgfpathlineto{\pgfqpoint{5.193975in}{0.479392in}}%
\pgfpathlineto{\pgfqpoint{5.247450in}{0.479392in}}%
\pgfpathlineto{\pgfqpoint{5.300925in}{0.479392in}}%
\pgfpathlineto{\pgfqpoint{5.354400in}{0.479392in}}%
\pgfpathlineto{\pgfqpoint{5.407875in}{0.479392in}}%
\pgfpathlineto{\pgfqpoint{5.461350in}{0.478820in}}%
\pgfpathlineto{\pgfqpoint{5.514825in}{0.478820in}}%
\pgfpathlineto{\pgfqpoint{5.568300in}{0.478820in}}%
\pgfpathlineto{\pgfqpoint{5.621775in}{0.478820in}}%
\pgfpathlineto{\pgfqpoint{5.675250in}{0.478820in}}%
\pgfpathlineto{\pgfqpoint{5.728725in}{0.478820in}}%
\pgfpathlineto{\pgfqpoint{5.782200in}{0.478820in}}%
\pgfpathlineto{\pgfqpoint{5.835675in}{0.478820in}}%
\pgfpathlineto{\pgfqpoint{5.889150in}{0.478820in}}%
\pgfpathlineto{\pgfqpoint{5.942625in}{0.478820in}}%
\pgfpathlineto{\pgfqpoint{5.996100in}{0.478820in}}%
\pgfpathlineto{\pgfqpoint{6.049575in}{0.478820in}}%
\pgfpathlineto{\pgfqpoint{6.103050in}{0.478820in}}%
\pgfpathlineto{\pgfqpoint{6.156525in}{0.478820in}}%
\pgfpathlineto{\pgfqpoint{6.210000in}{0.478820in}}%
\pgfpathlineto{\pgfqpoint{6.223889in}{0.478820in}}%
\pgfusepath{stroke}%
\end{pgfscope}%
\begin{pgfscope}%
\pgfpathrectangle{\pgfqpoint{0.862500in}{0.375000in}}{\pgfqpoint{5.347500in}{2.265000in}}%
\pgfusepath{clip}%
\pgfsetroundcap%
\pgfsetroundjoin%
\pgfsetlinewidth{1.505625pt}%
\definecolor{currentstroke}{rgb}{0.890196,0.466667,0.760784}%
\pgfsetstrokecolor{currentstroke}%
\pgfsetdash{}{0pt}%
\pgfpathmoveto{\pgfqpoint{0.862500in}{1.786526in}}%
\pgfpathlineto{\pgfqpoint{0.915975in}{1.572112in}}%
\pgfpathlineto{\pgfqpoint{0.969450in}{1.329725in}}%
\pgfpathlineto{\pgfqpoint{1.022925in}{1.324825in}}%
\pgfpathlineto{\pgfqpoint{1.076400in}{1.324825in}}%
\pgfpathlineto{\pgfqpoint{1.129875in}{1.253832in}}%
\pgfpathlineto{\pgfqpoint{1.183350in}{1.253832in}}%
\pgfpathlineto{\pgfqpoint{1.236825in}{1.253832in}}%
\pgfpathlineto{\pgfqpoint{1.290300in}{1.253832in}}%
\pgfpathlineto{\pgfqpoint{1.343775in}{1.253832in}}%
\pgfpathlineto{\pgfqpoint{1.397250in}{1.253832in}}%
\pgfpathlineto{\pgfqpoint{1.450725in}{1.253832in}}%
\pgfpathlineto{\pgfqpoint{1.504200in}{1.223336in}}%
\pgfpathlineto{\pgfqpoint{1.557675in}{1.150115in}}%
\pgfpathlineto{\pgfqpoint{1.611150in}{1.067117in}}%
\pgfpathlineto{\pgfqpoint{1.664625in}{1.067117in}}%
\pgfpathlineto{\pgfqpoint{1.718100in}{1.067117in}}%
\pgfpathlineto{\pgfqpoint{1.771575in}{1.067117in}}%
\pgfpathlineto{\pgfqpoint{1.825050in}{1.067117in}}%
\pgfpathlineto{\pgfqpoint{1.878525in}{1.067117in}}%
\pgfpathlineto{\pgfqpoint{1.932000in}{1.067117in}}%
\pgfpathlineto{\pgfqpoint{1.985475in}{1.067117in}}%
\pgfpathlineto{\pgfqpoint{2.038950in}{1.067117in}}%
\pgfpathlineto{\pgfqpoint{2.092425in}{1.067117in}}%
\pgfpathlineto{\pgfqpoint{2.145900in}{1.067117in}}%
\pgfpathlineto{\pgfqpoint{2.199375in}{1.067117in}}%
\pgfpathlineto{\pgfqpoint{2.252850in}{1.067117in}}%
\pgfpathlineto{\pgfqpoint{2.306325in}{1.067117in}}%
\pgfpathlineto{\pgfqpoint{2.359800in}{1.067117in}}%
\pgfpathlineto{\pgfqpoint{2.413275in}{1.067117in}}%
\pgfpathlineto{\pgfqpoint{2.466750in}{1.067117in}}%
\pgfpathlineto{\pgfqpoint{2.520225in}{1.023385in}}%
\pgfpathlineto{\pgfqpoint{2.573700in}{1.023385in}}%
\pgfpathlineto{\pgfqpoint{2.627175in}{1.023385in}}%
\pgfpathlineto{\pgfqpoint{2.680650in}{1.023385in}}%
\pgfpathlineto{\pgfqpoint{2.734125in}{1.023385in}}%
\pgfpathlineto{\pgfqpoint{2.787600in}{0.961784in}}%
\pgfpathlineto{\pgfqpoint{2.841075in}{0.961784in}}%
\pgfpathlineto{\pgfqpoint{2.894550in}{0.961784in}}%
\pgfpathlineto{\pgfqpoint{2.948025in}{0.961784in}}%
\pgfpathlineto{\pgfqpoint{3.001500in}{0.961784in}}%
\pgfpathlineto{\pgfqpoint{3.054975in}{0.918382in}}%
\pgfpathlineto{\pgfqpoint{3.108450in}{0.918382in}}%
\pgfpathlineto{\pgfqpoint{3.161925in}{0.918382in}}%
\pgfpathlineto{\pgfqpoint{3.215400in}{0.918382in}}%
\pgfpathlineto{\pgfqpoint{3.268875in}{0.918382in}}%
\pgfpathlineto{\pgfqpoint{3.322350in}{0.918382in}}%
\pgfpathlineto{\pgfqpoint{3.375825in}{0.918382in}}%
\pgfpathlineto{\pgfqpoint{3.429300in}{0.918382in}}%
\pgfpathlineto{\pgfqpoint{3.482775in}{0.918382in}}%
\pgfpathlineto{\pgfqpoint{3.536250in}{0.918382in}}%
\pgfpathlineto{\pgfqpoint{3.589725in}{0.918382in}}%
\pgfpathlineto{\pgfqpoint{3.643200in}{0.918382in}}%
\pgfpathlineto{\pgfqpoint{3.696675in}{0.918382in}}%
\pgfpathlineto{\pgfqpoint{3.750150in}{0.918382in}}%
\pgfpathlineto{\pgfqpoint{3.803625in}{0.918382in}}%
\pgfpathlineto{\pgfqpoint{3.857100in}{0.918382in}}%
\pgfpathlineto{\pgfqpoint{3.910575in}{0.918382in}}%
\pgfpathlineto{\pgfqpoint{3.964050in}{0.918382in}}%
\pgfpathlineto{\pgfqpoint{4.017525in}{0.790005in}}%
\pgfpathlineto{\pgfqpoint{4.071000in}{0.790005in}}%
\pgfpathlineto{\pgfqpoint{4.124475in}{0.790005in}}%
\pgfpathlineto{\pgfqpoint{4.177950in}{0.790005in}}%
\pgfpathlineto{\pgfqpoint{4.231425in}{0.790005in}}%
\pgfpathlineto{\pgfqpoint{4.284900in}{0.790005in}}%
\pgfpathlineto{\pgfqpoint{4.338375in}{0.790005in}}%
\pgfpathlineto{\pgfqpoint{4.391850in}{0.790005in}}%
\pgfpathlineto{\pgfqpoint{4.445325in}{0.790005in}}%
\pgfpathlineto{\pgfqpoint{4.498800in}{0.790005in}}%
\pgfpathlineto{\pgfqpoint{4.552275in}{0.790005in}}%
\pgfpathlineto{\pgfqpoint{4.605750in}{0.790005in}}%
\pgfpathlineto{\pgfqpoint{4.659225in}{0.790005in}}%
\pgfpathlineto{\pgfqpoint{4.712700in}{0.790005in}}%
\pgfpathlineto{\pgfqpoint{4.766175in}{0.790005in}}%
\pgfpathlineto{\pgfqpoint{4.819650in}{0.790005in}}%
\pgfpathlineto{\pgfqpoint{4.873125in}{0.790005in}}%
\pgfpathlineto{\pgfqpoint{4.926600in}{0.790005in}}%
\pgfpathlineto{\pgfqpoint{4.980075in}{0.790005in}}%
\pgfpathlineto{\pgfqpoint{5.033550in}{0.790005in}}%
\pgfpathlineto{\pgfqpoint{5.087025in}{0.790005in}}%
\pgfpathlineto{\pgfqpoint{5.140500in}{0.790005in}}%
\pgfpathlineto{\pgfqpoint{5.193975in}{0.790005in}}%
\pgfpathlineto{\pgfqpoint{5.247450in}{0.790005in}}%
\pgfpathlineto{\pgfqpoint{5.300925in}{0.790005in}}%
\pgfpathlineto{\pgfqpoint{5.354400in}{0.790005in}}%
\pgfpathlineto{\pgfqpoint{5.407875in}{0.790005in}}%
\pgfpathlineto{\pgfqpoint{5.461350in}{0.790005in}}%
\pgfpathlineto{\pgfqpoint{5.514825in}{0.790005in}}%
\pgfpathlineto{\pgfqpoint{5.568300in}{0.790005in}}%
\pgfpathlineto{\pgfqpoint{5.621775in}{0.790005in}}%
\pgfpathlineto{\pgfqpoint{5.675250in}{0.790005in}}%
\pgfpathlineto{\pgfqpoint{5.728725in}{0.790005in}}%
\pgfpathlineto{\pgfqpoint{5.782200in}{0.790005in}}%
\pgfpathlineto{\pgfqpoint{5.835675in}{0.790005in}}%
\pgfpathlineto{\pgfqpoint{5.889150in}{0.790005in}}%
\pgfpathlineto{\pgfqpoint{5.942625in}{0.790005in}}%
\pgfpathlineto{\pgfqpoint{5.996100in}{0.790005in}}%
\pgfpathlineto{\pgfqpoint{6.049575in}{0.790005in}}%
\pgfpathlineto{\pgfqpoint{6.103050in}{0.790005in}}%
\pgfpathlineto{\pgfqpoint{6.156525in}{0.790005in}}%
\pgfpathlineto{\pgfqpoint{6.210000in}{0.790005in}}%
\pgfpathlineto{\pgfqpoint{6.223889in}{0.790005in}}%
\pgfusepath{stroke}%
\end{pgfscope}%
\begin{pgfscope}%
\pgfsetrectcap%
\pgfsetmiterjoin%
\pgfsetlinewidth{0.000000pt}%
\definecolor{currentstroke}{rgb}{1.000000,1.000000,1.000000}%
\pgfsetstrokecolor{currentstroke}%
\pgfsetdash{}{0pt}%
\pgfpathmoveto{\pgfqpoint{0.862500in}{0.375000in}}%
\pgfpathlineto{\pgfqpoint{0.862500in}{2.640000in}}%
\pgfusepath{}%
\end{pgfscope}%
\begin{pgfscope}%
\pgfsetrectcap%
\pgfsetmiterjoin%
\pgfsetlinewidth{0.000000pt}%
\definecolor{currentstroke}{rgb}{1.000000,1.000000,1.000000}%
\pgfsetstrokecolor{currentstroke}%
\pgfsetdash{}{0pt}%
\pgfpathmoveto{\pgfqpoint{6.210000in}{0.375000in}}%
\pgfpathlineto{\pgfqpoint{6.210000in}{2.640000in}}%
\pgfusepath{}%
\end{pgfscope}%
\begin{pgfscope}%
\pgfsetrectcap%
\pgfsetmiterjoin%
\pgfsetlinewidth{0.000000pt}%
\definecolor{currentstroke}{rgb}{1.000000,1.000000,1.000000}%
\pgfsetstrokecolor{currentstroke}%
\pgfsetdash{}{0pt}%
\pgfpathmoveto{\pgfqpoint{0.862500in}{0.375000in}}%
\pgfpathlineto{\pgfqpoint{6.210000in}{0.375000in}}%
\pgfusepath{}%
\end{pgfscope}%
\begin{pgfscope}%
\pgfsetrectcap%
\pgfsetmiterjoin%
\pgfsetlinewidth{0.000000pt}%
\definecolor{currentstroke}{rgb}{1.000000,1.000000,1.000000}%
\pgfsetstrokecolor{currentstroke}%
\pgfsetdash{}{0pt}%
\pgfpathmoveto{\pgfqpoint{0.862500in}{2.640000in}}%
\pgfpathlineto{\pgfqpoint{6.210000in}{2.640000in}}%
\pgfusepath{}%
\end{pgfscope}%
\begin{pgfscope}%
\definecolor{textcolor}{rgb}{0.150000,0.150000,0.150000}%
\pgfsetstrokecolor{textcolor}%
\pgfsetfillcolor{textcolor}%
\pgftext[x=3.536250in,y=2.723333in,,base]{\color{textcolor}\rmfamily\fontsize{8.000000}{9.600000}\selectfont Logistic Regression MNIST}%
\end{pgfscope}%
\begin{pgfscope}%
\pgfsetroundcap%
\pgfsetroundjoin%
\pgfsetlinewidth{1.505625pt}%
\definecolor{currentstroke}{rgb}{0.121569,0.466667,0.705882}%
\pgfsetstrokecolor{currentstroke}%
\pgfsetdash{}{0pt}%
\pgfpathmoveto{\pgfqpoint{4.250712in}{2.494470in}}%
\pgfpathlineto{\pgfqpoint{4.472934in}{2.494470in}}%
\pgfusepath{stroke}%
\end{pgfscope}%
\begin{pgfscope}%
\definecolor{textcolor}{rgb}{0.150000,0.150000,0.150000}%
\pgfsetstrokecolor{textcolor}%
\pgfsetfillcolor{textcolor}%
\pgftext[x=4.561823in,y=2.455582in,left,base]{\color{textcolor}\rmfamily\fontsize{8.000000}{9.600000}\selectfont 5 x DNGO fixed}%
\end{pgfscope}%
\begin{pgfscope}%
\pgfsetroundcap%
\pgfsetroundjoin%
\pgfsetlinewidth{1.505625pt}%
\definecolor{currentstroke}{rgb}{1.000000,0.498039,0.054902}%
\pgfsetstrokecolor{currentstroke}%
\pgfsetdash{}{0pt}%
\pgfpathmoveto{\pgfqpoint{4.250712in}{2.331385in}}%
\pgfpathlineto{\pgfqpoint{4.472934in}{2.331385in}}%
\pgfusepath{stroke}%
\end{pgfscope}%
\begin{pgfscope}%
\definecolor{textcolor}{rgb}{0.150000,0.150000,0.150000}%
\pgfsetstrokecolor{textcolor}%
\pgfsetfillcolor{textcolor}%
\pgftext[x=4.561823in,y=2.292496in,left,base]{\color{textcolor}\rmfamily\fontsize{8.000000}{9.600000}\selectfont DNGO fixed}%
\end{pgfscope}%
\begin{pgfscope}%
\pgfsetroundcap%
\pgfsetroundjoin%
\pgfsetlinewidth{1.505625pt}%
\definecolor{currentstroke}{rgb}{0.172549,0.627451,0.172549}%
\pgfsetstrokecolor{currentstroke}%
\pgfsetdash{}{0pt}%
\pgfpathmoveto{\pgfqpoint{4.250712in}{2.168299in}}%
\pgfpathlineto{\pgfqpoint{4.472934in}{2.168299in}}%
\pgfusepath{stroke}%
\end{pgfscope}%
\begin{pgfscope}%
\definecolor{textcolor}{rgb}{0.150000,0.150000,0.150000}%
\pgfsetstrokecolor{textcolor}%
\pgfsetfillcolor{textcolor}%
\pgftext[x=4.561823in,y=2.129410in,left,base]{\color{textcolor}\rmfamily\fontsize{8.000000}{9.600000}\selectfont DNGO retrain-reset}%
\end{pgfscope}%
\begin{pgfscope}%
\pgfsetroundcap%
\pgfsetroundjoin%
\pgfsetlinewidth{1.505625pt}%
\definecolor{currentstroke}{rgb}{0.839216,0.152941,0.156863}%
\pgfsetstrokecolor{currentstroke}%
\pgfsetdash{}{0pt}%
\pgfpathmoveto{\pgfqpoint{4.250712in}{2.005213in}}%
\pgfpathlineto{\pgfqpoint{4.472934in}{2.005213in}}%
\pgfusepath{stroke}%
\end{pgfscope}%
\begin{pgfscope}%
\definecolor{textcolor}{rgb}{0.150000,0.150000,0.150000}%
\pgfsetstrokecolor{textcolor}%
\pgfsetfillcolor{textcolor}%
\pgftext[x=4.561823in,y=1.966324in,left,base]{\color{textcolor}\rmfamily\fontsize{8.000000}{9.600000}\selectfont DNGO retrain-reset MCMC}%
\end{pgfscope}%
\begin{pgfscope}%
\pgfsetroundcap%
\pgfsetroundjoin%
\pgfsetlinewidth{1.505625pt}%
\definecolor{currentstroke}{rgb}{0.580392,0.403922,0.741176}%
\pgfsetstrokecolor{currentstroke}%
\pgfsetdash{}{0pt}%
\pgfpathmoveto{\pgfqpoint{4.250712in}{1.842127in}}%
\pgfpathlineto{\pgfqpoint{4.472934in}{1.842127in}}%
\pgfusepath{stroke}%
\end{pgfscope}%
\begin{pgfscope}%
\definecolor{textcolor}{rgb}{0.150000,0.150000,0.150000}%
\pgfsetstrokecolor{textcolor}%
\pgfsetfillcolor{textcolor}%
\pgftext[x=4.561823in,y=1.803238in,left,base]{\color{textcolor}\rmfamily\fontsize{8.000000}{9.600000}\selectfont GP}%
\end{pgfscope}%
\begin{pgfscope}%
\pgfsetroundcap%
\pgfsetroundjoin%
\pgfsetlinewidth{1.505625pt}%
\definecolor{currentstroke}{rgb}{0.549020,0.337255,0.294118}%
\pgfsetstrokecolor{currentstroke}%
\pgfsetdash{}{0pt}%
\pgfpathmoveto{\pgfqpoint{4.250712in}{1.679041in}}%
\pgfpathlineto{\pgfqpoint{4.472934in}{1.679041in}}%
\pgfusepath{stroke}%
\end{pgfscope}%
\begin{pgfscope}%
\definecolor{textcolor}{rgb}{0.150000,0.150000,0.150000}%
\pgfsetstrokecolor{textcolor}%
\pgfsetfillcolor{textcolor}%
\pgftext[x=4.561823in,y=1.640152in,left,base]{\color{textcolor}\rmfamily\fontsize{8.000000}{9.600000}\selectfont GP MCMC}%
\end{pgfscope}%
\begin{pgfscope}%
\pgfsetroundcap%
\pgfsetroundjoin%
\pgfsetlinewidth{1.505625pt}%
\definecolor{currentstroke}{rgb}{0.890196,0.466667,0.760784}%
\pgfsetstrokecolor{currentstroke}%
\pgfsetdash{}{0pt}%
\pgfpathmoveto{\pgfqpoint{4.250712in}{1.515955in}}%
\pgfpathlineto{\pgfqpoint{4.472934in}{1.515955in}}%
\pgfusepath{stroke}%
\end{pgfscope}%
\begin{pgfscope}%
\definecolor{textcolor}{rgb}{0.150000,0.150000,0.150000}%
\pgfsetstrokecolor{textcolor}%
\pgfsetfillcolor{textcolor}%
\pgftext[x=4.561823in,y=1.477067in,left,base]{\color{textcolor}\rmfamily\fontsize{8.000000}{9.600000}\selectfont Rand}%
\end{pgfscope}%
\end{pgfpicture}%
\makeatother%
\endgroup%

    \end{minipage}
    \vfill
    \subsection{Miscellaneous}\label{sec:extra}

    \begin{minipage}{\linewidth}
        \centering
        %% Creator: Matplotlib, PGF backend
%%
%% To include the figure in your LaTeX document, write
%%   \input{<filename>.pgf}
%%
%% Make sure the required packages are loaded in your preamble
%%   \usepackage{pgf}
%%
%% Figures using additional raster images can only be included by \input if
%% they are in the same directory as the main LaTeX file. For loading figures
%% from other directories you can use the `import` package
%%   \usepackage{import}
%% and then include the figures with
%%   \import{<path to file>}{<filename>.pgf}
%%
%% Matplotlib used the following preamble
%%   \usepackage{gensymb}
%%   \usepackage{fontspec}
%%   \setmainfont{DejaVu Serif}
%%   \setsansfont{Arial}
%%   \setmonofont{DejaVu Sans Mono}
%%
\begingroup%
\makeatletter%
\begin{pgfpicture}%
\pgfpathrectangle{\pgfpointorigin}{\pgfqpoint{6.900000in}{3.000000in}}%
\pgfusepath{use as bounding box, clip}%
\begin{pgfscope}%
\pgfsetbuttcap%
\pgfsetmiterjoin%
\definecolor{currentfill}{rgb}{1.000000,1.000000,1.000000}%
\pgfsetfillcolor{currentfill}%
\pgfsetlinewidth{0.000000pt}%
\definecolor{currentstroke}{rgb}{1.000000,1.000000,1.000000}%
\pgfsetstrokecolor{currentstroke}%
\pgfsetdash{}{0pt}%
\pgfpathmoveto{\pgfqpoint{0.000000in}{0.000000in}}%
\pgfpathlineto{\pgfqpoint{6.900000in}{0.000000in}}%
\pgfpathlineto{\pgfqpoint{6.900000in}{3.000000in}}%
\pgfpathlineto{\pgfqpoint{0.000000in}{3.000000in}}%
\pgfpathclose%
\pgfusepath{fill}%
\end{pgfscope}%
\begin{pgfscope}%
\pgfsetbuttcap%
\pgfsetmiterjoin%
\definecolor{currentfill}{rgb}{0.917647,0.917647,0.949020}%
\pgfsetfillcolor{currentfill}%
\pgfsetlinewidth{0.000000pt}%
\definecolor{currentstroke}{rgb}{0.000000,0.000000,0.000000}%
\pgfsetstrokecolor{currentstroke}%
\pgfsetstrokeopacity{0.000000}%
\pgfsetdash{}{0pt}%
\pgfpathmoveto{\pgfqpoint{0.862500in}{0.375000in}}%
\pgfpathlineto{\pgfqpoint{6.210000in}{0.375000in}}%
\pgfpathlineto{\pgfqpoint{6.210000in}{2.640000in}}%
\pgfpathlineto{\pgfqpoint{0.862500in}{2.640000in}}%
\pgfpathclose%
\pgfusepath{fill}%
\end{pgfscope}%
\begin{pgfscope}%
\pgfpathrectangle{\pgfqpoint{0.862500in}{0.375000in}}{\pgfqpoint{5.347500in}{2.265000in}}%
\pgfusepath{clip}%
\pgfsetroundcap%
\pgfsetroundjoin%
\pgfsetlinewidth{0.803000pt}%
\definecolor{currentstroke}{rgb}{1.000000,1.000000,1.000000}%
\pgfsetstrokecolor{currentstroke}%
\pgfsetdash{}{0pt}%
\pgfpathmoveto{\pgfqpoint{1.105568in}{0.375000in}}%
\pgfpathlineto{\pgfqpoint{1.105568in}{2.640000in}}%
\pgfusepath{stroke}%
\end{pgfscope}%
\begin{pgfscope}%
\definecolor{textcolor}{rgb}{0.150000,0.150000,0.150000}%
\pgfsetstrokecolor{textcolor}%
\pgfsetfillcolor{textcolor}%
\pgftext[x=1.105568in,y=0.326389in,,top]{\color{textcolor}\rmfamily\fontsize{8.000000}{9.600000}\selectfont \(\displaystyle 0\)}%
\end{pgfscope}%
\begin{pgfscope}%
\pgfpathrectangle{\pgfqpoint{0.862500in}{0.375000in}}{\pgfqpoint{5.347500in}{2.265000in}}%
\pgfusepath{clip}%
\pgfsetroundcap%
\pgfsetroundjoin%
\pgfsetlinewidth{0.803000pt}%
\definecolor{currentstroke}{rgb}{1.000000,1.000000,1.000000}%
\pgfsetstrokecolor{currentstroke}%
\pgfsetdash{}{0pt}%
\pgfpathmoveto{\pgfqpoint{2.081746in}{0.375000in}}%
\pgfpathlineto{\pgfqpoint{2.081746in}{2.640000in}}%
\pgfusepath{stroke}%
\end{pgfscope}%
\begin{pgfscope}%
\definecolor{textcolor}{rgb}{0.150000,0.150000,0.150000}%
\pgfsetstrokecolor{textcolor}%
\pgfsetfillcolor{textcolor}%
\pgftext[x=2.081746in,y=0.326389in,,top]{\color{textcolor}\rmfamily\fontsize{8.000000}{9.600000}\selectfont \(\displaystyle 50\)}%
\end{pgfscope}%
\begin{pgfscope}%
\pgfpathrectangle{\pgfqpoint{0.862500in}{0.375000in}}{\pgfqpoint{5.347500in}{2.265000in}}%
\pgfusepath{clip}%
\pgfsetroundcap%
\pgfsetroundjoin%
\pgfsetlinewidth{0.803000pt}%
\definecolor{currentstroke}{rgb}{1.000000,1.000000,1.000000}%
\pgfsetstrokecolor{currentstroke}%
\pgfsetdash{}{0pt}%
\pgfpathmoveto{\pgfqpoint{3.057923in}{0.375000in}}%
\pgfpathlineto{\pgfqpoint{3.057923in}{2.640000in}}%
\pgfusepath{stroke}%
\end{pgfscope}%
\begin{pgfscope}%
\definecolor{textcolor}{rgb}{0.150000,0.150000,0.150000}%
\pgfsetstrokecolor{textcolor}%
\pgfsetfillcolor{textcolor}%
\pgftext[x=3.057923in,y=0.326389in,,top]{\color{textcolor}\rmfamily\fontsize{8.000000}{9.600000}\selectfont \(\displaystyle 100\)}%
\end{pgfscope}%
\begin{pgfscope}%
\pgfpathrectangle{\pgfqpoint{0.862500in}{0.375000in}}{\pgfqpoint{5.347500in}{2.265000in}}%
\pgfusepath{clip}%
\pgfsetroundcap%
\pgfsetroundjoin%
\pgfsetlinewidth{0.803000pt}%
\definecolor{currentstroke}{rgb}{1.000000,1.000000,1.000000}%
\pgfsetstrokecolor{currentstroke}%
\pgfsetdash{}{0pt}%
\pgfpathmoveto{\pgfqpoint{4.034100in}{0.375000in}}%
\pgfpathlineto{\pgfqpoint{4.034100in}{2.640000in}}%
\pgfusepath{stroke}%
\end{pgfscope}%
\begin{pgfscope}%
\definecolor{textcolor}{rgb}{0.150000,0.150000,0.150000}%
\pgfsetstrokecolor{textcolor}%
\pgfsetfillcolor{textcolor}%
\pgftext[x=4.034100in,y=0.326389in,,top]{\color{textcolor}\rmfamily\fontsize{8.000000}{9.600000}\selectfont \(\displaystyle 150\)}%
\end{pgfscope}%
\begin{pgfscope}%
\pgfpathrectangle{\pgfqpoint{0.862500in}{0.375000in}}{\pgfqpoint{5.347500in}{2.265000in}}%
\pgfusepath{clip}%
\pgfsetroundcap%
\pgfsetroundjoin%
\pgfsetlinewidth{0.803000pt}%
\definecolor{currentstroke}{rgb}{1.000000,1.000000,1.000000}%
\pgfsetstrokecolor{currentstroke}%
\pgfsetdash{}{0pt}%
\pgfpathmoveto{\pgfqpoint{5.010278in}{0.375000in}}%
\pgfpathlineto{\pgfqpoint{5.010278in}{2.640000in}}%
\pgfusepath{stroke}%
\end{pgfscope}%
\begin{pgfscope}%
\definecolor{textcolor}{rgb}{0.150000,0.150000,0.150000}%
\pgfsetstrokecolor{textcolor}%
\pgfsetfillcolor{textcolor}%
\pgftext[x=5.010278in,y=0.326389in,,top]{\color{textcolor}\rmfamily\fontsize{8.000000}{9.600000}\selectfont \(\displaystyle 200\)}%
\end{pgfscope}%
\begin{pgfscope}%
\pgfpathrectangle{\pgfqpoint{0.862500in}{0.375000in}}{\pgfqpoint{5.347500in}{2.265000in}}%
\pgfusepath{clip}%
\pgfsetroundcap%
\pgfsetroundjoin%
\pgfsetlinewidth{0.803000pt}%
\definecolor{currentstroke}{rgb}{1.000000,1.000000,1.000000}%
\pgfsetstrokecolor{currentstroke}%
\pgfsetdash{}{0pt}%
\pgfpathmoveto{\pgfqpoint{5.986455in}{0.375000in}}%
\pgfpathlineto{\pgfqpoint{5.986455in}{2.640000in}}%
\pgfusepath{stroke}%
\end{pgfscope}%
\begin{pgfscope}%
\definecolor{textcolor}{rgb}{0.150000,0.150000,0.150000}%
\pgfsetstrokecolor{textcolor}%
\pgfsetfillcolor{textcolor}%
\pgftext[x=5.986455in,y=0.326389in,,top]{\color{textcolor}\rmfamily\fontsize{8.000000}{9.600000}\selectfont \(\displaystyle 250\)}%
\end{pgfscope}%
\begin{pgfscope}%
\definecolor{textcolor}{rgb}{0.150000,0.150000,0.150000}%
\pgfsetstrokecolor{textcolor}%
\pgfsetfillcolor{textcolor}%
\pgftext[x=3.536250in,y=0.163303in,,top]{\color{textcolor}\rmfamily\fontsize{8.000000}{9.600000}\selectfont Step}%
\end{pgfscope}%
\begin{pgfscope}%
\pgfpathrectangle{\pgfqpoint{0.862500in}{0.375000in}}{\pgfqpoint{5.347500in}{2.265000in}}%
\pgfusepath{clip}%
\pgfsetroundcap%
\pgfsetroundjoin%
\pgfsetlinewidth{0.803000pt}%
\definecolor{currentstroke}{rgb}{1.000000,1.000000,1.000000}%
\pgfsetstrokecolor{currentstroke}%
\pgfsetdash{}{0pt}%
\pgfpathmoveto{\pgfqpoint{0.862500in}{0.419743in}}%
\pgfpathlineto{\pgfqpoint{6.210000in}{0.419743in}}%
\pgfusepath{stroke}%
\end{pgfscope}%
\begin{pgfscope}%
\definecolor{textcolor}{rgb}{0.150000,0.150000,0.150000}%
\pgfsetstrokecolor{textcolor}%
\pgfsetfillcolor{textcolor}%
\pgftext[x=0.557716in,y=0.377533in,left,base]{\color{textcolor}\rmfamily\fontsize{8.000000}{9.600000}\selectfont \(\displaystyle 10^{-4}\)}%
\end{pgfscope}%
\begin{pgfscope}%
\pgfpathrectangle{\pgfqpoint{0.862500in}{0.375000in}}{\pgfqpoint{5.347500in}{2.265000in}}%
\pgfusepath{clip}%
\pgfsetroundcap%
\pgfsetroundjoin%
\pgfsetlinewidth{0.803000pt}%
\definecolor{currentstroke}{rgb}{1.000000,1.000000,1.000000}%
\pgfsetstrokecolor{currentstroke}%
\pgfsetdash{}{0pt}%
\pgfpathmoveto{\pgfqpoint{0.862500in}{0.883447in}}%
\pgfpathlineto{\pgfqpoint{6.210000in}{0.883447in}}%
\pgfusepath{stroke}%
\end{pgfscope}%
\begin{pgfscope}%
\definecolor{textcolor}{rgb}{0.150000,0.150000,0.150000}%
\pgfsetstrokecolor{textcolor}%
\pgfsetfillcolor{textcolor}%
\pgftext[x=0.557716in,y=0.841238in,left,base]{\color{textcolor}\rmfamily\fontsize{8.000000}{9.600000}\selectfont \(\displaystyle 10^{-3}\)}%
\end{pgfscope}%
\begin{pgfscope}%
\pgfpathrectangle{\pgfqpoint{0.862500in}{0.375000in}}{\pgfqpoint{5.347500in}{2.265000in}}%
\pgfusepath{clip}%
\pgfsetroundcap%
\pgfsetroundjoin%
\pgfsetlinewidth{0.803000pt}%
\definecolor{currentstroke}{rgb}{1.000000,1.000000,1.000000}%
\pgfsetstrokecolor{currentstroke}%
\pgfsetdash{}{0pt}%
\pgfpathmoveto{\pgfqpoint{0.862500in}{1.347152in}}%
\pgfpathlineto{\pgfqpoint{6.210000in}{1.347152in}}%
\pgfusepath{stroke}%
\end{pgfscope}%
\begin{pgfscope}%
\definecolor{textcolor}{rgb}{0.150000,0.150000,0.150000}%
\pgfsetstrokecolor{textcolor}%
\pgfsetfillcolor{textcolor}%
\pgftext[x=0.557716in,y=1.304943in,left,base]{\color{textcolor}\rmfamily\fontsize{8.000000}{9.600000}\selectfont \(\displaystyle 10^{-2}\)}%
\end{pgfscope}%
\begin{pgfscope}%
\pgfpathrectangle{\pgfqpoint{0.862500in}{0.375000in}}{\pgfqpoint{5.347500in}{2.265000in}}%
\pgfusepath{clip}%
\pgfsetroundcap%
\pgfsetroundjoin%
\pgfsetlinewidth{0.803000pt}%
\definecolor{currentstroke}{rgb}{1.000000,1.000000,1.000000}%
\pgfsetstrokecolor{currentstroke}%
\pgfsetdash{}{0pt}%
\pgfpathmoveto{\pgfqpoint{0.862500in}{1.810857in}}%
\pgfpathlineto{\pgfqpoint{6.210000in}{1.810857in}}%
\pgfusepath{stroke}%
\end{pgfscope}%
\begin{pgfscope}%
\definecolor{textcolor}{rgb}{0.150000,0.150000,0.150000}%
\pgfsetstrokecolor{textcolor}%
\pgfsetfillcolor{textcolor}%
\pgftext[x=0.557716in,y=1.768648in,left,base]{\color{textcolor}\rmfamily\fontsize{8.000000}{9.600000}\selectfont \(\displaystyle 10^{-1}\)}%
\end{pgfscope}%
\begin{pgfscope}%
\pgfpathrectangle{\pgfqpoint{0.862500in}{0.375000in}}{\pgfqpoint{5.347500in}{2.265000in}}%
\pgfusepath{clip}%
\pgfsetroundcap%
\pgfsetroundjoin%
\pgfsetlinewidth{0.803000pt}%
\definecolor{currentstroke}{rgb}{1.000000,1.000000,1.000000}%
\pgfsetstrokecolor{currentstroke}%
\pgfsetdash{}{0pt}%
\pgfpathmoveto{\pgfqpoint{0.862500in}{2.274562in}}%
\pgfpathlineto{\pgfqpoint{6.210000in}{2.274562in}}%
\pgfusepath{stroke}%
\end{pgfscope}%
\begin{pgfscope}%
\definecolor{textcolor}{rgb}{0.150000,0.150000,0.150000}%
\pgfsetstrokecolor{textcolor}%
\pgfsetfillcolor{textcolor}%
\pgftext[x=0.637962in,y=2.232353in,left,base]{\color{textcolor}\rmfamily\fontsize{8.000000}{9.600000}\selectfont \(\displaystyle 10^{0}\)}%
\end{pgfscope}%
\begin{pgfscope}%
\definecolor{textcolor}{rgb}{0.150000,0.150000,0.150000}%
\pgfsetstrokecolor{textcolor}%
\pgfsetfillcolor{textcolor}%
\pgftext[x=0.502160in,y=1.507500in,,bottom,rotate=90.000000]{\color{textcolor}\rmfamily\fontsize{8.000000}{9.600000}\selectfont Simple Regret}%
\end{pgfscope}%
\begin{pgfscope}%
\pgfpathrectangle{\pgfqpoint{0.862500in}{0.375000in}}{\pgfqpoint{5.347500in}{2.265000in}}%
\pgfusepath{clip}%
\pgfsetbuttcap%
\pgfsetroundjoin%
\definecolor{currentfill}{rgb}{0.121569,0.466667,0.705882}%
\pgfsetfillcolor{currentfill}%
\pgfsetfillopacity{0.200000}%
\pgfsetlinewidth{0.000000pt}%
\definecolor{currentstroke}{rgb}{0.000000,0.000000,0.000000}%
\pgfsetstrokecolor{currentstroke}%
\pgfsetdash{}{0pt}%
\pgfpathmoveto{\pgfqpoint{1.105568in}{2.501020in}}%
\pgfpathlineto{\pgfqpoint{1.105568in}{2.537045in}}%
\pgfpathlineto{\pgfqpoint{1.125092in}{2.503180in}}%
\pgfpathlineto{\pgfqpoint{1.144615in}{2.490730in}}%
\pgfpathlineto{\pgfqpoint{1.164139in}{2.467225in}}%
\pgfpathlineto{\pgfqpoint{1.183662in}{2.439668in}}%
\pgfpathlineto{\pgfqpoint{1.203186in}{2.411714in}}%
\pgfpathlineto{\pgfqpoint{1.222709in}{2.411714in}}%
\pgfpathlineto{\pgfqpoint{1.242233in}{2.374223in}}%
\pgfpathlineto{\pgfqpoint{1.261757in}{2.372552in}}%
\pgfpathlineto{\pgfqpoint{1.281280in}{2.372552in}}%
\pgfpathlineto{\pgfqpoint{1.300804in}{2.366659in}}%
\pgfpathlineto{\pgfqpoint{1.320327in}{2.345968in}}%
\pgfpathlineto{\pgfqpoint{1.339851in}{2.345968in}}%
\pgfpathlineto{\pgfqpoint{1.359374in}{2.345968in}}%
\pgfpathlineto{\pgfqpoint{1.378898in}{2.322568in}}%
\pgfpathlineto{\pgfqpoint{1.398421in}{2.322568in}}%
\pgfpathlineto{\pgfqpoint{1.417945in}{2.293599in}}%
\pgfpathlineto{\pgfqpoint{1.437469in}{2.293599in}}%
\pgfpathlineto{\pgfqpoint{1.456992in}{2.293599in}}%
\pgfpathlineto{\pgfqpoint{1.476516in}{2.293599in}}%
\pgfpathlineto{\pgfqpoint{1.496039in}{2.285192in}}%
\pgfpathlineto{\pgfqpoint{1.515563in}{2.285192in}}%
\pgfpathlineto{\pgfqpoint{1.535086in}{2.275426in}}%
\pgfpathlineto{\pgfqpoint{1.554610in}{2.275426in}}%
\pgfpathlineto{\pgfqpoint{1.574133in}{2.261779in}}%
\pgfpathlineto{\pgfqpoint{1.593657in}{2.247985in}}%
\pgfpathlineto{\pgfqpoint{1.613180in}{2.226201in}}%
\pgfpathlineto{\pgfqpoint{1.632704in}{2.226201in}}%
\pgfpathlineto{\pgfqpoint{1.652228in}{2.226201in}}%
\pgfpathlineto{\pgfqpoint{1.671751in}{2.226201in}}%
\pgfpathlineto{\pgfqpoint{1.691275in}{2.226201in}}%
\pgfpathlineto{\pgfqpoint{1.710798in}{2.226201in}}%
\pgfpathlineto{\pgfqpoint{1.730322in}{2.225234in}}%
\pgfpathlineto{\pgfqpoint{1.749845in}{2.205908in}}%
\pgfpathlineto{\pgfqpoint{1.769369in}{2.205908in}}%
\pgfpathlineto{\pgfqpoint{1.788892in}{2.205908in}}%
\pgfpathlineto{\pgfqpoint{1.808416in}{2.205908in}}%
\pgfpathlineto{\pgfqpoint{1.827939in}{2.205908in}}%
\pgfpathlineto{\pgfqpoint{1.847463in}{2.205908in}}%
\pgfpathlineto{\pgfqpoint{1.866987in}{2.205908in}}%
\pgfpathlineto{\pgfqpoint{1.886510in}{2.203902in}}%
\pgfpathlineto{\pgfqpoint{1.906034in}{2.203902in}}%
\pgfpathlineto{\pgfqpoint{1.925557in}{2.203902in}}%
\pgfpathlineto{\pgfqpoint{1.945081in}{2.203902in}}%
\pgfpathlineto{\pgfqpoint{1.964604in}{2.203902in}}%
\pgfpathlineto{\pgfqpoint{1.984128in}{2.185764in}}%
\pgfpathlineto{\pgfqpoint{2.003651in}{2.168831in}}%
\pgfpathlineto{\pgfqpoint{2.023175in}{2.165054in}}%
\pgfpathlineto{\pgfqpoint{2.042699in}{2.165054in}}%
\pgfpathlineto{\pgfqpoint{2.062222in}{2.165054in}}%
\pgfpathlineto{\pgfqpoint{2.081746in}{2.165054in}}%
\pgfpathlineto{\pgfqpoint{2.101269in}{2.165054in}}%
\pgfpathlineto{\pgfqpoint{2.120793in}{2.165054in}}%
\pgfpathlineto{\pgfqpoint{2.140316in}{2.165054in}}%
\pgfpathlineto{\pgfqpoint{2.159840in}{2.125615in}}%
\pgfpathlineto{\pgfqpoint{2.179363in}{2.125615in}}%
\pgfpathlineto{\pgfqpoint{2.198887in}{2.120971in}}%
\pgfpathlineto{\pgfqpoint{2.218410in}{2.120971in}}%
\pgfpathlineto{\pgfqpoint{2.237934in}{2.120971in}}%
\pgfpathlineto{\pgfqpoint{2.257458in}{2.120971in}}%
\pgfpathlineto{\pgfqpoint{2.276981in}{2.120971in}}%
\pgfpathlineto{\pgfqpoint{2.296505in}{2.120971in}}%
\pgfpathlineto{\pgfqpoint{2.316028in}{2.120971in}}%
\pgfpathlineto{\pgfqpoint{2.335552in}{2.120971in}}%
\pgfpathlineto{\pgfqpoint{2.355075in}{2.120971in}}%
\pgfpathlineto{\pgfqpoint{2.374599in}{2.120971in}}%
\pgfpathlineto{\pgfqpoint{2.394122in}{2.120971in}}%
\pgfpathlineto{\pgfqpoint{2.413646in}{2.120971in}}%
\pgfpathlineto{\pgfqpoint{2.433169in}{2.120971in}}%
\pgfpathlineto{\pgfqpoint{2.452693in}{2.120971in}}%
\pgfpathlineto{\pgfqpoint{2.472217in}{2.120971in}}%
\pgfpathlineto{\pgfqpoint{2.491740in}{2.120971in}}%
\pgfpathlineto{\pgfqpoint{2.511264in}{2.120971in}}%
\pgfpathlineto{\pgfqpoint{2.530787in}{2.120971in}}%
\pgfpathlineto{\pgfqpoint{2.550311in}{2.120971in}}%
\pgfpathlineto{\pgfqpoint{2.569834in}{2.120971in}}%
\pgfpathlineto{\pgfqpoint{2.589358in}{2.120971in}}%
\pgfpathlineto{\pgfqpoint{2.608881in}{2.120971in}}%
\pgfpathlineto{\pgfqpoint{2.628405in}{2.120971in}}%
\pgfpathlineto{\pgfqpoint{2.647929in}{2.120971in}}%
\pgfpathlineto{\pgfqpoint{2.667452in}{2.108715in}}%
\pgfpathlineto{\pgfqpoint{2.686976in}{2.108715in}}%
\pgfpathlineto{\pgfqpoint{2.706499in}{2.108715in}}%
\pgfpathlineto{\pgfqpoint{2.726023in}{2.108715in}}%
\pgfpathlineto{\pgfqpoint{2.745546in}{2.108715in}}%
\pgfpathlineto{\pgfqpoint{2.765070in}{2.080724in}}%
\pgfpathlineto{\pgfqpoint{2.784593in}{2.080724in}}%
\pgfpathlineto{\pgfqpoint{2.804117in}{2.080724in}}%
\pgfpathlineto{\pgfqpoint{2.823640in}{2.080724in}}%
\pgfpathlineto{\pgfqpoint{2.843164in}{2.074129in}}%
\pgfpathlineto{\pgfqpoint{2.862688in}{2.074129in}}%
\pgfpathlineto{\pgfqpoint{2.882211in}{2.074129in}}%
\pgfpathlineto{\pgfqpoint{2.901735in}{2.074129in}}%
\pgfpathlineto{\pgfqpoint{2.921258in}{2.074129in}}%
\pgfpathlineto{\pgfqpoint{2.940782in}{2.067961in}}%
\pgfpathlineto{\pgfqpoint{2.960305in}{2.067961in}}%
\pgfpathlineto{\pgfqpoint{2.979829in}{2.067961in}}%
\pgfpathlineto{\pgfqpoint{2.999352in}{2.067961in}}%
\pgfpathlineto{\pgfqpoint{3.018876in}{2.067961in}}%
\pgfpathlineto{\pgfqpoint{3.038400in}{2.067961in}}%
\pgfpathlineto{\pgfqpoint{3.057923in}{2.067961in}}%
\pgfpathlineto{\pgfqpoint{3.077447in}{2.067961in}}%
\pgfpathlineto{\pgfqpoint{3.096970in}{2.067961in}}%
\pgfpathlineto{\pgfqpoint{3.116494in}{2.067961in}}%
\pgfpathlineto{\pgfqpoint{3.136017in}{2.067961in}}%
\pgfpathlineto{\pgfqpoint{3.155541in}{2.067961in}}%
\pgfpathlineto{\pgfqpoint{3.175064in}{2.067399in}}%
\pgfpathlineto{\pgfqpoint{3.194588in}{2.063744in}}%
\pgfpathlineto{\pgfqpoint{3.214111in}{2.063744in}}%
\pgfpathlineto{\pgfqpoint{3.233635in}{2.063744in}}%
\pgfpathlineto{\pgfqpoint{3.253159in}{2.063744in}}%
\pgfpathlineto{\pgfqpoint{3.272682in}{2.063744in}}%
\pgfpathlineto{\pgfqpoint{3.292206in}{2.063744in}}%
\pgfpathlineto{\pgfqpoint{3.311729in}{2.063744in}}%
\pgfpathlineto{\pgfqpoint{3.331253in}{2.063744in}}%
\pgfpathlineto{\pgfqpoint{3.350776in}{2.063744in}}%
\pgfpathlineto{\pgfqpoint{3.370300in}{2.063744in}}%
\pgfpathlineto{\pgfqpoint{3.389823in}{2.063744in}}%
\pgfpathlineto{\pgfqpoint{3.409347in}{2.063744in}}%
\pgfpathlineto{\pgfqpoint{3.428870in}{2.063744in}}%
\pgfpathlineto{\pgfqpoint{3.448394in}{2.063744in}}%
\pgfpathlineto{\pgfqpoint{3.467918in}{2.063744in}}%
\pgfpathlineto{\pgfqpoint{3.487441in}{2.063744in}}%
\pgfpathlineto{\pgfqpoint{3.506965in}{2.063744in}}%
\pgfpathlineto{\pgfqpoint{3.526488in}{2.063744in}}%
\pgfpathlineto{\pgfqpoint{3.546012in}{2.063744in}}%
\pgfpathlineto{\pgfqpoint{3.565535in}{2.063744in}}%
\pgfpathlineto{\pgfqpoint{3.585059in}{2.063744in}}%
\pgfpathlineto{\pgfqpoint{3.604582in}{2.063744in}}%
\pgfpathlineto{\pgfqpoint{3.624106in}{2.063744in}}%
\pgfpathlineto{\pgfqpoint{3.643630in}{2.063744in}}%
\pgfpathlineto{\pgfqpoint{3.663153in}{2.063744in}}%
\pgfpathlineto{\pgfqpoint{3.682677in}{2.063744in}}%
\pgfpathlineto{\pgfqpoint{3.702200in}{2.063744in}}%
\pgfpathlineto{\pgfqpoint{3.721724in}{2.063744in}}%
\pgfpathlineto{\pgfqpoint{3.741247in}{2.063744in}}%
\pgfpathlineto{\pgfqpoint{3.760771in}{2.063744in}}%
\pgfpathlineto{\pgfqpoint{3.780294in}{2.063744in}}%
\pgfpathlineto{\pgfqpoint{3.799818in}{2.063744in}}%
\pgfpathlineto{\pgfqpoint{3.819341in}{2.063744in}}%
\pgfpathlineto{\pgfqpoint{3.838865in}{2.063744in}}%
\pgfpathlineto{\pgfqpoint{3.858389in}{2.063744in}}%
\pgfpathlineto{\pgfqpoint{3.877912in}{2.063744in}}%
\pgfpathlineto{\pgfqpoint{3.897436in}{2.017240in}}%
\pgfpathlineto{\pgfqpoint{3.916959in}{2.017240in}}%
\pgfpathlineto{\pgfqpoint{3.936483in}{2.017240in}}%
\pgfpathlineto{\pgfqpoint{3.956006in}{2.017240in}}%
\pgfpathlineto{\pgfqpoint{3.975530in}{2.017240in}}%
\pgfpathlineto{\pgfqpoint{3.995053in}{2.017240in}}%
\pgfpathlineto{\pgfqpoint{4.014577in}{2.017240in}}%
\pgfpathlineto{\pgfqpoint{4.034100in}{2.017240in}}%
\pgfpathlineto{\pgfqpoint{4.053624in}{2.017240in}}%
\pgfpathlineto{\pgfqpoint{4.073148in}{2.017240in}}%
\pgfpathlineto{\pgfqpoint{4.092671in}{2.017240in}}%
\pgfpathlineto{\pgfqpoint{4.112195in}{2.017240in}}%
\pgfpathlineto{\pgfqpoint{4.131718in}{2.017240in}}%
\pgfpathlineto{\pgfqpoint{4.151242in}{1.980139in}}%
\pgfpathlineto{\pgfqpoint{4.170765in}{1.980139in}}%
\pgfpathlineto{\pgfqpoint{4.190289in}{1.980139in}}%
\pgfpathlineto{\pgfqpoint{4.209812in}{1.980139in}}%
\pgfpathlineto{\pgfqpoint{4.229336in}{1.980139in}}%
\pgfpathlineto{\pgfqpoint{4.248860in}{1.980139in}}%
\pgfpathlineto{\pgfqpoint{4.268383in}{1.967738in}}%
\pgfpathlineto{\pgfqpoint{4.287907in}{1.967738in}}%
\pgfpathlineto{\pgfqpoint{4.307430in}{1.967738in}}%
\pgfpathlineto{\pgfqpoint{4.326954in}{1.967738in}}%
\pgfpathlineto{\pgfqpoint{4.346477in}{1.966727in}}%
\pgfpathlineto{\pgfqpoint{4.366001in}{1.966727in}}%
\pgfpathlineto{\pgfqpoint{4.385524in}{1.966727in}}%
\pgfpathlineto{\pgfqpoint{4.405048in}{1.966727in}}%
\pgfpathlineto{\pgfqpoint{4.424571in}{1.966727in}}%
\pgfpathlineto{\pgfqpoint{4.444095in}{1.966727in}}%
\pgfpathlineto{\pgfqpoint{4.463619in}{1.966727in}}%
\pgfpathlineto{\pgfqpoint{4.483142in}{1.966727in}}%
\pgfpathlineto{\pgfqpoint{4.502666in}{1.966727in}}%
\pgfpathlineto{\pgfqpoint{4.522189in}{1.966727in}}%
\pgfpathlineto{\pgfqpoint{4.541713in}{1.966727in}}%
\pgfpathlineto{\pgfqpoint{4.561236in}{1.966727in}}%
\pgfpathlineto{\pgfqpoint{4.580760in}{1.966727in}}%
\pgfpathlineto{\pgfqpoint{4.600283in}{1.966727in}}%
\pgfpathlineto{\pgfqpoint{4.619807in}{1.966727in}}%
\pgfpathlineto{\pgfqpoint{4.639331in}{1.966727in}}%
\pgfpathlineto{\pgfqpoint{4.658854in}{1.966727in}}%
\pgfpathlineto{\pgfqpoint{4.678378in}{1.966727in}}%
\pgfpathlineto{\pgfqpoint{4.697901in}{1.966727in}}%
\pgfpathlineto{\pgfqpoint{4.717425in}{1.966727in}}%
\pgfpathlineto{\pgfqpoint{4.736948in}{1.966727in}}%
\pgfpathlineto{\pgfqpoint{4.756472in}{1.966727in}}%
\pgfpathlineto{\pgfqpoint{4.775995in}{1.966727in}}%
\pgfpathlineto{\pgfqpoint{4.795519in}{1.966727in}}%
\pgfpathlineto{\pgfqpoint{4.815042in}{1.966727in}}%
\pgfpathlineto{\pgfqpoint{4.834566in}{1.966727in}}%
\pgfpathlineto{\pgfqpoint{4.854090in}{1.966727in}}%
\pgfpathlineto{\pgfqpoint{4.873613in}{1.966727in}}%
\pgfpathlineto{\pgfqpoint{4.893137in}{1.966727in}}%
\pgfpathlineto{\pgfqpoint{4.912660in}{1.966727in}}%
\pgfpathlineto{\pgfqpoint{4.932184in}{1.966727in}}%
\pgfpathlineto{\pgfqpoint{4.951707in}{1.966727in}}%
\pgfpathlineto{\pgfqpoint{4.971231in}{1.966727in}}%
\pgfpathlineto{\pgfqpoint{4.990754in}{1.966727in}}%
\pgfpathlineto{\pgfqpoint{5.010278in}{1.966727in}}%
\pgfpathlineto{\pgfqpoint{5.029801in}{1.966727in}}%
\pgfpathlineto{\pgfqpoint{5.049325in}{1.966727in}}%
\pgfpathlineto{\pgfqpoint{5.068849in}{1.966727in}}%
\pgfpathlineto{\pgfqpoint{5.088372in}{1.966727in}}%
\pgfpathlineto{\pgfqpoint{5.107896in}{1.966727in}}%
\pgfpathlineto{\pgfqpoint{5.127419in}{1.966727in}}%
\pgfpathlineto{\pgfqpoint{5.146943in}{1.966727in}}%
\pgfpathlineto{\pgfqpoint{5.166466in}{1.966727in}}%
\pgfpathlineto{\pgfqpoint{5.185990in}{1.966727in}}%
\pgfpathlineto{\pgfqpoint{5.205513in}{1.966727in}}%
\pgfpathlineto{\pgfqpoint{5.225037in}{1.966727in}}%
\pgfpathlineto{\pgfqpoint{5.244561in}{1.966727in}}%
\pgfpathlineto{\pgfqpoint{5.264084in}{1.966727in}}%
\pgfpathlineto{\pgfqpoint{5.283608in}{1.966727in}}%
\pgfpathlineto{\pgfqpoint{5.303131in}{1.966727in}}%
\pgfpathlineto{\pgfqpoint{5.322655in}{1.966727in}}%
\pgfpathlineto{\pgfqpoint{5.342178in}{1.966727in}}%
\pgfpathlineto{\pgfqpoint{5.361702in}{1.966727in}}%
\pgfpathlineto{\pgfqpoint{5.381225in}{1.966727in}}%
\pgfpathlineto{\pgfqpoint{5.400749in}{1.966727in}}%
\pgfpathlineto{\pgfqpoint{5.420272in}{1.966727in}}%
\pgfpathlineto{\pgfqpoint{5.439796in}{1.966727in}}%
\pgfpathlineto{\pgfqpoint{5.459320in}{1.966727in}}%
\pgfpathlineto{\pgfqpoint{5.478843in}{1.952491in}}%
\pgfpathlineto{\pgfqpoint{5.498367in}{1.952491in}}%
\pgfpathlineto{\pgfqpoint{5.517890in}{1.952491in}}%
\pgfpathlineto{\pgfqpoint{5.537414in}{1.952491in}}%
\pgfpathlineto{\pgfqpoint{5.556937in}{1.952491in}}%
\pgfpathlineto{\pgfqpoint{5.576461in}{1.952491in}}%
\pgfpathlineto{\pgfqpoint{5.595984in}{1.952491in}}%
\pgfpathlineto{\pgfqpoint{5.615508in}{1.952491in}}%
\pgfpathlineto{\pgfqpoint{5.635031in}{1.952491in}}%
\pgfpathlineto{\pgfqpoint{5.654555in}{1.952491in}}%
\pgfpathlineto{\pgfqpoint{5.674079in}{1.952491in}}%
\pgfpathlineto{\pgfqpoint{5.693602in}{1.952491in}}%
\pgfpathlineto{\pgfqpoint{5.713126in}{1.952491in}}%
\pgfpathlineto{\pgfqpoint{5.732649in}{1.952491in}}%
\pgfpathlineto{\pgfqpoint{5.752173in}{1.952491in}}%
\pgfpathlineto{\pgfqpoint{5.771696in}{1.952491in}}%
\pgfpathlineto{\pgfqpoint{5.791220in}{1.952491in}}%
\pgfpathlineto{\pgfqpoint{5.810743in}{1.952491in}}%
\pgfpathlineto{\pgfqpoint{5.830267in}{1.952491in}}%
\pgfpathlineto{\pgfqpoint{5.849791in}{1.952491in}}%
\pgfpathlineto{\pgfqpoint{5.869314in}{1.952491in}}%
\pgfpathlineto{\pgfqpoint{5.888838in}{1.952491in}}%
\pgfpathlineto{\pgfqpoint{5.908361in}{1.952491in}}%
\pgfpathlineto{\pgfqpoint{5.927885in}{1.952491in}}%
\pgfpathlineto{\pgfqpoint{5.947408in}{1.952491in}}%
\pgfpathlineto{\pgfqpoint{5.966932in}{1.952491in}}%
\pgfpathlineto{\pgfqpoint{5.966932in}{1.739908in}}%
\pgfpathlineto{\pgfqpoint{5.966932in}{1.739908in}}%
\pgfpathlineto{\pgfqpoint{5.947408in}{1.739908in}}%
\pgfpathlineto{\pgfqpoint{5.927885in}{1.739908in}}%
\pgfpathlineto{\pgfqpoint{5.908361in}{1.739908in}}%
\pgfpathlineto{\pgfqpoint{5.888838in}{1.739908in}}%
\pgfpathlineto{\pgfqpoint{5.869314in}{1.739908in}}%
\pgfpathlineto{\pgfqpoint{5.849791in}{1.739908in}}%
\pgfpathlineto{\pgfqpoint{5.830267in}{1.739908in}}%
\pgfpathlineto{\pgfqpoint{5.810743in}{1.739908in}}%
\pgfpathlineto{\pgfqpoint{5.791220in}{1.739908in}}%
\pgfpathlineto{\pgfqpoint{5.771696in}{1.739908in}}%
\pgfpathlineto{\pgfqpoint{5.752173in}{1.739908in}}%
\pgfpathlineto{\pgfqpoint{5.732649in}{1.739908in}}%
\pgfpathlineto{\pgfqpoint{5.713126in}{1.739908in}}%
\pgfpathlineto{\pgfqpoint{5.693602in}{1.739908in}}%
\pgfpathlineto{\pgfqpoint{5.674079in}{1.739908in}}%
\pgfpathlineto{\pgfqpoint{5.654555in}{1.739908in}}%
\pgfpathlineto{\pgfqpoint{5.635031in}{1.739908in}}%
\pgfpathlineto{\pgfqpoint{5.615508in}{1.739908in}}%
\pgfpathlineto{\pgfqpoint{5.595984in}{1.739908in}}%
\pgfpathlineto{\pgfqpoint{5.576461in}{1.739908in}}%
\pgfpathlineto{\pgfqpoint{5.556937in}{1.739908in}}%
\pgfpathlineto{\pgfqpoint{5.537414in}{1.739908in}}%
\pgfpathlineto{\pgfqpoint{5.517890in}{1.739908in}}%
\pgfpathlineto{\pgfqpoint{5.498367in}{1.739908in}}%
\pgfpathlineto{\pgfqpoint{5.478843in}{1.739908in}}%
\pgfpathlineto{\pgfqpoint{5.459320in}{1.765352in}}%
\pgfpathlineto{\pgfqpoint{5.439796in}{1.765352in}}%
\pgfpathlineto{\pgfqpoint{5.420272in}{1.765352in}}%
\pgfpathlineto{\pgfqpoint{5.400749in}{1.765352in}}%
\pgfpathlineto{\pgfqpoint{5.381225in}{1.765352in}}%
\pgfpathlineto{\pgfqpoint{5.361702in}{1.765352in}}%
\pgfpathlineto{\pgfqpoint{5.342178in}{1.765352in}}%
\pgfpathlineto{\pgfqpoint{5.322655in}{1.765352in}}%
\pgfpathlineto{\pgfqpoint{5.303131in}{1.765352in}}%
\pgfpathlineto{\pgfqpoint{5.283608in}{1.765352in}}%
\pgfpathlineto{\pgfqpoint{5.264084in}{1.765352in}}%
\pgfpathlineto{\pgfqpoint{5.244561in}{1.765352in}}%
\pgfpathlineto{\pgfqpoint{5.225037in}{1.765352in}}%
\pgfpathlineto{\pgfqpoint{5.205513in}{1.765352in}}%
\pgfpathlineto{\pgfqpoint{5.185990in}{1.765352in}}%
\pgfpathlineto{\pgfqpoint{5.166466in}{1.765352in}}%
\pgfpathlineto{\pgfqpoint{5.146943in}{1.765352in}}%
\pgfpathlineto{\pgfqpoint{5.127419in}{1.765352in}}%
\pgfpathlineto{\pgfqpoint{5.107896in}{1.765352in}}%
\pgfpathlineto{\pgfqpoint{5.088372in}{1.765352in}}%
\pgfpathlineto{\pgfqpoint{5.068849in}{1.765352in}}%
\pgfpathlineto{\pgfqpoint{5.049325in}{1.765352in}}%
\pgfpathlineto{\pgfqpoint{5.029801in}{1.765352in}}%
\pgfpathlineto{\pgfqpoint{5.010278in}{1.765352in}}%
\pgfpathlineto{\pgfqpoint{4.990754in}{1.765352in}}%
\pgfpathlineto{\pgfqpoint{4.971231in}{1.765352in}}%
\pgfpathlineto{\pgfqpoint{4.951707in}{1.765352in}}%
\pgfpathlineto{\pgfqpoint{4.932184in}{1.765352in}}%
\pgfpathlineto{\pgfqpoint{4.912660in}{1.765352in}}%
\pgfpathlineto{\pgfqpoint{4.893137in}{1.765352in}}%
\pgfpathlineto{\pgfqpoint{4.873613in}{1.765352in}}%
\pgfpathlineto{\pgfqpoint{4.854090in}{1.765352in}}%
\pgfpathlineto{\pgfqpoint{4.834566in}{1.765352in}}%
\pgfpathlineto{\pgfqpoint{4.815042in}{1.765352in}}%
\pgfpathlineto{\pgfqpoint{4.795519in}{1.765352in}}%
\pgfpathlineto{\pgfqpoint{4.775995in}{1.765352in}}%
\pgfpathlineto{\pgfqpoint{4.756472in}{1.765352in}}%
\pgfpathlineto{\pgfqpoint{4.736948in}{1.765352in}}%
\pgfpathlineto{\pgfqpoint{4.717425in}{1.765352in}}%
\pgfpathlineto{\pgfqpoint{4.697901in}{1.765352in}}%
\pgfpathlineto{\pgfqpoint{4.678378in}{1.765352in}}%
\pgfpathlineto{\pgfqpoint{4.658854in}{1.765352in}}%
\pgfpathlineto{\pgfqpoint{4.639331in}{1.765352in}}%
\pgfpathlineto{\pgfqpoint{4.619807in}{1.765352in}}%
\pgfpathlineto{\pgfqpoint{4.600283in}{1.765352in}}%
\pgfpathlineto{\pgfqpoint{4.580760in}{1.765352in}}%
\pgfpathlineto{\pgfqpoint{4.561236in}{1.765352in}}%
\pgfpathlineto{\pgfqpoint{4.541713in}{1.765352in}}%
\pgfpathlineto{\pgfqpoint{4.522189in}{1.765352in}}%
\pgfpathlineto{\pgfqpoint{4.502666in}{1.765352in}}%
\pgfpathlineto{\pgfqpoint{4.483142in}{1.765352in}}%
\pgfpathlineto{\pgfqpoint{4.463619in}{1.765352in}}%
\pgfpathlineto{\pgfqpoint{4.444095in}{1.765352in}}%
\pgfpathlineto{\pgfqpoint{4.424571in}{1.765352in}}%
\pgfpathlineto{\pgfqpoint{4.405048in}{1.765352in}}%
\pgfpathlineto{\pgfqpoint{4.385524in}{1.765352in}}%
\pgfpathlineto{\pgfqpoint{4.366001in}{1.765352in}}%
\pgfpathlineto{\pgfqpoint{4.346477in}{1.765352in}}%
\pgfpathlineto{\pgfqpoint{4.326954in}{1.767531in}}%
\pgfpathlineto{\pgfqpoint{4.307430in}{1.767531in}}%
\pgfpathlineto{\pgfqpoint{4.287907in}{1.767531in}}%
\pgfpathlineto{\pgfqpoint{4.268383in}{1.767531in}}%
\pgfpathlineto{\pgfqpoint{4.248860in}{1.800194in}}%
\pgfpathlineto{\pgfqpoint{4.229336in}{1.800194in}}%
\pgfpathlineto{\pgfqpoint{4.209812in}{1.800194in}}%
\pgfpathlineto{\pgfqpoint{4.190289in}{1.800194in}}%
\pgfpathlineto{\pgfqpoint{4.170765in}{1.800194in}}%
\pgfpathlineto{\pgfqpoint{4.151242in}{1.800194in}}%
\pgfpathlineto{\pgfqpoint{4.131718in}{1.829706in}}%
\pgfpathlineto{\pgfqpoint{4.112195in}{1.829706in}}%
\pgfpathlineto{\pgfqpoint{4.092671in}{1.829706in}}%
\pgfpathlineto{\pgfqpoint{4.073148in}{1.829706in}}%
\pgfpathlineto{\pgfqpoint{4.053624in}{1.829706in}}%
\pgfpathlineto{\pgfqpoint{4.034100in}{1.829706in}}%
\pgfpathlineto{\pgfqpoint{4.014577in}{1.829706in}}%
\pgfpathlineto{\pgfqpoint{3.995053in}{1.829706in}}%
\pgfpathlineto{\pgfqpoint{3.975530in}{1.829706in}}%
\pgfpathlineto{\pgfqpoint{3.956006in}{1.829706in}}%
\pgfpathlineto{\pgfqpoint{3.936483in}{1.829706in}}%
\pgfpathlineto{\pgfqpoint{3.916959in}{1.829706in}}%
\pgfpathlineto{\pgfqpoint{3.897436in}{1.829706in}}%
\pgfpathlineto{\pgfqpoint{3.877912in}{1.881764in}}%
\pgfpathlineto{\pgfqpoint{3.858389in}{1.881764in}}%
\pgfpathlineto{\pgfqpoint{3.838865in}{1.881764in}}%
\pgfpathlineto{\pgfqpoint{3.819341in}{1.881764in}}%
\pgfpathlineto{\pgfqpoint{3.799818in}{1.881764in}}%
\pgfpathlineto{\pgfqpoint{3.780294in}{1.881764in}}%
\pgfpathlineto{\pgfqpoint{3.760771in}{1.881764in}}%
\pgfpathlineto{\pgfqpoint{3.741247in}{1.881764in}}%
\pgfpathlineto{\pgfqpoint{3.721724in}{1.881764in}}%
\pgfpathlineto{\pgfqpoint{3.702200in}{1.881764in}}%
\pgfpathlineto{\pgfqpoint{3.682677in}{1.881764in}}%
\pgfpathlineto{\pgfqpoint{3.663153in}{1.881764in}}%
\pgfpathlineto{\pgfqpoint{3.643630in}{1.881764in}}%
\pgfpathlineto{\pgfqpoint{3.624106in}{1.881764in}}%
\pgfpathlineto{\pgfqpoint{3.604582in}{1.881764in}}%
\pgfpathlineto{\pgfqpoint{3.585059in}{1.881764in}}%
\pgfpathlineto{\pgfqpoint{3.565535in}{1.881764in}}%
\pgfpathlineto{\pgfqpoint{3.546012in}{1.881764in}}%
\pgfpathlineto{\pgfqpoint{3.526488in}{1.881764in}}%
\pgfpathlineto{\pgfqpoint{3.506965in}{1.881764in}}%
\pgfpathlineto{\pgfqpoint{3.487441in}{1.881764in}}%
\pgfpathlineto{\pgfqpoint{3.467918in}{1.881764in}}%
\pgfpathlineto{\pgfqpoint{3.448394in}{1.881764in}}%
\pgfpathlineto{\pgfqpoint{3.428870in}{1.881764in}}%
\pgfpathlineto{\pgfqpoint{3.409347in}{1.881764in}}%
\pgfpathlineto{\pgfqpoint{3.389823in}{1.881764in}}%
\pgfpathlineto{\pgfqpoint{3.370300in}{1.881764in}}%
\pgfpathlineto{\pgfqpoint{3.350776in}{1.881764in}}%
\pgfpathlineto{\pgfqpoint{3.331253in}{1.881764in}}%
\pgfpathlineto{\pgfqpoint{3.311729in}{1.881764in}}%
\pgfpathlineto{\pgfqpoint{3.292206in}{1.881764in}}%
\pgfpathlineto{\pgfqpoint{3.272682in}{1.881764in}}%
\pgfpathlineto{\pgfqpoint{3.253159in}{1.881764in}}%
\pgfpathlineto{\pgfqpoint{3.233635in}{1.881764in}}%
\pgfpathlineto{\pgfqpoint{3.214111in}{1.881764in}}%
\pgfpathlineto{\pgfqpoint{3.194588in}{1.881764in}}%
\pgfpathlineto{\pgfqpoint{3.175064in}{1.895135in}}%
\pgfpathlineto{\pgfqpoint{3.155541in}{1.896640in}}%
\pgfpathlineto{\pgfqpoint{3.136017in}{1.896640in}}%
\pgfpathlineto{\pgfqpoint{3.116494in}{1.896640in}}%
\pgfpathlineto{\pgfqpoint{3.096970in}{1.896640in}}%
\pgfpathlineto{\pgfqpoint{3.077447in}{1.896640in}}%
\pgfpathlineto{\pgfqpoint{3.057923in}{1.896640in}}%
\pgfpathlineto{\pgfqpoint{3.038400in}{1.896640in}}%
\pgfpathlineto{\pgfqpoint{3.018876in}{1.896640in}}%
\pgfpathlineto{\pgfqpoint{2.999352in}{1.896640in}}%
\pgfpathlineto{\pgfqpoint{2.979829in}{1.896640in}}%
\pgfpathlineto{\pgfqpoint{2.960305in}{1.896640in}}%
\pgfpathlineto{\pgfqpoint{2.940782in}{1.896640in}}%
\pgfpathlineto{\pgfqpoint{2.921258in}{1.928332in}}%
\pgfpathlineto{\pgfqpoint{2.901735in}{1.928332in}}%
\pgfpathlineto{\pgfqpoint{2.882211in}{1.928332in}}%
\pgfpathlineto{\pgfqpoint{2.862688in}{1.928332in}}%
\pgfpathlineto{\pgfqpoint{2.843164in}{1.928332in}}%
\pgfpathlineto{\pgfqpoint{2.823640in}{1.931345in}}%
\pgfpathlineto{\pgfqpoint{2.804117in}{1.931345in}}%
\pgfpathlineto{\pgfqpoint{2.784593in}{1.931345in}}%
\pgfpathlineto{\pgfqpoint{2.765070in}{1.931345in}}%
\pgfpathlineto{\pgfqpoint{2.745546in}{1.987780in}}%
\pgfpathlineto{\pgfqpoint{2.726023in}{1.987780in}}%
\pgfpathlineto{\pgfqpoint{2.706499in}{1.987780in}}%
\pgfpathlineto{\pgfqpoint{2.686976in}{1.987780in}}%
\pgfpathlineto{\pgfqpoint{2.667452in}{1.987780in}}%
\pgfpathlineto{\pgfqpoint{2.647929in}{2.028139in}}%
\pgfpathlineto{\pgfqpoint{2.628405in}{2.028139in}}%
\pgfpathlineto{\pgfqpoint{2.608881in}{2.028139in}}%
\pgfpathlineto{\pgfqpoint{2.589358in}{2.028139in}}%
\pgfpathlineto{\pgfqpoint{2.569834in}{2.028139in}}%
\pgfpathlineto{\pgfqpoint{2.550311in}{2.028139in}}%
\pgfpathlineto{\pgfqpoint{2.530787in}{2.028139in}}%
\pgfpathlineto{\pgfqpoint{2.511264in}{2.028139in}}%
\pgfpathlineto{\pgfqpoint{2.491740in}{2.028139in}}%
\pgfpathlineto{\pgfqpoint{2.472217in}{2.028139in}}%
\pgfpathlineto{\pgfqpoint{2.452693in}{2.028139in}}%
\pgfpathlineto{\pgfqpoint{2.433169in}{2.028139in}}%
\pgfpathlineto{\pgfqpoint{2.413646in}{2.028139in}}%
\pgfpathlineto{\pgfqpoint{2.394122in}{2.028139in}}%
\pgfpathlineto{\pgfqpoint{2.374599in}{2.028139in}}%
\pgfpathlineto{\pgfqpoint{2.355075in}{2.028139in}}%
\pgfpathlineto{\pgfqpoint{2.335552in}{2.028139in}}%
\pgfpathlineto{\pgfqpoint{2.316028in}{2.028139in}}%
\pgfpathlineto{\pgfqpoint{2.296505in}{2.028139in}}%
\pgfpathlineto{\pgfqpoint{2.276981in}{2.028139in}}%
\pgfpathlineto{\pgfqpoint{2.257458in}{2.028139in}}%
\pgfpathlineto{\pgfqpoint{2.237934in}{2.028139in}}%
\pgfpathlineto{\pgfqpoint{2.218410in}{2.028139in}}%
\pgfpathlineto{\pgfqpoint{2.198887in}{2.028139in}}%
\pgfpathlineto{\pgfqpoint{2.179363in}{2.032189in}}%
\pgfpathlineto{\pgfqpoint{2.159840in}{2.032189in}}%
\pgfpathlineto{\pgfqpoint{2.140316in}{2.061351in}}%
\pgfpathlineto{\pgfqpoint{2.120793in}{2.061351in}}%
\pgfpathlineto{\pgfqpoint{2.101269in}{2.061351in}}%
\pgfpathlineto{\pgfqpoint{2.081746in}{2.061351in}}%
\pgfpathlineto{\pgfqpoint{2.062222in}{2.061351in}}%
\pgfpathlineto{\pgfqpoint{2.042699in}{2.061351in}}%
\pgfpathlineto{\pgfqpoint{2.023175in}{2.061351in}}%
\pgfpathlineto{\pgfqpoint{2.003651in}{2.066014in}}%
\pgfpathlineto{\pgfqpoint{1.984128in}{2.091908in}}%
\pgfpathlineto{\pgfqpoint{1.964604in}{2.099609in}}%
\pgfpathlineto{\pgfqpoint{1.945081in}{2.099609in}}%
\pgfpathlineto{\pgfqpoint{1.925557in}{2.099609in}}%
\pgfpathlineto{\pgfqpoint{1.906034in}{2.099609in}}%
\pgfpathlineto{\pgfqpoint{1.886510in}{2.099609in}}%
\pgfpathlineto{\pgfqpoint{1.866987in}{2.100164in}}%
\pgfpathlineto{\pgfqpoint{1.847463in}{2.100164in}}%
\pgfpathlineto{\pgfqpoint{1.827939in}{2.100164in}}%
\pgfpathlineto{\pgfqpoint{1.808416in}{2.100164in}}%
\pgfpathlineto{\pgfqpoint{1.788892in}{2.100164in}}%
\pgfpathlineto{\pgfqpoint{1.769369in}{2.100164in}}%
\pgfpathlineto{\pgfqpoint{1.749845in}{2.100164in}}%
\pgfpathlineto{\pgfqpoint{1.730322in}{2.135799in}}%
\pgfpathlineto{\pgfqpoint{1.710798in}{2.144472in}}%
\pgfpathlineto{\pgfqpoint{1.691275in}{2.144472in}}%
\pgfpathlineto{\pgfqpoint{1.671751in}{2.144472in}}%
\pgfpathlineto{\pgfqpoint{1.652228in}{2.144472in}}%
\pgfpathlineto{\pgfqpoint{1.632704in}{2.144472in}}%
\pgfpathlineto{\pgfqpoint{1.613180in}{2.144472in}}%
\pgfpathlineto{\pgfqpoint{1.593657in}{2.171652in}}%
\pgfpathlineto{\pgfqpoint{1.574133in}{2.203977in}}%
\pgfpathlineto{\pgfqpoint{1.554610in}{2.208557in}}%
\pgfpathlineto{\pgfqpoint{1.535086in}{2.208557in}}%
\pgfpathlineto{\pgfqpoint{1.515563in}{2.211000in}}%
\pgfpathlineto{\pgfqpoint{1.496039in}{2.211000in}}%
\pgfpathlineto{\pgfqpoint{1.476516in}{2.227210in}}%
\pgfpathlineto{\pgfqpoint{1.456992in}{2.227210in}}%
\pgfpathlineto{\pgfqpoint{1.437469in}{2.227210in}}%
\pgfpathlineto{\pgfqpoint{1.417945in}{2.227210in}}%
\pgfpathlineto{\pgfqpoint{1.398421in}{2.250695in}}%
\pgfpathlineto{\pgfqpoint{1.378898in}{2.250695in}}%
\pgfpathlineto{\pgfqpoint{1.359374in}{2.278379in}}%
\pgfpathlineto{\pgfqpoint{1.339851in}{2.278379in}}%
\pgfpathlineto{\pgfqpoint{1.320327in}{2.278379in}}%
\pgfpathlineto{\pgfqpoint{1.300804in}{2.299884in}}%
\pgfpathlineto{\pgfqpoint{1.281280in}{2.315718in}}%
\pgfpathlineto{\pgfqpoint{1.261757in}{2.315718in}}%
\pgfpathlineto{\pgfqpoint{1.242233in}{2.317536in}}%
\pgfpathlineto{\pgfqpoint{1.222709in}{2.330568in}}%
\pgfpathlineto{\pgfqpoint{1.203186in}{2.330568in}}%
\pgfpathlineto{\pgfqpoint{1.183662in}{2.365104in}}%
\pgfpathlineto{\pgfqpoint{1.164139in}{2.389453in}}%
\pgfpathlineto{\pgfqpoint{1.144615in}{2.405967in}}%
\pgfpathlineto{\pgfqpoint{1.125092in}{2.417867in}}%
\pgfpathlineto{\pgfqpoint{1.105568in}{2.501020in}}%
\pgfpathclose%
\pgfusepath{fill}%
\end{pgfscope}%
\begin{pgfscope}%
\pgfpathrectangle{\pgfqpoint{0.862500in}{0.375000in}}{\pgfqpoint{5.347500in}{2.265000in}}%
\pgfusepath{clip}%
\pgfsetbuttcap%
\pgfsetroundjoin%
\definecolor{currentfill}{rgb}{1.000000,0.498039,0.054902}%
\pgfsetfillcolor{currentfill}%
\pgfsetfillopacity{0.200000}%
\pgfsetlinewidth{0.000000pt}%
\definecolor{currentstroke}{rgb}{0.000000,0.000000,0.000000}%
\pgfsetstrokecolor{currentstroke}%
\pgfsetdash{}{0pt}%
\pgfpathmoveto{\pgfqpoint{1.105568in}{2.451037in}}%
\pgfpathlineto{\pgfqpoint{1.105568in}{2.524160in}}%
\pgfpathlineto{\pgfqpoint{1.125092in}{2.515240in}}%
\pgfpathlineto{\pgfqpoint{1.144615in}{2.472563in}}%
\pgfpathlineto{\pgfqpoint{1.164139in}{2.374428in}}%
\pgfpathlineto{\pgfqpoint{1.183662in}{2.374428in}}%
\pgfpathlineto{\pgfqpoint{1.203186in}{2.343508in}}%
\pgfpathlineto{\pgfqpoint{1.222709in}{2.311765in}}%
\pgfpathlineto{\pgfqpoint{1.242233in}{2.311765in}}%
\pgfpathlineto{\pgfqpoint{1.261757in}{2.311765in}}%
\pgfpathlineto{\pgfqpoint{1.281280in}{2.311765in}}%
\pgfpathlineto{\pgfqpoint{1.300804in}{2.311765in}}%
\pgfpathlineto{\pgfqpoint{1.320327in}{2.311765in}}%
\pgfpathlineto{\pgfqpoint{1.339851in}{2.309982in}}%
\pgfpathlineto{\pgfqpoint{1.359374in}{2.309982in}}%
\pgfpathlineto{\pgfqpoint{1.378898in}{2.235650in}}%
\pgfpathlineto{\pgfqpoint{1.398421in}{2.235650in}}%
\pgfpathlineto{\pgfqpoint{1.417945in}{2.235650in}}%
\pgfpathlineto{\pgfqpoint{1.437469in}{2.235650in}}%
\pgfpathlineto{\pgfqpoint{1.456992in}{2.210862in}}%
\pgfpathlineto{\pgfqpoint{1.476516in}{2.210862in}}%
\pgfpathlineto{\pgfqpoint{1.496039in}{2.210862in}}%
\pgfpathlineto{\pgfqpoint{1.515563in}{2.210862in}}%
\pgfpathlineto{\pgfqpoint{1.535086in}{2.210862in}}%
\pgfpathlineto{\pgfqpoint{1.554610in}{2.210862in}}%
\pgfpathlineto{\pgfqpoint{1.574133in}{2.210862in}}%
\pgfpathlineto{\pgfqpoint{1.593657in}{2.210862in}}%
\pgfpathlineto{\pgfqpoint{1.613180in}{2.210862in}}%
\pgfpathlineto{\pgfqpoint{1.632704in}{2.210862in}}%
\pgfpathlineto{\pgfqpoint{1.652228in}{2.210862in}}%
\pgfpathlineto{\pgfqpoint{1.671751in}{2.210862in}}%
\pgfpathlineto{\pgfqpoint{1.691275in}{2.170221in}}%
\pgfpathlineto{\pgfqpoint{1.710798in}{2.170221in}}%
\pgfpathlineto{\pgfqpoint{1.730322in}{2.165418in}}%
\pgfpathlineto{\pgfqpoint{1.749845in}{2.147482in}}%
\pgfpathlineto{\pgfqpoint{1.769369in}{2.147482in}}%
\pgfpathlineto{\pgfqpoint{1.788892in}{2.147482in}}%
\pgfpathlineto{\pgfqpoint{1.808416in}{2.147482in}}%
\pgfpathlineto{\pgfqpoint{1.827939in}{2.147482in}}%
\pgfpathlineto{\pgfqpoint{1.847463in}{2.147482in}}%
\pgfpathlineto{\pgfqpoint{1.866987in}{2.147482in}}%
\pgfpathlineto{\pgfqpoint{1.886510in}{2.147482in}}%
\pgfpathlineto{\pgfqpoint{1.906034in}{2.147482in}}%
\pgfpathlineto{\pgfqpoint{1.925557in}{2.147482in}}%
\pgfpathlineto{\pgfqpoint{1.945081in}{2.147482in}}%
\pgfpathlineto{\pgfqpoint{1.964604in}{2.147482in}}%
\pgfpathlineto{\pgfqpoint{1.984128in}{2.147482in}}%
\pgfpathlineto{\pgfqpoint{2.003651in}{2.147482in}}%
\pgfpathlineto{\pgfqpoint{2.023175in}{2.147482in}}%
\pgfpathlineto{\pgfqpoint{2.042699in}{2.147482in}}%
\pgfpathlineto{\pgfqpoint{2.062222in}{2.147482in}}%
\pgfpathlineto{\pgfqpoint{2.081746in}{2.084592in}}%
\pgfpathlineto{\pgfqpoint{2.101269in}{2.084592in}}%
\pgfpathlineto{\pgfqpoint{2.120793in}{2.084592in}}%
\pgfpathlineto{\pgfqpoint{2.140316in}{2.084592in}}%
\pgfpathlineto{\pgfqpoint{2.159840in}{2.084592in}}%
\pgfpathlineto{\pgfqpoint{2.179363in}{2.084592in}}%
\pgfpathlineto{\pgfqpoint{2.198887in}{2.002408in}}%
\pgfpathlineto{\pgfqpoint{2.218410in}{2.002408in}}%
\pgfpathlineto{\pgfqpoint{2.237934in}{1.954384in}}%
\pgfpathlineto{\pgfqpoint{2.257458in}{1.895863in}}%
\pgfpathlineto{\pgfqpoint{2.276981in}{1.895631in}}%
\pgfpathlineto{\pgfqpoint{2.296505in}{1.895631in}}%
\pgfpathlineto{\pgfqpoint{2.316028in}{1.886248in}}%
\pgfpathlineto{\pgfqpoint{2.335552in}{1.770610in}}%
\pgfpathlineto{\pgfqpoint{2.355075in}{1.686714in}}%
\pgfpathlineto{\pgfqpoint{2.374599in}{1.686400in}}%
\pgfpathlineto{\pgfqpoint{2.394122in}{1.686400in}}%
\pgfpathlineto{\pgfqpoint{2.413646in}{1.686400in}}%
\pgfpathlineto{\pgfqpoint{2.433169in}{1.636073in}}%
\pgfpathlineto{\pgfqpoint{2.452693in}{1.636073in}}%
\pgfpathlineto{\pgfqpoint{2.472217in}{1.636073in}}%
\pgfpathlineto{\pgfqpoint{2.491740in}{1.627572in}}%
\pgfpathlineto{\pgfqpoint{2.511264in}{1.625152in}}%
\pgfpathlineto{\pgfqpoint{2.530787in}{1.601290in}}%
\pgfpathlineto{\pgfqpoint{2.550311in}{1.568751in}}%
\pgfpathlineto{\pgfqpoint{2.569834in}{1.549564in}}%
\pgfpathlineto{\pgfqpoint{2.589358in}{1.549564in}}%
\pgfpathlineto{\pgfqpoint{2.608881in}{1.549564in}}%
\pgfpathlineto{\pgfqpoint{2.628405in}{1.548950in}}%
\pgfpathlineto{\pgfqpoint{2.647929in}{1.517463in}}%
\pgfpathlineto{\pgfqpoint{2.667452in}{1.517463in}}%
\pgfpathlineto{\pgfqpoint{2.686976in}{1.517463in}}%
\pgfpathlineto{\pgfqpoint{2.706499in}{1.517463in}}%
\pgfpathlineto{\pgfqpoint{2.726023in}{1.462605in}}%
\pgfpathlineto{\pgfqpoint{2.745546in}{1.462605in}}%
\pgfpathlineto{\pgfqpoint{2.765070in}{1.461900in}}%
\pgfpathlineto{\pgfqpoint{2.784593in}{1.461900in}}%
\pgfpathlineto{\pgfqpoint{2.804117in}{1.461900in}}%
\pgfpathlineto{\pgfqpoint{2.823640in}{1.461900in}}%
\pgfpathlineto{\pgfqpoint{2.843164in}{1.461900in}}%
\pgfpathlineto{\pgfqpoint{2.862688in}{1.420448in}}%
\pgfpathlineto{\pgfqpoint{2.882211in}{1.420448in}}%
\pgfpathlineto{\pgfqpoint{2.901735in}{1.420448in}}%
\pgfpathlineto{\pgfqpoint{2.921258in}{1.420448in}}%
\pgfpathlineto{\pgfqpoint{2.940782in}{1.420448in}}%
\pgfpathlineto{\pgfqpoint{2.960305in}{1.420448in}}%
\pgfpathlineto{\pgfqpoint{2.979829in}{1.420448in}}%
\pgfpathlineto{\pgfqpoint{2.999352in}{1.420448in}}%
\pgfpathlineto{\pgfqpoint{3.018876in}{1.396083in}}%
\pgfpathlineto{\pgfqpoint{3.038400in}{1.396083in}}%
\pgfpathlineto{\pgfqpoint{3.057923in}{1.396083in}}%
\pgfpathlineto{\pgfqpoint{3.077447in}{1.390611in}}%
\pgfpathlineto{\pgfqpoint{3.096970in}{1.390611in}}%
\pgfpathlineto{\pgfqpoint{3.116494in}{1.390611in}}%
\pgfpathlineto{\pgfqpoint{3.136017in}{1.329815in}}%
\pgfpathlineto{\pgfqpoint{3.155541in}{1.329815in}}%
\pgfpathlineto{\pgfqpoint{3.175064in}{1.324037in}}%
\pgfpathlineto{\pgfqpoint{3.194588in}{1.271966in}}%
\pgfpathlineto{\pgfqpoint{3.214111in}{1.266415in}}%
\pgfpathlineto{\pgfqpoint{3.233635in}{1.266415in}}%
\pgfpathlineto{\pgfqpoint{3.253159in}{1.266415in}}%
\pgfpathlineto{\pgfqpoint{3.272682in}{1.236353in}}%
\pgfpathlineto{\pgfqpoint{3.292206in}{1.236353in}}%
\pgfpathlineto{\pgfqpoint{3.311729in}{1.236353in}}%
\pgfpathlineto{\pgfqpoint{3.331253in}{1.236353in}}%
\pgfpathlineto{\pgfqpoint{3.350776in}{1.236353in}}%
\pgfpathlineto{\pgfqpoint{3.370300in}{1.236353in}}%
\pgfpathlineto{\pgfqpoint{3.389823in}{1.236353in}}%
\pgfpathlineto{\pgfqpoint{3.409347in}{1.236353in}}%
\pgfpathlineto{\pgfqpoint{3.428870in}{1.236353in}}%
\pgfpathlineto{\pgfqpoint{3.448394in}{1.236353in}}%
\pgfpathlineto{\pgfqpoint{3.467918in}{1.236353in}}%
\pgfpathlineto{\pgfqpoint{3.487441in}{1.235876in}}%
\pgfpathlineto{\pgfqpoint{3.506965in}{1.142072in}}%
\pgfpathlineto{\pgfqpoint{3.526488in}{1.142072in}}%
\pgfpathlineto{\pgfqpoint{3.546012in}{1.142072in}}%
\pgfpathlineto{\pgfqpoint{3.565535in}{1.142072in}}%
\pgfpathlineto{\pgfqpoint{3.585059in}{1.142072in}}%
\pgfpathlineto{\pgfqpoint{3.604582in}{1.142072in}}%
\pgfpathlineto{\pgfqpoint{3.624106in}{1.142072in}}%
\pgfpathlineto{\pgfqpoint{3.643630in}{1.142072in}}%
\pgfpathlineto{\pgfqpoint{3.663153in}{1.142072in}}%
\pgfpathlineto{\pgfqpoint{3.682677in}{1.142072in}}%
\pgfpathlineto{\pgfqpoint{3.702200in}{1.142072in}}%
\pgfpathlineto{\pgfqpoint{3.721724in}{1.140269in}}%
\pgfpathlineto{\pgfqpoint{3.741247in}{1.140269in}}%
\pgfpathlineto{\pgfqpoint{3.760771in}{1.140269in}}%
\pgfpathlineto{\pgfqpoint{3.780294in}{1.140269in}}%
\pgfpathlineto{\pgfqpoint{3.799818in}{1.140269in}}%
\pgfpathlineto{\pgfqpoint{3.819341in}{1.140269in}}%
\pgfpathlineto{\pgfqpoint{3.838865in}{1.140269in}}%
\pgfpathlineto{\pgfqpoint{3.858389in}{1.140269in}}%
\pgfpathlineto{\pgfqpoint{3.877912in}{1.140269in}}%
\pgfpathlineto{\pgfqpoint{3.897436in}{1.140269in}}%
\pgfpathlineto{\pgfqpoint{3.916959in}{1.140269in}}%
\pgfpathlineto{\pgfqpoint{3.936483in}{1.140269in}}%
\pgfpathlineto{\pgfqpoint{3.956006in}{1.140269in}}%
\pgfpathlineto{\pgfqpoint{3.975530in}{1.140269in}}%
\pgfpathlineto{\pgfqpoint{3.995053in}{1.140269in}}%
\pgfpathlineto{\pgfqpoint{4.014577in}{1.140269in}}%
\pgfpathlineto{\pgfqpoint{4.034100in}{1.140269in}}%
\pgfpathlineto{\pgfqpoint{4.053624in}{1.140269in}}%
\pgfpathlineto{\pgfqpoint{4.073148in}{1.140269in}}%
\pgfpathlineto{\pgfqpoint{4.092671in}{1.140269in}}%
\pgfpathlineto{\pgfqpoint{4.112195in}{1.140269in}}%
\pgfpathlineto{\pgfqpoint{4.131718in}{1.140269in}}%
\pgfpathlineto{\pgfqpoint{4.151242in}{1.140269in}}%
\pgfpathlineto{\pgfqpoint{4.170765in}{1.140269in}}%
\pgfpathlineto{\pgfqpoint{4.190289in}{1.122554in}}%
\pgfpathlineto{\pgfqpoint{4.209812in}{1.122554in}}%
\pgfpathlineto{\pgfqpoint{4.229336in}{1.122554in}}%
\pgfpathlineto{\pgfqpoint{4.248860in}{1.122554in}}%
\pgfpathlineto{\pgfqpoint{4.268383in}{1.122554in}}%
\pgfpathlineto{\pgfqpoint{4.287907in}{1.112651in}}%
\pgfpathlineto{\pgfqpoint{4.307430in}{1.065353in}}%
\pgfpathlineto{\pgfqpoint{4.326954in}{1.065353in}}%
\pgfpathlineto{\pgfqpoint{4.346477in}{1.065353in}}%
\pgfpathlineto{\pgfqpoint{4.366001in}{1.065353in}}%
\pgfpathlineto{\pgfqpoint{4.385524in}{1.065353in}}%
\pgfpathlineto{\pgfqpoint{4.405048in}{1.065353in}}%
\pgfpathlineto{\pgfqpoint{4.424571in}{1.065353in}}%
\pgfpathlineto{\pgfqpoint{4.444095in}{1.065353in}}%
\pgfpathlineto{\pgfqpoint{4.463619in}{1.065353in}}%
\pgfpathlineto{\pgfqpoint{4.483142in}{1.065353in}}%
\pgfpathlineto{\pgfqpoint{4.502666in}{1.065353in}}%
\pgfpathlineto{\pgfqpoint{4.522189in}{1.065353in}}%
\pgfpathlineto{\pgfqpoint{4.541713in}{1.065353in}}%
\pgfpathlineto{\pgfqpoint{4.561236in}{1.065353in}}%
\pgfpathlineto{\pgfqpoint{4.580760in}{1.061576in}}%
\pgfpathlineto{\pgfqpoint{4.600283in}{0.932733in}}%
\pgfpathlineto{\pgfqpoint{4.619807in}{0.921670in}}%
\pgfpathlineto{\pgfqpoint{4.639331in}{0.914606in}}%
\pgfpathlineto{\pgfqpoint{4.658854in}{0.914606in}}%
\pgfpathlineto{\pgfqpoint{4.678378in}{0.914606in}}%
\pgfpathlineto{\pgfqpoint{4.697901in}{0.914606in}}%
\pgfpathlineto{\pgfqpoint{4.717425in}{0.914606in}}%
\pgfpathlineto{\pgfqpoint{4.736948in}{0.914606in}}%
\pgfpathlineto{\pgfqpoint{4.756472in}{0.914606in}}%
\pgfpathlineto{\pgfqpoint{4.775995in}{0.914606in}}%
\pgfpathlineto{\pgfqpoint{4.795519in}{0.914606in}}%
\pgfpathlineto{\pgfqpoint{4.815042in}{0.914606in}}%
\pgfpathlineto{\pgfqpoint{4.834566in}{0.914606in}}%
\pgfpathlineto{\pgfqpoint{4.854090in}{0.914606in}}%
\pgfpathlineto{\pgfqpoint{4.873613in}{0.912311in}}%
\pgfpathlineto{\pgfqpoint{4.893137in}{0.912311in}}%
\pgfpathlineto{\pgfqpoint{4.912660in}{0.912311in}}%
\pgfpathlineto{\pgfqpoint{4.932184in}{0.912311in}}%
\pgfpathlineto{\pgfqpoint{4.951707in}{0.912311in}}%
\pgfpathlineto{\pgfqpoint{4.971231in}{0.912311in}}%
\pgfpathlineto{\pgfqpoint{4.990754in}{0.912311in}}%
\pgfpathlineto{\pgfqpoint{5.010278in}{0.912311in}}%
\pgfpathlineto{\pgfqpoint{5.029801in}{0.912311in}}%
\pgfpathlineto{\pgfqpoint{5.049325in}{0.912311in}}%
\pgfpathlineto{\pgfqpoint{5.068849in}{0.912311in}}%
\pgfpathlineto{\pgfqpoint{5.088372in}{0.912311in}}%
\pgfpathlineto{\pgfqpoint{5.107896in}{0.912311in}}%
\pgfpathlineto{\pgfqpoint{5.127419in}{0.912311in}}%
\pgfpathlineto{\pgfqpoint{5.146943in}{0.912311in}}%
\pgfpathlineto{\pgfqpoint{5.166466in}{0.912311in}}%
\pgfpathlineto{\pgfqpoint{5.185990in}{0.912311in}}%
\pgfpathlineto{\pgfqpoint{5.205513in}{0.912311in}}%
\pgfpathlineto{\pgfqpoint{5.225037in}{0.912311in}}%
\pgfpathlineto{\pgfqpoint{5.244561in}{0.912311in}}%
\pgfpathlineto{\pgfqpoint{5.264084in}{0.912311in}}%
\pgfpathlineto{\pgfqpoint{5.283608in}{0.912311in}}%
\pgfpathlineto{\pgfqpoint{5.303131in}{0.912311in}}%
\pgfpathlineto{\pgfqpoint{5.322655in}{0.912311in}}%
\pgfpathlineto{\pgfqpoint{5.342178in}{0.912311in}}%
\pgfpathlineto{\pgfqpoint{5.361702in}{0.912311in}}%
\pgfpathlineto{\pgfqpoint{5.381225in}{0.912311in}}%
\pgfpathlineto{\pgfqpoint{5.400749in}{0.912311in}}%
\pgfpathlineto{\pgfqpoint{5.420272in}{0.912311in}}%
\pgfpathlineto{\pgfqpoint{5.439796in}{0.912311in}}%
\pgfpathlineto{\pgfqpoint{5.459320in}{0.912311in}}%
\pgfpathlineto{\pgfqpoint{5.478843in}{0.912311in}}%
\pgfpathlineto{\pgfqpoint{5.498367in}{0.912311in}}%
\pgfpathlineto{\pgfqpoint{5.517890in}{0.912311in}}%
\pgfpathlineto{\pgfqpoint{5.537414in}{0.912311in}}%
\pgfpathlineto{\pgfqpoint{5.556937in}{0.912311in}}%
\pgfpathlineto{\pgfqpoint{5.576461in}{0.912311in}}%
\pgfpathlineto{\pgfqpoint{5.595984in}{0.912311in}}%
\pgfpathlineto{\pgfqpoint{5.615508in}{0.912311in}}%
\pgfpathlineto{\pgfqpoint{5.635031in}{0.912311in}}%
\pgfpathlineto{\pgfqpoint{5.654555in}{0.912311in}}%
\pgfpathlineto{\pgfqpoint{5.674079in}{0.912311in}}%
\pgfpathlineto{\pgfqpoint{5.693602in}{0.912311in}}%
\pgfpathlineto{\pgfqpoint{5.713126in}{0.912311in}}%
\pgfpathlineto{\pgfqpoint{5.732649in}{0.912311in}}%
\pgfpathlineto{\pgfqpoint{5.752173in}{0.912311in}}%
\pgfpathlineto{\pgfqpoint{5.771696in}{0.900671in}}%
\pgfpathlineto{\pgfqpoint{5.791220in}{0.900671in}}%
\pgfpathlineto{\pgfqpoint{5.810743in}{0.900671in}}%
\pgfpathlineto{\pgfqpoint{5.830267in}{0.900671in}}%
\pgfpathlineto{\pgfqpoint{5.849791in}{0.900671in}}%
\pgfpathlineto{\pgfqpoint{5.869314in}{0.900671in}}%
\pgfpathlineto{\pgfqpoint{5.888838in}{0.900671in}}%
\pgfpathlineto{\pgfqpoint{5.908361in}{0.900671in}}%
\pgfpathlineto{\pgfqpoint{5.927885in}{0.900671in}}%
\pgfpathlineto{\pgfqpoint{5.947408in}{0.900671in}}%
\pgfpathlineto{\pgfqpoint{5.966932in}{0.900671in}}%
\pgfpathlineto{\pgfqpoint{5.966932in}{0.760731in}}%
\pgfpathlineto{\pgfqpoint{5.966932in}{0.760731in}}%
\pgfpathlineto{\pgfqpoint{5.947408in}{0.760731in}}%
\pgfpathlineto{\pgfqpoint{5.927885in}{0.760731in}}%
\pgfpathlineto{\pgfqpoint{5.908361in}{0.760731in}}%
\pgfpathlineto{\pgfqpoint{5.888838in}{0.760731in}}%
\pgfpathlineto{\pgfqpoint{5.869314in}{0.760731in}}%
\pgfpathlineto{\pgfqpoint{5.849791in}{0.760731in}}%
\pgfpathlineto{\pgfqpoint{5.830267in}{0.760731in}}%
\pgfpathlineto{\pgfqpoint{5.810743in}{0.760731in}}%
\pgfpathlineto{\pgfqpoint{5.791220in}{0.760731in}}%
\pgfpathlineto{\pgfqpoint{5.771696in}{0.760731in}}%
\pgfpathlineto{\pgfqpoint{5.752173in}{0.792141in}}%
\pgfpathlineto{\pgfqpoint{5.732649in}{0.792141in}}%
\pgfpathlineto{\pgfqpoint{5.713126in}{0.792141in}}%
\pgfpathlineto{\pgfqpoint{5.693602in}{0.792141in}}%
\pgfpathlineto{\pgfqpoint{5.674079in}{0.792141in}}%
\pgfpathlineto{\pgfqpoint{5.654555in}{0.792141in}}%
\pgfpathlineto{\pgfqpoint{5.635031in}{0.792141in}}%
\pgfpathlineto{\pgfqpoint{5.615508in}{0.792141in}}%
\pgfpathlineto{\pgfqpoint{5.595984in}{0.792141in}}%
\pgfpathlineto{\pgfqpoint{5.576461in}{0.792141in}}%
\pgfpathlineto{\pgfqpoint{5.556937in}{0.792141in}}%
\pgfpathlineto{\pgfqpoint{5.537414in}{0.792141in}}%
\pgfpathlineto{\pgfqpoint{5.517890in}{0.792141in}}%
\pgfpathlineto{\pgfqpoint{5.498367in}{0.792141in}}%
\pgfpathlineto{\pgfqpoint{5.478843in}{0.792141in}}%
\pgfpathlineto{\pgfqpoint{5.459320in}{0.792141in}}%
\pgfpathlineto{\pgfqpoint{5.439796in}{0.792141in}}%
\pgfpathlineto{\pgfqpoint{5.420272in}{0.792141in}}%
\pgfpathlineto{\pgfqpoint{5.400749in}{0.792141in}}%
\pgfpathlineto{\pgfqpoint{5.381225in}{0.792141in}}%
\pgfpathlineto{\pgfqpoint{5.361702in}{0.792141in}}%
\pgfpathlineto{\pgfqpoint{5.342178in}{0.792141in}}%
\pgfpathlineto{\pgfqpoint{5.322655in}{0.792141in}}%
\pgfpathlineto{\pgfqpoint{5.303131in}{0.792141in}}%
\pgfpathlineto{\pgfqpoint{5.283608in}{0.792141in}}%
\pgfpathlineto{\pgfqpoint{5.264084in}{0.792141in}}%
\pgfpathlineto{\pgfqpoint{5.244561in}{0.792141in}}%
\pgfpathlineto{\pgfqpoint{5.225037in}{0.792141in}}%
\pgfpathlineto{\pgfqpoint{5.205513in}{0.792141in}}%
\pgfpathlineto{\pgfqpoint{5.185990in}{0.792141in}}%
\pgfpathlineto{\pgfqpoint{5.166466in}{0.792141in}}%
\pgfpathlineto{\pgfqpoint{5.146943in}{0.792141in}}%
\pgfpathlineto{\pgfqpoint{5.127419in}{0.792141in}}%
\pgfpathlineto{\pgfqpoint{5.107896in}{0.792141in}}%
\pgfpathlineto{\pgfqpoint{5.088372in}{0.792141in}}%
\pgfpathlineto{\pgfqpoint{5.068849in}{0.792141in}}%
\pgfpathlineto{\pgfqpoint{5.049325in}{0.792141in}}%
\pgfpathlineto{\pgfqpoint{5.029801in}{0.792141in}}%
\pgfpathlineto{\pgfqpoint{5.010278in}{0.792141in}}%
\pgfpathlineto{\pgfqpoint{4.990754in}{0.792141in}}%
\pgfpathlineto{\pgfqpoint{4.971231in}{0.792141in}}%
\pgfpathlineto{\pgfqpoint{4.951707in}{0.792141in}}%
\pgfpathlineto{\pgfqpoint{4.932184in}{0.792141in}}%
\pgfpathlineto{\pgfqpoint{4.912660in}{0.792141in}}%
\pgfpathlineto{\pgfqpoint{4.893137in}{0.792141in}}%
\pgfpathlineto{\pgfqpoint{4.873613in}{0.792141in}}%
\pgfpathlineto{\pgfqpoint{4.854090in}{0.806672in}}%
\pgfpathlineto{\pgfqpoint{4.834566in}{0.806672in}}%
\pgfpathlineto{\pgfqpoint{4.815042in}{0.806672in}}%
\pgfpathlineto{\pgfqpoint{4.795519in}{0.806672in}}%
\pgfpathlineto{\pgfqpoint{4.775995in}{0.806672in}}%
\pgfpathlineto{\pgfqpoint{4.756472in}{0.806672in}}%
\pgfpathlineto{\pgfqpoint{4.736948in}{0.806672in}}%
\pgfpathlineto{\pgfqpoint{4.717425in}{0.806672in}}%
\pgfpathlineto{\pgfqpoint{4.697901in}{0.806672in}}%
\pgfpathlineto{\pgfqpoint{4.678378in}{0.806672in}}%
\pgfpathlineto{\pgfqpoint{4.658854in}{0.806672in}}%
\pgfpathlineto{\pgfqpoint{4.639331in}{0.806672in}}%
\pgfpathlineto{\pgfqpoint{4.619807in}{0.833951in}}%
\pgfpathlineto{\pgfqpoint{4.600283in}{0.836604in}}%
\pgfpathlineto{\pgfqpoint{4.580760in}{0.857381in}}%
\pgfpathlineto{\pgfqpoint{4.561236in}{0.877000in}}%
\pgfpathlineto{\pgfqpoint{4.541713in}{0.877000in}}%
\pgfpathlineto{\pgfqpoint{4.522189in}{0.877000in}}%
\pgfpathlineto{\pgfqpoint{4.502666in}{0.877000in}}%
\pgfpathlineto{\pgfqpoint{4.483142in}{0.877000in}}%
\pgfpathlineto{\pgfqpoint{4.463619in}{0.877000in}}%
\pgfpathlineto{\pgfqpoint{4.444095in}{0.877000in}}%
\pgfpathlineto{\pgfqpoint{4.424571in}{0.877000in}}%
\pgfpathlineto{\pgfqpoint{4.405048in}{0.877000in}}%
\pgfpathlineto{\pgfqpoint{4.385524in}{0.877000in}}%
\pgfpathlineto{\pgfqpoint{4.366001in}{0.877000in}}%
\pgfpathlineto{\pgfqpoint{4.346477in}{0.877000in}}%
\pgfpathlineto{\pgfqpoint{4.326954in}{0.877000in}}%
\pgfpathlineto{\pgfqpoint{4.307430in}{0.877000in}}%
\pgfpathlineto{\pgfqpoint{4.287907in}{0.965697in}}%
\pgfpathlineto{\pgfqpoint{4.268383in}{1.002891in}}%
\pgfpathlineto{\pgfqpoint{4.248860in}{1.002891in}}%
\pgfpathlineto{\pgfqpoint{4.229336in}{1.002891in}}%
\pgfpathlineto{\pgfqpoint{4.209812in}{1.002891in}}%
\pgfpathlineto{\pgfqpoint{4.190289in}{1.002891in}}%
\pgfpathlineto{\pgfqpoint{4.170765in}{1.014501in}}%
\pgfpathlineto{\pgfqpoint{4.151242in}{1.014501in}}%
\pgfpathlineto{\pgfqpoint{4.131718in}{1.014501in}}%
\pgfpathlineto{\pgfqpoint{4.112195in}{1.014501in}}%
\pgfpathlineto{\pgfqpoint{4.092671in}{1.014501in}}%
\pgfpathlineto{\pgfqpoint{4.073148in}{1.014501in}}%
\pgfpathlineto{\pgfqpoint{4.053624in}{1.014501in}}%
\pgfpathlineto{\pgfqpoint{4.034100in}{1.014501in}}%
\pgfpathlineto{\pgfqpoint{4.014577in}{1.014501in}}%
\pgfpathlineto{\pgfqpoint{3.995053in}{1.014501in}}%
\pgfpathlineto{\pgfqpoint{3.975530in}{1.014501in}}%
\pgfpathlineto{\pgfqpoint{3.956006in}{1.014501in}}%
\pgfpathlineto{\pgfqpoint{3.936483in}{1.014501in}}%
\pgfpathlineto{\pgfqpoint{3.916959in}{1.014501in}}%
\pgfpathlineto{\pgfqpoint{3.897436in}{1.014501in}}%
\pgfpathlineto{\pgfqpoint{3.877912in}{1.014501in}}%
\pgfpathlineto{\pgfqpoint{3.858389in}{1.014501in}}%
\pgfpathlineto{\pgfqpoint{3.838865in}{1.014501in}}%
\pgfpathlineto{\pgfqpoint{3.819341in}{1.014501in}}%
\pgfpathlineto{\pgfqpoint{3.799818in}{1.014501in}}%
\pgfpathlineto{\pgfqpoint{3.780294in}{1.014501in}}%
\pgfpathlineto{\pgfqpoint{3.760771in}{1.014501in}}%
\pgfpathlineto{\pgfqpoint{3.741247in}{1.014501in}}%
\pgfpathlineto{\pgfqpoint{3.721724in}{1.014501in}}%
\pgfpathlineto{\pgfqpoint{3.702200in}{1.019503in}}%
\pgfpathlineto{\pgfqpoint{3.682677in}{1.019503in}}%
\pgfpathlineto{\pgfqpoint{3.663153in}{1.019503in}}%
\pgfpathlineto{\pgfqpoint{3.643630in}{1.019503in}}%
\pgfpathlineto{\pgfqpoint{3.624106in}{1.019503in}}%
\pgfpathlineto{\pgfqpoint{3.604582in}{1.019503in}}%
\pgfpathlineto{\pgfqpoint{3.585059in}{1.019503in}}%
\pgfpathlineto{\pgfqpoint{3.565535in}{1.019503in}}%
\pgfpathlineto{\pgfqpoint{3.546012in}{1.019503in}}%
\pgfpathlineto{\pgfqpoint{3.526488in}{1.019503in}}%
\pgfpathlineto{\pgfqpoint{3.506965in}{1.019503in}}%
\pgfpathlineto{\pgfqpoint{3.487441in}{1.106155in}}%
\pgfpathlineto{\pgfqpoint{3.467918in}{1.108107in}}%
\pgfpathlineto{\pgfqpoint{3.448394in}{1.108107in}}%
\pgfpathlineto{\pgfqpoint{3.428870in}{1.108107in}}%
\pgfpathlineto{\pgfqpoint{3.409347in}{1.108107in}}%
\pgfpathlineto{\pgfqpoint{3.389823in}{1.108107in}}%
\pgfpathlineto{\pgfqpoint{3.370300in}{1.108107in}}%
\pgfpathlineto{\pgfqpoint{3.350776in}{1.108107in}}%
\pgfpathlineto{\pgfqpoint{3.331253in}{1.108107in}}%
\pgfpathlineto{\pgfqpoint{3.311729in}{1.108107in}}%
\pgfpathlineto{\pgfqpoint{3.292206in}{1.108107in}}%
\pgfpathlineto{\pgfqpoint{3.272682in}{1.108107in}}%
\pgfpathlineto{\pgfqpoint{3.253159in}{1.114831in}}%
\pgfpathlineto{\pgfqpoint{3.233635in}{1.114831in}}%
\pgfpathlineto{\pgfqpoint{3.214111in}{1.114831in}}%
\pgfpathlineto{\pgfqpoint{3.194588in}{1.144924in}}%
\pgfpathlineto{\pgfqpoint{3.175064in}{1.220928in}}%
\pgfpathlineto{\pgfqpoint{3.155541in}{1.235200in}}%
\pgfpathlineto{\pgfqpoint{3.136017in}{1.235200in}}%
\pgfpathlineto{\pgfqpoint{3.116494in}{1.254026in}}%
\pgfpathlineto{\pgfqpoint{3.096970in}{1.254026in}}%
\pgfpathlineto{\pgfqpoint{3.077447in}{1.254026in}}%
\pgfpathlineto{\pgfqpoint{3.057923in}{1.268157in}}%
\pgfpathlineto{\pgfqpoint{3.038400in}{1.268157in}}%
\pgfpathlineto{\pgfqpoint{3.018876in}{1.268157in}}%
\pgfpathlineto{\pgfqpoint{2.999352in}{1.292334in}}%
\pgfpathlineto{\pgfqpoint{2.979829in}{1.292334in}}%
\pgfpathlineto{\pgfqpoint{2.960305in}{1.292334in}}%
\pgfpathlineto{\pgfqpoint{2.940782in}{1.292334in}}%
\pgfpathlineto{\pgfqpoint{2.921258in}{1.292334in}}%
\pgfpathlineto{\pgfqpoint{2.901735in}{1.292334in}}%
\pgfpathlineto{\pgfqpoint{2.882211in}{1.292334in}}%
\pgfpathlineto{\pgfqpoint{2.862688in}{1.292334in}}%
\pgfpathlineto{\pgfqpoint{2.843164in}{1.374770in}}%
\pgfpathlineto{\pgfqpoint{2.823640in}{1.374770in}}%
\pgfpathlineto{\pgfqpoint{2.804117in}{1.374770in}}%
\pgfpathlineto{\pgfqpoint{2.784593in}{1.374770in}}%
\pgfpathlineto{\pgfqpoint{2.765070in}{1.374770in}}%
\pgfpathlineto{\pgfqpoint{2.745546in}{1.381064in}}%
\pgfpathlineto{\pgfqpoint{2.726023in}{1.381064in}}%
\pgfpathlineto{\pgfqpoint{2.706499in}{1.399285in}}%
\pgfpathlineto{\pgfqpoint{2.686976in}{1.399285in}}%
\pgfpathlineto{\pgfqpoint{2.667452in}{1.399285in}}%
\pgfpathlineto{\pgfqpoint{2.647929in}{1.399285in}}%
\pgfpathlineto{\pgfqpoint{2.628405in}{1.424817in}}%
\pgfpathlineto{\pgfqpoint{2.608881in}{1.425366in}}%
\pgfpathlineto{\pgfqpoint{2.589358in}{1.425366in}}%
\pgfpathlineto{\pgfqpoint{2.569834in}{1.425366in}}%
\pgfpathlineto{\pgfqpoint{2.550311in}{1.482907in}}%
\pgfpathlineto{\pgfqpoint{2.530787in}{1.562302in}}%
\pgfpathlineto{\pgfqpoint{2.511264in}{1.577495in}}%
\pgfpathlineto{\pgfqpoint{2.491740in}{1.586122in}}%
\pgfpathlineto{\pgfqpoint{2.472217in}{1.599881in}}%
\pgfpathlineto{\pgfqpoint{2.452693in}{1.599881in}}%
\pgfpathlineto{\pgfqpoint{2.433169in}{1.599881in}}%
\pgfpathlineto{\pgfqpoint{2.413646in}{1.659028in}}%
\pgfpathlineto{\pgfqpoint{2.394122in}{1.659028in}}%
\pgfpathlineto{\pgfqpoint{2.374599in}{1.659028in}}%
\pgfpathlineto{\pgfqpoint{2.355075in}{1.660648in}}%
\pgfpathlineto{\pgfqpoint{2.335552in}{1.668926in}}%
\pgfpathlineto{\pgfqpoint{2.316028in}{1.681796in}}%
\pgfpathlineto{\pgfqpoint{2.296505in}{1.717239in}}%
\pgfpathlineto{\pgfqpoint{2.276981in}{1.717239in}}%
\pgfpathlineto{\pgfqpoint{2.257458in}{1.718177in}}%
\pgfpathlineto{\pgfqpoint{2.237934in}{1.859451in}}%
\pgfpathlineto{\pgfqpoint{2.218410in}{1.924413in}}%
\pgfpathlineto{\pgfqpoint{2.198887in}{1.924413in}}%
\pgfpathlineto{\pgfqpoint{2.179363in}{1.953050in}}%
\pgfpathlineto{\pgfqpoint{2.159840in}{1.953050in}}%
\pgfpathlineto{\pgfqpoint{2.140316in}{1.953050in}}%
\pgfpathlineto{\pgfqpoint{2.120793in}{1.953050in}}%
\pgfpathlineto{\pgfqpoint{2.101269in}{1.953050in}}%
\pgfpathlineto{\pgfqpoint{2.081746in}{1.953050in}}%
\pgfpathlineto{\pgfqpoint{2.062222in}{1.996884in}}%
\pgfpathlineto{\pgfqpoint{2.042699in}{1.996884in}}%
\pgfpathlineto{\pgfqpoint{2.023175in}{1.996884in}}%
\pgfpathlineto{\pgfqpoint{2.003651in}{1.996884in}}%
\pgfpathlineto{\pgfqpoint{1.984128in}{1.996884in}}%
\pgfpathlineto{\pgfqpoint{1.964604in}{1.996884in}}%
\pgfpathlineto{\pgfqpoint{1.945081in}{1.996884in}}%
\pgfpathlineto{\pgfqpoint{1.925557in}{1.996884in}}%
\pgfpathlineto{\pgfqpoint{1.906034in}{1.996884in}}%
\pgfpathlineto{\pgfqpoint{1.886510in}{1.996884in}}%
\pgfpathlineto{\pgfqpoint{1.866987in}{1.996884in}}%
\pgfpathlineto{\pgfqpoint{1.847463in}{1.996884in}}%
\pgfpathlineto{\pgfqpoint{1.827939in}{1.996884in}}%
\pgfpathlineto{\pgfqpoint{1.808416in}{1.996884in}}%
\pgfpathlineto{\pgfqpoint{1.788892in}{1.996884in}}%
\pgfpathlineto{\pgfqpoint{1.769369in}{1.996884in}}%
\pgfpathlineto{\pgfqpoint{1.749845in}{1.996884in}}%
\pgfpathlineto{\pgfqpoint{1.730322in}{2.048019in}}%
\pgfpathlineto{\pgfqpoint{1.710798in}{2.068813in}}%
\pgfpathlineto{\pgfqpoint{1.691275in}{2.068813in}}%
\pgfpathlineto{\pgfqpoint{1.671751in}{2.090218in}}%
\pgfpathlineto{\pgfqpoint{1.652228in}{2.090218in}}%
\pgfpathlineto{\pgfqpoint{1.632704in}{2.090218in}}%
\pgfpathlineto{\pgfqpoint{1.613180in}{2.090218in}}%
\pgfpathlineto{\pgfqpoint{1.593657in}{2.090218in}}%
\pgfpathlineto{\pgfqpoint{1.574133in}{2.090218in}}%
\pgfpathlineto{\pgfqpoint{1.554610in}{2.090218in}}%
\pgfpathlineto{\pgfqpoint{1.535086in}{2.090218in}}%
\pgfpathlineto{\pgfqpoint{1.515563in}{2.090218in}}%
\pgfpathlineto{\pgfqpoint{1.496039in}{2.090218in}}%
\pgfpathlineto{\pgfqpoint{1.476516in}{2.090218in}}%
\pgfpathlineto{\pgfqpoint{1.456992in}{2.090218in}}%
\pgfpathlineto{\pgfqpoint{1.437469in}{2.158254in}}%
\pgfpathlineto{\pgfqpoint{1.417945in}{2.158254in}}%
\pgfpathlineto{\pgfqpoint{1.398421in}{2.158254in}}%
\pgfpathlineto{\pgfqpoint{1.378898in}{2.158254in}}%
\pgfpathlineto{\pgfqpoint{1.359374in}{2.225213in}}%
\pgfpathlineto{\pgfqpoint{1.339851in}{2.225213in}}%
\pgfpathlineto{\pgfqpoint{1.320327in}{2.227343in}}%
\pgfpathlineto{\pgfqpoint{1.300804in}{2.227343in}}%
\pgfpathlineto{\pgfqpoint{1.281280in}{2.227343in}}%
\pgfpathlineto{\pgfqpoint{1.261757in}{2.227343in}}%
\pgfpathlineto{\pgfqpoint{1.242233in}{2.227343in}}%
\pgfpathlineto{\pgfqpoint{1.222709in}{2.227343in}}%
\pgfpathlineto{\pgfqpoint{1.203186in}{2.261825in}}%
\pgfpathlineto{\pgfqpoint{1.183662in}{2.288316in}}%
\pgfpathlineto{\pgfqpoint{1.164139in}{2.288316in}}%
\pgfpathlineto{\pgfqpoint{1.144615in}{2.365232in}}%
\pgfpathlineto{\pgfqpoint{1.125092in}{2.437937in}}%
\pgfpathlineto{\pgfqpoint{1.105568in}{2.451037in}}%
\pgfpathclose%
\pgfusepath{fill}%
\end{pgfscope}%
\begin{pgfscope}%
\pgfpathrectangle{\pgfqpoint{0.862500in}{0.375000in}}{\pgfqpoint{5.347500in}{2.265000in}}%
\pgfusepath{clip}%
\pgfsetbuttcap%
\pgfsetroundjoin%
\definecolor{currentfill}{rgb}{0.172549,0.627451,0.172549}%
\pgfsetfillcolor{currentfill}%
\pgfsetfillopacity{0.200000}%
\pgfsetlinewidth{0.000000pt}%
\definecolor{currentstroke}{rgb}{0.000000,0.000000,0.000000}%
\pgfsetstrokecolor{currentstroke}%
\pgfsetdash{}{0pt}%
\pgfpathmoveto{\pgfqpoint{1.105568in}{2.468219in}}%
\pgfpathlineto{\pgfqpoint{1.105568in}{2.499890in}}%
\pgfpathlineto{\pgfqpoint{1.125092in}{2.493734in}}%
\pgfpathlineto{\pgfqpoint{1.144615in}{2.475221in}}%
\pgfpathlineto{\pgfqpoint{1.164139in}{2.475221in}}%
\pgfpathlineto{\pgfqpoint{1.183662in}{2.437004in}}%
\pgfpathlineto{\pgfqpoint{1.203186in}{2.327944in}}%
\pgfpathlineto{\pgfqpoint{1.222709in}{2.306641in}}%
\pgfpathlineto{\pgfqpoint{1.242233in}{2.299436in}}%
\pgfpathlineto{\pgfqpoint{1.261757in}{2.299436in}}%
\pgfpathlineto{\pgfqpoint{1.281280in}{2.299436in}}%
\pgfpathlineto{\pgfqpoint{1.300804in}{2.299436in}}%
\pgfpathlineto{\pgfqpoint{1.320327in}{2.299436in}}%
\pgfpathlineto{\pgfqpoint{1.339851in}{2.277997in}}%
\pgfpathlineto{\pgfqpoint{1.359374in}{2.277997in}}%
\pgfpathlineto{\pgfqpoint{1.378898in}{2.277997in}}%
\pgfpathlineto{\pgfqpoint{1.398421in}{2.272482in}}%
\pgfpathlineto{\pgfqpoint{1.417945in}{2.243546in}}%
\pgfpathlineto{\pgfqpoint{1.437469in}{2.243546in}}%
\pgfpathlineto{\pgfqpoint{1.456992in}{2.243546in}}%
\pgfpathlineto{\pgfqpoint{1.476516in}{2.243546in}}%
\pgfpathlineto{\pgfqpoint{1.496039in}{2.243546in}}%
\pgfpathlineto{\pgfqpoint{1.515563in}{2.243546in}}%
\pgfpathlineto{\pgfqpoint{1.535086in}{2.243546in}}%
\pgfpathlineto{\pgfqpoint{1.554610in}{2.243546in}}%
\pgfpathlineto{\pgfqpoint{1.574133in}{2.243546in}}%
\pgfpathlineto{\pgfqpoint{1.593657in}{2.243546in}}%
\pgfpathlineto{\pgfqpoint{1.613180in}{2.216267in}}%
\pgfpathlineto{\pgfqpoint{1.632704in}{2.216267in}}%
\pgfpathlineto{\pgfqpoint{1.652228in}{2.216267in}}%
\pgfpathlineto{\pgfqpoint{1.671751in}{2.216267in}}%
\pgfpathlineto{\pgfqpoint{1.691275in}{2.216267in}}%
\pgfpathlineto{\pgfqpoint{1.710798in}{2.216267in}}%
\pgfpathlineto{\pgfqpoint{1.730322in}{2.216267in}}%
\pgfpathlineto{\pgfqpoint{1.749845in}{2.216267in}}%
\pgfpathlineto{\pgfqpoint{1.769369in}{2.216267in}}%
\pgfpathlineto{\pgfqpoint{1.788892in}{2.190149in}}%
\pgfpathlineto{\pgfqpoint{1.808416in}{2.190149in}}%
\pgfpathlineto{\pgfqpoint{1.827939in}{2.190149in}}%
\pgfpathlineto{\pgfqpoint{1.847463in}{2.190149in}}%
\pgfpathlineto{\pgfqpoint{1.866987in}{2.190149in}}%
\pgfpathlineto{\pgfqpoint{1.886510in}{2.190149in}}%
\pgfpathlineto{\pgfqpoint{1.906034in}{2.190149in}}%
\pgfpathlineto{\pgfqpoint{1.925557in}{2.190149in}}%
\pgfpathlineto{\pgfqpoint{1.945081in}{2.190149in}}%
\pgfpathlineto{\pgfqpoint{1.964604in}{2.190149in}}%
\pgfpathlineto{\pgfqpoint{1.984128in}{2.190149in}}%
\pgfpathlineto{\pgfqpoint{2.003651in}{2.190149in}}%
\pgfpathlineto{\pgfqpoint{2.023175in}{2.190149in}}%
\pgfpathlineto{\pgfqpoint{2.042699in}{2.190149in}}%
\pgfpathlineto{\pgfqpoint{2.062222in}{2.190149in}}%
\pgfpathlineto{\pgfqpoint{2.081746in}{2.153142in}}%
\pgfpathlineto{\pgfqpoint{2.101269in}{2.153002in}}%
\pgfpathlineto{\pgfqpoint{2.120793in}{2.003067in}}%
\pgfpathlineto{\pgfqpoint{2.140316in}{2.003067in}}%
\pgfpathlineto{\pgfqpoint{2.159840in}{2.001606in}}%
\pgfpathlineto{\pgfqpoint{2.179363in}{1.945597in}}%
\pgfpathlineto{\pgfqpoint{2.198887in}{1.917415in}}%
\pgfpathlineto{\pgfqpoint{2.218410in}{1.905963in}}%
\pgfpathlineto{\pgfqpoint{2.237934in}{1.804818in}}%
\pgfpathlineto{\pgfqpoint{2.257458in}{1.804818in}}%
\pgfpathlineto{\pgfqpoint{2.276981in}{1.800718in}}%
\pgfpathlineto{\pgfqpoint{2.296505in}{1.752307in}}%
\pgfpathlineto{\pgfqpoint{2.316028in}{1.751812in}}%
\pgfpathlineto{\pgfqpoint{2.335552in}{1.736978in}}%
\pgfpathlineto{\pgfqpoint{2.355075in}{1.736978in}}%
\pgfpathlineto{\pgfqpoint{2.374599in}{1.730964in}}%
\pgfpathlineto{\pgfqpoint{2.394122in}{1.730964in}}%
\pgfpathlineto{\pgfqpoint{2.413646in}{1.730964in}}%
\pgfpathlineto{\pgfqpoint{2.433169in}{1.730964in}}%
\pgfpathlineto{\pgfqpoint{2.452693in}{1.730964in}}%
\pgfpathlineto{\pgfqpoint{2.472217in}{1.730964in}}%
\pgfpathlineto{\pgfqpoint{2.491740in}{1.728733in}}%
\pgfpathlineto{\pgfqpoint{2.511264in}{1.728733in}}%
\pgfpathlineto{\pgfqpoint{2.530787in}{1.714262in}}%
\pgfpathlineto{\pgfqpoint{2.550311in}{1.696649in}}%
\pgfpathlineto{\pgfqpoint{2.569834in}{1.696649in}}%
\pgfpathlineto{\pgfqpoint{2.589358in}{1.696163in}}%
\pgfpathlineto{\pgfqpoint{2.608881in}{1.696163in}}%
\pgfpathlineto{\pgfqpoint{2.628405in}{1.696163in}}%
\pgfpathlineto{\pgfqpoint{2.647929in}{1.696163in}}%
\pgfpathlineto{\pgfqpoint{2.667452in}{1.696163in}}%
\pgfpathlineto{\pgfqpoint{2.686976in}{1.657534in}}%
\pgfpathlineto{\pgfqpoint{2.706499in}{1.643387in}}%
\pgfpathlineto{\pgfqpoint{2.726023in}{1.643387in}}%
\pgfpathlineto{\pgfqpoint{2.745546in}{1.620510in}}%
\pgfpathlineto{\pgfqpoint{2.765070in}{1.620510in}}%
\pgfpathlineto{\pgfqpoint{2.784593in}{1.620510in}}%
\pgfpathlineto{\pgfqpoint{2.804117in}{1.600196in}}%
\pgfpathlineto{\pgfqpoint{2.823640in}{1.600196in}}%
\pgfpathlineto{\pgfqpoint{2.843164in}{1.597454in}}%
\pgfpathlineto{\pgfqpoint{2.862688in}{1.597454in}}%
\pgfpathlineto{\pgfqpoint{2.882211in}{1.593032in}}%
\pgfpathlineto{\pgfqpoint{2.901735in}{1.574070in}}%
\pgfpathlineto{\pgfqpoint{2.921258in}{1.574070in}}%
\pgfpathlineto{\pgfqpoint{2.940782in}{1.574070in}}%
\pgfpathlineto{\pgfqpoint{2.960305in}{1.574070in}}%
\pgfpathlineto{\pgfqpoint{2.979829in}{1.572426in}}%
\pgfpathlineto{\pgfqpoint{2.999352in}{1.558000in}}%
\pgfpathlineto{\pgfqpoint{3.018876in}{1.555242in}}%
\pgfpathlineto{\pgfqpoint{3.038400in}{1.555242in}}%
\pgfpathlineto{\pgfqpoint{3.057923in}{1.512307in}}%
\pgfpathlineto{\pgfqpoint{3.077447in}{1.489455in}}%
\pgfpathlineto{\pgfqpoint{3.096970in}{1.489455in}}%
\pgfpathlineto{\pgfqpoint{3.116494in}{1.463921in}}%
\pgfpathlineto{\pgfqpoint{3.136017in}{1.442064in}}%
\pgfpathlineto{\pgfqpoint{3.155541in}{1.408568in}}%
\pgfpathlineto{\pgfqpoint{3.175064in}{1.408568in}}%
\pgfpathlineto{\pgfqpoint{3.194588in}{1.408568in}}%
\pgfpathlineto{\pgfqpoint{3.214111in}{1.408568in}}%
\pgfpathlineto{\pgfqpoint{3.233635in}{1.408568in}}%
\pgfpathlineto{\pgfqpoint{3.253159in}{1.385266in}}%
\pgfpathlineto{\pgfqpoint{3.272682in}{1.385266in}}%
\pgfpathlineto{\pgfqpoint{3.292206in}{1.385266in}}%
\pgfpathlineto{\pgfqpoint{3.311729in}{1.372120in}}%
\pgfpathlineto{\pgfqpoint{3.331253in}{1.372120in}}%
\pgfpathlineto{\pgfqpoint{3.350776in}{1.372120in}}%
\pgfpathlineto{\pgfqpoint{3.370300in}{1.360407in}}%
\pgfpathlineto{\pgfqpoint{3.389823in}{1.360407in}}%
\pgfpathlineto{\pgfqpoint{3.409347in}{1.320591in}}%
\pgfpathlineto{\pgfqpoint{3.428870in}{1.320591in}}%
\pgfpathlineto{\pgfqpoint{3.448394in}{1.320591in}}%
\pgfpathlineto{\pgfqpoint{3.467918in}{1.307073in}}%
\pgfpathlineto{\pgfqpoint{3.487441in}{1.307073in}}%
\pgfpathlineto{\pgfqpoint{3.506965in}{1.307073in}}%
\pgfpathlineto{\pgfqpoint{3.526488in}{1.307073in}}%
\pgfpathlineto{\pgfqpoint{3.546012in}{1.307073in}}%
\pgfpathlineto{\pgfqpoint{3.565535in}{1.307073in}}%
\pgfpathlineto{\pgfqpoint{3.585059in}{1.307073in}}%
\pgfpathlineto{\pgfqpoint{3.604582in}{1.307073in}}%
\pgfpathlineto{\pgfqpoint{3.624106in}{1.305289in}}%
\pgfpathlineto{\pgfqpoint{3.643630in}{1.305289in}}%
\pgfpathlineto{\pgfqpoint{3.663153in}{1.305289in}}%
\pgfpathlineto{\pgfqpoint{3.682677in}{1.305289in}}%
\pgfpathlineto{\pgfqpoint{3.702200in}{1.305289in}}%
\pgfpathlineto{\pgfqpoint{3.721724in}{1.305289in}}%
\pgfpathlineto{\pgfqpoint{3.741247in}{1.305289in}}%
\pgfpathlineto{\pgfqpoint{3.760771in}{1.305289in}}%
\pgfpathlineto{\pgfqpoint{3.780294in}{1.305289in}}%
\pgfpathlineto{\pgfqpoint{3.799818in}{1.305289in}}%
\pgfpathlineto{\pgfqpoint{3.819341in}{1.305289in}}%
\pgfpathlineto{\pgfqpoint{3.838865in}{1.299053in}}%
\pgfpathlineto{\pgfqpoint{3.858389in}{1.299053in}}%
\pgfpathlineto{\pgfqpoint{3.877912in}{1.299053in}}%
\pgfpathlineto{\pgfqpoint{3.897436in}{1.299053in}}%
\pgfpathlineto{\pgfqpoint{3.916959in}{1.299053in}}%
\pgfpathlineto{\pgfqpoint{3.936483in}{1.275416in}}%
\pgfpathlineto{\pgfqpoint{3.956006in}{1.249048in}}%
\pgfpathlineto{\pgfqpoint{3.975530in}{1.249048in}}%
\pgfpathlineto{\pgfqpoint{3.995053in}{1.178939in}}%
\pgfpathlineto{\pgfqpoint{4.014577in}{1.178939in}}%
\pgfpathlineto{\pgfqpoint{4.034100in}{1.178939in}}%
\pgfpathlineto{\pgfqpoint{4.053624in}{1.178939in}}%
\pgfpathlineto{\pgfqpoint{4.073148in}{1.178939in}}%
\pgfpathlineto{\pgfqpoint{4.092671in}{1.151331in}}%
\pgfpathlineto{\pgfqpoint{4.112195in}{1.151331in}}%
\pgfpathlineto{\pgfqpoint{4.131718in}{1.148504in}}%
\pgfpathlineto{\pgfqpoint{4.151242in}{1.148504in}}%
\pgfpathlineto{\pgfqpoint{4.170765in}{1.148504in}}%
\pgfpathlineto{\pgfqpoint{4.190289in}{1.148504in}}%
\pgfpathlineto{\pgfqpoint{4.209812in}{1.148504in}}%
\pgfpathlineto{\pgfqpoint{4.229336in}{1.148504in}}%
\pgfpathlineto{\pgfqpoint{4.248860in}{1.148504in}}%
\pgfpathlineto{\pgfqpoint{4.268383in}{1.148504in}}%
\pgfpathlineto{\pgfqpoint{4.287907in}{1.101879in}}%
\pgfpathlineto{\pgfqpoint{4.307430in}{1.101879in}}%
\pgfpathlineto{\pgfqpoint{4.326954in}{1.101879in}}%
\pgfpathlineto{\pgfqpoint{4.346477in}{1.101879in}}%
\pgfpathlineto{\pgfqpoint{4.366001in}{1.101879in}}%
\pgfpathlineto{\pgfqpoint{4.385524in}{1.101879in}}%
\pgfpathlineto{\pgfqpoint{4.405048in}{1.101879in}}%
\pgfpathlineto{\pgfqpoint{4.424571in}{1.101879in}}%
\pgfpathlineto{\pgfqpoint{4.444095in}{1.101879in}}%
\pgfpathlineto{\pgfqpoint{4.463619in}{1.101879in}}%
\pgfpathlineto{\pgfqpoint{4.483142in}{1.101879in}}%
\pgfpathlineto{\pgfqpoint{4.502666in}{1.101879in}}%
\pgfpathlineto{\pgfqpoint{4.522189in}{1.101879in}}%
\pgfpathlineto{\pgfqpoint{4.541713in}{1.101879in}}%
\pgfpathlineto{\pgfqpoint{4.561236in}{1.098627in}}%
\pgfpathlineto{\pgfqpoint{4.580760in}{1.098627in}}%
\pgfpathlineto{\pgfqpoint{4.600283in}{1.098627in}}%
\pgfpathlineto{\pgfqpoint{4.619807in}{1.098627in}}%
\pgfpathlineto{\pgfqpoint{4.639331in}{1.098627in}}%
\pgfpathlineto{\pgfqpoint{4.658854in}{0.975541in}}%
\pgfpathlineto{\pgfqpoint{4.678378in}{0.975541in}}%
\pgfpathlineto{\pgfqpoint{4.697901in}{0.975541in}}%
\pgfpathlineto{\pgfqpoint{4.717425in}{0.947812in}}%
\pgfpathlineto{\pgfqpoint{4.736948in}{0.947812in}}%
\pgfpathlineto{\pgfqpoint{4.756472in}{0.947812in}}%
\pgfpathlineto{\pgfqpoint{4.775995in}{0.947812in}}%
\pgfpathlineto{\pgfqpoint{4.795519in}{0.947812in}}%
\pgfpathlineto{\pgfqpoint{4.815042in}{0.947812in}}%
\pgfpathlineto{\pgfqpoint{4.834566in}{0.926726in}}%
\pgfpathlineto{\pgfqpoint{4.854090in}{0.926726in}}%
\pgfpathlineto{\pgfqpoint{4.873613in}{0.926726in}}%
\pgfpathlineto{\pgfqpoint{4.893137in}{0.926726in}}%
\pgfpathlineto{\pgfqpoint{4.912660in}{0.926726in}}%
\pgfpathlineto{\pgfqpoint{4.932184in}{0.926726in}}%
\pgfpathlineto{\pgfqpoint{4.951707in}{0.926726in}}%
\pgfpathlineto{\pgfqpoint{4.971231in}{0.926726in}}%
\pgfpathlineto{\pgfqpoint{4.990754in}{0.926726in}}%
\pgfpathlineto{\pgfqpoint{5.010278in}{0.926726in}}%
\pgfpathlineto{\pgfqpoint{5.029801in}{0.926726in}}%
\pgfpathlineto{\pgfqpoint{5.049325in}{0.926726in}}%
\pgfpathlineto{\pgfqpoint{5.068849in}{0.926726in}}%
\pgfpathlineto{\pgfqpoint{5.088372in}{0.926726in}}%
\pgfpathlineto{\pgfqpoint{5.107896in}{0.926726in}}%
\pgfpathlineto{\pgfqpoint{5.127419in}{0.926726in}}%
\pgfpathlineto{\pgfqpoint{5.146943in}{0.897754in}}%
\pgfpathlineto{\pgfqpoint{5.166466in}{0.897754in}}%
\pgfpathlineto{\pgfqpoint{5.185990in}{0.897754in}}%
\pgfpathlineto{\pgfqpoint{5.205513in}{0.897754in}}%
\pgfpathlineto{\pgfqpoint{5.225037in}{0.897754in}}%
\pgfpathlineto{\pgfqpoint{5.244561in}{0.897754in}}%
\pgfpathlineto{\pgfqpoint{5.264084in}{0.897754in}}%
\pgfpathlineto{\pgfqpoint{5.283608in}{0.897754in}}%
\pgfpathlineto{\pgfqpoint{5.303131in}{0.897754in}}%
\pgfpathlineto{\pgfqpoint{5.322655in}{0.897754in}}%
\pgfpathlineto{\pgfqpoint{5.342178in}{0.897754in}}%
\pgfpathlineto{\pgfqpoint{5.361702in}{0.897754in}}%
\pgfpathlineto{\pgfqpoint{5.381225in}{0.897754in}}%
\pgfpathlineto{\pgfqpoint{5.400749in}{0.897754in}}%
\pgfpathlineto{\pgfqpoint{5.420272in}{0.897754in}}%
\pgfpathlineto{\pgfqpoint{5.439796in}{0.897754in}}%
\pgfpathlineto{\pgfqpoint{5.459320in}{0.897754in}}%
\pgfpathlineto{\pgfqpoint{5.478843in}{0.897754in}}%
\pgfpathlineto{\pgfqpoint{5.498367in}{0.897754in}}%
\pgfpathlineto{\pgfqpoint{5.517890in}{0.897754in}}%
\pgfpathlineto{\pgfqpoint{5.537414in}{0.897754in}}%
\pgfpathlineto{\pgfqpoint{5.556937in}{0.897754in}}%
\pgfpathlineto{\pgfqpoint{5.576461in}{0.897754in}}%
\pgfpathlineto{\pgfqpoint{5.595984in}{0.897754in}}%
\pgfpathlineto{\pgfqpoint{5.615508in}{0.897754in}}%
\pgfpathlineto{\pgfqpoint{5.635031in}{0.897754in}}%
\pgfpathlineto{\pgfqpoint{5.654555in}{0.897754in}}%
\pgfpathlineto{\pgfqpoint{5.674079in}{0.831076in}}%
\pgfpathlineto{\pgfqpoint{5.693602in}{0.831076in}}%
\pgfpathlineto{\pgfqpoint{5.713126in}{0.794638in}}%
\pgfpathlineto{\pgfqpoint{5.732649in}{0.794638in}}%
\pgfpathlineto{\pgfqpoint{5.752173in}{0.794638in}}%
\pgfpathlineto{\pgfqpoint{5.771696in}{0.794638in}}%
\pgfpathlineto{\pgfqpoint{5.791220in}{0.794638in}}%
\pgfpathlineto{\pgfqpoint{5.810743in}{0.750883in}}%
\pgfpathlineto{\pgfqpoint{5.830267in}{0.750883in}}%
\pgfpathlineto{\pgfqpoint{5.849791in}{0.745531in}}%
\pgfpathlineto{\pgfqpoint{5.869314in}{0.745531in}}%
\pgfpathlineto{\pgfqpoint{5.888838in}{0.724184in}}%
\pgfpathlineto{\pgfqpoint{5.908361in}{0.724184in}}%
\pgfpathlineto{\pgfqpoint{5.927885in}{0.724184in}}%
\pgfpathlineto{\pgfqpoint{5.947408in}{0.724184in}}%
\pgfpathlineto{\pgfqpoint{5.966932in}{0.724184in}}%
\pgfpathlineto{\pgfqpoint{5.966932in}{0.619166in}}%
\pgfpathlineto{\pgfqpoint{5.966932in}{0.619166in}}%
\pgfpathlineto{\pgfqpoint{5.947408in}{0.619166in}}%
\pgfpathlineto{\pgfqpoint{5.927885in}{0.619166in}}%
\pgfpathlineto{\pgfqpoint{5.908361in}{0.619166in}}%
\pgfpathlineto{\pgfqpoint{5.888838in}{0.619166in}}%
\pgfpathlineto{\pgfqpoint{5.869314in}{0.632082in}}%
\pgfpathlineto{\pgfqpoint{5.849791in}{0.632082in}}%
\pgfpathlineto{\pgfqpoint{5.830267in}{0.677664in}}%
\pgfpathlineto{\pgfqpoint{5.810743in}{0.677664in}}%
\pgfpathlineto{\pgfqpoint{5.791220in}{0.686648in}}%
\pgfpathlineto{\pgfqpoint{5.771696in}{0.686648in}}%
\pgfpathlineto{\pgfqpoint{5.752173in}{0.686648in}}%
\pgfpathlineto{\pgfqpoint{5.732649in}{0.686648in}}%
\pgfpathlineto{\pgfqpoint{5.713126in}{0.686648in}}%
\pgfpathlineto{\pgfqpoint{5.693602in}{0.718721in}}%
\pgfpathlineto{\pgfqpoint{5.674079in}{0.718721in}}%
\pgfpathlineto{\pgfqpoint{5.654555in}{0.738035in}}%
\pgfpathlineto{\pgfqpoint{5.635031in}{0.738035in}}%
\pgfpathlineto{\pgfqpoint{5.615508in}{0.738035in}}%
\pgfpathlineto{\pgfqpoint{5.595984in}{0.738035in}}%
\pgfpathlineto{\pgfqpoint{5.576461in}{0.738035in}}%
\pgfpathlineto{\pgfqpoint{5.556937in}{0.738035in}}%
\pgfpathlineto{\pgfqpoint{5.537414in}{0.738035in}}%
\pgfpathlineto{\pgfqpoint{5.517890in}{0.738035in}}%
\pgfpathlineto{\pgfqpoint{5.498367in}{0.738035in}}%
\pgfpathlineto{\pgfqpoint{5.478843in}{0.738035in}}%
\pgfpathlineto{\pgfqpoint{5.459320in}{0.738035in}}%
\pgfpathlineto{\pgfqpoint{5.439796in}{0.738035in}}%
\pgfpathlineto{\pgfqpoint{5.420272in}{0.738035in}}%
\pgfpathlineto{\pgfqpoint{5.400749in}{0.738035in}}%
\pgfpathlineto{\pgfqpoint{5.381225in}{0.738035in}}%
\pgfpathlineto{\pgfqpoint{5.361702in}{0.738035in}}%
\pgfpathlineto{\pgfqpoint{5.342178in}{0.738035in}}%
\pgfpathlineto{\pgfqpoint{5.322655in}{0.738035in}}%
\pgfpathlineto{\pgfqpoint{5.303131in}{0.738035in}}%
\pgfpathlineto{\pgfqpoint{5.283608in}{0.738035in}}%
\pgfpathlineto{\pgfqpoint{5.264084in}{0.738035in}}%
\pgfpathlineto{\pgfqpoint{5.244561in}{0.738035in}}%
\pgfpathlineto{\pgfqpoint{5.225037in}{0.738035in}}%
\pgfpathlineto{\pgfqpoint{5.205513in}{0.738035in}}%
\pgfpathlineto{\pgfqpoint{5.185990in}{0.738035in}}%
\pgfpathlineto{\pgfqpoint{5.166466in}{0.738035in}}%
\pgfpathlineto{\pgfqpoint{5.146943in}{0.738035in}}%
\pgfpathlineto{\pgfqpoint{5.127419in}{0.809897in}}%
\pgfpathlineto{\pgfqpoint{5.107896in}{0.809897in}}%
\pgfpathlineto{\pgfqpoint{5.088372in}{0.809897in}}%
\pgfpathlineto{\pgfqpoint{5.068849in}{0.809897in}}%
\pgfpathlineto{\pgfqpoint{5.049325in}{0.809897in}}%
\pgfpathlineto{\pgfqpoint{5.029801in}{0.809897in}}%
\pgfpathlineto{\pgfqpoint{5.010278in}{0.809897in}}%
\pgfpathlineto{\pgfqpoint{4.990754in}{0.809897in}}%
\pgfpathlineto{\pgfqpoint{4.971231in}{0.809897in}}%
\pgfpathlineto{\pgfqpoint{4.951707in}{0.809897in}}%
\pgfpathlineto{\pgfqpoint{4.932184in}{0.809897in}}%
\pgfpathlineto{\pgfqpoint{4.912660in}{0.809897in}}%
\pgfpathlineto{\pgfqpoint{4.893137in}{0.809897in}}%
\pgfpathlineto{\pgfqpoint{4.873613in}{0.809897in}}%
\pgfpathlineto{\pgfqpoint{4.854090in}{0.809897in}}%
\pgfpathlineto{\pgfqpoint{4.834566in}{0.809897in}}%
\pgfpathlineto{\pgfqpoint{4.815042in}{0.828789in}}%
\pgfpathlineto{\pgfqpoint{4.795519in}{0.828789in}}%
\pgfpathlineto{\pgfqpoint{4.775995in}{0.828789in}}%
\pgfpathlineto{\pgfqpoint{4.756472in}{0.828789in}}%
\pgfpathlineto{\pgfqpoint{4.736948in}{0.828789in}}%
\pgfpathlineto{\pgfqpoint{4.717425in}{0.828789in}}%
\pgfpathlineto{\pgfqpoint{4.697901in}{0.839213in}}%
\pgfpathlineto{\pgfqpoint{4.678378in}{0.839213in}}%
\pgfpathlineto{\pgfqpoint{4.658854in}{0.839213in}}%
\pgfpathlineto{\pgfqpoint{4.639331in}{0.894748in}}%
\pgfpathlineto{\pgfqpoint{4.619807in}{0.894748in}}%
\pgfpathlineto{\pgfqpoint{4.600283in}{0.894748in}}%
\pgfpathlineto{\pgfqpoint{4.580760in}{0.894748in}}%
\pgfpathlineto{\pgfqpoint{4.561236in}{0.894748in}}%
\pgfpathlineto{\pgfqpoint{4.541713in}{0.911938in}}%
\pgfpathlineto{\pgfqpoint{4.522189in}{0.911938in}}%
\pgfpathlineto{\pgfqpoint{4.502666in}{0.911938in}}%
\pgfpathlineto{\pgfqpoint{4.483142in}{0.911938in}}%
\pgfpathlineto{\pgfqpoint{4.463619in}{0.911938in}}%
\pgfpathlineto{\pgfqpoint{4.444095in}{0.911938in}}%
\pgfpathlineto{\pgfqpoint{4.424571in}{0.911938in}}%
\pgfpathlineto{\pgfqpoint{4.405048in}{0.911938in}}%
\pgfpathlineto{\pgfqpoint{4.385524in}{0.911938in}}%
\pgfpathlineto{\pgfqpoint{4.366001in}{0.911938in}}%
\pgfpathlineto{\pgfqpoint{4.346477in}{0.911938in}}%
\pgfpathlineto{\pgfqpoint{4.326954in}{0.911938in}}%
\pgfpathlineto{\pgfqpoint{4.307430in}{0.911938in}}%
\pgfpathlineto{\pgfqpoint{4.287907in}{0.911938in}}%
\pgfpathlineto{\pgfqpoint{4.268383in}{0.922046in}}%
\pgfpathlineto{\pgfqpoint{4.248860in}{0.922046in}}%
\pgfpathlineto{\pgfqpoint{4.229336in}{0.922046in}}%
\pgfpathlineto{\pgfqpoint{4.209812in}{0.922046in}}%
\pgfpathlineto{\pgfqpoint{4.190289in}{0.922046in}}%
\pgfpathlineto{\pgfqpoint{4.170765in}{0.922046in}}%
\pgfpathlineto{\pgfqpoint{4.151242in}{0.922046in}}%
\pgfpathlineto{\pgfqpoint{4.131718in}{0.922046in}}%
\pgfpathlineto{\pgfqpoint{4.112195in}{0.930126in}}%
\pgfpathlineto{\pgfqpoint{4.092671in}{0.930126in}}%
\pgfpathlineto{\pgfqpoint{4.073148in}{0.951240in}}%
\pgfpathlineto{\pgfqpoint{4.053624in}{0.951240in}}%
\pgfpathlineto{\pgfqpoint{4.034100in}{0.951240in}}%
\pgfpathlineto{\pgfqpoint{4.014577in}{0.951240in}}%
\pgfpathlineto{\pgfqpoint{3.995053in}{0.951240in}}%
\pgfpathlineto{\pgfqpoint{3.975530in}{1.050638in}}%
\pgfpathlineto{\pgfqpoint{3.956006in}{1.050638in}}%
\pgfpathlineto{\pgfqpoint{3.936483in}{1.068927in}}%
\pgfpathlineto{\pgfqpoint{3.916959in}{1.157349in}}%
\pgfpathlineto{\pgfqpoint{3.897436in}{1.157349in}}%
\pgfpathlineto{\pgfqpoint{3.877912in}{1.157349in}}%
\pgfpathlineto{\pgfqpoint{3.858389in}{1.157349in}}%
\pgfpathlineto{\pgfqpoint{3.838865in}{1.157349in}}%
\pgfpathlineto{\pgfqpoint{3.819341in}{1.186885in}}%
\pgfpathlineto{\pgfqpoint{3.799818in}{1.186885in}}%
\pgfpathlineto{\pgfqpoint{3.780294in}{1.186885in}}%
\pgfpathlineto{\pgfqpoint{3.760771in}{1.186885in}}%
\pgfpathlineto{\pgfqpoint{3.741247in}{1.186885in}}%
\pgfpathlineto{\pgfqpoint{3.721724in}{1.186885in}}%
\pgfpathlineto{\pgfqpoint{3.702200in}{1.186885in}}%
\pgfpathlineto{\pgfqpoint{3.682677in}{1.186885in}}%
\pgfpathlineto{\pgfqpoint{3.663153in}{1.186885in}}%
\pgfpathlineto{\pgfqpoint{3.643630in}{1.186885in}}%
\pgfpathlineto{\pgfqpoint{3.624106in}{1.186885in}}%
\pgfpathlineto{\pgfqpoint{3.604582in}{1.192903in}}%
\pgfpathlineto{\pgfqpoint{3.585059in}{1.192903in}}%
\pgfpathlineto{\pgfqpoint{3.565535in}{1.192903in}}%
\pgfpathlineto{\pgfqpoint{3.546012in}{1.192903in}}%
\pgfpathlineto{\pgfqpoint{3.526488in}{1.192903in}}%
\pgfpathlineto{\pgfqpoint{3.506965in}{1.192903in}}%
\pgfpathlineto{\pgfqpoint{3.487441in}{1.192903in}}%
\pgfpathlineto{\pgfqpoint{3.467918in}{1.192903in}}%
\pgfpathlineto{\pgfqpoint{3.448394in}{1.202060in}}%
\pgfpathlineto{\pgfqpoint{3.428870in}{1.202060in}}%
\pgfpathlineto{\pgfqpoint{3.409347in}{1.202060in}}%
\pgfpathlineto{\pgfqpoint{3.389823in}{1.227478in}}%
\pgfpathlineto{\pgfqpoint{3.370300in}{1.227478in}}%
\pgfpathlineto{\pgfqpoint{3.350776in}{1.230892in}}%
\pgfpathlineto{\pgfqpoint{3.331253in}{1.230892in}}%
\pgfpathlineto{\pgfqpoint{3.311729in}{1.230892in}}%
\pgfpathlineto{\pgfqpoint{3.292206in}{1.259547in}}%
\pgfpathlineto{\pgfqpoint{3.272682in}{1.259547in}}%
\pgfpathlineto{\pgfqpoint{3.253159in}{1.259547in}}%
\pgfpathlineto{\pgfqpoint{3.233635in}{1.307887in}}%
\pgfpathlineto{\pgfqpoint{3.214111in}{1.307887in}}%
\pgfpathlineto{\pgfqpoint{3.194588in}{1.307887in}}%
\pgfpathlineto{\pgfqpoint{3.175064in}{1.307887in}}%
\pgfpathlineto{\pgfqpoint{3.155541in}{1.307887in}}%
\pgfpathlineto{\pgfqpoint{3.136017in}{1.332736in}}%
\pgfpathlineto{\pgfqpoint{3.116494in}{1.349827in}}%
\pgfpathlineto{\pgfqpoint{3.096970in}{1.361641in}}%
\pgfpathlineto{\pgfqpoint{3.077447in}{1.361641in}}%
\pgfpathlineto{\pgfqpoint{3.057923in}{1.369027in}}%
\pgfpathlineto{\pgfqpoint{3.038400in}{1.431935in}}%
\pgfpathlineto{\pgfqpoint{3.018876in}{1.431935in}}%
\pgfpathlineto{\pgfqpoint{2.999352in}{1.435887in}}%
\pgfpathlineto{\pgfqpoint{2.979829in}{1.472463in}}%
\pgfpathlineto{\pgfqpoint{2.960305in}{1.473753in}}%
\pgfpathlineto{\pgfqpoint{2.940782in}{1.473753in}}%
\pgfpathlineto{\pgfqpoint{2.921258in}{1.473753in}}%
\pgfpathlineto{\pgfqpoint{2.901735in}{1.473753in}}%
\pgfpathlineto{\pgfqpoint{2.882211in}{1.493496in}}%
\pgfpathlineto{\pgfqpoint{2.862688in}{1.497505in}}%
\pgfpathlineto{\pgfqpoint{2.843164in}{1.497505in}}%
\pgfpathlineto{\pgfqpoint{2.823640in}{1.502649in}}%
\pgfpathlineto{\pgfqpoint{2.804117in}{1.502649in}}%
\pgfpathlineto{\pgfqpoint{2.784593in}{1.516361in}}%
\pgfpathlineto{\pgfqpoint{2.765070in}{1.516361in}}%
\pgfpathlineto{\pgfqpoint{2.745546in}{1.516361in}}%
\pgfpathlineto{\pgfqpoint{2.726023in}{1.534866in}}%
\pgfpathlineto{\pgfqpoint{2.706499in}{1.534866in}}%
\pgfpathlineto{\pgfqpoint{2.686976in}{1.544013in}}%
\pgfpathlineto{\pgfqpoint{2.667452in}{1.589519in}}%
\pgfpathlineto{\pgfqpoint{2.647929in}{1.589519in}}%
\pgfpathlineto{\pgfqpoint{2.628405in}{1.589519in}}%
\pgfpathlineto{\pgfqpoint{2.608881in}{1.589519in}}%
\pgfpathlineto{\pgfqpoint{2.589358in}{1.589519in}}%
\pgfpathlineto{\pgfqpoint{2.569834in}{1.590074in}}%
\pgfpathlineto{\pgfqpoint{2.550311in}{1.590074in}}%
\pgfpathlineto{\pgfqpoint{2.530787in}{1.604860in}}%
\pgfpathlineto{\pgfqpoint{2.511264in}{1.619899in}}%
\pgfpathlineto{\pgfqpoint{2.491740in}{1.619899in}}%
\pgfpathlineto{\pgfqpoint{2.472217in}{1.625024in}}%
\pgfpathlineto{\pgfqpoint{2.452693in}{1.625024in}}%
\pgfpathlineto{\pgfqpoint{2.433169in}{1.625024in}}%
\pgfpathlineto{\pgfqpoint{2.413646in}{1.625024in}}%
\pgfpathlineto{\pgfqpoint{2.394122in}{1.625024in}}%
\pgfpathlineto{\pgfqpoint{2.374599in}{1.625024in}}%
\pgfpathlineto{\pgfqpoint{2.355075in}{1.638603in}}%
\pgfpathlineto{\pgfqpoint{2.335552in}{1.638603in}}%
\pgfpathlineto{\pgfqpoint{2.316028in}{1.655849in}}%
\pgfpathlineto{\pgfqpoint{2.296505in}{1.670433in}}%
\pgfpathlineto{\pgfqpoint{2.276981in}{1.687601in}}%
\pgfpathlineto{\pgfqpoint{2.257458in}{1.698206in}}%
\pgfpathlineto{\pgfqpoint{2.237934in}{1.698206in}}%
\pgfpathlineto{\pgfqpoint{2.218410in}{1.752071in}}%
\pgfpathlineto{\pgfqpoint{2.198887in}{1.782590in}}%
\pgfpathlineto{\pgfqpoint{2.179363in}{1.828338in}}%
\pgfpathlineto{\pgfqpoint{2.159840in}{1.858149in}}%
\pgfpathlineto{\pgfqpoint{2.140316in}{1.861795in}}%
\pgfpathlineto{\pgfqpoint{2.120793in}{1.861795in}}%
\pgfpathlineto{\pgfqpoint{2.101269in}{1.961759in}}%
\pgfpathlineto{\pgfqpoint{2.081746in}{1.962225in}}%
\pgfpathlineto{\pgfqpoint{2.062222in}{2.055116in}}%
\pgfpathlineto{\pgfqpoint{2.042699in}{2.055116in}}%
\pgfpathlineto{\pgfqpoint{2.023175in}{2.055116in}}%
\pgfpathlineto{\pgfqpoint{2.003651in}{2.055116in}}%
\pgfpathlineto{\pgfqpoint{1.984128in}{2.055116in}}%
\pgfpathlineto{\pgfqpoint{1.964604in}{2.055116in}}%
\pgfpathlineto{\pgfqpoint{1.945081in}{2.055116in}}%
\pgfpathlineto{\pgfqpoint{1.925557in}{2.055116in}}%
\pgfpathlineto{\pgfqpoint{1.906034in}{2.055116in}}%
\pgfpathlineto{\pgfqpoint{1.886510in}{2.055116in}}%
\pgfpathlineto{\pgfqpoint{1.866987in}{2.055116in}}%
\pgfpathlineto{\pgfqpoint{1.847463in}{2.055116in}}%
\pgfpathlineto{\pgfqpoint{1.827939in}{2.055116in}}%
\pgfpathlineto{\pgfqpoint{1.808416in}{2.055116in}}%
\pgfpathlineto{\pgfqpoint{1.788892in}{2.055116in}}%
\pgfpathlineto{\pgfqpoint{1.769369in}{2.076942in}}%
\pgfpathlineto{\pgfqpoint{1.749845in}{2.076942in}}%
\pgfpathlineto{\pgfqpoint{1.730322in}{2.076942in}}%
\pgfpathlineto{\pgfqpoint{1.710798in}{2.076942in}}%
\pgfpathlineto{\pgfqpoint{1.691275in}{2.076942in}}%
\pgfpathlineto{\pgfqpoint{1.671751in}{2.076942in}}%
\pgfpathlineto{\pgfqpoint{1.652228in}{2.076942in}}%
\pgfpathlineto{\pgfqpoint{1.632704in}{2.076942in}}%
\pgfpathlineto{\pgfqpoint{1.613180in}{2.076942in}}%
\pgfpathlineto{\pgfqpoint{1.593657in}{2.090712in}}%
\pgfpathlineto{\pgfqpoint{1.574133in}{2.090712in}}%
\pgfpathlineto{\pgfqpoint{1.554610in}{2.090712in}}%
\pgfpathlineto{\pgfqpoint{1.535086in}{2.090712in}}%
\pgfpathlineto{\pgfqpoint{1.515563in}{2.090712in}}%
\pgfpathlineto{\pgfqpoint{1.496039in}{2.090712in}}%
\pgfpathlineto{\pgfqpoint{1.476516in}{2.090712in}}%
\pgfpathlineto{\pgfqpoint{1.456992in}{2.090712in}}%
\pgfpathlineto{\pgfqpoint{1.437469in}{2.090712in}}%
\pgfpathlineto{\pgfqpoint{1.417945in}{2.090712in}}%
\pgfpathlineto{\pgfqpoint{1.398421in}{2.139912in}}%
\pgfpathlineto{\pgfqpoint{1.378898in}{2.162193in}}%
\pgfpathlineto{\pgfqpoint{1.359374in}{2.162193in}}%
\pgfpathlineto{\pgfqpoint{1.339851in}{2.162193in}}%
\pgfpathlineto{\pgfqpoint{1.320327in}{2.178242in}}%
\pgfpathlineto{\pgfqpoint{1.300804in}{2.178242in}}%
\pgfpathlineto{\pgfqpoint{1.281280in}{2.178242in}}%
\pgfpathlineto{\pgfqpoint{1.261757in}{2.178242in}}%
\pgfpathlineto{\pgfqpoint{1.242233in}{2.178242in}}%
\pgfpathlineto{\pgfqpoint{1.222709in}{2.232839in}}%
\pgfpathlineto{\pgfqpoint{1.203186in}{2.251063in}}%
\pgfpathlineto{\pgfqpoint{1.183662in}{2.309079in}}%
\pgfpathlineto{\pgfqpoint{1.164139in}{2.380608in}}%
\pgfpathlineto{\pgfqpoint{1.144615in}{2.380608in}}%
\pgfpathlineto{\pgfqpoint{1.125092in}{2.442850in}}%
\pgfpathlineto{\pgfqpoint{1.105568in}{2.468219in}}%
\pgfpathclose%
\pgfusepath{fill}%
\end{pgfscope}%
\begin{pgfscope}%
\pgfpathrectangle{\pgfqpoint{0.862500in}{0.375000in}}{\pgfqpoint{5.347500in}{2.265000in}}%
\pgfusepath{clip}%
\pgfsetbuttcap%
\pgfsetroundjoin%
\definecolor{currentfill}{rgb}{0.839216,0.152941,0.156863}%
\pgfsetfillcolor{currentfill}%
\pgfsetfillopacity{0.200000}%
\pgfsetlinewidth{0.000000pt}%
\definecolor{currentstroke}{rgb}{0.000000,0.000000,0.000000}%
\pgfsetstrokecolor{currentstroke}%
\pgfsetdash{}{0pt}%
\pgfpathmoveto{\pgfqpoint{1.105568in}{2.417025in}}%
\pgfpathlineto{\pgfqpoint{1.105568in}{2.511852in}}%
\pgfpathlineto{\pgfqpoint{1.125092in}{2.435088in}}%
\pgfpathlineto{\pgfqpoint{1.144615in}{2.435088in}}%
\pgfpathlineto{\pgfqpoint{1.164139in}{2.435088in}}%
\pgfpathlineto{\pgfqpoint{1.183662in}{2.431441in}}%
\pgfpathlineto{\pgfqpoint{1.203186in}{2.427861in}}%
\pgfpathlineto{\pgfqpoint{1.222709in}{2.427861in}}%
\pgfpathlineto{\pgfqpoint{1.242233in}{2.348668in}}%
\pgfpathlineto{\pgfqpoint{1.261757in}{2.348668in}}%
\pgfpathlineto{\pgfqpoint{1.281280in}{2.335358in}}%
\pgfpathlineto{\pgfqpoint{1.300804in}{2.335358in}}%
\pgfpathlineto{\pgfqpoint{1.320327in}{2.335358in}}%
\pgfpathlineto{\pgfqpoint{1.339851in}{2.335358in}}%
\pgfpathlineto{\pgfqpoint{1.359374in}{2.317216in}}%
\pgfpathlineto{\pgfqpoint{1.378898in}{2.279899in}}%
\pgfpathlineto{\pgfqpoint{1.398421in}{2.279899in}}%
\pgfpathlineto{\pgfqpoint{1.417945in}{2.266533in}}%
\pgfpathlineto{\pgfqpoint{1.437469in}{2.266533in}}%
\pgfpathlineto{\pgfqpoint{1.456992in}{2.266533in}}%
\pgfpathlineto{\pgfqpoint{1.476516in}{2.251895in}}%
\pgfpathlineto{\pgfqpoint{1.496039in}{2.251895in}}%
\pgfpathlineto{\pgfqpoint{1.515563in}{2.251895in}}%
\pgfpathlineto{\pgfqpoint{1.535086in}{2.251895in}}%
\pgfpathlineto{\pgfqpoint{1.554610in}{2.248937in}}%
\pgfpathlineto{\pgfqpoint{1.574133in}{2.248937in}}%
\pgfpathlineto{\pgfqpoint{1.593657in}{2.248937in}}%
\pgfpathlineto{\pgfqpoint{1.613180in}{2.248937in}}%
\pgfpathlineto{\pgfqpoint{1.632704in}{2.248937in}}%
\pgfpathlineto{\pgfqpoint{1.652228in}{2.248937in}}%
\pgfpathlineto{\pgfqpoint{1.671751in}{2.248937in}}%
\pgfpathlineto{\pgfqpoint{1.691275in}{2.248937in}}%
\pgfpathlineto{\pgfqpoint{1.710798in}{2.248937in}}%
\pgfpathlineto{\pgfqpoint{1.730322in}{2.248937in}}%
\pgfpathlineto{\pgfqpoint{1.749845in}{2.248937in}}%
\pgfpathlineto{\pgfqpoint{1.769369in}{2.248937in}}%
\pgfpathlineto{\pgfqpoint{1.788892in}{2.248937in}}%
\pgfpathlineto{\pgfqpoint{1.808416in}{2.248937in}}%
\pgfpathlineto{\pgfqpoint{1.827939in}{2.248937in}}%
\pgfpathlineto{\pgfqpoint{1.847463in}{2.248937in}}%
\pgfpathlineto{\pgfqpoint{1.866987in}{2.248937in}}%
\pgfpathlineto{\pgfqpoint{1.886510in}{2.248937in}}%
\pgfpathlineto{\pgfqpoint{1.906034in}{2.248937in}}%
\pgfpathlineto{\pgfqpoint{1.925557in}{2.248937in}}%
\pgfpathlineto{\pgfqpoint{1.945081in}{2.248937in}}%
\pgfpathlineto{\pgfqpoint{1.964604in}{2.230540in}}%
\pgfpathlineto{\pgfqpoint{1.984128in}{2.230540in}}%
\pgfpathlineto{\pgfqpoint{2.003651in}{2.230540in}}%
\pgfpathlineto{\pgfqpoint{2.023175in}{2.220542in}}%
\pgfpathlineto{\pgfqpoint{2.042699in}{2.216173in}}%
\pgfpathlineto{\pgfqpoint{2.062222in}{2.216173in}}%
\pgfpathlineto{\pgfqpoint{2.081746in}{2.179311in}}%
\pgfpathlineto{\pgfqpoint{2.101269in}{2.118334in}}%
\pgfpathlineto{\pgfqpoint{2.120793in}{1.996860in}}%
\pgfpathlineto{\pgfqpoint{2.140316in}{1.960831in}}%
\pgfpathlineto{\pgfqpoint{2.159840in}{1.869870in}}%
\pgfpathlineto{\pgfqpoint{2.179363in}{1.860375in}}%
\pgfpathlineto{\pgfqpoint{2.198887in}{1.860375in}}%
\pgfpathlineto{\pgfqpoint{2.218410in}{1.846638in}}%
\pgfpathlineto{\pgfqpoint{2.237934in}{1.846638in}}%
\pgfpathlineto{\pgfqpoint{2.257458in}{1.846638in}}%
\pgfpathlineto{\pgfqpoint{2.276981in}{1.846638in}}%
\pgfpathlineto{\pgfqpoint{2.296505in}{1.834604in}}%
\pgfpathlineto{\pgfqpoint{2.316028in}{1.797863in}}%
\pgfpathlineto{\pgfqpoint{2.335552in}{1.797863in}}%
\pgfpathlineto{\pgfqpoint{2.355075in}{1.791168in}}%
\pgfpathlineto{\pgfqpoint{2.374599in}{1.788767in}}%
\pgfpathlineto{\pgfqpoint{2.394122in}{1.781932in}}%
\pgfpathlineto{\pgfqpoint{2.413646in}{1.780610in}}%
\pgfpathlineto{\pgfqpoint{2.433169in}{1.758382in}}%
\pgfpathlineto{\pgfqpoint{2.452693in}{1.758382in}}%
\pgfpathlineto{\pgfqpoint{2.472217in}{1.757637in}}%
\pgfpathlineto{\pgfqpoint{2.491740in}{1.757016in}}%
\pgfpathlineto{\pgfqpoint{2.511264in}{1.748704in}}%
\pgfpathlineto{\pgfqpoint{2.530787in}{1.746137in}}%
\pgfpathlineto{\pgfqpoint{2.550311in}{1.744547in}}%
\pgfpathlineto{\pgfqpoint{2.569834in}{1.742695in}}%
\pgfpathlineto{\pgfqpoint{2.589358in}{1.725925in}}%
\pgfpathlineto{\pgfqpoint{2.608881in}{1.725925in}}%
\pgfpathlineto{\pgfqpoint{2.628405in}{1.725925in}}%
\pgfpathlineto{\pgfqpoint{2.647929in}{1.725925in}}%
\pgfpathlineto{\pgfqpoint{2.667452in}{1.721538in}}%
\pgfpathlineto{\pgfqpoint{2.686976in}{1.717882in}}%
\pgfpathlineto{\pgfqpoint{2.706499in}{1.707181in}}%
\pgfpathlineto{\pgfqpoint{2.726023in}{1.707181in}}%
\pgfpathlineto{\pgfqpoint{2.745546in}{1.707181in}}%
\pgfpathlineto{\pgfqpoint{2.765070in}{1.707181in}}%
\pgfpathlineto{\pgfqpoint{2.784593in}{1.695042in}}%
\pgfpathlineto{\pgfqpoint{2.804117in}{1.695042in}}%
\pgfpathlineto{\pgfqpoint{2.823640in}{1.693829in}}%
\pgfpathlineto{\pgfqpoint{2.843164in}{1.693829in}}%
\pgfpathlineto{\pgfqpoint{2.862688in}{1.680685in}}%
\pgfpathlineto{\pgfqpoint{2.882211in}{1.647607in}}%
\pgfpathlineto{\pgfqpoint{2.901735in}{1.629307in}}%
\pgfpathlineto{\pgfqpoint{2.921258in}{1.629307in}}%
\pgfpathlineto{\pgfqpoint{2.940782in}{1.600715in}}%
\pgfpathlineto{\pgfqpoint{2.960305in}{1.600580in}}%
\pgfpathlineto{\pgfqpoint{2.979829in}{1.594451in}}%
\pgfpathlineto{\pgfqpoint{2.999352in}{1.581565in}}%
\pgfpathlineto{\pgfqpoint{3.018876in}{1.574866in}}%
\pgfpathlineto{\pgfqpoint{3.038400in}{1.574866in}}%
\pgfpathlineto{\pgfqpoint{3.057923in}{1.572874in}}%
\pgfpathlineto{\pgfqpoint{3.077447in}{1.563122in}}%
\pgfpathlineto{\pgfqpoint{3.096970in}{1.544541in}}%
\pgfpathlineto{\pgfqpoint{3.116494in}{1.544541in}}%
\pgfpathlineto{\pgfqpoint{3.136017in}{1.543021in}}%
\pgfpathlineto{\pgfqpoint{3.155541in}{1.518447in}}%
\pgfpathlineto{\pgfqpoint{3.175064in}{1.515140in}}%
\pgfpathlineto{\pgfqpoint{3.194588in}{1.515140in}}%
\pgfpathlineto{\pgfqpoint{3.214111in}{1.515140in}}%
\pgfpathlineto{\pgfqpoint{3.233635in}{1.515140in}}%
\pgfpathlineto{\pgfqpoint{3.253159in}{1.507300in}}%
\pgfpathlineto{\pgfqpoint{3.272682in}{1.507300in}}%
\pgfpathlineto{\pgfqpoint{3.292206in}{1.459407in}}%
\pgfpathlineto{\pgfqpoint{3.311729in}{1.457075in}}%
\pgfpathlineto{\pgfqpoint{3.331253in}{1.456849in}}%
\pgfpathlineto{\pgfqpoint{3.350776in}{1.456849in}}%
\pgfpathlineto{\pgfqpoint{3.370300in}{1.456849in}}%
\pgfpathlineto{\pgfqpoint{3.389823in}{1.456849in}}%
\pgfpathlineto{\pgfqpoint{3.409347in}{1.456849in}}%
\pgfpathlineto{\pgfqpoint{3.428870in}{1.456849in}}%
\pgfpathlineto{\pgfqpoint{3.448394in}{1.456849in}}%
\pgfpathlineto{\pgfqpoint{3.467918in}{1.444085in}}%
\pgfpathlineto{\pgfqpoint{3.487441in}{1.444085in}}%
\pgfpathlineto{\pgfqpoint{3.506965in}{1.444085in}}%
\pgfpathlineto{\pgfqpoint{3.526488in}{1.444085in}}%
\pgfpathlineto{\pgfqpoint{3.546012in}{1.444085in}}%
\pgfpathlineto{\pgfqpoint{3.565535in}{1.444085in}}%
\pgfpathlineto{\pgfqpoint{3.585059in}{1.444085in}}%
\pgfpathlineto{\pgfqpoint{3.604582in}{1.444085in}}%
\pgfpathlineto{\pgfqpoint{3.624106in}{1.443954in}}%
\pgfpathlineto{\pgfqpoint{3.643630in}{1.437872in}}%
\pgfpathlineto{\pgfqpoint{3.663153in}{1.437009in}}%
\pgfpathlineto{\pgfqpoint{3.682677in}{1.437009in}}%
\pgfpathlineto{\pgfqpoint{3.702200in}{1.422215in}}%
\pgfpathlineto{\pgfqpoint{3.721724in}{1.367549in}}%
\pgfpathlineto{\pgfqpoint{3.741247in}{1.367549in}}%
\pgfpathlineto{\pgfqpoint{3.760771in}{1.367549in}}%
\pgfpathlineto{\pgfqpoint{3.780294in}{1.367549in}}%
\pgfpathlineto{\pgfqpoint{3.799818in}{1.367549in}}%
\pgfpathlineto{\pgfqpoint{3.819341in}{1.338884in}}%
\pgfpathlineto{\pgfqpoint{3.838865in}{0.794388in}}%
\pgfpathlineto{\pgfqpoint{3.858389in}{0.794388in}}%
\pgfpathlineto{\pgfqpoint{3.877912in}{0.794388in}}%
\pgfpathlineto{\pgfqpoint{3.897436in}{0.794388in}}%
\pgfpathlineto{\pgfqpoint{3.916959in}{0.794388in}}%
\pgfpathlineto{\pgfqpoint{3.936483in}{0.794388in}}%
\pgfpathlineto{\pgfqpoint{3.956006in}{0.794388in}}%
\pgfpathlineto{\pgfqpoint{3.975530in}{0.776715in}}%
\pgfpathlineto{\pgfqpoint{3.995053in}{0.763391in}}%
\pgfpathlineto{\pgfqpoint{4.014577in}{0.763391in}}%
\pgfpathlineto{\pgfqpoint{4.034100in}{0.763391in}}%
\pgfpathlineto{\pgfqpoint{4.053624in}{0.763391in}}%
\pgfpathlineto{\pgfqpoint{4.073148in}{0.657523in}}%
\pgfpathlineto{\pgfqpoint{4.092671in}{0.657523in}}%
\pgfpathlineto{\pgfqpoint{4.112195in}{0.657523in}}%
\pgfpathlineto{\pgfqpoint{4.131718in}{0.657523in}}%
\pgfpathlineto{\pgfqpoint{4.151242in}{0.657523in}}%
\pgfpathlineto{\pgfqpoint{4.170765in}{0.657523in}}%
\pgfpathlineto{\pgfqpoint{4.190289in}{0.657523in}}%
\pgfpathlineto{\pgfqpoint{4.209812in}{0.657523in}}%
\pgfpathlineto{\pgfqpoint{4.229336in}{0.657523in}}%
\pgfpathlineto{\pgfqpoint{4.248860in}{0.657523in}}%
\pgfpathlineto{\pgfqpoint{4.268383in}{0.657523in}}%
\pgfpathlineto{\pgfqpoint{4.287907in}{0.657523in}}%
\pgfpathlineto{\pgfqpoint{4.307430in}{0.657523in}}%
\pgfpathlineto{\pgfqpoint{4.326954in}{0.657523in}}%
\pgfpathlineto{\pgfqpoint{4.346477in}{0.657523in}}%
\pgfpathlineto{\pgfqpoint{4.366001in}{0.657523in}}%
\pgfpathlineto{\pgfqpoint{4.385524in}{0.657523in}}%
\pgfpathlineto{\pgfqpoint{4.405048in}{0.657523in}}%
\pgfpathlineto{\pgfqpoint{4.424571in}{0.657523in}}%
\pgfpathlineto{\pgfqpoint{4.444095in}{0.657523in}}%
\pgfpathlineto{\pgfqpoint{4.463619in}{0.657523in}}%
\pgfpathlineto{\pgfqpoint{4.483142in}{0.657523in}}%
\pgfpathlineto{\pgfqpoint{4.502666in}{0.657523in}}%
\pgfpathlineto{\pgfqpoint{4.522189in}{0.657523in}}%
\pgfpathlineto{\pgfqpoint{4.541713in}{0.657523in}}%
\pgfpathlineto{\pgfqpoint{4.561236in}{0.657523in}}%
\pgfpathlineto{\pgfqpoint{4.580760in}{0.657523in}}%
\pgfpathlineto{\pgfqpoint{4.600283in}{0.639287in}}%
\pgfpathlineto{\pgfqpoint{4.619807in}{0.639287in}}%
\pgfpathlineto{\pgfqpoint{4.639331in}{0.639287in}}%
\pgfpathlineto{\pgfqpoint{4.658854in}{0.639287in}}%
\pgfpathlineto{\pgfqpoint{4.678378in}{0.639287in}}%
\pgfpathlineto{\pgfqpoint{4.697901in}{0.639287in}}%
\pgfpathlineto{\pgfqpoint{4.717425in}{0.639287in}}%
\pgfpathlineto{\pgfqpoint{4.736948in}{0.639287in}}%
\pgfpathlineto{\pgfqpoint{4.756472in}{0.639287in}}%
\pgfpathlineto{\pgfqpoint{4.775995in}{0.639287in}}%
\pgfpathlineto{\pgfqpoint{4.795519in}{0.639287in}}%
\pgfpathlineto{\pgfqpoint{4.815042in}{0.639287in}}%
\pgfpathlineto{\pgfqpoint{4.834566in}{0.639287in}}%
\pgfpathlineto{\pgfqpoint{4.854090in}{0.639287in}}%
\pgfpathlineto{\pgfqpoint{4.873613in}{0.639287in}}%
\pgfpathlineto{\pgfqpoint{4.893137in}{0.639287in}}%
\pgfpathlineto{\pgfqpoint{4.912660in}{0.639287in}}%
\pgfpathlineto{\pgfqpoint{4.932184in}{0.614339in}}%
\pgfpathlineto{\pgfqpoint{4.951707in}{0.614339in}}%
\pgfpathlineto{\pgfqpoint{4.971231in}{0.614339in}}%
\pgfpathlineto{\pgfqpoint{4.990754in}{0.614339in}}%
\pgfpathlineto{\pgfqpoint{5.010278in}{0.614339in}}%
\pgfpathlineto{\pgfqpoint{5.029801in}{0.614339in}}%
\pgfpathlineto{\pgfqpoint{5.049325in}{0.614339in}}%
\pgfpathlineto{\pgfqpoint{5.068849in}{0.614339in}}%
\pgfpathlineto{\pgfqpoint{5.088372in}{0.614339in}}%
\pgfpathlineto{\pgfqpoint{5.107896in}{0.614339in}}%
\pgfpathlineto{\pgfqpoint{5.127419in}{0.614339in}}%
\pgfpathlineto{\pgfqpoint{5.146943in}{0.614339in}}%
\pgfpathlineto{\pgfqpoint{5.166466in}{0.614339in}}%
\pgfpathlineto{\pgfqpoint{5.185990in}{0.614339in}}%
\pgfpathlineto{\pgfqpoint{5.205513in}{0.614339in}}%
\pgfpathlineto{\pgfqpoint{5.225037in}{0.614339in}}%
\pgfpathlineto{\pgfqpoint{5.244561in}{0.614339in}}%
\pgfpathlineto{\pgfqpoint{5.264084in}{0.614339in}}%
\pgfpathlineto{\pgfqpoint{5.283608in}{0.614339in}}%
\pgfpathlineto{\pgfqpoint{5.303131in}{0.614339in}}%
\pgfpathlineto{\pgfqpoint{5.322655in}{0.614339in}}%
\pgfpathlineto{\pgfqpoint{5.342178in}{0.614339in}}%
\pgfpathlineto{\pgfqpoint{5.361702in}{0.614339in}}%
\pgfpathlineto{\pgfqpoint{5.381225in}{0.614339in}}%
\pgfpathlineto{\pgfqpoint{5.400749in}{0.614339in}}%
\pgfpathlineto{\pgfqpoint{5.420272in}{0.614339in}}%
\pgfpathlineto{\pgfqpoint{5.439796in}{0.614339in}}%
\pgfpathlineto{\pgfqpoint{5.459320in}{0.614339in}}%
\pgfpathlineto{\pgfqpoint{5.478843in}{0.614339in}}%
\pgfpathlineto{\pgfqpoint{5.498367in}{0.614339in}}%
\pgfpathlineto{\pgfqpoint{5.517890in}{0.614339in}}%
\pgfpathlineto{\pgfqpoint{5.537414in}{0.614339in}}%
\pgfpathlineto{\pgfqpoint{5.556937in}{0.614339in}}%
\pgfpathlineto{\pgfqpoint{5.576461in}{0.614339in}}%
\pgfpathlineto{\pgfqpoint{5.595984in}{0.614339in}}%
\pgfpathlineto{\pgfqpoint{5.615508in}{0.614339in}}%
\pgfpathlineto{\pgfqpoint{5.635031in}{0.614339in}}%
\pgfpathlineto{\pgfqpoint{5.654555in}{0.614339in}}%
\pgfpathlineto{\pgfqpoint{5.674079in}{0.614339in}}%
\pgfpathlineto{\pgfqpoint{5.693602in}{0.614339in}}%
\pgfpathlineto{\pgfqpoint{5.713126in}{0.614339in}}%
\pgfpathlineto{\pgfqpoint{5.732649in}{0.614339in}}%
\pgfpathlineto{\pgfqpoint{5.752173in}{0.614339in}}%
\pgfpathlineto{\pgfqpoint{5.771696in}{0.614339in}}%
\pgfpathlineto{\pgfqpoint{5.791220in}{0.644698in}}%
\pgfpathlineto{\pgfqpoint{5.810743in}{0.644698in}}%
\pgfpathlineto{\pgfqpoint{5.830267in}{0.644698in}}%
\pgfpathlineto{\pgfqpoint{5.849791in}{0.644698in}}%
\pgfpathlineto{\pgfqpoint{5.869314in}{0.644698in}}%
\pgfpathlineto{\pgfqpoint{5.888838in}{0.644698in}}%
\pgfpathlineto{\pgfqpoint{5.908361in}{0.644698in}}%
\pgfpathlineto{\pgfqpoint{5.927885in}{0.644698in}}%
\pgfpathlineto{\pgfqpoint{5.947408in}{0.644698in}}%
\pgfpathlineto{\pgfqpoint{5.966932in}{0.644698in}}%
\pgfpathlineto{\pgfqpoint{5.966932in}{0.551496in}}%
\pgfpathlineto{\pgfqpoint{5.966932in}{0.551496in}}%
\pgfpathlineto{\pgfqpoint{5.947408in}{0.551496in}}%
\pgfpathlineto{\pgfqpoint{5.927885in}{0.551496in}}%
\pgfpathlineto{\pgfqpoint{5.908361in}{0.551496in}}%
\pgfpathlineto{\pgfqpoint{5.888838in}{0.551496in}}%
\pgfpathlineto{\pgfqpoint{5.869314in}{0.551496in}}%
\pgfpathlineto{\pgfqpoint{5.849791in}{0.551496in}}%
\pgfpathlineto{\pgfqpoint{5.830267in}{0.551496in}}%
\pgfpathlineto{\pgfqpoint{5.810743in}{0.551496in}}%
\pgfpathlineto{\pgfqpoint{5.791220in}{0.551496in}}%
\pgfpathlineto{\pgfqpoint{5.771696in}{0.477955in}}%
\pgfpathlineto{\pgfqpoint{5.752173in}{0.477955in}}%
\pgfpathlineto{\pgfqpoint{5.732649in}{0.477955in}}%
\pgfpathlineto{\pgfqpoint{5.713126in}{0.477955in}}%
\pgfpathlineto{\pgfqpoint{5.693602in}{0.477955in}}%
\pgfpathlineto{\pgfqpoint{5.674079in}{0.477955in}}%
\pgfpathlineto{\pgfqpoint{5.654555in}{0.477955in}}%
\pgfpathlineto{\pgfqpoint{5.635031in}{0.477955in}}%
\pgfpathlineto{\pgfqpoint{5.615508in}{0.477955in}}%
\pgfpathlineto{\pgfqpoint{5.595984in}{0.477955in}}%
\pgfpathlineto{\pgfqpoint{5.576461in}{0.477955in}}%
\pgfpathlineto{\pgfqpoint{5.556937in}{0.477955in}}%
\pgfpathlineto{\pgfqpoint{5.537414in}{0.477955in}}%
\pgfpathlineto{\pgfqpoint{5.517890in}{0.477955in}}%
\pgfpathlineto{\pgfqpoint{5.498367in}{0.477955in}}%
\pgfpathlineto{\pgfqpoint{5.478843in}{0.477955in}}%
\pgfpathlineto{\pgfqpoint{5.459320in}{0.477955in}}%
\pgfpathlineto{\pgfqpoint{5.439796in}{0.477955in}}%
\pgfpathlineto{\pgfqpoint{5.420272in}{0.477955in}}%
\pgfpathlineto{\pgfqpoint{5.400749in}{0.477955in}}%
\pgfpathlineto{\pgfqpoint{5.381225in}{0.477955in}}%
\pgfpathlineto{\pgfqpoint{5.361702in}{0.477955in}}%
\pgfpathlineto{\pgfqpoint{5.342178in}{0.477955in}}%
\pgfpathlineto{\pgfqpoint{5.322655in}{0.477955in}}%
\pgfpathlineto{\pgfqpoint{5.303131in}{0.477955in}}%
\pgfpathlineto{\pgfqpoint{5.283608in}{0.477955in}}%
\pgfpathlineto{\pgfqpoint{5.264084in}{0.477955in}}%
\pgfpathlineto{\pgfqpoint{5.244561in}{0.477955in}}%
\pgfpathlineto{\pgfqpoint{5.225037in}{0.477955in}}%
\pgfpathlineto{\pgfqpoint{5.205513in}{0.477955in}}%
\pgfpathlineto{\pgfqpoint{5.185990in}{0.477955in}}%
\pgfpathlineto{\pgfqpoint{5.166466in}{0.477955in}}%
\pgfpathlineto{\pgfqpoint{5.146943in}{0.477955in}}%
\pgfpathlineto{\pgfqpoint{5.127419in}{0.477955in}}%
\pgfpathlineto{\pgfqpoint{5.107896in}{0.477955in}}%
\pgfpathlineto{\pgfqpoint{5.088372in}{0.477955in}}%
\pgfpathlineto{\pgfqpoint{5.068849in}{0.477955in}}%
\pgfpathlineto{\pgfqpoint{5.049325in}{0.477955in}}%
\pgfpathlineto{\pgfqpoint{5.029801in}{0.477955in}}%
\pgfpathlineto{\pgfqpoint{5.010278in}{0.477955in}}%
\pgfpathlineto{\pgfqpoint{4.990754in}{0.477955in}}%
\pgfpathlineto{\pgfqpoint{4.971231in}{0.477955in}}%
\pgfpathlineto{\pgfqpoint{4.951707in}{0.477955in}}%
\pgfpathlineto{\pgfqpoint{4.932184in}{0.477955in}}%
\pgfpathlineto{\pgfqpoint{4.912660in}{0.530992in}}%
\pgfpathlineto{\pgfqpoint{4.893137in}{0.530992in}}%
\pgfpathlineto{\pgfqpoint{4.873613in}{0.530992in}}%
\pgfpathlineto{\pgfqpoint{4.854090in}{0.530992in}}%
\pgfpathlineto{\pgfqpoint{4.834566in}{0.530992in}}%
\pgfpathlineto{\pgfqpoint{4.815042in}{0.530992in}}%
\pgfpathlineto{\pgfqpoint{4.795519in}{0.530992in}}%
\pgfpathlineto{\pgfqpoint{4.775995in}{0.530992in}}%
\pgfpathlineto{\pgfqpoint{4.756472in}{0.530992in}}%
\pgfpathlineto{\pgfqpoint{4.736948in}{0.530992in}}%
\pgfpathlineto{\pgfqpoint{4.717425in}{0.530992in}}%
\pgfpathlineto{\pgfqpoint{4.697901in}{0.530992in}}%
\pgfpathlineto{\pgfqpoint{4.678378in}{0.530992in}}%
\pgfpathlineto{\pgfqpoint{4.658854in}{0.530992in}}%
\pgfpathlineto{\pgfqpoint{4.639331in}{0.530992in}}%
\pgfpathlineto{\pgfqpoint{4.619807in}{0.530992in}}%
\pgfpathlineto{\pgfqpoint{4.600283in}{0.530992in}}%
\pgfpathlineto{\pgfqpoint{4.580760in}{0.550121in}}%
\pgfpathlineto{\pgfqpoint{4.561236in}{0.550121in}}%
\pgfpathlineto{\pgfqpoint{4.541713in}{0.550121in}}%
\pgfpathlineto{\pgfqpoint{4.522189in}{0.550121in}}%
\pgfpathlineto{\pgfqpoint{4.502666in}{0.550121in}}%
\pgfpathlineto{\pgfqpoint{4.483142in}{0.550121in}}%
\pgfpathlineto{\pgfqpoint{4.463619in}{0.550121in}}%
\pgfpathlineto{\pgfqpoint{4.444095in}{0.550121in}}%
\pgfpathlineto{\pgfqpoint{4.424571in}{0.550121in}}%
\pgfpathlineto{\pgfqpoint{4.405048in}{0.550121in}}%
\pgfpathlineto{\pgfqpoint{4.385524in}{0.550121in}}%
\pgfpathlineto{\pgfqpoint{4.366001in}{0.550121in}}%
\pgfpathlineto{\pgfqpoint{4.346477in}{0.550121in}}%
\pgfpathlineto{\pgfqpoint{4.326954in}{0.550121in}}%
\pgfpathlineto{\pgfqpoint{4.307430in}{0.550121in}}%
\pgfpathlineto{\pgfqpoint{4.287907in}{0.550121in}}%
\pgfpathlineto{\pgfqpoint{4.268383in}{0.550121in}}%
\pgfpathlineto{\pgfqpoint{4.248860in}{0.550121in}}%
\pgfpathlineto{\pgfqpoint{4.229336in}{0.550121in}}%
\pgfpathlineto{\pgfqpoint{4.209812in}{0.550121in}}%
\pgfpathlineto{\pgfqpoint{4.190289in}{0.550121in}}%
\pgfpathlineto{\pgfqpoint{4.170765in}{0.550121in}}%
\pgfpathlineto{\pgfqpoint{4.151242in}{0.550121in}}%
\pgfpathlineto{\pgfqpoint{4.131718in}{0.550121in}}%
\pgfpathlineto{\pgfqpoint{4.112195in}{0.550121in}}%
\pgfpathlineto{\pgfqpoint{4.092671in}{0.550121in}}%
\pgfpathlineto{\pgfqpoint{4.073148in}{0.550121in}}%
\pgfpathlineto{\pgfqpoint{4.053624in}{0.592498in}}%
\pgfpathlineto{\pgfqpoint{4.034100in}{0.592498in}}%
\pgfpathlineto{\pgfqpoint{4.014577in}{0.592498in}}%
\pgfpathlineto{\pgfqpoint{3.995053in}{0.592498in}}%
\pgfpathlineto{\pgfqpoint{3.975530in}{0.620284in}}%
\pgfpathlineto{\pgfqpoint{3.956006in}{0.645318in}}%
\pgfpathlineto{\pgfqpoint{3.936483in}{0.645318in}}%
\pgfpathlineto{\pgfqpoint{3.916959in}{0.645318in}}%
\pgfpathlineto{\pgfqpoint{3.897436in}{0.645318in}}%
\pgfpathlineto{\pgfqpoint{3.877912in}{0.645318in}}%
\pgfpathlineto{\pgfqpoint{3.858389in}{0.645318in}}%
\pgfpathlineto{\pgfqpoint{3.838865in}{0.645318in}}%
\pgfpathlineto{\pgfqpoint{3.819341in}{0.866593in}}%
\pgfpathlineto{\pgfqpoint{3.799818in}{0.926511in}}%
\pgfpathlineto{\pgfqpoint{3.780294in}{0.926511in}}%
\pgfpathlineto{\pgfqpoint{3.760771in}{0.926511in}}%
\pgfpathlineto{\pgfqpoint{3.741247in}{0.926511in}}%
\pgfpathlineto{\pgfqpoint{3.721724in}{0.926511in}}%
\pgfpathlineto{\pgfqpoint{3.702200in}{0.953163in}}%
\pgfpathlineto{\pgfqpoint{3.682677in}{0.960977in}}%
\pgfpathlineto{\pgfqpoint{3.663153in}{0.960977in}}%
\pgfpathlineto{\pgfqpoint{3.643630in}{0.975787in}}%
\pgfpathlineto{\pgfqpoint{3.624106in}{1.055213in}}%
\pgfpathlineto{\pgfqpoint{3.604582in}{1.056790in}}%
\pgfpathlineto{\pgfqpoint{3.585059in}{1.056790in}}%
\pgfpathlineto{\pgfqpoint{3.565535in}{1.056790in}}%
\pgfpathlineto{\pgfqpoint{3.546012in}{1.056790in}}%
\pgfpathlineto{\pgfqpoint{3.526488in}{1.056790in}}%
\pgfpathlineto{\pgfqpoint{3.506965in}{1.056790in}}%
\pgfpathlineto{\pgfqpoint{3.487441in}{1.056790in}}%
\pgfpathlineto{\pgfqpoint{3.467918in}{1.056790in}}%
\pgfpathlineto{\pgfqpoint{3.448394in}{1.146164in}}%
\pgfpathlineto{\pgfqpoint{3.428870in}{1.146164in}}%
\pgfpathlineto{\pgfqpoint{3.409347in}{1.146164in}}%
\pgfpathlineto{\pgfqpoint{3.389823in}{1.146164in}}%
\pgfpathlineto{\pgfqpoint{3.370300in}{1.146164in}}%
\pgfpathlineto{\pgfqpoint{3.350776in}{1.146164in}}%
\pgfpathlineto{\pgfqpoint{3.331253in}{1.146164in}}%
\pgfpathlineto{\pgfqpoint{3.311729in}{1.148005in}}%
\pgfpathlineto{\pgfqpoint{3.292206in}{1.148675in}}%
\pgfpathlineto{\pgfqpoint{3.272682in}{1.162987in}}%
\pgfpathlineto{\pgfqpoint{3.253159in}{1.162987in}}%
\pgfpathlineto{\pgfqpoint{3.233635in}{1.205909in}}%
\pgfpathlineto{\pgfqpoint{3.214111in}{1.205909in}}%
\pgfpathlineto{\pgfqpoint{3.194588in}{1.205909in}}%
\pgfpathlineto{\pgfqpoint{3.175064in}{1.205909in}}%
\pgfpathlineto{\pgfqpoint{3.155541in}{1.220134in}}%
\pgfpathlineto{\pgfqpoint{3.136017in}{1.289023in}}%
\pgfpathlineto{\pgfqpoint{3.116494in}{1.299545in}}%
\pgfpathlineto{\pgfqpoint{3.096970in}{1.299545in}}%
\pgfpathlineto{\pgfqpoint{3.077447in}{1.307170in}}%
\pgfpathlineto{\pgfqpoint{3.057923in}{1.326922in}}%
\pgfpathlineto{\pgfqpoint{3.038400in}{1.330436in}}%
\pgfpathlineto{\pgfqpoint{3.018876in}{1.330436in}}%
\pgfpathlineto{\pgfqpoint{2.999352in}{1.341223in}}%
\pgfpathlineto{\pgfqpoint{2.979829in}{1.358440in}}%
\pgfpathlineto{\pgfqpoint{2.960305in}{1.365379in}}%
\pgfpathlineto{\pgfqpoint{2.940782in}{1.366420in}}%
\pgfpathlineto{\pgfqpoint{2.921258in}{1.387576in}}%
\pgfpathlineto{\pgfqpoint{2.901735in}{1.387576in}}%
\pgfpathlineto{\pgfqpoint{2.882211in}{1.410921in}}%
\pgfpathlineto{\pgfqpoint{2.862688in}{1.431799in}}%
\pgfpathlineto{\pgfqpoint{2.843164in}{1.441208in}}%
\pgfpathlineto{\pgfqpoint{2.823640in}{1.441208in}}%
\pgfpathlineto{\pgfqpoint{2.804117in}{1.450366in}}%
\pgfpathlineto{\pgfqpoint{2.784593in}{1.450366in}}%
\pgfpathlineto{\pgfqpoint{2.765070in}{1.468559in}}%
\pgfpathlineto{\pgfqpoint{2.745546in}{1.468559in}}%
\pgfpathlineto{\pgfqpoint{2.726023in}{1.468559in}}%
\pgfpathlineto{\pgfqpoint{2.706499in}{1.468559in}}%
\pgfpathlineto{\pgfqpoint{2.686976in}{1.482399in}}%
\pgfpathlineto{\pgfqpoint{2.667452in}{1.486342in}}%
\pgfpathlineto{\pgfqpoint{2.647929in}{1.513286in}}%
\pgfpathlineto{\pgfqpoint{2.628405in}{1.513286in}}%
\pgfpathlineto{\pgfqpoint{2.608881in}{1.513286in}}%
\pgfpathlineto{\pgfqpoint{2.589358in}{1.513286in}}%
\pgfpathlineto{\pgfqpoint{2.569834in}{1.535978in}}%
\pgfpathlineto{\pgfqpoint{2.550311in}{1.546188in}}%
\pgfpathlineto{\pgfqpoint{2.530787in}{1.551537in}}%
\pgfpathlineto{\pgfqpoint{2.511264in}{1.552868in}}%
\pgfpathlineto{\pgfqpoint{2.491740in}{1.556944in}}%
\pgfpathlineto{\pgfqpoint{2.472217in}{1.561377in}}%
\pgfpathlineto{\pgfqpoint{2.452693in}{1.566601in}}%
\pgfpathlineto{\pgfqpoint{2.433169in}{1.566601in}}%
\pgfpathlineto{\pgfqpoint{2.413646in}{1.575599in}}%
\pgfpathlineto{\pgfqpoint{2.394122in}{1.581393in}}%
\pgfpathlineto{\pgfqpoint{2.374599in}{1.591061in}}%
\pgfpathlineto{\pgfqpoint{2.355075in}{1.593034in}}%
\pgfpathlineto{\pgfqpoint{2.335552in}{1.617452in}}%
\pgfpathlineto{\pgfqpoint{2.316028in}{1.617452in}}%
\pgfpathlineto{\pgfqpoint{2.296505in}{1.675018in}}%
\pgfpathlineto{\pgfqpoint{2.276981in}{1.681879in}}%
\pgfpathlineto{\pgfqpoint{2.257458in}{1.681879in}}%
\pgfpathlineto{\pgfqpoint{2.237934in}{1.681879in}}%
\pgfpathlineto{\pgfqpoint{2.218410in}{1.681879in}}%
\pgfpathlineto{\pgfqpoint{2.198887in}{1.717559in}}%
\pgfpathlineto{\pgfqpoint{2.179363in}{1.717559in}}%
\pgfpathlineto{\pgfqpoint{2.159840in}{1.724126in}}%
\pgfpathlineto{\pgfqpoint{2.140316in}{1.787951in}}%
\pgfpathlineto{\pgfqpoint{2.120793in}{1.800124in}}%
\pgfpathlineto{\pgfqpoint{2.101269in}{1.926527in}}%
\pgfpathlineto{\pgfqpoint{2.081746in}{2.023077in}}%
\pgfpathlineto{\pgfqpoint{2.062222in}{2.122731in}}%
\pgfpathlineto{\pgfqpoint{2.042699in}{2.122731in}}%
\pgfpathlineto{\pgfqpoint{2.023175in}{2.126737in}}%
\pgfpathlineto{\pgfqpoint{2.003651in}{2.134012in}}%
\pgfpathlineto{\pgfqpoint{1.984128in}{2.134012in}}%
\pgfpathlineto{\pgfqpoint{1.964604in}{2.134012in}}%
\pgfpathlineto{\pgfqpoint{1.945081in}{2.151822in}}%
\pgfpathlineto{\pgfqpoint{1.925557in}{2.151822in}}%
\pgfpathlineto{\pgfqpoint{1.906034in}{2.151822in}}%
\pgfpathlineto{\pgfqpoint{1.886510in}{2.151822in}}%
\pgfpathlineto{\pgfqpoint{1.866987in}{2.151822in}}%
\pgfpathlineto{\pgfqpoint{1.847463in}{2.151822in}}%
\pgfpathlineto{\pgfqpoint{1.827939in}{2.151822in}}%
\pgfpathlineto{\pgfqpoint{1.808416in}{2.151822in}}%
\pgfpathlineto{\pgfqpoint{1.788892in}{2.151822in}}%
\pgfpathlineto{\pgfqpoint{1.769369in}{2.151822in}}%
\pgfpathlineto{\pgfqpoint{1.749845in}{2.151822in}}%
\pgfpathlineto{\pgfqpoint{1.730322in}{2.151822in}}%
\pgfpathlineto{\pgfqpoint{1.710798in}{2.151822in}}%
\pgfpathlineto{\pgfqpoint{1.691275in}{2.151822in}}%
\pgfpathlineto{\pgfqpoint{1.671751in}{2.151822in}}%
\pgfpathlineto{\pgfqpoint{1.652228in}{2.151822in}}%
\pgfpathlineto{\pgfqpoint{1.632704in}{2.151822in}}%
\pgfpathlineto{\pgfqpoint{1.613180in}{2.151822in}}%
\pgfpathlineto{\pgfqpoint{1.593657in}{2.151822in}}%
\pgfpathlineto{\pgfqpoint{1.574133in}{2.151822in}}%
\pgfpathlineto{\pgfqpoint{1.554610in}{2.151822in}}%
\pgfpathlineto{\pgfqpoint{1.535086in}{2.153903in}}%
\pgfpathlineto{\pgfqpoint{1.515563in}{2.153903in}}%
\pgfpathlineto{\pgfqpoint{1.496039in}{2.153903in}}%
\pgfpathlineto{\pgfqpoint{1.476516in}{2.153903in}}%
\pgfpathlineto{\pgfqpoint{1.456992in}{2.162124in}}%
\pgfpathlineto{\pgfqpoint{1.437469in}{2.162124in}}%
\pgfpathlineto{\pgfqpoint{1.417945in}{2.162124in}}%
\pgfpathlineto{\pgfqpoint{1.398421in}{2.167664in}}%
\pgfpathlineto{\pgfqpoint{1.378898in}{2.167664in}}%
\pgfpathlineto{\pgfqpoint{1.359374in}{2.194078in}}%
\pgfpathlineto{\pgfqpoint{1.339851in}{2.239045in}}%
\pgfpathlineto{\pgfqpoint{1.320327in}{2.239045in}}%
\pgfpathlineto{\pgfqpoint{1.300804in}{2.239045in}}%
\pgfpathlineto{\pgfqpoint{1.281280in}{2.239045in}}%
\pgfpathlineto{\pgfqpoint{1.261757in}{2.253567in}}%
\pgfpathlineto{\pgfqpoint{1.242233in}{2.253567in}}%
\pgfpathlineto{\pgfqpoint{1.222709in}{2.307311in}}%
\pgfpathlineto{\pgfqpoint{1.203186in}{2.307311in}}%
\pgfpathlineto{\pgfqpoint{1.183662in}{2.308186in}}%
\pgfpathlineto{\pgfqpoint{1.164139in}{2.315594in}}%
\pgfpathlineto{\pgfqpoint{1.144615in}{2.315594in}}%
\pgfpathlineto{\pgfqpoint{1.125092in}{2.315594in}}%
\pgfpathlineto{\pgfqpoint{1.105568in}{2.417025in}}%
\pgfpathclose%
\pgfusepath{fill}%
\end{pgfscope}%
\begin{pgfscope}%
\pgfpathrectangle{\pgfqpoint{0.862500in}{0.375000in}}{\pgfqpoint{5.347500in}{2.265000in}}%
\pgfusepath{clip}%
\pgfsetroundcap%
\pgfsetroundjoin%
\pgfsetlinewidth{1.505625pt}%
\definecolor{currentstroke}{rgb}{0.121569,0.466667,0.705882}%
\pgfsetstrokecolor{currentstroke}%
\pgfsetdash{}{0pt}%
\pgfpathmoveto{\pgfqpoint{1.105568in}{2.519837in}}%
\pgfpathlineto{\pgfqpoint{1.125092in}{2.465008in}}%
\pgfpathlineto{\pgfqpoint{1.144615in}{2.452776in}}%
\pgfpathlineto{\pgfqpoint{1.164139in}{2.432070in}}%
\pgfpathlineto{\pgfqpoint{1.183662in}{2.405817in}}%
\pgfpathlineto{\pgfqpoint{1.203186in}{2.375201in}}%
\pgfpathlineto{\pgfqpoint{1.222709in}{2.375201in}}%
\pgfpathlineto{\pgfqpoint{1.242233in}{2.347868in}}%
\pgfpathlineto{\pgfqpoint{1.261757in}{2.346134in}}%
\pgfpathlineto{\pgfqpoint{1.281280in}{2.346134in}}%
\pgfpathlineto{\pgfqpoint{1.300804in}{2.336027in}}%
\pgfpathlineto{\pgfqpoint{1.320327in}{2.314996in}}%
\pgfpathlineto{\pgfqpoint{1.359374in}{2.314996in}}%
\pgfpathlineto{\pgfqpoint{1.378898in}{2.289821in}}%
\pgfpathlineto{\pgfqpoint{1.398421in}{2.289821in}}%
\pgfpathlineto{\pgfqpoint{1.417945in}{2.263128in}}%
\pgfpathlineto{\pgfqpoint{1.476516in}{2.263128in}}%
\pgfpathlineto{\pgfqpoint{1.496039in}{2.251493in}}%
\pgfpathlineto{\pgfqpoint{1.515563in}{2.251493in}}%
\pgfpathlineto{\pgfqpoint{1.535086in}{2.244754in}}%
\pgfpathlineto{\pgfqpoint{1.554610in}{2.244754in}}%
\pgfpathlineto{\pgfqpoint{1.574133in}{2.234945in}}%
\pgfpathlineto{\pgfqpoint{1.593657in}{2.213414in}}%
\pgfpathlineto{\pgfqpoint{1.613180in}{2.189455in}}%
\pgfpathlineto{\pgfqpoint{1.710798in}{2.189455in}}%
\pgfpathlineto{\pgfqpoint{1.730322in}{2.185441in}}%
\pgfpathlineto{\pgfqpoint{1.749845in}{2.159898in}}%
\pgfpathlineto{\pgfqpoint{1.964604in}{2.158433in}}%
\pgfpathlineto{\pgfqpoint{1.984128in}{2.144255in}}%
\pgfpathlineto{\pgfqpoint{2.003651in}{2.123914in}}%
\pgfpathlineto{\pgfqpoint{2.023175in}{2.119805in}}%
\pgfpathlineto{\pgfqpoint{2.140316in}{2.119805in}}%
\pgfpathlineto{\pgfqpoint{2.159840in}{2.084271in}}%
\pgfpathlineto{\pgfqpoint{2.179363in}{2.084271in}}%
\pgfpathlineto{\pgfqpoint{2.198887in}{2.079857in}}%
\pgfpathlineto{\pgfqpoint{2.647929in}{2.079857in}}%
\pgfpathlineto{\pgfqpoint{2.667452in}{2.057192in}}%
\pgfpathlineto{\pgfqpoint{2.745546in}{2.057192in}}%
\pgfpathlineto{\pgfqpoint{2.765070in}{2.019579in}}%
\pgfpathlineto{\pgfqpoint{2.823640in}{2.019579in}}%
\pgfpathlineto{\pgfqpoint{2.843164in}{2.014146in}}%
\pgfpathlineto{\pgfqpoint{2.921258in}{2.014146in}}%
\pgfpathlineto{\pgfqpoint{2.940782in}{1.999995in}}%
\pgfpathlineto{\pgfqpoint{3.175064in}{1.999151in}}%
\pgfpathlineto{\pgfqpoint{3.194588in}{1.992646in}}%
\pgfpathlineto{\pgfqpoint{3.877912in}{1.992646in}}%
\pgfpathlineto{\pgfqpoint{3.897436in}{1.944557in}}%
\pgfpathlineto{\pgfqpoint{4.131718in}{1.944557in}}%
\pgfpathlineto{\pgfqpoint{4.151242in}{1.909630in}}%
\pgfpathlineto{\pgfqpoint{4.248860in}{1.909630in}}%
\pgfpathlineto{\pgfqpoint{4.268383in}{1.891552in}}%
\pgfpathlineto{\pgfqpoint{5.459320in}{1.890227in}}%
\pgfpathlineto{\pgfqpoint{5.478843in}{1.873037in}}%
\pgfpathlineto{\pgfqpoint{5.966932in}{1.873037in}}%
\pgfpathlineto{\pgfqpoint{5.966932in}{1.873037in}}%
\pgfusepath{stroke}%
\end{pgfscope}%
\begin{pgfscope}%
\pgfpathrectangle{\pgfqpoint{0.862500in}{0.375000in}}{\pgfqpoint{5.347500in}{2.265000in}}%
\pgfusepath{clip}%
\pgfsetroundcap%
\pgfsetroundjoin%
\pgfsetlinewidth{1.505625pt}%
\definecolor{currentstroke}{rgb}{1.000000,0.498039,0.054902}%
\pgfsetstrokecolor{currentstroke}%
\pgfsetdash{}{0pt}%
\pgfpathmoveto{\pgfqpoint{1.105568in}{2.488987in}}%
\pgfpathlineto{\pgfqpoint{1.125092in}{2.478256in}}%
\pgfpathlineto{\pgfqpoint{1.144615in}{2.423186in}}%
\pgfpathlineto{\pgfqpoint{1.164139in}{2.333696in}}%
\pgfpathlineto{\pgfqpoint{1.183662in}{2.333696in}}%
\pgfpathlineto{\pgfqpoint{1.203186in}{2.304648in}}%
\pgfpathlineto{\pgfqpoint{1.222709in}{2.271744in}}%
\pgfpathlineto{\pgfqpoint{1.320327in}{2.271744in}}%
\pgfpathlineto{\pgfqpoint{1.339851in}{2.269815in}}%
\pgfpathlineto{\pgfqpoint{1.359374in}{2.269815in}}%
\pgfpathlineto{\pgfqpoint{1.378898in}{2.198625in}}%
\pgfpathlineto{\pgfqpoint{1.437469in}{2.198625in}}%
\pgfpathlineto{\pgfqpoint{1.456992in}{2.156336in}}%
\pgfpathlineto{\pgfqpoint{1.671751in}{2.156336in}}%
\pgfpathlineto{\pgfqpoint{1.691275in}{2.123203in}}%
\pgfpathlineto{\pgfqpoint{1.710798in}{2.123203in}}%
\pgfpathlineto{\pgfqpoint{1.730322in}{2.112127in}}%
\pgfpathlineto{\pgfqpoint{1.749845in}{2.082120in}}%
\pgfpathlineto{\pgfqpoint{2.062222in}{2.082120in}}%
\pgfpathlineto{\pgfqpoint{2.081746in}{2.026004in}}%
\pgfpathlineto{\pgfqpoint{2.179363in}{2.026004in}}%
\pgfpathlineto{\pgfqpoint{2.198887in}{1.965126in}}%
\pgfpathlineto{\pgfqpoint{2.218410in}{1.965126in}}%
\pgfpathlineto{\pgfqpoint{2.237934in}{1.909992in}}%
\pgfpathlineto{\pgfqpoint{2.257458in}{1.821569in}}%
\pgfpathlineto{\pgfqpoint{2.296505in}{1.821116in}}%
\pgfpathlineto{\pgfqpoint{2.316028in}{1.803901in}}%
\pgfpathlineto{\pgfqpoint{2.335552in}{1.723481in}}%
\pgfpathlineto{\pgfqpoint{2.355075in}{1.673416in}}%
\pgfpathlineto{\pgfqpoint{2.394122in}{1.672458in}}%
\pgfpathlineto{\pgfqpoint{2.413646in}{1.672458in}}%
\pgfpathlineto{\pgfqpoint{2.433169in}{1.617837in}}%
\pgfpathlineto{\pgfqpoint{2.472217in}{1.617837in}}%
\pgfpathlineto{\pgfqpoint{2.491740in}{1.606822in}}%
\pgfpathlineto{\pgfqpoint{2.511264in}{1.601478in}}%
\pgfpathlineto{\pgfqpoint{2.530787in}{1.581712in}}%
\pgfpathlineto{\pgfqpoint{2.550311in}{1.528131in}}%
\pgfpathlineto{\pgfqpoint{2.569834in}{1.493698in}}%
\pgfpathlineto{\pgfqpoint{2.628405in}{1.493108in}}%
\pgfpathlineto{\pgfqpoint{2.647929in}{1.463874in}}%
\pgfpathlineto{\pgfqpoint{2.706499in}{1.463874in}}%
\pgfpathlineto{\pgfqpoint{2.726023in}{1.423805in}}%
\pgfpathlineto{\pgfqpoint{2.745546in}{1.423805in}}%
\pgfpathlineto{\pgfqpoint{2.765070in}{1.420740in}}%
\pgfpathlineto{\pgfqpoint{2.843164in}{1.420740in}}%
\pgfpathlineto{\pgfqpoint{2.862688in}{1.363123in}}%
\pgfpathlineto{\pgfqpoint{2.999352in}{1.363123in}}%
\pgfpathlineto{\pgfqpoint{3.018876in}{1.338827in}}%
\pgfpathlineto{\pgfqpoint{3.057923in}{1.338827in}}%
\pgfpathlineto{\pgfqpoint{3.077447in}{1.330190in}}%
\pgfpathlineto{\pgfqpoint{3.116494in}{1.330190in}}%
\pgfpathlineto{\pgfqpoint{3.136017in}{1.285554in}}%
\pgfpathlineto{\pgfqpoint{3.155541in}{1.285554in}}%
\pgfpathlineto{\pgfqpoint{3.175064in}{1.276337in}}%
\pgfpathlineto{\pgfqpoint{3.194588in}{1.215039in}}%
\pgfpathlineto{\pgfqpoint{3.214111in}{1.200714in}}%
\pgfpathlineto{\pgfqpoint{3.253159in}{1.200714in}}%
\pgfpathlineto{\pgfqpoint{3.272682in}{1.178979in}}%
\pgfpathlineto{\pgfqpoint{3.487441in}{1.177957in}}%
\pgfpathlineto{\pgfqpoint{3.506965in}{1.086818in}}%
\pgfpathlineto{\pgfqpoint{3.702200in}{1.086818in}}%
\pgfpathlineto{\pgfqpoint{3.721724in}{1.083815in}}%
\pgfpathlineto{\pgfqpoint{4.170765in}{1.083815in}}%
\pgfpathlineto{\pgfqpoint{4.190289in}{1.068399in}}%
\pgfpathlineto{\pgfqpoint{4.268383in}{1.068399in}}%
\pgfpathlineto{\pgfqpoint{4.287907in}{1.048552in}}%
\pgfpathlineto{\pgfqpoint{4.307430in}{0.987760in}}%
\pgfpathlineto{\pgfqpoint{4.561236in}{0.987760in}}%
\pgfpathlineto{\pgfqpoint{4.580760in}{0.979302in}}%
\pgfpathlineto{\pgfqpoint{4.600283in}{0.887853in}}%
\pgfpathlineto{\pgfqpoint{4.619807in}{0.880264in}}%
\pgfpathlineto{\pgfqpoint{4.639331in}{0.864991in}}%
\pgfpathlineto{\pgfqpoint{4.854090in}{0.864991in}}%
\pgfpathlineto{\pgfqpoint{4.873613in}{0.857964in}}%
\pgfpathlineto{\pgfqpoint{5.752173in}{0.857964in}}%
\pgfpathlineto{\pgfqpoint{5.771696in}{0.839046in}}%
\pgfpathlineto{\pgfqpoint{5.966932in}{0.839046in}}%
\pgfpathlineto{\pgfqpoint{5.966932in}{0.839046in}}%
\pgfusepath{stroke}%
\end{pgfscope}%
\begin{pgfscope}%
\pgfpathrectangle{\pgfqpoint{0.862500in}{0.375000in}}{\pgfqpoint{5.347500in}{2.265000in}}%
\pgfusepath{clip}%
\pgfsetroundcap%
\pgfsetroundjoin%
\pgfsetlinewidth{1.505625pt}%
\definecolor{currentstroke}{rgb}{0.172549,0.627451,0.172549}%
\pgfsetstrokecolor{currentstroke}%
\pgfsetdash{}{0pt}%
\pgfpathmoveto{\pgfqpoint{1.105568in}{2.483843in}}%
\pgfpathlineto{\pgfqpoint{1.125092in}{2.468559in}}%
\pgfpathlineto{\pgfqpoint{1.144615in}{2.430960in}}%
\pgfpathlineto{\pgfqpoint{1.164139in}{2.430960in}}%
\pgfpathlineto{\pgfqpoint{1.183662in}{2.379749in}}%
\pgfpathlineto{\pgfqpoint{1.203186in}{2.291141in}}%
\pgfpathlineto{\pgfqpoint{1.222709in}{2.271172in}}%
\pgfpathlineto{\pgfqpoint{1.242233in}{2.244701in}}%
\pgfpathlineto{\pgfqpoint{1.320327in}{2.244701in}}%
\pgfpathlineto{\pgfqpoint{1.339851in}{2.225318in}}%
\pgfpathlineto{\pgfqpoint{1.378898in}{2.225318in}}%
\pgfpathlineto{\pgfqpoint{1.398421in}{2.213518in}}%
\pgfpathlineto{\pgfqpoint{1.417945in}{2.177416in}}%
\pgfpathlineto{\pgfqpoint{1.593657in}{2.177416in}}%
\pgfpathlineto{\pgfqpoint{1.613180in}{2.154862in}}%
\pgfpathlineto{\pgfqpoint{1.769369in}{2.154862in}}%
\pgfpathlineto{\pgfqpoint{1.788892in}{2.130289in}}%
\pgfpathlineto{\pgfqpoint{2.062222in}{2.130289in}}%
\pgfpathlineto{\pgfqpoint{2.081746in}{2.074774in}}%
\pgfpathlineto{\pgfqpoint{2.101269in}{2.074536in}}%
\pgfpathlineto{\pgfqpoint{2.120793in}{1.940969in}}%
\pgfpathlineto{\pgfqpoint{2.140316in}{1.940969in}}%
\pgfpathlineto{\pgfqpoint{2.159840in}{1.938734in}}%
\pgfpathlineto{\pgfqpoint{2.179363in}{1.892360in}}%
\pgfpathlineto{\pgfqpoint{2.198887in}{1.857630in}}%
\pgfpathlineto{\pgfqpoint{2.218410in}{1.839472in}}%
\pgfpathlineto{\pgfqpoint{2.237934in}{1.755725in}}%
\pgfpathlineto{\pgfqpoint{2.257458in}{1.755725in}}%
\pgfpathlineto{\pgfqpoint{2.276981in}{1.749078in}}%
\pgfpathlineto{\pgfqpoint{2.296505in}{1.713366in}}%
\pgfpathlineto{\pgfqpoint{2.316028in}{1.706999in}}%
\pgfpathlineto{\pgfqpoint{2.335552in}{1.691184in}}%
\pgfpathlineto{\pgfqpoint{2.355075in}{1.691184in}}%
\pgfpathlineto{\pgfqpoint{2.374599in}{1.682138in}}%
\pgfpathlineto{\pgfqpoint{2.472217in}{1.682138in}}%
\pgfpathlineto{\pgfqpoint{2.491740in}{1.678765in}}%
\pgfpathlineto{\pgfqpoint{2.511264in}{1.678765in}}%
\pgfpathlineto{\pgfqpoint{2.550311in}{1.647571in}}%
\pgfpathlineto{\pgfqpoint{2.667452in}{1.647058in}}%
\pgfpathlineto{\pgfqpoint{2.686976in}{1.605737in}}%
\pgfpathlineto{\pgfqpoint{2.706499in}{1.593542in}}%
\pgfpathlineto{\pgfqpoint{2.726023in}{1.593542in}}%
\pgfpathlineto{\pgfqpoint{2.745546in}{1.572395in}}%
\pgfpathlineto{\pgfqpoint{2.784593in}{1.572395in}}%
\pgfpathlineto{\pgfqpoint{2.804117in}{1.554738in}}%
\pgfpathlineto{\pgfqpoint{2.823640in}{1.554738in}}%
\pgfpathlineto{\pgfqpoint{2.843164in}{1.551023in}}%
\pgfpathlineto{\pgfqpoint{2.862688in}{1.551023in}}%
\pgfpathlineto{\pgfqpoint{2.882211in}{1.546768in}}%
\pgfpathlineto{\pgfqpoint{2.901735in}{1.527491in}}%
\pgfpathlineto{\pgfqpoint{2.979829in}{1.525990in}}%
\pgfpathlineto{\pgfqpoint{2.999352in}{1.502918in}}%
\pgfpathlineto{\pgfqpoint{3.018876in}{1.499710in}}%
\pgfpathlineto{\pgfqpoint{3.038400in}{1.499710in}}%
\pgfpathlineto{\pgfqpoint{3.057923in}{1.449498in}}%
\pgfpathlineto{\pgfqpoint{3.077447in}{1.432241in}}%
\pgfpathlineto{\pgfqpoint{3.096970in}{1.432241in}}%
\pgfpathlineto{\pgfqpoint{3.136017in}{1.391902in}}%
\pgfpathlineto{\pgfqpoint{3.155541in}{1.361842in}}%
\pgfpathlineto{\pgfqpoint{3.233635in}{1.361842in}}%
\pgfpathlineto{\pgfqpoint{3.253159in}{1.328831in}}%
\pgfpathlineto{\pgfqpoint{3.292206in}{1.328831in}}%
\pgfpathlineto{\pgfqpoint{3.311729in}{1.310037in}}%
\pgfpathlineto{\pgfqpoint{3.350776in}{1.310037in}}%
\pgfpathlineto{\pgfqpoint{3.370300in}{1.301312in}}%
\pgfpathlineto{\pgfqpoint{3.389823in}{1.301312in}}%
\pgfpathlineto{\pgfqpoint{3.409347in}{1.266868in}}%
\pgfpathlineto{\pgfqpoint{3.448394in}{1.266868in}}%
\pgfpathlineto{\pgfqpoint{3.467918in}{1.255025in}}%
\pgfpathlineto{\pgfqpoint{3.604582in}{1.255025in}}%
\pgfpathlineto{\pgfqpoint{3.624106in}{1.251614in}}%
\pgfpathlineto{\pgfqpoint{3.819341in}{1.251614in}}%
\pgfpathlineto{\pgfqpoint{3.838865in}{1.236801in}}%
\pgfpathlineto{\pgfqpoint{3.916959in}{1.236801in}}%
\pgfpathlineto{\pgfqpoint{3.936483in}{1.192487in}}%
\pgfpathlineto{\pgfqpoint{3.956006in}{1.168454in}}%
\pgfpathlineto{\pgfqpoint{3.975530in}{1.168454in}}%
\pgfpathlineto{\pgfqpoint{3.995053in}{1.090189in}}%
\pgfpathlineto{\pgfqpoint{4.073148in}{1.090189in}}%
\pgfpathlineto{\pgfqpoint{4.092671in}{1.064316in}}%
\pgfpathlineto{\pgfqpoint{4.112195in}{1.064316in}}%
\pgfpathlineto{\pgfqpoint{4.131718in}{1.060083in}}%
\pgfpathlineto{\pgfqpoint{4.268383in}{1.060083in}}%
\pgfpathlineto{\pgfqpoint{4.287907in}{1.023805in}}%
\pgfpathlineto{\pgfqpoint{4.541713in}{1.023805in}}%
\pgfpathlineto{\pgfqpoint{4.561236in}{1.016444in}}%
\pgfpathlineto{\pgfqpoint{4.639331in}{1.016444in}}%
\pgfpathlineto{\pgfqpoint{4.658854in}{0.915213in}}%
\pgfpathlineto{\pgfqpoint{4.697901in}{0.915213in}}%
\pgfpathlineto{\pgfqpoint{4.717425in}{0.893901in}}%
\pgfpathlineto{\pgfqpoint{4.815042in}{0.893901in}}%
\pgfpathlineto{\pgfqpoint{4.834566in}{0.873654in}}%
\pgfpathlineto{\pgfqpoint{5.127419in}{0.873654in}}%
\pgfpathlineto{\pgfqpoint{5.146943in}{0.829294in}}%
\pgfpathlineto{\pgfqpoint{5.654555in}{0.829294in}}%
\pgfpathlineto{\pgfqpoint{5.674079in}{0.779731in}}%
\pgfpathlineto{\pgfqpoint{5.693602in}{0.779731in}}%
\pgfpathlineto{\pgfqpoint{5.713126in}{0.745001in}}%
\pgfpathlineto{\pgfqpoint{5.791220in}{0.745001in}}%
\pgfpathlineto{\pgfqpoint{5.810743in}{0.715668in}}%
\pgfpathlineto{\pgfqpoint{5.830267in}{0.715668in}}%
\pgfpathlineto{\pgfqpoint{5.849791in}{0.693762in}}%
\pgfpathlineto{\pgfqpoint{5.869314in}{0.693762in}}%
\pgfpathlineto{\pgfqpoint{5.888838in}{0.675723in}}%
\pgfpathlineto{\pgfqpoint{5.966932in}{0.675723in}}%
\pgfpathlineto{\pgfqpoint{5.966932in}{0.675723in}}%
\pgfusepath{stroke}%
\end{pgfscope}%
\begin{pgfscope}%
\pgfpathrectangle{\pgfqpoint{0.862500in}{0.375000in}}{\pgfqpoint{5.347500in}{2.265000in}}%
\pgfusepath{clip}%
\pgfsetroundcap%
\pgfsetroundjoin%
\pgfsetlinewidth{1.505625pt}%
\definecolor{currentstroke}{rgb}{0.839216,0.152941,0.156863}%
\pgfsetstrokecolor{currentstroke}%
\pgfsetdash{}{0pt}%
\pgfpathmoveto{\pgfqpoint{1.105568in}{2.467504in}}%
\pgfpathlineto{\pgfqpoint{1.125092in}{2.380998in}}%
\pgfpathlineto{\pgfqpoint{1.164139in}{2.380998in}}%
\pgfpathlineto{\pgfqpoint{1.183662in}{2.375929in}}%
\pgfpathlineto{\pgfqpoint{1.203186in}{2.373370in}}%
\pgfpathlineto{\pgfqpoint{1.222709in}{2.373370in}}%
\pgfpathlineto{\pgfqpoint{1.242233in}{2.304208in}}%
\pgfpathlineto{\pgfqpoint{1.261757in}{2.304208in}}%
\pgfpathlineto{\pgfqpoint{1.281280in}{2.290402in}}%
\pgfpathlineto{\pgfqpoint{1.339851in}{2.290402in}}%
\pgfpathlineto{\pgfqpoint{1.359374in}{2.261747in}}%
\pgfpathlineto{\pgfqpoint{1.378898in}{2.228601in}}%
\pgfpathlineto{\pgfqpoint{1.398421in}{2.228601in}}%
\pgfpathlineto{\pgfqpoint{1.417945in}{2.218315in}}%
\pgfpathlineto{\pgfqpoint{1.456992in}{2.218315in}}%
\pgfpathlineto{\pgfqpoint{1.476516in}{2.206257in}}%
\pgfpathlineto{\pgfqpoint{1.535086in}{2.206257in}}%
\pgfpathlineto{\pgfqpoint{1.554610in}{2.203655in}}%
\pgfpathlineto{\pgfqpoint{1.945081in}{2.203655in}}%
\pgfpathlineto{\pgfqpoint{1.964604in}{2.185497in}}%
\pgfpathlineto{\pgfqpoint{2.003651in}{2.185497in}}%
\pgfpathlineto{\pgfqpoint{2.023175in}{2.176613in}}%
\pgfpathlineto{\pgfqpoint{2.042699in}{2.172393in}}%
\pgfpathlineto{\pgfqpoint{2.062222in}{2.172393in}}%
\pgfpathlineto{\pgfqpoint{2.081746in}{2.112024in}}%
\pgfpathlineto{\pgfqpoint{2.101269in}{2.039698in}}%
\pgfpathlineto{\pgfqpoint{2.120793in}{1.916758in}}%
\pgfpathlineto{\pgfqpoint{2.140316in}{1.888063in}}%
\pgfpathlineto{\pgfqpoint{2.159840in}{1.806194in}}%
\pgfpathlineto{\pgfqpoint{2.179363in}{1.797729in}}%
\pgfpathlineto{\pgfqpoint{2.198887in}{1.797729in}}%
\pgfpathlineto{\pgfqpoint{2.218410in}{1.776506in}}%
\pgfpathlineto{\pgfqpoint{2.276981in}{1.776506in}}%
\pgfpathlineto{\pgfqpoint{2.296505in}{1.766189in}}%
\pgfpathlineto{\pgfqpoint{2.316028in}{1.722715in}}%
\pgfpathlineto{\pgfqpoint{2.335552in}{1.722715in}}%
\pgfpathlineto{\pgfqpoint{2.355075in}{1.710655in}}%
\pgfpathlineto{\pgfqpoint{2.374599in}{1.708380in}}%
\pgfpathlineto{\pgfqpoint{2.394122in}{1.700716in}}%
\pgfpathlineto{\pgfqpoint{2.413646in}{1.698102in}}%
\pgfpathlineto{\pgfqpoint{2.433169in}{1.679754in}}%
\pgfpathlineto{\pgfqpoint{2.452693in}{1.679754in}}%
\pgfpathlineto{\pgfqpoint{2.491740in}{1.675936in}}%
\pgfpathlineto{\pgfqpoint{2.511264in}{1.668868in}}%
\pgfpathlineto{\pgfqpoint{2.550311in}{1.663968in}}%
\pgfpathlineto{\pgfqpoint{2.569834in}{1.659701in}}%
\pgfpathlineto{\pgfqpoint{2.589358in}{1.641262in}}%
\pgfpathlineto{\pgfqpoint{2.647929in}{1.641262in}}%
\pgfpathlineto{\pgfqpoint{2.667452in}{1.630836in}}%
\pgfpathlineto{\pgfqpoint{2.686976in}{1.627107in}}%
\pgfpathlineto{\pgfqpoint{2.706499in}{1.615605in}}%
\pgfpathlineto{\pgfqpoint{2.765070in}{1.615605in}}%
\pgfpathlineto{\pgfqpoint{2.784593in}{1.601948in}}%
\pgfpathlineto{\pgfqpoint{2.804117in}{1.601948in}}%
\pgfpathlineto{\pgfqpoint{2.823640in}{1.598794in}}%
\pgfpathlineto{\pgfqpoint{2.843164in}{1.598794in}}%
\pgfpathlineto{\pgfqpoint{2.862688in}{1.586556in}}%
\pgfpathlineto{\pgfqpoint{2.882211in}{1.556524in}}%
\pgfpathlineto{\pgfqpoint{2.901735in}{1.536947in}}%
\pgfpathlineto{\pgfqpoint{2.921258in}{1.536947in}}%
\pgfpathlineto{\pgfqpoint{2.940782in}{1.510245in}}%
\pgfpathlineto{\pgfqpoint{2.960305in}{1.509877in}}%
\pgfpathlineto{\pgfqpoint{2.979829in}{1.503541in}}%
\pgfpathlineto{\pgfqpoint{2.999352in}{1.489554in}}%
\pgfpathlineto{\pgfqpoint{3.018876in}{1.481833in}}%
\pgfpathlineto{\pgfqpoint{3.038400in}{1.481833in}}%
\pgfpathlineto{\pgfqpoint{3.057923in}{1.479464in}}%
\pgfpathlineto{\pgfqpoint{3.077447in}{1.467290in}}%
\pgfpathlineto{\pgfqpoint{3.096970in}{1.451368in}}%
\pgfpathlineto{\pgfqpoint{3.116494in}{1.451368in}}%
\pgfpathlineto{\pgfqpoint{3.136017in}{1.447655in}}%
\pgfpathlineto{\pgfqpoint{3.155541in}{1.413326in}}%
\pgfpathlineto{\pgfqpoint{3.175064in}{1.407864in}}%
\pgfpathlineto{\pgfqpoint{3.233635in}{1.407864in}}%
\pgfpathlineto{\pgfqpoint{3.253159in}{1.393710in}}%
\pgfpathlineto{\pgfqpoint{3.272682in}{1.393710in}}%
\pgfpathlineto{\pgfqpoint{3.292206in}{1.351842in}}%
\pgfpathlineto{\pgfqpoint{3.331253in}{1.349293in}}%
\pgfpathlineto{\pgfqpoint{3.448394in}{1.349293in}}%
\pgfpathlineto{\pgfqpoint{3.467918in}{1.323911in}}%
\pgfpathlineto{\pgfqpoint{3.624106in}{1.323578in}}%
\pgfpathlineto{\pgfqpoint{3.643630in}{1.308775in}}%
\pgfpathlineto{\pgfqpoint{3.663153in}{1.306550in}}%
\pgfpathlineto{\pgfqpoint{3.682677in}{1.306550in}}%
\pgfpathlineto{\pgfqpoint{3.702200in}{1.292427in}}%
\pgfpathlineto{\pgfqpoint{3.721724in}{1.240675in}}%
\pgfpathlineto{\pgfqpoint{3.799818in}{1.240675in}}%
\pgfpathlineto{\pgfqpoint{3.819341in}{1.208783in}}%
\pgfpathlineto{\pgfqpoint{3.838865in}{0.729554in}}%
\pgfpathlineto{\pgfqpoint{3.956006in}{0.729554in}}%
\pgfpathlineto{\pgfqpoint{3.995053in}{0.691262in}}%
\pgfpathlineto{\pgfqpoint{4.053624in}{0.691262in}}%
\pgfpathlineto{\pgfqpoint{4.073148in}{0.608118in}}%
\pgfpathlineto{\pgfqpoint{4.580760in}{0.608118in}}%
\pgfpathlineto{\pgfqpoint{4.600283in}{0.589530in}}%
\pgfpathlineto{\pgfqpoint{4.912660in}{0.589530in}}%
\pgfpathlineto{\pgfqpoint{4.932184in}{0.553990in}}%
\pgfpathlineto{\pgfqpoint{5.771696in}{0.553990in}}%
\pgfpathlineto{\pgfqpoint{5.791220in}{0.601017in}}%
\pgfpathlineto{\pgfqpoint{5.966932in}{0.601017in}}%
\pgfpathlineto{\pgfqpoint{5.966932in}{0.601017in}}%
\pgfusepath{stroke}%
\end{pgfscope}%
\begin{pgfscope}%
\pgfsetrectcap%
\pgfsetmiterjoin%
\pgfsetlinewidth{0.000000pt}%
\definecolor{currentstroke}{rgb}{1.000000,1.000000,1.000000}%
\pgfsetstrokecolor{currentstroke}%
\pgfsetdash{}{0pt}%
\pgfpathmoveto{\pgfqpoint{0.862500in}{0.375000in}}%
\pgfpathlineto{\pgfqpoint{0.862500in}{2.640000in}}%
\pgfusepath{}%
\end{pgfscope}%
\begin{pgfscope}%
\pgfsetrectcap%
\pgfsetmiterjoin%
\pgfsetlinewidth{0.000000pt}%
\definecolor{currentstroke}{rgb}{1.000000,1.000000,1.000000}%
\pgfsetstrokecolor{currentstroke}%
\pgfsetdash{}{0pt}%
\pgfpathmoveto{\pgfqpoint{6.210000in}{0.375000in}}%
\pgfpathlineto{\pgfqpoint{6.210000in}{2.640000in}}%
\pgfusepath{}%
\end{pgfscope}%
\begin{pgfscope}%
\pgfsetrectcap%
\pgfsetmiterjoin%
\pgfsetlinewidth{0.000000pt}%
\definecolor{currentstroke}{rgb}{1.000000,1.000000,1.000000}%
\pgfsetstrokecolor{currentstroke}%
\pgfsetdash{}{0pt}%
\pgfpathmoveto{\pgfqpoint{0.862500in}{0.375000in}}%
\pgfpathlineto{\pgfqpoint{6.210000in}{0.375000in}}%
\pgfusepath{}%
\end{pgfscope}%
\begin{pgfscope}%
\pgfsetrectcap%
\pgfsetmiterjoin%
\pgfsetlinewidth{0.000000pt}%
\definecolor{currentstroke}{rgb}{1.000000,1.000000,1.000000}%
\pgfsetstrokecolor{currentstroke}%
\pgfsetdash{}{0pt}%
\pgfpathmoveto{\pgfqpoint{0.862500in}{2.640000in}}%
\pgfpathlineto{\pgfqpoint{6.210000in}{2.640000in}}%
\pgfusepath{}%
\end{pgfscope}%
\begin{pgfscope}%
\definecolor{textcolor}{rgb}{0.150000,0.150000,0.150000}%
\pgfsetstrokecolor{textcolor}%
\pgfsetfillcolor{textcolor}%
\pgftext[x=3.536250in,y=2.723333in,,base]{\color{textcolor}\rmfamily\fontsize{8.000000}{9.600000}\selectfont Hartmann3 Epochs}%
\end{pgfscope}%
\begin{pgfscope}%
\pgfsetroundcap%
\pgfsetroundjoin%
\pgfsetlinewidth{1.505625pt}%
\definecolor{currentstroke}{rgb}{0.121569,0.466667,0.705882}%
\pgfsetstrokecolor{currentstroke}%
\pgfsetdash{}{0pt}%
\pgfpathmoveto{\pgfqpoint{0.962500in}{1.026258in}}%
\pgfpathlineto{\pgfqpoint{1.184722in}{1.026258in}}%
\pgfusepath{stroke}%
\end{pgfscope}%
\begin{pgfscope}%
\definecolor{textcolor}{rgb}{0.150000,0.150000,0.150000}%
\pgfsetstrokecolor{textcolor}%
\pgfsetfillcolor{textcolor}%
\pgftext[x=1.273611in,y=0.987369in,left,base]{\color{textcolor}\rmfamily\fontsize{8.000000}{9.600000}\selectfont random}%
\end{pgfscope}%
\begin{pgfscope}%
\pgfsetroundcap%
\pgfsetroundjoin%
\pgfsetlinewidth{1.505625pt}%
\definecolor{currentstroke}{rgb}{1.000000,0.498039,0.054902}%
\pgfsetstrokecolor{currentstroke}%
\pgfsetdash{}{0pt}%
\pgfpathmoveto{\pgfqpoint{0.962500in}{0.863172in}}%
\pgfpathlineto{\pgfqpoint{1.184722in}{0.863172in}}%
\pgfusepath{stroke}%
\end{pgfscope}%
\begin{pgfscope}%
\definecolor{textcolor}{rgb}{0.150000,0.150000,0.150000}%
\pgfsetstrokecolor{textcolor}%
\pgfsetfillcolor{textcolor}%
\pgftext[x=1.273611in,y=0.824283in,left,base]{\color{textcolor}\rmfamily\fontsize{8.000000}{9.600000}\selectfont DNGO retrain-reset with 100 epochs}%
\end{pgfscope}%
\begin{pgfscope}%
\pgfsetroundcap%
\pgfsetroundjoin%
\pgfsetlinewidth{1.505625pt}%
\definecolor{currentstroke}{rgb}{0.172549,0.627451,0.172549}%
\pgfsetstrokecolor{currentstroke}%
\pgfsetdash{}{0pt}%
\pgfpathmoveto{\pgfqpoint{0.962500in}{0.700087in}}%
\pgfpathlineto{\pgfqpoint{1.184722in}{0.700087in}}%
\pgfusepath{stroke}%
\end{pgfscope}%
\begin{pgfscope}%
\definecolor{textcolor}{rgb}{0.150000,0.150000,0.150000}%
\pgfsetstrokecolor{textcolor}%
\pgfsetfillcolor{textcolor}%
\pgftext[x=1.273611in,y=0.661198in,left,base]{\color{textcolor}\rmfamily\fontsize{8.000000}{9.600000}\selectfont DNGO retrain-reset with 1000 epochs}%
\end{pgfscope}%
\begin{pgfscope}%
\pgfsetroundcap%
\pgfsetroundjoin%
\pgfsetlinewidth{1.505625pt}%
\definecolor{currentstroke}{rgb}{0.839216,0.152941,0.156863}%
\pgfsetstrokecolor{currentstroke}%
\pgfsetdash{}{0pt}%
\pgfpathmoveto{\pgfqpoint{0.962500in}{0.537001in}}%
\pgfpathlineto{\pgfqpoint{1.184722in}{0.537001in}}%
\pgfusepath{stroke}%
\end{pgfscope}%
\begin{pgfscope}%
\definecolor{textcolor}{rgb}{0.150000,0.150000,0.150000}%
\pgfsetstrokecolor{textcolor}%
\pgfsetfillcolor{textcolor}%
\pgftext[x=1.273611in,y=0.498112in,left,base]{\color{textcolor}\rmfamily\fontsize{8.000000}{9.600000}\selectfont DNGO retrain-reset with 10000 epochs}%
\end{pgfscope}%
\end{pgfpicture}%
\makeatother%
\endgroup%

        %\caption{Epochs}
    \end{minipage}
    \begin{minipage}{\linewidth}
        \centering
        %% Creator: Matplotlib, PGF backend
%%
%% To include the figure in your LaTeX document, write
%%   \input{<filename>.pgf}
%%
%% Make sure the required packages are loaded in your preamble
%%   \usepackage{pgf}
%%
%% Figures using additional raster images can only be included by \input if
%% they are in the same directory as the main LaTeX file. For loading figures
%% from other directories you can use the `import` package
%%   \usepackage{import}
%% and then include the figures with
%%   \import{<path to file>}{<filename>.pgf}
%%
%% Matplotlib used the following preamble
%%   \usepackage{gensymb}
%%   \usepackage{fontspec}
%%   \setmainfont{DejaVu Serif}
%%   \setsansfont{Arial}
%%   \setmonofont{DejaVu Sans Mono}
%%
\begingroup%
\makeatletter%
\begin{pgfpicture}%
\pgfpathrectangle{\pgfpointorigin}{\pgfqpoint{6.900000in}{3.000000in}}%
\pgfusepath{use as bounding box, clip}%
\begin{pgfscope}%
\pgfsetbuttcap%
\pgfsetmiterjoin%
\definecolor{currentfill}{rgb}{1.000000,1.000000,1.000000}%
\pgfsetfillcolor{currentfill}%
\pgfsetlinewidth{0.000000pt}%
\definecolor{currentstroke}{rgb}{1.000000,1.000000,1.000000}%
\pgfsetstrokecolor{currentstroke}%
\pgfsetdash{}{0pt}%
\pgfpathmoveto{\pgfqpoint{0.000000in}{0.000000in}}%
\pgfpathlineto{\pgfqpoint{6.900000in}{0.000000in}}%
\pgfpathlineto{\pgfqpoint{6.900000in}{3.000000in}}%
\pgfpathlineto{\pgfqpoint{0.000000in}{3.000000in}}%
\pgfpathclose%
\pgfusepath{fill}%
\end{pgfscope}%
\begin{pgfscope}%
\pgfsetbuttcap%
\pgfsetmiterjoin%
\definecolor{currentfill}{rgb}{0.917647,0.917647,0.949020}%
\pgfsetfillcolor{currentfill}%
\pgfsetlinewidth{0.000000pt}%
\definecolor{currentstroke}{rgb}{0.000000,0.000000,0.000000}%
\pgfsetstrokecolor{currentstroke}%
\pgfsetstrokeopacity{0.000000}%
\pgfsetdash{}{0pt}%
\pgfpathmoveto{\pgfqpoint{0.862500in}{0.375000in}}%
\pgfpathlineto{\pgfqpoint{6.210000in}{0.375000in}}%
\pgfpathlineto{\pgfqpoint{6.210000in}{2.640000in}}%
\pgfpathlineto{\pgfqpoint{0.862500in}{2.640000in}}%
\pgfpathclose%
\pgfusepath{fill}%
\end{pgfscope}%
\begin{pgfscope}%
\pgfpathrectangle{\pgfqpoint{0.862500in}{0.375000in}}{\pgfqpoint{5.347500in}{2.265000in}}%
\pgfusepath{clip}%
\pgfsetroundcap%
\pgfsetroundjoin%
\pgfsetlinewidth{0.803000pt}%
\definecolor{currentstroke}{rgb}{1.000000,1.000000,1.000000}%
\pgfsetstrokecolor{currentstroke}%
\pgfsetdash{}{0pt}%
\pgfpathmoveto{\pgfqpoint{0.862500in}{0.375000in}}%
\pgfpathlineto{\pgfqpoint{0.862500in}{2.640000in}}%
\pgfusepath{stroke}%
\end{pgfscope}%
\begin{pgfscope}%
\definecolor{textcolor}{rgb}{0.150000,0.150000,0.150000}%
\pgfsetstrokecolor{textcolor}%
\pgfsetfillcolor{textcolor}%
\pgftext[x=0.862500in,y=0.326389in,,top]{\color{textcolor}\rmfamily\fontsize{8.000000}{9.600000}\selectfont \(\displaystyle 0\)}%
\end{pgfscope}%
\begin{pgfscope}%
\pgfpathrectangle{\pgfqpoint{0.862500in}{0.375000in}}{\pgfqpoint{5.347500in}{2.265000in}}%
\pgfusepath{clip}%
\pgfsetroundcap%
\pgfsetroundjoin%
\pgfsetlinewidth{0.803000pt}%
\definecolor{currentstroke}{rgb}{1.000000,1.000000,1.000000}%
\pgfsetstrokecolor{currentstroke}%
\pgfsetdash{}{0pt}%
\pgfpathmoveto{\pgfqpoint{1.881071in}{0.375000in}}%
\pgfpathlineto{\pgfqpoint{1.881071in}{2.640000in}}%
\pgfusepath{stroke}%
\end{pgfscope}%
\begin{pgfscope}%
\definecolor{textcolor}{rgb}{0.150000,0.150000,0.150000}%
\pgfsetstrokecolor{textcolor}%
\pgfsetfillcolor{textcolor}%
\pgftext[x=1.881071in,y=0.326389in,,top]{\color{textcolor}\rmfamily\fontsize{8.000000}{9.600000}\selectfont \(\displaystyle 20\)}%
\end{pgfscope}%
\begin{pgfscope}%
\pgfpathrectangle{\pgfqpoint{0.862500in}{0.375000in}}{\pgfqpoint{5.347500in}{2.265000in}}%
\pgfusepath{clip}%
\pgfsetroundcap%
\pgfsetroundjoin%
\pgfsetlinewidth{0.803000pt}%
\definecolor{currentstroke}{rgb}{1.000000,1.000000,1.000000}%
\pgfsetstrokecolor{currentstroke}%
\pgfsetdash{}{0pt}%
\pgfpathmoveto{\pgfqpoint{2.899643in}{0.375000in}}%
\pgfpathlineto{\pgfqpoint{2.899643in}{2.640000in}}%
\pgfusepath{stroke}%
\end{pgfscope}%
\begin{pgfscope}%
\definecolor{textcolor}{rgb}{0.150000,0.150000,0.150000}%
\pgfsetstrokecolor{textcolor}%
\pgfsetfillcolor{textcolor}%
\pgftext[x=2.899643in,y=0.326389in,,top]{\color{textcolor}\rmfamily\fontsize{8.000000}{9.600000}\selectfont \(\displaystyle 40\)}%
\end{pgfscope}%
\begin{pgfscope}%
\pgfpathrectangle{\pgfqpoint{0.862500in}{0.375000in}}{\pgfqpoint{5.347500in}{2.265000in}}%
\pgfusepath{clip}%
\pgfsetroundcap%
\pgfsetroundjoin%
\pgfsetlinewidth{0.803000pt}%
\definecolor{currentstroke}{rgb}{1.000000,1.000000,1.000000}%
\pgfsetstrokecolor{currentstroke}%
\pgfsetdash{}{0pt}%
\pgfpathmoveto{\pgfqpoint{3.918214in}{0.375000in}}%
\pgfpathlineto{\pgfqpoint{3.918214in}{2.640000in}}%
\pgfusepath{stroke}%
\end{pgfscope}%
\begin{pgfscope}%
\definecolor{textcolor}{rgb}{0.150000,0.150000,0.150000}%
\pgfsetstrokecolor{textcolor}%
\pgfsetfillcolor{textcolor}%
\pgftext[x=3.918214in,y=0.326389in,,top]{\color{textcolor}\rmfamily\fontsize{8.000000}{9.600000}\selectfont \(\displaystyle 60\)}%
\end{pgfscope}%
\begin{pgfscope}%
\pgfpathrectangle{\pgfqpoint{0.862500in}{0.375000in}}{\pgfqpoint{5.347500in}{2.265000in}}%
\pgfusepath{clip}%
\pgfsetroundcap%
\pgfsetroundjoin%
\pgfsetlinewidth{0.803000pt}%
\definecolor{currentstroke}{rgb}{1.000000,1.000000,1.000000}%
\pgfsetstrokecolor{currentstroke}%
\pgfsetdash{}{0pt}%
\pgfpathmoveto{\pgfqpoint{4.936786in}{0.375000in}}%
\pgfpathlineto{\pgfqpoint{4.936786in}{2.640000in}}%
\pgfusepath{stroke}%
\end{pgfscope}%
\begin{pgfscope}%
\definecolor{textcolor}{rgb}{0.150000,0.150000,0.150000}%
\pgfsetstrokecolor{textcolor}%
\pgfsetfillcolor{textcolor}%
\pgftext[x=4.936786in,y=0.326389in,,top]{\color{textcolor}\rmfamily\fontsize{8.000000}{9.600000}\selectfont \(\displaystyle 80\)}%
\end{pgfscope}%
\begin{pgfscope}%
\pgfpathrectangle{\pgfqpoint{0.862500in}{0.375000in}}{\pgfqpoint{5.347500in}{2.265000in}}%
\pgfusepath{clip}%
\pgfsetroundcap%
\pgfsetroundjoin%
\pgfsetlinewidth{0.803000pt}%
\definecolor{currentstroke}{rgb}{1.000000,1.000000,1.000000}%
\pgfsetstrokecolor{currentstroke}%
\pgfsetdash{}{0pt}%
\pgfpathmoveto{\pgfqpoint{5.955357in}{0.375000in}}%
\pgfpathlineto{\pgfqpoint{5.955357in}{2.640000in}}%
\pgfusepath{stroke}%
\end{pgfscope}%
\begin{pgfscope}%
\definecolor{textcolor}{rgb}{0.150000,0.150000,0.150000}%
\pgfsetstrokecolor{textcolor}%
\pgfsetfillcolor{textcolor}%
\pgftext[x=5.955357in,y=0.326389in,,top]{\color{textcolor}\rmfamily\fontsize{8.000000}{9.600000}\selectfont \(\displaystyle 100\)}%
\end{pgfscope}%
\begin{pgfscope}%
\definecolor{textcolor}{rgb}{0.150000,0.150000,0.150000}%
\pgfsetstrokecolor{textcolor}%
\pgfsetfillcolor{textcolor}%
\pgftext[x=3.536250in,y=0.163303in,,top]{\color{textcolor}\rmfamily\fontsize{8.000000}{9.600000}\selectfont Step}%
\end{pgfscope}%
\begin{pgfscope}%
\pgfpathrectangle{\pgfqpoint{0.862500in}{0.375000in}}{\pgfqpoint{5.347500in}{2.265000in}}%
\pgfusepath{clip}%
\pgfsetroundcap%
\pgfsetroundjoin%
\pgfsetlinewidth{0.803000pt}%
\definecolor{currentstroke}{rgb}{1.000000,1.000000,1.000000}%
\pgfsetstrokecolor{currentstroke}%
\pgfsetdash{}{0pt}%
\pgfpathmoveto{\pgfqpoint{0.862500in}{1.050744in}}%
\pgfpathlineto{\pgfqpoint{6.210000in}{1.050744in}}%
\pgfusepath{stroke}%
\end{pgfscope}%
\begin{pgfscope}%
\definecolor{textcolor}{rgb}{0.150000,0.150000,0.150000}%
\pgfsetstrokecolor{textcolor}%
\pgfsetfillcolor{textcolor}%
\pgftext[x=0.557716in,y=1.008534in,left,base]{\color{textcolor}\rmfamily\fontsize{8.000000}{9.600000}\selectfont \(\displaystyle 10^{-1}\)}%
\end{pgfscope}%
\begin{pgfscope}%
\pgfpathrectangle{\pgfqpoint{0.862500in}{0.375000in}}{\pgfqpoint{5.347500in}{2.265000in}}%
\pgfusepath{clip}%
\pgfsetroundcap%
\pgfsetroundjoin%
\pgfsetlinewidth{0.803000pt}%
\definecolor{currentstroke}{rgb}{1.000000,1.000000,1.000000}%
\pgfsetstrokecolor{currentstroke}%
\pgfsetdash{}{0pt}%
\pgfpathmoveto{\pgfqpoint{0.862500in}{2.030590in}}%
\pgfpathlineto{\pgfqpoint{6.210000in}{2.030590in}}%
\pgfusepath{stroke}%
\end{pgfscope}%
\begin{pgfscope}%
\definecolor{textcolor}{rgb}{0.150000,0.150000,0.150000}%
\pgfsetstrokecolor{textcolor}%
\pgfsetfillcolor{textcolor}%
\pgftext[x=0.637962in,y=1.988381in,left,base]{\color{textcolor}\rmfamily\fontsize{8.000000}{9.600000}\selectfont \(\displaystyle 10^{0}\)}%
\end{pgfscope}%
\begin{pgfscope}%
\definecolor{textcolor}{rgb}{0.150000,0.150000,0.150000}%
\pgfsetstrokecolor{textcolor}%
\pgfsetfillcolor{textcolor}%
\pgftext[x=0.502160in,y=1.507500in,,bottom,rotate=90.000000]{\color{textcolor}\rmfamily\fontsize{8.000000}{9.600000}\selectfont Simple Regret}%
\end{pgfscope}%
\begin{pgfscope}%
\pgfpathrectangle{\pgfqpoint{0.862500in}{0.375000in}}{\pgfqpoint{5.347500in}{2.265000in}}%
\pgfusepath{clip}%
\pgfsetbuttcap%
\pgfsetroundjoin%
\definecolor{currentfill}{rgb}{0.121569,0.466667,0.705882}%
\pgfsetfillcolor{currentfill}%
\pgfsetfillopacity{0.200000}%
\pgfsetlinewidth{0.000000pt}%
\definecolor{currentstroke}{rgb}{0.000000,0.000000,0.000000}%
\pgfsetstrokecolor{currentstroke}%
\pgfsetdash{}{0pt}%
\pgfpathmoveto{\pgfqpoint{0.862500in}{2.482319in}}%
\pgfpathlineto{\pgfqpoint{0.862500in}{2.523964in}}%
\pgfpathlineto{\pgfqpoint{0.913429in}{2.523683in}}%
\pgfpathlineto{\pgfqpoint{0.964357in}{2.506471in}}%
\pgfpathlineto{\pgfqpoint{1.015286in}{2.506471in}}%
\pgfpathlineto{\pgfqpoint{1.066214in}{2.487944in}}%
\pgfpathlineto{\pgfqpoint{1.117143in}{2.476115in}}%
\pgfpathlineto{\pgfqpoint{1.168071in}{2.469030in}}%
\pgfpathlineto{\pgfqpoint{1.219000in}{2.464079in}}%
\pgfpathlineto{\pgfqpoint{1.269929in}{2.464079in}}%
\pgfpathlineto{\pgfqpoint{1.320857in}{2.463649in}}%
\pgfpathlineto{\pgfqpoint{1.371786in}{2.439295in}}%
\pgfpathlineto{\pgfqpoint{1.422714in}{2.429388in}}%
\pgfpathlineto{\pgfqpoint{1.473643in}{2.426525in}}%
\pgfpathlineto{\pgfqpoint{1.524571in}{2.422803in}}%
\pgfpathlineto{\pgfqpoint{1.575500in}{2.396401in}}%
\pgfpathlineto{\pgfqpoint{1.626429in}{2.388610in}}%
\pgfpathlineto{\pgfqpoint{1.677357in}{2.388610in}}%
\pgfpathlineto{\pgfqpoint{1.728286in}{2.388610in}}%
\pgfpathlineto{\pgfqpoint{1.779214in}{2.359605in}}%
\pgfpathlineto{\pgfqpoint{1.830143in}{2.359605in}}%
\pgfpathlineto{\pgfqpoint{1.881071in}{2.359605in}}%
\pgfpathlineto{\pgfqpoint{1.932000in}{2.319091in}}%
\pgfpathlineto{\pgfqpoint{1.982929in}{2.319091in}}%
\pgfpathlineto{\pgfqpoint{2.033857in}{2.319091in}}%
\pgfpathlineto{\pgfqpoint{2.084786in}{2.319091in}}%
\pgfpathlineto{\pgfqpoint{2.135714in}{2.319091in}}%
\pgfpathlineto{\pgfqpoint{2.186643in}{2.319091in}}%
\pgfpathlineto{\pgfqpoint{2.237571in}{2.319091in}}%
\pgfpathlineto{\pgfqpoint{2.288500in}{2.319091in}}%
\pgfpathlineto{\pgfqpoint{2.339429in}{2.312958in}}%
\pgfpathlineto{\pgfqpoint{2.390357in}{2.300078in}}%
\pgfpathlineto{\pgfqpoint{2.441286in}{2.300078in}}%
\pgfpathlineto{\pgfqpoint{2.492214in}{2.300078in}}%
\pgfpathlineto{\pgfqpoint{2.543143in}{2.300078in}}%
\pgfpathlineto{\pgfqpoint{2.594071in}{2.255385in}}%
\pgfpathlineto{\pgfqpoint{2.645000in}{2.255385in}}%
\pgfpathlineto{\pgfqpoint{2.695929in}{2.255385in}}%
\pgfpathlineto{\pgfqpoint{2.746857in}{2.255385in}}%
\pgfpathlineto{\pgfqpoint{2.797786in}{2.252664in}}%
\pgfpathlineto{\pgfqpoint{2.848714in}{2.252664in}}%
\pgfpathlineto{\pgfqpoint{2.899643in}{2.252664in}}%
\pgfpathlineto{\pgfqpoint{2.950571in}{2.252664in}}%
\pgfpathlineto{\pgfqpoint{3.001500in}{2.252664in}}%
\pgfpathlineto{\pgfqpoint{3.052429in}{2.252664in}}%
\pgfpathlineto{\pgfqpoint{3.103357in}{2.252664in}}%
\pgfpathlineto{\pgfqpoint{3.154286in}{2.252664in}}%
\pgfpathlineto{\pgfqpoint{3.205214in}{2.252664in}}%
\pgfpathlineto{\pgfqpoint{3.256143in}{2.225323in}}%
\pgfpathlineto{\pgfqpoint{3.307071in}{2.225323in}}%
\pgfpathlineto{\pgfqpoint{3.358000in}{2.225323in}}%
\pgfpathlineto{\pgfqpoint{3.408929in}{2.225323in}}%
\pgfpathlineto{\pgfqpoint{3.459857in}{2.225323in}}%
\pgfpathlineto{\pgfqpoint{3.510786in}{2.225323in}}%
\pgfpathlineto{\pgfqpoint{3.561714in}{2.225323in}}%
\pgfpathlineto{\pgfqpoint{3.612643in}{2.225323in}}%
\pgfpathlineto{\pgfqpoint{3.663571in}{2.225323in}}%
\pgfpathlineto{\pgfqpoint{3.714500in}{2.192484in}}%
\pgfpathlineto{\pgfqpoint{3.765429in}{2.192484in}}%
\pgfpathlineto{\pgfqpoint{3.816357in}{2.192484in}}%
\pgfpathlineto{\pgfqpoint{3.867286in}{2.192484in}}%
\pgfpathlineto{\pgfqpoint{3.918214in}{2.192484in}}%
\pgfpathlineto{\pgfqpoint{3.969143in}{2.192484in}}%
\pgfpathlineto{\pgfqpoint{4.020071in}{2.192484in}}%
\pgfpathlineto{\pgfqpoint{4.071000in}{2.192484in}}%
\pgfpathlineto{\pgfqpoint{4.121929in}{2.172418in}}%
\pgfpathlineto{\pgfqpoint{4.172857in}{2.172418in}}%
\pgfpathlineto{\pgfqpoint{4.223786in}{2.172418in}}%
\pgfpathlineto{\pgfqpoint{4.274714in}{2.172418in}}%
\pgfpathlineto{\pgfqpoint{4.325643in}{2.172418in}}%
\pgfpathlineto{\pgfqpoint{4.376571in}{2.172418in}}%
\pgfpathlineto{\pgfqpoint{4.427500in}{2.172418in}}%
\pgfpathlineto{\pgfqpoint{4.478429in}{2.172418in}}%
\pgfpathlineto{\pgfqpoint{4.529357in}{2.172418in}}%
\pgfpathlineto{\pgfqpoint{4.580286in}{2.172418in}}%
\pgfpathlineto{\pgfqpoint{4.631214in}{2.172418in}}%
\pgfpathlineto{\pgfqpoint{4.682143in}{2.172418in}}%
\pgfpathlineto{\pgfqpoint{4.733071in}{2.172418in}}%
\pgfpathlineto{\pgfqpoint{4.784000in}{2.172418in}}%
\pgfpathlineto{\pgfqpoint{4.834929in}{2.172418in}}%
\pgfpathlineto{\pgfqpoint{4.885857in}{2.172418in}}%
\pgfpathlineto{\pgfqpoint{4.936786in}{2.154736in}}%
\pgfpathlineto{\pgfqpoint{4.987714in}{2.154736in}}%
\pgfpathlineto{\pgfqpoint{5.038643in}{2.147508in}}%
\pgfpathlineto{\pgfqpoint{5.089571in}{2.147508in}}%
\pgfpathlineto{\pgfqpoint{5.140500in}{2.147508in}}%
\pgfpathlineto{\pgfqpoint{5.191429in}{2.147508in}}%
\pgfpathlineto{\pgfqpoint{5.242357in}{2.147508in}}%
\pgfpathlineto{\pgfqpoint{5.293286in}{2.147508in}}%
\pgfpathlineto{\pgfqpoint{5.344214in}{2.147508in}}%
\pgfpathlineto{\pgfqpoint{5.395143in}{2.147508in}}%
\pgfpathlineto{\pgfqpoint{5.446071in}{2.147508in}}%
\pgfpathlineto{\pgfqpoint{5.497000in}{2.147508in}}%
\pgfpathlineto{\pgfqpoint{5.547929in}{2.147508in}}%
\pgfpathlineto{\pgfqpoint{5.598857in}{2.147508in}}%
\pgfpathlineto{\pgfqpoint{5.649786in}{2.147508in}}%
\pgfpathlineto{\pgfqpoint{5.700714in}{2.147508in}}%
\pgfpathlineto{\pgfqpoint{5.751643in}{2.147508in}}%
\pgfpathlineto{\pgfqpoint{5.802571in}{2.147508in}}%
\pgfpathlineto{\pgfqpoint{5.853500in}{2.147508in}}%
\pgfpathlineto{\pgfqpoint{5.904429in}{2.147508in}}%
\pgfpathlineto{\pgfqpoint{5.955357in}{2.147508in}}%
\pgfpathlineto{\pgfqpoint{6.006286in}{2.147508in}}%
\pgfpathlineto{\pgfqpoint{6.057214in}{2.147508in}}%
\pgfpathlineto{\pgfqpoint{6.108143in}{2.147508in}}%
\pgfpathlineto{\pgfqpoint{6.159071in}{2.147508in}}%
\pgfpathlineto{\pgfqpoint{6.210000in}{2.147508in}}%
\pgfpathlineto{\pgfqpoint{6.260929in}{2.147508in}}%
\pgfpathlineto{\pgfqpoint{6.311857in}{2.147508in}}%
\pgfpathlineto{\pgfqpoint{6.362786in}{2.147508in}}%
\pgfpathlineto{\pgfqpoint{6.413714in}{2.147508in}}%
\pgfpathlineto{\pgfqpoint{6.464643in}{2.147508in}}%
\pgfpathlineto{\pgfqpoint{6.515571in}{2.147508in}}%
\pgfpathlineto{\pgfqpoint{6.566500in}{2.147508in}}%
\pgfpathlineto{\pgfqpoint{6.617429in}{2.147508in}}%
\pgfpathlineto{\pgfqpoint{6.668357in}{2.147508in}}%
\pgfpathlineto{\pgfqpoint{6.719286in}{2.147508in}}%
\pgfpathlineto{\pgfqpoint{6.770214in}{2.147508in}}%
\pgfpathlineto{\pgfqpoint{6.821143in}{2.147508in}}%
\pgfpathlineto{\pgfqpoint{6.872071in}{2.147508in}}%
\pgfpathlineto{\pgfqpoint{6.923000in}{2.147508in}}%
\pgfpathlineto{\pgfqpoint{6.973929in}{2.147508in}}%
\pgfpathlineto{\pgfqpoint{7.024857in}{2.147508in}}%
\pgfpathlineto{\pgfqpoint{7.075786in}{2.104426in}}%
\pgfpathlineto{\pgfqpoint{7.126714in}{2.104426in}}%
\pgfpathlineto{\pgfqpoint{7.177643in}{2.104426in}}%
\pgfpathlineto{\pgfqpoint{7.228571in}{2.104426in}}%
\pgfpathlineto{\pgfqpoint{7.279500in}{2.104426in}}%
\pgfpathlineto{\pgfqpoint{7.330429in}{2.104426in}}%
\pgfpathlineto{\pgfqpoint{7.381357in}{2.104426in}}%
\pgfpathlineto{\pgfqpoint{7.432286in}{2.104426in}}%
\pgfpathlineto{\pgfqpoint{7.483214in}{2.104426in}}%
\pgfpathlineto{\pgfqpoint{7.534143in}{2.104426in}}%
\pgfpathlineto{\pgfqpoint{7.585071in}{2.104426in}}%
\pgfpathlineto{\pgfqpoint{7.636000in}{2.104426in}}%
\pgfpathlineto{\pgfqpoint{7.686929in}{2.104426in}}%
\pgfpathlineto{\pgfqpoint{7.737857in}{2.104426in}}%
\pgfpathlineto{\pgfqpoint{7.788786in}{2.104426in}}%
\pgfpathlineto{\pgfqpoint{7.839714in}{2.104426in}}%
\pgfpathlineto{\pgfqpoint{7.890643in}{2.104426in}}%
\pgfpathlineto{\pgfqpoint{7.941571in}{2.104426in}}%
\pgfpathlineto{\pgfqpoint{7.992500in}{2.104426in}}%
\pgfpathlineto{\pgfqpoint{8.043429in}{2.104426in}}%
\pgfpathlineto{\pgfqpoint{8.094357in}{2.104426in}}%
\pgfpathlineto{\pgfqpoint{8.145286in}{2.100982in}}%
\pgfpathlineto{\pgfqpoint{8.196214in}{2.100982in}}%
\pgfpathlineto{\pgfqpoint{8.247143in}{2.100982in}}%
\pgfpathlineto{\pgfqpoint{8.298071in}{2.097254in}}%
\pgfpathlineto{\pgfqpoint{8.349000in}{2.097254in}}%
\pgfpathlineto{\pgfqpoint{8.399929in}{2.097254in}}%
\pgfpathlineto{\pgfqpoint{8.450857in}{2.097254in}}%
\pgfpathlineto{\pgfqpoint{8.501786in}{2.097254in}}%
\pgfpathlineto{\pgfqpoint{8.552714in}{2.097254in}}%
\pgfpathlineto{\pgfqpoint{8.603643in}{2.097254in}}%
\pgfpathlineto{\pgfqpoint{8.654571in}{2.097254in}}%
\pgfpathlineto{\pgfqpoint{8.705500in}{2.097254in}}%
\pgfpathlineto{\pgfqpoint{8.756429in}{2.097254in}}%
\pgfpathlineto{\pgfqpoint{8.807357in}{2.097254in}}%
\pgfpathlineto{\pgfqpoint{8.858286in}{2.097254in}}%
\pgfpathlineto{\pgfqpoint{8.909214in}{2.097254in}}%
\pgfpathlineto{\pgfqpoint{8.960143in}{2.097254in}}%
\pgfpathlineto{\pgfqpoint{9.011071in}{2.097254in}}%
\pgfpathlineto{\pgfqpoint{9.062000in}{2.097254in}}%
\pgfpathlineto{\pgfqpoint{9.112929in}{2.097254in}}%
\pgfpathlineto{\pgfqpoint{9.163857in}{2.097254in}}%
\pgfpathlineto{\pgfqpoint{9.214786in}{2.097254in}}%
\pgfpathlineto{\pgfqpoint{9.265714in}{2.097254in}}%
\pgfpathlineto{\pgfqpoint{9.316643in}{2.097254in}}%
\pgfpathlineto{\pgfqpoint{9.367571in}{2.097254in}}%
\pgfpathlineto{\pgfqpoint{9.418500in}{2.097254in}}%
\pgfpathlineto{\pgfqpoint{9.469429in}{2.097254in}}%
\pgfpathlineto{\pgfqpoint{9.469429in}{1.978616in}}%
\pgfpathlineto{\pgfqpoint{9.469429in}{1.978616in}}%
\pgfpathlineto{\pgfqpoint{9.418500in}{1.978616in}}%
\pgfpathlineto{\pgfqpoint{9.367571in}{1.978616in}}%
\pgfpathlineto{\pgfqpoint{9.316643in}{1.978616in}}%
\pgfpathlineto{\pgfqpoint{9.265714in}{1.978616in}}%
\pgfpathlineto{\pgfqpoint{9.214786in}{1.978616in}}%
\pgfpathlineto{\pgfqpoint{9.163857in}{1.978616in}}%
\pgfpathlineto{\pgfqpoint{9.112929in}{1.978616in}}%
\pgfpathlineto{\pgfqpoint{9.062000in}{1.978616in}}%
\pgfpathlineto{\pgfqpoint{9.011071in}{1.978616in}}%
\pgfpathlineto{\pgfqpoint{8.960143in}{1.978616in}}%
\pgfpathlineto{\pgfqpoint{8.909214in}{1.978616in}}%
\pgfpathlineto{\pgfqpoint{8.858286in}{1.978616in}}%
\pgfpathlineto{\pgfqpoint{8.807357in}{1.978616in}}%
\pgfpathlineto{\pgfqpoint{8.756429in}{1.978616in}}%
\pgfpathlineto{\pgfqpoint{8.705500in}{1.978616in}}%
\pgfpathlineto{\pgfqpoint{8.654571in}{1.978616in}}%
\pgfpathlineto{\pgfqpoint{8.603643in}{1.978616in}}%
\pgfpathlineto{\pgfqpoint{8.552714in}{1.978616in}}%
\pgfpathlineto{\pgfqpoint{8.501786in}{1.978616in}}%
\pgfpathlineto{\pgfqpoint{8.450857in}{1.978616in}}%
\pgfpathlineto{\pgfqpoint{8.399929in}{1.978616in}}%
\pgfpathlineto{\pgfqpoint{8.349000in}{1.978616in}}%
\pgfpathlineto{\pgfqpoint{8.298071in}{1.978616in}}%
\pgfpathlineto{\pgfqpoint{8.247143in}{1.980610in}}%
\pgfpathlineto{\pgfqpoint{8.196214in}{1.980610in}}%
\pgfpathlineto{\pgfqpoint{8.145286in}{1.980610in}}%
\pgfpathlineto{\pgfqpoint{8.094357in}{2.002566in}}%
\pgfpathlineto{\pgfqpoint{8.043429in}{2.002566in}}%
\pgfpathlineto{\pgfqpoint{7.992500in}{2.002566in}}%
\pgfpathlineto{\pgfqpoint{7.941571in}{2.002566in}}%
\pgfpathlineto{\pgfqpoint{7.890643in}{2.002566in}}%
\pgfpathlineto{\pgfqpoint{7.839714in}{2.002566in}}%
\pgfpathlineto{\pgfqpoint{7.788786in}{2.002566in}}%
\pgfpathlineto{\pgfqpoint{7.737857in}{2.002566in}}%
\pgfpathlineto{\pgfqpoint{7.686929in}{2.002566in}}%
\pgfpathlineto{\pgfqpoint{7.636000in}{2.002566in}}%
\pgfpathlineto{\pgfqpoint{7.585071in}{2.002566in}}%
\pgfpathlineto{\pgfqpoint{7.534143in}{2.002566in}}%
\pgfpathlineto{\pgfqpoint{7.483214in}{2.002566in}}%
\pgfpathlineto{\pgfqpoint{7.432286in}{2.002566in}}%
\pgfpathlineto{\pgfqpoint{7.381357in}{2.002566in}}%
\pgfpathlineto{\pgfqpoint{7.330429in}{2.002566in}}%
\pgfpathlineto{\pgfqpoint{7.279500in}{2.002566in}}%
\pgfpathlineto{\pgfqpoint{7.228571in}{2.002566in}}%
\pgfpathlineto{\pgfqpoint{7.177643in}{2.002566in}}%
\pgfpathlineto{\pgfqpoint{7.126714in}{2.002566in}}%
\pgfpathlineto{\pgfqpoint{7.075786in}{2.002566in}}%
\pgfpathlineto{\pgfqpoint{7.024857in}{2.013369in}}%
\pgfpathlineto{\pgfqpoint{6.973929in}{2.013369in}}%
\pgfpathlineto{\pgfqpoint{6.923000in}{2.013369in}}%
\pgfpathlineto{\pgfqpoint{6.872071in}{2.013369in}}%
\pgfpathlineto{\pgfqpoint{6.821143in}{2.013369in}}%
\pgfpathlineto{\pgfqpoint{6.770214in}{2.013369in}}%
\pgfpathlineto{\pgfqpoint{6.719286in}{2.013369in}}%
\pgfpathlineto{\pgfqpoint{6.668357in}{2.013369in}}%
\pgfpathlineto{\pgfqpoint{6.617429in}{2.013369in}}%
\pgfpathlineto{\pgfqpoint{6.566500in}{2.013369in}}%
\pgfpathlineto{\pgfqpoint{6.515571in}{2.013369in}}%
\pgfpathlineto{\pgfqpoint{6.464643in}{2.013369in}}%
\pgfpathlineto{\pgfqpoint{6.413714in}{2.013369in}}%
\pgfpathlineto{\pgfqpoint{6.362786in}{2.013369in}}%
\pgfpathlineto{\pgfqpoint{6.311857in}{2.013369in}}%
\pgfpathlineto{\pgfqpoint{6.260929in}{2.013369in}}%
\pgfpathlineto{\pgfqpoint{6.210000in}{2.013369in}}%
\pgfpathlineto{\pgfqpoint{6.159071in}{2.013369in}}%
\pgfpathlineto{\pgfqpoint{6.108143in}{2.013369in}}%
\pgfpathlineto{\pgfqpoint{6.057214in}{2.013369in}}%
\pgfpathlineto{\pgfqpoint{6.006286in}{2.013369in}}%
\pgfpathlineto{\pgfqpoint{5.955357in}{2.013369in}}%
\pgfpathlineto{\pgfqpoint{5.904429in}{2.013369in}}%
\pgfpathlineto{\pgfqpoint{5.853500in}{2.013369in}}%
\pgfpathlineto{\pgfqpoint{5.802571in}{2.013369in}}%
\pgfpathlineto{\pgfqpoint{5.751643in}{2.013369in}}%
\pgfpathlineto{\pgfqpoint{5.700714in}{2.013369in}}%
\pgfpathlineto{\pgfqpoint{5.649786in}{2.013369in}}%
\pgfpathlineto{\pgfqpoint{5.598857in}{2.013369in}}%
\pgfpathlineto{\pgfqpoint{5.547929in}{2.013369in}}%
\pgfpathlineto{\pgfqpoint{5.497000in}{2.013369in}}%
\pgfpathlineto{\pgfqpoint{5.446071in}{2.013369in}}%
\pgfpathlineto{\pgfqpoint{5.395143in}{2.013369in}}%
\pgfpathlineto{\pgfqpoint{5.344214in}{2.013369in}}%
\pgfpathlineto{\pgfqpoint{5.293286in}{2.013369in}}%
\pgfpathlineto{\pgfqpoint{5.242357in}{2.013369in}}%
\pgfpathlineto{\pgfqpoint{5.191429in}{2.013369in}}%
\pgfpathlineto{\pgfqpoint{5.140500in}{2.013369in}}%
\pgfpathlineto{\pgfqpoint{5.089571in}{2.013369in}}%
\pgfpathlineto{\pgfqpoint{5.038643in}{2.013369in}}%
\pgfpathlineto{\pgfqpoint{4.987714in}{2.012932in}}%
\pgfpathlineto{\pgfqpoint{4.936786in}{2.012932in}}%
\pgfpathlineto{\pgfqpoint{4.885857in}{2.018815in}}%
\pgfpathlineto{\pgfqpoint{4.834929in}{2.018815in}}%
\pgfpathlineto{\pgfqpoint{4.784000in}{2.018815in}}%
\pgfpathlineto{\pgfqpoint{4.733071in}{2.018815in}}%
\pgfpathlineto{\pgfqpoint{4.682143in}{2.018815in}}%
\pgfpathlineto{\pgfqpoint{4.631214in}{2.018815in}}%
\pgfpathlineto{\pgfqpoint{4.580286in}{2.018815in}}%
\pgfpathlineto{\pgfqpoint{4.529357in}{2.018815in}}%
\pgfpathlineto{\pgfqpoint{4.478429in}{2.018815in}}%
\pgfpathlineto{\pgfqpoint{4.427500in}{2.018815in}}%
\pgfpathlineto{\pgfqpoint{4.376571in}{2.018815in}}%
\pgfpathlineto{\pgfqpoint{4.325643in}{2.018815in}}%
\pgfpathlineto{\pgfqpoint{4.274714in}{2.018815in}}%
\pgfpathlineto{\pgfqpoint{4.223786in}{2.018815in}}%
\pgfpathlineto{\pgfqpoint{4.172857in}{2.018815in}}%
\pgfpathlineto{\pgfqpoint{4.121929in}{2.018815in}}%
\pgfpathlineto{\pgfqpoint{4.071000in}{2.032069in}}%
\pgfpathlineto{\pgfqpoint{4.020071in}{2.032069in}}%
\pgfpathlineto{\pgfqpoint{3.969143in}{2.032069in}}%
\pgfpathlineto{\pgfqpoint{3.918214in}{2.032069in}}%
\pgfpathlineto{\pgfqpoint{3.867286in}{2.032069in}}%
\pgfpathlineto{\pgfqpoint{3.816357in}{2.032069in}}%
\pgfpathlineto{\pgfqpoint{3.765429in}{2.032069in}}%
\pgfpathlineto{\pgfqpoint{3.714500in}{2.032069in}}%
\pgfpathlineto{\pgfqpoint{3.663571in}{2.071856in}}%
\pgfpathlineto{\pgfqpoint{3.612643in}{2.071856in}}%
\pgfpathlineto{\pgfqpoint{3.561714in}{2.071856in}}%
\pgfpathlineto{\pgfqpoint{3.510786in}{2.071856in}}%
\pgfpathlineto{\pgfqpoint{3.459857in}{2.071856in}}%
\pgfpathlineto{\pgfqpoint{3.408929in}{2.071856in}}%
\pgfpathlineto{\pgfqpoint{3.358000in}{2.071856in}}%
\pgfpathlineto{\pgfqpoint{3.307071in}{2.071856in}}%
\pgfpathlineto{\pgfqpoint{3.256143in}{2.071856in}}%
\pgfpathlineto{\pgfqpoint{3.205214in}{2.092050in}}%
\pgfpathlineto{\pgfqpoint{3.154286in}{2.092050in}}%
\pgfpathlineto{\pgfqpoint{3.103357in}{2.092050in}}%
\pgfpathlineto{\pgfqpoint{3.052429in}{2.092050in}}%
\pgfpathlineto{\pgfqpoint{3.001500in}{2.092050in}}%
\pgfpathlineto{\pgfqpoint{2.950571in}{2.092050in}}%
\pgfpathlineto{\pgfqpoint{2.899643in}{2.092050in}}%
\pgfpathlineto{\pgfqpoint{2.848714in}{2.092050in}}%
\pgfpathlineto{\pgfqpoint{2.797786in}{2.092050in}}%
\pgfpathlineto{\pgfqpoint{2.746857in}{2.093861in}}%
\pgfpathlineto{\pgfqpoint{2.695929in}{2.093861in}}%
\pgfpathlineto{\pgfqpoint{2.645000in}{2.093861in}}%
\pgfpathlineto{\pgfqpoint{2.594071in}{2.093861in}}%
\pgfpathlineto{\pgfqpoint{2.543143in}{2.133841in}}%
\pgfpathlineto{\pgfqpoint{2.492214in}{2.133841in}}%
\pgfpathlineto{\pgfqpoint{2.441286in}{2.133841in}}%
\pgfpathlineto{\pgfqpoint{2.390357in}{2.133841in}}%
\pgfpathlineto{\pgfqpoint{2.339429in}{2.142489in}}%
\pgfpathlineto{\pgfqpoint{2.288500in}{2.146069in}}%
\pgfpathlineto{\pgfqpoint{2.237571in}{2.146069in}}%
\pgfpathlineto{\pgfqpoint{2.186643in}{2.146069in}}%
\pgfpathlineto{\pgfqpoint{2.135714in}{2.146069in}}%
\pgfpathlineto{\pgfqpoint{2.084786in}{2.146069in}}%
\pgfpathlineto{\pgfqpoint{2.033857in}{2.146069in}}%
\pgfpathlineto{\pgfqpoint{1.982929in}{2.146069in}}%
\pgfpathlineto{\pgfqpoint{1.932000in}{2.146069in}}%
\pgfpathlineto{\pgfqpoint{1.881071in}{2.165053in}}%
\pgfpathlineto{\pgfqpoint{1.830143in}{2.165053in}}%
\pgfpathlineto{\pgfqpoint{1.779214in}{2.165053in}}%
\pgfpathlineto{\pgfqpoint{1.728286in}{2.219864in}}%
\pgfpathlineto{\pgfqpoint{1.677357in}{2.219864in}}%
\pgfpathlineto{\pgfqpoint{1.626429in}{2.219864in}}%
\pgfpathlineto{\pgfqpoint{1.575500in}{2.225877in}}%
\pgfpathlineto{\pgfqpoint{1.524571in}{2.274635in}}%
\pgfpathlineto{\pgfqpoint{1.473643in}{2.277354in}}%
\pgfpathlineto{\pgfqpoint{1.422714in}{2.320845in}}%
\pgfpathlineto{\pgfqpoint{1.371786in}{2.340979in}}%
\pgfpathlineto{\pgfqpoint{1.320857in}{2.366357in}}%
\pgfpathlineto{\pgfqpoint{1.269929in}{2.366498in}}%
\pgfpathlineto{\pgfqpoint{1.219000in}{2.366498in}}%
\pgfpathlineto{\pgfqpoint{1.168071in}{2.377022in}}%
\pgfpathlineto{\pgfqpoint{1.117143in}{2.430198in}}%
\pgfpathlineto{\pgfqpoint{1.066214in}{2.440456in}}%
\pgfpathlineto{\pgfqpoint{1.015286in}{2.457097in}}%
\pgfpathlineto{\pgfqpoint{0.964357in}{2.457097in}}%
\pgfpathlineto{\pgfqpoint{0.913429in}{2.482204in}}%
\pgfpathlineto{\pgfqpoint{0.862500in}{2.482319in}}%
\pgfpathclose%
\pgfusepath{fill}%
\end{pgfscope}%
\begin{pgfscope}%
\pgfpathrectangle{\pgfqpoint{0.862500in}{0.375000in}}{\pgfqpoint{5.347500in}{2.265000in}}%
\pgfusepath{clip}%
\pgfsetbuttcap%
\pgfsetroundjoin%
\definecolor{currentfill}{rgb}{1.000000,0.498039,0.054902}%
\pgfsetfillcolor{currentfill}%
\pgfsetfillopacity{0.200000}%
\pgfsetlinewidth{0.000000pt}%
\definecolor{currentstroke}{rgb}{0.000000,0.000000,0.000000}%
\pgfsetstrokecolor{currentstroke}%
\pgfsetdash{}{0pt}%
\pgfpathmoveto{\pgfqpoint{0.862500in}{2.517079in}}%
\pgfpathlineto{\pgfqpoint{0.862500in}{2.532092in}}%
\pgfpathlineto{\pgfqpoint{0.913429in}{2.532092in}}%
\pgfpathlineto{\pgfqpoint{0.964357in}{2.508457in}}%
\pgfpathlineto{\pgfqpoint{1.015286in}{2.502136in}}%
\pgfpathlineto{\pgfqpoint{1.066214in}{2.460301in}}%
\pgfpathlineto{\pgfqpoint{1.117143in}{2.425319in}}%
\pgfpathlineto{\pgfqpoint{1.168071in}{2.425319in}}%
\pgfpathlineto{\pgfqpoint{1.219000in}{2.425319in}}%
\pgfpathlineto{\pgfqpoint{1.269929in}{2.420588in}}%
\pgfpathlineto{\pgfqpoint{1.320857in}{2.420588in}}%
\pgfpathlineto{\pgfqpoint{1.371786in}{2.420588in}}%
\pgfpathlineto{\pgfqpoint{1.422714in}{2.420588in}}%
\pgfpathlineto{\pgfqpoint{1.473643in}{2.420588in}}%
\pgfpathlineto{\pgfqpoint{1.524571in}{2.410422in}}%
\pgfpathlineto{\pgfqpoint{1.575500in}{2.378821in}}%
\pgfpathlineto{\pgfqpoint{1.626429in}{2.378821in}}%
\pgfpathlineto{\pgfqpoint{1.677357in}{2.314381in}}%
\pgfpathlineto{\pgfqpoint{1.728286in}{2.311752in}}%
\pgfpathlineto{\pgfqpoint{1.779214in}{2.311752in}}%
\pgfpathlineto{\pgfqpoint{1.830143in}{2.311752in}}%
\pgfpathlineto{\pgfqpoint{1.881071in}{2.311752in}}%
\pgfpathlineto{\pgfqpoint{1.932000in}{2.311752in}}%
\pgfpathlineto{\pgfqpoint{1.982929in}{2.307528in}}%
\pgfpathlineto{\pgfqpoint{2.033857in}{2.307528in}}%
\pgfpathlineto{\pgfqpoint{2.084786in}{2.307528in}}%
\pgfpathlineto{\pgfqpoint{2.135714in}{2.307528in}}%
\pgfpathlineto{\pgfqpoint{2.186643in}{2.240815in}}%
\pgfpathlineto{\pgfqpoint{2.237571in}{2.240630in}}%
\pgfpathlineto{\pgfqpoint{2.288500in}{2.240630in}}%
\pgfpathlineto{\pgfqpoint{2.339429in}{2.240630in}}%
\pgfpathlineto{\pgfqpoint{2.390357in}{2.192336in}}%
\pgfpathlineto{\pgfqpoint{2.441286in}{2.192336in}}%
\pgfpathlineto{\pgfqpoint{2.492214in}{2.192336in}}%
\pgfpathlineto{\pgfqpoint{2.543143in}{2.189251in}}%
\pgfpathlineto{\pgfqpoint{2.594071in}{2.158556in}}%
\pgfpathlineto{\pgfqpoint{2.645000in}{2.094269in}}%
\pgfpathlineto{\pgfqpoint{2.695929in}{2.077824in}}%
\pgfpathlineto{\pgfqpoint{2.746857in}{2.024830in}}%
\pgfpathlineto{\pgfqpoint{2.797786in}{2.024830in}}%
\pgfpathlineto{\pgfqpoint{2.848714in}{1.953779in}}%
\pgfpathlineto{\pgfqpoint{2.899643in}{1.953779in}}%
\pgfpathlineto{\pgfqpoint{2.950571in}{1.921815in}}%
\pgfpathlineto{\pgfqpoint{3.001500in}{1.921815in}}%
\pgfpathlineto{\pgfqpoint{3.052429in}{1.903379in}}%
\pgfpathlineto{\pgfqpoint{3.103357in}{1.844416in}}%
\pgfpathlineto{\pgfqpoint{3.154286in}{1.760218in}}%
\pgfpathlineto{\pgfqpoint{3.205214in}{1.760218in}}%
\pgfpathlineto{\pgfqpoint{3.256143in}{1.756914in}}%
\pgfpathlineto{\pgfqpoint{3.307071in}{1.756914in}}%
\pgfpathlineto{\pgfqpoint{3.358000in}{1.756914in}}%
\pgfpathlineto{\pgfqpoint{3.408929in}{1.754706in}}%
\pgfpathlineto{\pgfqpoint{3.459857in}{1.754706in}}%
\pgfpathlineto{\pgfqpoint{3.510786in}{1.724196in}}%
\pgfpathlineto{\pgfqpoint{3.561714in}{1.720300in}}%
\pgfpathlineto{\pgfqpoint{3.612643in}{1.717873in}}%
\pgfpathlineto{\pgfqpoint{3.663571in}{1.717873in}}%
\pgfpathlineto{\pgfqpoint{3.714500in}{1.709167in}}%
\pgfpathlineto{\pgfqpoint{3.765429in}{1.622219in}}%
\pgfpathlineto{\pgfqpoint{3.816357in}{1.560485in}}%
\pgfpathlineto{\pgfqpoint{3.867286in}{1.560485in}}%
\pgfpathlineto{\pgfqpoint{3.918214in}{1.557420in}}%
\pgfpathlineto{\pgfqpoint{3.969143in}{1.501855in}}%
\pgfpathlineto{\pgfqpoint{4.020071in}{1.415777in}}%
\pgfpathlineto{\pgfqpoint{4.071000in}{1.411250in}}%
\pgfpathlineto{\pgfqpoint{4.121929in}{1.411250in}}%
\pgfpathlineto{\pgfqpoint{4.172857in}{1.371847in}}%
\pgfpathlineto{\pgfqpoint{4.223786in}{1.371847in}}%
\pgfpathlineto{\pgfqpoint{4.274714in}{1.371847in}}%
\pgfpathlineto{\pgfqpoint{4.325643in}{1.337868in}}%
\pgfpathlineto{\pgfqpoint{4.376571in}{1.331264in}}%
\pgfpathlineto{\pgfqpoint{4.427500in}{1.331264in}}%
\pgfpathlineto{\pgfqpoint{4.478429in}{1.309663in}}%
\pgfpathlineto{\pgfqpoint{4.529357in}{1.309663in}}%
\pgfpathlineto{\pgfqpoint{4.580286in}{1.309663in}}%
\pgfpathlineto{\pgfqpoint{4.631214in}{1.309663in}}%
\pgfpathlineto{\pgfqpoint{4.682143in}{1.281559in}}%
\pgfpathlineto{\pgfqpoint{4.733071in}{1.281559in}}%
\pgfpathlineto{\pgfqpoint{4.784000in}{1.281559in}}%
\pgfpathlineto{\pgfqpoint{4.834929in}{1.281559in}}%
\pgfpathlineto{\pgfqpoint{4.885857in}{1.272732in}}%
\pgfpathlineto{\pgfqpoint{4.936786in}{1.268534in}}%
\pgfpathlineto{\pgfqpoint{4.987714in}{1.268534in}}%
\pgfpathlineto{\pgfqpoint{5.038643in}{1.267178in}}%
\pgfpathlineto{\pgfqpoint{5.089571in}{1.267178in}}%
\pgfpathlineto{\pgfqpoint{5.140500in}{1.210576in}}%
\pgfpathlineto{\pgfqpoint{5.191429in}{1.210576in}}%
\pgfpathlineto{\pgfqpoint{5.242357in}{1.210576in}}%
\pgfpathlineto{\pgfqpoint{5.293286in}{1.210576in}}%
\pgfpathlineto{\pgfqpoint{5.344214in}{1.210576in}}%
\pgfpathlineto{\pgfqpoint{5.395143in}{1.195156in}}%
\pgfpathlineto{\pgfqpoint{5.446071in}{1.195156in}}%
\pgfpathlineto{\pgfqpoint{5.497000in}{1.190353in}}%
\pgfpathlineto{\pgfqpoint{5.547929in}{1.167388in}}%
\pgfpathlineto{\pgfqpoint{5.598857in}{1.156475in}}%
\pgfpathlineto{\pgfqpoint{5.649786in}{1.156475in}}%
\pgfpathlineto{\pgfqpoint{5.700714in}{1.156475in}}%
\pgfpathlineto{\pgfqpoint{5.751643in}{1.156475in}}%
\pgfpathlineto{\pgfqpoint{5.802571in}{1.150896in}}%
\pgfpathlineto{\pgfqpoint{5.853500in}{1.150896in}}%
\pgfpathlineto{\pgfqpoint{5.904429in}{1.150896in}}%
\pgfpathlineto{\pgfqpoint{5.955357in}{1.150896in}}%
\pgfpathlineto{\pgfqpoint{6.006286in}{1.150896in}}%
\pgfpathlineto{\pgfqpoint{6.057214in}{1.150896in}}%
\pgfpathlineto{\pgfqpoint{6.108143in}{1.150896in}}%
\pgfpathlineto{\pgfqpoint{6.159071in}{1.150896in}}%
\pgfpathlineto{\pgfqpoint{6.210000in}{1.149000in}}%
\pgfpathlineto{\pgfqpoint{6.260929in}{1.132188in}}%
\pgfpathlineto{\pgfqpoint{6.311857in}{1.132188in}}%
\pgfpathlineto{\pgfqpoint{6.362786in}{1.131625in}}%
\pgfpathlineto{\pgfqpoint{6.413714in}{1.131625in}}%
\pgfpathlineto{\pgfqpoint{6.464643in}{1.131625in}}%
\pgfpathlineto{\pgfqpoint{6.515571in}{1.131625in}}%
\pgfpathlineto{\pgfqpoint{6.566500in}{1.131625in}}%
\pgfpathlineto{\pgfqpoint{6.617429in}{1.131625in}}%
\pgfpathlineto{\pgfqpoint{6.668357in}{1.090699in}}%
\pgfpathlineto{\pgfqpoint{6.719286in}{1.090699in}}%
\pgfpathlineto{\pgfqpoint{6.770214in}{1.090699in}}%
\pgfpathlineto{\pgfqpoint{6.821143in}{1.088059in}}%
\pgfpathlineto{\pgfqpoint{6.872071in}{1.088059in}}%
\pgfpathlineto{\pgfqpoint{6.923000in}{1.079737in}}%
\pgfpathlineto{\pgfqpoint{6.973929in}{1.079737in}}%
\pgfpathlineto{\pgfqpoint{7.024857in}{1.079737in}}%
\pgfpathlineto{\pgfqpoint{7.075786in}{1.079737in}}%
\pgfpathlineto{\pgfqpoint{7.126714in}{1.079737in}}%
\pgfpathlineto{\pgfqpoint{7.177643in}{1.079737in}}%
\pgfpathlineto{\pgfqpoint{7.228571in}{1.079737in}}%
\pgfpathlineto{\pgfqpoint{7.279500in}{1.079737in}}%
\pgfpathlineto{\pgfqpoint{7.330429in}{1.079737in}}%
\pgfpathlineto{\pgfqpoint{7.381357in}{1.078305in}}%
\pgfpathlineto{\pgfqpoint{7.432286in}{1.078305in}}%
\pgfpathlineto{\pgfqpoint{7.483214in}{1.078305in}}%
\pgfpathlineto{\pgfqpoint{7.534143in}{1.078305in}}%
\pgfpathlineto{\pgfqpoint{7.585071in}{1.078305in}}%
\pgfpathlineto{\pgfqpoint{7.636000in}{1.078305in}}%
\pgfpathlineto{\pgfqpoint{7.686929in}{1.078305in}}%
\pgfpathlineto{\pgfqpoint{7.737857in}{1.078298in}}%
\pgfpathlineto{\pgfqpoint{7.788786in}{1.078298in}}%
\pgfpathlineto{\pgfqpoint{7.839714in}{1.078298in}}%
\pgfpathlineto{\pgfqpoint{7.890643in}{1.066886in}}%
\pgfpathlineto{\pgfqpoint{7.941571in}{1.066886in}}%
\pgfpathlineto{\pgfqpoint{7.992500in}{1.066886in}}%
\pgfpathlineto{\pgfqpoint{8.043429in}{1.066886in}}%
\pgfpathlineto{\pgfqpoint{8.094357in}{1.066886in}}%
\pgfpathlineto{\pgfqpoint{8.145286in}{1.066886in}}%
\pgfpathlineto{\pgfqpoint{8.196214in}{1.066886in}}%
\pgfpathlineto{\pgfqpoint{8.247143in}{1.066886in}}%
\pgfpathlineto{\pgfqpoint{8.298071in}{1.044474in}}%
\pgfpathlineto{\pgfqpoint{8.349000in}{1.044474in}}%
\pgfpathlineto{\pgfqpoint{8.399929in}{1.038105in}}%
\pgfpathlineto{\pgfqpoint{8.450857in}{1.038105in}}%
\pgfpathlineto{\pgfqpoint{8.501786in}{1.038105in}}%
\pgfpathlineto{\pgfqpoint{8.552714in}{1.038105in}}%
\pgfpathlineto{\pgfqpoint{8.603643in}{1.037900in}}%
\pgfpathlineto{\pgfqpoint{8.654571in}{1.037900in}}%
\pgfpathlineto{\pgfqpoint{8.705500in}{1.032749in}}%
\pgfpathlineto{\pgfqpoint{8.756429in}{1.032749in}}%
\pgfpathlineto{\pgfqpoint{8.807357in}{1.032749in}}%
\pgfpathlineto{\pgfqpoint{8.858286in}{1.032749in}}%
\pgfpathlineto{\pgfqpoint{8.909214in}{1.032749in}}%
\pgfpathlineto{\pgfqpoint{8.960143in}{0.984389in}}%
\pgfpathlineto{\pgfqpoint{9.011071in}{0.984389in}}%
\pgfpathlineto{\pgfqpoint{9.062000in}{0.984389in}}%
\pgfpathlineto{\pgfqpoint{9.112929in}{0.984389in}}%
\pgfpathlineto{\pgfqpoint{9.163857in}{0.984389in}}%
\pgfpathlineto{\pgfqpoint{9.214786in}{0.984389in}}%
\pgfpathlineto{\pgfqpoint{9.265714in}{0.984389in}}%
\pgfpathlineto{\pgfqpoint{9.316643in}{0.984389in}}%
\pgfpathlineto{\pgfqpoint{9.367571in}{0.984389in}}%
\pgfpathlineto{\pgfqpoint{9.418500in}{0.983437in}}%
\pgfpathlineto{\pgfqpoint{9.469429in}{0.983437in}}%
\pgfpathlineto{\pgfqpoint{9.469429in}{0.477955in}}%
\pgfpathlineto{\pgfqpoint{9.469429in}{0.477955in}}%
\pgfpathlineto{\pgfqpoint{9.418500in}{0.477955in}}%
\pgfpathlineto{\pgfqpoint{9.367571in}{0.485084in}}%
\pgfpathlineto{\pgfqpoint{9.316643in}{0.485084in}}%
\pgfpathlineto{\pgfqpoint{9.265714in}{0.485084in}}%
\pgfpathlineto{\pgfqpoint{9.214786in}{0.485084in}}%
\pgfpathlineto{\pgfqpoint{9.163857in}{0.485084in}}%
\pgfpathlineto{\pgfqpoint{9.112929in}{0.485084in}}%
\pgfpathlineto{\pgfqpoint{9.062000in}{0.485084in}}%
\pgfpathlineto{\pgfqpoint{9.011071in}{0.485084in}}%
\pgfpathlineto{\pgfqpoint{8.960143in}{0.485084in}}%
\pgfpathlineto{\pgfqpoint{8.909214in}{0.522858in}}%
\pgfpathlineto{\pgfqpoint{8.858286in}{0.522858in}}%
\pgfpathlineto{\pgfqpoint{8.807357in}{0.522858in}}%
\pgfpathlineto{\pgfqpoint{8.756429in}{0.522858in}}%
\pgfpathlineto{\pgfqpoint{8.705500in}{0.522858in}}%
\pgfpathlineto{\pgfqpoint{8.654571in}{0.559671in}}%
\pgfpathlineto{\pgfqpoint{8.603643in}{0.559671in}}%
\pgfpathlineto{\pgfqpoint{8.552714in}{0.561007in}}%
\pgfpathlineto{\pgfqpoint{8.501786in}{0.561007in}}%
\pgfpathlineto{\pgfqpoint{8.450857in}{0.561007in}}%
\pgfpathlineto{\pgfqpoint{8.399929in}{0.561007in}}%
\pgfpathlineto{\pgfqpoint{8.349000in}{0.602529in}}%
\pgfpathlineto{\pgfqpoint{8.298071in}{0.602529in}}%
\pgfpathlineto{\pgfqpoint{8.247143in}{0.614759in}}%
\pgfpathlineto{\pgfqpoint{8.196214in}{0.614759in}}%
\pgfpathlineto{\pgfqpoint{8.145286in}{0.614759in}}%
\pgfpathlineto{\pgfqpoint{8.094357in}{0.614759in}}%
\pgfpathlineto{\pgfqpoint{8.043429in}{0.614759in}}%
\pgfpathlineto{\pgfqpoint{7.992500in}{0.614759in}}%
\pgfpathlineto{\pgfqpoint{7.941571in}{0.614759in}}%
\pgfpathlineto{\pgfqpoint{7.890643in}{0.614759in}}%
\pgfpathlineto{\pgfqpoint{7.839714in}{0.628069in}}%
\pgfpathlineto{\pgfqpoint{7.788786in}{0.628069in}}%
\pgfpathlineto{\pgfqpoint{7.737857in}{0.628069in}}%
\pgfpathlineto{\pgfqpoint{7.686929in}{0.628113in}}%
\pgfpathlineto{\pgfqpoint{7.636000in}{0.628113in}}%
\pgfpathlineto{\pgfqpoint{7.585071in}{0.628113in}}%
\pgfpathlineto{\pgfqpoint{7.534143in}{0.628113in}}%
\pgfpathlineto{\pgfqpoint{7.483214in}{0.628113in}}%
\pgfpathlineto{\pgfqpoint{7.432286in}{0.628113in}}%
\pgfpathlineto{\pgfqpoint{7.381357in}{0.628113in}}%
\pgfpathlineto{\pgfqpoint{7.330429in}{0.628930in}}%
\pgfpathlineto{\pgfqpoint{7.279500in}{0.628930in}}%
\pgfpathlineto{\pgfqpoint{7.228571in}{0.628930in}}%
\pgfpathlineto{\pgfqpoint{7.177643in}{0.628930in}}%
\pgfpathlineto{\pgfqpoint{7.126714in}{0.628930in}}%
\pgfpathlineto{\pgfqpoint{7.075786in}{0.628930in}}%
\pgfpathlineto{\pgfqpoint{7.024857in}{0.628930in}}%
\pgfpathlineto{\pgfqpoint{6.973929in}{0.628930in}}%
\pgfpathlineto{\pgfqpoint{6.923000in}{0.628930in}}%
\pgfpathlineto{\pgfqpoint{6.872071in}{0.637945in}}%
\pgfpathlineto{\pgfqpoint{6.821143in}{0.637945in}}%
\pgfpathlineto{\pgfqpoint{6.770214in}{0.656321in}}%
\pgfpathlineto{\pgfqpoint{6.719286in}{0.656321in}}%
\pgfpathlineto{\pgfqpoint{6.668357in}{0.656321in}}%
\pgfpathlineto{\pgfqpoint{6.617429in}{0.678004in}}%
\pgfpathlineto{\pgfqpoint{6.566500in}{0.678004in}}%
\pgfpathlineto{\pgfqpoint{6.515571in}{0.678004in}}%
\pgfpathlineto{\pgfqpoint{6.464643in}{0.678004in}}%
\pgfpathlineto{\pgfqpoint{6.413714in}{0.678004in}}%
\pgfpathlineto{\pgfqpoint{6.362786in}{0.678004in}}%
\pgfpathlineto{\pgfqpoint{6.311857in}{0.681458in}}%
\pgfpathlineto{\pgfqpoint{6.260929in}{0.681458in}}%
\pgfpathlineto{\pgfqpoint{6.210000in}{0.689245in}}%
\pgfpathlineto{\pgfqpoint{6.159071in}{0.691932in}}%
\pgfpathlineto{\pgfqpoint{6.108143in}{0.691932in}}%
\pgfpathlineto{\pgfqpoint{6.057214in}{0.691932in}}%
\pgfpathlineto{\pgfqpoint{6.006286in}{0.691932in}}%
\pgfpathlineto{\pgfqpoint{5.955357in}{0.691932in}}%
\pgfpathlineto{\pgfqpoint{5.904429in}{0.691932in}}%
\pgfpathlineto{\pgfqpoint{5.853500in}{0.691932in}}%
\pgfpathlineto{\pgfqpoint{5.802571in}{0.691932in}}%
\pgfpathlineto{\pgfqpoint{5.751643in}{0.726000in}}%
\pgfpathlineto{\pgfqpoint{5.700714in}{0.726000in}}%
\pgfpathlineto{\pgfqpoint{5.649786in}{0.726000in}}%
\pgfpathlineto{\pgfqpoint{5.598857in}{0.726000in}}%
\pgfpathlineto{\pgfqpoint{5.547929in}{0.787435in}}%
\pgfpathlineto{\pgfqpoint{5.497000in}{0.878402in}}%
\pgfpathlineto{\pgfqpoint{5.446071in}{0.907606in}}%
\pgfpathlineto{\pgfqpoint{5.395143in}{0.907606in}}%
\pgfpathlineto{\pgfqpoint{5.344214in}{0.911799in}}%
\pgfpathlineto{\pgfqpoint{5.293286in}{0.911799in}}%
\pgfpathlineto{\pgfqpoint{5.242357in}{0.911799in}}%
\pgfpathlineto{\pgfqpoint{5.191429in}{0.911799in}}%
\pgfpathlineto{\pgfqpoint{5.140500in}{0.911799in}}%
\pgfpathlineto{\pgfqpoint{5.089571in}{0.961470in}}%
\pgfpathlineto{\pgfqpoint{5.038643in}{0.961470in}}%
\pgfpathlineto{\pgfqpoint{4.987714in}{0.965944in}}%
\pgfpathlineto{\pgfqpoint{4.936786in}{0.965944in}}%
\pgfpathlineto{\pgfqpoint{4.885857in}{0.988100in}}%
\pgfpathlineto{\pgfqpoint{4.834929in}{1.030262in}}%
\pgfpathlineto{\pgfqpoint{4.784000in}{1.030262in}}%
\pgfpathlineto{\pgfqpoint{4.733071in}{1.030262in}}%
\pgfpathlineto{\pgfqpoint{4.682143in}{1.030262in}}%
\pgfpathlineto{\pgfqpoint{4.631214in}{1.095510in}}%
\pgfpathlineto{\pgfqpoint{4.580286in}{1.095510in}}%
\pgfpathlineto{\pgfqpoint{4.529357in}{1.095510in}}%
\pgfpathlineto{\pgfqpoint{4.478429in}{1.095510in}}%
\pgfpathlineto{\pgfqpoint{4.427500in}{1.179251in}}%
\pgfpathlineto{\pgfqpoint{4.376571in}{1.179251in}}%
\pgfpathlineto{\pgfqpoint{4.325643in}{1.224715in}}%
\pgfpathlineto{\pgfqpoint{4.274714in}{1.270786in}}%
\pgfpathlineto{\pgfqpoint{4.223786in}{1.270786in}}%
\pgfpathlineto{\pgfqpoint{4.172857in}{1.270786in}}%
\pgfpathlineto{\pgfqpoint{4.121929in}{1.283396in}}%
\pgfpathlineto{\pgfqpoint{4.071000in}{1.283396in}}%
\pgfpathlineto{\pgfqpoint{4.020071in}{1.295851in}}%
\pgfpathlineto{\pgfqpoint{3.969143in}{1.337151in}}%
\pgfpathlineto{\pgfqpoint{3.918214in}{1.420532in}}%
\pgfpathlineto{\pgfqpoint{3.867286in}{1.421764in}}%
\pgfpathlineto{\pgfqpoint{3.816357in}{1.421764in}}%
\pgfpathlineto{\pgfqpoint{3.765429in}{1.461543in}}%
\pgfpathlineto{\pgfqpoint{3.714500in}{1.483626in}}%
\pgfpathlineto{\pgfqpoint{3.663571in}{1.529849in}}%
\pgfpathlineto{\pgfqpoint{3.612643in}{1.529849in}}%
\pgfpathlineto{\pgfqpoint{3.561714in}{1.537746in}}%
\pgfpathlineto{\pgfqpoint{3.510786in}{1.552660in}}%
\pgfpathlineto{\pgfqpoint{3.459857in}{1.587307in}}%
\pgfpathlineto{\pgfqpoint{3.408929in}{1.587307in}}%
\pgfpathlineto{\pgfqpoint{3.358000in}{1.592313in}}%
\pgfpathlineto{\pgfqpoint{3.307071in}{1.592313in}}%
\pgfpathlineto{\pgfqpoint{3.256143in}{1.592313in}}%
\pgfpathlineto{\pgfqpoint{3.205214in}{1.599062in}}%
\pgfpathlineto{\pgfqpoint{3.154286in}{1.599062in}}%
\pgfpathlineto{\pgfqpoint{3.103357in}{1.607146in}}%
\pgfpathlineto{\pgfqpoint{3.052429in}{1.613123in}}%
\pgfpathlineto{\pgfqpoint{3.001500in}{1.665195in}}%
\pgfpathlineto{\pgfqpoint{2.950571in}{1.665195in}}%
\pgfpathlineto{\pgfqpoint{2.899643in}{1.714370in}}%
\pgfpathlineto{\pgfqpoint{2.848714in}{1.714370in}}%
\pgfpathlineto{\pgfqpoint{2.797786in}{1.726497in}}%
\pgfpathlineto{\pgfqpoint{2.746857in}{1.726497in}}%
\pgfpathlineto{\pgfqpoint{2.695929in}{1.734685in}}%
\pgfpathlineto{\pgfqpoint{2.645000in}{1.789281in}}%
\pgfpathlineto{\pgfqpoint{2.594071in}{1.875415in}}%
\pgfpathlineto{\pgfqpoint{2.543143in}{1.970108in}}%
\pgfpathlineto{\pgfqpoint{2.492214in}{1.975726in}}%
\pgfpathlineto{\pgfqpoint{2.441286in}{1.975726in}}%
\pgfpathlineto{\pgfqpoint{2.390357in}{1.975726in}}%
\pgfpathlineto{\pgfqpoint{2.339429in}{2.060343in}}%
\pgfpathlineto{\pgfqpoint{2.288500in}{2.060343in}}%
\pgfpathlineto{\pgfqpoint{2.237571in}{2.060343in}}%
\pgfpathlineto{\pgfqpoint{2.186643in}{2.063415in}}%
\pgfpathlineto{\pgfqpoint{2.135714in}{2.099508in}}%
\pgfpathlineto{\pgfqpoint{2.084786in}{2.099508in}}%
\pgfpathlineto{\pgfqpoint{2.033857in}{2.099508in}}%
\pgfpathlineto{\pgfqpoint{1.982929in}{2.099508in}}%
\pgfpathlineto{\pgfqpoint{1.932000in}{2.101311in}}%
\pgfpathlineto{\pgfqpoint{1.881071in}{2.101311in}}%
\pgfpathlineto{\pgfqpoint{1.830143in}{2.101311in}}%
\pgfpathlineto{\pgfqpoint{1.779214in}{2.101311in}}%
\pgfpathlineto{\pgfqpoint{1.728286in}{2.101311in}}%
\pgfpathlineto{\pgfqpoint{1.677357in}{2.103032in}}%
\pgfpathlineto{\pgfqpoint{1.626429in}{2.244597in}}%
\pgfpathlineto{\pgfqpoint{1.575500in}{2.244597in}}%
\pgfpathlineto{\pgfqpoint{1.524571in}{2.298543in}}%
\pgfpathlineto{\pgfqpoint{1.473643in}{2.353542in}}%
\pgfpathlineto{\pgfqpoint{1.422714in}{2.353542in}}%
\pgfpathlineto{\pgfqpoint{1.371786in}{2.353542in}}%
\pgfpathlineto{\pgfqpoint{1.320857in}{2.353542in}}%
\pgfpathlineto{\pgfqpoint{1.269929in}{2.353542in}}%
\pgfpathlineto{\pgfqpoint{1.219000in}{2.357614in}}%
\pgfpathlineto{\pgfqpoint{1.168071in}{2.357614in}}%
\pgfpathlineto{\pgfqpoint{1.117143in}{2.357614in}}%
\pgfpathlineto{\pgfqpoint{1.066214in}{2.372344in}}%
\pgfpathlineto{\pgfqpoint{1.015286in}{2.476628in}}%
\pgfpathlineto{\pgfqpoint{0.964357in}{2.482520in}}%
\pgfpathlineto{\pgfqpoint{0.913429in}{2.517079in}}%
\pgfpathlineto{\pgfqpoint{0.862500in}{2.517079in}}%
\pgfpathclose%
\pgfusepath{fill}%
\end{pgfscope}%
\begin{pgfscope}%
\pgfpathrectangle{\pgfqpoint{0.862500in}{0.375000in}}{\pgfqpoint{5.347500in}{2.265000in}}%
\pgfusepath{clip}%
\pgfsetbuttcap%
\pgfsetroundjoin%
\definecolor{currentfill}{rgb}{0.172549,0.627451,0.172549}%
\pgfsetfillcolor{currentfill}%
\pgfsetfillopacity{0.200000}%
\pgfsetlinewidth{0.000000pt}%
\definecolor{currentstroke}{rgb}{0.000000,0.000000,0.000000}%
\pgfsetstrokecolor{currentstroke}%
\pgfsetdash{}{0pt}%
\pgfpathmoveto{\pgfqpoint{0.862500in}{2.512866in}}%
\pgfpathlineto{\pgfqpoint{0.862500in}{2.530152in}}%
\pgfpathlineto{\pgfqpoint{0.913429in}{2.505352in}}%
\pgfpathlineto{\pgfqpoint{0.964357in}{2.505352in}}%
\pgfpathlineto{\pgfqpoint{1.015286in}{2.490412in}}%
\pgfpathlineto{\pgfqpoint{1.066214in}{2.448972in}}%
\pgfpathlineto{\pgfqpoint{1.117143in}{2.443268in}}%
\pgfpathlineto{\pgfqpoint{1.168071in}{2.443268in}}%
\pgfpathlineto{\pgfqpoint{1.219000in}{2.443268in}}%
\pgfpathlineto{\pgfqpoint{1.269929in}{2.443268in}}%
\pgfpathlineto{\pgfqpoint{1.320857in}{2.443268in}}%
\pgfpathlineto{\pgfqpoint{1.371786in}{2.427960in}}%
\pgfpathlineto{\pgfqpoint{1.422714in}{2.427960in}}%
\pgfpathlineto{\pgfqpoint{1.473643in}{2.427960in}}%
\pgfpathlineto{\pgfqpoint{1.524571in}{2.427960in}}%
\pgfpathlineto{\pgfqpoint{1.575500in}{2.427960in}}%
\pgfpathlineto{\pgfqpoint{1.626429in}{2.390960in}}%
\pgfpathlineto{\pgfqpoint{1.677357in}{2.390960in}}%
\pgfpathlineto{\pgfqpoint{1.728286in}{2.390960in}}%
\pgfpathlineto{\pgfqpoint{1.779214in}{2.390960in}}%
\pgfpathlineto{\pgfqpoint{1.830143in}{2.390960in}}%
\pgfpathlineto{\pgfqpoint{1.881071in}{2.390960in}}%
\pgfpathlineto{\pgfqpoint{1.932000in}{2.390960in}}%
\pgfpathlineto{\pgfqpoint{1.982929in}{2.386508in}}%
\pgfpathlineto{\pgfqpoint{2.033857in}{2.386508in}}%
\pgfpathlineto{\pgfqpoint{2.084786in}{2.386508in}}%
\pgfpathlineto{\pgfqpoint{2.135714in}{2.362743in}}%
\pgfpathlineto{\pgfqpoint{2.186643in}{2.362743in}}%
\pgfpathlineto{\pgfqpoint{2.237571in}{2.362743in}}%
\pgfpathlineto{\pgfqpoint{2.288500in}{2.354585in}}%
\pgfpathlineto{\pgfqpoint{2.339429in}{2.287694in}}%
\pgfpathlineto{\pgfqpoint{2.390357in}{2.287694in}}%
\pgfpathlineto{\pgfqpoint{2.441286in}{2.278135in}}%
\pgfpathlineto{\pgfqpoint{2.492214in}{2.278135in}}%
\pgfpathlineto{\pgfqpoint{2.543143in}{2.242884in}}%
\pgfpathlineto{\pgfqpoint{2.594071in}{2.242884in}}%
\pgfpathlineto{\pgfqpoint{2.645000in}{2.207553in}}%
\pgfpathlineto{\pgfqpoint{2.695929in}{2.144865in}}%
\pgfpathlineto{\pgfqpoint{2.746857in}{2.144865in}}%
\pgfpathlineto{\pgfqpoint{2.797786in}{2.144865in}}%
\pgfpathlineto{\pgfqpoint{2.848714in}{2.131701in}}%
\pgfpathlineto{\pgfqpoint{2.899643in}{2.019714in}}%
\pgfpathlineto{\pgfqpoint{2.950571in}{2.019714in}}%
\pgfpathlineto{\pgfqpoint{3.001500in}{2.019714in}}%
\pgfpathlineto{\pgfqpoint{3.052429in}{2.011852in}}%
\pgfpathlineto{\pgfqpoint{3.103357in}{1.943427in}}%
\pgfpathlineto{\pgfqpoint{3.154286in}{1.943427in}}%
\pgfpathlineto{\pgfqpoint{3.205214in}{1.943427in}}%
\pgfpathlineto{\pgfqpoint{3.256143in}{1.943427in}}%
\pgfpathlineto{\pgfqpoint{3.307071in}{1.943427in}}%
\pgfpathlineto{\pgfqpoint{3.358000in}{1.811492in}}%
\pgfpathlineto{\pgfqpoint{3.408929in}{1.757367in}}%
\pgfpathlineto{\pgfqpoint{3.459857in}{1.742918in}}%
\pgfpathlineto{\pgfqpoint{3.510786in}{1.620965in}}%
\pgfpathlineto{\pgfqpoint{3.561714in}{1.589112in}}%
\pgfpathlineto{\pgfqpoint{3.612643in}{1.546598in}}%
\pgfpathlineto{\pgfqpoint{3.663571in}{1.546598in}}%
\pgfpathlineto{\pgfqpoint{3.714500in}{1.546598in}}%
\pgfpathlineto{\pgfqpoint{3.765429in}{1.546598in}}%
\pgfpathlineto{\pgfqpoint{3.816357in}{1.546598in}}%
\pgfpathlineto{\pgfqpoint{3.867286in}{1.535880in}}%
\pgfpathlineto{\pgfqpoint{3.918214in}{1.524344in}}%
\pgfpathlineto{\pgfqpoint{3.969143in}{1.524344in}}%
\pgfpathlineto{\pgfqpoint{4.020071in}{1.524344in}}%
\pgfpathlineto{\pgfqpoint{4.071000in}{1.524344in}}%
\pgfpathlineto{\pgfqpoint{4.121929in}{1.439091in}}%
\pgfpathlineto{\pgfqpoint{4.172857in}{1.439091in}}%
\pgfpathlineto{\pgfqpoint{4.223786in}{1.384922in}}%
\pgfpathlineto{\pgfqpoint{4.274714in}{1.384922in}}%
\pgfpathlineto{\pgfqpoint{4.325643in}{1.384922in}}%
\pgfpathlineto{\pgfqpoint{4.376571in}{1.384922in}}%
\pgfpathlineto{\pgfqpoint{4.427500in}{1.384922in}}%
\pgfpathlineto{\pgfqpoint{4.478429in}{1.384922in}}%
\pgfpathlineto{\pgfqpoint{4.529357in}{1.342616in}}%
\pgfpathlineto{\pgfqpoint{4.580286in}{1.342616in}}%
\pgfpathlineto{\pgfqpoint{4.631214in}{1.342616in}}%
\pgfpathlineto{\pgfqpoint{4.682143in}{1.342616in}}%
\pgfpathlineto{\pgfqpoint{4.733071in}{1.342616in}}%
\pgfpathlineto{\pgfqpoint{4.784000in}{1.314896in}}%
\pgfpathlineto{\pgfqpoint{4.834929in}{1.314896in}}%
\pgfpathlineto{\pgfqpoint{4.885857in}{1.314896in}}%
\pgfpathlineto{\pgfqpoint{4.936786in}{1.290798in}}%
\pgfpathlineto{\pgfqpoint{4.987714in}{1.290798in}}%
\pgfpathlineto{\pgfqpoint{5.038643in}{1.290798in}}%
\pgfpathlineto{\pgfqpoint{5.089571in}{1.290798in}}%
\pgfpathlineto{\pgfqpoint{5.140500in}{1.290798in}}%
\pgfpathlineto{\pgfqpoint{5.191429in}{1.288876in}}%
\pgfpathlineto{\pgfqpoint{5.242357in}{1.288876in}}%
\pgfpathlineto{\pgfqpoint{5.293286in}{1.288876in}}%
\pgfpathlineto{\pgfqpoint{5.344214in}{1.288876in}}%
\pgfpathlineto{\pgfqpoint{5.395143in}{1.231916in}}%
\pgfpathlineto{\pgfqpoint{5.446071in}{1.231916in}}%
\pgfpathlineto{\pgfqpoint{5.497000in}{1.231916in}}%
\pgfpathlineto{\pgfqpoint{5.547929in}{1.231916in}}%
\pgfpathlineto{\pgfqpoint{5.598857in}{1.223007in}}%
\pgfpathlineto{\pgfqpoint{5.649786in}{1.223007in}}%
\pgfpathlineto{\pgfqpoint{5.700714in}{1.223007in}}%
\pgfpathlineto{\pgfqpoint{5.751643in}{1.223007in}}%
\pgfpathlineto{\pgfqpoint{5.802571in}{1.223007in}}%
\pgfpathlineto{\pgfqpoint{5.853500in}{1.207380in}}%
\pgfpathlineto{\pgfqpoint{5.904429in}{1.207380in}}%
\pgfpathlineto{\pgfqpoint{5.955357in}{1.207380in}}%
\pgfpathlineto{\pgfqpoint{6.006286in}{1.207380in}}%
\pgfpathlineto{\pgfqpoint{6.057214in}{1.207380in}}%
\pgfpathlineto{\pgfqpoint{6.108143in}{1.207380in}}%
\pgfpathlineto{\pgfqpoint{6.159071in}{1.200728in}}%
\pgfpathlineto{\pgfqpoint{6.210000in}{1.200728in}}%
\pgfpathlineto{\pgfqpoint{6.260929in}{1.200728in}}%
\pgfpathlineto{\pgfqpoint{6.311857in}{1.200728in}}%
\pgfpathlineto{\pgfqpoint{6.362786in}{1.200728in}}%
\pgfpathlineto{\pgfqpoint{6.413714in}{1.200728in}}%
\pgfpathlineto{\pgfqpoint{6.464643in}{1.200728in}}%
\pgfpathlineto{\pgfqpoint{6.515571in}{1.200728in}}%
\pgfpathlineto{\pgfqpoint{6.566500in}{1.200728in}}%
\pgfpathlineto{\pgfqpoint{6.617429in}{1.196937in}}%
\pgfpathlineto{\pgfqpoint{6.668357in}{1.196937in}}%
\pgfpathlineto{\pgfqpoint{6.719286in}{1.196937in}}%
\pgfpathlineto{\pgfqpoint{6.770214in}{1.191966in}}%
\pgfpathlineto{\pgfqpoint{6.821143in}{1.191966in}}%
\pgfpathlineto{\pgfqpoint{6.872071in}{1.191966in}}%
\pgfpathlineto{\pgfqpoint{6.923000in}{1.188538in}}%
\pgfpathlineto{\pgfqpoint{6.973929in}{1.174138in}}%
\pgfpathlineto{\pgfqpoint{7.024857in}{1.174138in}}%
\pgfpathlineto{\pgfqpoint{7.075786in}{1.174138in}}%
\pgfpathlineto{\pgfqpoint{7.126714in}{1.174138in}}%
\pgfpathlineto{\pgfqpoint{7.177643in}{1.174138in}}%
\pgfpathlineto{\pgfqpoint{7.228571in}{1.174138in}}%
\pgfpathlineto{\pgfqpoint{7.279500in}{1.166358in}}%
\pgfpathlineto{\pgfqpoint{7.330429in}{1.155855in}}%
\pgfpathlineto{\pgfqpoint{7.381357in}{1.155855in}}%
\pgfpathlineto{\pgfqpoint{7.432286in}{1.155855in}}%
\pgfpathlineto{\pgfqpoint{7.483214in}{1.147016in}}%
\pgfpathlineto{\pgfqpoint{7.534143in}{1.133687in}}%
\pgfpathlineto{\pgfqpoint{7.585071in}{1.133687in}}%
\pgfpathlineto{\pgfqpoint{7.636000in}{1.133687in}}%
\pgfpathlineto{\pgfqpoint{7.686929in}{1.133687in}}%
\pgfpathlineto{\pgfqpoint{7.737857in}{1.133687in}}%
\pgfpathlineto{\pgfqpoint{7.788786in}{1.130038in}}%
\pgfpathlineto{\pgfqpoint{7.839714in}{1.130038in}}%
\pgfpathlineto{\pgfqpoint{7.890643in}{1.130038in}}%
\pgfpathlineto{\pgfqpoint{7.941571in}{1.130038in}}%
\pgfpathlineto{\pgfqpoint{7.992500in}{1.130038in}}%
\pgfpathlineto{\pgfqpoint{8.043429in}{1.130038in}}%
\pgfpathlineto{\pgfqpoint{8.094357in}{1.130038in}}%
\pgfpathlineto{\pgfqpoint{8.145286in}{1.130038in}}%
\pgfpathlineto{\pgfqpoint{8.196214in}{1.125410in}}%
\pgfpathlineto{\pgfqpoint{8.247143in}{1.125410in}}%
\pgfpathlineto{\pgfqpoint{8.298071in}{1.091419in}}%
\pgfpathlineto{\pgfqpoint{8.349000in}{1.091419in}}%
\pgfpathlineto{\pgfqpoint{8.399929in}{1.086400in}}%
\pgfpathlineto{\pgfqpoint{8.450857in}{1.086400in}}%
\pgfpathlineto{\pgfqpoint{8.501786in}{1.086400in}}%
\pgfpathlineto{\pgfqpoint{8.552714in}{1.086400in}}%
\pgfpathlineto{\pgfqpoint{8.603643in}{1.083949in}}%
\pgfpathlineto{\pgfqpoint{8.654571in}{1.083949in}}%
\pgfpathlineto{\pgfqpoint{8.705500in}{1.083949in}}%
\pgfpathlineto{\pgfqpoint{8.756429in}{1.083949in}}%
\pgfpathlineto{\pgfqpoint{8.807357in}{1.083949in}}%
\pgfpathlineto{\pgfqpoint{8.858286in}{1.078102in}}%
\pgfpathlineto{\pgfqpoint{8.909214in}{1.078102in}}%
\pgfpathlineto{\pgfqpoint{8.960143in}{1.078102in}}%
\pgfpathlineto{\pgfqpoint{9.011071in}{1.077582in}}%
\pgfpathlineto{\pgfqpoint{9.062000in}{1.077582in}}%
\pgfpathlineto{\pgfqpoint{9.112929in}{1.077582in}}%
\pgfpathlineto{\pgfqpoint{9.163857in}{1.077582in}}%
\pgfpathlineto{\pgfqpoint{9.214786in}{1.077582in}}%
\pgfpathlineto{\pgfqpoint{9.265714in}{1.077582in}}%
\pgfpathlineto{\pgfqpoint{9.316643in}{1.077582in}}%
\pgfpathlineto{\pgfqpoint{9.367571in}{1.077582in}}%
\pgfpathlineto{\pgfqpoint{9.418500in}{1.077582in}}%
\pgfpathlineto{\pgfqpoint{9.469429in}{1.077582in}}%
\pgfpathlineto{\pgfqpoint{9.469429in}{0.705648in}}%
\pgfpathlineto{\pgfqpoint{9.469429in}{0.705648in}}%
\pgfpathlineto{\pgfqpoint{9.418500in}{0.705648in}}%
\pgfpathlineto{\pgfqpoint{9.367571in}{0.705648in}}%
\pgfpathlineto{\pgfqpoint{9.316643in}{0.705648in}}%
\pgfpathlineto{\pgfqpoint{9.265714in}{0.705648in}}%
\pgfpathlineto{\pgfqpoint{9.214786in}{0.705648in}}%
\pgfpathlineto{\pgfqpoint{9.163857in}{0.705648in}}%
\pgfpathlineto{\pgfqpoint{9.112929in}{0.705648in}}%
\pgfpathlineto{\pgfqpoint{9.062000in}{0.705648in}}%
\pgfpathlineto{\pgfqpoint{9.011071in}{0.705648in}}%
\pgfpathlineto{\pgfqpoint{8.960143in}{0.707585in}}%
\pgfpathlineto{\pgfqpoint{8.909214in}{0.707585in}}%
\pgfpathlineto{\pgfqpoint{8.858286in}{0.707585in}}%
\pgfpathlineto{\pgfqpoint{8.807357in}{0.744363in}}%
\pgfpathlineto{\pgfqpoint{8.756429in}{0.744363in}}%
\pgfpathlineto{\pgfqpoint{8.705500in}{0.744363in}}%
\pgfpathlineto{\pgfqpoint{8.654571in}{0.744363in}}%
\pgfpathlineto{\pgfqpoint{8.603643in}{0.744363in}}%
\pgfpathlineto{\pgfqpoint{8.552714in}{0.757377in}}%
\pgfpathlineto{\pgfqpoint{8.501786in}{0.757377in}}%
\pgfpathlineto{\pgfqpoint{8.450857in}{0.757377in}}%
\pgfpathlineto{\pgfqpoint{8.399929in}{0.757377in}}%
\pgfpathlineto{\pgfqpoint{8.349000in}{0.787408in}}%
\pgfpathlineto{\pgfqpoint{8.298071in}{0.787408in}}%
\pgfpathlineto{\pgfqpoint{8.247143in}{0.835052in}}%
\pgfpathlineto{\pgfqpoint{8.196214in}{0.835052in}}%
\pgfpathlineto{\pgfqpoint{8.145286in}{0.859049in}}%
\pgfpathlineto{\pgfqpoint{8.094357in}{0.859049in}}%
\pgfpathlineto{\pgfqpoint{8.043429in}{0.859049in}}%
\pgfpathlineto{\pgfqpoint{7.992500in}{0.859049in}}%
\pgfpathlineto{\pgfqpoint{7.941571in}{0.859049in}}%
\pgfpathlineto{\pgfqpoint{7.890643in}{0.859049in}}%
\pgfpathlineto{\pgfqpoint{7.839714in}{0.859049in}}%
\pgfpathlineto{\pgfqpoint{7.788786in}{0.859049in}}%
\pgfpathlineto{\pgfqpoint{7.737857in}{0.869333in}}%
\pgfpathlineto{\pgfqpoint{7.686929in}{0.869333in}}%
\pgfpathlineto{\pgfqpoint{7.636000in}{0.869333in}}%
\pgfpathlineto{\pgfqpoint{7.585071in}{0.869333in}}%
\pgfpathlineto{\pgfqpoint{7.534143in}{0.869333in}}%
\pgfpathlineto{\pgfqpoint{7.483214in}{0.875674in}}%
\pgfpathlineto{\pgfqpoint{7.432286in}{0.919649in}}%
\pgfpathlineto{\pgfqpoint{7.381357in}{0.919649in}}%
\pgfpathlineto{\pgfqpoint{7.330429in}{0.919649in}}%
\pgfpathlineto{\pgfqpoint{7.279500in}{0.945707in}}%
\pgfpathlineto{\pgfqpoint{7.228571in}{0.950320in}}%
\pgfpathlineto{\pgfqpoint{7.177643in}{0.950320in}}%
\pgfpathlineto{\pgfqpoint{7.126714in}{0.950320in}}%
\pgfpathlineto{\pgfqpoint{7.075786in}{0.950320in}}%
\pgfpathlineto{\pgfqpoint{7.024857in}{0.950320in}}%
\pgfpathlineto{\pgfqpoint{6.973929in}{0.950320in}}%
\pgfpathlineto{\pgfqpoint{6.923000in}{0.956360in}}%
\pgfpathlineto{\pgfqpoint{6.872071in}{0.970472in}}%
\pgfpathlineto{\pgfqpoint{6.821143in}{0.970472in}}%
\pgfpathlineto{\pgfqpoint{6.770214in}{0.970472in}}%
\pgfpathlineto{\pgfqpoint{6.719286in}{0.997971in}}%
\pgfpathlineto{\pgfqpoint{6.668357in}{0.997971in}}%
\pgfpathlineto{\pgfqpoint{6.617429in}{0.997971in}}%
\pgfpathlineto{\pgfqpoint{6.566500in}{1.012527in}}%
\pgfpathlineto{\pgfqpoint{6.515571in}{1.012527in}}%
\pgfpathlineto{\pgfqpoint{6.464643in}{1.012527in}}%
\pgfpathlineto{\pgfqpoint{6.413714in}{1.012527in}}%
\pgfpathlineto{\pgfqpoint{6.362786in}{1.012527in}}%
\pgfpathlineto{\pgfqpoint{6.311857in}{1.012527in}}%
\pgfpathlineto{\pgfqpoint{6.260929in}{1.012527in}}%
\pgfpathlineto{\pgfqpoint{6.210000in}{1.012527in}}%
\pgfpathlineto{\pgfqpoint{6.159071in}{1.012527in}}%
\pgfpathlineto{\pgfqpoint{6.108143in}{1.027441in}}%
\pgfpathlineto{\pgfqpoint{6.057214in}{1.027441in}}%
\pgfpathlineto{\pgfqpoint{6.006286in}{1.027441in}}%
\pgfpathlineto{\pgfqpoint{5.955357in}{1.027441in}}%
\pgfpathlineto{\pgfqpoint{5.904429in}{1.027441in}}%
\pgfpathlineto{\pgfqpoint{5.853500in}{1.027441in}}%
\pgfpathlineto{\pgfqpoint{5.802571in}{1.069178in}}%
\pgfpathlineto{\pgfqpoint{5.751643in}{1.069178in}}%
\pgfpathlineto{\pgfqpoint{5.700714in}{1.069178in}}%
\pgfpathlineto{\pgfqpoint{5.649786in}{1.069178in}}%
\pgfpathlineto{\pgfqpoint{5.598857in}{1.069178in}}%
\pgfpathlineto{\pgfqpoint{5.547929in}{1.074309in}}%
\pgfpathlineto{\pgfqpoint{5.497000in}{1.074309in}}%
\pgfpathlineto{\pgfqpoint{5.446071in}{1.074309in}}%
\pgfpathlineto{\pgfqpoint{5.395143in}{1.074309in}}%
\pgfpathlineto{\pgfqpoint{5.344214in}{1.095943in}}%
\pgfpathlineto{\pgfqpoint{5.293286in}{1.095943in}}%
\pgfpathlineto{\pgfqpoint{5.242357in}{1.095943in}}%
\pgfpathlineto{\pgfqpoint{5.191429in}{1.095943in}}%
\pgfpathlineto{\pgfqpoint{5.140500in}{1.096473in}}%
\pgfpathlineto{\pgfqpoint{5.089571in}{1.096473in}}%
\pgfpathlineto{\pgfqpoint{5.038643in}{1.096473in}}%
\pgfpathlineto{\pgfqpoint{4.987714in}{1.096473in}}%
\pgfpathlineto{\pgfqpoint{4.936786in}{1.096473in}}%
\pgfpathlineto{\pgfqpoint{4.885857in}{1.112296in}}%
\pgfpathlineto{\pgfqpoint{4.834929in}{1.112296in}}%
\pgfpathlineto{\pgfqpoint{4.784000in}{1.112296in}}%
\pgfpathlineto{\pgfqpoint{4.733071in}{1.166563in}}%
\pgfpathlineto{\pgfqpoint{4.682143in}{1.166563in}}%
\pgfpathlineto{\pgfqpoint{4.631214in}{1.166563in}}%
\pgfpathlineto{\pgfqpoint{4.580286in}{1.166563in}}%
\pgfpathlineto{\pgfqpoint{4.529357in}{1.166563in}}%
\pgfpathlineto{\pgfqpoint{4.478429in}{1.186737in}}%
\pgfpathlineto{\pgfqpoint{4.427500in}{1.186737in}}%
\pgfpathlineto{\pgfqpoint{4.376571in}{1.186737in}}%
\pgfpathlineto{\pgfqpoint{4.325643in}{1.186737in}}%
\pgfpathlineto{\pgfqpoint{4.274714in}{1.186737in}}%
\pgfpathlineto{\pgfqpoint{4.223786in}{1.186737in}}%
\pgfpathlineto{\pgfqpoint{4.172857in}{1.271044in}}%
\pgfpathlineto{\pgfqpoint{4.121929in}{1.271044in}}%
\pgfpathlineto{\pgfqpoint{4.071000in}{1.423220in}}%
\pgfpathlineto{\pgfqpoint{4.020071in}{1.423220in}}%
\pgfpathlineto{\pgfqpoint{3.969143in}{1.423220in}}%
\pgfpathlineto{\pgfqpoint{3.918214in}{1.423220in}}%
\pgfpathlineto{\pgfqpoint{3.867286in}{1.438717in}}%
\pgfpathlineto{\pgfqpoint{3.816357in}{1.455294in}}%
\pgfpathlineto{\pgfqpoint{3.765429in}{1.455294in}}%
\pgfpathlineto{\pgfqpoint{3.714500in}{1.455294in}}%
\pgfpathlineto{\pgfqpoint{3.663571in}{1.455294in}}%
\pgfpathlineto{\pgfqpoint{3.612643in}{1.455294in}}%
\pgfpathlineto{\pgfqpoint{3.561714in}{1.482161in}}%
\pgfpathlineto{\pgfqpoint{3.510786in}{1.565110in}}%
\pgfpathlineto{\pgfqpoint{3.459857in}{1.586924in}}%
\pgfpathlineto{\pgfqpoint{3.408929in}{1.605231in}}%
\pgfpathlineto{\pgfqpoint{3.358000in}{1.682964in}}%
\pgfpathlineto{\pgfqpoint{3.307071in}{1.729798in}}%
\pgfpathlineto{\pgfqpoint{3.256143in}{1.729798in}}%
\pgfpathlineto{\pgfqpoint{3.205214in}{1.729798in}}%
\pgfpathlineto{\pgfqpoint{3.154286in}{1.729798in}}%
\pgfpathlineto{\pgfqpoint{3.103357in}{1.729798in}}%
\pgfpathlineto{\pgfqpoint{3.052429in}{1.842062in}}%
\pgfpathlineto{\pgfqpoint{3.001500in}{1.891631in}}%
\pgfpathlineto{\pgfqpoint{2.950571in}{1.891631in}}%
\pgfpathlineto{\pgfqpoint{2.899643in}{1.891631in}}%
\pgfpathlineto{\pgfqpoint{2.848714in}{1.909971in}}%
\pgfpathlineto{\pgfqpoint{2.797786in}{1.932108in}}%
\pgfpathlineto{\pgfqpoint{2.746857in}{1.932108in}}%
\pgfpathlineto{\pgfqpoint{2.695929in}{1.932108in}}%
\pgfpathlineto{\pgfqpoint{2.645000in}{2.044602in}}%
\pgfpathlineto{\pgfqpoint{2.594071in}{2.051879in}}%
\pgfpathlineto{\pgfqpoint{2.543143in}{2.051879in}}%
\pgfpathlineto{\pgfqpoint{2.492214in}{2.104780in}}%
\pgfpathlineto{\pgfqpoint{2.441286in}{2.104780in}}%
\pgfpathlineto{\pgfqpoint{2.390357in}{2.107196in}}%
\pgfpathlineto{\pgfqpoint{2.339429in}{2.107196in}}%
\pgfpathlineto{\pgfqpoint{2.288500in}{2.185767in}}%
\pgfpathlineto{\pgfqpoint{2.237571in}{2.198918in}}%
\pgfpathlineto{\pgfqpoint{2.186643in}{2.198918in}}%
\pgfpathlineto{\pgfqpoint{2.135714in}{2.198918in}}%
\pgfpathlineto{\pgfqpoint{2.084786in}{2.278856in}}%
\pgfpathlineto{\pgfqpoint{2.033857in}{2.278856in}}%
\pgfpathlineto{\pgfqpoint{1.982929in}{2.278856in}}%
\pgfpathlineto{\pgfqpoint{1.932000in}{2.281765in}}%
\pgfpathlineto{\pgfqpoint{1.881071in}{2.281765in}}%
\pgfpathlineto{\pgfqpoint{1.830143in}{2.281765in}}%
\pgfpathlineto{\pgfqpoint{1.779214in}{2.281765in}}%
\pgfpathlineto{\pgfqpoint{1.728286in}{2.281765in}}%
\pgfpathlineto{\pgfqpoint{1.677357in}{2.281765in}}%
\pgfpathlineto{\pgfqpoint{1.626429in}{2.281765in}}%
\pgfpathlineto{\pgfqpoint{1.575500in}{2.372575in}}%
\pgfpathlineto{\pgfqpoint{1.524571in}{2.372575in}}%
\pgfpathlineto{\pgfqpoint{1.473643in}{2.372575in}}%
\pgfpathlineto{\pgfqpoint{1.422714in}{2.372575in}}%
\pgfpathlineto{\pgfqpoint{1.371786in}{2.372575in}}%
\pgfpathlineto{\pgfqpoint{1.320857in}{2.377261in}}%
\pgfpathlineto{\pgfqpoint{1.269929in}{2.377261in}}%
\pgfpathlineto{\pgfqpoint{1.219000in}{2.377261in}}%
\pgfpathlineto{\pgfqpoint{1.168071in}{2.377261in}}%
\pgfpathlineto{\pgfqpoint{1.117143in}{2.377261in}}%
\pgfpathlineto{\pgfqpoint{1.066214in}{2.378547in}}%
\pgfpathlineto{\pgfqpoint{1.015286in}{2.431992in}}%
\pgfpathlineto{\pgfqpoint{0.964357in}{2.445539in}}%
\pgfpathlineto{\pgfqpoint{0.913429in}{2.445539in}}%
\pgfpathlineto{\pgfqpoint{0.862500in}{2.512866in}}%
\pgfpathclose%
\pgfusepath{fill}%
\end{pgfscope}%
\begin{pgfscope}%
\pgfpathrectangle{\pgfqpoint{0.862500in}{0.375000in}}{\pgfqpoint{5.347500in}{2.265000in}}%
\pgfusepath{clip}%
\pgfsetbuttcap%
\pgfsetroundjoin%
\definecolor{currentfill}{rgb}{0.839216,0.152941,0.156863}%
\pgfsetfillcolor{currentfill}%
\pgfsetfillopacity{0.200000}%
\pgfsetlinewidth{0.000000pt}%
\definecolor{currentstroke}{rgb}{0.000000,0.000000,0.000000}%
\pgfsetstrokecolor{currentstroke}%
\pgfsetdash{}{0pt}%
\pgfpathmoveto{\pgfqpoint{0.862500in}{2.471353in}}%
\pgfpathlineto{\pgfqpoint{0.862500in}{2.516669in}}%
\pgfpathlineto{\pgfqpoint{0.913429in}{2.509175in}}%
\pgfpathlineto{\pgfqpoint{0.964357in}{2.505078in}}%
\pgfpathlineto{\pgfqpoint{1.015286in}{2.505078in}}%
\pgfpathlineto{\pgfqpoint{1.066214in}{2.503044in}}%
\pgfpathlineto{\pgfqpoint{1.117143in}{2.476310in}}%
\pgfpathlineto{\pgfqpoint{1.168071in}{2.476310in}}%
\pgfpathlineto{\pgfqpoint{1.219000in}{2.469399in}}%
\pgfpathlineto{\pgfqpoint{1.269929in}{2.469399in}}%
\pgfpathlineto{\pgfqpoint{1.320857in}{2.469399in}}%
\pgfpathlineto{\pgfqpoint{1.371786in}{2.448459in}}%
\pgfpathlineto{\pgfqpoint{1.422714in}{2.445928in}}%
\pgfpathlineto{\pgfqpoint{1.473643in}{2.445928in}}%
\pgfpathlineto{\pgfqpoint{1.524571in}{2.445928in}}%
\pgfpathlineto{\pgfqpoint{1.575500in}{2.445928in}}%
\pgfpathlineto{\pgfqpoint{1.626429in}{2.419916in}}%
\pgfpathlineto{\pgfqpoint{1.677357in}{2.395261in}}%
\pgfpathlineto{\pgfqpoint{1.728286in}{2.395261in}}%
\pgfpathlineto{\pgfqpoint{1.779214in}{2.395261in}}%
\pgfpathlineto{\pgfqpoint{1.830143in}{2.395261in}}%
\pgfpathlineto{\pgfqpoint{1.881071in}{2.384151in}}%
\pgfpathlineto{\pgfqpoint{1.932000in}{2.384151in}}%
\pgfpathlineto{\pgfqpoint{1.982929in}{2.309034in}}%
\pgfpathlineto{\pgfqpoint{2.033857in}{2.287070in}}%
\pgfpathlineto{\pgfqpoint{2.084786in}{2.280080in}}%
\pgfpathlineto{\pgfqpoint{2.135714in}{2.276295in}}%
\pgfpathlineto{\pgfqpoint{2.186643in}{2.270670in}}%
\pgfpathlineto{\pgfqpoint{2.237571in}{2.225885in}}%
\pgfpathlineto{\pgfqpoint{2.288500in}{2.218914in}}%
\pgfpathlineto{\pgfqpoint{2.339429in}{2.194037in}}%
\pgfpathlineto{\pgfqpoint{2.390357in}{2.194037in}}%
\pgfpathlineto{\pgfqpoint{2.441286in}{2.194037in}}%
\pgfpathlineto{\pgfqpoint{2.492214in}{2.140532in}}%
\pgfpathlineto{\pgfqpoint{2.543143in}{2.072577in}}%
\pgfpathlineto{\pgfqpoint{2.594071in}{1.994018in}}%
\pgfpathlineto{\pgfqpoint{2.645000in}{1.937841in}}%
\pgfpathlineto{\pgfqpoint{2.695929in}{1.833053in}}%
\pgfpathlineto{\pgfqpoint{2.746857in}{1.817017in}}%
\pgfpathlineto{\pgfqpoint{2.797786in}{1.785407in}}%
\pgfpathlineto{\pgfqpoint{2.848714in}{1.770991in}}%
\pgfpathlineto{\pgfqpoint{2.899643in}{1.745343in}}%
\pgfpathlineto{\pgfqpoint{2.950571in}{1.724095in}}%
\pgfpathlineto{\pgfqpoint{3.001500in}{1.710625in}}%
\pgfpathlineto{\pgfqpoint{3.052429in}{1.710625in}}%
\pgfpathlineto{\pgfqpoint{3.103357in}{1.710480in}}%
\pgfpathlineto{\pgfqpoint{3.154286in}{1.536641in}}%
\pgfpathlineto{\pgfqpoint{3.205214in}{1.501844in}}%
\pgfpathlineto{\pgfqpoint{3.256143in}{1.501844in}}%
\pgfpathlineto{\pgfqpoint{3.307071in}{1.501844in}}%
\pgfpathlineto{\pgfqpoint{3.358000in}{1.428316in}}%
\pgfpathlineto{\pgfqpoint{3.408929in}{1.428316in}}%
\pgfpathlineto{\pgfqpoint{3.459857in}{1.383704in}}%
\pgfpathlineto{\pgfqpoint{3.510786in}{1.378335in}}%
\pgfpathlineto{\pgfqpoint{3.561714in}{1.378335in}}%
\pgfpathlineto{\pgfqpoint{3.612643in}{1.376962in}}%
\pgfpathlineto{\pgfqpoint{3.663571in}{1.348409in}}%
\pgfpathlineto{\pgfqpoint{3.714500in}{1.348409in}}%
\pgfpathlineto{\pgfqpoint{3.765429in}{1.344505in}}%
\pgfpathlineto{\pgfqpoint{3.816357in}{1.343017in}}%
\pgfpathlineto{\pgfqpoint{3.867286in}{1.340806in}}%
\pgfpathlineto{\pgfqpoint{3.918214in}{1.340806in}}%
\pgfpathlineto{\pgfqpoint{3.969143in}{1.334062in}}%
\pgfpathlineto{\pgfqpoint{4.020071in}{1.334062in}}%
\pgfpathlineto{\pgfqpoint{4.071000in}{1.334062in}}%
\pgfpathlineto{\pgfqpoint{4.121929in}{1.327228in}}%
\pgfpathlineto{\pgfqpoint{4.172857in}{1.327228in}}%
\pgfpathlineto{\pgfqpoint{4.223786in}{1.327228in}}%
\pgfpathlineto{\pgfqpoint{4.274714in}{1.326825in}}%
\pgfpathlineto{\pgfqpoint{4.325643in}{1.326825in}}%
\pgfpathlineto{\pgfqpoint{4.376571in}{1.326825in}}%
\pgfpathlineto{\pgfqpoint{4.427500in}{1.326825in}}%
\pgfpathlineto{\pgfqpoint{4.478429in}{1.326825in}}%
\pgfpathlineto{\pgfqpoint{4.529357in}{1.326825in}}%
\pgfpathlineto{\pgfqpoint{4.580286in}{1.326825in}}%
\pgfpathlineto{\pgfqpoint{4.631214in}{1.326825in}}%
\pgfpathlineto{\pgfqpoint{4.682143in}{1.323349in}}%
\pgfpathlineto{\pgfqpoint{4.733071in}{1.320465in}}%
\pgfpathlineto{\pgfqpoint{4.784000in}{1.320465in}}%
\pgfpathlineto{\pgfqpoint{4.834929in}{1.320465in}}%
\pgfpathlineto{\pgfqpoint{4.885857in}{1.316356in}}%
\pgfpathlineto{\pgfqpoint{4.936786in}{1.316356in}}%
\pgfpathlineto{\pgfqpoint{4.987714in}{1.316356in}}%
\pgfpathlineto{\pgfqpoint{5.038643in}{1.309018in}}%
\pgfpathlineto{\pgfqpoint{5.089571in}{1.309018in}}%
\pgfpathlineto{\pgfqpoint{5.140500in}{1.307026in}}%
\pgfpathlineto{\pgfqpoint{5.191429in}{1.299060in}}%
\pgfpathlineto{\pgfqpoint{5.242357in}{1.298544in}}%
\pgfpathlineto{\pgfqpoint{5.293286in}{1.296212in}}%
\pgfpathlineto{\pgfqpoint{5.344214in}{1.296212in}}%
\pgfpathlineto{\pgfqpoint{5.395143in}{1.296041in}}%
\pgfpathlineto{\pgfqpoint{5.446071in}{1.293782in}}%
\pgfpathlineto{\pgfqpoint{5.497000in}{1.292723in}}%
\pgfpathlineto{\pgfqpoint{5.547929in}{1.291888in}}%
\pgfpathlineto{\pgfqpoint{5.598857in}{1.291165in}}%
\pgfpathlineto{\pgfqpoint{5.649786in}{1.291154in}}%
\pgfpathlineto{\pgfqpoint{5.700714in}{1.290515in}}%
\pgfpathlineto{\pgfqpoint{5.751643in}{1.289262in}}%
\pgfpathlineto{\pgfqpoint{5.802571in}{1.289262in}}%
\pgfpathlineto{\pgfqpoint{5.853500in}{1.283779in}}%
\pgfpathlineto{\pgfqpoint{5.904429in}{1.283779in}}%
\pgfpathlineto{\pgfqpoint{5.955357in}{1.283542in}}%
\pgfpathlineto{\pgfqpoint{6.006286in}{1.283542in}}%
\pgfpathlineto{\pgfqpoint{6.057214in}{1.283397in}}%
\pgfpathlineto{\pgfqpoint{6.108143in}{1.283279in}}%
\pgfpathlineto{\pgfqpoint{6.159071in}{1.279922in}}%
\pgfpathlineto{\pgfqpoint{6.210000in}{1.279776in}}%
\pgfpathlineto{\pgfqpoint{6.260929in}{1.279384in}}%
\pgfpathlineto{\pgfqpoint{6.311857in}{1.279123in}}%
\pgfpathlineto{\pgfqpoint{6.362786in}{1.279123in}}%
\pgfpathlineto{\pgfqpoint{6.413714in}{1.277920in}}%
\pgfpathlineto{\pgfqpoint{6.464643in}{1.277889in}}%
\pgfpathlineto{\pgfqpoint{6.515571in}{1.277889in}}%
\pgfpathlineto{\pgfqpoint{6.566500in}{1.269364in}}%
\pgfpathlineto{\pgfqpoint{6.617429in}{1.269158in}}%
\pgfpathlineto{\pgfqpoint{6.668357in}{1.269158in}}%
\pgfpathlineto{\pgfqpoint{6.719286in}{1.269158in}}%
\pgfpathlineto{\pgfqpoint{6.770214in}{1.268814in}}%
\pgfpathlineto{\pgfqpoint{6.821143in}{1.268814in}}%
\pgfpathlineto{\pgfqpoint{6.872071in}{1.268497in}}%
\pgfpathlineto{\pgfqpoint{6.923000in}{1.268497in}}%
\pgfpathlineto{\pgfqpoint{6.973929in}{1.268428in}}%
\pgfpathlineto{\pgfqpoint{7.024857in}{1.268192in}}%
\pgfpathlineto{\pgfqpoint{7.075786in}{1.268088in}}%
\pgfpathlineto{\pgfqpoint{7.126714in}{1.267439in}}%
\pgfpathlineto{\pgfqpoint{7.177643in}{1.267144in}}%
\pgfpathlineto{\pgfqpoint{7.228571in}{1.267100in}}%
\pgfpathlineto{\pgfqpoint{7.279500in}{1.267100in}}%
\pgfpathlineto{\pgfqpoint{7.330429in}{1.266642in}}%
\pgfpathlineto{\pgfqpoint{7.381357in}{1.261501in}}%
\pgfpathlineto{\pgfqpoint{7.432286in}{1.261129in}}%
\pgfpathlineto{\pgfqpoint{7.483214in}{1.261129in}}%
\pgfpathlineto{\pgfqpoint{7.534143in}{1.261041in}}%
\pgfpathlineto{\pgfqpoint{7.585071in}{1.261021in}}%
\pgfpathlineto{\pgfqpoint{7.636000in}{1.260959in}}%
\pgfpathlineto{\pgfqpoint{7.686929in}{1.191141in}}%
\pgfpathlineto{\pgfqpoint{7.737857in}{1.191029in}}%
\pgfpathlineto{\pgfqpoint{7.788786in}{1.191029in}}%
\pgfpathlineto{\pgfqpoint{7.839714in}{1.190923in}}%
\pgfpathlineto{\pgfqpoint{7.890643in}{1.190785in}}%
\pgfpathlineto{\pgfqpoint{7.941571in}{1.190774in}}%
\pgfpathlineto{\pgfqpoint{7.992500in}{1.190765in}}%
\pgfpathlineto{\pgfqpoint{8.043429in}{1.186644in}}%
\pgfpathlineto{\pgfqpoint{8.094357in}{1.186633in}}%
\pgfpathlineto{\pgfqpoint{8.145286in}{1.186617in}}%
\pgfpathlineto{\pgfqpoint{8.196214in}{1.184662in}}%
\pgfpathlineto{\pgfqpoint{8.247143in}{1.184604in}}%
\pgfpathlineto{\pgfqpoint{8.298071in}{1.184464in}}%
\pgfpathlineto{\pgfqpoint{8.349000in}{1.165179in}}%
\pgfpathlineto{\pgfqpoint{8.399929in}{1.165153in}}%
\pgfpathlineto{\pgfqpoint{8.450857in}{1.165125in}}%
\pgfpathlineto{\pgfqpoint{8.501786in}{1.159118in}}%
\pgfpathlineto{\pgfqpoint{8.552714in}{1.159101in}}%
\pgfpathlineto{\pgfqpoint{8.603643in}{1.159073in}}%
\pgfpathlineto{\pgfqpoint{8.654571in}{1.159044in}}%
\pgfpathlineto{\pgfqpoint{8.705500in}{1.159023in}}%
\pgfpathlineto{\pgfqpoint{8.756429in}{1.158967in}}%
\pgfpathlineto{\pgfqpoint{8.807357in}{1.143354in}}%
\pgfpathlineto{\pgfqpoint{8.858286in}{1.131657in}}%
\pgfpathlineto{\pgfqpoint{8.909214in}{1.131604in}}%
\pgfpathlineto{\pgfqpoint{8.960143in}{1.131583in}}%
\pgfpathlineto{\pgfqpoint{9.011071in}{1.131566in}}%
\pgfpathlineto{\pgfqpoint{9.062000in}{1.131508in}}%
\pgfpathlineto{\pgfqpoint{9.112929in}{1.130419in}}%
\pgfpathlineto{\pgfqpoint{9.163857in}{1.130410in}}%
\pgfpathlineto{\pgfqpoint{9.214786in}{1.130401in}}%
\pgfpathlineto{\pgfqpoint{9.265714in}{1.130388in}}%
\pgfpathlineto{\pgfqpoint{9.316643in}{1.130373in}}%
\pgfpathlineto{\pgfqpoint{9.367571in}{1.130363in}}%
\pgfpathlineto{\pgfqpoint{9.418500in}{1.130352in}}%
\pgfpathlineto{\pgfqpoint{9.469429in}{1.130244in}}%
\pgfpathlineto{\pgfqpoint{9.469429in}{0.586610in}}%
\pgfpathlineto{\pgfqpoint{9.469429in}{0.586610in}}%
\pgfpathlineto{\pgfqpoint{9.418500in}{0.586739in}}%
\pgfpathlineto{\pgfqpoint{9.367571in}{0.586761in}}%
\pgfpathlineto{\pgfqpoint{9.316643in}{0.586845in}}%
\pgfpathlineto{\pgfqpoint{9.265714in}{0.586970in}}%
\pgfpathlineto{\pgfqpoint{9.214786in}{0.587077in}}%
\pgfpathlineto{\pgfqpoint{9.163857in}{0.587151in}}%
\pgfpathlineto{\pgfqpoint{9.112929in}{0.587219in}}%
\pgfpathlineto{\pgfqpoint{9.062000in}{0.588503in}}%
\pgfpathlineto{\pgfqpoint{9.011071in}{0.588759in}}%
\pgfpathlineto{\pgfqpoint{8.960143in}{0.588896in}}%
\pgfpathlineto{\pgfqpoint{8.909214in}{0.589071in}}%
\pgfpathlineto{\pgfqpoint{8.858286in}{0.589504in}}%
\pgfpathlineto{\pgfqpoint{8.807357in}{0.600787in}}%
\pgfpathlineto{\pgfqpoint{8.756429in}{0.614656in}}%
\pgfpathlineto{\pgfqpoint{8.705500in}{0.615117in}}%
\pgfpathlineto{\pgfqpoint{8.654571in}{0.615289in}}%
\pgfpathlineto{\pgfqpoint{8.603643in}{0.615532in}}%
\pgfpathlineto{\pgfqpoint{8.552714in}{0.615762in}}%
\pgfpathlineto{\pgfqpoint{8.501786in}{0.615901in}}%
\pgfpathlineto{\pgfqpoint{8.450857in}{0.620857in}}%
\pgfpathlineto{\pgfqpoint{8.399929in}{0.621090in}}%
\pgfpathlineto{\pgfqpoint{8.349000in}{0.621305in}}%
\pgfpathlineto{\pgfqpoint{8.298071in}{0.643819in}}%
\pgfpathlineto{\pgfqpoint{8.247143in}{0.644972in}}%
\pgfpathlineto{\pgfqpoint{8.196214in}{0.645444in}}%
\pgfpathlineto{\pgfqpoint{8.145286in}{0.647588in}}%
\pgfpathlineto{\pgfqpoint{8.094357in}{0.647725in}}%
\pgfpathlineto{\pgfqpoint{8.043429in}{0.647814in}}%
\pgfpathlineto{\pgfqpoint{7.992500in}{0.651926in}}%
\pgfpathlineto{\pgfqpoint{7.941571in}{0.652004in}}%
\pgfpathlineto{\pgfqpoint{7.890643in}{0.652093in}}%
\pgfpathlineto{\pgfqpoint{7.839714in}{0.653214in}}%
\pgfpathlineto{\pgfqpoint{7.788786in}{0.654074in}}%
\pgfpathlineto{\pgfqpoint{7.737857in}{0.654074in}}%
\pgfpathlineto{\pgfqpoint{7.686929in}{0.654974in}}%
\pgfpathlineto{\pgfqpoint{7.636000in}{0.706530in}}%
\pgfpathlineto{\pgfqpoint{7.585071in}{0.707042in}}%
\pgfpathlineto{\pgfqpoint{7.534143in}{0.707209in}}%
\pgfpathlineto{\pgfqpoint{7.483214in}{0.707932in}}%
\pgfpathlineto{\pgfqpoint{7.432286in}{0.707932in}}%
\pgfpathlineto{\pgfqpoint{7.381357in}{0.711010in}}%
\pgfpathlineto{\pgfqpoint{7.330429in}{0.714169in}}%
\pgfpathlineto{\pgfqpoint{7.279500in}{0.717921in}}%
\pgfpathlineto{\pgfqpoint{7.228571in}{0.717921in}}%
\pgfpathlineto{\pgfqpoint{7.177643in}{0.718280in}}%
\pgfpathlineto{\pgfqpoint{7.126714in}{0.720661in}}%
\pgfpathlineto{\pgfqpoint{7.075786in}{0.725873in}}%
\pgfpathlineto{\pgfqpoint{7.024857in}{0.726726in}}%
\pgfpathlineto{\pgfqpoint{6.973929in}{0.728596in}}%
\pgfpathlineto{\pgfqpoint{6.923000in}{0.729153in}}%
\pgfpathlineto{\pgfqpoint{6.872071in}{0.729153in}}%
\pgfpathlineto{\pgfqpoint{6.821143in}{0.731681in}}%
\pgfpathlineto{\pgfqpoint{6.770214in}{0.731681in}}%
\pgfpathlineto{\pgfqpoint{6.719286in}{0.734407in}}%
\pgfpathlineto{\pgfqpoint{6.668357in}{0.734407in}}%
\pgfpathlineto{\pgfqpoint{6.617429in}{0.734407in}}%
\pgfpathlineto{\pgfqpoint{6.566500in}{0.736054in}}%
\pgfpathlineto{\pgfqpoint{6.515571in}{0.740902in}}%
\pgfpathlineto{\pgfqpoint{6.464643in}{0.740902in}}%
\pgfpathlineto{\pgfqpoint{6.413714in}{0.741142in}}%
\pgfpathlineto{\pgfqpoint{6.362786in}{0.750566in}}%
\pgfpathlineto{\pgfqpoint{6.311857in}{0.750566in}}%
\pgfpathlineto{\pgfqpoint{6.260929in}{0.752569in}}%
\pgfpathlineto{\pgfqpoint{6.210000in}{0.755580in}}%
\pgfpathlineto{\pgfqpoint{6.159071in}{0.756716in}}%
\pgfpathlineto{\pgfqpoint{6.108143in}{0.758514in}}%
\pgfpathlineto{\pgfqpoint{6.057214in}{0.759414in}}%
\pgfpathlineto{\pgfqpoint{6.006286in}{0.760534in}}%
\pgfpathlineto{\pgfqpoint{5.955357in}{0.760534in}}%
\pgfpathlineto{\pgfqpoint{5.904429in}{0.762360in}}%
\pgfpathlineto{\pgfqpoint{5.853500in}{0.762360in}}%
\pgfpathlineto{\pgfqpoint{5.802571in}{0.780244in}}%
\pgfpathlineto{\pgfqpoint{5.751643in}{0.780244in}}%
\pgfpathlineto{\pgfqpoint{5.700714in}{0.782843in}}%
\pgfpathlineto{\pgfqpoint{5.649786in}{0.787521in}}%
\pgfpathlineto{\pgfqpoint{5.598857in}{0.787538in}}%
\pgfpathlineto{\pgfqpoint{5.547929in}{0.792837in}}%
\pgfpathlineto{\pgfqpoint{5.497000in}{0.798673in}}%
\pgfpathlineto{\pgfqpoint{5.446071in}{0.800324in}}%
\pgfpathlineto{\pgfqpoint{5.395143in}{0.815894in}}%
\pgfpathlineto{\pgfqpoint{5.344214in}{0.817124in}}%
\pgfpathlineto{\pgfqpoint{5.293286in}{0.817124in}}%
\pgfpathlineto{\pgfqpoint{5.242357in}{0.833186in}}%
\pgfpathlineto{\pgfqpoint{5.191429in}{0.836595in}}%
\pgfpathlineto{\pgfqpoint{5.140500in}{0.847791in}}%
\pgfpathlineto{\pgfqpoint{5.089571in}{0.860572in}}%
\pgfpathlineto{\pgfqpoint{5.038643in}{0.860572in}}%
\pgfpathlineto{\pgfqpoint{4.987714in}{0.870769in}}%
\pgfpathlineto{\pgfqpoint{4.936786in}{0.870769in}}%
\pgfpathlineto{\pgfqpoint{4.885857in}{0.870769in}}%
\pgfpathlineto{\pgfqpoint{4.834929in}{0.894639in}}%
\pgfpathlineto{\pgfqpoint{4.784000in}{0.894639in}}%
\pgfpathlineto{\pgfqpoint{4.733071in}{0.894639in}}%
\pgfpathlineto{\pgfqpoint{4.682143in}{0.909633in}}%
\pgfpathlineto{\pgfqpoint{4.631214in}{0.930950in}}%
\pgfpathlineto{\pgfqpoint{4.580286in}{0.930950in}}%
\pgfpathlineto{\pgfqpoint{4.529357in}{0.930950in}}%
\pgfpathlineto{\pgfqpoint{4.478429in}{0.930950in}}%
\pgfpathlineto{\pgfqpoint{4.427500in}{0.930950in}}%
\pgfpathlineto{\pgfqpoint{4.376571in}{0.930950in}}%
\pgfpathlineto{\pgfqpoint{4.325643in}{0.930950in}}%
\pgfpathlineto{\pgfqpoint{4.274714in}{0.930950in}}%
\pgfpathlineto{\pgfqpoint{4.223786in}{0.933249in}}%
\pgfpathlineto{\pgfqpoint{4.172857in}{0.933249in}}%
\pgfpathlineto{\pgfqpoint{4.121929in}{0.933249in}}%
\pgfpathlineto{\pgfqpoint{4.071000in}{0.963837in}}%
\pgfpathlineto{\pgfqpoint{4.020071in}{0.963837in}}%
\pgfpathlineto{\pgfqpoint{3.969143in}{0.963837in}}%
\pgfpathlineto{\pgfqpoint{3.918214in}{0.992403in}}%
\pgfpathlineto{\pgfqpoint{3.867286in}{0.992403in}}%
\pgfpathlineto{\pgfqpoint{3.816357in}{1.000005in}}%
\pgfpathlineto{\pgfqpoint{3.765429in}{1.001548in}}%
\pgfpathlineto{\pgfqpoint{3.714500in}{1.021454in}}%
\pgfpathlineto{\pgfqpoint{3.663571in}{1.021454in}}%
\pgfpathlineto{\pgfqpoint{3.612643in}{1.091249in}}%
\pgfpathlineto{\pgfqpoint{3.561714in}{1.098190in}}%
\pgfpathlineto{\pgfqpoint{3.510786in}{1.098190in}}%
\pgfpathlineto{\pgfqpoint{3.459857in}{1.121887in}}%
\pgfpathlineto{\pgfqpoint{3.408929in}{1.177293in}}%
\pgfpathlineto{\pgfqpoint{3.358000in}{1.177293in}}%
\pgfpathlineto{\pgfqpoint{3.307071in}{1.307176in}}%
\pgfpathlineto{\pgfqpoint{3.256143in}{1.307176in}}%
\pgfpathlineto{\pgfqpoint{3.205214in}{1.307176in}}%
\pgfpathlineto{\pgfqpoint{3.154286in}{1.337584in}}%
\pgfpathlineto{\pgfqpoint{3.103357in}{1.498845in}}%
\pgfpathlineto{\pgfqpoint{3.052429in}{1.498913in}}%
\pgfpathlineto{\pgfqpoint{3.001500in}{1.498913in}}%
\pgfpathlineto{\pgfqpoint{2.950571in}{1.505645in}}%
\pgfpathlineto{\pgfqpoint{2.899643in}{1.531285in}}%
\pgfpathlineto{\pgfqpoint{2.848714in}{1.546393in}}%
\pgfpathlineto{\pgfqpoint{2.797786in}{1.555834in}}%
\pgfpathlineto{\pgfqpoint{2.746857in}{1.614273in}}%
\pgfpathlineto{\pgfqpoint{2.695929in}{1.679656in}}%
\pgfpathlineto{\pgfqpoint{2.645000in}{1.816371in}}%
\pgfpathlineto{\pgfqpoint{2.594071in}{1.860432in}}%
\pgfpathlineto{\pgfqpoint{2.543143in}{1.918376in}}%
\pgfpathlineto{\pgfqpoint{2.492214in}{1.950207in}}%
\pgfpathlineto{\pgfqpoint{2.441286in}{2.075560in}}%
\pgfpathlineto{\pgfqpoint{2.390357in}{2.075560in}}%
\pgfpathlineto{\pgfqpoint{2.339429in}{2.075560in}}%
\pgfpathlineto{\pgfqpoint{2.288500in}{2.107355in}}%
\pgfpathlineto{\pgfqpoint{2.237571in}{2.111854in}}%
\pgfpathlineto{\pgfqpoint{2.186643in}{2.155014in}}%
\pgfpathlineto{\pgfqpoint{2.135714in}{2.159793in}}%
\pgfpathlineto{\pgfqpoint{2.084786in}{2.164241in}}%
\pgfpathlineto{\pgfqpoint{2.033857in}{2.199986in}}%
\pgfpathlineto{\pgfqpoint{1.982929in}{2.215750in}}%
\pgfpathlineto{\pgfqpoint{1.932000in}{2.255542in}}%
\pgfpathlineto{\pgfqpoint{1.881071in}{2.255542in}}%
\pgfpathlineto{\pgfqpoint{1.830143in}{2.276158in}}%
\pgfpathlineto{\pgfqpoint{1.779214in}{2.276158in}}%
\pgfpathlineto{\pgfqpoint{1.728286in}{2.276158in}}%
\pgfpathlineto{\pgfqpoint{1.677357in}{2.276158in}}%
\pgfpathlineto{\pgfqpoint{1.626429in}{2.298024in}}%
\pgfpathlineto{\pgfqpoint{1.575500in}{2.394276in}}%
\pgfpathlineto{\pgfqpoint{1.524571in}{2.394276in}}%
\pgfpathlineto{\pgfqpoint{1.473643in}{2.394276in}}%
\pgfpathlineto{\pgfqpoint{1.422714in}{2.394276in}}%
\pgfpathlineto{\pgfqpoint{1.371786in}{2.419405in}}%
\pgfpathlineto{\pgfqpoint{1.320857in}{2.436514in}}%
\pgfpathlineto{\pgfqpoint{1.269929in}{2.436514in}}%
\pgfpathlineto{\pgfqpoint{1.219000in}{2.436514in}}%
\pgfpathlineto{\pgfqpoint{1.168071in}{2.443010in}}%
\pgfpathlineto{\pgfqpoint{1.117143in}{2.443010in}}%
\pgfpathlineto{\pgfqpoint{1.066214in}{2.461507in}}%
\pgfpathlineto{\pgfqpoint{1.015286in}{2.462380in}}%
\pgfpathlineto{\pgfqpoint{0.964357in}{2.462380in}}%
\pgfpathlineto{\pgfqpoint{0.913429in}{2.464390in}}%
\pgfpathlineto{\pgfqpoint{0.862500in}{2.471353in}}%
\pgfpathclose%
\pgfusepath{fill}%
\end{pgfscope}%
\begin{pgfscope}%
\pgfpathrectangle{\pgfqpoint{0.862500in}{0.375000in}}{\pgfqpoint{5.347500in}{2.265000in}}%
\pgfusepath{clip}%
\pgfsetbuttcap%
\pgfsetroundjoin%
\definecolor{currentfill}{rgb}{0.580392,0.403922,0.741176}%
\pgfsetfillcolor{currentfill}%
\pgfsetfillopacity{0.200000}%
\pgfsetlinewidth{0.000000pt}%
\definecolor{currentstroke}{rgb}{0.000000,0.000000,0.000000}%
\pgfsetstrokecolor{currentstroke}%
\pgfsetdash{}{0pt}%
\pgfpathmoveto{\pgfqpoint{0.862500in}{2.531271in}}%
\pgfpathlineto{\pgfqpoint{0.862500in}{2.537045in}}%
\pgfpathlineto{\pgfqpoint{0.913429in}{2.510593in}}%
\pgfpathlineto{\pgfqpoint{0.964357in}{2.510593in}}%
\pgfpathlineto{\pgfqpoint{1.015286in}{2.448803in}}%
\pgfpathlineto{\pgfqpoint{1.066214in}{2.436843in}}%
\pgfpathlineto{\pgfqpoint{1.117143in}{2.436843in}}%
\pgfpathlineto{\pgfqpoint{1.168071in}{2.388435in}}%
\pgfpathlineto{\pgfqpoint{1.219000in}{2.388435in}}%
\pgfpathlineto{\pgfqpoint{1.269929in}{2.359548in}}%
\pgfpathlineto{\pgfqpoint{1.320857in}{2.359548in}}%
\pgfpathlineto{\pgfqpoint{1.371786in}{2.359548in}}%
\pgfpathlineto{\pgfqpoint{1.422714in}{2.350568in}}%
\pgfpathlineto{\pgfqpoint{1.473643in}{2.350568in}}%
\pgfpathlineto{\pgfqpoint{1.524571in}{2.344717in}}%
\pgfpathlineto{\pgfqpoint{1.575500in}{2.331448in}}%
\pgfpathlineto{\pgfqpoint{1.626429in}{2.331448in}}%
\pgfpathlineto{\pgfqpoint{1.677357in}{2.331448in}}%
\pgfpathlineto{\pgfqpoint{1.728286in}{2.331448in}}%
\pgfpathlineto{\pgfqpoint{1.779214in}{2.331448in}}%
\pgfpathlineto{\pgfqpoint{1.830143in}{2.331448in}}%
\pgfpathlineto{\pgfqpoint{1.881071in}{2.325153in}}%
\pgfpathlineto{\pgfqpoint{1.932000in}{2.309525in}}%
\pgfpathlineto{\pgfqpoint{1.982929in}{2.302396in}}%
\pgfpathlineto{\pgfqpoint{2.033857in}{2.262789in}}%
\pgfpathlineto{\pgfqpoint{2.084786in}{2.245398in}}%
\pgfpathlineto{\pgfqpoint{2.135714in}{2.241152in}}%
\pgfpathlineto{\pgfqpoint{2.186643in}{2.188868in}}%
\pgfpathlineto{\pgfqpoint{2.237571in}{2.176710in}}%
\pgfpathlineto{\pgfqpoint{2.288500in}{2.173675in}}%
\pgfpathlineto{\pgfqpoint{2.339429in}{2.122737in}}%
\pgfpathlineto{\pgfqpoint{2.390357in}{2.077345in}}%
\pgfpathlineto{\pgfqpoint{2.441286in}{1.756796in}}%
\pgfpathlineto{\pgfqpoint{2.492214in}{1.703631in}}%
\pgfpathlineto{\pgfqpoint{2.543143in}{1.703631in}}%
\pgfpathlineto{\pgfqpoint{2.594071in}{1.703631in}}%
\pgfpathlineto{\pgfqpoint{2.645000in}{1.610946in}}%
\pgfpathlineto{\pgfqpoint{2.695929in}{1.610946in}}%
\pgfpathlineto{\pgfqpoint{2.746857in}{1.610001in}}%
\pgfpathlineto{\pgfqpoint{2.797786in}{1.610001in}}%
\pgfpathlineto{\pgfqpoint{2.848714in}{1.601874in}}%
\pgfpathlineto{\pgfqpoint{2.899643in}{1.601874in}}%
\pgfpathlineto{\pgfqpoint{2.950571in}{1.601874in}}%
\pgfpathlineto{\pgfqpoint{3.001500in}{1.601874in}}%
\pgfpathlineto{\pgfqpoint{3.052429in}{1.597674in}}%
\pgfpathlineto{\pgfqpoint{3.103357in}{1.568930in}}%
\pgfpathlineto{\pgfqpoint{3.154286in}{1.568268in}}%
\pgfpathlineto{\pgfqpoint{3.205214in}{1.511039in}}%
\pgfpathlineto{\pgfqpoint{3.256143in}{1.491502in}}%
\pgfpathlineto{\pgfqpoint{3.307071in}{1.458070in}}%
\pgfpathlineto{\pgfqpoint{3.358000in}{1.458070in}}%
\pgfpathlineto{\pgfqpoint{3.408929in}{1.430957in}}%
\pgfpathlineto{\pgfqpoint{3.459857in}{1.421807in}}%
\pgfpathlineto{\pgfqpoint{3.510786in}{1.421807in}}%
\pgfpathlineto{\pgfqpoint{3.561714in}{1.388322in}}%
\pgfpathlineto{\pgfqpoint{3.612643in}{1.388322in}}%
\pgfpathlineto{\pgfqpoint{3.663571in}{1.388322in}}%
\pgfpathlineto{\pgfqpoint{3.714500in}{1.367394in}}%
\pgfpathlineto{\pgfqpoint{3.765429in}{1.291670in}}%
\pgfpathlineto{\pgfqpoint{3.816357in}{1.286772in}}%
\pgfpathlineto{\pgfqpoint{3.867286in}{1.285436in}}%
\pgfpathlineto{\pgfqpoint{3.918214in}{1.285436in}}%
\pgfpathlineto{\pgfqpoint{3.969143in}{1.284745in}}%
\pgfpathlineto{\pgfqpoint{4.020071in}{1.284745in}}%
\pgfpathlineto{\pgfqpoint{4.071000in}{1.271736in}}%
\pgfpathlineto{\pgfqpoint{4.121929in}{1.265682in}}%
\pgfpathlineto{\pgfqpoint{4.172857in}{1.265682in}}%
\pgfpathlineto{\pgfqpoint{4.223786in}{1.265682in}}%
\pgfpathlineto{\pgfqpoint{4.274714in}{1.265682in}}%
\pgfpathlineto{\pgfqpoint{4.325643in}{1.265682in}}%
\pgfpathlineto{\pgfqpoint{4.376571in}{1.263283in}}%
\pgfpathlineto{\pgfqpoint{4.427500in}{1.263283in}}%
\pgfpathlineto{\pgfqpoint{4.478429in}{1.263283in}}%
\pgfpathlineto{\pgfqpoint{4.529357in}{1.263283in}}%
\pgfpathlineto{\pgfqpoint{4.580286in}{1.259629in}}%
\pgfpathlineto{\pgfqpoint{4.631214in}{1.256454in}}%
\pgfpathlineto{\pgfqpoint{4.682143in}{1.256389in}}%
\pgfpathlineto{\pgfqpoint{4.733071in}{1.252109in}}%
\pgfpathlineto{\pgfqpoint{4.784000in}{1.252109in}}%
\pgfpathlineto{\pgfqpoint{4.834929in}{1.249508in}}%
\pgfpathlineto{\pgfqpoint{4.885857in}{1.248283in}}%
\pgfpathlineto{\pgfqpoint{4.936786in}{1.248283in}}%
\pgfpathlineto{\pgfqpoint{4.987714in}{1.248283in}}%
\pgfpathlineto{\pgfqpoint{5.038643in}{1.248283in}}%
\pgfpathlineto{\pgfqpoint{5.089571in}{1.243569in}}%
\pgfpathlineto{\pgfqpoint{5.140500in}{1.243569in}}%
\pgfpathlineto{\pgfqpoint{5.191429in}{1.243569in}}%
\pgfpathlineto{\pgfqpoint{5.242357in}{1.243569in}}%
\pgfpathlineto{\pgfqpoint{5.293286in}{1.243143in}}%
\pgfpathlineto{\pgfqpoint{5.344214in}{1.243143in}}%
\pgfpathlineto{\pgfqpoint{5.395143in}{1.242431in}}%
\pgfpathlineto{\pgfqpoint{5.446071in}{1.242431in}}%
\pgfpathlineto{\pgfqpoint{5.497000in}{1.242431in}}%
\pgfpathlineto{\pgfqpoint{5.547929in}{1.242431in}}%
\pgfpathlineto{\pgfqpoint{5.598857in}{1.242431in}}%
\pgfpathlineto{\pgfqpoint{5.649786in}{1.242431in}}%
\pgfpathlineto{\pgfqpoint{5.700714in}{1.242190in}}%
\pgfpathlineto{\pgfqpoint{5.751643in}{1.241816in}}%
\pgfpathlineto{\pgfqpoint{5.802571in}{1.241816in}}%
\pgfpathlineto{\pgfqpoint{5.853500in}{1.241816in}}%
\pgfpathlineto{\pgfqpoint{5.904429in}{1.239718in}}%
\pgfpathlineto{\pgfqpoint{5.955357in}{1.239573in}}%
\pgfpathlineto{\pgfqpoint{6.006286in}{1.239573in}}%
\pgfpathlineto{\pgfqpoint{6.057214in}{1.238624in}}%
\pgfpathlineto{\pgfqpoint{6.108143in}{1.238624in}}%
\pgfpathlineto{\pgfqpoint{6.159071in}{1.238576in}}%
\pgfpathlineto{\pgfqpoint{6.210000in}{1.238576in}}%
\pgfpathlineto{\pgfqpoint{6.260929in}{1.237526in}}%
\pgfpathlineto{\pgfqpoint{6.311857in}{1.236995in}}%
\pgfpathlineto{\pgfqpoint{6.362786in}{1.236995in}}%
\pgfpathlineto{\pgfqpoint{6.413714in}{1.236995in}}%
\pgfpathlineto{\pgfqpoint{6.464643in}{1.236995in}}%
\pgfpathlineto{\pgfqpoint{6.515571in}{1.236995in}}%
\pgfpathlineto{\pgfqpoint{6.566500in}{1.236995in}}%
\pgfpathlineto{\pgfqpoint{6.617429in}{1.236775in}}%
\pgfpathlineto{\pgfqpoint{6.668357in}{1.236775in}}%
\pgfpathlineto{\pgfqpoint{6.719286in}{1.226380in}}%
\pgfpathlineto{\pgfqpoint{6.770214in}{1.226380in}}%
\pgfpathlineto{\pgfqpoint{6.821143in}{1.226380in}}%
\pgfpathlineto{\pgfqpoint{6.872071in}{1.219879in}}%
\pgfpathlineto{\pgfqpoint{6.923000in}{1.219686in}}%
\pgfpathlineto{\pgfqpoint{6.973929in}{1.219686in}}%
\pgfpathlineto{\pgfqpoint{7.024857in}{1.219259in}}%
\pgfpathlineto{\pgfqpoint{7.075786in}{1.205643in}}%
\pgfpathlineto{\pgfqpoint{7.126714in}{1.205641in}}%
\pgfpathlineto{\pgfqpoint{7.177643in}{1.205641in}}%
\pgfpathlineto{\pgfqpoint{7.228571in}{1.205641in}}%
\pgfpathlineto{\pgfqpoint{7.279500in}{1.203816in}}%
\pgfpathlineto{\pgfqpoint{7.330429in}{1.203223in}}%
\pgfpathlineto{\pgfqpoint{7.381357in}{1.203207in}}%
\pgfpathlineto{\pgfqpoint{7.432286in}{1.202917in}}%
\pgfpathlineto{\pgfqpoint{7.483214in}{1.202864in}}%
\pgfpathlineto{\pgfqpoint{7.534143in}{1.202756in}}%
\pgfpathlineto{\pgfqpoint{7.585071in}{1.202756in}}%
\pgfpathlineto{\pgfqpoint{7.636000in}{1.202555in}}%
\pgfpathlineto{\pgfqpoint{7.686929in}{1.201763in}}%
\pgfpathlineto{\pgfqpoint{7.737857in}{1.197121in}}%
\pgfpathlineto{\pgfqpoint{7.788786in}{1.197121in}}%
\pgfpathlineto{\pgfqpoint{7.839714in}{1.194027in}}%
\pgfpathlineto{\pgfqpoint{7.890643in}{1.193920in}}%
\pgfpathlineto{\pgfqpoint{7.941571in}{1.193630in}}%
\pgfpathlineto{\pgfqpoint{7.992500in}{1.187786in}}%
\pgfpathlineto{\pgfqpoint{8.043429in}{1.186943in}}%
\pgfpathlineto{\pgfqpoint{8.094357in}{1.186943in}}%
\pgfpathlineto{\pgfqpoint{8.145286in}{1.186803in}}%
\pgfpathlineto{\pgfqpoint{8.196214in}{1.184912in}}%
\pgfpathlineto{\pgfqpoint{8.247143in}{1.184823in}}%
\pgfpathlineto{\pgfqpoint{8.298071in}{1.183763in}}%
\pgfpathlineto{\pgfqpoint{8.349000in}{1.183483in}}%
\pgfpathlineto{\pgfqpoint{8.399929in}{1.180323in}}%
\pgfpathlineto{\pgfqpoint{8.450857in}{1.180323in}}%
\pgfpathlineto{\pgfqpoint{8.501786in}{1.180323in}}%
\pgfpathlineto{\pgfqpoint{8.552714in}{1.180262in}}%
\pgfpathlineto{\pgfqpoint{8.603643in}{1.180256in}}%
\pgfpathlineto{\pgfqpoint{8.654571in}{1.180254in}}%
\pgfpathlineto{\pgfqpoint{8.705500in}{1.180251in}}%
\pgfpathlineto{\pgfqpoint{8.756429in}{1.180251in}}%
\pgfpathlineto{\pgfqpoint{8.807357in}{1.175396in}}%
\pgfpathlineto{\pgfqpoint{8.858286in}{1.171371in}}%
\pgfpathlineto{\pgfqpoint{8.909214in}{1.171371in}}%
\pgfpathlineto{\pgfqpoint{8.960143in}{1.171365in}}%
\pgfpathlineto{\pgfqpoint{9.011071in}{1.171299in}}%
\pgfpathlineto{\pgfqpoint{9.062000in}{1.171255in}}%
\pgfpathlineto{\pgfqpoint{9.112929in}{1.171255in}}%
\pgfpathlineto{\pgfqpoint{9.163857in}{1.171253in}}%
\pgfpathlineto{\pgfqpoint{9.214786in}{1.171253in}}%
\pgfpathlineto{\pgfqpoint{9.265714in}{1.144612in}}%
\pgfpathlineto{\pgfqpoint{9.316643in}{1.144612in}}%
\pgfpathlineto{\pgfqpoint{9.367571in}{1.144612in}}%
\pgfpathlineto{\pgfqpoint{9.418500in}{1.134912in}}%
\pgfpathlineto{\pgfqpoint{9.469429in}{1.067187in}}%
\pgfpathlineto{\pgfqpoint{9.469429in}{0.505099in}}%
\pgfpathlineto{\pgfqpoint{9.469429in}{0.505099in}}%
\pgfpathlineto{\pgfqpoint{9.418500in}{0.591544in}}%
\pgfpathlineto{\pgfqpoint{9.367571in}{0.600886in}}%
\pgfpathlineto{\pgfqpoint{9.316643in}{0.600886in}}%
\pgfpathlineto{\pgfqpoint{9.265714in}{0.600886in}}%
\pgfpathlineto{\pgfqpoint{9.214786in}{0.627027in}}%
\pgfpathlineto{\pgfqpoint{9.163857in}{0.627027in}}%
\pgfpathlineto{\pgfqpoint{9.112929in}{0.627042in}}%
\pgfpathlineto{\pgfqpoint{9.062000in}{0.627042in}}%
\pgfpathlineto{\pgfqpoint{9.011071in}{0.627409in}}%
\pgfpathlineto{\pgfqpoint{8.960143in}{0.627949in}}%
\pgfpathlineto{\pgfqpoint{8.909214in}{0.628002in}}%
\pgfpathlineto{\pgfqpoint{8.858286in}{0.628002in}}%
\pgfpathlineto{\pgfqpoint{8.807357in}{0.632538in}}%
\pgfpathlineto{\pgfqpoint{8.756429in}{0.637663in}}%
\pgfpathlineto{\pgfqpoint{8.705500in}{0.637663in}}%
\pgfpathlineto{\pgfqpoint{8.654571in}{0.637689in}}%
\pgfpathlineto{\pgfqpoint{8.603643in}{0.637702in}}%
\pgfpathlineto{\pgfqpoint{8.552714in}{0.637747in}}%
\pgfpathlineto{\pgfqpoint{8.501786in}{0.638253in}}%
\pgfpathlineto{\pgfqpoint{8.450857in}{0.638253in}}%
\pgfpathlineto{\pgfqpoint{8.399929in}{0.638253in}}%
\pgfpathlineto{\pgfqpoint{8.349000in}{0.641772in}}%
\pgfpathlineto{\pgfqpoint{8.298071in}{0.642056in}}%
\pgfpathlineto{\pgfqpoint{8.247143in}{0.643423in}}%
\pgfpathlineto{\pgfqpoint{8.196214in}{0.644152in}}%
\pgfpathlineto{\pgfqpoint{8.145286in}{0.645969in}}%
\pgfpathlineto{\pgfqpoint{8.094357in}{0.647117in}}%
\pgfpathlineto{\pgfqpoint{8.043429in}{0.647117in}}%
\pgfpathlineto{\pgfqpoint{7.992500in}{0.648541in}}%
\pgfpathlineto{\pgfqpoint{7.941571in}{0.654724in}}%
\pgfpathlineto{\pgfqpoint{7.890643in}{0.656284in}}%
\pgfpathlineto{\pgfqpoint{7.839714in}{0.657153in}}%
\pgfpathlineto{\pgfqpoint{7.788786in}{0.661450in}}%
\pgfpathlineto{\pgfqpoint{7.737857in}{0.661450in}}%
\pgfpathlineto{\pgfqpoint{7.686929in}{0.666160in}}%
\pgfpathlineto{\pgfqpoint{7.636000in}{0.672513in}}%
\pgfpathlineto{\pgfqpoint{7.585071in}{0.674095in}}%
\pgfpathlineto{\pgfqpoint{7.534143in}{0.674095in}}%
\pgfpathlineto{\pgfqpoint{7.483214in}{0.674962in}}%
\pgfpathlineto{\pgfqpoint{7.432286in}{0.675394in}}%
\pgfpathlineto{\pgfqpoint{7.381357in}{0.675678in}}%
\pgfpathlineto{\pgfqpoint{7.330429in}{0.675810in}}%
\pgfpathlineto{\pgfqpoint{7.279500in}{0.680441in}}%
\pgfpathlineto{\pgfqpoint{7.228571in}{0.682201in}}%
\pgfpathlineto{\pgfqpoint{7.177643in}{0.682201in}}%
\pgfpathlineto{\pgfqpoint{7.126714in}{0.682201in}}%
\pgfpathlineto{\pgfqpoint{7.075786in}{0.682218in}}%
\pgfpathlineto{\pgfqpoint{7.024857in}{0.696666in}}%
\pgfpathlineto{\pgfqpoint{6.973929in}{0.700075in}}%
\pgfpathlineto{\pgfqpoint{6.923000in}{0.700075in}}%
\pgfpathlineto{\pgfqpoint{6.872071in}{0.701613in}}%
\pgfpathlineto{\pgfqpoint{6.821143in}{0.707165in}}%
\pgfpathlineto{\pgfqpoint{6.770214in}{0.707165in}}%
\pgfpathlineto{\pgfqpoint{6.719286in}{0.707165in}}%
\pgfpathlineto{\pgfqpoint{6.668357in}{0.715626in}}%
\pgfpathlineto{\pgfqpoint{6.617429in}{0.715626in}}%
\pgfpathlineto{\pgfqpoint{6.566500in}{0.715800in}}%
\pgfpathlineto{\pgfqpoint{6.515571in}{0.715800in}}%
\pgfpathlineto{\pgfqpoint{6.464643in}{0.715800in}}%
\pgfpathlineto{\pgfqpoint{6.413714in}{0.715800in}}%
\pgfpathlineto{\pgfqpoint{6.362786in}{0.715800in}}%
\pgfpathlineto{\pgfqpoint{6.311857in}{0.715800in}}%
\pgfpathlineto{\pgfqpoint{6.260929in}{0.719940in}}%
\pgfpathlineto{\pgfqpoint{6.210000in}{0.728120in}}%
\pgfpathlineto{\pgfqpoint{6.159071in}{0.728120in}}%
\pgfpathlineto{\pgfqpoint{6.108143in}{0.728488in}}%
\pgfpathlineto{\pgfqpoint{6.057214in}{0.728488in}}%
\pgfpathlineto{\pgfqpoint{6.006286in}{0.735610in}}%
\pgfpathlineto{\pgfqpoint{5.955357in}{0.735610in}}%
\pgfpathlineto{\pgfqpoint{5.904429in}{0.736703in}}%
\pgfpathlineto{\pgfqpoint{5.853500in}{0.752012in}}%
\pgfpathlineto{\pgfqpoint{5.802571in}{0.752012in}}%
\pgfpathlineto{\pgfqpoint{5.751643in}{0.752012in}}%
\pgfpathlineto{\pgfqpoint{5.700714in}{0.754612in}}%
\pgfpathlineto{\pgfqpoint{5.649786in}{0.756394in}}%
\pgfpathlineto{\pgfqpoint{5.598857in}{0.756394in}}%
\pgfpathlineto{\pgfqpoint{5.547929in}{0.756394in}}%
\pgfpathlineto{\pgfqpoint{5.497000in}{0.756394in}}%
\pgfpathlineto{\pgfqpoint{5.446071in}{0.756394in}}%
\pgfpathlineto{\pgfqpoint{5.395143in}{0.756394in}}%
\pgfpathlineto{\pgfqpoint{5.344214in}{0.761564in}}%
\pgfpathlineto{\pgfqpoint{5.293286in}{0.761564in}}%
\pgfpathlineto{\pgfqpoint{5.242357in}{0.764595in}}%
\pgfpathlineto{\pgfqpoint{5.191429in}{0.764595in}}%
\pgfpathlineto{\pgfqpoint{5.140500in}{0.764595in}}%
\pgfpathlineto{\pgfqpoint{5.089571in}{0.764595in}}%
\pgfpathlineto{\pgfqpoint{5.038643in}{0.796526in}}%
\pgfpathlineto{\pgfqpoint{4.987714in}{0.796526in}}%
\pgfpathlineto{\pgfqpoint{4.936786in}{0.796526in}}%
\pgfpathlineto{\pgfqpoint{4.885857in}{0.796526in}}%
\pgfpathlineto{\pgfqpoint{4.834929in}{0.804784in}}%
\pgfpathlineto{\pgfqpoint{4.784000in}{0.808843in}}%
\pgfpathlineto{\pgfqpoint{4.733071in}{0.808843in}}%
\pgfpathlineto{\pgfqpoint{4.682143in}{0.834942in}}%
\pgfpathlineto{\pgfqpoint{4.631214in}{0.835386in}}%
\pgfpathlineto{\pgfqpoint{4.580286in}{0.852485in}}%
\pgfpathlineto{\pgfqpoint{4.529357in}{0.865086in}}%
\pgfpathlineto{\pgfqpoint{4.478429in}{0.865086in}}%
\pgfpathlineto{\pgfqpoint{4.427500in}{0.865086in}}%
\pgfpathlineto{\pgfqpoint{4.376571in}{0.865086in}}%
\pgfpathlineto{\pgfqpoint{4.325643in}{0.878378in}}%
\pgfpathlineto{\pgfqpoint{4.274714in}{0.878378in}}%
\pgfpathlineto{\pgfqpoint{4.223786in}{0.878378in}}%
\pgfpathlineto{\pgfqpoint{4.172857in}{0.878378in}}%
\pgfpathlineto{\pgfqpoint{4.121929in}{0.878378in}}%
\pgfpathlineto{\pgfqpoint{4.071000in}{0.908734in}}%
\pgfpathlineto{\pgfqpoint{4.020071in}{0.963519in}}%
\pgfpathlineto{\pgfqpoint{3.969143in}{0.963519in}}%
\pgfpathlineto{\pgfqpoint{3.918214in}{0.968583in}}%
\pgfpathlineto{\pgfqpoint{3.867286in}{0.968583in}}%
\pgfpathlineto{\pgfqpoint{3.816357in}{0.974564in}}%
\pgfpathlineto{\pgfqpoint{3.765429in}{0.994326in}}%
\pgfpathlineto{\pgfqpoint{3.714500in}{1.024026in}}%
\pgfpathlineto{\pgfqpoint{3.663571in}{1.090875in}}%
\pgfpathlineto{\pgfqpoint{3.612643in}{1.090875in}}%
\pgfpathlineto{\pgfqpoint{3.561714in}{1.090875in}}%
\pgfpathlineto{\pgfqpoint{3.510786in}{1.123505in}}%
\pgfpathlineto{\pgfqpoint{3.459857in}{1.123505in}}%
\pgfpathlineto{\pgfqpoint{3.408929in}{1.165084in}}%
\pgfpathlineto{\pgfqpoint{3.358000in}{1.218991in}}%
\pgfpathlineto{\pgfqpoint{3.307071in}{1.218991in}}%
\pgfpathlineto{\pgfqpoint{3.256143in}{1.242269in}}%
\pgfpathlineto{\pgfqpoint{3.205214in}{1.270067in}}%
\pgfpathlineto{\pgfqpoint{3.154286in}{1.297978in}}%
\pgfpathlineto{\pgfqpoint{3.103357in}{1.303975in}}%
\pgfpathlineto{\pgfqpoint{3.052429in}{1.340412in}}%
\pgfpathlineto{\pgfqpoint{3.001500in}{1.374712in}}%
\pgfpathlineto{\pgfqpoint{2.950571in}{1.374712in}}%
\pgfpathlineto{\pgfqpoint{2.899643in}{1.374712in}}%
\pgfpathlineto{\pgfqpoint{2.848714in}{1.374712in}}%
\pgfpathlineto{\pgfqpoint{2.797786in}{1.411786in}}%
\pgfpathlineto{\pgfqpoint{2.746857in}{1.411786in}}%
\pgfpathlineto{\pgfqpoint{2.695929in}{1.420256in}}%
\pgfpathlineto{\pgfqpoint{2.645000in}{1.420256in}}%
\pgfpathlineto{\pgfqpoint{2.594071in}{1.566129in}}%
\pgfpathlineto{\pgfqpoint{2.543143in}{1.566129in}}%
\pgfpathlineto{\pgfqpoint{2.492214in}{1.566129in}}%
\pgfpathlineto{\pgfqpoint{2.441286in}{1.609889in}}%
\pgfpathlineto{\pgfqpoint{2.390357in}{1.736550in}}%
\pgfpathlineto{\pgfqpoint{2.339429in}{1.837798in}}%
\pgfpathlineto{\pgfqpoint{2.288500in}{1.899684in}}%
\pgfpathlineto{\pgfqpoint{2.237571in}{1.908900in}}%
\pgfpathlineto{\pgfqpoint{2.186643in}{1.941409in}}%
\pgfpathlineto{\pgfqpoint{2.135714in}{1.988435in}}%
\pgfpathlineto{\pgfqpoint{2.084786in}{2.007546in}}%
\pgfpathlineto{\pgfqpoint{2.033857in}{2.085944in}}%
\pgfpathlineto{\pgfqpoint{1.982929in}{2.217882in}}%
\pgfpathlineto{\pgfqpoint{1.932000in}{2.241092in}}%
\pgfpathlineto{\pgfqpoint{1.881071in}{2.266764in}}%
\pgfpathlineto{\pgfqpoint{1.830143in}{2.271832in}}%
\pgfpathlineto{\pgfqpoint{1.779214in}{2.271832in}}%
\pgfpathlineto{\pgfqpoint{1.728286in}{2.271832in}}%
\pgfpathlineto{\pgfqpoint{1.677357in}{2.271832in}}%
\pgfpathlineto{\pgfqpoint{1.626429in}{2.271832in}}%
\pgfpathlineto{\pgfqpoint{1.575500in}{2.271832in}}%
\pgfpathlineto{\pgfqpoint{1.524571in}{2.278695in}}%
\pgfpathlineto{\pgfqpoint{1.473643in}{2.280823in}}%
\pgfpathlineto{\pgfqpoint{1.422714in}{2.280823in}}%
\pgfpathlineto{\pgfqpoint{1.371786in}{2.283951in}}%
\pgfpathlineto{\pgfqpoint{1.320857in}{2.283951in}}%
\pgfpathlineto{\pgfqpoint{1.269929in}{2.283951in}}%
\pgfpathlineto{\pgfqpoint{1.219000in}{2.290624in}}%
\pgfpathlineto{\pgfqpoint{1.168071in}{2.290624in}}%
\pgfpathlineto{\pgfqpoint{1.117143in}{2.328501in}}%
\pgfpathlineto{\pgfqpoint{1.066214in}{2.328501in}}%
\pgfpathlineto{\pgfqpoint{1.015286in}{2.332558in}}%
\pgfpathlineto{\pgfqpoint{0.964357in}{2.408396in}}%
\pgfpathlineto{\pgfqpoint{0.913429in}{2.408396in}}%
\pgfpathlineto{\pgfqpoint{0.862500in}{2.531271in}}%
\pgfpathclose%
\pgfusepath{fill}%
\end{pgfscope}%
\begin{pgfscope}%
\pgfpathrectangle{\pgfqpoint{0.862500in}{0.375000in}}{\pgfqpoint{5.347500in}{2.265000in}}%
\pgfusepath{clip}%
\pgfsetroundcap%
\pgfsetroundjoin%
\pgfsetlinewidth{1.505625pt}%
\definecolor{currentstroke}{rgb}{0.121569,0.466667,0.705882}%
\pgfsetstrokecolor{currentstroke}%
\pgfsetdash{}{0pt}%
\pgfpathmoveto{\pgfqpoint{0.862500in}{2.503651in}}%
\pgfpathlineto{\pgfqpoint{0.913429in}{2.503449in}}%
\pgfpathlineto{\pgfqpoint{0.964357in}{2.482500in}}%
\pgfpathlineto{\pgfqpoint{1.015286in}{2.482500in}}%
\pgfpathlineto{\pgfqpoint{1.066214in}{2.464862in}}%
\pgfpathlineto{\pgfqpoint{1.117143in}{2.453776in}}%
\pgfpathlineto{\pgfqpoint{1.168071in}{2.425507in}}%
\pgfpathlineto{\pgfqpoint{1.219000in}{2.418079in}}%
\pgfpathlineto{\pgfqpoint{1.320857in}{2.417778in}}%
\pgfpathlineto{\pgfqpoint{1.371786in}{2.392970in}}%
\pgfpathlineto{\pgfqpoint{1.422714in}{2.378568in}}%
\pgfpathlineto{\pgfqpoint{1.473643in}{2.358442in}}%
\pgfpathlineto{\pgfqpoint{1.524571in}{2.355136in}}%
\pgfpathlineto{\pgfqpoint{1.575500in}{2.319624in}}%
\pgfpathlineto{\pgfqpoint{1.626429in}{2.312547in}}%
\pgfpathlineto{\pgfqpoint{1.728286in}{2.312547in}}%
\pgfpathlineto{\pgfqpoint{1.779214in}{2.273351in}}%
\pgfpathlineto{\pgfqpoint{1.881071in}{2.273351in}}%
\pgfpathlineto{\pgfqpoint{1.932000in}{2.241314in}}%
\pgfpathlineto{\pgfqpoint{2.288500in}{2.241314in}}%
\pgfpathlineto{\pgfqpoint{2.339429in}{2.236203in}}%
\pgfpathlineto{\pgfqpoint{2.390357in}{2.225026in}}%
\pgfpathlineto{\pgfqpoint{2.543143in}{2.225026in}}%
\pgfpathlineto{\pgfqpoint{2.594071in}{2.182241in}}%
\pgfpathlineto{\pgfqpoint{2.746857in}{2.182241in}}%
\pgfpathlineto{\pgfqpoint{2.797786in}{2.179890in}}%
\pgfpathlineto{\pgfqpoint{3.205214in}{2.179890in}}%
\pgfpathlineto{\pgfqpoint{3.256143in}{2.155471in}}%
\pgfpathlineto{\pgfqpoint{3.663571in}{2.155471in}}%
\pgfpathlineto{\pgfqpoint{3.714500in}{2.119791in}}%
\pgfpathlineto{\pgfqpoint{4.071000in}{2.119791in}}%
\pgfpathlineto{\pgfqpoint{4.121929in}{2.102509in}}%
\pgfpathlineto{\pgfqpoint{4.885857in}{2.102509in}}%
\pgfpathlineto{\pgfqpoint{4.936786in}{2.089713in}}%
\pgfpathlineto{\pgfqpoint{4.987714in}{2.089713in}}%
\pgfpathlineto{\pgfqpoint{5.038643in}{2.085702in}}%
\pgfpathlineto{\pgfqpoint{6.223889in}{2.085702in}}%
\pgfpathlineto{\pgfqpoint{6.223889in}{2.085702in}}%
\pgfusepath{stroke}%
\end{pgfscope}%
\begin{pgfscope}%
\pgfpathrectangle{\pgfqpoint{0.862500in}{0.375000in}}{\pgfqpoint{5.347500in}{2.265000in}}%
\pgfusepath{clip}%
\pgfsetroundcap%
\pgfsetroundjoin%
\pgfsetlinewidth{1.505625pt}%
\definecolor{currentstroke}{rgb}{1.000000,0.498039,0.054902}%
\pgfsetstrokecolor{currentstroke}%
\pgfsetdash{}{0pt}%
\pgfpathmoveto{\pgfqpoint{0.862500in}{2.524256in}}%
\pgfpathlineto{\pgfqpoint{0.913429in}{2.524256in}}%
\pgfpathlineto{\pgfqpoint{0.964357in}{2.495003in}}%
\pgfpathlineto{\pgfqpoint{1.015286in}{2.488901in}}%
\pgfpathlineto{\pgfqpoint{1.066214in}{2.416278in}}%
\pgfpathlineto{\pgfqpoint{1.117143in}{2.391030in}}%
\pgfpathlineto{\pgfqpoint{1.219000in}{2.391030in}}%
\pgfpathlineto{\pgfqpoint{1.269929in}{2.386619in}}%
\pgfpathlineto{\pgfqpoint{1.473643in}{2.386619in}}%
\pgfpathlineto{\pgfqpoint{1.524571in}{2.355211in}}%
\pgfpathlineto{\pgfqpoint{1.575500in}{2.313462in}}%
\pgfpathlineto{\pgfqpoint{1.626429in}{2.313462in}}%
\pgfpathlineto{\pgfqpoint{1.677357in}{2.216209in}}%
\pgfpathlineto{\pgfqpoint{1.728286in}{2.213947in}}%
\pgfpathlineto{\pgfqpoint{1.932000in}{2.213947in}}%
\pgfpathlineto{\pgfqpoint{1.982929in}{2.210703in}}%
\pgfpathlineto{\pgfqpoint{2.135714in}{2.210703in}}%
\pgfpathlineto{\pgfqpoint{2.186643in}{2.156666in}}%
\pgfpathlineto{\pgfqpoint{2.288500in}{2.155263in}}%
\pgfpathlineto{\pgfqpoint{2.339429in}{2.155263in}}%
\pgfpathlineto{\pgfqpoint{2.390357in}{2.092050in}}%
\pgfpathlineto{\pgfqpoint{2.492214in}{2.092050in}}%
\pgfpathlineto{\pgfqpoint{2.543143in}{2.087952in}}%
\pgfpathlineto{\pgfqpoint{2.594071in}{2.032863in}}%
\pgfpathlineto{\pgfqpoint{2.645000in}{1.960763in}}%
\pgfpathlineto{\pgfqpoint{2.695929in}{1.931285in}}%
\pgfpathlineto{\pgfqpoint{2.746857in}{1.893677in}}%
\pgfpathlineto{\pgfqpoint{2.797786in}{1.893677in}}%
\pgfpathlineto{\pgfqpoint{2.848714in}{1.844510in}}%
\pgfpathlineto{\pgfqpoint{2.899643in}{1.844510in}}%
\pgfpathlineto{\pgfqpoint{2.950571in}{1.805956in}}%
\pgfpathlineto{\pgfqpoint{3.001500in}{1.805956in}}%
\pgfpathlineto{\pgfqpoint{3.052429in}{1.775114in}}%
\pgfpathlineto{\pgfqpoint{3.103357in}{1.735977in}}%
\pgfpathlineto{\pgfqpoint{3.154286in}{1.683012in}}%
\pgfpathlineto{\pgfqpoint{3.205214in}{1.683012in}}%
\pgfpathlineto{\pgfqpoint{3.256143in}{1.678223in}}%
\pgfpathlineto{\pgfqpoint{3.358000in}{1.678223in}}%
\pgfpathlineto{\pgfqpoint{3.408929in}{1.674814in}}%
\pgfpathlineto{\pgfqpoint{3.459857in}{1.674814in}}%
\pgfpathlineto{\pgfqpoint{3.510786in}{1.642536in}}%
\pgfpathlineto{\pgfqpoint{3.561714in}{1.633980in}}%
\pgfpathlineto{\pgfqpoint{3.612643in}{1.629265in}}%
\pgfpathlineto{\pgfqpoint{3.663571in}{1.629265in}}%
\pgfpathlineto{\pgfqpoint{3.714500in}{1.605327in}}%
\pgfpathlineto{\pgfqpoint{3.765429in}{1.545221in}}%
\pgfpathlineto{\pgfqpoint{3.816357in}{1.493118in}}%
\pgfpathlineto{\pgfqpoint{3.867286in}{1.493118in}}%
\pgfpathlineto{\pgfqpoint{3.918214in}{1.490870in}}%
\pgfpathlineto{\pgfqpoint{4.020071in}{1.356878in}}%
\pgfpathlineto{\pgfqpoint{4.071000in}{1.348754in}}%
\pgfpathlineto{\pgfqpoint{4.121929in}{1.348754in}}%
\pgfpathlineto{\pgfqpoint{4.172857in}{1.321654in}}%
\pgfpathlineto{\pgfqpoint{4.274714in}{1.321654in}}%
\pgfpathlineto{\pgfqpoint{4.325643in}{1.282071in}}%
\pgfpathlineto{\pgfqpoint{4.376571in}{1.258033in}}%
\pgfpathlineto{\pgfqpoint{4.427500in}{1.258033in}}%
\pgfpathlineto{\pgfqpoint{4.478429in}{1.210362in}}%
\pgfpathlineto{\pgfqpoint{4.631214in}{1.210362in}}%
\pgfpathlineto{\pgfqpoint{4.682143in}{1.167721in}}%
\pgfpathlineto{\pgfqpoint{4.834929in}{1.167721in}}%
\pgfpathlineto{\pgfqpoint{4.885857in}{1.146498in}}%
\pgfpathlineto{\pgfqpoint{4.936786in}{1.135873in}}%
\pgfpathlineto{\pgfqpoint{4.987714in}{1.135873in}}%
\pgfpathlineto{\pgfqpoint{5.038643in}{1.133419in}}%
\pgfpathlineto{\pgfqpoint{5.089571in}{1.133419in}}%
\pgfpathlineto{\pgfqpoint{5.140500in}{1.079265in}}%
\pgfpathlineto{\pgfqpoint{5.344214in}{1.079265in}}%
\pgfpathlineto{\pgfqpoint{5.395143in}{1.067866in}}%
\pgfpathlineto{\pgfqpoint{5.446071in}{1.067866in}}%
\pgfpathlineto{\pgfqpoint{5.497000in}{1.054411in}}%
\pgfpathlineto{\pgfqpoint{5.547929in}{1.008991in}}%
\pgfpathlineto{\pgfqpoint{5.598857in}{0.982914in}}%
\pgfpathlineto{\pgfqpoint{5.751643in}{0.982914in}}%
\pgfpathlineto{\pgfqpoint{5.802571in}{0.969330in}}%
\pgfpathlineto{\pgfqpoint{6.159071in}{0.969330in}}%
\pgfpathlineto{\pgfqpoint{6.210000in}{0.967217in}}%
\pgfpathlineto{\pgfqpoint{6.223889in}{0.963312in}}%
\pgfpathlineto{\pgfqpoint{6.223889in}{0.963312in}}%
\pgfusepath{stroke}%
\end{pgfscope}%
\begin{pgfscope}%
\pgfpathrectangle{\pgfqpoint{0.862500in}{0.375000in}}{\pgfqpoint{5.347500in}{2.265000in}}%
\pgfusepath{clip}%
\pgfsetroundcap%
\pgfsetroundjoin%
\pgfsetlinewidth{1.505625pt}%
\definecolor{currentstroke}{rgb}{0.172549,0.627451,0.172549}%
\pgfsetstrokecolor{currentstroke}%
\pgfsetdash{}{0pt}%
\pgfpathmoveto{\pgfqpoint{0.862500in}{2.521142in}}%
\pgfpathlineto{\pgfqpoint{0.913429in}{2.474921in}}%
\pgfpathlineto{\pgfqpoint{0.964357in}{2.474921in}}%
\pgfpathlineto{\pgfqpoint{1.015286in}{2.460666in}}%
\pgfpathlineto{\pgfqpoint{1.066214in}{2.413362in}}%
\pgfpathlineto{\pgfqpoint{1.117143in}{2.409806in}}%
\pgfpathlineto{\pgfqpoint{1.320857in}{2.409806in}}%
\pgfpathlineto{\pgfqpoint{1.371786in}{2.399710in}}%
\pgfpathlineto{\pgfqpoint{1.575500in}{2.399710in}}%
\pgfpathlineto{\pgfqpoint{1.626429in}{2.336988in}}%
\pgfpathlineto{\pgfqpoint{1.932000in}{2.336988in}}%
\pgfpathlineto{\pgfqpoint{1.982929in}{2.333250in}}%
\pgfpathlineto{\pgfqpoint{2.084786in}{2.333250in}}%
\pgfpathlineto{\pgfqpoint{2.135714in}{2.284386in}}%
\pgfpathlineto{\pgfqpoint{2.237571in}{2.284386in}}%
\pgfpathlineto{\pgfqpoint{2.288500in}{2.274085in}}%
\pgfpathlineto{\pgfqpoint{2.339429in}{2.202238in}}%
\pgfpathlineto{\pgfqpoint{2.390357in}{2.202238in}}%
\pgfpathlineto{\pgfqpoint{2.441286in}{2.195701in}}%
\pgfpathlineto{\pgfqpoint{2.492214in}{2.195701in}}%
\pgfpathlineto{\pgfqpoint{2.543143in}{2.153037in}}%
\pgfpathlineto{\pgfqpoint{2.594071in}{2.153037in}}%
\pgfpathlineto{\pgfqpoint{2.645000in}{2.129572in}}%
\pgfpathlineto{\pgfqpoint{2.695929in}{2.046126in}}%
\pgfpathlineto{\pgfqpoint{2.797786in}{2.046126in}}%
\pgfpathlineto{\pgfqpoint{2.848714in}{2.029372in}}%
\pgfpathlineto{\pgfqpoint{2.899643in}{1.957115in}}%
\pgfpathlineto{\pgfqpoint{3.001500in}{1.957115in}}%
\pgfpathlineto{\pgfqpoint{3.052429in}{1.930937in}}%
\pgfpathlineto{\pgfqpoint{3.103357in}{1.844337in}}%
\pgfpathlineto{\pgfqpoint{3.307071in}{1.844337in}}%
\pgfpathlineto{\pgfqpoint{3.358000in}{1.748692in}}%
\pgfpathlineto{\pgfqpoint{3.408929in}{1.684082in}}%
\pgfpathlineto{\pgfqpoint{3.459857in}{1.667950in}}%
\pgfpathlineto{\pgfqpoint{3.510786in}{1.592483in}}%
\pgfpathlineto{\pgfqpoint{3.561714in}{1.536179in}}%
\pgfpathlineto{\pgfqpoint{3.612643in}{1.500990in}}%
\pgfpathlineto{\pgfqpoint{3.816357in}{1.500990in}}%
\pgfpathlineto{\pgfqpoint{3.918214in}{1.474122in}}%
\pgfpathlineto{\pgfqpoint{4.071000in}{1.474122in}}%
\pgfpathlineto{\pgfqpoint{4.121929in}{1.358921in}}%
\pgfpathlineto{\pgfqpoint{4.172857in}{1.358921in}}%
\pgfpathlineto{\pgfqpoint{4.223786in}{1.292110in}}%
\pgfpathlineto{\pgfqpoint{4.478429in}{1.292110in}}%
\pgfpathlineto{\pgfqpoint{4.529357in}{1.259037in}}%
\pgfpathlineto{\pgfqpoint{4.733071in}{1.259037in}}%
\pgfpathlineto{\pgfqpoint{4.784000in}{1.220275in}}%
\pgfpathlineto{\pgfqpoint{4.885857in}{1.220275in}}%
\pgfpathlineto{\pgfqpoint{4.936786in}{1.199576in}}%
\pgfpathlineto{\pgfqpoint{5.344214in}{1.198229in}}%
\pgfpathlineto{\pgfqpoint{5.395143in}{1.156247in}}%
\pgfpathlineto{\pgfqpoint{5.547929in}{1.156247in}}%
\pgfpathlineto{\pgfqpoint{5.598857in}{1.148983in}}%
\pgfpathlineto{\pgfqpoint{5.802571in}{1.148983in}}%
\pgfpathlineto{\pgfqpoint{5.853500in}{1.122160in}}%
\pgfpathlineto{\pgfqpoint{6.108143in}{1.122160in}}%
\pgfpathlineto{\pgfqpoint{6.159071in}{1.112047in}}%
\pgfpathlineto{\pgfqpoint{6.223889in}{1.112047in}}%
\pgfpathlineto{\pgfqpoint{6.223889in}{1.112047in}}%
\pgfusepath{stroke}%
\end{pgfscope}%
\begin{pgfscope}%
\pgfpathrectangle{\pgfqpoint{0.862500in}{0.375000in}}{\pgfqpoint{5.347500in}{2.265000in}}%
\pgfusepath{clip}%
\pgfsetroundcap%
\pgfsetroundjoin%
\pgfsetlinewidth{1.505625pt}%
\definecolor{currentstroke}{rgb}{0.839216,0.152941,0.156863}%
\pgfsetstrokecolor{currentstroke}%
\pgfsetdash{}{0pt}%
\pgfpathmoveto{\pgfqpoint{0.862500in}{2.493421in}}%
\pgfpathlineto{\pgfqpoint{0.913429in}{2.486192in}}%
\pgfpathlineto{\pgfqpoint{0.964357in}{2.483140in}}%
\pgfpathlineto{\pgfqpoint{1.066214in}{2.481688in}}%
\pgfpathlineto{\pgfqpoint{1.117143in}{2.459109in}}%
\pgfpathlineto{\pgfqpoint{1.168071in}{2.459109in}}%
\pgfpathlineto{\pgfqpoint{1.219000in}{2.452408in}}%
\pgfpathlineto{\pgfqpoint{1.320857in}{2.452408in}}%
\pgfpathlineto{\pgfqpoint{1.371786in}{2.433415in}}%
\pgfpathlineto{\pgfqpoint{1.422714in}{2.419525in}}%
\pgfpathlineto{\pgfqpoint{1.575500in}{2.419525in}}%
\pgfpathlineto{\pgfqpoint{1.626429in}{2.360122in}}%
\pgfpathlineto{\pgfqpoint{1.677357in}{2.336737in}}%
\pgfpathlineto{\pgfqpoint{1.830143in}{2.336737in}}%
\pgfpathlineto{\pgfqpoint{1.881071in}{2.321315in}}%
\pgfpathlineto{\pgfqpoint{1.932000in}{2.321315in}}%
\pgfpathlineto{\pgfqpoint{1.982929in}{2.262491in}}%
\pgfpathlineto{\pgfqpoint{2.084786in}{2.223049in}}%
\pgfpathlineto{\pgfqpoint{2.186643in}{2.213723in}}%
\pgfpathlineto{\pgfqpoint{2.237571in}{2.169684in}}%
\pgfpathlineto{\pgfqpoint{2.288500in}{2.163851in}}%
\pgfpathlineto{\pgfqpoint{2.339429in}{2.135799in}}%
\pgfpathlineto{\pgfqpoint{2.441286in}{2.135799in}}%
\pgfpathlineto{\pgfqpoint{2.492214in}{2.050967in}}%
\pgfpathlineto{\pgfqpoint{2.543143in}{1.998390in}}%
\pgfpathlineto{\pgfqpoint{2.594071in}{1.928944in}}%
\pgfpathlineto{\pgfqpoint{2.645000in}{1.878239in}}%
\pgfpathlineto{\pgfqpoint{2.695929in}{1.759217in}}%
\pgfpathlineto{\pgfqpoint{2.746857in}{1.722338in}}%
\pgfpathlineto{\pgfqpoint{2.797786in}{1.679978in}}%
\pgfpathlineto{\pgfqpoint{2.848714in}{1.667524in}}%
\pgfpathlineto{\pgfqpoint{2.950571in}{1.623073in}}%
\pgfpathlineto{\pgfqpoint{3.001500in}{1.612307in}}%
\pgfpathlineto{\pgfqpoint{3.103357in}{1.612193in}}%
\pgfpathlineto{\pgfqpoint{3.154286in}{1.443471in}}%
\pgfpathlineto{\pgfqpoint{3.205214in}{1.410481in}}%
\pgfpathlineto{\pgfqpoint{3.307071in}{1.410481in}}%
\pgfpathlineto{\pgfqpoint{3.358000in}{1.314582in}}%
\pgfpathlineto{\pgfqpoint{3.408929in}{1.314582in}}%
\pgfpathlineto{\pgfqpoint{3.459857in}{1.265888in}}%
\pgfpathlineto{\pgfqpoint{3.510786in}{1.253734in}}%
\pgfpathlineto{\pgfqpoint{3.561714in}{1.253734in}}%
\pgfpathlineto{\pgfqpoint{3.612643in}{1.250336in}}%
\pgfpathlineto{\pgfqpoint{3.663571in}{1.207305in}}%
\pgfpathlineto{\pgfqpoint{3.714500in}{1.207305in}}%
\pgfpathlineto{\pgfqpoint{3.765429in}{1.198026in}}%
\pgfpathlineto{\pgfqpoint{3.816357in}{1.196520in}}%
\pgfpathlineto{\pgfqpoint{3.867286in}{1.192528in}}%
\pgfpathlineto{\pgfqpoint{3.918214in}{1.192528in}}%
\pgfpathlineto{\pgfqpoint{3.969143in}{1.178731in}}%
\pgfpathlineto{\pgfqpoint{4.071000in}{1.178731in}}%
\pgfpathlineto{\pgfqpoint{4.121929in}{1.164494in}}%
\pgfpathlineto{\pgfqpoint{4.631214in}{1.163513in}}%
\pgfpathlineto{\pgfqpoint{4.682143in}{1.154679in}}%
\pgfpathlineto{\pgfqpoint{4.733071in}{1.148247in}}%
\pgfpathlineto{\pgfqpoint{4.834929in}{1.148247in}}%
\pgfpathlineto{\pgfqpoint{4.885857in}{1.138501in}}%
\pgfpathlineto{\pgfqpoint{4.987714in}{1.138501in}}%
\pgfpathlineto{\pgfqpoint{5.038643in}{1.130363in}}%
\pgfpathlineto{\pgfqpoint{5.089571in}{1.130363in}}%
\pgfpathlineto{\pgfqpoint{5.140500in}{1.125386in}}%
\pgfpathlineto{\pgfqpoint{5.191429in}{1.116537in}}%
\pgfpathlineto{\pgfqpoint{5.242357in}{1.115235in}}%
\pgfpathlineto{\pgfqpoint{5.293286in}{1.109221in}}%
\pgfpathlineto{\pgfqpoint{5.395143in}{1.108770in}}%
\pgfpathlineto{\pgfqpoint{5.446071in}{1.103032in}}%
\pgfpathlineto{\pgfqpoint{5.547929in}{1.099699in}}%
\pgfpathlineto{\pgfqpoint{5.598857in}{1.097810in}}%
\pgfpathlineto{\pgfqpoint{5.649786in}{1.097797in}}%
\pgfpathlineto{\pgfqpoint{5.751643in}{1.094544in}}%
\pgfpathlineto{\pgfqpoint{5.802571in}{1.094544in}}%
\pgfpathlineto{\pgfqpoint{5.853500in}{1.085977in}}%
\pgfpathlineto{\pgfqpoint{6.108143in}{1.084656in}}%
\pgfpathlineto{\pgfqpoint{6.159071in}{1.081682in}}%
\pgfpathlineto{\pgfqpoint{6.223889in}{1.081011in}}%
\pgfpathlineto{\pgfqpoint{6.223889in}{1.081011in}}%
\pgfusepath{stroke}%
\end{pgfscope}%
\begin{pgfscope}%
\pgfpathrectangle{\pgfqpoint{0.862500in}{0.375000in}}{\pgfqpoint{5.347500in}{2.265000in}}%
\pgfusepath{clip}%
\pgfsetroundcap%
\pgfsetroundjoin%
\pgfsetlinewidth{1.505625pt}%
\definecolor{currentstroke}{rgb}{0.580392,0.403922,0.741176}%
\pgfsetstrokecolor{currentstroke}%
\pgfsetdash{}{0pt}%
\pgfpathmoveto{\pgfqpoint{0.862500in}{2.534016in}}%
\pgfpathlineto{\pgfqpoint{0.913429in}{2.459870in}}%
\pgfpathlineto{\pgfqpoint{0.964357in}{2.459870in}}%
\pgfpathlineto{\pgfqpoint{1.015286in}{2.391586in}}%
\pgfpathlineto{\pgfqpoint{1.066214in}{2.383266in}}%
\pgfpathlineto{\pgfqpoint{1.117143in}{2.383266in}}%
\pgfpathlineto{\pgfqpoint{1.168071in}{2.339763in}}%
\pgfpathlineto{\pgfqpoint{1.219000in}{2.339763in}}%
\pgfpathlineto{\pgfqpoint{1.269929in}{2.321438in}}%
\pgfpathlineto{\pgfqpoint{1.371786in}{2.321438in}}%
\pgfpathlineto{\pgfqpoint{1.422714in}{2.315287in}}%
\pgfpathlineto{\pgfqpoint{1.473643in}{2.315287in}}%
\pgfpathlineto{\pgfqpoint{1.524571in}{2.311248in}}%
\pgfpathlineto{\pgfqpoint{1.575500in}{2.301114in}}%
\pgfpathlineto{\pgfqpoint{1.830143in}{2.301114in}}%
\pgfpathlineto{\pgfqpoint{1.881071in}{2.295422in}}%
\pgfpathlineto{\pgfqpoint{1.932000in}{2.274882in}}%
\pgfpathlineto{\pgfqpoint{1.982929in}{2.260011in}}%
\pgfpathlineto{\pgfqpoint{2.033857in}{2.178875in}}%
\pgfpathlineto{\pgfqpoint{2.084786in}{2.136733in}}%
\pgfpathlineto{\pgfqpoint{2.135714in}{2.126773in}}%
\pgfpathlineto{\pgfqpoint{2.186643in}{2.076496in}}%
\pgfpathlineto{\pgfqpoint{2.237571in}{2.056655in}}%
\pgfpathlineto{\pgfqpoint{2.288500in}{2.051332in}}%
\pgfpathlineto{\pgfqpoint{2.339429in}{1.996392in}}%
\pgfpathlineto{\pgfqpoint{2.390357in}{1.931585in}}%
\pgfpathlineto{\pgfqpoint{2.441286in}{1.685805in}}%
\pgfpathlineto{\pgfqpoint{2.492214in}{1.636807in}}%
\pgfpathlineto{\pgfqpoint{2.594071in}{1.636807in}}%
\pgfpathlineto{\pgfqpoint{2.645000in}{1.521230in}}%
\pgfpathlineto{\pgfqpoint{2.695929in}{1.521230in}}%
\pgfpathlineto{\pgfqpoint{2.746857in}{1.517177in}}%
\pgfpathlineto{\pgfqpoint{2.797786in}{1.517177in}}%
\pgfpathlineto{\pgfqpoint{2.848714in}{1.497394in}}%
\pgfpathlineto{\pgfqpoint{3.001500in}{1.497394in}}%
\pgfpathlineto{\pgfqpoint{3.052429in}{1.481573in}}%
\pgfpathlineto{\pgfqpoint{3.103357in}{1.449939in}}%
\pgfpathlineto{\pgfqpoint{3.154286in}{1.447293in}}%
\pgfpathlineto{\pgfqpoint{3.205214in}{1.401165in}}%
\pgfpathlineto{\pgfqpoint{3.256143in}{1.378451in}}%
\pgfpathlineto{\pgfqpoint{3.307071in}{1.348928in}}%
\pgfpathlineto{\pgfqpoint{3.358000in}{1.348928in}}%
\pgfpathlineto{\pgfqpoint{3.408929in}{1.311623in}}%
\pgfpathlineto{\pgfqpoint{3.459857in}{1.290665in}}%
\pgfpathlineto{\pgfqpoint{3.510786in}{1.290665in}}%
\pgfpathlineto{\pgfqpoint{3.561714in}{1.257484in}}%
\pgfpathlineto{\pgfqpoint{3.663571in}{1.257484in}}%
\pgfpathlineto{\pgfqpoint{3.714500in}{1.220779in}}%
\pgfpathlineto{\pgfqpoint{3.765429in}{1.160868in}}%
\pgfpathlineto{\pgfqpoint{3.816357in}{1.150740in}}%
\pgfpathlineto{\pgfqpoint{3.867286in}{1.147794in}}%
\pgfpathlineto{\pgfqpoint{3.918214in}{1.147794in}}%
\pgfpathlineto{\pgfqpoint{3.969143in}{1.145598in}}%
\pgfpathlineto{\pgfqpoint{4.020071in}{1.145598in}}%
\pgfpathlineto{\pgfqpoint{4.071000in}{1.118712in}}%
\pgfpathlineto{\pgfqpoint{4.121929in}{1.105000in}}%
\pgfpathlineto{\pgfqpoint{4.325643in}{1.105000in}}%
\pgfpathlineto{\pgfqpoint{4.376571in}{1.099265in}}%
\pgfpathlineto{\pgfqpoint{4.529357in}{1.099265in}}%
\pgfpathlineto{\pgfqpoint{4.631214in}{1.085622in}}%
\pgfpathlineto{\pgfqpoint{4.682143in}{1.085446in}}%
\pgfpathlineto{\pgfqpoint{4.733071in}{1.074907in}}%
\pgfpathlineto{\pgfqpoint{4.784000in}{1.074907in}}%
\pgfpathlineto{\pgfqpoint{4.885857in}{1.068706in}}%
\pgfpathlineto{\pgfqpoint{5.038643in}{1.068706in}}%
\pgfpathlineto{\pgfqpoint{5.089571in}{1.056609in}}%
\pgfpathlineto{\pgfqpoint{5.344214in}{1.055494in}}%
\pgfpathlineto{\pgfqpoint{5.395143in}{1.053611in}}%
\pgfpathlineto{\pgfqpoint{5.700714in}{1.052967in}}%
\pgfpathlineto{\pgfqpoint{5.802571in}{1.052013in}}%
\pgfpathlineto{\pgfqpoint{5.853500in}{1.052013in}}%
\pgfpathlineto{\pgfqpoint{5.904429in}{1.046518in}}%
\pgfpathlineto{\pgfqpoint{6.006286in}{1.046132in}}%
\pgfpathlineto{\pgfqpoint{6.057214in}{1.043626in}}%
\pgfpathlineto{\pgfqpoint{6.223889in}{1.042726in}}%
\pgfpathlineto{\pgfqpoint{6.223889in}{1.042726in}}%
\pgfusepath{stroke}%
\end{pgfscope}%
\begin{pgfscope}%
\pgfsetrectcap%
\pgfsetmiterjoin%
\pgfsetlinewidth{0.000000pt}%
\definecolor{currentstroke}{rgb}{1.000000,1.000000,1.000000}%
\pgfsetstrokecolor{currentstroke}%
\pgfsetdash{}{0pt}%
\pgfpathmoveto{\pgfqpoint{0.862500in}{0.375000in}}%
\pgfpathlineto{\pgfqpoint{0.862500in}{2.640000in}}%
\pgfusepath{}%
\end{pgfscope}%
\begin{pgfscope}%
\pgfsetrectcap%
\pgfsetmiterjoin%
\pgfsetlinewidth{0.000000pt}%
\definecolor{currentstroke}{rgb}{1.000000,1.000000,1.000000}%
\pgfsetstrokecolor{currentstroke}%
\pgfsetdash{}{0pt}%
\pgfpathmoveto{\pgfqpoint{6.210000in}{0.375000in}}%
\pgfpathlineto{\pgfqpoint{6.210000in}{2.640000in}}%
\pgfusepath{}%
\end{pgfscope}%
\begin{pgfscope}%
\pgfsetrectcap%
\pgfsetmiterjoin%
\pgfsetlinewidth{0.000000pt}%
\definecolor{currentstroke}{rgb}{1.000000,1.000000,1.000000}%
\pgfsetstrokecolor{currentstroke}%
\pgfsetdash{}{0pt}%
\pgfpathmoveto{\pgfqpoint{0.862500in}{0.375000in}}%
\pgfpathlineto{\pgfqpoint{6.210000in}{0.375000in}}%
\pgfusepath{}%
\end{pgfscope}%
\begin{pgfscope}%
\pgfsetrectcap%
\pgfsetmiterjoin%
\pgfsetlinewidth{0.000000pt}%
\definecolor{currentstroke}{rgb}{1.000000,1.000000,1.000000}%
\pgfsetstrokecolor{currentstroke}%
\pgfsetdash{}{0pt}%
\pgfpathmoveto{\pgfqpoint{0.862500in}{2.640000in}}%
\pgfpathlineto{\pgfqpoint{6.210000in}{2.640000in}}%
\pgfusepath{}%
\end{pgfscope}%
\begin{pgfscope}%
\definecolor{textcolor}{rgb}{0.150000,0.150000,0.150000}%
\pgfsetstrokecolor{textcolor}%
\pgfsetfillcolor{textcolor}%
\pgftext[x=3.536250in,y=2.723333in,,base]{\color{textcolor}\rmfamily\fontsize{8.000000}{9.600000}\selectfont Hartmann6}%
\end{pgfscope}%
\begin{pgfscope}%
\pgfsetroundcap%
\pgfsetroundjoin%
\pgfsetlinewidth{1.505625pt}%
\definecolor{currentstroke}{rgb}{0.121569,0.466667,0.705882}%
\pgfsetstrokecolor{currentstroke}%
\pgfsetdash{}{0pt}%
\pgfpathmoveto{\pgfqpoint{0.962500in}{1.189344in}}%
\pgfpathlineto{\pgfqpoint{1.184722in}{1.189344in}}%
\pgfusepath{stroke}%
\end{pgfscope}%
\begin{pgfscope}%
\definecolor{textcolor}{rgb}{0.150000,0.150000,0.150000}%
\pgfsetstrokecolor{textcolor}%
\pgfsetfillcolor{textcolor}%
\pgftext[x=1.273611in,y=1.150455in,left,base]{\color{textcolor}\rmfamily\fontsize{8.000000}{9.600000}\selectfont random}%
\end{pgfscope}%
\begin{pgfscope}%
\pgfsetroundcap%
\pgfsetroundjoin%
\pgfsetlinewidth{1.505625pt}%
\definecolor{currentstroke}{rgb}{1.000000,0.498039,0.054902}%
\pgfsetstrokecolor{currentstroke}%
\pgfsetdash{}{0pt}%
\pgfpathmoveto{\pgfqpoint{0.962500in}{1.026258in}}%
\pgfpathlineto{\pgfqpoint{1.184722in}{1.026258in}}%
\pgfusepath{stroke}%
\end{pgfscope}%
\begin{pgfscope}%
\definecolor{textcolor}{rgb}{0.150000,0.150000,0.150000}%
\pgfsetstrokecolor{textcolor}%
\pgfsetfillcolor{textcolor}%
\pgftext[x=1.273611in,y=0.987369in,left,base]{\color{textcolor}\rmfamily\fontsize{8.000000}{9.600000}\selectfont 5 x DNGO retrain-reset}%
\end{pgfscope}%
\begin{pgfscope}%
\pgfsetroundcap%
\pgfsetroundjoin%
\pgfsetlinewidth{1.505625pt}%
\definecolor{currentstroke}{rgb}{0.172549,0.627451,0.172549}%
\pgfsetstrokecolor{currentstroke}%
\pgfsetdash{}{0pt}%
\pgfpathmoveto{\pgfqpoint{0.962500in}{0.863172in}}%
\pgfpathlineto{\pgfqpoint{1.184722in}{0.863172in}}%
\pgfusepath{stroke}%
\end{pgfscope}%
\begin{pgfscope}%
\definecolor{textcolor}{rgb}{0.150000,0.150000,0.150000}%
\pgfsetstrokecolor{textcolor}%
\pgfsetfillcolor{textcolor}%
\pgftext[x=1.273611in,y=0.824283in,left,base]{\color{textcolor}\rmfamily\fontsize{8.000000}{9.600000}\selectfont DNGO retrain-reset}%
\end{pgfscope}%
\begin{pgfscope}%
\pgfsetroundcap%
\pgfsetroundjoin%
\pgfsetlinewidth{1.505625pt}%
\definecolor{currentstroke}{rgb}{0.839216,0.152941,0.156863}%
\pgfsetstrokecolor{currentstroke}%
\pgfsetdash{}{0pt}%
\pgfpathmoveto{\pgfqpoint{0.962500in}{0.700087in}}%
\pgfpathlineto{\pgfqpoint{1.184722in}{0.700087in}}%
\pgfusepath{stroke}%
\end{pgfscope}%
\begin{pgfscope}%
\definecolor{textcolor}{rgb}{0.150000,0.150000,0.150000}%
\pgfsetstrokecolor{textcolor}%
\pgfsetfillcolor{textcolor}%
\pgftext[x=1.273611in,y=0.661198in,left,base]{\color{textcolor}\rmfamily\fontsize{8.000000}{9.600000}\selectfont GP}%
\end{pgfscope}%
\begin{pgfscope}%
\pgfsetroundcap%
\pgfsetroundjoin%
\pgfsetlinewidth{1.505625pt}%
\definecolor{currentstroke}{rgb}{0.580392,0.403922,0.741176}%
\pgfsetstrokecolor{currentstroke}%
\pgfsetdash{}{0pt}%
\pgfpathmoveto{\pgfqpoint{0.962500in}{0.537001in}}%
\pgfpathlineto{\pgfqpoint{1.184722in}{0.537001in}}%
\pgfusepath{stroke}%
\end{pgfscope}%
\begin{pgfscope}%
\definecolor{textcolor}{rgb}{0.150000,0.150000,0.150000}%
\pgfsetstrokecolor{textcolor}%
\pgfsetfillcolor{textcolor}%
\pgftext[x=1.273611in,y=0.498112in,left,base]{\color{textcolor}\rmfamily\fontsize{8.000000}{9.600000}\selectfont GP MCMC}%
\end{pgfscope}%
\end{pgfpicture}%
\makeatother%
\endgroup%

        %\caption{Hartmann6}
    \end{minipage}

\end{document}

% - Problems:
%   - DNGO exploring corners
%   - Why is the speed dropping dramatically?

% - How it was done:
%   - Finding good weight decay:
%     - use gpy points
%     - run one step of DNGO BO
%     - Plot the mean and acq restricted to the exploited area

% Properties of method:
% - gp very similar to nn
% - low variance..
% - Sudden spikes in variance because of basis function fit
% - local exploration bad (exploitation?)

% - Removing data: doesn't this lead to miscalibrated uncertainty estimates? => prediction wrong about something of which it should be certain. (inspired by (2))
% - Test Dropout? (Inspired by (1) and (2))
% - Need to train for longer to overfit/be different. Is this good/desireable?
% - Better for e.g. noisy data?
